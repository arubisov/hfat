\chapter{Results}

In this chapter we explore the dynamics of the continuous time and discrete time models, and perform in-sample and out-of-sample backtests to compare the performance of the continuous and discrete time solutions to the stochastic optimal control problem.

\section{Calibration}

All tests in this chapter were run using the following global set of parameters:
\begin{center}
\begin{tabular}{rll}
time window for computing price change & $\Delta t_S$ & 1000ms \\
time window for averaging order imbalance & $\Delta t_I$ & 1000ms \\
number of imbalance bins & $\#_{bins}$ & 5 \\
fill probability constant & $\kappa$ & 100 \\
&& \\
\multicolumn{3}{p{\linewidth}}{For each daily calibration, we then computed the remaining parameters utilizing the following formulae:} \\
&& \\
infinitesimal generator matrix & $\mat{G}$ & \autoref{eq:MLEG} \\
transition probability matrix & $\mat{P}$ & \autoref{eq:CTMCPG} \\
market order arrival intensities & $\mu^\pm$ & \autoref{eq:MLElambda}
\end{tabular}
\end{center}

Additionally, $\xi$ was computed as half of the simple average of the bid-ask spread observed during the trading day, rounded to the nearest half-cent; and the imbalance bins $\rho$ were computed as the partitioning of the interval $[-1,1]$ into percentile bins symmetric around zero, where the percentile interval was $100 \div \#_{bins}$.

As was mentioned in Chapter 2, the exploratory data analysis done on the data made use of an unorthodox Markov chain, where its state at time $t$ was actually not determinable at time $t$ because the price change $\Delta S(t)$ was computed over the \emph{future} time interval $\Delta t_S$. (See \autoref{sec:2DCTMC}.) In the optimal stochastic control formulations, the Markov chain was defined instead such the price change was computed over the \emph{past} interval $\Delta t_S$. However, it was of interest to explore what results would be obtained if the calibration was still done using the non $\cF$-predictable method. A justification for doing so is that in a given Markov state $Z$, there is a state-dependent arrival rate of price updates, and there is a state-dependent distribution of jumps when a price update occurs. So that although the price change is measured over the future when calibrating, this really is a way of getting at the state-dependence of those price changes. In the following tests, this calibration method is denoted `w nFPC', standing for `with non-$\cF$-predictable calibration'.

\section{Dynamics of \texorpdfstring{${\delta^{\pm}}^*$}{delta}}

We have solved the same stochastic control problem using both continuous and discrete time, which have yielded markedly different resulting formulae for the optimal limit order posting depths. In this section we explore the calibrated results obtained on an in-sample backtest on \texttt{ORCL}, both calibrating and backtesting using data from 2013-05-15. The dynamics of $\delta^\pm$ were obtained using a time-to-maturity of 100sec to best depict the behavior as we approached the end of day trading; as the time-to-maturity horizon increases, the postings depths tend to stabilize. 

\begin{figure}
\centering
\begin{subfigure}{.45\linewidth}
  \centering
  \setlength\figureheight{\linewidth} 
  \setlength\figurewidth{\linewidth}
  \tikzsetnextfilename{dp_cts_z1}
  % This file was created by matlab2tikz.
%
%The latest updates can be retrieved from
%  http://www.mathworks.com/matlabcentral/fileexchange/22022-matlab2tikz-matlab2tikz
%where you can also make suggestions and rate matlab2tikz.
%
\definecolor{mycolor1}{rgb}{0.00000,1.00000,0.14286}%
\definecolor{mycolor2}{rgb}{0.00000,1.00000,0.28571}%
\definecolor{mycolor3}{rgb}{0.00000,1.00000,0.42857}%
\definecolor{mycolor4}{rgb}{0.00000,1.00000,0.57143}%
\definecolor{mycolor5}{rgb}{0.00000,1.00000,0.71429}%
\definecolor{mycolor6}{rgb}{0.00000,1.00000,0.85714}%
\definecolor{mycolor7}{rgb}{0.00000,1.00000,1.00000}%
\definecolor{mycolor8}{rgb}{0.00000,0.87500,1.00000}%
\definecolor{mycolor9}{rgb}{0.00000,0.62500,1.00000}%
\definecolor{mycolor10}{rgb}{0.12500,0.00000,1.00000}%
\definecolor{mycolor11}{rgb}{0.25000,0.00000,1.00000}%
\definecolor{mycolor12}{rgb}{0.37500,0.00000,1.00000}%
\definecolor{mycolor13}{rgb}{0.50000,0.00000,1.00000}%
\definecolor{mycolor14}{rgb}{0.62500,0.00000,1.00000}%
\definecolor{mycolor15}{rgb}{0.75000,0.00000,1.00000}%
\definecolor{mycolor16}{rgb}{0.87500,0.00000,1.00000}%
\definecolor{mycolor17}{rgb}{1.00000,0.00000,1.00000}%
\definecolor{mycolor18}{rgb}{1.00000,0.00000,0.87500}%
\definecolor{mycolor19}{rgb}{1.00000,0.00000,0.62500}%
\definecolor{mycolor20}{rgb}{0.85714,0.00000,0.00000}%
\definecolor{mycolor21}{rgb}{0.71429,0.00000,0.00000}%
%
\begin{tikzpicture}

\begin{axis}[%
width=4.1in,
height=3.803in,
at={(0.809in,0.513in)},
scale only axis,
point meta min=0,
point meta max=1,
every outer x axis line/.append style={black},
every x tick label/.append style={font=\color{black}},
xmin=0,
xmax=600,
every outer y axis line/.append style={black},
every y tick label/.append style={font=\color{black}},
ymin=0,
ymax=0.01,
axis background/.style={fill=white},
axis x line*=bottom,
axis y line*=left,
colormap={mymap}{[1pt] rgb(0pt)=(0,1,0); rgb(7pt)=(0,1,1); rgb(15pt)=(0,0,1); rgb(23pt)=(1,0,1); rgb(31pt)=(1,0,0); rgb(38pt)=(0,0,0)},
colorbar,
colorbar style={separate axis lines,every outer x axis line/.append style={black},every x tick label/.append style={font=\color{black}},every outer y axis line/.append style={black},every y tick label/.append style={font=\color{black}},yticklabels={{-19},{-17},{-15},{-13},{-11},{-9},{-7},{-5},{-3},{-1},{1},{3},{5},{7},{9},{11},{13},{15},{17},{19}}}
]
\addplot [color=green,solid,forget plot]
  table[row sep=crcr]{%
0.01	0\\
1.01	0\\
2.01	0\\
3.01	0\\
4.01	0\\
5.01	0\\
6.01	0\\
7.01	0\\
8.01	0\\
9.01	0\\
10.01	0\\
11.01	0\\
12.01	0\\
13.01	0\\
14.01	0\\
15.01	0\\
16.01	0\\
17.01	0\\
18.01	0\\
19.01	0\\
20.01	0\\
21.01	0\\
22.01	0\\
23.01	0\\
24.01	0\\
25.01	0\\
26.01	0\\
27.01	0\\
28.01	0\\
29.01	0\\
30.01	0\\
31.01	0\\
32.01	0\\
33.01	0\\
34.01	0\\
35.01	0\\
36.01	0\\
37.01	0\\
38.01	0\\
39.01	0\\
40.01	0\\
41.01	0\\
42.01	0\\
43.01	0\\
44.01	0\\
45.01	0\\
46.01	0\\
47.01	0\\
48.01	0\\
49.01	0\\
50.01	0\\
51.01	0\\
52.01	0\\
53.01	0\\
54.01	0\\
55.01	0\\
56.01	0\\
57.01	0\\
58.01	0\\
59.01	0\\
60.01	0\\
61.01	0\\
62.01	0\\
63.01	0\\
64.01	0\\
65.01	0\\
66.01	0\\
67.01	0\\
68.01	0\\
69.01	0\\
70.01	0\\
71.01	0\\
72.01	0\\
73.01	0\\
74.01	0\\
75.01	0\\
76.01	0\\
77.01	0\\
78.01	0\\
79.01	0\\
80.01	0\\
81.01	0\\
82.01	0\\
83.01	0\\
84.01	0\\
85.01	0\\
86.01	0\\
87.01	0\\
88.01	0\\
89.01	0\\
90.01	0\\
91.01	0\\
92.01	0\\
93.01	0\\
94.01	0\\
95.01	0\\
96.01	0\\
97.01	0\\
98.01	0\\
99.01	0\\
100.01	0\\
101.01	0\\
102.01	0\\
103.01	0\\
104.01	0\\
105.01	0\\
106.01	0\\
107.01	0\\
108.01	0\\
109.01	0\\
110.01	0\\
111.01	0\\
112.01	0\\
113.01	0\\
114.01	0\\
115.01	0\\
116.01	0\\
117.01	0\\
118.01	0\\
119.01	0\\
120.01	0\\
121.01	0\\
122.01	0\\
123.01	0\\
124.01	0\\
125.01	0\\
126.01	0\\
127.01	0\\
128.01	0\\
129.01	0\\
130.01	0\\
131.01	0\\
132.01	0\\
133.01	0\\
134.01	0\\
135.01	0\\
136.01	0\\
137.01	0\\
138.01	0\\
139.01	0\\
140.01	0\\
141.01	0\\
142.01	0\\
143.01	0\\
144.01	0\\
145.01	0\\
146.01	0\\
147.01	0\\
148.01	0\\
149.01	0\\
150.01	0\\
151.01	0\\
152.01	0\\
153.01	0\\
154.01	0\\
155.01	0\\
156.01	0\\
157.01	0\\
158.01	0\\
159.01	0\\
160.01	0\\
161.01	0\\
162.01	0\\
163.01	0\\
164.01	0\\
165.01	0\\
166.01	0\\
167.01	0\\
168.01	0\\
169.01	0\\
170.01	0\\
171.01	0\\
172.01	0\\
173.01	0\\
174.01	0\\
175.01	0\\
176.01	0\\
177.01	0\\
178.01	0\\
179.01	0\\
180.01	0\\
181.01	0\\
182.01	0\\
183.01	0\\
184.01	0\\
185.01	0\\
186.01	0\\
187.01	0\\
188.01	0\\
189.01	0\\
190.01	0\\
191.01	0\\
192.01	0\\
193.01	0\\
194.01	0\\
195.01	0\\
196.01	0\\
197.01	0\\
198.01	0\\
199.01	0\\
200.01	0\\
201.01	0\\
202.01	0\\
203.01	0\\
204.01	0\\
205.01	0\\
206.01	0\\
207.01	0\\
208.01	0\\
209.01	0\\
210.01	0\\
211.01	0\\
212.01	0\\
213.01	0\\
214.01	0\\
215.01	0\\
216.01	0\\
217.01	0\\
218.01	0\\
219.01	0\\
220.01	0\\
221.01	0\\
222.01	0\\
223.01	0\\
224.01	0\\
225.01	0\\
226.01	0\\
227.01	0\\
228.01	0\\
229.01	0\\
230.01	0\\
231.01	0\\
232.01	0\\
233.01	0\\
234.01	0\\
235.01	0\\
236.01	0\\
237.01	0\\
238.01	0\\
239.01	0\\
240.01	0\\
241.01	0\\
242.01	0\\
243.01	0\\
244.01	0\\
245.01	0\\
246.01	0\\
247.01	0\\
248.01	0\\
249.01	0\\
250.01	0\\
251.01	0\\
252.01	0\\
253.01	0\\
254.01	0\\
255.01	0\\
256.01	0\\
257.01	0\\
258.01	0\\
259.01	0\\
260.01	0\\
261.01	0\\
262.01	0\\
263.01	0\\
264.01	0\\
265.01	0\\
266.01	0\\
267.01	0\\
268.01	0\\
269.01	0\\
270.01	0\\
271.01	0\\
272.01	0\\
273.01	0\\
274.01	0\\
275.01	0\\
276.01	0\\
277.01	0\\
278.01	0\\
279.01	0\\
280.01	0\\
281.01	0\\
282.01	0\\
283.01	0\\
284.01	0\\
285.01	0\\
286.01	0\\
287.01	0\\
288.01	0\\
289.01	0\\
290.01	0\\
291.01	0\\
292.01	0\\
293.01	0\\
294.01	0\\
295.01	0\\
296.01	0\\
297.01	0\\
298.01	0\\
299.01	0\\
300.01	0\\
301.01	0\\
302.01	0\\
303.01	0\\
304.01	0\\
305.01	0\\
306.01	0\\
307.01	0\\
308.01	0\\
309.01	0\\
310.01	0\\
311.01	0\\
312.01	0\\
313.01	0\\
314.01	0\\
315.01	0\\
316.01	0\\
317.01	0\\
318.01	0\\
319.01	0\\
320.01	0\\
321.01	0\\
322.01	0\\
323.01	0\\
324.01	0\\
325.01	0\\
326.01	0\\
327.01	0\\
328.01	0\\
329.01	0\\
330.01	0\\
331.01	0\\
332.01	0\\
333.01	0\\
334.01	0\\
335.01	0\\
336.01	0\\
337.01	0\\
338.01	0\\
339.01	0\\
340.01	0\\
341.01	0\\
342.01	0\\
343.01	0\\
344.01	0\\
345.01	0\\
346.01	0\\
347.01	0\\
348.01	0\\
349.01	0\\
350.01	0\\
351.01	0\\
352.01	0\\
353.01	0\\
354.01	0\\
355.01	0\\
356.01	0\\
357.01	0\\
358.01	0\\
359.01	0\\
360.01	0\\
361.01	0\\
362.01	0\\
363.01	0\\
364.01	0\\
365.01	0\\
366.01	0\\
367.01	0\\
368.01	0\\
369.01	0\\
370.01	0\\
371.01	0\\
372.01	0\\
373.01	0\\
374.01	0\\
375.01	0\\
376.01	0\\
377.01	0\\
378.01	0\\
379.01	0\\
380.01	0\\
381.01	0\\
382.01	0\\
383.01	0\\
384.01	0\\
385.01	0\\
386.01	0\\
387.01	0\\
388.01	0\\
389.01	0\\
390.01	0\\
391.01	0\\
392.01	0\\
393.01	0\\
394.01	0\\
395.01	0\\
396.01	0\\
397.01	0\\
398.01	0\\
399.01	0\\
400.01	0\\
401.01	0\\
402.01	0\\
403.01	0\\
404.01	0\\
405.01	0\\
406.01	0\\
407.01	0\\
408.01	0\\
409.01	0\\
410.01	0\\
411.01	0\\
412.01	0\\
413.01	0\\
414.01	0\\
415.01	0\\
416.01	0\\
417.01	0\\
418.01	0\\
419.01	0\\
420.01	0\\
421.01	0\\
422.01	0\\
423.01	0\\
424.01	0\\
425.01	0\\
426.01	0\\
427.01	0\\
428.01	0\\
429.01	0\\
430.01	0\\
431.01	0\\
432.01	0\\
433.01	0\\
434.01	0\\
435.01	0\\
436.01	0\\
437.01	0\\
438.01	0\\
439.01	0\\
440.01	0\\
441.01	0\\
442.01	0\\
443.01	0\\
444.01	0\\
445.01	0\\
446.01	0\\
447.01	0\\
448.01	0\\
449.01	0\\
450.01	0\\
451.01	1.73472347597681e-18\\
452.01	0\\
453.01	0\\
454.01	0\\
455.01	0\\
456.01	0\\
457.01	0\\
458.01	0\\
459.01	0\\
460.01	0\\
461.01	0\\
462.01	0\\
463.01	0\\
464.01	0\\
465.01	0\\
466.01	0\\
467.01	0\\
468.01	0\\
469.01	1.73472347597681e-18\\
470.01	0\\
471.01	0\\
472.01	0\\
473.01	0\\
474.01	0\\
475.01	0\\
476.01	1.73472347597681e-18\\
477.01	0\\
478.01	0\\
479.01	0\\
480.01	0\\
481.01	0\\
482.01	0\\
483.01	0\\
484.01	0\\
485.01	0\\
486.01	0\\
487.01	0\\
488.01	0\\
489.01	0\\
490.01	0\\
491.01	0\\
492.01	0\\
493.01	0\\
494.01	0\\
495.01	0\\
496.01	0\\
497.01	0\\
498.01	0\\
499.01	0\\
500.01	0\\
501.01	0\\
502.01	0\\
503.01	0\\
504.01	0\\
505.01	0\\
506.01	0\\
507.01	0\\
508.01	0\\
509.01	0\\
510.01	0\\
511.01	0\\
512.01	1.73472347597681e-18\\
513.01	0\\
514.01	0\\
515.01	0\\
516.01	0\\
517.01	0\\
518.01	0\\
519.01	0\\
520.01	0\\
521.01	0\\
522.01	0\\
523.01	1.73472347597681e-18\\
524.01	0\\
525.01	0\\
526.01	0\\
527.01	0\\
528.01	0\\
529.01	0\\
530.01	0\\
531.01	0\\
532.01	0\\
533.01	0\\
534.01	0\\
535.01	0\\
536.01	1.73472347597681e-18\\
537.01	1.73472347597681e-18\\
538.01	1.73472347597681e-18\\
539.01	0\\
540.01	0\\
541.01	0\\
542.01	0\\
543.01	0\\
544.01	0\\
545.01	1.73472347597681e-18\\
546.01	1.73472347597681e-18\\
547.01	0\\
548.01	0\\
549.01	0\\
550.01	0\\
551.01	0\\
552.01	0\\
553.01	0\\
554.01	0\\
555.01	0\\
556.01	1.73472347597681e-18\\
557.01	0\\
558.01	0\\
559.01	0\\
560.01	1.73472347597681e-18\\
561.01	1.73472347597681e-18\\
562.01	0\\
563.01	0\\
564.01	0\\
565.01	0\\
566.01	0\\
567.01	0\\
568.01	0\\
569.01	0\\
570.01	0\\
571.01	0\\
572.01	0\\
573.01	0\\
574.01	0\\
575.01	0\\
576.01	0\\
577.01	0\\
578.01	1.73472347597681e-18\\
579.01	0\\
580.01	0\\
581.01	0\\
582.01	0\\
583.01	1.73472347597681e-18\\
584.01	0\\
585.01	0\\
586.01	0\\
587.01	0\\
588.01	0\\
589.01	0\\
590.01	0\\
591.01	0\\
592.01	0\\
593.01	0\\
594.01	0\\
595.01	0\\
596.01	0\\
597.01	0\\
598.01	0\\
599.01	0\\
599.02	0\\
599.03	0\\
599.04	0\\
599.05	0\\
599.06	0\\
599.07	0\\
599.08	0\\
599.09	0\\
599.1	0\\
599.11	0\\
599.12	0\\
599.13	0\\
599.14	0\\
599.15	0\\
599.16	0\\
599.17	0\\
599.18	0\\
599.19	0\\
599.2	0\\
599.21	0\\
599.22	0\\
599.23	0\\
599.24	0\\
599.25	0\\
599.26	0\\
599.27	0\\
599.28	0\\
599.29	0\\
599.3	0\\
599.31	0\\
599.32	0\\
599.33	0\\
599.34	0\\
599.35	0\\
599.36	0\\
599.37	0\\
599.38	0\\
599.39	0\\
599.4	0\\
599.41	0\\
599.42	0\\
599.43	0\\
599.44	0\\
599.45	0\\
599.46	0\\
599.47	0\\
599.48	0\\
599.49	0\\
599.5	0\\
599.51	0\\
599.52	0\\
599.53	0\\
599.54	0\\
599.55	0\\
599.56	0\\
599.57	0\\
599.58	0\\
599.59	0\\
599.6	0\\
599.61	0\\
599.62	0\\
599.63	0\\
599.64	0\\
599.65	0\\
599.66	0\\
599.67	0\\
599.68	0\\
599.69	0\\
599.7	0\\
599.71	0\\
599.72	0\\
599.73	0\\
599.74	0\\
599.75	0\\
599.76	0\\
599.77	0\\
599.78	0\\
599.79	0\\
599.8	0\\
599.81	0\\
599.82	0\\
599.83	0\\
599.84	0\\
599.85	0\\
599.86	0\\
599.87	0\\
599.88	0\\
599.89	0\\
599.9	0\\
599.91	0\\
599.92	0\\
599.93	0\\
599.94	0\\
599.95	0\\
599.96	0\\
599.97	0\\
599.98	0\\
599.99	0\\
600	0\\
};
\addplot [color=mycolor1,solid,forget plot]
  table[row sep=crcr]{%
0.01	0\\
1.01	0\\
2.01	0\\
3.01	0\\
4.01	0\\
5.01	0\\
6.01	0\\
7.01	0\\
8.01	0\\
9.01	0\\
10.01	0\\
11.01	0\\
12.01	0\\
13.01	0\\
14.01	0\\
15.01	0\\
16.01	0\\
17.01	0\\
18.01	0\\
19.01	0\\
20.01	0\\
21.01	0\\
22.01	0\\
23.01	0\\
24.01	0\\
25.01	0\\
26.01	0\\
27.01	0\\
28.01	0\\
29.01	0\\
30.01	0\\
31.01	0\\
32.01	0\\
33.01	0\\
34.01	0\\
35.01	0\\
36.01	0\\
37.01	0\\
38.01	0\\
39.01	0\\
40.01	0\\
41.01	0\\
42.01	0\\
43.01	0\\
44.01	0\\
45.01	0\\
46.01	0\\
47.01	0\\
48.01	0\\
49.01	0\\
50.01	0\\
51.01	0\\
52.01	0\\
53.01	0\\
54.01	0\\
55.01	0\\
56.01	0\\
57.01	0\\
58.01	0\\
59.01	0\\
60.01	0\\
61.01	0\\
62.01	0\\
63.01	0\\
64.01	0\\
65.01	0\\
66.01	0\\
67.01	0\\
68.01	0\\
69.01	0\\
70.01	0\\
71.01	0\\
72.01	0\\
73.01	0\\
74.01	0\\
75.01	0\\
76.01	0\\
77.01	0\\
78.01	0\\
79.01	0\\
80.01	0\\
81.01	0\\
82.01	0\\
83.01	0\\
84.01	0\\
85.01	0\\
86.01	0\\
87.01	0\\
88.01	0\\
89.01	0\\
90.01	0\\
91.01	0\\
92.01	0\\
93.01	0\\
94.01	0\\
95.01	0\\
96.01	0\\
97.01	0\\
98.01	0\\
99.01	0\\
100.01	0\\
101.01	0\\
102.01	0\\
103.01	0\\
104.01	0\\
105.01	0\\
106.01	0\\
107.01	0\\
108.01	0\\
109.01	0\\
110.01	0\\
111.01	0\\
112.01	0\\
113.01	0\\
114.01	0\\
115.01	0\\
116.01	0\\
117.01	0\\
118.01	0\\
119.01	0\\
120.01	0\\
121.01	0\\
122.01	0\\
123.01	0\\
124.01	0\\
125.01	0\\
126.01	0\\
127.01	0\\
128.01	0\\
129.01	0\\
130.01	0\\
131.01	0\\
132.01	0\\
133.01	0\\
134.01	0\\
135.01	0\\
136.01	0\\
137.01	0\\
138.01	0\\
139.01	0\\
140.01	0\\
141.01	0\\
142.01	0\\
143.01	0\\
144.01	0\\
145.01	0\\
146.01	0\\
147.01	0\\
148.01	0\\
149.01	0\\
150.01	0\\
151.01	0\\
152.01	0\\
153.01	0\\
154.01	0\\
155.01	0\\
156.01	0\\
157.01	0\\
158.01	0\\
159.01	0\\
160.01	0\\
161.01	0\\
162.01	0\\
163.01	0\\
164.01	0\\
165.01	0\\
166.01	0\\
167.01	0\\
168.01	0\\
169.01	0\\
170.01	0\\
171.01	0\\
172.01	0\\
173.01	0\\
174.01	0\\
175.01	0\\
176.01	0\\
177.01	0\\
178.01	0\\
179.01	0\\
180.01	0\\
181.01	0\\
182.01	0\\
183.01	0\\
184.01	0\\
185.01	0\\
186.01	0\\
187.01	0\\
188.01	0\\
189.01	0\\
190.01	0\\
191.01	0\\
192.01	0\\
193.01	0\\
194.01	0\\
195.01	0\\
196.01	0\\
197.01	0\\
198.01	0\\
199.01	0\\
200.01	0\\
201.01	0\\
202.01	0\\
203.01	0\\
204.01	0\\
205.01	0\\
206.01	0\\
207.01	0\\
208.01	0\\
209.01	0\\
210.01	0\\
211.01	0\\
212.01	0\\
213.01	0\\
214.01	0\\
215.01	0\\
216.01	0\\
217.01	0\\
218.01	0\\
219.01	0\\
220.01	0\\
221.01	0\\
222.01	0\\
223.01	0\\
224.01	0\\
225.01	0\\
226.01	0\\
227.01	0\\
228.01	0\\
229.01	0\\
230.01	0\\
231.01	0\\
232.01	0\\
233.01	0\\
234.01	0\\
235.01	0\\
236.01	0\\
237.01	0\\
238.01	0\\
239.01	0\\
240.01	0\\
241.01	0\\
242.01	0\\
243.01	0\\
244.01	0\\
245.01	0\\
246.01	0\\
247.01	0\\
248.01	0\\
249.01	0\\
250.01	0\\
251.01	0\\
252.01	0\\
253.01	0\\
254.01	0\\
255.01	0\\
256.01	0\\
257.01	0\\
258.01	0\\
259.01	0\\
260.01	0\\
261.01	0\\
262.01	0\\
263.01	0\\
264.01	0\\
265.01	0\\
266.01	0\\
267.01	0\\
268.01	0\\
269.01	0\\
270.01	0\\
271.01	0\\
272.01	0\\
273.01	0\\
274.01	0\\
275.01	0\\
276.01	0\\
277.01	0\\
278.01	0\\
279.01	0\\
280.01	0\\
281.01	0\\
282.01	0\\
283.01	0\\
284.01	0\\
285.01	0\\
286.01	0\\
287.01	0\\
288.01	0\\
289.01	0\\
290.01	0\\
291.01	0\\
292.01	0\\
293.01	0\\
294.01	0\\
295.01	0\\
296.01	0\\
297.01	0\\
298.01	0\\
299.01	0\\
300.01	0\\
301.01	0\\
302.01	0\\
303.01	0\\
304.01	0\\
305.01	0\\
306.01	0\\
307.01	0\\
308.01	0\\
309.01	0\\
310.01	0\\
311.01	0\\
312.01	0\\
313.01	0\\
314.01	0\\
315.01	0\\
316.01	0\\
317.01	0\\
318.01	0\\
319.01	0\\
320.01	0\\
321.01	0\\
322.01	0\\
323.01	0\\
324.01	0\\
325.01	0\\
326.01	0\\
327.01	0\\
328.01	0\\
329.01	0\\
330.01	0\\
331.01	0\\
332.01	0\\
333.01	0\\
334.01	0\\
335.01	0\\
336.01	0\\
337.01	0\\
338.01	0\\
339.01	0\\
340.01	0\\
341.01	0\\
342.01	0\\
343.01	0\\
344.01	0\\
345.01	0\\
346.01	0\\
347.01	0\\
348.01	0\\
349.01	0\\
350.01	0\\
351.01	0\\
352.01	0\\
353.01	0\\
354.01	0\\
355.01	0\\
356.01	0\\
357.01	0\\
358.01	0\\
359.01	0\\
360.01	0\\
361.01	0\\
362.01	0\\
363.01	0\\
364.01	0\\
365.01	0\\
366.01	0\\
367.01	0\\
368.01	0\\
369.01	0\\
370.01	0\\
371.01	0\\
372.01	0\\
373.01	0\\
374.01	0\\
375.01	0\\
376.01	0\\
377.01	0\\
378.01	0\\
379.01	0\\
380.01	0\\
381.01	0\\
382.01	0\\
383.01	0\\
384.01	0\\
385.01	0\\
386.01	0\\
387.01	0\\
388.01	0\\
389.01	0\\
390.01	0\\
391.01	0\\
392.01	0\\
393.01	0\\
394.01	0\\
395.01	0\\
396.01	0\\
397.01	0\\
398.01	0\\
399.01	0\\
400.01	0\\
401.01	0\\
402.01	0\\
403.01	0\\
404.01	0\\
405.01	0\\
406.01	0\\
407.01	0\\
408.01	0\\
409.01	0\\
410.01	0\\
411.01	0\\
412.01	0\\
413.01	0\\
414.01	0\\
415.01	0\\
416.01	0\\
417.01	0\\
418.01	0\\
419.01	0\\
420.01	0\\
421.01	0\\
422.01	0\\
423.01	0\\
424.01	0\\
425.01	0\\
426.01	0\\
427.01	0\\
428.01	0\\
429.01	0\\
430.01	0\\
431.01	0\\
432.01	0\\
433.01	0\\
434.01	0\\
435.01	0\\
436.01	0\\
437.01	0\\
438.01	0\\
439.01	0\\
440.01	0\\
441.01	0\\
442.01	0\\
443.01	0\\
444.01	0\\
445.01	0\\
446.01	0\\
447.01	0\\
448.01	0\\
449.01	0\\
450.01	0\\
451.01	1.73472347597681e-18\\
452.01	0\\
453.01	0\\
454.01	0\\
455.01	0\\
456.01	0\\
457.01	0\\
458.01	0\\
459.01	0\\
460.01	0\\
461.01	0\\
462.01	0\\
463.01	0\\
464.01	0\\
465.01	0\\
466.01	0\\
467.01	0\\
468.01	0\\
469.01	1.73472347597681e-18\\
470.01	0\\
471.01	0\\
472.01	0\\
473.01	0\\
474.01	0\\
475.01	0\\
476.01	1.73472347597681e-18\\
477.01	0\\
478.01	0\\
479.01	0\\
480.01	0\\
481.01	0\\
482.01	0\\
483.01	0\\
484.01	0\\
485.01	0\\
486.01	0\\
487.01	0\\
488.01	0\\
489.01	0\\
490.01	0\\
491.01	0\\
492.01	0\\
493.01	0\\
494.01	0\\
495.01	0\\
496.01	0\\
497.01	0\\
498.01	0\\
499.01	0\\
500.01	0\\
501.01	0\\
502.01	0\\
503.01	0\\
504.01	0\\
505.01	0\\
506.01	0\\
507.01	0\\
508.01	0\\
509.01	0\\
510.01	0\\
511.01	0\\
512.01	1.73472347597681e-18\\
513.01	0\\
514.01	0\\
515.01	0\\
516.01	0\\
517.01	0\\
518.01	0\\
519.01	0\\
520.01	0\\
521.01	0\\
522.01	0\\
523.01	1.73472347597681e-18\\
524.01	0\\
525.01	0\\
526.01	0\\
527.01	0\\
528.01	0\\
529.01	0\\
530.01	0\\
531.01	0\\
532.01	0\\
533.01	0\\
534.01	0\\
535.01	0\\
536.01	1.73472347597681e-18\\
537.01	1.73472347597681e-18\\
538.01	1.73472347597681e-18\\
539.01	0\\
540.01	0\\
541.01	0\\
542.01	0\\
543.01	0\\
544.01	0\\
545.01	1.73472347597681e-18\\
546.01	1.73472347597681e-18\\
547.01	0\\
548.01	0\\
549.01	0\\
550.01	0\\
551.01	0\\
552.01	0\\
553.01	0\\
554.01	0\\
555.01	0\\
556.01	1.73472347597681e-18\\
557.01	0\\
558.01	0\\
559.01	0\\
560.01	1.73472347597681e-18\\
561.01	1.73472347597681e-18\\
562.01	0\\
563.01	0\\
564.01	0\\
565.01	0\\
566.01	0\\
567.01	0\\
568.01	0\\
569.01	0\\
570.01	0\\
571.01	0\\
572.01	0\\
573.01	0\\
574.01	0\\
575.01	0\\
576.01	0\\
577.01	0\\
578.01	1.73472347597681e-18\\
579.01	0\\
580.01	0\\
581.01	0\\
582.01	0\\
583.01	1.73472347597681e-18\\
584.01	0\\
585.01	0\\
586.01	0\\
587.01	0\\
588.01	0\\
589.01	0\\
590.01	0\\
591.01	0\\
592.01	0\\
593.01	0\\
594.01	0\\
595.01	0\\
596.01	0\\
597.01	0\\
598.01	0\\
599.01	0\\
599.02	0\\
599.03	0\\
599.04	0\\
599.05	0\\
599.06	0\\
599.07	0\\
599.08	0\\
599.09	0\\
599.1	0\\
599.11	0\\
599.12	0\\
599.13	0\\
599.14	0\\
599.15	0\\
599.16	0\\
599.17	0\\
599.18	0\\
599.19	0\\
599.2	0\\
599.21	0\\
599.22	0\\
599.23	0\\
599.24	0\\
599.25	0\\
599.26	0\\
599.27	0\\
599.28	0\\
599.29	0\\
599.3	0\\
599.31	0\\
599.32	0\\
599.33	0\\
599.34	0\\
599.35	0\\
599.36	0\\
599.37	0\\
599.38	0\\
599.39	0\\
599.4	0\\
599.41	0\\
599.42	0\\
599.43	0\\
599.44	0\\
599.45	0\\
599.46	0\\
599.47	0\\
599.48	0\\
599.49	0\\
599.5	0\\
599.51	0\\
599.52	0\\
599.53	0\\
599.54	0\\
599.55	0\\
599.56	0\\
599.57	0\\
599.58	0\\
599.59	0\\
599.6	0\\
599.61	0\\
599.62	0\\
599.63	0\\
599.64	0\\
599.65	0\\
599.66	0\\
599.67	0\\
599.68	0\\
599.69	0\\
599.7	0\\
599.71	0\\
599.72	0\\
599.73	0\\
599.74	0\\
599.75	0\\
599.76	0\\
599.77	0\\
599.78	0\\
599.79	0\\
599.8	0\\
599.81	0\\
599.82	0\\
599.83	0\\
599.84	0\\
599.85	0\\
599.86	0\\
599.87	0\\
599.88	0\\
599.89	0\\
599.9	0\\
599.91	0\\
599.92	0\\
599.93	0\\
599.94	0\\
599.95	0\\
599.96	0\\
599.97	0\\
599.98	0\\
599.99	0\\
600	0\\
};
\addplot [color=mycolor2,solid,forget plot]
  table[row sep=crcr]{%
0.01	0\\
1.01	0\\
2.01	0\\
3.01	0\\
4.01	0\\
5.01	0\\
6.01	0\\
7.01	0\\
8.01	0\\
9.01	0\\
10.01	0\\
11.01	0\\
12.01	0\\
13.01	0\\
14.01	0\\
15.01	0\\
16.01	0\\
17.01	0\\
18.01	0\\
19.01	0\\
20.01	0\\
21.01	0\\
22.01	0\\
23.01	0\\
24.01	0\\
25.01	0\\
26.01	0\\
27.01	0\\
28.01	0\\
29.01	0\\
30.01	0\\
31.01	0\\
32.01	0\\
33.01	0\\
34.01	0\\
35.01	0\\
36.01	0\\
37.01	0\\
38.01	0\\
39.01	0\\
40.01	0\\
41.01	0\\
42.01	0\\
43.01	0\\
44.01	0\\
45.01	0\\
46.01	0\\
47.01	0\\
48.01	0\\
49.01	0\\
50.01	0\\
51.01	0\\
52.01	0\\
53.01	0\\
54.01	0\\
55.01	0\\
56.01	0\\
57.01	0\\
58.01	0\\
59.01	0\\
60.01	0\\
61.01	0\\
62.01	0\\
63.01	0\\
64.01	0\\
65.01	0\\
66.01	0\\
67.01	0\\
68.01	0\\
69.01	0\\
70.01	0\\
71.01	0\\
72.01	0\\
73.01	0\\
74.01	0\\
75.01	0\\
76.01	0\\
77.01	0\\
78.01	0\\
79.01	0\\
80.01	0\\
81.01	0\\
82.01	0\\
83.01	0\\
84.01	0\\
85.01	0\\
86.01	0\\
87.01	0\\
88.01	0\\
89.01	0\\
90.01	0\\
91.01	0\\
92.01	0\\
93.01	0\\
94.01	0\\
95.01	0\\
96.01	0\\
97.01	0\\
98.01	0\\
99.01	0\\
100.01	0\\
101.01	0\\
102.01	0\\
103.01	0\\
104.01	0\\
105.01	0\\
106.01	0\\
107.01	0\\
108.01	0\\
109.01	0\\
110.01	0\\
111.01	0\\
112.01	0\\
113.01	0\\
114.01	0\\
115.01	0\\
116.01	0\\
117.01	0\\
118.01	0\\
119.01	0\\
120.01	0\\
121.01	0\\
122.01	0\\
123.01	0\\
124.01	0\\
125.01	0\\
126.01	0\\
127.01	0\\
128.01	0\\
129.01	0\\
130.01	0\\
131.01	0\\
132.01	0\\
133.01	0\\
134.01	0\\
135.01	0\\
136.01	0\\
137.01	0\\
138.01	0\\
139.01	0\\
140.01	0\\
141.01	0\\
142.01	0\\
143.01	0\\
144.01	0\\
145.01	0\\
146.01	0\\
147.01	0\\
148.01	0\\
149.01	0\\
150.01	0\\
151.01	0\\
152.01	0\\
153.01	0\\
154.01	0\\
155.01	0\\
156.01	0\\
157.01	0\\
158.01	0\\
159.01	0\\
160.01	0\\
161.01	0\\
162.01	0\\
163.01	0\\
164.01	0\\
165.01	0\\
166.01	0\\
167.01	0\\
168.01	0\\
169.01	0\\
170.01	0\\
171.01	0\\
172.01	0\\
173.01	0\\
174.01	0\\
175.01	0\\
176.01	0\\
177.01	0\\
178.01	0\\
179.01	0\\
180.01	0\\
181.01	0\\
182.01	0\\
183.01	0\\
184.01	0\\
185.01	0\\
186.01	0\\
187.01	0\\
188.01	0\\
189.01	0\\
190.01	0\\
191.01	0\\
192.01	0\\
193.01	0\\
194.01	0\\
195.01	0\\
196.01	0\\
197.01	0\\
198.01	0\\
199.01	0\\
200.01	0\\
201.01	0\\
202.01	0\\
203.01	0\\
204.01	0\\
205.01	0\\
206.01	0\\
207.01	0\\
208.01	0\\
209.01	0\\
210.01	0\\
211.01	0\\
212.01	0\\
213.01	0\\
214.01	0\\
215.01	0\\
216.01	0\\
217.01	0\\
218.01	0\\
219.01	0\\
220.01	0\\
221.01	0\\
222.01	0\\
223.01	0\\
224.01	0\\
225.01	0\\
226.01	0\\
227.01	0\\
228.01	0\\
229.01	0\\
230.01	0\\
231.01	0\\
232.01	0\\
233.01	0\\
234.01	0\\
235.01	0\\
236.01	0\\
237.01	0\\
238.01	0\\
239.01	0\\
240.01	0\\
241.01	0\\
242.01	0\\
243.01	0\\
244.01	0\\
245.01	0\\
246.01	0\\
247.01	0\\
248.01	0\\
249.01	0\\
250.01	0\\
251.01	0\\
252.01	0\\
253.01	0\\
254.01	0\\
255.01	0\\
256.01	0\\
257.01	0\\
258.01	0\\
259.01	0\\
260.01	0\\
261.01	0\\
262.01	0\\
263.01	0\\
264.01	0\\
265.01	0\\
266.01	0\\
267.01	0\\
268.01	0\\
269.01	0\\
270.01	0\\
271.01	0\\
272.01	0\\
273.01	0\\
274.01	0\\
275.01	0\\
276.01	0\\
277.01	0\\
278.01	0\\
279.01	0\\
280.01	0\\
281.01	0\\
282.01	0\\
283.01	0\\
284.01	0\\
285.01	0\\
286.01	0\\
287.01	0\\
288.01	0\\
289.01	0\\
290.01	0\\
291.01	0\\
292.01	0\\
293.01	0\\
294.01	0\\
295.01	0\\
296.01	0\\
297.01	0\\
298.01	0\\
299.01	0\\
300.01	0\\
301.01	0\\
302.01	0\\
303.01	0\\
304.01	0\\
305.01	0\\
306.01	0\\
307.01	0\\
308.01	0\\
309.01	0\\
310.01	0\\
311.01	0\\
312.01	0\\
313.01	0\\
314.01	0\\
315.01	0\\
316.01	0\\
317.01	0\\
318.01	0\\
319.01	0\\
320.01	0\\
321.01	0\\
322.01	0\\
323.01	0\\
324.01	0\\
325.01	0\\
326.01	0\\
327.01	0\\
328.01	0\\
329.01	0\\
330.01	0\\
331.01	0\\
332.01	0\\
333.01	0\\
334.01	0\\
335.01	0\\
336.01	0\\
337.01	0\\
338.01	0\\
339.01	0\\
340.01	0\\
341.01	0\\
342.01	0\\
343.01	0\\
344.01	0\\
345.01	0\\
346.01	0\\
347.01	0\\
348.01	0\\
349.01	0\\
350.01	0\\
351.01	0\\
352.01	0\\
353.01	0\\
354.01	0\\
355.01	0\\
356.01	0\\
357.01	0\\
358.01	0\\
359.01	0\\
360.01	0\\
361.01	0\\
362.01	0\\
363.01	0\\
364.01	0\\
365.01	0\\
366.01	0\\
367.01	0\\
368.01	0\\
369.01	0\\
370.01	0\\
371.01	0\\
372.01	0\\
373.01	0\\
374.01	0\\
375.01	0\\
376.01	0\\
377.01	0\\
378.01	0\\
379.01	0\\
380.01	0\\
381.01	0\\
382.01	0\\
383.01	0\\
384.01	0\\
385.01	0\\
386.01	0\\
387.01	0\\
388.01	0\\
389.01	0\\
390.01	0\\
391.01	0\\
392.01	0\\
393.01	0\\
394.01	0\\
395.01	0\\
396.01	0\\
397.01	0\\
398.01	0\\
399.01	0\\
400.01	0\\
401.01	0\\
402.01	0\\
403.01	0\\
404.01	0\\
405.01	0\\
406.01	0\\
407.01	0\\
408.01	0\\
409.01	0\\
410.01	0\\
411.01	0\\
412.01	0\\
413.01	0\\
414.01	0\\
415.01	0\\
416.01	0\\
417.01	0\\
418.01	0\\
419.01	0\\
420.01	0\\
421.01	0\\
422.01	0\\
423.01	0\\
424.01	0\\
425.01	0\\
426.01	0\\
427.01	0\\
428.01	0\\
429.01	0\\
430.01	0\\
431.01	0\\
432.01	0\\
433.01	0\\
434.01	0\\
435.01	0\\
436.01	0\\
437.01	0\\
438.01	0\\
439.01	0\\
440.01	0\\
441.01	0\\
442.01	0\\
443.01	0\\
444.01	0\\
445.01	0\\
446.01	0\\
447.01	0\\
448.01	0\\
449.01	0\\
450.01	0\\
451.01	1.73472347597681e-18\\
452.01	0\\
453.01	0\\
454.01	0\\
455.01	0\\
456.01	0\\
457.01	0\\
458.01	0\\
459.01	0\\
460.01	0\\
461.01	0\\
462.01	0\\
463.01	0\\
464.01	0\\
465.01	0\\
466.01	0\\
467.01	0\\
468.01	0\\
469.01	1.73472347597681e-18\\
470.01	0\\
471.01	0\\
472.01	0\\
473.01	0\\
474.01	0\\
475.01	0\\
476.01	1.73472347597681e-18\\
477.01	0\\
478.01	0\\
479.01	0\\
480.01	0\\
481.01	0\\
482.01	0\\
483.01	0\\
484.01	0\\
485.01	0\\
486.01	0\\
487.01	0\\
488.01	0\\
489.01	0\\
490.01	0\\
491.01	0\\
492.01	0\\
493.01	0\\
494.01	0\\
495.01	0\\
496.01	0\\
497.01	0\\
498.01	0\\
499.01	0\\
500.01	0\\
501.01	0\\
502.01	0\\
503.01	0\\
504.01	0\\
505.01	0\\
506.01	0\\
507.01	0\\
508.01	0\\
509.01	0\\
510.01	0\\
511.01	0\\
512.01	1.73472347597681e-18\\
513.01	0\\
514.01	0\\
515.01	0\\
516.01	0\\
517.01	0\\
518.01	0\\
519.01	0\\
520.01	0\\
521.01	0\\
522.01	0\\
523.01	1.73472347597681e-18\\
524.01	0\\
525.01	0\\
526.01	0\\
527.01	0\\
528.01	0\\
529.01	0\\
530.01	0\\
531.01	0\\
532.01	0\\
533.01	0\\
534.01	0\\
535.01	0\\
536.01	1.73472347597681e-18\\
537.01	1.73472347597681e-18\\
538.01	1.73472347597681e-18\\
539.01	0\\
540.01	0\\
541.01	0\\
542.01	0\\
543.01	0\\
544.01	0\\
545.01	1.73472347597681e-18\\
546.01	1.73472347597681e-18\\
547.01	0\\
548.01	0\\
549.01	0\\
550.01	0\\
551.01	0\\
552.01	0\\
553.01	0\\
554.01	0\\
555.01	0\\
556.01	1.73472347597681e-18\\
557.01	0\\
558.01	0\\
559.01	0\\
560.01	1.73472347597681e-18\\
561.01	1.73472347597681e-18\\
562.01	0\\
563.01	0\\
564.01	0\\
565.01	0\\
566.01	0\\
567.01	0\\
568.01	0\\
569.01	0\\
570.01	0\\
571.01	0\\
572.01	0\\
573.01	0\\
574.01	0\\
575.01	0\\
576.01	0\\
577.01	0\\
578.01	1.73472347597681e-18\\
579.01	0\\
580.01	0\\
581.01	0\\
582.01	0\\
583.01	1.73472347597681e-18\\
584.01	0\\
585.01	0\\
586.01	0\\
587.01	0\\
588.01	0\\
589.01	0\\
590.01	0\\
591.01	0\\
592.01	0\\
593.01	0\\
594.01	0\\
595.01	0\\
596.01	0\\
597.01	0\\
598.01	0\\
599.01	0\\
599.02	0\\
599.03	0\\
599.04	0\\
599.05	0\\
599.06	0\\
599.07	0\\
599.08	0\\
599.09	0\\
599.1	0\\
599.11	0\\
599.12	0\\
599.13	0\\
599.14	0\\
599.15	0\\
599.16	0\\
599.17	0\\
599.18	0\\
599.19	0\\
599.2	0\\
599.21	0\\
599.22	0\\
599.23	0\\
599.24	0\\
599.25	0\\
599.26	0\\
599.27	0\\
599.28	0\\
599.29	0\\
599.3	0\\
599.31	0\\
599.32	0\\
599.33	0\\
599.34	0\\
599.35	0\\
599.36	0\\
599.37	0\\
599.38	0\\
599.39	0\\
599.4	0\\
599.41	0\\
599.42	0\\
599.43	0\\
599.44	0\\
599.45	0\\
599.46	0\\
599.47	0\\
599.48	0\\
599.49	0\\
599.5	0\\
599.51	0\\
599.52	0\\
599.53	0\\
599.54	0\\
599.55	0\\
599.56	0\\
599.57	0\\
599.58	0\\
599.59	0\\
599.6	0\\
599.61	0\\
599.62	0\\
599.63	0\\
599.64	0\\
599.65	0\\
599.66	0\\
599.67	0\\
599.68	0\\
599.69	0\\
599.7	0\\
599.71	0\\
599.72	0\\
599.73	0\\
599.74	0\\
599.75	0\\
599.76	0\\
599.77	0\\
599.78	0\\
599.79	0\\
599.8	0\\
599.81	0\\
599.82	0\\
599.83	0\\
599.84	0\\
599.85	0\\
599.86	0\\
599.87	0\\
599.88	0\\
599.89	0\\
599.9	0\\
599.91	0\\
599.92	0\\
599.93	0\\
599.94	0\\
599.95	0\\
599.96	0\\
599.97	0\\
599.98	0\\
599.99	0\\
600	0\\
};
\addplot [color=mycolor3,solid,forget plot]
  table[row sep=crcr]{%
0.01	0\\
1.01	0\\
2.01	0\\
3.01	0\\
4.01	0\\
5.01	0\\
6.01	0\\
7.01	0\\
8.01	0\\
9.01	0\\
10.01	0\\
11.01	0\\
12.01	0\\
13.01	0\\
14.01	0\\
15.01	0\\
16.01	0\\
17.01	0\\
18.01	0\\
19.01	0\\
20.01	0\\
21.01	0\\
22.01	0\\
23.01	0\\
24.01	0\\
25.01	0\\
26.01	0\\
27.01	0\\
28.01	0\\
29.01	0\\
30.01	0\\
31.01	0\\
32.01	0\\
33.01	0\\
34.01	0\\
35.01	0\\
36.01	0\\
37.01	0\\
38.01	0\\
39.01	0\\
40.01	0\\
41.01	0\\
42.01	0\\
43.01	0\\
44.01	0\\
45.01	0\\
46.01	0\\
47.01	0\\
48.01	0\\
49.01	0\\
50.01	0\\
51.01	0\\
52.01	0\\
53.01	0\\
54.01	0\\
55.01	0\\
56.01	0\\
57.01	0\\
58.01	0\\
59.01	0\\
60.01	0\\
61.01	0\\
62.01	0\\
63.01	0\\
64.01	0\\
65.01	0\\
66.01	0\\
67.01	0\\
68.01	0\\
69.01	0\\
70.01	0\\
71.01	0\\
72.01	0\\
73.01	0\\
74.01	0\\
75.01	0\\
76.01	0\\
77.01	0\\
78.01	0\\
79.01	0\\
80.01	0\\
81.01	0\\
82.01	0\\
83.01	0\\
84.01	0\\
85.01	0\\
86.01	0\\
87.01	0\\
88.01	0\\
89.01	0\\
90.01	0\\
91.01	0\\
92.01	0\\
93.01	0\\
94.01	0\\
95.01	0\\
96.01	0\\
97.01	0\\
98.01	0\\
99.01	0\\
100.01	0\\
101.01	0\\
102.01	0\\
103.01	0\\
104.01	0\\
105.01	0\\
106.01	0\\
107.01	0\\
108.01	0\\
109.01	0\\
110.01	0\\
111.01	0\\
112.01	0\\
113.01	0\\
114.01	0\\
115.01	0\\
116.01	0\\
117.01	0\\
118.01	0\\
119.01	0\\
120.01	0\\
121.01	0\\
122.01	0\\
123.01	0\\
124.01	0\\
125.01	0\\
126.01	0\\
127.01	0\\
128.01	0\\
129.01	0\\
130.01	0\\
131.01	0\\
132.01	0\\
133.01	0\\
134.01	0\\
135.01	0\\
136.01	0\\
137.01	0\\
138.01	0\\
139.01	0\\
140.01	0\\
141.01	0\\
142.01	0\\
143.01	0\\
144.01	0\\
145.01	0\\
146.01	0\\
147.01	0\\
148.01	0\\
149.01	0\\
150.01	0\\
151.01	0\\
152.01	0\\
153.01	0\\
154.01	0\\
155.01	0\\
156.01	0\\
157.01	0\\
158.01	0\\
159.01	0\\
160.01	0\\
161.01	0\\
162.01	0\\
163.01	0\\
164.01	0\\
165.01	0\\
166.01	0\\
167.01	0\\
168.01	0\\
169.01	0\\
170.01	0\\
171.01	0\\
172.01	0\\
173.01	0\\
174.01	0\\
175.01	0\\
176.01	0\\
177.01	0\\
178.01	0\\
179.01	0\\
180.01	0\\
181.01	0\\
182.01	0\\
183.01	0\\
184.01	0\\
185.01	0\\
186.01	0\\
187.01	0\\
188.01	0\\
189.01	0\\
190.01	0\\
191.01	0\\
192.01	0\\
193.01	0\\
194.01	0\\
195.01	0\\
196.01	0\\
197.01	0\\
198.01	0\\
199.01	0\\
200.01	0\\
201.01	0\\
202.01	0\\
203.01	0\\
204.01	0\\
205.01	0\\
206.01	0\\
207.01	0\\
208.01	0\\
209.01	0\\
210.01	0\\
211.01	0\\
212.01	0\\
213.01	0\\
214.01	0\\
215.01	0\\
216.01	0\\
217.01	0\\
218.01	0\\
219.01	0\\
220.01	0\\
221.01	0\\
222.01	0\\
223.01	0\\
224.01	0\\
225.01	0\\
226.01	0\\
227.01	0\\
228.01	0\\
229.01	0\\
230.01	0\\
231.01	0\\
232.01	0\\
233.01	0\\
234.01	0\\
235.01	0\\
236.01	0\\
237.01	0\\
238.01	0\\
239.01	0\\
240.01	0\\
241.01	0\\
242.01	0\\
243.01	0\\
244.01	0\\
245.01	0\\
246.01	0\\
247.01	0\\
248.01	0\\
249.01	0\\
250.01	0\\
251.01	0\\
252.01	0\\
253.01	0\\
254.01	0\\
255.01	0\\
256.01	0\\
257.01	0\\
258.01	0\\
259.01	0\\
260.01	0\\
261.01	0\\
262.01	0\\
263.01	0\\
264.01	0\\
265.01	0\\
266.01	0\\
267.01	0\\
268.01	0\\
269.01	0\\
270.01	0\\
271.01	0\\
272.01	0\\
273.01	0\\
274.01	0\\
275.01	0\\
276.01	0\\
277.01	0\\
278.01	0\\
279.01	0\\
280.01	0\\
281.01	0\\
282.01	0\\
283.01	0\\
284.01	0\\
285.01	0\\
286.01	0\\
287.01	0\\
288.01	0\\
289.01	0\\
290.01	0\\
291.01	0\\
292.01	0\\
293.01	0\\
294.01	0\\
295.01	0\\
296.01	0\\
297.01	0\\
298.01	0\\
299.01	0\\
300.01	0\\
301.01	0\\
302.01	0\\
303.01	0\\
304.01	0\\
305.01	0\\
306.01	0\\
307.01	0\\
308.01	0\\
309.01	0\\
310.01	0\\
311.01	0\\
312.01	0\\
313.01	0\\
314.01	0\\
315.01	0\\
316.01	0\\
317.01	0\\
318.01	0\\
319.01	0\\
320.01	0\\
321.01	0\\
322.01	0\\
323.01	0\\
324.01	0\\
325.01	0\\
326.01	0\\
327.01	0\\
328.01	0\\
329.01	0\\
330.01	0\\
331.01	0\\
332.01	0\\
333.01	0\\
334.01	0\\
335.01	0\\
336.01	0\\
337.01	0\\
338.01	0\\
339.01	0\\
340.01	0\\
341.01	0\\
342.01	0\\
343.01	0\\
344.01	0\\
345.01	0\\
346.01	0\\
347.01	0\\
348.01	0\\
349.01	0\\
350.01	0\\
351.01	0\\
352.01	0\\
353.01	0\\
354.01	0\\
355.01	0\\
356.01	0\\
357.01	0\\
358.01	0\\
359.01	0\\
360.01	0\\
361.01	0\\
362.01	0\\
363.01	0\\
364.01	0\\
365.01	0\\
366.01	0\\
367.01	0\\
368.01	0\\
369.01	0\\
370.01	0\\
371.01	0\\
372.01	0\\
373.01	0\\
374.01	0\\
375.01	0\\
376.01	0\\
377.01	0\\
378.01	0\\
379.01	0\\
380.01	0\\
381.01	0\\
382.01	0\\
383.01	0\\
384.01	0\\
385.01	0\\
386.01	0\\
387.01	0\\
388.01	0\\
389.01	0\\
390.01	0\\
391.01	0\\
392.01	0\\
393.01	0\\
394.01	0\\
395.01	0\\
396.01	0\\
397.01	0\\
398.01	0\\
399.01	0\\
400.01	0\\
401.01	0\\
402.01	0\\
403.01	0\\
404.01	0\\
405.01	0\\
406.01	0\\
407.01	0\\
408.01	0\\
409.01	0\\
410.01	0\\
411.01	0\\
412.01	0\\
413.01	0\\
414.01	0\\
415.01	0\\
416.01	0\\
417.01	0\\
418.01	0\\
419.01	0\\
420.01	0\\
421.01	0\\
422.01	0\\
423.01	0\\
424.01	0\\
425.01	0\\
426.01	0\\
427.01	0\\
428.01	0\\
429.01	0\\
430.01	0\\
431.01	0\\
432.01	0\\
433.01	0\\
434.01	0\\
435.01	0\\
436.01	0\\
437.01	0\\
438.01	0\\
439.01	0\\
440.01	0\\
441.01	0\\
442.01	0\\
443.01	0\\
444.01	0\\
445.01	0\\
446.01	0\\
447.01	0\\
448.01	0\\
449.01	0\\
450.01	0\\
451.01	1.73472347597681e-18\\
452.01	0\\
453.01	0\\
454.01	0\\
455.01	0\\
456.01	0\\
457.01	0\\
458.01	0\\
459.01	0\\
460.01	0\\
461.01	0\\
462.01	0\\
463.01	0\\
464.01	0\\
465.01	0\\
466.01	0\\
467.01	0\\
468.01	0\\
469.01	1.73472347597681e-18\\
470.01	0\\
471.01	0\\
472.01	0\\
473.01	0\\
474.01	0\\
475.01	0\\
476.01	1.73472347597681e-18\\
477.01	0\\
478.01	0\\
479.01	0\\
480.01	0\\
481.01	0\\
482.01	0\\
483.01	0\\
484.01	0\\
485.01	0\\
486.01	0\\
487.01	0\\
488.01	0\\
489.01	0\\
490.01	0\\
491.01	0\\
492.01	0\\
493.01	0\\
494.01	0\\
495.01	0\\
496.01	0\\
497.01	0\\
498.01	0\\
499.01	0\\
500.01	0\\
501.01	0\\
502.01	0\\
503.01	0\\
504.01	0\\
505.01	0\\
506.01	0\\
507.01	0\\
508.01	0\\
509.01	0\\
510.01	0\\
511.01	0\\
512.01	1.73472347597681e-18\\
513.01	0\\
514.01	0\\
515.01	0\\
516.01	0\\
517.01	0\\
518.01	0\\
519.01	0\\
520.01	0\\
521.01	0\\
522.01	0\\
523.01	1.73472347597681e-18\\
524.01	0\\
525.01	0\\
526.01	0\\
527.01	0\\
528.01	0\\
529.01	0\\
530.01	0\\
531.01	0\\
532.01	0\\
533.01	0\\
534.01	0\\
535.01	0\\
536.01	1.73472347597681e-18\\
537.01	1.73472347597681e-18\\
538.01	1.73472347597681e-18\\
539.01	0\\
540.01	0\\
541.01	0\\
542.01	0\\
543.01	0\\
544.01	0\\
545.01	1.73472347597681e-18\\
546.01	1.73472347597681e-18\\
547.01	0\\
548.01	0\\
549.01	0\\
550.01	0\\
551.01	0\\
552.01	0\\
553.01	0\\
554.01	0\\
555.01	0\\
556.01	1.73472347597681e-18\\
557.01	0\\
558.01	0\\
559.01	0\\
560.01	1.73472347597681e-18\\
561.01	1.73472347597681e-18\\
562.01	0\\
563.01	0\\
564.01	0\\
565.01	0\\
566.01	0\\
567.01	0\\
568.01	0\\
569.01	0\\
570.01	0\\
571.01	0\\
572.01	0\\
573.01	0\\
574.01	0\\
575.01	0\\
576.01	0\\
577.01	0\\
578.01	1.73472347597681e-18\\
579.01	0\\
580.01	0\\
581.01	0\\
582.01	0\\
583.01	1.73472347597681e-18\\
584.01	0\\
585.01	0\\
586.01	0\\
587.01	0\\
588.01	0\\
589.01	0\\
590.01	0\\
591.01	0\\
592.01	0\\
593.01	0\\
594.01	0\\
595.01	0\\
596.01	0\\
597.01	0\\
598.01	0\\
599.01	0\\
599.02	0\\
599.03	0\\
599.04	0\\
599.05	0\\
599.06	0\\
599.07	0\\
599.08	0\\
599.09	0\\
599.1	0\\
599.11	0\\
599.12	0\\
599.13	0\\
599.14	0\\
599.15	0\\
599.16	0\\
599.17	0\\
599.18	0\\
599.19	0\\
599.2	0\\
599.21	0\\
599.22	0\\
599.23	0\\
599.24	0\\
599.25	0\\
599.26	0\\
599.27	0\\
599.28	0\\
599.29	0\\
599.3	0\\
599.31	0\\
599.32	0\\
599.33	0\\
599.34	0\\
599.35	0\\
599.36	0\\
599.37	0\\
599.38	0\\
599.39	0\\
599.4	0\\
599.41	0\\
599.42	0\\
599.43	0\\
599.44	0\\
599.45	0\\
599.46	0\\
599.47	0\\
599.48	0\\
599.49	0\\
599.5	0\\
599.51	0\\
599.52	0\\
599.53	0\\
599.54	0\\
599.55	0\\
599.56	0\\
599.57	0\\
599.58	0\\
599.59	0\\
599.6	0\\
599.61	0\\
599.62	0\\
599.63	0\\
599.64	0\\
599.65	0\\
599.66	0\\
599.67	0\\
599.68	0\\
599.69	0\\
599.7	0\\
599.71	0\\
599.72	0\\
599.73	0\\
599.74	0\\
599.75	0\\
599.76	0\\
599.77	0\\
599.78	0\\
599.79	0\\
599.8	0\\
599.81	0\\
599.82	0\\
599.83	0\\
599.84	0\\
599.85	0\\
599.86	0\\
599.87	0\\
599.88	0\\
599.89	0\\
599.9	0\\
599.91	0\\
599.92	0\\
599.93	0\\
599.94	0\\
599.95	0\\
599.96	0\\
599.97	0\\
599.98	0\\
599.99	0\\
600	0\\
};
\addplot [color=mycolor4,solid,forget plot]
  table[row sep=crcr]{%
0.01	0\\
1.01	0\\
2.01	0\\
3.01	0\\
4.01	0\\
5.01	0\\
6.01	0\\
7.01	0\\
8.01	0\\
9.01	0\\
10.01	0\\
11.01	0\\
12.01	0\\
13.01	0\\
14.01	0\\
15.01	0\\
16.01	0\\
17.01	0\\
18.01	0\\
19.01	0\\
20.01	0\\
21.01	0\\
22.01	0\\
23.01	0\\
24.01	0\\
25.01	0\\
26.01	0\\
27.01	0\\
28.01	0\\
29.01	0\\
30.01	0\\
31.01	0\\
32.01	0\\
33.01	0\\
34.01	0\\
35.01	0\\
36.01	0\\
37.01	0\\
38.01	0\\
39.01	0\\
40.01	0\\
41.01	0\\
42.01	0\\
43.01	0\\
44.01	0\\
45.01	0\\
46.01	0\\
47.01	0\\
48.01	0\\
49.01	0\\
50.01	0\\
51.01	0\\
52.01	0\\
53.01	0\\
54.01	0\\
55.01	0\\
56.01	0\\
57.01	0\\
58.01	0\\
59.01	0\\
60.01	0\\
61.01	0\\
62.01	0\\
63.01	0\\
64.01	0\\
65.01	0\\
66.01	0\\
67.01	0\\
68.01	0\\
69.01	0\\
70.01	0\\
71.01	0\\
72.01	0\\
73.01	0\\
74.01	0\\
75.01	0\\
76.01	0\\
77.01	0\\
78.01	0\\
79.01	0\\
80.01	0\\
81.01	0\\
82.01	0\\
83.01	0\\
84.01	0\\
85.01	0\\
86.01	0\\
87.01	0\\
88.01	0\\
89.01	0\\
90.01	0\\
91.01	0\\
92.01	0\\
93.01	0\\
94.01	0\\
95.01	0\\
96.01	0\\
97.01	0\\
98.01	0\\
99.01	0\\
100.01	0\\
101.01	0\\
102.01	0\\
103.01	0\\
104.01	0\\
105.01	0\\
106.01	0\\
107.01	0\\
108.01	0\\
109.01	0\\
110.01	0\\
111.01	0\\
112.01	0\\
113.01	0\\
114.01	0\\
115.01	0\\
116.01	0\\
117.01	0\\
118.01	0\\
119.01	0\\
120.01	0\\
121.01	0\\
122.01	0\\
123.01	0\\
124.01	0\\
125.01	0\\
126.01	0\\
127.01	0\\
128.01	0\\
129.01	0\\
130.01	0\\
131.01	0\\
132.01	0\\
133.01	0\\
134.01	0\\
135.01	0\\
136.01	0\\
137.01	0\\
138.01	0\\
139.01	0\\
140.01	0\\
141.01	0\\
142.01	0\\
143.01	0\\
144.01	0\\
145.01	0\\
146.01	0\\
147.01	0\\
148.01	0\\
149.01	0\\
150.01	0\\
151.01	0\\
152.01	0\\
153.01	0\\
154.01	0\\
155.01	0\\
156.01	0\\
157.01	0\\
158.01	0\\
159.01	0\\
160.01	0\\
161.01	0\\
162.01	0\\
163.01	0\\
164.01	0\\
165.01	0\\
166.01	0\\
167.01	0\\
168.01	0\\
169.01	0\\
170.01	0\\
171.01	0\\
172.01	0\\
173.01	0\\
174.01	0\\
175.01	0\\
176.01	0\\
177.01	0\\
178.01	0\\
179.01	0\\
180.01	0\\
181.01	0\\
182.01	0\\
183.01	0\\
184.01	0\\
185.01	0\\
186.01	0\\
187.01	0\\
188.01	0\\
189.01	0\\
190.01	0\\
191.01	0\\
192.01	0\\
193.01	0\\
194.01	0\\
195.01	0\\
196.01	0\\
197.01	0\\
198.01	0\\
199.01	0\\
200.01	0\\
201.01	0\\
202.01	0\\
203.01	0\\
204.01	0\\
205.01	0\\
206.01	0\\
207.01	0\\
208.01	0\\
209.01	0\\
210.01	0\\
211.01	0\\
212.01	0\\
213.01	0\\
214.01	0\\
215.01	0\\
216.01	0\\
217.01	0\\
218.01	0\\
219.01	0\\
220.01	0\\
221.01	0\\
222.01	0\\
223.01	0\\
224.01	0\\
225.01	0\\
226.01	0\\
227.01	0\\
228.01	0\\
229.01	0\\
230.01	0\\
231.01	0\\
232.01	0\\
233.01	0\\
234.01	0\\
235.01	0\\
236.01	0\\
237.01	0\\
238.01	0\\
239.01	0\\
240.01	0\\
241.01	0\\
242.01	0\\
243.01	0\\
244.01	0\\
245.01	0\\
246.01	0\\
247.01	0\\
248.01	0\\
249.01	0\\
250.01	0\\
251.01	0\\
252.01	0\\
253.01	0\\
254.01	0\\
255.01	0\\
256.01	0\\
257.01	0\\
258.01	0\\
259.01	0\\
260.01	0\\
261.01	0\\
262.01	0\\
263.01	0\\
264.01	0\\
265.01	0\\
266.01	0\\
267.01	0\\
268.01	0\\
269.01	0\\
270.01	0\\
271.01	0\\
272.01	0\\
273.01	0\\
274.01	0\\
275.01	0\\
276.01	0\\
277.01	0\\
278.01	0\\
279.01	0\\
280.01	0\\
281.01	0\\
282.01	0\\
283.01	0\\
284.01	0\\
285.01	0\\
286.01	0\\
287.01	0\\
288.01	0\\
289.01	0\\
290.01	0\\
291.01	0\\
292.01	0\\
293.01	0\\
294.01	0\\
295.01	0\\
296.01	0\\
297.01	0\\
298.01	0\\
299.01	0\\
300.01	0\\
301.01	0\\
302.01	0\\
303.01	0\\
304.01	0\\
305.01	0\\
306.01	0\\
307.01	0\\
308.01	0\\
309.01	0\\
310.01	0\\
311.01	0\\
312.01	0\\
313.01	0\\
314.01	0\\
315.01	0\\
316.01	0\\
317.01	0\\
318.01	0\\
319.01	0\\
320.01	0\\
321.01	0\\
322.01	0\\
323.01	0\\
324.01	0\\
325.01	0\\
326.01	0\\
327.01	0\\
328.01	0\\
329.01	0\\
330.01	0\\
331.01	0\\
332.01	0\\
333.01	0\\
334.01	0\\
335.01	0\\
336.01	0\\
337.01	0\\
338.01	0\\
339.01	0\\
340.01	0\\
341.01	0\\
342.01	0\\
343.01	0\\
344.01	0\\
345.01	0\\
346.01	0\\
347.01	0\\
348.01	0\\
349.01	0\\
350.01	0\\
351.01	0\\
352.01	0\\
353.01	0\\
354.01	0\\
355.01	0\\
356.01	0\\
357.01	0\\
358.01	0\\
359.01	0\\
360.01	0\\
361.01	0\\
362.01	0\\
363.01	0\\
364.01	0\\
365.01	0\\
366.01	0\\
367.01	0\\
368.01	0\\
369.01	0\\
370.01	0\\
371.01	0\\
372.01	0\\
373.01	0\\
374.01	0\\
375.01	0\\
376.01	0\\
377.01	0\\
378.01	0\\
379.01	0\\
380.01	0\\
381.01	0\\
382.01	0\\
383.01	0\\
384.01	0\\
385.01	0\\
386.01	0\\
387.01	0\\
388.01	0\\
389.01	0\\
390.01	0\\
391.01	0\\
392.01	0\\
393.01	0\\
394.01	0\\
395.01	0\\
396.01	0\\
397.01	0\\
398.01	0\\
399.01	0\\
400.01	0\\
401.01	0\\
402.01	0\\
403.01	0\\
404.01	0\\
405.01	0\\
406.01	0\\
407.01	0\\
408.01	0\\
409.01	0\\
410.01	0\\
411.01	0\\
412.01	0\\
413.01	0\\
414.01	0\\
415.01	0\\
416.01	0\\
417.01	0\\
418.01	0\\
419.01	0\\
420.01	0\\
421.01	0\\
422.01	0\\
423.01	0\\
424.01	0\\
425.01	0\\
426.01	0\\
427.01	0\\
428.01	0\\
429.01	0\\
430.01	0\\
431.01	0\\
432.01	0\\
433.01	0\\
434.01	0\\
435.01	0\\
436.01	0\\
437.01	0\\
438.01	0\\
439.01	0\\
440.01	0\\
441.01	0\\
442.01	0\\
443.01	0\\
444.01	0\\
445.01	0\\
446.01	0\\
447.01	0\\
448.01	0\\
449.01	0\\
450.01	0\\
451.01	1.73472347597681e-18\\
452.01	0\\
453.01	0\\
454.01	0\\
455.01	0\\
456.01	0\\
457.01	0\\
458.01	0\\
459.01	0\\
460.01	0\\
461.01	0\\
462.01	0\\
463.01	0\\
464.01	0\\
465.01	0\\
466.01	0\\
467.01	0\\
468.01	0\\
469.01	1.73472347597681e-18\\
470.01	0\\
471.01	0\\
472.01	0\\
473.01	0\\
474.01	0\\
475.01	0\\
476.01	1.73472347597681e-18\\
477.01	0\\
478.01	0\\
479.01	0\\
480.01	0\\
481.01	0\\
482.01	0\\
483.01	0\\
484.01	0\\
485.01	0\\
486.01	0\\
487.01	0\\
488.01	0\\
489.01	0\\
490.01	0\\
491.01	0\\
492.01	0\\
493.01	0\\
494.01	0\\
495.01	0\\
496.01	0\\
497.01	0\\
498.01	0\\
499.01	0\\
500.01	0\\
501.01	0\\
502.01	0\\
503.01	0\\
504.01	0\\
505.01	0\\
506.01	0\\
507.01	0\\
508.01	0\\
509.01	0\\
510.01	0\\
511.01	0\\
512.01	1.73472347597681e-18\\
513.01	0\\
514.01	0\\
515.01	0\\
516.01	0\\
517.01	0\\
518.01	0\\
519.01	0\\
520.01	0\\
521.01	0\\
522.01	0\\
523.01	1.73472347597681e-18\\
524.01	0\\
525.01	0\\
526.01	0\\
527.01	0\\
528.01	0\\
529.01	0\\
530.01	0\\
531.01	0\\
532.01	0\\
533.01	0\\
534.01	0\\
535.01	0\\
536.01	1.73472347597681e-18\\
537.01	1.73472347597681e-18\\
538.01	1.73472347597681e-18\\
539.01	0\\
540.01	0\\
541.01	0\\
542.01	0\\
543.01	0\\
544.01	0\\
545.01	1.73472347597681e-18\\
546.01	1.73472347597681e-18\\
547.01	0\\
548.01	0\\
549.01	0\\
550.01	0\\
551.01	0\\
552.01	0\\
553.01	0\\
554.01	0\\
555.01	0\\
556.01	1.73472347597681e-18\\
557.01	0\\
558.01	0\\
559.01	0\\
560.01	1.73472347597681e-18\\
561.01	1.73472347597681e-18\\
562.01	0\\
563.01	0\\
564.01	0\\
565.01	0\\
566.01	0\\
567.01	0\\
568.01	0\\
569.01	0\\
570.01	0\\
571.01	0\\
572.01	0\\
573.01	0\\
574.01	0\\
575.01	0\\
576.01	0\\
577.01	0\\
578.01	1.73472347597681e-18\\
579.01	0\\
580.01	0\\
581.01	0\\
582.01	0\\
583.01	1.73472347597681e-18\\
584.01	0\\
585.01	0\\
586.01	0\\
587.01	0\\
588.01	0\\
589.01	0\\
590.01	0\\
591.01	0\\
592.01	0\\
593.01	0\\
594.01	0\\
595.01	0\\
596.01	0\\
597.01	0\\
598.01	0\\
599.01	0\\
599.02	0\\
599.03	0\\
599.04	0\\
599.05	0\\
599.06	0\\
599.07	0\\
599.08	0\\
599.09	0\\
599.1	0\\
599.11	0\\
599.12	0\\
599.13	0\\
599.14	0\\
599.15	0\\
599.16	0\\
599.17	0\\
599.18	0\\
599.19	0\\
599.2	0\\
599.21	0\\
599.22	0\\
599.23	0\\
599.24	0\\
599.25	0\\
599.26	0\\
599.27	0\\
599.28	0\\
599.29	0\\
599.3	0\\
599.31	0\\
599.32	0\\
599.33	0\\
599.34	0\\
599.35	0\\
599.36	0\\
599.37	0\\
599.38	0\\
599.39	0\\
599.4	0\\
599.41	0\\
599.42	0\\
599.43	0\\
599.44	0\\
599.45	0\\
599.46	0\\
599.47	0\\
599.48	0\\
599.49	0\\
599.5	0\\
599.51	0\\
599.52	0\\
599.53	0\\
599.54	0\\
599.55	0\\
599.56	0\\
599.57	0\\
599.58	0\\
599.59	0\\
599.6	0\\
599.61	0\\
599.62	0\\
599.63	0\\
599.64	0\\
599.65	0\\
599.66	0\\
599.67	0\\
599.68	0\\
599.69	0\\
599.7	0\\
599.71	0\\
599.72	0\\
599.73	0\\
599.74	0\\
599.75	0\\
599.76	0\\
599.77	0\\
599.78	0\\
599.79	0\\
599.8	0\\
599.81	0\\
599.82	0\\
599.83	0\\
599.84	0\\
599.85	0\\
599.86	0\\
599.87	0\\
599.88	0\\
599.89	0\\
599.9	0\\
599.91	0\\
599.92	0\\
599.93	0\\
599.94	0\\
599.95	0\\
599.96	0\\
599.97	0\\
599.98	0\\
599.99	0\\
600	0\\
};
\addplot [color=mycolor5,solid,forget plot]
  table[row sep=crcr]{%
0.01	0\\
1.01	0\\
2.01	0\\
3.01	0\\
4.01	0\\
5.01	0\\
6.01	0\\
7.01	0\\
8.01	0\\
9.01	0\\
10.01	0\\
11.01	0\\
12.01	0\\
13.01	0\\
14.01	0\\
15.01	0\\
16.01	0\\
17.01	0\\
18.01	0\\
19.01	0\\
20.01	0\\
21.01	0\\
22.01	0\\
23.01	0\\
24.01	0\\
25.01	0\\
26.01	0\\
27.01	0\\
28.01	0\\
29.01	0\\
30.01	0\\
31.01	0\\
32.01	0\\
33.01	0\\
34.01	0\\
35.01	0\\
36.01	0\\
37.01	0\\
38.01	0\\
39.01	0\\
40.01	0\\
41.01	0\\
42.01	0\\
43.01	0\\
44.01	0\\
45.01	0\\
46.01	0\\
47.01	0\\
48.01	0\\
49.01	0\\
50.01	0\\
51.01	0\\
52.01	0\\
53.01	0\\
54.01	0\\
55.01	0\\
56.01	0\\
57.01	0\\
58.01	0\\
59.01	0\\
60.01	0\\
61.01	0\\
62.01	0\\
63.01	0\\
64.01	0\\
65.01	0\\
66.01	0\\
67.01	0\\
68.01	0\\
69.01	0\\
70.01	0\\
71.01	0\\
72.01	0\\
73.01	0\\
74.01	0\\
75.01	0\\
76.01	0\\
77.01	0\\
78.01	0\\
79.01	0\\
80.01	0\\
81.01	0\\
82.01	0\\
83.01	0\\
84.01	0\\
85.01	0\\
86.01	0\\
87.01	0\\
88.01	0\\
89.01	0\\
90.01	0\\
91.01	0\\
92.01	0\\
93.01	0\\
94.01	0\\
95.01	0\\
96.01	0\\
97.01	0\\
98.01	0\\
99.01	0\\
100.01	0\\
101.01	0\\
102.01	0\\
103.01	0\\
104.01	0\\
105.01	0\\
106.01	0\\
107.01	0\\
108.01	0\\
109.01	0\\
110.01	0\\
111.01	0\\
112.01	0\\
113.01	0\\
114.01	0\\
115.01	0\\
116.01	0\\
117.01	0\\
118.01	0\\
119.01	0\\
120.01	0\\
121.01	0\\
122.01	0\\
123.01	0\\
124.01	0\\
125.01	0\\
126.01	0\\
127.01	0\\
128.01	0\\
129.01	0\\
130.01	0\\
131.01	0\\
132.01	0\\
133.01	0\\
134.01	0\\
135.01	0\\
136.01	0\\
137.01	0\\
138.01	0\\
139.01	0\\
140.01	0\\
141.01	0\\
142.01	0\\
143.01	0\\
144.01	0\\
145.01	0\\
146.01	0\\
147.01	0\\
148.01	0\\
149.01	0\\
150.01	0\\
151.01	0\\
152.01	0\\
153.01	0\\
154.01	0\\
155.01	0\\
156.01	0\\
157.01	0\\
158.01	0\\
159.01	0\\
160.01	0\\
161.01	0\\
162.01	0\\
163.01	0\\
164.01	0\\
165.01	0\\
166.01	0\\
167.01	0\\
168.01	0\\
169.01	0\\
170.01	0\\
171.01	0\\
172.01	0\\
173.01	0\\
174.01	0\\
175.01	0\\
176.01	0\\
177.01	0\\
178.01	0\\
179.01	0\\
180.01	0\\
181.01	0\\
182.01	0\\
183.01	0\\
184.01	0\\
185.01	0\\
186.01	0\\
187.01	0\\
188.01	0\\
189.01	0\\
190.01	0\\
191.01	0\\
192.01	0\\
193.01	0\\
194.01	0\\
195.01	0\\
196.01	0\\
197.01	0\\
198.01	0\\
199.01	0\\
200.01	0\\
201.01	0\\
202.01	0\\
203.01	0\\
204.01	0\\
205.01	0\\
206.01	0\\
207.01	0\\
208.01	0\\
209.01	0\\
210.01	0\\
211.01	0\\
212.01	0\\
213.01	0\\
214.01	0\\
215.01	0\\
216.01	0\\
217.01	0\\
218.01	0\\
219.01	0\\
220.01	0\\
221.01	0\\
222.01	0\\
223.01	0\\
224.01	0\\
225.01	0\\
226.01	0\\
227.01	0\\
228.01	0\\
229.01	0\\
230.01	0\\
231.01	0\\
232.01	0\\
233.01	0\\
234.01	0\\
235.01	0\\
236.01	0\\
237.01	0\\
238.01	0\\
239.01	0\\
240.01	0\\
241.01	0\\
242.01	0\\
243.01	0\\
244.01	0\\
245.01	0\\
246.01	0\\
247.01	0\\
248.01	0\\
249.01	0\\
250.01	0\\
251.01	0\\
252.01	0\\
253.01	0\\
254.01	0\\
255.01	0\\
256.01	0\\
257.01	0\\
258.01	0\\
259.01	0\\
260.01	0\\
261.01	0\\
262.01	0\\
263.01	0\\
264.01	0\\
265.01	0\\
266.01	0\\
267.01	0\\
268.01	0\\
269.01	0\\
270.01	0\\
271.01	0\\
272.01	0\\
273.01	0\\
274.01	0\\
275.01	0\\
276.01	0\\
277.01	0\\
278.01	0\\
279.01	0\\
280.01	0\\
281.01	0\\
282.01	0\\
283.01	0\\
284.01	0\\
285.01	0\\
286.01	0\\
287.01	0\\
288.01	0\\
289.01	0\\
290.01	0\\
291.01	0\\
292.01	0\\
293.01	0\\
294.01	0\\
295.01	0\\
296.01	0\\
297.01	0\\
298.01	0\\
299.01	0\\
300.01	0\\
301.01	0\\
302.01	0\\
303.01	0\\
304.01	0\\
305.01	0\\
306.01	0\\
307.01	0\\
308.01	0\\
309.01	0\\
310.01	0\\
311.01	0\\
312.01	0\\
313.01	0\\
314.01	0\\
315.01	0\\
316.01	0\\
317.01	0\\
318.01	0\\
319.01	0\\
320.01	0\\
321.01	0\\
322.01	0\\
323.01	0\\
324.01	0\\
325.01	0\\
326.01	0\\
327.01	0\\
328.01	0\\
329.01	0\\
330.01	0\\
331.01	0\\
332.01	0\\
333.01	0\\
334.01	0\\
335.01	0\\
336.01	0\\
337.01	0\\
338.01	0\\
339.01	0\\
340.01	0\\
341.01	0\\
342.01	0\\
343.01	0\\
344.01	0\\
345.01	0\\
346.01	0\\
347.01	0\\
348.01	0\\
349.01	0\\
350.01	0\\
351.01	0\\
352.01	0\\
353.01	0\\
354.01	0\\
355.01	0\\
356.01	0\\
357.01	0\\
358.01	0\\
359.01	0\\
360.01	0\\
361.01	0\\
362.01	0\\
363.01	0\\
364.01	0\\
365.01	0\\
366.01	0\\
367.01	0\\
368.01	0\\
369.01	0\\
370.01	0\\
371.01	0\\
372.01	0\\
373.01	0\\
374.01	0\\
375.01	0\\
376.01	0\\
377.01	0\\
378.01	0\\
379.01	0\\
380.01	0\\
381.01	0\\
382.01	0\\
383.01	0\\
384.01	0\\
385.01	0\\
386.01	0\\
387.01	0\\
388.01	0\\
389.01	0\\
390.01	0\\
391.01	0\\
392.01	0\\
393.01	0\\
394.01	0\\
395.01	0\\
396.01	0\\
397.01	0\\
398.01	0\\
399.01	0\\
400.01	0\\
401.01	0\\
402.01	0\\
403.01	0\\
404.01	0\\
405.01	0\\
406.01	0\\
407.01	0\\
408.01	0\\
409.01	0\\
410.01	0\\
411.01	0\\
412.01	0\\
413.01	0\\
414.01	0\\
415.01	0\\
416.01	0\\
417.01	0\\
418.01	0\\
419.01	0\\
420.01	0\\
421.01	0\\
422.01	0\\
423.01	0\\
424.01	0\\
425.01	0\\
426.01	0\\
427.01	0\\
428.01	0\\
429.01	0\\
430.01	0\\
431.01	0\\
432.01	0\\
433.01	0\\
434.01	0\\
435.01	0\\
436.01	0\\
437.01	0\\
438.01	0\\
439.01	0\\
440.01	0\\
441.01	0\\
442.01	0\\
443.01	0\\
444.01	0\\
445.01	0\\
446.01	0\\
447.01	0\\
448.01	0\\
449.01	0\\
450.01	0\\
451.01	1.73472347597681e-18\\
452.01	0\\
453.01	0\\
454.01	0\\
455.01	0\\
456.01	0\\
457.01	0\\
458.01	0\\
459.01	0\\
460.01	0\\
461.01	0\\
462.01	0\\
463.01	0\\
464.01	0\\
465.01	0\\
466.01	0\\
467.01	0\\
468.01	0\\
469.01	1.73472347597681e-18\\
470.01	0\\
471.01	0\\
472.01	0\\
473.01	0\\
474.01	0\\
475.01	0\\
476.01	1.73472347597681e-18\\
477.01	0\\
478.01	0\\
479.01	0\\
480.01	0\\
481.01	0\\
482.01	0\\
483.01	0\\
484.01	0\\
485.01	0\\
486.01	0\\
487.01	0\\
488.01	0\\
489.01	0\\
490.01	0\\
491.01	0\\
492.01	0\\
493.01	0\\
494.01	0\\
495.01	0\\
496.01	0\\
497.01	0\\
498.01	0\\
499.01	0\\
500.01	0\\
501.01	0\\
502.01	0\\
503.01	0\\
504.01	0\\
505.01	0\\
506.01	0\\
507.01	0\\
508.01	0\\
509.01	0\\
510.01	0\\
511.01	0\\
512.01	1.73472347597681e-18\\
513.01	0\\
514.01	0\\
515.01	0\\
516.01	0\\
517.01	0\\
518.01	0\\
519.01	0\\
520.01	0\\
521.01	0\\
522.01	0\\
523.01	1.73472347597681e-18\\
524.01	0\\
525.01	0\\
526.01	0\\
527.01	0\\
528.01	0\\
529.01	0\\
530.01	0\\
531.01	0\\
532.01	0\\
533.01	0\\
534.01	0\\
535.01	0\\
536.01	1.73472347597681e-18\\
537.01	1.73472347597681e-18\\
538.01	1.73472347597681e-18\\
539.01	0\\
540.01	0\\
541.01	0\\
542.01	0\\
543.01	0\\
544.01	0\\
545.01	1.73472347597681e-18\\
546.01	1.73472347597681e-18\\
547.01	0\\
548.01	0\\
549.01	0\\
550.01	0\\
551.01	0\\
552.01	0\\
553.01	0\\
554.01	0\\
555.01	0\\
556.01	1.73472347597681e-18\\
557.01	0\\
558.01	0\\
559.01	0\\
560.01	1.73472347597681e-18\\
561.01	1.73472347597681e-18\\
562.01	0\\
563.01	0\\
564.01	0\\
565.01	0\\
566.01	0\\
567.01	0\\
568.01	0\\
569.01	0\\
570.01	0\\
571.01	0\\
572.01	0\\
573.01	0\\
574.01	0\\
575.01	0\\
576.01	0\\
577.01	0\\
578.01	1.73472347597681e-18\\
579.01	0\\
580.01	0\\
581.01	0\\
582.01	0\\
583.01	1.73472347597681e-18\\
584.01	0\\
585.01	0\\
586.01	0\\
587.01	0\\
588.01	0\\
589.01	0\\
590.01	0\\
591.01	0\\
592.01	0\\
593.01	0\\
594.01	0\\
595.01	0\\
596.01	0\\
597.01	0\\
598.01	0\\
599.01	0\\
599.02	0\\
599.03	0\\
599.04	0\\
599.05	0\\
599.06	0\\
599.07	0\\
599.08	0\\
599.09	0\\
599.1	0\\
599.11	0\\
599.12	0\\
599.13	0\\
599.14	0\\
599.15	0\\
599.16	0\\
599.17	0\\
599.18	0\\
599.19	0\\
599.2	0\\
599.21	0\\
599.22	0\\
599.23	0\\
599.24	0\\
599.25	0\\
599.26	0\\
599.27	0\\
599.28	0\\
599.29	0\\
599.3	0\\
599.31	0\\
599.32	0\\
599.33	0\\
599.34	0\\
599.35	0\\
599.36	0\\
599.37	0\\
599.38	0\\
599.39	0\\
599.4	0\\
599.41	0\\
599.42	0\\
599.43	0\\
599.44	0\\
599.45	0\\
599.46	0\\
599.47	0\\
599.48	0\\
599.49	0\\
599.5	0\\
599.51	0\\
599.52	0\\
599.53	0\\
599.54	0\\
599.55	0\\
599.56	0\\
599.57	0\\
599.58	0\\
599.59	0\\
599.6	0\\
599.61	0\\
599.62	0\\
599.63	0\\
599.64	0\\
599.65	0\\
599.66	0\\
599.67	0\\
599.68	0\\
599.69	0\\
599.7	0\\
599.71	0\\
599.72	0\\
599.73	0\\
599.74	0\\
599.75	0\\
599.76	0\\
599.77	0\\
599.78	0\\
599.79	0\\
599.8	0\\
599.81	0\\
599.82	0\\
599.83	0\\
599.84	0\\
599.85	0\\
599.86	0\\
599.87	0\\
599.88	0\\
599.89	0\\
599.9	0\\
599.91	0\\
599.92	0\\
599.93	0\\
599.94	0\\
599.95	0\\
599.96	0\\
599.97	0\\
599.98	0\\
599.99	0\\
600	0\\
};
\addplot [color=mycolor6,solid,forget plot]
  table[row sep=crcr]{%
0.01	0\\
1.01	0\\
2.01	0\\
3.01	0\\
4.01	0\\
5.01	0\\
6.01	0\\
7.01	0\\
8.01	0\\
9.01	0\\
10.01	0\\
11.01	0\\
12.01	0\\
13.01	0\\
14.01	0\\
15.01	0\\
16.01	0\\
17.01	0\\
18.01	0\\
19.01	0\\
20.01	0\\
21.01	0\\
22.01	0\\
23.01	0\\
24.01	0\\
25.01	0\\
26.01	0\\
27.01	0\\
28.01	0\\
29.01	0\\
30.01	0\\
31.01	0\\
32.01	0\\
33.01	0\\
34.01	0\\
35.01	0\\
36.01	0\\
37.01	0\\
38.01	0\\
39.01	0\\
40.01	0\\
41.01	0\\
42.01	0\\
43.01	0\\
44.01	0\\
45.01	0\\
46.01	0\\
47.01	0\\
48.01	0\\
49.01	0\\
50.01	0\\
51.01	0\\
52.01	0\\
53.01	0\\
54.01	0\\
55.01	0\\
56.01	0\\
57.01	0\\
58.01	0\\
59.01	0\\
60.01	0\\
61.01	0\\
62.01	0\\
63.01	0\\
64.01	0\\
65.01	0\\
66.01	0\\
67.01	0\\
68.01	0\\
69.01	0\\
70.01	0\\
71.01	0\\
72.01	0\\
73.01	0\\
74.01	0\\
75.01	0\\
76.01	0\\
77.01	0\\
78.01	0\\
79.01	0\\
80.01	0\\
81.01	0\\
82.01	0\\
83.01	0\\
84.01	0\\
85.01	0\\
86.01	0\\
87.01	0\\
88.01	0\\
89.01	0\\
90.01	0\\
91.01	0\\
92.01	0\\
93.01	0\\
94.01	0\\
95.01	0\\
96.01	0\\
97.01	0\\
98.01	0\\
99.01	0\\
100.01	0\\
101.01	0\\
102.01	0\\
103.01	0\\
104.01	0\\
105.01	0\\
106.01	0\\
107.01	0\\
108.01	0\\
109.01	0\\
110.01	0\\
111.01	0\\
112.01	0\\
113.01	0\\
114.01	0\\
115.01	0\\
116.01	0\\
117.01	0\\
118.01	0\\
119.01	0\\
120.01	0\\
121.01	0\\
122.01	0\\
123.01	0\\
124.01	0\\
125.01	0\\
126.01	0\\
127.01	0\\
128.01	0\\
129.01	0\\
130.01	0\\
131.01	0\\
132.01	0\\
133.01	0\\
134.01	0\\
135.01	0\\
136.01	0\\
137.01	0\\
138.01	0\\
139.01	0\\
140.01	0\\
141.01	0\\
142.01	0\\
143.01	0\\
144.01	0\\
145.01	0\\
146.01	0\\
147.01	0\\
148.01	0\\
149.01	0\\
150.01	0\\
151.01	0\\
152.01	0\\
153.01	0\\
154.01	0\\
155.01	0\\
156.01	0\\
157.01	0\\
158.01	0\\
159.01	0\\
160.01	0\\
161.01	0\\
162.01	0\\
163.01	0\\
164.01	0\\
165.01	0\\
166.01	0\\
167.01	0\\
168.01	0\\
169.01	0\\
170.01	0\\
171.01	0\\
172.01	0\\
173.01	0\\
174.01	0\\
175.01	0\\
176.01	0\\
177.01	0\\
178.01	0\\
179.01	0\\
180.01	0\\
181.01	0\\
182.01	0\\
183.01	0\\
184.01	0\\
185.01	0\\
186.01	0\\
187.01	0\\
188.01	0\\
189.01	0\\
190.01	0\\
191.01	0\\
192.01	0\\
193.01	0\\
194.01	0\\
195.01	0\\
196.01	0\\
197.01	0\\
198.01	0\\
199.01	0\\
200.01	0\\
201.01	0\\
202.01	0\\
203.01	0\\
204.01	0\\
205.01	0\\
206.01	0\\
207.01	0\\
208.01	0\\
209.01	0\\
210.01	0\\
211.01	0\\
212.01	0\\
213.01	0\\
214.01	0\\
215.01	0\\
216.01	0\\
217.01	0\\
218.01	0\\
219.01	0\\
220.01	0\\
221.01	0\\
222.01	0\\
223.01	0\\
224.01	0\\
225.01	0\\
226.01	0\\
227.01	0\\
228.01	0\\
229.01	0\\
230.01	0\\
231.01	0\\
232.01	0\\
233.01	0\\
234.01	0\\
235.01	0\\
236.01	0\\
237.01	0\\
238.01	0\\
239.01	0\\
240.01	0\\
241.01	0\\
242.01	0\\
243.01	0\\
244.01	0\\
245.01	0\\
246.01	0\\
247.01	0\\
248.01	0\\
249.01	0\\
250.01	0\\
251.01	0\\
252.01	0\\
253.01	0\\
254.01	0\\
255.01	0\\
256.01	0\\
257.01	0\\
258.01	0\\
259.01	0\\
260.01	0\\
261.01	0\\
262.01	0\\
263.01	0\\
264.01	0\\
265.01	0\\
266.01	0\\
267.01	0\\
268.01	0\\
269.01	0\\
270.01	0\\
271.01	0\\
272.01	0\\
273.01	0\\
274.01	0\\
275.01	0\\
276.01	0\\
277.01	0\\
278.01	0\\
279.01	0\\
280.01	0\\
281.01	0\\
282.01	0\\
283.01	0\\
284.01	0\\
285.01	0\\
286.01	0\\
287.01	0\\
288.01	0\\
289.01	0\\
290.01	0\\
291.01	0\\
292.01	0\\
293.01	0\\
294.01	0\\
295.01	0\\
296.01	0\\
297.01	0\\
298.01	0\\
299.01	0\\
300.01	0\\
301.01	0\\
302.01	0\\
303.01	0\\
304.01	0\\
305.01	0\\
306.01	0\\
307.01	0\\
308.01	0\\
309.01	0\\
310.01	0\\
311.01	0\\
312.01	0\\
313.01	0\\
314.01	0\\
315.01	0\\
316.01	0\\
317.01	0\\
318.01	0\\
319.01	0\\
320.01	0\\
321.01	0\\
322.01	0\\
323.01	0\\
324.01	0\\
325.01	0\\
326.01	0\\
327.01	0\\
328.01	0\\
329.01	0\\
330.01	0\\
331.01	0\\
332.01	0\\
333.01	0\\
334.01	0\\
335.01	0\\
336.01	0\\
337.01	0\\
338.01	0\\
339.01	0\\
340.01	0\\
341.01	0\\
342.01	0\\
343.01	0\\
344.01	0\\
345.01	0\\
346.01	0\\
347.01	0\\
348.01	0\\
349.01	0\\
350.01	0\\
351.01	0\\
352.01	0\\
353.01	0\\
354.01	0\\
355.01	0\\
356.01	0\\
357.01	0\\
358.01	0\\
359.01	0\\
360.01	0\\
361.01	0\\
362.01	0\\
363.01	0\\
364.01	0\\
365.01	0\\
366.01	0\\
367.01	0\\
368.01	0\\
369.01	0\\
370.01	0\\
371.01	0\\
372.01	0\\
373.01	0\\
374.01	0\\
375.01	0\\
376.01	0\\
377.01	0\\
378.01	0\\
379.01	0\\
380.01	0\\
381.01	0\\
382.01	0\\
383.01	0\\
384.01	0\\
385.01	0\\
386.01	0\\
387.01	0\\
388.01	0\\
389.01	0\\
390.01	0\\
391.01	0\\
392.01	0\\
393.01	0\\
394.01	0\\
395.01	0\\
396.01	0\\
397.01	0\\
398.01	0\\
399.01	0\\
400.01	0\\
401.01	0\\
402.01	0\\
403.01	0\\
404.01	0\\
405.01	0\\
406.01	0\\
407.01	0\\
408.01	0\\
409.01	0\\
410.01	0\\
411.01	0\\
412.01	0\\
413.01	0\\
414.01	0\\
415.01	0\\
416.01	0\\
417.01	0\\
418.01	0\\
419.01	0\\
420.01	0\\
421.01	0\\
422.01	0\\
423.01	0\\
424.01	0\\
425.01	0\\
426.01	0\\
427.01	0\\
428.01	0\\
429.01	0\\
430.01	0\\
431.01	0\\
432.01	0\\
433.01	0\\
434.01	0\\
435.01	0\\
436.01	0\\
437.01	0\\
438.01	0\\
439.01	0\\
440.01	0\\
441.01	0\\
442.01	0\\
443.01	0\\
444.01	0\\
445.01	0\\
446.01	0\\
447.01	0\\
448.01	0\\
449.01	0\\
450.01	0\\
451.01	1.73472347597681e-18\\
452.01	0\\
453.01	0\\
454.01	0\\
455.01	0\\
456.01	0\\
457.01	0\\
458.01	0\\
459.01	0\\
460.01	0\\
461.01	0\\
462.01	0\\
463.01	0\\
464.01	0\\
465.01	0\\
466.01	0\\
467.01	0\\
468.01	0\\
469.01	1.73472347597681e-18\\
470.01	0\\
471.01	0\\
472.01	0\\
473.01	0\\
474.01	0\\
475.01	0\\
476.01	1.73472347597681e-18\\
477.01	0\\
478.01	0\\
479.01	0\\
480.01	0\\
481.01	0\\
482.01	0\\
483.01	0\\
484.01	0\\
485.01	0\\
486.01	0\\
487.01	0\\
488.01	0\\
489.01	0\\
490.01	0\\
491.01	0\\
492.01	0\\
493.01	0\\
494.01	0\\
495.01	0\\
496.01	0\\
497.01	0\\
498.01	0\\
499.01	0\\
500.01	0\\
501.01	0\\
502.01	0\\
503.01	0\\
504.01	0\\
505.01	0\\
506.01	0\\
507.01	0\\
508.01	0\\
509.01	0\\
510.01	0\\
511.01	0\\
512.01	1.73472347597681e-18\\
513.01	0\\
514.01	0\\
515.01	0\\
516.01	0\\
517.01	0\\
518.01	0\\
519.01	0\\
520.01	0\\
521.01	0\\
522.01	0\\
523.01	1.73472347597681e-18\\
524.01	0\\
525.01	0\\
526.01	0\\
527.01	0\\
528.01	0\\
529.01	0\\
530.01	0\\
531.01	0\\
532.01	0\\
533.01	0\\
534.01	0\\
535.01	0\\
536.01	1.73472347597681e-18\\
537.01	1.73472347597681e-18\\
538.01	1.73472347597681e-18\\
539.01	0\\
540.01	0\\
541.01	0\\
542.01	0\\
543.01	0\\
544.01	0\\
545.01	1.73472347597681e-18\\
546.01	1.73472347597681e-18\\
547.01	0\\
548.01	0\\
549.01	0\\
550.01	0\\
551.01	0\\
552.01	0\\
553.01	0\\
554.01	0\\
555.01	0\\
556.01	1.73472347597681e-18\\
557.01	0\\
558.01	0\\
559.01	0\\
560.01	1.73472347597681e-18\\
561.01	1.73472347597681e-18\\
562.01	0\\
563.01	0\\
564.01	0\\
565.01	0\\
566.01	0\\
567.01	0\\
568.01	0\\
569.01	0\\
570.01	0\\
571.01	0\\
572.01	0\\
573.01	0\\
574.01	0\\
575.01	0\\
576.01	0\\
577.01	0\\
578.01	1.73472347597681e-18\\
579.01	0\\
580.01	0\\
581.01	0\\
582.01	0\\
583.01	1.73472347597681e-18\\
584.01	0\\
585.01	0\\
586.01	0\\
587.01	0\\
588.01	0\\
589.01	0\\
590.01	0\\
591.01	0\\
592.01	0\\
593.01	0\\
594.01	0\\
595.01	0\\
596.01	0\\
597.01	0\\
598.01	0\\
599.01	0\\
599.02	0\\
599.03	0\\
599.04	0\\
599.05	0\\
599.06	0\\
599.07	0\\
599.08	0\\
599.09	0\\
599.1	0\\
599.11	0\\
599.12	0\\
599.13	0\\
599.14	0\\
599.15	0\\
599.16	0\\
599.17	0\\
599.18	0\\
599.19	0\\
599.2	0\\
599.21	0\\
599.22	0\\
599.23	0\\
599.24	0\\
599.25	0\\
599.26	0\\
599.27	0\\
599.28	0\\
599.29	0\\
599.3	0\\
599.31	0\\
599.32	0\\
599.33	0\\
599.34	0\\
599.35	0\\
599.36	0\\
599.37	0\\
599.38	0\\
599.39	0\\
599.4	0\\
599.41	0\\
599.42	0\\
599.43	0\\
599.44	0\\
599.45	0\\
599.46	0\\
599.47	0\\
599.48	0\\
599.49	0\\
599.5	0\\
599.51	0\\
599.52	0\\
599.53	0\\
599.54	0\\
599.55	0\\
599.56	0\\
599.57	0\\
599.58	0\\
599.59	0\\
599.6	0\\
599.61	0\\
599.62	0\\
599.63	0\\
599.64	0\\
599.65	0\\
599.66	0\\
599.67	0\\
599.68	0\\
599.69	0\\
599.7	0\\
599.71	0\\
599.72	0\\
599.73	0\\
599.74	0\\
599.75	0\\
599.76	0\\
599.77	0\\
599.78	0\\
599.79	0\\
599.8	0\\
599.81	0\\
599.82	0\\
599.83	0\\
599.84	0\\
599.85	0\\
599.86	0\\
599.87	0\\
599.88	0\\
599.89	0\\
599.9	0\\
599.91	0\\
599.92	0\\
599.93	0\\
599.94	0\\
599.95	0\\
599.96	0\\
599.97	0\\
599.98	0\\
599.99	0\\
600	0\\
};
\addplot [color=mycolor7,solid,forget plot]
  table[row sep=crcr]{%
0.01	0\\
1.01	0\\
2.01	0\\
3.01	0\\
4.01	0\\
5.01	0\\
6.01	0\\
7.01	0\\
8.01	0\\
9.01	0\\
10.01	0\\
11.01	0\\
12.01	0\\
13.01	0\\
14.01	0\\
15.01	0\\
16.01	0\\
17.01	0\\
18.01	0\\
19.01	0\\
20.01	0\\
21.01	0\\
22.01	0\\
23.01	0\\
24.01	0\\
25.01	0\\
26.01	0\\
27.01	0\\
28.01	0\\
29.01	0\\
30.01	0\\
31.01	0\\
32.01	0\\
33.01	0\\
34.01	0\\
35.01	0\\
36.01	0\\
37.01	0\\
38.01	0\\
39.01	0\\
40.01	0\\
41.01	0\\
42.01	0\\
43.01	0\\
44.01	0\\
45.01	0\\
46.01	0\\
47.01	0\\
48.01	0\\
49.01	0\\
50.01	0\\
51.01	0\\
52.01	0\\
53.01	0\\
54.01	0\\
55.01	0\\
56.01	0\\
57.01	0\\
58.01	0\\
59.01	0\\
60.01	0\\
61.01	0\\
62.01	0\\
63.01	0\\
64.01	0\\
65.01	0\\
66.01	0\\
67.01	0\\
68.01	0\\
69.01	0\\
70.01	0\\
71.01	0\\
72.01	0\\
73.01	0\\
74.01	0\\
75.01	0\\
76.01	0\\
77.01	0\\
78.01	0\\
79.01	0\\
80.01	0\\
81.01	0\\
82.01	0\\
83.01	0\\
84.01	0\\
85.01	0\\
86.01	0\\
87.01	0\\
88.01	0\\
89.01	0\\
90.01	0\\
91.01	0\\
92.01	0\\
93.01	0\\
94.01	0\\
95.01	0\\
96.01	0\\
97.01	0\\
98.01	0\\
99.01	0\\
100.01	0\\
101.01	0\\
102.01	0\\
103.01	0\\
104.01	0\\
105.01	0\\
106.01	0\\
107.01	0\\
108.01	0\\
109.01	0\\
110.01	0\\
111.01	0\\
112.01	0\\
113.01	0\\
114.01	0\\
115.01	0\\
116.01	0\\
117.01	0\\
118.01	0\\
119.01	0\\
120.01	0\\
121.01	0\\
122.01	0\\
123.01	0\\
124.01	0\\
125.01	0\\
126.01	0\\
127.01	0\\
128.01	0\\
129.01	0\\
130.01	0\\
131.01	0\\
132.01	0\\
133.01	0\\
134.01	0\\
135.01	0\\
136.01	0\\
137.01	0\\
138.01	0\\
139.01	0\\
140.01	0\\
141.01	0\\
142.01	0\\
143.01	0\\
144.01	0\\
145.01	0\\
146.01	0\\
147.01	0\\
148.01	0\\
149.01	0\\
150.01	0\\
151.01	0\\
152.01	0\\
153.01	0\\
154.01	0\\
155.01	0\\
156.01	0\\
157.01	0\\
158.01	0\\
159.01	0\\
160.01	0\\
161.01	0\\
162.01	0\\
163.01	0\\
164.01	0\\
165.01	0\\
166.01	0\\
167.01	0\\
168.01	0\\
169.01	0\\
170.01	0\\
171.01	0\\
172.01	0\\
173.01	0\\
174.01	0\\
175.01	0\\
176.01	0\\
177.01	0\\
178.01	0\\
179.01	0\\
180.01	0\\
181.01	0\\
182.01	0\\
183.01	0\\
184.01	0\\
185.01	0\\
186.01	0\\
187.01	0\\
188.01	0\\
189.01	0\\
190.01	0\\
191.01	0\\
192.01	0\\
193.01	0\\
194.01	0\\
195.01	0\\
196.01	0\\
197.01	0\\
198.01	0\\
199.01	0\\
200.01	0\\
201.01	0\\
202.01	0\\
203.01	0\\
204.01	0\\
205.01	0\\
206.01	0\\
207.01	0\\
208.01	0\\
209.01	0\\
210.01	0\\
211.01	0\\
212.01	0\\
213.01	0\\
214.01	0\\
215.01	0\\
216.01	0\\
217.01	0\\
218.01	0\\
219.01	0\\
220.01	0\\
221.01	0\\
222.01	0\\
223.01	0\\
224.01	0\\
225.01	0\\
226.01	0\\
227.01	0\\
228.01	0\\
229.01	0\\
230.01	0\\
231.01	0\\
232.01	0\\
233.01	0\\
234.01	0\\
235.01	0\\
236.01	0\\
237.01	0\\
238.01	0\\
239.01	0\\
240.01	0\\
241.01	0\\
242.01	0\\
243.01	0\\
244.01	0\\
245.01	0\\
246.01	0\\
247.01	0\\
248.01	0\\
249.01	0\\
250.01	0\\
251.01	0\\
252.01	0\\
253.01	0\\
254.01	0\\
255.01	0\\
256.01	0\\
257.01	0\\
258.01	0\\
259.01	0\\
260.01	0\\
261.01	0\\
262.01	0\\
263.01	0\\
264.01	0\\
265.01	0\\
266.01	0\\
267.01	0\\
268.01	0\\
269.01	0\\
270.01	0\\
271.01	0\\
272.01	0\\
273.01	0\\
274.01	0\\
275.01	0\\
276.01	0\\
277.01	0\\
278.01	0\\
279.01	0\\
280.01	0\\
281.01	0\\
282.01	0\\
283.01	0\\
284.01	0\\
285.01	0\\
286.01	0\\
287.01	0\\
288.01	0\\
289.01	0\\
290.01	0\\
291.01	0\\
292.01	0\\
293.01	0\\
294.01	0\\
295.01	0\\
296.01	0\\
297.01	0\\
298.01	0\\
299.01	0\\
300.01	0\\
301.01	0\\
302.01	0\\
303.01	0\\
304.01	0\\
305.01	0\\
306.01	0\\
307.01	0\\
308.01	0\\
309.01	0\\
310.01	0\\
311.01	0\\
312.01	0\\
313.01	0\\
314.01	0\\
315.01	0\\
316.01	0\\
317.01	0\\
318.01	0\\
319.01	0\\
320.01	0\\
321.01	0\\
322.01	0\\
323.01	0\\
324.01	0\\
325.01	0\\
326.01	0\\
327.01	0\\
328.01	0\\
329.01	0\\
330.01	0\\
331.01	0\\
332.01	0\\
333.01	0\\
334.01	0\\
335.01	0\\
336.01	0\\
337.01	0\\
338.01	0\\
339.01	0\\
340.01	0\\
341.01	0\\
342.01	0\\
343.01	0\\
344.01	0\\
345.01	0\\
346.01	0\\
347.01	0\\
348.01	0\\
349.01	0\\
350.01	0\\
351.01	0\\
352.01	0\\
353.01	0\\
354.01	0\\
355.01	0\\
356.01	0\\
357.01	0\\
358.01	0\\
359.01	0\\
360.01	0\\
361.01	0\\
362.01	0\\
363.01	0\\
364.01	0\\
365.01	0\\
366.01	0\\
367.01	0\\
368.01	0\\
369.01	0\\
370.01	0\\
371.01	0\\
372.01	0\\
373.01	0\\
374.01	0\\
375.01	0\\
376.01	0\\
377.01	0\\
378.01	0\\
379.01	0\\
380.01	0\\
381.01	0\\
382.01	0\\
383.01	0\\
384.01	0\\
385.01	0\\
386.01	0\\
387.01	0\\
388.01	0\\
389.01	0\\
390.01	0\\
391.01	0\\
392.01	0\\
393.01	0\\
394.01	0\\
395.01	0\\
396.01	0\\
397.01	0\\
398.01	0\\
399.01	0\\
400.01	0\\
401.01	0\\
402.01	0\\
403.01	0\\
404.01	0\\
405.01	0\\
406.01	0\\
407.01	0\\
408.01	0\\
409.01	0\\
410.01	0\\
411.01	0\\
412.01	0\\
413.01	0\\
414.01	0\\
415.01	0\\
416.01	0\\
417.01	0\\
418.01	0\\
419.01	0\\
420.01	0\\
421.01	0\\
422.01	0\\
423.01	0\\
424.01	0\\
425.01	0\\
426.01	0\\
427.01	0\\
428.01	0\\
429.01	0\\
430.01	0\\
431.01	0\\
432.01	0\\
433.01	0\\
434.01	0\\
435.01	0\\
436.01	0\\
437.01	0\\
438.01	0\\
439.01	0\\
440.01	0\\
441.01	0\\
442.01	0\\
443.01	0\\
444.01	0\\
445.01	0\\
446.01	0\\
447.01	0\\
448.01	0\\
449.01	0\\
450.01	0\\
451.01	1.73472347597681e-18\\
452.01	0\\
453.01	0\\
454.01	0\\
455.01	0\\
456.01	0\\
457.01	0\\
458.01	0\\
459.01	0\\
460.01	0\\
461.01	0\\
462.01	0\\
463.01	0\\
464.01	0\\
465.01	0\\
466.01	0\\
467.01	0\\
468.01	0\\
469.01	1.73472347597681e-18\\
470.01	0\\
471.01	0\\
472.01	0\\
473.01	0\\
474.01	0\\
475.01	0\\
476.01	1.73472347597681e-18\\
477.01	0\\
478.01	0\\
479.01	0\\
480.01	0\\
481.01	0\\
482.01	0\\
483.01	0\\
484.01	0\\
485.01	0\\
486.01	0\\
487.01	0\\
488.01	0\\
489.01	0\\
490.01	0\\
491.01	0\\
492.01	0\\
493.01	0\\
494.01	0\\
495.01	0\\
496.01	0\\
497.01	0\\
498.01	0\\
499.01	0\\
500.01	0\\
501.01	0\\
502.01	0\\
503.01	0\\
504.01	0\\
505.01	0\\
506.01	0\\
507.01	0\\
508.01	0\\
509.01	0\\
510.01	0\\
511.01	0\\
512.01	1.73472347597681e-18\\
513.01	0\\
514.01	0\\
515.01	0\\
516.01	0\\
517.01	0\\
518.01	0\\
519.01	0\\
520.01	0\\
521.01	0\\
522.01	0\\
523.01	1.73472347597681e-18\\
524.01	0\\
525.01	0\\
526.01	0\\
527.01	0\\
528.01	0\\
529.01	0\\
530.01	0\\
531.01	0\\
532.01	0\\
533.01	0\\
534.01	0\\
535.01	0\\
536.01	1.73472347597681e-18\\
537.01	1.73472347597681e-18\\
538.01	1.73472347597681e-18\\
539.01	0\\
540.01	0\\
541.01	0\\
542.01	0\\
543.01	0\\
544.01	0\\
545.01	1.73472347597681e-18\\
546.01	1.73472347597681e-18\\
547.01	0\\
548.01	0\\
549.01	0\\
550.01	0\\
551.01	0\\
552.01	0\\
553.01	0\\
554.01	0\\
555.01	0\\
556.01	1.73472347597681e-18\\
557.01	0\\
558.01	0\\
559.01	0\\
560.01	1.73472347597681e-18\\
561.01	1.73472347597681e-18\\
562.01	0\\
563.01	0\\
564.01	0\\
565.01	0\\
566.01	0\\
567.01	0\\
568.01	0\\
569.01	0\\
570.01	0\\
571.01	0\\
572.01	0\\
573.01	0\\
574.01	0\\
575.01	0\\
576.01	0\\
577.01	0\\
578.01	1.73472347597681e-18\\
579.01	0\\
580.01	0\\
581.01	0\\
582.01	0\\
583.01	1.73472347597681e-18\\
584.01	0\\
585.01	0\\
586.01	0\\
587.01	0\\
588.01	0\\
589.01	0\\
590.01	0\\
591.01	0\\
592.01	0\\
593.01	0\\
594.01	0\\
595.01	0\\
596.01	0\\
597.01	0\\
598.01	0\\
599.01	0\\
599.02	0\\
599.03	0\\
599.04	0\\
599.05	0\\
599.06	0\\
599.07	0\\
599.08	0\\
599.09	0\\
599.1	0\\
599.11	0\\
599.12	0\\
599.13	0\\
599.14	0\\
599.15	0\\
599.16	0\\
599.17	0\\
599.18	0\\
599.19	0\\
599.2	0\\
599.21	0\\
599.22	0\\
599.23	0\\
599.24	0\\
599.25	0\\
599.26	0\\
599.27	0\\
599.28	0\\
599.29	0\\
599.3	0\\
599.31	0\\
599.32	0\\
599.33	0\\
599.34	0\\
599.35	0\\
599.36	0\\
599.37	0\\
599.38	0\\
599.39	0\\
599.4	0\\
599.41	0\\
599.42	0\\
599.43	0\\
599.44	0\\
599.45	0\\
599.46	0\\
599.47	0\\
599.48	0\\
599.49	0\\
599.5	0\\
599.51	0\\
599.52	0\\
599.53	0\\
599.54	0\\
599.55	0\\
599.56	0\\
599.57	0\\
599.58	0\\
599.59	0\\
599.6	0\\
599.61	0\\
599.62	0\\
599.63	0\\
599.64	0\\
599.65	0\\
599.66	0\\
599.67	0\\
599.68	0\\
599.69	0\\
599.7	0\\
599.71	0\\
599.72	0\\
599.73	0\\
599.74	0\\
599.75	0\\
599.76	0\\
599.77	0\\
599.78	0\\
599.79	0\\
599.8	0\\
599.81	0\\
599.82	0\\
599.83	0\\
599.84	0\\
599.85	0\\
599.86	0\\
599.87	0\\
599.88	0\\
599.89	0\\
599.9	0\\
599.91	0\\
599.92	0\\
599.93	0\\
599.94	0\\
599.95	0\\
599.96	0\\
599.97	0\\
599.98	0\\
599.99	0\\
600	0\\
};
\addplot [color=mycolor8,solid,forget plot]
  table[row sep=crcr]{%
0.01	0\\
1.01	0\\
2.01	0\\
3.01	0\\
4.01	0\\
5.01	0\\
6.01	0\\
7.01	0\\
8.01	0\\
9.01	0\\
10.01	0\\
11.01	0\\
12.01	0\\
13.01	0\\
14.01	0\\
15.01	0\\
16.01	0\\
17.01	0\\
18.01	0\\
19.01	0\\
20.01	0\\
21.01	0\\
22.01	0\\
23.01	0\\
24.01	0\\
25.01	0\\
26.01	0\\
27.01	0\\
28.01	0\\
29.01	0\\
30.01	0\\
31.01	0\\
32.01	0\\
33.01	0\\
34.01	0\\
35.01	0\\
36.01	0\\
37.01	0\\
38.01	0\\
39.01	0\\
40.01	0\\
41.01	0\\
42.01	0\\
43.01	0\\
44.01	0\\
45.01	0\\
46.01	0\\
47.01	0\\
48.01	0\\
49.01	0\\
50.01	0\\
51.01	0\\
52.01	0\\
53.01	0\\
54.01	0\\
55.01	0\\
56.01	0\\
57.01	0\\
58.01	0\\
59.01	0\\
60.01	0\\
61.01	0\\
62.01	0\\
63.01	0\\
64.01	0\\
65.01	0\\
66.01	0\\
67.01	0\\
68.01	0\\
69.01	0\\
70.01	0\\
71.01	0\\
72.01	0\\
73.01	0\\
74.01	0\\
75.01	0\\
76.01	0\\
77.01	0\\
78.01	0\\
79.01	0\\
80.01	0\\
81.01	0\\
82.01	0\\
83.01	0\\
84.01	0\\
85.01	0\\
86.01	0\\
87.01	0\\
88.01	0\\
89.01	0\\
90.01	0\\
91.01	0\\
92.01	0\\
93.01	0\\
94.01	0\\
95.01	0\\
96.01	0\\
97.01	0\\
98.01	0\\
99.01	0\\
100.01	0\\
101.01	0\\
102.01	0\\
103.01	0\\
104.01	0\\
105.01	0\\
106.01	0\\
107.01	0\\
108.01	0\\
109.01	0\\
110.01	0\\
111.01	0\\
112.01	0\\
113.01	0\\
114.01	0\\
115.01	0\\
116.01	0\\
117.01	0\\
118.01	0\\
119.01	0\\
120.01	0\\
121.01	0\\
122.01	0\\
123.01	0\\
124.01	0\\
125.01	0\\
126.01	0\\
127.01	0\\
128.01	0\\
129.01	0\\
130.01	0\\
131.01	0\\
132.01	0\\
133.01	0\\
134.01	0\\
135.01	0\\
136.01	0\\
137.01	0\\
138.01	0\\
139.01	0\\
140.01	0\\
141.01	0\\
142.01	0\\
143.01	0\\
144.01	0\\
145.01	0\\
146.01	0\\
147.01	0\\
148.01	0\\
149.01	0\\
150.01	0\\
151.01	0\\
152.01	0\\
153.01	0\\
154.01	0\\
155.01	0\\
156.01	0\\
157.01	0\\
158.01	0\\
159.01	0\\
160.01	0\\
161.01	0\\
162.01	0\\
163.01	0\\
164.01	0\\
165.01	0\\
166.01	0\\
167.01	0\\
168.01	0\\
169.01	0\\
170.01	0\\
171.01	0\\
172.01	0\\
173.01	0\\
174.01	0\\
175.01	0\\
176.01	0\\
177.01	0\\
178.01	0\\
179.01	0\\
180.01	0\\
181.01	0\\
182.01	0\\
183.01	0\\
184.01	0\\
185.01	0\\
186.01	0\\
187.01	0\\
188.01	0\\
189.01	0\\
190.01	0\\
191.01	0\\
192.01	0\\
193.01	0\\
194.01	0\\
195.01	0\\
196.01	0\\
197.01	0\\
198.01	0\\
199.01	0\\
200.01	0\\
201.01	0\\
202.01	0\\
203.01	0\\
204.01	0\\
205.01	0\\
206.01	0\\
207.01	0\\
208.01	0\\
209.01	0\\
210.01	0\\
211.01	0\\
212.01	0\\
213.01	0\\
214.01	0\\
215.01	0\\
216.01	0\\
217.01	0\\
218.01	0\\
219.01	0\\
220.01	0\\
221.01	0\\
222.01	0\\
223.01	0\\
224.01	0\\
225.01	0\\
226.01	0\\
227.01	0\\
228.01	0\\
229.01	0\\
230.01	0\\
231.01	0\\
232.01	0\\
233.01	0\\
234.01	0\\
235.01	0\\
236.01	0\\
237.01	0\\
238.01	0\\
239.01	0\\
240.01	0\\
241.01	0\\
242.01	0\\
243.01	0\\
244.01	0\\
245.01	0\\
246.01	0\\
247.01	0\\
248.01	0\\
249.01	0\\
250.01	0\\
251.01	0\\
252.01	0\\
253.01	0\\
254.01	0\\
255.01	0\\
256.01	0\\
257.01	0\\
258.01	0\\
259.01	0\\
260.01	0\\
261.01	0\\
262.01	0\\
263.01	0\\
264.01	0\\
265.01	0\\
266.01	0\\
267.01	0\\
268.01	0\\
269.01	0\\
270.01	0\\
271.01	0\\
272.01	0\\
273.01	0\\
274.01	0\\
275.01	0\\
276.01	0\\
277.01	0\\
278.01	0\\
279.01	0\\
280.01	0\\
281.01	0\\
282.01	0\\
283.01	0\\
284.01	0\\
285.01	0\\
286.01	0\\
287.01	0\\
288.01	0\\
289.01	0\\
290.01	0\\
291.01	0\\
292.01	0\\
293.01	0\\
294.01	0\\
295.01	0\\
296.01	0\\
297.01	0\\
298.01	0\\
299.01	0\\
300.01	0\\
301.01	0\\
302.01	0\\
303.01	0\\
304.01	0\\
305.01	0\\
306.01	0\\
307.01	0\\
308.01	0\\
309.01	0\\
310.01	0\\
311.01	0\\
312.01	0\\
313.01	0\\
314.01	0\\
315.01	0\\
316.01	0\\
317.01	0\\
318.01	0\\
319.01	0\\
320.01	0\\
321.01	0\\
322.01	0\\
323.01	0\\
324.01	0\\
325.01	0\\
326.01	0\\
327.01	0\\
328.01	0\\
329.01	0\\
330.01	0\\
331.01	0\\
332.01	0\\
333.01	0\\
334.01	0\\
335.01	0\\
336.01	0\\
337.01	0\\
338.01	0\\
339.01	0\\
340.01	0\\
341.01	0\\
342.01	0\\
343.01	0\\
344.01	0\\
345.01	0\\
346.01	0\\
347.01	0\\
348.01	0\\
349.01	0\\
350.01	0\\
351.01	0\\
352.01	0\\
353.01	0\\
354.01	0\\
355.01	0\\
356.01	0\\
357.01	0\\
358.01	0\\
359.01	0\\
360.01	0\\
361.01	0\\
362.01	0\\
363.01	0\\
364.01	0\\
365.01	0\\
366.01	0\\
367.01	0\\
368.01	0\\
369.01	0\\
370.01	0\\
371.01	0\\
372.01	0\\
373.01	0\\
374.01	0\\
375.01	0\\
376.01	0\\
377.01	0\\
378.01	0\\
379.01	0\\
380.01	0\\
381.01	0\\
382.01	0\\
383.01	0\\
384.01	0\\
385.01	0\\
386.01	0\\
387.01	0\\
388.01	0\\
389.01	0\\
390.01	0\\
391.01	0\\
392.01	0\\
393.01	0\\
394.01	0\\
395.01	0\\
396.01	0\\
397.01	0\\
398.01	0\\
399.01	0\\
400.01	0\\
401.01	0\\
402.01	0\\
403.01	0\\
404.01	0\\
405.01	0\\
406.01	0\\
407.01	0\\
408.01	0\\
409.01	0\\
410.01	0\\
411.01	0\\
412.01	0\\
413.01	0\\
414.01	0\\
415.01	0\\
416.01	0\\
417.01	0\\
418.01	0\\
419.01	0\\
420.01	0\\
421.01	0\\
422.01	0\\
423.01	0\\
424.01	0\\
425.01	0\\
426.01	0\\
427.01	0\\
428.01	0\\
429.01	0\\
430.01	0\\
431.01	0\\
432.01	0\\
433.01	0\\
434.01	0\\
435.01	0\\
436.01	0\\
437.01	0\\
438.01	0\\
439.01	0\\
440.01	0\\
441.01	0\\
442.01	0\\
443.01	0\\
444.01	0\\
445.01	0\\
446.01	0\\
447.01	0\\
448.01	0\\
449.01	0\\
450.01	0\\
451.01	1.73472347597681e-18\\
452.01	0\\
453.01	0\\
454.01	0\\
455.01	0\\
456.01	0\\
457.01	0\\
458.01	0\\
459.01	0\\
460.01	0\\
461.01	0\\
462.01	0\\
463.01	0\\
464.01	0\\
465.01	0\\
466.01	0\\
467.01	0\\
468.01	0\\
469.01	1.73472347597681e-18\\
470.01	0\\
471.01	0\\
472.01	0\\
473.01	0\\
474.01	0\\
475.01	0\\
476.01	1.73472347597681e-18\\
477.01	0\\
478.01	0\\
479.01	0\\
480.01	0\\
481.01	0\\
482.01	0\\
483.01	0\\
484.01	0\\
485.01	0\\
486.01	0\\
487.01	0\\
488.01	0\\
489.01	0\\
490.01	0\\
491.01	0\\
492.01	0\\
493.01	0\\
494.01	0\\
495.01	0\\
496.01	0\\
497.01	0\\
498.01	0\\
499.01	0\\
500.01	0\\
501.01	0\\
502.01	0\\
503.01	0\\
504.01	0\\
505.01	0\\
506.01	0\\
507.01	0\\
508.01	0\\
509.01	0\\
510.01	0\\
511.01	0\\
512.01	1.73472347597681e-18\\
513.01	0\\
514.01	0\\
515.01	0\\
516.01	0\\
517.01	0\\
518.01	0\\
519.01	0\\
520.01	0\\
521.01	0\\
522.01	0\\
523.01	1.73472347597681e-18\\
524.01	0\\
525.01	0\\
526.01	0\\
527.01	0\\
528.01	0\\
529.01	0\\
530.01	0\\
531.01	0\\
532.01	0\\
533.01	0\\
534.01	0\\
535.01	0\\
536.01	1.73472347597681e-18\\
537.01	1.73472347597681e-18\\
538.01	1.73472347597681e-18\\
539.01	0\\
540.01	0\\
541.01	0\\
542.01	0\\
543.01	0\\
544.01	0\\
545.01	1.73472347597681e-18\\
546.01	1.73472347597681e-18\\
547.01	0\\
548.01	0\\
549.01	0\\
550.01	0\\
551.01	0\\
552.01	0\\
553.01	0\\
554.01	0\\
555.01	0\\
556.01	1.73472347597681e-18\\
557.01	0\\
558.01	0\\
559.01	0\\
560.01	1.73472347597681e-18\\
561.01	1.73472347597681e-18\\
562.01	0\\
563.01	0\\
564.01	0\\
565.01	0\\
566.01	0\\
567.01	0\\
568.01	0\\
569.01	0\\
570.01	0\\
571.01	0\\
572.01	0\\
573.01	0\\
574.01	0\\
575.01	0\\
576.01	0\\
577.01	0\\
578.01	1.73472347597681e-18\\
579.01	0\\
580.01	0\\
581.01	0\\
582.01	0\\
583.01	1.73472347597681e-18\\
584.01	0\\
585.01	0\\
586.01	0\\
587.01	0\\
588.01	0\\
589.01	0\\
590.01	0\\
591.01	0\\
592.01	0\\
593.01	0\\
594.01	0\\
595.01	0\\
596.01	0\\
597.01	0\\
598.01	0\\
599.01	0\\
599.02	0\\
599.03	0\\
599.04	0\\
599.05	0\\
599.06	0\\
599.07	0\\
599.08	0\\
599.09	0\\
599.1	0\\
599.11	0\\
599.12	0\\
599.13	0\\
599.14	0\\
599.15	0\\
599.16	0\\
599.17	0\\
599.18	0\\
599.19	0\\
599.2	0\\
599.21	0\\
599.22	0\\
599.23	0\\
599.24	0\\
599.25	0\\
599.26	0\\
599.27	0\\
599.28	0\\
599.29	0\\
599.3	0\\
599.31	0\\
599.32	0\\
599.33	0\\
599.34	0\\
599.35	0\\
599.36	0\\
599.37	0\\
599.38	0\\
599.39	0\\
599.4	0\\
599.41	0\\
599.42	0\\
599.43	0\\
599.44	0\\
599.45	0\\
599.46	0\\
599.47	0\\
599.48	0\\
599.49	0\\
599.5	0\\
599.51	0\\
599.52	0\\
599.53	0\\
599.54	0\\
599.55	0\\
599.56	0\\
599.57	0\\
599.58	0\\
599.59	0\\
599.6	0\\
599.61	0\\
599.62	0\\
599.63	0\\
599.64	0\\
599.65	0\\
599.66	0\\
599.67	0\\
599.68	0\\
599.69	0\\
599.7	0\\
599.71	0\\
599.72	0\\
599.73	0\\
599.74	0\\
599.75	0\\
599.76	0\\
599.77	0\\
599.78	0\\
599.79	0\\
599.8	0\\
599.81	0\\
599.82	0\\
599.83	0\\
599.84	0\\
599.85	0\\
599.86	0\\
599.87	0\\
599.88	0\\
599.89	0\\
599.9	0\\
599.91	0\\
599.92	0\\
599.93	0\\
599.94	0\\
599.95	0\\
599.96	0\\
599.97	0\\
599.98	0\\
599.99	0\\
600	0\\
};
\addplot [color=blue!25!mycolor7,solid,forget plot]
  table[row sep=crcr]{%
0.01	0\\
1.01	0\\
2.01	0\\
3.01	0\\
4.01	0\\
5.01	0\\
6.01	0\\
7.01	0\\
8.01	0\\
9.01	0\\
10.01	0\\
11.01	0\\
12.01	0\\
13.01	0\\
14.01	0\\
15.01	0\\
16.01	0\\
17.01	0\\
18.01	0\\
19.01	0\\
20.01	0\\
21.01	0\\
22.01	0\\
23.01	0\\
24.01	0\\
25.01	0\\
26.01	0\\
27.01	0\\
28.01	0\\
29.01	0\\
30.01	0\\
31.01	0\\
32.01	0\\
33.01	0\\
34.01	0\\
35.01	0\\
36.01	0\\
37.01	0\\
38.01	0\\
39.01	0\\
40.01	0\\
41.01	0\\
42.01	0\\
43.01	0\\
44.01	0\\
45.01	0\\
46.01	0\\
47.01	0\\
48.01	0\\
49.01	0\\
50.01	0\\
51.01	0\\
52.01	0\\
53.01	0\\
54.01	0\\
55.01	0\\
56.01	0\\
57.01	0\\
58.01	0\\
59.01	0\\
60.01	0\\
61.01	0\\
62.01	0\\
63.01	0\\
64.01	0\\
65.01	0\\
66.01	0\\
67.01	0\\
68.01	0\\
69.01	0\\
70.01	0\\
71.01	0\\
72.01	0\\
73.01	0\\
74.01	0\\
75.01	0\\
76.01	0\\
77.01	0\\
78.01	0\\
79.01	0\\
80.01	0\\
81.01	0\\
82.01	0\\
83.01	0\\
84.01	0\\
85.01	0\\
86.01	0\\
87.01	0\\
88.01	0\\
89.01	0\\
90.01	0\\
91.01	0\\
92.01	0\\
93.01	0\\
94.01	0\\
95.01	0\\
96.01	0\\
97.01	0\\
98.01	0\\
99.01	0\\
100.01	0\\
101.01	0\\
102.01	0\\
103.01	0\\
104.01	0\\
105.01	0\\
106.01	0\\
107.01	0\\
108.01	0\\
109.01	0\\
110.01	0\\
111.01	0\\
112.01	0\\
113.01	0\\
114.01	0\\
115.01	0\\
116.01	0\\
117.01	0\\
118.01	0\\
119.01	0\\
120.01	0\\
121.01	0\\
122.01	0\\
123.01	0\\
124.01	0\\
125.01	0\\
126.01	0\\
127.01	0\\
128.01	0\\
129.01	0\\
130.01	0\\
131.01	0\\
132.01	0\\
133.01	0\\
134.01	0\\
135.01	0\\
136.01	0\\
137.01	0\\
138.01	0\\
139.01	0\\
140.01	0\\
141.01	0\\
142.01	0\\
143.01	0\\
144.01	0\\
145.01	0\\
146.01	0\\
147.01	0\\
148.01	0\\
149.01	0\\
150.01	0\\
151.01	0\\
152.01	0\\
153.01	0\\
154.01	0\\
155.01	0\\
156.01	0\\
157.01	0\\
158.01	0\\
159.01	0\\
160.01	0\\
161.01	0\\
162.01	0\\
163.01	0\\
164.01	0\\
165.01	0\\
166.01	0\\
167.01	0\\
168.01	0\\
169.01	0\\
170.01	0\\
171.01	0\\
172.01	0\\
173.01	0\\
174.01	0\\
175.01	0\\
176.01	0\\
177.01	0\\
178.01	0\\
179.01	0\\
180.01	0\\
181.01	0\\
182.01	0\\
183.01	0\\
184.01	0\\
185.01	0\\
186.01	0\\
187.01	0\\
188.01	0\\
189.01	0\\
190.01	0\\
191.01	0\\
192.01	0\\
193.01	0\\
194.01	0\\
195.01	0\\
196.01	0\\
197.01	0\\
198.01	0\\
199.01	0\\
200.01	0\\
201.01	0\\
202.01	0\\
203.01	0\\
204.01	0\\
205.01	0\\
206.01	0\\
207.01	0\\
208.01	0\\
209.01	0\\
210.01	0\\
211.01	0\\
212.01	0\\
213.01	0\\
214.01	0\\
215.01	0\\
216.01	0\\
217.01	0\\
218.01	0\\
219.01	0\\
220.01	0\\
221.01	0\\
222.01	0\\
223.01	0\\
224.01	0\\
225.01	0\\
226.01	0\\
227.01	0\\
228.01	0\\
229.01	0\\
230.01	0\\
231.01	0\\
232.01	0\\
233.01	0\\
234.01	0\\
235.01	0\\
236.01	0\\
237.01	0\\
238.01	0\\
239.01	0\\
240.01	0\\
241.01	0\\
242.01	0\\
243.01	0\\
244.01	0\\
245.01	0\\
246.01	0\\
247.01	0\\
248.01	0\\
249.01	0\\
250.01	0\\
251.01	0\\
252.01	0\\
253.01	0\\
254.01	0\\
255.01	0\\
256.01	0\\
257.01	0\\
258.01	0\\
259.01	0\\
260.01	0\\
261.01	0\\
262.01	0\\
263.01	0\\
264.01	0\\
265.01	0\\
266.01	0\\
267.01	0\\
268.01	0\\
269.01	0\\
270.01	0\\
271.01	0\\
272.01	0\\
273.01	0\\
274.01	0\\
275.01	0\\
276.01	0\\
277.01	0\\
278.01	0\\
279.01	0\\
280.01	0\\
281.01	0\\
282.01	0\\
283.01	0\\
284.01	0\\
285.01	0\\
286.01	0\\
287.01	0\\
288.01	0\\
289.01	0\\
290.01	0\\
291.01	0\\
292.01	0\\
293.01	0\\
294.01	0\\
295.01	0\\
296.01	0\\
297.01	0\\
298.01	0\\
299.01	0\\
300.01	0\\
301.01	0\\
302.01	0\\
303.01	0\\
304.01	0\\
305.01	0\\
306.01	0\\
307.01	0\\
308.01	0\\
309.01	0\\
310.01	0\\
311.01	0\\
312.01	0\\
313.01	0\\
314.01	0\\
315.01	0\\
316.01	0\\
317.01	0\\
318.01	0\\
319.01	0\\
320.01	0\\
321.01	0\\
322.01	0\\
323.01	0\\
324.01	0\\
325.01	0\\
326.01	0\\
327.01	0\\
328.01	0\\
329.01	0\\
330.01	0\\
331.01	0\\
332.01	0\\
333.01	0\\
334.01	0\\
335.01	0\\
336.01	0\\
337.01	0\\
338.01	0\\
339.01	0\\
340.01	0\\
341.01	0\\
342.01	0\\
343.01	0\\
344.01	0\\
345.01	0\\
346.01	0\\
347.01	0\\
348.01	0\\
349.01	0\\
350.01	0\\
351.01	0\\
352.01	0\\
353.01	0\\
354.01	0\\
355.01	0\\
356.01	0\\
357.01	0\\
358.01	0\\
359.01	0\\
360.01	0\\
361.01	0\\
362.01	0\\
363.01	0\\
364.01	0\\
365.01	0\\
366.01	0\\
367.01	0\\
368.01	0\\
369.01	0\\
370.01	0\\
371.01	0\\
372.01	0\\
373.01	0\\
374.01	0\\
375.01	0\\
376.01	0\\
377.01	0\\
378.01	0\\
379.01	0\\
380.01	0\\
381.01	0\\
382.01	0\\
383.01	0\\
384.01	0\\
385.01	0\\
386.01	0\\
387.01	0\\
388.01	0\\
389.01	0\\
390.01	0\\
391.01	0\\
392.01	0\\
393.01	0\\
394.01	0\\
395.01	0\\
396.01	0\\
397.01	0\\
398.01	0\\
399.01	0\\
400.01	0\\
401.01	0\\
402.01	0\\
403.01	0\\
404.01	0\\
405.01	0\\
406.01	0\\
407.01	0\\
408.01	0\\
409.01	0\\
410.01	0\\
411.01	0\\
412.01	0\\
413.01	0\\
414.01	0\\
415.01	0\\
416.01	0\\
417.01	0\\
418.01	0\\
419.01	0\\
420.01	0\\
421.01	0\\
422.01	0\\
423.01	0\\
424.01	0\\
425.01	0\\
426.01	0\\
427.01	0\\
428.01	0\\
429.01	0\\
430.01	0\\
431.01	0\\
432.01	0\\
433.01	0\\
434.01	0\\
435.01	0\\
436.01	0\\
437.01	0\\
438.01	0\\
439.01	0\\
440.01	0\\
441.01	0\\
442.01	0\\
443.01	0\\
444.01	0\\
445.01	0\\
446.01	0\\
447.01	0\\
448.01	0\\
449.01	0\\
450.01	0\\
451.01	1.73472347597681e-18\\
452.01	0\\
453.01	0\\
454.01	0\\
455.01	0\\
456.01	0\\
457.01	0\\
458.01	0\\
459.01	0\\
460.01	0\\
461.01	0\\
462.01	0\\
463.01	0\\
464.01	0\\
465.01	0\\
466.01	0\\
467.01	0\\
468.01	0\\
469.01	1.73472347597681e-18\\
470.01	0\\
471.01	0\\
472.01	0\\
473.01	0\\
474.01	0\\
475.01	0\\
476.01	1.73472347597681e-18\\
477.01	0\\
478.01	0\\
479.01	0\\
480.01	0\\
481.01	0\\
482.01	0\\
483.01	0\\
484.01	0\\
485.01	0\\
486.01	0\\
487.01	0\\
488.01	0\\
489.01	0\\
490.01	0\\
491.01	0\\
492.01	0\\
493.01	0\\
494.01	0\\
495.01	0\\
496.01	0\\
497.01	0\\
498.01	0\\
499.01	0\\
500.01	0\\
501.01	0\\
502.01	0\\
503.01	0\\
504.01	0\\
505.01	0\\
506.01	0\\
507.01	0\\
508.01	0\\
509.01	0\\
510.01	0\\
511.01	0\\
512.01	1.73472347597681e-18\\
513.01	0\\
514.01	0\\
515.01	0\\
516.01	0\\
517.01	0\\
518.01	0\\
519.01	0\\
520.01	0\\
521.01	0\\
522.01	0\\
523.01	1.73472347597681e-18\\
524.01	0\\
525.01	0\\
526.01	0\\
527.01	0\\
528.01	0\\
529.01	0\\
530.01	0\\
531.01	0\\
532.01	0\\
533.01	0\\
534.01	0\\
535.01	0\\
536.01	1.73472347597681e-18\\
537.01	1.73472347597681e-18\\
538.01	1.73472347597681e-18\\
539.01	0\\
540.01	0\\
541.01	0\\
542.01	0\\
543.01	0\\
544.01	0\\
545.01	1.73472347597681e-18\\
546.01	1.73472347597681e-18\\
547.01	0\\
548.01	0\\
549.01	0\\
550.01	0\\
551.01	0\\
552.01	0\\
553.01	0\\
554.01	0\\
555.01	0\\
556.01	1.73472347597681e-18\\
557.01	0\\
558.01	0\\
559.01	0\\
560.01	1.73472347597681e-18\\
561.01	1.73472347597681e-18\\
562.01	0\\
563.01	0\\
564.01	0\\
565.01	0\\
566.01	0\\
567.01	0\\
568.01	0\\
569.01	0\\
570.01	0\\
571.01	0\\
572.01	0\\
573.01	0\\
574.01	0\\
575.01	0\\
576.01	0\\
577.01	0\\
578.01	1.73472347597681e-18\\
579.01	0\\
580.01	0\\
581.01	0\\
582.01	0\\
583.01	1.73472347597681e-18\\
584.01	0\\
585.01	0\\
586.01	0\\
587.01	0\\
588.01	0\\
589.01	0\\
590.01	0\\
591.01	0\\
592.01	0\\
593.01	0\\
594.01	0\\
595.01	0\\
596.01	0\\
597.01	0\\
598.01	0\\
599.01	0\\
599.02	0\\
599.03	0\\
599.04	0\\
599.05	0\\
599.06	0\\
599.07	0\\
599.08	0\\
599.09	0\\
599.1	0\\
599.11	0\\
599.12	0\\
599.13	0\\
599.14	0\\
599.15	0\\
599.16	0\\
599.17	0\\
599.18	0\\
599.19	0\\
599.2	0\\
599.21	0\\
599.22	0\\
599.23	0\\
599.24	0\\
599.25	0\\
599.26	0\\
599.27	0\\
599.28	0\\
599.29	0\\
599.3	0\\
599.31	0\\
599.32	0\\
599.33	0\\
599.34	0\\
599.35	0\\
599.36	0\\
599.37	0\\
599.38	0\\
599.39	0\\
599.4	0\\
599.41	0\\
599.42	0\\
599.43	0\\
599.44	0\\
599.45	0\\
599.46	0\\
599.47	0\\
599.48	0\\
599.49	0\\
599.5	0\\
599.51	0\\
599.52	0\\
599.53	0\\
599.54	0\\
599.55	0\\
599.56	0\\
599.57	0\\
599.58	0\\
599.59	0\\
599.6	0\\
599.61	0\\
599.62	0\\
599.63	0\\
599.64	0\\
599.65	0\\
599.66	0\\
599.67	0\\
599.68	0\\
599.69	0\\
599.7	0\\
599.71	0\\
599.72	0\\
599.73	0\\
599.74	0\\
599.75	0\\
599.76	0\\
599.77	0\\
599.78	0\\
599.79	0\\
599.8	0\\
599.81	0\\
599.82	0\\
599.83	0\\
599.84	0\\
599.85	0\\
599.86	0\\
599.87	0\\
599.88	0\\
599.89	0\\
599.9	0\\
599.91	0\\
599.92	0\\
599.93	0\\
599.94	0\\
599.95	0\\
599.96	0\\
599.97	0\\
599.98	0\\
599.99	0\\
600	0\\
};
\addplot [color=mycolor9,solid,forget plot]
  table[row sep=crcr]{%
0.01	0\\
1.01	0\\
2.01	0\\
3.01	0\\
4.01	0\\
5.01	0\\
6.01	0\\
7.01	0\\
8.01	0\\
9.01	0\\
10.01	0\\
11.01	0\\
12.01	0\\
13.01	0\\
14.01	0\\
15.01	0\\
16.01	0\\
17.01	0\\
18.01	0\\
19.01	0\\
20.01	0\\
21.01	0\\
22.01	0\\
23.01	0\\
24.01	0\\
25.01	0\\
26.01	0\\
27.01	0\\
28.01	0\\
29.01	0\\
30.01	0\\
31.01	0\\
32.01	0\\
33.01	0\\
34.01	0\\
35.01	0\\
36.01	0\\
37.01	0\\
38.01	0\\
39.01	0\\
40.01	0\\
41.01	0\\
42.01	0\\
43.01	0\\
44.01	0\\
45.01	0\\
46.01	0\\
47.01	0\\
48.01	0\\
49.01	0\\
50.01	0\\
51.01	0\\
52.01	0\\
53.01	0\\
54.01	0\\
55.01	0\\
56.01	0\\
57.01	0\\
58.01	0\\
59.01	0\\
60.01	0\\
61.01	0\\
62.01	0\\
63.01	0\\
64.01	0\\
65.01	0\\
66.01	0\\
67.01	0\\
68.01	0\\
69.01	0\\
70.01	0\\
71.01	0\\
72.01	0\\
73.01	0\\
74.01	0\\
75.01	0\\
76.01	0\\
77.01	0\\
78.01	0\\
79.01	0\\
80.01	0\\
81.01	0\\
82.01	0\\
83.01	0\\
84.01	0\\
85.01	0\\
86.01	0\\
87.01	0\\
88.01	0\\
89.01	0\\
90.01	0\\
91.01	0\\
92.01	0\\
93.01	0\\
94.01	0\\
95.01	0\\
96.01	0\\
97.01	0\\
98.01	0\\
99.01	0\\
100.01	0\\
101.01	0\\
102.01	0\\
103.01	0\\
104.01	0\\
105.01	0\\
106.01	0\\
107.01	0\\
108.01	0\\
109.01	0\\
110.01	0\\
111.01	0\\
112.01	0\\
113.01	0\\
114.01	0\\
115.01	0\\
116.01	0\\
117.01	0\\
118.01	0\\
119.01	0\\
120.01	0\\
121.01	0\\
122.01	0\\
123.01	0\\
124.01	0\\
125.01	0\\
126.01	0\\
127.01	0\\
128.01	0\\
129.01	0\\
130.01	0\\
131.01	0\\
132.01	0\\
133.01	0\\
134.01	0\\
135.01	0\\
136.01	0\\
137.01	0\\
138.01	0\\
139.01	0\\
140.01	0\\
141.01	0\\
142.01	0\\
143.01	0\\
144.01	0\\
145.01	0\\
146.01	0\\
147.01	0\\
148.01	0\\
149.01	0\\
150.01	0\\
151.01	0\\
152.01	0\\
153.01	0\\
154.01	0\\
155.01	0\\
156.01	0\\
157.01	0\\
158.01	0\\
159.01	0\\
160.01	0\\
161.01	0\\
162.01	0\\
163.01	0\\
164.01	0\\
165.01	0\\
166.01	0\\
167.01	0\\
168.01	0\\
169.01	0\\
170.01	0\\
171.01	0\\
172.01	0\\
173.01	0\\
174.01	0\\
175.01	0\\
176.01	0\\
177.01	0\\
178.01	0\\
179.01	0\\
180.01	0\\
181.01	0\\
182.01	0\\
183.01	0\\
184.01	0\\
185.01	0\\
186.01	0\\
187.01	0\\
188.01	0\\
189.01	0\\
190.01	0\\
191.01	0\\
192.01	0\\
193.01	0\\
194.01	0\\
195.01	0\\
196.01	0\\
197.01	0\\
198.01	0\\
199.01	0\\
200.01	0\\
201.01	0\\
202.01	0\\
203.01	0\\
204.01	0\\
205.01	0\\
206.01	0\\
207.01	0\\
208.01	0\\
209.01	0\\
210.01	0\\
211.01	0\\
212.01	0\\
213.01	0\\
214.01	0\\
215.01	0\\
216.01	0\\
217.01	0\\
218.01	0\\
219.01	0\\
220.01	0\\
221.01	0\\
222.01	0\\
223.01	0\\
224.01	0\\
225.01	0\\
226.01	0\\
227.01	0\\
228.01	0\\
229.01	0\\
230.01	0\\
231.01	0\\
232.01	0\\
233.01	0\\
234.01	0\\
235.01	0\\
236.01	0\\
237.01	0\\
238.01	0\\
239.01	0\\
240.01	0\\
241.01	0\\
242.01	0\\
243.01	0\\
244.01	0\\
245.01	0\\
246.01	0\\
247.01	0\\
248.01	0\\
249.01	0\\
250.01	0\\
251.01	0\\
252.01	0\\
253.01	0\\
254.01	0\\
255.01	0\\
256.01	0\\
257.01	0\\
258.01	0\\
259.01	0\\
260.01	0\\
261.01	0\\
262.01	0\\
263.01	0\\
264.01	0\\
265.01	0\\
266.01	0\\
267.01	0\\
268.01	0\\
269.01	0\\
270.01	0\\
271.01	0\\
272.01	0\\
273.01	0\\
274.01	0\\
275.01	0\\
276.01	0\\
277.01	0\\
278.01	0\\
279.01	0\\
280.01	0\\
281.01	0\\
282.01	0\\
283.01	0\\
284.01	0\\
285.01	0\\
286.01	0\\
287.01	0\\
288.01	0\\
289.01	0\\
290.01	0\\
291.01	0\\
292.01	0\\
293.01	0\\
294.01	0\\
295.01	0\\
296.01	0\\
297.01	0\\
298.01	0\\
299.01	0\\
300.01	0\\
301.01	0\\
302.01	0\\
303.01	0\\
304.01	0\\
305.01	0\\
306.01	0\\
307.01	0\\
308.01	0\\
309.01	0\\
310.01	0\\
311.01	0\\
312.01	0\\
313.01	0\\
314.01	0\\
315.01	0\\
316.01	0\\
317.01	0\\
318.01	0\\
319.01	0\\
320.01	0\\
321.01	0\\
322.01	0\\
323.01	0\\
324.01	0\\
325.01	0\\
326.01	0\\
327.01	0\\
328.01	0\\
329.01	0\\
330.01	0\\
331.01	0\\
332.01	0\\
333.01	0\\
334.01	0\\
335.01	0\\
336.01	0\\
337.01	0\\
338.01	0\\
339.01	0\\
340.01	0\\
341.01	0\\
342.01	0\\
343.01	0\\
344.01	0\\
345.01	0\\
346.01	0\\
347.01	0\\
348.01	0\\
349.01	0\\
350.01	0\\
351.01	0\\
352.01	0\\
353.01	0\\
354.01	0\\
355.01	0\\
356.01	0\\
357.01	0\\
358.01	0\\
359.01	0\\
360.01	0\\
361.01	0\\
362.01	0\\
363.01	0\\
364.01	0\\
365.01	0\\
366.01	0\\
367.01	0\\
368.01	0\\
369.01	0\\
370.01	0\\
371.01	0\\
372.01	0\\
373.01	0\\
374.01	0\\
375.01	0\\
376.01	0\\
377.01	0\\
378.01	0\\
379.01	0\\
380.01	0\\
381.01	0\\
382.01	0\\
383.01	0\\
384.01	0\\
385.01	0\\
386.01	0\\
387.01	0\\
388.01	0\\
389.01	0\\
390.01	0\\
391.01	0\\
392.01	0\\
393.01	0\\
394.01	0\\
395.01	0\\
396.01	0\\
397.01	0\\
398.01	0\\
399.01	0\\
400.01	0\\
401.01	0\\
402.01	0\\
403.01	0\\
404.01	0\\
405.01	0\\
406.01	0\\
407.01	0\\
408.01	0\\
409.01	0\\
410.01	0\\
411.01	0\\
412.01	0\\
413.01	0\\
414.01	0\\
415.01	0\\
416.01	0\\
417.01	0\\
418.01	0\\
419.01	0\\
420.01	0\\
421.01	0\\
422.01	0\\
423.01	0\\
424.01	0\\
425.01	0\\
426.01	0\\
427.01	0\\
428.01	0\\
429.01	0\\
430.01	0\\
431.01	0\\
432.01	0\\
433.01	0\\
434.01	0\\
435.01	0\\
436.01	0\\
437.01	0\\
438.01	0\\
439.01	0\\
440.01	0\\
441.01	0\\
442.01	0\\
443.01	0\\
444.01	0\\
445.01	0\\
446.01	0\\
447.01	0\\
448.01	0\\
449.01	0\\
450.01	0\\
451.01	1.73472347597681e-18\\
452.01	0\\
453.01	0\\
454.01	0\\
455.01	0\\
456.01	0\\
457.01	0\\
458.01	0\\
459.01	0\\
460.01	0\\
461.01	0\\
462.01	0\\
463.01	0\\
464.01	0\\
465.01	0\\
466.01	0\\
467.01	0\\
468.01	0\\
469.01	1.73472347597681e-18\\
470.01	0\\
471.01	0\\
472.01	0\\
473.01	0\\
474.01	0\\
475.01	0\\
476.01	1.73472347597681e-18\\
477.01	0\\
478.01	0\\
479.01	0\\
480.01	0\\
481.01	0\\
482.01	0\\
483.01	0\\
484.01	0\\
485.01	0\\
486.01	0\\
487.01	0\\
488.01	0\\
489.01	0\\
490.01	0\\
491.01	0\\
492.01	0\\
493.01	0\\
494.01	0\\
495.01	0\\
496.01	0\\
497.01	0\\
498.01	0\\
499.01	0\\
500.01	0\\
501.01	0\\
502.01	0\\
503.01	0\\
504.01	0\\
505.01	0\\
506.01	0\\
507.01	0\\
508.01	0\\
509.01	0\\
510.01	0\\
511.01	0\\
512.01	1.73472347597681e-18\\
513.01	0\\
514.01	0\\
515.01	0\\
516.01	0\\
517.01	0\\
518.01	0\\
519.01	0\\
520.01	0\\
521.01	0\\
522.01	0\\
523.01	1.73472347597681e-18\\
524.01	0\\
525.01	0\\
526.01	0\\
527.01	0\\
528.01	0\\
529.01	0\\
530.01	0\\
531.01	0\\
532.01	0\\
533.01	0\\
534.01	0\\
535.01	0\\
536.01	1.73472347597681e-18\\
537.01	1.73472347597681e-18\\
538.01	1.73472347597681e-18\\
539.01	0\\
540.01	0\\
541.01	0\\
542.01	0\\
543.01	0\\
544.01	0\\
545.01	1.73472347597681e-18\\
546.01	1.73472347597681e-18\\
547.01	0\\
548.01	0\\
549.01	0\\
550.01	0\\
551.01	0\\
552.01	0\\
553.01	0\\
554.01	0\\
555.01	0\\
556.01	1.73472347597681e-18\\
557.01	0\\
558.01	0\\
559.01	0\\
560.01	1.73472347597681e-18\\
561.01	1.73472347597681e-18\\
562.01	0\\
563.01	0\\
564.01	0\\
565.01	0\\
566.01	0\\
567.01	0\\
568.01	0\\
569.01	0\\
570.01	0\\
571.01	0\\
572.01	0\\
573.01	0\\
574.01	0\\
575.01	0\\
576.01	0\\
577.01	0\\
578.01	1.73472347597681e-18\\
579.01	0\\
580.01	0\\
581.01	0\\
582.01	0\\
583.01	1.73472347597681e-18\\
584.01	0\\
585.01	0\\
586.01	0\\
587.01	0\\
588.01	0\\
589.01	0\\
590.01	0\\
591.01	0\\
592.01	0\\
593.01	0\\
594.01	0\\
595.01	0\\
596.01	0\\
597.01	0\\
598.01	0\\
599.01	0\\
599.02	0\\
599.03	0\\
599.04	0\\
599.05	0\\
599.06	0\\
599.07	0\\
599.08	0\\
599.09	0\\
599.1	0\\
599.11	0\\
599.12	0\\
599.13	0\\
599.14	0\\
599.15	0\\
599.16	0\\
599.17	0\\
599.18	0\\
599.19	0\\
599.2	0\\
599.21	0\\
599.22	0\\
599.23	0\\
599.24	0\\
599.25	0\\
599.26	0\\
599.27	0\\
599.28	0\\
599.29	0\\
599.3	0\\
599.31	0\\
599.32	0\\
599.33	0\\
599.34	0\\
599.35	0\\
599.36	0\\
599.37	0\\
599.38	0\\
599.39	0\\
599.4	0\\
599.41	0\\
599.42	0\\
599.43	0\\
599.44	0\\
599.45	0\\
599.46	0\\
599.47	0\\
599.48	0\\
599.49	0\\
599.5	0\\
599.51	0\\
599.52	0\\
599.53	0\\
599.54	0\\
599.55	0\\
599.56	0\\
599.57	0\\
599.58	0\\
599.59	0\\
599.6	0\\
599.61	0\\
599.62	0\\
599.63	0\\
599.64	0\\
599.65	0\\
599.66	0\\
599.67	0\\
599.68	0\\
599.69	0\\
599.7	0\\
599.71	0\\
599.72	0\\
599.73	0\\
599.74	0\\
599.75	0\\
599.76	0\\
599.77	0\\
599.78	0\\
599.79	0\\
599.8	0\\
599.81	0\\
599.82	0\\
599.83	0\\
599.84	0\\
599.85	0\\
599.86	0\\
599.87	0\\
599.88	0\\
599.89	0\\
599.9	0\\
599.91	0\\
599.92	0\\
599.93	0\\
599.94	0\\
599.95	0\\
599.96	0\\
599.97	0\\
599.98	0\\
599.99	0\\
600	0\\
};
\addplot [color=blue!50!mycolor7,solid,forget plot]
  table[row sep=crcr]{%
0.01	0\\
1.01	0\\
2.01	0\\
3.01	0\\
4.01	0\\
5.01	0\\
6.01	0\\
7.01	0\\
8.01	0\\
9.01	0\\
10.01	0\\
11.01	0\\
12.01	0\\
13.01	0\\
14.01	0\\
15.01	0\\
16.01	0\\
17.01	0\\
18.01	0\\
19.01	0\\
20.01	0\\
21.01	0\\
22.01	0\\
23.01	0\\
24.01	0\\
25.01	0\\
26.01	0\\
27.01	0\\
28.01	0\\
29.01	0\\
30.01	0\\
31.01	0\\
32.01	0\\
33.01	0\\
34.01	0\\
35.01	0\\
36.01	0\\
37.01	0\\
38.01	0\\
39.01	0\\
40.01	0\\
41.01	0\\
42.01	0\\
43.01	0\\
44.01	0\\
45.01	0\\
46.01	0\\
47.01	0\\
48.01	0\\
49.01	0\\
50.01	0\\
51.01	0\\
52.01	0\\
53.01	0\\
54.01	0\\
55.01	0\\
56.01	0\\
57.01	0\\
58.01	0\\
59.01	0\\
60.01	0\\
61.01	0\\
62.01	0\\
63.01	0\\
64.01	0\\
65.01	0\\
66.01	0\\
67.01	0\\
68.01	0\\
69.01	0\\
70.01	0\\
71.01	0\\
72.01	0\\
73.01	0\\
74.01	0\\
75.01	0\\
76.01	0\\
77.01	0\\
78.01	0\\
79.01	0\\
80.01	0\\
81.01	0\\
82.01	0\\
83.01	0\\
84.01	0\\
85.01	0\\
86.01	0\\
87.01	0\\
88.01	0\\
89.01	0\\
90.01	0\\
91.01	0\\
92.01	0\\
93.01	0\\
94.01	0\\
95.01	0\\
96.01	0\\
97.01	0\\
98.01	0\\
99.01	0\\
100.01	0\\
101.01	0\\
102.01	0\\
103.01	0\\
104.01	0\\
105.01	0\\
106.01	0\\
107.01	0\\
108.01	0\\
109.01	0\\
110.01	0\\
111.01	0\\
112.01	0\\
113.01	0\\
114.01	0\\
115.01	0\\
116.01	0\\
117.01	0\\
118.01	0\\
119.01	0\\
120.01	0\\
121.01	0\\
122.01	0\\
123.01	0\\
124.01	0\\
125.01	0\\
126.01	0\\
127.01	0\\
128.01	0\\
129.01	0\\
130.01	0\\
131.01	0\\
132.01	0\\
133.01	0\\
134.01	0\\
135.01	0\\
136.01	0\\
137.01	0\\
138.01	0\\
139.01	0\\
140.01	0\\
141.01	0\\
142.01	0\\
143.01	0\\
144.01	0\\
145.01	0\\
146.01	0\\
147.01	0\\
148.01	0\\
149.01	0\\
150.01	0\\
151.01	0\\
152.01	0\\
153.01	0\\
154.01	0\\
155.01	0\\
156.01	0\\
157.01	0\\
158.01	0\\
159.01	0\\
160.01	0\\
161.01	0\\
162.01	0\\
163.01	0\\
164.01	0\\
165.01	0\\
166.01	0\\
167.01	0\\
168.01	0\\
169.01	0\\
170.01	0\\
171.01	0\\
172.01	0\\
173.01	0\\
174.01	0\\
175.01	0\\
176.01	0\\
177.01	0\\
178.01	0\\
179.01	0\\
180.01	0\\
181.01	0\\
182.01	0\\
183.01	0\\
184.01	0\\
185.01	0\\
186.01	0\\
187.01	0\\
188.01	0\\
189.01	0\\
190.01	0\\
191.01	0\\
192.01	0\\
193.01	0\\
194.01	0\\
195.01	0\\
196.01	0\\
197.01	0\\
198.01	0\\
199.01	0\\
200.01	0\\
201.01	0\\
202.01	0\\
203.01	0\\
204.01	0\\
205.01	0\\
206.01	0\\
207.01	0\\
208.01	0\\
209.01	0\\
210.01	0\\
211.01	0\\
212.01	0\\
213.01	0\\
214.01	0\\
215.01	0\\
216.01	0\\
217.01	0\\
218.01	0\\
219.01	0\\
220.01	0\\
221.01	0\\
222.01	0\\
223.01	0\\
224.01	0\\
225.01	0\\
226.01	0\\
227.01	0\\
228.01	0\\
229.01	0\\
230.01	0\\
231.01	0\\
232.01	0\\
233.01	0\\
234.01	0\\
235.01	0\\
236.01	0\\
237.01	0\\
238.01	0\\
239.01	0\\
240.01	0\\
241.01	0\\
242.01	0\\
243.01	0\\
244.01	0\\
245.01	0\\
246.01	0\\
247.01	0\\
248.01	0\\
249.01	0\\
250.01	0\\
251.01	0\\
252.01	0\\
253.01	0\\
254.01	0\\
255.01	0\\
256.01	0\\
257.01	0\\
258.01	0\\
259.01	0\\
260.01	0\\
261.01	0\\
262.01	0\\
263.01	0\\
264.01	0\\
265.01	0\\
266.01	0\\
267.01	0\\
268.01	0\\
269.01	0\\
270.01	0\\
271.01	0\\
272.01	0\\
273.01	0\\
274.01	0\\
275.01	0\\
276.01	0\\
277.01	0\\
278.01	0\\
279.01	0\\
280.01	0\\
281.01	0\\
282.01	0\\
283.01	0\\
284.01	0\\
285.01	0\\
286.01	0\\
287.01	0\\
288.01	0\\
289.01	0\\
290.01	0\\
291.01	0\\
292.01	0\\
293.01	0\\
294.01	0\\
295.01	0\\
296.01	0\\
297.01	0\\
298.01	0\\
299.01	0\\
300.01	0\\
301.01	0\\
302.01	0\\
303.01	0\\
304.01	0\\
305.01	0\\
306.01	0\\
307.01	0\\
308.01	0\\
309.01	0\\
310.01	0\\
311.01	0\\
312.01	0\\
313.01	0\\
314.01	0\\
315.01	0\\
316.01	0\\
317.01	0\\
318.01	0\\
319.01	0\\
320.01	0\\
321.01	0\\
322.01	0\\
323.01	0\\
324.01	0\\
325.01	0\\
326.01	0\\
327.01	0\\
328.01	0\\
329.01	0\\
330.01	0\\
331.01	0\\
332.01	0\\
333.01	0\\
334.01	0\\
335.01	0\\
336.01	0\\
337.01	0\\
338.01	0\\
339.01	0\\
340.01	0\\
341.01	0\\
342.01	0\\
343.01	0\\
344.01	0\\
345.01	0\\
346.01	0\\
347.01	0\\
348.01	0\\
349.01	0\\
350.01	0\\
351.01	0\\
352.01	0\\
353.01	0\\
354.01	0\\
355.01	0\\
356.01	0\\
357.01	0\\
358.01	0\\
359.01	0\\
360.01	0\\
361.01	0\\
362.01	0\\
363.01	0\\
364.01	0\\
365.01	0\\
366.01	0\\
367.01	0\\
368.01	0\\
369.01	0\\
370.01	0\\
371.01	0\\
372.01	0\\
373.01	0\\
374.01	0\\
375.01	0\\
376.01	0\\
377.01	0\\
378.01	0\\
379.01	0\\
380.01	0\\
381.01	0\\
382.01	0\\
383.01	0\\
384.01	0\\
385.01	0\\
386.01	0\\
387.01	0\\
388.01	0\\
389.01	0\\
390.01	0\\
391.01	0\\
392.01	0\\
393.01	0\\
394.01	0\\
395.01	0\\
396.01	0\\
397.01	0\\
398.01	0\\
399.01	0\\
400.01	0\\
401.01	0\\
402.01	0\\
403.01	0\\
404.01	0\\
405.01	0\\
406.01	0\\
407.01	0\\
408.01	0\\
409.01	0\\
410.01	0\\
411.01	0\\
412.01	0\\
413.01	0\\
414.01	0\\
415.01	0\\
416.01	0\\
417.01	0\\
418.01	0\\
419.01	0\\
420.01	0\\
421.01	0\\
422.01	0\\
423.01	0\\
424.01	0\\
425.01	0\\
426.01	0\\
427.01	0\\
428.01	0\\
429.01	0\\
430.01	0\\
431.01	0\\
432.01	0\\
433.01	0\\
434.01	0\\
435.01	0\\
436.01	0\\
437.01	0\\
438.01	0\\
439.01	0\\
440.01	0\\
441.01	0\\
442.01	0\\
443.01	0\\
444.01	0\\
445.01	0\\
446.01	0\\
447.01	0\\
448.01	0\\
449.01	0\\
450.01	0\\
451.01	1.73472347597681e-18\\
452.01	0\\
453.01	0\\
454.01	0\\
455.01	0\\
456.01	0\\
457.01	0\\
458.01	0\\
459.01	0\\
460.01	0\\
461.01	0\\
462.01	0\\
463.01	0\\
464.01	0\\
465.01	0\\
466.01	0\\
467.01	0\\
468.01	0\\
469.01	1.73472347597681e-18\\
470.01	0\\
471.01	0\\
472.01	0\\
473.01	0\\
474.01	0\\
475.01	0\\
476.01	1.73472347597681e-18\\
477.01	0\\
478.01	0\\
479.01	0\\
480.01	0\\
481.01	0\\
482.01	0\\
483.01	0\\
484.01	0\\
485.01	0\\
486.01	0\\
487.01	0\\
488.01	0\\
489.01	0\\
490.01	0\\
491.01	0\\
492.01	0\\
493.01	0\\
494.01	0\\
495.01	0\\
496.01	0\\
497.01	0\\
498.01	0\\
499.01	0\\
500.01	0\\
501.01	0\\
502.01	0\\
503.01	0\\
504.01	0\\
505.01	0\\
506.01	0\\
507.01	0\\
508.01	0\\
509.01	0\\
510.01	0\\
511.01	0\\
512.01	1.73472347597681e-18\\
513.01	0\\
514.01	0\\
515.01	0\\
516.01	0\\
517.01	0\\
518.01	0\\
519.01	0\\
520.01	0\\
521.01	0\\
522.01	0\\
523.01	1.73472347597681e-18\\
524.01	0\\
525.01	0\\
526.01	0\\
527.01	0\\
528.01	0\\
529.01	0\\
530.01	0\\
531.01	0\\
532.01	0\\
533.01	0\\
534.01	0\\
535.01	0\\
536.01	1.73472347597681e-18\\
537.01	1.73472347597681e-18\\
538.01	1.73472347597681e-18\\
539.01	0\\
540.01	0\\
541.01	0\\
542.01	0\\
543.01	0\\
544.01	0\\
545.01	1.73472347597681e-18\\
546.01	1.73472347597681e-18\\
547.01	0\\
548.01	0\\
549.01	0\\
550.01	0\\
551.01	0\\
552.01	0\\
553.01	0\\
554.01	0\\
555.01	0\\
556.01	1.73472347597681e-18\\
557.01	0\\
558.01	0\\
559.01	0\\
560.01	1.73472347597681e-18\\
561.01	1.73472347597681e-18\\
562.01	0\\
563.01	0\\
564.01	0\\
565.01	0\\
566.01	0\\
567.01	0\\
568.01	0\\
569.01	0\\
570.01	0\\
571.01	0\\
572.01	0\\
573.01	0\\
574.01	0\\
575.01	0\\
576.01	0\\
577.01	0\\
578.01	1.73472347597681e-18\\
579.01	0\\
580.01	0\\
581.01	0\\
582.01	0\\
583.01	1.73472347597681e-18\\
584.01	0\\
585.01	0\\
586.01	0\\
587.01	0\\
588.01	0\\
589.01	0\\
590.01	0\\
591.01	0\\
592.01	0\\
593.01	0\\
594.01	0\\
595.01	0\\
596.01	0\\
597.01	0\\
598.01	0\\
599.01	0\\
599.02	0\\
599.03	0\\
599.04	0\\
599.05	0\\
599.06	0\\
599.07	0\\
599.08	0\\
599.09	0\\
599.1	0\\
599.11	0\\
599.12	0\\
599.13	0\\
599.14	0\\
599.15	0\\
599.16	0\\
599.17	0\\
599.18	0\\
599.19	0\\
599.2	0\\
599.21	0\\
599.22	0\\
599.23	0\\
599.24	0\\
599.25	0\\
599.26	0\\
599.27	0\\
599.28	0\\
599.29	0\\
599.3	0\\
599.31	0\\
599.32	0\\
599.33	0\\
599.34	0\\
599.35	0\\
599.36	0\\
599.37	0\\
599.38	0\\
599.39	0\\
599.4	0\\
599.41	0\\
599.42	0\\
599.43	0\\
599.44	0\\
599.45	0\\
599.46	0\\
599.47	0\\
599.48	0\\
599.49	0\\
599.5	0\\
599.51	0\\
599.52	0\\
599.53	0\\
599.54	0\\
599.55	0\\
599.56	0\\
599.57	0\\
599.58	0\\
599.59	0\\
599.6	0\\
599.61	0\\
599.62	0\\
599.63	0\\
599.64	0\\
599.65	0\\
599.66	0\\
599.67	0\\
599.68	0\\
599.69	0\\
599.7	0\\
599.71	0\\
599.72	0\\
599.73	0\\
599.74	0\\
599.75	0\\
599.76	0\\
599.77	0\\
599.78	0\\
599.79	0\\
599.8	0\\
599.81	0\\
599.82	0\\
599.83	0\\
599.84	0\\
599.85	0\\
599.86	0\\
599.87	0\\
599.88	0\\
599.89	0\\
599.9	0\\
599.91	0\\
599.92	0\\
599.93	0\\
599.94	0\\
599.95	0\\
599.96	0\\
599.97	0\\
599.98	0\\
599.99	0\\
600	0\\
};
\addplot [color=blue!40!mycolor9,solid,forget plot]
  table[row sep=crcr]{%
0.01	0\\
1.01	0\\
2.01	0\\
3.01	0\\
4.01	0\\
5.01	0\\
6.01	0\\
7.01	0\\
8.01	0\\
9.01	0\\
10.01	0\\
11.01	0\\
12.01	0\\
13.01	0\\
14.01	0\\
15.01	0\\
16.01	0\\
17.01	0\\
18.01	0\\
19.01	0\\
20.01	0\\
21.01	0\\
22.01	0\\
23.01	0\\
24.01	0\\
25.01	0\\
26.01	0\\
27.01	0\\
28.01	0\\
29.01	0\\
30.01	0\\
31.01	0\\
32.01	0\\
33.01	0\\
34.01	0\\
35.01	0\\
36.01	0\\
37.01	0\\
38.01	0\\
39.01	0\\
40.01	0\\
41.01	0\\
42.01	0\\
43.01	0\\
44.01	0\\
45.01	0\\
46.01	0\\
47.01	0\\
48.01	0\\
49.01	0\\
50.01	0\\
51.01	0\\
52.01	0\\
53.01	0\\
54.01	0\\
55.01	0\\
56.01	0\\
57.01	0\\
58.01	0\\
59.01	0\\
60.01	0\\
61.01	0\\
62.01	0\\
63.01	0\\
64.01	0\\
65.01	0\\
66.01	0\\
67.01	0\\
68.01	0\\
69.01	0\\
70.01	0\\
71.01	0\\
72.01	0\\
73.01	0\\
74.01	0\\
75.01	0\\
76.01	0\\
77.01	0\\
78.01	0\\
79.01	0\\
80.01	0\\
81.01	0\\
82.01	0\\
83.01	0\\
84.01	0\\
85.01	0\\
86.01	0\\
87.01	0\\
88.01	0\\
89.01	0\\
90.01	0\\
91.01	0\\
92.01	0\\
93.01	0\\
94.01	0\\
95.01	0\\
96.01	0\\
97.01	0\\
98.01	0\\
99.01	0\\
100.01	0\\
101.01	0\\
102.01	0\\
103.01	0\\
104.01	0\\
105.01	0\\
106.01	0\\
107.01	0\\
108.01	0\\
109.01	0\\
110.01	0\\
111.01	0\\
112.01	0\\
113.01	0\\
114.01	0\\
115.01	0\\
116.01	0\\
117.01	0\\
118.01	0\\
119.01	0\\
120.01	0\\
121.01	0\\
122.01	0\\
123.01	0\\
124.01	0\\
125.01	0\\
126.01	0\\
127.01	0\\
128.01	0\\
129.01	0\\
130.01	0\\
131.01	0\\
132.01	0\\
133.01	0\\
134.01	0\\
135.01	0\\
136.01	0\\
137.01	0\\
138.01	0\\
139.01	0\\
140.01	0\\
141.01	0\\
142.01	0\\
143.01	0\\
144.01	0\\
145.01	0\\
146.01	0\\
147.01	0\\
148.01	0\\
149.01	0\\
150.01	0\\
151.01	0\\
152.01	0\\
153.01	0\\
154.01	0\\
155.01	0\\
156.01	0\\
157.01	0\\
158.01	0\\
159.01	0\\
160.01	0\\
161.01	0\\
162.01	0\\
163.01	0\\
164.01	0\\
165.01	0\\
166.01	0\\
167.01	0\\
168.01	0\\
169.01	0\\
170.01	0\\
171.01	0\\
172.01	0\\
173.01	0\\
174.01	0\\
175.01	0\\
176.01	0\\
177.01	0\\
178.01	0\\
179.01	0\\
180.01	0\\
181.01	0\\
182.01	0\\
183.01	0\\
184.01	0\\
185.01	0\\
186.01	0\\
187.01	0\\
188.01	0\\
189.01	0\\
190.01	0\\
191.01	0\\
192.01	0\\
193.01	0\\
194.01	0\\
195.01	0\\
196.01	0\\
197.01	0\\
198.01	0\\
199.01	0\\
200.01	0\\
201.01	0\\
202.01	0\\
203.01	0\\
204.01	0\\
205.01	0\\
206.01	0\\
207.01	0\\
208.01	0\\
209.01	0\\
210.01	0\\
211.01	0\\
212.01	0\\
213.01	0\\
214.01	0\\
215.01	0\\
216.01	0\\
217.01	0\\
218.01	0\\
219.01	0\\
220.01	0\\
221.01	0\\
222.01	0\\
223.01	0\\
224.01	0\\
225.01	0\\
226.01	0\\
227.01	0\\
228.01	0\\
229.01	0\\
230.01	0\\
231.01	0\\
232.01	0\\
233.01	0\\
234.01	0\\
235.01	0\\
236.01	0\\
237.01	0\\
238.01	0\\
239.01	0\\
240.01	0\\
241.01	0\\
242.01	0\\
243.01	0\\
244.01	0\\
245.01	0\\
246.01	0\\
247.01	0\\
248.01	0\\
249.01	0\\
250.01	0\\
251.01	0\\
252.01	0\\
253.01	0\\
254.01	0\\
255.01	0\\
256.01	0\\
257.01	0\\
258.01	0\\
259.01	0\\
260.01	0\\
261.01	0\\
262.01	0\\
263.01	0\\
264.01	0\\
265.01	0\\
266.01	0\\
267.01	0\\
268.01	0\\
269.01	0\\
270.01	0\\
271.01	0\\
272.01	0\\
273.01	0\\
274.01	0\\
275.01	0\\
276.01	0\\
277.01	0\\
278.01	0\\
279.01	0\\
280.01	0\\
281.01	0\\
282.01	0\\
283.01	0\\
284.01	0\\
285.01	0\\
286.01	0\\
287.01	0\\
288.01	0\\
289.01	0\\
290.01	0\\
291.01	0\\
292.01	0\\
293.01	0\\
294.01	0\\
295.01	0\\
296.01	0\\
297.01	0\\
298.01	0\\
299.01	0\\
300.01	0\\
301.01	0\\
302.01	0\\
303.01	0\\
304.01	0\\
305.01	0\\
306.01	0\\
307.01	0\\
308.01	0\\
309.01	0\\
310.01	0\\
311.01	0\\
312.01	0\\
313.01	0\\
314.01	0\\
315.01	0\\
316.01	0\\
317.01	0\\
318.01	0\\
319.01	0\\
320.01	0\\
321.01	0\\
322.01	0\\
323.01	0\\
324.01	0\\
325.01	0\\
326.01	0\\
327.01	0\\
328.01	0\\
329.01	0\\
330.01	0\\
331.01	0\\
332.01	0\\
333.01	0\\
334.01	0\\
335.01	0\\
336.01	0\\
337.01	0\\
338.01	0\\
339.01	0\\
340.01	0\\
341.01	0\\
342.01	0\\
343.01	0\\
344.01	0\\
345.01	0\\
346.01	0\\
347.01	0\\
348.01	0\\
349.01	0\\
350.01	0\\
351.01	0\\
352.01	0\\
353.01	0\\
354.01	0\\
355.01	0\\
356.01	0\\
357.01	0\\
358.01	0\\
359.01	0\\
360.01	0\\
361.01	0\\
362.01	0\\
363.01	0\\
364.01	0\\
365.01	0\\
366.01	0\\
367.01	0\\
368.01	0\\
369.01	0\\
370.01	0\\
371.01	0\\
372.01	0\\
373.01	0\\
374.01	0\\
375.01	0\\
376.01	0\\
377.01	0\\
378.01	0\\
379.01	0\\
380.01	0\\
381.01	0\\
382.01	0\\
383.01	0\\
384.01	0\\
385.01	0\\
386.01	0\\
387.01	0\\
388.01	0\\
389.01	0\\
390.01	0\\
391.01	0\\
392.01	0\\
393.01	0\\
394.01	0\\
395.01	0\\
396.01	0\\
397.01	0\\
398.01	0\\
399.01	0\\
400.01	0\\
401.01	0\\
402.01	0\\
403.01	0\\
404.01	0\\
405.01	0\\
406.01	0\\
407.01	0\\
408.01	0\\
409.01	0\\
410.01	0\\
411.01	0\\
412.01	0\\
413.01	0\\
414.01	0\\
415.01	0\\
416.01	0\\
417.01	0\\
418.01	0\\
419.01	0\\
420.01	0\\
421.01	0\\
422.01	0\\
423.01	0\\
424.01	0\\
425.01	0\\
426.01	0\\
427.01	0\\
428.01	0\\
429.01	0\\
430.01	0\\
431.01	0\\
432.01	0\\
433.01	0\\
434.01	0\\
435.01	0\\
436.01	0\\
437.01	0\\
438.01	0\\
439.01	0\\
440.01	0\\
441.01	0\\
442.01	0\\
443.01	0\\
444.01	0\\
445.01	0\\
446.01	0\\
447.01	0\\
448.01	0\\
449.01	0\\
450.01	0\\
451.01	1.73472347597681e-18\\
452.01	0\\
453.01	0\\
454.01	0\\
455.01	0\\
456.01	0\\
457.01	0\\
458.01	0\\
459.01	0\\
460.01	0\\
461.01	0\\
462.01	0\\
463.01	0\\
464.01	0\\
465.01	0\\
466.01	0\\
467.01	0\\
468.01	0\\
469.01	1.73472347597681e-18\\
470.01	0\\
471.01	0\\
472.01	0\\
473.01	0\\
474.01	0\\
475.01	0\\
476.01	1.73472347597681e-18\\
477.01	0\\
478.01	0\\
479.01	0\\
480.01	0\\
481.01	0\\
482.01	0\\
483.01	0\\
484.01	0\\
485.01	0\\
486.01	0\\
487.01	0\\
488.01	0\\
489.01	0\\
490.01	0\\
491.01	0\\
492.01	0\\
493.01	0\\
494.01	0\\
495.01	0\\
496.01	0\\
497.01	0\\
498.01	0\\
499.01	0\\
500.01	0\\
501.01	0\\
502.01	0\\
503.01	0\\
504.01	0\\
505.01	0\\
506.01	0\\
507.01	0\\
508.01	0\\
509.01	0\\
510.01	0\\
511.01	0\\
512.01	1.73472347597681e-18\\
513.01	0\\
514.01	0\\
515.01	0\\
516.01	0\\
517.01	0\\
518.01	0\\
519.01	0\\
520.01	0\\
521.01	0\\
522.01	0\\
523.01	1.73472347597681e-18\\
524.01	0\\
525.01	0\\
526.01	0\\
527.01	0\\
528.01	0\\
529.01	0\\
530.01	0\\
531.01	0\\
532.01	0\\
533.01	0\\
534.01	0\\
535.01	0\\
536.01	1.73472347597681e-18\\
537.01	1.73472347597681e-18\\
538.01	1.73472347597681e-18\\
539.01	0\\
540.01	0\\
541.01	0\\
542.01	0\\
543.01	0\\
544.01	0\\
545.01	1.73472347597681e-18\\
546.01	1.73472347597681e-18\\
547.01	0\\
548.01	0\\
549.01	0\\
550.01	0\\
551.01	0\\
552.01	0\\
553.01	0\\
554.01	0\\
555.01	0\\
556.01	1.73472347597681e-18\\
557.01	0\\
558.01	0\\
559.01	0\\
560.01	1.73472347597681e-18\\
561.01	1.73472347597681e-18\\
562.01	0\\
563.01	0\\
564.01	0\\
565.01	0\\
566.01	0\\
567.01	0\\
568.01	0\\
569.01	0\\
570.01	0\\
571.01	0\\
572.01	0\\
573.01	0\\
574.01	0\\
575.01	0\\
576.01	0\\
577.01	0\\
578.01	1.73472347597681e-18\\
579.01	0\\
580.01	0\\
581.01	0\\
582.01	0\\
583.01	1.73472347597681e-18\\
584.01	0\\
585.01	0\\
586.01	0\\
587.01	0\\
588.01	0\\
589.01	0\\
590.01	0\\
591.01	0\\
592.01	0\\
593.01	0\\
594.01	0\\
595.01	0\\
596.01	0\\
597.01	0\\
598.01	0\\
599.01	0\\
599.02	0\\
599.03	0\\
599.04	0\\
599.05	0\\
599.06	0\\
599.07	0\\
599.08	0\\
599.09	0\\
599.1	0\\
599.11	0\\
599.12	0\\
599.13	0\\
599.14	0\\
599.15	0\\
599.16	0\\
599.17	0\\
599.18	0\\
599.19	0\\
599.2	0\\
599.21	0\\
599.22	0\\
599.23	0\\
599.24	0\\
599.25	0\\
599.26	0\\
599.27	0\\
599.28	0\\
599.29	0\\
599.3	0\\
599.31	0\\
599.32	0\\
599.33	0\\
599.34	0\\
599.35	0\\
599.36	0\\
599.37	0\\
599.38	0\\
599.39	0\\
599.4	0\\
599.41	0\\
599.42	0\\
599.43	0\\
599.44	0\\
599.45	0\\
599.46	0\\
599.47	0\\
599.48	0\\
599.49	0\\
599.5	0\\
599.51	0\\
599.52	0\\
599.53	0\\
599.54	0\\
599.55	0\\
599.56	0\\
599.57	0\\
599.58	0\\
599.59	0\\
599.6	0\\
599.61	0\\
599.62	0\\
599.63	0\\
599.64	0\\
599.65	0\\
599.66	0\\
599.67	0\\
599.68	0\\
599.69	0\\
599.7	0\\
599.71	0\\
599.72	0\\
599.73	0\\
599.74	0\\
599.75	0\\
599.76	0\\
599.77	0\\
599.78	0\\
599.79	0\\
599.8	0\\
599.81	0\\
599.82	0\\
599.83	0\\
599.84	0\\
599.85	0\\
599.86	0\\
599.87	0\\
599.88	0\\
599.89	0\\
599.9	0\\
599.91	0\\
599.92	0\\
599.93	0\\
599.94	0\\
599.95	0\\
599.96	0\\
599.97	0\\
599.98	0\\
599.99	0\\
600	0\\
};
\addplot [color=blue!75!mycolor7,solid,forget plot]
  table[row sep=crcr]{%
0.01	0\\
1.01	0\\
2.01	0\\
3.01	0\\
4.01	0\\
5.01	0\\
6.01	0\\
7.01	0\\
8.01	0\\
9.01	0\\
10.01	0\\
11.01	0\\
12.01	0\\
13.01	0\\
14.01	0\\
15.01	0\\
16.01	0\\
17.01	0\\
18.01	0\\
19.01	0\\
20.01	0\\
21.01	0\\
22.01	0\\
23.01	0\\
24.01	0\\
25.01	0\\
26.01	0\\
27.01	0\\
28.01	0\\
29.01	0\\
30.01	0\\
31.01	0\\
32.01	0\\
33.01	0\\
34.01	0\\
35.01	0\\
36.01	0\\
37.01	0\\
38.01	0\\
39.01	0\\
40.01	0\\
41.01	0\\
42.01	0\\
43.01	0\\
44.01	0\\
45.01	0\\
46.01	0\\
47.01	0\\
48.01	0\\
49.01	0\\
50.01	0\\
51.01	0\\
52.01	0\\
53.01	0\\
54.01	0\\
55.01	0\\
56.01	0\\
57.01	0\\
58.01	0\\
59.01	0\\
60.01	0\\
61.01	0\\
62.01	0\\
63.01	0\\
64.01	0\\
65.01	0\\
66.01	0\\
67.01	0\\
68.01	0\\
69.01	0\\
70.01	0\\
71.01	0\\
72.01	0\\
73.01	0\\
74.01	0\\
75.01	0\\
76.01	0\\
77.01	0\\
78.01	0\\
79.01	0\\
80.01	0\\
81.01	0\\
82.01	0\\
83.01	0\\
84.01	0\\
85.01	0\\
86.01	0\\
87.01	0\\
88.01	0\\
89.01	0\\
90.01	0\\
91.01	0\\
92.01	0\\
93.01	0\\
94.01	0\\
95.01	0\\
96.01	0\\
97.01	0\\
98.01	0\\
99.01	0\\
100.01	0\\
101.01	0\\
102.01	0\\
103.01	0\\
104.01	0\\
105.01	0\\
106.01	0\\
107.01	0\\
108.01	0\\
109.01	0\\
110.01	0\\
111.01	0\\
112.01	0\\
113.01	0\\
114.01	0\\
115.01	0\\
116.01	0\\
117.01	0\\
118.01	0\\
119.01	0\\
120.01	0\\
121.01	0\\
122.01	0\\
123.01	0\\
124.01	0\\
125.01	0\\
126.01	0\\
127.01	0\\
128.01	0\\
129.01	0\\
130.01	0\\
131.01	0\\
132.01	0\\
133.01	0\\
134.01	0\\
135.01	0\\
136.01	0\\
137.01	0\\
138.01	0\\
139.01	0\\
140.01	0\\
141.01	0\\
142.01	0\\
143.01	0\\
144.01	0\\
145.01	0\\
146.01	0\\
147.01	0\\
148.01	0\\
149.01	0\\
150.01	0\\
151.01	0\\
152.01	0\\
153.01	0\\
154.01	0\\
155.01	0\\
156.01	0\\
157.01	0\\
158.01	0\\
159.01	0\\
160.01	0\\
161.01	0\\
162.01	0\\
163.01	0\\
164.01	0\\
165.01	0\\
166.01	0\\
167.01	0\\
168.01	0\\
169.01	0\\
170.01	0\\
171.01	0\\
172.01	0\\
173.01	0\\
174.01	0\\
175.01	0\\
176.01	0\\
177.01	0\\
178.01	0\\
179.01	0\\
180.01	0\\
181.01	0\\
182.01	0\\
183.01	0\\
184.01	0\\
185.01	0\\
186.01	0\\
187.01	0\\
188.01	0\\
189.01	0\\
190.01	0\\
191.01	0\\
192.01	0\\
193.01	0\\
194.01	0\\
195.01	0\\
196.01	0\\
197.01	0\\
198.01	0\\
199.01	0\\
200.01	0\\
201.01	0\\
202.01	0\\
203.01	0\\
204.01	0\\
205.01	0\\
206.01	0\\
207.01	0\\
208.01	0\\
209.01	0\\
210.01	0\\
211.01	0\\
212.01	0\\
213.01	0\\
214.01	0\\
215.01	0\\
216.01	0\\
217.01	0\\
218.01	0\\
219.01	0\\
220.01	0\\
221.01	0\\
222.01	0\\
223.01	0\\
224.01	0\\
225.01	0\\
226.01	0\\
227.01	0\\
228.01	0\\
229.01	0\\
230.01	0\\
231.01	0\\
232.01	0\\
233.01	0\\
234.01	0\\
235.01	0\\
236.01	0\\
237.01	0\\
238.01	0\\
239.01	0\\
240.01	0\\
241.01	0\\
242.01	0\\
243.01	0\\
244.01	0\\
245.01	0\\
246.01	0\\
247.01	0\\
248.01	0\\
249.01	0\\
250.01	0\\
251.01	0\\
252.01	0\\
253.01	0\\
254.01	0\\
255.01	0\\
256.01	0\\
257.01	0\\
258.01	0\\
259.01	0\\
260.01	0\\
261.01	0\\
262.01	0\\
263.01	0\\
264.01	0\\
265.01	0\\
266.01	0\\
267.01	0\\
268.01	0\\
269.01	0\\
270.01	0\\
271.01	0\\
272.01	0\\
273.01	0\\
274.01	0\\
275.01	0\\
276.01	0\\
277.01	0\\
278.01	0\\
279.01	0\\
280.01	0\\
281.01	0\\
282.01	0\\
283.01	0\\
284.01	0\\
285.01	0\\
286.01	0\\
287.01	0\\
288.01	0\\
289.01	0\\
290.01	0\\
291.01	0\\
292.01	0\\
293.01	0\\
294.01	0\\
295.01	0\\
296.01	0\\
297.01	0\\
298.01	0\\
299.01	0\\
300.01	0\\
301.01	0\\
302.01	0\\
303.01	0\\
304.01	0\\
305.01	0\\
306.01	0\\
307.01	0\\
308.01	0\\
309.01	0\\
310.01	0\\
311.01	0\\
312.01	0\\
313.01	0\\
314.01	0\\
315.01	0\\
316.01	0\\
317.01	0\\
318.01	0\\
319.01	0\\
320.01	0\\
321.01	0\\
322.01	0\\
323.01	0\\
324.01	0\\
325.01	0\\
326.01	0\\
327.01	0\\
328.01	0\\
329.01	0\\
330.01	0\\
331.01	0\\
332.01	0\\
333.01	0\\
334.01	0\\
335.01	0\\
336.01	0\\
337.01	0\\
338.01	0\\
339.01	0\\
340.01	0\\
341.01	0\\
342.01	0\\
343.01	0\\
344.01	0\\
345.01	0\\
346.01	0\\
347.01	0\\
348.01	0\\
349.01	0\\
350.01	0\\
351.01	0\\
352.01	0\\
353.01	0\\
354.01	0\\
355.01	0\\
356.01	0\\
357.01	0\\
358.01	0\\
359.01	0\\
360.01	0\\
361.01	0\\
362.01	0\\
363.01	0\\
364.01	0\\
365.01	0\\
366.01	0\\
367.01	0\\
368.01	0\\
369.01	0\\
370.01	0\\
371.01	0\\
372.01	0\\
373.01	0\\
374.01	0\\
375.01	0\\
376.01	0\\
377.01	0\\
378.01	0\\
379.01	0\\
380.01	0\\
381.01	0\\
382.01	0\\
383.01	0\\
384.01	0\\
385.01	0\\
386.01	0\\
387.01	0\\
388.01	0\\
389.01	0\\
390.01	0\\
391.01	0\\
392.01	0\\
393.01	0\\
394.01	0\\
395.01	0\\
396.01	0\\
397.01	0\\
398.01	0\\
399.01	0\\
400.01	0\\
401.01	0\\
402.01	0\\
403.01	0\\
404.01	0\\
405.01	0\\
406.01	0\\
407.01	0\\
408.01	0\\
409.01	0\\
410.01	0\\
411.01	0\\
412.01	0\\
413.01	0\\
414.01	0\\
415.01	0\\
416.01	0\\
417.01	0\\
418.01	0\\
419.01	0\\
420.01	0\\
421.01	0\\
422.01	0\\
423.01	0\\
424.01	0\\
425.01	0\\
426.01	0\\
427.01	0\\
428.01	0\\
429.01	0\\
430.01	0\\
431.01	0\\
432.01	0\\
433.01	0\\
434.01	0\\
435.01	0\\
436.01	0\\
437.01	0\\
438.01	0\\
439.01	0\\
440.01	0\\
441.01	0\\
442.01	0\\
443.01	0\\
444.01	0\\
445.01	0\\
446.01	0\\
447.01	0\\
448.01	0\\
449.01	0\\
450.01	0\\
451.01	1.73472347597681e-18\\
452.01	0\\
453.01	0\\
454.01	0\\
455.01	0\\
456.01	0\\
457.01	0\\
458.01	0\\
459.01	0\\
460.01	0\\
461.01	0\\
462.01	0\\
463.01	0\\
464.01	0\\
465.01	0\\
466.01	0\\
467.01	0\\
468.01	0\\
469.01	1.73472347597681e-18\\
470.01	0\\
471.01	0\\
472.01	0\\
473.01	0\\
474.01	0\\
475.01	0\\
476.01	1.73472347597681e-18\\
477.01	0\\
478.01	0\\
479.01	0\\
480.01	0\\
481.01	0\\
482.01	0\\
483.01	0\\
484.01	0\\
485.01	0\\
486.01	0\\
487.01	0\\
488.01	0\\
489.01	0\\
490.01	0\\
491.01	0\\
492.01	0\\
493.01	0\\
494.01	0\\
495.01	0\\
496.01	0\\
497.01	0\\
498.01	0\\
499.01	0\\
500.01	0\\
501.01	0\\
502.01	0\\
503.01	0\\
504.01	0\\
505.01	0\\
506.01	0\\
507.01	0\\
508.01	0\\
509.01	0\\
510.01	0\\
511.01	0\\
512.01	1.73472347597681e-18\\
513.01	0\\
514.01	0\\
515.01	0\\
516.01	0\\
517.01	0\\
518.01	0\\
519.01	0\\
520.01	0\\
521.01	0\\
522.01	0\\
523.01	1.73472347597681e-18\\
524.01	0\\
525.01	0\\
526.01	0\\
527.01	0\\
528.01	0\\
529.01	0\\
530.01	0\\
531.01	0\\
532.01	0\\
533.01	0\\
534.01	0\\
535.01	0\\
536.01	1.73472347597681e-18\\
537.01	1.73472347597681e-18\\
538.01	1.73472347597681e-18\\
539.01	0\\
540.01	0\\
541.01	0\\
542.01	0\\
543.01	0\\
544.01	0\\
545.01	1.73472347597681e-18\\
546.01	1.73472347597681e-18\\
547.01	0\\
548.01	0\\
549.01	0\\
550.01	0\\
551.01	0\\
552.01	0\\
553.01	0\\
554.01	0\\
555.01	0\\
556.01	1.73472347597681e-18\\
557.01	0\\
558.01	0\\
559.01	0\\
560.01	1.73472347597681e-18\\
561.01	1.73472347597681e-18\\
562.01	0\\
563.01	0\\
564.01	0\\
565.01	0\\
566.01	0\\
567.01	0\\
568.01	0\\
569.01	0\\
570.01	0\\
571.01	0\\
572.01	0\\
573.01	0\\
574.01	0\\
575.01	0\\
576.01	0\\
577.01	0\\
578.01	1.73472347597681e-18\\
579.01	0\\
580.01	0\\
581.01	0\\
582.01	0\\
583.01	1.73472347597681e-18\\
584.01	0\\
585.01	0\\
586.01	0\\
587.01	0\\
588.01	0\\
589.01	0\\
590.01	0\\
591.01	0\\
592.01	0\\
593.01	0\\
594.01	0\\
595.01	0\\
596.01	0\\
597.01	0\\
598.01	0\\
599.01	0\\
599.02	0\\
599.03	0\\
599.04	0\\
599.05	0\\
599.06	0\\
599.07	0\\
599.08	0\\
599.09	0\\
599.1	0\\
599.11	0\\
599.12	0\\
599.13	0\\
599.14	0\\
599.15	0\\
599.16	0\\
599.17	0\\
599.18	0\\
599.19	0\\
599.2	0\\
599.21	0\\
599.22	0\\
599.23	0\\
599.24	0\\
599.25	0\\
599.26	0\\
599.27	0\\
599.28	0\\
599.29	0\\
599.3	0\\
599.31	0\\
599.32	0\\
599.33	0\\
599.34	0\\
599.35	0\\
599.36	0\\
599.37	0\\
599.38	0\\
599.39	0\\
599.4	0\\
599.41	0\\
599.42	0\\
599.43	0\\
599.44	0\\
599.45	0\\
599.46	0\\
599.47	0\\
599.48	0\\
599.49	0\\
599.5	0\\
599.51	0\\
599.52	0\\
599.53	0\\
599.54	0\\
599.55	0\\
599.56	0\\
599.57	0\\
599.58	0\\
599.59	0\\
599.6	0\\
599.61	0\\
599.62	0\\
599.63	0\\
599.64	0\\
599.65	0\\
599.66	0\\
599.67	0\\
599.68	0\\
599.69	0\\
599.7	0\\
599.71	0\\
599.72	0\\
599.73	0\\
599.74	0\\
599.75	0\\
599.76	0\\
599.77	0\\
599.78	0\\
599.79	0\\
599.8	0\\
599.81	0\\
599.82	0\\
599.83	0\\
599.84	0\\
599.85	0\\
599.86	0\\
599.87	0\\
599.88	0\\
599.89	0\\
599.9	0\\
599.91	0\\
599.92	0\\
599.93	0\\
599.94	0\\
599.95	0\\
599.96	0\\
599.97	0\\
599.98	0\\
599.99	0\\
600	0\\
};
\addplot [color=blue!80!mycolor9,solid,forget plot]
  table[row sep=crcr]{%
0.01	0\\
1.01	0\\
2.01	0\\
3.01	0\\
4.01	0\\
5.01	0\\
6.01	0\\
7.01	0\\
8.01	0\\
9.01	0\\
10.01	0\\
11.01	0\\
12.01	0\\
13.01	0\\
14.01	0\\
15.01	0\\
16.01	0\\
17.01	0\\
18.01	0\\
19.01	0\\
20.01	0\\
21.01	0\\
22.01	0\\
23.01	0\\
24.01	0\\
25.01	0\\
26.01	0\\
27.01	0\\
28.01	0\\
29.01	0\\
30.01	0\\
31.01	0\\
32.01	0\\
33.01	0\\
34.01	0\\
35.01	0\\
36.01	0\\
37.01	0\\
38.01	0\\
39.01	0\\
40.01	0\\
41.01	0\\
42.01	0\\
43.01	0\\
44.01	0\\
45.01	0\\
46.01	0\\
47.01	0\\
48.01	0\\
49.01	0\\
50.01	0\\
51.01	0\\
52.01	0\\
53.01	0\\
54.01	0\\
55.01	0\\
56.01	0\\
57.01	0\\
58.01	0\\
59.01	0\\
60.01	0\\
61.01	0\\
62.01	0\\
63.01	0\\
64.01	0\\
65.01	0\\
66.01	0\\
67.01	0\\
68.01	0\\
69.01	0\\
70.01	0\\
71.01	0\\
72.01	0\\
73.01	0\\
74.01	0\\
75.01	0\\
76.01	0\\
77.01	0\\
78.01	0\\
79.01	0\\
80.01	0\\
81.01	0\\
82.01	0\\
83.01	0\\
84.01	0\\
85.01	0\\
86.01	0\\
87.01	0\\
88.01	0\\
89.01	0\\
90.01	0\\
91.01	0\\
92.01	0\\
93.01	0\\
94.01	0\\
95.01	0\\
96.01	0\\
97.01	0\\
98.01	0\\
99.01	0\\
100.01	0\\
101.01	0\\
102.01	0\\
103.01	0\\
104.01	0\\
105.01	0\\
106.01	0\\
107.01	0\\
108.01	0\\
109.01	0\\
110.01	0\\
111.01	0\\
112.01	0\\
113.01	0\\
114.01	0\\
115.01	0\\
116.01	0\\
117.01	0\\
118.01	0\\
119.01	0\\
120.01	0\\
121.01	0\\
122.01	0\\
123.01	0\\
124.01	0\\
125.01	0\\
126.01	0\\
127.01	0\\
128.01	0\\
129.01	0\\
130.01	0\\
131.01	0\\
132.01	0\\
133.01	0\\
134.01	0\\
135.01	0\\
136.01	0\\
137.01	0\\
138.01	0\\
139.01	0\\
140.01	0\\
141.01	0\\
142.01	0\\
143.01	0\\
144.01	0\\
145.01	0\\
146.01	0\\
147.01	0\\
148.01	0\\
149.01	0\\
150.01	0\\
151.01	0\\
152.01	0\\
153.01	0\\
154.01	0\\
155.01	0\\
156.01	0\\
157.01	0\\
158.01	0\\
159.01	0\\
160.01	0\\
161.01	0\\
162.01	0\\
163.01	0\\
164.01	0\\
165.01	0\\
166.01	0\\
167.01	0\\
168.01	0\\
169.01	0\\
170.01	0\\
171.01	0\\
172.01	0\\
173.01	0\\
174.01	0\\
175.01	0\\
176.01	0\\
177.01	0\\
178.01	0\\
179.01	0\\
180.01	0\\
181.01	0\\
182.01	0\\
183.01	0\\
184.01	0\\
185.01	0\\
186.01	0\\
187.01	0\\
188.01	0\\
189.01	0\\
190.01	0\\
191.01	0\\
192.01	0\\
193.01	0\\
194.01	0\\
195.01	0\\
196.01	0\\
197.01	0\\
198.01	0\\
199.01	0\\
200.01	0\\
201.01	0\\
202.01	0\\
203.01	0\\
204.01	0\\
205.01	0\\
206.01	0\\
207.01	0\\
208.01	0\\
209.01	0\\
210.01	0\\
211.01	0\\
212.01	0\\
213.01	0\\
214.01	0\\
215.01	0\\
216.01	0\\
217.01	0\\
218.01	0\\
219.01	0\\
220.01	0\\
221.01	0\\
222.01	0\\
223.01	0\\
224.01	0\\
225.01	0\\
226.01	0\\
227.01	0\\
228.01	0\\
229.01	0\\
230.01	0\\
231.01	0\\
232.01	0\\
233.01	0\\
234.01	0\\
235.01	0\\
236.01	0\\
237.01	0\\
238.01	0\\
239.01	0\\
240.01	0\\
241.01	0\\
242.01	0\\
243.01	0\\
244.01	0\\
245.01	0\\
246.01	0\\
247.01	0\\
248.01	0\\
249.01	0\\
250.01	0\\
251.01	0\\
252.01	0\\
253.01	0\\
254.01	0\\
255.01	0\\
256.01	0\\
257.01	0\\
258.01	0\\
259.01	0\\
260.01	0\\
261.01	0\\
262.01	0\\
263.01	0\\
264.01	0\\
265.01	0\\
266.01	0\\
267.01	0\\
268.01	0\\
269.01	0\\
270.01	0\\
271.01	0\\
272.01	0\\
273.01	0\\
274.01	0\\
275.01	0\\
276.01	0\\
277.01	0\\
278.01	0\\
279.01	0\\
280.01	0\\
281.01	0\\
282.01	0\\
283.01	0\\
284.01	0\\
285.01	0\\
286.01	0\\
287.01	0\\
288.01	0\\
289.01	0\\
290.01	0\\
291.01	0\\
292.01	0\\
293.01	0\\
294.01	0\\
295.01	0\\
296.01	0\\
297.01	0\\
298.01	0\\
299.01	0\\
300.01	0\\
301.01	0\\
302.01	0\\
303.01	0\\
304.01	0\\
305.01	0\\
306.01	0\\
307.01	0\\
308.01	0\\
309.01	0\\
310.01	0\\
311.01	0\\
312.01	0\\
313.01	0\\
314.01	0\\
315.01	0\\
316.01	0\\
317.01	0\\
318.01	0\\
319.01	0\\
320.01	0\\
321.01	0\\
322.01	0\\
323.01	0\\
324.01	0\\
325.01	0\\
326.01	0\\
327.01	0\\
328.01	0\\
329.01	0\\
330.01	0\\
331.01	0\\
332.01	0\\
333.01	0\\
334.01	0\\
335.01	0\\
336.01	0\\
337.01	0\\
338.01	0\\
339.01	0\\
340.01	0\\
341.01	0\\
342.01	0\\
343.01	0\\
344.01	0\\
345.01	0\\
346.01	0\\
347.01	0\\
348.01	0\\
349.01	0\\
350.01	0\\
351.01	0\\
352.01	0\\
353.01	0\\
354.01	0\\
355.01	0\\
356.01	0\\
357.01	0\\
358.01	0\\
359.01	0\\
360.01	0\\
361.01	0\\
362.01	0\\
363.01	0\\
364.01	0\\
365.01	0\\
366.01	0\\
367.01	0\\
368.01	0\\
369.01	0\\
370.01	0\\
371.01	0\\
372.01	0\\
373.01	0\\
374.01	0\\
375.01	0\\
376.01	0\\
377.01	0\\
378.01	0\\
379.01	0\\
380.01	0\\
381.01	0\\
382.01	0\\
383.01	0\\
384.01	0\\
385.01	0\\
386.01	0\\
387.01	0\\
388.01	0\\
389.01	0\\
390.01	0\\
391.01	0\\
392.01	0\\
393.01	0\\
394.01	0\\
395.01	0\\
396.01	0\\
397.01	0\\
398.01	0\\
399.01	0\\
400.01	0\\
401.01	0\\
402.01	0\\
403.01	0\\
404.01	0\\
405.01	0\\
406.01	0\\
407.01	0\\
408.01	0\\
409.01	0\\
410.01	0\\
411.01	0\\
412.01	0\\
413.01	0\\
414.01	0\\
415.01	0\\
416.01	0\\
417.01	0\\
418.01	0\\
419.01	0\\
420.01	0\\
421.01	0\\
422.01	0\\
423.01	0\\
424.01	0\\
425.01	0\\
426.01	0\\
427.01	0\\
428.01	0\\
429.01	0\\
430.01	0\\
431.01	0\\
432.01	0\\
433.01	0\\
434.01	0\\
435.01	0\\
436.01	0\\
437.01	0\\
438.01	0\\
439.01	0\\
440.01	0\\
441.01	0\\
442.01	0\\
443.01	0\\
444.01	0\\
445.01	0\\
446.01	0\\
447.01	0\\
448.01	0\\
449.01	0\\
450.01	0\\
451.01	1.73472347597681e-18\\
452.01	0\\
453.01	0\\
454.01	0\\
455.01	0\\
456.01	0\\
457.01	0\\
458.01	0\\
459.01	0\\
460.01	0\\
461.01	0\\
462.01	0\\
463.01	0\\
464.01	0\\
465.01	0\\
466.01	0\\
467.01	0\\
468.01	0\\
469.01	1.73472347597681e-18\\
470.01	0\\
471.01	0\\
472.01	0\\
473.01	0\\
474.01	0\\
475.01	0\\
476.01	1.73472347597681e-18\\
477.01	0\\
478.01	0\\
479.01	0\\
480.01	0\\
481.01	0\\
482.01	0\\
483.01	0\\
484.01	0\\
485.01	0\\
486.01	0\\
487.01	0\\
488.01	0\\
489.01	0\\
490.01	0\\
491.01	0\\
492.01	0\\
493.01	0\\
494.01	0\\
495.01	0\\
496.01	0\\
497.01	0\\
498.01	0\\
499.01	0\\
500.01	0\\
501.01	0\\
502.01	0\\
503.01	0\\
504.01	0\\
505.01	0\\
506.01	0\\
507.01	0\\
508.01	0\\
509.01	0\\
510.01	0\\
511.01	0\\
512.01	1.73472347597681e-18\\
513.01	0\\
514.01	0\\
515.01	0\\
516.01	0\\
517.01	0\\
518.01	0\\
519.01	0\\
520.01	0\\
521.01	0\\
522.01	0\\
523.01	1.73472347597681e-18\\
524.01	0\\
525.01	0\\
526.01	0\\
527.01	0\\
528.01	0\\
529.01	0\\
530.01	0\\
531.01	0\\
532.01	0\\
533.01	0\\
534.01	0\\
535.01	0\\
536.01	1.73472347597681e-18\\
537.01	1.73472347597681e-18\\
538.01	1.73472347597681e-18\\
539.01	0\\
540.01	0\\
541.01	0\\
542.01	0\\
543.01	0\\
544.01	0\\
545.01	1.73472347597681e-18\\
546.01	1.73472347597681e-18\\
547.01	0\\
548.01	0\\
549.01	0\\
550.01	0\\
551.01	0\\
552.01	0\\
553.01	0\\
554.01	0\\
555.01	0\\
556.01	1.73472347597681e-18\\
557.01	0\\
558.01	0\\
559.01	0\\
560.01	1.73472347597681e-18\\
561.01	1.73472347597681e-18\\
562.01	0\\
563.01	0\\
564.01	0\\
565.01	0\\
566.01	0\\
567.01	0\\
568.01	0\\
569.01	0\\
570.01	0\\
571.01	0\\
572.01	0\\
573.01	0\\
574.01	0\\
575.01	0\\
576.01	0\\
577.01	0\\
578.01	1.73472347597681e-18\\
579.01	0\\
580.01	0\\
581.01	0\\
582.01	0\\
583.01	1.73472347597681e-18\\
584.01	0\\
585.01	0\\
586.01	0\\
587.01	0\\
588.01	0\\
589.01	0\\
590.01	0\\
591.01	0\\
592.01	0\\
593.01	0\\
594.01	0\\
595.01	0\\
596.01	0\\
597.01	0\\
598.01	0\\
599.01	0\\
599.02	0\\
599.03	0\\
599.04	0\\
599.05	0\\
599.06	0\\
599.07	0\\
599.08	0\\
599.09	0\\
599.1	0\\
599.11	0\\
599.12	0\\
599.13	0\\
599.14	0\\
599.15	0\\
599.16	0\\
599.17	0\\
599.18	0\\
599.19	0\\
599.2	0\\
599.21	0\\
599.22	0\\
599.23	0\\
599.24	0\\
599.25	0\\
599.26	0\\
599.27	0\\
599.28	0\\
599.29	0\\
599.3	0\\
599.31	0\\
599.32	0\\
599.33	0\\
599.34	0\\
599.35	0\\
599.36	0\\
599.37	0\\
599.38	0\\
599.39	0\\
599.4	0\\
599.41	0\\
599.42	0\\
599.43	0\\
599.44	0\\
599.45	0\\
599.46	0\\
599.47	0\\
599.48	0\\
599.49	0\\
599.5	0\\
599.51	0\\
599.52	0\\
599.53	0\\
599.54	0\\
599.55	0\\
599.56	0\\
599.57	0\\
599.58	0\\
599.59	0\\
599.6	0\\
599.61	0\\
599.62	0\\
599.63	0\\
599.64	0\\
599.65	0\\
599.66	0\\
599.67	0\\
599.68	0\\
599.69	0\\
599.7	0\\
599.71	0\\
599.72	0\\
599.73	0\\
599.74	0\\
599.75	0\\
599.76	0\\
599.77	0\\
599.78	0\\
599.79	0\\
599.8	0\\
599.81	0\\
599.82	0\\
599.83	0\\
599.84	0\\
599.85	0\\
599.86	0\\
599.87	0\\
599.88	0\\
599.89	0\\
599.9	0\\
599.91	0\\
599.92	0\\
599.93	0\\
599.94	0\\
599.95	0\\
599.96	0\\
599.97	0\\
599.98	0\\
599.99	0\\
600	0\\
};
\addplot [color=blue,solid,forget plot]
  table[row sep=crcr]{%
0.01	0\\
1.01	0\\
2.01	0\\
3.01	0\\
4.01	0\\
5.01	0\\
6.01	0\\
7.01	0\\
8.01	0\\
9.01	0\\
10.01	0\\
11.01	0\\
12.01	0\\
13.01	0\\
14.01	0\\
15.01	0\\
16.01	0\\
17.01	0\\
18.01	0\\
19.01	0\\
20.01	0\\
21.01	0\\
22.01	0\\
23.01	0\\
24.01	0\\
25.01	0\\
26.01	0\\
27.01	0\\
28.01	0\\
29.01	0\\
30.01	0\\
31.01	0\\
32.01	0\\
33.01	0\\
34.01	0\\
35.01	0\\
36.01	0\\
37.01	0\\
38.01	0\\
39.01	0\\
40.01	0\\
41.01	0\\
42.01	0\\
43.01	0\\
44.01	0\\
45.01	0\\
46.01	0\\
47.01	0\\
48.01	0\\
49.01	0\\
50.01	0\\
51.01	0\\
52.01	0\\
53.01	0\\
54.01	0\\
55.01	0\\
56.01	0\\
57.01	0\\
58.01	0\\
59.01	0\\
60.01	0\\
61.01	0\\
62.01	0\\
63.01	0\\
64.01	0\\
65.01	0\\
66.01	0\\
67.01	0\\
68.01	0\\
69.01	0\\
70.01	0\\
71.01	0\\
72.01	0\\
73.01	0\\
74.01	0\\
75.01	0\\
76.01	0\\
77.01	0\\
78.01	0\\
79.01	0\\
80.01	0\\
81.01	0\\
82.01	0\\
83.01	0\\
84.01	0\\
85.01	0\\
86.01	0\\
87.01	0\\
88.01	0\\
89.01	0\\
90.01	0\\
91.01	0\\
92.01	0\\
93.01	0\\
94.01	0\\
95.01	0\\
96.01	0\\
97.01	0\\
98.01	0\\
99.01	0\\
100.01	0\\
101.01	0\\
102.01	0\\
103.01	0\\
104.01	0\\
105.01	0\\
106.01	0\\
107.01	0\\
108.01	0\\
109.01	0\\
110.01	0\\
111.01	0\\
112.01	0\\
113.01	0\\
114.01	0\\
115.01	0\\
116.01	0\\
117.01	0\\
118.01	0\\
119.01	0\\
120.01	0\\
121.01	0\\
122.01	0\\
123.01	0\\
124.01	0\\
125.01	0\\
126.01	0\\
127.01	0\\
128.01	0\\
129.01	0\\
130.01	0\\
131.01	0\\
132.01	0\\
133.01	0\\
134.01	0\\
135.01	0\\
136.01	0\\
137.01	0\\
138.01	0\\
139.01	0\\
140.01	0\\
141.01	0\\
142.01	0\\
143.01	0\\
144.01	0\\
145.01	0\\
146.01	0\\
147.01	0\\
148.01	0\\
149.01	0\\
150.01	0\\
151.01	0\\
152.01	0\\
153.01	0\\
154.01	0\\
155.01	0\\
156.01	0\\
157.01	0\\
158.01	0\\
159.01	0\\
160.01	0\\
161.01	0\\
162.01	0\\
163.01	0\\
164.01	0\\
165.01	0\\
166.01	0\\
167.01	0\\
168.01	0\\
169.01	0\\
170.01	0\\
171.01	0\\
172.01	0\\
173.01	0\\
174.01	0\\
175.01	0\\
176.01	0\\
177.01	0\\
178.01	0\\
179.01	0\\
180.01	0\\
181.01	0\\
182.01	0\\
183.01	0\\
184.01	0\\
185.01	0\\
186.01	0\\
187.01	0\\
188.01	0\\
189.01	0\\
190.01	0\\
191.01	0\\
192.01	0\\
193.01	0\\
194.01	0\\
195.01	0\\
196.01	0\\
197.01	0\\
198.01	0\\
199.01	0\\
200.01	0\\
201.01	0\\
202.01	0\\
203.01	0\\
204.01	0\\
205.01	0\\
206.01	0\\
207.01	0\\
208.01	0\\
209.01	0\\
210.01	0\\
211.01	0\\
212.01	0\\
213.01	0\\
214.01	0\\
215.01	0\\
216.01	0\\
217.01	0\\
218.01	0\\
219.01	0\\
220.01	0\\
221.01	0\\
222.01	0\\
223.01	0\\
224.01	0\\
225.01	0\\
226.01	0\\
227.01	0\\
228.01	0\\
229.01	0\\
230.01	0\\
231.01	0\\
232.01	0\\
233.01	0\\
234.01	0\\
235.01	0\\
236.01	0\\
237.01	0\\
238.01	0\\
239.01	0\\
240.01	0\\
241.01	0\\
242.01	0\\
243.01	0\\
244.01	0\\
245.01	0\\
246.01	0\\
247.01	0\\
248.01	0\\
249.01	0\\
250.01	0\\
251.01	0\\
252.01	0\\
253.01	0\\
254.01	0\\
255.01	0\\
256.01	0\\
257.01	0\\
258.01	0\\
259.01	0\\
260.01	0\\
261.01	0\\
262.01	0\\
263.01	0\\
264.01	0\\
265.01	0\\
266.01	0\\
267.01	0\\
268.01	0\\
269.01	0\\
270.01	0\\
271.01	0\\
272.01	0\\
273.01	0\\
274.01	0\\
275.01	0\\
276.01	0\\
277.01	0\\
278.01	0\\
279.01	0\\
280.01	0\\
281.01	0\\
282.01	0\\
283.01	0\\
284.01	0\\
285.01	0\\
286.01	0\\
287.01	0\\
288.01	0\\
289.01	0\\
290.01	0\\
291.01	0\\
292.01	0\\
293.01	0\\
294.01	0\\
295.01	0\\
296.01	0\\
297.01	0\\
298.01	0\\
299.01	0\\
300.01	0\\
301.01	0\\
302.01	0\\
303.01	0\\
304.01	0\\
305.01	0\\
306.01	0\\
307.01	0\\
308.01	0\\
309.01	0\\
310.01	0\\
311.01	0\\
312.01	0\\
313.01	0\\
314.01	0\\
315.01	0\\
316.01	0\\
317.01	0\\
318.01	0\\
319.01	0\\
320.01	0\\
321.01	0\\
322.01	0\\
323.01	0\\
324.01	0\\
325.01	0\\
326.01	0\\
327.01	0\\
328.01	0\\
329.01	0\\
330.01	0\\
331.01	0\\
332.01	0\\
333.01	0\\
334.01	0\\
335.01	0\\
336.01	0\\
337.01	0\\
338.01	0\\
339.01	0\\
340.01	0\\
341.01	0\\
342.01	0\\
343.01	0\\
344.01	0\\
345.01	0\\
346.01	0\\
347.01	0\\
348.01	0\\
349.01	0\\
350.01	0\\
351.01	0\\
352.01	0\\
353.01	0\\
354.01	0\\
355.01	0\\
356.01	0\\
357.01	0\\
358.01	0\\
359.01	0\\
360.01	0\\
361.01	0\\
362.01	0\\
363.01	0\\
364.01	0\\
365.01	0\\
366.01	0\\
367.01	0\\
368.01	0\\
369.01	0\\
370.01	0\\
371.01	0\\
372.01	0\\
373.01	0\\
374.01	0\\
375.01	0\\
376.01	0\\
377.01	0\\
378.01	0\\
379.01	0\\
380.01	0\\
381.01	0\\
382.01	0\\
383.01	0\\
384.01	0\\
385.01	0\\
386.01	0\\
387.01	0\\
388.01	0\\
389.01	0\\
390.01	0\\
391.01	0\\
392.01	0\\
393.01	0\\
394.01	0\\
395.01	0\\
396.01	0\\
397.01	0\\
398.01	0\\
399.01	0\\
400.01	0\\
401.01	0\\
402.01	0\\
403.01	0\\
404.01	0\\
405.01	0\\
406.01	0\\
407.01	0\\
408.01	0\\
409.01	0\\
410.01	0\\
411.01	0\\
412.01	0\\
413.01	0\\
414.01	0\\
415.01	0\\
416.01	0\\
417.01	0\\
418.01	0\\
419.01	0\\
420.01	0\\
421.01	0\\
422.01	0\\
423.01	0\\
424.01	0\\
425.01	0\\
426.01	0\\
427.01	0\\
428.01	0\\
429.01	0\\
430.01	0\\
431.01	0\\
432.01	0\\
433.01	0\\
434.01	0\\
435.01	0\\
436.01	0\\
437.01	0\\
438.01	0\\
439.01	0\\
440.01	0\\
441.01	0\\
442.01	0\\
443.01	0\\
444.01	0\\
445.01	0\\
446.01	0\\
447.01	0\\
448.01	0\\
449.01	0\\
450.01	0\\
451.01	1.73472347597681e-18\\
452.01	0\\
453.01	0\\
454.01	0\\
455.01	0\\
456.01	0\\
457.01	0\\
458.01	0\\
459.01	0\\
460.01	0\\
461.01	0\\
462.01	0\\
463.01	0\\
464.01	0\\
465.01	0\\
466.01	0\\
467.01	0\\
468.01	0\\
469.01	1.73472347597681e-18\\
470.01	0\\
471.01	0\\
472.01	0\\
473.01	0\\
474.01	0\\
475.01	0\\
476.01	1.73472347597681e-18\\
477.01	0\\
478.01	0\\
479.01	0\\
480.01	0\\
481.01	0\\
482.01	0\\
483.01	0\\
484.01	0\\
485.01	0\\
486.01	0\\
487.01	0\\
488.01	0\\
489.01	0\\
490.01	0\\
491.01	0\\
492.01	0\\
493.01	0\\
494.01	0\\
495.01	0\\
496.01	0\\
497.01	0\\
498.01	0\\
499.01	0\\
500.01	0\\
501.01	0\\
502.01	0\\
503.01	0\\
504.01	0\\
505.01	0\\
506.01	0\\
507.01	0\\
508.01	0\\
509.01	0\\
510.01	0\\
511.01	0\\
512.01	1.73472347597681e-18\\
513.01	0\\
514.01	0\\
515.01	0\\
516.01	0\\
517.01	0\\
518.01	0\\
519.01	0\\
520.01	0\\
521.01	0\\
522.01	0\\
523.01	1.73472347597681e-18\\
524.01	0\\
525.01	0\\
526.01	0\\
527.01	0\\
528.01	0\\
529.01	0\\
530.01	0\\
531.01	0\\
532.01	0\\
533.01	0\\
534.01	0\\
535.01	0\\
536.01	1.73472347597681e-18\\
537.01	1.73472347597681e-18\\
538.01	1.73472347597681e-18\\
539.01	0\\
540.01	0\\
541.01	0\\
542.01	0\\
543.01	0\\
544.01	0\\
545.01	1.73472347597681e-18\\
546.01	1.73472347597681e-18\\
547.01	0\\
548.01	0\\
549.01	0\\
550.01	0\\
551.01	0\\
552.01	0\\
553.01	0\\
554.01	0\\
555.01	0\\
556.01	1.73472347597681e-18\\
557.01	0\\
558.01	0\\
559.01	0\\
560.01	1.73472347597681e-18\\
561.01	1.73472347597681e-18\\
562.01	0\\
563.01	0\\
564.01	0\\
565.01	0\\
566.01	0\\
567.01	0\\
568.01	0\\
569.01	0\\
570.01	0\\
571.01	0\\
572.01	0\\
573.01	0\\
574.01	0\\
575.01	0\\
576.01	0\\
577.01	0\\
578.01	1.73472347597681e-18\\
579.01	0\\
580.01	0\\
581.01	0\\
582.01	0\\
583.01	1.73472347597681e-18\\
584.01	0\\
585.01	0\\
586.01	0\\
587.01	0\\
588.01	0\\
589.01	0\\
590.01	0\\
591.01	0\\
592.01	0\\
593.01	0\\
594.01	0\\
595.01	0\\
596.01	0\\
597.01	0\\
598.01	0\\
599.01	0\\
599.02	0\\
599.03	0\\
599.04	0\\
599.05	0\\
599.06	0\\
599.07	0\\
599.08	0\\
599.09	0\\
599.1	0\\
599.11	0\\
599.12	0\\
599.13	0\\
599.14	0\\
599.15	0\\
599.16	0\\
599.17	0\\
599.18	0\\
599.19	0\\
599.2	0\\
599.21	0\\
599.22	0\\
599.23	0\\
599.24	0\\
599.25	0\\
599.26	0\\
599.27	0\\
599.28	0\\
599.29	0\\
599.3	0\\
599.31	0\\
599.32	0\\
599.33	0\\
599.34	0\\
599.35	0\\
599.36	0\\
599.37	0\\
599.38	0\\
599.39	0\\
599.4	0\\
599.41	0\\
599.42	0\\
599.43	0\\
599.44	0\\
599.45	0\\
599.46	0\\
599.47	0\\
599.48	0\\
599.49	0\\
599.5	0\\
599.51	0\\
599.52	0\\
599.53	0\\
599.54	0\\
599.55	0\\
599.56	0\\
599.57	0\\
599.58	0\\
599.59	0\\
599.6	0\\
599.61	0\\
599.62	0\\
599.63	0\\
599.64	0\\
599.65	0\\
599.66	0\\
599.67	0\\
599.68	0\\
599.69	0\\
599.7	0\\
599.71	0\\
599.72	0\\
599.73	0\\
599.74	0\\
599.75	0\\
599.76	0\\
599.77	0\\
599.78	0\\
599.79	0\\
599.8	0\\
599.81	0\\
599.82	0\\
599.83	0\\
599.84	0\\
599.85	0\\
599.86	0\\
599.87	0\\
599.88	0\\
599.89	0\\
599.9	0\\
599.91	0\\
599.92	0\\
599.93	0\\
599.94	0\\
599.95	0\\
599.96	0\\
599.97	0\\
599.98	0\\
599.99	0\\
600	0\\
};
\addplot [color=mycolor10,solid,forget plot]
  table[row sep=crcr]{%
0.01	3.6332990175629e-05\\
1.01	3.6332990175629e-05\\
2.01	3.6332990175629e-05\\
3.01	3.6332990175629e-05\\
4.01	3.6332990175629e-05\\
5.01	3.6332990175629e-05\\
6.01	3.6332990175629e-05\\
7.01	3.6332990175629e-05\\
8.01	3.6332990175629e-05\\
9.01	3.6332990175629e-05\\
10.01	3.6332990175629e-05\\
11.01	3.6332990175629e-05\\
12.01	3.6332990175629e-05\\
13.01	3.6332990175629e-05\\
14.01	3.6332990175629e-05\\
15.01	3.6332990175629e-05\\
16.01	3.6332990175629e-05\\
17.01	3.6332990175629e-05\\
18.01	3.6332990175629e-05\\
19.01	3.6332990175629e-05\\
20.01	3.6332990175629e-05\\
21.01	3.6332990175629e-05\\
22.01	3.6332990175629e-05\\
23.01	3.6332990175629e-05\\
24.01	3.6332990175629e-05\\
25.01	3.6332990175629e-05\\
26.01	3.6332990175629e-05\\
27.01	3.6332990175629e-05\\
28.01	3.6332990175629e-05\\
29.01	3.6332990175629e-05\\
30.01	3.6332990175629e-05\\
31.01	3.6332990175629e-05\\
32.01	3.6332990175629e-05\\
33.01	3.6332990175629e-05\\
34.01	3.6332990175629e-05\\
35.01	3.6332990175629e-05\\
36.01	3.6332990175629e-05\\
37.01	3.6332990175629e-05\\
38.01	3.6332990175629e-05\\
39.01	3.6332990175629e-05\\
40.01	3.6332990175629e-05\\
41.01	3.6332990175629e-05\\
42.01	3.6332990175629e-05\\
43.01	3.6332990175629e-05\\
44.01	3.6332990175629e-05\\
45.01	3.6332990175629e-05\\
46.01	3.6332990175629e-05\\
47.01	3.6332990175629e-05\\
48.01	3.6332990175629e-05\\
49.01	3.6332990175629e-05\\
50.01	3.6332990175629e-05\\
51.01	3.6332990175629e-05\\
52.01	3.6332990175629e-05\\
53.01	3.6332990175629e-05\\
54.01	3.6332990175629e-05\\
55.01	3.6332990175629e-05\\
56.01	3.6332990175629e-05\\
57.01	3.6332990175629e-05\\
58.01	3.6332990175629e-05\\
59.01	3.6332990175629e-05\\
60.01	3.6332990175629e-05\\
61.01	3.6332990175629e-05\\
62.01	3.6332990175629e-05\\
63.01	3.6332990175629e-05\\
64.01	3.6332990175629e-05\\
65.01	3.6332990175629e-05\\
66.01	3.6332990175629e-05\\
67.01	3.6332990175629e-05\\
68.01	3.6332990175629e-05\\
69.01	3.6332990175629e-05\\
70.01	3.6332990175629e-05\\
71.01	3.6332990175629e-05\\
72.01	3.6332990175629e-05\\
73.01	3.6332990175629e-05\\
74.01	3.6332990175629e-05\\
75.01	3.6332990175629e-05\\
76.01	3.6332990175629e-05\\
77.01	3.6332990175629e-05\\
78.01	3.6332990175629e-05\\
79.01	3.6332990175629e-05\\
80.01	3.6332990175629e-05\\
81.01	3.6332990175629e-05\\
82.01	3.6332990175629e-05\\
83.01	3.6332990175629e-05\\
84.01	3.6332990175629e-05\\
85.01	3.6332990175629e-05\\
86.01	3.6332990175629e-05\\
87.01	3.6332990175629e-05\\
88.01	3.6332990175629e-05\\
89.01	3.6332990175629e-05\\
90.01	3.6332990175629e-05\\
91.01	3.6332990175629e-05\\
92.01	3.6332990175629e-05\\
93.01	3.6332990175629e-05\\
94.01	3.6332990175629e-05\\
95.01	3.6332990175629e-05\\
96.01	3.6332990175629e-05\\
97.01	3.6332990175629e-05\\
98.01	3.6332990175629e-05\\
99.01	3.6332990175629e-05\\
100.01	3.6332990175629e-05\\
101.01	3.6332990175629e-05\\
102.01	3.6332990175629e-05\\
103.01	3.6332990175629e-05\\
104.01	3.6332990175629e-05\\
105.01	3.6332990175629e-05\\
106.01	3.6332990175629e-05\\
107.01	3.6332990175629e-05\\
108.01	3.6332990175629e-05\\
109.01	3.6332990175629e-05\\
110.01	3.6332990175629e-05\\
111.01	3.6332990175629e-05\\
112.01	3.6332990175629e-05\\
113.01	3.6332990175629e-05\\
114.01	3.6332990175629e-05\\
115.01	3.6332990175629e-05\\
116.01	3.6332990175629e-05\\
117.01	3.6332990175629e-05\\
118.01	3.6332990175629e-05\\
119.01	3.6332990175629e-05\\
120.01	3.6332990175629e-05\\
121.01	3.6332990175629e-05\\
122.01	3.6332990175629e-05\\
123.01	3.6332990175629e-05\\
124.01	3.6332990175629e-05\\
125.01	3.6332990175629e-05\\
126.01	3.6332990175629e-05\\
127.01	3.6332990175629e-05\\
128.01	3.6332990175629e-05\\
129.01	3.6332990175629e-05\\
130.01	3.6332990175629e-05\\
131.01	3.6332990175629e-05\\
132.01	3.6332990175629e-05\\
133.01	3.6332990175629e-05\\
134.01	3.6332990175629e-05\\
135.01	3.6332990175629e-05\\
136.01	3.6332990175629e-05\\
137.01	3.6332990175629e-05\\
138.01	3.6332990175629e-05\\
139.01	3.6332990175629e-05\\
140.01	3.6332990175629e-05\\
141.01	3.6332990175629e-05\\
142.01	3.6332990175629e-05\\
143.01	3.6332990175629e-05\\
144.01	3.6332990175629e-05\\
145.01	3.6332990175629e-05\\
146.01	3.6332990175629e-05\\
147.01	3.6332990175629e-05\\
148.01	3.6332990175629e-05\\
149.01	3.6332990175629e-05\\
150.01	3.6332990175629e-05\\
151.01	3.6332990175629e-05\\
152.01	3.6332990175629e-05\\
153.01	3.6332990175629e-05\\
154.01	3.6332990175629e-05\\
155.01	3.6332990175629e-05\\
156.01	3.6332990175629e-05\\
157.01	3.6332990175629e-05\\
158.01	3.6332990175629e-05\\
159.01	3.6332990175629e-05\\
160.01	3.6332990175629e-05\\
161.01	3.6332990175629e-05\\
162.01	3.6332990175629e-05\\
163.01	3.6332990175629e-05\\
164.01	3.6332990175629e-05\\
165.01	3.6332990175629e-05\\
166.01	3.6332990175629e-05\\
167.01	3.6332990175629e-05\\
168.01	3.6332990175629e-05\\
169.01	3.6332990175629e-05\\
170.01	3.6332990175629e-05\\
171.01	3.6332990175629e-05\\
172.01	3.6332990175629e-05\\
173.01	3.6332990175629e-05\\
174.01	3.6332990175629e-05\\
175.01	3.6332990175629e-05\\
176.01	3.6332990175629e-05\\
177.01	3.6332990175629e-05\\
178.01	3.6332990175629e-05\\
179.01	3.6332990175629e-05\\
180.01	3.6332990175629e-05\\
181.01	3.6332990175629e-05\\
182.01	3.6332990175629e-05\\
183.01	3.6332990175629e-05\\
184.01	3.6332990175629e-05\\
185.01	3.6332990175629e-05\\
186.01	3.6332990175629e-05\\
187.01	3.6332990175629e-05\\
188.01	3.6332990175629e-05\\
189.01	3.6332990175629e-05\\
190.01	3.6332990175629e-05\\
191.01	3.6332990175629e-05\\
192.01	3.6332990175629e-05\\
193.01	3.6332990175629e-05\\
194.01	3.6332990175629e-05\\
195.01	3.6332990175629e-05\\
196.01	3.6332990175629e-05\\
197.01	3.6332990175629e-05\\
198.01	3.6332990175629e-05\\
199.01	3.6332990175629e-05\\
200.01	3.6332990175629e-05\\
201.01	3.6332990175629e-05\\
202.01	3.6332990175629e-05\\
203.01	3.6332990175629e-05\\
204.01	3.6332990175629e-05\\
205.01	3.6332990175629e-05\\
206.01	3.6332990175629e-05\\
207.01	3.6332990175629e-05\\
208.01	3.6332990175629e-05\\
209.01	3.6332990175629e-05\\
210.01	3.6332990175629e-05\\
211.01	3.6332990175629e-05\\
212.01	3.6332990175629e-05\\
213.01	3.6332990175629e-05\\
214.01	3.6332990175629e-05\\
215.01	3.6332990175629e-05\\
216.01	3.6332990175629e-05\\
217.01	3.6332990175629e-05\\
218.01	3.6332990175629e-05\\
219.01	3.6332990175629e-05\\
220.01	3.6332990175629e-05\\
221.01	3.6332990175629e-05\\
222.01	3.6332990175629e-05\\
223.01	3.6332990175629e-05\\
224.01	3.6332990175629e-05\\
225.01	3.6332990175629e-05\\
226.01	3.6332990175629e-05\\
227.01	3.6332990175629e-05\\
228.01	3.6332990175629e-05\\
229.01	3.6332990175629e-05\\
230.01	3.6332990175629e-05\\
231.01	3.6332990175629e-05\\
232.01	3.6332990175629e-05\\
233.01	3.6332990175629e-05\\
234.01	3.6332990175629e-05\\
235.01	3.6332990175629e-05\\
236.01	3.6332990175629e-05\\
237.01	3.6332990175629e-05\\
238.01	3.6332990175629e-05\\
239.01	3.6332990175629e-05\\
240.01	3.6332990175629e-05\\
241.01	3.6332990175629e-05\\
242.01	3.6332990175629e-05\\
243.01	3.6332990175629e-05\\
244.01	3.6332990175629e-05\\
245.01	3.6332990175629e-05\\
246.01	3.6332990175629e-05\\
247.01	3.6332990175629e-05\\
248.01	3.6332990175629e-05\\
249.01	3.6332990175629e-05\\
250.01	3.6332990175629e-05\\
251.01	3.6332990175629e-05\\
252.01	3.6332990175629e-05\\
253.01	3.6332990175629e-05\\
254.01	3.6332990175629e-05\\
255.01	3.6332990175629e-05\\
256.01	3.6332990175629e-05\\
257.01	3.6332990175629e-05\\
258.01	3.6332990175629e-05\\
259.01	3.6332990175629e-05\\
260.01	3.6332990175629e-05\\
261.01	3.6332990175629e-05\\
262.01	3.6332990175629e-05\\
263.01	3.6332990175629e-05\\
264.01	3.6332990175629e-05\\
265.01	3.6332990175629e-05\\
266.01	3.6332990175629e-05\\
267.01	3.6332990175629e-05\\
268.01	3.6332990175629e-05\\
269.01	3.6332990175629e-05\\
270.01	3.6332990175629e-05\\
271.01	3.6332990175629e-05\\
272.01	3.6332990175629e-05\\
273.01	3.6332990175629e-05\\
274.01	3.6332990175629e-05\\
275.01	3.6332990175629e-05\\
276.01	3.6332990175629e-05\\
277.01	3.6332990175629e-05\\
278.01	3.6332990175629e-05\\
279.01	3.6332990175629e-05\\
280.01	3.6332990175629e-05\\
281.01	3.6332990175629e-05\\
282.01	3.6332990175629e-05\\
283.01	3.6332990175629e-05\\
284.01	3.6332990175629e-05\\
285.01	3.6332990175629e-05\\
286.01	3.6332990175629e-05\\
287.01	3.6332990175629e-05\\
288.01	3.6332990175629e-05\\
289.01	3.6332990175629e-05\\
290.01	3.6332990175629e-05\\
291.01	3.6332990175629e-05\\
292.01	3.6332990175629e-05\\
293.01	3.6332990175629e-05\\
294.01	3.6332990175629e-05\\
295.01	3.6332990175629e-05\\
296.01	3.6332990175629e-05\\
297.01	3.6332990175629e-05\\
298.01	3.6332990175629e-05\\
299.01	3.6332990175629e-05\\
300.01	3.6332990175629e-05\\
301.01	3.6332990175629e-05\\
302.01	3.6332990175629e-05\\
303.01	3.6332990175629e-05\\
304.01	3.6332990175629e-05\\
305.01	3.6332990175629e-05\\
306.01	3.6332990175629e-05\\
307.01	3.6332990175629e-05\\
308.01	3.6332990175629e-05\\
309.01	3.6332990175629e-05\\
310.01	3.6332990175629e-05\\
311.01	3.6332990175629e-05\\
312.01	3.6332990175629e-05\\
313.01	3.6332990175629e-05\\
314.01	3.6332990175629e-05\\
315.01	3.6332990175629e-05\\
316.01	3.6332990175629e-05\\
317.01	3.6332990175629e-05\\
318.01	3.6332990175629e-05\\
319.01	3.6332990175629e-05\\
320.01	3.6332990175629e-05\\
321.01	3.6332990175629e-05\\
322.01	3.6332990175629e-05\\
323.01	3.6332990175629e-05\\
324.01	3.6332990175629e-05\\
325.01	3.6332990175629e-05\\
326.01	3.6332990175629e-05\\
327.01	3.6332990175629e-05\\
328.01	3.6332990175629e-05\\
329.01	3.6332990175629e-05\\
330.01	3.6332990175629e-05\\
331.01	3.6332990175629e-05\\
332.01	3.6332990175629e-05\\
333.01	3.6332990175629e-05\\
334.01	3.6332990175629e-05\\
335.01	3.6332990175629e-05\\
336.01	3.6332990175629e-05\\
337.01	3.6332990175629e-05\\
338.01	3.6332990175629e-05\\
339.01	3.6332990175629e-05\\
340.01	3.6332990175629e-05\\
341.01	3.6332990175629e-05\\
342.01	3.6332990175629e-05\\
343.01	3.6332990175629e-05\\
344.01	3.6332990175629e-05\\
345.01	3.6332990175629e-05\\
346.01	3.6332990175629e-05\\
347.01	3.6332990175629e-05\\
348.01	3.6332990175629e-05\\
349.01	3.6332990175629e-05\\
350.01	3.6332990175629e-05\\
351.01	3.6332990175629e-05\\
352.01	3.6332990175629e-05\\
353.01	3.6332990175629e-05\\
354.01	3.6332990175629e-05\\
355.01	3.6332990175629e-05\\
356.01	3.6332990175629e-05\\
357.01	3.6332990175629e-05\\
358.01	3.6332990175629e-05\\
359.01	3.6332990175629e-05\\
360.01	3.6332990175629e-05\\
361.01	3.6332990175629e-05\\
362.01	3.6332990175629e-05\\
363.01	3.6332990175629e-05\\
364.01	3.6332990175629e-05\\
365.01	3.6332990175629e-05\\
366.01	3.6332990175629e-05\\
367.01	3.6332990175629e-05\\
368.01	3.6332990175629e-05\\
369.01	3.6332990175629e-05\\
370.01	3.6332990175629e-05\\
371.01	3.6332990175629e-05\\
372.01	3.6332990175629e-05\\
373.01	3.6332990175629e-05\\
374.01	3.6332990175629e-05\\
375.01	3.6332990175629e-05\\
376.01	3.6332990175629e-05\\
377.01	3.6332990175629e-05\\
378.01	3.6332990175629e-05\\
379.01	3.6332990175629e-05\\
380.01	3.6332990175629e-05\\
381.01	3.6332990175629e-05\\
382.01	3.6332990175629e-05\\
383.01	3.6332990175629e-05\\
384.01	3.6332990175629e-05\\
385.01	3.6332990175629e-05\\
386.01	3.6332990175629e-05\\
387.01	3.6332990175629e-05\\
388.01	3.6332990175629e-05\\
389.01	3.6332990175629e-05\\
390.01	3.6332990175629e-05\\
391.01	3.6332990175629e-05\\
392.01	3.6332990175629e-05\\
393.01	3.6332990175629e-05\\
394.01	3.6332990175629e-05\\
395.01	3.6332990175629e-05\\
396.01	3.6332990175629e-05\\
397.01	3.6332990175629e-05\\
398.01	3.6332990175629e-05\\
399.01	3.6332990175629e-05\\
400.01	3.6332990175629e-05\\
401.01	3.6332990175629e-05\\
402.01	3.6332990175629e-05\\
403.01	3.6332990175629e-05\\
404.01	3.6332990175629e-05\\
405.01	3.6332990175629e-05\\
406.01	3.6332990175629e-05\\
407.01	3.6332990175629e-05\\
408.01	3.6332990175629e-05\\
409.01	3.6332990175629e-05\\
410.01	3.6332990175629e-05\\
411.01	3.6332990175629e-05\\
412.01	3.6332990175629e-05\\
413.01	3.6332990175629e-05\\
414.01	3.6332990175629e-05\\
415.01	3.6332990175629e-05\\
416.01	3.6332990175629e-05\\
417.01	3.6332990175629e-05\\
418.01	3.6332990175629e-05\\
419.01	3.6332990175629e-05\\
420.01	3.6332990175629e-05\\
421.01	3.6332990175629e-05\\
422.01	3.6332990175629e-05\\
423.01	3.6332990175629e-05\\
424.01	3.6332990175629e-05\\
425.01	3.6332990175629e-05\\
426.01	3.6332990175629e-05\\
427.01	3.6332990175629e-05\\
428.01	3.6332990175629e-05\\
429.01	3.6332990175629e-05\\
430.01	3.6332990175629e-05\\
431.01	3.6332990175629e-05\\
432.01	3.6332990175629e-05\\
433.01	3.6332990175629e-05\\
434.01	3.6332990175629e-05\\
435.01	3.6332990175629e-05\\
436.01	3.6332990175629e-05\\
437.01	3.6332990175629e-05\\
438.01	3.6332990175629e-05\\
439.01	3.6332990175629e-05\\
440.01	3.6332990175629e-05\\
441.01	3.6332990175629e-05\\
442.01	3.6332990175629e-05\\
443.01	3.6332990175629e-05\\
444.01	3.6332990175629e-05\\
445.01	3.63329901756273e-05\\
446.01	3.63329901756099e-05\\
447.01	3.633299017557e-05\\
448.01	3.63329901754642e-05\\
449.01	3.63329901751815e-05\\
450.01	3.63329901744112e-05\\
451.01	3.63329901723244e-05\\
452.01	3.63329901666726e-05\\
453.01	3.63329901513759e-05\\
454.01	3.63329901100322e-05\\
455.01	3.63329899984756e-05\\
456.01	3.63329896982331e-05\\
457.01	3.63329888930751e-05\\
458.01	3.63329867448508e-05\\
459.01	3.63329810541529e-05\\
460.01	3.63329661298185e-05\\
461.01	3.63329275367384e-05\\
462.01	3.63328297068363e-05\\
463.01	3.63325887119854e-05\\
464.01	3.63320195061381e-05\\
465.01	3.63307585765377e-05\\
466.01	3.63282378046548e-05\\
467.01	3.63240050212116e-05\\
468.01	3.63186526850575e-05\\
469.01	3.63131005853765e-05\\
470.01	3.63074235777052e-05\\
471.01	3.6301634700522e-05\\
472.01	3.62957697946677e-05\\
473.01	3.62899174158309e-05\\
474.01	3.62842786465652e-05\\
475.01	3.62792618784288e-05\\
476.01	3.6275545979637e-05\\
477.01	3.62738038318083e-05\\
478.01	3.62736023077313e-05\\
479.01	3.62736023077313e-05\\
480.01	3.62736023077313e-05\\
481.01	3.62736023077313e-05\\
482.01	3.62736023077313e-05\\
483.01	3.62736023077313e-05\\
484.01	3.62736023077313e-05\\
485.01	3.62736023077313e-05\\
486.01	3.62736023077313e-05\\
487.01	3.62736023077313e-05\\
488.01	3.62736023077313e-05\\
489.01	3.62736023077313e-05\\
490.01	3.62736023077313e-05\\
491.01	3.62736023077313e-05\\
492.01	3.62736023077313e-05\\
493.01	3.62736023077313e-05\\
494.01	3.62736023077313e-05\\
495.01	3.62736023077313e-05\\
496.01	3.62736023077313e-05\\
497.01	3.62736023077313e-05\\
498.01	3.62736023077313e-05\\
499.01	3.62736023077313e-05\\
500.01	3.62736023077313e-05\\
501.01	3.62736023077313e-05\\
502.01	3.62736023077313e-05\\
503.01	3.62736023077313e-05\\
504.01	3.62736023077313e-05\\
505.01	3.62736023077313e-05\\
506.01	3.62736023077313e-05\\
507.01	3.62736023077331e-05\\
508.01	3.62736023077331e-05\\
509.01	3.6273602307714e-05\\
510.01	3.6273602307688e-05\\
511.01	3.62736023076082e-05\\
512.01	3.62736023073896e-05\\
513.01	3.62736023067703e-05\\
514.01	3.62736023050304e-05\\
515.01	3.62736023001211e-05\\
516.01	3.62736022862329e-05\\
517.01	3.62736022467593e-05\\
518.01	3.62736021340768e-05\\
519.01	3.62736018107348e-05\\
520.01	3.62736008775299e-05\\
521.01	3.62735981668961e-05\\
522.01	3.62735902382626e-05\\
523.01	3.62735668706579e-05\\
524.01	3.62734974387602e-05\\
525.01	3.6273289345785e-05\\
526.01	3.62726599625867e-05\\
527.01	3.62707381364666e-05\\
528.01	3.62648115461908e-05\\
529.01	3.62463488691057e-05\\
530.01	3.61882420091067e-05\\
531.01	3.6003504478764e-05\\
532.01	3.54104452939367e-05\\
533.01	3.34895990750511e-05\\
534.01	2.86780701776249e-05\\
535.01	2.33349897528021e-05\\
536.01	1.7891461666756e-05\\
537.01	1.23300729334211e-05\\
538.01	6.77559521603671e-06\\
539.01	1.90599427558891e-06\\
540.01	0\\
541.01	0\\
542.01	0\\
543.01	0\\
544.01	0\\
545.01	1.73472347597681e-18\\
546.01	1.73472347597681e-18\\
547.01	0\\
548.01	0\\
549.01	0\\
550.01	0\\
551.01	0\\
552.01	0\\
553.01	0\\
554.01	0\\
555.01	0\\
556.01	1.73472347597681e-18\\
557.01	0\\
558.01	0\\
559.01	0\\
560.01	1.73472347597681e-18\\
561.01	1.73472347597681e-18\\
562.01	0\\
563.01	0\\
564.01	0\\
565.01	0\\
566.01	0\\
567.01	0\\
568.01	0\\
569.01	0\\
570.01	0\\
571.01	0\\
572.01	0\\
573.01	0\\
574.01	0\\
575.01	0\\
576.01	0\\
577.01	0\\
578.01	1.73472347597681e-18\\
579.01	0\\
580.01	0\\
581.01	0\\
582.01	0\\
583.01	1.73472347597681e-18\\
584.01	0\\
585.01	0\\
586.01	0\\
587.01	0\\
588.01	0\\
589.01	0\\
590.01	0\\
591.01	0\\
592.01	0\\
593.01	0\\
594.01	0\\
595.01	0\\
596.01	0\\
597.01	0\\
598.01	0\\
599.01	0\\
599.02	0\\
599.03	0\\
599.04	0\\
599.05	0\\
599.06	0\\
599.07	0\\
599.08	0\\
599.09	0\\
599.1	0\\
599.11	0\\
599.12	0\\
599.13	0\\
599.14	0\\
599.15	0\\
599.16	0\\
599.17	0\\
599.18	0\\
599.19	0\\
599.2	0\\
599.21	0\\
599.22	0\\
599.23	0\\
599.24	0\\
599.25	0\\
599.26	0\\
599.27	0\\
599.28	0\\
599.29	0\\
599.3	0\\
599.31	0\\
599.32	0\\
599.33	0\\
599.34	0\\
599.35	0\\
599.36	0\\
599.37	0\\
599.38	0\\
599.39	0\\
599.4	0\\
599.41	0\\
599.42	0\\
599.43	0\\
599.44	0\\
599.45	0\\
599.46	0\\
599.47	0\\
599.48	0\\
599.49	0\\
599.5	0\\
599.51	0\\
599.52	0\\
599.53	0\\
599.54	0\\
599.55	0\\
599.56	0\\
599.57	0\\
599.58	0\\
599.59	0\\
599.6	0\\
599.61	0\\
599.62	0\\
599.63	0\\
599.64	0\\
599.65	0\\
599.66	0\\
599.67	0\\
599.68	0\\
599.69	0\\
599.7	0\\
599.71	0\\
599.72	0\\
599.73	0\\
599.74	0\\
599.75	0\\
599.76	0\\
599.77	0\\
599.78	0\\
599.79	0\\
599.8	0\\
599.81	0\\
599.82	0\\
599.83	0\\
599.84	0\\
599.85	0\\
599.86	0\\
599.87	0\\
599.88	0\\
599.89	0\\
599.9	0\\
599.91	0\\
599.92	0\\
599.93	0\\
599.94	0\\
599.95	0\\
599.96	0\\
599.97	0\\
599.98	0\\
599.99	0\\
600	0\\
};
\addplot [color=mycolor11,solid,forget plot]
  table[row sep=crcr]{%
0.01	0.00032348326751387\\
1.01	0.00032348326751387\\
2.01	0.00032348326751387\\
3.01	0.00032348326751387\\
4.01	0.00032348326751387\\
5.01	0.00032348326751387\\
6.01	0.00032348326751387\\
7.01	0.00032348326751387\\
8.01	0.00032348326751387\\
9.01	0.00032348326751387\\
10.01	0.00032348326751387\\
11.01	0.00032348326751387\\
12.01	0.00032348326751387\\
13.01	0.00032348326751387\\
14.01	0.00032348326751387\\
15.01	0.00032348326751387\\
16.01	0.00032348326751387\\
17.01	0.00032348326751387\\
18.01	0.00032348326751387\\
19.01	0.00032348326751387\\
20.01	0.00032348326751387\\
21.01	0.00032348326751387\\
22.01	0.00032348326751387\\
23.01	0.00032348326751387\\
24.01	0.00032348326751387\\
25.01	0.00032348326751387\\
26.01	0.00032348326751387\\
27.01	0.00032348326751387\\
28.01	0.00032348326751387\\
29.01	0.00032348326751387\\
30.01	0.00032348326751387\\
31.01	0.00032348326751387\\
32.01	0.00032348326751387\\
33.01	0.00032348326751387\\
34.01	0.00032348326751387\\
35.01	0.00032348326751387\\
36.01	0.00032348326751387\\
37.01	0.00032348326751387\\
38.01	0.00032348326751387\\
39.01	0.00032348326751387\\
40.01	0.00032348326751387\\
41.01	0.00032348326751387\\
42.01	0.00032348326751387\\
43.01	0.00032348326751387\\
44.01	0.00032348326751387\\
45.01	0.00032348326751387\\
46.01	0.00032348326751387\\
47.01	0.00032348326751387\\
48.01	0.00032348326751387\\
49.01	0.00032348326751387\\
50.01	0.00032348326751387\\
51.01	0.00032348326751387\\
52.01	0.00032348326751387\\
53.01	0.00032348326751387\\
54.01	0.00032348326751387\\
55.01	0.00032348326751387\\
56.01	0.00032348326751387\\
57.01	0.00032348326751387\\
58.01	0.00032348326751387\\
59.01	0.00032348326751387\\
60.01	0.00032348326751387\\
61.01	0.00032348326751387\\
62.01	0.00032348326751387\\
63.01	0.00032348326751387\\
64.01	0.00032348326751387\\
65.01	0.00032348326751387\\
66.01	0.00032348326751387\\
67.01	0.00032348326751387\\
68.01	0.00032348326751387\\
69.01	0.00032348326751387\\
70.01	0.00032348326751387\\
71.01	0.00032348326751387\\
72.01	0.00032348326751387\\
73.01	0.00032348326751387\\
74.01	0.00032348326751387\\
75.01	0.00032348326751387\\
76.01	0.00032348326751387\\
77.01	0.00032348326751387\\
78.01	0.00032348326751387\\
79.01	0.00032348326751387\\
80.01	0.00032348326751387\\
81.01	0.00032348326751387\\
82.01	0.00032348326751387\\
83.01	0.00032348326751387\\
84.01	0.00032348326751387\\
85.01	0.00032348326751387\\
86.01	0.00032348326751387\\
87.01	0.00032348326751387\\
88.01	0.00032348326751387\\
89.01	0.00032348326751387\\
90.01	0.00032348326751387\\
91.01	0.00032348326751387\\
92.01	0.00032348326751387\\
93.01	0.00032348326751387\\
94.01	0.00032348326751387\\
95.01	0.00032348326751387\\
96.01	0.00032348326751387\\
97.01	0.00032348326751387\\
98.01	0.00032348326751387\\
99.01	0.00032348326751387\\
100.01	0.00032348326751387\\
101.01	0.00032348326751387\\
102.01	0.00032348326751387\\
103.01	0.00032348326751387\\
104.01	0.00032348326751387\\
105.01	0.00032348326751387\\
106.01	0.00032348326751387\\
107.01	0.00032348326751387\\
108.01	0.00032348326751387\\
109.01	0.00032348326751387\\
110.01	0.00032348326751387\\
111.01	0.00032348326751387\\
112.01	0.00032348326751387\\
113.01	0.00032348326751387\\
114.01	0.00032348326751387\\
115.01	0.00032348326751387\\
116.01	0.00032348326751387\\
117.01	0.00032348326751387\\
118.01	0.00032348326751387\\
119.01	0.00032348326751387\\
120.01	0.00032348326751387\\
121.01	0.00032348326751387\\
122.01	0.00032348326751387\\
123.01	0.00032348326751387\\
124.01	0.00032348326751387\\
125.01	0.00032348326751387\\
126.01	0.00032348326751387\\
127.01	0.00032348326751387\\
128.01	0.00032348326751387\\
129.01	0.00032348326751387\\
130.01	0.00032348326751387\\
131.01	0.00032348326751387\\
132.01	0.00032348326751387\\
133.01	0.00032348326751387\\
134.01	0.00032348326751387\\
135.01	0.00032348326751387\\
136.01	0.00032348326751387\\
137.01	0.00032348326751387\\
138.01	0.00032348326751387\\
139.01	0.00032348326751387\\
140.01	0.00032348326751387\\
141.01	0.00032348326751387\\
142.01	0.00032348326751387\\
143.01	0.00032348326751387\\
144.01	0.00032348326751387\\
145.01	0.00032348326751387\\
146.01	0.00032348326751387\\
147.01	0.00032348326751387\\
148.01	0.00032348326751387\\
149.01	0.00032348326751387\\
150.01	0.00032348326751387\\
151.01	0.00032348326751387\\
152.01	0.00032348326751387\\
153.01	0.00032348326751387\\
154.01	0.00032348326751387\\
155.01	0.00032348326751387\\
156.01	0.00032348326751387\\
157.01	0.00032348326751387\\
158.01	0.00032348326751387\\
159.01	0.00032348326751387\\
160.01	0.00032348326751387\\
161.01	0.00032348326751387\\
162.01	0.00032348326751387\\
163.01	0.00032348326751387\\
164.01	0.00032348326751387\\
165.01	0.00032348326751387\\
166.01	0.00032348326751387\\
167.01	0.00032348326751387\\
168.01	0.00032348326751387\\
169.01	0.00032348326751387\\
170.01	0.00032348326751387\\
171.01	0.00032348326751387\\
172.01	0.00032348326751387\\
173.01	0.00032348326751387\\
174.01	0.00032348326751387\\
175.01	0.00032348326751387\\
176.01	0.00032348326751387\\
177.01	0.00032348326751387\\
178.01	0.00032348326751387\\
179.01	0.00032348326751387\\
180.01	0.00032348326751387\\
181.01	0.00032348326751387\\
182.01	0.00032348326751387\\
183.01	0.00032348326751387\\
184.01	0.00032348326751387\\
185.01	0.00032348326751387\\
186.01	0.00032348326751387\\
187.01	0.00032348326751387\\
188.01	0.00032348326751387\\
189.01	0.00032348326751387\\
190.01	0.00032348326751387\\
191.01	0.00032348326751387\\
192.01	0.00032348326751387\\
193.01	0.00032348326751387\\
194.01	0.00032348326751387\\
195.01	0.00032348326751387\\
196.01	0.00032348326751387\\
197.01	0.00032348326751387\\
198.01	0.00032348326751387\\
199.01	0.00032348326751387\\
200.01	0.00032348326751387\\
201.01	0.00032348326751387\\
202.01	0.00032348326751387\\
203.01	0.00032348326751387\\
204.01	0.00032348326751387\\
205.01	0.00032348326751387\\
206.01	0.00032348326751387\\
207.01	0.00032348326751387\\
208.01	0.00032348326751387\\
209.01	0.00032348326751387\\
210.01	0.00032348326751387\\
211.01	0.00032348326751387\\
212.01	0.00032348326751387\\
213.01	0.00032348326751387\\
214.01	0.00032348326751387\\
215.01	0.00032348326751387\\
216.01	0.00032348326751387\\
217.01	0.00032348326751387\\
218.01	0.00032348326751387\\
219.01	0.00032348326751387\\
220.01	0.00032348326751387\\
221.01	0.00032348326751387\\
222.01	0.00032348326751387\\
223.01	0.00032348326751387\\
224.01	0.00032348326751387\\
225.01	0.00032348326751387\\
226.01	0.00032348326751387\\
227.01	0.00032348326751387\\
228.01	0.00032348326751387\\
229.01	0.00032348326751387\\
230.01	0.00032348326751387\\
231.01	0.00032348326751387\\
232.01	0.00032348326751387\\
233.01	0.00032348326751387\\
234.01	0.00032348326751387\\
235.01	0.00032348326751387\\
236.01	0.00032348326751387\\
237.01	0.00032348326751387\\
238.01	0.00032348326751387\\
239.01	0.00032348326751387\\
240.01	0.00032348326751387\\
241.01	0.00032348326751387\\
242.01	0.00032348326751387\\
243.01	0.00032348326751387\\
244.01	0.00032348326751387\\
245.01	0.00032348326751387\\
246.01	0.00032348326751387\\
247.01	0.00032348326751387\\
248.01	0.00032348326751387\\
249.01	0.00032348326751387\\
250.01	0.00032348326751387\\
251.01	0.00032348326751387\\
252.01	0.00032348326751387\\
253.01	0.00032348326751387\\
254.01	0.00032348326751387\\
255.01	0.00032348326751387\\
256.01	0.00032348326751387\\
257.01	0.00032348326751387\\
258.01	0.00032348326751387\\
259.01	0.00032348326751387\\
260.01	0.00032348326751387\\
261.01	0.00032348326751387\\
262.01	0.00032348326751387\\
263.01	0.00032348326751387\\
264.01	0.00032348326751387\\
265.01	0.00032348326751387\\
266.01	0.00032348326751387\\
267.01	0.00032348326751387\\
268.01	0.00032348326751387\\
269.01	0.00032348326751387\\
270.01	0.00032348326751387\\
271.01	0.00032348326751387\\
272.01	0.00032348326751387\\
273.01	0.00032348326751387\\
274.01	0.00032348326751387\\
275.01	0.00032348326751387\\
276.01	0.00032348326751387\\
277.01	0.00032348326751387\\
278.01	0.00032348326751387\\
279.01	0.00032348326751387\\
280.01	0.00032348326751387\\
281.01	0.00032348326751387\\
282.01	0.00032348326751387\\
283.01	0.00032348326751387\\
284.01	0.00032348326751387\\
285.01	0.00032348326751387\\
286.01	0.00032348326751387\\
287.01	0.00032348326751387\\
288.01	0.00032348326751387\\
289.01	0.00032348326751387\\
290.01	0.00032348326751387\\
291.01	0.00032348326751387\\
292.01	0.00032348326751387\\
293.01	0.00032348326751387\\
294.01	0.00032348326751387\\
295.01	0.00032348326751387\\
296.01	0.00032348326751387\\
297.01	0.00032348326751387\\
298.01	0.00032348326751387\\
299.01	0.00032348326751387\\
300.01	0.00032348326751387\\
301.01	0.00032348326751387\\
302.01	0.00032348326751387\\
303.01	0.00032348326751387\\
304.01	0.00032348326751387\\
305.01	0.00032348326751387\\
306.01	0.00032348326751387\\
307.01	0.00032348326751387\\
308.01	0.00032348326751387\\
309.01	0.00032348326751387\\
310.01	0.00032348326751387\\
311.01	0.00032348326751387\\
312.01	0.00032348326751387\\
313.01	0.00032348326751387\\
314.01	0.00032348326751387\\
315.01	0.00032348326751387\\
316.01	0.00032348326751387\\
317.01	0.00032348326751387\\
318.01	0.00032348326751387\\
319.01	0.00032348326751387\\
320.01	0.00032348326751387\\
321.01	0.00032348326751387\\
322.01	0.00032348326751387\\
323.01	0.00032348326751387\\
324.01	0.00032348326751387\\
325.01	0.00032348326751387\\
326.01	0.00032348326751387\\
327.01	0.00032348326751387\\
328.01	0.00032348326751387\\
329.01	0.00032348326751387\\
330.01	0.00032348326751387\\
331.01	0.00032348326751387\\
332.01	0.00032348326751387\\
333.01	0.00032348326751387\\
334.01	0.00032348326751387\\
335.01	0.00032348326751387\\
336.01	0.00032348326751387\\
337.01	0.00032348326751387\\
338.01	0.00032348326751387\\
339.01	0.00032348326751387\\
340.01	0.00032348326751387\\
341.01	0.00032348326751387\\
342.01	0.00032348326751387\\
343.01	0.00032348326751387\\
344.01	0.00032348326751387\\
345.01	0.00032348326751387\\
346.01	0.00032348326751387\\
347.01	0.00032348326751387\\
348.01	0.00032348326751387\\
349.01	0.00032348326751387\\
350.01	0.00032348326751387\\
351.01	0.00032348326751387\\
352.01	0.00032348326751387\\
353.01	0.00032348326751387\\
354.01	0.00032348326751387\\
355.01	0.00032348326751387\\
356.01	0.00032348326751387\\
357.01	0.00032348326751387\\
358.01	0.00032348326751387\\
359.01	0.00032348326751387\\
360.01	0.00032348326751387\\
361.01	0.00032348326751387\\
362.01	0.00032348326751387\\
363.01	0.00032348326751387\\
364.01	0.00032348326751387\\
365.01	0.00032348326751387\\
366.01	0.00032348326751387\\
367.01	0.00032348326751387\\
368.01	0.00032348326751387\\
369.01	0.00032348326751387\\
370.01	0.00032348326751387\\
371.01	0.00032348326751387\\
372.01	0.00032348326751387\\
373.01	0.00032348326751387\\
374.01	0.00032348326751387\\
375.01	0.00032348326751387\\
376.01	0.00032348326751387\\
377.01	0.00032348326751387\\
378.01	0.00032348326751387\\
379.01	0.00032348326751387\\
380.01	0.00032348326751387\\
381.01	0.00032348326751387\\
382.01	0.00032348326751387\\
383.01	0.00032348326751387\\
384.01	0.00032348326751387\\
385.01	0.00032348326751387\\
386.01	0.00032348326751387\\
387.01	0.00032348326751387\\
388.01	0.00032348326751387\\
389.01	0.00032348326751387\\
390.01	0.00032348326751387\\
391.01	0.00032348326751387\\
392.01	0.00032348326751387\\
393.01	0.00032348326751387\\
394.01	0.00032348326751387\\
395.01	0.00032348326751387\\
396.01	0.00032348326751387\\
397.01	0.00032348326751387\\
398.01	0.00032348326751387\\
399.01	0.00032348326751387\\
400.01	0.00032348326751387\\
401.01	0.00032348326751387\\
402.01	0.00032348326751387\\
403.01	0.00032348326751387\\
404.01	0.00032348326751387\\
405.01	0.00032348326751387\\
406.01	0.00032348326751387\\
407.01	0.00032348326751387\\
408.01	0.00032348326751387\\
409.01	0.00032348326751387\\
410.01	0.00032348326751387\\
411.01	0.00032348326751387\\
412.01	0.00032348326751387\\
413.01	0.00032348326751387\\
414.01	0.00032348326751387\\
415.01	0.00032348326751387\\
416.01	0.00032348326751387\\
417.01	0.00032348326751387\\
418.01	0.00032348326751387\\
419.01	0.00032348326751387\\
420.01	0.00032348326751387\\
421.01	0.00032348326751387\\
422.01	0.00032348326751387\\
423.01	0.00032348326751387\\
424.01	0.00032348326751387\\
425.01	0.00032348326751387\\
426.01	0.00032348326751387\\
427.01	0.00032348326751387\\
428.01	0.00032348326751387\\
429.01	0.00032348326751387\\
430.01	0.00032348326751387\\
431.01	0.00032348326751387\\
432.01	0.00032348326751387\\
433.01	0.00032348326751387\\
434.01	0.00032348326751387\\
435.01	0.00032348326751387\\
436.01	0.00032348326751387\\
437.01	0.00032348326751387\\
438.01	0.00032348326751387\\
439.01	0.00032348326751387\\
440.01	0.00032348326751387\\
441.01	0.00032348326751387\\
442.01	0.00032348326751387\\
443.01	0.00032348326751387\\
444.01	0.000323483267513872\\
445.01	0.000323483267513872\\
446.01	0.000323483267513854\\
447.01	0.000323483267513818\\
448.01	0.000323483267513719\\
449.01	0.00032348326751346\\
450.01	0.000323483267512777\\
451.01	0.00032348326751098\\
452.01	0.000323483267506265\\
453.01	0.000323483267493957\\
454.01	0.00032348326746201\\
455.01	0.000323483267379661\\
456.01	0.000323483267169111\\
457.01	0.000323483266636138\\
458.01	0.000323483265303676\\
459.01	0.000323483262024351\\
460.01	0.000323483254114245\\
461.01	0.000323483235527324\\
462.01	0.000323483193342078\\
463.01	0.000323483101985822\\
464.01	0.000323482916574127\\
465.01	0.000323482573337544\\
466.01	0.000323482016935115\\
467.01	0.000323481268148873\\
468.01	0.000323480441666813\\
469.01	0.000323479595467246\\
470.01	0.000323478733894969\\
471.01	0.000323477863201176\\
472.01	0.000323476997294479\\
473.01	0.000323476164510996\\
474.01	0.000323475416246714\\
475.01	0.000323474829585597\\
476.01	0.000323474481653553\\
477.01	0.000323474371361708\\
478.01	0.000323474365043552\\
479.01	0.000323474365043552\\
480.01	0.000323474365043552\\
481.01	0.000323474365043552\\
482.01	0.000323474365043552\\
483.01	0.000323474365043552\\
484.01	0.000323474365043552\\
485.01	0.000323474365043552\\
486.01	0.000323474365043552\\
487.01	0.000323474365043552\\
488.01	0.000323474365043552\\
489.01	0.000323474365043552\\
490.01	0.000323474365043552\\
491.01	0.000323474365043552\\
492.01	0.000323474365043552\\
493.01	0.000323474365043552\\
494.01	0.000323474365043552\\
495.01	0.000323474365043552\\
496.01	0.000323474365043552\\
497.01	0.000323474365043552\\
498.01	0.000323474365043552\\
499.01	0.000323474365043552\\
500.01	0.000323474365043552\\
501.01	0.000323474365043552\\
502.01	0.000323474365043552\\
503.01	0.000323474365043552\\
504.01	0.000323474365043552\\
505.01	0.000323474365043552\\
506.01	0.000323474365043552\\
507.01	0.000323474365043552\\
508.01	0.000323474365043552\\
509.01	0.000323474365043541\\
510.01	0.000323474365043508\\
511.01	0.000323474365043427\\
512.01	0.000323474365043201\\
513.01	0.000323474365042582\\
514.01	0.000323474365040866\\
515.01	0.000323474365036124\\
516.01	0.000323474365023004\\
517.01	0.000323474364986712\\
518.01	0.000323474364886271\\
519.01	0.000323474364608155\\
520.01	0.000323474363837714\\
521.01	0.000323474361702531\\
522.01	0.000323474355783794\\
523.01	0.00032347433937965\\
524.01	0.000323474293952497\\
525.01	0.000323474168395149\\
526.01	0.000323473822617017\\
527.01	0.000323472876248249\\
528.01	0.000323470312193687\\
529.01	0.000323463476687595\\
530.01	0.0003234457191502\\
531.01	0.000323401507088766\\
532.01	0.000323299349623227\\
533.01	0.000323096583750704\\
534.01	0.000322820494104223\\
535.01	0.000322544563133137\\
536.01	0.000322274405627094\\
537.01	0.000322012012008299\\
538.01	0.000321781721082232\\
539.01	0.000321635878047019\\
540.01	0.000321609567441481\\
541.01	0.000321609567440196\\
542.01	0.00032160956743655\\
543.01	0.000321609567426226\\
544.01	0.000321609567396994\\
545.01	0.000321609567314168\\
546.01	0.000321609567079316\\
547.01	0.00032160956641262\\
548.01	0.000321609564516894\\
549.01	0.000321609559114573\\
550.01	0.000321609543674653\\
551.01	0.00032160949938256\\
552.01	0.00032160937172819\\
553.01	0.000321609001700078\\
554.01	0.000321607921710718\\
555.01	0.000321604744132882\\
556.01	0.00032159530906575\\
557.01	0.000321567009926508\\
558.01	0.00032148121484275\\
559.01	0.000321218239453271\\
560.01	0.000320403516967051\\
561.01	0.000317854518528006\\
562.01	0.000309812765753915\\
563.01	0.000290521225180146\\
564.01	0.000258815636837589\\
565.01	0.00022319689057226\\
566.01	0.000180005113740115\\
567.01	0.000117402829674417\\
568.01	5.15310646992881e-05\\
569.01	4.39279151884069e-06\\
570.01	0\\
571.01	0\\
572.01	0\\
573.01	0\\
574.01	0\\
575.01	0\\
576.01	0\\
577.01	0\\
578.01	1.73472347597681e-18\\
579.01	0\\
580.01	0\\
581.01	0\\
582.01	0\\
583.01	1.73472347597681e-18\\
584.01	0\\
585.01	0\\
586.01	0\\
587.01	0\\
588.01	0\\
589.01	0\\
590.01	0\\
591.01	0\\
592.01	0\\
593.01	0\\
594.01	0\\
595.01	0\\
596.01	0\\
597.01	0\\
598.01	0\\
599.01	0\\
599.02	0\\
599.03	0\\
599.04	0\\
599.05	0\\
599.06	0\\
599.07	0\\
599.08	0\\
599.09	0\\
599.1	0\\
599.11	0\\
599.12	0\\
599.13	0\\
599.14	0\\
599.15	0\\
599.16	0\\
599.17	0\\
599.18	0\\
599.19	0\\
599.2	0\\
599.21	0\\
599.22	0\\
599.23	0\\
599.24	0\\
599.25	0\\
599.26	0\\
599.27	0\\
599.28	0\\
599.29	0\\
599.3	0\\
599.31	0\\
599.32	0\\
599.33	0\\
599.34	0\\
599.35	0\\
599.36	0\\
599.37	0\\
599.38	0\\
599.39	0\\
599.4	0\\
599.41	0\\
599.42	0\\
599.43	0\\
599.44	0\\
599.45	0\\
599.46	0\\
599.47	0\\
599.48	0\\
599.49	0\\
599.5	0\\
599.51	0\\
599.52	0\\
599.53	0\\
599.54	0\\
599.55	0\\
599.56	0\\
599.57	0\\
599.58	0\\
599.59	0\\
599.6	0\\
599.61	0\\
599.62	0\\
599.63	0\\
599.64	0\\
599.65	0\\
599.66	0\\
599.67	0\\
599.68	0\\
599.69	0\\
599.7	0\\
599.71	0\\
599.72	0\\
599.73	0\\
599.74	0\\
599.75	0\\
599.76	0\\
599.77	0\\
599.78	0\\
599.79	0\\
599.8	0\\
599.81	0\\
599.82	0\\
599.83	0\\
599.84	0\\
599.85	0\\
599.86	0\\
599.87	0\\
599.88	0\\
599.89	0\\
599.9	0\\
599.91	0\\
599.92	0\\
599.93	0\\
599.94	0\\
599.95	0\\
599.96	0\\
599.97	0\\
599.98	0\\
599.99	0\\
600	0\\
};
\addplot [color=mycolor12,solid,forget plot]
  table[row sep=crcr]{%
0.01	0.00230832677588388\\
1.01	0.00230832677588388\\
2.01	0.00230832677588388\\
3.01	0.00230832677588388\\
4.01	0.00230832677588388\\
5.01	0.00230832677588388\\
6.01	0.00230832677588388\\
7.01	0.00230832677588388\\
8.01	0.00230832677588388\\
9.01	0.00230832677588388\\
10.01	0.00230832677588388\\
11.01	0.00230832677588388\\
12.01	0.00230832677588388\\
13.01	0.00230832677588388\\
14.01	0.00230832677588388\\
15.01	0.00230832677588388\\
16.01	0.00230832677588388\\
17.01	0.00230832677588388\\
18.01	0.00230832677588388\\
19.01	0.00230832677588388\\
20.01	0.00230832677588388\\
21.01	0.00230832677588388\\
22.01	0.00230832677588388\\
23.01	0.00230832677588388\\
24.01	0.00230832677588388\\
25.01	0.00230832677588388\\
26.01	0.00230832677588388\\
27.01	0.00230832677588388\\
28.01	0.00230832677588388\\
29.01	0.00230832677588388\\
30.01	0.00230832677588388\\
31.01	0.00230832677588388\\
32.01	0.00230832677588388\\
33.01	0.00230832677588388\\
34.01	0.00230832677588388\\
35.01	0.00230832677588388\\
36.01	0.00230832677588388\\
37.01	0.00230832677588388\\
38.01	0.00230832677588388\\
39.01	0.00230832677588388\\
40.01	0.00230832677588388\\
41.01	0.00230832677588388\\
42.01	0.00230832677588388\\
43.01	0.00230832677588388\\
44.01	0.00230832677588388\\
45.01	0.00230832677588388\\
46.01	0.00230832677588388\\
47.01	0.00230832677588388\\
48.01	0.00230832677588388\\
49.01	0.00230832677588388\\
50.01	0.00230832677588388\\
51.01	0.00230832677588388\\
52.01	0.00230832677588388\\
53.01	0.00230832677588388\\
54.01	0.00230832677588388\\
55.01	0.00230832677588388\\
56.01	0.00230832677588388\\
57.01	0.00230832677588388\\
58.01	0.00230832677588388\\
59.01	0.00230832677588388\\
60.01	0.00230832677588388\\
61.01	0.00230832677588388\\
62.01	0.00230832677588388\\
63.01	0.00230832677588388\\
64.01	0.00230832677588388\\
65.01	0.00230832677588388\\
66.01	0.00230832677588388\\
67.01	0.00230832677588388\\
68.01	0.00230832677588388\\
69.01	0.00230832677588388\\
70.01	0.00230832677588388\\
71.01	0.00230832677588388\\
72.01	0.00230832677588388\\
73.01	0.00230832677588388\\
74.01	0.00230832677588388\\
75.01	0.00230832677588388\\
76.01	0.00230832677588388\\
77.01	0.00230832677588388\\
78.01	0.00230832677588388\\
79.01	0.00230832677588388\\
80.01	0.00230832677588388\\
81.01	0.00230832677588388\\
82.01	0.00230832677588388\\
83.01	0.00230832677588388\\
84.01	0.00230832677588388\\
85.01	0.00230832677588388\\
86.01	0.00230832677588388\\
87.01	0.00230832677588388\\
88.01	0.00230832677588388\\
89.01	0.00230832677588388\\
90.01	0.00230832677588388\\
91.01	0.00230832677588388\\
92.01	0.00230832677588388\\
93.01	0.00230832677588388\\
94.01	0.00230832677588388\\
95.01	0.00230832677588388\\
96.01	0.00230832677588388\\
97.01	0.00230832677588388\\
98.01	0.00230832677588388\\
99.01	0.00230832677588388\\
100.01	0.00230832677588388\\
101.01	0.00230832677588388\\
102.01	0.00230832677588388\\
103.01	0.00230832677588388\\
104.01	0.00230832677588388\\
105.01	0.00230832677588388\\
106.01	0.00230832677588388\\
107.01	0.00230832677588388\\
108.01	0.00230832677588388\\
109.01	0.00230832677588388\\
110.01	0.00230832677588388\\
111.01	0.00230832677588388\\
112.01	0.00230832677588388\\
113.01	0.00230832677588388\\
114.01	0.00230832677588388\\
115.01	0.00230832677588388\\
116.01	0.00230832677588388\\
117.01	0.00230832677588388\\
118.01	0.00230832677588388\\
119.01	0.00230832677588388\\
120.01	0.00230832677588388\\
121.01	0.00230832677588388\\
122.01	0.00230832677588388\\
123.01	0.00230832677588388\\
124.01	0.00230832677588388\\
125.01	0.00230832677588388\\
126.01	0.00230832677588388\\
127.01	0.00230832677588388\\
128.01	0.00230832677588388\\
129.01	0.00230832677588388\\
130.01	0.00230832677588388\\
131.01	0.00230832677588388\\
132.01	0.00230832677588388\\
133.01	0.00230832677588388\\
134.01	0.00230832677588388\\
135.01	0.00230832677588388\\
136.01	0.00230832677588388\\
137.01	0.00230832677588388\\
138.01	0.00230832677588388\\
139.01	0.00230832677588388\\
140.01	0.00230832677588388\\
141.01	0.00230832677588388\\
142.01	0.00230832677588388\\
143.01	0.00230832677588388\\
144.01	0.00230832677588388\\
145.01	0.00230832677588388\\
146.01	0.00230832677588388\\
147.01	0.00230832677588388\\
148.01	0.00230832677588388\\
149.01	0.00230832677588388\\
150.01	0.00230832677588388\\
151.01	0.00230832677588388\\
152.01	0.00230832677588388\\
153.01	0.00230832677588388\\
154.01	0.00230832677588388\\
155.01	0.00230832677588388\\
156.01	0.00230832677588388\\
157.01	0.00230832677588388\\
158.01	0.00230832677588388\\
159.01	0.00230832677588388\\
160.01	0.00230832677588388\\
161.01	0.00230832677588388\\
162.01	0.00230832677588388\\
163.01	0.00230832677588388\\
164.01	0.00230832677588388\\
165.01	0.00230832677588388\\
166.01	0.00230832677588388\\
167.01	0.00230832677588388\\
168.01	0.00230832677588388\\
169.01	0.00230832677588388\\
170.01	0.00230832677588388\\
171.01	0.00230832677588388\\
172.01	0.00230832677588388\\
173.01	0.00230832677588388\\
174.01	0.00230832677588388\\
175.01	0.00230832677588388\\
176.01	0.00230832677588388\\
177.01	0.00230832677588388\\
178.01	0.00230832677588388\\
179.01	0.00230832677588388\\
180.01	0.00230832677588388\\
181.01	0.00230832677588388\\
182.01	0.00230832677588388\\
183.01	0.00230832677588388\\
184.01	0.00230832677588388\\
185.01	0.00230832677588388\\
186.01	0.00230832677588388\\
187.01	0.00230832677588388\\
188.01	0.00230832677588388\\
189.01	0.00230832677588388\\
190.01	0.00230832677588388\\
191.01	0.00230832677588388\\
192.01	0.00230832677588388\\
193.01	0.00230832677588388\\
194.01	0.00230832677588388\\
195.01	0.00230832677588388\\
196.01	0.00230832677588388\\
197.01	0.00230832677588388\\
198.01	0.00230832677588388\\
199.01	0.00230832677588388\\
200.01	0.00230832677588388\\
201.01	0.00230832677588388\\
202.01	0.00230832677588388\\
203.01	0.00230832677588388\\
204.01	0.00230832677588388\\
205.01	0.00230832677588388\\
206.01	0.00230832677588388\\
207.01	0.00230832677588388\\
208.01	0.00230832677588388\\
209.01	0.00230832677588388\\
210.01	0.00230832677588388\\
211.01	0.00230832677588388\\
212.01	0.00230832677588388\\
213.01	0.00230832677588388\\
214.01	0.00230832677588388\\
215.01	0.00230832677588388\\
216.01	0.00230832677588388\\
217.01	0.00230832677588388\\
218.01	0.00230832677588388\\
219.01	0.00230832677588388\\
220.01	0.00230832677588388\\
221.01	0.00230832677588388\\
222.01	0.00230832677588388\\
223.01	0.00230832677588388\\
224.01	0.00230832677588388\\
225.01	0.00230832677588388\\
226.01	0.00230832677588388\\
227.01	0.00230832677588388\\
228.01	0.00230832677588388\\
229.01	0.00230832677588388\\
230.01	0.00230832677588388\\
231.01	0.00230832677588388\\
232.01	0.00230832677588388\\
233.01	0.00230832677588388\\
234.01	0.00230832677588388\\
235.01	0.00230832677588388\\
236.01	0.00230832677588388\\
237.01	0.00230832677588388\\
238.01	0.00230832677588388\\
239.01	0.00230832677588388\\
240.01	0.00230832677588388\\
241.01	0.00230832677588388\\
242.01	0.00230832677588388\\
243.01	0.00230832677588388\\
244.01	0.00230832677588388\\
245.01	0.00230832677588388\\
246.01	0.00230832677588388\\
247.01	0.00230832677588388\\
248.01	0.00230832677588388\\
249.01	0.00230832677588388\\
250.01	0.00230832677588388\\
251.01	0.00230832677588388\\
252.01	0.00230832677588388\\
253.01	0.00230832677588388\\
254.01	0.00230832677588388\\
255.01	0.00230832677588388\\
256.01	0.00230832677588388\\
257.01	0.00230832677588388\\
258.01	0.00230832677588388\\
259.01	0.00230832677588388\\
260.01	0.00230832677588388\\
261.01	0.00230832677588388\\
262.01	0.00230832677588388\\
263.01	0.00230832677588388\\
264.01	0.00230832677588388\\
265.01	0.00230832677588388\\
266.01	0.00230832677588388\\
267.01	0.00230832677588388\\
268.01	0.00230832677588388\\
269.01	0.00230832677588388\\
270.01	0.00230832677588388\\
271.01	0.00230832677588388\\
272.01	0.00230832677588388\\
273.01	0.00230832677588388\\
274.01	0.00230832677588388\\
275.01	0.00230832677588388\\
276.01	0.00230832677588388\\
277.01	0.00230832677588388\\
278.01	0.00230832677588388\\
279.01	0.00230832677588388\\
280.01	0.00230832677588388\\
281.01	0.00230832677588388\\
282.01	0.00230832677588388\\
283.01	0.00230832677588388\\
284.01	0.00230832677588388\\
285.01	0.00230832677588388\\
286.01	0.00230832677588388\\
287.01	0.00230832677588388\\
288.01	0.00230832677588388\\
289.01	0.00230832677588388\\
290.01	0.00230832677588388\\
291.01	0.00230832677588388\\
292.01	0.00230832677588388\\
293.01	0.00230832677588388\\
294.01	0.00230832677588388\\
295.01	0.00230832677588388\\
296.01	0.00230832677588388\\
297.01	0.00230832677588388\\
298.01	0.00230832677588388\\
299.01	0.00230832677588388\\
300.01	0.00230832677588388\\
301.01	0.00230832677588388\\
302.01	0.00230832677588388\\
303.01	0.00230832677588388\\
304.01	0.00230832677588388\\
305.01	0.00230832677588388\\
306.01	0.00230832677588388\\
307.01	0.00230832677588388\\
308.01	0.00230832677588388\\
309.01	0.00230832677588388\\
310.01	0.00230832677588388\\
311.01	0.00230832677588388\\
312.01	0.00230832677588388\\
313.01	0.00230832677588388\\
314.01	0.00230832677588388\\
315.01	0.00230832677588388\\
316.01	0.00230832677588388\\
317.01	0.00230832677588388\\
318.01	0.00230832677588388\\
319.01	0.00230832677588388\\
320.01	0.00230832677588388\\
321.01	0.00230832677588388\\
322.01	0.00230832677588388\\
323.01	0.00230832677588388\\
324.01	0.00230832677588388\\
325.01	0.00230832677588388\\
326.01	0.00230832677588388\\
327.01	0.00230832677588388\\
328.01	0.00230832677588388\\
329.01	0.00230832677588388\\
330.01	0.00230832677588388\\
331.01	0.00230832677588388\\
332.01	0.00230832677588388\\
333.01	0.00230832677588388\\
334.01	0.00230832677588388\\
335.01	0.00230832677588388\\
336.01	0.00230832677588388\\
337.01	0.00230832677588388\\
338.01	0.00230832677588388\\
339.01	0.00230832677588388\\
340.01	0.00230832677588388\\
341.01	0.00230832677588388\\
342.01	0.00230832677588388\\
343.01	0.00230832677588388\\
344.01	0.00230832677588388\\
345.01	0.00230832677588388\\
346.01	0.00230832677588388\\
347.01	0.00230832677588388\\
348.01	0.00230832677588388\\
349.01	0.00230832677588388\\
350.01	0.00230832677588388\\
351.01	0.00230832677588388\\
352.01	0.00230832677588388\\
353.01	0.00230832677588388\\
354.01	0.00230832677588388\\
355.01	0.00230832677588388\\
356.01	0.00230832677588388\\
357.01	0.00230832677588388\\
358.01	0.00230832677588388\\
359.01	0.00230832677588388\\
360.01	0.00230832677588388\\
361.01	0.00230832677588388\\
362.01	0.00230832677588388\\
363.01	0.00230832677588388\\
364.01	0.00230832677588388\\
365.01	0.00230832677588388\\
366.01	0.00230832677588388\\
367.01	0.00230832677588388\\
368.01	0.00230832677588388\\
369.01	0.00230832677588388\\
370.01	0.00230832677588388\\
371.01	0.00230832677588388\\
372.01	0.00230832677588388\\
373.01	0.00230832677588388\\
374.01	0.00230832677588388\\
375.01	0.00230832677588388\\
376.01	0.00230832677588388\\
377.01	0.00230832677588388\\
378.01	0.00230832677588388\\
379.01	0.00230832677588388\\
380.01	0.00230832677588388\\
381.01	0.00230832677588388\\
382.01	0.00230832677588388\\
383.01	0.00230832677588388\\
384.01	0.00230832677588388\\
385.01	0.00230832677588388\\
386.01	0.00230832677588388\\
387.01	0.00230832677588388\\
388.01	0.00230832677588388\\
389.01	0.00230832677588388\\
390.01	0.00230832677588388\\
391.01	0.00230832677588388\\
392.01	0.00230832677588388\\
393.01	0.00230832677588388\\
394.01	0.00230832677588388\\
395.01	0.00230832677588388\\
396.01	0.00230832677588388\\
397.01	0.00230832677588388\\
398.01	0.00230832677588388\\
399.01	0.00230832677588388\\
400.01	0.00230832677588388\\
401.01	0.00230832677588388\\
402.01	0.00230832677588388\\
403.01	0.00230832677588388\\
404.01	0.00230832677588388\\
405.01	0.00230832677588388\\
406.01	0.00230832677588388\\
407.01	0.00230832677588388\\
408.01	0.00230832677588388\\
409.01	0.00230832677588388\\
410.01	0.00230832677588388\\
411.01	0.00230832677588388\\
412.01	0.00230832677588388\\
413.01	0.00230832677588388\\
414.01	0.00230832677588388\\
415.01	0.00230832677588388\\
416.01	0.00230832677588388\\
417.01	0.00230832677588388\\
418.01	0.00230832677588388\\
419.01	0.00230832677588388\\
420.01	0.00230832677588388\\
421.01	0.00230832677588388\\
422.01	0.00230832677588388\\
423.01	0.00230832677588388\\
424.01	0.00230832677588388\\
425.01	0.00230832677588388\\
426.01	0.00230832677588388\\
427.01	0.00230832677588388\\
428.01	0.00230832677588388\\
429.01	0.00230832677588388\\
430.01	0.00230832677588388\\
431.01	0.00230832677588388\\
432.01	0.00230832677588388\\
433.01	0.00230832677588388\\
434.01	0.00230832677588388\\
435.01	0.00230832677588388\\
436.01	0.00230832677588388\\
437.01	0.00230832677588388\\
438.01	0.00230832677588388\\
439.01	0.00230832677588388\\
440.01	0.00230832677588388\\
441.01	0.00230832677588388\\
442.01	0.00230832677588388\\
443.01	0.00230832677588388\\
444.01	0.00230832677588388\\
445.01	0.00230832677588388\\
446.01	0.00230832677588388\\
447.01	0.00230832677588385\\
448.01	0.00230832677588378\\
449.01	0.00230832677588358\\
450.01	0.00230832677588308\\
451.01	0.00230832677588178\\
452.01	0.00230832677587848\\
453.01	0.00230832677587017\\
454.01	0.00230832677584946\\
455.01	0.00230832677579852\\
456.01	0.00230832677567521\\
457.01	0.00230832677538211\\
458.01	0.00230832677470067\\
459.01	0.00230832677315791\\
460.01	0.00230832676977686\\
461.01	0.00230832676265952\\
462.01	0.00230832674841642\\
463.01	0.00230832672169671\\
464.01	0.00230832667559282\\
465.01	0.00230832660424001\\
466.01	0.00230832650799597\\
467.01	0.00230832639644697\\
468.01	0.00230832628012502\\
469.01	0.00230832616191159\\
470.01	0.00230832604287346\\
471.01	0.00230832592509532\\
472.01	0.00230832581240644\\
473.01	0.00230832571105018\\
474.01	0.00230832562949901\\
475.01	0.00230832557582186\\
476.01	0.00230832555157799\\
477.01	0.00230832554657476\\
478.01	0.00230832554643708\\
479.01	0.00230832554643708\\
480.01	0.00230832554643708\\
481.01	0.00230832554643708\\
482.01	0.00230832554643708\\
483.01	0.00230832554643708\\
484.01	0.00230832554643708\\
485.01	0.00230832554643708\\
486.01	0.00230832554643708\\
487.01	0.00230832554643708\\
488.01	0.00230832554643708\\
489.01	0.00230832554643708\\
490.01	0.00230832554643708\\
491.01	0.00230832554643708\\
492.01	0.00230832554643708\\
493.01	0.00230832554643708\\
494.01	0.00230832554643708\\
495.01	0.00230832554643708\\
496.01	0.00230832554643708\\
497.01	0.00230832554643708\\
498.01	0.00230832554643708\\
499.01	0.00230832554643708\\
500.01	0.00230832554643708\\
501.01	0.00230832554643708\\
502.01	0.00230832554643708\\
503.01	0.00230832554643708\\
504.01	0.00230832554643708\\
505.01	0.00230832554643708\\
506.01	0.00230832554643708\\
507.01	0.00230832554643708\\
508.01	0.00230832554643708\\
509.01	0.00230832554643708\\
510.01	0.00230832554643706\\
511.01	0.00230832554643699\\
512.01	0.00230832554643681\\
513.01	0.00230832554643632\\
514.01	0.00230832554643497\\
515.01	0.0023083255464313\\
516.01	0.00230832554642139\\
517.01	0.00230832554639465\\
518.01	0.00230832554632288\\
519.01	0.00230832554613126\\
520.01	0.00230832554562292\\
521.01	0.00230832554428438\\
522.01	0.00230832554079122\\
523.01	0.0023083255317732\\
524.01	0.00230832550879643\\
525.01	0.00230832545119599\\
526.01	0.002308325309691\\
527.01	0.00230832497087839\\
528.01	0.00230832418619398\\
529.01	0.00230832244749176\\
530.01	0.00230831882192436\\
531.01	0.00230831189250172\\
532.01	0.00230830028466374\\
533.01	0.00230828450470487\\
534.01	0.00230826815361379\\
535.01	0.00230825300739821\\
536.01	0.00230823904377992\\
537.01	0.00230822681893579\\
538.01	0.00230821812485519\\
539.01	0.00230821447379591\\
540.01	0.00230821417331235\\
541.01	0.002308214173311\\
542.01	0.0023082141733072\\
543.01	0.00230821417329649\\
544.01	0.00230821417326634\\
545.01	0.00230821417318166\\
546.01	0.00230821417294417\\
547.01	0.00230821417227934\\
548.01	0.00230821417042241\\
549.01	0.00230821416524946\\
550.01	0.00230821415088388\\
551.01	0.00230821411113991\\
552.01	0.0023082140016871\\
553.01	0.00230821370195917\\
554.01	0.00230821288695113\\
555.01	0.00230821069054941\\
556.01	0.00230820483924602\\
557.01	0.00230818948614138\\
558.01	0.00230815002122632\\
559.01	0.00230805146354952\\
560.01	0.00230781562226196\\
561.01	0.00230728866371109\\
562.01	0.00230625128843328\\
563.01	0.00230463539557428\\
564.01	0.00230261463947325\\
565.01	0.00230027902002805\\
566.01	0.00229739045615409\\
567.01	0.00229414404146691\\
568.01	0.00229160501562282\\
569.01	0.00229055117153649\\
570.01	0.00229052498630496\\
571.01	0.00229052423870807\\
572.01	0.00229052200145302\\
573.01	0.00229051531235272\\
574.01	0.00229049533245125\\
575.01	0.00229043571706054\\
576.01	0.00229025804248841\\
577.01	0.00228972916792507\\
578.01	0.00228815701010785\\
579.01	0.00228349029843025\\
580.01	0.00226965902075653\\
581.01	0.00222878167195284\\
582.01	0.00213964864116229\\
583.01	0.00201726281566172\\
584.01	0.00185939417254293\\
585.01	0.00164117735174267\\
586.01	0.00140842561566351\\
587.01	0.00116447021169677\\
588.01	0.000899880963021769\\
589.01	0.00058797839348404\\
590.01	0.000224847652437047\\
591.01	3.75983539933726e-06\\
592.01	0\\
593.01	0\\
594.01	0\\
595.01	0\\
596.01	0\\
597.01	0\\
598.01	0\\
599.01	0\\
599.02	0\\
599.03	0\\
599.04	0\\
599.05	0\\
599.06	0\\
599.07	0\\
599.08	0\\
599.09	0\\
599.1	0\\
599.11	0\\
599.12	0\\
599.13	0\\
599.14	0\\
599.15	0\\
599.16	0\\
599.17	0\\
599.18	0\\
599.19	0\\
599.2	0\\
599.21	0\\
599.22	0\\
599.23	0\\
599.24	0\\
599.25	0\\
599.26	0\\
599.27	0\\
599.28	0\\
599.29	0\\
599.3	0\\
599.31	0\\
599.32	0\\
599.33	0\\
599.34	0\\
599.35	0\\
599.36	0\\
599.37	0\\
599.38	0\\
599.39	0\\
599.4	0\\
599.41	0\\
599.42	0\\
599.43	0\\
599.44	0\\
599.45	0\\
599.46	0\\
599.47	0\\
599.48	0\\
599.49	0\\
599.5	0\\
599.51	0\\
599.52	0\\
599.53	0\\
599.54	0\\
599.55	0\\
599.56	0\\
599.57	0\\
599.58	0\\
599.59	0\\
599.6	0\\
599.61	0\\
599.62	0\\
599.63	0\\
599.64	0\\
599.65	0\\
599.66	0\\
599.67	0\\
599.68	0\\
599.69	0\\
599.7	0\\
599.71	0\\
599.72	0\\
599.73	0\\
599.74	0\\
599.75	0\\
599.76	0\\
599.77	0\\
599.78	0\\
599.79	0\\
599.8	0\\
599.81	0\\
599.82	0\\
599.83	0\\
599.84	0\\
599.85	0\\
599.86	0\\
599.87	0\\
599.88	0\\
599.89	0\\
599.9	0\\
599.91	0\\
599.92	0\\
599.93	0\\
599.94	0\\
599.95	0\\
599.96	0\\
599.97	0\\
599.98	0\\
599.99	0\\
600	0\\
};
\addplot [color=mycolor13,solid,forget plot]
  table[row sep=crcr]{%
0.01	0\\
1.01	0\\
2.01	0\\
3.01	0\\
4.01	0\\
5.01	0\\
6.01	0\\
7.01	0\\
8.01	0\\
9.01	0\\
10.01	0\\
11.01	0\\
12.01	0\\
13.01	0\\
14.01	0\\
15.01	0\\
16.01	0\\
17.01	0\\
18.01	0\\
19.01	0\\
20.01	0\\
21.01	0\\
22.01	0\\
23.01	0\\
24.01	0\\
25.01	0\\
26.01	0\\
27.01	0\\
28.01	0\\
29.01	0\\
30.01	0\\
31.01	0\\
32.01	0\\
33.01	0\\
34.01	0\\
35.01	0\\
36.01	0\\
37.01	0\\
38.01	0\\
39.01	0\\
40.01	0\\
41.01	0\\
42.01	0\\
43.01	0\\
44.01	0\\
45.01	0\\
46.01	0\\
47.01	0\\
48.01	0\\
49.01	0\\
50.01	0\\
51.01	0\\
52.01	0\\
53.01	0\\
54.01	0\\
55.01	0\\
56.01	0\\
57.01	0\\
58.01	0\\
59.01	0\\
60.01	0\\
61.01	0\\
62.01	0\\
63.01	0\\
64.01	0\\
65.01	0\\
66.01	0\\
67.01	0\\
68.01	0\\
69.01	0\\
70.01	0\\
71.01	0\\
72.01	0\\
73.01	0\\
74.01	0\\
75.01	0\\
76.01	0\\
77.01	0\\
78.01	0\\
79.01	0\\
80.01	0\\
81.01	0\\
82.01	0\\
83.01	0\\
84.01	0\\
85.01	0\\
86.01	0\\
87.01	0\\
88.01	0\\
89.01	0\\
90.01	0\\
91.01	0\\
92.01	0\\
93.01	0\\
94.01	0\\
95.01	0\\
96.01	0\\
97.01	0\\
98.01	0\\
99.01	0\\
100.01	0\\
101.01	0\\
102.01	0\\
103.01	0\\
104.01	0\\
105.01	0\\
106.01	0\\
107.01	0\\
108.01	0\\
109.01	0\\
110.01	0\\
111.01	0\\
112.01	0\\
113.01	0\\
114.01	0\\
115.01	0\\
116.01	0\\
117.01	0\\
118.01	0\\
119.01	0\\
120.01	0\\
121.01	0\\
122.01	0\\
123.01	0\\
124.01	0\\
125.01	0\\
126.01	0\\
127.01	0\\
128.01	0\\
129.01	0\\
130.01	0\\
131.01	0\\
132.01	0\\
133.01	0\\
134.01	0\\
135.01	0\\
136.01	0\\
137.01	0\\
138.01	0\\
139.01	0\\
140.01	0\\
141.01	0\\
142.01	0\\
143.01	0\\
144.01	0\\
145.01	0\\
146.01	0\\
147.01	0\\
148.01	0\\
149.01	0\\
150.01	0\\
151.01	0\\
152.01	0\\
153.01	0\\
154.01	0\\
155.01	0\\
156.01	0\\
157.01	0\\
158.01	0\\
159.01	0\\
160.01	0\\
161.01	0\\
162.01	0\\
163.01	0\\
164.01	0\\
165.01	0\\
166.01	0\\
167.01	0\\
168.01	0\\
169.01	0\\
170.01	0\\
171.01	0\\
172.01	0\\
173.01	0\\
174.01	0\\
175.01	0\\
176.01	0\\
177.01	0\\
178.01	0\\
179.01	0\\
180.01	0\\
181.01	0\\
182.01	0\\
183.01	0\\
184.01	0\\
185.01	0\\
186.01	0\\
187.01	0\\
188.01	0\\
189.01	0\\
190.01	0\\
191.01	0\\
192.01	0\\
193.01	0\\
194.01	0\\
195.01	0\\
196.01	0\\
197.01	0\\
198.01	0\\
199.01	0\\
200.01	0\\
201.01	0\\
202.01	0\\
203.01	0\\
204.01	0\\
205.01	0\\
206.01	0\\
207.01	0\\
208.01	0\\
209.01	0\\
210.01	0\\
211.01	0\\
212.01	0\\
213.01	0\\
214.01	0\\
215.01	0\\
216.01	0\\
217.01	0\\
218.01	0\\
219.01	0\\
220.01	0\\
221.01	0\\
222.01	0\\
223.01	0\\
224.01	0\\
225.01	0\\
226.01	0\\
227.01	0\\
228.01	0\\
229.01	0\\
230.01	0\\
231.01	0\\
232.01	0\\
233.01	0\\
234.01	0\\
235.01	0\\
236.01	0\\
237.01	0\\
238.01	0\\
239.01	0\\
240.01	0\\
241.01	0\\
242.01	0\\
243.01	0\\
244.01	0\\
245.01	0\\
246.01	0\\
247.01	0\\
248.01	0\\
249.01	0\\
250.01	0\\
251.01	0\\
252.01	0\\
253.01	0\\
254.01	0\\
255.01	0\\
256.01	0\\
257.01	0\\
258.01	0\\
259.01	0\\
260.01	0\\
261.01	0\\
262.01	0\\
263.01	0\\
264.01	0\\
265.01	0\\
266.01	0\\
267.01	0\\
268.01	0\\
269.01	0\\
270.01	0\\
271.01	0\\
272.01	0\\
273.01	0\\
274.01	0\\
275.01	0\\
276.01	0\\
277.01	0\\
278.01	0\\
279.01	0\\
280.01	0\\
281.01	0\\
282.01	0\\
283.01	0\\
284.01	0\\
285.01	0\\
286.01	0\\
287.01	0\\
288.01	0\\
289.01	0\\
290.01	0\\
291.01	0\\
292.01	0\\
293.01	0\\
294.01	0\\
295.01	0\\
296.01	0\\
297.01	0\\
298.01	0\\
299.01	0\\
300.01	0\\
301.01	0\\
302.01	0\\
303.01	0\\
304.01	0\\
305.01	0\\
306.01	0\\
307.01	0\\
308.01	0\\
309.01	0\\
310.01	0\\
311.01	0\\
312.01	0\\
313.01	0\\
314.01	0\\
315.01	0\\
316.01	0\\
317.01	0\\
318.01	0\\
319.01	0\\
320.01	0\\
321.01	0\\
322.01	0\\
323.01	0\\
324.01	0\\
325.01	0\\
326.01	0\\
327.01	0\\
328.01	0\\
329.01	0\\
330.01	0\\
331.01	0\\
332.01	0\\
333.01	0\\
334.01	0\\
335.01	0\\
336.01	0\\
337.01	0\\
338.01	0\\
339.01	0\\
340.01	0\\
341.01	0\\
342.01	0\\
343.01	0\\
344.01	0\\
345.01	0\\
346.01	0\\
347.01	0\\
348.01	0\\
349.01	0\\
350.01	0\\
351.01	0\\
352.01	0\\
353.01	0\\
354.01	0\\
355.01	0\\
356.01	0\\
357.01	0\\
358.01	0\\
359.01	0\\
360.01	0\\
361.01	0\\
362.01	0\\
363.01	0\\
364.01	0\\
365.01	0\\
366.01	0\\
367.01	0\\
368.01	0\\
369.01	0\\
370.01	0\\
371.01	0\\
372.01	0\\
373.01	0\\
374.01	0\\
375.01	0\\
376.01	0\\
377.01	0\\
378.01	0\\
379.01	0\\
380.01	0\\
381.01	0\\
382.01	0\\
383.01	0\\
384.01	0\\
385.01	0\\
386.01	0\\
387.01	0\\
388.01	0\\
389.01	0\\
390.01	0\\
391.01	0\\
392.01	0\\
393.01	0\\
394.01	0\\
395.01	0\\
396.01	0\\
397.01	0\\
398.01	0\\
399.01	0\\
400.01	0\\
401.01	0\\
402.01	0\\
403.01	0\\
404.01	0\\
405.01	0\\
406.01	0\\
407.01	0\\
408.01	0\\
409.01	0\\
410.01	0\\
411.01	0\\
412.01	0\\
413.01	0\\
414.01	0\\
415.01	0\\
416.01	0\\
417.01	0\\
418.01	0\\
419.01	0\\
420.01	0\\
421.01	0\\
422.01	0\\
423.01	0\\
424.01	0\\
425.01	0\\
426.01	0\\
427.01	0\\
428.01	0\\
429.01	0\\
430.01	0\\
431.01	0\\
432.01	0\\
433.01	0\\
434.01	0\\
435.01	0\\
436.01	0\\
437.01	0\\
438.01	0\\
439.01	0\\
440.01	0\\
441.01	0\\
442.01	0\\
443.01	0\\
444.01	0\\
445.01	0\\
446.01	0\\
447.01	0\\
448.01	0\\
449.01	0\\
450.01	0\\
451.01	0\\
452.01	0\\
453.01	0\\
454.01	0\\
455.01	0\\
456.01	0\\
457.01	0\\
458.01	0\\
459.01	0\\
460.01	0\\
461.01	0\\
462.01	0\\
463.01	0\\
464.01	0\\
465.01	0\\
466.01	0\\
467.01	0\\
468.01	0\\
469.01	0\\
470.01	0\\
471.01	0\\
472.01	0\\
473.01	0\\
474.01	0\\
475.01	0\\
476.01	0\\
477.01	0\\
478.01	0\\
479.01	0\\
480.01	0\\
481.01	0\\
482.01	0\\
483.01	0\\
484.01	0\\
485.01	0\\
486.01	0\\
487.01	0\\
488.01	0\\
489.01	0\\
490.01	0\\
491.01	0\\
492.01	0\\
493.01	0\\
494.01	0\\
495.01	0\\
496.01	0\\
497.01	0\\
498.01	0\\
499.01	0\\
500.01	0\\
501.01	0\\
502.01	0\\
503.01	0\\
504.01	0\\
505.01	0\\
506.01	0\\
507.01	0\\
508.01	0\\
509.01	0\\
510.01	0\\
511.01	0\\
512.01	0\\
513.01	0\\
514.01	0\\
515.01	0\\
516.01	0\\
517.01	0\\
518.01	0\\
519.01	0\\
520.01	0\\
521.01	0\\
522.01	0\\
523.01	0\\
524.01	0\\
525.01	0\\
526.01	0\\
527.01	0\\
528.01	0\\
529.01	0\\
530.01	0\\
531.01	0\\
532.01	0\\
533.01	0\\
534.01	0\\
535.01	0\\
536.01	0\\
537.01	0\\
538.01	0\\
539.01	0\\
540.01	0\\
541.01	0\\
542.01	0\\
543.01	0\\
544.01	0\\
545.01	0\\
546.01	0\\
547.01	0\\
548.01	0\\
549.01	0\\
550.01	0\\
551.01	0\\
552.01	0\\
553.01	0\\
554.01	0\\
555.01	0\\
556.01	0\\
557.01	0\\
558.01	0\\
559.01	0\\
560.01	0\\
561.01	0\\
562.01	0\\
563.01	0\\
564.01	0\\
565.01	0\\
566.01	0\\
567.01	0\\
568.01	0\\
569.01	0\\
570.01	0\\
571.01	0\\
572.01	0\\
573.01	0\\
574.01	0\\
575.01	0\\
576.01	0\\
577.01	0\\
578.01	0\\
579.01	0\\
580.01	0\\
581.01	0\\
582.01	0\\
583.01	0\\
584.01	0\\
585.01	0\\
586.01	0\\
587.01	0\\
588.01	0\\
589.01	0\\
590.01	0\\
591.01	0\\
592.01	0\\
593.01	0\\
594.01	0\\
595.01	0\\
596.01	0\\
597.01	0\\
598.01	0\\
599.01	0.0022382818960328\\
599.02	0.00228829715400941\\
599.03	0.00233870844988766\\
599.04	0.00238951982934219\\
599.05	0.00244073538661882\\
599.06	0.00249235926524579\\
599.07	0.00254439565875786\\
599.08	0.00259684881143348\\
599.09	0.00264972301904528\\
599.1	0.0027030226296243\\
599.11	0.00275675204423814\\
599.12	0.00281091571778341\\
599.13	0.00286551815979286\\
599.14	0.00292056393525741\\
599.15	0.00297605766546351\\
599.16	0.00303200402884618\\
599.17	0.00308840776185803\\
599.18	0.00314527365985481\\
599.19	0.00320260657799763\\
599.2	0.00326041143217246\\
599.21	0.00331869319992726\\
599.22	0.00337745692142712\\
599.23	0.00343670770042793\\
599.24	0.00349645070526896\\
599.25	0.00355669116988489\\
599.26	0.0036174343948377\\
599.27	0.00367868574836898\\
599.28	0.00374045066747309\\
599.29	0.00380273465899182\\
599.3	0.00386554330073107\\
599.31	0.00392888224260002\\
599.32	0.00399275720777354\\
599.33	0.00405717399387833\\
599.34	0.00412213847420342\\
599.35	0.00418765659893578\\
599.36	0.00425373439642159\\
599.37	0.00432037797445393\\
599.38	0.00438759352158759\\
599.39	0.00445538730848178\\
599.4	0.00452376568927147\\
599.41	0.0045927351029681\\
599.42	0.00466230207489068\\
599.43	0.0047324732181279\\
599.44	0.00480325523503231\\
599.45	0.0048746549187474\\
599.46	0.00494667915476945\\
599.47	0.005019334922583\\
599.48	0.00509262929729296\\
599.49	0.00516656945129368\\
599.5	0.00524116265597606\\
599.51	0.00531641628347389\\
599.52	0.00539233780845056\\
599.53	0.00546893480992739\\
599.54	0.00554621497315491\\
599.55	0.00562418609152839\\
599.56	0.00570285606854896\\
599.57	0.00578223291983187\\
599.58	0.00586232477516331\\
599.59	0.00594313988060741\\
599.6	0.006024686600665\\
599.61	0.00610697342048581\\
599.62	0.00619000894813599\\
599.63	0.00627380191692264\\
599.64	0.00635836118777732\\
599.65	0.00644369575170061\\
599.66	0.00652981473226971\\
599.67	0.00661672738821123\\
599.68	0.00670444311604154\\
599.69	0.00679297145277696\\
599.7	0.00688232207871629\\
599.71	0.00697250482029821\\
599.72	0.0070635296530363\\
599.73	0.00715540670453446\\
599.74	0.00724814625758565\\
599.75	0.00734175875335703\\
599.76	0.00743625479466474\\
599.77	0.0075316451493416\\
599.78	0.00762794075370135\\
599.79	0.00772515271610301\\
599.8	0.00782329232061922\\
599.81	0.00792237103081269\\
599.82	0.00802240049362481\\
599.83	0.00812339254338108\\
599.84	0.00822535920591771\\
599.85	0.00832831270283452\\
599.86	0.00843226545587912\\
599.87	0.00853723009146771\\
599.88	0.00864321944534821\\
599.89	0.00875024656741154\\
599.9	0.0088583247266573\\
599.91	0.00896746741632041\\
599.92	0.00907768835916546\\
599.93	0.0091890015129561\\
599.94	0.00930142107610699\\
599.95	0.00941496149352625\\
599.96	0.009529637462657\\
599.97	0.00964546393972657\\
599.98	0.00976245614621301\\
599.99	0.00988062957553847\\
600	0.01\\
};
\addplot [color=mycolor14,solid,forget plot]
  table[row sep=crcr]{%
0.01	0\\
1.01	0\\
2.01	0\\
3.01	0\\
4.01	0\\
5.01	0\\
6.01	0\\
7.01	0\\
8.01	0\\
9.01	0\\
10.01	0\\
11.01	0\\
12.01	0\\
13.01	0\\
14.01	0\\
15.01	0\\
16.01	0\\
17.01	0\\
18.01	0\\
19.01	0\\
20.01	0\\
21.01	0\\
22.01	0\\
23.01	0\\
24.01	0\\
25.01	0\\
26.01	0\\
27.01	0\\
28.01	0\\
29.01	0\\
30.01	0\\
31.01	0\\
32.01	0\\
33.01	0\\
34.01	0\\
35.01	0\\
36.01	0\\
37.01	0\\
38.01	0\\
39.01	0\\
40.01	0\\
41.01	0\\
42.01	0\\
43.01	0\\
44.01	0\\
45.01	0\\
46.01	0\\
47.01	0\\
48.01	0\\
49.01	0\\
50.01	0\\
51.01	0\\
52.01	0\\
53.01	0\\
54.01	0\\
55.01	0\\
56.01	0\\
57.01	0\\
58.01	0\\
59.01	0\\
60.01	0\\
61.01	0\\
62.01	0\\
63.01	0\\
64.01	0\\
65.01	0\\
66.01	0\\
67.01	0\\
68.01	0\\
69.01	0\\
70.01	0\\
71.01	0\\
72.01	0\\
73.01	0\\
74.01	0\\
75.01	0\\
76.01	0\\
77.01	0\\
78.01	0\\
79.01	0\\
80.01	0\\
81.01	0\\
82.01	0\\
83.01	0\\
84.01	0\\
85.01	0\\
86.01	0\\
87.01	0\\
88.01	0\\
89.01	0\\
90.01	0\\
91.01	0\\
92.01	0\\
93.01	0\\
94.01	0\\
95.01	0\\
96.01	0\\
97.01	0\\
98.01	0\\
99.01	0\\
100.01	0\\
101.01	0\\
102.01	0\\
103.01	0\\
104.01	0\\
105.01	0\\
106.01	0\\
107.01	0\\
108.01	0\\
109.01	0\\
110.01	0\\
111.01	0\\
112.01	0\\
113.01	0\\
114.01	0\\
115.01	0\\
116.01	0\\
117.01	0\\
118.01	0\\
119.01	0\\
120.01	0\\
121.01	0\\
122.01	0\\
123.01	0\\
124.01	0\\
125.01	0\\
126.01	0\\
127.01	0\\
128.01	0\\
129.01	0\\
130.01	0\\
131.01	0\\
132.01	0\\
133.01	0\\
134.01	0\\
135.01	0\\
136.01	0\\
137.01	0\\
138.01	0\\
139.01	0\\
140.01	0\\
141.01	0\\
142.01	0\\
143.01	0\\
144.01	0\\
145.01	0\\
146.01	0\\
147.01	0\\
148.01	0\\
149.01	0\\
150.01	0\\
151.01	0\\
152.01	0\\
153.01	0\\
154.01	0\\
155.01	0\\
156.01	0\\
157.01	0\\
158.01	0\\
159.01	0\\
160.01	0\\
161.01	0\\
162.01	0\\
163.01	0\\
164.01	0\\
165.01	0\\
166.01	0\\
167.01	0\\
168.01	0\\
169.01	0\\
170.01	0\\
171.01	0\\
172.01	0\\
173.01	0\\
174.01	0\\
175.01	0\\
176.01	0\\
177.01	0\\
178.01	0\\
179.01	0\\
180.01	0\\
181.01	0\\
182.01	0\\
183.01	0\\
184.01	0\\
185.01	0\\
186.01	0\\
187.01	0\\
188.01	0\\
189.01	0\\
190.01	0\\
191.01	0\\
192.01	0\\
193.01	0\\
194.01	0\\
195.01	0\\
196.01	0\\
197.01	0\\
198.01	0\\
199.01	0\\
200.01	0\\
201.01	0\\
202.01	0\\
203.01	0\\
204.01	0\\
205.01	0\\
206.01	0\\
207.01	0\\
208.01	0\\
209.01	0\\
210.01	0\\
211.01	0\\
212.01	0\\
213.01	0\\
214.01	0\\
215.01	0\\
216.01	0\\
217.01	0\\
218.01	0\\
219.01	0\\
220.01	0\\
221.01	0\\
222.01	0\\
223.01	0\\
224.01	0\\
225.01	0\\
226.01	0\\
227.01	0\\
228.01	0\\
229.01	0\\
230.01	0\\
231.01	0\\
232.01	0\\
233.01	0\\
234.01	0\\
235.01	0\\
236.01	0\\
237.01	0\\
238.01	0\\
239.01	0\\
240.01	0\\
241.01	0\\
242.01	0\\
243.01	0\\
244.01	0\\
245.01	0\\
246.01	0\\
247.01	0\\
248.01	0\\
249.01	0\\
250.01	0\\
251.01	0\\
252.01	0\\
253.01	0\\
254.01	0\\
255.01	0\\
256.01	0\\
257.01	0\\
258.01	0\\
259.01	0\\
260.01	0\\
261.01	0\\
262.01	0\\
263.01	0\\
264.01	0\\
265.01	0\\
266.01	0\\
267.01	0\\
268.01	0\\
269.01	0\\
270.01	0\\
271.01	0\\
272.01	0\\
273.01	0\\
274.01	0\\
275.01	0\\
276.01	0\\
277.01	0\\
278.01	0\\
279.01	0\\
280.01	0\\
281.01	0\\
282.01	0\\
283.01	0\\
284.01	0\\
285.01	0\\
286.01	0\\
287.01	0\\
288.01	0\\
289.01	0\\
290.01	0\\
291.01	0\\
292.01	0\\
293.01	0\\
294.01	0\\
295.01	0\\
296.01	0\\
297.01	0\\
298.01	0\\
299.01	0\\
300.01	0\\
301.01	0\\
302.01	0\\
303.01	0\\
304.01	0\\
305.01	0\\
306.01	0\\
307.01	0\\
308.01	0\\
309.01	0\\
310.01	0\\
311.01	0\\
312.01	0\\
313.01	0\\
314.01	0\\
315.01	0\\
316.01	0\\
317.01	0\\
318.01	0\\
319.01	0\\
320.01	0\\
321.01	0\\
322.01	0\\
323.01	0\\
324.01	0\\
325.01	0\\
326.01	0\\
327.01	0\\
328.01	0\\
329.01	0\\
330.01	0\\
331.01	0\\
332.01	0\\
333.01	0\\
334.01	0\\
335.01	0\\
336.01	0\\
337.01	0\\
338.01	0\\
339.01	0\\
340.01	0\\
341.01	0\\
342.01	0\\
343.01	0\\
344.01	0\\
345.01	0\\
346.01	0\\
347.01	0\\
348.01	0\\
349.01	0\\
350.01	0\\
351.01	0\\
352.01	0\\
353.01	0\\
354.01	0\\
355.01	0\\
356.01	0\\
357.01	0\\
358.01	0\\
359.01	0\\
360.01	0\\
361.01	0\\
362.01	0\\
363.01	0\\
364.01	0\\
365.01	0\\
366.01	0\\
367.01	0\\
368.01	0\\
369.01	0\\
370.01	0\\
371.01	0\\
372.01	0\\
373.01	0\\
374.01	0\\
375.01	0\\
376.01	0\\
377.01	0\\
378.01	0\\
379.01	0\\
380.01	0\\
381.01	0\\
382.01	0\\
383.01	0\\
384.01	0\\
385.01	0\\
386.01	0\\
387.01	0\\
388.01	0\\
389.01	0\\
390.01	0\\
391.01	0\\
392.01	0\\
393.01	0\\
394.01	0\\
395.01	0\\
396.01	0\\
397.01	0\\
398.01	0\\
399.01	0\\
400.01	0\\
401.01	0\\
402.01	0\\
403.01	0\\
404.01	0\\
405.01	0\\
406.01	0\\
407.01	0\\
408.01	0\\
409.01	0\\
410.01	0\\
411.01	0\\
412.01	0\\
413.01	0\\
414.01	0\\
415.01	0\\
416.01	0\\
417.01	0\\
418.01	0\\
419.01	0\\
420.01	0\\
421.01	0\\
422.01	0\\
423.01	0\\
424.01	0\\
425.01	0\\
426.01	0\\
427.01	0\\
428.01	0\\
429.01	0\\
430.01	0\\
431.01	0\\
432.01	0\\
433.01	0\\
434.01	0\\
435.01	0\\
436.01	0\\
437.01	0\\
438.01	0\\
439.01	0\\
440.01	0\\
441.01	0\\
442.01	0\\
443.01	0\\
444.01	0\\
445.01	0\\
446.01	0\\
447.01	0\\
448.01	0\\
449.01	0\\
450.01	0\\
451.01	0\\
452.01	0\\
453.01	0\\
454.01	0\\
455.01	0\\
456.01	0\\
457.01	0\\
458.01	0\\
459.01	0\\
460.01	0\\
461.01	0\\
462.01	0\\
463.01	0\\
464.01	0\\
465.01	0\\
466.01	0\\
467.01	0\\
468.01	0\\
469.01	0\\
470.01	0\\
471.01	0\\
472.01	0\\
473.01	0\\
474.01	0\\
475.01	0\\
476.01	0\\
477.01	0\\
478.01	0\\
479.01	0\\
480.01	0\\
481.01	0\\
482.01	0\\
483.01	0\\
484.01	0\\
485.01	0\\
486.01	0\\
487.01	0\\
488.01	0\\
489.01	0\\
490.01	0\\
491.01	0\\
492.01	0\\
493.01	0\\
494.01	0\\
495.01	0\\
496.01	0\\
497.01	0\\
498.01	0\\
499.01	0\\
500.01	0\\
501.01	0\\
502.01	0\\
503.01	0\\
504.01	0\\
505.01	0\\
506.01	0\\
507.01	0\\
508.01	0\\
509.01	0\\
510.01	0\\
511.01	0\\
512.01	0\\
513.01	0\\
514.01	0\\
515.01	0\\
516.01	0\\
517.01	0\\
518.01	0\\
519.01	0\\
520.01	0\\
521.01	0\\
522.01	0\\
523.01	0\\
524.01	0\\
525.01	0\\
526.01	0\\
527.01	0\\
528.01	0\\
529.01	0\\
530.01	0\\
531.01	0\\
532.01	0\\
533.01	0\\
534.01	0\\
535.01	0\\
536.01	0\\
537.01	0\\
538.01	0\\
539.01	0\\
540.01	0\\
541.01	0\\
542.01	0\\
543.01	0\\
544.01	0\\
545.01	0\\
546.01	0\\
547.01	0\\
548.01	0\\
549.01	0\\
550.01	0\\
551.01	0\\
552.01	0\\
553.01	0\\
554.01	0\\
555.01	0\\
556.01	0\\
557.01	0\\
558.01	0\\
559.01	0\\
560.01	0\\
561.01	0\\
562.01	0\\
563.01	0\\
564.01	0\\
565.01	0\\
566.01	0\\
567.01	0\\
568.01	0\\
569.01	0\\
570.01	0\\
571.01	0\\
572.01	0\\
573.01	0\\
574.01	0\\
575.01	0\\
576.01	0\\
577.01	0\\
578.01	0\\
579.01	0\\
580.01	0\\
581.01	0\\
582.01	0\\
583.01	0\\
584.01	0\\
585.01	0\\
586.01	0\\
587.01	0\\
588.01	0\\
589.01	0\\
590.01	0\\
591.01	0\\
592.01	0\\
593.01	0\\
594.01	0\\
595.01	0\\
596.01	0\\
597.01	0\\
598.01	3.69463863061463e-05\\
599.01	0.00379444900894968\\
599.02	0.00383270374370267\\
599.03	0.00387131777133739\\
599.04	0.00391029453419965\\
599.05	0.0039496375063856\\
599.06	0.003989350193997\\
599.07	0.00402943613539713\\
599.08	0.00406989890146739\\
599.09	0.00411074209586436\\
599.1	0.00415196935527735\\
599.11	0.00419358434968641\\
599.12	0.00423559078262063\\
599.13	0.00427799239141662\\
599.14	0.00432079294747728\\
599.15	0.00436399625653049\\
599.16	0.00440760615888791\\
599.17	0.00445162652970357\\
599.18	0.00449606127923222\\
599.19	0.00454091435308739\\
599.2	0.00458618973249894\\
599.21	0.00463189143457006\\
599.22	0.00467802351253352\\
599.23	0.00472459005600711\\
599.24	0.00477159519124805\\
599.25	0.00481904308140627\\
599.26	0.00486693792678842\\
599.27	0.00491528396512382\\
599.28	0.00496408547181374\\
599.29	0.00501334676017881\\
599.3	0.00506307218170441\\
599.31	0.00511326612628387\\
599.32	0.00516393302245909\\
599.33	0.00521507733765866\\
599.34	0.00526670357843294\\
599.35	0.00531881629068601\\
599.36	0.00537142005990418\\
599.37	0.00542451951138086\\
599.38	0.00547811931043734\\
599.39	0.00553222416263933\\
599.4	0.00558683881400888\\
599.41	0.00564196805123132\\
599.42	0.00569761670185686\\
599.43	0.00575378963449661\\
599.44	0.00581049175901236\\
599.45	0.00586772802670002\\
599.46	0.00592550342982103\\
599.47	0.00598382297386091\\
599.48	0.00604269170566519\\
599.49	0.00610211471358306\\
599.5	0.00616209712760199\\
599.51	0.00622264411947269\\
599.52	0.00628376090282387\\
599.53	0.00634545273326615\\
599.54	0.00640772490848447\\
599.55	0.00647058276831837\\
599.56	0.00653403169482929\\
599.57	0.00659807711235419\\
599.58	0.00666272448754477\\
599.59	0.00672797932939122\\
599.6	0.00679384718922988\\
599.61	0.0068603336607336\\
599.62	0.00692744437988406\\
599.63	0.00699518502492475\\
599.64	0.00706356131629375\\
599.65	0.007132579016535\\
599.66	0.00720224393018678\\
599.67	0.00727256190364638\\
599.68	0.00734353882500926\\
599.69	0.00741518062388145\\
599.7	0.00748749327116373\\
599.71	0.00756048277880581\\
599.72	0.00763415519952897\\
599.73	0.00770851662651528\\
599.74	0.00778357319306161\\
599.75	0.00785933107219639\\
599.76	0.00793579647625698\\
599.77	0.00801297565642555\\
599.78	0.00809087490222101\\
599.79	0.00816950054094457\\
599.8	0.00824885893707632\\
599.81	0.00832895649161999\\
599.82	0.00840979964139308\\
599.83	0.00849139485825914\\
599.84	0.00857374864829899\\
599.85	0.0086568675509174\\
599.86	0.00874075813788151\\
599.87	0.00882542701228713\\
599.88	0.00891088080744883\\
599.89	0.00899712618570936\\
599.9	0.00908416983716382\\
599.91	0.00917201847829369\\
599.92	0.00926067885050539\\
599.93	0.00935015771856803\\
599.94	0.00944046186894425\\
599.95	0.00953159810800815\\
599.96	0.00962357326014348\\
599.97	0.00971639416571523\\
599.98	0.00981006767890704\\
599.99	0.00990460066541651\\
600	0.01\\
};
\addplot [color=mycolor15,solid,forget plot]
  table[row sep=crcr]{%
0.01	0\\
1.01	0\\
2.01	0\\
3.01	0\\
4.01	0\\
5.01	0\\
6.01	0\\
7.01	0\\
8.01	0\\
9.01	0\\
10.01	0\\
11.01	0\\
12.01	0\\
13.01	0\\
14.01	0\\
15.01	0\\
16.01	0\\
17.01	0\\
18.01	0\\
19.01	0\\
20.01	0\\
21.01	0\\
22.01	0\\
23.01	0\\
24.01	0\\
25.01	0\\
26.01	0\\
27.01	0\\
28.01	0\\
29.01	0\\
30.01	0\\
31.01	0\\
32.01	0\\
33.01	0\\
34.01	0\\
35.01	0\\
36.01	0\\
37.01	0\\
38.01	0\\
39.01	0\\
40.01	0\\
41.01	0\\
42.01	0\\
43.01	0\\
44.01	0\\
45.01	0\\
46.01	0\\
47.01	0\\
48.01	0\\
49.01	0\\
50.01	0\\
51.01	0\\
52.01	0\\
53.01	0\\
54.01	0\\
55.01	0\\
56.01	0\\
57.01	0\\
58.01	0\\
59.01	0\\
60.01	0\\
61.01	0\\
62.01	0\\
63.01	0\\
64.01	0\\
65.01	0\\
66.01	0\\
67.01	0\\
68.01	0\\
69.01	0\\
70.01	0\\
71.01	0\\
72.01	0\\
73.01	0\\
74.01	0\\
75.01	0\\
76.01	0\\
77.01	0\\
78.01	0\\
79.01	0\\
80.01	0\\
81.01	0\\
82.01	0\\
83.01	0\\
84.01	0\\
85.01	0\\
86.01	0\\
87.01	0\\
88.01	0\\
89.01	0\\
90.01	0\\
91.01	0\\
92.01	0\\
93.01	0\\
94.01	0\\
95.01	0\\
96.01	0\\
97.01	0\\
98.01	0\\
99.01	0\\
100.01	0\\
101.01	0\\
102.01	0\\
103.01	0\\
104.01	0\\
105.01	0\\
106.01	0\\
107.01	0\\
108.01	0\\
109.01	0\\
110.01	0\\
111.01	0\\
112.01	0\\
113.01	0\\
114.01	0\\
115.01	0\\
116.01	0\\
117.01	0\\
118.01	0\\
119.01	0\\
120.01	0\\
121.01	0\\
122.01	0\\
123.01	0\\
124.01	0\\
125.01	0\\
126.01	0\\
127.01	0\\
128.01	0\\
129.01	0\\
130.01	0\\
131.01	0\\
132.01	0\\
133.01	0\\
134.01	0\\
135.01	0\\
136.01	0\\
137.01	0\\
138.01	0\\
139.01	0\\
140.01	0\\
141.01	0\\
142.01	0\\
143.01	0\\
144.01	0\\
145.01	0\\
146.01	0\\
147.01	0\\
148.01	0\\
149.01	0\\
150.01	0\\
151.01	0\\
152.01	0\\
153.01	0\\
154.01	0\\
155.01	0\\
156.01	0\\
157.01	0\\
158.01	0\\
159.01	0\\
160.01	0\\
161.01	0\\
162.01	0\\
163.01	0\\
164.01	0\\
165.01	0\\
166.01	0\\
167.01	0\\
168.01	0\\
169.01	0\\
170.01	0\\
171.01	0\\
172.01	0\\
173.01	0\\
174.01	0\\
175.01	0\\
176.01	0\\
177.01	0\\
178.01	0\\
179.01	0\\
180.01	0\\
181.01	0\\
182.01	0\\
183.01	0\\
184.01	0\\
185.01	0\\
186.01	0\\
187.01	0\\
188.01	0\\
189.01	0\\
190.01	0\\
191.01	0\\
192.01	0\\
193.01	0\\
194.01	0\\
195.01	0\\
196.01	0\\
197.01	0\\
198.01	0\\
199.01	0\\
200.01	0\\
201.01	0\\
202.01	0\\
203.01	0\\
204.01	0\\
205.01	0\\
206.01	0\\
207.01	0\\
208.01	0\\
209.01	0\\
210.01	0\\
211.01	0\\
212.01	0\\
213.01	0\\
214.01	0\\
215.01	0\\
216.01	0\\
217.01	0\\
218.01	0\\
219.01	0\\
220.01	0\\
221.01	0\\
222.01	0\\
223.01	0\\
224.01	0\\
225.01	0\\
226.01	0\\
227.01	0\\
228.01	0\\
229.01	0\\
230.01	0\\
231.01	0\\
232.01	0\\
233.01	0\\
234.01	0\\
235.01	0\\
236.01	0\\
237.01	0\\
238.01	0\\
239.01	0\\
240.01	0\\
241.01	0\\
242.01	0\\
243.01	0\\
244.01	0\\
245.01	0\\
246.01	0\\
247.01	0\\
248.01	0\\
249.01	0\\
250.01	0\\
251.01	0\\
252.01	0\\
253.01	0\\
254.01	0\\
255.01	0\\
256.01	0\\
257.01	0\\
258.01	0\\
259.01	0\\
260.01	0\\
261.01	0\\
262.01	0\\
263.01	0\\
264.01	0\\
265.01	0\\
266.01	0\\
267.01	0\\
268.01	0\\
269.01	0\\
270.01	0\\
271.01	0\\
272.01	0\\
273.01	0\\
274.01	0\\
275.01	0\\
276.01	0\\
277.01	0\\
278.01	0\\
279.01	0\\
280.01	0\\
281.01	0\\
282.01	0\\
283.01	0\\
284.01	0\\
285.01	0\\
286.01	0\\
287.01	0\\
288.01	0\\
289.01	0\\
290.01	0\\
291.01	0\\
292.01	0\\
293.01	0\\
294.01	0\\
295.01	0\\
296.01	0\\
297.01	0\\
298.01	0\\
299.01	0\\
300.01	0\\
301.01	0\\
302.01	0\\
303.01	0\\
304.01	0\\
305.01	0\\
306.01	0\\
307.01	0\\
308.01	0\\
309.01	0\\
310.01	0\\
311.01	0\\
312.01	0\\
313.01	0\\
314.01	0\\
315.01	0\\
316.01	0\\
317.01	0\\
318.01	0\\
319.01	0\\
320.01	0\\
321.01	0\\
322.01	0\\
323.01	0\\
324.01	0\\
325.01	0\\
326.01	0\\
327.01	0\\
328.01	0\\
329.01	0\\
330.01	0\\
331.01	0\\
332.01	0\\
333.01	0\\
334.01	0\\
335.01	0\\
336.01	0\\
337.01	0\\
338.01	0\\
339.01	0\\
340.01	0\\
341.01	0\\
342.01	0\\
343.01	0\\
344.01	0\\
345.01	0\\
346.01	0\\
347.01	0\\
348.01	0\\
349.01	0\\
350.01	0\\
351.01	0\\
352.01	0\\
353.01	0\\
354.01	0\\
355.01	0\\
356.01	0\\
357.01	0\\
358.01	0\\
359.01	0\\
360.01	0\\
361.01	0\\
362.01	0\\
363.01	0\\
364.01	0\\
365.01	0\\
366.01	0\\
367.01	0\\
368.01	0\\
369.01	0\\
370.01	0\\
371.01	0\\
372.01	0\\
373.01	0\\
374.01	0\\
375.01	0\\
376.01	0\\
377.01	0\\
378.01	0\\
379.01	0\\
380.01	0\\
381.01	0\\
382.01	0\\
383.01	0\\
384.01	0\\
385.01	0\\
386.01	0\\
387.01	0\\
388.01	0\\
389.01	0\\
390.01	0\\
391.01	0\\
392.01	0\\
393.01	0\\
394.01	0\\
395.01	0\\
396.01	0\\
397.01	0\\
398.01	0\\
399.01	0\\
400.01	0\\
401.01	0\\
402.01	0\\
403.01	0\\
404.01	0\\
405.01	0\\
406.01	0\\
407.01	0\\
408.01	0\\
409.01	0\\
410.01	0\\
411.01	0\\
412.01	0\\
413.01	0\\
414.01	0\\
415.01	0\\
416.01	0\\
417.01	0\\
418.01	0\\
419.01	0\\
420.01	0\\
421.01	0\\
422.01	0\\
423.01	0\\
424.01	0\\
425.01	0\\
426.01	0\\
427.01	0\\
428.01	0\\
429.01	0\\
430.01	0\\
431.01	0\\
432.01	0\\
433.01	0\\
434.01	0\\
435.01	0\\
436.01	0\\
437.01	0\\
438.01	0\\
439.01	0\\
440.01	0\\
441.01	0\\
442.01	0\\
443.01	0\\
444.01	0\\
445.01	0\\
446.01	0\\
447.01	0\\
448.01	0\\
449.01	0\\
450.01	0\\
451.01	0\\
452.01	0\\
453.01	0\\
454.01	0\\
455.01	0\\
456.01	0\\
457.01	0\\
458.01	0\\
459.01	0\\
460.01	0\\
461.01	0\\
462.01	0\\
463.01	0\\
464.01	0\\
465.01	0\\
466.01	0\\
467.01	0\\
468.01	0\\
469.01	0\\
470.01	0\\
471.01	0\\
472.01	0\\
473.01	0\\
474.01	0\\
475.01	0\\
476.01	0\\
477.01	0\\
478.01	0\\
479.01	0\\
480.01	0\\
481.01	0\\
482.01	0\\
483.01	0\\
484.01	0\\
485.01	0\\
486.01	0\\
487.01	0\\
488.01	0\\
489.01	0\\
490.01	0\\
491.01	0\\
492.01	0\\
493.01	0\\
494.01	0\\
495.01	0\\
496.01	0\\
497.01	0\\
498.01	0\\
499.01	0\\
500.01	0\\
501.01	0\\
502.01	0\\
503.01	0\\
504.01	0\\
505.01	0\\
506.01	0\\
507.01	0\\
508.01	0\\
509.01	0\\
510.01	0\\
511.01	0\\
512.01	0\\
513.01	0\\
514.01	0\\
515.01	0\\
516.01	0\\
517.01	0\\
518.01	0\\
519.01	0\\
520.01	0\\
521.01	0\\
522.01	0\\
523.01	0\\
524.01	0\\
525.01	0\\
526.01	0\\
527.01	0\\
528.01	0\\
529.01	0\\
530.01	0\\
531.01	0\\
532.01	0\\
533.01	0\\
534.01	0\\
535.01	0\\
536.01	0\\
537.01	0\\
538.01	0\\
539.01	0\\
540.01	0\\
541.01	0\\
542.01	0\\
543.01	0\\
544.01	0\\
545.01	0\\
546.01	0\\
547.01	0\\
548.01	0\\
549.01	0\\
550.01	0\\
551.01	0\\
552.01	0\\
553.01	0\\
554.01	0\\
555.01	0\\
556.01	0\\
557.01	0\\
558.01	0\\
559.01	0\\
560.01	0\\
561.01	0\\
562.01	0\\
563.01	0\\
564.01	0\\
565.01	0\\
566.01	0\\
567.01	0\\
568.01	0\\
569.01	0\\
570.01	0\\
571.01	0\\
572.01	0\\
573.01	0\\
574.01	0\\
575.01	0\\
576.01	0\\
577.01	0\\
578.01	0\\
579.01	0\\
580.01	0\\
581.01	0\\
582.01	0\\
583.01	0\\
584.01	0\\
585.01	0\\
586.01	0\\
587.01	0\\
588.01	0\\
589.01	0\\
590.01	0\\
591.01	0\\
592.01	0\\
593.01	0\\
594.01	0\\
595.01	0\\
596.01	0\\
597.01	0\\
598.01	0.00141368560759025\\
599.01	0.00385460611783226\\
599.02	0.00389217668082419\\
599.03	0.00393010478508591\\
599.04	0.00396839388118488\\
599.05	0.00400704745291263\\
599.06	0.00404606901760485\\
599.07	0.00408546212646469\\
599.08	0.00412523036488917\\
599.09	0.00416537735279883\\
599.1	0.00420590674497062\\
599.11	0.00424682223137403\\
599.12	0.0042881275375106\\
599.13	0.00432982642475674\\
599.14	0.0043719226907099\\
599.15	0.0044144201695383\\
599.16	0.00445732273233393\\
599.17	0.00450063428746929\\
599.18	0.00454435878095746\\
599.19	0.00458850019681592\\
599.2	0.00463306255743389\\
599.21	0.00467804992394345\\
599.22	0.0047234663965943\\
599.23	0.00476931611513231\\
599.24	0.00481560325918193\\
599.25	0.00486233204863238\\
599.26	0.00490950673609608\\
599.27	0.00495713160496067\\
599.28	0.0050052109796917\\
599.29	0.00505374922622871\\
599.3	0.00510275075238528\\
599.31	0.00515222000825318\\
599.32	0.00520216148661058\\
599.33	0.00525257972333446\\
599.34	0.00530347929781725\\
599.35	0.00535486483338777\\
599.36	0.00540674099773647\\
599.37	0.00545911250334512\\
599.38	0.00551198410792095\\
599.39	0.00556536061483532\\
599.4	0.00561924687356703\\
599.41	0.00567364778015027\\
599.42	0.00572856827762734\\
599.43	0.00578401335650615\\
599.44	0.00583998805522266\\
599.45	0.00589649746060833\\
599.46	0.00595354670836325\\
599.47	0.00601114098356581\\
599.48	0.00606928552115735\\
599.49	0.00612798560643227\\
599.5	0.00618724657553358\\
599.51	0.00624707381595411\\
599.52	0.00630747276704345\\
599.53	0.00636844892052068\\
599.54	0.00643000782099308\\
599.55	0.00649215506648088\\
599.56	0.00655489630894826\\
599.57	0.00661823725484062\\
599.58	0.00668218366562836\\
599.59	0.00674674135835722\\
599.6	0.00681191620620549\\
599.61	0.006877714139048\\
599.62	0.00694414114402729\\
599.63	0.00701120326613206\\
599.64	0.00707890660878295\\
599.65	0.00714725733442608\\
599.66	0.00721626166513437\\
599.67	0.0072859258832169\\
599.68	0.00735625633183653\\
599.69	0.00742725941563611\\
599.7	0.00749894160137328\\
599.71	0.00757130941856446\\
599.72	0.00764436946013798\\
599.73	0.00771812838309687\\
599.74	0.00779259290919147\\
599.75	0.00786776982560227\\
599.76	0.0079436659856333\\
599.77	0.00802028830941633\\
599.78	0.00809764378462653\\
599.79	0.00817573946720956\\
599.8	0.00825458248212104\\
599.81	0.00833418002407833\\
599.82	0.00841453935832557\\
599.83	0.00849566782141209\\
599.84	0.00857757282198503\\
599.85	0.00866026184159656\\
599.86	0.00874374243552642\\
599.87	0.00882802223362036\\
599.88	0.00891310894114515\\
599.89	0.00899901033966106\\
599.9	0.0090857342879123\\
599.91	0.00917328872273657\\
599.92	0.00926168165999438\\
599.93	0.00935092119551921\\
599.94	0.0094410155060894\\
599.95	0.009531972850423\\
599.96	0.00962380157019665\\
599.97	0.00971651009108966\\
599.98	0.0098101069238547\\
599.99	0.00990460066541651\\
600	0.01\\
};
\addplot [color=mycolor16,solid,forget plot]
  table[row sep=crcr]{%
0.01	0\\
1.01	0\\
2.01	0\\
3.01	0\\
4.01	0\\
5.01	0\\
6.01	0\\
7.01	0\\
8.01	0\\
9.01	0\\
10.01	0\\
11.01	0\\
12.01	0\\
13.01	0\\
14.01	0\\
15.01	0\\
16.01	0\\
17.01	0\\
18.01	0\\
19.01	0\\
20.01	0\\
21.01	0\\
22.01	0\\
23.01	0\\
24.01	0\\
25.01	0\\
26.01	0\\
27.01	0\\
28.01	0\\
29.01	0\\
30.01	0\\
31.01	0\\
32.01	0\\
33.01	0\\
34.01	0\\
35.01	0\\
36.01	0\\
37.01	0\\
38.01	0\\
39.01	0\\
40.01	0\\
41.01	0\\
42.01	0\\
43.01	0\\
44.01	0\\
45.01	0\\
46.01	0\\
47.01	0\\
48.01	0\\
49.01	0\\
50.01	0\\
51.01	0\\
52.01	0\\
53.01	0\\
54.01	0\\
55.01	0\\
56.01	0\\
57.01	0\\
58.01	0\\
59.01	0\\
60.01	0\\
61.01	0\\
62.01	0\\
63.01	0\\
64.01	0\\
65.01	0\\
66.01	0\\
67.01	0\\
68.01	0\\
69.01	0\\
70.01	0\\
71.01	0\\
72.01	0\\
73.01	0\\
74.01	0\\
75.01	0\\
76.01	0\\
77.01	0\\
78.01	0\\
79.01	0\\
80.01	0\\
81.01	0\\
82.01	0\\
83.01	0\\
84.01	0\\
85.01	0\\
86.01	0\\
87.01	0\\
88.01	0\\
89.01	0\\
90.01	0\\
91.01	0\\
92.01	0\\
93.01	0\\
94.01	0\\
95.01	0\\
96.01	0\\
97.01	0\\
98.01	0\\
99.01	0\\
100.01	0\\
101.01	0\\
102.01	0\\
103.01	0\\
104.01	0\\
105.01	0\\
106.01	0\\
107.01	0\\
108.01	0\\
109.01	0\\
110.01	0\\
111.01	0\\
112.01	0\\
113.01	0\\
114.01	0\\
115.01	0\\
116.01	0\\
117.01	0\\
118.01	0\\
119.01	0\\
120.01	0\\
121.01	0\\
122.01	0\\
123.01	0\\
124.01	0\\
125.01	0\\
126.01	0\\
127.01	0\\
128.01	0\\
129.01	0\\
130.01	0\\
131.01	0\\
132.01	0\\
133.01	0\\
134.01	0\\
135.01	0\\
136.01	0\\
137.01	0\\
138.01	0\\
139.01	0\\
140.01	0\\
141.01	0\\
142.01	0\\
143.01	0\\
144.01	0\\
145.01	0\\
146.01	0\\
147.01	0\\
148.01	0\\
149.01	0\\
150.01	0\\
151.01	0\\
152.01	0\\
153.01	0\\
154.01	0\\
155.01	0\\
156.01	0\\
157.01	0\\
158.01	0\\
159.01	0\\
160.01	0\\
161.01	0\\
162.01	0\\
163.01	0\\
164.01	0\\
165.01	0\\
166.01	0\\
167.01	0\\
168.01	0\\
169.01	0\\
170.01	0\\
171.01	0\\
172.01	0\\
173.01	0\\
174.01	0\\
175.01	0\\
176.01	0\\
177.01	0\\
178.01	0\\
179.01	0\\
180.01	0\\
181.01	0\\
182.01	0\\
183.01	0\\
184.01	0\\
185.01	0\\
186.01	0\\
187.01	0\\
188.01	0\\
189.01	0\\
190.01	0\\
191.01	0\\
192.01	0\\
193.01	0\\
194.01	0\\
195.01	0\\
196.01	0\\
197.01	0\\
198.01	0\\
199.01	0\\
200.01	0\\
201.01	0\\
202.01	0\\
203.01	0\\
204.01	0\\
205.01	0\\
206.01	0\\
207.01	0\\
208.01	0\\
209.01	0\\
210.01	0\\
211.01	0\\
212.01	0\\
213.01	0\\
214.01	0\\
215.01	0\\
216.01	0\\
217.01	0\\
218.01	0\\
219.01	0\\
220.01	0\\
221.01	0\\
222.01	0\\
223.01	0\\
224.01	0\\
225.01	0\\
226.01	0\\
227.01	0\\
228.01	0\\
229.01	0\\
230.01	0\\
231.01	0\\
232.01	0\\
233.01	0\\
234.01	0\\
235.01	0\\
236.01	0\\
237.01	0\\
238.01	0\\
239.01	0\\
240.01	0\\
241.01	0\\
242.01	0\\
243.01	0\\
244.01	0\\
245.01	0\\
246.01	0\\
247.01	0\\
248.01	0\\
249.01	0\\
250.01	0\\
251.01	0\\
252.01	0\\
253.01	0\\
254.01	0\\
255.01	0\\
256.01	0\\
257.01	0\\
258.01	0\\
259.01	0\\
260.01	0\\
261.01	0\\
262.01	0\\
263.01	0\\
264.01	0\\
265.01	0\\
266.01	0\\
267.01	0\\
268.01	0\\
269.01	0\\
270.01	0\\
271.01	0\\
272.01	0\\
273.01	0\\
274.01	0\\
275.01	0\\
276.01	0\\
277.01	0\\
278.01	0\\
279.01	0\\
280.01	0\\
281.01	0\\
282.01	0\\
283.01	0\\
284.01	0\\
285.01	0\\
286.01	0\\
287.01	0\\
288.01	0\\
289.01	0\\
290.01	0\\
291.01	0\\
292.01	0\\
293.01	0\\
294.01	0\\
295.01	0\\
296.01	0\\
297.01	0\\
298.01	0\\
299.01	0\\
300.01	0\\
301.01	0\\
302.01	0\\
303.01	0\\
304.01	0\\
305.01	0\\
306.01	0\\
307.01	0\\
308.01	0\\
309.01	0\\
310.01	0\\
311.01	0\\
312.01	0\\
313.01	0\\
314.01	0\\
315.01	0\\
316.01	0\\
317.01	0\\
318.01	0\\
319.01	0\\
320.01	0\\
321.01	0\\
322.01	0\\
323.01	0\\
324.01	0\\
325.01	0\\
326.01	0\\
327.01	0\\
328.01	0\\
329.01	0\\
330.01	0\\
331.01	0\\
332.01	0\\
333.01	0\\
334.01	0\\
335.01	0\\
336.01	0\\
337.01	0\\
338.01	0\\
339.01	0\\
340.01	0\\
341.01	0\\
342.01	0\\
343.01	0\\
344.01	0\\
345.01	0\\
346.01	0\\
347.01	0\\
348.01	0\\
349.01	0\\
350.01	0\\
351.01	0\\
352.01	0\\
353.01	0\\
354.01	0\\
355.01	0\\
356.01	0\\
357.01	0\\
358.01	0\\
359.01	0\\
360.01	0\\
361.01	0\\
362.01	0\\
363.01	0\\
364.01	0\\
365.01	0\\
366.01	0\\
367.01	0\\
368.01	0\\
369.01	0\\
370.01	0\\
371.01	0\\
372.01	0\\
373.01	0\\
374.01	0\\
375.01	0\\
376.01	0\\
377.01	0\\
378.01	0\\
379.01	0\\
380.01	0\\
381.01	0\\
382.01	0\\
383.01	0\\
384.01	0\\
385.01	0\\
386.01	0\\
387.01	0\\
388.01	0\\
389.01	0\\
390.01	0\\
391.01	0\\
392.01	0\\
393.01	0\\
394.01	0\\
395.01	0\\
396.01	0\\
397.01	0\\
398.01	0\\
399.01	0\\
400.01	0\\
401.01	0\\
402.01	0\\
403.01	0\\
404.01	0\\
405.01	0\\
406.01	0\\
407.01	0\\
408.01	0\\
409.01	0\\
410.01	0\\
411.01	0\\
412.01	0\\
413.01	0\\
414.01	0\\
415.01	0\\
416.01	0\\
417.01	0\\
418.01	0\\
419.01	0\\
420.01	0\\
421.01	0\\
422.01	0\\
423.01	0\\
424.01	0\\
425.01	0\\
426.01	0\\
427.01	0\\
428.01	0\\
429.01	0\\
430.01	0\\
431.01	0\\
432.01	0\\
433.01	0\\
434.01	0\\
435.01	0\\
436.01	0\\
437.01	0\\
438.01	0\\
439.01	0\\
440.01	0\\
441.01	0\\
442.01	0\\
443.01	0\\
444.01	0\\
445.01	0\\
446.01	0\\
447.01	0\\
448.01	0\\
449.01	0\\
450.01	0\\
451.01	0\\
452.01	0\\
453.01	0\\
454.01	0\\
455.01	0\\
456.01	0\\
457.01	0\\
458.01	0\\
459.01	0\\
460.01	0\\
461.01	0\\
462.01	0\\
463.01	0\\
464.01	0\\
465.01	0\\
466.01	0\\
467.01	0\\
468.01	0\\
469.01	0\\
470.01	0\\
471.01	0\\
472.01	0\\
473.01	0\\
474.01	0\\
475.01	0\\
476.01	0\\
477.01	0\\
478.01	0\\
479.01	0\\
480.01	0\\
481.01	0\\
482.01	0\\
483.01	0\\
484.01	0\\
485.01	0\\
486.01	0\\
487.01	0\\
488.01	0\\
489.01	0\\
490.01	0\\
491.01	0\\
492.01	0\\
493.01	0\\
494.01	0\\
495.01	0\\
496.01	0\\
497.01	0\\
498.01	0\\
499.01	0\\
500.01	0\\
501.01	0\\
502.01	0\\
503.01	0\\
504.01	0\\
505.01	0\\
506.01	0\\
507.01	0\\
508.01	0\\
509.01	0\\
510.01	0\\
511.01	0\\
512.01	0\\
513.01	0\\
514.01	0\\
515.01	0\\
516.01	0\\
517.01	0\\
518.01	0\\
519.01	0\\
520.01	0\\
521.01	0\\
522.01	0\\
523.01	0\\
524.01	0\\
525.01	0\\
526.01	0\\
527.01	0\\
528.01	0\\
529.01	0\\
530.01	0\\
531.01	0\\
532.01	0\\
533.01	0\\
534.01	0\\
535.01	0\\
536.01	0\\
537.01	0\\
538.01	0\\
539.01	0\\
540.01	0\\
541.01	0\\
542.01	0\\
543.01	0\\
544.01	0\\
545.01	0\\
546.01	0\\
547.01	0\\
548.01	0\\
549.01	0\\
550.01	0\\
551.01	0\\
552.01	0\\
553.01	0\\
554.01	0\\
555.01	0\\
556.01	0\\
557.01	0\\
558.01	0\\
559.01	0\\
560.01	0\\
561.01	0\\
562.01	0\\
563.01	0\\
564.01	0\\
565.01	0\\
566.01	0\\
567.01	0\\
568.01	0\\
569.01	0\\
570.01	0\\
571.01	0\\
572.01	0\\
573.01	0\\
574.01	0\\
575.01	0\\
576.01	0\\
577.01	0\\
578.01	0\\
579.01	0\\
580.01	0\\
581.01	0\\
582.01	0\\
583.01	0\\
584.01	0\\
585.01	0\\
586.01	0\\
587.01	0\\
588.01	0\\
589.01	0\\
590.01	0\\
591.01	0\\
592.01	0\\
593.01	0\\
594.01	0\\
595.01	0\\
596.01	0\\
597.01	0\\
598.01	0.0014374037248278\\
599.01	0.00385656036521221\\
599.02	0.00389409724555795\\
599.03	0.00393199176826388\\
599.04	0.00397024738727775\\
599.05	0.00400886758988391\\
599.06	0.00404785589702493\\
599.07	0.0040872158636263\\
599.08	0.00412695107892428\\
599.09	0.00416706516679696\\
599.1	0.00420756178609841\\
599.11	0.00424844463099618\\
599.12	0.00428971743131196\\
599.13	0.00433138395286554\\
599.14	0.00437344799782218\\
599.15	0.00441591340504319\\
599.16	0.00445878405044\\
599.17	0.00450206384733162\\
599.18	0.0045457567468055\\
599.19	0.00458986673808198\\
599.2	0.00463439784888212\\
599.21	0.00467935414579919\\
599.22	0.00472473973467372\\
599.23	0.00477055876097212\\
599.24	0.00481681541016898\\
599.25	0.0048635139081331\\
599.26	0.00491065851083179\\
599.27	0.00495825350811072\\
599.28	0.00500630323104035\\
599.29	0.00505481205231203\\
599.3	0.00510378438663774\\
599.31	0.00515322469115379\\
599.32	0.0052031374658283\\
599.33	0.00525352725387272\\
599.34	0.00530439864215714\\
599.35	0.00535575626162978\\
599.36	0.00540760478774036\\
599.37	0.00545994894086756\\
599.38	0.00551279348675071\\
599.39	0.00556614323692543\\
599.4	0.00562000304916365\\
599.41	0.00567437782791768\\
599.42	0.00572927252476872\\
599.43	0.00578469213887949\\
599.44	0.00584064171745141\\
599.45	0.00589712635618597\\
599.46	0.00595415119975069\\
599.47	0.00601172144224942\\
599.48	0.00606984232769716\\
599.49	0.00612851915049951\\
599.5	0.00618775725593663\\
599.51	0.00624756204065183\\
599.52	0.00630793895314495\\
599.53	0.00636889349427038\\
599.54	0.00643043121773979\\
599.55	0.00649255773062987\\
599.56	0.00655527869389471\\
599.57	0.00661859982288322\\
599.58	0.00668252688786147\\
599.59	0.00674706571453995\\
599.6	0.00681222218460597\\
599.61	0.00687800223626108\\
599.62	0.00694441186476363\\
599.63	0.00701145712297651\\
599.64	0.00707914412192007\\
599.65	0.00714747903133038\\
599.66	0.00721646808022274\\
599.67	0.00728611755746052\\
599.68	0.00735643381232947\\
599.69	0.00742742325511746\\
599.7	0.00749909235769963\\
599.71	0.00757144765412921\\
599.72	0.00764449574123379\\
599.73	0.00771824327921728\\
599.74	0.00779269699226756\\
599.75	0.00786786366916977\\
599.76	0.00794375016392541\\
599.77	0.0080203633963772\\
599.78	0.00809771035283981\\
599.79	0.00817579808673647\\
599.8	0.00825463371924141\\
599.81	0.00833422443992839\\
599.82	0.00841457750742512\\
599.83	0.00849570025007373\\
599.84	0.00857760006659731\\
599.85	0.00866028442677256\\
599.86	0.0087437608721085\\
599.87	0.00882803701653141\\
599.88	0.00891312054707585\\
599.89	0.00899901922458194\\
599.9	0.00908574088439886\\
599.91	0.00917329343709451\\
599.92	0.00926168486917145\\
599.93	0.00935092324378912\\
599.94	0.00944101670149225\\
599.95	0.00953197346094551\\
599.96	0.00962380181967443\\
599.97	0.00971651015481255\\
599.98	0.0098101069238547\\
599.99	0.00990460066541651\\
600	0.01\\
};
\addplot [color=mycolor17,solid,forget plot]
  table[row sep=crcr]{%
0.01	0\\
1.01	0\\
2.01	0\\
3.01	0\\
4.01	0\\
5.01	0\\
6.01	0\\
7.01	0\\
8.01	0\\
9.01	0\\
10.01	0\\
11.01	0\\
12.01	0\\
13.01	0\\
14.01	0\\
15.01	0\\
16.01	0\\
17.01	0\\
18.01	0\\
19.01	0\\
20.01	0\\
21.01	0\\
22.01	0\\
23.01	0\\
24.01	0\\
25.01	0\\
26.01	0\\
27.01	0\\
28.01	0\\
29.01	0\\
30.01	0\\
31.01	0\\
32.01	0\\
33.01	0\\
34.01	0\\
35.01	0\\
36.01	0\\
37.01	0\\
38.01	0\\
39.01	0\\
40.01	0\\
41.01	0\\
42.01	0\\
43.01	0\\
44.01	0\\
45.01	0\\
46.01	0\\
47.01	0\\
48.01	0\\
49.01	0\\
50.01	0\\
51.01	0\\
52.01	0\\
53.01	0\\
54.01	0\\
55.01	0\\
56.01	0\\
57.01	0\\
58.01	0\\
59.01	0\\
60.01	0\\
61.01	0\\
62.01	0\\
63.01	0\\
64.01	0\\
65.01	0\\
66.01	0\\
67.01	0\\
68.01	0\\
69.01	0\\
70.01	0\\
71.01	0\\
72.01	0\\
73.01	0\\
74.01	0\\
75.01	0\\
76.01	0\\
77.01	0\\
78.01	0\\
79.01	0\\
80.01	0\\
81.01	0\\
82.01	0\\
83.01	0\\
84.01	0\\
85.01	0\\
86.01	0\\
87.01	0\\
88.01	0\\
89.01	0\\
90.01	0\\
91.01	0\\
92.01	0\\
93.01	0\\
94.01	0\\
95.01	0\\
96.01	0\\
97.01	0\\
98.01	0\\
99.01	0\\
100.01	0\\
101.01	0\\
102.01	0\\
103.01	0\\
104.01	0\\
105.01	0\\
106.01	0\\
107.01	0\\
108.01	0\\
109.01	0\\
110.01	0\\
111.01	0\\
112.01	0\\
113.01	0\\
114.01	0\\
115.01	0\\
116.01	0\\
117.01	0\\
118.01	0\\
119.01	0\\
120.01	0\\
121.01	0\\
122.01	0\\
123.01	0\\
124.01	0\\
125.01	0\\
126.01	0\\
127.01	0\\
128.01	0\\
129.01	0\\
130.01	0\\
131.01	0\\
132.01	0\\
133.01	0\\
134.01	0\\
135.01	0\\
136.01	0\\
137.01	0\\
138.01	0\\
139.01	0\\
140.01	0\\
141.01	0\\
142.01	0\\
143.01	0\\
144.01	0\\
145.01	0\\
146.01	0\\
147.01	0\\
148.01	0\\
149.01	0\\
150.01	0\\
151.01	0\\
152.01	0\\
153.01	0\\
154.01	0\\
155.01	0\\
156.01	0\\
157.01	0\\
158.01	0\\
159.01	0\\
160.01	0\\
161.01	0\\
162.01	0\\
163.01	0\\
164.01	0\\
165.01	0\\
166.01	0\\
167.01	0\\
168.01	0\\
169.01	0\\
170.01	0\\
171.01	0\\
172.01	0\\
173.01	0\\
174.01	0\\
175.01	0\\
176.01	0\\
177.01	0\\
178.01	0\\
179.01	0\\
180.01	0\\
181.01	0\\
182.01	0\\
183.01	0\\
184.01	0\\
185.01	0\\
186.01	0\\
187.01	0\\
188.01	0\\
189.01	0\\
190.01	0\\
191.01	0\\
192.01	0\\
193.01	0\\
194.01	0\\
195.01	0\\
196.01	0\\
197.01	0\\
198.01	0\\
199.01	0\\
200.01	0\\
201.01	0\\
202.01	0\\
203.01	0\\
204.01	0\\
205.01	0\\
206.01	0\\
207.01	0\\
208.01	0\\
209.01	0\\
210.01	0\\
211.01	0\\
212.01	0\\
213.01	0\\
214.01	0\\
215.01	0\\
216.01	0\\
217.01	0\\
218.01	0\\
219.01	0\\
220.01	0\\
221.01	0\\
222.01	0\\
223.01	0\\
224.01	0\\
225.01	0\\
226.01	0\\
227.01	0\\
228.01	0\\
229.01	0\\
230.01	0\\
231.01	0\\
232.01	0\\
233.01	0\\
234.01	0\\
235.01	0\\
236.01	0\\
237.01	0\\
238.01	0\\
239.01	0\\
240.01	0\\
241.01	0\\
242.01	0\\
243.01	0\\
244.01	0\\
245.01	0\\
246.01	0\\
247.01	0\\
248.01	0\\
249.01	0\\
250.01	0\\
251.01	0\\
252.01	0\\
253.01	0\\
254.01	0\\
255.01	0\\
256.01	0\\
257.01	0\\
258.01	0\\
259.01	0\\
260.01	0\\
261.01	0\\
262.01	0\\
263.01	0\\
264.01	0\\
265.01	0\\
266.01	0\\
267.01	0\\
268.01	0\\
269.01	0\\
270.01	0\\
271.01	0\\
272.01	0\\
273.01	0\\
274.01	0\\
275.01	0\\
276.01	0\\
277.01	0\\
278.01	0\\
279.01	0\\
280.01	0\\
281.01	0\\
282.01	0\\
283.01	0\\
284.01	0\\
285.01	0\\
286.01	0\\
287.01	0\\
288.01	0\\
289.01	0\\
290.01	0\\
291.01	0\\
292.01	0\\
293.01	0\\
294.01	0\\
295.01	0\\
296.01	0\\
297.01	0\\
298.01	0\\
299.01	0\\
300.01	0\\
301.01	0\\
302.01	0\\
303.01	0\\
304.01	0\\
305.01	0\\
306.01	0\\
307.01	0\\
308.01	0\\
309.01	0\\
310.01	0\\
311.01	0\\
312.01	0\\
313.01	0\\
314.01	0\\
315.01	0\\
316.01	0\\
317.01	0\\
318.01	0\\
319.01	0\\
320.01	0\\
321.01	0\\
322.01	0\\
323.01	0\\
324.01	0\\
325.01	0\\
326.01	0\\
327.01	0\\
328.01	0\\
329.01	0\\
330.01	0\\
331.01	0\\
332.01	0\\
333.01	0\\
334.01	0\\
335.01	0\\
336.01	0\\
337.01	0\\
338.01	0\\
339.01	0\\
340.01	0\\
341.01	0\\
342.01	0\\
343.01	0\\
344.01	0\\
345.01	0\\
346.01	0\\
347.01	0\\
348.01	0\\
349.01	0\\
350.01	0\\
351.01	0\\
352.01	0\\
353.01	0\\
354.01	0\\
355.01	0\\
356.01	0\\
357.01	0\\
358.01	0\\
359.01	0\\
360.01	0\\
361.01	0\\
362.01	0\\
363.01	0\\
364.01	0\\
365.01	0\\
366.01	0\\
367.01	0\\
368.01	0\\
369.01	0\\
370.01	0\\
371.01	0\\
372.01	0\\
373.01	0\\
374.01	0\\
375.01	0\\
376.01	0\\
377.01	0\\
378.01	0\\
379.01	0\\
380.01	0\\
381.01	0\\
382.01	0\\
383.01	0\\
384.01	0\\
385.01	0\\
386.01	0\\
387.01	0\\
388.01	0\\
389.01	0\\
390.01	0\\
391.01	0\\
392.01	0\\
393.01	0\\
394.01	0\\
395.01	0\\
396.01	0\\
397.01	0\\
398.01	0\\
399.01	0\\
400.01	0\\
401.01	0\\
402.01	0\\
403.01	0\\
404.01	0\\
405.01	0\\
406.01	0\\
407.01	0\\
408.01	0\\
409.01	0\\
410.01	0\\
411.01	0\\
412.01	0\\
413.01	0\\
414.01	0\\
415.01	0\\
416.01	0\\
417.01	0\\
418.01	0\\
419.01	0\\
420.01	0\\
421.01	0\\
422.01	0\\
423.01	0\\
424.01	0\\
425.01	0\\
426.01	0\\
427.01	0\\
428.01	0\\
429.01	0\\
430.01	0\\
431.01	0\\
432.01	0\\
433.01	0\\
434.01	0\\
435.01	0\\
436.01	0\\
437.01	0\\
438.01	0\\
439.01	0\\
440.01	0\\
441.01	0\\
442.01	0\\
443.01	0\\
444.01	0\\
445.01	0\\
446.01	0\\
447.01	0\\
448.01	0\\
449.01	0\\
450.01	0\\
451.01	0\\
452.01	0\\
453.01	0\\
454.01	0\\
455.01	0\\
456.01	0\\
457.01	0\\
458.01	0\\
459.01	0\\
460.01	0\\
461.01	0\\
462.01	0\\
463.01	0\\
464.01	0\\
465.01	0\\
466.01	0\\
467.01	0\\
468.01	0\\
469.01	0\\
470.01	0\\
471.01	0\\
472.01	0\\
473.01	0\\
474.01	0\\
475.01	0\\
476.01	0\\
477.01	0\\
478.01	0\\
479.01	0\\
480.01	0\\
481.01	0\\
482.01	0\\
483.01	0\\
484.01	0\\
485.01	0\\
486.01	0\\
487.01	0\\
488.01	0\\
489.01	0\\
490.01	0\\
491.01	0\\
492.01	0\\
493.01	0\\
494.01	0\\
495.01	0\\
496.01	0\\
497.01	0\\
498.01	0\\
499.01	0\\
500.01	0\\
501.01	0\\
502.01	0\\
503.01	0\\
504.01	0\\
505.01	0\\
506.01	0\\
507.01	0\\
508.01	0\\
509.01	0\\
510.01	0\\
511.01	0\\
512.01	0\\
513.01	0\\
514.01	0\\
515.01	0\\
516.01	0\\
517.01	0\\
518.01	0\\
519.01	0\\
520.01	0\\
521.01	0\\
522.01	0\\
523.01	0\\
524.01	0\\
525.01	0\\
526.01	0\\
527.01	0\\
528.01	0\\
529.01	0\\
530.01	0\\
531.01	0\\
532.01	0\\
533.01	0\\
534.01	0\\
535.01	0\\
536.01	0\\
537.01	0\\
538.01	0\\
539.01	0\\
540.01	0\\
541.01	0\\
542.01	0\\
543.01	0\\
544.01	0\\
545.01	0\\
546.01	0\\
547.01	0\\
548.01	0\\
549.01	0\\
550.01	0\\
551.01	0\\
552.01	0\\
553.01	0\\
554.01	0\\
555.01	0\\
556.01	0\\
557.01	0\\
558.01	0\\
559.01	0\\
560.01	0\\
561.01	0\\
562.01	0\\
563.01	0\\
564.01	0\\
565.01	0\\
566.01	0\\
567.01	0\\
568.01	0\\
569.01	0\\
570.01	0\\
571.01	0\\
572.01	0\\
573.01	0\\
574.01	0\\
575.01	0\\
576.01	0\\
577.01	0\\
578.01	0\\
579.01	0\\
580.01	0\\
581.01	0\\
582.01	0\\
583.01	0\\
584.01	0\\
585.01	0\\
586.01	0\\
587.01	0\\
588.01	0\\
589.01	0\\
590.01	0\\
591.01	0\\
592.01	0\\
593.01	0\\
594.01	0\\
595.01	0\\
596.01	0\\
597.01	0\\
598.01	0.00143822749983459\\
599.01	0.00385661144683571\\
599.02	0.00389414709863901\\
599.03	0.00393204040529412\\
599.04	0.00397029482087115\\
599.05	0.00400891383277854\\
599.06	0.00404790096208447\\
599.07	0.00408725976384149\\
599.08	0.00412699382741433\\
599.09	0.0041671067768107\\
599.1	0.00420760227101545\\
599.11	0.00424848400432787\\
599.12	0.00428975570670221\\
599.13	0.00433142114409155\\
599.14	0.00437348411879493\\
599.15	0.00441594846980785\\
599.16	0.00445881807317617\\
599.17	0.00450209684235336\\
599.18	0.00454578872856121\\
599.19	0.00458989772115408\\
599.2	0.00463442784798657\\
599.21	0.00467938317578476\\
599.22	0.00472476781052109\\
599.23	0.00477058589779272\\
599.24	0.00481684162320371\\
599.25	0.00486353921275068\\
599.26	0.0049106829225159\\
599.27	0.00495827704247297\\
599.28	0.00500632590381774\\
599.29	0.00505483387936404\\
599.3	0.00510380538394311\\
599.31	0.00515324487480703\\
599.32	0.00520315685203588\\
599.33	0.00525354585894884\\
599.34	0.00530441648251928\\
599.35	0.00535577335379373\\
599.36	0.005407621148315\\
599.37	0.00545996458654921\\
599.38	0.00551280843431708\\
599.39	0.00556615750322915\\
599.4	0.00562001665112538\\
599.41	0.00567439078251885\\
599.42	0.00572928484904367\\
599.43	0.00578470384990735\\
599.44	0.00584065283234732\\
599.45	0.00589713689209195\\
599.46	0.00595416117382597\\
599.47	0.00601173087166026\\
599.48	0.00606985122960629\\
599.49	0.006128527542055\\
599.5	0.0061877651542603\\
599.51	0.00624756946282722\\
599.52	0.00630794591620473\\
599.53	0.00636890001518324\\
599.54	0.00643043731339698\\
599.55	0.00649256341783105\\
599.56	0.00655528398933347\\
599.57	0.00661860474313204\\
599.58	0.00668253144935614\\
599.59	0.00674706993356369\\
599.6	0.00681222607727294\\
599.61	0.0068780058184995\\
599.62	0.0069444151522985\\
599.63	0.00701146013131191\\
599.64	0.0070791468663211\\
599.65	0.00714748152680477\\
599.66	0.00721647034150212\\
599.67	0.00728611959898148\\
599.68	0.00735643564821442\\
599.69	0.00742742489915527\\
599.7	0.00749909382332627\\
599.71	0.00757144895440832\\
599.72	0.00764449688883733\\
599.73	0.00771824428640635\\
599.74	0.00779269787087345\\
599.75	0.00786786443057545\\
599.76	0.00794375081904746\\
599.77	0.0080203639556484\\
599.78	0.00809771082619259\\
599.79	0.00817579848358721\\
599.8	0.00825463404847609\\
599.81	0.0083342247098895\\
599.82	0.0084145777259002\\
599.83	0.00849570042428592\\
599.84	0.00857760020319794\\
599.85	0.00866028453183631\\
599.86	0.00874376095113144\\
599.87	0.00882803707443219\\
599.88	0.00891312058820066\\
599.89	0.00899901925271357\\
599.9	0.00908574090277037\\
599.91	0.00917329344840818\\
599.92	0.00926168487562357\\
599.93	0.00935092324710127\\
599.94	0.00944101670294984\\
599.95	0.00953197346144448\\
599.96	0.00962380181977693\\
599.97	0.00971651015481255\\
599.98	0.0098101069238547\\
599.99	0.00990460066541651\\
600	0.01\\
};
\addplot [color=mycolor18,solid,forget plot]
  table[row sep=crcr]{%
0.01	0\\
1.01	0\\
2.01	0\\
3.01	0\\
4.01	0\\
5.01	0\\
6.01	0\\
7.01	0\\
8.01	0\\
9.01	0\\
10.01	0\\
11.01	0\\
12.01	0\\
13.01	0\\
14.01	0\\
15.01	0\\
16.01	0\\
17.01	0\\
18.01	0\\
19.01	0\\
20.01	0\\
21.01	0\\
22.01	0\\
23.01	0\\
24.01	0\\
25.01	0\\
26.01	0\\
27.01	0\\
28.01	0\\
29.01	0\\
30.01	0\\
31.01	0\\
32.01	0\\
33.01	0\\
34.01	0\\
35.01	0\\
36.01	0\\
37.01	0\\
38.01	0\\
39.01	0\\
40.01	0\\
41.01	0\\
42.01	0\\
43.01	0\\
44.01	0\\
45.01	0\\
46.01	0\\
47.01	0\\
48.01	0\\
49.01	0\\
50.01	0\\
51.01	0\\
52.01	0\\
53.01	0\\
54.01	0\\
55.01	0\\
56.01	0\\
57.01	0\\
58.01	0\\
59.01	0\\
60.01	0\\
61.01	0\\
62.01	0\\
63.01	0\\
64.01	0\\
65.01	0\\
66.01	0\\
67.01	0\\
68.01	0\\
69.01	0\\
70.01	0\\
71.01	0\\
72.01	0\\
73.01	0\\
74.01	0\\
75.01	0\\
76.01	0\\
77.01	0\\
78.01	0\\
79.01	0\\
80.01	0\\
81.01	0\\
82.01	0\\
83.01	0\\
84.01	0\\
85.01	0\\
86.01	0\\
87.01	0\\
88.01	0\\
89.01	0\\
90.01	0\\
91.01	0\\
92.01	0\\
93.01	0\\
94.01	0\\
95.01	0\\
96.01	0\\
97.01	0\\
98.01	0\\
99.01	0\\
100.01	0\\
101.01	0\\
102.01	0\\
103.01	0\\
104.01	0\\
105.01	0\\
106.01	0\\
107.01	0\\
108.01	0\\
109.01	0\\
110.01	0\\
111.01	0\\
112.01	0\\
113.01	0\\
114.01	0\\
115.01	0\\
116.01	0\\
117.01	0\\
118.01	0\\
119.01	0\\
120.01	0\\
121.01	0\\
122.01	0\\
123.01	0\\
124.01	0\\
125.01	0\\
126.01	0\\
127.01	0\\
128.01	0\\
129.01	0\\
130.01	0\\
131.01	0\\
132.01	0\\
133.01	0\\
134.01	0\\
135.01	0\\
136.01	0\\
137.01	0\\
138.01	0\\
139.01	0\\
140.01	0\\
141.01	0\\
142.01	0\\
143.01	0\\
144.01	0\\
145.01	0\\
146.01	0\\
147.01	0\\
148.01	0\\
149.01	0\\
150.01	0\\
151.01	0\\
152.01	0\\
153.01	0\\
154.01	0\\
155.01	0\\
156.01	0\\
157.01	0\\
158.01	0\\
159.01	0\\
160.01	0\\
161.01	0\\
162.01	0\\
163.01	0\\
164.01	0\\
165.01	0\\
166.01	0\\
167.01	0\\
168.01	0\\
169.01	0\\
170.01	0\\
171.01	0\\
172.01	0\\
173.01	0\\
174.01	0\\
175.01	0\\
176.01	0\\
177.01	0\\
178.01	0\\
179.01	0\\
180.01	0\\
181.01	0\\
182.01	0\\
183.01	0\\
184.01	0\\
185.01	0\\
186.01	0\\
187.01	0\\
188.01	0\\
189.01	0\\
190.01	0\\
191.01	0\\
192.01	0\\
193.01	0\\
194.01	0\\
195.01	0\\
196.01	0\\
197.01	0\\
198.01	0\\
199.01	0\\
200.01	0\\
201.01	0\\
202.01	0\\
203.01	0\\
204.01	0\\
205.01	0\\
206.01	0\\
207.01	0\\
208.01	0\\
209.01	0\\
210.01	0\\
211.01	0\\
212.01	0\\
213.01	0\\
214.01	0\\
215.01	0\\
216.01	0\\
217.01	0\\
218.01	0\\
219.01	0\\
220.01	0\\
221.01	0\\
222.01	0\\
223.01	0\\
224.01	0\\
225.01	0\\
226.01	0\\
227.01	0\\
228.01	0\\
229.01	0\\
230.01	0\\
231.01	0\\
232.01	0\\
233.01	0\\
234.01	0\\
235.01	0\\
236.01	0\\
237.01	0\\
238.01	0\\
239.01	0\\
240.01	0\\
241.01	0\\
242.01	0\\
243.01	0\\
244.01	0\\
245.01	0\\
246.01	0\\
247.01	0\\
248.01	0\\
249.01	0\\
250.01	0\\
251.01	0\\
252.01	0\\
253.01	0\\
254.01	0\\
255.01	0\\
256.01	0\\
257.01	0\\
258.01	0\\
259.01	0\\
260.01	0\\
261.01	0\\
262.01	0\\
263.01	0\\
264.01	0\\
265.01	0\\
266.01	0\\
267.01	0\\
268.01	0\\
269.01	0\\
270.01	0\\
271.01	0\\
272.01	0\\
273.01	0\\
274.01	0\\
275.01	0\\
276.01	0\\
277.01	0\\
278.01	0\\
279.01	0\\
280.01	0\\
281.01	0\\
282.01	0\\
283.01	0\\
284.01	0\\
285.01	0\\
286.01	0\\
287.01	0\\
288.01	0\\
289.01	0\\
290.01	0\\
291.01	0\\
292.01	0\\
293.01	0\\
294.01	0\\
295.01	0\\
296.01	0\\
297.01	0\\
298.01	0\\
299.01	0\\
300.01	0\\
301.01	0\\
302.01	0\\
303.01	0\\
304.01	0\\
305.01	0\\
306.01	0\\
307.01	0\\
308.01	0\\
309.01	0\\
310.01	0\\
311.01	0\\
312.01	0\\
313.01	0\\
314.01	0\\
315.01	0\\
316.01	0\\
317.01	0\\
318.01	0\\
319.01	0\\
320.01	0\\
321.01	0\\
322.01	0\\
323.01	0\\
324.01	0\\
325.01	0\\
326.01	0\\
327.01	0\\
328.01	0\\
329.01	0\\
330.01	0\\
331.01	0\\
332.01	0\\
333.01	0\\
334.01	0\\
335.01	0\\
336.01	0\\
337.01	0\\
338.01	0\\
339.01	0\\
340.01	0\\
341.01	0\\
342.01	0\\
343.01	0\\
344.01	0\\
345.01	0\\
346.01	0\\
347.01	0\\
348.01	0\\
349.01	0\\
350.01	0\\
351.01	0\\
352.01	0\\
353.01	0\\
354.01	0\\
355.01	0\\
356.01	0\\
357.01	0\\
358.01	0\\
359.01	0\\
360.01	0\\
361.01	0\\
362.01	0\\
363.01	0\\
364.01	0\\
365.01	0\\
366.01	0\\
367.01	0\\
368.01	0\\
369.01	0\\
370.01	0\\
371.01	0\\
372.01	0\\
373.01	0\\
374.01	0\\
375.01	0\\
376.01	0\\
377.01	0\\
378.01	0\\
379.01	0\\
380.01	0\\
381.01	0\\
382.01	0\\
383.01	0\\
384.01	0\\
385.01	0\\
386.01	0\\
387.01	0\\
388.01	0\\
389.01	0\\
390.01	0\\
391.01	0\\
392.01	0\\
393.01	0\\
394.01	0\\
395.01	0\\
396.01	0\\
397.01	0\\
398.01	0\\
399.01	0\\
400.01	0\\
401.01	0\\
402.01	0\\
403.01	0\\
404.01	0\\
405.01	0\\
406.01	0\\
407.01	0\\
408.01	0\\
409.01	0\\
410.01	0\\
411.01	0\\
412.01	0\\
413.01	0\\
414.01	0\\
415.01	0\\
416.01	0\\
417.01	0\\
418.01	0\\
419.01	0\\
420.01	0\\
421.01	0\\
422.01	0\\
423.01	0\\
424.01	0\\
425.01	0\\
426.01	0\\
427.01	0\\
428.01	0\\
429.01	0\\
430.01	0\\
431.01	0\\
432.01	0\\
433.01	0\\
434.01	0\\
435.01	0\\
436.01	0\\
437.01	0\\
438.01	0\\
439.01	0\\
440.01	0\\
441.01	0\\
442.01	0\\
443.01	0\\
444.01	0\\
445.01	0\\
446.01	0\\
447.01	0\\
448.01	0\\
449.01	0\\
450.01	0\\
451.01	0\\
452.01	0\\
453.01	0\\
454.01	0\\
455.01	0\\
456.01	0\\
457.01	0\\
458.01	0\\
459.01	0\\
460.01	0\\
461.01	0\\
462.01	0\\
463.01	0\\
464.01	0\\
465.01	0\\
466.01	0\\
467.01	0\\
468.01	0\\
469.01	0\\
470.01	0\\
471.01	0\\
472.01	0\\
473.01	0\\
474.01	0\\
475.01	0\\
476.01	0\\
477.01	0\\
478.01	0\\
479.01	0\\
480.01	0\\
481.01	0\\
482.01	0\\
483.01	0\\
484.01	0\\
485.01	0\\
486.01	0\\
487.01	0\\
488.01	0\\
489.01	0\\
490.01	0\\
491.01	0\\
492.01	0\\
493.01	0\\
494.01	0\\
495.01	0\\
496.01	0\\
497.01	0\\
498.01	0\\
499.01	0\\
500.01	0\\
501.01	0\\
502.01	0\\
503.01	0\\
504.01	0\\
505.01	0\\
506.01	0\\
507.01	0\\
508.01	0\\
509.01	0\\
510.01	0\\
511.01	0\\
512.01	0\\
513.01	0\\
514.01	0\\
515.01	0\\
516.01	0\\
517.01	0\\
518.01	0\\
519.01	0\\
520.01	0\\
521.01	0\\
522.01	0\\
523.01	0\\
524.01	0\\
525.01	0\\
526.01	0\\
527.01	0\\
528.01	0\\
529.01	0\\
530.01	0\\
531.01	0\\
532.01	0\\
533.01	0\\
534.01	0\\
535.01	0\\
536.01	0\\
537.01	0\\
538.01	0\\
539.01	0\\
540.01	0\\
541.01	0\\
542.01	0\\
543.01	0\\
544.01	0\\
545.01	0\\
546.01	0\\
547.01	0\\
548.01	0\\
549.01	0\\
550.01	0\\
551.01	0\\
552.01	0\\
553.01	0\\
554.01	0\\
555.01	0\\
556.01	0\\
557.01	0\\
558.01	0\\
559.01	0\\
560.01	0\\
561.01	0\\
562.01	0\\
563.01	0\\
564.01	0\\
565.01	0\\
566.01	0\\
567.01	0\\
568.01	0\\
569.01	0\\
570.01	0\\
571.01	0\\
572.01	0\\
573.01	0\\
574.01	0\\
575.01	0\\
576.01	0\\
577.01	0\\
578.01	0\\
579.01	0\\
580.01	0\\
581.01	0\\
582.01	0\\
583.01	0\\
584.01	0\\
585.01	0\\
586.01	0\\
587.01	0\\
588.01	0\\
589.01	0\\
590.01	0\\
591.01	0\\
592.01	0\\
593.01	0\\
594.01	0\\
595.01	0\\
596.01	0\\
597.01	0\\
598.01	0.00143832017857744\\
599.01	0.00385661255145558\\
599.02	0.00389414816850622\\
599.03	0.00393204144103766\\
599.04	0.00397029582311886\\
599.05	0.00400891480215704\\
599.06	0.00404790189921909\\
599.07	0.00408726066935627\\
599.08	0.00412699470193188\\
599.09	0.00416710762095217\\
599.1	0.00420760308540043\\
599.11	0.00424848478957431\\
599.12	0.00428975646342631\\
599.13	0.00433142187290768\\
599.14	0.00437348482031551\\
599.15	0.00441594914464327\\
599.16	0.00445881872193463\\
599.17	0.00450209746564076\\
599.18	0.00454578932698104\\
599.19	0.00458989829530726\\
599.2	0.0046344283984713\\
599.21	0.0046793837031964\\
599.22	0.00472476831545197\\
599.23	0.00477058638083201\\
599.24	0.00481684208493721\\
599.25	0.00486353965376069\\
599.26	0.00491068334338074\\
599.27	0.00495827744376709\\
599.28	0.00500632628611153\\
599.29	0.0050548342432236\\
599.3	0.00510380572993011\\
599.31	0.00515324520347841\\
599.32	0.00520315716394368\\
599.33	0.00525354615463998\\
599.34	0.00530441676253529\\
599.35	0.00535577361867056\\
599.36	0.00540762139858276\\
599.37	0.00545996482273196\\
599.38	0.00551280865693252\\
599.39	0.00556615771278843\\
599.4	0.00562001684813284\\
599.41	0.00567439096747174\\
599.42	0.00572928502243194\\
599.43	0.00578470401221333\\
599.44	0.00584065298404554\\
599.45	0.00589713703364883\\
599.46	0.00595416130569955\\
599.47	0.00601173099430001\\
599.48	0.00606985134345282\\
599.49	0.00612852764753978\\
599.5	0.00618776525180547\\
599.51	0.0062475695528453\\
599.52	0.0063079459990984\\
599.53	0.00636890009134512\\
599.54	0.00643043738320938\\
599.55	0.00649256348166578\\
599.56	0.0065552840475516\\
599.57	0.0066186047960837\\
599.58	0.00668253149738038\\
599.59	0.00674706997698825\\
599.6	0.00681222611641409\\
599.61	0.00687800585366193\\
599.62	0.00694441518377516\\
599.63	0.00701146015938389\\
599.64	0.00707914689125756\\
599.65	0.00714748154886281\\
599.66	0.00721647036092677\\
599.67	0.00728611961600567\\
599.68	0.00735643566305893\\
599.69	0.00742742491202878\\
599.7	0.00749909383442539\\
599.71	0.00757144896391765\\
599.72	0.00764449689692958\\
599.73	0.00771824429324247\\
599.74	0.0077926978766028\\
599.75	0.00786786443533595\\
599.76	0.00794375082296586\\
599.77	0.00802036395884057\\
599.78	0.00809771082876376\\
599.79	0.00817579848563237\\
599.8	0.00825463405008032\\
599.81	0.0083342247111284\\
599.82	0.00841457772684034\\
599.83	0.00849570042498529\\
599.84	0.00857760020370652\\
599.85	0.00866028453219658\\
599.86	0.00874376095137895\\
599.87	0.00882803707459619\\
599.88	0.00891312058830471\\
599.89	0.00899901925277615\\
599.9	0.00908574090280557\\
599.91	0.00917329344842634\\
599.92	0.0092616848756319\\
599.93	0.00935092324710449\\
599.94	0.00944101670295078\\
599.95	0.00953197346144465\\
599.96	0.00962380181977693\\
599.97	0.00971651015481255\\
599.98	0.0098101069238547\\
599.99	0.00990460066541651\\
600	0.01\\
};
\addplot [color=red!25!mycolor17,solid,forget plot]
  table[row sep=crcr]{%
0.01	0\\
1.01	0\\
2.01	0\\
3.01	0\\
4.01	0\\
5.01	0\\
6.01	0\\
7.01	0\\
8.01	0\\
9.01	0\\
10.01	0\\
11.01	0\\
12.01	0\\
13.01	0\\
14.01	0\\
15.01	0\\
16.01	0\\
17.01	0\\
18.01	0\\
19.01	0\\
20.01	0\\
21.01	0\\
22.01	0\\
23.01	0\\
24.01	0\\
25.01	0\\
26.01	0\\
27.01	0\\
28.01	0\\
29.01	0\\
30.01	0\\
31.01	0\\
32.01	0\\
33.01	0\\
34.01	0\\
35.01	0\\
36.01	0\\
37.01	0\\
38.01	0\\
39.01	0\\
40.01	0\\
41.01	0\\
42.01	0\\
43.01	0\\
44.01	0\\
45.01	0\\
46.01	0\\
47.01	0\\
48.01	0\\
49.01	0\\
50.01	0\\
51.01	0\\
52.01	0\\
53.01	0\\
54.01	0\\
55.01	0\\
56.01	0\\
57.01	0\\
58.01	0\\
59.01	0\\
60.01	0\\
61.01	0\\
62.01	0\\
63.01	0\\
64.01	0\\
65.01	0\\
66.01	0\\
67.01	0\\
68.01	0\\
69.01	0\\
70.01	0\\
71.01	0\\
72.01	0\\
73.01	0\\
74.01	0\\
75.01	0\\
76.01	0\\
77.01	0\\
78.01	0\\
79.01	0\\
80.01	0\\
81.01	0\\
82.01	0\\
83.01	0\\
84.01	0\\
85.01	0\\
86.01	0\\
87.01	0\\
88.01	0\\
89.01	0\\
90.01	0\\
91.01	0\\
92.01	0\\
93.01	0\\
94.01	0\\
95.01	0\\
96.01	0\\
97.01	0\\
98.01	0\\
99.01	0\\
100.01	0\\
101.01	0\\
102.01	0\\
103.01	0\\
104.01	0\\
105.01	0\\
106.01	0\\
107.01	0\\
108.01	0\\
109.01	0\\
110.01	0\\
111.01	0\\
112.01	0\\
113.01	0\\
114.01	0\\
115.01	0\\
116.01	0\\
117.01	0\\
118.01	0\\
119.01	0\\
120.01	0\\
121.01	0\\
122.01	0\\
123.01	0\\
124.01	0\\
125.01	0\\
126.01	0\\
127.01	0\\
128.01	0\\
129.01	0\\
130.01	0\\
131.01	0\\
132.01	0\\
133.01	0\\
134.01	0\\
135.01	0\\
136.01	0\\
137.01	0\\
138.01	0\\
139.01	0\\
140.01	0\\
141.01	0\\
142.01	0\\
143.01	0\\
144.01	0\\
145.01	0\\
146.01	0\\
147.01	0\\
148.01	0\\
149.01	0\\
150.01	0\\
151.01	0\\
152.01	0\\
153.01	0\\
154.01	0\\
155.01	0\\
156.01	0\\
157.01	0\\
158.01	0\\
159.01	0\\
160.01	0\\
161.01	0\\
162.01	0\\
163.01	0\\
164.01	0\\
165.01	0\\
166.01	0\\
167.01	0\\
168.01	0\\
169.01	0\\
170.01	0\\
171.01	0\\
172.01	0\\
173.01	0\\
174.01	0\\
175.01	0\\
176.01	0\\
177.01	0\\
178.01	0\\
179.01	0\\
180.01	0\\
181.01	0\\
182.01	0\\
183.01	0\\
184.01	0\\
185.01	0\\
186.01	0\\
187.01	0\\
188.01	0\\
189.01	0\\
190.01	0\\
191.01	0\\
192.01	0\\
193.01	0\\
194.01	0\\
195.01	0\\
196.01	0\\
197.01	0\\
198.01	0\\
199.01	0\\
200.01	0\\
201.01	0\\
202.01	0\\
203.01	0\\
204.01	0\\
205.01	0\\
206.01	0\\
207.01	0\\
208.01	0\\
209.01	0\\
210.01	0\\
211.01	0\\
212.01	0\\
213.01	0\\
214.01	0\\
215.01	0\\
216.01	0\\
217.01	0\\
218.01	0\\
219.01	0\\
220.01	0\\
221.01	0\\
222.01	0\\
223.01	0\\
224.01	0\\
225.01	0\\
226.01	0\\
227.01	0\\
228.01	0\\
229.01	0\\
230.01	0\\
231.01	0\\
232.01	0\\
233.01	0\\
234.01	0\\
235.01	0\\
236.01	0\\
237.01	0\\
238.01	0\\
239.01	0\\
240.01	0\\
241.01	0\\
242.01	0\\
243.01	0\\
244.01	0\\
245.01	0\\
246.01	0\\
247.01	0\\
248.01	0\\
249.01	0\\
250.01	0\\
251.01	0\\
252.01	0\\
253.01	0\\
254.01	0\\
255.01	0\\
256.01	0\\
257.01	0\\
258.01	0\\
259.01	0\\
260.01	0\\
261.01	0\\
262.01	0\\
263.01	0\\
264.01	0\\
265.01	0\\
266.01	0\\
267.01	0\\
268.01	0\\
269.01	0\\
270.01	0\\
271.01	0\\
272.01	0\\
273.01	0\\
274.01	0\\
275.01	0\\
276.01	0\\
277.01	0\\
278.01	0\\
279.01	0\\
280.01	0\\
281.01	0\\
282.01	0\\
283.01	0\\
284.01	0\\
285.01	0\\
286.01	0\\
287.01	0\\
288.01	0\\
289.01	0\\
290.01	0\\
291.01	0\\
292.01	0\\
293.01	0\\
294.01	0\\
295.01	0\\
296.01	0\\
297.01	0\\
298.01	0\\
299.01	0\\
300.01	0\\
301.01	0\\
302.01	0\\
303.01	0\\
304.01	0\\
305.01	0\\
306.01	0\\
307.01	0\\
308.01	0\\
309.01	0\\
310.01	0\\
311.01	0\\
312.01	0\\
313.01	0\\
314.01	0\\
315.01	0\\
316.01	0\\
317.01	0\\
318.01	0\\
319.01	0\\
320.01	0\\
321.01	0\\
322.01	0\\
323.01	0\\
324.01	0\\
325.01	0\\
326.01	0\\
327.01	0\\
328.01	0\\
329.01	0\\
330.01	0\\
331.01	0\\
332.01	0\\
333.01	0\\
334.01	0\\
335.01	0\\
336.01	0\\
337.01	0\\
338.01	0\\
339.01	0\\
340.01	0\\
341.01	0\\
342.01	0\\
343.01	0\\
344.01	0\\
345.01	0\\
346.01	0\\
347.01	0\\
348.01	0\\
349.01	0\\
350.01	0\\
351.01	0\\
352.01	0\\
353.01	0\\
354.01	0\\
355.01	0\\
356.01	0\\
357.01	0\\
358.01	0\\
359.01	0\\
360.01	0\\
361.01	0\\
362.01	0\\
363.01	0\\
364.01	0\\
365.01	0\\
366.01	0\\
367.01	0\\
368.01	0\\
369.01	0\\
370.01	0\\
371.01	0\\
372.01	0\\
373.01	0\\
374.01	0\\
375.01	0\\
376.01	0\\
377.01	0\\
378.01	0\\
379.01	0\\
380.01	0\\
381.01	0\\
382.01	0\\
383.01	0\\
384.01	0\\
385.01	0\\
386.01	0\\
387.01	0\\
388.01	0\\
389.01	0\\
390.01	0\\
391.01	0\\
392.01	0\\
393.01	0\\
394.01	0\\
395.01	0\\
396.01	0\\
397.01	0\\
398.01	0\\
399.01	0\\
400.01	0\\
401.01	0\\
402.01	0\\
403.01	0\\
404.01	0\\
405.01	0\\
406.01	0\\
407.01	0\\
408.01	0\\
409.01	0\\
410.01	0\\
411.01	0\\
412.01	0\\
413.01	0\\
414.01	0\\
415.01	0\\
416.01	0\\
417.01	0\\
418.01	0\\
419.01	0\\
420.01	0\\
421.01	0\\
422.01	0\\
423.01	0\\
424.01	0\\
425.01	0\\
426.01	0\\
427.01	0\\
428.01	0\\
429.01	0\\
430.01	0\\
431.01	0\\
432.01	0\\
433.01	0\\
434.01	0\\
435.01	0\\
436.01	0\\
437.01	0\\
438.01	0\\
439.01	0\\
440.01	0\\
441.01	0\\
442.01	0\\
443.01	0\\
444.01	0\\
445.01	0\\
446.01	0\\
447.01	0\\
448.01	0\\
449.01	0\\
450.01	0\\
451.01	0\\
452.01	0\\
453.01	0\\
454.01	0\\
455.01	0\\
456.01	0\\
457.01	0\\
458.01	0\\
459.01	0\\
460.01	0\\
461.01	0\\
462.01	0\\
463.01	0\\
464.01	0\\
465.01	0\\
466.01	0\\
467.01	0\\
468.01	0\\
469.01	0\\
470.01	0\\
471.01	0\\
472.01	0\\
473.01	0\\
474.01	0\\
475.01	0\\
476.01	0\\
477.01	0\\
478.01	0\\
479.01	0\\
480.01	0\\
481.01	0\\
482.01	0\\
483.01	0\\
484.01	0\\
485.01	0\\
486.01	0\\
487.01	0\\
488.01	0\\
489.01	0\\
490.01	0\\
491.01	0\\
492.01	0\\
493.01	0\\
494.01	0\\
495.01	0\\
496.01	0\\
497.01	0\\
498.01	0\\
499.01	0\\
500.01	0\\
501.01	0\\
502.01	0\\
503.01	0\\
504.01	0\\
505.01	0\\
506.01	0\\
507.01	0\\
508.01	0\\
509.01	0\\
510.01	0\\
511.01	0\\
512.01	0\\
513.01	0\\
514.01	0\\
515.01	0\\
516.01	0\\
517.01	0\\
518.01	0\\
519.01	0\\
520.01	0\\
521.01	0\\
522.01	0\\
523.01	0\\
524.01	0\\
525.01	0\\
526.01	0\\
527.01	0\\
528.01	0\\
529.01	0\\
530.01	0\\
531.01	0\\
532.01	0\\
533.01	0\\
534.01	0\\
535.01	0\\
536.01	0\\
537.01	0\\
538.01	0\\
539.01	0\\
540.01	0\\
541.01	0\\
542.01	0\\
543.01	0\\
544.01	0\\
545.01	0\\
546.01	0\\
547.01	0\\
548.01	0\\
549.01	0\\
550.01	0\\
551.01	0\\
552.01	0\\
553.01	0\\
554.01	0\\
555.01	0\\
556.01	0\\
557.01	0\\
558.01	0\\
559.01	0\\
560.01	0\\
561.01	0\\
562.01	0\\
563.01	0\\
564.01	0\\
565.01	0\\
566.01	0\\
567.01	0\\
568.01	0\\
569.01	0\\
570.01	0\\
571.01	0\\
572.01	0\\
573.01	0\\
574.01	0\\
575.01	0\\
576.01	0\\
577.01	0\\
578.01	0\\
579.01	0\\
580.01	0\\
581.01	0\\
582.01	0\\
583.01	0\\
584.01	0\\
585.01	0\\
586.01	0\\
587.01	0\\
588.01	0\\
589.01	0\\
590.01	0\\
591.01	0\\
592.01	0\\
593.01	0\\
594.01	0\\
595.01	0\\
596.01	0\\
597.01	0\\
598.01	0.00143832067364899\\
599.01	0.00385661257178434\\
599.02	0.0038941481880362\\
599.03	0.00393204145978999\\
599.04	0.00397029584111443\\
599.05	0.00400891481941645\\
599.06	0.00404790191576273\\
599.07	0.00408726068520424\\
599.08	0.00412699471710402\\
599.09	0.00416710763546808\\
599.1	0.00420760309927944\\
599.11	0.00424848480283546\\
599.12	0.00428975647608841\\
599.13	0.00433142188498926\\
599.14	0.00437348483183484\\
599.15	0.00441594915561834\\
599.16	0.00445881873238316\\
599.17	0.00450209747558018\\
599.18	0.00454578933642853\\
599.19	0.00458989830427969\\
599.2	0.00463442840698527\\
599.21	0.00467938371126824\\
599.22	0.00472476832309771\\
599.23	0.00477058638806738\\
599.24	0.00481684209177767\\
599.25	0.0048635396602214\\
599.26	0.00491068334947654\\
599.27	0.00495827744951256\\
599.28	0.00500632629152094\\
599.29	0.00505483424831089\\
599.3	0.00510380573470894\\
599.31	0.00515324520796213\\
599.32	0.00520315716814533\\
599.33	0.00525354615857229\\
599.34	0.00530441676621069\\
599.35	0.00535577362210116\\
599.36	0.00540762140178036\\
599.37	0.00545996482570803\\
599.38	0.00551280865969823\\
599.39	0.00556615771535462\\
599.4	0.00562001685051006\\
599.41	0.00567439096967018\\
599.42	0.00572928502446148\\
599.43	0.00578470401408357\\
599.44	0.00584065298576571\\
599.45	0.00589713703522788\\
599.46	0.0059541613071461\\
599.47	0.00601173099562236\\
599.48	0.00606985134465893\\
599.49	0.00612852764863735\\
599.5	0.00618776525280185\\
599.51	0.00624756955374753\\
599.52	0.00630794599991322\\
599.53	0.00636890009207898\\
599.54	0.00643043738386841\\
599.55	0.00649256348225581\\
599.56	0.00655528404807817\\
599.57	0.00661860479655207\\
599.58	0.00668253149779552\\
599.59	0.00674706997735484\\
599.6	0.00681222611673654\\
599.61	0.00687800585394439\\
599.62	0.00694441518402151\\
599.63	0.00701146015959775\\
599.64	0.0070791468914423\\
599.65	0.00714748154902157\\
599.66	0.00721647036106244\\
599.67	0.00728611961612093\\
599.68	0.00735643566315622\\
599.69	0.00742742491211035\\
599.7	0.00749909383449329\\
599.71	0.00757144896397373\\
599.72	0.0076444968969755\\
599.73	0.00771824429327973\\
599.74	0.00779269787663272\\
599.75	0.00786786443535973\\
599.76	0.00794375082298454\\
599.77	0.00802036395885504\\
599.78	0.00809771082877482\\
599.79	0.00817579848564069\\
599.8	0.00825463405008646\\
599.81	0.00833422471113284\\
599.82	0.00841457772684348\\
599.83	0.00849570042498746\\
599.84	0.00857760020370797\\
599.85	0.00866028453219752\\
599.86	0.00874376095137954\\
599.87	0.00882803707459654\\
599.88	0.0089131205883049\\
599.89	0.00899901925277625\\
599.9	0.00908574090280562\\
599.91	0.00917329344842636\\
599.92	0.0092616848756319\\
599.93	0.00935092324710449\\
599.94	0.00944101670295079\\
599.95	0.00953197346144465\\
599.96	0.00962380181977694\\
599.97	0.00971651015481255\\
599.98	0.0098101069238547\\
599.99	0.00990460066541651\\
600	0.01\\
};
\addplot [color=mycolor19,solid,forget plot]
  table[row sep=crcr]{%
0.01	0\\
1.01	0\\
2.01	0\\
3.01	0\\
4.01	0\\
5.01	0\\
6.01	0\\
7.01	0\\
8.01	0\\
9.01	0\\
10.01	0\\
11.01	0\\
12.01	0\\
13.01	0\\
14.01	0\\
15.01	0\\
16.01	0\\
17.01	0\\
18.01	0\\
19.01	0\\
20.01	0\\
21.01	0\\
22.01	0\\
23.01	0\\
24.01	0\\
25.01	0\\
26.01	0\\
27.01	0\\
28.01	0\\
29.01	0\\
30.01	0\\
31.01	0\\
32.01	0\\
33.01	0\\
34.01	0\\
35.01	0\\
36.01	0\\
37.01	0\\
38.01	0\\
39.01	0\\
40.01	0\\
41.01	0\\
42.01	0\\
43.01	0\\
44.01	0\\
45.01	0\\
46.01	0\\
47.01	0\\
48.01	0\\
49.01	0\\
50.01	0\\
51.01	0\\
52.01	0\\
53.01	0\\
54.01	0\\
55.01	0\\
56.01	0\\
57.01	0\\
58.01	0\\
59.01	0\\
60.01	0\\
61.01	0\\
62.01	0\\
63.01	0\\
64.01	0\\
65.01	0\\
66.01	0\\
67.01	0\\
68.01	0\\
69.01	0\\
70.01	0\\
71.01	0\\
72.01	0\\
73.01	0\\
74.01	0\\
75.01	0\\
76.01	0\\
77.01	0\\
78.01	0\\
79.01	0\\
80.01	0\\
81.01	0\\
82.01	0\\
83.01	0\\
84.01	0\\
85.01	0\\
86.01	0\\
87.01	0\\
88.01	0\\
89.01	0\\
90.01	0\\
91.01	0\\
92.01	0\\
93.01	0\\
94.01	0\\
95.01	0\\
96.01	0\\
97.01	0\\
98.01	0\\
99.01	0\\
100.01	0\\
101.01	0\\
102.01	0\\
103.01	0\\
104.01	0\\
105.01	0\\
106.01	0\\
107.01	0\\
108.01	0\\
109.01	0\\
110.01	0\\
111.01	0\\
112.01	0\\
113.01	0\\
114.01	0\\
115.01	0\\
116.01	0\\
117.01	0\\
118.01	0\\
119.01	0\\
120.01	0\\
121.01	0\\
122.01	0\\
123.01	0\\
124.01	0\\
125.01	0\\
126.01	0\\
127.01	0\\
128.01	0\\
129.01	0\\
130.01	0\\
131.01	0\\
132.01	0\\
133.01	0\\
134.01	0\\
135.01	0\\
136.01	0\\
137.01	0\\
138.01	0\\
139.01	0\\
140.01	0\\
141.01	0\\
142.01	0\\
143.01	0\\
144.01	0\\
145.01	0\\
146.01	0\\
147.01	0\\
148.01	0\\
149.01	0\\
150.01	0\\
151.01	0\\
152.01	0\\
153.01	0\\
154.01	0\\
155.01	0\\
156.01	0\\
157.01	0\\
158.01	0\\
159.01	0\\
160.01	0\\
161.01	0\\
162.01	0\\
163.01	0\\
164.01	0\\
165.01	0\\
166.01	0\\
167.01	0\\
168.01	0\\
169.01	0\\
170.01	0\\
171.01	0\\
172.01	0\\
173.01	0\\
174.01	0\\
175.01	0\\
176.01	0\\
177.01	0\\
178.01	0\\
179.01	0\\
180.01	0\\
181.01	0\\
182.01	0\\
183.01	0\\
184.01	0\\
185.01	0\\
186.01	0\\
187.01	0\\
188.01	0\\
189.01	0\\
190.01	0\\
191.01	0\\
192.01	0\\
193.01	0\\
194.01	0\\
195.01	0\\
196.01	0\\
197.01	0\\
198.01	0\\
199.01	0\\
200.01	0\\
201.01	0\\
202.01	0\\
203.01	0\\
204.01	0\\
205.01	0\\
206.01	0\\
207.01	0\\
208.01	0\\
209.01	0\\
210.01	0\\
211.01	0\\
212.01	0\\
213.01	0\\
214.01	0\\
215.01	0\\
216.01	0\\
217.01	0\\
218.01	0\\
219.01	0\\
220.01	0\\
221.01	0\\
222.01	0\\
223.01	0\\
224.01	0\\
225.01	0\\
226.01	0\\
227.01	0\\
228.01	0\\
229.01	0\\
230.01	0\\
231.01	0\\
232.01	0\\
233.01	0\\
234.01	0\\
235.01	0\\
236.01	0\\
237.01	0\\
238.01	0\\
239.01	0\\
240.01	0\\
241.01	0\\
242.01	0\\
243.01	0\\
244.01	0\\
245.01	0\\
246.01	0\\
247.01	0\\
248.01	0\\
249.01	0\\
250.01	0\\
251.01	0\\
252.01	0\\
253.01	0\\
254.01	0\\
255.01	0\\
256.01	0\\
257.01	0\\
258.01	0\\
259.01	0\\
260.01	0\\
261.01	0\\
262.01	0\\
263.01	0\\
264.01	0\\
265.01	0\\
266.01	0\\
267.01	0\\
268.01	0\\
269.01	0\\
270.01	0\\
271.01	0\\
272.01	0\\
273.01	0\\
274.01	0\\
275.01	0\\
276.01	0\\
277.01	0\\
278.01	0\\
279.01	0\\
280.01	0\\
281.01	0\\
282.01	0\\
283.01	0\\
284.01	0\\
285.01	0\\
286.01	0\\
287.01	0\\
288.01	0\\
289.01	0\\
290.01	0\\
291.01	0\\
292.01	0\\
293.01	0\\
294.01	0\\
295.01	0\\
296.01	0\\
297.01	0\\
298.01	0\\
299.01	0\\
300.01	0\\
301.01	0\\
302.01	0\\
303.01	0\\
304.01	0\\
305.01	0\\
306.01	0\\
307.01	0\\
308.01	0\\
309.01	0\\
310.01	0\\
311.01	0\\
312.01	0\\
313.01	0\\
314.01	0\\
315.01	0\\
316.01	0\\
317.01	0\\
318.01	0\\
319.01	0\\
320.01	0\\
321.01	0\\
322.01	0\\
323.01	0\\
324.01	0\\
325.01	0\\
326.01	0\\
327.01	0\\
328.01	0\\
329.01	0\\
330.01	0\\
331.01	0\\
332.01	0\\
333.01	0\\
334.01	0\\
335.01	0\\
336.01	0\\
337.01	0\\
338.01	0\\
339.01	0\\
340.01	0\\
341.01	0\\
342.01	0\\
343.01	0\\
344.01	0\\
345.01	0\\
346.01	0\\
347.01	0\\
348.01	0\\
349.01	0\\
350.01	0\\
351.01	0\\
352.01	0\\
353.01	0\\
354.01	0\\
355.01	0\\
356.01	0\\
357.01	0\\
358.01	0\\
359.01	0\\
360.01	0\\
361.01	0\\
362.01	0\\
363.01	0\\
364.01	0\\
365.01	0\\
366.01	0\\
367.01	0\\
368.01	0\\
369.01	0\\
370.01	0\\
371.01	0\\
372.01	0\\
373.01	0\\
374.01	0\\
375.01	0\\
376.01	0\\
377.01	0\\
378.01	0\\
379.01	0\\
380.01	0\\
381.01	0\\
382.01	0\\
383.01	0\\
384.01	0\\
385.01	0\\
386.01	0\\
387.01	0\\
388.01	0\\
389.01	0\\
390.01	0\\
391.01	0\\
392.01	0\\
393.01	0\\
394.01	0\\
395.01	0\\
396.01	0\\
397.01	0\\
398.01	0\\
399.01	0\\
400.01	0\\
401.01	0\\
402.01	0\\
403.01	0\\
404.01	0\\
405.01	0\\
406.01	0\\
407.01	0\\
408.01	0\\
409.01	0\\
410.01	0\\
411.01	0\\
412.01	0\\
413.01	0\\
414.01	0\\
415.01	0\\
416.01	0\\
417.01	0\\
418.01	0\\
419.01	0\\
420.01	0\\
421.01	0\\
422.01	0\\
423.01	0\\
424.01	0\\
425.01	0\\
426.01	0\\
427.01	0\\
428.01	0\\
429.01	0\\
430.01	0\\
431.01	0\\
432.01	0\\
433.01	0\\
434.01	0\\
435.01	0\\
436.01	0\\
437.01	0\\
438.01	0\\
439.01	0\\
440.01	0\\
441.01	0\\
442.01	0\\
443.01	0\\
444.01	0\\
445.01	0\\
446.01	0\\
447.01	0\\
448.01	0\\
449.01	0\\
450.01	0\\
451.01	0\\
452.01	0\\
453.01	0\\
454.01	0\\
455.01	0\\
456.01	0\\
457.01	0\\
458.01	0\\
459.01	0\\
460.01	0\\
461.01	0\\
462.01	0\\
463.01	0\\
464.01	0\\
465.01	0\\
466.01	0\\
467.01	0\\
468.01	0\\
469.01	0\\
470.01	0\\
471.01	0\\
472.01	0\\
473.01	0\\
474.01	0\\
475.01	0\\
476.01	0\\
477.01	0\\
478.01	0\\
479.01	0\\
480.01	0\\
481.01	0\\
482.01	0\\
483.01	0\\
484.01	0\\
485.01	0\\
486.01	0\\
487.01	0\\
488.01	0\\
489.01	0\\
490.01	0\\
491.01	0\\
492.01	0\\
493.01	0\\
494.01	0\\
495.01	0\\
496.01	0\\
497.01	0\\
498.01	0\\
499.01	0\\
500.01	0\\
501.01	0\\
502.01	0\\
503.01	0\\
504.01	0\\
505.01	0\\
506.01	0\\
507.01	0\\
508.01	0\\
509.01	0\\
510.01	0\\
511.01	0\\
512.01	0\\
513.01	0\\
514.01	0\\
515.01	0\\
516.01	0\\
517.01	0\\
518.01	0\\
519.01	0\\
520.01	0\\
521.01	0\\
522.01	0\\
523.01	0\\
524.01	0\\
525.01	0\\
526.01	0\\
527.01	0\\
528.01	0\\
529.01	0\\
530.01	0\\
531.01	0\\
532.01	0\\
533.01	0\\
534.01	0\\
535.01	0\\
536.01	0\\
537.01	0\\
538.01	0\\
539.01	0\\
540.01	0\\
541.01	0\\
542.01	0\\
543.01	0\\
544.01	0\\
545.01	0\\
546.01	0\\
547.01	0\\
548.01	0\\
549.01	0\\
550.01	0\\
551.01	0\\
552.01	0\\
553.01	0\\
554.01	0\\
555.01	0\\
556.01	0\\
557.01	0\\
558.01	0\\
559.01	0\\
560.01	0\\
561.01	0\\
562.01	0\\
563.01	0\\
564.01	0\\
565.01	0\\
566.01	0\\
567.01	0\\
568.01	0\\
569.01	0\\
570.01	0\\
571.01	0\\
572.01	0\\
573.01	0\\
574.01	0\\
575.01	0\\
576.01	0\\
577.01	0\\
578.01	0\\
579.01	0\\
580.01	0\\
581.01	0\\
582.01	0\\
583.01	0\\
584.01	0\\
585.01	0\\
586.01	0\\
587.01	0\\
588.01	0\\
589.01	0\\
590.01	0\\
591.01	0\\
592.01	0\\
593.01	0\\
594.01	0\\
595.01	0\\
596.01	0\\
597.01	0\\
598.01	0.00143832068163977\\
599.01	0.00385661257211143\\
599.02	0.00389414818834777\\
599.03	0.00393204146008657\\
599.04	0.00397029584139656\\
599.05	0.00400891481968466\\
599.06	0.00404790191601751\\
599.07	0.00408726068544609\\
599.08	0.00412699471733344\\
599.09	0.00416710763568554\\
599.1	0.0042076030994854\\
599.11	0.00424848480303038\\
599.12	0.00428975647627273\\
599.13	0.0043314218851634\\
599.14	0.00437348483199923\\
599.15	0.00441594915577338\\
599.16	0.00445881873252926\\
599.17	0.00450209747571773\\
599.18	0.00454578933655789\\
599.19	0.00458989830440124\\
599.2	0.00463442840709936\\
599.21	0.00467938371137522\\
599.22	0.0047247683231979\\
599.23	0.00477058638816113\\
599.24	0.00481684209186529\\
599.25	0.00486353966030319\\
599.26	0.0049106833495528\\
599.27	0.00495827744958357\\
599.28	0.00500632629158696\\
599.29	0.00505483424837222\\
599.3	0.00510380573476581\\
599.31	0.00515324520801481\\
599.32	0.00520315716819406\\
599.33	0.00525354615861728\\
599.34	0.00530441676625217\\
599.35	0.00535577362213935\\
599.36	0.00540762140181545\\
599.37	0.00545996482574022\\
599.38	0.0055128086597277\\
599.39	0.00556615771538157\\
599.4	0.00562001685053463\\
599.41	0.00567439096969254\\
599.42	0.00572928502448181\\
599.43	0.00578470401410199\\
599.44	0.00584065298578238\\
599.45	0.00589713703524292\\
599.46	0.00595416130715964\\
599.47	0.00601173099563451\\
599.48	0.00606985134466982\\
599.49	0.00612852764864707\\
599.5	0.0061877652528105\\
599.51	0.00624756955375522\\
599.52	0.00630794599992002\\
599.53	0.00636890009208498\\
599.54	0.00643043738387368\\
599.55	0.00649256348226043\\
599.56	0.0065552840480822\\
599.57	0.00661860479655558\\
599.58	0.00668253149779855\\
599.59	0.00674706997735745\\
599.6	0.00681222611673879\\
599.61	0.00687800585394631\\
599.62	0.00694441518402313\\
599.63	0.00701146015959912\\
599.64	0.00707914689144344\\
599.65	0.00714748154902252\\
599.66	0.00721647036106323\\
599.67	0.00728611961612157\\
599.68	0.00735643566315675\\
599.69	0.00742742491211077\\
599.7	0.00749909383449363\\
599.71	0.00757144896397399\\
599.72	0.0076444968969757\\
599.73	0.00771824429327989\\
599.74	0.00779269787663284\\
599.75	0.00786786443535982\\
599.76	0.0079437508229846\\
599.77	0.00802036395885509\\
599.78	0.00809771082877485\\
599.79	0.00817579848564071\\
599.8	0.00825463405008648\\
599.81	0.00833422471113285\\
599.82	0.00841457772684349\\
599.83	0.00849570042498746\\
599.84	0.00857760020370797\\
599.85	0.00866028453219752\\
599.86	0.00874376095137953\\
599.87	0.00882803707459654\\
599.88	0.0089131205883049\\
599.89	0.00899901925277625\\
599.9	0.00908574090280562\\
599.91	0.00917329344842636\\
599.92	0.0092616848756319\\
599.93	0.00935092324710449\\
599.94	0.00944101670295078\\
599.95	0.00953197346144465\\
599.96	0.00962380181977693\\
599.97	0.00971651015481255\\
599.98	0.0098101069238547\\
599.99	0.00990460066541651\\
600	0.01\\
};
\addplot [color=red!50!mycolor17,solid,forget plot]
  table[row sep=crcr]{%
0.01	0\\
1.01	0\\
2.01	0\\
3.01	0\\
4.01	0\\
5.01	0\\
6.01	0\\
7.01	0\\
8.01	0\\
9.01	0\\
10.01	0\\
11.01	0\\
12.01	0\\
13.01	0\\
14.01	0\\
15.01	0\\
16.01	0\\
17.01	0\\
18.01	0\\
19.01	0\\
20.01	0\\
21.01	0\\
22.01	0\\
23.01	0\\
24.01	0\\
25.01	0\\
26.01	0\\
27.01	0\\
28.01	0\\
29.01	0\\
30.01	0\\
31.01	0\\
32.01	0\\
33.01	0\\
34.01	0\\
35.01	0\\
36.01	0\\
37.01	0\\
38.01	0\\
39.01	0\\
40.01	0\\
41.01	0\\
42.01	0\\
43.01	0\\
44.01	0\\
45.01	0\\
46.01	0\\
47.01	0\\
48.01	0\\
49.01	0\\
50.01	0\\
51.01	0\\
52.01	0\\
53.01	0\\
54.01	0\\
55.01	0\\
56.01	0\\
57.01	0\\
58.01	0\\
59.01	0\\
60.01	0\\
61.01	0\\
62.01	0\\
63.01	0\\
64.01	0\\
65.01	0\\
66.01	0\\
67.01	0\\
68.01	0\\
69.01	0\\
70.01	0\\
71.01	0\\
72.01	0\\
73.01	0\\
74.01	0\\
75.01	0\\
76.01	0\\
77.01	0\\
78.01	0\\
79.01	0\\
80.01	0\\
81.01	0\\
82.01	0\\
83.01	0\\
84.01	0\\
85.01	0\\
86.01	0\\
87.01	0\\
88.01	0\\
89.01	0\\
90.01	0\\
91.01	0\\
92.01	0\\
93.01	0\\
94.01	0\\
95.01	0\\
96.01	0\\
97.01	0\\
98.01	0\\
99.01	0\\
100.01	0\\
101.01	0\\
102.01	0\\
103.01	0\\
104.01	0\\
105.01	0\\
106.01	0\\
107.01	0\\
108.01	0\\
109.01	0\\
110.01	0\\
111.01	0\\
112.01	0\\
113.01	0\\
114.01	0\\
115.01	0\\
116.01	0\\
117.01	0\\
118.01	0\\
119.01	0\\
120.01	0\\
121.01	0\\
122.01	0\\
123.01	0\\
124.01	0\\
125.01	0\\
126.01	0\\
127.01	0\\
128.01	0\\
129.01	0\\
130.01	0\\
131.01	0\\
132.01	0\\
133.01	0\\
134.01	0\\
135.01	0\\
136.01	0\\
137.01	0\\
138.01	0\\
139.01	0\\
140.01	0\\
141.01	0\\
142.01	0\\
143.01	0\\
144.01	0\\
145.01	0\\
146.01	0\\
147.01	0\\
148.01	0\\
149.01	0\\
150.01	0\\
151.01	0\\
152.01	0\\
153.01	0\\
154.01	0\\
155.01	0\\
156.01	0\\
157.01	0\\
158.01	0\\
159.01	0\\
160.01	0\\
161.01	0\\
162.01	0\\
163.01	0\\
164.01	0\\
165.01	0\\
166.01	0\\
167.01	0\\
168.01	0\\
169.01	0\\
170.01	0\\
171.01	0\\
172.01	0\\
173.01	0\\
174.01	0\\
175.01	0\\
176.01	0\\
177.01	0\\
178.01	0\\
179.01	0\\
180.01	0\\
181.01	0\\
182.01	0\\
183.01	0\\
184.01	0\\
185.01	0\\
186.01	0\\
187.01	0\\
188.01	0\\
189.01	0\\
190.01	0\\
191.01	0\\
192.01	0\\
193.01	0\\
194.01	0\\
195.01	0\\
196.01	0\\
197.01	0\\
198.01	0\\
199.01	0\\
200.01	0\\
201.01	0\\
202.01	0\\
203.01	0\\
204.01	0\\
205.01	0\\
206.01	0\\
207.01	0\\
208.01	0\\
209.01	0\\
210.01	0\\
211.01	0\\
212.01	0\\
213.01	0\\
214.01	0\\
215.01	0\\
216.01	0\\
217.01	0\\
218.01	0\\
219.01	0\\
220.01	0\\
221.01	0\\
222.01	0\\
223.01	0\\
224.01	0\\
225.01	0\\
226.01	0\\
227.01	0\\
228.01	0\\
229.01	0\\
230.01	0\\
231.01	0\\
232.01	0\\
233.01	0\\
234.01	0\\
235.01	0\\
236.01	0\\
237.01	0\\
238.01	0\\
239.01	0\\
240.01	0\\
241.01	0\\
242.01	0\\
243.01	0\\
244.01	0\\
245.01	0\\
246.01	0\\
247.01	0\\
248.01	0\\
249.01	0\\
250.01	0\\
251.01	0\\
252.01	0\\
253.01	0\\
254.01	0\\
255.01	0\\
256.01	0\\
257.01	0\\
258.01	0\\
259.01	0\\
260.01	0\\
261.01	0\\
262.01	0\\
263.01	0\\
264.01	0\\
265.01	0\\
266.01	0\\
267.01	0\\
268.01	0\\
269.01	0\\
270.01	0\\
271.01	0\\
272.01	0\\
273.01	0\\
274.01	0\\
275.01	0\\
276.01	0\\
277.01	0\\
278.01	0\\
279.01	0\\
280.01	0\\
281.01	0\\
282.01	0\\
283.01	0\\
284.01	0\\
285.01	0\\
286.01	0\\
287.01	0\\
288.01	0\\
289.01	0\\
290.01	0\\
291.01	0\\
292.01	0\\
293.01	0\\
294.01	0\\
295.01	0\\
296.01	0\\
297.01	0\\
298.01	0\\
299.01	0\\
300.01	0\\
301.01	0\\
302.01	0\\
303.01	0\\
304.01	0\\
305.01	0\\
306.01	0\\
307.01	0\\
308.01	0\\
309.01	0\\
310.01	0\\
311.01	0\\
312.01	0\\
313.01	0\\
314.01	0\\
315.01	0\\
316.01	0\\
317.01	0\\
318.01	0\\
319.01	0\\
320.01	0\\
321.01	0\\
322.01	0\\
323.01	0\\
324.01	0\\
325.01	0\\
326.01	0\\
327.01	0\\
328.01	0\\
329.01	0\\
330.01	0\\
331.01	0\\
332.01	0\\
333.01	0\\
334.01	0\\
335.01	0\\
336.01	0\\
337.01	0\\
338.01	0\\
339.01	0\\
340.01	0\\
341.01	0\\
342.01	0\\
343.01	0\\
344.01	0\\
345.01	0\\
346.01	0\\
347.01	0\\
348.01	0\\
349.01	0\\
350.01	0\\
351.01	0\\
352.01	0\\
353.01	0\\
354.01	0\\
355.01	0\\
356.01	0\\
357.01	0\\
358.01	0\\
359.01	0\\
360.01	0\\
361.01	0\\
362.01	0\\
363.01	0\\
364.01	0\\
365.01	0\\
366.01	0\\
367.01	0\\
368.01	0\\
369.01	0\\
370.01	0\\
371.01	0\\
372.01	0\\
373.01	0\\
374.01	0\\
375.01	0\\
376.01	0\\
377.01	0\\
378.01	0\\
379.01	0\\
380.01	0\\
381.01	0\\
382.01	0\\
383.01	0\\
384.01	0\\
385.01	0\\
386.01	0\\
387.01	0\\
388.01	0\\
389.01	0\\
390.01	0\\
391.01	0\\
392.01	0\\
393.01	0\\
394.01	0\\
395.01	0\\
396.01	0\\
397.01	0\\
398.01	0\\
399.01	0\\
400.01	0\\
401.01	0\\
402.01	0\\
403.01	0\\
404.01	0\\
405.01	0\\
406.01	0\\
407.01	0\\
408.01	0\\
409.01	0\\
410.01	0\\
411.01	0\\
412.01	0\\
413.01	0\\
414.01	0\\
415.01	0\\
416.01	0\\
417.01	0\\
418.01	0\\
419.01	0\\
420.01	0\\
421.01	0\\
422.01	0\\
423.01	0\\
424.01	0\\
425.01	0\\
426.01	0\\
427.01	0\\
428.01	0\\
429.01	0\\
430.01	0\\
431.01	0\\
432.01	0\\
433.01	0\\
434.01	0\\
435.01	0\\
436.01	0\\
437.01	0\\
438.01	0\\
439.01	0\\
440.01	0\\
441.01	0\\
442.01	0\\
443.01	0\\
444.01	0\\
445.01	0\\
446.01	0\\
447.01	0\\
448.01	0\\
449.01	0\\
450.01	0\\
451.01	0\\
452.01	0\\
453.01	0\\
454.01	0\\
455.01	0\\
456.01	0\\
457.01	0\\
458.01	0\\
459.01	0\\
460.01	0\\
461.01	0\\
462.01	0\\
463.01	0\\
464.01	0\\
465.01	0\\
466.01	0\\
467.01	0\\
468.01	0\\
469.01	0\\
470.01	0\\
471.01	0\\
472.01	0\\
473.01	0\\
474.01	0\\
475.01	0\\
476.01	0\\
477.01	0\\
478.01	0\\
479.01	0\\
480.01	0\\
481.01	0\\
482.01	0\\
483.01	0\\
484.01	0\\
485.01	0\\
486.01	0\\
487.01	0\\
488.01	0\\
489.01	0\\
490.01	0\\
491.01	0\\
492.01	0\\
493.01	0\\
494.01	0\\
495.01	0\\
496.01	0\\
497.01	0\\
498.01	0\\
499.01	0\\
500.01	0\\
501.01	0\\
502.01	0\\
503.01	0\\
504.01	0\\
505.01	0\\
506.01	0\\
507.01	0\\
508.01	0\\
509.01	0\\
510.01	0\\
511.01	0\\
512.01	0\\
513.01	0\\
514.01	0\\
515.01	0\\
516.01	0\\
517.01	0\\
518.01	0\\
519.01	0\\
520.01	0\\
521.01	0\\
522.01	0\\
523.01	0\\
524.01	0\\
525.01	0\\
526.01	0\\
527.01	0\\
528.01	0\\
529.01	0\\
530.01	0\\
531.01	0\\
532.01	0\\
533.01	0\\
534.01	0\\
535.01	0\\
536.01	0\\
537.01	0\\
538.01	0\\
539.01	0\\
540.01	0\\
541.01	0\\
542.01	0\\
543.01	0\\
544.01	0\\
545.01	0\\
546.01	0\\
547.01	0\\
548.01	0\\
549.01	0\\
550.01	0\\
551.01	0\\
552.01	0\\
553.01	0\\
554.01	0\\
555.01	0\\
556.01	0\\
557.01	0\\
558.01	0\\
559.01	0\\
560.01	0\\
561.01	0\\
562.01	0\\
563.01	0\\
564.01	0\\
565.01	0\\
566.01	0\\
567.01	0\\
568.01	0\\
569.01	0\\
570.01	0\\
571.01	0\\
572.01	0\\
573.01	0\\
574.01	0\\
575.01	0\\
576.01	0\\
577.01	0\\
578.01	0\\
579.01	0\\
580.01	0\\
581.01	0\\
582.01	0\\
583.01	0\\
584.01	0\\
585.01	0\\
586.01	0\\
587.01	0\\
588.01	0\\
589.01	0\\
590.01	0\\
591.01	0\\
592.01	0\\
593.01	0\\
594.01	0\\
595.01	0\\
596.01	0\\
597.01	0.000469956016972223\\
598.01	0.00143832068184645\\
599.01	0.00385661257211617\\
599.02	0.00389414818835224\\
599.03	0.00393204146009079\\
599.04	0.00397029584140053\\
599.05	0.0040089148196884\\
599.06	0.00404790191602104\\
599.07	0.00408726068544941\\
599.08	0.00412699471733656\\
599.09	0.00416710763568846\\
599.1	0.00420760309948814\\
599.11	0.00424848480303295\\
599.12	0.00428975647627514\\
599.13	0.00433142188516566\\
599.14	0.00437348483200135\\
599.15	0.00441594915577536\\
599.16	0.00445881873253109\\
599.17	0.00450209747571945\\
599.18	0.00454578933655949\\
599.19	0.00458989830440273\\
599.2	0.00463442840710074\\
599.21	0.00467938371137649\\
599.22	0.00472476832319909\\
599.23	0.00477058638816223\\
599.24	0.0048168420918663\\
599.25	0.00486353966030412\\
599.26	0.00491068334955366\\
599.27	0.00495827744958435\\
599.28	0.00500632629158769\\
599.29	0.00505483424837289\\
599.3	0.00510380573476642\\
599.31	0.00515324520801536\\
599.32	0.00520315716819456\\
599.33	0.00525354615861774\\
599.34	0.00530441676625258\\
599.35	0.00535577362213972\\
599.36	0.00540762140181579\\
599.37	0.00545996482574053\\
599.38	0.00551280865972798\\
599.39	0.00556615771538182\\
599.4	0.00562001685053486\\
599.41	0.00567439096969275\\
599.42	0.00572928502448199\\
599.43	0.00578470401410216\\
599.44	0.00584065298578252\\
599.45	0.00589713703524304\\
599.46	0.00595416130715974\\
599.47	0.0060117309956346\\
599.48	0.0060698513446699\\
599.49	0.00612852764864714\\
599.5	0.00618776525281056\\
599.51	0.00624756955375527\\
599.52	0.00630794599992007\\
599.53	0.00636890009208501\\
599.54	0.00643043738387371\\
599.55	0.00649256348226045\\
599.56	0.00655528404808222\\
599.57	0.00661860479655559\\
599.58	0.00668253149779856\\
599.59	0.00674706997735745\\
599.6	0.00681222611673878\\
599.61	0.0068780058539463\\
599.62	0.00694441518402313\\
599.63	0.00701146015959912\\
599.64	0.00707914689144345\\
599.65	0.00714748154902253\\
599.66	0.00721647036106323\\
599.67	0.00728611961612158\\
599.68	0.00735643566315675\\
599.69	0.00742742491211078\\
599.7	0.00749909383449363\\
599.71	0.007571448963974\\
599.72	0.00764449689697571\\
599.73	0.00771824429327989\\
599.74	0.00779269787663284\\
599.75	0.00786786443535982\\
599.76	0.0079437508229846\\
599.77	0.0080203639588551\\
599.78	0.00809771082877486\\
599.79	0.00817579848564071\\
599.8	0.00825463405008648\\
599.81	0.00833422471113285\\
599.82	0.00841457772684349\\
599.83	0.00849570042498747\\
599.84	0.00857760020370797\\
599.85	0.00866028453219752\\
599.86	0.00874376095137953\\
599.87	0.00882803707459654\\
599.88	0.0089131205883049\\
599.89	0.00899901925277625\\
599.9	0.00908574090280562\\
599.91	0.00917329344842635\\
599.92	0.0092616848756319\\
599.93	0.00935092324710449\\
599.94	0.00944101670295078\\
599.95	0.00953197346144465\\
599.96	0.00962380181977693\\
599.97	0.00971651015481255\\
599.98	0.0098101069238547\\
599.99	0.00990460066541651\\
600	0.01\\
};
\addplot [color=red!40!mycolor19,solid,forget plot]
  table[row sep=crcr]{%
0.01	0\\
1.01	0\\
2.01	0\\
3.01	0\\
4.01	0\\
5.01	0\\
6.01	0\\
7.01	0\\
8.01	0\\
9.01	0\\
10.01	0\\
11.01	0\\
12.01	0\\
13.01	0\\
14.01	0\\
15.01	0\\
16.01	0\\
17.01	0\\
18.01	0\\
19.01	0\\
20.01	0\\
21.01	0\\
22.01	0\\
23.01	0\\
24.01	0\\
25.01	0\\
26.01	0\\
27.01	0\\
28.01	0\\
29.01	0\\
30.01	0\\
31.01	0\\
32.01	0\\
33.01	0\\
34.01	0\\
35.01	0\\
36.01	0\\
37.01	0\\
38.01	0\\
39.01	0\\
40.01	0\\
41.01	0\\
42.01	0\\
43.01	0\\
44.01	0\\
45.01	0\\
46.01	0\\
47.01	0\\
48.01	0\\
49.01	0\\
50.01	0\\
51.01	0\\
52.01	0\\
53.01	0\\
54.01	0\\
55.01	0\\
56.01	0\\
57.01	0\\
58.01	0\\
59.01	0\\
60.01	0\\
61.01	0\\
62.01	0\\
63.01	0\\
64.01	0\\
65.01	0\\
66.01	0\\
67.01	0\\
68.01	0\\
69.01	0\\
70.01	0\\
71.01	0\\
72.01	0\\
73.01	0\\
74.01	0\\
75.01	0\\
76.01	0\\
77.01	0\\
78.01	0\\
79.01	0\\
80.01	0\\
81.01	0\\
82.01	0\\
83.01	0\\
84.01	0\\
85.01	0\\
86.01	0\\
87.01	0\\
88.01	0\\
89.01	0\\
90.01	0\\
91.01	0\\
92.01	0\\
93.01	0\\
94.01	0\\
95.01	0\\
96.01	0\\
97.01	0\\
98.01	0\\
99.01	0\\
100.01	0\\
101.01	0\\
102.01	0\\
103.01	0\\
104.01	0\\
105.01	0\\
106.01	0\\
107.01	0\\
108.01	0\\
109.01	0\\
110.01	0\\
111.01	0\\
112.01	0\\
113.01	0\\
114.01	0\\
115.01	0\\
116.01	0\\
117.01	0\\
118.01	0\\
119.01	0\\
120.01	0\\
121.01	0\\
122.01	0\\
123.01	0\\
124.01	0\\
125.01	0\\
126.01	0\\
127.01	0\\
128.01	0\\
129.01	0\\
130.01	0\\
131.01	0\\
132.01	0\\
133.01	0\\
134.01	0\\
135.01	0\\
136.01	0\\
137.01	0\\
138.01	0\\
139.01	0\\
140.01	0\\
141.01	0\\
142.01	0\\
143.01	0\\
144.01	0\\
145.01	0\\
146.01	0\\
147.01	0\\
148.01	0\\
149.01	0\\
150.01	0\\
151.01	0\\
152.01	0\\
153.01	0\\
154.01	0\\
155.01	0\\
156.01	0\\
157.01	0\\
158.01	0\\
159.01	0\\
160.01	0\\
161.01	0\\
162.01	0\\
163.01	0\\
164.01	0\\
165.01	0\\
166.01	0\\
167.01	0\\
168.01	0\\
169.01	0\\
170.01	0\\
171.01	0\\
172.01	0\\
173.01	0\\
174.01	0\\
175.01	0\\
176.01	0\\
177.01	0\\
178.01	0\\
179.01	0\\
180.01	0\\
181.01	0\\
182.01	0\\
183.01	0\\
184.01	0\\
185.01	0\\
186.01	0\\
187.01	0\\
188.01	0\\
189.01	0\\
190.01	0\\
191.01	0\\
192.01	0\\
193.01	0\\
194.01	0\\
195.01	0\\
196.01	0\\
197.01	0\\
198.01	0\\
199.01	0\\
200.01	0\\
201.01	0\\
202.01	0\\
203.01	0\\
204.01	0\\
205.01	0\\
206.01	0\\
207.01	0\\
208.01	0\\
209.01	0\\
210.01	0\\
211.01	0\\
212.01	0\\
213.01	0\\
214.01	0\\
215.01	0\\
216.01	0\\
217.01	0\\
218.01	0\\
219.01	0\\
220.01	0\\
221.01	0\\
222.01	0\\
223.01	0\\
224.01	0\\
225.01	0\\
226.01	0\\
227.01	0\\
228.01	0\\
229.01	0\\
230.01	0\\
231.01	0\\
232.01	0\\
233.01	0\\
234.01	0\\
235.01	0\\
236.01	0\\
237.01	0\\
238.01	0\\
239.01	0\\
240.01	0\\
241.01	0\\
242.01	0\\
243.01	0\\
244.01	0\\
245.01	0\\
246.01	0\\
247.01	0\\
248.01	0\\
249.01	0\\
250.01	0\\
251.01	0\\
252.01	0\\
253.01	0\\
254.01	0\\
255.01	0\\
256.01	0\\
257.01	0\\
258.01	0\\
259.01	0\\
260.01	0\\
261.01	0\\
262.01	0\\
263.01	0\\
264.01	0\\
265.01	0\\
266.01	0\\
267.01	0\\
268.01	0\\
269.01	0\\
270.01	0\\
271.01	0\\
272.01	0\\
273.01	0\\
274.01	0\\
275.01	0\\
276.01	0\\
277.01	0\\
278.01	0\\
279.01	0\\
280.01	0\\
281.01	0\\
282.01	0\\
283.01	0\\
284.01	0\\
285.01	0\\
286.01	0\\
287.01	0\\
288.01	0\\
289.01	0\\
290.01	0\\
291.01	0\\
292.01	0\\
293.01	0\\
294.01	0\\
295.01	0\\
296.01	0\\
297.01	0\\
298.01	0\\
299.01	0\\
300.01	0\\
301.01	0\\
302.01	0\\
303.01	0\\
304.01	0\\
305.01	0\\
306.01	0\\
307.01	0\\
308.01	0\\
309.01	0\\
310.01	0\\
311.01	0\\
312.01	0\\
313.01	0\\
314.01	0\\
315.01	0\\
316.01	0\\
317.01	0\\
318.01	0\\
319.01	0\\
320.01	0\\
321.01	0\\
322.01	0\\
323.01	0\\
324.01	0\\
325.01	0\\
326.01	0\\
327.01	0\\
328.01	0\\
329.01	0\\
330.01	0\\
331.01	0\\
332.01	0\\
333.01	0\\
334.01	0\\
335.01	0\\
336.01	0\\
337.01	0\\
338.01	0\\
339.01	0\\
340.01	0\\
341.01	0\\
342.01	0\\
343.01	0\\
344.01	0\\
345.01	0\\
346.01	0\\
347.01	0\\
348.01	0\\
349.01	0\\
350.01	0\\
351.01	0\\
352.01	0\\
353.01	0\\
354.01	0\\
355.01	0\\
356.01	0\\
357.01	0\\
358.01	0\\
359.01	0\\
360.01	0\\
361.01	0\\
362.01	0\\
363.01	0\\
364.01	0\\
365.01	0\\
366.01	0\\
367.01	0\\
368.01	0\\
369.01	0\\
370.01	0\\
371.01	0\\
372.01	0\\
373.01	0\\
374.01	0\\
375.01	0\\
376.01	0\\
377.01	0\\
378.01	0\\
379.01	0\\
380.01	0\\
381.01	0\\
382.01	0\\
383.01	0\\
384.01	0\\
385.01	0\\
386.01	0\\
387.01	0\\
388.01	0\\
389.01	0\\
390.01	0\\
391.01	0\\
392.01	0\\
393.01	0\\
394.01	0\\
395.01	0\\
396.01	0\\
397.01	0\\
398.01	0\\
399.01	0\\
400.01	0\\
401.01	0\\
402.01	0\\
403.01	0\\
404.01	0\\
405.01	0\\
406.01	0\\
407.01	0\\
408.01	0\\
409.01	0\\
410.01	0\\
411.01	0\\
412.01	0\\
413.01	0\\
414.01	0\\
415.01	0\\
416.01	0\\
417.01	0\\
418.01	0\\
419.01	0\\
420.01	0\\
421.01	0\\
422.01	0\\
423.01	0\\
424.01	0\\
425.01	0\\
426.01	0\\
427.01	0\\
428.01	0\\
429.01	0\\
430.01	0\\
431.01	0\\
432.01	0\\
433.01	0\\
434.01	0\\
435.01	0\\
436.01	0\\
437.01	0\\
438.01	0\\
439.01	0\\
440.01	0\\
441.01	0\\
442.01	0\\
443.01	0\\
444.01	0\\
445.01	0\\
446.01	0\\
447.01	0\\
448.01	0\\
449.01	0\\
450.01	0\\
451.01	0\\
452.01	0\\
453.01	0\\
454.01	0\\
455.01	0\\
456.01	0\\
457.01	0\\
458.01	0\\
459.01	0\\
460.01	0\\
461.01	0\\
462.01	0\\
463.01	0\\
464.01	0\\
465.01	0\\
466.01	0\\
467.01	0\\
468.01	0\\
469.01	0\\
470.01	0\\
471.01	0\\
472.01	0\\
473.01	0\\
474.01	0\\
475.01	0\\
476.01	0\\
477.01	0\\
478.01	0\\
479.01	0\\
480.01	0\\
481.01	0\\
482.01	0\\
483.01	0\\
484.01	0\\
485.01	0\\
486.01	0\\
487.01	0\\
488.01	0\\
489.01	0\\
490.01	0\\
491.01	0\\
492.01	0\\
493.01	0\\
494.01	0\\
495.01	0\\
496.01	0\\
497.01	0\\
498.01	0\\
499.01	0\\
500.01	0\\
501.01	0\\
502.01	0\\
503.01	0\\
504.01	0\\
505.01	0\\
506.01	0\\
507.01	0\\
508.01	0\\
509.01	0\\
510.01	0\\
511.01	0\\
512.01	0\\
513.01	0\\
514.01	0\\
515.01	0\\
516.01	0\\
517.01	0\\
518.01	0\\
519.01	0\\
520.01	0\\
521.01	0\\
522.01	0\\
523.01	0\\
524.01	0\\
525.01	0\\
526.01	0\\
527.01	0\\
528.01	0\\
529.01	0\\
530.01	0\\
531.01	0\\
532.01	0\\
533.01	0\\
534.01	0\\
535.01	0\\
536.01	0\\
537.01	0\\
538.01	0\\
539.01	0\\
540.01	0\\
541.01	0\\
542.01	0\\
543.01	0\\
544.01	0\\
545.01	0\\
546.01	0\\
547.01	0\\
548.01	0\\
549.01	0\\
550.01	0\\
551.01	0\\
552.01	0\\
553.01	0\\
554.01	0\\
555.01	0\\
556.01	0\\
557.01	0\\
558.01	0\\
559.01	0\\
560.01	0\\
561.01	0\\
562.01	0\\
563.01	0\\
564.01	0\\
565.01	0\\
566.01	0\\
567.01	0\\
568.01	0\\
569.01	0\\
570.01	0\\
571.01	0\\
572.01	0\\
573.01	0\\
574.01	0\\
575.01	0\\
576.01	0\\
577.01	0\\
578.01	0\\
579.01	0\\
580.01	0\\
581.01	0\\
582.01	0\\
583.01	0\\
584.01	0\\
585.01	0\\
586.01	0\\
587.01	0\\
588.01	0\\
589.01	0\\
590.01	0\\
591.01	0\\
592.01	0\\
593.01	0\\
594.01	0\\
595.01	0\\
596.01	0\\
597.01	0.000480262484284122\\
598.01	0.00143832068185175\\
599.01	0.00385661257211624\\
599.02	0.00389414818835231\\
599.03	0.00393204146009084\\
599.04	0.00397029584140058\\
599.05	0.00400891481968844\\
599.06	0.00404790191602107\\
599.07	0.00408726068544944\\
599.08	0.00412699471733659\\
599.09	0.00416710763568848\\
599.1	0.00420760309948815\\
599.11	0.00424848480303296\\
599.12	0.00428975647627514\\
599.13	0.00433142188516566\\
599.14	0.00437348483200135\\
599.15	0.00441594915577536\\
599.16	0.0044588187325311\\
599.17	0.00450209747571944\\
599.18	0.00454578933655948\\
599.19	0.00458989830440271\\
599.2	0.00463442840710074\\
599.21	0.00467938371137649\\
599.22	0.00472476832319909\\
599.23	0.00477058638816223\\
599.24	0.0048168420918663\\
599.25	0.00486353966030412\\
599.26	0.00491068334955367\\
599.27	0.00495827744958437\\
599.28	0.0050063262915877\\
599.29	0.00505483424837289\\
599.3	0.00510380573476642\\
599.31	0.00515324520801536\\
599.32	0.00520315716819456\\
599.33	0.00525354615861774\\
599.34	0.00530441676625259\\
599.35	0.00535577362213972\\
599.36	0.00540762140181578\\
599.37	0.00545996482574052\\
599.38	0.00551280865972797\\
599.39	0.0055661577153818\\
599.4	0.00562001685053484\\
599.41	0.00567439096969274\\
599.42	0.00572928502448198\\
599.43	0.00578470401410214\\
599.44	0.00584065298578251\\
599.45	0.00589713703524303\\
599.46	0.00595416130715973\\
599.47	0.0060117309956346\\
599.48	0.0060698513446699\\
599.49	0.00612852764864714\\
599.5	0.00618776525281056\\
599.51	0.00624756955375527\\
599.52	0.00630794599992007\\
599.53	0.00636890009208502\\
599.54	0.00643043738387371\\
599.55	0.00649256348226045\\
599.56	0.00655528404808223\\
599.57	0.0066186047965556\\
599.58	0.00668253149779857\\
599.59	0.00674706997735746\\
599.6	0.0068122261167388\\
599.61	0.00687800585394632\\
599.62	0.00694441518402314\\
599.63	0.00701146015959913\\
599.64	0.00707914689144346\\
599.65	0.00714748154902253\\
599.66	0.00721647036106323\\
599.67	0.00728611961612158\\
599.68	0.00735643566315675\\
599.69	0.00742742491211078\\
599.7	0.00749909383449363\\
599.71	0.007571448963974\\
599.72	0.00764449689697571\\
599.73	0.00771824429327989\\
599.74	0.00779269787663285\\
599.75	0.00786786443535982\\
599.76	0.00794375082298461\\
599.77	0.0080203639588551\\
599.78	0.00809771082877486\\
599.79	0.00817579848564071\\
599.8	0.00825463405008648\\
599.81	0.00833422471113285\\
599.82	0.00841457772684349\\
599.83	0.00849570042498747\\
599.84	0.00857760020370798\\
599.85	0.00866028453219752\\
599.86	0.00874376095137954\\
599.87	0.00882803707459654\\
599.88	0.0089131205883049\\
599.89	0.00899901925277625\\
599.9	0.00908574090280562\\
599.91	0.00917329344842636\\
599.92	0.00926168487563191\\
599.93	0.00935092324710449\\
599.94	0.00944101670295079\\
599.95	0.00953197346144465\\
599.96	0.00962380181977694\\
599.97	0.00971651015481255\\
599.98	0.0098101069238547\\
599.99	0.00990460066541651\\
600	0.01\\
};
\addplot [color=red!75!mycolor17,solid,forget plot]
  table[row sep=crcr]{%
0.01	0\\
1.01	0\\
2.01	0\\
3.01	0\\
4.01	0\\
5.01	0\\
6.01	0\\
7.01	0\\
8.01	0\\
9.01	0\\
10.01	0\\
11.01	0\\
12.01	0\\
13.01	0\\
14.01	0\\
15.01	0\\
16.01	0\\
17.01	0\\
18.01	0\\
19.01	0\\
20.01	0\\
21.01	0\\
22.01	0\\
23.01	0\\
24.01	0\\
25.01	0\\
26.01	0\\
27.01	0\\
28.01	0\\
29.01	0\\
30.01	0\\
31.01	0\\
32.01	0\\
33.01	0\\
34.01	0\\
35.01	0\\
36.01	0\\
37.01	0\\
38.01	0\\
39.01	0\\
40.01	0\\
41.01	0\\
42.01	0\\
43.01	0\\
44.01	0\\
45.01	0\\
46.01	0\\
47.01	0\\
48.01	0\\
49.01	0\\
50.01	0\\
51.01	0\\
52.01	0\\
53.01	0\\
54.01	0\\
55.01	0\\
56.01	0\\
57.01	0\\
58.01	0\\
59.01	0\\
60.01	0\\
61.01	0\\
62.01	0\\
63.01	0\\
64.01	0\\
65.01	0\\
66.01	0\\
67.01	0\\
68.01	0\\
69.01	0\\
70.01	0\\
71.01	0\\
72.01	0\\
73.01	0\\
74.01	0\\
75.01	0\\
76.01	0\\
77.01	0\\
78.01	0\\
79.01	0\\
80.01	0\\
81.01	0\\
82.01	0\\
83.01	0\\
84.01	0\\
85.01	0\\
86.01	0\\
87.01	0\\
88.01	0\\
89.01	0\\
90.01	0\\
91.01	0\\
92.01	0\\
93.01	0\\
94.01	0\\
95.01	0\\
96.01	0\\
97.01	0\\
98.01	0\\
99.01	0\\
100.01	0\\
101.01	0\\
102.01	0\\
103.01	0\\
104.01	0\\
105.01	0\\
106.01	0\\
107.01	0\\
108.01	0\\
109.01	0\\
110.01	0\\
111.01	0\\
112.01	0\\
113.01	0\\
114.01	0\\
115.01	0\\
116.01	0\\
117.01	0\\
118.01	0\\
119.01	0\\
120.01	0\\
121.01	0\\
122.01	0\\
123.01	0\\
124.01	0\\
125.01	0\\
126.01	0\\
127.01	0\\
128.01	0\\
129.01	0\\
130.01	0\\
131.01	0\\
132.01	0\\
133.01	0\\
134.01	0\\
135.01	0\\
136.01	0\\
137.01	0\\
138.01	0\\
139.01	0\\
140.01	0\\
141.01	0\\
142.01	0\\
143.01	0\\
144.01	0\\
145.01	0\\
146.01	0\\
147.01	0\\
148.01	0\\
149.01	0\\
150.01	0\\
151.01	0\\
152.01	0\\
153.01	0\\
154.01	0\\
155.01	0\\
156.01	0\\
157.01	0\\
158.01	0\\
159.01	0\\
160.01	0\\
161.01	0\\
162.01	0\\
163.01	0\\
164.01	0\\
165.01	0\\
166.01	0\\
167.01	0\\
168.01	0\\
169.01	0\\
170.01	0\\
171.01	0\\
172.01	0\\
173.01	0\\
174.01	0\\
175.01	0\\
176.01	0\\
177.01	0\\
178.01	0\\
179.01	0\\
180.01	0\\
181.01	0\\
182.01	0\\
183.01	0\\
184.01	0\\
185.01	0\\
186.01	0\\
187.01	0\\
188.01	0\\
189.01	0\\
190.01	0\\
191.01	0\\
192.01	0\\
193.01	0\\
194.01	0\\
195.01	0\\
196.01	0\\
197.01	0\\
198.01	0\\
199.01	0\\
200.01	0\\
201.01	0\\
202.01	0\\
203.01	0\\
204.01	0\\
205.01	0\\
206.01	0\\
207.01	0\\
208.01	0\\
209.01	0\\
210.01	0\\
211.01	0\\
212.01	0\\
213.01	0\\
214.01	0\\
215.01	0\\
216.01	0\\
217.01	0\\
218.01	0\\
219.01	0\\
220.01	0\\
221.01	0\\
222.01	0\\
223.01	0\\
224.01	0\\
225.01	0\\
226.01	0\\
227.01	0\\
228.01	0\\
229.01	0\\
230.01	0\\
231.01	0\\
232.01	0\\
233.01	0\\
234.01	0\\
235.01	0\\
236.01	0\\
237.01	0\\
238.01	0\\
239.01	0\\
240.01	0\\
241.01	0\\
242.01	0\\
243.01	0\\
244.01	0\\
245.01	0\\
246.01	0\\
247.01	0\\
248.01	0\\
249.01	0\\
250.01	0\\
251.01	0\\
252.01	0\\
253.01	0\\
254.01	0\\
255.01	0\\
256.01	0\\
257.01	0\\
258.01	0\\
259.01	0\\
260.01	0\\
261.01	0\\
262.01	0\\
263.01	0\\
264.01	0\\
265.01	0\\
266.01	0\\
267.01	0\\
268.01	0\\
269.01	0\\
270.01	0\\
271.01	0\\
272.01	0\\
273.01	0\\
274.01	0\\
275.01	0\\
276.01	0\\
277.01	0\\
278.01	0\\
279.01	0\\
280.01	0\\
281.01	0\\
282.01	0\\
283.01	0\\
284.01	0\\
285.01	0\\
286.01	0\\
287.01	0\\
288.01	0\\
289.01	0\\
290.01	0\\
291.01	0\\
292.01	0\\
293.01	0\\
294.01	0\\
295.01	0\\
296.01	0\\
297.01	0\\
298.01	0\\
299.01	0\\
300.01	0\\
301.01	0\\
302.01	0\\
303.01	0\\
304.01	0\\
305.01	0\\
306.01	0\\
307.01	0\\
308.01	0\\
309.01	0\\
310.01	0\\
311.01	0\\
312.01	0\\
313.01	0\\
314.01	0\\
315.01	0\\
316.01	0\\
317.01	0\\
318.01	0\\
319.01	0\\
320.01	0\\
321.01	0\\
322.01	0\\
323.01	0\\
324.01	0\\
325.01	0\\
326.01	0\\
327.01	0\\
328.01	0\\
329.01	0\\
330.01	0\\
331.01	0\\
332.01	0\\
333.01	0\\
334.01	0\\
335.01	0\\
336.01	0\\
337.01	0\\
338.01	0\\
339.01	0\\
340.01	0\\
341.01	0\\
342.01	0\\
343.01	0\\
344.01	0\\
345.01	0\\
346.01	0\\
347.01	0\\
348.01	0\\
349.01	0\\
350.01	0\\
351.01	0\\
352.01	0\\
353.01	0\\
354.01	0\\
355.01	0\\
356.01	0\\
357.01	0\\
358.01	0\\
359.01	0\\
360.01	0\\
361.01	0\\
362.01	0\\
363.01	0\\
364.01	0\\
365.01	0\\
366.01	0\\
367.01	0\\
368.01	0\\
369.01	0\\
370.01	0\\
371.01	0\\
372.01	0\\
373.01	0\\
374.01	0\\
375.01	0\\
376.01	0\\
377.01	0\\
378.01	0\\
379.01	0\\
380.01	0\\
381.01	0\\
382.01	0\\
383.01	0\\
384.01	0\\
385.01	0\\
386.01	0\\
387.01	0\\
388.01	0\\
389.01	0\\
390.01	0\\
391.01	0\\
392.01	0\\
393.01	0\\
394.01	0\\
395.01	0\\
396.01	0\\
397.01	0\\
398.01	0\\
399.01	0\\
400.01	0\\
401.01	0\\
402.01	0\\
403.01	0\\
404.01	0\\
405.01	0\\
406.01	0\\
407.01	0\\
408.01	0\\
409.01	0\\
410.01	0\\
411.01	0\\
412.01	0\\
413.01	0\\
414.01	0\\
415.01	0\\
416.01	0\\
417.01	0\\
418.01	0\\
419.01	0\\
420.01	0\\
421.01	0\\
422.01	0\\
423.01	0\\
424.01	0\\
425.01	0\\
426.01	0\\
427.01	0\\
428.01	0\\
429.01	0\\
430.01	0\\
431.01	0\\
432.01	0\\
433.01	0\\
434.01	0\\
435.01	0\\
436.01	0\\
437.01	0\\
438.01	0\\
439.01	0\\
440.01	0\\
441.01	0\\
442.01	0\\
443.01	0\\
444.01	0\\
445.01	0\\
446.01	0\\
447.01	0\\
448.01	0\\
449.01	0\\
450.01	0\\
451.01	0\\
452.01	0\\
453.01	0\\
454.01	0\\
455.01	0\\
456.01	0\\
457.01	0\\
458.01	0\\
459.01	0\\
460.01	0\\
461.01	0\\
462.01	0\\
463.01	0\\
464.01	0\\
465.01	0\\
466.01	0\\
467.01	0\\
468.01	0\\
469.01	0\\
470.01	0\\
471.01	0\\
472.01	0\\
473.01	0\\
474.01	0\\
475.01	0\\
476.01	0\\
477.01	0\\
478.01	0\\
479.01	0\\
480.01	0\\
481.01	0\\
482.01	0\\
483.01	0\\
484.01	0\\
485.01	0\\
486.01	0\\
487.01	0\\
488.01	0\\
489.01	0\\
490.01	0\\
491.01	0\\
492.01	0\\
493.01	0\\
494.01	0\\
495.01	0\\
496.01	0\\
497.01	0\\
498.01	0\\
499.01	0\\
500.01	0\\
501.01	0\\
502.01	0\\
503.01	0\\
504.01	0\\
505.01	0\\
506.01	0\\
507.01	0\\
508.01	0\\
509.01	0\\
510.01	0\\
511.01	0\\
512.01	0\\
513.01	0\\
514.01	0\\
515.01	0\\
516.01	0\\
517.01	0\\
518.01	0\\
519.01	0\\
520.01	0\\
521.01	0\\
522.01	0\\
523.01	0\\
524.01	0\\
525.01	0\\
526.01	0\\
527.01	0\\
528.01	0\\
529.01	0\\
530.01	0\\
531.01	0\\
532.01	0\\
533.01	0\\
534.01	0\\
535.01	0\\
536.01	0\\
537.01	0\\
538.01	0\\
539.01	0\\
540.01	0\\
541.01	0\\
542.01	0\\
543.01	0\\
544.01	0\\
545.01	0\\
546.01	0\\
547.01	0\\
548.01	0\\
549.01	0\\
550.01	0\\
551.01	0\\
552.01	0\\
553.01	0\\
554.01	0\\
555.01	0\\
556.01	0\\
557.01	0\\
558.01	0\\
559.01	0\\
560.01	0\\
561.01	0\\
562.01	0\\
563.01	0\\
564.01	0\\
565.01	0\\
566.01	0\\
567.01	0\\
568.01	0\\
569.01	0\\
570.01	0\\
571.01	0\\
572.01	0\\
573.01	0\\
574.01	0\\
575.01	0\\
576.01	0\\
577.01	0\\
578.01	0\\
579.01	0\\
580.01	0\\
581.01	0\\
582.01	0\\
583.01	0\\
584.01	0\\
585.01	0\\
586.01	0\\
587.01	0\\
588.01	0\\
589.01	0\\
590.01	0\\
591.01	0\\
592.01	0\\
593.01	0\\
594.01	0\\
595.01	0\\
596.01	0\\
597.01	0.000480601427308949\\
598.01	0.0014383206818518\\
599.01	0.00385661257211624\\
599.02	0.00389414818835231\\
599.03	0.00393204146009085\\
599.04	0.00397029584140059\\
599.05	0.00400891481968844\\
599.06	0.00404790191602107\\
599.07	0.00408726068544946\\
599.08	0.00412699471733659\\
599.09	0.00416710763568849\\
599.1	0.00420760309948817\\
599.11	0.00424848480303298\\
599.12	0.00428975647627516\\
599.13	0.00433142188516568\\
599.14	0.00437348483200135\\
599.15	0.00441594915577537\\
599.16	0.00445881873253111\\
599.17	0.00450209747571945\\
599.18	0.00454578933655949\\
599.19	0.00458989830440273\\
599.2	0.00463442840710075\\
599.21	0.00467938371137651\\
599.22	0.00472476832319909\\
599.23	0.00477058638816223\\
599.24	0.00481684209186631\\
599.25	0.00486353966030413\\
599.26	0.00491068334955367\\
599.27	0.00495827744958437\\
599.28	0.0050063262915877\\
599.29	0.00505483424837289\\
599.3	0.00510380573476643\\
599.31	0.00515324520801537\\
599.32	0.00520315716819457\\
599.33	0.00525354615861775\\
599.34	0.0053044167662526\\
599.35	0.00535577362213973\\
599.36	0.0054076214018158\\
599.37	0.00545996482574054\\
599.38	0.00551280865972798\\
599.39	0.00556615771538182\\
599.4	0.00562001685053486\\
599.41	0.00567439096969275\\
599.42	0.00572928502448199\\
599.43	0.00578470401410217\\
599.44	0.00584065298578253\\
599.45	0.00589713703524306\\
599.46	0.00595416130715976\\
599.47	0.00601173099563462\\
599.48	0.00606985134466991\\
599.49	0.00612852764864715\\
599.5	0.00618776525281057\\
599.51	0.00624756955375528\\
599.52	0.00630794599992008\\
599.53	0.00636890009208502\\
599.54	0.00643043738387372\\
599.55	0.00649256348226046\\
599.56	0.00655528404808223\\
599.57	0.0066186047965556\\
599.58	0.00668253149779857\\
599.59	0.00674706997735747\\
599.6	0.0068122261167388\\
599.61	0.00687800585394632\\
599.62	0.00694441518402314\\
599.63	0.00701146015959912\\
599.64	0.00707914689144345\\
599.65	0.00714748154902253\\
599.66	0.00721647036106324\\
599.67	0.00728611961612158\\
599.68	0.00735643566315675\\
599.69	0.00742742491211078\\
599.7	0.00749909383449363\\
599.71	0.00757144896397399\\
599.72	0.0076444968969757\\
599.73	0.00771824429327989\\
599.74	0.00779269787663285\\
599.75	0.00786786443535982\\
599.76	0.00794375082298461\\
599.77	0.0080203639588551\\
599.78	0.00809771082877485\\
599.79	0.00817579848564071\\
599.8	0.00825463405008648\\
599.81	0.00833422471113285\\
599.82	0.00841457772684349\\
599.83	0.00849570042498746\\
599.84	0.00857760020370797\\
599.85	0.00866028453219752\\
599.86	0.00874376095137953\\
599.87	0.00882803707459653\\
599.88	0.0089131205883049\\
599.89	0.00899901925277625\\
599.9	0.00908574090280562\\
599.91	0.00917329344842636\\
599.92	0.0092616848756319\\
599.93	0.00935092324710449\\
599.94	0.00944101670295079\\
599.95	0.00953197346144465\\
599.96	0.00962380181977694\\
599.97	0.00971651015481255\\
599.98	0.0098101069238547\\
599.99	0.00990460066541651\\
600	0.01\\
};
\addplot [color=red!80!mycolor19,solid,forget plot]
  table[row sep=crcr]{%
0.01	0\\
1.01	0\\
2.01	0\\
3.01	0\\
4.01	0\\
5.01	0\\
6.01	0\\
7.01	0\\
8.01	0\\
9.01	0\\
10.01	0\\
11.01	0\\
12.01	0\\
13.01	0\\
14.01	0\\
15.01	0\\
16.01	0\\
17.01	0\\
18.01	0\\
19.01	0\\
20.01	0\\
21.01	0\\
22.01	0\\
23.01	0\\
24.01	0\\
25.01	0\\
26.01	0\\
27.01	0\\
28.01	0\\
29.01	0\\
30.01	0\\
31.01	0\\
32.01	0\\
33.01	0\\
34.01	0\\
35.01	0\\
36.01	0\\
37.01	0\\
38.01	0\\
39.01	0\\
40.01	0\\
41.01	0\\
42.01	0\\
43.01	0\\
44.01	0\\
45.01	0\\
46.01	0\\
47.01	0\\
48.01	0\\
49.01	0\\
50.01	0\\
51.01	0\\
52.01	0\\
53.01	0\\
54.01	0\\
55.01	0\\
56.01	0\\
57.01	0\\
58.01	0\\
59.01	0\\
60.01	0\\
61.01	0\\
62.01	0\\
63.01	0\\
64.01	0\\
65.01	0\\
66.01	0\\
67.01	0\\
68.01	0\\
69.01	0\\
70.01	0\\
71.01	0\\
72.01	0\\
73.01	0\\
74.01	0\\
75.01	0\\
76.01	0\\
77.01	0\\
78.01	0\\
79.01	0\\
80.01	0\\
81.01	0\\
82.01	0\\
83.01	0\\
84.01	0\\
85.01	0\\
86.01	0\\
87.01	0\\
88.01	0\\
89.01	0\\
90.01	0\\
91.01	0\\
92.01	0\\
93.01	0\\
94.01	0\\
95.01	0\\
96.01	0\\
97.01	0\\
98.01	0\\
99.01	0\\
100.01	0\\
101.01	0\\
102.01	0\\
103.01	0\\
104.01	0\\
105.01	0\\
106.01	0\\
107.01	0\\
108.01	0\\
109.01	0\\
110.01	0\\
111.01	0\\
112.01	0\\
113.01	0\\
114.01	0\\
115.01	0\\
116.01	0\\
117.01	0\\
118.01	0\\
119.01	0\\
120.01	0\\
121.01	0\\
122.01	0\\
123.01	0\\
124.01	0\\
125.01	0\\
126.01	0\\
127.01	0\\
128.01	0\\
129.01	0\\
130.01	0\\
131.01	0\\
132.01	0\\
133.01	0\\
134.01	0\\
135.01	0\\
136.01	0\\
137.01	0\\
138.01	0\\
139.01	0\\
140.01	0\\
141.01	0\\
142.01	0\\
143.01	0\\
144.01	0\\
145.01	0\\
146.01	0\\
147.01	0\\
148.01	0\\
149.01	0\\
150.01	0\\
151.01	0\\
152.01	0\\
153.01	0\\
154.01	0\\
155.01	0\\
156.01	0\\
157.01	0\\
158.01	0\\
159.01	0\\
160.01	0\\
161.01	0\\
162.01	0\\
163.01	0\\
164.01	0\\
165.01	0\\
166.01	0\\
167.01	0\\
168.01	0\\
169.01	0\\
170.01	0\\
171.01	0\\
172.01	0\\
173.01	0\\
174.01	0\\
175.01	0\\
176.01	0\\
177.01	0\\
178.01	0\\
179.01	0\\
180.01	0\\
181.01	0\\
182.01	0\\
183.01	0\\
184.01	0\\
185.01	0\\
186.01	0\\
187.01	0\\
188.01	0\\
189.01	0\\
190.01	0\\
191.01	0\\
192.01	0\\
193.01	0\\
194.01	0\\
195.01	0\\
196.01	0\\
197.01	0\\
198.01	0\\
199.01	0\\
200.01	0\\
201.01	0\\
202.01	0\\
203.01	0\\
204.01	0\\
205.01	0\\
206.01	0\\
207.01	0\\
208.01	0\\
209.01	0\\
210.01	0\\
211.01	0\\
212.01	0\\
213.01	0\\
214.01	0\\
215.01	0\\
216.01	0\\
217.01	0\\
218.01	0\\
219.01	0\\
220.01	0\\
221.01	0\\
222.01	0\\
223.01	0\\
224.01	0\\
225.01	0\\
226.01	0\\
227.01	0\\
228.01	0\\
229.01	0\\
230.01	0\\
231.01	0\\
232.01	0\\
233.01	0\\
234.01	0\\
235.01	0\\
236.01	0\\
237.01	0\\
238.01	0\\
239.01	0\\
240.01	0\\
241.01	0\\
242.01	0\\
243.01	0\\
244.01	0\\
245.01	0\\
246.01	0\\
247.01	0\\
248.01	0\\
249.01	0\\
250.01	0\\
251.01	0\\
252.01	0\\
253.01	0\\
254.01	0\\
255.01	0\\
256.01	0\\
257.01	0\\
258.01	0\\
259.01	0\\
260.01	0\\
261.01	0\\
262.01	0\\
263.01	0\\
264.01	0\\
265.01	0\\
266.01	0\\
267.01	0\\
268.01	0\\
269.01	0\\
270.01	0\\
271.01	0\\
272.01	0\\
273.01	0\\
274.01	0\\
275.01	0\\
276.01	0\\
277.01	0\\
278.01	0\\
279.01	0\\
280.01	0\\
281.01	0\\
282.01	0\\
283.01	0\\
284.01	0\\
285.01	0\\
286.01	0\\
287.01	0\\
288.01	0\\
289.01	0\\
290.01	0\\
291.01	0\\
292.01	0\\
293.01	0\\
294.01	0\\
295.01	0\\
296.01	0\\
297.01	0\\
298.01	0\\
299.01	0\\
300.01	0\\
301.01	0\\
302.01	0\\
303.01	0\\
304.01	0\\
305.01	0\\
306.01	0\\
307.01	0\\
308.01	0\\
309.01	0\\
310.01	0\\
311.01	0\\
312.01	0\\
313.01	0\\
314.01	0\\
315.01	0\\
316.01	0\\
317.01	0\\
318.01	0\\
319.01	0\\
320.01	0\\
321.01	0\\
322.01	0\\
323.01	0\\
324.01	0\\
325.01	0\\
326.01	0\\
327.01	0\\
328.01	0\\
329.01	0\\
330.01	0\\
331.01	0\\
332.01	0\\
333.01	0\\
334.01	0\\
335.01	0\\
336.01	0\\
337.01	0\\
338.01	0\\
339.01	0\\
340.01	0\\
341.01	0\\
342.01	0\\
343.01	0\\
344.01	0\\
345.01	0\\
346.01	0\\
347.01	0\\
348.01	0\\
349.01	0\\
350.01	0\\
351.01	0\\
352.01	0\\
353.01	0\\
354.01	0\\
355.01	0\\
356.01	0\\
357.01	0\\
358.01	0\\
359.01	0\\
360.01	0\\
361.01	0\\
362.01	0\\
363.01	0\\
364.01	0\\
365.01	0\\
366.01	0\\
367.01	0\\
368.01	0\\
369.01	0\\
370.01	0\\
371.01	0\\
372.01	0\\
373.01	0\\
374.01	0\\
375.01	0\\
376.01	0\\
377.01	0\\
378.01	0\\
379.01	0\\
380.01	0\\
381.01	0\\
382.01	0\\
383.01	0\\
384.01	0\\
385.01	0\\
386.01	0\\
387.01	0\\
388.01	0\\
389.01	0\\
390.01	0\\
391.01	0\\
392.01	0\\
393.01	0\\
394.01	0\\
395.01	0\\
396.01	0\\
397.01	0\\
398.01	0\\
399.01	0\\
400.01	0\\
401.01	0\\
402.01	0\\
403.01	0\\
404.01	0\\
405.01	0\\
406.01	0\\
407.01	0\\
408.01	0\\
409.01	0\\
410.01	0\\
411.01	0\\
412.01	0\\
413.01	0\\
414.01	0\\
415.01	0\\
416.01	0\\
417.01	0\\
418.01	0\\
419.01	0\\
420.01	0\\
421.01	0\\
422.01	0\\
423.01	0\\
424.01	0\\
425.01	0\\
426.01	0\\
427.01	0\\
428.01	0\\
429.01	0\\
430.01	0\\
431.01	0\\
432.01	0\\
433.01	0\\
434.01	0\\
435.01	0\\
436.01	0\\
437.01	0\\
438.01	0\\
439.01	0\\
440.01	0\\
441.01	0\\
442.01	0\\
443.01	0\\
444.01	0\\
445.01	0\\
446.01	0\\
447.01	0\\
448.01	0\\
449.01	0\\
450.01	0\\
451.01	0\\
452.01	0\\
453.01	0\\
454.01	0\\
455.01	0\\
456.01	0\\
457.01	0\\
458.01	0\\
459.01	0\\
460.01	0\\
461.01	0\\
462.01	0\\
463.01	0\\
464.01	0\\
465.01	0\\
466.01	0\\
467.01	0\\
468.01	0\\
469.01	0\\
470.01	0\\
471.01	0\\
472.01	0\\
473.01	0\\
474.01	0\\
475.01	0\\
476.01	0\\
477.01	0\\
478.01	0\\
479.01	0\\
480.01	0\\
481.01	0\\
482.01	0\\
483.01	0\\
484.01	0\\
485.01	0\\
486.01	0\\
487.01	0\\
488.01	0\\
489.01	0\\
490.01	0\\
491.01	0\\
492.01	0\\
493.01	0\\
494.01	0\\
495.01	0\\
496.01	0\\
497.01	0\\
498.01	0\\
499.01	0\\
500.01	0\\
501.01	0\\
502.01	0\\
503.01	0\\
504.01	0\\
505.01	0\\
506.01	0\\
507.01	0\\
508.01	0\\
509.01	0\\
510.01	0\\
511.01	0\\
512.01	0\\
513.01	0\\
514.01	0\\
515.01	0\\
516.01	0\\
517.01	0\\
518.01	0\\
519.01	0\\
520.01	0\\
521.01	0\\
522.01	0\\
523.01	0\\
524.01	0\\
525.01	0\\
526.01	0\\
527.01	0\\
528.01	0\\
529.01	0\\
530.01	0\\
531.01	0\\
532.01	0\\
533.01	0\\
534.01	0\\
535.01	0\\
536.01	0\\
537.01	0\\
538.01	0\\
539.01	0\\
540.01	0\\
541.01	0\\
542.01	0\\
543.01	0\\
544.01	0\\
545.01	0\\
546.01	0\\
547.01	0\\
548.01	0\\
549.01	0\\
550.01	0\\
551.01	0\\
552.01	0\\
553.01	0\\
554.01	0\\
555.01	0\\
556.01	0\\
557.01	0\\
558.01	0\\
559.01	0\\
560.01	0\\
561.01	0\\
562.01	0\\
563.01	0\\
564.01	0\\
565.01	0\\
566.01	0\\
567.01	0\\
568.01	0\\
569.01	0\\
570.01	0\\
571.01	0\\
572.01	0\\
573.01	0\\
574.01	0\\
575.01	0\\
576.01	0\\
577.01	0\\
578.01	0\\
579.01	0\\
580.01	0\\
581.01	0\\
582.01	0\\
583.01	0\\
584.01	0\\
585.01	0\\
586.01	0\\
587.01	0\\
588.01	0\\
589.01	0\\
590.01	0\\
591.01	0\\
592.01	0\\
593.01	0\\
594.01	0\\
595.01	0\\
596.01	0\\
597.01	0.000480804875715751\\
598.01	0.00143832068185185\\
599.01	0.00385661257211618\\
599.02	0.00389414818835225\\
599.03	0.0039320414600908\\
599.04	0.00397029584140053\\
599.05	0.00400891481968839\\
599.06	0.00404790191602102\\
599.07	0.00408726068544939\\
599.08	0.00412699471733653\\
599.09	0.00416710763568844\\
599.1	0.00420760309948812\\
599.11	0.00424848480303291\\
599.12	0.00428975647627511\\
599.13	0.00433142188516562\\
599.14	0.00437348483200131\\
599.15	0.00441594915577533\\
599.16	0.00445881873253107\\
599.17	0.00450209747571942\\
599.18	0.00454578933655946\\
599.19	0.00458989830440269\\
599.2	0.00463442840710071\\
599.21	0.00467938371137647\\
599.22	0.00472476832319907\\
599.23	0.0047705863881622\\
599.24	0.00481684209186628\\
599.25	0.00486353966030409\\
599.26	0.00491068334955363\\
599.27	0.00495827744958433\\
599.28	0.00500632629158766\\
599.29	0.00505483424837287\\
599.3	0.0051038057347664\\
599.31	0.00515324520801534\\
599.32	0.00520315716819453\\
599.33	0.00525354615861772\\
599.34	0.00530441676625257\\
599.35	0.0053557736221397\\
599.36	0.00540762140181576\\
599.37	0.0054599648257405\\
599.38	0.00551280865972795\\
599.39	0.0055661577153818\\
599.4	0.00562001685053483\\
599.41	0.00567439096969272\\
599.42	0.00572928502448197\\
599.43	0.00578470401410213\\
599.44	0.0058406529857825\\
599.45	0.00589713703524302\\
599.46	0.00595416130715973\\
599.47	0.00601173099563459\\
599.48	0.00606985134466988\\
599.49	0.00612852764864712\\
599.5	0.00618776525281054\\
599.51	0.00624756955375525\\
599.52	0.00630794599992006\\
599.53	0.006368900092085\\
599.54	0.0064304373838737\\
599.55	0.00649256348226045\\
599.56	0.00655528404808222\\
599.57	0.00661860479655559\\
599.58	0.00668253149779856\\
599.59	0.00674706997735745\\
599.6	0.00681222611673878\\
599.61	0.0068780058539463\\
599.62	0.00694441518402313\\
599.63	0.00701146015959912\\
599.64	0.00707914689144345\\
599.65	0.00714748154902253\\
599.66	0.00721647036106324\\
599.67	0.00728611961612158\\
599.68	0.00735643566315675\\
599.69	0.00742742491211078\\
599.7	0.00749909383449363\\
599.71	0.00757144896397399\\
599.72	0.00764449689697571\\
599.73	0.00771824429327988\\
599.74	0.00779269787663284\\
599.75	0.00786786443535982\\
599.76	0.0079437508229846\\
599.77	0.00802036395885509\\
599.78	0.00809771082877485\\
599.79	0.00817579848564071\\
599.8	0.00825463405008648\\
599.81	0.00833422471113284\\
599.82	0.00841457772684349\\
599.83	0.00849570042498746\\
599.84	0.00857760020370797\\
599.85	0.00866028453219752\\
599.86	0.00874376095137954\\
599.87	0.00882803707459654\\
599.88	0.0089131205883049\\
599.89	0.00899901925277625\\
599.9	0.00908574090280562\\
599.91	0.00917329344842635\\
599.92	0.0092616848756319\\
599.93	0.00935092324710449\\
599.94	0.00944101670295078\\
599.95	0.00953197346144465\\
599.96	0.00962380181977693\\
599.97	0.00971651015481255\\
599.98	0.0098101069238547\\
599.99	0.00990460066541651\\
600	0.01\\
};
\addplot [color=red,solid,forget plot]
  table[row sep=crcr]{%
0.01	0\\
1.01	0\\
2.01	0\\
3.01	0\\
4.01	0\\
5.01	0\\
6.01	0\\
7.01	0\\
8.01	0\\
9.01	0\\
10.01	0\\
11.01	0\\
12.01	0\\
13.01	0\\
14.01	0\\
15.01	0\\
16.01	0\\
17.01	0\\
18.01	0\\
19.01	0\\
20.01	0\\
21.01	0\\
22.01	0\\
23.01	0\\
24.01	0\\
25.01	0\\
26.01	0\\
27.01	0\\
28.01	0\\
29.01	0\\
30.01	0\\
31.01	0\\
32.01	0\\
33.01	0\\
34.01	0\\
35.01	0\\
36.01	0\\
37.01	0\\
38.01	0\\
39.01	0\\
40.01	0\\
41.01	0\\
42.01	0\\
43.01	0\\
44.01	0\\
45.01	0\\
46.01	0\\
47.01	0\\
48.01	0\\
49.01	0\\
50.01	0\\
51.01	0\\
52.01	0\\
53.01	0\\
54.01	0\\
55.01	0\\
56.01	0\\
57.01	0\\
58.01	0\\
59.01	0\\
60.01	0\\
61.01	0\\
62.01	0\\
63.01	0\\
64.01	0\\
65.01	0\\
66.01	0\\
67.01	0\\
68.01	0\\
69.01	0\\
70.01	0\\
71.01	0\\
72.01	0\\
73.01	0\\
74.01	0\\
75.01	0\\
76.01	0\\
77.01	0\\
78.01	0\\
79.01	0\\
80.01	0\\
81.01	0\\
82.01	0\\
83.01	0\\
84.01	0\\
85.01	0\\
86.01	0\\
87.01	0\\
88.01	0\\
89.01	0\\
90.01	0\\
91.01	0\\
92.01	0\\
93.01	0\\
94.01	0\\
95.01	0\\
96.01	0\\
97.01	0\\
98.01	0\\
99.01	0\\
100.01	0\\
101.01	0\\
102.01	0\\
103.01	0\\
104.01	0\\
105.01	0\\
106.01	0\\
107.01	0\\
108.01	0\\
109.01	0\\
110.01	0\\
111.01	0\\
112.01	0\\
113.01	0\\
114.01	0\\
115.01	0\\
116.01	0\\
117.01	0\\
118.01	0\\
119.01	0\\
120.01	0\\
121.01	0\\
122.01	0\\
123.01	0\\
124.01	0\\
125.01	0\\
126.01	0\\
127.01	0\\
128.01	0\\
129.01	0\\
130.01	0\\
131.01	0\\
132.01	0\\
133.01	0\\
134.01	0\\
135.01	0\\
136.01	0\\
137.01	0\\
138.01	0\\
139.01	0\\
140.01	0\\
141.01	0\\
142.01	0\\
143.01	0\\
144.01	0\\
145.01	0\\
146.01	0\\
147.01	0\\
148.01	0\\
149.01	0\\
150.01	0\\
151.01	0\\
152.01	0\\
153.01	0\\
154.01	0\\
155.01	0\\
156.01	0\\
157.01	0\\
158.01	0\\
159.01	0\\
160.01	0\\
161.01	0\\
162.01	0\\
163.01	0\\
164.01	0\\
165.01	0\\
166.01	0\\
167.01	0\\
168.01	0\\
169.01	0\\
170.01	0\\
171.01	0\\
172.01	0\\
173.01	0\\
174.01	0\\
175.01	0\\
176.01	0\\
177.01	0\\
178.01	0\\
179.01	0\\
180.01	0\\
181.01	0\\
182.01	0\\
183.01	0\\
184.01	0\\
185.01	0\\
186.01	0\\
187.01	0\\
188.01	0\\
189.01	0\\
190.01	0\\
191.01	0\\
192.01	0\\
193.01	0\\
194.01	0\\
195.01	0\\
196.01	0\\
197.01	0\\
198.01	0\\
199.01	0\\
200.01	0\\
201.01	0\\
202.01	0\\
203.01	0\\
204.01	0\\
205.01	0\\
206.01	0\\
207.01	0\\
208.01	0\\
209.01	0\\
210.01	0\\
211.01	0\\
212.01	0\\
213.01	0\\
214.01	0\\
215.01	0\\
216.01	0\\
217.01	0\\
218.01	0\\
219.01	0\\
220.01	0\\
221.01	0\\
222.01	0\\
223.01	0\\
224.01	0\\
225.01	0\\
226.01	0\\
227.01	0\\
228.01	0\\
229.01	0\\
230.01	0\\
231.01	0\\
232.01	0\\
233.01	0\\
234.01	0\\
235.01	0\\
236.01	0\\
237.01	0\\
238.01	0\\
239.01	0\\
240.01	0\\
241.01	0\\
242.01	0\\
243.01	0\\
244.01	0\\
245.01	0\\
246.01	0\\
247.01	0\\
248.01	0\\
249.01	0\\
250.01	0\\
251.01	0\\
252.01	0\\
253.01	0\\
254.01	0\\
255.01	0\\
256.01	0\\
257.01	0\\
258.01	0\\
259.01	0\\
260.01	0\\
261.01	0\\
262.01	0\\
263.01	0\\
264.01	0\\
265.01	0\\
266.01	0\\
267.01	0\\
268.01	0\\
269.01	0\\
270.01	0\\
271.01	0\\
272.01	0\\
273.01	0\\
274.01	0\\
275.01	0\\
276.01	0\\
277.01	0\\
278.01	0\\
279.01	0\\
280.01	0\\
281.01	0\\
282.01	0\\
283.01	0\\
284.01	0\\
285.01	0\\
286.01	0\\
287.01	0\\
288.01	0\\
289.01	0\\
290.01	0\\
291.01	0\\
292.01	0\\
293.01	0\\
294.01	0\\
295.01	0\\
296.01	0\\
297.01	0\\
298.01	0\\
299.01	0\\
300.01	0\\
301.01	0\\
302.01	0\\
303.01	0\\
304.01	0\\
305.01	0\\
306.01	0\\
307.01	0\\
308.01	0\\
309.01	0\\
310.01	0\\
311.01	0\\
312.01	0\\
313.01	0\\
314.01	0\\
315.01	0\\
316.01	0\\
317.01	0\\
318.01	0\\
319.01	0\\
320.01	0\\
321.01	0\\
322.01	0\\
323.01	0\\
324.01	0\\
325.01	0\\
326.01	0\\
327.01	0\\
328.01	0\\
329.01	0\\
330.01	0\\
331.01	0\\
332.01	0\\
333.01	0\\
334.01	0\\
335.01	0\\
336.01	0\\
337.01	0\\
338.01	0\\
339.01	0\\
340.01	0\\
341.01	0\\
342.01	0\\
343.01	0\\
344.01	0\\
345.01	0\\
346.01	0\\
347.01	0\\
348.01	0\\
349.01	0\\
350.01	0\\
351.01	0\\
352.01	0\\
353.01	0\\
354.01	0\\
355.01	0\\
356.01	0\\
357.01	0\\
358.01	0\\
359.01	0\\
360.01	0\\
361.01	0\\
362.01	0\\
363.01	0\\
364.01	0\\
365.01	0\\
366.01	0\\
367.01	0\\
368.01	0\\
369.01	0\\
370.01	0\\
371.01	0\\
372.01	0\\
373.01	0\\
374.01	0\\
375.01	0\\
376.01	0\\
377.01	0\\
378.01	0\\
379.01	0\\
380.01	0\\
381.01	0\\
382.01	0\\
383.01	0\\
384.01	0\\
385.01	0\\
386.01	0\\
387.01	0\\
388.01	0\\
389.01	0\\
390.01	0\\
391.01	0\\
392.01	0\\
393.01	0\\
394.01	0\\
395.01	0\\
396.01	0\\
397.01	0\\
398.01	0\\
399.01	0\\
400.01	0\\
401.01	0\\
402.01	0\\
403.01	0\\
404.01	0\\
405.01	0\\
406.01	0\\
407.01	0\\
408.01	0\\
409.01	0\\
410.01	0\\
411.01	0\\
412.01	0\\
413.01	0\\
414.01	0\\
415.01	0\\
416.01	0\\
417.01	0\\
418.01	0\\
419.01	0\\
420.01	0\\
421.01	0\\
422.01	0\\
423.01	0\\
424.01	0\\
425.01	0\\
426.01	0\\
427.01	0\\
428.01	0\\
429.01	0\\
430.01	0\\
431.01	0\\
432.01	0\\
433.01	0\\
434.01	0\\
435.01	0\\
436.01	0\\
437.01	0\\
438.01	0\\
439.01	0\\
440.01	0\\
441.01	0\\
442.01	0\\
443.01	0\\
444.01	0\\
445.01	0\\
446.01	0\\
447.01	0\\
448.01	0\\
449.01	0\\
450.01	0\\
451.01	0\\
452.01	0\\
453.01	0\\
454.01	0\\
455.01	0\\
456.01	0\\
457.01	0\\
458.01	0\\
459.01	0\\
460.01	0\\
461.01	0\\
462.01	0\\
463.01	0\\
464.01	0\\
465.01	0\\
466.01	0\\
467.01	0\\
468.01	0\\
469.01	0\\
470.01	0\\
471.01	0\\
472.01	0\\
473.01	0\\
474.01	0\\
475.01	0\\
476.01	0\\
477.01	0\\
478.01	0\\
479.01	0\\
480.01	0\\
481.01	0\\
482.01	0\\
483.01	0\\
484.01	0\\
485.01	0\\
486.01	0\\
487.01	0\\
488.01	0\\
489.01	0\\
490.01	0\\
491.01	0\\
492.01	0\\
493.01	0\\
494.01	0\\
495.01	0\\
496.01	0\\
497.01	0\\
498.01	0\\
499.01	0\\
500.01	0\\
501.01	0\\
502.01	0\\
503.01	0\\
504.01	0\\
505.01	0\\
506.01	0\\
507.01	0\\
508.01	0\\
509.01	0\\
510.01	0\\
511.01	0\\
512.01	0\\
513.01	0\\
514.01	0\\
515.01	0\\
516.01	0\\
517.01	0\\
518.01	0\\
519.01	0\\
520.01	0\\
521.01	0\\
522.01	0\\
523.01	0\\
524.01	0\\
525.01	0\\
526.01	0\\
527.01	0\\
528.01	0\\
529.01	0\\
530.01	0\\
531.01	0\\
532.01	0\\
533.01	0\\
534.01	0\\
535.01	0\\
536.01	0\\
537.01	0\\
538.01	0\\
539.01	0\\
540.01	0\\
541.01	0\\
542.01	0\\
543.01	0\\
544.01	0\\
545.01	0\\
546.01	0\\
547.01	0\\
548.01	0\\
549.01	0\\
550.01	0\\
551.01	0\\
552.01	0\\
553.01	0\\
554.01	0\\
555.01	0\\
556.01	0\\
557.01	0\\
558.01	0\\
559.01	0\\
560.01	0\\
561.01	0\\
562.01	0\\
563.01	0\\
564.01	0\\
565.01	0\\
566.01	0\\
567.01	0\\
568.01	0\\
569.01	0\\
570.01	0\\
571.01	0\\
572.01	0\\
573.01	0\\
574.01	0\\
575.01	0\\
576.01	0\\
577.01	0\\
578.01	0\\
579.01	0\\
580.01	0\\
581.01	0\\
582.01	0\\
583.01	0\\
584.01	0\\
585.01	0\\
586.01	0\\
587.01	0\\
588.01	0\\
589.01	0\\
590.01	0\\
591.01	0\\
592.01	0\\
593.01	0\\
594.01	0\\
595.01	0\\
596.01	0\\
597.01	0.000480974774395043\\
598.01	0.00143832068185183\\
599.01	0.00385661257211624\\
599.02	0.00389414818835231\\
599.03	0.00393204146009085\\
599.04	0.0039702958414006\\
599.05	0.00400891481968847\\
599.06	0.00404790191602111\\
599.07	0.00408726068544948\\
599.08	0.00412699471733663\\
599.09	0.00416710763568853\\
599.1	0.0042076030994882\\
599.11	0.00424848480303301\\
599.12	0.00428975647627519\\
599.13	0.00433142188516572\\
599.14	0.0043734848320014\\
599.15	0.00441594915577541\\
599.16	0.00445881873253114\\
599.17	0.00450209747571947\\
599.18	0.00454578933655952\\
599.19	0.00458989830440275\\
599.2	0.00463442840710077\\
599.21	0.00467938371137651\\
599.22	0.0047247683231991\\
599.23	0.00477058638816223\\
599.24	0.00481684209186631\\
599.25	0.00486353966030413\\
599.26	0.00491068334955368\\
599.27	0.00495827744958438\\
599.28	0.0050063262915877\\
599.29	0.0050548342483729\\
599.3	0.00510380573476643\\
599.31	0.00515324520801537\\
599.32	0.00520315716819457\\
599.33	0.00525354615861776\\
599.34	0.00530441676625259\\
599.35	0.00535577362213973\\
599.36	0.00540762140181579\\
599.37	0.00545996482574052\\
599.38	0.00551280865972798\\
599.39	0.00556615771538182\\
599.4	0.00562001685053486\\
599.41	0.00567439096969275\\
599.42	0.00572928502448199\\
599.43	0.00578470401410216\\
599.44	0.00584065298578253\\
599.45	0.00589713703524305\\
599.46	0.00595416130715976\\
599.47	0.00601173099563462\\
599.48	0.00606985134466992\\
599.49	0.00612852764864716\\
599.5	0.00618776525281058\\
599.51	0.00624756955375529\\
599.52	0.00630794599992009\\
599.53	0.00636890009208504\\
599.54	0.00643043738387374\\
599.55	0.00649256348226048\\
599.56	0.00655528404808225\\
599.57	0.00661860479655562\\
599.58	0.00668253149779859\\
599.59	0.00674706997735749\\
599.6	0.00681222611673882\\
599.61	0.00687800585394634\\
599.62	0.00694441518402316\\
599.63	0.00701146015959914\\
599.64	0.00707914689144347\\
599.65	0.00714748154902254\\
599.66	0.00721647036106325\\
599.67	0.00728611961612159\\
599.68	0.00735643566315677\\
599.69	0.0074274249121108\\
599.7	0.00749909383449365\\
599.71	0.00757144896397401\\
599.72	0.00764449689697572\\
599.73	0.0077182442932799\\
599.74	0.00779269787663286\\
599.75	0.00786786443535983\\
599.76	0.00794375082298461\\
599.77	0.0080203639588551\\
599.78	0.00809771082877486\\
599.79	0.00817579848564072\\
599.8	0.00825463405008648\\
599.81	0.00833422471113285\\
599.82	0.0084145777268435\\
599.83	0.00849570042498747\\
599.84	0.00857760020370798\\
599.85	0.00866028453219753\\
599.86	0.00874376095137954\\
599.87	0.00882803707459654\\
599.88	0.0089131205883049\\
599.89	0.00899901925277625\\
599.9	0.00908574090280562\\
599.91	0.00917329344842636\\
599.92	0.00926168487563191\\
599.93	0.00935092324710449\\
599.94	0.00944101670295078\\
599.95	0.00953197346144465\\
599.96	0.00962380181977693\\
599.97	0.00971651015481255\\
599.98	0.0098101069238547\\
599.99	0.00990460066541651\\
600	0.01\\
};
\addplot [color=mycolor20,solid,forget plot]
  table[row sep=crcr]{%
0.01	0\\
1.01	0\\
2.01	0\\
3.01	0\\
4.01	0\\
5.01	0\\
6.01	0\\
7.01	0\\
8.01	0\\
9.01	0\\
10.01	0\\
11.01	0\\
12.01	0\\
13.01	0\\
14.01	0\\
15.01	0\\
16.01	0\\
17.01	0\\
18.01	0\\
19.01	0\\
20.01	0\\
21.01	0\\
22.01	0\\
23.01	0\\
24.01	0\\
25.01	0\\
26.01	0\\
27.01	0\\
28.01	0\\
29.01	0\\
30.01	0\\
31.01	0\\
32.01	0\\
33.01	0\\
34.01	0\\
35.01	0\\
36.01	0\\
37.01	0\\
38.01	0\\
39.01	0\\
40.01	0\\
41.01	0\\
42.01	0\\
43.01	0\\
44.01	0\\
45.01	0\\
46.01	0\\
47.01	0\\
48.01	0\\
49.01	0\\
50.01	0\\
51.01	0\\
52.01	0\\
53.01	0\\
54.01	0\\
55.01	0\\
56.01	0\\
57.01	0\\
58.01	0\\
59.01	0\\
60.01	0\\
61.01	0\\
62.01	0\\
63.01	0\\
64.01	0\\
65.01	0\\
66.01	0\\
67.01	0\\
68.01	0\\
69.01	0\\
70.01	0\\
71.01	0\\
72.01	0\\
73.01	0\\
74.01	0\\
75.01	0\\
76.01	0\\
77.01	0\\
78.01	0\\
79.01	0\\
80.01	0\\
81.01	0\\
82.01	0\\
83.01	0\\
84.01	0\\
85.01	0\\
86.01	0\\
87.01	0\\
88.01	0\\
89.01	0\\
90.01	0\\
91.01	0\\
92.01	0\\
93.01	0\\
94.01	0\\
95.01	0\\
96.01	0\\
97.01	0\\
98.01	0\\
99.01	0\\
100.01	0\\
101.01	0\\
102.01	0\\
103.01	0\\
104.01	0\\
105.01	0\\
106.01	0\\
107.01	0\\
108.01	0\\
109.01	0\\
110.01	0\\
111.01	0\\
112.01	0\\
113.01	0\\
114.01	0\\
115.01	0\\
116.01	0\\
117.01	0\\
118.01	0\\
119.01	0\\
120.01	0\\
121.01	0\\
122.01	0\\
123.01	0\\
124.01	0\\
125.01	0\\
126.01	0\\
127.01	0\\
128.01	0\\
129.01	0\\
130.01	0\\
131.01	0\\
132.01	0\\
133.01	0\\
134.01	0\\
135.01	0\\
136.01	0\\
137.01	0\\
138.01	0\\
139.01	0\\
140.01	0\\
141.01	0\\
142.01	0\\
143.01	0\\
144.01	0\\
145.01	0\\
146.01	0\\
147.01	0\\
148.01	0\\
149.01	0\\
150.01	0\\
151.01	0\\
152.01	0\\
153.01	0\\
154.01	0\\
155.01	0\\
156.01	0\\
157.01	0\\
158.01	0\\
159.01	0\\
160.01	0\\
161.01	0\\
162.01	0\\
163.01	0\\
164.01	0\\
165.01	0\\
166.01	0\\
167.01	0\\
168.01	0\\
169.01	0\\
170.01	0\\
171.01	0\\
172.01	0\\
173.01	0\\
174.01	0\\
175.01	0\\
176.01	0\\
177.01	0\\
178.01	0\\
179.01	0\\
180.01	0\\
181.01	0\\
182.01	0\\
183.01	0\\
184.01	0\\
185.01	0\\
186.01	0\\
187.01	0\\
188.01	0\\
189.01	0\\
190.01	0\\
191.01	0\\
192.01	0\\
193.01	0\\
194.01	0\\
195.01	0\\
196.01	0\\
197.01	0\\
198.01	0\\
199.01	0\\
200.01	0\\
201.01	0\\
202.01	0\\
203.01	0\\
204.01	0\\
205.01	0\\
206.01	0\\
207.01	0\\
208.01	0\\
209.01	0\\
210.01	0\\
211.01	0\\
212.01	0\\
213.01	0\\
214.01	0\\
215.01	0\\
216.01	0\\
217.01	0\\
218.01	0\\
219.01	0\\
220.01	0\\
221.01	0\\
222.01	0\\
223.01	0\\
224.01	0\\
225.01	0\\
226.01	0\\
227.01	0\\
228.01	0\\
229.01	0\\
230.01	0\\
231.01	0\\
232.01	0\\
233.01	0\\
234.01	0\\
235.01	0\\
236.01	0\\
237.01	0\\
238.01	0\\
239.01	0\\
240.01	0\\
241.01	0\\
242.01	0\\
243.01	0\\
244.01	0\\
245.01	0\\
246.01	0\\
247.01	0\\
248.01	0\\
249.01	0\\
250.01	0\\
251.01	0\\
252.01	0\\
253.01	0\\
254.01	0\\
255.01	0\\
256.01	0\\
257.01	0\\
258.01	0\\
259.01	0\\
260.01	0\\
261.01	0\\
262.01	0\\
263.01	0\\
264.01	0\\
265.01	0\\
266.01	0\\
267.01	0\\
268.01	0\\
269.01	0\\
270.01	0\\
271.01	0\\
272.01	0\\
273.01	0\\
274.01	0\\
275.01	0\\
276.01	0\\
277.01	0\\
278.01	0\\
279.01	0\\
280.01	0\\
281.01	0\\
282.01	0\\
283.01	0\\
284.01	0\\
285.01	0\\
286.01	0\\
287.01	0\\
288.01	0\\
289.01	0\\
290.01	0\\
291.01	0\\
292.01	0\\
293.01	0\\
294.01	0\\
295.01	0\\
296.01	0\\
297.01	0\\
298.01	0\\
299.01	0\\
300.01	0\\
301.01	0\\
302.01	0\\
303.01	0\\
304.01	0\\
305.01	0\\
306.01	0\\
307.01	0\\
308.01	0\\
309.01	0\\
310.01	0\\
311.01	0\\
312.01	0\\
313.01	0\\
314.01	0\\
315.01	0\\
316.01	0\\
317.01	0\\
318.01	0\\
319.01	0\\
320.01	0\\
321.01	0\\
322.01	0\\
323.01	0\\
324.01	0\\
325.01	0\\
326.01	0\\
327.01	0\\
328.01	0\\
329.01	0\\
330.01	0\\
331.01	0\\
332.01	0\\
333.01	0\\
334.01	0\\
335.01	0\\
336.01	0\\
337.01	0\\
338.01	0\\
339.01	0\\
340.01	0\\
341.01	0\\
342.01	0\\
343.01	0\\
344.01	0\\
345.01	0\\
346.01	0\\
347.01	0\\
348.01	0\\
349.01	0\\
350.01	0\\
351.01	0\\
352.01	0\\
353.01	0\\
354.01	0\\
355.01	0\\
356.01	0\\
357.01	0\\
358.01	0\\
359.01	0\\
360.01	0\\
361.01	0\\
362.01	0\\
363.01	0\\
364.01	0\\
365.01	0\\
366.01	0\\
367.01	0\\
368.01	0\\
369.01	0\\
370.01	0\\
371.01	0\\
372.01	0\\
373.01	0\\
374.01	0\\
375.01	0\\
376.01	0\\
377.01	0\\
378.01	0\\
379.01	0\\
380.01	0\\
381.01	0\\
382.01	0\\
383.01	0\\
384.01	0\\
385.01	0\\
386.01	0\\
387.01	0\\
388.01	0\\
389.01	0\\
390.01	0\\
391.01	0\\
392.01	0\\
393.01	0\\
394.01	0\\
395.01	0\\
396.01	0\\
397.01	0\\
398.01	0\\
399.01	0\\
400.01	0\\
401.01	0\\
402.01	0\\
403.01	0\\
404.01	0\\
405.01	0\\
406.01	0\\
407.01	0\\
408.01	0\\
409.01	0\\
410.01	0\\
411.01	0\\
412.01	0\\
413.01	0\\
414.01	0\\
415.01	0\\
416.01	0\\
417.01	0\\
418.01	0\\
419.01	0\\
420.01	0\\
421.01	0\\
422.01	0\\
423.01	0\\
424.01	0\\
425.01	0\\
426.01	0\\
427.01	0\\
428.01	0\\
429.01	0\\
430.01	0\\
431.01	0\\
432.01	0\\
433.01	0\\
434.01	0\\
435.01	0\\
436.01	0\\
437.01	0\\
438.01	0\\
439.01	0\\
440.01	0\\
441.01	0\\
442.01	0\\
443.01	0\\
444.01	0\\
445.01	0\\
446.01	0\\
447.01	0\\
448.01	0\\
449.01	0\\
450.01	0\\
451.01	0\\
452.01	0\\
453.01	0\\
454.01	0\\
455.01	0\\
456.01	0\\
457.01	0\\
458.01	0\\
459.01	0\\
460.01	0\\
461.01	0\\
462.01	0\\
463.01	0\\
464.01	0\\
465.01	0\\
466.01	0\\
467.01	0\\
468.01	0\\
469.01	0\\
470.01	0\\
471.01	0\\
472.01	0\\
473.01	0\\
474.01	0\\
475.01	0\\
476.01	0\\
477.01	0\\
478.01	0\\
479.01	0\\
480.01	0\\
481.01	0\\
482.01	0\\
483.01	0\\
484.01	0\\
485.01	0\\
486.01	0\\
487.01	0\\
488.01	0\\
489.01	0\\
490.01	0\\
491.01	0\\
492.01	0\\
493.01	0\\
494.01	0\\
495.01	0\\
496.01	0\\
497.01	0\\
498.01	0\\
499.01	0\\
500.01	0\\
501.01	0\\
502.01	0\\
503.01	0\\
504.01	0\\
505.01	0\\
506.01	0\\
507.01	0\\
508.01	0\\
509.01	0\\
510.01	0\\
511.01	0\\
512.01	0\\
513.01	0\\
514.01	0\\
515.01	0\\
516.01	0\\
517.01	0\\
518.01	0\\
519.01	0\\
520.01	0\\
521.01	0\\
522.01	0\\
523.01	0\\
524.01	0\\
525.01	0\\
526.01	0\\
527.01	0\\
528.01	0\\
529.01	0\\
530.01	0\\
531.01	0\\
532.01	0\\
533.01	0\\
534.01	0\\
535.01	0\\
536.01	0\\
537.01	0\\
538.01	0\\
539.01	0\\
540.01	0\\
541.01	0\\
542.01	0\\
543.01	0\\
544.01	0\\
545.01	0\\
546.01	0\\
547.01	0\\
548.01	0\\
549.01	0\\
550.01	0\\
551.01	0\\
552.01	0\\
553.01	0\\
554.01	0\\
555.01	0\\
556.01	0\\
557.01	0\\
558.01	0\\
559.01	0\\
560.01	0\\
561.01	0\\
562.01	0\\
563.01	0\\
564.01	0\\
565.01	0\\
566.01	0\\
567.01	0\\
568.01	0\\
569.01	0\\
570.01	0\\
571.01	0\\
572.01	0\\
573.01	0\\
574.01	0\\
575.01	0\\
576.01	0\\
577.01	0\\
578.01	0\\
579.01	0\\
580.01	0\\
581.01	0\\
582.01	0\\
583.01	0\\
584.01	0\\
585.01	0\\
586.01	0\\
587.01	0\\
588.01	0\\
589.01	0\\
590.01	0\\
591.01	0\\
592.01	0\\
593.01	0\\
594.01	0\\
595.01	0\\
596.01	0\\
597.01	0.000481110898971432\\
598.01	0.00143832068185183\\
599.01	0.00385661257211628\\
599.02	0.00389414818835235\\
599.03	0.00393204146009088\\
599.04	0.00397029584140061\\
599.05	0.00400891481968847\\
599.06	0.0040479019160211\\
599.07	0.00408726068544947\\
599.08	0.00412699471733662\\
599.09	0.00416710763568851\\
599.1	0.00420760309948819\\
599.11	0.004248484803033\\
599.12	0.00428975647627518\\
599.13	0.00433142188516569\\
599.14	0.00437348483200137\\
599.15	0.00441594915577538\\
599.16	0.00445881873253112\\
599.17	0.00450209747571947\\
599.18	0.00454578933655951\\
599.19	0.00458989830440275\\
599.2	0.00463442840710077\\
599.21	0.00467938371137652\\
599.22	0.00472476832319911\\
599.23	0.00477058638816225\\
599.24	0.00481684209186631\\
599.25	0.00486353966030412\\
599.26	0.00491068334955366\\
599.27	0.00495827744958435\\
599.28	0.00500632629158769\\
599.29	0.00505483424837289\\
599.3	0.00510380573476643\\
599.31	0.00515324520801537\\
599.32	0.00520315716819457\\
599.33	0.00525354615861776\\
599.34	0.00530441676625261\\
599.35	0.00535577362213975\\
599.36	0.0054076214018158\\
599.37	0.00545996482574054\\
599.38	0.00551280865972799\\
599.39	0.00556615771538183\\
599.4	0.00562001685053486\\
599.41	0.00567439096969276\\
599.42	0.00572928502448199\\
599.43	0.00578470401410217\\
599.44	0.00584065298578253\\
599.45	0.00589713703524306\\
599.46	0.00595416130715976\\
599.47	0.00601173099563461\\
599.48	0.00606985134466991\\
599.49	0.00612852764864715\\
599.5	0.00618776525281058\\
599.51	0.00624756955375528\\
599.52	0.00630794599992008\\
599.53	0.00636890009208502\\
599.54	0.00643043738387371\\
599.55	0.00649256348226045\\
599.56	0.00655528404808223\\
599.57	0.00661860479655559\\
599.58	0.00668253149779856\\
599.59	0.00674706997735745\\
599.6	0.00681222611673879\\
599.61	0.00687800585394631\\
599.62	0.00694441518402313\\
599.63	0.00701146015959912\\
599.64	0.00707914689144344\\
599.65	0.00714748154902252\\
599.66	0.00721647036106322\\
599.67	0.00728611961612156\\
599.68	0.00735643566315675\\
599.69	0.00742742491211077\\
599.7	0.00749909383449363\\
599.71	0.00757144896397399\\
599.72	0.00764449689697571\\
599.73	0.00771824429327989\\
599.74	0.00779269787663284\\
599.75	0.00786786443535982\\
599.76	0.0079437508229846\\
599.77	0.00802036395885509\\
599.78	0.00809771082877485\\
599.79	0.0081757984856407\\
599.8	0.00825463405008647\\
599.81	0.00833422471113284\\
599.82	0.00841457772684348\\
599.83	0.00849570042498746\\
599.84	0.00857760020370798\\
599.85	0.00866028453219752\\
599.86	0.00874376095137953\\
599.87	0.00882803707459654\\
599.88	0.00891312058830489\\
599.89	0.00899901925277625\\
599.9	0.00908574090280562\\
599.91	0.00917329344842636\\
599.92	0.00926168487563191\\
599.93	0.00935092324710449\\
599.94	0.00944101670295078\\
599.95	0.00953197346144465\\
599.96	0.00962380181977694\\
599.97	0.00971651015481255\\
599.98	0.0098101069238547\\
599.99	0.00990460066541651\\
600	0.01\\
};
\addplot [color=mycolor21,solid,forget plot]
  table[row sep=crcr]{%
0.01	0\\
1.01	0\\
2.01	0\\
3.01	0\\
4.01	0\\
5.01	0\\
6.01	0\\
7.01	0\\
8.01	0\\
9.01	0\\
10.01	0\\
11.01	0\\
12.01	0\\
13.01	0\\
14.01	0\\
15.01	0\\
16.01	0\\
17.01	0\\
18.01	0\\
19.01	0\\
20.01	0\\
21.01	0\\
22.01	0\\
23.01	0\\
24.01	0\\
25.01	0\\
26.01	0\\
27.01	0\\
28.01	0\\
29.01	0\\
30.01	0\\
31.01	0\\
32.01	0\\
33.01	0\\
34.01	0\\
35.01	0\\
36.01	0\\
37.01	0\\
38.01	0\\
39.01	0\\
40.01	0\\
41.01	0\\
42.01	0\\
43.01	0\\
44.01	0\\
45.01	0\\
46.01	0\\
47.01	0\\
48.01	0\\
49.01	0\\
50.01	0\\
51.01	0\\
52.01	0\\
53.01	0\\
54.01	0\\
55.01	0\\
56.01	0\\
57.01	0\\
58.01	0\\
59.01	0\\
60.01	0\\
61.01	0\\
62.01	0\\
63.01	0\\
64.01	0\\
65.01	0\\
66.01	0\\
67.01	0\\
68.01	0\\
69.01	0\\
70.01	0\\
71.01	0\\
72.01	0\\
73.01	0\\
74.01	0\\
75.01	0\\
76.01	0\\
77.01	0\\
78.01	0\\
79.01	0\\
80.01	0\\
81.01	0\\
82.01	0\\
83.01	0\\
84.01	0\\
85.01	0\\
86.01	0\\
87.01	0\\
88.01	0\\
89.01	0\\
90.01	0\\
91.01	0\\
92.01	0\\
93.01	0\\
94.01	0\\
95.01	0\\
96.01	0\\
97.01	0\\
98.01	0\\
99.01	0\\
100.01	0\\
101.01	0\\
102.01	0\\
103.01	0\\
104.01	0\\
105.01	0\\
106.01	0\\
107.01	0\\
108.01	0\\
109.01	0\\
110.01	0\\
111.01	0\\
112.01	0\\
113.01	0\\
114.01	0\\
115.01	0\\
116.01	0\\
117.01	0\\
118.01	0\\
119.01	0\\
120.01	0\\
121.01	0\\
122.01	0\\
123.01	0\\
124.01	0\\
125.01	0\\
126.01	0\\
127.01	0\\
128.01	0\\
129.01	0\\
130.01	0\\
131.01	0\\
132.01	0\\
133.01	0\\
134.01	0\\
135.01	0\\
136.01	0\\
137.01	0\\
138.01	0\\
139.01	0\\
140.01	0\\
141.01	0\\
142.01	0\\
143.01	0\\
144.01	0\\
145.01	0\\
146.01	0\\
147.01	0\\
148.01	0\\
149.01	0\\
150.01	0\\
151.01	0\\
152.01	0\\
153.01	0\\
154.01	0\\
155.01	0\\
156.01	0\\
157.01	0\\
158.01	0\\
159.01	0\\
160.01	0\\
161.01	0\\
162.01	0\\
163.01	0\\
164.01	0\\
165.01	0\\
166.01	0\\
167.01	0\\
168.01	0\\
169.01	0\\
170.01	0\\
171.01	0\\
172.01	0\\
173.01	0\\
174.01	0\\
175.01	0\\
176.01	0\\
177.01	0\\
178.01	0\\
179.01	0\\
180.01	0\\
181.01	0\\
182.01	0\\
183.01	0\\
184.01	0\\
185.01	0\\
186.01	0\\
187.01	0\\
188.01	0\\
189.01	0\\
190.01	0\\
191.01	0\\
192.01	0\\
193.01	0\\
194.01	0\\
195.01	0\\
196.01	0\\
197.01	0\\
198.01	0\\
199.01	0\\
200.01	0\\
201.01	0\\
202.01	0\\
203.01	0\\
204.01	0\\
205.01	0\\
206.01	0\\
207.01	0\\
208.01	0\\
209.01	0\\
210.01	0\\
211.01	0\\
212.01	0\\
213.01	0\\
214.01	0\\
215.01	0\\
216.01	0\\
217.01	0\\
218.01	0\\
219.01	0\\
220.01	0\\
221.01	0\\
222.01	0\\
223.01	0\\
224.01	0\\
225.01	0\\
226.01	0\\
227.01	0\\
228.01	0\\
229.01	0\\
230.01	0\\
231.01	0\\
232.01	0\\
233.01	0\\
234.01	0\\
235.01	0\\
236.01	0\\
237.01	0\\
238.01	0\\
239.01	0\\
240.01	0\\
241.01	0\\
242.01	0\\
243.01	0\\
244.01	0\\
245.01	0\\
246.01	0\\
247.01	0\\
248.01	0\\
249.01	0\\
250.01	0\\
251.01	0\\
252.01	0\\
253.01	0\\
254.01	0\\
255.01	0\\
256.01	0\\
257.01	0\\
258.01	0\\
259.01	0\\
260.01	0\\
261.01	0\\
262.01	0\\
263.01	0\\
264.01	0\\
265.01	0\\
266.01	0\\
267.01	0\\
268.01	0\\
269.01	0\\
270.01	0\\
271.01	0\\
272.01	0\\
273.01	0\\
274.01	0\\
275.01	0\\
276.01	0\\
277.01	0\\
278.01	0\\
279.01	0\\
280.01	0\\
281.01	0\\
282.01	0\\
283.01	0\\
284.01	0\\
285.01	0\\
286.01	0\\
287.01	0\\
288.01	0\\
289.01	0\\
290.01	0\\
291.01	0\\
292.01	0\\
293.01	0\\
294.01	0\\
295.01	0\\
296.01	0\\
297.01	0\\
298.01	0\\
299.01	0\\
300.01	0\\
301.01	0\\
302.01	0\\
303.01	0\\
304.01	0\\
305.01	0\\
306.01	0\\
307.01	0\\
308.01	0\\
309.01	0\\
310.01	0\\
311.01	0\\
312.01	0\\
313.01	0\\
314.01	0\\
315.01	0\\
316.01	0\\
317.01	0\\
318.01	0\\
319.01	0\\
320.01	0\\
321.01	0\\
322.01	0\\
323.01	0\\
324.01	0\\
325.01	0\\
326.01	0\\
327.01	0\\
328.01	0\\
329.01	0\\
330.01	0\\
331.01	0\\
332.01	0\\
333.01	0\\
334.01	0\\
335.01	0\\
336.01	0\\
337.01	0\\
338.01	0\\
339.01	0\\
340.01	0\\
341.01	0\\
342.01	0\\
343.01	0\\
344.01	0\\
345.01	0\\
346.01	0\\
347.01	0\\
348.01	0\\
349.01	0\\
350.01	0\\
351.01	0\\
352.01	0\\
353.01	0\\
354.01	0\\
355.01	0\\
356.01	0\\
357.01	0\\
358.01	0\\
359.01	0\\
360.01	0\\
361.01	0\\
362.01	0\\
363.01	0\\
364.01	0\\
365.01	0\\
366.01	0\\
367.01	0\\
368.01	0\\
369.01	0\\
370.01	0\\
371.01	0\\
372.01	0\\
373.01	0\\
374.01	0\\
375.01	0\\
376.01	0\\
377.01	0\\
378.01	0\\
379.01	0\\
380.01	0\\
381.01	0\\
382.01	0\\
383.01	0\\
384.01	0\\
385.01	0\\
386.01	0\\
387.01	0\\
388.01	0\\
389.01	0\\
390.01	0\\
391.01	0\\
392.01	0\\
393.01	0\\
394.01	0\\
395.01	0\\
396.01	0\\
397.01	0\\
398.01	0\\
399.01	0\\
400.01	0\\
401.01	0\\
402.01	0\\
403.01	0\\
404.01	0\\
405.01	0\\
406.01	0\\
407.01	0\\
408.01	0\\
409.01	0\\
410.01	0\\
411.01	0\\
412.01	0\\
413.01	0\\
414.01	0\\
415.01	0\\
416.01	0\\
417.01	0\\
418.01	0\\
419.01	0\\
420.01	0\\
421.01	0\\
422.01	0\\
423.01	0\\
424.01	0\\
425.01	0\\
426.01	0\\
427.01	0\\
428.01	0\\
429.01	0\\
430.01	0\\
431.01	0\\
432.01	0\\
433.01	0\\
434.01	0\\
435.01	0\\
436.01	0\\
437.01	0\\
438.01	0\\
439.01	0\\
440.01	0\\
441.01	0\\
442.01	0\\
443.01	0\\
444.01	0\\
445.01	0\\
446.01	0\\
447.01	0\\
448.01	0\\
449.01	0\\
450.01	0\\
451.01	0\\
452.01	0\\
453.01	0\\
454.01	0\\
455.01	0\\
456.01	0\\
457.01	0\\
458.01	0\\
459.01	0\\
460.01	0\\
461.01	0\\
462.01	0\\
463.01	0\\
464.01	0\\
465.01	0\\
466.01	0\\
467.01	0\\
468.01	0\\
469.01	0\\
470.01	0\\
471.01	0\\
472.01	0\\
473.01	0\\
474.01	0\\
475.01	0\\
476.01	0\\
477.01	0\\
478.01	0\\
479.01	0\\
480.01	0\\
481.01	0\\
482.01	0\\
483.01	0\\
484.01	0\\
485.01	0\\
486.01	0\\
487.01	0\\
488.01	0\\
489.01	0\\
490.01	0\\
491.01	0\\
492.01	0\\
493.01	0\\
494.01	0\\
495.01	0\\
496.01	0\\
497.01	0\\
498.01	0\\
499.01	0\\
500.01	0\\
501.01	0\\
502.01	0\\
503.01	0\\
504.01	0\\
505.01	0\\
506.01	0\\
507.01	0\\
508.01	0\\
509.01	0\\
510.01	0\\
511.01	0\\
512.01	0\\
513.01	0\\
514.01	0\\
515.01	0\\
516.01	0\\
517.01	0\\
518.01	0\\
519.01	0\\
520.01	0\\
521.01	0\\
522.01	0\\
523.01	0\\
524.01	0\\
525.01	0\\
526.01	0\\
527.01	0\\
528.01	0\\
529.01	0\\
530.01	0\\
531.01	0\\
532.01	0\\
533.01	0\\
534.01	0\\
535.01	0\\
536.01	0\\
537.01	0\\
538.01	0\\
539.01	0\\
540.01	0\\
541.01	0\\
542.01	0\\
543.01	0\\
544.01	0\\
545.01	0\\
546.01	0\\
547.01	0\\
548.01	0\\
549.01	0\\
550.01	0\\
551.01	0\\
552.01	0\\
553.01	0\\
554.01	0\\
555.01	0\\
556.01	0\\
557.01	0\\
558.01	0\\
559.01	0\\
560.01	0\\
561.01	0\\
562.01	0\\
563.01	0\\
564.01	0\\
565.01	0\\
566.01	0\\
567.01	0\\
568.01	0\\
569.01	0\\
570.01	0\\
571.01	0\\
572.01	0\\
573.01	0\\
574.01	0\\
575.01	0\\
576.01	0\\
577.01	0\\
578.01	0\\
579.01	0\\
580.01	0\\
581.01	0\\
582.01	0\\
583.01	0\\
584.01	0\\
585.01	0\\
586.01	0\\
587.01	0\\
588.01	0\\
589.01	0\\
590.01	0\\
591.01	0\\
592.01	0\\
593.01	0\\
594.01	0\\
595.01	0\\
596.01	0\\
597.01	0.000481187488961005\\
598.01	0.00143832068185176\\
599.01	0.00385661257211614\\
599.02	0.00389414818835221\\
599.03	0.00393204146009075\\
599.04	0.00397029584140049\\
599.05	0.00400891481968836\\
599.06	0.004047901916021\\
599.07	0.00408726068544937\\
599.08	0.00412699471733652\\
599.09	0.00416710763568842\\
599.1	0.00420760309948809\\
599.11	0.0042484848030329\\
599.12	0.00428975647627509\\
599.13	0.00433142188516562\\
599.14	0.00437348483200131\\
599.15	0.00441594915577533\\
599.16	0.00445881873253107\\
599.17	0.0045020974757194\\
599.18	0.00454578933655944\\
599.19	0.00458989830440268\\
599.2	0.0046344284071007\\
599.21	0.00467938371137645\\
599.22	0.00472476832319906\\
599.23	0.0047705863881622\\
599.24	0.00481684209186628\\
599.25	0.0048635396603041\\
599.26	0.00491068334955364\\
599.27	0.00495827744958434\\
599.28	0.00500632629158766\\
599.29	0.00505483424837286\\
599.3	0.00510380573476639\\
599.31	0.00515324520801533\\
599.32	0.00520315716819451\\
599.33	0.0052535461586177\\
599.34	0.00530441676625255\\
599.35	0.00535577362213968\\
599.36	0.00540762140181575\\
599.37	0.00545996482574049\\
599.38	0.00551280865972795\\
599.39	0.00556615771538178\\
599.4	0.00562001685053481\\
599.41	0.0056743909696927\\
599.42	0.00572928502448196\\
599.43	0.00578470401410212\\
599.44	0.0058406529857825\\
599.45	0.00589713703524301\\
599.46	0.00595416130715971\\
599.47	0.00601173099563457\\
599.48	0.00606985134466987\\
599.49	0.00612852764864711\\
599.5	0.00618776525281053\\
599.51	0.00624756955375524\\
599.52	0.00630794599992004\\
599.53	0.006368900092085\\
599.54	0.00643043738387369\\
599.55	0.00649256348226044\\
599.56	0.00655528404808221\\
599.57	0.00661860479655558\\
599.58	0.00668253149779856\\
599.59	0.00674706997735745\\
599.6	0.00681222611673878\\
599.61	0.0068780058539463\\
599.62	0.00694441518402313\\
599.63	0.00701146015959912\\
599.64	0.00707914689144345\\
599.65	0.00714748154902253\\
599.66	0.00721647036106324\\
599.67	0.00728611961612158\\
599.68	0.00735643566315675\\
599.69	0.00742742491211078\\
599.7	0.00749909383449363\\
599.71	0.00757144896397399\\
599.72	0.00764449689697571\\
599.73	0.00771824429327989\\
599.74	0.00779269787663285\\
599.75	0.00786786443535983\\
599.76	0.00794375082298461\\
599.77	0.0080203639588551\\
599.78	0.00809771082877486\\
599.79	0.00817579848564071\\
599.8	0.00825463405008649\\
599.81	0.00833422471113285\\
599.82	0.00841457772684349\\
599.83	0.00849570042498747\\
599.84	0.00857760020370797\\
599.85	0.00866028453219752\\
599.86	0.00874376095137953\\
599.87	0.00882803707459654\\
599.88	0.0089131205883049\\
599.89	0.00899901925277625\\
599.9	0.00908574090280562\\
599.91	0.00917329344842636\\
599.92	0.0092616848756319\\
599.93	0.00935092324710449\\
599.94	0.00944101670295078\\
599.95	0.00953197346144465\\
599.96	0.00962380181977693\\
599.97	0.00971651015481255\\
599.98	0.0098101069238547\\
599.99	0.00990460066541651\\
600	0.01\\
};
\addplot [color=black!20!mycolor21,solid,forget plot]
  table[row sep=crcr]{%
0.01	0\\
1.01	0\\
2.01	0\\
3.01	0\\
4.01	0\\
5.01	0\\
6.01	0\\
7.01	0\\
8.01	0\\
9.01	0\\
10.01	0\\
11.01	0\\
12.01	0\\
13.01	0\\
14.01	0\\
15.01	0\\
16.01	0\\
17.01	0\\
18.01	0\\
19.01	0\\
20.01	0\\
21.01	0\\
22.01	0\\
23.01	0\\
24.01	0\\
25.01	0\\
26.01	0\\
27.01	0\\
28.01	0\\
29.01	0\\
30.01	0\\
31.01	0\\
32.01	0\\
33.01	0\\
34.01	0\\
35.01	0\\
36.01	0\\
37.01	0\\
38.01	0\\
39.01	0\\
40.01	0\\
41.01	0\\
42.01	0\\
43.01	0\\
44.01	0\\
45.01	0\\
46.01	0\\
47.01	0\\
48.01	0\\
49.01	0\\
50.01	0\\
51.01	0\\
52.01	0\\
53.01	0\\
54.01	0\\
55.01	0\\
56.01	0\\
57.01	0\\
58.01	0\\
59.01	0\\
60.01	0\\
61.01	0\\
62.01	0\\
63.01	0\\
64.01	0\\
65.01	0\\
66.01	0\\
67.01	0\\
68.01	0\\
69.01	0\\
70.01	0\\
71.01	0\\
72.01	0\\
73.01	0\\
74.01	0\\
75.01	0\\
76.01	0\\
77.01	0\\
78.01	0\\
79.01	0\\
80.01	0\\
81.01	0\\
82.01	0\\
83.01	0\\
84.01	0\\
85.01	0\\
86.01	0\\
87.01	0\\
88.01	0\\
89.01	0\\
90.01	0\\
91.01	0\\
92.01	0\\
93.01	0\\
94.01	0\\
95.01	0\\
96.01	0\\
97.01	0\\
98.01	0\\
99.01	0\\
100.01	0\\
101.01	0\\
102.01	0\\
103.01	0\\
104.01	0\\
105.01	0\\
106.01	0\\
107.01	0\\
108.01	0\\
109.01	0\\
110.01	0\\
111.01	0\\
112.01	0\\
113.01	0\\
114.01	0\\
115.01	0\\
116.01	0\\
117.01	0\\
118.01	0\\
119.01	0\\
120.01	0\\
121.01	0\\
122.01	0\\
123.01	0\\
124.01	0\\
125.01	0\\
126.01	0\\
127.01	0\\
128.01	0\\
129.01	0\\
130.01	0\\
131.01	0\\
132.01	0\\
133.01	0\\
134.01	0\\
135.01	0\\
136.01	0\\
137.01	0\\
138.01	0\\
139.01	0\\
140.01	0\\
141.01	0\\
142.01	0\\
143.01	0\\
144.01	0\\
145.01	0\\
146.01	0\\
147.01	0\\
148.01	0\\
149.01	0\\
150.01	0\\
151.01	0\\
152.01	0\\
153.01	0\\
154.01	0\\
155.01	0\\
156.01	0\\
157.01	0\\
158.01	0\\
159.01	0\\
160.01	0\\
161.01	0\\
162.01	0\\
163.01	0\\
164.01	0\\
165.01	0\\
166.01	0\\
167.01	0\\
168.01	0\\
169.01	0\\
170.01	0\\
171.01	0\\
172.01	0\\
173.01	0\\
174.01	0\\
175.01	0\\
176.01	0\\
177.01	0\\
178.01	0\\
179.01	0\\
180.01	0\\
181.01	0\\
182.01	0\\
183.01	0\\
184.01	0\\
185.01	0\\
186.01	0\\
187.01	0\\
188.01	0\\
189.01	0\\
190.01	0\\
191.01	0\\
192.01	0\\
193.01	0\\
194.01	0\\
195.01	0\\
196.01	0\\
197.01	0\\
198.01	0\\
199.01	0\\
200.01	0\\
201.01	0\\
202.01	0\\
203.01	0\\
204.01	0\\
205.01	0\\
206.01	0\\
207.01	0\\
208.01	0\\
209.01	0\\
210.01	0\\
211.01	0\\
212.01	0\\
213.01	0\\
214.01	0\\
215.01	0\\
216.01	0\\
217.01	0\\
218.01	0\\
219.01	0\\
220.01	0\\
221.01	0\\
222.01	0\\
223.01	0\\
224.01	0\\
225.01	0\\
226.01	0\\
227.01	0\\
228.01	0\\
229.01	0\\
230.01	0\\
231.01	0\\
232.01	0\\
233.01	0\\
234.01	0\\
235.01	0\\
236.01	0\\
237.01	0\\
238.01	0\\
239.01	0\\
240.01	0\\
241.01	0\\
242.01	0\\
243.01	0\\
244.01	0\\
245.01	0\\
246.01	0\\
247.01	0\\
248.01	0\\
249.01	0\\
250.01	0\\
251.01	0\\
252.01	0\\
253.01	0\\
254.01	0\\
255.01	0\\
256.01	0\\
257.01	0\\
258.01	0\\
259.01	0\\
260.01	0\\
261.01	0\\
262.01	0\\
263.01	0\\
264.01	0\\
265.01	0\\
266.01	0\\
267.01	0\\
268.01	0\\
269.01	0\\
270.01	0\\
271.01	0\\
272.01	0\\
273.01	0\\
274.01	0\\
275.01	0\\
276.01	0\\
277.01	0\\
278.01	0\\
279.01	0\\
280.01	0\\
281.01	0\\
282.01	0\\
283.01	0\\
284.01	0\\
285.01	0\\
286.01	0\\
287.01	0\\
288.01	0\\
289.01	0\\
290.01	0\\
291.01	0\\
292.01	0\\
293.01	0\\
294.01	0\\
295.01	0\\
296.01	0\\
297.01	0\\
298.01	0\\
299.01	0\\
300.01	0\\
301.01	0\\
302.01	0\\
303.01	0\\
304.01	0\\
305.01	0\\
306.01	0\\
307.01	0\\
308.01	0\\
309.01	0\\
310.01	0\\
311.01	0\\
312.01	0\\
313.01	0\\
314.01	0\\
315.01	0\\
316.01	0\\
317.01	0\\
318.01	0\\
319.01	0\\
320.01	0\\
321.01	0\\
322.01	0\\
323.01	0\\
324.01	0\\
325.01	0\\
326.01	0\\
327.01	0\\
328.01	0\\
329.01	0\\
330.01	0\\
331.01	0\\
332.01	0\\
333.01	0\\
334.01	0\\
335.01	0\\
336.01	0\\
337.01	0\\
338.01	0\\
339.01	0\\
340.01	0\\
341.01	0\\
342.01	0\\
343.01	0\\
344.01	0\\
345.01	0\\
346.01	0\\
347.01	0\\
348.01	0\\
349.01	0\\
350.01	0\\
351.01	0\\
352.01	0\\
353.01	0\\
354.01	0\\
355.01	0\\
356.01	0\\
357.01	0\\
358.01	0\\
359.01	0\\
360.01	0\\
361.01	0\\
362.01	0\\
363.01	0\\
364.01	0\\
365.01	0\\
366.01	0\\
367.01	0\\
368.01	0\\
369.01	0\\
370.01	0\\
371.01	0\\
372.01	0\\
373.01	0\\
374.01	0\\
375.01	0\\
376.01	0\\
377.01	0\\
378.01	0\\
379.01	0\\
380.01	0\\
381.01	0\\
382.01	0\\
383.01	0\\
384.01	0\\
385.01	0\\
386.01	0\\
387.01	0\\
388.01	0\\
389.01	0\\
390.01	0\\
391.01	0\\
392.01	0\\
393.01	0\\
394.01	0\\
395.01	0\\
396.01	0\\
397.01	0\\
398.01	0\\
399.01	0\\
400.01	0\\
401.01	0\\
402.01	0\\
403.01	0\\
404.01	0\\
405.01	0\\
406.01	0\\
407.01	0\\
408.01	0\\
409.01	0\\
410.01	0\\
411.01	0\\
412.01	0\\
413.01	0\\
414.01	0\\
415.01	0\\
416.01	0\\
417.01	0\\
418.01	0\\
419.01	0\\
420.01	0\\
421.01	0\\
422.01	0\\
423.01	0\\
424.01	0\\
425.01	0\\
426.01	0\\
427.01	0\\
428.01	0\\
429.01	0\\
430.01	0\\
431.01	0\\
432.01	0\\
433.01	0\\
434.01	0\\
435.01	0\\
436.01	0\\
437.01	0\\
438.01	0\\
439.01	0\\
440.01	0\\
441.01	0\\
442.01	0\\
443.01	0\\
444.01	0\\
445.01	0\\
446.01	0\\
447.01	0\\
448.01	0\\
449.01	0\\
450.01	0\\
451.01	0\\
452.01	0\\
453.01	0\\
454.01	0\\
455.01	0\\
456.01	0\\
457.01	0\\
458.01	0\\
459.01	0\\
460.01	0\\
461.01	0\\
462.01	0\\
463.01	0\\
464.01	0\\
465.01	0\\
466.01	0\\
467.01	0\\
468.01	0\\
469.01	0\\
470.01	0\\
471.01	0\\
472.01	0\\
473.01	0\\
474.01	0\\
475.01	0\\
476.01	0\\
477.01	0\\
478.01	0\\
479.01	0\\
480.01	0\\
481.01	0\\
482.01	0\\
483.01	0\\
484.01	0\\
485.01	0\\
486.01	0\\
487.01	0\\
488.01	0\\
489.01	0\\
490.01	0\\
491.01	0\\
492.01	0\\
493.01	0\\
494.01	0\\
495.01	0\\
496.01	0\\
497.01	0\\
498.01	0\\
499.01	0\\
500.01	0\\
501.01	0\\
502.01	0\\
503.01	0\\
504.01	0\\
505.01	0\\
506.01	0\\
507.01	0\\
508.01	0\\
509.01	0\\
510.01	0\\
511.01	0\\
512.01	0\\
513.01	0\\
514.01	0\\
515.01	0\\
516.01	0\\
517.01	0\\
518.01	0\\
519.01	0\\
520.01	0\\
521.01	0\\
522.01	0\\
523.01	0\\
524.01	0\\
525.01	0\\
526.01	0\\
527.01	0\\
528.01	0\\
529.01	0\\
530.01	0\\
531.01	0\\
532.01	0\\
533.01	0\\
534.01	0\\
535.01	0\\
536.01	0\\
537.01	0\\
538.01	0\\
539.01	0\\
540.01	0\\
541.01	0\\
542.01	0\\
543.01	0\\
544.01	0\\
545.01	0\\
546.01	0\\
547.01	0\\
548.01	0\\
549.01	0\\
550.01	0\\
551.01	0\\
552.01	0\\
553.01	0\\
554.01	0\\
555.01	0\\
556.01	0\\
557.01	0\\
558.01	0\\
559.01	0\\
560.01	0\\
561.01	0\\
562.01	0\\
563.01	0\\
564.01	0\\
565.01	0\\
566.01	0\\
567.01	0\\
568.01	0\\
569.01	0\\
570.01	0\\
571.01	0\\
572.01	0\\
573.01	0\\
574.01	0\\
575.01	0\\
576.01	0\\
577.01	0\\
578.01	0\\
579.01	0\\
580.01	0\\
581.01	0\\
582.01	0\\
583.01	0\\
584.01	0\\
585.01	0\\
586.01	0\\
587.01	0\\
588.01	0\\
589.01	0\\
590.01	0\\
591.01	0\\
592.01	0\\
593.01	0\\
594.01	0\\
595.01	0\\
596.01	0\\
597.01	0.000481249826739166\\
598.01	0.00143832068185196\\
599.01	0.00385661257211625\\
599.02	0.00389414818835232\\
599.03	0.00393204146009087\\
599.04	0.0039702958414006\\
599.05	0.00400891481968846\\
599.06	0.00404790191602108\\
599.07	0.00408726068544946\\
599.08	0.0041269947173366\\
599.09	0.00416710763568851\\
599.1	0.00420760309948819\\
599.11	0.00424848480303298\\
599.12	0.00428975647627516\\
599.13	0.00433142188516568\\
599.14	0.00437348483200135\\
599.15	0.00441594915577537\\
599.16	0.00445881873253111\\
599.17	0.00450209747571946\\
599.18	0.00454578933655951\\
599.19	0.00458989830440275\\
599.2	0.00463442840710078\\
599.21	0.00467938371137652\\
599.22	0.00472476832319911\\
599.23	0.00477058638816225\\
599.24	0.00481684209186631\\
599.25	0.00486353966030413\\
599.26	0.00491068334955368\\
599.27	0.00495827744958437\\
599.28	0.0050063262915877\\
599.29	0.0050548342483729\\
599.3	0.00510380573476643\\
599.31	0.00515324520801537\\
599.32	0.00520315716819457\\
599.33	0.00525354615861774\\
599.34	0.00530441676625258\\
599.35	0.00535577362213972\\
599.36	0.00540762140181579\\
599.37	0.00545996482574053\\
599.38	0.00551280865972797\\
599.39	0.00556615771538181\\
599.4	0.00562001685053486\\
599.41	0.00567439096969276\\
599.42	0.00572928502448199\\
599.43	0.00578470401410215\\
599.44	0.00584065298578251\\
599.45	0.00589713703524303\\
599.46	0.00595416130715974\\
599.47	0.00601173099563461\\
599.48	0.0060698513446699\\
599.49	0.00612852764864714\\
599.5	0.00618776525281057\\
599.51	0.00624756955375527\\
599.52	0.00630794599992007\\
599.53	0.00636890009208502\\
599.54	0.00643043738387371\\
599.55	0.00649256348226046\\
599.56	0.00655528404808223\\
599.57	0.00661860479655561\\
599.58	0.00668253149779858\\
599.59	0.00674706997735747\\
599.6	0.0068122261167388\\
599.61	0.00687800585394632\\
599.62	0.00694441518402315\\
599.63	0.00701146015959913\\
599.64	0.00707914689144346\\
599.65	0.00714748154902253\\
599.66	0.00721647036106324\\
599.67	0.00728611961612158\\
599.68	0.00735643566315675\\
599.69	0.00742742491211078\\
599.7	0.00749909383449363\\
599.71	0.00757144896397399\\
599.72	0.0076444968969757\\
599.73	0.00771824429327989\\
599.74	0.00779269787663284\\
599.75	0.00786786443535982\\
599.76	0.0079437508229846\\
599.77	0.0080203639588551\\
599.78	0.00809771082877485\\
599.79	0.00817579848564071\\
599.8	0.00825463405008648\\
599.81	0.00833422471113284\\
599.82	0.00841457772684348\\
599.83	0.00849570042498746\\
599.84	0.00857760020370798\\
599.85	0.00866028453219752\\
599.86	0.00874376095137953\\
599.87	0.00882803707459653\\
599.88	0.00891312058830489\\
599.89	0.00899901925277624\\
599.9	0.00908574090280562\\
599.91	0.00917329344842635\\
599.92	0.0092616848756319\\
599.93	0.00935092324710449\\
599.94	0.00944101670295078\\
599.95	0.00953197346144465\\
599.96	0.00962380181977694\\
599.97	0.00971651015481255\\
599.98	0.0098101069238547\\
599.99	0.00990460066541651\\
600	0.01\\
};
\addplot [color=black!50!mycolor20,solid,forget plot]
  table[row sep=crcr]{%
0.01	0\\
1.01	0\\
2.01	0\\
3.01	0\\
4.01	0\\
5.01	0\\
6.01	0\\
7.01	0\\
8.01	0\\
9.01	0\\
10.01	0\\
11.01	0\\
12.01	0\\
13.01	0\\
14.01	0\\
15.01	0\\
16.01	0\\
17.01	0\\
18.01	0\\
19.01	0\\
20.01	0\\
21.01	0\\
22.01	0\\
23.01	0\\
24.01	0\\
25.01	0\\
26.01	0\\
27.01	0\\
28.01	0\\
29.01	0\\
30.01	0\\
31.01	0\\
32.01	0\\
33.01	0\\
34.01	0\\
35.01	0\\
36.01	0\\
37.01	0\\
38.01	0\\
39.01	0\\
40.01	0\\
41.01	0\\
42.01	0\\
43.01	0\\
44.01	0\\
45.01	0\\
46.01	0\\
47.01	0\\
48.01	0\\
49.01	0\\
50.01	0\\
51.01	0\\
52.01	0\\
53.01	0\\
54.01	0\\
55.01	0\\
56.01	0\\
57.01	0\\
58.01	0\\
59.01	0\\
60.01	0\\
61.01	0\\
62.01	0\\
63.01	0\\
64.01	0\\
65.01	0\\
66.01	0\\
67.01	0\\
68.01	0\\
69.01	0\\
70.01	0\\
71.01	0\\
72.01	0\\
73.01	0\\
74.01	0\\
75.01	0\\
76.01	0\\
77.01	0\\
78.01	0\\
79.01	0\\
80.01	0\\
81.01	0\\
82.01	0\\
83.01	0\\
84.01	0\\
85.01	0\\
86.01	0\\
87.01	0\\
88.01	0\\
89.01	0\\
90.01	0\\
91.01	0\\
92.01	0\\
93.01	0\\
94.01	0\\
95.01	0\\
96.01	0\\
97.01	0\\
98.01	0\\
99.01	0\\
100.01	0\\
101.01	0\\
102.01	0\\
103.01	0\\
104.01	0\\
105.01	0\\
106.01	0\\
107.01	0\\
108.01	0\\
109.01	0\\
110.01	0\\
111.01	0\\
112.01	0\\
113.01	0\\
114.01	0\\
115.01	0\\
116.01	0\\
117.01	0\\
118.01	0\\
119.01	0\\
120.01	0\\
121.01	0\\
122.01	0\\
123.01	0\\
124.01	0\\
125.01	0\\
126.01	0\\
127.01	0\\
128.01	0\\
129.01	0\\
130.01	0\\
131.01	0\\
132.01	0\\
133.01	0\\
134.01	0\\
135.01	0\\
136.01	0\\
137.01	0\\
138.01	0\\
139.01	0\\
140.01	0\\
141.01	0\\
142.01	0\\
143.01	0\\
144.01	0\\
145.01	0\\
146.01	0\\
147.01	0\\
148.01	0\\
149.01	0\\
150.01	0\\
151.01	0\\
152.01	0\\
153.01	0\\
154.01	0\\
155.01	0\\
156.01	0\\
157.01	0\\
158.01	0\\
159.01	0\\
160.01	0\\
161.01	0\\
162.01	0\\
163.01	0\\
164.01	0\\
165.01	0\\
166.01	0\\
167.01	0\\
168.01	0\\
169.01	0\\
170.01	0\\
171.01	0\\
172.01	0\\
173.01	0\\
174.01	0\\
175.01	0\\
176.01	0\\
177.01	0\\
178.01	0\\
179.01	0\\
180.01	0\\
181.01	0\\
182.01	0\\
183.01	0\\
184.01	0\\
185.01	0\\
186.01	0\\
187.01	0\\
188.01	0\\
189.01	0\\
190.01	0\\
191.01	0\\
192.01	0\\
193.01	0\\
194.01	0\\
195.01	0\\
196.01	0\\
197.01	0\\
198.01	0\\
199.01	0\\
200.01	0\\
201.01	0\\
202.01	0\\
203.01	0\\
204.01	0\\
205.01	0\\
206.01	0\\
207.01	0\\
208.01	0\\
209.01	0\\
210.01	0\\
211.01	0\\
212.01	0\\
213.01	0\\
214.01	0\\
215.01	0\\
216.01	0\\
217.01	0\\
218.01	0\\
219.01	0\\
220.01	0\\
221.01	0\\
222.01	0\\
223.01	0\\
224.01	0\\
225.01	0\\
226.01	0\\
227.01	0\\
228.01	0\\
229.01	0\\
230.01	0\\
231.01	0\\
232.01	0\\
233.01	0\\
234.01	0\\
235.01	0\\
236.01	0\\
237.01	0\\
238.01	0\\
239.01	0\\
240.01	0\\
241.01	0\\
242.01	0\\
243.01	0\\
244.01	0\\
245.01	0\\
246.01	0\\
247.01	0\\
248.01	0\\
249.01	0\\
250.01	0\\
251.01	0\\
252.01	0\\
253.01	0\\
254.01	0\\
255.01	0\\
256.01	0\\
257.01	0\\
258.01	0\\
259.01	0\\
260.01	0\\
261.01	0\\
262.01	0\\
263.01	0\\
264.01	0\\
265.01	0\\
266.01	0\\
267.01	0\\
268.01	0\\
269.01	0\\
270.01	0\\
271.01	0\\
272.01	0\\
273.01	0\\
274.01	0\\
275.01	0\\
276.01	0\\
277.01	0\\
278.01	0\\
279.01	0\\
280.01	0\\
281.01	0\\
282.01	0\\
283.01	0\\
284.01	0\\
285.01	0\\
286.01	0\\
287.01	0\\
288.01	0\\
289.01	0\\
290.01	0\\
291.01	0\\
292.01	0\\
293.01	0\\
294.01	0\\
295.01	0\\
296.01	0\\
297.01	0\\
298.01	0\\
299.01	0\\
300.01	0\\
301.01	0\\
302.01	0\\
303.01	0\\
304.01	0\\
305.01	0\\
306.01	0\\
307.01	0\\
308.01	0\\
309.01	0\\
310.01	0\\
311.01	0\\
312.01	0\\
313.01	0\\
314.01	0\\
315.01	0\\
316.01	0\\
317.01	0\\
318.01	0\\
319.01	0\\
320.01	0\\
321.01	0\\
322.01	0\\
323.01	0\\
324.01	0\\
325.01	0\\
326.01	0\\
327.01	0\\
328.01	0\\
329.01	0\\
330.01	0\\
331.01	0\\
332.01	0\\
333.01	0\\
334.01	0\\
335.01	0\\
336.01	0\\
337.01	0\\
338.01	0\\
339.01	0\\
340.01	0\\
341.01	0\\
342.01	0\\
343.01	0\\
344.01	0\\
345.01	0\\
346.01	0\\
347.01	0\\
348.01	0\\
349.01	0\\
350.01	0\\
351.01	0\\
352.01	0\\
353.01	0\\
354.01	0\\
355.01	0\\
356.01	0\\
357.01	0\\
358.01	0\\
359.01	0\\
360.01	0\\
361.01	0\\
362.01	0\\
363.01	0\\
364.01	0\\
365.01	0\\
366.01	0\\
367.01	0\\
368.01	0\\
369.01	0\\
370.01	0\\
371.01	0\\
372.01	0\\
373.01	0\\
374.01	0\\
375.01	0\\
376.01	0\\
377.01	0\\
378.01	0\\
379.01	0\\
380.01	0\\
381.01	0\\
382.01	0\\
383.01	0\\
384.01	0\\
385.01	0\\
386.01	0\\
387.01	0\\
388.01	0\\
389.01	0\\
390.01	0\\
391.01	0\\
392.01	0\\
393.01	0\\
394.01	0\\
395.01	0\\
396.01	0\\
397.01	0\\
398.01	0\\
399.01	0\\
400.01	0\\
401.01	0\\
402.01	0\\
403.01	0\\
404.01	0\\
405.01	0\\
406.01	0\\
407.01	0\\
408.01	0\\
409.01	0\\
410.01	0\\
411.01	0\\
412.01	0\\
413.01	0\\
414.01	0\\
415.01	0\\
416.01	0\\
417.01	0\\
418.01	0\\
419.01	0\\
420.01	0\\
421.01	0\\
422.01	0\\
423.01	0\\
424.01	0\\
425.01	0\\
426.01	0\\
427.01	0\\
428.01	0\\
429.01	0\\
430.01	0\\
431.01	0\\
432.01	0\\
433.01	0\\
434.01	0\\
435.01	0\\
436.01	0\\
437.01	0\\
438.01	0\\
439.01	0\\
440.01	0\\
441.01	0\\
442.01	0\\
443.01	0\\
444.01	0\\
445.01	0\\
446.01	0\\
447.01	0\\
448.01	0\\
449.01	0\\
450.01	0\\
451.01	0\\
452.01	0\\
453.01	0\\
454.01	0\\
455.01	0\\
456.01	0\\
457.01	0\\
458.01	0\\
459.01	0\\
460.01	0\\
461.01	0\\
462.01	0\\
463.01	0\\
464.01	0\\
465.01	0\\
466.01	0\\
467.01	0\\
468.01	0\\
469.01	0\\
470.01	0\\
471.01	0\\
472.01	0\\
473.01	0\\
474.01	0\\
475.01	0\\
476.01	0\\
477.01	0\\
478.01	0\\
479.01	0\\
480.01	0\\
481.01	0\\
482.01	0\\
483.01	0\\
484.01	0\\
485.01	0\\
486.01	0\\
487.01	0\\
488.01	0\\
489.01	0\\
490.01	0\\
491.01	0\\
492.01	0\\
493.01	0\\
494.01	0\\
495.01	0\\
496.01	0\\
497.01	0\\
498.01	0\\
499.01	0\\
500.01	0\\
501.01	0\\
502.01	0\\
503.01	0\\
504.01	0\\
505.01	0\\
506.01	0\\
507.01	0\\
508.01	0\\
509.01	0\\
510.01	0\\
511.01	0\\
512.01	0\\
513.01	0\\
514.01	0\\
515.01	0\\
516.01	0\\
517.01	0\\
518.01	0\\
519.01	0\\
520.01	0\\
521.01	0\\
522.01	0\\
523.01	0\\
524.01	0\\
525.01	0\\
526.01	0\\
527.01	0\\
528.01	0\\
529.01	0\\
530.01	0\\
531.01	0\\
532.01	0\\
533.01	0\\
534.01	0\\
535.01	0\\
536.01	0\\
537.01	0\\
538.01	0\\
539.01	0\\
540.01	0\\
541.01	0\\
542.01	0\\
543.01	0\\
544.01	0\\
545.01	0\\
546.01	0\\
547.01	0\\
548.01	0\\
549.01	0\\
550.01	0\\
551.01	0\\
552.01	0\\
553.01	0\\
554.01	0\\
555.01	0\\
556.01	0\\
557.01	0\\
558.01	0\\
559.01	0\\
560.01	0\\
561.01	0\\
562.01	0\\
563.01	0\\
564.01	0\\
565.01	0\\
566.01	0\\
567.01	0\\
568.01	0\\
569.01	0\\
570.01	0\\
571.01	0\\
572.01	0\\
573.01	0\\
574.01	0\\
575.01	0\\
576.01	0\\
577.01	0\\
578.01	0\\
579.01	0\\
580.01	0\\
581.01	0\\
582.01	0\\
583.01	0\\
584.01	0\\
585.01	0\\
586.01	0\\
587.01	0\\
588.01	0\\
589.01	0\\
590.01	0\\
591.01	0\\
592.01	0\\
593.01	0\\
594.01	0\\
595.01	0\\
596.01	0\\
597.01	0.000481304278336731\\
598.01	0.0014383206818519\\
599.01	0.00385661257211628\\
599.02	0.00389414818835235\\
599.03	0.00393204146009089\\
599.04	0.00397029584140063\\
599.05	0.00400891481968849\\
599.06	0.00404790191602111\\
599.07	0.00408726068544948\\
599.08	0.00412699471733663\\
599.09	0.00416710763568853\\
599.1	0.00420760309948821\\
599.11	0.00424848480303303\\
599.12	0.00428975647627521\\
599.13	0.00433142188516572\\
599.14	0.0043734848320014\\
599.15	0.00441594915577541\\
599.16	0.00445881873253114\\
599.17	0.00450209747571947\\
599.18	0.00454578933655951\\
599.19	0.00458989830440273\\
599.2	0.00463442840710075\\
599.21	0.00467938371137651\\
599.22	0.0047247683231991\\
599.23	0.00477058638816223\\
599.24	0.00481684209186631\\
599.25	0.00486353966030412\\
599.26	0.00491068334955366\\
599.27	0.00495827744958437\\
599.28	0.00500632629158769\\
599.29	0.00505483424837289\\
599.3	0.00510380573476643\\
599.31	0.00515324520801538\\
599.32	0.00520315716819458\\
599.33	0.00525354615861777\\
599.34	0.00530441676625261\\
599.35	0.00535577362213975\\
599.36	0.00540762140181582\\
599.37	0.00545996482574056\\
599.38	0.005512808659728\\
599.39	0.00556615771538184\\
599.4	0.00562001685053487\\
599.41	0.00567439096969276\\
599.42	0.005729285024482\\
599.43	0.00578470401410217\\
599.44	0.00584065298578254\\
599.45	0.00589713703524307\\
599.46	0.00595416130715977\\
599.47	0.00601173099563464\\
599.48	0.00606985134466993\\
599.49	0.00612852764864717\\
599.5	0.00618776525281059\\
599.51	0.00624756955375529\\
599.52	0.00630794599992009\\
599.53	0.00636890009208503\\
599.54	0.00643043738387374\\
599.55	0.00649256348226047\\
599.56	0.00655528404808225\\
599.57	0.00661860479655561\\
599.58	0.00668253149779858\\
599.59	0.00674706997735747\\
599.6	0.0068122261167388\\
599.61	0.00687800585394632\\
599.62	0.00694441518402315\\
599.63	0.00701146015959914\\
599.64	0.00707914689144346\\
599.65	0.00714748154902254\\
599.66	0.00721647036106324\\
599.67	0.00728611961612158\\
599.68	0.00735643566315676\\
599.69	0.00742742491211078\\
599.7	0.00749909383449363\\
599.71	0.007571448963974\\
599.72	0.00764449689697572\\
599.73	0.00771824429327989\\
599.74	0.00779269787663285\\
599.75	0.00786786443535983\\
599.76	0.00794375082298461\\
599.77	0.0080203639588551\\
599.78	0.00809771082877486\\
599.79	0.00817579848564072\\
599.8	0.00825463405008649\\
599.81	0.00833422471113286\\
599.82	0.00841457772684349\\
599.83	0.00849570042498747\\
599.84	0.00857760020370798\\
599.85	0.00866028453219753\\
599.86	0.00874376095137954\\
599.87	0.00882803707459654\\
599.88	0.0089131205883049\\
599.89	0.00899901925277626\\
599.9	0.00908574090280562\\
599.91	0.00917329344842636\\
599.92	0.00926168487563191\\
599.93	0.00935092324710449\\
599.94	0.00944101670295079\\
599.95	0.00953197346144465\\
599.96	0.00962380181977693\\
599.97	0.00971651015481255\\
599.98	0.0098101069238547\\
599.99	0.00990460066541651\\
600	0.01\\
};
\addplot [color=black!60!mycolor21,solid,forget plot]
  table[row sep=crcr]{%
0.01	0\\
1.01	0\\
2.01	0\\
3.01	0\\
4.01	0\\
5.01	0\\
6.01	0\\
7.01	0\\
8.01	0\\
9.01	0\\
10.01	0\\
11.01	0\\
12.01	0\\
13.01	0\\
14.01	0\\
15.01	0\\
16.01	0\\
17.01	0\\
18.01	0\\
19.01	0\\
20.01	0\\
21.01	0\\
22.01	0\\
23.01	0\\
24.01	0\\
25.01	0\\
26.01	0\\
27.01	0\\
28.01	0\\
29.01	0\\
30.01	0\\
31.01	0\\
32.01	0\\
33.01	0\\
34.01	0\\
35.01	0\\
36.01	0\\
37.01	0\\
38.01	0\\
39.01	0\\
40.01	0\\
41.01	0\\
42.01	0\\
43.01	0\\
44.01	0\\
45.01	0\\
46.01	0\\
47.01	0\\
48.01	0\\
49.01	0\\
50.01	0\\
51.01	0\\
52.01	0\\
53.01	0\\
54.01	0\\
55.01	0\\
56.01	0\\
57.01	0\\
58.01	0\\
59.01	0\\
60.01	0\\
61.01	0\\
62.01	0\\
63.01	0\\
64.01	0\\
65.01	0\\
66.01	0\\
67.01	0\\
68.01	0\\
69.01	0\\
70.01	0\\
71.01	0\\
72.01	0\\
73.01	0\\
74.01	0\\
75.01	0\\
76.01	0\\
77.01	0\\
78.01	0\\
79.01	0\\
80.01	0\\
81.01	0\\
82.01	0\\
83.01	0\\
84.01	0\\
85.01	0\\
86.01	0\\
87.01	0\\
88.01	0\\
89.01	0\\
90.01	0\\
91.01	0\\
92.01	0\\
93.01	0\\
94.01	0\\
95.01	0\\
96.01	0\\
97.01	0\\
98.01	0\\
99.01	0\\
100.01	0\\
101.01	0\\
102.01	0\\
103.01	0\\
104.01	0\\
105.01	0\\
106.01	0\\
107.01	0\\
108.01	0\\
109.01	0\\
110.01	0\\
111.01	0\\
112.01	0\\
113.01	0\\
114.01	0\\
115.01	0\\
116.01	0\\
117.01	0\\
118.01	0\\
119.01	0\\
120.01	0\\
121.01	0\\
122.01	0\\
123.01	0\\
124.01	0\\
125.01	0\\
126.01	0\\
127.01	0\\
128.01	0\\
129.01	0\\
130.01	0\\
131.01	0\\
132.01	0\\
133.01	0\\
134.01	0\\
135.01	0\\
136.01	0\\
137.01	0\\
138.01	0\\
139.01	0\\
140.01	0\\
141.01	0\\
142.01	0\\
143.01	0\\
144.01	0\\
145.01	0\\
146.01	0\\
147.01	0\\
148.01	0\\
149.01	0\\
150.01	0\\
151.01	0\\
152.01	0\\
153.01	0\\
154.01	0\\
155.01	0\\
156.01	0\\
157.01	0\\
158.01	0\\
159.01	0\\
160.01	0\\
161.01	0\\
162.01	0\\
163.01	0\\
164.01	0\\
165.01	0\\
166.01	0\\
167.01	0\\
168.01	0\\
169.01	0\\
170.01	0\\
171.01	0\\
172.01	0\\
173.01	0\\
174.01	0\\
175.01	0\\
176.01	0\\
177.01	0\\
178.01	0\\
179.01	0\\
180.01	0\\
181.01	0\\
182.01	0\\
183.01	0\\
184.01	0\\
185.01	0\\
186.01	0\\
187.01	0\\
188.01	0\\
189.01	0\\
190.01	0\\
191.01	0\\
192.01	0\\
193.01	0\\
194.01	0\\
195.01	0\\
196.01	0\\
197.01	0\\
198.01	0\\
199.01	0\\
200.01	0\\
201.01	0\\
202.01	0\\
203.01	0\\
204.01	0\\
205.01	0\\
206.01	0\\
207.01	0\\
208.01	0\\
209.01	0\\
210.01	0\\
211.01	0\\
212.01	0\\
213.01	0\\
214.01	0\\
215.01	0\\
216.01	0\\
217.01	0\\
218.01	0\\
219.01	0\\
220.01	0\\
221.01	0\\
222.01	0\\
223.01	0\\
224.01	0\\
225.01	0\\
226.01	0\\
227.01	0\\
228.01	0\\
229.01	0\\
230.01	0\\
231.01	0\\
232.01	0\\
233.01	0\\
234.01	0\\
235.01	0\\
236.01	0\\
237.01	0\\
238.01	0\\
239.01	0\\
240.01	0\\
241.01	0\\
242.01	0\\
243.01	0\\
244.01	0\\
245.01	0\\
246.01	0\\
247.01	0\\
248.01	0\\
249.01	0\\
250.01	0\\
251.01	0\\
252.01	0\\
253.01	0\\
254.01	0\\
255.01	0\\
256.01	0\\
257.01	0\\
258.01	0\\
259.01	0\\
260.01	0\\
261.01	0\\
262.01	0\\
263.01	0\\
264.01	0\\
265.01	0\\
266.01	0\\
267.01	0\\
268.01	0\\
269.01	0\\
270.01	0\\
271.01	0\\
272.01	0\\
273.01	0\\
274.01	0\\
275.01	0\\
276.01	0\\
277.01	0\\
278.01	0\\
279.01	0\\
280.01	0\\
281.01	0\\
282.01	0\\
283.01	0\\
284.01	0\\
285.01	0\\
286.01	0\\
287.01	0\\
288.01	0\\
289.01	0\\
290.01	0\\
291.01	0\\
292.01	0\\
293.01	0\\
294.01	0\\
295.01	0\\
296.01	0\\
297.01	0\\
298.01	0\\
299.01	0\\
300.01	0\\
301.01	0\\
302.01	0\\
303.01	0\\
304.01	0\\
305.01	0\\
306.01	0\\
307.01	0\\
308.01	0\\
309.01	0\\
310.01	0\\
311.01	0\\
312.01	0\\
313.01	0\\
314.01	0\\
315.01	0\\
316.01	0\\
317.01	0\\
318.01	0\\
319.01	0\\
320.01	0\\
321.01	0\\
322.01	0\\
323.01	0\\
324.01	0\\
325.01	0\\
326.01	0\\
327.01	0\\
328.01	0\\
329.01	0\\
330.01	0\\
331.01	0\\
332.01	0\\
333.01	0\\
334.01	0\\
335.01	0\\
336.01	0\\
337.01	0\\
338.01	0\\
339.01	0\\
340.01	0\\
341.01	0\\
342.01	0\\
343.01	0\\
344.01	0\\
345.01	0\\
346.01	0\\
347.01	0\\
348.01	0\\
349.01	0\\
350.01	0\\
351.01	0\\
352.01	0\\
353.01	0\\
354.01	0\\
355.01	0\\
356.01	0\\
357.01	0\\
358.01	0\\
359.01	0\\
360.01	0\\
361.01	0\\
362.01	0\\
363.01	0\\
364.01	0\\
365.01	0\\
366.01	0\\
367.01	0\\
368.01	0\\
369.01	0\\
370.01	0\\
371.01	0\\
372.01	0\\
373.01	0\\
374.01	0\\
375.01	0\\
376.01	0\\
377.01	0\\
378.01	0\\
379.01	0\\
380.01	0\\
381.01	0\\
382.01	0\\
383.01	0\\
384.01	0\\
385.01	0\\
386.01	0\\
387.01	0\\
388.01	0\\
389.01	0\\
390.01	0\\
391.01	0\\
392.01	0\\
393.01	0\\
394.01	0\\
395.01	0\\
396.01	0\\
397.01	0\\
398.01	0\\
399.01	0\\
400.01	0\\
401.01	0\\
402.01	0\\
403.01	0\\
404.01	0\\
405.01	0\\
406.01	0\\
407.01	0\\
408.01	0\\
409.01	0\\
410.01	0\\
411.01	0\\
412.01	0\\
413.01	0\\
414.01	0\\
415.01	0\\
416.01	0\\
417.01	0\\
418.01	0\\
419.01	0\\
420.01	0\\
421.01	0\\
422.01	0\\
423.01	0\\
424.01	0\\
425.01	0\\
426.01	0\\
427.01	0\\
428.01	0\\
429.01	0\\
430.01	0\\
431.01	0\\
432.01	0\\
433.01	0\\
434.01	0\\
435.01	0\\
436.01	0\\
437.01	0\\
438.01	0\\
439.01	0\\
440.01	0\\
441.01	0\\
442.01	0\\
443.01	0\\
444.01	0\\
445.01	0\\
446.01	0\\
447.01	0\\
448.01	0\\
449.01	0\\
450.01	0\\
451.01	0\\
452.01	0\\
453.01	0\\
454.01	0\\
455.01	0\\
456.01	0\\
457.01	0\\
458.01	0\\
459.01	0\\
460.01	0\\
461.01	0\\
462.01	0\\
463.01	0\\
464.01	0\\
465.01	0\\
466.01	0\\
467.01	0\\
468.01	0\\
469.01	0\\
470.01	0\\
471.01	0\\
472.01	0\\
473.01	0\\
474.01	0\\
475.01	0\\
476.01	0\\
477.01	0\\
478.01	0\\
479.01	0\\
480.01	0\\
481.01	0\\
482.01	0\\
483.01	0\\
484.01	0\\
485.01	0\\
486.01	0\\
487.01	0\\
488.01	0\\
489.01	0\\
490.01	0\\
491.01	0\\
492.01	0\\
493.01	0\\
494.01	0\\
495.01	0\\
496.01	0\\
497.01	0\\
498.01	0\\
499.01	0\\
500.01	0\\
501.01	0\\
502.01	0\\
503.01	0\\
504.01	0\\
505.01	0\\
506.01	0\\
507.01	0\\
508.01	0\\
509.01	0\\
510.01	0\\
511.01	0\\
512.01	0\\
513.01	0\\
514.01	0\\
515.01	0\\
516.01	0\\
517.01	0\\
518.01	0\\
519.01	0\\
520.01	0\\
521.01	0\\
522.01	0\\
523.01	0\\
524.01	0\\
525.01	0\\
526.01	0\\
527.01	0\\
528.01	0\\
529.01	0\\
530.01	0\\
531.01	0\\
532.01	0\\
533.01	0\\
534.01	0\\
535.01	0\\
536.01	0\\
537.01	0\\
538.01	0\\
539.01	0\\
540.01	0\\
541.01	0\\
542.01	0\\
543.01	0\\
544.01	0\\
545.01	0\\
546.01	0\\
547.01	0\\
548.01	0\\
549.01	0\\
550.01	0\\
551.01	0\\
552.01	0\\
553.01	0\\
554.01	0\\
555.01	0\\
556.01	0\\
557.01	0\\
558.01	0\\
559.01	0\\
560.01	0\\
561.01	0\\
562.01	0\\
563.01	0\\
564.01	0\\
565.01	0\\
566.01	0\\
567.01	0\\
568.01	0\\
569.01	0\\
570.01	0\\
571.01	0\\
572.01	0\\
573.01	0\\
574.01	0\\
575.01	0\\
576.01	0\\
577.01	0\\
578.01	0\\
579.01	0\\
580.01	0\\
581.01	0\\
582.01	0\\
583.01	0\\
584.01	0\\
585.01	0\\
586.01	0\\
587.01	0\\
588.01	0\\
589.01	0\\
590.01	0\\
591.01	0\\
592.01	0\\
593.01	0\\
594.01	0\\
595.01	0\\
596.01	0\\
597.01	0.000481322506726112\\
598.01	0.00143832068185176\\
599.01	0.00385661257211617\\
599.02	0.00389414818835224\\
599.03	0.00393204146009077\\
599.04	0.00397029584140052\\
599.05	0.00400891481968839\\
599.06	0.00404790191602103\\
599.07	0.0040872606854494\\
599.08	0.00412699471733655\\
599.09	0.00416710763568844\\
599.1	0.0042076030994881\\
599.11	0.00424848480303291\\
599.12	0.00428975647627511\\
599.13	0.00433142188516564\\
599.14	0.00437348483200133\\
599.15	0.00441594915577534\\
599.16	0.00445881873253108\\
599.17	0.00450209747571942\\
599.18	0.00454578933655947\\
599.19	0.00458989830440271\\
599.2	0.00463442840710072\\
599.21	0.00467938371137648\\
599.22	0.00472476832319908\\
599.23	0.00477058638816223\\
599.24	0.00481684209186631\\
599.25	0.00486353966030413\\
599.26	0.00491068334955366\\
599.27	0.00495827744958435\\
599.28	0.00500632629158769\\
599.29	0.00505483424837289\\
599.3	0.00510380573476642\\
599.31	0.00515324520801536\\
599.32	0.00520315716819454\\
599.33	0.00525354615861773\\
599.34	0.00530441676625257\\
599.35	0.0053557736221397\\
599.36	0.00540762140181576\\
599.37	0.0054599648257405\\
599.38	0.00551280865972796\\
599.39	0.00556615771538181\\
599.4	0.00562001685053484\\
599.41	0.00567439096969274\\
599.42	0.00572928502448199\\
599.43	0.00578470401410215\\
599.44	0.00584065298578251\\
599.45	0.00589713703524303\\
599.46	0.00595416130715973\\
599.47	0.00601173099563458\\
599.48	0.00606985134466989\\
599.49	0.00612852764864713\\
599.5	0.00618776525281055\\
599.51	0.00624756955375526\\
599.52	0.00630794599992006\\
599.53	0.006368900092085\\
599.54	0.00643043738387369\\
599.55	0.00649256348226043\\
599.56	0.00655528404808221\\
599.57	0.00661860479655559\\
599.58	0.00668253149779856\\
599.59	0.00674706997735745\\
599.6	0.00681222611673879\\
599.61	0.00687800585394631\\
599.62	0.00694441518402313\\
599.63	0.00701146015959912\\
599.64	0.00707914689144344\\
599.65	0.00714748154902252\\
599.66	0.00721647036106323\\
599.67	0.00728611961612157\\
599.68	0.00735643566315675\\
599.69	0.00742742491211077\\
599.7	0.00749909383449363\\
599.71	0.00757144896397399\\
599.72	0.0076444968969757\\
599.73	0.00771824429327989\\
599.74	0.00779269787663284\\
599.75	0.00786786443535981\\
599.76	0.0079437508229846\\
599.77	0.00802036395885509\\
599.78	0.00809771082877485\\
599.79	0.00817579848564071\\
599.8	0.00825463405008648\\
599.81	0.00833422471113284\\
599.82	0.00841457772684349\\
599.83	0.00849570042498746\\
599.84	0.00857760020370797\\
599.85	0.00866028453219752\\
599.86	0.00874376095137954\\
599.87	0.00882803707459654\\
599.88	0.0089131205883049\\
599.89	0.00899901925277625\\
599.9	0.00908574090280562\\
599.91	0.00917329344842635\\
599.92	0.0092616848756319\\
599.93	0.00935092324710449\\
599.94	0.00944101670295078\\
599.95	0.00953197346144465\\
599.96	0.00962380181977693\\
599.97	0.00971651015481255\\
599.98	0.0098101069238547\\
599.99	0.00990460066541651\\
600	0.01\\
};
\addplot [color=black!80!mycolor21,solid,forget plot]
  table[row sep=crcr]{%
0.01	0\\
1.01	0\\
2.01	0\\
3.01	0\\
4.01	0\\
5.01	0\\
6.01	0\\
7.01	0\\
8.01	0\\
9.01	0\\
10.01	0\\
11.01	0\\
12.01	0\\
13.01	0\\
14.01	0\\
15.01	0\\
16.01	0\\
17.01	0\\
18.01	0\\
19.01	0\\
20.01	0\\
21.01	0\\
22.01	0\\
23.01	0\\
24.01	0\\
25.01	0\\
26.01	0\\
27.01	0\\
28.01	0\\
29.01	0\\
30.01	0\\
31.01	0\\
32.01	0\\
33.01	0\\
34.01	0\\
35.01	0\\
36.01	0\\
37.01	0\\
38.01	0\\
39.01	0\\
40.01	0\\
41.01	0\\
42.01	0\\
43.01	0\\
44.01	0\\
45.01	0\\
46.01	0\\
47.01	0\\
48.01	0\\
49.01	0\\
50.01	0\\
51.01	0\\
52.01	0\\
53.01	0\\
54.01	0\\
55.01	0\\
56.01	0\\
57.01	0\\
58.01	0\\
59.01	0\\
60.01	0\\
61.01	0\\
62.01	0\\
63.01	0\\
64.01	0\\
65.01	0\\
66.01	0\\
67.01	0\\
68.01	0\\
69.01	0\\
70.01	0\\
71.01	0\\
72.01	0\\
73.01	0\\
74.01	0\\
75.01	0\\
76.01	0\\
77.01	0\\
78.01	0\\
79.01	0\\
80.01	0\\
81.01	0\\
82.01	0\\
83.01	0\\
84.01	0\\
85.01	0\\
86.01	0\\
87.01	0\\
88.01	0\\
89.01	0\\
90.01	0\\
91.01	0\\
92.01	0\\
93.01	0\\
94.01	0\\
95.01	0\\
96.01	0\\
97.01	0\\
98.01	0\\
99.01	0\\
100.01	0\\
101.01	0\\
102.01	0\\
103.01	0\\
104.01	0\\
105.01	0\\
106.01	0\\
107.01	0\\
108.01	0\\
109.01	0\\
110.01	0\\
111.01	0\\
112.01	0\\
113.01	0\\
114.01	0\\
115.01	0\\
116.01	0\\
117.01	0\\
118.01	0\\
119.01	0\\
120.01	0\\
121.01	0\\
122.01	0\\
123.01	0\\
124.01	0\\
125.01	0\\
126.01	0\\
127.01	0\\
128.01	0\\
129.01	0\\
130.01	0\\
131.01	0\\
132.01	0\\
133.01	0\\
134.01	0\\
135.01	0\\
136.01	0\\
137.01	0\\
138.01	0\\
139.01	0\\
140.01	0\\
141.01	0\\
142.01	0\\
143.01	0\\
144.01	0\\
145.01	0\\
146.01	0\\
147.01	0\\
148.01	0\\
149.01	0\\
150.01	0\\
151.01	0\\
152.01	0\\
153.01	0\\
154.01	0\\
155.01	0\\
156.01	0\\
157.01	0\\
158.01	0\\
159.01	0\\
160.01	0\\
161.01	0\\
162.01	0\\
163.01	0\\
164.01	0\\
165.01	0\\
166.01	0\\
167.01	0\\
168.01	0\\
169.01	0\\
170.01	0\\
171.01	0\\
172.01	0\\
173.01	0\\
174.01	0\\
175.01	0\\
176.01	0\\
177.01	0\\
178.01	0\\
179.01	0\\
180.01	0\\
181.01	0\\
182.01	0\\
183.01	0\\
184.01	0\\
185.01	0\\
186.01	0\\
187.01	0\\
188.01	0\\
189.01	0\\
190.01	0\\
191.01	0\\
192.01	0\\
193.01	0\\
194.01	0\\
195.01	0\\
196.01	0\\
197.01	0\\
198.01	0\\
199.01	0\\
200.01	0\\
201.01	0\\
202.01	0\\
203.01	0\\
204.01	0\\
205.01	0\\
206.01	0\\
207.01	0\\
208.01	0\\
209.01	0\\
210.01	0\\
211.01	0\\
212.01	0\\
213.01	0\\
214.01	0\\
215.01	0\\
216.01	0\\
217.01	0\\
218.01	0\\
219.01	0\\
220.01	0\\
221.01	0\\
222.01	0\\
223.01	0\\
224.01	0\\
225.01	0\\
226.01	0\\
227.01	0\\
228.01	0\\
229.01	0\\
230.01	0\\
231.01	0\\
232.01	0\\
233.01	0\\
234.01	0\\
235.01	0\\
236.01	0\\
237.01	0\\
238.01	0\\
239.01	0\\
240.01	0\\
241.01	0\\
242.01	0\\
243.01	0\\
244.01	0\\
245.01	0\\
246.01	0\\
247.01	0\\
248.01	0\\
249.01	0\\
250.01	0\\
251.01	0\\
252.01	0\\
253.01	0\\
254.01	0\\
255.01	0\\
256.01	0\\
257.01	0\\
258.01	0\\
259.01	0\\
260.01	0\\
261.01	0\\
262.01	0\\
263.01	0\\
264.01	0\\
265.01	0\\
266.01	0\\
267.01	0\\
268.01	0\\
269.01	0\\
270.01	0\\
271.01	0\\
272.01	0\\
273.01	0\\
274.01	0\\
275.01	0\\
276.01	0\\
277.01	0\\
278.01	0\\
279.01	0\\
280.01	0\\
281.01	0\\
282.01	0\\
283.01	0\\
284.01	0\\
285.01	0\\
286.01	0\\
287.01	0\\
288.01	0\\
289.01	0\\
290.01	0\\
291.01	0\\
292.01	0\\
293.01	0\\
294.01	0\\
295.01	0\\
296.01	0\\
297.01	0\\
298.01	0\\
299.01	0\\
300.01	0\\
301.01	0\\
302.01	0\\
303.01	0\\
304.01	0\\
305.01	0\\
306.01	0\\
307.01	0\\
308.01	0\\
309.01	0\\
310.01	0\\
311.01	0\\
312.01	0\\
313.01	0\\
314.01	0\\
315.01	0\\
316.01	0\\
317.01	0\\
318.01	0\\
319.01	0\\
320.01	0\\
321.01	0\\
322.01	0\\
323.01	0\\
324.01	0\\
325.01	0\\
326.01	0\\
327.01	0\\
328.01	0\\
329.01	0\\
330.01	0\\
331.01	0\\
332.01	0\\
333.01	0\\
334.01	0\\
335.01	0\\
336.01	0\\
337.01	0\\
338.01	0\\
339.01	0\\
340.01	0\\
341.01	0\\
342.01	0\\
343.01	0\\
344.01	0\\
345.01	0\\
346.01	0\\
347.01	0\\
348.01	0\\
349.01	0\\
350.01	0\\
351.01	0\\
352.01	0\\
353.01	0\\
354.01	0\\
355.01	0\\
356.01	0\\
357.01	0\\
358.01	0\\
359.01	0\\
360.01	0\\
361.01	0\\
362.01	0\\
363.01	0\\
364.01	0\\
365.01	0\\
366.01	0\\
367.01	0\\
368.01	0\\
369.01	0\\
370.01	0\\
371.01	0\\
372.01	0\\
373.01	0\\
374.01	0\\
375.01	0\\
376.01	0\\
377.01	0\\
378.01	0\\
379.01	0\\
380.01	0\\
381.01	0\\
382.01	0\\
383.01	0\\
384.01	0\\
385.01	0\\
386.01	0\\
387.01	0\\
388.01	0\\
389.01	0\\
390.01	0\\
391.01	0\\
392.01	0\\
393.01	0\\
394.01	0\\
395.01	0\\
396.01	0\\
397.01	0\\
398.01	0\\
399.01	0\\
400.01	0\\
401.01	0\\
402.01	0\\
403.01	0\\
404.01	0\\
405.01	0\\
406.01	0\\
407.01	0\\
408.01	0\\
409.01	0\\
410.01	0\\
411.01	0\\
412.01	0\\
413.01	0\\
414.01	0\\
415.01	0\\
416.01	0\\
417.01	0\\
418.01	0\\
419.01	0\\
420.01	0\\
421.01	0\\
422.01	0\\
423.01	0\\
424.01	0\\
425.01	0\\
426.01	0\\
427.01	0\\
428.01	0\\
429.01	0\\
430.01	0\\
431.01	0\\
432.01	0\\
433.01	0\\
434.01	0\\
435.01	0\\
436.01	0\\
437.01	0\\
438.01	0\\
439.01	0\\
440.01	0\\
441.01	0\\
442.01	0\\
443.01	0\\
444.01	0\\
445.01	0\\
446.01	0\\
447.01	0\\
448.01	0\\
449.01	0\\
450.01	0\\
451.01	0\\
452.01	0\\
453.01	0\\
454.01	0\\
455.01	0\\
456.01	0\\
457.01	0\\
458.01	0\\
459.01	0\\
460.01	0\\
461.01	0\\
462.01	0\\
463.01	0\\
464.01	0\\
465.01	0\\
466.01	0\\
467.01	0\\
468.01	0\\
469.01	0\\
470.01	0\\
471.01	0\\
472.01	0\\
473.01	0\\
474.01	0\\
475.01	0\\
476.01	0\\
477.01	0\\
478.01	0\\
479.01	0\\
480.01	0\\
481.01	0\\
482.01	0\\
483.01	0\\
484.01	0\\
485.01	0\\
486.01	0\\
487.01	0\\
488.01	0\\
489.01	0\\
490.01	0\\
491.01	0\\
492.01	0\\
493.01	0\\
494.01	0\\
495.01	0\\
496.01	0\\
497.01	0\\
498.01	0\\
499.01	0\\
500.01	0\\
501.01	0\\
502.01	0\\
503.01	0\\
504.01	0\\
505.01	0\\
506.01	0\\
507.01	0\\
508.01	0\\
509.01	0\\
510.01	0\\
511.01	0\\
512.01	0\\
513.01	0\\
514.01	0\\
515.01	0\\
516.01	0\\
517.01	0\\
518.01	0\\
519.01	0\\
520.01	0\\
521.01	0\\
522.01	0\\
523.01	0\\
524.01	0\\
525.01	0\\
526.01	0\\
527.01	0\\
528.01	0\\
529.01	0\\
530.01	0\\
531.01	0\\
532.01	0\\
533.01	0\\
534.01	0\\
535.01	0\\
536.01	0\\
537.01	0\\
538.01	0\\
539.01	0\\
540.01	0\\
541.01	0\\
542.01	0\\
543.01	0\\
544.01	0\\
545.01	0\\
546.01	0\\
547.01	0\\
548.01	0\\
549.01	0\\
550.01	0\\
551.01	0\\
552.01	0\\
553.01	0\\
554.01	0\\
555.01	0\\
556.01	0\\
557.01	0\\
558.01	0\\
559.01	0\\
560.01	0\\
561.01	0\\
562.01	0\\
563.01	0\\
564.01	0\\
565.01	0\\
566.01	0\\
567.01	0\\
568.01	0\\
569.01	0\\
570.01	0\\
571.01	0\\
572.01	0\\
573.01	0\\
574.01	0\\
575.01	0\\
576.01	0\\
577.01	0\\
578.01	0\\
579.01	0\\
580.01	0\\
581.01	0\\
582.01	0\\
583.01	0\\
584.01	0\\
585.01	0\\
586.01	0\\
587.01	0\\
588.01	0\\
589.01	0\\
590.01	0\\
591.01	0\\
592.01	0\\
593.01	0\\
594.01	0\\
595.01	0\\
596.01	0\\
597.01	0.000481328854172736\\
598.01	0.00143832068185187\\
599.01	0.00385661257211624\\
599.02	0.00389414818835231\\
599.03	0.00393204146009085\\
599.04	0.00397029584140059\\
599.05	0.00400891481968844\\
599.06	0.00404790191602107\\
599.07	0.00408726068544944\\
599.08	0.00412699471733657\\
599.09	0.00416710763568848\\
599.1	0.00420760309948816\\
599.11	0.00424848480303296\\
599.12	0.00428975647627514\\
599.13	0.00433142188516565\\
599.14	0.00437348483200133\\
599.15	0.00441594915577534\\
599.16	0.00445881873253108\\
599.17	0.00450209747571943\\
599.18	0.00454578933655947\\
599.19	0.00458989830440271\\
599.2	0.00463442840710074\\
599.21	0.00467938371137648\\
599.22	0.00472476832319907\\
599.23	0.0047705863881622\\
599.24	0.00481684209186627\\
599.25	0.00486353966030409\\
599.26	0.00491068334955364\\
599.27	0.00495827744958434\\
599.28	0.00500632629158768\\
599.29	0.00505483424837287\\
599.3	0.0051038057347664\\
599.31	0.00515324520801534\\
599.32	0.00520315716819454\\
599.33	0.00525354615861773\\
599.34	0.00530441676625258\\
599.35	0.00535577362213972\\
599.36	0.00540762140181578\\
599.37	0.0054599648257405\\
599.38	0.00551280865972795\\
599.39	0.00556615771538178\\
599.4	0.00562001685053483\\
599.41	0.00567439096969272\\
599.42	0.00572928502448196\\
599.43	0.00578470401410212\\
599.44	0.00584065298578249\\
599.45	0.005897137035243\\
599.46	0.00595416130715971\\
599.47	0.00601173099563457\\
599.48	0.00606985134466986\\
599.49	0.0061285276486471\\
599.5	0.00618776525281052\\
599.51	0.00624756955375523\\
599.52	0.00630794599992004\\
599.53	0.006368900092085\\
599.54	0.00643043738387369\\
599.55	0.00649256348226045\\
599.56	0.00655528404808221\\
599.57	0.00661860479655557\\
599.58	0.00668253149779854\\
599.59	0.00674706997735744\\
599.6	0.00681222611673878\\
599.61	0.0068780058539463\\
599.62	0.00694441518402312\\
599.63	0.00701146015959912\\
599.64	0.00707914689144344\\
599.65	0.00714748154902253\\
599.66	0.00721647036106324\\
599.67	0.00728611961612158\\
599.68	0.00735643566315676\\
599.69	0.00742742491211079\\
599.7	0.00749909383449364\\
599.71	0.007571448963974\\
599.72	0.00764449689697572\\
599.73	0.0077182442932799\\
599.74	0.00779269787663286\\
599.75	0.00786786443535983\\
599.76	0.00794375082298462\\
599.77	0.00802036395885511\\
599.78	0.00809771082877486\\
599.79	0.00817579848564072\\
599.8	0.00825463405008649\\
599.81	0.00833422471113286\\
599.82	0.0084145777268435\\
599.83	0.00849570042498747\\
599.84	0.00857760020370798\\
599.85	0.00866028453219752\\
599.86	0.00874376095137954\\
599.87	0.00882803707459654\\
599.88	0.0089131205883049\\
599.89	0.00899901925277625\\
599.9	0.00908574090280562\\
599.91	0.00917329344842635\\
599.92	0.0092616848756319\\
599.93	0.00935092324710449\\
599.94	0.00944101670295079\\
599.95	0.00953197346144465\\
599.96	0.00962380181977693\\
599.97	0.00971651015481255\\
599.98	0.0098101069238547\\
599.99	0.00990460066541651\\
600	0.01\\
};
\addplot [color=black,solid,forget plot]
  table[row sep=crcr]{%
0.01	0\\
1.01	0\\
2.01	0\\
3.01	0\\
4.01	0\\
5.01	0\\
6.01	0\\
7.01	0\\
8.01	0\\
9.01	0\\
10.01	0\\
11.01	0\\
12.01	0\\
13.01	0\\
14.01	0\\
15.01	0\\
16.01	0\\
17.01	0\\
18.01	0\\
19.01	0\\
20.01	0\\
21.01	0\\
22.01	0\\
23.01	0\\
24.01	0\\
25.01	0\\
26.01	0\\
27.01	0\\
28.01	0\\
29.01	0\\
30.01	0\\
31.01	0\\
32.01	0\\
33.01	0\\
34.01	0\\
35.01	0\\
36.01	0\\
37.01	0\\
38.01	0\\
39.01	0\\
40.01	0\\
41.01	0\\
42.01	0\\
43.01	0\\
44.01	0\\
45.01	0\\
46.01	0\\
47.01	0\\
48.01	0\\
49.01	0\\
50.01	0\\
51.01	0\\
52.01	0\\
53.01	0\\
54.01	0\\
55.01	0\\
56.01	0\\
57.01	0\\
58.01	0\\
59.01	0\\
60.01	0\\
61.01	0\\
62.01	0\\
63.01	0\\
64.01	0\\
65.01	0\\
66.01	0\\
67.01	0\\
68.01	0\\
69.01	0\\
70.01	0\\
71.01	0\\
72.01	0\\
73.01	0\\
74.01	0\\
75.01	0\\
76.01	0\\
77.01	0\\
78.01	0\\
79.01	0\\
80.01	0\\
81.01	0\\
82.01	0\\
83.01	0\\
84.01	0\\
85.01	0\\
86.01	0\\
87.01	0\\
88.01	0\\
89.01	0\\
90.01	0\\
91.01	0\\
92.01	0\\
93.01	0\\
94.01	0\\
95.01	0\\
96.01	0\\
97.01	0\\
98.01	0\\
99.01	0\\
100.01	0\\
101.01	0\\
102.01	0\\
103.01	0\\
104.01	0\\
105.01	0\\
106.01	0\\
107.01	0\\
108.01	0\\
109.01	0\\
110.01	0\\
111.01	0\\
112.01	0\\
113.01	0\\
114.01	0\\
115.01	0\\
116.01	0\\
117.01	0\\
118.01	0\\
119.01	0\\
120.01	0\\
121.01	0\\
122.01	0\\
123.01	0\\
124.01	0\\
125.01	0\\
126.01	0\\
127.01	0\\
128.01	0\\
129.01	0\\
130.01	0\\
131.01	0\\
132.01	0\\
133.01	0\\
134.01	0\\
135.01	0\\
136.01	0\\
137.01	0\\
138.01	0\\
139.01	0\\
140.01	0\\
141.01	0\\
142.01	0\\
143.01	0\\
144.01	0\\
145.01	0\\
146.01	0\\
147.01	0\\
148.01	0\\
149.01	0\\
150.01	0\\
151.01	0\\
152.01	0\\
153.01	0\\
154.01	0\\
155.01	0\\
156.01	0\\
157.01	0\\
158.01	0\\
159.01	0\\
160.01	0\\
161.01	0\\
162.01	0\\
163.01	0\\
164.01	0\\
165.01	0\\
166.01	0\\
167.01	0\\
168.01	0\\
169.01	0\\
170.01	0\\
171.01	0\\
172.01	0\\
173.01	0\\
174.01	0\\
175.01	0\\
176.01	0\\
177.01	0\\
178.01	0\\
179.01	0\\
180.01	0\\
181.01	0\\
182.01	0\\
183.01	0\\
184.01	0\\
185.01	0\\
186.01	0\\
187.01	0\\
188.01	0\\
189.01	0\\
190.01	0\\
191.01	0\\
192.01	0\\
193.01	0\\
194.01	0\\
195.01	0\\
196.01	0\\
197.01	0\\
198.01	0\\
199.01	0\\
200.01	0\\
201.01	0\\
202.01	0\\
203.01	0\\
204.01	0\\
205.01	0\\
206.01	0\\
207.01	0\\
208.01	0\\
209.01	0\\
210.01	0\\
211.01	0\\
212.01	0\\
213.01	0\\
214.01	0\\
215.01	0\\
216.01	0\\
217.01	0\\
218.01	0\\
219.01	0\\
220.01	0\\
221.01	0\\
222.01	0\\
223.01	0\\
224.01	0\\
225.01	0\\
226.01	0\\
227.01	0\\
228.01	0\\
229.01	0\\
230.01	0\\
231.01	0\\
232.01	0\\
233.01	0\\
234.01	0\\
235.01	0\\
236.01	0\\
237.01	0\\
238.01	0\\
239.01	0\\
240.01	0\\
241.01	0\\
242.01	0\\
243.01	0\\
244.01	0\\
245.01	0\\
246.01	0\\
247.01	0\\
248.01	0\\
249.01	0\\
250.01	0\\
251.01	0\\
252.01	0\\
253.01	0\\
254.01	0\\
255.01	0\\
256.01	0\\
257.01	0\\
258.01	0\\
259.01	0\\
260.01	0\\
261.01	0\\
262.01	0\\
263.01	0\\
264.01	0\\
265.01	0\\
266.01	0\\
267.01	0\\
268.01	0\\
269.01	0\\
270.01	0\\
271.01	0\\
272.01	0\\
273.01	0\\
274.01	0\\
275.01	0\\
276.01	0\\
277.01	0\\
278.01	0\\
279.01	0\\
280.01	0\\
281.01	0\\
282.01	0\\
283.01	0\\
284.01	0\\
285.01	0\\
286.01	0\\
287.01	0\\
288.01	0\\
289.01	0\\
290.01	0\\
291.01	0\\
292.01	0\\
293.01	0\\
294.01	0\\
295.01	0\\
296.01	0\\
297.01	0\\
298.01	0\\
299.01	0\\
300.01	0\\
301.01	0\\
302.01	0\\
303.01	0\\
304.01	0\\
305.01	0\\
306.01	0\\
307.01	0\\
308.01	0\\
309.01	0\\
310.01	0\\
311.01	0\\
312.01	0\\
313.01	0\\
314.01	0\\
315.01	0\\
316.01	0\\
317.01	0\\
318.01	0\\
319.01	0\\
320.01	0\\
321.01	0\\
322.01	0\\
323.01	0\\
324.01	0\\
325.01	0\\
326.01	0\\
327.01	0\\
328.01	0\\
329.01	0\\
330.01	0\\
331.01	0\\
332.01	0\\
333.01	0\\
334.01	0\\
335.01	0\\
336.01	0\\
337.01	0\\
338.01	0\\
339.01	0\\
340.01	0\\
341.01	0\\
342.01	0\\
343.01	0\\
344.01	0\\
345.01	0\\
346.01	0\\
347.01	0\\
348.01	0\\
349.01	0\\
350.01	0\\
351.01	0\\
352.01	0\\
353.01	0\\
354.01	0\\
355.01	0\\
356.01	0\\
357.01	0\\
358.01	0\\
359.01	0\\
360.01	0\\
361.01	0\\
362.01	0\\
363.01	0\\
364.01	0\\
365.01	0\\
366.01	0\\
367.01	0\\
368.01	0\\
369.01	0\\
370.01	0\\
371.01	0\\
372.01	0\\
373.01	0\\
374.01	0\\
375.01	0\\
376.01	0\\
377.01	0\\
378.01	0\\
379.01	0\\
380.01	0\\
381.01	0\\
382.01	0\\
383.01	0\\
384.01	0\\
385.01	0\\
386.01	0\\
387.01	0\\
388.01	0\\
389.01	0\\
390.01	0\\
391.01	0\\
392.01	0\\
393.01	0\\
394.01	0\\
395.01	0\\
396.01	0\\
397.01	0\\
398.01	0\\
399.01	0\\
400.01	0\\
401.01	0\\
402.01	0\\
403.01	0\\
404.01	0\\
405.01	0\\
406.01	0\\
407.01	0\\
408.01	0\\
409.01	0\\
410.01	0\\
411.01	0\\
412.01	0\\
413.01	0\\
414.01	0\\
415.01	0\\
416.01	0\\
417.01	0\\
418.01	0\\
419.01	0\\
420.01	0\\
421.01	0\\
422.01	0\\
423.01	0\\
424.01	0\\
425.01	0\\
426.01	0\\
427.01	0\\
428.01	0\\
429.01	0\\
430.01	0\\
431.01	0\\
432.01	0\\
433.01	0\\
434.01	0\\
435.01	0\\
436.01	0\\
437.01	0\\
438.01	0\\
439.01	0\\
440.01	0\\
441.01	0\\
442.01	0\\
443.01	0\\
444.01	0\\
445.01	0\\
446.01	0\\
447.01	0\\
448.01	0\\
449.01	0\\
450.01	0\\
451.01	0\\
452.01	0\\
453.01	0\\
454.01	0\\
455.01	0\\
456.01	0\\
457.01	0\\
458.01	0\\
459.01	0\\
460.01	0\\
461.01	0\\
462.01	0\\
463.01	0\\
464.01	0\\
465.01	0\\
466.01	0\\
467.01	0\\
468.01	0\\
469.01	0\\
470.01	0\\
471.01	0\\
472.01	0\\
473.01	0\\
474.01	0\\
475.01	0\\
476.01	0\\
477.01	0\\
478.01	0\\
479.01	0\\
480.01	0\\
481.01	0\\
482.01	0\\
483.01	0\\
484.01	0\\
485.01	0\\
486.01	0\\
487.01	0\\
488.01	0\\
489.01	0\\
490.01	0\\
491.01	0\\
492.01	0\\
493.01	0\\
494.01	0\\
495.01	0\\
496.01	0\\
497.01	0\\
498.01	0\\
499.01	0\\
500.01	0\\
501.01	0\\
502.01	0\\
503.01	0\\
504.01	0\\
505.01	0\\
506.01	0\\
507.01	0\\
508.01	0\\
509.01	0\\
510.01	0\\
511.01	0\\
512.01	0\\
513.01	0\\
514.01	0\\
515.01	0\\
516.01	0\\
517.01	0\\
518.01	0\\
519.01	0\\
520.01	0\\
521.01	0\\
522.01	0\\
523.01	0\\
524.01	0\\
525.01	0\\
526.01	0\\
527.01	0\\
528.01	0\\
529.01	0\\
530.01	0\\
531.01	0\\
532.01	0\\
533.01	0\\
534.01	0\\
535.01	0\\
536.01	0\\
537.01	0\\
538.01	0\\
539.01	0\\
540.01	0\\
541.01	0\\
542.01	0\\
543.01	0\\
544.01	0\\
545.01	0\\
546.01	0\\
547.01	0\\
548.01	0\\
549.01	0\\
550.01	0\\
551.01	0\\
552.01	0\\
553.01	0\\
554.01	0\\
555.01	0\\
556.01	0\\
557.01	0\\
558.01	0\\
559.01	0\\
560.01	0\\
561.01	0\\
562.01	0\\
563.01	0\\
564.01	0\\
565.01	0\\
566.01	0\\
567.01	0\\
568.01	0\\
569.01	0\\
570.01	0\\
571.01	0\\
572.01	0\\
573.01	0\\
574.01	0\\
575.01	0\\
576.01	0\\
577.01	0\\
578.01	0\\
579.01	0\\
580.01	0\\
581.01	0\\
582.01	0\\
583.01	0\\
584.01	0\\
585.01	0\\
586.01	0\\
587.01	0\\
588.01	0\\
589.01	0\\
590.01	0\\
591.01	0\\
592.01	0\\
593.01	0\\
594.01	0\\
595.01	0\\
596.01	0\\
597.01	0.00048133368794398\\
598.01	0.00143832068185187\\
599.01	0.00385661257211624\\
599.02	0.00389414818835231\\
599.03	0.00393204146009084\\
599.04	0.00397029584140057\\
599.05	0.00400891481968844\\
599.06	0.00404790191602107\\
599.07	0.00408726068544944\\
599.08	0.00412699471733659\\
599.09	0.00416710763568849\\
599.1	0.00420760309948817\\
599.11	0.00424848480303298\\
599.12	0.00428975647627516\\
599.13	0.00433142188516569\\
599.14	0.00437348483200138\\
599.15	0.00441594915577539\\
599.16	0.00445881873253112\\
599.17	0.00450209747571946\\
599.18	0.0045457893365595\\
599.19	0.00458989830440273\\
599.2	0.00463442840710075\\
599.21	0.00467938371137651\\
599.22	0.0047247683231991\\
599.23	0.00477058638816223\\
599.24	0.00481684209186631\\
599.25	0.00486353966030412\\
599.26	0.00491068334955366\\
599.27	0.00495827744958435\\
599.28	0.00500632629158768\\
599.29	0.00505483424837287\\
599.3	0.00510380573476642\\
599.31	0.00515324520801536\\
599.32	0.00520315716819456\\
599.33	0.00525354615861774\\
599.34	0.00530441676625258\\
599.35	0.00535577362213972\\
599.36	0.00540762140181579\\
599.37	0.00545996482574053\\
599.38	0.00551280865972797\\
599.39	0.00556615771538181\\
599.4	0.00562001685053484\\
599.41	0.00567439096969274\\
599.42	0.00572928502448199\\
599.43	0.00578470401410215\\
599.44	0.00584065298578253\\
599.45	0.00589713703524306\\
599.46	0.00595416130715976\\
599.47	0.00601173099563462\\
599.48	0.00606985134466992\\
599.49	0.00612852764864716\\
599.5	0.00618776525281058\\
599.51	0.00624756955375529\\
599.52	0.00630794599992009\\
599.53	0.00636890009208503\\
599.54	0.00643043738387374\\
599.55	0.00649256348226047\\
599.56	0.00655528404808225\\
599.57	0.00661860479655563\\
599.58	0.0066825314977986\\
599.59	0.00674706997735748\\
599.6	0.0068122261167388\\
599.61	0.00687800585394632\\
599.62	0.00694441518402315\\
599.63	0.00701146015959914\\
599.64	0.00707914689144346\\
599.65	0.00714748154902254\\
599.66	0.00721647036106325\\
599.67	0.00728611961612159\\
599.68	0.00735643566315677\\
599.69	0.00742742491211079\\
599.7	0.00749909383449364\\
599.71	0.00757144896397401\\
599.72	0.00764449689697572\\
599.73	0.00771824429327989\\
599.74	0.00779269787663285\\
599.75	0.00786786443535983\\
599.76	0.00794375082298461\\
599.77	0.0080203639588551\\
599.78	0.00809771082877486\\
599.79	0.00817579848564071\\
599.8	0.00825463405008648\\
599.81	0.00833422471113285\\
599.82	0.00841457772684349\\
599.83	0.00849570042498746\\
599.84	0.00857760020370797\\
599.85	0.00866028453219752\\
599.86	0.00874376095137953\\
599.87	0.00882803707459654\\
599.88	0.0089131205883049\\
599.89	0.00899901925277625\\
599.9	0.00908574090280562\\
599.91	0.00917329344842635\\
599.92	0.0092616848756319\\
599.93	0.00935092324710449\\
599.94	0.00944101670295078\\
599.95	0.00953197346144464\\
599.96	0.00962380181977693\\
599.97	0.00971651015481255\\
599.98	0.0098101069238547\\
599.99	0.00990460066541651\\
600	0.01\\
};
\end{axis}
\end{tikzpicture}%
  \caption{Continuous Time}
\end{subfigure}%
\hfill%
\begin{subfigure}{.45\linewidth}
  \centering
  \setlength\figureheight{\linewidth} 
  \setlength\figurewidth{\linewidth}
  \tikzsetnextfilename{dp_dscr_z1}
  % This file was created by matlab2tikz.
%
%The latest updates can be retrieved from
%  http://www.mathworks.com/matlabcentral/fileexchange/22022-matlab2tikz-matlab2tikz
%where you can also make suggestions and rate matlab2tikz.
%
\definecolor{mycolor1}{rgb}{1.00000,0.00000,1.00000}%
%
\begin{tikzpicture}[trim axis left, trim axis right]

\begin{axis}[%
width=\figurewidth,
height=\figureheight,
at={(0\figurewidth,0\figureheight)},
scale only axis,
every outer x axis line/.append style={black},
every x tick label/.append style={font=\color{black}},
xmin=0,
xmax=100,
xlabel={Time},
every outer y axis line/.append style={black},
every y tick label/.append style={font=\color{black}},
ymin=0,
ymax=0.015,
ylabel={Depth $\delta$},
axis background/.style={fill=white},
title={Discrete Time\\$Z=(\rho = -1, \Delta S = -1)$},
axis x line*=bottom,
axis y line*=left,
]
\addplot [color=green,dashed]
  table[row sep=crcr]{%
1	0.0126967216220857\\
2	0.0126911766553508\\
3	0.0126854244957378\\
4	0.0126794571471091\\
5	0.0126732662900411\\
6	0.0126668432738205\\
7	0.0126601790839146\\
8	0.0126532642645715\\
9	0.0126460887824662\\
10	0.0126386414716695\\
11	0.0126309074598112\\
12	0.0126228534328581\\
13	0.0126144738803348\\
14	0.0126057825592762\\
15	0.0125967674054145\\
16	0.0125874158466489\\
17	0.0125777149221112\\
18	0.0125676518288635\\
19	0.0125572160964152\\
20	0.0125464065829753\\
21	0.012535255438371\\
22	0.0125238277676431\\
23	0.0125119439922694\\
24	0.0124995838925062\\
25	0.0124867253188147\\
26	0.0124733448453486\\
27	0.012459419125432\\
28	0.012444933178067\\
29	0.0124298672579279\\
30	0.0124141983529513\\
31	0.0123979000396031\\
32	0.0123809394392587\\
33	0.0123632758539379\\
34	0.0123448831206783\\
35	0.0123258029190969\\
36	0.0123063579176832\\
37	0.0122948431826702\\
38	0.0122826746106595\\
39	0.0122697489999676\\
40	0.0122559127763505\\
41	0.0122409423287981\\
42	0.0122247833367846\\
43	0.0122077090117757\\
44	0.0121896254167272\\
45	0.0121704206085886\\
46	0.0121499381712301\\
47	0.0121278682167617\\
48	0.0121033214534428\\
49	0.012057244947584\\
50	0.0119927062994268\\
51	0.0119258128569804\\
52	0.0118563553512363\\
53	0.0117841151637177\\
54	0.011708955006812\\
55	0.0116305529803687\\
56	0.0115208018328421\\
57	0.0113744009336203\\
58	0.0112246429079446\\
59	0.0110714809763842\\
60	0.0109144901049758\\
61	0.0107540938228196\\
62	0.0105907214355421\\
63	0.0104247831009525\\
64	0.0103389180097715\\
65	0.0102689495413488\\
66	0.0101986182069349\\
67	0.0101280884932016\\
68	0.0100581445773063\\
69	0.00998867805764777\\
70	0.00992023771067554\\
71	0.00985490987643991\\
72	0.00979055286639061\\
73	0.00972465253538295\\
74	0.00965715212588177\\
75	0.00958775863428604\\
76	0.00951625001316514\\
77	0.00944304654577188\\
78	0.0093676337122019\\
79	0.00929040728981676\\
80	0.00921125473484573\\
81	0.00912994292596067\\
82	0.00904545617036328\\
83	0.0089586369742139\\
84	0.00886976238995432\\
85	0.00877877634016461\\
86	0.00868545009930632\\
87	0.00858962774069853\\
88	0.00848937276806187\\
89	0.00838785675903304\\
90	0.00821118656248098\\
91	0.00801216764059995\\
92	0.00773569972584507\\
93	0.00725437032485447\\
94	0.00665687490612621\\
95	0.00613676242368112\\
96	0.00574362775386832\\
97	0.00505491021526543\\
98	0.00355084777151062\\
99	0\\
100	0\\
};
\addlegendentry{$q=-4$};

\addplot [color=mycolor1,dashed]
  table[row sep=crcr]{%
1	0.0129856045043362\\
2	0.0129822538654271\\
3	0.0129787746933184\\
4	0.0129751616878654\\
5	0.0129714092934589\\
6	0.0129675116766721\\
7	0.0129634626912123\\
8	0.0129592558149231\\
9	0.0129548839713748\\
10	0.0129503389645035\\
11	0.0129456098389892\\
12	0.012940681480507\\
13	0.0129355501669858\\
14	0.0129302165834978\\
15	0.0129246716083036\\
16	0.0129189056457788\\
17	0.0129129087727571\\
18	0.0129066712687333\\
19	0.0129001851521933\\
20	0.0128934479980573\\
21	0.0128864682679866\\
22	0.0128792467678925\\
23	0.012871707566661\\
24	0.0128638328128444\\
25	0.0128556033206079\\
26	0.0128469989321532\\
27	0.0128379985213246\\
28	0.0128285776587918\\
29	0.0128187088268875\\
30	0.0128083659511024\\
31	0.012797515380411\\
32	0.0127861180334234\\
33	0.0127741332109041\\
34	0.0127615242852429\\
35	0.0127482662811104\\
36	0.0127366827171888\\
37	0.0127190399444645\\
38	0.0127006520061875\\
39	0.0126814541108304\\
40	0.0126613711872214\\
41	0.0126403373095481\\
42	0.0126182723325326\\
43	0.0125951887515214\\
44	0.0125710562280958\\
45	0.0125457916001141\\
46	0.0125192740366137\\
47	0.0124913180328565\\
48	0.0124617207988576\\
49	0.0124306504650677\\
50	0.0123984429883102\\
51	0.0123648479623234\\
52	0.0123299879846532\\
53	0.0122933179431204\\
54	0.0122544667538672\\
55	0.0122125221004401\\
56	0.0121508932082245\\
57	0.0120667890621589\\
58	0.0119795987945139\\
59	0.0118911409837508\\
60	0.0118150951934112\\
61	0.0117359697918491\\
62	0.0116535008464507\\
63	0.0115668494192426\\
64	0.011425161894664\\
65	0.0112691603761261\\
66	0.0111090933800572\\
67	0.0109447517068695\\
68	0.0107760724444617\\
69	0.0106032322186129\\
70	0.0104266573272997\\
71	0.01024746461265\\
72	0.0101063155023793\\
73	0.010020043962939\\
74	0.00993220351405599\\
75	0.00984293956320838\\
76	0.00975326632229672\\
77	0.00966358591246109\\
78	0.00957552791947833\\
79	0.00948699342712069\\
80	0.00939843088438885\\
81	0.00931038240347798\\
82	0.00922020223285864\\
83	0.00912752453324444\\
84	0.00903245284317054\\
85	0.00893492163961726\\
86	0.00883485623418918\\
87	0.0087322177401472\\
88	0.00862691489199603\\
89	0.00851786573398347\\
90	0.00840566327603898\\
91	0.00829076007526614\\
92	0.00814755572284184\\
93	0.00793531516173855\\
94	0.00770050089829056\\
95	0.00715805987528115\\
96	0.00592456746978032\\
97	0.00505491021526543\\
98	0.00355084777151062\\
99	0\\
100	0\\
};
\addlegendentry{$q=-3$};

\addplot [color=red,dashed]
  table[row sep=crcr]{%
1	0.0133660041345742\\
2	0.0133649985425405\\
3	0.0133639521576256\\
4	0.0133628630839124\\
5	0.0133617293110373\\
6	0.0133605486986157\\
7	0.0133593189514349\\
8	0.0133580375734417\\
9	0.0133567017866217\\
10	0.0133553084353259\\
11	0.0133538541195436\\
12	0.0133523364007293\\
13	0.0133507528690506\\
14	0.0133491000120371\\
15	0.01334737409141\\
16	0.0133455711687318\\
17	0.0133436871912005\\
18	0.0133417181713269\\
19	0.013339660401462\\
20	0.013337510198378\\
21	0.0133352615354812\\
22	0.0133329022991876\\
23	0.0133304251252388\\
24	0.0133278219189346\\
25	0.0133250838125675\\
26	0.0133222010552851\\
27	0.0133191625390637\\
28	0.0133159558056849\\
29	0.0133125674314826\\
30	0.0133089832116995\\
31	0.0133051849132065\\
32	0.0133011505048681\\
33	0.0132968524947565\\
34	0.0132922486223346\\
35	0.01328725578988\\
36	0.0132800809359116\\
37	0.0132693409134395\\
38	0.0132581734571374\\
39	0.0132465486247589\\
40	0.0132344324925976\\
41	0.0132217924733293\\
42	0.0132085788742152\\
43	0.0131947883263311\\
44	0.0131803968337275\\
45	0.0131653605311704\\
46	0.0131496252304385\\
47	0.0131331326480878\\
48	0.0131158780985838\\
49	0.0130978752020875\\
50	0.0130791248421649\\
51	0.0130594825380877\\
52	0.0130386277888343\\
53	0.0130166419419974\\
54	0.0129932301314035\\
55	0.0129681276029774\\
56	0.0129413437312889\\
57	0.012912966270936\\
58	0.012882762385406\\
59	0.0128492213430169\\
60	0.0128037713468365\\
61	0.0127558045353237\\
62	0.0127047605146994\\
63	0.0126495225678913\\
64	0.0125607637022288\\
65	0.0124622749469665\\
66	0.0123611613224151\\
67	0.0122565881576478\\
68	0.0121477380418608\\
69	0.0120342752234703\\
70	0.0119195337674421\\
71	0.0118024553426227\\
72	0.011653872989352\\
73	0.011464738764733\\
74	0.0112707232604999\\
75	0.0110712846402284\\
76	0.010865535551453\\
77	0.0106762299032837\\
78	0.0104843448212728\\
79	0.0102896238788441\\
80	0.0100875909154579\\
81	0.00987756510301356\\
82	0.0097308864213279\\
83	0.00960224991070578\\
84	0.00947184311749152\\
85	0.00933914710917778\\
86	0.00920664389467955\\
87	0.00907652416718605\\
88	0.00894263834053335\\
89	0.00880463054234541\\
90	0.00866298575430099\\
91	0.00851835770472905\\
92	0.00837118372262871\\
93	0.00822205565507644\\
94	0.00807170093885098\\
95	0.00783250554120718\\
96	0.0074324720476095\\
97	0.00543592658335712\\
98	0.00355084777151062\\
99	0\\
100	0\\
};
\addlegendentry{$q=-2$};

\addplot [color=blue,dashed]
  table[row sep=crcr]{%
1	0.0135317146551453\\
2	0.0135315608261053\\
3	0.0135314005390358\\
4	0.0135312334767243\\
5	0.0135310593013548\\
6	0.0135308776511215\\
7	0.0135306881342496\\
8	0.0135304903186408\\
9	0.0135302837194055\\
10	0.0135300678132787\\
11	0.0135298421826273\\
12	0.0135296064071407\\
13	0.0135293599287123\\
14	0.013529102153335\\
15	0.013528832453455\\
16	0.0135285501769609\\
17	0.013528254666963\\
18	0.0135279452874482\\
19	0.0135276213979559\\
20	0.0135272820723854\\
21	0.0135269256598165\\
22	0.0135265510360669\\
23	0.0135261569698328\\
24	0.0135257421060612\\
25	0.0135253049431025\\
26	0.0135248438013243\\
27	0.0135243567981914\\
28	0.0135238418766312\\
29	0.0135232970190877\\
30	0.0135227196415019\\
31	0.013522106606347\\
32	0.0135214539712667\\
33	0.0135207560601477\\
34	0.0135200024793008\\
35	0.0135191666513325\\
36	0.0135172829937066\\
37	0.0135153213663433\\
38	0.0135132821840627\\
39	0.0135111594766569\\
40	0.013508946280691\\
41	0.0135066368013251\\
42	0.0135042284544917\\
43	0.0135017161848457\\
44	0.0134990929245518\\
45	0.0134963509555567\\
46	0.0134934822311073\\
47	0.0134904793381017\\
48	0.0134873357348835\\
49	0.0134840450438983\\
50	0.0134805862622556\\
51	0.0134769348974278\\
52	0.0134727995789019\\
53	0.0134682936817264\\
54	0.0134634607952326\\
55	0.0134582913927417\\
56	0.013452784758758\\
57	0.0134468901237203\\
58	0.0134405336742243\\
59	0.01343289578655\\
60	0.0134189018350896\\
61	0.013404150707504\\
62	0.0133884738742701\\
63	0.0133715682000753\\
64	0.0133537616925606\\
65	0.0133350339586812\\
66	0.0133142847501557\\
67	0.0132919259126887\\
68	0.0132683212132686\\
69	0.0132432959639189\\
70	0.0132141421005538\\
71	0.0131802680002577\\
72	0.0131288097559871\\
73	0.0130538307953416\\
74	0.0129760685627625\\
75	0.012895246920584\\
76	0.0128107313559607\\
77	0.0127083463508467\\
78	0.0125996771817469\\
79	0.0124844310082188\\
80	0.0123634245427691\\
81	0.0122360424318266\\
82	0.0120592949160788\\
83	0.0118621580300995\\
84	0.0116584140531364\\
85	0.0114463534785483\\
86	0.0112271122038623\\
87	0.0110012918727073\\
88	0.0107697666903376\\
89	0.0105268926754227\\
90	0.0102718492305835\\
91	0.0100037436489004\\
92	0.00972105195995043\\
93	0.00942146457752256\\
94	0.009101952469783\\
95	0.00875946745630175\\
96	0.0083888525844387\\
97	0.00758920978615654\\
98	0.00355084777151062\\
99	0\\
100	0\\
};
\addlegendentry{$q=-1$};

\addplot [color=black,solid]
  table[row sep=crcr]{%
1	0.00591981360054696\\
2	0.00591981360054696\\
3	0.00591981360054696\\
4	0.00591981360054696\\
5	0.00591981360054696\\
6	0.00591981360054696\\
7	0.00591981360054696\\
8	0.00591981360054696\\
9	0.00591981360054696\\
10	0.00591981360054696\\
11	0.00591981360054696\\
12	0.00591981360054696\\
13	0.00591981360054696\\
14	0.00591981360054696\\
15	0.00591981360054696\\
16	0.00591981360054696\\
17	0.00591981360054696\\
18	0.00591981360054696\\
19	0.00591981360054696\\
20	0.00591981360054696\\
21	0.00591981360054696\\
22	0.00591981360054696\\
23	0.00591981360054696\\
24	0.00591981360054696\\
25	0.00591981360054696\\
26	0.00591981360054696\\
27	0.00591981360054696\\
28	0.00591981360054696\\
29	0.00591981360054696\\
30	0.00591981360054696\\
31	0.00591981360054696\\
32	0.00591981360054696\\
33	0.00591981360054696\\
34	0.00591981360054696\\
35	0.00591981360054696\\
36	0.00591981360054696\\
37	0.00591981360054696\\
38	0.00591981360054696\\
39	0.00591981360054696\\
40	0.00591981360054696\\
41	0.00591981360054696\\
42	0.00591981360054696\\
43	0.00591981360054696\\
44	0.00591981360054696\\
45	0.00591981360054696\\
46	0.00591981360054696\\
47	0.00591981360054696\\
48	0.00591981360054696\\
49	0.00591981360054696\\
50	0.00591981360054696\\
51	0.00591981360054696\\
52	0.0059197314014602\\
53	0.00591957612551109\\
54	0.00591941017485126\\
55	0.00591923196648035\\
56	0.00591903973740182\\
57	0.00591883147966023\\
58	0.00591860481779524\\
59	0.00591835693701099\\
60	0.0059180844892575\\
61	0.00591778347486291\\
62	0.00591744916699088\\
63	0.00591707593995693\\
64	0.00591665713989317\\
65	0.00591618491490205\\
66	0.00596393855843097\\
67	0.00603036864143203\\
68	0.00610041656553486\\
69	0.0061743871334594\\
70	0.00625042468725859\\
71	0.00635396470808304\\
72	0.00646122841449285\\
73	0.00657204709940429\\
74	0.00672553666866284\\
75	0.00693282394483811\\
76	0.00714140644633385\\
77	0.00733859058735712\\
78	0.00753183746900076\\
79	0.00770503670347887\\
80	0.00788383015588474\\
81	0.00806847568100207\\
82	0.00823066202890321\\
83	0.00838615838767804\\
84	0.00854300538107573\\
85	0.00870117949536376\\
86	0.00886185907997313\\
87	0.00902167742403161\\
88	0.00918586905823183\\
89	0.00936316415379578\\
90	0.00955360283913768\\
91	0.00974697832940791\\
92	0.00994205264286296\\
93	0.0101405890941118\\
94	0.0103511869865303\\
95	0.0105897612397742\\
96	0.010890392895037\\
97	0.0113172670570267\\
98	0.0116898933452315\\
99	0\\
100	0\\
};
\addlegendentry{$q=0$};

\addplot [color=blue,solid]
  table[row sep=crcr]{%
1	0.00451614935682934\\
2	0.00452311676043263\\
3	0.00453027508233792\\
4	0.00453763279764905\\
5	0.00454520219945658\\
6	0.00455300549604212\\
7	0.00456108966975832\\
8	0.00456955742544878\\
9	0.00457859628509222\\
10	0.00458822014349924\\
11	0.00459810515859038\\
12	0.00460826524130698\\
13	0.00461872981492281\\
14	0.00462959478680276\\
15	0.00464124774693008\\
16	0.00465536313543588\\
17	0.00467930414929815\\
18	0.0047156224582261\\
19	0.00475305414923783\\
20	0.00479162545081955\\
21	0.00483132816510595\\
22	0.00487207179032189\\
23	0.00491360710969711\\
24	0.00495554809770774\\
25	0.00499842604340389\\
26	0.00504325978533359\\
27	0.00508970088284333\\
28	0.00513784254184666\\
29	0.00518778600467447\\
30	0.00523964153556477\\
31	0.00529352983872942\\
32	0.00534958458445824\\
33	0.00540795802037152\\
34	0.00546883551277045\\
35	0.00553247639970869\\
36	0.00559933305430107\\
37	0.00567040356050585\\
38	0.00574906251218784\\
39	0.00583787816885141\\
40	0.00592908614578806\\
41	0.00602278644817338\\
42	0.00611908733218572\\
43	0.00621809454499509\\
44	0.00631992047549828\\
45	0.0064246372639008\\
46	0.00653238053549258\\
47	0.00664329530107697\\
48	0.00675753619558864\\
49	0.00687526780032528\\
50	0.00699666755464488\\
51	0.00712194255786227\\
52	0.00725140172825184\\
53	0.00738571419701336\\
54	0.00752619676438501\\
55	0.00767734679906207\\
56	0.00783516490553262\\
57	0.008000939758594\\
58	0.00818375632458021\\
59	0.00837348572120451\\
60	0.0085681593084036\\
61	0.00876798883825665\\
62	0.00897314257728339\\
63	0.00918363852233112\\
64	0.00939904059078436\\
65	0.00961766273198416\\
66	0.00979911843910775\\
67	0.00995107288608682\\
68	0.010048972033477\\
69	0.0101472012981851\\
70	0.0102466420766812\\
71	0.0103272687637985\\
72	0.010407754383832\\
73	0.0104881556812313\\
74	0.0105709812055244\\
75	0.0106492339046446\\
76	0.0107264552869942\\
77	0.01080322383852\\
78	0.0108814839359471\\
79	0.0109617983024247\\
80	0.0110434235149282\\
81	0.0111220709407802\\
82	0.0112004869707467\\
83	0.0112800284837741\\
84	0.0113607313049829\\
85	0.0114424985291478\\
86	0.0115241054748618\\
87	0.0116073275602494\\
88	0.0116925866191128\\
89	0.0117844422105768\\
90	0.0118962795727799\\
91	0.0120265210753947\\
92	0.0121587847108071\\
93	0.0122936710395094\\
94	0.0124330125022277\\
95	0.0125820167352517\\
96	0.0127697443304968\\
97	0.0131086352216726\\
98	0.0135508477715106\\
99	0\\
100	0\\
};
\addlegendentry{$q=1$};

\addplot [color=red,solid]
  table[row sep=crcr]{%
1	0.00823461842574045\\
2	0.00826899602495442\\
3	0.00830442877152966\\
4	0.00834095967252183\\
5	0.00837863555996613\\
6	0.00841750975872967\\
7	0.00845765290518301\\
8	0.00849919332502241\\
9	0.00854249419452816\\
10	0.00858906396359684\\
11	0.00863990299031478\\
12	0.00869238162082222\\
13	0.00874655944120017\\
14	0.00880244889580168\\
15	0.00885983719858435\\
16	0.00891749908383751\\
17	0.0089697006586655\\
18	0.00901440876918852\\
19	0.00906043576208444\\
20	0.00910781177920156\\
21	0.00915653388374698\\
22	0.00920649599065296\\
23	0.00925726322100709\\
24	0.00930729853774357\\
25	0.00935113802307131\\
26	0.00938986627572017\\
27	0.00943282038810339\\
28	0.00947681869098303\\
29	0.00952186967365653\\
30	0.00956797914951497\\
31	0.00961514990469749\\
32	0.00966338147105382\\
33	0.00971267035956913\\
34	0.00976301173111001\\
35	0.00981440520254744\\
36	0.0098668713497011\\
37	0.00992048852035668\\
38	0.00997312244123629\\
39	0.0100211590770693\\
40	0.0100705380917287\\
41	0.0101212936933832\\
42	0.010173470373309\\
43	0.0102271425484395\\
44	0.0102823278688608\\
45	0.0103391539930373\\
46	0.0103974002909443\\
47	0.0104570996066606\\
48	0.0105182837235764\\
49	0.0105809817244718\\
50	0.0106452160441145\\
51	0.0107109919498944\\
52	0.0107782643963154\\
53	0.0108468095872655\\
54	0.0109171923145941\\
55	0.0109860994684566\\
56	0.0110570228370002\\
57	0.0111329976652417\\
58	0.0112159842069506\\
59	0.0113127764473596\\
60	0.0114130129655266\\
61	0.0115146276351316\\
62	0.0116175031699437\\
63	0.0117214496776731\\
64	0.0118261307886065\\
65	0.0119307526606845\\
66	0.0120203652489965\\
67	0.0120984408832417\\
68	0.0121568910226901\\
69	0.0122118012863447\\
70	0.0122652910119001\\
71	0.0123106732973323\\
72	0.0123549332729736\\
73	0.0123984456662247\\
74	0.0124414345932572\\
75	0.0124812997201541\\
76	0.0125194500415809\\
77	0.0125569664043024\\
78	0.0125943812824734\\
79	0.0126309404611539\\
80	0.012665963838849\\
81	0.0126994759494684\\
82	0.0127323389053315\\
83	0.0127647722281926\\
84	0.0127968511000606\\
85	0.0128286531811732\\
86	0.0128603301180877\\
87	0.0128963729173589\\
88	0.0129318836587637\\
89	0.0129675724876054\\
90	0.0130033415127081\\
91	0.0130390115143561\\
92	0.0130750839244785\\
93	0.0131135099546856\\
94	0.0131538072419449\\
95	0.0132031167059528\\
96	0.0132828620400256\\
97	0.0134008680318943\\
98	0.0135508477715106\\
99	0\\
100	0\\
};
\addlegendentry{$q=2$};

\addplot [color=mycolor1,solid]
  table[row sep=crcr]{%
1	0.0104070598822911\\
2	0.0104304807816848\\
3	0.0104545314997298\\
4	0.010479226888311\\
5	0.0105045825049643\\
6	0.0105306151605219\\
7	0.0105573425827522\\
8	0.0105847947376939\\
9	0.0106130709466891\\
10	0.0106426290787016\\
11	0.0106744578220093\\
12	0.010708476001213\\
13	0.0107432761490129\\
14	0.0107788359358436\\
15	0.0108150352469957\\
16	0.0108513302663978\\
17	0.0108853842988561\\
18	0.0109164722361298\\
19	0.0109482602322412\\
20	0.0109806318466408\\
21	0.0110135579561168\\
22	0.0110469730697013\\
23	0.011080699651873\\
24	0.011114173882034\\
25	0.011145388157608\\
26	0.0111741490791111\\
27	0.011201393028691\\
28	0.0112289339368627\\
29	0.011257015815042\\
30	0.0112856372065345\\
31	0.0113147950576296\\
32	0.0113444845167581\\
33	0.0113746987074325\\
34	0.0114054284472627\\
35	0.0114366617439412\\
36	0.0114683819974893\\
37	0.0115005581133852\\
38	0.0115322635464395\\
39	0.0115621491867184\\
40	0.0115928099751743\\
41	0.0116242661110847\\
42	0.011656543489315\\
43	0.01168970345219\\
44	0.0117248550277529\\
45	0.0117612297508009\\
46	0.0117978191438218\\
47	0.0118345701178826\\
48	0.0118714229622976\\
49	0.0119083102694157\\
50	0.0119451547256269\\
51	0.0119818630419612\\
52	0.0120183057633091\\
53	0.0120542430501936\\
54	0.0120896436841004\\
55	0.0121228176639475\\
56	0.0121552305302407\\
57	0.0121864443881354\\
58	0.0122167370122164\\
59	0.01224690152124\\
60	0.0122768848962182\\
61	0.0123066099007683\\
62	0.0123360050500266\\
63	0.0123648760476127\\
64	0.0123929406640282\\
65	0.0124202849765883\\
66	0.0124471178681301\\
67	0.0124732830497997\\
68	0.0124986575848265\\
69	0.0125224753421137\\
70	0.0125465845232579\\
71	0.0125712118834182\\
72	0.0125955234053651\\
73	0.0126196107228879\\
74	0.0126428678643362\\
75	0.0126673160897811\\
76	0.0126947312143728\\
77	0.0127224891406568\\
78	0.0127504433547237\\
79	0.012778096208618\\
80	0.0128056775984883\\
81	0.0128338097045545\\
82	0.0128624321267579\\
83	0.0128914557721687\\
84	0.0129209761633547\\
85	0.0129515363179389\\
86	0.0129843922586545\\
87	0.0130145080675542\\
88	0.0130448637247913\\
89	0.0130754225509084\\
90	0.01310614002025\\
91	0.0131370417925568\\
92	0.0131684013089337\\
93	0.0131992881190797\\
94	0.0132339377541053\\
95	0.0132837658672037\\
96	0.0133434148917366\\
97	0.0134129558428687\\
98	0.0135508477715106\\
99	0\\
100	0\\
};
\addlegendentry{$q=3$};

\addplot [color=green,solid]
  table[row sep=crcr]{%
1	0.0112173328374135\\
2	0.0112304770601278\\
3	0.0112439182131143\\
4	0.0112576571891033\\
5	0.01127169491468\\
6	0.0112860341128129\\
7	0.0113006833622851\\
8	0.0113156343059301\\
9	0.0113308631684719\\
10	0.0113463716299626\\
11	0.0113621611579808\\
12	0.0113782330923898\\
13	0.0113945884538718\\
14	0.0114112278659182\\
15	0.0114281513734847\\
16	0.0114453582960558\\
17	0.0114628742299139\\
18	0.0114807550685414\\
19	0.0114990214645766\\
20	0.0115176414336741\\
21	0.01153662032998\\
22	0.0115559632077397\\
23	0.0115756742070751\\
24	0.0115957546069088\\
25	0.0116161960754078\\
26	0.0116368416532018\\
27	0.0116569176878322\\
28	0.0116777533744845\\
29	0.0117002489592426\\
30	0.011722984948825\\
31	0.0117459461714951\\
32	0.0117691156809455\\
33	0.011792474633478\\
34	0.0118160021561575\\
35	0.0118396751971485\\
36	0.0118634683443689\\
37	0.0118873536627004\\
38	0.0119113147065166\\
39	0.0119353555472121\\
40	0.0119594389041357\\
41	0.011983522112995\\
42	0.0120075530995899\\
43	0.0120314523610817\\
44	0.012054465981151\\
45	0.0120770512355319\\
46	0.0120997330324277\\
47	0.0121224858909302\\
48	0.0121452824402504\\
49	0.0121680933715113\\
50	0.0121908873902304\\
51	0.0122136311461536\\
52	0.0122362886243086\\
53	0.0122588178504758\\
54	0.0122811820870626\\
55	0.0123033696464428\\
56	0.0123253090757543\\
57	0.0123465567831379\\
58	0.0123673593529313\\
59	0.0123881606288535\\
60	0.0124087928970967\\
61	0.0124294775489006\\
62	0.0124502172466953\\
63	0.0124709391635829\\
64	0.0124955518434812\\
65	0.012520604250692\\
66	0.0125454406786979\\
67	0.0125717908948519\\
68	0.0125979829530069\\
69	0.0126239699816152\\
70	0.0126488841755872\\
71	0.0126730572520101\\
72	0.012697528992541\\
73	0.0127223355983777\\
74	0.0127475049668732\\
75	0.0127714878551888\\
76	0.0127940838119219\\
77	0.012817096660328\\
78	0.0128407413392876\\
79	0.0128662424001927\\
80	0.0128927948367257\\
81	0.0129196990953362\\
82	0.0129469289363823\\
83	0.0129744655449398\\
84	0.0130023254225124\\
85	0.013030341472698\\
86	0.0130568081275538\\
87	0.0130818372586511\\
88	0.0131071398482185\\
89	0.0131326828575159\\
90	0.0131585197059812\\
91	0.0131841281185588\\
92	0.0132134166715202\\
93	0.013243972340567\\
94	0.0132770651881871\\
95	0.0133105595018936\\
96	0.0133500513908948\\
97	0.0134129558428687\\
98	0.0135508477715106\\
99	0\\
100	0\\
};
\addlegendentry{$q=4$};

\end{axis}
\end{tikzpicture}%
 
  \caption{Discrete Time}
\end{subfigure}\\
\vspace{1cm}
\begin{subfigure}{.45\linewidth}
  \centering
  \setlength\figureheight{\linewidth} 
  \setlength\figurewidth{\linewidth}
  \tikzsetnextfilename{dp_cts_nFPC_z1}
  % This file was created by matlab2tikz.
%
%The latest updates can be retrieved from
%  http://www.mathworks.com/matlabcentral/fileexchange/22022-matlab2tikz-matlab2tikz
%where you can also make suggestions and rate matlab2tikz.
%
\definecolor{mycolor1}{rgb}{1.00000,0.00000,1.00000}%
%
\begin{tikzpicture}

\begin{axis}[%
width=4.564in,
height=3.803in,
at={(1.067in,0.513in)},
scale only axis,
every outer x axis line/.append style={black},
every x tick label/.append style={font=\color{black}},
xmin=0,
xmax=100,
xlabel={Time},
every outer y axis line/.append style={black},
every y tick label/.append style={font=\color{black}},
ymin=0,
ymax=0.01,
ylabel={Depth $\delta$},
axis background/.style={fill=white},
title={Z=1},
axis x line*=bottom,
axis y line*=left,
legend style={legend cell align=left,align=left,draw=black}
]
\addplot [color=green,dashed,forget plot]
  table[row sep=crcr]{%
0.01	0\\
0.02	0\\
0.03	0\\
0.04	0\\
0.05	0\\
0.06	0\\
0.07	0\\
0.08	0\\
0.09	0\\
0.1	0\\
0.11	0\\
0.12	0\\
0.13	0\\
0.14	0\\
0.15	0\\
0.16	0\\
0.17	1.73472347597681e-18\\
0.18	1.73472347597681e-18\\
0.19	0\\
0.2	0\\
0.21	0\\
0.22	0\\
0.23	0\\
0.24	0\\
0.25	0\\
0.26	0\\
0.27	0\\
0.28	0\\
0.29	0\\
0.3	0\\
0.31	0\\
0.32	0\\
0.33	0\\
0.34	1.73472347597681e-18\\
0.35	1.73472347597681e-18\\
0.36	0\\
0.37	0\\
0.38	0\\
0.39	0\\
0.4	0\\
0.41	0\\
0.42	0\\
0.43	0\\
0.44	0\\
0.45	0\\
0.46	0\\
0.47	0\\
0.48	0\\
0.49	0\\
0.5	0\\
0.51	1.73472347597681e-18\\
0.52	1.73472347597681e-18\\
0.53	0\\
0.54	0\\
0.55	1.73472347597681e-18\\
0.56	0\\
0.57	0\\
0.58	0\\
0.59	0\\
0.6	1.73472347597681e-18\\
0.61	0\\
0.62	0\\
0.63	1.73472347597681e-18\\
0.64	1.73472347597681e-18\\
0.65	0\\
0.66	0\\
0.67	1.73472347597681e-18\\
0.68	0\\
0.69	0\\
0.7	0\\
0.71	0\\
0.72	0\\
0.73	0\\
0.74	0\\
0.75	0\\
0.76	0\\
0.77	1.73472347597681e-18\\
0.78	0\\
0.79	0\\
0.8	0\\
0.81	0\\
0.82	0\\
0.83	0\\
0.84	0\\
0.85	0\\
0.86	0\\
0.87	0\\
0.88	0\\
0.89	0\\
0.9	0\\
0.91	0\\
0.92	0\\
0.93	0\\
0.94	0\\
0.95	0\\
0.96	0\\
0.97	0\\
0.98	0\\
0.99	1.73472347597681e-18\\
1	0\\
1.01	1.73472347597681e-18\\
1.02	0\\
1.03	0\\
1.04	1.73472347597681e-18\\
1.05	0\\
1.06	0\\
1.07	0\\
1.08	0\\
1.09	0\\
1.1	0\\
1.11	0\\
1.12	0\\
1.13	1.73472347597681e-18\\
1.14	0\\
1.15	0\\
1.16	0\\
1.17	0\\
1.18	0\\
1.19	1.73472347597681e-18\\
1.2	0\\
1.21	0\\
1.22	1.73472347597681e-18\\
1.23	1.73472347597681e-18\\
1.24	0\\
1.25	1.73472347597681e-18\\
1.26	0\\
1.27	0\\
1.28	0\\
1.29	0\\
1.3	0\\
1.31	0\\
1.32	0\\
1.33	0\\
1.34	0\\
1.35	0\\
1.36	0\\
1.37	0\\
1.38	0\\
1.39	0\\
1.4	0\\
1.41	0\\
1.42	0\\
1.43	0\\
1.44	0\\
1.45	0\\
1.46	0\\
1.47	0\\
1.48	0\\
1.49	0\\
1.5	0\\
1.51	0\\
1.52	0\\
1.53	0\\
1.54	0\\
1.55	0\\
1.56	0\\
1.57	0\\
1.58	0\\
1.59	0\\
1.6	0\\
1.61	0\\
1.62	0\\
1.63	0\\
1.64	0\\
1.65	0\\
1.66	0\\
1.67	0\\
1.68	0\\
1.69	0\\
1.7	0\\
1.71	0\\
1.72	0\\
1.73	0\\
1.74	0\\
1.75	0\\
1.76	0\\
1.77	0\\
1.78	0\\
1.79	0\\
1.8	1.73472347597681e-18\\
1.81	0\\
1.82	0\\
1.83	0\\
1.84	0\\
1.85	0\\
1.86	0\\
1.87	0\\
1.88	0\\
1.89	0\\
1.9	0\\
1.91	0\\
1.92	0\\
1.93	0\\
1.94	0\\
1.95	0\\
1.96	1.73472347597681e-18\\
1.97	0\\
1.98	0\\
1.99	0\\
2	0\\
2.01	0\\
2.02	0\\
2.03	1.73472347597681e-18\\
2.04	0\\
2.05	1.73472347597681e-18\\
2.06	1.73472347597681e-18\\
2.07	0\\
2.08	0\\
2.09	1.73472347597681e-18\\
2.1	0\\
2.11	0\\
2.12	0\\
2.13	1.73472347597681e-18\\
2.14	0\\
2.15	0\\
2.16	0\\
2.17	0\\
2.18	0\\
2.19	0\\
2.2	0\\
2.21	0\\
2.22	0\\
2.23	0\\
2.24	1.73472347597681e-18\\
2.25	1.73472347597681e-18\\
2.26	0\\
2.27	0\\
2.28	0\\
2.29	0\\
2.3	1.73472347597681e-18\\
2.31	0\\
2.32	0\\
2.33	0\\
2.34	0\\
2.35	1.73472347597681e-18\\
2.36	0\\
2.37	0\\
2.38	0\\
2.39	0\\
2.4	0\\
2.41	0\\
2.42	0\\
2.43	0\\
2.44	0\\
2.45	1.73472347597681e-18\\
2.46	0\\
2.47	0\\
2.48	0\\
2.49	0\\
2.5	0\\
2.51	0\\
2.52	1.73472347597681e-18\\
2.53	0\\
2.54	0\\
2.55	1.73472347597681e-18\\
2.56	0\\
2.57	0\\
2.58	0\\
2.59	0\\
2.6	0\\
2.61	0\\
2.62	0\\
2.63	0\\
2.64	0\\
2.65	0\\
2.66	0\\
2.67	1.73472347597681e-18\\
2.68	0\\
2.69	0\\
2.7	0\\
2.71	0\\
2.72	0\\
2.73	0\\
2.74	0\\
2.75	0\\
2.76	0\\
2.77	0\\
2.78	0\\
2.79	0\\
2.8	0\\
2.81	0\\
2.82	0\\
2.83	0\\
2.84	0\\
2.85	0\\
2.86	0\\
2.87	0\\
2.88	0\\
2.89	0\\
2.9	0\\
2.91	0\\
2.92	0\\
2.93	0\\
2.94	0\\
2.95	0\\
2.96	1.73472347597681e-18\\
2.97	0\\
2.98	0\\
2.99	0\\
3	1.73472347597681e-18\\
3.01	0\\
3.02	0\\
3.03	0\\
3.04	0\\
3.05	0\\
3.06	0\\
3.07	0\\
3.08	0\\
3.09	0\\
3.1	0\\
3.11	0\\
3.12	0\\
3.13	1.73472347597681e-18\\
3.14	0\\
3.15	0\\
3.16	0\\
3.17	0\\
3.18	0\\
3.19	0\\
3.2	0\\
3.21	1.73472347597681e-18\\
3.22	0\\
3.23	1.73472347597681e-18\\
3.24	0\\
3.25	0\\
3.26	0\\
3.27	0\\
3.28	0\\
3.29	0\\
3.3	0\\
3.31	0\\
3.32	0\\
3.33	0\\
3.34	0\\
3.35	0\\
3.36	0\\
3.37	0\\
3.38	0\\
3.39	1.73472347597681e-18\\
3.4	0\\
3.41	1.73472347597681e-18\\
3.42	1.73472347597681e-18\\
3.43	0\\
3.44	1.73472347597681e-18\\
3.45	0\\
3.46	0\\
3.47	0\\
3.48	0\\
3.49	0\\
3.5	1.73472347597681e-18\\
3.51	0\\
3.52	0\\
3.53	0\\
3.54	0\\
3.55	0\\
3.56	0\\
3.57	0\\
3.58	0\\
3.59	1.73472347597681e-18\\
3.6	0\\
3.61	0\\
3.62	0\\
3.63	0\\
3.64	0\\
3.65	0\\
3.66	0\\
3.67	0\\
3.68	0\\
3.69	0\\
3.7	0\\
3.71	0\\
3.72	0\\
3.73	0\\
3.74	0\\
3.75	0\\
3.76	0\\
3.77	0\\
3.78	0\\
3.79	0\\
3.8	0\\
3.81	1.73472347597681e-18\\
3.82	0\\
3.83	0\\
3.84	0\\
3.85	0\\
3.86	0\\
3.87	0\\
3.88	0\\
3.89	0\\
3.9	0\\
3.91	0\\
3.92	0\\
3.93	0\\
3.94	0\\
3.95	0\\
3.96	0\\
3.97	0\\
3.98	1.73472347597681e-18\\
3.99	0\\
4	0\\
4.01	1.73472347597681e-18\\
4.02	0\\
4.03	0\\
4.04	0\\
4.05	1.73472347597681e-18\\
4.06	0\\
4.07	1.73472347597681e-18\\
4.08	0\\
4.09	0\\
4.1	0\\
4.11	0\\
4.12	0\\
4.13	0\\
4.14	0\\
4.15	1.73472347597681e-18\\
4.16	0\\
4.17	0\\
4.18	0\\
4.19	0\\
4.2	0\\
4.21	1.73472347597681e-18\\
4.22	0\\
4.23	0\\
4.24	0\\
4.25	0\\
4.26	1.73472347597681e-18\\
4.27	0\\
4.28	0\\
4.29	0\\
4.3	0\\
4.31	0\\
4.32	0\\
4.33	0\\
4.34	0\\
4.35	0\\
4.36	0\\
4.37	0\\
4.38	0\\
4.39	0\\
4.4	0\\
4.41	0\\
4.42	0\\
4.43	0\\
4.44	0\\
4.45	0\\
4.46	0\\
4.47	0\\
4.48	0\\
4.49	0\\
4.5	0\\
4.51	1.73472347597681e-18\\
4.52	0\\
4.53	0\\
4.54	0\\
4.55	0\\
4.56	0\\
4.57	0\\
4.58	0\\
4.59	0\\
4.6	0\\
4.61	0\\
4.62	1.73472347597681e-18\\
4.63	0\\
4.64	0\\
4.65	0\\
4.66	0\\
4.67	0\\
4.68	0\\
4.69	0\\
4.7	0\\
4.71	0\\
4.72	0\\
4.73	0\\
4.74	0\\
4.75	0\\
4.76	0\\
4.77	0\\
4.78	0\\
4.79	0\\
4.8	0\\
4.81	0\\
4.82	1.73472347597681e-18\\
4.83	0\\
4.84	0\\
4.85	1.73472347597681e-18\\
4.86	0\\
4.87	0\\
4.88	0\\
4.89	0\\
4.9	0\\
4.91	0\\
4.92	0\\
4.93	0\\
4.94	0\\
4.95	0\\
4.96	0\\
4.97	0\\
4.98	1.73472347597681e-18\\
4.99	0\\
5	0\\
5.01	0\\
5.02	0\\
5.03	0\\
5.04	0\\
5.05	0\\
5.06	1.73472347597681e-18\\
5.07	1.73472347597681e-18\\
5.08	0\\
5.09	0\\
5.1	0\\
5.11	1.73472347597681e-18\\
5.12	0\\
5.13	0\\
5.14	0\\
5.15	0\\
5.16	0\\
5.17	0\\
5.18	0\\
5.19	0\\
5.2	0\\
5.21	0\\
5.22	0\\
5.23	1.73472347597681e-18\\
5.24	0\\
5.25	0\\
5.26	0\\
5.27	0\\
5.28	0\\
5.29	0\\
5.3	0\\
5.31	0\\
5.32	1.73472347597681e-18\\
5.33	0\\
5.34	1.73472347597681e-18\\
5.35	1.73472347597681e-18\\
5.36	0\\
5.37	0\\
5.38	0\\
5.39	0\\
5.4	0\\
5.41	0\\
5.42	0\\
5.43	0\\
5.44	0\\
5.45	0\\
5.46	1.73472347597681e-18\\
5.47	0\\
5.48	0\\
5.49	1.73472347597681e-18\\
5.5	0\\
5.51	0\\
5.52	0\\
5.53	0\\
5.54	0\\
5.55	0\\
5.56	0\\
5.57	1.73472347597681e-18\\
5.58	0\\
5.59	0\\
5.6	0\\
5.61	0\\
5.62	0\\
5.63	1.73472347597681e-18\\
5.64	0\\
5.65	0\\
5.66	0\\
5.67	0\\
5.68	0\\
5.69	1.73472347597681e-18\\
5.7	0\\
5.71	1.73472347597681e-18\\
5.72	0\\
5.73	0\\
5.74	0\\
5.75	0\\
5.76	1.73472347597681e-18\\
5.77	0\\
5.78	0\\
5.79	0\\
5.8	0\\
5.81	0\\
5.82	0\\
5.83	1.73472347597681e-18\\
5.84	0\\
5.85	0\\
5.86	0\\
5.87	0\\
5.88	0\\
5.89	0\\
5.9	0\\
5.91	0\\
5.92	0\\
5.93	0\\
5.94	0\\
5.95	0\\
5.96	0\\
5.97	0\\
5.98	0\\
5.99	0\\
6	0\\
6.01	0\\
6.02	0\\
6.03	0\\
6.04	1.73472347597681e-18\\
6.05	0\\
6.06	0\\
6.07	0\\
6.08	0\\
6.09	0\\
6.1	0\\
6.11	1.73472347597681e-18\\
6.12	0\\
6.13	0\\
6.14	0\\
6.15	0\\
6.16	0\\
6.17	1.73472347597681e-18\\
6.18	0\\
6.19	1.73472347597681e-18\\
6.2	0\\
6.21	0\\
6.22	0\\
6.23	0\\
6.24	0\\
6.25	1.73472347597681e-18\\
6.26	0\\
6.27	0\\
6.28	1.73472347597681e-18\\
6.29	0\\
6.3	0\\
6.31	0\\
6.32	0\\
6.33	1.73472347597681e-18\\
6.34	0\\
6.35	1.73472347597681e-18\\
6.36	1.73472347597681e-18\\
6.37	0\\
6.38	1.73472347597681e-18\\
6.39	0\\
6.4	0\\
6.41	0\\
6.42	1.73472347597681e-18\\
6.43	0\\
6.44	0\\
6.45	0\\
6.46	1.73472347597681e-18\\
6.47	0\\
6.48	1.73472347597681e-18\\
6.49	0\\
6.5	0\\
6.51	0\\
6.52	0\\
6.53	0\\
6.54	0\\
6.55	0\\
6.56	1.73472347597681e-18\\
6.57	0\\
6.58	0\\
6.59	0\\
6.6	0\\
6.61	1.73472347597681e-18\\
6.62	0\\
6.63	0\\
6.64	0\\
6.65	0\\
6.66	0\\
6.67	0\\
6.68	0\\
6.69	0\\
6.7	0\\
6.71	0\\
6.72	0\\
6.73	0\\
6.74	0\\
6.75	0\\
6.76	0\\
6.77	0\\
6.78	0\\
6.79	0\\
6.8	0\\
6.81	0\\
6.82	0\\
6.83	0\\
6.84	0\\
6.85	0\\
6.86	1.73472347597681e-18\\
6.87	0\\
6.88	0\\
6.89	0\\
6.9	0\\
6.91	0\\
6.92	0\\
6.93	0\\
6.94	0\\
6.95	0\\
6.96	0\\
6.97	0\\
6.98	0\\
6.99	0\\
7	0\\
7.01	0\\
7.02	0\\
7.03	1.73472347597681e-18\\
7.04	0\\
7.05	0\\
7.06	0\\
7.07	0\\
7.08	0\\
7.09	0\\
7.1	0\\
7.11	0\\
7.12	0\\
7.13	0\\
7.14	0\\
7.15	0\\
7.16	0\\
7.17	0\\
7.18	0\\
7.19	1.73472347597681e-18\\
7.2	0\\
7.21	0\\
7.22	0\\
7.23	0\\
7.24	0\\
7.25	0\\
7.26	0\\
7.27	1.73472347597681e-18\\
7.28	0\\
7.29	0\\
7.3	0\\
7.31	0\\
7.32	0\\
7.33	0\\
7.34	0\\
7.35	0\\
7.36	0\\
7.37	1.73472347597681e-18\\
7.38	0\\
7.39	0\\
7.4	0\\
7.41	0\\
7.42	0\\
7.43	0\\
7.44	0\\
7.45	0\\
7.46	0\\
7.47	0\\
7.48	0\\
7.49	0\\
7.5	0\\
7.51	0\\
7.52	0\\
7.53	0\\
7.54	0\\
7.55	1.73472347597681e-18\\
7.56	1.73472347597681e-18\\
7.57	0\\
7.58	1.73472347597681e-18\\
7.59	0\\
7.6	0\\
7.61	0\\
7.62	0\\
7.63	0\\
7.64	0\\
7.65	0\\
7.66	0\\
7.67	0\\
7.68	0\\
7.69	0\\
7.7	0\\
7.71	0\\
7.72	0\\
7.73	0\\
7.74	0\\
7.75	0\\
7.76	0\\
7.77	0\\
7.78	0\\
7.79	0\\
7.8	0\\
7.81	0\\
7.82	0\\
7.83	0\\
7.84	0\\
7.85	1.73472347597681e-18\\
7.86	0\\
7.87	0\\
7.88	0\\
7.89	0\\
7.9	0\\
7.91	0\\
7.92	0\\
7.93	0\\
7.94	0\\
7.95	1.73472347597681e-18\\
7.96	0\\
7.97	0\\
7.98	0\\
7.99	0\\
8	0\\
8.01	0\\
8.02	0\\
8.03	1.73472347597681e-18\\
8.04	1.73472347597681e-18\\
8.05	1.73472347597681e-18\\
8.06	0\\
8.07	1.73472347597681e-18\\
8.08	0\\
8.09	1.73472347597681e-18\\
8.1	0\\
8.11	0\\
8.12	0\\
8.13	1.73472347597681e-18\\
8.14	0\\
8.15	0\\
8.16	0\\
8.17	0\\
8.18	0\\
8.19	0\\
8.2	0\\
8.21	0\\
8.22	0\\
8.23	0\\
8.24	0\\
8.25	0\\
8.26	0\\
8.27	0\\
8.28	0\\
8.29	0\\
8.3	0\\
8.31	0\\
8.32	0\\
8.33	0\\
8.34	0\\
8.35	0\\
8.36	0\\
8.37	0\\
8.38	0\\
8.39	1.73472347597681e-18\\
8.4	0\\
8.41	0\\
8.42	0\\
8.43	0\\
8.44	0\\
8.45	1.73472347597681e-18\\
8.46	0\\
8.47	0\\
8.48	0\\
8.49	0\\
8.5	0\\
8.51	0\\
8.52	0\\
8.53	0\\
8.54	0\\
8.55	0\\
8.56	0\\
8.57	0\\
8.58	0\\
8.59	0\\
8.6	0\\
8.61	0\\
8.62	0\\
8.63	0\\
8.64	1.73472347597681e-18\\
8.65	0\\
8.66	0\\
8.67	0\\
8.68	1.73472347597681e-18\\
8.69	0\\
8.7	0\\
8.71	0\\
8.72	0\\
8.73	0\\
8.74	0\\
8.75	0\\
8.76	0\\
8.77	0\\
8.78	0\\
8.79	1.73472347597681e-18\\
8.8	0\\
8.81	0\\
8.82	0\\
8.83	0\\
8.84	1.73472347597681e-18\\
8.85	0\\
8.86	1.73472347597681e-18\\
8.87	0\\
8.88	0\\
8.89	1.73472347597681e-18\\
8.9	0\\
8.91	0\\
8.92	0\\
8.93	0\\
8.94	0\\
8.95	0\\
8.96	0\\
8.97	0\\
8.98	0\\
8.99	0\\
9	0\\
9.01	0\\
9.02	0\\
9.03	1.73472347597681e-18\\
9.04	0\\
9.05	0\\
9.06	0\\
9.07	0\\
9.08	0\\
9.09	0\\
9.1	1.73472347597681e-18\\
9.11	0\\
9.12	0\\
9.13	0\\
9.14	0\\
9.15	1.73472347597681e-18\\
9.16	0\\
9.17	1.73472347597681e-18\\
9.18	0\\
9.19	1.73472347597681e-18\\
9.2	0\\
9.21	0\\
9.22	1.73472347597681e-18\\
9.23	0\\
9.24	0\\
9.25	1.73472347597681e-18\\
9.26	1.73472347597681e-18\\
9.27	0\\
9.28	0\\
9.29	0\\
9.3	0\\
9.31	1.73472347597681e-18\\
9.32	0\\
9.33	1.73472347597681e-18\\
9.34	0\\
9.35	0\\
9.36	0\\
9.37	0\\
9.38	0\\
9.39	1.73472347597681e-18\\
9.4	0\\
9.41	1.73472347597681e-18\\
9.42	0\\
9.43	0\\
9.44	0\\
9.45	0\\
9.46	0\\
9.47	0\\
9.48	0\\
9.49	0\\
9.5	0\\
9.51	1.73472347597681e-18\\
9.52	0\\
9.53	0\\
9.54	0\\
9.55	1.73472347597681e-18\\
9.56	0\\
9.57	0\\
9.58	0\\
9.59	0\\
9.6	0\\
9.61	0\\
9.62	0\\
9.63	0\\
9.64	1.73472347597681e-18\\
9.65	0\\
9.66	0\\
9.67	0\\
9.68	1.73472347597681e-18\\
9.69	0\\
9.7	0\\
9.71	0\\
9.72	1.73472347597681e-18\\
9.73	0\\
9.74	0\\
9.75	0\\
9.76	0\\
9.77	0\\
9.78	1.73472347597681e-18\\
9.79	0\\
9.8	0\\
9.81	1.73472347597681e-18\\
9.82	0\\
9.83	0\\
9.84	0\\
9.85	0\\
9.86	0\\
9.87	1.73472347597681e-18\\
9.88	0\\
9.89	0\\
9.9	1.73472347597681e-18\\
9.91	0\\
9.92	0\\
9.93	0\\
9.94	0\\
9.95	0\\
9.96	0\\
9.97	0\\
9.98	1.73472347597681e-18\\
9.99	0\\
10	0\\
10.01	0\\
10.02	0\\
10.03	0\\
10.04	0\\
10.05	0\\
10.06	0\\
10.07	0\\
10.08	0\\
10.09	0\\
10.1	0\\
10.11	0\\
10.12	0\\
10.13	0\\
10.14	0\\
10.15	0\\
10.16	0\\
10.17	0\\
10.18	0\\
10.19	0\\
10.2	0\\
10.21	0\\
10.22	0\\
10.23	0\\
10.24	0\\
10.25	0\\
10.26	0\\
10.27	0\\
10.28	0\\
10.29	0\\
10.3	0\\
10.31	0\\
10.32	0\\
10.33	0\\
10.34	0\\
10.35	0\\
10.36	1.73472347597681e-18\\
10.37	1.73472347597681e-18\\
10.38	0\\
10.39	0\\
10.4	0\\
10.41	0\\
10.42	1.73472347597681e-18\\
10.43	1.73472347597681e-18\\
10.44	0\\
10.45	0\\
10.46	0\\
10.47	0\\
10.48	0\\
10.49	0\\
10.5	0\\
10.51	0\\
10.52	0\\
10.53	0\\
10.54	0\\
10.55	0\\
10.56	0\\
10.57	0\\
10.58	0\\
10.59	1.73472347597681e-18\\
10.6	0\\
10.61	1.73472347597681e-18\\
10.62	0\\
10.63	0\\
10.64	0\\
10.65	0\\
10.66	0\\
10.67	0\\
10.68	0\\
10.69	0\\
10.7	0\\
10.71	0\\
10.72	0\\
10.73	0\\
10.74	0\\
10.75	0\\
10.76	1.73472347597681e-18\\
10.77	1.73472347597681e-18\\
10.78	0\\
10.79	1.73472347597681e-18\\
10.8	0\\
10.81	0\\
10.82	0\\
10.83	0\\
10.84	0\\
10.85	0\\
10.86	0\\
10.87	0\\
10.88	1.73472347597681e-18\\
10.89	0\\
10.9	1.73472347597681e-18\\
10.91	0\\
10.92	0\\
10.93	0\\
10.94	0\\
10.95	0\\
10.96	0\\
10.97	0\\
10.98	0\\
10.99	0\\
11	0\\
11.01	0\\
11.02	0\\
11.03	0\\
11.04	0\\
11.05	0\\
11.06	0\\
11.07	0\\
11.08	0\\
11.09	0\\
11.1	0\\
11.11	1.73472347597681e-18\\
11.12	0\\
11.13	0\\
11.14	0\\
11.15	0\\
11.16	1.73472347597681e-18\\
11.17	0\\
11.18	0\\
11.19	1.73472347597681e-18\\
11.2	0\\
11.21	0\\
11.22	0\\
11.23	0\\
11.24	1.73472347597681e-18\\
11.25	0\\
11.26	0\\
11.27	0\\
11.28	0\\
11.29	0\\
11.3	0\\
11.31	0\\
11.32	0\\
11.33	0\\
11.34	0\\
11.35	1.73472347597681e-18\\
11.36	0\\
11.37	0\\
11.38	0\\
11.39	1.73472347597681e-18\\
11.4	0\\
11.41	1.73472347597681e-18\\
11.42	0\\
11.43	0\\
11.44	0\\
11.45	0\\
11.46	0\\
11.47	0\\
11.48	0\\
11.49	0\\
11.5	0\\
11.51	0\\
11.52	0\\
11.53	0\\
11.54	0\\
11.55	0\\
11.56	0\\
11.57	0\\
11.58	1.73472347597681e-18\\
11.59	1.73472347597681e-18\\
11.6	0\\
11.61	0\\
11.62	0\\
11.63	1.73472347597681e-18\\
11.64	0\\
11.65	0\\
11.66	0\\
11.67	0\\
11.68	0\\
11.69	1.73472347597681e-18\\
11.7	0\\
11.71	0\\
11.72	0\\
11.73	0\\
11.74	0\\
11.75	0\\
11.76	0\\
11.77	1.73472347597681e-18\\
11.78	0\\
11.79	0\\
11.8	0\\
11.81	0\\
11.82	0\\
11.83	0\\
11.84	0\\
11.85	0\\
11.86	0\\
11.87	0\\
11.88	0\\
11.89	0\\
11.9	0\\
11.91	1.73472347597681e-18\\
11.92	0\\
11.93	0\\
11.94	1.73472347597681e-18\\
11.95	0\\
11.96	0\\
11.97	0\\
11.98	0\\
11.99	0\\
12	0\\
12.01	0\\
12.02	0\\
12.03	0\\
12.04	1.73472347597681e-18\\
12.05	0\\
12.06	0\\
12.07	0\\
12.08	0\\
12.09	1.73472347597681e-18\\
12.1	0\\
12.11	1.73472347597681e-18\\
12.12	0\\
12.13	0\\
12.14	0\\
12.15	0\\
12.16	0\\
12.17	1.73472347597681e-18\\
12.18	1.73472347597681e-18\\
12.19	1.73472347597681e-18\\
12.2	0\\
12.21	0\\
12.22	0\\
12.23	0\\
12.24	0\\
12.25	1.73472347597681e-18\\
12.26	0\\
12.27	0\\
12.28	1.73472347597681e-18\\
12.29	0\\
12.3	0\\
12.31	0\\
12.32	0\\
12.33	0\\
12.34	0\\
12.35	1.73472347597681e-18\\
12.36	0\\
12.37	0\\
12.38	1.73472347597681e-18\\
12.39	0\\
12.4	0\\
12.41	0\\
12.42	0\\
12.43	0\\
12.44	0\\
12.45	0\\
12.46	0\\
12.47	1.73472347597681e-18\\
12.48	0\\
12.49	0\\
12.5	0\\
12.51	0\\
12.52	0\\
12.53	0\\
12.54	0\\
12.55	0\\
12.56	0\\
12.57	0\\
12.58	0\\
12.59	0\\
12.6	0\\
12.61	0\\
12.62	0\\
12.63	0\\
12.64	0\\
12.65	0\\
12.66	0\\
12.67	0\\
12.68	0\\
12.69	0\\
12.7	0\\
12.71	0\\
12.72	1.73472347597681e-18\\
12.73	0\\
12.74	0\\
12.75	0\\
12.76	0\\
12.77	0\\
12.78	0\\
12.79	0\\
12.8	0\\
12.81	1.73472347597681e-18\\
12.82	0\\
12.83	0\\
12.84	1.73472347597681e-18\\
12.85	0\\
12.86	1.73472347597681e-18\\
12.87	0\\
12.88	0\\
12.89	0\\
12.9	0\\
12.91	0\\
12.92	0\\
12.93	0\\
12.94	0\\
12.95	1.73472347597681e-18\\
12.96	0\\
12.97	0\\
12.98	0\\
12.99	0\\
13	0\\
13.01	0\\
13.02	0\\
13.03	0\\
13.04	1.73472347597681e-18\\
13.05	0\\
13.06	0\\
13.07	1.73472347597681e-18\\
13.08	0\\
13.09	0\\
13.1	0\\
13.11	0\\
13.12	0\\
13.13	0\\
13.14	0\\
13.15	0\\
13.16	0\\
13.17	0\\
13.18	0\\
13.19	0\\
13.2	0\\
13.21	0\\
13.22	0\\
13.23	0\\
13.24	0\\
13.25	1.73472347597681e-18\\
13.26	0\\
13.27	0\\
13.28	0\\
13.29	0\\
13.3	0\\
13.31	1.73472347597681e-18\\
13.32	0\\
13.33	1.73472347597681e-18\\
13.34	1.73472347597681e-18\\
13.35	0\\
13.36	1.73472347597681e-18\\
13.37	0\\
13.38	1.73472347597681e-18\\
13.39	0\\
13.4	0\\
13.41	0\\
13.42	0\\
13.43	0\\
13.44	0\\
13.45	0\\
13.46	0\\
13.47	1.73472347597681e-18\\
13.48	0\\
13.49	0\\
13.5	0\\
13.51	0\\
13.52	0\\
13.53	0\\
13.54	0\\
13.55	0\\
13.56	0\\
13.57	0\\
13.58	0\\
13.59	0\\
13.6	0\\
13.61	1.73472347597681e-18\\
13.62	0\\
13.63	1.73472347597681e-18\\
13.64	0\\
13.65	0\\
13.66	1.73472347597681e-18\\
13.67	1.73472347597681e-18\\
13.68	0\\
13.69	0\\
13.7	1.73472347597681e-18\\
13.71	0\\
13.72	1.73472347597681e-18\\
13.73	0\\
13.74	1.73472347597681e-18\\
13.75	0\\
13.76	0\\
13.77	0\\
13.78	0\\
13.79	1.73472347597681e-18\\
13.8	0\\
13.81	1.73472347597681e-18\\
13.82	1.73472347597681e-18\\
13.83	0\\
13.84	0\\
13.85	0\\
13.86	0\\
13.87	0\\
13.88	0\\
13.89	1.73472347597681e-18\\
13.9	0\\
13.91	0\\
13.92	1.73472347597681e-18\\
13.93	0\\
13.94	0\\
13.95	0\\
13.96	0\\
13.97	0\\
13.98	0\\
13.99	1.73472347597681e-18\\
14	1.73472347597681e-18\\
14.01	1.73472347597681e-18\\
14.02	0\\
14.03	0\\
14.04	0\\
14.05	0\\
14.06	0\\
14.07	0\\
14.08	0\\
14.09	0\\
14.1	1.73472347597681e-18\\
14.11	0\\
14.12	0\\
14.13	0\\
14.14	0\\
14.15	0\\
14.16	0\\
14.17	0\\
14.18	0\\
14.19	1.73472347597681e-18\\
14.2	0\\
14.21	0\\
14.22	0\\
14.23	0\\
14.24	0\\
14.25	0\\
14.26	0\\
14.27	0\\
14.28	0\\
14.29	0\\
14.3	0\\
14.31	0\\
14.32	0\\
14.33	1.73472347597681e-18\\
14.34	0\\
14.35	0\\
14.36	0\\
14.37	0\\
14.38	0\\
14.39	1.73472347597681e-18\\
14.4	0\\
14.41	1.73472347597681e-18\\
14.42	0\\
14.43	0\\
14.44	0\\
14.45	0\\
14.46	1.73472347597681e-18\\
14.47	1.73472347597681e-18\\
14.48	0\\
14.49	0\\
14.5	0\\
14.51	0\\
14.52	0\\
14.53	0\\
14.54	0\\
14.55	0\\
14.56	0\\
14.57	0\\
14.58	0\\
14.59	0\\
14.6	0\\
14.61	0\\
14.62	0\\
14.63	1.73472347597681e-18\\
14.64	0\\
14.65	0\\
14.66	0\\
14.67	0\\
14.68	0\\
14.69	0\\
14.7	1.73472347597681e-18\\
14.71	0\\
14.72	1.73472347597681e-18\\
14.73	1.73472347597681e-18\\
14.74	0\\
14.75	0\\
14.76	0\\
14.77	0\\
14.78	0\\
14.79	0\\
14.8	0\\
14.81	0\\
14.82	0\\
14.83	0\\
14.84	1.73472347597681e-18\\
14.85	0\\
14.86	0\\
14.87	0\\
14.88	0\\
14.89	0\\
14.9	0\\
14.91	1.73472347597681e-18\\
14.92	0\\
14.93	0\\
14.94	0\\
14.95	1.73472347597681e-18\\
14.96	1.73472347597681e-18\\
14.97	0\\
14.98	0\\
14.99	0\\
15	0\\
15.01	1.73472347597681e-18\\
15.02	0\\
15.03	0\\
15.04	0\\
15.05	1.73472347597681e-18\\
15.06	0\\
15.07	0\\
15.08	0\\
15.09	0\\
15.1	0\\
15.11	0\\
15.12	0\\
15.13	0\\
15.14	0\\
15.15	0\\
15.16	0\\
15.17	0\\
15.18	0\\
15.19	0\\
15.2	1.73472347597681e-18\\
15.21	0\\
15.22	0\\
15.23	0\\
15.24	1.73472347597681e-18\\
15.25	1.73472347597681e-18\\
15.26	0\\
15.27	0\\
15.28	0\\
15.29	0\\
15.3	0\\
15.31	0\\
15.32	0\\
15.33	1.73472347597681e-18\\
15.34	0\\
15.35	0\\
15.36	0\\
15.37	0\\
15.38	0\\
15.39	0\\
15.4	0\\
15.41	0\\
15.42	0\\
15.43	0\\
15.44	0\\
15.45	0\\
15.46	0\\
15.47	0\\
15.48	0\\
15.49	0\\
15.5	0\\
15.51	0\\
15.52	0\\
15.53	0\\
15.54	0\\
15.55	0\\
15.56	0\\
15.57	0\\
15.58	1.73472347597681e-18\\
15.59	0\\
15.6	0\\
15.61	0\\
15.62	0\\
15.63	1.73472347597681e-18\\
15.64	0\\
15.65	0\\
15.66	0\\
15.67	0\\
15.68	1.73472347597681e-18\\
15.69	0\\
15.7	0\\
15.71	0\\
15.72	0\\
15.73	0\\
15.74	0\\
15.75	1.73472347597681e-18\\
15.76	0\\
15.77	0\\
15.78	0\\
15.79	0\\
15.8	0\\
15.81	0\\
15.82	0\\
15.83	0\\
15.84	0\\
15.85	0\\
15.86	0\\
15.87	0\\
15.88	0\\
15.89	0\\
15.9	0\\
15.91	0\\
15.92	0\\
15.93	0\\
15.94	0\\
15.95	0\\
15.96	0\\
15.97	0\\
15.98	0\\
15.99	0\\
16	1.73472347597681e-18\\
16.01	0\\
16.02	0\\
16.03	0\\
16.04	0\\
16.05	0\\
16.06	0\\
16.07	0\\
16.08	1.73472347597681e-18\\
16.09	0\\
16.1	0\\
16.11	0\\
16.12	0\\
16.13	0\\
16.14	0\\
16.15	0\\
16.16	0\\
16.17	0\\
16.18	0\\
16.19	0\\
16.2	0\\
16.21	0\\
16.22	0\\
16.23	0\\
16.24	0\\
16.25	0\\
16.26	0\\
16.27	0\\
16.28	0\\
16.29	0\\
16.3	0\\
16.31	0\\
16.32	0\\
16.33	0\\
16.34	0\\
16.35	1.73472347597681e-18\\
16.36	0\\
16.37	1.73472347597681e-18\\
16.38	0\\
16.39	0\\
16.4	0\\
16.41	0\\
16.42	0\\
16.43	0\\
16.44	0\\
16.45	0\\
16.46	0\\
16.47	0\\
16.48	0\\
16.49	0\\
16.5	0\\
16.51	0\\
16.52	0\\
16.53	0\\
16.54	0\\
16.55	0\\
16.56	0\\
16.57	0\\
16.58	0\\
16.59	0\\
16.6	0\\
16.61	0\\
16.62	0\\
16.63	0\\
16.64	1.73472347597681e-18\\
16.65	1.73472347597681e-18\\
16.66	0\\
16.67	0\\
16.68	1.73472347597681e-18\\
16.69	0\\
16.7	0\\
16.71	0\\
16.72	0\\
16.73	0\\
16.74	0\\
16.75	0\\
16.76	1.73472347597681e-18\\
16.77	0\\
16.78	1.73472347597681e-18\\
16.79	0\\
16.8	0\\
16.81	0\\
16.82	0\\
16.83	1.73472347597681e-18\\
16.84	1.73472347597681e-18\\
16.85	0\\
16.86	1.73472347597681e-18\\
16.87	0\\
16.88	0\\
16.89	1.73472347597681e-18\\
16.9	0\\
16.91	0\\
16.92	0\\
16.93	1.73472347597681e-18\\
16.94	0\\
16.95	1.73472347597681e-18\\
16.96	0\\
16.97	0\\
16.98	0\\
16.99	1.73472347597681e-18\\
17	1.73472347597681e-18\\
17.01	1.73472347597681e-18\\
17.02	0\\
17.03	0\\
17.04	0\\
17.05	0\\
17.06	0\\
17.07	0\\
17.08	0\\
17.09	1.73472347597681e-18\\
17.1	0\\
17.11	0\\
17.12	0\\
17.13	1.73472347597681e-18\\
17.14	0\\
17.15	0\\
17.16	0\\
17.17	1.73472347597681e-18\\
17.18	0\\
17.19	0\\
17.2	1.73472347597681e-18\\
17.21	0\\
17.22	1.73472347597681e-18\\
17.23	0\\
17.24	0\\
17.25	0\\
17.26	0\\
17.27	0\\
17.28	0\\
17.29	0\\
17.3	0\\
17.31	0\\
17.32	0\\
17.33	0\\
17.34	0\\
17.35	0\\
17.36	0\\
17.37	0\\
17.38	0\\
17.39	0\\
17.4	0\\
17.41	1.73472347597681e-18\\
17.42	1.73472347597681e-18\\
17.43	0\\
17.44	0\\
17.45	0\\
17.46	1.73472347597681e-18\\
17.47	0\\
17.48	0\\
17.49	0\\
17.5	0\\
17.51	0\\
17.52	0\\
17.53	0\\
17.54	0\\
17.55	0\\
17.56	0\\
17.57	0\\
17.58	0\\
17.59	0\\
17.6	0\\
17.61	0\\
17.62	0\\
17.63	0\\
17.64	0\\
17.65	0\\
17.66	0\\
17.67	0\\
17.68	0\\
17.69	0\\
17.7	0\\
17.71	0\\
17.72	0\\
17.73	0\\
17.74	0\\
17.75	0\\
17.76	0\\
17.77	0\\
17.78	0\\
17.79	1.73472347597681e-18\\
17.8	0\\
17.81	1.73472347597681e-18\\
17.82	0\\
17.83	1.73472347597681e-18\\
17.84	0\\
17.85	0\\
17.86	0\\
17.87	1.73472347597681e-18\\
17.88	0\\
17.89	0\\
17.9	0\\
17.91	0\\
17.92	1.73472347597681e-18\\
17.93	0\\
17.94	0\\
17.95	0\\
17.96	0\\
17.97	1.73472347597681e-18\\
17.98	0\\
17.99	0\\
18	0\\
18.01	0\\
18.02	0\\
18.03	0\\
18.04	0\\
18.05	0\\
18.06	0\\
18.07	0\\
18.08	0\\
18.09	0\\
18.1	0\\
18.11	1.73472347597681e-18\\
18.12	0\\
18.13	0\\
18.14	0\\
18.15	0\\
18.16	0\\
18.17	0\\
18.18	0\\
18.19	0\\
18.2	1.73472347597681e-18\\
18.21	0\\
18.22	0\\
18.23	0\\
18.24	0\\
18.25	0\\
18.26	0\\
18.27	1.73472347597681e-18\\
18.28	0\\
18.29	0\\
18.3	0\\
18.31	0\\
18.32	0\\
18.33	0\\
18.34	0\\
18.35	1.73472347597681e-18\\
18.36	1.73472347597681e-18\\
18.37	0\\
18.38	0\\
18.39	0\\
18.4	0\\
18.41	0\\
18.42	0\\
18.43	1.73472347597681e-18\\
18.44	0\\
18.45	0\\
18.46	0\\
18.47	0\\
18.48	0\\
18.49	0\\
18.5	0\\
18.51	0\\
18.52	0\\
18.53	0\\
18.54	0\\
18.55	0\\
18.56	0\\
18.57	0\\
18.58	0\\
18.59	0\\
18.6	0\\
18.61	0\\
18.62	0\\
18.63	0\\
18.64	0\\
18.65	0\\
18.66	0\\
18.67	0\\
18.68	0\\
18.69	0\\
18.7	0\\
18.71	0\\
18.72	0\\
18.73	1.73472347597681e-18\\
18.74	0\\
18.75	0\\
18.76	0\\
18.77	0\\
18.78	1.73472347597681e-18\\
18.79	0\\
18.8	0\\
18.81	0\\
18.82	0\\
18.83	0\\
18.84	0\\
18.85	0\\
18.86	0\\
18.87	0\\
18.88	0\\
18.89	0\\
18.9	0\\
18.91	0\\
18.92	0\\
18.93	0\\
18.94	0\\
18.95	0\\
18.96	0\\
18.97	0\\
18.98	0\\
18.99	0\\
19	0\\
19.01	1.73472347597681e-18\\
19.02	0\\
19.03	0\\
19.04	1.73472347597681e-18\\
19.05	0\\
19.06	0\\
19.07	1.73472347597681e-18\\
19.08	0\\
19.09	0\\
19.1	0\\
19.11	0\\
19.12	1.73472347597681e-18\\
19.13	0\\
19.14	0\\
19.15	0\\
19.16	0\\
19.17	0\\
19.18	0\\
19.19	1.73472347597681e-18\\
19.2	0\\
19.21	0\\
19.22	0\\
19.23	0\\
19.24	0\\
19.25	0\\
19.26	0\\
19.27	0\\
19.28	0\\
19.29	1.73472347597681e-18\\
19.3	0\\
19.31	0\\
19.32	0\\
19.33	0\\
19.34	0\\
19.35	0\\
19.36	0\\
19.37	0\\
19.38	0\\
19.39	0\\
19.4	0\\
19.41	0\\
19.42	0\\
19.43	0\\
19.44	0\\
19.45	0\\
19.46	0\\
19.47	0\\
19.48	0\\
19.49	1.73472347597681e-18\\
19.5	0\\
19.51	0\\
19.52	0\\
19.53	0\\
19.54	0\\
19.55	0\\
19.56	0\\
19.57	0\\
19.58	0\\
19.59	0\\
19.6	0\\
19.61	0\\
19.62	1.73472347597681e-18\\
19.63	1.73472347597681e-18\\
19.64	0\\
19.65	0\\
19.66	1.73472347597681e-18\\
19.67	0\\
19.68	0\\
19.69	0\\
19.7	0\\
19.71	0\\
19.72	0\\
19.73	0\\
19.74	0\\
19.75	0\\
19.76	0\\
19.77	0\\
19.78	0\\
19.79	0\\
19.8	0\\
19.81	1.73472347597681e-18\\
19.82	1.73472347597681e-18\\
19.83	0\\
19.84	0\\
19.85	0\\
19.86	1.73472347597681e-18\\
19.87	0\\
19.88	0\\
19.89	0\\
19.9	0\\
19.91	0\\
19.92	0\\
19.93	0\\
19.94	1.73472347597681e-18\\
19.95	0\\
19.96	0\\
19.97	0\\
19.98	0\\
19.99	0\\
20	0\\
20.01	0\\
20.02	1.73472347597681e-18\\
20.03	0\\
20.04	0\\
20.05	0\\
20.06	0\\
20.07	0\\
20.08	0\\
20.09	1.73472347597681e-18\\
20.1	0\\
20.11	0\\
20.12	0\\
20.13	1.73472347597681e-18\\
20.14	0\\
20.15	0\\
20.16	0\\
20.17	0\\
20.18	0\\
20.19	0\\
20.2	0\\
20.21	0\\
20.22	1.73472347597681e-18\\
20.23	0\\
20.24	0\\
20.25	0\\
20.26	0\\
20.27	0\\
20.28	0\\
20.29	0\\
20.3	0\\
20.31	0\\
20.32	0\\
20.33	0\\
20.34	0\\
20.35	0\\
20.36	0\\
20.37	0\\
20.38	1.73472347597681e-18\\
20.39	0\\
20.4	0\\
20.41	0\\
20.42	1.73472347597681e-18\\
20.43	0\\
20.44	1.73472347597681e-18\\
20.45	1.73472347597681e-18\\
20.46	0\\
20.47	0\\
20.48	0\\
20.49	0\\
20.5	0\\
20.51	0\\
20.52	0\\
20.53	0\\
20.54	0\\
20.55	0\\
20.56	0\\
20.57	0\\
20.58	0\\
20.59	0\\
20.6	1.73472347597681e-18\\
20.61	0\\
20.62	0\\
20.63	0\\
20.64	1.73472347597681e-18\\
20.65	1.73472347597681e-18\\
20.66	1.73472347597681e-18\\
20.67	1.73472347597681e-18\\
20.68	0\\
20.69	0\\
20.7	0\\
20.71	0\\
20.72	0\\
20.73	0\\
20.74	1.73472347597681e-18\\
20.75	0\\
20.76	0\\
20.77	1.73472347597681e-18\\
20.78	0\\
20.79	0\\
20.8	0\\
20.81	0\\
20.82	0\\
20.83	0\\
20.84	1.73472347597681e-18\\
20.85	0\\
20.86	0\\
20.87	0\\
20.88	0\\
20.89	0\\
20.9	0\\
20.91	0\\
20.92	0\\
20.93	0\\
20.94	0\\
20.95	0\\
20.96	0\\
20.97	0\\
20.98	0\\
20.99	0\\
21	1.73472347597681e-18\\
21.01	1.73472347597681e-18\\
21.02	0\\
21.03	0\\
21.04	1.73472347597681e-18\\
21.05	0\\
21.06	0\\
21.07	0\\
21.08	0\\
21.09	0\\
21.1	0\\
21.11	0\\
21.12	0\\
21.13	0\\
21.14	0\\
21.15	0\\
21.16	1.73472347597681e-18\\
21.17	0\\
21.18	0\\
21.19	0\\
21.2	0\\
21.21	1.73472347597681e-18\\
21.22	0\\
21.23	0\\
21.24	0\\
21.25	1.73472347597681e-18\\
21.26	1.73472347597681e-18\\
21.27	1.73472347597681e-18\\
21.28	0\\
21.29	0\\
21.3	0\\
21.31	0\\
21.32	1.73472347597681e-18\\
21.33	0\\
21.34	0\\
21.35	0\\
21.36	0\\
21.37	0\\
21.38	0\\
21.39	1.73472347597681e-18\\
21.4	0\\
21.41	0\\
21.42	0\\
21.43	0\\
21.44	0\\
21.45	0\\
21.46	1.73472347597681e-18\\
21.47	0\\
21.48	0\\
21.49	0\\
21.5	0\\
21.51	0\\
21.52	0\\
21.53	0\\
21.54	0\\
21.55	0\\
21.56	0\\
21.57	1.73472347597681e-18\\
21.58	0\\
21.59	0\\
21.6	0\\
21.61	0\\
21.62	0\\
21.63	0\\
21.64	0\\
21.65	0\\
21.66	0\\
21.67	1.73472347597681e-18\\
21.68	0\\
21.69	0\\
21.7	1.73472347597681e-18\\
21.71	0\\
21.72	0\\
21.73	1.73472347597681e-18\\
21.74	0\\
21.75	0\\
21.76	0\\
21.77	0\\
21.78	1.73472347597681e-18\\
21.79	0\\
21.8	0\\
21.81	0\\
21.82	0\\
21.83	0\\
21.84	0\\
21.85	0\\
21.86	0\\
21.87	0\\
21.88	1.73472347597681e-18\\
21.89	0\\
21.9	0\\
21.91	0\\
21.92	1.73472347597681e-18\\
21.93	0\\
21.94	0\\
21.95	0\\
21.96	0\\
21.97	0\\
21.98	0\\
21.99	0\\
22	0\\
22.01	0\\
22.02	0\\
22.03	0\\
22.04	0\\
22.05	0\\
22.06	0\\
22.07	0\\
22.08	0\\
22.09	0\\
22.1	1.73472347597681e-18\\
22.11	0\\
22.12	1.73472347597681e-18\\
22.13	0\\
22.14	1.73472347597681e-18\\
22.15	1.73472347597681e-18\\
22.16	0\\
22.17	0\\
22.18	0\\
22.19	0\\
22.2	0\\
22.21	0\\
22.22	0\\
22.23	0\\
22.24	0\\
22.25	1.73472347597681e-18\\
22.26	1.73472347597681e-18\\
22.27	0\\
22.28	1.73472347597681e-18\\
22.29	0\\
22.3	0\\
22.31	0\\
22.32	0\\
22.33	0\\
22.34	0\\
22.35	0\\
22.36	0\\
22.37	0\\
22.38	0\\
22.39	0\\
22.4	0\\
22.41	0\\
22.42	1.73472347597681e-18\\
22.43	0\\
22.44	0\\
22.45	0\\
22.46	1.73472347597681e-18\\
22.47	1.73472347597681e-18\\
22.48	1.73472347597681e-18\\
22.49	0\\
22.5	0\\
22.51	0\\
22.52	0\\
22.53	0\\
22.54	0\\
22.55	0\\
22.56	0\\
22.57	0\\
22.58	0\\
22.59	0\\
22.6	0\\
22.61	0\\
22.62	0\\
22.63	0\\
22.64	0\\
22.65	1.73472347597681e-18\\
22.66	0\\
22.67	0\\
22.68	1.73472347597681e-18\\
22.69	1.73472347597681e-18\\
22.7	0\\
22.71	0\\
22.72	1.73472347597681e-18\\
22.73	0\\
22.74	0\\
22.75	0\\
22.76	0\\
22.77	0\\
22.78	0\\
22.79	0\\
22.8	0\\
22.81	0\\
22.82	0\\
22.83	0\\
22.84	0\\
22.85	0\\
22.86	0\\
22.87	0\\
22.88	0\\
22.89	0\\
22.9	0\\
22.91	0\\
22.92	0\\
22.93	0\\
22.94	0\\
22.95	0\\
22.96	0\\
22.97	0\\
22.98	0\\
22.99	0\\
23	0\\
23.01	0\\
23.02	0\\
23.03	0\\
23.04	0\\
23.05	0\\
23.06	0\\
23.07	1.73472347597681e-18\\
23.08	0\\
23.09	0\\
23.1	0\\
23.11	0\\
23.12	0\\
23.13	1.73472347597681e-18\\
23.14	0\\
23.15	0\\
23.16	0\\
23.17	0\\
23.18	0\\
23.19	0\\
23.2	0\\
23.21	1.73472347597681e-18\\
23.22	0\\
23.23	0\\
23.24	0\\
23.25	0\\
23.26	0\\
23.27	0\\
23.28	0\\
23.29	0\\
23.3	0\\
23.31	0\\
23.32	0\\
23.33	0\\
23.34	0\\
23.35	0\\
23.36	0\\
23.37	0\\
23.38	1.73472347597681e-18\\
23.39	0\\
23.4	0\\
23.41	0\\
23.42	0\\
23.43	0\\
23.44	0\\
23.45	0\\
23.46	0\\
23.47	0\\
23.48	0\\
23.49	0\\
23.5	0\\
23.51	0\\
23.52	0\\
23.53	0\\
23.54	0\\
23.55	1.73472347597681e-18\\
23.56	0\\
23.57	0\\
23.58	0\\
23.59	0\\
23.6	0\\
23.61	0\\
23.62	0\\
23.63	0\\
23.64	0\\
23.65	0\\
23.66	0\\
23.67	0\\
23.68	0\\
23.69	0\\
23.7	0\\
23.71	0\\
23.72	0\\
23.73	0\\
23.74	0\\
23.75	0\\
23.76	0\\
23.77	0\\
23.78	0\\
23.79	0\\
23.8	1.73472347597681e-18\\
23.81	0\\
23.82	0\\
23.83	0\\
23.84	0\\
23.85	0\\
23.86	1.73472347597681e-18\\
23.87	0\\
23.88	0\\
23.89	0\\
23.9	0\\
23.91	0\\
23.92	0\\
23.93	0\\
23.94	0\\
23.95	0\\
23.96	0\\
23.97	0\\
23.98	0\\
23.99	0\\
24	0\\
24.01	1.73472347597681e-18\\
24.02	0\\
24.03	0\\
24.04	0\\
24.05	0\\
24.06	0\\
24.07	0\\
24.08	0\\
24.09	0\\
24.1	0\\
24.11	0\\
24.12	0\\
24.13	0\\
24.14	0\\
24.15	0\\
24.16	0\\
24.17	1.73472347597681e-18\\
24.18	1.73472347597681e-18\\
24.19	0\\
24.2	0\\
24.21	0\\
24.22	1.73472347597681e-18\\
24.23	0\\
24.24	0\\
24.25	0\\
24.26	0\\
24.27	0\\
24.28	0\\
24.29	0\\
24.3	0\\
24.31	0\\
24.32	0\\
24.33	1.73472347597681e-18\\
24.34	0\\
24.35	0\\
24.36	0\\
24.37	0\\
24.38	0\\
24.39	0\\
24.4	1.73472347597681e-18\\
24.41	1.73472347597681e-18\\
24.42	0\\
24.43	0\\
24.44	0\\
24.45	0\\
24.46	0\\
24.47	1.73472347597681e-18\\
24.48	1.73472347597681e-18\\
24.49	0\\
24.5	0\\
24.51	0\\
24.52	0\\
24.53	0\\
24.54	0\\
24.55	0\\
24.56	0\\
24.57	0\\
24.58	0\\
24.59	0\\
24.6	0\\
24.61	0\\
24.62	0\\
24.63	0\\
24.64	0\\
24.65	1.73472347597681e-18\\
24.66	0\\
24.67	1.73472347597681e-18\\
24.68	0\\
24.69	0\\
24.7	0\\
24.71	0\\
24.72	0\\
24.73	0\\
24.74	1.73472347597681e-18\\
24.75	0\\
24.76	0\\
24.77	0\\
24.78	0\\
24.79	0\\
24.8	0\\
24.81	0\\
24.82	0\\
24.83	0\\
24.84	0\\
24.85	0\\
24.86	0\\
24.87	0\\
24.88	0\\
24.89	0\\
24.9	0\\
24.91	0\\
24.92	1.73472347597681e-18\\
24.93	0\\
24.94	0\\
24.95	0\\
24.96	0\\
24.97	0\\
24.98	0\\
24.99	0\\
25	0\\
25.01	1.73472347597681e-18\\
25.02	1.73472347597681e-18\\
25.03	0\\
25.04	1.73472347597681e-18\\
25.05	0\\
25.06	0\\
25.07	0\\
25.08	0\\
25.09	0\\
25.1	1.73472347597681e-18\\
25.11	1.73472347597681e-18\\
25.12	0\\
25.13	0\\
25.14	0\\
25.15	0\\
25.16	1.73472347597681e-18\\
25.17	0\\
25.18	1.73472347597681e-18\\
25.19	0\\
25.2	0\\
25.21	0\\
25.22	0\\
25.23	0\\
25.24	1.73472347597681e-18\\
25.25	1.73472347597681e-18\\
25.26	0\\
25.27	0\\
25.28	0\\
25.29	0\\
25.3	0\\
25.31	0\\
25.32	0\\
25.33	0\\
25.34	0\\
25.35	0\\
25.36	0\\
25.37	1.73472347597681e-18\\
25.38	0\\
25.39	0\\
25.4	0\\
25.41	1.73472347597681e-18\\
25.42	0\\
25.43	0\\
25.44	0\\
25.45	0\\
25.46	0\\
25.47	0\\
25.48	0\\
25.49	0\\
25.5	0\\
25.51	0\\
25.52	0\\
25.53	0\\
25.54	0\\
25.55	0\\
25.56	0\\
25.57	0\\
25.58	0\\
25.59	0\\
25.6	0\\
25.61	0\\
25.62	0\\
25.63	0\\
25.64	0\\
25.65	0\\
25.66	0\\
25.67	0\\
25.68	0\\
25.69	0\\
25.7	0\\
25.71	0\\
25.72	1.73472347597681e-18\\
25.73	0\\
25.74	1.73472347597681e-18\\
25.75	0\\
25.76	1.73472347597681e-18\\
25.77	0\\
25.78	1.73472347597681e-18\\
25.79	0\\
25.8	0\\
25.81	0\\
25.82	0\\
25.83	0\\
25.84	0\\
25.85	0\\
25.86	0\\
25.87	1.73472347597681e-18\\
25.88	0\\
25.89	0\\
25.9	0\\
25.91	0\\
25.92	0\\
25.93	0\\
25.94	0\\
25.95	0\\
25.96	1.73472347597681e-18\\
25.97	0\\
25.98	0\\
25.99	0\\
26	0\\
26.01	0\\
26.02	0\\
26.03	0\\
26.04	0\\
26.05	0\\
26.06	1.73472347597681e-18\\
26.07	0\\
26.08	0\\
26.09	1.73472347597681e-18\\
26.1	0\\
26.11	0\\
26.12	0\\
26.13	0\\
26.14	0\\
26.15	0\\
26.16	0\\
26.17	0\\
26.18	0\\
26.19	0\\
26.2	0\\
26.21	0\\
26.22	0\\
26.23	0\\
26.24	0\\
26.25	0\\
26.26	0\\
26.27	0\\
26.28	0\\
26.29	0\\
26.3	0\\
26.31	0\\
26.32	0\\
26.33	0\\
26.34	0\\
26.35	0\\
26.36	0\\
26.37	0\\
26.38	0\\
26.39	1.73472347597681e-18\\
26.4	0\\
26.41	0\\
26.42	0\\
26.43	0\\
26.44	0\\
26.45	0\\
26.46	0\\
26.47	0\\
26.48	0\\
26.49	0\\
26.5	0\\
26.51	0\\
26.52	0\\
26.53	0\\
26.54	0\\
26.55	0\\
26.56	0\\
26.57	0\\
26.58	0\\
26.59	0\\
26.6	0\\
26.61	0\\
26.62	0\\
26.63	0\\
26.64	0\\
26.65	0\\
26.66	0\\
26.67	0\\
26.68	0\\
26.69	0\\
26.7	0\\
26.71	1.73472347597681e-18\\
26.72	0\\
26.73	0\\
26.74	0\\
26.75	0\\
26.76	0\\
26.77	0\\
26.78	0\\
26.79	0\\
26.8	1.73472347597681e-18\\
26.81	0\\
26.82	0\\
26.83	0\\
26.84	1.73472347597681e-18\\
26.85	0\\
26.86	1.73472347597681e-18\\
26.87	0\\
26.88	0\\
26.89	0\\
26.9	0\\
26.91	1.73472347597681e-18\\
26.92	0\\
26.93	0\\
26.94	0\\
26.95	0\\
26.96	0\\
26.97	0\\
26.98	1.73472347597681e-18\\
26.99	0\\
27	0\\
27.01	0\\
27.02	0\\
27.03	0\\
27.04	0\\
27.05	0\\
27.06	0\\
27.07	0\\
27.08	0\\
27.09	0\\
27.1	0\\
27.11	1.73472347597681e-18\\
27.12	0\\
27.13	0\\
27.14	0\\
27.15	0\\
27.16	0\\
27.17	0\\
27.18	0\\
27.19	0\\
27.2	0\\
27.21	0\\
27.22	0\\
27.23	0\\
27.24	0\\
27.25	0\\
27.26	0\\
27.27	0\\
27.28	0\\
27.29	0\\
27.3	0\\
27.31	1.73472347597681e-18\\
27.32	0\\
27.33	0\\
27.34	0\\
27.35	0\\
27.36	0\\
27.37	0\\
27.38	0\\
27.39	0\\
27.4	0\\
27.41	0\\
27.42	0\\
27.43	1.73472347597681e-18\\
27.44	0\\
27.45	0\\
27.46	0\\
27.47	0\\
27.48	0\\
27.49	1.73472347597681e-18\\
27.5	1.73472347597681e-18\\
27.51	0\\
27.52	0\\
27.53	0\\
27.54	0\\
27.55	0\\
27.56	0\\
27.57	0\\
27.58	0\\
27.59	0\\
27.6	0\\
27.61	0\\
27.62	0\\
27.63	1.73472347597681e-18\\
27.64	0\\
27.65	0\\
27.66	0\\
27.67	0\\
27.68	0\\
27.69	0\\
27.7	0\\
27.71	0\\
27.72	0\\
27.73	0\\
27.74	0\\
27.75	1.73472347597681e-18\\
27.76	0\\
27.77	0\\
27.78	0\\
27.79	0\\
27.8	0\\
27.81	0\\
27.82	0\\
27.83	0\\
27.84	0\\
27.85	1.73472347597681e-18\\
27.86	0\\
27.87	1.73472347597681e-18\\
27.88	0\\
27.89	0\\
27.9	0\\
27.91	0\\
27.92	0\\
27.93	0\\
27.94	0\\
27.95	0\\
27.96	0\\
27.97	0\\
27.98	0\\
27.99	0\\
28	1.73472347597681e-18\\
28.01	0\\
28.02	0\\
28.03	1.73472347597681e-18\\
28.04	1.73472347597681e-18\\
28.05	0\\
28.06	0\\
28.07	0\\
28.08	0\\
28.09	0\\
28.1	0\\
28.11	1.73472347597681e-18\\
28.12	0\\
28.13	0\\
28.14	0\\
28.15	0\\
28.16	0\\
28.17	0\\
28.18	0\\
28.19	0\\
28.2	0\\
28.21	0\\
28.22	0\\
28.23	0\\
28.24	0\\
28.25	0\\
28.26	0\\
28.27	0\\
28.28	0\\
28.29	0\\
28.3	0\\
28.31	0\\
28.32	0\\
28.33	0\\
28.34	0\\
28.35	1.73472347597681e-18\\
28.36	0\\
28.37	0\\
28.38	0\\
28.39	0\\
28.4	0\\
28.41	0\\
28.42	0\\
28.43	0\\
28.44	0\\
28.45	0\\
28.46	0\\
28.47	0\\
28.48	0\\
28.49	0\\
28.5	0\\
28.51	0\\
28.52	0\\
28.53	0\\
28.54	1.73472347597681e-18\\
28.55	0\\
28.56	0\\
28.57	0\\
28.58	1.73472347597681e-18\\
28.59	1.73472347597681e-18\\
28.6	0\\
28.61	0\\
28.62	0\\
28.63	0\\
28.64	0\\
28.65	0\\
28.66	0\\
28.67	0\\
28.68	0\\
28.69	1.73472347597681e-18\\
28.7	0\\
28.71	0\\
28.72	0\\
28.73	0\\
28.74	0\\
28.75	0\\
28.76	0\\
28.77	0\\
28.78	0\\
28.79	0\\
28.8	0\\
28.81	1.73472347597681e-18\\
28.82	0\\
28.83	0\\
28.84	0\\
28.85	0\\
28.86	0\\
28.87	0\\
28.88	0\\
28.89	0\\
28.9	0\\
28.91	1.73472347597681e-18\\
28.92	1.73472347597681e-18\\
28.93	0\\
28.94	0\\
28.95	0\\
28.96	0\\
28.97	0\\
28.98	0\\
28.99	0\\
29	0\\
29.01	0\\
29.02	0\\
29.03	0\\
29.04	0\\
29.05	0\\
29.06	0\\
29.07	0\\
29.08	0\\
29.09	0\\
29.1	0\\
29.11	0\\
29.12	0\\
29.13	0\\
29.14	0\\
29.15	0\\
29.16	0\\
29.17	0\\
29.18	1.73472347597681e-18\\
29.19	0\\
29.2	0\\
29.21	0\\
29.22	0\\
29.23	0\\
29.24	0\\
29.25	0\\
29.26	0\\
29.27	0\\
29.28	0\\
29.29	0\\
29.3	0\\
29.31	1.73472347597681e-18\\
29.32	0\\
29.33	0\\
29.34	0\\
29.35	0\\
29.36	0\\
29.37	0\\
29.38	0\\
29.39	0\\
29.4	0\\
29.41	0\\
29.42	0\\
29.43	0\\
29.44	0\\
29.45	0\\
29.46	0\\
29.47	0\\
29.48	0\\
29.49	0\\
29.5	0\\
29.51	0\\
29.52	0\\
29.53	1.73472347597681e-18\\
29.54	0\\
29.55	0\\
29.56	0\\
29.57	0\\
29.58	0\\
29.59	0\\
29.6	0\\
29.61	0\\
29.62	0\\
29.63	1.73472347597681e-18\\
29.64	0\\
29.65	1.73472347597681e-18\\
29.66	1.73472347597681e-18\\
29.67	0\\
29.68	1.73472347597681e-18\\
29.69	0\\
29.7	0\\
29.71	0\\
29.72	1.73472347597681e-18\\
29.73	0\\
29.74	0\\
29.75	0\\
29.76	0\\
29.77	0\\
29.78	0\\
29.79	0\\
29.8	1.73472347597681e-18\\
29.81	0\\
29.82	0\\
29.83	0\\
29.84	0\\
29.85	0\\
29.86	0\\
29.87	1.73472347597681e-18\\
29.88	0\\
29.89	0\\
29.9	0\\
29.91	0\\
29.92	0\\
29.93	1.73472347597681e-18\\
29.94	0\\
29.95	0\\
29.96	0\\
29.97	0\\
29.98	0\\
29.99	0\\
30	1.73472347597681e-18\\
30.01	1.73472347597681e-18\\
30.02	0\\
30.03	0\\
30.04	0\\
30.05	0\\
30.06	0\\
30.07	0\\
30.08	0\\
30.09	0\\
30.1	0\\
30.11	1.73472347597681e-18\\
30.12	0\\
30.13	0\\
30.14	0\\
30.15	0\\
30.16	0\\
30.17	0\\
30.18	0\\
30.19	0\\
30.2	0\\
30.21	0\\
30.22	0\\
30.23	0\\
30.24	0\\
30.25	0\\
30.26	0\\
30.27	0\\
30.28	0\\
30.29	0\\
30.3	1.73472347597681e-18\\
30.31	0\\
30.32	0\\
30.33	0\\
30.34	0\\
30.35	0\\
30.36	0\\
30.37	0\\
30.38	1.73472347597681e-18\\
30.39	0\\
30.4	1.73472347597681e-18\\
30.41	0\\
30.42	0\\
30.43	0\\
30.44	0\\
30.45	0\\
30.46	0\\
30.47	0\\
30.48	0\\
30.49	0\\
30.5	0\\
30.51	0\\
30.52	0\\
30.53	0\\
30.54	0\\
30.55	0\\
30.56	0\\
30.57	0\\
30.58	0\\
30.59	0\\
30.6	0\\
30.61	0\\
30.62	0\\
30.63	0\\
30.64	0\\
30.65	0\\
30.66	0\\
30.67	0\\
30.68	1.73472347597681e-18\\
30.69	0\\
30.7	0\\
30.71	0\\
30.72	1.73472347597681e-18\\
30.73	0\\
30.74	0\\
30.75	0\\
30.76	0\\
30.77	0\\
30.78	0\\
30.79	0\\
30.8	1.73472347597681e-18\\
30.81	1.73472347597681e-18\\
30.82	0\\
30.83	1.73472347597681e-18\\
30.84	0\\
30.85	1.73472347597681e-18\\
30.86	0\\
30.87	0\\
30.88	0\\
30.89	0\\
30.9	0\\
30.91	1.73472347597681e-18\\
30.92	0\\
30.93	0\\
30.94	1.73472347597681e-18\\
30.95	0\\
30.96	0\\
30.97	1.73472347597681e-18\\
30.98	0\\
30.99	1.73472347597681e-18\\
31	0\\
31.01	0\\
31.02	0\\
31.03	0\\
31.04	0\\
31.05	0\\
31.06	0\\
31.07	0\\
31.08	1.73472347597681e-18\\
31.09	0\\
31.1	0\\
31.11	0\\
31.12	0\\
31.13	0\\
31.14	0\\
31.15	0\\
31.16	0\\
31.17	0\\
31.18	0\\
31.19	0\\
31.2	0\\
31.21	1.73472347597681e-18\\
31.22	1.73472347597681e-18\\
31.23	0\\
31.24	0\\
31.25	0\\
31.26	1.73472347597681e-18\\
31.27	0\\
31.28	0\\
31.29	0\\
31.3	0\\
31.31	0\\
31.32	0\\
31.33	0\\
31.34	0\\
31.35	0\\
31.36	0\\
31.37	0\\
31.38	0\\
31.39	1.73472347597681e-18\\
31.4	0\\
31.41	0\\
31.42	1.73472347597681e-18\\
31.43	1.73472347597681e-18\\
31.44	1.73472347597681e-18\\
31.45	0\\
31.46	0\\
31.47	0\\
31.48	0\\
31.49	0\\
31.5	0\\
31.51	0\\
31.52	0\\
31.53	0\\
31.54	0\\
31.55	1.73472347597681e-18\\
31.56	0\\
31.57	0\\
31.58	0\\
31.59	0\\
31.6	0\\
31.61	0\\
31.62	0\\
31.63	0\\
31.64	0\\
31.65	0\\
31.66	0\\
31.67	1.73472347597681e-18\\
31.68	0\\
31.69	0\\
31.7	0\\
31.71	0\\
31.72	0\\
31.73	0\\
31.74	0\\
31.75	0\\
31.76	0\\
31.77	0\\
31.78	0\\
31.79	0\\
31.8	0\\
31.81	0\\
31.82	0\\
31.83	1.73472347597681e-18\\
31.84	0\\
31.85	0\\
31.86	1.73472347597681e-18\\
31.87	0\\
31.88	1.73472347597681e-18\\
31.89	1.73472347597681e-18\\
31.9	0\\
31.91	0\\
31.92	0\\
31.93	0\\
31.94	1.73472347597681e-18\\
31.95	1.73472347597681e-18\\
31.96	0\\
31.97	1.73472347597681e-18\\
31.98	0\\
31.99	0\\
32	0\\
32.01	0\\
32.02	0\\
32.03	0\\
32.04	0\\
32.05	0\\
32.06	0\\
32.07	0\\
32.08	0\\
32.09	0\\
32.1	0\\
32.11	0\\
32.12	0\\
32.13	0\\
32.14	0\\
32.15	0\\
32.16	0\\
32.17	0\\
32.18	0\\
32.19	0\\
32.2	0\\
32.21	0\\
32.22	0\\
32.23	0\\
32.24	0\\
32.25	0\\
32.26	0\\
32.27	0\\
32.28	0\\
32.29	0\\
32.3	0\\
32.31	0\\
32.32	0\\
32.33	0\\
32.34	0\\
32.35	0\\
32.36	0\\
32.37	0\\
32.38	0\\
32.39	0\\
32.4	0\\
32.41	0\\
32.42	0\\
32.43	0\\
32.44	1.73472347597681e-18\\
32.45	1.73472347597681e-18\\
32.46	0\\
32.47	0\\
32.48	0\\
32.49	0\\
32.5	0\\
32.51	0\\
32.52	0\\
32.53	0\\
32.54	0\\
32.55	0\\
32.56	0\\
32.57	0\\
32.58	0\\
32.59	0\\
32.6	0\\
32.61	1.73472347597681e-18\\
32.62	0\\
32.63	0\\
32.64	0\\
32.65	0\\
32.66	0\\
32.67	0\\
32.68	0\\
32.69	0\\
32.7	0\\
32.71	0\\
32.72	0\\
32.73	0\\
32.74	0\\
32.75	0\\
32.76	0\\
32.77	0\\
32.78	0\\
32.79	0\\
32.8	0\\
32.81	0\\
32.82	0\\
32.83	1.73472347597681e-18\\
32.84	1.73472347597681e-18\\
32.85	0\\
32.86	0\\
32.87	0\\
32.88	0\\
32.89	1.73472347597681e-18\\
32.9	0\\
32.91	0\\
32.92	0\\
32.93	0\\
32.94	0\\
32.95	0\\
32.96	0\\
32.97	1.73472347597681e-18\\
32.98	0\\
32.99	0\\
33	0\\
33.01	0\\
33.02	0\\
33.03	0\\
33.04	0\\
33.05	0\\
33.06	0\\
33.07	0\\
33.08	0\\
33.09	0\\
33.1	0\\
33.11	1.73472347597681e-18\\
33.12	0\\
33.13	0\\
33.14	0\\
33.15	0\\
33.16	1.73472347597681e-18\\
33.17	0\\
33.18	0\\
33.19	0\\
33.2	0\\
33.21	0\\
33.22	0\\
33.23	0\\
33.24	0\\
33.25	1.73472347597681e-18\\
33.26	0\\
33.27	1.73472347597681e-18\\
33.28	1.73472347597681e-18\\
33.29	0\\
33.3	1.73472347597681e-18\\
33.31	0\\
33.32	0\\
33.33	0\\
33.34	0\\
33.35	0\\
33.36	1.73472347597681e-18\\
33.37	0\\
33.38	1.73472347597681e-18\\
33.39	0\\
33.4	0\\
33.41	0\\
33.42	0\\
33.43	0\\
33.44	0\\
33.45	0\\
33.46	0\\
33.47	0\\
33.48	0\\
33.49	0\\
33.5	0\\
33.51	0\\
33.52	0\\
33.53	0\\
33.54	0\\
33.55	0\\
33.56	0\\
33.57	0\\
33.58	1.73472347597681e-18\\
33.59	0\\
33.6	1.73472347597681e-18\\
33.61	0\\
33.62	0\\
33.63	0\\
33.64	0\\
33.65	0\\
33.66	0\\
33.67	0\\
33.68	0\\
33.69	0\\
33.7	0\\
33.71	0\\
33.72	0\\
33.73	0\\
33.74	0\\
33.75	0\\
33.76	1.73472347597681e-18\\
33.77	0\\
33.78	0\\
33.79	0\\
33.8	0\\
33.81	0\\
33.82	0\\
33.83	0\\
33.84	0\\
33.85	0\\
33.86	0\\
33.87	0\\
33.88	1.73472347597681e-18\\
33.89	0\\
33.9	0\\
33.91	0\\
33.92	0\\
33.93	0\\
33.94	1.73472347597681e-18\\
33.95	0\\
33.96	0\\
33.97	0\\
33.98	0\\
33.99	1.73472347597681e-18\\
34	0\\
34.01	0\\
34.02	0\\
34.03	0\\
34.04	0\\
34.05	0\\
34.06	0\\
34.07	0\\
34.08	0\\
34.09	0\\
34.1	0\\
34.11	0\\
34.12	0\\
34.13	0\\
34.14	0\\
34.15	0\\
34.16	1.73472347597681e-18\\
34.17	0\\
34.18	0\\
34.19	0\\
34.2	0\\
34.21	0\\
34.22	0\\
34.23	0\\
34.24	0\\
34.25	0\\
34.26	1.73472347597681e-18\\
34.27	0\\
34.28	0\\
34.29	0\\
34.3	0\\
34.31	0\\
34.32	0\\
34.33	0\\
34.34	1.73472347597681e-18\\
34.35	0\\
34.36	0\\
34.37	0\\
34.38	0\\
34.39	0\\
34.4	0\\
34.41	0\\
34.42	1.73472347597681e-18\\
34.43	0\\
34.44	0\\
34.45	0\\
34.46	0\\
34.47	0\\
34.48	0\\
34.49	1.73472347597681e-18\\
34.5	0\\
34.51	0\\
34.52	0\\
34.53	0\\
34.54	0\\
34.55	0\\
34.56	0\\
34.57	0\\
34.58	1.73472347597681e-18\\
34.59	0\\
34.6	0\\
34.61	0\\
34.62	0\\
34.63	0\\
34.64	0\\
34.65	0\\
34.66	0\\
34.67	0\\
34.68	0\\
34.69	0\\
34.7	1.73472347597681e-18\\
34.71	0\\
34.72	0\\
34.73	0\\
34.74	1.73472347597681e-18\\
34.75	1.73472347597681e-18\\
34.76	0\\
34.77	1.73472347597681e-18\\
34.78	1.73472347597681e-18\\
34.79	0\\
34.8	0\\
34.81	0\\
34.82	0\\
34.83	0\\
34.84	0\\
34.85	0\\
34.86	0\\
34.87	0\\
34.88	0\\
34.89	0\\
34.9	0\\
34.91	0\\
34.92	0\\
34.93	0\\
34.94	0\\
34.95	1.73472347597681e-18\\
34.96	1.73472347597681e-18\\
34.97	1.73472347597681e-18\\
34.98	0\\
34.99	0\\
35	0\\
35.01	0\\
35.02	0\\
35.03	0\\
35.04	0\\
35.05	0\\
35.06	0\\
35.07	1.73472347597681e-18\\
35.08	0\\
35.09	1.73472347597681e-18\\
35.1	0\\
35.11	0\\
35.12	0\\
35.13	1.73472347597681e-18\\
35.14	0\\
35.15	0\\
35.16	1.73472347597681e-18\\
35.17	0\\
35.18	0\\
35.19	0\\
35.2	0\\
35.21	0\\
35.22	0\\
35.23	0\\
35.24	0\\
35.25	0\\
35.26	0\\
35.27	0\\
35.28	0\\
35.29	0\\
35.3	0\\
35.31	0\\
35.32	0\\
35.33	0\\
35.34	0\\
35.35	0\\
35.36	0\\
35.37	0\\
35.38	0\\
35.39	0\\
35.4	0\\
35.41	0\\
35.42	0\\
35.43	1.73472347597681e-18\\
35.44	0\\
35.45	0\\
35.46	0\\
35.47	0\\
35.48	0\\
35.49	0\\
35.5	0\\
35.51	0\\
35.52	0\\
35.53	0\\
35.54	0\\
35.55	0\\
35.56	0\\
35.57	0\\
35.58	0\\
35.59	0\\
35.6	0\\
35.61	0\\
35.62	0\\
35.63	0\\
35.64	0\\
35.65	0\\
35.66	0\\
35.67	0\\
35.68	1.73472347597681e-18\\
35.69	1.73472347597681e-18\\
35.7	0\\
35.71	0\\
35.72	0\\
35.73	0\\
35.74	0\\
35.75	0\\
35.76	0\\
35.77	0\\
35.78	0\\
35.79	0\\
35.8	0\\
35.81	0\\
35.82	0\\
35.83	0\\
35.84	0\\
35.85	1.73472347597681e-18\\
35.86	0\\
35.87	0\\
35.88	0\\
35.89	0\\
35.9	0\\
35.91	1.73472347597681e-18\\
35.92	0\\
35.93	0\\
35.94	0\\
35.95	0\\
35.96	0\\
35.97	0\\
35.98	1.73472347597681e-18\\
35.99	0\\
36	0\\
36.01	0\\
36.02	0\\
36.03	0\\
36.04	0\\
36.05	0\\
36.06	0\\
36.07	1.73472347597681e-18\\
36.08	1.73472347597681e-18\\
36.09	0\\
36.1	0\\
36.11	0\\
36.12	0\\
36.13	0\\
36.14	0\\
36.15	0\\
36.16	1.73472347597681e-18\\
36.17	0\\
36.18	0\\
36.19	0\\
36.2	0\\
36.21	0\\
36.22	0\\
36.23	0\\
36.24	0\\
36.25	1.73472347597681e-18\\
36.26	0\\
36.27	0\\
36.28	0\\
36.29	0\\
36.3	0\\
36.31	0\\
36.32	1.73472347597681e-18\\
36.33	0\\
36.34	0\\
36.35	0\\
36.36	0\\
36.37	0\\
36.38	1.73472347597681e-18\\
36.39	0\\
36.4	0\\
36.41	0\\
36.42	0\\
36.43	0\\
36.44	1.73472347597681e-18\\
36.45	0\\
36.46	0\\
36.47	0\\
36.48	0\\
36.49	0\\
36.5	1.73472347597681e-18\\
36.51	0\\
36.52	0\\
36.53	0\\
36.54	0\\
36.55	0\\
36.56	0\\
36.57	0\\
36.58	1.73472347597681e-18\\
36.59	0\\
36.6	1.73472347597681e-18\\
36.61	0\\
36.62	0\\
36.63	1.73472347597681e-18\\
36.64	0\\
36.65	0\\
36.66	0\\
36.67	0\\
36.68	0\\
36.69	0\\
36.7	0\\
36.71	0\\
36.72	0\\
36.73	0\\
36.74	0\\
36.75	0\\
36.76	0\\
36.77	0\\
36.78	0\\
36.79	0\\
36.8	0\\
36.81	1.73472347597681e-18\\
36.82	0\\
36.83	0\\
36.84	0\\
36.85	0\\
36.86	0\\
36.87	0\\
36.88	0\\
36.89	0\\
36.9	0\\
36.91	0\\
36.92	0\\
36.93	0\\
36.94	0\\
36.95	0\\
36.96	0\\
36.97	0\\
36.98	1.73472347597681e-18\\
36.99	0\\
37	0\\
37.01	0\\
37.02	1.73472347597681e-18\\
37.03	0\\
37.04	0\\
37.05	0\\
37.06	0\\
37.07	0\\
37.08	0\\
37.09	0\\
37.1	0\\
37.11	0\\
37.12	0\\
37.13	0\\
37.14	0\\
37.15	0\\
37.16	0\\
37.17	0\\
37.18	0\\
37.19	0\\
37.2	0\\
37.21	0\\
37.22	0\\
37.23	0\\
37.24	0\\
37.25	0\\
37.26	0\\
37.27	0\\
37.28	0\\
37.29	0\\
37.3	1.73472347597681e-18\\
37.31	0\\
37.32	0\\
37.33	0\\
37.34	0\\
37.35	0\\
37.36	0\\
37.37	0\\
37.38	0\\
37.39	0\\
37.4	0\\
37.41	0\\
37.42	0\\
37.43	0\\
37.44	0\\
37.45	0\\
37.46	0\\
37.47	0\\
37.48	0\\
37.49	1.73472347597681e-18\\
37.5	0\\
37.51	0\\
37.52	0\\
37.53	0\\
37.54	0\\
37.55	0\\
37.56	0\\
37.57	0\\
37.58	0\\
37.59	0\\
37.6	0\\
37.61	0\\
37.62	0\\
37.63	0\\
37.64	0\\
37.65	0\\
37.66	0\\
37.67	0\\
37.68	0\\
37.69	0\\
37.7	0\\
37.71	0\\
37.72	0\\
37.73	0\\
37.74	0\\
37.75	0\\
37.76	1.73472347597681e-18\\
37.77	0\\
37.78	0\\
37.79	0\\
37.8	0\\
37.81	0\\
37.82	0\\
37.83	0\\
37.84	0\\
37.85	0\\
37.86	0\\
37.87	0\\
37.88	0\\
37.89	0\\
37.9	0\\
37.91	0\\
37.92	0\\
37.93	0\\
37.94	0\\
37.95	0\\
37.96	0\\
37.97	0\\
37.98	0\\
37.99	0\\
38	1.73472347597681e-18\\
38.01	0\\
38.02	0\\
38.03	0\\
38.04	0\\
38.05	0\\
38.06	0\\
38.07	0\\
38.08	0\\
38.09	0\\
38.1	0\\
38.11	1.73472347597681e-18\\
38.12	1.73472347597681e-18\\
38.13	0\\
38.14	0\\
38.15	0\\
38.16	0\\
38.17	0\\
38.18	0\\
38.19	0\\
38.2	0\\
38.21	0\\
38.22	0\\
38.23	0\\
38.24	0\\
38.25	0\\
38.26	0\\
38.27	1.73472347597681e-18\\
38.28	0\\
38.29	0\\
38.3	0\\
38.31	1.73472347597681e-18\\
38.32	0\\
38.33	0\\
38.34	0\\
38.35	1.73472347597681e-18\\
38.36	0\\
38.37	0\\
38.38	0\\
38.39	0\\
38.4	0\\
38.41	0\\
38.42	0\\
38.43	0\\
38.44	0\\
38.45	0\\
38.46	0\\
38.47	0\\
38.48	1.73472347597681e-18\\
38.49	0\\
38.5	0\\
38.51	0\\
38.52	0\\
38.53	0\\
38.54	0\\
38.55	0\\
38.56	0\\
38.57	0\\
38.58	0\\
38.59	0\\
38.6	0\\
38.61	1.73472347597681e-18\\
38.62	0\\
38.63	1.73472347597681e-18\\
38.64	0\\
38.65	1.73472347597681e-18\\
38.66	0\\
38.67	0\\
38.68	0\\
38.69	0\\
38.7	0\\
38.71	0\\
38.72	0\\
38.73	0\\
38.74	1.73472347597681e-18\\
38.75	0\\
38.76	0\\
38.77	0\\
38.78	0\\
38.79	0\\
38.8	0\\
38.81	0\\
38.82	1.73472347597681e-18\\
38.83	0\\
38.84	0\\
38.85	0\\
38.86	0\\
38.87	1.73472347597681e-18\\
38.88	0\\
38.89	1.73472347597681e-18\\
38.9	0\\
38.91	0\\
38.92	0\\
38.93	0\\
38.94	0\\
38.95	0\\
38.96	0\\
38.97	0\\
38.98	0\\
38.99	0\\
39	0\\
39.01	0\\
39.02	0\\
39.03	0\\
39.04	0\\
39.05	0\\
39.06	0\\
39.07	0\\
39.08	0\\
39.09	1.73472347597681e-18\\
39.1	0\\
39.11	0\\
39.12	0\\
39.13	0\\
39.14	0\\
39.15	0\\
39.16	0\\
39.17	0\\
39.18	0\\
39.19	1.73472347597681e-18\\
39.2	0\\
39.21	0\\
39.22	0\\
39.23	0\\
39.24	0\\
39.25	0\\
39.26	0\\
39.27	0\\
39.28	0\\
39.29	0\\
39.3	0\\
39.31	0\\
39.32	0\\
39.33	0\\
39.34	1.73472347597681e-18\\
39.35	0\\
39.36	0\\
39.37	0\\
39.38	0\\
39.39	0\\
39.4	0\\
39.41	0\\
39.42	0\\
39.43	0\\
39.44	0\\
39.45	0\\
39.46	0\\
39.47	0\\
39.48	0\\
39.49	0\\
39.5	0\\
39.51	0\\
39.52	1.73472347597681e-18\\
39.53	0\\
39.54	0\\
39.55	0\\
39.56	0\\
39.57	0\\
39.58	0\\
39.59	0\\
39.6	0\\
39.61	0\\
39.62	0\\
39.63	0\\
39.64	0\\
39.65	0\\
39.66	0\\
39.67	1.73472347597681e-18\\
39.68	0\\
39.69	0\\
39.7	0\\
39.71	0\\
39.72	0\\
39.73	0\\
39.74	0\\
39.75	1.73472347597681e-18\\
39.76	0\\
39.77	0\\
39.78	0\\
39.79	0\\
39.8	0\\
39.81	0\\
39.82	1.73472347597681e-18\\
39.83	0\\
39.84	0\\
39.85	0\\
39.86	0\\
39.87	0\\
39.88	0\\
39.89	0\\
39.9	0\\
39.91	0\\
39.92	1.73472347597681e-18\\
39.93	0\\
39.94	1.73472347597681e-18\\
39.95	0\\
39.96	0\\
39.97	0\\
39.98	0\\
39.99	0\\
40	0\\
40.01	1.73472347597681e-18\\
};
\addplot [color=green,dashed,forget plot]
  table[row sep=crcr]{%
40.01	1.73472347597681e-18\\
40.02	0\\
40.03	1.73472347597681e-18\\
40.04	0\\
40.05	1.73472347597681e-18\\
40.06	0\\
40.07	1.73472347597681e-18\\
40.08	0\\
40.09	1.73472347597681e-18\\
40.1	1.73472347597681e-18\\
40.11	0\\
40.12	0\\
40.13	0\\
40.14	0\\
40.15	1.73472347597681e-18\\
40.16	0\\
40.17	0\\
40.18	0\\
40.19	0\\
40.2	0\\
40.21	0\\
40.22	1.73472347597681e-18\\
40.23	0\\
40.24	0\\
40.25	1.73472347597681e-18\\
40.26	0\\
40.27	0\\
40.28	0\\
40.29	0\\
40.3	0\\
40.31	1.73472347597681e-18\\
40.32	1.73472347597681e-18\\
40.33	0\\
40.34	0\\
40.35	1.73472347597681e-18\\
40.36	0\\
40.37	0\\
40.38	0\\
40.39	0\\
40.4	1.73472347597681e-18\\
40.41	0\\
40.42	0\\
40.43	1.73472347597681e-18\\
40.44	0\\
40.45	0\\
40.46	0\\
40.47	0\\
40.48	0\\
40.49	0\\
40.5	1.73472347597681e-18\\
40.51	0\\
40.52	0\\
40.53	0\\
40.54	0\\
40.55	0\\
40.56	0\\
40.57	0\\
40.58	1.73472347597681e-18\\
40.59	0\\
40.6	0\\
40.61	1.73472347597681e-18\\
40.62	0\\
40.63	0\\
40.64	0\\
40.65	0\\
40.66	0\\
40.67	0\\
40.68	0\\
40.69	0\\
40.7	0\\
40.71	0\\
40.72	0\\
40.73	0\\
40.74	0\\
40.75	0\\
40.76	0\\
40.77	0\\
40.78	0\\
40.79	0\\
40.8	0\\
40.81	0\\
40.82	0\\
40.83	0\\
40.84	0\\
40.85	0\\
40.86	0\\
40.87	1.73472347597681e-18\\
40.88	0\\
40.89	0\\
40.9	1.73472347597681e-18\\
40.91	0\\
40.92	0\\
40.93	0\\
40.94	0\\
40.95	1.73472347597681e-18\\
40.96	0\\
40.97	1.73472347597681e-18\\
40.98	0\\
40.99	0\\
41	0\\
41.01	0\\
41.02	0\\
41.03	0\\
41.04	0\\
41.05	0\\
41.06	0\\
41.07	0\\
41.08	0\\
41.09	0\\
41.1	0\\
41.11	0\\
41.12	0\\
41.13	0\\
41.14	0\\
41.15	1.73472347597681e-18\\
41.16	0\\
41.17	0\\
41.18	1.73472347597681e-18\\
41.19	0\\
41.2	0\\
41.21	0\\
41.22	0\\
41.23	0\\
41.24	0\\
41.25	0\\
41.26	0\\
41.27	1.73472347597681e-18\\
41.28	0\\
41.29	1.73472347597681e-18\\
41.3	0\\
41.31	0\\
41.32	0\\
41.33	0\\
41.34	0\\
41.35	0\\
41.36	0\\
41.37	0\\
41.38	1.73472347597681e-18\\
41.39	0\\
41.4	0\\
41.41	1.73472347597681e-18\\
41.42	0\\
41.43	0\\
41.44	0\\
41.45	0\\
41.46	0\\
41.47	0\\
41.48	0\\
41.49	0\\
41.5	0\\
41.51	0\\
41.52	0\\
41.53	0\\
41.54	0\\
41.55	0\\
41.56	0\\
41.57	0\\
41.58	0\\
41.59	0\\
41.6	0\\
41.61	1.73472347597681e-18\\
41.62	0\\
41.63	0\\
41.64	0\\
41.65	0\\
41.66	0\\
41.67	0\\
41.68	1.73472347597681e-18\\
41.69	0\\
41.7	0\\
41.71	0\\
41.72	0\\
41.73	0\\
41.74	0\\
41.75	0\\
41.76	0\\
41.77	0\\
41.78	0\\
41.79	0\\
41.8	0\\
41.81	1.73472347597681e-18\\
41.82	0\\
41.83	0\\
41.84	0\\
41.85	1.73472347597681e-18\\
41.86	0\\
41.87	0\\
41.88	0\\
41.89	0\\
41.9	0\\
41.91	0\\
41.92	1.73472347597681e-18\\
41.93	0\\
41.94	0\\
41.95	0\\
41.96	0\\
41.97	0\\
41.98	0\\
41.99	0\\
42	0\\
42.01	0\\
42.02	0\\
42.03	0\\
42.04	0\\
42.05	0\\
42.06	0\\
42.07	0\\
42.08	0\\
42.09	0\\
42.1	0\\
42.11	0\\
42.12	0\\
42.13	0\\
42.14	1.73472347597681e-18\\
42.15	0\\
42.16	1.73472347597681e-18\\
42.17	0\\
42.18	1.73472347597681e-18\\
42.19	0\\
42.2	0\\
42.21	0\\
42.22	0\\
42.23	0\\
42.24	1.73472347597681e-18\\
42.25	0\\
42.26	0\\
42.27	0\\
42.28	0\\
42.29	0\\
42.3	1.73472347597681e-18\\
42.31	0\\
42.32	0\\
42.33	0\\
42.34	0\\
42.35	0\\
42.36	1.73472347597681e-18\\
42.37	0\\
42.38	0\\
42.39	0\\
42.4	0\\
42.41	0\\
42.42	0\\
42.43	0\\
42.44	0\\
42.45	0\\
42.46	0\\
42.47	0\\
42.48	0\\
42.49	0\\
42.5	0\\
42.51	0\\
42.52	0\\
42.53	0\\
42.54	0\\
42.55	0\\
42.56	0\\
42.57	0\\
42.58	0\\
42.59	0\\
42.6	0\\
42.61	0\\
42.62	0\\
42.63	0\\
42.64	0\\
42.65	0\\
42.66	0\\
42.67	0\\
42.68	0\\
42.69	0\\
42.7	0\\
42.71	1.73472347597681e-18\\
42.72	0\\
42.73	0\\
42.74	1.73472347597681e-18\\
42.75	0\\
42.76	0\\
42.77	1.73472347597681e-18\\
42.78	0\\
42.79	0\\
42.8	0\\
42.81	0\\
42.82	0\\
42.83	0\\
42.84	0\\
42.85	0\\
42.86	0\\
42.87	0\\
42.88	0\\
42.89	1.73472347597681e-18\\
42.9	0\\
42.91	0\\
42.92	0\\
42.93	0\\
42.94	0\\
42.95	0\\
42.96	0\\
42.97	0\\
42.98	0\\
42.99	0\\
43	1.73472347597681e-18\\
43.01	0\\
43.02	0\\
43.03	0\\
43.04	0\\
43.05	0\\
43.06	0\\
43.07	0\\
43.08	0\\
43.09	1.73472347597681e-18\\
43.1	0\\
43.11	0\\
43.12	0\\
43.13	1.73472347597681e-18\\
43.14	0\\
43.15	0\\
43.16	0\\
43.17	0\\
43.18	0\\
43.19	0\\
43.2	0\\
43.21	0\\
43.22	0\\
43.23	0\\
43.24	0\\
43.25	0\\
43.26	0\\
43.27	0\\
43.28	0\\
43.29	0\\
43.3	0\\
43.31	0\\
43.32	0\\
43.33	1.73472347597681e-18\\
43.34	0\\
43.35	0\\
43.36	0\\
43.37	0\\
43.38	1.73472347597681e-18\\
43.39	0\\
43.4	0\\
43.41	0\\
43.42	0\\
43.43	0\\
43.44	0\\
43.45	0\\
43.46	0\\
43.47	0\\
43.48	0\\
43.49	0\\
43.5	0\\
43.51	0\\
43.52	1.73472347597681e-18\\
43.53	0\\
43.54	0\\
43.55	1.73472347597681e-18\\
43.56	0\\
43.57	0\\
43.58	0\\
43.59	0\\
43.6	0\\
43.61	0\\
43.62	0\\
43.63	0\\
43.64	0\\
43.65	0\\
43.66	0\\
43.67	0\\
43.68	0\\
43.69	0\\
43.7	0\\
43.71	0\\
43.72	1.73472347597681e-18\\
43.73	0\\
43.74	1.73472347597681e-18\\
43.75	0\\
43.76	0\\
43.77	0\\
43.78	0\\
43.79	0\\
43.8	0\\
43.81	1.73472347597681e-18\\
43.82	0\\
43.83	1.73472347597681e-18\\
43.84	1.73472347597681e-18\\
43.85	0\\
43.86	0\\
43.87	0\\
43.88	0\\
43.89	0\\
43.9	0\\
43.91	0\\
43.92	0\\
43.93	1.73472347597681e-18\\
43.94	0\\
43.95	0\\
43.96	1.73472347597681e-18\\
43.97	0\\
43.98	0\\
43.99	0\\
44	0\\
44.01	0\\
44.02	0\\
44.03	1.73472347597681e-18\\
44.04	0\\
44.05	0\\
44.06	0\\
44.07	0\\
44.08	0\\
44.09	0\\
44.1	0\\
44.11	0\\
44.12	0\\
44.13	0\\
44.14	0\\
44.15	0\\
44.16	0\\
44.17	0\\
44.18	0\\
44.19	0\\
44.2	0\\
44.21	0\\
44.22	0\\
44.23	0\\
44.24	0\\
44.25	0\\
44.26	1.73472347597681e-18\\
44.27	0\\
44.28	0\\
44.29	0\\
44.3	0\\
44.31	0\\
44.32	0\\
44.33	1.73472347597681e-18\\
44.34	0\\
44.35	1.73472347597681e-18\\
44.36	1.73472347597681e-18\\
44.37	0\\
44.38	0\\
44.39	0\\
44.4	0\\
44.41	0\\
44.42	0\\
44.43	0\\
44.44	0\\
44.45	0\\
44.46	1.73472347597681e-18\\
44.47	0\\
44.48	0\\
44.49	0\\
44.5	0\\
44.51	0\\
44.52	0\\
44.53	0\\
44.54	0\\
44.55	0\\
44.56	0\\
44.57	0\\
44.58	0\\
44.59	0\\
44.6	0\\
44.61	0\\
44.62	0\\
44.63	0\\
44.64	1.73472347597681e-18\\
44.65	0\\
44.66	0\\
44.67	0\\
44.68	0\\
44.69	1.73472347597681e-18\\
44.7	0\\
44.71	0\\
44.72	0\\
44.73	0\\
44.74	0\\
44.75	0\\
44.76	0\\
44.77	0\\
44.78	0\\
44.79	0\\
44.8	0\\
44.81	0\\
44.82	0\\
44.83	0\\
44.84	1.73472347597681e-18\\
44.85	1.73472347597681e-18\\
44.86	0\\
44.87	0\\
44.88	0\\
44.89	0\\
44.9	0\\
44.91	1.73472347597681e-18\\
44.92	0\\
44.93	0\\
44.94	0\\
44.95	0\\
44.96	0\\
44.97	0\\
44.98	0\\
44.99	0\\
45	0\\
45.01	0\\
45.02	0\\
45.03	0\\
45.04	0\\
45.05	0\\
45.06	0\\
45.07	0\\
45.08	0\\
45.09	0\\
45.1	0\\
45.11	0\\
45.12	0\\
45.13	1.73472347597681e-18\\
45.14	0\\
45.15	0\\
45.16	0\\
45.17	0\\
45.18	0\\
45.19	0\\
45.2	0\\
45.21	0\\
45.22	0\\
45.23	0\\
45.24	0\\
45.25	0\\
45.26	0\\
45.27	0\\
45.28	0\\
45.29	0\\
45.3	0\\
45.31	0\\
45.32	0\\
45.33	0\\
45.34	0\\
45.35	0\\
45.36	0\\
45.37	0\\
45.38	0\\
45.39	0\\
45.4	1.73472347597681e-18\\
45.41	0\\
45.42	0\\
45.43	0\\
45.44	0\\
45.45	0\\
45.46	0\\
45.47	0\\
45.48	0\\
45.49	0\\
45.5	0\\
45.51	0\\
45.52	0\\
45.53	0\\
45.54	0\\
45.55	0\\
45.56	0\\
45.57	0\\
45.58	0\\
45.59	0\\
45.6	1.73472347597681e-18\\
45.61	0\\
45.62	0\\
45.63	1.73472347597681e-18\\
45.64	0\\
45.65	0\\
45.66	0\\
45.67	0\\
45.68	0\\
45.69	0\\
45.7	0\\
45.71	1.73472347597681e-18\\
45.72	0\\
45.73	0\\
45.74	0\\
45.75	0\\
45.76	0\\
45.77	0\\
45.78	0\\
45.79	0\\
45.8	0\\
45.81	0\\
45.82	0\\
45.83	1.73472347597681e-18\\
45.84	0\\
45.85	0\\
45.86	0\\
45.87	0\\
45.88	0\\
45.89	0\\
45.9	0\\
45.91	0\\
45.92	0\\
45.93	1.73472347597681e-18\\
45.94	0\\
45.95	0\\
45.96	0\\
45.97	0\\
45.98	0\\
45.99	0\\
46	0\\
46.01	0\\
46.02	0\\
46.03	0\\
46.04	0\\
46.05	0\\
46.06	0\\
46.07	0\\
46.08	1.73472347597681e-18\\
46.09	0\\
46.1	0\\
46.11	0\\
46.12	0\\
46.13	0\\
46.14	0\\
46.15	0\\
46.16	0\\
46.17	0\\
46.18	0\\
46.19	0\\
46.2	0\\
46.21	0\\
46.22	0\\
46.23	0\\
46.24	0\\
46.25	1.73472347597681e-18\\
46.26	0\\
46.27	0\\
46.28	0\\
46.29	1.73472347597681e-18\\
46.3	1.73472347597681e-18\\
46.31	0\\
46.32	1.73472347597681e-18\\
46.33	0\\
46.34	0\\
46.35	0\\
46.36	0\\
46.37	0\\
46.38	1.73472347597681e-18\\
46.39	0\\
46.4	0\\
46.41	0\\
46.42	0\\
46.43	0\\
46.44	0\\
46.45	0\\
46.46	0\\
46.47	0\\
46.48	0\\
46.49	0\\
46.5	0\\
46.51	0\\
46.52	0\\
46.53	0\\
46.54	0\\
46.55	0\\
46.56	0\\
46.57	0\\
46.58	0\\
46.59	1.73472347597681e-18\\
46.6	1.73472347597681e-18\\
46.61	0\\
46.62	0\\
46.63	0\\
46.64	0\\
46.65	0\\
46.66	0\\
46.67	0\\
46.68	1.73472347597681e-18\\
46.69	0\\
46.7	0\\
46.71	0\\
46.72	0\\
46.73	0\\
46.74	0\\
46.75	0\\
46.76	0\\
46.77	0\\
46.78	0\\
46.79	0\\
46.8	0\\
46.81	0\\
46.82	0\\
46.83	0\\
46.84	0\\
46.85	0\\
46.86	0\\
46.87	1.73472347597681e-18\\
46.88	0\\
46.89	1.73472347597681e-18\\
46.9	0\\
46.91	0\\
46.92	0\\
46.93	0\\
46.94	0\\
46.95	0\\
46.96	0\\
46.97	0\\
46.98	0\\
46.99	0\\
47	0\\
47.01	0\\
47.02	0\\
47.03	0\\
47.04	0\\
47.05	1.73472347597681e-18\\
47.06	0\\
47.07	0\\
47.08	0\\
47.09	0\\
47.1	1.73472347597681e-18\\
47.11	0\\
47.12	0\\
47.13	0\\
47.14	1.73472347597681e-18\\
47.15	0\\
47.16	1.73472347597681e-18\\
47.17	0\\
47.18	0\\
47.19	0\\
47.2	0\\
47.21	0\\
47.22	0\\
47.23	0\\
47.24	0\\
47.25	0\\
47.26	0\\
47.27	0\\
47.28	0\\
47.29	0\\
47.3	0\\
47.31	0\\
47.32	0\\
47.33	0\\
47.34	0\\
47.35	0\\
47.36	1.73472347597681e-18\\
47.37	0\\
47.38	1.73472347597681e-18\\
47.39	0\\
47.4	0\\
47.41	0\\
47.42	0\\
47.43	0\\
47.44	0\\
47.45	0\\
47.46	1.73472347597681e-18\\
47.47	0\\
47.48	0\\
47.49	0\\
47.5	0\\
47.51	1.73472347597681e-18\\
47.52	1.73472347597681e-18\\
47.53	0\\
47.54	0\\
47.55	0\\
47.56	0\\
47.57	1.73472347597681e-18\\
47.58	1.73472347597681e-18\\
47.59	0\\
47.6	0\\
47.61	0\\
47.62	0\\
47.63	0\\
47.64	0\\
47.65	0\\
47.66	0\\
47.67	0\\
47.68	0\\
47.69	0\\
47.7	0\\
47.71	0\\
47.72	0\\
47.73	0\\
47.74	0\\
47.75	0\\
47.76	1.73472347597681e-18\\
47.77	0\\
47.78	0\\
47.79	0\\
47.8	0\\
47.81	0\\
47.82	0\\
47.83	0\\
47.84	1.73472347597681e-18\\
47.85	0\\
47.86	0\\
47.87	0\\
47.88	0\\
47.89	0\\
47.9	0\\
47.91	0\\
47.92	0\\
47.93	0\\
47.94	0\\
47.95	0\\
47.96	0\\
47.97	1.73472347597681e-18\\
47.98	0\\
47.99	0\\
48	1.73472347597681e-18\\
48.01	0\\
48.02	0\\
48.03	0\\
48.04	0\\
48.05	0\\
48.06	0\\
48.07	1.73472347597681e-18\\
48.08	1.73472347597681e-18\\
48.09	0\\
48.1	0\\
48.11	1.73472347597681e-18\\
48.12	0\\
48.13	0\\
48.14	0\\
48.15	0\\
48.16	0\\
48.17	0\\
48.18	0\\
48.19	0\\
48.2	0\\
48.21	0\\
48.22	0\\
48.23	0\\
48.24	0\\
48.25	0\\
48.26	0\\
48.27	0\\
48.28	0\\
48.29	0\\
48.3	0\\
48.31	0\\
48.32	0\\
48.33	0\\
48.34	0\\
48.35	0\\
48.36	0\\
48.37	0\\
48.38	0\\
48.39	0\\
48.4	0\\
48.41	0\\
48.42	0\\
48.43	0\\
48.44	0\\
48.45	0\\
48.46	0\\
48.47	0\\
48.48	0\\
48.49	0\\
48.5	1.73472347597681e-18\\
48.51	0\\
48.52	0\\
48.53	0\\
48.54	0\\
48.55	0\\
48.56	0\\
48.57	0\\
48.58	0\\
48.59	0\\
48.6	0\\
48.61	0\\
48.62	0\\
48.63	0\\
48.64	0\\
48.65	0\\
48.66	0\\
48.67	0\\
48.68	0\\
48.69	0\\
48.7	0\\
48.71	0\\
48.72	0\\
48.73	1.73472347597681e-18\\
48.74	0\\
48.75	0\\
48.76	0\\
48.77	0\\
48.78	0\\
48.79	0\\
48.8	1.73472347597681e-18\\
48.81	0\\
48.82	0\\
48.83	0\\
48.84	0\\
48.85	0\\
48.86	0\\
48.87	1.73472347597681e-18\\
48.88	1.73472347597681e-18\\
48.89	0\\
48.9	0\\
48.91	0\\
48.92	0\\
48.93	0\\
48.94	0\\
48.95	1.73472347597681e-18\\
48.96	1.73472347597681e-18\\
48.97	0\\
48.98	0\\
48.99	0\\
49	0\\
49.01	0\\
49.02	0\\
49.03	0\\
49.04	1.73472347597681e-18\\
49.05	1.73472347597681e-18\\
49.06	0\\
49.07	0\\
49.08	0\\
49.09	0\\
49.1	0\\
49.11	0\\
49.12	0\\
49.13	1.73472347597681e-18\\
49.14	0\\
49.15	0\\
49.16	0\\
49.17	0\\
49.18	0\\
49.19	0\\
49.2	0\\
49.21	0\\
49.22	0\\
49.23	0\\
49.24	0\\
49.25	0\\
49.26	0\\
49.27	0\\
49.28	0\\
49.29	0\\
49.3	0\\
49.31	0\\
49.32	1.73472347597681e-18\\
49.33	0\\
49.34	0\\
49.35	0\\
49.36	0\\
49.37	0\\
49.38	0\\
49.39	0\\
49.4	0\\
49.41	0\\
49.42	0\\
49.43	0\\
49.44	0\\
49.45	0\\
49.46	0\\
49.47	0\\
49.48	0\\
49.49	0\\
49.5	0\\
49.51	0\\
49.52	0\\
49.53	0\\
49.54	0\\
49.55	0\\
49.56	0\\
49.57	0\\
49.58	0\\
49.59	0\\
49.6	0\\
49.61	0\\
49.62	0\\
49.63	0\\
49.64	0\\
49.65	0\\
49.66	0\\
49.67	1.73472347597681e-18\\
49.68	1.73472347597681e-18\\
49.69	0\\
49.7	0\\
49.71	0\\
49.72	0\\
49.73	0\\
49.74	0\\
49.75	0\\
49.76	0\\
49.77	0\\
49.78	0\\
49.79	0\\
49.8	0\\
49.81	0\\
49.82	0\\
49.83	0\\
49.84	0\\
49.85	0\\
49.86	0\\
49.87	0\\
49.88	0\\
49.89	0\\
49.9	0\\
49.91	0\\
49.92	0\\
49.93	0\\
49.94	1.73472347597681e-18\\
49.95	0\\
49.96	0\\
49.97	0\\
49.98	0\\
49.99	0\\
50	0\\
50.01	0\\
50.02	0\\
50.03	0\\
50.04	0\\
50.05	0\\
50.06	0\\
50.07	0\\
50.08	0\\
50.09	0\\
50.1	0\\
50.11	0\\
50.12	0\\
50.13	0\\
50.14	1.73472347597681e-18\\
50.15	0\\
50.16	0\\
50.17	0\\
50.18	0\\
50.19	0\\
50.2	0\\
50.21	1.73472347597681e-18\\
50.22	0\\
50.23	0\\
50.24	0\\
50.25	0\\
50.26	0\\
50.27	0\\
50.28	0\\
50.29	0\\
50.3	0\\
50.31	1.73472347597681e-18\\
50.32	0\\
50.33	0\\
50.34	0\\
50.35	0\\
50.36	0\\
50.37	1.73472347597681e-18\\
50.38	0\\
50.39	0\\
50.4	0\\
50.41	0\\
50.42	0\\
50.43	0\\
50.44	0\\
50.45	0\\
50.46	0\\
50.47	1.73472347597681e-18\\
50.48	0\\
50.49	0\\
50.5	0\\
50.51	1.73472347597681e-18\\
50.52	0\\
50.53	0\\
50.54	0\\
50.55	0\\
50.56	1.73472347597681e-18\\
50.57	0\\
50.58	0\\
50.59	0\\
50.6	0\\
50.61	0\\
50.62	0\\
50.63	0\\
50.64	0\\
50.65	0\\
50.66	0\\
50.67	0\\
50.68	0\\
50.69	0\\
50.7	0\\
50.71	0\\
50.72	0\\
50.73	0\\
50.74	0\\
50.75	0\\
50.76	0\\
50.77	0\\
50.78	1.73472347597681e-18\\
50.79	0\\
50.8	0\\
50.81	0\\
50.82	0\\
50.83	0\\
50.84	0\\
50.85	0\\
50.86	0\\
50.87	1.73472347597681e-18\\
50.88	1.73472347597681e-18\\
50.89	0\\
50.9	0\\
50.91	0\\
50.92	0\\
50.93	1.73472347597681e-18\\
50.94	0\\
50.95	0\\
50.96	1.73472347597681e-18\\
50.97	0\\
50.98	0\\
50.99	0\\
51	0\\
51.01	1.73472347597681e-18\\
51.02	0\\
51.03	0\\
51.04	0\\
51.05	0\\
51.06	0\\
51.07	0\\
51.08	0\\
51.09	0\\
51.1	0\\
51.11	0\\
51.12	0\\
51.13	0\\
51.14	0\\
51.15	0\\
51.16	0\\
51.17	0\\
51.18	0\\
51.19	1.73472347597681e-18\\
51.2	0\\
51.21	0\\
51.22	0\\
51.23	1.73472347597681e-18\\
51.24	0\\
51.25	0\\
51.26	0\\
51.27	0\\
51.28	0\\
51.29	0\\
51.3	0\\
51.31	0\\
51.32	1.73472347597681e-18\\
51.33	0\\
51.34	0\\
51.35	0\\
51.36	0\\
51.37	1.73472347597681e-18\\
51.38	0\\
51.39	0\\
51.4	0\\
51.41	0\\
51.42	1.73472347597681e-18\\
51.43	0\\
51.44	0\\
51.45	0\\
51.46	1.73472347597681e-18\\
51.47	0\\
51.48	0\\
51.49	1.73472347597681e-18\\
51.5	0\\
51.51	0\\
51.52	0\\
51.53	0\\
51.54	0\\
51.55	0\\
51.56	0\\
51.57	0\\
51.58	1.73472347597681e-18\\
51.59	0\\
51.6	0\\
51.61	0\\
51.62	0\\
51.63	0\\
51.64	0\\
51.65	0\\
51.66	0\\
51.67	0\\
51.68	1.73472347597681e-18\\
51.69	0\\
51.7	1.73472347597681e-18\\
51.71	0\\
51.72	0\\
51.73	0\\
51.74	0\\
51.75	0\\
51.76	0\\
51.77	0\\
51.78	0\\
51.79	0\\
51.8	0\\
51.81	0\\
51.82	0\\
51.83	0\\
51.84	0\\
51.85	0\\
51.86	0\\
51.87	0\\
51.88	0\\
51.89	0\\
51.9	0\\
51.91	0\\
51.92	0\\
51.93	1.73472347597681e-18\\
51.94	0\\
51.95	0\\
51.96	0\\
51.97	0\\
51.98	0\\
51.99	0\\
52	0\\
52.01	0\\
52.02	0\\
52.03	0\\
52.04	0\\
52.05	0\\
52.06	0\\
52.07	0\\
52.08	0\\
52.09	0\\
52.1	0\\
52.11	0\\
52.12	0\\
52.13	0\\
52.14	0\\
52.15	0\\
52.16	1.73472347597681e-18\\
52.17	1.73472347597681e-18\\
52.18	0\\
52.19	0\\
52.2	0\\
52.21	0\\
52.22	1.73472347597681e-18\\
52.23	0\\
52.24	0\\
52.25	0\\
52.26	0\\
52.27	0\\
52.28	0\\
52.29	0\\
52.3	0\\
52.31	0\\
52.32	0\\
52.33	0\\
52.34	0\\
52.35	0\\
52.36	0\\
52.37	1.73472347597681e-18\\
52.38	0\\
52.39	1.73472347597681e-18\\
52.4	1.73472347597681e-18\\
52.41	0\\
52.42	0\\
52.43	0\\
52.44	0\\
52.45	0\\
52.46	0\\
52.47	0\\
52.48	0\\
52.49	0\\
52.5	0\\
52.51	0\\
52.52	0\\
52.53	0\\
52.54	0\\
52.55	0\\
52.56	0\\
52.57	0\\
52.58	0\\
52.59	0\\
52.6	0\\
52.61	1.73472347597681e-18\\
52.62	1.73472347597681e-18\\
52.63	0\\
52.64	1.73472347597681e-18\\
52.65	0\\
52.66	0\\
52.67	0\\
52.68	0\\
52.69	0\\
52.7	0\\
52.71	0\\
52.72	1.73472347597681e-18\\
52.73	0\\
52.74	0\\
52.75	0\\
52.76	0\\
52.77	0\\
52.78	0\\
52.79	0\\
52.8	0\\
52.81	0\\
52.82	0\\
52.83	0\\
52.84	0\\
52.85	1.73472347597681e-18\\
52.86	0\\
52.87	1.73472347597681e-18\\
52.88	0\\
52.89	0\\
52.9	1.73472347597681e-18\\
52.91	0\\
52.92	0\\
52.93	0\\
52.94	0\\
52.95	1.73472347597681e-18\\
52.96	0\\
52.97	0\\
52.98	0\\
52.99	0\\
53	0\\
53.01	1.73472347597681e-18\\
53.02	0\\
53.03	0\\
53.04	0\\
53.05	0\\
53.06	0\\
53.07	0\\
53.08	0\\
53.09	0\\
53.1	1.73472347597681e-18\\
53.11	1.73472347597681e-18\\
53.12	0\\
53.13	0\\
53.14	0\\
53.15	0\\
53.16	0\\
53.17	0\\
53.18	0\\
53.19	0\\
53.2	1.73472347597681e-18\\
53.21	0\\
53.22	0\\
53.23	0\\
53.24	0\\
53.25	0\\
53.26	0\\
53.27	1.73472347597681e-18\\
53.28	0\\
53.29	0\\
53.3	0\\
53.31	0\\
53.32	0\\
53.33	0\\
53.34	0\\
53.35	0\\
53.36	0\\
53.37	0\\
53.38	0\\
53.39	0\\
53.4	0\\
53.41	0\\
53.42	0\\
53.43	0\\
53.44	0\\
53.45	1.73472347597681e-18\\
53.46	0\\
53.47	0\\
53.48	0\\
53.49	0\\
53.5	0\\
53.51	0\\
53.52	1.73472347597681e-18\\
53.53	1.73472347597681e-18\\
53.54	0\\
53.55	0\\
53.56	0\\
53.57	0\\
53.58	1.73472347597681e-18\\
53.59	0\\
53.6	0\\
53.61	0\\
53.62	0\\
53.63	0\\
53.64	0\\
53.65	0\\
53.66	0\\
53.67	0\\
53.68	0\\
53.69	0\\
53.7	0\\
53.71	0\\
53.72	0\\
53.73	0\\
53.74	0\\
53.75	0\\
53.76	0\\
53.77	0\\
53.78	0\\
53.79	1.73472347597681e-18\\
53.8	0\\
53.81	0\\
53.82	0\\
53.83	0\\
53.84	0\\
53.85	1.73472347597681e-18\\
53.86	0\\
53.87	0\\
53.88	0\\
53.89	0\\
53.9	0\\
53.91	0\\
53.92	0\\
53.93	0\\
53.94	0\\
53.95	0\\
53.96	0\\
53.97	0\\
53.98	0\\
53.99	0\\
54	0\\
54.01	0\\
54.02	0\\
54.03	0\\
54.04	0\\
54.05	0\\
54.06	0\\
54.07	1.73472347597681e-18\\
54.08	0\\
54.09	1.73472347597681e-18\\
54.1	0\\
54.11	1.73472347597681e-18\\
54.12	0\\
54.13	1.73472347597681e-18\\
54.14	0\\
54.15	1.73472347597681e-18\\
54.16	0\\
54.17	0\\
54.18	0\\
54.19	0\\
54.2	1.73472347597681e-18\\
54.21	1.73472347597681e-18\\
54.22	0\\
54.23	0\\
54.24	0\\
54.25	0\\
54.26	0\\
54.27	0\\
54.28	0\\
54.29	0\\
54.3	0\\
54.31	0\\
54.32	1.73472347597681e-18\\
54.33	0\\
54.34	0\\
54.35	0\\
54.36	0\\
54.37	0\\
54.38	0\\
54.39	0\\
54.4	0\\
54.41	0\\
54.42	0\\
54.43	0\\
54.44	0\\
54.45	1.73472347597681e-18\\
54.46	0\\
54.47	0\\
54.48	0\\
54.49	0\\
54.5	0\\
54.51	0\\
54.52	0\\
54.53	0\\
54.54	0\\
54.55	0\\
54.56	0\\
54.57	0\\
54.58	0\\
54.59	0\\
54.6	0\\
54.61	0\\
54.62	0\\
54.63	0\\
54.64	0\\
54.65	1.73472347597681e-18\\
54.66	0\\
54.67	0\\
54.68	0\\
54.69	1.73472347597681e-18\\
54.7	0\\
54.71	0\\
54.72	0\\
54.73	0\\
54.74	0\\
54.75	0\\
54.76	0\\
54.77	0\\
54.78	0\\
54.79	0\\
54.8	0\\
54.81	0\\
54.82	0\\
54.83	0\\
54.84	0\\
54.85	0\\
54.86	0\\
54.87	0\\
54.88	0\\
54.89	0\\
54.9	1.73472347597681e-18\\
54.91	0\\
54.92	0\\
54.93	0\\
54.94	1.73472347597681e-18\\
54.95	0\\
54.96	0\\
54.97	0\\
54.98	1.73472347597681e-18\\
54.99	0\\
55	0\\
55.01	0\\
55.02	0\\
55.03	0\\
55.04	0\\
55.05	0\\
55.06	0\\
55.07	1.73472347597681e-18\\
55.08	1.73472347597681e-18\\
55.09	0\\
55.1	0\\
55.11	0\\
55.12	0\\
55.13	0\\
55.14	1.73472347597681e-18\\
55.15	0\\
55.16	1.73472347597681e-18\\
55.17	1.73472347597681e-18\\
55.18	0\\
55.19	1.73472347597681e-18\\
55.2	0\\
55.21	0\\
55.22	0\\
55.23	0\\
55.24	0\\
55.25	0\\
55.26	0\\
55.27	0\\
55.28	0\\
55.29	0\\
55.3	0\\
55.31	0\\
55.32	0\\
55.33	0\\
55.34	0\\
55.35	0\\
55.36	1.73472347597681e-18\\
55.37	0\\
55.38	0\\
55.39	0\\
55.4	0\\
55.41	0\\
55.42	0\\
55.43	0\\
55.44	0\\
55.45	0\\
55.46	0\\
55.47	1.73472347597681e-18\\
55.48	0\\
55.49	1.73472347597681e-18\\
55.5	0\\
55.51	0\\
55.52	0\\
55.53	0\\
55.54	0\\
55.55	0\\
55.56	1.73472347597681e-18\\
55.57	0\\
55.58	0\\
55.59	0\\
55.6	0\\
55.61	0\\
55.62	0\\
55.63	0\\
55.64	0\\
55.65	0\\
55.66	0\\
55.67	0\\
55.68	0\\
55.69	0\\
55.7	0\\
55.71	0\\
55.72	0\\
55.73	1.73472347597681e-18\\
55.74	0\\
55.75	0\\
55.76	0\\
55.77	1.73472347597681e-18\\
55.78	0\\
55.79	0\\
55.8	0\\
55.81	0\\
55.82	0\\
55.83	0\\
55.84	0\\
55.85	0\\
55.86	0\\
55.87	0\\
55.88	0\\
55.89	0\\
55.9	0\\
55.91	0\\
55.92	0\\
55.93	0\\
55.94	0\\
55.95	1.73472347597681e-18\\
55.96	0\\
55.97	0\\
55.98	0\\
55.99	0\\
56	0\\
56.01	1.73472347597681e-18\\
56.02	0\\
56.03	0\\
56.04	1.73472347597681e-18\\
56.05	1.73472347597681e-18\\
56.06	0\\
56.07	1.73472347597681e-18\\
56.08	0\\
56.09	0\\
56.1	0\\
56.11	0\\
56.12	0\\
56.13	0\\
56.14	0\\
56.15	0\\
56.16	0\\
56.17	0\\
56.18	0\\
56.19	0\\
56.2	0\\
56.21	0\\
56.22	0\\
56.23	0\\
56.24	1.73472347597681e-18\\
56.25	0\\
56.26	0\\
56.27	0\\
56.28	0\\
56.29	1.73472347597681e-18\\
56.3	0\\
56.31	0\\
56.32	0\\
56.33	0\\
56.34	0\\
56.35	0\\
56.36	0\\
56.37	0\\
56.38	0\\
56.39	0\\
56.4	0\\
56.41	0\\
56.42	0\\
56.43	0\\
56.44	0\\
56.45	0\\
56.46	0\\
56.47	0\\
56.48	0\\
56.49	0\\
56.5	0\\
56.51	0\\
56.52	0\\
56.53	0\\
56.54	0\\
56.55	0\\
56.56	0\\
56.57	0\\
56.58	0\\
56.59	0\\
56.6	0\\
56.61	0\\
56.62	0\\
56.63	0\\
56.64	0\\
56.65	0\\
56.66	0\\
56.67	0\\
56.68	0\\
56.69	0\\
56.7	1.73472347597681e-18\\
56.71	1.73472347597681e-18\\
56.72	0\\
56.73	0\\
56.74	1.73472347597681e-18\\
56.75	1.73472347597681e-18\\
56.76	0\\
56.77	0\\
56.78	0\\
56.79	1.73472347597681e-18\\
56.8	0\\
56.81	0\\
56.82	1.73472347597681e-18\\
56.83	0\\
56.84	0\\
56.85	0\\
56.86	0\\
56.87	0\\
56.88	0\\
56.89	0\\
56.9	0\\
56.91	0\\
56.92	0\\
56.93	0\\
56.94	0\\
56.95	0\\
56.96	0\\
56.97	0\\
56.98	0\\
56.99	0\\
57	1.73472347597681e-18\\
57.01	0\\
57.02	0\\
57.03	0\\
57.04	0\\
57.05	1.73472347597681e-18\\
57.06	1.73472347597681e-18\\
57.07	0\\
57.08	0\\
57.09	0\\
57.1	0\\
57.11	0\\
57.12	0\\
57.13	0\\
57.14	0\\
57.15	0\\
57.16	0\\
57.17	0\\
57.18	0\\
57.19	0\\
57.2	0\\
57.21	0\\
57.22	0\\
57.23	0\\
57.24	0\\
57.25	0\\
57.26	0\\
57.27	1.73472347597681e-18\\
57.28	0\\
57.29	0\\
57.3	0\\
57.31	0\\
57.32	0\\
57.33	0\\
57.34	0\\
57.35	0\\
57.36	0\\
57.37	0\\
57.38	1.73472347597681e-18\\
57.39	0\\
57.4	0\\
57.41	0\\
57.42	0\\
57.43	0\\
57.44	0\\
57.45	0\\
57.46	0\\
57.47	0\\
57.48	0\\
57.49	0\\
57.5	0\\
57.51	1.73472347597681e-18\\
57.52	0\\
57.53	0\\
57.54	0\\
57.55	0\\
57.56	0\\
57.57	0\\
57.58	0\\
57.59	0\\
57.6	0\\
57.61	0\\
57.62	0\\
57.63	0\\
57.64	0\\
57.65	0\\
57.66	0\\
57.67	0\\
57.68	1.73472347597681e-18\\
57.69	1.73472347597681e-18\\
57.7	0\\
57.71	0\\
57.72	0\\
57.73	0\\
57.74	0\\
57.75	0\\
57.76	0\\
57.77	1.73472347597681e-18\\
57.78	0\\
57.79	0\\
57.8	0\\
57.81	0\\
57.82	0\\
57.83	0\\
57.84	1.73472347597681e-18\\
57.85	1.73472347597681e-18\\
57.86	0\\
57.87	1.73472347597681e-18\\
57.88	0\\
57.89	0\\
57.9	0\\
57.91	0\\
57.92	0\\
57.93	0\\
57.94	0\\
57.95	0\\
57.96	0\\
57.97	0\\
57.98	0\\
57.99	0\\
58	1.73472347597681e-18\\
58.01	0\\
58.02	0\\
58.03	0\\
58.04	0\\
58.05	0\\
58.06	0\\
58.07	0\\
58.08	0\\
58.09	0\\
58.1	0\\
58.11	1.73472347597681e-18\\
58.12	0\\
58.13	1.73472347597681e-18\\
58.14	0\\
58.15	0\\
58.16	0\\
58.17	0\\
58.18	0\\
58.19	0\\
58.2	0\\
58.21	0\\
58.22	1.73472347597681e-18\\
58.23	0\\
58.24	0\\
58.25	1.73472347597681e-18\\
58.26	0\\
58.27	0\\
58.28	0\\
58.29	1.73472347597681e-18\\
58.3	0\\
58.31	1.73472347597681e-18\\
58.32	0\\
58.33	0\\
58.34	1.73472347597681e-18\\
58.35	1.73472347597681e-18\\
58.36	0\\
58.37	0\\
58.38	0\\
58.39	0\\
58.4	0\\
58.41	1.73472347597681e-18\\
58.42	1.73472347597681e-18\\
58.43	0\\
58.44	0\\
58.45	0\\
58.46	0\\
58.47	0\\
58.48	0\\
58.49	0\\
58.5	0\\
58.51	0\\
58.52	1.73472347597681e-18\\
58.53	0\\
58.54	0\\
58.55	0\\
58.56	0\\
58.57	0\\
58.58	0\\
58.59	0\\
58.6	0\\
58.61	0\\
58.62	0\\
58.63	0\\
58.64	0\\
58.65	0\\
58.66	0\\
58.67	1.73472347597681e-18\\
58.68	0\\
58.69	0\\
58.7	0\\
58.71	0\\
58.72	0\\
58.73	0\\
58.74	1.73472347597681e-18\\
58.75	0\\
58.76	0\\
58.77	0\\
58.78	0\\
58.79	1.73472347597681e-18\\
58.8	0\\
58.81	0\\
58.82	0\\
58.83	0\\
58.84	0\\
58.85	0\\
58.86	0\\
58.87	0\\
58.88	1.73472347597681e-18\\
58.89	0\\
58.9	0\\
58.91	0\\
58.92	0\\
58.93	0\\
58.94	0\\
58.95	0\\
58.96	0\\
58.97	0\\
58.98	1.73472347597681e-18\\
58.99	0\\
59	0\\
59.01	0\\
59.02	0\\
59.03	0\\
59.04	0\\
59.05	0\\
59.06	0\\
59.07	0\\
59.08	0\\
59.09	0\\
59.1	0\\
59.11	0\\
59.12	0\\
59.13	0\\
59.14	0\\
59.15	0\\
59.16	0\\
59.17	1.73472347597681e-18\\
59.18	0\\
59.19	0\\
59.2	1.73472347597681e-18\\
59.21	0\\
59.22	0\\
59.23	0\\
59.24	0\\
59.25	1.73472347597681e-18\\
59.26	0\\
59.27	0\\
59.28	0\\
59.29	0\\
59.3	1.73472347597681e-18\\
59.31	0\\
59.32	1.73472347597681e-18\\
59.33	0\\
59.34	0\\
59.35	0\\
59.36	0\\
59.37	0\\
59.38	0\\
59.39	0\\
59.4	0\\
59.41	1.73472347597681e-18\\
59.42	1.73472347597681e-18\\
59.43	0\\
59.44	0\\
59.45	0\\
59.46	0\\
59.47	0\\
59.48	0\\
59.49	0\\
59.5	1.73472347597681e-18\\
59.51	1.73472347597681e-18\\
59.52	0\\
59.53	0\\
59.54	0\\
59.55	0\\
59.56	0\\
59.57	0\\
59.58	0\\
59.59	1.73472347597681e-18\\
59.6	0\\
59.61	0\\
59.62	1.73472347597681e-18\\
59.63	0\\
59.64	0\\
59.65	0\\
59.66	1.73472347597681e-18\\
59.67	0\\
59.68	0\\
59.69	0\\
59.7	0\\
59.71	0\\
59.72	0\\
59.73	0\\
59.74	0\\
59.75	0\\
59.76	0\\
59.77	0\\
59.78	0\\
59.79	0\\
59.8	1.73472347597681e-18\\
59.81	0\\
59.82	0\\
59.83	0\\
59.84	0\\
59.85	0\\
59.86	1.73472347597681e-18\\
59.87	0\\
59.88	1.73472347597681e-18\\
59.89	0\\
59.9	0\\
59.91	0\\
59.92	0\\
59.93	1.73472347597681e-18\\
59.94	0\\
59.95	0\\
59.96	0\\
59.97	0\\
59.98	0\\
59.99	0\\
60	0\\
60.01	1.73472347597681e-18\\
60.02	0\\
60.03	0\\
60.04	0\\
60.05	0\\
60.06	0\\
60.07	0\\
60.08	0\\
60.09	0\\
60.1	0\\
60.11	0\\
60.12	0\\
60.13	1.73472347597681e-18\\
60.14	0\\
60.15	0\\
60.16	0\\
60.17	0\\
60.18	0\\
60.19	0\\
60.2	0\\
60.21	0\\
60.22	0\\
60.23	0\\
60.24	0\\
60.25	0\\
60.26	0\\
60.27	0\\
60.28	0\\
60.29	0\\
60.3	0\\
60.31	0\\
60.32	0\\
60.33	0\\
60.34	0\\
60.35	0\\
60.36	0\\
60.37	0\\
60.38	0\\
60.39	1.73472347597681e-18\\
60.4	0\\
60.41	0\\
60.42	0\\
60.43	0\\
60.44	0\\
60.45	0\\
60.46	0\\
60.47	0\\
60.48	0\\
60.49	1.73472347597681e-18\\
60.5	0\\
60.51	0\\
60.52	0\\
60.53	0\\
60.54	0\\
60.55	0\\
60.56	0\\
60.57	0\\
60.58	0\\
60.59	0\\
60.6	0\\
60.61	0\\
60.62	0\\
60.63	0\\
60.64	0\\
60.65	0\\
60.66	1.73472347597681e-18\\
60.67	0\\
60.68	0\\
60.69	0\\
60.7	0\\
60.71	0\\
60.72	0\\
60.73	0\\
60.74	0\\
60.75	0\\
60.76	0\\
60.77	1.73472347597681e-18\\
60.78	0\\
60.79	0\\
60.8	0\\
60.81	0\\
60.82	0\\
60.83	0\\
60.84	0\\
60.85	0\\
60.86	0\\
60.87	0\\
60.88	0\\
60.89	0\\
60.9	0\\
60.91	0\\
60.92	0\\
60.93	0\\
60.94	0\\
60.95	0\\
60.96	0\\
60.97	0\\
60.98	0\\
60.99	0\\
61	0\\
61.01	1.73472347597681e-18\\
61.02	0\\
61.03	0\\
61.04	0\\
61.05	0\\
61.06	1.73472347597681e-18\\
61.07	0\\
61.08	0\\
61.09	0\\
61.1	0\\
61.11	0\\
61.12	0\\
61.13	0\\
61.14	1.73472347597681e-18\\
61.15	0\\
61.16	0\\
61.17	0\\
61.18	0\\
61.19	0\\
61.2	0\\
61.21	0\\
61.22	0\\
61.23	0\\
61.24	0\\
61.25	0\\
61.26	0\\
61.27	0\\
61.28	1.73472347597681e-18\\
61.29	0\\
61.3	0\\
61.31	0\\
61.32	0\\
61.33	0\\
61.34	0\\
61.35	0\\
61.36	0\\
61.37	0\\
61.38	0\\
61.39	0\\
61.4	0\\
61.41	0\\
61.42	0\\
61.43	1.73472347597681e-18\\
61.44	0\\
61.45	0\\
61.46	1.73472347597681e-18\\
61.47	1.73472347597681e-18\\
61.48	0\\
61.49	1.73472347597681e-18\\
61.5	0\\
61.51	0\\
61.52	0\\
61.53	0\\
61.54	0\\
61.55	0\\
61.56	0\\
61.57	0\\
61.58	0\\
61.59	0\\
61.6	0\\
61.61	0\\
61.62	0\\
61.63	0\\
61.64	0\\
61.65	1.73472347597681e-18\\
61.66	0\\
61.67	1.73472347597681e-18\\
61.68	0\\
61.69	0\\
61.7	0\\
61.71	0\\
61.72	0\\
61.73	0\\
61.74	0\\
61.75	0\\
61.76	0\\
61.77	1.73472347597681e-18\\
61.78	0\\
61.79	0\\
61.8	0\\
61.81	0\\
61.82	1.73472347597681e-18\\
61.83	0\\
61.84	1.73472347597681e-18\\
61.85	0\\
61.86	0\\
61.87	0\\
61.88	1.73472347597681e-18\\
61.89	0\\
61.9	0\\
61.91	0\\
61.92	0\\
61.93	0\\
61.94	0\\
61.95	0\\
61.96	0\\
61.97	1.73472347597681e-18\\
61.98	0\\
61.99	0\\
62	0\\
62.01	0\\
62.02	0\\
62.03	0\\
62.04	1.73472347597681e-18\\
62.05	0\\
62.06	0\\
62.07	0\\
62.08	0\\
62.09	0\\
62.1	0\\
62.11	0\\
62.12	0\\
62.13	0\\
62.14	0\\
62.15	0\\
62.16	0\\
62.17	1.73472347597681e-18\\
62.18	0\\
62.19	0\\
62.2	0\\
62.21	0\\
62.22	0\\
62.23	0\\
62.24	0\\
62.25	0\\
62.26	0\\
62.27	0\\
62.28	0\\
62.29	0\\
62.3	0\\
62.31	0\\
62.32	0\\
62.33	0\\
62.34	0\\
62.35	0\\
62.36	0\\
62.37	0\\
62.38	0\\
62.39	0\\
62.4	0\\
62.41	0\\
62.42	0\\
62.43	0\\
62.44	0\\
62.45	0\\
62.46	1.73472347597681e-18\\
62.47	0\\
62.48	0\\
62.49	1.73472347597681e-18\\
62.5	0\\
62.51	0\\
62.52	0\\
62.53	0\\
62.54	0\\
62.55	0\\
62.56	0\\
62.57	0\\
62.58	0\\
62.59	0\\
62.6	0\\
62.61	0\\
62.62	0\\
62.63	0\\
62.64	0\\
62.65	0\\
62.66	0\\
62.67	0\\
62.68	0\\
62.69	0\\
62.7	0\\
62.71	0\\
62.72	0\\
62.73	0\\
62.74	0\\
62.75	0\\
62.76	0\\
62.77	1.73472347597681e-18\\
62.78	0\\
62.79	0\\
62.8	0\\
62.81	0\\
62.82	0\\
62.83	0\\
62.84	1.73472347597681e-18\\
62.85	0\\
62.86	0\\
62.87	0\\
62.88	0\\
62.89	0\\
62.9	0\\
62.91	0\\
62.92	0\\
62.93	0\\
62.94	0\\
62.95	0\\
62.96	0\\
62.97	0\\
62.98	0\\
62.99	1.73472347597681e-18\\
63	0\\
63.01	0\\
63.02	1.73472347597681e-18\\
63.03	0\\
63.04	0\\
63.05	1.73472347597681e-18\\
63.06	0\\
63.07	0\\
63.08	0\\
63.09	0\\
63.1	0\\
63.11	0\\
63.12	0\\
63.13	0\\
63.14	0\\
63.15	0\\
63.16	0\\
63.17	0\\
63.18	1.73472347597681e-18\\
63.19	0\\
63.2	0\\
63.21	0\\
63.22	0\\
63.23	0\\
63.24	0\\
63.25	0\\
63.26	0\\
63.27	0\\
63.28	0\\
63.29	0\\
63.3	0\\
63.31	0\\
63.32	0\\
63.33	0\\
63.34	0\\
63.35	0\\
63.36	0\\
63.37	1.73472347597681e-18\\
63.38	0\\
63.39	0\\
63.4	0\\
63.41	0\\
63.42	0\\
63.43	0\\
63.44	0\\
63.45	0\\
63.46	0\\
63.47	0\\
63.48	0\\
63.49	1.73472347597681e-18\\
63.5	0\\
63.51	0\\
63.52	0\\
63.53	0\\
63.54	0\\
63.55	0\\
63.56	0\\
63.57	0\\
63.58	0\\
63.59	0\\
63.6	0\\
63.61	0\\
63.62	0\\
63.63	0\\
63.64	0\\
63.65	0\\
63.66	0\\
63.67	0\\
63.68	0\\
63.69	0\\
63.7	0\\
63.71	0\\
63.72	0\\
63.73	0\\
63.74	0\\
63.75	0\\
63.76	0\\
63.77	0\\
63.78	0\\
63.79	0\\
63.8	1.73472347597681e-18\\
63.81	0\\
63.82	0\\
63.83	0\\
63.84	0\\
63.85	0\\
63.86	0\\
63.87	0\\
63.88	0\\
63.89	0\\
63.9	0\\
63.91	0\\
63.92	1.73472347597681e-18\\
63.93	0\\
63.94	0\\
63.95	0\\
63.96	1.73472347597681e-18\\
63.97	1.73472347597681e-18\\
63.98	0\\
63.99	0\\
64	0\\
64.01	0\\
64.02	0\\
64.03	0\\
64.04	0\\
64.05	0\\
64.06	1.73472347597681e-18\\
64.07	0\\
64.08	0\\
64.09	0\\
64.1	0\\
64.11	0\\
64.12	0\\
64.13	1.73472347597681e-18\\
64.14	0\\
64.15	0\\
64.16	0\\
64.17	0\\
64.18	0\\
64.19	0\\
64.2	0\\
64.21	0\\
64.22	0\\
64.23	0\\
64.24	0\\
64.25	0\\
64.26	0\\
64.27	0\\
64.28	0\\
64.29	1.73472347597681e-18\\
64.3	0\\
64.31	0\\
64.32	0\\
64.33	0\\
64.34	1.73472347597681e-18\\
64.35	0\\
64.36	0\\
64.37	0\\
64.38	0\\
64.39	0\\
64.4	0\\
64.41	0\\
64.42	1.73472347597681e-18\\
64.43	0\\
64.44	0\\
64.45	0\\
64.46	1.73472347597681e-18\\
64.47	0\\
64.48	0\\
64.49	0\\
64.5	0\\
64.51	0\\
64.52	0\\
64.53	0\\
64.54	0\\
64.55	0\\
64.56	0\\
64.57	0\\
64.58	0\\
64.59	1.73472347597681e-18\\
64.6	0\\
64.61	0\\
64.62	0\\
64.63	0\\
64.64	0\\
64.65	1.73472347597681e-18\\
64.66	0\\
64.67	0\\
64.68	0\\
64.69	0\\
64.7	0\\
64.71	0\\
64.72	0\\
64.73	0\\
64.74	0\\
64.75	1.73472347597681e-18\\
64.76	0\\
64.77	0\\
64.78	0\\
64.79	0\\
64.8	0\\
64.81	0\\
64.82	0\\
64.83	1.73472347597681e-18\\
64.84	1.73472347597681e-18\\
64.85	0\\
64.86	0\\
64.87	1.73472347597681e-18\\
64.88	0\\
64.89	0\\
64.9	0\\
64.91	0\\
64.92	0\\
64.93	0\\
64.94	0\\
64.95	0\\
64.96	0\\
64.97	1.73472347597681e-18\\
64.98	0\\
64.99	1.73472347597681e-18\\
65	0\\
65.01	0\\
65.02	0\\
65.03	0\\
65.04	0\\
65.05	0\\
65.06	0\\
65.07	1.73472347597681e-18\\
65.08	0\\
65.09	0\\
65.1	0\\
65.11	0\\
65.12	0\\
65.13	0\\
65.14	0\\
65.15	0\\
65.16	0\\
65.17	0\\
65.18	1.73472347597681e-18\\
65.19	0\\
65.2	0\\
65.21	0\\
65.22	0\\
65.23	0\\
65.24	0\\
65.25	0\\
65.26	0\\
65.27	0\\
65.28	0\\
65.29	0\\
65.3	0\\
65.31	0\\
65.32	0\\
65.33	0\\
65.34	0\\
65.35	0\\
65.36	0\\
65.37	1.73472347597681e-18\\
65.38	0\\
65.39	0\\
65.4	0\\
65.41	0\\
65.42	0\\
65.43	0\\
65.44	0\\
65.45	0\\
65.46	0\\
65.47	0\\
65.48	0\\
65.49	0\\
65.5	0\\
65.51	0\\
65.52	0\\
65.53	0\\
65.54	0\\
65.55	0\\
65.56	0\\
65.57	0\\
65.58	0\\
65.59	0\\
65.6	0\\
65.61	0\\
65.62	0\\
65.63	0\\
65.64	0\\
65.65	1.73472347597681e-18\\
65.66	0\\
65.67	0\\
65.68	1.73472347597681e-18\\
65.69	0\\
65.7	0\\
65.71	0\\
65.72	0\\
65.73	0\\
65.74	0\\
65.75	0\\
65.76	0\\
65.77	0\\
65.78	0\\
65.79	0\\
65.8	1.73472347597681e-18\\
65.81	1.73472347597681e-18\\
65.82	0\\
65.83	1.73472347597681e-18\\
65.84	0\\
65.85	0\\
65.86	1.73472347597681e-18\\
65.87	0\\
65.88	0\\
65.89	0\\
65.9	0\\
65.91	0\\
65.92	1.73472347597681e-18\\
65.93	1.73472347597681e-18\\
65.94	0\\
65.95	0\\
65.96	0\\
65.97	0\\
65.98	0\\
65.99	0\\
66	0\\
66.01	0\\
66.02	0\\
66.03	0\\
66.04	0\\
66.05	0\\
66.06	1.73472347597681e-18\\
66.07	0\\
66.08	0\\
66.09	0\\
66.1	0\\
66.11	0\\
66.12	0\\
66.13	0\\
66.14	0\\
66.15	1.73472347597681e-18\\
66.16	0\\
66.17	0\\
66.18	0\\
66.19	0\\
66.2	0\\
66.21	0\\
66.22	0\\
66.23	1.73472347597681e-18\\
66.24	0\\
66.25	0\\
66.26	0\\
66.27	0\\
66.28	0\\
66.29	1.73472347597681e-18\\
66.3	0\\
66.31	0\\
66.32	0\\
66.33	0\\
66.34	0\\
66.35	1.73472347597681e-18\\
66.36	0\\
66.37	1.73472347597681e-18\\
66.38	0\\
66.39	0\\
66.4	0\\
66.41	0\\
66.42	0\\
66.43	0\\
66.44	0\\
66.45	0\\
66.46	0\\
66.47	0\\
66.48	0\\
66.49	1.73472347597681e-18\\
66.5	0\\
66.51	0\\
66.52	0\\
66.53	0\\
66.54	0\\
66.55	0\\
66.56	0\\
66.57	1.73472347597681e-18\\
66.58	0\\
66.59	0\\
66.6	0\\
66.61	0\\
66.62	0\\
66.63	0\\
66.64	0\\
66.65	1.73472347597681e-18\\
66.66	0\\
66.67	0\\
66.68	1.73472347597681e-18\\
66.69	0\\
66.7	0\\
66.71	0\\
66.72	0\\
66.73	0\\
66.74	0\\
66.75	0\\
66.76	0\\
66.77	0\\
66.78	0\\
66.79	0\\
66.8	0\\
66.81	0\\
66.82	0\\
66.83	0\\
66.84	0\\
66.85	0\\
66.86	0\\
66.87	0\\
66.88	0\\
66.89	0\\
66.9	0\\
66.91	1.73472347597681e-18\\
66.92	0\\
66.93	0\\
66.94	1.73472347597681e-18\\
66.95	0\\
66.96	0\\
66.97	1.73472347597681e-18\\
66.98	0\\
66.99	0\\
67	0\\
67.01	0\\
67.02	1.73472347597681e-18\\
67.03	0\\
67.04	0\\
67.05	1.73472347597681e-18\\
67.06	0\\
67.07	0\\
67.08	1.73472347597681e-18\\
67.09	0\\
67.1	0\\
67.11	0\\
67.12	0\\
67.13	0\\
67.14	0\\
67.15	0\\
67.16	0\\
67.17	0\\
67.18	0\\
67.19	0\\
67.2	0\\
67.21	0\\
67.22	0\\
67.23	0\\
67.24	0\\
67.25	0\\
67.26	0\\
67.27	0\\
67.28	0\\
67.29	0\\
67.3	0\\
67.31	0\\
67.32	0\\
67.33	0\\
67.34	0\\
67.35	0\\
67.36	1.73472347597681e-18\\
67.37	0\\
67.38	0\\
67.39	0\\
67.4	0\\
67.41	1.73472347597681e-18\\
67.42	0\\
67.43	0\\
67.44	0\\
67.45	0\\
67.46	1.73472347597681e-18\\
67.47	0\\
67.48	0\\
67.49	0\\
67.5	1.73472347597681e-18\\
67.51	0\\
67.52	0\\
67.53	0\\
67.54	0\\
67.55	0\\
67.56	0\\
67.57	0\\
67.58	0\\
67.59	0\\
67.6	0\\
67.61	1.73472347597681e-18\\
67.62	0\\
67.63	0\\
67.64	1.73472347597681e-18\\
67.65	0\\
67.66	0\\
67.67	0\\
67.68	0\\
67.69	0\\
67.7	0\\
67.71	0\\
67.72	0\\
67.73	0\\
67.74	0\\
67.75	0\\
67.76	0\\
67.77	0\\
67.78	0\\
67.79	0\\
67.8	0\\
67.81	0\\
67.82	0\\
67.83	0\\
67.84	0\\
67.85	1.73472347597681e-18\\
67.86	0\\
67.87	0\\
67.88	0\\
67.89	0\\
67.9	0\\
67.91	0\\
67.92	0\\
67.93	0\\
67.94	0\\
67.95	0\\
67.96	0\\
67.97	0\\
67.98	0\\
67.99	0\\
68	0\\
68.01	0\\
68.02	1.73472347597681e-18\\
68.03	0\\
68.04	0\\
68.05	0\\
68.06	0\\
68.07	0\\
68.08	0\\
68.09	1.73472347597681e-18\\
68.1	0\\
68.11	0\\
68.12	0\\
68.13	0\\
68.14	0\\
68.15	0\\
68.16	0\\
68.17	1.73472347597681e-18\\
68.18	0\\
68.19	1.73472347597681e-18\\
68.2	0\\
68.21	0\\
68.22	0\\
68.23	0\\
68.24	0\\
68.25	0\\
68.26	0\\
68.27	0\\
68.28	0\\
68.29	0\\
68.3	1.73472347597681e-18\\
68.31	0\\
68.32	0\\
68.33	0\\
68.34	0\\
68.35	0\\
68.36	1.73472347597681e-18\\
68.37	0\\
68.38	0\\
68.39	0\\
68.4	0\\
68.41	0\\
68.42	0\\
68.43	1.73472347597681e-18\\
68.44	0\\
68.45	0\\
68.46	0\\
68.47	0\\
68.48	0\\
68.49	0\\
68.5	0\\
68.51	0\\
68.52	0\\
68.53	0\\
68.54	0\\
68.55	0\\
68.56	0\\
68.57	0\\
68.58	0\\
68.59	0\\
68.6	0\\
68.61	0\\
68.62	0\\
68.63	0\\
68.64	0\\
68.65	0\\
68.66	0\\
68.67	0\\
68.68	0\\
68.69	0\\
68.7	0\\
68.71	1.73472347597681e-18\\
68.72	0\\
68.73	0\\
68.74	0\\
68.75	0\\
68.76	0\\
68.77	0\\
68.78	0\\
68.79	0\\
68.8	0\\
68.81	0\\
68.82	0\\
68.83	0\\
68.84	0\\
68.85	0\\
68.86	0\\
68.87	0\\
68.88	0\\
68.89	0\\
68.9	0\\
68.91	0\\
68.92	0\\
68.93	0\\
68.94	0\\
68.95	0\\
68.96	0\\
68.97	0\\
68.98	0\\
68.99	0\\
69	1.73472347597681e-18\\
69.01	0\\
69.02	0\\
69.03	0\\
69.04	0\\
69.05	1.73472347597681e-18\\
69.06	0\\
69.07	0\\
69.08	0\\
69.09	0\\
69.1	0\\
69.11	0\\
69.12	0\\
69.13	1.73472347597681e-18\\
69.14	0\\
69.15	1.73472347597681e-18\\
69.16	0\\
69.17	0\\
69.18	0\\
69.19	1.73472347597681e-18\\
69.2	0\\
69.21	1.73472347597681e-18\\
69.22	0\\
69.23	0\\
69.24	0\\
69.25	0\\
69.26	0\\
69.27	0\\
69.28	0\\
69.29	0\\
69.3	0\\
69.31	0\\
69.32	0\\
69.33	0\\
69.34	0\\
69.35	0\\
69.36	0\\
69.37	1.73472347597681e-18\\
69.38	0\\
69.39	0\\
69.4	0\\
69.41	0\\
69.42	0\\
69.43	1.73472347597681e-18\\
69.44	0\\
69.45	1.73472347597681e-18\\
69.46	0\\
69.47	0\\
69.48	0\\
69.49	0\\
69.5	0\\
69.51	0\\
69.52	0\\
69.53	0\\
69.54	0\\
69.55	0\\
69.56	0\\
69.57	1.73472347597681e-18\\
69.58	1.73472347597681e-18\\
69.59	0\\
69.6	0\\
69.61	0\\
69.62	0\\
69.63	0\\
69.64	1.73472347597681e-18\\
69.65	0\\
69.66	0\\
69.67	1.73472347597681e-18\\
69.68	0\\
69.69	0\\
69.7	0\\
69.71	0\\
69.72	0\\
69.73	0\\
69.74	0\\
69.75	0\\
69.76	0\\
69.77	0\\
69.78	0\\
69.79	0\\
69.8	0\\
69.81	0\\
69.82	0\\
69.83	0\\
69.84	0\\
69.85	0\\
69.86	0\\
69.87	0\\
69.88	1.73472347597681e-18\\
69.89	0\\
69.9	0\\
69.91	0\\
69.92	0\\
69.93	0\\
69.94	0\\
69.95	1.73472347597681e-18\\
69.96	0\\
69.97	0\\
69.98	0\\
69.99	0\\
70	0\\
70.01	0\\
70.02	0\\
70.03	0\\
70.04	0\\
70.05	0\\
70.06	0\\
70.07	0\\
70.08	0\\
70.09	0\\
70.1	1.73472347597681e-18\\
70.11	0\\
70.12	0\\
70.13	0\\
70.14	1.73472347597681e-18\\
70.15	0\\
70.16	0\\
70.17	0\\
70.18	0\\
70.19	0\\
70.2	0\\
70.21	0\\
70.22	0\\
70.23	0\\
70.24	0\\
70.25	0\\
70.26	0\\
70.27	0\\
70.28	0\\
70.29	0\\
70.3	0\\
70.31	0\\
70.32	0\\
70.33	1.73472347597681e-18\\
70.34	0\\
70.35	0\\
70.36	0\\
70.37	0\\
70.38	0\\
70.39	0\\
70.4	1.73472347597681e-18\\
70.41	0\\
70.42	0\\
70.43	0\\
70.44	0\\
70.45	0\\
70.46	1.73472347597681e-18\\
70.47	0\\
70.48	0\\
70.49	0\\
70.5	0\\
70.51	0\\
70.52	0\\
70.53	0\\
70.54	0\\
70.55	0\\
70.56	0\\
70.57	0\\
70.58	0\\
70.59	0\\
70.6	0\\
70.61	0\\
70.62	0\\
70.63	0\\
70.64	0\\
70.65	1.73472347597681e-18\\
70.66	0\\
70.67	0\\
70.68	0\\
70.69	0\\
70.7	0\\
70.71	1.73472347597681e-18\\
70.72	0\\
70.73	0\\
70.74	0\\
70.75	0\\
70.76	0\\
70.77	0\\
70.78	0\\
70.79	0\\
70.8	0\\
70.81	0\\
70.82	1.73472347597681e-18\\
70.83	0\\
70.84	0\\
70.85	1.73472347597681e-18\\
70.86	0\\
70.87	0\\
70.88	0\\
70.89	0\\
70.9	0\\
70.91	0\\
70.92	1.73472347597681e-18\\
70.93	0\\
70.94	0\\
70.95	0\\
70.96	0\\
70.97	0\\
70.98	0\\
70.99	0\\
71	0\\
71.01	0\\
71.02	0\\
71.03	0\\
71.04	0\\
71.05	0\\
71.06	0\\
71.07	0\\
71.08	1.73472347597681e-18\\
71.09	1.73472347597681e-18\\
71.1	0\\
71.11	0\\
71.12	0\\
71.13	0\\
71.14	0\\
71.15	0\\
71.16	0\\
71.17	0\\
71.18	1.73472347597681e-18\\
71.19	0\\
71.2	0\\
71.21	0\\
71.22	0\\
71.23	0\\
71.24	0\\
71.25	0\\
71.26	0\\
71.27	1.73472347597681e-18\\
71.28	0\\
71.29	0\\
71.3	0\\
71.31	0\\
71.32	1.73472347597681e-18\\
71.33	0\\
71.34	0\\
71.35	0\\
71.36	1.73472347597681e-18\\
71.37	0\\
71.38	0\\
71.39	0\\
71.4	0\\
71.41	0\\
71.42	0\\
71.43	1.73472347597681e-18\\
71.44	0\\
71.45	0\\
71.46	0\\
71.47	0\\
71.48	0\\
71.49	0\\
71.5	0\\
71.51	0\\
71.52	1.73472347597681e-18\\
71.53	0\\
71.54	0\\
71.55	0\\
71.56	0\\
71.57	0\\
71.58	0\\
71.59	0\\
71.6	0\\
71.61	1.73472347597681e-18\\
71.62	0\\
71.63	1.73472347597681e-18\\
71.64	0\\
71.65	0\\
71.66	0\\
71.67	0\\
71.68	0\\
71.69	0\\
71.7	0\\
71.71	1.73472347597681e-18\\
71.72	0\\
71.73	0\\
71.74	0\\
71.75	0\\
71.76	0\\
71.77	0\\
71.78	0\\
71.79	0\\
71.8	0\\
71.81	0\\
71.82	0\\
71.83	1.73472347597681e-18\\
71.84	0\\
71.85	0\\
71.86	0\\
71.87	0\\
71.88	0\\
71.89	1.73472347597681e-18\\
71.9	0\\
71.91	0\\
71.92	0\\
71.93	0\\
71.94	0\\
71.95	0\\
71.96	1.73472347597681e-18\\
71.97	0\\
71.98	0\\
71.99	0\\
72	0\\
72.01	0\\
72.02	0\\
72.03	0\\
72.04	0\\
72.05	0\\
72.06	0\\
72.07	0\\
72.08	0\\
72.09	0\\
72.1	0\\
72.11	0\\
72.12	1.73472347597681e-18\\
72.13	0\\
72.14	0\\
72.15	0\\
72.16	0\\
72.17	0\\
72.18	0\\
72.19	1.73472347597681e-18\\
72.2	0\\
72.21	0\\
72.22	0\\
72.23	0\\
72.24	0\\
72.25	0\\
72.26	0\\
72.27	0\\
72.28	0\\
72.29	0\\
72.3	0\\
72.31	0\\
72.32	0\\
72.33	0\\
72.34	0\\
72.35	0\\
72.36	0\\
72.37	0\\
72.38	0\\
72.39	0\\
72.4	0\\
72.41	0\\
72.42	0\\
72.43	0\\
72.44	0\\
72.45	1.73472347597681e-18\\
72.46	0\\
72.47	0\\
72.48	0\\
72.49	0\\
72.5	0\\
72.51	0\\
72.52	0\\
72.53	0\\
72.54	0\\
72.55	1.73472347597681e-18\\
72.56	0\\
72.57	0\\
72.58	0\\
72.59	0\\
72.6	0\\
72.61	0\\
72.62	1.73472347597681e-18\\
72.63	0\\
72.64	0\\
72.65	0\\
72.66	0\\
72.67	0\\
72.68	0\\
72.69	0\\
72.7	1.73472347597681e-18\\
72.71	0\\
72.72	0\\
72.73	1.73472347597681e-18\\
72.74	0\\
72.75	0\\
72.76	0\\
72.77	0\\
72.78	1.73472347597681e-18\\
72.79	0\\
72.8	0\\
72.81	0\\
72.82	0\\
72.83	0\\
72.84	0\\
72.85	1.73472347597681e-18\\
72.86	0\\
72.87	0\\
72.88	0\\
72.89	1.73472347597681e-18\\
72.9	0\\
72.91	0\\
72.92	0\\
72.93	0\\
72.94	0\\
72.95	0\\
72.96	0\\
72.97	0\\
72.98	0\\
72.99	0\\
73	0\\
73.01	0\\
73.02	1.73472347597681e-18\\
73.03	0\\
73.04	0\\
73.05	0\\
73.06	0\\
73.07	0\\
73.08	0\\
73.09	0\\
73.1	0\\
73.11	0\\
73.12	0\\
73.13	1.73472347597681e-18\\
73.14	0\\
73.15	0\\
73.16	0\\
73.17	0\\
73.18	0\\
73.19	0\\
73.2	0\\
73.21	0\\
73.22	0\\
73.23	1.73472347597681e-18\\
73.24	0\\
73.25	0\\
73.26	0\\
73.27	1.73472347597681e-18\\
73.28	0\\
73.29	0\\
73.3	0\\
73.31	0\\
73.32	0\\
73.33	0\\
73.34	0\\
73.35	0\\
73.36	0\\
73.37	0\\
73.38	0\\
73.39	1.73472347597681e-18\\
73.4	0\\
73.41	0\\
73.42	0\\
73.43	0\\
73.44	0\\
73.45	0\\
73.46	0\\
73.47	0\\
73.48	0\\
73.49	0\\
73.5	0\\
73.51	1.73472347597681e-18\\
73.52	1.73472347597681e-18\\
73.53	0\\
73.54	0\\
73.55	1.73472347597681e-18\\
73.56	0\\
73.57	0\\
73.58	0\\
73.59	0\\
73.6	0\\
73.61	0\\
73.62	0\\
73.63	0\\
73.64	0\\
73.65	0\\
73.66	1.73472347597681e-18\\
73.67	0\\
73.68	0\\
73.69	0\\
73.7	0\\
73.71	0\\
73.72	0\\
73.73	1.73472347597681e-18\\
73.74	0\\
73.75	0\\
73.76	1.73472347597681e-18\\
73.77	0\\
73.78	0\\
73.79	0\\
73.8	1.73472347597681e-18\\
73.81	0\\
73.82	0\\
73.83	0\\
73.84	0\\
73.85	0\\
73.86	0\\
73.87	0\\
73.88	0\\
73.89	0\\
73.9	0\\
73.91	0\\
73.92	0\\
73.93	0\\
73.94	0\\
73.95	0\\
73.96	0\\
73.97	0\\
73.98	1.73472347597681e-18\\
73.99	0\\
74	0\\
74.01	1.73472347597681e-18\\
74.02	0\\
74.03	0\\
74.04	0\\
74.05	0\\
74.06	0\\
74.07	0\\
74.08	0\\
74.09	0\\
74.1	1.73472347597681e-18\\
74.11	1.73472347597681e-18\\
74.12	1.73472347597681e-18\\
74.13	0\\
74.14	1.73472347597681e-18\\
74.15	0\\
74.16	0\\
74.17	1.73472347597681e-18\\
74.18	0\\
74.19	0\\
74.2	0\\
74.21	0\\
74.22	0\\
74.23	0\\
74.24	0\\
74.25	0\\
74.26	0\\
74.27	0\\
74.28	0\\
74.29	0\\
74.3	1.73472347597681e-18\\
74.31	0\\
74.32	0\\
74.33	0\\
74.34	0\\
74.35	0\\
74.36	0\\
74.37	1.73472347597681e-18\\
74.38	0\\
74.39	0\\
74.4	0\\
74.41	0\\
74.42	0\\
74.43	0\\
74.44	0\\
74.45	0\\
74.46	0\\
74.47	0\\
74.48	0\\
74.49	0\\
74.5	0\\
74.51	0\\
74.52	0\\
74.53	0\\
74.54	0\\
74.55	0\\
74.56	0\\
74.57	0\\
74.58	0\\
74.59	0\\
74.6	1.73472347597681e-18\\
74.61	0\\
74.62	1.73472347597681e-18\\
74.63	0\\
74.64	0\\
74.65	1.73472347597681e-18\\
74.66	0\\
74.67	0\\
74.68	0\\
74.69	0\\
74.7	1.73472347597681e-18\\
74.71	0\\
74.72	0\\
74.73	0\\
74.74	0\\
74.75	0\\
74.76	0\\
74.77	0\\
74.78	0\\
74.79	0\\
74.8	0\\
74.81	0\\
74.82	0\\
74.83	1.73472347597681e-18\\
74.84	0\\
74.85	0\\
74.86	0\\
74.87	1.73472347597681e-18\\
74.88	0\\
74.89	0\\
74.9	0\\
74.91	0\\
74.92	0\\
74.93	0\\
74.94	0\\
74.95	0\\
74.96	0\\
74.97	0\\
74.98	0\\
74.99	0\\
75	1.73472347597681e-18\\
75.01	0\\
75.02	0\\
75.03	1.73472347597681e-18\\
75.04	0\\
75.05	0\\
75.06	1.73472347597681e-18\\
75.07	0\\
75.08	0\\
75.09	0\\
75.1	0\\
75.11	0\\
75.12	0\\
75.13	0\\
75.14	1.73472347597681e-18\\
75.15	0\\
75.16	0\\
75.17	0\\
75.18	0\\
75.19	0\\
75.2	0\\
75.21	0\\
75.22	0\\
75.23	0\\
75.24	0\\
75.25	0\\
75.26	0\\
75.27	0\\
75.28	0\\
75.29	0\\
75.3	0\\
75.31	0\\
75.32	0\\
75.33	0\\
75.34	0\\
75.35	0\\
75.36	1.73472347597681e-18\\
75.37	0\\
75.38	0\\
75.39	0\\
75.4	0\\
75.41	0\\
75.42	0\\
75.43	0\\
75.44	1.73472347597681e-18\\
75.45	0\\
75.46	1.73472347597681e-18\\
75.47	0\\
75.48	0\\
75.49	0\\
75.5	0\\
75.51	0\\
75.52	0\\
75.53	0\\
75.54	0\\
75.55	0\\
75.56	1.73472347597681e-18\\
75.57	0\\
75.58	0\\
75.59	0\\
75.6	0\\
75.61	0\\
75.62	0\\
75.63	0\\
75.64	0\\
75.65	0\\
75.66	0\\
75.67	0\\
75.68	0\\
75.69	0\\
75.7	0\\
75.71	0\\
75.72	0\\
75.73	1.73472347597681e-18\\
75.74	0\\
75.75	0\\
75.76	0\\
75.77	0\\
75.78	0\\
75.79	0\\
75.8	0\\
75.81	0\\
75.82	0\\
75.83	0\\
75.84	0\\
75.85	0\\
75.86	0\\
75.87	0\\
75.88	0\\
75.89	0\\
75.9	0\\
75.91	0\\
75.92	0\\
75.93	1.73472347597681e-18\\
75.94	0\\
75.95	0\\
75.96	0\\
75.97	0\\
75.98	0\\
75.99	0\\
76	0\\
76.01	0\\
76.02	1.73472347597681e-18\\
76.03	0\\
76.04	0\\
76.05	0\\
76.06	0\\
76.07	0\\
76.08	0\\
76.09	0\\
76.1	0\\
76.11	0\\
76.12	0\\
76.13	1.73472347597681e-18\\
76.14	0\\
76.15	0\\
76.16	1.73472347597681e-18\\
76.17	1.73472347597681e-18\\
76.18	0\\
76.19	0\\
76.2	0\\
76.21	0\\
76.22	0\\
76.23	0\\
76.24	0\\
76.25	0\\
76.26	0\\
76.27	0\\
76.28	0\\
76.29	0\\
76.3	0\\
76.31	0\\
76.32	0\\
76.33	1.73472347597681e-18\\
76.34	0\\
76.35	0\\
76.36	0\\
76.37	0\\
76.38	1.73472347597681e-18\\
76.39	0\\
76.4	0\\
76.41	1.73472347597681e-18\\
76.42	1.73472347597681e-18\\
76.43	0\\
76.44	0\\
76.45	0\\
76.46	1.73472347597681e-18\\
76.47	0\\
76.48	0\\
76.49	0\\
76.5	0\\
76.51	0\\
76.52	1.73472347597681e-18\\
76.53	0\\
76.54	1.73472347597681e-18\\
76.55	1.73472347597681e-18\\
76.56	0\\
76.57	0\\
76.58	0\\
76.59	0\\
76.6	0\\
76.61	0\\
76.62	0\\
76.63	0\\
76.64	0\\
76.65	0\\
76.66	0\\
76.67	0\\
76.68	0\\
76.69	0\\
76.7	0\\
76.71	0\\
76.72	0\\
76.73	0\\
76.74	0\\
76.75	0\\
76.76	0\\
76.77	0\\
76.78	0\\
76.79	0\\
76.8	0\\
76.81	0\\
76.82	0\\
76.83	0\\
76.84	0\\
76.85	0\\
76.86	0\\
76.87	0\\
76.88	0\\
76.89	1.73472347597681e-18\\
76.9	0\\
76.91	0\\
76.92	0\\
76.93	0\\
76.94	1.73472347597681e-18\\
76.95	0\\
76.96	0\\
76.97	0\\
76.98	0\\
76.99	0\\
77	0\\
77.01	0\\
77.02	0\\
77.03	0\\
77.04	1.73472347597681e-18\\
77.05	0\\
77.06	0\\
77.07	0\\
77.08	0\\
77.09	0\\
77.1	0\\
77.11	0\\
77.12	0\\
77.13	0\\
77.14	0\\
77.15	0\\
77.16	1.73472347597681e-18\\
77.17	0\\
77.18	0\\
77.19	1.73472347597681e-18\\
77.2	0\\
77.21	0\\
77.22	0\\
77.23	0\\
77.24	0\\
77.25	0\\
77.26	0\\
77.27	0\\
77.28	0\\
77.29	0\\
77.3	0\\
77.31	0\\
77.32	0\\
77.33	0\\
77.34	0\\
77.35	0\\
77.36	0\\
77.37	0\\
77.38	0\\
77.39	0\\
77.4	0\\
77.41	1.73472347597681e-18\\
77.42	0\\
77.43	0\\
77.44	0\\
77.45	0\\
77.46	0\\
77.47	0\\
77.48	0\\
77.49	1.73472347597681e-18\\
77.5	0\\
77.51	0\\
77.52	0\\
77.53	0\\
77.54	1.73472347597681e-18\\
77.55	0\\
77.56	0\\
77.57	0\\
77.58	0\\
77.59	0\\
77.6	0\\
77.61	0\\
77.62	0\\
77.63	0\\
77.64	0\\
77.65	1.73472347597681e-18\\
77.66	0\\
77.67	0\\
77.68	0\\
77.69	0\\
77.7	0\\
77.71	0\\
77.72	0\\
77.73	0\\
77.74	0\\
77.75	1.73472347597681e-18\\
77.76	0\\
77.77	1.73472347597681e-18\\
77.78	0\\
77.79	1.73472347597681e-18\\
77.8	0\\
77.81	0\\
77.82	0\\
77.83	0\\
77.84	0\\
77.85	1.73472347597681e-18\\
77.86	0\\
77.87	0\\
77.88	0\\
77.89	0\\
77.9	0\\
77.91	0\\
77.92	0\\
77.93	0\\
77.94	0\\
77.95	0\\
77.96	0\\
77.97	0\\
77.98	0\\
77.99	0\\
78	0\\
78.01	0\\
78.02	0\\
78.03	0\\
78.04	1.73472347597681e-18\\
78.05	0\\
78.06	0\\
78.07	0\\
78.08	0\\
78.09	0\\
78.1	1.73472347597681e-18\\
78.11	0\\
78.12	0\\
78.13	0\\
78.14	0\\
78.15	0\\
78.16	0\\
78.17	0\\
78.18	0\\
78.19	0\\
78.2	1.73472347597681e-18\\
78.21	1.73472347597681e-18\\
78.22	0\\
78.23	0\\
78.24	0\\
78.25	1.73472347597681e-18\\
78.26	0\\
78.27	0\\
78.28	0\\
78.29	1.73472347597681e-18\\
78.3	0\\
78.31	0\\
78.32	0\\
78.33	0\\
78.34	0\\
78.35	0\\
78.36	0\\
78.37	0\\
78.38	0\\
78.39	0\\
78.4	1.73472347597681e-18\\
78.41	1.73472347597681e-18\\
78.42	0\\
78.43	0\\
78.44	0\\
78.45	0\\
78.46	0\\
78.47	0\\
78.48	0\\
78.49	0\\
78.5	0\\
78.51	0\\
78.52	0\\
78.53	0\\
78.54	1.73472347597681e-18\\
78.55	0\\
78.56	0\\
78.57	0\\
78.58	0\\
78.59	0\\
78.6	0\\
78.61	0\\
78.62	1.73472347597681e-18\\
78.63	0\\
78.64	0\\
78.65	0\\
78.66	0\\
78.67	0\\
78.68	0\\
78.69	0\\
78.7	0\\
78.71	0\\
78.72	0\\
78.73	0\\
78.74	0\\
78.75	0\\
78.76	0\\
78.77	0\\
78.78	0\\
78.79	0\\
78.8	0\\
78.81	0\\
78.82	0\\
78.83	0\\
78.84	0\\
78.85	0\\
78.86	0\\
78.87	1.73472347597681e-18\\
78.88	0\\
78.89	0\\
78.9	1.73472347597681e-18\\
78.91	0\\
78.92	1.73472347597681e-18\\
78.93	0\\
78.94	0\\
78.95	1.73472347597681e-18\\
78.96	0\\
78.97	0\\
78.98	1.73472347597681e-18\\
78.99	0\\
79	0\\
79.01	0\\
79.02	0\\
79.03	0\\
79.04	0\\
79.05	0\\
79.06	0\\
79.07	0\\
79.08	1.73472347597681e-18\\
79.09	0\\
79.1	0\\
79.11	0\\
79.12	0\\
79.13	0\\
79.14	0\\
79.15	0\\
79.16	0\\
79.17	0\\
79.18	0\\
79.19	0\\
79.2	0\\
79.21	0\\
79.22	0\\
79.23	0\\
79.24	0\\
79.25	0\\
79.26	0\\
79.27	0\\
79.28	0\\
79.29	0\\
79.3	1.73472347597681e-18\\
79.31	0\\
79.32	0\\
79.33	0\\
79.34	0\\
79.35	1.73472347597681e-18\\
79.36	0\\
79.37	0\\
79.38	0\\
79.39	0\\
79.4	0\\
79.41	0\\
79.42	0\\
79.43	0\\
79.44	0\\
79.45	0\\
79.46	1.73472347597681e-18\\
79.47	0\\
79.48	0\\
79.49	0\\
79.5	0\\
79.51	0\\
79.52	0\\
79.53	1.73472347597681e-18\\
79.54	0\\
79.55	0\\
79.56	1.73472347597681e-18\\
79.57	1.73472347597681e-18\\
79.58	0\\
79.59	1.73472347597681e-18\\
79.6	0\\
79.61	0\\
79.62	0\\
79.63	0\\
79.64	0\\
79.65	0\\
79.66	0\\
79.67	0\\
79.68	0\\
79.69	0\\
79.7	0\\
79.71	0\\
79.72	0\\
79.73	0\\
79.74	0\\
79.75	0\\
79.76	0\\
79.77	0\\
79.78	0\\
79.79	1.73472347597681e-18\\
79.8	0\\
79.81	0\\
79.82	1.73472347597681e-18\\
79.83	0\\
79.84	0\\
79.85	0\\
79.86	0\\
79.87	1.73472347597681e-18\\
79.88	0\\
79.89	0\\
79.9	0\\
79.91	0\\
79.92	0\\
79.93	0\\
79.94	1.73472347597681e-18\\
79.95	0\\
79.96	0\\
79.97	1.73472347597681e-18\\
79.98	1.73472347597681e-18\\
79.99	1.73472347597681e-18\\
80	0\\
80.01	0\\
};
\addplot [color=green,dashed]
  table[row sep=crcr]{%
80.01	0\\
80.02	0\\
80.03	1.73472347597681e-18\\
80.04	0\\
80.05	0\\
80.06	0\\
80.07	0\\
80.08	0\\
80.09	0\\
80.1	0\\
80.11	1.73472347597681e-18\\
80.12	0\\
80.13	1.73472347597681e-18\\
80.14	0\\
80.15	0\\
80.16	0\\
80.17	1.73472347597681e-18\\
80.18	1.73472347597681e-18\\
80.19	0\\
80.2	0\\
80.21	1.73472347597681e-18\\
80.22	0\\
80.23	1.73472347597681e-18\\
80.24	0\\
80.25	1.73472347597681e-18\\
80.26	0\\
80.27	0\\
80.28	0\\
80.29	0\\
80.3	0\\
80.31	0\\
80.32	0\\
80.33	0\\
80.34	0\\
80.35	0\\
80.36	0\\
80.37	1.73472347597681e-18\\
80.38	0\\
80.39	0\\
80.4	1.73472347597681e-18\\
80.41	0\\
80.42	0\\
80.43	1.73472347597681e-18\\
80.44	0\\
80.45	0\\
80.46	1.73472347597681e-18\\
80.47	0\\
80.48	0\\
80.49	0\\
80.5	0\\
80.51	0\\
80.52	0\\
80.53	0\\
80.54	0\\
80.55	0\\
80.56	1.73472347597681e-18\\
80.57	0\\
80.58	0\\
80.59	0\\
80.6	1.73472347597681e-18\\
80.61	0\\
80.62	0\\
80.63	0\\
80.64	0\\
80.65	0\\
80.66	0\\
80.67	0\\
80.68	0\\
80.69	1.73472347597681e-18\\
80.7	1.73472347597681e-18\\
80.71	0\\
80.72	0\\
80.73	1.73472347597681e-18\\
80.74	0\\
80.75	0\\
80.76	0\\
80.77	0\\
80.78	0\\
80.79	0\\
80.8	0\\
80.81	1.73472347597681e-18\\
80.82	0\\
80.83	0\\
80.84	0\\
80.85	0\\
80.86	0\\
80.87	0\\
80.88	0\\
80.89	0\\
80.9	1.73472347597681e-18\\
80.91	0\\
80.92	0\\
80.93	0\\
80.94	0\\
80.95	1.73472347597681e-18\\
80.96	0\\
80.97	0\\
80.98	0\\
80.99	0\\
81	1.73472347597681e-18\\
81.01	0\\
81.02	0\\
81.03	0\\
81.04	0\\
81.05	0\\
81.06	1.73472347597681e-18\\
81.07	1.73472347597681e-18\\
81.08	0\\
81.09	0\\
81.1	0\\
81.11	0\\
81.12	0\\
81.13	0\\
81.14	0\\
81.15	0\\
81.16	0\\
81.17	0\\
81.18	0\\
81.19	0\\
81.2	0\\
81.21	0\\
81.22	0\\
81.23	0\\
81.24	0\\
81.25	0\\
81.26	0\\
81.27	0\\
81.28	1.73472347597681e-18\\
81.29	1.73472347597681e-18\\
81.3	0\\
81.31	0\\
81.32	0\\
81.33	0\\
81.34	1.73472347597681e-18\\
81.35	0\\
81.36	0\\
81.37	0\\
81.38	0\\
81.39	0\\
81.4	1.73472347597681e-18\\
81.41	0\\
81.42	0\\
81.43	0\\
81.44	0\\
81.45	0\\
81.46	0\\
81.47	0\\
81.48	0\\
81.49	1.73472347597681e-18\\
81.5	0\\
81.51	0\\
81.52	0\\
81.53	0\\
81.54	1.73472347597681e-18\\
81.55	0\\
81.56	0\\
81.57	0\\
81.58	0\\
81.59	0\\
81.6	0\\
81.61	0\\
81.62	0\\
81.63	0\\
81.64	0\\
81.65	0\\
81.66	0\\
81.67	0\\
81.68	1.73472347597681e-18\\
81.69	0\\
81.7	0\\
81.71	1.73472347597681e-18\\
81.72	0\\
81.73	0\\
81.74	0\\
81.75	0\\
81.76	1.73472347597681e-18\\
81.77	0\\
81.78	0\\
81.79	1.73472347597681e-18\\
81.8	0\\
81.81	0\\
81.82	0\\
81.83	0\\
81.84	0\\
81.85	0\\
81.86	0\\
81.87	0\\
81.88	0\\
81.89	0\\
81.9	0\\
81.91	0\\
81.92	0\\
81.93	0\\
81.94	1.73472347597681e-18\\
81.95	0\\
81.96	1.73472347597681e-18\\
81.97	0\\
81.98	0\\
81.99	1.73472347597681e-18\\
82	0\\
82.01	1.73472347597681e-18\\
82.02	1.73472347597681e-18\\
82.03	1.73472347597681e-18\\
82.04	0\\
82.05	1.73472347597681e-18\\
82.06	0\\
82.07	1.73472347597681e-18\\
82.08	0\\
82.09	0\\
82.1	0\\
82.11	0\\
82.12	0\\
82.13	0\\
82.14	0\\
82.15	0\\
82.16	0\\
82.17	0\\
82.18	0\\
82.19	0\\
82.2	0\\
82.21	0\\
82.22	0\\
82.23	0\\
82.24	0\\
82.25	0\\
82.26	0\\
82.27	0\\
82.28	0\\
82.29	0\\
82.3	0\\
82.31	0\\
82.32	0\\
82.33	1.73472347597681e-18\\
82.34	0\\
82.35	1.73472347597681e-18\\
82.36	0\\
82.37	0\\
82.38	0\\
82.39	0\\
82.4	0\\
82.41	0\\
82.42	0\\
82.43	0\\
82.44	0\\
82.45	0\\
82.46	0\\
82.47	0\\
82.48	0\\
82.49	0\\
82.5	1.73472347597681e-18\\
82.51	0\\
82.52	0\\
82.53	0\\
82.54	0\\
82.55	0\\
82.56	0\\
82.57	0\\
82.58	0\\
82.59	0\\
82.6	0\\
82.61	0\\
82.62	0\\
82.63	1.73472347597681e-18\\
82.64	1.73472347597681e-18\\
82.65	1.73472347597681e-18\\
82.66	0\\
82.67	0\\
82.68	0\\
82.69	0\\
82.7	0\\
82.71	0\\
82.72	0\\
82.73	0\\
82.74	0\\
82.75	0\\
82.76	0\\
82.77	1.73472347597681e-18\\
82.78	0\\
82.79	0\\
82.8	0\\
82.81	0\\
82.82	0\\
82.83	0\\
82.84	0\\
82.85	1.73472347597681e-18\\
82.86	0\\
82.87	0\\
82.88	0\\
82.89	1.73472347597681e-18\\
82.9	0\\
82.91	0\\
82.92	0\\
82.93	0\\
82.94	0\\
82.95	0\\
82.96	0\\
82.97	0\\
82.98	0\\
82.99	0\\
83	0\\
83.01	0\\
83.02	0\\
83.03	0\\
83.04	0\\
83.05	1.73472347597681e-18\\
83.06	0\\
83.07	0\\
83.08	0\\
83.09	0\\
83.1	1.73472347597681e-18\\
83.11	1.73472347597681e-18\\
83.12	1.73472347597681e-18\\
83.13	0\\
83.14	0\\
83.15	0\\
83.16	0\\
83.17	0\\
83.18	0\\
83.19	0\\
83.2	1.73472347597681e-18\\
83.21	0\\
83.22	0\\
83.23	1.73472347597681e-18\\
83.24	0\\
83.25	0\\
83.26	1.73472347597681e-18\\
83.27	0\\
83.28	0\\
83.29	1.73472347597681e-18\\
83.3	0\\
83.31	0\\
83.32	0\\
83.33	0\\
83.34	0\\
83.35	0\\
83.36	0\\
83.37	0\\
83.38	0\\
83.39	0\\
83.4	0\\
83.41	0\\
83.42	0\\
83.43	0\\
83.44	0\\
83.45	0\\
83.46	0\\
83.47	1.73472347597681e-18\\
83.48	0\\
83.49	1.73472347597681e-18\\
83.5	0\\
83.51	0\\
83.52	0\\
83.53	0\\
83.54	0\\
83.55	0\\
83.56	0\\
83.57	0\\
83.58	0\\
83.59	0\\
83.6	1.73472347597681e-18\\
83.61	1.73472347597681e-18\\
83.62	0\\
83.63	1.73472347597681e-18\\
83.64	0\\
83.65	0\\
83.66	0\\
83.67	0\\
83.68	0\\
83.69	1.73472347597681e-18\\
83.7	0\\
83.71	0\\
83.72	0\\
83.73	0\\
83.74	0\\
83.75	0\\
83.76	0\\
83.77	0\\
83.78	0\\
83.79	1.73472347597681e-18\\
83.8	0\\
83.81	0\\
83.82	0\\
83.83	0\\
83.84	0\\
83.85	0\\
83.86	0\\
83.87	0\\
83.88	0\\
83.89	0\\
83.9	0\\
83.91	0\\
83.92	0\\
83.93	0\\
83.94	0\\
83.95	0\\
83.96	0\\
83.97	0\\
83.98	0\\
83.99	0\\
84	0\\
84.01	0\\
84.02	0\\
84.03	0\\
84.04	0\\
84.05	0\\
84.06	0\\
84.07	1.73472347597681e-18\\
84.08	0\\
84.09	0\\
84.1	0\\
84.11	1.73472347597681e-18\\
84.12	0\\
84.13	0\\
84.14	0\\
84.15	0\\
84.16	0\\
84.17	0\\
84.18	0\\
84.19	0\\
84.2	0\\
84.21	0\\
84.22	1.73472347597681e-18\\
84.23	0\\
84.24	0\\
84.25	0\\
84.26	1.73472347597681e-18\\
84.27	0\\
84.28	0\\
84.29	0\\
84.3	0\\
84.31	0\\
84.32	1.73472347597681e-18\\
84.33	0\\
84.34	0\\
84.35	1.73472347597681e-18\\
84.36	1.73472347597681e-18\\
84.37	0\\
84.38	0\\
84.39	0\\
84.4	0\\
84.41	1.73472347597681e-18\\
84.42	1.73472347597681e-18\\
84.43	0\\
84.44	0\\
84.45	0\\
84.46	0\\
84.47	0\\
84.48	0\\
84.49	0\\
84.5	1.73472347597681e-18\\
84.51	0\\
84.52	0\\
84.53	0\\
84.54	0\\
84.55	0\\
84.56	0\\
84.57	0\\
84.58	0\\
84.59	1.73472347597681e-18\\
84.6	0\\
84.61	0\\
84.62	0\\
84.63	0\\
84.64	0\\
84.65	0\\
84.66	1.73472347597681e-18\\
84.67	1.73472347597681e-18\\
84.68	0\\
84.69	0\\
84.7	0\\
84.71	0\\
84.72	0\\
84.73	0\\
84.74	0\\
84.75	0\\
84.76	1.73472347597681e-18\\
84.77	0\\
84.78	0\\
84.79	0\\
84.8	0\\
84.81	0\\
84.82	0\\
84.83	0\\
84.84	0\\
84.85	0\\
84.86	0\\
84.87	0\\
84.88	0\\
84.89	0\\
84.9	0\\
84.91	0\\
84.92	0\\
84.93	0\\
84.94	0\\
84.95	0\\
84.96	0\\
84.97	0\\
84.98	0\\
84.99	0\\
85	0\\
85.01	0\\
85.02	0\\
85.03	0\\
85.04	0\\
85.05	0\\
85.06	0\\
85.07	0\\
85.08	0\\
85.09	0\\
85.1	0\\
85.11	0\\
85.12	0\\
85.13	0\\
85.14	0\\
85.15	0\\
85.16	0\\
85.17	0\\
85.18	0\\
85.19	0\\
85.2	0\\
85.21	0\\
85.22	0\\
85.23	1.73472347597681e-18\\
85.24	0\\
85.25	0\\
85.26	1.73472347597681e-18\\
85.27	0\\
85.28	1.73472347597681e-18\\
85.29	0\\
85.3	0\\
85.31	1.73472347597681e-18\\
85.32	0\\
85.33	1.73472347597681e-18\\
85.34	0\\
85.35	0\\
85.36	0\\
85.37	0\\
85.38	0\\
85.39	0\\
85.4	0\\
85.41	0\\
85.42	0\\
85.43	0\\
85.44	0\\
85.45	0\\
85.46	0\\
85.47	0\\
85.48	0\\
85.49	0\\
85.5	0\\
85.51	0\\
85.52	0\\
85.53	1.73472347597681e-18\\
85.54	0\\
85.55	0\\
85.56	0\\
85.57	0\\
85.58	0\\
85.59	0\\
85.6	0\\
85.61	0\\
85.62	0\\
85.63	0\\
85.64	0\\
85.65	0\\
85.66	0\\
85.67	0\\
85.68	0\\
85.69	0\\
85.7	0\\
85.71	0\\
85.72	0\\
85.73	0\\
85.74	0\\
85.75	0\\
85.76	0\\
85.77	0\\
85.78	0\\
85.79	0\\
85.8	0\\
85.81	1.73472347597681e-18\\
85.82	0\\
85.83	0\\
85.84	0\\
85.85	0\\
85.86	0\\
85.87	0\\
85.88	0\\
85.89	0\\
85.9	0\\
85.91	0\\
85.92	0\\
85.93	0\\
85.94	0\\
85.95	0\\
85.96	0\\
85.97	1.73472347597681e-18\\
85.98	0\\
85.99	0\\
86	0\\
86.01	0\\
86.02	1.73472347597681e-18\\
86.03	1.73472347597681e-18\\
86.04	0\\
86.05	0\\
86.06	0\\
86.07	0\\
86.08	0\\
86.09	0\\
86.1	0\\
86.11	0\\
86.12	0\\
86.13	0\\
86.14	0\\
86.15	0\\
86.16	0\\
86.17	0\\
86.18	0\\
86.19	0\\
86.2	0\\
86.21	0\\
86.22	0\\
86.23	0\\
86.24	0\\
86.25	1.73472347597681e-18\\
86.26	1.73472347597681e-18\\
86.27	0\\
86.28	0\\
86.29	0\\
86.3	0\\
86.31	0\\
86.32	0\\
86.33	1.73472347597681e-18\\
86.34	0\\
86.35	0\\
86.36	0\\
86.37	0\\
86.38	0\\
86.39	0\\
86.4	0\\
86.41	0\\
86.42	0\\
86.43	0\\
86.44	0\\
86.45	0\\
86.46	0\\
86.47	0\\
86.48	0\\
86.49	0\\
86.5	0\\
86.51	0\\
86.52	0\\
86.53	0\\
86.54	0\\
86.55	1.73472347597681e-18\\
86.56	0\\
86.57	0\\
86.58	0\\
86.59	0\\
86.6	0\\
86.61	0\\
86.62	0\\
86.63	0\\
86.64	0\\
86.65	0\\
86.66	0\\
86.67	0\\
86.68	0\\
86.69	0\\
86.7	1.73472347597681e-18\\
86.71	1.73472347597681e-18\\
86.72	0\\
86.73	0\\
86.74	0\\
86.75	0\\
86.76	0\\
86.77	0\\
86.78	0\\
86.79	0\\
86.8	0\\
86.81	0\\
86.82	0\\
86.83	0\\
86.84	0\\
86.85	0\\
86.86	0\\
86.87	0\\
86.88	0\\
86.89	0\\
86.9	0\\
86.91	0\\
86.92	0\\
86.93	0\\
86.94	0\\
86.95	0\\
86.96	0\\
86.97	1.73472347597681e-18\\
86.98	0\\
86.99	0\\
87	0\\
87.01	0\\
87.02	0\\
87.03	0\\
87.04	0\\
87.05	0\\
87.06	0\\
87.07	0\\
87.08	0\\
87.09	0\\
87.1	0\\
87.11	0\\
87.12	0\\
87.13	0\\
87.14	0\\
87.15	0\\
87.16	0\\
87.17	0\\
87.18	0\\
87.19	0\\
87.2	0\\
87.21	0\\
87.22	0\\
87.23	0\\
87.24	0\\
87.25	0\\
87.26	0\\
87.27	1.73472347597681e-18\\
87.28	0\\
87.29	0\\
87.3	0\\
87.31	0\\
87.32	0\\
87.33	0\\
87.34	0\\
87.35	0\\
87.36	0\\
87.37	0\\
87.38	0\\
87.39	0\\
87.4	0\\
87.41	0\\
87.42	0\\
87.43	0\\
87.44	0\\
87.45	0\\
87.46	0\\
87.47	0\\
87.48	0\\
87.49	0\\
87.5	0\\
87.51	1.73472347597681e-18\\
87.52	0\\
87.53	0\\
87.54	0\\
87.55	0\\
87.56	0\\
87.57	0\\
87.58	0\\
87.59	0\\
87.6	0\\
87.61	0\\
87.62	1.73472347597681e-18\\
87.63	0\\
87.64	0\\
87.65	0\\
87.66	0\\
87.67	0\\
87.68	0\\
87.69	1.73472347597681e-18\\
87.7	1.73472347597681e-18\\
87.71	0\\
87.72	0\\
87.73	0\\
87.74	0\\
87.75	0\\
87.76	0\\
87.77	0\\
87.78	0\\
87.79	0\\
87.8	0\\
87.81	0\\
87.82	0\\
87.83	1.73472347597681e-18\\
87.84	0\\
87.85	0\\
87.86	0\\
87.87	0\\
87.88	0\\
87.89	0\\
87.9	0\\
87.91	0\\
87.92	0\\
87.93	0\\
87.94	0\\
87.95	0\\
87.96	0\\
87.97	0\\
87.98	0\\
87.99	0\\
88	0\\
88.01	0\\
88.02	0\\
88.03	0\\
88.04	0\\
88.05	0\\
88.06	0\\
88.07	0\\
88.08	0\\
88.09	0\\
88.1	0\\
88.11	0\\
88.12	0\\
88.13	0\\
88.14	0\\
88.15	1.73472347597681e-18\\
88.16	0\\
88.17	0\\
88.18	0\\
88.19	0\\
88.2	0\\
88.21	0\\
88.22	0\\
88.23	0\\
88.24	0\\
88.25	0\\
88.26	0\\
88.27	0\\
88.28	0\\
88.29	0\\
88.3	0\\
88.31	0\\
88.32	0\\
88.33	0\\
88.34	0\\
88.35	0\\
88.36	0\\
88.37	0\\
88.38	0\\
88.39	0\\
88.4	0\\
88.41	0\\
88.42	0\\
88.43	0\\
88.44	0\\
88.45	0\\
88.46	1.73472347597681e-18\\
88.47	0\\
88.48	0\\
88.49	0\\
88.5	0\\
88.51	0\\
88.52	0\\
88.53	0\\
88.54	0\\
88.55	0\\
88.56	0\\
88.57	0\\
88.58	0\\
88.59	0\\
88.6	0\\
88.61	0\\
88.62	0\\
88.63	0\\
88.64	0\\
88.65	0\\
88.66	0\\
88.67	0\\
88.68	0\\
88.69	0\\
88.7	0\\
88.71	0\\
88.72	0\\
88.73	0\\
88.74	0\\
88.75	0\\
88.76	0\\
88.77	0\\
88.78	0\\
88.79	0\\
88.8	0\\
88.81	0\\
88.82	0\\
88.83	0\\
88.84	0\\
88.85	0\\
88.86	0\\
88.87	0\\
88.88	0\\
88.89	0\\
88.9	0\\
88.91	0\\
88.92	0\\
88.93	0\\
88.94	0\\
88.95	0\\
88.96	0\\
88.97	0\\
88.98	0\\
88.99	0\\
89	0\\
89.01	0\\
89.02	0\\
89.03	0\\
89.04	0\\
89.05	0\\
89.06	0\\
89.07	0\\
89.08	0\\
89.09	0\\
89.1	0\\
89.11	0\\
89.12	0\\
89.13	0\\
89.14	0\\
89.15	0\\
89.16	0\\
89.17	0\\
89.18	0\\
89.19	0\\
89.2	0\\
89.21	0\\
89.22	0\\
89.23	0\\
89.24	0\\
89.25	0\\
89.26	0\\
89.27	0\\
89.28	0\\
89.29	0\\
89.3	0\\
89.31	0\\
89.32	0\\
89.33	0\\
89.34	0\\
89.35	0\\
89.36	0\\
89.37	0\\
89.38	1.73472347597681e-18\\
89.39	0\\
89.4	0\\
89.41	0\\
89.42	0\\
89.43	0\\
89.44	0\\
89.45	0\\
89.46	0\\
89.47	0\\
89.48	0\\
89.49	0\\
89.5	0\\
89.51	0\\
89.52	0\\
89.53	0\\
89.54	0\\
89.55	0\\
89.56	0\\
89.57	0\\
89.58	0\\
89.59	0\\
89.6	0\\
89.61	0\\
89.62	0\\
89.63	0\\
89.64	0\\
89.65	1.73472347597681e-18\\
89.66	0\\
89.67	0\\
89.68	1.73472347597681e-18\\
89.69	0\\
89.7	0\\
89.71	0\\
89.72	0\\
89.73	0\\
89.74	0\\
89.75	0\\
89.76	0\\
89.77	0\\
89.78	0\\
89.79	0\\
89.8	0\\
89.81	0\\
89.82	0\\
89.83	0\\
89.84	0\\
89.85	0\\
89.86	0\\
89.87	0\\
89.88	0\\
89.89	0\\
89.9	0\\
89.91	0\\
89.92	0\\
89.93	0\\
89.94	0\\
89.95	0\\
89.96	0\\
89.97	0\\
89.98	0\\
89.99	0\\
90	0\\
90.01	0\\
90.02	0\\
90.03	0\\
90.04	0\\
90.05	0\\
90.06	1.73472347597681e-18\\
90.07	0\\
90.08	0\\
90.09	0\\
90.1	0\\
90.11	0\\
90.12	0\\
90.13	0\\
90.14	0\\
90.15	0\\
90.16	0\\
90.17	0\\
90.18	1.73472347597681e-18\\
90.19	0\\
90.2	0\\
90.21	0\\
90.22	0\\
90.23	0\\
90.24	0\\
90.25	0\\
90.26	0\\
90.27	0\\
90.28	0\\
90.29	0\\
90.3	0\\
90.31	0\\
90.32	0\\
90.33	1.73472347597681e-18\\
90.34	0\\
90.35	0\\
90.36	0\\
90.37	0\\
90.38	0\\
90.39	0\\
90.4	0\\
90.41	0\\
90.42	0\\
90.43	0\\
90.44	0\\
90.45	0\\
90.46	0\\
90.47	0\\
90.48	0\\
90.49	0\\
90.5	0\\
90.51	0\\
90.52	0\\
90.53	0\\
90.54	0\\
90.55	0\\
90.56	0\\
90.57	0\\
90.58	0\\
90.59	0\\
90.6	0\\
90.61	0\\
90.62	0\\
90.63	0\\
90.64	0\\
90.65	0\\
90.66	0\\
90.67	0\\
90.68	0\\
90.69	0\\
90.7	0\\
90.71	0\\
90.72	0\\
90.73	0\\
90.74	0\\
90.75	0\\
90.76	0\\
90.77	0\\
90.78	0\\
90.79	0\\
90.8	0\\
90.81	0\\
90.82	0\\
90.83	0\\
90.84	0\\
90.85	0\\
90.86	0\\
90.87	0\\
90.88	0\\
90.89	0\\
90.9	0\\
90.91	0\\
90.92	0\\
90.93	0\\
90.94	0\\
90.95	0\\
90.96	0\\
90.97	0\\
90.98	0\\
90.99	0\\
91	0\\
91.01	0\\
91.02	0\\
91.03	0\\
91.04	0\\
91.05	0\\
91.06	0\\
91.07	0\\
91.08	0\\
91.09	0\\
91.1	0\\
91.11	0\\
91.12	0\\
91.13	0\\
91.14	0\\
91.15	0\\
91.16	0\\
91.17	0\\
91.18	0\\
91.19	0\\
91.2	0\\
91.21	0\\
91.22	0\\
91.23	0\\
91.24	0\\
91.25	0\\
91.26	0\\
91.27	0\\
91.28	0\\
91.29	0\\
91.3	0\\
91.31	0\\
91.32	0\\
91.33	0\\
91.34	0\\
91.35	0\\
91.36	0\\
91.37	0\\
91.38	0\\
91.39	0\\
91.4	0\\
91.41	0\\
91.42	0\\
91.43	0\\
91.44	0\\
91.45	0\\
91.46	0\\
91.47	0\\
91.48	0\\
91.49	0\\
91.5	0\\
91.51	0\\
91.52	0\\
91.53	0\\
91.54	0\\
91.55	0\\
91.56	0\\
91.57	0\\
91.58	0\\
91.59	0\\
91.6	0\\
91.61	0\\
91.62	0\\
91.63	0\\
91.64	0\\
91.65	0\\
91.66	0\\
91.67	0\\
91.68	0\\
91.69	0\\
91.7	0\\
91.71	0\\
91.72	0\\
91.73	0\\
91.74	0\\
91.75	0\\
91.76	0\\
91.77	0\\
91.78	0\\
91.79	0\\
91.8	0\\
91.81	0\\
91.82	0\\
91.83	0\\
91.84	0\\
91.85	0\\
91.86	0\\
91.87	0\\
91.88	0\\
91.89	0\\
91.9	0\\
91.91	0\\
91.92	0\\
91.93	0\\
91.94	0\\
91.95	0\\
91.96	0\\
91.97	0\\
91.98	0\\
91.99	0\\
92	0\\
92.01	0\\
92.02	0\\
92.03	0\\
92.04	0\\
92.05	0\\
92.06	0\\
92.07	0\\
92.08	0\\
92.09	0\\
92.1	0\\
92.11	0\\
92.12	0\\
92.13	0\\
92.14	0\\
92.15	0\\
92.16	0\\
92.17	0\\
92.18	0\\
92.19	0\\
92.2	0\\
92.21	0\\
92.22	0\\
92.23	0\\
92.24	0\\
92.25	0\\
92.26	0\\
92.27	0\\
92.28	0\\
92.29	0\\
92.3	0\\
92.31	0\\
92.32	0\\
92.33	0\\
92.34	0\\
92.35	0\\
92.36	0\\
92.37	0\\
92.38	0\\
92.39	0\\
92.4	0\\
92.41	0\\
92.42	0\\
92.43	0\\
92.44	0\\
92.45	0\\
92.46	0\\
92.47	0\\
92.48	0\\
92.49	0\\
92.5	0\\
92.51	0\\
92.52	0\\
92.53	0\\
92.54	0\\
92.55	0\\
92.56	0\\
92.57	0\\
92.58	0\\
92.59	0\\
92.6	0\\
92.61	0\\
92.62	0\\
92.63	0\\
92.64	0\\
92.65	0\\
92.66	0\\
92.67	0\\
92.68	0\\
92.69	0\\
92.7	0\\
92.71	0\\
92.72	0\\
92.73	0\\
92.74	0\\
92.75	0\\
92.76	0\\
92.77	0\\
92.78	0\\
92.79	0\\
92.8	0\\
92.81	0\\
92.82	0\\
92.83	0\\
92.84	0\\
92.85	0\\
92.86	0\\
92.87	0\\
92.88	0\\
92.89	0\\
92.9	0\\
92.91	0\\
92.92	0\\
92.93	0\\
92.94	0\\
92.95	0\\
92.96	0\\
92.97	0\\
92.98	0\\
92.99	0\\
93	0\\
93.01	0\\
93.02	0\\
93.03	0\\
93.04	0\\
93.05	0\\
93.06	0\\
93.07	0\\
93.08	0\\
93.09	0\\
93.1	0\\
93.11	0\\
93.12	0\\
93.13	0\\
93.14	0\\
93.15	0\\
93.16	0\\
93.17	0\\
93.18	0\\
93.19	0\\
93.2	0\\
93.21	0\\
93.22	0\\
93.23	0\\
93.24	0\\
93.25	0\\
93.26	0\\
93.27	0\\
93.28	0\\
93.29	0\\
93.3	0\\
93.31	0\\
93.32	0\\
93.33	0\\
93.34	0\\
93.35	0\\
93.36	0\\
93.37	0\\
93.38	0\\
93.39	0\\
93.4	0\\
93.41	0\\
93.42	0\\
93.43	0\\
93.44	0\\
93.45	0\\
93.46	0\\
93.47	0\\
93.48	0\\
93.49	0\\
93.5	0\\
93.51	0\\
93.52	0\\
93.53	0\\
93.54	0\\
93.55	0\\
93.56	0\\
93.57	0\\
93.58	0\\
93.59	0\\
93.6	0\\
93.61	0\\
93.62	0\\
93.63	0\\
93.64	0\\
93.65	0\\
93.66	0\\
93.67	0\\
93.68	0\\
93.69	0\\
93.7	0\\
93.71	0\\
93.72	0\\
93.73	0\\
93.74	0\\
93.75	0\\
93.76	0\\
93.77	0\\
93.78	0\\
93.79	0\\
93.8	0\\
93.81	0\\
93.82	0\\
93.83	0\\
93.84	0\\
93.85	0\\
93.86	0\\
93.87	0\\
93.88	0\\
93.89	0\\
93.9	0\\
93.91	0\\
93.92	0\\
93.93	0\\
93.94	0\\
93.95	0\\
93.96	0\\
93.97	0\\
93.98	0\\
93.99	0\\
94	0\\
94.01	0\\
94.02	0\\
94.03	0\\
94.04	0\\
94.05	0\\
94.06	0\\
94.07	0\\
94.08	0\\
94.09	0\\
94.1	0\\
94.11	0\\
94.12	0\\
94.13	0\\
94.14	0\\
94.15	0\\
94.16	0\\
94.17	0\\
94.18	0\\
94.19	0\\
94.2	0\\
94.21	0\\
94.22	0\\
94.23	0\\
94.24	0\\
94.25	0\\
94.26	0\\
94.27	0\\
94.28	0\\
94.29	0\\
94.3	0\\
94.31	0\\
94.32	0\\
94.33	0\\
94.34	0\\
94.35	0\\
94.36	0\\
94.37	0\\
94.38	0\\
94.39	0\\
94.4	0\\
94.41	0\\
94.42	0\\
94.43	0\\
94.44	0\\
94.45	0\\
94.46	0\\
94.47	0\\
94.48	0\\
94.49	0\\
94.5	0\\
94.51	0\\
94.52	0\\
94.53	0\\
94.54	0\\
94.55	0\\
94.56	0\\
94.57	0\\
94.58	0\\
94.59	0\\
94.6	0\\
94.61	0\\
94.62	0\\
94.63	0\\
94.64	0\\
94.65	0\\
94.66	0\\
94.67	0\\
94.68	0\\
94.69	0\\
94.7	0\\
94.71	0\\
94.72	0\\
94.73	0\\
94.74	0\\
94.75	0\\
94.76	0\\
94.77	0\\
94.78	0\\
94.79	0\\
94.8	0\\
94.81	0\\
94.82	0\\
94.83	0\\
94.84	0\\
94.85	0\\
94.86	0\\
94.87	0\\
94.88	0\\
94.89	0\\
94.9	0\\
94.91	0\\
94.92	0\\
94.93	0\\
94.94	0\\
94.95	0\\
94.96	0\\
94.97	0\\
94.98	0\\
94.99	0\\
95	0\\
95.01	0\\
95.02	0\\
95.03	0\\
95.04	0\\
95.05	0\\
95.06	0\\
95.07	0\\
95.08	0\\
95.09	0\\
95.1	0\\
95.11	0\\
95.12	0\\
95.13	0\\
95.14	0\\
95.15	0\\
95.16	0\\
95.17	0\\
95.18	0\\
95.19	0\\
95.2	0\\
95.21	0\\
95.22	0\\
95.23	0\\
95.24	0\\
95.25	0\\
95.26	0\\
95.27	0\\
95.28	0\\
95.29	0\\
95.3	0\\
95.31	0\\
95.32	0\\
95.33	0\\
95.34	0\\
95.35	0\\
95.36	0\\
95.37	0\\
95.38	0\\
95.39	0\\
95.4	0\\
95.41	0\\
95.42	0\\
95.43	0\\
95.44	0\\
95.45	0\\
95.46	0\\
95.47	0\\
95.48	0\\
95.49	0\\
95.5	0\\
95.51	0\\
95.52	0\\
95.53	0\\
95.54	0\\
95.55	0\\
95.56	0\\
95.57	0\\
95.58	0\\
95.59	0\\
95.6	0\\
95.61	0\\
95.62	0\\
95.63	0\\
95.64	0\\
95.65	0\\
95.66	0\\
95.67	0\\
95.68	0\\
95.69	0\\
95.7	0\\
95.71	0\\
95.72	0\\
95.73	0\\
95.74	0\\
95.75	0\\
95.76	0\\
95.77	0\\
95.78	0\\
95.79	0\\
95.8	0\\
95.81	0\\
95.82	0\\
95.83	0\\
95.84	0\\
95.85	0\\
95.86	0\\
95.87	0\\
95.88	0\\
95.89	0\\
95.9	0\\
95.91	0\\
95.92	0\\
95.93	0\\
95.94	0\\
95.95	0\\
95.96	0\\
95.97	0\\
95.98	0\\
95.99	0\\
96	0\\
96.01	0\\
96.02	0\\
96.03	0\\
96.04	0\\
96.05	0\\
96.06	0\\
96.07	0\\
96.08	0\\
96.09	0\\
96.1	0\\
96.11	0\\
96.12	0\\
96.13	0\\
96.14	0\\
96.15	0\\
96.16	0\\
96.17	0\\
96.18	0\\
96.19	0\\
96.2	0\\
96.21	0\\
96.22	0\\
96.23	0\\
96.24	0\\
96.25	0\\
96.26	0\\
96.27	0\\
96.28	0\\
96.29	0\\
96.3	0\\
96.31	0\\
96.32	0\\
96.33	0\\
96.34	0\\
96.35	0\\
96.36	0\\
96.37	0\\
96.38	0\\
96.39	0\\
96.4	0\\
96.41	0\\
96.42	0\\
96.43	0\\
96.44	0\\
96.45	0\\
96.46	0\\
96.47	0\\
96.48	0\\
96.49	0\\
96.5	0\\
96.51	0\\
96.52	0\\
96.53	0\\
96.54	0\\
96.55	0\\
96.56	0\\
96.57	0\\
96.58	0\\
96.59	0\\
96.6	0\\
96.61	0\\
96.62	0\\
96.63	0\\
96.64	0\\
96.65	0\\
96.66	0\\
96.67	0\\
96.68	0\\
96.69	0\\
96.7	0\\
96.71	0\\
96.72	0\\
96.73	0\\
96.74	0\\
96.75	0\\
96.76	0\\
96.77	0\\
96.78	0\\
96.79	0\\
96.8	0\\
96.81	0\\
96.82	0\\
96.83	0\\
96.84	0\\
96.85	0\\
96.86	0\\
96.87	0\\
96.88	0\\
96.89	0\\
96.9	0\\
96.91	0\\
96.92	0\\
96.93	0\\
96.94	0\\
96.95	0\\
96.96	0\\
96.97	0\\
96.98	0\\
96.99	0\\
97	0\\
97.01	0\\
97.02	0\\
97.03	0\\
97.04	0\\
97.05	0\\
97.06	0\\
97.07	0\\
97.08	0\\
97.09	0\\
97.1	0\\
97.11	0\\
97.12	0\\
97.13	0\\
97.14	0\\
97.15	0\\
97.16	0\\
97.17	0\\
97.18	0\\
97.19	0\\
97.2	0\\
97.21	0\\
97.22	0\\
97.23	0\\
97.24	0\\
97.25	0\\
97.26	0\\
97.27	0\\
97.28	0\\
97.29	0\\
97.3	0\\
97.31	0\\
97.32	0\\
97.33	0\\
97.34	0\\
97.35	0\\
97.36	0\\
97.37	0\\
97.38	0\\
97.39	0\\
97.4	0\\
97.41	0\\
97.42	0\\
97.43	0\\
97.44	0\\
97.45	0\\
97.46	0\\
97.47	0\\
97.48	0\\
97.49	0\\
97.5	0\\
97.51	0\\
97.52	0\\
97.53	0\\
97.54	0\\
97.55	0\\
97.56	0\\
97.57	0\\
97.58	0\\
97.59	0\\
97.6	0\\
97.61	0\\
97.62	0\\
97.63	0\\
97.64	0\\
97.65	0\\
97.66	0\\
97.67	0\\
97.68	0\\
97.69	0\\
97.7	0\\
97.71	0\\
97.72	0\\
97.73	0\\
97.74	0\\
97.75	0\\
97.76	0\\
97.77	0\\
97.78	0\\
97.79	0\\
97.8	0\\
97.81	0\\
97.82	0\\
97.83	0\\
97.84	0\\
97.85	0\\
97.86	0\\
97.87	0\\
97.88	0\\
97.89	0\\
97.9	0\\
97.91	0\\
97.92	0\\
97.93	0\\
97.94	0\\
97.95	0\\
97.96	0\\
97.97	0\\
97.98	0\\
97.99	0\\
98	0\\
98.01	0\\
98.02	0\\
98.03	0\\
98.04	0\\
98.05	0\\
98.06	0\\
98.07	0\\
98.08	0\\
98.09	0\\
98.1	0\\
98.11	0\\
98.12	0\\
98.13	0\\
98.14	0\\
98.15	0\\
98.16	0\\
98.17	0\\
98.18	0\\
98.19	0\\
98.2	0\\
98.21	0\\
98.22	0\\
98.23	0\\
98.24	0\\
98.25	0\\
98.26	0\\
98.27	0\\
98.28	0\\
98.29	0\\
98.3	0\\
98.31	0\\
98.32	0\\
98.33	0\\
98.34	0\\
98.35	0\\
98.36	0\\
98.37	0\\
98.38	0\\
98.39	0\\
98.4	0\\
98.41	0\\
98.42	0\\
98.43	0\\
98.44	0\\
98.45	0\\
98.46	0\\
98.47	0\\
98.48	0\\
98.49	0\\
98.5	0\\
98.51	0\\
98.52	0\\
98.53	0\\
98.54	0\\
98.55	0\\
98.56	0\\
98.57	0\\
98.58	0\\
98.59	0\\
98.6	0\\
98.61	0\\
98.62	0\\
98.63	0\\
98.64	0\\
98.65	0\\
98.66	0\\
98.67	0\\
98.68	0\\
98.69	0\\
98.7	0\\
98.71	0\\
98.72	0\\
98.73	0\\
98.74	0\\
98.75	0\\
98.76	0\\
98.77	0\\
98.78	0\\
98.79	0\\
98.8	0\\
98.81	0\\
98.82	0\\
98.83	0\\
98.84	0\\
98.85	0\\
98.86	0\\
98.87	0\\
98.88	0\\
98.89	0\\
98.9	0\\
98.91	0\\
98.92	0\\
98.93	0\\
98.94	0\\
98.95	0\\
98.96	0\\
98.97	0\\
98.98	0\\
98.99	0\\
99	0\\
99.01	0\\
99.02	0\\
99.03	0\\
99.04	0\\
99.05	0\\
99.06	0\\
99.07	0\\
99.08	0\\
99.09	0\\
99.1	0\\
99.11	0\\
99.12	0\\
99.13	0\\
99.14	0\\
99.15	0\\
99.16	0\\
99.17	0\\
99.18	0\\
99.19	0\\
99.2	0\\
99.21	0\\
99.22	0\\
99.23	0\\
99.24	0\\
99.25	0\\
99.26	0\\
99.27	0\\
99.28	0\\
99.29	0\\
99.3	0\\
99.31	0\\
99.32	0\\
99.33	0\\
99.34	0\\
99.35	0\\
99.36	0\\
99.37	0\\
99.38	0\\
99.39	0\\
99.4	0\\
99.41	0\\
99.42	0\\
99.43	0\\
99.44	0\\
99.45	0\\
99.46	0\\
99.47	0\\
99.48	0\\
99.49	0\\
99.5	0\\
99.51	0\\
99.52	0\\
99.53	0\\
99.54	0\\
99.55	0\\
99.56	0\\
99.57	0\\
99.58	0\\
99.59	0\\
99.6	0\\
99.61	0\\
99.62	0\\
99.63	0\\
99.64	0\\
99.65	0\\
99.66	0\\
99.67	0\\
99.68	0\\
99.69	0\\
99.7	0\\
99.71	0\\
99.72	0\\
99.73	0\\
99.74	0\\
99.75	0\\
99.76	0\\
99.77	0\\
99.78	0\\
99.79	0\\
99.8	0\\
99.81	0\\
99.82	0\\
99.83	0\\
99.84	0\\
99.85	0\\
99.86	0\\
99.87	0\\
99.88	0\\
99.89	0\\
99.9	0\\
99.91	0\\
99.92	0\\
99.93	0\\
99.94	0\\
99.95	0\\
99.96	0\\
99.97	0\\
99.98	0\\
99.99	0\\
100	0\\
};
\addlegendentry{$q=-4$};

\addplot [color=mycolor1,dashed,forget plot]
  table[row sep=crcr]{%
0.01	0\\
0.02	0\\
0.03	0\\
0.04	0\\
0.05	0\\
0.06	0\\
0.07	0\\
0.08	0\\
0.09	0\\
0.1	0\\
0.11	0\\
0.12	0\\
0.13	0\\
0.14	0\\
0.15	0\\
0.16	0\\
0.17	1.73472347597681e-18\\
0.18	1.73472347597681e-18\\
0.19	0\\
0.2	0\\
0.21	0\\
0.22	0\\
0.23	0\\
0.24	0\\
0.25	0\\
0.26	0\\
0.27	0\\
0.28	0\\
0.29	0\\
0.3	0\\
0.31	0\\
0.32	0\\
0.33	0\\
0.34	1.73472347597681e-18\\
0.35	1.73472347597681e-18\\
0.36	0\\
0.37	0\\
0.38	0\\
0.39	0\\
0.4	0\\
0.41	0\\
0.42	0\\
0.43	0\\
0.44	0\\
0.45	0\\
0.46	0\\
0.47	0\\
0.48	0\\
0.49	0\\
0.5	0\\
0.51	1.73472347597681e-18\\
0.52	1.73472347597681e-18\\
0.53	0\\
0.54	0\\
0.55	1.73472347597681e-18\\
0.56	0\\
0.57	0\\
0.58	0\\
0.59	0\\
0.6	1.73472347597681e-18\\
0.61	0\\
0.62	0\\
0.63	1.73472347597681e-18\\
0.64	1.73472347597681e-18\\
0.65	0\\
0.66	0\\
0.67	1.73472347597681e-18\\
0.68	0\\
0.69	0\\
0.7	0\\
0.71	0\\
0.72	0\\
0.73	0\\
0.74	0\\
0.75	0\\
0.76	0\\
0.77	1.73472347597681e-18\\
0.78	0\\
0.79	0\\
0.8	0\\
0.81	0\\
0.82	0\\
0.83	0\\
0.84	0\\
0.85	0\\
0.86	0\\
0.87	0\\
0.88	0\\
0.89	0\\
0.9	0\\
0.91	0\\
0.92	0\\
0.93	0\\
0.94	0\\
0.95	0\\
0.96	0\\
0.97	0\\
0.98	0\\
0.99	1.73472347597681e-18\\
1	0\\
1.01	1.73472347597681e-18\\
1.02	0\\
1.03	0\\
1.04	1.73472347597681e-18\\
1.05	0\\
1.06	0\\
1.07	0\\
1.08	0\\
1.09	0\\
1.1	0\\
1.11	0\\
1.12	0\\
1.13	1.73472347597681e-18\\
1.14	0\\
1.15	0\\
1.16	0\\
1.17	0\\
1.18	0\\
1.19	1.73472347597681e-18\\
1.2	0\\
1.21	0\\
1.22	1.73472347597681e-18\\
1.23	1.73472347597681e-18\\
1.24	0\\
1.25	1.73472347597681e-18\\
1.26	0\\
1.27	0\\
1.28	0\\
1.29	0\\
1.3	0\\
1.31	0\\
1.32	0\\
1.33	0\\
1.34	0\\
1.35	0\\
1.36	0\\
1.37	0\\
1.38	0\\
1.39	0\\
1.4	0\\
1.41	0\\
1.42	0\\
1.43	0\\
1.44	0\\
1.45	0\\
1.46	0\\
1.47	0\\
1.48	0\\
1.49	0\\
1.5	0\\
1.51	0\\
1.52	0\\
1.53	0\\
1.54	0\\
1.55	0\\
1.56	0\\
1.57	0\\
1.58	0\\
1.59	0\\
1.6	0\\
1.61	0\\
1.62	0\\
1.63	0\\
1.64	0\\
1.65	0\\
1.66	0\\
1.67	0\\
1.68	0\\
1.69	0\\
1.7	0\\
1.71	0\\
1.72	0\\
1.73	0\\
1.74	0\\
1.75	0\\
1.76	0\\
1.77	0\\
1.78	0\\
1.79	0\\
1.8	1.73472347597681e-18\\
1.81	0\\
1.82	0\\
1.83	0\\
1.84	0\\
1.85	0\\
1.86	0\\
1.87	0\\
1.88	0\\
1.89	0\\
1.9	0\\
1.91	0\\
1.92	0\\
1.93	0\\
1.94	0\\
1.95	0\\
1.96	1.73472347597681e-18\\
1.97	0\\
1.98	0\\
1.99	0\\
2	0\\
2.01	0\\
2.02	0\\
2.03	1.73472347597681e-18\\
2.04	0\\
2.05	1.73472347597681e-18\\
2.06	1.73472347597681e-18\\
2.07	0\\
2.08	0\\
2.09	1.73472347597681e-18\\
2.1	0\\
2.11	0\\
2.12	0\\
2.13	1.73472347597681e-18\\
2.14	0\\
2.15	0\\
2.16	0\\
2.17	0\\
2.18	0\\
2.19	0\\
2.2	0\\
2.21	0\\
2.22	0\\
2.23	0\\
2.24	1.73472347597681e-18\\
2.25	1.73472347597681e-18\\
2.26	0\\
2.27	0\\
2.28	0\\
2.29	0\\
2.3	1.73472347597681e-18\\
2.31	0\\
2.32	0\\
2.33	0\\
2.34	0\\
2.35	1.73472347597681e-18\\
2.36	0\\
2.37	0\\
2.38	0\\
2.39	0\\
2.4	0\\
2.41	0\\
2.42	0\\
2.43	0\\
2.44	0\\
2.45	1.73472347597681e-18\\
2.46	0\\
2.47	0\\
2.48	0\\
2.49	0\\
2.5	0\\
2.51	0\\
2.52	1.73472347597681e-18\\
2.53	0\\
2.54	0\\
2.55	1.73472347597681e-18\\
2.56	0\\
2.57	0\\
2.58	0\\
2.59	0\\
2.6	0\\
2.61	0\\
2.62	0\\
2.63	0\\
2.64	0\\
2.65	0\\
2.66	0\\
2.67	1.73472347597681e-18\\
2.68	0\\
2.69	0\\
2.7	0\\
2.71	0\\
2.72	0\\
2.73	0\\
2.74	0\\
2.75	0\\
2.76	0\\
2.77	0\\
2.78	0\\
2.79	0\\
2.8	0\\
2.81	0\\
2.82	0\\
2.83	0\\
2.84	0\\
2.85	0\\
2.86	0\\
2.87	0\\
2.88	0\\
2.89	0\\
2.9	0\\
2.91	0\\
2.92	0\\
2.93	0\\
2.94	0\\
2.95	0\\
2.96	1.73472347597681e-18\\
2.97	0\\
2.98	0\\
2.99	0\\
3	1.73472347597681e-18\\
3.01	0\\
3.02	0\\
3.03	0\\
3.04	0\\
3.05	0\\
3.06	0\\
3.07	0\\
3.08	0\\
3.09	0\\
3.1	0\\
3.11	0\\
3.12	0\\
3.13	1.73472347597681e-18\\
3.14	0\\
3.15	0\\
3.16	0\\
3.17	0\\
3.18	0\\
3.19	0\\
3.2	0\\
3.21	1.73472347597681e-18\\
3.22	0\\
3.23	1.73472347597681e-18\\
3.24	0\\
3.25	0\\
3.26	0\\
3.27	0\\
3.28	0\\
3.29	0\\
3.3	0\\
3.31	0\\
3.32	0\\
3.33	0\\
3.34	0\\
3.35	0\\
3.36	0\\
3.37	0\\
3.38	0\\
3.39	1.73472347597681e-18\\
3.4	0\\
3.41	1.73472347597681e-18\\
3.42	1.73472347597681e-18\\
3.43	0\\
3.44	1.73472347597681e-18\\
3.45	0\\
3.46	0\\
3.47	0\\
3.48	0\\
3.49	0\\
3.5	1.73472347597681e-18\\
3.51	0\\
3.52	0\\
3.53	0\\
3.54	0\\
3.55	0\\
3.56	0\\
3.57	0\\
3.58	0\\
3.59	1.73472347597681e-18\\
3.6	0\\
3.61	0\\
3.62	0\\
3.63	0\\
3.64	0\\
3.65	0\\
3.66	0\\
3.67	0\\
3.68	0\\
3.69	0\\
3.7	0\\
3.71	0\\
3.72	0\\
3.73	0\\
3.74	0\\
3.75	0\\
3.76	0\\
3.77	0\\
3.78	0\\
3.79	0\\
3.8	0\\
3.81	1.73472347597681e-18\\
3.82	0\\
3.83	0\\
3.84	0\\
3.85	0\\
3.86	0\\
3.87	0\\
3.88	0\\
3.89	0\\
3.9	0\\
3.91	0\\
3.92	0\\
3.93	0\\
3.94	0\\
3.95	0\\
3.96	0\\
3.97	0\\
3.98	1.73472347597681e-18\\
3.99	0\\
4	0\\
4.01	1.73472347597681e-18\\
4.02	0\\
4.03	0\\
4.04	0\\
4.05	1.73472347597681e-18\\
4.06	0\\
4.07	1.73472347597681e-18\\
4.08	0\\
4.09	0\\
4.1	0\\
4.11	0\\
4.12	0\\
4.13	0\\
4.14	0\\
4.15	1.73472347597681e-18\\
4.16	0\\
4.17	0\\
4.18	0\\
4.19	0\\
4.2	0\\
4.21	1.73472347597681e-18\\
4.22	0\\
4.23	0\\
4.24	0\\
4.25	0\\
4.26	1.73472347597681e-18\\
4.27	0\\
4.28	0\\
4.29	0\\
4.3	0\\
4.31	0\\
4.32	0\\
4.33	0\\
4.34	0\\
4.35	0\\
4.36	0\\
4.37	0\\
4.38	0\\
4.39	0\\
4.4	0\\
4.41	0\\
4.42	0\\
4.43	0\\
4.44	0\\
4.45	0\\
4.46	0\\
4.47	0\\
4.48	0\\
4.49	0\\
4.5	0\\
4.51	1.73472347597681e-18\\
4.52	0\\
4.53	0\\
4.54	0\\
4.55	0\\
4.56	0\\
4.57	0\\
4.58	0\\
4.59	0\\
4.6	0\\
4.61	0\\
4.62	1.73472347597681e-18\\
4.63	0\\
4.64	0\\
4.65	0\\
4.66	0\\
4.67	0\\
4.68	0\\
4.69	0\\
4.7	0\\
4.71	0\\
4.72	0\\
4.73	0\\
4.74	0\\
4.75	0\\
4.76	0\\
4.77	0\\
4.78	0\\
4.79	0\\
4.8	0\\
4.81	0\\
4.82	1.73472347597681e-18\\
4.83	0\\
4.84	0\\
4.85	1.73472347597681e-18\\
4.86	0\\
4.87	0\\
4.88	0\\
4.89	0\\
4.9	0\\
4.91	0\\
4.92	0\\
4.93	0\\
4.94	0\\
4.95	0\\
4.96	0\\
4.97	0\\
4.98	1.73472347597681e-18\\
4.99	0\\
5	0\\
5.01	0\\
5.02	0\\
5.03	0\\
5.04	0\\
5.05	0\\
5.06	1.73472347597681e-18\\
5.07	1.73472347597681e-18\\
5.08	0\\
5.09	0\\
5.1	0\\
5.11	1.73472347597681e-18\\
5.12	0\\
5.13	0\\
5.14	0\\
5.15	0\\
5.16	0\\
5.17	0\\
5.18	0\\
5.19	0\\
5.2	0\\
5.21	0\\
5.22	0\\
5.23	1.73472347597681e-18\\
5.24	0\\
5.25	0\\
5.26	0\\
5.27	0\\
5.28	0\\
5.29	0\\
5.3	0\\
5.31	0\\
5.32	1.73472347597681e-18\\
5.33	0\\
5.34	1.73472347597681e-18\\
5.35	1.73472347597681e-18\\
5.36	0\\
5.37	0\\
5.38	0\\
5.39	0\\
5.4	0\\
5.41	0\\
5.42	0\\
5.43	0\\
5.44	0\\
5.45	0\\
5.46	1.73472347597681e-18\\
5.47	0\\
5.48	0\\
5.49	1.73472347597681e-18\\
5.5	0\\
5.51	0\\
5.52	0\\
5.53	0\\
5.54	0\\
5.55	0\\
5.56	0\\
5.57	1.73472347597681e-18\\
5.58	0\\
5.59	0\\
5.6	0\\
5.61	0\\
5.62	0\\
5.63	1.73472347597681e-18\\
5.64	0\\
5.65	0\\
5.66	0\\
5.67	0\\
5.68	0\\
5.69	1.73472347597681e-18\\
5.7	0\\
5.71	1.73472347597681e-18\\
5.72	0\\
5.73	0\\
5.74	0\\
5.75	0\\
5.76	1.73472347597681e-18\\
5.77	0\\
5.78	0\\
5.79	0\\
5.8	0\\
5.81	0\\
5.82	0\\
5.83	1.73472347597681e-18\\
5.84	0\\
5.85	0\\
5.86	0\\
5.87	0\\
5.88	0\\
5.89	0\\
5.9	0\\
5.91	0\\
5.92	0\\
5.93	0\\
5.94	0\\
5.95	0\\
5.96	0\\
5.97	0\\
5.98	0\\
5.99	0\\
6	0\\
6.01	0\\
6.02	0\\
6.03	0\\
6.04	1.73472347597681e-18\\
6.05	0\\
6.06	0\\
6.07	0\\
6.08	0\\
6.09	0\\
6.1	0\\
6.11	1.73472347597681e-18\\
6.12	0\\
6.13	0\\
6.14	0\\
6.15	0\\
6.16	0\\
6.17	1.73472347597681e-18\\
6.18	0\\
6.19	1.73472347597681e-18\\
6.2	0\\
6.21	0\\
6.22	0\\
6.23	0\\
6.24	0\\
6.25	1.73472347597681e-18\\
6.26	0\\
6.27	0\\
6.28	1.73472347597681e-18\\
6.29	0\\
6.3	0\\
6.31	0\\
6.32	0\\
6.33	1.73472347597681e-18\\
6.34	0\\
6.35	1.73472347597681e-18\\
6.36	1.73472347597681e-18\\
6.37	0\\
6.38	1.73472347597681e-18\\
6.39	0\\
6.4	0\\
6.41	0\\
6.42	1.73472347597681e-18\\
6.43	0\\
6.44	0\\
6.45	0\\
6.46	1.73472347597681e-18\\
6.47	0\\
6.48	1.73472347597681e-18\\
6.49	0\\
6.5	0\\
6.51	0\\
6.52	0\\
6.53	0\\
6.54	0\\
6.55	0\\
6.56	1.73472347597681e-18\\
6.57	0\\
6.58	0\\
6.59	0\\
6.6	0\\
6.61	1.73472347597681e-18\\
6.62	0\\
6.63	0\\
6.64	0\\
6.65	0\\
6.66	0\\
6.67	0\\
6.68	0\\
6.69	0\\
6.7	0\\
6.71	0\\
6.72	0\\
6.73	0\\
6.74	0\\
6.75	0\\
6.76	0\\
6.77	0\\
6.78	0\\
6.79	0\\
6.8	0\\
6.81	0\\
6.82	0\\
6.83	0\\
6.84	0\\
6.85	0\\
6.86	1.73472347597681e-18\\
6.87	0\\
6.88	0\\
6.89	0\\
6.9	0\\
6.91	0\\
6.92	0\\
6.93	0\\
6.94	0\\
6.95	0\\
6.96	0\\
6.97	0\\
6.98	0\\
6.99	0\\
7	0\\
7.01	0\\
7.02	0\\
7.03	1.73472347597681e-18\\
7.04	0\\
7.05	0\\
7.06	0\\
7.07	0\\
7.08	0\\
7.09	0\\
7.1	0\\
7.11	0\\
7.12	0\\
7.13	0\\
7.14	0\\
7.15	0\\
7.16	0\\
7.17	0\\
7.18	0\\
7.19	1.73472347597681e-18\\
7.2	0\\
7.21	0\\
7.22	0\\
7.23	0\\
7.24	0\\
7.25	0\\
7.26	0\\
7.27	1.73472347597681e-18\\
7.28	0\\
7.29	0\\
7.3	0\\
7.31	0\\
7.32	0\\
7.33	0\\
7.34	0\\
7.35	0\\
7.36	0\\
7.37	1.73472347597681e-18\\
7.38	0\\
7.39	0\\
7.4	0\\
7.41	0\\
7.42	0\\
7.43	0\\
7.44	0\\
7.45	0\\
7.46	0\\
7.47	0\\
7.48	0\\
7.49	0\\
7.5	0\\
7.51	0\\
7.52	0\\
7.53	0\\
7.54	0\\
7.55	1.73472347597681e-18\\
7.56	1.73472347597681e-18\\
7.57	0\\
7.58	1.73472347597681e-18\\
7.59	0\\
7.6	0\\
7.61	0\\
7.62	0\\
7.63	0\\
7.64	0\\
7.65	0\\
7.66	0\\
7.67	0\\
7.68	0\\
7.69	0\\
7.7	0\\
7.71	0\\
7.72	0\\
7.73	0\\
7.74	0\\
7.75	0\\
7.76	0\\
7.77	0\\
7.78	0\\
7.79	0\\
7.8	0\\
7.81	0\\
7.82	0\\
7.83	0\\
7.84	0\\
7.85	1.73472347597681e-18\\
7.86	0\\
7.87	0\\
7.88	0\\
7.89	0\\
7.9	0\\
7.91	0\\
7.92	0\\
7.93	0\\
7.94	0\\
7.95	1.73472347597681e-18\\
7.96	0\\
7.97	0\\
7.98	0\\
7.99	0\\
8	0\\
8.01	0\\
8.02	0\\
8.03	1.73472347597681e-18\\
8.04	1.73472347597681e-18\\
8.05	1.73472347597681e-18\\
8.06	0\\
8.07	1.73472347597681e-18\\
8.08	0\\
8.09	1.73472347597681e-18\\
8.1	0\\
8.11	0\\
8.12	0\\
8.13	1.73472347597681e-18\\
8.14	0\\
8.15	0\\
8.16	0\\
8.17	0\\
8.18	0\\
8.19	0\\
8.2	0\\
8.21	0\\
8.22	0\\
8.23	0\\
8.24	0\\
8.25	0\\
8.26	0\\
8.27	0\\
8.28	0\\
8.29	0\\
8.3	0\\
8.31	0\\
8.32	0\\
8.33	0\\
8.34	0\\
8.35	0\\
8.36	0\\
8.37	0\\
8.38	0\\
8.39	1.73472347597681e-18\\
8.4	0\\
8.41	0\\
8.42	0\\
8.43	0\\
8.44	0\\
8.45	1.73472347597681e-18\\
8.46	0\\
8.47	0\\
8.48	0\\
8.49	0\\
8.5	0\\
8.51	0\\
8.52	0\\
8.53	0\\
8.54	0\\
8.55	0\\
8.56	0\\
8.57	0\\
8.58	0\\
8.59	0\\
8.6	0\\
8.61	0\\
8.62	0\\
8.63	0\\
8.64	1.73472347597681e-18\\
8.65	0\\
8.66	0\\
8.67	0\\
8.68	1.73472347597681e-18\\
8.69	0\\
8.7	0\\
8.71	0\\
8.72	0\\
8.73	0\\
8.74	0\\
8.75	0\\
8.76	0\\
8.77	0\\
8.78	0\\
8.79	1.73472347597681e-18\\
8.8	0\\
8.81	0\\
8.82	0\\
8.83	0\\
8.84	1.73472347597681e-18\\
8.85	0\\
8.86	1.73472347597681e-18\\
8.87	0\\
8.88	0\\
8.89	1.73472347597681e-18\\
8.9	0\\
8.91	0\\
8.92	0\\
8.93	0\\
8.94	0\\
8.95	0\\
8.96	0\\
8.97	0\\
8.98	0\\
8.99	0\\
9	0\\
9.01	0\\
9.02	0\\
9.03	1.73472347597681e-18\\
9.04	0\\
9.05	0\\
9.06	0\\
9.07	0\\
9.08	0\\
9.09	0\\
9.1	1.73472347597681e-18\\
9.11	0\\
9.12	0\\
9.13	0\\
9.14	0\\
9.15	1.73472347597681e-18\\
9.16	0\\
9.17	1.73472347597681e-18\\
9.18	0\\
9.19	1.73472347597681e-18\\
9.2	0\\
9.21	0\\
9.22	1.73472347597681e-18\\
9.23	0\\
9.24	0\\
9.25	1.73472347597681e-18\\
9.26	1.73472347597681e-18\\
9.27	0\\
9.28	0\\
9.29	0\\
9.3	0\\
9.31	1.73472347597681e-18\\
9.32	0\\
9.33	1.73472347597681e-18\\
9.34	0\\
9.35	0\\
9.36	0\\
9.37	0\\
9.38	0\\
9.39	1.73472347597681e-18\\
9.4	0\\
9.41	1.73472347597681e-18\\
9.42	0\\
9.43	0\\
9.44	0\\
9.45	0\\
9.46	0\\
9.47	0\\
9.48	0\\
9.49	0\\
9.5	0\\
9.51	1.73472347597681e-18\\
9.52	0\\
9.53	0\\
9.54	0\\
9.55	1.73472347597681e-18\\
9.56	0\\
9.57	0\\
9.58	0\\
9.59	0\\
9.6	0\\
9.61	0\\
9.62	0\\
9.63	0\\
9.64	1.73472347597681e-18\\
9.65	0\\
9.66	0\\
9.67	0\\
9.68	1.73472347597681e-18\\
9.69	0\\
9.7	0\\
9.71	0\\
9.72	1.73472347597681e-18\\
9.73	0\\
9.74	0\\
9.75	0\\
9.76	0\\
9.77	0\\
9.78	1.73472347597681e-18\\
9.79	0\\
9.8	0\\
9.81	1.73472347597681e-18\\
9.82	0\\
9.83	0\\
9.84	0\\
9.85	0\\
9.86	0\\
9.87	1.73472347597681e-18\\
9.88	0\\
9.89	0\\
9.9	1.73472347597681e-18\\
9.91	0\\
9.92	0\\
9.93	0\\
9.94	0\\
9.95	0\\
9.96	0\\
9.97	0\\
9.98	1.73472347597681e-18\\
9.99	0\\
10	0\\
10.01	0\\
10.02	0\\
10.03	0\\
10.04	0\\
10.05	0\\
10.06	0\\
10.07	0\\
10.08	0\\
10.09	0\\
10.1	0\\
10.11	0\\
10.12	0\\
10.13	0\\
10.14	0\\
10.15	0\\
10.16	0\\
10.17	0\\
10.18	0\\
10.19	0\\
10.2	0\\
10.21	0\\
10.22	0\\
10.23	0\\
10.24	0\\
10.25	0\\
10.26	0\\
10.27	0\\
10.28	0\\
10.29	0\\
10.3	0\\
10.31	0\\
10.32	0\\
10.33	0\\
10.34	0\\
10.35	0\\
10.36	1.73472347597681e-18\\
10.37	1.73472347597681e-18\\
10.38	0\\
10.39	0\\
10.4	0\\
10.41	0\\
10.42	1.73472347597681e-18\\
10.43	1.73472347597681e-18\\
10.44	0\\
10.45	0\\
10.46	0\\
10.47	0\\
10.48	0\\
10.49	0\\
10.5	0\\
10.51	0\\
10.52	0\\
10.53	0\\
10.54	0\\
10.55	0\\
10.56	0\\
10.57	0\\
10.58	0\\
10.59	1.73472347597681e-18\\
10.6	0\\
10.61	1.73472347597681e-18\\
10.62	0\\
10.63	0\\
10.64	0\\
10.65	0\\
10.66	0\\
10.67	0\\
10.68	0\\
10.69	0\\
10.7	0\\
10.71	0\\
10.72	0\\
10.73	0\\
10.74	0\\
10.75	0\\
10.76	1.73472347597681e-18\\
10.77	1.73472347597681e-18\\
10.78	0\\
10.79	1.73472347597681e-18\\
10.8	0\\
10.81	0\\
10.82	0\\
10.83	0\\
10.84	0\\
10.85	0\\
10.86	0\\
10.87	0\\
10.88	1.73472347597681e-18\\
10.89	0\\
10.9	1.73472347597681e-18\\
10.91	0\\
10.92	0\\
10.93	0\\
10.94	0\\
10.95	0\\
10.96	0\\
10.97	0\\
10.98	0\\
10.99	0\\
11	0\\
11.01	0\\
11.02	0\\
11.03	0\\
11.04	0\\
11.05	0\\
11.06	0\\
11.07	0\\
11.08	0\\
11.09	0\\
11.1	0\\
11.11	1.73472347597681e-18\\
11.12	0\\
11.13	0\\
11.14	0\\
11.15	0\\
11.16	1.73472347597681e-18\\
11.17	0\\
11.18	0\\
11.19	1.73472347597681e-18\\
11.2	0\\
11.21	0\\
11.22	0\\
11.23	0\\
11.24	1.73472347597681e-18\\
11.25	0\\
11.26	0\\
11.27	0\\
11.28	0\\
11.29	0\\
11.3	0\\
11.31	0\\
11.32	0\\
11.33	0\\
11.34	0\\
11.35	1.73472347597681e-18\\
11.36	0\\
11.37	0\\
11.38	0\\
11.39	1.73472347597681e-18\\
11.4	0\\
11.41	1.73472347597681e-18\\
11.42	0\\
11.43	0\\
11.44	0\\
11.45	0\\
11.46	0\\
11.47	0\\
11.48	0\\
11.49	0\\
11.5	0\\
11.51	0\\
11.52	0\\
11.53	0\\
11.54	0\\
11.55	0\\
11.56	0\\
11.57	0\\
11.58	1.73472347597681e-18\\
11.59	1.73472347597681e-18\\
11.6	0\\
11.61	0\\
11.62	0\\
11.63	1.73472347597681e-18\\
11.64	0\\
11.65	0\\
11.66	0\\
11.67	0\\
11.68	0\\
11.69	1.73472347597681e-18\\
11.7	0\\
11.71	0\\
11.72	0\\
11.73	0\\
11.74	0\\
11.75	0\\
11.76	0\\
11.77	1.73472347597681e-18\\
11.78	0\\
11.79	0\\
11.8	0\\
11.81	0\\
11.82	0\\
11.83	0\\
11.84	0\\
11.85	0\\
11.86	0\\
11.87	0\\
11.88	0\\
11.89	0\\
11.9	0\\
11.91	1.73472347597681e-18\\
11.92	0\\
11.93	0\\
11.94	1.73472347597681e-18\\
11.95	0\\
11.96	0\\
11.97	0\\
11.98	0\\
11.99	0\\
12	0\\
12.01	0\\
12.02	0\\
12.03	0\\
12.04	1.73472347597681e-18\\
12.05	0\\
12.06	0\\
12.07	0\\
12.08	0\\
12.09	1.73472347597681e-18\\
12.1	0\\
12.11	1.73472347597681e-18\\
12.12	0\\
12.13	0\\
12.14	0\\
12.15	0\\
12.16	0\\
12.17	1.73472347597681e-18\\
12.18	1.73472347597681e-18\\
12.19	1.73472347597681e-18\\
12.2	0\\
12.21	0\\
12.22	0\\
12.23	0\\
12.24	0\\
12.25	1.73472347597681e-18\\
12.26	0\\
12.27	0\\
12.28	1.73472347597681e-18\\
12.29	0\\
12.3	0\\
12.31	0\\
12.32	0\\
12.33	0\\
12.34	0\\
12.35	1.73472347597681e-18\\
12.36	0\\
12.37	0\\
12.38	1.73472347597681e-18\\
12.39	0\\
12.4	0\\
12.41	0\\
12.42	0\\
12.43	0\\
12.44	0\\
12.45	0\\
12.46	0\\
12.47	1.73472347597681e-18\\
12.48	0\\
12.49	0\\
12.5	0\\
12.51	0\\
12.52	0\\
12.53	0\\
12.54	0\\
12.55	0\\
12.56	0\\
12.57	0\\
12.58	0\\
12.59	0\\
12.6	0\\
12.61	0\\
12.62	0\\
12.63	0\\
12.64	0\\
12.65	0\\
12.66	0\\
12.67	0\\
12.68	0\\
12.69	0\\
12.7	0\\
12.71	0\\
12.72	1.73472347597681e-18\\
12.73	0\\
12.74	0\\
12.75	0\\
12.76	0\\
12.77	0\\
12.78	0\\
12.79	0\\
12.8	0\\
12.81	1.73472347597681e-18\\
12.82	0\\
12.83	0\\
12.84	1.73472347597681e-18\\
12.85	0\\
12.86	1.73472347597681e-18\\
12.87	0\\
12.88	0\\
12.89	0\\
12.9	0\\
12.91	0\\
12.92	0\\
12.93	0\\
12.94	0\\
12.95	1.73472347597681e-18\\
12.96	0\\
12.97	0\\
12.98	0\\
12.99	0\\
13	0\\
13.01	0\\
13.02	0\\
13.03	0\\
13.04	1.73472347597681e-18\\
13.05	0\\
13.06	0\\
13.07	1.73472347597681e-18\\
13.08	0\\
13.09	0\\
13.1	0\\
13.11	0\\
13.12	0\\
13.13	0\\
13.14	0\\
13.15	0\\
13.16	0\\
13.17	0\\
13.18	0\\
13.19	0\\
13.2	0\\
13.21	0\\
13.22	0\\
13.23	0\\
13.24	0\\
13.25	1.73472347597681e-18\\
13.26	0\\
13.27	0\\
13.28	0\\
13.29	0\\
13.3	0\\
13.31	1.73472347597681e-18\\
13.32	0\\
13.33	1.73472347597681e-18\\
13.34	1.73472347597681e-18\\
13.35	0\\
13.36	1.73472347597681e-18\\
13.37	0\\
13.38	1.73472347597681e-18\\
13.39	0\\
13.4	0\\
13.41	0\\
13.42	0\\
13.43	0\\
13.44	0\\
13.45	0\\
13.46	0\\
13.47	1.73472347597681e-18\\
13.48	0\\
13.49	0\\
13.5	0\\
13.51	0\\
13.52	0\\
13.53	0\\
13.54	0\\
13.55	0\\
13.56	0\\
13.57	0\\
13.58	0\\
13.59	0\\
13.6	0\\
13.61	1.73472347597681e-18\\
13.62	0\\
13.63	1.73472347597681e-18\\
13.64	0\\
13.65	0\\
13.66	1.73472347597681e-18\\
13.67	1.73472347597681e-18\\
13.68	0\\
13.69	0\\
13.7	1.73472347597681e-18\\
13.71	0\\
13.72	1.73472347597681e-18\\
13.73	0\\
13.74	1.73472347597681e-18\\
13.75	0\\
13.76	0\\
13.77	0\\
13.78	0\\
13.79	1.73472347597681e-18\\
13.8	0\\
13.81	1.73472347597681e-18\\
13.82	1.73472347597681e-18\\
13.83	0\\
13.84	0\\
13.85	0\\
13.86	0\\
13.87	0\\
13.88	0\\
13.89	1.73472347597681e-18\\
13.9	0\\
13.91	0\\
13.92	1.73472347597681e-18\\
13.93	0\\
13.94	0\\
13.95	0\\
13.96	0\\
13.97	0\\
13.98	0\\
13.99	1.73472347597681e-18\\
14	1.73472347597681e-18\\
14.01	1.73472347597681e-18\\
14.02	0\\
14.03	0\\
14.04	0\\
14.05	0\\
14.06	0\\
14.07	0\\
14.08	0\\
14.09	0\\
14.1	1.73472347597681e-18\\
14.11	0\\
14.12	0\\
14.13	0\\
14.14	0\\
14.15	0\\
14.16	0\\
14.17	0\\
14.18	0\\
14.19	1.73472347597681e-18\\
14.2	0\\
14.21	0\\
14.22	0\\
14.23	0\\
14.24	0\\
14.25	0\\
14.26	0\\
14.27	0\\
14.28	0\\
14.29	0\\
14.3	0\\
14.31	0\\
14.32	0\\
14.33	1.73472347597681e-18\\
14.34	0\\
14.35	0\\
14.36	0\\
14.37	0\\
14.38	0\\
14.39	1.73472347597681e-18\\
14.4	0\\
14.41	1.73472347597681e-18\\
14.42	0\\
14.43	0\\
14.44	0\\
14.45	0\\
14.46	1.73472347597681e-18\\
14.47	1.73472347597681e-18\\
14.48	0\\
14.49	0\\
14.5	0\\
14.51	0\\
14.52	0\\
14.53	0\\
14.54	0\\
14.55	0\\
14.56	0\\
14.57	0\\
14.58	0\\
14.59	0\\
14.6	0\\
14.61	0\\
14.62	0\\
14.63	1.73472347597681e-18\\
14.64	0\\
14.65	0\\
14.66	0\\
14.67	0\\
14.68	0\\
14.69	0\\
14.7	1.73472347597681e-18\\
14.71	0\\
14.72	1.73472347597681e-18\\
14.73	1.73472347597681e-18\\
14.74	0\\
14.75	0\\
14.76	0\\
14.77	0\\
14.78	0\\
14.79	0\\
14.8	0\\
14.81	0\\
14.82	0\\
14.83	0\\
14.84	1.73472347597681e-18\\
14.85	0\\
14.86	0\\
14.87	0\\
14.88	0\\
14.89	0\\
14.9	0\\
14.91	1.73472347597681e-18\\
14.92	0\\
14.93	0\\
14.94	0\\
14.95	1.73472347597681e-18\\
14.96	1.73472347597681e-18\\
14.97	0\\
14.98	0\\
14.99	0\\
15	0\\
15.01	1.73472347597681e-18\\
15.02	0\\
15.03	0\\
15.04	0\\
15.05	1.73472347597681e-18\\
15.06	0\\
15.07	0\\
15.08	0\\
15.09	0\\
15.1	0\\
15.11	0\\
15.12	0\\
15.13	0\\
15.14	0\\
15.15	0\\
15.16	0\\
15.17	0\\
15.18	0\\
15.19	0\\
15.2	1.73472347597681e-18\\
15.21	0\\
15.22	0\\
15.23	0\\
15.24	1.73472347597681e-18\\
15.25	1.73472347597681e-18\\
15.26	0\\
15.27	0\\
15.28	0\\
15.29	0\\
15.3	0\\
15.31	0\\
15.32	0\\
15.33	1.73472347597681e-18\\
15.34	0\\
15.35	0\\
15.36	0\\
15.37	0\\
15.38	0\\
15.39	0\\
15.4	0\\
15.41	0\\
15.42	0\\
15.43	0\\
15.44	0\\
15.45	0\\
15.46	0\\
15.47	0\\
15.48	0\\
15.49	0\\
15.5	0\\
15.51	0\\
15.52	0\\
15.53	0\\
15.54	0\\
15.55	0\\
15.56	0\\
15.57	0\\
15.58	1.73472347597681e-18\\
15.59	0\\
15.6	0\\
15.61	0\\
15.62	0\\
15.63	1.73472347597681e-18\\
15.64	0\\
15.65	0\\
15.66	0\\
15.67	0\\
15.68	1.73472347597681e-18\\
15.69	0\\
15.7	0\\
15.71	0\\
15.72	0\\
15.73	0\\
15.74	0\\
15.75	1.73472347597681e-18\\
15.76	0\\
15.77	0\\
15.78	0\\
15.79	0\\
15.8	0\\
15.81	0\\
15.82	0\\
15.83	0\\
15.84	0\\
15.85	0\\
15.86	0\\
15.87	0\\
15.88	0\\
15.89	0\\
15.9	0\\
15.91	0\\
15.92	0\\
15.93	0\\
15.94	0\\
15.95	0\\
15.96	0\\
15.97	0\\
15.98	0\\
15.99	0\\
16	1.73472347597681e-18\\
16.01	0\\
16.02	0\\
16.03	0\\
16.04	0\\
16.05	0\\
16.06	0\\
16.07	0\\
16.08	1.73472347597681e-18\\
16.09	0\\
16.1	0\\
16.11	0\\
16.12	0\\
16.13	0\\
16.14	0\\
16.15	0\\
16.16	0\\
16.17	0\\
16.18	0\\
16.19	0\\
16.2	0\\
16.21	0\\
16.22	0\\
16.23	0\\
16.24	0\\
16.25	0\\
16.26	0\\
16.27	0\\
16.28	0\\
16.29	0\\
16.3	0\\
16.31	0\\
16.32	0\\
16.33	0\\
16.34	0\\
16.35	1.73472347597681e-18\\
16.36	0\\
16.37	1.73472347597681e-18\\
16.38	0\\
16.39	0\\
16.4	0\\
16.41	0\\
16.42	0\\
16.43	0\\
16.44	0\\
16.45	0\\
16.46	0\\
16.47	0\\
16.48	0\\
16.49	0\\
16.5	0\\
16.51	0\\
16.52	0\\
16.53	0\\
16.54	0\\
16.55	0\\
16.56	0\\
16.57	0\\
16.58	0\\
16.59	0\\
16.6	0\\
16.61	0\\
16.62	0\\
16.63	0\\
16.64	1.73472347597681e-18\\
16.65	1.73472347597681e-18\\
16.66	0\\
16.67	0\\
16.68	1.73472347597681e-18\\
16.69	0\\
16.7	0\\
16.71	0\\
16.72	0\\
16.73	0\\
16.74	0\\
16.75	0\\
16.76	1.73472347597681e-18\\
16.77	0\\
16.78	1.73472347597681e-18\\
16.79	0\\
16.8	0\\
16.81	0\\
16.82	0\\
16.83	1.73472347597681e-18\\
16.84	1.73472347597681e-18\\
16.85	0\\
16.86	1.73472347597681e-18\\
16.87	0\\
16.88	0\\
16.89	1.73472347597681e-18\\
16.9	0\\
16.91	0\\
16.92	0\\
16.93	1.73472347597681e-18\\
16.94	0\\
16.95	1.73472347597681e-18\\
16.96	0\\
16.97	0\\
16.98	0\\
16.99	1.73472347597681e-18\\
17	1.73472347597681e-18\\
17.01	1.73472347597681e-18\\
17.02	0\\
17.03	0\\
17.04	0\\
17.05	0\\
17.06	0\\
17.07	0\\
17.08	0\\
17.09	1.73472347597681e-18\\
17.1	0\\
17.11	0\\
17.12	0\\
17.13	1.73472347597681e-18\\
17.14	0\\
17.15	0\\
17.16	0\\
17.17	1.73472347597681e-18\\
17.18	0\\
17.19	0\\
17.2	1.73472347597681e-18\\
17.21	0\\
17.22	1.73472347597681e-18\\
17.23	0\\
17.24	0\\
17.25	0\\
17.26	0\\
17.27	0\\
17.28	0\\
17.29	0\\
17.3	0\\
17.31	0\\
17.32	0\\
17.33	0\\
17.34	0\\
17.35	0\\
17.36	0\\
17.37	0\\
17.38	0\\
17.39	0\\
17.4	0\\
17.41	1.73472347597681e-18\\
17.42	1.73472347597681e-18\\
17.43	0\\
17.44	0\\
17.45	0\\
17.46	1.73472347597681e-18\\
17.47	0\\
17.48	0\\
17.49	0\\
17.5	0\\
17.51	0\\
17.52	0\\
17.53	0\\
17.54	0\\
17.55	0\\
17.56	0\\
17.57	0\\
17.58	0\\
17.59	0\\
17.6	0\\
17.61	0\\
17.62	0\\
17.63	0\\
17.64	0\\
17.65	0\\
17.66	0\\
17.67	0\\
17.68	0\\
17.69	0\\
17.7	0\\
17.71	0\\
17.72	0\\
17.73	0\\
17.74	0\\
17.75	0\\
17.76	0\\
17.77	0\\
17.78	0\\
17.79	1.73472347597681e-18\\
17.8	0\\
17.81	1.73472347597681e-18\\
17.82	0\\
17.83	1.73472347597681e-18\\
17.84	0\\
17.85	0\\
17.86	0\\
17.87	1.73472347597681e-18\\
17.88	0\\
17.89	0\\
17.9	0\\
17.91	0\\
17.92	1.73472347597681e-18\\
17.93	0\\
17.94	0\\
17.95	0\\
17.96	0\\
17.97	1.73472347597681e-18\\
17.98	0\\
17.99	0\\
18	0\\
18.01	0\\
18.02	0\\
18.03	0\\
18.04	0\\
18.05	0\\
18.06	0\\
18.07	0\\
18.08	0\\
18.09	0\\
18.1	0\\
18.11	1.73472347597681e-18\\
18.12	0\\
18.13	0\\
18.14	0\\
18.15	0\\
18.16	0\\
18.17	0\\
18.18	0\\
18.19	0\\
18.2	1.73472347597681e-18\\
18.21	0\\
18.22	0\\
18.23	0\\
18.24	0\\
18.25	0\\
18.26	0\\
18.27	1.73472347597681e-18\\
18.28	0\\
18.29	0\\
18.3	0\\
18.31	0\\
18.32	0\\
18.33	0\\
18.34	0\\
18.35	1.73472347597681e-18\\
18.36	1.73472347597681e-18\\
18.37	0\\
18.38	0\\
18.39	0\\
18.4	0\\
18.41	0\\
18.42	0\\
18.43	1.73472347597681e-18\\
18.44	0\\
18.45	0\\
18.46	0\\
18.47	0\\
18.48	0\\
18.49	0\\
18.5	0\\
18.51	0\\
18.52	0\\
18.53	0\\
18.54	0\\
18.55	0\\
18.56	0\\
18.57	0\\
18.58	0\\
18.59	0\\
18.6	0\\
18.61	0\\
18.62	0\\
18.63	0\\
18.64	0\\
18.65	0\\
18.66	0\\
18.67	0\\
18.68	0\\
18.69	0\\
18.7	0\\
18.71	0\\
18.72	0\\
18.73	1.73472347597681e-18\\
18.74	0\\
18.75	0\\
18.76	0\\
18.77	0\\
18.78	1.73472347597681e-18\\
18.79	0\\
18.8	0\\
18.81	0\\
18.82	0\\
18.83	0\\
18.84	0\\
18.85	0\\
18.86	0\\
18.87	0\\
18.88	0\\
18.89	0\\
18.9	0\\
18.91	0\\
18.92	0\\
18.93	0\\
18.94	0\\
18.95	0\\
18.96	0\\
18.97	0\\
18.98	0\\
18.99	0\\
19	0\\
19.01	1.73472347597681e-18\\
19.02	0\\
19.03	0\\
19.04	1.73472347597681e-18\\
19.05	0\\
19.06	0\\
19.07	1.73472347597681e-18\\
19.08	0\\
19.09	0\\
19.1	0\\
19.11	0\\
19.12	1.73472347597681e-18\\
19.13	0\\
19.14	0\\
19.15	0\\
19.16	0\\
19.17	0\\
19.18	0\\
19.19	1.73472347597681e-18\\
19.2	0\\
19.21	0\\
19.22	0\\
19.23	0\\
19.24	0\\
19.25	0\\
19.26	0\\
19.27	0\\
19.28	0\\
19.29	1.73472347597681e-18\\
19.3	0\\
19.31	0\\
19.32	0\\
19.33	0\\
19.34	0\\
19.35	0\\
19.36	0\\
19.37	0\\
19.38	0\\
19.39	0\\
19.4	0\\
19.41	0\\
19.42	0\\
19.43	0\\
19.44	0\\
19.45	0\\
19.46	0\\
19.47	0\\
19.48	0\\
19.49	1.73472347597681e-18\\
19.5	0\\
19.51	0\\
19.52	0\\
19.53	0\\
19.54	0\\
19.55	0\\
19.56	0\\
19.57	0\\
19.58	0\\
19.59	0\\
19.6	0\\
19.61	0\\
19.62	1.73472347597681e-18\\
19.63	1.73472347597681e-18\\
19.64	0\\
19.65	0\\
19.66	1.73472347597681e-18\\
19.67	0\\
19.68	0\\
19.69	0\\
19.7	0\\
19.71	0\\
19.72	0\\
19.73	0\\
19.74	0\\
19.75	0\\
19.76	0\\
19.77	0\\
19.78	0\\
19.79	0\\
19.8	0\\
19.81	1.73472347597681e-18\\
19.82	1.73472347597681e-18\\
19.83	0\\
19.84	0\\
19.85	0\\
19.86	1.73472347597681e-18\\
19.87	0\\
19.88	0\\
19.89	0\\
19.9	0\\
19.91	0\\
19.92	0\\
19.93	0\\
19.94	1.73472347597681e-18\\
19.95	0\\
19.96	0\\
19.97	0\\
19.98	0\\
19.99	0\\
20	0\\
20.01	0\\
20.02	1.73472347597681e-18\\
20.03	0\\
20.04	0\\
20.05	0\\
20.06	0\\
20.07	0\\
20.08	0\\
20.09	1.73472347597681e-18\\
20.1	0\\
20.11	0\\
20.12	0\\
20.13	1.73472347597681e-18\\
20.14	0\\
20.15	0\\
20.16	0\\
20.17	0\\
20.18	0\\
20.19	0\\
20.2	0\\
20.21	0\\
20.22	1.73472347597681e-18\\
20.23	0\\
20.24	0\\
20.25	0\\
20.26	0\\
20.27	0\\
20.28	0\\
20.29	0\\
20.3	0\\
20.31	0\\
20.32	0\\
20.33	0\\
20.34	0\\
20.35	0\\
20.36	0\\
20.37	0\\
20.38	1.73472347597681e-18\\
20.39	0\\
20.4	0\\
20.41	0\\
20.42	1.73472347597681e-18\\
20.43	0\\
20.44	1.73472347597681e-18\\
20.45	1.73472347597681e-18\\
20.46	0\\
20.47	0\\
20.48	0\\
20.49	0\\
20.5	0\\
20.51	0\\
20.52	0\\
20.53	0\\
20.54	0\\
20.55	0\\
20.56	0\\
20.57	0\\
20.58	0\\
20.59	0\\
20.6	1.73472347597681e-18\\
20.61	0\\
20.62	0\\
20.63	0\\
20.64	1.73472347597681e-18\\
20.65	1.73472347597681e-18\\
20.66	1.73472347597681e-18\\
20.67	1.73472347597681e-18\\
20.68	0\\
20.69	0\\
20.7	0\\
20.71	0\\
20.72	0\\
20.73	0\\
20.74	1.73472347597681e-18\\
20.75	0\\
20.76	0\\
20.77	1.73472347597681e-18\\
20.78	0\\
20.79	0\\
20.8	0\\
20.81	0\\
20.82	0\\
20.83	0\\
20.84	1.73472347597681e-18\\
20.85	0\\
20.86	0\\
20.87	0\\
20.88	0\\
20.89	0\\
20.9	0\\
20.91	0\\
20.92	0\\
20.93	0\\
20.94	0\\
20.95	0\\
20.96	0\\
20.97	0\\
20.98	0\\
20.99	0\\
21	1.73472347597681e-18\\
21.01	1.73472347597681e-18\\
21.02	0\\
21.03	0\\
21.04	1.73472347597681e-18\\
21.05	0\\
21.06	0\\
21.07	0\\
21.08	0\\
21.09	0\\
21.1	0\\
21.11	0\\
21.12	0\\
21.13	0\\
21.14	0\\
21.15	0\\
21.16	1.73472347597681e-18\\
21.17	0\\
21.18	0\\
21.19	0\\
21.2	0\\
21.21	1.73472347597681e-18\\
21.22	0\\
21.23	0\\
21.24	0\\
21.25	1.73472347597681e-18\\
21.26	1.73472347597681e-18\\
21.27	1.73472347597681e-18\\
21.28	0\\
21.29	0\\
21.3	0\\
21.31	0\\
21.32	1.73472347597681e-18\\
21.33	0\\
21.34	0\\
21.35	0\\
21.36	0\\
21.37	0\\
21.38	0\\
21.39	1.73472347597681e-18\\
21.4	0\\
21.41	0\\
21.42	0\\
21.43	0\\
21.44	0\\
21.45	0\\
21.46	1.73472347597681e-18\\
21.47	0\\
21.48	0\\
21.49	0\\
21.5	0\\
21.51	0\\
21.52	0\\
21.53	0\\
21.54	0\\
21.55	0\\
21.56	0\\
21.57	1.73472347597681e-18\\
21.58	0\\
21.59	0\\
21.6	0\\
21.61	0\\
21.62	0\\
21.63	0\\
21.64	0\\
21.65	0\\
21.66	0\\
21.67	1.73472347597681e-18\\
21.68	0\\
21.69	0\\
21.7	1.73472347597681e-18\\
21.71	0\\
21.72	0\\
21.73	1.73472347597681e-18\\
21.74	0\\
21.75	0\\
21.76	0\\
21.77	0\\
21.78	1.73472347597681e-18\\
21.79	0\\
21.8	0\\
21.81	0\\
21.82	0\\
21.83	0\\
21.84	0\\
21.85	0\\
21.86	0\\
21.87	0\\
21.88	1.73472347597681e-18\\
21.89	0\\
21.9	0\\
21.91	0\\
21.92	1.73472347597681e-18\\
21.93	0\\
21.94	0\\
21.95	0\\
21.96	0\\
21.97	0\\
21.98	0\\
21.99	0\\
22	0\\
22.01	0\\
22.02	0\\
22.03	0\\
22.04	0\\
22.05	0\\
22.06	0\\
22.07	0\\
22.08	0\\
22.09	0\\
22.1	1.73472347597681e-18\\
22.11	0\\
22.12	1.73472347597681e-18\\
22.13	0\\
22.14	1.73472347597681e-18\\
22.15	1.73472347597681e-18\\
22.16	0\\
22.17	0\\
22.18	0\\
22.19	0\\
22.2	0\\
22.21	0\\
22.22	0\\
22.23	0\\
22.24	0\\
22.25	1.73472347597681e-18\\
22.26	1.73472347597681e-18\\
22.27	0\\
22.28	1.73472347597681e-18\\
22.29	0\\
22.3	0\\
22.31	0\\
22.32	0\\
22.33	0\\
22.34	0\\
22.35	0\\
22.36	0\\
22.37	0\\
22.38	0\\
22.39	0\\
22.4	0\\
22.41	0\\
22.42	1.73472347597681e-18\\
22.43	0\\
22.44	0\\
22.45	0\\
22.46	1.73472347597681e-18\\
22.47	1.73472347597681e-18\\
22.48	1.73472347597681e-18\\
22.49	0\\
22.5	0\\
22.51	0\\
22.52	0\\
22.53	0\\
22.54	0\\
22.55	0\\
22.56	0\\
22.57	0\\
22.58	0\\
22.59	0\\
22.6	0\\
22.61	0\\
22.62	0\\
22.63	0\\
22.64	0\\
22.65	1.73472347597681e-18\\
22.66	0\\
22.67	0\\
22.68	1.73472347597681e-18\\
22.69	1.73472347597681e-18\\
22.7	0\\
22.71	0\\
22.72	1.73472347597681e-18\\
22.73	0\\
22.74	0\\
22.75	0\\
22.76	0\\
22.77	0\\
22.78	0\\
22.79	0\\
22.8	0\\
22.81	0\\
22.82	0\\
22.83	0\\
22.84	0\\
22.85	0\\
22.86	0\\
22.87	0\\
22.88	0\\
22.89	0\\
22.9	0\\
22.91	0\\
22.92	0\\
22.93	0\\
22.94	0\\
22.95	0\\
22.96	0\\
22.97	0\\
22.98	0\\
22.99	0\\
23	0\\
23.01	0\\
23.02	0\\
23.03	0\\
23.04	0\\
23.05	0\\
23.06	0\\
23.07	1.73472347597681e-18\\
23.08	0\\
23.09	0\\
23.1	0\\
23.11	0\\
23.12	0\\
23.13	1.73472347597681e-18\\
23.14	0\\
23.15	0\\
23.16	0\\
23.17	0\\
23.18	0\\
23.19	0\\
23.2	0\\
23.21	1.73472347597681e-18\\
23.22	0\\
23.23	0\\
23.24	0\\
23.25	0\\
23.26	0\\
23.27	0\\
23.28	0\\
23.29	0\\
23.3	0\\
23.31	0\\
23.32	0\\
23.33	0\\
23.34	0\\
23.35	0\\
23.36	0\\
23.37	0\\
23.38	1.73472347597681e-18\\
23.39	0\\
23.4	0\\
23.41	0\\
23.42	0\\
23.43	0\\
23.44	0\\
23.45	0\\
23.46	0\\
23.47	0\\
23.48	0\\
23.49	0\\
23.5	0\\
23.51	0\\
23.52	0\\
23.53	0\\
23.54	0\\
23.55	1.73472347597681e-18\\
23.56	0\\
23.57	0\\
23.58	0\\
23.59	0\\
23.6	0\\
23.61	0\\
23.62	0\\
23.63	0\\
23.64	0\\
23.65	0\\
23.66	0\\
23.67	0\\
23.68	0\\
23.69	0\\
23.7	0\\
23.71	0\\
23.72	0\\
23.73	0\\
23.74	0\\
23.75	0\\
23.76	0\\
23.77	0\\
23.78	0\\
23.79	0\\
23.8	1.73472347597681e-18\\
23.81	0\\
23.82	0\\
23.83	0\\
23.84	0\\
23.85	0\\
23.86	1.73472347597681e-18\\
23.87	0\\
23.88	0\\
23.89	0\\
23.9	0\\
23.91	0\\
23.92	0\\
23.93	0\\
23.94	0\\
23.95	0\\
23.96	0\\
23.97	0\\
23.98	0\\
23.99	0\\
24	0\\
24.01	1.73472347597681e-18\\
24.02	0\\
24.03	0\\
24.04	0\\
24.05	0\\
24.06	0\\
24.07	0\\
24.08	0\\
24.09	0\\
24.1	0\\
24.11	0\\
24.12	0\\
24.13	0\\
24.14	0\\
24.15	0\\
24.16	0\\
24.17	1.73472347597681e-18\\
24.18	1.73472347597681e-18\\
24.19	0\\
24.2	0\\
24.21	0\\
24.22	1.73472347597681e-18\\
24.23	0\\
24.24	0\\
24.25	0\\
24.26	0\\
24.27	0\\
24.28	0\\
24.29	0\\
24.3	0\\
24.31	0\\
24.32	0\\
24.33	1.73472347597681e-18\\
24.34	0\\
24.35	0\\
24.36	0\\
24.37	0\\
24.38	0\\
24.39	0\\
24.4	1.73472347597681e-18\\
24.41	1.73472347597681e-18\\
24.42	0\\
24.43	0\\
24.44	0\\
24.45	0\\
24.46	0\\
24.47	1.73472347597681e-18\\
24.48	1.73472347597681e-18\\
24.49	0\\
24.5	0\\
24.51	0\\
24.52	0\\
24.53	0\\
24.54	0\\
24.55	0\\
24.56	0\\
24.57	0\\
24.58	0\\
24.59	0\\
24.6	0\\
24.61	0\\
24.62	0\\
24.63	0\\
24.64	0\\
24.65	1.73472347597681e-18\\
24.66	0\\
24.67	1.73472347597681e-18\\
24.68	0\\
24.69	0\\
24.7	0\\
24.71	0\\
24.72	0\\
24.73	0\\
24.74	1.73472347597681e-18\\
24.75	0\\
24.76	0\\
24.77	0\\
24.78	0\\
24.79	0\\
24.8	0\\
24.81	0\\
24.82	0\\
24.83	0\\
24.84	0\\
24.85	0\\
24.86	0\\
24.87	0\\
24.88	0\\
24.89	0\\
24.9	0\\
24.91	0\\
24.92	1.73472347597681e-18\\
24.93	0\\
24.94	0\\
24.95	0\\
24.96	0\\
24.97	0\\
24.98	0\\
24.99	0\\
25	0\\
25.01	1.73472347597681e-18\\
25.02	1.73472347597681e-18\\
25.03	0\\
25.04	1.73472347597681e-18\\
25.05	0\\
25.06	0\\
25.07	0\\
25.08	0\\
25.09	0\\
25.1	1.73472347597681e-18\\
25.11	1.73472347597681e-18\\
25.12	0\\
25.13	0\\
25.14	0\\
25.15	0\\
25.16	1.73472347597681e-18\\
25.17	0\\
25.18	1.73472347597681e-18\\
25.19	0\\
25.2	0\\
25.21	0\\
25.22	0\\
25.23	0\\
25.24	1.73472347597681e-18\\
25.25	1.73472347597681e-18\\
25.26	0\\
25.27	0\\
25.28	0\\
25.29	0\\
25.3	0\\
25.31	0\\
25.32	0\\
25.33	0\\
25.34	0\\
25.35	0\\
25.36	0\\
25.37	1.73472347597681e-18\\
25.38	0\\
25.39	0\\
25.4	0\\
25.41	1.73472347597681e-18\\
25.42	0\\
25.43	0\\
25.44	0\\
25.45	0\\
25.46	0\\
25.47	0\\
25.48	0\\
25.49	0\\
25.5	0\\
25.51	0\\
25.52	0\\
25.53	0\\
25.54	0\\
25.55	0\\
25.56	0\\
25.57	0\\
25.58	0\\
25.59	0\\
25.6	0\\
25.61	0\\
25.62	0\\
25.63	0\\
25.64	0\\
25.65	0\\
25.66	0\\
25.67	0\\
25.68	0\\
25.69	0\\
25.7	0\\
25.71	0\\
25.72	1.73472347597681e-18\\
25.73	0\\
25.74	1.73472347597681e-18\\
25.75	0\\
25.76	1.73472347597681e-18\\
25.77	0\\
25.78	1.73472347597681e-18\\
25.79	0\\
25.8	0\\
25.81	0\\
25.82	0\\
25.83	0\\
25.84	0\\
25.85	0\\
25.86	0\\
25.87	1.73472347597681e-18\\
25.88	0\\
25.89	0\\
25.9	0\\
25.91	0\\
25.92	0\\
25.93	0\\
25.94	0\\
25.95	0\\
25.96	1.73472347597681e-18\\
25.97	0\\
25.98	0\\
25.99	0\\
26	0\\
26.01	0\\
26.02	0\\
26.03	0\\
26.04	0\\
26.05	0\\
26.06	1.73472347597681e-18\\
26.07	0\\
26.08	0\\
26.09	1.73472347597681e-18\\
26.1	0\\
26.11	0\\
26.12	0\\
26.13	0\\
26.14	0\\
26.15	0\\
26.16	0\\
26.17	0\\
26.18	0\\
26.19	0\\
26.2	0\\
26.21	0\\
26.22	0\\
26.23	0\\
26.24	0\\
26.25	0\\
26.26	0\\
26.27	0\\
26.28	0\\
26.29	0\\
26.3	0\\
26.31	0\\
26.32	0\\
26.33	0\\
26.34	0\\
26.35	0\\
26.36	0\\
26.37	0\\
26.38	0\\
26.39	1.73472347597681e-18\\
26.4	0\\
26.41	0\\
26.42	0\\
26.43	0\\
26.44	0\\
26.45	0\\
26.46	0\\
26.47	0\\
26.48	0\\
26.49	0\\
26.5	0\\
26.51	0\\
26.52	0\\
26.53	0\\
26.54	0\\
26.55	0\\
26.56	0\\
26.57	0\\
26.58	0\\
26.59	0\\
26.6	0\\
26.61	0\\
26.62	0\\
26.63	0\\
26.64	0\\
26.65	0\\
26.66	0\\
26.67	0\\
26.68	0\\
26.69	0\\
26.7	0\\
26.71	1.73472347597681e-18\\
26.72	0\\
26.73	0\\
26.74	0\\
26.75	0\\
26.76	0\\
26.77	0\\
26.78	0\\
26.79	0\\
26.8	1.73472347597681e-18\\
26.81	0\\
26.82	0\\
26.83	0\\
26.84	1.73472347597681e-18\\
26.85	0\\
26.86	1.73472347597681e-18\\
26.87	0\\
26.88	0\\
26.89	0\\
26.9	0\\
26.91	1.73472347597681e-18\\
26.92	0\\
26.93	0\\
26.94	0\\
26.95	0\\
26.96	0\\
26.97	0\\
26.98	1.73472347597681e-18\\
26.99	0\\
27	0\\
27.01	0\\
27.02	0\\
27.03	0\\
27.04	0\\
27.05	0\\
27.06	0\\
27.07	0\\
27.08	0\\
27.09	0\\
27.1	0\\
27.11	1.73472347597681e-18\\
27.12	0\\
27.13	0\\
27.14	0\\
27.15	0\\
27.16	0\\
27.17	0\\
27.18	0\\
27.19	0\\
27.2	0\\
27.21	0\\
27.22	0\\
27.23	0\\
27.24	0\\
27.25	0\\
27.26	0\\
27.27	0\\
27.28	0\\
27.29	0\\
27.3	0\\
27.31	1.73472347597681e-18\\
27.32	0\\
27.33	0\\
27.34	0\\
27.35	0\\
27.36	0\\
27.37	0\\
27.38	0\\
27.39	0\\
27.4	0\\
27.41	0\\
27.42	0\\
27.43	1.73472347597681e-18\\
27.44	0\\
27.45	0\\
27.46	0\\
27.47	0\\
27.48	0\\
27.49	1.73472347597681e-18\\
27.5	1.73472347597681e-18\\
27.51	0\\
27.52	0\\
27.53	0\\
27.54	0\\
27.55	0\\
27.56	0\\
27.57	0\\
27.58	0\\
27.59	0\\
27.6	0\\
27.61	0\\
27.62	0\\
27.63	1.73472347597681e-18\\
27.64	0\\
27.65	0\\
27.66	0\\
27.67	0\\
27.68	0\\
27.69	0\\
27.7	0\\
27.71	0\\
27.72	0\\
27.73	0\\
27.74	0\\
27.75	1.73472347597681e-18\\
27.76	0\\
27.77	0\\
27.78	0\\
27.79	0\\
27.8	0\\
27.81	0\\
27.82	0\\
27.83	0\\
27.84	0\\
27.85	1.73472347597681e-18\\
27.86	0\\
27.87	1.73472347597681e-18\\
27.88	0\\
27.89	0\\
27.9	0\\
27.91	0\\
27.92	0\\
27.93	0\\
27.94	0\\
27.95	0\\
27.96	0\\
27.97	0\\
27.98	0\\
27.99	0\\
28	1.73472347597681e-18\\
28.01	0\\
28.02	0\\
28.03	1.73472347597681e-18\\
28.04	1.73472347597681e-18\\
28.05	0\\
28.06	0\\
28.07	0\\
28.08	0\\
28.09	0\\
28.1	0\\
28.11	1.73472347597681e-18\\
28.12	0\\
28.13	0\\
28.14	0\\
28.15	0\\
28.16	0\\
28.17	0\\
28.18	0\\
28.19	0\\
28.2	0\\
28.21	0\\
28.22	0\\
28.23	0\\
28.24	0\\
28.25	0\\
28.26	0\\
28.27	0\\
28.28	0\\
28.29	0\\
28.3	0\\
28.31	0\\
28.32	0\\
28.33	0\\
28.34	0\\
28.35	1.73472347597681e-18\\
28.36	0\\
28.37	0\\
28.38	0\\
28.39	0\\
28.4	0\\
28.41	0\\
28.42	0\\
28.43	0\\
28.44	0\\
28.45	0\\
28.46	0\\
28.47	0\\
28.48	0\\
28.49	0\\
28.5	0\\
28.51	0\\
28.52	0\\
28.53	0\\
28.54	1.73472347597681e-18\\
28.55	0\\
28.56	0\\
28.57	0\\
28.58	1.73472347597681e-18\\
28.59	1.73472347597681e-18\\
28.6	0\\
28.61	0\\
28.62	0\\
28.63	0\\
28.64	0\\
28.65	0\\
28.66	0\\
28.67	0\\
28.68	0\\
28.69	1.73472347597681e-18\\
28.7	0\\
28.71	0\\
28.72	0\\
28.73	0\\
28.74	0\\
28.75	0\\
28.76	0\\
28.77	0\\
28.78	0\\
28.79	0\\
28.8	0\\
28.81	1.73472347597681e-18\\
28.82	0\\
28.83	0\\
28.84	0\\
28.85	0\\
28.86	0\\
28.87	0\\
28.88	0\\
28.89	0\\
28.9	0\\
28.91	1.73472347597681e-18\\
28.92	1.73472347597681e-18\\
28.93	0\\
28.94	0\\
28.95	0\\
28.96	0\\
28.97	0\\
28.98	0\\
28.99	0\\
29	0\\
29.01	0\\
29.02	0\\
29.03	0\\
29.04	0\\
29.05	0\\
29.06	0\\
29.07	0\\
29.08	0\\
29.09	0\\
29.1	0\\
29.11	0\\
29.12	0\\
29.13	0\\
29.14	0\\
29.15	0\\
29.16	0\\
29.17	0\\
29.18	1.73472347597681e-18\\
29.19	0\\
29.2	0\\
29.21	0\\
29.22	0\\
29.23	0\\
29.24	0\\
29.25	0\\
29.26	0\\
29.27	0\\
29.28	0\\
29.29	0\\
29.3	0\\
29.31	1.73472347597681e-18\\
29.32	0\\
29.33	0\\
29.34	0\\
29.35	0\\
29.36	0\\
29.37	0\\
29.38	0\\
29.39	0\\
29.4	0\\
29.41	0\\
29.42	0\\
29.43	0\\
29.44	0\\
29.45	0\\
29.46	0\\
29.47	0\\
29.48	0\\
29.49	0\\
29.5	0\\
29.51	0\\
29.52	0\\
29.53	1.73472347597681e-18\\
29.54	0\\
29.55	0\\
29.56	0\\
29.57	0\\
29.58	0\\
29.59	0\\
29.6	0\\
29.61	0\\
29.62	0\\
29.63	1.73472347597681e-18\\
29.64	0\\
29.65	1.73472347597681e-18\\
29.66	1.73472347597681e-18\\
29.67	0\\
29.68	1.73472347597681e-18\\
29.69	0\\
29.7	0\\
29.71	0\\
29.72	1.73472347597681e-18\\
29.73	0\\
29.74	0\\
29.75	0\\
29.76	0\\
29.77	0\\
29.78	0\\
29.79	0\\
29.8	1.73472347597681e-18\\
29.81	0\\
29.82	0\\
29.83	0\\
29.84	0\\
29.85	0\\
29.86	0\\
29.87	1.73472347597681e-18\\
29.88	0\\
29.89	0\\
29.9	0\\
29.91	0\\
29.92	0\\
29.93	1.73472347597681e-18\\
29.94	0\\
29.95	0\\
29.96	0\\
29.97	0\\
29.98	0\\
29.99	0\\
30	1.73472347597681e-18\\
30.01	1.73472347597681e-18\\
30.02	0\\
30.03	0\\
30.04	0\\
30.05	0\\
30.06	0\\
30.07	0\\
30.08	0\\
30.09	0\\
30.1	0\\
30.11	1.73472347597681e-18\\
30.12	0\\
30.13	0\\
30.14	0\\
30.15	0\\
30.16	0\\
30.17	0\\
30.18	0\\
30.19	0\\
30.2	0\\
30.21	0\\
30.22	0\\
30.23	0\\
30.24	0\\
30.25	0\\
30.26	0\\
30.27	0\\
30.28	0\\
30.29	0\\
30.3	1.73472347597681e-18\\
30.31	0\\
30.32	0\\
30.33	0\\
30.34	0\\
30.35	0\\
30.36	0\\
30.37	0\\
30.38	1.73472347597681e-18\\
30.39	0\\
30.4	1.73472347597681e-18\\
30.41	0\\
30.42	0\\
30.43	0\\
30.44	0\\
30.45	0\\
30.46	0\\
30.47	0\\
30.48	0\\
30.49	0\\
30.5	0\\
30.51	0\\
30.52	0\\
30.53	0\\
30.54	0\\
30.55	0\\
30.56	0\\
30.57	0\\
30.58	0\\
30.59	0\\
30.6	0\\
30.61	0\\
30.62	0\\
30.63	0\\
30.64	0\\
30.65	0\\
30.66	0\\
30.67	0\\
30.68	1.73472347597681e-18\\
30.69	0\\
30.7	0\\
30.71	0\\
30.72	1.73472347597681e-18\\
30.73	0\\
30.74	0\\
30.75	0\\
30.76	0\\
30.77	0\\
30.78	0\\
30.79	0\\
30.8	1.73472347597681e-18\\
30.81	1.73472347597681e-18\\
30.82	0\\
30.83	1.73472347597681e-18\\
30.84	0\\
30.85	1.73472347597681e-18\\
30.86	0\\
30.87	0\\
30.88	0\\
30.89	0\\
30.9	0\\
30.91	1.73472347597681e-18\\
30.92	0\\
30.93	0\\
30.94	1.73472347597681e-18\\
30.95	0\\
30.96	0\\
30.97	1.73472347597681e-18\\
30.98	0\\
30.99	1.73472347597681e-18\\
31	0\\
31.01	0\\
31.02	0\\
31.03	0\\
31.04	0\\
31.05	0\\
31.06	0\\
31.07	0\\
31.08	1.73472347597681e-18\\
31.09	0\\
31.1	0\\
31.11	0\\
31.12	0\\
31.13	0\\
31.14	0\\
31.15	0\\
31.16	0\\
31.17	0\\
31.18	0\\
31.19	0\\
31.2	0\\
31.21	1.73472347597681e-18\\
31.22	1.73472347597681e-18\\
31.23	0\\
31.24	0\\
31.25	0\\
31.26	1.73472347597681e-18\\
31.27	0\\
31.28	0\\
31.29	0\\
31.3	0\\
31.31	0\\
31.32	0\\
31.33	0\\
31.34	0\\
31.35	0\\
31.36	0\\
31.37	0\\
31.38	0\\
31.39	1.73472347597681e-18\\
31.4	0\\
31.41	0\\
31.42	1.73472347597681e-18\\
31.43	1.73472347597681e-18\\
31.44	1.73472347597681e-18\\
31.45	0\\
31.46	0\\
31.47	0\\
31.48	0\\
31.49	0\\
31.5	0\\
31.51	0\\
31.52	0\\
31.53	0\\
31.54	0\\
31.55	1.73472347597681e-18\\
31.56	0\\
31.57	0\\
31.58	0\\
31.59	0\\
31.6	0\\
31.61	0\\
31.62	0\\
31.63	0\\
31.64	0\\
31.65	0\\
31.66	0\\
31.67	1.73472347597681e-18\\
31.68	0\\
31.69	0\\
31.7	0\\
31.71	0\\
31.72	0\\
31.73	0\\
31.74	0\\
31.75	0\\
31.76	0\\
31.77	0\\
31.78	0\\
31.79	0\\
31.8	0\\
31.81	0\\
31.82	0\\
31.83	1.73472347597681e-18\\
31.84	0\\
31.85	0\\
31.86	1.73472347597681e-18\\
31.87	0\\
31.88	1.73472347597681e-18\\
31.89	1.73472347597681e-18\\
31.9	0\\
31.91	0\\
31.92	0\\
31.93	0\\
31.94	1.73472347597681e-18\\
31.95	1.73472347597681e-18\\
31.96	0\\
31.97	1.73472347597681e-18\\
31.98	0\\
31.99	0\\
32	0\\
32.01	0\\
32.02	0\\
32.03	0\\
32.04	0\\
32.05	0\\
32.06	0\\
32.07	0\\
32.08	0\\
32.09	0\\
32.1	0\\
32.11	0\\
32.12	0\\
32.13	0\\
32.14	0\\
32.15	0\\
32.16	0\\
32.17	0\\
32.18	0\\
32.19	0\\
32.2	0\\
32.21	0\\
32.22	0\\
32.23	0\\
32.24	0\\
32.25	0\\
32.26	0\\
32.27	0\\
32.28	0\\
32.29	0\\
32.3	0\\
32.31	0\\
32.32	0\\
32.33	0\\
32.34	0\\
32.35	0\\
32.36	0\\
32.37	0\\
32.38	0\\
32.39	0\\
32.4	0\\
32.41	0\\
32.42	0\\
32.43	0\\
32.44	1.73472347597681e-18\\
32.45	1.73472347597681e-18\\
32.46	0\\
32.47	0\\
32.48	0\\
32.49	0\\
32.5	0\\
32.51	0\\
32.52	0\\
32.53	0\\
32.54	0\\
32.55	0\\
32.56	0\\
32.57	0\\
32.58	0\\
32.59	0\\
32.6	0\\
32.61	1.73472347597681e-18\\
32.62	0\\
32.63	0\\
32.64	0\\
32.65	0\\
32.66	0\\
32.67	0\\
32.68	0\\
32.69	0\\
32.7	0\\
32.71	0\\
32.72	0\\
32.73	0\\
32.74	0\\
32.75	0\\
32.76	0\\
32.77	0\\
32.78	0\\
32.79	0\\
32.8	0\\
32.81	0\\
32.82	0\\
32.83	1.73472347597681e-18\\
32.84	1.73472347597681e-18\\
32.85	0\\
32.86	0\\
32.87	0\\
32.88	0\\
32.89	1.73472347597681e-18\\
32.9	0\\
32.91	0\\
32.92	0\\
32.93	0\\
32.94	0\\
32.95	0\\
32.96	0\\
32.97	1.73472347597681e-18\\
32.98	0\\
32.99	0\\
33	0\\
33.01	0\\
33.02	0\\
33.03	0\\
33.04	0\\
33.05	0\\
33.06	0\\
33.07	0\\
33.08	0\\
33.09	0\\
33.1	0\\
33.11	1.73472347597681e-18\\
33.12	0\\
33.13	0\\
33.14	0\\
33.15	0\\
33.16	1.73472347597681e-18\\
33.17	0\\
33.18	0\\
33.19	0\\
33.2	0\\
33.21	0\\
33.22	0\\
33.23	0\\
33.24	0\\
33.25	1.73472347597681e-18\\
33.26	0\\
33.27	1.73472347597681e-18\\
33.28	1.73472347597681e-18\\
33.29	0\\
33.3	1.73472347597681e-18\\
33.31	0\\
33.32	0\\
33.33	0\\
33.34	0\\
33.35	0\\
33.36	1.73472347597681e-18\\
33.37	0\\
33.38	1.73472347597681e-18\\
33.39	0\\
33.4	0\\
33.41	0\\
33.42	0\\
33.43	0\\
33.44	0\\
33.45	0\\
33.46	0\\
33.47	0\\
33.48	0\\
33.49	0\\
33.5	0\\
33.51	0\\
33.52	0\\
33.53	0\\
33.54	0\\
33.55	0\\
33.56	0\\
33.57	0\\
33.58	1.73472347597681e-18\\
33.59	0\\
33.6	1.73472347597681e-18\\
33.61	0\\
33.62	0\\
33.63	0\\
33.64	0\\
33.65	0\\
33.66	0\\
33.67	0\\
33.68	0\\
33.69	0\\
33.7	0\\
33.71	0\\
33.72	0\\
33.73	0\\
33.74	0\\
33.75	0\\
33.76	1.73472347597681e-18\\
33.77	0\\
33.78	0\\
33.79	0\\
33.8	0\\
33.81	0\\
33.82	0\\
33.83	0\\
33.84	0\\
33.85	0\\
33.86	0\\
33.87	0\\
33.88	1.73472347597681e-18\\
33.89	0\\
33.9	0\\
33.91	0\\
33.92	0\\
33.93	0\\
33.94	1.73472347597681e-18\\
33.95	0\\
33.96	0\\
33.97	0\\
33.98	0\\
33.99	1.73472347597681e-18\\
34	0\\
34.01	0\\
34.02	0\\
34.03	0\\
34.04	0\\
34.05	0\\
34.06	0\\
34.07	0\\
34.08	0\\
34.09	0\\
34.1	0\\
34.11	0\\
34.12	0\\
34.13	0\\
34.14	0\\
34.15	0\\
34.16	1.73472347597681e-18\\
34.17	0\\
34.18	0\\
34.19	0\\
34.2	0\\
34.21	0\\
34.22	0\\
34.23	0\\
34.24	0\\
34.25	0\\
34.26	1.73472347597681e-18\\
34.27	0\\
34.28	0\\
34.29	0\\
34.3	0\\
34.31	0\\
34.32	0\\
34.33	0\\
34.34	1.73472347597681e-18\\
34.35	0\\
34.36	0\\
34.37	0\\
34.38	0\\
34.39	0\\
34.4	0\\
34.41	0\\
34.42	1.73472347597681e-18\\
34.43	0\\
34.44	0\\
34.45	0\\
34.46	0\\
34.47	0\\
34.48	0\\
34.49	1.73472347597681e-18\\
34.5	0\\
34.51	0\\
34.52	0\\
34.53	0\\
34.54	0\\
34.55	0\\
34.56	0\\
34.57	0\\
34.58	1.73472347597681e-18\\
34.59	0\\
34.6	0\\
34.61	0\\
34.62	0\\
34.63	0\\
34.64	0\\
34.65	0\\
34.66	0\\
34.67	0\\
34.68	0\\
34.69	0\\
34.7	1.73472347597681e-18\\
34.71	0\\
34.72	0\\
34.73	0\\
34.74	1.73472347597681e-18\\
34.75	1.73472347597681e-18\\
34.76	0\\
34.77	1.73472347597681e-18\\
34.78	1.73472347597681e-18\\
34.79	0\\
34.8	0\\
34.81	0\\
34.82	0\\
34.83	0\\
34.84	0\\
34.85	0\\
34.86	0\\
34.87	0\\
34.88	0\\
34.89	0\\
34.9	0\\
34.91	0\\
34.92	0\\
34.93	0\\
34.94	0\\
34.95	1.73472347597681e-18\\
34.96	1.73472347597681e-18\\
34.97	1.73472347597681e-18\\
34.98	0\\
34.99	0\\
35	0\\
35.01	0\\
35.02	0\\
35.03	0\\
35.04	0\\
35.05	0\\
35.06	0\\
35.07	1.73472347597681e-18\\
35.08	0\\
35.09	1.73472347597681e-18\\
35.1	0\\
35.11	0\\
35.12	0\\
35.13	1.73472347597681e-18\\
35.14	0\\
35.15	0\\
35.16	1.73472347597681e-18\\
35.17	0\\
35.18	0\\
35.19	0\\
35.2	0\\
35.21	0\\
35.22	0\\
35.23	0\\
35.24	0\\
35.25	0\\
35.26	0\\
35.27	0\\
35.28	0\\
35.29	0\\
35.3	0\\
35.31	0\\
35.32	0\\
35.33	0\\
35.34	0\\
35.35	0\\
35.36	0\\
35.37	0\\
35.38	0\\
35.39	0\\
35.4	0\\
35.41	0\\
35.42	0\\
35.43	1.73472347597681e-18\\
35.44	0\\
35.45	0\\
35.46	0\\
35.47	0\\
35.48	0\\
35.49	0\\
35.5	0\\
35.51	0\\
35.52	0\\
35.53	0\\
35.54	0\\
35.55	0\\
35.56	0\\
35.57	0\\
35.58	0\\
35.59	0\\
35.6	0\\
35.61	0\\
35.62	0\\
35.63	0\\
35.64	0\\
35.65	0\\
35.66	0\\
35.67	0\\
35.68	1.73472347597681e-18\\
35.69	1.73472347597681e-18\\
35.7	0\\
35.71	0\\
35.72	0\\
35.73	0\\
35.74	0\\
35.75	0\\
35.76	0\\
35.77	0\\
35.78	0\\
35.79	0\\
35.8	0\\
35.81	0\\
35.82	0\\
35.83	0\\
35.84	0\\
35.85	1.73472347597681e-18\\
35.86	0\\
35.87	0\\
35.88	0\\
35.89	0\\
35.9	0\\
35.91	1.73472347597681e-18\\
35.92	0\\
35.93	0\\
35.94	0\\
35.95	0\\
35.96	0\\
35.97	0\\
35.98	1.73472347597681e-18\\
35.99	0\\
36	0\\
36.01	0\\
36.02	0\\
36.03	0\\
36.04	0\\
36.05	0\\
36.06	0\\
36.07	1.73472347597681e-18\\
36.08	1.73472347597681e-18\\
36.09	0\\
36.1	0\\
36.11	0\\
36.12	0\\
36.13	0\\
36.14	0\\
36.15	0\\
36.16	1.73472347597681e-18\\
36.17	0\\
36.18	0\\
36.19	0\\
36.2	0\\
36.21	0\\
36.22	0\\
36.23	0\\
36.24	0\\
36.25	1.73472347597681e-18\\
36.26	0\\
36.27	0\\
36.28	0\\
36.29	0\\
36.3	0\\
36.31	0\\
36.32	1.73472347597681e-18\\
36.33	0\\
36.34	0\\
36.35	0\\
36.36	0\\
36.37	0\\
36.38	1.73472347597681e-18\\
36.39	0\\
36.4	0\\
36.41	0\\
36.42	0\\
36.43	0\\
36.44	1.73472347597681e-18\\
36.45	0\\
36.46	0\\
36.47	0\\
36.48	0\\
36.49	0\\
36.5	1.73472347597681e-18\\
36.51	0\\
36.52	0\\
36.53	0\\
36.54	0\\
36.55	0\\
36.56	0\\
36.57	0\\
36.58	1.73472347597681e-18\\
36.59	0\\
36.6	1.73472347597681e-18\\
36.61	0\\
36.62	0\\
36.63	1.73472347597681e-18\\
36.64	0\\
36.65	0\\
36.66	0\\
36.67	0\\
36.68	0\\
36.69	0\\
36.7	0\\
36.71	0\\
36.72	0\\
36.73	0\\
36.74	0\\
36.75	0\\
36.76	0\\
36.77	0\\
36.78	0\\
36.79	0\\
36.8	0\\
36.81	1.73472347597681e-18\\
36.82	0\\
36.83	0\\
36.84	0\\
36.85	0\\
36.86	0\\
36.87	0\\
36.88	0\\
36.89	0\\
36.9	0\\
36.91	0\\
36.92	0\\
36.93	0\\
36.94	0\\
36.95	0\\
36.96	0\\
36.97	0\\
36.98	1.73472347597681e-18\\
36.99	0\\
37	0\\
37.01	0\\
37.02	1.73472347597681e-18\\
37.03	0\\
37.04	0\\
37.05	0\\
37.06	0\\
37.07	0\\
37.08	0\\
37.09	0\\
37.1	0\\
37.11	0\\
37.12	0\\
37.13	0\\
37.14	0\\
37.15	0\\
37.16	0\\
37.17	0\\
37.18	0\\
37.19	0\\
37.2	0\\
37.21	0\\
37.22	0\\
37.23	0\\
37.24	0\\
37.25	0\\
37.26	0\\
37.27	0\\
37.28	0\\
37.29	0\\
37.3	1.73472347597681e-18\\
37.31	0\\
37.32	0\\
37.33	0\\
37.34	0\\
37.35	0\\
37.36	0\\
37.37	0\\
37.38	0\\
37.39	0\\
37.4	0\\
37.41	0\\
37.42	0\\
37.43	0\\
37.44	0\\
37.45	0\\
37.46	0\\
37.47	0\\
37.48	0\\
37.49	1.73472347597681e-18\\
37.5	0\\
37.51	0\\
37.52	0\\
37.53	0\\
37.54	0\\
37.55	0\\
37.56	0\\
37.57	0\\
37.58	0\\
37.59	0\\
37.6	0\\
37.61	0\\
37.62	0\\
37.63	0\\
37.64	0\\
37.65	0\\
37.66	0\\
37.67	0\\
37.68	0\\
37.69	0\\
37.7	0\\
37.71	0\\
37.72	0\\
37.73	0\\
37.74	0\\
37.75	0\\
37.76	1.73472347597681e-18\\
37.77	0\\
37.78	0\\
37.79	0\\
37.8	0\\
37.81	0\\
37.82	0\\
37.83	0\\
37.84	0\\
37.85	0\\
37.86	0\\
37.87	0\\
37.88	0\\
37.89	0\\
37.9	0\\
37.91	0\\
37.92	0\\
37.93	0\\
37.94	0\\
37.95	0\\
37.96	0\\
37.97	0\\
37.98	0\\
37.99	0\\
38	1.73472347597681e-18\\
38.01	0\\
38.02	0\\
38.03	0\\
38.04	0\\
38.05	0\\
38.06	0\\
38.07	0\\
38.08	0\\
38.09	0\\
38.1	0\\
38.11	1.73472347597681e-18\\
38.12	1.73472347597681e-18\\
38.13	0\\
38.14	0\\
38.15	0\\
38.16	0\\
38.17	0\\
38.18	0\\
38.19	0\\
38.2	0\\
38.21	0\\
38.22	0\\
38.23	0\\
38.24	0\\
38.25	0\\
38.26	0\\
38.27	1.73472347597681e-18\\
38.28	0\\
38.29	0\\
38.3	0\\
38.31	1.73472347597681e-18\\
38.32	0\\
38.33	0\\
38.34	0\\
38.35	1.73472347597681e-18\\
38.36	0\\
38.37	0\\
38.38	0\\
38.39	0\\
38.4	0\\
38.41	0\\
38.42	0\\
38.43	0\\
38.44	0\\
38.45	0\\
38.46	0\\
38.47	0\\
38.48	1.73472347597681e-18\\
38.49	0\\
38.5	0\\
38.51	0\\
38.52	0\\
38.53	0\\
38.54	0\\
38.55	0\\
38.56	0\\
38.57	0\\
38.58	0\\
38.59	0\\
38.6	0\\
38.61	1.73472347597681e-18\\
38.62	0\\
38.63	1.73472347597681e-18\\
38.64	0\\
38.65	1.73472347597681e-18\\
38.66	0\\
38.67	0\\
38.68	0\\
38.69	0\\
38.7	0\\
38.71	0\\
38.72	0\\
38.73	0\\
38.74	1.73472347597681e-18\\
38.75	0\\
38.76	0\\
38.77	0\\
38.78	0\\
38.79	0\\
38.8	0\\
38.81	0\\
38.82	1.73472347597681e-18\\
38.83	0\\
38.84	0\\
38.85	0\\
38.86	0\\
38.87	1.73472347597681e-18\\
38.88	0\\
38.89	1.73472347597681e-18\\
38.9	0\\
38.91	0\\
38.92	0\\
38.93	0\\
38.94	0\\
38.95	0\\
38.96	0\\
38.97	0\\
38.98	0\\
38.99	0\\
39	0\\
39.01	0\\
39.02	0\\
39.03	0\\
39.04	0\\
39.05	0\\
39.06	0\\
39.07	0\\
39.08	0\\
39.09	1.73472347597681e-18\\
39.1	0\\
39.11	0\\
39.12	0\\
39.13	0\\
39.14	0\\
39.15	0\\
39.16	0\\
39.17	0\\
39.18	0\\
39.19	1.73472347597681e-18\\
39.2	0\\
39.21	0\\
39.22	0\\
39.23	0\\
39.24	0\\
39.25	0\\
39.26	0\\
39.27	0\\
39.28	0\\
39.29	0\\
39.3	0\\
39.31	0\\
39.32	0\\
39.33	0\\
39.34	1.73472347597681e-18\\
39.35	0\\
39.36	0\\
39.37	0\\
39.38	0\\
39.39	0\\
39.4	0\\
39.41	0\\
39.42	0\\
39.43	0\\
39.44	0\\
39.45	0\\
39.46	0\\
39.47	0\\
39.48	0\\
39.49	0\\
39.5	0\\
39.51	0\\
39.52	1.73472347597681e-18\\
39.53	0\\
39.54	0\\
39.55	0\\
39.56	0\\
39.57	0\\
39.58	0\\
39.59	0\\
39.6	0\\
39.61	0\\
39.62	0\\
39.63	0\\
39.64	0\\
39.65	0\\
39.66	0\\
39.67	1.73472347597681e-18\\
39.68	0\\
39.69	0\\
39.7	0\\
39.71	0\\
39.72	0\\
39.73	0\\
39.74	0\\
39.75	1.73472347597681e-18\\
39.76	0\\
39.77	0\\
39.78	0\\
39.79	0\\
39.8	0\\
39.81	0\\
39.82	1.73472347597681e-18\\
39.83	0\\
39.84	0\\
39.85	0\\
39.86	0\\
39.87	0\\
39.88	0\\
39.89	0\\
39.9	0\\
39.91	0\\
39.92	1.73472347597681e-18\\
39.93	0\\
39.94	1.73472347597681e-18\\
39.95	0\\
39.96	0\\
39.97	0\\
39.98	0\\
39.99	0\\
40	0\\
40.01	1.73472347597681e-18\\
};
\addplot [color=mycolor1,dashed,forget plot]
  table[row sep=crcr]{%
40.01	1.73472347597681e-18\\
40.02	0\\
40.03	1.73472347597681e-18\\
40.04	0\\
40.05	1.73472347597681e-18\\
40.06	0\\
40.07	1.73472347597681e-18\\
40.08	0\\
40.09	1.73472347597681e-18\\
40.1	1.73472347597681e-18\\
40.11	0\\
40.12	0\\
40.13	0\\
40.14	0\\
40.15	1.73472347597681e-18\\
40.16	0\\
40.17	0\\
40.18	0\\
40.19	0\\
40.2	0\\
40.21	0\\
40.22	1.73472347597681e-18\\
40.23	0\\
40.24	0\\
40.25	1.73472347597681e-18\\
40.26	0\\
40.27	0\\
40.28	0\\
40.29	0\\
40.3	0\\
40.31	1.73472347597681e-18\\
40.32	1.73472347597681e-18\\
40.33	0\\
40.34	0\\
40.35	1.73472347597681e-18\\
40.36	0\\
40.37	0\\
40.38	0\\
40.39	0\\
40.4	1.73472347597681e-18\\
40.41	0\\
40.42	0\\
40.43	1.73472347597681e-18\\
40.44	0\\
40.45	0\\
40.46	0\\
40.47	0\\
40.48	0\\
40.49	0\\
40.5	1.73472347597681e-18\\
40.51	0\\
40.52	0\\
40.53	0\\
40.54	0\\
40.55	0\\
40.56	0\\
40.57	0\\
40.58	1.73472347597681e-18\\
40.59	0\\
40.6	0\\
40.61	1.73472347597681e-18\\
40.62	0\\
40.63	0\\
40.64	0\\
40.65	0\\
40.66	0\\
40.67	0\\
40.68	0\\
40.69	0\\
40.7	0\\
40.71	0\\
40.72	0\\
40.73	0\\
40.74	0\\
40.75	0\\
40.76	0\\
40.77	0\\
40.78	0\\
40.79	0\\
40.8	0\\
40.81	0\\
40.82	0\\
40.83	0\\
40.84	0\\
40.85	0\\
40.86	0\\
40.87	1.73472347597681e-18\\
40.88	0\\
40.89	0\\
40.9	1.73472347597681e-18\\
40.91	0\\
40.92	0\\
40.93	0\\
40.94	0\\
40.95	1.73472347597681e-18\\
40.96	0\\
40.97	1.73472347597681e-18\\
40.98	0\\
40.99	0\\
41	0\\
41.01	0\\
41.02	0\\
41.03	0\\
41.04	0\\
41.05	0\\
41.06	0\\
41.07	0\\
41.08	0\\
41.09	0\\
41.1	0\\
41.11	0\\
41.12	0\\
41.13	0\\
41.14	0\\
41.15	1.73472347597681e-18\\
41.16	0\\
41.17	0\\
41.18	1.73472347597681e-18\\
41.19	0\\
41.2	0\\
41.21	0\\
41.22	0\\
41.23	0\\
41.24	0\\
41.25	0\\
41.26	0\\
41.27	1.73472347597681e-18\\
41.28	0\\
41.29	1.73472347597681e-18\\
41.3	0\\
41.31	0\\
41.32	0\\
41.33	0\\
41.34	0\\
41.35	0\\
41.36	0\\
41.37	0\\
41.38	1.73472347597681e-18\\
41.39	0\\
41.4	0\\
41.41	1.73472347597681e-18\\
41.42	0\\
41.43	0\\
41.44	0\\
41.45	0\\
41.46	0\\
41.47	0\\
41.48	0\\
41.49	0\\
41.5	0\\
41.51	0\\
41.52	0\\
41.53	0\\
41.54	0\\
41.55	0\\
41.56	0\\
41.57	0\\
41.58	0\\
41.59	0\\
41.6	0\\
41.61	1.73472347597681e-18\\
41.62	0\\
41.63	0\\
41.64	0\\
41.65	0\\
41.66	0\\
41.67	0\\
41.68	1.73472347597681e-18\\
41.69	0\\
41.7	0\\
41.71	0\\
41.72	0\\
41.73	0\\
41.74	0\\
41.75	0\\
41.76	0\\
41.77	0\\
41.78	0\\
41.79	0\\
41.8	0\\
41.81	1.73472347597681e-18\\
41.82	0\\
41.83	0\\
41.84	0\\
41.85	1.73472347597681e-18\\
41.86	0\\
41.87	0\\
41.88	0\\
41.89	0\\
41.9	0\\
41.91	0\\
41.92	1.73472347597681e-18\\
41.93	0\\
41.94	0\\
41.95	0\\
41.96	0\\
41.97	0\\
41.98	0\\
41.99	0\\
42	0\\
42.01	0\\
42.02	0\\
42.03	0\\
42.04	0\\
42.05	0\\
42.06	0\\
42.07	0\\
42.08	0\\
42.09	0\\
42.1	0\\
42.11	0\\
42.12	0\\
42.13	0\\
42.14	1.73472347597681e-18\\
42.15	0\\
42.16	1.73472347597681e-18\\
42.17	0\\
42.18	1.73472347597681e-18\\
42.19	0\\
42.2	0\\
42.21	0\\
42.22	0\\
42.23	0\\
42.24	1.73472347597681e-18\\
42.25	0\\
42.26	0\\
42.27	0\\
42.28	0\\
42.29	0\\
42.3	1.73472347597681e-18\\
42.31	0\\
42.32	0\\
42.33	0\\
42.34	0\\
42.35	0\\
42.36	1.73472347597681e-18\\
42.37	0\\
42.38	0\\
42.39	0\\
42.4	0\\
42.41	0\\
42.42	0\\
42.43	0\\
42.44	0\\
42.45	0\\
42.46	0\\
42.47	0\\
42.48	0\\
42.49	0\\
42.5	0\\
42.51	0\\
42.52	0\\
42.53	0\\
42.54	0\\
42.55	0\\
42.56	0\\
42.57	0\\
42.58	0\\
42.59	0\\
42.6	0\\
42.61	0\\
42.62	0\\
42.63	0\\
42.64	0\\
42.65	0\\
42.66	0\\
42.67	0\\
42.68	0\\
42.69	0\\
42.7	0\\
42.71	1.73472347597681e-18\\
42.72	0\\
42.73	0\\
42.74	1.73472347597681e-18\\
42.75	0\\
42.76	0\\
42.77	1.73472347597681e-18\\
42.78	0\\
42.79	0\\
42.8	0\\
42.81	0\\
42.82	0\\
42.83	0\\
42.84	0\\
42.85	0\\
42.86	0\\
42.87	0\\
42.88	0\\
42.89	1.73472347597681e-18\\
42.9	0\\
42.91	0\\
42.92	0\\
42.93	0\\
42.94	0\\
42.95	0\\
42.96	0\\
42.97	0\\
42.98	0\\
42.99	0\\
43	1.73472347597681e-18\\
43.01	0\\
43.02	0\\
43.03	0\\
43.04	0\\
43.05	0\\
43.06	0\\
43.07	0\\
43.08	0\\
43.09	1.73472347597681e-18\\
43.1	0\\
43.11	0\\
43.12	0\\
43.13	1.73472347597681e-18\\
43.14	0\\
43.15	0\\
43.16	0\\
43.17	0\\
43.18	0\\
43.19	0\\
43.2	0\\
43.21	0\\
43.22	0\\
43.23	0\\
43.24	0\\
43.25	0\\
43.26	0\\
43.27	0\\
43.28	0\\
43.29	0\\
43.3	0\\
43.31	0\\
43.32	0\\
43.33	1.73472347597681e-18\\
43.34	0\\
43.35	0\\
43.36	0\\
43.37	0\\
43.38	1.73472347597681e-18\\
43.39	0\\
43.4	0\\
43.41	0\\
43.42	0\\
43.43	0\\
43.44	0\\
43.45	0\\
43.46	0\\
43.47	0\\
43.48	0\\
43.49	0\\
43.5	0\\
43.51	0\\
43.52	1.73472347597681e-18\\
43.53	0\\
43.54	0\\
43.55	1.73472347597681e-18\\
43.56	0\\
43.57	0\\
43.58	0\\
43.59	0\\
43.6	0\\
43.61	0\\
43.62	0\\
43.63	0\\
43.64	0\\
43.65	0\\
43.66	0\\
43.67	0\\
43.68	0\\
43.69	0\\
43.7	0\\
43.71	0\\
43.72	1.73472347597681e-18\\
43.73	0\\
43.74	1.73472347597681e-18\\
43.75	0\\
43.76	0\\
43.77	0\\
43.78	0\\
43.79	0\\
43.8	0\\
43.81	1.73472347597681e-18\\
43.82	0\\
43.83	1.73472347597681e-18\\
43.84	1.73472347597681e-18\\
43.85	0\\
43.86	0\\
43.87	0\\
43.88	0\\
43.89	0\\
43.9	0\\
43.91	0\\
43.92	0\\
43.93	1.73472347597681e-18\\
43.94	0\\
43.95	0\\
43.96	1.73472347597681e-18\\
43.97	0\\
43.98	0\\
43.99	0\\
44	0\\
44.01	0\\
44.02	0\\
44.03	1.73472347597681e-18\\
44.04	0\\
44.05	0\\
44.06	0\\
44.07	0\\
44.08	0\\
44.09	0\\
44.1	0\\
44.11	0\\
44.12	0\\
44.13	0\\
44.14	0\\
44.15	0\\
44.16	0\\
44.17	0\\
44.18	0\\
44.19	0\\
44.2	0\\
44.21	0\\
44.22	0\\
44.23	0\\
44.24	0\\
44.25	0\\
44.26	1.73472347597681e-18\\
44.27	0\\
44.28	0\\
44.29	0\\
44.3	0\\
44.31	0\\
44.32	0\\
44.33	1.73472347597681e-18\\
44.34	0\\
44.35	1.73472347597681e-18\\
44.36	1.73472347597681e-18\\
44.37	0\\
44.38	0\\
44.39	0\\
44.4	0\\
44.41	0\\
44.42	0\\
44.43	0\\
44.44	0\\
44.45	0\\
44.46	1.73472347597681e-18\\
44.47	0\\
44.48	0\\
44.49	0\\
44.5	0\\
44.51	0\\
44.52	0\\
44.53	0\\
44.54	0\\
44.55	0\\
44.56	0\\
44.57	0\\
44.58	0\\
44.59	0\\
44.6	0\\
44.61	0\\
44.62	0\\
44.63	0\\
44.64	1.73472347597681e-18\\
44.65	0\\
44.66	0\\
44.67	0\\
44.68	0\\
44.69	1.73472347597681e-18\\
44.7	0\\
44.71	0\\
44.72	0\\
44.73	0\\
44.74	0\\
44.75	0\\
44.76	0\\
44.77	0\\
44.78	0\\
44.79	0\\
44.8	0\\
44.81	0\\
44.82	0\\
44.83	0\\
44.84	1.73472347597681e-18\\
44.85	1.73472347597681e-18\\
44.86	0\\
44.87	0\\
44.88	0\\
44.89	0\\
44.9	0\\
44.91	1.73472347597681e-18\\
44.92	0\\
44.93	0\\
44.94	0\\
44.95	0\\
44.96	0\\
44.97	0\\
44.98	0\\
44.99	0\\
45	0\\
45.01	0\\
45.02	0\\
45.03	0\\
45.04	0\\
45.05	0\\
45.06	0\\
45.07	0\\
45.08	0\\
45.09	0\\
45.1	0\\
45.11	0\\
45.12	0\\
45.13	1.73472347597681e-18\\
45.14	0\\
45.15	0\\
45.16	0\\
45.17	0\\
45.18	0\\
45.19	0\\
45.2	0\\
45.21	0\\
45.22	0\\
45.23	0\\
45.24	0\\
45.25	0\\
45.26	0\\
45.27	0\\
45.28	0\\
45.29	0\\
45.3	0\\
45.31	0\\
45.32	0\\
45.33	0\\
45.34	0\\
45.35	0\\
45.36	0\\
45.37	0\\
45.38	0\\
45.39	0\\
45.4	1.73472347597681e-18\\
45.41	0\\
45.42	0\\
45.43	0\\
45.44	0\\
45.45	0\\
45.46	0\\
45.47	0\\
45.48	0\\
45.49	0\\
45.5	0\\
45.51	0\\
45.52	0\\
45.53	0\\
45.54	0\\
45.55	0\\
45.56	0\\
45.57	0\\
45.58	0\\
45.59	0\\
45.6	1.73472347597681e-18\\
45.61	0\\
45.62	0\\
45.63	1.73472347597681e-18\\
45.64	0\\
45.65	0\\
45.66	0\\
45.67	0\\
45.68	0\\
45.69	0\\
45.7	0\\
45.71	1.73472347597681e-18\\
45.72	0\\
45.73	0\\
45.74	0\\
45.75	0\\
45.76	0\\
45.77	0\\
45.78	0\\
45.79	0\\
45.8	0\\
45.81	0\\
45.82	0\\
45.83	1.73472347597681e-18\\
45.84	0\\
45.85	0\\
45.86	0\\
45.87	0\\
45.88	0\\
45.89	0\\
45.9	0\\
45.91	0\\
45.92	0\\
45.93	1.73472347597681e-18\\
45.94	0\\
45.95	0\\
45.96	0\\
45.97	0\\
45.98	0\\
45.99	0\\
46	0\\
46.01	0\\
46.02	0\\
46.03	0\\
46.04	0\\
46.05	0\\
46.06	0\\
46.07	0\\
46.08	1.73472347597681e-18\\
46.09	0\\
46.1	0\\
46.11	0\\
46.12	0\\
46.13	0\\
46.14	0\\
46.15	0\\
46.16	0\\
46.17	0\\
46.18	0\\
46.19	0\\
46.2	0\\
46.21	0\\
46.22	0\\
46.23	0\\
46.24	0\\
46.25	1.73472347597681e-18\\
46.26	0\\
46.27	0\\
46.28	0\\
46.29	1.73472347597681e-18\\
46.3	1.73472347597681e-18\\
46.31	0\\
46.32	1.73472347597681e-18\\
46.33	0\\
46.34	0\\
46.35	0\\
46.36	0\\
46.37	0\\
46.38	1.73472347597681e-18\\
46.39	0\\
46.4	0\\
46.41	0\\
46.42	0\\
46.43	0\\
46.44	0\\
46.45	0\\
46.46	0\\
46.47	0\\
46.48	0\\
46.49	0\\
46.5	0\\
46.51	0\\
46.52	0\\
46.53	0\\
46.54	0\\
46.55	0\\
46.56	0\\
46.57	0\\
46.58	0\\
46.59	1.73472347597681e-18\\
46.6	1.73472347597681e-18\\
46.61	0\\
46.62	0\\
46.63	0\\
46.64	0\\
46.65	0\\
46.66	0\\
46.67	0\\
46.68	1.73472347597681e-18\\
46.69	0\\
46.7	0\\
46.71	0\\
46.72	0\\
46.73	0\\
46.74	0\\
46.75	0\\
46.76	0\\
46.77	0\\
46.78	0\\
46.79	0\\
46.8	0\\
46.81	0\\
46.82	0\\
46.83	0\\
46.84	0\\
46.85	0\\
46.86	0\\
46.87	1.73472347597681e-18\\
46.88	0\\
46.89	1.73472347597681e-18\\
46.9	0\\
46.91	0\\
46.92	0\\
46.93	0\\
46.94	0\\
46.95	0\\
46.96	0\\
46.97	0\\
46.98	0\\
46.99	0\\
47	0\\
47.01	0\\
47.02	0\\
47.03	0\\
47.04	0\\
47.05	1.73472347597681e-18\\
47.06	0\\
47.07	0\\
47.08	0\\
47.09	0\\
47.1	1.73472347597681e-18\\
47.11	0\\
47.12	0\\
47.13	0\\
47.14	1.73472347597681e-18\\
47.15	0\\
47.16	1.73472347597681e-18\\
47.17	0\\
47.18	0\\
47.19	0\\
47.2	0\\
47.21	0\\
47.22	0\\
47.23	0\\
47.24	0\\
47.25	0\\
47.26	0\\
47.27	0\\
47.28	0\\
47.29	0\\
47.3	0\\
47.31	0\\
47.32	0\\
47.33	0\\
47.34	0\\
47.35	0\\
47.36	1.73472347597681e-18\\
47.37	0\\
47.38	1.73472347597681e-18\\
47.39	0\\
47.4	0\\
47.41	0\\
47.42	0\\
47.43	0\\
47.44	0\\
47.45	0\\
47.46	1.73472347597681e-18\\
47.47	0\\
47.48	0\\
47.49	0\\
47.5	0\\
47.51	1.73472347597681e-18\\
47.52	1.73472347597681e-18\\
47.53	0\\
47.54	0\\
47.55	0\\
47.56	0\\
47.57	1.73472347597681e-18\\
47.58	1.73472347597681e-18\\
47.59	0\\
47.6	0\\
47.61	0\\
47.62	0\\
47.63	0\\
47.64	0\\
47.65	0\\
47.66	0\\
47.67	0\\
47.68	0\\
47.69	0\\
47.7	0\\
47.71	0\\
47.72	0\\
47.73	0\\
47.74	0\\
47.75	0\\
47.76	1.73472347597681e-18\\
47.77	0\\
47.78	0\\
47.79	0\\
47.8	0\\
47.81	0\\
47.82	0\\
47.83	0\\
47.84	1.73472347597681e-18\\
47.85	0\\
47.86	0\\
47.87	0\\
47.88	0\\
47.89	0\\
47.9	0\\
47.91	0\\
47.92	0\\
47.93	0\\
47.94	0\\
47.95	0\\
47.96	0\\
47.97	1.73472347597681e-18\\
47.98	0\\
47.99	0\\
48	1.73472347597681e-18\\
48.01	0\\
48.02	0\\
48.03	0\\
48.04	0\\
48.05	0\\
48.06	0\\
48.07	1.73472347597681e-18\\
48.08	1.73472347597681e-18\\
48.09	0\\
48.1	0\\
48.11	1.73472347597681e-18\\
48.12	0\\
48.13	0\\
48.14	0\\
48.15	0\\
48.16	0\\
48.17	0\\
48.18	0\\
48.19	0\\
48.2	0\\
48.21	0\\
48.22	0\\
48.23	0\\
48.24	0\\
48.25	0\\
48.26	0\\
48.27	0\\
48.28	0\\
48.29	0\\
48.3	0\\
48.31	0\\
48.32	0\\
48.33	0\\
48.34	0\\
48.35	0\\
48.36	0\\
48.37	0\\
48.38	0\\
48.39	0\\
48.4	0\\
48.41	0\\
48.42	0\\
48.43	0\\
48.44	0\\
48.45	0\\
48.46	0\\
48.47	0\\
48.48	0\\
48.49	0\\
48.5	1.73472347597681e-18\\
48.51	0\\
48.52	0\\
48.53	0\\
48.54	0\\
48.55	0\\
48.56	0\\
48.57	0\\
48.58	0\\
48.59	0\\
48.6	0\\
48.61	0\\
48.62	0\\
48.63	0\\
48.64	0\\
48.65	0\\
48.66	0\\
48.67	0\\
48.68	0\\
48.69	0\\
48.7	0\\
48.71	0\\
48.72	0\\
48.73	1.73472347597681e-18\\
48.74	0\\
48.75	0\\
48.76	0\\
48.77	0\\
48.78	0\\
48.79	0\\
48.8	1.73472347597681e-18\\
48.81	0\\
48.82	0\\
48.83	0\\
48.84	0\\
48.85	0\\
48.86	0\\
48.87	1.73472347597681e-18\\
48.88	1.73472347597681e-18\\
48.89	0\\
48.9	0\\
48.91	0\\
48.92	0\\
48.93	0\\
48.94	0\\
48.95	1.73472347597681e-18\\
48.96	1.73472347597681e-18\\
48.97	0\\
48.98	0\\
48.99	0\\
49	0\\
49.01	0\\
49.02	0\\
49.03	0\\
49.04	1.73472347597681e-18\\
49.05	1.73472347597681e-18\\
49.06	0\\
49.07	0\\
49.08	0\\
49.09	0\\
49.1	0\\
49.11	0\\
49.12	0\\
49.13	1.73472347597681e-18\\
49.14	0\\
49.15	0\\
49.16	0\\
49.17	0\\
49.18	0\\
49.19	0\\
49.2	0\\
49.21	0\\
49.22	0\\
49.23	0\\
49.24	0\\
49.25	0\\
49.26	0\\
49.27	0\\
49.28	0\\
49.29	0\\
49.3	0\\
49.31	0\\
49.32	1.73472347597681e-18\\
49.33	0\\
49.34	0\\
49.35	0\\
49.36	0\\
49.37	0\\
49.38	0\\
49.39	0\\
49.4	0\\
49.41	0\\
49.42	0\\
49.43	0\\
49.44	0\\
49.45	0\\
49.46	0\\
49.47	0\\
49.48	0\\
49.49	0\\
49.5	0\\
49.51	0\\
49.52	0\\
49.53	0\\
49.54	0\\
49.55	0\\
49.56	0\\
49.57	0\\
49.58	0\\
49.59	0\\
49.6	0\\
49.61	0\\
49.62	0\\
49.63	0\\
49.64	0\\
49.65	0\\
49.66	0\\
49.67	1.73472347597681e-18\\
49.68	1.73472347597681e-18\\
49.69	0\\
49.7	0\\
49.71	0\\
49.72	0\\
49.73	0\\
49.74	0\\
49.75	0\\
49.76	0\\
49.77	0\\
49.78	0\\
49.79	0\\
49.8	0\\
49.81	0\\
49.82	0\\
49.83	0\\
49.84	0\\
49.85	0\\
49.86	0\\
49.87	0\\
49.88	0\\
49.89	0\\
49.9	0\\
49.91	0\\
49.92	0\\
49.93	0\\
49.94	1.73472347597681e-18\\
49.95	0\\
49.96	0\\
49.97	0\\
49.98	0\\
49.99	0\\
50	0\\
50.01	0\\
50.02	0\\
50.03	0\\
50.04	0\\
50.05	0\\
50.06	0\\
50.07	0\\
50.08	0\\
50.09	0\\
50.1	0\\
50.11	0\\
50.12	0\\
50.13	0\\
50.14	1.73472347597681e-18\\
50.15	0\\
50.16	0\\
50.17	0\\
50.18	0\\
50.19	0\\
50.2	0\\
50.21	1.73472347597681e-18\\
50.22	0\\
50.23	0\\
50.24	0\\
50.25	0\\
50.26	0\\
50.27	0\\
50.28	0\\
50.29	0\\
50.3	0\\
50.31	1.73472347597681e-18\\
50.32	0\\
50.33	0\\
50.34	0\\
50.35	0\\
50.36	0\\
50.37	1.73472347597681e-18\\
50.38	0\\
50.39	0\\
50.4	0\\
50.41	0\\
50.42	0\\
50.43	0\\
50.44	0\\
50.45	0\\
50.46	0\\
50.47	1.73472347597681e-18\\
50.48	0\\
50.49	0\\
50.5	0\\
50.51	1.73472347597681e-18\\
50.52	0\\
50.53	0\\
50.54	0\\
50.55	0\\
50.56	1.73472347597681e-18\\
50.57	0\\
50.58	0\\
50.59	0\\
50.6	0\\
50.61	0\\
50.62	0\\
50.63	0\\
50.64	0\\
50.65	0\\
50.66	0\\
50.67	0\\
50.68	0\\
50.69	0\\
50.7	0\\
50.71	0\\
50.72	0\\
50.73	0\\
50.74	0\\
50.75	0\\
50.76	0\\
50.77	0\\
50.78	1.73472347597681e-18\\
50.79	0\\
50.8	0\\
50.81	0\\
50.82	0\\
50.83	0\\
50.84	0\\
50.85	0\\
50.86	0\\
50.87	1.73472347597681e-18\\
50.88	1.73472347597681e-18\\
50.89	0\\
50.9	0\\
50.91	0\\
50.92	0\\
50.93	1.73472347597681e-18\\
50.94	0\\
50.95	0\\
50.96	1.73472347597681e-18\\
50.97	0\\
50.98	0\\
50.99	0\\
51	0\\
51.01	1.73472347597681e-18\\
51.02	0\\
51.03	0\\
51.04	0\\
51.05	0\\
51.06	0\\
51.07	0\\
51.08	0\\
51.09	0\\
51.1	0\\
51.11	0\\
51.12	0\\
51.13	0\\
51.14	0\\
51.15	0\\
51.16	0\\
51.17	0\\
51.18	0\\
51.19	1.73472347597681e-18\\
51.2	0\\
51.21	0\\
51.22	0\\
51.23	1.73472347597681e-18\\
51.24	0\\
51.25	0\\
51.26	0\\
51.27	0\\
51.28	0\\
51.29	0\\
51.3	0\\
51.31	0\\
51.32	1.73472347597681e-18\\
51.33	0\\
51.34	0\\
51.35	0\\
51.36	0\\
51.37	1.73472347597681e-18\\
51.38	0\\
51.39	0\\
51.4	0\\
51.41	0\\
51.42	1.73472347597681e-18\\
51.43	0\\
51.44	0\\
51.45	0\\
51.46	1.73472347597681e-18\\
51.47	0\\
51.48	0\\
51.49	1.73472347597681e-18\\
51.5	0\\
51.51	0\\
51.52	0\\
51.53	0\\
51.54	0\\
51.55	0\\
51.56	0\\
51.57	0\\
51.58	1.73472347597681e-18\\
51.59	0\\
51.6	0\\
51.61	0\\
51.62	0\\
51.63	0\\
51.64	0\\
51.65	0\\
51.66	0\\
51.67	0\\
51.68	1.73472347597681e-18\\
51.69	0\\
51.7	1.73472347597681e-18\\
51.71	0\\
51.72	0\\
51.73	0\\
51.74	0\\
51.75	0\\
51.76	0\\
51.77	0\\
51.78	0\\
51.79	0\\
51.8	0\\
51.81	0\\
51.82	0\\
51.83	0\\
51.84	0\\
51.85	0\\
51.86	0\\
51.87	0\\
51.88	0\\
51.89	0\\
51.9	0\\
51.91	0\\
51.92	0\\
51.93	1.73472347597681e-18\\
51.94	0\\
51.95	0\\
51.96	0\\
51.97	0\\
51.98	0\\
51.99	0\\
52	0\\
52.01	0\\
52.02	0\\
52.03	0\\
52.04	0\\
52.05	0\\
52.06	0\\
52.07	0\\
52.08	0\\
52.09	0\\
52.1	0\\
52.11	0\\
52.12	0\\
52.13	0\\
52.14	0\\
52.15	0\\
52.16	1.73472347597681e-18\\
52.17	1.73472347597681e-18\\
52.18	0\\
52.19	0\\
52.2	0\\
52.21	0\\
52.22	1.73472347597681e-18\\
52.23	0\\
52.24	0\\
52.25	0\\
52.26	0\\
52.27	0\\
52.28	0\\
52.29	0\\
52.3	0\\
52.31	0\\
52.32	0\\
52.33	0\\
52.34	0\\
52.35	0\\
52.36	0\\
52.37	1.73472347597681e-18\\
52.38	0\\
52.39	1.73472347597681e-18\\
52.4	1.73472347597681e-18\\
52.41	0\\
52.42	0\\
52.43	0\\
52.44	0\\
52.45	0\\
52.46	0\\
52.47	0\\
52.48	0\\
52.49	0\\
52.5	0\\
52.51	0\\
52.52	0\\
52.53	0\\
52.54	0\\
52.55	0\\
52.56	0\\
52.57	0\\
52.58	0\\
52.59	0\\
52.6	0\\
52.61	1.73472347597681e-18\\
52.62	1.73472347597681e-18\\
52.63	0\\
52.64	1.73472347597681e-18\\
52.65	0\\
52.66	0\\
52.67	0\\
52.68	0\\
52.69	0\\
52.7	0\\
52.71	0\\
52.72	1.73472347597681e-18\\
52.73	0\\
52.74	0\\
52.75	0\\
52.76	0\\
52.77	0\\
52.78	0\\
52.79	0\\
52.8	0\\
52.81	0\\
52.82	0\\
52.83	0\\
52.84	0\\
52.85	1.73472347597681e-18\\
52.86	0\\
52.87	1.73472347597681e-18\\
52.88	0\\
52.89	0\\
52.9	1.73472347597681e-18\\
52.91	0\\
52.92	0\\
52.93	0\\
52.94	0\\
52.95	1.73472347597681e-18\\
52.96	0\\
52.97	0\\
52.98	0\\
52.99	0\\
53	0\\
53.01	1.73472347597681e-18\\
53.02	0\\
53.03	0\\
53.04	0\\
53.05	0\\
53.06	0\\
53.07	0\\
53.08	0\\
53.09	0\\
53.1	1.73472347597681e-18\\
53.11	1.73472347597681e-18\\
53.12	0\\
53.13	0\\
53.14	0\\
53.15	0\\
53.16	0\\
53.17	0\\
53.18	0\\
53.19	0\\
53.2	1.73472347597681e-18\\
53.21	0\\
53.22	0\\
53.23	0\\
53.24	0\\
53.25	0\\
53.26	0\\
53.27	1.73472347597681e-18\\
53.28	0\\
53.29	0\\
53.3	0\\
53.31	0\\
53.32	0\\
53.33	0\\
53.34	0\\
53.35	0\\
53.36	0\\
53.37	0\\
53.38	0\\
53.39	0\\
53.4	0\\
53.41	0\\
53.42	0\\
53.43	0\\
53.44	0\\
53.45	1.73472347597681e-18\\
53.46	0\\
53.47	0\\
53.48	0\\
53.49	0\\
53.5	0\\
53.51	0\\
53.52	1.73472347597681e-18\\
53.53	1.73472347597681e-18\\
53.54	0\\
53.55	0\\
53.56	0\\
53.57	0\\
53.58	1.73472347597681e-18\\
53.59	0\\
53.6	0\\
53.61	0\\
53.62	0\\
53.63	0\\
53.64	0\\
53.65	0\\
53.66	0\\
53.67	0\\
53.68	0\\
53.69	0\\
53.7	0\\
53.71	0\\
53.72	0\\
53.73	0\\
53.74	0\\
53.75	0\\
53.76	0\\
53.77	0\\
53.78	0\\
53.79	1.73472347597681e-18\\
53.8	0\\
53.81	0\\
53.82	0\\
53.83	0\\
53.84	0\\
53.85	1.73472347597681e-18\\
53.86	0\\
53.87	0\\
53.88	0\\
53.89	0\\
53.9	0\\
53.91	0\\
53.92	0\\
53.93	0\\
53.94	0\\
53.95	0\\
53.96	0\\
53.97	0\\
53.98	0\\
53.99	0\\
54	0\\
54.01	0\\
54.02	0\\
54.03	0\\
54.04	0\\
54.05	0\\
54.06	0\\
54.07	1.73472347597681e-18\\
54.08	0\\
54.09	1.73472347597681e-18\\
54.1	0\\
54.11	1.73472347597681e-18\\
54.12	0\\
54.13	1.73472347597681e-18\\
54.14	0\\
54.15	1.73472347597681e-18\\
54.16	0\\
54.17	0\\
54.18	0\\
54.19	0\\
54.2	1.73472347597681e-18\\
54.21	1.73472347597681e-18\\
54.22	0\\
54.23	0\\
54.24	0\\
54.25	0\\
54.26	0\\
54.27	0\\
54.28	0\\
54.29	0\\
54.3	0\\
54.31	0\\
54.32	1.73472347597681e-18\\
54.33	0\\
54.34	0\\
54.35	0\\
54.36	0\\
54.37	0\\
54.38	0\\
54.39	0\\
54.4	0\\
54.41	0\\
54.42	0\\
54.43	0\\
54.44	0\\
54.45	1.73472347597681e-18\\
54.46	0\\
54.47	0\\
54.48	0\\
54.49	0\\
54.5	0\\
54.51	0\\
54.52	0\\
54.53	0\\
54.54	0\\
54.55	0\\
54.56	0\\
54.57	0\\
54.58	0\\
54.59	0\\
54.6	0\\
54.61	0\\
54.62	0\\
54.63	0\\
54.64	0\\
54.65	1.73472347597681e-18\\
54.66	0\\
54.67	0\\
54.68	0\\
54.69	1.73472347597681e-18\\
54.7	0\\
54.71	0\\
54.72	0\\
54.73	0\\
54.74	0\\
54.75	0\\
54.76	0\\
54.77	0\\
54.78	0\\
54.79	0\\
54.8	0\\
54.81	0\\
54.82	0\\
54.83	0\\
54.84	0\\
54.85	0\\
54.86	0\\
54.87	0\\
54.88	0\\
54.89	0\\
54.9	1.73472347597681e-18\\
54.91	0\\
54.92	0\\
54.93	0\\
54.94	1.73472347597681e-18\\
54.95	0\\
54.96	0\\
54.97	0\\
54.98	1.73472347597681e-18\\
54.99	0\\
55	0\\
55.01	0\\
55.02	0\\
55.03	0\\
55.04	0\\
55.05	0\\
55.06	0\\
55.07	1.73472347597681e-18\\
55.08	1.73472347597681e-18\\
55.09	0\\
55.1	0\\
55.11	0\\
55.12	0\\
55.13	0\\
55.14	1.73472347597681e-18\\
55.15	0\\
55.16	1.73472347597681e-18\\
55.17	1.73472347597681e-18\\
55.18	0\\
55.19	1.73472347597681e-18\\
55.2	0\\
55.21	0\\
55.22	0\\
55.23	0\\
55.24	0\\
55.25	0\\
55.26	0\\
55.27	0\\
55.28	0\\
55.29	0\\
55.3	0\\
55.31	0\\
55.32	0\\
55.33	0\\
55.34	0\\
55.35	0\\
55.36	1.73472347597681e-18\\
55.37	0\\
55.38	0\\
55.39	0\\
55.4	0\\
55.41	0\\
55.42	0\\
55.43	0\\
55.44	0\\
55.45	0\\
55.46	0\\
55.47	1.73472347597681e-18\\
55.48	0\\
55.49	1.73472347597681e-18\\
55.5	0\\
55.51	0\\
55.52	0\\
55.53	0\\
55.54	0\\
55.55	0\\
55.56	1.73472347597681e-18\\
55.57	0\\
55.58	0\\
55.59	0\\
55.6	0\\
55.61	0\\
55.62	0\\
55.63	0\\
55.64	0\\
55.65	0\\
55.66	0\\
55.67	0\\
55.68	0\\
55.69	0\\
55.7	0\\
55.71	0\\
55.72	0\\
55.73	1.73472347597681e-18\\
55.74	0\\
55.75	0\\
55.76	0\\
55.77	1.73472347597681e-18\\
55.78	0\\
55.79	0\\
55.8	0\\
55.81	0\\
55.82	0\\
55.83	0\\
55.84	0\\
55.85	0\\
55.86	0\\
55.87	0\\
55.88	0\\
55.89	0\\
55.9	0\\
55.91	0\\
55.92	0\\
55.93	0\\
55.94	0\\
55.95	1.73472347597681e-18\\
55.96	0\\
55.97	0\\
55.98	0\\
55.99	0\\
56	0\\
56.01	1.73472347597681e-18\\
56.02	0\\
56.03	0\\
56.04	1.73472347597681e-18\\
56.05	1.73472347597681e-18\\
56.06	0\\
56.07	1.73472347597681e-18\\
56.08	0\\
56.09	0\\
56.1	0\\
56.11	0\\
56.12	0\\
56.13	0\\
56.14	0\\
56.15	0\\
56.16	0\\
56.17	0\\
56.18	0\\
56.19	0\\
56.2	0\\
56.21	0\\
56.22	0\\
56.23	0\\
56.24	1.73472347597681e-18\\
56.25	0\\
56.26	0\\
56.27	0\\
56.28	0\\
56.29	1.73472347597681e-18\\
56.3	0\\
56.31	0\\
56.32	0\\
56.33	0\\
56.34	0\\
56.35	0\\
56.36	0\\
56.37	0\\
56.38	0\\
56.39	0\\
56.4	0\\
56.41	0\\
56.42	0\\
56.43	0\\
56.44	0\\
56.45	0\\
56.46	0\\
56.47	0\\
56.48	0\\
56.49	0\\
56.5	0\\
56.51	0\\
56.52	0\\
56.53	0\\
56.54	0\\
56.55	0\\
56.56	0\\
56.57	0\\
56.58	0\\
56.59	0\\
56.6	0\\
56.61	0\\
56.62	0\\
56.63	0\\
56.64	0\\
56.65	0\\
56.66	0\\
56.67	0\\
56.68	0\\
56.69	0\\
56.7	1.73472347597681e-18\\
56.71	1.73472347597681e-18\\
56.72	0\\
56.73	0\\
56.74	1.73472347597681e-18\\
56.75	1.73472347597681e-18\\
56.76	0\\
56.77	0\\
56.78	0\\
56.79	1.73472347597681e-18\\
56.8	0\\
56.81	0\\
56.82	1.73472347597681e-18\\
56.83	0\\
56.84	0\\
56.85	0\\
56.86	0\\
56.87	0\\
56.88	0\\
56.89	0\\
56.9	0\\
56.91	0\\
56.92	0\\
56.93	0\\
56.94	0\\
56.95	0\\
56.96	0\\
56.97	0\\
56.98	0\\
56.99	0\\
57	1.73472347597681e-18\\
57.01	0\\
57.02	0\\
57.03	0\\
57.04	0\\
57.05	1.73472347597681e-18\\
57.06	1.73472347597681e-18\\
57.07	0\\
57.08	0\\
57.09	0\\
57.1	0\\
57.11	0\\
57.12	0\\
57.13	0\\
57.14	0\\
57.15	0\\
57.16	0\\
57.17	0\\
57.18	0\\
57.19	0\\
57.2	0\\
57.21	0\\
57.22	0\\
57.23	0\\
57.24	0\\
57.25	0\\
57.26	0\\
57.27	1.73472347597681e-18\\
57.28	0\\
57.29	0\\
57.3	0\\
57.31	0\\
57.32	0\\
57.33	0\\
57.34	0\\
57.35	0\\
57.36	0\\
57.37	0\\
57.38	1.73472347597681e-18\\
57.39	0\\
57.4	0\\
57.41	0\\
57.42	0\\
57.43	0\\
57.44	0\\
57.45	0\\
57.46	0\\
57.47	0\\
57.48	0\\
57.49	0\\
57.5	0\\
57.51	1.73472347597681e-18\\
57.52	0\\
57.53	0\\
57.54	0\\
57.55	0\\
57.56	0\\
57.57	0\\
57.58	0\\
57.59	0\\
57.6	0\\
57.61	0\\
57.62	0\\
57.63	0\\
57.64	0\\
57.65	0\\
57.66	0\\
57.67	0\\
57.68	1.73472347597681e-18\\
57.69	1.73472347597681e-18\\
57.7	0\\
57.71	0\\
57.72	0\\
57.73	0\\
57.74	0\\
57.75	0\\
57.76	0\\
57.77	1.73472347597681e-18\\
57.78	0\\
57.79	0\\
57.8	0\\
57.81	0\\
57.82	0\\
57.83	0\\
57.84	1.73472347597681e-18\\
57.85	1.73472347597681e-18\\
57.86	0\\
57.87	1.73472347597681e-18\\
57.88	0\\
57.89	0\\
57.9	0\\
57.91	0\\
57.92	0\\
57.93	0\\
57.94	0\\
57.95	0\\
57.96	0\\
57.97	0\\
57.98	0\\
57.99	0\\
58	1.73472347597681e-18\\
58.01	0\\
58.02	0\\
58.03	0\\
58.04	0\\
58.05	0\\
58.06	0\\
58.07	0\\
58.08	0\\
58.09	0\\
58.1	0\\
58.11	1.73472347597681e-18\\
58.12	0\\
58.13	1.73472347597681e-18\\
58.14	0\\
58.15	0\\
58.16	0\\
58.17	0\\
58.18	0\\
58.19	0\\
58.2	0\\
58.21	0\\
58.22	1.73472347597681e-18\\
58.23	0\\
58.24	0\\
58.25	1.73472347597681e-18\\
58.26	0\\
58.27	0\\
58.28	0\\
58.29	1.73472347597681e-18\\
58.3	0\\
58.31	1.73472347597681e-18\\
58.32	0\\
58.33	0\\
58.34	1.73472347597681e-18\\
58.35	1.73472347597681e-18\\
58.36	0\\
58.37	0\\
58.38	0\\
58.39	0\\
58.4	0\\
58.41	1.73472347597681e-18\\
58.42	1.73472347597681e-18\\
58.43	0\\
58.44	0\\
58.45	0\\
58.46	0\\
58.47	0\\
58.48	0\\
58.49	0\\
58.5	0\\
58.51	0\\
58.52	1.73472347597681e-18\\
58.53	0\\
58.54	0\\
58.55	0\\
58.56	0\\
58.57	0\\
58.58	0\\
58.59	0\\
58.6	0\\
58.61	0\\
58.62	0\\
58.63	0\\
58.64	0\\
58.65	0\\
58.66	0\\
58.67	1.73472347597681e-18\\
58.68	0\\
58.69	0\\
58.7	0\\
58.71	0\\
58.72	0\\
58.73	0\\
58.74	1.73472347597681e-18\\
58.75	0\\
58.76	0\\
58.77	0\\
58.78	0\\
58.79	1.73472347597681e-18\\
58.8	0\\
58.81	0\\
58.82	0\\
58.83	0\\
58.84	0\\
58.85	0\\
58.86	0\\
58.87	0\\
58.88	1.73472347597681e-18\\
58.89	0\\
58.9	0\\
58.91	0\\
58.92	0\\
58.93	0\\
58.94	0\\
58.95	0\\
58.96	0\\
58.97	0\\
58.98	1.73472347597681e-18\\
58.99	0\\
59	0\\
59.01	0\\
59.02	0\\
59.03	0\\
59.04	0\\
59.05	0\\
59.06	0\\
59.07	0\\
59.08	0\\
59.09	0\\
59.1	0\\
59.11	0\\
59.12	0\\
59.13	0\\
59.14	0\\
59.15	0\\
59.16	0\\
59.17	1.73472347597681e-18\\
59.18	0\\
59.19	0\\
59.2	1.73472347597681e-18\\
59.21	0\\
59.22	0\\
59.23	0\\
59.24	0\\
59.25	1.73472347597681e-18\\
59.26	0\\
59.27	0\\
59.28	0\\
59.29	0\\
59.3	1.73472347597681e-18\\
59.31	0\\
59.32	1.73472347597681e-18\\
59.33	0\\
59.34	0\\
59.35	0\\
59.36	0\\
59.37	0\\
59.38	0\\
59.39	0\\
59.4	0\\
59.41	1.73472347597681e-18\\
59.42	1.73472347597681e-18\\
59.43	0\\
59.44	0\\
59.45	0\\
59.46	0\\
59.47	0\\
59.48	0\\
59.49	0\\
59.5	1.73472347597681e-18\\
59.51	1.73472347597681e-18\\
59.52	0\\
59.53	0\\
59.54	0\\
59.55	0\\
59.56	0\\
59.57	0\\
59.58	0\\
59.59	1.73472347597681e-18\\
59.6	0\\
59.61	0\\
59.62	1.73472347597681e-18\\
59.63	0\\
59.64	0\\
59.65	0\\
59.66	1.73472347597681e-18\\
59.67	0\\
59.68	0\\
59.69	0\\
59.7	0\\
59.71	0\\
59.72	0\\
59.73	0\\
59.74	0\\
59.75	0\\
59.76	0\\
59.77	0\\
59.78	0\\
59.79	0\\
59.8	1.73472347597681e-18\\
59.81	0\\
59.82	0\\
59.83	0\\
59.84	0\\
59.85	0\\
59.86	1.73472347597681e-18\\
59.87	0\\
59.88	1.73472347597681e-18\\
59.89	0\\
59.9	0\\
59.91	0\\
59.92	0\\
59.93	1.73472347597681e-18\\
59.94	0\\
59.95	0\\
59.96	0\\
59.97	0\\
59.98	0\\
59.99	0\\
60	0\\
60.01	1.73472347597681e-18\\
60.02	0\\
60.03	0\\
60.04	0\\
60.05	0\\
60.06	0\\
60.07	0\\
60.08	0\\
60.09	0\\
60.1	0\\
60.11	0\\
60.12	0\\
60.13	1.73472347597681e-18\\
60.14	0\\
60.15	0\\
60.16	0\\
60.17	0\\
60.18	0\\
60.19	0\\
60.2	0\\
60.21	0\\
60.22	0\\
60.23	0\\
60.24	0\\
60.25	0\\
60.26	0\\
60.27	0\\
60.28	0\\
60.29	0\\
60.3	0\\
60.31	0\\
60.32	0\\
60.33	0\\
60.34	0\\
60.35	0\\
60.36	0\\
60.37	0\\
60.38	0\\
60.39	1.73472347597681e-18\\
60.4	0\\
60.41	0\\
60.42	0\\
60.43	0\\
60.44	0\\
60.45	0\\
60.46	0\\
60.47	0\\
60.48	0\\
60.49	1.73472347597681e-18\\
60.5	0\\
60.51	0\\
60.52	0\\
60.53	0\\
60.54	0\\
60.55	0\\
60.56	0\\
60.57	0\\
60.58	0\\
60.59	0\\
60.6	0\\
60.61	0\\
60.62	0\\
60.63	0\\
60.64	0\\
60.65	0\\
60.66	1.73472347597681e-18\\
60.67	0\\
60.68	0\\
60.69	0\\
60.7	0\\
60.71	0\\
60.72	0\\
60.73	0\\
60.74	0\\
60.75	0\\
60.76	0\\
60.77	1.73472347597681e-18\\
60.78	0\\
60.79	0\\
60.8	0\\
60.81	0\\
60.82	0\\
60.83	0\\
60.84	0\\
60.85	0\\
60.86	0\\
60.87	0\\
60.88	0\\
60.89	0\\
60.9	0\\
60.91	0\\
60.92	0\\
60.93	0\\
60.94	0\\
60.95	0\\
60.96	0\\
60.97	0\\
60.98	0\\
60.99	0\\
61	0\\
61.01	1.73472347597681e-18\\
61.02	0\\
61.03	0\\
61.04	0\\
61.05	0\\
61.06	1.73472347597681e-18\\
61.07	0\\
61.08	0\\
61.09	0\\
61.1	0\\
61.11	0\\
61.12	0\\
61.13	0\\
61.14	1.73472347597681e-18\\
61.15	0\\
61.16	0\\
61.17	0\\
61.18	0\\
61.19	0\\
61.2	0\\
61.21	0\\
61.22	0\\
61.23	0\\
61.24	0\\
61.25	0\\
61.26	0\\
61.27	0\\
61.28	1.73472347597681e-18\\
61.29	0\\
61.3	0\\
61.31	0\\
61.32	0\\
61.33	0\\
61.34	0\\
61.35	0\\
61.36	0\\
61.37	0\\
61.38	0\\
61.39	0\\
61.4	0\\
61.41	0\\
61.42	0\\
61.43	1.73472347597681e-18\\
61.44	0\\
61.45	0\\
61.46	1.73472347597681e-18\\
61.47	1.73472347597681e-18\\
61.48	0\\
61.49	1.73472347597681e-18\\
61.5	0\\
61.51	0\\
61.52	0\\
61.53	0\\
61.54	0\\
61.55	0\\
61.56	0\\
61.57	0\\
61.58	0\\
61.59	0\\
61.6	0\\
61.61	0\\
61.62	0\\
61.63	0\\
61.64	0\\
61.65	1.73472347597681e-18\\
61.66	0\\
61.67	1.73472347597681e-18\\
61.68	0\\
61.69	0\\
61.7	0\\
61.71	0\\
61.72	0\\
61.73	0\\
61.74	0\\
61.75	0\\
61.76	0\\
61.77	1.73472347597681e-18\\
61.78	0\\
61.79	0\\
61.8	0\\
61.81	0\\
61.82	1.73472347597681e-18\\
61.83	0\\
61.84	1.73472347597681e-18\\
61.85	0\\
61.86	0\\
61.87	0\\
61.88	1.73472347597681e-18\\
61.89	0\\
61.9	0\\
61.91	0\\
61.92	0\\
61.93	0\\
61.94	0\\
61.95	0\\
61.96	0\\
61.97	1.73472347597681e-18\\
61.98	0\\
61.99	0\\
62	0\\
62.01	0\\
62.02	0\\
62.03	0\\
62.04	1.73472347597681e-18\\
62.05	0\\
62.06	0\\
62.07	0\\
62.08	0\\
62.09	0\\
62.1	0\\
62.11	0\\
62.12	0\\
62.13	0\\
62.14	0\\
62.15	0\\
62.16	0\\
62.17	1.73472347597681e-18\\
62.18	0\\
62.19	0\\
62.2	0\\
62.21	0\\
62.22	0\\
62.23	0\\
62.24	0\\
62.25	0\\
62.26	0\\
62.27	0\\
62.28	0\\
62.29	0\\
62.3	0\\
62.31	0\\
62.32	0\\
62.33	0\\
62.34	0\\
62.35	0\\
62.36	0\\
62.37	0\\
62.38	0\\
62.39	0\\
62.4	0\\
62.41	0\\
62.42	0\\
62.43	0\\
62.44	0\\
62.45	0\\
62.46	1.73472347597681e-18\\
62.47	0\\
62.48	0\\
62.49	1.73472347597681e-18\\
62.5	0\\
62.51	0\\
62.52	0\\
62.53	0\\
62.54	0\\
62.55	0\\
62.56	0\\
62.57	0\\
62.58	0\\
62.59	0\\
62.6	0\\
62.61	0\\
62.62	0\\
62.63	0\\
62.64	0\\
62.65	0\\
62.66	0\\
62.67	0\\
62.68	0\\
62.69	0\\
62.7	0\\
62.71	0\\
62.72	0\\
62.73	0\\
62.74	0\\
62.75	0\\
62.76	0\\
62.77	1.73472347597681e-18\\
62.78	0\\
62.79	0\\
62.8	0\\
62.81	0\\
62.82	0\\
62.83	0\\
62.84	1.73472347597681e-18\\
62.85	0\\
62.86	0\\
62.87	0\\
62.88	0\\
62.89	0\\
62.9	0\\
62.91	0\\
62.92	0\\
62.93	0\\
62.94	0\\
62.95	0\\
62.96	0\\
62.97	0\\
62.98	0\\
62.99	1.73472347597681e-18\\
63	0\\
63.01	0\\
63.02	1.73472347597681e-18\\
63.03	0\\
63.04	0\\
63.05	1.73472347597681e-18\\
63.06	0\\
63.07	0\\
63.08	0\\
63.09	0\\
63.1	0\\
63.11	0\\
63.12	0\\
63.13	0\\
63.14	0\\
63.15	0\\
63.16	0\\
63.17	0\\
63.18	1.73472347597681e-18\\
63.19	0\\
63.2	0\\
63.21	0\\
63.22	0\\
63.23	0\\
63.24	0\\
63.25	0\\
63.26	0\\
63.27	0\\
63.28	0\\
63.29	0\\
63.3	0\\
63.31	0\\
63.32	0\\
63.33	0\\
63.34	0\\
63.35	0\\
63.36	0\\
63.37	1.73472347597681e-18\\
63.38	0\\
63.39	0\\
63.4	0\\
63.41	0\\
63.42	0\\
63.43	0\\
63.44	0\\
63.45	0\\
63.46	0\\
63.47	0\\
63.48	0\\
63.49	1.73472347597681e-18\\
63.5	0\\
63.51	0\\
63.52	0\\
63.53	0\\
63.54	0\\
63.55	0\\
63.56	0\\
63.57	0\\
63.58	0\\
63.59	0\\
63.6	0\\
63.61	0\\
63.62	0\\
63.63	0\\
63.64	0\\
63.65	0\\
63.66	0\\
63.67	0\\
63.68	0\\
63.69	0\\
63.7	0\\
63.71	0\\
63.72	0\\
63.73	0\\
63.74	0\\
63.75	0\\
63.76	0\\
63.77	0\\
63.78	0\\
63.79	0\\
63.8	1.73472347597681e-18\\
63.81	0\\
63.82	0\\
63.83	0\\
63.84	0\\
63.85	0\\
63.86	0\\
63.87	0\\
63.88	0\\
63.89	0\\
63.9	0\\
63.91	0\\
63.92	1.73472347597681e-18\\
63.93	0\\
63.94	0\\
63.95	0\\
63.96	1.73472347597681e-18\\
63.97	1.73472347597681e-18\\
63.98	0\\
63.99	0\\
64	0\\
64.01	0\\
64.02	0\\
64.03	0\\
64.04	0\\
64.05	0\\
64.06	1.73472347597681e-18\\
64.07	0\\
64.08	0\\
64.09	0\\
64.1	0\\
64.11	0\\
64.12	0\\
64.13	1.73472347597681e-18\\
64.14	0\\
64.15	0\\
64.16	0\\
64.17	0\\
64.18	0\\
64.19	0\\
64.2	0\\
64.21	0\\
64.22	0\\
64.23	0\\
64.24	0\\
64.25	0\\
64.26	0\\
64.27	0\\
64.28	0\\
64.29	1.73472347597681e-18\\
64.3	0\\
64.31	0\\
64.32	0\\
64.33	0\\
64.34	1.73472347597681e-18\\
64.35	0\\
64.36	0\\
64.37	0\\
64.38	0\\
64.39	0\\
64.4	0\\
64.41	0\\
64.42	1.73472347597681e-18\\
64.43	0\\
64.44	0\\
64.45	0\\
64.46	1.73472347597681e-18\\
64.47	0\\
64.48	0\\
64.49	0\\
64.5	0\\
64.51	0\\
64.52	0\\
64.53	0\\
64.54	0\\
64.55	0\\
64.56	0\\
64.57	0\\
64.58	0\\
64.59	1.73472347597681e-18\\
64.6	0\\
64.61	0\\
64.62	0\\
64.63	0\\
64.64	0\\
64.65	1.73472347597681e-18\\
64.66	0\\
64.67	0\\
64.68	0\\
64.69	0\\
64.7	0\\
64.71	0\\
64.72	0\\
64.73	0\\
64.74	0\\
64.75	1.73472347597681e-18\\
64.76	0\\
64.77	0\\
64.78	0\\
64.79	0\\
64.8	0\\
64.81	0\\
64.82	0\\
64.83	1.73472347597681e-18\\
64.84	1.73472347597681e-18\\
64.85	0\\
64.86	0\\
64.87	1.73472347597681e-18\\
64.88	0\\
64.89	0\\
64.9	0\\
64.91	0\\
64.92	0\\
64.93	0\\
64.94	0\\
64.95	0\\
64.96	0\\
64.97	1.73472347597681e-18\\
64.98	0\\
64.99	1.73472347597681e-18\\
65	0\\
65.01	0\\
65.02	0\\
65.03	0\\
65.04	0\\
65.05	0\\
65.06	0\\
65.07	1.73472347597681e-18\\
65.08	0\\
65.09	0\\
65.1	0\\
65.11	0\\
65.12	0\\
65.13	0\\
65.14	0\\
65.15	0\\
65.16	0\\
65.17	0\\
65.18	1.73472347597681e-18\\
65.19	0\\
65.2	0\\
65.21	0\\
65.22	0\\
65.23	0\\
65.24	0\\
65.25	0\\
65.26	0\\
65.27	0\\
65.28	0\\
65.29	0\\
65.3	0\\
65.31	0\\
65.32	0\\
65.33	0\\
65.34	0\\
65.35	0\\
65.36	0\\
65.37	1.73472347597681e-18\\
65.38	0\\
65.39	0\\
65.4	0\\
65.41	0\\
65.42	0\\
65.43	0\\
65.44	0\\
65.45	0\\
65.46	0\\
65.47	0\\
65.48	0\\
65.49	0\\
65.5	0\\
65.51	0\\
65.52	0\\
65.53	0\\
65.54	0\\
65.55	0\\
65.56	0\\
65.57	0\\
65.58	0\\
65.59	0\\
65.6	0\\
65.61	0\\
65.62	0\\
65.63	0\\
65.64	0\\
65.65	1.73472347597681e-18\\
65.66	0\\
65.67	0\\
65.68	1.73472347597681e-18\\
65.69	0\\
65.7	0\\
65.71	0\\
65.72	0\\
65.73	0\\
65.74	0\\
65.75	0\\
65.76	0\\
65.77	0\\
65.78	0\\
65.79	0\\
65.8	1.73472347597681e-18\\
65.81	1.73472347597681e-18\\
65.82	0\\
65.83	1.73472347597681e-18\\
65.84	0\\
65.85	0\\
65.86	1.73472347597681e-18\\
65.87	0\\
65.88	0\\
65.89	0\\
65.9	0\\
65.91	0\\
65.92	1.73472347597681e-18\\
65.93	1.73472347597681e-18\\
65.94	0\\
65.95	0\\
65.96	0\\
65.97	0\\
65.98	0\\
65.99	0\\
66	0\\
66.01	0\\
66.02	0\\
66.03	0\\
66.04	0\\
66.05	0\\
66.06	1.73472347597681e-18\\
66.07	0\\
66.08	0\\
66.09	0\\
66.1	0\\
66.11	0\\
66.12	0\\
66.13	0\\
66.14	0\\
66.15	1.73472347597681e-18\\
66.16	0\\
66.17	0\\
66.18	0\\
66.19	0\\
66.2	0\\
66.21	0\\
66.22	0\\
66.23	1.73472347597681e-18\\
66.24	0\\
66.25	0\\
66.26	0\\
66.27	0\\
66.28	0\\
66.29	1.73472347597681e-18\\
66.3	0\\
66.31	0\\
66.32	0\\
66.33	0\\
66.34	0\\
66.35	1.73472347597681e-18\\
66.36	0\\
66.37	1.73472347597681e-18\\
66.38	0\\
66.39	0\\
66.4	0\\
66.41	0\\
66.42	0\\
66.43	0\\
66.44	0\\
66.45	0\\
66.46	0\\
66.47	0\\
66.48	0\\
66.49	1.73472347597681e-18\\
66.5	0\\
66.51	0\\
66.52	0\\
66.53	0\\
66.54	0\\
66.55	0\\
66.56	0\\
66.57	1.73472347597681e-18\\
66.58	0\\
66.59	0\\
66.6	0\\
66.61	0\\
66.62	0\\
66.63	0\\
66.64	0\\
66.65	1.73472347597681e-18\\
66.66	0\\
66.67	0\\
66.68	1.73472347597681e-18\\
66.69	0\\
66.7	0\\
66.71	0\\
66.72	0\\
66.73	0\\
66.74	0\\
66.75	0\\
66.76	0\\
66.77	0\\
66.78	0\\
66.79	0\\
66.8	0\\
66.81	0\\
66.82	0\\
66.83	0\\
66.84	0\\
66.85	0\\
66.86	0\\
66.87	0\\
66.88	0\\
66.89	0\\
66.9	0\\
66.91	1.73472347597681e-18\\
66.92	0\\
66.93	0\\
66.94	1.73472347597681e-18\\
66.95	0\\
66.96	0\\
66.97	1.73472347597681e-18\\
66.98	0\\
66.99	0\\
67	0\\
67.01	0\\
67.02	1.73472347597681e-18\\
67.03	0\\
67.04	0\\
67.05	1.73472347597681e-18\\
67.06	0\\
67.07	0\\
67.08	1.73472347597681e-18\\
67.09	0\\
67.1	0\\
67.11	0\\
67.12	0\\
67.13	0\\
67.14	0\\
67.15	0\\
67.16	0\\
67.17	0\\
67.18	0\\
67.19	0\\
67.2	0\\
67.21	0\\
67.22	0\\
67.23	0\\
67.24	0\\
67.25	0\\
67.26	0\\
67.27	0\\
67.28	0\\
67.29	0\\
67.3	0\\
67.31	0\\
67.32	0\\
67.33	0\\
67.34	0\\
67.35	0\\
67.36	1.73472347597681e-18\\
67.37	0\\
67.38	0\\
67.39	0\\
67.4	0\\
67.41	1.73472347597681e-18\\
67.42	0\\
67.43	0\\
67.44	0\\
67.45	0\\
67.46	1.73472347597681e-18\\
67.47	0\\
67.48	0\\
67.49	0\\
67.5	1.73472347597681e-18\\
67.51	0\\
67.52	0\\
67.53	0\\
67.54	0\\
67.55	0\\
67.56	0\\
67.57	0\\
67.58	0\\
67.59	0\\
67.6	0\\
67.61	1.73472347597681e-18\\
67.62	0\\
67.63	0\\
67.64	1.73472347597681e-18\\
67.65	0\\
67.66	0\\
67.67	0\\
67.68	0\\
67.69	0\\
67.7	0\\
67.71	0\\
67.72	0\\
67.73	0\\
67.74	0\\
67.75	0\\
67.76	0\\
67.77	0\\
67.78	0\\
67.79	0\\
67.8	0\\
67.81	0\\
67.82	0\\
67.83	0\\
67.84	0\\
67.85	1.73472347597681e-18\\
67.86	0\\
67.87	0\\
67.88	0\\
67.89	0\\
67.9	0\\
67.91	0\\
67.92	0\\
67.93	0\\
67.94	0\\
67.95	0\\
67.96	0\\
67.97	0\\
67.98	0\\
67.99	0\\
68	0\\
68.01	0\\
68.02	1.73472347597681e-18\\
68.03	0\\
68.04	0\\
68.05	0\\
68.06	0\\
68.07	0\\
68.08	0\\
68.09	1.73472347597681e-18\\
68.1	0\\
68.11	0\\
68.12	0\\
68.13	0\\
68.14	0\\
68.15	0\\
68.16	0\\
68.17	1.73472347597681e-18\\
68.18	0\\
68.19	1.73472347597681e-18\\
68.2	0\\
68.21	0\\
68.22	0\\
68.23	0\\
68.24	0\\
68.25	0\\
68.26	0\\
68.27	0\\
68.28	0\\
68.29	0\\
68.3	1.73472347597681e-18\\
68.31	0\\
68.32	0\\
68.33	0\\
68.34	0\\
68.35	0\\
68.36	1.73472347597681e-18\\
68.37	0\\
68.38	0\\
68.39	0\\
68.4	0\\
68.41	0\\
68.42	0\\
68.43	1.73472347597681e-18\\
68.44	0\\
68.45	0\\
68.46	0\\
68.47	0\\
68.48	0\\
68.49	0\\
68.5	0\\
68.51	0\\
68.52	0\\
68.53	0\\
68.54	0\\
68.55	0\\
68.56	0\\
68.57	0\\
68.58	0\\
68.59	0\\
68.6	0\\
68.61	0\\
68.62	0\\
68.63	0\\
68.64	0\\
68.65	0\\
68.66	0\\
68.67	0\\
68.68	0\\
68.69	0\\
68.7	0\\
68.71	1.73472347597681e-18\\
68.72	0\\
68.73	0\\
68.74	0\\
68.75	0\\
68.76	0\\
68.77	0\\
68.78	0\\
68.79	0\\
68.8	0\\
68.81	0\\
68.82	0\\
68.83	0\\
68.84	0\\
68.85	0\\
68.86	0\\
68.87	0\\
68.88	0\\
68.89	0\\
68.9	0\\
68.91	0\\
68.92	0\\
68.93	0\\
68.94	0\\
68.95	0\\
68.96	0\\
68.97	0\\
68.98	0\\
68.99	0\\
69	1.73472347597681e-18\\
69.01	0\\
69.02	0\\
69.03	0\\
69.04	0\\
69.05	1.73472347597681e-18\\
69.06	0\\
69.07	0\\
69.08	0\\
69.09	0\\
69.1	0\\
69.11	0\\
69.12	0\\
69.13	1.73472347597681e-18\\
69.14	0\\
69.15	1.73472347597681e-18\\
69.16	0\\
69.17	0\\
69.18	0\\
69.19	1.73472347597681e-18\\
69.2	0\\
69.21	1.73472347597681e-18\\
69.22	0\\
69.23	0\\
69.24	0\\
69.25	0\\
69.26	0\\
69.27	0\\
69.28	0\\
69.29	0\\
69.3	0\\
69.31	0\\
69.32	0\\
69.33	0\\
69.34	0\\
69.35	0\\
69.36	0\\
69.37	1.73472347597681e-18\\
69.38	0\\
69.39	0\\
69.4	0\\
69.41	0\\
69.42	0\\
69.43	1.73472347597681e-18\\
69.44	0\\
69.45	1.73472347597681e-18\\
69.46	0\\
69.47	0\\
69.48	0\\
69.49	0\\
69.5	0\\
69.51	0\\
69.52	0\\
69.53	0\\
69.54	0\\
69.55	0\\
69.56	0\\
69.57	1.73472347597681e-18\\
69.58	1.73472347597681e-18\\
69.59	0\\
69.6	0\\
69.61	0\\
69.62	0\\
69.63	0\\
69.64	1.73472347597681e-18\\
69.65	0\\
69.66	0\\
69.67	1.73472347597681e-18\\
69.68	0\\
69.69	0\\
69.7	0\\
69.71	0\\
69.72	0\\
69.73	0\\
69.74	0\\
69.75	0\\
69.76	0\\
69.77	0\\
69.78	0\\
69.79	0\\
69.8	0\\
69.81	0\\
69.82	0\\
69.83	0\\
69.84	0\\
69.85	0\\
69.86	0\\
69.87	0\\
69.88	1.73472347597681e-18\\
69.89	0\\
69.9	0\\
69.91	0\\
69.92	0\\
69.93	0\\
69.94	0\\
69.95	1.73472347597681e-18\\
69.96	0\\
69.97	0\\
69.98	0\\
69.99	0\\
70	0\\
70.01	0\\
70.02	0\\
70.03	0\\
70.04	0\\
70.05	0\\
70.06	0\\
70.07	0\\
70.08	0\\
70.09	0\\
70.1	1.73472347597681e-18\\
70.11	0\\
70.12	0\\
70.13	0\\
70.14	1.73472347597681e-18\\
70.15	0\\
70.16	0\\
70.17	0\\
70.18	0\\
70.19	0\\
70.2	0\\
70.21	0\\
70.22	0\\
70.23	0\\
70.24	0\\
70.25	0\\
70.26	0\\
70.27	0\\
70.28	0\\
70.29	0\\
70.3	0\\
70.31	0\\
70.32	0\\
70.33	1.73472347597681e-18\\
70.34	0\\
70.35	0\\
70.36	0\\
70.37	0\\
70.38	0\\
70.39	0\\
70.4	1.73472347597681e-18\\
70.41	0\\
70.42	0\\
70.43	0\\
70.44	0\\
70.45	0\\
70.46	1.73472347597681e-18\\
70.47	0\\
70.48	0\\
70.49	0\\
70.5	0\\
70.51	0\\
70.52	0\\
70.53	0\\
70.54	0\\
70.55	0\\
70.56	0\\
70.57	0\\
70.58	0\\
70.59	0\\
70.6	0\\
70.61	0\\
70.62	0\\
70.63	0\\
70.64	0\\
70.65	1.73472347597681e-18\\
70.66	0\\
70.67	0\\
70.68	0\\
70.69	0\\
70.7	0\\
70.71	1.73472347597681e-18\\
70.72	0\\
70.73	0\\
70.74	0\\
70.75	0\\
70.76	0\\
70.77	0\\
70.78	0\\
70.79	0\\
70.8	0\\
70.81	0\\
70.82	1.73472347597681e-18\\
70.83	0\\
70.84	0\\
70.85	1.73472347597681e-18\\
70.86	0\\
70.87	0\\
70.88	0\\
70.89	0\\
70.9	0\\
70.91	0\\
70.92	1.73472347597681e-18\\
70.93	0\\
70.94	0\\
70.95	0\\
70.96	0\\
70.97	0\\
70.98	0\\
70.99	0\\
71	0\\
71.01	0\\
71.02	0\\
71.03	0\\
71.04	0\\
71.05	0\\
71.06	0\\
71.07	0\\
71.08	1.73472347597681e-18\\
71.09	1.73472347597681e-18\\
71.1	0\\
71.11	0\\
71.12	0\\
71.13	0\\
71.14	0\\
71.15	0\\
71.16	0\\
71.17	0\\
71.18	1.73472347597681e-18\\
71.19	0\\
71.2	0\\
71.21	0\\
71.22	0\\
71.23	0\\
71.24	0\\
71.25	0\\
71.26	0\\
71.27	1.73472347597681e-18\\
71.28	0\\
71.29	0\\
71.3	0\\
71.31	0\\
71.32	1.73472347597681e-18\\
71.33	0\\
71.34	0\\
71.35	0\\
71.36	1.73472347597681e-18\\
71.37	0\\
71.38	0\\
71.39	0\\
71.4	0\\
71.41	0\\
71.42	0\\
71.43	1.73472347597681e-18\\
71.44	0\\
71.45	0\\
71.46	0\\
71.47	0\\
71.48	0\\
71.49	0\\
71.5	0\\
71.51	0\\
71.52	1.73472347597681e-18\\
71.53	0\\
71.54	0\\
71.55	0\\
71.56	0\\
71.57	0\\
71.58	0\\
71.59	0\\
71.6	0\\
71.61	1.73472347597681e-18\\
71.62	0\\
71.63	1.73472347597681e-18\\
71.64	0\\
71.65	0\\
71.66	0\\
71.67	0\\
71.68	0\\
71.69	0\\
71.7	0\\
71.71	1.73472347597681e-18\\
71.72	0\\
71.73	0\\
71.74	0\\
71.75	0\\
71.76	0\\
71.77	0\\
71.78	0\\
71.79	0\\
71.8	0\\
71.81	0\\
71.82	0\\
71.83	1.73472347597681e-18\\
71.84	0\\
71.85	0\\
71.86	0\\
71.87	0\\
71.88	0\\
71.89	1.73472347597681e-18\\
71.9	0\\
71.91	0\\
71.92	0\\
71.93	0\\
71.94	0\\
71.95	0\\
71.96	1.73472347597681e-18\\
71.97	0\\
71.98	0\\
71.99	0\\
72	0\\
72.01	0\\
72.02	0\\
72.03	0\\
72.04	0\\
72.05	0\\
72.06	0\\
72.07	0\\
72.08	0\\
72.09	0\\
72.1	0\\
72.11	0\\
72.12	1.73472347597681e-18\\
72.13	0\\
72.14	0\\
72.15	0\\
72.16	0\\
72.17	0\\
72.18	0\\
72.19	1.73472347597681e-18\\
72.2	0\\
72.21	0\\
72.22	0\\
72.23	0\\
72.24	0\\
72.25	0\\
72.26	0\\
72.27	0\\
72.28	0\\
72.29	0\\
72.3	0\\
72.31	0\\
72.32	0\\
72.33	0\\
72.34	0\\
72.35	0\\
72.36	0\\
72.37	0\\
72.38	0\\
72.39	0\\
72.4	0\\
72.41	0\\
72.42	0\\
72.43	0\\
72.44	0\\
72.45	1.73472347597681e-18\\
72.46	0\\
72.47	0\\
72.48	0\\
72.49	0\\
72.5	0\\
72.51	0\\
72.52	0\\
72.53	0\\
72.54	0\\
72.55	1.73472347597681e-18\\
72.56	0\\
72.57	0\\
72.58	0\\
72.59	0\\
72.6	0\\
72.61	0\\
72.62	1.73472347597681e-18\\
72.63	0\\
72.64	0\\
72.65	0\\
72.66	0\\
72.67	0\\
72.68	0\\
72.69	0\\
72.7	1.73472347597681e-18\\
72.71	0\\
72.72	0\\
72.73	1.73472347597681e-18\\
72.74	0\\
72.75	0\\
72.76	0\\
72.77	0\\
72.78	1.73472347597681e-18\\
72.79	0\\
72.8	0\\
72.81	0\\
72.82	0\\
72.83	0\\
72.84	0\\
72.85	1.73472347597681e-18\\
72.86	0\\
72.87	0\\
72.88	0\\
72.89	1.73472347597681e-18\\
72.9	0\\
72.91	0\\
72.92	0\\
72.93	0\\
72.94	0\\
72.95	0\\
72.96	0\\
72.97	0\\
72.98	0\\
72.99	0\\
73	0\\
73.01	0\\
73.02	1.73472347597681e-18\\
73.03	0\\
73.04	0\\
73.05	0\\
73.06	0\\
73.07	0\\
73.08	0\\
73.09	0\\
73.1	0\\
73.11	0\\
73.12	0\\
73.13	1.73472347597681e-18\\
73.14	0\\
73.15	0\\
73.16	0\\
73.17	0\\
73.18	0\\
73.19	0\\
73.2	0\\
73.21	0\\
73.22	0\\
73.23	1.73472347597681e-18\\
73.24	0\\
73.25	0\\
73.26	0\\
73.27	1.73472347597681e-18\\
73.28	0\\
73.29	0\\
73.3	0\\
73.31	0\\
73.32	0\\
73.33	0\\
73.34	0\\
73.35	0\\
73.36	0\\
73.37	0\\
73.38	0\\
73.39	1.73472347597681e-18\\
73.4	0\\
73.41	0\\
73.42	0\\
73.43	0\\
73.44	0\\
73.45	0\\
73.46	0\\
73.47	0\\
73.48	0\\
73.49	0\\
73.5	0\\
73.51	1.73472347597681e-18\\
73.52	1.73472347597681e-18\\
73.53	0\\
73.54	0\\
73.55	1.73472347597681e-18\\
73.56	0\\
73.57	0\\
73.58	0\\
73.59	0\\
73.6	0\\
73.61	0\\
73.62	0\\
73.63	0\\
73.64	0\\
73.65	0\\
73.66	1.73472347597681e-18\\
73.67	0\\
73.68	0\\
73.69	0\\
73.7	0\\
73.71	0\\
73.72	0\\
73.73	1.73472347597681e-18\\
73.74	0\\
73.75	0\\
73.76	1.73472347597681e-18\\
73.77	0\\
73.78	0\\
73.79	0\\
73.8	1.73472347597681e-18\\
73.81	0\\
73.82	0\\
73.83	0\\
73.84	0\\
73.85	0\\
73.86	0\\
73.87	0\\
73.88	0\\
73.89	0\\
73.9	0\\
73.91	0\\
73.92	0\\
73.93	0\\
73.94	0\\
73.95	0\\
73.96	0\\
73.97	0\\
73.98	1.73472347597681e-18\\
73.99	0\\
74	0\\
74.01	1.73472347597681e-18\\
74.02	0\\
74.03	0\\
74.04	0\\
74.05	0\\
74.06	0\\
74.07	0\\
74.08	0\\
74.09	0\\
74.1	1.73472347597681e-18\\
74.11	1.73472347597681e-18\\
74.12	1.73472347597681e-18\\
74.13	0\\
74.14	1.73472347597681e-18\\
74.15	0\\
74.16	0\\
74.17	1.73472347597681e-18\\
74.18	0\\
74.19	0\\
74.2	0\\
74.21	0\\
74.22	0\\
74.23	0\\
74.24	0\\
74.25	0\\
74.26	0\\
74.27	0\\
74.28	0\\
74.29	0\\
74.3	1.73472347597681e-18\\
74.31	0\\
74.32	0\\
74.33	0\\
74.34	0\\
74.35	0\\
74.36	0\\
74.37	1.73472347597681e-18\\
74.38	0\\
74.39	0\\
74.4	0\\
74.41	0\\
74.42	0\\
74.43	0\\
74.44	0\\
74.45	0\\
74.46	0\\
74.47	0\\
74.48	0\\
74.49	0\\
74.5	0\\
74.51	0\\
74.52	0\\
74.53	0\\
74.54	0\\
74.55	0\\
74.56	0\\
74.57	0\\
74.58	0\\
74.59	0\\
74.6	1.73472347597681e-18\\
74.61	0\\
74.62	1.73472347597681e-18\\
74.63	0\\
74.64	0\\
74.65	1.73472347597681e-18\\
74.66	0\\
74.67	0\\
74.68	0\\
74.69	0\\
74.7	1.73472347597681e-18\\
74.71	0\\
74.72	0\\
74.73	0\\
74.74	0\\
74.75	0\\
74.76	0\\
74.77	0\\
74.78	0\\
74.79	0\\
74.8	0\\
74.81	0\\
74.82	0\\
74.83	1.73472347597681e-18\\
74.84	0\\
74.85	0\\
74.86	0\\
74.87	1.73472347597681e-18\\
74.88	0\\
74.89	0\\
74.9	0\\
74.91	0\\
74.92	0\\
74.93	0\\
74.94	0\\
74.95	0\\
74.96	0\\
74.97	0\\
74.98	0\\
74.99	0\\
75	1.73472347597681e-18\\
75.01	0\\
75.02	0\\
75.03	1.73472347597681e-18\\
75.04	0\\
75.05	0\\
75.06	1.73472347597681e-18\\
75.07	0\\
75.08	0\\
75.09	0\\
75.1	0\\
75.11	0\\
75.12	0\\
75.13	0\\
75.14	1.73472347597681e-18\\
75.15	0\\
75.16	0\\
75.17	0\\
75.18	0\\
75.19	0\\
75.2	0\\
75.21	0\\
75.22	0\\
75.23	0\\
75.24	0\\
75.25	0\\
75.26	0\\
75.27	0\\
75.28	0\\
75.29	0\\
75.3	0\\
75.31	0\\
75.32	0\\
75.33	0\\
75.34	0\\
75.35	0\\
75.36	1.73472347597681e-18\\
75.37	0\\
75.38	0\\
75.39	0\\
75.4	0\\
75.41	0\\
75.42	0\\
75.43	0\\
75.44	1.73472347597681e-18\\
75.45	0\\
75.46	1.73472347597681e-18\\
75.47	0\\
75.48	0\\
75.49	0\\
75.5	0\\
75.51	0\\
75.52	0\\
75.53	0\\
75.54	0\\
75.55	0\\
75.56	1.73472347597681e-18\\
75.57	0\\
75.58	0\\
75.59	0\\
75.6	0\\
75.61	0\\
75.62	0\\
75.63	0\\
75.64	0\\
75.65	0\\
75.66	0\\
75.67	0\\
75.68	0\\
75.69	0\\
75.7	0\\
75.71	0\\
75.72	0\\
75.73	1.73472347597681e-18\\
75.74	0\\
75.75	0\\
75.76	0\\
75.77	0\\
75.78	0\\
75.79	0\\
75.8	0\\
75.81	0\\
75.82	0\\
75.83	0\\
75.84	0\\
75.85	0\\
75.86	0\\
75.87	0\\
75.88	0\\
75.89	0\\
75.9	0\\
75.91	0\\
75.92	0\\
75.93	1.73472347597681e-18\\
75.94	0\\
75.95	0\\
75.96	0\\
75.97	0\\
75.98	0\\
75.99	0\\
76	0\\
76.01	0\\
76.02	1.73472347597681e-18\\
76.03	0\\
76.04	0\\
76.05	0\\
76.06	0\\
76.07	0\\
76.08	0\\
76.09	0\\
76.1	0\\
76.11	0\\
76.12	0\\
76.13	1.73472347597681e-18\\
76.14	0\\
76.15	0\\
76.16	1.73472347597681e-18\\
76.17	1.73472347597681e-18\\
76.18	0\\
76.19	0\\
76.2	0\\
76.21	0\\
76.22	0\\
76.23	0\\
76.24	0\\
76.25	0\\
76.26	0\\
76.27	0\\
76.28	0\\
76.29	0\\
76.3	0\\
76.31	0\\
76.32	0\\
76.33	1.73472347597681e-18\\
76.34	0\\
76.35	0\\
76.36	0\\
76.37	0\\
76.38	1.73472347597681e-18\\
76.39	0\\
76.4	0\\
76.41	1.73472347597681e-18\\
76.42	1.73472347597681e-18\\
76.43	0\\
76.44	0\\
76.45	0\\
76.46	1.73472347597681e-18\\
76.47	0\\
76.48	0\\
76.49	0\\
76.5	0\\
76.51	0\\
76.52	1.73472347597681e-18\\
76.53	0\\
76.54	1.73472347597681e-18\\
76.55	1.73472347597681e-18\\
76.56	0\\
76.57	0\\
76.58	0\\
76.59	0\\
76.6	0\\
76.61	0\\
76.62	0\\
76.63	0\\
76.64	0\\
76.65	0\\
76.66	0\\
76.67	0\\
76.68	0\\
76.69	0\\
76.7	0\\
76.71	0\\
76.72	0\\
76.73	0\\
76.74	0\\
76.75	0\\
76.76	0\\
76.77	0\\
76.78	0\\
76.79	0\\
76.8	0\\
76.81	0\\
76.82	0\\
76.83	0\\
76.84	0\\
76.85	0\\
76.86	0\\
76.87	0\\
76.88	0\\
76.89	1.73472347597681e-18\\
76.9	0\\
76.91	0\\
76.92	0\\
76.93	0\\
76.94	1.73472347597681e-18\\
76.95	0\\
76.96	0\\
76.97	0\\
76.98	0\\
76.99	0\\
77	0\\
77.01	0\\
77.02	0\\
77.03	0\\
77.04	1.73472347597681e-18\\
77.05	0\\
77.06	0\\
77.07	0\\
77.08	0\\
77.09	0\\
77.1	0\\
77.11	0\\
77.12	0\\
77.13	0\\
77.14	0\\
77.15	0\\
77.16	1.73472347597681e-18\\
77.17	0\\
77.18	0\\
77.19	1.73472347597681e-18\\
77.2	0\\
77.21	0\\
77.22	0\\
77.23	0\\
77.24	0\\
77.25	0\\
77.26	0\\
77.27	0\\
77.28	0\\
77.29	0\\
77.3	0\\
77.31	0\\
77.32	0\\
77.33	0\\
77.34	0\\
77.35	0\\
77.36	0\\
77.37	0\\
77.38	0\\
77.39	0\\
77.4	0\\
77.41	1.73472347597681e-18\\
77.42	0\\
77.43	0\\
77.44	0\\
77.45	0\\
77.46	0\\
77.47	0\\
77.48	0\\
77.49	1.73472347597681e-18\\
77.5	0\\
77.51	0\\
77.52	0\\
77.53	0\\
77.54	1.73472347597681e-18\\
77.55	0\\
77.56	0\\
77.57	0\\
77.58	0\\
77.59	0\\
77.6	0\\
77.61	0\\
77.62	0\\
77.63	0\\
77.64	0\\
77.65	1.73472347597681e-18\\
77.66	0\\
77.67	0\\
77.68	0\\
77.69	0\\
77.7	0\\
77.71	0\\
77.72	0\\
77.73	0\\
77.74	0\\
77.75	1.73472347597681e-18\\
77.76	0\\
77.77	1.73472347597681e-18\\
77.78	0\\
77.79	1.73472347597681e-18\\
77.8	0\\
77.81	0\\
77.82	0\\
77.83	0\\
77.84	0\\
77.85	1.73472347597681e-18\\
77.86	0\\
77.87	0\\
77.88	0\\
77.89	0\\
77.9	0\\
77.91	0\\
77.92	0\\
77.93	0\\
77.94	0\\
77.95	0\\
77.96	0\\
77.97	0\\
77.98	0\\
77.99	0\\
78	0\\
78.01	0\\
78.02	0\\
78.03	0\\
78.04	1.73472347597681e-18\\
78.05	0\\
78.06	0\\
78.07	0\\
78.08	0\\
78.09	0\\
78.1	1.73472347597681e-18\\
78.11	0\\
78.12	0\\
78.13	0\\
78.14	0\\
78.15	0\\
78.16	0\\
78.17	0\\
78.18	0\\
78.19	0\\
78.2	1.73472347597681e-18\\
78.21	1.73472347597681e-18\\
78.22	0\\
78.23	0\\
78.24	0\\
78.25	1.73472347597681e-18\\
78.26	0\\
78.27	0\\
78.28	0\\
78.29	1.73472347597681e-18\\
78.3	0\\
78.31	0\\
78.32	0\\
78.33	0\\
78.34	0\\
78.35	0\\
78.36	0\\
78.37	0\\
78.38	0\\
78.39	0\\
78.4	1.73472347597681e-18\\
78.41	1.73472347597681e-18\\
78.42	0\\
78.43	0\\
78.44	0\\
78.45	0\\
78.46	0\\
78.47	0\\
78.48	0\\
78.49	0\\
78.5	0\\
78.51	0\\
78.52	0\\
78.53	0\\
78.54	1.73472347597681e-18\\
78.55	0\\
78.56	0\\
78.57	0\\
78.58	0\\
78.59	0\\
78.6	0\\
78.61	0\\
78.62	1.73472347597681e-18\\
78.63	0\\
78.64	0\\
78.65	0\\
78.66	0\\
78.67	0\\
78.68	0\\
78.69	0\\
78.7	0\\
78.71	0\\
78.72	0\\
78.73	0\\
78.74	0\\
78.75	0\\
78.76	0\\
78.77	0\\
78.78	0\\
78.79	0\\
78.8	0\\
78.81	0\\
78.82	0\\
78.83	0\\
78.84	0\\
78.85	0\\
78.86	0\\
78.87	1.73472347597681e-18\\
78.88	0\\
78.89	0\\
78.9	1.73472347597681e-18\\
78.91	0\\
78.92	1.73472347597681e-18\\
78.93	0\\
78.94	0\\
78.95	1.73472347597681e-18\\
78.96	0\\
78.97	0\\
78.98	1.73472347597681e-18\\
78.99	0\\
79	0\\
79.01	0\\
79.02	0\\
79.03	0\\
79.04	0\\
79.05	0\\
79.06	0\\
79.07	0\\
79.08	1.73472347597681e-18\\
79.09	0\\
79.1	0\\
79.11	0\\
79.12	0\\
79.13	0\\
79.14	0\\
79.15	0\\
79.16	0\\
79.17	0\\
79.18	0\\
79.19	0\\
79.2	0\\
79.21	0\\
79.22	0\\
79.23	0\\
79.24	0\\
79.25	0\\
79.26	0\\
79.27	0\\
79.28	0\\
79.29	0\\
79.3	1.73472347597681e-18\\
79.31	0\\
79.32	0\\
79.33	0\\
79.34	0\\
79.35	1.73472347597681e-18\\
79.36	0\\
79.37	0\\
79.38	0\\
79.39	0\\
79.4	0\\
79.41	0\\
79.42	0\\
79.43	0\\
79.44	0\\
79.45	0\\
79.46	1.73472347597681e-18\\
79.47	0\\
79.48	0\\
79.49	0\\
79.5	0\\
79.51	0\\
79.52	0\\
79.53	1.73472347597681e-18\\
79.54	0\\
79.55	0\\
79.56	1.73472347597681e-18\\
79.57	1.73472347597681e-18\\
79.58	0\\
79.59	1.73472347597681e-18\\
79.6	0\\
79.61	0\\
79.62	0\\
79.63	0\\
79.64	0\\
79.65	0\\
79.66	0\\
79.67	0\\
79.68	0\\
79.69	0\\
79.7	0\\
79.71	0\\
79.72	0\\
79.73	0\\
79.74	0\\
79.75	0\\
79.76	0\\
79.77	0\\
79.78	0\\
79.79	1.73472347597681e-18\\
79.8	0\\
79.81	0\\
79.82	1.73472347597681e-18\\
79.83	0\\
79.84	0\\
79.85	0\\
79.86	0\\
79.87	1.73472347597681e-18\\
79.88	0\\
79.89	0\\
79.9	0\\
79.91	0\\
79.92	0\\
79.93	0\\
79.94	1.73472347597681e-18\\
79.95	0\\
79.96	0\\
79.97	1.73472347597681e-18\\
79.98	1.73472347597681e-18\\
79.99	1.73472347597681e-18\\
80	0\\
80.01	0\\
};
\addplot [color=mycolor1,dashed]
  table[row sep=crcr]{%
80.01	0\\
80.02	0\\
80.03	1.73472347597681e-18\\
80.04	0\\
80.05	0\\
80.06	0\\
80.07	0\\
80.08	0\\
80.09	0\\
80.1	0\\
80.11	1.73472347597681e-18\\
80.12	0\\
80.13	1.73472347597681e-18\\
80.14	0\\
80.15	0\\
80.16	0\\
80.17	1.73472347597681e-18\\
80.18	1.73472347597681e-18\\
80.19	0\\
80.2	0\\
80.21	1.73472347597681e-18\\
80.22	0\\
80.23	1.73472347597681e-18\\
80.24	0\\
80.25	1.73472347597681e-18\\
80.26	0\\
80.27	0\\
80.28	0\\
80.29	0\\
80.3	0\\
80.31	0\\
80.32	0\\
80.33	0\\
80.34	0\\
80.35	0\\
80.36	0\\
80.37	1.73472347597681e-18\\
80.38	0\\
80.39	0\\
80.4	1.73472347597681e-18\\
80.41	0\\
80.42	0\\
80.43	1.73472347597681e-18\\
80.44	0\\
80.45	0\\
80.46	1.73472347597681e-18\\
80.47	0\\
80.48	0\\
80.49	0\\
80.5	0\\
80.51	0\\
80.52	0\\
80.53	0\\
80.54	0\\
80.55	0\\
80.56	1.73472347597681e-18\\
80.57	0\\
80.58	0\\
80.59	0\\
80.6	1.73472347597681e-18\\
80.61	0\\
80.62	0\\
80.63	0\\
80.64	0\\
80.65	0\\
80.66	0\\
80.67	0\\
80.68	0\\
80.69	1.73472347597681e-18\\
80.7	1.73472347597681e-18\\
80.71	0\\
80.72	0\\
80.73	1.73472347597681e-18\\
80.74	0\\
80.75	0\\
80.76	0\\
80.77	0\\
80.78	0\\
80.79	0\\
80.8	0\\
80.81	1.73472347597681e-18\\
80.82	0\\
80.83	0\\
80.84	0\\
80.85	0\\
80.86	0\\
80.87	0\\
80.88	0\\
80.89	0\\
80.9	1.73472347597681e-18\\
80.91	0\\
80.92	0\\
80.93	0\\
80.94	0\\
80.95	1.73472347597681e-18\\
80.96	0\\
80.97	0\\
80.98	0\\
80.99	0\\
81	1.73472347597681e-18\\
81.01	0\\
81.02	0\\
81.03	0\\
81.04	0\\
81.05	0\\
81.06	1.73472347597681e-18\\
81.07	1.73472347597681e-18\\
81.08	0\\
81.09	0\\
81.1	0\\
81.11	0\\
81.12	0\\
81.13	0\\
81.14	0\\
81.15	0\\
81.16	0\\
81.17	0\\
81.18	0\\
81.19	0\\
81.2	0\\
81.21	0\\
81.22	0\\
81.23	0\\
81.24	0\\
81.25	0\\
81.26	0\\
81.27	0\\
81.28	1.73472347597681e-18\\
81.29	1.73472347597681e-18\\
81.3	0\\
81.31	0\\
81.32	0\\
81.33	0\\
81.34	1.73472347597681e-18\\
81.35	0\\
81.36	0\\
81.37	0\\
81.38	0\\
81.39	0\\
81.4	1.73472347597681e-18\\
81.41	0\\
81.42	0\\
81.43	0\\
81.44	0\\
81.45	0\\
81.46	0\\
81.47	0\\
81.48	0\\
81.49	1.73472347597681e-18\\
81.5	0\\
81.51	0\\
81.52	0\\
81.53	0\\
81.54	1.73472347597681e-18\\
81.55	0\\
81.56	0\\
81.57	0\\
81.58	0\\
81.59	0\\
81.6	0\\
81.61	0\\
81.62	0\\
81.63	0\\
81.64	0\\
81.65	0\\
81.66	0\\
81.67	0\\
81.68	1.73472347597681e-18\\
81.69	0\\
81.7	0\\
81.71	1.73472347597681e-18\\
81.72	0\\
81.73	0\\
81.74	0\\
81.75	0\\
81.76	1.73472347597681e-18\\
81.77	0\\
81.78	0\\
81.79	1.73472347597681e-18\\
81.8	0\\
81.81	0\\
81.82	0\\
81.83	0\\
81.84	0\\
81.85	0\\
81.86	0\\
81.87	0\\
81.88	0\\
81.89	0\\
81.9	0\\
81.91	0\\
81.92	0\\
81.93	0\\
81.94	1.73472347597681e-18\\
81.95	0\\
81.96	1.73472347597681e-18\\
81.97	0\\
81.98	0\\
81.99	1.73472347597681e-18\\
82	0\\
82.01	1.73472347597681e-18\\
82.02	1.73472347597681e-18\\
82.03	1.73472347597681e-18\\
82.04	0\\
82.05	1.73472347597681e-18\\
82.06	0\\
82.07	1.73472347597681e-18\\
82.08	0\\
82.09	0\\
82.1	0\\
82.11	0\\
82.12	0\\
82.13	0\\
82.14	0\\
82.15	0\\
82.16	0\\
82.17	0\\
82.18	0\\
82.19	0\\
82.2	0\\
82.21	0\\
82.22	0\\
82.23	0\\
82.24	0\\
82.25	0\\
82.26	0\\
82.27	0\\
82.28	0\\
82.29	0\\
82.3	0\\
82.31	0\\
82.32	0\\
82.33	1.73472347597681e-18\\
82.34	0\\
82.35	1.73472347597681e-18\\
82.36	0\\
82.37	0\\
82.38	0\\
82.39	0\\
82.4	0\\
82.41	0\\
82.42	0\\
82.43	0\\
82.44	0\\
82.45	0\\
82.46	0\\
82.47	0\\
82.48	0\\
82.49	0\\
82.5	1.73472347597681e-18\\
82.51	0\\
82.52	0\\
82.53	0\\
82.54	0\\
82.55	0\\
82.56	0\\
82.57	0\\
82.58	0\\
82.59	0\\
82.6	0\\
82.61	0\\
82.62	0\\
82.63	1.73472347597681e-18\\
82.64	1.73472347597681e-18\\
82.65	1.73472347597681e-18\\
82.66	0\\
82.67	0\\
82.68	0\\
82.69	0\\
82.7	0\\
82.71	0\\
82.72	0\\
82.73	0\\
82.74	0\\
82.75	0\\
82.76	0\\
82.77	1.73472347597681e-18\\
82.78	0\\
82.79	0\\
82.8	0\\
82.81	0\\
82.82	0\\
82.83	0\\
82.84	0\\
82.85	1.73472347597681e-18\\
82.86	0\\
82.87	0\\
82.88	0\\
82.89	1.73472347597681e-18\\
82.9	0\\
82.91	0\\
82.92	0\\
82.93	0\\
82.94	0\\
82.95	0\\
82.96	0\\
82.97	0\\
82.98	0\\
82.99	0\\
83	0\\
83.01	0\\
83.02	0\\
83.03	0\\
83.04	0\\
83.05	1.73472347597681e-18\\
83.06	0\\
83.07	0\\
83.08	0\\
83.09	0\\
83.1	1.73472347597681e-18\\
83.11	1.73472347597681e-18\\
83.12	1.73472347597681e-18\\
83.13	0\\
83.14	0\\
83.15	0\\
83.16	0\\
83.17	0\\
83.18	0\\
83.19	0\\
83.2	1.73472347597681e-18\\
83.21	0\\
83.22	0\\
83.23	1.73472347597681e-18\\
83.24	0\\
83.25	0\\
83.26	1.73472347597681e-18\\
83.27	0\\
83.28	0\\
83.29	1.73472347597681e-18\\
83.3	0\\
83.31	0\\
83.32	0\\
83.33	0\\
83.34	0\\
83.35	0\\
83.36	0\\
83.37	0\\
83.38	0\\
83.39	0\\
83.4	0\\
83.41	0\\
83.42	0\\
83.43	0\\
83.44	0\\
83.45	0\\
83.46	0\\
83.47	1.73472347597681e-18\\
83.48	0\\
83.49	1.73472347597681e-18\\
83.5	0\\
83.51	0\\
83.52	0\\
83.53	0\\
83.54	0\\
83.55	0\\
83.56	0\\
83.57	0\\
83.58	0\\
83.59	0\\
83.6	1.73472347597681e-18\\
83.61	1.73472347597681e-18\\
83.62	0\\
83.63	1.73472347597681e-18\\
83.64	0\\
83.65	0\\
83.66	0\\
83.67	0\\
83.68	0\\
83.69	1.73472347597681e-18\\
83.7	0\\
83.71	0\\
83.72	0\\
83.73	0\\
83.74	0\\
83.75	0\\
83.76	0\\
83.77	0\\
83.78	0\\
83.79	1.73472347597681e-18\\
83.8	0\\
83.81	0\\
83.82	0\\
83.83	0\\
83.84	0\\
83.85	0\\
83.86	0\\
83.87	0\\
83.88	0\\
83.89	0\\
83.9	0\\
83.91	0\\
83.92	0\\
83.93	0\\
83.94	0\\
83.95	0\\
83.96	0\\
83.97	0\\
83.98	0\\
83.99	0\\
84	0\\
84.01	0\\
84.02	0\\
84.03	0\\
84.04	0\\
84.05	0\\
84.06	0\\
84.07	1.73472347597681e-18\\
84.08	0\\
84.09	0\\
84.1	0\\
84.11	1.73472347597681e-18\\
84.12	0\\
84.13	0\\
84.14	0\\
84.15	0\\
84.16	0\\
84.17	0\\
84.18	0\\
84.19	0\\
84.2	0\\
84.21	0\\
84.22	1.73472347597681e-18\\
84.23	0\\
84.24	0\\
84.25	0\\
84.26	1.73472347597681e-18\\
84.27	0\\
84.28	0\\
84.29	0\\
84.3	0\\
84.31	0\\
84.32	1.73472347597681e-18\\
84.33	0\\
84.34	0\\
84.35	1.73472347597681e-18\\
84.36	1.73472347597681e-18\\
84.37	0\\
84.38	0\\
84.39	0\\
84.4	0\\
84.41	1.73472347597681e-18\\
84.42	1.73472347597681e-18\\
84.43	0\\
84.44	0\\
84.45	0\\
84.46	0\\
84.47	0\\
84.48	0\\
84.49	0\\
84.5	1.73472347597681e-18\\
84.51	0\\
84.52	0\\
84.53	0\\
84.54	0\\
84.55	0\\
84.56	0\\
84.57	0\\
84.58	0\\
84.59	1.73472347597681e-18\\
84.6	0\\
84.61	0\\
84.62	0\\
84.63	0\\
84.64	0\\
84.65	0\\
84.66	1.73472347597681e-18\\
84.67	1.73472347597681e-18\\
84.68	0\\
84.69	0\\
84.7	0\\
84.71	0\\
84.72	0\\
84.73	0\\
84.74	0\\
84.75	0\\
84.76	1.73472347597681e-18\\
84.77	0\\
84.78	0\\
84.79	0\\
84.8	0\\
84.81	0\\
84.82	0\\
84.83	0\\
84.84	0\\
84.85	0\\
84.86	0\\
84.87	0\\
84.88	0\\
84.89	0\\
84.9	0\\
84.91	0\\
84.92	0\\
84.93	0\\
84.94	0\\
84.95	0\\
84.96	0\\
84.97	0\\
84.98	0\\
84.99	0\\
85	0\\
85.01	0\\
85.02	0\\
85.03	0\\
85.04	0\\
85.05	0\\
85.06	0\\
85.07	0\\
85.08	0\\
85.09	0\\
85.1	0\\
85.11	0\\
85.12	0\\
85.13	0\\
85.14	0\\
85.15	0\\
85.16	0\\
85.17	0\\
85.18	0\\
85.19	0\\
85.2	0\\
85.21	0\\
85.22	0\\
85.23	1.73472347597681e-18\\
85.24	0\\
85.25	0\\
85.26	1.73472347597681e-18\\
85.27	0\\
85.28	1.73472347597681e-18\\
85.29	0\\
85.3	0\\
85.31	1.73472347597681e-18\\
85.32	0\\
85.33	1.73472347597681e-18\\
85.34	0\\
85.35	0\\
85.36	0\\
85.37	0\\
85.38	0\\
85.39	0\\
85.4	0\\
85.41	0\\
85.42	0\\
85.43	0\\
85.44	0\\
85.45	0\\
85.46	0\\
85.47	0\\
85.48	0\\
85.49	0\\
85.5	0\\
85.51	0\\
85.52	0\\
85.53	1.73472347597681e-18\\
85.54	0\\
85.55	0\\
85.56	0\\
85.57	0\\
85.58	0\\
85.59	0\\
85.6	0\\
85.61	0\\
85.62	0\\
85.63	0\\
85.64	0\\
85.65	0\\
85.66	0\\
85.67	0\\
85.68	0\\
85.69	0\\
85.7	0\\
85.71	0\\
85.72	0\\
85.73	0\\
85.74	0\\
85.75	0\\
85.76	0\\
85.77	0\\
85.78	0\\
85.79	0\\
85.8	0\\
85.81	1.73472347597681e-18\\
85.82	0\\
85.83	0\\
85.84	0\\
85.85	0\\
85.86	0\\
85.87	0\\
85.88	0\\
85.89	0\\
85.9	0\\
85.91	0\\
85.92	0\\
85.93	0\\
85.94	0\\
85.95	0\\
85.96	0\\
85.97	1.73472347597681e-18\\
85.98	0\\
85.99	0\\
86	0\\
86.01	0\\
86.02	1.73472347597681e-18\\
86.03	1.73472347597681e-18\\
86.04	0\\
86.05	0\\
86.06	0\\
86.07	0\\
86.08	0\\
86.09	0\\
86.1	0\\
86.11	0\\
86.12	0\\
86.13	0\\
86.14	0\\
86.15	0\\
86.16	0\\
86.17	0\\
86.18	0\\
86.19	0\\
86.2	0\\
86.21	0\\
86.22	0\\
86.23	0\\
86.24	0\\
86.25	1.73472347597681e-18\\
86.26	1.73472347597681e-18\\
86.27	0\\
86.28	0\\
86.29	0\\
86.3	0\\
86.31	0\\
86.32	0\\
86.33	1.73472347597681e-18\\
86.34	0\\
86.35	0\\
86.36	0\\
86.37	0\\
86.38	0\\
86.39	0\\
86.4	0\\
86.41	0\\
86.42	0\\
86.43	0\\
86.44	0\\
86.45	0\\
86.46	0\\
86.47	0\\
86.48	0\\
86.49	0\\
86.5	0\\
86.51	0\\
86.52	0\\
86.53	0\\
86.54	0\\
86.55	1.73472347597681e-18\\
86.56	0\\
86.57	0\\
86.58	0\\
86.59	0\\
86.6	0\\
86.61	0\\
86.62	0\\
86.63	0\\
86.64	0\\
86.65	0\\
86.66	0\\
86.67	0\\
86.68	0\\
86.69	0\\
86.7	1.73472347597681e-18\\
86.71	1.73472347597681e-18\\
86.72	0\\
86.73	0\\
86.74	0\\
86.75	0\\
86.76	0\\
86.77	0\\
86.78	0\\
86.79	0\\
86.8	0\\
86.81	0\\
86.82	0\\
86.83	0\\
86.84	0\\
86.85	0\\
86.86	0\\
86.87	0\\
86.88	0\\
86.89	0\\
86.9	0\\
86.91	0\\
86.92	0\\
86.93	0\\
86.94	0\\
86.95	0\\
86.96	0\\
86.97	1.73472347597681e-18\\
86.98	0\\
86.99	0\\
87	0\\
87.01	0\\
87.02	0\\
87.03	0\\
87.04	0\\
87.05	0\\
87.06	0\\
87.07	0\\
87.08	0\\
87.09	0\\
87.1	0\\
87.11	0\\
87.12	0\\
87.13	0\\
87.14	0\\
87.15	0\\
87.16	0\\
87.17	0\\
87.18	0\\
87.19	0\\
87.2	0\\
87.21	0\\
87.22	0\\
87.23	0\\
87.24	0\\
87.25	0\\
87.26	0\\
87.27	1.73472347597681e-18\\
87.28	0\\
87.29	0\\
87.3	0\\
87.31	0\\
87.32	0\\
87.33	0\\
87.34	0\\
87.35	0\\
87.36	0\\
87.37	0\\
87.38	0\\
87.39	0\\
87.4	0\\
87.41	0\\
87.42	0\\
87.43	0\\
87.44	0\\
87.45	0\\
87.46	0\\
87.47	0\\
87.48	0\\
87.49	0\\
87.5	0\\
87.51	1.73472347597681e-18\\
87.52	0\\
87.53	0\\
87.54	0\\
87.55	0\\
87.56	0\\
87.57	0\\
87.58	0\\
87.59	0\\
87.6	0\\
87.61	0\\
87.62	1.73472347597681e-18\\
87.63	0\\
87.64	0\\
87.65	0\\
87.66	0\\
87.67	0\\
87.68	0\\
87.69	1.73472347597681e-18\\
87.7	1.73472347597681e-18\\
87.71	0\\
87.72	0\\
87.73	0\\
87.74	0\\
87.75	0\\
87.76	0\\
87.77	0\\
87.78	0\\
87.79	0\\
87.8	0\\
87.81	0\\
87.82	0\\
87.83	1.73472347597681e-18\\
87.84	0\\
87.85	0\\
87.86	0\\
87.87	0\\
87.88	0\\
87.89	0\\
87.9	0\\
87.91	0\\
87.92	0\\
87.93	0\\
87.94	0\\
87.95	0\\
87.96	0\\
87.97	0\\
87.98	0\\
87.99	0\\
88	0\\
88.01	0\\
88.02	0\\
88.03	0\\
88.04	0\\
88.05	0\\
88.06	0\\
88.07	0\\
88.08	0\\
88.09	0\\
88.1	0\\
88.11	0\\
88.12	0\\
88.13	0\\
88.14	0\\
88.15	1.73472347597681e-18\\
88.16	0\\
88.17	0\\
88.18	0\\
88.19	0\\
88.2	0\\
88.21	0\\
88.22	0\\
88.23	0\\
88.24	0\\
88.25	0\\
88.26	0\\
88.27	0\\
88.28	0\\
88.29	0\\
88.3	0\\
88.31	0\\
88.32	0\\
88.33	0\\
88.34	0\\
88.35	0\\
88.36	0\\
88.37	0\\
88.38	0\\
88.39	0\\
88.4	0\\
88.41	0\\
88.42	0\\
88.43	0\\
88.44	0\\
88.45	0\\
88.46	1.73472347597681e-18\\
88.47	0\\
88.48	0\\
88.49	0\\
88.5	0\\
88.51	0\\
88.52	0\\
88.53	0\\
88.54	0\\
88.55	0\\
88.56	0\\
88.57	0\\
88.58	0\\
88.59	0\\
88.6	0\\
88.61	0\\
88.62	0\\
88.63	0\\
88.64	0\\
88.65	0\\
88.66	0\\
88.67	0\\
88.68	0\\
88.69	0\\
88.7	0\\
88.71	0\\
88.72	0\\
88.73	0\\
88.74	0\\
88.75	0\\
88.76	0\\
88.77	0\\
88.78	0\\
88.79	0\\
88.8	0\\
88.81	0\\
88.82	0\\
88.83	0\\
88.84	0\\
88.85	0\\
88.86	0\\
88.87	0\\
88.88	0\\
88.89	0\\
88.9	0\\
88.91	0\\
88.92	0\\
88.93	0\\
88.94	0\\
88.95	0\\
88.96	0\\
88.97	0\\
88.98	0\\
88.99	0\\
89	0\\
89.01	0\\
89.02	0\\
89.03	0\\
89.04	0\\
89.05	0\\
89.06	0\\
89.07	0\\
89.08	0\\
89.09	0\\
89.1	0\\
89.11	0\\
89.12	0\\
89.13	0\\
89.14	0\\
89.15	0\\
89.16	0\\
89.17	0\\
89.18	0\\
89.19	0\\
89.2	0\\
89.21	0\\
89.22	0\\
89.23	0\\
89.24	0\\
89.25	0\\
89.26	0\\
89.27	0\\
89.28	0\\
89.29	0\\
89.3	0\\
89.31	0\\
89.32	0\\
89.33	0\\
89.34	0\\
89.35	0\\
89.36	0\\
89.37	0\\
89.38	1.73472347597681e-18\\
89.39	0\\
89.4	0\\
89.41	0\\
89.42	0\\
89.43	0\\
89.44	0\\
89.45	0\\
89.46	0\\
89.47	0\\
89.48	0\\
89.49	0\\
89.5	0\\
89.51	0\\
89.52	0\\
89.53	0\\
89.54	0\\
89.55	0\\
89.56	0\\
89.57	0\\
89.58	0\\
89.59	0\\
89.6	0\\
89.61	0\\
89.62	0\\
89.63	0\\
89.64	0\\
89.65	1.73472347597681e-18\\
89.66	0\\
89.67	0\\
89.68	1.73472347597681e-18\\
89.69	0\\
89.7	0\\
89.71	0\\
89.72	0\\
89.73	0\\
89.74	0\\
89.75	0\\
89.76	0\\
89.77	0\\
89.78	0\\
89.79	0\\
89.8	0\\
89.81	0\\
89.82	0\\
89.83	0\\
89.84	0\\
89.85	0\\
89.86	0\\
89.87	0\\
89.88	0\\
89.89	0\\
89.9	0\\
89.91	0\\
89.92	0\\
89.93	0\\
89.94	0\\
89.95	0\\
89.96	0\\
89.97	0\\
89.98	0\\
89.99	0\\
90	0\\
90.01	0\\
90.02	0\\
90.03	0\\
90.04	0\\
90.05	0\\
90.06	1.73472347597681e-18\\
90.07	0\\
90.08	0\\
90.09	0\\
90.1	0\\
90.11	0\\
90.12	0\\
90.13	0\\
90.14	0\\
90.15	0\\
90.16	0\\
90.17	0\\
90.18	1.73472347597681e-18\\
90.19	0\\
90.2	0\\
90.21	0\\
90.22	0\\
90.23	0\\
90.24	0\\
90.25	0\\
90.26	0\\
90.27	0\\
90.28	0\\
90.29	0\\
90.3	0\\
90.31	0\\
90.32	0\\
90.33	1.73472347597681e-18\\
90.34	0\\
90.35	0\\
90.36	0\\
90.37	0\\
90.38	0\\
90.39	0\\
90.4	0\\
90.41	0\\
90.42	0\\
90.43	0\\
90.44	0\\
90.45	0\\
90.46	0\\
90.47	0\\
90.48	0\\
90.49	0\\
90.5	0\\
90.51	0\\
90.52	0\\
90.53	0\\
90.54	0\\
90.55	0\\
90.56	0\\
90.57	0\\
90.58	0\\
90.59	0\\
90.6	0\\
90.61	0\\
90.62	0\\
90.63	0\\
90.64	0\\
90.65	0\\
90.66	0\\
90.67	0\\
90.68	0\\
90.69	0\\
90.7	0\\
90.71	0\\
90.72	0\\
90.73	0\\
90.74	0\\
90.75	0\\
90.76	0\\
90.77	0\\
90.78	0\\
90.79	0\\
90.8	0\\
90.81	0\\
90.82	0\\
90.83	0\\
90.84	0\\
90.85	0\\
90.86	0\\
90.87	0\\
90.88	0\\
90.89	0\\
90.9	0\\
90.91	0\\
90.92	0\\
90.93	0\\
90.94	0\\
90.95	0\\
90.96	0\\
90.97	0\\
90.98	0\\
90.99	0\\
91	0\\
91.01	0\\
91.02	0\\
91.03	0\\
91.04	0\\
91.05	0\\
91.06	0\\
91.07	0\\
91.08	0\\
91.09	0\\
91.1	0\\
91.11	0\\
91.12	0\\
91.13	0\\
91.14	0\\
91.15	0\\
91.16	0\\
91.17	0\\
91.18	0\\
91.19	0\\
91.2	0\\
91.21	0\\
91.22	0\\
91.23	0\\
91.24	0\\
91.25	0\\
91.26	0\\
91.27	0\\
91.28	0\\
91.29	0\\
91.3	0\\
91.31	0\\
91.32	0\\
91.33	0\\
91.34	0\\
91.35	0\\
91.36	0\\
91.37	0\\
91.38	0\\
91.39	0\\
91.4	0\\
91.41	0\\
91.42	0\\
91.43	0\\
91.44	0\\
91.45	0\\
91.46	0\\
91.47	0\\
91.48	0\\
91.49	0\\
91.5	0\\
91.51	0\\
91.52	0\\
91.53	0\\
91.54	0\\
91.55	0\\
91.56	0\\
91.57	0\\
91.58	0\\
91.59	0\\
91.6	0\\
91.61	0\\
91.62	0\\
91.63	0\\
91.64	0\\
91.65	0\\
91.66	0\\
91.67	0\\
91.68	0\\
91.69	0\\
91.7	0\\
91.71	0\\
91.72	0\\
91.73	0\\
91.74	0\\
91.75	0\\
91.76	0\\
91.77	0\\
91.78	0\\
91.79	0\\
91.8	0\\
91.81	0\\
91.82	0\\
91.83	0\\
91.84	0\\
91.85	0\\
91.86	0\\
91.87	0\\
91.88	0\\
91.89	0\\
91.9	0\\
91.91	0\\
91.92	0\\
91.93	0\\
91.94	0\\
91.95	0\\
91.96	0\\
91.97	0\\
91.98	0\\
91.99	0\\
92	0\\
92.01	0\\
92.02	0\\
92.03	0\\
92.04	0\\
92.05	0\\
92.06	0\\
92.07	0\\
92.08	0\\
92.09	0\\
92.1	0\\
92.11	0\\
92.12	0\\
92.13	0\\
92.14	0\\
92.15	0\\
92.16	0\\
92.17	0\\
92.18	0\\
92.19	0\\
92.2	0\\
92.21	0\\
92.22	0\\
92.23	0\\
92.24	0\\
92.25	0\\
92.26	0\\
92.27	0\\
92.28	0\\
92.29	0\\
92.3	0\\
92.31	0\\
92.32	0\\
92.33	0\\
92.34	0\\
92.35	0\\
92.36	0\\
92.37	0\\
92.38	0\\
92.39	0\\
92.4	0\\
92.41	0\\
92.42	0\\
92.43	0\\
92.44	0\\
92.45	0\\
92.46	0\\
92.47	0\\
92.48	0\\
92.49	0\\
92.5	0\\
92.51	0\\
92.52	0\\
92.53	0\\
92.54	0\\
92.55	0\\
92.56	0\\
92.57	0\\
92.58	0\\
92.59	0\\
92.6	0\\
92.61	0\\
92.62	0\\
92.63	0\\
92.64	0\\
92.65	0\\
92.66	0\\
92.67	0\\
92.68	0\\
92.69	0\\
92.7	0\\
92.71	0\\
92.72	0\\
92.73	0\\
92.74	0\\
92.75	0\\
92.76	0\\
92.77	0\\
92.78	0\\
92.79	0\\
92.8	0\\
92.81	0\\
92.82	0\\
92.83	0\\
92.84	0\\
92.85	0\\
92.86	0\\
92.87	0\\
92.88	0\\
92.89	0\\
92.9	0\\
92.91	0\\
92.92	0\\
92.93	0\\
92.94	0\\
92.95	0\\
92.96	0\\
92.97	0\\
92.98	0\\
92.99	0\\
93	0\\
93.01	0\\
93.02	0\\
93.03	0\\
93.04	0\\
93.05	0\\
93.06	0\\
93.07	0\\
93.08	0\\
93.09	0\\
93.1	0\\
93.11	0\\
93.12	0\\
93.13	0\\
93.14	0\\
93.15	0\\
93.16	0\\
93.17	0\\
93.18	0\\
93.19	0\\
93.2	0\\
93.21	0\\
93.22	0\\
93.23	0\\
93.24	0\\
93.25	0\\
93.26	0\\
93.27	0\\
93.28	0\\
93.29	0\\
93.3	0\\
93.31	0\\
93.32	0\\
93.33	0\\
93.34	0\\
93.35	0\\
93.36	0\\
93.37	0\\
93.38	0\\
93.39	0\\
93.4	0\\
93.41	0\\
93.42	0\\
93.43	0\\
93.44	0\\
93.45	0\\
93.46	0\\
93.47	0\\
93.48	0\\
93.49	0\\
93.5	0\\
93.51	0\\
93.52	0\\
93.53	0\\
93.54	0\\
93.55	0\\
93.56	0\\
93.57	0\\
93.58	0\\
93.59	0\\
93.6	0\\
93.61	0\\
93.62	0\\
93.63	0\\
93.64	0\\
93.65	0\\
93.66	0\\
93.67	0\\
93.68	0\\
93.69	0\\
93.7	0\\
93.71	0\\
93.72	0\\
93.73	0\\
93.74	0\\
93.75	0\\
93.76	0\\
93.77	0\\
93.78	0\\
93.79	0\\
93.8	0\\
93.81	0\\
93.82	0\\
93.83	0\\
93.84	0\\
93.85	0\\
93.86	0\\
93.87	0\\
93.88	0\\
93.89	0\\
93.9	0\\
93.91	0\\
93.92	0\\
93.93	0\\
93.94	0\\
93.95	0\\
93.96	0\\
93.97	0\\
93.98	0\\
93.99	0\\
94	0\\
94.01	0\\
94.02	0\\
94.03	0\\
94.04	0\\
94.05	0\\
94.06	0\\
94.07	0\\
94.08	0\\
94.09	0\\
94.1	0\\
94.11	0\\
94.12	0\\
94.13	0\\
94.14	0\\
94.15	0\\
94.16	0\\
94.17	0\\
94.18	0\\
94.19	0\\
94.2	0\\
94.21	0\\
94.22	0\\
94.23	0\\
94.24	0\\
94.25	0\\
94.26	0\\
94.27	0\\
94.28	0\\
94.29	0\\
94.3	0\\
94.31	0\\
94.32	0\\
94.33	0\\
94.34	0\\
94.35	0\\
94.36	0\\
94.37	0\\
94.38	0\\
94.39	0\\
94.4	0\\
94.41	0\\
94.42	0\\
94.43	0\\
94.44	0\\
94.45	0\\
94.46	0\\
94.47	0\\
94.48	0\\
94.49	0\\
94.5	0\\
94.51	0\\
94.52	0\\
94.53	0\\
94.54	0\\
94.55	0\\
94.56	0\\
94.57	0\\
94.58	0\\
94.59	0\\
94.6	0\\
94.61	0\\
94.62	0\\
94.63	0\\
94.64	0\\
94.65	0\\
94.66	0\\
94.67	0\\
94.68	0\\
94.69	0\\
94.7	0\\
94.71	0\\
94.72	0\\
94.73	0\\
94.74	0\\
94.75	0\\
94.76	0\\
94.77	0\\
94.78	0\\
94.79	0\\
94.8	0\\
94.81	0\\
94.82	0\\
94.83	0\\
94.84	0\\
94.85	0\\
94.86	0\\
94.87	0\\
94.88	0\\
94.89	0\\
94.9	0\\
94.91	0\\
94.92	0\\
94.93	0\\
94.94	0\\
94.95	0\\
94.96	0\\
94.97	0\\
94.98	0\\
94.99	0\\
95	0\\
95.01	0\\
95.02	0\\
95.03	0\\
95.04	0\\
95.05	0\\
95.06	0\\
95.07	0\\
95.08	0\\
95.09	0\\
95.1	0\\
95.11	0\\
95.12	0\\
95.13	0\\
95.14	0\\
95.15	0\\
95.16	0\\
95.17	0\\
95.18	0\\
95.19	0\\
95.2	0\\
95.21	0\\
95.22	0\\
95.23	0\\
95.24	0\\
95.25	0\\
95.26	0\\
95.27	0\\
95.28	0\\
95.29	0\\
95.3	0\\
95.31	0\\
95.32	0\\
95.33	0\\
95.34	0\\
95.35	0\\
95.36	0\\
95.37	0\\
95.38	0\\
95.39	0\\
95.4	0\\
95.41	0\\
95.42	0\\
95.43	0\\
95.44	0\\
95.45	0\\
95.46	0\\
95.47	0\\
95.48	0\\
95.49	0\\
95.5	0\\
95.51	0\\
95.52	0\\
95.53	0\\
95.54	0\\
95.55	0\\
95.56	0\\
95.57	0\\
95.58	0\\
95.59	0\\
95.6	0\\
95.61	0\\
95.62	0\\
95.63	0\\
95.64	0\\
95.65	0\\
95.66	0\\
95.67	0\\
95.68	0\\
95.69	0\\
95.7	0\\
95.71	0\\
95.72	0\\
95.73	0\\
95.74	0\\
95.75	0\\
95.76	0\\
95.77	0\\
95.78	0\\
95.79	0\\
95.8	0\\
95.81	0\\
95.82	0\\
95.83	0\\
95.84	0\\
95.85	0\\
95.86	0\\
95.87	0\\
95.88	0\\
95.89	0\\
95.9	0\\
95.91	0\\
95.92	0\\
95.93	0\\
95.94	0\\
95.95	0\\
95.96	0\\
95.97	0\\
95.98	0\\
95.99	0\\
96	0\\
96.01	0\\
96.02	0\\
96.03	0\\
96.04	0\\
96.05	0\\
96.06	0\\
96.07	0\\
96.08	0\\
96.09	0\\
96.1	0\\
96.11	0\\
96.12	0\\
96.13	0\\
96.14	0\\
96.15	0\\
96.16	0\\
96.17	0\\
96.18	0\\
96.19	0\\
96.2	0\\
96.21	0\\
96.22	0\\
96.23	0\\
96.24	0\\
96.25	0\\
96.26	0\\
96.27	0\\
96.28	0\\
96.29	0\\
96.3	0\\
96.31	0\\
96.32	0\\
96.33	0\\
96.34	0\\
96.35	0\\
96.36	0\\
96.37	0\\
96.38	0\\
96.39	0\\
96.4	0\\
96.41	0\\
96.42	0\\
96.43	0\\
96.44	0\\
96.45	0\\
96.46	0\\
96.47	0\\
96.48	0\\
96.49	0\\
96.5	0\\
96.51	0\\
96.52	0\\
96.53	0\\
96.54	0\\
96.55	0\\
96.56	0\\
96.57	0\\
96.58	0\\
96.59	0\\
96.6	0\\
96.61	0\\
96.62	0\\
96.63	0\\
96.64	0\\
96.65	0\\
96.66	0\\
96.67	0\\
96.68	0\\
96.69	0\\
96.7	0\\
96.71	0\\
96.72	0\\
96.73	0\\
96.74	0\\
96.75	0\\
96.76	0\\
96.77	0\\
96.78	0\\
96.79	0\\
96.8	0\\
96.81	0\\
96.82	0\\
96.83	0\\
96.84	0\\
96.85	0\\
96.86	0\\
96.87	0\\
96.88	0\\
96.89	0\\
96.9	0\\
96.91	0\\
96.92	0\\
96.93	0\\
96.94	0\\
96.95	0\\
96.96	0\\
96.97	0\\
96.98	0\\
96.99	0\\
97	0\\
97.01	0\\
97.02	0\\
97.03	0\\
97.04	0\\
97.05	0\\
97.06	0\\
97.07	0\\
97.08	0\\
97.09	0\\
97.1	0\\
97.11	0\\
97.12	0\\
97.13	0\\
97.14	0\\
97.15	0\\
97.16	0\\
97.17	0\\
97.18	0\\
97.19	0\\
97.2	0\\
97.21	0\\
97.22	0\\
97.23	0\\
97.24	0\\
97.25	0\\
97.26	0\\
97.27	0\\
97.28	0\\
97.29	0\\
97.3	0\\
97.31	0\\
97.32	0\\
97.33	0\\
97.34	0\\
97.35	0\\
97.36	0\\
97.37	0\\
97.38	0\\
97.39	0\\
97.4	0\\
97.41	0\\
97.42	0\\
97.43	0\\
97.44	0\\
97.45	0\\
97.46	0\\
97.47	0\\
97.48	0\\
97.49	0\\
97.5	0\\
97.51	0\\
97.52	0\\
97.53	0\\
97.54	0\\
97.55	0\\
97.56	0\\
97.57	0\\
97.58	0\\
97.59	0\\
97.6	0\\
97.61	0\\
97.62	0\\
97.63	0\\
97.64	0\\
97.65	0\\
97.66	0\\
97.67	0\\
97.68	0\\
97.69	0\\
97.7	0\\
97.71	0\\
97.72	0\\
97.73	0\\
97.74	0\\
97.75	0\\
97.76	0\\
97.77	0\\
97.78	0\\
97.79	0\\
97.8	0\\
97.81	0\\
97.82	0\\
97.83	0\\
97.84	0\\
97.85	0\\
97.86	0\\
97.87	0\\
97.88	0\\
97.89	0\\
97.9	0\\
97.91	0\\
97.92	0\\
97.93	0\\
97.94	0\\
97.95	0\\
97.96	0\\
97.97	0\\
97.98	0\\
97.99	0\\
98	0\\
98.01	0\\
98.02	0\\
98.03	0\\
98.04	0\\
98.05	0\\
98.06	0\\
98.07	0\\
98.08	0\\
98.09	0\\
98.1	0\\
98.11	0\\
98.12	0\\
98.13	0\\
98.14	0\\
98.15	0\\
98.16	0\\
98.17	0\\
98.18	0\\
98.19	0\\
98.2	0\\
98.21	0\\
98.22	0\\
98.23	0\\
98.24	0\\
98.25	0\\
98.26	0\\
98.27	0\\
98.28	0\\
98.29	0\\
98.3	0\\
98.31	0\\
98.32	0\\
98.33	0\\
98.34	0\\
98.35	0\\
98.36	0\\
98.37	0\\
98.38	0\\
98.39	0\\
98.4	0\\
98.41	0\\
98.42	0\\
98.43	0\\
98.44	0\\
98.45	0\\
98.46	0\\
98.47	0\\
98.48	0\\
98.49	0\\
98.5	0\\
98.51	0\\
98.52	0\\
98.53	0\\
98.54	0\\
98.55	0\\
98.56	0\\
98.57	0\\
98.58	0\\
98.59	0\\
98.6	0\\
98.61	0\\
98.62	0\\
98.63	0\\
98.64	0\\
98.65	0\\
98.66	0\\
98.67	0\\
98.68	0\\
98.69	0\\
98.7	0\\
98.71	0\\
98.72	0\\
98.73	0\\
98.74	0\\
98.75	0\\
98.76	0\\
98.77	0\\
98.78	0\\
98.79	0\\
98.8	0\\
98.81	0\\
98.82	0\\
98.83	0\\
98.84	0\\
98.85	0\\
98.86	0\\
98.87	0\\
98.88	0\\
98.89	0\\
98.9	0\\
98.91	0\\
98.92	0\\
98.93	0\\
98.94	0\\
98.95	0\\
98.96	0\\
98.97	0\\
98.98	0\\
98.99	0\\
99	0\\
99.01	0\\
99.02	0\\
99.03	0\\
99.04	0\\
99.05	0\\
99.06	0\\
99.07	0\\
99.08	0\\
99.09	0\\
99.1	0\\
99.11	0\\
99.12	0\\
99.13	0\\
99.14	0\\
99.15	0\\
99.16	0\\
99.17	0\\
99.18	0\\
99.19	0\\
99.2	0\\
99.21	0\\
99.22	0\\
99.23	0\\
99.24	0\\
99.25	0\\
99.26	0\\
99.27	0\\
99.28	0\\
99.29	0\\
99.3	0\\
99.31	0\\
99.32	0\\
99.33	0\\
99.34	0\\
99.35	0\\
99.36	0\\
99.37	0\\
99.38	0\\
99.39	0\\
99.4	0\\
99.41	0\\
99.42	0\\
99.43	0\\
99.44	0\\
99.45	0\\
99.46	0\\
99.47	0\\
99.48	0\\
99.49	0\\
99.5	0\\
99.51	0\\
99.52	0\\
99.53	0\\
99.54	0\\
99.55	0\\
99.56	0\\
99.57	0\\
99.58	0\\
99.59	0\\
99.6	0\\
99.61	0\\
99.62	0\\
99.63	0\\
99.64	0\\
99.65	0\\
99.66	0\\
99.67	0\\
99.68	0\\
99.69	0\\
99.7	0\\
99.71	0\\
99.72	0\\
99.73	0\\
99.74	0\\
99.75	0\\
99.76	0\\
99.77	0\\
99.78	0\\
99.79	0\\
99.8	0\\
99.81	0\\
99.82	0\\
99.83	0\\
99.84	0\\
99.85	0\\
99.86	0\\
99.87	0\\
99.88	0\\
99.89	0\\
99.9	0\\
99.91	0\\
99.92	0\\
99.93	0\\
99.94	0\\
99.95	0\\
99.96	0\\
99.97	0\\
99.98	0\\
99.99	0\\
100	0\\
};
\addlegendentry{$q=-3$};

\addplot [color=red,dashed,forget plot]
  table[row sep=crcr]{%
0.01	0\\
0.02	0\\
0.03	0\\
0.04	0\\
0.05	0\\
0.06	0\\
0.07	0\\
0.08	0\\
0.09	0\\
0.1	0\\
0.11	0\\
0.12	0\\
0.13	0\\
0.14	0\\
0.15	0\\
0.16	0\\
0.17	1.73472347597681e-18\\
0.18	1.73472347597681e-18\\
0.19	0\\
0.2	0\\
0.21	0\\
0.22	0\\
0.23	0\\
0.24	0\\
0.25	0\\
0.26	0\\
0.27	0\\
0.28	0\\
0.29	0\\
0.3	0\\
0.31	0\\
0.32	0\\
0.33	0\\
0.34	1.73472347597681e-18\\
0.35	1.73472347597681e-18\\
0.36	0\\
0.37	0\\
0.38	0\\
0.39	0\\
0.4	0\\
0.41	0\\
0.42	0\\
0.43	0\\
0.44	0\\
0.45	0\\
0.46	0\\
0.47	0\\
0.48	0\\
0.49	0\\
0.5	0\\
0.51	1.73472347597681e-18\\
0.52	1.73472347597681e-18\\
0.53	0\\
0.54	0\\
0.55	1.73472347597681e-18\\
0.56	0\\
0.57	0\\
0.58	0\\
0.59	0\\
0.6	1.73472347597681e-18\\
0.61	0\\
0.62	0\\
0.63	1.73472347597681e-18\\
0.64	1.73472347597681e-18\\
0.65	0\\
0.66	0\\
0.67	1.73472347597681e-18\\
0.68	0\\
0.69	0\\
0.7	0\\
0.71	0\\
0.72	0\\
0.73	0\\
0.74	0\\
0.75	0\\
0.76	0\\
0.77	1.73472347597681e-18\\
0.78	0\\
0.79	0\\
0.8	0\\
0.81	0\\
0.82	0\\
0.83	0\\
0.84	0\\
0.85	0\\
0.86	0\\
0.87	0\\
0.88	0\\
0.89	0\\
0.9	0\\
0.91	0\\
0.92	0\\
0.93	0\\
0.94	0\\
0.95	0\\
0.96	0\\
0.97	0\\
0.98	0\\
0.99	1.73472347597681e-18\\
1	0\\
1.01	1.73472347597681e-18\\
1.02	0\\
1.03	0\\
1.04	1.73472347597681e-18\\
1.05	0\\
1.06	0\\
1.07	0\\
1.08	0\\
1.09	0\\
1.1	0\\
1.11	0\\
1.12	0\\
1.13	1.73472347597681e-18\\
1.14	0\\
1.15	0\\
1.16	0\\
1.17	0\\
1.18	0\\
1.19	1.73472347597681e-18\\
1.2	0\\
1.21	0\\
1.22	1.73472347597681e-18\\
1.23	1.73472347597681e-18\\
1.24	0\\
1.25	1.73472347597681e-18\\
1.26	0\\
1.27	0\\
1.28	0\\
1.29	0\\
1.3	0\\
1.31	0\\
1.32	0\\
1.33	0\\
1.34	0\\
1.35	0\\
1.36	0\\
1.37	0\\
1.38	0\\
1.39	0\\
1.4	0\\
1.41	0\\
1.42	0\\
1.43	0\\
1.44	0\\
1.45	0\\
1.46	0\\
1.47	0\\
1.48	0\\
1.49	0\\
1.5	0\\
1.51	0\\
1.52	0\\
1.53	0\\
1.54	0\\
1.55	0\\
1.56	0\\
1.57	0\\
1.58	0\\
1.59	0\\
1.6	0\\
1.61	0\\
1.62	0\\
1.63	0\\
1.64	0\\
1.65	0\\
1.66	0\\
1.67	0\\
1.68	0\\
1.69	0\\
1.7	0\\
1.71	0\\
1.72	0\\
1.73	0\\
1.74	0\\
1.75	0\\
1.76	0\\
1.77	0\\
1.78	0\\
1.79	0\\
1.8	1.73472347597681e-18\\
1.81	0\\
1.82	0\\
1.83	0\\
1.84	0\\
1.85	0\\
1.86	0\\
1.87	0\\
1.88	0\\
1.89	0\\
1.9	0\\
1.91	0\\
1.92	0\\
1.93	0\\
1.94	0\\
1.95	0\\
1.96	1.73472347597681e-18\\
1.97	0\\
1.98	0\\
1.99	0\\
2	0\\
2.01	0\\
2.02	0\\
2.03	1.73472347597681e-18\\
2.04	0\\
2.05	1.73472347597681e-18\\
2.06	1.73472347597681e-18\\
2.07	0\\
2.08	0\\
2.09	1.73472347597681e-18\\
2.1	0\\
2.11	0\\
2.12	0\\
2.13	1.73472347597681e-18\\
2.14	0\\
2.15	0\\
2.16	0\\
2.17	0\\
2.18	0\\
2.19	0\\
2.2	0\\
2.21	0\\
2.22	0\\
2.23	0\\
2.24	1.73472347597681e-18\\
2.25	1.73472347597681e-18\\
2.26	0\\
2.27	0\\
2.28	0\\
2.29	0\\
2.3	1.73472347597681e-18\\
2.31	0\\
2.32	0\\
2.33	0\\
2.34	0\\
2.35	1.73472347597681e-18\\
2.36	0\\
2.37	0\\
2.38	0\\
2.39	0\\
2.4	0\\
2.41	0\\
2.42	0\\
2.43	0\\
2.44	0\\
2.45	1.73472347597681e-18\\
2.46	0\\
2.47	0\\
2.48	0\\
2.49	0\\
2.5	0\\
2.51	0\\
2.52	1.73472347597681e-18\\
2.53	0\\
2.54	0\\
2.55	1.73472347597681e-18\\
2.56	0\\
2.57	0\\
2.58	0\\
2.59	0\\
2.6	0\\
2.61	0\\
2.62	0\\
2.63	0\\
2.64	0\\
2.65	0\\
2.66	0\\
2.67	1.73472347597681e-18\\
2.68	0\\
2.69	0\\
2.7	0\\
2.71	0\\
2.72	0\\
2.73	0\\
2.74	0\\
2.75	0\\
2.76	0\\
2.77	0\\
2.78	0\\
2.79	0\\
2.8	0\\
2.81	0\\
2.82	0\\
2.83	0\\
2.84	0\\
2.85	0\\
2.86	0\\
2.87	0\\
2.88	0\\
2.89	0\\
2.9	0\\
2.91	0\\
2.92	0\\
2.93	0\\
2.94	0\\
2.95	0\\
2.96	1.73472347597681e-18\\
2.97	0\\
2.98	0\\
2.99	0\\
3	1.73472347597681e-18\\
3.01	0\\
3.02	0\\
3.03	0\\
3.04	0\\
3.05	0\\
3.06	0\\
3.07	0\\
3.08	0\\
3.09	0\\
3.1	0\\
3.11	0\\
3.12	0\\
3.13	1.73472347597681e-18\\
3.14	0\\
3.15	0\\
3.16	0\\
3.17	0\\
3.18	0\\
3.19	0\\
3.2	0\\
3.21	1.73472347597681e-18\\
3.22	0\\
3.23	1.73472347597681e-18\\
3.24	0\\
3.25	0\\
3.26	0\\
3.27	0\\
3.28	0\\
3.29	0\\
3.3	0\\
3.31	0\\
3.32	0\\
3.33	0\\
3.34	0\\
3.35	0\\
3.36	0\\
3.37	0\\
3.38	0\\
3.39	1.73472347597681e-18\\
3.4	0\\
3.41	1.73472347597681e-18\\
3.42	1.73472347597681e-18\\
3.43	0\\
3.44	1.73472347597681e-18\\
3.45	0\\
3.46	0\\
3.47	0\\
3.48	0\\
3.49	0\\
3.5	1.73472347597681e-18\\
3.51	0\\
3.52	0\\
3.53	0\\
3.54	0\\
3.55	0\\
3.56	0\\
3.57	0\\
3.58	0\\
3.59	1.73472347597681e-18\\
3.6	0\\
3.61	0\\
3.62	0\\
3.63	0\\
3.64	0\\
3.65	0\\
3.66	0\\
3.67	0\\
3.68	0\\
3.69	0\\
3.7	0\\
3.71	0\\
3.72	0\\
3.73	0\\
3.74	0\\
3.75	0\\
3.76	0\\
3.77	0\\
3.78	0\\
3.79	0\\
3.8	0\\
3.81	1.73472347597681e-18\\
3.82	0\\
3.83	0\\
3.84	0\\
3.85	0\\
3.86	0\\
3.87	0\\
3.88	0\\
3.89	0\\
3.9	0\\
3.91	0\\
3.92	0\\
3.93	0\\
3.94	0\\
3.95	0\\
3.96	0\\
3.97	0\\
3.98	1.73472347597681e-18\\
3.99	0\\
4	0\\
4.01	1.73472347597681e-18\\
4.02	0\\
4.03	0\\
4.04	0\\
4.05	1.73472347597681e-18\\
4.06	0\\
4.07	1.73472347597681e-18\\
4.08	0\\
4.09	0\\
4.1	0\\
4.11	0\\
4.12	0\\
4.13	0\\
4.14	0\\
4.15	1.73472347597681e-18\\
4.16	0\\
4.17	0\\
4.18	0\\
4.19	0\\
4.2	0\\
4.21	1.73472347597681e-18\\
4.22	0\\
4.23	0\\
4.24	0\\
4.25	0\\
4.26	1.73472347597681e-18\\
4.27	0\\
4.28	0\\
4.29	0\\
4.3	0\\
4.31	0\\
4.32	0\\
4.33	0\\
4.34	0\\
4.35	0\\
4.36	0\\
4.37	0\\
4.38	0\\
4.39	0\\
4.4	0\\
4.41	0\\
4.42	0\\
4.43	0\\
4.44	0\\
4.45	0\\
4.46	0\\
4.47	0\\
4.48	0\\
4.49	0\\
4.5	0\\
4.51	1.73472347597681e-18\\
4.52	0\\
4.53	0\\
4.54	0\\
4.55	0\\
4.56	0\\
4.57	0\\
4.58	0\\
4.59	0\\
4.6	0\\
4.61	0\\
4.62	1.73472347597681e-18\\
4.63	0\\
4.64	0\\
4.65	0\\
4.66	0\\
4.67	0\\
4.68	0\\
4.69	0\\
4.7	0\\
4.71	0\\
4.72	0\\
4.73	0\\
4.74	0\\
4.75	0\\
4.76	0\\
4.77	0\\
4.78	0\\
4.79	0\\
4.8	0\\
4.81	0\\
4.82	1.73472347597681e-18\\
4.83	0\\
4.84	0\\
4.85	1.73472347597681e-18\\
4.86	0\\
4.87	0\\
4.88	0\\
4.89	0\\
4.9	0\\
4.91	0\\
4.92	0\\
4.93	0\\
4.94	0\\
4.95	0\\
4.96	0\\
4.97	0\\
4.98	1.73472347597681e-18\\
4.99	0\\
5	0\\
5.01	0\\
5.02	0\\
5.03	0\\
5.04	0\\
5.05	0\\
5.06	1.73472347597681e-18\\
5.07	1.73472347597681e-18\\
5.08	0\\
5.09	0\\
5.1	0\\
5.11	1.73472347597681e-18\\
5.12	0\\
5.13	0\\
5.14	0\\
5.15	0\\
5.16	0\\
5.17	0\\
5.18	0\\
5.19	0\\
5.2	0\\
5.21	0\\
5.22	0\\
5.23	1.73472347597681e-18\\
5.24	0\\
5.25	0\\
5.26	0\\
5.27	0\\
5.28	0\\
5.29	0\\
5.3	0\\
5.31	0\\
5.32	1.73472347597681e-18\\
5.33	0\\
5.34	1.73472347597681e-18\\
5.35	1.73472347597681e-18\\
5.36	0\\
5.37	0\\
5.38	0\\
5.39	0\\
5.4	0\\
5.41	0\\
5.42	0\\
5.43	0\\
5.44	0\\
5.45	0\\
5.46	1.73472347597681e-18\\
5.47	0\\
5.48	0\\
5.49	1.73472347597681e-18\\
5.5	0\\
5.51	0\\
5.52	0\\
5.53	0\\
5.54	0\\
5.55	0\\
5.56	0\\
5.57	1.73472347597681e-18\\
5.58	0\\
5.59	0\\
5.6	0\\
5.61	0\\
5.62	0\\
5.63	1.73472347597681e-18\\
5.64	0\\
5.65	0\\
5.66	0\\
5.67	0\\
5.68	0\\
5.69	1.73472347597681e-18\\
5.7	0\\
5.71	1.73472347597681e-18\\
5.72	0\\
5.73	0\\
5.74	0\\
5.75	0\\
5.76	1.73472347597681e-18\\
5.77	0\\
5.78	0\\
5.79	0\\
5.8	0\\
5.81	0\\
5.82	0\\
5.83	1.73472347597681e-18\\
5.84	0\\
5.85	0\\
5.86	0\\
5.87	0\\
5.88	0\\
5.89	0\\
5.9	0\\
5.91	0\\
5.92	0\\
5.93	0\\
5.94	0\\
5.95	0\\
5.96	0\\
5.97	0\\
5.98	0\\
5.99	0\\
6	0\\
6.01	0\\
6.02	0\\
6.03	0\\
6.04	1.73472347597681e-18\\
6.05	0\\
6.06	0\\
6.07	0\\
6.08	0\\
6.09	0\\
6.1	0\\
6.11	1.73472347597681e-18\\
6.12	0\\
6.13	0\\
6.14	0\\
6.15	0\\
6.16	0\\
6.17	1.73472347597681e-18\\
6.18	0\\
6.19	1.73472347597681e-18\\
6.2	0\\
6.21	0\\
6.22	0\\
6.23	0\\
6.24	0\\
6.25	1.73472347597681e-18\\
6.26	0\\
6.27	0\\
6.28	1.73472347597681e-18\\
6.29	0\\
6.3	0\\
6.31	0\\
6.32	0\\
6.33	1.73472347597681e-18\\
6.34	0\\
6.35	1.73472347597681e-18\\
6.36	1.73472347597681e-18\\
6.37	0\\
6.38	1.73472347597681e-18\\
6.39	0\\
6.4	0\\
6.41	0\\
6.42	1.73472347597681e-18\\
6.43	0\\
6.44	0\\
6.45	0\\
6.46	1.73472347597681e-18\\
6.47	0\\
6.48	1.73472347597681e-18\\
6.49	0\\
6.5	0\\
6.51	0\\
6.52	0\\
6.53	0\\
6.54	0\\
6.55	0\\
6.56	1.73472347597681e-18\\
6.57	0\\
6.58	0\\
6.59	0\\
6.6	0\\
6.61	1.73472347597681e-18\\
6.62	0\\
6.63	0\\
6.64	0\\
6.65	0\\
6.66	0\\
6.67	0\\
6.68	0\\
6.69	0\\
6.7	0\\
6.71	0\\
6.72	0\\
6.73	0\\
6.74	0\\
6.75	0\\
6.76	0\\
6.77	0\\
6.78	0\\
6.79	0\\
6.8	0\\
6.81	0\\
6.82	0\\
6.83	0\\
6.84	0\\
6.85	0\\
6.86	1.73472347597681e-18\\
6.87	0\\
6.88	0\\
6.89	0\\
6.9	0\\
6.91	0\\
6.92	0\\
6.93	0\\
6.94	0\\
6.95	0\\
6.96	0\\
6.97	0\\
6.98	0\\
6.99	0\\
7	0\\
7.01	0\\
7.02	0\\
7.03	1.73472347597681e-18\\
7.04	0\\
7.05	0\\
7.06	0\\
7.07	0\\
7.08	0\\
7.09	0\\
7.1	0\\
7.11	0\\
7.12	0\\
7.13	0\\
7.14	0\\
7.15	0\\
7.16	0\\
7.17	0\\
7.18	0\\
7.19	1.73472347597681e-18\\
7.2	0\\
7.21	0\\
7.22	0\\
7.23	0\\
7.24	0\\
7.25	0\\
7.26	0\\
7.27	1.73472347597681e-18\\
7.28	0\\
7.29	0\\
7.3	0\\
7.31	0\\
7.32	0\\
7.33	0\\
7.34	0\\
7.35	0\\
7.36	0\\
7.37	1.73472347597681e-18\\
7.38	0\\
7.39	0\\
7.4	0\\
7.41	0\\
7.42	0\\
7.43	0\\
7.44	0\\
7.45	0\\
7.46	0\\
7.47	0\\
7.48	0\\
7.49	0\\
7.5	0\\
7.51	0\\
7.52	0\\
7.53	0\\
7.54	0\\
7.55	1.73472347597681e-18\\
7.56	1.73472347597681e-18\\
7.57	0\\
7.58	1.73472347597681e-18\\
7.59	0\\
7.6	0\\
7.61	0\\
7.62	0\\
7.63	0\\
7.64	0\\
7.65	0\\
7.66	0\\
7.67	0\\
7.68	0\\
7.69	0\\
7.7	0\\
7.71	0\\
7.72	0\\
7.73	0\\
7.74	0\\
7.75	0\\
7.76	0\\
7.77	0\\
7.78	0\\
7.79	0\\
7.8	0\\
7.81	0\\
7.82	0\\
7.83	0\\
7.84	0\\
7.85	1.73472347597681e-18\\
7.86	0\\
7.87	0\\
7.88	0\\
7.89	0\\
7.9	0\\
7.91	0\\
7.92	0\\
7.93	0\\
7.94	0\\
7.95	1.73472347597681e-18\\
7.96	0\\
7.97	0\\
7.98	0\\
7.99	0\\
8	0\\
8.01	0\\
8.02	0\\
8.03	1.73472347597681e-18\\
8.04	1.73472347597681e-18\\
8.05	1.73472347597681e-18\\
8.06	0\\
8.07	1.73472347597681e-18\\
8.08	0\\
8.09	1.73472347597681e-18\\
8.1	0\\
8.11	0\\
8.12	0\\
8.13	1.73472347597681e-18\\
8.14	0\\
8.15	0\\
8.16	0\\
8.17	0\\
8.18	0\\
8.19	0\\
8.2	0\\
8.21	0\\
8.22	0\\
8.23	0\\
8.24	0\\
8.25	0\\
8.26	0\\
8.27	0\\
8.28	0\\
8.29	0\\
8.3	0\\
8.31	0\\
8.32	0\\
8.33	0\\
8.34	0\\
8.35	0\\
8.36	0\\
8.37	0\\
8.38	0\\
8.39	1.73472347597681e-18\\
8.4	0\\
8.41	0\\
8.42	0\\
8.43	0\\
8.44	0\\
8.45	1.73472347597681e-18\\
8.46	0\\
8.47	0\\
8.48	0\\
8.49	0\\
8.5	0\\
8.51	0\\
8.52	0\\
8.53	0\\
8.54	0\\
8.55	0\\
8.56	0\\
8.57	0\\
8.58	0\\
8.59	0\\
8.6	0\\
8.61	0\\
8.62	0\\
8.63	0\\
8.64	1.73472347597681e-18\\
8.65	0\\
8.66	0\\
8.67	0\\
8.68	1.73472347597681e-18\\
8.69	0\\
8.7	0\\
8.71	0\\
8.72	0\\
8.73	0\\
8.74	0\\
8.75	0\\
8.76	0\\
8.77	0\\
8.78	0\\
8.79	1.73472347597681e-18\\
8.8	0\\
8.81	0\\
8.82	0\\
8.83	0\\
8.84	1.73472347597681e-18\\
8.85	0\\
8.86	1.73472347597681e-18\\
8.87	0\\
8.88	0\\
8.89	1.73472347597681e-18\\
8.9	0\\
8.91	0\\
8.92	0\\
8.93	0\\
8.94	0\\
8.95	0\\
8.96	0\\
8.97	0\\
8.98	0\\
8.99	0\\
9	0\\
9.01	0\\
9.02	0\\
9.03	1.73472347597681e-18\\
9.04	0\\
9.05	0\\
9.06	0\\
9.07	0\\
9.08	0\\
9.09	0\\
9.1	1.73472347597681e-18\\
9.11	0\\
9.12	0\\
9.13	0\\
9.14	0\\
9.15	1.73472347597681e-18\\
9.16	0\\
9.17	1.73472347597681e-18\\
9.18	0\\
9.19	1.73472347597681e-18\\
9.2	0\\
9.21	0\\
9.22	1.73472347597681e-18\\
9.23	0\\
9.24	0\\
9.25	1.73472347597681e-18\\
9.26	1.73472347597681e-18\\
9.27	0\\
9.28	0\\
9.29	0\\
9.3	0\\
9.31	1.73472347597681e-18\\
9.32	0\\
9.33	1.73472347597681e-18\\
9.34	0\\
9.35	0\\
9.36	0\\
9.37	0\\
9.38	0\\
9.39	1.73472347597681e-18\\
9.4	0\\
9.41	1.73472347597681e-18\\
9.42	0\\
9.43	0\\
9.44	0\\
9.45	0\\
9.46	0\\
9.47	0\\
9.48	0\\
9.49	0\\
9.5	0\\
9.51	1.73472347597681e-18\\
9.52	0\\
9.53	0\\
9.54	0\\
9.55	1.73472347597681e-18\\
9.56	0\\
9.57	0\\
9.58	0\\
9.59	0\\
9.6	0\\
9.61	0\\
9.62	0\\
9.63	0\\
9.64	1.73472347597681e-18\\
9.65	0\\
9.66	0\\
9.67	0\\
9.68	1.73472347597681e-18\\
9.69	0\\
9.7	0\\
9.71	0\\
9.72	1.73472347597681e-18\\
9.73	0\\
9.74	0\\
9.75	0\\
9.76	0\\
9.77	0\\
9.78	1.73472347597681e-18\\
9.79	0\\
9.8	0\\
9.81	1.73472347597681e-18\\
9.82	0\\
9.83	0\\
9.84	0\\
9.85	0\\
9.86	0\\
9.87	1.73472347597681e-18\\
9.88	0\\
9.89	0\\
9.9	1.73472347597681e-18\\
9.91	0\\
9.92	0\\
9.93	0\\
9.94	0\\
9.95	0\\
9.96	0\\
9.97	0\\
9.98	1.73472347597681e-18\\
9.99	0\\
10	0\\
10.01	0\\
10.02	0\\
10.03	0\\
10.04	0\\
10.05	0\\
10.06	0\\
10.07	0\\
10.08	0\\
10.09	0\\
10.1	0\\
10.11	0\\
10.12	0\\
10.13	0\\
10.14	0\\
10.15	0\\
10.16	0\\
10.17	0\\
10.18	0\\
10.19	0\\
10.2	0\\
10.21	0\\
10.22	0\\
10.23	0\\
10.24	0\\
10.25	0\\
10.26	0\\
10.27	0\\
10.28	0\\
10.29	0\\
10.3	0\\
10.31	0\\
10.32	0\\
10.33	0\\
10.34	0\\
10.35	0\\
10.36	1.73472347597681e-18\\
10.37	1.73472347597681e-18\\
10.38	0\\
10.39	0\\
10.4	0\\
10.41	0\\
10.42	1.73472347597681e-18\\
10.43	1.73472347597681e-18\\
10.44	0\\
10.45	0\\
10.46	0\\
10.47	0\\
10.48	0\\
10.49	0\\
10.5	0\\
10.51	0\\
10.52	0\\
10.53	0\\
10.54	0\\
10.55	0\\
10.56	0\\
10.57	0\\
10.58	0\\
10.59	1.73472347597681e-18\\
10.6	0\\
10.61	1.73472347597681e-18\\
10.62	0\\
10.63	0\\
10.64	0\\
10.65	0\\
10.66	0\\
10.67	0\\
10.68	0\\
10.69	0\\
10.7	0\\
10.71	0\\
10.72	0\\
10.73	0\\
10.74	0\\
10.75	0\\
10.76	1.73472347597681e-18\\
10.77	1.73472347597681e-18\\
10.78	0\\
10.79	1.73472347597681e-18\\
10.8	0\\
10.81	0\\
10.82	0\\
10.83	0\\
10.84	0\\
10.85	0\\
10.86	0\\
10.87	0\\
10.88	1.73472347597681e-18\\
10.89	0\\
10.9	1.73472347597681e-18\\
10.91	0\\
10.92	0\\
10.93	0\\
10.94	0\\
10.95	0\\
10.96	0\\
10.97	0\\
10.98	0\\
10.99	0\\
11	0\\
11.01	0\\
11.02	0\\
11.03	0\\
11.04	0\\
11.05	0\\
11.06	0\\
11.07	0\\
11.08	0\\
11.09	0\\
11.1	0\\
11.11	1.73472347597681e-18\\
11.12	0\\
11.13	0\\
11.14	0\\
11.15	0\\
11.16	1.73472347597681e-18\\
11.17	0\\
11.18	0\\
11.19	1.73472347597681e-18\\
11.2	0\\
11.21	0\\
11.22	0\\
11.23	0\\
11.24	1.73472347597681e-18\\
11.25	0\\
11.26	0\\
11.27	0\\
11.28	0\\
11.29	0\\
11.3	0\\
11.31	0\\
11.32	0\\
11.33	0\\
11.34	0\\
11.35	1.73472347597681e-18\\
11.36	0\\
11.37	0\\
11.38	0\\
11.39	1.73472347597681e-18\\
11.4	0\\
11.41	1.73472347597681e-18\\
11.42	0\\
11.43	0\\
11.44	0\\
11.45	0\\
11.46	0\\
11.47	0\\
11.48	0\\
11.49	0\\
11.5	0\\
11.51	0\\
11.52	0\\
11.53	0\\
11.54	0\\
11.55	0\\
11.56	0\\
11.57	0\\
11.58	1.73472347597681e-18\\
11.59	1.73472347597681e-18\\
11.6	0\\
11.61	0\\
11.62	0\\
11.63	1.73472347597681e-18\\
11.64	0\\
11.65	0\\
11.66	0\\
11.67	0\\
11.68	0\\
11.69	1.73472347597681e-18\\
11.7	0\\
11.71	0\\
11.72	0\\
11.73	0\\
11.74	0\\
11.75	0\\
11.76	0\\
11.77	1.73472347597681e-18\\
11.78	0\\
11.79	0\\
11.8	0\\
11.81	0\\
11.82	0\\
11.83	0\\
11.84	0\\
11.85	0\\
11.86	0\\
11.87	0\\
11.88	0\\
11.89	0\\
11.9	0\\
11.91	1.73472347597681e-18\\
11.92	0\\
11.93	0\\
11.94	1.73472347597681e-18\\
11.95	0\\
11.96	0\\
11.97	0\\
11.98	0\\
11.99	0\\
12	0\\
12.01	0\\
12.02	0\\
12.03	0\\
12.04	1.73472347597681e-18\\
12.05	0\\
12.06	0\\
12.07	0\\
12.08	0\\
12.09	1.73472347597681e-18\\
12.1	0\\
12.11	1.73472347597681e-18\\
12.12	0\\
12.13	0\\
12.14	0\\
12.15	0\\
12.16	0\\
12.17	1.73472347597681e-18\\
12.18	1.73472347597681e-18\\
12.19	1.73472347597681e-18\\
12.2	0\\
12.21	0\\
12.22	0\\
12.23	0\\
12.24	0\\
12.25	1.73472347597681e-18\\
12.26	0\\
12.27	0\\
12.28	1.73472347597681e-18\\
12.29	0\\
12.3	0\\
12.31	0\\
12.32	0\\
12.33	0\\
12.34	0\\
12.35	1.73472347597681e-18\\
12.36	0\\
12.37	0\\
12.38	1.73472347597681e-18\\
12.39	0\\
12.4	0\\
12.41	0\\
12.42	0\\
12.43	0\\
12.44	0\\
12.45	0\\
12.46	0\\
12.47	1.73472347597681e-18\\
12.48	0\\
12.49	0\\
12.5	0\\
12.51	0\\
12.52	0\\
12.53	0\\
12.54	0\\
12.55	0\\
12.56	0\\
12.57	0\\
12.58	0\\
12.59	0\\
12.6	0\\
12.61	0\\
12.62	0\\
12.63	0\\
12.64	0\\
12.65	0\\
12.66	0\\
12.67	0\\
12.68	0\\
12.69	0\\
12.7	0\\
12.71	0\\
12.72	1.73472347597681e-18\\
12.73	0\\
12.74	0\\
12.75	0\\
12.76	0\\
12.77	0\\
12.78	0\\
12.79	0\\
12.8	0\\
12.81	1.73472347597681e-18\\
12.82	0\\
12.83	0\\
12.84	1.73472347597681e-18\\
12.85	0\\
12.86	1.73472347597681e-18\\
12.87	0\\
12.88	0\\
12.89	0\\
12.9	0\\
12.91	0\\
12.92	0\\
12.93	0\\
12.94	0\\
12.95	1.73472347597681e-18\\
12.96	0\\
12.97	0\\
12.98	0\\
12.99	0\\
13	0\\
13.01	0\\
13.02	0\\
13.03	0\\
13.04	1.73472347597681e-18\\
13.05	0\\
13.06	0\\
13.07	1.73472347597681e-18\\
13.08	0\\
13.09	0\\
13.1	0\\
13.11	0\\
13.12	0\\
13.13	0\\
13.14	0\\
13.15	0\\
13.16	0\\
13.17	0\\
13.18	0\\
13.19	0\\
13.2	0\\
13.21	0\\
13.22	0\\
13.23	0\\
13.24	0\\
13.25	1.73472347597681e-18\\
13.26	0\\
13.27	0\\
13.28	0\\
13.29	0\\
13.3	0\\
13.31	1.73472347597681e-18\\
13.32	0\\
13.33	1.73472347597681e-18\\
13.34	1.73472347597681e-18\\
13.35	0\\
13.36	1.73472347597681e-18\\
13.37	0\\
13.38	1.73472347597681e-18\\
13.39	0\\
13.4	0\\
13.41	0\\
13.42	0\\
13.43	0\\
13.44	0\\
13.45	0\\
13.46	0\\
13.47	1.73472347597681e-18\\
13.48	0\\
13.49	0\\
13.5	0\\
13.51	0\\
13.52	0\\
13.53	0\\
13.54	0\\
13.55	0\\
13.56	0\\
13.57	0\\
13.58	0\\
13.59	0\\
13.6	0\\
13.61	1.73472347597681e-18\\
13.62	0\\
13.63	1.73472347597681e-18\\
13.64	0\\
13.65	0\\
13.66	1.73472347597681e-18\\
13.67	1.73472347597681e-18\\
13.68	0\\
13.69	0\\
13.7	1.73472347597681e-18\\
13.71	0\\
13.72	1.73472347597681e-18\\
13.73	0\\
13.74	1.73472347597681e-18\\
13.75	0\\
13.76	0\\
13.77	0\\
13.78	0\\
13.79	1.73472347597681e-18\\
13.8	0\\
13.81	1.73472347597681e-18\\
13.82	1.73472347597681e-18\\
13.83	0\\
13.84	0\\
13.85	0\\
13.86	0\\
13.87	0\\
13.88	0\\
13.89	1.73472347597681e-18\\
13.9	0\\
13.91	0\\
13.92	1.73472347597681e-18\\
13.93	0\\
13.94	0\\
13.95	0\\
13.96	0\\
13.97	0\\
13.98	0\\
13.99	1.73472347597681e-18\\
14	1.73472347597681e-18\\
14.01	1.73472347597681e-18\\
14.02	0\\
14.03	0\\
14.04	0\\
14.05	0\\
14.06	0\\
14.07	0\\
14.08	0\\
14.09	0\\
14.1	1.73472347597681e-18\\
14.11	0\\
14.12	0\\
14.13	0\\
14.14	0\\
14.15	0\\
14.16	0\\
14.17	0\\
14.18	0\\
14.19	1.73472347597681e-18\\
14.2	0\\
14.21	0\\
14.22	0\\
14.23	0\\
14.24	0\\
14.25	0\\
14.26	0\\
14.27	0\\
14.28	0\\
14.29	0\\
14.3	0\\
14.31	0\\
14.32	0\\
14.33	1.73472347597681e-18\\
14.34	0\\
14.35	0\\
14.36	0\\
14.37	0\\
14.38	0\\
14.39	1.73472347597681e-18\\
14.4	0\\
14.41	1.73472347597681e-18\\
14.42	0\\
14.43	0\\
14.44	0\\
14.45	0\\
14.46	1.73472347597681e-18\\
14.47	1.73472347597681e-18\\
14.48	0\\
14.49	0\\
14.5	0\\
14.51	0\\
14.52	0\\
14.53	0\\
14.54	0\\
14.55	0\\
14.56	0\\
14.57	0\\
14.58	0\\
14.59	0\\
14.6	0\\
14.61	0\\
14.62	0\\
14.63	1.73472347597681e-18\\
14.64	0\\
14.65	0\\
14.66	0\\
14.67	0\\
14.68	0\\
14.69	0\\
14.7	1.73472347597681e-18\\
14.71	0\\
14.72	1.73472347597681e-18\\
14.73	1.73472347597681e-18\\
14.74	0\\
14.75	0\\
14.76	0\\
14.77	0\\
14.78	0\\
14.79	0\\
14.8	0\\
14.81	0\\
14.82	0\\
14.83	0\\
14.84	1.73472347597681e-18\\
14.85	0\\
14.86	0\\
14.87	0\\
14.88	0\\
14.89	0\\
14.9	0\\
14.91	1.73472347597681e-18\\
14.92	0\\
14.93	0\\
14.94	0\\
14.95	1.73472347597681e-18\\
14.96	1.73472347597681e-18\\
14.97	0\\
14.98	0\\
14.99	0\\
15	0\\
15.01	1.73472347597681e-18\\
15.02	0\\
15.03	0\\
15.04	0\\
15.05	1.73472347597681e-18\\
15.06	0\\
15.07	0\\
15.08	0\\
15.09	0\\
15.1	0\\
15.11	0\\
15.12	0\\
15.13	0\\
15.14	0\\
15.15	0\\
15.16	0\\
15.17	0\\
15.18	0\\
15.19	0\\
15.2	1.73472347597681e-18\\
15.21	0\\
15.22	0\\
15.23	0\\
15.24	1.73472347597681e-18\\
15.25	1.73472347597681e-18\\
15.26	0\\
15.27	0\\
15.28	0\\
15.29	0\\
15.3	0\\
15.31	0\\
15.32	0\\
15.33	1.73472347597681e-18\\
15.34	0\\
15.35	0\\
15.36	0\\
15.37	0\\
15.38	0\\
15.39	0\\
15.4	0\\
15.41	0\\
15.42	0\\
15.43	0\\
15.44	0\\
15.45	0\\
15.46	0\\
15.47	0\\
15.48	0\\
15.49	0\\
15.5	0\\
15.51	0\\
15.52	0\\
15.53	0\\
15.54	0\\
15.55	0\\
15.56	0\\
15.57	0\\
15.58	1.73472347597681e-18\\
15.59	0\\
15.6	0\\
15.61	0\\
15.62	0\\
15.63	1.73472347597681e-18\\
15.64	0\\
15.65	0\\
15.66	0\\
15.67	0\\
15.68	1.73472347597681e-18\\
15.69	0\\
15.7	0\\
15.71	0\\
15.72	0\\
15.73	0\\
15.74	0\\
15.75	1.73472347597681e-18\\
15.76	0\\
15.77	0\\
15.78	0\\
15.79	0\\
15.8	0\\
15.81	0\\
15.82	0\\
15.83	0\\
15.84	0\\
15.85	0\\
15.86	0\\
15.87	0\\
15.88	0\\
15.89	0\\
15.9	0\\
15.91	0\\
15.92	0\\
15.93	0\\
15.94	0\\
15.95	0\\
15.96	0\\
15.97	0\\
15.98	0\\
15.99	0\\
16	1.73472347597681e-18\\
16.01	0\\
16.02	0\\
16.03	0\\
16.04	0\\
16.05	0\\
16.06	0\\
16.07	0\\
16.08	1.73472347597681e-18\\
16.09	0\\
16.1	0\\
16.11	0\\
16.12	0\\
16.13	0\\
16.14	0\\
16.15	0\\
16.16	0\\
16.17	0\\
16.18	0\\
16.19	0\\
16.2	0\\
16.21	0\\
16.22	0\\
16.23	0\\
16.24	0\\
16.25	0\\
16.26	0\\
16.27	0\\
16.28	0\\
16.29	0\\
16.3	0\\
16.31	0\\
16.32	0\\
16.33	0\\
16.34	0\\
16.35	1.73472347597681e-18\\
16.36	0\\
16.37	1.73472347597681e-18\\
16.38	0\\
16.39	0\\
16.4	0\\
16.41	0\\
16.42	0\\
16.43	0\\
16.44	0\\
16.45	0\\
16.46	0\\
16.47	0\\
16.48	0\\
16.49	0\\
16.5	0\\
16.51	0\\
16.52	0\\
16.53	0\\
16.54	0\\
16.55	0\\
16.56	0\\
16.57	0\\
16.58	0\\
16.59	0\\
16.6	0\\
16.61	0\\
16.62	0\\
16.63	0\\
16.64	1.73472347597681e-18\\
16.65	1.73472347597681e-18\\
16.66	0\\
16.67	0\\
16.68	1.73472347597681e-18\\
16.69	0\\
16.7	0\\
16.71	0\\
16.72	0\\
16.73	0\\
16.74	0\\
16.75	0\\
16.76	1.73472347597681e-18\\
16.77	0\\
16.78	1.73472347597681e-18\\
16.79	0\\
16.8	0\\
16.81	0\\
16.82	0\\
16.83	1.73472347597681e-18\\
16.84	1.73472347597681e-18\\
16.85	0\\
16.86	1.73472347597681e-18\\
16.87	0\\
16.88	0\\
16.89	1.73472347597681e-18\\
16.9	0\\
16.91	0\\
16.92	0\\
16.93	1.73472347597681e-18\\
16.94	0\\
16.95	1.73472347597681e-18\\
16.96	0\\
16.97	0\\
16.98	0\\
16.99	1.73472347597681e-18\\
17	1.73472347597681e-18\\
17.01	1.73472347597681e-18\\
17.02	0\\
17.03	0\\
17.04	0\\
17.05	0\\
17.06	0\\
17.07	0\\
17.08	0\\
17.09	1.73472347597681e-18\\
17.1	0\\
17.11	0\\
17.12	0\\
17.13	1.73472347597681e-18\\
17.14	0\\
17.15	0\\
17.16	0\\
17.17	1.73472347597681e-18\\
17.18	0\\
17.19	0\\
17.2	1.73472347597681e-18\\
17.21	0\\
17.22	1.73472347597681e-18\\
17.23	0\\
17.24	0\\
17.25	0\\
17.26	0\\
17.27	0\\
17.28	0\\
17.29	0\\
17.3	0\\
17.31	0\\
17.32	0\\
17.33	0\\
17.34	0\\
17.35	0\\
17.36	0\\
17.37	0\\
17.38	0\\
17.39	0\\
17.4	0\\
17.41	1.73472347597681e-18\\
17.42	1.73472347597681e-18\\
17.43	0\\
17.44	0\\
17.45	0\\
17.46	1.73472347597681e-18\\
17.47	0\\
17.48	0\\
17.49	0\\
17.5	0\\
17.51	0\\
17.52	0\\
17.53	0\\
17.54	0\\
17.55	0\\
17.56	0\\
17.57	0\\
17.58	0\\
17.59	0\\
17.6	0\\
17.61	0\\
17.62	0\\
17.63	0\\
17.64	0\\
17.65	0\\
17.66	0\\
17.67	0\\
17.68	0\\
17.69	0\\
17.7	0\\
17.71	0\\
17.72	0\\
17.73	0\\
17.74	0\\
17.75	0\\
17.76	0\\
17.77	0\\
17.78	0\\
17.79	1.73472347597681e-18\\
17.8	0\\
17.81	1.73472347597681e-18\\
17.82	0\\
17.83	1.73472347597681e-18\\
17.84	0\\
17.85	0\\
17.86	0\\
17.87	1.73472347597681e-18\\
17.88	0\\
17.89	0\\
17.9	0\\
17.91	0\\
17.92	1.73472347597681e-18\\
17.93	0\\
17.94	0\\
17.95	0\\
17.96	0\\
17.97	1.73472347597681e-18\\
17.98	0\\
17.99	0\\
18	0\\
18.01	0\\
18.02	0\\
18.03	0\\
18.04	0\\
18.05	0\\
18.06	0\\
18.07	0\\
18.08	0\\
18.09	0\\
18.1	0\\
18.11	1.73472347597681e-18\\
18.12	0\\
18.13	0\\
18.14	0\\
18.15	0\\
18.16	0\\
18.17	0\\
18.18	0\\
18.19	0\\
18.2	1.73472347597681e-18\\
18.21	0\\
18.22	0\\
18.23	0\\
18.24	0\\
18.25	0\\
18.26	0\\
18.27	1.73472347597681e-18\\
18.28	0\\
18.29	0\\
18.3	0\\
18.31	0\\
18.32	0\\
18.33	0\\
18.34	0\\
18.35	1.73472347597681e-18\\
18.36	1.73472347597681e-18\\
18.37	0\\
18.38	0\\
18.39	0\\
18.4	0\\
18.41	0\\
18.42	0\\
18.43	1.73472347597681e-18\\
18.44	0\\
18.45	0\\
18.46	0\\
18.47	0\\
18.48	0\\
18.49	0\\
18.5	0\\
18.51	0\\
18.52	0\\
18.53	0\\
18.54	0\\
18.55	0\\
18.56	0\\
18.57	0\\
18.58	0\\
18.59	0\\
18.6	0\\
18.61	0\\
18.62	0\\
18.63	0\\
18.64	0\\
18.65	0\\
18.66	0\\
18.67	0\\
18.68	0\\
18.69	0\\
18.7	0\\
18.71	0\\
18.72	0\\
18.73	1.73472347597681e-18\\
18.74	0\\
18.75	0\\
18.76	0\\
18.77	0\\
18.78	1.73472347597681e-18\\
18.79	0\\
18.8	0\\
18.81	0\\
18.82	0\\
18.83	0\\
18.84	0\\
18.85	0\\
18.86	0\\
18.87	0\\
18.88	0\\
18.89	0\\
18.9	0\\
18.91	0\\
18.92	0\\
18.93	0\\
18.94	0\\
18.95	0\\
18.96	0\\
18.97	0\\
18.98	0\\
18.99	0\\
19	0\\
19.01	1.73472347597681e-18\\
19.02	0\\
19.03	0\\
19.04	1.73472347597681e-18\\
19.05	0\\
19.06	0\\
19.07	1.73472347597681e-18\\
19.08	0\\
19.09	0\\
19.1	0\\
19.11	0\\
19.12	1.73472347597681e-18\\
19.13	0\\
19.14	0\\
19.15	0\\
19.16	0\\
19.17	0\\
19.18	0\\
19.19	1.73472347597681e-18\\
19.2	0\\
19.21	0\\
19.22	0\\
19.23	0\\
19.24	0\\
19.25	0\\
19.26	0\\
19.27	0\\
19.28	0\\
19.29	1.73472347597681e-18\\
19.3	0\\
19.31	0\\
19.32	0\\
19.33	0\\
19.34	0\\
19.35	0\\
19.36	0\\
19.37	0\\
19.38	0\\
19.39	0\\
19.4	0\\
19.41	0\\
19.42	0\\
19.43	0\\
19.44	0\\
19.45	0\\
19.46	0\\
19.47	0\\
19.48	0\\
19.49	1.73472347597681e-18\\
19.5	0\\
19.51	0\\
19.52	0\\
19.53	0\\
19.54	0\\
19.55	0\\
19.56	0\\
19.57	0\\
19.58	0\\
19.59	0\\
19.6	0\\
19.61	0\\
19.62	1.73472347597681e-18\\
19.63	1.73472347597681e-18\\
19.64	0\\
19.65	0\\
19.66	1.73472347597681e-18\\
19.67	0\\
19.68	0\\
19.69	0\\
19.7	0\\
19.71	0\\
19.72	0\\
19.73	0\\
19.74	0\\
19.75	0\\
19.76	0\\
19.77	0\\
19.78	0\\
19.79	0\\
19.8	0\\
19.81	1.73472347597681e-18\\
19.82	1.73472347597681e-18\\
19.83	0\\
19.84	0\\
19.85	0\\
19.86	1.73472347597681e-18\\
19.87	0\\
19.88	0\\
19.89	0\\
19.9	0\\
19.91	0\\
19.92	0\\
19.93	0\\
19.94	1.73472347597681e-18\\
19.95	0\\
19.96	0\\
19.97	0\\
19.98	0\\
19.99	0\\
20	0\\
20.01	0\\
20.02	1.73472347597681e-18\\
20.03	0\\
20.04	0\\
20.05	0\\
20.06	0\\
20.07	0\\
20.08	0\\
20.09	1.73472347597681e-18\\
20.1	0\\
20.11	0\\
20.12	0\\
20.13	1.73472347597681e-18\\
20.14	0\\
20.15	0\\
20.16	0\\
20.17	0\\
20.18	0\\
20.19	0\\
20.2	0\\
20.21	0\\
20.22	1.73472347597681e-18\\
20.23	0\\
20.24	0\\
20.25	0\\
20.26	0\\
20.27	0\\
20.28	0\\
20.29	0\\
20.3	0\\
20.31	0\\
20.32	0\\
20.33	0\\
20.34	0\\
20.35	0\\
20.36	0\\
20.37	0\\
20.38	1.73472347597681e-18\\
20.39	0\\
20.4	0\\
20.41	0\\
20.42	1.73472347597681e-18\\
20.43	0\\
20.44	1.73472347597681e-18\\
20.45	1.73472347597681e-18\\
20.46	0\\
20.47	0\\
20.48	0\\
20.49	0\\
20.5	0\\
20.51	0\\
20.52	0\\
20.53	0\\
20.54	0\\
20.55	0\\
20.56	0\\
20.57	0\\
20.58	0\\
20.59	0\\
20.6	1.73472347597681e-18\\
20.61	0\\
20.62	0\\
20.63	0\\
20.64	1.73472347597681e-18\\
20.65	1.73472347597681e-18\\
20.66	1.73472347597681e-18\\
20.67	1.73472347597681e-18\\
20.68	0\\
20.69	0\\
20.7	0\\
20.71	0\\
20.72	0\\
20.73	0\\
20.74	1.73472347597681e-18\\
20.75	0\\
20.76	0\\
20.77	1.73472347597681e-18\\
20.78	0\\
20.79	0\\
20.8	0\\
20.81	0\\
20.82	0\\
20.83	0\\
20.84	1.73472347597681e-18\\
20.85	0\\
20.86	0\\
20.87	0\\
20.88	0\\
20.89	0\\
20.9	0\\
20.91	0\\
20.92	0\\
20.93	0\\
20.94	0\\
20.95	0\\
20.96	0\\
20.97	0\\
20.98	0\\
20.99	0\\
21	1.73472347597681e-18\\
21.01	1.73472347597681e-18\\
21.02	0\\
21.03	0\\
21.04	1.73472347597681e-18\\
21.05	0\\
21.06	0\\
21.07	0\\
21.08	0\\
21.09	0\\
21.1	0\\
21.11	0\\
21.12	0\\
21.13	0\\
21.14	0\\
21.15	0\\
21.16	1.73472347597681e-18\\
21.17	0\\
21.18	0\\
21.19	0\\
21.2	0\\
21.21	1.73472347597681e-18\\
21.22	0\\
21.23	0\\
21.24	0\\
21.25	1.73472347597681e-18\\
21.26	1.73472347597681e-18\\
21.27	1.73472347597681e-18\\
21.28	0\\
21.29	0\\
21.3	0\\
21.31	0\\
21.32	1.73472347597681e-18\\
21.33	0\\
21.34	0\\
21.35	0\\
21.36	0\\
21.37	0\\
21.38	0\\
21.39	1.73472347597681e-18\\
21.4	0\\
21.41	0\\
21.42	0\\
21.43	0\\
21.44	0\\
21.45	0\\
21.46	1.73472347597681e-18\\
21.47	0\\
21.48	0\\
21.49	0\\
21.5	0\\
21.51	0\\
21.52	0\\
21.53	0\\
21.54	0\\
21.55	0\\
21.56	0\\
21.57	1.73472347597681e-18\\
21.58	0\\
21.59	0\\
21.6	0\\
21.61	0\\
21.62	0\\
21.63	0\\
21.64	0\\
21.65	0\\
21.66	0\\
21.67	1.73472347597681e-18\\
21.68	0\\
21.69	0\\
21.7	1.73472347597681e-18\\
21.71	0\\
21.72	0\\
21.73	1.73472347597681e-18\\
21.74	0\\
21.75	0\\
21.76	0\\
21.77	0\\
21.78	1.73472347597681e-18\\
21.79	0\\
21.8	0\\
21.81	0\\
21.82	0\\
21.83	0\\
21.84	0\\
21.85	0\\
21.86	0\\
21.87	0\\
21.88	1.73472347597681e-18\\
21.89	0\\
21.9	0\\
21.91	0\\
21.92	1.73472347597681e-18\\
21.93	0\\
21.94	0\\
21.95	0\\
21.96	0\\
21.97	0\\
21.98	0\\
21.99	0\\
22	0\\
22.01	0\\
22.02	0\\
22.03	0\\
22.04	0\\
22.05	0\\
22.06	0\\
22.07	0\\
22.08	0\\
22.09	0\\
22.1	1.73472347597681e-18\\
22.11	0\\
22.12	1.73472347597681e-18\\
22.13	0\\
22.14	1.73472347597681e-18\\
22.15	1.73472347597681e-18\\
22.16	0\\
22.17	0\\
22.18	0\\
22.19	0\\
22.2	0\\
22.21	0\\
22.22	0\\
22.23	0\\
22.24	0\\
22.25	1.73472347597681e-18\\
22.26	1.73472347597681e-18\\
22.27	0\\
22.28	1.73472347597681e-18\\
22.29	0\\
22.3	0\\
22.31	0\\
22.32	0\\
22.33	0\\
22.34	0\\
22.35	0\\
22.36	0\\
22.37	0\\
22.38	0\\
22.39	0\\
22.4	0\\
22.41	0\\
22.42	1.73472347597681e-18\\
22.43	0\\
22.44	0\\
22.45	0\\
22.46	1.73472347597681e-18\\
22.47	1.73472347597681e-18\\
22.48	1.73472347597681e-18\\
22.49	0\\
22.5	0\\
22.51	0\\
22.52	0\\
22.53	0\\
22.54	0\\
22.55	0\\
22.56	0\\
22.57	0\\
22.58	0\\
22.59	0\\
22.6	0\\
22.61	0\\
22.62	0\\
22.63	0\\
22.64	0\\
22.65	1.73472347597681e-18\\
22.66	0\\
22.67	0\\
22.68	1.73472347597681e-18\\
22.69	1.73472347597681e-18\\
22.7	0\\
22.71	0\\
22.72	1.73472347597681e-18\\
22.73	0\\
22.74	0\\
22.75	0\\
22.76	0\\
22.77	0\\
22.78	0\\
22.79	0\\
22.8	0\\
22.81	0\\
22.82	0\\
22.83	0\\
22.84	0\\
22.85	0\\
22.86	0\\
22.87	0\\
22.88	0\\
22.89	0\\
22.9	0\\
22.91	0\\
22.92	0\\
22.93	0\\
22.94	0\\
22.95	0\\
22.96	0\\
22.97	0\\
22.98	0\\
22.99	0\\
23	0\\
23.01	0\\
23.02	0\\
23.03	0\\
23.04	0\\
23.05	0\\
23.06	0\\
23.07	1.73472347597681e-18\\
23.08	0\\
23.09	0\\
23.1	0\\
23.11	0\\
23.12	0\\
23.13	1.73472347597681e-18\\
23.14	0\\
23.15	0\\
23.16	0\\
23.17	0\\
23.18	0\\
23.19	0\\
23.2	0\\
23.21	1.73472347597681e-18\\
23.22	0\\
23.23	0\\
23.24	0\\
23.25	0\\
23.26	0\\
23.27	0\\
23.28	0\\
23.29	0\\
23.3	0\\
23.31	0\\
23.32	0\\
23.33	0\\
23.34	0\\
23.35	0\\
23.36	0\\
23.37	0\\
23.38	1.73472347597681e-18\\
23.39	0\\
23.4	0\\
23.41	0\\
23.42	0\\
23.43	0\\
23.44	0\\
23.45	0\\
23.46	0\\
23.47	0\\
23.48	0\\
23.49	0\\
23.5	0\\
23.51	0\\
23.52	0\\
23.53	0\\
23.54	0\\
23.55	1.73472347597681e-18\\
23.56	0\\
23.57	0\\
23.58	0\\
23.59	0\\
23.6	0\\
23.61	0\\
23.62	0\\
23.63	0\\
23.64	0\\
23.65	0\\
23.66	0\\
23.67	0\\
23.68	0\\
23.69	0\\
23.7	0\\
23.71	0\\
23.72	0\\
23.73	0\\
23.74	0\\
23.75	0\\
23.76	0\\
23.77	0\\
23.78	0\\
23.79	0\\
23.8	1.73472347597681e-18\\
23.81	0\\
23.82	0\\
23.83	0\\
23.84	0\\
23.85	0\\
23.86	1.73472347597681e-18\\
23.87	0\\
23.88	0\\
23.89	0\\
23.9	0\\
23.91	0\\
23.92	0\\
23.93	0\\
23.94	0\\
23.95	0\\
23.96	0\\
23.97	0\\
23.98	0\\
23.99	0\\
24	0\\
24.01	1.73472347597681e-18\\
24.02	0\\
24.03	0\\
24.04	0\\
24.05	0\\
24.06	0\\
24.07	0\\
24.08	0\\
24.09	0\\
24.1	0\\
24.11	0\\
24.12	0\\
24.13	0\\
24.14	0\\
24.15	0\\
24.16	0\\
24.17	1.73472347597681e-18\\
24.18	1.73472347597681e-18\\
24.19	0\\
24.2	0\\
24.21	0\\
24.22	1.73472347597681e-18\\
24.23	0\\
24.24	0\\
24.25	0\\
24.26	0\\
24.27	0\\
24.28	0\\
24.29	0\\
24.3	0\\
24.31	0\\
24.32	0\\
24.33	1.73472347597681e-18\\
24.34	0\\
24.35	0\\
24.36	0\\
24.37	0\\
24.38	0\\
24.39	0\\
24.4	1.73472347597681e-18\\
24.41	1.73472347597681e-18\\
24.42	0\\
24.43	0\\
24.44	0\\
24.45	0\\
24.46	0\\
24.47	1.73472347597681e-18\\
24.48	1.73472347597681e-18\\
24.49	0\\
24.5	0\\
24.51	0\\
24.52	0\\
24.53	0\\
24.54	0\\
24.55	0\\
24.56	0\\
24.57	0\\
24.58	0\\
24.59	0\\
24.6	0\\
24.61	0\\
24.62	0\\
24.63	0\\
24.64	0\\
24.65	1.73472347597681e-18\\
24.66	0\\
24.67	1.73472347597681e-18\\
24.68	0\\
24.69	0\\
24.7	0\\
24.71	0\\
24.72	0\\
24.73	0\\
24.74	1.73472347597681e-18\\
24.75	0\\
24.76	0\\
24.77	0\\
24.78	0\\
24.79	0\\
24.8	0\\
24.81	0\\
24.82	0\\
24.83	0\\
24.84	0\\
24.85	0\\
24.86	0\\
24.87	0\\
24.88	0\\
24.89	0\\
24.9	0\\
24.91	0\\
24.92	1.73472347597681e-18\\
24.93	0\\
24.94	0\\
24.95	0\\
24.96	0\\
24.97	0\\
24.98	0\\
24.99	0\\
25	0\\
25.01	1.73472347597681e-18\\
25.02	1.73472347597681e-18\\
25.03	0\\
25.04	1.73472347597681e-18\\
25.05	0\\
25.06	0\\
25.07	0\\
25.08	0\\
25.09	0\\
25.1	1.73472347597681e-18\\
25.11	1.73472347597681e-18\\
25.12	0\\
25.13	0\\
25.14	0\\
25.15	0\\
25.16	1.73472347597681e-18\\
25.17	0\\
25.18	1.73472347597681e-18\\
25.19	0\\
25.2	0\\
25.21	0\\
25.22	0\\
25.23	0\\
25.24	1.73472347597681e-18\\
25.25	1.73472347597681e-18\\
25.26	0\\
25.27	0\\
25.28	0\\
25.29	0\\
25.3	0\\
25.31	0\\
25.32	0\\
25.33	0\\
25.34	0\\
25.35	0\\
25.36	0\\
25.37	1.73472347597681e-18\\
25.38	0\\
25.39	0\\
25.4	0\\
25.41	1.73472347597681e-18\\
25.42	0\\
25.43	0\\
25.44	0\\
25.45	0\\
25.46	0\\
25.47	0\\
25.48	0\\
25.49	0\\
25.5	0\\
25.51	0\\
25.52	0\\
25.53	0\\
25.54	0\\
25.55	0\\
25.56	0\\
25.57	0\\
25.58	0\\
25.59	0\\
25.6	0\\
25.61	0\\
25.62	0\\
25.63	0\\
25.64	0\\
25.65	0\\
25.66	0\\
25.67	0\\
25.68	0\\
25.69	0\\
25.7	0\\
25.71	0\\
25.72	1.73472347597681e-18\\
25.73	0\\
25.74	1.73472347597681e-18\\
25.75	0\\
25.76	1.73472347597681e-18\\
25.77	0\\
25.78	1.73472347597681e-18\\
25.79	0\\
25.8	0\\
25.81	0\\
25.82	0\\
25.83	0\\
25.84	0\\
25.85	0\\
25.86	0\\
25.87	1.73472347597681e-18\\
25.88	0\\
25.89	0\\
25.9	0\\
25.91	0\\
25.92	0\\
25.93	0\\
25.94	0\\
25.95	0\\
25.96	1.73472347597681e-18\\
25.97	0\\
25.98	0\\
25.99	0\\
26	0\\
26.01	0\\
26.02	0\\
26.03	0\\
26.04	0\\
26.05	0\\
26.06	1.73472347597681e-18\\
26.07	0\\
26.08	0\\
26.09	1.73472347597681e-18\\
26.1	0\\
26.11	0\\
26.12	0\\
26.13	0\\
26.14	0\\
26.15	0\\
26.16	0\\
26.17	0\\
26.18	0\\
26.19	0\\
26.2	0\\
26.21	0\\
26.22	0\\
26.23	0\\
26.24	0\\
26.25	0\\
26.26	0\\
26.27	0\\
26.28	0\\
26.29	0\\
26.3	0\\
26.31	0\\
26.32	0\\
26.33	0\\
26.34	0\\
26.35	0\\
26.36	0\\
26.37	0\\
26.38	0\\
26.39	1.73472347597681e-18\\
26.4	0\\
26.41	0\\
26.42	0\\
26.43	0\\
26.44	0\\
26.45	0\\
26.46	0\\
26.47	0\\
26.48	0\\
26.49	0\\
26.5	0\\
26.51	0\\
26.52	0\\
26.53	0\\
26.54	0\\
26.55	0\\
26.56	0\\
26.57	0\\
26.58	0\\
26.59	0\\
26.6	0\\
26.61	0\\
26.62	0\\
26.63	0\\
26.64	0\\
26.65	0\\
26.66	0\\
26.67	0\\
26.68	0\\
26.69	0\\
26.7	0\\
26.71	1.73472347597681e-18\\
26.72	0\\
26.73	0\\
26.74	0\\
26.75	0\\
26.76	0\\
26.77	0\\
26.78	0\\
26.79	0\\
26.8	1.73472347597681e-18\\
26.81	0\\
26.82	0\\
26.83	0\\
26.84	1.73472347597681e-18\\
26.85	0\\
26.86	1.73472347597681e-18\\
26.87	0\\
26.88	0\\
26.89	0\\
26.9	0\\
26.91	1.73472347597681e-18\\
26.92	0\\
26.93	0\\
26.94	0\\
26.95	0\\
26.96	0\\
26.97	0\\
26.98	1.73472347597681e-18\\
26.99	0\\
27	0\\
27.01	0\\
27.02	0\\
27.03	0\\
27.04	0\\
27.05	0\\
27.06	0\\
27.07	0\\
27.08	0\\
27.09	0\\
27.1	0\\
27.11	1.73472347597681e-18\\
27.12	0\\
27.13	0\\
27.14	0\\
27.15	0\\
27.16	0\\
27.17	0\\
27.18	0\\
27.19	0\\
27.2	0\\
27.21	0\\
27.22	0\\
27.23	0\\
27.24	0\\
27.25	0\\
27.26	0\\
27.27	0\\
27.28	0\\
27.29	0\\
27.3	0\\
27.31	1.73472347597681e-18\\
27.32	0\\
27.33	0\\
27.34	0\\
27.35	0\\
27.36	0\\
27.37	0\\
27.38	0\\
27.39	0\\
27.4	0\\
27.41	0\\
27.42	0\\
27.43	1.73472347597681e-18\\
27.44	0\\
27.45	0\\
27.46	0\\
27.47	0\\
27.48	0\\
27.49	1.73472347597681e-18\\
27.5	1.73472347597681e-18\\
27.51	0\\
27.52	0\\
27.53	0\\
27.54	0\\
27.55	0\\
27.56	0\\
27.57	0\\
27.58	0\\
27.59	0\\
27.6	0\\
27.61	0\\
27.62	0\\
27.63	1.73472347597681e-18\\
27.64	0\\
27.65	0\\
27.66	0\\
27.67	0\\
27.68	0\\
27.69	0\\
27.7	0\\
27.71	0\\
27.72	0\\
27.73	0\\
27.74	0\\
27.75	1.73472347597681e-18\\
27.76	0\\
27.77	0\\
27.78	0\\
27.79	0\\
27.8	0\\
27.81	0\\
27.82	0\\
27.83	0\\
27.84	0\\
27.85	1.73472347597681e-18\\
27.86	0\\
27.87	1.73472347597681e-18\\
27.88	0\\
27.89	0\\
27.9	0\\
27.91	0\\
27.92	0\\
27.93	0\\
27.94	0\\
27.95	0\\
27.96	0\\
27.97	0\\
27.98	0\\
27.99	0\\
28	1.73472347597681e-18\\
28.01	0\\
28.02	0\\
28.03	1.73472347597681e-18\\
28.04	1.73472347597681e-18\\
28.05	0\\
28.06	0\\
28.07	0\\
28.08	0\\
28.09	0\\
28.1	0\\
28.11	1.73472347597681e-18\\
28.12	0\\
28.13	0\\
28.14	0\\
28.15	0\\
28.16	0\\
28.17	0\\
28.18	0\\
28.19	0\\
28.2	0\\
28.21	0\\
28.22	0\\
28.23	0\\
28.24	0\\
28.25	0\\
28.26	0\\
28.27	0\\
28.28	0\\
28.29	0\\
28.3	0\\
28.31	0\\
28.32	0\\
28.33	0\\
28.34	0\\
28.35	1.73472347597681e-18\\
28.36	0\\
28.37	0\\
28.38	0\\
28.39	0\\
28.4	0\\
28.41	0\\
28.42	0\\
28.43	0\\
28.44	0\\
28.45	0\\
28.46	0\\
28.47	0\\
28.48	0\\
28.49	0\\
28.5	0\\
28.51	0\\
28.52	0\\
28.53	0\\
28.54	1.73472347597681e-18\\
28.55	0\\
28.56	0\\
28.57	0\\
28.58	1.73472347597681e-18\\
28.59	1.73472347597681e-18\\
28.6	0\\
28.61	0\\
28.62	0\\
28.63	0\\
28.64	0\\
28.65	0\\
28.66	0\\
28.67	0\\
28.68	0\\
28.69	1.73472347597681e-18\\
28.7	0\\
28.71	0\\
28.72	0\\
28.73	0\\
28.74	0\\
28.75	0\\
28.76	0\\
28.77	0\\
28.78	0\\
28.79	0\\
28.8	0\\
28.81	1.73472347597681e-18\\
28.82	0\\
28.83	0\\
28.84	0\\
28.85	0\\
28.86	0\\
28.87	0\\
28.88	0\\
28.89	0\\
28.9	0\\
28.91	1.73472347597681e-18\\
28.92	1.73472347597681e-18\\
28.93	0\\
28.94	0\\
28.95	0\\
28.96	0\\
28.97	0\\
28.98	0\\
28.99	0\\
29	0\\
29.01	0\\
29.02	0\\
29.03	0\\
29.04	0\\
29.05	0\\
29.06	0\\
29.07	0\\
29.08	0\\
29.09	0\\
29.1	0\\
29.11	0\\
29.12	0\\
29.13	0\\
29.14	0\\
29.15	0\\
29.16	0\\
29.17	0\\
29.18	1.73472347597681e-18\\
29.19	0\\
29.2	0\\
29.21	0\\
29.22	0\\
29.23	0\\
29.24	0\\
29.25	0\\
29.26	0\\
29.27	0\\
29.28	0\\
29.29	0\\
29.3	0\\
29.31	1.73472347597681e-18\\
29.32	0\\
29.33	0\\
29.34	0\\
29.35	0\\
29.36	0\\
29.37	0\\
29.38	0\\
29.39	0\\
29.4	0\\
29.41	0\\
29.42	0\\
29.43	0\\
29.44	0\\
29.45	0\\
29.46	0\\
29.47	0\\
29.48	0\\
29.49	0\\
29.5	0\\
29.51	0\\
29.52	0\\
29.53	1.73472347597681e-18\\
29.54	0\\
29.55	0\\
29.56	0\\
29.57	0\\
29.58	0\\
29.59	0\\
29.6	0\\
29.61	0\\
29.62	0\\
29.63	1.73472347597681e-18\\
29.64	0\\
29.65	1.73472347597681e-18\\
29.66	1.73472347597681e-18\\
29.67	0\\
29.68	1.73472347597681e-18\\
29.69	0\\
29.7	0\\
29.71	0\\
29.72	1.73472347597681e-18\\
29.73	0\\
29.74	0\\
29.75	0\\
29.76	0\\
29.77	0\\
29.78	0\\
29.79	0\\
29.8	1.73472347597681e-18\\
29.81	0\\
29.82	0\\
29.83	0\\
29.84	0\\
29.85	0\\
29.86	0\\
29.87	1.73472347597681e-18\\
29.88	0\\
29.89	0\\
29.9	0\\
29.91	0\\
29.92	0\\
29.93	1.73472347597681e-18\\
29.94	0\\
29.95	0\\
29.96	0\\
29.97	0\\
29.98	0\\
29.99	0\\
30	1.73472347597681e-18\\
30.01	1.73472347597681e-18\\
30.02	0\\
30.03	0\\
30.04	0\\
30.05	0\\
30.06	0\\
30.07	0\\
30.08	0\\
30.09	0\\
30.1	0\\
30.11	1.73472347597681e-18\\
30.12	0\\
30.13	0\\
30.14	0\\
30.15	0\\
30.16	0\\
30.17	0\\
30.18	0\\
30.19	0\\
30.2	0\\
30.21	0\\
30.22	0\\
30.23	0\\
30.24	0\\
30.25	0\\
30.26	0\\
30.27	0\\
30.28	0\\
30.29	0\\
30.3	1.73472347597681e-18\\
30.31	0\\
30.32	0\\
30.33	0\\
30.34	0\\
30.35	0\\
30.36	0\\
30.37	0\\
30.38	1.73472347597681e-18\\
30.39	0\\
30.4	1.73472347597681e-18\\
30.41	0\\
30.42	0\\
30.43	0\\
30.44	0\\
30.45	0\\
30.46	0\\
30.47	0\\
30.48	0\\
30.49	0\\
30.5	0\\
30.51	0\\
30.52	0\\
30.53	0\\
30.54	0\\
30.55	0\\
30.56	0\\
30.57	0\\
30.58	0\\
30.59	0\\
30.6	0\\
30.61	0\\
30.62	0\\
30.63	0\\
30.64	0\\
30.65	0\\
30.66	0\\
30.67	0\\
30.68	1.73472347597681e-18\\
30.69	0\\
30.7	0\\
30.71	0\\
30.72	1.73472347597681e-18\\
30.73	0\\
30.74	0\\
30.75	0\\
30.76	0\\
30.77	0\\
30.78	0\\
30.79	0\\
30.8	1.73472347597681e-18\\
30.81	1.73472347597681e-18\\
30.82	0\\
30.83	1.73472347597681e-18\\
30.84	0\\
30.85	1.73472347597681e-18\\
30.86	0\\
30.87	0\\
30.88	0\\
30.89	0\\
30.9	0\\
30.91	1.73472347597681e-18\\
30.92	0\\
30.93	0\\
30.94	1.73472347597681e-18\\
30.95	0\\
30.96	0\\
30.97	1.73472347597681e-18\\
30.98	0\\
30.99	1.73472347597681e-18\\
31	0\\
31.01	0\\
31.02	0\\
31.03	0\\
31.04	0\\
31.05	0\\
31.06	0\\
31.07	0\\
31.08	1.73472347597681e-18\\
31.09	0\\
31.1	0\\
31.11	0\\
31.12	0\\
31.13	0\\
31.14	0\\
31.15	0\\
31.16	0\\
31.17	0\\
31.18	0\\
31.19	0\\
31.2	0\\
31.21	1.73472347597681e-18\\
31.22	1.73472347597681e-18\\
31.23	0\\
31.24	0\\
31.25	0\\
31.26	1.73472347597681e-18\\
31.27	0\\
31.28	0\\
31.29	0\\
31.3	0\\
31.31	0\\
31.32	0\\
31.33	0\\
31.34	0\\
31.35	0\\
31.36	0\\
31.37	0\\
31.38	0\\
31.39	1.73472347597681e-18\\
31.4	0\\
31.41	0\\
31.42	1.73472347597681e-18\\
31.43	1.73472347597681e-18\\
31.44	1.73472347597681e-18\\
31.45	0\\
31.46	0\\
31.47	0\\
31.48	0\\
31.49	0\\
31.5	0\\
31.51	0\\
31.52	0\\
31.53	0\\
31.54	0\\
31.55	1.73472347597681e-18\\
31.56	0\\
31.57	0\\
31.58	0\\
31.59	0\\
31.6	0\\
31.61	0\\
31.62	0\\
31.63	0\\
31.64	0\\
31.65	0\\
31.66	0\\
31.67	1.73472347597681e-18\\
31.68	0\\
31.69	0\\
31.7	0\\
31.71	0\\
31.72	0\\
31.73	0\\
31.74	0\\
31.75	0\\
31.76	0\\
31.77	0\\
31.78	0\\
31.79	0\\
31.8	0\\
31.81	0\\
31.82	0\\
31.83	1.73472347597681e-18\\
31.84	0\\
31.85	0\\
31.86	1.73472347597681e-18\\
31.87	0\\
31.88	1.73472347597681e-18\\
31.89	1.73472347597681e-18\\
31.9	0\\
31.91	0\\
31.92	0\\
31.93	0\\
31.94	1.73472347597681e-18\\
31.95	1.73472347597681e-18\\
31.96	0\\
31.97	1.73472347597681e-18\\
31.98	0\\
31.99	0\\
32	0\\
32.01	0\\
32.02	0\\
32.03	0\\
32.04	0\\
32.05	0\\
32.06	0\\
32.07	0\\
32.08	0\\
32.09	0\\
32.1	0\\
32.11	0\\
32.12	0\\
32.13	0\\
32.14	0\\
32.15	0\\
32.16	0\\
32.17	0\\
32.18	0\\
32.19	0\\
32.2	0\\
32.21	0\\
32.22	0\\
32.23	0\\
32.24	0\\
32.25	0\\
32.26	0\\
32.27	0\\
32.28	0\\
32.29	0\\
32.3	0\\
32.31	0\\
32.32	0\\
32.33	0\\
32.34	0\\
32.35	0\\
32.36	0\\
32.37	0\\
32.38	0\\
32.39	0\\
32.4	0\\
32.41	0\\
32.42	0\\
32.43	0\\
32.44	1.73472347597681e-18\\
32.45	1.73472347597681e-18\\
32.46	0\\
32.47	0\\
32.48	0\\
32.49	0\\
32.5	0\\
32.51	0\\
32.52	0\\
32.53	0\\
32.54	0\\
32.55	0\\
32.56	0\\
32.57	0\\
32.58	0\\
32.59	0\\
32.6	0\\
32.61	1.73472347597681e-18\\
32.62	0\\
32.63	0\\
32.64	0\\
32.65	0\\
32.66	0\\
32.67	0\\
32.68	0\\
32.69	0\\
32.7	0\\
32.71	0\\
32.72	0\\
32.73	0\\
32.74	0\\
32.75	0\\
32.76	0\\
32.77	0\\
32.78	0\\
32.79	0\\
32.8	0\\
32.81	0\\
32.82	0\\
32.83	1.73472347597681e-18\\
32.84	1.73472347597681e-18\\
32.85	0\\
32.86	0\\
32.87	0\\
32.88	0\\
32.89	1.73472347597681e-18\\
32.9	0\\
32.91	0\\
32.92	0\\
32.93	0\\
32.94	0\\
32.95	0\\
32.96	0\\
32.97	1.73472347597681e-18\\
32.98	0\\
32.99	0\\
33	0\\
33.01	0\\
33.02	0\\
33.03	0\\
33.04	0\\
33.05	0\\
33.06	0\\
33.07	0\\
33.08	0\\
33.09	0\\
33.1	0\\
33.11	1.73472347597681e-18\\
33.12	0\\
33.13	0\\
33.14	0\\
33.15	0\\
33.16	1.73472347597681e-18\\
33.17	0\\
33.18	0\\
33.19	0\\
33.2	0\\
33.21	0\\
33.22	0\\
33.23	0\\
33.24	0\\
33.25	1.73472347597681e-18\\
33.26	0\\
33.27	1.73472347597681e-18\\
33.28	1.73472347597681e-18\\
33.29	0\\
33.3	1.73472347597681e-18\\
33.31	0\\
33.32	0\\
33.33	0\\
33.34	0\\
33.35	0\\
33.36	1.73472347597681e-18\\
33.37	0\\
33.38	1.73472347597681e-18\\
33.39	0\\
33.4	0\\
33.41	0\\
33.42	0\\
33.43	0\\
33.44	0\\
33.45	0\\
33.46	0\\
33.47	0\\
33.48	0\\
33.49	0\\
33.5	0\\
33.51	0\\
33.52	0\\
33.53	0\\
33.54	0\\
33.55	0\\
33.56	0\\
33.57	0\\
33.58	1.73472347597681e-18\\
33.59	0\\
33.6	1.73472347597681e-18\\
33.61	0\\
33.62	0\\
33.63	0\\
33.64	0\\
33.65	0\\
33.66	0\\
33.67	0\\
33.68	0\\
33.69	0\\
33.7	0\\
33.71	0\\
33.72	0\\
33.73	0\\
33.74	0\\
33.75	0\\
33.76	1.73472347597681e-18\\
33.77	0\\
33.78	0\\
33.79	0\\
33.8	0\\
33.81	0\\
33.82	0\\
33.83	0\\
33.84	0\\
33.85	0\\
33.86	0\\
33.87	0\\
33.88	1.73472347597681e-18\\
33.89	0\\
33.9	0\\
33.91	0\\
33.92	0\\
33.93	0\\
33.94	1.73472347597681e-18\\
33.95	0\\
33.96	0\\
33.97	0\\
33.98	0\\
33.99	1.73472347597681e-18\\
34	0\\
34.01	0\\
34.02	0\\
34.03	0\\
34.04	0\\
34.05	0\\
34.06	0\\
34.07	0\\
34.08	0\\
34.09	0\\
34.1	0\\
34.11	0\\
34.12	0\\
34.13	0\\
34.14	0\\
34.15	0\\
34.16	1.73472347597681e-18\\
34.17	0\\
34.18	0\\
34.19	0\\
34.2	0\\
34.21	0\\
34.22	0\\
34.23	0\\
34.24	0\\
34.25	0\\
34.26	1.73472347597681e-18\\
34.27	0\\
34.28	0\\
34.29	0\\
34.3	0\\
34.31	0\\
34.32	0\\
34.33	0\\
34.34	1.73472347597681e-18\\
34.35	0\\
34.36	0\\
34.37	0\\
34.38	0\\
34.39	0\\
34.4	0\\
34.41	0\\
34.42	1.73472347597681e-18\\
34.43	0\\
34.44	0\\
34.45	0\\
34.46	0\\
34.47	0\\
34.48	0\\
34.49	1.73472347597681e-18\\
34.5	0\\
34.51	0\\
34.52	0\\
34.53	0\\
34.54	0\\
34.55	0\\
34.56	0\\
34.57	0\\
34.58	1.73472347597681e-18\\
34.59	0\\
34.6	0\\
34.61	0\\
34.62	0\\
34.63	0\\
34.64	0\\
34.65	0\\
34.66	0\\
34.67	0\\
34.68	0\\
34.69	0\\
34.7	1.73472347597681e-18\\
34.71	0\\
34.72	0\\
34.73	0\\
34.74	1.73472347597681e-18\\
34.75	1.73472347597681e-18\\
34.76	0\\
34.77	1.73472347597681e-18\\
34.78	1.73472347597681e-18\\
34.79	0\\
34.8	0\\
34.81	0\\
34.82	0\\
34.83	0\\
34.84	0\\
34.85	0\\
34.86	0\\
34.87	0\\
34.88	0\\
34.89	0\\
34.9	0\\
34.91	0\\
34.92	0\\
34.93	0\\
34.94	0\\
34.95	1.73472347597681e-18\\
34.96	1.73472347597681e-18\\
34.97	1.73472347597681e-18\\
34.98	0\\
34.99	0\\
35	0\\
35.01	0\\
35.02	0\\
35.03	0\\
35.04	0\\
35.05	0\\
35.06	0\\
35.07	1.73472347597681e-18\\
35.08	0\\
35.09	1.73472347597681e-18\\
35.1	0\\
35.11	0\\
35.12	0\\
35.13	1.73472347597681e-18\\
35.14	0\\
35.15	0\\
35.16	1.73472347597681e-18\\
35.17	0\\
35.18	0\\
35.19	0\\
35.2	0\\
35.21	0\\
35.22	0\\
35.23	0\\
35.24	0\\
35.25	0\\
35.26	0\\
35.27	0\\
35.28	0\\
35.29	0\\
35.3	0\\
35.31	0\\
35.32	0\\
35.33	0\\
35.34	0\\
35.35	0\\
35.36	0\\
35.37	0\\
35.38	0\\
35.39	0\\
35.4	0\\
35.41	0\\
35.42	0\\
35.43	1.73472347597681e-18\\
35.44	0\\
35.45	0\\
35.46	0\\
35.47	0\\
35.48	0\\
35.49	0\\
35.5	0\\
35.51	0\\
35.52	0\\
35.53	0\\
35.54	0\\
35.55	0\\
35.56	0\\
35.57	0\\
35.58	0\\
35.59	0\\
35.6	0\\
35.61	0\\
35.62	0\\
35.63	0\\
35.64	0\\
35.65	0\\
35.66	0\\
35.67	0\\
35.68	1.73472347597681e-18\\
35.69	1.73472347597681e-18\\
35.7	0\\
35.71	0\\
35.72	0\\
35.73	0\\
35.74	0\\
35.75	0\\
35.76	0\\
35.77	0\\
35.78	0\\
35.79	0\\
35.8	0\\
35.81	0\\
35.82	0\\
35.83	0\\
35.84	0\\
35.85	1.73472347597681e-18\\
35.86	0\\
35.87	0\\
35.88	0\\
35.89	0\\
35.9	0\\
35.91	1.73472347597681e-18\\
35.92	0\\
35.93	0\\
35.94	0\\
35.95	0\\
35.96	0\\
35.97	0\\
35.98	1.73472347597681e-18\\
35.99	0\\
36	0\\
36.01	0\\
36.02	0\\
36.03	0\\
36.04	0\\
36.05	0\\
36.06	0\\
36.07	1.73472347597681e-18\\
36.08	1.73472347597681e-18\\
36.09	0\\
36.1	0\\
36.11	0\\
36.12	0\\
36.13	0\\
36.14	0\\
36.15	0\\
36.16	1.73472347597681e-18\\
36.17	0\\
36.18	0\\
36.19	0\\
36.2	0\\
36.21	0\\
36.22	0\\
36.23	0\\
36.24	0\\
36.25	1.73472347597681e-18\\
36.26	0\\
36.27	0\\
36.28	0\\
36.29	0\\
36.3	0\\
36.31	0\\
36.32	1.73472347597681e-18\\
36.33	0\\
36.34	0\\
36.35	0\\
36.36	0\\
36.37	0\\
36.38	1.73472347597681e-18\\
36.39	0\\
36.4	0\\
36.41	0\\
36.42	0\\
36.43	0\\
36.44	1.73472347597681e-18\\
36.45	0\\
36.46	0\\
36.47	0\\
36.48	0\\
36.49	0\\
36.5	1.73472347597681e-18\\
36.51	0\\
36.52	0\\
36.53	0\\
36.54	0\\
36.55	0\\
36.56	0\\
36.57	0\\
36.58	1.73472347597681e-18\\
36.59	0\\
36.6	1.73472347597681e-18\\
36.61	0\\
36.62	0\\
36.63	1.73472347597681e-18\\
36.64	0\\
36.65	0\\
36.66	0\\
36.67	0\\
36.68	0\\
36.69	0\\
36.7	0\\
36.71	0\\
36.72	0\\
36.73	0\\
36.74	0\\
36.75	0\\
36.76	0\\
36.77	0\\
36.78	0\\
36.79	0\\
36.8	0\\
36.81	1.73472347597681e-18\\
36.82	0\\
36.83	0\\
36.84	0\\
36.85	0\\
36.86	0\\
36.87	0\\
36.88	0\\
36.89	0\\
36.9	0\\
36.91	0\\
36.92	0\\
36.93	0\\
36.94	0\\
36.95	0\\
36.96	0\\
36.97	0\\
36.98	1.73472347597681e-18\\
36.99	0\\
37	0\\
37.01	0\\
37.02	1.73472347597681e-18\\
37.03	0\\
37.04	0\\
37.05	0\\
37.06	0\\
37.07	0\\
37.08	0\\
37.09	0\\
37.1	0\\
37.11	0\\
37.12	0\\
37.13	0\\
37.14	0\\
37.15	0\\
37.16	0\\
37.17	0\\
37.18	0\\
37.19	0\\
37.2	0\\
37.21	0\\
37.22	0\\
37.23	0\\
37.24	0\\
37.25	0\\
37.26	0\\
37.27	0\\
37.28	0\\
37.29	0\\
37.3	1.73472347597681e-18\\
37.31	0\\
37.32	0\\
37.33	0\\
37.34	0\\
37.35	0\\
37.36	0\\
37.37	0\\
37.38	0\\
37.39	0\\
37.4	0\\
37.41	0\\
37.42	0\\
37.43	0\\
37.44	0\\
37.45	0\\
37.46	0\\
37.47	0\\
37.48	0\\
37.49	1.73472347597681e-18\\
37.5	0\\
37.51	0\\
37.52	0\\
37.53	0\\
37.54	0\\
37.55	0\\
37.56	0\\
37.57	0\\
37.58	0\\
37.59	0\\
37.6	0\\
37.61	0\\
37.62	0\\
37.63	0\\
37.64	0\\
37.65	0\\
37.66	0\\
37.67	0\\
37.68	0\\
37.69	0\\
37.7	0\\
37.71	0\\
37.72	0\\
37.73	0\\
37.74	0\\
37.75	0\\
37.76	1.73472347597681e-18\\
37.77	0\\
37.78	0\\
37.79	0\\
37.8	0\\
37.81	0\\
37.82	0\\
37.83	0\\
37.84	0\\
37.85	0\\
37.86	0\\
37.87	0\\
37.88	0\\
37.89	0\\
37.9	0\\
37.91	0\\
37.92	0\\
37.93	0\\
37.94	0\\
37.95	0\\
37.96	0\\
37.97	0\\
37.98	0\\
37.99	0\\
38	1.73472347597681e-18\\
38.01	0\\
38.02	0\\
38.03	0\\
38.04	0\\
38.05	0\\
38.06	0\\
38.07	0\\
38.08	0\\
38.09	0\\
38.1	0\\
38.11	1.73472347597681e-18\\
38.12	1.73472347597681e-18\\
38.13	0\\
38.14	0\\
38.15	0\\
38.16	0\\
38.17	0\\
38.18	0\\
38.19	0\\
38.2	0\\
38.21	0\\
38.22	0\\
38.23	0\\
38.24	0\\
38.25	0\\
38.26	0\\
38.27	1.73472347597681e-18\\
38.28	0\\
38.29	0\\
38.3	0\\
38.31	1.73472347597681e-18\\
38.32	0\\
38.33	0\\
38.34	0\\
38.35	1.73472347597681e-18\\
38.36	0\\
38.37	0\\
38.38	0\\
38.39	0\\
38.4	0\\
38.41	0\\
38.42	0\\
38.43	0\\
38.44	0\\
38.45	0\\
38.46	0\\
38.47	0\\
38.48	1.73472347597681e-18\\
38.49	0\\
38.5	0\\
38.51	0\\
38.52	0\\
38.53	0\\
38.54	0\\
38.55	0\\
38.56	0\\
38.57	0\\
38.58	0\\
38.59	0\\
38.6	0\\
38.61	1.73472347597681e-18\\
38.62	0\\
38.63	1.73472347597681e-18\\
38.64	0\\
38.65	1.73472347597681e-18\\
38.66	0\\
38.67	0\\
38.68	0\\
38.69	0\\
38.7	0\\
38.71	0\\
38.72	0\\
38.73	0\\
38.74	1.73472347597681e-18\\
38.75	0\\
38.76	0\\
38.77	0\\
38.78	0\\
38.79	0\\
38.8	0\\
38.81	0\\
38.82	1.73472347597681e-18\\
38.83	0\\
38.84	0\\
38.85	0\\
38.86	0\\
38.87	1.73472347597681e-18\\
38.88	0\\
38.89	1.73472347597681e-18\\
38.9	0\\
38.91	0\\
38.92	0\\
38.93	0\\
38.94	0\\
38.95	0\\
38.96	0\\
38.97	0\\
38.98	0\\
38.99	0\\
39	0\\
39.01	0\\
39.02	0\\
39.03	0\\
39.04	0\\
39.05	0\\
39.06	0\\
39.07	0\\
39.08	0\\
39.09	1.73472347597681e-18\\
39.1	0\\
39.11	0\\
39.12	0\\
39.13	0\\
39.14	0\\
39.15	0\\
39.16	0\\
39.17	0\\
39.18	0\\
39.19	1.73472347597681e-18\\
39.2	0\\
39.21	0\\
39.22	0\\
39.23	0\\
39.24	0\\
39.25	0\\
39.26	0\\
39.27	0\\
39.28	0\\
39.29	0\\
39.3	0\\
39.31	0\\
39.32	0\\
39.33	0\\
39.34	1.73472347597681e-18\\
39.35	0\\
39.36	0\\
39.37	0\\
39.38	0\\
39.39	0\\
39.4	0\\
39.41	0\\
39.42	0\\
39.43	0\\
39.44	0\\
39.45	0\\
39.46	0\\
39.47	0\\
39.48	0\\
39.49	0\\
39.5	0\\
39.51	0\\
39.52	1.73472347597681e-18\\
39.53	0\\
39.54	0\\
39.55	0\\
39.56	0\\
39.57	0\\
39.58	0\\
39.59	0\\
39.6	0\\
39.61	0\\
39.62	0\\
39.63	0\\
39.64	0\\
39.65	0\\
39.66	0\\
39.67	1.73472347597681e-18\\
39.68	0\\
39.69	0\\
39.7	0\\
39.71	0\\
39.72	0\\
39.73	0\\
39.74	0\\
39.75	1.73472347597681e-18\\
39.76	0\\
39.77	0\\
39.78	0\\
39.79	0\\
39.8	0\\
39.81	0\\
39.82	1.73472347597681e-18\\
39.83	0\\
39.84	0\\
39.85	0\\
39.86	0\\
39.87	0\\
39.88	0\\
39.89	0\\
39.9	0\\
39.91	0\\
39.92	1.73472347597681e-18\\
39.93	0\\
39.94	1.73472347597681e-18\\
39.95	0\\
39.96	0\\
39.97	0\\
39.98	0\\
39.99	0\\
40	0\\
40.01	1.73472347597681e-18\\
};
\addplot [color=red,dashed,forget plot]
  table[row sep=crcr]{%
40.01	1.73472347597681e-18\\
40.02	0\\
40.03	1.73472347597681e-18\\
40.04	0\\
40.05	1.73472347597681e-18\\
40.06	0\\
40.07	1.73472347597681e-18\\
40.08	0\\
40.09	1.73472347597681e-18\\
40.1	1.73472347597681e-18\\
40.11	0\\
40.12	0\\
40.13	0\\
40.14	0\\
40.15	1.73472347597681e-18\\
40.16	0\\
40.17	0\\
40.18	0\\
40.19	0\\
40.2	0\\
40.21	0\\
40.22	1.73472347597681e-18\\
40.23	0\\
40.24	0\\
40.25	1.73472347597681e-18\\
40.26	0\\
40.27	0\\
40.28	0\\
40.29	0\\
40.3	0\\
40.31	1.73472347597681e-18\\
40.32	1.73472347597681e-18\\
40.33	0\\
40.34	0\\
40.35	1.73472347597681e-18\\
40.36	0\\
40.37	0\\
40.38	0\\
40.39	0\\
40.4	1.73472347597681e-18\\
40.41	0\\
40.42	0\\
40.43	1.73472347597681e-18\\
40.44	0\\
40.45	0\\
40.46	0\\
40.47	0\\
40.48	0\\
40.49	0\\
40.5	1.73472347597681e-18\\
40.51	0\\
40.52	0\\
40.53	0\\
40.54	0\\
40.55	0\\
40.56	0\\
40.57	0\\
40.58	1.73472347597681e-18\\
40.59	0\\
40.6	0\\
40.61	1.73472347597681e-18\\
40.62	0\\
40.63	0\\
40.64	0\\
40.65	0\\
40.66	0\\
40.67	0\\
40.68	0\\
40.69	0\\
40.7	0\\
40.71	0\\
40.72	0\\
40.73	0\\
40.74	0\\
40.75	0\\
40.76	0\\
40.77	0\\
40.78	0\\
40.79	0\\
40.8	0\\
40.81	0\\
40.82	0\\
40.83	0\\
40.84	0\\
40.85	0\\
40.86	0\\
40.87	1.73472347597681e-18\\
40.88	0\\
40.89	0\\
40.9	1.73472347597681e-18\\
40.91	0\\
40.92	0\\
40.93	0\\
40.94	0\\
40.95	1.73472347597681e-18\\
40.96	0\\
40.97	1.73472347597681e-18\\
40.98	0\\
40.99	0\\
41	0\\
41.01	0\\
41.02	0\\
41.03	0\\
41.04	0\\
41.05	0\\
41.06	0\\
41.07	0\\
41.08	0\\
41.09	0\\
41.1	0\\
41.11	0\\
41.12	0\\
41.13	0\\
41.14	0\\
41.15	1.73472347597681e-18\\
41.16	0\\
41.17	0\\
41.18	1.73472347597681e-18\\
41.19	0\\
41.2	0\\
41.21	0\\
41.22	0\\
41.23	0\\
41.24	0\\
41.25	0\\
41.26	0\\
41.27	1.73472347597681e-18\\
41.28	0\\
41.29	1.73472347597681e-18\\
41.3	0\\
41.31	0\\
41.32	0\\
41.33	0\\
41.34	0\\
41.35	0\\
41.36	0\\
41.37	0\\
41.38	1.73472347597681e-18\\
41.39	0\\
41.4	0\\
41.41	1.73472347597681e-18\\
41.42	0\\
41.43	0\\
41.44	0\\
41.45	0\\
41.46	0\\
41.47	0\\
41.48	0\\
41.49	0\\
41.5	0\\
41.51	0\\
41.52	0\\
41.53	0\\
41.54	0\\
41.55	0\\
41.56	0\\
41.57	0\\
41.58	0\\
41.59	0\\
41.6	0\\
41.61	1.73472347597681e-18\\
41.62	0\\
41.63	0\\
41.64	0\\
41.65	0\\
41.66	0\\
41.67	0\\
41.68	1.73472347597681e-18\\
41.69	0\\
41.7	0\\
41.71	0\\
41.72	0\\
41.73	0\\
41.74	0\\
41.75	0\\
41.76	0\\
41.77	0\\
41.78	0\\
41.79	0\\
41.8	0\\
41.81	1.73472347597681e-18\\
41.82	0\\
41.83	0\\
41.84	0\\
41.85	1.73472347597681e-18\\
41.86	0\\
41.87	0\\
41.88	0\\
41.89	0\\
41.9	0\\
41.91	0\\
41.92	1.73472347597681e-18\\
41.93	0\\
41.94	0\\
41.95	0\\
41.96	0\\
41.97	0\\
41.98	0\\
41.99	0\\
42	0\\
42.01	0\\
42.02	0\\
42.03	0\\
42.04	0\\
42.05	0\\
42.06	0\\
42.07	0\\
42.08	0\\
42.09	0\\
42.1	0\\
42.11	0\\
42.12	0\\
42.13	0\\
42.14	1.73472347597681e-18\\
42.15	0\\
42.16	1.73472347597681e-18\\
42.17	0\\
42.18	1.73472347597681e-18\\
42.19	0\\
42.2	0\\
42.21	0\\
42.22	0\\
42.23	0\\
42.24	1.73472347597681e-18\\
42.25	0\\
42.26	0\\
42.27	0\\
42.28	0\\
42.29	0\\
42.3	1.73472347597681e-18\\
42.31	0\\
42.32	0\\
42.33	0\\
42.34	0\\
42.35	0\\
42.36	1.73472347597681e-18\\
42.37	0\\
42.38	0\\
42.39	0\\
42.4	0\\
42.41	0\\
42.42	0\\
42.43	0\\
42.44	0\\
42.45	0\\
42.46	0\\
42.47	0\\
42.48	0\\
42.49	0\\
42.5	0\\
42.51	0\\
42.52	0\\
42.53	0\\
42.54	0\\
42.55	0\\
42.56	0\\
42.57	0\\
42.58	0\\
42.59	0\\
42.6	0\\
42.61	0\\
42.62	0\\
42.63	0\\
42.64	0\\
42.65	0\\
42.66	0\\
42.67	0\\
42.68	0\\
42.69	0\\
42.7	0\\
42.71	1.73472347597681e-18\\
42.72	0\\
42.73	0\\
42.74	1.73472347597681e-18\\
42.75	0\\
42.76	0\\
42.77	1.73472347597681e-18\\
42.78	0\\
42.79	0\\
42.8	0\\
42.81	0\\
42.82	0\\
42.83	0\\
42.84	0\\
42.85	0\\
42.86	0\\
42.87	0\\
42.88	0\\
42.89	1.73472347597681e-18\\
42.9	0\\
42.91	0\\
42.92	0\\
42.93	0\\
42.94	0\\
42.95	0\\
42.96	0\\
42.97	0\\
42.98	0\\
42.99	0\\
43	1.73472347597681e-18\\
43.01	0\\
43.02	0\\
43.03	0\\
43.04	0\\
43.05	0\\
43.06	0\\
43.07	0\\
43.08	0\\
43.09	1.73472347597681e-18\\
43.1	0\\
43.11	0\\
43.12	0\\
43.13	1.73472347597681e-18\\
43.14	0\\
43.15	0\\
43.16	0\\
43.17	0\\
43.18	0\\
43.19	0\\
43.2	0\\
43.21	0\\
43.22	0\\
43.23	0\\
43.24	0\\
43.25	0\\
43.26	0\\
43.27	0\\
43.28	0\\
43.29	0\\
43.3	0\\
43.31	0\\
43.32	0\\
43.33	1.73472347597681e-18\\
43.34	0\\
43.35	0\\
43.36	0\\
43.37	0\\
43.38	1.73472347597681e-18\\
43.39	0\\
43.4	0\\
43.41	0\\
43.42	0\\
43.43	0\\
43.44	0\\
43.45	0\\
43.46	0\\
43.47	0\\
43.48	0\\
43.49	0\\
43.5	0\\
43.51	0\\
43.52	1.73472347597681e-18\\
43.53	0\\
43.54	0\\
43.55	1.73472347597681e-18\\
43.56	0\\
43.57	0\\
43.58	0\\
43.59	0\\
43.6	0\\
43.61	0\\
43.62	0\\
43.63	0\\
43.64	0\\
43.65	0\\
43.66	0\\
43.67	0\\
43.68	0\\
43.69	0\\
43.7	0\\
43.71	0\\
43.72	1.73472347597681e-18\\
43.73	0\\
43.74	1.73472347597681e-18\\
43.75	0\\
43.76	0\\
43.77	0\\
43.78	0\\
43.79	0\\
43.8	0\\
43.81	1.73472347597681e-18\\
43.82	0\\
43.83	1.73472347597681e-18\\
43.84	1.73472347597681e-18\\
43.85	0\\
43.86	0\\
43.87	0\\
43.88	0\\
43.89	0\\
43.9	0\\
43.91	0\\
43.92	0\\
43.93	1.73472347597681e-18\\
43.94	0\\
43.95	0\\
43.96	1.73472347597681e-18\\
43.97	0\\
43.98	0\\
43.99	0\\
44	0\\
44.01	0\\
44.02	0\\
44.03	1.73472347597681e-18\\
44.04	0\\
44.05	0\\
44.06	0\\
44.07	0\\
44.08	0\\
44.09	0\\
44.1	0\\
44.11	0\\
44.12	0\\
44.13	0\\
44.14	0\\
44.15	0\\
44.16	0\\
44.17	0\\
44.18	0\\
44.19	0\\
44.2	0\\
44.21	0\\
44.22	0\\
44.23	0\\
44.24	0\\
44.25	0\\
44.26	1.73472347597681e-18\\
44.27	0\\
44.28	0\\
44.29	0\\
44.3	0\\
44.31	0\\
44.32	0\\
44.33	1.73472347597681e-18\\
44.34	0\\
44.35	1.73472347597681e-18\\
44.36	1.73472347597681e-18\\
44.37	0\\
44.38	0\\
44.39	0\\
44.4	0\\
44.41	0\\
44.42	0\\
44.43	0\\
44.44	0\\
44.45	0\\
44.46	1.73472347597681e-18\\
44.47	0\\
44.48	0\\
44.49	0\\
44.5	0\\
44.51	0\\
44.52	0\\
44.53	0\\
44.54	0\\
44.55	0\\
44.56	0\\
44.57	0\\
44.58	0\\
44.59	0\\
44.6	0\\
44.61	0\\
44.62	0\\
44.63	0\\
44.64	1.73472347597681e-18\\
44.65	0\\
44.66	0\\
44.67	0\\
44.68	0\\
44.69	1.73472347597681e-18\\
44.7	0\\
44.71	0\\
44.72	0\\
44.73	0\\
44.74	0\\
44.75	0\\
44.76	0\\
44.77	0\\
44.78	0\\
44.79	0\\
44.8	0\\
44.81	0\\
44.82	0\\
44.83	0\\
44.84	1.73472347597681e-18\\
44.85	1.73472347597681e-18\\
44.86	0\\
44.87	0\\
44.88	0\\
44.89	0\\
44.9	0\\
44.91	1.73472347597681e-18\\
44.92	0\\
44.93	0\\
44.94	0\\
44.95	0\\
44.96	0\\
44.97	0\\
44.98	0\\
44.99	0\\
45	0\\
45.01	0\\
45.02	0\\
45.03	0\\
45.04	0\\
45.05	0\\
45.06	0\\
45.07	0\\
45.08	0\\
45.09	0\\
45.1	0\\
45.11	0\\
45.12	0\\
45.13	1.73472347597681e-18\\
45.14	0\\
45.15	0\\
45.16	0\\
45.17	0\\
45.18	0\\
45.19	0\\
45.2	0\\
45.21	0\\
45.22	0\\
45.23	0\\
45.24	0\\
45.25	0\\
45.26	0\\
45.27	0\\
45.28	0\\
45.29	0\\
45.3	0\\
45.31	0\\
45.32	0\\
45.33	0\\
45.34	0\\
45.35	0\\
45.36	0\\
45.37	0\\
45.38	0\\
45.39	0\\
45.4	1.73472347597681e-18\\
45.41	0\\
45.42	0\\
45.43	0\\
45.44	0\\
45.45	0\\
45.46	0\\
45.47	0\\
45.48	0\\
45.49	0\\
45.5	0\\
45.51	0\\
45.52	0\\
45.53	0\\
45.54	0\\
45.55	0\\
45.56	0\\
45.57	0\\
45.58	0\\
45.59	0\\
45.6	1.73472347597681e-18\\
45.61	0\\
45.62	0\\
45.63	1.73472347597681e-18\\
45.64	0\\
45.65	0\\
45.66	0\\
45.67	0\\
45.68	0\\
45.69	0\\
45.7	0\\
45.71	1.73472347597681e-18\\
45.72	0\\
45.73	0\\
45.74	0\\
45.75	0\\
45.76	0\\
45.77	0\\
45.78	0\\
45.79	0\\
45.8	0\\
45.81	0\\
45.82	0\\
45.83	1.73472347597681e-18\\
45.84	0\\
45.85	0\\
45.86	0\\
45.87	0\\
45.88	0\\
45.89	0\\
45.9	0\\
45.91	0\\
45.92	0\\
45.93	1.73472347597681e-18\\
45.94	0\\
45.95	0\\
45.96	0\\
45.97	0\\
45.98	0\\
45.99	0\\
46	0\\
46.01	0\\
46.02	0\\
46.03	0\\
46.04	0\\
46.05	0\\
46.06	0\\
46.07	0\\
46.08	1.73472347597681e-18\\
46.09	0\\
46.1	0\\
46.11	0\\
46.12	0\\
46.13	0\\
46.14	0\\
46.15	0\\
46.16	0\\
46.17	0\\
46.18	0\\
46.19	0\\
46.2	0\\
46.21	0\\
46.22	0\\
46.23	0\\
46.24	0\\
46.25	1.73472347597681e-18\\
46.26	0\\
46.27	0\\
46.28	0\\
46.29	1.73472347597681e-18\\
46.3	1.73472347597681e-18\\
46.31	0\\
46.32	1.73472347597681e-18\\
46.33	0\\
46.34	0\\
46.35	0\\
46.36	0\\
46.37	0\\
46.38	1.73472347597681e-18\\
46.39	0\\
46.4	0\\
46.41	0\\
46.42	0\\
46.43	0\\
46.44	0\\
46.45	0\\
46.46	0\\
46.47	0\\
46.48	0\\
46.49	0\\
46.5	0\\
46.51	0\\
46.52	0\\
46.53	0\\
46.54	0\\
46.55	0\\
46.56	0\\
46.57	0\\
46.58	0\\
46.59	1.73472347597681e-18\\
46.6	1.73472347597681e-18\\
46.61	0\\
46.62	0\\
46.63	0\\
46.64	0\\
46.65	0\\
46.66	0\\
46.67	0\\
46.68	1.73472347597681e-18\\
46.69	0\\
46.7	0\\
46.71	0\\
46.72	0\\
46.73	0\\
46.74	0\\
46.75	0\\
46.76	0\\
46.77	0\\
46.78	0\\
46.79	0\\
46.8	0\\
46.81	0\\
46.82	0\\
46.83	0\\
46.84	0\\
46.85	0\\
46.86	0\\
46.87	1.73472347597681e-18\\
46.88	0\\
46.89	1.73472347597681e-18\\
46.9	0\\
46.91	0\\
46.92	0\\
46.93	0\\
46.94	0\\
46.95	0\\
46.96	0\\
46.97	0\\
46.98	0\\
46.99	0\\
47	0\\
47.01	0\\
47.02	0\\
47.03	0\\
47.04	0\\
47.05	1.73472347597681e-18\\
47.06	0\\
47.07	0\\
47.08	0\\
47.09	0\\
47.1	1.73472347597681e-18\\
47.11	0\\
47.12	0\\
47.13	0\\
47.14	1.73472347597681e-18\\
47.15	0\\
47.16	1.73472347597681e-18\\
47.17	0\\
47.18	0\\
47.19	0\\
47.2	0\\
47.21	0\\
47.22	0\\
47.23	0\\
47.24	0\\
47.25	0\\
47.26	0\\
47.27	0\\
47.28	0\\
47.29	0\\
47.3	0\\
47.31	0\\
47.32	0\\
47.33	0\\
47.34	0\\
47.35	0\\
47.36	1.73472347597681e-18\\
47.37	0\\
47.38	1.73472347597681e-18\\
47.39	0\\
47.4	0\\
47.41	0\\
47.42	0\\
47.43	0\\
47.44	0\\
47.45	0\\
47.46	1.73472347597681e-18\\
47.47	0\\
47.48	0\\
47.49	0\\
47.5	0\\
47.51	1.73472347597681e-18\\
47.52	1.73472347597681e-18\\
47.53	0\\
47.54	0\\
47.55	0\\
47.56	0\\
47.57	1.73472347597681e-18\\
47.58	1.73472347597681e-18\\
47.59	0\\
47.6	0\\
47.61	0\\
47.62	0\\
47.63	0\\
47.64	0\\
47.65	0\\
47.66	0\\
47.67	0\\
47.68	0\\
47.69	0\\
47.7	0\\
47.71	0\\
47.72	0\\
47.73	0\\
47.74	0\\
47.75	0\\
47.76	1.73472347597681e-18\\
47.77	0\\
47.78	0\\
47.79	0\\
47.8	0\\
47.81	0\\
47.82	0\\
47.83	0\\
47.84	1.73472347597681e-18\\
47.85	0\\
47.86	0\\
47.87	0\\
47.88	0\\
47.89	0\\
47.9	0\\
47.91	0\\
47.92	0\\
47.93	0\\
47.94	0\\
47.95	0\\
47.96	0\\
47.97	1.73472347597681e-18\\
47.98	0\\
47.99	0\\
48	1.73472347597681e-18\\
48.01	0\\
48.02	0\\
48.03	0\\
48.04	0\\
48.05	0\\
48.06	0\\
48.07	1.73472347597681e-18\\
48.08	1.73472347597681e-18\\
48.09	0\\
48.1	0\\
48.11	1.73472347597681e-18\\
48.12	0\\
48.13	0\\
48.14	0\\
48.15	0\\
48.16	0\\
48.17	0\\
48.18	0\\
48.19	0\\
48.2	0\\
48.21	0\\
48.22	0\\
48.23	0\\
48.24	0\\
48.25	0\\
48.26	0\\
48.27	0\\
48.28	0\\
48.29	0\\
48.3	0\\
48.31	0\\
48.32	0\\
48.33	0\\
48.34	0\\
48.35	0\\
48.36	0\\
48.37	0\\
48.38	0\\
48.39	0\\
48.4	0\\
48.41	0\\
48.42	0\\
48.43	0\\
48.44	0\\
48.45	0\\
48.46	0\\
48.47	0\\
48.48	0\\
48.49	0\\
48.5	1.73472347597681e-18\\
48.51	0\\
48.52	0\\
48.53	0\\
48.54	0\\
48.55	0\\
48.56	0\\
48.57	0\\
48.58	0\\
48.59	0\\
48.6	0\\
48.61	0\\
48.62	0\\
48.63	0\\
48.64	0\\
48.65	0\\
48.66	0\\
48.67	0\\
48.68	0\\
48.69	0\\
48.7	0\\
48.71	0\\
48.72	0\\
48.73	1.73472347597681e-18\\
48.74	0\\
48.75	0\\
48.76	0\\
48.77	0\\
48.78	0\\
48.79	0\\
48.8	1.73472347597681e-18\\
48.81	0\\
48.82	0\\
48.83	0\\
48.84	0\\
48.85	0\\
48.86	0\\
48.87	1.73472347597681e-18\\
48.88	1.73472347597681e-18\\
48.89	0\\
48.9	0\\
48.91	0\\
48.92	0\\
48.93	0\\
48.94	0\\
48.95	1.73472347597681e-18\\
48.96	1.73472347597681e-18\\
48.97	0\\
48.98	0\\
48.99	0\\
49	0\\
49.01	0\\
49.02	0\\
49.03	0\\
49.04	1.73472347597681e-18\\
49.05	1.73472347597681e-18\\
49.06	0\\
49.07	0\\
49.08	0\\
49.09	0\\
49.1	0\\
49.11	0\\
49.12	0\\
49.13	1.73472347597681e-18\\
49.14	0\\
49.15	0\\
49.16	0\\
49.17	0\\
49.18	0\\
49.19	0\\
49.2	0\\
49.21	0\\
49.22	0\\
49.23	0\\
49.24	0\\
49.25	0\\
49.26	0\\
49.27	0\\
49.28	0\\
49.29	0\\
49.3	0\\
49.31	0\\
49.32	1.73472347597681e-18\\
49.33	0\\
49.34	0\\
49.35	0\\
49.36	0\\
49.37	0\\
49.38	0\\
49.39	0\\
49.4	0\\
49.41	0\\
49.42	0\\
49.43	0\\
49.44	0\\
49.45	0\\
49.46	0\\
49.47	0\\
49.48	0\\
49.49	0\\
49.5	0\\
49.51	0\\
49.52	0\\
49.53	0\\
49.54	0\\
49.55	0\\
49.56	0\\
49.57	0\\
49.58	0\\
49.59	0\\
49.6	0\\
49.61	0\\
49.62	0\\
49.63	0\\
49.64	0\\
49.65	0\\
49.66	0\\
49.67	1.73472347597681e-18\\
49.68	1.73472347597681e-18\\
49.69	0\\
49.7	0\\
49.71	0\\
49.72	0\\
49.73	0\\
49.74	0\\
49.75	0\\
49.76	0\\
49.77	0\\
49.78	0\\
49.79	0\\
49.8	0\\
49.81	0\\
49.82	0\\
49.83	0\\
49.84	0\\
49.85	0\\
49.86	0\\
49.87	0\\
49.88	0\\
49.89	0\\
49.9	0\\
49.91	0\\
49.92	0\\
49.93	0\\
49.94	1.73472347597681e-18\\
49.95	0\\
49.96	0\\
49.97	0\\
49.98	0\\
49.99	0\\
50	0\\
50.01	0\\
50.02	0\\
50.03	0\\
50.04	0\\
50.05	0\\
50.06	0\\
50.07	0\\
50.08	0\\
50.09	0\\
50.1	0\\
50.11	0\\
50.12	0\\
50.13	0\\
50.14	1.73472347597681e-18\\
50.15	0\\
50.16	0\\
50.17	0\\
50.18	0\\
50.19	0\\
50.2	0\\
50.21	1.73472347597681e-18\\
50.22	0\\
50.23	0\\
50.24	0\\
50.25	0\\
50.26	0\\
50.27	0\\
50.28	0\\
50.29	0\\
50.3	0\\
50.31	1.73472347597681e-18\\
50.32	0\\
50.33	0\\
50.34	0\\
50.35	0\\
50.36	0\\
50.37	1.73472347597681e-18\\
50.38	0\\
50.39	0\\
50.4	0\\
50.41	0\\
50.42	0\\
50.43	0\\
50.44	0\\
50.45	0\\
50.46	0\\
50.47	1.73472347597681e-18\\
50.48	0\\
50.49	0\\
50.5	0\\
50.51	1.73472347597681e-18\\
50.52	0\\
50.53	0\\
50.54	0\\
50.55	0\\
50.56	1.73472347597681e-18\\
50.57	0\\
50.58	0\\
50.59	0\\
50.6	0\\
50.61	0\\
50.62	0\\
50.63	0\\
50.64	0\\
50.65	0\\
50.66	0\\
50.67	0\\
50.68	0\\
50.69	0\\
50.7	0\\
50.71	0\\
50.72	0\\
50.73	0\\
50.74	0\\
50.75	0\\
50.76	0\\
50.77	0\\
50.78	1.73472347597681e-18\\
50.79	0\\
50.8	0\\
50.81	0\\
50.82	0\\
50.83	0\\
50.84	0\\
50.85	0\\
50.86	0\\
50.87	1.73472347597681e-18\\
50.88	1.73472347597681e-18\\
50.89	0\\
50.9	0\\
50.91	0\\
50.92	0\\
50.93	1.73472347597681e-18\\
50.94	0\\
50.95	0\\
50.96	1.73472347597681e-18\\
50.97	0\\
50.98	0\\
50.99	0\\
51	0\\
51.01	1.73472347597681e-18\\
51.02	0\\
51.03	0\\
51.04	0\\
51.05	0\\
51.06	0\\
51.07	0\\
51.08	0\\
51.09	0\\
51.1	0\\
51.11	0\\
51.12	0\\
51.13	0\\
51.14	0\\
51.15	0\\
51.16	0\\
51.17	0\\
51.18	0\\
51.19	1.73472347597681e-18\\
51.2	0\\
51.21	0\\
51.22	0\\
51.23	1.73472347597681e-18\\
51.24	0\\
51.25	0\\
51.26	0\\
51.27	0\\
51.28	0\\
51.29	0\\
51.3	0\\
51.31	0\\
51.32	1.73472347597681e-18\\
51.33	0\\
51.34	0\\
51.35	0\\
51.36	0\\
51.37	1.73472347597681e-18\\
51.38	0\\
51.39	0\\
51.4	0\\
51.41	0\\
51.42	1.73472347597681e-18\\
51.43	0\\
51.44	0\\
51.45	0\\
51.46	1.73472347597681e-18\\
51.47	0\\
51.48	0\\
51.49	1.73472347597681e-18\\
51.5	0\\
51.51	0\\
51.52	0\\
51.53	0\\
51.54	0\\
51.55	0\\
51.56	0\\
51.57	0\\
51.58	1.73472347597681e-18\\
51.59	0\\
51.6	0\\
51.61	0\\
51.62	0\\
51.63	0\\
51.64	0\\
51.65	0\\
51.66	0\\
51.67	0\\
51.68	1.73472347597681e-18\\
51.69	0\\
51.7	1.73472347597681e-18\\
51.71	0\\
51.72	0\\
51.73	0\\
51.74	0\\
51.75	0\\
51.76	0\\
51.77	0\\
51.78	0\\
51.79	0\\
51.8	0\\
51.81	0\\
51.82	0\\
51.83	0\\
51.84	0\\
51.85	0\\
51.86	0\\
51.87	0\\
51.88	0\\
51.89	0\\
51.9	0\\
51.91	0\\
51.92	0\\
51.93	1.73472347597681e-18\\
51.94	0\\
51.95	0\\
51.96	0\\
51.97	0\\
51.98	0\\
51.99	0\\
52	0\\
52.01	0\\
52.02	0\\
52.03	0\\
52.04	0\\
52.05	0\\
52.06	0\\
52.07	0\\
52.08	0\\
52.09	0\\
52.1	0\\
52.11	0\\
52.12	0\\
52.13	0\\
52.14	0\\
52.15	0\\
52.16	1.73472347597681e-18\\
52.17	1.73472347597681e-18\\
52.18	0\\
52.19	0\\
52.2	0\\
52.21	0\\
52.22	1.73472347597681e-18\\
52.23	0\\
52.24	0\\
52.25	0\\
52.26	0\\
52.27	0\\
52.28	0\\
52.29	0\\
52.3	0\\
52.31	0\\
52.32	0\\
52.33	0\\
52.34	0\\
52.35	0\\
52.36	0\\
52.37	1.73472347597681e-18\\
52.38	0\\
52.39	1.73472347597681e-18\\
52.4	1.73472347597681e-18\\
52.41	0\\
52.42	0\\
52.43	0\\
52.44	0\\
52.45	0\\
52.46	0\\
52.47	0\\
52.48	0\\
52.49	0\\
52.5	0\\
52.51	0\\
52.52	0\\
52.53	0\\
52.54	0\\
52.55	0\\
52.56	0\\
52.57	0\\
52.58	0\\
52.59	0\\
52.6	0\\
52.61	1.73472347597681e-18\\
52.62	1.73472347597681e-18\\
52.63	0\\
52.64	1.73472347597681e-18\\
52.65	0\\
52.66	0\\
52.67	0\\
52.68	0\\
52.69	0\\
52.7	0\\
52.71	0\\
52.72	1.73472347597681e-18\\
52.73	0\\
52.74	0\\
52.75	0\\
52.76	0\\
52.77	0\\
52.78	0\\
52.79	0\\
52.8	0\\
52.81	0\\
52.82	0\\
52.83	0\\
52.84	0\\
52.85	1.73472347597681e-18\\
52.86	0\\
52.87	1.73472347597681e-18\\
52.88	0\\
52.89	0\\
52.9	1.73472347597681e-18\\
52.91	0\\
52.92	0\\
52.93	0\\
52.94	0\\
52.95	1.73472347597681e-18\\
52.96	0\\
52.97	0\\
52.98	0\\
52.99	0\\
53	0\\
53.01	1.73472347597681e-18\\
53.02	0\\
53.03	0\\
53.04	0\\
53.05	0\\
53.06	0\\
53.07	0\\
53.08	0\\
53.09	0\\
53.1	1.73472347597681e-18\\
53.11	1.73472347597681e-18\\
53.12	0\\
53.13	0\\
53.14	0\\
53.15	0\\
53.16	0\\
53.17	0\\
53.18	0\\
53.19	0\\
53.2	1.73472347597681e-18\\
53.21	0\\
53.22	0\\
53.23	0\\
53.24	0\\
53.25	0\\
53.26	0\\
53.27	1.73472347597681e-18\\
53.28	0\\
53.29	0\\
53.3	0\\
53.31	0\\
53.32	0\\
53.33	0\\
53.34	0\\
53.35	0\\
53.36	0\\
53.37	0\\
53.38	0\\
53.39	0\\
53.4	0\\
53.41	0\\
53.42	0\\
53.43	0\\
53.44	0\\
53.45	1.73472347597681e-18\\
53.46	0\\
53.47	0\\
53.48	0\\
53.49	0\\
53.5	0\\
53.51	0\\
53.52	1.73472347597681e-18\\
53.53	1.73472347597681e-18\\
53.54	0\\
53.55	0\\
53.56	0\\
53.57	0\\
53.58	1.73472347597681e-18\\
53.59	0\\
53.6	0\\
53.61	0\\
53.62	0\\
53.63	0\\
53.64	0\\
53.65	0\\
53.66	0\\
53.67	0\\
53.68	0\\
53.69	0\\
53.7	0\\
53.71	0\\
53.72	0\\
53.73	0\\
53.74	0\\
53.75	0\\
53.76	0\\
53.77	0\\
53.78	0\\
53.79	1.73472347597681e-18\\
53.8	0\\
53.81	0\\
53.82	0\\
53.83	0\\
53.84	0\\
53.85	1.73472347597681e-18\\
53.86	0\\
53.87	0\\
53.88	0\\
53.89	0\\
53.9	0\\
53.91	0\\
53.92	0\\
53.93	0\\
53.94	0\\
53.95	0\\
53.96	0\\
53.97	0\\
53.98	0\\
53.99	0\\
54	0\\
54.01	0\\
54.02	0\\
54.03	0\\
54.04	0\\
54.05	0\\
54.06	0\\
54.07	1.73472347597681e-18\\
54.08	0\\
54.09	1.73472347597681e-18\\
54.1	0\\
54.11	1.73472347597681e-18\\
54.12	0\\
54.13	1.73472347597681e-18\\
54.14	0\\
54.15	1.73472347597681e-18\\
54.16	0\\
54.17	0\\
54.18	0\\
54.19	0\\
54.2	1.73472347597681e-18\\
54.21	1.73472347597681e-18\\
54.22	0\\
54.23	0\\
54.24	0\\
54.25	0\\
54.26	0\\
54.27	0\\
54.28	0\\
54.29	0\\
54.3	0\\
54.31	0\\
54.32	1.73472347597681e-18\\
54.33	0\\
54.34	0\\
54.35	0\\
54.36	0\\
54.37	0\\
54.38	0\\
54.39	0\\
54.4	0\\
54.41	0\\
54.42	0\\
54.43	0\\
54.44	0\\
54.45	1.73472347597681e-18\\
54.46	0\\
54.47	0\\
54.48	0\\
54.49	0\\
54.5	0\\
54.51	0\\
54.52	0\\
54.53	0\\
54.54	0\\
54.55	0\\
54.56	0\\
54.57	0\\
54.58	0\\
54.59	0\\
54.6	0\\
54.61	0\\
54.62	0\\
54.63	0\\
54.64	0\\
54.65	1.73472347597681e-18\\
54.66	0\\
54.67	0\\
54.68	0\\
54.69	1.73472347597681e-18\\
54.7	0\\
54.71	0\\
54.72	0\\
54.73	0\\
54.74	0\\
54.75	0\\
54.76	0\\
54.77	0\\
54.78	0\\
54.79	0\\
54.8	0\\
54.81	0\\
54.82	0\\
54.83	0\\
54.84	0\\
54.85	0\\
54.86	0\\
54.87	0\\
54.88	0\\
54.89	0\\
54.9	1.73472347597681e-18\\
54.91	0\\
54.92	0\\
54.93	0\\
54.94	1.73472347597681e-18\\
54.95	0\\
54.96	0\\
54.97	0\\
54.98	1.73472347597681e-18\\
54.99	0\\
55	0\\
55.01	0\\
55.02	0\\
55.03	0\\
55.04	0\\
55.05	0\\
55.06	0\\
55.07	1.73472347597681e-18\\
55.08	1.73472347597681e-18\\
55.09	0\\
55.1	0\\
55.11	0\\
55.12	0\\
55.13	0\\
55.14	1.73472347597681e-18\\
55.15	0\\
55.16	1.73472347597681e-18\\
55.17	1.73472347597681e-18\\
55.18	0\\
55.19	1.73472347597681e-18\\
55.2	0\\
55.21	0\\
55.22	0\\
55.23	0\\
55.24	0\\
55.25	0\\
55.26	0\\
55.27	0\\
55.28	0\\
55.29	0\\
55.3	0\\
55.31	0\\
55.32	0\\
55.33	0\\
55.34	0\\
55.35	0\\
55.36	1.73472347597681e-18\\
55.37	0\\
55.38	0\\
55.39	0\\
55.4	0\\
55.41	0\\
55.42	0\\
55.43	0\\
55.44	0\\
55.45	0\\
55.46	0\\
55.47	1.73472347597681e-18\\
55.48	0\\
55.49	1.73472347597681e-18\\
55.5	0\\
55.51	0\\
55.52	0\\
55.53	0\\
55.54	0\\
55.55	0\\
55.56	1.73472347597681e-18\\
55.57	0\\
55.58	0\\
55.59	0\\
55.6	0\\
55.61	0\\
55.62	0\\
55.63	0\\
55.64	0\\
55.65	0\\
55.66	0\\
55.67	0\\
55.68	0\\
55.69	0\\
55.7	0\\
55.71	0\\
55.72	0\\
55.73	1.73472347597681e-18\\
55.74	0\\
55.75	0\\
55.76	0\\
55.77	1.73472347597681e-18\\
55.78	0\\
55.79	0\\
55.8	0\\
55.81	0\\
55.82	0\\
55.83	0\\
55.84	0\\
55.85	0\\
55.86	0\\
55.87	0\\
55.88	0\\
55.89	0\\
55.9	0\\
55.91	0\\
55.92	0\\
55.93	0\\
55.94	0\\
55.95	1.73472347597681e-18\\
55.96	0\\
55.97	0\\
55.98	0\\
55.99	0\\
56	0\\
56.01	1.73472347597681e-18\\
56.02	0\\
56.03	0\\
56.04	1.73472347597681e-18\\
56.05	1.73472347597681e-18\\
56.06	0\\
56.07	1.73472347597681e-18\\
56.08	0\\
56.09	0\\
56.1	0\\
56.11	0\\
56.12	0\\
56.13	0\\
56.14	0\\
56.15	0\\
56.16	0\\
56.17	0\\
56.18	0\\
56.19	0\\
56.2	0\\
56.21	0\\
56.22	0\\
56.23	0\\
56.24	1.73472347597681e-18\\
56.25	0\\
56.26	0\\
56.27	0\\
56.28	0\\
56.29	1.73472347597681e-18\\
56.3	0\\
56.31	0\\
56.32	0\\
56.33	0\\
56.34	0\\
56.35	0\\
56.36	0\\
56.37	0\\
56.38	0\\
56.39	0\\
56.4	0\\
56.41	0\\
56.42	0\\
56.43	0\\
56.44	0\\
56.45	0\\
56.46	0\\
56.47	0\\
56.48	0\\
56.49	0\\
56.5	0\\
56.51	0\\
56.52	0\\
56.53	0\\
56.54	0\\
56.55	0\\
56.56	0\\
56.57	0\\
56.58	0\\
56.59	0\\
56.6	0\\
56.61	0\\
56.62	0\\
56.63	0\\
56.64	0\\
56.65	0\\
56.66	0\\
56.67	0\\
56.68	0\\
56.69	0\\
56.7	1.73472347597681e-18\\
56.71	1.73472347597681e-18\\
56.72	0\\
56.73	0\\
56.74	1.73472347597681e-18\\
56.75	1.73472347597681e-18\\
56.76	0\\
56.77	0\\
56.78	0\\
56.79	1.73472347597681e-18\\
56.8	0\\
56.81	0\\
56.82	1.73472347597681e-18\\
56.83	0\\
56.84	0\\
56.85	0\\
56.86	0\\
56.87	0\\
56.88	0\\
56.89	0\\
56.9	0\\
56.91	0\\
56.92	0\\
56.93	0\\
56.94	0\\
56.95	0\\
56.96	0\\
56.97	0\\
56.98	0\\
56.99	0\\
57	1.73472347597681e-18\\
57.01	0\\
57.02	0\\
57.03	0\\
57.04	0\\
57.05	1.73472347597681e-18\\
57.06	1.73472347597681e-18\\
57.07	0\\
57.08	0\\
57.09	0\\
57.1	0\\
57.11	0\\
57.12	0\\
57.13	0\\
57.14	0\\
57.15	0\\
57.16	0\\
57.17	0\\
57.18	0\\
57.19	0\\
57.2	0\\
57.21	0\\
57.22	0\\
57.23	0\\
57.24	0\\
57.25	0\\
57.26	0\\
57.27	1.73472347597681e-18\\
57.28	0\\
57.29	0\\
57.3	0\\
57.31	0\\
57.32	0\\
57.33	0\\
57.34	0\\
57.35	0\\
57.36	0\\
57.37	0\\
57.38	1.73472347597681e-18\\
57.39	0\\
57.4	0\\
57.41	0\\
57.42	0\\
57.43	0\\
57.44	0\\
57.45	0\\
57.46	0\\
57.47	0\\
57.48	0\\
57.49	0\\
57.5	0\\
57.51	1.73472347597681e-18\\
57.52	0\\
57.53	0\\
57.54	0\\
57.55	0\\
57.56	0\\
57.57	0\\
57.58	0\\
57.59	0\\
57.6	0\\
57.61	0\\
57.62	0\\
57.63	0\\
57.64	0\\
57.65	0\\
57.66	0\\
57.67	0\\
57.68	1.73472347597681e-18\\
57.69	1.73472347597681e-18\\
57.7	0\\
57.71	0\\
57.72	0\\
57.73	0\\
57.74	0\\
57.75	0\\
57.76	0\\
57.77	1.73472347597681e-18\\
57.78	0\\
57.79	0\\
57.8	0\\
57.81	0\\
57.82	0\\
57.83	0\\
57.84	1.73472347597681e-18\\
57.85	1.73472347597681e-18\\
57.86	0\\
57.87	1.73472347597681e-18\\
57.88	0\\
57.89	0\\
57.9	0\\
57.91	0\\
57.92	0\\
57.93	0\\
57.94	0\\
57.95	0\\
57.96	0\\
57.97	0\\
57.98	0\\
57.99	0\\
58	1.73472347597681e-18\\
58.01	0\\
58.02	0\\
58.03	0\\
58.04	0\\
58.05	0\\
58.06	0\\
58.07	0\\
58.08	0\\
58.09	0\\
58.1	0\\
58.11	1.73472347597681e-18\\
58.12	0\\
58.13	1.73472347597681e-18\\
58.14	0\\
58.15	0\\
58.16	0\\
58.17	0\\
58.18	0\\
58.19	0\\
58.2	0\\
58.21	0\\
58.22	1.73472347597681e-18\\
58.23	0\\
58.24	0\\
58.25	1.73472347597681e-18\\
58.26	0\\
58.27	0\\
58.28	0\\
58.29	1.73472347597681e-18\\
58.3	0\\
58.31	1.73472347597681e-18\\
58.32	0\\
58.33	0\\
58.34	1.73472347597681e-18\\
58.35	1.73472347597681e-18\\
58.36	0\\
58.37	0\\
58.38	0\\
58.39	0\\
58.4	0\\
58.41	1.73472347597681e-18\\
58.42	1.73472347597681e-18\\
58.43	0\\
58.44	0\\
58.45	0\\
58.46	0\\
58.47	0\\
58.48	0\\
58.49	0\\
58.5	0\\
58.51	0\\
58.52	1.73472347597681e-18\\
58.53	0\\
58.54	0\\
58.55	0\\
58.56	0\\
58.57	0\\
58.58	0\\
58.59	0\\
58.6	0\\
58.61	0\\
58.62	0\\
58.63	0\\
58.64	0\\
58.65	0\\
58.66	0\\
58.67	1.73472347597681e-18\\
58.68	0\\
58.69	0\\
58.7	0\\
58.71	0\\
58.72	0\\
58.73	0\\
58.74	1.73472347597681e-18\\
58.75	0\\
58.76	0\\
58.77	0\\
58.78	0\\
58.79	1.73472347597681e-18\\
58.8	0\\
58.81	0\\
58.82	0\\
58.83	0\\
58.84	0\\
58.85	0\\
58.86	0\\
58.87	0\\
58.88	1.73472347597681e-18\\
58.89	0\\
58.9	0\\
58.91	0\\
58.92	0\\
58.93	0\\
58.94	0\\
58.95	0\\
58.96	0\\
58.97	0\\
58.98	1.73472347597681e-18\\
58.99	0\\
59	0\\
59.01	0\\
59.02	0\\
59.03	0\\
59.04	0\\
59.05	0\\
59.06	0\\
59.07	0\\
59.08	0\\
59.09	0\\
59.1	0\\
59.11	0\\
59.12	0\\
59.13	0\\
59.14	0\\
59.15	0\\
59.16	0\\
59.17	1.73472347597681e-18\\
59.18	0\\
59.19	0\\
59.2	1.73472347597681e-18\\
59.21	0\\
59.22	0\\
59.23	0\\
59.24	0\\
59.25	1.73472347597681e-18\\
59.26	0\\
59.27	0\\
59.28	0\\
59.29	0\\
59.3	1.73472347597681e-18\\
59.31	0\\
59.32	1.73472347597681e-18\\
59.33	0\\
59.34	0\\
59.35	0\\
59.36	0\\
59.37	0\\
59.38	0\\
59.39	0\\
59.4	0\\
59.41	1.73472347597681e-18\\
59.42	1.73472347597681e-18\\
59.43	0\\
59.44	0\\
59.45	0\\
59.46	0\\
59.47	0\\
59.48	0\\
59.49	0\\
59.5	1.73472347597681e-18\\
59.51	1.73472347597681e-18\\
59.52	0\\
59.53	0\\
59.54	0\\
59.55	0\\
59.56	0\\
59.57	0\\
59.58	0\\
59.59	1.73472347597681e-18\\
59.6	0\\
59.61	0\\
59.62	1.73472347597681e-18\\
59.63	0\\
59.64	0\\
59.65	0\\
59.66	1.73472347597681e-18\\
59.67	0\\
59.68	0\\
59.69	0\\
59.7	0\\
59.71	0\\
59.72	0\\
59.73	0\\
59.74	0\\
59.75	0\\
59.76	0\\
59.77	0\\
59.78	0\\
59.79	0\\
59.8	1.73472347597681e-18\\
59.81	0\\
59.82	0\\
59.83	0\\
59.84	0\\
59.85	0\\
59.86	1.73472347597681e-18\\
59.87	0\\
59.88	1.73472347597681e-18\\
59.89	0\\
59.9	0\\
59.91	0\\
59.92	0\\
59.93	1.73472347597681e-18\\
59.94	0\\
59.95	0\\
59.96	0\\
59.97	0\\
59.98	0\\
59.99	0\\
60	0\\
60.01	1.73472347597681e-18\\
60.02	0\\
60.03	0\\
60.04	0\\
60.05	0\\
60.06	0\\
60.07	0\\
60.08	0\\
60.09	0\\
60.1	0\\
60.11	0\\
60.12	0\\
60.13	1.73472347597681e-18\\
60.14	0\\
60.15	0\\
60.16	0\\
60.17	0\\
60.18	0\\
60.19	0\\
60.2	0\\
60.21	0\\
60.22	0\\
60.23	0\\
60.24	0\\
60.25	0\\
60.26	0\\
60.27	0\\
60.28	0\\
60.29	0\\
60.3	0\\
60.31	0\\
60.32	0\\
60.33	0\\
60.34	0\\
60.35	0\\
60.36	0\\
60.37	0\\
60.38	0\\
60.39	1.73472347597681e-18\\
60.4	0\\
60.41	0\\
60.42	0\\
60.43	0\\
60.44	0\\
60.45	0\\
60.46	0\\
60.47	0\\
60.48	0\\
60.49	1.73472347597681e-18\\
60.5	0\\
60.51	0\\
60.52	0\\
60.53	0\\
60.54	0\\
60.55	0\\
60.56	0\\
60.57	0\\
60.58	0\\
60.59	0\\
60.6	0\\
60.61	0\\
60.62	0\\
60.63	0\\
60.64	0\\
60.65	0\\
60.66	1.73472347597681e-18\\
60.67	0\\
60.68	0\\
60.69	0\\
60.7	0\\
60.71	0\\
60.72	0\\
60.73	0\\
60.74	0\\
60.75	0\\
60.76	0\\
60.77	1.73472347597681e-18\\
60.78	0\\
60.79	0\\
60.8	0\\
60.81	0\\
60.82	0\\
60.83	0\\
60.84	0\\
60.85	0\\
60.86	0\\
60.87	0\\
60.88	0\\
60.89	0\\
60.9	0\\
60.91	0\\
60.92	0\\
60.93	0\\
60.94	0\\
60.95	0\\
60.96	0\\
60.97	0\\
60.98	0\\
60.99	0\\
61	0\\
61.01	1.73472347597681e-18\\
61.02	0\\
61.03	0\\
61.04	0\\
61.05	0\\
61.06	1.73472347597681e-18\\
61.07	0\\
61.08	0\\
61.09	0\\
61.1	0\\
61.11	0\\
61.12	0\\
61.13	0\\
61.14	1.73472347597681e-18\\
61.15	0\\
61.16	0\\
61.17	0\\
61.18	0\\
61.19	0\\
61.2	0\\
61.21	0\\
61.22	0\\
61.23	0\\
61.24	0\\
61.25	0\\
61.26	0\\
61.27	0\\
61.28	1.73472347597681e-18\\
61.29	0\\
61.3	0\\
61.31	0\\
61.32	0\\
61.33	0\\
61.34	0\\
61.35	0\\
61.36	0\\
61.37	0\\
61.38	0\\
61.39	0\\
61.4	0\\
61.41	0\\
61.42	0\\
61.43	1.73472347597681e-18\\
61.44	0\\
61.45	0\\
61.46	1.73472347597681e-18\\
61.47	1.73472347597681e-18\\
61.48	0\\
61.49	1.73472347597681e-18\\
61.5	0\\
61.51	0\\
61.52	0\\
61.53	0\\
61.54	0\\
61.55	0\\
61.56	0\\
61.57	0\\
61.58	0\\
61.59	0\\
61.6	0\\
61.61	0\\
61.62	0\\
61.63	0\\
61.64	0\\
61.65	1.73472347597681e-18\\
61.66	0\\
61.67	1.73472347597681e-18\\
61.68	0\\
61.69	0\\
61.7	0\\
61.71	0\\
61.72	0\\
61.73	0\\
61.74	0\\
61.75	0\\
61.76	0\\
61.77	1.73472347597681e-18\\
61.78	0\\
61.79	0\\
61.8	0\\
61.81	0\\
61.82	1.73472347597681e-18\\
61.83	0\\
61.84	1.73472347597681e-18\\
61.85	0\\
61.86	0\\
61.87	0\\
61.88	1.73472347597681e-18\\
61.89	0\\
61.9	0\\
61.91	0\\
61.92	0\\
61.93	0\\
61.94	0\\
61.95	0\\
61.96	0\\
61.97	1.73472347597681e-18\\
61.98	0\\
61.99	0\\
62	0\\
62.01	0\\
62.02	0\\
62.03	0\\
62.04	1.73472347597681e-18\\
62.05	0\\
62.06	0\\
62.07	0\\
62.08	0\\
62.09	0\\
62.1	0\\
62.11	0\\
62.12	0\\
62.13	0\\
62.14	0\\
62.15	0\\
62.16	0\\
62.17	1.73472347597681e-18\\
62.18	0\\
62.19	0\\
62.2	0\\
62.21	0\\
62.22	0\\
62.23	0\\
62.24	0\\
62.25	0\\
62.26	0\\
62.27	0\\
62.28	0\\
62.29	0\\
62.3	0\\
62.31	0\\
62.32	0\\
62.33	0\\
62.34	0\\
62.35	0\\
62.36	0\\
62.37	0\\
62.38	0\\
62.39	0\\
62.4	0\\
62.41	0\\
62.42	0\\
62.43	0\\
62.44	0\\
62.45	0\\
62.46	1.73472347597681e-18\\
62.47	0\\
62.48	0\\
62.49	1.73472347597681e-18\\
62.5	0\\
62.51	0\\
62.52	0\\
62.53	0\\
62.54	0\\
62.55	0\\
62.56	0\\
62.57	0\\
62.58	0\\
62.59	0\\
62.6	0\\
62.61	0\\
62.62	0\\
62.63	0\\
62.64	0\\
62.65	0\\
62.66	0\\
62.67	0\\
62.68	0\\
62.69	0\\
62.7	0\\
62.71	0\\
62.72	0\\
62.73	0\\
62.74	0\\
62.75	0\\
62.76	0\\
62.77	1.73472347597681e-18\\
62.78	0\\
62.79	0\\
62.8	0\\
62.81	0\\
62.82	0\\
62.83	0\\
62.84	1.73472347597681e-18\\
62.85	0\\
62.86	0\\
62.87	0\\
62.88	0\\
62.89	0\\
62.9	0\\
62.91	0\\
62.92	0\\
62.93	0\\
62.94	0\\
62.95	0\\
62.96	0\\
62.97	0\\
62.98	0\\
62.99	1.73472347597681e-18\\
63	0\\
63.01	0\\
63.02	1.73472347597681e-18\\
63.03	0\\
63.04	0\\
63.05	1.73472347597681e-18\\
63.06	0\\
63.07	0\\
63.08	0\\
63.09	0\\
63.1	0\\
63.11	0\\
63.12	0\\
63.13	0\\
63.14	0\\
63.15	0\\
63.16	0\\
63.17	0\\
63.18	1.73472347597681e-18\\
63.19	0\\
63.2	0\\
63.21	0\\
63.22	0\\
63.23	0\\
63.24	0\\
63.25	0\\
63.26	0\\
63.27	0\\
63.28	0\\
63.29	0\\
63.3	0\\
63.31	0\\
63.32	0\\
63.33	0\\
63.34	0\\
63.35	0\\
63.36	0\\
63.37	1.73472347597681e-18\\
63.38	0\\
63.39	0\\
63.4	0\\
63.41	0\\
63.42	0\\
63.43	0\\
63.44	0\\
63.45	0\\
63.46	0\\
63.47	0\\
63.48	0\\
63.49	1.73472347597681e-18\\
63.5	0\\
63.51	0\\
63.52	0\\
63.53	0\\
63.54	0\\
63.55	0\\
63.56	0\\
63.57	0\\
63.58	0\\
63.59	0\\
63.6	0\\
63.61	0\\
63.62	0\\
63.63	0\\
63.64	0\\
63.65	0\\
63.66	0\\
63.67	0\\
63.68	0\\
63.69	0\\
63.7	0\\
63.71	0\\
63.72	0\\
63.73	0\\
63.74	0\\
63.75	0\\
63.76	0\\
63.77	0\\
63.78	0\\
63.79	0\\
63.8	1.73472347597681e-18\\
63.81	0\\
63.82	0\\
63.83	0\\
63.84	0\\
63.85	0\\
63.86	0\\
63.87	0\\
63.88	0\\
63.89	0\\
63.9	0\\
63.91	0\\
63.92	1.73472347597681e-18\\
63.93	0\\
63.94	0\\
63.95	0\\
63.96	1.73472347597681e-18\\
63.97	1.73472347597681e-18\\
63.98	0\\
63.99	0\\
64	0\\
64.01	0\\
64.02	0\\
64.03	0\\
64.04	0\\
64.05	0\\
64.06	1.73472347597681e-18\\
64.07	0\\
64.08	0\\
64.09	0\\
64.1	0\\
64.11	0\\
64.12	0\\
64.13	1.73472347597681e-18\\
64.14	0\\
64.15	0\\
64.16	0\\
64.17	0\\
64.18	0\\
64.19	0\\
64.2	0\\
64.21	0\\
64.22	0\\
64.23	0\\
64.24	0\\
64.25	0\\
64.26	0\\
64.27	0\\
64.28	0\\
64.29	1.73472347597681e-18\\
64.3	0\\
64.31	0\\
64.32	0\\
64.33	0\\
64.34	1.73472347597681e-18\\
64.35	0\\
64.36	0\\
64.37	0\\
64.38	0\\
64.39	0\\
64.4	0\\
64.41	0\\
64.42	1.73472347597681e-18\\
64.43	0\\
64.44	0\\
64.45	0\\
64.46	1.73472347597681e-18\\
64.47	0\\
64.48	0\\
64.49	0\\
64.5	0\\
64.51	0\\
64.52	0\\
64.53	0\\
64.54	0\\
64.55	0\\
64.56	0\\
64.57	0\\
64.58	0\\
64.59	1.73472347597681e-18\\
64.6	0\\
64.61	0\\
64.62	0\\
64.63	0\\
64.64	0\\
64.65	1.73472347597681e-18\\
64.66	0\\
64.67	0\\
64.68	0\\
64.69	0\\
64.7	0\\
64.71	0\\
64.72	0\\
64.73	0\\
64.74	0\\
64.75	1.73472347597681e-18\\
64.76	0\\
64.77	0\\
64.78	0\\
64.79	0\\
64.8	0\\
64.81	0\\
64.82	0\\
64.83	1.73472347597681e-18\\
64.84	1.73472347597681e-18\\
64.85	0\\
64.86	0\\
64.87	1.73472347597681e-18\\
64.88	0\\
64.89	0\\
64.9	0\\
64.91	0\\
64.92	0\\
64.93	0\\
64.94	0\\
64.95	0\\
64.96	0\\
64.97	1.73472347597681e-18\\
64.98	0\\
64.99	1.73472347597681e-18\\
65	0\\
65.01	0\\
65.02	0\\
65.03	0\\
65.04	0\\
65.05	0\\
65.06	0\\
65.07	1.73472347597681e-18\\
65.08	0\\
65.09	0\\
65.1	0\\
65.11	0\\
65.12	0\\
65.13	0\\
65.14	0\\
65.15	0\\
65.16	0\\
65.17	0\\
65.18	1.73472347597681e-18\\
65.19	0\\
65.2	0\\
65.21	0\\
65.22	0\\
65.23	0\\
65.24	0\\
65.25	0\\
65.26	0\\
65.27	0\\
65.28	0\\
65.29	0\\
65.3	0\\
65.31	0\\
65.32	0\\
65.33	0\\
65.34	0\\
65.35	0\\
65.36	0\\
65.37	1.73472347597681e-18\\
65.38	0\\
65.39	0\\
65.4	0\\
65.41	0\\
65.42	0\\
65.43	0\\
65.44	0\\
65.45	0\\
65.46	0\\
65.47	0\\
65.48	0\\
65.49	0\\
65.5	0\\
65.51	0\\
65.52	0\\
65.53	0\\
65.54	0\\
65.55	0\\
65.56	0\\
65.57	0\\
65.58	0\\
65.59	0\\
65.6	0\\
65.61	0\\
65.62	0\\
65.63	0\\
65.64	0\\
65.65	1.73472347597681e-18\\
65.66	0\\
65.67	0\\
65.68	1.73472347597681e-18\\
65.69	0\\
65.7	0\\
65.71	0\\
65.72	0\\
65.73	0\\
65.74	0\\
65.75	0\\
65.76	0\\
65.77	0\\
65.78	0\\
65.79	0\\
65.8	1.73472347597681e-18\\
65.81	1.73472347597681e-18\\
65.82	0\\
65.83	1.73472347597681e-18\\
65.84	0\\
65.85	0\\
65.86	1.73472347597681e-18\\
65.87	0\\
65.88	0\\
65.89	0\\
65.9	0\\
65.91	0\\
65.92	1.73472347597681e-18\\
65.93	1.73472347597681e-18\\
65.94	0\\
65.95	0\\
65.96	0\\
65.97	0\\
65.98	0\\
65.99	0\\
66	0\\
66.01	0\\
66.02	0\\
66.03	0\\
66.04	0\\
66.05	0\\
66.06	1.73472347597681e-18\\
66.07	0\\
66.08	0\\
66.09	0\\
66.1	0\\
66.11	0\\
66.12	0\\
66.13	0\\
66.14	0\\
66.15	1.73472347597681e-18\\
66.16	0\\
66.17	0\\
66.18	0\\
66.19	0\\
66.2	0\\
66.21	0\\
66.22	0\\
66.23	1.73472347597681e-18\\
66.24	0\\
66.25	0\\
66.26	0\\
66.27	0\\
66.28	0\\
66.29	1.73472347597681e-18\\
66.3	0\\
66.31	0\\
66.32	0\\
66.33	0\\
66.34	0\\
66.35	1.73472347597681e-18\\
66.36	0\\
66.37	1.73472347597681e-18\\
66.38	0\\
66.39	0\\
66.4	0\\
66.41	0\\
66.42	0\\
66.43	0\\
66.44	0\\
66.45	0\\
66.46	0\\
66.47	0\\
66.48	0\\
66.49	1.73472347597681e-18\\
66.5	0\\
66.51	0\\
66.52	0\\
66.53	0\\
66.54	0\\
66.55	0\\
66.56	0\\
66.57	1.73472347597681e-18\\
66.58	0\\
66.59	0\\
66.6	0\\
66.61	0\\
66.62	0\\
66.63	0\\
66.64	0\\
66.65	1.73472347597681e-18\\
66.66	0\\
66.67	0\\
66.68	1.73472347597681e-18\\
66.69	0\\
66.7	0\\
66.71	0\\
66.72	0\\
66.73	0\\
66.74	0\\
66.75	0\\
66.76	0\\
66.77	0\\
66.78	0\\
66.79	0\\
66.8	0\\
66.81	0\\
66.82	0\\
66.83	0\\
66.84	0\\
66.85	0\\
66.86	0\\
66.87	0\\
66.88	0\\
66.89	0\\
66.9	0\\
66.91	1.73472347597681e-18\\
66.92	0\\
66.93	0\\
66.94	1.73472347597681e-18\\
66.95	0\\
66.96	0\\
66.97	1.73472347597681e-18\\
66.98	0\\
66.99	0\\
67	0\\
67.01	0\\
67.02	1.73472347597681e-18\\
67.03	0\\
67.04	0\\
67.05	1.73472347597681e-18\\
67.06	0\\
67.07	0\\
67.08	1.73472347597681e-18\\
67.09	0\\
67.1	0\\
67.11	0\\
67.12	0\\
67.13	0\\
67.14	0\\
67.15	0\\
67.16	0\\
67.17	0\\
67.18	0\\
67.19	0\\
67.2	0\\
67.21	0\\
67.22	0\\
67.23	0\\
67.24	0\\
67.25	0\\
67.26	0\\
67.27	0\\
67.28	0\\
67.29	0\\
67.3	0\\
67.31	0\\
67.32	0\\
67.33	0\\
67.34	0\\
67.35	0\\
67.36	1.73472347597681e-18\\
67.37	0\\
67.38	0\\
67.39	0\\
67.4	0\\
67.41	1.73472347597681e-18\\
67.42	0\\
67.43	0\\
67.44	0\\
67.45	0\\
67.46	1.73472347597681e-18\\
67.47	0\\
67.48	0\\
67.49	0\\
67.5	1.73472347597681e-18\\
67.51	0\\
67.52	0\\
67.53	0\\
67.54	0\\
67.55	0\\
67.56	0\\
67.57	0\\
67.58	0\\
67.59	0\\
67.6	0\\
67.61	1.73472347597681e-18\\
67.62	0\\
67.63	0\\
67.64	1.73472347597681e-18\\
67.65	0\\
67.66	0\\
67.67	0\\
67.68	0\\
67.69	0\\
67.7	0\\
67.71	0\\
67.72	0\\
67.73	0\\
67.74	0\\
67.75	0\\
67.76	0\\
67.77	0\\
67.78	0\\
67.79	0\\
67.8	0\\
67.81	0\\
67.82	0\\
67.83	0\\
67.84	0\\
67.85	1.73472347597681e-18\\
67.86	0\\
67.87	0\\
67.88	0\\
67.89	0\\
67.9	0\\
67.91	0\\
67.92	0\\
67.93	0\\
67.94	0\\
67.95	0\\
67.96	0\\
67.97	0\\
67.98	0\\
67.99	0\\
68	0\\
68.01	0\\
68.02	1.73472347597681e-18\\
68.03	0\\
68.04	0\\
68.05	0\\
68.06	0\\
68.07	0\\
68.08	0\\
68.09	1.73472347597681e-18\\
68.1	0\\
68.11	0\\
68.12	0\\
68.13	0\\
68.14	0\\
68.15	0\\
68.16	0\\
68.17	1.73472347597681e-18\\
68.18	0\\
68.19	1.73472347597681e-18\\
68.2	0\\
68.21	0\\
68.22	0\\
68.23	0\\
68.24	0\\
68.25	0\\
68.26	0\\
68.27	0\\
68.28	0\\
68.29	0\\
68.3	1.73472347597681e-18\\
68.31	0\\
68.32	0\\
68.33	0\\
68.34	0\\
68.35	0\\
68.36	1.73472347597681e-18\\
68.37	0\\
68.38	0\\
68.39	0\\
68.4	0\\
68.41	0\\
68.42	0\\
68.43	1.73472347597681e-18\\
68.44	0\\
68.45	0\\
68.46	0\\
68.47	0\\
68.48	0\\
68.49	0\\
68.5	0\\
68.51	0\\
68.52	0\\
68.53	0\\
68.54	0\\
68.55	0\\
68.56	0\\
68.57	0\\
68.58	0\\
68.59	0\\
68.6	0\\
68.61	0\\
68.62	0\\
68.63	0\\
68.64	0\\
68.65	0\\
68.66	0\\
68.67	0\\
68.68	0\\
68.69	0\\
68.7	0\\
68.71	1.73472347597681e-18\\
68.72	0\\
68.73	0\\
68.74	0\\
68.75	0\\
68.76	0\\
68.77	0\\
68.78	0\\
68.79	0\\
68.8	0\\
68.81	0\\
68.82	0\\
68.83	0\\
68.84	0\\
68.85	0\\
68.86	0\\
68.87	0\\
68.88	0\\
68.89	0\\
68.9	0\\
68.91	0\\
68.92	0\\
68.93	0\\
68.94	0\\
68.95	0\\
68.96	0\\
68.97	0\\
68.98	0\\
68.99	0\\
69	1.73472347597681e-18\\
69.01	0\\
69.02	0\\
69.03	0\\
69.04	0\\
69.05	1.73472347597681e-18\\
69.06	0\\
69.07	0\\
69.08	0\\
69.09	0\\
69.1	0\\
69.11	0\\
69.12	0\\
69.13	1.73472347597681e-18\\
69.14	0\\
69.15	1.73472347597681e-18\\
69.16	0\\
69.17	0\\
69.18	0\\
69.19	1.73472347597681e-18\\
69.2	0\\
69.21	1.73472347597681e-18\\
69.22	0\\
69.23	0\\
69.24	0\\
69.25	0\\
69.26	0\\
69.27	0\\
69.28	0\\
69.29	0\\
69.3	0\\
69.31	0\\
69.32	0\\
69.33	0\\
69.34	0\\
69.35	0\\
69.36	0\\
69.37	1.73472347597681e-18\\
69.38	0\\
69.39	0\\
69.4	0\\
69.41	0\\
69.42	0\\
69.43	1.73472347597681e-18\\
69.44	0\\
69.45	1.73472347597681e-18\\
69.46	0\\
69.47	0\\
69.48	0\\
69.49	0\\
69.5	0\\
69.51	0\\
69.52	0\\
69.53	0\\
69.54	0\\
69.55	0\\
69.56	0\\
69.57	1.73472347597681e-18\\
69.58	1.73472347597681e-18\\
69.59	0\\
69.6	0\\
69.61	0\\
69.62	0\\
69.63	0\\
69.64	1.73472347597681e-18\\
69.65	0\\
69.66	0\\
69.67	1.73472347597681e-18\\
69.68	0\\
69.69	0\\
69.7	0\\
69.71	0\\
69.72	0\\
69.73	0\\
69.74	0\\
69.75	0\\
69.76	0\\
69.77	0\\
69.78	0\\
69.79	0\\
69.8	0\\
69.81	0\\
69.82	0\\
69.83	0\\
69.84	0\\
69.85	0\\
69.86	0\\
69.87	0\\
69.88	1.73472347597681e-18\\
69.89	0\\
69.9	0\\
69.91	0\\
69.92	0\\
69.93	0\\
69.94	0\\
69.95	1.73472347597681e-18\\
69.96	0\\
69.97	0\\
69.98	0\\
69.99	0\\
70	0\\
70.01	0\\
70.02	0\\
70.03	0\\
70.04	0\\
70.05	0\\
70.06	0\\
70.07	0\\
70.08	0\\
70.09	0\\
70.1	1.73472347597681e-18\\
70.11	0\\
70.12	0\\
70.13	0\\
70.14	1.73472347597681e-18\\
70.15	0\\
70.16	0\\
70.17	0\\
70.18	0\\
70.19	0\\
70.2	0\\
70.21	0\\
70.22	0\\
70.23	0\\
70.24	0\\
70.25	0\\
70.26	0\\
70.27	0\\
70.28	0\\
70.29	0\\
70.3	0\\
70.31	0\\
70.32	0\\
70.33	1.73472347597681e-18\\
70.34	0\\
70.35	0\\
70.36	0\\
70.37	0\\
70.38	0\\
70.39	0\\
70.4	1.73472347597681e-18\\
70.41	0\\
70.42	0\\
70.43	0\\
70.44	0\\
70.45	0\\
70.46	1.73472347597681e-18\\
70.47	0\\
70.48	0\\
70.49	0\\
70.5	0\\
70.51	0\\
70.52	0\\
70.53	0\\
70.54	0\\
70.55	0\\
70.56	0\\
70.57	0\\
70.58	0\\
70.59	0\\
70.6	0\\
70.61	0\\
70.62	0\\
70.63	0\\
70.64	0\\
70.65	1.73472347597681e-18\\
70.66	0\\
70.67	0\\
70.68	0\\
70.69	0\\
70.7	0\\
70.71	1.73472347597681e-18\\
70.72	0\\
70.73	0\\
70.74	0\\
70.75	0\\
70.76	0\\
70.77	0\\
70.78	0\\
70.79	0\\
70.8	0\\
70.81	0\\
70.82	1.73472347597681e-18\\
70.83	0\\
70.84	0\\
70.85	1.73472347597681e-18\\
70.86	0\\
70.87	0\\
70.88	0\\
70.89	0\\
70.9	0\\
70.91	0\\
70.92	1.73472347597681e-18\\
70.93	0\\
70.94	0\\
70.95	0\\
70.96	0\\
70.97	0\\
70.98	0\\
70.99	0\\
71	0\\
71.01	0\\
71.02	0\\
71.03	0\\
71.04	0\\
71.05	0\\
71.06	0\\
71.07	0\\
71.08	1.73472347597681e-18\\
71.09	1.73472347597681e-18\\
71.1	0\\
71.11	0\\
71.12	0\\
71.13	0\\
71.14	0\\
71.15	0\\
71.16	0\\
71.17	0\\
71.18	1.73472347597681e-18\\
71.19	0\\
71.2	0\\
71.21	0\\
71.22	0\\
71.23	0\\
71.24	0\\
71.25	0\\
71.26	0\\
71.27	1.73472347597681e-18\\
71.28	0\\
71.29	0\\
71.3	0\\
71.31	0\\
71.32	1.73472347597681e-18\\
71.33	0\\
71.34	0\\
71.35	0\\
71.36	1.73472347597681e-18\\
71.37	0\\
71.38	0\\
71.39	0\\
71.4	0\\
71.41	0\\
71.42	0\\
71.43	1.73472347597681e-18\\
71.44	0\\
71.45	0\\
71.46	0\\
71.47	0\\
71.48	0\\
71.49	0\\
71.5	0\\
71.51	0\\
71.52	1.73472347597681e-18\\
71.53	0\\
71.54	0\\
71.55	0\\
71.56	0\\
71.57	0\\
71.58	0\\
71.59	0\\
71.6	0\\
71.61	1.73472347597681e-18\\
71.62	0\\
71.63	1.73472347597681e-18\\
71.64	0\\
71.65	0\\
71.66	0\\
71.67	0\\
71.68	0\\
71.69	0\\
71.7	0\\
71.71	1.73472347597681e-18\\
71.72	0\\
71.73	0\\
71.74	0\\
71.75	0\\
71.76	0\\
71.77	0\\
71.78	0\\
71.79	0\\
71.8	0\\
71.81	0\\
71.82	0\\
71.83	1.73472347597681e-18\\
71.84	0\\
71.85	0\\
71.86	0\\
71.87	0\\
71.88	0\\
71.89	1.73472347597681e-18\\
71.9	0\\
71.91	0\\
71.92	0\\
71.93	0\\
71.94	0\\
71.95	0\\
71.96	1.73472347597681e-18\\
71.97	0\\
71.98	0\\
71.99	0\\
72	0\\
72.01	0\\
72.02	0\\
72.03	0\\
72.04	0\\
72.05	0\\
72.06	0\\
72.07	0\\
72.08	0\\
72.09	0\\
72.1	0\\
72.11	0\\
72.12	1.73472347597681e-18\\
72.13	0\\
72.14	0\\
72.15	0\\
72.16	0\\
72.17	0\\
72.18	0\\
72.19	1.73472347597681e-18\\
72.2	0\\
72.21	0\\
72.22	0\\
72.23	0\\
72.24	0\\
72.25	0\\
72.26	0\\
72.27	0\\
72.28	0\\
72.29	0\\
72.3	0\\
72.31	0\\
72.32	0\\
72.33	0\\
72.34	0\\
72.35	0\\
72.36	0\\
72.37	0\\
72.38	0\\
72.39	0\\
72.4	0\\
72.41	0\\
72.42	0\\
72.43	0\\
72.44	0\\
72.45	1.73472347597681e-18\\
72.46	0\\
72.47	0\\
72.48	0\\
72.49	0\\
72.5	0\\
72.51	0\\
72.52	0\\
72.53	0\\
72.54	0\\
72.55	1.73472347597681e-18\\
72.56	0\\
72.57	0\\
72.58	0\\
72.59	0\\
72.6	0\\
72.61	0\\
72.62	1.73472347597681e-18\\
72.63	0\\
72.64	0\\
72.65	0\\
72.66	0\\
72.67	0\\
72.68	0\\
72.69	0\\
72.7	1.73472347597681e-18\\
72.71	0\\
72.72	0\\
72.73	1.73472347597681e-18\\
72.74	0\\
72.75	0\\
72.76	0\\
72.77	0\\
72.78	1.73472347597681e-18\\
72.79	0\\
72.8	0\\
72.81	0\\
72.82	0\\
72.83	0\\
72.84	0\\
72.85	1.73472347597681e-18\\
72.86	0\\
72.87	0\\
72.88	0\\
72.89	1.73472347597681e-18\\
72.9	0\\
72.91	0\\
72.92	0\\
72.93	0\\
72.94	0\\
72.95	0\\
72.96	0\\
72.97	0\\
72.98	0\\
72.99	0\\
73	0\\
73.01	0\\
73.02	1.73472347597681e-18\\
73.03	0\\
73.04	0\\
73.05	0\\
73.06	0\\
73.07	0\\
73.08	0\\
73.09	0\\
73.1	0\\
73.11	0\\
73.12	0\\
73.13	1.73472347597681e-18\\
73.14	0\\
73.15	0\\
73.16	0\\
73.17	0\\
73.18	0\\
73.19	0\\
73.2	0\\
73.21	0\\
73.22	0\\
73.23	1.73472347597681e-18\\
73.24	0\\
73.25	0\\
73.26	0\\
73.27	1.73472347597681e-18\\
73.28	0\\
73.29	0\\
73.3	0\\
73.31	0\\
73.32	0\\
73.33	0\\
73.34	0\\
73.35	0\\
73.36	0\\
73.37	0\\
73.38	0\\
73.39	1.73472347597681e-18\\
73.4	0\\
73.41	0\\
73.42	0\\
73.43	0\\
73.44	0\\
73.45	0\\
73.46	0\\
73.47	0\\
73.48	0\\
73.49	0\\
73.5	0\\
73.51	1.73472347597681e-18\\
73.52	1.73472347597681e-18\\
73.53	0\\
73.54	0\\
73.55	1.73472347597681e-18\\
73.56	0\\
73.57	0\\
73.58	0\\
73.59	0\\
73.6	0\\
73.61	0\\
73.62	0\\
73.63	0\\
73.64	0\\
73.65	0\\
73.66	1.73472347597681e-18\\
73.67	0\\
73.68	0\\
73.69	0\\
73.7	0\\
73.71	0\\
73.72	0\\
73.73	1.73472347597681e-18\\
73.74	0\\
73.75	0\\
73.76	1.73472347597681e-18\\
73.77	0\\
73.78	0\\
73.79	0\\
73.8	1.73472347597681e-18\\
73.81	0\\
73.82	0\\
73.83	0\\
73.84	0\\
73.85	0\\
73.86	0\\
73.87	0\\
73.88	0\\
73.89	0\\
73.9	0\\
73.91	0\\
73.92	0\\
73.93	0\\
73.94	0\\
73.95	0\\
73.96	0\\
73.97	0\\
73.98	1.73472347597681e-18\\
73.99	0\\
74	0\\
74.01	1.73472347597681e-18\\
74.02	0\\
74.03	0\\
74.04	0\\
74.05	0\\
74.06	0\\
74.07	0\\
74.08	0\\
74.09	0\\
74.1	1.73472347597681e-18\\
74.11	1.73472347597681e-18\\
74.12	1.73472347597681e-18\\
74.13	0\\
74.14	1.73472347597681e-18\\
74.15	0\\
74.16	0\\
74.17	1.73472347597681e-18\\
74.18	0\\
74.19	0\\
74.2	0\\
74.21	0\\
74.22	0\\
74.23	0\\
74.24	0\\
74.25	0\\
74.26	0\\
74.27	0\\
74.28	0\\
74.29	0\\
74.3	1.73472347597681e-18\\
74.31	0\\
74.32	0\\
74.33	0\\
74.34	0\\
74.35	0\\
74.36	0\\
74.37	1.73472347597681e-18\\
74.38	0\\
74.39	0\\
74.4	0\\
74.41	0\\
74.42	0\\
74.43	0\\
74.44	0\\
74.45	0\\
74.46	0\\
74.47	0\\
74.48	0\\
74.49	0\\
74.5	0\\
74.51	0\\
74.52	0\\
74.53	0\\
74.54	0\\
74.55	0\\
74.56	0\\
74.57	0\\
74.58	0\\
74.59	0\\
74.6	1.73472347597681e-18\\
74.61	0\\
74.62	1.73472347597681e-18\\
74.63	0\\
74.64	0\\
74.65	1.73472347597681e-18\\
74.66	0\\
74.67	0\\
74.68	0\\
74.69	0\\
74.7	1.73472347597681e-18\\
74.71	0\\
74.72	0\\
74.73	0\\
74.74	0\\
74.75	0\\
74.76	0\\
74.77	0\\
74.78	0\\
74.79	0\\
74.8	0\\
74.81	0\\
74.82	0\\
74.83	1.73472347597681e-18\\
74.84	0\\
74.85	0\\
74.86	0\\
74.87	1.73472347597681e-18\\
74.88	0\\
74.89	0\\
74.9	0\\
74.91	0\\
74.92	0\\
74.93	0\\
74.94	0\\
74.95	0\\
74.96	0\\
74.97	0\\
74.98	0\\
74.99	0\\
75	1.73472347597681e-18\\
75.01	0\\
75.02	0\\
75.03	1.73472347597681e-18\\
75.04	0\\
75.05	0\\
75.06	1.73472347597681e-18\\
75.07	0\\
75.08	0\\
75.09	0\\
75.1	0\\
75.11	0\\
75.12	0\\
75.13	0\\
75.14	1.73472347597681e-18\\
75.15	0\\
75.16	0\\
75.17	0\\
75.18	0\\
75.19	0\\
75.2	0\\
75.21	0\\
75.22	0\\
75.23	0\\
75.24	0\\
75.25	0\\
75.26	0\\
75.27	0\\
75.28	0\\
75.29	0\\
75.3	0\\
75.31	0\\
75.32	0\\
75.33	0\\
75.34	0\\
75.35	0\\
75.36	1.73472347597681e-18\\
75.37	0\\
75.38	0\\
75.39	0\\
75.4	0\\
75.41	0\\
75.42	0\\
75.43	0\\
75.44	1.73472347597681e-18\\
75.45	0\\
75.46	1.73472347597681e-18\\
75.47	0\\
75.48	0\\
75.49	0\\
75.5	0\\
75.51	0\\
75.52	0\\
75.53	0\\
75.54	0\\
75.55	0\\
75.56	1.73472347597681e-18\\
75.57	0\\
75.58	0\\
75.59	0\\
75.6	0\\
75.61	0\\
75.62	0\\
75.63	0\\
75.64	0\\
75.65	0\\
75.66	0\\
75.67	0\\
75.68	0\\
75.69	0\\
75.7	0\\
75.71	0\\
75.72	0\\
75.73	1.73472347597681e-18\\
75.74	0\\
75.75	0\\
75.76	0\\
75.77	0\\
75.78	0\\
75.79	0\\
75.8	0\\
75.81	0\\
75.82	0\\
75.83	0\\
75.84	0\\
75.85	0\\
75.86	0\\
75.87	0\\
75.88	0\\
75.89	0\\
75.9	0\\
75.91	0\\
75.92	0\\
75.93	1.73472347597681e-18\\
75.94	0\\
75.95	0\\
75.96	0\\
75.97	0\\
75.98	0\\
75.99	0\\
76	0\\
76.01	0\\
76.02	1.73472347597681e-18\\
76.03	0\\
76.04	0\\
76.05	0\\
76.06	0\\
76.07	0\\
76.08	0\\
76.09	0\\
76.1	0\\
76.11	0\\
76.12	0\\
76.13	1.73472347597681e-18\\
76.14	0\\
76.15	0\\
76.16	1.73472347597681e-18\\
76.17	1.73472347597681e-18\\
76.18	0\\
76.19	0\\
76.2	0\\
76.21	0\\
76.22	0\\
76.23	0\\
76.24	0\\
76.25	0\\
76.26	0\\
76.27	0\\
76.28	0\\
76.29	0\\
76.3	0\\
76.31	0\\
76.32	0\\
76.33	1.73472347597681e-18\\
76.34	0\\
76.35	0\\
76.36	0\\
76.37	0\\
76.38	1.73472347597681e-18\\
76.39	0\\
76.4	0\\
76.41	1.73472347597681e-18\\
76.42	1.73472347597681e-18\\
76.43	0\\
76.44	0\\
76.45	0\\
76.46	1.73472347597681e-18\\
76.47	0\\
76.48	0\\
76.49	0\\
76.5	0\\
76.51	0\\
76.52	1.73472347597681e-18\\
76.53	0\\
76.54	1.73472347597681e-18\\
76.55	1.73472347597681e-18\\
76.56	0\\
76.57	0\\
76.58	0\\
76.59	0\\
76.6	0\\
76.61	0\\
76.62	0\\
76.63	0\\
76.64	0\\
76.65	0\\
76.66	0\\
76.67	0\\
76.68	0\\
76.69	0\\
76.7	0\\
76.71	0\\
76.72	0\\
76.73	0\\
76.74	0\\
76.75	0\\
76.76	0\\
76.77	0\\
76.78	0\\
76.79	0\\
76.8	0\\
76.81	0\\
76.82	0\\
76.83	0\\
76.84	0\\
76.85	0\\
76.86	0\\
76.87	0\\
76.88	0\\
76.89	1.73472347597681e-18\\
76.9	0\\
76.91	0\\
76.92	0\\
76.93	0\\
76.94	1.73472347597681e-18\\
76.95	0\\
76.96	0\\
76.97	0\\
76.98	0\\
76.99	0\\
77	0\\
77.01	0\\
77.02	0\\
77.03	0\\
77.04	1.73472347597681e-18\\
77.05	0\\
77.06	0\\
77.07	0\\
77.08	0\\
77.09	0\\
77.1	0\\
77.11	0\\
77.12	0\\
77.13	0\\
77.14	0\\
77.15	0\\
77.16	1.73472347597681e-18\\
77.17	0\\
77.18	0\\
77.19	1.73472347597681e-18\\
77.2	0\\
77.21	0\\
77.22	0\\
77.23	0\\
77.24	0\\
77.25	0\\
77.26	0\\
77.27	0\\
77.28	0\\
77.29	0\\
77.3	0\\
77.31	0\\
77.32	0\\
77.33	0\\
77.34	0\\
77.35	0\\
77.36	0\\
77.37	0\\
77.38	0\\
77.39	0\\
77.4	0\\
77.41	1.73472347597681e-18\\
77.42	0\\
77.43	0\\
77.44	0\\
77.45	0\\
77.46	0\\
77.47	0\\
77.48	0\\
77.49	1.73472347597681e-18\\
77.5	0\\
77.51	0\\
77.52	0\\
77.53	0\\
77.54	1.73472347597681e-18\\
77.55	0\\
77.56	0\\
77.57	0\\
77.58	0\\
77.59	0\\
77.6	0\\
77.61	0\\
77.62	0\\
77.63	0\\
77.64	0\\
77.65	1.73472347597681e-18\\
77.66	0\\
77.67	0\\
77.68	0\\
77.69	0\\
77.7	0\\
77.71	0\\
77.72	0\\
77.73	0\\
77.74	0\\
77.75	1.73472347597681e-18\\
77.76	0\\
77.77	1.73472347597681e-18\\
77.78	0\\
77.79	1.73472347597681e-18\\
77.8	0\\
77.81	0\\
77.82	0\\
77.83	0\\
77.84	0\\
77.85	1.73472347597681e-18\\
77.86	0\\
77.87	0\\
77.88	0\\
77.89	0\\
77.9	0\\
77.91	0\\
77.92	0\\
77.93	0\\
77.94	0\\
77.95	0\\
77.96	0\\
77.97	0\\
77.98	0\\
77.99	0\\
78	0\\
78.01	0\\
78.02	0\\
78.03	0\\
78.04	1.73472347597681e-18\\
78.05	0\\
78.06	0\\
78.07	0\\
78.08	0\\
78.09	0\\
78.1	1.73472347597681e-18\\
78.11	0\\
78.12	0\\
78.13	0\\
78.14	0\\
78.15	0\\
78.16	0\\
78.17	0\\
78.18	0\\
78.19	0\\
78.2	1.73472347597681e-18\\
78.21	1.73472347597681e-18\\
78.22	0\\
78.23	0\\
78.24	0\\
78.25	1.73472347597681e-18\\
78.26	0\\
78.27	0\\
78.28	0\\
78.29	1.73472347597681e-18\\
78.3	0\\
78.31	0\\
78.32	0\\
78.33	0\\
78.34	0\\
78.35	0\\
78.36	0\\
78.37	0\\
78.38	0\\
78.39	0\\
78.4	1.73472347597681e-18\\
78.41	1.73472347597681e-18\\
78.42	0\\
78.43	0\\
78.44	0\\
78.45	0\\
78.46	0\\
78.47	0\\
78.48	0\\
78.49	0\\
78.5	0\\
78.51	0\\
78.52	0\\
78.53	0\\
78.54	1.73472347597681e-18\\
78.55	0\\
78.56	0\\
78.57	0\\
78.58	0\\
78.59	0\\
78.6	0\\
78.61	0\\
78.62	1.73472347597681e-18\\
78.63	0\\
78.64	0\\
78.65	0\\
78.66	0\\
78.67	0\\
78.68	0\\
78.69	0\\
78.7	0\\
78.71	0\\
78.72	0\\
78.73	0\\
78.74	0\\
78.75	0\\
78.76	0\\
78.77	0\\
78.78	0\\
78.79	0\\
78.8	0\\
78.81	0\\
78.82	0\\
78.83	0\\
78.84	0\\
78.85	0\\
78.86	0\\
78.87	1.73472347597681e-18\\
78.88	0\\
78.89	0\\
78.9	1.73472347597681e-18\\
78.91	0\\
78.92	1.73472347597681e-18\\
78.93	0\\
78.94	0\\
78.95	1.73472347597681e-18\\
78.96	0\\
78.97	0\\
78.98	1.73472347597681e-18\\
78.99	0\\
79	0\\
79.01	0\\
79.02	0\\
79.03	0\\
79.04	0\\
79.05	0\\
79.06	0\\
79.07	0\\
79.08	1.73472347597681e-18\\
79.09	0\\
79.1	0\\
79.11	0\\
79.12	0\\
79.13	0\\
79.14	0\\
79.15	0\\
79.16	0\\
79.17	0\\
79.18	0\\
79.19	0\\
79.2	0\\
79.21	0\\
79.22	0\\
79.23	0\\
79.24	0\\
79.25	0\\
79.26	0\\
79.27	0\\
79.28	0\\
79.29	0\\
79.3	1.73472347597681e-18\\
79.31	0\\
79.32	0\\
79.33	0\\
79.34	0\\
79.35	1.73472347597681e-18\\
79.36	0\\
79.37	0\\
79.38	0\\
79.39	0\\
79.4	0\\
79.41	0\\
79.42	0\\
79.43	0\\
79.44	0\\
79.45	0\\
79.46	1.73472347597681e-18\\
79.47	0\\
79.48	0\\
79.49	0\\
79.5	0\\
79.51	0\\
79.52	0\\
79.53	1.73472347597681e-18\\
79.54	0\\
79.55	0\\
79.56	1.73472347597681e-18\\
79.57	1.73472347597681e-18\\
79.58	0\\
79.59	1.73472347597681e-18\\
79.6	0\\
79.61	0\\
79.62	0\\
79.63	0\\
79.64	0\\
79.65	0\\
79.66	0\\
79.67	0\\
79.68	0\\
79.69	0\\
79.7	0\\
79.71	0\\
79.72	0\\
79.73	0\\
79.74	0\\
79.75	0\\
79.76	0\\
79.77	0\\
79.78	0\\
79.79	1.73472347597681e-18\\
79.8	0\\
79.81	0\\
79.82	1.73472347597681e-18\\
79.83	0\\
79.84	0\\
79.85	0\\
79.86	0\\
79.87	1.73472347597681e-18\\
79.88	0\\
79.89	0\\
79.9	0\\
79.91	0\\
79.92	0\\
79.93	0\\
79.94	1.73472347597681e-18\\
79.95	0\\
79.96	0\\
79.97	1.73472347597681e-18\\
79.98	1.73472347597681e-18\\
79.99	1.73472347597681e-18\\
80	0\\
80.01	0\\
};
\addplot [color=red,dashed]
  table[row sep=crcr]{%
80.01	0\\
80.02	0\\
80.03	1.73472347597681e-18\\
80.04	0\\
80.05	0\\
80.06	0\\
80.07	0\\
80.08	0\\
80.09	0\\
80.1	0\\
80.11	1.73472347597681e-18\\
80.12	0\\
80.13	1.73472347597681e-18\\
80.14	0\\
80.15	0\\
80.16	0\\
80.17	1.73472347597681e-18\\
80.18	1.73472347597681e-18\\
80.19	0\\
80.2	0\\
80.21	1.73472347597681e-18\\
80.22	0\\
80.23	1.73472347597681e-18\\
80.24	0\\
80.25	1.73472347597681e-18\\
80.26	0\\
80.27	0\\
80.28	0\\
80.29	0\\
80.3	0\\
80.31	0\\
80.32	0\\
80.33	0\\
80.34	0\\
80.35	0\\
80.36	0\\
80.37	1.73472347597681e-18\\
80.38	0\\
80.39	0\\
80.4	1.73472347597681e-18\\
80.41	0\\
80.42	0\\
80.43	1.73472347597681e-18\\
80.44	0\\
80.45	0\\
80.46	1.73472347597681e-18\\
80.47	0\\
80.48	0\\
80.49	0\\
80.5	0\\
80.51	0\\
80.52	0\\
80.53	0\\
80.54	0\\
80.55	0\\
80.56	1.73472347597681e-18\\
80.57	0\\
80.58	0\\
80.59	0\\
80.6	1.73472347597681e-18\\
80.61	0\\
80.62	0\\
80.63	0\\
80.64	0\\
80.65	0\\
80.66	0\\
80.67	0\\
80.68	0\\
80.69	1.73472347597681e-18\\
80.7	1.73472347597681e-18\\
80.71	0\\
80.72	0\\
80.73	1.73472347597681e-18\\
80.74	0\\
80.75	0\\
80.76	0\\
80.77	0\\
80.78	0\\
80.79	0\\
80.8	0\\
80.81	1.73472347597681e-18\\
80.82	0\\
80.83	0\\
80.84	0\\
80.85	0\\
80.86	0\\
80.87	0\\
80.88	0\\
80.89	0\\
80.9	1.73472347597681e-18\\
80.91	0\\
80.92	0\\
80.93	0\\
80.94	0\\
80.95	1.73472347597681e-18\\
80.96	0\\
80.97	0\\
80.98	0\\
80.99	0\\
81	1.73472347597681e-18\\
81.01	0\\
81.02	0\\
81.03	0\\
81.04	0\\
81.05	0\\
81.06	1.73472347597681e-18\\
81.07	1.73472347597681e-18\\
81.08	0\\
81.09	0\\
81.1	0\\
81.11	0\\
81.12	0\\
81.13	0\\
81.14	0\\
81.15	0\\
81.16	0\\
81.17	0\\
81.18	0\\
81.19	0\\
81.2	0\\
81.21	0\\
81.22	0\\
81.23	0\\
81.24	0\\
81.25	0\\
81.26	0\\
81.27	0\\
81.28	1.73472347597681e-18\\
81.29	1.73472347597681e-18\\
81.3	0\\
81.31	0\\
81.32	0\\
81.33	0\\
81.34	1.73472347597681e-18\\
81.35	0\\
81.36	0\\
81.37	0\\
81.38	0\\
81.39	0\\
81.4	1.73472347597681e-18\\
81.41	0\\
81.42	0\\
81.43	0\\
81.44	0\\
81.45	0\\
81.46	0\\
81.47	0\\
81.48	0\\
81.49	1.73472347597681e-18\\
81.5	0\\
81.51	0\\
81.52	0\\
81.53	0\\
81.54	1.73472347597681e-18\\
81.55	0\\
81.56	0\\
81.57	0\\
81.58	0\\
81.59	0\\
81.6	0\\
81.61	0\\
81.62	0\\
81.63	0\\
81.64	0\\
81.65	0\\
81.66	0\\
81.67	0\\
81.68	1.73472347597681e-18\\
81.69	0\\
81.7	0\\
81.71	1.73472347597681e-18\\
81.72	0\\
81.73	0\\
81.74	0\\
81.75	0\\
81.76	1.73472347597681e-18\\
81.77	0\\
81.78	0\\
81.79	1.73472347597681e-18\\
81.8	0\\
81.81	0\\
81.82	0\\
81.83	0\\
81.84	0\\
81.85	0\\
81.86	0\\
81.87	0\\
81.88	0\\
81.89	0\\
81.9	0\\
81.91	0\\
81.92	0\\
81.93	0\\
81.94	1.73472347597681e-18\\
81.95	0\\
81.96	1.73472347597681e-18\\
81.97	0\\
81.98	0\\
81.99	1.73472347597681e-18\\
82	0\\
82.01	1.73472347597681e-18\\
82.02	1.73472347597681e-18\\
82.03	1.73472347597681e-18\\
82.04	0\\
82.05	1.73472347597681e-18\\
82.06	0\\
82.07	1.73472347597681e-18\\
82.08	0\\
82.09	0\\
82.1	0\\
82.11	0\\
82.12	0\\
82.13	0\\
82.14	0\\
82.15	0\\
82.16	0\\
82.17	0\\
82.18	0\\
82.19	0\\
82.2	0\\
82.21	0\\
82.22	0\\
82.23	0\\
82.24	0\\
82.25	0\\
82.26	0\\
82.27	0\\
82.28	0\\
82.29	0\\
82.3	0\\
82.31	0\\
82.32	0\\
82.33	1.73472347597681e-18\\
82.34	0\\
82.35	1.73472347597681e-18\\
82.36	0\\
82.37	0\\
82.38	0\\
82.39	0\\
82.4	0\\
82.41	0\\
82.42	0\\
82.43	0\\
82.44	0\\
82.45	0\\
82.46	0\\
82.47	0\\
82.48	0\\
82.49	0\\
82.5	1.73472347597681e-18\\
82.51	0\\
82.52	0\\
82.53	0\\
82.54	0\\
82.55	0\\
82.56	0\\
82.57	0\\
82.58	0\\
82.59	0\\
82.6	0\\
82.61	0\\
82.62	0\\
82.63	1.73472347597681e-18\\
82.64	1.73472347597681e-18\\
82.65	1.73472347597681e-18\\
82.66	0\\
82.67	0\\
82.68	0\\
82.69	0\\
82.7	0\\
82.71	0\\
82.72	0\\
82.73	0\\
82.74	0\\
82.75	0\\
82.76	0\\
82.77	1.73472347597681e-18\\
82.78	0\\
82.79	0\\
82.8	0\\
82.81	0\\
82.82	0\\
82.83	0\\
82.84	0\\
82.85	1.73472347597681e-18\\
82.86	0\\
82.87	0\\
82.88	0\\
82.89	1.73472347597681e-18\\
82.9	0\\
82.91	0\\
82.92	0\\
82.93	0\\
82.94	0\\
82.95	0\\
82.96	0\\
82.97	0\\
82.98	0\\
82.99	0\\
83	0\\
83.01	0\\
83.02	0\\
83.03	0\\
83.04	0\\
83.05	1.73472347597681e-18\\
83.06	0\\
83.07	0\\
83.08	0\\
83.09	0\\
83.1	1.73472347597681e-18\\
83.11	1.73472347597681e-18\\
83.12	1.73472347597681e-18\\
83.13	0\\
83.14	0\\
83.15	0\\
83.16	0\\
83.17	0\\
83.18	0\\
83.19	0\\
83.2	1.73472347597681e-18\\
83.21	0\\
83.22	0\\
83.23	1.73472347597681e-18\\
83.24	0\\
83.25	0\\
83.26	1.73472347597681e-18\\
83.27	0\\
83.28	0\\
83.29	1.73472347597681e-18\\
83.3	0\\
83.31	0\\
83.32	0\\
83.33	0\\
83.34	0\\
83.35	0\\
83.36	0\\
83.37	0\\
83.38	0\\
83.39	0\\
83.4	0\\
83.41	0\\
83.42	0\\
83.43	0\\
83.44	0\\
83.45	0\\
83.46	0\\
83.47	1.73472347597681e-18\\
83.48	0\\
83.49	1.73472347597681e-18\\
83.5	0\\
83.51	0\\
83.52	0\\
83.53	0\\
83.54	0\\
83.55	0\\
83.56	0\\
83.57	0\\
83.58	0\\
83.59	0\\
83.6	1.73472347597681e-18\\
83.61	1.73472347597681e-18\\
83.62	0\\
83.63	1.73472347597681e-18\\
83.64	0\\
83.65	0\\
83.66	0\\
83.67	0\\
83.68	0\\
83.69	1.73472347597681e-18\\
83.7	0\\
83.71	0\\
83.72	0\\
83.73	0\\
83.74	0\\
83.75	0\\
83.76	0\\
83.77	0\\
83.78	0\\
83.79	1.73472347597681e-18\\
83.8	0\\
83.81	0\\
83.82	0\\
83.83	0\\
83.84	0\\
83.85	0\\
83.86	0\\
83.87	0\\
83.88	0\\
83.89	0\\
83.9	0\\
83.91	0\\
83.92	0\\
83.93	0\\
83.94	0\\
83.95	0\\
83.96	0\\
83.97	0\\
83.98	0\\
83.99	0\\
84	0\\
84.01	0\\
84.02	0\\
84.03	0\\
84.04	0\\
84.05	0\\
84.06	0\\
84.07	1.73472347597681e-18\\
84.08	0\\
84.09	0\\
84.1	0\\
84.11	1.73472347597681e-18\\
84.12	0\\
84.13	0\\
84.14	0\\
84.15	0\\
84.16	0\\
84.17	0\\
84.18	0\\
84.19	0\\
84.2	0\\
84.21	0\\
84.22	1.73472347597681e-18\\
84.23	0\\
84.24	0\\
84.25	0\\
84.26	1.73472347597681e-18\\
84.27	0\\
84.28	0\\
84.29	0\\
84.3	0\\
84.31	0\\
84.32	1.73472347597681e-18\\
84.33	0\\
84.34	0\\
84.35	1.73472347597681e-18\\
84.36	1.73472347597681e-18\\
84.37	0\\
84.38	0\\
84.39	0\\
84.4	0\\
84.41	1.73472347597681e-18\\
84.42	1.73472347597681e-18\\
84.43	0\\
84.44	0\\
84.45	0\\
84.46	0\\
84.47	0\\
84.48	0\\
84.49	0\\
84.5	1.73472347597681e-18\\
84.51	0\\
84.52	0\\
84.53	0\\
84.54	0\\
84.55	0\\
84.56	0\\
84.57	0\\
84.58	0\\
84.59	1.73472347597681e-18\\
84.6	0\\
84.61	0\\
84.62	0\\
84.63	0\\
84.64	0\\
84.65	0\\
84.66	1.73472347597681e-18\\
84.67	1.73472347597681e-18\\
84.68	0\\
84.69	0\\
84.7	0\\
84.71	0\\
84.72	0\\
84.73	0\\
84.74	0\\
84.75	0\\
84.76	1.73472347597681e-18\\
84.77	0\\
84.78	0\\
84.79	0\\
84.8	0\\
84.81	0\\
84.82	0\\
84.83	0\\
84.84	0\\
84.85	0\\
84.86	0\\
84.87	0\\
84.88	0\\
84.89	0\\
84.9	0\\
84.91	0\\
84.92	0\\
84.93	0\\
84.94	0\\
84.95	0\\
84.96	0\\
84.97	0\\
84.98	0\\
84.99	0\\
85	0\\
85.01	0\\
85.02	0\\
85.03	0\\
85.04	0\\
85.05	0\\
85.06	0\\
85.07	0\\
85.08	0\\
85.09	0\\
85.1	0\\
85.11	0\\
85.12	0\\
85.13	0\\
85.14	0\\
85.15	0\\
85.16	0\\
85.17	0\\
85.18	0\\
85.19	0\\
85.2	0\\
85.21	0\\
85.22	0\\
85.23	1.73472347597681e-18\\
85.24	0\\
85.25	0\\
85.26	1.73472347597681e-18\\
85.27	0\\
85.28	1.73472347597681e-18\\
85.29	0\\
85.3	0\\
85.31	1.73472347597681e-18\\
85.32	0\\
85.33	1.73472347597681e-18\\
85.34	0\\
85.35	0\\
85.36	0\\
85.37	0\\
85.38	0\\
85.39	0\\
85.4	0\\
85.41	0\\
85.42	0\\
85.43	0\\
85.44	0\\
85.45	0\\
85.46	0\\
85.47	0\\
85.48	0\\
85.49	0\\
85.5	0\\
85.51	0\\
85.52	0\\
85.53	1.73472347597681e-18\\
85.54	0\\
85.55	0\\
85.56	0\\
85.57	0\\
85.58	0\\
85.59	0\\
85.6	0\\
85.61	0\\
85.62	0\\
85.63	0\\
85.64	0\\
85.65	0\\
85.66	0\\
85.67	0\\
85.68	0\\
85.69	0\\
85.7	0\\
85.71	0\\
85.72	0\\
85.73	0\\
85.74	0\\
85.75	0\\
85.76	0\\
85.77	0\\
85.78	0\\
85.79	0\\
85.8	0\\
85.81	1.73472347597681e-18\\
85.82	0\\
85.83	0\\
85.84	0\\
85.85	0\\
85.86	0\\
85.87	0\\
85.88	0\\
85.89	0\\
85.9	0\\
85.91	0\\
85.92	0\\
85.93	0\\
85.94	0\\
85.95	0\\
85.96	0\\
85.97	1.73472347597681e-18\\
85.98	0\\
85.99	0\\
86	0\\
86.01	0\\
86.02	1.73472347597681e-18\\
86.03	1.73472347597681e-18\\
86.04	0\\
86.05	0\\
86.06	0\\
86.07	0\\
86.08	0\\
86.09	0\\
86.1	0\\
86.11	0\\
86.12	0\\
86.13	0\\
86.14	0\\
86.15	0\\
86.16	0\\
86.17	0\\
86.18	0\\
86.19	0\\
86.2	0\\
86.21	0\\
86.22	0\\
86.23	0\\
86.24	0\\
86.25	1.73472347597681e-18\\
86.26	1.73472347597681e-18\\
86.27	0\\
86.28	0\\
86.29	0\\
86.3	0\\
86.31	0\\
86.32	0\\
86.33	1.73472347597681e-18\\
86.34	0\\
86.35	0\\
86.36	0\\
86.37	0\\
86.38	0\\
86.39	0\\
86.4	0\\
86.41	0\\
86.42	0\\
86.43	0\\
86.44	0\\
86.45	0\\
86.46	0\\
86.47	0\\
86.48	0\\
86.49	0\\
86.5	0\\
86.51	0\\
86.52	0\\
86.53	0\\
86.54	0\\
86.55	1.73472347597681e-18\\
86.56	0\\
86.57	0\\
86.58	0\\
86.59	0\\
86.6	0\\
86.61	0\\
86.62	0\\
86.63	0\\
86.64	0\\
86.65	0\\
86.66	0\\
86.67	0\\
86.68	0\\
86.69	0\\
86.7	1.73472347597681e-18\\
86.71	1.73472347597681e-18\\
86.72	0\\
86.73	0\\
86.74	0\\
86.75	0\\
86.76	0\\
86.77	0\\
86.78	0\\
86.79	0\\
86.8	0\\
86.81	0\\
86.82	0\\
86.83	0\\
86.84	0\\
86.85	0\\
86.86	0\\
86.87	0\\
86.88	0\\
86.89	0\\
86.9	0\\
86.91	0\\
86.92	0\\
86.93	0\\
86.94	0\\
86.95	0\\
86.96	0\\
86.97	1.73472347597681e-18\\
86.98	0\\
86.99	0\\
87	0\\
87.01	0\\
87.02	0\\
87.03	0\\
87.04	0\\
87.05	0\\
87.06	0\\
87.07	0\\
87.08	0\\
87.09	0\\
87.1	0\\
87.11	0\\
87.12	0\\
87.13	0\\
87.14	0\\
87.15	0\\
87.16	0\\
87.17	0\\
87.18	0\\
87.19	0\\
87.2	0\\
87.21	0\\
87.22	0\\
87.23	0\\
87.24	0\\
87.25	0\\
87.26	0\\
87.27	1.73472347597681e-18\\
87.28	0\\
87.29	0\\
87.3	0\\
87.31	0\\
87.32	0\\
87.33	0\\
87.34	0\\
87.35	0\\
87.36	0\\
87.37	0\\
87.38	0\\
87.39	0\\
87.4	0\\
87.41	0\\
87.42	0\\
87.43	0\\
87.44	0\\
87.45	0\\
87.46	0\\
87.47	0\\
87.48	0\\
87.49	0\\
87.5	0\\
87.51	1.73472347597681e-18\\
87.52	0\\
87.53	0\\
87.54	0\\
87.55	0\\
87.56	0\\
87.57	0\\
87.58	0\\
87.59	0\\
87.6	0\\
87.61	0\\
87.62	1.73472347597681e-18\\
87.63	0\\
87.64	0\\
87.65	0\\
87.66	0\\
87.67	0\\
87.68	0\\
87.69	1.73472347597681e-18\\
87.7	1.73472347597681e-18\\
87.71	0\\
87.72	0\\
87.73	0\\
87.74	0\\
87.75	0\\
87.76	0\\
87.77	0\\
87.78	0\\
87.79	0\\
87.8	0\\
87.81	0\\
87.82	0\\
87.83	1.73472347597681e-18\\
87.84	0\\
87.85	0\\
87.86	0\\
87.87	0\\
87.88	0\\
87.89	0\\
87.9	0\\
87.91	0\\
87.92	0\\
87.93	0\\
87.94	0\\
87.95	0\\
87.96	0\\
87.97	0\\
87.98	0\\
87.99	0\\
88	0\\
88.01	0\\
88.02	0\\
88.03	0\\
88.04	0\\
88.05	0\\
88.06	0\\
88.07	0\\
88.08	0\\
88.09	0\\
88.1	0\\
88.11	0\\
88.12	0\\
88.13	0\\
88.14	0\\
88.15	1.73472347597681e-18\\
88.16	0\\
88.17	0\\
88.18	0\\
88.19	0\\
88.2	0\\
88.21	0\\
88.22	0\\
88.23	0\\
88.24	0\\
88.25	0\\
88.26	0\\
88.27	0\\
88.28	0\\
88.29	0\\
88.3	0\\
88.31	0\\
88.32	0\\
88.33	0\\
88.34	0\\
88.35	0\\
88.36	0\\
88.37	0\\
88.38	0\\
88.39	0\\
88.4	0\\
88.41	0\\
88.42	0\\
88.43	0\\
88.44	0\\
88.45	0\\
88.46	1.73472347597681e-18\\
88.47	0\\
88.48	0\\
88.49	0\\
88.5	0\\
88.51	0\\
88.52	0\\
88.53	0\\
88.54	0\\
88.55	0\\
88.56	0\\
88.57	0\\
88.58	0\\
88.59	0\\
88.6	0\\
88.61	0\\
88.62	0\\
88.63	0\\
88.64	0\\
88.65	0\\
88.66	0\\
88.67	0\\
88.68	0\\
88.69	0\\
88.7	0\\
88.71	0\\
88.72	0\\
88.73	0\\
88.74	0\\
88.75	0\\
88.76	0\\
88.77	0\\
88.78	0\\
88.79	0\\
88.8	0\\
88.81	0\\
88.82	0\\
88.83	0\\
88.84	0\\
88.85	0\\
88.86	0\\
88.87	0\\
88.88	0\\
88.89	0\\
88.9	0\\
88.91	0\\
88.92	0\\
88.93	0\\
88.94	0\\
88.95	0\\
88.96	0\\
88.97	0\\
88.98	0\\
88.99	0\\
89	0\\
89.01	0\\
89.02	0\\
89.03	0\\
89.04	0\\
89.05	0\\
89.06	0\\
89.07	0\\
89.08	0\\
89.09	0\\
89.1	0\\
89.11	0\\
89.12	0\\
89.13	0\\
89.14	0\\
89.15	0\\
89.16	0\\
89.17	0\\
89.18	0\\
89.19	0\\
89.2	0\\
89.21	0\\
89.22	0\\
89.23	0\\
89.24	0\\
89.25	0\\
89.26	0\\
89.27	0\\
89.28	0\\
89.29	0\\
89.3	0\\
89.31	0\\
89.32	0\\
89.33	0\\
89.34	0\\
89.35	0\\
89.36	0\\
89.37	0\\
89.38	1.73472347597681e-18\\
89.39	0\\
89.4	0\\
89.41	0\\
89.42	0\\
89.43	0\\
89.44	0\\
89.45	0\\
89.46	0\\
89.47	0\\
89.48	0\\
89.49	0\\
89.5	0\\
89.51	0\\
89.52	0\\
89.53	0\\
89.54	0\\
89.55	0\\
89.56	0\\
89.57	0\\
89.58	0\\
89.59	0\\
89.6	0\\
89.61	0\\
89.62	0\\
89.63	0\\
89.64	0\\
89.65	1.73472347597681e-18\\
89.66	0\\
89.67	0\\
89.68	1.73472347597681e-18\\
89.69	0\\
89.7	0\\
89.71	0\\
89.72	0\\
89.73	0\\
89.74	0\\
89.75	0\\
89.76	0\\
89.77	0\\
89.78	0\\
89.79	0\\
89.8	0\\
89.81	0\\
89.82	0\\
89.83	0\\
89.84	0\\
89.85	0\\
89.86	0\\
89.87	0\\
89.88	0\\
89.89	0\\
89.9	0\\
89.91	0\\
89.92	0\\
89.93	0\\
89.94	0\\
89.95	0\\
89.96	0\\
89.97	0\\
89.98	0\\
89.99	0\\
90	0\\
90.01	0\\
90.02	0\\
90.03	0\\
90.04	0\\
90.05	0\\
90.06	1.73472347597681e-18\\
90.07	0\\
90.08	0\\
90.09	0\\
90.1	0\\
90.11	0\\
90.12	0\\
90.13	0\\
90.14	0\\
90.15	0\\
90.16	0\\
90.17	0\\
90.18	1.73472347597681e-18\\
90.19	0\\
90.2	0\\
90.21	0\\
90.22	0\\
90.23	0\\
90.24	0\\
90.25	0\\
90.26	0\\
90.27	0\\
90.28	0\\
90.29	0\\
90.3	0\\
90.31	0\\
90.32	0\\
90.33	1.73472347597681e-18\\
90.34	0\\
90.35	0\\
90.36	0\\
90.37	0\\
90.38	0\\
90.39	0\\
90.4	0\\
90.41	0\\
90.42	0\\
90.43	0\\
90.44	0\\
90.45	0\\
90.46	0\\
90.47	0\\
90.48	0\\
90.49	0\\
90.5	0\\
90.51	0\\
90.52	0\\
90.53	0\\
90.54	0\\
90.55	0\\
90.56	0\\
90.57	0\\
90.58	0\\
90.59	0\\
90.6	0\\
90.61	0\\
90.62	0\\
90.63	0\\
90.64	0\\
90.65	0\\
90.66	0\\
90.67	0\\
90.68	0\\
90.69	0\\
90.7	0\\
90.71	0\\
90.72	0\\
90.73	0\\
90.74	0\\
90.75	0\\
90.76	0\\
90.77	0\\
90.78	0\\
90.79	0\\
90.8	0\\
90.81	0\\
90.82	0\\
90.83	0\\
90.84	0\\
90.85	0\\
90.86	0\\
90.87	0\\
90.88	0\\
90.89	0\\
90.9	0\\
90.91	0\\
90.92	0\\
90.93	0\\
90.94	0\\
90.95	0\\
90.96	0\\
90.97	0\\
90.98	0\\
90.99	0\\
91	0\\
91.01	0\\
91.02	0\\
91.03	0\\
91.04	0\\
91.05	0\\
91.06	0\\
91.07	0\\
91.08	0\\
91.09	0\\
91.1	0\\
91.11	0\\
91.12	0\\
91.13	0\\
91.14	0\\
91.15	0\\
91.16	0\\
91.17	0\\
91.18	0\\
91.19	0\\
91.2	0\\
91.21	0\\
91.22	0\\
91.23	0\\
91.24	0\\
91.25	0\\
91.26	0\\
91.27	0\\
91.28	0\\
91.29	0\\
91.3	0\\
91.31	0\\
91.32	0\\
91.33	0\\
91.34	0\\
91.35	0\\
91.36	0\\
91.37	0\\
91.38	0\\
91.39	0\\
91.4	0\\
91.41	0\\
91.42	0\\
91.43	0\\
91.44	0\\
91.45	0\\
91.46	0\\
91.47	0\\
91.48	0\\
91.49	0\\
91.5	0\\
91.51	0\\
91.52	0\\
91.53	0\\
91.54	0\\
91.55	0\\
91.56	0\\
91.57	0\\
91.58	0\\
91.59	0\\
91.6	0\\
91.61	0\\
91.62	0\\
91.63	0\\
91.64	0\\
91.65	0\\
91.66	0\\
91.67	0\\
91.68	0\\
91.69	0\\
91.7	0\\
91.71	0\\
91.72	0\\
91.73	0\\
91.74	0\\
91.75	0\\
91.76	0\\
91.77	0\\
91.78	0\\
91.79	0\\
91.8	0\\
91.81	0\\
91.82	0\\
91.83	0\\
91.84	0\\
91.85	0\\
91.86	0\\
91.87	0\\
91.88	0\\
91.89	0\\
91.9	0\\
91.91	0\\
91.92	0\\
91.93	0\\
91.94	0\\
91.95	0\\
91.96	0\\
91.97	0\\
91.98	0\\
91.99	0\\
92	0\\
92.01	0\\
92.02	0\\
92.03	0\\
92.04	0\\
92.05	0\\
92.06	0\\
92.07	0\\
92.08	0\\
92.09	0\\
92.1	0\\
92.11	0\\
92.12	0\\
92.13	0\\
92.14	0\\
92.15	0\\
92.16	0\\
92.17	0\\
92.18	0\\
92.19	0\\
92.2	0\\
92.21	0\\
92.22	0\\
92.23	0\\
92.24	0\\
92.25	0\\
92.26	0\\
92.27	0\\
92.28	0\\
92.29	0\\
92.3	0\\
92.31	0\\
92.32	0\\
92.33	0\\
92.34	0\\
92.35	0\\
92.36	0\\
92.37	0\\
92.38	0\\
92.39	0\\
92.4	0\\
92.41	0\\
92.42	0\\
92.43	0\\
92.44	0\\
92.45	0\\
92.46	0\\
92.47	0\\
92.48	0\\
92.49	0\\
92.5	0\\
92.51	0\\
92.52	0\\
92.53	0\\
92.54	0\\
92.55	0\\
92.56	0\\
92.57	0\\
92.58	0\\
92.59	0\\
92.6	0\\
92.61	0\\
92.62	0\\
92.63	0\\
92.64	0\\
92.65	0\\
92.66	0\\
92.67	0\\
92.68	0\\
92.69	0\\
92.7	0\\
92.71	0\\
92.72	0\\
92.73	0\\
92.74	0\\
92.75	0\\
92.76	0\\
92.77	0\\
92.78	0\\
92.79	0\\
92.8	0\\
92.81	0\\
92.82	0\\
92.83	0\\
92.84	0\\
92.85	0\\
92.86	0\\
92.87	0\\
92.88	0\\
92.89	0\\
92.9	0\\
92.91	0\\
92.92	0\\
92.93	0\\
92.94	0\\
92.95	0\\
92.96	0\\
92.97	0\\
92.98	0\\
92.99	0\\
93	0\\
93.01	0\\
93.02	0\\
93.03	0\\
93.04	0\\
93.05	0\\
93.06	0\\
93.07	0\\
93.08	0\\
93.09	0\\
93.1	0\\
93.11	0\\
93.12	0\\
93.13	0\\
93.14	0\\
93.15	0\\
93.16	0\\
93.17	0\\
93.18	0\\
93.19	0\\
93.2	0\\
93.21	0\\
93.22	0\\
93.23	0\\
93.24	0\\
93.25	0\\
93.26	0\\
93.27	0\\
93.28	0\\
93.29	0\\
93.3	0\\
93.31	0\\
93.32	0\\
93.33	0\\
93.34	0\\
93.35	0\\
93.36	0\\
93.37	0\\
93.38	0\\
93.39	0\\
93.4	0\\
93.41	0\\
93.42	0\\
93.43	0\\
93.44	0\\
93.45	0\\
93.46	0\\
93.47	0\\
93.48	0\\
93.49	0\\
93.5	0\\
93.51	0\\
93.52	0\\
93.53	0\\
93.54	0\\
93.55	0\\
93.56	0\\
93.57	0\\
93.58	0\\
93.59	0\\
93.6	0\\
93.61	0\\
93.62	0\\
93.63	0\\
93.64	0\\
93.65	0\\
93.66	0\\
93.67	0\\
93.68	0\\
93.69	0\\
93.7	0\\
93.71	0\\
93.72	0\\
93.73	0\\
93.74	0\\
93.75	0\\
93.76	0\\
93.77	0\\
93.78	0\\
93.79	0\\
93.8	0\\
93.81	0\\
93.82	0\\
93.83	0\\
93.84	0\\
93.85	0\\
93.86	0\\
93.87	0\\
93.88	0\\
93.89	0\\
93.9	0\\
93.91	0\\
93.92	0\\
93.93	0\\
93.94	0\\
93.95	0\\
93.96	0\\
93.97	0\\
93.98	0\\
93.99	0\\
94	0\\
94.01	0\\
94.02	0\\
94.03	0\\
94.04	0\\
94.05	0\\
94.06	0\\
94.07	0\\
94.08	0\\
94.09	0\\
94.1	0\\
94.11	0\\
94.12	0\\
94.13	0\\
94.14	0\\
94.15	0\\
94.16	0\\
94.17	0\\
94.18	0\\
94.19	0\\
94.2	0\\
94.21	0\\
94.22	0\\
94.23	0\\
94.24	0\\
94.25	0\\
94.26	0\\
94.27	0\\
94.28	0\\
94.29	0\\
94.3	0\\
94.31	0\\
94.32	0\\
94.33	0\\
94.34	0\\
94.35	0\\
94.36	0\\
94.37	0\\
94.38	0\\
94.39	0\\
94.4	0\\
94.41	0\\
94.42	0\\
94.43	0\\
94.44	0\\
94.45	0\\
94.46	0\\
94.47	0\\
94.48	0\\
94.49	0\\
94.5	0\\
94.51	0\\
94.52	0\\
94.53	0\\
94.54	0\\
94.55	0\\
94.56	0\\
94.57	0\\
94.58	0\\
94.59	0\\
94.6	0\\
94.61	0\\
94.62	0\\
94.63	0\\
94.64	0\\
94.65	0\\
94.66	0\\
94.67	0\\
94.68	0\\
94.69	0\\
94.7	0\\
94.71	0\\
94.72	0\\
94.73	0\\
94.74	0\\
94.75	0\\
94.76	0\\
94.77	0\\
94.78	0\\
94.79	0\\
94.8	0\\
94.81	0\\
94.82	0\\
94.83	0\\
94.84	0\\
94.85	0\\
94.86	0\\
94.87	0\\
94.88	0\\
94.89	0\\
94.9	0\\
94.91	0\\
94.92	0\\
94.93	0\\
94.94	0\\
94.95	0\\
94.96	0\\
94.97	0\\
94.98	0\\
94.99	0\\
95	0\\
95.01	0\\
95.02	0\\
95.03	0\\
95.04	0\\
95.05	0\\
95.06	0\\
95.07	0\\
95.08	0\\
95.09	0\\
95.1	0\\
95.11	0\\
95.12	0\\
95.13	0\\
95.14	0\\
95.15	0\\
95.16	0\\
95.17	0\\
95.18	0\\
95.19	0\\
95.2	0\\
95.21	0\\
95.22	0\\
95.23	0\\
95.24	0\\
95.25	0\\
95.26	0\\
95.27	0\\
95.28	0\\
95.29	0\\
95.3	0\\
95.31	0\\
95.32	0\\
95.33	0\\
95.34	0\\
95.35	0\\
95.36	0\\
95.37	0\\
95.38	0\\
95.39	0\\
95.4	0\\
95.41	0\\
95.42	0\\
95.43	0\\
95.44	0\\
95.45	0\\
95.46	0\\
95.47	0\\
95.48	0\\
95.49	0\\
95.5	0\\
95.51	0\\
95.52	0\\
95.53	0\\
95.54	0\\
95.55	0\\
95.56	0\\
95.57	0\\
95.58	0\\
95.59	0\\
95.6	0\\
95.61	0\\
95.62	0\\
95.63	0\\
95.64	0\\
95.65	0\\
95.66	0\\
95.67	0\\
95.68	0\\
95.69	0\\
95.7	0\\
95.71	0\\
95.72	0\\
95.73	0\\
95.74	0\\
95.75	0\\
95.76	0\\
95.77	0\\
95.78	0\\
95.79	0\\
95.8	0\\
95.81	0\\
95.82	0\\
95.83	0\\
95.84	0\\
95.85	0\\
95.86	0\\
95.87	0\\
95.88	0\\
95.89	0\\
95.9	0\\
95.91	0\\
95.92	0\\
95.93	0\\
95.94	0\\
95.95	0\\
95.96	0\\
95.97	0\\
95.98	0\\
95.99	0\\
96	0\\
96.01	0\\
96.02	0\\
96.03	0\\
96.04	0\\
96.05	0\\
96.06	0\\
96.07	0\\
96.08	0\\
96.09	0\\
96.1	0\\
96.11	0\\
96.12	0\\
96.13	0\\
96.14	0\\
96.15	0\\
96.16	0\\
96.17	0\\
96.18	0\\
96.19	0\\
96.2	0\\
96.21	0\\
96.22	0\\
96.23	0\\
96.24	0\\
96.25	0\\
96.26	0\\
96.27	0\\
96.28	0\\
96.29	0\\
96.3	0\\
96.31	0\\
96.32	0\\
96.33	0\\
96.34	0\\
96.35	0\\
96.36	0\\
96.37	0\\
96.38	0\\
96.39	0\\
96.4	0\\
96.41	0\\
96.42	0\\
96.43	0\\
96.44	0\\
96.45	0\\
96.46	0\\
96.47	0\\
96.48	0\\
96.49	0\\
96.5	0\\
96.51	0\\
96.52	0\\
96.53	0\\
96.54	0\\
96.55	0\\
96.56	0\\
96.57	0\\
96.58	0\\
96.59	0\\
96.6	0\\
96.61	0\\
96.62	0\\
96.63	0\\
96.64	0\\
96.65	0\\
96.66	0\\
96.67	0\\
96.68	0\\
96.69	0\\
96.7	0\\
96.71	0\\
96.72	0\\
96.73	0\\
96.74	0\\
96.75	0\\
96.76	0\\
96.77	0\\
96.78	0\\
96.79	0\\
96.8	0\\
96.81	0\\
96.82	0\\
96.83	0\\
96.84	0\\
96.85	0\\
96.86	0\\
96.87	0\\
96.88	0\\
96.89	0\\
96.9	0\\
96.91	0\\
96.92	0\\
96.93	0\\
96.94	0\\
96.95	0\\
96.96	0\\
96.97	0\\
96.98	0\\
96.99	0\\
97	0\\
97.01	0\\
97.02	0\\
97.03	0\\
97.04	0\\
97.05	0\\
97.06	0\\
97.07	0\\
97.08	0\\
97.09	0\\
97.1	0\\
97.11	0\\
97.12	0\\
97.13	0\\
97.14	0\\
97.15	0\\
97.16	0\\
97.17	0\\
97.18	0\\
97.19	0\\
97.2	0\\
97.21	0\\
97.22	0\\
97.23	0\\
97.24	0\\
97.25	0\\
97.26	0\\
97.27	0\\
97.28	0\\
97.29	0\\
97.3	0\\
97.31	0\\
97.32	0\\
97.33	0\\
97.34	0\\
97.35	0\\
97.36	0\\
97.37	0\\
97.38	0\\
97.39	0\\
97.4	0\\
97.41	0\\
97.42	0\\
97.43	0\\
97.44	0\\
97.45	0\\
97.46	0\\
97.47	0\\
97.48	0\\
97.49	0\\
97.5	0\\
97.51	0\\
97.52	0\\
97.53	0\\
97.54	0\\
97.55	0\\
97.56	0\\
97.57	0\\
97.58	0\\
97.59	0\\
97.6	0\\
97.61	0\\
97.62	0\\
97.63	0\\
97.64	0\\
97.65	0\\
97.66	0\\
97.67	0\\
97.68	0\\
97.69	0\\
97.7	0\\
97.71	0\\
97.72	0\\
97.73	0\\
97.74	0\\
97.75	0\\
97.76	0\\
97.77	0\\
97.78	0\\
97.79	0\\
97.8	0\\
97.81	0\\
97.82	0\\
97.83	0\\
97.84	0\\
97.85	0\\
97.86	0\\
97.87	0\\
97.88	0\\
97.89	0\\
97.9	0\\
97.91	0\\
97.92	0\\
97.93	0\\
97.94	0\\
97.95	0\\
97.96	0\\
97.97	0\\
97.98	0\\
97.99	0\\
98	0\\
98.01	0\\
98.02	0\\
98.03	0\\
98.04	0\\
98.05	0\\
98.06	0\\
98.07	0\\
98.08	0\\
98.09	0\\
98.1	0\\
98.11	0\\
98.12	0\\
98.13	0\\
98.14	0\\
98.15	0\\
98.16	0\\
98.17	0\\
98.18	0\\
98.19	0\\
98.2	0\\
98.21	0\\
98.22	0\\
98.23	0\\
98.24	0\\
98.25	0\\
98.26	0\\
98.27	0\\
98.28	0\\
98.29	0\\
98.3	0\\
98.31	0\\
98.32	0\\
98.33	0\\
98.34	0\\
98.35	0\\
98.36	0\\
98.37	0\\
98.38	0\\
98.39	0\\
98.4	0\\
98.41	0\\
98.42	0\\
98.43	0\\
98.44	0\\
98.45	0\\
98.46	0\\
98.47	0\\
98.48	0\\
98.49	0\\
98.5	0\\
98.51	0\\
98.52	0\\
98.53	0\\
98.54	0\\
98.55	0\\
98.56	0\\
98.57	0\\
98.58	0\\
98.59	0\\
98.6	0\\
98.61	0\\
98.62	0\\
98.63	0\\
98.64	0\\
98.65	0\\
98.66	0\\
98.67	0\\
98.68	0\\
98.69	0\\
98.7	0\\
98.71	0\\
98.72	0\\
98.73	0\\
98.74	0\\
98.75	0\\
98.76	0\\
98.77	0\\
98.78	0\\
98.79	0\\
98.8	0\\
98.81	0\\
98.82	0\\
98.83	0\\
98.84	0\\
98.85	0\\
98.86	0\\
98.87	0\\
98.88	0\\
98.89	0\\
98.9	0\\
98.91	0\\
98.92	0\\
98.93	0\\
98.94	0\\
98.95	0\\
98.96	0\\
98.97	0\\
98.98	0\\
98.99	0\\
99	0\\
99.01	0\\
99.02	0\\
99.03	0\\
99.04	0\\
99.05	0\\
99.06	0\\
99.07	0\\
99.08	0\\
99.09	0\\
99.1	0\\
99.11	0\\
99.12	0\\
99.13	0\\
99.14	0\\
99.15	0\\
99.16	0\\
99.17	0\\
99.18	0\\
99.19	0\\
99.2	0\\
99.21	0\\
99.22	0\\
99.23	0\\
99.24	0\\
99.25	0\\
99.26	0\\
99.27	0\\
99.28	0\\
99.29	0\\
99.3	0\\
99.31	0\\
99.32	0\\
99.33	0\\
99.34	0\\
99.35	0\\
99.36	0\\
99.37	0\\
99.38	0\\
99.39	0\\
99.4	0\\
99.41	0\\
99.42	0\\
99.43	0\\
99.44	0\\
99.45	0\\
99.46	0\\
99.47	0\\
99.48	0\\
99.49	0\\
99.5	0\\
99.51	0\\
99.52	0\\
99.53	0\\
99.54	0\\
99.55	0\\
99.56	0\\
99.57	0\\
99.58	0\\
99.59	0\\
99.6	0\\
99.61	0\\
99.62	0\\
99.63	0\\
99.64	0\\
99.65	0\\
99.66	0\\
99.67	0\\
99.68	0\\
99.69	0\\
99.7	0\\
99.71	0\\
99.72	0\\
99.73	0\\
99.74	0\\
99.75	0\\
99.76	0\\
99.77	0\\
99.78	0\\
99.79	0\\
99.8	0\\
99.81	0\\
99.82	0\\
99.83	0\\
99.84	0\\
99.85	0\\
99.86	0\\
99.87	0\\
99.88	0\\
99.89	0\\
99.9	0\\
99.91	0\\
99.92	0\\
99.93	0\\
99.94	0\\
99.95	0\\
99.96	0\\
99.97	0\\
99.98	0\\
99.99	0\\
100	0\\
};
\addlegendentry{$q=-2$};

\addplot [color=blue,dashed,forget plot]
  table[row sep=crcr]{%
0.01	0.00201272129290892\\
0.02	0.00201272129049512\\
0.03	0.00201272128807998\\
0.04	0.00201272128566351\\
0.05	0.0020127212832457\\
0.06	0.00201272128082656\\
0.07	0.00201272127840608\\
0.08	0.00201272127598427\\
0.09	0.00201272127356111\\
0.1	0.00201272127113662\\
0.11	0.00201272126871078\\
0.12	0.00201272126628361\\
0.13	0.00201272126385509\\
0.14	0.00201272126142523\\
0.15	0.00201272125899402\\
0.16	0.00201272125656147\\
0.17	0.00201272125412757\\
0.18	0.00201272125169232\\
0.19	0.00201272124925573\\
0.2	0.00201272124681778\\
0.21	0.00201272124437849\\
0.22	0.00201272124193784\\
0.23	0.00201272123949584\\
0.24	0.00201272123705249\\
0.25	0.00201272123460778\\
0.26	0.00201272123216172\\
0.27	0.00201272122971429\\
0.28	0.00201272122726552\\
0.29	0.00201272122481538\\
0.3	0.00201272122236388\\
0.31	0.00201272121991103\\
0.32	0.00201272121745681\\
0.33	0.00201272121500122\\
0.34	0.00201272121254428\\
0.35	0.00201272121008597\\
0.36	0.00201272120762629\\
0.37	0.00201272120516524\\
0.38	0.00201272120270283\\
0.39	0.00201272120023905\\
0.4	0.0020127211977739\\
0.41	0.00201272119530737\\
0.42	0.00201272119283948\\
0.43	0.00201272119037021\\
0.44	0.00201272118789957\\
0.45	0.00201272118542755\\
0.46	0.00201272118295416\\
0.47	0.00201272118047939\\
0.48	0.00201272117800324\\
0.49	0.00201272117552571\\
0.5	0.0020127211730468\\
0.51	0.00201272117056651\\
0.52	0.00201272116808483\\
0.53	0.00201272116560177\\
0.54	0.00201272116311733\\
0.55	0.0020127211606315\\
0.56	0.00201272115814428\\
0.57	0.00201272115565568\\
0.58	0.00201272115316569\\
0.59	0.00201272115067431\\
0.6	0.00201272114818153\\
0.61	0.00201272114568737\\
0.62	0.00201272114319181\\
0.63	0.00201272114069486\\
0.64	0.00201272113819651\\
0.65	0.00201272113569676\\
0.66	0.00201272113319562\\
0.67	0.00201272113069308\\
0.68	0.00201272112818914\\
0.69	0.0020127211256838\\
0.7	0.00201272112317705\\
0.71	0.00201272112066891\\
0.72	0.00201272111815936\\
0.73	0.0020127211156484\\
0.74	0.00201272111313604\\
0.75	0.00201272111062227\\
0.76	0.0020127211081071\\
0.77	0.00201272110559051\\
0.78	0.00201272110307252\\
0.79	0.00201272110055311\\
0.8	0.00201272109803229\\
0.81	0.00201272109551006\\
0.82	0.00201272109298641\\
0.83	0.00201272109046134\\
0.84	0.00201272108793486\\
0.85	0.00201272108540696\\
0.86	0.00201272108287765\\
0.87	0.00201272108034691\\
0.88	0.00201272107781475\\
0.89	0.00201272107528117\\
0.9	0.00201272107274616\\
0.91	0.00201272107020973\\
0.92	0.00201272106767188\\
0.93	0.0020127210651326\\
0.94	0.00201272106259189\\
0.95	0.00201272106004975\\
0.96	0.00201272105750618\\
0.97	0.00201272105496118\\
0.98	0.00201272105241475\\
0.99	0.00201272104986688\\
1	0.00201272104731758\\
1.01	0.00201272104476685\\
1.02	0.00201272104221467\\
1.03	0.00201272103966106\\
1.04	0.00201272103710602\\
1.05	0.00201272103454953\\
1.06	0.0020127210319916\\
1.07	0.00201272102943222\\
1.08	0.00201272102687141\\
1.09	0.00201272102430915\\
1.1	0.00201272102174544\\
1.11	0.00201272101918029\\
1.12	0.00201272101661369\\
1.13	0.00201272101404564\\
1.14	0.00201272101147614\\
1.15	0.00201272100890519\\
1.16	0.00201272100633279\\
1.17	0.00201272100375893\\
1.18	0.00201272100118362\\
1.19	0.00201272099860686\\
1.2	0.00201272099602863\\
1.21	0.00201272099344895\\
1.22	0.00201272099086781\\
1.23	0.00201272098828521\\
1.24	0.00201272098570115\\
1.25	0.00201272098311562\\
1.26	0.00201272098052863\\
1.27	0.00201272097794018\\
1.28	0.00201272097535026\\
1.29	0.00201272097275888\\
1.3	0.00201272097016602\\
1.31	0.0020127209675717\\
1.32	0.0020127209649759\\
1.33	0.00201272096237864\\
1.34	0.0020127209597799\\
1.35	0.00201272095717968\\
1.36	0.002012720954578\\
1.37	0.00201272095197483\\
1.38	0.00201272094937019\\
1.39	0.00201272094676407\\
1.4	0.00201272094415647\\
1.41	0.00201272094154738\\
1.42	0.00201272093893682\\
1.43	0.00201272093632477\\
1.44	0.00201272093371124\\
1.45	0.00201272093109623\\
1.46	0.00201272092847972\\
1.47	0.00201272092586173\\
1.48	0.00201272092324225\\
1.49	0.00201272092062128\\
1.5	0.00201272091799881\\
1.51	0.00201272091537486\\
1.52	0.00201272091274941\\
1.53	0.00201272091012247\\
1.54	0.00201272090749403\\
1.55	0.00201272090486409\\
1.56	0.00201272090223265\\
1.57	0.00201272089959971\\
1.58	0.00201272089696528\\
1.59	0.00201272089432934\\
1.6	0.0020127208916919\\
1.61	0.00201272088905295\\
1.62	0.0020127208864125\\
1.63	0.00201272088377054\\
1.64	0.00201272088112707\\
1.65	0.0020127208784821\\
1.66	0.00201272087583561\\
1.67	0.00201272087318761\\
1.68	0.0020127208705381\\
1.69	0.00201272086788708\\
1.7	0.00201272086523454\\
1.71	0.00201272086258048\\
1.72	0.00201272085992491\\
1.73	0.00201272085726781\\
1.74	0.0020127208546092\\
1.75	0.00201272085194907\\
1.76	0.00201272084928741\\
1.77	0.00201272084662423\\
1.78	0.00201272084395953\\
1.79	0.00201272084129329\\
1.8	0.00201272083862554\\
1.81	0.00201272083595625\\
1.82	0.00201272083328543\\
1.83	0.00201272083061309\\
1.84	0.00201272082793921\\
1.85	0.0020127208252638\\
1.86	0.00201272082258685\\
1.87	0.00201272081990836\\
1.88	0.00201272081722834\\
1.89	0.00201272081454679\\
1.9	0.00201272081186369\\
1.91	0.00201272080917905\\
1.92	0.00201272080649287\\
1.93	0.00201272080380515\\
1.94	0.00201272080111588\\
1.95	0.00201272079842507\\
1.96	0.00201272079573271\\
1.97	0.0020127207930388\\
1.98	0.00201272079034334\\
1.99	0.00201272078764634\\
2	0.00201272078494778\\
2.01	0.00201272078224766\\
2.02	0.002012720779546\\
2.03	0.00201272077684277\\
2.04	0.00201272077413799\\
2.05	0.00201272077143166\\
2.06	0.00201272076872376\\
2.07	0.0020127207660143\\
2.08	0.00201272076330329\\
2.09	0.00201272076059071\\
2.1	0.00201272075787656\\
2.11	0.00201272075516085\\
2.12	0.00201272075244357\\
2.13	0.00201272074972472\\
2.14	0.00201272074700431\\
2.15	0.00201272074428232\\
2.16	0.00201272074155876\\
2.17	0.00201272073883363\\
2.18	0.00201272073610693\\
2.19	0.00201272073337865\\
2.2	0.00201272073064879\\
2.21	0.00201272072791735\\
2.22	0.00201272072518434\\
2.23	0.00201272072244974\\
2.24	0.00201272071971356\\
2.25	0.0020127207169758\\
2.26	0.00201272071423646\\
2.27	0.00201272071149552\\
2.28	0.00201272070875301\\
2.29	0.0020127207060089\\
2.3	0.0020127207032632\\
2.31	0.00201272070051591\\
2.32	0.00201272069776703\\
2.33	0.00201272069501656\\
2.34	0.00201272069226449\\
2.35	0.00201272068951083\\
2.36	0.00201272068675556\\
2.37	0.0020127206839987\\
2.38	0.00201272068124024\\
2.39	0.00201272067848018\\
2.4	0.00201272067571852\\
2.41	0.00201272067295525\\
2.42	0.00201272067019038\\
2.43	0.0020127206674239\\
2.44	0.00201272066465581\\
2.45	0.00201272066188611\\
2.46	0.00201272065911481\\
2.47	0.00201272065634189\\
2.48	0.00201272065356735\\
2.49	0.00201272065079121\\
2.5	0.00201272064801345\\
2.51	0.00201272064523407\\
2.52	0.00201272064245307\\
2.53	0.00201272063967045\\
2.54	0.00201272063688621\\
2.55	0.00201272063410036\\
2.56	0.00201272063131287\\
2.57	0.00201272062852376\\
2.58	0.00201272062573303\\
2.59	0.00201272062294067\\
2.6	0.00201272062014668\\
2.61	0.00201272061735105\\
2.62	0.0020127206145538\\
2.63	0.00201272061175491\\
2.64	0.0020127206089544\\
2.65	0.00201272060615224\\
2.66	0.00201272060334845\\
2.67	0.00201272060054302\\
2.68	0.00201272059773595\\
2.69	0.00201272059492724\\
2.7	0.00201272059211688\\
2.71	0.00201272058930489\\
2.72	0.00201272058649125\\
2.73	0.00201272058367596\\
2.74	0.00201272058085902\\
2.75	0.00201272057804044\\
2.76	0.0020127205752202\\
2.77	0.00201272057239832\\
2.78	0.00201272056957478\\
2.79	0.00201272056674958\\
2.8	0.00201272056392273\\
2.81	0.00201272056109423\\
2.82	0.00201272055826406\\
2.83	0.00201272055543224\\
2.84	0.00201272055259875\\
2.85	0.0020127205497636\\
2.86	0.00201272054692679\\
2.87	0.00201272054408831\\
2.88	0.00201272054124816\\
2.89	0.00201272053840635\\
2.9	0.00201272053556287\\
2.91	0.00201272053271771\\
2.92	0.00201272052987089\\
2.93	0.00201272052702239\\
2.94	0.00201272052417221\\
2.95	0.00201272052132036\\
2.96	0.00201272051846684\\
2.97	0.00201272051561163\\
2.98	0.00201272051275474\\
2.99	0.00201272050989617\\
3	0.00201272050703592\\
3.01	0.00201272050417398\\
3.02	0.00201272050131036\\
3.03	0.00201272049844504\\
3.04	0.00201272049557804\\
3.05	0.00201272049270935\\
3.06	0.00201272048983897\\
3.07	0.00201272048696689\\
3.08	0.00201272048409312\\
3.09	0.00201272048121766\\
3.1	0.00201272047834049\\
3.11	0.00201272047546163\\
3.12	0.00201272047258107\\
3.13	0.00201272046969881\\
3.14	0.00201272046681484\\
3.15	0.00201272046392917\\
3.16	0.00201272046104179\\
3.17	0.00201272045815271\\
3.18	0.00201272045526192\\
3.19	0.00201272045236941\\
3.2	0.0020127204494752\\
3.21	0.00201272044657927\\
3.22	0.00201272044368163\\
3.23	0.00201272044078228\\
3.24	0.0020127204378812\\
3.25	0.00201272043497841\\
3.26	0.00201272043207389\\
3.27	0.00201272042916766\\
3.28	0.0020127204262597\\
3.29	0.00201272042335002\\
3.3	0.00201272042043861\\
3.31	0.00201272041752548\\
3.32	0.00201272041461061\\
3.33	0.00201272041169402\\
3.34	0.00201272040877569\\
3.35	0.00201272040585564\\
3.36	0.00201272040293384\\
3.37	0.00201272040001031\\
3.38	0.00201272039708505\\
3.39	0.00201272039415804\\
3.4	0.0020127203912293\\
3.41	0.00201272038829881\\
3.42	0.00201272038536658\\
3.43	0.0020127203824326\\
3.44	0.00201272037949688\\
3.45	0.00201272037655941\\
3.46	0.00201272037362019\\
3.47	0.00201272037067922\\
3.48	0.0020127203677365\\
3.49	0.00201272036479203\\
3.5	0.0020127203618458\\
3.51	0.00201272035889781\\
3.52	0.00201272035594807\\
3.53	0.00201272035299657\\
3.54	0.0020127203500433\\
3.55	0.00201272034708828\\
3.56	0.00201272034413148\\
3.57	0.00201272034117293\\
3.58	0.00201272033821261\\
3.59	0.00201272033525052\\
3.6	0.00201272033228665\\
3.61	0.00201272032932102\\
3.62	0.00201272032635362\\
3.63	0.00201272032338444\\
3.64	0.00201272032041348\\
3.65	0.00201272031744075\\
3.66	0.00201272031446624\\
3.67	0.00201272031148994\\
3.68	0.00201272030851187\\
3.69	0.00201272030553201\\
3.7	0.00201272030255037\\
3.71	0.00201272029956694\\
3.72	0.00201272029658173\\
3.73	0.00201272029359472\\
3.74	0.00201272029060592\\
3.75	0.00201272028761534\\
3.76	0.00201272028462295\\
3.77	0.00201272028162877\\
3.78	0.0020127202786328\\
3.79	0.00201272027563502\\
3.8	0.00201272027263545\\
3.81	0.00201272026963407\\
3.82	0.00201272026663089\\
3.83	0.00201272026362591\\
3.84	0.00201272026061912\\
3.85	0.00201272025761052\\
3.86	0.00201272025460011\\
3.87	0.0020127202515879\\
3.88	0.00201272024857387\\
3.89	0.00201272024555802\\
3.9	0.00201272024254036\\
3.91	0.00201272023952088\\
3.92	0.00201272023649959\\
3.93	0.00201272023347648\\
3.94	0.00201272023045154\\
3.95	0.00201272022742478\\
3.96	0.00201272022439619\\
3.97	0.00201272022136578\\
3.98	0.00201272021833354\\
3.99	0.00201272021529947\\
4	0.00201272021226357\\
4.01	0.00201272020922584\\
4.02	0.00201272020618628\\
4.03	0.00201272020314487\\
4.04	0.00201272020010163\\
4.05	0.00201272019705656\\
4.06	0.00201272019400964\\
4.07	0.00201272019096088\\
4.08	0.00201272018791027\\
4.09	0.00201272018485782\\
4.1	0.00201272018180352\\
4.11	0.00201272017874738\\
4.12	0.00201272017568939\\
4.13	0.00201272017262954\\
4.14	0.00201272016956784\\
4.15	0.00201272016650429\\
4.16	0.00201272016343888\\
4.17	0.00201272016037161\\
4.18	0.00201272015730248\\
4.19	0.00201272015423149\\
4.2	0.00201272015115864\\
4.21	0.00201272014808392\\
4.22	0.00201272014500734\\
4.23	0.00201272014192889\\
4.24	0.00201272013884857\\
4.25	0.00201272013576639\\
4.26	0.00201272013268232\\
4.27	0.00201272012959639\\
4.28	0.00201272012650858\\
4.29	0.00201272012341889\\
4.3	0.00201272012032732\\
4.31	0.00201272011723387\\
4.32	0.00201272011413854\\
4.33	0.00201272011104133\\
4.34	0.00201272010794223\\
4.35	0.00201272010484125\\
4.36	0.00201272010173837\\
4.37	0.00201272009863361\\
4.38	0.00201272009552695\\
4.39	0.0020127200924184\\
4.4	0.00201272008930796\\
4.41	0.00201272008619562\\
4.42	0.00201272008308138\\
4.43	0.00201272007996524\\
4.44	0.0020127200768472\\
4.45	0.00201272007372725\\
4.46	0.0020127200706054\\
4.47	0.00201272006748164\\
4.48	0.00201272006435598\\
4.49	0.0020127200612284\\
4.5	0.00201272005809892\\
4.51	0.00201272005496752\\
4.52	0.0020127200518342\\
4.53	0.00201272004869897\\
4.54	0.00201272004556182\\
4.55	0.00201272004242275\\
4.56	0.00201272003928176\\
4.57	0.00201272003613884\\
4.58	0.002012720032994\\
4.59	0.00201272002984724\\
4.6	0.00201272002669854\\
4.61	0.00201272002354792\\
4.62	0.00201272002039536\\
4.63	0.00201272001724087\\
4.64	0.00201272001408445\\
4.65	0.00201272001092609\\
4.66	0.00201272000776579\\
4.67	0.00201272000460354\\
4.68	0.00201272000143936\\
4.69	0.00201271999827324\\
4.7	0.00201271999510517\\
4.71	0.00201271999193515\\
4.72	0.00201271998876318\\
4.73	0.00201271998558926\\
4.74	0.0020127199824134\\
4.75	0.00201271997923557\\
4.76	0.00201271997605579\\
4.77	0.00201271997287406\\
4.78	0.00201271996969037\\
4.79	0.00201271996650471\\
4.8	0.00201271996331709\\
4.81	0.00201271996012751\\
4.82	0.00201271995693597\\
4.83	0.00201271995374245\\
4.84	0.00201271995054697\\
4.85	0.00201271994734951\\
4.86	0.00201271994415008\\
4.87	0.00201271994094868\\
4.88	0.0020127199377453\\
4.89	0.00201271993453995\\
4.9	0.00201271993133261\\
4.91	0.00201271992812329\\
4.92	0.002012719924912\\
4.93	0.00201271992169871\\
4.94	0.00201271991848344\\
4.95	0.00201271991526618\\
4.96	0.00201271991204693\\
4.97	0.00201271990882569\\
4.98	0.00201271990560245\\
4.99	0.00201271990237722\\
5	0.00201271989914999\\
5.01	0.00201271989592076\\
5.02	0.00201271989268954\\
5.03	0.00201271988945631\\
5.04	0.00201271988622107\\
5.05	0.00201271988298383\\
5.06	0.00201271987974458\\
5.07	0.00201271987650333\\
5.08	0.00201271987326006\\
5.09	0.00201271987001477\\
5.1	0.00201271986676748\\
5.11	0.00201271986351816\\
5.12	0.00201271986026683\\
5.13	0.00201271985701348\\
5.14	0.00201271985375811\\
5.15	0.00201271985050071\\
5.16	0.00201271984724128\\
5.17	0.00201271984397983\\
5.18	0.00201271984071635\\
5.19	0.00201271983745084\\
5.2	0.0020127198341833\\
5.21	0.00201271983091372\\
5.22	0.0020127198276421\\
5.23	0.00201271982436845\\
5.24	0.00201271982109276\\
5.25	0.00201271981781502\\
5.26	0.00201271981453524\\
5.27	0.00201271981125342\\
5.28	0.00201271980796955\\
5.29	0.00201271980468362\\
5.3	0.00201271980139565\\
5.31	0.00201271979810563\\
5.32	0.00201271979481355\\
5.33	0.00201271979151941\\
5.34	0.00201271978822322\\
5.35	0.00201271978492497\\
5.36	0.00201271978162465\\
5.37	0.00201271977832227\\
5.38	0.00201271977501783\\
5.39	0.00201271977171131\\
5.4	0.00201271976840273\\
5.41	0.00201271976509208\\
5.42	0.00201271976177935\\
5.43	0.00201271975846455\\
5.44	0.00201271975514767\\
5.45	0.00201271975182872\\
5.46	0.00201271974850768\\
5.47	0.00201271974518456\\
5.48	0.00201271974185936\\
5.49	0.00201271973853207\\
5.5	0.00201271973520269\\
5.51	0.00201271973187122\\
5.52	0.00201271972853767\\
5.53	0.00201271972520202\\
5.54	0.00201271972186427\\
5.55	0.00201271971852442\\
5.56	0.00201271971518248\\
5.57	0.00201271971183844\\
5.58	0.00201271970849229\\
5.59	0.00201271970514404\\
5.6	0.00201271970179368\\
5.61	0.00201271969844121\\
5.62	0.00201271969508663\\
5.63	0.00201271969172994\\
5.64	0.00201271968837114\\
5.65	0.00201271968501022\\
5.66	0.00201271968164718\\
5.67	0.00201271967828202\\
5.68	0.00201271967491474\\
5.69	0.00201271967154533\\
5.7	0.0020127196681738\\
5.71	0.00201271966480015\\
5.72	0.00201271966142436\\
5.73	0.00201271965804644\\
5.74	0.00201271965466639\\
5.75	0.0020127196512842\\
5.76	0.00201271964789988\\
5.77	0.00201271964451341\\
5.78	0.00201271964112481\\
5.79	0.00201271963773406\\
5.8	0.00201271963434117\\
5.81	0.00201271963094613\\
5.82	0.00201271962754894\\
5.83	0.0020127196241496\\
5.84	0.00201271962074811\\
5.85	0.00201271961734446\\
5.86	0.00201271961393866\\
5.87	0.0020127196105307\\
5.88	0.00201271960712058\\
5.89	0.00201271960370829\\
5.9	0.00201271960029384\\
5.91	0.00201271959687723\\
5.92	0.00201271959345844\\
5.93	0.00201271959003749\\
5.94	0.00201271958661436\\
5.95	0.00201271958318906\\
5.96	0.00201271957976158\\
5.97	0.00201271957633193\\
5.98	0.00201271957290009\\
5.99	0.00201271956946608\\
6	0.00201271956602987\\
6.01	0.00201271956259149\\
6.02	0.00201271955915091\\
6.03	0.00201271955570814\\
6.04	0.00201271955226319\\
6.05	0.00201271954881604\\
6.06	0.00201271954536669\\
6.07	0.00201271954191514\\
6.08	0.00201271953846139\\
6.09	0.00201271953500545\\
6.1	0.00201271953154729\\
6.11	0.00201271952808694\\
6.12	0.00201271952462437\\
6.13	0.00201271952115959\\
6.14	0.0020127195176926\\
6.15	0.0020127195142234\\
6.16	0.00201271951075198\\
6.17	0.00201271950727835\\
6.18	0.00201271950380249\\
6.19	0.00201271950032441\\
6.2	0.0020127194968441\\
6.21	0.00201271949336157\\
6.22	0.00201271948987681\\
6.23	0.00201271948638983\\
6.24	0.0020127194829006\\
6.25	0.00201271947940915\\
6.26	0.00201271947591546\\
6.27	0.00201271947241952\\
6.28	0.00201271946892135\\
6.29	0.00201271946542094\\
6.3	0.00201271946191828\\
6.31	0.00201271945841337\\
6.32	0.00201271945490622\\
6.33	0.00201271945139682\\
6.34	0.00201271944788516\\
6.35	0.00201271944437125\\
6.36	0.00201271944085508\\
6.37	0.00201271943733665\\
6.38	0.00201271943381596\\
6.39	0.002012719430293\\
6.4	0.00201271942676778\\
6.41	0.0020127194232403\\
6.42	0.00201271941971054\\
6.43	0.00201271941617851\\
6.44	0.00201271941264421\\
6.45	0.00201271940910763\\
6.46	0.00201271940556878\\
6.47	0.00201271940202765\\
6.48	0.00201271939848423\\
6.49	0.00201271939493853\\
6.5	0.00201271939139054\\
6.51	0.00201271938784026\\
6.52	0.0020127193842877\\
6.53	0.00201271938073284\\
6.54	0.00201271937717569\\
6.55	0.00201271937361624\\
6.56	0.00201271937005449\\
6.57	0.00201271936649044\\
6.58	0.00201271936292408\\
6.59	0.00201271935935543\\
6.6	0.00201271935578446\\
6.61	0.00201271935221119\\
6.62	0.0020127193486356\\
6.63	0.0020127193450577\\
6.64	0.00201271934147748\\
6.65	0.00201271933789494\\
6.66	0.00201271933431009\\
6.67	0.00201271933072291\\
6.68	0.00201271932713341\\
6.69	0.00201271932354158\\
6.7	0.00201271931994742\\
6.71	0.00201271931635093\\
6.72	0.00201271931275211\\
6.73	0.00201271930915096\\
6.74	0.00201271930554746\\
6.75	0.00201271930194163\\
6.76	0.00201271929833345\\
6.77	0.00201271929472293\\
6.78	0.00201271929111006\\
6.79	0.00201271928749485\\
6.8	0.00201271928387729\\
6.81	0.00201271928025737\\
6.82	0.0020127192766351\\
6.83	0.00201271927301047\\
6.84	0.00201271926938348\\
6.85	0.00201271926575413\\
6.86	0.00201271926212242\\
6.87	0.00201271925848833\\
6.88	0.00201271925485189\\
6.89	0.00201271925121307\\
6.9	0.00201271924757188\\
6.91	0.00201271924392831\\
6.92	0.00201271924028237\\
6.93	0.00201271923663405\\
6.94	0.00201271923298335\\
6.95	0.00201271922933027\\
6.96	0.00201271922567479\\
6.97	0.00201271922201694\\
6.98	0.00201271921835669\\
6.99	0.00201271921469405\\
7	0.00201271921102901\\
7.01	0.00201271920736158\\
7.02	0.00201271920369175\\
7.03	0.00201271920001951\\
7.04	0.00201271919634488\\
7.05	0.00201271919266783\\
7.06	0.00201271918898838\\
7.07	0.00201271918530652\\
7.08	0.00201271918162225\\
7.09	0.00201271917793556\\
7.1	0.00201271917424646\\
7.11	0.00201271917055493\\
7.12	0.00201271916686098\\
7.13	0.00201271916316462\\
7.14	0.00201271915946582\\
7.15	0.00201271915576459\\
7.16	0.00201271915206094\\
7.17	0.00201271914835485\\
7.18	0.00201271914464633\\
7.19	0.00201271914093537\\
7.2	0.00201271913722197\\
7.21	0.00201271913350613\\
7.22	0.00201271912978784\\
7.23	0.00201271912606711\\
7.24	0.00201271912234392\\
7.25	0.00201271911861829\\
7.26	0.0020127191148902\\
7.27	0.00201271911115966\\
7.28	0.00201271910742666\\
7.29	0.0020127191036912\\
7.3	0.00201271909995328\\
7.31	0.00201271909621289\\
7.32	0.00201271909247003\\
7.33	0.00201271908872471\\
7.34	0.00201271908497691\\
7.35	0.00201271908122664\\
7.36	0.00201271907747389\\
7.37	0.00201271907371866\\
7.38	0.00201271906996095\\
7.39	0.00201271906620076\\
7.4	0.00201271906243808\\
7.41	0.00201271905867291\\
7.42	0.00201271905490526\\
7.43	0.0020127190511351\\
7.44	0.00201271904736246\\
7.45	0.00201271904358732\\
7.46	0.00201271903980967\\
7.47	0.00201271903602953\\
7.48	0.00201271903224687\\
7.49	0.00201271902846172\\
7.5	0.00201271902467405\\
7.51	0.00201271902088387\\
7.52	0.00201271901709118\\
7.53	0.00201271901329597\\
7.54	0.00201271900949824\\
7.55	0.00201271900569799\\
7.56	0.00201271900189521\\
7.57	0.00201271899808991\\
7.58	0.00201271899428209\\
7.59	0.00201271899047173\\
7.6	0.00201271898665883\\
7.61	0.0020127189828434\\
7.62	0.00201271897902544\\
7.63	0.00201271897520493\\
7.64	0.00201271897138188\\
7.65	0.00201271896755629\\
7.66	0.00201271896372814\\
7.67	0.00201271895989745\\
7.68	0.0020127189560642\\
7.69	0.0020127189522284\\
7.7	0.00201271894839004\\
7.71	0.00201271894454913\\
7.72	0.00201271894070565\\
7.73	0.0020127189368596\\
7.74	0.00201271893301099\\
7.75	0.00201271892915981\\
7.76	0.00201271892530606\\
7.77	0.00201271892144973\\
7.78	0.00201271891759082\\
7.79	0.00201271891372934\\
7.8	0.00201271890986528\\
7.81	0.00201271890599863\\
7.82	0.00201271890212939\\
7.83	0.00201271889825757\\
7.84	0.00201271889438315\\
7.85	0.00201271889050614\\
7.86	0.00201271888662653\\
7.87	0.00201271888274433\\
7.88	0.00201271887885952\\
7.89	0.00201271887497211\\
7.9	0.00201271887108209\\
7.91	0.00201271886718946\\
7.92	0.00201271886329422\\
7.93	0.00201271885939637\\
7.94	0.00201271885549591\\
7.95	0.00201271885159282\\
7.96	0.00201271884768711\\
7.97	0.00201271884377878\\
7.98	0.00201271883986782\\
7.99	0.00201271883595423\\
8	0.00201271883203802\\
8.01	0.00201271882811916\\
8.02	0.00201271882419768\\
8.03	0.00201271882027355\\
8.04	0.00201271881634678\\
8.05	0.00201271881241737\\
8.06	0.00201271880848531\\
8.07	0.0020127188045506\\
8.08	0.00201271880061324\\
8.09	0.00201271879667323\\
8.1	0.00201271879273056\\
8.11	0.00201271878878523\\
8.12	0.00201271878483724\\
8.13	0.00201271878088659\\
8.14	0.00201271877693326\\
8.15	0.00201271877297727\\
8.16	0.00201271876901861\\
8.17	0.00201271876505728\\
8.18	0.00201271876109326\\
8.19	0.00201271875712657\\
8.2	0.00201271875315719\\
8.21	0.00201271874918514\\
8.22	0.00201271874521039\\
8.23	0.00201271874123295\\
8.24	0.00201271873725282\\
8.25	0.00201271873327\\
8.26	0.00201271872928448\\
8.27	0.00201271872529626\\
8.28	0.00201271872130534\\
8.29	0.00201271871731171\\
8.3	0.00201271871331537\\
8.31	0.00201271870931632\\
8.32	0.00201271870531456\\
8.33	0.00201271870131009\\
8.34	0.00201271869730289\\
8.35	0.00201271869329298\\
8.36	0.00201271868928034\\
8.37	0.00201271868526498\\
8.38	0.00201271868124688\\
8.39	0.00201271867722606\\
8.4	0.0020127186732025\\
8.41	0.0020127186691762\\
8.42	0.00201271866514717\\
8.43	0.00201271866111539\\
8.44	0.00201271865708087\\
8.45	0.00201271865304361\\
8.46	0.00201271864900359\\
8.47	0.00201271864496082\\
8.48	0.00201271864091529\\
8.49	0.00201271863686701\\
8.5	0.00201271863281597\\
8.51	0.00201271862876216\\
8.52	0.00201271862470559\\
8.53	0.00201271862064625\\
8.54	0.00201271861658414\\
8.55	0.00201271861251926\\
8.56	0.0020127186084516\\
8.57	0.00201271860438116\\
8.58	0.00201271860030794\\
8.59	0.00201271859623193\\
8.6	0.00201271859215314\\
8.61	0.00201271858807155\\
8.62	0.00201271858398718\\
8.63	0.00201271857990001\\
8.64	0.00201271857581004\\
8.65	0.00201271857171727\\
8.66	0.0020127185676217\\
8.67	0.00201271856352332\\
8.68	0.00201271855942214\\
8.69	0.00201271855531814\\
8.7	0.00201271855121132\\
8.71	0.0020127185471017\\
8.72	0.00201271854298925\\
8.73	0.00201271853887398\\
8.74	0.00201271853475588\\
8.75	0.00201271853063496\\
8.76	0.0020127185265112\\
8.77	0.00201271852238462\\
8.78	0.00201271851825519\\
8.79	0.00201271851412293\\
8.8	0.00201271850998783\\
8.81	0.00201271850584988\\
8.82	0.00201271850170909\\
8.83	0.00201271849756545\\
8.84	0.00201271849341895\\
8.85	0.0020127184892696\\
8.86	0.00201271848511739\\
8.87	0.00201271848096232\\
8.88	0.00201271847680438\\
8.89	0.00201271847264359\\
8.9	0.00201271846847992\\
8.91	0.00201271846431337\\
8.92	0.00201271846014396\\
8.93	0.00201271845597167\\
8.94	0.00201271845179649\\
8.95	0.00201271844761844\\
8.96	0.0020127184434375\\
8.97	0.00201271843925367\\
8.98	0.00201271843506694\\
8.99	0.00201271843087733\\
9	0.00201271842668482\\
9.01	0.0020127184224894\\
9.02	0.00201271841829109\\
9.03	0.00201271841408987\\
9.04	0.00201271840988574\\
9.05	0.0020127184056787\\
9.06	0.00201271840146874\\
9.07	0.00201271839725587\\
9.08	0.00201271839304008\\
9.09	0.00201271838882137\\
9.1	0.00201271838459973\\
9.11	0.00201271838037517\\
9.12	0.00201271837614767\\
9.13	0.00201271837191724\\
9.14	0.00201271836768387\\
9.15	0.00201271836344756\\
9.16	0.00201271835920831\\
9.17	0.00201271835496612\\
9.18	0.00201271835072097\\
9.19	0.00201271834647288\\
9.2	0.00201271834222183\\
9.21	0.00201271833796783\\
9.22	0.00201271833371086\\
9.23	0.00201271832945094\\
9.24	0.00201271832518804\\
9.25	0.00201271832092218\\
9.26	0.00201271831665335\\
9.27	0.00201271831238154\\
9.28	0.00201271830810676\\
9.29	0.002012718303829\\
9.3	0.00201271829954825\\
9.31	0.00201271829526452\\
9.32	0.0020127182909778\\
9.33	0.00201271828668809\\
9.34	0.00201271828239538\\
9.35	0.00201271827809968\\
9.36	0.00201271827380098\\
9.37	0.00201271826949927\\
9.38	0.00201271826519456\\
9.39	0.00201271826088684\\
9.4	0.0020127182565761\\
9.41	0.00201271825226236\\
9.42	0.00201271824794559\\
9.43	0.0020127182436258\\
9.44	0.00201271823930299\\
9.45	0.00201271823497715\\
9.46	0.00201271823064828\\
9.47	0.00201271822631638\\
9.48	0.00201271822198145\\
9.49	0.00201271821764347\\
9.5	0.00201271821330245\\
9.51	0.00201271820895839\\
9.52	0.00201271820461128\\
9.53	0.00201271820026113\\
9.54	0.00201271819590791\\
9.55	0.00201271819155164\\
9.56	0.00201271818719232\\
9.57	0.00201271818282992\\
9.58	0.00201271817846447\\
9.59	0.00201271817409594\\
9.6	0.00201271816972435\\
9.61	0.00201271816534968\\
9.62	0.00201271816097193\\
9.63	0.0020127181565911\\
9.64	0.00201271815220719\\
9.65	0.00201271814782019\\
9.66	0.0020127181434301\\
9.67	0.00201271813903692\\
9.68	0.00201271813464064\\
9.69	0.00201271813024126\\
9.7	0.00201271812583879\\
9.71	0.00201271812143321\\
9.72	0.00201271811702452\\
9.73	0.00201271811261272\\
9.74	0.0020127181081978\\
9.75	0.00201271810377977\\
9.76	0.00201271809935862\\
9.77	0.00201271809493434\\
9.78	0.00201271809050694\\
9.79	0.00201271808607641\\
9.8	0.00201271808164275\\
9.81	0.00201271807720596\\
9.82	0.00201271807276602\\
9.83	0.00201271806832294\\
9.84	0.00201271806387672\\
9.85	0.00201271805942735\\
9.86	0.00201271805497483\\
9.87	0.00201271805051916\\
9.88	0.00201271804606032\\
9.89	0.00201271804159833\\
9.9	0.00201271803713318\\
9.91	0.00201271803266486\\
9.92	0.00201271802819337\\
9.93	0.00201271802371871\\
9.94	0.00201271801924087\\
9.95	0.00201271801475985\\
9.96	0.00201271801027565\\
9.97	0.00201271800578827\\
9.98	0.0020127180012977\\
9.99	0.00201271799680394\\
10	0.00201271799230698\\
10.01	0.00201271798780682\\
10.02	0.00201271798330347\\
10.03	0.00201271797879691\\
10.04	0.00201271797428714\\
10.05	0.00201271796977417\\
10.06	0.00201271796525798\\
10.07	0.00201271796073857\\
10.08	0.00201271795621595\\
10.09	0.0020127179516901\\
10.1	0.00201271794716102\\
10.11	0.00201271794262872\\
10.12	0.00201271793809319\\
10.13	0.00201271793355441\\
10.14	0.0020127179290124\\
10.15	0.00201271792446715\\
10.16	0.00201271791991866\\
10.17	0.00201271791536691\\
10.18	0.00201271791081191\\
10.19	0.00201271790625366\\
10.2	0.00201271790169215\\
10.21	0.00201271789712739\\
10.22	0.00201271789255935\\
10.23	0.00201271788798805\\
10.24	0.00201271788341348\\
10.25	0.00201271787883563\\
10.26	0.00201271787425451\\
10.27	0.0020127178696701\\
10.28	0.00201271786508241\\
10.29	0.00201271786049144\\
10.3	0.00201271785589717\\
10.31	0.00201271785129962\\
10.32	0.00201271784669876\\
10.33	0.00201271784209461\\
10.34	0.00201271783748715\\
10.35	0.00201271783287639\\
10.36	0.00201271782826232\\
10.37	0.00201271782364493\\
10.38	0.00201271781902423\\
10.39	0.0020127178144002\\
10.4	0.00201271780977286\\
10.41	0.00201271780514219\\
10.42	0.00201271780050819\\
10.43	0.00201271779587086\\
10.44	0.00201271779123019\\
10.45	0.00201271778658618\\
10.46	0.00201271778193883\\
10.47	0.00201271777728814\\
10.48	0.00201271777263409\\
10.49	0.0020127177679767\\
10.5	0.00201271776331594\\
10.51	0.00201271775865183\\
10.52	0.00201271775398436\\
10.53	0.00201271774931352\\
10.54	0.00201271774463931\\
10.55	0.00201271773996173\\
10.56	0.00201271773528077\\
10.57	0.00201271773059644\\
10.58	0.00201271772590872\\
10.59	0.00201271772121762\\
10.6	0.00201271771652313\\
10.61	0.00201271771182524\\
10.62	0.00201271770712396\\
10.63	0.00201271770241928\\
10.64	0.0020127176977112\\
10.65	0.00201271769299972\\
10.66	0.00201271768828482\\
10.67	0.00201271768356651\\
10.68	0.00201271767884479\\
10.69	0.00201271767411965\\
10.7	0.00201271766939108\\
10.71	0.00201271766465909\\
10.72	0.00201271765992367\\
10.73	0.00201271765518482\\
10.74	0.00201271765044253\\
10.75	0.0020127176456968\\
10.76	0.00201271764094762\\
10.77	0.00201271763619501\\
10.78	0.00201271763143894\\
10.79	0.00201271762667942\\
10.8	0.00201271762191644\\
10.81	0.00201271761715\\
10.82	0.0020127176123801\\
10.83	0.00201271760760673\\
10.84	0.00201271760282989\\
10.85	0.00201271759804958\\
10.86	0.00201271759326579\\
10.87	0.00201271758847852\\
10.88	0.00201271758368777\\
10.89	0.00201271757889353\\
10.9	0.00201271757409579\\
10.91	0.00201271756929457\\
10.92	0.00201271756448984\\
10.93	0.00201271755968162\\
10.94	0.00201271755486989\\
10.95	0.00201271755005465\\
10.96	0.0020127175452359\\
10.97	0.00201271754041363\\
10.98	0.00201271753558785\\
10.99	0.00201271753075854\\
11	0.00201271752592571\\
11.01	0.00201271752108935\\
11.02	0.00201271751624946\\
11.03	0.00201271751140603\\
11.04	0.00201271750655906\\
11.05	0.00201271750170855\\
11.06	0.00201271749685449\\
11.07	0.00201271749199688\\
11.08	0.00201271748713572\\
11.09	0.002012717482271\\
11.1	0.00201271747740272\\
11.11	0.00201271747253088\\
11.12	0.00201271746765547\\
11.13	0.00201271746277648\\
11.14	0.00201271745789392\\
11.15	0.00201271745300779\\
11.16	0.00201271744811807\\
11.17	0.00201271744322477\\
11.18	0.00201271743832788\\
11.19	0.00201271743342739\\
11.2	0.00201271742852331\\
11.21	0.00201271742361563\\
11.22	0.00201271741870435\\
11.23	0.00201271741378946\\
11.24	0.00201271740887096\\
11.25	0.00201271740394885\\
11.26	0.00201271739902311\\
11.27	0.00201271739409376\\
11.28	0.00201271738916078\\
11.29	0.00201271738422418\\
11.3	0.00201271737928394\\
11.31	0.00201271737434007\\
11.32	0.00201271736939256\\
11.33	0.0020127173644414\\
11.34	0.0020127173594866\\
11.35	0.00201271735452816\\
11.36	0.00201271734956605\\
11.37	0.00201271734460029\\
11.38	0.00201271733963087\\
11.39	0.00201271733465779\\
11.4	0.00201271732968103\\
11.41	0.00201271732470061\\
11.42	0.00201271731971651\\
11.43	0.00201271731472873\\
11.44	0.00201271730973727\\
11.45	0.00201271730474212\\
11.46	0.00201271729974329\\
11.47	0.00201271729474076\\
11.48	0.00201271728973453\\
11.49	0.0020127172847246\\
11.5	0.00201271727971097\\
11.51	0.00201271727469363\\
11.52	0.00201271726967258\\
11.53	0.00201271726464781\\
11.54	0.00201271725961932\\
11.55	0.00201271725458712\\
11.56	0.00201271724955118\\
11.57	0.00201271724451152\\
11.58	0.00201271723946812\\
11.59	0.00201271723442098\\
11.6	0.0020127172293701\\
11.61	0.00201271722431548\\
11.62	0.00201271721925711\\
11.63	0.00201271721419499\\
11.64	0.00201271720912911\\
11.65	0.00201271720405947\\
11.66	0.00201271719898607\\
11.67	0.0020127171939089\\
11.68	0.00201271718882796\\
11.69	0.00201271718374324\\
11.7	0.00201271717865475\\
11.71	0.00201271717356247\\
11.72	0.00201271716846641\\
11.73	0.00201271716336656\\
11.74	0.00201271715826292\\
11.75	0.00201271715315548\\
11.76	0.00201271714804423\\
11.77	0.00201271714292919\\
11.78	0.00201271713781033\\
11.79	0.00201271713268767\\
11.8	0.00201271712756118\\
11.81	0.00201271712243088\\
11.82	0.00201271711729675\\
11.83	0.0020127171121588\\
11.84	0.00201271710701702\\
11.85	0.0020127171018714\\
11.86	0.00201271709672194\\
11.87	0.00201271709156864\\
11.88	0.0020127170864115\\
11.89	0.0020127170812505\\
11.9	0.00201271707608565\\
11.91	0.00201271707091695\\
11.92	0.00201271706574438\\
11.93	0.00201271706056795\\
11.94	0.00201271705538764\\
11.95	0.00201271705020347\\
11.96	0.00201271704501542\\
11.97	0.00201271703982349\\
11.98	0.00201271703462767\\
11.99	0.00201271702942796\\
12	0.00201271702422437\\
12.01	0.00201271701901688\\
12.02	0.00201271701380548\\
12.03	0.00201271700859019\\
12.04	0.00201271700337099\\
12.05	0.00201271699814787\\
12.06	0.00201271699292084\\
12.07	0.00201271698768989\\
12.08	0.00201271698245502\\
12.09	0.00201271697721623\\
12.1	0.0020127169719735\\
12.11	0.00201271696672684\\
12.12	0.00201271696147623\\
12.13	0.00201271695622169\\
12.14	0.0020127169509632\\
12.15	0.00201271694570076\\
12.16	0.00201271694043437\\
12.17	0.00201271693516402\\
12.18	0.00201271692988971\\
12.19	0.00201271692461143\\
12.2	0.00201271691932918\\
12.21	0.00201271691404296\\
12.22	0.00201271690875277\\
12.23	0.00201271690345859\\
12.24	0.00201271689816042\\
12.25	0.00201271689285827\\
12.26	0.00201271688755212\\
12.27	0.00201271688224198\\
12.28	0.00201271687692784\\
12.29	0.00201271687160968\\
12.3	0.00201271686628753\\
12.31	0.00201271686096136\\
12.32	0.00201271685563117\\
12.33	0.00201271685029696\\
12.34	0.00201271684495873\\
12.35	0.00201271683961647\\
12.36	0.00201271683427017\\
12.37	0.00201271682891985\\
12.38	0.00201271682356548\\
12.39	0.00201271681820706\\
12.4	0.0020127168128446\\
12.41	0.00201271680747808\\
12.42	0.00201271680210751\\
12.43	0.00201271679673288\\
12.44	0.00201271679135418\\
12.45	0.00201271678597142\\
12.46	0.00201271678058458\\
12.47	0.00201271677519367\\
12.48	0.00201271676979867\\
12.49	0.00201271676439959\\
12.5	0.00201271675899643\\
12.51	0.00201271675358917\\
12.52	0.00201271674817781\\
12.53	0.00201271674276235\\
12.54	0.00201271673734279\\
12.55	0.00201271673191913\\
12.56	0.00201271672649134\\
12.57	0.00201271672105945\\
12.58	0.00201271671562343\\
12.59	0.00201271671018328\\
12.6	0.00201271670473901\\
12.61	0.00201271669929061\\
12.62	0.00201271669383806\\
12.63	0.00201271668838138\\
12.64	0.00201271668292055\\
12.65	0.00201271667745558\\
12.66	0.00201271667198645\\
12.67	0.00201271666651316\\
12.68	0.00201271666103571\\
12.69	0.0020127166555541\\
12.7	0.00201271665006832\\
12.71	0.00201271664457836\\
12.72	0.00201271663908423\\
12.73	0.00201271663358591\\
12.74	0.00201271662808341\\
12.75	0.00201271662257672\\
12.76	0.00201271661706584\\
12.77	0.00201271661155075\\
12.78	0.00201271660603147\\
12.79	0.00201271660050798\\
12.8	0.00201271659498028\\
12.81	0.00201271658944836\\
12.82	0.00201271658391223\\
12.83	0.00201271657837187\\
12.84	0.00201271657282729\\
12.85	0.00201271656727847\\
12.86	0.00201271656172542\\
12.87	0.00201271655616813\\
12.88	0.00201271655060659\\
12.89	0.00201271654504081\\
12.9	0.00201271653947078\\
12.91	0.00201271653389649\\
12.92	0.00201271652831794\\
12.93	0.00201271652273513\\
12.94	0.00201271651714805\\
12.95	0.0020127165115567\\
12.96	0.00201271650596107\\
12.97	0.00201271650036116\\
12.98	0.00201271649475696\\
12.99	0.00201271648914847\\
13	0.0020127164835357\\
13.01	0.00201271647791862\\
13.02	0.00201271647229724\\
13.03	0.00201271646667156\\
13.04	0.00201271646104157\\
13.05	0.00201271645540726\\
13.06	0.00201271644976863\\
13.07	0.00201271644412568\\
13.08	0.0020127164384784\\
13.09	0.00201271643282679\\
13.1	0.00201271642717085\\
13.11	0.00201271642151057\\
13.12	0.00201271641584594\\
13.13	0.00201271641017697\\
13.14	0.00201271640450364\\
13.15	0.00201271639882595\\
13.16	0.00201271639314391\\
13.17	0.0020127163874575\\
13.18	0.00201271638176672\\
13.19	0.00201271637607157\\
13.2	0.00201271637037204\\
13.21	0.00201271636466813\\
13.22	0.00201271635895983\\
13.23	0.00201271635324714\\
13.24	0.00201271634753006\\
13.25	0.00201271634180857\\
13.26	0.00201271633608269\\
13.27	0.00201271633035239\\
13.28	0.00201271632461769\\
13.29	0.00201271631887857\\
13.3	0.00201271631313502\\
13.31	0.00201271630738706\\
13.32	0.00201271630163466\\
13.33	0.00201271629587784\\
13.34	0.00201271629011657\\
13.35	0.00201271628435086\\
13.36	0.00201271627858071\\
13.37	0.0020127162728061\\
13.38	0.00201271626702704\\
13.39	0.00201271626124352\\
13.4	0.00201271625545554\\
13.41	0.00201271624966309\\
13.42	0.00201271624386617\\
13.43	0.00201271623806477\\
13.44	0.00201271623225889\\
13.45	0.00201271622644853\\
13.46	0.00201271622063368\\
13.47	0.00201271621481433\\
13.48	0.00201271620899049\\
13.49	0.00201271620316214\\
13.5	0.00201271619732929\\
13.51	0.00201271619149192\\
13.52	0.00201271618565004\\
13.53	0.00201271617980364\\
13.54	0.00201271617395272\\
13.55	0.00201271616809727\\
13.56	0.00201271616223728\\
13.57	0.00201271615637276\\
13.58	0.0020127161505037\\
13.59	0.00201271614463009\\
13.6	0.00201271613875193\\
13.61	0.00201271613286922\\
13.62	0.00201271612698195\\
13.63	0.00201271612109011\\
13.64	0.0020127161151937\\
13.65	0.00201271610929273\\
13.66	0.00201271610338718\\
13.67	0.00201271609747704\\
13.68	0.00201271609156232\\
13.69	0.00201271608564301\\
13.7	0.00201271607971911\\
13.71	0.00201271607379061\\
13.72	0.00201271606785751\\
13.73	0.0020127160619198\\
13.74	0.00201271605597747\\
13.75	0.00201271605003053\\
13.76	0.00201271604407897\\
13.77	0.00201271603812279\\
13.78	0.00201271603216197\\
13.79	0.00201271602619652\\
13.8	0.00201271602022644\\
13.81	0.00201271601425171\\
13.82	0.00201271600827233\\
13.83	0.0020127160022883\\
13.84	0.00201271599629962\\
13.85	0.00201271599030627\\
13.86	0.00201271598430826\\
13.87	0.00201271597830558\\
13.88	0.00201271597229823\\
13.89	0.0020127159662862\\
13.9	0.00201271596026948\\
13.91	0.00201271595424808\\
13.92	0.00201271594822199\\
13.93	0.0020127159421912\\
13.94	0.00201271593615571\\
13.95	0.00201271593011551\\
13.96	0.0020127159240706\\
13.97	0.00201271591802099\\
13.98	0.00201271591196665\\
13.99	0.00201271590590759\\
14	0.0020127158998438\\
14.01	0.00201271589377528\\
14.02	0.00201271588770202\\
14.03	0.00201271588162402\\
14.04	0.00201271587554128\\
14.05	0.00201271586945379\\
14.06	0.00201271586336154\\
14.07	0.00201271585726453\\
14.08	0.00201271585116276\\
14.09	0.00201271584505622\\
14.1	0.00201271583894491\\
14.11	0.00201271583282882\\
14.12	0.00201271582670795\\
14.13	0.00201271582058229\\
14.14	0.00201271581445185\\
14.15	0.0020127158083166\\
14.16	0.00201271580217656\\
14.17	0.00201271579603172\\
14.18	0.00201271578988206\\
14.19	0.00201271578372759\\
14.2	0.0020127157775683\\
14.21	0.0020127157714042\\
14.22	0.00201271576523526\\
14.23	0.00201271575906149\\
14.24	0.00201271575288289\\
14.25	0.00201271574669945\\
14.26	0.00201271574051116\\
14.27	0.00201271573431802\\
14.28	0.00201271572812002\\
14.29	0.00201271572191717\\
14.3	0.00201271571570946\\
14.31	0.00201271570949687\\
14.32	0.00201271570327942\\
14.33	0.00201271569705708\\
14.34	0.00201271569082987\\
14.35	0.00201271568459776\\
14.36	0.00201271567836077\\
14.37	0.00201271567211888\\
14.38	0.0020127156658721\\
14.39	0.0020127156596204\\
14.4	0.0020127156533638\\
14.41	0.00201271564710228\\
14.42	0.00201271564083585\\
14.43	0.00201271563456449\\
14.44	0.00201271562828821\\
14.45	0.00201271562200699\\
14.46	0.00201271561572084\\
14.47	0.00201271560942974\\
14.48	0.0020127156031337\\
14.49	0.0020127155968327\\
14.5	0.00201271559052676\\
14.51	0.00201271558421585\\
14.52	0.00201271557789997\\
14.53	0.00201271557157913\\
14.54	0.00201271556525332\\
14.55	0.00201271555892252\\
14.56	0.00201271555258675\\
14.57	0.00201271554624598\\
14.58	0.00201271553990022\\
14.59	0.00201271553354947\\
14.6	0.00201271552719371\\
14.61	0.00201271552083295\\
14.62	0.00201271551446718\\
14.63	0.00201271550809639\\
14.64	0.00201271550172058\\
14.65	0.00201271549533975\\
14.66	0.00201271548895388\\
14.67	0.00201271548256299\\
14.68	0.00201271547616705\\
14.69	0.00201271546976607\\
14.7	0.00201271546336004\\
14.71	0.00201271545694896\\
14.72	0.00201271545053283\\
14.73	0.00201271544411163\\
14.74	0.00201271543768536\\
14.75	0.00201271543125402\\
14.76	0.0020127154248176\\
14.77	0.00201271541837611\\
14.78	0.00201271541192953\\
14.79	0.00201271540547786\\
14.8	0.00201271539902109\\
14.81	0.00201271539255922\\
14.82	0.00201271538609226\\
14.83	0.00201271537962018\\
14.84	0.00201271537314299\\
14.85	0.00201271536666067\\
14.86	0.00201271536017324\\
14.87	0.00201271535368068\\
14.88	0.00201271534718298\\
14.89	0.00201271534068015\\
14.9	0.00201271533417218\\
14.91	0.00201271532765906\\
14.92	0.00201271532114079\\
14.93	0.00201271531461737\\
14.94	0.00201271530808878\\
14.95	0.00201271530155503\\
14.96	0.00201271529501611\\
14.97	0.00201271528847201\\
14.98	0.00201271528192273\\
14.99	0.00201271527536827\\
15	0.00201271526880862\\
15.01	0.00201271526224378\\
15.02	0.00201271525567374\\
15.03	0.00201271524909849\\
15.04	0.00201271524251804\\
15.05	0.00201271523593237\\
15.06	0.00201271522934149\\
15.07	0.00201271522274539\\
15.08	0.00201271521614406\\
15.09	0.00201271520953749\\
15.1	0.00201271520292569\\
15.11	0.00201271519630866\\
15.12	0.00201271518968637\\
15.13	0.00201271518305883\\
15.14	0.00201271517642604\\
15.15	0.00201271516978799\\
15.16	0.00201271516314467\\
15.17	0.00201271515649609\\
15.18	0.00201271514984223\\
15.19	0.00201271514318309\\
15.2	0.00201271513651867\\
15.21	0.00201271512984896\\
15.22	0.00201271512317396\\
15.23	0.00201271511649365\\
15.24	0.00201271510980805\\
15.25	0.00201271510311714\\
15.26	0.00201271509642091\\
15.27	0.00201271508971937\\
15.28	0.00201271508301251\\
15.29	0.00201271507630032\\
15.3	0.0020127150695828\\
15.31	0.00201271506285994\\
15.32	0.00201271505613175\\
15.33	0.0020127150493982\\
15.34	0.00201271504265931\\
15.35	0.00201271503591506\\
15.36	0.00201271502916545\\
15.37	0.00201271502241048\\
15.38	0.00201271501565014\\
15.39	0.00201271500888442\\
15.4	0.00201271500211332\\
15.41	0.00201271499533684\\
15.42	0.00201271498855497\\
15.43	0.00201271498176771\\
15.44	0.00201271497497505\\
15.45	0.00201271496817699\\
15.46	0.00201271496137351\\
15.47	0.00201271495456463\\
15.48	0.00201271494775033\\
15.49	0.0020127149409306\\
15.5	0.00201271493410545\\
15.51	0.00201271492727487\\
15.52	0.00201271492043885\\
15.53	0.00201271491359739\\
15.54	0.00201271490675049\\
15.55	0.00201271489989813\\
15.56	0.00201271489304032\\
15.57	0.00201271488617704\\
15.58	0.0020127148793083\\
15.59	0.00201271487243409\\
15.6	0.00201271486555441\\
15.61	0.00201271485866925\\
15.62	0.0020127148517786\\
15.63	0.00201271484488246\\
15.64	0.00201271483798083\\
15.65	0.00201271483107369\\
15.66	0.00201271482416106\\
15.67	0.00201271481724292\\
15.68	0.00201271481031926\\
15.69	0.00201271480339008\\
15.7	0.00201271479645538\\
15.71	0.00201271478951516\\
15.72	0.0020127147825694\\
15.73	0.0020127147756181\\
15.74	0.00201271476866126\\
15.75	0.00201271476169888\\
15.76	0.00201271475473094\\
15.77	0.00201271474775744\\
15.78	0.00201271474077839\\
15.79	0.00201271473379376\\
15.8	0.00201271472680357\\
15.81	0.0020127147198078\\
15.82	0.00201271471280645\\
15.83	0.00201271470579951\\
15.84	0.00201271469878698\\
15.85	0.00201271469176886\\
15.86	0.00201271468474514\\
15.87	0.00201271467771581\\
15.88	0.00201271467068087\\
15.89	0.00201271466364032\\
15.9	0.00201271465659414\\
15.91	0.00201271464954235\\
15.92	0.00201271464248492\\
15.93	0.00201271463542186\\
15.94	0.00201271462835316\\
15.95	0.00201271462127882\\
15.96	0.00201271461419883\\
15.97	0.00201271460711318\\
15.98	0.00201271460002188\\
15.99	0.00201271459292491\\
16	0.00201271458582228\\
16.01	0.00201271457871397\\
16.02	0.00201271457159999\\
16.03	0.00201271456448032\\
16.04	0.00201271455735497\\
16.05	0.00201271455022392\\
16.06	0.00201271454308718\\
16.07	0.00201271453594474\\
16.08	0.00201271452879659\\
16.09	0.00201271452164272\\
16.1	0.00201271451448315\\
16.11	0.00201271450731785\\
16.12	0.00201271450014682\\
16.13	0.00201271449297007\\
16.14	0.00201271448578757\\
16.15	0.00201271447859934\\
16.16	0.00201271447140537\\
16.17	0.00201271446420564\\
16.18	0.00201271445700016\\
16.19	0.00201271444978891\\
16.2	0.00201271444257191\\
16.21	0.00201271443534913\\
16.22	0.00201271442812058\\
16.23	0.00201271442088626\\
16.24	0.00201271441364614\\
16.25	0.00201271440640024\\
16.26	0.00201271439914854\\
16.27	0.00201271439189105\\
16.28	0.00201271438462775\\
16.29	0.00201271437735865\\
16.3	0.00201271437008373\\
16.31	0.002012714362803\\
16.32	0.00201271435551644\\
16.33	0.00201271434822405\\
16.34	0.00201271434092583\\
16.35	0.00201271433362178\\
16.36	0.00201271432631188\\
16.37	0.00201271431899614\\
16.38	0.00201271431167454\\
16.39	0.00201271430434709\\
16.4	0.00201271429701378\\
16.41	0.0020127142896746\\
16.42	0.00201271428232955\\
16.43	0.00201271427497862\\
16.44	0.00201271426762182\\
16.45	0.00201271426025912\\
16.46	0.00201271425289054\\
16.47	0.00201271424551606\\
16.48	0.00201271423813569\\
16.49	0.0020127142307494\\
16.5	0.00201271422335721\\
16.51	0.00201271421595911\\
16.52	0.00201271420855508\\
16.53	0.00201271420114513\\
16.54	0.00201271419372925\\
16.55	0.00201271418630744\\
16.56	0.00201271417887969\\
16.57	0.002012714171446\\
16.58	0.00201271416400635\\
16.59	0.00201271415656076\\
16.6	0.0020127141491092\\
16.61	0.00201271414165169\\
16.62	0.0020127141341882\\
16.63	0.00201271412671875\\
16.64	0.00201271411924332\\
16.65	0.0020127141117619\\
16.66	0.0020127141042745\\
16.67	0.00201271409678111\\
16.68	0.00201271408928172\\
16.69	0.00201271408177633\\
16.7	0.00201271407426493\\
16.71	0.00201271406674752\\
16.72	0.0020127140592241\\
16.73	0.00201271405169465\\
16.74	0.00201271404415919\\
16.75	0.00201271403661769\\
16.76	0.00201271402907015\\
16.77	0.00201271402151658\\
16.78	0.00201271401395696\\
16.79	0.00201271400639129\\
16.8	0.00201271399881957\\
16.81	0.00201271399124179\\
16.82	0.00201271398365794\\
16.83	0.00201271397606803\\
16.84	0.00201271396847205\\
16.85	0.00201271396086998\\
16.86	0.00201271395326184\\
16.87	0.0020127139456476\\
16.88	0.00201271393802727\\
16.89	0.00201271393040085\\
16.9	0.00201271392276832\\
16.91	0.00201271391512969\\
16.92	0.00201271390748494\\
16.93	0.00201271389983408\\
16.94	0.0020127138921771\\
16.95	0.00201271388451399\\
16.96	0.00201271387684475\\
16.97	0.00201271386916937\\
16.98	0.00201271386148785\\
16.99	0.00201271385380019\\
17	0.00201271384610638\\
17.01	0.00201271383840642\\
17.02	0.00201271383070029\\
17.03	0.002012713822988\\
17.04	0.00201271381526954\\
17.05	0.00201271380754491\\
17.06	0.0020127137998141\\
17.07	0.00201271379207711\\
17.08	0.00201271378433393\\
17.09	0.00201271377658455\\
17.1	0.00201271376882898\\
17.11	0.0020127137610672\\
17.12	0.00201271375329922\\
17.13	0.00201271374552503\\
17.14	0.00201271373774462\\
17.15	0.00201271372995799\\
17.16	0.00201271372216513\\
17.17	0.00201271371436604\\
17.18	0.00201271370656071\\
17.19	0.00201271369874915\\
17.2	0.00201271369093134\\
17.21	0.00201271368310728\\
17.22	0.00201271367527697\\
17.23	0.00201271366744039\\
17.24	0.00201271365959755\\
17.25	0.00201271365174845\\
17.26	0.00201271364389307\\
17.27	0.00201271363603141\\
17.28	0.00201271362816347\\
17.29	0.00201271362028924\\
17.3	0.00201271361240872\\
17.31	0.00201271360452191\\
17.32	0.00201271359662879\\
17.33	0.00201271358872936\\
17.34	0.00201271358082362\\
17.35	0.00201271357291157\\
17.36	0.0020127135649932\\
17.37	0.0020127135570685\\
17.38	0.00201271354913747\\
17.39	0.0020127135412001\\
17.4	0.0020127135332564\\
17.41	0.00201271352530636\\
17.42	0.00201271351734996\\
17.43	0.00201271350938721\\
17.44	0.0020127135014181\\
17.45	0.00201271349344263\\
17.46	0.00201271348546079\\
17.47	0.00201271347747258\\
17.48	0.002012713469478\\
17.49	0.00201271346147703\\
17.5	0.00201271345346967\\
17.51	0.00201271344545593\\
17.52	0.00201271343743579\\
17.53	0.00201271342940925\\
17.54	0.0020127134213763\\
17.55	0.00201271341333695\\
17.56	0.00201271340529118\\
17.57	0.00201271339723899\\
17.58	0.00201271338918038\\
17.59	0.00201271338111534\\
17.6	0.00201271337304386\\
17.61	0.00201271336496595\\
17.62	0.0020127133568816\\
17.63	0.00201271334879081\\
17.64	0.00201271334069356\\
17.65	0.00201271333258985\\
17.66	0.00201271332447969\\
17.67	0.00201271331636305\\
17.68	0.00201271330823996\\
17.69	0.00201271330011038\\
17.7	0.00201271329197433\\
17.71	0.00201271328383179\\
17.72	0.00201271327568277\\
17.73	0.00201271326752725\\
17.74	0.00201271325936523\\
17.75	0.00201271325119672\\
17.76	0.00201271324302169\\
17.77	0.00201271323484016\\
17.78	0.00201271322665211\\
17.79	0.00201271321845754\\
17.8	0.00201271321025645\\
17.81	0.00201271320204882\\
17.82	0.00201271319383466\\
17.83	0.00201271318561397\\
17.84	0.00201271317738673\\
17.85	0.00201271316915294\\
17.86	0.00201271316091261\\
17.87	0.00201271315266571\\
17.88	0.00201271314441225\\
17.89	0.00201271313615223\\
17.9	0.00201271312788564\\
17.91	0.00201271311961247\\
17.92	0.00201271311133273\\
17.93	0.0020127131030464\\
17.94	0.00201271309475348\\
17.95	0.00201271308645397\\
17.96	0.00201271307814786\\
17.97	0.00201271306983515\\
17.98	0.00201271306151584\\
17.99	0.00201271305318991\\
18	0.00201271304485737\\
18.01	0.0020127130365182\\
18.02	0.00201271302817242\\
18.03	0.00201271301982\\
18.04	0.00201271301146096\\
18.05	0.00201271300309527\\
18.06	0.00201271299472294\\
18.07	0.00201271298634397\\
18.08	0.00201271297795834\\
18.09	0.00201271296956606\\
18.1	0.00201271296116712\\
18.11	0.00201271295276152\\
18.12	0.00201271294434924\\
18.13	0.00201271293593029\\
18.14	0.00201271292750467\\
18.15	0.00201271291907236\\
18.16	0.00201271291063337\\
18.17	0.00201271290218769\\
18.18	0.00201271289373531\\
18.19	0.00201271288527623\\
18.2	0.00201271287681045\\
18.21	0.00201271286833795\\
18.22	0.00201271285985875\\
18.23	0.00201271285137283\\
18.24	0.00201271284288018\\
18.25	0.00201271283438081\\
18.26	0.00201271282587471\\
18.27	0.00201271281736188\\
18.28	0.00201271280884231\\
18.29	0.00201271280031599\\
18.3	0.00201271279178293\\
18.31	0.00201271278324311\\
18.32	0.00201271277469654\\
18.33	0.0020127127661432\\
18.34	0.0020127127575831\\
18.35	0.00201271274901624\\
18.36	0.00201271274044259\\
18.37	0.00201271273186217\\
18.38	0.00201271272327497\\
18.39	0.00201271271468098\\
18.4	0.0020127127060802\\
18.41	0.00201271269747263\\
18.42	0.00201271268885825\\
18.43	0.00201271268023708\\
18.44	0.00201271267160909\\
18.45	0.00201271266297429\\
18.46	0.00201271265433268\\
18.47	0.00201271264568424\\
18.48	0.00201271263702898\\
18.49	0.00201271262836689\\
18.5	0.00201271261969797\\
18.51	0.00201271261102221\\
18.52	0.00201271260233961\\
18.53	0.00201271259365016\\
18.54	0.00201271258495386\\
18.55	0.00201271257625071\\
18.56	0.0020127125675407\\
18.57	0.00201271255882382\\
18.58	0.00201271255010008\\
18.59	0.00201271254136947\\
18.6	0.00201271253263198\\
18.61	0.00201271252388761\\
18.62	0.00201271251513636\\
18.63	0.00201271250637823\\
18.64	0.0020127124976132\\
18.65	0.00201271248884127\\
18.66	0.00201271248006244\\
18.67	0.00201271247127671\\
18.68	0.00201271246248407\\
18.69	0.00201271245368452\\
18.7	0.00201271244487805\\
18.71	0.00201271243606467\\
18.72	0.00201271242724435\\
18.73	0.00201271241841711\\
18.74	0.00201271240958294\\
18.75	0.00201271240074182\\
18.76	0.00201271239189377\\
18.77	0.00201271238303878\\
18.78	0.00201271237417683\\
18.79	0.00201271236530793\\
18.8	0.00201271235643207\\
18.81	0.00201271234754926\\
18.82	0.00201271233865947\\
18.83	0.00201271232976272\\
18.84	0.002012712320859\\
18.85	0.0020127123119483\\
18.86	0.00201271230303062\\
18.87	0.00201271229410595\\
18.88	0.0020127122851743\\
18.89	0.00201271227623565\\
18.9	0.00201271226729\\
18.91	0.00201271225833736\\
18.92	0.00201271224937771\\
18.93	0.00201271224041105\\
18.94	0.00201271223143738\\
18.95	0.00201271222245669\\
18.96	0.00201271221346898\\
18.97	0.00201271220447425\\
18.98	0.0020127121954725\\
18.99	0.00201271218646371\\
19	0.00201271217744788\\
19.01	0.00201271216842502\\
19.02	0.00201271215939511\\
19.03	0.00201271215035815\\
19.04	0.00201271214131415\\
19.05	0.00201271213226308\\
19.06	0.00201271212320497\\
19.07	0.00201271211413979\\
19.08	0.00201271210506754\\
19.09	0.00201271209598822\\
19.1	0.00201271208690183\\
19.11	0.00201271207780837\\
19.12	0.00201271206870782\\
19.13	0.00201271205960019\\
19.14	0.00201271205048546\\
19.15	0.00201271204136365\\
19.16	0.00201271203223474\\
19.17	0.00201271202309874\\
19.18	0.00201271201395563\\
19.19	0.00201271200480541\\
19.2	0.00201271199564808\\
19.21	0.00201271198648364\\
19.22	0.00201271197731208\\
19.23	0.0020127119681334\\
19.24	0.0020127119589476\\
19.25	0.00201271194975467\\
19.26	0.0020127119405546\\
19.27	0.0020127119313474\\
19.28	0.00201271192213306\\
19.29	0.00201271191291158\\
19.3	0.00201271190368295\\
19.31	0.00201271189444718\\
19.32	0.00201271188520425\\
19.33	0.00201271187595416\\
19.34	0.00201271186669691\\
19.35	0.0020127118574325\\
19.36	0.00201271184816092\\
19.37	0.00201271183888218\\
19.38	0.00201271182959625\\
19.39	0.00201271182030316\\
19.4	0.00201271181100287\\
19.41	0.00201271180169541\\
19.42	0.00201271179238076\\
19.43	0.00201271178305892\\
19.44	0.00201271177372988\\
19.45	0.00201271176439365\\
19.46	0.00201271175505021\\
19.47	0.00201271174569957\\
19.48	0.00201271173634173\\
19.49	0.00201271172697667\\
19.5	0.0020127117176044\\
19.51	0.00201271170822491\\
19.52	0.0020127116988382\\
19.53	0.00201271168944427\\
19.54	0.00201271168004311\\
19.55	0.00201271167063472\\
19.56	0.0020127116612191\\
19.57	0.00201271165179624\\
19.58	0.00201271164236614\\
19.59	0.0020127116329288\\
19.6	0.00201271162348421\\
19.61	0.00201271161403237\\
19.62	0.00201271160457328\\
19.63	0.00201271159510694\\
19.64	0.00201271158563333\\
19.65	0.00201271157615247\\
19.66	0.00201271156666434\\
19.67	0.00201271155716895\\
19.68	0.00201271154766628\\
19.69	0.00201271153815634\\
19.7	0.00201271152863912\\
19.71	0.00201271151911463\\
19.72	0.00201271150958285\\
19.73	0.00201271150004379\\
19.74	0.00201271149049744\\
19.75	0.0020127114809438\\
19.76	0.00201271147138286\\
19.77	0.00201271146181463\\
19.78	0.0020127114522391\\
19.79	0.00201271144265626\\
19.8	0.00201271143306612\\
19.81	0.00201271142346868\\
19.82	0.00201271141386392\\
19.83	0.00201271140425185\\
19.84	0.00201271139463246\\
19.85	0.00201271138500575\\
19.86	0.00201271137537173\\
19.87	0.00201271136573037\\
19.88	0.00201271135608169\\
19.89	0.00201271134642569\\
19.9	0.00201271133676235\\
19.91	0.00201271132709167\\
19.92	0.00201271131741366\\
19.93	0.00201271130772831\\
19.94	0.00201271129803562\\
19.95	0.00201271128833558\\
19.96	0.00201271127862819\\
19.97	0.00201271126891345\\
19.98	0.00201271125919137\\
19.99	0.00201271124946193\\
20	0.00201271123972513\\
20.01	0.00201271122998097\\
20.02	0.00201271122022945\\
20.03	0.00201271121047056\\
20.04	0.00201271120070431\\
20.05	0.00201271119093069\\
20.06	0.0020127111811497\\
20.07	0.00201271117136134\\
20.08	0.0020127111615656\\
20.09	0.00201271115176249\\
20.1	0.00201271114195199\\
20.11	0.00201271113213411\\
20.12	0.00201271112230885\\
20.13	0.0020127111124762\\
20.14	0.00201271110263617\\
20.15	0.00201271109278874\\
20.16	0.00201271108293392\\
20.17	0.00201271107307171\\
20.18	0.0020127110632021\\
20.19	0.00201271105332509\\
20.2	0.00201271104344069\\
20.21	0.00201271103354888\\
20.22	0.00201271102364967\\
20.23	0.00201271101374305\\
20.24	0.00201271100382902\\
20.25	0.00201271099390759\\
20.26	0.00201271098397874\\
20.27	0.00201271097404248\\
20.28	0.00201271096409881\\
20.29	0.00201271095414772\\
20.3	0.00201271094418921\\
20.31	0.00201271093422328\\
20.32	0.00201271092424994\\
20.33	0.00201271091426917\\
20.34	0.00201271090428097\\
20.35	0.00201271089428535\\
20.36	0.0020127108842823\\
20.37	0.00201271087427183\\
20.38	0.00201271086425392\\
20.39	0.00201271085422859\\
20.4	0.00201271084419582\\
20.41	0.00201271083415561\\
20.42	0.00201271082410798\\
20.43	0.0020127108140529\\
20.44	0.00201271080399039\\
20.45	0.00201271079392044\\
20.46	0.00201271078384304\\
20.47	0.00201271077375821\\
20.48	0.00201271076366593\\
20.49	0.00201271075356622\\
20.5	0.00201271074345905\\
20.51	0.00201271073334444\\
20.52	0.00201271072322239\\
20.53	0.00201271071309289\\
20.54	0.00201271070295594\\
20.55	0.00201271069281154\\
20.56	0.00201271068265969\\
20.57	0.00201271067250039\\
20.58	0.00201271066233364\\
20.59	0.00201271065215944\\
20.6	0.00201271064197778\\
20.61	0.00201271063178867\\
20.62	0.00201271062159211\\
20.63	0.0020127106113881\\
20.64	0.00201271060117662\\
20.65	0.0020127105909577\\
20.66	0.00201271058073132\\
20.67	0.00201271057049748\\
20.68	0.00201271056025618\\
20.69	0.00201271055000743\\
20.7	0.00201271053975122\\
20.71	0.00201271052948756\\
20.72	0.00201271051921643\\
20.73	0.00201271050893785\\
20.74	0.00201271049865182\\
20.75	0.00201271048835832\\
20.76	0.00201271047805737\\
20.77	0.00201271046774896\\
20.78	0.00201271045743309\\
20.79	0.00201271044710976\\
20.8	0.00201271043677898\\
20.81	0.00201271042644074\\
20.82	0.00201271041609504\\
20.83	0.00201271040574189\\
20.84	0.00201271039538128\\
20.85	0.00201271038501321\\
20.86	0.00201271037463769\\
20.87	0.00201271036425472\\
20.88	0.00201271035386429\\
20.89	0.0020127103434664\\
20.9	0.00201271033306107\\
20.91	0.00201271032264828\\
20.92	0.00201271031222804\\
20.93	0.00201271030180035\\
20.94	0.00201271029136521\\
20.95	0.00201271028092262\\
20.96	0.00201271027047258\\
20.97	0.0020127102600151\\
20.98	0.00201271024955017\\
20.99	0.0020127102390778\\
21	0.00201271022859798\\
21.01	0.00201271021811072\\
21.02	0.00201271020761602\\
21.03	0.00201271019711388\\
21.04	0.0020127101866043\\
21.05	0.00201271017608728\\
21.06	0.00201271016556283\\
21.07	0.00201271015503094\\
21.08	0.00201271014449162\\
21.09	0.00201271013394488\\
21.1	0.0020127101233907\\
21.11	0.00201271011282909\\
21.12	0.00201271010226006\\
21.13	0.00201271009168361\\
21.14	0.00201271008109974\\
21.15	0.00201271007050845\\
21.16	0.00201271005990974\\
21.17	0.00201271004930361\\
21.18	0.00201271003869007\\
21.19	0.00201271002806912\\
21.2	0.00201271001744077\\
21.21	0.00201271000680501\\
21.22	0.00201270999616184\\
21.23	0.00201270998551128\\
21.24	0.00201270997485331\\
21.25	0.00201270996418795\\
21.26	0.0020127099535152\\
21.27	0.00201270994283506\\
21.28	0.00201270993214753\\
21.29	0.00201270992145262\\
21.3	0.00201270991075033\\
21.31	0.00201270990004066\\
21.32	0.00201270988932362\\
21.33	0.0020127098785992\\
21.34	0.00201270986786742\\
21.35	0.00201270985712827\\
21.36	0.00201270984638176\\
21.37	0.00201270983562789\\
21.38	0.00201270982486667\\
21.39	0.0020127098140981\\
21.4	0.00201270980332219\\
21.41	0.00201270979253893\\
21.42	0.00201270978174833\\
21.43	0.0020127097709504\\
21.44	0.00201270976014513\\
21.45	0.00201270974933254\\
21.46	0.00201270973851263\\
21.47	0.0020127097276854\\
21.48	0.00201270971685085\\
21.49	0.002012709706009\\
21.5	0.00201270969515984\\
21.51	0.00201270968430339\\
21.52	0.00201270967343963\\
21.53	0.00201270966256859\\
21.54	0.00201270965169026\\
21.55	0.00201270964080465\\
21.56	0.00201270962991177\\
21.57	0.00201270961901162\\
21.58	0.0020127096081042\\
21.59	0.00201270959718952\\
21.6	0.00201270958626758\\
21.61	0.0020127095753384\\
21.62	0.00201270956440198\\
21.63	0.00201270955345831\\
21.64	0.00201270954250742\\
21.65	0.0020127095315493\\
21.66	0.00201270952058396\\
21.67	0.0020127095096114\\
21.68	0.00201270949863164\\
21.69	0.00201270948764468\\
21.7	0.00201270947665052\\
21.71	0.00201270946564917\\
21.72	0.00201270945464064\\
21.73	0.00201270944362494\\
21.74	0.00201270943260206\\
21.75	0.00201270942157203\\
21.76	0.00201270941053483\\
21.77	0.0020127093994905\\
21.78	0.00201270938843902\\
21.79	0.0020127093773804\\
21.8	0.00201270936631466\\
21.81	0.0020127093552418\\
21.82	0.00201270934416183\\
21.83	0.00201270933307475\\
21.84	0.00201270932198058\\
21.85	0.00201270931087932\\
21.86	0.00201270929977098\\
21.87	0.00201270928865557\\
21.88	0.00201270927753309\\
21.89	0.00201270926640356\\
21.9	0.00201270925526697\\
21.91	0.00201270924412335\\
21.92	0.0020127092329727\\
21.93	0.00201270922181503\\
21.94	0.00201270921065034\\
21.95	0.00201270919947865\\
21.96	0.00201270918829997\\
21.97	0.00201270917711429\\
21.98	0.00201270916592164\\
21.99	0.00201270915472203\\
22	0.00201270914351545\\
22.01	0.00201270913230193\\
22.02	0.00201270912108147\\
22.03	0.00201270910985408\\
22.04	0.00201270909861977\\
22.05	0.00201270908737855\\
22.06	0.00201270907613043\\
22.07	0.00201270906487542\\
22.08	0.00201270905361354\\
22.09	0.00201270904234479\\
22.1	0.00201270903106918\\
22.11	0.00201270901978672\\
22.12	0.00201270900849743\\
22.13	0.00201270899720131\\
22.14	0.00201270898589839\\
22.15	0.00201270897458866\\
22.16	0.00201270896327213\\
22.17	0.00201270895194883\\
22.18	0.00201270894061877\\
22.19	0.00201270892928194\\
22.2	0.00201270891793837\\
22.21	0.00201270890658807\\
22.22	0.00201270889523105\\
22.23	0.00201270888386733\\
22.24	0.0020127088724969\\
22.25	0.0020127088611198\\
22.26	0.00201270884973602\\
22.27	0.00201270883834559\\
22.28	0.00201270882694851\\
22.29	0.0020127088155448\\
22.3	0.00201270880413447\\
22.31	0.00201270879271754\\
22.32	0.00201270878129401\\
22.33	0.00201270876986391\\
22.34	0.00201270875842724\\
22.35	0.00201270874698402\\
22.36	0.00201270873553427\\
22.37	0.00201270872407799\\
22.38	0.0020127087126152\\
22.39	0.00201270870114592\\
22.4	0.00201270868967015\\
22.41	0.00201270867818793\\
22.42	0.00201270866669925\\
22.43	0.00201270865520413\\
22.44	0.0020127086437026\\
22.45	0.00201270863219465\\
22.46	0.00201270862068032\\
22.47	0.00201270860915962\\
22.48	0.00201270859763255\\
22.49	0.00201270858609913\\
22.5	0.0020127085745594\\
22.51	0.00201270856301334\\
22.52	0.00201270855146099\\
22.53	0.00201270853990236\\
22.54	0.00201270852833747\\
22.55	0.00201270851676633\\
22.56	0.00201270850518895\\
22.57	0.00201270849360537\\
22.58	0.00201270848201558\\
22.59	0.00201270847041961\\
22.6	0.00201270845881749\\
22.61	0.00201270844720921\\
22.62	0.00201270843559481\\
22.63	0.00201270842397429\\
22.64	0.00201270841234768\\
22.65	0.00201270840071499\\
22.66	0.00201270838907625\\
22.67	0.00201270837743146\\
22.68	0.00201270836578066\\
22.69	0.00201270835412385\\
22.7	0.00201270834246105\\
22.71	0.00201270833079228\\
22.72	0.00201270831911757\\
22.73	0.00201270830743693\\
22.74	0.00201270829575038\\
22.75	0.00201270828405794\\
22.76	0.00201270827235962\\
22.77	0.00201270826065545\\
22.78	0.00201270824894545\\
22.79	0.00201270823722963\\
22.8	0.00201270822550802\\
22.81	0.00201270821378063\\
22.82	0.00201270820204749\\
22.83	0.00201270819030862\\
22.84	0.00201270817856403\\
22.85	0.00201270816681374\\
22.86	0.00201270815505778\\
22.87	0.00201270814329617\\
22.88	0.00201270813152893\\
22.89	0.00201270811975607\\
22.9	0.00201270810797762\\
22.91	0.00201270809619361\\
22.92	0.00201270808440404\\
22.93	0.00201270807260895\\
22.94	0.00201270806080834\\
22.95	0.00201270804900226\\
22.96	0.00201270803719071\\
22.97	0.00201270802537371\\
22.98	0.0020127080135513\\
22.99	0.00201270800172349\\
23	0.0020127079898903\\
23.01	0.00201270797805176\\
23.02	0.00201270796620789\\
23.03	0.0020127079543587\\
23.04	0.00201270794250423\\
23.05	0.00201270793064449\\
23.06	0.00201270791877952\\
23.07	0.00201270790690932\\
23.08	0.00201270789503392\\
23.09	0.00201270788315334\\
23.1	0.00201270787126762\\
23.11	0.00201270785937677\\
23.12	0.00201270784748081\\
23.13	0.00201270783557977\\
23.14	0.00201270782367366\\
23.15	0.00201270781176252\\
23.16	0.00201270779984637\\
23.17	0.00201270778792523\\
23.18	0.00201270777599913\\
23.19	0.00201270776406808\\
23.2	0.00201270775213211\\
23.21	0.00201270774019124\\
23.22	0.0020127077282455\\
23.23	0.00201270771629492\\
23.24	0.00201270770433951\\
23.25	0.0020127076923793\\
23.26	0.00201270768041432\\
23.27	0.00201270766844458\\
23.28	0.00201270765647011\\
23.29	0.00201270764449094\\
23.3	0.00201270763250708\\
23.31	0.00201270762051857\\
23.32	0.00201270760852543\\
23.33	0.00201270759652768\\
23.34	0.00201270758452535\\
23.35	0.00201270757251845\\
23.36	0.00201270756050702\\
23.37	0.00201270754849107\\
23.38	0.00201270753647064\\
23.39	0.00201270752444575\\
23.4	0.00201270751241641\\
23.41	0.00201270750038266\\
23.42	0.00201270748834452\\
23.43	0.00201270747630201\\
23.44	0.00201270746425516\\
23.45	0.00201270745220398\\
23.46	0.00201270744014852\\
23.47	0.00201270742808878\\
23.48	0.00201270741602479\\
23.49	0.00201270740395658\\
23.5	0.00201270739188418\\
23.51	0.00201270737980759\\
23.52	0.00201270736772686\\
23.53	0.00201270735564199\\
23.54	0.00201270734355302\\
23.55	0.00201270733145997\\
23.56	0.00201270731936287\\
23.57	0.00201270730726172\\
23.58	0.00201270729515657\\
23.59	0.00201270728304743\\
23.6	0.00201270727093433\\
23.61	0.00201270725881728\\
23.62	0.00201270724669631\\
23.63	0.00201270723457145\\
23.64	0.00201270722244272\\
23.65	0.00201270721031013\\
23.66	0.00201270719817372\\
23.67	0.00201270718603349\\
23.68	0.00201270717388949\\
23.69	0.00201270716174172\\
23.7	0.00201270714959022\\
23.71	0.00201270713743499\\
23.72	0.00201270712527607\\
23.73	0.00201270711311347\\
23.74	0.00201270710094721\\
23.75	0.00201270708877733\\
23.76	0.00201270707660383\\
23.77	0.00201270706442673\\
23.78	0.00201270705224607\\
23.79	0.00201270704006185\\
23.8	0.0020127070278741\\
23.81	0.00201270701568284\\
23.82	0.00201270700348808\\
23.83	0.00201270699128985\\
23.84	0.00201270697908816\\
23.85	0.00201270696688304\\
23.86	0.0020127069546745\\
23.87	0.00201270694246256\\
23.88	0.00201270693024724\\
23.89	0.00201270691802855\\
23.9	0.00201270690580651\\
23.91	0.00201270689358115\\
23.92	0.00201270688135247\\
23.93	0.00201270686912049\\
23.94	0.00201270685688524\\
23.95	0.00201270684464671\\
23.96	0.00201270683240494\\
23.97	0.00201270682015993\\
23.98	0.0020127068079117\\
23.99	0.00201270679566026\\
24	0.00201270678340564\\
24.01	0.00201270677114783\\
24.02	0.00201270675888686\\
24.03	0.00201270674662273\\
24.04	0.00201270673435546\\
24.05	0.00201270672208507\\
24.06	0.00201270670981155\\
24.07	0.00201270669753494\\
24.08	0.00201270668525522\\
24.09	0.00201270667297242\\
24.1	0.00201270666068655\\
24.11	0.0020127066483976\\
24.12	0.00201270663610561\\
24.13	0.00201270662381056\\
24.14	0.00201270661151247\\
24.15	0.00201270659921135\\
24.16	0.0020127065869072\\
24.17	0.00201270657460003\\
24.18	0.00201270656228984\\
24.19	0.00201270654997665\\
24.2	0.00201270653766045\\
24.21	0.00201270652534125\\
24.22	0.00201270651301906\\
24.23	0.00201270650069387\\
24.24	0.0020127064883657\\
24.25	0.00201270647603453\\
24.26	0.00201270646370038\\
24.27	0.00201270645136324\\
24.28	0.00201270643902312\\
24.29	0.00201270642668001\\
24.3	0.00201270641433392\\
24.31	0.00201270640198483\\
24.32	0.00201270638963276\\
24.33	0.0020127063772777\\
24.34	0.00201270636491963\\
24.35	0.00201270635255857\\
24.36	0.00201270634019451\\
24.37	0.00201270632782743\\
24.38	0.00201270631545734\\
24.39	0.00201270630308423\\
24.4	0.00201270629070809\\
24.41	0.00201270627832891\\
24.42	0.00201270626594669\\
24.43	0.00201270625356141\\
24.44	0.00201270624117307\\
24.45	0.00201270622878165\\
24.46	0.00201270621638715\\
24.47	0.00201270620398956\\
24.48	0.00201270619158885\\
24.49	0.00201270617918501\\
24.5	0.00201270616677804\\
24.51	0.00201270615436792\\
24.52	0.00201270614195464\\
24.53	0.00201270612953817\\
24.54	0.0020127061171185\\
24.55	0.00201270610469561\\
24.56	0.00201270609226949\\
24.57	0.00201270607984012\\
24.58	0.00201270606740747\\
24.59	0.00201270605497153\\
24.6	0.00201270604253228\\
24.61	0.0020127060300897\\
24.62	0.00201270601764375\\
24.63	0.00201270600519443\\
24.64	0.00201270599274171\\
24.65	0.00201270598028556\\
24.66	0.00201270596782596\\
24.67	0.00201270595536288\\
24.68	0.0020127059428963\\
24.69	0.00201270593042619\\
24.7	0.00201270591795253\\
24.71	0.00201270590547528\\
24.72	0.00201270589299442\\
24.73	0.00201270588050992\\
24.74	0.00201270586802176\\
24.75	0.00201270585552989\\
24.76	0.00201270584303429\\
24.77	0.00201270583053493\\
24.78	0.00201270581803178\\
24.79	0.00201270580552481\\
24.8	0.00201270579301398\\
24.81	0.00201270578049926\\
24.82	0.00201270576798061\\
24.83	0.00201270575545801\\
24.84	0.00201270574293143\\
24.85	0.00201270573040081\\
24.86	0.00201270571786615\\
24.87	0.00201270570532738\\
24.88	0.00201270569278449\\
24.89	0.00201270568023744\\
24.9	0.00201270566768618\\
24.91	0.0020127056551307\\
24.92	0.00201270564257094\\
24.93	0.00201270563000688\\
24.94	0.00201270561743848\\
24.95	0.0020127056048657\\
24.96	0.00201270559228851\\
24.97	0.00201270557970686\\
24.98	0.00201270556712074\\
24.99	0.00201270555453009\\
25	0.00201270554193489\\
25.01	0.00201270552933509\\
25.02	0.00201270551673066\\
25.03	0.00201270550412157\\
25.04	0.00201270549150777\\
25.05	0.00201270547888924\\
25.06	0.00201270546626593\\
25.07	0.00201270545363781\\
25.08	0.00201270544100484\\
25.09	0.00201270542836699\\
25.1	0.00201270541572422\\
25.11	0.0020127054030765\\
25.12	0.00201270539042379\\
25.13	0.00201270537776605\\
25.14	0.00201270536510324\\
25.15	0.00201270535243534\\
25.16	0.00201270533976231\\
25.17	0.0020127053270841\\
25.18	0.00201270531440069\\
25.19	0.00201270530171205\\
25.2	0.00201270528901813\\
25.21	0.00201270527631889\\
25.22	0.00201270526361432\\
25.23	0.00201270525090437\\
25.24	0.00201270523818901\\
25.25	0.00201270522546819\\
25.26	0.0020127052127419\\
25.27	0.0020127052000101\\
25.28	0.00201270518727275\\
25.29	0.00201270517452982\\
25.3	0.00201270516178128\\
25.31	0.00201270514902709\\
25.32	0.00201270513626722\\
25.33	0.00201270512350165\\
25.34	0.00201270511073033\\
25.35	0.00201270509795325\\
25.36	0.00201270508517036\\
25.37	0.00201270507238164\\
25.38	0.00201270505958705\\
25.39	0.00201270504678657\\
25.4	0.00201270503398017\\
25.41	0.00201270502116782\\
25.42	0.00201270500834948\\
25.43	0.00201270499552514\\
25.44	0.00201270498269476\\
25.45	0.00201270496985831\\
25.46	0.00201270495701578\\
25.47	0.00201270494416713\\
25.48	0.00201270493131233\\
25.49	0.00201270491845136\\
25.5	0.00201270490558421\\
25.51	0.00201270489271083\\
25.52	0.0020127048798312\\
25.53	0.00201270486694531\\
25.54	0.00201270485405314\\
25.55	0.00201270484115464\\
25.56	0.00201270482824982\\
25.57	0.00201270481533864\\
25.58	0.00201270480242107\\
25.59	0.00201270478949712\\
25.6	0.00201270477656674\\
25.61	0.00201270476362993\\
25.62	0.00201270475068666\\
25.63	0.00201270473773691\\
25.64	0.00201270472478068\\
25.65	0.00201270471181793\\
25.66	0.00201270469884866\\
25.67	0.00201270468587284\\
25.68	0.00201270467289047\\
25.69	0.00201270465990152\\
25.7	0.00201270464690599\\
25.71	0.00201270463390386\\
25.72	0.00201270462089511\\
25.73	0.00201270460787973\\
25.74	0.00201270459485772\\
25.75	0.00201270458182905\\
25.76	0.00201270456879372\\
25.77	0.00201270455575172\\
25.78	0.00201270454270304\\
25.79	0.00201270452964766\\
25.8	0.00201270451658558\\
25.81	0.00201270450351679\\
25.82	0.00201270449044129\\
25.83	0.00201270447735906\\
25.84	0.00201270446427009\\
25.85	0.00201270445117438\\
25.86	0.00201270443807193\\
25.87	0.00201270442496273\\
25.88	0.00201270441184677\\
25.89	0.00201270439872404\\
25.9	0.00201270438559455\\
25.91	0.00201270437245828\\
25.92	0.00201270435931524\\
25.93	0.00201270434616541\\
25.94	0.0020127043330088\\
25.95	0.0020127043198454\\
25.96	0.00201270430667521\\
25.97	0.00201270429349822\\
25.98	0.00201270428031442\\
25.99	0.00201270426712383\\
26	0.00201270425392642\\
26.01	0.0020127042407222\\
26.02	0.00201270422751117\\
26.03	0.00201270421429332\\
26.04	0.00201270420106865\\
26.05	0.00201270418783715\\
26.06	0.00201270417459882\\
26.07	0.00201270416135365\\
26.08	0.00201270414810165\\
26.09	0.00201270413484281\\
26.1	0.00201270412157712\\
26.11	0.00201270410830458\\
26.12	0.00201270409502519\\
26.13	0.00201270408173894\\
26.14	0.00201270406844584\\
26.15	0.00201270405514587\\
26.16	0.00201270404183903\\
26.17	0.00201270402852533\\
26.18	0.00201270401520475\\
26.19	0.00201270400187729\\
26.2	0.00201270398854294\\
26.21	0.00201270397520172\\
26.22	0.0020127039618536\\
26.23	0.00201270394849859\\
26.24	0.00201270393513668\\
26.25	0.00201270392176788\\
26.26	0.00201270390839216\\
26.27	0.00201270389500955\\
26.28	0.00201270388162001\\
26.29	0.00201270386822357\\
26.3	0.0020127038548202\\
26.31	0.00201270384140991\\
26.32	0.00201270382799269\\
26.33	0.00201270381456854\\
26.34	0.00201270380113746\\
26.35	0.00201270378769944\\
26.36	0.00201270377425448\\
26.37	0.00201270376080257\\
26.38	0.00201270374734372\\
26.39	0.00201270373387791\\
26.4	0.00201270372040514\\
26.41	0.00201270370692541\\
26.42	0.00201270369343872\\
26.43	0.00201270367994506\\
26.44	0.00201270366644443\\
26.45	0.00201270365293682\\
26.46	0.00201270363942224\\
26.47	0.00201270362590066\\
26.48	0.00201270361237211\\
26.49	0.00201270359883656\\
26.5	0.00201270358529401\\
26.51	0.00201270357174447\\
26.52	0.00201270355818793\\
26.53	0.00201270354462438\\
26.54	0.00201270353105382\\
26.55	0.00201270351747624\\
26.56	0.00201270350389165\\
26.57	0.00201270349030004\\
26.58	0.0020127034767014\\
26.59	0.00201270346309573\\
26.6	0.00201270344948303\\
26.61	0.00201270343586329\\
26.62	0.00201270342223652\\
26.63	0.00201270340860269\\
26.64	0.00201270339496182\\
26.65	0.0020127033813139\\
26.66	0.00201270336765892\\
26.67	0.00201270335399688\\
26.68	0.00201270334032778\\
26.69	0.00201270332665161\\
26.7	0.00201270331296837\\
26.71	0.00201270329927806\\
26.72	0.00201270328558066\\
26.73	0.00201270327187619\\
26.74	0.00201270325816462\\
26.75	0.00201270324444597\\
26.76	0.00201270323072022\\
26.77	0.00201270321698737\\
26.78	0.00201270320324743\\
26.79	0.00201270318950037\\
26.8	0.00201270317574621\\
26.81	0.00201270316198492\\
26.82	0.00201270314821653\\
26.83	0.00201270313444101\\
26.84	0.00201270312065837\\
26.85	0.00201270310686859\\
26.86	0.00201270309307169\\
26.87	0.00201270307926764\\
26.88	0.00201270306545646\\
26.89	0.00201270305163813\\
26.9	0.00201270303781265\\
26.91	0.00201270302398002\\
26.92	0.00201270301014023\\
26.93	0.00201270299629329\\
26.94	0.00201270298243917\\
26.95	0.00201270296857789\\
26.96	0.00201270295470944\\
26.97	0.00201270294083381\\
26.98	0.002012702926951\\
26.99	0.002012702913061\\
27	0.00201270289916382\\
27.01	0.00201270288525945\\
27.02	0.00201270287134788\\
27.03	0.00201270285742911\\
27.04	0.00201270284350313\\
27.05	0.00201270282956995\\
27.06	0.00201270281562956\\
27.07	0.00201270280168195\\
27.08	0.00201270278772712\\
27.09	0.00201270277376507\\
27.1	0.00201270275979579\\
27.11	0.00201270274581927\\
27.12	0.00201270273183553\\
27.13	0.00201270271784454\\
27.14	0.00201270270384631\\
27.15	0.00201270268984083\\
27.16	0.0020127026758281\\
27.17	0.00201270266180812\\
27.18	0.00201270264778087\\
27.19	0.00201270263374636\\
27.2	0.00201270261970459\\
27.21	0.00201270260565554\\
27.22	0.00201270259159922\\
27.23	0.00201270257753562\\
27.24	0.00201270256346474\\
27.25	0.00201270254938657\\
27.26	0.0020127025353011\\
27.27	0.00201270252120834\\
27.28	0.00201270250710829\\
27.29	0.00201270249300092\\
27.3	0.00201270247888626\\
27.31	0.00201270246476428\\
27.32	0.00201270245063498\\
27.33	0.00201270243649836\\
27.34	0.00201270242235442\\
27.35	0.00201270240820316\\
27.36	0.00201270239404456\\
27.37	0.00201270237987862\\
27.38	0.00201270236570535\\
27.39	0.00201270235152473\\
27.4	0.00201270233733677\\
27.41	0.00201270232314145\\
27.42	0.00201270230893877\\
27.43	0.00201270229472874\\
27.44	0.00201270228051134\\
27.45	0.00201270226628658\\
27.46	0.00201270225205444\\
27.47	0.00201270223781492\\
27.48	0.00201270222356803\\
27.49	0.00201270220931375\\
27.5	0.00201270219505208\\
27.51	0.00201270218078302\\
27.52	0.00201270216650656\\
27.53	0.0020127021522227\\
27.54	0.00201270213793144\\
27.55	0.00201270212363277\\
27.56	0.00201270210932668\\
27.57	0.00201270209501318\\
27.58	0.00201270208069226\\
27.59	0.00201270206636391\\
27.6	0.00201270205202813\\
27.61	0.00201270203768492\\
27.62	0.00201270202333426\\
27.63	0.00201270200897617\\
27.64	0.00201270199461063\\
27.65	0.00201270198023765\\
27.66	0.0020127019658572\\
27.67	0.0020127019514693\\
27.68	0.00201270193707394\\
27.69	0.00201270192267111\\
27.7	0.00201270190826081\\
27.71	0.00201270189384303\\
27.72	0.00201270187941777\\
27.73	0.00201270186498504\\
27.74	0.00201270185054481\\
27.75	0.00201270183609709\\
27.76	0.00201270182164188\\
27.77	0.00201270180717916\\
27.78	0.00201270179270894\\
27.79	0.00201270177823122\\
27.8	0.00201270176374598\\
27.81	0.00201270174925322\\
27.82	0.00201270173475295\\
27.83	0.00201270172024515\\
27.84	0.00201270170572982\\
27.85	0.00201270169120695\\
27.86	0.00201270167667655\\
27.87	0.00201270166213861\\
27.88	0.00201270164759312\\
27.89	0.00201270163304008\\
27.9	0.00201270161847949\\
27.91	0.00201270160391133\\
27.92	0.00201270158933562\\
27.93	0.00201270157475233\\
27.94	0.00201270156016148\\
27.95	0.00201270154556305\\
27.96	0.00201270153095704\\
27.97	0.00201270151634345\\
27.98	0.00201270150172227\\
27.99	0.00201270148709349\\
28	0.00201270147245713\\
28.01	0.00201270145781315\\
28.02	0.00201270144316158\\
28.03	0.00201270142850239\\
28.04	0.00201270141383559\\
28.05	0.00201270139916118\\
28.06	0.00201270138447914\\
28.07	0.00201270136978947\\
28.08	0.00201270135509217\\
28.09	0.00201270134038724\\
28.1	0.00201270132567467\\
28.11	0.00201270131095446\\
28.12	0.0020127012962266\\
28.13	0.00201270128149108\\
28.14	0.00201270126674791\\
28.15	0.00201270125199708\\
28.16	0.00201270123723859\\
28.17	0.00201270122247242\\
28.18	0.00201270120769858\\
28.19	0.00201270119291707\\
28.2	0.00201270117812787\\
28.21	0.00201270116333098\\
28.22	0.00201270114852641\\
28.23	0.00201270113371414\\
28.24	0.00201270111889417\\
28.25	0.00201270110406649\\
28.26	0.00201270108923111\\
28.27	0.00201270107438802\\
28.28	0.00201270105953721\\
28.29	0.00201270104467867\\
28.3	0.00201270102981242\\
28.31	0.00201270101493843\\
28.32	0.00201270100005671\\
28.33	0.00201270098516725\\
28.34	0.00201270097027005\\
28.35	0.0020127009553651\\
28.36	0.0020127009404524\\
28.37	0.00201270092553194\\
28.38	0.00201270091060373\\
28.39	0.00201270089566774\\
28.4	0.00201270088072399\\
28.41	0.00201270086577247\\
28.42	0.00201270085081317\\
28.43	0.00201270083584609\\
28.44	0.00201270082087122\\
28.45	0.00201270080588856\\
28.46	0.0020127007908981\\
28.47	0.00201270077589985\\
28.48	0.00201270076089379\\
28.49	0.00201270074587992\\
28.5	0.00201270073085824\\
28.51	0.00201270071582874\\
28.52	0.00201270070079142\\
28.53	0.00201270068574627\\
28.54	0.0020127006706933\\
28.55	0.00201270065563249\\
28.56	0.00201270064056384\\
28.57	0.00201270062548735\\
28.58	0.002012700610403\\
28.59	0.00201270059531081\\
28.6	0.00201270058021076\\
28.61	0.00201270056510284\\
28.62	0.00201270054998706\\
28.63	0.00201270053486341\\
28.64	0.00201270051973189\\
28.65	0.00201270050459249\\
28.66	0.0020127004894452\\
28.67	0.00201270047429002\\
28.68	0.00201270045912695\\
28.69	0.00201270044395599\\
28.7	0.00201270042877712\\
28.71	0.00201270041359035\\
28.72	0.00201270039839566\\
28.73	0.00201270038319306\\
28.74	0.00201270036798254\\
28.75	0.0020127003527641\\
28.76	0.00201270033753773\\
28.77	0.00201270032230342\\
28.78	0.00201270030706118\\
28.79	0.002012700291811\\
28.8	0.00201270027655287\\
28.81	0.00201270026128678\\
28.82	0.00201270024601275\\
28.83	0.00201270023073075\\
28.84	0.00201270021544079\\
28.85	0.00201270020014285\\
28.86	0.00201270018483695\\
28.87	0.00201270016952306\\
28.88	0.0020127001542012\\
28.89	0.00201270013887134\\
28.9	0.0020127001235335\\
28.91	0.00201270010818766\\
28.92	0.00201270009283381\\
28.93	0.00201270007747197\\
28.94	0.00201270006210211\\
28.95	0.00201270004672423\\
28.96	0.00201270003133834\\
28.97	0.00201270001594442\\
28.98	0.00201270000054248\\
28.99	0.0020126999851325\\
29	0.00201269996971449\\
29.01	0.00201269995428843\\
29.02	0.00201269993885432\\
29.03	0.00201269992341217\\
29.04	0.00201269990796196\\
29.05	0.00201269989250369\\
29.06	0.00201269987703735\\
29.07	0.00201269986156294\\
29.08	0.00201269984608046\\
29.09	0.0020126998305899\\
29.1	0.00201269981509126\\
29.11	0.00201269979958453\\
29.12	0.0020126997840697\\
29.13	0.00201269976854678\\
29.14	0.00201269975301576\\
29.15	0.00201269973747663\\
29.16	0.00201269972192938\\
29.17	0.00201269970637403\\
29.18	0.00201269969081055\\
29.19	0.00201269967523894\\
29.2	0.00201269965965921\\
29.21	0.00201269964407134\\
29.22	0.00201269962847534\\
29.23	0.00201269961287118\\
29.24	0.00201269959725888\\
29.25	0.00201269958163843\\
29.26	0.00201269956600982\\
29.27	0.00201269955037305\\
29.28	0.00201269953472811\\
29.29	0.00201269951907499\\
29.3	0.0020126995034137\\
29.31	0.00201269948774424\\
29.32	0.00201269947206658\\
29.33	0.00201269945638073\\
29.34	0.00201269944068669\\
29.35	0.00201269942498444\\
29.36	0.002012699409274\\
29.37	0.00201269939355534\\
29.38	0.00201269937782847\\
29.39	0.00201269936209338\\
29.4	0.00201269934635006\\
29.41	0.00201269933059852\\
29.42	0.00201269931483874\\
29.43	0.00201269929907073\\
29.44	0.00201269928329447\\
29.45	0.00201269926750997\\
29.46	0.00201269925171721\\
29.47	0.0020126992359162\\
29.48	0.00201269922010692\\
29.49	0.00201269920428938\\
29.5	0.00201269918846357\\
29.51	0.00201269917262948\\
29.52	0.00201269915678711\\
29.53	0.00201269914093645\\
29.54	0.00201269912507751\\
29.55	0.00201269910921027\\
29.56	0.00201269909333473\\
29.57	0.00201269907745088\\
29.58	0.00201269906155873\\
29.59	0.00201269904565826\\
29.6	0.00201269902974947\\
29.61	0.00201269901383236\\
29.62	0.00201269899790692\\
29.63	0.00201269898197314\\
29.64	0.00201269896603103\\
29.65	0.00201269895008058\\
29.66	0.00201269893412177\\
29.67	0.00201269891815461\\
29.68	0.0020126989021791\\
29.69	0.00201269888619522\\
29.7	0.00201269887020298\\
29.71	0.00201269885420236\\
29.72	0.00201269883819337\\
29.73	0.00201269882217599\\
29.74	0.00201269880615023\\
29.75	0.00201269879011608\\
29.76	0.00201269877407353\\
29.77	0.00201269875802258\\
29.78	0.00201269874196322\\
29.79	0.00201269872589545\\
29.8	0.00201269870981927\\
29.81	0.00201269869373466\\
29.82	0.00201269867764163\\
29.83	0.00201269866154017\\
29.84	0.00201269864543027\\
29.85	0.00201269862931193\\
29.86	0.00201269861318515\\
29.87	0.00201269859704992\\
29.88	0.00201269858090623\\
29.89	0.00201269856475408\\
29.9	0.00201269854859347\\
29.91	0.00201269853242438\\
29.92	0.00201269851624683\\
29.93	0.00201269850006079\\
29.94	0.00201269848386627\\
29.95	0.00201269846766325\\
29.96	0.00201269845145175\\
29.97	0.00201269843523174\\
29.98	0.00201269841900323\\
29.99	0.00201269840276621\\
30	0.00201269838652068\\
30.01	0.00201269837026662\\
30.02	0.00201269835400405\\
30.03	0.00201269833773294\\
30.04	0.0020126983214533\\
30.05	0.00201269830516512\\
30.06	0.00201269828886839\\
30.07	0.00201269827256312\\
30.08	0.00201269825624929\\
30.09	0.0020126982399269\\
30.1	0.00201269822359595\\
30.11	0.00201269820725643\\
30.12	0.00201269819090834\\
30.13	0.00201269817455166\\
30.14	0.00201269815818641\\
30.15	0.00201269814181256\\
30.16	0.00201269812543012\\
30.17	0.00201269810903908\\
30.18	0.00201269809263943\\
30.19	0.00201269807623118\\
30.2	0.00201269805981431\\
30.21	0.00201269804338882\\
30.22	0.00201269802695471\\
30.23	0.00201269801051197\\
30.24	0.00201269799406059\\
30.25	0.00201269797760057\\
30.26	0.00201269796113191\\
30.27	0.0020126979446546\\
30.28	0.00201269792816864\\
30.29	0.00201269791167401\\
30.3	0.00201269789517072\\
30.31	0.00201269787865876\\
30.32	0.00201269786213813\\
30.33	0.00201269784560881\\
30.34	0.00201269782907081\\
30.35	0.00201269781252412\\
30.36	0.00201269779596874\\
30.37	0.00201269777940465\\
30.38	0.00201269776283186\\
30.39	0.00201269774625036\\
30.4	0.00201269772966014\\
30.41	0.0020126977130612\\
30.42	0.00201269769645353\\
30.43	0.00201269767983713\\
30.44	0.002012697663212\\
30.45	0.00201269764657812\\
30.46	0.0020126976299355\\
30.47	0.00201269761328413\\
30.48	0.002012697596624\\
30.49	0.0020126975799551\\
30.5	0.00201269756327744\\
30.51	0.00201269754659101\\
30.52	0.0020126975298958\\
30.53	0.00201269751319181\\
30.54	0.00201269749647902\\
30.55	0.00201269747975745\\
30.56	0.00201269746302708\\
30.57	0.0020126974462879\\
30.58	0.00201269742953992\\
30.59	0.00201269741278312\\
30.6	0.0020126973960175\\
30.61	0.00201269737924306\\
30.62	0.00201269736245979\\
30.63	0.00201269734566768\\
30.64	0.00201269732886673\\
30.65	0.00201269731205694\\
30.66	0.0020126972952383\\
30.67	0.0020126972784108\\
30.68	0.00201269726157445\\
30.69	0.00201269724472922\\
30.7	0.00201269722787512\\
30.71	0.00201269721101215\\
30.72	0.0020126971941403\\
30.73	0.00201269717725956\\
30.74	0.00201269716036992\\
30.75	0.0020126971434714\\
30.76	0.00201269712656396\\
30.77	0.00201269710964762\\
30.78	0.00201269709272237\\
30.79	0.00201269707578819\\
30.8	0.00201269705884509\\
30.81	0.00201269704189307\\
30.82	0.00201269702493211\\
30.83	0.00201269700796221\\
30.84	0.00201269699098336\\
30.85	0.00201269697399556\\
30.86	0.00201269695699881\\
30.87	0.0020126969399931\\
30.88	0.00201269692297842\\
30.89	0.00201269690595477\\
30.9	0.00201269688892214\\
30.91	0.00201269687188053\\
30.92	0.00201269685482993\\
30.93	0.00201269683777034\\
30.94	0.00201269682070176\\
30.95	0.00201269680362416\\
30.96	0.00201269678653756\\
30.97	0.00201269676944195\\
30.98	0.00201269675233731\\
30.99	0.00201269673522365\\
31	0.00201269671810096\\
31.01	0.00201269670096924\\
31.02	0.00201269668382847\\
31.03	0.00201269666667865\\
31.04	0.00201269664951979\\
31.05	0.00201269663235186\\
31.06	0.00201269661517488\\
31.07	0.00201269659798883\\
31.08	0.0020126965807937\\
31.09	0.00201269656358949\\
31.1	0.0020126965463762\\
31.11	0.00201269652915382\\
31.12	0.00201269651192235\\
31.13	0.00201269649468178\\
31.14	0.0020126964774321\\
31.15	0.0020126964601733\\
31.16	0.00201269644290539\\
31.17	0.00201269642562836\\
31.18	0.00201269640834221\\
31.19	0.00201269639104691\\
31.2	0.00201269637374248\\
31.21	0.00201269635642891\\
31.22	0.00201269633910619\\
31.23	0.00201269632177431\\
31.24	0.00201269630443327\\
31.25	0.00201269628708307\\
31.26	0.00201269626972369\\
31.27	0.00201269625235514\\
31.28	0.0020126962349774\\
31.29	0.00201269621759048\\
31.3	0.00201269620019437\\
31.31	0.00201269618278905\\
31.32	0.00201269616537453\\
31.33	0.00201269614795081\\
31.34	0.00201269613051786\\
31.35	0.0020126961130757\\
31.36	0.00201269609562431\\
31.37	0.00201269607816369\\
31.38	0.00201269606069383\\
31.39	0.00201269604321473\\
31.4	0.00201269602572638\\
31.41	0.00201269600822878\\
31.42	0.00201269599072191\\
31.43	0.00201269597320579\\
31.44	0.00201269595568039\\
31.45	0.00201269593814571\\
31.46	0.00201269592060176\\
31.47	0.00201269590304852\\
31.48	0.00201269588548598\\
31.49	0.00201269586791415\\
31.5	0.00201269585033301\\
31.51	0.00201269583274256\\
31.52	0.0020126958151428\\
31.53	0.00201269579753372\\
31.54	0.00201269577991531\\
31.55	0.00201269576228757\\
31.56	0.0020126957446505\\
31.57	0.00201269572700408\\
31.58	0.00201269570934831\\
31.59	0.00201269569168318\\
31.6	0.0020126956740087\\
31.61	0.00201269565632485\\
31.62	0.00201269563863163\\
31.63	0.00201269562092904\\
31.64	0.00201269560321706\\
31.65	0.0020126955854957\\
31.66	0.00201269556776494\\
31.67	0.00201269555002478\\
31.68	0.00201269553227522\\
31.69	0.00201269551451624\\
31.7	0.00201269549674785\\
31.71	0.00201269547897004\\
31.72	0.0020126954611828\\
31.73	0.00201269544338613\\
31.74	0.00201269542558002\\
31.75	0.00201269540776446\\
31.76	0.00201269538993946\\
31.77	0.002012695372105\\
31.78	0.00201269535426108\\
31.79	0.00201269533640769\\
31.8	0.00201269531854483\\
31.81	0.00201269530067249\\
31.82	0.00201269528279066\\
31.83	0.00201269526489935\\
31.84	0.00201269524699853\\
31.85	0.00201269522908822\\
31.86	0.0020126952111684\\
31.87	0.00201269519323907\\
31.88	0.00201269517530022\\
31.89	0.00201269515735184\\
31.9	0.00201269513939394\\
31.91	0.0020126951214265\\
31.92	0.00201269510344951\\
31.93	0.00201269508546298\\
31.94	0.0020126950674669\\
31.95	0.00201269504946126\\
31.96	0.00201269503144605\\
31.97	0.00201269501342127\\
31.98	0.00201269499538692\\
31.99	0.00201269497734298\\
32	0.00201269495928946\\
32.01	0.00201269494122635\\
32.02	0.00201269492315363\\
32.03	0.00201269490507131\\
32.04	0.00201269488697938\\
32.05	0.00201269486887783\\
32.06	0.00201269485076666\\
32.07	0.00201269483264586\\
32.08	0.00201269481451543\\
32.09	0.00201269479637535\\
32.1	0.00201269477822563\\
32.11	0.00201269476006626\\
32.12	0.00201269474189724\\
32.13	0.00201269472371854\\
32.14	0.00201269470553019\\
32.15	0.00201269468733215\\
32.16	0.00201269466912443\\
32.17	0.00201269465090703\\
32.18	0.00201269463267994\\
32.19	0.00201269461444314\\
32.2	0.00201269459619664\\
32.21	0.00201269457794044\\
32.22	0.00201269455967451\\
32.23	0.00201269454139887\\
32.24	0.0020126945231135\\
32.25	0.00201269450481839\\
32.26	0.00201269448651354\\
32.27	0.00201269446819895\\
32.28	0.00201269444987461\\
32.29	0.00201269443154051\\
32.3	0.00201269441319664\\
32.31	0.00201269439484301\\
32.32	0.00201269437647961\\
32.33	0.00201269435810642\\
32.34	0.00201269433972344\\
32.35	0.00201269432133067\\
32.36	0.00201269430292811\\
32.37	0.00201269428451574\\
32.38	0.00201269426609356\\
32.39	0.00201269424766156\\
32.4	0.00201269422921974\\
32.41	0.00201269421076809\\
32.42	0.00201269419230661\\
32.43	0.00201269417383528\\
32.44	0.00201269415535412\\
32.45	0.00201269413686309\\
32.46	0.00201269411836221\\
32.47	0.00201269409985147\\
32.48	0.00201269408133086\\
32.49	0.00201269406280037\\
32.5	0.00201269404425999\\
32.51	0.00201269402570973\\
32.52	0.00201269400714958\\
32.53	0.00201269398857952\\
32.54	0.00201269396999956\\
32.55	0.00201269395140969\\
32.56	0.0020126939328099\\
32.57	0.00201269391420019\\
32.58	0.00201269389558054\\
32.59	0.00201269387695096\\
32.6	0.00201269385831144\\
32.61	0.00201269383966197\\
32.62	0.00201269382100254\\
32.63	0.00201269380233315\\
32.64	0.0020126937836538\\
32.65	0.00201269376496447\\
32.66	0.00201269374626517\\
32.67	0.00201269372755588\\
32.68	0.0020126937088366\\
32.69	0.00201269369010733\\
32.7	0.00201269367136805\\
32.71	0.00201269365261876\\
32.72	0.00201269363385946\\
32.73	0.00201269361509014\\
32.74	0.00201269359631079\\
32.75	0.0020126935775214\\
32.76	0.00201269355872198\\
32.77	0.00201269353991251\\
32.78	0.00201269352109299\\
32.79	0.00201269350226342\\
32.8	0.00201269348342378\\
32.81	0.00201269346457407\\
32.82	0.00201269344571428\\
32.83	0.00201269342684441\\
32.84	0.00201269340796446\\
32.85	0.0020126933890744\\
32.86	0.00201269337017425\\
32.87	0.00201269335126399\\
32.88	0.00201269333234362\\
32.89	0.00201269331341313\\
32.9	0.00201269329447252\\
32.91	0.00201269327552177\\
32.92	0.00201269325656089\\
32.93	0.00201269323758986\\
32.94	0.00201269321860868\\
32.95	0.00201269319961735\\
32.96	0.00201269318061586\\
32.97	0.0020126931616042\\
32.98	0.00201269314258236\\
32.99	0.00201269312355034\\
33	0.00201269310450814\\
33.01	0.00201269308545574\\
33.02	0.00201269306639314\\
33.03	0.00201269304732034\\
33.04	0.00201269302823733\\
33.05	0.00201269300914409\\
33.06	0.00201269299004064\\
33.07	0.00201269297092695\\
33.08	0.00201269295180303\\
33.09	0.00201269293266887\\
33.1	0.00201269291352445\\
33.11	0.00201269289436978\\
33.12	0.00201269287520485\\
33.13	0.00201269285602966\\
33.14	0.00201269283684418\\
33.15	0.00201269281764843\\
33.16	0.00201269279844239\\
33.17	0.00201269277922606\\
33.18	0.00201269275999943\\
33.19	0.0020126927407625\\
33.2	0.00201269272151525\\
33.21	0.00201269270225768\\
33.22	0.0020126926829898\\
33.23	0.00201269266371158\\
33.24	0.00201269264442302\\
33.25	0.00201269262512412\\
33.26	0.00201269260581488\\
33.27	0.00201269258649527\\
33.28	0.00201269256716531\\
33.29	0.00201269254782498\\
33.3	0.00201269252847427\\
33.31	0.00201269250911318\\
33.32	0.00201269248974171\\
33.33	0.00201269247035984\\
33.34	0.00201269245096757\\
33.35	0.0020126924315649\\
33.36	0.00201269241215181\\
33.37	0.0020126923927283\\
33.38	0.00201269237329438\\
33.39	0.00201269235385001\\
33.4	0.00201269233439521\\
33.41	0.00201269231492997\\
33.42	0.00201269229545428\\
33.43	0.00201269227596813\\
33.44	0.00201269225647152\\
33.45	0.00201269223696443\\
33.46	0.00201269221744687\\
33.47	0.00201269219791883\\
33.48	0.0020126921783803\\
33.49	0.00201269215883128\\
33.5	0.00201269213927175\\
33.51	0.00201269211970172\\
33.52	0.00201269210012117\\
33.53	0.0020126920805301\\
33.54	0.00201269206092851\\
33.55	0.00201269204131638\\
33.56	0.00201269202169371\\
33.57	0.0020126920020605\\
33.58	0.00201269198241673\\
33.59	0.0020126919627624\\
33.6	0.00201269194309751\\
33.61	0.00201269192342205\\
33.62	0.002012691903736\\
33.63	0.00201269188403938\\
33.64	0.00201269186433216\\
33.65	0.00201269184461435\\
33.66	0.00201269182488593\\
33.67	0.0020126918051469\\
33.68	0.00201269178539725\\
33.69	0.00201269176563698\\
33.7	0.00201269174586608\\
33.71	0.00201269172608454\\
33.72	0.00201269170629236\\
33.73	0.00201269168648953\\
33.74	0.00201269166667605\\
33.75	0.0020126916468519\\
33.76	0.00201269162701709\\
33.77	0.0020126916071716\\
33.78	0.00201269158731543\\
33.79	0.00201269156744857\\
33.8	0.00201269154757101\\
33.81	0.00201269152768276\\
33.82	0.00201269150778379\\
33.83	0.00201269148787411\\
33.84	0.00201269146795371\\
33.85	0.00201269144802259\\
33.86	0.00201269142808073\\
33.87	0.00201269140812813\\
33.88	0.00201269138816478\\
33.89	0.00201269136819068\\
33.9	0.00201269134820582\\
33.91	0.0020126913282102\\
33.92	0.0020126913082038\\
33.93	0.00201269128818662\\
33.94	0.00201269126815866\\
33.95	0.0020126912481199\\
33.96	0.00201269122807034\\
33.97	0.00201269120800998\\
33.98	0.00201269118793881\\
33.99	0.00201269116785682\\
34	0.002012691147764\\
34.01	0.00201269112766035\\
34.02	0.00201269110754586\\
34.03	0.00201269108742053\\
34.04	0.00201269106728434\\
34.05	0.0020126910471373\\
34.06	0.00201269102697939\\
34.07	0.00201269100681062\\
34.08	0.00201269098663096\\
34.09	0.00201269096644042\\
34.1	0.00201269094623898\\
34.11	0.00201269092602665\\
34.12	0.00201269090580342\\
34.13	0.00201269088556927\\
34.14	0.00201269086532421\\
34.15	0.00201269084506822\\
34.16	0.00201269082480131\\
34.17	0.00201269080452345\\
34.18	0.00201269078423465\\
34.19	0.0020126907639349\\
34.2	0.00201269074362419\\
34.21	0.00201269072330252\\
34.22	0.00201269070296988\\
34.23	0.00201269068262626\\
34.24	0.00201269066227166\\
34.25	0.00201269064190606\\
34.26	0.00201269062152947\\
34.27	0.00201269060114188\\
34.28	0.00201269058074327\\
34.29	0.00201269056033365\\
34.3	0.002012690539913\\
34.31	0.00201269051948133\\
34.32	0.00201269049903861\\
34.33	0.00201269047858486\\
34.34	0.00201269045812005\\
34.35	0.00201269043764418\\
34.36	0.00201269041715725\\
34.37	0.00201269039665925\\
34.38	0.00201269037615018\\
34.39	0.00201269035563002\\
34.4	0.00201269033509876\\
34.41	0.00201269031455641\\
34.42	0.00201269029400296\\
34.43	0.0020126902734384\\
34.44	0.00201269025286272\\
34.45	0.00201269023227591\\
34.46	0.00201269021167798\\
34.47	0.0020126901910689\\
34.48	0.00201269017044869\\
34.49	0.00201269014981732\\
34.5	0.00201269012917479\\
34.51	0.0020126901085211\\
34.52	0.00201269008785624\\
34.53	0.0020126900671802\\
34.54	0.00201269004649297\\
34.55	0.00201269002579455\\
34.56	0.00201269000508494\\
34.57	0.00201268998436412\\
34.58	0.00201268996363209\\
34.59	0.00201268994288884\\
34.6	0.00201268992213437\\
34.61	0.00201268990136866\\
34.62	0.00201268988059171\\
34.63	0.00201268985980352\\
34.64	0.00201268983900408\\
34.65	0.00201268981819338\\
34.66	0.00201268979737141\\
34.67	0.00201268977653817\\
34.68	0.00201268975569365\\
34.69	0.00201268973483784\\
34.7	0.00201268971397074\\
34.71	0.00201268969309234\\
34.72	0.00201268967220264\\
34.73	0.00201268965130162\\
34.74	0.00201268963038928\\
34.75	0.00201268960946562\\
34.76	0.00201268958853062\\
34.77	0.00201268956758428\\
34.78	0.00201268954662659\\
34.79	0.00201268952565755\\
34.8	0.00201268950467715\\
34.81	0.00201268948368538\\
34.82	0.00201268946268223\\
34.83	0.00201268944166771\\
34.84	0.00201268942064179\\
34.85	0.00201268939960448\\
34.86	0.00201268937855577\\
34.87	0.00201268935749565\\
34.88	0.00201268933642412\\
34.89	0.00201268931534116\\
34.9	0.00201268929424677\\
34.91	0.00201268927314095\\
34.92	0.00201268925202369\\
34.93	0.00201268923089497\\
34.94	0.0020126892097548\\
34.95	0.00201268918860316\\
34.96	0.00201268916744006\\
34.97	0.00201268914626548\\
34.98	0.00201268912507941\\
34.99	0.00201268910388185\\
35	0.00201268908267279\\
35.01	0.00201268906145223\\
35.02	0.00201268904022016\\
35.03	0.00201268901897657\\
35.04	0.00201268899772145\\
35.05	0.00201268897645481\\
35.06	0.00201268895517662\\
35.07	0.00201268893388689\\
35.08	0.0020126889125856\\
35.09	0.00201268889127276\\
35.1	0.00201268886994834\\
35.11	0.00201268884861236\\
35.12	0.00201268882726479\\
35.13	0.00201268880590564\\
35.14	0.00201268878453489\\
35.15	0.00201268876315254\\
35.16	0.00201268874175859\\
35.17	0.00201268872035302\\
35.18	0.00201268869893582\\
35.19	0.002012688677507\\
35.2	0.00201268865606654\\
35.21	0.00201268863461444\\
35.22	0.00201268861315069\\
35.23	0.00201268859167529\\
35.24	0.00201268857018822\\
35.25	0.00201268854868948\\
35.26	0.00201268852717906\\
35.27	0.00201268850565696\\
35.28	0.00201268848412317\\
35.29	0.00201268846257768\\
35.3	0.00201268844102048\\
35.31	0.00201268841945157\\
35.32	0.00201268839787095\\
35.33	0.00201268837627859\\
35.34	0.00201268835467451\\
35.35	0.00201268833305869\\
35.36	0.00201268831143111\\
35.37	0.00201268828979179\\
35.38	0.0020126882681407\\
35.39	0.00201268824647785\\
35.4	0.00201268822480322\\
35.41	0.00201268820311681\\
35.42	0.00201268818141861\\
35.43	0.00201268815970862\\
35.44	0.00201268813798683\\
35.45	0.00201268811625322\\
35.46	0.0020126880945078\\
35.47	0.00201268807275055\\
35.48	0.00201268805098148\\
35.49	0.00201268802920057\\
35.5	0.00201268800740781\\
35.51	0.0020126879856032\\
35.52	0.00201268796378673\\
35.53	0.0020126879419584\\
35.54	0.0020126879201182\\
35.55	0.00201268789826611\\
35.56	0.00201268787640214\\
35.57	0.00201268785452628\\
35.58	0.00201268783263852\\
35.59	0.00201268781073885\\
35.6	0.00201268778882726\\
35.61	0.00201268776690375\\
35.62	0.00201268774496832\\
35.63	0.00201268772302095\\
35.64	0.00201268770106164\\
35.65	0.00201268767909038\\
35.66	0.00201268765710716\\
35.67	0.00201268763511198\\
35.68	0.00201268761310482\\
35.69	0.0020126875910857\\
35.7	0.00201268756905458\\
35.71	0.00201268754701147\\
35.72	0.00201268752495637\\
35.73	0.00201268750288926\\
35.74	0.00201268748081014\\
35.75	0.00201268745871899\\
35.76	0.00201268743661583\\
35.77	0.00201268741450062\\
35.78	0.00201268739237338\\
35.79	0.00201268737023409\\
35.8	0.00201268734808274\\
35.81	0.00201268732591933\\
35.82	0.00201268730374385\\
35.83	0.0020126872815563\\
35.84	0.00201268725935666\\
35.85	0.00201268723714494\\
35.86	0.00201268721492111\\
35.87	0.00201268719268518\\
35.88	0.00201268717043714\\
35.89	0.00201268714817698\\
35.9	0.0020126871259047\\
35.91	0.00201268710362028\\
35.92	0.00201268708132373\\
35.93	0.00201268705901502\\
35.94	0.00201268703669417\\
35.95	0.00201268701436115\\
35.96	0.00201268699201596\\
35.97	0.0020126869696586\\
35.98	0.00201268694728906\\
35.99	0.00201268692490732\\
36	0.0020126869025134\\
36.01	0.00201268688010726\\
36.02	0.00201268685768892\\
36.03	0.00201268683525836\\
36.04	0.00201268681281557\\
36.05	0.00201268679036056\\
36.06	0.0020126867678933\\
36.07	0.0020126867454138\\
36.08	0.00201268672292205\\
36.09	0.00201268670041803\\
36.1	0.00201268667790175\\
36.11	0.00201268665537319\\
36.12	0.00201268663283235\\
36.13	0.00201268661027923\\
36.14	0.0020126865877138\\
36.15	0.00201268656513608\\
36.16	0.00201268654254604\\
36.17	0.00201268651994369\\
36.18	0.00201268649732901\\
36.19	0.00201268647470201\\
36.2	0.00201268645206266\\
36.21	0.00201268642941097\\
36.22	0.00201268640674692\\
36.23	0.00201268638407052\\
36.24	0.00201268636138175\\
36.25	0.0020126863386806\\
36.26	0.00201268631596708\\
36.27	0.00201268629324116\\
36.28	0.00201268627050285\\
36.29	0.00201268624775214\\
36.3	0.00201268622498901\\
36.31	0.00201268620221347\\
36.32	0.00201268617942551\\
36.33	0.00201268615662512\\
36.34	0.00201268613381228\\
36.35	0.002012686110987\\
36.36	0.00201268608814927\\
36.37	0.00201268606529908\\
36.38	0.00201268604243642\\
36.39	0.00201268601956129\\
36.4	0.00201268599667368\\
36.41	0.00201268597377358\\
36.42	0.00201268595086098\\
36.43	0.00201268592793588\\
36.44	0.00201268590499827\\
36.45	0.00201268588204814\\
36.46	0.0020126858590855\\
36.47	0.00201268583611031\\
36.48	0.00201268581312259\\
36.49	0.00201268579012233\\
36.5	0.00201268576710951\\
36.51	0.00201268574408413\\
36.52	0.00201268572104618\\
36.53	0.00201268569799566\\
36.54	0.00201268567493256\\
36.55	0.00201268565185687\\
36.56	0.00201268562876859\\
36.57	0.0020126856056677\\
36.58	0.0020126855825542\\
36.59	0.00201268555942808\\
36.6	0.00201268553628934\\
36.61	0.00201268551313797\\
36.62	0.00201268548997396\\
36.63	0.0020126854667973\\
36.64	0.00201268544360799\\
36.65	0.00201268542040602\\
36.66	0.00201268539719138\\
36.67	0.00201268537396407\\
36.68	0.00201268535072407\\
36.69	0.00201268532747139\\
36.7	0.00201268530420601\\
36.71	0.00201268528092793\\
36.72	0.00201268525763713\\
36.73	0.00201268523433362\\
36.74	0.00201268521101738\\
36.75	0.00201268518768841\\
36.76	0.0020126851643467\\
36.77	0.00201268514099224\\
36.78	0.00201268511762503\\
36.79	0.00201268509424506\\
36.8	0.00201268507085232\\
36.81	0.00201268504744681\\
36.82	0.00201268502402851\\
36.83	0.00201268500059742\\
36.84	0.00201268497715353\\
36.85	0.00201268495369684\\
36.86	0.00201268493022734\\
36.87	0.00201268490674502\\
36.88	0.00201268488324987\\
36.89	0.0020126848597419\\
36.9	0.00201268483622108\\
36.91	0.00201268481268741\\
36.92	0.00201268478914089\\
36.93	0.0020126847655815\\
36.94	0.00201268474200925\\
36.95	0.00201268471842412\\
36.96	0.00201268469482611\\
36.97	0.0020126846712152\\
36.98	0.0020126846475914\\
36.99	0.00201268462395469\\
37	0.00201268460030507\\
37.01	0.00201268457664253\\
37.02	0.00201268455296706\\
37.03	0.00201268452927866\\
37.04	0.00201268450557732\\
37.05	0.00201268448186303\\
37.06	0.00201268445813578\\
37.07	0.00201268443439557\\
37.08	0.00201268441064239\\
37.09	0.00201268438687623\\
37.1	0.00201268436309708\\
37.11	0.00201268433930495\\
37.12	0.00201268431549981\\
37.13	0.00201268429168167\\
37.14	0.00201268426785052\\
37.15	0.00201268424400634\\
37.16	0.00201268422014914\\
37.17	0.0020126841962789\\
37.18	0.00201268417239561\\
37.19	0.00201268414849928\\
37.2	0.00201268412458989\\
37.21	0.00201268410066744\\
37.22	0.00201268407673191\\
37.23	0.00201268405278331\\
37.24	0.00201268402882162\\
37.25	0.00201268400484683\\
37.26	0.00201268398085895\\
37.27	0.00201268395685796\\
37.28	0.00201268393284385\\
37.29	0.00201268390881662\\
37.3	0.00201268388477626\\
37.31	0.00201268386072276\\
37.32	0.00201268383665612\\
37.33	0.00201268381257633\\
37.34	0.00201268378848338\\
37.35	0.00201268376437726\\
37.36	0.00201268374025797\\
37.37	0.0020126837161255\\
37.38	0.00201268369197985\\
37.39	0.002012683667821\\
37.4	0.00201268364364894\\
37.41	0.00201268361946368\\
37.42	0.0020126835952652\\
37.43	0.0020126835710535\\
37.44	0.00201268354682856\\
37.45	0.00201268352259039\\
37.46	0.00201268349833897\\
37.47	0.0020126834740743\\
37.48	0.00201268344979637\\
37.49	0.00201268342550517\\
37.5	0.0020126834012007\\
37.51	0.00201268337688294\\
37.52	0.0020126833525519\\
37.53	0.00201268332820756\\
37.54	0.00201268330384992\\
37.55	0.00201268327947896\\
37.56	0.00201268325509469\\
37.57	0.00201268323069709\\
37.58	0.00201268320628616\\
37.59	0.00201268318186189\\
37.6	0.00201268315742428\\
37.61	0.00201268313297331\\
37.62	0.00201268310850897\\
37.63	0.00201268308403127\\
37.64	0.0020126830595402\\
37.65	0.00201268303503573\\
37.66	0.00201268301051788\\
37.67	0.00201268298598663\\
37.68	0.00201268296144198\\
37.69	0.00201268293688391\\
37.7	0.00201268291231243\\
37.71	0.00201268288772751\\
37.72	0.00201268286312916\\
37.73	0.00201268283851738\\
37.74	0.00201268281389214\\
37.75	0.00201268278925344\\
37.76	0.00201268276460129\\
37.77	0.00201268273993566\\
37.78	0.00201268271525656\\
37.79	0.00201268269056397\\
37.8	0.00201268266585788\\
37.81	0.0020126826411383\\
37.82	0.00201268261640522\\
37.83	0.00201268259165861\\
37.84	0.00201268256689849\\
37.85	0.00201268254212484\\
37.86	0.00201268251733765\\
37.87	0.00201268249253692\\
37.88	0.00201268246772264\\
37.89	0.0020126824428948\\
37.9	0.0020126824180534\\
37.91	0.00201268239319843\\
37.92	0.00201268236832987\\
37.93	0.00201268234344773\\
37.94	0.002012682318552\\
37.95	0.00201268229364266\\
37.96	0.00201268226871972\\
37.97	0.00201268224378316\\
37.98	0.00201268221883298\\
37.99	0.00201268219386917\\
38	0.00201268216889172\\
38.01	0.00201268214390062\\
38.02	0.00201268211889588\\
38.03	0.00201268209387747\\
38.04	0.0020126820688454\\
38.05	0.00201268204379965\\
38.06	0.00201268201874023\\
38.07	0.00201268199366712\\
38.08	0.00201268196858031\\
38.09	0.0020126819434798\\
38.1	0.00201268191836558\\
38.11	0.00201268189323764\\
38.12	0.00201268186809597\\
38.13	0.00201268184294058\\
38.14	0.00201268181777144\\
38.15	0.00201268179258857\\
38.16	0.00201268176739193\\
38.17	0.00201268174218154\\
38.18	0.00201268171695738\\
38.19	0.00201268169171944\\
38.2	0.00201268166646772\\
38.21	0.00201268164120222\\
38.22	0.00201268161592291\\
38.23	0.0020126815906298\\
38.24	0.00201268156532288\\
38.25	0.00201268154000214\\
38.26	0.00201268151466758\\
38.27	0.00201268148931918\\
38.28	0.00201268146395694\\
38.29	0.00201268143858085\\
38.3	0.00201268141319091\\
38.31	0.00201268138778711\\
38.32	0.00201268136236944\\
38.33	0.00201268133693789\\
38.34	0.00201268131149246\\
38.35	0.00201268128603314\\
38.36	0.00201268126055991\\
38.37	0.00201268123507278\\
38.38	0.00201268120957175\\
38.39	0.00201268118405679\\
38.4	0.0020126811585279\\
38.41	0.00201268113298508\\
38.42	0.00201268110742831\\
38.43	0.0020126810818576\\
38.44	0.00201268105627293\\
38.45	0.0020126810306743\\
38.46	0.0020126810050617\\
38.47	0.00201268097943512\\
38.48	0.00201268095379455\\
38.49	0.00201268092814\\
38.5	0.00201268090247144\\
38.51	0.00201268087678888\\
38.52	0.0020126808510923\\
38.53	0.0020126808253817\\
38.54	0.00201268079965707\\
38.55	0.00201268077391841\\
38.56	0.0020126807481657\\
38.57	0.00201268072239895\\
38.58	0.00201268069661813\\
38.59	0.00201268067082325\\
38.6	0.0020126806450143\\
38.61	0.00201268061919128\\
38.62	0.00201268059335416\\
38.63	0.00201268056750295\\
38.64	0.00201268054163764\\
38.65	0.00201268051575823\\
38.66	0.00201268048986469\\
38.67	0.00201268046395704\\
38.68	0.00201268043803525\\
38.69	0.00201268041209933\\
38.7	0.00201268038614927\\
38.71	0.00201268036018505\\
38.72	0.00201268033420668\\
38.73	0.00201268030821414\\
38.74	0.00201268028220743\\
38.75	0.00201268025618653\\
38.76	0.00201268023015145\\
38.77	0.00201268020410218\\
38.78	0.0020126801780387\\
38.79	0.00201268015196102\\
38.8	0.00201268012586911\\
38.81	0.00201268009976299\\
38.82	0.00201268007364264\\
38.83	0.00201268004750805\\
38.84	0.00201268002135921\\
38.85	0.00201267999519612\\
38.86	0.00201267996901877\\
38.87	0.00201267994282716\\
38.88	0.00201267991662127\\
38.89	0.0020126798904011\\
38.9	0.00201267986416664\\
38.91	0.00201267983791789\\
38.92	0.00201267981165484\\
38.93	0.00201267978537747\\
38.94	0.00201267975908579\\
38.95	0.00201267973277979\\
38.96	0.00201267970645946\\
38.97	0.00201267968012478\\
38.98	0.00201267965377576\\
38.99	0.00201267962741239\\
39	0.00201267960103466\\
39.01	0.00201267957464256\\
39.02	0.00201267954823609\\
39.03	0.00201267952181524\\
39.04	0.00201267949537999\\
39.05	0.00201267946893036\\
39.06	0.00201267944246632\\
39.07	0.00201267941598787\\
39.08	0.002012679389495\\
39.09	0.00201267936298771\\
39.1	0.00201267933646598\\
39.11	0.00201267930992982\\
39.12	0.00201267928337921\\
39.13	0.00201267925681415\\
39.14	0.00201267923023463\\
39.15	0.00201267920364064\\
39.16	0.00201267917703218\\
39.17	0.00201267915040924\\
39.18	0.00201267912377181\\
39.19	0.00201267909711988\\
39.2	0.00201267907045345\\
39.21	0.00201267904377251\\
39.22	0.00201267901707705\\
39.23	0.00201267899036706\\
39.24	0.00201267896364255\\
39.25	0.0020126789369035\\
39.26	0.0020126789101499\\
39.27	0.00201267888338175\\
39.28	0.00201267885659904\\
39.29	0.00201267882980176\\
39.3	0.0020126788029899\\
39.31	0.00201267877616347\\
39.32	0.00201267874932244\\
39.33	0.00201267872246682\\
39.34	0.0020126786955966\\
39.35	0.00201267866871177\\
39.36	0.00201267864181232\\
39.37	0.00201267861489825\\
39.38	0.00201267858796954\\
39.39	0.0020126785610262\\
39.4	0.00201267853406821\\
39.41	0.00201267850709557\\
39.42	0.00201267848010827\\
39.43	0.0020126784531063\\
39.44	0.00201267842608966\\
39.45	0.00201267839905834\\
39.46	0.00201267837201233\\
39.47	0.00201267834495162\\
39.48	0.00201267831787621\\
39.49	0.0020126782907861\\
39.5	0.00201267826368126\\
39.51	0.0020126782365617\\
39.52	0.00201267820942742\\
39.53	0.00201267818227839\\
39.54	0.00201267815511462\\
39.55	0.00201267812793609\\
39.56	0.00201267810074281\\
39.57	0.00201267807353476\\
39.58	0.00201267804631194\\
39.59	0.00201267801907434\\
39.6	0.00201267799182195\\
39.61	0.00201267796455477\\
39.62	0.00201267793727278\\
39.63	0.00201267790997599\\
39.64	0.00201267788266438\\
39.65	0.00201267785533794\\
39.66	0.00201267782799668\\
39.67	0.00201267780064058\\
39.68	0.00201267777326964\\
39.69	0.00201267774588385\\
39.7	0.0020126777184832\\
39.71	0.00201267769106768\\
39.72	0.00201267766363729\\
39.73	0.00201267763619202\\
39.74	0.00201267760873187\\
39.75	0.00201267758125682\\
39.76	0.00201267755376688\\
39.77	0.00201267752626202\\
39.78	0.00201267749874225\\
39.79	0.00201267747120756\\
39.8	0.00201267744365794\\
39.81	0.00201267741609339\\
39.82	0.00201267738851389\\
39.83	0.00201267736091945\\
39.84	0.00201267733331005\\
39.85	0.00201267730568568\\
39.86	0.00201267727804634\\
39.87	0.00201267725039203\\
39.88	0.00201267722272273\\
39.89	0.00201267719503844\\
39.9	0.00201267716733915\\
39.91	0.00201267713962486\\
39.92	0.00201267711189555\\
39.93	0.00201267708415123\\
39.94	0.00201267705639188\\
39.95	0.00201267702861749\\
39.96	0.00201267700082806\\
39.97	0.00201267697302359\\
39.98	0.00201267694520406\\
39.99	0.00201267691736947\\
40	0.00201267688951981\\
40.01	0.00201267686165508\\
};
\addplot [color=blue,dashed,forget plot]
  table[row sep=crcr]{%
40.01	0.00201267686165508\\
40.02	0.00201267683377526\\
40.03	0.00201267680588036\\
40.04	0.00201267677797036\\
40.05	0.00201267675004525\\
40.06	0.00201267672210504\\
40.07	0.0020126766941497\\
40.08	0.00201267666617925\\
40.09	0.00201267663819366\\
40.1	0.00201267661019293\\
40.11	0.00201267658217706\\
40.12	0.00201267655414603\\
40.13	0.00201267652609985\\
40.14	0.0020126764980385\\
40.15	0.00201267646996198\\
40.16	0.00201267644187028\\
40.17	0.00201267641376339\\
40.18	0.00201267638564131\\
40.19	0.00201267635750403\\
40.2	0.00201267632935154\\
40.21	0.00201267630118383\\
40.22	0.00201267627300091\\
40.23	0.00201267624480275\\
40.24	0.00201267621658936\\
40.25	0.00201267618836073\\
40.26	0.00201267616011685\\
40.27	0.00201267613185771\\
40.28	0.00201267610358331\\
40.29	0.00201267607529364\\
40.3	0.00201267604698869\\
40.31	0.00201267601866846\\
40.32	0.00201267599033294\\
40.33	0.00201267596198212\\
40.34	0.002012675933616\\
40.35	0.00201267590523456\\
40.36	0.00201267587683781\\
40.37	0.00201267584842573\\
40.38	0.00201267581999832\\
40.39	0.00201267579155557\\
40.4	0.00201267576309748\\
40.41	0.00201267573462403\\
40.42	0.00201267570613523\\
40.43	0.00201267567763106\\
40.44	0.00201267564911151\\
40.45	0.00201267562057658\\
40.46	0.00201267559202627\\
40.47	0.00201267556346056\\
40.48	0.00201267553487946\\
40.49	0.00201267550628294\\
40.5	0.00201267547767101\\
40.51	0.00201267544904366\\
40.52	0.00201267542040088\\
40.53	0.00201267539174267\\
40.54	0.00201267536306902\\
40.55	0.00201267533437991\\
40.56	0.00201267530567536\\
40.57	0.00201267527695533\\
40.58	0.00201267524821984\\
40.59	0.00201267521946888\\
40.6	0.00201267519070243\\
40.61	0.00201267516192049\\
40.62	0.00201267513312306\\
40.63	0.00201267510431012\\
40.64	0.00201267507548167\\
40.65	0.00201267504663771\\
40.66	0.00201267501777822\\
40.67	0.0020126749889032\\
40.68	0.00201267496001264\\
40.69	0.00201267493110654\\
40.7	0.00201267490218489\\
40.71	0.00201267487324768\\
40.72	0.00201267484429491\\
40.73	0.00201267481532657\\
40.74	0.00201267478634265\\
40.75	0.00201267475734314\\
40.76	0.00201267472832804\\
40.77	0.00201267469929735\\
40.78	0.00201267467025105\\
40.79	0.00201267464118914\\
40.8	0.0020126746121116\\
40.81	0.00201267458301845\\
40.82	0.00201267455390966\\
40.83	0.00201267452478523\\
40.84	0.00201267449564516\\
40.85	0.00201267446648943\\
40.86	0.00201267443731805\\
40.87	0.002012674408131\\
40.88	0.00201267437892827\\
40.89	0.00201267434970987\\
40.9	0.00201267432047579\\
40.91	0.002012674291226\\
40.92	0.00201267426196052\\
40.93	0.00201267423267934\\
40.94	0.00201267420338244\\
40.95	0.00201267417406982\\
40.96	0.00201267414474147\\
40.97	0.00201267411539739\\
40.98	0.00201267408603757\\
40.99	0.00201267405666201\\
41	0.00201267402727069\\
41.01	0.00201267399786361\\
41.02	0.00201267396844077\\
41.03	0.00201267393900215\\
41.04	0.00201267390954775\\
41.05	0.00201267388007757\\
41.06	0.00201267385059159\\
41.07	0.00201267382108981\\
41.08	0.00201267379157222\\
41.09	0.00201267376203883\\
41.1	0.00201267373248961\\
41.11	0.00201267370292456\\
41.12	0.00201267367334368\\
41.13	0.00201267364374696\\
41.14	0.0020126736141344\\
41.15	0.00201267358450598\\
41.16	0.0020126735548617\\
41.17	0.00201267352520155\\
41.18	0.00201267349552554\\
41.19	0.00201267346583364\\
41.2	0.00201267343612585\\
41.21	0.00201267340640217\\
41.22	0.00201267337666259\\
41.23	0.0020126733469071\\
41.24	0.0020126733171357\\
41.25	0.00201267328734838\\
41.26	0.00201267325754514\\
41.27	0.00201267322772596\\
41.28	0.00201267319789084\\
41.29	0.00201267316803978\\
41.3	0.00201267313817276\\
41.31	0.00201267310828978\\
41.32	0.00201267307839084\\
41.33	0.00201267304847592\\
41.34	0.00201267301854503\\
41.35	0.00201267298859814\\
41.36	0.00201267295863527\\
41.37	0.0020126729286564\\
41.38	0.00201267289866152\\
41.39	0.00201267286865063\\
41.4	0.00201267283862372\\
41.41	0.00201267280858078\\
41.42	0.00201267277852181\\
41.43	0.0020126727484468\\
41.44	0.00201267271835575\\
41.45	0.00201267268824865\\
41.46	0.00201267265812549\\
41.47	0.00201267262798626\\
41.48	0.00201267259783096\\
41.49	0.00201267256765958\\
41.5	0.00201267253747212\\
41.51	0.00201267250726857\\
41.52	0.00201267247704892\\
41.53	0.00201267244681317\\
41.54	0.0020126724165613\\
41.55	0.00201267238629332\\
41.56	0.00201267235600921\\
41.57	0.00201267232570898\\
41.58	0.0020126722953926\\
41.59	0.00201267226506009\\
41.6	0.00201267223471142\\
41.61	0.0020126722043466\\
41.62	0.00201267217396561\\
41.63	0.00201267214356845\\
41.64	0.00201267211315512\\
41.65	0.00201267208272561\\
41.66	0.0020126720522799\\
41.67	0.002012672021818\\
41.68	0.0020126719913399\\
41.69	0.00201267196084559\\
41.7	0.00201267193033507\\
41.71	0.00201267189980832\\
41.72	0.00201267186926535\\
41.73	0.00201267183870614\\
41.74	0.00201267180813069\\
41.75	0.00201267177753899\\
41.76	0.00201267174693104\\
41.77	0.00201267171630683\\
41.78	0.00201267168566635\\
41.79	0.0020126716550096\\
41.8	0.00201267162433657\\
41.81	0.00201267159364725\\
41.82	0.00201267156294164\\
41.83	0.00201267153221973\\
41.84	0.00201267150148151\\
41.85	0.00201267147072699\\
41.86	0.00201267143995614\\
41.87	0.00201267140916897\\
41.88	0.00201267137836547\\
41.89	0.00201267134754562\\
41.9	0.00201267131670944\\
41.91	0.0020126712858569\\
41.92	0.00201267125498801\\
41.93	0.00201267122410275\\
41.94	0.00201267119320113\\
41.95	0.00201267116228313\\
41.96	0.00201267113134874\\
41.97	0.00201267110039796\\
41.98	0.00201267106943079\\
41.99	0.00201267103844722\\
42	0.00201267100744724\\
42.01	0.00201267097643084\\
42.02	0.00201267094539802\\
42.03	0.00201267091434877\\
42.04	0.0020126708832831\\
42.05	0.00201267085220097\\
42.06	0.00201267082110241\\
42.07	0.00201267078998738\\
42.08	0.0020126707588559\\
42.09	0.00201267072770796\\
42.1	0.00201267069654354\\
42.11	0.00201267066536264\\
42.12	0.00201267063416525\\
42.13	0.00201267060295138\\
42.14	0.002012670571721\\
42.15	0.00201267054047412\\
42.16	0.00201267050921074\\
42.17	0.00201267047793083\\
42.18	0.0020126704466344\\
42.19	0.00201267041532144\\
42.2	0.00201267038399194\\
42.21	0.0020126703526459\\
42.22	0.00201267032128332\\
42.23	0.00201267028990417\\
42.24	0.00201267025850847\\
42.25	0.00201267022709619\\
42.26	0.00201267019566735\\
42.27	0.00201267016422192\\
42.28	0.0020126701327599\\
42.29	0.00201267010128129\\
42.3	0.00201267006978609\\
42.31	0.00201267003827427\\
42.32	0.00201267000674584\\
42.33	0.0020126699752008\\
42.34	0.00201266994363913\\
42.35	0.00201266991206082\\
42.36	0.00201266988046588\\
42.37	0.0020126698488543\\
42.38	0.00201266981722606\\
42.39	0.00201266978558117\\
42.4	0.00201266975391962\\
42.41	0.00201266972224139\\
42.42	0.00201266969054649\\
42.43	0.00201266965883491\\
42.44	0.00201266962710664\\
42.45	0.00201266959536168\\
42.46	0.00201266956360001\\
42.47	0.00201266953182164\\
42.48	0.00201266950002655\\
42.49	0.00201266946821474\\
42.5	0.00201266943638621\\
42.51	0.00201266940454095\\
42.52	0.00201266937267894\\
42.53	0.0020126693408002\\
42.54	0.0020126693089047\\
42.55	0.00201266927699244\\
42.56	0.00201266924506342\\
42.57	0.00201266921311762\\
42.58	0.00201266918115506\\
42.59	0.0020126691491757\\
42.6	0.00201266911717956\\
42.61	0.00201266908516663\\
42.62	0.00201266905313689\\
42.63	0.00201266902109035\\
42.64	0.00201266898902699\\
42.65	0.00201266895694681\\
42.66	0.0020126689248498\\
42.67	0.00201266889273596\\
42.68	0.00201266886060528\\
42.69	0.00201266882845776\\
42.7	0.00201266879629338\\
42.71	0.00201266876411215\\
42.72	0.00201266873191405\\
42.73	0.00201266869969908\\
42.74	0.00201266866746723\\
42.75	0.0020126686352185\\
42.76	0.00201266860295289\\
42.77	0.00201266857067037\\
42.78	0.00201266853837096\\
42.79	0.00201266850605463\\
42.8	0.00201266847372139\\
42.81	0.00201266844137123\\
42.82	0.00201266840900415\\
42.83	0.00201266837662012\\
42.84	0.00201266834421917\\
42.85	0.00201266831180126\\
42.86	0.0020126682793664\\
42.87	0.00201266824691459\\
42.88	0.00201266821444581\\
42.89	0.00201266818196006\\
42.9	0.00201266814945733\\
42.91	0.00201266811693762\\
42.92	0.00201266808440092\\
42.93	0.00201266805184722\\
42.94	0.00201266801927653\\
42.95	0.00201266798668882\\
42.96	0.0020126679540841\\
42.97	0.00201266792146236\\
42.98	0.00201266788882359\\
42.99	0.00201266785616779\\
43	0.00201266782349495\\
43.01	0.00201266779080506\\
43.02	0.00201266775809812\\
43.03	0.00201266772537412\\
43.04	0.00201266769263306\\
43.05	0.00201266765987493\\
43.06	0.00201266762709972\\
43.07	0.00201266759430743\\
43.08	0.00201266756149805\\
43.09	0.00201266752867157\\
43.1	0.00201266749582799\\
43.11	0.00201266746296731\\
43.12	0.0020126674300895\\
43.13	0.00201266739719458\\
43.14	0.00201266736428253\\
43.15	0.00201266733135335\\
43.16	0.00201266729840703\\
43.17	0.00201266726544356\\
43.18	0.00201266723246295\\
43.19	0.00201266719946517\\
43.2	0.00201266716645023\\
43.21	0.00201266713341812\\
43.22	0.00201266710036883\\
43.23	0.00201266706730236\\
43.24	0.00201266703421869\\
43.25	0.00201266700111784\\
43.26	0.00201266696799978\\
43.27	0.00201266693486451\\
43.28	0.00201266690171203\\
43.29	0.00201266686854233\\
43.3	0.00201266683535541\\
43.31	0.00201266680215124\\
43.32	0.00201266676892984\\
43.33	0.0020126667356912\\
43.34	0.0020126667024353\\
43.35	0.00201266666916215\\
43.36	0.00201266663587173\\
43.37	0.00201266660256404\\
43.38	0.00201266656923907\\
43.39	0.00201266653589682\\
43.4	0.00201266650253728\\
43.41	0.00201266646916044\\
43.42	0.00201266643576631\\
43.43	0.00201266640235486\\
43.44	0.0020126663689261\\
43.45	0.00201266633548002\\
43.46	0.00201266630201662\\
43.47	0.00201266626853588\\
43.48	0.0020126662350378\\
43.49	0.00201266620152237\\
43.5	0.00201266616798959\\
43.51	0.00201266613443946\\
43.52	0.00201266610087196\\
43.53	0.00201266606728709\\
43.54	0.00201266603368484\\
43.55	0.00201266600006521\\
43.56	0.00201266596642818\\
43.57	0.00201266593277377\\
43.58	0.00201266589910195\\
43.59	0.00201266586541272\\
43.6	0.00201266583170607\\
43.61	0.00201266579798201\\
43.62	0.00201266576424052\\
43.63	0.00201266573048159\\
43.64	0.00201266569670523\\
43.65	0.00201266566291141\\
43.66	0.00201266562910015\\
43.67	0.00201266559527142\\
43.68	0.00201266556142523\\
43.69	0.00201266552756157\\
43.7	0.00201266549368043\\
43.71	0.00201266545978181\\
43.72	0.0020126654258657\\
43.73	0.00201266539193208\\
43.74	0.00201266535798097\\
43.75	0.00201266532401235\\
43.76	0.0020126652900262\\
43.77	0.00201266525602254\\
43.78	0.00201266522200135\\
43.79	0.00201266518796262\\
43.8	0.00201266515390636\\
43.81	0.00201266511983254\\
43.82	0.00201266508574117\\
43.83	0.00201266505163224\\
43.84	0.00201266501750575\\
43.85	0.00201266498336167\\
43.86	0.00201266494920002\\
43.87	0.00201266491502079\\
43.88	0.00201266488082396\\
43.89	0.00201266484660954\\
43.9	0.0020126648123775\\
43.91	0.00201266477812786\\
43.92	0.0020126647438606\\
43.93	0.00201266470957571\\
43.94	0.0020126646752732\\
43.95	0.00201266464095304\\
43.96	0.00201266460661525\\
43.97	0.0020126645722598\\
43.98	0.0020126645378867\\
43.99	0.00201266450349593\\
44	0.00201266446908749\\
44.01	0.00201266443466138\\
44.02	0.00201266440021759\\
44.03	0.00201266436575611\\
44.04	0.00201266433127693\\
44.05	0.00201266429678005\\
44.06	0.00201266426226547\\
44.07	0.00201266422773316\\
44.08	0.00201266419318314\\
44.09	0.00201266415861539\\
44.1	0.00201266412402991\\
44.11	0.00201266408942669\\
44.12	0.00201266405480572\\
44.13	0.00201266402016699\\
44.14	0.00201266398551051\\
44.15	0.00201266395083626\\
44.16	0.00201266391614424\\
44.17	0.00201266388143444\\
44.18	0.00201266384670685\\
44.19	0.00201266381196147\\
44.2	0.00201266377719829\\
44.21	0.0020126637424173\\
44.22	0.0020126637076185\\
44.23	0.00201266367280189\\
44.24	0.00201266363796745\\
44.25	0.00201266360311517\\
44.26	0.00201266356824506\\
44.27	0.0020126635333571\\
44.28	0.0020126634984513\\
44.29	0.00201266346352763\\
44.3	0.0020126634285861\\
44.31	0.0020126633936267\\
44.32	0.00201266335864941\\
44.33	0.00201266332365425\\
44.34	0.00201266328864119\\
44.35	0.00201266325361024\\
44.36	0.00201266321856138\\
44.37	0.0020126631834946\\
44.38	0.00201266314840992\\
44.39	0.0020126631133073\\
44.4	0.00201266307818676\\
44.41	0.00201266304304827\\
44.42	0.00201266300789185\\
44.43	0.00201266297271747\\
44.44	0.00201266293752513\\
44.45	0.00201266290231483\\
44.46	0.00201266286708656\\
44.47	0.0020126628318403\\
44.48	0.00201266279657607\\
44.49	0.00201266276129384\\
44.5	0.00201266272599361\\
44.51	0.00201266269067538\\
44.52	0.00201266265533913\\
44.53	0.00201266261998487\\
44.54	0.00201266258461258\\
44.55	0.00201266254922226\\
44.56	0.00201266251381389\\
44.57	0.00201266247838749\\
44.58	0.00201266244294303\\
44.59	0.0020126624074805\\
44.6	0.00201266237199992\\
44.61	0.00201266233650126\\
44.62	0.00201266230098452\\
44.63	0.00201266226544969\\
44.64	0.00201266222989676\\
44.65	0.00201266219432574\\
44.66	0.00201266215873661\\
44.67	0.00201266212312936\\
44.68	0.00201266208750399\\
44.69	0.0020126620518605\\
44.7	0.00201266201619886\\
44.71	0.00201266198051908\\
44.72	0.00201266194482116\\
44.73	0.00201266190910507\\
44.74	0.00201266187337083\\
44.75	0.00201266183761841\\
44.76	0.00201266180184782\\
44.77	0.00201266176605904\\
44.78	0.00201266173025206\\
44.79	0.0020126616944269\\
44.8	0.00201266165858352\\
44.81	0.00201266162272193\\
44.82	0.00201266158684212\\
44.83	0.00201266155094408\\
44.84	0.00201266151502781\\
44.85	0.0020126614790933\\
44.86	0.00201266144314054\\
44.87	0.00201266140716952\\
44.88	0.00201266137118024\\
44.89	0.00201266133517269\\
44.9	0.00201266129914687\\
44.91	0.00201266126310276\\
44.92	0.00201266122704035\\
44.93	0.00201266119095965\\
44.94	0.00201266115486064\\
44.95	0.00201266111874332\\
44.96	0.00201266108260768\\
44.97	0.00201266104645371\\
44.98	0.00201266101028141\\
44.99	0.00201266097409076\\
45	0.00201266093788176\\
45.01	0.00201266090165441\\
45.02	0.00201266086540869\\
45.03	0.0020126608291446\\
45.04	0.00201266079286213\\
45.05	0.00201266075656128\\
45.06	0.00201266072024203\\
45.07	0.00201266068390438\\
45.08	0.00201266064754832\\
45.09	0.00201266061117384\\
45.1	0.00201266057478094\\
45.11	0.00201266053836961\\
45.12	0.00201266050193985\\
45.13	0.00201266046549163\\
45.14	0.00201266042902496\\
45.15	0.00201266039253984\\
45.16	0.00201266035603624\\
45.17	0.00201266031951416\\
45.18	0.00201266028297361\\
45.19	0.00201266024641456\\
45.2	0.00201266020983701\\
45.21	0.00201266017324096\\
45.22	0.00201266013662639\\
45.23	0.0020126600999933\\
45.24	0.00201266006334168\\
45.25	0.00201266002667152\\
45.26	0.00201265998998282\\
45.27	0.00201265995327556\\
45.28	0.00201265991654975\\
45.29	0.00201265987980537\\
45.3	0.00201265984304241\\
45.31	0.00201265980626087\\
45.32	0.00201265976946074\\
45.33	0.002012659732642\\
45.34	0.00201265969580467\\
45.35	0.00201265965894871\\
45.36	0.00201265962207414\\
45.37	0.00201265958518093\\
45.38	0.00201265954826908\\
45.39	0.00201265951133859\\
45.4	0.00201265947438945\\
45.41	0.00201265943742164\\
45.42	0.00201265940043516\\
45.43	0.00201265936343\\
45.44	0.00201265932640616\\
45.45	0.00201265928936362\\
45.46	0.00201265925230238\\
45.47	0.00201265921522243\\
45.48	0.00201265917812376\\
45.49	0.00201265914100637\\
45.5	0.00201265910387024\\
45.51	0.00201265906671537\\
45.52	0.00201265902954175\\
45.53	0.00201265899234936\\
45.54	0.00201265895513821\\
45.55	0.00201265891790829\\
45.56	0.00201265888065958\\
45.57	0.00201265884339208\\
45.58	0.00201265880610578\\
45.59	0.00201265876880067\\
45.6	0.00201265873147675\\
45.61	0.00201265869413399\\
45.62	0.00201265865677241\\
45.63	0.00201265861939198\\
45.64	0.00201265858199271\\
45.65	0.00201265854457457\\
45.66	0.00201265850713757\\
45.67	0.00201265846968169\\
45.68	0.00201265843220693\\
45.69	0.00201265839471327\\
45.7	0.00201265835720072\\
45.71	0.00201265831966926\\
45.72	0.00201265828211887\\
45.73	0.00201265824454956\\
45.74	0.00201265820696131\\
45.75	0.00201265816935413\\
45.76	0.00201265813172798\\
45.77	0.00201265809408288\\
45.78	0.00201265805641881\\
45.79	0.00201265801873576\\
45.8	0.00201265798103372\\
45.81	0.00201265794331268\\
45.82	0.00201265790557264\\
45.83	0.00201265786781359\\
45.84	0.00201265783003551\\
45.85	0.00201265779223841\\
45.86	0.00201265775442226\\
45.87	0.00201265771658706\\
45.88	0.00201265767873281\\
45.89	0.00201265764085948\\
45.9	0.00201265760296709\\
45.91	0.0020126575650556\\
45.92	0.00201265752712503\\
45.93	0.00201265748917535\\
45.94	0.00201265745120656\\
45.95	0.00201265741321864\\
45.96	0.0020126573752116\\
45.97	0.00201265733718542\\
45.98	0.00201265729914009\\
45.99	0.0020126572610756\\
46	0.00201265722299194\\
46.01	0.00201265718488911\\
46.02	0.00201265714676709\\
46.03	0.00201265710862588\\
46.04	0.00201265707046547\\
46.05	0.00201265703228584\\
46.06	0.00201265699408699\\
46.07	0.0020126569558689\\
46.08	0.00201265691763158\\
46.09	0.002012656879375\\
46.1	0.00201265684109917\\
46.11	0.00201265680280406\\
46.12	0.00201265676448968\\
46.13	0.00201265672615601\\
46.14	0.00201265668780303\\
46.15	0.00201265664943076\\
46.16	0.00201265661103916\\
46.17	0.00201265657262823\\
46.18	0.00201265653419797\\
46.19	0.00201265649574837\\
46.2	0.0020126564572794\\
46.21	0.00201265641879107\\
46.22	0.00201265638028337\\
46.23	0.00201265634175628\\
46.24	0.00201265630320979\\
46.25	0.0020126562646439\\
46.26	0.00201265622605859\\
46.27	0.00201265618745386\\
46.28	0.00201265614882969\\
46.29	0.00201265611018609\\
46.3	0.00201265607152302\\
46.31	0.00201265603284049\\
46.32	0.00201265599413848\\
46.33	0.00201265595541699\\
46.34	0.00201265591667601\\
46.35	0.00201265587791552\\
46.36	0.00201265583913551\\
46.37	0.00201265580033598\\
46.38	0.00201265576151691\\
46.39	0.0020126557226783\\
46.4	0.00201265568382013\\
46.41	0.00201265564494239\\
46.42	0.00201265560604508\\
46.43	0.00201265556712818\\
46.44	0.00201265552819168\\
46.45	0.00201265548923558\\
46.46	0.00201265545025985\\
46.47	0.0020126554112645\\
46.48	0.00201265537224951\\
46.49	0.00201265533321487\\
46.5	0.00201265529416056\\
46.51	0.00201265525508659\\
46.52	0.00201265521599293\\
46.53	0.00201265517687959\\
46.54	0.00201265513774654\\
46.55	0.00201265509859377\\
46.56	0.00201265505942128\\
46.57	0.00201265502022906\\
46.58	0.00201265498101709\\
46.59	0.00201265494178536\\
46.6	0.00201265490253387\\
46.61	0.00201265486326259\\
46.62	0.00201265482397153\\
46.63	0.00201265478466067\\
46.64	0.00201265474532999\\
46.65	0.0020126547059795\\
46.66	0.00201265466660917\\
46.67	0.00201265462721899\\
46.68	0.00201265458780896\\
46.69	0.00201265454837906\\
46.7	0.00201265450892929\\
46.71	0.00201265446945962\\
46.72	0.00201265442997006\\
46.73	0.00201265439046058\\
46.74	0.00201265435093119\\
46.75	0.00201265431138185\\
46.76	0.00201265427181257\\
46.77	0.00201265423222334\\
46.78	0.00201265419261413\\
46.79	0.00201265415298495\\
46.8	0.00201265411333578\\
46.81	0.0020126540736666\\
46.82	0.00201265403397741\\
46.83	0.00201265399426819\\
46.84	0.00201265395453893\\
46.85	0.00201265391478963\\
46.86	0.00201265387502026\\
46.87	0.00201265383523083\\
46.88	0.0020126537954213\\
46.89	0.00201265375559169\\
46.9	0.00201265371574196\\
46.91	0.00201265367587212\\
46.92	0.00201265363598214\\
46.93	0.00201265359607202\\
46.94	0.00201265355614175\\
46.95	0.0020126535161913\\
46.96	0.00201265347622068\\
46.97	0.00201265343622987\\
46.98	0.00201265339621886\\
46.99	0.00201265335618763\\
47	0.00201265331613617\\
47.01	0.00201265327606447\\
47.02	0.00201265323597252\\
47.03	0.00201265319586031\\
47.04	0.00201265315572782\\
47.05	0.00201265311557504\\
47.06	0.00201265307540197\\
47.07	0.00201265303520857\\
47.08	0.00201265299499486\\
47.09	0.0020126529547608\\
47.1	0.0020126529145064\\
47.11	0.00201265287423163\\
47.12	0.00201265283393649\\
47.13	0.00201265279362096\\
47.14	0.00201265275328503\\
47.15	0.00201265271292869\\
47.16	0.00201265267255192\\
47.17	0.00201265263215471\\
47.18	0.00201265259173705\\
47.19	0.00201265255129893\\
47.2	0.00201265251084034\\
47.21	0.00201265247036125\\
47.22	0.00201265242986166\\
47.23	0.00201265238934156\\
47.24	0.00201265234880093\\
47.25	0.00201265230823976\\
47.26	0.00201265226765803\\
47.27	0.00201265222705574\\
47.28	0.00201265218643287\\
47.29	0.00201265214578941\\
47.3	0.00201265210512534\\
47.31	0.00201265206444065\\
47.32	0.00201265202373533\\
47.33	0.00201265198300937\\
47.34	0.00201265194226274\\
47.35	0.00201265190149545\\
47.36	0.00201265186070747\\
47.37	0.0020126518198988\\
47.38	0.00201265177906941\\
47.39	0.0020126517382193\\
47.4	0.00201265169734845\\
47.41	0.00201265165645684\\
47.42	0.00201265161554448\\
47.43	0.00201265157461133\\
47.44	0.00201265153365739\\
47.45	0.00201265149268265\\
47.46	0.00201265145168709\\
47.47	0.00201265141067069\\
47.48	0.00201265136963345\\
47.49	0.00201265132857535\\
47.5	0.00201265128749638\\
47.51	0.00201265124639652\\
47.52	0.00201265120527575\\
47.53	0.00201265116413408\\
47.54	0.00201265112297147\\
47.55	0.00201265108178792\\
47.56	0.00201265104058341\\
47.57	0.00201265099935794\\
47.58	0.00201265095811148\\
47.59	0.00201265091684402\\
47.6	0.00201265087555554\\
47.61	0.00201265083424605\\
47.62	0.00201265079291551\\
47.63	0.00201265075156391\\
47.64	0.00201265071019125\\
47.65	0.00201265066879751\\
47.66	0.00201265062738266\\
47.67	0.00201265058594671\\
47.68	0.00201265054448963\\
47.69	0.00201265050301141\\
47.7	0.00201265046151204\\
47.71	0.0020126504199915\\
47.72	0.00201265037844978\\
47.73	0.00201265033688685\\
47.74	0.00201265029530272\\
47.75	0.00201265025369736\\
47.76	0.00201265021207077\\
47.77	0.00201265017042291\\
47.78	0.00201265012875379\\
47.79	0.00201265008706338\\
47.8	0.00201265004535168\\
47.81	0.00201265000361866\\
47.82	0.00201264996186431\\
47.83	0.00201264992008862\\
47.84	0.00201264987829157\\
47.85	0.00201264983647315\\
47.86	0.00201264979463335\\
47.87	0.00201264975277214\\
47.88	0.00201264971088951\\
47.89	0.00201264966898546\\
47.9	0.00201264962705996\\
47.91	0.00201264958511299\\
47.92	0.00201264954314456\\
47.93	0.00201264950115463\\
47.94	0.00201264945914319\\
47.95	0.00201264941711024\\
47.96	0.00201264937505575\\
47.97	0.00201264933297971\\
47.98	0.0020126492908821\\
47.99	0.00201264924876291\\
48	0.00201264920662213\\
48.01	0.00201264916445973\\
48.02	0.00201264912227571\\
48.03	0.00201264908007004\\
48.04	0.00201264903784272\\
48.05	0.00201264899559373\\
48.06	0.00201264895332305\\
48.07	0.00201264891103067\\
48.08	0.00201264886871657\\
48.09	0.00201264882638073\\
48.1	0.00201264878402315\\
48.11	0.00201264874164381\\
48.12	0.00201264869924268\\
48.13	0.00201264865681976\\
48.14	0.00201264861437503\\
48.15	0.00201264857190847\\
48.16	0.00201264852942007\\
48.17	0.00201264848690982\\
48.18	0.00201264844437769\\
48.19	0.00201264840182368\\
48.2	0.00201264835924776\\
48.21	0.00201264831664992\\
48.22	0.00201264827403015\\
48.23	0.00201264823138843\\
48.24	0.00201264818872474\\
48.25	0.00201264814603907\\
48.26	0.0020126481033314\\
48.27	0.00201264806060172\\
48.28	0.00201264801785001\\
48.29	0.00201264797507626\\
48.3	0.00201264793228044\\
48.31	0.00201264788946255\\
48.32	0.00201264784662256\\
48.33	0.00201264780376047\\
48.34	0.00201264776087626\\
48.35	0.0020126477179699\\
48.36	0.00201264767504138\\
48.37	0.0020126476320907\\
48.38	0.00201264758911783\\
48.39	0.00201264754612275\\
48.4	0.00201264750310545\\
48.41	0.00201264746006592\\
48.42	0.00201264741700414\\
48.43	0.00201264737392008\\
48.44	0.00201264733081375\\
48.45	0.00201264728768511\\
48.46	0.00201264724453415\\
48.47	0.00201264720136087\\
48.48	0.00201264715816523\\
48.49	0.00201264711494723\\
48.5	0.00201264707170685\\
48.51	0.00201264702844407\\
48.52	0.00201264698515887\\
48.53	0.00201264694185125\\
48.54	0.00201264689852118\\
48.55	0.00201264685516865\\
48.56	0.00201264681179363\\
48.57	0.00201264676839613\\
48.58	0.00201264672497611\\
48.59	0.00201264668153356\\
48.6	0.00201264663806846\\
48.61	0.00201264659458081\\
48.62	0.00201264655107058\\
48.63	0.00201264650753775\\
48.64	0.00201264646398231\\
48.65	0.00201264642040425\\
48.66	0.00201264637680354\\
48.67	0.00201264633318017\\
48.68	0.00201264628953412\\
48.69	0.00201264624586538\\
48.7	0.00201264620217393\\
48.71	0.00201264615845976\\
48.72	0.00201264611472283\\
48.73	0.00201264607096315\\
48.74	0.00201264602718069\\
48.75	0.00201264598337544\\
48.76	0.00201264593954738\\
48.77	0.00201264589569649\\
48.78	0.00201264585182275\\
48.79	0.00201264580792615\\
48.8	0.00201264576400668\\
48.81	0.00201264572006431\\
48.82	0.00201264567609903\\
48.83	0.00201264563211082\\
48.84	0.00201264558809966\\
48.85	0.00201264554406554\\
48.86	0.00201264550000844\\
48.87	0.00201264545592835\\
48.88	0.00201264541182524\\
48.89	0.00201264536769909\\
48.9	0.0020126453235499\\
48.91	0.00201264527937765\\
48.92	0.00201264523518232\\
48.93	0.00201264519096388\\
48.94	0.00201264514672233\\
48.95	0.00201264510245764\\
48.96	0.0020126450581698\\
48.97	0.0020126450138588\\
48.98	0.00201264496952461\\
48.99	0.00201264492516721\\
49	0.0020126448807866\\
49.01	0.00201264483638275\\
49.02	0.00201264479195564\\
49.03	0.00201264474750526\\
49.04	0.00201264470303159\\
49.05	0.00201264465853462\\
49.06	0.00201264461401432\\
49.07	0.00201264456947067\\
49.08	0.00201264452490367\\
49.09	0.00201264448031329\\
49.1	0.00201264443569952\\
49.11	0.00201264439106233\\
49.12	0.00201264434640172\\
49.13	0.00201264430171766\\
49.14	0.00201264425701013\\
49.15	0.00201264421227912\\
49.16	0.00201264416752461\\
49.17	0.00201264412274658\\
49.18	0.00201264407794501\\
49.19	0.00201264403311989\\
49.2	0.0020126439882712\\
49.21	0.00201264394339892\\
49.22	0.00201264389850304\\
49.23	0.00201264385358352\\
49.24	0.00201264380864037\\
49.25	0.00201264376367355\\
49.26	0.00201264371868306\\
49.27	0.00201264367366887\\
49.28	0.00201264362863097\\
49.29	0.00201264358356933\\
49.3	0.00201264353848394\\
49.31	0.00201264349337479\\
49.32	0.00201264344824184\\
49.33	0.00201264340308509\\
49.34	0.00201264335790452\\
49.35	0.00201264331270011\\
49.36	0.00201264326747184\\
49.37	0.00201264322221969\\
49.38	0.00201264317694365\\
49.39	0.00201264313164369\\
49.4	0.0020126430863198\\
49.41	0.00201264304097196\\
49.42	0.00201264299560015\\
49.43	0.00201264295020435\\
49.44	0.00201264290478455\\
49.45	0.00201264285934072\\
49.46	0.00201264281387285\\
49.47	0.00201264276838092\\
49.48	0.00201264272286492\\
49.49	0.00201264267732481\\
49.5	0.00201264263176059\\
49.51	0.00201264258617223\\
49.52	0.00201264254055972\\
49.53	0.00201264249492304\\
49.54	0.00201264244926216\\
49.55	0.00201264240357708\\
49.56	0.00201264235786777\\
49.57	0.00201264231213421\\
49.58	0.00201264226637639\\
49.59	0.00201264222059428\\
49.6	0.00201264217478787\\
49.61	0.00201264212895714\\
49.62	0.00201264208310207\\
49.63	0.00201264203722264\\
49.64	0.00201264199131883\\
49.65	0.00201264194539062\\
49.66	0.002012641899438\\
49.67	0.00201264185346095\\
49.68	0.00201264180745943\\
49.69	0.00201264176143345\\
49.7	0.00201264171538297\\
49.71	0.00201264166930798\\
49.72	0.00201264162320846\\
49.73	0.0020126415770844\\
49.74	0.00201264153093576\\
49.75	0.00201264148476254\\
49.76	0.00201264143856471\\
49.77	0.00201264139234225\\
49.78	0.00201264134609515\\
49.79	0.00201264129982338\\
49.8	0.00201264125352693\\
49.81	0.00201264120720578\\
49.82	0.0020126411608599\\
49.83	0.00201264111448928\\
49.84	0.0020126410680939\\
49.85	0.00201264102167374\\
49.86	0.00201264097522878\\
49.87	0.002012640928759\\
49.88	0.00201264088226438\\
49.89	0.0020126408357449\\
49.9	0.00201264078920054\\
49.91	0.00201264074263128\\
49.92	0.00201264069603711\\
49.93	0.002012640649418\\
49.94	0.00201264060277393\\
49.95	0.00201264055610488\\
49.96	0.00201264050941084\\
49.97	0.00201264046269178\\
49.98	0.00201264041594769\\
49.99	0.00201264036917854\\
50	0.00201264032238431\\
50.01	0.00201264027556499\\
50.02	0.00201264022872055\\
50.03	0.00201264018185098\\
50.04	0.00201264013495625\\
50.05	0.00201264008803635\\
50.06	0.00201264004109125\\
50.07	0.00201263999412093\\
50.08	0.00201263994712538\\
50.09	0.00201263990010457\\
50.1	0.00201263985305849\\
50.11	0.00201263980598711\\
50.12	0.00201263975889042\\
50.13	0.00201263971176838\\
50.14	0.002012639664621\\
50.15	0.00201263961744823\\
50.16	0.00201263957025006\\
50.17	0.00201263952302648\\
50.18	0.00201263947577746\\
50.19	0.00201263942850298\\
50.2	0.00201263938120303\\
50.21	0.00201263933387757\\
50.22	0.0020126392865266\\
50.23	0.00201263923915008\\
50.24	0.002012639191748\\
50.25	0.00201263914432034\\
50.26	0.00201263909686708\\
50.27	0.00201263904938819\\
50.28	0.00201263900188367\\
50.29	0.00201263895435347\\
50.3	0.0020126389067976\\
50.31	0.00201263885921601\\
50.32	0.0020126388116087\\
50.33	0.00201263876397565\\
50.34	0.00201263871631682\\
50.35	0.0020126386686322\\
50.36	0.00201263862092178\\
50.37	0.00201263857318553\\
50.38	0.00201263852542342\\
50.39	0.00201263847763544\\
50.4	0.00201263842982156\\
50.41	0.00201263838198177\\
50.42	0.00201263833411605\\
50.43	0.00201263828622436\\
50.44	0.0020126382383067\\
50.45	0.00201263819036304\\
50.46	0.00201263814239336\\
50.47	0.00201263809439764\\
50.48	0.00201263804637585\\
50.49	0.00201263799832798\\
50.5	0.002012637950254\\
50.51	0.0020126379021539\\
50.52	0.00201263785402764\\
50.53	0.00201263780587522\\
50.54	0.0020126377576966\\
50.55	0.00201263770949177\\
50.56	0.00201263766126071\\
50.57	0.00201263761300339\\
50.58	0.00201263756471979\\
50.59	0.00201263751640989\\
50.6	0.00201263746807367\\
50.61	0.00201263741971111\\
50.62	0.00201263737132218\\
50.63	0.00201263732290687\\
50.64	0.00201263727446515\\
50.65	0.002012637225997\\
50.66	0.0020126371775024\\
50.67	0.00201263712898132\\
50.68	0.00201263708043375\\
50.69	0.00201263703185966\\
50.7	0.00201263698325903\\
50.71	0.00201263693463183\\
50.72	0.00201263688597806\\
50.73	0.00201263683729768\\
50.74	0.00201263678859067\\
50.75	0.00201263673985701\\
50.76	0.00201263669109668\\
50.77	0.00201263664230965\\
50.78	0.00201263659349591\\
50.79	0.00201263654465543\\
50.8	0.00201263649578819\\
50.81	0.00201263644689416\\
50.82	0.00201263639797333\\
50.83	0.00201263634902567\\
50.84	0.00201263630005115\\
50.85	0.00201263625104977\\
50.86	0.00201263620202149\\
50.87	0.00201263615296629\\
50.88	0.00201263610388415\\
50.89	0.00201263605477504\\
50.9	0.00201263600563895\\
50.91	0.00201263595647585\\
50.92	0.00201263590728572\\
50.93	0.00201263585806853\\
50.94	0.00201263580882427\\
50.95	0.0020126357595529\\
50.96	0.00201263571025442\\
50.97	0.00201263566092878\\
50.98	0.00201263561157598\\
50.99	0.00201263556219599\\
51	0.00201263551278878\\
51.01	0.00201263546335433\\
51.02	0.00201263541389263\\
51.03	0.00201263536440363\\
51.04	0.00201263531488734\\
51.05	0.00201263526534371\\
51.06	0.00201263521577272\\
51.07	0.00201263516617436\\
51.08	0.0020126351165486\\
51.09	0.00201263506689542\\
51.1	0.00201263501721479\\
51.11	0.00201263496750669\\
51.12	0.0020126349177711\\
51.13	0.00201263486800798\\
51.14	0.00201263481821733\\
51.15	0.00201263476839912\\
51.16	0.00201263471855331\\
51.17	0.0020126346686799\\
51.18	0.00201263461877885\\
51.19	0.00201263456885014\\
51.2	0.00201263451889374\\
51.21	0.00201263446890965\\
51.22	0.00201263441889782\\
51.23	0.00201263436885824\\
51.24	0.00201263431879088\\
51.25	0.00201263426869572\\
51.26	0.00201263421857273\\
51.27	0.0020126341684219\\
51.28	0.00201263411824319\\
51.29	0.00201263406803658\\
51.3	0.00201263401780205\\
51.31	0.00201263396753958\\
51.32	0.00201263391724914\\
51.33	0.0020126338669307\\
51.34	0.00201263381658424\\
51.35	0.00201263376620974\\
51.36	0.00201263371580717\\
51.37	0.00201263366537652\\
51.38	0.00201263361491774\\
51.39	0.00201263356443083\\
51.4	0.00201263351391575\\
51.41	0.00201263346337248\\
51.42	0.002012633412801\\
51.43	0.00201263336220128\\
51.44	0.0020126333115733\\
51.45	0.00201263326091703\\
51.46	0.00201263321023244\\
51.47	0.00201263315951952\\
51.48	0.00201263310877824\\
51.49	0.00201263305800858\\
51.5	0.00201263300721049\\
51.51	0.00201263295638398\\
51.52	0.002012632905529\\
51.53	0.00201263285464554\\
51.54	0.00201263280373356\\
51.55	0.00201263275279305\\
51.56	0.00201263270182398\\
51.57	0.00201263265082632\\
51.58	0.00201263259980005\\
51.59	0.00201263254874514\\
51.6	0.00201263249766157\\
51.61	0.00201263244654931\\
51.62	0.00201263239540834\\
51.63	0.00201263234423863\\
51.64	0.00201263229304016\\
51.65	0.0020126322418129\\
51.66	0.00201263219055682\\
51.67	0.0020126321392719\\
51.68	0.00201263208795812\\
51.69	0.00201263203661545\\
51.7	0.00201263198524386\\
51.71	0.00201263193384332\\
51.72	0.00201263188241382\\
51.73	0.00201263183095532\\
51.74	0.00201263177946781\\
51.75	0.00201263172795125\\
51.76	0.00201263167640561\\
51.77	0.00201263162483088\\
51.78	0.00201263157322703\\
51.79	0.00201263152159402\\
51.8	0.00201263146993184\\
51.81	0.00201263141824046\\
51.82	0.00201263136651984\\
51.83	0.00201263131476998\\
51.84	0.00201263126299083\\
51.85	0.00201263121118238\\
51.86	0.00201263115934459\\
51.87	0.00201263110747745\\
51.88	0.00201263105558092\\
51.89	0.00201263100365497\\
51.9	0.00201263095169959\\
51.91	0.00201263089971474\\
51.92	0.0020126308477004\\
51.93	0.00201263079565654\\
51.94	0.00201263074358313\\
51.95	0.00201263069148016\\
51.96	0.00201263063934758\\
51.97	0.00201263058718538\\
51.98	0.00201263053499352\\
51.99	0.00201263048277199\\
52	0.00201263043052075\\
52.01	0.00201263037823978\\
52.02	0.00201263032592905\\
52.03	0.00201263027358852\\
52.04	0.00201263022121819\\
52.05	0.00201263016881801\\
52.06	0.00201263011638797\\
52.07	0.00201263006392803\\
52.08	0.00201263001143816\\
52.09	0.00201262995891835\\
52.1	0.00201262990636855\\
52.11	0.00201262985378875\\
52.12	0.00201262980117892\\
52.13	0.00201262974853903\\
52.14	0.00201262969586905\\
52.15	0.00201262964316895\\
52.16	0.00201262959043871\\
52.17	0.0020126295376783\\
52.18	0.00201262948488769\\
52.19	0.00201262943206685\\
52.2	0.00201262937921576\\
52.21	0.00201262932633439\\
52.22	0.00201262927342271\\
52.23	0.00201262922048069\\
52.24	0.0020126291675083\\
52.25	0.00201262911450552\\
52.26	0.00201262906147232\\
52.27	0.00201262900840867\\
52.28	0.00201262895531453\\
52.29	0.0020126289021899\\
52.3	0.00201262884903472\\
52.31	0.00201262879584898\\
52.32	0.00201262874263266\\
52.33	0.00201262868938571\\
52.34	0.00201262863610811\\
52.35	0.00201262858279983\\
52.36	0.00201262852946085\\
52.37	0.00201262847609113\\
52.38	0.00201262842269065\\
52.39	0.00201262836925938\\
52.4	0.00201262831579729\\
52.41	0.00201262826230435\\
52.42	0.00201262820878052\\
52.43	0.0020126281552258\\
52.44	0.00201262810164013\\
52.45	0.0020126280480235\\
52.46	0.00201262799437588\\
52.47	0.00201262794069723\\
52.48	0.00201262788698753\\
52.49	0.00201262783324675\\
52.5	0.00201262777947486\\
52.51	0.00201262772567182\\
52.52	0.00201262767183762\\
52.53	0.00201262761797222\\
52.54	0.00201262756407559\\
52.55	0.0020126275101477\\
52.56	0.00201262745618852\\
52.57	0.00201262740219803\\
52.58	0.00201262734817618\\
52.59	0.00201262729412297\\
52.6	0.00201262724003834\\
52.61	0.00201262718592228\\
52.62	0.00201262713177476\\
52.63	0.00201262707759573\\
52.64	0.00201262702338518\\
52.65	0.00201262696914308\\
52.66	0.00201262691486939\\
52.67	0.00201262686056408\\
52.68	0.00201262680622713\\
52.69	0.0020126267518585\\
52.7	0.00201262669745816\\
52.71	0.00201262664302609\\
52.72	0.00201262658856225\\
52.73	0.00201262653406661\\
52.74	0.00201262647953914\\
52.75	0.00201262642497981\\
52.76	0.00201262637038859\\
52.77	0.00201262631576546\\
52.78	0.00201262626111037\\
52.79	0.0020126262064233\\
52.8	0.00201262615170422\\
52.81	0.0020126260969531\\
52.82	0.0020126260421699\\
52.83	0.00201262598735459\\
52.84	0.00201262593250716\\
52.85	0.00201262587762755\\
52.86	0.00201262582271574\\
52.87	0.00201262576777171\\
52.88	0.00201262571279541\\
52.89	0.00201262565778683\\
52.9	0.00201262560274592\\
52.91	0.00201262554767265\\
52.92	0.002012625492567\\
52.93	0.00201262543742893\\
52.94	0.00201262538225842\\
52.95	0.00201262532705542\\
52.96	0.00201262527181991\\
52.97	0.00201262521655186\\
52.98	0.00201262516125122\\
52.99	0.00201262510591799\\
53	0.00201262505055211\\
53.01	0.00201262499515356\\
53.02	0.00201262493972231\\
53.03	0.00201262488425832\\
53.04	0.00201262482876157\\
53.05	0.00201262477323201\\
53.06	0.00201262471766963\\
53.07	0.00201262466207438\\
53.08	0.00201262460644623\\
53.09	0.00201262455078516\\
53.1	0.00201262449509112\\
53.11	0.00201262443936409\\
53.12	0.00201262438360403\\
53.13	0.00201262432781092\\
53.14	0.00201262427198471\\
53.15	0.00201262421612538\\
53.16	0.00201262416023289\\
53.17	0.00201262410430721\\
53.18	0.00201262404834831\\
53.19	0.00201262399235615\\
53.2	0.00201262393633071\\
53.21	0.00201262388027194\\
53.22	0.00201262382417982\\
53.23	0.00201262376805431\\
53.24	0.00201262371189537\\
53.25	0.00201262365570299\\
53.26	0.00201262359947711\\
53.27	0.00201262354321772\\
53.28	0.00201262348692476\\
53.29	0.00201262343059823\\
53.3	0.00201262337423806\\
53.31	0.00201262331784425\\
53.32	0.00201262326141674\\
53.33	0.00201262320495551\\
53.34	0.00201262314846052\\
53.35	0.00201262309193175\\
53.36	0.00201262303536914\\
53.37	0.00201262297877268\\
53.38	0.00201262292214233\\
53.39	0.00201262286547804\\
53.4	0.0020126228087798\\
53.41	0.00201262275204756\\
53.42	0.00201262269528129\\
53.43	0.00201262263848096\\
53.44	0.00201262258164653\\
53.45	0.00201262252477796\\
53.46	0.00201262246787523\\
53.47	0.0020126224109383\\
53.48	0.00201262235396712\\
53.49	0.00201262229696168\\
53.5	0.00201262223992193\\
53.51	0.00201262218284783\\
53.52	0.00201262212573937\\
53.53	0.00201262206859649\\
53.54	0.00201262201141916\\
53.55	0.00201262195420735\\
53.56	0.00201262189696102\\
53.57	0.00201262183968015\\
53.58	0.00201262178236468\\
53.59	0.00201262172501459\\
53.6	0.00201262166762984\\
53.61	0.0020126216102104\\
53.62	0.00201262155275623\\
53.63	0.0020126214952673\\
53.64	0.00201262143774356\\
53.65	0.00201262138018499\\
53.66	0.00201262132259154\\
53.67	0.00201262126496319\\
53.68	0.00201262120729989\\
53.69	0.00201262114960161\\
53.7	0.00201262109186832\\
53.71	0.00201262103409997\\
53.72	0.00201262097629654\\
53.73	0.00201262091845798\\
53.74	0.00201262086058425\\
53.75	0.00201262080267534\\
53.76	0.00201262074473118\\
53.77	0.00201262068675176\\
53.78	0.00201262062873702\\
53.79	0.00201262057068695\\
53.8	0.00201262051260149\\
53.81	0.00201262045448061\\
53.82	0.00201262039632428\\
53.83	0.00201262033813246\\
53.84	0.00201262027990511\\
53.85	0.00201262022164219\\
53.86	0.00201262016334367\\
53.87	0.00201262010500951\\
53.88	0.00201262004663967\\
53.89	0.00201261998823412\\
53.9	0.00201261992979282\\
53.91	0.00201261987131572\\
53.92	0.0020126198128028\\
53.93	0.00201261975425401\\
53.94	0.00201261969566932\\
53.95	0.00201261963704869\\
53.96	0.00201261957839208\\
53.97	0.00201261951969946\\
53.98	0.00201261946097078\\
53.99	0.002012619402206\\
54	0.0020126193434051\\
54.01	0.00201261928456803\\
54.02	0.00201261922569475\\
54.03	0.00201261916678522\\
54.04	0.00201261910783942\\
54.05	0.00201261904885728\\
54.06	0.00201261898983879\\
54.07	0.0020126189307839\\
54.08	0.00201261887169257\\
54.09	0.00201261881256477\\
54.1	0.00201261875340045\\
54.11	0.00201261869419957\\
54.12	0.0020126186349621\\
54.13	0.00201261857568799\\
54.14	0.00201261851637722\\
54.15	0.00201261845702973\\
54.16	0.00201261839764549\\
54.17	0.00201261833822446\\
54.18	0.0020126182787666\\
54.19	0.00201261821927188\\
54.2	0.00201261815974024\\
54.21	0.00201261810017166\\
54.22	0.00201261804056609\\
54.23	0.00201261798092349\\
54.24	0.00201261792124383\\
54.25	0.00201261786152705\\
54.26	0.00201261780177313\\
54.27	0.00201261774198202\\
54.28	0.00201261768215369\\
54.29	0.00201261762228808\\
54.3	0.00201261756238517\\
54.31	0.0020126175024449\\
54.32	0.00201261744246725\\
54.33	0.00201261738245217\\
54.34	0.00201261732239962\\
54.35	0.00201261726230956\\
54.36	0.00201261720218194\\
54.37	0.00201261714201673\\
54.38	0.00201261708181389\\
54.39	0.00201261702157337\\
54.4	0.00201261696129514\\
54.41	0.00201261690097915\\
54.42	0.00201261684062536\\
54.43	0.00201261678023373\\
54.44	0.00201261671980422\\
54.45	0.00201261665933678\\
54.46	0.00201261659883138\\
54.47	0.00201261653828798\\
54.48	0.00201261647770653\\
54.49	0.00201261641708699\\
54.5	0.00201261635642931\\
54.51	0.00201261629573347\\
54.52	0.0020126162349994\\
54.53	0.00201261617422708\\
54.54	0.00201261611341646\\
54.55	0.0020126160525675\\
54.56	0.00201261599168016\\
54.57	0.00201261593075438\\
54.58	0.00201261586979014\\
54.59	0.00201261580878738\\
54.6	0.00201261574774607\\
54.61	0.00201261568666616\\
54.62	0.00201261562554761\\
54.63	0.00201261556439038\\
54.64	0.00201261550319442\\
54.65	0.00201261544195969\\
54.66	0.00201261538068615\\
54.67	0.00201261531937375\\
54.68	0.00201261525802245\\
54.69	0.00201261519663221\\
54.7	0.00201261513520298\\
54.71	0.00201261507373472\\
54.72	0.00201261501222739\\
54.73	0.00201261495068094\\
54.74	0.00201261488909533\\
54.75	0.00201261482747051\\
54.76	0.00201261476580644\\
54.77	0.00201261470410308\\
54.78	0.00201261464236038\\
54.79	0.0020126145805783\\
54.8	0.00201261451875679\\
54.81	0.00201261445689581\\
54.82	0.00201261439499532\\
54.83	0.00201261433305526\\
54.84	0.0020126142710756\\
54.85	0.00201261420905629\\
54.86	0.00201261414699729\\
54.87	0.00201261408489854\\
54.88	0.00201261402276001\\
54.89	0.00201261396058165\\
54.9	0.00201261389836341\\
54.91	0.00201261383610525\\
54.92	0.00201261377380713\\
54.93	0.00201261371146899\\
54.94	0.0020126136490908\\
54.95	0.0020126135866725\\
54.96	0.00201261352421405\\
54.97	0.00201261346171541\\
54.98	0.00201261339917653\\
54.99	0.00201261333659736\\
55	0.00201261327397786\\
55.01	0.00201261321131798\\
55.02	0.00201261314861767\\
55.03	0.00201261308587689\\
55.04	0.00201261302309559\\
55.05	0.00201261296027373\\
55.06	0.00201261289741125\\
55.07	0.00201261283450811\\
55.08	0.00201261277156427\\
55.09	0.00201261270857968\\
55.1	0.00201261264555428\\
55.11	0.00201261258248804\\
55.12	0.00201261251938091\\
55.13	0.00201261245623283\\
55.14	0.00201261239304376\\
55.15	0.00201261232981366\\
55.16	0.00201261226654247\\
55.17	0.00201261220323015\\
55.18	0.00201261213987665\\
55.19	0.00201261207648192\\
55.2	0.00201261201304591\\
55.21	0.00201261194956858\\
55.22	0.00201261188604988\\
55.23	0.00201261182248976\\
55.24	0.00201261175888817\\
55.25	0.00201261169524506\\
55.26	0.00201261163156038\\
55.27	0.00201261156783409\\
55.28	0.00201261150406614\\
55.29	0.00201261144025647\\
55.3	0.00201261137640504\\
55.31	0.0020126113125118\\
55.32	0.00201261124857669\\
55.33	0.00201261118459968\\
55.34	0.00201261112058071\\
55.35	0.00201261105651973\\
55.36	0.0020126109924167\\
55.37	0.00201261092827155\\
55.38	0.00201261086408425\\
55.39	0.00201261079985474\\
55.4	0.00201261073558297\\
55.41	0.0020126106712689\\
55.42	0.00201261060691246\\
55.43	0.00201261054251362\\
55.44	0.00201261047807232\\
55.45	0.00201261041358852\\
55.46	0.00201261034906215\\
55.47	0.00201261028449317\\
55.48	0.00201261021988153\\
55.49	0.00201261015522717\\
55.5	0.00201261009053006\\
55.51	0.00201261002579013\\
55.52	0.00201260996100733\\
55.53	0.00201260989618162\\
55.54	0.00201260983131293\\
55.55	0.00201260976640123\\
55.56	0.00201260970144646\\
55.57	0.00201260963644856\\
55.58	0.00201260957140748\\
55.59	0.00201260950632318\\
55.6	0.0020126094411956\\
55.61	0.00201260937602469\\
55.62	0.00201260931081039\\
55.63	0.00201260924555266\\
55.64	0.00201260918025144\\
55.65	0.00201260911490667\\
55.66	0.00201260904951831\\
55.67	0.00201260898408631\\
55.68	0.0020126089186106\\
55.69	0.00201260885309114\\
55.7	0.00201260878752788\\
55.71	0.00201260872192075\\
55.72	0.00201260865626971\\
55.73	0.0020126085905747\\
55.74	0.00201260852483568\\
55.75	0.00201260845905258\\
55.76	0.00201260839322535\\
55.77	0.00201260832735394\\
55.78	0.00201260826143829\\
55.79	0.00201260819547835\\
55.8	0.00201260812947407\\
55.81	0.00201260806342539\\
55.82	0.00201260799733225\\
55.83	0.00201260793119461\\
55.84	0.0020126078650124\\
55.85	0.00201260779878558\\
55.86	0.00201260773251408\\
55.87	0.00201260766619786\\
55.88	0.00201260759983685\\
55.89	0.00201260753343101\\
55.9	0.00201260746698027\\
55.91	0.00201260740048458\\
55.92	0.00201260733394388\\
55.93	0.00201260726735813\\
55.94	0.00201260720072725\\
55.95	0.00201260713405121\\
55.96	0.00201260706732993\\
55.97	0.00201260700056337\\
55.98	0.00201260693375147\\
55.99	0.00201260686689417\\
56	0.00201260679999142\\
56.01	0.00201260673304315\\
56.02	0.00201260666604931\\
56.03	0.00201260659900985\\
56.04	0.00201260653192471\\
56.05	0.00201260646479382\\
56.06	0.00201260639761714\\
56.07	0.00201260633039461\\
56.08	0.00201260626312616\\
56.09	0.00201260619581174\\
56.1	0.00201260612845129\\
56.11	0.00201260606104476\\
56.12	0.00201260599359208\\
56.13	0.0020126059260932\\
56.14	0.00201260585854806\\
56.15	0.0020126057909566\\
56.16	0.00201260572331876\\
56.17	0.00201260565563449\\
56.18	0.00201260558790372\\
56.19	0.0020126055201264\\
56.2	0.00201260545230246\\
56.21	0.00201260538443185\\
56.22	0.00201260531651451\\
56.23	0.00201260524855038\\
56.24	0.00201260518053939\\
56.25	0.0020126051124815\\
56.26	0.00201260504437663\\
56.27	0.00201260497622474\\
56.28	0.00201260490802575\\
56.29	0.00201260483977961\\
56.3	0.00201260477148626\\
56.31	0.00201260470314563\\
56.32	0.00201260463475768\\
56.33	0.00201260456632233\\
56.34	0.00201260449783953\\
56.35	0.00201260442930921\\
56.36	0.00201260436073131\\
56.37	0.00201260429210578\\
56.38	0.00201260422343254\\
56.39	0.00201260415471155\\
56.4	0.00201260408594273\\
56.41	0.00201260401712603\\
56.42	0.00201260394826138\\
56.43	0.00201260387934872\\
56.44	0.00201260381038799\\
56.45	0.00201260374137912\\
56.46	0.00201260367232206\\
56.47	0.00201260360321674\\
56.48	0.0020126035340631\\
56.49	0.00201260346486108\\
56.5	0.0020126033956106\\
56.51	0.00201260332631162\\
56.52	0.00201260325696406\\
56.53	0.00201260318756786\\
56.54	0.00201260311812297\\
56.55	0.0020126030486293\\
56.56	0.00201260297908681\\
56.57	0.00201260290949542\\
56.58	0.00201260283985508\\
56.59	0.00201260277016571\\
56.6	0.00201260270042726\\
56.61	0.00201260263063965\\
56.62	0.00201260256080283\\
56.63	0.00201260249091673\\
56.64	0.00201260242098128\\
56.65	0.00201260235099642\\
56.66	0.00201260228096209\\
56.67	0.00201260221087821\\
56.68	0.00201260214074472\\
56.69	0.00201260207056156\\
56.7	0.00201260200032866\\
56.71	0.00201260193004595\\
56.72	0.00201260185971337\\
56.73	0.00201260178933086\\
56.74	0.00201260171889833\\
56.75	0.00201260164841574\\
56.76	0.002012601577883\\
56.77	0.00201260150730006\\
56.78	0.00201260143666685\\
56.79	0.0020126013659833\\
56.8	0.00201260129524933\\
56.81	0.0020126012244649\\
56.82	0.00201260115362992\\
56.83	0.00201260108274432\\
56.84	0.00201260101180805\\
56.85	0.00201260094082103\\
56.86	0.00201260086978319\\
56.87	0.00201260079869447\\
56.88	0.0020126007275548\\
56.89	0.0020126006563641\\
56.9	0.00201260058512231\\
56.91	0.00201260051382936\\
56.92	0.00201260044248518\\
56.93	0.0020126003710897\\
56.94	0.00201260029964285\\
56.95	0.00201260022814456\\
56.96	0.00201260015659476\\
56.97	0.00201260008499339\\
56.98	0.00201260001334036\\
56.99	0.00201259994163561\\
57	0.00201259986987907\\
57.01	0.00201259979807067\\
57.02	0.00201259972621034\\
57.03	0.002012599654298\\
57.04	0.00201259958233359\\
57.05	0.00201259951031703\\
57.06	0.00201259943824825\\
57.07	0.00201259936612718\\
57.08	0.00201259929395375\\
57.09	0.00201259922172789\\
57.1	0.00201259914944951\\
57.11	0.00201259907711856\\
57.12	0.00201259900473496\\
57.13	0.00201259893229863\\
57.14	0.0020125988598095\\
57.15	0.0020125987872675\\
57.16	0.00201259871467256\\
57.17	0.0020125986420246\\
57.18	0.00201259856932355\\
57.19	0.00201259849656933\\
57.2	0.00201259842376187\\
57.21	0.0020125983509011\\
57.22	0.00201259827798694\\
57.23	0.00201259820501932\\
57.24	0.00201259813199815\\
57.25	0.00201259805892338\\
57.26	0.00201259798579492\\
57.27	0.00201259791261269\\
57.28	0.00201259783937663\\
57.29	0.00201259776608665\\
57.3	0.00201259769274268\\
57.31	0.00201259761934464\\
57.32	0.00201259754589247\\
57.33	0.00201259747238607\\
57.34	0.00201259739882538\\
57.35	0.00201259732521031\\
57.36	0.0020125972515408\\
57.37	0.00201259717781676\\
57.38	0.00201259710403811\\
57.39	0.00201259703020478\\
57.4	0.0020125969563167\\
57.41	0.00201259688237377\\
57.42	0.00201259680837594\\
57.43	0.00201259673432311\\
57.44	0.0020125966602152\\
57.45	0.00201259658605215\\
57.46	0.00201259651183386\\
57.47	0.00201259643756027\\
57.48	0.00201259636323129\\
57.49	0.00201259628884685\\
57.5	0.00201259621440685\\
57.51	0.00201259613991123\\
57.52	0.0020125960653599\\
57.53	0.00201259599075279\\
57.54	0.00201259591608981\\
57.55	0.00201259584137088\\
57.56	0.00201259576659592\\
57.57	0.00201259569176485\\
57.58	0.00201259561687759\\
57.59	0.00201259554193405\\
57.6	0.00201259546693417\\
57.61	0.00201259539187784\\
57.62	0.002012595316765\\
57.63	0.00201259524159556\\
57.64	0.00201259516636943\\
57.65	0.00201259509108654\\
57.66	0.0020125950157468\\
57.67	0.00201259494035013\\
57.68	0.00201259486489644\\
57.69	0.00201259478938566\\
57.7	0.00201259471381769\\
57.71	0.00201259463819246\\
57.72	0.00201259456250987\\
57.73	0.00201259448676985\\
57.74	0.00201259441097232\\
57.75	0.00201259433511717\\
57.76	0.00201259425920434\\
57.77	0.00201259418323374\\
57.78	0.00201259410720527\\
57.79	0.00201259403111886\\
57.8	0.00201259395497442\\
57.81	0.00201259387877186\\
57.82	0.00201259380251109\\
57.83	0.00201259372619204\\
57.84	0.00201259364981461\\
57.85	0.00201259357337871\\
57.86	0.00201259349688427\\
57.87	0.00201259342033118\\
57.88	0.00201259334371937\\
57.89	0.00201259326704874\\
57.9	0.00201259319031921\\
57.91	0.00201259311353069\\
57.92	0.00201259303668309\\
57.93	0.00201259295977632\\
57.94	0.00201259288281029\\
57.95	0.00201259280578492\\
57.96	0.00201259272870011\\
57.97	0.00201259265155577\\
57.98	0.00201259257435182\\
57.99	0.00201259249708816\\
58	0.00201259241976471\\
58.01	0.00201259234238136\\
58.02	0.00201259226493804\\
58.03	0.00201259218743465\\
58.04	0.0020125921098711\\
58.05	0.00201259203224729\\
58.06	0.00201259195456314\\
58.07	0.00201259187681855\\
58.08	0.00201259179901343\\
58.09	0.00201259172114769\\
58.1	0.00201259164322123\\
58.11	0.00201259156523397\\
58.12	0.0020125914871858\\
58.13	0.00201259140907664\\
58.14	0.00201259133090638\\
58.15	0.00201259125267495\\
58.16	0.00201259117438223\\
58.17	0.00201259109602814\\
58.18	0.00201259101761259\\
58.19	0.00201259093913547\\
58.2	0.00201259086059669\\
58.21	0.00201259078199616\\
58.22	0.00201259070333378\\
58.23	0.00201259062460945\\
58.24	0.00201259054582307\\
58.25	0.00201259046697456\\
58.26	0.00201259038806381\\
58.27	0.00201259030909072\\
58.28	0.0020125902300552\\
58.29	0.00201259015095716\\
58.3	0.00201259007179648\\
58.31	0.00201258999257308\\
58.32	0.00201258991328685\\
58.33	0.00201258983393769\\
58.34	0.00201258975452552\\
58.35	0.00201258967505021\\
58.36	0.00201258959551169\\
58.37	0.00201258951590984\\
58.38	0.00201258943624457\\
58.39	0.00201258935651577\\
58.4	0.00201258927672335\\
58.41	0.0020125891968672\\
58.42	0.00201258911694723\\
58.43	0.00201258903696332\\
58.44	0.00201258895691538\\
58.45	0.00201258887680331\\
58.46	0.00201258879662699\\
58.47	0.00201258871638634\\
58.48	0.00201258863608125\\
58.49	0.0020125885557116\\
58.5	0.00201258847527731\\
58.51	0.00201258839477826\\
58.52	0.00201258831421435\\
58.53	0.00201258823358548\\
58.54	0.00201258815289154\\
58.55	0.00201258807213242\\
58.56	0.00201258799130802\\
58.57	0.00201258791041823\\
58.58	0.00201258782946295\\
58.59	0.00201258774844207\\
58.6	0.00201258766735548\\
58.61	0.00201258758620308\\
58.62	0.00201258750498475\\
58.63	0.0020125874237004\\
58.64	0.00201258734234991\\
58.65	0.00201258726093317\\
58.66	0.00201258717945007\\
58.67	0.00201258709790052\\
58.68	0.00201258701628438\\
58.69	0.00201258693460156\\
58.7	0.00201258685285195\\
58.71	0.00201258677103543\\
58.72	0.0020125866891519\\
58.73	0.00201258660720124\\
58.74	0.00201258652518334\\
58.75	0.00201258644309809\\
58.76	0.00201258636094538\\
58.77	0.00201258627872509\\
58.78	0.00201258619643712\\
58.79	0.00201258611408134\\
58.8	0.00201258603165765\\
58.81	0.00201258594916593\\
58.82	0.00201258586660607\\
58.83	0.00201258578397794\\
58.84	0.00201258570128145\\
58.85	0.00201258561851647\\
58.86	0.00201258553568288\\
58.87	0.00201258545278057\\
58.88	0.00201258536980943\\
58.89	0.00201258528676933\\
58.9	0.00201258520366017\\
58.91	0.00201258512048181\\
58.92	0.00201258503723415\\
58.93	0.00201258495391707\\
58.94	0.00201258487053044\\
58.95	0.00201258478707415\\
58.96	0.00201258470354808\\
58.97	0.00201258461995211\\
58.98	0.00201258453628612\\
58.99	0.00201258445254998\\
59	0.00201258436874359\\
59.01	0.00201258428486681\\
59.02	0.00201258420091952\\
59.03	0.00201258411690161\\
59.04	0.00201258403281295\\
59.05	0.00201258394865341\\
59.06	0.00201258386442288\\
59.07	0.00201258378012122\\
59.08	0.00201258369574833\\
59.09	0.00201258361130406\\
59.1	0.00201258352678831\\
59.11	0.00201258344220093\\
59.12	0.00201258335754181\\
59.13	0.00201258327281082\\
59.14	0.00201258318800783\\
59.15	0.00201258310313271\\
59.16	0.00201258301818535\\
59.17	0.00201258293316561\\
59.18	0.00201258284807336\\
59.19	0.00201258276290847\\
59.2	0.00201258267767082\\
59.21	0.00201258259236027\\
59.22	0.0020125825069767\\
59.23	0.00201258242151998\\
59.24	0.00201258233598997\\
59.25	0.00201258225038654\\
59.26	0.00201258216470956\\
59.27	0.0020125820789589\\
59.28	0.00201258199313443\\
59.29	0.00201258190723602\\
59.3	0.00201258182126352\\
59.31	0.00201258173521681\\
59.32	0.00201258164909575\\
59.33	0.00201258156290021\\
59.34	0.00201258147663005\\
59.35	0.00201258139028513\\
59.36	0.00201258130386533\\
59.37	0.00201258121737049\\
59.38	0.0020125811308005\\
59.39	0.0020125810441552\\
59.4	0.00201258095743446\\
59.41	0.00201258087063814\\
59.42	0.0020125807837661\\
59.43	0.00201258069681821\\
59.44	0.00201258060979431\\
59.45	0.00201258052269428\\
59.46	0.00201258043551797\\
59.47	0.00201258034826524\\
59.48	0.00201258026093594\\
59.49	0.00201258017352994\\
59.5	0.00201258008604709\\
59.51	0.00201257999848724\\
59.52	0.00201257991085026\\
59.53	0.002012579823136\\
59.54	0.0020125797353443\\
59.55	0.00201257964747504\\
59.56	0.00201257955952806\\
59.57	0.00201257947150321\\
59.58	0.00201257938340035\\
59.59	0.00201257929521933\\
59.6	0.00201257920695999\\
59.61	0.0020125791186222\\
59.62	0.00201257903020581\\
59.63	0.00201257894171065\\
59.64	0.00201257885313658\\
59.65	0.00201257876448346\\
59.66	0.00201257867575112\\
59.67	0.00201257858693942\\
59.68	0.0020125784980482\\
59.69	0.00201257840907731\\
59.7	0.0020125783200266\\
59.71	0.00201257823089591\\
59.72	0.00201257814168509\\
59.73	0.00201257805239397\\
59.74	0.00201257796302242\\
59.75	0.00201257787357025\\
59.76	0.00201257778403733\\
59.77	0.00201257769442349\\
59.78	0.00201257760472857\\
59.79	0.00201257751495241\\
59.8	0.00201257742509486\\
59.81	0.00201257733515575\\
59.82	0.00201257724513492\\
59.83	0.00201257715503221\\
59.84	0.00201257706484746\\
59.85	0.0020125769745805\\
59.86	0.00201257688423117\\
59.87	0.00201257679379931\\
59.88	0.00201257670328475\\
59.89	0.00201257661268733\\
59.9	0.00201257652200687\\
59.91	0.00201257643124322\\
59.92	0.0020125763403962\\
59.93	0.00201257624946565\\
59.94	0.00201257615845139\\
59.95	0.00201257606735327\\
59.96	0.0020125759761711\\
59.97	0.00201257588490471\\
59.98	0.00201257579355395\\
59.99	0.00201257570211862\\
60	0.00201257561059856\\
60.01	0.0020125755189936\\
60.02	0.00201257542730356\\
60.03	0.00201257533552827\\
60.04	0.00201257524366754\\
60.05	0.00201257515172121\\
60.06	0.0020125750596891\\
60.07	0.00201257496757102\\
60.08	0.00201257487536681\\
60.09	0.00201257478307627\\
60.1	0.00201257469069923\\
60.11	0.00201257459823551\\
60.12	0.00201257450568493\\
60.13	0.00201257441304731\\
60.14	0.00201257432032245\\
60.15	0.00201257422751018\\
60.16	0.00201257413461031\\
60.17	0.00201257404162266\\
60.18	0.00201257394854704\\
60.19	0.00201257385538326\\
60.2	0.00201257376213114\\
60.21	0.00201257366879049\\
60.22	0.00201257357536111\\
60.23	0.00201257348184281\\
60.24	0.00201257338823542\\
60.25	0.00201257329453872\\
60.26	0.00201257320075254\\
60.27	0.00201257310687667\\
60.28	0.00201257301291093\\
60.29	0.00201257291885511\\
60.3	0.00201257282470902\\
60.31	0.00201257273047247\\
60.32	0.00201257263614525\\
60.33	0.00201257254172717\\
60.34	0.00201257244721803\\
60.35	0.00201257235261763\\
60.36	0.00201257225792575\\
60.37	0.00201257216314222\\
60.38	0.00201257206826681\\
60.39	0.00201257197329933\\
60.4	0.00201257187823957\\
60.41	0.00201257178308732\\
60.42	0.00201257168784239\\
60.43	0.00201257159250455\\
60.44	0.00201257149707361\\
60.45	0.00201257140154934\\
60.46	0.00201257130593155\\
60.47	0.00201257121022002\\
60.48	0.00201257111441454\\
60.49	0.00201257101851489\\
60.5	0.00201257092252086\\
60.51	0.00201257082643224\\
60.52	0.0020125707302488\\
60.53	0.00201257063397033\\
60.54	0.00201257053759662\\
60.55	0.00201257044112744\\
60.56	0.00201257034456257\\
60.57	0.00201257024790179\\
60.58	0.00201257015114488\\
60.59	0.00201257005429161\\
60.6	0.00201256995734177\\
60.61	0.00201256986029512\\
60.62	0.00201256976315144\\
60.63	0.0020125696659105\\
60.64	0.00201256956857207\\
60.65	0.00201256947113592\\
60.66	0.00201256937360183\\
60.67	0.00201256927596955\\
60.68	0.00201256917823886\\
60.69	0.00201256908040952\\
60.7	0.00201256898248129\\
60.71	0.00201256888445395\\
60.72	0.00201256878632724\\
60.73	0.00201256868810094\\
60.74	0.0020125685897748\\
60.75	0.00201256849134859\\
60.76	0.00201256839282205\\
60.77	0.00201256829419495\\
60.78	0.00201256819546704\\
60.79	0.00201256809663808\\
60.8	0.00201256799770782\\
60.81	0.002012567898676\\
60.82	0.00201256779954239\\
60.83	0.00201256770030673\\
60.84	0.00201256760096876\\
60.85	0.00201256750152824\\
60.86	0.00201256740198492\\
60.87	0.00201256730233852\\
60.88	0.0020125672025888\\
60.89	0.00201256710273551\\
60.9	0.00201256700277837\\
60.91	0.00201256690271713\\
60.92	0.00201256680255153\\
60.93	0.0020125667022813\\
60.94	0.00201256660190618\\
60.95	0.0020125665014259\\
60.96	0.0020125664008402\\
60.97	0.0020125663001488\\
60.98	0.00201256619935144\\
60.99	0.00201256609844784\\
61	0.00201256599743774\\
61.01	0.00201256589632085\\
61.02	0.0020125657950969\\
61.03	0.00201256569376561\\
61.04	0.0020125655923267\\
61.05	0.00201256549077991\\
61.06	0.00201256538912493\\
61.07	0.00201256528736149\\
61.08	0.00201256518548931\\
61.09	0.0020125650835081\\
61.1	0.00201256498141756\\
61.11	0.00201256487921742\\
61.12	0.00201256477690738\\
61.13	0.00201256467448715\\
61.14	0.00201256457195644\\
61.15	0.00201256446931495\\
61.16	0.00201256436656238\\
61.17	0.00201256426369843\\
61.18	0.00201256416072281\\
61.19	0.00201256405763522\\
61.2	0.00201256395443534\\
61.21	0.00201256385112288\\
61.22	0.00201256374769753\\
61.23	0.00201256364415898\\
61.24	0.00201256354050692\\
61.25	0.00201256343674104\\
61.26	0.00201256333286103\\
61.27	0.00201256322886657\\
61.28	0.00201256312475735\\
61.29	0.00201256302053304\\
61.3	0.00201256291619333\\
61.31	0.00201256281173789\\
61.32	0.00201256270716641\\
61.33	0.00201256260247855\\
61.34	0.00201256249767399\\
61.35	0.0020125623927524\\
61.36	0.00201256228771345\\
61.37	0.00201256218255681\\
61.38	0.00201256207728213\\
61.39	0.0020125619718891\\
61.4	0.00201256186637736\\
61.41	0.00201256176074658\\
61.42	0.00201256165499642\\
61.43	0.00201256154912652\\
61.44	0.00201256144313655\\
61.45	0.00201256133702617\\
61.46	0.00201256123079501\\
61.47	0.00201256112444272\\
61.48	0.00201256101796896\\
61.49	0.00201256091137336\\
61.5	0.00201256080465558\\
61.51	0.00201256069781524\\
61.52	0.00201256059085199\\
61.53	0.00201256048376546\\
61.54	0.0020125603765553\\
61.55	0.00201256026922112\\
61.56	0.00201256016176255\\
61.57	0.00201256005417924\\
61.58	0.0020125599464708\\
61.59	0.00201255983863685\\
61.6	0.00201255973067702\\
61.61	0.00201255962259092\\
61.62	0.00201255951437817\\
61.63	0.00201255940603839\\
61.64	0.00201255929757119\\
61.65	0.00201255918897618\\
61.66	0.00201255908025296\\
61.67	0.00201255897140114\\
61.68	0.00201255886242032\\
61.69	0.00201255875331011\\
61.7	0.0020125586440701\\
61.71	0.00201255853469989\\
61.72	0.00201255842519907\\
61.73	0.00201255831556724\\
61.74	0.00201255820580397\\
61.75	0.00201255809590886\\
61.76	0.00201255798588149\\
61.77	0.00201255787572145\\
61.78	0.0020125577654283\\
61.79	0.00201255765500163\\
61.8	0.00201255754444101\\
61.81	0.00201255743374601\\
61.82	0.0020125573229162\\
61.83	0.00201255721195114\\
61.84	0.00201255710085041\\
61.85	0.00201255698961355\\
61.86	0.00201255687824013\\
61.87	0.00201255676672969\\
61.88	0.00201255665508181\\
61.89	0.00201255654329601\\
61.9	0.00201255643137185\\
61.91	0.00201255631930888\\
61.92	0.00201255620710664\\
61.93	0.00201255609476465\\
61.94	0.00201255598228247\\
61.95	0.00201255586965962\\
61.96	0.00201255575689563\\
61.97	0.00201255564399003\\
61.98	0.00201255553094233\\
61.99	0.00201255541775208\\
62	0.00201255530441877\\
62.01	0.00201255519094193\\
62.02	0.00201255507732107\\
62.03	0.0020125549635557\\
62.04	0.00201255484964532\\
62.05	0.00201255473558943\\
62.06	0.00201255462138754\\
62.07	0.00201255450703914\\
62.08	0.00201255439254372\\
62.09	0.00201255427790078\\
62.1	0.0020125541631098\\
62.11	0.00201255404817027\\
62.12	0.00201255393308166\\
62.13	0.00201255381784345\\
62.14	0.00201255370245512\\
62.15	0.00201255358691614\\
62.16	0.00201255347122597\\
62.17	0.00201255335538407\\
62.18	0.00201255323938991\\
62.19	0.00201255312324294\\
62.2	0.00201255300694262\\
62.21	0.0020125528904884\\
62.22	0.00201255277387971\\
62.23	0.00201255265711601\\
62.24	0.00201255254019674\\
62.25	0.00201255242312132\\
62.26	0.00201255230588919\\
62.27	0.00201255218849977\\
62.28	0.0020125520709525\\
62.29	0.00201255195324679\\
62.3	0.00201255183538206\\
62.31	0.00201255171735772\\
62.32	0.00201255159917319\\
62.33	0.00201255148082785\\
62.34	0.00201255136232113\\
62.35	0.0020125512436524\\
62.36	0.00201255112482108\\
62.37	0.00201255100582654\\
62.38	0.00201255088666817\\
62.39	0.00201255076734536\\
62.4	0.00201255064785748\\
62.41	0.0020125505282039\\
62.42	0.00201255040838399\\
62.43	0.00201255028839713\\
62.44	0.00201255016824266\\
62.45	0.00201255004791994\\
62.46	0.00201254992742833\\
62.47	0.00201254980676718\\
62.48	0.00201254968593582\\
62.49	0.0020125495649336\\
62.5	0.00201254944375986\\
62.51	0.00201254932241391\\
62.52	0.00201254920089509\\
62.53	0.00201254907920272\\
62.54	0.00201254895733612\\
62.55	0.0020125488352946\\
62.56	0.00201254871307746\\
62.57	0.00201254859068401\\
62.58	0.00201254846811354\\
62.59	0.00201254834536536\\
62.6	0.00201254822243874\\
62.61	0.00201254809933298\\
62.62	0.00201254797604735\\
62.63	0.00201254785258112\\
62.64	0.00201254772893357\\
62.65	0.00201254760510396\\
62.66	0.00201254748109155\\
62.67	0.00201254735689559\\
62.68	0.00201254723251533\\
62.69	0.00201254710795002\\
62.7	0.0020125469831989\\
62.71	0.0020125468582612\\
62.72	0.00201254673313614\\
62.73	0.00201254660782296\\
62.74	0.00201254648232087\\
62.75	0.00201254635662908\\
62.76	0.0020125462307468\\
62.77	0.00201254610467323\\
62.78	0.00201254597840757\\
62.79	0.00201254585194901\\
62.8	0.00201254572529674\\
62.81	0.00201254559844993\\
62.82	0.00201254547140776\\
62.83	0.0020125453441694\\
62.84	0.00201254521673402\\
62.85	0.00201254508910075\\
62.86	0.00201254496126877\\
62.87	0.00201254483323722\\
62.88	0.00201254470500523\\
62.89	0.00201254457657193\\
62.9	0.00201254444793647\\
62.91	0.00201254431909795\\
62.92	0.00201254419005549\\
62.93	0.00201254406080821\\
62.94	0.0020125439313552\\
62.95	0.00201254380169557\\
62.96	0.0020125436718284\\
62.97	0.00201254354175278\\
62.98	0.00201254341146779\\
62.99	0.00201254328097249\\
63	0.00201254315026596\\
63.01	0.00201254301934725\\
63.02	0.00201254288821542\\
63.03	0.00201254275686951\\
63.04	0.00201254262530856\\
63.05	0.0020125424935316\\
63.06	0.00201254236153766\\
63.07	0.00201254222932576\\
63.08	0.00201254209689491\\
63.09	0.00201254196424411\\
63.1	0.00201254183137236\\
63.11	0.00201254169827866\\
63.12	0.00201254156496198\\
63.13	0.00201254143142131\\
63.14	0.00201254129765562\\
63.15	0.00201254116366387\\
63.16	0.002012541029445\\
63.17	0.00201254089499798\\
63.18	0.00201254076032174\\
63.19	0.00201254062541522\\
63.2	0.00201254049027734\\
63.21	0.00201254035490702\\
63.22	0.00201254021930317\\
63.23	0.00201254008346469\\
63.24	0.00201253994739048\\
63.25	0.00201253981107944\\
63.26	0.00201253967453042\\
63.27	0.00201253953774232\\
63.28	0.00201253940071399\\
63.29	0.00201253926344429\\
63.3	0.00201253912593206\\
63.31	0.00201253898817616\\
63.32	0.0020125388501754\\
63.33	0.00201253871192862\\
63.34	0.00201253857343463\\
63.35	0.00201253843469223\\
63.36	0.00201253829570022\\
63.37	0.0020125381564574\\
63.38	0.00201253801696255\\
63.39	0.00201253787721443\\
63.4	0.00201253773721182\\
63.41	0.00201253759695347\\
63.42	0.00201253745643813\\
63.43	0.00201253731566453\\
63.44	0.0020125371746314\\
63.45	0.00201253703333747\\
63.46	0.00201253689178144\\
63.47	0.00201253674996203\\
63.48	0.00201253660787791\\
63.49	0.00201253646552778\\
63.5	0.00201253632291031\\
63.51	0.00201253618002416\\
63.52	0.002012536036868\\
63.53	0.00201253589344046\\
63.54	0.00201253574974019\\
63.55	0.00201253560576581\\
63.56	0.00201253546151594\\
63.57	0.00201253531698919\\
63.58	0.00201253517218415\\
63.59	0.00201253502709942\\
63.6	0.00201253488173357\\
63.61	0.00201253473608517\\
63.62	0.00201253459015278\\
63.63	0.00201253444393495\\
63.64	0.00201253429743021\\
63.65	0.00201253415063709\\
63.66	0.00201253400355412\\
63.67	0.00201253385617979\\
63.68	0.0020125337085126\\
63.69	0.00201253356055104\\
63.7	0.00201253341229359\\
63.71	0.0020125332637387\\
63.72	0.00201253311488484\\
63.73	0.00201253296573044\\
63.74	0.00201253281627393\\
63.75	0.00201253266651374\\
63.76	0.00201253251644828\\
63.77	0.00201253236607595\\
63.78	0.00201253221539512\\
63.79	0.00201253206440419\\
63.8	0.0020125319131015\\
63.81	0.00201253176148543\\
63.82	0.0020125316095543\\
63.83	0.00201253145730644\\
63.84	0.00201253130474019\\
63.85	0.00201253115185383\\
63.86	0.00201253099864568\\
63.87	0.002012530845114\\
63.88	0.00201253069125707\\
63.89	0.00201253053707315\\
63.9	0.00201253038256049\\
63.91	0.00201253022771732\\
63.92	0.00201253007254186\\
63.93	0.00201252991703232\\
63.94	0.0020125297611869\\
63.95	0.00201252960500378\\
63.96	0.00201252944848113\\
63.97	0.00201252929161712\\
63.98	0.00201252913440989\\
63.99	0.00201252897685757\\
64	0.00201252881895828\\
64.01	0.00201252866071014\\
64.02	0.00201252850211123\\
64.03	0.00201252834315964\\
64.04	0.00201252818385343\\
64.05	0.00201252802419066\\
64.06	0.00201252786416937\\
64.07	0.00201252770378759\\
64.08	0.00201252754304333\\
64.09	0.00201252738193458\\
64.1	0.00201252722045934\\
64.11	0.00201252705861559\\
64.12	0.00201252689640126\\
64.13	0.00201252673381432\\
64.14	0.00201252657085269\\
64.15	0.00201252640751429\\
64.16	0.00201252624379702\\
64.17	0.00201252607969876\\
64.18	0.00201252591521739\\
64.19	0.00201252575035076\\
64.2	0.00201252558509672\\
64.21	0.00201252541945311\\
64.22	0.00201252525341772\\
64.23	0.00201252508698836\\
64.24	0.00201252492016282\\
64.25	0.00201252475293886\\
64.26	0.00201252458531423\\
64.27	0.00201252441728668\\
64.28	0.00201252424885393\\
64.29	0.00201252408001368\\
64.3	0.00201252391076362\\
64.31	0.00201252374110143\\
64.32	0.00201252357102478\\
64.33	0.0020125234005313\\
64.34	0.00201252322961863\\
64.35	0.00201252305828438\\
64.36	0.00201252288652613\\
64.37	0.00201252271434149\\
64.38	0.00201252254172799\\
64.39	0.00201252236868321\\
64.4	0.00201252219520466\\
64.41	0.00201252202128985\\
64.42	0.0020125218469363\\
64.43	0.00201252167214147\\
64.44	0.00201252149690284\\
64.45	0.00201252132121785\\
64.46	0.00201252114508393\\
64.47	0.00201252096849849\\
64.48	0.00201252079145893\\
64.49	0.00201252061396263\\
64.5	0.00201252043600695\\
64.51	0.00201252025758923\\
64.52	0.00201252007870679\\
64.53	0.00201251989935696\\
64.54	0.00201251971953701\\
64.55	0.00201251953924423\\
64.56	0.00201251935847586\\
64.57	0.00201251917722914\\
64.58	0.00201251899550129\\
64.59	0.00201251881328951\\
64.6	0.00201251863059099\\
64.61	0.00201251844740288\\
64.62	0.00201251826372233\\
64.63	0.00201251807954648\\
64.64	0.00201251789487242\\
64.65	0.00201251770969724\\
64.66	0.00201251752401802\\
64.67	0.00201251733783181\\
64.68	0.00201251715113564\\
64.69	0.00201251696392651\\
64.7	0.00201251677620144\\
64.71	0.00201251658795738\\
64.72	0.0020125163991913\\
64.73	0.00201251620990013\\
64.74	0.00201251602008079\\
64.75	0.00201251582973017\\
64.76	0.00201251563884515\\
64.77	0.00201251544742258\\
64.78	0.00201251525545931\\
64.79	0.00201251506295215\\
64.8	0.00201251486989789\\
64.81	0.00201251467629331\\
64.82	0.00201251448213517\\
64.83	0.0020125142874202\\
64.84	0.00201251409214512\\
64.85	0.00201251389630662\\
64.86	0.00201251369990138\\
64.87	0.00201251350292604\\
64.88	0.00201251330537725\\
64.89	0.0020125131072516\\
64.9	0.0020125129085457\\
64.91	0.0020125127092561\\
64.92	0.00201251250937937\\
64.93	0.00201251230891202\\
64.94	0.00201251210785055\\
64.95	0.00201251190619146\\
64.96	0.00201251170393121\\
64.97	0.00201251150106623\\
64.98	0.00201251129759294\\
64.99	0.00201251109350775\\
65	0.00201251088880703\\
65.01	0.00201251068348712\\
65.02	0.00201251047754437\\
65.03	0.00201251027097507\\
65.04	0.00201251006377553\\
65.05	0.002012509855942\\
65.06	0.00201250964747072\\
65.07	0.00201250943835792\\
65.08	0.00201250922859979\\
65.09	0.00201250901819251\\
65.1	0.00201250880713222\\
65.11	0.00201250859541507\\
65.12	0.00201250838303716\\
65.13	0.00201250816999457\\
65.14	0.00201250795628336\\
65.15	0.00201250774189957\\
65.16	0.00201250752683922\\
65.17	0.0020125073110983\\
65.18	0.00201250709467278\\
65.19	0.0020125068775586\\
65.2	0.00201250665975169\\
65.21	0.00201250644124795\\
65.22	0.00201250622204324\\
65.23	0.00201250600213343\\
65.24	0.00201250578151435\\
65.25	0.00201250556018179\\
65.26	0.00201250533813154\\
65.27	0.00201250511535936\\
65.28	0.00201250489186098\\
65.29	0.00201250466763212\\
65.3	0.00201250444266845\\
65.31	0.00201250421696565\\
65.32	0.00201250399051934\\
65.33	0.00201250376332516\\
65.34	0.00201250353537867\\
65.35	0.00201250330667546\\
65.36	0.00201250307721107\\
65.37	0.00201250284698101\\
65.38	0.00201250261598077\\
65.39	0.00201250238420583\\
65.4	0.00201250215165164\\
65.41	0.00201250191831361\\
65.42	0.00201250168418714\\
65.43	0.0020125014492676\\
65.44	0.00201250121355035\\
65.45	0.0020125009770307\\
65.46	0.00201250073970395\\
65.47	0.00201250050156539\\
65.48	0.00201250026261025\\
65.49	0.00201250002283377\\
65.5	0.00201249978223114\\
65.51	0.00201249954079755\\
65.52	0.00201249929852814\\
65.53	0.00201249905541805\\
65.54	0.00201249881146237\\
65.55	0.00201249856665619\\
65.56	0.00201249832099455\\
65.57	0.00201249807447249\\
65.58	0.00201249782708501\\
65.59	0.0020124975788271\\
65.6	0.0020124973296937\\
65.61	0.00201249707967975\\
65.62	0.00201249682878015\\
65.63	0.00201249657698979\\
65.64	0.00201249632430352\\
65.65	0.00201249607071618\\
65.66	0.00201249581622257\\
65.67	0.00201249556081748\\
65.68	0.00201249530449567\\
65.69	0.00201249504725187\\
65.7	0.00201249478908079\\
65.71	0.00201249452997712\\
65.72	0.00201249426993552\\
65.73	0.00201249400895062\\
65.74	0.00201249374701705\\
65.75	0.00201249348412939\\
65.76	0.0020124932202822\\
65.77	0.00201249295547003\\
65.78	0.00201249268968739\\
65.79	0.00201249242292878\\
65.8	0.00201249215518866\\
65.81	0.00201249188646148\\
65.82	0.00201249161674166\\
65.83	0.0020124913460236\\
65.84	0.00201249107430167\\
65.85	0.00201249080157023\\
65.86	0.0020124905278236\\
65.87	0.00201249025305608\\
65.88	0.00201248997726195\\
65.89	0.00201248970043548\\
65.9	0.00201248942257089\\
65.91	0.0020124891436624\\
65.92	0.0020124888637042\\
65.93	0.00201248858269044\\
65.94	0.00201248830061528\\
65.95	0.00201248801747284\\
65.96	0.0020124877332572\\
65.97	0.00201248744796246\\
65.98	0.00201248716158265\\
65.99	0.00201248687411182\\
66	0.00201248658554398\\
66.01	0.0020124862958731\\
66.02	0.00201248600509316\\
66.03	0.0020124857131981\\
66.04	0.00201248542018185\\
66.05	0.00201248512603831\\
66.06	0.00201248483076136\\
66.07	0.00201248453434486\\
66.08	0.00201248423678265\\
66.09	0.00201248393806855\\
66.1	0.00201248363819637\\
66.11	0.00201248333715988\\
66.12	0.00201248303495284\\
66.13	0.00201248273156899\\
66.14	0.00201248242700205\\
66.15	0.00201248212124573\\
66.16	0.00201248181429371\\
66.17	0.00201248150613964\\
66.18	0.00201248119677719\\
66.19	0.00201248088619997\\
66.2	0.00201248057440159\\
66.21	0.00201248026137565\\
66.22	0.00201247994711571\\
66.23	0.00201247963161535\\
66.24	0.00201247931486808\\
66.25	0.00201247899686745\\
66.26	0.00201247867760695\\
66.27	0.00201247835708008\\
66.28	0.0020124780352803\\
66.29	0.00201247771220108\\
66.3	0.00201247738783587\\
66.31	0.00201247706217808\\
66.32	0.00201247673522114\\
66.33	0.00201247640695844\\
66.34	0.00201247607738337\\
66.35	0.0020124757464893\\
66.36	0.00201247541426958\\
66.37	0.00201247508071757\\
66.38	0.0020124747458266\\
66.39	0.00201247440958998\\
66.4	0.00201247407200103\\
66.41	0.00201247373305304\\
66.42	0.0020124733927393\\
66.43	0.00201247305105308\\
66.44	0.00201247270798765\\
66.45	0.00201247236353626\\
66.46	0.00201247201769216\\
66.47	0.00201247167044858\\
66.48	0.00201247132179875\\
66.49	0.00201247097173589\\
66.5	0.0020124706202532\\
66.51	0.0020124702673439\\
66.52	0.00201246991300118\\
66.53	0.00201246955721823\\
66.54	0.00201246919998824\\
66.55	0.00201246884130438\\
66.56	0.00201246848115982\\
66.57	0.00201246811954773\\
66.58	0.00201246775646129\\
66.59	0.00201246739189365\\
66.6	0.00201246702583796\\
66.61	0.0020124666582874\\
66.62	0.0020124662892351\\
66.63	0.00201246591867422\\
66.64	0.00201246554659792\\
66.65	0.00201246517299934\\
66.66	0.00201246479787163\\
66.67	0.00201246442120796\\
66.68	0.00201246404300146\\
66.69	0.00201246366324531\\
66.7	0.00201246328193265\\
66.71	0.00201246289905665\\
66.72	0.00201246251461047\\
66.73	0.00201246212858729\\
66.74	0.00201246174098028\\
66.75	0.00201246135178261\\
66.76	0.00201246096098749\\
66.77	0.00201246056858809\\
66.78	0.00201246017457764\\
66.79	0.00201245977894932\\
66.8	0.00201245938169638\\
66.81	0.00201245898281202\\
66.82	0.00201245858228951\\
66.83	0.00201245818012208\\
66.84	0.002012457776303\\
66.85	0.00201245737082555\\
66.86	0.00201245696368302\\
66.87	0.00201245655486871\\
66.88	0.00201245614437594\\
66.89	0.00201245573219805\\
66.9	0.00201245531832839\\
66.91	0.00201245490276032\\
66.92	0.00201245448548724\\
66.93	0.00201245406650255\\
66.94	0.00201245364579968\\
66.95	0.00201245322337207\\
66.96	0.0020124527992132\\
66.97	0.00201245237331654\\
66.98	0.00201245194567563\\
66.99	0.00201245151628399\\
67	0.00201245108513519\\
67.01	0.00201245065222281\\
67.02	0.00201245021754048\\
67.03	0.00201244978108183\\
67.04	0.00201244934284054\\
67.05	0.00201244890281031\\
67.06	0.00201244846098488\\
67.07	0.00201244801735799\\
67.08	0.00201244757192346\\
67.09	0.0020124471246751\\
67.1	0.00201244667560678\\
67.11	0.0020124462247124\\
67.12	0.0020124457719859\\
67.13	0.00201244531742125\\
67.14	0.00201244486101244\\
67.15	0.00201244440275355\\
67.16	0.00201244394263864\\
67.17	0.00201244348066187\\
67.18	0.00201244301681739\\
67.19	0.00201244255109942\\
67.2	0.00201244208350223\\
67.21	0.00201244161402012\\
67.22	0.00201244114264745\\
67.23	0.00201244066937861\\
67.24	0.00201244019420805\\
67.25	0.00201243971713028\\
67.26	0.00201243923813985\\
67.27	0.00201243875723135\\
67.28	0.00201243827439943\\
67.29	0.00201243778963882\\
67.3	0.00201243730294428\\
67.31	0.00201243681431062\\
67.32	0.00201243632373273\\
67.33	0.00201243583120554\\
67.34	0.00201243533672406\\
67.35	0.00201243484028333\\
67.36	0.00201243434187849\\
67.37	0.00201243384150471\\
67.38	0.00201243333915725\\
67.39	0.00201243283483143\\
67.4	0.00201243232852261\\
67.41	0.00201243182022627\\
67.42	0.0020124313099379\\
67.43	0.00201243079765311\\
67.44	0.00201243028336756\\
67.45	0.00201242976707698\\
67.46	0.00201242924877718\\
67.47	0.00201242872846405\\
67.48	0.00201242820613355\\
67.49	0.00201242768178171\\
67.5	0.00201242715540466\\
67.51	0.0020124266269986\\
67.52	0.00201242609655979\\
67.53	0.00201242556408462\\
67.54	0.00201242502956951\\
67.55	0.00201242449301101\\
67.56	0.00201242395440573\\
67.57	0.00201242341375038\\
67.58	0.00201242287104175\\
67.59	0.00201242232627673\\
67.6	0.00201242177945228\\
67.61	0.00201242123056549\\
67.62	0.00201242067961349\\
67.63	0.00201242012659357\\
67.64	0.00201241957150306\\
67.65	0.00201241901433941\\
67.66	0.00201241845510017\\
67.67	0.00201241789378298\\
67.68	0.0020124173303856\\
67.69	0.00201241676490587\\
67.7	0.00201241619734173\\
67.71	0.00201241562769126\\
67.72	0.0020124150559526\\
67.73	0.00201241448212402\\
67.74	0.00201241390620389\\
67.75	0.0020124133281907\\
67.76	0.00201241274808304\\
67.77	0.0020124121658796\\
67.78	0.00201241158157921\\
67.79	0.00201241099518078\\
67.8	0.00201241040668335\\
67.81	0.00201240981608608\\
67.82	0.00201240922338824\\
67.83	0.0020124086285892\\
67.84	0.00201240803168848\\
67.85	0.00201240743268569\\
67.86	0.00201240683158057\\
67.87	0.00201240622837299\\
67.88	0.00201240562306291\\
67.89	0.00201240501565044\\
67.9	0.00201240440613581\\
67.91	0.00201240379451936\\
67.92	0.00201240318080156\\
67.93	0.002012402564983\\
67.94	0.00201240194706441\\
67.95	0.00201240132704663\\
67.96	0.00201240070493063\\
67.97	0.00201240008071751\\
67.98	0.00201239945440849\\
67.99	0.00201239882600494\\
68	0.00201239819550832\\
68.01	0.00201239756292024\\
68.02	0.00201239692824246\\
68.03	0.00201239629147682\\
68.04	0.00201239565262534\\
68.05	0.00201239501169013\\
68.06	0.00201239436867345\\
68.07	0.00201239372357768\\
68.08	0.00201239307640535\\
68.09	0.0020123924271591\\
68.1	0.00201239177584169\\
68.11	0.00201239112245603\\
68.12	0.00201239046700516\\
68.13	0.00201238980949224\\
68.14	0.00201238914992056\\
68.15	0.00201238848829353\\
68.16	0.0020123878246147\\
68.17	0.00201238715888774\\
68.18	0.00201238649111646\\
68.19	0.00201238582130477\\
68.2	0.00201238514945673\\
68.21	0.00201238447557651\\
68.22	0.0020123837996684\\
68.23	0.00201238312173683\\
68.24	0.00201238244178633\\
68.25	0.00201238175982156\\
68.26	0.0020123810758473\\
68.27	0.00201238038986845\\
68.28	0.00201237970189\\
68.29	0.00201237901191708\\
68.3	0.00201237831995492\\
68.31	0.00201237762600887\\
68.32	0.00201237693008438\\
68.33	0.00201237623218701\\
68.34	0.0020123755323224\\
68.35	0.00201237483049633\\
68.36	0.00201237412671466\\
68.37	0.00201237342098333\\
68.38	0.00201237271330839\\
68.39	0.002012372003696\\
68.4	0.00201237129215236\\
68.41	0.00201237057868379\\
68.42	0.00201236986329668\\
68.43	0.00201236914599749\\
68.44	0.00201236842679277\\
68.45	0.00201236770568913\\
68.46	0.00201236698269324\\
68.47	0.00201236625781184\\
68.48	0.00201236553105171\\
68.49	0.00201236480241972\\
68.5	0.00201236407192274\\
68.51	0.00201236333956772\\
68.52	0.00201236260536163\\
68.53	0.00201236186931148\\
68.54	0.0020123611314243\\
68.55	0.00201236039170714\\
68.56	0.00201235965016707\\
68.57	0.00201235890681117\\
68.58	0.00201235816164652\\
68.59	0.0020123574146802\\
68.6	0.00201235666591926\\
68.61	0.00201235591537077\\
68.62	0.00201235516304173\\
68.63	0.00201235440893913\\
68.64	0.00201235365306992\\
68.65	0.002012352895441\\
68.66	0.0020123521360592\\
68.67	0.0020123513749313\\
68.68	0.00201235061206399\\
68.69	0.00201234984746388\\
68.7	0.00201234908113748\\
68.71	0.00201234831309121\\
68.72	0.00201234754333136\\
68.73	0.0020123467718641\\
68.74	0.00201234599869547\\
68.75	0.00201234522383134\\
68.76	0.00201234444727744\\
68.77	0.00201234366903932\\
68.78	0.00201234288912236\\
68.79	0.00201234210753171\\
68.8	0.00201234132427235\\
68.81	0.002012340539349\\
68.82	0.00201233975276617\\
68.83	0.00201233896452809\\
68.84	0.00201233817463876\\
68.85	0.00201233738310186\\
68.86	0.0020123365899208\\
68.87	0.00201233579509865\\
68.88	0.00201233499863816\\
68.89	0.00201233420054175\\
68.9	0.00201233340081144\\
68.91	0.0020123325994489\\
68.92	0.00201233179645537\\
68.93	0.00201233099183168\\
68.94	0.00201233018557822\\
68.95	0.00201232937769492\\
68.96	0.00201232856818121\\
68.97	0.00201232775703604\\
68.98	0.00201232694425782\\
68.99	0.00201232612984441\\
69	0.00201232531379312\\
69.01	0.00201232449610062\\
69.02	0.00201232367676309\\
69.03	0.00201232285577674\\
69.04	0.00201232203313781\\
69.05	0.00201232120884264\\
69.06	0.00201232038288757\\
69.07	0.00201231955526902\\
69.08	0.00201231872598347\\
69.09	0.00201231789502744\\
69.1	0.00201231706239753\\
69.11	0.00201231622809036\\
69.12	0.00201231539210266\\
69.13	0.00201231455443119\\
69.14	0.00201231371507277\\
69.15	0.00201231287402429\\
69.16	0.00201231203128273\\
69.17	0.0020123111868451\\
69.18	0.00201231034070849\\
69.19	0.00201230949287008\\
69.2	0.00201230864332709\\
69.21	0.00201230779207684\\
69.22	0.00201230693911671\\
69.23	0.00201230608444415\\
69.24	0.00201230522805671\\
69.25	0.00201230436995199\\
69.26	0.00201230351012769\\
69.27	0.00201230264858159\\
69.28	0.00201230178531154\\
69.29	0.00201230092031549\\
69.3	0.00201230005359147\\
69.31	0.0020122991851376\\
69.32	0.00201229831495209\\
69.33	0.00201229744303325\\
69.34	0.00201229656937945\\
69.35	0.00201229569398921\\
69.36	0.00201229481686109\\
69.37	0.00201229393799378\\
69.38	0.00201229305738608\\
69.39	0.00201229217503685\\
69.4	0.00201229129094511\\
69.41	0.00201229040510993\\
69.42	0.00201228951753052\\
69.43	0.0020122886282062\\
69.44	0.00201228773713638\\
69.45	0.00201228684432061\\
69.46	0.00201228594975852\\
69.47	0.0020122850534499\\
69.48	0.00201228415539462\\
69.49	0.00201228325559269\\
69.5	0.00201228235404424\\
69.51	0.00201228145074953\\
69.52	0.00201228054570892\\
69.53	0.00201227963892293\\
69.54	0.0020122787303922\\
69.55	0.0020122778201175\\
69.56	0.00201227690809973\\
69.57	0.00201227599433994\\
69.58	0.0020122750788393\\
69.59	0.00201227416159913\\
69.6	0.00201227324262091\\
69.61	0.00201227232190623\\
69.62	0.00201227139945687\\
69.63	0.00201227047527471\\
69.64	0.00201226954936183\\
69.65	0.00201226862172042\\
69.66	0.00201226769235287\\
69.67	0.00201226676126169\\
69.68	0.00201226582844957\\
69.69	0.00201226489391936\\
69.7	0.00201226395767408\\
69.71	0.0020122630197169\\
69.72	0.00201226208005117\\
69.73	0.00201226113868042\\
69.74	0.00201226019560835\\
69.75	0.00201225925083883\\
69.76	0.00201225830437591\\
69.77	0.00201225735622383\\
69.78	0.00201225640638701\\
69.79	0.00201225545487004\\
69.8	0.00201225450167773\\
69.81	0.00201225354681505\\
69.82	0.00201225259028718\\
69.83	0.00201225163209951\\
69.84	0.00201225067225759\\
69.85	0.00201224971076721\\
69.86	0.00201224874763433\\
69.87	0.00201224778286515\\
69.88	0.00201224681646604\\
69.89	0.00201224584844362\\
69.9	0.00201224487880469\\
69.91	0.00201224390755629\\
69.92	0.00201224293470566\\
69.93	0.00201224196026028\\
69.94	0.00201224098422784\\
69.95	0.00201224000661625\\
69.96	0.00201223902743367\\
69.97	0.00201223804668848\\
69.98	0.00201223706438929\\
69.99	0.00201223608054495\\
70	0.00201223509516454\\
70.01	0.0020122341082574\\
70.02	0.0020122331198331\\
70.03	0.00201223212990147\\
70.04	0.00201223113847256\\
70.05	0.00201223014555671\\
70.06	0.00201222915116449\\
70.07	0.00201222815530674\\
70.08	0.00201222715799454\\
70.09	0.00201222615923925\\
70.1	0.0020122251590525\\
70.11	0.00201222415744616\\
70.12	0.0020122231544324\\
70.13	0.00201222215002364\\
70.14	0.0020122211442326\\
70.15	0.00201222013707224\\
70.16	0.00201221912855584\\
70.17	0.00201221811869693\\
70.18	0.00201221710750935\\
70.19	0.0020122160950072\\
70.2	0.00201221508120491\\
70.21	0.00201221406611717\\
70.22	0.00201221304975896\\
70.23	0.00201221203214558\\
70.24	0.00201221101329263\\
70.25	0.00201220999321599\\
70.26	0.00201220897193186\\
70.27	0.00201220794945674\\
70.28	0.00201220692580744\\
70.29	0.0020122059010011\\
70.3	0.00201220487505514\\
70.31	0.00201220384798732\\
70.32	0.0020122028198157\\
70.33	0.00201220179055868\\
70.34	0.00201220076023498\\
70.35	0.00201219972886362\\
70.36	0.00201219869646396\\
70.37	0.00201219766305571\\
70.38	0.00201219662865889\\
70.39	0.00201219559329383\\
70.4	0.00201219455698124\\
70.41	0.00201219351974213\\
70.42	0.00201219248159787\\
70.43	0.00201219144257015\\
70.44	0.00201219040268101\\
70.45	0.00201218936195285\\
70.46	0.00201218832040838\\
70.47	0.00201218727807067\\
70.48	0.00201218623496315\\
70.49	0.00201218519110959\\
70.5	0.0020121841465341\\
70.51	0.00201218310126115\\
70.52	0.00201218205531555\\
70.53	0.00201218100872249\\
70.54	0.00201217996150749\\
70.55	0.00201217891369643\\
70.56	0.00201217786531554\\
70.57	0.00201217681639143\\
70.58	0.00201217576695103\\
70.59	0.00201217471702167\\
70.6	0.002012173666631\\
70.61	0.00201217261580704\\
70.62	0.00201217156457819\\
70.63	0.00201217051297318\\
70.64	0.00201216946102111\\
70.65	0.00201216840875144\\
70.66	0.00201216735619399\\
70.67	0.00201216630337892\\
70.68	0.00201216525033678\\
70.69	0.00201216419709845\\
70.7	0.00201216314369517\\
70.71	0.00201216209015855\\
70.72	0.00201216103652054\\
70.73	0.00201215998281344\\
70.74	0.00201215892906992\\
70.75	0.00201215787532298\\
70.76	0.00201215682160597\\
70.77	0.00201215576795262\\
70.78	0.00201215471439695\\
70.79	0.00201215366097337\\
70.8	0.0020121526077166\\
70.81	0.00201215155466172\\
70.82	0.00201215050184412\\
70.83	0.00201214944929955\\
70.84	0.00201214839706407\\
70.85	0.00201214734517407\\
70.86	0.00201214629366627\\
70.87	0.00201214524257769\\
70.88	0.00201214419194567\\
70.89	0.00201214314180789\\
70.9	0.00201214209220229\\
70.91	0.00201214104316714\\
70.92	0.002012139994741\\
70.93	0.00201213894696272\\
70.94	0.00201213789987143\\
70.95	0.00201213685350654\\
70.96	0.00201213580790775\\
70.97	0.00201213476311501\\
70.98	0.00201213371916853\\
70.99	0.00201213267610879\\
71	0.0020121316339765\\
71.01	0.00201213059281261\\
71.02	0.00201212955265831\\
71.03	0.00201212851355501\\
71.04	0.00201212747554434\\
71.05	0.00201212643866812\\
71.06	0.00201212540296838\\
71.07	0.00201212436848733\\
71.08	0.00201212333526735\\
71.09	0.00201212230335101\\
71.1	0.00201212127278102\\
71.11	0.00201212024360022\\
71.12	0.00201211921585161\\
71.13	0.00201211818957829\\
71.14	0.00201211716482348\\
71.15	0.0020121161416305\\
71.16	0.00201211512004272\\
71.17	0.00201211410010363\\
71.18	0.00201211308185673\\
71.19	0.00201211206534558\\
71.2	0.00201211105061375\\
71.21	0.00201211003770483\\
71.22	0.00201210902666239\\
71.23	0.00201210801753\\
71.24	0.00201210701035115\\
71.25	0.0020121060051693\\
71.26	0.00201210500202781\\
71.27	0.00201210400096995\\
71.28	0.00201210300203886\\
71.29	0.00201210200527757\\
71.3	0.00201210101072893\\
71.31	0.00201210001843559\\
71.32	0.00201209902844003\\
71.33	0.00201209804078448\\
71.34	0.00201209705551092\\
71.35	0.00201209607266108\\
71.36	0.00201209509227635\\
71.37	0.00201209411439781\\
71.38	0.00201209313906621\\
71.39	0.00201209216632187\\
71.4	0.00201209119620475\\
71.41	0.00201209022875433\\
71.42	0.00201208926400965\\
71.43	0.00201208830200924\\
71.44	0.00201208734279109\\
71.45	0.00201208638639265\\
71.46	0.00201208543285075\\
71.47	0.0020120844822016\\
71.48	0.00201208353448074\\
71.49	0.002012082589723\\
71.5	0.00201208164796247\\
71.51	0.00201208070923249\\
71.52	0.00201207977356553\\
71.53	0.00201207884099325\\
71.54	0.00201207791154637\\
71.55	0.0020120769852547\\
71.56	0.00201207606214703\\
71.57	0.00201207514225116\\
71.58	0.00201207422559376\\
71.59	0.00201207331220042\\
71.6	0.00201207240209553\\
71.61	0.00201207149530227\\
71.62	0.00201207059184254\\
71.63	0.00201206969173694\\
71.64	0.00201206879500608\\
71.65	0.00201206790167053\\
71.66	0.00201206701175088\\
71.67	0.00201206612526767\\
71.68	0.00201206524224141\\
71.69	0.00201206436269262\\
71.7	0.00201206348664177\\
71.71	0.0020120626141093\\
71.72	0.00201206174511563\\
71.73	0.00201206087968113\\
71.74	0.00201206001782616\\
71.75	0.00201205915957102\\
71.76	0.00201205830493596\\
71.77	0.00201205745394122\\
71.78	0.00201205660660695\\
71.79	0.00201205576295328\\
71.8	0.00201205492300029\\
71.81	0.00201205408676798\\
71.82	0.00201205325427632\\
71.83	0.00201205242554519\\
71.84	0.00201205160059443\\
71.85	0.00201205077944381\\
71.86	0.00201204996211301\\
71.87	0.00201204914862166\\
71.88	0.00201204833898932\\
71.89	0.00201204753323544\\
71.9	0.0020120467313794\\
71.91	0.00201204593344053\\
71.92	0.00201204513943801\\
71.93	0.00201204434939098\\
71.94	0.00201204356331846\\
71.95	0.00201204278123938\\
71.96	0.00201204200317256\\
71.97	0.00201204122913673\\
71.98	0.00201204045915049\\
71.99	0.00201203969323235\\
72	0.0020120389314007\\
72.01	0.00201203817367379\\
72.02	0.00201203742006977\\
72.03	0.00201203667060666\\
72.04	0.00201203592530235\\
72.05	0.00201203518417459\\
72.06	0.00201203444724099\\
72.07	0.00201203371451904\\
72.08	0.00201203298602605\\
72.09	0.00201203226177922\\
72.1	0.00201203154179556\\
72.11	0.00201203082609195\\
72.12	0.00201203011468511\\
72.13	0.00201202940759157\\
72.14	0.00201202870482771\\
72.15	0.00201202800640973\\
72.16	0.00201202731235367\\
72.17	0.00201202662267536\\
72.18	0.00201202593739046\\
72.19	0.00201202525651445\\
72.2	0.00201202458006259\\
72.21	0.00201202390804997\\
72.22	0.00201202324049145\\
72.23	0.00201202257740171\\
72.24	0.00201202191879518\\
72.25	0.00201202126468612\\
72.26	0.00201202061508854\\
72.27	0.00201201997001622\\
72.28	0.00201201932948273\\
72.29	0.0020120186935014\\
72.3	0.00201201806208531\\
72.31	0.0020120174352473\\
72.32	0.00201201681299996\\
72.33	0.00201201619535564\\
72.34	0.00201201558232641\\
72.35	0.0020120149739241\\
72.36	0.00201201437016025\\
72.37	0.00201201377104614\\
72.38	0.00201201317659276\\
72.39	0.00201201258681084\\
72.4	0.0020120120017108\\
72.41	0.00201201142130276\\
72.42	0.00201201084559658\\
72.43	0.00201201027460178\\
72.44	0.00201200970832759\\
72.45	0.00201200914678292\\
72.46	0.00201200858997635\\
72.47	0.00201200803791617\\
72.48	0.0020120074906103\\
72.49	0.00201200694806637\\
72.5	0.00201200641029162\\
72.51	0.002012005877293\\
72.52	0.00201200534907705\\
72.53	0.00201200482565002\\
72.54	0.00201200430701775\\
72.55	0.00201200379318573\\
72.56	0.00201200328415909\\
72.57	0.00201200277994256\\
72.58	0.00201200228054052\\
72.59	0.00201200178595693\\
72.6	0.00201200129619539\\
72.61	0.00201200081125906\\
72.62	0.00201200033115075\\
72.63	0.00201199985587281\\
72.64	0.00201199938542722\\
72.65	0.00201199891981551\\
72.66	0.0020119984590388\\
72.67	0.00201199800309777\\
72.68	0.00201199755199269\\
72.69	0.00201199710572335\\
72.7	0.00201199666428913\\
72.71	0.00201199622768894\\
72.72	0.00201199579592124\\
72.73	0.00201199536898403\\
72.74	0.00201199494687483\\
72.75	0.00201199452959071\\
72.76	0.00201199411712825\\
72.77	0.00201199370948354\\
72.78	0.00201199330665221\\
72.79	0.00201199290862935\\
72.8	0.00201199251540962\\
72.81	0.00201199212698711\\
72.82	0.00201199174335545\\
72.83	0.00201199136450773\\
72.84	0.00201199099043654\\
72.85	0.00201199062113394\\
72.86	0.00201199025659147\\
72.87	0.00201198989680012\\
72.88	0.00201198954175038\\
72.89	0.00201198919143217\\
72.9	0.00201198884583486\\
72.91	0.00201198850494731\\
72.92	0.00201198816875778\\
72.93	0.002011987837254\\
72.94	0.00201198751042314\\
72.95	0.0020119871882518\\
72.96	0.00201198687072599\\
72.97	0.00201198655783118\\
72.98	0.00201198624955224\\
72.99	0.00201198594587347\\
73	0.00201198564677859\\
73.01	0.00201198535225071\\
73.02	0.00201198506227237\\
73.03	0.00201198477682552\\
73.04	0.0020119844958915\\
73.05	0.00201198421945106\\
73.06	0.00201198394748434\\
73.07	0.00201198367997088\\
73.08	0.00201198341688962\\
73.09	0.00201198315821888\\
73.1	0.00201198290393637\\
73.11	0.00201198265401919\\
73.12	0.00201198240844384\\
73.13	0.00201198216718619\\
73.14	0.00201198193022149\\
73.15	0.00201198169752438\\
73.16	0.00201198146906888\\
73.17	0.0020119812448284\\
73.18	0.00201198102477572\\
73.19	0.002011980808883\\
73.2	0.00201198059712179\\
73.21	0.00201198038946302\\
73.22	0.002011980185877\\
73.23	0.00201197998633342\\
73.24	0.00201197979080136\\
73.25	0.00201197959924929\\
73.26	0.00201197941164507\\
73.27	0.00201197922795594\\
73.28	0.00201197904814854\\
73.29	0.00201197887218891\\
73.3	0.00201197870004249\\
73.31	0.00201197853167411\\
73.32	0.00201197836704803\\
73.33	0.0020119782061279\\
73.34	0.00201197804887681\\
73.35	0.00201197789525725\\
73.36	0.00201197774523115\\
73.37	0.00201197759875986\\
73.38	0.00201197745580421\\
73.39	0.00201197731632442\\
73.4	0.0020119771802802\\
73.41	0.00201197704763073\\
73.42	0.00201197691833464\\
73.43	0.00201197679235004\\
73.44	0.00201197666963454\\
73.45	0.00201197655014525\\
73.46	0.0020119764338388\\
73.47	0.0020119763206713\\
73.48	0.00201197621059844\\
73.49	0.00201197610357544\\
73.5	0.00201197599955706\\
73.51	0.00201197589849765\\
73.52	0.00201197580035115\\
73.53	0.0020119757050711\\
73.54	0.00201197561261063\\
73.55	0.00201197552292254\\
73.56	0.00201197543595926\\
73.57	0.00201197535167289\\
73.58	0.00201197527001522\\
73.59	0.00201197519093775\\
73.6	0.00201197511439169\\
73.61	0.00201197504032801\\
73.62	0.00201197496869745\\
73.63	0.00201197489945053\\
73.64	0.0020119748325376\\
73.65	0.00201197476790884\\
73.66	0.00201197470551431\\
73.67	0.00201197464530393\\
73.68	0.00201197458722757\\
73.69	0.00201197453123505\\
73.7	0.00201197447727613\\
73.71	0.00201197442530062\\
73.72	0.00201197437525834\\
73.73	0.00201197432709921\\
73.74	0.00201197428077323\\
73.75	0.00201197423623056\\
73.76	0.00201197419342153\\
73.77	0.00201197415229668\\
73.78	0.00201197411280681\\
73.79	0.00201197407490303\\
73.8	0.00201197403853676\\
73.81	0.00201197400365982\\
73.82	0.00201197397022443\\
73.83	0.00201197393818331\\
73.84	0.00201197390748967\\
73.85	0.0020119738780973\\
73.86	0.0020119738499606\\
73.87	0.00201197382303464\\
73.88	0.0020119737972752\\
73.89	0.00201197377263885\\
73.9	0.00201197374908298\\
73.91	0.00201197372656588\\
73.92	0.00201197370504679\\
73.93	0.00201197368448597\\
73.94	0.00201197366484473\\
73.95	0.00201197364608554\\
73.96	0.00201197362817211\\
73.97	0.00201197361106937\\
73.98	0.00201197359474366\\
73.99	0.0020119735791627\\
74	0.00201197356429568\\
74.01	0.00201197355011188\\
74.02	0.00201197353658076\\
74.03	0.00201197352367196\\
74.04	0.00201197351135532\\
74.05	0.00201197349960091\\
74.06	0.00201197348837906\\
74.07	0.00201197347766039\\
74.08	0.00201197346741581\\
74.09	0.00201197345761659\\
74.1	0.00201197344823439\\
74.11	0.00201197343924122\\
74.12	0.00201197343060957\\
74.13	0.00201197342231237\\
74.14	0.00201197341432307\\
74.15	0.00201197340661565\\
74.16	0.00201197339916464\\
74.17	0.00201197339194521\\
74.18	0.00201197338493315\\
74.19	0.00201197337810496\\
74.2	0.00201197337143783\\
74.21	0.00201197336490976\\
74.22	0.00201197335849953\\
74.23	0.00201197335218677\\
74.24	0.00201197334595204\\
74.25	0.00201197333977682\\
74.26	0.00201197333364359\\
74.27	0.00201197332753587\\
74.28	0.00201197332143827\\
74.29	0.00201197331533657\\
74.3	0.00201197330921771\\
74.31	0.00201197330306991\\
74.32	0.00201197329688269\\
74.33	0.00201197329064695\\
74.34	0.002011973284355\\
74.35	0.00201197327800066\\
74.36	0.00201197327157927\\
74.37	0.00201197326508741\\
74.38	0.00201197325852225\\
74.39	0.00201197325188167\\
74.4	0.00201197324516419\\
74.41	0.00201197323836893\\
74.42	0.00201197323149498\\
74.43	0.00201197322454145\\
74.44	0.0020119732175074\\
74.45	0.00201197321039192\\
74.46	0.00201197320319406\\
74.47	0.00201197319591287\\
74.48	0.00201197318854738\\
74.49	0.00201197318109663\\
74.5	0.00201197317355963\\
74.51	0.00201197316593538\\
74.52	0.00201197315822287\\
74.53	0.00201197315042109\\
74.54	0.00201197314252901\\
74.55	0.00201197313454558\\
74.56	0.00201197312646974\\
74.57	0.00201197311830044\\
74.58	0.00201197311003659\\
74.59	0.00201197310167711\\
74.6	0.00201197309322088\\
74.61	0.00201197308466679\\
74.62	0.00201197307601371\\
74.63	0.00201197306726049\\
74.64	0.002011973058406\\
74.65	0.00201197304944904\\
74.66	0.00201197304038844\\
74.67	0.00201197303122301\\
74.68	0.00201197302195154\\
74.69	0.00201197301257279\\
74.7	0.00201197300308554\\
74.71	0.00201197299348853\\
74.72	0.00201197298378049\\
74.73	0.00201197297396015\\
74.74	0.0020119729640262\\
74.75	0.00201197295397734\\
74.76	0.00201197294381224\\
74.77	0.00201197293352956\\
74.78	0.00201197292312795\\
74.79	0.00201197291260602\\
74.8	0.0020119729019624\\
74.81	0.00201197289119568\\
74.82	0.00201197288030443\\
74.83	0.00201197286928723\\
74.84	0.00201197285814262\\
74.85	0.00201197284686913\\
74.86	0.00201197283546528\\
74.87	0.00201197282392955\\
74.88	0.00201197281226044\\
74.89	0.00201197280045639\\
74.9	0.00201197278851586\\
74.91	0.00201197277643727\\
74.92	0.00201197276421903\\
74.93	0.00201197275185953\\
74.94	0.00201197273935714\\
74.95	0.00201197272671021\\
74.96	0.00201197271391708\\
74.97	0.00201197270097605\\
74.98	0.00201197268788544\\
74.99	0.0020119726746435\\
75	0.0020119726612485\\
75.01	0.00201197264769867\\
75.02	0.00201197263399223\\
75.03	0.00201197262012737\\
75.04	0.00201197260610226\\
75.05	0.00201197259191506\\
75.06	0.0020119725775639\\
75.07	0.00201197256304689\\
75.08	0.00201197254836212\\
75.09	0.00201197253350766\\
75.1	0.00201197251848154\\
75.11	0.0020119725032818\\
75.12	0.00201197248790643\\
75.13	0.00201197247235341\\
75.14	0.00201197245662068\\
75.15	0.00201197244070618\\
75.16	0.00201197242460782\\
75.17	0.00201197240832348\\
75.18	0.00201197239185101\\
75.19	0.00201197237518824\\
75.2	0.00201197235833299\\
75.21	0.00201197234128304\\
75.22	0.00201197232403614\\
75.23	0.00201197230659003\\
75.24	0.0020119722889424\\
75.25	0.00201197227109095\\
75.26	0.00201197225303332\\
75.27	0.00201197223476713\\
75.28	0.00201197221628999\\
75.29	0.00201197219759946\\
75.3	0.0020119721786931\\
75.31	0.0020119721595684\\
75.32	0.00201197214022287\\
75.33	0.00201197212065394\\
75.34	0.00201197210085907\\
75.35	0.00201197208083563\\
75.36	0.002011972060581\\
75.37	0.00201197204009253\\
75.38	0.0020119720193675\\
75.39	0.00201197199840321\\
75.4	0.0020119719771969\\
75.41	0.00201197195574578\\
75.42	0.00201197193404703\\
75.43	0.00201197191209781\\
75.44	0.00201197188989523\\
75.45	0.00201197186743637\\
75.46	0.00201197184471829\\
75.47	0.00201197182173799\\
75.48	0.00201197179849247\\
75.49	0.00201197177497868\\
75.5	0.00201197175119351\\
75.51	0.00201197172713386\\
75.52	0.00201197170279656\\
75.53	0.00201197167817841\\
75.54	0.00201197165327619\\
75.55	0.00201197162808663\\
75.56	0.00201197160260643\\
75.57	0.00201197157683222\\
75.58	0.00201197155076065\\
75.59	0.00201197152438828\\
75.6	0.00201197149771165\\
75.61	0.00201197147072727\\
75.62	0.00201197144343159\\
75.63	0.00201197141582103\\
75.64	0.00201197138789198\\
75.65	0.00201197135964076\\
75.66	0.00201197133106368\\
75.67	0.00201197130215698\\
75.68	0.00201197127291688\\
75.69	0.00201197124333954\\
75.7	0.00201197121342109\\
75.71	0.00201197118315759\\
75.72	0.00201197115254509\\
75.73	0.00201197112157956\\
75.74	0.00201197109025696\\
75.75	0.00201197105857317\\
75.76	0.00201197102652405\\
75.77	0.00201197099410539\\
75.78	0.00201197096131294\\
75.79	0.00201197092814241\\
75.8	0.00201197089458946\\
75.81	0.00201197086064968\\
75.82	0.00201197082631863\\
75.83	0.00201197079159182\\
75.84	0.00201197075646469\\
75.85	0.00201197072093264\\
75.86	0.00201197068499102\\
75.87	0.00201197064863513\\
75.88	0.0020119706118602\\
75.89	0.00201197057466142\\
75.9	0.00201197053703392\\
75.91	0.00201197049897276\\
75.92	0.00201197046047298\\
75.93	0.00201197042152953\\
75.94	0.0020119703821373\\
75.95	0.00201197034229116\\
75.96	0.00201197030198588\\
75.97	0.00201197026121618\\
75.98	0.00201197021997674\\
75.99	0.00201197017826216\\
76	0.00201197013606697\\
76.01	0.00201197009338567\\
76.02	0.00201197005021266\\
76.03	0.0020119700065423\\
76.04	0.00201196996236888\\
76.05	0.00201196991768663\\
76.06	0.00201196987248969\\
76.07	0.00201196982677215\\
76.08	0.00201196978052805\\
76.09	0.00201196973375133\\
76.1	0.00201196968643588\\
76.11	0.00201196963857552\\
76.12	0.00201196959016398\\
76.13	0.00201196954119495\\
76.14	0.00201196949166202\\
76.15	0.00201196944155871\\
76.16	0.00201196939087848\\
76.17	0.00201196933961472\\
76.18	0.00201196928776071\\
76.19	0.00201196923530969\\
76.2	0.0020119691822548\\
76.21	0.00201196912858912\\
76.22	0.00201196907430564\\
76.23	0.00201196901939726\\
76.24	0.00201196896385681\\
76.25	0.00201196890767705\\
76.26	0.00201196885085063\\
76.27	0.00201196879337015\\
76.28	0.00201196873522808\\
76.29	0.00201196867641685\\
76.3	0.00201196861692878\\
76.31	0.0020119685567561\\
76.32	0.00201196849589095\\
76.33	0.00201196843432541\\
76.34	0.00201196837205142\\
76.35	0.00201196830906087\\
76.36	0.00201196824534555\\
76.37	0.00201196818089712\\
76.38	0.0020119681157072\\
76.39	0.00201196804976727\\
76.4	0.00201196798306873\\
76.41	0.0020119679156029\\
76.42	0.00201196784736096\\
76.43	0.00201196777833402\\
76.44	0.00201196770851307\\
76.45	0.00201196763788903\\
76.46	0.00201196756645267\\
76.47	0.00201196749419469\\
76.48	0.00201196742110567\\
76.49	0.00201196734717608\\
76.5	0.00201196727239628\\
76.51	0.00201196719675654\\
76.52	0.00201196712024699\\
76.53	0.00201196704285767\\
76.54	0.0020119669645785\\
76.55	0.00201196688539927\\
76.56	0.00201196680530967\\
76.57	0.00201196672429928\\
76.58	0.00201196664235753\\
76.59	0.00201196655947376\\
76.6	0.00201196647563718\\
76.61	0.00201196639083686\\
76.62	0.00201196630506177\\
76.63	0.00201196621830075\\
76.64	0.00201196613054249\\
76.65	0.00201196604177557\\
76.66	0.00201196595198844\\
76.67	0.00201196586116941\\
76.68	0.00201196576930667\\
76.69	0.00201196567638826\\
76.7	0.0020119655824021\\
76.71	0.00201196548733594\\
76.72	0.00201196539117743\\
76.73	0.00201196529391406\\
76.74	0.00201196519553317\\
76.75	0.00201196509602198\\
76.76	0.00201196499536753\\
76.77	0.00201196489355675\\
76.78	0.00201196479057638\\
76.79	0.00201196468641305\\
76.8	0.00201196458105321\\
76.81	0.00201196447448317\\
76.82	0.00201196436668906\\
76.83	0.00201196425765689\\
76.84	0.00201196414737248\\
76.85	0.0020119640358215\\
76.86	0.00201196392298945\\
76.87	0.00201196380886168\\
76.88	0.00201196369342336\\
76.89	0.00201196357665948\\
76.9	0.00201196345855489\\
76.91	0.00201196333909425\\
76.92	0.00201196321826203\\
76.93	0.00201196309604255\\
76.94	0.00201196297241994\\
76.95	0.00201196284737814\\
76.96	0.00201196272090092\\
76.97	0.00201196259297186\\
76.98	0.00201196246357436\\
76.99	0.00201196233269162\\
77	0.00201196220030666\\
77.01	0.00201196206640228\\
77.02	0.00201196193096113\\
77.03	0.00201196179396562\\
77.04	0.00201196165539797\\
77.05	0.00201196151524022\\
77.06	0.00201196137347419\\
77.07	0.00201196123008148\\
77.08	0.00201196108504349\\
77.09	0.00201196093834143\\
77.1	0.00201196078995626\\
77.11	0.00201196063986874\\
77.12	0.00201196048805942\\
77.13	0.00201196033450862\\
77.14	0.00201196017919643\\
77.15	0.00201196002210272\\
77.16	0.00201195986320713\\
77.17	0.00201195970248907\\
77.18	0.00201195953992772\\
77.19	0.002011959375502\\
77.2	0.00201195920919062\\
77.21	0.00201195904097203\\
77.22	0.00201195887082445\\
77.23	0.00201195869872583\\
77.24	0.00201195852465389\\
77.25	0.00201195834858607\\
77.26	0.0020119581704996\\
77.27	0.0020119579903714\\
77.28	0.00201195780817815\\
77.29	0.00201195762389628\\
77.3	0.00201195743750192\\
77.31	0.00201195724897095\\
77.32	0.00201195705827898\\
77.33	0.00201195686540132\\
77.34	0.00201195667031303\\
77.35	0.00201195647298884\\
77.36	0.00201195627340325\\
77.37	0.00201195607153043\\
77.38	0.00201195586734427\\
77.39	0.00201195566081835\\
77.4	0.00201195545192598\\
77.41	0.00201195524064014\\
77.42	0.00201195502693351\\
77.43	0.00201195481077846\\
77.44	0.00201195459214705\\
77.45	0.00201195437101101\\
77.46	0.00201195414734177\\
77.47	0.00201195392111041\\
77.48	0.0020119536922877\\
77.49	0.00201195346084408\\
77.5	0.00201195322674964\\
77.51	0.00201195298997414\\
77.52	0.00201195275048698\\
77.53	0.00201195250825725\\
77.54	0.00201195226325364\\
77.55	0.00201195201544453\\
77.56	0.00201195176479791\\
77.57	0.00201195151128141\\
77.58	0.00201195125486231\\
77.59	0.00201195099550751\\
77.6	0.00201195073318352\\
77.61	0.0020119504678565\\
77.62	0.00201195019949218\\
77.63	0.00201194992805596\\
77.64	0.00201194965351279\\
77.65	0.00201194937582727\\
77.66	0.00201194909496355\\
77.67	0.00201194881088543\\
77.68	0.00201194852355624\\
77.69	0.00201194823293895\\
77.7	0.00201194793899606\\
77.71	0.00201194764168968\\
77.72	0.00201194734098148\\
77.73	0.00201194703683267\\
77.74	0.00201194672920407\\
77.75	0.00201194641805602\\
77.76	0.0020119461033484\\
77.77	0.00201194578504068\\
77.78	0.00201194546309182\\
77.79	0.00201194513746036\\
77.8	0.00201194480810433\\
77.81	0.00201194447498131\\
77.82	0.00201194413804839\\
77.83	0.00201194379726218\\
77.84	0.00201194345257878\\
77.85	0.00201194310395382\\
77.86	0.0020119427513424\\
77.87	0.00201194239469913\\
77.88	0.0020119420339781\\
77.89	0.00201194166913288\\
77.9	0.0020119413001165\\
77.91	0.00201194092688148\\
77.92	0.00201194054937979\\
77.93	0.00201194016756285\\
77.94	0.00201193978138156\\
77.95	0.00201193939078622\\
77.96	0.00201193899572659\\
77.97	0.00201193859615187\\
77.98	0.00201193819201067\\
77.99	0.00201193778325102\\
78	0.00201193736982036\\
78.01	0.00201193695166555\\
78.02	0.00201193652873282\\
78.03	0.00201193610096781\\
78.04	0.00201193566831555\\
78.05	0.00201193523072044\\
78.06	0.00201193478812624\\
78.07	0.00201193434047609\\
78.08	0.00201193388771248\\
78.09	0.00201193342977725\\
78.1	0.00201193296661157\\
78.11	0.00201193249815597\\
78.12	0.00201193202435029\\
78.13	0.00201193154513368\\
78.14	0.00201193106044462\\
78.15	0.00201193057022088\\
78.16	0.00201193007439954\\
78.17	0.00201192957291696\\
78.18	0.00201192906570878\\
78.19	0.0020119285527099\\
78.2	0.00201192803385452\\
78.21	0.00201192750907605\\
78.22	0.00201192697830718\\
78.23	0.00201192644147983\\
78.24	0.00201192589852514\\
78.25	0.00201192534937348\\
78.26	0.00201192479395444\\
78.27	0.0020119242321968\\
78.28	0.00201192366402854\\
78.29	0.00201192308937683\\
78.3	0.00201192250816801\\
78.31	0.0020119219203276\\
78.32	0.00201192132578026\\
78.33	0.0020119207244498\\
78.34	0.0020119201162592\\
78.35	0.00201191950113053\\
78.36	0.002011918878985\\
78.37	0.00201191824974293\\
78.38	0.00201191761332373\\
78.39	0.0020119169696459\\
78.4	0.00201191631862704\\
78.41	0.00201191566018379\\
78.42	0.00201191499423187\\
78.43	0.00201191432068604\\
78.44	0.0020119136394601\\
78.45	0.00201191295046686\\
78.46	0.00201191225361818\\
78.47	0.0020119115488249\\
78.48	0.00201191083599686\\
78.49	0.00201191011504287\\
78.5	0.00201190938587073\\
78.51	0.00201190864838718\\
78.52	0.00201190790249794\\
78.53	0.00201190714810762\\
78.54	0.0020119063851198\\
78.55	0.00201190561343693\\
78.56	0.00201190483296039\\
78.57	0.00201190404359044\\
78.58	0.0020119032452262\\
78.59	0.00201190243776567\\
78.6	0.00201190162110568\\
78.61	0.00201190079514193\\
78.62	0.00201189995976891\\
78.63	0.00201189911487994\\
78.64	0.00201189826036711\\
78.65	0.00201189739612134\\
78.66	0.00201189652203227\\
78.67	0.00201189563798833\\
78.68	0.00201189474387668\\
78.69	0.00201189383958321\\
78.7	0.00201189292499251\\
78.71	0.0020118919999879\\
78.72	0.00201189106445136\\
78.73	0.00201189011826354\\
78.74	0.00201188916130375\\
78.75	0.00201188819344995\\
78.76	0.00201188721457872\\
78.77	0.00201188622456524\\
78.78	0.00201188522328328\\
78.79	0.00201188421060521\\
78.8	0.00201188318640196\\
78.81	0.00201188215054297\\
78.82	0.00201188110289626\\
78.83	0.00201188004332833\\
78.84	0.00201187897170418\\
78.85	0.00201187788788731\\
78.86	0.00201187679173966\\
78.87	0.00201187568312163\\
78.88	0.00201187456189205\\
78.89	0.00201187342790814\\
78.9	0.00201187228102554\\
78.91	0.00201187112109824\\
78.92	0.00201186994797861\\
78.93	0.00201186876151735\\
78.94	0.00201186756156348\\
78.95	0.00201186634796431\\
78.96	0.00201186512056544\\
78.97	0.00201186387921075\\
78.98	0.00201186262374234\\
78.99	0.00201186135400054\\
79	0.00201186006982389\\
79.01	0.00201185877104911\\
79.02	0.00201185745751109\\
79.03	0.00201185612904286\\
79.04	0.00201185478547556\\
79.05	0.00201185342663844\\
79.06	0.00201185205235885\\
79.07	0.00201185066246218\\
79.08	0.00201184925677184\\
79.09	0.0020118478351093\\
79.1	0.00201184639729398\\
79.11	0.00201184494314329\\
79.12	0.0020118434724726\\
79.13	0.00201184198509519\\
79.14	0.00201184048082224\\
79.15	0.00201183895946281\\
79.16	0.00201183742082382\\
79.17	0.00201183586471003\\
79.18	0.00201183429092399\\
79.19	0.00201183269926604\\
79.2	0.00201183108953429\\
79.21	0.00201182946152455\\
79.22	0.00201182781503037\\
79.23	0.00201182614984297\\
79.24	0.00201182446575123\\
79.25	0.00201182276254165\\
79.26	0.00201182103999835\\
79.27	0.002011819297903\\
79.28	0.00201181753603485\\
79.29	0.00201181575417067\\
79.3	0.0020118139520847\\
79.31	0.00201181212954866\\
79.32	0.00201181028633172\\
79.33	0.00201180842220045\\
79.34	0.00201180653691881\\
79.35	0.00201180463024809\\
79.36	0.00201180270194694\\
79.37	0.00201180075177126\\
79.38	0.00201179877947426\\
79.39	0.00201179678480633\\
79.4	0.00201179476751511\\
79.41	0.00201179272734539\\
79.42	0.00201179066403909\\
79.43	0.00201178857733526\\
79.44	0.00201178646696999\\
79.45	0.00201178433267646\\
79.46	0.00201178217418482\\
79.47	0.00201177999122222\\
79.48	0.00201177778351273\\
79.49	0.00201177555077735\\
79.5	0.00201177329273396\\
79.51	0.00201177100909725\\
79.52	0.00201176869957874\\
79.53	0.00201176636388671\\
79.54	0.00201176400172617\\
79.55	0.00201176161279884\\
79.56	0.00201175919680308\\
79.57	0.0020117567534339\\
79.58	0.00201175428238287\\
79.59	0.00201175178333811\\
79.6	0.00201174925598428\\
79.61	0.00201174670000246\\
79.62	0.00201174411507021\\
79.63	0.00201174150086146\\
79.64	0.00201173885704648\\
79.65	0.00201173618329188\\
79.66	0.00201173347926051\\
79.67	0.00201173074461147\\
79.68	0.00201172797900003\\
79.69	0.00201172518207763\\
79.7	0.00201172235349179\\
79.71	0.0020117194928861\\
79.72	0.00201171659990015\\
79.73	0.00201171367416952\\
79.74	0.0020117107153257\\
79.75	0.00201170772299608\\
79.76	0.00201170469680386\\
79.77	0.00201170163636805\\
79.78	0.00201169854130339\\
79.79	0.00201169541122033\\
79.8	0.00201169224572495\\
79.81	0.00201168904441893\\
79.82	0.00201168580689953\\
79.83	0.00201168253275946\\
79.84	0.00201167922158693\\
79.85	0.00201167587296552\\
79.86	0.00201167248647418\\
79.87	0.00201166906168715\\
79.88	0.0020116655981739\\
79.89	0.00201166209549913\\
79.9	0.00201165855322264\\
79.91	0.00201165497089935\\
79.92	0.0020116513480792\\
79.93	0.00201164768430711\\
79.94	0.00201164397912291\\
79.95	0.0020116402320613\\
79.96	0.00201163644265181\\
79.97	0.0020116326104187\\
79.98	0.00201162873488092\\
79.99	0.00201162481555207\\
80	0.00201162085194031\\
80.01	0.00201161684354834\\
};
\addplot [color=blue,dashed]
  table[row sep=crcr]{%
80.01	0.00201161684354834\\
80.02	0.00201161278987329\\
80.03	0.00201160869040669\\
80.04	0.0020116045446344\\
80.05	0.00201160035203657\\
80.06	0.00201159611208752\\
80.07	0.00201159182425574\\
80.08	0.00201158748800378\\
80.09	0.00201158310278822\\
80.1	0.00201157866805956\\
80.11	0.00201157418326219\\
80.12	0.00201156964783432\\
80.13	0.00201156506120789\\
80.14	0.00201156042280853\\
80.15	0.00201155573205544\\
80.16	0.00201155098836138\\
80.17	0.00201154619113258\\
80.18	0.00201154133976863\\
80.19	0.00201153643366246\\
80.2	0.00201153147220022\\
80.21	0.00201152645476126\\
80.22	0.00201152138071799\\
80.23	0.00201151624943585\\
80.24	0.00201151106027323\\
80.25	0.00201150581258137\\
80.26	0.00201150050570428\\
80.27	0.00201149513897871\\
80.28	0.002011489711734\\
80.29	0.00201148422329205\\
80.3	0.0020114786729672\\
80.31	0.00201147306006618\\
80.32	0.00201146738388802\\
80.33	0.00201146164372395\\
80.34	0.00201145583885732\\
80.35	0.0020114499685635\\
80.36	0.00201144403210984\\
80.37	0.00201143802875551\\
80.38	0.0020114319577515\\
80.39	0.00201142581834042\\
80.4	0.00201141960975651\\
80.41	0.00201141333122548\\
80.42	0.00201140698196446\\
80.43	0.00201140056118186\\
80.44	0.00201139406807733\\
80.45	0.00201138750184162\\
80.46	0.00201138086165648\\
80.47	0.0020113741466946\\
80.48	0.00201136735611948\\
80.49	0.00201136048908533\\
80.5	0.00201135354473699\\
80.51	0.00201134652220979\\
80.52	0.00201133942062949\\
80.53	0.00201133223911212\\
80.54	0.00201132497676394\\
80.55	0.00201131763268127\\
80.56	0.00201131020595043\\
80.57	0.00201130269564758\\
80.58	0.00201129510083867\\
80.59	0.00201128742057926\\
80.6	0.00201127965391448\\
80.61	0.00201127179987884\\
80.62	0.00201126385749616\\
80.63	0.00201125582577946\\
80.64	0.0020112477037308\\
80.65	0.0020112394903412\\
80.66	0.00201123118459049\\
80.67	0.00201122278544721\\
80.68	0.00201121429186847\\
80.69	0.00201120570279983\\
80.7	0.00201119701717517\\
80.71	0.00201118823391657\\
80.72	0.00201117935193419\\
80.73	0.0020111703701261\\
80.74	0.00201116128737819\\
80.75	0.00201115210256402\\
80.76	0.00201114281454468\\
80.77	0.00201113342216866\\
80.78	0.00201112392427171\\
80.79	0.00201111431967669\\
80.8	0.00201110460719346\\
80.81	0.00201109478561871\\
80.82	0.00201108485373581\\
80.83	0.0020110748103147\\
80.84	0.00201106465411169\\
80.85	0.00201105438386938\\
80.86	0.00201104399831644\\
80.87	0.00201103349616749\\
80.88	0.00201102287612297\\
80.89	0.00201101213686894\\
80.9	0.00201100127707695\\
80.91	0.00201099029540387\\
80.92	0.00201097919049173\\
80.93	0.00201096796096758\\
80.94	0.00201095660544329\\
80.95	0.00201094512251541\\
80.96	0.00201093351076498\\
80.97	0.00201092176875741\\
80.98	0.00201090989504224\\
80.99	0.00201089788815302\\
81	0.0020108857466071\\
81.01	0.0020108734689055\\
81.02	0.00201086105353267\\
81.03	0.00201084849895636\\
81.04	0.00201083580362741\\
81.05	0.00201082296597958\\
81.06	0.00201080998442936\\
81.07	0.00201079685737576\\
81.08	0.00201078358320017\\
81.09	0.00201077016026611\\
81.1	0.00201075658691907\\
81.11	0.00201074286148633\\
81.12	0.00201072898227669\\
81.13	0.00201071494758035\\
81.14	0.00201070075566868\\
81.15	0.00201068640479397\\
81.16	0.00201067189318929\\
81.17	0.00201065721906824\\
81.18	0.00201064238062476\\
81.19	0.0020106273760329\\
81.2	0.00201061220344659\\
81.21	0.00201059686099947\\
81.22	0.00201058134680461\\
81.23	0.00201056565895434\\
81.24	0.00201054979551997\\
81.25	0.00201053375455163\\
81.26	0.00201051753407796\\
81.27	0.00201050113210594\\
81.28	0.00201048454662062\\
81.29	0.0020104677755849\\
81.3	0.00201045081693929\\
81.31	0.00201043366860163\\
81.32	0.0020104163284669\\
81.33	0.00201039879440694\\
81.34	0.00201038106427019\\
81.35	0.00201036313588145\\
81.36	0.00201034500704163\\
81.37	0.00201032667552749\\
81.38	0.00201030813909136\\
81.39	0.00201028939546087\\
81.4	0.00201027044233872\\
81.41	0.00201025127740238\\
81.42	0.00201023189830383\\
81.43	0.00201021230266926\\
81.44	0.00201019248809881\\
81.45	0.00201017245216631\\
81.46	0.00201015219241893\\
81.47	0.00201013170637696\\
81.48	0.00201011099153348\\
81.49	0.00201009004535407\\
81.5	0.00201006886527652\\
81.51	0.00201004744871053\\
81.52	0.00201002579303739\\
81.53	0.0020100038956097\\
81.54	0.00200998175375102\\
81.55	0.0020099593647556\\
81.56	0.00200993672588803\\
81.57	0.00200991383438294\\
81.58	0.00200989068744465\\
81.59	0.00200986728224688\\
81.6	0.00200984361593237\\
81.61	0.00200981968561261\\
81.62	0.00200979548836743\\
81.63	0.00200977102124471\\
81.64	0.00200974628126001\\
81.65	0.00200972126539624\\
81.66	0.00200969597060327\\
81.67	0.00200967039379764\\
81.68	0.00200964453186212\\
81.69	0.0020096183816454\\
81.7	0.0020095919399617\\
81.71	0.00200956520359041\\
81.72	0.00200953816927571\\
81.73	0.00200951083372619\\
81.74	0.00200948319361444\\
81.75	0.00200945524557672\\
81.76	0.00200942698621251\\
81.77	0.00200939841208414\\
81.78	0.00200936951971641\\
81.79	0.00200934030559612\\
81.8	0.00200931076617175\\
81.81	0.00200928089785297\\
81.82	0.00200925069701026\\
81.83	0.00200922015997448\\
81.84	0.00200918928303644\\
81.85	0.00200915806244648\\
81.86	0.00200912649441402\\
81.87	0.00200909457510712\\
81.88	0.00200906230065206\\
81.89	0.00200902966713285\\
81.9	0.00200899667059082\\
81.91	0.0020089633070241\\
81.92	0.00200892957238724\\
81.93	0.00200889546259066\\
81.94	0.00200886097350023\\
81.95	0.00200882610093676\\
81.96	0.00200879084067554\\
81.97	0.00200875518844581\\
81.98	0.00200871913993033\\
81.99	0.00200868269076483\\
82	0.00200864583653751\\
82.01	0.00200860857278855\\
82.02	0.00200857089500959\\
82.03	0.00200853279864319\\
82.04	0.00200849427908232\\
82.05	0.00200845533166983\\
82.06	0.00200841595169792\\
82.07	0.00200837613440755\\
82.08	0.00200833587498795\\
82.09	0.00200829516857605\\
82.1	0.00200825401025588\\
82.11	0.00200821239505806\\
82.12	0.0020081703179592\\
82.13	0.0020081277738813\\
82.14	0.00200808475769121\\
82.15	0.00200804126420002\\
82.16	0.00200799728816245\\
82.17	0.00200795282427626\\
82.18	0.00200790786718165\\
82.19	0.00200786241146061\\
82.2	0.00200781645163635\\
82.21	0.00200776998217263\\
82.22	0.00200772299747312\\
82.23	0.00200767549188081\\
82.24	0.0020076274596773\\
82.25	0.00200757889508217\\
82.26	0.00200752979225234\\
82.27	0.00200748014528135\\
82.28	0.00200742994819875\\
82.29	0.00200737919496933\\
82.3	0.00200732787949253\\
82.31	0.00200727599560166\\
82.32	0.00200722353706323\\
82.33	0.00200717049757624\\
82.34	0.00200711687077146\\
82.35	0.00200706265021068\\
82.36	0.00200700782938599\\
82.37	0.00200695240171906\\
82.38	0.00200689636056036\\
82.39	0.0020068396991884\\
82.4	0.00200678241080897\\
82.41	0.0020067244885544\\
82.42	0.00200666592548272\\
82.43	0.00200660671457691\\
82.44	0.00200654684874408\\
82.45	0.00200648632081467\\
82.46	0.00200642512354166\\
82.47	0.00200636324959969\\
82.48	0.00200630069158428\\
82.49	0.00200623744201097\\
82.5	0.00200617349331446\\
82.51	0.00200610883784778\\
82.52	0.00200604346788137\\
82.53	0.00200597737560228\\
82.54	0.0020059105531132\\
82.55	0.00200584299243164\\
82.56	0.00200577468548897\\
82.57	0.00200570562412954\\
82.58	0.00200563580010974\\
82.59	0.00200556520509708\\
82.6	0.00200549383066925\\
82.61	0.00200542166831313\\
82.62	0.00200534870942389\\
82.63	0.00200527494530396\\
82.64	0.00200520036716209\\
82.65	0.00200512496611235\\
82.66	0.00200504873317308\\
82.67	0.00200497194232735\\
82.68	0.00200489471204742\\
82.69	0.00200481703781737\\
82.7	0.00200473891507041\\
82.71	0.00200466033918833\\
82.72	0.0020045813055009\\
82.73	0.00200450180928527\\
82.74	0.00200442184576537\\
82.75	0.00200434141011133\\
82.76	0.00200426049743881\\
82.77	0.00200417910280845\\
82.78	0.00200409722122516\\
82.79	0.00200401484763756\\
82.8	0.00200393197693729\\
82.81	0.00200384860395838\\
82.82	0.00200376472347659\\
82.83	0.00200368033020871\\
82.84	0.00200359541881195\\
82.85	0.00200350998388322\\
82.86	0.00200342401995844\\
82.87	0.00200333752151185\\
82.88	0.00200325048295532\\
82.89	0.00200316289863762\\
82.9	0.0020030747628437\\
82.91	0.00200298606979398\\
82.92	0.00200289681364357\\
82.93	0.00200280698848159\\
82.94	0.00200271658833037\\
82.95	0.00200262560714467\\
82.96	0.00200253403881098\\
82.97	0.00200244187714667\\
82.98	0.00200234911589924\\
82.99	0.0020022557487455\\
83	0.00200216176929078\\
83.01	0.0020020671710681\\
83.02	0.00200197194753738\\
83.03	0.00200187609208454\\
83.04	0.00200177959802071\\
83.05	0.00200168245858136\\
83.06	0.00200158466692543\\
83.07	0.00200148621613448\\
83.08	0.00200138709921175\\
83.09	0.00200128730908135\\
83.1	0.00200118683858727\\
83.11	0.00200108568049254\\
83.12	0.00200098382747824\\
83.13	0.00200088127214263\\
83.14	0.00200077800700016\\
83.15	0.0020006740244805\\
83.16	0.00200056931692764\\
83.17	0.00200046387659883\\
83.18	0.00200035769566366\\
83.19	0.00200025076620301\\
83.2	0.00200014308020807\\
83.21	0.00200003462957928\\
83.22	0.00199992540612534\\
83.23	0.00199981540156213\\
83.24	0.00199970460751167\\
83.25	0.00199959301550102\\
83.26	0.00199948061696124\\
83.27	0.00199936740322625\\
83.28	0.00199925336553175\\
83.29	0.00199913849501408\\
83.3	0.00199902278270913\\
83.31	0.00199890621955112\\
83.32	0.0019987887963715\\
83.33	0.00199867050389776\\
83.34	0.00199855133275223\\
83.35	0.00199843127345091\\
83.36	0.00199831031640222\\
83.37	0.00199818845190579\\
83.38	0.00199806567015124\\
83.39	0.00199794196121689\\
83.4	0.00199781731506849\\
83.41	0.00199769172155798\\
83.42	0.00199756517042215\\
83.43	0.00199743765128132\\
83.44	0.00199730915363806\\
83.45	0.00199717966687579\\
83.46	0.00199704918025745\\
83.47	0.00199691768292414\\
83.48	0.00199678516389369\\
83.49	0.00199665161205929\\
83.5	0.00199651701618805\\
83.51	0.00199638136491956\\
83.52	0.00199624464676445\\
83.53	0.0019961068501029\\
83.54	0.00199596796318318\\
83.55	0.0019958279741201\\
83.56	0.00199568687089354\\
83.57	0.00199554464134688\\
83.58	0.00199540127318543\\
83.59	0.0019952567539749\\
83.6	0.00199511107113978\\
83.61	0.00199496421196171\\
83.62	0.00199481616357789\\
83.63	0.00199466691297943\\
83.64	0.00199451644700962\\
83.65	0.00199436475236235\\
83.66	0.00199421181558032\\
83.67	0.00199405762305335\\
83.68	0.00199390216101663\\
83.69	0.00199374541554896\\
83.7	0.00199358737257097\\
83.71	0.00199342801784332\\
83.72	0.00199326733696484\\
83.73	0.00199310531537074\\
83.74	0.00199294193833072\\
83.75	0.00199277719094705\\
83.76	0.00199261105815272\\
83.77	0.00199244352470947\\
83.78	0.00199227457520585\\
83.79	0.00199210419405524\\
83.8	0.00199193236549388\\
83.81	0.0019917590735788\\
83.82	0.00199158430218581\\
83.83	0.00199140803500744\\
83.84	0.00199123025555084\\
83.85	0.00199105094713566\\
83.86	0.00199087009289192\\
83.87	0.00199068767575784\\
83.88	0.00199050367847769\\
83.89	0.00199031808359951\\
83.9	0.00199013087347295\\
83.91	0.00198994203024695\\
83.92	0.0019897515358675\\
83.93	0.00198955937207527\\
83.94	0.00198936552040332\\
83.95	0.00198916996217472\\
83.96	0.00198897267850016\\
83.97	0.00198877365027549\\
83.98	0.00198857285817933\\
83.99	0.00198837028267058\\
84	0.00198816590398587\\
84.01	0.00198795970213709\\
84.02	0.00198775165690879\\
84.03	0.00198754174785561\\
84.04	0.00198732995429964\\
84.05	0.00198711625532778\\
84.06	0.00198690062978907\\
84.07	0.00198668305629197\\
84.08	0.00198646351320161\\
84.09	0.00198624197863704\\
84.1	0.0019860184304684\\
84.11	0.00198579284631413\\
84.12	0.00198556520353802\\
84.13	0.00198533547924641\\
84.14	0.00198510365028516\\
84.15	0.00198486969323677\\
84.16	0.00198463358441731\\
84.17	0.00198439529987343\\
84.18	0.00198415481537927\\
84.19	0.00198391210643335\\
84.2	0.00198366714825545\\
84.21	0.00198341991578344\\
84.22	0.00198317038367002\\
84.23	0.00198291852627953\\
84.24	0.00198266431768465\\
84.25	0.00198240773166304\\
84.26	0.00198214874169403\\
84.27	0.00198188732095521\\
84.28	0.00198162344231895\\
84.29	0.00198135707834899\\
84.3	0.00198108820129689\\
84.31	0.00198081678309846\\
84.32	0.00198054279537021\\
84.33	0.00198026620940566\\
84.34	0.00197998699617169\\
84.35	0.00197970512630485\\
84.36	0.00197942057010751\\
84.37	0.00197913329754416\\
84.38	0.00197884327823749\\
84.39	0.00197855048146453\\
84.4	0.0019782548761527\\
84.41	0.00197795643087582\\
84.42	0.00197765511385011\\
84.43	0.00197735089293011\\
84.44	0.00197704373560456\\
84.45	0.00197673360899221\\
84.46	0.00197642047983764\\
84.47	0.001976104314507\\
84.48	0.00197578507898367\\
84.49	0.00197546273886393\\
84.5	0.00197513725935256\\
84.51	0.00197480860525833\\
84.52	0.00197447674098956\\
84.53	0.00197414163054952\\
84.54	0.00197380323753183\\
84.55	0.00197346152511577\\
84.56	0.00197311645606162\\
84.57	0.00197276799270583\\
84.58	0.00197241609695624\\
84.59	0.00197206073028717\\
84.6	0.00197170185373449\\
84.61	0.00197133942789065\\
84.62	0.0019709734128996\\
84.63	0.00197060376845172\\
84.64	0.00197023045377862\\
84.65	0.00196985342764794\\
84.66	0.00196947264835806\\
84.67	0.00196908807373277\\
84.68	0.00196869966111584\\
84.69	0.00196830736736559\\
84.7	0.00196791114884935\\
84.71	0.00196751096143785\\
84.72	0.00196710676049961\\
84.73	0.00196669850089517\\
84.74	0.00196628613697135\\
84.75	0.0019658696225554\\
84.76	0.00196544891094906\\
84.77	0.00196502395492258\\
84.78	0.0019645947067087\\
84.79	0.00196416111799652\\
84.8	0.00196372313992528\\
84.81	0.00196328072307813\\
84.82	0.00196283381747582\\
84.83	0.00196238237257025\\
84.84	0.00196192633723801\\
84.85	0.00196146565977387\\
84.86	0.0019610002878841\\
84.87	0.00196053016867981\\
84.88	0.00196005524867015\\
84.89	0.00195957547375548\\
84.9	0.00195909078922043\\
84.91	0.00195860113972688\\
84.92	0.0019581064693069\\
84.93	0.00195760672135553\\
84.94	0.0019571018386236\\
84.95	0.00195659176321033\\
84.96	0.00195607643655594\\
84.97	0.00195555579943416\\
84.98	0.00195502979194461\\
84.99	0.00195449835350517\\
85	0.00195396142284418\\
85.01	0.00195341893799259\\
85.02	0.00195287083627606\\
85.03	0.0019523170543069\\
85.04	0.00195175752797594\\
85.05	0.00195119219244435\\
85.06	0.00195062098213528\\
85.07	0.00195004383072554\\
85.08	0.001949460671137\\
85.09	0.00194887143552808\\
85.1	0.00194827605528502\\
85.11	0.00194767446101308\\
85.12	0.00194706658252768\\
85.13	0.00194645234884537\\
85.14	0.00194583168817475\\
85.15	0.00194520452790728\\
85.16	0.00194457079460797\\
85.17	0.00194393041400595\\
85.18	0.00194328331098496\\
85.19	0.00194262940957375\\
85.2	0.00194196863293631\\
85.21	0.00194130090336203\\
85.22	0.00194062614225575\\
85.23	0.00193994427012766\\
85.24	0.00193925520658315\\
85.25	0.00193855887031248\\
85.26	0.00193785517908034\\
85.27	0.00193714404971534\\
85.28	0.00193642539809934\\
85.29	0.00193569913915664\\
85.3	0.00193496518684312\\
85.31	0.00193422345413515\\
85.32	0.00193347385301846\\
85.33	0.00193271629447688\\
85.34	0.00193195068848087\\
85.35	0.00193117694397601\\
85.36	0.00193039496887133\\
85.37	0.00192960467002745\\
85.38	0.0019288059532447\\
85.39	0.00192799872325097\\
85.4	0.00192718288368956\\
85.41	0.00192635833710674\\
85.42	0.00192552498493933\\
85.43	0.00192468272750196\\
85.44	0.00192383146397438\\
85.45	0.00192297109238844\\
85.46	0.00192210150961506\\
85.47	0.00192122261135096\\
85.48	0.0019203342921053\\
85.49	0.00191943644518612\\
85.5	0.00191852896268664\\
85.51	0.00191761173547144\\
85.52	0.00191668465316241\\
85.53	0.00191574760412463\\
85.54	0.00191480178146727\\
85.55	0.00191384987596485\\
85.56	0.00191289182513978\\
85.57	0.00191192756580686\\
85.58	0.00191095703406505\\
85.59	0.00190998016528918\\
85.6	0.00190899689412164\\
85.61	0.00190800715446382\\
85.62	0.00190701087946759\\
85.63	0.00190600945949907\\
85.64	0.00190500301460894\\
85.65	0.00190399149098689\\
85.66	0.00190297483419296\\
85.67	0.00190195298915008\\
85.68	0.00190092590013648\\
85.69	0.00189989351077811\\
85.7	0.00189885576404078\\
85.71	0.00189781260222242\\
85.72	0.00189676396694508\\
85.73	0.00189570979914696\\
85.74	0.0018946500390742\\
85.75	0.00189358462627275\\
85.76	0.00189251349957998\\
85.77	0.0018914365971163\\
85.78	0.00189035385627661\\
85.79	0.0018892652137217\\
85.8	0.0018881706053695\\
85.81	0.00188706996638627\\
85.82	0.00188596323117765\\
85.83	0.00188485033337961\\
85.84	0.0018837312058493\\
85.85	0.00188260578065581\\
85.86	0.00188147398907074\\
85.87	0.00188033576155878\\
85.88	0.00187919102776806\\
85.89	0.00187803971652045\\
85.9	0.00187688175580174\\
85.91	0.00187571707275166\\
85.92	0.00187454559365385\\
85.93	0.00187336724392562\\
85.94	0.00187218194810767\\
85.95	0.00187098962985365\\
85.96	0.00186979021191959\\
85.97	0.00186858361615321\\
85.98	0.00186736976348309\\
85.99	0.00186614857390774\\
86	0.00186491996648453\\
86.01	0.00186368385931843\\
86.02	0.00186244016955071\\
86.03	0.00186118881334742\\
86.04	0.0018599297058878\\
86.05	0.00185866276135249\\
86.06	0.00185738789291162\\
86.07	0.00185610501271279\\
86.08	0.00185481403186887\\
86.09	0.00185351486044562\\
86.1	0.00185220740744925\\
86.11	0.00185089158081377\\
86.12	0.00184956728738816\\
86.13	0.00184823443292349\\
86.14	0.00184689292205975\\
86.15	0.00184554265831263\\
86.16	0.00184418354406008\\
86.17	0.00184281548052876\\
86.18	0.00184143836778023\\
86.19	0.00184005210469711\\
86.2	0.00183865658896894\\
86.21	0.00183725171707798\\
86.22	0.00183583738428473\\
86.23	0.00183441348461339\\
86.24	0.00183297991083704\\
86.25	0.0018315365544627\\
86.26	0.00183008330571624\\
86.27	0.00182862005352698\\
86.28	0.00182714668551225\\
86.29	0.0018256630879617\\
86.3	0.00182416914582138\\
86.31	0.0018226647426777\\
86.32	0.00182114976074115\\
86.33	0.00181962408082984\\
86.34	0.00181808758235285\\
86.35	0.00181654014329332\\
86.36	0.00181498164019144\\
86.37	0.00181341194812712\\
86.38	0.00181183094070253\\
86.39	0.0018102384900244\\
86.4	0.00180863446668609\\
86.41	0.00180701873974947\\
86.42	0.00180539117672658\\
86.43	0.001803751643561\\
86.44	0.00180210000460914\\
86.45	0.00180043612262111\\
86.46	0.00179875985872152\\
86.47	0.00179707107238998\\
86.48	0.00179536962144134\\
86.49	0.00179365536200573\\
86.5	0.00179192814850834\\
86.51	0.00179018783364899\\
86.52	0.00178843426838133\\
86.53	0.00178666730189198\\
86.54	0.00178488678157921\\
86.55	0.00178309255303155\\
86.56	0.00178128446000598\\
86.57	0.00177946234440593\\
86.58	0.00177762604625904\\
86.59	0.00177577540369457\\
86.6	0.00177391025292061\\
86.61	0.00177203042820092\\
86.62	0.00177013576183161\\
86.63	0.0017682260841174\\
86.64	0.00176630122334772\\
86.65	0.00176436100577241\\
86.66	0.00176240525557715\\
86.67	0.00176043379485866\\
86.68	0.0017584464435995\\
86.69	0.0017564430196426\\
86.7	0.00175442333866549\\
86.71	0.00175238721415419\\
86.72	0.00175033445737681\\
86.73	0.00174826487735677\\
86.74	0.00174617828084576\\
86.75	0.00174407447229632\\
86.76	0.00174195325383408\\
86.77	0.00173981442522972\\
86.78	0.00173765778387048\\
86.79	0.0017354831247314\\
86.8	0.00173329024034621\\
86.81	0.0017310789207778\\
86.82	0.00172884895358837\\
86.83	0.00172660012380917\\
86.84	0.00172433221390998\\
86.85	0.00172204500376802\\
86.86	0.00171973827063665\\
86.87	0.00171741178911363\\
86.88	0.0017150653311089\\
86.89	0.00171269866581212\\
86.9	0.00171031155965966\\
86.91	0.00170790377630128\\
86.92	0.00170547507656632\\
86.93	0.00170302521842955\\
86.94	0.00170056345832739\\
86.95	0.00169809020899514\\
86.96	0.00169560534287889\\
86.97	0.00169310873087927\\
86.98	0.00169060024233235\\
86.99	0.00168807974499035\\
87	0.00168554710500204\\
87.01	0.00168300218689294\\
87.02	0.00168044485354525\\
87.03	0.00167787496617754\\
87.04	0.00167529238432416\\
87.05	0.00167269696581444\\
87.06	0.00167008856675156\\
87.07	0.00166746704149123\\
87.08	0.00166483224262001\\
87.09	0.00166218402093346\\
87.1	0.00165952222541394\\
87.11	0.00165684670320814\\
87.12	0.00165415729960437\\
87.13	0.0016514538580095\\
87.14	0.00164873621992566\\
87.15	0.00164600422492662\\
87.16	0.00164325771063388\\
87.17	0.00164049651269244\\
87.18	0.00163772046474633\\
87.19	0.0016349293984137\\
87.2	0.00163212314326174\\
87.21	0.00162930152678117\\
87.22	0.0016264643743605\\
87.23	0.00162361150925986\\
87.24	0.0016207427525846\\
87.25	0.0016178579232585\\
87.26	0.0016149568379966\\
87.27	0.00161203931127783\\
87.28	0.00160910515531709\\
87.29	0.00160615418003715\\
87.3	0.00160318619304011\\
87.31	0.0016002009995785\\
87.32	0.00159719840252601\\
87.33	0.0015941782023479\\
87.34	0.00159114019707094\\
87.35	0.00158808418225306\\
87.36	0.00158500995095255\\
87.37	0.00158191729369691\\
87.38	0.00157880599845126\\
87.39	0.00157567585058635\\
87.4	0.00157252663284622\\
87.41	0.0015693581253154\\
87.42	0.00156617010538566\\
87.43	0.00156296234772238\\
87.44	0.00155973462423051\\
87.45	0.00155648670402002\\
87.46	0.00155321835337099\\
87.47	0.00154992933569815\\
87.48	0.00154661941151511\\
87.49	0.00154328833839795\\
87.5	0.00153993587094853\\
87.51	0.00153656176075717\\
87.52	0.00153316575636493\\
87.53	0.00152974760322542\\
87.54	0.00152630704366604\\
87.55	0.00152284381684883\\
87.56	0.00151935765873068\\
87.57	0.00151584830202315\\
87.58	0.00151231547615171\\
87.59	0.00150875890721445\\
87.6	0.00150517831794029\\
87.61	0.00150157342764662\\
87.62	0.00149794395219637\\
87.63	0.0014942896039546\\
87.64	0.00149061009174448\\
87.65	0.00148690512080265\\
87.66	0.0014831743927341\\
87.67	0.00147941760546638\\
87.68	0.00147563445320327\\
87.69	0.00147182462637782\\
87.7	0.00146798781160478\\
87.71	0.00146412369163241\\
87.72	0.00146023194529367\\
87.73	0.00145631224745677\\
87.74	0.00145239085706545\\
87.75	0.00144846855178409\\
87.76	0.00144454533384675\\
87.77	0.00144062120552747\\
87.78	0.00143669616914078\\
87.79	0.00143277022704227\\
87.8	0.00142884338162906\\
87.81	0.0014249156353404\\
87.82	0.00142098699065818\\
87.83	0.0014170574501075\\
87.84	0.00141312701625725\\
87.85	0.00140919569172062\\
87.86	0.00140526347915577\\
87.87	0.00140133038126631\\
87.88	0.00139739640080197\\
87.89	0.00139346154055918\\
87.9	0.00138952580338165\\
87.91	0.00138558919216101\\
87.92	0.00138165170983742\\
87.93	0.00137771335940021\\
87.94	0.00137377414388853\\
87.95	0.00136983406639194\\
87.96	0.00136589313005115\\
87.97	0.00136195133805861\\
87.98	0.00135800869365922\\
87.99	0.001354065200151\\
88	0.00135012086088577\\
88.01	0.00134617567926986\\
88.02	0.00134222965876482\\
88.03	0.00133828280288813\\
88.04	0.0013343351152139\\
88.05	0.00133038659937365\\
88.06	0.00132643725905702\\
88.07	0.00132248709801254\\
88.08	0.00131853612004838\\
88.09	0.00131458432903312\\
88.1	0.00131063172889655\\
88.11	0.00130667832363045\\
88.12	0.0013027241172894\\
88.13	0.00129876911399156\\
88.14	0.00129481331791955\\
88.15	0.00129085673332124\\
88.16	0.00128689936451059\\
88.17	0.00128294121586854\\
88.18	0.00127898229184386\\
88.19	0.001275022596954\\
88.2	0.00127106213578603\\
88.21	0.00126710091299747\\
88.22	0.0012631389333173\\
88.23	0.00125917620154675\\
88.24	0.00125521272256038\\
88.25	0.0012512485013069\\
88.26	0.00124728354281021\\
88.27	0.00124331785217033\\
88.28	0.0012393514345644\\
88.29	0.00123538429524767\\
88.3	0.00123141643955451\\
88.31	0.00122744787289944\\
88.32	0.00122347860077813\\
88.33	0.00121950862876849\\
88.34	0.00121553796253172\\
88.35	0.00121156660781336\\
88.36	0.0012075945704444\\
88.37	0.00120362185634239\\
88.38	0.00119964847151252\\
88.39	0.00119567442204879\\
88.4	0.00119169971413515\\
88.41	0.00118772435404663\\
88.42	0.00118374834815054\\
88.43	0.00117977170290766\\
88.44	0.00117579442487342\\
88.45	0.00117181652069915\\
88.46	0.00116783799713331\\
88.47	0.00116385886102273\\
88.48	0.00115987911931388\\
88.49	0.00115589877905415\\
88.5	0.00115191784739318\\
88.51	0.00114793633158413\\
88.52	0.00114395423898506\\
88.53	0.00113997157706025\\
88.54	0.00113598835338156\\
88.55	0.00113200457562987\\
88.56	0.00112802025159641\\
88.57	0.00112403538918426\\
88.58	0.00112004999640971\\
88.59	0.00111606408140377\\
88.6	0.00111207765241364\\
88.61	0.0011080907178042\\
88.62	0.00110410328605953\\
88.63	0.00110011536578441\\
88.64	0.00109612696570594\\
88.65	0.00109213809467506\\
88.66	0.00108814876166816\\
88.67	0.0010841589757887\\
88.68	0.00108016874626886\\
88.69	0.00107617808247115\\
88.7	0.00107218699389012\\
88.71	0.00106819549015409\\
88.72	0.0010642035810268\\
88.73	0.0010602112764092\\
88.74	0.00105621858634121\\
88.75	0.00105222552100351\\
88.76	0.00104823209071932\\
88.77	0.00104423830595629\\
88.78	0.00104024417732831\\
88.79	0.0010362497155974\\
88.8	0.00103225493167565\\
88.81	0.0010282598366271\\
88.82	0.00102426444166974\\
88.83	0.00102026875817745\\
88.84	0.00101627279768205\\
88.85	0.00101227657187529\\
88.86	0.00100828009261092\\
88.87	0.0010042833719068\\
88.88	0.00100028642194697\\
88.89	0.000996289255083821\\
88.9	0.000992291883840236\\
88.91	0.000988294320911801\\
88.92	0.000984296579169014\\
88.93	0.000980298671659545\\
88.94	0.000976300611610505\\
88.95	0.000972302412430761\\
88.96	0.00096830408771327\\
88.97	0.00096430565123745\\
88.98	0.000960307116971572\\
88.99	0.000956308499075196\\
89	0.000952309811901628\\
89.01	0.00094831107000041\\
89.02	0.000944312288119848\\
89.03	0.000940313481209567\\
89.04	0.000936314664423094\\
89.05	0.00093231585312049\\
89.06	0.000928317062871001\\
89.07	0.00092431830945575\\
89.08	0.000920319608870457\\
89.09	0.000916320977328208\\
89.1	0.000912322431262241\\
89.11	0.000908323987328787\\
89.12	0.000904325662409928\\
89.13	0.000900327473616508\\
89.14	0.000896329438291075\\
89.15	0.000892331574010856\\
89.16	0.000888333898590781\\
89.17	0.00088433643008654\\
89.18	0.000880339186797674\\
89.19	0.000876342187270717\\
89.2	0.000872345450302367\\
89.21	0.00086834899494271\\
89.22	0.000864352840498475\\
89.23	0.000860357006536337\\
89.24	0.000856361512886258\\
89.25	0.000852366379644875\\
89.26	0.000848371627178931\\
89.27	0.000844377276128745\\
89.28	0.000840383347411737\\
89.29	0.000836389862225988\\
89.3	0.000832396842053855\\
89.31	0.000828404308665615\\
89.32	0.000824412284123189\\
89.33	0.000820420790783873\\
89.34	0.000816429851304154\\
89.35	0.000812439488643547\\
89.36	0.000808449726068503\\
89.37	0.000804460587156346\\
89.38	0.000800472095799283\\
89.39	0.00079648427620845\\
89.4	0.000792497152918018\\
89.41	0.000788510750789343\\
89.42	0.000784525095015183\\
89.43	0.000780540211123951\\
89.44	0.000776556124984052\\
89.45	0.000772572862808235\\
89.46	0.000768590451158043\\
89.47	0.000764608916948286\\
89.48	0.000760628287451597\\
89.49	0.000756648590303031\\
89.5	0.000752669853504728\\
89.51	0.000748692105430644\\
89.52	0.000744715374831324\\
89.53	0.000740739690838758\\
89.54	0.000736765082971285\\
89.55	0.000732791581138566\\
89.56	0.000728819215646619\\
89.57	0.000724848017202922\\
89.58	0.000720878016921582\\
89.59	0.000716909246328558\\
89.6	0.000712941737366978\\
89.61	0.000708975522402492\\
89.62	0.000705010634228724\\
89.63	0.000701047106072768\\
89.64	0.000697084971600781\\
89.65	0.000693124264923626\\
89.66	0.000689165020602593\\
89.67	0.000685207273655208\\
89.68	0.000681251059561102\\
89.69	0.000677296414267951\\
89.7	0.00067334337419752\\
89.71	0.000669391976251744\\
89.72	0.000665442257818933\\
89.73	0.000661494256780013\\
89.74	0.00065754801151488\\
89.75	0.000653603560908827\\
89.76	0.00064966094435904\\
89.77	0.000645720201781202\\
89.78	0.000641781373616155\\
89.79	0.000637844500836674\\
89.8	0.000633909624954318\\
89.81	0.000629976788026353\\
89.82	0.000626046032662798\\
89.83	0.000622117402033523\\
89.84	0.000618190939875482\\
89.85	0.0006142666905\\
89.86	0.000610344698800168\\
89.87	0.00060642501025835\\
89.88	0.000602507670953762\\
89.89	0.000598592727570157\\
89.9	0.000594680227403622\\
89.91	0.000590770218370455\\
89.92	0.000586862749015154\\
89.93	0.000582957868518521\\
89.94	0.000579055626705838\\
89.95	0.000575156074055189\\
89.96	0.000571259261705852\\
89.97	0.000567365241466829\\
89.98	0.000563474065825466\\
89.99	0.000559585787956193\\
90	0.00055570046172938\\
90.01	0.000551818141720294\\
90.02	0.000547938883218189\\
90.03	0.000544062742235498\\
90.04	0.000540189775517152\\
90.05	0.000536320040550018\\
90.06	0.00053245359557246\\
90.07	0.000528590499584014\\
90.08	0.000524730812355203\\
90.09	0.000520874594437469\\
90.1	0.00051702190717323\\
90.11	0.00051317281270608\\
90.12	0.000509327373991101\\
90.13	0.00050548565480533\\
90.14	0.000501647719758344\\
90.15	0.000497813634302988\\
90.16	0.000493983464746247\\
90.17	0.000490157278260239\\
90.18	0.00048633514289338\\
90.19	0.000482517127581662\\
90.2	0.000478703302160098\\
90.21	0.000474893737374305\\
90.22	0.000471088504892235\\
90.23	0.00046728767731607\\
90.24	0.000463491328194246\\
90.25	0.000459699532033663\\
90.26	0.000455912364312024\\
90.27	0.000452129901490352\\
90.28	0.00044835222102566\\
90.29	0.000444579401383784\\
90.3	0.00044081152205239\\
90.31	0.000437048663554131\\
90.32	0.000433290907459994\\
90.33	0.000429538336402807\\
90.34	0.000425791034090918\\
90.35	0.000422049085322064\\
90.36	0.000418312575997401\\
90.37	0.000414581593135727\\
90.38	0.000410856224887882\\
90.39	0.000407136560551339\\
90.4	0.000403422690584977\\
90.41	0.00039971470662405\\
90.42	0.000396012701495339\\
90.43	0.000392316769232515\\
90.44	0.000388627005091683\\
90.45	0.000384943505567141\\
90.46	0.000381266368407322\\
90.47	0.000377595692630966\\
90.48	0.000373931578543485\\
90.49	0.000370274127753535\\
90.5	0.000366623443189811\\
90.51	0.000362979629118057\\
90.52	0.000359342791158292\\
90.53	0.000355713036302248\\
90.54	0.000352090472931058\\
90.55	0.000348475210833143\\
90.56	0.000344867361222342\\
90.57	0.000341267036756283\\
90.58	0.000337674351554971\\
90.59	0.00033408942121963\\
90.6	0.000330512362851783\\
90.61	0.000326943295072579\\
90.62	0.000323382338042352\\
90.63	0.000319829613480463\\
90.64	0.000316285244685365\\
90.65	0.000312749356554946\\
90.66	0.000309222075607123\\
90.67	0.000305703530000702\\
90.68	0.000302193849556511\\
90.69	0.000298693165778801\\
90.7	0.000295201611876914\\
90.71	0.000291719322787237\\
90.72	0.000288246435195443\\
90.73	0.000284783087559004\\
90.74	0.000281329420129994\\
90.75	0.000277885574978201\\
90.76	0.000274451696014512\\
90.77	0.00027102792901461\\
90.78	0.00026761442164298\\
90.79	0.000264211323477205\\
90.8	0.000260818786032587\\
90.81	0.000257436962787082\\
90.82	0.000254066009206546\\
90.83	0.000250706082770311\\
90.84	0.000247357342997084\\
90.85	0.000244019951471185\\
90.86	0.000240694071869104\\
90.87	0.000237379869986419\\
90.88	0.000234077513765035\\
90.89	0.000230787173320791\\
90.9	0.000227509020971408\\
90.91	0.000224243231264791\\
90.92	0.000220989981007707\\
90.93	0.000217749449294811\\
90.94	0.000214521817538053\\
90.95	0.000211307269496464\\
90.96	0.000208105991306298\\
90.97	0.000204918171511592\\
90.98	0.00020174400109509\\
90.99	0.000198583673509565\\
91	0.000195437384709556\\
91.01	0.000192305333183479\\
91.02	0.00018918771998618\\
91.03	0.000186084748771872\\
91.04	0.000182996625827513\\
91.05	0.000179923560106592\\
91.06	0.000176865763263355\\
91.07	0.000173823449687466\\
91.08	0.000170796836539092\\
91.09	0.000167786143784454\\
91.1	0.000164791594231817\\
91.11	0.000161813413567935\\
91.12	0.000158851830394964\\
91.13	0.000155907076267841\\
91.14	0.000152979385732135\\
91.15	0.000150068996362369\\
91.16	0.000147176148800847\\
91.17	0.000144301086796949\\
91.18	0.000141444057246939\\
91.19	0.000138605310234269\\
91.2	0.000135785099070388\\
91.21	0.000132983680336077\\
91.22	0.000130201313923295\\
91.23	0.000127438263077555\\
91.24	0.000124694794440843\\
91.25	0.00012197117809506\\
91.26	0.000119267687606027\\
91.27	0.000116584600068036\\
91.28	0.000113922196148962\\
91.29	0.000111280760135946\\
91.3	0.000108660579981636\\
91.31	0.000106061947351037\\
91.32	0.000103485157668925\\
91.33	0.000100930510167857\\
91.34	9.83983079367989e-05\\
91.35	9.58888579703446e-05\\
91.36	9.34024712185618e-05\\
91.37	9.09394626374578e-05\\
91.38	8.85001512400726e-05\\
91.39	8.60848601482202e-05\\
91.4	8.36939166448617e-05\\
91.41	8.13276522271447e-05\\
91.42	7.89864026600993e-05\\
91.43	7.66705080310001e-05\\
91.44	7.43803128044045e-05\\
91.45	7.21161658778832e-05\\
91.46	6.98784206384397e-05\\
91.47	6.76674350196249e-05\\
91.48	6.54835715593691e-05\\
91.49	6.33271974585237e-05\\
91.5	6.11986846401399e-05\\
91.51	5.90984098094664e-05\\
91.52	5.70267545147024e-05\\
91.53	5.49841052084939e-05\\
91.54	5.29708533101945e-05\\
91.55	5.09873952688894e-05\\
91.56	4.9034132627206e-05\\
91.57	4.71114720859006e-05\\
91.58	4.52198255692458e-05\\
91.59	4.33596102912184e-05\\
91.6	4.15312488225121e-05\\
91.61	3.97351691583513e-05\\
91.62	3.79718047871635e-05\\
91.63	3.62415947600721e-05\\
91.64	3.4544983761247e-05\\
91.65	3.28824221791212e-05\\
91.66	3.12543661784726e-05\\
91.67	2.96612777733827e-05\\
91.68	2.81036249010908e-05\\
91.69	2.65818814967449e-05\\
91.7	2.50965275690653e-05\\
91.71	2.36480492769257e-05\\
91.72	2.22369390068602e-05\\
91.73	2.08636954515239e-05\\
91.74	1.952882368908e-05\\
91.75	1.8232835263584e-05\\
91.76	1.69762482662948e-05\\
91.77	1.57595874180124e-05\\
91.78	1.45833841523734e-05\\
91.79	1.34481767001694e-05\\
91.8	1.2354510174676e-05\\
91.81	1.13029366580007e-05\\
91.82	1.02940152884728e-05\\
91.83	9.32831234908699e-06\\
91.84	8.40640135698484e-06\\
91.85	7.52886315403233e-06\\
91.86	6.6962859984563e-06\\
91.87	5.90926565758633e-06\\
91.88	5.16840550169803e-06\\
91.89	4.47431659896837e-06\\
91.9	3.82761781155851e-06\\
91.91	3.22893589284334e-06\\
91.92	2.67890558577899e-06\\
91.93	2.17816972244295e-06\\
91.94	1.7273793247452e-06\\
91.95	1.32719370632764e-06\\
91.96	9.78280575653426e-07\\
91.97	6.81316140322893e-07\\
91.98	4.3698521259676e-07\\
91.99	2.45981316161423e-07\\
92	1.0900679414845e-07\\
92.01	2.67729183944104e-08\\
92.02	0\\
92.03	0\\
92.04	0\\
92.05	0\\
92.06	0\\
92.07	0\\
92.08	0\\
92.09	0\\
92.1	0\\
92.11	0\\
92.12	0\\
92.13	0\\
92.14	0\\
92.15	0\\
92.16	0\\
92.17	0\\
92.18	0\\
92.19	0\\
92.2	0\\
92.21	0\\
92.22	0\\
92.23	0\\
92.24	0\\
92.25	0\\
92.26	0\\
92.27	0\\
92.28	0\\
92.29	0\\
92.3	0\\
92.31	0\\
92.32	0\\
92.33	0\\
92.34	0\\
92.35	0\\
92.36	0\\
92.37	0\\
92.38	0\\
92.39	0\\
92.4	0\\
92.41	0\\
92.42	0\\
92.43	0\\
92.44	0\\
92.45	0\\
92.46	0\\
92.47	0\\
92.48	0\\
92.49	0\\
92.5	0\\
92.51	0\\
92.52	0\\
92.53	0\\
92.54	0\\
92.55	0\\
92.56	0\\
92.57	0\\
92.58	0\\
92.59	0\\
92.6	0\\
92.61	0\\
92.62	0\\
92.63	0\\
92.64	0\\
92.65	0\\
92.66	0\\
92.67	0\\
92.68	0\\
92.69	0\\
92.7	0\\
92.71	0\\
92.72	0\\
92.73	0\\
92.74	0\\
92.75	0\\
92.76	0\\
92.77	0\\
92.78	0\\
92.79	0\\
92.8	0\\
92.81	0\\
92.82	0\\
92.83	0\\
92.84	0\\
92.85	0\\
92.86	0\\
92.87	0\\
92.88	0\\
92.89	0\\
92.9	0\\
92.91	0\\
92.92	0\\
92.93	0\\
92.94	0\\
92.95	0\\
92.96	0\\
92.97	0\\
92.98	0\\
92.99	0\\
93	0\\
93.01	0\\
93.02	0\\
93.03	0\\
93.04	0\\
93.05	0\\
93.06	0\\
93.07	0\\
93.08	0\\
93.09	0\\
93.1	0\\
93.11	0\\
93.12	0\\
93.13	0\\
93.14	0\\
93.15	0\\
93.16	0\\
93.17	0\\
93.18	0\\
93.19	0\\
93.2	0\\
93.21	0\\
93.22	0\\
93.23	0\\
93.24	0\\
93.25	0\\
93.26	0\\
93.27	0\\
93.28	0\\
93.29	0\\
93.3	0\\
93.31	0\\
93.32	0\\
93.33	0\\
93.34	0\\
93.35	0\\
93.36	0\\
93.37	0\\
93.38	0\\
93.39	0\\
93.4	0\\
93.41	0\\
93.42	0\\
93.43	0\\
93.44	0\\
93.45	0\\
93.46	0\\
93.47	0\\
93.48	0\\
93.49	0\\
93.5	0\\
93.51	0\\
93.52	0\\
93.53	0\\
93.54	0\\
93.55	0\\
93.56	0\\
93.57	0\\
93.58	0\\
93.59	0\\
93.6	0\\
93.61	0\\
93.62	0\\
93.63	0\\
93.64	0\\
93.65	0\\
93.66	0\\
93.67	0\\
93.68	0\\
93.69	0\\
93.7	0\\
93.71	0\\
93.72	0\\
93.73	0\\
93.74	0\\
93.75	0\\
93.76	0\\
93.77	0\\
93.78	0\\
93.79	0\\
93.8	0\\
93.81	0\\
93.82	0\\
93.83	0\\
93.84	0\\
93.85	0\\
93.86	0\\
93.87	0\\
93.88	0\\
93.89	0\\
93.9	0\\
93.91	0\\
93.92	0\\
93.93	0\\
93.94	0\\
93.95	0\\
93.96	0\\
93.97	0\\
93.98	0\\
93.99	0\\
94	0\\
94.01	0\\
94.02	0\\
94.03	0\\
94.04	0\\
94.05	0\\
94.06	0\\
94.07	0\\
94.08	0\\
94.09	0\\
94.1	0\\
94.11	0\\
94.12	0\\
94.13	0\\
94.14	0\\
94.15	0\\
94.16	0\\
94.17	0\\
94.18	0\\
94.19	0\\
94.2	0\\
94.21	0\\
94.22	0\\
94.23	0\\
94.24	0\\
94.25	0\\
94.26	0\\
94.27	0\\
94.28	0\\
94.29	0\\
94.3	0\\
94.31	0\\
94.32	0\\
94.33	0\\
94.34	0\\
94.35	0\\
94.36	0\\
94.37	0\\
94.38	0\\
94.39	0\\
94.4	0\\
94.41	0\\
94.42	0\\
94.43	0\\
94.44	0\\
94.45	0\\
94.46	0\\
94.47	0\\
94.48	0\\
94.49	0\\
94.5	0\\
94.51	0\\
94.52	0\\
94.53	0\\
94.54	0\\
94.55	0\\
94.56	0\\
94.57	0\\
94.58	0\\
94.59	0\\
94.6	0\\
94.61	0\\
94.62	0\\
94.63	0\\
94.64	0\\
94.65	0\\
94.66	0\\
94.67	0\\
94.68	0\\
94.69	0\\
94.7	0\\
94.71	0\\
94.72	0\\
94.73	0\\
94.74	0\\
94.75	0\\
94.76	0\\
94.77	0\\
94.78	0\\
94.79	0\\
94.8	0\\
94.81	0\\
94.82	0\\
94.83	0\\
94.84	0\\
94.85	0\\
94.86	0\\
94.87	0\\
94.88	0\\
94.89	0\\
94.9	0\\
94.91	0\\
94.92	0\\
94.93	0\\
94.94	0\\
94.95	0\\
94.96	0\\
94.97	0\\
94.98	0\\
94.99	0\\
95	0\\
95.01	0\\
95.02	0\\
95.03	0\\
95.04	0\\
95.05	0\\
95.06	0\\
95.07	0\\
95.08	0\\
95.09	0\\
95.1	0\\
95.11	0\\
95.12	0\\
95.13	0\\
95.14	0\\
95.15	0\\
95.16	0\\
95.17	0\\
95.18	0\\
95.19	0\\
95.2	0\\
95.21	0\\
95.22	0\\
95.23	0\\
95.24	0\\
95.25	0\\
95.26	0\\
95.27	0\\
95.28	0\\
95.29	0\\
95.3	0\\
95.31	0\\
95.32	0\\
95.33	0\\
95.34	0\\
95.35	0\\
95.36	0\\
95.37	0\\
95.38	0\\
95.39	0\\
95.4	0\\
95.41	0\\
95.42	0\\
95.43	0\\
95.44	0\\
95.45	0\\
95.46	0\\
95.47	0\\
95.48	0\\
95.49	0\\
95.5	0\\
95.51	0\\
95.52	0\\
95.53	0\\
95.54	0\\
95.55	0\\
95.56	0\\
95.57	0\\
95.58	0\\
95.59	0\\
95.6	0\\
95.61	0\\
95.62	0\\
95.63	0\\
95.64	0\\
95.65	0\\
95.66	0\\
95.67	0\\
95.68	0\\
95.69	0\\
95.7	0\\
95.71	0\\
95.72	0\\
95.73	0\\
95.74	0\\
95.75	0\\
95.76	0\\
95.77	0\\
95.78	0\\
95.79	0\\
95.8	0\\
95.81	0\\
95.82	0\\
95.83	0\\
95.84	0\\
95.85	0\\
95.86	0\\
95.87	0\\
95.88	0\\
95.89	0\\
95.9	0\\
95.91	0\\
95.92	0\\
95.93	0\\
95.94	0\\
95.95	0\\
95.96	0\\
95.97	0\\
95.98	0\\
95.99	0\\
96	0\\
96.01	0\\
96.02	0\\
96.03	0\\
96.04	0\\
96.05	0\\
96.06	0\\
96.07	0\\
96.08	0\\
96.09	0\\
96.1	0\\
96.11	0\\
96.12	0\\
96.13	0\\
96.14	0\\
96.15	0\\
96.16	0\\
96.17	0\\
96.18	0\\
96.19	0\\
96.2	0\\
96.21	0\\
96.22	0\\
96.23	0\\
96.24	0\\
96.25	0\\
96.26	0\\
96.27	0\\
96.28	0\\
96.29	0\\
96.3	0\\
96.31	0\\
96.32	0\\
96.33	0\\
96.34	0\\
96.35	0\\
96.36	0\\
96.37	0\\
96.38	0\\
96.39	0\\
96.4	0\\
96.41	0\\
96.42	0\\
96.43	0\\
96.44	0\\
96.45	0\\
96.46	0\\
96.47	0\\
96.48	0\\
96.49	0\\
96.5	0\\
96.51	0\\
96.52	0\\
96.53	0\\
96.54	0\\
96.55	0\\
96.56	0\\
96.57	0\\
96.58	0\\
96.59	0\\
96.6	0\\
96.61	0\\
96.62	0\\
96.63	0\\
96.64	0\\
96.65	0\\
96.66	0\\
96.67	0\\
96.68	0\\
96.69	0\\
96.7	0\\
96.71	0\\
96.72	0\\
96.73	0\\
96.74	0\\
96.75	0\\
96.76	0\\
96.77	0\\
96.78	0\\
96.79	0\\
96.8	0\\
96.81	0\\
96.82	0\\
96.83	0\\
96.84	0\\
96.85	0\\
96.86	0\\
96.87	0\\
96.88	0\\
96.89	0\\
96.9	0\\
96.91	0\\
96.92	0\\
96.93	0\\
96.94	0\\
96.95	0\\
96.96	0\\
96.97	0\\
96.98	0\\
96.99	0\\
97	0\\
97.01	0\\
97.02	0\\
97.03	0\\
97.04	0\\
97.05	0\\
97.06	0\\
97.07	0\\
97.08	0\\
97.09	0\\
97.1	0\\
97.11	0\\
97.12	0\\
97.13	0\\
97.14	0\\
97.15	0\\
97.16	0\\
97.17	0\\
97.18	0\\
97.19	0\\
97.2	0\\
97.21	0\\
97.22	0\\
97.23	0\\
97.24	0\\
97.25	0\\
97.26	0\\
97.27	0\\
97.28	0\\
97.29	0\\
97.3	0\\
97.31	0\\
97.32	0\\
97.33	0\\
97.34	0\\
97.35	0\\
97.36	0\\
97.37	0\\
97.38	0\\
97.39	0\\
97.4	0\\
97.41	0\\
97.42	0\\
97.43	0\\
97.44	0\\
97.45	0\\
97.46	0\\
97.47	0\\
97.48	0\\
97.49	0\\
97.5	0\\
97.51	0\\
97.52	0\\
97.53	0\\
97.54	0\\
97.55	0\\
97.56	0\\
97.57	0\\
97.58	0\\
97.59	0\\
97.6	0\\
97.61	0\\
97.62	0\\
97.63	0\\
97.64	0\\
97.65	0\\
97.66	0\\
97.67	0\\
97.68	0\\
97.69	0\\
97.7	0\\
97.71	0\\
97.72	0\\
97.73	0\\
97.74	0\\
97.75	0\\
97.76	0\\
97.77	0\\
97.78	0\\
97.79	0\\
97.8	0\\
97.81	0\\
97.82	0\\
97.83	0\\
97.84	0\\
97.85	0\\
97.86	0\\
97.87	0\\
97.88	0\\
97.89	0\\
97.9	0\\
97.91	0\\
97.92	0\\
97.93	0\\
97.94	0\\
97.95	0\\
97.96	0\\
97.97	0\\
97.98	0\\
97.99	0\\
98	0\\
98.01	0\\
98.02	0\\
98.03	0\\
98.04	0\\
98.05	0\\
98.06	0\\
98.07	0\\
98.08	0\\
98.09	0\\
98.1	0\\
98.11	0\\
98.12	0\\
98.13	0\\
98.14	0\\
98.15	0\\
98.16	0\\
98.17	0\\
98.18	0\\
98.19	0\\
98.2	0\\
98.21	0\\
98.22	0\\
98.23	0\\
98.24	0\\
98.25	0\\
98.26	0\\
98.27	0\\
98.28	0\\
98.29	0\\
98.3	0\\
98.31	0\\
98.32	0\\
98.33	0\\
98.34	0\\
98.35	0\\
98.36	0\\
98.37	0\\
98.38	0\\
98.39	0\\
98.4	0\\
98.41	0\\
98.42	0\\
98.43	0\\
98.44	0\\
98.45	0\\
98.46	0\\
98.47	0\\
98.48	0\\
98.49	0\\
98.5	0\\
98.51	0\\
98.52	0\\
98.53	0\\
98.54	0\\
98.55	0\\
98.56	0\\
98.57	0\\
98.58	0\\
98.59	0\\
98.6	0\\
98.61	0\\
98.62	0\\
98.63	0\\
98.64	0\\
98.65	0\\
98.66	0\\
98.67	0\\
98.68	0\\
98.69	0\\
98.7	0\\
98.71	0\\
98.72	0\\
98.73	0\\
98.74	0\\
98.75	0\\
98.76	0\\
98.77	0\\
98.78	0\\
98.79	0\\
98.8	0\\
98.81	0\\
98.82	0\\
98.83	0\\
98.84	0\\
98.85	0\\
98.86	0\\
98.87	0\\
98.88	0\\
98.89	0\\
98.9	0\\
98.91	0\\
98.92	0\\
98.93	0\\
98.94	0\\
98.95	0\\
98.96	0\\
98.97	0\\
98.98	0\\
98.99	0\\
99	0\\
99.01	0\\
99.02	0\\
99.03	0\\
99.04	0\\
99.05	0\\
99.06	0\\
99.07	0\\
99.08	0\\
99.09	0\\
99.1	0\\
99.11	0\\
99.12	0\\
99.13	0\\
99.14	0\\
99.15	0\\
99.16	0\\
99.17	0\\
99.18	0\\
99.19	0\\
99.2	0\\
99.21	0\\
99.22	0\\
99.23	0\\
99.24	0\\
99.25	0\\
99.26	0\\
99.27	0\\
99.28	0\\
99.29	0\\
99.3	0\\
99.31	0\\
99.32	0\\
99.33	0\\
99.34	0\\
99.35	0\\
99.36	0\\
99.37	0\\
99.38	0\\
99.39	0\\
99.4	0\\
99.41	0\\
99.42	0\\
99.43	0\\
99.44	0\\
99.45	0\\
99.46	0\\
99.47	0\\
99.48	0\\
99.49	0\\
99.5	0\\
99.51	0\\
99.52	0\\
99.53	0\\
99.54	0\\
99.55	0\\
99.56	0\\
99.57	0\\
99.58	0\\
99.59	0\\
99.6	0\\
99.61	0\\
99.62	0\\
99.63	0\\
99.64	0\\
99.65	0\\
99.66	0\\
99.67	0\\
99.68	0\\
99.69	0\\
99.7	0\\
99.71	0\\
99.72	0\\
99.73	0\\
99.74	0\\
99.75	0\\
99.76	0\\
99.77	0\\
99.78	0\\
99.79	0\\
99.8	0\\
99.81	0\\
99.82	0\\
99.83	0\\
99.84	0\\
99.85	0\\
99.86	0\\
99.87	0\\
99.88	0\\
99.89	0\\
99.9	0\\
99.91	0\\
99.92	0\\
99.93	0\\
99.94	0\\
99.95	0\\
99.96	0\\
99.97	0\\
99.98	0\\
99.99	0\\
100	0\\
};
\addlegendentry{$q=-1$};

\addplot [color=black,solid,forget plot]
  table[row sep=crcr]{%
0.01	0\\
0.02	0\\
0.03	0\\
0.04	0\\
0.05	0\\
0.06	0\\
0.07	0\\
0.08	0\\
0.09	0\\
0.1	0\\
0.11	0\\
0.12	0\\
0.13	0\\
0.14	0\\
0.15	0\\
0.16	0\\
0.17	0\\
0.18	0\\
0.19	0\\
0.2	0\\
0.21	0\\
0.22	0\\
0.23	0\\
0.24	0\\
0.25	0\\
0.26	0\\
0.27	0\\
0.28	0\\
0.29	0\\
0.3	0\\
0.31	0\\
0.32	0\\
0.33	0\\
0.34	0\\
0.35	0\\
0.36	0\\
0.37	0\\
0.38	0\\
0.39	0\\
0.4	0\\
0.41	0\\
0.42	0\\
0.43	0\\
0.44	0\\
0.45	0\\
0.46	0\\
0.47	0\\
0.48	0\\
0.49	0\\
0.5	0\\
0.51	0\\
0.52	0\\
0.53	0\\
0.54	0\\
0.55	0\\
0.56	0\\
0.57	0\\
0.58	0\\
0.59	0\\
0.6	0\\
0.61	0\\
0.62	0\\
0.63	0\\
0.64	0\\
0.65	0\\
0.66	0\\
0.67	0\\
0.68	0\\
0.69	0\\
0.7	0\\
0.71	0\\
0.72	0\\
0.73	0\\
0.74	0\\
0.75	0\\
0.76	0\\
0.77	0\\
0.78	0\\
0.79	0\\
0.8	0\\
0.81	0\\
0.82	0\\
0.83	0\\
0.84	0\\
0.85	0\\
0.86	0\\
0.87	0\\
0.88	0\\
0.89	0\\
0.9	0\\
0.91	0\\
0.92	0\\
0.93	0\\
0.94	0\\
0.95	0\\
0.96	0\\
0.97	0\\
0.98	0\\
0.99	0\\
1	0\\
1.01	0\\
1.02	0\\
1.03	0\\
1.04	0\\
1.05	0\\
1.06	0\\
1.07	0\\
1.08	0\\
1.09	0\\
1.1	0\\
1.11	0\\
1.12	0\\
1.13	0\\
1.14	0\\
1.15	0\\
1.16	0\\
1.17	0\\
1.18	0\\
1.19	0\\
1.2	0\\
1.21	0\\
1.22	0\\
1.23	0\\
1.24	0\\
1.25	0\\
1.26	0\\
1.27	0\\
1.28	0\\
1.29	0\\
1.3	0\\
1.31	0\\
1.32	0\\
1.33	0\\
1.34	0\\
1.35	0\\
1.36	0\\
1.37	0\\
1.38	0\\
1.39	0\\
1.4	0\\
1.41	0\\
1.42	0\\
1.43	0\\
1.44	0\\
1.45	0\\
1.46	0\\
1.47	0\\
1.48	0\\
1.49	0\\
1.5	0\\
1.51	0\\
1.52	0\\
1.53	0\\
1.54	0\\
1.55	0\\
1.56	0\\
1.57	0\\
1.58	0\\
1.59	0\\
1.6	0\\
1.61	0\\
1.62	0\\
1.63	0\\
1.64	0\\
1.65	0\\
1.66	0\\
1.67	0\\
1.68	0\\
1.69	0\\
1.7	0\\
1.71	0\\
1.72	0\\
1.73	0\\
1.74	0\\
1.75	0\\
1.76	0\\
1.77	0\\
1.78	0\\
1.79	0\\
1.8	0\\
1.81	0\\
1.82	0\\
1.83	0\\
1.84	0\\
1.85	0\\
1.86	0\\
1.87	0\\
1.88	0\\
1.89	0\\
1.9	0\\
1.91	0\\
1.92	0\\
1.93	0\\
1.94	0\\
1.95	0\\
1.96	0\\
1.97	0\\
1.98	0\\
1.99	0\\
2	0\\
2.01	0\\
2.02	0\\
2.03	0\\
2.04	0\\
2.05	0\\
2.06	0\\
2.07	0\\
2.08	0\\
2.09	0\\
2.1	0\\
2.11	0\\
2.12	0\\
2.13	0\\
2.14	0\\
2.15	0\\
2.16	0\\
2.17	0\\
2.18	0\\
2.19	0\\
2.2	0\\
2.21	0\\
2.22	0\\
2.23	0\\
2.24	0\\
2.25	0\\
2.26	0\\
2.27	0\\
2.28	0\\
2.29	0\\
2.3	0\\
2.31	0\\
2.32	0\\
2.33	0\\
2.34	0\\
2.35	0\\
2.36	0\\
2.37	0\\
2.38	0\\
2.39	0\\
2.4	0\\
2.41	0\\
2.42	0\\
2.43	0\\
2.44	0\\
2.45	0\\
2.46	0\\
2.47	0\\
2.48	0\\
2.49	0\\
2.5	0\\
2.51	0\\
2.52	0\\
2.53	0\\
2.54	0\\
2.55	0\\
2.56	0\\
2.57	0\\
2.58	0\\
2.59	0\\
2.6	0\\
2.61	0\\
2.62	0\\
2.63	0\\
2.64	0\\
2.65	0\\
2.66	0\\
2.67	0\\
2.68	0\\
2.69	0\\
2.7	0\\
2.71	0\\
2.72	0\\
2.73	0\\
2.74	0\\
2.75	0\\
2.76	0\\
2.77	0\\
2.78	0\\
2.79	0\\
2.8	0\\
2.81	0\\
2.82	0\\
2.83	0\\
2.84	0\\
2.85	0\\
2.86	0\\
2.87	0\\
2.88	0\\
2.89	0\\
2.9	0\\
2.91	0\\
2.92	0\\
2.93	0\\
2.94	0\\
2.95	0\\
2.96	0\\
2.97	0\\
2.98	0\\
2.99	0\\
3	0\\
3.01	0\\
3.02	0\\
3.03	0\\
3.04	0\\
3.05	0\\
3.06	0\\
3.07	0\\
3.08	0\\
3.09	0\\
3.1	0\\
3.11	0\\
3.12	0\\
3.13	0\\
3.14	0\\
3.15	0\\
3.16	0\\
3.17	0\\
3.18	0\\
3.19	0\\
3.2	0\\
3.21	0\\
3.22	0\\
3.23	0\\
3.24	0\\
3.25	0\\
3.26	0\\
3.27	0\\
3.28	0\\
3.29	0\\
3.3	0\\
3.31	0\\
3.32	0\\
3.33	0\\
3.34	0\\
3.35	0\\
3.36	0\\
3.37	0\\
3.38	0\\
3.39	0\\
3.4	0\\
3.41	0\\
3.42	0\\
3.43	0\\
3.44	0\\
3.45	0\\
3.46	0\\
3.47	0\\
3.48	0\\
3.49	0\\
3.5	0\\
3.51	0\\
3.52	0\\
3.53	0\\
3.54	0\\
3.55	0\\
3.56	0\\
3.57	0\\
3.58	0\\
3.59	0\\
3.6	0\\
3.61	0\\
3.62	0\\
3.63	0\\
3.64	0\\
3.65	0\\
3.66	0\\
3.67	0\\
3.68	0\\
3.69	0\\
3.7	0\\
3.71	0\\
3.72	0\\
3.73	0\\
3.74	0\\
3.75	0\\
3.76	0\\
3.77	0\\
3.78	0\\
3.79	0\\
3.8	0\\
3.81	0\\
3.82	0\\
3.83	0\\
3.84	0\\
3.85	0\\
3.86	0\\
3.87	0\\
3.88	0\\
3.89	0\\
3.9	0\\
3.91	0\\
3.92	0\\
3.93	0\\
3.94	0\\
3.95	0\\
3.96	0\\
3.97	0\\
3.98	0\\
3.99	0\\
4	0\\
4.01	0\\
4.02	0\\
4.03	0\\
4.04	0\\
4.05	0\\
4.06	0\\
4.07	0\\
4.08	0\\
4.09	0\\
4.1	0\\
4.11	0\\
4.12	0\\
4.13	0\\
4.14	0\\
4.15	0\\
4.16	0\\
4.17	0\\
4.18	0\\
4.19	0\\
4.2	0\\
4.21	0\\
4.22	0\\
4.23	0\\
4.24	0\\
4.25	0\\
4.26	0\\
4.27	0\\
4.28	0\\
4.29	0\\
4.3	0\\
4.31	0\\
4.32	0\\
4.33	0\\
4.34	0\\
4.35	0\\
4.36	0\\
4.37	0\\
4.38	0\\
4.39	0\\
4.4	0\\
4.41	0\\
4.42	0\\
4.43	0\\
4.44	0\\
4.45	0\\
4.46	0\\
4.47	0\\
4.48	0\\
4.49	0\\
4.5	0\\
4.51	0\\
4.52	0\\
4.53	0\\
4.54	0\\
4.55	0\\
4.56	0\\
4.57	0\\
4.58	0\\
4.59	0\\
4.6	0\\
4.61	0\\
4.62	0\\
4.63	0\\
4.64	0\\
4.65	0\\
4.66	0\\
4.67	0\\
4.68	0\\
4.69	0\\
4.7	0\\
4.71	0\\
4.72	0\\
4.73	0\\
4.74	0\\
4.75	0\\
4.76	0\\
4.77	0\\
4.78	0\\
4.79	0\\
4.8	0\\
4.81	0\\
4.82	0\\
4.83	0\\
4.84	0\\
4.85	0\\
4.86	0\\
4.87	0\\
4.88	0\\
4.89	0\\
4.9	0\\
4.91	0\\
4.92	0\\
4.93	0\\
4.94	0\\
4.95	0\\
4.96	0\\
4.97	0\\
4.98	0\\
4.99	0\\
5	0\\
5.01	0\\
5.02	0\\
5.03	0\\
5.04	0\\
5.05	0\\
5.06	0\\
5.07	0\\
5.08	0\\
5.09	0\\
5.1	0\\
5.11	0\\
5.12	0\\
5.13	0\\
5.14	0\\
5.15	0\\
5.16	0\\
5.17	0\\
5.18	0\\
5.19	0\\
5.2	0\\
5.21	0\\
5.22	0\\
5.23	0\\
5.24	0\\
5.25	0\\
5.26	0\\
5.27	0\\
5.28	0\\
5.29	0\\
5.3	0\\
5.31	0\\
5.32	0\\
5.33	0\\
5.34	0\\
5.35	0\\
5.36	0\\
5.37	0\\
5.38	0\\
5.39	0\\
5.4	0\\
5.41	0\\
5.42	0\\
5.43	0\\
5.44	0\\
5.45	0\\
5.46	0\\
5.47	0\\
5.48	0\\
5.49	0\\
5.5	0\\
5.51	0\\
5.52	0\\
5.53	0\\
5.54	0\\
5.55	0\\
5.56	0\\
5.57	0\\
5.58	0\\
5.59	0\\
5.6	0\\
5.61	0\\
5.62	0\\
5.63	0\\
5.64	0\\
5.65	0\\
5.66	0\\
5.67	0\\
5.68	0\\
5.69	0\\
5.7	0\\
5.71	0\\
5.72	0\\
5.73	0\\
5.74	0\\
5.75	0\\
5.76	0\\
5.77	0\\
5.78	0\\
5.79	0\\
5.8	0\\
5.81	0\\
5.82	0\\
5.83	0\\
5.84	0\\
5.85	0\\
5.86	0\\
5.87	0\\
5.88	0\\
5.89	0\\
5.9	0\\
5.91	0\\
5.92	0\\
5.93	0\\
5.94	0\\
5.95	0\\
5.96	0\\
5.97	0\\
5.98	0\\
5.99	0\\
6	0\\
6.01	0\\
6.02	0\\
6.03	0\\
6.04	0\\
6.05	0\\
6.06	0\\
6.07	0\\
6.08	0\\
6.09	0\\
6.1	0\\
6.11	0\\
6.12	0\\
6.13	0\\
6.14	0\\
6.15	0\\
6.16	0\\
6.17	0\\
6.18	0\\
6.19	0\\
6.2	0\\
6.21	0\\
6.22	0\\
6.23	0\\
6.24	0\\
6.25	0\\
6.26	0\\
6.27	0\\
6.28	0\\
6.29	0\\
6.3	0\\
6.31	0\\
6.32	0\\
6.33	0\\
6.34	0\\
6.35	0\\
6.36	0\\
6.37	0\\
6.38	0\\
6.39	0\\
6.4	0\\
6.41	0\\
6.42	0\\
6.43	0\\
6.44	0\\
6.45	0\\
6.46	0\\
6.47	0\\
6.48	0\\
6.49	0\\
6.5	0\\
6.51	0\\
6.52	0\\
6.53	0\\
6.54	0\\
6.55	0\\
6.56	0\\
6.57	0\\
6.58	0\\
6.59	0\\
6.6	0\\
6.61	0\\
6.62	0\\
6.63	0\\
6.64	0\\
6.65	0\\
6.66	0\\
6.67	0\\
6.68	0\\
6.69	0\\
6.7	0\\
6.71	0\\
6.72	0\\
6.73	0\\
6.74	0\\
6.75	0\\
6.76	0\\
6.77	0\\
6.78	0\\
6.79	0\\
6.8	0\\
6.81	0\\
6.82	0\\
6.83	0\\
6.84	0\\
6.85	0\\
6.86	0\\
6.87	0\\
6.88	0\\
6.89	0\\
6.9	0\\
6.91	0\\
6.92	0\\
6.93	0\\
6.94	0\\
6.95	0\\
6.96	0\\
6.97	0\\
6.98	0\\
6.99	0\\
7	0\\
7.01	0\\
7.02	0\\
7.03	0\\
7.04	0\\
7.05	0\\
7.06	0\\
7.07	0\\
7.08	0\\
7.09	0\\
7.1	0\\
7.11	0\\
7.12	0\\
7.13	0\\
7.14	0\\
7.15	0\\
7.16	0\\
7.17	0\\
7.18	0\\
7.19	0\\
7.2	0\\
7.21	0\\
7.22	0\\
7.23	0\\
7.24	0\\
7.25	0\\
7.26	0\\
7.27	0\\
7.28	0\\
7.29	0\\
7.3	0\\
7.31	0\\
7.32	0\\
7.33	0\\
7.34	0\\
7.35	0\\
7.36	0\\
7.37	0\\
7.38	0\\
7.39	0\\
7.4	0\\
7.41	0\\
7.42	0\\
7.43	0\\
7.44	0\\
7.45	0\\
7.46	0\\
7.47	0\\
7.48	0\\
7.49	0\\
7.5	0\\
7.51	0\\
7.52	0\\
7.53	0\\
7.54	0\\
7.55	0\\
7.56	0\\
7.57	0\\
7.58	0\\
7.59	0\\
7.6	0\\
7.61	0\\
7.62	0\\
7.63	0\\
7.64	0\\
7.65	0\\
7.66	0\\
7.67	0\\
7.68	0\\
7.69	0\\
7.7	0\\
7.71	0\\
7.72	0\\
7.73	0\\
7.74	0\\
7.75	0\\
7.76	0\\
7.77	0\\
7.78	0\\
7.79	0\\
7.8	0\\
7.81	0\\
7.82	0\\
7.83	0\\
7.84	0\\
7.85	0\\
7.86	0\\
7.87	0\\
7.88	0\\
7.89	0\\
7.9	0\\
7.91	0\\
7.92	0\\
7.93	0\\
7.94	0\\
7.95	0\\
7.96	0\\
7.97	0\\
7.98	0\\
7.99	0\\
8	0\\
8.01	0\\
8.02	0\\
8.03	0\\
8.04	0\\
8.05	0\\
8.06	0\\
8.07	0\\
8.08	0\\
8.09	0\\
8.1	0\\
8.11	0\\
8.12	0\\
8.13	0\\
8.14	0\\
8.15	0\\
8.16	0\\
8.17	0\\
8.18	0\\
8.19	0\\
8.2	0\\
8.21	0\\
8.22	0\\
8.23	0\\
8.24	0\\
8.25	0\\
8.26	0\\
8.27	0\\
8.28	0\\
8.29	0\\
8.3	0\\
8.31	0\\
8.32	0\\
8.33	0\\
8.34	0\\
8.35	0\\
8.36	0\\
8.37	0\\
8.38	0\\
8.39	0\\
8.4	0\\
8.41	0\\
8.42	0\\
8.43	0\\
8.44	0\\
8.45	0\\
8.46	0\\
8.47	0\\
8.48	0\\
8.49	0\\
8.5	0\\
8.51	0\\
8.52	0\\
8.53	0\\
8.54	0\\
8.55	0\\
8.56	0\\
8.57	0\\
8.58	0\\
8.59	0\\
8.6	0\\
8.61	0\\
8.62	0\\
8.63	0\\
8.64	0\\
8.65	0\\
8.66	0\\
8.67	0\\
8.68	0\\
8.69	0\\
8.7	0\\
8.71	0\\
8.72	0\\
8.73	0\\
8.74	0\\
8.75	0\\
8.76	0\\
8.77	0\\
8.78	0\\
8.79	0\\
8.8	0\\
8.81	0\\
8.82	0\\
8.83	0\\
8.84	0\\
8.85	0\\
8.86	0\\
8.87	0\\
8.88	0\\
8.89	0\\
8.9	0\\
8.91	0\\
8.92	0\\
8.93	0\\
8.94	0\\
8.95	0\\
8.96	0\\
8.97	0\\
8.98	0\\
8.99	0\\
9	0\\
9.01	0\\
9.02	0\\
9.03	0\\
9.04	0\\
9.05	0\\
9.06	0\\
9.07	0\\
9.08	0\\
9.09	0\\
9.1	0\\
9.11	0\\
9.12	0\\
9.13	0\\
9.14	0\\
9.15	0\\
9.16	0\\
9.17	0\\
9.18	0\\
9.19	0\\
9.2	0\\
9.21	0\\
9.22	0\\
9.23	0\\
9.24	0\\
9.25	0\\
9.26	0\\
9.27	0\\
9.28	0\\
9.29	0\\
9.3	0\\
9.31	0\\
9.32	0\\
9.33	0\\
9.34	0\\
9.35	0\\
9.36	0\\
9.37	0\\
9.38	0\\
9.39	0\\
9.4	0\\
9.41	0\\
9.42	0\\
9.43	0\\
9.44	0\\
9.45	0\\
9.46	0\\
9.47	0\\
9.48	0\\
9.49	0\\
9.5	0\\
9.51	0\\
9.52	0\\
9.53	0\\
9.54	0\\
9.55	0\\
9.56	0\\
9.57	0\\
9.58	0\\
9.59	0\\
9.6	0\\
9.61	0\\
9.62	0\\
9.63	0\\
9.64	0\\
9.65	0\\
9.66	0\\
9.67	0\\
9.68	0\\
9.69	0\\
9.7	0\\
9.71	0\\
9.72	0\\
9.73	0\\
9.74	0\\
9.75	0\\
9.76	0\\
9.77	0\\
9.78	0\\
9.79	0\\
9.8	0\\
9.81	0\\
9.82	0\\
9.83	0\\
9.84	0\\
9.85	0\\
9.86	0\\
9.87	0\\
9.88	0\\
9.89	0\\
9.9	0\\
9.91	0\\
9.92	0\\
9.93	0\\
9.94	0\\
9.95	0\\
9.96	0\\
9.97	0\\
9.98	0\\
9.99	0\\
10	0\\
10.01	0\\
10.02	0\\
10.03	0\\
10.04	0\\
10.05	0\\
10.06	0\\
10.07	0\\
10.08	0\\
10.09	0\\
10.1	0\\
10.11	0\\
10.12	0\\
10.13	0\\
10.14	0\\
10.15	0\\
10.16	0\\
10.17	0\\
10.18	0\\
10.19	0\\
10.2	0\\
10.21	0\\
10.22	0\\
10.23	0\\
10.24	0\\
10.25	0\\
10.26	0\\
10.27	0\\
10.28	0\\
10.29	0\\
10.3	0\\
10.31	0\\
10.32	0\\
10.33	0\\
10.34	0\\
10.35	0\\
10.36	0\\
10.37	0\\
10.38	0\\
10.39	0\\
10.4	0\\
10.41	0\\
10.42	0\\
10.43	0\\
10.44	0\\
10.45	0\\
10.46	0\\
10.47	0\\
10.48	0\\
10.49	0\\
10.5	0\\
10.51	0\\
10.52	0\\
10.53	0\\
10.54	0\\
10.55	0\\
10.56	0\\
10.57	0\\
10.58	0\\
10.59	0\\
10.6	0\\
10.61	0\\
10.62	0\\
10.63	0\\
10.64	0\\
10.65	0\\
10.66	0\\
10.67	0\\
10.68	0\\
10.69	0\\
10.7	0\\
10.71	0\\
10.72	0\\
10.73	0\\
10.74	0\\
10.75	0\\
10.76	0\\
10.77	0\\
10.78	0\\
10.79	0\\
10.8	0\\
10.81	0\\
10.82	0\\
10.83	0\\
10.84	0\\
10.85	0\\
10.86	0\\
10.87	0\\
10.88	0\\
10.89	0\\
10.9	0\\
10.91	0\\
10.92	0\\
10.93	0\\
10.94	0\\
10.95	0\\
10.96	0\\
10.97	0\\
10.98	0\\
10.99	0\\
11	0\\
11.01	0\\
11.02	0\\
11.03	0\\
11.04	0\\
11.05	0\\
11.06	0\\
11.07	0\\
11.08	0\\
11.09	0\\
11.1	0\\
11.11	0\\
11.12	0\\
11.13	0\\
11.14	0\\
11.15	0\\
11.16	0\\
11.17	0\\
11.18	0\\
11.19	0\\
11.2	0\\
11.21	0\\
11.22	0\\
11.23	0\\
11.24	0\\
11.25	0\\
11.26	0\\
11.27	0\\
11.28	0\\
11.29	0\\
11.3	0\\
11.31	0\\
11.32	0\\
11.33	0\\
11.34	0\\
11.35	0\\
11.36	0\\
11.37	0\\
11.38	0\\
11.39	0\\
11.4	0\\
11.41	0\\
11.42	0\\
11.43	0\\
11.44	0\\
11.45	0\\
11.46	0\\
11.47	0\\
11.48	0\\
11.49	0\\
11.5	0\\
11.51	0\\
11.52	0\\
11.53	0\\
11.54	0\\
11.55	0\\
11.56	0\\
11.57	0\\
11.58	0\\
11.59	0\\
11.6	0\\
11.61	0\\
11.62	0\\
11.63	0\\
11.64	0\\
11.65	0\\
11.66	0\\
11.67	0\\
11.68	0\\
11.69	0\\
11.7	0\\
11.71	0\\
11.72	0\\
11.73	0\\
11.74	0\\
11.75	0\\
11.76	0\\
11.77	0\\
11.78	0\\
11.79	0\\
11.8	0\\
11.81	0\\
11.82	0\\
11.83	0\\
11.84	0\\
11.85	0\\
11.86	0\\
11.87	0\\
11.88	0\\
11.89	0\\
11.9	0\\
11.91	0\\
11.92	0\\
11.93	0\\
11.94	0\\
11.95	0\\
11.96	0\\
11.97	0\\
11.98	0\\
11.99	0\\
12	0\\
12.01	0\\
12.02	0\\
12.03	0\\
12.04	0\\
12.05	0\\
12.06	0\\
12.07	0\\
12.08	0\\
12.09	0\\
12.1	0\\
12.11	0\\
12.12	0\\
12.13	0\\
12.14	0\\
12.15	0\\
12.16	0\\
12.17	0\\
12.18	0\\
12.19	0\\
12.2	0\\
12.21	0\\
12.22	0\\
12.23	0\\
12.24	0\\
12.25	0\\
12.26	0\\
12.27	0\\
12.28	0\\
12.29	0\\
12.3	0\\
12.31	0\\
12.32	0\\
12.33	0\\
12.34	0\\
12.35	0\\
12.36	0\\
12.37	0\\
12.38	0\\
12.39	0\\
12.4	0\\
12.41	0\\
12.42	0\\
12.43	0\\
12.44	0\\
12.45	0\\
12.46	0\\
12.47	0\\
12.48	0\\
12.49	0\\
12.5	0\\
12.51	0\\
12.52	0\\
12.53	0\\
12.54	0\\
12.55	0\\
12.56	0\\
12.57	0\\
12.58	0\\
12.59	0\\
12.6	0\\
12.61	0\\
12.62	0\\
12.63	0\\
12.64	0\\
12.65	0\\
12.66	0\\
12.67	0\\
12.68	0\\
12.69	0\\
12.7	0\\
12.71	0\\
12.72	0\\
12.73	0\\
12.74	0\\
12.75	0\\
12.76	0\\
12.77	0\\
12.78	0\\
12.79	0\\
12.8	0\\
12.81	0\\
12.82	0\\
12.83	0\\
12.84	0\\
12.85	0\\
12.86	0\\
12.87	0\\
12.88	0\\
12.89	0\\
12.9	0\\
12.91	0\\
12.92	0\\
12.93	0\\
12.94	0\\
12.95	0\\
12.96	0\\
12.97	0\\
12.98	0\\
12.99	0\\
13	0\\
13.01	0\\
13.02	0\\
13.03	0\\
13.04	0\\
13.05	0\\
13.06	0\\
13.07	0\\
13.08	0\\
13.09	0\\
13.1	0\\
13.11	0\\
13.12	0\\
13.13	0\\
13.14	0\\
13.15	0\\
13.16	0\\
13.17	0\\
13.18	0\\
13.19	0\\
13.2	0\\
13.21	0\\
13.22	0\\
13.23	0\\
13.24	0\\
13.25	0\\
13.26	0\\
13.27	0\\
13.28	0\\
13.29	0\\
13.3	0\\
13.31	0\\
13.32	0\\
13.33	0\\
13.34	0\\
13.35	0\\
13.36	0\\
13.37	0\\
13.38	0\\
13.39	0\\
13.4	0\\
13.41	0\\
13.42	0\\
13.43	0\\
13.44	0\\
13.45	0\\
13.46	0\\
13.47	0\\
13.48	0\\
13.49	0\\
13.5	0\\
13.51	0\\
13.52	0\\
13.53	0\\
13.54	0\\
13.55	0\\
13.56	0\\
13.57	0\\
13.58	0\\
13.59	0\\
13.6	0\\
13.61	0\\
13.62	0\\
13.63	0\\
13.64	0\\
13.65	0\\
13.66	0\\
13.67	0\\
13.68	0\\
13.69	0\\
13.7	0\\
13.71	0\\
13.72	0\\
13.73	0\\
13.74	0\\
13.75	0\\
13.76	0\\
13.77	0\\
13.78	0\\
13.79	0\\
13.8	0\\
13.81	0\\
13.82	0\\
13.83	0\\
13.84	0\\
13.85	0\\
13.86	0\\
13.87	0\\
13.88	0\\
13.89	0\\
13.9	0\\
13.91	0\\
13.92	0\\
13.93	0\\
13.94	0\\
13.95	0\\
13.96	0\\
13.97	0\\
13.98	0\\
13.99	0\\
14	0\\
14.01	0\\
14.02	0\\
14.03	0\\
14.04	0\\
14.05	0\\
14.06	0\\
14.07	0\\
14.08	0\\
14.09	0\\
14.1	0\\
14.11	0\\
14.12	0\\
14.13	0\\
14.14	0\\
14.15	0\\
14.16	0\\
14.17	0\\
14.18	0\\
14.19	0\\
14.2	0\\
14.21	0\\
14.22	0\\
14.23	0\\
14.24	0\\
14.25	0\\
14.26	0\\
14.27	0\\
14.28	0\\
14.29	0\\
14.3	0\\
14.31	0\\
14.32	0\\
14.33	0\\
14.34	0\\
14.35	0\\
14.36	0\\
14.37	0\\
14.38	0\\
14.39	0\\
14.4	0\\
14.41	0\\
14.42	0\\
14.43	0\\
14.44	0\\
14.45	0\\
14.46	0\\
14.47	0\\
14.48	0\\
14.49	0\\
14.5	0\\
14.51	0\\
14.52	0\\
14.53	0\\
14.54	0\\
14.55	0\\
14.56	0\\
14.57	0\\
14.58	0\\
14.59	0\\
14.6	0\\
14.61	0\\
14.62	0\\
14.63	0\\
14.64	0\\
14.65	0\\
14.66	0\\
14.67	0\\
14.68	0\\
14.69	0\\
14.7	0\\
14.71	0\\
14.72	0\\
14.73	0\\
14.74	0\\
14.75	0\\
14.76	0\\
14.77	0\\
14.78	0\\
14.79	0\\
14.8	0\\
14.81	0\\
14.82	0\\
14.83	0\\
14.84	0\\
14.85	0\\
14.86	0\\
14.87	0\\
14.88	0\\
14.89	0\\
14.9	0\\
14.91	0\\
14.92	0\\
14.93	0\\
14.94	0\\
14.95	0\\
14.96	0\\
14.97	0\\
14.98	0\\
14.99	0\\
15	0\\
15.01	0\\
15.02	0\\
15.03	0\\
15.04	0\\
15.05	0\\
15.06	0\\
15.07	0\\
15.08	0\\
15.09	0\\
15.1	0\\
15.11	0\\
15.12	0\\
15.13	0\\
15.14	0\\
15.15	0\\
15.16	0\\
15.17	0\\
15.18	0\\
15.19	0\\
15.2	0\\
15.21	0\\
15.22	0\\
15.23	0\\
15.24	0\\
15.25	0\\
15.26	0\\
15.27	0\\
15.28	0\\
15.29	0\\
15.3	0\\
15.31	0\\
15.32	0\\
15.33	0\\
15.34	0\\
15.35	0\\
15.36	0\\
15.37	0\\
15.38	0\\
15.39	0\\
15.4	0\\
15.41	0\\
15.42	0\\
15.43	0\\
15.44	0\\
15.45	0\\
15.46	0\\
15.47	0\\
15.48	0\\
15.49	0\\
15.5	0\\
15.51	0\\
15.52	0\\
15.53	0\\
15.54	0\\
15.55	0\\
15.56	0\\
15.57	0\\
15.58	0\\
15.59	0\\
15.6	0\\
15.61	0\\
15.62	0\\
15.63	0\\
15.64	0\\
15.65	0\\
15.66	0\\
15.67	0\\
15.68	0\\
15.69	0\\
15.7	0\\
15.71	0\\
15.72	0\\
15.73	0\\
15.74	0\\
15.75	0\\
15.76	0\\
15.77	0\\
15.78	0\\
15.79	0\\
15.8	0\\
15.81	0\\
15.82	0\\
15.83	0\\
15.84	0\\
15.85	0\\
15.86	0\\
15.87	0\\
15.88	0\\
15.89	0\\
15.9	0\\
15.91	0\\
15.92	0\\
15.93	0\\
15.94	0\\
15.95	0\\
15.96	0\\
15.97	0\\
15.98	0\\
15.99	0\\
16	0\\
16.01	0\\
16.02	0\\
16.03	0\\
16.04	0\\
16.05	0\\
16.06	0\\
16.07	0\\
16.08	0\\
16.09	0\\
16.1	0\\
16.11	0\\
16.12	0\\
16.13	0\\
16.14	0\\
16.15	0\\
16.16	0\\
16.17	0\\
16.18	0\\
16.19	0\\
16.2	0\\
16.21	0\\
16.22	0\\
16.23	0\\
16.24	0\\
16.25	0\\
16.26	0\\
16.27	0\\
16.28	0\\
16.29	0\\
16.3	0\\
16.31	0\\
16.32	0\\
16.33	0\\
16.34	0\\
16.35	0\\
16.36	0\\
16.37	0\\
16.38	0\\
16.39	0\\
16.4	0\\
16.41	0\\
16.42	0\\
16.43	0\\
16.44	0\\
16.45	0\\
16.46	0\\
16.47	0\\
16.48	0\\
16.49	0\\
16.5	0\\
16.51	0\\
16.52	0\\
16.53	0\\
16.54	0\\
16.55	0\\
16.56	0\\
16.57	0\\
16.58	0\\
16.59	0\\
16.6	0\\
16.61	0\\
16.62	0\\
16.63	0\\
16.64	0\\
16.65	0\\
16.66	0\\
16.67	0\\
16.68	0\\
16.69	0\\
16.7	0\\
16.71	0\\
16.72	0\\
16.73	0\\
16.74	0\\
16.75	0\\
16.76	0\\
16.77	0\\
16.78	0\\
16.79	0\\
16.8	0\\
16.81	0\\
16.82	0\\
16.83	0\\
16.84	0\\
16.85	0\\
16.86	0\\
16.87	0\\
16.88	0\\
16.89	0\\
16.9	0\\
16.91	0\\
16.92	0\\
16.93	0\\
16.94	0\\
16.95	0\\
16.96	0\\
16.97	0\\
16.98	0\\
16.99	0\\
17	0\\
17.01	0\\
17.02	0\\
17.03	0\\
17.04	0\\
17.05	0\\
17.06	0\\
17.07	0\\
17.08	0\\
17.09	0\\
17.1	0\\
17.11	0\\
17.12	0\\
17.13	0\\
17.14	0\\
17.15	0\\
17.16	0\\
17.17	0\\
17.18	0\\
17.19	0\\
17.2	0\\
17.21	0\\
17.22	0\\
17.23	0\\
17.24	0\\
17.25	0\\
17.26	0\\
17.27	0\\
17.28	0\\
17.29	0\\
17.3	0\\
17.31	0\\
17.32	0\\
17.33	0\\
17.34	0\\
17.35	0\\
17.36	0\\
17.37	0\\
17.38	0\\
17.39	0\\
17.4	0\\
17.41	0\\
17.42	0\\
17.43	0\\
17.44	0\\
17.45	0\\
17.46	0\\
17.47	0\\
17.48	0\\
17.49	0\\
17.5	0\\
17.51	0\\
17.52	0\\
17.53	0\\
17.54	0\\
17.55	0\\
17.56	0\\
17.57	0\\
17.58	0\\
17.59	0\\
17.6	0\\
17.61	0\\
17.62	0\\
17.63	0\\
17.64	0\\
17.65	0\\
17.66	0\\
17.67	0\\
17.68	0\\
17.69	0\\
17.7	0\\
17.71	0\\
17.72	0\\
17.73	0\\
17.74	0\\
17.75	0\\
17.76	0\\
17.77	0\\
17.78	0\\
17.79	0\\
17.8	0\\
17.81	0\\
17.82	0\\
17.83	0\\
17.84	0\\
17.85	0\\
17.86	0\\
17.87	0\\
17.88	0\\
17.89	0\\
17.9	0\\
17.91	0\\
17.92	0\\
17.93	0\\
17.94	0\\
17.95	0\\
17.96	0\\
17.97	0\\
17.98	0\\
17.99	0\\
18	0\\
18.01	0\\
18.02	0\\
18.03	0\\
18.04	0\\
18.05	0\\
18.06	0\\
18.07	0\\
18.08	0\\
18.09	0\\
18.1	0\\
18.11	0\\
18.12	0\\
18.13	0\\
18.14	0\\
18.15	0\\
18.16	0\\
18.17	0\\
18.18	0\\
18.19	0\\
18.2	0\\
18.21	0\\
18.22	0\\
18.23	0\\
18.24	0\\
18.25	0\\
18.26	0\\
18.27	0\\
18.28	0\\
18.29	0\\
18.3	0\\
18.31	0\\
18.32	0\\
18.33	0\\
18.34	0\\
18.35	0\\
18.36	0\\
18.37	0\\
18.38	0\\
18.39	0\\
18.4	0\\
18.41	0\\
18.42	0\\
18.43	0\\
18.44	0\\
18.45	0\\
18.46	0\\
18.47	0\\
18.48	0\\
18.49	0\\
18.5	0\\
18.51	0\\
18.52	0\\
18.53	0\\
18.54	0\\
18.55	0\\
18.56	0\\
18.57	0\\
18.58	0\\
18.59	0\\
18.6	0\\
18.61	0\\
18.62	0\\
18.63	0\\
18.64	0\\
18.65	0\\
18.66	0\\
18.67	0\\
18.68	0\\
18.69	0\\
18.7	0\\
18.71	0\\
18.72	0\\
18.73	0\\
18.74	0\\
18.75	0\\
18.76	0\\
18.77	0\\
18.78	0\\
18.79	0\\
18.8	0\\
18.81	0\\
18.82	0\\
18.83	0\\
18.84	0\\
18.85	0\\
18.86	0\\
18.87	0\\
18.88	0\\
18.89	0\\
18.9	0\\
18.91	0\\
18.92	0\\
18.93	0\\
18.94	0\\
18.95	0\\
18.96	0\\
18.97	0\\
18.98	0\\
18.99	0\\
19	0\\
19.01	0\\
19.02	0\\
19.03	0\\
19.04	0\\
19.05	0\\
19.06	0\\
19.07	0\\
19.08	0\\
19.09	0\\
19.1	0\\
19.11	0\\
19.12	0\\
19.13	0\\
19.14	0\\
19.15	0\\
19.16	0\\
19.17	0\\
19.18	0\\
19.19	0\\
19.2	0\\
19.21	0\\
19.22	0\\
19.23	0\\
19.24	0\\
19.25	0\\
19.26	0\\
19.27	0\\
19.28	0\\
19.29	0\\
19.3	0\\
19.31	0\\
19.32	0\\
19.33	0\\
19.34	0\\
19.35	0\\
19.36	0\\
19.37	0\\
19.38	0\\
19.39	0\\
19.4	0\\
19.41	0\\
19.42	0\\
19.43	0\\
19.44	0\\
19.45	0\\
19.46	0\\
19.47	0\\
19.48	0\\
19.49	0\\
19.5	0\\
19.51	0\\
19.52	0\\
19.53	0\\
19.54	0\\
19.55	0\\
19.56	0\\
19.57	0\\
19.58	0\\
19.59	0\\
19.6	0\\
19.61	0\\
19.62	0\\
19.63	0\\
19.64	0\\
19.65	0\\
19.66	0\\
19.67	0\\
19.68	0\\
19.69	0\\
19.7	0\\
19.71	0\\
19.72	0\\
19.73	0\\
19.74	0\\
19.75	0\\
19.76	0\\
19.77	0\\
19.78	0\\
19.79	0\\
19.8	0\\
19.81	0\\
19.82	0\\
19.83	0\\
19.84	0\\
19.85	0\\
19.86	0\\
19.87	0\\
19.88	0\\
19.89	0\\
19.9	0\\
19.91	0\\
19.92	0\\
19.93	0\\
19.94	0\\
19.95	0\\
19.96	0\\
19.97	0\\
19.98	0\\
19.99	0\\
20	0\\
20.01	0\\
20.02	0\\
20.03	0\\
20.04	0\\
20.05	0\\
20.06	0\\
20.07	0\\
20.08	0\\
20.09	0\\
20.1	0\\
20.11	0\\
20.12	0\\
20.13	0\\
20.14	0\\
20.15	0\\
20.16	0\\
20.17	0\\
20.18	0\\
20.19	0\\
20.2	0\\
20.21	0\\
20.22	0\\
20.23	0\\
20.24	0\\
20.25	0\\
20.26	0\\
20.27	0\\
20.28	0\\
20.29	0\\
20.3	0\\
20.31	0\\
20.32	0\\
20.33	0\\
20.34	0\\
20.35	0\\
20.36	0\\
20.37	0\\
20.38	0\\
20.39	0\\
20.4	0\\
20.41	0\\
20.42	0\\
20.43	0\\
20.44	0\\
20.45	0\\
20.46	0\\
20.47	0\\
20.48	0\\
20.49	0\\
20.5	0\\
20.51	0\\
20.52	0\\
20.53	0\\
20.54	0\\
20.55	0\\
20.56	0\\
20.57	0\\
20.58	0\\
20.59	0\\
20.6	0\\
20.61	0\\
20.62	0\\
20.63	0\\
20.64	0\\
20.65	0\\
20.66	0\\
20.67	0\\
20.68	0\\
20.69	0\\
20.7	0\\
20.71	0\\
20.72	0\\
20.73	0\\
20.74	0\\
20.75	0\\
20.76	0\\
20.77	0\\
20.78	0\\
20.79	0\\
20.8	0\\
20.81	0\\
20.82	0\\
20.83	0\\
20.84	0\\
20.85	0\\
20.86	0\\
20.87	0\\
20.88	0\\
20.89	0\\
20.9	0\\
20.91	0\\
20.92	0\\
20.93	0\\
20.94	0\\
20.95	0\\
20.96	0\\
20.97	0\\
20.98	0\\
20.99	0\\
21	0\\
21.01	0\\
21.02	0\\
21.03	0\\
21.04	0\\
21.05	0\\
21.06	0\\
21.07	0\\
21.08	0\\
21.09	0\\
21.1	0\\
21.11	0\\
21.12	0\\
21.13	0\\
21.14	0\\
21.15	0\\
21.16	0\\
21.17	0\\
21.18	0\\
21.19	0\\
21.2	0\\
21.21	0\\
21.22	0\\
21.23	0\\
21.24	0\\
21.25	0\\
21.26	0\\
21.27	0\\
21.28	0\\
21.29	0\\
21.3	0\\
21.31	0\\
21.32	0\\
21.33	0\\
21.34	0\\
21.35	0\\
21.36	0\\
21.37	0\\
21.38	0\\
21.39	0\\
21.4	0\\
21.41	0\\
21.42	0\\
21.43	0\\
21.44	0\\
21.45	0\\
21.46	0\\
21.47	0\\
21.48	0\\
21.49	0\\
21.5	0\\
21.51	0\\
21.52	0\\
21.53	0\\
21.54	0\\
21.55	0\\
21.56	0\\
21.57	0\\
21.58	0\\
21.59	0\\
21.6	0\\
21.61	0\\
21.62	0\\
21.63	0\\
21.64	0\\
21.65	0\\
21.66	0\\
21.67	0\\
21.68	0\\
21.69	0\\
21.7	0\\
21.71	0\\
21.72	0\\
21.73	0\\
21.74	0\\
21.75	0\\
21.76	0\\
21.77	0\\
21.78	0\\
21.79	0\\
21.8	0\\
21.81	0\\
21.82	0\\
21.83	0\\
21.84	0\\
21.85	0\\
21.86	0\\
21.87	0\\
21.88	0\\
21.89	0\\
21.9	0\\
21.91	0\\
21.92	0\\
21.93	0\\
21.94	0\\
21.95	0\\
21.96	0\\
21.97	0\\
21.98	0\\
21.99	0\\
22	0\\
22.01	0\\
22.02	0\\
22.03	0\\
22.04	0\\
22.05	0\\
22.06	0\\
22.07	0\\
22.08	0\\
22.09	0\\
22.1	0\\
22.11	0\\
22.12	0\\
22.13	0\\
22.14	0\\
22.15	0\\
22.16	0\\
22.17	0\\
22.18	0\\
22.19	0\\
22.2	0\\
22.21	0\\
22.22	0\\
22.23	0\\
22.24	0\\
22.25	0\\
22.26	0\\
22.27	0\\
22.28	0\\
22.29	0\\
22.3	0\\
22.31	0\\
22.32	0\\
22.33	0\\
22.34	0\\
22.35	0\\
22.36	0\\
22.37	0\\
22.38	0\\
22.39	0\\
22.4	0\\
22.41	0\\
22.42	0\\
22.43	0\\
22.44	0\\
22.45	0\\
22.46	0\\
22.47	0\\
22.48	0\\
22.49	0\\
22.5	0\\
22.51	0\\
22.52	0\\
22.53	0\\
22.54	0\\
22.55	0\\
22.56	0\\
22.57	0\\
22.58	0\\
22.59	0\\
22.6	0\\
22.61	0\\
22.62	0\\
22.63	0\\
22.64	0\\
22.65	0\\
22.66	0\\
22.67	0\\
22.68	0\\
22.69	0\\
22.7	0\\
22.71	0\\
22.72	0\\
22.73	0\\
22.74	0\\
22.75	0\\
22.76	0\\
22.77	0\\
22.78	0\\
22.79	0\\
22.8	0\\
22.81	0\\
22.82	0\\
22.83	0\\
22.84	0\\
22.85	0\\
22.86	0\\
22.87	0\\
22.88	0\\
22.89	0\\
22.9	0\\
22.91	0\\
22.92	0\\
22.93	0\\
22.94	0\\
22.95	0\\
22.96	0\\
22.97	0\\
22.98	0\\
22.99	0\\
23	0\\
23.01	0\\
23.02	0\\
23.03	0\\
23.04	0\\
23.05	0\\
23.06	0\\
23.07	0\\
23.08	0\\
23.09	0\\
23.1	0\\
23.11	0\\
23.12	0\\
23.13	0\\
23.14	0\\
23.15	0\\
23.16	0\\
23.17	0\\
23.18	0\\
23.19	0\\
23.2	0\\
23.21	0\\
23.22	0\\
23.23	0\\
23.24	0\\
23.25	0\\
23.26	0\\
23.27	0\\
23.28	0\\
23.29	0\\
23.3	0\\
23.31	0\\
23.32	0\\
23.33	0\\
23.34	0\\
23.35	0\\
23.36	0\\
23.37	0\\
23.38	0\\
23.39	0\\
23.4	0\\
23.41	0\\
23.42	0\\
23.43	0\\
23.44	0\\
23.45	0\\
23.46	0\\
23.47	0\\
23.48	0\\
23.49	0\\
23.5	0\\
23.51	0\\
23.52	0\\
23.53	0\\
23.54	0\\
23.55	0\\
23.56	0\\
23.57	0\\
23.58	0\\
23.59	0\\
23.6	0\\
23.61	0\\
23.62	0\\
23.63	0\\
23.64	0\\
23.65	0\\
23.66	0\\
23.67	0\\
23.68	0\\
23.69	0\\
23.7	0\\
23.71	0\\
23.72	0\\
23.73	0\\
23.74	0\\
23.75	0\\
23.76	0\\
23.77	0\\
23.78	0\\
23.79	0\\
23.8	0\\
23.81	0\\
23.82	0\\
23.83	0\\
23.84	0\\
23.85	0\\
23.86	0\\
23.87	0\\
23.88	0\\
23.89	0\\
23.9	0\\
23.91	0\\
23.92	0\\
23.93	0\\
23.94	0\\
23.95	0\\
23.96	0\\
23.97	0\\
23.98	0\\
23.99	0\\
24	0\\
24.01	0\\
24.02	0\\
24.03	0\\
24.04	0\\
24.05	0\\
24.06	0\\
24.07	0\\
24.08	0\\
24.09	0\\
24.1	0\\
24.11	0\\
24.12	0\\
24.13	0\\
24.14	0\\
24.15	0\\
24.16	0\\
24.17	0\\
24.18	0\\
24.19	0\\
24.2	0\\
24.21	0\\
24.22	0\\
24.23	0\\
24.24	0\\
24.25	0\\
24.26	0\\
24.27	0\\
24.28	0\\
24.29	0\\
24.3	0\\
24.31	0\\
24.32	0\\
24.33	0\\
24.34	0\\
24.35	0\\
24.36	0\\
24.37	0\\
24.38	0\\
24.39	0\\
24.4	0\\
24.41	0\\
24.42	0\\
24.43	0\\
24.44	0\\
24.45	0\\
24.46	0\\
24.47	0\\
24.48	0\\
24.49	0\\
24.5	0\\
24.51	0\\
24.52	0\\
24.53	0\\
24.54	0\\
24.55	0\\
24.56	0\\
24.57	0\\
24.58	0\\
24.59	0\\
24.6	0\\
24.61	0\\
24.62	0\\
24.63	0\\
24.64	0\\
24.65	0\\
24.66	0\\
24.67	0\\
24.68	0\\
24.69	0\\
24.7	0\\
24.71	0\\
24.72	0\\
24.73	0\\
24.74	0\\
24.75	0\\
24.76	0\\
24.77	0\\
24.78	0\\
24.79	0\\
24.8	0\\
24.81	0\\
24.82	0\\
24.83	0\\
24.84	0\\
24.85	0\\
24.86	0\\
24.87	0\\
24.88	0\\
24.89	0\\
24.9	0\\
24.91	0\\
24.92	0\\
24.93	0\\
24.94	0\\
24.95	0\\
24.96	0\\
24.97	0\\
24.98	0\\
24.99	0\\
25	0\\
25.01	0\\
25.02	0\\
25.03	0\\
25.04	0\\
25.05	0\\
25.06	0\\
25.07	0\\
25.08	0\\
25.09	0\\
25.1	0\\
25.11	0\\
25.12	0\\
25.13	0\\
25.14	0\\
25.15	0\\
25.16	0\\
25.17	0\\
25.18	0\\
25.19	0\\
25.2	0\\
25.21	0\\
25.22	0\\
25.23	0\\
25.24	0\\
25.25	0\\
25.26	0\\
25.27	0\\
25.28	0\\
25.29	0\\
25.3	0\\
25.31	0\\
25.32	0\\
25.33	0\\
25.34	0\\
25.35	0\\
25.36	0\\
25.37	0\\
25.38	0\\
25.39	0\\
25.4	0\\
25.41	0\\
25.42	0\\
25.43	0\\
25.44	0\\
25.45	0\\
25.46	0\\
25.47	0\\
25.48	0\\
25.49	0\\
25.5	0\\
25.51	0\\
25.52	0\\
25.53	0\\
25.54	0\\
25.55	0\\
25.56	0\\
25.57	0\\
25.58	0\\
25.59	0\\
25.6	0\\
25.61	0\\
25.62	0\\
25.63	0\\
25.64	0\\
25.65	0\\
25.66	0\\
25.67	0\\
25.68	0\\
25.69	0\\
25.7	0\\
25.71	0\\
25.72	0\\
25.73	0\\
25.74	0\\
25.75	0\\
25.76	0\\
25.77	0\\
25.78	0\\
25.79	0\\
25.8	0\\
25.81	0\\
25.82	0\\
25.83	0\\
25.84	0\\
25.85	0\\
25.86	0\\
25.87	0\\
25.88	0\\
25.89	0\\
25.9	0\\
25.91	0\\
25.92	0\\
25.93	0\\
25.94	0\\
25.95	0\\
25.96	0\\
25.97	0\\
25.98	0\\
25.99	0\\
26	0\\
26.01	0\\
26.02	0\\
26.03	0\\
26.04	0\\
26.05	0\\
26.06	0\\
26.07	0\\
26.08	0\\
26.09	0\\
26.1	0\\
26.11	0\\
26.12	0\\
26.13	0\\
26.14	0\\
26.15	0\\
26.16	0\\
26.17	0\\
26.18	0\\
26.19	0\\
26.2	0\\
26.21	0\\
26.22	0\\
26.23	0\\
26.24	0\\
26.25	0\\
26.26	0\\
26.27	0\\
26.28	0\\
26.29	0\\
26.3	0\\
26.31	0\\
26.32	0\\
26.33	0\\
26.34	0\\
26.35	0\\
26.36	0\\
26.37	0\\
26.38	0\\
26.39	0\\
26.4	0\\
26.41	0\\
26.42	0\\
26.43	0\\
26.44	0\\
26.45	0\\
26.46	0\\
26.47	0\\
26.48	0\\
26.49	0\\
26.5	0\\
26.51	0\\
26.52	0\\
26.53	0\\
26.54	0\\
26.55	0\\
26.56	0\\
26.57	0\\
26.58	0\\
26.59	0\\
26.6	0\\
26.61	0\\
26.62	0\\
26.63	0\\
26.64	0\\
26.65	0\\
26.66	0\\
26.67	0\\
26.68	0\\
26.69	0\\
26.7	0\\
26.71	0\\
26.72	0\\
26.73	0\\
26.74	0\\
26.75	0\\
26.76	0\\
26.77	0\\
26.78	0\\
26.79	0\\
26.8	0\\
26.81	0\\
26.82	0\\
26.83	0\\
26.84	0\\
26.85	0\\
26.86	0\\
26.87	0\\
26.88	0\\
26.89	0\\
26.9	0\\
26.91	0\\
26.92	0\\
26.93	0\\
26.94	0\\
26.95	0\\
26.96	0\\
26.97	0\\
26.98	0\\
26.99	0\\
27	0\\
27.01	0\\
27.02	0\\
27.03	0\\
27.04	0\\
27.05	0\\
27.06	0\\
27.07	0\\
27.08	0\\
27.09	0\\
27.1	0\\
27.11	0\\
27.12	0\\
27.13	0\\
27.14	0\\
27.15	0\\
27.16	0\\
27.17	0\\
27.18	0\\
27.19	0\\
27.2	0\\
27.21	0\\
27.22	0\\
27.23	0\\
27.24	0\\
27.25	0\\
27.26	0\\
27.27	0\\
27.28	0\\
27.29	0\\
27.3	0\\
27.31	0\\
27.32	0\\
27.33	0\\
27.34	0\\
27.35	0\\
27.36	0\\
27.37	0\\
27.38	0\\
27.39	0\\
27.4	0\\
27.41	0\\
27.42	0\\
27.43	0\\
27.44	0\\
27.45	0\\
27.46	0\\
27.47	0\\
27.48	0\\
27.49	0\\
27.5	0\\
27.51	0\\
27.52	0\\
27.53	0\\
27.54	0\\
27.55	0\\
27.56	0\\
27.57	0\\
27.58	0\\
27.59	0\\
27.6	0\\
27.61	0\\
27.62	0\\
27.63	0\\
27.64	0\\
27.65	0\\
27.66	0\\
27.67	0\\
27.68	0\\
27.69	0\\
27.7	0\\
27.71	0\\
27.72	0\\
27.73	0\\
27.74	0\\
27.75	0\\
27.76	0\\
27.77	0\\
27.78	0\\
27.79	0\\
27.8	0\\
27.81	0\\
27.82	0\\
27.83	0\\
27.84	0\\
27.85	0\\
27.86	0\\
27.87	0\\
27.88	0\\
27.89	0\\
27.9	0\\
27.91	0\\
27.92	0\\
27.93	0\\
27.94	0\\
27.95	0\\
27.96	0\\
27.97	0\\
27.98	0\\
27.99	0\\
28	0\\
28.01	0\\
28.02	0\\
28.03	0\\
28.04	0\\
28.05	0\\
28.06	0\\
28.07	0\\
28.08	0\\
28.09	0\\
28.1	0\\
28.11	0\\
28.12	0\\
28.13	0\\
28.14	0\\
28.15	0\\
28.16	0\\
28.17	0\\
28.18	0\\
28.19	0\\
28.2	0\\
28.21	0\\
28.22	0\\
28.23	0\\
28.24	0\\
28.25	0\\
28.26	0\\
28.27	0\\
28.28	0\\
28.29	0\\
28.3	0\\
28.31	0\\
28.32	0\\
28.33	0\\
28.34	0\\
28.35	0\\
28.36	0\\
28.37	0\\
28.38	0\\
28.39	0\\
28.4	0\\
28.41	0\\
28.42	0\\
28.43	0\\
28.44	0\\
28.45	0\\
28.46	0\\
28.47	0\\
28.48	0\\
28.49	0\\
28.5	0\\
28.51	0\\
28.52	0\\
28.53	0\\
28.54	0\\
28.55	0\\
28.56	0\\
28.57	0\\
28.58	0\\
28.59	0\\
28.6	0\\
28.61	0\\
28.62	0\\
28.63	0\\
28.64	0\\
28.65	0\\
28.66	0\\
28.67	0\\
28.68	0\\
28.69	0\\
28.7	0\\
28.71	0\\
28.72	0\\
28.73	0\\
28.74	0\\
28.75	0\\
28.76	0\\
28.77	0\\
28.78	0\\
28.79	0\\
28.8	0\\
28.81	0\\
28.82	0\\
28.83	0\\
28.84	0\\
28.85	0\\
28.86	0\\
28.87	0\\
28.88	0\\
28.89	0\\
28.9	0\\
28.91	0\\
28.92	0\\
28.93	0\\
28.94	0\\
28.95	0\\
28.96	0\\
28.97	0\\
28.98	0\\
28.99	0\\
29	0\\
29.01	0\\
29.02	0\\
29.03	0\\
29.04	0\\
29.05	0\\
29.06	0\\
29.07	0\\
29.08	0\\
29.09	0\\
29.1	0\\
29.11	0\\
29.12	0\\
29.13	0\\
29.14	0\\
29.15	0\\
29.16	0\\
29.17	0\\
29.18	0\\
29.19	0\\
29.2	0\\
29.21	0\\
29.22	0\\
29.23	0\\
29.24	0\\
29.25	0\\
29.26	0\\
29.27	0\\
29.28	0\\
29.29	0\\
29.3	0\\
29.31	0\\
29.32	0\\
29.33	0\\
29.34	0\\
29.35	0\\
29.36	0\\
29.37	0\\
29.38	0\\
29.39	0\\
29.4	0\\
29.41	0\\
29.42	0\\
29.43	0\\
29.44	0\\
29.45	0\\
29.46	0\\
29.47	0\\
29.48	0\\
29.49	0\\
29.5	0\\
29.51	0\\
29.52	0\\
29.53	0\\
29.54	0\\
29.55	0\\
29.56	0\\
29.57	0\\
29.58	0\\
29.59	0\\
29.6	0\\
29.61	0\\
29.62	0\\
29.63	0\\
29.64	0\\
29.65	0\\
29.66	0\\
29.67	0\\
29.68	0\\
29.69	0\\
29.7	0\\
29.71	0\\
29.72	0\\
29.73	0\\
29.74	0\\
29.75	0\\
29.76	0\\
29.77	0\\
29.78	0\\
29.79	0\\
29.8	0\\
29.81	0\\
29.82	0\\
29.83	0\\
29.84	0\\
29.85	0\\
29.86	0\\
29.87	0\\
29.88	0\\
29.89	0\\
29.9	0\\
29.91	0\\
29.92	0\\
29.93	0\\
29.94	0\\
29.95	0\\
29.96	0\\
29.97	0\\
29.98	0\\
29.99	0\\
30	0\\
30.01	0\\
30.02	0\\
30.03	0\\
30.04	0\\
30.05	0\\
30.06	0\\
30.07	0\\
30.08	0\\
30.09	0\\
30.1	0\\
30.11	0\\
30.12	0\\
30.13	0\\
30.14	0\\
30.15	0\\
30.16	0\\
30.17	0\\
30.18	0\\
30.19	0\\
30.2	0\\
30.21	0\\
30.22	0\\
30.23	0\\
30.24	0\\
30.25	0\\
30.26	0\\
30.27	0\\
30.28	0\\
30.29	0\\
30.3	0\\
30.31	0\\
30.32	0\\
30.33	0\\
30.34	0\\
30.35	0\\
30.36	0\\
30.37	0\\
30.38	0\\
30.39	0\\
30.4	0\\
30.41	0\\
30.42	0\\
30.43	0\\
30.44	0\\
30.45	0\\
30.46	0\\
30.47	0\\
30.48	0\\
30.49	0\\
30.5	0\\
30.51	0\\
30.52	0\\
30.53	0\\
30.54	0\\
30.55	0\\
30.56	0\\
30.57	0\\
30.58	0\\
30.59	0\\
30.6	0\\
30.61	0\\
30.62	0\\
30.63	0\\
30.64	0\\
30.65	0\\
30.66	0\\
30.67	0\\
30.68	0\\
30.69	0\\
30.7	0\\
30.71	0\\
30.72	0\\
30.73	0\\
30.74	0\\
30.75	0\\
30.76	0\\
30.77	0\\
30.78	0\\
30.79	0\\
30.8	0\\
30.81	0\\
30.82	0\\
30.83	0\\
30.84	0\\
30.85	0\\
30.86	0\\
30.87	0\\
30.88	0\\
30.89	0\\
30.9	0\\
30.91	0\\
30.92	0\\
30.93	0\\
30.94	0\\
30.95	0\\
30.96	0\\
30.97	0\\
30.98	0\\
30.99	0\\
31	0\\
31.01	0\\
31.02	0\\
31.03	0\\
31.04	0\\
31.05	0\\
31.06	0\\
31.07	0\\
31.08	0\\
31.09	0\\
31.1	0\\
31.11	0\\
31.12	0\\
31.13	0\\
31.14	0\\
31.15	0\\
31.16	0\\
31.17	0\\
31.18	0\\
31.19	0\\
31.2	0\\
31.21	0\\
31.22	0\\
31.23	0\\
31.24	0\\
31.25	0\\
31.26	0\\
31.27	0\\
31.28	0\\
31.29	0\\
31.3	0\\
31.31	0\\
31.32	0\\
31.33	0\\
31.34	0\\
31.35	0\\
31.36	0\\
31.37	0\\
31.38	0\\
31.39	0\\
31.4	0\\
31.41	0\\
31.42	0\\
31.43	0\\
31.44	0\\
31.45	0\\
31.46	0\\
31.47	0\\
31.48	0\\
31.49	0\\
31.5	0\\
31.51	0\\
31.52	0\\
31.53	0\\
31.54	0\\
31.55	0\\
31.56	0\\
31.57	0\\
31.58	0\\
31.59	0\\
31.6	0\\
31.61	0\\
31.62	0\\
31.63	0\\
31.64	0\\
31.65	0\\
31.66	0\\
31.67	0\\
31.68	0\\
31.69	0\\
31.7	0\\
31.71	0\\
31.72	0\\
31.73	0\\
31.74	0\\
31.75	0\\
31.76	0\\
31.77	0\\
31.78	0\\
31.79	0\\
31.8	0\\
31.81	0\\
31.82	0\\
31.83	0\\
31.84	0\\
31.85	0\\
31.86	0\\
31.87	0\\
31.88	0\\
31.89	0\\
31.9	0\\
31.91	0\\
31.92	0\\
31.93	0\\
31.94	0\\
31.95	0\\
31.96	0\\
31.97	0\\
31.98	0\\
31.99	0\\
32	0\\
32.01	0\\
32.02	0\\
32.03	0\\
32.04	0\\
32.05	0\\
32.06	0\\
32.07	0\\
32.08	0\\
32.09	0\\
32.1	0\\
32.11	0\\
32.12	0\\
32.13	0\\
32.14	0\\
32.15	0\\
32.16	0\\
32.17	0\\
32.18	0\\
32.19	0\\
32.2	0\\
32.21	0\\
32.22	0\\
32.23	0\\
32.24	0\\
32.25	0\\
32.26	0\\
32.27	0\\
32.28	0\\
32.29	0\\
32.3	0\\
32.31	0\\
32.32	0\\
32.33	0\\
32.34	0\\
32.35	0\\
32.36	0\\
32.37	0\\
32.38	0\\
32.39	0\\
32.4	0\\
32.41	0\\
32.42	0\\
32.43	0\\
32.44	0\\
32.45	0\\
32.46	0\\
32.47	0\\
32.48	0\\
32.49	0\\
32.5	0\\
32.51	0\\
32.52	0\\
32.53	0\\
32.54	0\\
32.55	0\\
32.56	0\\
32.57	0\\
32.58	0\\
32.59	0\\
32.6	0\\
32.61	0\\
32.62	0\\
32.63	0\\
32.64	0\\
32.65	0\\
32.66	0\\
32.67	0\\
32.68	0\\
32.69	0\\
32.7	0\\
32.71	0\\
32.72	0\\
32.73	0\\
32.74	0\\
32.75	0\\
32.76	0\\
32.77	0\\
32.78	0\\
32.79	0\\
32.8	0\\
32.81	0\\
32.82	0\\
32.83	0\\
32.84	0\\
32.85	0\\
32.86	0\\
32.87	0\\
32.88	0\\
32.89	0\\
32.9	0\\
32.91	0\\
32.92	0\\
32.93	0\\
32.94	0\\
32.95	0\\
32.96	0\\
32.97	0\\
32.98	0\\
32.99	0\\
33	0\\
33.01	0\\
33.02	0\\
33.03	0\\
33.04	0\\
33.05	0\\
33.06	0\\
33.07	0\\
33.08	0\\
33.09	0\\
33.1	0\\
33.11	0\\
33.12	0\\
33.13	0\\
33.14	0\\
33.15	0\\
33.16	0\\
33.17	0\\
33.18	0\\
33.19	0\\
33.2	0\\
33.21	0\\
33.22	0\\
33.23	0\\
33.24	0\\
33.25	0\\
33.26	0\\
33.27	0\\
33.28	0\\
33.29	0\\
33.3	0\\
33.31	0\\
33.32	0\\
33.33	0\\
33.34	0\\
33.35	0\\
33.36	0\\
33.37	0\\
33.38	0\\
33.39	0\\
33.4	0\\
33.41	0\\
33.42	0\\
33.43	0\\
33.44	0\\
33.45	0\\
33.46	0\\
33.47	0\\
33.48	0\\
33.49	0\\
33.5	0\\
33.51	0\\
33.52	0\\
33.53	0\\
33.54	0\\
33.55	0\\
33.56	0\\
33.57	0\\
33.58	0\\
33.59	0\\
33.6	0\\
33.61	0\\
33.62	0\\
33.63	0\\
33.64	0\\
33.65	0\\
33.66	0\\
33.67	0\\
33.68	0\\
33.69	0\\
33.7	0\\
33.71	0\\
33.72	0\\
33.73	0\\
33.74	0\\
33.75	0\\
33.76	0\\
33.77	0\\
33.78	0\\
33.79	0\\
33.8	0\\
33.81	0\\
33.82	0\\
33.83	0\\
33.84	0\\
33.85	0\\
33.86	0\\
33.87	0\\
33.88	0\\
33.89	0\\
33.9	0\\
33.91	0\\
33.92	0\\
33.93	0\\
33.94	0\\
33.95	0\\
33.96	0\\
33.97	0\\
33.98	0\\
33.99	0\\
34	0\\
34.01	0\\
34.02	0\\
34.03	0\\
34.04	0\\
34.05	0\\
34.06	0\\
34.07	0\\
34.08	0\\
34.09	0\\
34.1	0\\
34.11	0\\
34.12	0\\
34.13	0\\
34.14	0\\
34.15	0\\
34.16	0\\
34.17	0\\
34.18	0\\
34.19	0\\
34.2	0\\
34.21	0\\
34.22	0\\
34.23	0\\
34.24	0\\
34.25	0\\
34.26	0\\
34.27	0\\
34.28	0\\
34.29	0\\
34.3	0\\
34.31	0\\
34.32	0\\
34.33	0\\
34.34	0\\
34.35	0\\
34.36	0\\
34.37	0\\
34.38	0\\
34.39	0\\
34.4	0\\
34.41	0\\
34.42	0\\
34.43	0\\
34.44	0\\
34.45	0\\
34.46	0\\
34.47	0\\
34.48	0\\
34.49	0\\
34.5	0\\
34.51	0\\
34.52	0\\
34.53	0\\
34.54	0\\
34.55	0\\
34.56	0\\
34.57	0\\
34.58	0\\
34.59	0\\
34.6	0\\
34.61	0\\
34.62	0\\
34.63	0\\
34.64	0\\
34.65	0\\
34.66	0\\
34.67	0\\
34.68	0\\
34.69	0\\
34.7	0\\
34.71	0\\
34.72	0\\
34.73	0\\
34.74	0\\
34.75	0\\
34.76	0\\
34.77	0\\
34.78	0\\
34.79	0\\
34.8	0\\
34.81	0\\
34.82	0\\
34.83	0\\
34.84	0\\
34.85	0\\
34.86	0\\
34.87	0\\
34.88	0\\
34.89	0\\
34.9	0\\
34.91	0\\
34.92	0\\
34.93	0\\
34.94	0\\
34.95	0\\
34.96	0\\
34.97	0\\
34.98	0\\
34.99	0\\
35	0\\
35.01	0\\
35.02	0\\
35.03	0\\
35.04	0\\
35.05	0\\
35.06	0\\
35.07	0\\
35.08	0\\
35.09	0\\
35.1	0\\
35.11	0\\
35.12	0\\
35.13	0\\
35.14	0\\
35.15	0\\
35.16	0\\
35.17	0\\
35.18	0\\
35.19	0\\
35.2	0\\
35.21	0\\
35.22	0\\
35.23	0\\
35.24	0\\
35.25	0\\
35.26	0\\
35.27	0\\
35.28	0\\
35.29	0\\
35.3	0\\
35.31	0\\
35.32	0\\
35.33	0\\
35.34	0\\
35.35	0\\
35.36	0\\
35.37	0\\
35.38	0\\
35.39	0\\
35.4	0\\
35.41	0\\
35.42	0\\
35.43	0\\
35.44	0\\
35.45	0\\
35.46	0\\
35.47	0\\
35.48	0\\
35.49	0\\
35.5	0\\
35.51	0\\
35.52	0\\
35.53	0\\
35.54	0\\
35.55	0\\
35.56	0\\
35.57	0\\
35.58	0\\
35.59	0\\
35.6	0\\
35.61	0\\
35.62	0\\
35.63	0\\
35.64	0\\
35.65	0\\
35.66	0\\
35.67	0\\
35.68	0\\
35.69	0\\
35.7	0\\
35.71	0\\
35.72	0\\
35.73	0\\
35.74	0\\
35.75	0\\
35.76	0\\
35.77	0\\
35.78	0\\
35.79	0\\
35.8	0\\
35.81	0\\
35.82	0\\
35.83	0\\
35.84	0\\
35.85	0\\
35.86	0\\
35.87	0\\
35.88	0\\
35.89	0\\
35.9	0\\
35.91	0\\
35.92	0\\
35.93	0\\
35.94	0\\
35.95	0\\
35.96	0\\
35.97	0\\
35.98	0\\
35.99	0\\
36	0\\
36.01	0\\
36.02	0\\
36.03	0\\
36.04	0\\
36.05	0\\
36.06	0\\
36.07	0\\
36.08	0\\
36.09	0\\
36.1	0\\
36.11	0\\
36.12	0\\
36.13	0\\
36.14	0\\
36.15	0\\
36.16	0\\
36.17	0\\
36.18	0\\
36.19	0\\
36.2	0\\
36.21	0\\
36.22	0\\
36.23	0\\
36.24	0\\
36.25	0\\
36.26	0\\
36.27	0\\
36.28	0\\
36.29	0\\
36.3	0\\
36.31	0\\
36.32	0\\
36.33	0\\
36.34	0\\
36.35	0\\
36.36	0\\
36.37	0\\
36.38	0\\
36.39	0\\
36.4	0\\
36.41	0\\
36.42	0\\
36.43	0\\
36.44	0\\
36.45	0\\
36.46	0\\
36.47	0\\
36.48	0\\
36.49	0\\
36.5	0\\
36.51	0\\
36.52	0\\
36.53	0\\
36.54	0\\
36.55	0\\
36.56	0\\
36.57	0\\
36.58	0\\
36.59	0\\
36.6	0\\
36.61	0\\
36.62	0\\
36.63	0\\
36.64	0\\
36.65	0\\
36.66	0\\
36.67	0\\
36.68	0\\
36.69	0\\
36.7	0\\
36.71	0\\
36.72	0\\
36.73	0\\
36.74	0\\
36.75	0\\
36.76	0\\
36.77	0\\
36.78	0\\
36.79	0\\
36.8	0\\
36.81	0\\
36.82	0\\
36.83	0\\
36.84	0\\
36.85	0\\
36.86	0\\
36.87	0\\
36.88	0\\
36.89	0\\
36.9	0\\
36.91	0\\
36.92	0\\
36.93	0\\
36.94	0\\
36.95	0\\
36.96	0\\
36.97	0\\
36.98	0\\
36.99	0\\
37	0\\
37.01	0\\
37.02	0\\
37.03	0\\
37.04	0\\
37.05	0\\
37.06	0\\
37.07	0\\
37.08	0\\
37.09	0\\
37.1	0\\
37.11	0\\
37.12	0\\
37.13	0\\
37.14	0\\
37.15	0\\
37.16	0\\
37.17	0\\
37.18	0\\
37.19	0\\
37.2	0\\
37.21	0\\
37.22	0\\
37.23	0\\
37.24	0\\
37.25	0\\
37.26	0\\
37.27	0\\
37.28	0\\
37.29	0\\
37.3	0\\
37.31	0\\
37.32	0\\
37.33	0\\
37.34	0\\
37.35	0\\
37.36	0\\
37.37	0\\
37.38	0\\
37.39	0\\
37.4	0\\
37.41	0\\
37.42	0\\
37.43	0\\
37.44	0\\
37.45	0\\
37.46	0\\
37.47	0\\
37.48	0\\
37.49	0\\
37.5	0\\
37.51	0\\
37.52	0\\
37.53	0\\
37.54	0\\
37.55	0\\
37.56	0\\
37.57	0\\
37.58	0\\
37.59	0\\
37.6	0\\
37.61	0\\
37.62	0\\
37.63	0\\
37.64	0\\
37.65	0\\
37.66	0\\
37.67	0\\
37.68	0\\
37.69	0\\
37.7	0\\
37.71	0\\
37.72	0\\
37.73	0\\
37.74	0\\
37.75	0\\
37.76	0\\
37.77	0\\
37.78	0\\
37.79	0\\
37.8	0\\
37.81	0\\
37.82	0\\
37.83	0\\
37.84	0\\
37.85	0\\
37.86	0\\
37.87	0\\
37.88	0\\
37.89	0\\
37.9	0\\
37.91	0\\
37.92	0\\
37.93	0\\
37.94	0\\
37.95	0\\
37.96	0\\
37.97	0\\
37.98	0\\
37.99	0\\
38	0\\
38.01	0\\
38.02	0\\
38.03	0\\
38.04	0\\
38.05	0\\
38.06	0\\
38.07	0\\
38.08	0\\
38.09	0\\
38.1	0\\
38.11	0\\
38.12	0\\
38.13	0\\
38.14	0\\
38.15	0\\
38.16	0\\
38.17	0\\
38.18	0\\
38.19	0\\
38.2	0\\
38.21	0\\
38.22	0\\
38.23	0\\
38.24	0\\
38.25	0\\
38.26	0\\
38.27	0\\
38.28	0\\
38.29	0\\
38.3	0\\
38.31	0\\
38.32	0\\
38.33	0\\
38.34	0\\
38.35	0\\
38.36	0\\
38.37	0\\
38.38	0\\
38.39	0\\
38.4	0\\
38.41	0\\
38.42	0\\
38.43	0\\
38.44	0\\
38.45	0\\
38.46	0\\
38.47	0\\
38.48	0\\
38.49	0\\
38.5	0\\
38.51	0\\
38.52	0\\
38.53	0\\
38.54	0\\
38.55	0\\
38.56	0\\
38.57	0\\
38.58	0\\
38.59	0\\
38.6	0\\
38.61	0\\
38.62	0\\
38.63	0\\
38.64	0\\
38.65	0\\
38.66	0\\
38.67	0\\
38.68	0\\
38.69	0\\
38.7	0\\
38.71	0\\
38.72	0\\
38.73	0\\
38.74	0\\
38.75	0\\
38.76	0\\
38.77	0\\
38.78	0\\
38.79	0\\
38.8	0\\
38.81	0\\
38.82	0\\
38.83	0\\
38.84	0\\
38.85	0\\
38.86	0\\
38.87	0\\
38.88	0\\
38.89	0\\
38.9	0\\
38.91	0\\
38.92	0\\
38.93	0\\
38.94	0\\
38.95	0\\
38.96	0\\
38.97	0\\
38.98	0\\
38.99	0\\
39	0\\
39.01	0\\
39.02	0\\
39.03	0\\
39.04	0\\
39.05	0\\
39.06	0\\
39.07	0\\
39.08	0\\
39.09	0\\
39.1	0\\
39.11	0\\
39.12	0\\
39.13	0\\
39.14	0\\
39.15	0\\
39.16	0\\
39.17	0\\
39.18	0\\
39.19	0\\
39.2	0\\
39.21	0\\
39.22	0\\
39.23	0\\
39.24	0\\
39.25	0\\
39.26	0\\
39.27	0\\
39.28	0\\
39.29	0\\
39.3	0\\
39.31	0\\
39.32	0\\
39.33	0\\
39.34	0\\
39.35	0\\
39.36	0\\
39.37	0\\
39.38	0\\
39.39	0\\
39.4	0\\
39.41	0\\
39.42	0\\
39.43	0\\
39.44	0\\
39.45	0\\
39.46	0\\
39.47	0\\
39.48	0\\
39.49	0\\
39.5	0\\
39.51	0\\
39.52	0\\
39.53	0\\
39.54	0\\
39.55	0\\
39.56	0\\
39.57	0\\
39.58	0\\
39.59	0\\
39.6	0\\
39.61	0\\
39.62	0\\
39.63	0\\
39.64	0\\
39.65	0\\
39.66	0\\
39.67	0\\
39.68	0\\
39.69	0\\
39.7	0\\
39.71	0\\
39.72	0\\
39.73	0\\
39.74	0\\
39.75	0\\
39.76	0\\
39.77	0\\
39.78	0\\
39.79	0\\
39.8	0\\
39.81	0\\
39.82	0\\
39.83	0\\
39.84	0\\
39.85	0\\
39.86	0\\
39.87	0\\
39.88	0\\
39.89	0\\
39.9	0\\
39.91	0\\
39.92	0\\
39.93	0\\
39.94	0\\
39.95	0\\
39.96	0\\
39.97	0\\
39.98	0\\
39.99	0\\
40	0\\
40.01	0\\
};
\addplot [color=black,solid,forget plot]
  table[row sep=crcr]{%
40.01	0\\
40.02	0\\
40.03	0\\
40.04	0\\
40.05	0\\
40.06	0\\
40.07	0\\
40.08	0\\
40.09	0\\
40.1	0\\
40.11	0\\
40.12	0\\
40.13	0\\
40.14	0\\
40.15	0\\
40.16	0\\
40.17	0\\
40.18	0\\
40.19	0\\
40.2	0\\
40.21	0\\
40.22	0\\
40.23	0\\
40.24	0\\
40.25	0\\
40.26	0\\
40.27	0\\
40.28	0\\
40.29	0\\
40.3	0\\
40.31	0\\
40.32	0\\
40.33	0\\
40.34	0\\
40.35	0\\
40.36	0\\
40.37	0\\
40.38	0\\
40.39	0\\
40.4	0\\
40.41	0\\
40.42	0\\
40.43	0\\
40.44	0\\
40.45	0\\
40.46	0\\
40.47	0\\
40.48	0\\
40.49	0\\
40.5	0\\
40.51	0\\
40.52	0\\
40.53	0\\
40.54	0\\
40.55	0\\
40.56	0\\
40.57	0\\
40.58	0\\
40.59	0\\
40.6	0\\
40.61	0\\
40.62	0\\
40.63	0\\
40.64	0\\
40.65	0\\
40.66	0\\
40.67	0\\
40.68	0\\
40.69	0\\
40.7	0\\
40.71	0\\
40.72	0\\
40.73	0\\
40.74	0\\
40.75	0\\
40.76	0\\
40.77	0\\
40.78	0\\
40.79	0\\
40.8	0\\
40.81	0\\
40.82	0\\
40.83	0\\
40.84	0\\
40.85	0\\
40.86	0\\
40.87	0\\
40.88	0\\
40.89	0\\
40.9	0\\
40.91	0\\
40.92	0\\
40.93	0\\
40.94	0\\
40.95	0\\
40.96	0\\
40.97	0\\
40.98	0\\
40.99	0\\
41	0\\
41.01	0\\
41.02	0\\
41.03	0\\
41.04	0\\
41.05	0\\
41.06	0\\
41.07	0\\
41.08	0\\
41.09	0\\
41.1	0\\
41.11	0\\
41.12	0\\
41.13	0\\
41.14	0\\
41.15	0\\
41.16	0\\
41.17	0\\
41.18	0\\
41.19	0\\
41.2	0\\
41.21	0\\
41.22	0\\
41.23	0\\
41.24	0\\
41.25	0\\
41.26	0\\
41.27	0\\
41.28	0\\
41.29	0\\
41.3	0\\
41.31	0\\
41.32	0\\
41.33	0\\
41.34	0\\
41.35	0\\
41.36	0\\
41.37	0\\
41.38	0\\
41.39	0\\
41.4	0\\
41.41	0\\
41.42	0\\
41.43	0\\
41.44	0\\
41.45	0\\
41.46	0\\
41.47	0\\
41.48	0\\
41.49	0\\
41.5	0\\
41.51	0\\
41.52	0\\
41.53	0\\
41.54	0\\
41.55	0\\
41.56	0\\
41.57	0\\
41.58	0\\
41.59	0\\
41.6	0\\
41.61	0\\
41.62	0\\
41.63	0\\
41.64	0\\
41.65	0\\
41.66	0\\
41.67	0\\
41.68	0\\
41.69	0\\
41.7	0\\
41.71	0\\
41.72	0\\
41.73	0\\
41.74	0\\
41.75	0\\
41.76	0\\
41.77	0\\
41.78	0\\
41.79	0\\
41.8	0\\
41.81	0\\
41.82	0\\
41.83	0\\
41.84	0\\
41.85	0\\
41.86	0\\
41.87	0\\
41.88	0\\
41.89	0\\
41.9	0\\
41.91	0\\
41.92	0\\
41.93	0\\
41.94	0\\
41.95	0\\
41.96	0\\
41.97	0\\
41.98	0\\
41.99	0\\
42	0\\
42.01	0\\
42.02	0\\
42.03	0\\
42.04	0\\
42.05	0\\
42.06	0\\
42.07	0\\
42.08	0\\
42.09	0\\
42.1	0\\
42.11	0\\
42.12	0\\
42.13	0\\
42.14	0\\
42.15	0\\
42.16	0\\
42.17	0\\
42.18	0\\
42.19	0\\
42.2	0\\
42.21	0\\
42.22	0\\
42.23	0\\
42.24	0\\
42.25	0\\
42.26	0\\
42.27	0\\
42.28	0\\
42.29	0\\
42.3	0\\
42.31	0\\
42.32	0\\
42.33	0\\
42.34	0\\
42.35	0\\
42.36	0\\
42.37	0\\
42.38	0\\
42.39	0\\
42.4	0\\
42.41	0\\
42.42	0\\
42.43	0\\
42.44	0\\
42.45	0\\
42.46	0\\
42.47	0\\
42.48	0\\
42.49	0\\
42.5	0\\
42.51	0\\
42.52	0\\
42.53	0\\
42.54	0\\
42.55	0\\
42.56	0\\
42.57	0\\
42.58	0\\
42.59	0\\
42.6	0\\
42.61	0\\
42.62	0\\
42.63	0\\
42.64	0\\
42.65	0\\
42.66	0\\
42.67	0\\
42.68	0\\
42.69	0\\
42.7	0\\
42.71	0\\
42.72	0\\
42.73	0\\
42.74	0\\
42.75	0\\
42.76	0\\
42.77	0\\
42.78	0\\
42.79	0\\
42.8	0\\
42.81	0\\
42.82	0\\
42.83	0\\
42.84	0\\
42.85	0\\
42.86	0\\
42.87	0\\
42.88	0\\
42.89	0\\
42.9	0\\
42.91	0\\
42.92	0\\
42.93	0\\
42.94	0\\
42.95	0\\
42.96	0\\
42.97	0\\
42.98	0\\
42.99	0\\
43	0\\
43.01	0\\
43.02	0\\
43.03	0\\
43.04	0\\
43.05	0\\
43.06	0\\
43.07	0\\
43.08	0\\
43.09	0\\
43.1	0\\
43.11	0\\
43.12	0\\
43.13	0\\
43.14	0\\
43.15	0\\
43.16	0\\
43.17	0\\
43.18	0\\
43.19	0\\
43.2	0\\
43.21	0\\
43.22	0\\
43.23	0\\
43.24	0\\
43.25	0\\
43.26	0\\
43.27	0\\
43.28	0\\
43.29	0\\
43.3	0\\
43.31	0\\
43.32	0\\
43.33	0\\
43.34	0\\
43.35	0\\
43.36	0\\
43.37	0\\
43.38	0\\
43.39	0\\
43.4	0\\
43.41	0\\
43.42	0\\
43.43	0\\
43.44	0\\
43.45	0\\
43.46	0\\
43.47	0\\
43.48	0\\
43.49	0\\
43.5	0\\
43.51	0\\
43.52	0\\
43.53	0\\
43.54	0\\
43.55	0\\
43.56	0\\
43.57	0\\
43.58	0\\
43.59	0\\
43.6	0\\
43.61	0\\
43.62	0\\
43.63	0\\
43.64	0\\
43.65	0\\
43.66	0\\
43.67	0\\
43.68	0\\
43.69	0\\
43.7	0\\
43.71	0\\
43.72	0\\
43.73	0\\
43.74	0\\
43.75	0\\
43.76	0\\
43.77	0\\
43.78	0\\
43.79	0\\
43.8	0\\
43.81	0\\
43.82	0\\
43.83	0\\
43.84	0\\
43.85	0\\
43.86	0\\
43.87	0\\
43.88	0\\
43.89	0\\
43.9	0\\
43.91	0\\
43.92	0\\
43.93	0\\
43.94	0\\
43.95	0\\
43.96	0\\
43.97	0\\
43.98	0\\
43.99	0\\
44	0\\
44.01	0\\
44.02	0\\
44.03	0\\
44.04	0\\
44.05	0\\
44.06	0\\
44.07	0\\
44.08	0\\
44.09	0\\
44.1	0\\
44.11	0\\
44.12	0\\
44.13	0\\
44.14	0\\
44.15	0\\
44.16	0\\
44.17	0\\
44.18	0\\
44.19	0\\
44.2	0\\
44.21	0\\
44.22	0\\
44.23	0\\
44.24	0\\
44.25	0\\
44.26	0\\
44.27	0\\
44.28	0\\
44.29	0\\
44.3	0\\
44.31	0\\
44.32	0\\
44.33	0\\
44.34	0\\
44.35	0\\
44.36	0\\
44.37	0\\
44.38	0\\
44.39	0\\
44.4	0\\
44.41	0\\
44.42	0\\
44.43	0\\
44.44	0\\
44.45	0\\
44.46	0\\
44.47	0\\
44.48	0\\
44.49	0\\
44.5	0\\
44.51	0\\
44.52	0\\
44.53	0\\
44.54	0\\
44.55	0\\
44.56	0\\
44.57	0\\
44.58	0\\
44.59	0\\
44.6	0\\
44.61	0\\
44.62	0\\
44.63	0\\
44.64	0\\
44.65	0\\
44.66	0\\
44.67	0\\
44.68	0\\
44.69	0\\
44.7	0\\
44.71	0\\
44.72	0\\
44.73	0\\
44.74	0\\
44.75	0\\
44.76	0\\
44.77	0\\
44.78	0\\
44.79	0\\
44.8	0\\
44.81	0\\
44.82	0\\
44.83	0\\
44.84	0\\
44.85	0\\
44.86	0\\
44.87	0\\
44.88	0\\
44.89	0\\
44.9	0\\
44.91	0\\
44.92	0\\
44.93	0\\
44.94	0\\
44.95	0\\
44.96	0\\
44.97	0\\
44.98	0\\
44.99	0\\
45	0\\
45.01	0\\
45.02	0\\
45.03	0\\
45.04	0\\
45.05	0\\
45.06	0\\
45.07	0\\
45.08	0\\
45.09	0\\
45.1	0\\
45.11	0\\
45.12	0\\
45.13	0\\
45.14	0\\
45.15	0\\
45.16	0\\
45.17	0\\
45.18	0\\
45.19	0\\
45.2	0\\
45.21	0\\
45.22	0\\
45.23	0\\
45.24	0\\
45.25	0\\
45.26	0\\
45.27	0\\
45.28	0\\
45.29	0\\
45.3	0\\
45.31	0\\
45.32	0\\
45.33	0\\
45.34	0\\
45.35	0\\
45.36	0\\
45.37	0\\
45.38	0\\
45.39	0\\
45.4	0\\
45.41	0\\
45.42	0\\
45.43	0\\
45.44	0\\
45.45	0\\
45.46	0\\
45.47	0\\
45.48	0\\
45.49	0\\
45.5	0\\
45.51	0\\
45.52	0\\
45.53	0\\
45.54	0\\
45.55	0\\
45.56	0\\
45.57	0\\
45.58	0\\
45.59	0\\
45.6	0\\
45.61	0\\
45.62	0\\
45.63	0\\
45.64	0\\
45.65	0\\
45.66	0\\
45.67	0\\
45.68	0\\
45.69	0\\
45.7	0\\
45.71	0\\
45.72	0\\
45.73	0\\
45.74	0\\
45.75	0\\
45.76	0\\
45.77	0\\
45.78	0\\
45.79	0\\
45.8	0\\
45.81	0\\
45.82	0\\
45.83	0\\
45.84	0\\
45.85	0\\
45.86	0\\
45.87	0\\
45.88	0\\
45.89	0\\
45.9	0\\
45.91	0\\
45.92	0\\
45.93	0\\
45.94	0\\
45.95	0\\
45.96	0\\
45.97	0\\
45.98	0\\
45.99	0\\
46	0\\
46.01	0\\
46.02	0\\
46.03	0\\
46.04	0\\
46.05	0\\
46.06	0\\
46.07	0\\
46.08	0\\
46.09	0\\
46.1	0\\
46.11	0\\
46.12	0\\
46.13	0\\
46.14	0\\
46.15	0\\
46.16	0\\
46.17	0\\
46.18	0\\
46.19	0\\
46.2	0\\
46.21	0\\
46.22	0\\
46.23	0\\
46.24	0\\
46.25	0\\
46.26	0\\
46.27	0\\
46.28	0\\
46.29	0\\
46.3	0\\
46.31	0\\
46.32	0\\
46.33	0\\
46.34	0\\
46.35	0\\
46.36	0\\
46.37	0\\
46.38	0\\
46.39	0\\
46.4	0\\
46.41	0\\
46.42	0\\
46.43	0\\
46.44	0\\
46.45	0\\
46.46	0\\
46.47	0\\
46.48	0\\
46.49	0\\
46.5	0\\
46.51	0\\
46.52	0\\
46.53	0\\
46.54	0\\
46.55	0\\
46.56	0\\
46.57	0\\
46.58	0\\
46.59	0\\
46.6	0\\
46.61	0\\
46.62	0\\
46.63	0\\
46.64	0\\
46.65	0\\
46.66	0\\
46.67	0\\
46.68	0\\
46.69	0\\
46.7	0\\
46.71	0\\
46.72	0\\
46.73	0\\
46.74	0\\
46.75	0\\
46.76	0\\
46.77	0\\
46.78	0\\
46.79	0\\
46.8	0\\
46.81	0\\
46.82	0\\
46.83	0\\
46.84	0\\
46.85	0\\
46.86	0\\
46.87	0\\
46.88	0\\
46.89	0\\
46.9	0\\
46.91	0\\
46.92	0\\
46.93	0\\
46.94	0\\
46.95	0\\
46.96	0\\
46.97	0\\
46.98	0\\
46.99	0\\
47	0\\
47.01	0\\
47.02	0\\
47.03	0\\
47.04	0\\
47.05	0\\
47.06	0\\
47.07	0\\
47.08	0\\
47.09	0\\
47.1	0\\
47.11	0\\
47.12	0\\
47.13	0\\
47.14	0\\
47.15	0\\
47.16	0\\
47.17	0\\
47.18	0\\
47.19	0\\
47.2	0\\
47.21	0\\
47.22	0\\
47.23	0\\
47.24	0\\
47.25	0\\
47.26	0\\
47.27	0\\
47.28	0\\
47.29	0\\
47.3	0\\
47.31	0\\
47.32	0\\
47.33	0\\
47.34	0\\
47.35	0\\
47.36	0\\
47.37	0\\
47.38	0\\
47.39	0\\
47.4	0\\
47.41	0\\
47.42	0\\
47.43	0\\
47.44	0\\
47.45	0\\
47.46	0\\
47.47	0\\
47.48	0\\
47.49	0\\
47.5	0\\
47.51	0\\
47.52	0\\
47.53	0\\
47.54	0\\
47.55	0\\
47.56	0\\
47.57	0\\
47.58	0\\
47.59	0\\
47.6	0\\
47.61	0\\
47.62	0\\
47.63	0\\
47.64	0\\
47.65	0\\
47.66	0\\
47.67	0\\
47.68	0\\
47.69	0\\
47.7	0\\
47.71	0\\
47.72	0\\
47.73	0\\
47.74	0\\
47.75	0\\
47.76	0\\
47.77	0\\
47.78	0\\
47.79	0\\
47.8	0\\
47.81	0\\
47.82	0\\
47.83	0\\
47.84	0\\
47.85	0\\
47.86	0\\
47.87	0\\
47.88	0\\
47.89	0\\
47.9	0\\
47.91	0\\
47.92	0\\
47.93	0\\
47.94	0\\
47.95	0\\
47.96	0\\
47.97	0\\
47.98	0\\
47.99	0\\
48	0\\
48.01	0\\
48.02	0\\
48.03	0\\
48.04	0\\
48.05	0\\
48.06	0\\
48.07	0\\
48.08	0\\
48.09	0\\
48.1	0\\
48.11	0\\
48.12	0\\
48.13	0\\
48.14	0\\
48.15	0\\
48.16	0\\
48.17	0\\
48.18	0\\
48.19	0\\
48.2	0\\
48.21	0\\
48.22	0\\
48.23	0\\
48.24	0\\
48.25	0\\
48.26	0\\
48.27	0\\
48.28	0\\
48.29	0\\
48.3	0\\
48.31	0\\
48.32	0\\
48.33	0\\
48.34	0\\
48.35	0\\
48.36	0\\
48.37	0\\
48.38	0\\
48.39	0\\
48.4	0\\
48.41	0\\
48.42	0\\
48.43	0\\
48.44	0\\
48.45	0\\
48.46	0\\
48.47	0\\
48.48	0\\
48.49	0\\
48.5	0\\
48.51	0\\
48.52	0\\
48.53	0\\
48.54	0\\
48.55	0\\
48.56	0\\
48.57	0\\
48.58	0\\
48.59	0\\
48.6	0\\
48.61	0\\
48.62	0\\
48.63	0\\
48.64	0\\
48.65	0\\
48.66	0\\
48.67	0\\
48.68	0\\
48.69	0\\
48.7	0\\
48.71	0\\
48.72	0\\
48.73	0\\
48.74	0\\
48.75	0\\
48.76	0\\
48.77	0\\
48.78	0\\
48.79	0\\
48.8	0\\
48.81	0\\
48.82	0\\
48.83	0\\
48.84	0\\
48.85	0\\
48.86	0\\
48.87	0\\
48.88	0\\
48.89	0\\
48.9	0\\
48.91	0\\
48.92	0\\
48.93	0\\
48.94	0\\
48.95	0\\
48.96	0\\
48.97	0\\
48.98	0\\
48.99	0\\
49	0\\
49.01	0\\
49.02	0\\
49.03	0\\
49.04	0\\
49.05	0\\
49.06	0\\
49.07	0\\
49.08	0\\
49.09	0\\
49.1	0\\
49.11	0\\
49.12	0\\
49.13	0\\
49.14	0\\
49.15	0\\
49.16	0\\
49.17	0\\
49.18	0\\
49.19	0\\
49.2	0\\
49.21	0\\
49.22	0\\
49.23	0\\
49.24	0\\
49.25	0\\
49.26	0\\
49.27	0\\
49.28	0\\
49.29	0\\
49.3	0\\
49.31	0\\
49.32	0\\
49.33	0\\
49.34	0\\
49.35	0\\
49.36	0\\
49.37	0\\
49.38	0\\
49.39	0\\
49.4	0\\
49.41	0\\
49.42	0\\
49.43	0\\
49.44	0\\
49.45	0\\
49.46	0\\
49.47	0\\
49.48	0\\
49.49	0\\
49.5	0\\
49.51	0\\
49.52	0\\
49.53	0\\
49.54	0\\
49.55	0\\
49.56	0\\
49.57	0\\
49.58	0\\
49.59	0\\
49.6	0\\
49.61	0\\
49.62	0\\
49.63	0\\
49.64	0\\
49.65	0\\
49.66	0\\
49.67	0\\
49.68	0\\
49.69	0\\
49.7	0\\
49.71	0\\
49.72	0\\
49.73	0\\
49.74	0\\
49.75	0\\
49.76	0\\
49.77	0\\
49.78	0\\
49.79	0\\
49.8	0\\
49.81	0\\
49.82	0\\
49.83	0\\
49.84	0\\
49.85	0\\
49.86	0\\
49.87	0\\
49.88	0\\
49.89	0\\
49.9	0\\
49.91	0\\
49.92	0\\
49.93	0\\
49.94	0\\
49.95	0\\
49.96	0\\
49.97	0\\
49.98	0\\
49.99	0\\
50	0\\
50.01	0\\
50.02	0\\
50.03	0\\
50.04	0\\
50.05	0\\
50.06	0\\
50.07	0\\
50.08	0\\
50.09	0\\
50.1	0\\
50.11	0\\
50.12	0\\
50.13	0\\
50.14	0\\
50.15	0\\
50.16	0\\
50.17	0\\
50.18	0\\
50.19	0\\
50.2	0\\
50.21	0\\
50.22	0\\
50.23	0\\
50.24	0\\
50.25	0\\
50.26	0\\
50.27	0\\
50.28	0\\
50.29	0\\
50.3	0\\
50.31	0\\
50.32	0\\
50.33	0\\
50.34	0\\
50.35	0\\
50.36	0\\
50.37	0\\
50.38	0\\
50.39	0\\
50.4	0\\
50.41	0\\
50.42	0\\
50.43	0\\
50.44	0\\
50.45	0\\
50.46	0\\
50.47	0\\
50.48	0\\
50.49	0\\
50.5	0\\
50.51	0\\
50.52	0\\
50.53	0\\
50.54	0\\
50.55	0\\
50.56	0\\
50.57	0\\
50.58	0\\
50.59	0\\
50.6	0\\
50.61	0\\
50.62	0\\
50.63	0\\
50.64	0\\
50.65	0\\
50.66	0\\
50.67	0\\
50.68	0\\
50.69	0\\
50.7	0\\
50.71	0\\
50.72	0\\
50.73	0\\
50.74	0\\
50.75	0\\
50.76	0\\
50.77	0\\
50.78	0\\
50.79	0\\
50.8	0\\
50.81	0\\
50.82	0\\
50.83	0\\
50.84	0\\
50.85	0\\
50.86	0\\
50.87	0\\
50.88	0\\
50.89	0\\
50.9	0\\
50.91	0\\
50.92	0\\
50.93	0\\
50.94	0\\
50.95	0\\
50.96	0\\
50.97	0\\
50.98	0\\
50.99	0\\
51	0\\
51.01	0\\
51.02	0\\
51.03	0\\
51.04	0\\
51.05	0\\
51.06	0\\
51.07	0\\
51.08	0\\
51.09	0\\
51.1	0\\
51.11	0\\
51.12	0\\
51.13	0\\
51.14	0\\
51.15	0\\
51.16	0\\
51.17	0\\
51.18	0\\
51.19	0\\
51.2	0\\
51.21	0\\
51.22	0\\
51.23	0\\
51.24	0\\
51.25	0\\
51.26	0\\
51.27	0\\
51.28	0\\
51.29	0\\
51.3	0\\
51.31	0\\
51.32	0\\
51.33	0\\
51.34	0\\
51.35	0\\
51.36	0\\
51.37	0\\
51.38	0\\
51.39	0\\
51.4	0\\
51.41	0\\
51.42	0\\
51.43	0\\
51.44	0\\
51.45	0\\
51.46	0\\
51.47	0\\
51.48	0\\
51.49	0\\
51.5	0\\
51.51	0\\
51.52	0\\
51.53	0\\
51.54	0\\
51.55	0\\
51.56	0\\
51.57	0\\
51.58	0\\
51.59	0\\
51.6	0\\
51.61	0\\
51.62	0\\
51.63	0\\
51.64	0\\
51.65	0\\
51.66	0\\
51.67	0\\
51.68	0\\
51.69	0\\
51.7	0\\
51.71	0\\
51.72	0\\
51.73	0\\
51.74	0\\
51.75	0\\
51.76	0\\
51.77	0\\
51.78	0\\
51.79	0\\
51.8	0\\
51.81	0\\
51.82	0\\
51.83	0\\
51.84	0\\
51.85	0\\
51.86	0\\
51.87	0\\
51.88	0\\
51.89	0\\
51.9	0\\
51.91	0\\
51.92	0\\
51.93	0\\
51.94	0\\
51.95	0\\
51.96	0\\
51.97	0\\
51.98	0\\
51.99	0\\
52	0\\
52.01	0\\
52.02	0\\
52.03	0\\
52.04	0\\
52.05	0\\
52.06	0\\
52.07	0\\
52.08	0\\
52.09	0\\
52.1	0\\
52.11	0\\
52.12	0\\
52.13	0\\
52.14	0\\
52.15	0\\
52.16	0\\
52.17	0\\
52.18	0\\
52.19	0\\
52.2	0\\
52.21	0\\
52.22	0\\
52.23	0\\
52.24	0\\
52.25	0\\
52.26	0\\
52.27	0\\
52.28	0\\
52.29	0\\
52.3	0\\
52.31	0\\
52.32	0\\
52.33	0\\
52.34	0\\
52.35	0\\
52.36	0\\
52.37	0\\
52.38	0\\
52.39	0\\
52.4	0\\
52.41	0\\
52.42	0\\
52.43	0\\
52.44	0\\
52.45	0\\
52.46	0\\
52.47	0\\
52.48	0\\
52.49	0\\
52.5	0\\
52.51	0\\
52.52	0\\
52.53	0\\
52.54	0\\
52.55	0\\
52.56	0\\
52.57	0\\
52.58	0\\
52.59	0\\
52.6	0\\
52.61	0\\
52.62	0\\
52.63	0\\
52.64	0\\
52.65	0\\
52.66	0\\
52.67	0\\
52.68	0\\
52.69	0\\
52.7	0\\
52.71	0\\
52.72	0\\
52.73	0\\
52.74	0\\
52.75	0\\
52.76	0\\
52.77	0\\
52.78	0\\
52.79	0\\
52.8	0\\
52.81	0\\
52.82	0\\
52.83	0\\
52.84	0\\
52.85	0\\
52.86	0\\
52.87	0\\
52.88	0\\
52.89	0\\
52.9	0\\
52.91	0\\
52.92	0\\
52.93	0\\
52.94	0\\
52.95	0\\
52.96	0\\
52.97	0\\
52.98	0\\
52.99	0\\
53	0\\
53.01	0\\
53.02	0\\
53.03	0\\
53.04	0\\
53.05	0\\
53.06	0\\
53.07	0\\
53.08	0\\
53.09	0\\
53.1	0\\
53.11	0\\
53.12	0\\
53.13	0\\
53.14	0\\
53.15	0\\
53.16	0\\
53.17	0\\
53.18	0\\
53.19	0\\
53.2	0\\
53.21	0\\
53.22	0\\
53.23	0\\
53.24	0\\
53.25	0\\
53.26	0\\
53.27	0\\
53.28	0\\
53.29	0\\
53.3	0\\
53.31	0\\
53.32	0\\
53.33	0\\
53.34	0\\
53.35	0\\
53.36	0\\
53.37	0\\
53.38	0\\
53.39	0\\
53.4	0\\
53.41	0\\
53.42	0\\
53.43	0\\
53.44	0\\
53.45	0\\
53.46	0\\
53.47	0\\
53.48	0\\
53.49	0\\
53.5	0\\
53.51	0\\
53.52	0\\
53.53	0\\
53.54	0\\
53.55	0\\
53.56	0\\
53.57	0\\
53.58	0\\
53.59	0\\
53.6	0\\
53.61	0\\
53.62	0\\
53.63	0\\
53.64	0\\
53.65	0\\
53.66	0\\
53.67	0\\
53.68	0\\
53.69	0\\
53.7	0\\
53.71	0\\
53.72	0\\
53.73	0\\
53.74	0\\
53.75	0\\
53.76	0\\
53.77	0\\
53.78	0\\
53.79	0\\
53.8	0\\
53.81	0\\
53.82	0\\
53.83	0\\
53.84	0\\
53.85	0\\
53.86	0\\
53.87	0\\
53.88	0\\
53.89	0\\
53.9	0\\
53.91	0\\
53.92	0\\
53.93	0\\
53.94	0\\
53.95	0\\
53.96	0\\
53.97	0\\
53.98	0\\
53.99	0\\
54	0\\
54.01	0\\
54.02	0\\
54.03	0\\
54.04	0\\
54.05	0\\
54.06	0\\
54.07	0\\
54.08	0\\
54.09	0\\
54.1	0\\
54.11	0\\
54.12	0\\
54.13	0\\
54.14	0\\
54.15	0\\
54.16	0\\
54.17	0\\
54.18	0\\
54.19	0\\
54.2	0\\
54.21	0\\
54.22	0\\
54.23	0\\
54.24	0\\
54.25	0\\
54.26	0\\
54.27	0\\
54.28	0\\
54.29	0\\
54.3	0\\
54.31	0\\
54.32	0\\
54.33	0\\
54.34	0\\
54.35	0\\
54.36	0\\
54.37	0\\
54.38	0\\
54.39	0\\
54.4	0\\
54.41	0\\
54.42	0\\
54.43	0\\
54.44	0\\
54.45	0\\
54.46	0\\
54.47	0\\
54.48	0\\
54.49	0\\
54.5	0\\
54.51	0\\
54.52	0\\
54.53	0\\
54.54	0\\
54.55	0\\
54.56	0\\
54.57	0\\
54.58	0\\
54.59	0\\
54.6	0\\
54.61	0\\
54.62	0\\
54.63	0\\
54.64	0\\
54.65	0\\
54.66	0\\
54.67	0\\
54.68	0\\
54.69	0\\
54.7	0\\
54.71	0\\
54.72	0\\
54.73	0\\
54.74	0\\
54.75	0\\
54.76	0\\
54.77	0\\
54.78	0\\
54.79	0\\
54.8	0\\
54.81	0\\
54.82	0\\
54.83	0\\
54.84	0\\
54.85	0\\
54.86	0\\
54.87	0\\
54.88	0\\
54.89	0\\
54.9	0\\
54.91	0\\
54.92	0\\
54.93	0\\
54.94	0\\
54.95	0\\
54.96	0\\
54.97	0\\
54.98	0\\
54.99	0\\
55	0\\
55.01	0\\
55.02	0\\
55.03	0\\
55.04	0\\
55.05	0\\
55.06	0\\
55.07	0\\
55.08	0\\
55.09	0\\
55.1	0\\
55.11	0\\
55.12	0\\
55.13	0\\
55.14	0\\
55.15	0\\
55.16	0\\
55.17	0\\
55.18	0\\
55.19	0\\
55.2	0\\
55.21	0\\
55.22	0\\
55.23	0\\
55.24	0\\
55.25	0\\
55.26	0\\
55.27	0\\
55.28	0\\
55.29	0\\
55.3	0\\
55.31	0\\
55.32	0\\
55.33	0\\
55.34	0\\
55.35	0\\
55.36	0\\
55.37	0\\
55.38	0\\
55.39	0\\
55.4	0\\
55.41	0\\
55.42	0\\
55.43	0\\
55.44	0\\
55.45	0\\
55.46	0\\
55.47	0\\
55.48	0\\
55.49	0\\
55.5	0\\
55.51	0\\
55.52	0\\
55.53	0\\
55.54	0\\
55.55	0\\
55.56	0\\
55.57	0\\
55.58	0\\
55.59	0\\
55.6	0\\
55.61	0\\
55.62	0\\
55.63	0\\
55.64	0\\
55.65	0\\
55.66	0\\
55.67	0\\
55.68	0\\
55.69	0\\
55.7	0\\
55.71	0\\
55.72	0\\
55.73	0\\
55.74	0\\
55.75	0\\
55.76	0\\
55.77	0\\
55.78	0\\
55.79	0\\
55.8	0\\
55.81	0\\
55.82	0\\
55.83	0\\
55.84	0\\
55.85	0\\
55.86	0\\
55.87	0\\
55.88	0\\
55.89	0\\
55.9	0\\
55.91	0\\
55.92	0\\
55.93	0\\
55.94	0\\
55.95	0\\
55.96	0\\
55.97	0\\
55.98	0\\
55.99	0\\
56	0\\
56.01	0\\
56.02	0\\
56.03	0\\
56.04	0\\
56.05	0\\
56.06	0\\
56.07	0\\
56.08	0\\
56.09	0\\
56.1	0\\
56.11	0\\
56.12	0\\
56.13	0\\
56.14	0\\
56.15	0\\
56.16	0\\
56.17	0\\
56.18	0\\
56.19	0\\
56.2	0\\
56.21	0\\
56.22	0\\
56.23	0\\
56.24	0\\
56.25	0\\
56.26	0\\
56.27	0\\
56.28	0\\
56.29	0\\
56.3	0\\
56.31	0\\
56.32	0\\
56.33	0\\
56.34	0\\
56.35	0\\
56.36	0\\
56.37	0\\
56.38	0\\
56.39	0\\
56.4	0\\
56.41	0\\
56.42	0\\
56.43	0\\
56.44	0\\
56.45	0\\
56.46	0\\
56.47	0\\
56.48	0\\
56.49	0\\
56.5	0\\
56.51	0\\
56.52	0\\
56.53	0\\
56.54	0\\
56.55	0\\
56.56	0\\
56.57	0\\
56.58	0\\
56.59	0\\
56.6	0\\
56.61	0\\
56.62	0\\
56.63	0\\
56.64	0\\
56.65	0\\
56.66	0\\
56.67	0\\
56.68	0\\
56.69	0\\
56.7	0\\
56.71	0\\
56.72	0\\
56.73	0\\
56.74	0\\
56.75	0\\
56.76	0\\
56.77	0\\
56.78	0\\
56.79	0\\
56.8	0\\
56.81	0\\
56.82	0\\
56.83	0\\
56.84	0\\
56.85	0\\
56.86	0\\
56.87	0\\
56.88	0\\
56.89	0\\
56.9	0\\
56.91	0\\
56.92	0\\
56.93	0\\
56.94	0\\
56.95	0\\
56.96	0\\
56.97	0\\
56.98	0\\
56.99	0\\
57	0\\
57.01	0\\
57.02	0\\
57.03	0\\
57.04	0\\
57.05	0\\
57.06	0\\
57.07	0\\
57.08	0\\
57.09	0\\
57.1	0\\
57.11	0\\
57.12	0\\
57.13	0\\
57.14	0\\
57.15	0\\
57.16	0\\
57.17	0\\
57.18	0\\
57.19	0\\
57.2	0\\
57.21	0\\
57.22	0\\
57.23	0\\
57.24	0\\
57.25	0\\
57.26	0\\
57.27	0\\
57.28	0\\
57.29	0\\
57.3	0\\
57.31	0\\
57.32	0\\
57.33	0\\
57.34	0\\
57.35	0\\
57.36	0\\
57.37	0\\
57.38	0\\
57.39	0\\
57.4	0\\
57.41	0\\
57.42	0\\
57.43	0\\
57.44	0\\
57.45	0\\
57.46	0\\
57.47	0\\
57.48	0\\
57.49	0\\
57.5	0\\
57.51	0\\
57.52	0\\
57.53	0\\
57.54	0\\
57.55	0\\
57.56	0\\
57.57	0\\
57.58	0\\
57.59	0\\
57.6	0\\
57.61	0\\
57.62	0\\
57.63	0\\
57.64	0\\
57.65	0\\
57.66	0\\
57.67	0\\
57.68	0\\
57.69	0\\
57.7	0\\
57.71	0\\
57.72	0\\
57.73	0\\
57.74	0\\
57.75	0\\
57.76	0\\
57.77	0\\
57.78	0\\
57.79	0\\
57.8	0\\
57.81	0\\
57.82	0\\
57.83	0\\
57.84	0\\
57.85	0\\
57.86	0\\
57.87	0\\
57.88	0\\
57.89	0\\
57.9	0\\
57.91	0\\
57.92	0\\
57.93	0\\
57.94	0\\
57.95	0\\
57.96	0\\
57.97	0\\
57.98	0\\
57.99	0\\
58	0\\
58.01	0\\
58.02	0\\
58.03	0\\
58.04	0\\
58.05	0\\
58.06	0\\
58.07	0\\
58.08	0\\
58.09	0\\
58.1	0\\
58.11	0\\
58.12	0\\
58.13	0\\
58.14	0\\
58.15	0\\
58.16	0\\
58.17	0\\
58.18	0\\
58.19	0\\
58.2	0\\
58.21	0\\
58.22	0\\
58.23	0\\
58.24	0\\
58.25	0\\
58.26	0\\
58.27	0\\
58.28	0\\
58.29	0\\
58.3	0\\
58.31	0\\
58.32	0\\
58.33	0\\
58.34	0\\
58.35	0\\
58.36	0\\
58.37	0\\
58.38	0\\
58.39	0\\
58.4	0\\
58.41	0\\
58.42	0\\
58.43	0\\
58.44	0\\
58.45	0\\
58.46	0\\
58.47	0\\
58.48	0\\
58.49	0\\
58.5	0\\
58.51	0\\
58.52	0\\
58.53	0\\
58.54	0\\
58.55	0\\
58.56	0\\
58.57	0\\
58.58	0\\
58.59	0\\
58.6	0\\
58.61	0\\
58.62	0\\
58.63	0\\
58.64	0\\
58.65	0\\
58.66	0\\
58.67	0\\
58.68	0\\
58.69	0\\
58.7	0\\
58.71	0\\
58.72	0\\
58.73	0\\
58.74	0\\
58.75	0\\
58.76	0\\
58.77	0\\
58.78	0\\
58.79	0\\
58.8	0\\
58.81	0\\
58.82	0\\
58.83	0\\
58.84	0\\
58.85	0\\
58.86	0\\
58.87	0\\
58.88	0\\
58.89	0\\
58.9	0\\
58.91	0\\
58.92	0\\
58.93	0\\
58.94	0\\
58.95	0\\
58.96	0\\
58.97	0\\
58.98	0\\
58.99	0\\
59	0\\
59.01	0\\
59.02	0\\
59.03	0\\
59.04	0\\
59.05	0\\
59.06	0\\
59.07	0\\
59.08	0\\
59.09	0\\
59.1	0\\
59.11	0\\
59.12	0\\
59.13	0\\
59.14	0\\
59.15	0\\
59.16	0\\
59.17	0\\
59.18	0\\
59.19	0\\
59.2	0\\
59.21	0\\
59.22	0\\
59.23	0\\
59.24	0\\
59.25	0\\
59.26	0\\
59.27	0\\
59.28	0\\
59.29	0\\
59.3	0\\
59.31	0\\
59.32	0\\
59.33	0\\
59.34	0\\
59.35	0\\
59.36	0\\
59.37	0\\
59.38	0\\
59.39	0\\
59.4	0\\
59.41	0\\
59.42	0\\
59.43	0\\
59.44	0\\
59.45	0\\
59.46	0\\
59.47	0\\
59.48	0\\
59.49	0\\
59.5	0\\
59.51	0\\
59.52	0\\
59.53	0\\
59.54	0\\
59.55	0\\
59.56	0\\
59.57	0\\
59.58	0\\
59.59	0\\
59.6	0\\
59.61	0\\
59.62	0\\
59.63	0\\
59.64	0\\
59.65	0\\
59.66	0\\
59.67	0\\
59.68	0\\
59.69	0\\
59.7	0\\
59.71	0\\
59.72	0\\
59.73	0\\
59.74	0\\
59.75	0\\
59.76	0\\
59.77	0\\
59.78	0\\
59.79	0\\
59.8	0\\
59.81	0\\
59.82	0\\
59.83	0\\
59.84	0\\
59.85	0\\
59.86	0\\
59.87	0\\
59.88	0\\
59.89	0\\
59.9	0\\
59.91	0\\
59.92	0\\
59.93	0\\
59.94	0\\
59.95	0\\
59.96	0\\
59.97	0\\
59.98	0\\
59.99	0\\
60	0\\
60.01	0\\
60.02	0\\
60.03	0\\
60.04	0\\
60.05	0\\
60.06	0\\
60.07	0\\
60.08	0\\
60.09	0\\
60.1	0\\
60.11	0\\
60.12	0\\
60.13	0\\
60.14	0\\
60.15	0\\
60.16	0\\
60.17	0\\
60.18	0\\
60.19	0\\
60.2	0\\
60.21	0\\
60.22	0\\
60.23	0\\
60.24	0\\
60.25	0\\
60.26	0\\
60.27	0\\
60.28	0\\
60.29	0\\
60.3	0\\
60.31	0\\
60.32	0\\
60.33	0\\
60.34	0\\
60.35	0\\
60.36	0\\
60.37	0\\
60.38	0\\
60.39	0\\
60.4	0\\
60.41	0\\
60.42	0\\
60.43	0\\
60.44	0\\
60.45	0\\
60.46	0\\
60.47	0\\
60.48	0\\
60.49	0\\
60.5	0\\
60.51	0\\
60.52	0\\
60.53	0\\
60.54	0\\
60.55	0\\
60.56	0\\
60.57	0\\
60.58	0\\
60.59	0\\
60.6	0\\
60.61	0\\
60.62	0\\
60.63	0\\
60.64	0\\
60.65	0\\
60.66	0\\
60.67	0\\
60.68	0\\
60.69	0\\
60.7	0\\
60.71	0\\
60.72	0\\
60.73	0\\
60.74	0\\
60.75	0\\
60.76	0\\
60.77	0\\
60.78	0\\
60.79	0\\
60.8	0\\
60.81	0\\
60.82	0\\
60.83	0\\
60.84	0\\
60.85	0\\
60.86	0\\
60.87	0\\
60.88	0\\
60.89	0\\
60.9	0\\
60.91	0\\
60.92	0\\
60.93	0\\
60.94	0\\
60.95	0\\
60.96	0\\
60.97	0\\
60.98	0\\
60.99	0\\
61	0\\
61.01	0\\
61.02	0\\
61.03	0\\
61.04	0\\
61.05	0\\
61.06	0\\
61.07	0\\
61.08	0\\
61.09	0\\
61.1	0\\
61.11	0\\
61.12	0\\
61.13	0\\
61.14	0\\
61.15	0\\
61.16	0\\
61.17	0\\
61.18	0\\
61.19	0\\
61.2	0\\
61.21	0\\
61.22	0\\
61.23	0\\
61.24	0\\
61.25	0\\
61.26	0\\
61.27	0\\
61.28	0\\
61.29	0\\
61.3	0\\
61.31	0\\
61.32	0\\
61.33	0\\
61.34	0\\
61.35	0\\
61.36	0\\
61.37	0\\
61.38	0\\
61.39	0\\
61.4	0\\
61.41	0\\
61.42	0\\
61.43	0\\
61.44	0\\
61.45	0\\
61.46	0\\
61.47	0\\
61.48	0\\
61.49	0\\
61.5	0\\
61.51	0\\
61.52	0\\
61.53	0\\
61.54	0\\
61.55	0\\
61.56	0\\
61.57	0\\
61.58	0\\
61.59	0\\
61.6	0\\
61.61	0\\
61.62	0\\
61.63	0\\
61.64	0\\
61.65	0\\
61.66	0\\
61.67	0\\
61.68	0\\
61.69	0\\
61.7	0\\
61.71	0\\
61.72	0\\
61.73	0\\
61.74	0\\
61.75	0\\
61.76	0\\
61.77	0\\
61.78	0\\
61.79	0\\
61.8	0\\
61.81	0\\
61.82	0\\
61.83	0\\
61.84	0\\
61.85	0\\
61.86	0\\
61.87	0\\
61.88	0\\
61.89	0\\
61.9	0\\
61.91	0\\
61.92	0\\
61.93	0\\
61.94	0\\
61.95	0\\
61.96	0\\
61.97	0\\
61.98	0\\
61.99	0\\
62	0\\
62.01	0\\
62.02	0\\
62.03	0\\
62.04	0\\
62.05	0\\
62.06	0\\
62.07	0\\
62.08	0\\
62.09	0\\
62.1	0\\
62.11	0\\
62.12	0\\
62.13	0\\
62.14	0\\
62.15	0\\
62.16	0\\
62.17	0\\
62.18	0\\
62.19	0\\
62.2	0\\
62.21	0\\
62.22	0\\
62.23	0\\
62.24	0\\
62.25	0\\
62.26	0\\
62.27	0\\
62.28	0\\
62.29	0\\
62.3	0\\
62.31	0\\
62.32	0\\
62.33	0\\
62.34	0\\
62.35	0\\
62.36	0\\
62.37	0\\
62.38	0\\
62.39	0\\
62.4	0\\
62.41	0\\
62.42	0\\
62.43	0\\
62.44	0\\
62.45	0\\
62.46	0\\
62.47	0\\
62.48	0\\
62.49	0\\
62.5	0\\
62.51	0\\
62.52	0\\
62.53	0\\
62.54	0\\
62.55	0\\
62.56	0\\
62.57	0\\
62.58	0\\
62.59	0\\
62.6	0\\
62.61	0\\
62.62	0\\
62.63	0\\
62.64	0\\
62.65	0\\
62.66	0\\
62.67	0\\
62.68	0\\
62.69	0\\
62.7	0\\
62.71	0\\
62.72	0\\
62.73	0\\
62.74	0\\
62.75	0\\
62.76	0\\
62.77	0\\
62.78	0\\
62.79	0\\
62.8	0\\
62.81	0\\
62.82	0\\
62.83	0\\
62.84	0\\
62.85	0\\
62.86	0\\
62.87	0\\
62.88	0\\
62.89	0\\
62.9	0\\
62.91	0\\
62.92	0\\
62.93	0\\
62.94	0\\
62.95	0\\
62.96	0\\
62.97	0\\
62.98	0\\
62.99	0\\
63	0\\
63.01	0\\
63.02	0\\
63.03	0\\
63.04	0\\
63.05	0\\
63.06	0\\
63.07	0\\
63.08	0\\
63.09	0\\
63.1	0\\
63.11	0\\
63.12	0\\
63.13	0\\
63.14	0\\
63.15	0\\
63.16	0\\
63.17	0\\
63.18	0\\
63.19	0\\
63.2	0\\
63.21	0\\
63.22	0\\
63.23	0\\
63.24	0\\
63.25	0\\
63.26	0\\
63.27	0\\
63.28	0\\
63.29	0\\
63.3	0\\
63.31	0\\
63.32	0\\
63.33	0\\
63.34	0\\
63.35	0\\
63.36	0\\
63.37	0\\
63.38	0\\
63.39	0\\
63.4	0\\
63.41	0\\
63.42	0\\
63.43	0\\
63.44	0\\
63.45	0\\
63.46	0\\
63.47	0\\
63.48	0\\
63.49	0\\
63.5	0\\
63.51	0\\
63.52	0\\
63.53	0\\
63.54	0\\
63.55	0\\
63.56	0\\
63.57	0\\
63.58	0\\
63.59	0\\
63.6	0\\
63.61	0\\
63.62	0\\
63.63	0\\
63.64	0\\
63.65	0\\
63.66	0\\
63.67	0\\
63.68	0\\
63.69	0\\
63.7	0\\
63.71	0\\
63.72	0\\
63.73	0\\
63.74	0\\
63.75	0\\
63.76	0\\
63.77	0\\
63.78	0\\
63.79	0\\
63.8	0\\
63.81	0\\
63.82	0\\
63.83	0\\
63.84	0\\
63.85	0\\
63.86	0\\
63.87	0\\
63.88	0\\
63.89	0\\
63.9	0\\
63.91	0\\
63.92	0\\
63.93	0\\
63.94	0\\
63.95	0\\
63.96	0\\
63.97	0\\
63.98	0\\
63.99	0\\
64	0\\
64.01	0\\
64.02	0\\
64.03	0\\
64.04	0\\
64.05	0\\
64.06	0\\
64.07	0\\
64.08	0\\
64.09	0\\
64.1	0\\
64.11	0\\
64.12	0\\
64.13	0\\
64.14	0\\
64.15	0\\
64.16	0\\
64.17	0\\
64.18	0\\
64.19	0\\
64.2	0\\
64.21	0\\
64.22	0\\
64.23	0\\
64.24	0\\
64.25	0\\
64.26	0\\
64.27	0\\
64.28	0\\
64.29	0\\
64.3	0\\
64.31	0\\
64.32	0\\
64.33	0\\
64.34	0\\
64.35	0\\
64.36	0\\
64.37	0\\
64.38	0\\
64.39	0\\
64.4	0\\
64.41	0\\
64.42	0\\
64.43	0\\
64.44	0\\
64.45	0\\
64.46	0\\
64.47	0\\
64.48	0\\
64.49	0\\
64.5	0\\
64.51	0\\
64.52	0\\
64.53	0\\
64.54	0\\
64.55	0\\
64.56	0\\
64.57	0\\
64.58	0\\
64.59	0\\
64.6	0\\
64.61	0\\
64.62	0\\
64.63	0\\
64.64	0\\
64.65	0\\
64.66	0\\
64.67	0\\
64.68	0\\
64.69	0\\
64.7	0\\
64.71	0\\
64.72	0\\
64.73	0\\
64.74	0\\
64.75	0\\
64.76	0\\
64.77	0\\
64.78	0\\
64.79	0\\
64.8	0\\
64.81	0\\
64.82	0\\
64.83	0\\
64.84	0\\
64.85	0\\
64.86	0\\
64.87	0\\
64.88	0\\
64.89	0\\
64.9	0\\
64.91	0\\
64.92	0\\
64.93	0\\
64.94	0\\
64.95	0\\
64.96	0\\
64.97	0\\
64.98	0\\
64.99	0\\
65	0\\
65.01	0\\
65.02	0\\
65.03	0\\
65.04	0\\
65.05	0\\
65.06	0\\
65.07	0\\
65.08	0\\
65.09	0\\
65.1	0\\
65.11	0\\
65.12	0\\
65.13	0\\
65.14	0\\
65.15	0\\
65.16	0\\
65.17	0\\
65.18	0\\
65.19	0\\
65.2	0\\
65.21	0\\
65.22	0\\
65.23	0\\
65.24	0\\
65.25	0\\
65.26	0\\
65.27	0\\
65.28	0\\
65.29	0\\
65.3	0\\
65.31	0\\
65.32	0\\
65.33	0\\
65.34	0\\
65.35	0\\
65.36	0\\
65.37	0\\
65.38	0\\
65.39	0\\
65.4	0\\
65.41	0\\
65.42	0\\
65.43	0\\
65.44	0\\
65.45	0\\
65.46	0\\
65.47	0\\
65.48	0\\
65.49	0\\
65.5	0\\
65.51	0\\
65.52	0\\
65.53	0\\
65.54	0\\
65.55	0\\
65.56	0\\
65.57	0\\
65.58	0\\
65.59	0\\
65.6	0\\
65.61	0\\
65.62	0\\
65.63	0\\
65.64	0\\
65.65	0\\
65.66	0\\
65.67	0\\
65.68	0\\
65.69	0\\
65.7	0\\
65.71	0\\
65.72	0\\
65.73	0\\
65.74	0\\
65.75	0\\
65.76	0\\
65.77	0\\
65.78	0\\
65.79	0\\
65.8	0\\
65.81	0\\
65.82	0\\
65.83	0\\
65.84	0\\
65.85	0\\
65.86	0\\
65.87	0\\
65.88	0\\
65.89	0\\
65.9	0\\
65.91	0\\
65.92	0\\
65.93	0\\
65.94	0\\
65.95	0\\
65.96	0\\
65.97	0\\
65.98	0\\
65.99	0\\
66	0\\
66.01	0\\
66.02	0\\
66.03	0\\
66.04	0\\
66.05	0\\
66.06	0\\
66.07	0\\
66.08	0\\
66.09	0\\
66.1	0\\
66.11	0\\
66.12	0\\
66.13	0\\
66.14	0\\
66.15	0\\
66.16	0\\
66.17	0\\
66.18	0\\
66.19	0\\
66.2	0\\
66.21	0\\
66.22	0\\
66.23	0\\
66.24	0\\
66.25	0\\
66.26	0\\
66.27	0\\
66.28	0\\
66.29	0\\
66.3	0\\
66.31	0\\
66.32	0\\
66.33	0\\
66.34	0\\
66.35	0\\
66.36	0\\
66.37	0\\
66.38	0\\
66.39	0\\
66.4	0\\
66.41	0\\
66.42	0\\
66.43	0\\
66.44	0\\
66.45	0\\
66.46	0\\
66.47	0\\
66.48	0\\
66.49	0\\
66.5	0\\
66.51	0\\
66.52	0\\
66.53	0\\
66.54	0\\
66.55	0\\
66.56	0\\
66.57	0\\
66.58	0\\
66.59	0\\
66.6	0\\
66.61	0\\
66.62	0\\
66.63	0\\
66.64	0\\
66.65	0\\
66.66	0\\
66.67	0\\
66.68	0\\
66.69	0\\
66.7	0\\
66.71	0\\
66.72	0\\
66.73	0\\
66.74	0\\
66.75	0\\
66.76	0\\
66.77	0\\
66.78	0\\
66.79	0\\
66.8	0\\
66.81	0\\
66.82	0\\
66.83	0\\
66.84	0\\
66.85	0\\
66.86	0\\
66.87	0\\
66.88	0\\
66.89	0\\
66.9	0\\
66.91	0\\
66.92	0\\
66.93	0\\
66.94	0\\
66.95	0\\
66.96	0\\
66.97	0\\
66.98	0\\
66.99	0\\
67	0\\
67.01	0\\
67.02	0\\
67.03	0\\
67.04	0\\
67.05	0\\
67.06	0\\
67.07	0\\
67.08	0\\
67.09	0\\
67.1	0\\
67.11	0\\
67.12	0\\
67.13	0\\
67.14	0\\
67.15	0\\
67.16	0\\
67.17	0\\
67.18	0\\
67.19	0\\
67.2	0\\
67.21	0\\
67.22	0\\
67.23	0\\
67.24	0\\
67.25	0\\
67.26	0\\
67.27	0\\
67.28	0\\
67.29	0\\
67.3	0\\
67.31	0\\
67.32	0\\
67.33	0\\
67.34	0\\
67.35	0\\
67.36	0\\
67.37	0\\
67.38	0\\
67.39	0\\
67.4	0\\
67.41	0\\
67.42	0\\
67.43	0\\
67.44	0\\
67.45	0\\
67.46	0\\
67.47	0\\
67.48	0\\
67.49	0\\
67.5	0\\
67.51	0\\
67.52	0\\
67.53	0\\
67.54	0\\
67.55	0\\
67.56	0\\
67.57	0\\
67.58	0\\
67.59	0\\
67.6	0\\
67.61	0\\
67.62	0\\
67.63	0\\
67.64	0\\
67.65	0\\
67.66	0\\
67.67	0\\
67.68	0\\
67.69	0\\
67.7	0\\
67.71	0\\
67.72	0\\
67.73	0\\
67.74	0\\
67.75	0\\
67.76	0\\
67.77	0\\
67.78	0\\
67.79	0\\
67.8	0\\
67.81	0\\
67.82	0\\
67.83	0\\
67.84	0\\
67.85	0\\
67.86	0\\
67.87	0\\
67.88	0\\
67.89	0\\
67.9	0\\
67.91	0\\
67.92	0\\
67.93	0\\
67.94	0\\
67.95	0\\
67.96	0\\
67.97	0\\
67.98	0\\
67.99	0\\
68	0\\
68.01	0\\
68.02	0\\
68.03	0\\
68.04	0\\
68.05	0\\
68.06	0\\
68.07	0\\
68.08	0\\
68.09	0\\
68.1	0\\
68.11	0\\
68.12	0\\
68.13	0\\
68.14	0\\
68.15	0\\
68.16	0\\
68.17	0\\
68.18	0\\
68.19	0\\
68.2	0\\
68.21	0\\
68.22	0\\
68.23	0\\
68.24	0\\
68.25	0\\
68.26	0\\
68.27	0\\
68.28	0\\
68.29	0\\
68.3	0\\
68.31	0\\
68.32	0\\
68.33	0\\
68.34	0\\
68.35	0\\
68.36	0\\
68.37	0\\
68.38	0\\
68.39	0\\
68.4	0\\
68.41	0\\
68.42	0\\
68.43	0\\
68.44	0\\
68.45	0\\
68.46	0\\
68.47	0\\
68.48	0\\
68.49	0\\
68.5	0\\
68.51	0\\
68.52	0\\
68.53	0\\
68.54	0\\
68.55	0\\
68.56	0\\
68.57	0\\
68.58	0\\
68.59	0\\
68.6	0\\
68.61	0\\
68.62	0\\
68.63	0\\
68.64	0\\
68.65	0\\
68.66	0\\
68.67	0\\
68.68	0\\
68.69	0\\
68.7	0\\
68.71	0\\
68.72	0\\
68.73	0\\
68.74	0\\
68.75	0\\
68.76	0\\
68.77	0\\
68.78	0\\
68.79	0\\
68.8	0\\
68.81	0\\
68.82	0\\
68.83	0\\
68.84	0\\
68.85	0\\
68.86	0\\
68.87	0\\
68.88	0\\
68.89	0\\
68.9	0\\
68.91	0\\
68.92	0\\
68.93	0\\
68.94	0\\
68.95	0\\
68.96	0\\
68.97	0\\
68.98	0\\
68.99	0\\
69	0\\
69.01	0\\
69.02	0\\
69.03	0\\
69.04	0\\
69.05	0\\
69.06	0\\
69.07	0\\
69.08	0\\
69.09	0\\
69.1	0\\
69.11	0\\
69.12	0\\
69.13	0\\
69.14	0\\
69.15	0\\
69.16	0\\
69.17	0\\
69.18	0\\
69.19	0\\
69.2	0\\
69.21	0\\
69.22	0\\
69.23	0\\
69.24	0\\
69.25	0\\
69.26	0\\
69.27	0\\
69.28	0\\
69.29	0\\
69.3	0\\
69.31	0\\
69.32	0\\
69.33	0\\
69.34	0\\
69.35	0\\
69.36	0\\
69.37	0\\
69.38	0\\
69.39	0\\
69.4	0\\
69.41	0\\
69.42	0\\
69.43	0\\
69.44	0\\
69.45	0\\
69.46	0\\
69.47	0\\
69.48	0\\
69.49	0\\
69.5	0\\
69.51	0\\
69.52	0\\
69.53	0\\
69.54	0\\
69.55	0\\
69.56	0\\
69.57	0\\
69.58	0\\
69.59	0\\
69.6	0\\
69.61	0\\
69.62	0\\
69.63	0\\
69.64	0\\
69.65	0\\
69.66	0\\
69.67	0\\
69.68	0\\
69.69	0\\
69.7	0\\
69.71	0\\
69.72	0\\
69.73	0\\
69.74	0\\
69.75	0\\
69.76	0\\
69.77	0\\
69.78	0\\
69.79	0\\
69.8	0\\
69.81	0\\
69.82	0\\
69.83	0\\
69.84	0\\
69.85	0\\
69.86	0\\
69.87	0\\
69.88	0\\
69.89	0\\
69.9	0\\
69.91	0\\
69.92	0\\
69.93	0\\
69.94	0\\
69.95	0\\
69.96	0\\
69.97	0\\
69.98	0\\
69.99	0\\
70	0\\
70.01	0\\
70.02	0\\
70.03	0\\
70.04	0\\
70.05	0\\
70.06	0\\
70.07	0\\
70.08	0\\
70.09	0\\
70.1	0\\
70.11	0\\
70.12	0\\
70.13	0\\
70.14	0\\
70.15	0\\
70.16	0\\
70.17	0\\
70.18	0\\
70.19	0\\
70.2	0\\
70.21	0\\
70.22	0\\
70.23	0\\
70.24	0\\
70.25	0\\
70.26	0\\
70.27	0\\
70.28	0\\
70.29	0\\
70.3	0\\
70.31	0\\
70.32	0\\
70.33	0\\
70.34	0\\
70.35	0\\
70.36	0\\
70.37	0\\
70.38	0\\
70.39	0\\
70.4	0\\
70.41	0\\
70.42	0\\
70.43	0\\
70.44	0\\
70.45	0\\
70.46	0\\
70.47	0\\
70.48	0\\
70.49	0\\
70.5	0\\
70.51	0\\
70.52	0\\
70.53	0\\
70.54	0\\
70.55	0\\
70.56	0\\
70.57	0\\
70.58	0\\
70.59	0\\
70.6	0\\
70.61	0\\
70.62	0\\
70.63	0\\
70.64	0\\
70.65	0\\
70.66	0\\
70.67	0\\
70.68	0\\
70.69	0\\
70.7	0\\
70.71	0\\
70.72	0\\
70.73	0\\
70.74	0\\
70.75	0\\
70.76	0\\
70.77	0\\
70.78	0\\
70.79	0\\
70.8	0\\
70.81	0\\
70.82	0\\
70.83	0\\
70.84	0\\
70.85	0\\
70.86	0\\
70.87	0\\
70.88	0\\
70.89	0\\
70.9	0\\
70.91	0\\
70.92	0\\
70.93	0\\
70.94	0\\
70.95	0\\
70.96	0\\
70.97	0\\
70.98	0\\
70.99	0\\
71	0\\
71.01	0\\
71.02	0\\
71.03	0\\
71.04	0\\
71.05	0\\
71.06	0\\
71.07	0\\
71.08	0\\
71.09	0\\
71.1	0\\
71.11	0\\
71.12	0\\
71.13	0\\
71.14	0\\
71.15	0\\
71.16	0\\
71.17	0\\
71.18	0\\
71.19	0\\
71.2	0\\
71.21	0\\
71.22	0\\
71.23	0\\
71.24	0\\
71.25	0\\
71.26	0\\
71.27	0\\
71.28	0\\
71.29	0\\
71.3	0\\
71.31	0\\
71.32	0\\
71.33	0\\
71.34	0\\
71.35	0\\
71.36	0\\
71.37	0\\
71.38	0\\
71.39	0\\
71.4	0\\
71.41	0\\
71.42	0\\
71.43	0\\
71.44	0\\
71.45	0\\
71.46	0\\
71.47	0\\
71.48	0\\
71.49	0\\
71.5	0\\
71.51	0\\
71.52	0\\
71.53	0\\
71.54	0\\
71.55	0\\
71.56	0\\
71.57	0\\
71.58	0\\
71.59	0\\
71.6	0\\
71.61	0\\
71.62	0\\
71.63	0\\
71.64	0\\
71.65	0\\
71.66	0\\
71.67	0\\
71.68	0\\
71.69	0\\
71.7	0\\
71.71	0\\
71.72	0\\
71.73	0\\
71.74	0\\
71.75	0\\
71.76	0\\
71.77	0\\
71.78	0\\
71.79	0\\
71.8	0\\
71.81	0\\
71.82	0\\
71.83	0\\
71.84	0\\
71.85	0\\
71.86	0\\
71.87	0\\
71.88	0\\
71.89	0\\
71.9	0\\
71.91	0\\
71.92	0\\
71.93	0\\
71.94	0\\
71.95	0\\
71.96	0\\
71.97	0\\
71.98	0\\
71.99	0\\
72	0\\
72.01	0\\
72.02	0\\
72.03	0\\
72.04	0\\
72.05	0\\
72.06	0\\
72.07	0\\
72.08	0\\
72.09	0\\
72.1	0\\
72.11	0\\
72.12	0\\
72.13	0\\
72.14	0\\
72.15	0\\
72.16	0\\
72.17	0\\
72.18	0\\
72.19	0\\
72.2	0\\
72.21	0\\
72.22	0\\
72.23	0\\
72.24	0\\
72.25	0\\
72.26	0\\
72.27	0\\
72.28	0\\
72.29	0\\
72.3	0\\
72.31	0\\
72.32	0\\
72.33	0\\
72.34	0\\
72.35	0\\
72.36	0\\
72.37	0\\
72.38	0\\
72.39	0\\
72.4	0\\
72.41	0\\
72.42	0\\
72.43	0\\
72.44	0\\
72.45	0\\
72.46	0\\
72.47	0\\
72.48	0\\
72.49	0\\
72.5	0\\
72.51	0\\
72.52	0\\
72.53	0\\
72.54	0\\
72.55	0\\
72.56	0\\
72.57	0\\
72.58	0\\
72.59	0\\
72.6	0\\
72.61	0\\
72.62	0\\
72.63	0\\
72.64	0\\
72.65	0\\
72.66	0\\
72.67	0\\
72.68	0\\
72.69	0\\
72.7	0\\
72.71	0\\
72.72	0\\
72.73	0\\
72.74	0\\
72.75	0\\
72.76	0\\
72.77	0\\
72.78	0\\
72.79	0\\
72.8	0\\
72.81	0\\
72.82	0\\
72.83	0\\
72.84	0\\
72.85	0\\
72.86	0\\
72.87	0\\
72.88	0\\
72.89	0\\
72.9	0\\
72.91	0\\
72.92	0\\
72.93	0\\
72.94	0\\
72.95	0\\
72.96	0\\
72.97	0\\
72.98	0\\
72.99	0\\
73	0\\
73.01	0\\
73.02	0\\
73.03	0\\
73.04	0\\
73.05	0\\
73.06	0\\
73.07	0\\
73.08	0\\
73.09	0\\
73.1	0\\
73.11	0\\
73.12	0\\
73.13	0\\
73.14	0\\
73.15	0\\
73.16	0\\
73.17	0\\
73.18	0\\
73.19	0\\
73.2	0\\
73.21	0\\
73.22	0\\
73.23	0\\
73.24	0\\
73.25	0\\
73.26	0\\
73.27	0\\
73.28	0\\
73.29	0\\
73.3	0\\
73.31	0\\
73.32	0\\
73.33	0\\
73.34	0\\
73.35	0\\
73.36	0\\
73.37	0\\
73.38	0\\
73.39	0\\
73.4	0\\
73.41	0\\
73.42	0\\
73.43	0\\
73.44	0\\
73.45	0\\
73.46	0\\
73.47	0\\
73.48	0\\
73.49	0\\
73.5	0\\
73.51	0\\
73.52	0\\
73.53	0\\
73.54	0\\
73.55	0\\
73.56	0\\
73.57	0\\
73.58	0\\
73.59	0\\
73.6	0\\
73.61	0\\
73.62	0\\
73.63	0\\
73.64	0\\
73.65	0\\
73.66	0\\
73.67	0\\
73.68	0\\
73.69	0\\
73.7	0\\
73.71	0\\
73.72	0\\
73.73	0\\
73.74	0\\
73.75	0\\
73.76	0\\
73.77	0\\
73.78	0\\
73.79	0\\
73.8	0\\
73.81	0\\
73.82	0\\
73.83	0\\
73.84	0\\
73.85	0\\
73.86	0\\
73.87	0\\
73.88	0\\
73.89	0\\
73.9	0\\
73.91	0\\
73.92	0\\
73.93	0\\
73.94	0\\
73.95	0\\
73.96	0\\
73.97	0\\
73.98	0\\
73.99	0\\
74	0\\
74.01	0\\
74.02	0\\
74.03	0\\
74.04	0\\
74.05	0\\
74.06	0\\
74.07	0\\
74.08	0\\
74.09	0\\
74.1	0\\
74.11	0\\
74.12	0\\
74.13	0\\
74.14	0\\
74.15	0\\
74.16	0\\
74.17	0\\
74.18	0\\
74.19	0\\
74.2	0\\
74.21	0\\
74.22	0\\
74.23	0\\
74.24	0\\
74.25	0\\
74.26	0\\
74.27	0\\
74.28	0\\
74.29	0\\
74.3	0\\
74.31	0\\
74.32	0\\
74.33	0\\
74.34	0\\
74.35	0\\
74.36	0\\
74.37	0\\
74.38	0\\
74.39	0\\
74.4	0\\
74.41	0\\
74.42	0\\
74.43	0\\
74.44	0\\
74.45	0\\
74.46	0\\
74.47	0\\
74.48	0\\
74.49	0\\
74.5	0\\
74.51	0\\
74.52	0\\
74.53	0\\
74.54	0\\
74.55	0\\
74.56	0\\
74.57	0\\
74.58	0\\
74.59	0\\
74.6	0\\
74.61	0\\
74.62	0\\
74.63	0\\
74.64	0\\
74.65	0\\
74.66	0\\
74.67	0\\
74.68	0\\
74.69	0\\
74.7	0\\
74.71	0\\
74.72	0\\
74.73	0\\
74.74	0\\
74.75	0\\
74.76	0\\
74.77	0\\
74.78	0\\
74.79	0\\
74.8	0\\
74.81	0\\
74.82	0\\
74.83	0\\
74.84	0\\
74.85	0\\
74.86	0\\
74.87	0\\
74.88	0\\
74.89	0\\
74.9	0\\
74.91	0\\
74.92	0\\
74.93	0\\
74.94	0\\
74.95	0\\
74.96	0\\
74.97	0\\
74.98	0\\
74.99	0\\
75	0\\
75.01	0\\
75.02	0\\
75.03	0\\
75.04	0\\
75.05	0\\
75.06	0\\
75.07	0\\
75.08	0\\
75.09	0\\
75.1	0\\
75.11	0\\
75.12	0\\
75.13	0\\
75.14	0\\
75.15	0\\
75.16	0\\
75.17	0\\
75.18	0\\
75.19	0\\
75.2	0\\
75.21	0\\
75.22	0\\
75.23	0\\
75.24	0\\
75.25	0\\
75.26	0\\
75.27	0\\
75.28	0\\
75.29	0\\
75.3	0\\
75.31	0\\
75.32	0\\
75.33	0\\
75.34	0\\
75.35	0\\
75.36	0\\
75.37	0\\
75.38	0\\
75.39	0\\
75.4	0\\
75.41	0\\
75.42	0\\
75.43	0\\
75.44	0\\
75.45	0\\
75.46	0\\
75.47	0\\
75.48	0\\
75.49	0\\
75.5	0\\
75.51	0\\
75.52	0\\
75.53	0\\
75.54	0\\
75.55	0\\
75.56	0\\
75.57	0\\
75.58	0\\
75.59	0\\
75.6	0\\
75.61	0\\
75.62	0\\
75.63	0\\
75.64	0\\
75.65	0\\
75.66	0\\
75.67	0\\
75.68	0\\
75.69	0\\
75.7	0\\
75.71	0\\
75.72	0\\
75.73	0\\
75.74	0\\
75.75	0\\
75.76	0\\
75.77	0\\
75.78	0\\
75.79	0\\
75.8	0\\
75.81	0\\
75.82	0\\
75.83	0\\
75.84	0\\
75.85	0\\
75.86	0\\
75.87	0\\
75.88	0\\
75.89	0\\
75.9	0\\
75.91	0\\
75.92	0\\
75.93	0\\
75.94	0\\
75.95	0\\
75.96	0\\
75.97	0\\
75.98	0\\
75.99	0\\
76	0\\
76.01	0\\
76.02	0\\
76.03	0\\
76.04	0\\
76.05	0\\
76.06	0\\
76.07	0\\
76.08	0\\
76.09	0\\
76.1	0\\
76.11	0\\
76.12	0\\
76.13	0\\
76.14	0\\
76.15	0\\
76.16	0\\
76.17	0\\
76.18	0\\
76.19	0\\
76.2	0\\
76.21	0\\
76.22	0\\
76.23	0\\
76.24	0\\
76.25	0\\
76.26	0\\
76.27	0\\
76.28	0\\
76.29	0\\
76.3	0\\
76.31	0\\
76.32	0\\
76.33	0\\
76.34	0\\
76.35	0\\
76.36	0\\
76.37	0\\
76.38	0\\
76.39	0\\
76.4	0\\
76.41	0\\
76.42	0\\
76.43	0\\
76.44	0\\
76.45	0\\
76.46	0\\
76.47	0\\
76.48	0\\
76.49	0\\
76.5	0\\
76.51	0\\
76.52	0\\
76.53	0\\
76.54	0\\
76.55	0\\
76.56	0\\
76.57	0\\
76.58	0\\
76.59	0\\
76.6	0\\
76.61	0\\
76.62	0\\
76.63	0\\
76.64	0\\
76.65	0\\
76.66	0\\
76.67	0\\
76.68	0\\
76.69	0\\
76.7	0\\
76.71	0\\
76.72	0\\
76.73	0\\
76.74	0\\
76.75	0\\
76.76	0\\
76.77	0\\
76.78	0\\
76.79	0\\
76.8	0\\
76.81	0\\
76.82	0\\
76.83	0\\
76.84	0\\
76.85	0\\
76.86	0\\
76.87	0\\
76.88	0\\
76.89	0\\
76.9	0\\
76.91	0\\
76.92	0\\
76.93	0\\
76.94	0\\
76.95	0\\
76.96	0\\
76.97	0\\
76.98	0\\
76.99	0\\
77	0\\
77.01	0\\
77.02	0\\
77.03	0\\
77.04	0\\
77.05	0\\
77.06	0\\
77.07	0\\
77.08	0\\
77.09	0\\
77.1	0\\
77.11	0\\
77.12	0\\
77.13	0\\
77.14	0\\
77.15	0\\
77.16	0\\
77.17	0\\
77.18	0\\
77.19	0\\
77.2	0\\
77.21	0\\
77.22	0\\
77.23	0\\
77.24	0\\
77.25	0\\
77.26	0\\
77.27	0\\
77.28	0\\
77.29	0\\
77.3	0\\
77.31	0\\
77.32	0\\
77.33	0\\
77.34	0\\
77.35	0\\
77.36	0\\
77.37	0\\
77.38	0\\
77.39	0\\
77.4	0\\
77.41	0\\
77.42	0\\
77.43	0\\
77.44	0\\
77.45	0\\
77.46	0\\
77.47	0\\
77.48	0\\
77.49	0\\
77.5	0\\
77.51	0\\
77.52	0\\
77.53	0\\
77.54	0\\
77.55	0\\
77.56	0\\
77.57	0\\
77.58	0\\
77.59	0\\
77.6	0\\
77.61	0\\
77.62	0\\
77.63	0\\
77.64	0\\
77.65	0\\
77.66	0\\
77.67	0\\
77.68	0\\
77.69	0\\
77.7	0\\
77.71	0\\
77.72	0\\
77.73	0\\
77.74	0\\
77.75	0\\
77.76	0\\
77.77	0\\
77.78	0\\
77.79	0\\
77.8	0\\
77.81	0\\
77.82	0\\
77.83	0\\
77.84	0\\
77.85	0\\
77.86	0\\
77.87	0\\
77.88	0\\
77.89	0\\
77.9	0\\
77.91	0\\
77.92	0\\
77.93	0\\
77.94	0\\
77.95	0\\
77.96	0\\
77.97	0\\
77.98	0\\
77.99	0\\
78	0\\
78.01	0\\
78.02	0\\
78.03	0\\
78.04	0\\
78.05	0\\
78.06	0\\
78.07	0\\
78.08	0\\
78.09	0\\
78.1	0\\
78.11	0\\
78.12	0\\
78.13	0\\
78.14	0\\
78.15	0\\
78.16	0\\
78.17	0\\
78.18	0\\
78.19	0\\
78.2	0\\
78.21	0\\
78.22	0\\
78.23	0\\
78.24	0\\
78.25	0\\
78.26	0\\
78.27	0\\
78.28	0\\
78.29	0\\
78.3	0\\
78.31	0\\
78.32	0\\
78.33	0\\
78.34	0\\
78.35	0\\
78.36	0\\
78.37	0\\
78.38	0\\
78.39	0\\
78.4	0\\
78.41	0\\
78.42	0\\
78.43	0\\
78.44	0\\
78.45	0\\
78.46	0\\
78.47	0\\
78.48	0\\
78.49	0\\
78.5	0\\
78.51	0\\
78.52	0\\
78.53	0\\
78.54	0\\
78.55	0\\
78.56	0\\
78.57	0\\
78.58	0\\
78.59	0\\
78.6	0\\
78.61	0\\
78.62	0\\
78.63	0\\
78.64	0\\
78.65	0\\
78.66	0\\
78.67	0\\
78.68	0\\
78.69	0\\
78.7	0\\
78.71	0\\
78.72	0\\
78.73	0\\
78.74	0\\
78.75	0\\
78.76	0\\
78.77	0\\
78.78	0\\
78.79	0\\
78.8	0\\
78.81	0\\
78.82	0\\
78.83	0\\
78.84	0\\
78.85	0\\
78.86	0\\
78.87	0\\
78.88	0\\
78.89	0\\
78.9	0\\
78.91	0\\
78.92	0\\
78.93	0\\
78.94	0\\
78.95	0\\
78.96	0\\
78.97	0\\
78.98	0\\
78.99	0\\
79	0\\
79.01	0\\
79.02	0\\
79.03	0\\
79.04	0\\
79.05	0\\
79.06	0\\
79.07	0\\
79.08	0\\
79.09	0\\
79.1	0\\
79.11	0\\
79.12	0\\
79.13	0\\
79.14	0\\
79.15	0\\
79.16	0\\
79.17	0\\
79.18	0\\
79.19	0\\
79.2	0\\
79.21	0\\
79.22	0\\
79.23	0\\
79.24	0\\
79.25	0\\
79.26	0\\
79.27	0\\
79.28	0\\
79.29	0\\
79.3	0\\
79.31	0\\
79.32	0\\
79.33	0\\
79.34	0\\
79.35	0\\
79.36	0\\
79.37	0\\
79.38	0\\
79.39	0\\
79.4	0\\
79.41	0\\
79.42	0\\
79.43	0\\
79.44	0\\
79.45	0\\
79.46	0\\
79.47	0\\
79.48	0\\
79.49	0\\
79.5	0\\
79.51	0\\
79.52	0\\
79.53	0\\
79.54	0\\
79.55	0\\
79.56	0\\
79.57	0\\
79.58	0\\
79.59	0\\
79.6	0\\
79.61	0\\
79.62	0\\
79.63	0\\
79.64	0\\
79.65	0\\
79.66	0\\
79.67	0\\
79.68	0\\
79.69	0\\
79.7	0\\
79.71	0\\
79.72	0\\
79.73	0\\
79.74	0\\
79.75	0\\
79.76	0\\
79.77	0\\
79.78	0\\
79.79	0\\
79.8	0\\
79.81	0\\
79.82	0\\
79.83	0\\
79.84	0\\
79.85	0\\
79.86	0\\
79.87	0\\
79.88	0\\
79.89	0\\
79.9	0\\
79.91	0\\
79.92	0\\
79.93	0\\
79.94	0\\
79.95	0\\
79.96	0\\
79.97	0\\
79.98	0\\
79.99	0\\
80	0\\
80.01	0\\
};
\addplot [color=black,solid]
  table[row sep=crcr]{%
80.01	0\\
80.02	0\\
80.03	0\\
80.04	0\\
80.05	0\\
80.06	0\\
80.07	0\\
80.08	0\\
80.09	0\\
80.1	0\\
80.11	0\\
80.12	0\\
80.13	0\\
80.14	0\\
80.15	0\\
80.16	0\\
80.17	0\\
80.18	0\\
80.19	0\\
80.2	0\\
80.21	0\\
80.22	0\\
80.23	0\\
80.24	0\\
80.25	0\\
80.26	0\\
80.27	0\\
80.28	0\\
80.29	0\\
80.3	0\\
80.31	0\\
80.32	0\\
80.33	0\\
80.34	0\\
80.35	0\\
80.36	0\\
80.37	0\\
80.38	0\\
80.39	0\\
80.4	0\\
80.41	0\\
80.42	0\\
80.43	0\\
80.44	0\\
80.45	0\\
80.46	0\\
80.47	0\\
80.48	0\\
80.49	0\\
80.5	0\\
80.51	0\\
80.52	0\\
80.53	0\\
80.54	0\\
80.55	0\\
80.56	0\\
80.57	0\\
80.58	0\\
80.59	0\\
80.6	0\\
80.61	0\\
80.62	0\\
80.63	0\\
80.64	0\\
80.65	0\\
80.66	0\\
80.67	0\\
80.68	0\\
80.69	0\\
80.7	0\\
80.71	0\\
80.72	0\\
80.73	0\\
80.74	0\\
80.75	0\\
80.76	0\\
80.77	0\\
80.78	0\\
80.79	0\\
80.8	0\\
80.81	0\\
80.82	0\\
80.83	0\\
80.84	0\\
80.85	0\\
80.86	0\\
80.87	0\\
80.88	0\\
80.89	0\\
80.9	0\\
80.91	0\\
80.92	0\\
80.93	0\\
80.94	0\\
80.95	0\\
80.96	0\\
80.97	0\\
80.98	0\\
80.99	0\\
81	0\\
81.01	0\\
81.02	0\\
81.03	0\\
81.04	0\\
81.05	0\\
81.06	0\\
81.07	0\\
81.08	0\\
81.09	0\\
81.1	0\\
81.11	0\\
81.12	0\\
81.13	0\\
81.14	0\\
81.15	0\\
81.16	0\\
81.17	0\\
81.18	0\\
81.19	0\\
81.2	0\\
81.21	0\\
81.22	0\\
81.23	0\\
81.24	0\\
81.25	0\\
81.26	0\\
81.27	0\\
81.28	0\\
81.29	0\\
81.3	0\\
81.31	0\\
81.32	0\\
81.33	0\\
81.34	0\\
81.35	0\\
81.36	0\\
81.37	0\\
81.38	0\\
81.39	0\\
81.4	0\\
81.41	0\\
81.42	0\\
81.43	0\\
81.44	0\\
81.45	0\\
81.46	0\\
81.47	0\\
81.48	0\\
81.49	0\\
81.5	0\\
81.51	0\\
81.52	0\\
81.53	0\\
81.54	0\\
81.55	0\\
81.56	0\\
81.57	0\\
81.58	0\\
81.59	0\\
81.6	0\\
81.61	0\\
81.62	0\\
81.63	0\\
81.64	0\\
81.65	0\\
81.66	0\\
81.67	0\\
81.68	0\\
81.69	0\\
81.7	0\\
81.71	0\\
81.72	0\\
81.73	0\\
81.74	0\\
81.75	0\\
81.76	0\\
81.77	0\\
81.78	0\\
81.79	0\\
81.8	0\\
81.81	0\\
81.82	0\\
81.83	0\\
81.84	0\\
81.85	0\\
81.86	0\\
81.87	0\\
81.88	0\\
81.89	0\\
81.9	0\\
81.91	0\\
81.92	0\\
81.93	0\\
81.94	0\\
81.95	0\\
81.96	0\\
81.97	0\\
81.98	0\\
81.99	0\\
82	0\\
82.01	0\\
82.02	0\\
82.03	0\\
82.04	0\\
82.05	0\\
82.06	0\\
82.07	0\\
82.08	0\\
82.09	0\\
82.1	0\\
82.11	0\\
82.12	0\\
82.13	0\\
82.14	0\\
82.15	0\\
82.16	0\\
82.17	0\\
82.18	0\\
82.19	0\\
82.2	0\\
82.21	0\\
82.22	0\\
82.23	0\\
82.24	0\\
82.25	0\\
82.26	0\\
82.27	0\\
82.28	0\\
82.29	0\\
82.3	0\\
82.31	0\\
82.32	0\\
82.33	0\\
82.34	0\\
82.35	0\\
82.36	0\\
82.37	0\\
82.38	0\\
82.39	0\\
82.4	0\\
82.41	0\\
82.42	0\\
82.43	0\\
82.44	0\\
82.45	0\\
82.46	0\\
82.47	0\\
82.48	0\\
82.49	0\\
82.5	0\\
82.51	0\\
82.52	0\\
82.53	0\\
82.54	0\\
82.55	0\\
82.56	0\\
82.57	0\\
82.58	0\\
82.59	0\\
82.6	0\\
82.61	0\\
82.62	0\\
82.63	0\\
82.64	0\\
82.65	0\\
82.66	0\\
82.67	0\\
82.68	0\\
82.69	0\\
82.7	0\\
82.71	0\\
82.72	0\\
82.73	0\\
82.74	0\\
82.75	0\\
82.76	0\\
82.77	0\\
82.78	0\\
82.79	0\\
82.8	0\\
82.81	0\\
82.82	0\\
82.83	0\\
82.84	0\\
82.85	0\\
82.86	0\\
82.87	0\\
82.88	0\\
82.89	0\\
82.9	0\\
82.91	0\\
82.92	0\\
82.93	0\\
82.94	0\\
82.95	0\\
82.96	0\\
82.97	0\\
82.98	0\\
82.99	0\\
83	0\\
83.01	0\\
83.02	0\\
83.03	0\\
83.04	0\\
83.05	0\\
83.06	0\\
83.07	0\\
83.08	0\\
83.09	0\\
83.1	0\\
83.11	0\\
83.12	0\\
83.13	0\\
83.14	0\\
83.15	0\\
83.16	0\\
83.17	0\\
83.18	0\\
83.19	0\\
83.2	0\\
83.21	0\\
83.22	0\\
83.23	0\\
83.24	0\\
83.25	0\\
83.26	0\\
83.27	0\\
83.28	0\\
83.29	0\\
83.3	0\\
83.31	0\\
83.32	0\\
83.33	0\\
83.34	0\\
83.35	0\\
83.36	0\\
83.37	0\\
83.38	0\\
83.39	0\\
83.4	0\\
83.41	0\\
83.42	0\\
83.43	0\\
83.44	0\\
83.45	0\\
83.46	0\\
83.47	0\\
83.48	0\\
83.49	0\\
83.5	0\\
83.51	0\\
83.52	0\\
83.53	0\\
83.54	0\\
83.55	0\\
83.56	0\\
83.57	0\\
83.58	0\\
83.59	0\\
83.6	0\\
83.61	0\\
83.62	0\\
83.63	0\\
83.64	0\\
83.65	0\\
83.66	0\\
83.67	0\\
83.68	0\\
83.69	0\\
83.7	0\\
83.71	0\\
83.72	0\\
83.73	0\\
83.74	0\\
83.75	0\\
83.76	0\\
83.77	0\\
83.78	0\\
83.79	0\\
83.8	0\\
83.81	0\\
83.82	0\\
83.83	0\\
83.84	0\\
83.85	0\\
83.86	0\\
83.87	0\\
83.88	0\\
83.89	0\\
83.9	0\\
83.91	0\\
83.92	0\\
83.93	0\\
83.94	0\\
83.95	0\\
83.96	0\\
83.97	0\\
83.98	0\\
83.99	0\\
84	0\\
84.01	0\\
84.02	0\\
84.03	0\\
84.04	0\\
84.05	0\\
84.06	0\\
84.07	0\\
84.08	0\\
84.09	0\\
84.1	0\\
84.11	0\\
84.12	0\\
84.13	0\\
84.14	0\\
84.15	0\\
84.16	0\\
84.17	0\\
84.18	0\\
84.19	0\\
84.2	0\\
84.21	0\\
84.22	0\\
84.23	0\\
84.24	0\\
84.25	0\\
84.26	0\\
84.27	0\\
84.28	0\\
84.29	0\\
84.3	0\\
84.31	0\\
84.32	0\\
84.33	0\\
84.34	0\\
84.35	0\\
84.36	0\\
84.37	0\\
84.38	0\\
84.39	0\\
84.4	0\\
84.41	0\\
84.42	0\\
84.43	0\\
84.44	0\\
84.45	0\\
84.46	0\\
84.47	0\\
84.48	0\\
84.49	0\\
84.5	0\\
84.51	0\\
84.52	0\\
84.53	0\\
84.54	0\\
84.55	0\\
84.56	0\\
84.57	0\\
84.58	0\\
84.59	0\\
84.6	0\\
84.61	0\\
84.62	0\\
84.63	0\\
84.64	0\\
84.65	0\\
84.66	0\\
84.67	0\\
84.68	0\\
84.69	0\\
84.7	0\\
84.71	0\\
84.72	0\\
84.73	0\\
84.74	0\\
84.75	0\\
84.76	0\\
84.77	0\\
84.78	0\\
84.79	0\\
84.8	0\\
84.81	0\\
84.82	0\\
84.83	0\\
84.84	0\\
84.85	0\\
84.86	0\\
84.87	0\\
84.88	0\\
84.89	0\\
84.9	0\\
84.91	0\\
84.92	0\\
84.93	0\\
84.94	0\\
84.95	0\\
84.96	0\\
84.97	0\\
84.98	0\\
84.99	0\\
85	0\\
85.01	0\\
85.02	0\\
85.03	0\\
85.04	0\\
85.05	0\\
85.06	0\\
85.07	0\\
85.08	0\\
85.09	0\\
85.1	0\\
85.11	0\\
85.12	0\\
85.13	0\\
85.14	0\\
85.15	0\\
85.16	0\\
85.17	0\\
85.18	0\\
85.19	0\\
85.2	0\\
85.21	0\\
85.22	0\\
85.23	0\\
85.24	0\\
85.25	0\\
85.26	0\\
85.27	0\\
85.28	0\\
85.29	0\\
85.3	0\\
85.31	0\\
85.32	0\\
85.33	0\\
85.34	0\\
85.35	0\\
85.36	0\\
85.37	0\\
85.38	0\\
85.39	0\\
85.4	0\\
85.41	0\\
85.42	0\\
85.43	0\\
85.44	0\\
85.45	0\\
85.46	0\\
85.47	0\\
85.48	0\\
85.49	0\\
85.5	0\\
85.51	0\\
85.52	0\\
85.53	0\\
85.54	0\\
85.55	0\\
85.56	0\\
85.57	0\\
85.58	0\\
85.59	0\\
85.6	0\\
85.61	0\\
85.62	0\\
85.63	0\\
85.64	0\\
85.65	0\\
85.66	0\\
85.67	0\\
85.68	0\\
85.69	0\\
85.7	0\\
85.71	0\\
85.72	0\\
85.73	0\\
85.74	0\\
85.75	0\\
85.76	0\\
85.77	0\\
85.78	0\\
85.79	0\\
85.8	0\\
85.81	0\\
85.82	0\\
85.83	0\\
85.84	0\\
85.85	0\\
85.86	0\\
85.87	0\\
85.88	0\\
85.89	0\\
85.9	0\\
85.91	0\\
85.92	0\\
85.93	0\\
85.94	0\\
85.95	0\\
85.96	0\\
85.97	0\\
85.98	0\\
85.99	0\\
86	0\\
86.01	0\\
86.02	0\\
86.03	0\\
86.04	0\\
86.05	0\\
86.06	0\\
86.07	0\\
86.08	0\\
86.09	0\\
86.1	0\\
86.11	0\\
86.12	0\\
86.13	0\\
86.14	0\\
86.15	0\\
86.16	0\\
86.17	0\\
86.18	0\\
86.19	0\\
86.2	0\\
86.21	0\\
86.22	0\\
86.23	0\\
86.24	0\\
86.25	0\\
86.26	0\\
86.27	0\\
86.28	0\\
86.29	0\\
86.3	0\\
86.31	0\\
86.32	0\\
86.33	0\\
86.34	0\\
86.35	0\\
86.36	0\\
86.37	0\\
86.38	0\\
86.39	0\\
86.4	0\\
86.41	0\\
86.42	0\\
86.43	0\\
86.44	0\\
86.45	0\\
86.46	0\\
86.47	0\\
86.48	0\\
86.49	0\\
86.5	0\\
86.51	0\\
86.52	0\\
86.53	0\\
86.54	0\\
86.55	0\\
86.56	0\\
86.57	0\\
86.58	0\\
86.59	0\\
86.6	0\\
86.61	0\\
86.62	0\\
86.63	0\\
86.64	0\\
86.65	0\\
86.66	0\\
86.67	0\\
86.68	0\\
86.69	0\\
86.7	0\\
86.71	0\\
86.72	0\\
86.73	0\\
86.74	0\\
86.75	0\\
86.76	0\\
86.77	0\\
86.78	0\\
86.79	0\\
86.8	0\\
86.81	0\\
86.82	0\\
86.83	0\\
86.84	0\\
86.85	0\\
86.86	0\\
86.87	0\\
86.88	0\\
86.89	0\\
86.9	0\\
86.91	0\\
86.92	0\\
86.93	0\\
86.94	0\\
86.95	0\\
86.96	0\\
86.97	0\\
86.98	0\\
86.99	0\\
87	0\\
87.01	0\\
87.02	0\\
87.03	0\\
87.04	0\\
87.05	0\\
87.06	0\\
87.07	0\\
87.08	0\\
87.09	0\\
87.1	0\\
87.11	0\\
87.12	0\\
87.13	0\\
87.14	0\\
87.15	0\\
87.16	0\\
87.17	0\\
87.18	0\\
87.19	0\\
87.2	0\\
87.21	0\\
87.22	0\\
87.23	0\\
87.24	0\\
87.25	0\\
87.26	0\\
87.27	0\\
87.28	0\\
87.29	0\\
87.3	0\\
87.31	0\\
87.32	0\\
87.33	0\\
87.34	0\\
87.35	0\\
87.36	0\\
87.37	0\\
87.38	0\\
87.39	0\\
87.4	0\\
87.41	0\\
87.42	0\\
87.43	0\\
87.44	0\\
87.45	0\\
87.46	0\\
87.47	0\\
87.48	0\\
87.49	0\\
87.5	0\\
87.51	0\\
87.52	0\\
87.53	0\\
87.54	0\\
87.55	0\\
87.56	0\\
87.57	0\\
87.58	0\\
87.59	0\\
87.6	0\\
87.61	0\\
87.62	0\\
87.63	0\\
87.64	0\\
87.65	0\\
87.66	0\\
87.67	0\\
87.68	0\\
87.69	0\\
87.7	0\\
87.71	0\\
87.72	0\\
87.73	0\\
87.74	0\\
87.75	0\\
87.76	0\\
87.77	0\\
87.78	0\\
87.79	0\\
87.8	0\\
87.81	0\\
87.82	0\\
87.83	0\\
87.84	0\\
87.85	0\\
87.86	0\\
87.87	0\\
87.88	0\\
87.89	0\\
87.9	0\\
87.91	0\\
87.92	0\\
87.93	0\\
87.94	0\\
87.95	0\\
87.96	0\\
87.97	0\\
87.98	0\\
87.99	0\\
88	0\\
88.01	0\\
88.02	0\\
88.03	0\\
88.04	0\\
88.05	0\\
88.06	0\\
88.07	0\\
88.08	0\\
88.09	0\\
88.1	0\\
88.11	0\\
88.12	0\\
88.13	0\\
88.14	0\\
88.15	0\\
88.16	0\\
88.17	0\\
88.18	0\\
88.19	0\\
88.2	0\\
88.21	0\\
88.22	0\\
88.23	0\\
88.24	0\\
88.25	0\\
88.26	0\\
88.27	0\\
88.28	0\\
88.29	0\\
88.3	0\\
88.31	0\\
88.32	0\\
88.33	0\\
88.34	0\\
88.35	0\\
88.36	0\\
88.37	0\\
88.38	0\\
88.39	0\\
88.4	0\\
88.41	0\\
88.42	0\\
88.43	0\\
88.44	0\\
88.45	0\\
88.46	0\\
88.47	0\\
88.48	0\\
88.49	0\\
88.5	0\\
88.51	0\\
88.52	0\\
88.53	0\\
88.54	0\\
88.55	0\\
88.56	0\\
88.57	0\\
88.58	0\\
88.59	0\\
88.6	0\\
88.61	0\\
88.62	0\\
88.63	0\\
88.64	0\\
88.65	0\\
88.66	0\\
88.67	0\\
88.68	0\\
88.69	0\\
88.7	0\\
88.71	0\\
88.72	0\\
88.73	0\\
88.74	0\\
88.75	0\\
88.76	0\\
88.77	0\\
88.78	0\\
88.79	0\\
88.8	0\\
88.81	0\\
88.82	0\\
88.83	0\\
88.84	0\\
88.85	0\\
88.86	0\\
88.87	0\\
88.88	0\\
88.89	0\\
88.9	0\\
88.91	0\\
88.92	0\\
88.93	0\\
88.94	0\\
88.95	0\\
88.96	0\\
88.97	0\\
88.98	0\\
88.99	0\\
89	0\\
89.01	0\\
89.02	0\\
89.03	0\\
89.04	0\\
89.05	0\\
89.06	0\\
89.07	0\\
89.08	0\\
89.09	0\\
89.1	0\\
89.11	0\\
89.12	0\\
89.13	0\\
89.14	0\\
89.15	0\\
89.16	0\\
89.17	0\\
89.18	0\\
89.19	0\\
89.2	0\\
89.21	0\\
89.22	0\\
89.23	0\\
89.24	0\\
89.25	0\\
89.26	0\\
89.27	0\\
89.28	0\\
89.29	0\\
89.3	0\\
89.31	0\\
89.32	0\\
89.33	0\\
89.34	0\\
89.35	0\\
89.36	0\\
89.37	0\\
89.38	0\\
89.39	0\\
89.4	0\\
89.41	0\\
89.42	0\\
89.43	0\\
89.44	0\\
89.45	0\\
89.46	0\\
89.47	0\\
89.48	0\\
89.49	0\\
89.5	0\\
89.51	0\\
89.52	0\\
89.53	0\\
89.54	0\\
89.55	0\\
89.56	0\\
89.57	0\\
89.58	0\\
89.59	0\\
89.6	0\\
89.61	0\\
89.62	0\\
89.63	0\\
89.64	0\\
89.65	0\\
89.66	0\\
89.67	0\\
89.68	0\\
89.69	0\\
89.7	0\\
89.71	0\\
89.72	0\\
89.73	0\\
89.74	0\\
89.75	0\\
89.76	0\\
89.77	0\\
89.78	0\\
89.79	0\\
89.8	0\\
89.81	0\\
89.82	0\\
89.83	0\\
89.84	0\\
89.85	0\\
89.86	0\\
89.87	0\\
89.88	0\\
89.89	0\\
89.9	0\\
89.91	0\\
89.92	0\\
89.93	0\\
89.94	0\\
89.95	0\\
89.96	0\\
89.97	0\\
89.98	0\\
89.99	0\\
90	0\\
90.01	0\\
90.02	0\\
90.03	0\\
90.04	0\\
90.05	0\\
90.06	0\\
90.07	0\\
90.08	0\\
90.09	0\\
90.1	0\\
90.11	0\\
90.12	0\\
90.13	0\\
90.14	0\\
90.15	0\\
90.16	0\\
90.17	0\\
90.18	0\\
90.19	0\\
90.2	0\\
90.21	0\\
90.22	0\\
90.23	0\\
90.24	0\\
90.25	0\\
90.26	0\\
90.27	0\\
90.28	0\\
90.29	0\\
90.3	0\\
90.31	0\\
90.32	0\\
90.33	0\\
90.34	0\\
90.35	0\\
90.36	0\\
90.37	0\\
90.38	0\\
90.39	0\\
90.4	0\\
90.41	0\\
90.42	0\\
90.43	0\\
90.44	0\\
90.45	0\\
90.46	0\\
90.47	0\\
90.48	0\\
90.49	0\\
90.5	0\\
90.51	0\\
90.52	0\\
90.53	0\\
90.54	0\\
90.55	0\\
90.56	0\\
90.57	0\\
90.58	0\\
90.59	0\\
90.6	0\\
90.61	0\\
90.62	0\\
90.63	0\\
90.64	0\\
90.65	0\\
90.66	0\\
90.67	0\\
90.68	0\\
90.69	0\\
90.7	0\\
90.71	0\\
90.72	0\\
90.73	0\\
90.74	0\\
90.75	0\\
90.76	0\\
90.77	0\\
90.78	0\\
90.79	0\\
90.8	0\\
90.81	0\\
90.82	0\\
90.83	0\\
90.84	0\\
90.85	0\\
90.86	0\\
90.87	0\\
90.88	0\\
90.89	0\\
90.9	0\\
90.91	0\\
90.92	0\\
90.93	0\\
90.94	0\\
90.95	0\\
90.96	0\\
90.97	0\\
90.98	0\\
90.99	0\\
91	0\\
91.01	0\\
91.02	0\\
91.03	0\\
91.04	0\\
91.05	0\\
91.06	0\\
91.07	0\\
91.08	0\\
91.09	0\\
91.1	0\\
91.11	0\\
91.12	0\\
91.13	0\\
91.14	0\\
91.15	0\\
91.16	0\\
91.17	0\\
91.18	0\\
91.19	0\\
91.2	0\\
91.21	0\\
91.22	0\\
91.23	0\\
91.24	0\\
91.25	0\\
91.26	0\\
91.27	0\\
91.28	0\\
91.29	0\\
91.3	0\\
91.31	0\\
91.32	0\\
91.33	0\\
91.34	0\\
91.35	0\\
91.36	0\\
91.37	0\\
91.38	0\\
91.39	0\\
91.4	0\\
91.41	0\\
91.42	0\\
91.43	0\\
91.44	0\\
91.45	0\\
91.46	0\\
91.47	0\\
91.48	0\\
91.49	0\\
91.5	0\\
91.51	0\\
91.52	0\\
91.53	0\\
91.54	0\\
91.55	0\\
91.56	0\\
91.57	0\\
91.58	0\\
91.59	0\\
91.6	0\\
91.61	0\\
91.62	0\\
91.63	0\\
91.64	0\\
91.65	0\\
91.66	0\\
91.67	0\\
91.68	0\\
91.69	0\\
91.7	0\\
91.71	0\\
91.72	0\\
91.73	0\\
91.74	0\\
91.75	0\\
91.76	0\\
91.77	0\\
91.78	0\\
91.79	0\\
91.8	0\\
91.81	0\\
91.82	0\\
91.83	0\\
91.84	0\\
91.85	0\\
91.86	0\\
91.87	0\\
91.88	0\\
91.89	0\\
91.9	0\\
91.91	0\\
91.92	0\\
91.93	0\\
91.94	0\\
91.95	0\\
91.96	0\\
91.97	0\\
91.98	0\\
91.99	0\\
92	0\\
92.01	0\\
92.02	0\\
92.03	0\\
92.04	0\\
92.05	0\\
92.06	0\\
92.07	0\\
92.08	0\\
92.09	0\\
92.1	0\\
92.11	0\\
92.12	0\\
92.13	0\\
92.14	0\\
92.15	0\\
92.16	0\\
92.17	0\\
92.18	0\\
92.19	0\\
92.2	0\\
92.21	0\\
92.22	0\\
92.23	0\\
92.24	0\\
92.25	0\\
92.26	0\\
92.27	0\\
92.28	0\\
92.29	0\\
92.3	0\\
92.31	0\\
92.32	0\\
92.33	0\\
92.34	0\\
92.35	0\\
92.36	0\\
92.37	0\\
92.38	0\\
92.39	0\\
92.4	0\\
92.41	0\\
92.42	0\\
92.43	0\\
92.44	0\\
92.45	0\\
92.46	0\\
92.47	0\\
92.48	0\\
92.49	0\\
92.5	0\\
92.51	0\\
92.52	0\\
92.53	0\\
92.54	0\\
92.55	0\\
92.56	0\\
92.57	0\\
92.58	0\\
92.59	0\\
92.6	0\\
92.61	0\\
92.62	0\\
92.63	0\\
92.64	0\\
92.65	0\\
92.66	0\\
92.67	0\\
92.68	0\\
92.69	0\\
92.7	0\\
92.71	0\\
92.72	0\\
92.73	0\\
92.74	0\\
92.75	0\\
92.76	0\\
92.77	0\\
92.78	0\\
92.79	0\\
92.8	0\\
92.81	0\\
92.82	0\\
92.83	0\\
92.84	0\\
92.85	0\\
92.86	0\\
92.87	0\\
92.88	0\\
92.89	0\\
92.9	0\\
92.91	0\\
92.92	0\\
92.93	0\\
92.94	0\\
92.95	0\\
92.96	0\\
92.97	0\\
92.98	0\\
92.99	0\\
93	0\\
93.01	0\\
93.02	0\\
93.03	0\\
93.04	0\\
93.05	0\\
93.06	0\\
93.07	0\\
93.08	0\\
93.09	0\\
93.1	0\\
93.11	0\\
93.12	0\\
93.13	0\\
93.14	0\\
93.15	0\\
93.16	0\\
93.17	0\\
93.18	0\\
93.19	0\\
93.2	0\\
93.21	0\\
93.22	0\\
93.23	0\\
93.24	0\\
93.25	0\\
93.26	0\\
93.27	0\\
93.28	0\\
93.29	0\\
93.3	0\\
93.31	0\\
93.32	0\\
93.33	0\\
93.34	0\\
93.35	0\\
93.36	0\\
93.37	0\\
93.38	0\\
93.39	0\\
93.4	0\\
93.41	0\\
93.42	0\\
93.43	0\\
93.44	0\\
93.45	0\\
93.46	0\\
93.47	0\\
93.48	0\\
93.49	0\\
93.5	0\\
93.51	0\\
93.52	0\\
93.53	0\\
93.54	0\\
93.55	0\\
93.56	0\\
93.57	0\\
93.58	0\\
93.59	0\\
93.6	0\\
93.61	0\\
93.62	0\\
93.63	0\\
93.64	0\\
93.65	0\\
93.66	0\\
93.67	0\\
93.68	0\\
93.69	0\\
93.7	0\\
93.71	0\\
93.72	0\\
93.73	0\\
93.74	0\\
93.75	0\\
93.76	0\\
93.77	0\\
93.78	0\\
93.79	0\\
93.8	0\\
93.81	0\\
93.82	0\\
93.83	0\\
93.84	0\\
93.85	0\\
93.86	0\\
93.87	0\\
93.88	0\\
93.89	0\\
93.9	0\\
93.91	0\\
93.92	0\\
93.93	0\\
93.94	0\\
93.95	0\\
93.96	0\\
93.97	0\\
93.98	0\\
93.99	0\\
94	0\\
94.01	0\\
94.02	0\\
94.03	0\\
94.04	0\\
94.05	0\\
94.06	0\\
94.07	0\\
94.08	0\\
94.09	0\\
94.1	0\\
94.11	0\\
94.12	0\\
94.13	0\\
94.14	0\\
94.15	0\\
94.16	0\\
94.17	0\\
94.18	0\\
94.19	0\\
94.2	0\\
94.21	0\\
94.22	0\\
94.23	0\\
94.24	0\\
94.25	0\\
94.26	0\\
94.27	0\\
94.28	0\\
94.29	0\\
94.3	0\\
94.31	0\\
94.32	0\\
94.33	0\\
94.34	0\\
94.35	0\\
94.36	0\\
94.37	0\\
94.38	0\\
94.39	0\\
94.4	0\\
94.41	0\\
94.42	0\\
94.43	0\\
94.44	0\\
94.45	0\\
94.46	0\\
94.47	0\\
94.48	0\\
94.49	0\\
94.5	0\\
94.51	0\\
94.52	0\\
94.53	0\\
94.54	0\\
94.55	0\\
94.56	0\\
94.57	0\\
94.58	0\\
94.59	0\\
94.6	0\\
94.61	0\\
94.62	0\\
94.63	0\\
94.64	0\\
94.65	0\\
94.66	0\\
94.67	0\\
94.68	0\\
94.69	0\\
94.7	0\\
94.71	0\\
94.72	0\\
94.73	0\\
94.74	0\\
94.75	0\\
94.76	0\\
94.77	0\\
94.78	0\\
94.79	0\\
94.8	0\\
94.81	0\\
94.82	0\\
94.83	0\\
94.84	0\\
94.85	0\\
94.86	0\\
94.87	0\\
94.88	0\\
94.89	0\\
94.9	0\\
94.91	0\\
94.92	0\\
94.93	0\\
94.94	0\\
94.95	0\\
94.96	0\\
94.97	0\\
94.98	0\\
94.99	0\\
95	0\\
95.01	0\\
95.02	0\\
95.03	0\\
95.04	0\\
95.05	0\\
95.06	0\\
95.07	0\\
95.08	0\\
95.09	0\\
95.1	0\\
95.11	0\\
95.12	0\\
95.13	0\\
95.14	0\\
95.15	0\\
95.16	0\\
95.17	0\\
95.18	0\\
95.19	0\\
95.2	0\\
95.21	0\\
95.22	0\\
95.23	0\\
95.24	0\\
95.25	0\\
95.26	0\\
95.27	0\\
95.28	0\\
95.29	0\\
95.3	0\\
95.31	0\\
95.32	0\\
95.33	0\\
95.34	0\\
95.35	0\\
95.36	0\\
95.37	0\\
95.38	0\\
95.39	0\\
95.4	0\\
95.41	0\\
95.42	0\\
95.43	0\\
95.44	0\\
95.45	0\\
95.46	0\\
95.47	0\\
95.48	0\\
95.49	0\\
95.5	0\\
95.51	0\\
95.52	0\\
95.53	0\\
95.54	0\\
95.55	0\\
95.56	0\\
95.57	0\\
95.58	0\\
95.59	0\\
95.6	0\\
95.61	0\\
95.62	0\\
95.63	0\\
95.64	0\\
95.65	0\\
95.66	0\\
95.67	0\\
95.68	0\\
95.69	0\\
95.7	0\\
95.71	0\\
95.72	0\\
95.73	0\\
95.74	0\\
95.75	0\\
95.76	0\\
95.77	0\\
95.78	0\\
95.79	0\\
95.8	0\\
95.81	0\\
95.82	0\\
95.83	0\\
95.84	0\\
95.85	0\\
95.86	0\\
95.87	0\\
95.88	0\\
95.89	0\\
95.9	0\\
95.91	0\\
95.92	0\\
95.93	0\\
95.94	0\\
95.95	0\\
95.96	0\\
95.97	0\\
95.98	0\\
95.99	0\\
96	0\\
96.01	0\\
96.02	0\\
96.03	0\\
96.04	0\\
96.05	0\\
96.06	0\\
96.07	0\\
96.08	0\\
96.09	0\\
96.1	0\\
96.11	0\\
96.12	0\\
96.13	0\\
96.14	0\\
96.15	0\\
96.16	0\\
96.17	0\\
96.18	0\\
96.19	0\\
96.2	0\\
96.21	0\\
96.22	0\\
96.23	0\\
96.24	0\\
96.25	0\\
96.26	0\\
96.27	0\\
96.28	0\\
96.29	0\\
96.3	0\\
96.31	0\\
96.32	0\\
96.33	0\\
96.34	0\\
96.35	0\\
96.36	0\\
96.37	0\\
96.38	0\\
96.39	0\\
96.4	0\\
96.41	0\\
96.42	0\\
96.43	0\\
96.44	0\\
96.45	0\\
96.46	0\\
96.47	0\\
96.48	0\\
96.49	0\\
96.5	0\\
96.51	0\\
96.52	0\\
96.53	0\\
96.54	0\\
96.55	0\\
96.56	0\\
96.57	0\\
96.58	0\\
96.59	0\\
96.6	0\\
96.61	0\\
96.62	0\\
96.63	0\\
96.64	0\\
96.65	0\\
96.66	0\\
96.67	0\\
96.68	0\\
96.69	0\\
96.7	0\\
96.71	0\\
96.72	0\\
96.73	0\\
96.74	0\\
96.75	0\\
96.76	0\\
96.77	0\\
96.78	0\\
96.79	0\\
96.8	0\\
96.81	0\\
96.82	0\\
96.83	0\\
96.84	0\\
96.85	0\\
96.86	0\\
96.87	0\\
96.88	0\\
96.89	0\\
96.9	0\\
96.91	0\\
96.92	0\\
96.93	0\\
96.94	0\\
96.95	0\\
96.96	0\\
96.97	0\\
96.98	0\\
96.99	0\\
97	0\\
97.01	0\\
97.02	0\\
97.03	0\\
97.04	0\\
97.05	0\\
97.06	0\\
97.07	0\\
97.08	0\\
97.09	0\\
97.1	0\\
97.11	0\\
97.12	0\\
97.13	0\\
97.14	0\\
97.15	0\\
97.16	0\\
97.17	0\\
97.18	0\\
97.19	0\\
97.2	0\\
97.21	0\\
97.22	0\\
97.23	0\\
97.24	0\\
97.25	0\\
97.26	0\\
97.27	0\\
97.28	0\\
97.29	0\\
97.3	0\\
97.31	0\\
97.32	0\\
97.33	0\\
97.34	0\\
97.35	0\\
97.36	0\\
97.37	0\\
97.38	0\\
97.39	0\\
97.4	0\\
97.41	0\\
97.42	0\\
97.43	0\\
97.44	0\\
97.45	0\\
97.46	0\\
97.47	0\\
97.48	0\\
97.49	0\\
97.5	0\\
97.51	0\\
97.52	0\\
97.53	0\\
97.54	0\\
97.55	0\\
97.56	0\\
97.57	0\\
97.58	0\\
97.59	0\\
97.6	0\\
97.61	0\\
97.62	0\\
97.63	0\\
97.64	0\\
97.65	0\\
97.66	0\\
97.67	0\\
97.68	0\\
97.69	0\\
97.7	0\\
97.71	0\\
97.72	0\\
97.73	0\\
97.74	0\\
97.75	0\\
97.76	0\\
97.77	0\\
97.78	0\\
97.79	0\\
97.8	0\\
97.81	0\\
97.82	0\\
97.83	0\\
97.84	0\\
97.85	0\\
97.86	0\\
97.87	0\\
97.88	0\\
97.89	0\\
97.9	0\\
97.91	0\\
97.92	0\\
97.93	0\\
97.94	0\\
97.95	0\\
97.96	0\\
97.97	0\\
97.98	0\\
97.99	0\\
98	0\\
98.01	0\\
98.02	0\\
98.03	0\\
98.04	0\\
98.05	0\\
98.06	0\\
98.07	0\\
98.08	0\\
98.09	0\\
98.1	0\\
98.11	0\\
98.12	0\\
98.13	0\\
98.14	0\\
98.15	0\\
98.16	0\\
98.17	0\\
98.18	0\\
98.19	0\\
98.2	0\\
98.21	0\\
98.22	0\\
98.23	0\\
98.24	0\\
98.25	0\\
98.26	0\\
98.27	0\\
98.28	0\\
98.29	0\\
98.3	0\\
98.31	0\\
98.32	0\\
98.33	0\\
98.34	0\\
98.35	0\\
98.36	0\\
98.37	0\\
98.38	0\\
98.39	0\\
98.4	0\\
98.41	0\\
98.42	0\\
98.43	0\\
98.44	0\\
98.45	0\\
98.46	0\\
98.47	0\\
98.48	0\\
98.49	0\\
98.5	0\\
98.51	0\\
98.52	0\\
98.53	0\\
98.54	0\\
98.55	0\\
98.56	0\\
98.57	0\\
98.58	0\\
98.59	0\\
98.6	0\\
98.61	0\\
98.62	0\\
98.63	0\\
98.64	0\\
98.65	0\\
98.66	0\\
98.67	0\\
98.68	0\\
98.69	3.45443584492192e-05\\
98.7	8.03198772388932e-05\\
98.71	0.000126406080126856\\
98.72	0.000172805774561973\\
98.73	0.000219521789666057\\
98.74	0.000266554423900341\\
98.75	0.000313904475297423\\
98.76	0.000361574794340553\\
98.77	0.00040956826207514\\
98.78	0.000457887790512465\\
98.79	0.000506528840464412\\
98.8	0.000555492119411654\\
98.81	0.000604780473917106\\
98.82	0.000654396780925144\\
98.83	0.000704343948166776\\
98.84	0.00075462491457162\\
98.85	0.000805242650687451\\
98.86	0.000856200159106401\\
98.87	0.000907500474897953\\
98.88	0.000959146666048846\\
98.89	0.00101114183391\\
98.9	0.00106348911365062\\
98.91	0.00111619167471953\\
98.92	0.00116925272131401\\
98.93	0.00122267549285609\\
98.94	0.00127646326447658\\
98.95	0.00133061934750694\\
98.96	0.00138514708997907\\
98.97	0.00144004987713325\\
98.98	0.00149533113193433\\
98.99	0.00155099431559634\\
99	0.00160704292811567\\
99.01	0.00166348050881296\\
99.02	0.00172031063688394\\
99.03	0.00177753693195926\\
99.04	0.0018351630546736\\
99.05	0.00189319270724418\\
99.06	0.0019516296340588\\
99.07	0.00201047762227375\\
99.08	0.00206974050251821\\
99.09	0.00212942214959332\\
99.1	0.00218952648310286\\
99.11	0.00225005746809431\\
99.12	0.00231101911644198\\
99.13	0.00237241548697314\\
99.14	0.00243425068596524\\
99.15	0.00249652886783157\\
99.16	0.00255925423581837\\
99.17	0.0026224310427139\\
99.18	0.00268606359156946\\
99.19	0.00275015623643286\\
99.2	0.00281471338309431\\
99.21	0.00287973948984525\\
99.22	0.00294523906825022\\
99.23	0.00301121668393211\\
99.24	0.00307767695737102\\
99.25	0.00314462456471708\\
99.26	0.0032120642386175\\
99.27	0.00328000076905816\\
99.28	0.00334843900422\\
99.29	0.00341738385135061\\
99.3	0.00348684027765132\\
99.31	0.00355681331118016\\
99.32	0.00362730804177092\\
99.33	0.00369832962196888\\
99.34	0.00376988326798339\\
99.35	0.00384197426065781\\
99.36	0.00391460794645714\\
99.37	0.00398778973847378\\
99.38	0.00406152511745181\\
99.39	0.00413581963283032\\
99.4	0.00421067890380606\\
99.41	0.00428610862041612\\
99.42	0.00436211454464085\\
99.43	0.00443870251152773\\
99.44	0.00451587843033661\\
99.45	0.00459364828570688\\
99.46	0.004672018138847\\
99.47	0.00475099412874719\\
99.48	0.00483058247341566\\
99.49	0.00491078947113902\\
99.5	0.0049916215017676\\
99.51	0.00507308502802622\\
99.52	0.00515518659685116\\
99.53	0.0052379328407539\\
99.54	0.00532133047921257\\
99.55	0.0054053863200916\\
99.56	0.00549010726109058\\
99.57	0.00557550029122293\\
99.58	0.00566157249232538\\
99.59	0.00574833104059902\\
99.6	0.00583578320818287\\
99.61	0.00592393636476083\\
99.62	0.00601279797920297\\
99.63	0.00610237562124226\\
99.64	0.00619267696318778\\
99.65	0.00628370978167542\\
99.66	0.0063754819594573\\
99.67	0.006468001487231\\
99.68	0.00656127646550995\\
99.69	0.00665531510653603\\
99.7	0.00675012573623595\\
99.71	0.00684571679629706\\
99.72	0.00694209684622123\\
99.73	0.00703927456542089\\
99.74	0.00713725875536444\\
99.75	0.00723605834177294\\
99.76	0.00733568237686946\\
99.77	0.00743614004168323\\
99.78	0.00753744064841003\\
99.79	0.00763959364283112\\
99.8	0.00774260860679244\\
99.81	0.00784649526074633\\
99.82	0.00795126346635793\\
99.83	0.00805692322917855\\
99.84	0.00816348470138832\\
99.85	0.00827095818461085\\
99.86	0.00837935413280225\\
99.87	0.00848868315521739\\
99.88	0.00859895601945634\\
99.89	0.00871018365459377\\
99.9	0.0088223771543947\\
99.91	0.00893554778061969\\
99.92	0.00904970696642306\\
99.93	0.00916486631984768\\
99.94	0.00928103762742014\\
99.95	0.00939823285785025\\
99.96	0.00951646416583911\\
99.97	0.00963574389600006\\
99.98	0.00975608458689704\\
99.99	0.00987749897520541\\
100	0.01\\
};
\addlegendentry{$q=0$};

\addplot [color=blue,solid,forget plot]
  table[row sep=crcr]{%
0.01	0\\
0.02	0\\
0.03	0\\
0.04	0\\
0.05	0\\
0.06	0\\
0.07	0\\
0.08	0\\
0.09	0\\
0.1	0\\
0.11	0\\
0.12	0\\
0.13	0\\
0.14	0\\
0.15	0\\
0.16	0\\
0.17	0\\
0.18	0\\
0.19	0\\
0.2	0\\
0.21	0\\
0.22	0\\
0.23	0\\
0.24	0\\
0.25	0\\
0.26	0\\
0.27	0\\
0.28	0\\
0.29	0\\
0.3	0\\
0.31	0\\
0.32	0\\
0.33	0\\
0.34	0\\
0.35	0\\
0.36	0\\
0.37	0\\
0.38	0\\
0.39	0\\
0.4	0\\
0.41	0\\
0.42	0\\
0.43	0\\
0.44	0\\
0.45	0\\
0.46	0\\
0.47	0\\
0.48	0\\
0.49	0\\
0.5	0\\
0.51	0\\
0.52	0\\
0.53	0\\
0.54	0\\
0.55	0\\
0.56	0\\
0.57	0\\
0.58	0\\
0.59	0\\
0.6	0\\
0.61	0\\
0.62	0\\
0.63	0\\
0.64	0\\
0.65	0\\
0.66	0\\
0.67	0\\
0.68	0\\
0.69	0\\
0.7	0\\
0.71	0\\
0.72	0\\
0.73	0\\
0.74	0\\
0.75	0\\
0.76	0\\
0.77	0\\
0.78	0\\
0.79	0\\
0.8	0\\
0.81	0\\
0.82	0\\
0.83	0\\
0.84	0\\
0.85	0\\
0.86	0\\
0.87	0\\
0.88	0\\
0.89	0\\
0.9	0\\
0.91	0\\
0.92	0\\
0.93	0\\
0.94	0\\
0.95	0\\
0.96	0\\
0.97	0\\
0.98	0\\
0.99	0\\
1	0\\
1.01	0\\
1.02	0\\
1.03	0\\
1.04	0\\
1.05	0\\
1.06	0\\
1.07	0\\
1.08	0\\
1.09	0\\
1.1	0\\
1.11	0\\
1.12	0\\
1.13	0\\
1.14	0\\
1.15	0\\
1.16	0\\
1.17	0\\
1.18	0\\
1.19	0\\
1.2	0\\
1.21	0\\
1.22	0\\
1.23	0\\
1.24	0\\
1.25	0\\
1.26	0\\
1.27	0\\
1.28	0\\
1.29	0\\
1.3	0\\
1.31	0\\
1.32	0\\
1.33	0\\
1.34	0\\
1.35	0\\
1.36	0\\
1.37	0\\
1.38	0\\
1.39	0\\
1.4	0\\
1.41	0\\
1.42	0\\
1.43	0\\
1.44	0\\
1.45	0\\
1.46	0\\
1.47	0\\
1.48	0\\
1.49	0\\
1.5	0\\
1.51	0\\
1.52	0\\
1.53	0\\
1.54	0\\
1.55	0\\
1.56	0\\
1.57	0\\
1.58	0\\
1.59	0\\
1.6	0\\
1.61	0\\
1.62	0\\
1.63	0\\
1.64	0\\
1.65	0\\
1.66	0\\
1.67	0\\
1.68	0\\
1.69	0\\
1.7	0\\
1.71	0\\
1.72	0\\
1.73	0\\
1.74	0\\
1.75	0\\
1.76	0\\
1.77	0\\
1.78	0\\
1.79	0\\
1.8	0\\
1.81	0\\
1.82	0\\
1.83	0\\
1.84	0\\
1.85	0\\
1.86	0\\
1.87	0\\
1.88	0\\
1.89	0\\
1.9	0\\
1.91	0\\
1.92	0\\
1.93	0\\
1.94	0\\
1.95	0\\
1.96	0\\
1.97	0\\
1.98	0\\
1.99	0\\
2	0\\
2.01	0\\
2.02	0\\
2.03	0\\
2.04	0\\
2.05	0\\
2.06	0\\
2.07	0\\
2.08	0\\
2.09	0\\
2.1	0\\
2.11	0\\
2.12	0\\
2.13	0\\
2.14	0\\
2.15	0\\
2.16	0\\
2.17	0\\
2.18	0\\
2.19	0\\
2.2	0\\
2.21	0\\
2.22	0\\
2.23	0\\
2.24	0\\
2.25	0\\
2.26	0\\
2.27	0\\
2.28	0\\
2.29	0\\
2.3	0\\
2.31	0\\
2.32	0\\
2.33	0\\
2.34	0\\
2.35	0\\
2.36	0\\
2.37	0\\
2.38	0\\
2.39	0\\
2.4	0\\
2.41	0\\
2.42	0\\
2.43	0\\
2.44	0\\
2.45	0\\
2.46	0\\
2.47	0\\
2.48	0\\
2.49	0\\
2.5	0\\
2.51	0\\
2.52	0\\
2.53	0\\
2.54	0\\
2.55	0\\
2.56	0\\
2.57	0\\
2.58	0\\
2.59	0\\
2.6	0\\
2.61	0\\
2.62	0\\
2.63	0\\
2.64	0\\
2.65	0\\
2.66	0\\
2.67	0\\
2.68	0\\
2.69	0\\
2.7	0\\
2.71	0\\
2.72	0\\
2.73	0\\
2.74	0\\
2.75	0\\
2.76	0\\
2.77	0\\
2.78	0\\
2.79	0\\
2.8	0\\
2.81	0\\
2.82	0\\
2.83	0\\
2.84	0\\
2.85	0\\
2.86	0\\
2.87	0\\
2.88	0\\
2.89	0\\
2.9	0\\
2.91	0\\
2.92	0\\
2.93	0\\
2.94	0\\
2.95	0\\
2.96	0\\
2.97	0\\
2.98	0\\
2.99	0\\
3	0\\
3.01	0\\
3.02	0\\
3.03	0\\
3.04	0\\
3.05	0\\
3.06	0\\
3.07	0\\
3.08	0\\
3.09	0\\
3.1	0\\
3.11	0\\
3.12	0\\
3.13	0\\
3.14	0\\
3.15	0\\
3.16	0\\
3.17	0\\
3.18	0\\
3.19	0\\
3.2	0\\
3.21	0\\
3.22	0\\
3.23	0\\
3.24	0\\
3.25	0\\
3.26	0\\
3.27	0\\
3.28	0\\
3.29	0\\
3.3	0\\
3.31	0\\
3.32	0\\
3.33	0\\
3.34	0\\
3.35	0\\
3.36	0\\
3.37	0\\
3.38	0\\
3.39	0\\
3.4	0\\
3.41	0\\
3.42	0\\
3.43	0\\
3.44	0\\
3.45	0\\
3.46	0\\
3.47	0\\
3.48	0\\
3.49	0\\
3.5	0\\
3.51	0\\
3.52	0\\
3.53	0\\
3.54	0\\
3.55	0\\
3.56	0\\
3.57	0\\
3.58	0\\
3.59	0\\
3.6	0\\
3.61	0\\
3.62	0\\
3.63	0\\
3.64	0\\
3.65	0\\
3.66	0\\
3.67	0\\
3.68	0\\
3.69	0\\
3.7	0\\
3.71	0\\
3.72	0\\
3.73	0\\
3.74	0\\
3.75	0\\
3.76	0\\
3.77	0\\
3.78	0\\
3.79	0\\
3.8	0\\
3.81	0\\
3.82	0\\
3.83	0\\
3.84	0\\
3.85	0\\
3.86	0\\
3.87	0\\
3.88	0\\
3.89	0\\
3.9	0\\
3.91	0\\
3.92	0\\
3.93	0\\
3.94	0\\
3.95	0\\
3.96	0\\
3.97	0\\
3.98	0\\
3.99	0\\
4	0\\
4.01	0\\
4.02	0\\
4.03	0\\
4.04	0\\
4.05	0\\
4.06	0\\
4.07	0\\
4.08	0\\
4.09	0\\
4.1	0\\
4.11	0\\
4.12	0\\
4.13	0\\
4.14	0\\
4.15	0\\
4.16	0\\
4.17	0\\
4.18	0\\
4.19	0\\
4.2	0\\
4.21	0\\
4.22	0\\
4.23	0\\
4.24	0\\
4.25	0\\
4.26	0\\
4.27	0\\
4.28	0\\
4.29	0\\
4.3	0\\
4.31	0\\
4.32	0\\
4.33	0\\
4.34	0\\
4.35	0\\
4.36	0\\
4.37	0\\
4.38	0\\
4.39	0\\
4.4	0\\
4.41	0\\
4.42	0\\
4.43	0\\
4.44	0\\
4.45	0\\
4.46	0\\
4.47	0\\
4.48	0\\
4.49	0\\
4.5	0\\
4.51	0\\
4.52	0\\
4.53	0\\
4.54	0\\
4.55	0\\
4.56	0\\
4.57	0\\
4.58	0\\
4.59	0\\
4.6	0\\
4.61	0\\
4.62	0\\
4.63	0\\
4.64	0\\
4.65	0\\
4.66	0\\
4.67	0\\
4.68	0\\
4.69	0\\
4.7	0\\
4.71	0\\
4.72	0\\
4.73	0\\
4.74	0\\
4.75	0\\
4.76	0\\
4.77	0\\
4.78	0\\
4.79	0\\
4.8	0\\
4.81	0\\
4.82	0\\
4.83	0\\
4.84	0\\
4.85	0\\
4.86	0\\
4.87	0\\
4.88	0\\
4.89	0\\
4.9	0\\
4.91	0\\
4.92	0\\
4.93	0\\
4.94	0\\
4.95	0\\
4.96	0\\
4.97	0\\
4.98	0\\
4.99	0\\
5	0\\
5.01	0\\
5.02	0\\
5.03	0\\
5.04	0\\
5.05	0\\
5.06	0\\
5.07	0\\
5.08	0\\
5.09	0\\
5.1	0\\
5.11	0\\
5.12	0\\
5.13	0\\
5.14	0\\
5.15	0\\
5.16	0\\
5.17	0\\
5.18	0\\
5.19	0\\
5.2	0\\
5.21	0\\
5.22	0\\
5.23	0\\
5.24	0\\
5.25	0\\
5.26	0\\
5.27	0\\
5.28	0\\
5.29	0\\
5.3	0\\
5.31	0\\
5.32	0\\
5.33	0\\
5.34	0\\
5.35	0\\
5.36	0\\
5.37	0\\
5.38	0\\
5.39	0\\
5.4	0\\
5.41	0\\
5.42	0\\
5.43	0\\
5.44	0\\
5.45	0\\
5.46	0\\
5.47	0\\
5.48	0\\
5.49	0\\
5.5	0\\
5.51	0\\
5.52	0\\
5.53	0\\
5.54	0\\
5.55	0\\
5.56	0\\
5.57	0\\
5.58	0\\
5.59	0\\
5.6	0\\
5.61	0\\
5.62	0\\
5.63	0\\
5.64	0\\
5.65	0\\
5.66	0\\
5.67	0\\
5.68	0\\
5.69	0\\
5.7	0\\
5.71	0\\
5.72	0\\
5.73	0\\
5.74	0\\
5.75	0\\
5.76	0\\
5.77	0\\
5.78	0\\
5.79	0\\
5.8	0\\
5.81	0\\
5.82	0\\
5.83	0\\
5.84	0\\
5.85	0\\
5.86	0\\
5.87	0\\
5.88	0\\
5.89	0\\
5.9	0\\
5.91	0\\
5.92	0\\
5.93	0\\
5.94	0\\
5.95	0\\
5.96	0\\
5.97	0\\
5.98	0\\
5.99	0\\
6	0\\
6.01	0\\
6.02	0\\
6.03	0\\
6.04	0\\
6.05	0\\
6.06	0\\
6.07	0\\
6.08	0\\
6.09	0\\
6.1	0\\
6.11	0\\
6.12	0\\
6.13	0\\
6.14	0\\
6.15	0\\
6.16	0\\
6.17	0\\
6.18	0\\
6.19	0\\
6.2	0\\
6.21	0\\
6.22	0\\
6.23	0\\
6.24	0\\
6.25	0\\
6.26	0\\
6.27	0\\
6.28	0\\
6.29	0\\
6.3	0\\
6.31	0\\
6.32	0\\
6.33	0\\
6.34	0\\
6.35	0\\
6.36	0\\
6.37	0\\
6.38	0\\
6.39	0\\
6.4	0\\
6.41	0\\
6.42	0\\
6.43	0\\
6.44	0\\
6.45	0\\
6.46	0\\
6.47	0\\
6.48	0\\
6.49	0\\
6.5	0\\
6.51	0\\
6.52	0\\
6.53	0\\
6.54	0\\
6.55	0\\
6.56	0\\
6.57	0\\
6.58	0\\
6.59	0\\
6.6	0\\
6.61	0\\
6.62	0\\
6.63	0\\
6.64	0\\
6.65	0\\
6.66	0\\
6.67	0\\
6.68	0\\
6.69	0\\
6.7	0\\
6.71	0\\
6.72	0\\
6.73	0\\
6.74	0\\
6.75	0\\
6.76	0\\
6.77	0\\
6.78	0\\
6.79	0\\
6.8	0\\
6.81	0\\
6.82	0\\
6.83	0\\
6.84	0\\
6.85	0\\
6.86	0\\
6.87	0\\
6.88	0\\
6.89	0\\
6.9	0\\
6.91	0\\
6.92	0\\
6.93	0\\
6.94	0\\
6.95	0\\
6.96	0\\
6.97	0\\
6.98	0\\
6.99	0\\
7	0\\
7.01	0\\
7.02	0\\
7.03	0\\
7.04	0\\
7.05	0\\
7.06	0\\
7.07	0\\
7.08	0\\
7.09	0\\
7.1	0\\
7.11	0\\
7.12	0\\
7.13	0\\
7.14	0\\
7.15	0\\
7.16	0\\
7.17	0\\
7.18	0\\
7.19	0\\
7.2	0\\
7.21	0\\
7.22	0\\
7.23	0\\
7.24	0\\
7.25	0\\
7.26	0\\
7.27	0\\
7.28	0\\
7.29	0\\
7.3	0\\
7.31	0\\
7.32	0\\
7.33	0\\
7.34	0\\
7.35	0\\
7.36	0\\
7.37	0\\
7.38	0\\
7.39	0\\
7.4	0\\
7.41	0\\
7.42	0\\
7.43	0\\
7.44	0\\
7.45	0\\
7.46	0\\
7.47	0\\
7.48	0\\
7.49	0\\
7.5	0\\
7.51	0\\
7.52	0\\
7.53	0\\
7.54	0\\
7.55	0\\
7.56	0\\
7.57	0\\
7.58	0\\
7.59	0\\
7.6	0\\
7.61	0\\
7.62	0\\
7.63	0\\
7.64	0\\
7.65	0\\
7.66	0\\
7.67	0\\
7.68	0\\
7.69	0\\
7.7	0\\
7.71	0\\
7.72	0\\
7.73	0\\
7.74	0\\
7.75	0\\
7.76	0\\
7.77	0\\
7.78	0\\
7.79	0\\
7.8	0\\
7.81	0\\
7.82	0\\
7.83	0\\
7.84	0\\
7.85	0\\
7.86	0\\
7.87	0\\
7.88	0\\
7.89	0\\
7.9	0\\
7.91	0\\
7.92	0\\
7.93	0\\
7.94	0\\
7.95	0\\
7.96	0\\
7.97	0\\
7.98	0\\
7.99	0\\
8	0\\
8.01	0\\
8.02	0\\
8.03	0\\
8.04	0\\
8.05	0\\
8.06	0\\
8.07	0\\
8.08	0\\
8.09	0\\
8.1	0\\
8.11	0\\
8.12	0\\
8.13	0\\
8.14	0\\
8.15	0\\
8.16	0\\
8.17	0\\
8.18	0\\
8.19	0\\
8.2	0\\
8.21	0\\
8.22	0\\
8.23	0\\
8.24	0\\
8.25	0\\
8.26	0\\
8.27	0\\
8.28	0\\
8.29	0\\
8.3	0\\
8.31	0\\
8.32	0\\
8.33	0\\
8.34	0\\
8.35	0\\
8.36	0\\
8.37	0\\
8.38	0\\
8.39	0\\
8.4	0\\
8.41	0\\
8.42	0\\
8.43	0\\
8.44	0\\
8.45	0\\
8.46	0\\
8.47	0\\
8.48	0\\
8.49	0\\
8.5	0\\
8.51	0\\
8.52	0\\
8.53	0\\
8.54	0\\
8.55	0\\
8.56	0\\
8.57	0\\
8.58	0\\
8.59	0\\
8.6	0\\
8.61	0\\
8.62	0\\
8.63	0\\
8.64	0\\
8.65	0\\
8.66	0\\
8.67	0\\
8.68	0\\
8.69	0\\
8.7	0\\
8.71	0\\
8.72	0\\
8.73	0\\
8.74	0\\
8.75	0\\
8.76	0\\
8.77	0\\
8.78	0\\
8.79	0\\
8.8	0\\
8.81	0\\
8.82	0\\
8.83	0\\
8.84	0\\
8.85	0\\
8.86	0\\
8.87	0\\
8.88	0\\
8.89	0\\
8.9	0\\
8.91	0\\
8.92	0\\
8.93	0\\
8.94	0\\
8.95	0\\
8.96	0\\
8.97	0\\
8.98	0\\
8.99	0\\
9	0\\
9.01	0\\
9.02	0\\
9.03	0\\
9.04	0\\
9.05	0\\
9.06	0\\
9.07	0\\
9.08	0\\
9.09	0\\
9.1	0\\
9.11	0\\
9.12	0\\
9.13	0\\
9.14	0\\
9.15	0\\
9.16	0\\
9.17	0\\
9.18	0\\
9.19	0\\
9.2	0\\
9.21	0\\
9.22	0\\
9.23	0\\
9.24	0\\
9.25	0\\
9.26	0\\
9.27	0\\
9.28	0\\
9.29	0\\
9.3	0\\
9.31	0\\
9.32	0\\
9.33	0\\
9.34	0\\
9.35	0\\
9.36	0\\
9.37	0\\
9.38	0\\
9.39	0\\
9.4	0\\
9.41	0\\
9.42	0\\
9.43	0\\
9.44	0\\
9.45	0\\
9.46	0\\
9.47	0\\
9.48	0\\
9.49	0\\
9.5	0\\
9.51	0\\
9.52	0\\
9.53	0\\
9.54	0\\
9.55	0\\
9.56	0\\
9.57	0\\
9.58	0\\
9.59	0\\
9.6	0\\
9.61	0\\
9.62	0\\
9.63	0\\
9.64	0\\
9.65	0\\
9.66	0\\
9.67	0\\
9.68	0\\
9.69	0\\
9.7	0\\
9.71	0\\
9.72	0\\
9.73	0\\
9.74	0\\
9.75	0\\
9.76	0\\
9.77	0\\
9.78	0\\
9.79	0\\
9.8	0\\
9.81	0\\
9.82	0\\
9.83	0\\
9.84	0\\
9.85	0\\
9.86	0\\
9.87	0\\
9.88	0\\
9.89	0\\
9.9	0\\
9.91	0\\
9.92	0\\
9.93	0\\
9.94	0\\
9.95	0\\
9.96	0\\
9.97	0\\
9.98	0\\
9.99	0\\
10	0\\
10.01	0\\
10.02	0\\
10.03	0\\
10.04	0\\
10.05	0\\
10.06	0\\
10.07	0\\
10.08	0\\
10.09	0\\
10.1	0\\
10.11	0\\
10.12	0\\
10.13	0\\
10.14	0\\
10.15	0\\
10.16	0\\
10.17	0\\
10.18	0\\
10.19	0\\
10.2	0\\
10.21	0\\
10.22	0\\
10.23	0\\
10.24	0\\
10.25	0\\
10.26	0\\
10.27	0\\
10.28	0\\
10.29	0\\
10.3	0\\
10.31	0\\
10.32	0\\
10.33	0\\
10.34	0\\
10.35	0\\
10.36	0\\
10.37	0\\
10.38	0\\
10.39	0\\
10.4	0\\
10.41	0\\
10.42	0\\
10.43	0\\
10.44	0\\
10.45	0\\
10.46	0\\
10.47	0\\
10.48	0\\
10.49	0\\
10.5	0\\
10.51	0\\
10.52	0\\
10.53	0\\
10.54	0\\
10.55	0\\
10.56	0\\
10.57	0\\
10.58	0\\
10.59	0\\
10.6	0\\
10.61	0\\
10.62	0\\
10.63	0\\
10.64	0\\
10.65	0\\
10.66	0\\
10.67	0\\
10.68	0\\
10.69	0\\
10.7	0\\
10.71	0\\
10.72	0\\
10.73	0\\
10.74	0\\
10.75	0\\
10.76	0\\
10.77	0\\
10.78	0\\
10.79	0\\
10.8	0\\
10.81	0\\
10.82	0\\
10.83	0\\
10.84	0\\
10.85	0\\
10.86	0\\
10.87	0\\
10.88	0\\
10.89	0\\
10.9	0\\
10.91	0\\
10.92	0\\
10.93	0\\
10.94	0\\
10.95	0\\
10.96	0\\
10.97	0\\
10.98	0\\
10.99	0\\
11	0\\
11.01	0\\
11.02	0\\
11.03	0\\
11.04	0\\
11.05	0\\
11.06	0\\
11.07	0\\
11.08	0\\
11.09	0\\
11.1	0\\
11.11	0\\
11.12	0\\
11.13	0\\
11.14	0\\
11.15	0\\
11.16	0\\
11.17	0\\
11.18	0\\
11.19	0\\
11.2	0\\
11.21	0\\
11.22	0\\
11.23	0\\
11.24	0\\
11.25	0\\
11.26	0\\
11.27	0\\
11.28	0\\
11.29	0\\
11.3	0\\
11.31	0\\
11.32	0\\
11.33	0\\
11.34	0\\
11.35	0\\
11.36	0\\
11.37	0\\
11.38	0\\
11.39	0\\
11.4	0\\
11.41	0\\
11.42	0\\
11.43	0\\
11.44	0\\
11.45	0\\
11.46	0\\
11.47	0\\
11.48	0\\
11.49	0\\
11.5	0\\
11.51	0\\
11.52	0\\
11.53	0\\
11.54	0\\
11.55	0\\
11.56	0\\
11.57	0\\
11.58	0\\
11.59	0\\
11.6	0\\
11.61	0\\
11.62	0\\
11.63	0\\
11.64	0\\
11.65	0\\
11.66	0\\
11.67	0\\
11.68	0\\
11.69	0\\
11.7	0\\
11.71	0\\
11.72	0\\
11.73	0\\
11.74	0\\
11.75	0\\
11.76	0\\
11.77	0\\
11.78	0\\
11.79	0\\
11.8	0\\
11.81	0\\
11.82	0\\
11.83	0\\
11.84	0\\
11.85	0\\
11.86	0\\
11.87	0\\
11.88	0\\
11.89	0\\
11.9	0\\
11.91	0\\
11.92	0\\
11.93	0\\
11.94	0\\
11.95	0\\
11.96	0\\
11.97	0\\
11.98	0\\
11.99	0\\
12	0\\
12.01	0\\
12.02	0\\
12.03	0\\
12.04	0\\
12.05	0\\
12.06	0\\
12.07	0\\
12.08	0\\
12.09	0\\
12.1	0\\
12.11	0\\
12.12	0\\
12.13	0\\
12.14	0\\
12.15	0\\
12.16	0\\
12.17	0\\
12.18	0\\
12.19	0\\
12.2	0\\
12.21	0\\
12.22	0\\
12.23	0\\
12.24	0\\
12.25	0\\
12.26	0\\
12.27	0\\
12.28	0\\
12.29	0\\
12.3	0\\
12.31	0\\
12.32	0\\
12.33	0\\
12.34	0\\
12.35	0\\
12.36	0\\
12.37	0\\
12.38	0\\
12.39	0\\
12.4	0\\
12.41	0\\
12.42	0\\
12.43	0\\
12.44	0\\
12.45	0\\
12.46	0\\
12.47	0\\
12.48	0\\
12.49	0\\
12.5	0\\
12.51	0\\
12.52	0\\
12.53	0\\
12.54	0\\
12.55	0\\
12.56	0\\
12.57	0\\
12.58	0\\
12.59	0\\
12.6	0\\
12.61	0\\
12.62	0\\
12.63	0\\
12.64	0\\
12.65	0\\
12.66	0\\
12.67	0\\
12.68	0\\
12.69	0\\
12.7	0\\
12.71	0\\
12.72	0\\
12.73	0\\
12.74	0\\
12.75	0\\
12.76	0\\
12.77	0\\
12.78	0\\
12.79	0\\
12.8	0\\
12.81	0\\
12.82	0\\
12.83	0\\
12.84	0\\
12.85	0\\
12.86	0\\
12.87	0\\
12.88	0\\
12.89	0\\
12.9	0\\
12.91	0\\
12.92	0\\
12.93	0\\
12.94	0\\
12.95	0\\
12.96	0\\
12.97	0\\
12.98	0\\
12.99	0\\
13	0\\
13.01	0\\
13.02	0\\
13.03	0\\
13.04	0\\
13.05	0\\
13.06	0\\
13.07	0\\
13.08	0\\
13.09	0\\
13.1	0\\
13.11	0\\
13.12	0\\
13.13	0\\
13.14	0\\
13.15	0\\
13.16	0\\
13.17	0\\
13.18	0\\
13.19	0\\
13.2	0\\
13.21	0\\
13.22	0\\
13.23	0\\
13.24	0\\
13.25	0\\
13.26	0\\
13.27	0\\
13.28	0\\
13.29	0\\
13.3	0\\
13.31	0\\
13.32	0\\
13.33	0\\
13.34	0\\
13.35	0\\
13.36	0\\
13.37	0\\
13.38	0\\
13.39	0\\
13.4	0\\
13.41	0\\
13.42	0\\
13.43	0\\
13.44	0\\
13.45	0\\
13.46	0\\
13.47	0\\
13.48	0\\
13.49	0\\
13.5	0\\
13.51	0\\
13.52	0\\
13.53	0\\
13.54	0\\
13.55	0\\
13.56	0\\
13.57	0\\
13.58	0\\
13.59	0\\
13.6	0\\
13.61	0\\
13.62	0\\
13.63	0\\
13.64	0\\
13.65	0\\
13.66	0\\
13.67	0\\
13.68	0\\
13.69	0\\
13.7	0\\
13.71	0\\
13.72	0\\
13.73	0\\
13.74	0\\
13.75	0\\
13.76	0\\
13.77	0\\
13.78	0\\
13.79	0\\
13.8	0\\
13.81	0\\
13.82	0\\
13.83	0\\
13.84	0\\
13.85	0\\
13.86	0\\
13.87	0\\
13.88	0\\
13.89	0\\
13.9	0\\
13.91	0\\
13.92	0\\
13.93	0\\
13.94	0\\
13.95	0\\
13.96	0\\
13.97	0\\
13.98	0\\
13.99	0\\
14	0\\
14.01	0\\
14.02	0\\
14.03	0\\
14.04	0\\
14.05	0\\
14.06	0\\
14.07	0\\
14.08	0\\
14.09	0\\
14.1	0\\
14.11	0\\
14.12	0\\
14.13	0\\
14.14	0\\
14.15	0\\
14.16	0\\
14.17	0\\
14.18	0\\
14.19	0\\
14.2	0\\
14.21	0\\
14.22	0\\
14.23	0\\
14.24	0\\
14.25	0\\
14.26	0\\
14.27	0\\
14.28	0\\
14.29	0\\
14.3	0\\
14.31	0\\
14.32	0\\
14.33	0\\
14.34	0\\
14.35	0\\
14.36	0\\
14.37	0\\
14.38	0\\
14.39	0\\
14.4	0\\
14.41	0\\
14.42	0\\
14.43	0\\
14.44	0\\
14.45	0\\
14.46	0\\
14.47	0\\
14.48	0\\
14.49	0\\
14.5	0\\
14.51	0\\
14.52	0\\
14.53	0\\
14.54	0\\
14.55	0\\
14.56	0\\
14.57	0\\
14.58	0\\
14.59	0\\
14.6	0\\
14.61	0\\
14.62	0\\
14.63	0\\
14.64	0\\
14.65	0\\
14.66	0\\
14.67	0\\
14.68	0\\
14.69	0\\
14.7	0\\
14.71	0\\
14.72	0\\
14.73	0\\
14.74	0\\
14.75	0\\
14.76	0\\
14.77	0\\
14.78	0\\
14.79	0\\
14.8	0\\
14.81	0\\
14.82	0\\
14.83	0\\
14.84	0\\
14.85	0\\
14.86	0\\
14.87	0\\
14.88	0\\
14.89	0\\
14.9	0\\
14.91	0\\
14.92	0\\
14.93	0\\
14.94	0\\
14.95	0\\
14.96	0\\
14.97	0\\
14.98	0\\
14.99	0\\
15	0\\
15.01	0\\
15.02	0\\
15.03	0\\
15.04	0\\
15.05	0\\
15.06	0\\
15.07	0\\
15.08	0\\
15.09	0\\
15.1	0\\
15.11	0\\
15.12	0\\
15.13	0\\
15.14	0\\
15.15	0\\
15.16	0\\
15.17	0\\
15.18	0\\
15.19	0\\
15.2	0\\
15.21	0\\
15.22	0\\
15.23	0\\
15.24	0\\
15.25	0\\
15.26	0\\
15.27	0\\
15.28	0\\
15.29	0\\
15.3	0\\
15.31	0\\
15.32	0\\
15.33	0\\
15.34	0\\
15.35	0\\
15.36	0\\
15.37	0\\
15.38	0\\
15.39	0\\
15.4	0\\
15.41	0\\
15.42	0\\
15.43	0\\
15.44	0\\
15.45	0\\
15.46	0\\
15.47	0\\
15.48	0\\
15.49	0\\
15.5	0\\
15.51	0\\
15.52	0\\
15.53	0\\
15.54	0\\
15.55	0\\
15.56	0\\
15.57	0\\
15.58	0\\
15.59	0\\
15.6	0\\
15.61	0\\
15.62	0\\
15.63	0\\
15.64	0\\
15.65	0\\
15.66	0\\
15.67	0\\
15.68	0\\
15.69	0\\
15.7	0\\
15.71	0\\
15.72	0\\
15.73	0\\
15.74	0\\
15.75	0\\
15.76	0\\
15.77	0\\
15.78	0\\
15.79	0\\
15.8	0\\
15.81	0\\
15.82	0\\
15.83	0\\
15.84	0\\
15.85	0\\
15.86	0\\
15.87	0\\
15.88	0\\
15.89	0\\
15.9	0\\
15.91	0\\
15.92	0\\
15.93	0\\
15.94	0\\
15.95	0\\
15.96	0\\
15.97	0\\
15.98	0\\
15.99	0\\
16	0\\
16.01	0\\
16.02	0\\
16.03	0\\
16.04	0\\
16.05	0\\
16.06	0\\
16.07	0\\
16.08	0\\
16.09	0\\
16.1	0\\
16.11	0\\
16.12	0\\
16.13	0\\
16.14	0\\
16.15	0\\
16.16	0\\
16.17	0\\
16.18	0\\
16.19	0\\
16.2	0\\
16.21	0\\
16.22	0\\
16.23	0\\
16.24	0\\
16.25	0\\
16.26	0\\
16.27	0\\
16.28	0\\
16.29	0\\
16.3	0\\
16.31	0\\
16.32	0\\
16.33	0\\
16.34	0\\
16.35	0\\
16.36	0\\
16.37	0\\
16.38	0\\
16.39	0\\
16.4	0\\
16.41	0\\
16.42	0\\
16.43	0\\
16.44	0\\
16.45	0\\
16.46	0\\
16.47	0\\
16.48	0\\
16.49	0\\
16.5	0\\
16.51	0\\
16.52	0\\
16.53	0\\
16.54	0\\
16.55	0\\
16.56	0\\
16.57	0\\
16.58	0\\
16.59	0\\
16.6	0\\
16.61	0\\
16.62	0\\
16.63	0\\
16.64	0\\
16.65	0\\
16.66	0\\
16.67	0\\
16.68	0\\
16.69	0\\
16.7	0\\
16.71	0\\
16.72	0\\
16.73	0\\
16.74	0\\
16.75	0\\
16.76	0\\
16.77	0\\
16.78	0\\
16.79	0\\
16.8	0\\
16.81	0\\
16.82	0\\
16.83	0\\
16.84	0\\
16.85	0\\
16.86	0\\
16.87	0\\
16.88	0\\
16.89	0\\
16.9	0\\
16.91	0\\
16.92	0\\
16.93	0\\
16.94	0\\
16.95	0\\
16.96	0\\
16.97	0\\
16.98	0\\
16.99	0\\
17	0\\
17.01	0\\
17.02	0\\
17.03	0\\
17.04	0\\
17.05	0\\
17.06	0\\
17.07	0\\
17.08	0\\
17.09	0\\
17.1	0\\
17.11	0\\
17.12	0\\
17.13	0\\
17.14	0\\
17.15	0\\
17.16	0\\
17.17	0\\
17.18	0\\
17.19	0\\
17.2	0\\
17.21	0\\
17.22	0\\
17.23	0\\
17.24	0\\
17.25	0\\
17.26	0\\
17.27	0\\
17.28	0\\
17.29	0\\
17.3	0\\
17.31	0\\
17.32	0\\
17.33	0\\
17.34	0\\
17.35	0\\
17.36	0\\
17.37	0\\
17.38	0\\
17.39	0\\
17.4	0\\
17.41	0\\
17.42	0\\
17.43	0\\
17.44	0\\
17.45	0\\
17.46	0\\
17.47	0\\
17.48	0\\
17.49	0\\
17.5	0\\
17.51	0\\
17.52	0\\
17.53	0\\
17.54	0\\
17.55	0\\
17.56	0\\
17.57	0\\
17.58	0\\
17.59	0\\
17.6	0\\
17.61	0\\
17.62	0\\
17.63	0\\
17.64	0\\
17.65	0\\
17.66	0\\
17.67	0\\
17.68	0\\
17.69	0\\
17.7	0\\
17.71	0\\
17.72	0\\
17.73	0\\
17.74	0\\
17.75	0\\
17.76	0\\
17.77	0\\
17.78	0\\
17.79	0\\
17.8	0\\
17.81	0\\
17.82	0\\
17.83	0\\
17.84	0\\
17.85	0\\
17.86	0\\
17.87	0\\
17.88	0\\
17.89	0\\
17.9	0\\
17.91	0\\
17.92	0\\
17.93	0\\
17.94	0\\
17.95	0\\
17.96	0\\
17.97	0\\
17.98	0\\
17.99	0\\
18	0\\
18.01	0\\
18.02	0\\
18.03	0\\
18.04	0\\
18.05	0\\
18.06	0\\
18.07	0\\
18.08	0\\
18.09	0\\
18.1	0\\
18.11	0\\
18.12	0\\
18.13	0\\
18.14	0\\
18.15	0\\
18.16	0\\
18.17	0\\
18.18	0\\
18.19	0\\
18.2	0\\
18.21	0\\
18.22	0\\
18.23	0\\
18.24	0\\
18.25	0\\
18.26	0\\
18.27	0\\
18.28	0\\
18.29	0\\
18.3	0\\
18.31	0\\
18.32	0\\
18.33	0\\
18.34	0\\
18.35	0\\
18.36	0\\
18.37	0\\
18.38	0\\
18.39	0\\
18.4	0\\
18.41	0\\
18.42	0\\
18.43	0\\
18.44	0\\
18.45	0\\
18.46	0\\
18.47	0\\
18.48	0\\
18.49	0\\
18.5	0\\
18.51	0\\
18.52	0\\
18.53	0\\
18.54	0\\
18.55	0\\
18.56	0\\
18.57	0\\
18.58	0\\
18.59	0\\
18.6	0\\
18.61	0\\
18.62	0\\
18.63	0\\
18.64	0\\
18.65	0\\
18.66	0\\
18.67	0\\
18.68	0\\
18.69	0\\
18.7	0\\
18.71	0\\
18.72	0\\
18.73	0\\
18.74	0\\
18.75	0\\
18.76	0\\
18.77	0\\
18.78	0\\
18.79	0\\
18.8	0\\
18.81	0\\
18.82	0\\
18.83	0\\
18.84	0\\
18.85	0\\
18.86	0\\
18.87	0\\
18.88	0\\
18.89	0\\
18.9	0\\
18.91	0\\
18.92	0\\
18.93	0\\
18.94	0\\
18.95	0\\
18.96	0\\
18.97	0\\
18.98	0\\
18.99	0\\
19	0\\
19.01	0\\
19.02	0\\
19.03	0\\
19.04	0\\
19.05	0\\
19.06	0\\
19.07	0\\
19.08	0\\
19.09	0\\
19.1	0\\
19.11	0\\
19.12	0\\
19.13	0\\
19.14	0\\
19.15	0\\
19.16	0\\
19.17	0\\
19.18	0\\
19.19	0\\
19.2	0\\
19.21	0\\
19.22	0\\
19.23	0\\
19.24	0\\
19.25	0\\
19.26	0\\
19.27	0\\
19.28	0\\
19.29	0\\
19.3	0\\
19.31	0\\
19.32	0\\
19.33	0\\
19.34	0\\
19.35	0\\
19.36	0\\
19.37	0\\
19.38	0\\
19.39	0\\
19.4	0\\
19.41	0\\
19.42	0\\
19.43	0\\
19.44	0\\
19.45	0\\
19.46	0\\
19.47	0\\
19.48	0\\
19.49	0\\
19.5	0\\
19.51	0\\
19.52	0\\
19.53	0\\
19.54	0\\
19.55	0\\
19.56	0\\
19.57	0\\
19.58	0\\
19.59	0\\
19.6	0\\
19.61	0\\
19.62	0\\
19.63	0\\
19.64	0\\
19.65	0\\
19.66	0\\
19.67	0\\
19.68	0\\
19.69	0\\
19.7	0\\
19.71	0\\
19.72	0\\
19.73	0\\
19.74	0\\
19.75	0\\
19.76	0\\
19.77	0\\
19.78	0\\
19.79	0\\
19.8	0\\
19.81	0\\
19.82	0\\
19.83	0\\
19.84	0\\
19.85	0\\
19.86	0\\
19.87	0\\
19.88	0\\
19.89	0\\
19.9	0\\
19.91	0\\
19.92	0\\
19.93	0\\
19.94	0\\
19.95	0\\
19.96	0\\
19.97	0\\
19.98	0\\
19.99	0\\
20	0\\
20.01	0\\
20.02	0\\
20.03	0\\
20.04	0\\
20.05	0\\
20.06	0\\
20.07	0\\
20.08	0\\
20.09	0\\
20.1	0\\
20.11	0\\
20.12	0\\
20.13	0\\
20.14	0\\
20.15	0\\
20.16	0\\
20.17	0\\
20.18	0\\
20.19	0\\
20.2	0\\
20.21	0\\
20.22	0\\
20.23	0\\
20.24	0\\
20.25	0\\
20.26	0\\
20.27	0\\
20.28	0\\
20.29	0\\
20.3	0\\
20.31	0\\
20.32	0\\
20.33	0\\
20.34	0\\
20.35	0\\
20.36	0\\
20.37	0\\
20.38	0\\
20.39	0\\
20.4	0\\
20.41	0\\
20.42	0\\
20.43	0\\
20.44	0\\
20.45	0\\
20.46	0\\
20.47	0\\
20.48	0\\
20.49	0\\
20.5	0\\
20.51	0\\
20.52	0\\
20.53	0\\
20.54	0\\
20.55	0\\
20.56	0\\
20.57	0\\
20.58	0\\
20.59	0\\
20.6	0\\
20.61	0\\
20.62	0\\
20.63	0\\
20.64	0\\
20.65	0\\
20.66	0\\
20.67	0\\
20.68	0\\
20.69	0\\
20.7	0\\
20.71	0\\
20.72	0\\
20.73	0\\
20.74	0\\
20.75	0\\
20.76	0\\
20.77	0\\
20.78	0\\
20.79	0\\
20.8	0\\
20.81	0\\
20.82	0\\
20.83	0\\
20.84	0\\
20.85	0\\
20.86	0\\
20.87	0\\
20.88	0\\
20.89	0\\
20.9	0\\
20.91	0\\
20.92	0\\
20.93	0\\
20.94	0\\
20.95	0\\
20.96	0\\
20.97	0\\
20.98	0\\
20.99	0\\
21	0\\
21.01	0\\
21.02	0\\
21.03	0\\
21.04	0\\
21.05	0\\
21.06	0\\
21.07	0\\
21.08	0\\
21.09	0\\
21.1	0\\
21.11	0\\
21.12	0\\
21.13	0\\
21.14	0\\
21.15	0\\
21.16	0\\
21.17	0\\
21.18	0\\
21.19	0\\
21.2	0\\
21.21	0\\
21.22	0\\
21.23	0\\
21.24	0\\
21.25	0\\
21.26	0\\
21.27	0\\
21.28	0\\
21.29	0\\
21.3	0\\
21.31	0\\
21.32	0\\
21.33	0\\
21.34	0\\
21.35	0\\
21.36	0\\
21.37	0\\
21.38	0\\
21.39	0\\
21.4	0\\
21.41	0\\
21.42	0\\
21.43	0\\
21.44	0\\
21.45	0\\
21.46	0\\
21.47	0\\
21.48	0\\
21.49	0\\
21.5	0\\
21.51	0\\
21.52	0\\
21.53	0\\
21.54	0\\
21.55	0\\
21.56	0\\
21.57	0\\
21.58	0\\
21.59	0\\
21.6	0\\
21.61	0\\
21.62	0\\
21.63	0\\
21.64	0\\
21.65	0\\
21.66	0\\
21.67	0\\
21.68	0\\
21.69	0\\
21.7	0\\
21.71	0\\
21.72	0\\
21.73	0\\
21.74	0\\
21.75	0\\
21.76	0\\
21.77	0\\
21.78	0\\
21.79	0\\
21.8	0\\
21.81	0\\
21.82	0\\
21.83	0\\
21.84	0\\
21.85	0\\
21.86	0\\
21.87	0\\
21.88	0\\
21.89	0\\
21.9	0\\
21.91	0\\
21.92	0\\
21.93	0\\
21.94	0\\
21.95	0\\
21.96	0\\
21.97	0\\
21.98	0\\
21.99	0\\
22	0\\
22.01	0\\
22.02	0\\
22.03	0\\
22.04	0\\
22.05	0\\
22.06	0\\
22.07	0\\
22.08	0\\
22.09	0\\
22.1	0\\
22.11	0\\
22.12	0\\
22.13	0\\
22.14	0\\
22.15	0\\
22.16	0\\
22.17	0\\
22.18	0\\
22.19	0\\
22.2	0\\
22.21	0\\
22.22	0\\
22.23	0\\
22.24	0\\
22.25	0\\
22.26	0\\
22.27	0\\
22.28	0\\
22.29	0\\
22.3	0\\
22.31	0\\
22.32	0\\
22.33	0\\
22.34	0\\
22.35	0\\
22.36	0\\
22.37	0\\
22.38	0\\
22.39	0\\
22.4	0\\
22.41	0\\
22.42	0\\
22.43	0\\
22.44	0\\
22.45	0\\
22.46	0\\
22.47	0\\
22.48	0\\
22.49	0\\
22.5	0\\
22.51	0\\
22.52	0\\
22.53	0\\
22.54	0\\
22.55	0\\
22.56	0\\
22.57	0\\
22.58	0\\
22.59	0\\
22.6	0\\
22.61	0\\
22.62	0\\
22.63	0\\
22.64	0\\
22.65	0\\
22.66	0\\
22.67	0\\
22.68	0\\
22.69	0\\
22.7	0\\
22.71	0\\
22.72	0\\
22.73	0\\
22.74	0\\
22.75	0\\
22.76	0\\
22.77	0\\
22.78	0\\
22.79	0\\
22.8	0\\
22.81	0\\
22.82	0\\
22.83	0\\
22.84	0\\
22.85	0\\
22.86	0\\
22.87	0\\
22.88	0\\
22.89	0\\
22.9	0\\
22.91	0\\
22.92	0\\
22.93	0\\
22.94	0\\
22.95	0\\
22.96	0\\
22.97	0\\
22.98	0\\
22.99	0\\
23	0\\
23.01	0\\
23.02	0\\
23.03	0\\
23.04	0\\
23.05	0\\
23.06	0\\
23.07	0\\
23.08	0\\
23.09	0\\
23.1	0\\
23.11	0\\
23.12	0\\
23.13	0\\
23.14	0\\
23.15	0\\
23.16	0\\
23.17	0\\
23.18	0\\
23.19	0\\
23.2	0\\
23.21	0\\
23.22	0\\
23.23	0\\
23.24	0\\
23.25	0\\
23.26	0\\
23.27	0\\
23.28	0\\
23.29	0\\
23.3	0\\
23.31	0\\
23.32	0\\
23.33	0\\
23.34	0\\
23.35	0\\
23.36	0\\
23.37	0\\
23.38	0\\
23.39	0\\
23.4	0\\
23.41	0\\
23.42	0\\
23.43	0\\
23.44	0\\
23.45	0\\
23.46	0\\
23.47	0\\
23.48	0\\
23.49	0\\
23.5	0\\
23.51	0\\
23.52	0\\
23.53	0\\
23.54	0\\
23.55	0\\
23.56	0\\
23.57	0\\
23.58	0\\
23.59	0\\
23.6	0\\
23.61	0\\
23.62	0\\
23.63	0\\
23.64	0\\
23.65	0\\
23.66	0\\
23.67	0\\
23.68	0\\
23.69	0\\
23.7	0\\
23.71	0\\
23.72	0\\
23.73	0\\
23.74	0\\
23.75	0\\
23.76	0\\
23.77	0\\
23.78	0\\
23.79	0\\
23.8	0\\
23.81	0\\
23.82	0\\
23.83	0\\
23.84	0\\
23.85	0\\
23.86	0\\
23.87	0\\
23.88	0\\
23.89	0\\
23.9	0\\
23.91	0\\
23.92	0\\
23.93	0\\
23.94	0\\
23.95	0\\
23.96	0\\
23.97	0\\
23.98	0\\
23.99	0\\
24	0\\
24.01	0\\
24.02	0\\
24.03	0\\
24.04	0\\
24.05	0\\
24.06	0\\
24.07	0\\
24.08	0\\
24.09	0\\
24.1	0\\
24.11	0\\
24.12	0\\
24.13	0\\
24.14	0\\
24.15	0\\
24.16	0\\
24.17	0\\
24.18	0\\
24.19	0\\
24.2	0\\
24.21	0\\
24.22	0\\
24.23	0\\
24.24	0\\
24.25	0\\
24.26	0\\
24.27	0\\
24.28	0\\
24.29	0\\
24.3	0\\
24.31	0\\
24.32	0\\
24.33	0\\
24.34	0\\
24.35	0\\
24.36	0\\
24.37	0\\
24.38	0\\
24.39	0\\
24.4	0\\
24.41	0\\
24.42	0\\
24.43	0\\
24.44	0\\
24.45	0\\
24.46	0\\
24.47	0\\
24.48	0\\
24.49	0\\
24.5	0\\
24.51	0\\
24.52	0\\
24.53	0\\
24.54	0\\
24.55	0\\
24.56	0\\
24.57	0\\
24.58	0\\
24.59	0\\
24.6	0\\
24.61	0\\
24.62	0\\
24.63	0\\
24.64	0\\
24.65	0\\
24.66	0\\
24.67	0\\
24.68	0\\
24.69	0\\
24.7	0\\
24.71	0\\
24.72	0\\
24.73	0\\
24.74	0\\
24.75	0\\
24.76	0\\
24.77	0\\
24.78	0\\
24.79	0\\
24.8	0\\
24.81	0\\
24.82	0\\
24.83	0\\
24.84	0\\
24.85	0\\
24.86	0\\
24.87	0\\
24.88	0\\
24.89	0\\
24.9	0\\
24.91	0\\
24.92	0\\
24.93	0\\
24.94	0\\
24.95	0\\
24.96	0\\
24.97	0\\
24.98	0\\
24.99	0\\
25	0\\
25.01	0\\
25.02	0\\
25.03	0\\
25.04	0\\
25.05	0\\
25.06	0\\
25.07	0\\
25.08	0\\
25.09	0\\
25.1	0\\
25.11	0\\
25.12	0\\
25.13	0\\
25.14	0\\
25.15	0\\
25.16	0\\
25.17	0\\
25.18	0\\
25.19	0\\
25.2	0\\
25.21	0\\
25.22	0\\
25.23	0\\
25.24	0\\
25.25	0\\
25.26	0\\
25.27	0\\
25.28	0\\
25.29	0\\
25.3	0\\
25.31	0\\
25.32	0\\
25.33	0\\
25.34	0\\
25.35	0\\
25.36	0\\
25.37	0\\
25.38	0\\
25.39	0\\
25.4	0\\
25.41	0\\
25.42	0\\
25.43	0\\
25.44	0\\
25.45	0\\
25.46	0\\
25.47	0\\
25.48	0\\
25.49	0\\
25.5	0\\
25.51	0\\
25.52	0\\
25.53	0\\
25.54	0\\
25.55	0\\
25.56	0\\
25.57	0\\
25.58	0\\
25.59	0\\
25.6	0\\
25.61	0\\
25.62	0\\
25.63	0\\
25.64	0\\
25.65	0\\
25.66	0\\
25.67	0\\
25.68	0\\
25.69	0\\
25.7	0\\
25.71	0\\
25.72	0\\
25.73	0\\
25.74	0\\
25.75	0\\
25.76	0\\
25.77	0\\
25.78	0\\
25.79	0\\
25.8	0\\
25.81	0\\
25.82	0\\
25.83	0\\
25.84	0\\
25.85	0\\
25.86	0\\
25.87	0\\
25.88	0\\
25.89	0\\
25.9	0\\
25.91	0\\
25.92	0\\
25.93	0\\
25.94	0\\
25.95	0\\
25.96	0\\
25.97	0\\
25.98	0\\
25.99	0\\
26	0\\
26.01	0\\
26.02	0\\
26.03	0\\
26.04	0\\
26.05	0\\
26.06	0\\
26.07	0\\
26.08	0\\
26.09	0\\
26.1	0\\
26.11	0\\
26.12	0\\
26.13	0\\
26.14	0\\
26.15	0\\
26.16	0\\
26.17	0\\
26.18	0\\
26.19	0\\
26.2	0\\
26.21	0\\
26.22	0\\
26.23	0\\
26.24	0\\
26.25	0\\
26.26	0\\
26.27	0\\
26.28	0\\
26.29	0\\
26.3	0\\
26.31	0\\
26.32	0\\
26.33	0\\
26.34	0\\
26.35	0\\
26.36	0\\
26.37	0\\
26.38	0\\
26.39	0\\
26.4	0\\
26.41	0\\
26.42	0\\
26.43	0\\
26.44	0\\
26.45	0\\
26.46	0\\
26.47	0\\
26.48	0\\
26.49	0\\
26.5	0\\
26.51	0\\
26.52	0\\
26.53	0\\
26.54	0\\
26.55	0\\
26.56	0\\
26.57	0\\
26.58	0\\
26.59	0\\
26.6	0\\
26.61	0\\
26.62	0\\
26.63	0\\
26.64	0\\
26.65	0\\
26.66	0\\
26.67	0\\
26.68	0\\
26.69	0\\
26.7	0\\
26.71	0\\
26.72	0\\
26.73	0\\
26.74	0\\
26.75	0\\
26.76	0\\
26.77	0\\
26.78	0\\
26.79	0\\
26.8	0\\
26.81	0\\
26.82	0\\
26.83	0\\
26.84	0\\
26.85	0\\
26.86	0\\
26.87	0\\
26.88	0\\
26.89	0\\
26.9	0\\
26.91	0\\
26.92	0\\
26.93	0\\
26.94	0\\
26.95	0\\
26.96	0\\
26.97	0\\
26.98	0\\
26.99	0\\
27	0\\
27.01	0\\
27.02	0\\
27.03	0\\
27.04	0\\
27.05	0\\
27.06	0\\
27.07	0\\
27.08	0\\
27.09	0\\
27.1	0\\
27.11	0\\
27.12	0\\
27.13	0\\
27.14	0\\
27.15	0\\
27.16	0\\
27.17	0\\
27.18	0\\
27.19	0\\
27.2	0\\
27.21	0\\
27.22	0\\
27.23	0\\
27.24	0\\
27.25	0\\
27.26	0\\
27.27	0\\
27.28	0\\
27.29	0\\
27.3	0\\
27.31	0\\
27.32	0\\
27.33	0\\
27.34	0\\
27.35	0\\
27.36	0\\
27.37	0\\
27.38	0\\
27.39	0\\
27.4	0\\
27.41	0\\
27.42	0\\
27.43	0\\
27.44	0\\
27.45	0\\
27.46	0\\
27.47	0\\
27.48	0\\
27.49	0\\
27.5	0\\
27.51	0\\
27.52	0\\
27.53	0\\
27.54	0\\
27.55	0\\
27.56	0\\
27.57	0\\
27.58	0\\
27.59	0\\
27.6	0\\
27.61	0\\
27.62	0\\
27.63	0\\
27.64	0\\
27.65	0\\
27.66	0\\
27.67	0\\
27.68	0\\
27.69	0\\
27.7	0\\
27.71	0\\
27.72	0\\
27.73	0\\
27.74	0\\
27.75	0\\
27.76	0\\
27.77	0\\
27.78	0\\
27.79	0\\
27.8	0\\
27.81	0\\
27.82	0\\
27.83	0\\
27.84	0\\
27.85	0\\
27.86	0\\
27.87	0\\
27.88	0\\
27.89	0\\
27.9	0\\
27.91	0\\
27.92	0\\
27.93	0\\
27.94	0\\
27.95	0\\
27.96	0\\
27.97	0\\
27.98	0\\
27.99	0\\
28	0\\
28.01	0\\
28.02	0\\
28.03	0\\
28.04	0\\
28.05	0\\
28.06	0\\
28.07	0\\
28.08	0\\
28.09	0\\
28.1	0\\
28.11	0\\
28.12	0\\
28.13	0\\
28.14	0\\
28.15	0\\
28.16	0\\
28.17	0\\
28.18	0\\
28.19	0\\
28.2	0\\
28.21	0\\
28.22	0\\
28.23	0\\
28.24	0\\
28.25	0\\
28.26	0\\
28.27	0\\
28.28	0\\
28.29	0\\
28.3	0\\
28.31	0\\
28.32	0\\
28.33	0\\
28.34	0\\
28.35	0\\
28.36	0\\
28.37	0\\
28.38	0\\
28.39	0\\
28.4	0\\
28.41	0\\
28.42	0\\
28.43	0\\
28.44	0\\
28.45	0\\
28.46	0\\
28.47	0\\
28.48	0\\
28.49	0\\
28.5	0\\
28.51	0\\
28.52	0\\
28.53	0\\
28.54	0\\
28.55	0\\
28.56	0\\
28.57	0\\
28.58	0\\
28.59	0\\
28.6	0\\
28.61	0\\
28.62	0\\
28.63	0\\
28.64	0\\
28.65	0\\
28.66	0\\
28.67	0\\
28.68	0\\
28.69	0\\
28.7	0\\
28.71	0\\
28.72	0\\
28.73	0\\
28.74	0\\
28.75	0\\
28.76	0\\
28.77	0\\
28.78	0\\
28.79	0\\
28.8	0\\
28.81	0\\
28.82	0\\
28.83	0\\
28.84	0\\
28.85	0\\
28.86	0\\
28.87	0\\
28.88	0\\
28.89	0\\
28.9	0\\
28.91	0\\
28.92	0\\
28.93	0\\
28.94	0\\
28.95	0\\
28.96	0\\
28.97	0\\
28.98	0\\
28.99	0\\
29	0\\
29.01	0\\
29.02	0\\
29.03	0\\
29.04	0\\
29.05	0\\
29.06	0\\
29.07	0\\
29.08	0\\
29.09	0\\
29.1	0\\
29.11	0\\
29.12	0\\
29.13	0\\
29.14	0\\
29.15	0\\
29.16	0\\
29.17	0\\
29.18	0\\
29.19	0\\
29.2	0\\
29.21	0\\
29.22	0\\
29.23	0\\
29.24	0\\
29.25	0\\
29.26	0\\
29.27	0\\
29.28	0\\
29.29	0\\
29.3	0\\
29.31	0\\
29.32	0\\
29.33	0\\
29.34	0\\
29.35	0\\
29.36	0\\
29.37	0\\
29.38	0\\
29.39	0\\
29.4	0\\
29.41	0\\
29.42	0\\
29.43	0\\
29.44	0\\
29.45	0\\
29.46	0\\
29.47	0\\
29.48	0\\
29.49	0\\
29.5	0\\
29.51	0\\
29.52	0\\
29.53	0\\
29.54	0\\
29.55	0\\
29.56	0\\
29.57	0\\
29.58	0\\
29.59	0\\
29.6	0\\
29.61	0\\
29.62	0\\
29.63	0\\
29.64	0\\
29.65	0\\
29.66	0\\
29.67	0\\
29.68	0\\
29.69	0\\
29.7	0\\
29.71	0\\
29.72	0\\
29.73	0\\
29.74	0\\
29.75	0\\
29.76	0\\
29.77	0\\
29.78	0\\
29.79	0\\
29.8	0\\
29.81	0\\
29.82	0\\
29.83	0\\
29.84	0\\
29.85	0\\
29.86	0\\
29.87	0\\
29.88	0\\
29.89	0\\
29.9	0\\
29.91	0\\
29.92	0\\
29.93	0\\
29.94	0\\
29.95	0\\
29.96	0\\
29.97	0\\
29.98	0\\
29.99	0\\
30	0\\
30.01	0\\
30.02	0\\
30.03	0\\
30.04	0\\
30.05	0\\
30.06	0\\
30.07	0\\
30.08	0\\
30.09	0\\
30.1	0\\
30.11	0\\
30.12	0\\
30.13	0\\
30.14	0\\
30.15	0\\
30.16	0\\
30.17	0\\
30.18	0\\
30.19	0\\
30.2	0\\
30.21	0\\
30.22	0\\
30.23	0\\
30.24	0\\
30.25	0\\
30.26	0\\
30.27	0\\
30.28	0\\
30.29	0\\
30.3	0\\
30.31	0\\
30.32	0\\
30.33	0\\
30.34	0\\
30.35	0\\
30.36	0\\
30.37	0\\
30.38	0\\
30.39	0\\
30.4	0\\
30.41	0\\
30.42	0\\
30.43	0\\
30.44	0\\
30.45	0\\
30.46	0\\
30.47	0\\
30.48	0\\
30.49	0\\
30.5	0\\
30.51	0\\
30.52	0\\
30.53	0\\
30.54	0\\
30.55	0\\
30.56	0\\
30.57	0\\
30.58	0\\
30.59	0\\
30.6	0\\
30.61	0\\
30.62	0\\
30.63	0\\
30.64	0\\
30.65	0\\
30.66	0\\
30.67	0\\
30.68	0\\
30.69	0\\
30.7	0\\
30.71	0\\
30.72	0\\
30.73	0\\
30.74	0\\
30.75	0\\
30.76	0\\
30.77	0\\
30.78	0\\
30.79	0\\
30.8	0\\
30.81	0\\
30.82	0\\
30.83	0\\
30.84	0\\
30.85	0\\
30.86	0\\
30.87	0\\
30.88	0\\
30.89	0\\
30.9	0\\
30.91	0\\
30.92	0\\
30.93	0\\
30.94	0\\
30.95	0\\
30.96	0\\
30.97	0\\
30.98	0\\
30.99	0\\
31	0\\
31.01	0\\
31.02	0\\
31.03	0\\
31.04	0\\
31.05	0\\
31.06	0\\
31.07	0\\
31.08	0\\
31.09	0\\
31.1	0\\
31.11	0\\
31.12	0\\
31.13	0\\
31.14	0\\
31.15	0\\
31.16	0\\
31.17	0\\
31.18	0\\
31.19	0\\
31.2	0\\
31.21	0\\
31.22	0\\
31.23	0\\
31.24	0\\
31.25	0\\
31.26	0\\
31.27	0\\
31.28	0\\
31.29	0\\
31.3	0\\
31.31	0\\
31.32	0\\
31.33	0\\
31.34	0\\
31.35	0\\
31.36	0\\
31.37	0\\
31.38	0\\
31.39	0\\
31.4	0\\
31.41	0\\
31.42	0\\
31.43	0\\
31.44	0\\
31.45	0\\
31.46	0\\
31.47	0\\
31.48	0\\
31.49	0\\
31.5	0\\
31.51	0\\
31.52	0\\
31.53	0\\
31.54	0\\
31.55	0\\
31.56	0\\
31.57	0\\
31.58	0\\
31.59	0\\
31.6	0\\
31.61	0\\
31.62	0\\
31.63	0\\
31.64	0\\
31.65	0\\
31.66	0\\
31.67	0\\
31.68	0\\
31.69	0\\
31.7	0\\
31.71	0\\
31.72	0\\
31.73	0\\
31.74	0\\
31.75	0\\
31.76	0\\
31.77	0\\
31.78	0\\
31.79	0\\
31.8	0\\
31.81	0\\
31.82	0\\
31.83	0\\
31.84	0\\
31.85	0\\
31.86	0\\
31.87	0\\
31.88	0\\
31.89	0\\
31.9	0\\
31.91	0\\
31.92	0\\
31.93	0\\
31.94	0\\
31.95	0\\
31.96	0\\
31.97	0\\
31.98	0\\
31.99	0\\
32	0\\
32.01	0\\
32.02	0\\
32.03	0\\
32.04	0\\
32.05	0\\
32.06	0\\
32.07	0\\
32.08	0\\
32.09	0\\
32.1	0\\
32.11	0\\
32.12	0\\
32.13	0\\
32.14	0\\
32.15	0\\
32.16	0\\
32.17	0\\
32.18	0\\
32.19	0\\
32.2	0\\
32.21	0\\
32.22	0\\
32.23	0\\
32.24	0\\
32.25	0\\
32.26	0\\
32.27	0\\
32.28	0\\
32.29	0\\
32.3	0\\
32.31	0\\
32.32	0\\
32.33	0\\
32.34	0\\
32.35	0\\
32.36	0\\
32.37	0\\
32.38	0\\
32.39	0\\
32.4	0\\
32.41	0\\
32.42	0\\
32.43	0\\
32.44	0\\
32.45	0\\
32.46	0\\
32.47	0\\
32.48	0\\
32.49	0\\
32.5	0\\
32.51	0\\
32.52	0\\
32.53	0\\
32.54	0\\
32.55	0\\
32.56	0\\
32.57	0\\
32.58	0\\
32.59	0\\
32.6	0\\
32.61	0\\
32.62	0\\
32.63	0\\
32.64	0\\
32.65	0\\
32.66	0\\
32.67	0\\
32.68	0\\
32.69	0\\
32.7	0\\
32.71	0\\
32.72	0\\
32.73	0\\
32.74	0\\
32.75	0\\
32.76	0\\
32.77	0\\
32.78	0\\
32.79	0\\
32.8	0\\
32.81	0\\
32.82	0\\
32.83	0\\
32.84	0\\
32.85	0\\
32.86	0\\
32.87	0\\
32.88	0\\
32.89	0\\
32.9	0\\
32.91	0\\
32.92	0\\
32.93	0\\
32.94	0\\
32.95	0\\
32.96	0\\
32.97	0\\
32.98	0\\
32.99	0\\
33	0\\
33.01	0\\
33.02	0\\
33.03	0\\
33.04	0\\
33.05	0\\
33.06	0\\
33.07	0\\
33.08	0\\
33.09	0\\
33.1	0\\
33.11	0\\
33.12	0\\
33.13	0\\
33.14	0\\
33.15	0\\
33.16	0\\
33.17	0\\
33.18	0\\
33.19	0\\
33.2	0\\
33.21	0\\
33.22	0\\
33.23	0\\
33.24	0\\
33.25	0\\
33.26	0\\
33.27	0\\
33.28	0\\
33.29	0\\
33.3	0\\
33.31	0\\
33.32	0\\
33.33	0\\
33.34	0\\
33.35	0\\
33.36	0\\
33.37	0\\
33.38	0\\
33.39	0\\
33.4	0\\
33.41	0\\
33.42	0\\
33.43	0\\
33.44	0\\
33.45	0\\
33.46	0\\
33.47	0\\
33.48	0\\
33.49	0\\
33.5	0\\
33.51	0\\
33.52	0\\
33.53	0\\
33.54	0\\
33.55	0\\
33.56	0\\
33.57	0\\
33.58	0\\
33.59	0\\
33.6	0\\
33.61	0\\
33.62	0\\
33.63	0\\
33.64	0\\
33.65	0\\
33.66	0\\
33.67	0\\
33.68	0\\
33.69	0\\
33.7	0\\
33.71	0\\
33.72	0\\
33.73	0\\
33.74	0\\
33.75	0\\
33.76	0\\
33.77	0\\
33.78	0\\
33.79	0\\
33.8	0\\
33.81	0\\
33.82	0\\
33.83	0\\
33.84	0\\
33.85	0\\
33.86	0\\
33.87	0\\
33.88	0\\
33.89	0\\
33.9	0\\
33.91	0\\
33.92	0\\
33.93	0\\
33.94	0\\
33.95	0\\
33.96	0\\
33.97	0\\
33.98	0\\
33.99	0\\
34	0\\
34.01	0\\
34.02	0\\
34.03	0\\
34.04	0\\
34.05	0\\
34.06	0\\
34.07	0\\
34.08	0\\
34.09	0\\
34.1	0\\
34.11	0\\
34.12	0\\
34.13	0\\
34.14	0\\
34.15	0\\
34.16	0\\
34.17	0\\
34.18	0\\
34.19	0\\
34.2	0\\
34.21	0\\
34.22	0\\
34.23	0\\
34.24	0\\
34.25	0\\
34.26	0\\
34.27	0\\
34.28	0\\
34.29	0\\
34.3	0\\
34.31	0\\
34.32	0\\
34.33	0\\
34.34	0\\
34.35	0\\
34.36	0\\
34.37	0\\
34.38	0\\
34.39	0\\
34.4	0\\
34.41	0\\
34.42	0\\
34.43	0\\
34.44	0\\
34.45	0\\
34.46	0\\
34.47	0\\
34.48	0\\
34.49	0\\
34.5	0\\
34.51	0\\
34.52	0\\
34.53	0\\
34.54	0\\
34.55	0\\
34.56	0\\
34.57	0\\
34.58	0\\
34.59	0\\
34.6	0\\
34.61	0\\
34.62	0\\
34.63	0\\
34.64	0\\
34.65	0\\
34.66	0\\
34.67	0\\
34.68	0\\
34.69	0\\
34.7	0\\
34.71	0\\
34.72	0\\
34.73	0\\
34.74	0\\
34.75	0\\
34.76	0\\
34.77	0\\
34.78	0\\
34.79	0\\
34.8	0\\
34.81	0\\
34.82	0\\
34.83	0\\
34.84	0\\
34.85	0\\
34.86	0\\
34.87	0\\
34.88	0\\
34.89	0\\
34.9	0\\
34.91	0\\
34.92	0\\
34.93	0\\
34.94	0\\
34.95	0\\
34.96	0\\
34.97	0\\
34.98	0\\
34.99	0\\
35	0\\
35.01	0\\
35.02	0\\
35.03	0\\
35.04	0\\
35.05	0\\
35.06	0\\
35.07	0\\
35.08	0\\
35.09	0\\
35.1	0\\
35.11	0\\
35.12	0\\
35.13	0\\
35.14	0\\
35.15	0\\
35.16	0\\
35.17	0\\
35.18	0\\
35.19	0\\
35.2	0\\
35.21	0\\
35.22	0\\
35.23	0\\
35.24	0\\
35.25	0\\
35.26	0\\
35.27	0\\
35.28	0\\
35.29	0\\
35.3	0\\
35.31	0\\
35.32	0\\
35.33	0\\
35.34	0\\
35.35	0\\
35.36	0\\
35.37	0\\
35.38	0\\
35.39	0\\
35.4	0\\
35.41	0\\
35.42	0\\
35.43	0\\
35.44	0\\
35.45	0\\
35.46	0\\
35.47	0\\
35.48	0\\
35.49	0\\
35.5	0\\
35.51	0\\
35.52	0\\
35.53	0\\
35.54	0\\
35.55	0\\
35.56	0\\
35.57	0\\
35.58	0\\
35.59	0\\
35.6	0\\
35.61	0\\
35.62	0\\
35.63	0\\
35.64	0\\
35.65	0\\
35.66	0\\
35.67	0\\
35.68	0\\
35.69	0\\
35.7	0\\
35.71	0\\
35.72	0\\
35.73	0\\
35.74	0\\
35.75	0\\
35.76	0\\
35.77	0\\
35.78	0\\
35.79	0\\
35.8	0\\
35.81	0\\
35.82	0\\
35.83	0\\
35.84	0\\
35.85	0\\
35.86	0\\
35.87	0\\
35.88	0\\
35.89	0\\
35.9	0\\
35.91	0\\
35.92	0\\
35.93	0\\
35.94	0\\
35.95	0\\
35.96	0\\
35.97	0\\
35.98	0\\
35.99	0\\
36	0\\
36.01	0\\
36.02	0\\
36.03	0\\
36.04	0\\
36.05	0\\
36.06	0\\
36.07	0\\
36.08	0\\
36.09	0\\
36.1	0\\
36.11	0\\
36.12	0\\
36.13	0\\
36.14	0\\
36.15	0\\
36.16	0\\
36.17	0\\
36.18	0\\
36.19	0\\
36.2	0\\
36.21	0\\
36.22	0\\
36.23	0\\
36.24	0\\
36.25	0\\
36.26	0\\
36.27	0\\
36.28	0\\
36.29	0\\
36.3	0\\
36.31	0\\
36.32	0\\
36.33	0\\
36.34	0\\
36.35	0\\
36.36	0\\
36.37	0\\
36.38	0\\
36.39	0\\
36.4	0\\
36.41	0\\
36.42	0\\
36.43	0\\
36.44	0\\
36.45	0\\
36.46	0\\
36.47	0\\
36.48	0\\
36.49	0\\
36.5	0\\
36.51	0\\
36.52	0\\
36.53	0\\
36.54	0\\
36.55	0\\
36.56	0\\
36.57	0\\
36.58	0\\
36.59	0\\
36.6	0\\
36.61	0\\
36.62	0\\
36.63	0\\
36.64	0\\
36.65	0\\
36.66	0\\
36.67	0\\
36.68	0\\
36.69	0\\
36.7	0\\
36.71	0\\
36.72	0\\
36.73	0\\
36.74	0\\
36.75	0\\
36.76	0\\
36.77	0\\
36.78	0\\
36.79	0\\
36.8	0\\
36.81	0\\
36.82	0\\
36.83	0\\
36.84	0\\
36.85	0\\
36.86	0\\
36.87	0\\
36.88	0\\
36.89	0\\
36.9	0\\
36.91	0\\
36.92	0\\
36.93	0\\
36.94	0\\
36.95	0\\
36.96	0\\
36.97	0\\
36.98	0\\
36.99	0\\
37	0\\
37.01	0\\
37.02	0\\
37.03	0\\
37.04	0\\
37.05	0\\
37.06	0\\
37.07	0\\
37.08	0\\
37.09	0\\
37.1	0\\
37.11	0\\
37.12	0\\
37.13	0\\
37.14	0\\
37.15	0\\
37.16	0\\
37.17	0\\
37.18	0\\
37.19	0\\
37.2	0\\
37.21	0\\
37.22	0\\
37.23	0\\
37.24	0\\
37.25	0\\
37.26	0\\
37.27	0\\
37.28	0\\
37.29	0\\
37.3	0\\
37.31	0\\
37.32	0\\
37.33	0\\
37.34	0\\
37.35	0\\
37.36	0\\
37.37	0\\
37.38	0\\
37.39	0\\
37.4	0\\
37.41	0\\
37.42	0\\
37.43	0\\
37.44	0\\
37.45	0\\
37.46	0\\
37.47	0\\
37.48	0\\
37.49	0\\
37.5	0\\
37.51	0\\
37.52	0\\
37.53	0\\
37.54	0\\
37.55	0\\
37.56	0\\
37.57	0\\
37.58	0\\
37.59	0\\
37.6	0\\
37.61	0\\
37.62	0\\
37.63	0\\
37.64	0\\
37.65	0\\
37.66	0\\
37.67	0\\
37.68	0\\
37.69	0\\
37.7	0\\
37.71	0\\
37.72	0\\
37.73	0\\
37.74	0\\
37.75	0\\
37.76	0\\
37.77	0\\
37.78	0\\
37.79	0\\
37.8	0\\
37.81	0\\
37.82	0\\
37.83	0\\
37.84	0\\
37.85	0\\
37.86	0\\
37.87	0\\
37.88	0\\
37.89	0\\
37.9	0\\
37.91	0\\
37.92	0\\
37.93	0\\
37.94	0\\
37.95	0\\
37.96	0\\
37.97	0\\
37.98	0\\
37.99	0\\
38	0\\
38.01	0\\
38.02	0\\
38.03	0\\
38.04	0\\
38.05	0\\
38.06	0\\
38.07	0\\
38.08	0\\
38.09	0\\
38.1	0\\
38.11	0\\
38.12	0\\
38.13	0\\
38.14	0\\
38.15	0\\
38.16	0\\
38.17	0\\
38.18	0\\
38.19	0\\
38.2	0\\
38.21	0\\
38.22	0\\
38.23	0\\
38.24	0\\
38.25	0\\
38.26	0\\
38.27	0\\
38.28	0\\
38.29	0\\
38.3	0\\
38.31	0\\
38.32	0\\
38.33	0\\
38.34	0\\
38.35	0\\
38.36	0\\
38.37	0\\
38.38	0\\
38.39	0\\
38.4	0\\
38.41	0\\
38.42	0\\
38.43	0\\
38.44	0\\
38.45	0\\
38.46	0\\
38.47	0\\
38.48	0\\
38.49	0\\
38.5	0\\
38.51	0\\
38.52	0\\
38.53	0\\
38.54	0\\
38.55	0\\
38.56	0\\
38.57	0\\
38.58	0\\
38.59	0\\
38.6	0\\
38.61	0\\
38.62	0\\
38.63	0\\
38.64	0\\
38.65	0\\
38.66	0\\
38.67	0\\
38.68	0\\
38.69	0\\
38.7	0\\
38.71	0\\
38.72	0\\
38.73	0\\
38.74	0\\
38.75	0\\
38.76	0\\
38.77	0\\
38.78	0\\
38.79	0\\
38.8	0\\
38.81	0\\
38.82	0\\
38.83	0\\
38.84	0\\
38.85	0\\
38.86	0\\
38.87	0\\
38.88	0\\
38.89	0\\
38.9	0\\
38.91	0\\
38.92	0\\
38.93	0\\
38.94	0\\
38.95	0\\
38.96	0\\
38.97	0\\
38.98	0\\
38.99	0\\
39	0\\
39.01	0\\
39.02	0\\
39.03	0\\
39.04	0\\
39.05	0\\
39.06	0\\
39.07	0\\
39.08	0\\
39.09	0\\
39.1	0\\
39.11	0\\
39.12	0\\
39.13	0\\
39.14	0\\
39.15	0\\
39.16	0\\
39.17	0\\
39.18	0\\
39.19	0\\
39.2	0\\
39.21	0\\
39.22	0\\
39.23	0\\
39.24	0\\
39.25	0\\
39.26	0\\
39.27	0\\
39.28	0\\
39.29	0\\
39.3	0\\
39.31	0\\
39.32	0\\
39.33	0\\
39.34	0\\
39.35	0\\
39.36	0\\
39.37	0\\
39.38	0\\
39.39	0\\
39.4	0\\
39.41	0\\
39.42	0\\
39.43	0\\
39.44	0\\
39.45	0\\
39.46	0\\
39.47	0\\
39.48	0\\
39.49	0\\
39.5	0\\
39.51	0\\
39.52	0\\
39.53	0\\
39.54	0\\
39.55	0\\
39.56	0\\
39.57	0\\
39.58	0\\
39.59	0\\
39.6	0\\
39.61	0\\
39.62	0\\
39.63	0\\
39.64	0\\
39.65	0\\
39.66	0\\
39.67	0\\
39.68	0\\
39.69	0\\
39.7	0\\
39.71	0\\
39.72	0\\
39.73	0\\
39.74	0\\
39.75	0\\
39.76	0\\
39.77	0\\
39.78	0\\
39.79	0\\
39.8	0\\
39.81	0\\
39.82	0\\
39.83	0\\
39.84	0\\
39.85	0\\
39.86	0\\
39.87	0\\
39.88	0\\
39.89	0\\
39.9	0\\
39.91	0\\
39.92	0\\
39.93	0\\
39.94	0\\
39.95	0\\
39.96	0\\
39.97	0\\
39.98	0\\
39.99	0\\
40	0\\
40.01	0\\
};
\addplot [color=blue,solid,forget plot]
  table[row sep=crcr]{%
40.01	0\\
40.02	0\\
40.03	0\\
40.04	0\\
40.05	0\\
40.06	0\\
40.07	0\\
40.08	0\\
40.09	0\\
40.1	0\\
40.11	0\\
40.12	0\\
40.13	0\\
40.14	0\\
40.15	0\\
40.16	0\\
40.17	0\\
40.18	0\\
40.19	0\\
40.2	0\\
40.21	0\\
40.22	0\\
40.23	0\\
40.24	0\\
40.25	0\\
40.26	0\\
40.27	0\\
40.28	0\\
40.29	0\\
40.3	0\\
40.31	0\\
40.32	0\\
40.33	0\\
40.34	0\\
40.35	0\\
40.36	0\\
40.37	0\\
40.38	0\\
40.39	0\\
40.4	0\\
40.41	0\\
40.42	0\\
40.43	0\\
40.44	0\\
40.45	0\\
40.46	0\\
40.47	0\\
40.48	0\\
40.49	0\\
40.5	0\\
40.51	0\\
40.52	0\\
40.53	0\\
40.54	0\\
40.55	0\\
40.56	0\\
40.57	0\\
40.58	0\\
40.59	0\\
40.6	0\\
40.61	0\\
40.62	0\\
40.63	0\\
40.64	0\\
40.65	0\\
40.66	0\\
40.67	0\\
40.68	0\\
40.69	0\\
40.7	0\\
40.71	0\\
40.72	0\\
40.73	0\\
40.74	0\\
40.75	0\\
40.76	0\\
40.77	0\\
40.78	0\\
40.79	0\\
40.8	0\\
40.81	0\\
40.82	0\\
40.83	0\\
40.84	0\\
40.85	0\\
40.86	0\\
40.87	0\\
40.88	0\\
40.89	0\\
40.9	0\\
40.91	0\\
40.92	0\\
40.93	0\\
40.94	0\\
40.95	0\\
40.96	0\\
40.97	0\\
40.98	0\\
40.99	0\\
41	0\\
41.01	0\\
41.02	0\\
41.03	0\\
41.04	0\\
41.05	0\\
41.06	0\\
41.07	0\\
41.08	0\\
41.09	0\\
41.1	0\\
41.11	0\\
41.12	0\\
41.13	0\\
41.14	0\\
41.15	0\\
41.16	0\\
41.17	0\\
41.18	0\\
41.19	0\\
41.2	0\\
41.21	0\\
41.22	0\\
41.23	0\\
41.24	0\\
41.25	0\\
41.26	0\\
41.27	0\\
41.28	0\\
41.29	0\\
41.3	0\\
41.31	0\\
41.32	0\\
41.33	0\\
41.34	0\\
41.35	0\\
41.36	0\\
41.37	0\\
41.38	0\\
41.39	0\\
41.4	0\\
41.41	0\\
41.42	0\\
41.43	0\\
41.44	0\\
41.45	0\\
41.46	0\\
41.47	0\\
41.48	0\\
41.49	0\\
41.5	0\\
41.51	0\\
41.52	0\\
41.53	0\\
41.54	0\\
41.55	0\\
41.56	0\\
41.57	0\\
41.58	0\\
41.59	0\\
41.6	0\\
41.61	0\\
41.62	0\\
41.63	0\\
41.64	0\\
41.65	0\\
41.66	0\\
41.67	0\\
41.68	0\\
41.69	0\\
41.7	0\\
41.71	0\\
41.72	0\\
41.73	0\\
41.74	0\\
41.75	0\\
41.76	0\\
41.77	0\\
41.78	0\\
41.79	0\\
41.8	0\\
41.81	0\\
41.82	0\\
41.83	0\\
41.84	0\\
41.85	0\\
41.86	0\\
41.87	0\\
41.88	0\\
41.89	0\\
41.9	0\\
41.91	0\\
41.92	0\\
41.93	0\\
41.94	0\\
41.95	0\\
41.96	0\\
41.97	0\\
41.98	0\\
41.99	0\\
42	0\\
42.01	0\\
42.02	0\\
42.03	0\\
42.04	0\\
42.05	0\\
42.06	0\\
42.07	0\\
42.08	0\\
42.09	0\\
42.1	0\\
42.11	0\\
42.12	0\\
42.13	0\\
42.14	0\\
42.15	0\\
42.16	0\\
42.17	0\\
42.18	0\\
42.19	0\\
42.2	0\\
42.21	0\\
42.22	0\\
42.23	0\\
42.24	0\\
42.25	0\\
42.26	0\\
42.27	0\\
42.28	0\\
42.29	0\\
42.3	0\\
42.31	0\\
42.32	0\\
42.33	0\\
42.34	0\\
42.35	0\\
42.36	0\\
42.37	0\\
42.38	0\\
42.39	0\\
42.4	0\\
42.41	0\\
42.42	0\\
42.43	0\\
42.44	0\\
42.45	0\\
42.46	0\\
42.47	0\\
42.48	0\\
42.49	0\\
42.5	0\\
42.51	0\\
42.52	0\\
42.53	0\\
42.54	0\\
42.55	0\\
42.56	0\\
42.57	0\\
42.58	0\\
42.59	0\\
42.6	0\\
42.61	0\\
42.62	0\\
42.63	0\\
42.64	0\\
42.65	0\\
42.66	0\\
42.67	0\\
42.68	0\\
42.69	0\\
42.7	0\\
42.71	0\\
42.72	0\\
42.73	0\\
42.74	0\\
42.75	0\\
42.76	0\\
42.77	0\\
42.78	0\\
42.79	0\\
42.8	0\\
42.81	0\\
42.82	0\\
42.83	0\\
42.84	0\\
42.85	0\\
42.86	0\\
42.87	0\\
42.88	0\\
42.89	0\\
42.9	0\\
42.91	0\\
42.92	0\\
42.93	0\\
42.94	0\\
42.95	0\\
42.96	0\\
42.97	0\\
42.98	0\\
42.99	0\\
43	0\\
43.01	0\\
43.02	0\\
43.03	0\\
43.04	0\\
43.05	0\\
43.06	0\\
43.07	0\\
43.08	0\\
43.09	0\\
43.1	0\\
43.11	0\\
43.12	0\\
43.13	0\\
43.14	0\\
43.15	0\\
43.16	0\\
43.17	0\\
43.18	0\\
43.19	0\\
43.2	0\\
43.21	0\\
43.22	0\\
43.23	0\\
43.24	0\\
43.25	0\\
43.26	0\\
43.27	0\\
43.28	0\\
43.29	0\\
43.3	0\\
43.31	0\\
43.32	0\\
43.33	0\\
43.34	0\\
43.35	0\\
43.36	0\\
43.37	0\\
43.38	0\\
43.39	0\\
43.4	0\\
43.41	0\\
43.42	0\\
43.43	0\\
43.44	0\\
43.45	0\\
43.46	0\\
43.47	0\\
43.48	0\\
43.49	0\\
43.5	0\\
43.51	0\\
43.52	0\\
43.53	0\\
43.54	0\\
43.55	0\\
43.56	0\\
43.57	0\\
43.58	0\\
43.59	0\\
43.6	0\\
43.61	0\\
43.62	0\\
43.63	0\\
43.64	0\\
43.65	0\\
43.66	0\\
43.67	0\\
43.68	0\\
43.69	0\\
43.7	0\\
43.71	0\\
43.72	0\\
43.73	0\\
43.74	0\\
43.75	0\\
43.76	0\\
43.77	0\\
43.78	0\\
43.79	0\\
43.8	0\\
43.81	0\\
43.82	0\\
43.83	0\\
43.84	0\\
43.85	0\\
43.86	0\\
43.87	0\\
43.88	0\\
43.89	0\\
43.9	0\\
43.91	0\\
43.92	0\\
43.93	0\\
43.94	0\\
43.95	0\\
43.96	0\\
43.97	0\\
43.98	0\\
43.99	0\\
44	0\\
44.01	0\\
44.02	0\\
44.03	0\\
44.04	0\\
44.05	0\\
44.06	0\\
44.07	0\\
44.08	0\\
44.09	0\\
44.1	0\\
44.11	0\\
44.12	0\\
44.13	0\\
44.14	0\\
44.15	0\\
44.16	0\\
44.17	0\\
44.18	0\\
44.19	0\\
44.2	0\\
44.21	0\\
44.22	0\\
44.23	0\\
44.24	0\\
44.25	0\\
44.26	0\\
44.27	0\\
44.28	0\\
44.29	0\\
44.3	0\\
44.31	0\\
44.32	0\\
44.33	0\\
44.34	0\\
44.35	0\\
44.36	0\\
44.37	0\\
44.38	0\\
44.39	0\\
44.4	0\\
44.41	0\\
44.42	0\\
44.43	0\\
44.44	0\\
44.45	0\\
44.46	0\\
44.47	0\\
44.48	0\\
44.49	0\\
44.5	0\\
44.51	0\\
44.52	0\\
44.53	0\\
44.54	0\\
44.55	0\\
44.56	0\\
44.57	0\\
44.58	0\\
44.59	0\\
44.6	0\\
44.61	0\\
44.62	0\\
44.63	0\\
44.64	0\\
44.65	0\\
44.66	0\\
44.67	0\\
44.68	0\\
44.69	0\\
44.7	0\\
44.71	0\\
44.72	0\\
44.73	0\\
44.74	0\\
44.75	0\\
44.76	0\\
44.77	0\\
44.78	0\\
44.79	0\\
44.8	0\\
44.81	0\\
44.82	0\\
44.83	0\\
44.84	0\\
44.85	0\\
44.86	0\\
44.87	0\\
44.88	0\\
44.89	0\\
44.9	0\\
44.91	0\\
44.92	0\\
44.93	0\\
44.94	0\\
44.95	0\\
44.96	0\\
44.97	0\\
44.98	0\\
44.99	0\\
45	0\\
45.01	0\\
45.02	0\\
45.03	0\\
45.04	0\\
45.05	0\\
45.06	0\\
45.07	0\\
45.08	0\\
45.09	0\\
45.1	0\\
45.11	0\\
45.12	0\\
45.13	0\\
45.14	0\\
45.15	0\\
45.16	0\\
45.17	0\\
45.18	0\\
45.19	0\\
45.2	0\\
45.21	0\\
45.22	0\\
45.23	0\\
45.24	0\\
45.25	0\\
45.26	0\\
45.27	0\\
45.28	0\\
45.29	0\\
45.3	0\\
45.31	0\\
45.32	0\\
45.33	0\\
45.34	0\\
45.35	0\\
45.36	0\\
45.37	0\\
45.38	0\\
45.39	0\\
45.4	0\\
45.41	0\\
45.42	0\\
45.43	0\\
45.44	0\\
45.45	0\\
45.46	0\\
45.47	0\\
45.48	0\\
45.49	0\\
45.5	0\\
45.51	0\\
45.52	0\\
45.53	0\\
45.54	0\\
45.55	0\\
45.56	0\\
45.57	0\\
45.58	0\\
45.59	0\\
45.6	0\\
45.61	0\\
45.62	0\\
45.63	0\\
45.64	0\\
45.65	0\\
45.66	0\\
45.67	0\\
45.68	0\\
45.69	0\\
45.7	0\\
45.71	0\\
45.72	0\\
45.73	0\\
45.74	0\\
45.75	0\\
45.76	0\\
45.77	0\\
45.78	0\\
45.79	0\\
45.8	0\\
45.81	0\\
45.82	0\\
45.83	0\\
45.84	0\\
45.85	0\\
45.86	0\\
45.87	0\\
45.88	0\\
45.89	0\\
45.9	0\\
45.91	0\\
45.92	0\\
45.93	0\\
45.94	0\\
45.95	0\\
45.96	0\\
45.97	0\\
45.98	0\\
45.99	0\\
46	0\\
46.01	0\\
46.02	0\\
46.03	0\\
46.04	0\\
46.05	0\\
46.06	0\\
46.07	0\\
46.08	0\\
46.09	0\\
46.1	0\\
46.11	0\\
46.12	0\\
46.13	0\\
46.14	0\\
46.15	0\\
46.16	0\\
46.17	0\\
46.18	0\\
46.19	0\\
46.2	0\\
46.21	0\\
46.22	0\\
46.23	0\\
46.24	0\\
46.25	0\\
46.26	0\\
46.27	0\\
46.28	0\\
46.29	0\\
46.3	0\\
46.31	0\\
46.32	0\\
46.33	0\\
46.34	0\\
46.35	0\\
46.36	0\\
46.37	0\\
46.38	0\\
46.39	0\\
46.4	0\\
46.41	0\\
46.42	0\\
46.43	0\\
46.44	0\\
46.45	0\\
46.46	0\\
46.47	0\\
46.48	0\\
46.49	0\\
46.5	0\\
46.51	0\\
46.52	0\\
46.53	0\\
46.54	0\\
46.55	0\\
46.56	0\\
46.57	0\\
46.58	0\\
46.59	0\\
46.6	0\\
46.61	0\\
46.62	0\\
46.63	0\\
46.64	0\\
46.65	0\\
46.66	0\\
46.67	0\\
46.68	0\\
46.69	0\\
46.7	0\\
46.71	0\\
46.72	0\\
46.73	0\\
46.74	0\\
46.75	0\\
46.76	0\\
46.77	0\\
46.78	0\\
46.79	0\\
46.8	0\\
46.81	0\\
46.82	0\\
46.83	0\\
46.84	0\\
46.85	0\\
46.86	0\\
46.87	0\\
46.88	0\\
46.89	0\\
46.9	0\\
46.91	0\\
46.92	0\\
46.93	0\\
46.94	0\\
46.95	0\\
46.96	0\\
46.97	0\\
46.98	0\\
46.99	0\\
47	0\\
47.01	0\\
47.02	0\\
47.03	0\\
47.04	0\\
47.05	0\\
47.06	0\\
47.07	0\\
47.08	0\\
47.09	0\\
47.1	0\\
47.11	0\\
47.12	0\\
47.13	0\\
47.14	0\\
47.15	0\\
47.16	0\\
47.17	0\\
47.18	0\\
47.19	0\\
47.2	0\\
47.21	0\\
47.22	0\\
47.23	0\\
47.24	0\\
47.25	0\\
47.26	0\\
47.27	0\\
47.28	0\\
47.29	0\\
47.3	0\\
47.31	0\\
47.32	0\\
47.33	0\\
47.34	0\\
47.35	0\\
47.36	0\\
47.37	0\\
47.38	0\\
47.39	0\\
47.4	0\\
47.41	0\\
47.42	0\\
47.43	0\\
47.44	0\\
47.45	0\\
47.46	0\\
47.47	0\\
47.48	0\\
47.49	0\\
47.5	0\\
47.51	0\\
47.52	0\\
47.53	0\\
47.54	0\\
47.55	0\\
47.56	0\\
47.57	0\\
47.58	0\\
47.59	0\\
47.6	0\\
47.61	0\\
47.62	0\\
47.63	0\\
47.64	0\\
47.65	0\\
47.66	0\\
47.67	0\\
47.68	0\\
47.69	0\\
47.7	0\\
47.71	0\\
47.72	0\\
47.73	0\\
47.74	0\\
47.75	0\\
47.76	0\\
47.77	0\\
47.78	0\\
47.79	0\\
47.8	0\\
47.81	0\\
47.82	0\\
47.83	0\\
47.84	0\\
47.85	0\\
47.86	0\\
47.87	0\\
47.88	0\\
47.89	0\\
47.9	0\\
47.91	0\\
47.92	0\\
47.93	0\\
47.94	0\\
47.95	0\\
47.96	0\\
47.97	0\\
47.98	0\\
47.99	0\\
48	0\\
48.01	0\\
48.02	0\\
48.03	0\\
48.04	0\\
48.05	0\\
48.06	0\\
48.07	0\\
48.08	0\\
48.09	0\\
48.1	0\\
48.11	0\\
48.12	0\\
48.13	0\\
48.14	0\\
48.15	0\\
48.16	0\\
48.17	0\\
48.18	0\\
48.19	0\\
48.2	0\\
48.21	0\\
48.22	0\\
48.23	0\\
48.24	0\\
48.25	0\\
48.26	0\\
48.27	0\\
48.28	0\\
48.29	0\\
48.3	0\\
48.31	0\\
48.32	0\\
48.33	0\\
48.34	0\\
48.35	0\\
48.36	0\\
48.37	0\\
48.38	0\\
48.39	0\\
48.4	0\\
48.41	0\\
48.42	0\\
48.43	0\\
48.44	0\\
48.45	0\\
48.46	0\\
48.47	0\\
48.48	0\\
48.49	0\\
48.5	0\\
48.51	0\\
48.52	0\\
48.53	0\\
48.54	0\\
48.55	0\\
48.56	0\\
48.57	0\\
48.58	0\\
48.59	0\\
48.6	0\\
48.61	0\\
48.62	0\\
48.63	0\\
48.64	0\\
48.65	0\\
48.66	0\\
48.67	0\\
48.68	0\\
48.69	0\\
48.7	0\\
48.71	0\\
48.72	0\\
48.73	0\\
48.74	0\\
48.75	0\\
48.76	0\\
48.77	0\\
48.78	0\\
48.79	0\\
48.8	0\\
48.81	0\\
48.82	0\\
48.83	0\\
48.84	0\\
48.85	0\\
48.86	0\\
48.87	0\\
48.88	0\\
48.89	0\\
48.9	0\\
48.91	0\\
48.92	0\\
48.93	0\\
48.94	0\\
48.95	0\\
48.96	0\\
48.97	0\\
48.98	0\\
48.99	0\\
49	0\\
49.01	0\\
49.02	0\\
49.03	0\\
49.04	0\\
49.05	0\\
49.06	0\\
49.07	0\\
49.08	0\\
49.09	0\\
49.1	0\\
49.11	0\\
49.12	0\\
49.13	0\\
49.14	0\\
49.15	0\\
49.16	0\\
49.17	0\\
49.18	0\\
49.19	0\\
49.2	0\\
49.21	0\\
49.22	0\\
49.23	0\\
49.24	0\\
49.25	0\\
49.26	0\\
49.27	0\\
49.28	0\\
49.29	0\\
49.3	0\\
49.31	0\\
49.32	0\\
49.33	0\\
49.34	0\\
49.35	0\\
49.36	0\\
49.37	0\\
49.38	0\\
49.39	0\\
49.4	0\\
49.41	0\\
49.42	0\\
49.43	0\\
49.44	0\\
49.45	0\\
49.46	0\\
49.47	0\\
49.48	0\\
49.49	0\\
49.5	0\\
49.51	0\\
49.52	0\\
49.53	0\\
49.54	0\\
49.55	0\\
49.56	0\\
49.57	0\\
49.58	0\\
49.59	0\\
49.6	0\\
49.61	0\\
49.62	0\\
49.63	0\\
49.64	0\\
49.65	0\\
49.66	0\\
49.67	0\\
49.68	0\\
49.69	0\\
49.7	0\\
49.71	0\\
49.72	0\\
49.73	0\\
49.74	0\\
49.75	0\\
49.76	0\\
49.77	0\\
49.78	0\\
49.79	0\\
49.8	0\\
49.81	0\\
49.82	0\\
49.83	0\\
49.84	0\\
49.85	0\\
49.86	0\\
49.87	0\\
49.88	0\\
49.89	0\\
49.9	0\\
49.91	0\\
49.92	0\\
49.93	0\\
49.94	0\\
49.95	0\\
49.96	0\\
49.97	0\\
49.98	0\\
49.99	0\\
50	0\\
50.01	0\\
50.02	0\\
50.03	0\\
50.04	0\\
50.05	0\\
50.06	0\\
50.07	0\\
50.08	0\\
50.09	0\\
50.1	0\\
50.11	0\\
50.12	0\\
50.13	0\\
50.14	0\\
50.15	0\\
50.16	0\\
50.17	0\\
50.18	0\\
50.19	0\\
50.2	0\\
50.21	0\\
50.22	0\\
50.23	0\\
50.24	0\\
50.25	0\\
50.26	0\\
50.27	0\\
50.28	0\\
50.29	0\\
50.3	0\\
50.31	0\\
50.32	0\\
50.33	0\\
50.34	0\\
50.35	0\\
50.36	0\\
50.37	0\\
50.38	0\\
50.39	0\\
50.4	0\\
50.41	0\\
50.42	0\\
50.43	0\\
50.44	0\\
50.45	0\\
50.46	0\\
50.47	0\\
50.48	0\\
50.49	0\\
50.5	0\\
50.51	0\\
50.52	0\\
50.53	0\\
50.54	0\\
50.55	0\\
50.56	0\\
50.57	0\\
50.58	0\\
50.59	0\\
50.6	0\\
50.61	0\\
50.62	0\\
50.63	0\\
50.64	0\\
50.65	0\\
50.66	0\\
50.67	0\\
50.68	0\\
50.69	0\\
50.7	0\\
50.71	0\\
50.72	0\\
50.73	0\\
50.74	0\\
50.75	0\\
50.76	0\\
50.77	0\\
50.78	0\\
50.79	0\\
50.8	0\\
50.81	0\\
50.82	0\\
50.83	0\\
50.84	0\\
50.85	0\\
50.86	0\\
50.87	0\\
50.88	0\\
50.89	0\\
50.9	0\\
50.91	0\\
50.92	0\\
50.93	0\\
50.94	0\\
50.95	0\\
50.96	0\\
50.97	0\\
50.98	0\\
50.99	0\\
51	0\\
51.01	0\\
51.02	0\\
51.03	0\\
51.04	0\\
51.05	0\\
51.06	0\\
51.07	0\\
51.08	0\\
51.09	0\\
51.1	0\\
51.11	0\\
51.12	0\\
51.13	0\\
51.14	0\\
51.15	0\\
51.16	0\\
51.17	0\\
51.18	0\\
51.19	0\\
51.2	0\\
51.21	0\\
51.22	0\\
51.23	0\\
51.24	0\\
51.25	0\\
51.26	0\\
51.27	0\\
51.28	0\\
51.29	0\\
51.3	0\\
51.31	0\\
51.32	0\\
51.33	0\\
51.34	0\\
51.35	0\\
51.36	0\\
51.37	0\\
51.38	0\\
51.39	0\\
51.4	0\\
51.41	0\\
51.42	0\\
51.43	0\\
51.44	0\\
51.45	0\\
51.46	0\\
51.47	0\\
51.48	0\\
51.49	0\\
51.5	0\\
51.51	0\\
51.52	0\\
51.53	0\\
51.54	0\\
51.55	0\\
51.56	0\\
51.57	0\\
51.58	0\\
51.59	0\\
51.6	0\\
51.61	0\\
51.62	0\\
51.63	0\\
51.64	0\\
51.65	0\\
51.66	0\\
51.67	0\\
51.68	0\\
51.69	0\\
51.7	0\\
51.71	0\\
51.72	0\\
51.73	0\\
51.74	0\\
51.75	0\\
51.76	0\\
51.77	0\\
51.78	0\\
51.79	0\\
51.8	0\\
51.81	0\\
51.82	0\\
51.83	0\\
51.84	0\\
51.85	0\\
51.86	0\\
51.87	0\\
51.88	0\\
51.89	0\\
51.9	0\\
51.91	0\\
51.92	0\\
51.93	0\\
51.94	0\\
51.95	0\\
51.96	0\\
51.97	0\\
51.98	0\\
51.99	0\\
52	0\\
52.01	0\\
52.02	0\\
52.03	0\\
52.04	0\\
52.05	0\\
52.06	0\\
52.07	0\\
52.08	0\\
52.09	0\\
52.1	0\\
52.11	0\\
52.12	0\\
52.13	0\\
52.14	0\\
52.15	0\\
52.16	0\\
52.17	0\\
52.18	0\\
52.19	0\\
52.2	0\\
52.21	0\\
52.22	0\\
52.23	0\\
52.24	0\\
52.25	0\\
52.26	0\\
52.27	0\\
52.28	0\\
52.29	0\\
52.3	0\\
52.31	0\\
52.32	0\\
52.33	0\\
52.34	0\\
52.35	0\\
52.36	0\\
52.37	0\\
52.38	0\\
52.39	0\\
52.4	0\\
52.41	0\\
52.42	0\\
52.43	0\\
52.44	0\\
52.45	0\\
52.46	0\\
52.47	0\\
52.48	0\\
52.49	0\\
52.5	0\\
52.51	0\\
52.52	0\\
52.53	0\\
52.54	0\\
52.55	0\\
52.56	0\\
52.57	0\\
52.58	0\\
52.59	0\\
52.6	0\\
52.61	0\\
52.62	0\\
52.63	0\\
52.64	0\\
52.65	0\\
52.66	0\\
52.67	0\\
52.68	0\\
52.69	0\\
52.7	0\\
52.71	0\\
52.72	0\\
52.73	0\\
52.74	0\\
52.75	0\\
52.76	0\\
52.77	0\\
52.78	0\\
52.79	0\\
52.8	0\\
52.81	0\\
52.82	0\\
52.83	0\\
52.84	0\\
52.85	0\\
52.86	0\\
52.87	0\\
52.88	0\\
52.89	0\\
52.9	0\\
52.91	0\\
52.92	0\\
52.93	0\\
52.94	0\\
52.95	0\\
52.96	0\\
52.97	0\\
52.98	0\\
52.99	0\\
53	0\\
53.01	0\\
53.02	0\\
53.03	0\\
53.04	0\\
53.05	0\\
53.06	0\\
53.07	0\\
53.08	0\\
53.09	0\\
53.1	0\\
53.11	0\\
53.12	0\\
53.13	0\\
53.14	0\\
53.15	0\\
53.16	0\\
53.17	0\\
53.18	0\\
53.19	0\\
53.2	0\\
53.21	0\\
53.22	0\\
53.23	0\\
53.24	0\\
53.25	0\\
53.26	0\\
53.27	0\\
53.28	0\\
53.29	0\\
53.3	0\\
53.31	0\\
53.32	0\\
53.33	0\\
53.34	0\\
53.35	0\\
53.36	0\\
53.37	0\\
53.38	0\\
53.39	0\\
53.4	0\\
53.41	0\\
53.42	0\\
53.43	0\\
53.44	0\\
53.45	0\\
53.46	0\\
53.47	0\\
53.48	0\\
53.49	0\\
53.5	0\\
53.51	0\\
53.52	0\\
53.53	0\\
53.54	0\\
53.55	0\\
53.56	0\\
53.57	0\\
53.58	0\\
53.59	0\\
53.6	0\\
53.61	0\\
53.62	0\\
53.63	0\\
53.64	0\\
53.65	0\\
53.66	0\\
53.67	0\\
53.68	0\\
53.69	0\\
53.7	0\\
53.71	0\\
53.72	0\\
53.73	0\\
53.74	0\\
53.75	0\\
53.76	0\\
53.77	0\\
53.78	0\\
53.79	0\\
53.8	0\\
53.81	0\\
53.82	0\\
53.83	0\\
53.84	0\\
53.85	0\\
53.86	0\\
53.87	0\\
53.88	0\\
53.89	0\\
53.9	0\\
53.91	0\\
53.92	0\\
53.93	0\\
53.94	0\\
53.95	0\\
53.96	0\\
53.97	0\\
53.98	0\\
53.99	0\\
54	0\\
54.01	0\\
54.02	0\\
54.03	0\\
54.04	0\\
54.05	0\\
54.06	0\\
54.07	0\\
54.08	0\\
54.09	0\\
54.1	0\\
54.11	0\\
54.12	0\\
54.13	0\\
54.14	0\\
54.15	0\\
54.16	0\\
54.17	0\\
54.18	0\\
54.19	0\\
54.2	0\\
54.21	0\\
54.22	0\\
54.23	0\\
54.24	0\\
54.25	0\\
54.26	0\\
54.27	0\\
54.28	0\\
54.29	0\\
54.3	0\\
54.31	0\\
54.32	0\\
54.33	0\\
54.34	0\\
54.35	0\\
54.36	0\\
54.37	0\\
54.38	0\\
54.39	0\\
54.4	0\\
54.41	0\\
54.42	0\\
54.43	0\\
54.44	0\\
54.45	0\\
54.46	0\\
54.47	0\\
54.48	0\\
54.49	0\\
54.5	0\\
54.51	0\\
54.52	0\\
54.53	0\\
54.54	0\\
54.55	0\\
54.56	0\\
54.57	0\\
54.58	0\\
54.59	0\\
54.6	0\\
54.61	0\\
54.62	0\\
54.63	0\\
54.64	0\\
54.65	0\\
54.66	0\\
54.67	0\\
54.68	0\\
54.69	0\\
54.7	0\\
54.71	0\\
54.72	0\\
54.73	0\\
54.74	0\\
54.75	0\\
54.76	0\\
54.77	0\\
54.78	0\\
54.79	0\\
54.8	0\\
54.81	0\\
54.82	0\\
54.83	0\\
54.84	0\\
54.85	0\\
54.86	0\\
54.87	0\\
54.88	0\\
54.89	0\\
54.9	0\\
54.91	0\\
54.92	0\\
54.93	0\\
54.94	0\\
54.95	0\\
54.96	0\\
54.97	0\\
54.98	0\\
54.99	0\\
55	0\\
55.01	0\\
55.02	0\\
55.03	0\\
55.04	0\\
55.05	0\\
55.06	0\\
55.07	0\\
55.08	0\\
55.09	0\\
55.1	0\\
55.11	0\\
55.12	0\\
55.13	0\\
55.14	0\\
55.15	0\\
55.16	0\\
55.17	0\\
55.18	0\\
55.19	0\\
55.2	0\\
55.21	0\\
55.22	0\\
55.23	0\\
55.24	0\\
55.25	0\\
55.26	0\\
55.27	0\\
55.28	0\\
55.29	0\\
55.3	0\\
55.31	0\\
55.32	0\\
55.33	0\\
55.34	0\\
55.35	0\\
55.36	0\\
55.37	0\\
55.38	0\\
55.39	0\\
55.4	0\\
55.41	0\\
55.42	0\\
55.43	0\\
55.44	0\\
55.45	0\\
55.46	0\\
55.47	0\\
55.48	0\\
55.49	0\\
55.5	0\\
55.51	0\\
55.52	0\\
55.53	0\\
55.54	0\\
55.55	0\\
55.56	0\\
55.57	0\\
55.58	0\\
55.59	0\\
55.6	0\\
55.61	0\\
55.62	0\\
55.63	0\\
55.64	0\\
55.65	0\\
55.66	0\\
55.67	0\\
55.68	0\\
55.69	0\\
55.7	0\\
55.71	0\\
55.72	0\\
55.73	0\\
55.74	0\\
55.75	0\\
55.76	0\\
55.77	0\\
55.78	0\\
55.79	0\\
55.8	0\\
55.81	0\\
55.82	0\\
55.83	0\\
55.84	0\\
55.85	0\\
55.86	0\\
55.87	0\\
55.88	0\\
55.89	0\\
55.9	0\\
55.91	0\\
55.92	0\\
55.93	0\\
55.94	0\\
55.95	0\\
55.96	0\\
55.97	0\\
55.98	0\\
55.99	0\\
56	0\\
56.01	0\\
56.02	0\\
56.03	0\\
56.04	0\\
56.05	0\\
56.06	0\\
56.07	0\\
56.08	0\\
56.09	0\\
56.1	0\\
56.11	0\\
56.12	0\\
56.13	0\\
56.14	0\\
56.15	0\\
56.16	0\\
56.17	0\\
56.18	0\\
56.19	0\\
56.2	0\\
56.21	0\\
56.22	0\\
56.23	0\\
56.24	0\\
56.25	0\\
56.26	0\\
56.27	0\\
56.28	0\\
56.29	0\\
56.3	0\\
56.31	0\\
56.32	0\\
56.33	0\\
56.34	0\\
56.35	0\\
56.36	0\\
56.37	0\\
56.38	0\\
56.39	0\\
56.4	0\\
56.41	0\\
56.42	0\\
56.43	0\\
56.44	0\\
56.45	0\\
56.46	0\\
56.47	0\\
56.48	0\\
56.49	0\\
56.5	0\\
56.51	0\\
56.52	0\\
56.53	0\\
56.54	0\\
56.55	0\\
56.56	0\\
56.57	0\\
56.58	0\\
56.59	0\\
56.6	0\\
56.61	0\\
56.62	0\\
56.63	0\\
56.64	0\\
56.65	0\\
56.66	0\\
56.67	0\\
56.68	0\\
56.69	0\\
56.7	0\\
56.71	0\\
56.72	0\\
56.73	0\\
56.74	0\\
56.75	0\\
56.76	0\\
56.77	0\\
56.78	0\\
56.79	0\\
56.8	0\\
56.81	0\\
56.82	0\\
56.83	0\\
56.84	0\\
56.85	0\\
56.86	0\\
56.87	0\\
56.88	0\\
56.89	0\\
56.9	0\\
56.91	0\\
56.92	0\\
56.93	0\\
56.94	0\\
56.95	0\\
56.96	0\\
56.97	0\\
56.98	0\\
56.99	0\\
57	0\\
57.01	0\\
57.02	0\\
57.03	0\\
57.04	0\\
57.05	0\\
57.06	0\\
57.07	0\\
57.08	0\\
57.09	0\\
57.1	0\\
57.11	0\\
57.12	0\\
57.13	0\\
57.14	0\\
57.15	0\\
57.16	0\\
57.17	0\\
57.18	0\\
57.19	0\\
57.2	0\\
57.21	0\\
57.22	0\\
57.23	0\\
57.24	0\\
57.25	0\\
57.26	0\\
57.27	0\\
57.28	0\\
57.29	0\\
57.3	0\\
57.31	0\\
57.32	0\\
57.33	0\\
57.34	0\\
57.35	0\\
57.36	0\\
57.37	0\\
57.38	0\\
57.39	0\\
57.4	0\\
57.41	0\\
57.42	0\\
57.43	0\\
57.44	0\\
57.45	0\\
57.46	0\\
57.47	0\\
57.48	0\\
57.49	0\\
57.5	0\\
57.51	0\\
57.52	0\\
57.53	0\\
57.54	0\\
57.55	0\\
57.56	0\\
57.57	0\\
57.58	0\\
57.59	0\\
57.6	0\\
57.61	0\\
57.62	0\\
57.63	0\\
57.64	0\\
57.65	0\\
57.66	0\\
57.67	0\\
57.68	0\\
57.69	0\\
57.7	0\\
57.71	0\\
57.72	0\\
57.73	0\\
57.74	0\\
57.75	0\\
57.76	0\\
57.77	0\\
57.78	0\\
57.79	0\\
57.8	0\\
57.81	0\\
57.82	0\\
57.83	0\\
57.84	0\\
57.85	0\\
57.86	0\\
57.87	0\\
57.88	0\\
57.89	0\\
57.9	0\\
57.91	0\\
57.92	0\\
57.93	0\\
57.94	0\\
57.95	0\\
57.96	0\\
57.97	0\\
57.98	0\\
57.99	0\\
58	0\\
58.01	0\\
58.02	0\\
58.03	0\\
58.04	0\\
58.05	0\\
58.06	0\\
58.07	0\\
58.08	0\\
58.09	0\\
58.1	0\\
58.11	0\\
58.12	0\\
58.13	0\\
58.14	0\\
58.15	0\\
58.16	0\\
58.17	0\\
58.18	0\\
58.19	0\\
58.2	0\\
58.21	0\\
58.22	0\\
58.23	0\\
58.24	0\\
58.25	0\\
58.26	0\\
58.27	0\\
58.28	0\\
58.29	0\\
58.3	0\\
58.31	0\\
58.32	0\\
58.33	0\\
58.34	0\\
58.35	0\\
58.36	0\\
58.37	0\\
58.38	0\\
58.39	0\\
58.4	0\\
58.41	0\\
58.42	0\\
58.43	0\\
58.44	0\\
58.45	0\\
58.46	0\\
58.47	0\\
58.48	0\\
58.49	0\\
58.5	0\\
58.51	0\\
58.52	0\\
58.53	0\\
58.54	0\\
58.55	0\\
58.56	0\\
58.57	0\\
58.58	0\\
58.59	0\\
58.6	0\\
58.61	0\\
58.62	0\\
58.63	0\\
58.64	0\\
58.65	0\\
58.66	0\\
58.67	0\\
58.68	0\\
58.69	0\\
58.7	0\\
58.71	0\\
58.72	0\\
58.73	0\\
58.74	0\\
58.75	0\\
58.76	0\\
58.77	0\\
58.78	0\\
58.79	0\\
58.8	0\\
58.81	0\\
58.82	0\\
58.83	0\\
58.84	0\\
58.85	0\\
58.86	0\\
58.87	0\\
58.88	0\\
58.89	0\\
58.9	0\\
58.91	0\\
58.92	0\\
58.93	0\\
58.94	0\\
58.95	0\\
58.96	0\\
58.97	0\\
58.98	0\\
58.99	0\\
59	0\\
59.01	0\\
59.02	0\\
59.03	0\\
59.04	0\\
59.05	0\\
59.06	0\\
59.07	0\\
59.08	0\\
59.09	0\\
59.1	0\\
59.11	0\\
59.12	0\\
59.13	0\\
59.14	0\\
59.15	0\\
59.16	0\\
59.17	0\\
59.18	0\\
59.19	0\\
59.2	0\\
59.21	0\\
59.22	0\\
59.23	0\\
59.24	0\\
59.25	0\\
59.26	0\\
59.27	0\\
59.28	0\\
59.29	0\\
59.3	0\\
59.31	0\\
59.32	0\\
59.33	0\\
59.34	0\\
59.35	0\\
59.36	0\\
59.37	0\\
59.38	0\\
59.39	0\\
59.4	0\\
59.41	0\\
59.42	0\\
59.43	0\\
59.44	0\\
59.45	0\\
59.46	0\\
59.47	0\\
59.48	0\\
59.49	0\\
59.5	0\\
59.51	0\\
59.52	0\\
59.53	0\\
59.54	0\\
59.55	0\\
59.56	0\\
59.57	0\\
59.58	0\\
59.59	0\\
59.6	0\\
59.61	0\\
59.62	0\\
59.63	0\\
59.64	0\\
59.65	0\\
59.66	0\\
59.67	0\\
59.68	0\\
59.69	0\\
59.7	0\\
59.71	0\\
59.72	0\\
59.73	0\\
59.74	0\\
59.75	0\\
59.76	0\\
59.77	0\\
59.78	0\\
59.79	0\\
59.8	0\\
59.81	0\\
59.82	0\\
59.83	0\\
59.84	0\\
59.85	0\\
59.86	0\\
59.87	0\\
59.88	0\\
59.89	0\\
59.9	0\\
59.91	0\\
59.92	0\\
59.93	0\\
59.94	0\\
59.95	0\\
59.96	0\\
59.97	0\\
59.98	0\\
59.99	0\\
60	0\\
60.01	0\\
60.02	0\\
60.03	0\\
60.04	0\\
60.05	0\\
60.06	0\\
60.07	0\\
60.08	0\\
60.09	0\\
60.1	0\\
60.11	0\\
60.12	0\\
60.13	0\\
60.14	0\\
60.15	0\\
60.16	0\\
60.17	0\\
60.18	0\\
60.19	0\\
60.2	0\\
60.21	0\\
60.22	0\\
60.23	0\\
60.24	0\\
60.25	0\\
60.26	0\\
60.27	0\\
60.28	0\\
60.29	0\\
60.3	0\\
60.31	0\\
60.32	0\\
60.33	0\\
60.34	0\\
60.35	0\\
60.36	0\\
60.37	0\\
60.38	0\\
60.39	0\\
60.4	0\\
60.41	0\\
60.42	0\\
60.43	0\\
60.44	0\\
60.45	0\\
60.46	0\\
60.47	0\\
60.48	0\\
60.49	0\\
60.5	0\\
60.51	0\\
60.52	0\\
60.53	0\\
60.54	0\\
60.55	0\\
60.56	0\\
60.57	0\\
60.58	0\\
60.59	0\\
60.6	0\\
60.61	0\\
60.62	0\\
60.63	0\\
60.64	0\\
60.65	0\\
60.66	0\\
60.67	0\\
60.68	0\\
60.69	0\\
60.7	0\\
60.71	0\\
60.72	0\\
60.73	0\\
60.74	0\\
60.75	0\\
60.76	0\\
60.77	0\\
60.78	0\\
60.79	0\\
60.8	0\\
60.81	0\\
60.82	0\\
60.83	0\\
60.84	0\\
60.85	0\\
60.86	0\\
60.87	0\\
60.88	0\\
60.89	0\\
60.9	0\\
60.91	0\\
60.92	0\\
60.93	0\\
60.94	0\\
60.95	0\\
60.96	0\\
60.97	0\\
60.98	0\\
60.99	0\\
61	0\\
61.01	0\\
61.02	0\\
61.03	0\\
61.04	0\\
61.05	0\\
61.06	0\\
61.07	0\\
61.08	0\\
61.09	0\\
61.1	0\\
61.11	0\\
61.12	0\\
61.13	0\\
61.14	0\\
61.15	0\\
61.16	0\\
61.17	0\\
61.18	0\\
61.19	0\\
61.2	0\\
61.21	0\\
61.22	0\\
61.23	0\\
61.24	0\\
61.25	0\\
61.26	0\\
61.27	0\\
61.28	0\\
61.29	0\\
61.3	0\\
61.31	0\\
61.32	0\\
61.33	0\\
61.34	0\\
61.35	0\\
61.36	0\\
61.37	0\\
61.38	0\\
61.39	0\\
61.4	0\\
61.41	0\\
61.42	0\\
61.43	0\\
61.44	0\\
61.45	0\\
61.46	0\\
61.47	0\\
61.48	0\\
61.49	0\\
61.5	0\\
61.51	0\\
61.52	0\\
61.53	0\\
61.54	0\\
61.55	0\\
61.56	0\\
61.57	0\\
61.58	0\\
61.59	0\\
61.6	0\\
61.61	0\\
61.62	0\\
61.63	0\\
61.64	0\\
61.65	0\\
61.66	0\\
61.67	0\\
61.68	0\\
61.69	0\\
61.7	0\\
61.71	0\\
61.72	0\\
61.73	0\\
61.74	0\\
61.75	0\\
61.76	0\\
61.77	0\\
61.78	0\\
61.79	0\\
61.8	0\\
61.81	0\\
61.82	0\\
61.83	0\\
61.84	0\\
61.85	0\\
61.86	0\\
61.87	0\\
61.88	0\\
61.89	0\\
61.9	0\\
61.91	0\\
61.92	0\\
61.93	0\\
61.94	0\\
61.95	0\\
61.96	0\\
61.97	0\\
61.98	0\\
61.99	0\\
62	0\\
62.01	0\\
62.02	0\\
62.03	0\\
62.04	0\\
62.05	0\\
62.06	0\\
62.07	0\\
62.08	0\\
62.09	0\\
62.1	0\\
62.11	0\\
62.12	0\\
62.13	0\\
62.14	0\\
62.15	0\\
62.16	0\\
62.17	0\\
62.18	0\\
62.19	0\\
62.2	0\\
62.21	0\\
62.22	0\\
62.23	0\\
62.24	0\\
62.25	0\\
62.26	0\\
62.27	0\\
62.28	0\\
62.29	0\\
62.3	0\\
62.31	0\\
62.32	0\\
62.33	0\\
62.34	0\\
62.35	0\\
62.36	0\\
62.37	0\\
62.38	0\\
62.39	0\\
62.4	0\\
62.41	0\\
62.42	0\\
62.43	0\\
62.44	0\\
62.45	0\\
62.46	0\\
62.47	0\\
62.48	0\\
62.49	0\\
62.5	0\\
62.51	0\\
62.52	0\\
62.53	0\\
62.54	0\\
62.55	0\\
62.56	0\\
62.57	0\\
62.58	0\\
62.59	0\\
62.6	0\\
62.61	0\\
62.62	0\\
62.63	0\\
62.64	0\\
62.65	0\\
62.66	0\\
62.67	0\\
62.68	0\\
62.69	0\\
62.7	0\\
62.71	0\\
62.72	0\\
62.73	0\\
62.74	0\\
62.75	0\\
62.76	0\\
62.77	0\\
62.78	0\\
62.79	0\\
62.8	0\\
62.81	0\\
62.82	0\\
62.83	0\\
62.84	0\\
62.85	0\\
62.86	0\\
62.87	0\\
62.88	0\\
62.89	0\\
62.9	0\\
62.91	0\\
62.92	0\\
62.93	0\\
62.94	0\\
62.95	0\\
62.96	0\\
62.97	0\\
62.98	0\\
62.99	0\\
63	0\\
63.01	0\\
63.02	0\\
63.03	0\\
63.04	0\\
63.05	0\\
63.06	0\\
63.07	0\\
63.08	0\\
63.09	0\\
63.1	0\\
63.11	0\\
63.12	0\\
63.13	0\\
63.14	0\\
63.15	0\\
63.16	0\\
63.17	0\\
63.18	0\\
63.19	0\\
63.2	0\\
63.21	0\\
63.22	0\\
63.23	0\\
63.24	0\\
63.25	0\\
63.26	0\\
63.27	0\\
63.28	0\\
63.29	0\\
63.3	0\\
63.31	0\\
63.32	0\\
63.33	0\\
63.34	0\\
63.35	0\\
63.36	0\\
63.37	0\\
63.38	0\\
63.39	0\\
63.4	0\\
63.41	0\\
63.42	0\\
63.43	0\\
63.44	0\\
63.45	0\\
63.46	0\\
63.47	0\\
63.48	0\\
63.49	0\\
63.5	0\\
63.51	0\\
63.52	0\\
63.53	0\\
63.54	0\\
63.55	0\\
63.56	0\\
63.57	0\\
63.58	0\\
63.59	0\\
63.6	0\\
63.61	0\\
63.62	0\\
63.63	0\\
63.64	0\\
63.65	0\\
63.66	0\\
63.67	0\\
63.68	0\\
63.69	0\\
63.7	0\\
63.71	0\\
63.72	0\\
63.73	0\\
63.74	0\\
63.75	0\\
63.76	0\\
63.77	0\\
63.78	0\\
63.79	0\\
63.8	0\\
63.81	0\\
63.82	0\\
63.83	0\\
63.84	0\\
63.85	0\\
63.86	0\\
63.87	0\\
63.88	0\\
63.89	0\\
63.9	0\\
63.91	0\\
63.92	0\\
63.93	0\\
63.94	0\\
63.95	0\\
63.96	0\\
63.97	0\\
63.98	0\\
63.99	0\\
64	0\\
64.01	0\\
64.02	0\\
64.03	0\\
64.04	0\\
64.05	0\\
64.06	0\\
64.07	0\\
64.08	0\\
64.09	0\\
64.1	0\\
64.11	0\\
64.12	0\\
64.13	0\\
64.14	0\\
64.15	0\\
64.16	0\\
64.17	0\\
64.18	0\\
64.19	0\\
64.2	0\\
64.21	0\\
64.22	0\\
64.23	0\\
64.24	0\\
64.25	0\\
64.26	0\\
64.27	0\\
64.28	0\\
64.29	0\\
64.3	0\\
64.31	0\\
64.32	0\\
64.33	0\\
64.34	0\\
64.35	0\\
64.36	0\\
64.37	0\\
64.38	0\\
64.39	0\\
64.4	0\\
64.41	0\\
64.42	0\\
64.43	0\\
64.44	0\\
64.45	0\\
64.46	0\\
64.47	0\\
64.48	0\\
64.49	0\\
64.5	0\\
64.51	0\\
64.52	0\\
64.53	0\\
64.54	0\\
64.55	0\\
64.56	0\\
64.57	0\\
64.58	0\\
64.59	0\\
64.6	0\\
64.61	0\\
64.62	0\\
64.63	0\\
64.64	0\\
64.65	0\\
64.66	0\\
64.67	0\\
64.68	0\\
64.69	0\\
64.7	0\\
64.71	0\\
64.72	0\\
64.73	0\\
64.74	0\\
64.75	0\\
64.76	0\\
64.77	0\\
64.78	0\\
64.79	0\\
64.8	0\\
64.81	0\\
64.82	0\\
64.83	0\\
64.84	0\\
64.85	0\\
64.86	0\\
64.87	0\\
64.88	0\\
64.89	0\\
64.9	0\\
64.91	0\\
64.92	0\\
64.93	0\\
64.94	0\\
64.95	0\\
64.96	0\\
64.97	0\\
64.98	0\\
64.99	0\\
65	0\\
65.01	0\\
65.02	0\\
65.03	0\\
65.04	0\\
65.05	0\\
65.06	0\\
65.07	0\\
65.08	0\\
65.09	0\\
65.1	0\\
65.11	0\\
65.12	0\\
65.13	0\\
65.14	0\\
65.15	0\\
65.16	0\\
65.17	0\\
65.18	0\\
65.19	0\\
65.2	0\\
65.21	0\\
65.22	0\\
65.23	0\\
65.24	0\\
65.25	0\\
65.26	0\\
65.27	0\\
65.28	0\\
65.29	0\\
65.3	0\\
65.31	0\\
65.32	0\\
65.33	0\\
65.34	0\\
65.35	0\\
65.36	0\\
65.37	0\\
65.38	0\\
65.39	0\\
65.4	0\\
65.41	0\\
65.42	0\\
65.43	0\\
65.44	0\\
65.45	0\\
65.46	0\\
65.47	0\\
65.48	0\\
65.49	0\\
65.5	0\\
65.51	0\\
65.52	0\\
65.53	0\\
65.54	0\\
65.55	0\\
65.56	0\\
65.57	0\\
65.58	0\\
65.59	0\\
65.6	0\\
65.61	0\\
65.62	0\\
65.63	0\\
65.64	0\\
65.65	0\\
65.66	0\\
65.67	0\\
65.68	0\\
65.69	0\\
65.7	0\\
65.71	0\\
65.72	0\\
65.73	0\\
65.74	0\\
65.75	0\\
65.76	0\\
65.77	0\\
65.78	0\\
65.79	0\\
65.8	0\\
65.81	0\\
65.82	0\\
65.83	0\\
65.84	0\\
65.85	0\\
65.86	0\\
65.87	0\\
65.88	0\\
65.89	0\\
65.9	0\\
65.91	0\\
65.92	0\\
65.93	0\\
65.94	0\\
65.95	0\\
65.96	0\\
65.97	0\\
65.98	0\\
65.99	0\\
66	0\\
66.01	0\\
66.02	0\\
66.03	0\\
66.04	0\\
66.05	0\\
66.06	0\\
66.07	0\\
66.08	0\\
66.09	0\\
66.1	0\\
66.11	0\\
66.12	0\\
66.13	0\\
66.14	0\\
66.15	0\\
66.16	0\\
66.17	0\\
66.18	0\\
66.19	0\\
66.2	0\\
66.21	0\\
66.22	0\\
66.23	0\\
66.24	0\\
66.25	0\\
66.26	0\\
66.27	0\\
66.28	0\\
66.29	0\\
66.3	0\\
66.31	0\\
66.32	0\\
66.33	0\\
66.34	0\\
66.35	0\\
66.36	0\\
66.37	0\\
66.38	0\\
66.39	0\\
66.4	0\\
66.41	0\\
66.42	0\\
66.43	0\\
66.44	0\\
66.45	0\\
66.46	0\\
66.47	0\\
66.48	0\\
66.49	0\\
66.5	0\\
66.51	0\\
66.52	0\\
66.53	0\\
66.54	0\\
66.55	0\\
66.56	0\\
66.57	0\\
66.58	0\\
66.59	0\\
66.6	0\\
66.61	0\\
66.62	0\\
66.63	0\\
66.64	0\\
66.65	0\\
66.66	0\\
66.67	0\\
66.68	0\\
66.69	0\\
66.7	0\\
66.71	0\\
66.72	0\\
66.73	0\\
66.74	0\\
66.75	0\\
66.76	0\\
66.77	0\\
66.78	0\\
66.79	0\\
66.8	0\\
66.81	0\\
66.82	0\\
66.83	0\\
66.84	0\\
66.85	0\\
66.86	0\\
66.87	0\\
66.88	0\\
66.89	0\\
66.9	0\\
66.91	0\\
66.92	0\\
66.93	0\\
66.94	0\\
66.95	0\\
66.96	0\\
66.97	0\\
66.98	0\\
66.99	0\\
67	0\\
67.01	0\\
67.02	0\\
67.03	0\\
67.04	0\\
67.05	0\\
67.06	0\\
67.07	0\\
67.08	0\\
67.09	0\\
67.1	0\\
67.11	0\\
67.12	0\\
67.13	0\\
67.14	0\\
67.15	0\\
67.16	0\\
67.17	0\\
67.18	0\\
67.19	0\\
67.2	0\\
67.21	0\\
67.22	0\\
67.23	0\\
67.24	0\\
67.25	0\\
67.26	0\\
67.27	0\\
67.28	0\\
67.29	0\\
67.3	0\\
67.31	0\\
67.32	0\\
67.33	0\\
67.34	0\\
67.35	0\\
67.36	0\\
67.37	0\\
67.38	0\\
67.39	0\\
67.4	0\\
67.41	0\\
67.42	0\\
67.43	0\\
67.44	0\\
67.45	0\\
67.46	0\\
67.47	0\\
67.48	0\\
67.49	0\\
67.5	0\\
67.51	0\\
67.52	0\\
67.53	0\\
67.54	0\\
67.55	0\\
67.56	0\\
67.57	0\\
67.58	0\\
67.59	0\\
67.6	0\\
67.61	0\\
67.62	0\\
67.63	0\\
67.64	0\\
67.65	0\\
67.66	0\\
67.67	0\\
67.68	0\\
67.69	0\\
67.7	0\\
67.71	0\\
67.72	0\\
67.73	0\\
67.74	0\\
67.75	0\\
67.76	0\\
67.77	0\\
67.78	0\\
67.79	0\\
67.8	0\\
67.81	0\\
67.82	0\\
67.83	0\\
67.84	0\\
67.85	0\\
67.86	0\\
67.87	0\\
67.88	0\\
67.89	0\\
67.9	0\\
67.91	0\\
67.92	0\\
67.93	0\\
67.94	0\\
67.95	0\\
67.96	0\\
67.97	0\\
67.98	0\\
67.99	0\\
68	0\\
68.01	0\\
68.02	0\\
68.03	0\\
68.04	0\\
68.05	0\\
68.06	0\\
68.07	0\\
68.08	0\\
68.09	0\\
68.1	0\\
68.11	0\\
68.12	0\\
68.13	0\\
68.14	0\\
68.15	0\\
68.16	0\\
68.17	0\\
68.18	0\\
68.19	0\\
68.2	0\\
68.21	0\\
68.22	0\\
68.23	0\\
68.24	0\\
68.25	0\\
68.26	0\\
68.27	0\\
68.28	0\\
68.29	0\\
68.3	0\\
68.31	0\\
68.32	0\\
68.33	0\\
68.34	0\\
68.35	0\\
68.36	0\\
68.37	0\\
68.38	0\\
68.39	0\\
68.4	0\\
68.41	0\\
68.42	0\\
68.43	0\\
68.44	0\\
68.45	0\\
68.46	0\\
68.47	0\\
68.48	0\\
68.49	0\\
68.5	0\\
68.51	0\\
68.52	0\\
68.53	0\\
68.54	0\\
68.55	0\\
68.56	0\\
68.57	0\\
68.58	0\\
68.59	0\\
68.6	0\\
68.61	0\\
68.62	0\\
68.63	0\\
68.64	0\\
68.65	0\\
68.66	0\\
68.67	0\\
68.68	0\\
68.69	0\\
68.7	0\\
68.71	0\\
68.72	0\\
68.73	0\\
68.74	0\\
68.75	0\\
68.76	0\\
68.77	0\\
68.78	0\\
68.79	0\\
68.8	0\\
68.81	0\\
68.82	0\\
68.83	0\\
68.84	0\\
68.85	0\\
68.86	0\\
68.87	0\\
68.88	0\\
68.89	0\\
68.9	0\\
68.91	0\\
68.92	0\\
68.93	0\\
68.94	0\\
68.95	0\\
68.96	0\\
68.97	0\\
68.98	0\\
68.99	0\\
69	0\\
69.01	0\\
69.02	0\\
69.03	0\\
69.04	0\\
69.05	0\\
69.06	0\\
69.07	0\\
69.08	0\\
69.09	0\\
69.1	0\\
69.11	0\\
69.12	0\\
69.13	0\\
69.14	0\\
69.15	0\\
69.16	0\\
69.17	0\\
69.18	0\\
69.19	0\\
69.2	0\\
69.21	0\\
69.22	0\\
69.23	0\\
69.24	0\\
69.25	0\\
69.26	0\\
69.27	0\\
69.28	0\\
69.29	0\\
69.3	0\\
69.31	0\\
69.32	0\\
69.33	0\\
69.34	0\\
69.35	0\\
69.36	0\\
69.37	0\\
69.38	0\\
69.39	0\\
69.4	0\\
69.41	0\\
69.42	0\\
69.43	0\\
69.44	0\\
69.45	0\\
69.46	0\\
69.47	0\\
69.48	0\\
69.49	0\\
69.5	0\\
69.51	0\\
69.52	0\\
69.53	0\\
69.54	0\\
69.55	0\\
69.56	0\\
69.57	0\\
69.58	0\\
69.59	0\\
69.6	0\\
69.61	0\\
69.62	0\\
69.63	0\\
69.64	0\\
69.65	0\\
69.66	0\\
69.67	0\\
69.68	0\\
69.69	0\\
69.7	0\\
69.71	0\\
69.72	0\\
69.73	0\\
69.74	0\\
69.75	0\\
69.76	0\\
69.77	0\\
69.78	0\\
69.79	0\\
69.8	0\\
69.81	0\\
69.82	0\\
69.83	0\\
69.84	0\\
69.85	0\\
69.86	0\\
69.87	0\\
69.88	0\\
69.89	0\\
69.9	0\\
69.91	0\\
69.92	0\\
69.93	0\\
69.94	0\\
69.95	0\\
69.96	0\\
69.97	0\\
69.98	0\\
69.99	0\\
70	0\\
70.01	0\\
70.02	0\\
70.03	0\\
70.04	0\\
70.05	0\\
70.06	0\\
70.07	0\\
70.08	0\\
70.09	0\\
70.1	0\\
70.11	0\\
70.12	0\\
70.13	0\\
70.14	0\\
70.15	0\\
70.16	0\\
70.17	0\\
70.18	0\\
70.19	0\\
70.2	0\\
70.21	0\\
70.22	0\\
70.23	0\\
70.24	0\\
70.25	0\\
70.26	0\\
70.27	0\\
70.28	0\\
70.29	0\\
70.3	0\\
70.31	0\\
70.32	0\\
70.33	0\\
70.34	0\\
70.35	0\\
70.36	0\\
70.37	0\\
70.38	0\\
70.39	0\\
70.4	0\\
70.41	0\\
70.42	0\\
70.43	0\\
70.44	0\\
70.45	0\\
70.46	0\\
70.47	0\\
70.48	0\\
70.49	0\\
70.5	0\\
70.51	0\\
70.52	0\\
70.53	0\\
70.54	0\\
70.55	0\\
70.56	0\\
70.57	0\\
70.58	0\\
70.59	0\\
70.6	0\\
70.61	0\\
70.62	0\\
70.63	0\\
70.64	0\\
70.65	0\\
70.66	0\\
70.67	0\\
70.68	0\\
70.69	0\\
70.7	0\\
70.71	0\\
70.72	0\\
70.73	0\\
70.74	0\\
70.75	0\\
70.76	0\\
70.77	0\\
70.78	0\\
70.79	0\\
70.8	0\\
70.81	0\\
70.82	0\\
70.83	0\\
70.84	0\\
70.85	0\\
70.86	0\\
70.87	0\\
70.88	0\\
70.89	0\\
70.9	0\\
70.91	0\\
70.92	0\\
70.93	0\\
70.94	0\\
70.95	0\\
70.96	0\\
70.97	0\\
70.98	0\\
70.99	0\\
71	0\\
71.01	0\\
71.02	0\\
71.03	0\\
71.04	0\\
71.05	0\\
71.06	0\\
71.07	0\\
71.08	0\\
71.09	0\\
71.1	0\\
71.11	0\\
71.12	0\\
71.13	0\\
71.14	0\\
71.15	0\\
71.16	0\\
71.17	0\\
71.18	0\\
71.19	0\\
71.2	0\\
71.21	0\\
71.22	0\\
71.23	0\\
71.24	0\\
71.25	0\\
71.26	0\\
71.27	0\\
71.28	0\\
71.29	0\\
71.3	0\\
71.31	0\\
71.32	0\\
71.33	0\\
71.34	0\\
71.35	0\\
71.36	0\\
71.37	0\\
71.38	0\\
71.39	0\\
71.4	0\\
71.41	0\\
71.42	0\\
71.43	0\\
71.44	0\\
71.45	0\\
71.46	0\\
71.47	0\\
71.48	0\\
71.49	0\\
71.5	0\\
71.51	0\\
71.52	0\\
71.53	0\\
71.54	0\\
71.55	0\\
71.56	0\\
71.57	0\\
71.58	0\\
71.59	0\\
71.6	0\\
71.61	0\\
71.62	0\\
71.63	0\\
71.64	0\\
71.65	0\\
71.66	0\\
71.67	0\\
71.68	0\\
71.69	0\\
71.7	0\\
71.71	0\\
71.72	0\\
71.73	0\\
71.74	0\\
71.75	0\\
71.76	0\\
71.77	0\\
71.78	0\\
71.79	0\\
71.8	0\\
71.81	0\\
71.82	0\\
71.83	0\\
71.84	0\\
71.85	0\\
71.86	0\\
71.87	0\\
71.88	0\\
71.89	0\\
71.9	0\\
71.91	0\\
71.92	0\\
71.93	0\\
71.94	0\\
71.95	0\\
71.96	0\\
71.97	0\\
71.98	0\\
71.99	0\\
72	0\\
72.01	0\\
72.02	0\\
72.03	0\\
72.04	0\\
72.05	0\\
72.06	0\\
72.07	0\\
72.08	0\\
72.09	0\\
72.1	0\\
72.11	0\\
72.12	0\\
72.13	0\\
72.14	0\\
72.15	0\\
72.16	0\\
72.17	0\\
72.18	0\\
72.19	0\\
72.2	0\\
72.21	0\\
72.22	0\\
72.23	0\\
72.24	0\\
72.25	0\\
72.26	0\\
72.27	0\\
72.28	0\\
72.29	0\\
72.3	0\\
72.31	0\\
72.32	0\\
72.33	0\\
72.34	0\\
72.35	0\\
72.36	0\\
72.37	0\\
72.38	0\\
72.39	0\\
72.4	0\\
72.41	0\\
72.42	0\\
72.43	0\\
72.44	0\\
72.45	0\\
72.46	0\\
72.47	0\\
72.48	0\\
72.49	0\\
72.5	0\\
72.51	0\\
72.52	0\\
72.53	0\\
72.54	0\\
72.55	0\\
72.56	0\\
72.57	0\\
72.58	0\\
72.59	0\\
72.6	0\\
72.61	0\\
72.62	0\\
72.63	0\\
72.64	0\\
72.65	0\\
72.66	0\\
72.67	0\\
72.68	0\\
72.69	0\\
72.7	0\\
72.71	0\\
72.72	0\\
72.73	0\\
72.74	0\\
72.75	0\\
72.76	0\\
72.77	0\\
72.78	0\\
72.79	0\\
72.8	0\\
72.81	0\\
72.82	0\\
72.83	0\\
72.84	0\\
72.85	0\\
72.86	0\\
72.87	0\\
72.88	0\\
72.89	0\\
72.9	0\\
72.91	0\\
72.92	0\\
72.93	0\\
72.94	0\\
72.95	0\\
72.96	0\\
72.97	0\\
72.98	0\\
72.99	0\\
73	0\\
73.01	0\\
73.02	0\\
73.03	0\\
73.04	0\\
73.05	0\\
73.06	0\\
73.07	0\\
73.08	0\\
73.09	0\\
73.1	0\\
73.11	0\\
73.12	0\\
73.13	0\\
73.14	0\\
73.15	0\\
73.16	0\\
73.17	0\\
73.18	0\\
73.19	0\\
73.2	0\\
73.21	0\\
73.22	0\\
73.23	0\\
73.24	0\\
73.25	0\\
73.26	0\\
73.27	0\\
73.28	0\\
73.29	0\\
73.3	0\\
73.31	0\\
73.32	0\\
73.33	0\\
73.34	0\\
73.35	0\\
73.36	0\\
73.37	0\\
73.38	0\\
73.39	0\\
73.4	0\\
73.41	0\\
73.42	0\\
73.43	0\\
73.44	0\\
73.45	0\\
73.46	0\\
73.47	0\\
73.48	0\\
73.49	0\\
73.5	0\\
73.51	0\\
73.52	0\\
73.53	0\\
73.54	0\\
73.55	0\\
73.56	0\\
73.57	0\\
73.58	0\\
73.59	0\\
73.6	0\\
73.61	0\\
73.62	0\\
73.63	0\\
73.64	0\\
73.65	0\\
73.66	0\\
73.67	0\\
73.68	0\\
73.69	0\\
73.7	0\\
73.71	0\\
73.72	0\\
73.73	0\\
73.74	0\\
73.75	0\\
73.76	0\\
73.77	0\\
73.78	0\\
73.79	0\\
73.8	0\\
73.81	0\\
73.82	0\\
73.83	0\\
73.84	0\\
73.85	0\\
73.86	0\\
73.87	0\\
73.88	0\\
73.89	0\\
73.9	0\\
73.91	0\\
73.92	0\\
73.93	0\\
73.94	0\\
73.95	0\\
73.96	0\\
73.97	0\\
73.98	0\\
73.99	0\\
74	0\\
74.01	0\\
74.02	0\\
74.03	0\\
74.04	0\\
74.05	0\\
74.06	0\\
74.07	0\\
74.08	0\\
74.09	0\\
74.1	0\\
74.11	0\\
74.12	0\\
74.13	0\\
74.14	0\\
74.15	0\\
74.16	0\\
74.17	0\\
74.18	0\\
74.19	0\\
74.2	0\\
74.21	0\\
74.22	0\\
74.23	0\\
74.24	0\\
74.25	0\\
74.26	0\\
74.27	0\\
74.28	0\\
74.29	0\\
74.3	0\\
74.31	0\\
74.32	0\\
74.33	0\\
74.34	0\\
74.35	0\\
74.36	0\\
74.37	0\\
74.38	0\\
74.39	0\\
74.4	0\\
74.41	0\\
74.42	0\\
74.43	0\\
74.44	0\\
74.45	0\\
74.46	0\\
74.47	0\\
74.48	0\\
74.49	0\\
74.5	0\\
74.51	0\\
74.52	0\\
74.53	0\\
74.54	0\\
74.55	0\\
74.56	0\\
74.57	0\\
74.58	0\\
74.59	0\\
74.6	0\\
74.61	0\\
74.62	0\\
74.63	0\\
74.64	0\\
74.65	0\\
74.66	0\\
74.67	0\\
74.68	0\\
74.69	0\\
74.7	0\\
74.71	0\\
74.72	0\\
74.73	0\\
74.74	0\\
74.75	0\\
74.76	0\\
74.77	0\\
74.78	0\\
74.79	0\\
74.8	0\\
74.81	0\\
74.82	0\\
74.83	0\\
74.84	0\\
74.85	0\\
74.86	0\\
74.87	0\\
74.88	0\\
74.89	0\\
74.9	0\\
74.91	0\\
74.92	0\\
74.93	0\\
74.94	0\\
74.95	0\\
74.96	0\\
74.97	0\\
74.98	0\\
74.99	0\\
75	0\\
75.01	0\\
75.02	0\\
75.03	0\\
75.04	0\\
75.05	0\\
75.06	0\\
75.07	0\\
75.08	0\\
75.09	0\\
75.1	0\\
75.11	0\\
75.12	0\\
75.13	0\\
75.14	0\\
75.15	0\\
75.16	0\\
75.17	0\\
75.18	0\\
75.19	0\\
75.2	0\\
75.21	0\\
75.22	0\\
75.23	0\\
75.24	0\\
75.25	0\\
75.26	0\\
75.27	0\\
75.28	0\\
75.29	0\\
75.3	0\\
75.31	0\\
75.32	0\\
75.33	0\\
75.34	0\\
75.35	0\\
75.36	0\\
75.37	0\\
75.38	0\\
75.39	0\\
75.4	0\\
75.41	0\\
75.42	0\\
75.43	0\\
75.44	0\\
75.45	0\\
75.46	0\\
75.47	0\\
75.48	0\\
75.49	0\\
75.5	0\\
75.51	0\\
75.52	0\\
75.53	0\\
75.54	0\\
75.55	0\\
75.56	0\\
75.57	0\\
75.58	0\\
75.59	0\\
75.6	0\\
75.61	0\\
75.62	0\\
75.63	0\\
75.64	0\\
75.65	0\\
75.66	0\\
75.67	0\\
75.68	0\\
75.69	0\\
75.7	0\\
75.71	0\\
75.72	0\\
75.73	0\\
75.74	0\\
75.75	0\\
75.76	0\\
75.77	0\\
75.78	0\\
75.79	0\\
75.8	0\\
75.81	0\\
75.82	0\\
75.83	0\\
75.84	0\\
75.85	0\\
75.86	0\\
75.87	0\\
75.88	0\\
75.89	0\\
75.9	0\\
75.91	0\\
75.92	0\\
75.93	0\\
75.94	0\\
75.95	0\\
75.96	0\\
75.97	0\\
75.98	0\\
75.99	0\\
76	0\\
76.01	0\\
76.02	0\\
76.03	0\\
76.04	0\\
76.05	0\\
76.06	0\\
76.07	0\\
76.08	0\\
76.09	0\\
76.1	0\\
76.11	0\\
76.12	0\\
76.13	0\\
76.14	0\\
76.15	0\\
76.16	0\\
76.17	0\\
76.18	0\\
76.19	0\\
76.2	0\\
76.21	0\\
76.22	0\\
76.23	0\\
76.24	0\\
76.25	0\\
76.26	0\\
76.27	0\\
76.28	0\\
76.29	0\\
76.3	0\\
76.31	0\\
76.32	0\\
76.33	0\\
76.34	0\\
76.35	0\\
76.36	0\\
76.37	0\\
76.38	0\\
76.39	0\\
76.4	0\\
76.41	0\\
76.42	0\\
76.43	0\\
76.44	0\\
76.45	0\\
76.46	0\\
76.47	0\\
76.48	0\\
76.49	0\\
76.5	0\\
76.51	0\\
76.52	0\\
76.53	0\\
76.54	0\\
76.55	0\\
76.56	0\\
76.57	0\\
76.58	0\\
76.59	0\\
76.6	0\\
76.61	0\\
76.62	0\\
76.63	0\\
76.64	0\\
76.65	0\\
76.66	0\\
76.67	0\\
76.68	0\\
76.69	0\\
76.7	0\\
76.71	0\\
76.72	0\\
76.73	0\\
76.74	0\\
76.75	0\\
76.76	0\\
76.77	0\\
76.78	0\\
76.79	0\\
76.8	0\\
76.81	0\\
76.82	0\\
76.83	0\\
76.84	0\\
76.85	0\\
76.86	0\\
76.87	0\\
76.88	0\\
76.89	0\\
76.9	0\\
76.91	0\\
76.92	0\\
76.93	0\\
76.94	0\\
76.95	0\\
76.96	0\\
76.97	0\\
76.98	0\\
76.99	0\\
77	0\\
77.01	0\\
77.02	0\\
77.03	0\\
77.04	0\\
77.05	0\\
77.06	0\\
77.07	0\\
77.08	0\\
77.09	0\\
77.1	0\\
77.11	0\\
77.12	0\\
77.13	0\\
77.14	0\\
77.15	0\\
77.16	0\\
77.17	0\\
77.18	0\\
77.19	0\\
77.2	0\\
77.21	0\\
77.22	0\\
77.23	0\\
77.24	0\\
77.25	0\\
77.26	0\\
77.27	0\\
77.28	0\\
77.29	0\\
77.3	0\\
77.31	0\\
77.32	0\\
77.33	0\\
77.34	0\\
77.35	0\\
77.36	0\\
77.37	0\\
77.38	0\\
77.39	0\\
77.4	0\\
77.41	0\\
77.42	0\\
77.43	0\\
77.44	0\\
77.45	0\\
77.46	0\\
77.47	0\\
77.48	0\\
77.49	0\\
77.5	0\\
77.51	0\\
77.52	0\\
77.53	0\\
77.54	0\\
77.55	0\\
77.56	0\\
77.57	0\\
77.58	0\\
77.59	0\\
77.6	0\\
77.61	0\\
77.62	0\\
77.63	0\\
77.64	0\\
77.65	0\\
77.66	0\\
77.67	0\\
77.68	0\\
77.69	0\\
77.7	0\\
77.71	0\\
77.72	0\\
77.73	0\\
77.74	0\\
77.75	0\\
77.76	0\\
77.77	0\\
77.78	0\\
77.79	0\\
77.8	0\\
77.81	0\\
77.82	0\\
77.83	0\\
77.84	0\\
77.85	0\\
77.86	0\\
77.87	0\\
77.88	0\\
77.89	0\\
77.9	0\\
77.91	0\\
77.92	0\\
77.93	0\\
77.94	0\\
77.95	0\\
77.96	0\\
77.97	0\\
77.98	0\\
77.99	0\\
78	0\\
78.01	0\\
78.02	0\\
78.03	0\\
78.04	0\\
78.05	0\\
78.06	0\\
78.07	0\\
78.08	0\\
78.09	0\\
78.1	0\\
78.11	0\\
78.12	0\\
78.13	0\\
78.14	0\\
78.15	0\\
78.16	0\\
78.17	0\\
78.18	0\\
78.19	0\\
78.2	0\\
78.21	0\\
78.22	0\\
78.23	0\\
78.24	0\\
78.25	0\\
78.26	0\\
78.27	0\\
78.28	0\\
78.29	0\\
78.3	0\\
78.31	0\\
78.32	0\\
78.33	0\\
78.34	0\\
78.35	0\\
78.36	0\\
78.37	0\\
78.38	0\\
78.39	0\\
78.4	0\\
78.41	0\\
78.42	0\\
78.43	0\\
78.44	0\\
78.45	0\\
78.46	0\\
78.47	0\\
78.48	0\\
78.49	0\\
78.5	0\\
78.51	0\\
78.52	0\\
78.53	0\\
78.54	0\\
78.55	0\\
78.56	0\\
78.57	0\\
78.58	0\\
78.59	0\\
78.6	0\\
78.61	0\\
78.62	0\\
78.63	0\\
78.64	0\\
78.65	0\\
78.66	0\\
78.67	0\\
78.68	0\\
78.69	0\\
78.7	0\\
78.71	0\\
78.72	0\\
78.73	0\\
78.74	0\\
78.75	0\\
78.76	0\\
78.77	0\\
78.78	0\\
78.79	0\\
78.8	0\\
78.81	0\\
78.82	0\\
78.83	0\\
78.84	0\\
78.85	0\\
78.86	0\\
78.87	0\\
78.88	0\\
78.89	0\\
78.9	0\\
78.91	0\\
78.92	0\\
78.93	0\\
78.94	0\\
78.95	0\\
78.96	0\\
78.97	0\\
78.98	0\\
78.99	0\\
79	0\\
79.01	0\\
79.02	0\\
79.03	0\\
79.04	0\\
79.05	0\\
79.06	0\\
79.07	0\\
79.08	0\\
79.09	0\\
79.1	0\\
79.11	0\\
79.12	0\\
79.13	0\\
79.14	0\\
79.15	0\\
79.16	0\\
79.17	0\\
79.18	0\\
79.19	0\\
79.2	0\\
79.21	0\\
79.22	0\\
79.23	0\\
79.24	0\\
79.25	0\\
79.26	0\\
79.27	0\\
79.28	0\\
79.29	0\\
79.3	0\\
79.31	0\\
79.32	0\\
79.33	0\\
79.34	0\\
79.35	0\\
79.36	0\\
79.37	0\\
79.38	0\\
79.39	0\\
79.4	0\\
79.41	0\\
79.42	0\\
79.43	0\\
79.44	0\\
79.45	0\\
79.46	0\\
79.47	0\\
79.48	0\\
79.49	0\\
79.5	0\\
79.51	0\\
79.52	0\\
79.53	0\\
79.54	0\\
79.55	0\\
79.56	0\\
79.57	0\\
79.58	0\\
79.59	0\\
79.6	0\\
79.61	0\\
79.62	0\\
79.63	0\\
79.64	0\\
79.65	0\\
79.66	0\\
79.67	0\\
79.68	0\\
79.69	0\\
79.7	0\\
79.71	0\\
79.72	0\\
79.73	0\\
79.74	0\\
79.75	0\\
79.76	0\\
79.77	0\\
79.78	0\\
79.79	0\\
79.8	0\\
79.81	0\\
79.82	0\\
79.83	0\\
79.84	0\\
79.85	0\\
79.86	0\\
79.87	0\\
79.88	0\\
79.89	0\\
79.9	0\\
79.91	0\\
79.92	0\\
79.93	0\\
79.94	0\\
79.95	0\\
79.96	0\\
79.97	0\\
79.98	0\\
79.99	0\\
80	0\\
80.01	0\\
};
\addplot [color=blue,solid]
  table[row sep=crcr]{%
80.01	0\\
80.02	0\\
80.03	0\\
80.04	0\\
80.05	0\\
80.06	0\\
80.07	0\\
80.08	0\\
80.09	0\\
80.1	0\\
80.11	0\\
80.12	0\\
80.13	0\\
80.14	0\\
80.15	0\\
80.16	0\\
80.17	0\\
80.18	0\\
80.19	0\\
80.2	0\\
80.21	0\\
80.22	0\\
80.23	0\\
80.24	0\\
80.25	0\\
80.26	0\\
80.27	0\\
80.28	0\\
80.29	0\\
80.3	0\\
80.31	0\\
80.32	0\\
80.33	0\\
80.34	0\\
80.35	0\\
80.36	0\\
80.37	0\\
80.38	0\\
80.39	0\\
80.4	0\\
80.41	0\\
80.42	0\\
80.43	0\\
80.44	0\\
80.45	0\\
80.46	0\\
80.47	0\\
80.48	0\\
80.49	0\\
80.5	0\\
80.51	0\\
80.52	0\\
80.53	0\\
80.54	0\\
80.55	0\\
80.56	0\\
80.57	0\\
80.58	0\\
80.59	0\\
80.6	0\\
80.61	0\\
80.62	0\\
80.63	0\\
80.64	0\\
80.65	0\\
80.66	0\\
80.67	0\\
80.68	0\\
80.69	0\\
80.7	0\\
80.71	0\\
80.72	0\\
80.73	0\\
80.74	0\\
80.75	0\\
80.76	0\\
80.77	0\\
80.78	0\\
80.79	0\\
80.8	0\\
80.81	0\\
80.82	0\\
80.83	0\\
80.84	0\\
80.85	0\\
80.86	0\\
80.87	0\\
80.88	0\\
80.89	0\\
80.9	0\\
80.91	0\\
80.92	0\\
80.93	0\\
80.94	0\\
80.95	0\\
80.96	0\\
80.97	0\\
80.98	0\\
80.99	0\\
81	0\\
81.01	0\\
81.02	0\\
81.03	0\\
81.04	0\\
81.05	0\\
81.06	0\\
81.07	0\\
81.08	0\\
81.09	0\\
81.1	0\\
81.11	0\\
81.12	0\\
81.13	0\\
81.14	0\\
81.15	0\\
81.16	0\\
81.17	0\\
81.18	0\\
81.19	0\\
81.2	0\\
81.21	0\\
81.22	0\\
81.23	0\\
81.24	0\\
81.25	0\\
81.26	0\\
81.27	0\\
81.28	0\\
81.29	0\\
81.3	0\\
81.31	0\\
81.32	0\\
81.33	0\\
81.34	0\\
81.35	0\\
81.36	0\\
81.37	0\\
81.38	0\\
81.39	0\\
81.4	0\\
81.41	0\\
81.42	0\\
81.43	0\\
81.44	0\\
81.45	0\\
81.46	0\\
81.47	0\\
81.48	0\\
81.49	0\\
81.5	0\\
81.51	0\\
81.52	0\\
81.53	0\\
81.54	0\\
81.55	0\\
81.56	0\\
81.57	0\\
81.58	0\\
81.59	0\\
81.6	0\\
81.61	0\\
81.62	0\\
81.63	0\\
81.64	0\\
81.65	0\\
81.66	0\\
81.67	0\\
81.68	0\\
81.69	0\\
81.7	0\\
81.71	0\\
81.72	0\\
81.73	0\\
81.74	0\\
81.75	0\\
81.76	0\\
81.77	0\\
81.78	0\\
81.79	0\\
81.8	0\\
81.81	0\\
81.82	0\\
81.83	0\\
81.84	0\\
81.85	0\\
81.86	0\\
81.87	0\\
81.88	0\\
81.89	0\\
81.9	0\\
81.91	0\\
81.92	0\\
81.93	0\\
81.94	0\\
81.95	0\\
81.96	0\\
81.97	0\\
81.98	0\\
81.99	0\\
82	0\\
82.01	0\\
82.02	0\\
82.03	0\\
82.04	0\\
82.05	0\\
82.06	0\\
82.07	0\\
82.08	0\\
82.09	0\\
82.1	0\\
82.11	0\\
82.12	0\\
82.13	0\\
82.14	0\\
82.15	0\\
82.16	0\\
82.17	0\\
82.18	0\\
82.19	0\\
82.2	0\\
82.21	0\\
82.22	0\\
82.23	0\\
82.24	0\\
82.25	0\\
82.26	0\\
82.27	0\\
82.28	0\\
82.29	0\\
82.3	0\\
82.31	0\\
82.32	0\\
82.33	0\\
82.34	0\\
82.35	0\\
82.36	0\\
82.37	0\\
82.38	0\\
82.39	0\\
82.4	0\\
82.41	0\\
82.42	0\\
82.43	0\\
82.44	0\\
82.45	0\\
82.46	0\\
82.47	0\\
82.48	0\\
82.49	0\\
82.5	0\\
82.51	0\\
82.52	0\\
82.53	0\\
82.54	0\\
82.55	0\\
82.56	0\\
82.57	0\\
82.58	0\\
82.59	0\\
82.6	0\\
82.61	0\\
82.62	0\\
82.63	0\\
82.64	0\\
82.65	0\\
82.66	0\\
82.67	0\\
82.68	0\\
82.69	0\\
82.7	0\\
82.71	0\\
82.72	0\\
82.73	0\\
82.74	0\\
82.75	0\\
82.76	0\\
82.77	0\\
82.78	0\\
82.79	0\\
82.8	0\\
82.81	0\\
82.82	0\\
82.83	0\\
82.84	0\\
82.85	0\\
82.86	0\\
82.87	0\\
82.88	0\\
82.89	0\\
82.9	0\\
82.91	0\\
82.92	0\\
82.93	0\\
82.94	0\\
82.95	0\\
82.96	0\\
82.97	0\\
82.98	0\\
82.99	0\\
83	0\\
83.01	0\\
83.02	0\\
83.03	0\\
83.04	0\\
83.05	0\\
83.06	0\\
83.07	0\\
83.08	0\\
83.09	0\\
83.1	0\\
83.11	0\\
83.12	0\\
83.13	0\\
83.14	0\\
83.15	0\\
83.16	0\\
83.17	0\\
83.18	0\\
83.19	0\\
83.2	0\\
83.21	0\\
83.22	0\\
83.23	0\\
83.24	0\\
83.25	0\\
83.26	0\\
83.27	0\\
83.28	0\\
83.29	0\\
83.3	0\\
83.31	0\\
83.32	0\\
83.33	0\\
83.34	0\\
83.35	0\\
83.36	0\\
83.37	0\\
83.38	0\\
83.39	0\\
83.4	0\\
83.41	0\\
83.42	0\\
83.43	0\\
83.44	0\\
83.45	0\\
83.46	0\\
83.47	0\\
83.48	0\\
83.49	0\\
83.5	0\\
83.51	0\\
83.52	0\\
83.53	0\\
83.54	0\\
83.55	0\\
83.56	0\\
83.57	0\\
83.58	0\\
83.59	0\\
83.6	0\\
83.61	0\\
83.62	0\\
83.63	0\\
83.64	0\\
83.65	0\\
83.66	0\\
83.67	0\\
83.68	0\\
83.69	0\\
83.7	0\\
83.71	0\\
83.72	0\\
83.73	0\\
83.74	0\\
83.75	0\\
83.76	0\\
83.77	0\\
83.78	0\\
83.79	0\\
83.8	0\\
83.81	0\\
83.82	0\\
83.83	0\\
83.84	0\\
83.85	0\\
83.86	0\\
83.87	0\\
83.88	0\\
83.89	0\\
83.9	0\\
83.91	0\\
83.92	0\\
83.93	0\\
83.94	0\\
83.95	0\\
83.96	0\\
83.97	0\\
83.98	0\\
83.99	0\\
84	0\\
84.01	0\\
84.02	0\\
84.03	0\\
84.04	0\\
84.05	0\\
84.06	0\\
84.07	0\\
84.08	0\\
84.09	0\\
84.1	0\\
84.11	0\\
84.12	0\\
84.13	0\\
84.14	0\\
84.15	0\\
84.16	0\\
84.17	0\\
84.18	0\\
84.19	0\\
84.2	0\\
84.21	0\\
84.22	0\\
84.23	0\\
84.24	0\\
84.25	0\\
84.26	0\\
84.27	0\\
84.28	0\\
84.29	0\\
84.3	0\\
84.31	0\\
84.32	0\\
84.33	0\\
84.34	0\\
84.35	0\\
84.36	0\\
84.37	0\\
84.38	0\\
84.39	0\\
84.4	0\\
84.41	0\\
84.42	0\\
84.43	0\\
84.44	0\\
84.45	0\\
84.46	0\\
84.47	0\\
84.48	0\\
84.49	0\\
84.5	0\\
84.51	0\\
84.52	0\\
84.53	0\\
84.54	0\\
84.55	0\\
84.56	0\\
84.57	0\\
84.58	0\\
84.59	0\\
84.6	0\\
84.61	0\\
84.62	0\\
84.63	0\\
84.64	0\\
84.65	0\\
84.66	0\\
84.67	0\\
84.68	0\\
84.69	0\\
84.7	0\\
84.71	0\\
84.72	0\\
84.73	0\\
84.74	0\\
84.75	0\\
84.76	0\\
84.77	0\\
84.78	0\\
84.79	0\\
84.8	0\\
84.81	0\\
84.82	0\\
84.83	0\\
84.84	0\\
84.85	0\\
84.86	0\\
84.87	0\\
84.88	0\\
84.89	0\\
84.9	0\\
84.91	0\\
84.92	0\\
84.93	0\\
84.94	0\\
84.95	0\\
84.96	0\\
84.97	0\\
84.98	0\\
84.99	0\\
85	0\\
85.01	0\\
85.02	0\\
85.03	0\\
85.04	0\\
85.05	0\\
85.06	0\\
85.07	0\\
85.08	0\\
85.09	0\\
85.1	0\\
85.11	0\\
85.12	0\\
85.13	0\\
85.14	0\\
85.15	0\\
85.16	0\\
85.17	0\\
85.18	0\\
85.19	0\\
85.2	0\\
85.21	0\\
85.22	0\\
85.23	0\\
85.24	0\\
85.25	0\\
85.26	0\\
85.27	0\\
85.28	0\\
85.29	0\\
85.3	0\\
85.31	0\\
85.32	0\\
85.33	0\\
85.34	0\\
85.35	0\\
85.36	0\\
85.37	0\\
85.38	0\\
85.39	0\\
85.4	0\\
85.41	0\\
85.42	0\\
85.43	0\\
85.44	0\\
85.45	0\\
85.46	0\\
85.47	0\\
85.48	0\\
85.49	0\\
85.5	0\\
85.51	0\\
85.52	0\\
85.53	0\\
85.54	0\\
85.55	0\\
85.56	0\\
85.57	0\\
85.58	0\\
85.59	0\\
85.6	0\\
85.61	0\\
85.62	0\\
85.63	0\\
85.64	0\\
85.65	0\\
85.66	0\\
85.67	0\\
85.68	0\\
85.69	0\\
85.7	0\\
85.71	0\\
85.72	0\\
85.73	0\\
85.74	0\\
85.75	0\\
85.76	0\\
85.77	0\\
85.78	0\\
85.79	0\\
85.8	0\\
85.81	0\\
85.82	0\\
85.83	0\\
85.84	0\\
85.85	0\\
85.86	0\\
85.87	0\\
85.88	0\\
85.89	0\\
85.9	0\\
85.91	0\\
85.92	0\\
85.93	0\\
85.94	0\\
85.95	0\\
85.96	0\\
85.97	0\\
85.98	0\\
85.99	0\\
86	0\\
86.01	0\\
86.02	0\\
86.03	0\\
86.04	0\\
86.05	0\\
86.06	0\\
86.07	0\\
86.08	0\\
86.09	0\\
86.1	0\\
86.11	0\\
86.12	0\\
86.13	0\\
86.14	0\\
86.15	0\\
86.16	0\\
86.17	0\\
86.18	0\\
86.19	0\\
86.2	0\\
86.21	0\\
86.22	0\\
86.23	0\\
86.24	0\\
86.25	0\\
86.26	0\\
86.27	0\\
86.28	0\\
86.29	0\\
86.3	0\\
86.31	0\\
86.32	0\\
86.33	0\\
86.34	0\\
86.35	0\\
86.36	0\\
86.37	0\\
86.38	0\\
86.39	0\\
86.4	0\\
86.41	0\\
86.42	0\\
86.43	0\\
86.44	0\\
86.45	0\\
86.46	0\\
86.47	0\\
86.48	0\\
86.49	0\\
86.5	0\\
86.51	0\\
86.52	0\\
86.53	0\\
86.54	0\\
86.55	0\\
86.56	0\\
86.57	0\\
86.58	0\\
86.59	0\\
86.6	0\\
86.61	0\\
86.62	0\\
86.63	0\\
86.64	0\\
86.65	0\\
86.66	0\\
86.67	0\\
86.68	0\\
86.69	0\\
86.7	0\\
86.71	0\\
86.72	0\\
86.73	0\\
86.74	0\\
86.75	0\\
86.76	0\\
86.77	0\\
86.78	0\\
86.79	0\\
86.8	0\\
86.81	0\\
86.82	0\\
86.83	0\\
86.84	0\\
86.85	0\\
86.86	0\\
86.87	0\\
86.88	0\\
86.89	0\\
86.9	0\\
86.91	0\\
86.92	0\\
86.93	0\\
86.94	0\\
86.95	0\\
86.96	0\\
86.97	0\\
86.98	0\\
86.99	0\\
87	0\\
87.01	0\\
87.02	0\\
87.03	0\\
87.04	0\\
87.05	0\\
87.06	0\\
87.07	0\\
87.08	0\\
87.09	0\\
87.1	0\\
87.11	0\\
87.12	0\\
87.13	0\\
87.14	0\\
87.15	0\\
87.16	0\\
87.17	0\\
87.18	0\\
87.19	0\\
87.2	0\\
87.21	0\\
87.22	0\\
87.23	0\\
87.24	0\\
87.25	0\\
87.26	0\\
87.27	0\\
87.28	0\\
87.29	0\\
87.3	0\\
87.31	0\\
87.32	0\\
87.33	0\\
87.34	0\\
87.35	0\\
87.36	0\\
87.37	0\\
87.38	0\\
87.39	0\\
87.4	0\\
87.41	0\\
87.42	0\\
87.43	0\\
87.44	0\\
87.45	0\\
87.46	0\\
87.47	0\\
87.48	0\\
87.49	0\\
87.5	0\\
87.51	0\\
87.52	0\\
87.53	0\\
87.54	0\\
87.55	0\\
87.56	0\\
87.57	0\\
87.58	0\\
87.59	0\\
87.6	0\\
87.61	0\\
87.62	0\\
87.63	0\\
87.64	0\\
87.65	0\\
87.66	0\\
87.67	0\\
87.68	0\\
87.69	0\\
87.7	0\\
87.71	0\\
87.72	0\\
87.73	0\\
87.74	0\\
87.75	0\\
87.76	0\\
87.77	0\\
87.78	0\\
87.79	0\\
87.8	0\\
87.81	0\\
87.82	0\\
87.83	0\\
87.84	0\\
87.85	0\\
87.86	0\\
87.87	0\\
87.88	0\\
87.89	0\\
87.9	0\\
87.91	0\\
87.92	0\\
87.93	0\\
87.94	0\\
87.95	0\\
87.96	0\\
87.97	0\\
87.98	0\\
87.99	0\\
88	0\\
88.01	0\\
88.02	0\\
88.03	0\\
88.04	0\\
88.05	0\\
88.06	0\\
88.07	0\\
88.08	0\\
88.09	0\\
88.1	0\\
88.11	0\\
88.12	0\\
88.13	0\\
88.14	0\\
88.15	0\\
88.16	0\\
88.17	0\\
88.18	0\\
88.19	0\\
88.2	0\\
88.21	0\\
88.22	0\\
88.23	0\\
88.24	0\\
88.25	0\\
88.26	0\\
88.27	0\\
88.28	0\\
88.29	0\\
88.3	0\\
88.31	0\\
88.32	0\\
88.33	0\\
88.34	0\\
88.35	0\\
88.36	0\\
88.37	0\\
88.38	0\\
88.39	0\\
88.4	0\\
88.41	0\\
88.42	0\\
88.43	0\\
88.44	0\\
88.45	0\\
88.46	0\\
88.47	0\\
88.48	0\\
88.49	0\\
88.5	0\\
88.51	0\\
88.52	0\\
88.53	0\\
88.54	0\\
88.55	0\\
88.56	0\\
88.57	0\\
88.58	0\\
88.59	0\\
88.6	0\\
88.61	0\\
88.62	0\\
88.63	0\\
88.64	0\\
88.65	0\\
88.66	0\\
88.67	0\\
88.68	0\\
88.69	0\\
88.7	0\\
88.71	0\\
88.72	0\\
88.73	0\\
88.74	0\\
88.75	0\\
88.76	0\\
88.77	0\\
88.78	0\\
88.79	0\\
88.8	0\\
88.81	0\\
88.82	0\\
88.83	0\\
88.84	0\\
88.85	0\\
88.86	0\\
88.87	0\\
88.88	0\\
88.89	0\\
88.9	0\\
88.91	0\\
88.92	0\\
88.93	0\\
88.94	0\\
88.95	0\\
88.96	0\\
88.97	0\\
88.98	0\\
88.99	0\\
89	0\\
89.01	0\\
89.02	0\\
89.03	0\\
89.04	0\\
89.05	0\\
89.06	0\\
89.07	0\\
89.08	0\\
89.09	0\\
89.1	0\\
89.11	0\\
89.12	0\\
89.13	0\\
89.14	0\\
89.15	0\\
89.16	0\\
89.17	0\\
89.18	0\\
89.19	0\\
89.2	0\\
89.21	0\\
89.22	0\\
89.23	0\\
89.24	0\\
89.25	0\\
89.26	0\\
89.27	0\\
89.28	0\\
89.29	0\\
89.3	0\\
89.31	0\\
89.32	0\\
89.33	0\\
89.34	0\\
89.35	0\\
89.36	0\\
89.37	0\\
89.38	0\\
89.39	0\\
89.4	0\\
89.41	0\\
89.42	0\\
89.43	0\\
89.44	0\\
89.45	0\\
89.46	0\\
89.47	0\\
89.48	0\\
89.49	0\\
89.5	0\\
89.51	0\\
89.52	0\\
89.53	0\\
89.54	0\\
89.55	0\\
89.56	0\\
89.57	0\\
89.58	0\\
89.59	0\\
89.6	0\\
89.61	0\\
89.62	0\\
89.63	0\\
89.64	0\\
89.65	0\\
89.66	0\\
89.67	0\\
89.68	0\\
89.69	0\\
89.7	0\\
89.71	0\\
89.72	0\\
89.73	0\\
89.74	0\\
89.75	0\\
89.76	0\\
89.77	0\\
89.78	0\\
89.79	0\\
89.8	0\\
89.81	0\\
89.82	0\\
89.83	0\\
89.84	0\\
89.85	0\\
89.86	0\\
89.87	0\\
89.88	0\\
89.89	0\\
89.9	0\\
89.91	0\\
89.92	0\\
89.93	0\\
89.94	0\\
89.95	0\\
89.96	0\\
89.97	0\\
89.98	0\\
89.99	0\\
90	0\\
90.01	0\\
90.02	0\\
90.03	0\\
90.04	0\\
90.05	0\\
90.06	0\\
90.07	0\\
90.08	0\\
90.09	0\\
90.1	0\\
90.11	0\\
90.12	0\\
90.13	0\\
90.14	0\\
90.15	0\\
90.16	0\\
90.17	0\\
90.18	0\\
90.19	0\\
90.2	0\\
90.21	0\\
90.22	0\\
90.23	0\\
90.24	0\\
90.25	0\\
90.26	0\\
90.27	0\\
90.28	0\\
90.29	0\\
90.3	0\\
90.31	0\\
90.32	0\\
90.33	0\\
90.34	0\\
90.35	0\\
90.36	0\\
90.37	0\\
90.38	0\\
90.39	0\\
90.4	0\\
90.41	0\\
90.42	0\\
90.43	0\\
90.44	0\\
90.45	0\\
90.46	0\\
90.47	0\\
90.48	0\\
90.49	0\\
90.5	0\\
90.51	0\\
90.52	0\\
90.53	0\\
90.54	0\\
90.55	0\\
90.56	0\\
90.57	0\\
90.58	0\\
90.59	0\\
90.6	0\\
90.61	0\\
90.62	0\\
90.63	0\\
90.64	0\\
90.65	0\\
90.66	0\\
90.67	0\\
90.68	0\\
90.69	0\\
90.7	0\\
90.71	0\\
90.72	0\\
90.73	0\\
90.74	0\\
90.75	0\\
90.76	0\\
90.77	0\\
90.78	0\\
90.79	0\\
90.8	0\\
90.81	0\\
90.82	0\\
90.83	0\\
90.84	0\\
90.85	0\\
90.86	0\\
90.87	0\\
90.88	0\\
90.89	0\\
90.9	0\\
90.91	0\\
90.92	0\\
90.93	0\\
90.94	0\\
90.95	0\\
90.96	0\\
90.97	0\\
90.98	0\\
90.99	0\\
91	0\\
91.01	0\\
91.02	0\\
91.03	0\\
91.04	0\\
91.05	0\\
91.06	0\\
91.07	0\\
91.08	0\\
91.09	0\\
91.1	0\\
91.11	0\\
91.12	0\\
91.13	0\\
91.14	0\\
91.15	0\\
91.16	0\\
91.17	0\\
91.18	0\\
91.19	0\\
91.2	0\\
91.21	0\\
91.22	0\\
91.23	0\\
91.24	0\\
91.25	0\\
91.26	0\\
91.27	0\\
91.28	0\\
91.29	0\\
91.3	0\\
91.31	0\\
91.32	0\\
91.33	0\\
91.34	0\\
91.35	0\\
91.36	0\\
91.37	0\\
91.38	0\\
91.39	0\\
91.4	0\\
91.41	0\\
91.42	0\\
91.43	0\\
91.44	0\\
91.45	0\\
91.46	0\\
91.47	0\\
91.48	0\\
91.49	0\\
91.5	0\\
91.51	0\\
91.52	0\\
91.53	0\\
91.54	0\\
91.55	0\\
91.56	0\\
91.57	0\\
91.58	0\\
91.59	0\\
91.6	0\\
91.61	0\\
91.62	0\\
91.63	0\\
91.64	0\\
91.65	0\\
91.66	0\\
91.67	0\\
91.68	0\\
91.69	0\\
91.7	0\\
91.71	0\\
91.72	0\\
91.73	0\\
91.74	0\\
91.75	0\\
91.76	0\\
91.77	0\\
91.78	0\\
91.79	0\\
91.8	0\\
91.81	0\\
91.82	0\\
91.83	0\\
91.84	0\\
91.85	0\\
91.86	0\\
91.87	0\\
91.88	0\\
91.89	0\\
91.9	0\\
91.91	0\\
91.92	0\\
91.93	0\\
91.94	0\\
91.95	0\\
91.96	0\\
91.97	0\\
91.98	0\\
91.99	0\\
92	0\\
92.01	0\\
92.02	0\\
92.03	0\\
92.04	0\\
92.05	0\\
92.06	0\\
92.07	0\\
92.08	0\\
92.09	0\\
92.1	0\\
92.11	0\\
92.12	0\\
92.13	0\\
92.14	0\\
92.15	0\\
92.16	0\\
92.17	0\\
92.18	0\\
92.19	0\\
92.2	0\\
92.21	0\\
92.22	0\\
92.23	0\\
92.24	0\\
92.25	0\\
92.26	0\\
92.27	0\\
92.28	0\\
92.29	0\\
92.3	0\\
92.31	0\\
92.32	0\\
92.33	0\\
92.34	0\\
92.35	0\\
92.36	0\\
92.37	0\\
92.38	0\\
92.39	0\\
92.4	0\\
92.41	0\\
92.42	0\\
92.43	0\\
92.44	0\\
92.45	0\\
92.46	0\\
92.47	0\\
92.48	0\\
92.49	0\\
92.5	0\\
92.51	0\\
92.52	0\\
92.53	0\\
92.54	0\\
92.55	0\\
92.56	0\\
92.57	0\\
92.58	0\\
92.59	0\\
92.6	0\\
92.61	0\\
92.62	0\\
92.63	0\\
92.64	0\\
92.65	0\\
92.66	0\\
92.67	0\\
92.68	0\\
92.69	0\\
92.7	0\\
92.71	0\\
92.72	0\\
92.73	0\\
92.74	0\\
92.75	0\\
92.76	0\\
92.77	0\\
92.78	0\\
92.79	0\\
92.8	0\\
92.81	0\\
92.82	0\\
92.83	0\\
92.84	0\\
92.85	0\\
92.86	0\\
92.87	0\\
92.88	0\\
92.89	0\\
92.9	0\\
92.91	0\\
92.92	0\\
92.93	0\\
92.94	0\\
92.95	0\\
92.96	0\\
92.97	0\\
92.98	0\\
92.99	0\\
93	0\\
93.01	0\\
93.02	0\\
93.03	0\\
93.04	0\\
93.05	0\\
93.06	0\\
93.07	0\\
93.08	0\\
93.09	0\\
93.1	0\\
93.11	0\\
93.12	0\\
93.13	0\\
93.14	0\\
93.15	0\\
93.16	0\\
93.17	0\\
93.18	0\\
93.19	0\\
93.2	0\\
93.21	0\\
93.22	0\\
93.23	0\\
93.24	0\\
93.25	0\\
93.26	0\\
93.27	0\\
93.28	0\\
93.29	0\\
93.3	0\\
93.31	0\\
93.32	0\\
93.33	0\\
93.34	0\\
93.35	0\\
93.36	0\\
93.37	0\\
93.38	0\\
93.39	0\\
93.4	0\\
93.41	0\\
93.42	0\\
93.43	0\\
93.44	0\\
93.45	0\\
93.46	0\\
93.47	0\\
93.48	0\\
93.49	0\\
93.5	0\\
93.51	0\\
93.52	0\\
93.53	0\\
93.54	0\\
93.55	0\\
93.56	0\\
93.57	0\\
93.58	0\\
93.59	0\\
93.6	0\\
93.61	0\\
93.62	0\\
93.63	0\\
93.64	0\\
93.65	0\\
93.66	0\\
93.67	0\\
93.68	0\\
93.69	0\\
93.7	0\\
93.71	0\\
93.72	0\\
93.73	0\\
93.74	0\\
93.75	0\\
93.76	0\\
93.77	0\\
93.78	0\\
93.79	0\\
93.8	0\\
93.81	0\\
93.82	0\\
93.83	0\\
93.84	0\\
93.85	0\\
93.86	0\\
93.87	0\\
93.88	0\\
93.89	0\\
93.9	0\\
93.91	0\\
93.92	0\\
93.93	0\\
93.94	0\\
93.95	0\\
93.96	0\\
93.97	0\\
93.98	0\\
93.99	0\\
94	0\\
94.01	0\\
94.02	0\\
94.03	0\\
94.04	0\\
94.05	0\\
94.06	0\\
94.07	0\\
94.08	0\\
94.09	0\\
94.1	0\\
94.11	0\\
94.12	0\\
94.13	0\\
94.14	0\\
94.15	0\\
94.16	0\\
94.17	0\\
94.18	0\\
94.19	0\\
94.2	0\\
94.21	0\\
94.22	0\\
94.23	0\\
94.24	0\\
94.25	0\\
94.26	0\\
94.27	0\\
94.28	0\\
94.29	0\\
94.3	0\\
94.31	0\\
94.32	0\\
94.33	0\\
94.34	0\\
94.35	0\\
94.36	0\\
94.37	0\\
94.38	0\\
94.39	0\\
94.4	0\\
94.41	0\\
94.42	0\\
94.43	0\\
94.44	0\\
94.45	0\\
94.46	0\\
94.47	0\\
94.48	0\\
94.49	0\\
94.5	0\\
94.51	0\\
94.52	0\\
94.53	0\\
94.54	0\\
94.55	0\\
94.56	0\\
94.57	0\\
94.58	0\\
94.59	0\\
94.6	0\\
94.61	0\\
94.62	0\\
94.63	0\\
94.64	0\\
94.65	0\\
94.66	0\\
94.67	0\\
94.68	0\\
94.69	0\\
94.7	0\\
94.71	0\\
94.72	0\\
94.73	0\\
94.74	0\\
94.75	0\\
94.76	0\\
94.77	0\\
94.78	0\\
94.79	0\\
94.8	0\\
94.81	0\\
94.82	0\\
94.83	0\\
94.84	0\\
94.85	0\\
94.86	0\\
94.87	0\\
94.88	0\\
94.89	0\\
94.9	0\\
94.91	0\\
94.92	0\\
94.93	0\\
94.94	0\\
94.95	0\\
94.96	0\\
94.97	0\\
94.98	0\\
94.99	0\\
95	0\\
95.01	0\\
95.02	0\\
95.03	0\\
95.04	0\\
95.05	0\\
95.06	0\\
95.07	0\\
95.08	0\\
95.09	0\\
95.1	0\\
95.11	0\\
95.12	0\\
95.13	0\\
95.14	0\\
95.15	0\\
95.16	0\\
95.17	0\\
95.18	0\\
95.19	0\\
95.2	0\\
95.21	0\\
95.22	0\\
95.23	0\\
95.24	0\\
95.25	0\\
95.26	0\\
95.27	0\\
95.28	0\\
95.29	0\\
95.3	0\\
95.31	0\\
95.32	0\\
95.33	0\\
95.34	0\\
95.35	0\\
95.36	0\\
95.37	0\\
95.38	0\\
95.39	0\\
95.4	0\\
95.41	0\\
95.42	0\\
95.43	0\\
95.44	0\\
95.45	0\\
95.46	0\\
95.47	0\\
95.48	0\\
95.49	0\\
95.5	0\\
95.51	0\\
95.52	0\\
95.53	0\\
95.54	0\\
95.55	0\\
95.56	0\\
95.57	0\\
95.58	0\\
95.59	0\\
95.6	0\\
95.61	0\\
95.62	0\\
95.63	0\\
95.64	0\\
95.65	0\\
95.66	0\\
95.67	0\\
95.68	0\\
95.69	0\\
95.7	0\\
95.71	0\\
95.72	0\\
95.73	0\\
95.74	0\\
95.75	0\\
95.76	0\\
95.77	0\\
95.78	0\\
95.79	0\\
95.8	0\\
95.81	0\\
95.82	0\\
95.83	0\\
95.84	0\\
95.85	0\\
95.86	0\\
95.87	0\\
95.88	0\\
95.89	0\\
95.9	0\\
95.91	0\\
95.92	0\\
95.93	0\\
95.94	0\\
95.95	0\\
95.96	0\\
95.97	0\\
95.98	0\\
95.99	0\\
96	0\\
96.01	0\\
96.02	0\\
96.03	0\\
96.04	0\\
96.05	0\\
96.06	0\\
96.07	0\\
96.08	0\\
96.09	0\\
96.1	0\\
96.11	0\\
96.12	0\\
96.13	0\\
96.14	0\\
96.15	0\\
96.16	0\\
96.17	0\\
96.18	0\\
96.19	0\\
96.2	0\\
96.21	0\\
96.22	0\\
96.23	0\\
96.24	0\\
96.25	0\\
96.26	0\\
96.27	0\\
96.28	0\\
96.29	0\\
96.3	0\\
96.31	0\\
96.32	0\\
96.33	0\\
96.34	0\\
96.35	0\\
96.36	0\\
96.37	0\\
96.38	0\\
96.39	0\\
96.4	0\\
96.41	0\\
96.42	0\\
96.43	0\\
96.44	0\\
96.45	0\\
96.46	0\\
96.47	0\\
96.48	0\\
96.49	0\\
96.5	0\\
96.51	0\\
96.52	0\\
96.53	0\\
96.54	0\\
96.55	0\\
96.56	0\\
96.57	0\\
96.58	0\\
96.59	0\\
96.6	0\\
96.61	0\\
96.62	0\\
96.63	0\\
96.64	0\\
96.65	0\\
96.66	0\\
96.67	0\\
96.68	0\\
96.69	0\\
96.7	0\\
96.71	0\\
96.72	0\\
96.73	0\\
96.74	0\\
96.75	0\\
96.76	0\\
96.77	0\\
96.78	0\\
96.79	0\\
96.8	0\\
96.81	0\\
96.82	0\\
96.83	0\\
96.84	0\\
96.85	0\\
96.86	0\\
96.87	0\\
96.88	0\\
96.89	0\\
96.9	0\\
96.91	0\\
96.92	0\\
96.93	0\\
96.94	0\\
96.95	0\\
96.96	0\\
96.97	0\\
96.98	0\\
96.99	0\\
97	0\\
97.01	0\\
97.02	0\\
97.03	0\\
97.04	0\\
97.05	0\\
97.06	0\\
97.07	0\\
97.08	0\\
97.09	0\\
97.1	0\\
97.11	0\\
97.12	0\\
97.13	0\\
97.14	0\\
97.15	0\\
97.16	0\\
97.17	0\\
97.18	0\\
97.19	0\\
97.2	0\\
97.21	0\\
97.22	0\\
97.23	0\\
97.24	0\\
97.25	0\\
97.26	0\\
97.27	0\\
97.28	0\\
97.29	0\\
97.3	0\\
97.31	0\\
97.32	0\\
97.33	0\\
97.34	0\\
97.35	0\\
97.36	0\\
97.37	0\\
97.38	0\\
97.39	0\\
97.4	0\\
97.41	0\\
97.42	0\\
97.43	0\\
97.44	0\\
97.45	0\\
97.46	0\\
97.47	0\\
97.48	0\\
97.49	0\\
97.5	0\\
97.51	0\\
97.52	0\\
97.53	0\\
97.54	0\\
97.55	0\\
97.56	0\\
97.57	0\\
97.58	0\\
97.59	0\\
97.6	0\\
97.61	0\\
97.62	0\\
97.63	0\\
97.64	0\\
97.65	0\\
97.66	0\\
97.67	0\\
97.68	0\\
97.69	0\\
97.7	0\\
97.71	0\\
97.72	0\\
97.73	0\\
97.74	0\\
97.75	0\\
97.76	0\\
97.77	0\\
97.78	0\\
97.79	0\\
97.8	0\\
97.81	0\\
97.82	0\\
97.83	0\\
97.84	0\\
97.85	0\\
97.86	0\\
97.87	0\\
97.88	0\\
97.89	0\\
97.9	0\\
97.91	0\\
97.92	0\\
97.93	0\\
97.94	0\\
97.95	0\\
97.96	0\\
97.97	0\\
97.98	0\\
97.99	0\\
98	0\\
98.01	0\\
98.02	0\\
98.03	0\\
98.04	0\\
98.05	0\\
98.06	0\\
98.07	0\\
98.08	0\\
98.09	0\\
98.1	0\\
98.11	0\\
98.12	0\\
98.13	0\\
98.14	0\\
98.15	0\\
98.16	0\\
98.17	0\\
98.18	0\\
98.19	0\\
98.2	0\\
98.21	0\\
98.22	0\\
98.23	0\\
98.24	0\\
98.25	0\\
98.26	0\\
98.27	0\\
98.28	0\\
98.29	0\\
98.3	0\\
98.31	0\\
98.32	0\\
98.33	0\\
98.34	0\\
98.35	0\\
98.36	0\\
98.37	5.7998100325321e-05\\
98.38	0.000118506413573239\\
98.39	0.000179467720816638\\
98.4	0.000240886391479865\\
98.41	0.000302766842697373\\
98.42	0.000365113539938074\\
98.43	0.000427921760477259\\
98.44	0.000491193024538615\\
98.45	0.000554931731968803\\
98.46	0.000619142330284417\\
98.47	0.000683829315301882\\
98.48	0.00074899723177797\\
98.49	0.000814650674061151\\
98.5	0.000880794286754041\\
98.51	0.000947432765387158\\
98.52	0.00101457085710427\\
98.53	0.00108221336135956\\
98.54	0.00115036513062691\\
98.55	0.00121903107112158\\
98.56	0.00128821614353447\\
98.57	0.00135792536377948\\
98.58	0.00142816380375393\\
98.59	0.00149893659211269\\
98.6	0.00157024891505607\\
98.61	0.00164210601713198\\
98.62	0.00171451320205261\\
98.63	0.00178747583352593\\
98.64	0.0018609993361025\\
98.65	0.00193508919603787\\
98.66	0.00200975096216816\\
98.67	0.00208499024680375\\
98.68	0.0021608127266408\\
98.69	0.00220266311440171\\
98.7	0.00223385457855723\\
98.71	0.0022653127949539\\
98.72	0.00229704020679703\\
98.73	0.00232903928724017\\
98.74	0.00236131509615186\\
98.75	0.00239387224952333\\
98.76	0.00242671336449341\\
98.77	0.00245984108183159\\
98.78	0.00249325806613635\\
98.79	0.00252697449899538\\
98.8	0.00256099537475504\\
98.81	0.00259532360587348\\
98.82	0.00262996213227076\\
98.83	0.00266491392227347\\
98.84	0.00270018197272896\\
98.85	0.00273576930885851\\
98.86	0.002771678984517\\
98.87	0.00280791408245459\\
98.88	0.00284447771458059\\
98.89	0.00288137302222933\\
98.9	0.00291860317642809\\
98.91	0.00295617137816723\\
98.92	0.00299408085867234\\
98.93	0.00303233487967847\\
98.94	0.00307093673370654\\
98.95	0.00310988974434178\\
98.96	0.00314919726651431\\
98.97	0.00318886268678185\\
98.98	0.00322888942361441\\
98.99	0.00326928092768132\\
99	0.00331004068214008\\
99.01	0.00335117220292753\\
99.02	0.00339267903905301\\
99.03	0.00343456477289363\\
99.04	0.00347683302049166\\
99.05	0.00351948743185394\\
99.06	0.00356253169125339\\
99.07	0.00360596951753261\\
99.08	0.00364980462385488\\
99.09	0.00369404072412283\\
99.1	0.00373868156623795\\
99.11	0.00378373093239057\\
99.12	0.00382919263861687\\
99.13	0.00387507053562815\\
99.14	0.00392136850928232\\
99.15	0.00396809048088036\\
99.16	0.00401524040746425\\
99.17	0.00406282228211657\\
99.18	0.00411084013426162\\
99.19	0.00415929802996813\\
99.2	0.00420820007225344\\
99.21	0.00425755040138913\\
99.22	0.00430735319520808\\
99.23	0.0043576126694129\\
99.24	0.00440833307788572\\
99.25	0.0044595187129992\\
99.26	0.00451117390592883\\
99.27	0.00456330302696634\\
99.28	0.0046159104858343\\
99.29	0.00466900073200166\\
99.3	0.0047225782550004\\
99.31	0.00477664758474295\\
99.32	0.00483121329184059\\
99.33	0.00488627998792247\\
99.34	0.00494185232595538\\
99.35	0.00499793500056411\\
99.36	0.00505453274835219\\
99.37	0.0051116503482232\\
99.38	0.00516929262170212\\
99.39	0.00522746443325709\\
99.4	0.00528617069062104\\
99.41	0.0053454163451133\\
99.42	0.00540520639196104\\
99.43	0.0054655458706203\\
99.44	0.00552643986509657\\
99.45	0.00558789350426468\\
99.46	0.00564991196218789\\
99.47	0.00571250045843599\\
99.48	0.00577566425840224\\
99.49	0.00583940867361894\\
99.5	0.00590373906207135\\
99.51	0.00596866082851\\
99.52	0.00603417942476082\\
99.53	0.00610030035003307\\
99.54	0.00616702915122485\\
99.55	0.00623437142322573\\
99.56	0.00630233280921639\\
99.57	0.00637091900096489\\
99.58	0.00644013573911934\\
99.59	0.00650998881349655\\
99.6	0.0065804840633663\\
99.61	0.00665162737777062\\
99.62	0.00672342469581863\\
99.63	0.00679588200695593\\
99.64	0.00686900535122805\\
99.65	0.00694280081953754\\
99.66	0.00701727455389419\\
99.67	0.00709243274765783\\
99.68	0.00716828164577332\\
99.69	0.00724482754499706\\
99.7	0.0073220767941144\\
99.71	0.00740003575010869\\
99.72	0.0074787108180037\\
99.73	0.0075581084544986\\
99.74	0.00763823516812948\\
99.75	0.00771909751941802\\
99.76	0.00780070212100672\\
99.77	0.00788305563777966\\
99.78	0.00796616478696797\\
99.79	0.00805003633823904\\
99.8	0.00813467711376831\\
99.81	0.00822009398829281\\
99.82	0.0083062938891451\\
99.83	0.00839328379626649\\
99.84	0.00848107074219826\\
99.85	0.00856966181204955\\
99.86	0.00865906414344045\\
99.87	0.00874928492641887\\
99.88	0.00884033140334948\\
99.89	0.00893221086877317\\
99.9	0.00902493066923513\\
99.91	0.00911849820307978\\
99.92	0.0092129209202104\\
99.93	0.00930820632181158\\
99.94	0.00940436196003195\\
99.95	0.00950139543762514\\
99.96	0.00959931440754618\\
99.97	0.00969812657250083\\
99.98	0.00979783968444508\\
99.99	0.00989846154403157\\
100	0.01\\
};
\addlegendentry{$q=1$};

\addplot [color=red,solid,forget plot]
  table[row sep=crcr]{%
0.01	0\\
0.02	0\\
0.03	0\\
0.04	0\\
0.05	0\\
0.06	0\\
0.07	0\\
0.08	0\\
0.09	0\\
0.1	0\\
0.11	0\\
0.12	0\\
0.13	0\\
0.14	0\\
0.15	0\\
0.16	0\\
0.17	0\\
0.18	0\\
0.19	0\\
0.2	0\\
0.21	0\\
0.22	0\\
0.23	0\\
0.24	0\\
0.25	0\\
0.26	0\\
0.27	0\\
0.28	0\\
0.29	0\\
0.3	0\\
0.31	0\\
0.32	0\\
0.33	0\\
0.34	0\\
0.35	0\\
0.36	0\\
0.37	0\\
0.38	0\\
0.39	0\\
0.4	0\\
0.41	0\\
0.42	0\\
0.43	0\\
0.44	0\\
0.45	0\\
0.46	0\\
0.47	0\\
0.48	0\\
0.49	0\\
0.5	0\\
0.51	0\\
0.52	0\\
0.53	0\\
0.54	0\\
0.55	0\\
0.56	0\\
0.57	0\\
0.58	0\\
0.59	0\\
0.6	0\\
0.61	0\\
0.62	0\\
0.63	0\\
0.64	0\\
0.65	0\\
0.66	0\\
0.67	0\\
0.68	0\\
0.69	0\\
0.7	0\\
0.71	0\\
0.72	0\\
0.73	0\\
0.74	0\\
0.75	0\\
0.76	0\\
0.77	0\\
0.78	0\\
0.79	0\\
0.8	0\\
0.81	0\\
0.82	0\\
0.83	0\\
0.84	0\\
0.85	0\\
0.86	0\\
0.87	0\\
0.88	0\\
0.89	0\\
0.9	0\\
0.91	0\\
0.92	0\\
0.93	0\\
0.94	0\\
0.95	0\\
0.96	0\\
0.97	0\\
0.98	0\\
0.99	0\\
1	0\\
1.01	0\\
1.02	0\\
1.03	0\\
1.04	0\\
1.05	0\\
1.06	0\\
1.07	0\\
1.08	0\\
1.09	0\\
1.1	0\\
1.11	0\\
1.12	0\\
1.13	0\\
1.14	0\\
1.15	0\\
1.16	0\\
1.17	0\\
1.18	0\\
1.19	0\\
1.2	0\\
1.21	0\\
1.22	0\\
1.23	0\\
1.24	0\\
1.25	0\\
1.26	0\\
1.27	0\\
1.28	0\\
1.29	0\\
1.3	0\\
1.31	0\\
1.32	0\\
1.33	0\\
1.34	0\\
1.35	0\\
1.36	0\\
1.37	0\\
1.38	0\\
1.39	0\\
1.4	0\\
1.41	0\\
1.42	0\\
1.43	0\\
1.44	0\\
1.45	0\\
1.46	0\\
1.47	0\\
1.48	0\\
1.49	0\\
1.5	0\\
1.51	0\\
1.52	0\\
1.53	0\\
1.54	0\\
1.55	0\\
1.56	0\\
1.57	0\\
1.58	0\\
1.59	0\\
1.6	0\\
1.61	0\\
1.62	0\\
1.63	0\\
1.64	0\\
1.65	0\\
1.66	0\\
1.67	0\\
1.68	0\\
1.69	0\\
1.7	0\\
1.71	0\\
1.72	0\\
1.73	0\\
1.74	0\\
1.75	0\\
1.76	0\\
1.77	0\\
1.78	0\\
1.79	0\\
1.8	0\\
1.81	0\\
1.82	0\\
1.83	0\\
1.84	0\\
1.85	0\\
1.86	0\\
1.87	0\\
1.88	0\\
1.89	0\\
1.9	0\\
1.91	0\\
1.92	0\\
1.93	0\\
1.94	0\\
1.95	0\\
1.96	0\\
1.97	0\\
1.98	0\\
1.99	0\\
2	0\\
2.01	0\\
2.02	0\\
2.03	0\\
2.04	0\\
2.05	0\\
2.06	0\\
2.07	0\\
2.08	0\\
2.09	0\\
2.1	0\\
2.11	0\\
2.12	0\\
2.13	0\\
2.14	0\\
2.15	0\\
2.16	0\\
2.17	0\\
2.18	0\\
2.19	0\\
2.2	0\\
2.21	0\\
2.22	0\\
2.23	0\\
2.24	0\\
2.25	0\\
2.26	0\\
2.27	0\\
2.28	0\\
2.29	0\\
2.3	0\\
2.31	0\\
2.32	0\\
2.33	0\\
2.34	0\\
2.35	0\\
2.36	0\\
2.37	0\\
2.38	0\\
2.39	0\\
2.4	0\\
2.41	0\\
2.42	0\\
2.43	0\\
2.44	0\\
2.45	0\\
2.46	0\\
2.47	0\\
2.48	0\\
2.49	0\\
2.5	0\\
2.51	0\\
2.52	0\\
2.53	0\\
2.54	0\\
2.55	0\\
2.56	0\\
2.57	0\\
2.58	0\\
2.59	0\\
2.6	0\\
2.61	0\\
2.62	0\\
2.63	0\\
2.64	0\\
2.65	0\\
2.66	0\\
2.67	0\\
2.68	0\\
2.69	0\\
2.7	0\\
2.71	0\\
2.72	0\\
2.73	0\\
2.74	0\\
2.75	0\\
2.76	0\\
2.77	0\\
2.78	0\\
2.79	0\\
2.8	0\\
2.81	0\\
2.82	0\\
2.83	0\\
2.84	0\\
2.85	0\\
2.86	0\\
2.87	0\\
2.88	0\\
2.89	0\\
2.9	0\\
2.91	0\\
2.92	0\\
2.93	0\\
2.94	0\\
2.95	0\\
2.96	0\\
2.97	0\\
2.98	0\\
2.99	0\\
3	0\\
3.01	0\\
3.02	0\\
3.03	0\\
3.04	0\\
3.05	0\\
3.06	0\\
3.07	0\\
3.08	0\\
3.09	0\\
3.1	0\\
3.11	0\\
3.12	0\\
3.13	0\\
3.14	0\\
3.15	0\\
3.16	0\\
3.17	0\\
3.18	0\\
3.19	0\\
3.2	0\\
3.21	0\\
3.22	0\\
3.23	0\\
3.24	0\\
3.25	0\\
3.26	0\\
3.27	0\\
3.28	0\\
3.29	0\\
3.3	0\\
3.31	0\\
3.32	0\\
3.33	0\\
3.34	0\\
3.35	0\\
3.36	0\\
3.37	0\\
3.38	0\\
3.39	0\\
3.4	0\\
3.41	0\\
3.42	0\\
3.43	0\\
3.44	0\\
3.45	0\\
3.46	0\\
3.47	0\\
3.48	0\\
3.49	0\\
3.5	0\\
3.51	0\\
3.52	0\\
3.53	0\\
3.54	0\\
3.55	0\\
3.56	0\\
3.57	0\\
3.58	0\\
3.59	0\\
3.6	0\\
3.61	0\\
3.62	0\\
3.63	0\\
3.64	0\\
3.65	0\\
3.66	0\\
3.67	0\\
3.68	0\\
3.69	0\\
3.7	0\\
3.71	0\\
3.72	0\\
3.73	0\\
3.74	0\\
3.75	0\\
3.76	0\\
3.77	0\\
3.78	0\\
3.79	0\\
3.8	0\\
3.81	0\\
3.82	0\\
3.83	0\\
3.84	0\\
3.85	0\\
3.86	0\\
3.87	0\\
3.88	0\\
3.89	0\\
3.9	0\\
3.91	0\\
3.92	0\\
3.93	0\\
3.94	0\\
3.95	0\\
3.96	0\\
3.97	0\\
3.98	0\\
3.99	0\\
4	0\\
4.01	0\\
4.02	0\\
4.03	0\\
4.04	0\\
4.05	0\\
4.06	0\\
4.07	0\\
4.08	0\\
4.09	0\\
4.1	0\\
4.11	0\\
4.12	0\\
4.13	0\\
4.14	0\\
4.15	0\\
4.16	0\\
4.17	0\\
4.18	0\\
4.19	0\\
4.2	0\\
4.21	0\\
4.22	0\\
4.23	0\\
4.24	0\\
4.25	0\\
4.26	0\\
4.27	0\\
4.28	0\\
4.29	0\\
4.3	0\\
4.31	0\\
4.32	0\\
4.33	0\\
4.34	0\\
4.35	0\\
4.36	0\\
4.37	0\\
4.38	0\\
4.39	0\\
4.4	0\\
4.41	0\\
4.42	0\\
4.43	0\\
4.44	0\\
4.45	0\\
4.46	0\\
4.47	0\\
4.48	0\\
4.49	0\\
4.5	0\\
4.51	0\\
4.52	0\\
4.53	0\\
4.54	0\\
4.55	0\\
4.56	0\\
4.57	0\\
4.58	0\\
4.59	0\\
4.6	0\\
4.61	0\\
4.62	0\\
4.63	0\\
4.64	0\\
4.65	0\\
4.66	0\\
4.67	0\\
4.68	0\\
4.69	0\\
4.7	0\\
4.71	0\\
4.72	0\\
4.73	0\\
4.74	0\\
4.75	0\\
4.76	0\\
4.77	0\\
4.78	0\\
4.79	0\\
4.8	0\\
4.81	0\\
4.82	0\\
4.83	0\\
4.84	0\\
4.85	0\\
4.86	0\\
4.87	0\\
4.88	0\\
4.89	0\\
4.9	0\\
4.91	0\\
4.92	0\\
4.93	0\\
4.94	0\\
4.95	0\\
4.96	0\\
4.97	0\\
4.98	0\\
4.99	0\\
5	0\\
5.01	0\\
5.02	0\\
5.03	0\\
5.04	0\\
5.05	0\\
5.06	0\\
5.07	0\\
5.08	0\\
5.09	0\\
5.1	0\\
5.11	0\\
5.12	0\\
5.13	0\\
5.14	0\\
5.15	0\\
5.16	0\\
5.17	0\\
5.18	0\\
5.19	0\\
5.2	0\\
5.21	0\\
5.22	0\\
5.23	0\\
5.24	0\\
5.25	0\\
5.26	0\\
5.27	0\\
5.28	0\\
5.29	0\\
5.3	0\\
5.31	0\\
5.32	0\\
5.33	0\\
5.34	0\\
5.35	0\\
5.36	0\\
5.37	0\\
5.38	0\\
5.39	0\\
5.4	0\\
5.41	0\\
5.42	0\\
5.43	0\\
5.44	0\\
5.45	0\\
5.46	0\\
5.47	0\\
5.48	0\\
5.49	0\\
5.5	0\\
5.51	0\\
5.52	0\\
5.53	0\\
5.54	0\\
5.55	0\\
5.56	0\\
5.57	0\\
5.58	0\\
5.59	0\\
5.6	0\\
5.61	0\\
5.62	0\\
5.63	0\\
5.64	0\\
5.65	0\\
5.66	0\\
5.67	0\\
5.68	0\\
5.69	0\\
5.7	0\\
5.71	0\\
5.72	0\\
5.73	0\\
5.74	0\\
5.75	0\\
5.76	0\\
5.77	0\\
5.78	0\\
5.79	0\\
5.8	0\\
5.81	0\\
5.82	0\\
5.83	0\\
5.84	0\\
5.85	0\\
5.86	0\\
5.87	0\\
5.88	0\\
5.89	0\\
5.9	0\\
5.91	0\\
5.92	0\\
5.93	0\\
5.94	0\\
5.95	0\\
5.96	0\\
5.97	0\\
5.98	0\\
5.99	0\\
6	0\\
6.01	0\\
6.02	0\\
6.03	0\\
6.04	0\\
6.05	0\\
6.06	0\\
6.07	0\\
6.08	0\\
6.09	0\\
6.1	0\\
6.11	0\\
6.12	0\\
6.13	0\\
6.14	0\\
6.15	0\\
6.16	0\\
6.17	0\\
6.18	0\\
6.19	0\\
6.2	0\\
6.21	0\\
6.22	0\\
6.23	0\\
6.24	0\\
6.25	0\\
6.26	0\\
6.27	0\\
6.28	0\\
6.29	0\\
6.3	0\\
6.31	0\\
6.32	0\\
6.33	0\\
6.34	0\\
6.35	0\\
6.36	0\\
6.37	0\\
6.38	0\\
6.39	0\\
6.4	0\\
6.41	0\\
6.42	0\\
6.43	0\\
6.44	0\\
6.45	0\\
6.46	0\\
6.47	0\\
6.48	0\\
6.49	0\\
6.5	0\\
6.51	0\\
6.52	0\\
6.53	0\\
6.54	0\\
6.55	0\\
6.56	0\\
6.57	0\\
6.58	0\\
6.59	0\\
6.6	0\\
6.61	0\\
6.62	0\\
6.63	0\\
6.64	0\\
6.65	0\\
6.66	0\\
6.67	0\\
6.68	0\\
6.69	0\\
6.7	0\\
6.71	0\\
6.72	0\\
6.73	0\\
6.74	0\\
6.75	0\\
6.76	0\\
6.77	0\\
6.78	0\\
6.79	0\\
6.8	0\\
6.81	0\\
6.82	0\\
6.83	0\\
6.84	0\\
6.85	0\\
6.86	0\\
6.87	0\\
6.88	0\\
6.89	0\\
6.9	0\\
6.91	0\\
6.92	0\\
6.93	0\\
6.94	0\\
6.95	0\\
6.96	0\\
6.97	0\\
6.98	0\\
6.99	0\\
7	0\\
7.01	0\\
7.02	0\\
7.03	0\\
7.04	0\\
7.05	0\\
7.06	0\\
7.07	0\\
7.08	0\\
7.09	0\\
7.1	0\\
7.11	0\\
7.12	0\\
7.13	0\\
7.14	0\\
7.15	0\\
7.16	0\\
7.17	0\\
7.18	0\\
7.19	0\\
7.2	0\\
7.21	0\\
7.22	0\\
7.23	0\\
7.24	0\\
7.25	0\\
7.26	0\\
7.27	0\\
7.28	0\\
7.29	0\\
7.3	0\\
7.31	0\\
7.32	0\\
7.33	0\\
7.34	0\\
7.35	0\\
7.36	0\\
7.37	0\\
7.38	0\\
7.39	0\\
7.4	0\\
7.41	0\\
7.42	0\\
7.43	0\\
7.44	0\\
7.45	0\\
7.46	0\\
7.47	0\\
7.48	0\\
7.49	0\\
7.5	0\\
7.51	0\\
7.52	0\\
7.53	0\\
7.54	0\\
7.55	0\\
7.56	0\\
7.57	0\\
7.58	0\\
7.59	0\\
7.6	0\\
7.61	0\\
7.62	0\\
7.63	0\\
7.64	0\\
7.65	0\\
7.66	0\\
7.67	0\\
7.68	0\\
7.69	0\\
7.7	0\\
7.71	0\\
7.72	0\\
7.73	0\\
7.74	0\\
7.75	0\\
7.76	0\\
7.77	0\\
7.78	0\\
7.79	0\\
7.8	0\\
7.81	0\\
7.82	0\\
7.83	0\\
7.84	0\\
7.85	0\\
7.86	0\\
7.87	0\\
7.88	0\\
7.89	0\\
7.9	0\\
7.91	0\\
7.92	0\\
7.93	0\\
7.94	0\\
7.95	0\\
7.96	0\\
7.97	0\\
7.98	0\\
7.99	0\\
8	0\\
8.01	0\\
8.02	0\\
8.03	0\\
8.04	0\\
8.05	0\\
8.06	0\\
8.07	0\\
8.08	0\\
8.09	0\\
8.1	0\\
8.11	0\\
8.12	0\\
8.13	0\\
8.14	0\\
8.15	0\\
8.16	0\\
8.17	0\\
8.18	0\\
8.19	0\\
8.2	0\\
8.21	0\\
8.22	0\\
8.23	0\\
8.24	0\\
8.25	0\\
8.26	0\\
8.27	0\\
8.28	0\\
8.29	0\\
8.3	0\\
8.31	0\\
8.32	0\\
8.33	0\\
8.34	0\\
8.35	0\\
8.36	0\\
8.37	0\\
8.38	0\\
8.39	0\\
8.4	0\\
8.41	0\\
8.42	0\\
8.43	0\\
8.44	0\\
8.45	0\\
8.46	0\\
8.47	0\\
8.48	0\\
8.49	0\\
8.5	0\\
8.51	0\\
8.52	0\\
8.53	0\\
8.54	0\\
8.55	0\\
8.56	0\\
8.57	0\\
8.58	0\\
8.59	0\\
8.6	0\\
8.61	0\\
8.62	0\\
8.63	0\\
8.64	0\\
8.65	0\\
8.66	0\\
8.67	0\\
8.68	0\\
8.69	0\\
8.7	0\\
8.71	0\\
8.72	0\\
8.73	0\\
8.74	0\\
8.75	0\\
8.76	0\\
8.77	0\\
8.78	0\\
8.79	0\\
8.8	0\\
8.81	0\\
8.82	0\\
8.83	0\\
8.84	0\\
8.85	0\\
8.86	0\\
8.87	0\\
8.88	0\\
8.89	0\\
8.9	0\\
8.91	0\\
8.92	0\\
8.93	0\\
8.94	0\\
8.95	0\\
8.96	0\\
8.97	0\\
8.98	0\\
8.99	0\\
9	0\\
9.01	0\\
9.02	0\\
9.03	0\\
9.04	0\\
9.05	0\\
9.06	0\\
9.07	0\\
9.08	0\\
9.09	0\\
9.1	0\\
9.11	0\\
9.12	0\\
9.13	0\\
9.14	0\\
9.15	0\\
9.16	0\\
9.17	0\\
9.18	0\\
9.19	0\\
9.2	0\\
9.21	0\\
9.22	0\\
9.23	0\\
9.24	0\\
9.25	0\\
9.26	0\\
9.27	0\\
9.28	0\\
9.29	0\\
9.3	0\\
9.31	0\\
9.32	0\\
9.33	0\\
9.34	0\\
9.35	0\\
9.36	0\\
9.37	0\\
9.38	0\\
9.39	0\\
9.4	0\\
9.41	0\\
9.42	0\\
9.43	0\\
9.44	0\\
9.45	0\\
9.46	0\\
9.47	0\\
9.48	0\\
9.49	0\\
9.5	0\\
9.51	0\\
9.52	0\\
9.53	0\\
9.54	0\\
9.55	0\\
9.56	0\\
9.57	0\\
9.58	0\\
9.59	0\\
9.6	0\\
9.61	0\\
9.62	0\\
9.63	0\\
9.64	0\\
9.65	0\\
9.66	0\\
9.67	0\\
9.68	0\\
9.69	0\\
9.7	0\\
9.71	0\\
9.72	0\\
9.73	0\\
9.74	0\\
9.75	0\\
9.76	0\\
9.77	0\\
9.78	0\\
9.79	0\\
9.8	0\\
9.81	0\\
9.82	0\\
9.83	0\\
9.84	0\\
9.85	0\\
9.86	0\\
9.87	0\\
9.88	0\\
9.89	0\\
9.9	0\\
9.91	0\\
9.92	0\\
9.93	0\\
9.94	0\\
9.95	0\\
9.96	0\\
9.97	0\\
9.98	0\\
9.99	0\\
10	0\\
10.01	0\\
10.02	0\\
10.03	0\\
10.04	0\\
10.05	0\\
10.06	0\\
10.07	0\\
10.08	0\\
10.09	0\\
10.1	0\\
10.11	0\\
10.12	0\\
10.13	0\\
10.14	0\\
10.15	0\\
10.16	0\\
10.17	0\\
10.18	0\\
10.19	0\\
10.2	0\\
10.21	0\\
10.22	0\\
10.23	0\\
10.24	0\\
10.25	0\\
10.26	0\\
10.27	0\\
10.28	0\\
10.29	0\\
10.3	0\\
10.31	0\\
10.32	0\\
10.33	0\\
10.34	0\\
10.35	0\\
10.36	0\\
10.37	0\\
10.38	0\\
10.39	0\\
10.4	0\\
10.41	0\\
10.42	0\\
10.43	0\\
10.44	0\\
10.45	0\\
10.46	0\\
10.47	0\\
10.48	0\\
10.49	0\\
10.5	0\\
10.51	0\\
10.52	0\\
10.53	0\\
10.54	0\\
10.55	0\\
10.56	0\\
10.57	0\\
10.58	0\\
10.59	0\\
10.6	0\\
10.61	0\\
10.62	0\\
10.63	0\\
10.64	0\\
10.65	0\\
10.66	0\\
10.67	0\\
10.68	0\\
10.69	0\\
10.7	0\\
10.71	0\\
10.72	0\\
10.73	0\\
10.74	0\\
10.75	0\\
10.76	0\\
10.77	0\\
10.78	0\\
10.79	0\\
10.8	0\\
10.81	0\\
10.82	0\\
10.83	0\\
10.84	0\\
10.85	0\\
10.86	0\\
10.87	0\\
10.88	0\\
10.89	0\\
10.9	0\\
10.91	0\\
10.92	0\\
10.93	0\\
10.94	0\\
10.95	0\\
10.96	0\\
10.97	0\\
10.98	0\\
10.99	0\\
11	0\\
11.01	0\\
11.02	0\\
11.03	0\\
11.04	0\\
11.05	0\\
11.06	0\\
11.07	0\\
11.08	0\\
11.09	0\\
11.1	0\\
11.11	0\\
11.12	0\\
11.13	0\\
11.14	0\\
11.15	0\\
11.16	0\\
11.17	0\\
11.18	0\\
11.19	0\\
11.2	0\\
11.21	0\\
11.22	0\\
11.23	0\\
11.24	0\\
11.25	0\\
11.26	0\\
11.27	0\\
11.28	0\\
11.29	0\\
11.3	0\\
11.31	0\\
11.32	0\\
11.33	0\\
11.34	0\\
11.35	0\\
11.36	0\\
11.37	0\\
11.38	0\\
11.39	0\\
11.4	0\\
11.41	0\\
11.42	0\\
11.43	0\\
11.44	0\\
11.45	0\\
11.46	0\\
11.47	0\\
11.48	0\\
11.49	0\\
11.5	0\\
11.51	0\\
11.52	0\\
11.53	0\\
11.54	0\\
11.55	0\\
11.56	0\\
11.57	0\\
11.58	0\\
11.59	0\\
11.6	0\\
11.61	0\\
11.62	0\\
11.63	0\\
11.64	0\\
11.65	0\\
11.66	0\\
11.67	0\\
11.68	0\\
11.69	0\\
11.7	0\\
11.71	0\\
11.72	0\\
11.73	0\\
11.74	0\\
11.75	0\\
11.76	0\\
11.77	0\\
11.78	0\\
11.79	0\\
11.8	0\\
11.81	0\\
11.82	0\\
11.83	0\\
11.84	0\\
11.85	0\\
11.86	0\\
11.87	0\\
11.88	0\\
11.89	0\\
11.9	0\\
11.91	0\\
11.92	0\\
11.93	0\\
11.94	0\\
11.95	0\\
11.96	0\\
11.97	0\\
11.98	0\\
11.99	0\\
12	0\\
12.01	0\\
12.02	0\\
12.03	0\\
12.04	0\\
12.05	0\\
12.06	0\\
12.07	0\\
12.08	0\\
12.09	0\\
12.1	0\\
12.11	0\\
12.12	0\\
12.13	0\\
12.14	0\\
12.15	0\\
12.16	0\\
12.17	0\\
12.18	0\\
12.19	0\\
12.2	0\\
12.21	0\\
12.22	0\\
12.23	0\\
12.24	0\\
12.25	0\\
12.26	0\\
12.27	0\\
12.28	0\\
12.29	0\\
12.3	0\\
12.31	0\\
12.32	0\\
12.33	0\\
12.34	0\\
12.35	0\\
12.36	0\\
12.37	0\\
12.38	0\\
12.39	0\\
12.4	0\\
12.41	0\\
12.42	0\\
12.43	0\\
12.44	0\\
12.45	0\\
12.46	0\\
12.47	0\\
12.48	0\\
12.49	0\\
12.5	0\\
12.51	0\\
12.52	0\\
12.53	0\\
12.54	0\\
12.55	0\\
12.56	0\\
12.57	0\\
12.58	0\\
12.59	0\\
12.6	0\\
12.61	0\\
12.62	0\\
12.63	0\\
12.64	0\\
12.65	0\\
12.66	0\\
12.67	0\\
12.68	0\\
12.69	0\\
12.7	0\\
12.71	0\\
12.72	0\\
12.73	0\\
12.74	0\\
12.75	0\\
12.76	0\\
12.77	0\\
12.78	0\\
12.79	0\\
12.8	0\\
12.81	0\\
12.82	0\\
12.83	0\\
12.84	0\\
12.85	0\\
12.86	0\\
12.87	0\\
12.88	0\\
12.89	0\\
12.9	0\\
12.91	0\\
12.92	0\\
12.93	0\\
12.94	0\\
12.95	0\\
12.96	0\\
12.97	0\\
12.98	0\\
12.99	0\\
13	0\\
13.01	0\\
13.02	0\\
13.03	0\\
13.04	0\\
13.05	0\\
13.06	0\\
13.07	0\\
13.08	0\\
13.09	0\\
13.1	0\\
13.11	0\\
13.12	0\\
13.13	0\\
13.14	0\\
13.15	0\\
13.16	0\\
13.17	0\\
13.18	0\\
13.19	0\\
13.2	0\\
13.21	0\\
13.22	0\\
13.23	0\\
13.24	0\\
13.25	0\\
13.26	0\\
13.27	0\\
13.28	0\\
13.29	0\\
13.3	0\\
13.31	0\\
13.32	0\\
13.33	0\\
13.34	0\\
13.35	0\\
13.36	0\\
13.37	0\\
13.38	0\\
13.39	0\\
13.4	0\\
13.41	0\\
13.42	0\\
13.43	0\\
13.44	0\\
13.45	0\\
13.46	0\\
13.47	0\\
13.48	0\\
13.49	0\\
13.5	0\\
13.51	0\\
13.52	0\\
13.53	0\\
13.54	0\\
13.55	0\\
13.56	0\\
13.57	0\\
13.58	0\\
13.59	0\\
13.6	0\\
13.61	0\\
13.62	0\\
13.63	0\\
13.64	0\\
13.65	0\\
13.66	0\\
13.67	0\\
13.68	0\\
13.69	0\\
13.7	0\\
13.71	0\\
13.72	0\\
13.73	0\\
13.74	0\\
13.75	0\\
13.76	0\\
13.77	0\\
13.78	0\\
13.79	0\\
13.8	0\\
13.81	0\\
13.82	0\\
13.83	0\\
13.84	0\\
13.85	0\\
13.86	0\\
13.87	0\\
13.88	0\\
13.89	0\\
13.9	0\\
13.91	0\\
13.92	0\\
13.93	0\\
13.94	0\\
13.95	0\\
13.96	0\\
13.97	0\\
13.98	0\\
13.99	0\\
14	0\\
14.01	0\\
14.02	0\\
14.03	0\\
14.04	0\\
14.05	0\\
14.06	0\\
14.07	0\\
14.08	0\\
14.09	0\\
14.1	0\\
14.11	0\\
14.12	0\\
14.13	0\\
14.14	0\\
14.15	0\\
14.16	0\\
14.17	0\\
14.18	0\\
14.19	0\\
14.2	0\\
14.21	0\\
14.22	0\\
14.23	0\\
14.24	0\\
14.25	0\\
14.26	0\\
14.27	0\\
14.28	0\\
14.29	0\\
14.3	0\\
14.31	0\\
14.32	0\\
14.33	0\\
14.34	0\\
14.35	0\\
14.36	0\\
14.37	0\\
14.38	0\\
14.39	0\\
14.4	0\\
14.41	0\\
14.42	0\\
14.43	0\\
14.44	0\\
14.45	0\\
14.46	0\\
14.47	0\\
14.48	0\\
14.49	0\\
14.5	0\\
14.51	0\\
14.52	0\\
14.53	0\\
14.54	0\\
14.55	0\\
14.56	0\\
14.57	0\\
14.58	0\\
14.59	0\\
14.6	0\\
14.61	0\\
14.62	0\\
14.63	0\\
14.64	0\\
14.65	0\\
14.66	0\\
14.67	0\\
14.68	0\\
14.69	0\\
14.7	0\\
14.71	0\\
14.72	0\\
14.73	0\\
14.74	0\\
14.75	0\\
14.76	0\\
14.77	0\\
14.78	0\\
14.79	0\\
14.8	0\\
14.81	0\\
14.82	0\\
14.83	0\\
14.84	0\\
14.85	0\\
14.86	0\\
14.87	0\\
14.88	0\\
14.89	0\\
14.9	0\\
14.91	0\\
14.92	0\\
14.93	0\\
14.94	0\\
14.95	0\\
14.96	0\\
14.97	0\\
14.98	0\\
14.99	0\\
15	0\\
15.01	0\\
15.02	0\\
15.03	0\\
15.04	0\\
15.05	0\\
15.06	0\\
15.07	0\\
15.08	0\\
15.09	0\\
15.1	0\\
15.11	0\\
15.12	0\\
15.13	0\\
15.14	0\\
15.15	0\\
15.16	0\\
15.17	0\\
15.18	0\\
15.19	0\\
15.2	0\\
15.21	0\\
15.22	0\\
15.23	0\\
15.24	0\\
15.25	0\\
15.26	0\\
15.27	0\\
15.28	0\\
15.29	0\\
15.3	0\\
15.31	0\\
15.32	0\\
15.33	0\\
15.34	0\\
15.35	0\\
15.36	0\\
15.37	0\\
15.38	0\\
15.39	0\\
15.4	0\\
15.41	0\\
15.42	0\\
15.43	0\\
15.44	0\\
15.45	0\\
15.46	0\\
15.47	0\\
15.48	0\\
15.49	0\\
15.5	0\\
15.51	0\\
15.52	0\\
15.53	0\\
15.54	0\\
15.55	0\\
15.56	0\\
15.57	0\\
15.58	0\\
15.59	0\\
15.6	0\\
15.61	0\\
15.62	0\\
15.63	0\\
15.64	0\\
15.65	0\\
15.66	0\\
15.67	0\\
15.68	0\\
15.69	0\\
15.7	0\\
15.71	0\\
15.72	0\\
15.73	0\\
15.74	0\\
15.75	0\\
15.76	0\\
15.77	0\\
15.78	0\\
15.79	0\\
15.8	0\\
15.81	0\\
15.82	0\\
15.83	0\\
15.84	0\\
15.85	0\\
15.86	0\\
15.87	0\\
15.88	0\\
15.89	0\\
15.9	0\\
15.91	0\\
15.92	0\\
15.93	0\\
15.94	0\\
15.95	0\\
15.96	0\\
15.97	0\\
15.98	0\\
15.99	0\\
16	0\\
16.01	0\\
16.02	0\\
16.03	0\\
16.04	0\\
16.05	0\\
16.06	0\\
16.07	0\\
16.08	0\\
16.09	0\\
16.1	0\\
16.11	0\\
16.12	0\\
16.13	0\\
16.14	0\\
16.15	0\\
16.16	0\\
16.17	0\\
16.18	0\\
16.19	0\\
16.2	0\\
16.21	0\\
16.22	0\\
16.23	0\\
16.24	0\\
16.25	0\\
16.26	0\\
16.27	0\\
16.28	0\\
16.29	0\\
16.3	0\\
16.31	0\\
16.32	0\\
16.33	0\\
16.34	0\\
16.35	0\\
16.36	0\\
16.37	0\\
16.38	0\\
16.39	0\\
16.4	0\\
16.41	0\\
16.42	0\\
16.43	0\\
16.44	0\\
16.45	0\\
16.46	0\\
16.47	0\\
16.48	0\\
16.49	0\\
16.5	0\\
16.51	0\\
16.52	0\\
16.53	0\\
16.54	0\\
16.55	0\\
16.56	0\\
16.57	0\\
16.58	0\\
16.59	0\\
16.6	0\\
16.61	0\\
16.62	0\\
16.63	0\\
16.64	0\\
16.65	0\\
16.66	0\\
16.67	0\\
16.68	0\\
16.69	0\\
16.7	0\\
16.71	0\\
16.72	0\\
16.73	0\\
16.74	0\\
16.75	0\\
16.76	0\\
16.77	0\\
16.78	0\\
16.79	0\\
16.8	0\\
16.81	0\\
16.82	0\\
16.83	0\\
16.84	0\\
16.85	0\\
16.86	0\\
16.87	0\\
16.88	0\\
16.89	0\\
16.9	0\\
16.91	0\\
16.92	0\\
16.93	0\\
16.94	0\\
16.95	0\\
16.96	0\\
16.97	0\\
16.98	0\\
16.99	0\\
17	0\\
17.01	0\\
17.02	0\\
17.03	0\\
17.04	0\\
17.05	0\\
17.06	0\\
17.07	0\\
17.08	0\\
17.09	0\\
17.1	0\\
17.11	0\\
17.12	0\\
17.13	0\\
17.14	0\\
17.15	0\\
17.16	0\\
17.17	0\\
17.18	0\\
17.19	0\\
17.2	0\\
17.21	0\\
17.22	0\\
17.23	0\\
17.24	0\\
17.25	0\\
17.26	0\\
17.27	0\\
17.28	0\\
17.29	0\\
17.3	0\\
17.31	0\\
17.32	0\\
17.33	0\\
17.34	0\\
17.35	0\\
17.36	0\\
17.37	0\\
17.38	0\\
17.39	0\\
17.4	0\\
17.41	0\\
17.42	0\\
17.43	0\\
17.44	0\\
17.45	0\\
17.46	0\\
17.47	0\\
17.48	0\\
17.49	0\\
17.5	0\\
17.51	0\\
17.52	0\\
17.53	0\\
17.54	0\\
17.55	0\\
17.56	0\\
17.57	0\\
17.58	0\\
17.59	0\\
17.6	0\\
17.61	0\\
17.62	0\\
17.63	0\\
17.64	0\\
17.65	0\\
17.66	0\\
17.67	0\\
17.68	0\\
17.69	0\\
17.7	0\\
17.71	0\\
17.72	0\\
17.73	0\\
17.74	0\\
17.75	0\\
17.76	0\\
17.77	0\\
17.78	0\\
17.79	0\\
17.8	0\\
17.81	0\\
17.82	0\\
17.83	0\\
17.84	0\\
17.85	0\\
17.86	0\\
17.87	0\\
17.88	0\\
17.89	0\\
17.9	0\\
17.91	0\\
17.92	0\\
17.93	0\\
17.94	0\\
17.95	0\\
17.96	0\\
17.97	0\\
17.98	0\\
17.99	0\\
18	0\\
18.01	0\\
18.02	0\\
18.03	0\\
18.04	0\\
18.05	0\\
18.06	0\\
18.07	0\\
18.08	0\\
18.09	0\\
18.1	0\\
18.11	0\\
18.12	0\\
18.13	0\\
18.14	0\\
18.15	0\\
18.16	0\\
18.17	0\\
18.18	0\\
18.19	0\\
18.2	0\\
18.21	0\\
18.22	0\\
18.23	0\\
18.24	0\\
18.25	0\\
18.26	0\\
18.27	0\\
18.28	0\\
18.29	0\\
18.3	0\\
18.31	0\\
18.32	0\\
18.33	0\\
18.34	0\\
18.35	0\\
18.36	0\\
18.37	0\\
18.38	0\\
18.39	0\\
18.4	0\\
18.41	0\\
18.42	0\\
18.43	0\\
18.44	0\\
18.45	0\\
18.46	0\\
18.47	0\\
18.48	0\\
18.49	0\\
18.5	0\\
18.51	0\\
18.52	0\\
18.53	0\\
18.54	0\\
18.55	0\\
18.56	0\\
18.57	0\\
18.58	0\\
18.59	0\\
18.6	0\\
18.61	0\\
18.62	0\\
18.63	0\\
18.64	0\\
18.65	0\\
18.66	0\\
18.67	0\\
18.68	0\\
18.69	0\\
18.7	0\\
18.71	0\\
18.72	0\\
18.73	0\\
18.74	0\\
18.75	0\\
18.76	0\\
18.77	0\\
18.78	0\\
18.79	0\\
18.8	0\\
18.81	0\\
18.82	0\\
18.83	0\\
18.84	0\\
18.85	0\\
18.86	0\\
18.87	0\\
18.88	0\\
18.89	0\\
18.9	0\\
18.91	0\\
18.92	0\\
18.93	0\\
18.94	0\\
18.95	0\\
18.96	0\\
18.97	0\\
18.98	0\\
18.99	0\\
19	0\\
19.01	0\\
19.02	0\\
19.03	0\\
19.04	0\\
19.05	0\\
19.06	0\\
19.07	0\\
19.08	0\\
19.09	0\\
19.1	0\\
19.11	0\\
19.12	0\\
19.13	0\\
19.14	0\\
19.15	0\\
19.16	0\\
19.17	0\\
19.18	0\\
19.19	0\\
19.2	0\\
19.21	0\\
19.22	0\\
19.23	0\\
19.24	0\\
19.25	0\\
19.26	0\\
19.27	0\\
19.28	0\\
19.29	0\\
19.3	0\\
19.31	0\\
19.32	0\\
19.33	0\\
19.34	0\\
19.35	0\\
19.36	0\\
19.37	0\\
19.38	0\\
19.39	0\\
19.4	0\\
19.41	0\\
19.42	0\\
19.43	0\\
19.44	0\\
19.45	0\\
19.46	0\\
19.47	0\\
19.48	0\\
19.49	0\\
19.5	0\\
19.51	0\\
19.52	0\\
19.53	0\\
19.54	0\\
19.55	0\\
19.56	0\\
19.57	0\\
19.58	0\\
19.59	0\\
19.6	0\\
19.61	0\\
19.62	0\\
19.63	0\\
19.64	0\\
19.65	0\\
19.66	0\\
19.67	0\\
19.68	0\\
19.69	0\\
19.7	0\\
19.71	0\\
19.72	0\\
19.73	0\\
19.74	0\\
19.75	0\\
19.76	0\\
19.77	0\\
19.78	0\\
19.79	0\\
19.8	0\\
19.81	0\\
19.82	0\\
19.83	0\\
19.84	0\\
19.85	0\\
19.86	0\\
19.87	0\\
19.88	0\\
19.89	0\\
19.9	0\\
19.91	0\\
19.92	0\\
19.93	0\\
19.94	0\\
19.95	0\\
19.96	0\\
19.97	0\\
19.98	0\\
19.99	0\\
20	0\\
20.01	0\\
20.02	0\\
20.03	0\\
20.04	0\\
20.05	0\\
20.06	0\\
20.07	0\\
20.08	0\\
20.09	0\\
20.1	0\\
20.11	0\\
20.12	0\\
20.13	0\\
20.14	0\\
20.15	0\\
20.16	0\\
20.17	0\\
20.18	0\\
20.19	0\\
20.2	0\\
20.21	0\\
20.22	0\\
20.23	0\\
20.24	0\\
20.25	0\\
20.26	0\\
20.27	0\\
20.28	0\\
20.29	0\\
20.3	0\\
20.31	0\\
20.32	0\\
20.33	0\\
20.34	0\\
20.35	0\\
20.36	0\\
20.37	0\\
20.38	0\\
20.39	0\\
20.4	0\\
20.41	0\\
20.42	0\\
20.43	0\\
20.44	0\\
20.45	0\\
20.46	0\\
20.47	0\\
20.48	0\\
20.49	0\\
20.5	0\\
20.51	0\\
20.52	0\\
20.53	0\\
20.54	0\\
20.55	0\\
20.56	0\\
20.57	0\\
20.58	0\\
20.59	0\\
20.6	0\\
20.61	0\\
20.62	0\\
20.63	0\\
20.64	0\\
20.65	0\\
20.66	0\\
20.67	0\\
20.68	0\\
20.69	0\\
20.7	0\\
20.71	0\\
20.72	0\\
20.73	0\\
20.74	0\\
20.75	0\\
20.76	0\\
20.77	0\\
20.78	0\\
20.79	0\\
20.8	0\\
20.81	0\\
20.82	0\\
20.83	0\\
20.84	0\\
20.85	0\\
20.86	0\\
20.87	0\\
20.88	0\\
20.89	0\\
20.9	0\\
20.91	0\\
20.92	0\\
20.93	0\\
20.94	0\\
20.95	0\\
20.96	0\\
20.97	0\\
20.98	0\\
20.99	0\\
21	0\\
21.01	0\\
21.02	0\\
21.03	0\\
21.04	0\\
21.05	0\\
21.06	0\\
21.07	0\\
21.08	0\\
21.09	0\\
21.1	0\\
21.11	0\\
21.12	0\\
21.13	0\\
21.14	0\\
21.15	0\\
21.16	0\\
21.17	0\\
21.18	0\\
21.19	0\\
21.2	0\\
21.21	0\\
21.22	0\\
21.23	0\\
21.24	0\\
21.25	0\\
21.26	0\\
21.27	0\\
21.28	0\\
21.29	0\\
21.3	0\\
21.31	0\\
21.32	0\\
21.33	0\\
21.34	0\\
21.35	0\\
21.36	0\\
21.37	0\\
21.38	0\\
21.39	0\\
21.4	0\\
21.41	0\\
21.42	0\\
21.43	0\\
21.44	0\\
21.45	0\\
21.46	0\\
21.47	0\\
21.48	0\\
21.49	0\\
21.5	0\\
21.51	0\\
21.52	0\\
21.53	0\\
21.54	0\\
21.55	0\\
21.56	0\\
21.57	0\\
21.58	0\\
21.59	0\\
21.6	0\\
21.61	0\\
21.62	0\\
21.63	0\\
21.64	0\\
21.65	0\\
21.66	0\\
21.67	0\\
21.68	0\\
21.69	0\\
21.7	0\\
21.71	0\\
21.72	0\\
21.73	0\\
21.74	0\\
21.75	0\\
21.76	0\\
21.77	0\\
21.78	0\\
21.79	0\\
21.8	0\\
21.81	0\\
21.82	0\\
21.83	0\\
21.84	0\\
21.85	0\\
21.86	0\\
21.87	0\\
21.88	0\\
21.89	0\\
21.9	0\\
21.91	0\\
21.92	0\\
21.93	0\\
21.94	0\\
21.95	0\\
21.96	0\\
21.97	0\\
21.98	0\\
21.99	0\\
22	0\\
22.01	0\\
22.02	0\\
22.03	0\\
22.04	0\\
22.05	0\\
22.06	0\\
22.07	0\\
22.08	0\\
22.09	0\\
22.1	0\\
22.11	0\\
22.12	0\\
22.13	0\\
22.14	0\\
22.15	0\\
22.16	0\\
22.17	0\\
22.18	0\\
22.19	0\\
22.2	0\\
22.21	0\\
22.22	0\\
22.23	0\\
22.24	0\\
22.25	0\\
22.26	0\\
22.27	0\\
22.28	0\\
22.29	0\\
22.3	0\\
22.31	0\\
22.32	0\\
22.33	0\\
22.34	0\\
22.35	0\\
22.36	0\\
22.37	0\\
22.38	0\\
22.39	0\\
22.4	0\\
22.41	0\\
22.42	0\\
22.43	0\\
22.44	0\\
22.45	0\\
22.46	0\\
22.47	0\\
22.48	0\\
22.49	0\\
22.5	0\\
22.51	0\\
22.52	0\\
22.53	0\\
22.54	0\\
22.55	0\\
22.56	0\\
22.57	0\\
22.58	0\\
22.59	0\\
22.6	0\\
22.61	0\\
22.62	0\\
22.63	0\\
22.64	0\\
22.65	0\\
22.66	0\\
22.67	0\\
22.68	0\\
22.69	0\\
22.7	0\\
22.71	0\\
22.72	0\\
22.73	0\\
22.74	0\\
22.75	0\\
22.76	0\\
22.77	0\\
22.78	0\\
22.79	0\\
22.8	0\\
22.81	0\\
22.82	0\\
22.83	0\\
22.84	0\\
22.85	0\\
22.86	0\\
22.87	0\\
22.88	0\\
22.89	0\\
22.9	0\\
22.91	0\\
22.92	0\\
22.93	0\\
22.94	0\\
22.95	0\\
22.96	0\\
22.97	0\\
22.98	0\\
22.99	0\\
23	0\\
23.01	0\\
23.02	0\\
23.03	0\\
23.04	0\\
23.05	0\\
23.06	0\\
23.07	0\\
23.08	0\\
23.09	0\\
23.1	0\\
23.11	0\\
23.12	0\\
23.13	0\\
23.14	0\\
23.15	0\\
23.16	0\\
23.17	0\\
23.18	0\\
23.19	0\\
23.2	0\\
23.21	0\\
23.22	0\\
23.23	0\\
23.24	0\\
23.25	0\\
23.26	0\\
23.27	0\\
23.28	0\\
23.29	0\\
23.3	0\\
23.31	0\\
23.32	0\\
23.33	0\\
23.34	0\\
23.35	0\\
23.36	0\\
23.37	0\\
23.38	0\\
23.39	0\\
23.4	0\\
23.41	0\\
23.42	0\\
23.43	0\\
23.44	0\\
23.45	0\\
23.46	0\\
23.47	0\\
23.48	0\\
23.49	0\\
23.5	0\\
23.51	0\\
23.52	0\\
23.53	0\\
23.54	0\\
23.55	0\\
23.56	0\\
23.57	0\\
23.58	0\\
23.59	0\\
23.6	0\\
23.61	0\\
23.62	0\\
23.63	0\\
23.64	0\\
23.65	0\\
23.66	0\\
23.67	0\\
23.68	0\\
23.69	0\\
23.7	0\\
23.71	0\\
23.72	0\\
23.73	0\\
23.74	0\\
23.75	0\\
23.76	0\\
23.77	0\\
23.78	0\\
23.79	0\\
23.8	0\\
23.81	0\\
23.82	0\\
23.83	0\\
23.84	0\\
23.85	0\\
23.86	0\\
23.87	0\\
23.88	0\\
23.89	0\\
23.9	0\\
23.91	0\\
23.92	0\\
23.93	0\\
23.94	0\\
23.95	0\\
23.96	0\\
23.97	0\\
23.98	0\\
23.99	0\\
24	0\\
24.01	0\\
24.02	0\\
24.03	0\\
24.04	0\\
24.05	0\\
24.06	0\\
24.07	0\\
24.08	0\\
24.09	0\\
24.1	0\\
24.11	0\\
24.12	0\\
24.13	0\\
24.14	0\\
24.15	0\\
24.16	0\\
24.17	0\\
24.18	0\\
24.19	0\\
24.2	0\\
24.21	0\\
24.22	0\\
24.23	0\\
24.24	0\\
24.25	0\\
24.26	0\\
24.27	0\\
24.28	0\\
24.29	0\\
24.3	0\\
24.31	0\\
24.32	0\\
24.33	0\\
24.34	0\\
24.35	0\\
24.36	0\\
24.37	0\\
24.38	0\\
24.39	0\\
24.4	0\\
24.41	0\\
24.42	0\\
24.43	0\\
24.44	0\\
24.45	0\\
24.46	0\\
24.47	0\\
24.48	0\\
24.49	0\\
24.5	0\\
24.51	0\\
24.52	0\\
24.53	0\\
24.54	0\\
24.55	0\\
24.56	0\\
24.57	0\\
24.58	0\\
24.59	0\\
24.6	0\\
24.61	0\\
24.62	0\\
24.63	0\\
24.64	0\\
24.65	0\\
24.66	0\\
24.67	0\\
24.68	0\\
24.69	0\\
24.7	0\\
24.71	0\\
24.72	0\\
24.73	0\\
24.74	0\\
24.75	0\\
24.76	0\\
24.77	0\\
24.78	0\\
24.79	0\\
24.8	0\\
24.81	0\\
24.82	0\\
24.83	0\\
24.84	0\\
24.85	0\\
24.86	0\\
24.87	0\\
24.88	0\\
24.89	0\\
24.9	0\\
24.91	0\\
24.92	0\\
24.93	0\\
24.94	0\\
24.95	0\\
24.96	0\\
24.97	0\\
24.98	0\\
24.99	0\\
25	0\\
25.01	0\\
25.02	0\\
25.03	0\\
25.04	0\\
25.05	0\\
25.06	0\\
25.07	0\\
25.08	0\\
25.09	0\\
25.1	0\\
25.11	0\\
25.12	0\\
25.13	0\\
25.14	0\\
25.15	0\\
25.16	0\\
25.17	0\\
25.18	0\\
25.19	0\\
25.2	0\\
25.21	0\\
25.22	0\\
25.23	0\\
25.24	0\\
25.25	0\\
25.26	0\\
25.27	0\\
25.28	0\\
25.29	0\\
25.3	0\\
25.31	0\\
25.32	0\\
25.33	0\\
25.34	0\\
25.35	0\\
25.36	0\\
25.37	0\\
25.38	0\\
25.39	0\\
25.4	0\\
25.41	0\\
25.42	0\\
25.43	0\\
25.44	0\\
25.45	0\\
25.46	0\\
25.47	0\\
25.48	0\\
25.49	0\\
25.5	0\\
25.51	0\\
25.52	0\\
25.53	0\\
25.54	0\\
25.55	0\\
25.56	0\\
25.57	0\\
25.58	0\\
25.59	0\\
25.6	0\\
25.61	0\\
25.62	0\\
25.63	0\\
25.64	0\\
25.65	0\\
25.66	0\\
25.67	0\\
25.68	0\\
25.69	0\\
25.7	0\\
25.71	0\\
25.72	0\\
25.73	0\\
25.74	0\\
25.75	0\\
25.76	0\\
25.77	0\\
25.78	0\\
25.79	0\\
25.8	0\\
25.81	0\\
25.82	0\\
25.83	0\\
25.84	0\\
25.85	0\\
25.86	0\\
25.87	0\\
25.88	0\\
25.89	0\\
25.9	0\\
25.91	0\\
25.92	0\\
25.93	0\\
25.94	0\\
25.95	0\\
25.96	0\\
25.97	0\\
25.98	0\\
25.99	0\\
26	0\\
26.01	0\\
26.02	0\\
26.03	0\\
26.04	0\\
26.05	0\\
26.06	0\\
26.07	0\\
26.08	0\\
26.09	0\\
26.1	0\\
26.11	0\\
26.12	0\\
26.13	0\\
26.14	0\\
26.15	0\\
26.16	0\\
26.17	0\\
26.18	0\\
26.19	0\\
26.2	0\\
26.21	0\\
26.22	0\\
26.23	0\\
26.24	0\\
26.25	0\\
26.26	0\\
26.27	0\\
26.28	0\\
26.29	0\\
26.3	0\\
26.31	0\\
26.32	0\\
26.33	0\\
26.34	0\\
26.35	0\\
26.36	0\\
26.37	0\\
26.38	0\\
26.39	0\\
26.4	0\\
26.41	0\\
26.42	0\\
26.43	0\\
26.44	0\\
26.45	0\\
26.46	0\\
26.47	0\\
26.48	0\\
26.49	0\\
26.5	0\\
26.51	0\\
26.52	0\\
26.53	0\\
26.54	0\\
26.55	0\\
26.56	0\\
26.57	0\\
26.58	0\\
26.59	0\\
26.6	0\\
26.61	0\\
26.62	0\\
26.63	0\\
26.64	0\\
26.65	0\\
26.66	0\\
26.67	0\\
26.68	0\\
26.69	0\\
26.7	0\\
26.71	0\\
26.72	0\\
26.73	0\\
26.74	0\\
26.75	0\\
26.76	0\\
26.77	0\\
26.78	0\\
26.79	0\\
26.8	0\\
26.81	0\\
26.82	0\\
26.83	0\\
26.84	0\\
26.85	0\\
26.86	0\\
26.87	0\\
26.88	0\\
26.89	0\\
26.9	0\\
26.91	0\\
26.92	0\\
26.93	0\\
26.94	0\\
26.95	0\\
26.96	0\\
26.97	0\\
26.98	0\\
26.99	0\\
27	0\\
27.01	0\\
27.02	0\\
27.03	0\\
27.04	0\\
27.05	0\\
27.06	0\\
27.07	0\\
27.08	0\\
27.09	0\\
27.1	0\\
27.11	0\\
27.12	0\\
27.13	0\\
27.14	0\\
27.15	0\\
27.16	0\\
27.17	0\\
27.18	0\\
27.19	0\\
27.2	0\\
27.21	0\\
27.22	0\\
27.23	0\\
27.24	0\\
27.25	0\\
27.26	0\\
27.27	0\\
27.28	0\\
27.29	0\\
27.3	0\\
27.31	0\\
27.32	0\\
27.33	0\\
27.34	0\\
27.35	0\\
27.36	0\\
27.37	0\\
27.38	0\\
27.39	0\\
27.4	0\\
27.41	0\\
27.42	0\\
27.43	0\\
27.44	0\\
27.45	0\\
27.46	0\\
27.47	0\\
27.48	0\\
27.49	0\\
27.5	0\\
27.51	0\\
27.52	0\\
27.53	0\\
27.54	0\\
27.55	0\\
27.56	0\\
27.57	0\\
27.58	0\\
27.59	0\\
27.6	0\\
27.61	0\\
27.62	0\\
27.63	0\\
27.64	0\\
27.65	0\\
27.66	0\\
27.67	0\\
27.68	0\\
27.69	0\\
27.7	0\\
27.71	0\\
27.72	0\\
27.73	0\\
27.74	0\\
27.75	0\\
27.76	0\\
27.77	0\\
27.78	0\\
27.79	0\\
27.8	0\\
27.81	0\\
27.82	0\\
27.83	0\\
27.84	0\\
27.85	0\\
27.86	0\\
27.87	0\\
27.88	0\\
27.89	0\\
27.9	0\\
27.91	0\\
27.92	0\\
27.93	0\\
27.94	0\\
27.95	0\\
27.96	0\\
27.97	0\\
27.98	0\\
27.99	0\\
28	0\\
28.01	0\\
28.02	0\\
28.03	0\\
28.04	0\\
28.05	0\\
28.06	0\\
28.07	0\\
28.08	0\\
28.09	0\\
28.1	0\\
28.11	0\\
28.12	0\\
28.13	0\\
28.14	0\\
28.15	0\\
28.16	0\\
28.17	0\\
28.18	0\\
28.19	0\\
28.2	0\\
28.21	0\\
28.22	0\\
28.23	0\\
28.24	0\\
28.25	0\\
28.26	0\\
28.27	0\\
28.28	0\\
28.29	0\\
28.3	0\\
28.31	0\\
28.32	0\\
28.33	0\\
28.34	0\\
28.35	0\\
28.36	0\\
28.37	0\\
28.38	0\\
28.39	0\\
28.4	0\\
28.41	0\\
28.42	0\\
28.43	0\\
28.44	0\\
28.45	0\\
28.46	0\\
28.47	0\\
28.48	0\\
28.49	0\\
28.5	0\\
28.51	0\\
28.52	0\\
28.53	0\\
28.54	0\\
28.55	0\\
28.56	0\\
28.57	0\\
28.58	0\\
28.59	0\\
28.6	0\\
28.61	0\\
28.62	0\\
28.63	0\\
28.64	0\\
28.65	0\\
28.66	0\\
28.67	0\\
28.68	0\\
28.69	0\\
28.7	0\\
28.71	0\\
28.72	0\\
28.73	0\\
28.74	0\\
28.75	0\\
28.76	0\\
28.77	0\\
28.78	0\\
28.79	0\\
28.8	0\\
28.81	0\\
28.82	0\\
28.83	0\\
28.84	0\\
28.85	0\\
28.86	0\\
28.87	0\\
28.88	0\\
28.89	0\\
28.9	0\\
28.91	0\\
28.92	0\\
28.93	0\\
28.94	0\\
28.95	0\\
28.96	0\\
28.97	0\\
28.98	0\\
28.99	0\\
29	0\\
29.01	0\\
29.02	0\\
29.03	0\\
29.04	0\\
29.05	0\\
29.06	0\\
29.07	0\\
29.08	0\\
29.09	0\\
29.1	0\\
29.11	0\\
29.12	0\\
29.13	0\\
29.14	0\\
29.15	0\\
29.16	0\\
29.17	0\\
29.18	0\\
29.19	0\\
29.2	0\\
29.21	0\\
29.22	0\\
29.23	0\\
29.24	0\\
29.25	0\\
29.26	0\\
29.27	0\\
29.28	0\\
29.29	0\\
29.3	0\\
29.31	0\\
29.32	0\\
29.33	0\\
29.34	0\\
29.35	0\\
29.36	0\\
29.37	0\\
29.38	0\\
29.39	0\\
29.4	0\\
29.41	0\\
29.42	0\\
29.43	0\\
29.44	0\\
29.45	0\\
29.46	0\\
29.47	0\\
29.48	0\\
29.49	0\\
29.5	0\\
29.51	0\\
29.52	0\\
29.53	0\\
29.54	0\\
29.55	0\\
29.56	0\\
29.57	0\\
29.58	0\\
29.59	0\\
29.6	0\\
29.61	0\\
29.62	0\\
29.63	0\\
29.64	0\\
29.65	0\\
29.66	0\\
29.67	0\\
29.68	0\\
29.69	0\\
29.7	0\\
29.71	0\\
29.72	0\\
29.73	0\\
29.74	0\\
29.75	0\\
29.76	0\\
29.77	0\\
29.78	0\\
29.79	0\\
29.8	0\\
29.81	0\\
29.82	0\\
29.83	0\\
29.84	0\\
29.85	0\\
29.86	0\\
29.87	0\\
29.88	0\\
29.89	0\\
29.9	0\\
29.91	0\\
29.92	0\\
29.93	0\\
29.94	0\\
29.95	0\\
29.96	0\\
29.97	0\\
29.98	0\\
29.99	0\\
30	0\\
30.01	0\\
30.02	0\\
30.03	0\\
30.04	0\\
30.05	0\\
30.06	0\\
30.07	0\\
30.08	0\\
30.09	0\\
30.1	0\\
30.11	0\\
30.12	0\\
30.13	0\\
30.14	0\\
30.15	0\\
30.16	0\\
30.17	0\\
30.18	0\\
30.19	0\\
30.2	0\\
30.21	0\\
30.22	0\\
30.23	0\\
30.24	0\\
30.25	0\\
30.26	0\\
30.27	0\\
30.28	0\\
30.29	0\\
30.3	0\\
30.31	0\\
30.32	0\\
30.33	0\\
30.34	0\\
30.35	0\\
30.36	0\\
30.37	0\\
30.38	0\\
30.39	0\\
30.4	0\\
30.41	0\\
30.42	0\\
30.43	0\\
30.44	0\\
30.45	0\\
30.46	0\\
30.47	0\\
30.48	0\\
30.49	0\\
30.5	0\\
30.51	0\\
30.52	0\\
30.53	0\\
30.54	0\\
30.55	0\\
30.56	0\\
30.57	0\\
30.58	0\\
30.59	0\\
30.6	0\\
30.61	0\\
30.62	0\\
30.63	0\\
30.64	0\\
30.65	0\\
30.66	0\\
30.67	0\\
30.68	0\\
30.69	0\\
30.7	0\\
30.71	0\\
30.72	0\\
30.73	0\\
30.74	0\\
30.75	0\\
30.76	0\\
30.77	0\\
30.78	0\\
30.79	0\\
30.8	0\\
30.81	0\\
30.82	0\\
30.83	0\\
30.84	0\\
30.85	0\\
30.86	0\\
30.87	0\\
30.88	0\\
30.89	0\\
30.9	0\\
30.91	0\\
30.92	0\\
30.93	0\\
30.94	0\\
30.95	0\\
30.96	0\\
30.97	0\\
30.98	0\\
30.99	0\\
31	0\\
31.01	0\\
31.02	0\\
31.03	0\\
31.04	0\\
31.05	0\\
31.06	0\\
31.07	0\\
31.08	0\\
31.09	0\\
31.1	0\\
31.11	0\\
31.12	0\\
31.13	0\\
31.14	0\\
31.15	0\\
31.16	0\\
31.17	0\\
31.18	0\\
31.19	0\\
31.2	0\\
31.21	0\\
31.22	0\\
31.23	0\\
31.24	0\\
31.25	0\\
31.26	0\\
31.27	0\\
31.28	0\\
31.29	0\\
31.3	0\\
31.31	0\\
31.32	0\\
31.33	0\\
31.34	0\\
31.35	0\\
31.36	0\\
31.37	0\\
31.38	0\\
31.39	0\\
31.4	0\\
31.41	0\\
31.42	0\\
31.43	0\\
31.44	0\\
31.45	0\\
31.46	0\\
31.47	0\\
31.48	0\\
31.49	0\\
31.5	0\\
31.51	0\\
31.52	0\\
31.53	0\\
31.54	0\\
31.55	0\\
31.56	0\\
31.57	0\\
31.58	0\\
31.59	0\\
31.6	0\\
31.61	0\\
31.62	0\\
31.63	0\\
31.64	0\\
31.65	0\\
31.66	0\\
31.67	0\\
31.68	0\\
31.69	0\\
31.7	0\\
31.71	0\\
31.72	0\\
31.73	0\\
31.74	0\\
31.75	0\\
31.76	0\\
31.77	0\\
31.78	0\\
31.79	0\\
31.8	0\\
31.81	0\\
31.82	0\\
31.83	0\\
31.84	0\\
31.85	0\\
31.86	0\\
31.87	0\\
31.88	0\\
31.89	0\\
31.9	0\\
31.91	0\\
31.92	0\\
31.93	0\\
31.94	0\\
31.95	0\\
31.96	0\\
31.97	0\\
31.98	0\\
31.99	0\\
32	0\\
32.01	0\\
32.02	0\\
32.03	0\\
32.04	0\\
32.05	0\\
32.06	0\\
32.07	0\\
32.08	0\\
32.09	0\\
32.1	0\\
32.11	0\\
32.12	0\\
32.13	0\\
32.14	0\\
32.15	0\\
32.16	0\\
32.17	0\\
32.18	0\\
32.19	0\\
32.2	0\\
32.21	0\\
32.22	0\\
32.23	0\\
32.24	0\\
32.25	0\\
32.26	0\\
32.27	0\\
32.28	0\\
32.29	0\\
32.3	0\\
32.31	0\\
32.32	0\\
32.33	0\\
32.34	0\\
32.35	0\\
32.36	0\\
32.37	0\\
32.38	0\\
32.39	0\\
32.4	0\\
32.41	0\\
32.42	0\\
32.43	0\\
32.44	0\\
32.45	0\\
32.46	0\\
32.47	0\\
32.48	0\\
32.49	0\\
32.5	0\\
32.51	0\\
32.52	0\\
32.53	0\\
32.54	0\\
32.55	0\\
32.56	0\\
32.57	0\\
32.58	0\\
32.59	0\\
32.6	0\\
32.61	0\\
32.62	0\\
32.63	0\\
32.64	0\\
32.65	0\\
32.66	0\\
32.67	0\\
32.68	0\\
32.69	0\\
32.7	0\\
32.71	0\\
32.72	0\\
32.73	0\\
32.74	0\\
32.75	0\\
32.76	0\\
32.77	0\\
32.78	0\\
32.79	0\\
32.8	0\\
32.81	0\\
32.82	0\\
32.83	0\\
32.84	0\\
32.85	0\\
32.86	0\\
32.87	0\\
32.88	0\\
32.89	0\\
32.9	0\\
32.91	0\\
32.92	0\\
32.93	0\\
32.94	0\\
32.95	0\\
32.96	0\\
32.97	0\\
32.98	0\\
32.99	0\\
33	0\\
33.01	0\\
33.02	0\\
33.03	0\\
33.04	0\\
33.05	0\\
33.06	0\\
33.07	0\\
33.08	0\\
33.09	0\\
33.1	0\\
33.11	0\\
33.12	0\\
33.13	0\\
33.14	0\\
33.15	0\\
33.16	0\\
33.17	0\\
33.18	0\\
33.19	0\\
33.2	0\\
33.21	0\\
33.22	0\\
33.23	0\\
33.24	0\\
33.25	0\\
33.26	0\\
33.27	0\\
33.28	0\\
33.29	0\\
33.3	0\\
33.31	0\\
33.32	0\\
33.33	0\\
33.34	0\\
33.35	0\\
33.36	0\\
33.37	0\\
33.38	0\\
33.39	0\\
33.4	0\\
33.41	0\\
33.42	0\\
33.43	0\\
33.44	0\\
33.45	0\\
33.46	0\\
33.47	0\\
33.48	0\\
33.49	0\\
33.5	0\\
33.51	0\\
33.52	0\\
33.53	0\\
33.54	0\\
33.55	0\\
33.56	0\\
33.57	0\\
33.58	0\\
33.59	0\\
33.6	0\\
33.61	0\\
33.62	0\\
33.63	0\\
33.64	0\\
33.65	0\\
33.66	0\\
33.67	0\\
33.68	0\\
33.69	0\\
33.7	0\\
33.71	0\\
33.72	0\\
33.73	0\\
33.74	0\\
33.75	0\\
33.76	0\\
33.77	0\\
33.78	0\\
33.79	0\\
33.8	0\\
33.81	0\\
33.82	0\\
33.83	0\\
33.84	0\\
33.85	0\\
33.86	0\\
33.87	0\\
33.88	0\\
33.89	0\\
33.9	0\\
33.91	0\\
33.92	0\\
33.93	0\\
33.94	0\\
33.95	0\\
33.96	0\\
33.97	0\\
33.98	0\\
33.99	0\\
34	0\\
34.01	0\\
34.02	0\\
34.03	0\\
34.04	0\\
34.05	0\\
34.06	0\\
34.07	0\\
34.08	0\\
34.09	0\\
34.1	0\\
34.11	0\\
34.12	0\\
34.13	0\\
34.14	0\\
34.15	0\\
34.16	0\\
34.17	0\\
34.18	0\\
34.19	0\\
34.2	0\\
34.21	0\\
34.22	0\\
34.23	0\\
34.24	0\\
34.25	0\\
34.26	0\\
34.27	0\\
34.28	0\\
34.29	0\\
34.3	0\\
34.31	0\\
34.32	0\\
34.33	0\\
34.34	0\\
34.35	0\\
34.36	0\\
34.37	0\\
34.38	0\\
34.39	0\\
34.4	0\\
34.41	0\\
34.42	0\\
34.43	0\\
34.44	0\\
34.45	0\\
34.46	0\\
34.47	0\\
34.48	0\\
34.49	0\\
34.5	0\\
34.51	0\\
34.52	0\\
34.53	0\\
34.54	0\\
34.55	0\\
34.56	0\\
34.57	0\\
34.58	0\\
34.59	0\\
34.6	0\\
34.61	0\\
34.62	0\\
34.63	0\\
34.64	0\\
34.65	0\\
34.66	0\\
34.67	0\\
34.68	0\\
34.69	0\\
34.7	0\\
34.71	0\\
34.72	0\\
34.73	0\\
34.74	0\\
34.75	0\\
34.76	0\\
34.77	0\\
34.78	0\\
34.79	0\\
34.8	0\\
34.81	0\\
34.82	0\\
34.83	0\\
34.84	0\\
34.85	0\\
34.86	0\\
34.87	0\\
34.88	0\\
34.89	0\\
34.9	0\\
34.91	0\\
34.92	0\\
34.93	0\\
34.94	0\\
34.95	0\\
34.96	0\\
34.97	0\\
34.98	0\\
34.99	0\\
35	0\\
35.01	0\\
35.02	0\\
35.03	0\\
35.04	0\\
35.05	0\\
35.06	0\\
35.07	0\\
35.08	0\\
35.09	0\\
35.1	0\\
35.11	0\\
35.12	0\\
35.13	0\\
35.14	0\\
35.15	0\\
35.16	0\\
35.17	0\\
35.18	0\\
35.19	0\\
35.2	0\\
35.21	0\\
35.22	0\\
35.23	0\\
35.24	0\\
35.25	0\\
35.26	0\\
35.27	0\\
35.28	0\\
35.29	0\\
35.3	0\\
35.31	0\\
35.32	0\\
35.33	0\\
35.34	0\\
35.35	0\\
35.36	0\\
35.37	0\\
35.38	0\\
35.39	0\\
35.4	0\\
35.41	0\\
35.42	0\\
35.43	0\\
35.44	0\\
35.45	0\\
35.46	0\\
35.47	0\\
35.48	0\\
35.49	0\\
35.5	0\\
35.51	0\\
35.52	0\\
35.53	0\\
35.54	0\\
35.55	0\\
35.56	0\\
35.57	0\\
35.58	0\\
35.59	0\\
35.6	0\\
35.61	0\\
35.62	0\\
35.63	0\\
35.64	0\\
35.65	0\\
35.66	0\\
35.67	0\\
35.68	0\\
35.69	0\\
35.7	0\\
35.71	0\\
35.72	0\\
35.73	0\\
35.74	0\\
35.75	0\\
35.76	0\\
35.77	0\\
35.78	0\\
35.79	0\\
35.8	0\\
35.81	0\\
35.82	0\\
35.83	0\\
35.84	0\\
35.85	0\\
35.86	0\\
35.87	0\\
35.88	0\\
35.89	0\\
35.9	0\\
35.91	0\\
35.92	0\\
35.93	0\\
35.94	0\\
35.95	0\\
35.96	0\\
35.97	0\\
35.98	0\\
35.99	0\\
36	0\\
36.01	0\\
36.02	0\\
36.03	0\\
36.04	0\\
36.05	0\\
36.06	0\\
36.07	0\\
36.08	0\\
36.09	0\\
36.1	0\\
36.11	0\\
36.12	0\\
36.13	0\\
36.14	0\\
36.15	0\\
36.16	0\\
36.17	0\\
36.18	0\\
36.19	0\\
36.2	0\\
36.21	0\\
36.22	0\\
36.23	0\\
36.24	0\\
36.25	0\\
36.26	0\\
36.27	0\\
36.28	0\\
36.29	0\\
36.3	0\\
36.31	0\\
36.32	0\\
36.33	0\\
36.34	0\\
36.35	0\\
36.36	0\\
36.37	0\\
36.38	0\\
36.39	0\\
36.4	0\\
36.41	0\\
36.42	0\\
36.43	0\\
36.44	0\\
36.45	0\\
36.46	0\\
36.47	0\\
36.48	0\\
36.49	0\\
36.5	0\\
36.51	0\\
36.52	0\\
36.53	0\\
36.54	0\\
36.55	0\\
36.56	0\\
36.57	0\\
36.58	0\\
36.59	0\\
36.6	0\\
36.61	0\\
36.62	0\\
36.63	0\\
36.64	0\\
36.65	0\\
36.66	0\\
36.67	0\\
36.68	0\\
36.69	0\\
36.7	0\\
36.71	0\\
36.72	0\\
36.73	0\\
36.74	0\\
36.75	0\\
36.76	0\\
36.77	0\\
36.78	0\\
36.79	0\\
36.8	0\\
36.81	0\\
36.82	0\\
36.83	0\\
36.84	0\\
36.85	0\\
36.86	0\\
36.87	0\\
36.88	0\\
36.89	0\\
36.9	0\\
36.91	0\\
36.92	0\\
36.93	0\\
36.94	0\\
36.95	0\\
36.96	0\\
36.97	0\\
36.98	0\\
36.99	0\\
37	0\\
37.01	0\\
37.02	0\\
37.03	0\\
37.04	0\\
37.05	0\\
37.06	0\\
37.07	0\\
37.08	0\\
37.09	0\\
37.1	0\\
37.11	0\\
37.12	0\\
37.13	0\\
37.14	0\\
37.15	0\\
37.16	0\\
37.17	0\\
37.18	0\\
37.19	0\\
37.2	0\\
37.21	0\\
37.22	0\\
37.23	0\\
37.24	0\\
37.25	0\\
37.26	0\\
37.27	0\\
37.28	0\\
37.29	0\\
37.3	0\\
37.31	0\\
37.32	0\\
37.33	0\\
37.34	0\\
37.35	0\\
37.36	0\\
37.37	0\\
37.38	0\\
37.39	0\\
37.4	0\\
37.41	0\\
37.42	0\\
37.43	0\\
37.44	0\\
37.45	0\\
37.46	0\\
37.47	0\\
37.48	0\\
37.49	0\\
37.5	0\\
37.51	0\\
37.52	0\\
37.53	0\\
37.54	0\\
37.55	0\\
37.56	0\\
37.57	0\\
37.58	0\\
37.59	0\\
37.6	0\\
37.61	0\\
37.62	0\\
37.63	0\\
37.64	0\\
37.65	0\\
37.66	0\\
37.67	0\\
37.68	0\\
37.69	0\\
37.7	0\\
37.71	0\\
37.72	0\\
37.73	0\\
37.74	0\\
37.75	0\\
37.76	0\\
37.77	0\\
37.78	0\\
37.79	0\\
37.8	0\\
37.81	0\\
37.82	0\\
37.83	0\\
37.84	0\\
37.85	0\\
37.86	0\\
37.87	0\\
37.88	0\\
37.89	0\\
37.9	0\\
37.91	0\\
37.92	0\\
37.93	0\\
37.94	0\\
37.95	0\\
37.96	0\\
37.97	0\\
37.98	0\\
37.99	0\\
38	0\\
38.01	0\\
38.02	0\\
38.03	0\\
38.04	0\\
38.05	0\\
38.06	0\\
38.07	0\\
38.08	0\\
38.09	0\\
38.1	0\\
38.11	0\\
38.12	0\\
38.13	0\\
38.14	0\\
38.15	0\\
38.16	0\\
38.17	0\\
38.18	0\\
38.19	0\\
38.2	0\\
38.21	0\\
38.22	0\\
38.23	0\\
38.24	0\\
38.25	0\\
38.26	0\\
38.27	0\\
38.28	0\\
38.29	0\\
38.3	0\\
38.31	0\\
38.32	0\\
38.33	0\\
38.34	0\\
38.35	0\\
38.36	0\\
38.37	0\\
38.38	0\\
38.39	0\\
38.4	0\\
38.41	0\\
38.42	0\\
38.43	0\\
38.44	0\\
38.45	0\\
38.46	0\\
38.47	0\\
38.48	0\\
38.49	0\\
38.5	0\\
38.51	0\\
38.52	0\\
38.53	0\\
38.54	0\\
38.55	0\\
38.56	0\\
38.57	0\\
38.58	0\\
38.59	0\\
38.6	0\\
38.61	0\\
38.62	0\\
38.63	0\\
38.64	0\\
38.65	0\\
38.66	0\\
38.67	0\\
38.68	0\\
38.69	0\\
38.7	0\\
38.71	0\\
38.72	0\\
38.73	0\\
38.74	0\\
38.75	0\\
38.76	0\\
38.77	0\\
38.78	0\\
38.79	0\\
38.8	0\\
38.81	0\\
38.82	0\\
38.83	0\\
38.84	0\\
38.85	0\\
38.86	0\\
38.87	0\\
38.88	0\\
38.89	0\\
38.9	0\\
38.91	0\\
38.92	0\\
38.93	0\\
38.94	0\\
38.95	0\\
38.96	0\\
38.97	0\\
38.98	0\\
38.99	0\\
39	0\\
39.01	0\\
39.02	0\\
39.03	0\\
39.04	0\\
39.05	0\\
39.06	0\\
39.07	0\\
39.08	0\\
39.09	0\\
39.1	0\\
39.11	0\\
39.12	0\\
39.13	0\\
39.14	0\\
39.15	0\\
39.16	0\\
39.17	0\\
39.18	0\\
39.19	0\\
39.2	0\\
39.21	0\\
39.22	0\\
39.23	0\\
39.24	0\\
39.25	0\\
39.26	0\\
39.27	0\\
39.28	0\\
39.29	0\\
39.3	0\\
39.31	0\\
39.32	0\\
39.33	0\\
39.34	0\\
39.35	0\\
39.36	0\\
39.37	0\\
39.38	0\\
39.39	0\\
39.4	0\\
39.41	0\\
39.42	0\\
39.43	0\\
39.44	0\\
39.45	0\\
39.46	0\\
39.47	0\\
39.48	0\\
39.49	0\\
39.5	0\\
39.51	0\\
39.52	0\\
39.53	0\\
39.54	0\\
39.55	0\\
39.56	0\\
39.57	0\\
39.58	0\\
39.59	0\\
39.6	0\\
39.61	0\\
39.62	0\\
39.63	0\\
39.64	0\\
39.65	0\\
39.66	0\\
39.67	0\\
39.68	0\\
39.69	0\\
39.7	0\\
39.71	0\\
39.72	0\\
39.73	0\\
39.74	0\\
39.75	0\\
39.76	0\\
39.77	0\\
39.78	0\\
39.79	0\\
39.8	0\\
39.81	0\\
39.82	0\\
39.83	0\\
39.84	0\\
39.85	0\\
39.86	0\\
39.87	0\\
39.88	0\\
39.89	0\\
39.9	0\\
39.91	0\\
39.92	0\\
39.93	0\\
39.94	0\\
39.95	0\\
39.96	0\\
39.97	0\\
39.98	0\\
39.99	0\\
40	0\\
40.01	0\\
};
\addplot [color=red,solid,forget plot]
  table[row sep=crcr]{%
40.01	0\\
40.02	0\\
40.03	0\\
40.04	0\\
40.05	0\\
40.06	0\\
40.07	0\\
40.08	0\\
40.09	0\\
40.1	0\\
40.11	0\\
40.12	0\\
40.13	0\\
40.14	0\\
40.15	0\\
40.16	0\\
40.17	0\\
40.18	0\\
40.19	0\\
40.2	0\\
40.21	0\\
40.22	0\\
40.23	0\\
40.24	0\\
40.25	0\\
40.26	0\\
40.27	0\\
40.28	0\\
40.29	0\\
40.3	0\\
40.31	0\\
40.32	0\\
40.33	0\\
40.34	0\\
40.35	0\\
40.36	0\\
40.37	0\\
40.38	0\\
40.39	0\\
40.4	0\\
40.41	0\\
40.42	0\\
40.43	0\\
40.44	0\\
40.45	0\\
40.46	0\\
40.47	0\\
40.48	0\\
40.49	0\\
40.5	0\\
40.51	0\\
40.52	0\\
40.53	0\\
40.54	0\\
40.55	0\\
40.56	0\\
40.57	0\\
40.58	0\\
40.59	0\\
40.6	0\\
40.61	0\\
40.62	0\\
40.63	0\\
40.64	0\\
40.65	0\\
40.66	0\\
40.67	0\\
40.68	0\\
40.69	0\\
40.7	0\\
40.71	0\\
40.72	0\\
40.73	0\\
40.74	0\\
40.75	0\\
40.76	0\\
40.77	0\\
40.78	0\\
40.79	0\\
40.8	0\\
40.81	0\\
40.82	0\\
40.83	0\\
40.84	0\\
40.85	0\\
40.86	0\\
40.87	0\\
40.88	0\\
40.89	0\\
40.9	0\\
40.91	0\\
40.92	0\\
40.93	0\\
40.94	0\\
40.95	0\\
40.96	0\\
40.97	0\\
40.98	0\\
40.99	0\\
41	0\\
41.01	0\\
41.02	0\\
41.03	0\\
41.04	0\\
41.05	0\\
41.06	0\\
41.07	0\\
41.08	0\\
41.09	0\\
41.1	0\\
41.11	0\\
41.12	0\\
41.13	0\\
41.14	0\\
41.15	0\\
41.16	0\\
41.17	0\\
41.18	0\\
41.19	0\\
41.2	0\\
41.21	0\\
41.22	0\\
41.23	0\\
41.24	0\\
41.25	0\\
41.26	0\\
41.27	0\\
41.28	0\\
41.29	0\\
41.3	0\\
41.31	0\\
41.32	0\\
41.33	0\\
41.34	0\\
41.35	0\\
41.36	0\\
41.37	0\\
41.38	0\\
41.39	0\\
41.4	0\\
41.41	0\\
41.42	0\\
41.43	0\\
41.44	0\\
41.45	0\\
41.46	0\\
41.47	0\\
41.48	0\\
41.49	0\\
41.5	0\\
41.51	0\\
41.52	0\\
41.53	0\\
41.54	0\\
41.55	0\\
41.56	0\\
41.57	0\\
41.58	0\\
41.59	0\\
41.6	0\\
41.61	0\\
41.62	0\\
41.63	0\\
41.64	0\\
41.65	0\\
41.66	0\\
41.67	0\\
41.68	0\\
41.69	0\\
41.7	0\\
41.71	0\\
41.72	0\\
41.73	0\\
41.74	0\\
41.75	0\\
41.76	0\\
41.77	0\\
41.78	0\\
41.79	0\\
41.8	0\\
41.81	0\\
41.82	0\\
41.83	0\\
41.84	0\\
41.85	0\\
41.86	0\\
41.87	0\\
41.88	0\\
41.89	0\\
41.9	0\\
41.91	0\\
41.92	0\\
41.93	0\\
41.94	0\\
41.95	0\\
41.96	0\\
41.97	0\\
41.98	0\\
41.99	0\\
42	0\\
42.01	0\\
42.02	0\\
42.03	0\\
42.04	0\\
42.05	0\\
42.06	0\\
42.07	0\\
42.08	0\\
42.09	0\\
42.1	0\\
42.11	0\\
42.12	0\\
42.13	0\\
42.14	0\\
42.15	0\\
42.16	0\\
42.17	0\\
42.18	0\\
42.19	0\\
42.2	0\\
42.21	0\\
42.22	0\\
42.23	0\\
42.24	0\\
42.25	0\\
42.26	0\\
42.27	0\\
42.28	0\\
42.29	0\\
42.3	0\\
42.31	0\\
42.32	0\\
42.33	0\\
42.34	0\\
42.35	0\\
42.36	0\\
42.37	0\\
42.38	0\\
42.39	0\\
42.4	0\\
42.41	0\\
42.42	0\\
42.43	0\\
42.44	0\\
42.45	0\\
42.46	0\\
42.47	0\\
42.48	0\\
42.49	0\\
42.5	0\\
42.51	0\\
42.52	0\\
42.53	0\\
42.54	0\\
42.55	0\\
42.56	0\\
42.57	0\\
42.58	0\\
42.59	0\\
42.6	0\\
42.61	0\\
42.62	0\\
42.63	0\\
42.64	0\\
42.65	0\\
42.66	0\\
42.67	0\\
42.68	0\\
42.69	0\\
42.7	0\\
42.71	0\\
42.72	0\\
42.73	0\\
42.74	0\\
42.75	0\\
42.76	0\\
42.77	0\\
42.78	0\\
42.79	0\\
42.8	0\\
42.81	0\\
42.82	0\\
42.83	0\\
42.84	0\\
42.85	0\\
42.86	0\\
42.87	0\\
42.88	0\\
42.89	0\\
42.9	0\\
42.91	0\\
42.92	0\\
42.93	0\\
42.94	0\\
42.95	0\\
42.96	0\\
42.97	0\\
42.98	0\\
42.99	0\\
43	0\\
43.01	0\\
43.02	0\\
43.03	0\\
43.04	0\\
43.05	0\\
43.06	0\\
43.07	0\\
43.08	0\\
43.09	0\\
43.1	0\\
43.11	0\\
43.12	0\\
43.13	0\\
43.14	0\\
43.15	0\\
43.16	0\\
43.17	0\\
43.18	0\\
43.19	0\\
43.2	0\\
43.21	0\\
43.22	0\\
43.23	0\\
43.24	0\\
43.25	0\\
43.26	0\\
43.27	0\\
43.28	0\\
43.29	0\\
43.3	0\\
43.31	0\\
43.32	0\\
43.33	0\\
43.34	0\\
43.35	0\\
43.36	0\\
43.37	0\\
43.38	0\\
43.39	0\\
43.4	0\\
43.41	0\\
43.42	0\\
43.43	0\\
43.44	0\\
43.45	0\\
43.46	0\\
43.47	0\\
43.48	0\\
43.49	0\\
43.5	0\\
43.51	0\\
43.52	0\\
43.53	0\\
43.54	0\\
43.55	0\\
43.56	0\\
43.57	0\\
43.58	0\\
43.59	0\\
43.6	0\\
43.61	0\\
43.62	0\\
43.63	0\\
43.64	0\\
43.65	0\\
43.66	0\\
43.67	0\\
43.68	0\\
43.69	0\\
43.7	0\\
43.71	0\\
43.72	0\\
43.73	0\\
43.74	0\\
43.75	0\\
43.76	0\\
43.77	0\\
43.78	0\\
43.79	0\\
43.8	0\\
43.81	0\\
43.82	0\\
43.83	0\\
43.84	0\\
43.85	0\\
43.86	0\\
43.87	0\\
43.88	0\\
43.89	0\\
43.9	0\\
43.91	0\\
43.92	0\\
43.93	0\\
43.94	0\\
43.95	0\\
43.96	0\\
43.97	0\\
43.98	0\\
43.99	0\\
44	0\\
44.01	0\\
44.02	0\\
44.03	0\\
44.04	0\\
44.05	0\\
44.06	0\\
44.07	0\\
44.08	0\\
44.09	0\\
44.1	0\\
44.11	0\\
44.12	0\\
44.13	0\\
44.14	0\\
44.15	0\\
44.16	0\\
44.17	0\\
44.18	0\\
44.19	0\\
44.2	0\\
44.21	0\\
44.22	0\\
44.23	0\\
44.24	0\\
44.25	0\\
44.26	0\\
44.27	0\\
44.28	0\\
44.29	0\\
44.3	0\\
44.31	0\\
44.32	0\\
44.33	0\\
44.34	0\\
44.35	0\\
44.36	0\\
44.37	0\\
44.38	0\\
44.39	0\\
44.4	0\\
44.41	0\\
44.42	0\\
44.43	0\\
44.44	0\\
44.45	0\\
44.46	0\\
44.47	0\\
44.48	0\\
44.49	0\\
44.5	0\\
44.51	0\\
44.52	0\\
44.53	0\\
44.54	0\\
44.55	0\\
44.56	0\\
44.57	0\\
44.58	0\\
44.59	0\\
44.6	0\\
44.61	0\\
44.62	0\\
44.63	0\\
44.64	0\\
44.65	0\\
44.66	0\\
44.67	0\\
44.68	0\\
44.69	0\\
44.7	0\\
44.71	0\\
44.72	0\\
44.73	0\\
44.74	0\\
44.75	0\\
44.76	0\\
44.77	0\\
44.78	0\\
44.79	0\\
44.8	0\\
44.81	0\\
44.82	0\\
44.83	0\\
44.84	0\\
44.85	0\\
44.86	0\\
44.87	0\\
44.88	0\\
44.89	0\\
44.9	0\\
44.91	0\\
44.92	0\\
44.93	0\\
44.94	0\\
44.95	0\\
44.96	0\\
44.97	0\\
44.98	0\\
44.99	0\\
45	0\\
45.01	0\\
45.02	0\\
45.03	0\\
45.04	0\\
45.05	0\\
45.06	0\\
45.07	0\\
45.08	0\\
45.09	0\\
45.1	0\\
45.11	0\\
45.12	0\\
45.13	0\\
45.14	0\\
45.15	0\\
45.16	0\\
45.17	0\\
45.18	0\\
45.19	0\\
45.2	0\\
45.21	0\\
45.22	0\\
45.23	0\\
45.24	0\\
45.25	0\\
45.26	0\\
45.27	0\\
45.28	0\\
45.29	0\\
45.3	0\\
45.31	0\\
45.32	0\\
45.33	0\\
45.34	0\\
45.35	0\\
45.36	0\\
45.37	0\\
45.38	0\\
45.39	0\\
45.4	0\\
45.41	0\\
45.42	0\\
45.43	0\\
45.44	0\\
45.45	0\\
45.46	0\\
45.47	0\\
45.48	0\\
45.49	0\\
45.5	0\\
45.51	0\\
45.52	0\\
45.53	0\\
45.54	0\\
45.55	0\\
45.56	0\\
45.57	0\\
45.58	0\\
45.59	0\\
45.6	0\\
45.61	0\\
45.62	0\\
45.63	0\\
45.64	0\\
45.65	0\\
45.66	0\\
45.67	0\\
45.68	0\\
45.69	0\\
45.7	0\\
45.71	0\\
45.72	0\\
45.73	0\\
45.74	0\\
45.75	0\\
45.76	0\\
45.77	0\\
45.78	0\\
45.79	0\\
45.8	0\\
45.81	0\\
45.82	0\\
45.83	0\\
45.84	0\\
45.85	0\\
45.86	0\\
45.87	0\\
45.88	0\\
45.89	0\\
45.9	0\\
45.91	0\\
45.92	0\\
45.93	0\\
45.94	0\\
45.95	0\\
45.96	0\\
45.97	0\\
45.98	0\\
45.99	0\\
46	0\\
46.01	0\\
46.02	0\\
46.03	0\\
46.04	0\\
46.05	0\\
46.06	0\\
46.07	0\\
46.08	0\\
46.09	0\\
46.1	0\\
46.11	0\\
46.12	0\\
46.13	0\\
46.14	0\\
46.15	0\\
46.16	0\\
46.17	0\\
46.18	0\\
46.19	0\\
46.2	0\\
46.21	0\\
46.22	0\\
46.23	0\\
46.24	0\\
46.25	0\\
46.26	0\\
46.27	0\\
46.28	0\\
46.29	0\\
46.3	0\\
46.31	0\\
46.32	0\\
46.33	0\\
46.34	0\\
46.35	0\\
46.36	0\\
46.37	0\\
46.38	0\\
46.39	0\\
46.4	0\\
46.41	0\\
46.42	0\\
46.43	0\\
46.44	0\\
46.45	0\\
46.46	0\\
46.47	0\\
46.48	0\\
46.49	0\\
46.5	0\\
46.51	0\\
46.52	0\\
46.53	0\\
46.54	0\\
46.55	0\\
46.56	0\\
46.57	0\\
46.58	0\\
46.59	0\\
46.6	0\\
46.61	0\\
46.62	0\\
46.63	0\\
46.64	0\\
46.65	0\\
46.66	0\\
46.67	0\\
46.68	0\\
46.69	0\\
46.7	0\\
46.71	0\\
46.72	0\\
46.73	0\\
46.74	0\\
46.75	0\\
46.76	0\\
46.77	0\\
46.78	0\\
46.79	0\\
46.8	0\\
46.81	0\\
46.82	0\\
46.83	0\\
46.84	0\\
46.85	0\\
46.86	0\\
46.87	0\\
46.88	0\\
46.89	0\\
46.9	0\\
46.91	0\\
46.92	0\\
46.93	0\\
46.94	0\\
46.95	0\\
46.96	0\\
46.97	0\\
46.98	0\\
46.99	0\\
47	0\\
47.01	0\\
47.02	0\\
47.03	0\\
47.04	0\\
47.05	0\\
47.06	0\\
47.07	0\\
47.08	0\\
47.09	0\\
47.1	0\\
47.11	0\\
47.12	0\\
47.13	0\\
47.14	0\\
47.15	0\\
47.16	0\\
47.17	0\\
47.18	0\\
47.19	0\\
47.2	0\\
47.21	0\\
47.22	0\\
47.23	0\\
47.24	0\\
47.25	0\\
47.26	0\\
47.27	0\\
47.28	0\\
47.29	0\\
47.3	0\\
47.31	0\\
47.32	0\\
47.33	0\\
47.34	0\\
47.35	0\\
47.36	0\\
47.37	0\\
47.38	0\\
47.39	0\\
47.4	0\\
47.41	0\\
47.42	0\\
47.43	0\\
47.44	0\\
47.45	0\\
47.46	0\\
47.47	0\\
47.48	0\\
47.49	0\\
47.5	0\\
47.51	0\\
47.52	0\\
47.53	0\\
47.54	0\\
47.55	0\\
47.56	0\\
47.57	0\\
47.58	0\\
47.59	0\\
47.6	0\\
47.61	0\\
47.62	0\\
47.63	0\\
47.64	0\\
47.65	0\\
47.66	0\\
47.67	0\\
47.68	0\\
47.69	0\\
47.7	0\\
47.71	0\\
47.72	0\\
47.73	0\\
47.74	0\\
47.75	0\\
47.76	0\\
47.77	0\\
47.78	0\\
47.79	0\\
47.8	0\\
47.81	0\\
47.82	0\\
47.83	0\\
47.84	0\\
47.85	0\\
47.86	0\\
47.87	0\\
47.88	0\\
47.89	0\\
47.9	0\\
47.91	0\\
47.92	0\\
47.93	0\\
47.94	0\\
47.95	0\\
47.96	0\\
47.97	0\\
47.98	0\\
47.99	0\\
48	0\\
48.01	0\\
48.02	0\\
48.03	0\\
48.04	0\\
48.05	0\\
48.06	0\\
48.07	0\\
48.08	0\\
48.09	0\\
48.1	0\\
48.11	0\\
48.12	0\\
48.13	0\\
48.14	0\\
48.15	0\\
48.16	0\\
48.17	0\\
48.18	0\\
48.19	0\\
48.2	0\\
48.21	0\\
48.22	0\\
48.23	0\\
48.24	0\\
48.25	0\\
48.26	0\\
48.27	0\\
48.28	0\\
48.29	0\\
48.3	0\\
48.31	0\\
48.32	0\\
48.33	0\\
48.34	0\\
48.35	0\\
48.36	0\\
48.37	0\\
48.38	0\\
48.39	0\\
48.4	0\\
48.41	0\\
48.42	0\\
48.43	0\\
48.44	0\\
48.45	0\\
48.46	0\\
48.47	0\\
48.48	0\\
48.49	0\\
48.5	0\\
48.51	0\\
48.52	0\\
48.53	0\\
48.54	0\\
48.55	0\\
48.56	0\\
48.57	0\\
48.58	0\\
48.59	0\\
48.6	0\\
48.61	0\\
48.62	0\\
48.63	0\\
48.64	0\\
48.65	0\\
48.66	0\\
48.67	0\\
48.68	0\\
48.69	0\\
48.7	0\\
48.71	0\\
48.72	0\\
48.73	0\\
48.74	0\\
48.75	0\\
48.76	0\\
48.77	0\\
48.78	0\\
48.79	0\\
48.8	0\\
48.81	0\\
48.82	0\\
48.83	0\\
48.84	0\\
48.85	0\\
48.86	0\\
48.87	0\\
48.88	0\\
48.89	0\\
48.9	0\\
48.91	0\\
48.92	0\\
48.93	0\\
48.94	0\\
48.95	0\\
48.96	0\\
48.97	0\\
48.98	0\\
48.99	0\\
49	0\\
49.01	0\\
49.02	0\\
49.03	0\\
49.04	0\\
49.05	0\\
49.06	0\\
49.07	0\\
49.08	0\\
49.09	0\\
49.1	0\\
49.11	0\\
49.12	0\\
49.13	0\\
49.14	0\\
49.15	0\\
49.16	0\\
49.17	0\\
49.18	0\\
49.19	0\\
49.2	0\\
49.21	0\\
49.22	0\\
49.23	0\\
49.24	0\\
49.25	0\\
49.26	0\\
49.27	0\\
49.28	0\\
49.29	0\\
49.3	0\\
49.31	0\\
49.32	0\\
49.33	0\\
49.34	0\\
49.35	0\\
49.36	0\\
49.37	0\\
49.38	0\\
49.39	0\\
49.4	0\\
49.41	0\\
49.42	0\\
49.43	0\\
49.44	0\\
49.45	0\\
49.46	0\\
49.47	0\\
49.48	0\\
49.49	0\\
49.5	0\\
49.51	0\\
49.52	0\\
49.53	0\\
49.54	0\\
49.55	0\\
49.56	0\\
49.57	0\\
49.58	0\\
49.59	0\\
49.6	0\\
49.61	0\\
49.62	0\\
49.63	0\\
49.64	0\\
49.65	0\\
49.66	0\\
49.67	0\\
49.68	0\\
49.69	0\\
49.7	0\\
49.71	0\\
49.72	0\\
49.73	0\\
49.74	0\\
49.75	0\\
49.76	0\\
49.77	0\\
49.78	0\\
49.79	0\\
49.8	0\\
49.81	0\\
49.82	0\\
49.83	0\\
49.84	0\\
49.85	0\\
49.86	0\\
49.87	0\\
49.88	0\\
49.89	0\\
49.9	0\\
49.91	0\\
49.92	0\\
49.93	0\\
49.94	0\\
49.95	0\\
49.96	0\\
49.97	0\\
49.98	0\\
49.99	0\\
50	0\\
50.01	0\\
50.02	0\\
50.03	0\\
50.04	0\\
50.05	0\\
50.06	0\\
50.07	0\\
50.08	0\\
50.09	0\\
50.1	0\\
50.11	0\\
50.12	0\\
50.13	0\\
50.14	0\\
50.15	0\\
50.16	0\\
50.17	0\\
50.18	0\\
50.19	0\\
50.2	0\\
50.21	0\\
50.22	0\\
50.23	0\\
50.24	0\\
50.25	0\\
50.26	0\\
50.27	0\\
50.28	0\\
50.29	0\\
50.3	0\\
50.31	0\\
50.32	0\\
50.33	0\\
50.34	0\\
50.35	0\\
50.36	0\\
50.37	0\\
50.38	0\\
50.39	0\\
50.4	0\\
50.41	0\\
50.42	0\\
50.43	0\\
50.44	0\\
50.45	0\\
50.46	0\\
50.47	0\\
50.48	0\\
50.49	0\\
50.5	0\\
50.51	0\\
50.52	0\\
50.53	0\\
50.54	0\\
50.55	0\\
50.56	0\\
50.57	0\\
50.58	0\\
50.59	0\\
50.6	0\\
50.61	0\\
50.62	0\\
50.63	0\\
50.64	0\\
50.65	0\\
50.66	0\\
50.67	0\\
50.68	0\\
50.69	0\\
50.7	0\\
50.71	0\\
50.72	0\\
50.73	0\\
50.74	0\\
50.75	0\\
50.76	0\\
50.77	0\\
50.78	0\\
50.79	0\\
50.8	0\\
50.81	0\\
50.82	0\\
50.83	0\\
50.84	0\\
50.85	0\\
50.86	0\\
50.87	0\\
50.88	0\\
50.89	0\\
50.9	0\\
50.91	0\\
50.92	0\\
50.93	0\\
50.94	0\\
50.95	0\\
50.96	0\\
50.97	0\\
50.98	0\\
50.99	0\\
51	0\\
51.01	0\\
51.02	0\\
51.03	0\\
51.04	0\\
51.05	0\\
51.06	0\\
51.07	0\\
51.08	0\\
51.09	0\\
51.1	0\\
51.11	0\\
51.12	0\\
51.13	0\\
51.14	0\\
51.15	0\\
51.16	0\\
51.17	0\\
51.18	0\\
51.19	0\\
51.2	0\\
51.21	0\\
51.22	0\\
51.23	0\\
51.24	0\\
51.25	0\\
51.26	0\\
51.27	0\\
51.28	0\\
51.29	0\\
51.3	0\\
51.31	0\\
51.32	0\\
51.33	0\\
51.34	0\\
51.35	0\\
51.36	0\\
51.37	0\\
51.38	0\\
51.39	0\\
51.4	0\\
51.41	0\\
51.42	0\\
51.43	0\\
51.44	0\\
51.45	0\\
51.46	0\\
51.47	0\\
51.48	0\\
51.49	0\\
51.5	0\\
51.51	0\\
51.52	0\\
51.53	0\\
51.54	0\\
51.55	0\\
51.56	0\\
51.57	0\\
51.58	0\\
51.59	0\\
51.6	0\\
51.61	0\\
51.62	0\\
51.63	0\\
51.64	0\\
51.65	0\\
51.66	0\\
51.67	0\\
51.68	0\\
51.69	0\\
51.7	0\\
51.71	0\\
51.72	0\\
51.73	0\\
51.74	0\\
51.75	0\\
51.76	0\\
51.77	0\\
51.78	0\\
51.79	0\\
51.8	0\\
51.81	0\\
51.82	0\\
51.83	0\\
51.84	0\\
51.85	0\\
51.86	0\\
51.87	0\\
51.88	0\\
51.89	0\\
51.9	0\\
51.91	0\\
51.92	0\\
51.93	0\\
51.94	0\\
51.95	0\\
51.96	0\\
51.97	0\\
51.98	0\\
51.99	0\\
52	0\\
52.01	0\\
52.02	0\\
52.03	0\\
52.04	0\\
52.05	0\\
52.06	0\\
52.07	0\\
52.08	0\\
52.09	0\\
52.1	0\\
52.11	0\\
52.12	0\\
52.13	0\\
52.14	0\\
52.15	0\\
52.16	0\\
52.17	0\\
52.18	0\\
52.19	0\\
52.2	0\\
52.21	0\\
52.22	0\\
52.23	0\\
52.24	0\\
52.25	0\\
52.26	0\\
52.27	0\\
52.28	0\\
52.29	0\\
52.3	0\\
52.31	0\\
52.32	0\\
52.33	0\\
52.34	0\\
52.35	0\\
52.36	0\\
52.37	0\\
52.38	0\\
52.39	0\\
52.4	0\\
52.41	0\\
52.42	0\\
52.43	0\\
52.44	0\\
52.45	0\\
52.46	0\\
52.47	0\\
52.48	0\\
52.49	0\\
52.5	0\\
52.51	0\\
52.52	0\\
52.53	0\\
52.54	0\\
52.55	0\\
52.56	0\\
52.57	0\\
52.58	0\\
52.59	0\\
52.6	0\\
52.61	0\\
52.62	0\\
52.63	0\\
52.64	0\\
52.65	0\\
52.66	0\\
52.67	0\\
52.68	0\\
52.69	0\\
52.7	0\\
52.71	0\\
52.72	0\\
52.73	0\\
52.74	0\\
52.75	0\\
52.76	0\\
52.77	0\\
52.78	0\\
52.79	0\\
52.8	0\\
52.81	0\\
52.82	0\\
52.83	0\\
52.84	0\\
52.85	0\\
52.86	0\\
52.87	0\\
52.88	0\\
52.89	0\\
52.9	0\\
52.91	0\\
52.92	0\\
52.93	0\\
52.94	0\\
52.95	0\\
52.96	0\\
52.97	0\\
52.98	0\\
52.99	0\\
53	0\\
53.01	0\\
53.02	0\\
53.03	0\\
53.04	0\\
53.05	0\\
53.06	0\\
53.07	0\\
53.08	0\\
53.09	0\\
53.1	0\\
53.11	0\\
53.12	0\\
53.13	0\\
53.14	0\\
53.15	0\\
53.16	0\\
53.17	0\\
53.18	0\\
53.19	0\\
53.2	0\\
53.21	0\\
53.22	0\\
53.23	0\\
53.24	0\\
53.25	0\\
53.26	0\\
53.27	0\\
53.28	0\\
53.29	0\\
53.3	0\\
53.31	0\\
53.32	0\\
53.33	0\\
53.34	0\\
53.35	0\\
53.36	0\\
53.37	0\\
53.38	0\\
53.39	0\\
53.4	0\\
53.41	0\\
53.42	0\\
53.43	0\\
53.44	0\\
53.45	0\\
53.46	0\\
53.47	0\\
53.48	0\\
53.49	0\\
53.5	0\\
53.51	0\\
53.52	0\\
53.53	0\\
53.54	0\\
53.55	0\\
53.56	0\\
53.57	0\\
53.58	0\\
53.59	0\\
53.6	0\\
53.61	0\\
53.62	0\\
53.63	0\\
53.64	0\\
53.65	0\\
53.66	0\\
53.67	0\\
53.68	0\\
53.69	0\\
53.7	0\\
53.71	0\\
53.72	0\\
53.73	0\\
53.74	0\\
53.75	0\\
53.76	0\\
53.77	0\\
53.78	0\\
53.79	0\\
53.8	0\\
53.81	0\\
53.82	0\\
53.83	0\\
53.84	0\\
53.85	0\\
53.86	0\\
53.87	0\\
53.88	0\\
53.89	0\\
53.9	0\\
53.91	0\\
53.92	0\\
53.93	0\\
53.94	0\\
53.95	0\\
53.96	0\\
53.97	0\\
53.98	0\\
53.99	0\\
54	0\\
54.01	0\\
54.02	0\\
54.03	0\\
54.04	0\\
54.05	0\\
54.06	0\\
54.07	0\\
54.08	0\\
54.09	0\\
54.1	0\\
54.11	0\\
54.12	0\\
54.13	0\\
54.14	0\\
54.15	0\\
54.16	0\\
54.17	0\\
54.18	0\\
54.19	0\\
54.2	0\\
54.21	0\\
54.22	0\\
54.23	0\\
54.24	0\\
54.25	0\\
54.26	0\\
54.27	0\\
54.28	0\\
54.29	0\\
54.3	0\\
54.31	0\\
54.32	0\\
54.33	0\\
54.34	0\\
54.35	0\\
54.36	0\\
54.37	0\\
54.38	0\\
54.39	0\\
54.4	0\\
54.41	0\\
54.42	0\\
54.43	0\\
54.44	0\\
54.45	0\\
54.46	0\\
54.47	0\\
54.48	0\\
54.49	0\\
54.5	0\\
54.51	0\\
54.52	0\\
54.53	0\\
54.54	0\\
54.55	0\\
54.56	0\\
54.57	0\\
54.58	0\\
54.59	0\\
54.6	0\\
54.61	0\\
54.62	0\\
54.63	0\\
54.64	0\\
54.65	0\\
54.66	0\\
54.67	0\\
54.68	0\\
54.69	0\\
54.7	0\\
54.71	0\\
54.72	0\\
54.73	0\\
54.74	0\\
54.75	0\\
54.76	0\\
54.77	0\\
54.78	0\\
54.79	0\\
54.8	0\\
54.81	0\\
54.82	0\\
54.83	0\\
54.84	0\\
54.85	0\\
54.86	0\\
54.87	0\\
54.88	0\\
54.89	0\\
54.9	0\\
54.91	0\\
54.92	0\\
54.93	0\\
54.94	0\\
54.95	0\\
54.96	0\\
54.97	0\\
54.98	0\\
54.99	0\\
55	0\\
55.01	0\\
55.02	0\\
55.03	0\\
55.04	0\\
55.05	0\\
55.06	0\\
55.07	0\\
55.08	0\\
55.09	0\\
55.1	0\\
55.11	0\\
55.12	0\\
55.13	0\\
55.14	0\\
55.15	0\\
55.16	0\\
55.17	0\\
55.18	0\\
55.19	0\\
55.2	0\\
55.21	0\\
55.22	0\\
55.23	0\\
55.24	0\\
55.25	0\\
55.26	0\\
55.27	0\\
55.28	0\\
55.29	0\\
55.3	0\\
55.31	0\\
55.32	0\\
55.33	0\\
55.34	0\\
55.35	0\\
55.36	0\\
55.37	0\\
55.38	0\\
55.39	0\\
55.4	0\\
55.41	0\\
55.42	0\\
55.43	0\\
55.44	0\\
55.45	0\\
55.46	0\\
55.47	0\\
55.48	0\\
55.49	0\\
55.5	0\\
55.51	0\\
55.52	0\\
55.53	0\\
55.54	0\\
55.55	0\\
55.56	0\\
55.57	0\\
55.58	0\\
55.59	0\\
55.6	0\\
55.61	0\\
55.62	0\\
55.63	0\\
55.64	0\\
55.65	0\\
55.66	0\\
55.67	0\\
55.68	0\\
55.69	0\\
55.7	0\\
55.71	0\\
55.72	0\\
55.73	0\\
55.74	0\\
55.75	0\\
55.76	0\\
55.77	0\\
55.78	0\\
55.79	0\\
55.8	0\\
55.81	0\\
55.82	0\\
55.83	0\\
55.84	0\\
55.85	0\\
55.86	0\\
55.87	0\\
55.88	0\\
55.89	0\\
55.9	0\\
55.91	0\\
55.92	0\\
55.93	0\\
55.94	0\\
55.95	0\\
55.96	0\\
55.97	0\\
55.98	0\\
55.99	0\\
56	0\\
56.01	0\\
56.02	0\\
56.03	0\\
56.04	0\\
56.05	0\\
56.06	0\\
56.07	0\\
56.08	0\\
56.09	0\\
56.1	0\\
56.11	0\\
56.12	0\\
56.13	0\\
56.14	0\\
56.15	0\\
56.16	0\\
56.17	0\\
56.18	0\\
56.19	0\\
56.2	0\\
56.21	0\\
56.22	0\\
56.23	0\\
56.24	0\\
56.25	0\\
56.26	0\\
56.27	0\\
56.28	0\\
56.29	0\\
56.3	0\\
56.31	0\\
56.32	0\\
56.33	0\\
56.34	0\\
56.35	0\\
56.36	0\\
56.37	0\\
56.38	0\\
56.39	0\\
56.4	0\\
56.41	0\\
56.42	0\\
56.43	0\\
56.44	0\\
56.45	0\\
56.46	0\\
56.47	0\\
56.48	0\\
56.49	0\\
56.5	0\\
56.51	0\\
56.52	0\\
56.53	0\\
56.54	0\\
56.55	0\\
56.56	0\\
56.57	0\\
56.58	0\\
56.59	0\\
56.6	0\\
56.61	0\\
56.62	0\\
56.63	0\\
56.64	0\\
56.65	0\\
56.66	0\\
56.67	0\\
56.68	0\\
56.69	0\\
56.7	0\\
56.71	0\\
56.72	0\\
56.73	0\\
56.74	0\\
56.75	0\\
56.76	0\\
56.77	0\\
56.78	0\\
56.79	0\\
56.8	0\\
56.81	0\\
56.82	0\\
56.83	0\\
56.84	0\\
56.85	0\\
56.86	0\\
56.87	0\\
56.88	0\\
56.89	0\\
56.9	0\\
56.91	0\\
56.92	0\\
56.93	0\\
56.94	0\\
56.95	0\\
56.96	0\\
56.97	0\\
56.98	0\\
56.99	0\\
57	0\\
57.01	0\\
57.02	0\\
57.03	0\\
57.04	0\\
57.05	0\\
57.06	0\\
57.07	0\\
57.08	0\\
57.09	0\\
57.1	0\\
57.11	0\\
57.12	0\\
57.13	0\\
57.14	0\\
57.15	0\\
57.16	0\\
57.17	0\\
57.18	0\\
57.19	0\\
57.2	0\\
57.21	0\\
57.22	0\\
57.23	0\\
57.24	0\\
57.25	0\\
57.26	0\\
57.27	0\\
57.28	0\\
57.29	0\\
57.3	0\\
57.31	0\\
57.32	0\\
57.33	0\\
57.34	0\\
57.35	0\\
57.36	0\\
57.37	0\\
57.38	0\\
57.39	0\\
57.4	0\\
57.41	0\\
57.42	0\\
57.43	0\\
57.44	0\\
57.45	0\\
57.46	0\\
57.47	0\\
57.48	0\\
57.49	0\\
57.5	0\\
57.51	0\\
57.52	0\\
57.53	0\\
57.54	0\\
57.55	0\\
57.56	0\\
57.57	0\\
57.58	0\\
57.59	0\\
57.6	0\\
57.61	0\\
57.62	0\\
57.63	0\\
57.64	0\\
57.65	0\\
57.66	0\\
57.67	0\\
57.68	0\\
57.69	0\\
57.7	0\\
57.71	0\\
57.72	0\\
57.73	0\\
57.74	0\\
57.75	0\\
57.76	0\\
57.77	0\\
57.78	0\\
57.79	0\\
57.8	0\\
57.81	0\\
57.82	0\\
57.83	0\\
57.84	0\\
57.85	0\\
57.86	0\\
57.87	0\\
57.88	0\\
57.89	0\\
57.9	0\\
57.91	0\\
57.92	0\\
57.93	0\\
57.94	0\\
57.95	0\\
57.96	0\\
57.97	0\\
57.98	0\\
57.99	0\\
58	0\\
58.01	0\\
58.02	0\\
58.03	0\\
58.04	0\\
58.05	0\\
58.06	0\\
58.07	0\\
58.08	0\\
58.09	0\\
58.1	0\\
58.11	0\\
58.12	0\\
58.13	0\\
58.14	0\\
58.15	0\\
58.16	0\\
58.17	0\\
58.18	0\\
58.19	0\\
58.2	0\\
58.21	0\\
58.22	0\\
58.23	0\\
58.24	0\\
58.25	0\\
58.26	0\\
58.27	0\\
58.28	0\\
58.29	0\\
58.3	0\\
58.31	0\\
58.32	0\\
58.33	0\\
58.34	0\\
58.35	0\\
58.36	0\\
58.37	0\\
58.38	0\\
58.39	0\\
58.4	0\\
58.41	0\\
58.42	0\\
58.43	0\\
58.44	0\\
58.45	0\\
58.46	0\\
58.47	0\\
58.48	0\\
58.49	0\\
58.5	0\\
58.51	0\\
58.52	0\\
58.53	0\\
58.54	0\\
58.55	0\\
58.56	0\\
58.57	0\\
58.58	0\\
58.59	0\\
58.6	0\\
58.61	0\\
58.62	0\\
58.63	0\\
58.64	0\\
58.65	0\\
58.66	0\\
58.67	0\\
58.68	0\\
58.69	0\\
58.7	0\\
58.71	0\\
58.72	0\\
58.73	0\\
58.74	0\\
58.75	0\\
58.76	0\\
58.77	0\\
58.78	0\\
58.79	0\\
58.8	0\\
58.81	0\\
58.82	0\\
58.83	0\\
58.84	0\\
58.85	0\\
58.86	0\\
58.87	0\\
58.88	0\\
58.89	0\\
58.9	0\\
58.91	0\\
58.92	0\\
58.93	0\\
58.94	0\\
58.95	0\\
58.96	0\\
58.97	0\\
58.98	0\\
58.99	0\\
59	0\\
59.01	0\\
59.02	0\\
59.03	0\\
59.04	0\\
59.05	0\\
59.06	0\\
59.07	0\\
59.08	0\\
59.09	0\\
59.1	0\\
59.11	0\\
59.12	0\\
59.13	0\\
59.14	0\\
59.15	0\\
59.16	0\\
59.17	0\\
59.18	0\\
59.19	0\\
59.2	0\\
59.21	0\\
59.22	0\\
59.23	0\\
59.24	0\\
59.25	0\\
59.26	0\\
59.27	0\\
59.28	0\\
59.29	0\\
59.3	0\\
59.31	0\\
59.32	0\\
59.33	0\\
59.34	0\\
59.35	0\\
59.36	0\\
59.37	0\\
59.38	0\\
59.39	0\\
59.4	0\\
59.41	0\\
59.42	0\\
59.43	0\\
59.44	0\\
59.45	0\\
59.46	0\\
59.47	0\\
59.48	0\\
59.49	0\\
59.5	0\\
59.51	0\\
59.52	0\\
59.53	0\\
59.54	0\\
59.55	0\\
59.56	0\\
59.57	0\\
59.58	0\\
59.59	0\\
59.6	0\\
59.61	0\\
59.62	0\\
59.63	0\\
59.64	0\\
59.65	0\\
59.66	0\\
59.67	0\\
59.68	0\\
59.69	0\\
59.7	0\\
59.71	0\\
59.72	0\\
59.73	0\\
59.74	0\\
59.75	0\\
59.76	0\\
59.77	0\\
59.78	0\\
59.79	0\\
59.8	0\\
59.81	0\\
59.82	0\\
59.83	0\\
59.84	0\\
59.85	0\\
59.86	0\\
59.87	0\\
59.88	0\\
59.89	0\\
59.9	0\\
59.91	0\\
59.92	0\\
59.93	0\\
59.94	0\\
59.95	0\\
59.96	0\\
59.97	0\\
59.98	0\\
59.99	0\\
60	0\\
60.01	0\\
60.02	0\\
60.03	0\\
60.04	0\\
60.05	0\\
60.06	0\\
60.07	0\\
60.08	0\\
60.09	0\\
60.1	0\\
60.11	0\\
60.12	0\\
60.13	0\\
60.14	0\\
60.15	0\\
60.16	0\\
60.17	0\\
60.18	0\\
60.19	0\\
60.2	0\\
60.21	0\\
60.22	0\\
60.23	0\\
60.24	0\\
60.25	0\\
60.26	0\\
60.27	0\\
60.28	0\\
60.29	0\\
60.3	0\\
60.31	0\\
60.32	0\\
60.33	0\\
60.34	0\\
60.35	0\\
60.36	0\\
60.37	0\\
60.38	0\\
60.39	0\\
60.4	0\\
60.41	0\\
60.42	0\\
60.43	0\\
60.44	0\\
60.45	0\\
60.46	0\\
60.47	0\\
60.48	0\\
60.49	0\\
60.5	0\\
60.51	0\\
60.52	0\\
60.53	0\\
60.54	0\\
60.55	0\\
60.56	0\\
60.57	0\\
60.58	0\\
60.59	0\\
60.6	0\\
60.61	0\\
60.62	0\\
60.63	0\\
60.64	0\\
60.65	0\\
60.66	0\\
60.67	0\\
60.68	0\\
60.69	0\\
60.7	0\\
60.71	0\\
60.72	0\\
60.73	0\\
60.74	0\\
60.75	0\\
60.76	0\\
60.77	0\\
60.78	0\\
60.79	0\\
60.8	0\\
60.81	0\\
60.82	0\\
60.83	0\\
60.84	0\\
60.85	0\\
60.86	0\\
60.87	0\\
60.88	0\\
60.89	0\\
60.9	0\\
60.91	0\\
60.92	0\\
60.93	0\\
60.94	0\\
60.95	0\\
60.96	0\\
60.97	0\\
60.98	0\\
60.99	0\\
61	0\\
61.01	0\\
61.02	0\\
61.03	0\\
61.04	0\\
61.05	0\\
61.06	0\\
61.07	0\\
61.08	0\\
61.09	0\\
61.1	0\\
61.11	0\\
61.12	0\\
61.13	0\\
61.14	0\\
61.15	0\\
61.16	0\\
61.17	0\\
61.18	0\\
61.19	0\\
61.2	0\\
61.21	0\\
61.22	0\\
61.23	0\\
61.24	0\\
61.25	0\\
61.26	0\\
61.27	0\\
61.28	0\\
61.29	0\\
61.3	0\\
61.31	0\\
61.32	0\\
61.33	0\\
61.34	0\\
61.35	0\\
61.36	0\\
61.37	0\\
61.38	0\\
61.39	0\\
61.4	0\\
61.41	0\\
61.42	0\\
61.43	0\\
61.44	0\\
61.45	0\\
61.46	0\\
61.47	0\\
61.48	0\\
61.49	0\\
61.5	0\\
61.51	0\\
61.52	0\\
61.53	0\\
61.54	0\\
61.55	0\\
61.56	0\\
61.57	0\\
61.58	0\\
61.59	0\\
61.6	0\\
61.61	0\\
61.62	0\\
61.63	0\\
61.64	0\\
61.65	0\\
61.66	0\\
61.67	0\\
61.68	0\\
61.69	0\\
61.7	0\\
61.71	0\\
61.72	0\\
61.73	0\\
61.74	0\\
61.75	0\\
61.76	0\\
61.77	0\\
61.78	0\\
61.79	0\\
61.8	0\\
61.81	0\\
61.82	0\\
61.83	0\\
61.84	0\\
61.85	0\\
61.86	0\\
61.87	0\\
61.88	0\\
61.89	0\\
61.9	0\\
61.91	0\\
61.92	0\\
61.93	0\\
61.94	0\\
61.95	0\\
61.96	0\\
61.97	0\\
61.98	0\\
61.99	0\\
62	0\\
62.01	0\\
62.02	0\\
62.03	0\\
62.04	0\\
62.05	0\\
62.06	0\\
62.07	0\\
62.08	0\\
62.09	0\\
62.1	0\\
62.11	0\\
62.12	0\\
62.13	0\\
62.14	0\\
62.15	0\\
62.16	0\\
62.17	0\\
62.18	0\\
62.19	0\\
62.2	0\\
62.21	0\\
62.22	0\\
62.23	0\\
62.24	0\\
62.25	0\\
62.26	0\\
62.27	0\\
62.28	0\\
62.29	0\\
62.3	0\\
62.31	0\\
62.32	0\\
62.33	0\\
62.34	0\\
62.35	0\\
62.36	0\\
62.37	0\\
62.38	0\\
62.39	0\\
62.4	0\\
62.41	0\\
62.42	0\\
62.43	0\\
62.44	0\\
62.45	0\\
62.46	0\\
62.47	0\\
62.48	0\\
62.49	0\\
62.5	0\\
62.51	0\\
62.52	0\\
62.53	0\\
62.54	0\\
62.55	0\\
62.56	0\\
62.57	0\\
62.58	0\\
62.59	0\\
62.6	0\\
62.61	0\\
62.62	0\\
62.63	0\\
62.64	0\\
62.65	0\\
62.66	0\\
62.67	0\\
62.68	0\\
62.69	0\\
62.7	0\\
62.71	0\\
62.72	0\\
62.73	0\\
62.74	0\\
62.75	0\\
62.76	0\\
62.77	0\\
62.78	0\\
62.79	0\\
62.8	0\\
62.81	0\\
62.82	0\\
62.83	0\\
62.84	0\\
62.85	0\\
62.86	0\\
62.87	0\\
62.88	0\\
62.89	0\\
62.9	0\\
62.91	0\\
62.92	0\\
62.93	0\\
62.94	0\\
62.95	0\\
62.96	0\\
62.97	0\\
62.98	0\\
62.99	0\\
63	0\\
63.01	0\\
63.02	0\\
63.03	0\\
63.04	0\\
63.05	0\\
63.06	0\\
63.07	0\\
63.08	0\\
63.09	0\\
63.1	0\\
63.11	0\\
63.12	0\\
63.13	0\\
63.14	0\\
63.15	0\\
63.16	0\\
63.17	0\\
63.18	0\\
63.19	0\\
63.2	0\\
63.21	0\\
63.22	0\\
63.23	0\\
63.24	0\\
63.25	0\\
63.26	0\\
63.27	0\\
63.28	0\\
63.29	0\\
63.3	0\\
63.31	0\\
63.32	0\\
63.33	0\\
63.34	0\\
63.35	0\\
63.36	0\\
63.37	0\\
63.38	0\\
63.39	0\\
63.4	0\\
63.41	0\\
63.42	0\\
63.43	0\\
63.44	0\\
63.45	0\\
63.46	0\\
63.47	0\\
63.48	0\\
63.49	0\\
63.5	0\\
63.51	0\\
63.52	0\\
63.53	0\\
63.54	0\\
63.55	0\\
63.56	0\\
63.57	0\\
63.58	0\\
63.59	0\\
63.6	0\\
63.61	0\\
63.62	0\\
63.63	0\\
63.64	0\\
63.65	0\\
63.66	0\\
63.67	0\\
63.68	0\\
63.69	0\\
63.7	0\\
63.71	0\\
63.72	0\\
63.73	0\\
63.74	0\\
63.75	0\\
63.76	0\\
63.77	0\\
63.78	0\\
63.79	0\\
63.8	0\\
63.81	0\\
63.82	0\\
63.83	0\\
63.84	0\\
63.85	0\\
63.86	0\\
63.87	0\\
63.88	0\\
63.89	0\\
63.9	0\\
63.91	0\\
63.92	0\\
63.93	0\\
63.94	0\\
63.95	0\\
63.96	0\\
63.97	0\\
63.98	0\\
63.99	0\\
64	0\\
64.01	0\\
64.02	0\\
64.03	0\\
64.04	0\\
64.05	0\\
64.06	0\\
64.07	0\\
64.08	0\\
64.09	0\\
64.1	0\\
64.11	0\\
64.12	0\\
64.13	0\\
64.14	0\\
64.15	0\\
64.16	0\\
64.17	0\\
64.18	0\\
64.19	0\\
64.2	0\\
64.21	0\\
64.22	0\\
64.23	0\\
64.24	0\\
64.25	0\\
64.26	0\\
64.27	0\\
64.28	0\\
64.29	0\\
64.3	0\\
64.31	0\\
64.32	0\\
64.33	0\\
64.34	0\\
64.35	0\\
64.36	0\\
64.37	0\\
64.38	0\\
64.39	0\\
64.4	0\\
64.41	0\\
64.42	0\\
64.43	0\\
64.44	0\\
64.45	0\\
64.46	0\\
64.47	0\\
64.48	0\\
64.49	0\\
64.5	0\\
64.51	0\\
64.52	0\\
64.53	0\\
64.54	0\\
64.55	0\\
64.56	0\\
64.57	0\\
64.58	0\\
64.59	0\\
64.6	0\\
64.61	0\\
64.62	0\\
64.63	0\\
64.64	0\\
64.65	0\\
64.66	0\\
64.67	0\\
64.68	0\\
64.69	0\\
64.7	0\\
64.71	0\\
64.72	0\\
64.73	0\\
64.74	0\\
64.75	0\\
64.76	0\\
64.77	0\\
64.78	0\\
64.79	0\\
64.8	0\\
64.81	0\\
64.82	0\\
64.83	0\\
64.84	0\\
64.85	0\\
64.86	0\\
64.87	0\\
64.88	0\\
64.89	0\\
64.9	0\\
64.91	0\\
64.92	0\\
64.93	0\\
64.94	0\\
64.95	0\\
64.96	0\\
64.97	0\\
64.98	0\\
64.99	0\\
65	0\\
65.01	0\\
65.02	0\\
65.03	0\\
65.04	0\\
65.05	0\\
65.06	0\\
65.07	0\\
65.08	0\\
65.09	0\\
65.1	0\\
65.11	0\\
65.12	0\\
65.13	0\\
65.14	0\\
65.15	0\\
65.16	0\\
65.17	0\\
65.18	0\\
65.19	0\\
65.2	0\\
65.21	0\\
65.22	0\\
65.23	0\\
65.24	0\\
65.25	0\\
65.26	0\\
65.27	0\\
65.28	0\\
65.29	0\\
65.3	0\\
65.31	0\\
65.32	0\\
65.33	0\\
65.34	0\\
65.35	0\\
65.36	0\\
65.37	0\\
65.38	0\\
65.39	0\\
65.4	0\\
65.41	0\\
65.42	0\\
65.43	0\\
65.44	0\\
65.45	0\\
65.46	0\\
65.47	0\\
65.48	0\\
65.49	0\\
65.5	0\\
65.51	0\\
65.52	0\\
65.53	0\\
65.54	0\\
65.55	0\\
65.56	0\\
65.57	0\\
65.58	0\\
65.59	0\\
65.6	0\\
65.61	0\\
65.62	0\\
65.63	0\\
65.64	0\\
65.65	0\\
65.66	0\\
65.67	0\\
65.68	0\\
65.69	0\\
65.7	0\\
65.71	0\\
65.72	0\\
65.73	0\\
65.74	0\\
65.75	0\\
65.76	0\\
65.77	0\\
65.78	0\\
65.79	0\\
65.8	0\\
65.81	0\\
65.82	0\\
65.83	0\\
65.84	0\\
65.85	0\\
65.86	0\\
65.87	0\\
65.88	0\\
65.89	0\\
65.9	0\\
65.91	0\\
65.92	0\\
65.93	0\\
65.94	0\\
65.95	0\\
65.96	0\\
65.97	0\\
65.98	0\\
65.99	0\\
66	0\\
66.01	0\\
66.02	0\\
66.03	0\\
66.04	0\\
66.05	0\\
66.06	0\\
66.07	0\\
66.08	0\\
66.09	0\\
66.1	0\\
66.11	0\\
66.12	0\\
66.13	0\\
66.14	0\\
66.15	0\\
66.16	0\\
66.17	0\\
66.18	0\\
66.19	0\\
66.2	0\\
66.21	0\\
66.22	0\\
66.23	0\\
66.24	0\\
66.25	0\\
66.26	0\\
66.27	0\\
66.28	0\\
66.29	0\\
66.3	0\\
66.31	0\\
66.32	0\\
66.33	0\\
66.34	0\\
66.35	0\\
66.36	0\\
66.37	0\\
66.38	0\\
66.39	0\\
66.4	0\\
66.41	0\\
66.42	0\\
66.43	0\\
66.44	0\\
66.45	0\\
66.46	0\\
66.47	0\\
66.48	0\\
66.49	0\\
66.5	0\\
66.51	0\\
66.52	0\\
66.53	0\\
66.54	0\\
66.55	0\\
66.56	0\\
66.57	0\\
66.58	0\\
66.59	0\\
66.6	0\\
66.61	0\\
66.62	0\\
66.63	0\\
66.64	0\\
66.65	0\\
66.66	0\\
66.67	0\\
66.68	0\\
66.69	0\\
66.7	0\\
66.71	0\\
66.72	0\\
66.73	0\\
66.74	0\\
66.75	0\\
66.76	0\\
66.77	0\\
66.78	0\\
66.79	0\\
66.8	0\\
66.81	0\\
66.82	0\\
66.83	0\\
66.84	0\\
66.85	0\\
66.86	0\\
66.87	0\\
66.88	0\\
66.89	0\\
66.9	0\\
66.91	0\\
66.92	0\\
66.93	0\\
66.94	0\\
66.95	0\\
66.96	0\\
66.97	0\\
66.98	0\\
66.99	0\\
67	0\\
67.01	0\\
67.02	0\\
67.03	0\\
67.04	0\\
67.05	0\\
67.06	0\\
67.07	0\\
67.08	0\\
67.09	0\\
67.1	0\\
67.11	0\\
67.12	0\\
67.13	0\\
67.14	0\\
67.15	0\\
67.16	0\\
67.17	0\\
67.18	0\\
67.19	0\\
67.2	0\\
67.21	0\\
67.22	0\\
67.23	0\\
67.24	0\\
67.25	0\\
67.26	0\\
67.27	0\\
67.28	0\\
67.29	0\\
67.3	0\\
67.31	0\\
67.32	0\\
67.33	0\\
67.34	0\\
67.35	0\\
67.36	0\\
67.37	0\\
67.38	0\\
67.39	0\\
67.4	0\\
67.41	0\\
67.42	0\\
67.43	0\\
67.44	0\\
67.45	0\\
67.46	0\\
67.47	0\\
67.48	0\\
67.49	0\\
67.5	0\\
67.51	0\\
67.52	0\\
67.53	0\\
67.54	0\\
67.55	0\\
67.56	0\\
67.57	0\\
67.58	0\\
67.59	0\\
67.6	0\\
67.61	0\\
67.62	0\\
67.63	0\\
67.64	0\\
67.65	0\\
67.66	0\\
67.67	0\\
67.68	0\\
67.69	0\\
67.7	0\\
67.71	0\\
67.72	0\\
67.73	0\\
67.74	0\\
67.75	0\\
67.76	0\\
67.77	0\\
67.78	0\\
67.79	0\\
67.8	0\\
67.81	0\\
67.82	0\\
67.83	0\\
67.84	0\\
67.85	0\\
67.86	0\\
67.87	0\\
67.88	0\\
67.89	0\\
67.9	0\\
67.91	0\\
67.92	0\\
67.93	0\\
67.94	0\\
67.95	0\\
67.96	0\\
67.97	0\\
67.98	0\\
67.99	0\\
68	0\\
68.01	0\\
68.02	0\\
68.03	0\\
68.04	0\\
68.05	0\\
68.06	0\\
68.07	0\\
68.08	0\\
68.09	0\\
68.1	0\\
68.11	0\\
68.12	0\\
68.13	0\\
68.14	0\\
68.15	0\\
68.16	0\\
68.17	0\\
68.18	0\\
68.19	0\\
68.2	0\\
68.21	0\\
68.22	0\\
68.23	0\\
68.24	0\\
68.25	0\\
68.26	0\\
68.27	0\\
68.28	0\\
68.29	0\\
68.3	0\\
68.31	0\\
68.32	0\\
68.33	0\\
68.34	0\\
68.35	0\\
68.36	0\\
68.37	0\\
68.38	0\\
68.39	0\\
68.4	0\\
68.41	0\\
68.42	0\\
68.43	0\\
68.44	0\\
68.45	0\\
68.46	0\\
68.47	0\\
68.48	0\\
68.49	0\\
68.5	0\\
68.51	0\\
68.52	0\\
68.53	0\\
68.54	0\\
68.55	0\\
68.56	0\\
68.57	0\\
68.58	0\\
68.59	0\\
68.6	0\\
68.61	0\\
68.62	0\\
68.63	0\\
68.64	0\\
68.65	0\\
68.66	0\\
68.67	0\\
68.68	0\\
68.69	0\\
68.7	0\\
68.71	0\\
68.72	0\\
68.73	0\\
68.74	0\\
68.75	0\\
68.76	0\\
68.77	0\\
68.78	0\\
68.79	0\\
68.8	0\\
68.81	0\\
68.82	0\\
68.83	0\\
68.84	0\\
68.85	0\\
68.86	0\\
68.87	0\\
68.88	0\\
68.89	0\\
68.9	0\\
68.91	0\\
68.92	0\\
68.93	0\\
68.94	0\\
68.95	0\\
68.96	0\\
68.97	0\\
68.98	0\\
68.99	0\\
69	0\\
69.01	0\\
69.02	0\\
69.03	0\\
69.04	0\\
69.05	0\\
69.06	0\\
69.07	0\\
69.08	0\\
69.09	0\\
69.1	0\\
69.11	0\\
69.12	0\\
69.13	0\\
69.14	0\\
69.15	0\\
69.16	0\\
69.17	0\\
69.18	0\\
69.19	0\\
69.2	0\\
69.21	0\\
69.22	0\\
69.23	0\\
69.24	0\\
69.25	0\\
69.26	0\\
69.27	0\\
69.28	0\\
69.29	0\\
69.3	0\\
69.31	0\\
69.32	0\\
69.33	0\\
69.34	0\\
69.35	0\\
69.36	0\\
69.37	0\\
69.38	0\\
69.39	0\\
69.4	0\\
69.41	0\\
69.42	0\\
69.43	0\\
69.44	0\\
69.45	0\\
69.46	0\\
69.47	0\\
69.48	0\\
69.49	0\\
69.5	0\\
69.51	0\\
69.52	0\\
69.53	0\\
69.54	0\\
69.55	0\\
69.56	0\\
69.57	0\\
69.58	0\\
69.59	0\\
69.6	0\\
69.61	0\\
69.62	0\\
69.63	0\\
69.64	0\\
69.65	0\\
69.66	0\\
69.67	0\\
69.68	0\\
69.69	0\\
69.7	0\\
69.71	0\\
69.72	0\\
69.73	0\\
69.74	0\\
69.75	0\\
69.76	0\\
69.77	0\\
69.78	0\\
69.79	0\\
69.8	0\\
69.81	0\\
69.82	0\\
69.83	0\\
69.84	0\\
69.85	0\\
69.86	0\\
69.87	0\\
69.88	0\\
69.89	0\\
69.9	0\\
69.91	0\\
69.92	0\\
69.93	0\\
69.94	0\\
69.95	0\\
69.96	0\\
69.97	0\\
69.98	0\\
69.99	0\\
70	0\\
70.01	0\\
70.02	0\\
70.03	0\\
70.04	0\\
70.05	0\\
70.06	0\\
70.07	0\\
70.08	0\\
70.09	0\\
70.1	0\\
70.11	0\\
70.12	0\\
70.13	0\\
70.14	0\\
70.15	0\\
70.16	0\\
70.17	0\\
70.18	0\\
70.19	0\\
70.2	0\\
70.21	0\\
70.22	0\\
70.23	0\\
70.24	0\\
70.25	0\\
70.26	0\\
70.27	0\\
70.28	0\\
70.29	0\\
70.3	0\\
70.31	0\\
70.32	0\\
70.33	0\\
70.34	0\\
70.35	0\\
70.36	0\\
70.37	0\\
70.38	0\\
70.39	0\\
70.4	0\\
70.41	0\\
70.42	0\\
70.43	0\\
70.44	0\\
70.45	0\\
70.46	0\\
70.47	0\\
70.48	0\\
70.49	0\\
70.5	0\\
70.51	0\\
70.52	0\\
70.53	0\\
70.54	0\\
70.55	0\\
70.56	0\\
70.57	0\\
70.58	0\\
70.59	0\\
70.6	0\\
70.61	0\\
70.62	0\\
70.63	0\\
70.64	0\\
70.65	0\\
70.66	0\\
70.67	0\\
70.68	0\\
70.69	0\\
70.7	0\\
70.71	0\\
70.72	0\\
70.73	0\\
70.74	0\\
70.75	0\\
70.76	0\\
70.77	0\\
70.78	0\\
70.79	0\\
70.8	0\\
70.81	0\\
70.82	0\\
70.83	0\\
70.84	0\\
70.85	0\\
70.86	0\\
70.87	0\\
70.88	0\\
70.89	0\\
70.9	0\\
70.91	0\\
70.92	0\\
70.93	0\\
70.94	0\\
70.95	0\\
70.96	0\\
70.97	0\\
70.98	0\\
70.99	0\\
71	0\\
71.01	0\\
71.02	0\\
71.03	0\\
71.04	0\\
71.05	0\\
71.06	0\\
71.07	0\\
71.08	0\\
71.09	0\\
71.1	0\\
71.11	0\\
71.12	0\\
71.13	0\\
71.14	0\\
71.15	0\\
71.16	0\\
71.17	0\\
71.18	0\\
71.19	0\\
71.2	0\\
71.21	0\\
71.22	0\\
71.23	0\\
71.24	0\\
71.25	0\\
71.26	0\\
71.27	0\\
71.28	0\\
71.29	0\\
71.3	0\\
71.31	0\\
71.32	0\\
71.33	0\\
71.34	0\\
71.35	0\\
71.36	0\\
71.37	0\\
71.38	0\\
71.39	0\\
71.4	0\\
71.41	0\\
71.42	0\\
71.43	0\\
71.44	0\\
71.45	0\\
71.46	0\\
71.47	0\\
71.48	0\\
71.49	0\\
71.5	0\\
71.51	0\\
71.52	0\\
71.53	0\\
71.54	0\\
71.55	0\\
71.56	0\\
71.57	0\\
71.58	0\\
71.59	0\\
71.6	0\\
71.61	0\\
71.62	0\\
71.63	0\\
71.64	0\\
71.65	0\\
71.66	0\\
71.67	0\\
71.68	0\\
71.69	0\\
71.7	0\\
71.71	0\\
71.72	0\\
71.73	0\\
71.74	0\\
71.75	0\\
71.76	0\\
71.77	0\\
71.78	0\\
71.79	0\\
71.8	0\\
71.81	0\\
71.82	0\\
71.83	0\\
71.84	0\\
71.85	0\\
71.86	0\\
71.87	0\\
71.88	0\\
71.89	0\\
71.9	0\\
71.91	0\\
71.92	0\\
71.93	0\\
71.94	0\\
71.95	0\\
71.96	0\\
71.97	0\\
71.98	0\\
71.99	0\\
72	0\\
72.01	0\\
72.02	0\\
72.03	0\\
72.04	0\\
72.05	0\\
72.06	0\\
72.07	0\\
72.08	0\\
72.09	0\\
72.1	0\\
72.11	0\\
72.12	0\\
72.13	0\\
72.14	0\\
72.15	0\\
72.16	0\\
72.17	0\\
72.18	0\\
72.19	0\\
72.2	0\\
72.21	0\\
72.22	0\\
72.23	0\\
72.24	0\\
72.25	0\\
72.26	0\\
72.27	0\\
72.28	0\\
72.29	0\\
72.3	0\\
72.31	0\\
72.32	0\\
72.33	0\\
72.34	0\\
72.35	0\\
72.36	0\\
72.37	0\\
72.38	0\\
72.39	0\\
72.4	0\\
72.41	0\\
72.42	0\\
72.43	0\\
72.44	0\\
72.45	0\\
72.46	0\\
72.47	0\\
72.48	0\\
72.49	0\\
72.5	0\\
72.51	0\\
72.52	0\\
72.53	0\\
72.54	0\\
72.55	0\\
72.56	0\\
72.57	0\\
72.58	0\\
72.59	0\\
72.6	0\\
72.61	0\\
72.62	0\\
72.63	0\\
72.64	0\\
72.65	0\\
72.66	0\\
72.67	0\\
72.68	0\\
72.69	0\\
72.7	0\\
72.71	0\\
72.72	0\\
72.73	0\\
72.74	0\\
72.75	0\\
72.76	0\\
72.77	0\\
72.78	0\\
72.79	0\\
72.8	0\\
72.81	0\\
72.82	0\\
72.83	0\\
72.84	0\\
72.85	0\\
72.86	0\\
72.87	0\\
72.88	0\\
72.89	0\\
72.9	0\\
72.91	0\\
72.92	0\\
72.93	0\\
72.94	0\\
72.95	0\\
72.96	0\\
72.97	0\\
72.98	0\\
72.99	0\\
73	0\\
73.01	0\\
73.02	0\\
73.03	0\\
73.04	0\\
73.05	0\\
73.06	0\\
73.07	0\\
73.08	0\\
73.09	0\\
73.1	0\\
73.11	0\\
73.12	0\\
73.13	0\\
73.14	0\\
73.15	0\\
73.16	0\\
73.17	0\\
73.18	0\\
73.19	0\\
73.2	0\\
73.21	0\\
73.22	0\\
73.23	0\\
73.24	0\\
73.25	0\\
73.26	0\\
73.27	0\\
73.28	0\\
73.29	0\\
73.3	0\\
73.31	0\\
73.32	0\\
73.33	0\\
73.34	0\\
73.35	0\\
73.36	0\\
73.37	0\\
73.38	0\\
73.39	0\\
73.4	0\\
73.41	0\\
73.42	0\\
73.43	0\\
73.44	0\\
73.45	0\\
73.46	0\\
73.47	0\\
73.48	0\\
73.49	0\\
73.5	0\\
73.51	0\\
73.52	0\\
73.53	0\\
73.54	0\\
73.55	0\\
73.56	0\\
73.57	0\\
73.58	0\\
73.59	0\\
73.6	0\\
73.61	0\\
73.62	0\\
73.63	0\\
73.64	0\\
73.65	0\\
73.66	0\\
73.67	0\\
73.68	0\\
73.69	0\\
73.7	0\\
73.71	0\\
73.72	0\\
73.73	0\\
73.74	0\\
73.75	0\\
73.76	0\\
73.77	0\\
73.78	0\\
73.79	0\\
73.8	0\\
73.81	0\\
73.82	0\\
73.83	0\\
73.84	0\\
73.85	0\\
73.86	0\\
73.87	0\\
73.88	0\\
73.89	0\\
73.9	0\\
73.91	0\\
73.92	0\\
73.93	0\\
73.94	0\\
73.95	0\\
73.96	0\\
73.97	0\\
73.98	0\\
73.99	0\\
74	0\\
74.01	0\\
74.02	0\\
74.03	0\\
74.04	0\\
74.05	0\\
74.06	0\\
74.07	0\\
74.08	0\\
74.09	0\\
74.1	0\\
74.11	0\\
74.12	0\\
74.13	0\\
74.14	0\\
74.15	0\\
74.16	0\\
74.17	0\\
74.18	0\\
74.19	0\\
74.2	0\\
74.21	0\\
74.22	0\\
74.23	0\\
74.24	0\\
74.25	0\\
74.26	0\\
74.27	0\\
74.28	0\\
74.29	0\\
74.3	0\\
74.31	0\\
74.32	0\\
74.33	0\\
74.34	0\\
74.35	0\\
74.36	0\\
74.37	0\\
74.38	0\\
74.39	0\\
74.4	0\\
74.41	0\\
74.42	0\\
74.43	0\\
74.44	0\\
74.45	0\\
74.46	0\\
74.47	0\\
74.48	0\\
74.49	0\\
74.5	0\\
74.51	0\\
74.52	0\\
74.53	0\\
74.54	0\\
74.55	0\\
74.56	0\\
74.57	0\\
74.58	0\\
74.59	0\\
74.6	0\\
74.61	0\\
74.62	0\\
74.63	0\\
74.64	0\\
74.65	0\\
74.66	0\\
74.67	0\\
74.68	0\\
74.69	0\\
74.7	0\\
74.71	0\\
74.72	0\\
74.73	0\\
74.74	0\\
74.75	0\\
74.76	0\\
74.77	0\\
74.78	0\\
74.79	0\\
74.8	0\\
74.81	0\\
74.82	0\\
74.83	0\\
74.84	0\\
74.85	0\\
74.86	0\\
74.87	0\\
74.88	0\\
74.89	0\\
74.9	0\\
74.91	0\\
74.92	0\\
74.93	0\\
74.94	0\\
74.95	0\\
74.96	0\\
74.97	0\\
74.98	0\\
74.99	0\\
75	0\\
75.01	0\\
75.02	0\\
75.03	0\\
75.04	0\\
75.05	0\\
75.06	0\\
75.07	0\\
75.08	0\\
75.09	0\\
75.1	0\\
75.11	0\\
75.12	0\\
75.13	0\\
75.14	0\\
75.15	0\\
75.16	0\\
75.17	0\\
75.18	0\\
75.19	0\\
75.2	0\\
75.21	0\\
75.22	0\\
75.23	0\\
75.24	0\\
75.25	0\\
75.26	0\\
75.27	0\\
75.28	0\\
75.29	0\\
75.3	0\\
75.31	0\\
75.32	0\\
75.33	0\\
75.34	0\\
75.35	0\\
75.36	0\\
75.37	0\\
75.38	0\\
75.39	0\\
75.4	0\\
75.41	0\\
75.42	0\\
75.43	0\\
75.44	0\\
75.45	0\\
75.46	0\\
75.47	0\\
75.48	0\\
75.49	0\\
75.5	0\\
75.51	0\\
75.52	0\\
75.53	0\\
75.54	0\\
75.55	0\\
75.56	0\\
75.57	0\\
75.58	0\\
75.59	0\\
75.6	0\\
75.61	0\\
75.62	0\\
75.63	0\\
75.64	0\\
75.65	0\\
75.66	0\\
75.67	0\\
75.68	0\\
75.69	0\\
75.7	0\\
75.71	0\\
75.72	0\\
75.73	0\\
75.74	0\\
75.75	0\\
75.76	0\\
75.77	0\\
75.78	0\\
75.79	0\\
75.8	0\\
75.81	0\\
75.82	0\\
75.83	0\\
75.84	0\\
75.85	0\\
75.86	0\\
75.87	0\\
75.88	0\\
75.89	0\\
75.9	0\\
75.91	0\\
75.92	0\\
75.93	0\\
75.94	0\\
75.95	0\\
75.96	0\\
75.97	0\\
75.98	0\\
75.99	0\\
76	0\\
76.01	0\\
76.02	0\\
76.03	0\\
76.04	0\\
76.05	0\\
76.06	0\\
76.07	0\\
76.08	0\\
76.09	0\\
76.1	0\\
76.11	0\\
76.12	0\\
76.13	0\\
76.14	0\\
76.15	0\\
76.16	0\\
76.17	0\\
76.18	0\\
76.19	0\\
76.2	0\\
76.21	0\\
76.22	0\\
76.23	0\\
76.24	0\\
76.25	0\\
76.26	0\\
76.27	0\\
76.28	0\\
76.29	0\\
76.3	0\\
76.31	0\\
76.32	0\\
76.33	0\\
76.34	0\\
76.35	0\\
76.36	0\\
76.37	0\\
76.38	0\\
76.39	0\\
76.4	0\\
76.41	0\\
76.42	0\\
76.43	0\\
76.44	0\\
76.45	0\\
76.46	0\\
76.47	0\\
76.48	0\\
76.49	0\\
76.5	0\\
76.51	0\\
76.52	0\\
76.53	0\\
76.54	0\\
76.55	0\\
76.56	0\\
76.57	0\\
76.58	0\\
76.59	0\\
76.6	0\\
76.61	0\\
76.62	0\\
76.63	0\\
76.64	0\\
76.65	0\\
76.66	0\\
76.67	0\\
76.68	0\\
76.69	0\\
76.7	0\\
76.71	0\\
76.72	0\\
76.73	0\\
76.74	0\\
76.75	0\\
76.76	0\\
76.77	0\\
76.78	0\\
76.79	0\\
76.8	0\\
76.81	0\\
76.82	0\\
76.83	0\\
76.84	0\\
76.85	0\\
76.86	0\\
76.87	0\\
76.88	0\\
76.89	0\\
76.9	0\\
76.91	0\\
76.92	0\\
76.93	0\\
76.94	0\\
76.95	0\\
76.96	0\\
76.97	0\\
76.98	0\\
76.99	0\\
77	0\\
77.01	0\\
77.02	0\\
77.03	0\\
77.04	0\\
77.05	0\\
77.06	0\\
77.07	0\\
77.08	0\\
77.09	0\\
77.1	0\\
77.11	0\\
77.12	0\\
77.13	0\\
77.14	0\\
77.15	0\\
77.16	0\\
77.17	0\\
77.18	0\\
77.19	0\\
77.2	0\\
77.21	0\\
77.22	0\\
77.23	0\\
77.24	0\\
77.25	0\\
77.26	0\\
77.27	0\\
77.28	0\\
77.29	0\\
77.3	0\\
77.31	0\\
77.32	0\\
77.33	0\\
77.34	0\\
77.35	0\\
77.36	0\\
77.37	0\\
77.38	0\\
77.39	0\\
77.4	0\\
77.41	0\\
77.42	0\\
77.43	0\\
77.44	0\\
77.45	0\\
77.46	0\\
77.47	0\\
77.48	0\\
77.49	0\\
77.5	0\\
77.51	0\\
77.52	0\\
77.53	0\\
77.54	0\\
77.55	0\\
77.56	0\\
77.57	0\\
77.58	0\\
77.59	0\\
77.6	0\\
77.61	0\\
77.62	0\\
77.63	0\\
77.64	0\\
77.65	0\\
77.66	0\\
77.67	0\\
77.68	0\\
77.69	0\\
77.7	0\\
77.71	0\\
77.72	0\\
77.73	0\\
77.74	0\\
77.75	0\\
77.76	0\\
77.77	0\\
77.78	0\\
77.79	0\\
77.8	0\\
77.81	0\\
77.82	0\\
77.83	0\\
77.84	0\\
77.85	0\\
77.86	0\\
77.87	0\\
77.88	0\\
77.89	0\\
77.9	0\\
77.91	0\\
77.92	0\\
77.93	0\\
77.94	0\\
77.95	0\\
77.96	0\\
77.97	0\\
77.98	0\\
77.99	0\\
78	0\\
78.01	0\\
78.02	0\\
78.03	0\\
78.04	0\\
78.05	0\\
78.06	0\\
78.07	0\\
78.08	0\\
78.09	0\\
78.1	0\\
78.11	0\\
78.12	0\\
78.13	0\\
78.14	0\\
78.15	0\\
78.16	0\\
78.17	0\\
78.18	0\\
78.19	0\\
78.2	0\\
78.21	0\\
78.22	0\\
78.23	0\\
78.24	0\\
78.25	0\\
78.26	0\\
78.27	0\\
78.28	0\\
78.29	0\\
78.3	0\\
78.31	0\\
78.32	0\\
78.33	0\\
78.34	0\\
78.35	0\\
78.36	0\\
78.37	0\\
78.38	0\\
78.39	0\\
78.4	0\\
78.41	0\\
78.42	0\\
78.43	0\\
78.44	0\\
78.45	0\\
78.46	0\\
78.47	0\\
78.48	0\\
78.49	0\\
78.5	0\\
78.51	0\\
78.52	0\\
78.53	0\\
78.54	0\\
78.55	0\\
78.56	0\\
78.57	0\\
78.58	0\\
78.59	0\\
78.6	0\\
78.61	0\\
78.62	0\\
78.63	0\\
78.64	0\\
78.65	0\\
78.66	0\\
78.67	0\\
78.68	0\\
78.69	0\\
78.7	0\\
78.71	0\\
78.72	0\\
78.73	0\\
78.74	0\\
78.75	0\\
78.76	0\\
78.77	0\\
78.78	0\\
78.79	0\\
78.8	0\\
78.81	0\\
78.82	0\\
78.83	0\\
78.84	0\\
78.85	0\\
78.86	0\\
78.87	0\\
78.88	0\\
78.89	0\\
78.9	0\\
78.91	0\\
78.92	0\\
78.93	0\\
78.94	0\\
78.95	0\\
78.96	0\\
78.97	0\\
78.98	0\\
78.99	0\\
79	0\\
79.01	0\\
79.02	0\\
79.03	0\\
79.04	0\\
79.05	0\\
79.06	0\\
79.07	0\\
79.08	0\\
79.09	0\\
79.1	0\\
79.11	0\\
79.12	0\\
79.13	0\\
79.14	0\\
79.15	0\\
79.16	0\\
79.17	0\\
79.18	0\\
79.19	0\\
79.2	0\\
79.21	0\\
79.22	0\\
79.23	0\\
79.24	0\\
79.25	0\\
79.26	0\\
79.27	0\\
79.28	0\\
79.29	0\\
79.3	0\\
79.31	0\\
79.32	0\\
79.33	0\\
79.34	0\\
79.35	0\\
79.36	0\\
79.37	0\\
79.38	0\\
79.39	0\\
79.4	0\\
79.41	0\\
79.42	0\\
79.43	0\\
79.44	0\\
79.45	0\\
79.46	0\\
79.47	0\\
79.48	0\\
79.49	0\\
79.5	0\\
79.51	0\\
79.52	0\\
79.53	0\\
79.54	0\\
79.55	0\\
79.56	0\\
79.57	0\\
79.58	0\\
79.59	0\\
79.6	0\\
79.61	0\\
79.62	0\\
79.63	0\\
79.64	0\\
79.65	0\\
79.66	0\\
79.67	0\\
79.68	0\\
79.69	0\\
79.7	0\\
79.71	0\\
79.72	0\\
79.73	0\\
79.74	0\\
79.75	0\\
79.76	0\\
79.77	0\\
79.78	0\\
79.79	0\\
79.8	0\\
79.81	0\\
79.82	0\\
79.83	0\\
79.84	0\\
79.85	0\\
79.86	0\\
79.87	0\\
79.88	0\\
79.89	0\\
79.9	0\\
79.91	0\\
79.92	0\\
79.93	0\\
79.94	0\\
79.95	0\\
79.96	0\\
79.97	0\\
79.98	0\\
79.99	0\\
80	0\\
80.01	0\\
};
\addplot [color=red,solid]
  table[row sep=crcr]{%
80.01	0\\
80.02	0\\
80.03	0\\
80.04	0\\
80.05	0\\
80.06	0\\
80.07	0\\
80.08	0\\
80.09	0\\
80.1	0\\
80.11	0\\
80.12	0\\
80.13	0\\
80.14	0\\
80.15	0\\
80.16	0\\
80.17	0\\
80.18	0\\
80.19	0\\
80.2	0\\
80.21	0\\
80.22	0\\
80.23	0\\
80.24	0\\
80.25	0\\
80.26	0\\
80.27	0\\
80.28	0\\
80.29	0\\
80.3	0\\
80.31	0\\
80.32	0\\
80.33	0\\
80.34	0\\
80.35	0\\
80.36	0\\
80.37	0\\
80.38	0\\
80.39	0\\
80.4	0\\
80.41	0\\
80.42	0\\
80.43	0\\
80.44	0\\
80.45	0\\
80.46	0\\
80.47	0\\
80.48	0\\
80.49	0\\
80.5	0\\
80.51	0\\
80.52	0\\
80.53	0\\
80.54	0\\
80.55	0\\
80.56	0\\
80.57	0\\
80.58	0\\
80.59	0\\
80.6	0\\
80.61	0\\
80.62	0\\
80.63	0\\
80.64	0\\
80.65	0\\
80.66	0\\
80.67	0\\
80.68	0\\
80.69	0\\
80.7	0\\
80.71	0\\
80.72	0\\
80.73	0\\
80.74	0\\
80.75	0\\
80.76	0\\
80.77	0\\
80.78	0\\
80.79	0\\
80.8	0\\
80.81	0\\
80.82	0\\
80.83	0\\
80.84	0\\
80.85	0\\
80.86	0\\
80.87	0\\
80.88	0\\
80.89	0\\
80.9	0\\
80.91	0\\
80.92	0\\
80.93	0\\
80.94	0\\
80.95	0\\
80.96	0\\
80.97	0\\
80.98	0\\
80.99	0\\
81	0\\
81.01	0\\
81.02	0\\
81.03	0\\
81.04	0\\
81.05	0\\
81.06	0\\
81.07	0\\
81.08	0\\
81.09	0\\
81.1	0\\
81.11	0\\
81.12	0\\
81.13	0\\
81.14	0\\
81.15	0\\
81.16	0\\
81.17	0\\
81.18	0\\
81.19	0\\
81.2	0\\
81.21	0\\
81.22	0\\
81.23	0\\
81.24	0\\
81.25	0\\
81.26	0\\
81.27	0\\
81.28	0\\
81.29	0\\
81.3	0\\
81.31	0\\
81.32	0\\
81.33	0\\
81.34	0\\
81.35	0\\
81.36	0\\
81.37	0\\
81.38	0\\
81.39	0\\
81.4	0\\
81.41	0\\
81.42	0\\
81.43	0\\
81.44	0\\
81.45	0\\
81.46	0\\
81.47	0\\
81.48	0\\
81.49	0\\
81.5	0\\
81.51	0\\
81.52	0\\
81.53	0\\
81.54	0\\
81.55	0\\
81.56	0\\
81.57	0\\
81.58	0\\
81.59	0\\
81.6	0\\
81.61	0\\
81.62	0\\
81.63	0\\
81.64	0\\
81.65	0\\
81.66	0\\
81.67	0\\
81.68	0\\
81.69	0\\
81.7	0\\
81.71	0\\
81.72	0\\
81.73	0\\
81.74	0\\
81.75	0\\
81.76	0\\
81.77	0\\
81.78	0\\
81.79	0\\
81.8	0\\
81.81	0\\
81.82	0\\
81.83	0\\
81.84	0\\
81.85	0\\
81.86	0\\
81.87	0\\
81.88	0\\
81.89	0\\
81.9	0\\
81.91	0\\
81.92	0\\
81.93	0\\
81.94	0\\
81.95	0\\
81.96	0\\
81.97	0\\
81.98	0\\
81.99	0\\
82	0\\
82.01	0\\
82.02	0\\
82.03	0\\
82.04	0\\
82.05	0\\
82.06	0\\
82.07	0\\
82.08	0\\
82.09	0\\
82.1	0\\
82.11	0\\
82.12	0\\
82.13	0\\
82.14	0\\
82.15	0\\
82.16	0\\
82.17	0\\
82.18	0\\
82.19	0\\
82.2	0\\
82.21	0\\
82.22	0\\
82.23	0\\
82.24	0\\
82.25	0\\
82.26	0\\
82.27	0\\
82.28	0\\
82.29	0\\
82.3	0\\
82.31	0\\
82.32	0\\
82.33	0\\
82.34	0\\
82.35	0\\
82.36	0\\
82.37	0\\
82.38	0\\
82.39	0\\
82.4	0\\
82.41	0\\
82.42	0\\
82.43	0\\
82.44	0\\
82.45	0\\
82.46	0\\
82.47	0\\
82.48	0\\
82.49	0\\
82.5	0\\
82.51	0\\
82.52	0\\
82.53	0\\
82.54	0\\
82.55	0\\
82.56	0\\
82.57	0\\
82.58	0\\
82.59	0\\
82.6	0\\
82.61	0\\
82.62	0\\
82.63	0\\
82.64	0\\
82.65	0\\
82.66	0\\
82.67	0\\
82.68	0\\
82.69	0\\
82.7	0\\
82.71	0\\
82.72	0\\
82.73	0\\
82.74	0\\
82.75	0\\
82.76	0\\
82.77	0\\
82.78	0\\
82.79	0\\
82.8	0\\
82.81	0\\
82.82	0\\
82.83	0\\
82.84	0\\
82.85	0\\
82.86	0\\
82.87	0\\
82.88	0\\
82.89	0\\
82.9	0\\
82.91	0\\
82.92	0\\
82.93	0\\
82.94	0\\
82.95	0\\
82.96	0\\
82.97	0\\
82.98	0\\
82.99	0\\
83	0\\
83.01	0\\
83.02	0\\
83.03	0\\
83.04	0\\
83.05	0\\
83.06	0\\
83.07	0\\
83.08	0\\
83.09	0\\
83.1	0\\
83.11	0\\
83.12	0\\
83.13	0\\
83.14	0\\
83.15	0\\
83.16	0\\
83.17	0\\
83.18	0\\
83.19	0\\
83.2	0\\
83.21	0\\
83.22	0\\
83.23	0\\
83.24	0\\
83.25	0\\
83.26	0\\
83.27	0\\
83.28	0\\
83.29	0\\
83.3	0\\
83.31	0\\
83.32	0\\
83.33	0\\
83.34	0\\
83.35	0\\
83.36	0\\
83.37	0\\
83.38	0\\
83.39	0\\
83.4	0\\
83.41	0\\
83.42	0\\
83.43	0\\
83.44	0\\
83.45	0\\
83.46	0\\
83.47	0\\
83.48	0\\
83.49	0\\
83.5	0\\
83.51	0\\
83.52	0\\
83.53	0\\
83.54	0\\
83.55	0\\
83.56	0\\
83.57	0\\
83.58	0\\
83.59	0\\
83.6	0\\
83.61	0\\
83.62	0\\
83.63	0\\
83.64	0\\
83.65	0\\
83.66	0\\
83.67	0\\
83.68	0\\
83.69	0\\
83.7	0\\
83.71	0\\
83.72	0\\
83.73	0\\
83.74	0\\
83.75	0\\
83.76	0\\
83.77	0\\
83.78	0\\
83.79	0\\
83.8	0\\
83.81	0\\
83.82	0\\
83.83	0\\
83.84	0\\
83.85	0\\
83.86	0\\
83.87	0\\
83.88	0\\
83.89	0\\
83.9	0\\
83.91	0\\
83.92	0\\
83.93	0\\
83.94	0\\
83.95	0\\
83.96	0\\
83.97	0\\
83.98	0\\
83.99	0\\
84	0\\
84.01	0\\
84.02	0\\
84.03	0\\
84.04	0\\
84.05	0\\
84.06	0\\
84.07	0\\
84.08	0\\
84.09	0\\
84.1	0\\
84.11	0\\
84.12	0\\
84.13	0\\
84.14	0\\
84.15	0\\
84.16	0\\
84.17	0\\
84.18	0\\
84.19	0\\
84.2	0\\
84.21	0\\
84.22	0\\
84.23	0\\
84.24	0\\
84.25	0\\
84.26	0\\
84.27	0\\
84.28	0\\
84.29	0\\
84.3	0\\
84.31	0\\
84.32	0\\
84.33	0\\
84.34	0\\
84.35	0\\
84.36	0\\
84.37	0\\
84.38	0\\
84.39	0\\
84.4	0\\
84.41	0\\
84.42	0\\
84.43	0\\
84.44	0\\
84.45	0\\
84.46	0\\
84.47	0\\
84.48	0\\
84.49	0\\
84.5	0\\
84.51	0\\
84.52	0\\
84.53	0\\
84.54	0\\
84.55	0\\
84.56	0\\
84.57	0\\
84.58	0\\
84.59	0\\
84.6	0\\
84.61	0\\
84.62	0\\
84.63	0\\
84.64	0\\
84.65	0\\
84.66	0\\
84.67	0\\
84.68	0\\
84.69	0\\
84.7	0\\
84.71	0\\
84.72	0\\
84.73	0\\
84.74	0\\
84.75	0\\
84.76	0\\
84.77	0\\
84.78	0\\
84.79	0\\
84.8	0\\
84.81	0\\
84.82	0\\
84.83	0\\
84.84	0\\
84.85	0\\
84.86	0\\
84.87	0\\
84.88	0\\
84.89	0\\
84.9	0\\
84.91	0\\
84.92	0\\
84.93	0\\
84.94	0\\
84.95	0\\
84.96	0\\
84.97	0\\
84.98	0\\
84.99	0\\
85	0\\
85.01	0\\
85.02	0\\
85.03	0\\
85.04	0\\
85.05	0\\
85.06	0\\
85.07	0\\
85.08	0\\
85.09	0\\
85.1	0\\
85.11	0\\
85.12	0\\
85.13	0\\
85.14	0\\
85.15	0\\
85.16	0\\
85.17	0\\
85.18	0\\
85.19	0\\
85.2	0\\
85.21	0\\
85.22	0\\
85.23	0\\
85.24	0\\
85.25	0\\
85.26	0\\
85.27	0\\
85.28	0\\
85.29	0\\
85.3	0\\
85.31	0\\
85.32	0\\
85.33	0\\
85.34	0\\
85.35	0\\
85.36	0\\
85.37	0\\
85.38	0\\
85.39	0\\
85.4	0\\
85.41	0\\
85.42	0\\
85.43	0\\
85.44	0\\
85.45	0\\
85.46	0\\
85.47	0\\
85.48	0\\
85.49	0\\
85.5	0\\
85.51	0\\
85.52	0\\
85.53	0\\
85.54	0\\
85.55	0\\
85.56	0\\
85.57	0\\
85.58	0\\
85.59	0\\
85.6	0\\
85.61	0\\
85.62	0\\
85.63	0\\
85.64	0\\
85.65	0\\
85.66	0\\
85.67	0\\
85.68	0\\
85.69	0\\
85.7	0\\
85.71	0\\
85.72	0\\
85.73	0\\
85.74	0\\
85.75	0\\
85.76	0\\
85.77	0\\
85.78	0\\
85.79	0\\
85.8	0\\
85.81	0\\
85.82	0\\
85.83	0\\
85.84	0\\
85.85	0\\
85.86	0\\
85.87	0\\
85.88	0\\
85.89	0\\
85.9	0\\
85.91	0\\
85.92	0\\
85.93	0\\
85.94	0\\
85.95	0\\
85.96	0\\
85.97	0\\
85.98	0\\
85.99	0\\
86	0\\
86.01	0\\
86.02	0\\
86.03	0\\
86.04	0\\
86.05	0\\
86.06	0\\
86.07	0\\
86.08	0\\
86.09	0\\
86.1	0\\
86.11	0\\
86.12	0\\
86.13	0\\
86.14	0\\
86.15	0\\
86.16	0\\
86.17	0\\
86.18	0\\
86.19	0\\
86.2	0\\
86.21	0\\
86.22	0\\
86.23	0\\
86.24	0\\
86.25	0\\
86.26	0\\
86.27	0\\
86.28	0\\
86.29	0\\
86.3	0\\
86.31	0\\
86.32	0\\
86.33	0\\
86.34	0\\
86.35	0\\
86.36	0\\
86.37	0\\
86.38	0\\
86.39	0\\
86.4	0\\
86.41	0\\
86.42	0\\
86.43	0\\
86.44	0\\
86.45	0\\
86.46	0\\
86.47	0\\
86.48	0\\
86.49	0\\
86.5	0\\
86.51	0\\
86.52	0\\
86.53	0\\
86.54	0\\
86.55	0\\
86.56	0\\
86.57	0\\
86.58	0\\
86.59	0\\
86.6	0\\
86.61	0\\
86.62	0\\
86.63	0\\
86.64	0\\
86.65	0\\
86.66	0\\
86.67	0\\
86.68	0\\
86.69	0\\
86.7	0\\
86.71	0\\
86.72	0\\
86.73	0\\
86.74	0\\
86.75	0\\
86.76	0\\
86.77	0\\
86.78	0\\
86.79	0\\
86.8	0\\
86.81	0\\
86.82	0\\
86.83	0\\
86.84	0\\
86.85	0\\
86.86	0\\
86.87	0\\
86.88	0\\
86.89	0\\
86.9	0\\
86.91	0\\
86.92	0\\
86.93	0\\
86.94	0\\
86.95	0\\
86.96	0\\
86.97	0\\
86.98	0\\
86.99	0\\
87	0\\
87.01	0\\
87.02	0\\
87.03	0\\
87.04	0\\
87.05	0\\
87.06	0\\
87.07	0\\
87.08	0\\
87.09	0\\
87.1	0\\
87.11	0\\
87.12	0\\
87.13	0\\
87.14	0\\
87.15	0\\
87.16	0\\
87.17	0\\
87.18	0\\
87.19	0\\
87.2	0\\
87.21	0\\
87.22	0\\
87.23	0\\
87.24	0\\
87.25	0\\
87.26	0\\
87.27	0\\
87.28	0\\
87.29	0\\
87.3	0\\
87.31	0\\
87.32	0\\
87.33	0\\
87.34	0\\
87.35	0\\
87.36	0\\
87.37	0\\
87.38	0\\
87.39	0\\
87.4	0\\
87.41	0\\
87.42	0\\
87.43	0\\
87.44	0\\
87.45	0\\
87.46	0\\
87.47	0\\
87.48	0\\
87.49	0\\
87.5	0\\
87.51	0\\
87.52	0\\
87.53	0\\
87.54	0\\
87.55	0\\
87.56	0\\
87.57	0\\
87.58	0\\
87.59	0\\
87.6	0\\
87.61	0\\
87.62	0\\
87.63	0\\
87.64	0\\
87.65	0\\
87.66	0\\
87.67	0\\
87.68	0\\
87.69	0\\
87.7	0\\
87.71	0\\
87.72	0\\
87.73	0\\
87.74	0\\
87.75	0\\
87.76	0\\
87.77	0\\
87.78	0\\
87.79	0\\
87.8	0\\
87.81	0\\
87.82	0\\
87.83	0\\
87.84	0\\
87.85	0\\
87.86	0\\
87.87	0\\
87.88	0\\
87.89	0\\
87.9	0\\
87.91	0\\
87.92	0\\
87.93	0\\
87.94	0\\
87.95	0\\
87.96	0\\
87.97	0\\
87.98	0\\
87.99	0\\
88	0\\
88.01	0\\
88.02	0\\
88.03	0\\
88.04	0\\
88.05	0\\
88.06	0\\
88.07	0\\
88.08	0\\
88.09	0\\
88.1	0\\
88.11	0\\
88.12	0\\
88.13	0\\
88.14	0\\
88.15	0\\
88.16	0\\
88.17	0\\
88.18	0\\
88.19	0\\
88.2	0\\
88.21	0\\
88.22	0\\
88.23	0\\
88.24	0\\
88.25	0\\
88.26	0\\
88.27	0\\
88.28	0\\
88.29	0\\
88.3	0\\
88.31	0\\
88.32	0\\
88.33	0\\
88.34	0\\
88.35	0\\
88.36	0\\
88.37	0\\
88.38	0\\
88.39	0\\
88.4	0\\
88.41	0\\
88.42	0\\
88.43	0\\
88.44	0\\
88.45	0\\
88.46	0\\
88.47	0\\
88.48	0\\
88.49	0\\
88.5	0\\
88.51	0\\
88.52	0\\
88.53	0\\
88.54	0\\
88.55	0\\
88.56	0\\
88.57	0\\
88.58	0\\
88.59	0\\
88.6	0\\
88.61	0\\
88.62	0\\
88.63	0\\
88.64	0\\
88.65	0\\
88.66	0\\
88.67	0\\
88.68	0\\
88.69	0\\
88.7	0\\
88.71	0\\
88.72	0\\
88.73	0\\
88.74	0\\
88.75	0\\
88.76	0\\
88.77	0\\
88.78	0\\
88.79	0\\
88.8	0\\
88.81	0\\
88.82	0\\
88.83	0\\
88.84	0\\
88.85	0\\
88.86	0\\
88.87	0\\
88.88	0\\
88.89	0\\
88.9	0\\
88.91	0\\
88.92	0\\
88.93	0\\
88.94	0\\
88.95	0\\
88.96	0\\
88.97	0\\
88.98	0\\
88.99	0\\
89	0\\
89.01	0\\
89.02	0\\
89.03	0\\
89.04	0\\
89.05	0\\
89.06	0\\
89.07	0\\
89.08	0\\
89.09	0\\
89.1	0\\
89.11	0\\
89.12	0\\
89.13	0\\
89.14	0\\
89.15	0\\
89.16	0\\
89.17	0\\
89.18	0\\
89.19	0\\
89.2	0\\
89.21	0\\
89.22	0\\
89.23	0\\
89.24	0\\
89.25	0\\
89.26	0\\
89.27	0\\
89.28	0\\
89.29	0\\
89.3	0\\
89.31	0\\
89.32	0\\
89.33	0\\
89.34	0\\
89.35	0\\
89.36	0\\
89.37	0\\
89.38	0\\
89.39	0\\
89.4	0\\
89.41	0\\
89.42	0\\
89.43	0\\
89.44	0\\
89.45	0\\
89.46	0\\
89.47	0\\
89.48	0\\
89.49	0\\
89.5	0\\
89.51	0\\
89.52	0\\
89.53	0\\
89.54	0\\
89.55	0\\
89.56	0\\
89.57	0\\
89.58	0\\
89.59	0\\
89.6	0\\
89.61	0\\
89.62	0\\
89.63	0\\
89.64	0\\
89.65	0\\
89.66	0\\
89.67	0\\
89.68	0\\
89.69	0\\
89.7	0\\
89.71	0\\
89.72	0\\
89.73	0\\
89.74	0\\
89.75	0\\
89.76	0\\
89.77	0\\
89.78	0\\
89.79	0\\
89.8	0\\
89.81	0\\
89.82	0\\
89.83	0\\
89.84	0\\
89.85	0\\
89.86	0\\
89.87	0\\
89.88	0\\
89.89	0\\
89.9	0\\
89.91	0\\
89.92	0\\
89.93	0\\
89.94	0\\
89.95	0\\
89.96	0\\
89.97	0\\
89.98	0\\
89.99	0\\
90	0\\
90.01	0\\
90.02	0\\
90.03	0\\
90.04	0\\
90.05	0\\
90.06	0\\
90.07	0\\
90.08	0\\
90.09	0\\
90.1	0\\
90.11	0\\
90.12	0\\
90.13	0\\
90.14	0\\
90.15	0\\
90.16	0\\
90.17	0\\
90.18	0\\
90.19	0\\
90.2	0\\
90.21	0\\
90.22	0\\
90.23	0\\
90.24	0\\
90.25	0\\
90.26	0\\
90.27	0\\
90.28	0\\
90.29	0\\
90.3	0\\
90.31	0\\
90.32	0\\
90.33	0\\
90.34	0\\
90.35	0\\
90.36	0\\
90.37	0\\
90.38	0\\
90.39	0\\
90.4	0\\
90.41	0\\
90.42	0\\
90.43	0\\
90.44	0\\
90.45	0\\
90.46	0\\
90.47	0\\
90.48	0\\
90.49	0\\
90.5	0\\
90.51	0\\
90.52	0\\
90.53	0\\
90.54	0\\
90.55	0\\
90.56	0\\
90.57	0\\
90.58	0\\
90.59	0\\
90.6	0\\
90.61	0\\
90.62	0\\
90.63	0\\
90.64	0\\
90.65	0\\
90.66	0\\
90.67	0\\
90.68	0\\
90.69	0\\
90.7	0\\
90.71	0\\
90.72	0\\
90.73	0\\
90.74	0\\
90.75	0\\
90.76	0\\
90.77	0\\
90.78	0\\
90.79	0\\
90.8	0\\
90.81	0\\
90.82	0\\
90.83	0\\
90.84	0\\
90.85	0\\
90.86	0\\
90.87	0\\
90.88	0\\
90.89	0\\
90.9	0\\
90.91	0\\
90.92	0\\
90.93	0\\
90.94	0\\
90.95	0\\
90.96	0\\
90.97	0\\
90.98	0\\
90.99	0\\
91	0\\
91.01	0\\
91.02	0\\
91.03	0\\
91.04	0\\
91.05	0\\
91.06	0\\
91.07	0\\
91.08	0\\
91.09	0\\
91.1	0\\
91.11	0\\
91.12	0\\
91.13	0\\
91.14	0\\
91.15	0\\
91.16	0\\
91.17	0\\
91.18	0\\
91.19	0\\
91.2	0\\
91.21	0\\
91.22	0\\
91.23	0\\
91.24	0\\
91.25	0\\
91.26	0\\
91.27	0\\
91.28	0\\
91.29	0\\
91.3	0\\
91.31	0\\
91.32	0\\
91.33	0\\
91.34	0\\
91.35	0\\
91.36	0\\
91.37	0\\
91.38	0\\
91.39	0\\
91.4	0\\
91.41	0\\
91.42	0\\
91.43	0\\
91.44	0\\
91.45	0\\
91.46	0\\
91.47	0\\
91.48	0\\
91.49	0\\
91.5	0\\
91.51	0\\
91.52	0\\
91.53	0\\
91.54	0\\
91.55	0\\
91.56	0\\
91.57	0\\
91.58	0\\
91.59	0\\
91.6	0\\
91.61	0\\
91.62	0\\
91.63	0\\
91.64	0\\
91.65	0\\
91.66	0\\
91.67	0\\
91.68	0\\
91.69	0\\
91.7	0\\
91.71	0\\
91.72	0\\
91.73	0\\
91.74	0\\
91.75	0\\
91.76	0\\
91.77	0\\
91.78	0\\
91.79	0\\
91.8	0\\
91.81	0\\
91.82	0\\
91.83	0\\
91.84	0\\
91.85	0\\
91.86	0\\
91.87	0\\
91.88	0\\
91.89	0\\
91.9	0\\
91.91	0\\
91.92	0\\
91.93	0\\
91.94	0\\
91.95	0\\
91.96	0\\
91.97	0\\
91.98	0\\
91.99	0\\
92	0\\
92.01	0\\
92.02	0\\
92.03	0\\
92.04	0\\
92.05	0\\
92.06	0\\
92.07	0\\
92.08	0\\
92.09	0\\
92.1	0\\
92.11	0\\
92.12	0\\
92.13	0\\
92.14	0\\
92.15	0\\
92.16	0\\
92.17	0\\
92.18	0\\
92.19	0\\
92.2	0\\
92.21	0\\
92.22	0\\
92.23	0\\
92.24	0\\
92.25	0\\
92.26	0\\
92.27	0\\
92.28	0\\
92.29	0\\
92.3	0\\
92.31	0\\
92.32	0\\
92.33	0\\
92.34	0\\
92.35	0\\
92.36	0\\
92.37	0\\
92.38	0\\
92.39	0\\
92.4	0\\
92.41	0\\
92.42	0\\
92.43	0\\
92.44	0\\
92.45	0\\
92.46	0\\
92.47	0\\
92.48	0\\
92.49	0\\
92.5	0\\
92.51	0\\
92.52	0\\
92.53	0\\
92.54	0\\
92.55	0\\
92.56	0\\
92.57	0\\
92.58	0\\
92.59	0\\
92.6	0\\
92.61	0\\
92.62	0\\
92.63	0\\
92.64	0\\
92.65	0\\
92.66	0\\
92.67	0\\
92.68	0\\
92.69	0\\
92.7	0\\
92.71	0\\
92.72	0\\
92.73	0\\
92.74	0\\
92.75	0\\
92.76	0\\
92.77	0\\
92.78	0\\
92.79	0\\
92.8	0\\
92.81	0\\
92.82	0\\
92.83	0\\
92.84	0\\
92.85	0\\
92.86	0\\
92.87	0\\
92.88	0\\
92.89	0\\
92.9	0\\
92.91	0\\
92.92	0\\
92.93	0\\
92.94	0\\
92.95	0\\
92.96	0\\
92.97	0\\
92.98	0\\
92.99	0\\
93	0\\
93.01	0\\
93.02	0\\
93.03	0\\
93.04	0\\
93.05	0\\
93.06	0\\
93.07	0\\
93.08	0\\
93.09	0\\
93.1	0\\
93.11	0\\
93.12	0\\
93.13	0\\
93.14	0\\
93.15	0\\
93.16	0\\
93.17	0\\
93.18	0\\
93.19	0\\
93.2	0\\
93.21	0\\
93.22	0\\
93.23	0\\
93.24	0\\
93.25	0\\
93.26	0\\
93.27	0\\
93.28	0\\
93.29	0\\
93.3	0\\
93.31	0\\
93.32	0\\
93.33	0\\
93.34	0\\
93.35	0\\
93.36	0\\
93.37	0\\
93.38	0\\
93.39	0\\
93.4	0\\
93.41	0\\
93.42	0\\
93.43	0\\
93.44	0\\
93.45	0\\
93.46	0\\
93.47	0\\
93.48	0\\
93.49	0\\
93.5	0\\
93.51	0\\
93.52	0\\
93.53	0\\
93.54	0\\
93.55	0\\
93.56	0\\
93.57	0\\
93.58	0\\
93.59	0\\
93.6	0\\
93.61	0\\
93.62	0\\
93.63	0\\
93.64	0\\
93.65	0\\
93.66	0\\
93.67	0\\
93.68	0\\
93.69	0\\
93.7	0\\
93.71	0\\
93.72	0\\
93.73	0\\
93.74	0\\
93.75	0\\
93.76	0\\
93.77	0\\
93.78	0\\
93.79	0\\
93.8	0\\
93.81	0\\
93.82	0\\
93.83	0\\
93.84	0\\
93.85	0\\
93.86	0\\
93.87	0\\
93.88	0\\
93.89	0\\
93.9	0\\
93.91	0\\
93.92	0\\
93.93	0\\
93.94	0\\
93.95	0\\
93.96	0\\
93.97	0\\
93.98	0\\
93.99	0\\
94	0\\
94.01	0\\
94.02	0\\
94.03	0\\
94.04	0\\
94.05	0\\
94.06	0\\
94.07	0\\
94.08	0\\
94.09	0\\
94.1	0\\
94.11	0\\
94.12	0\\
94.13	0\\
94.14	0\\
94.15	0\\
94.16	0\\
94.17	0\\
94.18	0\\
94.19	0\\
94.2	0\\
94.21	0\\
94.22	0\\
94.23	0\\
94.24	0\\
94.25	0\\
94.26	0\\
94.27	0\\
94.28	0\\
94.29	0\\
94.3	0\\
94.31	0\\
94.32	0\\
94.33	0\\
94.34	0\\
94.35	0\\
94.36	0\\
94.37	0\\
94.38	0\\
94.39	0\\
94.4	0\\
94.41	0\\
94.42	0\\
94.43	0\\
94.44	0\\
94.45	0\\
94.46	0\\
94.47	0\\
94.48	0\\
94.49	0\\
94.5	0\\
94.51	0\\
94.52	0\\
94.53	0\\
94.54	0\\
94.55	0\\
94.56	0\\
94.57	0\\
94.58	0\\
94.59	0\\
94.6	0\\
94.61	0\\
94.62	0\\
94.63	0\\
94.64	0\\
94.65	0\\
94.66	0\\
94.67	0\\
94.68	0\\
94.69	0\\
94.7	0\\
94.71	0\\
94.72	0\\
94.73	0\\
94.74	0\\
94.75	0\\
94.76	0\\
94.77	0\\
94.78	0\\
94.79	0\\
94.8	0\\
94.81	0\\
94.82	0\\
94.83	0\\
94.84	0\\
94.85	0\\
94.86	0\\
94.87	0\\
94.88	0\\
94.89	0\\
94.9	0\\
94.91	0\\
94.92	0\\
94.93	0\\
94.94	0\\
94.95	0\\
94.96	0\\
94.97	0\\
94.98	0\\
94.99	0\\
95	0\\
95.01	0\\
95.02	0\\
95.03	0\\
95.04	0\\
95.05	0\\
95.06	0\\
95.07	0\\
95.08	0\\
95.09	0\\
95.1	0\\
95.11	0\\
95.12	0\\
95.13	0\\
95.14	0\\
95.15	0\\
95.16	0\\
95.17	0\\
95.18	0\\
95.19	0\\
95.2	0\\
95.21	0\\
95.22	0\\
95.23	0\\
95.24	0\\
95.25	0\\
95.26	0\\
95.27	0\\
95.28	0\\
95.29	0\\
95.3	0\\
95.31	0\\
95.32	0\\
95.33	0\\
95.34	0\\
95.35	0\\
95.36	0\\
95.37	0\\
95.38	0\\
95.39	0\\
95.4	0\\
95.41	0\\
95.42	0\\
95.43	0\\
95.44	0\\
95.45	0\\
95.46	0\\
95.47	0\\
95.48	0\\
95.49	0\\
95.5	0\\
95.51	0\\
95.52	0\\
95.53	0\\
95.54	0\\
95.55	0\\
95.56	0\\
95.57	0\\
95.58	0\\
95.59	0\\
95.6	0\\
95.61	0\\
95.62	0\\
95.63	0\\
95.64	0\\
95.65	0\\
95.66	0\\
95.67	0\\
95.68	0\\
95.69	0\\
95.7	0\\
95.71	0\\
95.72	0\\
95.73	0\\
95.74	0\\
95.75	0\\
95.76	0\\
95.77	0\\
95.78	0\\
95.79	0\\
95.8	0\\
95.81	0\\
95.82	0\\
95.83	0\\
95.84	0\\
95.85	0\\
95.86	0\\
95.87	0\\
95.88	0\\
95.89	0\\
95.9	0\\
95.91	0\\
95.92	0\\
95.93	0\\
95.94	0\\
95.95	0\\
95.96	0\\
95.97	0\\
95.98	0\\
95.99	0\\
96	0\\
96.01	0\\
96.02	0\\
96.03	0\\
96.04	0\\
96.05	0\\
96.06	0\\
96.07	0\\
96.08	0\\
96.09	0\\
96.1	0\\
96.11	0\\
96.12	0\\
96.13	0\\
96.14	0\\
96.15	0\\
96.16	0\\
96.17	0\\
96.18	0\\
96.19	0\\
96.2	0\\
96.21	0\\
96.22	0\\
96.23	0\\
96.24	0\\
96.25	0\\
96.26	0\\
96.27	0\\
96.28	0\\
96.29	0\\
96.3	0\\
96.31	0\\
96.32	0\\
96.33	0\\
96.34	0\\
96.35	0\\
96.36	0\\
96.37	0\\
96.38	0\\
96.39	0\\
96.4	0\\
96.41	0\\
96.42	0\\
96.43	0\\
96.44	0\\
96.45	0\\
96.46	0\\
96.47	0\\
96.48	0\\
96.49	0\\
96.5	0\\
96.51	0\\
96.52	0\\
96.53	0\\
96.54	0\\
96.55	0\\
96.56	0\\
96.57	0\\
96.58	0\\
96.59	0\\
96.6	0\\
96.61	0\\
96.62	0\\
96.63	0\\
96.64	0\\
96.65	0\\
96.66	0\\
96.67	0\\
96.68	0\\
96.69	0\\
96.7	0\\
96.71	0\\
96.72	0\\
96.73	0\\
96.74	0\\
96.75	0\\
96.76	0\\
96.77	0\\
96.78	0\\
96.79	0\\
96.8	0\\
96.81	0\\
96.82	0\\
96.83	0\\
96.84	0\\
96.85	0\\
96.86	0\\
96.87	0\\
96.88	0\\
96.89	0\\
96.9	0\\
96.91	0\\
96.92	0\\
96.93	0\\
96.94	0\\
96.95	0\\
96.96	0\\
96.97	0\\
96.98	0\\
96.99	0\\
97	0\\
97.01	0\\
97.02	0\\
97.03	0\\
97.04	0\\
97.05	0\\
97.06	0\\
97.07	0\\
97.08	0\\
97.09	0\\
97.1	0\\
97.11	0\\
97.12	0\\
97.13	0\\
97.14	0\\
97.15	0\\
97.16	0\\
97.17	0\\
97.18	0\\
97.19	0\\
97.2	0\\
97.21	0\\
97.22	0\\
97.23	0\\
97.24	0\\
97.25	0\\
97.26	0\\
97.27	0\\
97.28	0\\
97.29	0\\
97.3	0\\
97.31	0\\
97.32	0\\
97.33	0\\
97.34	0\\
97.35	0\\
97.36	0\\
97.37	0\\
97.38	0\\
97.39	0\\
97.4	0\\
97.41	0\\
97.42	0\\
97.43	0\\
97.44	0\\
97.45	0\\
97.46	0\\
97.47	0\\
97.48	0\\
97.49	0\\
97.5	0\\
97.51	0\\
97.52	0\\
97.53	0\\
97.54	0\\
97.55	0\\
97.56	0\\
97.57	0\\
97.58	0\\
97.59	0\\
97.6	0\\
97.61	0\\
97.62	0\\
97.63	0\\
97.64	0\\
97.65	0\\
97.66	0\\
97.67	0\\
97.68	0\\
97.69	0\\
97.7	0\\
97.71	0\\
97.72	0\\
97.73	0\\
97.74	0\\
97.75	0\\
97.76	0\\
97.77	0\\
97.78	0\\
97.79	0\\
97.8	0\\
97.81	0\\
97.82	0\\
97.83	0\\
97.84	0\\
97.85	0\\
97.86	0\\
97.87	0\\
97.88	0\\
97.89	0\\
97.9	0\\
97.91	0\\
97.92	0\\
97.93	0\\
97.94	0\\
97.95	0\\
97.96	0\\
97.97	0\\
97.98	0\\
97.99	0\\
98	0\\
98.01	0\\
98.02	0\\
98.03	0\\
98.04	0\\
98.05	0\\
98.06	0\\
98.07	0\\
98.08	0\\
98.09	0\\
98.1	0\\
98.11	0\\
98.12	0\\
98.13	0\\
98.14	0\\
98.15	0\\
98.16	0\\
98.17	0\\
98.18	0\\
98.19	5.87091993147215e-05\\
98.2	0.000131660668289388\\
98.21	0.000205183667401541\\
98.22	0.000279272272188956\\
98.23	0.000353932011134473\\
98.24	0.000429168473808594\\
98.25	0.000504987311700112\\
98.26	0.000581394239061801\\
98.27	0.000658395033771495\\
98.28	0.000735995538209001\\
98.29	0.000814201660149175\\
98.3	0.000893019375794994\\
98.31	0.000972454729098664\\
98.32	0.00105251383229726\\
98.33	0.00113320286707123\\
98.34	0.00121452808741002\\
98.35	0.00129649581862847\\
98.36	0.00137911245839192\\
98.37	0.00140436050083681\\
98.38	0.0014277326476487\\
98.39	0.00145129188951867\\
98.4	0.00147503968331663\\
98.41	0.00149897749327697\\
98.42	0.0015231067909245\\
98.43	0.00154743830384437\\
98.44	0.0015719765775428\\
98.45	0.00159672333583745\\
98.46	0.00162168031340307\\
98.47	0.00164684925574105\\
98.48	0.00167223191914474\\
98.49	0.00169783007066085\\
98.5	0.00172364548804649\\
98.51	0.00174967995972184\\
98.52	0.00177593528471809\\
98.53	0.00180241327262077\\
98.54	0.00182911574350748\\
98.55	0.00185604452787694\\
98.56	0.00188320146657662\\
98.57	0.00191058841072475\\
98.58	0.0019382072216267\\
98.59	0.0019660597706852\\
98.6	0.00199414793930437\\
98.61	0.00202247361878732\\
98.62	0.00205103871022702\\
98.63	0.00207984512439009\\
98.64	0.00210889478159352\\
98.65	0.00213818961157364\\
98.66	0.00216773155443861\\
98.67	0.00219752255983394\\
98.68	0.00222756458641881\\
98.69	0.00225787628061744\\
98.7	0.0022884654676135\\
98.71	0.00231933481963858\\
98.72	0.0023504870349779\\
98.73	0.00238192483822133\\
98.74	0.00241365097665809\\
98.75	0.00244566822147595\\
98.76	0.00247797937070885\\
98.77	0.00251058724950424\\
98.78	0.00254349471039308\\
98.79	0.00257670462316625\\
98.8	0.00261021988212598\\
98.81	0.002644043409222\\
98.82	0.0026781781543157\\
98.83	0.00271262709475195\\
98.84	0.00274739322467683\\
98.85	0.00278247952927148\\
98.86	0.00281788902175932\\
98.87	0.00285362474366859\\
98.88	0.00288968976509735\\
98.89	0.00292608718498091\\
98.9	0.00296282013136184\\
98.91	0.00299989176166234\\
98.92	0.00303730526295931\\
98.93	0.00307506385226188\\
98.94	0.00311317077679147\\
98.95	0.00315162931426461\\
98.96	0.00319044277317824\\
98.97	0.00322961449309772\\
98.98	0.0032691478449475\\
98.99	0.00330904623130455\\
99	0.00334931308669438\\
99.01	0.00338995187788997\\
99.02	0.00343096610421334\\
99.03	0.00347235929784001\\
99.04	0.00351413502410622\\
99.05	0.00355629688181904\\
99.06	0.00359884850356936\\
99.07	0.00364179355604775\\
99.08	0.00368513574038585\\
99.09	0.00372887879249803\\
99.1	0.00377302648340829\\
99.11	0.00381758261958013\\
99.12	0.00386255104325286\\
99.13	0.00390793563277871\\
99.14	0.00395374030296244\\
99.15	0.00399996900540418\\
99.16	0.00404662572884551\\
99.17	0.0040937144995188\\
99.18	0.00414123938149992\\
99.19	0.00418920447706429\\
99.2	0.0042376139270463\\
99.21	0.00428647191120219\\
99.22	0.00433578264857637\\
99.23	0.00438555039787123\\
99.24	0.00443577945782051\\
99.25	0.00448647416756618\\
99.26	0.00453763890703903\\
99.27	0.00458927809734277\\
99.28	0.00464139620114196\\
99.29	0.0046939977230535\\
99.3	0.00474708721004204\\
99.31	0.00480066925181908\\
99.32	0.00485474848124597\\
99.33	0.00490932957474073\\
99.34	0.00496441725268891\\
99.35	0.00502001627985832\\
99.36	0.00507613146581777\\
99.37	0.00513276766535998\\
99.38	0.00518992977892843\\
99.39	0.00524762275304852\\
99.4	0.00530585158076283\\
99.41	0.00536462130207064\\
99.42	0.00542393700437178\\
99.43	0.00548380382291478\\
99.44	0.00554422694124944\\
99.45	0.00560521159168388\\
99.46	0.00566676305574604\\
99.47	0.00572888666464975\\
99.48	0.00579158779976541\\
99.49	0.00585487189309542\\
99.5	0.0059187444277542\\
99.51	0.00598321093845309\\
99.52	0.00604827701199012\\
99.53	0.00611394828774462\\
99.54	0.00618023045817687\\
99.55	0.00624712926933275\\
99.56	0.00631465052135359\\
99.57	0.00638280006899107\\
99.58	0.0064515838221275\\
99.59	0.00652100774630139\\
99.6	0.00659107786323844\\
99.61	0.00666180022938315\\
99.62	0.00673318094678823\\
99.63	0.00680522617466349\\
99.64	0.00687794212991873\\
99.65	0.00695133508771244\\
99.66	0.00702541138200604\\
99.67	0.00710017740612412\\
99.68	0.00717563961332046\\
99.69	0.00725180451735017\\
99.7	0.00732867869304791\\
99.71	0.00740626877694791\\
99.72	0.00748458146787993\\
99.73	0.00756362352756889\\
99.74	0.0076434017812414\\
99.75	0.00772392311823898\\
99.76	0.0078051944926383\\
99.77	0.00788722292387853\\
99.78	0.00797001549739592\\
99.79	0.00805357936526582\\
99.8	0.00813792174685226\\
99.81	0.00822304992946529\\
99.82	0.00830897126902625\\
99.83	0.00839569319074109\\
99.84	0.00848322318978214\\
99.85	0.00857156883197824\\
99.86	0.00866073775451372\\
99.87	0.0087507376666363\\
99.88	0.00884157635037421\\
99.89	0.00893326166126276\\
99.9	0.00902580152908066\\
99.91	0.00911920395859638\\
99.92	0.00921347703032472\\
99.93	0.00930862890129417\\
99.94	0.00940466780582511\\
99.95	0.0095016020563194\\
99.96	0.00959944004406162\\
99.97	0.00969819024003246\\
99.98	0.0097978611957346\\
99.99	0.00989846154403157\\
100	0.01\\
};
\addlegendentry{$q=2$};

\addplot [color=mycolor1,solid,forget plot]
  table[row sep=crcr]{%
0.01	0\\
0.02	0\\
0.03	0\\
0.04	0\\
0.05	0\\
0.06	0\\
0.07	0\\
0.08	0\\
0.09	0\\
0.1	0\\
0.11	0\\
0.12	0\\
0.13	0\\
0.14	0\\
0.15	0\\
0.16	0\\
0.17	0\\
0.18	0\\
0.19	0\\
0.2	0\\
0.21	0\\
0.22	0\\
0.23	0\\
0.24	0\\
0.25	0\\
0.26	0\\
0.27	0\\
0.28	0\\
0.29	0\\
0.3	0\\
0.31	0\\
0.32	0\\
0.33	0\\
0.34	0\\
0.35	0\\
0.36	0\\
0.37	0\\
0.38	0\\
0.39	0\\
0.4	0\\
0.41	0\\
0.42	0\\
0.43	0\\
0.44	0\\
0.45	0\\
0.46	0\\
0.47	0\\
0.48	0\\
0.49	0\\
0.5	0\\
0.51	0\\
0.52	0\\
0.53	0\\
0.54	0\\
0.55	0\\
0.56	0\\
0.57	0\\
0.58	0\\
0.59	0\\
0.6	0\\
0.61	0\\
0.62	0\\
0.63	0\\
0.64	0\\
0.65	0\\
0.66	0\\
0.67	0\\
0.68	0\\
0.69	0\\
0.7	0\\
0.71	0\\
0.72	0\\
0.73	0\\
0.74	0\\
0.75	0\\
0.76	0\\
0.77	0\\
0.78	0\\
0.79	0\\
0.8	0\\
0.81	0\\
0.82	0\\
0.83	0\\
0.84	0\\
0.85	0\\
0.86	0\\
0.87	0\\
0.88	0\\
0.89	0\\
0.9	0\\
0.91	0\\
0.92	0\\
0.93	0\\
0.94	0\\
0.95	0\\
0.96	0\\
0.97	0\\
0.98	0\\
0.99	0\\
1	0\\
1.01	0\\
1.02	0\\
1.03	0\\
1.04	0\\
1.05	0\\
1.06	0\\
1.07	0\\
1.08	0\\
1.09	0\\
1.1	0\\
1.11	0\\
1.12	0\\
1.13	0\\
1.14	0\\
1.15	0\\
1.16	0\\
1.17	0\\
1.18	0\\
1.19	0\\
1.2	0\\
1.21	0\\
1.22	0\\
1.23	0\\
1.24	0\\
1.25	0\\
1.26	0\\
1.27	0\\
1.28	0\\
1.29	0\\
1.3	0\\
1.31	0\\
1.32	0\\
1.33	0\\
1.34	0\\
1.35	0\\
1.36	0\\
1.37	0\\
1.38	0\\
1.39	0\\
1.4	0\\
1.41	0\\
1.42	0\\
1.43	0\\
1.44	0\\
1.45	0\\
1.46	0\\
1.47	0\\
1.48	0\\
1.49	0\\
1.5	0\\
1.51	0\\
1.52	0\\
1.53	0\\
1.54	0\\
1.55	0\\
1.56	0\\
1.57	0\\
1.58	0\\
1.59	0\\
1.6	0\\
1.61	0\\
1.62	0\\
1.63	0\\
1.64	0\\
1.65	0\\
1.66	0\\
1.67	0\\
1.68	0\\
1.69	0\\
1.7	0\\
1.71	0\\
1.72	0\\
1.73	0\\
1.74	0\\
1.75	0\\
1.76	0\\
1.77	0\\
1.78	0\\
1.79	0\\
1.8	0\\
1.81	0\\
1.82	0\\
1.83	0\\
1.84	0\\
1.85	0\\
1.86	0\\
1.87	0\\
1.88	0\\
1.89	0\\
1.9	0\\
1.91	0\\
1.92	0\\
1.93	0\\
1.94	0\\
1.95	0\\
1.96	0\\
1.97	0\\
1.98	0\\
1.99	0\\
2	0\\
2.01	0\\
2.02	0\\
2.03	0\\
2.04	0\\
2.05	0\\
2.06	0\\
2.07	0\\
2.08	0\\
2.09	0\\
2.1	0\\
2.11	0\\
2.12	0\\
2.13	0\\
2.14	0\\
2.15	0\\
2.16	0\\
2.17	0\\
2.18	0\\
2.19	0\\
2.2	0\\
2.21	0\\
2.22	0\\
2.23	0\\
2.24	0\\
2.25	0\\
2.26	0\\
2.27	0\\
2.28	0\\
2.29	0\\
2.3	0\\
2.31	0\\
2.32	0\\
2.33	0\\
2.34	0\\
2.35	0\\
2.36	0\\
2.37	0\\
2.38	0\\
2.39	0\\
2.4	0\\
2.41	0\\
2.42	0\\
2.43	0\\
2.44	0\\
2.45	0\\
2.46	0\\
2.47	0\\
2.48	0\\
2.49	0\\
2.5	0\\
2.51	0\\
2.52	0\\
2.53	0\\
2.54	0\\
2.55	0\\
2.56	0\\
2.57	0\\
2.58	0\\
2.59	0\\
2.6	0\\
2.61	0\\
2.62	0\\
2.63	0\\
2.64	0\\
2.65	0\\
2.66	0\\
2.67	0\\
2.68	0\\
2.69	0\\
2.7	0\\
2.71	0\\
2.72	0\\
2.73	0\\
2.74	0\\
2.75	0\\
2.76	0\\
2.77	0\\
2.78	0\\
2.79	0\\
2.8	0\\
2.81	0\\
2.82	0\\
2.83	0\\
2.84	0\\
2.85	0\\
2.86	0\\
2.87	0\\
2.88	0\\
2.89	0\\
2.9	0\\
2.91	0\\
2.92	0\\
2.93	0\\
2.94	0\\
2.95	0\\
2.96	0\\
2.97	0\\
2.98	0\\
2.99	0\\
3	0\\
3.01	0\\
3.02	0\\
3.03	0\\
3.04	0\\
3.05	0\\
3.06	0\\
3.07	0\\
3.08	0\\
3.09	0\\
3.1	0\\
3.11	0\\
3.12	0\\
3.13	0\\
3.14	0\\
3.15	0\\
3.16	0\\
3.17	0\\
3.18	0\\
3.19	0\\
3.2	0\\
3.21	0\\
3.22	0\\
3.23	0\\
3.24	0\\
3.25	0\\
3.26	0\\
3.27	0\\
3.28	0\\
3.29	0\\
3.3	0\\
3.31	0\\
3.32	0\\
3.33	0\\
3.34	0\\
3.35	0\\
3.36	0\\
3.37	0\\
3.38	0\\
3.39	0\\
3.4	0\\
3.41	0\\
3.42	0\\
3.43	0\\
3.44	0\\
3.45	0\\
3.46	0\\
3.47	0\\
3.48	0\\
3.49	0\\
3.5	0\\
3.51	0\\
3.52	0\\
3.53	0\\
3.54	0\\
3.55	0\\
3.56	0\\
3.57	0\\
3.58	0\\
3.59	0\\
3.6	0\\
3.61	0\\
3.62	0\\
3.63	0\\
3.64	0\\
3.65	0\\
3.66	0\\
3.67	0\\
3.68	0\\
3.69	0\\
3.7	0\\
3.71	0\\
3.72	0\\
3.73	0\\
3.74	0\\
3.75	0\\
3.76	0\\
3.77	0\\
3.78	0\\
3.79	0\\
3.8	0\\
3.81	0\\
3.82	0\\
3.83	0\\
3.84	0\\
3.85	0\\
3.86	0\\
3.87	0\\
3.88	0\\
3.89	0\\
3.9	0\\
3.91	0\\
3.92	0\\
3.93	0\\
3.94	0\\
3.95	0\\
3.96	0\\
3.97	0\\
3.98	0\\
3.99	0\\
4	0\\
4.01	0\\
4.02	0\\
4.03	0\\
4.04	0\\
4.05	0\\
4.06	0\\
4.07	0\\
4.08	0\\
4.09	0\\
4.1	0\\
4.11	0\\
4.12	0\\
4.13	0\\
4.14	0\\
4.15	0\\
4.16	0\\
4.17	0\\
4.18	0\\
4.19	0\\
4.2	0\\
4.21	0\\
4.22	0\\
4.23	0\\
4.24	0\\
4.25	0\\
4.26	0\\
4.27	0\\
4.28	0\\
4.29	0\\
4.3	0\\
4.31	0\\
4.32	0\\
4.33	0\\
4.34	0\\
4.35	0\\
4.36	0\\
4.37	0\\
4.38	0\\
4.39	0\\
4.4	0\\
4.41	0\\
4.42	0\\
4.43	0\\
4.44	0\\
4.45	0\\
4.46	0\\
4.47	0\\
4.48	0\\
4.49	0\\
4.5	0\\
4.51	0\\
4.52	0\\
4.53	0\\
4.54	0\\
4.55	0\\
4.56	0\\
4.57	0\\
4.58	0\\
4.59	0\\
4.6	0\\
4.61	0\\
4.62	0\\
4.63	0\\
4.64	0\\
4.65	0\\
4.66	0\\
4.67	0\\
4.68	0\\
4.69	0\\
4.7	0\\
4.71	0\\
4.72	0\\
4.73	0\\
4.74	0\\
4.75	0\\
4.76	0\\
4.77	0\\
4.78	0\\
4.79	0\\
4.8	0\\
4.81	0\\
4.82	0\\
4.83	0\\
4.84	0\\
4.85	0\\
4.86	0\\
4.87	0\\
4.88	0\\
4.89	0\\
4.9	0\\
4.91	0\\
4.92	0\\
4.93	0\\
4.94	0\\
4.95	0\\
4.96	0\\
4.97	0\\
4.98	0\\
4.99	0\\
5	0\\
5.01	0\\
5.02	0\\
5.03	0\\
5.04	0\\
5.05	0\\
5.06	0\\
5.07	0\\
5.08	0\\
5.09	0\\
5.1	0\\
5.11	0\\
5.12	0\\
5.13	0\\
5.14	0\\
5.15	0\\
5.16	0\\
5.17	0\\
5.18	0\\
5.19	0\\
5.2	0\\
5.21	0\\
5.22	0\\
5.23	0\\
5.24	0\\
5.25	0\\
5.26	0\\
5.27	0\\
5.28	0\\
5.29	0\\
5.3	0\\
5.31	0\\
5.32	0\\
5.33	0\\
5.34	0\\
5.35	0\\
5.36	0\\
5.37	0\\
5.38	0\\
5.39	0\\
5.4	0\\
5.41	0\\
5.42	0\\
5.43	0\\
5.44	0\\
5.45	0\\
5.46	0\\
5.47	0\\
5.48	0\\
5.49	0\\
5.5	0\\
5.51	0\\
5.52	0\\
5.53	0\\
5.54	0\\
5.55	0\\
5.56	0\\
5.57	0\\
5.58	0\\
5.59	0\\
5.6	0\\
5.61	0\\
5.62	0\\
5.63	0\\
5.64	0\\
5.65	0\\
5.66	0\\
5.67	0\\
5.68	0\\
5.69	0\\
5.7	0\\
5.71	0\\
5.72	0\\
5.73	0\\
5.74	0\\
5.75	0\\
5.76	0\\
5.77	0\\
5.78	0\\
5.79	0\\
5.8	0\\
5.81	0\\
5.82	0\\
5.83	0\\
5.84	0\\
5.85	0\\
5.86	0\\
5.87	0\\
5.88	0\\
5.89	0\\
5.9	0\\
5.91	0\\
5.92	0\\
5.93	0\\
5.94	0\\
5.95	0\\
5.96	0\\
5.97	0\\
5.98	0\\
5.99	0\\
6	0\\
6.01	0\\
6.02	0\\
6.03	0\\
6.04	0\\
6.05	0\\
6.06	0\\
6.07	0\\
6.08	0\\
6.09	0\\
6.1	0\\
6.11	0\\
6.12	0\\
6.13	0\\
6.14	0\\
6.15	0\\
6.16	0\\
6.17	0\\
6.18	0\\
6.19	0\\
6.2	0\\
6.21	0\\
6.22	0\\
6.23	0\\
6.24	0\\
6.25	0\\
6.26	0\\
6.27	0\\
6.28	0\\
6.29	0\\
6.3	0\\
6.31	0\\
6.32	0\\
6.33	0\\
6.34	0\\
6.35	0\\
6.36	0\\
6.37	0\\
6.38	0\\
6.39	0\\
6.4	0\\
6.41	0\\
6.42	0\\
6.43	0\\
6.44	0\\
6.45	0\\
6.46	0\\
6.47	0\\
6.48	0\\
6.49	0\\
6.5	0\\
6.51	0\\
6.52	0\\
6.53	0\\
6.54	0\\
6.55	0\\
6.56	0\\
6.57	0\\
6.58	0\\
6.59	0\\
6.6	0\\
6.61	0\\
6.62	0\\
6.63	0\\
6.64	0\\
6.65	0\\
6.66	0\\
6.67	0\\
6.68	0\\
6.69	0\\
6.7	0\\
6.71	0\\
6.72	0\\
6.73	0\\
6.74	0\\
6.75	0\\
6.76	0\\
6.77	0\\
6.78	0\\
6.79	0\\
6.8	0\\
6.81	0\\
6.82	0\\
6.83	0\\
6.84	0\\
6.85	0\\
6.86	0\\
6.87	0\\
6.88	0\\
6.89	0\\
6.9	0\\
6.91	0\\
6.92	0\\
6.93	0\\
6.94	0\\
6.95	0\\
6.96	0\\
6.97	0\\
6.98	0\\
6.99	0\\
7	0\\
7.01	0\\
7.02	0\\
7.03	0\\
7.04	0\\
7.05	0\\
7.06	0\\
7.07	0\\
7.08	0\\
7.09	0\\
7.1	0\\
7.11	0\\
7.12	0\\
7.13	0\\
7.14	0\\
7.15	0\\
7.16	0\\
7.17	0\\
7.18	0\\
7.19	0\\
7.2	0\\
7.21	0\\
7.22	0\\
7.23	0\\
7.24	0\\
7.25	0\\
7.26	0\\
7.27	0\\
7.28	0\\
7.29	0\\
7.3	0\\
7.31	0\\
7.32	0\\
7.33	0\\
7.34	0\\
7.35	0\\
7.36	0\\
7.37	0\\
7.38	0\\
7.39	0\\
7.4	0\\
7.41	0\\
7.42	0\\
7.43	0\\
7.44	0\\
7.45	0\\
7.46	0\\
7.47	0\\
7.48	0\\
7.49	0\\
7.5	0\\
7.51	0\\
7.52	0\\
7.53	0\\
7.54	0\\
7.55	0\\
7.56	0\\
7.57	0\\
7.58	0\\
7.59	0\\
7.6	0\\
7.61	0\\
7.62	0\\
7.63	0\\
7.64	0\\
7.65	0\\
7.66	0\\
7.67	0\\
7.68	0\\
7.69	0\\
7.7	0\\
7.71	0\\
7.72	0\\
7.73	0\\
7.74	0\\
7.75	0\\
7.76	0\\
7.77	0\\
7.78	0\\
7.79	0\\
7.8	0\\
7.81	0\\
7.82	0\\
7.83	0\\
7.84	0\\
7.85	0\\
7.86	0\\
7.87	0\\
7.88	0\\
7.89	0\\
7.9	0\\
7.91	0\\
7.92	0\\
7.93	0\\
7.94	0\\
7.95	0\\
7.96	0\\
7.97	0\\
7.98	0\\
7.99	0\\
8	0\\
8.01	0\\
8.02	0\\
8.03	0\\
8.04	0\\
8.05	0\\
8.06	0\\
8.07	0\\
8.08	0\\
8.09	0\\
8.1	0\\
8.11	0\\
8.12	0\\
8.13	0\\
8.14	0\\
8.15	0\\
8.16	0\\
8.17	0\\
8.18	0\\
8.19	0\\
8.2	0\\
8.21	0\\
8.22	0\\
8.23	0\\
8.24	0\\
8.25	0\\
8.26	0\\
8.27	0\\
8.28	0\\
8.29	0\\
8.3	0\\
8.31	0\\
8.32	0\\
8.33	0\\
8.34	0\\
8.35	0\\
8.36	0\\
8.37	0\\
8.38	0\\
8.39	0\\
8.4	0\\
8.41	0\\
8.42	0\\
8.43	0\\
8.44	0\\
8.45	0\\
8.46	0\\
8.47	0\\
8.48	0\\
8.49	0\\
8.5	0\\
8.51	0\\
8.52	0\\
8.53	0\\
8.54	0\\
8.55	0\\
8.56	0\\
8.57	0\\
8.58	0\\
8.59	0\\
8.6	0\\
8.61	0\\
8.62	0\\
8.63	0\\
8.64	0\\
8.65	0\\
8.66	0\\
8.67	0\\
8.68	0\\
8.69	0\\
8.7	0\\
8.71	0\\
8.72	0\\
8.73	0\\
8.74	0\\
8.75	0\\
8.76	0\\
8.77	0\\
8.78	0\\
8.79	0\\
8.8	0\\
8.81	0\\
8.82	0\\
8.83	0\\
8.84	0\\
8.85	0\\
8.86	0\\
8.87	0\\
8.88	0\\
8.89	0\\
8.9	0\\
8.91	0\\
8.92	0\\
8.93	0\\
8.94	0\\
8.95	0\\
8.96	0\\
8.97	0\\
8.98	0\\
8.99	0\\
9	0\\
9.01	0\\
9.02	0\\
9.03	0\\
9.04	0\\
9.05	0\\
9.06	0\\
9.07	0\\
9.08	0\\
9.09	0\\
9.1	0\\
9.11	0\\
9.12	0\\
9.13	0\\
9.14	0\\
9.15	0\\
9.16	0\\
9.17	0\\
9.18	0\\
9.19	0\\
9.2	0\\
9.21	0\\
9.22	0\\
9.23	0\\
9.24	0\\
9.25	0\\
9.26	0\\
9.27	0\\
9.28	0\\
9.29	0\\
9.3	0\\
9.31	0\\
9.32	0\\
9.33	0\\
9.34	0\\
9.35	0\\
9.36	0\\
9.37	0\\
9.38	0\\
9.39	0\\
9.4	0\\
9.41	0\\
9.42	0\\
9.43	0\\
9.44	0\\
9.45	0\\
9.46	0\\
9.47	0\\
9.48	0\\
9.49	0\\
9.5	0\\
9.51	0\\
9.52	0\\
9.53	0\\
9.54	0\\
9.55	0\\
9.56	0\\
9.57	0\\
9.58	0\\
9.59	0\\
9.6	0\\
9.61	0\\
9.62	0\\
9.63	0\\
9.64	0\\
9.65	0\\
9.66	0\\
9.67	0\\
9.68	0\\
9.69	0\\
9.7	0\\
9.71	0\\
9.72	0\\
9.73	0\\
9.74	0\\
9.75	0\\
9.76	0\\
9.77	0\\
9.78	0\\
9.79	0\\
9.8	0\\
9.81	0\\
9.82	0\\
9.83	0\\
9.84	0\\
9.85	0\\
9.86	0\\
9.87	0\\
9.88	0\\
9.89	0\\
9.9	0\\
9.91	0\\
9.92	0\\
9.93	0\\
9.94	0\\
9.95	0\\
9.96	0\\
9.97	0\\
9.98	0\\
9.99	0\\
10	0\\
10.01	0\\
10.02	0\\
10.03	0\\
10.04	0\\
10.05	0\\
10.06	0\\
10.07	0\\
10.08	0\\
10.09	0\\
10.1	0\\
10.11	0\\
10.12	0\\
10.13	0\\
10.14	0\\
10.15	0\\
10.16	0\\
10.17	0\\
10.18	0\\
10.19	0\\
10.2	0\\
10.21	0\\
10.22	0\\
10.23	0\\
10.24	0\\
10.25	0\\
10.26	0\\
10.27	0\\
10.28	0\\
10.29	0\\
10.3	0\\
10.31	0\\
10.32	0\\
10.33	0\\
10.34	0\\
10.35	0\\
10.36	0\\
10.37	0\\
10.38	0\\
10.39	0\\
10.4	0\\
10.41	0\\
10.42	0\\
10.43	0\\
10.44	0\\
10.45	0\\
10.46	0\\
10.47	0\\
10.48	0\\
10.49	0\\
10.5	0\\
10.51	0\\
10.52	0\\
10.53	0\\
10.54	0\\
10.55	0\\
10.56	0\\
10.57	0\\
10.58	0\\
10.59	0\\
10.6	0\\
10.61	0\\
10.62	0\\
10.63	0\\
10.64	0\\
10.65	0\\
10.66	0\\
10.67	0\\
10.68	0\\
10.69	0\\
10.7	0\\
10.71	0\\
10.72	0\\
10.73	0\\
10.74	0\\
10.75	0\\
10.76	0\\
10.77	0\\
10.78	0\\
10.79	0\\
10.8	0\\
10.81	0\\
10.82	0\\
10.83	0\\
10.84	0\\
10.85	0\\
10.86	0\\
10.87	0\\
10.88	0\\
10.89	0\\
10.9	0\\
10.91	0\\
10.92	0\\
10.93	0\\
10.94	0\\
10.95	0\\
10.96	0\\
10.97	0\\
10.98	0\\
10.99	0\\
11	0\\
11.01	0\\
11.02	0\\
11.03	0\\
11.04	0\\
11.05	0\\
11.06	0\\
11.07	0\\
11.08	0\\
11.09	0\\
11.1	0\\
11.11	0\\
11.12	0\\
11.13	0\\
11.14	0\\
11.15	0\\
11.16	0\\
11.17	0\\
11.18	0\\
11.19	0\\
11.2	0\\
11.21	0\\
11.22	0\\
11.23	0\\
11.24	0\\
11.25	0\\
11.26	0\\
11.27	0\\
11.28	0\\
11.29	0\\
11.3	0\\
11.31	0\\
11.32	0\\
11.33	0\\
11.34	0\\
11.35	0\\
11.36	0\\
11.37	0\\
11.38	0\\
11.39	0\\
11.4	0\\
11.41	0\\
11.42	0\\
11.43	0\\
11.44	0\\
11.45	0\\
11.46	0\\
11.47	0\\
11.48	0\\
11.49	0\\
11.5	0\\
11.51	0\\
11.52	0\\
11.53	0\\
11.54	0\\
11.55	0\\
11.56	0\\
11.57	0\\
11.58	0\\
11.59	0\\
11.6	0\\
11.61	0\\
11.62	0\\
11.63	0\\
11.64	0\\
11.65	0\\
11.66	0\\
11.67	0\\
11.68	0\\
11.69	0\\
11.7	0\\
11.71	0\\
11.72	0\\
11.73	0\\
11.74	0\\
11.75	0\\
11.76	0\\
11.77	0\\
11.78	0\\
11.79	0\\
11.8	0\\
11.81	0\\
11.82	0\\
11.83	0\\
11.84	0\\
11.85	0\\
11.86	0\\
11.87	0\\
11.88	0\\
11.89	0\\
11.9	0\\
11.91	0\\
11.92	0\\
11.93	0\\
11.94	0\\
11.95	0\\
11.96	0\\
11.97	0\\
11.98	0\\
11.99	0\\
12	0\\
12.01	0\\
12.02	0\\
12.03	0\\
12.04	0\\
12.05	0\\
12.06	0\\
12.07	0\\
12.08	0\\
12.09	0\\
12.1	0\\
12.11	0\\
12.12	0\\
12.13	0\\
12.14	0\\
12.15	0\\
12.16	0\\
12.17	0\\
12.18	0\\
12.19	0\\
12.2	0\\
12.21	0\\
12.22	0\\
12.23	0\\
12.24	0\\
12.25	0\\
12.26	0\\
12.27	0\\
12.28	0\\
12.29	0\\
12.3	0\\
12.31	0\\
12.32	0\\
12.33	0\\
12.34	0\\
12.35	0\\
12.36	0\\
12.37	0\\
12.38	0\\
12.39	0\\
12.4	0\\
12.41	0\\
12.42	0\\
12.43	0\\
12.44	0\\
12.45	0\\
12.46	0\\
12.47	0\\
12.48	0\\
12.49	0\\
12.5	0\\
12.51	0\\
12.52	0\\
12.53	0\\
12.54	0\\
12.55	0\\
12.56	0\\
12.57	0\\
12.58	0\\
12.59	0\\
12.6	0\\
12.61	0\\
12.62	0\\
12.63	0\\
12.64	0\\
12.65	0\\
12.66	0\\
12.67	0\\
12.68	0\\
12.69	0\\
12.7	0\\
12.71	0\\
12.72	0\\
12.73	0\\
12.74	0\\
12.75	0\\
12.76	0\\
12.77	0\\
12.78	0\\
12.79	0\\
12.8	0\\
12.81	0\\
12.82	0\\
12.83	0\\
12.84	0\\
12.85	0\\
12.86	0\\
12.87	0\\
12.88	0\\
12.89	0\\
12.9	0\\
12.91	0\\
12.92	0\\
12.93	0\\
12.94	0\\
12.95	0\\
12.96	0\\
12.97	0\\
12.98	0\\
12.99	0\\
13	0\\
13.01	0\\
13.02	0\\
13.03	0\\
13.04	0\\
13.05	0\\
13.06	0\\
13.07	0\\
13.08	0\\
13.09	0\\
13.1	0\\
13.11	0\\
13.12	0\\
13.13	0\\
13.14	0\\
13.15	0\\
13.16	0\\
13.17	0\\
13.18	0\\
13.19	0\\
13.2	0\\
13.21	0\\
13.22	0\\
13.23	0\\
13.24	0\\
13.25	0\\
13.26	0\\
13.27	0\\
13.28	0\\
13.29	0\\
13.3	0\\
13.31	0\\
13.32	0\\
13.33	0\\
13.34	0\\
13.35	0\\
13.36	0\\
13.37	0\\
13.38	0\\
13.39	0\\
13.4	0\\
13.41	0\\
13.42	0\\
13.43	0\\
13.44	0\\
13.45	0\\
13.46	0\\
13.47	0\\
13.48	0\\
13.49	0\\
13.5	0\\
13.51	0\\
13.52	0\\
13.53	0\\
13.54	0\\
13.55	0\\
13.56	0\\
13.57	0\\
13.58	0\\
13.59	0\\
13.6	0\\
13.61	0\\
13.62	0\\
13.63	0\\
13.64	0\\
13.65	0\\
13.66	0\\
13.67	0\\
13.68	0\\
13.69	0\\
13.7	0\\
13.71	0\\
13.72	0\\
13.73	0\\
13.74	0\\
13.75	0\\
13.76	0\\
13.77	0\\
13.78	0\\
13.79	0\\
13.8	0\\
13.81	0\\
13.82	0\\
13.83	0\\
13.84	0\\
13.85	0\\
13.86	0\\
13.87	0\\
13.88	0\\
13.89	0\\
13.9	0\\
13.91	0\\
13.92	0\\
13.93	0\\
13.94	0\\
13.95	0\\
13.96	0\\
13.97	0\\
13.98	0\\
13.99	0\\
14	0\\
14.01	0\\
14.02	0\\
14.03	0\\
14.04	0\\
14.05	0\\
14.06	0\\
14.07	0\\
14.08	0\\
14.09	0\\
14.1	0\\
14.11	0\\
14.12	0\\
14.13	0\\
14.14	0\\
14.15	0\\
14.16	0\\
14.17	0\\
14.18	0\\
14.19	0\\
14.2	0\\
14.21	0\\
14.22	0\\
14.23	0\\
14.24	0\\
14.25	0\\
14.26	0\\
14.27	0\\
14.28	0\\
14.29	0\\
14.3	0\\
14.31	0\\
14.32	0\\
14.33	0\\
14.34	0\\
14.35	0\\
14.36	0\\
14.37	0\\
14.38	0\\
14.39	0\\
14.4	0\\
14.41	0\\
14.42	0\\
14.43	0\\
14.44	0\\
14.45	0\\
14.46	0\\
14.47	0\\
14.48	0\\
14.49	0\\
14.5	0\\
14.51	0\\
14.52	0\\
14.53	0\\
14.54	0\\
14.55	0\\
14.56	0\\
14.57	0\\
14.58	0\\
14.59	0\\
14.6	0\\
14.61	0\\
14.62	0\\
14.63	0\\
14.64	0\\
14.65	0\\
14.66	0\\
14.67	0\\
14.68	0\\
14.69	0\\
14.7	0\\
14.71	0\\
14.72	0\\
14.73	0\\
14.74	0\\
14.75	0\\
14.76	0\\
14.77	0\\
14.78	0\\
14.79	0\\
14.8	0\\
14.81	0\\
14.82	0\\
14.83	0\\
14.84	0\\
14.85	0\\
14.86	0\\
14.87	0\\
14.88	0\\
14.89	0\\
14.9	0\\
14.91	0\\
14.92	0\\
14.93	0\\
14.94	0\\
14.95	0\\
14.96	0\\
14.97	0\\
14.98	0\\
14.99	0\\
15	0\\
15.01	0\\
15.02	0\\
15.03	0\\
15.04	0\\
15.05	0\\
15.06	0\\
15.07	0\\
15.08	0\\
15.09	0\\
15.1	0\\
15.11	0\\
15.12	0\\
15.13	0\\
15.14	0\\
15.15	0\\
15.16	0\\
15.17	0\\
15.18	0\\
15.19	0\\
15.2	0\\
15.21	0\\
15.22	0\\
15.23	0\\
15.24	0\\
15.25	0\\
15.26	0\\
15.27	0\\
15.28	0\\
15.29	0\\
15.3	0\\
15.31	0\\
15.32	0\\
15.33	0\\
15.34	0\\
15.35	0\\
15.36	0\\
15.37	0\\
15.38	0\\
15.39	0\\
15.4	0\\
15.41	0\\
15.42	0\\
15.43	0\\
15.44	0\\
15.45	0\\
15.46	0\\
15.47	0\\
15.48	0\\
15.49	0\\
15.5	0\\
15.51	0\\
15.52	0\\
15.53	0\\
15.54	0\\
15.55	0\\
15.56	0\\
15.57	0\\
15.58	0\\
15.59	0\\
15.6	0\\
15.61	0\\
15.62	0\\
15.63	0\\
15.64	0\\
15.65	0\\
15.66	0\\
15.67	0\\
15.68	0\\
15.69	0\\
15.7	0\\
15.71	0\\
15.72	0\\
15.73	0\\
15.74	0\\
15.75	0\\
15.76	0\\
15.77	0\\
15.78	0\\
15.79	0\\
15.8	0\\
15.81	0\\
15.82	0\\
15.83	0\\
15.84	0\\
15.85	0\\
15.86	0\\
15.87	0\\
15.88	0\\
15.89	0\\
15.9	0\\
15.91	0\\
15.92	0\\
15.93	0\\
15.94	0\\
15.95	0\\
15.96	0\\
15.97	0\\
15.98	0\\
15.99	0\\
16	0\\
16.01	0\\
16.02	0\\
16.03	0\\
16.04	0\\
16.05	0\\
16.06	0\\
16.07	0\\
16.08	0\\
16.09	0\\
16.1	0\\
16.11	0\\
16.12	0\\
16.13	0\\
16.14	0\\
16.15	0\\
16.16	0\\
16.17	0\\
16.18	0\\
16.19	0\\
16.2	0\\
16.21	0\\
16.22	0\\
16.23	0\\
16.24	0\\
16.25	0\\
16.26	0\\
16.27	0\\
16.28	0\\
16.29	0\\
16.3	0\\
16.31	0\\
16.32	0\\
16.33	0\\
16.34	0\\
16.35	0\\
16.36	0\\
16.37	0\\
16.38	0\\
16.39	0\\
16.4	0\\
16.41	0\\
16.42	0\\
16.43	0\\
16.44	0\\
16.45	0\\
16.46	0\\
16.47	0\\
16.48	0\\
16.49	0\\
16.5	0\\
16.51	0\\
16.52	0\\
16.53	0\\
16.54	0\\
16.55	0\\
16.56	0\\
16.57	0\\
16.58	0\\
16.59	0\\
16.6	0\\
16.61	0\\
16.62	0\\
16.63	0\\
16.64	0\\
16.65	0\\
16.66	0\\
16.67	0\\
16.68	0\\
16.69	0\\
16.7	0\\
16.71	0\\
16.72	0\\
16.73	0\\
16.74	0\\
16.75	0\\
16.76	0\\
16.77	0\\
16.78	0\\
16.79	0\\
16.8	0\\
16.81	0\\
16.82	0\\
16.83	0\\
16.84	0\\
16.85	0\\
16.86	0\\
16.87	0\\
16.88	0\\
16.89	0\\
16.9	0\\
16.91	0\\
16.92	0\\
16.93	0\\
16.94	0\\
16.95	0\\
16.96	0\\
16.97	0\\
16.98	0\\
16.99	0\\
17	0\\
17.01	0\\
17.02	0\\
17.03	0\\
17.04	0\\
17.05	0\\
17.06	0\\
17.07	0\\
17.08	0\\
17.09	0\\
17.1	0\\
17.11	0\\
17.12	0\\
17.13	0\\
17.14	0\\
17.15	0\\
17.16	0\\
17.17	0\\
17.18	0\\
17.19	0\\
17.2	0\\
17.21	0\\
17.22	0\\
17.23	0\\
17.24	0\\
17.25	0\\
17.26	0\\
17.27	0\\
17.28	0\\
17.29	0\\
17.3	0\\
17.31	0\\
17.32	0\\
17.33	0\\
17.34	0\\
17.35	0\\
17.36	0\\
17.37	0\\
17.38	0\\
17.39	0\\
17.4	0\\
17.41	0\\
17.42	0\\
17.43	0\\
17.44	0\\
17.45	0\\
17.46	0\\
17.47	0\\
17.48	0\\
17.49	0\\
17.5	0\\
17.51	0\\
17.52	0\\
17.53	0\\
17.54	0\\
17.55	0\\
17.56	0\\
17.57	0\\
17.58	0\\
17.59	0\\
17.6	0\\
17.61	0\\
17.62	0\\
17.63	0\\
17.64	0\\
17.65	0\\
17.66	0\\
17.67	0\\
17.68	0\\
17.69	0\\
17.7	0\\
17.71	0\\
17.72	0\\
17.73	0\\
17.74	0\\
17.75	0\\
17.76	0\\
17.77	0\\
17.78	0\\
17.79	0\\
17.8	0\\
17.81	0\\
17.82	0\\
17.83	0\\
17.84	0\\
17.85	0\\
17.86	0\\
17.87	0\\
17.88	0\\
17.89	0\\
17.9	0\\
17.91	0\\
17.92	0\\
17.93	0\\
17.94	0\\
17.95	0\\
17.96	0\\
17.97	0\\
17.98	0\\
17.99	0\\
18	0\\
18.01	0\\
18.02	0\\
18.03	0\\
18.04	0\\
18.05	0\\
18.06	0\\
18.07	0\\
18.08	0\\
18.09	0\\
18.1	0\\
18.11	0\\
18.12	0\\
18.13	0\\
18.14	0\\
18.15	0\\
18.16	0\\
18.17	0\\
18.18	0\\
18.19	0\\
18.2	0\\
18.21	0\\
18.22	0\\
18.23	0\\
18.24	0\\
18.25	0\\
18.26	0\\
18.27	0\\
18.28	0\\
18.29	0\\
18.3	0\\
18.31	0\\
18.32	0\\
18.33	0\\
18.34	0\\
18.35	0\\
18.36	0\\
18.37	0\\
18.38	0\\
18.39	0\\
18.4	0\\
18.41	0\\
18.42	0\\
18.43	0\\
18.44	0\\
18.45	0\\
18.46	0\\
18.47	0\\
18.48	0\\
18.49	0\\
18.5	0\\
18.51	0\\
18.52	0\\
18.53	0\\
18.54	0\\
18.55	0\\
18.56	0\\
18.57	0\\
18.58	0\\
18.59	0\\
18.6	0\\
18.61	0\\
18.62	0\\
18.63	0\\
18.64	0\\
18.65	0\\
18.66	0\\
18.67	0\\
18.68	0\\
18.69	0\\
18.7	0\\
18.71	0\\
18.72	0\\
18.73	0\\
18.74	0\\
18.75	0\\
18.76	0\\
18.77	0\\
18.78	0\\
18.79	0\\
18.8	0\\
18.81	0\\
18.82	0\\
18.83	0\\
18.84	0\\
18.85	0\\
18.86	0\\
18.87	0\\
18.88	0\\
18.89	0\\
18.9	0\\
18.91	0\\
18.92	0\\
18.93	0\\
18.94	0\\
18.95	0\\
18.96	0\\
18.97	0\\
18.98	0\\
18.99	0\\
19	0\\
19.01	0\\
19.02	0\\
19.03	0\\
19.04	0\\
19.05	0\\
19.06	0\\
19.07	0\\
19.08	0\\
19.09	0\\
19.1	0\\
19.11	0\\
19.12	0\\
19.13	0\\
19.14	0\\
19.15	0\\
19.16	0\\
19.17	0\\
19.18	0\\
19.19	0\\
19.2	0\\
19.21	0\\
19.22	0\\
19.23	0\\
19.24	0\\
19.25	0\\
19.26	0\\
19.27	0\\
19.28	0\\
19.29	0\\
19.3	0\\
19.31	0\\
19.32	0\\
19.33	0\\
19.34	0\\
19.35	0\\
19.36	0\\
19.37	0\\
19.38	0\\
19.39	0\\
19.4	0\\
19.41	0\\
19.42	0\\
19.43	0\\
19.44	0\\
19.45	0\\
19.46	0\\
19.47	0\\
19.48	0\\
19.49	0\\
19.5	0\\
19.51	0\\
19.52	0\\
19.53	0\\
19.54	0\\
19.55	0\\
19.56	0\\
19.57	0\\
19.58	0\\
19.59	0\\
19.6	0\\
19.61	0\\
19.62	0\\
19.63	0\\
19.64	0\\
19.65	0\\
19.66	0\\
19.67	0\\
19.68	0\\
19.69	0\\
19.7	0\\
19.71	0\\
19.72	0\\
19.73	0\\
19.74	0\\
19.75	0\\
19.76	0\\
19.77	0\\
19.78	0\\
19.79	0\\
19.8	0\\
19.81	0\\
19.82	0\\
19.83	0\\
19.84	0\\
19.85	0\\
19.86	0\\
19.87	0\\
19.88	0\\
19.89	0\\
19.9	0\\
19.91	0\\
19.92	0\\
19.93	0\\
19.94	0\\
19.95	0\\
19.96	0\\
19.97	0\\
19.98	0\\
19.99	0\\
20	0\\
20.01	0\\
20.02	0\\
20.03	0\\
20.04	0\\
20.05	0\\
20.06	0\\
20.07	0\\
20.08	0\\
20.09	0\\
20.1	0\\
20.11	0\\
20.12	0\\
20.13	0\\
20.14	0\\
20.15	0\\
20.16	0\\
20.17	0\\
20.18	0\\
20.19	0\\
20.2	0\\
20.21	0\\
20.22	0\\
20.23	0\\
20.24	0\\
20.25	0\\
20.26	0\\
20.27	0\\
20.28	0\\
20.29	0\\
20.3	0\\
20.31	0\\
20.32	0\\
20.33	0\\
20.34	0\\
20.35	0\\
20.36	0\\
20.37	0\\
20.38	0\\
20.39	0\\
20.4	0\\
20.41	0\\
20.42	0\\
20.43	0\\
20.44	0\\
20.45	0\\
20.46	0\\
20.47	0\\
20.48	0\\
20.49	0\\
20.5	0\\
20.51	0\\
20.52	0\\
20.53	0\\
20.54	0\\
20.55	0\\
20.56	0\\
20.57	0\\
20.58	0\\
20.59	0\\
20.6	0\\
20.61	0\\
20.62	0\\
20.63	0\\
20.64	0\\
20.65	0\\
20.66	0\\
20.67	0\\
20.68	0\\
20.69	0\\
20.7	0\\
20.71	0\\
20.72	0\\
20.73	0\\
20.74	0\\
20.75	0\\
20.76	0\\
20.77	0\\
20.78	0\\
20.79	0\\
20.8	0\\
20.81	0\\
20.82	0\\
20.83	0\\
20.84	0\\
20.85	0\\
20.86	0\\
20.87	0\\
20.88	0\\
20.89	0\\
20.9	0\\
20.91	0\\
20.92	0\\
20.93	0\\
20.94	0\\
20.95	0\\
20.96	0\\
20.97	0\\
20.98	0\\
20.99	0\\
21	0\\
21.01	0\\
21.02	0\\
21.03	0\\
21.04	0\\
21.05	0\\
21.06	0\\
21.07	0\\
21.08	0\\
21.09	0\\
21.1	0\\
21.11	0\\
21.12	0\\
21.13	0\\
21.14	0\\
21.15	0\\
21.16	0\\
21.17	0\\
21.18	0\\
21.19	0\\
21.2	0\\
21.21	0\\
21.22	0\\
21.23	0\\
21.24	0\\
21.25	0\\
21.26	0\\
21.27	0\\
21.28	0\\
21.29	0\\
21.3	0\\
21.31	0\\
21.32	0\\
21.33	0\\
21.34	0\\
21.35	0\\
21.36	0\\
21.37	0\\
21.38	0\\
21.39	0\\
21.4	0\\
21.41	0\\
21.42	0\\
21.43	0\\
21.44	0\\
21.45	0\\
21.46	0\\
21.47	0\\
21.48	0\\
21.49	0\\
21.5	0\\
21.51	0\\
21.52	0\\
21.53	0\\
21.54	0\\
21.55	0\\
21.56	0\\
21.57	0\\
21.58	0\\
21.59	0\\
21.6	0\\
21.61	0\\
21.62	0\\
21.63	0\\
21.64	0\\
21.65	0\\
21.66	0\\
21.67	0\\
21.68	0\\
21.69	0\\
21.7	0\\
21.71	0\\
21.72	0\\
21.73	0\\
21.74	0\\
21.75	0\\
21.76	0\\
21.77	0\\
21.78	0\\
21.79	0\\
21.8	0\\
21.81	0\\
21.82	0\\
21.83	0\\
21.84	0\\
21.85	0\\
21.86	0\\
21.87	0\\
21.88	0\\
21.89	0\\
21.9	0\\
21.91	0\\
21.92	0\\
21.93	0\\
21.94	0\\
21.95	0\\
21.96	0\\
21.97	0\\
21.98	0\\
21.99	0\\
22	0\\
22.01	0\\
22.02	0\\
22.03	0\\
22.04	0\\
22.05	0\\
22.06	0\\
22.07	0\\
22.08	0\\
22.09	0\\
22.1	0\\
22.11	0\\
22.12	0\\
22.13	0\\
22.14	0\\
22.15	0\\
22.16	0\\
22.17	0\\
22.18	0\\
22.19	0\\
22.2	0\\
22.21	0\\
22.22	0\\
22.23	0\\
22.24	0\\
22.25	0\\
22.26	0\\
22.27	0\\
22.28	0\\
22.29	0\\
22.3	0\\
22.31	0\\
22.32	0\\
22.33	0\\
22.34	0\\
22.35	0\\
22.36	0\\
22.37	0\\
22.38	0\\
22.39	0\\
22.4	0\\
22.41	0\\
22.42	0\\
22.43	0\\
22.44	0\\
22.45	0\\
22.46	0\\
22.47	0\\
22.48	0\\
22.49	0\\
22.5	0\\
22.51	0\\
22.52	0\\
22.53	0\\
22.54	0\\
22.55	0\\
22.56	0\\
22.57	0\\
22.58	0\\
22.59	0\\
22.6	0\\
22.61	0\\
22.62	0\\
22.63	0\\
22.64	0\\
22.65	0\\
22.66	0\\
22.67	0\\
22.68	0\\
22.69	0\\
22.7	0\\
22.71	0\\
22.72	0\\
22.73	0\\
22.74	0\\
22.75	0\\
22.76	0\\
22.77	0\\
22.78	0\\
22.79	0\\
22.8	0\\
22.81	0\\
22.82	0\\
22.83	0\\
22.84	0\\
22.85	0\\
22.86	0\\
22.87	0\\
22.88	0\\
22.89	0\\
22.9	0\\
22.91	0\\
22.92	0\\
22.93	0\\
22.94	0\\
22.95	0\\
22.96	0\\
22.97	0\\
22.98	0\\
22.99	0\\
23	0\\
23.01	0\\
23.02	0\\
23.03	0\\
23.04	0\\
23.05	0\\
23.06	0\\
23.07	0\\
23.08	0\\
23.09	0\\
23.1	0\\
23.11	0\\
23.12	0\\
23.13	0\\
23.14	0\\
23.15	0\\
23.16	0\\
23.17	0\\
23.18	0\\
23.19	0\\
23.2	0\\
23.21	0\\
23.22	0\\
23.23	0\\
23.24	0\\
23.25	0\\
23.26	0\\
23.27	0\\
23.28	0\\
23.29	0\\
23.3	0\\
23.31	0\\
23.32	0\\
23.33	0\\
23.34	0\\
23.35	0\\
23.36	0\\
23.37	0\\
23.38	0\\
23.39	0\\
23.4	0\\
23.41	0\\
23.42	0\\
23.43	0\\
23.44	0\\
23.45	0\\
23.46	0\\
23.47	0\\
23.48	0\\
23.49	0\\
23.5	0\\
23.51	0\\
23.52	0\\
23.53	0\\
23.54	0\\
23.55	0\\
23.56	0\\
23.57	0\\
23.58	0\\
23.59	0\\
23.6	0\\
23.61	0\\
23.62	0\\
23.63	0\\
23.64	0\\
23.65	0\\
23.66	0\\
23.67	0\\
23.68	0\\
23.69	0\\
23.7	0\\
23.71	0\\
23.72	0\\
23.73	0\\
23.74	0\\
23.75	0\\
23.76	0\\
23.77	0\\
23.78	0\\
23.79	0\\
23.8	0\\
23.81	0\\
23.82	0\\
23.83	0\\
23.84	0\\
23.85	0\\
23.86	0\\
23.87	0\\
23.88	0\\
23.89	0\\
23.9	0\\
23.91	0\\
23.92	0\\
23.93	0\\
23.94	0\\
23.95	0\\
23.96	0\\
23.97	0\\
23.98	0\\
23.99	0\\
24	0\\
24.01	0\\
24.02	0\\
24.03	0\\
24.04	0\\
24.05	0\\
24.06	0\\
24.07	0\\
24.08	0\\
24.09	0\\
24.1	0\\
24.11	0\\
24.12	0\\
24.13	0\\
24.14	0\\
24.15	0\\
24.16	0\\
24.17	0\\
24.18	0\\
24.19	0\\
24.2	0\\
24.21	0\\
24.22	0\\
24.23	0\\
24.24	0\\
24.25	0\\
24.26	0\\
24.27	0\\
24.28	0\\
24.29	0\\
24.3	0\\
24.31	0\\
24.32	0\\
24.33	0\\
24.34	0\\
24.35	0\\
24.36	0\\
24.37	0\\
24.38	0\\
24.39	0\\
24.4	0\\
24.41	0\\
24.42	0\\
24.43	0\\
24.44	0\\
24.45	0\\
24.46	0\\
24.47	0\\
24.48	0\\
24.49	0\\
24.5	0\\
24.51	0\\
24.52	0\\
24.53	0\\
24.54	0\\
24.55	0\\
24.56	0\\
24.57	0\\
24.58	0\\
24.59	0\\
24.6	0\\
24.61	0\\
24.62	0\\
24.63	0\\
24.64	0\\
24.65	0\\
24.66	0\\
24.67	0\\
24.68	0\\
24.69	0\\
24.7	0\\
24.71	0\\
24.72	0\\
24.73	0\\
24.74	0\\
24.75	0\\
24.76	0\\
24.77	0\\
24.78	0\\
24.79	0\\
24.8	0\\
24.81	0\\
24.82	0\\
24.83	0\\
24.84	0\\
24.85	0\\
24.86	0\\
24.87	0\\
24.88	0\\
24.89	0\\
24.9	0\\
24.91	0\\
24.92	0\\
24.93	0\\
24.94	0\\
24.95	0\\
24.96	0\\
24.97	0\\
24.98	0\\
24.99	0\\
25	0\\
25.01	0\\
25.02	0\\
25.03	0\\
25.04	0\\
25.05	0\\
25.06	0\\
25.07	0\\
25.08	0\\
25.09	0\\
25.1	0\\
25.11	0\\
25.12	0\\
25.13	0\\
25.14	0\\
25.15	0\\
25.16	0\\
25.17	0\\
25.18	0\\
25.19	0\\
25.2	0\\
25.21	0\\
25.22	0\\
25.23	0\\
25.24	0\\
25.25	0\\
25.26	0\\
25.27	0\\
25.28	0\\
25.29	0\\
25.3	0\\
25.31	0\\
25.32	0\\
25.33	0\\
25.34	0\\
25.35	0\\
25.36	0\\
25.37	0\\
25.38	0\\
25.39	0\\
25.4	0\\
25.41	0\\
25.42	0\\
25.43	0\\
25.44	0\\
25.45	0\\
25.46	0\\
25.47	0\\
25.48	0\\
25.49	0\\
25.5	0\\
25.51	0\\
25.52	0\\
25.53	0\\
25.54	0\\
25.55	0\\
25.56	0\\
25.57	0\\
25.58	0\\
25.59	0\\
25.6	0\\
25.61	0\\
25.62	0\\
25.63	0\\
25.64	0\\
25.65	0\\
25.66	0\\
25.67	0\\
25.68	0\\
25.69	0\\
25.7	0\\
25.71	0\\
25.72	0\\
25.73	0\\
25.74	0\\
25.75	0\\
25.76	0\\
25.77	0\\
25.78	0\\
25.79	0\\
25.8	0\\
25.81	0\\
25.82	0\\
25.83	0\\
25.84	0\\
25.85	0\\
25.86	0\\
25.87	0\\
25.88	0\\
25.89	0\\
25.9	0\\
25.91	0\\
25.92	0\\
25.93	0\\
25.94	0\\
25.95	0\\
25.96	0\\
25.97	0\\
25.98	0\\
25.99	0\\
26	0\\
26.01	0\\
26.02	0\\
26.03	0\\
26.04	0\\
26.05	0\\
26.06	0\\
26.07	0\\
26.08	0\\
26.09	0\\
26.1	0\\
26.11	0\\
26.12	0\\
26.13	0\\
26.14	0\\
26.15	0\\
26.16	0\\
26.17	0\\
26.18	0\\
26.19	0\\
26.2	0\\
26.21	0\\
26.22	0\\
26.23	0\\
26.24	0\\
26.25	0\\
26.26	0\\
26.27	0\\
26.28	0\\
26.29	0\\
26.3	0\\
26.31	0\\
26.32	0\\
26.33	0\\
26.34	0\\
26.35	0\\
26.36	0\\
26.37	0\\
26.38	0\\
26.39	0\\
26.4	0\\
26.41	0\\
26.42	0\\
26.43	0\\
26.44	0\\
26.45	0\\
26.46	0\\
26.47	0\\
26.48	0\\
26.49	0\\
26.5	0\\
26.51	0\\
26.52	0\\
26.53	0\\
26.54	0\\
26.55	0\\
26.56	0\\
26.57	0\\
26.58	0\\
26.59	0\\
26.6	0\\
26.61	0\\
26.62	0\\
26.63	0\\
26.64	0\\
26.65	0\\
26.66	0\\
26.67	0\\
26.68	0\\
26.69	0\\
26.7	0\\
26.71	0\\
26.72	0\\
26.73	0\\
26.74	0\\
26.75	0\\
26.76	0\\
26.77	0\\
26.78	0\\
26.79	0\\
26.8	0\\
26.81	0\\
26.82	0\\
26.83	0\\
26.84	0\\
26.85	0\\
26.86	0\\
26.87	0\\
26.88	0\\
26.89	0\\
26.9	0\\
26.91	0\\
26.92	0\\
26.93	0\\
26.94	0\\
26.95	0\\
26.96	0\\
26.97	0\\
26.98	0\\
26.99	0\\
27	0\\
27.01	0\\
27.02	0\\
27.03	0\\
27.04	0\\
27.05	0\\
27.06	0\\
27.07	0\\
27.08	0\\
27.09	0\\
27.1	0\\
27.11	0\\
27.12	0\\
27.13	0\\
27.14	0\\
27.15	0\\
27.16	0\\
27.17	0\\
27.18	0\\
27.19	0\\
27.2	0\\
27.21	0\\
27.22	0\\
27.23	0\\
27.24	0\\
27.25	0\\
27.26	0\\
27.27	0\\
27.28	0\\
27.29	0\\
27.3	0\\
27.31	0\\
27.32	0\\
27.33	0\\
27.34	0\\
27.35	0\\
27.36	0\\
27.37	0\\
27.38	0\\
27.39	0\\
27.4	0\\
27.41	0\\
27.42	0\\
27.43	0\\
27.44	0\\
27.45	0\\
27.46	0\\
27.47	0\\
27.48	0\\
27.49	0\\
27.5	0\\
27.51	0\\
27.52	0\\
27.53	0\\
27.54	0\\
27.55	0\\
27.56	0\\
27.57	0\\
27.58	0\\
27.59	0\\
27.6	0\\
27.61	0\\
27.62	0\\
27.63	0\\
27.64	0\\
27.65	0\\
27.66	0\\
27.67	0\\
27.68	0\\
27.69	0\\
27.7	0\\
27.71	0\\
27.72	0\\
27.73	0\\
27.74	0\\
27.75	0\\
27.76	0\\
27.77	0\\
27.78	0\\
27.79	0\\
27.8	0\\
27.81	0\\
27.82	0\\
27.83	0\\
27.84	0\\
27.85	0\\
27.86	0\\
27.87	0\\
27.88	0\\
27.89	0\\
27.9	0\\
27.91	0\\
27.92	0\\
27.93	0\\
27.94	0\\
27.95	0\\
27.96	0\\
27.97	0\\
27.98	0\\
27.99	0\\
28	0\\
28.01	0\\
28.02	0\\
28.03	0\\
28.04	0\\
28.05	0\\
28.06	0\\
28.07	0\\
28.08	0\\
28.09	0\\
28.1	0\\
28.11	0\\
28.12	0\\
28.13	0\\
28.14	0\\
28.15	0\\
28.16	0\\
28.17	0\\
28.18	0\\
28.19	0\\
28.2	0\\
28.21	0\\
28.22	0\\
28.23	0\\
28.24	0\\
28.25	0\\
28.26	0\\
28.27	0\\
28.28	0\\
28.29	0\\
28.3	0\\
28.31	0\\
28.32	0\\
28.33	0\\
28.34	0\\
28.35	0\\
28.36	0\\
28.37	0\\
28.38	0\\
28.39	0\\
28.4	0\\
28.41	0\\
28.42	0\\
28.43	0\\
28.44	0\\
28.45	0\\
28.46	0\\
28.47	0\\
28.48	0\\
28.49	0\\
28.5	0\\
28.51	0\\
28.52	0\\
28.53	0\\
28.54	0\\
28.55	0\\
28.56	0\\
28.57	0\\
28.58	0\\
28.59	0\\
28.6	0\\
28.61	0\\
28.62	0\\
28.63	0\\
28.64	0\\
28.65	0\\
28.66	0\\
28.67	0\\
28.68	0\\
28.69	0\\
28.7	0\\
28.71	0\\
28.72	0\\
28.73	0\\
28.74	0\\
28.75	0\\
28.76	0\\
28.77	0\\
28.78	0\\
28.79	0\\
28.8	0\\
28.81	0\\
28.82	0\\
28.83	0\\
28.84	0\\
28.85	0\\
28.86	0\\
28.87	0\\
28.88	0\\
28.89	0\\
28.9	0\\
28.91	0\\
28.92	0\\
28.93	0\\
28.94	0\\
28.95	0\\
28.96	0\\
28.97	0\\
28.98	0\\
28.99	0\\
29	0\\
29.01	0\\
29.02	0\\
29.03	0\\
29.04	0\\
29.05	0\\
29.06	0\\
29.07	0\\
29.08	0\\
29.09	0\\
29.1	0\\
29.11	0\\
29.12	0\\
29.13	0\\
29.14	0\\
29.15	0\\
29.16	0\\
29.17	0\\
29.18	0\\
29.19	0\\
29.2	0\\
29.21	0\\
29.22	0\\
29.23	0\\
29.24	0\\
29.25	0\\
29.26	0\\
29.27	0\\
29.28	0\\
29.29	0\\
29.3	0\\
29.31	0\\
29.32	0\\
29.33	0\\
29.34	0\\
29.35	0\\
29.36	0\\
29.37	0\\
29.38	0\\
29.39	0\\
29.4	0\\
29.41	0\\
29.42	0\\
29.43	0\\
29.44	0\\
29.45	0\\
29.46	0\\
29.47	0\\
29.48	0\\
29.49	0\\
29.5	0\\
29.51	0\\
29.52	0\\
29.53	0\\
29.54	0\\
29.55	0\\
29.56	0\\
29.57	0\\
29.58	0\\
29.59	0\\
29.6	0\\
29.61	0\\
29.62	0\\
29.63	0\\
29.64	0\\
29.65	0\\
29.66	0\\
29.67	0\\
29.68	0\\
29.69	0\\
29.7	0\\
29.71	0\\
29.72	0\\
29.73	0\\
29.74	0\\
29.75	0\\
29.76	0\\
29.77	0\\
29.78	0\\
29.79	0\\
29.8	0\\
29.81	0\\
29.82	0\\
29.83	0\\
29.84	0\\
29.85	0\\
29.86	0\\
29.87	0\\
29.88	0\\
29.89	0\\
29.9	0\\
29.91	0\\
29.92	0\\
29.93	0\\
29.94	0\\
29.95	0\\
29.96	0\\
29.97	0\\
29.98	0\\
29.99	0\\
30	0\\
30.01	0\\
30.02	0\\
30.03	0\\
30.04	0\\
30.05	0\\
30.06	0\\
30.07	0\\
30.08	0\\
30.09	0\\
30.1	0\\
30.11	0\\
30.12	0\\
30.13	0\\
30.14	0\\
30.15	0\\
30.16	0\\
30.17	0\\
30.18	0\\
30.19	0\\
30.2	0\\
30.21	0\\
30.22	0\\
30.23	0\\
30.24	0\\
30.25	0\\
30.26	0\\
30.27	0\\
30.28	0\\
30.29	0\\
30.3	0\\
30.31	0\\
30.32	0\\
30.33	0\\
30.34	0\\
30.35	0\\
30.36	0\\
30.37	0\\
30.38	0\\
30.39	0\\
30.4	0\\
30.41	0\\
30.42	0\\
30.43	0\\
30.44	0\\
30.45	0\\
30.46	0\\
30.47	0\\
30.48	0\\
30.49	0\\
30.5	0\\
30.51	0\\
30.52	0\\
30.53	0\\
30.54	0\\
30.55	0\\
30.56	0\\
30.57	0\\
30.58	0\\
30.59	0\\
30.6	0\\
30.61	0\\
30.62	0\\
30.63	0\\
30.64	0\\
30.65	0\\
30.66	0\\
30.67	0\\
30.68	0\\
30.69	0\\
30.7	0\\
30.71	0\\
30.72	0\\
30.73	0\\
30.74	0\\
30.75	0\\
30.76	0\\
30.77	0\\
30.78	0\\
30.79	0\\
30.8	0\\
30.81	0\\
30.82	0\\
30.83	0\\
30.84	0\\
30.85	0\\
30.86	0\\
30.87	0\\
30.88	0\\
30.89	0\\
30.9	0\\
30.91	0\\
30.92	0\\
30.93	0\\
30.94	0\\
30.95	0\\
30.96	0\\
30.97	0\\
30.98	0\\
30.99	0\\
31	0\\
31.01	0\\
31.02	0\\
31.03	0\\
31.04	0\\
31.05	0\\
31.06	0\\
31.07	0\\
31.08	0\\
31.09	0\\
31.1	0\\
31.11	0\\
31.12	0\\
31.13	0\\
31.14	0\\
31.15	0\\
31.16	0\\
31.17	0\\
31.18	0\\
31.19	0\\
31.2	0\\
31.21	0\\
31.22	0\\
31.23	0\\
31.24	0\\
31.25	0\\
31.26	0\\
31.27	0\\
31.28	0\\
31.29	0\\
31.3	0\\
31.31	0\\
31.32	0\\
31.33	0\\
31.34	0\\
31.35	0\\
31.36	0\\
31.37	0\\
31.38	0\\
31.39	0\\
31.4	0\\
31.41	0\\
31.42	0\\
31.43	0\\
31.44	0\\
31.45	0\\
31.46	0\\
31.47	0\\
31.48	0\\
31.49	0\\
31.5	0\\
31.51	0\\
31.52	0\\
31.53	0\\
31.54	0\\
31.55	0\\
31.56	0\\
31.57	0\\
31.58	0\\
31.59	0\\
31.6	0\\
31.61	0\\
31.62	0\\
31.63	0\\
31.64	0\\
31.65	0\\
31.66	0\\
31.67	0\\
31.68	0\\
31.69	0\\
31.7	0\\
31.71	0\\
31.72	0\\
31.73	0\\
31.74	0\\
31.75	0\\
31.76	0\\
31.77	0\\
31.78	0\\
31.79	0\\
31.8	0\\
31.81	0\\
31.82	0\\
31.83	0\\
31.84	0\\
31.85	0\\
31.86	0\\
31.87	0\\
31.88	0\\
31.89	0\\
31.9	0\\
31.91	0\\
31.92	0\\
31.93	0\\
31.94	0\\
31.95	0\\
31.96	0\\
31.97	0\\
31.98	0\\
31.99	0\\
32	0\\
32.01	0\\
32.02	0\\
32.03	0\\
32.04	0\\
32.05	0\\
32.06	0\\
32.07	0\\
32.08	0\\
32.09	0\\
32.1	0\\
32.11	0\\
32.12	0\\
32.13	0\\
32.14	0\\
32.15	0\\
32.16	0\\
32.17	0\\
32.18	0\\
32.19	0\\
32.2	0\\
32.21	0\\
32.22	0\\
32.23	0\\
32.24	0\\
32.25	0\\
32.26	0\\
32.27	0\\
32.28	0\\
32.29	0\\
32.3	0\\
32.31	0\\
32.32	0\\
32.33	0\\
32.34	0\\
32.35	0\\
32.36	0\\
32.37	0\\
32.38	0\\
32.39	0\\
32.4	0\\
32.41	0\\
32.42	0\\
32.43	0\\
32.44	0\\
32.45	0\\
32.46	0\\
32.47	0\\
32.48	0\\
32.49	0\\
32.5	0\\
32.51	0\\
32.52	0\\
32.53	0\\
32.54	0\\
32.55	0\\
32.56	0\\
32.57	0\\
32.58	0\\
32.59	0\\
32.6	0\\
32.61	0\\
32.62	0\\
32.63	0\\
32.64	0\\
32.65	0\\
32.66	0\\
32.67	0\\
32.68	0\\
32.69	0\\
32.7	0\\
32.71	0\\
32.72	0\\
32.73	0\\
32.74	0\\
32.75	0\\
32.76	0\\
32.77	0\\
32.78	0\\
32.79	0\\
32.8	0\\
32.81	0\\
32.82	0\\
32.83	0\\
32.84	0\\
32.85	0\\
32.86	0\\
32.87	0\\
32.88	0\\
32.89	0\\
32.9	0\\
32.91	0\\
32.92	0\\
32.93	0\\
32.94	0\\
32.95	0\\
32.96	0\\
32.97	0\\
32.98	0\\
32.99	0\\
33	0\\
33.01	0\\
33.02	0\\
33.03	0\\
33.04	0\\
33.05	0\\
33.06	0\\
33.07	0\\
33.08	0\\
33.09	0\\
33.1	0\\
33.11	0\\
33.12	0\\
33.13	0\\
33.14	0\\
33.15	0\\
33.16	0\\
33.17	0\\
33.18	0\\
33.19	0\\
33.2	0\\
33.21	0\\
33.22	0\\
33.23	0\\
33.24	0\\
33.25	0\\
33.26	0\\
33.27	0\\
33.28	0\\
33.29	0\\
33.3	0\\
33.31	0\\
33.32	0\\
33.33	0\\
33.34	0\\
33.35	0\\
33.36	0\\
33.37	0\\
33.38	0\\
33.39	0\\
33.4	0\\
33.41	0\\
33.42	0\\
33.43	0\\
33.44	0\\
33.45	0\\
33.46	0\\
33.47	0\\
33.48	0\\
33.49	0\\
33.5	0\\
33.51	0\\
33.52	0\\
33.53	0\\
33.54	0\\
33.55	0\\
33.56	0\\
33.57	0\\
33.58	0\\
33.59	0\\
33.6	0\\
33.61	0\\
33.62	0\\
33.63	0\\
33.64	0\\
33.65	0\\
33.66	0\\
33.67	0\\
33.68	0\\
33.69	0\\
33.7	0\\
33.71	0\\
33.72	0\\
33.73	0\\
33.74	0\\
33.75	0\\
33.76	0\\
33.77	0\\
33.78	0\\
33.79	0\\
33.8	0\\
33.81	0\\
33.82	0\\
33.83	0\\
33.84	0\\
33.85	0\\
33.86	0\\
33.87	0\\
33.88	0\\
33.89	0\\
33.9	0\\
33.91	0\\
33.92	0\\
33.93	0\\
33.94	0\\
33.95	0\\
33.96	0\\
33.97	0\\
33.98	0\\
33.99	0\\
34	0\\
34.01	0\\
34.02	0\\
34.03	0\\
34.04	0\\
34.05	0\\
34.06	0\\
34.07	0\\
34.08	0\\
34.09	0\\
34.1	0\\
34.11	0\\
34.12	0\\
34.13	0\\
34.14	0\\
34.15	0\\
34.16	0\\
34.17	0\\
34.18	0\\
34.19	0\\
34.2	0\\
34.21	0\\
34.22	0\\
34.23	0\\
34.24	0\\
34.25	0\\
34.26	0\\
34.27	0\\
34.28	0\\
34.29	0\\
34.3	0\\
34.31	0\\
34.32	0\\
34.33	0\\
34.34	0\\
34.35	0\\
34.36	0\\
34.37	0\\
34.38	0\\
34.39	0\\
34.4	0\\
34.41	0\\
34.42	0\\
34.43	0\\
34.44	0\\
34.45	0\\
34.46	0\\
34.47	0\\
34.48	0\\
34.49	0\\
34.5	0\\
34.51	0\\
34.52	0\\
34.53	0\\
34.54	0\\
34.55	0\\
34.56	0\\
34.57	0\\
34.58	0\\
34.59	0\\
34.6	0\\
34.61	0\\
34.62	0\\
34.63	0\\
34.64	0\\
34.65	0\\
34.66	0\\
34.67	0\\
34.68	0\\
34.69	0\\
34.7	0\\
34.71	0\\
34.72	0\\
34.73	0\\
34.74	0\\
34.75	0\\
34.76	0\\
34.77	0\\
34.78	0\\
34.79	0\\
34.8	0\\
34.81	0\\
34.82	0\\
34.83	0\\
34.84	0\\
34.85	0\\
34.86	0\\
34.87	0\\
34.88	0\\
34.89	0\\
34.9	0\\
34.91	0\\
34.92	0\\
34.93	0\\
34.94	0\\
34.95	0\\
34.96	0\\
34.97	0\\
34.98	0\\
34.99	0\\
35	0\\
35.01	0\\
35.02	0\\
35.03	0\\
35.04	0\\
35.05	0\\
35.06	0\\
35.07	0\\
35.08	0\\
35.09	0\\
35.1	0\\
35.11	0\\
35.12	0\\
35.13	0\\
35.14	0\\
35.15	0\\
35.16	0\\
35.17	0\\
35.18	0\\
35.19	0\\
35.2	0\\
35.21	0\\
35.22	0\\
35.23	0\\
35.24	0\\
35.25	0\\
35.26	0\\
35.27	0\\
35.28	0\\
35.29	0\\
35.3	0\\
35.31	0\\
35.32	0\\
35.33	0\\
35.34	0\\
35.35	0\\
35.36	0\\
35.37	0\\
35.38	0\\
35.39	0\\
35.4	0\\
35.41	0\\
35.42	0\\
35.43	0\\
35.44	0\\
35.45	0\\
35.46	0\\
35.47	0\\
35.48	0\\
35.49	0\\
35.5	0\\
35.51	0\\
35.52	0\\
35.53	0\\
35.54	0\\
35.55	0\\
35.56	0\\
35.57	0\\
35.58	0\\
35.59	0\\
35.6	0\\
35.61	0\\
35.62	0\\
35.63	0\\
35.64	0\\
35.65	0\\
35.66	0\\
35.67	0\\
35.68	0\\
35.69	0\\
35.7	0\\
35.71	0\\
35.72	0\\
35.73	0\\
35.74	0\\
35.75	0\\
35.76	0\\
35.77	0\\
35.78	0\\
35.79	0\\
35.8	0\\
35.81	0\\
35.82	0\\
35.83	0\\
35.84	0\\
35.85	0\\
35.86	0\\
35.87	0\\
35.88	0\\
35.89	0\\
35.9	0\\
35.91	0\\
35.92	0\\
35.93	0\\
35.94	0\\
35.95	0\\
35.96	0\\
35.97	0\\
35.98	0\\
35.99	0\\
36	0\\
36.01	0\\
36.02	0\\
36.03	0\\
36.04	0\\
36.05	0\\
36.06	0\\
36.07	0\\
36.08	0\\
36.09	0\\
36.1	0\\
36.11	0\\
36.12	0\\
36.13	0\\
36.14	0\\
36.15	0\\
36.16	0\\
36.17	0\\
36.18	0\\
36.19	0\\
36.2	0\\
36.21	0\\
36.22	0\\
36.23	0\\
36.24	0\\
36.25	0\\
36.26	0\\
36.27	0\\
36.28	0\\
36.29	0\\
36.3	0\\
36.31	0\\
36.32	0\\
36.33	0\\
36.34	0\\
36.35	0\\
36.36	0\\
36.37	0\\
36.38	0\\
36.39	0\\
36.4	0\\
36.41	0\\
36.42	0\\
36.43	0\\
36.44	0\\
36.45	0\\
36.46	0\\
36.47	0\\
36.48	0\\
36.49	0\\
36.5	0\\
36.51	0\\
36.52	0\\
36.53	0\\
36.54	0\\
36.55	0\\
36.56	0\\
36.57	0\\
36.58	0\\
36.59	0\\
36.6	0\\
36.61	0\\
36.62	0\\
36.63	0\\
36.64	0\\
36.65	0\\
36.66	0\\
36.67	0\\
36.68	0\\
36.69	0\\
36.7	0\\
36.71	0\\
36.72	0\\
36.73	0\\
36.74	0\\
36.75	0\\
36.76	0\\
36.77	0\\
36.78	0\\
36.79	0\\
36.8	0\\
36.81	0\\
36.82	0\\
36.83	0\\
36.84	0\\
36.85	0\\
36.86	0\\
36.87	0\\
36.88	0\\
36.89	0\\
36.9	0\\
36.91	0\\
36.92	0\\
36.93	0\\
36.94	0\\
36.95	0\\
36.96	0\\
36.97	0\\
36.98	0\\
36.99	0\\
37	0\\
37.01	0\\
37.02	0\\
37.03	0\\
37.04	0\\
37.05	0\\
37.06	0\\
37.07	0\\
37.08	0\\
37.09	0\\
37.1	0\\
37.11	0\\
37.12	0\\
37.13	0\\
37.14	0\\
37.15	0\\
37.16	0\\
37.17	0\\
37.18	0\\
37.19	0\\
37.2	0\\
37.21	0\\
37.22	0\\
37.23	0\\
37.24	0\\
37.25	0\\
37.26	0\\
37.27	0\\
37.28	0\\
37.29	0\\
37.3	0\\
37.31	0\\
37.32	0\\
37.33	0\\
37.34	0\\
37.35	0\\
37.36	0\\
37.37	0\\
37.38	0\\
37.39	0\\
37.4	0\\
37.41	0\\
37.42	0\\
37.43	0\\
37.44	0\\
37.45	0\\
37.46	0\\
37.47	0\\
37.48	0\\
37.49	0\\
37.5	0\\
37.51	0\\
37.52	0\\
37.53	0\\
37.54	0\\
37.55	0\\
37.56	0\\
37.57	0\\
37.58	0\\
37.59	0\\
37.6	0\\
37.61	0\\
37.62	0\\
37.63	0\\
37.64	0\\
37.65	0\\
37.66	0\\
37.67	0\\
37.68	0\\
37.69	0\\
37.7	0\\
37.71	0\\
37.72	0\\
37.73	0\\
37.74	0\\
37.75	0\\
37.76	0\\
37.77	0\\
37.78	0\\
37.79	0\\
37.8	0\\
37.81	0\\
37.82	0\\
37.83	0\\
37.84	0\\
37.85	0\\
37.86	0\\
37.87	0\\
37.88	0\\
37.89	0\\
37.9	0\\
37.91	0\\
37.92	0\\
37.93	0\\
37.94	0\\
37.95	0\\
37.96	0\\
37.97	0\\
37.98	0\\
37.99	0\\
38	0\\
38.01	0\\
38.02	0\\
38.03	0\\
38.04	0\\
38.05	0\\
38.06	0\\
38.07	0\\
38.08	0\\
38.09	0\\
38.1	0\\
38.11	0\\
38.12	0\\
38.13	0\\
38.14	0\\
38.15	0\\
38.16	0\\
38.17	0\\
38.18	0\\
38.19	0\\
38.2	0\\
38.21	0\\
38.22	0\\
38.23	0\\
38.24	0\\
38.25	0\\
38.26	0\\
38.27	0\\
38.28	0\\
38.29	0\\
38.3	0\\
38.31	0\\
38.32	0\\
38.33	0\\
38.34	0\\
38.35	0\\
38.36	0\\
38.37	0\\
38.38	0\\
38.39	0\\
38.4	0\\
38.41	0\\
38.42	0\\
38.43	0\\
38.44	0\\
38.45	0\\
38.46	0\\
38.47	0\\
38.48	0\\
38.49	0\\
38.5	0\\
38.51	0\\
38.52	0\\
38.53	0\\
38.54	0\\
38.55	0\\
38.56	0\\
38.57	0\\
38.58	0\\
38.59	0\\
38.6	0\\
38.61	0\\
38.62	0\\
38.63	0\\
38.64	0\\
38.65	0\\
38.66	0\\
38.67	0\\
38.68	0\\
38.69	0\\
38.7	0\\
38.71	0\\
38.72	0\\
38.73	0\\
38.74	0\\
38.75	0\\
38.76	0\\
38.77	0\\
38.78	0\\
38.79	0\\
38.8	0\\
38.81	0\\
38.82	0\\
38.83	0\\
38.84	0\\
38.85	0\\
38.86	0\\
38.87	0\\
38.88	0\\
38.89	0\\
38.9	0\\
38.91	0\\
38.92	0\\
38.93	0\\
38.94	0\\
38.95	0\\
38.96	0\\
38.97	0\\
38.98	0\\
38.99	0\\
39	0\\
39.01	0\\
39.02	0\\
39.03	0\\
39.04	0\\
39.05	0\\
39.06	0\\
39.07	0\\
39.08	0\\
39.09	0\\
39.1	0\\
39.11	0\\
39.12	0\\
39.13	0\\
39.14	0\\
39.15	0\\
39.16	0\\
39.17	0\\
39.18	0\\
39.19	0\\
39.2	0\\
39.21	0\\
39.22	0\\
39.23	0\\
39.24	0\\
39.25	0\\
39.26	0\\
39.27	0\\
39.28	0\\
39.29	0\\
39.3	0\\
39.31	0\\
39.32	0\\
39.33	0\\
39.34	0\\
39.35	0\\
39.36	0\\
39.37	0\\
39.38	0\\
39.39	0\\
39.4	0\\
39.41	0\\
39.42	0\\
39.43	0\\
39.44	0\\
39.45	0\\
39.46	0\\
39.47	0\\
39.48	0\\
39.49	0\\
39.5	0\\
39.51	0\\
39.52	0\\
39.53	0\\
39.54	0\\
39.55	0\\
39.56	0\\
39.57	0\\
39.58	0\\
39.59	0\\
39.6	0\\
39.61	0\\
39.62	0\\
39.63	0\\
39.64	0\\
39.65	0\\
39.66	0\\
39.67	0\\
39.68	0\\
39.69	0\\
39.7	0\\
39.71	0\\
39.72	0\\
39.73	0\\
39.74	0\\
39.75	0\\
39.76	0\\
39.77	0\\
39.78	0\\
39.79	0\\
39.8	0\\
39.81	0\\
39.82	0\\
39.83	0\\
39.84	0\\
39.85	0\\
39.86	0\\
39.87	0\\
39.88	0\\
39.89	0\\
39.9	0\\
39.91	0\\
39.92	0\\
39.93	0\\
39.94	0\\
39.95	0\\
39.96	0\\
39.97	0\\
39.98	0\\
39.99	0\\
40	0\\
40.01	0\\
};
\addplot [color=mycolor1,solid,forget plot]
  table[row sep=crcr]{%
40.01	0\\
40.02	0\\
40.03	0\\
40.04	0\\
40.05	0\\
40.06	0\\
40.07	0\\
40.08	0\\
40.09	0\\
40.1	0\\
40.11	0\\
40.12	0\\
40.13	0\\
40.14	0\\
40.15	0\\
40.16	0\\
40.17	0\\
40.18	0\\
40.19	0\\
40.2	0\\
40.21	0\\
40.22	0\\
40.23	0\\
40.24	0\\
40.25	0\\
40.26	0\\
40.27	0\\
40.28	0\\
40.29	0\\
40.3	0\\
40.31	0\\
40.32	0\\
40.33	0\\
40.34	0\\
40.35	0\\
40.36	0\\
40.37	0\\
40.38	0\\
40.39	0\\
40.4	0\\
40.41	0\\
40.42	0\\
40.43	0\\
40.44	0\\
40.45	0\\
40.46	0\\
40.47	0\\
40.48	0\\
40.49	0\\
40.5	0\\
40.51	0\\
40.52	0\\
40.53	0\\
40.54	0\\
40.55	0\\
40.56	0\\
40.57	0\\
40.58	0\\
40.59	0\\
40.6	0\\
40.61	0\\
40.62	0\\
40.63	0\\
40.64	0\\
40.65	0\\
40.66	0\\
40.67	0\\
40.68	0\\
40.69	0\\
40.7	0\\
40.71	0\\
40.72	0\\
40.73	0\\
40.74	0\\
40.75	0\\
40.76	0\\
40.77	0\\
40.78	0\\
40.79	0\\
40.8	0\\
40.81	0\\
40.82	0\\
40.83	0\\
40.84	0\\
40.85	0\\
40.86	0\\
40.87	0\\
40.88	0\\
40.89	0\\
40.9	0\\
40.91	0\\
40.92	0\\
40.93	0\\
40.94	0\\
40.95	0\\
40.96	0\\
40.97	0\\
40.98	0\\
40.99	0\\
41	0\\
41.01	0\\
41.02	0\\
41.03	0\\
41.04	0\\
41.05	0\\
41.06	0\\
41.07	0\\
41.08	0\\
41.09	0\\
41.1	0\\
41.11	0\\
41.12	0\\
41.13	0\\
41.14	0\\
41.15	0\\
41.16	0\\
41.17	0\\
41.18	0\\
41.19	0\\
41.2	0\\
41.21	0\\
41.22	0\\
41.23	0\\
41.24	0\\
41.25	0\\
41.26	0\\
41.27	0\\
41.28	0\\
41.29	0\\
41.3	0\\
41.31	0\\
41.32	0\\
41.33	0\\
41.34	0\\
41.35	0\\
41.36	0\\
41.37	0\\
41.38	0\\
41.39	0\\
41.4	0\\
41.41	0\\
41.42	0\\
41.43	0\\
41.44	0\\
41.45	0\\
41.46	0\\
41.47	0\\
41.48	0\\
41.49	0\\
41.5	0\\
41.51	0\\
41.52	0\\
41.53	0\\
41.54	0\\
41.55	0\\
41.56	0\\
41.57	0\\
41.58	0\\
41.59	0\\
41.6	0\\
41.61	0\\
41.62	0\\
41.63	0\\
41.64	0\\
41.65	0\\
41.66	0\\
41.67	0\\
41.68	0\\
41.69	0\\
41.7	0\\
41.71	0\\
41.72	0\\
41.73	0\\
41.74	0\\
41.75	0\\
41.76	0\\
41.77	0\\
41.78	0\\
41.79	0\\
41.8	0\\
41.81	0\\
41.82	0\\
41.83	0\\
41.84	0\\
41.85	0\\
41.86	0\\
41.87	0\\
41.88	0\\
41.89	0\\
41.9	0\\
41.91	0\\
41.92	0\\
41.93	0\\
41.94	0\\
41.95	0\\
41.96	0\\
41.97	0\\
41.98	0\\
41.99	0\\
42	0\\
42.01	0\\
42.02	0\\
42.03	0\\
42.04	0\\
42.05	0\\
42.06	0\\
42.07	0\\
42.08	0\\
42.09	0\\
42.1	0\\
42.11	0\\
42.12	0\\
42.13	0\\
42.14	0\\
42.15	0\\
42.16	0\\
42.17	0\\
42.18	0\\
42.19	0\\
42.2	0\\
42.21	0\\
42.22	0\\
42.23	0\\
42.24	0\\
42.25	0\\
42.26	0\\
42.27	0\\
42.28	0\\
42.29	0\\
42.3	0\\
42.31	0\\
42.32	0\\
42.33	0\\
42.34	0\\
42.35	0\\
42.36	0\\
42.37	0\\
42.38	0\\
42.39	0\\
42.4	0\\
42.41	0\\
42.42	0\\
42.43	0\\
42.44	0\\
42.45	0\\
42.46	0\\
42.47	0\\
42.48	0\\
42.49	0\\
42.5	0\\
42.51	0\\
42.52	0\\
42.53	0\\
42.54	0\\
42.55	0\\
42.56	0\\
42.57	0\\
42.58	0\\
42.59	0\\
42.6	0\\
42.61	0\\
42.62	0\\
42.63	0\\
42.64	0\\
42.65	0\\
42.66	0\\
42.67	0\\
42.68	0\\
42.69	0\\
42.7	0\\
42.71	0\\
42.72	0\\
42.73	0\\
42.74	0\\
42.75	0\\
42.76	0\\
42.77	0\\
42.78	0\\
42.79	0\\
42.8	0\\
42.81	0\\
42.82	0\\
42.83	0\\
42.84	0\\
42.85	0\\
42.86	0\\
42.87	0\\
42.88	0\\
42.89	0\\
42.9	0\\
42.91	0\\
42.92	0\\
42.93	0\\
42.94	0\\
42.95	0\\
42.96	0\\
42.97	0\\
42.98	0\\
42.99	0\\
43	0\\
43.01	0\\
43.02	0\\
43.03	0\\
43.04	0\\
43.05	0\\
43.06	0\\
43.07	0\\
43.08	0\\
43.09	0\\
43.1	0\\
43.11	0\\
43.12	0\\
43.13	0\\
43.14	0\\
43.15	0\\
43.16	0\\
43.17	0\\
43.18	0\\
43.19	0\\
43.2	0\\
43.21	0\\
43.22	0\\
43.23	0\\
43.24	0\\
43.25	0\\
43.26	0\\
43.27	0\\
43.28	0\\
43.29	0\\
43.3	0\\
43.31	0\\
43.32	0\\
43.33	0\\
43.34	0\\
43.35	0\\
43.36	0\\
43.37	0\\
43.38	0\\
43.39	0\\
43.4	0\\
43.41	0\\
43.42	0\\
43.43	0\\
43.44	0\\
43.45	0\\
43.46	0\\
43.47	0\\
43.48	0\\
43.49	0\\
43.5	0\\
43.51	0\\
43.52	0\\
43.53	0\\
43.54	0\\
43.55	0\\
43.56	0\\
43.57	0\\
43.58	0\\
43.59	0\\
43.6	0\\
43.61	0\\
43.62	0\\
43.63	0\\
43.64	0\\
43.65	0\\
43.66	0\\
43.67	0\\
43.68	0\\
43.69	0\\
43.7	0\\
43.71	0\\
43.72	0\\
43.73	0\\
43.74	0\\
43.75	0\\
43.76	0\\
43.77	0\\
43.78	0\\
43.79	0\\
43.8	0\\
43.81	0\\
43.82	0\\
43.83	0\\
43.84	0\\
43.85	0\\
43.86	0\\
43.87	0\\
43.88	0\\
43.89	0\\
43.9	0\\
43.91	0\\
43.92	0\\
43.93	0\\
43.94	0\\
43.95	0\\
43.96	0\\
43.97	0\\
43.98	0\\
43.99	0\\
44	0\\
44.01	0\\
44.02	0\\
44.03	0\\
44.04	0\\
44.05	0\\
44.06	0\\
44.07	0\\
44.08	0\\
44.09	0\\
44.1	0\\
44.11	0\\
44.12	0\\
44.13	0\\
44.14	0\\
44.15	0\\
44.16	0\\
44.17	0\\
44.18	0\\
44.19	0\\
44.2	0\\
44.21	0\\
44.22	0\\
44.23	0\\
44.24	0\\
44.25	0\\
44.26	0\\
44.27	0\\
44.28	0\\
44.29	0\\
44.3	0\\
44.31	0\\
44.32	0\\
44.33	0\\
44.34	0\\
44.35	0\\
44.36	0\\
44.37	0\\
44.38	0\\
44.39	0\\
44.4	0\\
44.41	0\\
44.42	0\\
44.43	0\\
44.44	0\\
44.45	0\\
44.46	0\\
44.47	0\\
44.48	0\\
44.49	0\\
44.5	0\\
44.51	0\\
44.52	0\\
44.53	0\\
44.54	0\\
44.55	0\\
44.56	0\\
44.57	0\\
44.58	0\\
44.59	0\\
44.6	0\\
44.61	0\\
44.62	0\\
44.63	0\\
44.64	0\\
44.65	0\\
44.66	0\\
44.67	0\\
44.68	0\\
44.69	0\\
44.7	0\\
44.71	0\\
44.72	0\\
44.73	0\\
44.74	0\\
44.75	0\\
44.76	0\\
44.77	0\\
44.78	0\\
44.79	0\\
44.8	0\\
44.81	0\\
44.82	0\\
44.83	0\\
44.84	0\\
44.85	0\\
44.86	0\\
44.87	0\\
44.88	0\\
44.89	0\\
44.9	0\\
44.91	0\\
44.92	0\\
44.93	0\\
44.94	0\\
44.95	0\\
44.96	0\\
44.97	0\\
44.98	0\\
44.99	0\\
45	0\\
45.01	0\\
45.02	0\\
45.03	0\\
45.04	0\\
45.05	0\\
45.06	0\\
45.07	0\\
45.08	0\\
45.09	0\\
45.1	0\\
45.11	0\\
45.12	0\\
45.13	0\\
45.14	0\\
45.15	0\\
45.16	0\\
45.17	0\\
45.18	0\\
45.19	0\\
45.2	0\\
45.21	0\\
45.22	0\\
45.23	0\\
45.24	0\\
45.25	0\\
45.26	0\\
45.27	0\\
45.28	0\\
45.29	0\\
45.3	0\\
45.31	0\\
45.32	0\\
45.33	0\\
45.34	0\\
45.35	0\\
45.36	0\\
45.37	0\\
45.38	0\\
45.39	0\\
45.4	0\\
45.41	0\\
45.42	0\\
45.43	0\\
45.44	0\\
45.45	0\\
45.46	0\\
45.47	0\\
45.48	0\\
45.49	0\\
45.5	0\\
45.51	0\\
45.52	0\\
45.53	0\\
45.54	0\\
45.55	0\\
45.56	0\\
45.57	0\\
45.58	0\\
45.59	0\\
45.6	0\\
45.61	0\\
45.62	0\\
45.63	0\\
45.64	0\\
45.65	0\\
45.66	0\\
45.67	0\\
45.68	0\\
45.69	0\\
45.7	0\\
45.71	0\\
45.72	0\\
45.73	0\\
45.74	0\\
45.75	0\\
45.76	0\\
45.77	0\\
45.78	0\\
45.79	0\\
45.8	0\\
45.81	0\\
45.82	0\\
45.83	0\\
45.84	0\\
45.85	0\\
45.86	0\\
45.87	0\\
45.88	0\\
45.89	0\\
45.9	0\\
45.91	0\\
45.92	0\\
45.93	0\\
45.94	0\\
45.95	0\\
45.96	0\\
45.97	0\\
45.98	0\\
45.99	0\\
46	0\\
46.01	0\\
46.02	0\\
46.03	0\\
46.04	0\\
46.05	0\\
46.06	0\\
46.07	0\\
46.08	0\\
46.09	0\\
46.1	0\\
46.11	0\\
46.12	0\\
46.13	0\\
46.14	0\\
46.15	0\\
46.16	0\\
46.17	0\\
46.18	0\\
46.19	0\\
46.2	0\\
46.21	0\\
46.22	0\\
46.23	0\\
46.24	0\\
46.25	0\\
46.26	0\\
46.27	0\\
46.28	0\\
46.29	0\\
46.3	0\\
46.31	0\\
46.32	0\\
46.33	0\\
46.34	0\\
46.35	0\\
46.36	0\\
46.37	0\\
46.38	0\\
46.39	0\\
46.4	0\\
46.41	0\\
46.42	0\\
46.43	0\\
46.44	0\\
46.45	0\\
46.46	0\\
46.47	0\\
46.48	0\\
46.49	0\\
46.5	0\\
46.51	0\\
46.52	0\\
46.53	0\\
46.54	0\\
46.55	0\\
46.56	0\\
46.57	0\\
46.58	0\\
46.59	0\\
46.6	0\\
46.61	0\\
46.62	0\\
46.63	0\\
46.64	0\\
46.65	0\\
46.66	0\\
46.67	0\\
46.68	0\\
46.69	0\\
46.7	0\\
46.71	0\\
46.72	0\\
46.73	0\\
46.74	0\\
46.75	0\\
46.76	0\\
46.77	0\\
46.78	0\\
46.79	0\\
46.8	0\\
46.81	0\\
46.82	0\\
46.83	0\\
46.84	0\\
46.85	0\\
46.86	0\\
46.87	0\\
46.88	0\\
46.89	0\\
46.9	0\\
46.91	0\\
46.92	0\\
46.93	0\\
46.94	0\\
46.95	0\\
46.96	0\\
46.97	0\\
46.98	0\\
46.99	0\\
47	0\\
47.01	0\\
47.02	0\\
47.03	0\\
47.04	0\\
47.05	0\\
47.06	0\\
47.07	0\\
47.08	0\\
47.09	0\\
47.1	0\\
47.11	0\\
47.12	0\\
47.13	0\\
47.14	0\\
47.15	0\\
47.16	0\\
47.17	0\\
47.18	0\\
47.19	0\\
47.2	0\\
47.21	0\\
47.22	0\\
47.23	0\\
47.24	0\\
47.25	0\\
47.26	0\\
47.27	0\\
47.28	0\\
47.29	0\\
47.3	0\\
47.31	0\\
47.32	0\\
47.33	0\\
47.34	0\\
47.35	0\\
47.36	0\\
47.37	0\\
47.38	0\\
47.39	0\\
47.4	0\\
47.41	0\\
47.42	0\\
47.43	0\\
47.44	0\\
47.45	0\\
47.46	0\\
47.47	0\\
47.48	0\\
47.49	0\\
47.5	0\\
47.51	0\\
47.52	0\\
47.53	0\\
47.54	0\\
47.55	0\\
47.56	0\\
47.57	0\\
47.58	0\\
47.59	0\\
47.6	0\\
47.61	0\\
47.62	0\\
47.63	0\\
47.64	0\\
47.65	0\\
47.66	0\\
47.67	0\\
47.68	0\\
47.69	0\\
47.7	0\\
47.71	0\\
47.72	0\\
47.73	0\\
47.74	0\\
47.75	0\\
47.76	0\\
47.77	0\\
47.78	0\\
47.79	0\\
47.8	0\\
47.81	0\\
47.82	0\\
47.83	0\\
47.84	0\\
47.85	0\\
47.86	0\\
47.87	0\\
47.88	0\\
47.89	0\\
47.9	0\\
47.91	0\\
47.92	0\\
47.93	0\\
47.94	0\\
47.95	0\\
47.96	0\\
47.97	0\\
47.98	0\\
47.99	0\\
48	0\\
48.01	0\\
48.02	0\\
48.03	0\\
48.04	0\\
48.05	0\\
48.06	0\\
48.07	0\\
48.08	0\\
48.09	0\\
48.1	0\\
48.11	0\\
48.12	0\\
48.13	0\\
48.14	0\\
48.15	0\\
48.16	0\\
48.17	0\\
48.18	0\\
48.19	0\\
48.2	0\\
48.21	0\\
48.22	0\\
48.23	0\\
48.24	0\\
48.25	0\\
48.26	0\\
48.27	0\\
48.28	0\\
48.29	0\\
48.3	0\\
48.31	0\\
48.32	0\\
48.33	0\\
48.34	0\\
48.35	0\\
48.36	0\\
48.37	0\\
48.38	0\\
48.39	0\\
48.4	0\\
48.41	0\\
48.42	0\\
48.43	0\\
48.44	0\\
48.45	0\\
48.46	0\\
48.47	0\\
48.48	0\\
48.49	0\\
48.5	0\\
48.51	0\\
48.52	0\\
48.53	0\\
48.54	0\\
48.55	0\\
48.56	0\\
48.57	0\\
48.58	0\\
48.59	0\\
48.6	0\\
48.61	0\\
48.62	0\\
48.63	0\\
48.64	0\\
48.65	0\\
48.66	0\\
48.67	0\\
48.68	0\\
48.69	0\\
48.7	0\\
48.71	0\\
48.72	0\\
48.73	0\\
48.74	0\\
48.75	0\\
48.76	0\\
48.77	0\\
48.78	0\\
48.79	0\\
48.8	0\\
48.81	0\\
48.82	0\\
48.83	0\\
48.84	0\\
48.85	0\\
48.86	0\\
48.87	0\\
48.88	0\\
48.89	0\\
48.9	0\\
48.91	0\\
48.92	0\\
48.93	0\\
48.94	0\\
48.95	0\\
48.96	0\\
48.97	0\\
48.98	0\\
48.99	0\\
49	0\\
49.01	0\\
49.02	0\\
49.03	0\\
49.04	0\\
49.05	0\\
49.06	0\\
49.07	0\\
49.08	0\\
49.09	0\\
49.1	0\\
49.11	0\\
49.12	0\\
49.13	0\\
49.14	0\\
49.15	0\\
49.16	0\\
49.17	0\\
49.18	0\\
49.19	0\\
49.2	0\\
49.21	0\\
49.22	0\\
49.23	0\\
49.24	0\\
49.25	0\\
49.26	0\\
49.27	0\\
49.28	0\\
49.29	0\\
49.3	0\\
49.31	0\\
49.32	0\\
49.33	0\\
49.34	0\\
49.35	0\\
49.36	0\\
49.37	0\\
49.38	0\\
49.39	0\\
49.4	0\\
49.41	0\\
49.42	0\\
49.43	0\\
49.44	0\\
49.45	0\\
49.46	0\\
49.47	0\\
49.48	0\\
49.49	0\\
49.5	0\\
49.51	0\\
49.52	0\\
49.53	0\\
49.54	0\\
49.55	0\\
49.56	0\\
49.57	0\\
49.58	0\\
49.59	0\\
49.6	0\\
49.61	0\\
49.62	0\\
49.63	0\\
49.64	0\\
49.65	0\\
49.66	0\\
49.67	0\\
49.68	0\\
49.69	0\\
49.7	0\\
49.71	0\\
49.72	0\\
49.73	0\\
49.74	0\\
49.75	0\\
49.76	0\\
49.77	0\\
49.78	0\\
49.79	0\\
49.8	0\\
49.81	0\\
49.82	0\\
49.83	0\\
49.84	0\\
49.85	0\\
49.86	0\\
49.87	0\\
49.88	0\\
49.89	0\\
49.9	0\\
49.91	0\\
49.92	0\\
49.93	0\\
49.94	0\\
49.95	0\\
49.96	0\\
49.97	0\\
49.98	0\\
49.99	0\\
50	0\\
50.01	0\\
50.02	0\\
50.03	0\\
50.04	0\\
50.05	0\\
50.06	0\\
50.07	0\\
50.08	0\\
50.09	0\\
50.1	0\\
50.11	0\\
50.12	0\\
50.13	0\\
50.14	0\\
50.15	0\\
50.16	0\\
50.17	0\\
50.18	0\\
50.19	0\\
50.2	0\\
50.21	0\\
50.22	0\\
50.23	0\\
50.24	0\\
50.25	0\\
50.26	0\\
50.27	0\\
50.28	0\\
50.29	0\\
50.3	0\\
50.31	0\\
50.32	0\\
50.33	0\\
50.34	0\\
50.35	0\\
50.36	0\\
50.37	0\\
50.38	0\\
50.39	0\\
50.4	0\\
50.41	0\\
50.42	0\\
50.43	0\\
50.44	0\\
50.45	0\\
50.46	0\\
50.47	0\\
50.48	0\\
50.49	0\\
50.5	0\\
50.51	0\\
50.52	0\\
50.53	0\\
50.54	0\\
50.55	0\\
50.56	0\\
50.57	0\\
50.58	0\\
50.59	0\\
50.6	0\\
50.61	0\\
50.62	0\\
50.63	0\\
50.64	0\\
50.65	0\\
50.66	0\\
50.67	0\\
50.68	0\\
50.69	0\\
50.7	0\\
50.71	0\\
50.72	0\\
50.73	0\\
50.74	0\\
50.75	0\\
50.76	0\\
50.77	0\\
50.78	0\\
50.79	0\\
50.8	0\\
50.81	0\\
50.82	0\\
50.83	0\\
50.84	0\\
50.85	0\\
50.86	0\\
50.87	0\\
50.88	0\\
50.89	0\\
50.9	0\\
50.91	0\\
50.92	0\\
50.93	0\\
50.94	0\\
50.95	0\\
50.96	0\\
50.97	0\\
50.98	0\\
50.99	0\\
51	0\\
51.01	0\\
51.02	0\\
51.03	0\\
51.04	0\\
51.05	0\\
51.06	0\\
51.07	0\\
51.08	0\\
51.09	0\\
51.1	0\\
51.11	0\\
51.12	0\\
51.13	0\\
51.14	0\\
51.15	0\\
51.16	0\\
51.17	0\\
51.18	0\\
51.19	0\\
51.2	0\\
51.21	0\\
51.22	0\\
51.23	0\\
51.24	0\\
51.25	0\\
51.26	0\\
51.27	0\\
51.28	0\\
51.29	0\\
51.3	0\\
51.31	0\\
51.32	0\\
51.33	0\\
51.34	0\\
51.35	0\\
51.36	0\\
51.37	0\\
51.38	0\\
51.39	0\\
51.4	0\\
51.41	0\\
51.42	0\\
51.43	0\\
51.44	0\\
51.45	0\\
51.46	0\\
51.47	0\\
51.48	0\\
51.49	0\\
51.5	0\\
51.51	0\\
51.52	0\\
51.53	0\\
51.54	0\\
51.55	0\\
51.56	0\\
51.57	0\\
51.58	0\\
51.59	0\\
51.6	0\\
51.61	0\\
51.62	0\\
51.63	0\\
51.64	0\\
51.65	0\\
51.66	0\\
51.67	0\\
51.68	0\\
51.69	0\\
51.7	0\\
51.71	0\\
51.72	0\\
51.73	0\\
51.74	0\\
51.75	0\\
51.76	0\\
51.77	0\\
51.78	0\\
51.79	0\\
51.8	0\\
51.81	0\\
51.82	0\\
51.83	0\\
51.84	0\\
51.85	0\\
51.86	0\\
51.87	0\\
51.88	0\\
51.89	0\\
51.9	0\\
51.91	0\\
51.92	0\\
51.93	0\\
51.94	0\\
51.95	0\\
51.96	0\\
51.97	0\\
51.98	0\\
51.99	0\\
52	0\\
52.01	0\\
52.02	0\\
52.03	0\\
52.04	0\\
52.05	0\\
52.06	0\\
52.07	0\\
52.08	0\\
52.09	0\\
52.1	0\\
52.11	0\\
52.12	0\\
52.13	0\\
52.14	0\\
52.15	0\\
52.16	0\\
52.17	0\\
52.18	0\\
52.19	0\\
52.2	0\\
52.21	0\\
52.22	0\\
52.23	0\\
52.24	0\\
52.25	0\\
52.26	0\\
52.27	0\\
52.28	0\\
52.29	0\\
52.3	0\\
52.31	0\\
52.32	0\\
52.33	0\\
52.34	0\\
52.35	0\\
52.36	0\\
52.37	0\\
52.38	0\\
52.39	0\\
52.4	0\\
52.41	0\\
52.42	0\\
52.43	0\\
52.44	0\\
52.45	0\\
52.46	0\\
52.47	0\\
52.48	0\\
52.49	0\\
52.5	0\\
52.51	0\\
52.52	0\\
52.53	0\\
52.54	0\\
52.55	0\\
52.56	0\\
52.57	0\\
52.58	0\\
52.59	0\\
52.6	0\\
52.61	0\\
52.62	0\\
52.63	0\\
52.64	0\\
52.65	0\\
52.66	0\\
52.67	0\\
52.68	0\\
52.69	0\\
52.7	0\\
52.71	0\\
52.72	0\\
52.73	0\\
52.74	0\\
52.75	0\\
52.76	0\\
52.77	0\\
52.78	0\\
52.79	0\\
52.8	0\\
52.81	0\\
52.82	0\\
52.83	0\\
52.84	0\\
52.85	0\\
52.86	0\\
52.87	0\\
52.88	0\\
52.89	0\\
52.9	0\\
52.91	0\\
52.92	0\\
52.93	0\\
52.94	0\\
52.95	0\\
52.96	0\\
52.97	0\\
52.98	0\\
52.99	0\\
53	0\\
53.01	0\\
53.02	0\\
53.03	0\\
53.04	0\\
53.05	0\\
53.06	0\\
53.07	0\\
53.08	0\\
53.09	0\\
53.1	0\\
53.11	0\\
53.12	0\\
53.13	0\\
53.14	0\\
53.15	0\\
53.16	0\\
53.17	0\\
53.18	0\\
53.19	0\\
53.2	0\\
53.21	0\\
53.22	0\\
53.23	0\\
53.24	0\\
53.25	0\\
53.26	0\\
53.27	0\\
53.28	0\\
53.29	0\\
53.3	0\\
53.31	0\\
53.32	0\\
53.33	0\\
53.34	0\\
53.35	0\\
53.36	0\\
53.37	0\\
53.38	0\\
53.39	0\\
53.4	0\\
53.41	0\\
53.42	0\\
53.43	0\\
53.44	0\\
53.45	0\\
53.46	0\\
53.47	0\\
53.48	0\\
53.49	0\\
53.5	0\\
53.51	0\\
53.52	0\\
53.53	0\\
53.54	0\\
53.55	0\\
53.56	0\\
53.57	0\\
53.58	0\\
53.59	0\\
53.6	0\\
53.61	0\\
53.62	0\\
53.63	0\\
53.64	0\\
53.65	0\\
53.66	0\\
53.67	0\\
53.68	0\\
53.69	0\\
53.7	0\\
53.71	0\\
53.72	0\\
53.73	0\\
53.74	0\\
53.75	0\\
53.76	0\\
53.77	0\\
53.78	0\\
53.79	0\\
53.8	0\\
53.81	0\\
53.82	0\\
53.83	0\\
53.84	0\\
53.85	0\\
53.86	0\\
53.87	0\\
53.88	0\\
53.89	0\\
53.9	0\\
53.91	0\\
53.92	0\\
53.93	0\\
53.94	0\\
53.95	0\\
53.96	0\\
53.97	0\\
53.98	0\\
53.99	0\\
54	0\\
54.01	0\\
54.02	0\\
54.03	0\\
54.04	0\\
54.05	0\\
54.06	0\\
54.07	0\\
54.08	0\\
54.09	0\\
54.1	0\\
54.11	0\\
54.12	0\\
54.13	0\\
54.14	0\\
54.15	0\\
54.16	0\\
54.17	0\\
54.18	0\\
54.19	0\\
54.2	0\\
54.21	0\\
54.22	0\\
54.23	0\\
54.24	0\\
54.25	0\\
54.26	0\\
54.27	0\\
54.28	0\\
54.29	0\\
54.3	0\\
54.31	0\\
54.32	0\\
54.33	0\\
54.34	0\\
54.35	0\\
54.36	0\\
54.37	0\\
54.38	0\\
54.39	0\\
54.4	0\\
54.41	0\\
54.42	0\\
54.43	0\\
54.44	0\\
54.45	0\\
54.46	0\\
54.47	0\\
54.48	0\\
54.49	0\\
54.5	0\\
54.51	0\\
54.52	0\\
54.53	0\\
54.54	0\\
54.55	0\\
54.56	0\\
54.57	0\\
54.58	0\\
54.59	0\\
54.6	0\\
54.61	0\\
54.62	0\\
54.63	0\\
54.64	0\\
54.65	0\\
54.66	0\\
54.67	0\\
54.68	0\\
54.69	0\\
54.7	0\\
54.71	0\\
54.72	0\\
54.73	0\\
54.74	0\\
54.75	0\\
54.76	0\\
54.77	0\\
54.78	0\\
54.79	0\\
54.8	0\\
54.81	0\\
54.82	0\\
54.83	0\\
54.84	0\\
54.85	0\\
54.86	0\\
54.87	0\\
54.88	0\\
54.89	0\\
54.9	0\\
54.91	0\\
54.92	0\\
54.93	0\\
54.94	0\\
54.95	0\\
54.96	0\\
54.97	0\\
54.98	0\\
54.99	0\\
55	0\\
55.01	0\\
55.02	0\\
55.03	0\\
55.04	0\\
55.05	0\\
55.06	0\\
55.07	0\\
55.08	0\\
55.09	0\\
55.1	0\\
55.11	0\\
55.12	0\\
55.13	0\\
55.14	0\\
55.15	0\\
55.16	0\\
55.17	0\\
55.18	0\\
55.19	0\\
55.2	0\\
55.21	0\\
55.22	0\\
55.23	0\\
55.24	0\\
55.25	0\\
55.26	0\\
55.27	0\\
55.28	0\\
55.29	0\\
55.3	0\\
55.31	0\\
55.32	0\\
55.33	0\\
55.34	0\\
55.35	0\\
55.36	0\\
55.37	0\\
55.38	0\\
55.39	0\\
55.4	0\\
55.41	0\\
55.42	0\\
55.43	0\\
55.44	0\\
55.45	0\\
55.46	0\\
55.47	0\\
55.48	0\\
55.49	0\\
55.5	0\\
55.51	0\\
55.52	0\\
55.53	0\\
55.54	0\\
55.55	0\\
55.56	0\\
55.57	0\\
55.58	0\\
55.59	0\\
55.6	0\\
55.61	0\\
55.62	0\\
55.63	0\\
55.64	0\\
55.65	0\\
55.66	0\\
55.67	0\\
55.68	0\\
55.69	0\\
55.7	0\\
55.71	0\\
55.72	0\\
55.73	0\\
55.74	0\\
55.75	0\\
55.76	0\\
55.77	0\\
55.78	0\\
55.79	0\\
55.8	0\\
55.81	0\\
55.82	0\\
55.83	0\\
55.84	0\\
55.85	0\\
55.86	0\\
55.87	0\\
55.88	0\\
55.89	0\\
55.9	0\\
55.91	0\\
55.92	0\\
55.93	0\\
55.94	0\\
55.95	0\\
55.96	0\\
55.97	0\\
55.98	0\\
55.99	0\\
56	0\\
56.01	0\\
56.02	0\\
56.03	0\\
56.04	0\\
56.05	0\\
56.06	0\\
56.07	0\\
56.08	0\\
56.09	0\\
56.1	0\\
56.11	0\\
56.12	0\\
56.13	0\\
56.14	0\\
56.15	0\\
56.16	0\\
56.17	0\\
56.18	0\\
56.19	0\\
56.2	0\\
56.21	0\\
56.22	0\\
56.23	0\\
56.24	0\\
56.25	0\\
56.26	0\\
56.27	0\\
56.28	0\\
56.29	0\\
56.3	0\\
56.31	0\\
56.32	0\\
56.33	0\\
56.34	0\\
56.35	0\\
56.36	0\\
56.37	0\\
56.38	0\\
56.39	0\\
56.4	0\\
56.41	0\\
56.42	0\\
56.43	0\\
56.44	0\\
56.45	0\\
56.46	0\\
56.47	0\\
56.48	0\\
56.49	0\\
56.5	0\\
56.51	0\\
56.52	0\\
56.53	0\\
56.54	0\\
56.55	0\\
56.56	0\\
56.57	0\\
56.58	0\\
56.59	0\\
56.6	0\\
56.61	0\\
56.62	0\\
56.63	0\\
56.64	0\\
56.65	0\\
56.66	0\\
56.67	0\\
56.68	0\\
56.69	0\\
56.7	0\\
56.71	0\\
56.72	0\\
56.73	0\\
56.74	0\\
56.75	0\\
56.76	0\\
56.77	0\\
56.78	0\\
56.79	0\\
56.8	0\\
56.81	0\\
56.82	0\\
56.83	0\\
56.84	0\\
56.85	0\\
56.86	0\\
56.87	0\\
56.88	0\\
56.89	0\\
56.9	0\\
56.91	0\\
56.92	0\\
56.93	0\\
56.94	0\\
56.95	0\\
56.96	0\\
56.97	0\\
56.98	0\\
56.99	0\\
57	0\\
57.01	0\\
57.02	0\\
57.03	0\\
57.04	0\\
57.05	0\\
57.06	0\\
57.07	0\\
57.08	0\\
57.09	0\\
57.1	0\\
57.11	0\\
57.12	0\\
57.13	0\\
57.14	0\\
57.15	0\\
57.16	0\\
57.17	0\\
57.18	0\\
57.19	0\\
57.2	0\\
57.21	0\\
57.22	0\\
57.23	0\\
57.24	0\\
57.25	0\\
57.26	0\\
57.27	0\\
57.28	0\\
57.29	0\\
57.3	0\\
57.31	0\\
57.32	0\\
57.33	0\\
57.34	0\\
57.35	0\\
57.36	0\\
57.37	0\\
57.38	0\\
57.39	0\\
57.4	0\\
57.41	0\\
57.42	0\\
57.43	0\\
57.44	0\\
57.45	0\\
57.46	0\\
57.47	0\\
57.48	0\\
57.49	0\\
57.5	0\\
57.51	0\\
57.52	0\\
57.53	0\\
57.54	0\\
57.55	0\\
57.56	0\\
57.57	0\\
57.58	0\\
57.59	0\\
57.6	0\\
57.61	0\\
57.62	0\\
57.63	0\\
57.64	0\\
57.65	0\\
57.66	0\\
57.67	0\\
57.68	0\\
57.69	0\\
57.7	0\\
57.71	0\\
57.72	0\\
57.73	0\\
57.74	0\\
57.75	0\\
57.76	0\\
57.77	0\\
57.78	0\\
57.79	0\\
57.8	0\\
57.81	0\\
57.82	0\\
57.83	0\\
57.84	0\\
57.85	0\\
57.86	0\\
57.87	0\\
57.88	0\\
57.89	0\\
57.9	0\\
57.91	0\\
57.92	0\\
57.93	0\\
57.94	0\\
57.95	0\\
57.96	0\\
57.97	0\\
57.98	0\\
57.99	0\\
58	0\\
58.01	0\\
58.02	0\\
58.03	0\\
58.04	0\\
58.05	0\\
58.06	0\\
58.07	0\\
58.08	0\\
58.09	0\\
58.1	0\\
58.11	0\\
58.12	0\\
58.13	0\\
58.14	0\\
58.15	0\\
58.16	0\\
58.17	0\\
58.18	0\\
58.19	0\\
58.2	0\\
58.21	0\\
58.22	0\\
58.23	0\\
58.24	0\\
58.25	0\\
58.26	0\\
58.27	0\\
58.28	0\\
58.29	0\\
58.3	0\\
58.31	0\\
58.32	0\\
58.33	0\\
58.34	0\\
58.35	0\\
58.36	0\\
58.37	0\\
58.38	0\\
58.39	0\\
58.4	0\\
58.41	0\\
58.42	0\\
58.43	0\\
58.44	0\\
58.45	0\\
58.46	0\\
58.47	0\\
58.48	0\\
58.49	0\\
58.5	0\\
58.51	0\\
58.52	0\\
58.53	0\\
58.54	0\\
58.55	0\\
58.56	0\\
58.57	0\\
58.58	0\\
58.59	0\\
58.6	0\\
58.61	0\\
58.62	0\\
58.63	0\\
58.64	0\\
58.65	0\\
58.66	0\\
58.67	0\\
58.68	0\\
58.69	0\\
58.7	0\\
58.71	0\\
58.72	0\\
58.73	0\\
58.74	0\\
58.75	0\\
58.76	0\\
58.77	0\\
58.78	0\\
58.79	0\\
58.8	0\\
58.81	0\\
58.82	0\\
58.83	0\\
58.84	0\\
58.85	0\\
58.86	0\\
58.87	0\\
58.88	0\\
58.89	0\\
58.9	0\\
58.91	0\\
58.92	0\\
58.93	0\\
58.94	0\\
58.95	0\\
58.96	0\\
58.97	0\\
58.98	0\\
58.99	0\\
59	0\\
59.01	0\\
59.02	0\\
59.03	0\\
59.04	0\\
59.05	0\\
59.06	0\\
59.07	0\\
59.08	0\\
59.09	0\\
59.1	0\\
59.11	0\\
59.12	0\\
59.13	0\\
59.14	0\\
59.15	0\\
59.16	0\\
59.17	0\\
59.18	0\\
59.19	0\\
59.2	0\\
59.21	0\\
59.22	0\\
59.23	0\\
59.24	0\\
59.25	0\\
59.26	0\\
59.27	0\\
59.28	0\\
59.29	0\\
59.3	0\\
59.31	0\\
59.32	0\\
59.33	0\\
59.34	0\\
59.35	0\\
59.36	0\\
59.37	0\\
59.38	0\\
59.39	0\\
59.4	0\\
59.41	0\\
59.42	0\\
59.43	0\\
59.44	0\\
59.45	0\\
59.46	0\\
59.47	0\\
59.48	0\\
59.49	0\\
59.5	0\\
59.51	0\\
59.52	0\\
59.53	0\\
59.54	0\\
59.55	0\\
59.56	0\\
59.57	0\\
59.58	0\\
59.59	0\\
59.6	0\\
59.61	0\\
59.62	0\\
59.63	0\\
59.64	0\\
59.65	0\\
59.66	0\\
59.67	0\\
59.68	0\\
59.69	0\\
59.7	0\\
59.71	0\\
59.72	0\\
59.73	0\\
59.74	0\\
59.75	0\\
59.76	0\\
59.77	0\\
59.78	0\\
59.79	0\\
59.8	0\\
59.81	0\\
59.82	0\\
59.83	0\\
59.84	0\\
59.85	0\\
59.86	0\\
59.87	0\\
59.88	0\\
59.89	0\\
59.9	0\\
59.91	0\\
59.92	0\\
59.93	0\\
59.94	0\\
59.95	0\\
59.96	0\\
59.97	0\\
59.98	0\\
59.99	0\\
60	0\\
60.01	0\\
60.02	0\\
60.03	0\\
60.04	0\\
60.05	0\\
60.06	0\\
60.07	0\\
60.08	0\\
60.09	0\\
60.1	0\\
60.11	0\\
60.12	0\\
60.13	0\\
60.14	0\\
60.15	0\\
60.16	0\\
60.17	0\\
60.18	0\\
60.19	0\\
60.2	0\\
60.21	0\\
60.22	0\\
60.23	0\\
60.24	0\\
60.25	0\\
60.26	0\\
60.27	0\\
60.28	0\\
60.29	0\\
60.3	0\\
60.31	0\\
60.32	0\\
60.33	0\\
60.34	0\\
60.35	0\\
60.36	0\\
60.37	0\\
60.38	0\\
60.39	0\\
60.4	0\\
60.41	0\\
60.42	0\\
60.43	0\\
60.44	0\\
60.45	0\\
60.46	0\\
60.47	0\\
60.48	0\\
60.49	0\\
60.5	0\\
60.51	0\\
60.52	0\\
60.53	0\\
60.54	0\\
60.55	0\\
60.56	0\\
60.57	0\\
60.58	0\\
60.59	0\\
60.6	0\\
60.61	0\\
60.62	0\\
60.63	0\\
60.64	0\\
60.65	0\\
60.66	0\\
60.67	0\\
60.68	0\\
60.69	0\\
60.7	0\\
60.71	0\\
60.72	0\\
60.73	0\\
60.74	0\\
60.75	0\\
60.76	0\\
60.77	0\\
60.78	0\\
60.79	0\\
60.8	0\\
60.81	0\\
60.82	0\\
60.83	0\\
60.84	0\\
60.85	0\\
60.86	0\\
60.87	0\\
60.88	0\\
60.89	0\\
60.9	0\\
60.91	0\\
60.92	0\\
60.93	0\\
60.94	0\\
60.95	0\\
60.96	0\\
60.97	0\\
60.98	0\\
60.99	0\\
61	0\\
61.01	0\\
61.02	0\\
61.03	0\\
61.04	0\\
61.05	0\\
61.06	0\\
61.07	0\\
61.08	0\\
61.09	0\\
61.1	0\\
61.11	0\\
61.12	0\\
61.13	0\\
61.14	0\\
61.15	0\\
61.16	0\\
61.17	0\\
61.18	0\\
61.19	0\\
61.2	0\\
61.21	0\\
61.22	0\\
61.23	0\\
61.24	0\\
61.25	0\\
61.26	0\\
61.27	0\\
61.28	0\\
61.29	0\\
61.3	0\\
61.31	0\\
61.32	0\\
61.33	0\\
61.34	0\\
61.35	0\\
61.36	0\\
61.37	0\\
61.38	0\\
61.39	0\\
61.4	0\\
61.41	0\\
61.42	0\\
61.43	0\\
61.44	0\\
61.45	0\\
61.46	0\\
61.47	0\\
61.48	0\\
61.49	0\\
61.5	0\\
61.51	0\\
61.52	0\\
61.53	0\\
61.54	0\\
61.55	0\\
61.56	0\\
61.57	0\\
61.58	0\\
61.59	0\\
61.6	0\\
61.61	0\\
61.62	0\\
61.63	0\\
61.64	0\\
61.65	0\\
61.66	0\\
61.67	0\\
61.68	0\\
61.69	0\\
61.7	0\\
61.71	0\\
61.72	0\\
61.73	0\\
61.74	0\\
61.75	0\\
61.76	0\\
61.77	0\\
61.78	0\\
61.79	0\\
61.8	0\\
61.81	0\\
61.82	0\\
61.83	0\\
61.84	0\\
61.85	0\\
61.86	0\\
61.87	0\\
61.88	0\\
61.89	0\\
61.9	0\\
61.91	0\\
61.92	0\\
61.93	0\\
61.94	0\\
61.95	0\\
61.96	0\\
61.97	0\\
61.98	0\\
61.99	0\\
62	0\\
62.01	0\\
62.02	0\\
62.03	0\\
62.04	0\\
62.05	0\\
62.06	0\\
62.07	0\\
62.08	0\\
62.09	0\\
62.1	0\\
62.11	0\\
62.12	0\\
62.13	0\\
62.14	0\\
62.15	0\\
62.16	0\\
62.17	0\\
62.18	0\\
62.19	0\\
62.2	0\\
62.21	0\\
62.22	0\\
62.23	0\\
62.24	0\\
62.25	0\\
62.26	0\\
62.27	0\\
62.28	0\\
62.29	0\\
62.3	0\\
62.31	0\\
62.32	0\\
62.33	0\\
62.34	0\\
62.35	0\\
62.36	0\\
62.37	0\\
62.38	0\\
62.39	0\\
62.4	0\\
62.41	0\\
62.42	0\\
62.43	0\\
62.44	0\\
62.45	0\\
62.46	0\\
62.47	0\\
62.48	0\\
62.49	0\\
62.5	0\\
62.51	0\\
62.52	0\\
62.53	0\\
62.54	0\\
62.55	0\\
62.56	0\\
62.57	0\\
62.58	0\\
62.59	0\\
62.6	0\\
62.61	0\\
62.62	0\\
62.63	0\\
62.64	0\\
62.65	0\\
62.66	0\\
62.67	0\\
62.68	0\\
62.69	0\\
62.7	0\\
62.71	0\\
62.72	0\\
62.73	0\\
62.74	0\\
62.75	0\\
62.76	0\\
62.77	0\\
62.78	0\\
62.79	0\\
62.8	0\\
62.81	0\\
62.82	0\\
62.83	0\\
62.84	0\\
62.85	0\\
62.86	0\\
62.87	0\\
62.88	0\\
62.89	0\\
62.9	0\\
62.91	0\\
62.92	0\\
62.93	0\\
62.94	0\\
62.95	0\\
62.96	0\\
62.97	0\\
62.98	0\\
62.99	0\\
63	0\\
63.01	0\\
63.02	0\\
63.03	0\\
63.04	0\\
63.05	0\\
63.06	0\\
63.07	0\\
63.08	0\\
63.09	0\\
63.1	0\\
63.11	0\\
63.12	0\\
63.13	0\\
63.14	0\\
63.15	0\\
63.16	0\\
63.17	0\\
63.18	0\\
63.19	0\\
63.2	0\\
63.21	0\\
63.22	0\\
63.23	0\\
63.24	0\\
63.25	0\\
63.26	0\\
63.27	0\\
63.28	0\\
63.29	0\\
63.3	0\\
63.31	0\\
63.32	0\\
63.33	0\\
63.34	0\\
63.35	0\\
63.36	0\\
63.37	0\\
63.38	0\\
63.39	0\\
63.4	0\\
63.41	0\\
63.42	0\\
63.43	0\\
63.44	0\\
63.45	0\\
63.46	0\\
63.47	0\\
63.48	0\\
63.49	0\\
63.5	0\\
63.51	0\\
63.52	0\\
63.53	0\\
63.54	0\\
63.55	0\\
63.56	0\\
63.57	0\\
63.58	0\\
63.59	0\\
63.6	0\\
63.61	0\\
63.62	0\\
63.63	0\\
63.64	0\\
63.65	0\\
63.66	0\\
63.67	0\\
63.68	0\\
63.69	0\\
63.7	0\\
63.71	0\\
63.72	0\\
63.73	0\\
63.74	0\\
63.75	0\\
63.76	0\\
63.77	0\\
63.78	0\\
63.79	0\\
63.8	0\\
63.81	0\\
63.82	0\\
63.83	0\\
63.84	0\\
63.85	0\\
63.86	0\\
63.87	0\\
63.88	0\\
63.89	0\\
63.9	0\\
63.91	0\\
63.92	0\\
63.93	0\\
63.94	0\\
63.95	0\\
63.96	0\\
63.97	0\\
63.98	0\\
63.99	0\\
64	0\\
64.01	0\\
64.02	0\\
64.03	0\\
64.04	0\\
64.05	0\\
64.06	0\\
64.07	0\\
64.08	0\\
64.09	0\\
64.1	0\\
64.11	0\\
64.12	0\\
64.13	0\\
64.14	0\\
64.15	0\\
64.16	0\\
64.17	0\\
64.18	0\\
64.19	0\\
64.2	0\\
64.21	0\\
64.22	0\\
64.23	0\\
64.24	0\\
64.25	0\\
64.26	0\\
64.27	0\\
64.28	0\\
64.29	0\\
64.3	0\\
64.31	0\\
64.32	0\\
64.33	0\\
64.34	0\\
64.35	0\\
64.36	0\\
64.37	0\\
64.38	0\\
64.39	0\\
64.4	0\\
64.41	0\\
64.42	0\\
64.43	0\\
64.44	0\\
64.45	0\\
64.46	0\\
64.47	0\\
64.48	0\\
64.49	0\\
64.5	0\\
64.51	0\\
64.52	0\\
64.53	0\\
64.54	0\\
64.55	0\\
64.56	0\\
64.57	0\\
64.58	0\\
64.59	0\\
64.6	0\\
64.61	0\\
64.62	0\\
64.63	0\\
64.64	0\\
64.65	0\\
64.66	0\\
64.67	0\\
64.68	0\\
64.69	0\\
64.7	0\\
64.71	0\\
64.72	0\\
64.73	0\\
64.74	0\\
64.75	0\\
64.76	0\\
64.77	0\\
64.78	0\\
64.79	0\\
64.8	0\\
64.81	0\\
64.82	0\\
64.83	0\\
64.84	0\\
64.85	0\\
64.86	0\\
64.87	0\\
64.88	0\\
64.89	0\\
64.9	0\\
64.91	0\\
64.92	0\\
64.93	0\\
64.94	0\\
64.95	0\\
64.96	0\\
64.97	0\\
64.98	0\\
64.99	0\\
65	0\\
65.01	0\\
65.02	0\\
65.03	0\\
65.04	0\\
65.05	0\\
65.06	0\\
65.07	0\\
65.08	0\\
65.09	0\\
65.1	0\\
65.11	0\\
65.12	0\\
65.13	0\\
65.14	0\\
65.15	0\\
65.16	0\\
65.17	0\\
65.18	0\\
65.19	0\\
65.2	0\\
65.21	0\\
65.22	0\\
65.23	0\\
65.24	0\\
65.25	0\\
65.26	0\\
65.27	0\\
65.28	0\\
65.29	0\\
65.3	0\\
65.31	0\\
65.32	0\\
65.33	0\\
65.34	0\\
65.35	0\\
65.36	0\\
65.37	0\\
65.38	0\\
65.39	0\\
65.4	0\\
65.41	0\\
65.42	0\\
65.43	0\\
65.44	0\\
65.45	0\\
65.46	0\\
65.47	0\\
65.48	0\\
65.49	0\\
65.5	0\\
65.51	0\\
65.52	0\\
65.53	0\\
65.54	0\\
65.55	0\\
65.56	0\\
65.57	0\\
65.58	0\\
65.59	0\\
65.6	0\\
65.61	0\\
65.62	0\\
65.63	0\\
65.64	0\\
65.65	0\\
65.66	0\\
65.67	0\\
65.68	0\\
65.69	0\\
65.7	0\\
65.71	0\\
65.72	0\\
65.73	0\\
65.74	0\\
65.75	0\\
65.76	0\\
65.77	0\\
65.78	0\\
65.79	0\\
65.8	0\\
65.81	0\\
65.82	0\\
65.83	0\\
65.84	0\\
65.85	0\\
65.86	0\\
65.87	0\\
65.88	0\\
65.89	0\\
65.9	0\\
65.91	0\\
65.92	0\\
65.93	0\\
65.94	0\\
65.95	0\\
65.96	0\\
65.97	0\\
65.98	0\\
65.99	0\\
66	0\\
66.01	0\\
66.02	0\\
66.03	0\\
66.04	0\\
66.05	0\\
66.06	0\\
66.07	0\\
66.08	0\\
66.09	0\\
66.1	0\\
66.11	0\\
66.12	0\\
66.13	0\\
66.14	0\\
66.15	0\\
66.16	0\\
66.17	0\\
66.18	0\\
66.19	0\\
66.2	0\\
66.21	0\\
66.22	0\\
66.23	0\\
66.24	0\\
66.25	0\\
66.26	0\\
66.27	0\\
66.28	0\\
66.29	0\\
66.3	0\\
66.31	0\\
66.32	0\\
66.33	0\\
66.34	0\\
66.35	0\\
66.36	0\\
66.37	0\\
66.38	0\\
66.39	0\\
66.4	0\\
66.41	0\\
66.42	0\\
66.43	0\\
66.44	0\\
66.45	0\\
66.46	0\\
66.47	0\\
66.48	0\\
66.49	0\\
66.5	0\\
66.51	0\\
66.52	0\\
66.53	0\\
66.54	0\\
66.55	0\\
66.56	0\\
66.57	0\\
66.58	0\\
66.59	0\\
66.6	0\\
66.61	0\\
66.62	0\\
66.63	0\\
66.64	0\\
66.65	0\\
66.66	0\\
66.67	0\\
66.68	0\\
66.69	0\\
66.7	0\\
66.71	0\\
66.72	0\\
66.73	0\\
66.74	0\\
66.75	0\\
66.76	0\\
66.77	0\\
66.78	0\\
66.79	0\\
66.8	0\\
66.81	0\\
66.82	0\\
66.83	0\\
66.84	0\\
66.85	0\\
66.86	0\\
66.87	0\\
66.88	0\\
66.89	0\\
66.9	0\\
66.91	0\\
66.92	0\\
66.93	0\\
66.94	0\\
66.95	0\\
66.96	0\\
66.97	0\\
66.98	0\\
66.99	0\\
67	0\\
67.01	0\\
67.02	0\\
67.03	0\\
67.04	0\\
67.05	0\\
67.06	0\\
67.07	0\\
67.08	0\\
67.09	0\\
67.1	0\\
67.11	0\\
67.12	0\\
67.13	0\\
67.14	0\\
67.15	0\\
67.16	0\\
67.17	0\\
67.18	0\\
67.19	0\\
67.2	0\\
67.21	0\\
67.22	0\\
67.23	0\\
67.24	0\\
67.25	0\\
67.26	0\\
67.27	0\\
67.28	0\\
67.29	0\\
67.3	0\\
67.31	0\\
67.32	0\\
67.33	0\\
67.34	0\\
67.35	0\\
67.36	0\\
67.37	0\\
67.38	0\\
67.39	0\\
67.4	0\\
67.41	0\\
67.42	0\\
67.43	0\\
67.44	0\\
67.45	0\\
67.46	0\\
67.47	0\\
67.48	0\\
67.49	0\\
67.5	0\\
67.51	0\\
67.52	0\\
67.53	0\\
67.54	0\\
67.55	0\\
67.56	0\\
67.57	0\\
67.58	0\\
67.59	0\\
67.6	0\\
67.61	0\\
67.62	0\\
67.63	0\\
67.64	0\\
67.65	0\\
67.66	0\\
67.67	0\\
67.68	0\\
67.69	0\\
67.7	0\\
67.71	0\\
67.72	0\\
67.73	0\\
67.74	0\\
67.75	0\\
67.76	0\\
67.77	0\\
67.78	0\\
67.79	0\\
67.8	0\\
67.81	0\\
67.82	0\\
67.83	0\\
67.84	0\\
67.85	0\\
67.86	0\\
67.87	0\\
67.88	0\\
67.89	0\\
67.9	0\\
67.91	0\\
67.92	0\\
67.93	0\\
67.94	0\\
67.95	0\\
67.96	0\\
67.97	0\\
67.98	0\\
67.99	0\\
68	0\\
68.01	0\\
68.02	0\\
68.03	0\\
68.04	0\\
68.05	0\\
68.06	0\\
68.07	0\\
68.08	0\\
68.09	0\\
68.1	0\\
68.11	0\\
68.12	0\\
68.13	0\\
68.14	0\\
68.15	0\\
68.16	0\\
68.17	0\\
68.18	0\\
68.19	0\\
68.2	0\\
68.21	0\\
68.22	0\\
68.23	0\\
68.24	0\\
68.25	0\\
68.26	0\\
68.27	0\\
68.28	0\\
68.29	0\\
68.3	0\\
68.31	0\\
68.32	0\\
68.33	0\\
68.34	0\\
68.35	0\\
68.36	0\\
68.37	0\\
68.38	0\\
68.39	0\\
68.4	0\\
68.41	0\\
68.42	0\\
68.43	0\\
68.44	0\\
68.45	0\\
68.46	0\\
68.47	0\\
68.48	0\\
68.49	0\\
68.5	0\\
68.51	0\\
68.52	0\\
68.53	0\\
68.54	0\\
68.55	0\\
68.56	0\\
68.57	0\\
68.58	0\\
68.59	0\\
68.6	0\\
68.61	0\\
68.62	0\\
68.63	0\\
68.64	0\\
68.65	0\\
68.66	0\\
68.67	0\\
68.68	0\\
68.69	0\\
68.7	0\\
68.71	0\\
68.72	0\\
68.73	0\\
68.74	0\\
68.75	0\\
68.76	0\\
68.77	0\\
68.78	0\\
68.79	0\\
68.8	0\\
68.81	0\\
68.82	0\\
68.83	0\\
68.84	0\\
68.85	0\\
68.86	0\\
68.87	0\\
68.88	0\\
68.89	0\\
68.9	0\\
68.91	0\\
68.92	0\\
68.93	0\\
68.94	0\\
68.95	0\\
68.96	0\\
68.97	0\\
68.98	0\\
68.99	0\\
69	0\\
69.01	0\\
69.02	0\\
69.03	0\\
69.04	0\\
69.05	0\\
69.06	0\\
69.07	0\\
69.08	0\\
69.09	0\\
69.1	0\\
69.11	0\\
69.12	0\\
69.13	0\\
69.14	0\\
69.15	0\\
69.16	0\\
69.17	0\\
69.18	0\\
69.19	0\\
69.2	0\\
69.21	0\\
69.22	0\\
69.23	0\\
69.24	0\\
69.25	0\\
69.26	0\\
69.27	0\\
69.28	0\\
69.29	0\\
69.3	0\\
69.31	0\\
69.32	0\\
69.33	0\\
69.34	0\\
69.35	0\\
69.36	0\\
69.37	0\\
69.38	0\\
69.39	0\\
69.4	0\\
69.41	0\\
69.42	0\\
69.43	0\\
69.44	0\\
69.45	0\\
69.46	0\\
69.47	0\\
69.48	0\\
69.49	0\\
69.5	0\\
69.51	0\\
69.52	0\\
69.53	0\\
69.54	0\\
69.55	0\\
69.56	0\\
69.57	0\\
69.58	0\\
69.59	0\\
69.6	0\\
69.61	0\\
69.62	0\\
69.63	0\\
69.64	0\\
69.65	0\\
69.66	0\\
69.67	0\\
69.68	0\\
69.69	0\\
69.7	0\\
69.71	0\\
69.72	0\\
69.73	0\\
69.74	0\\
69.75	0\\
69.76	0\\
69.77	0\\
69.78	0\\
69.79	0\\
69.8	0\\
69.81	0\\
69.82	0\\
69.83	0\\
69.84	0\\
69.85	0\\
69.86	0\\
69.87	0\\
69.88	0\\
69.89	0\\
69.9	0\\
69.91	0\\
69.92	0\\
69.93	0\\
69.94	0\\
69.95	0\\
69.96	0\\
69.97	0\\
69.98	0\\
69.99	0\\
70	0\\
70.01	0\\
70.02	0\\
70.03	0\\
70.04	0\\
70.05	0\\
70.06	0\\
70.07	0\\
70.08	0\\
70.09	0\\
70.1	0\\
70.11	0\\
70.12	0\\
70.13	0\\
70.14	0\\
70.15	0\\
70.16	0\\
70.17	0\\
70.18	0\\
70.19	0\\
70.2	0\\
70.21	0\\
70.22	0\\
70.23	0\\
70.24	0\\
70.25	0\\
70.26	0\\
70.27	0\\
70.28	0\\
70.29	0\\
70.3	0\\
70.31	0\\
70.32	0\\
70.33	0\\
70.34	0\\
70.35	0\\
70.36	0\\
70.37	0\\
70.38	0\\
70.39	0\\
70.4	0\\
70.41	0\\
70.42	0\\
70.43	0\\
70.44	0\\
70.45	0\\
70.46	0\\
70.47	0\\
70.48	0\\
70.49	0\\
70.5	0\\
70.51	0\\
70.52	0\\
70.53	0\\
70.54	0\\
70.55	0\\
70.56	0\\
70.57	0\\
70.58	0\\
70.59	0\\
70.6	0\\
70.61	0\\
70.62	0\\
70.63	0\\
70.64	0\\
70.65	0\\
70.66	0\\
70.67	0\\
70.68	0\\
70.69	0\\
70.7	0\\
70.71	0\\
70.72	0\\
70.73	0\\
70.74	0\\
70.75	0\\
70.76	0\\
70.77	0\\
70.78	0\\
70.79	0\\
70.8	0\\
70.81	0\\
70.82	0\\
70.83	0\\
70.84	0\\
70.85	0\\
70.86	0\\
70.87	0\\
70.88	0\\
70.89	0\\
70.9	0\\
70.91	0\\
70.92	0\\
70.93	0\\
70.94	0\\
70.95	0\\
70.96	0\\
70.97	0\\
70.98	0\\
70.99	0\\
71	0\\
71.01	0\\
71.02	0\\
71.03	0\\
71.04	0\\
71.05	0\\
71.06	0\\
71.07	0\\
71.08	0\\
71.09	0\\
71.1	0\\
71.11	0\\
71.12	0\\
71.13	0\\
71.14	0\\
71.15	0\\
71.16	0\\
71.17	0\\
71.18	0\\
71.19	0\\
71.2	0\\
71.21	0\\
71.22	0\\
71.23	0\\
71.24	0\\
71.25	0\\
71.26	0\\
71.27	0\\
71.28	0\\
71.29	0\\
71.3	0\\
71.31	0\\
71.32	0\\
71.33	0\\
71.34	0\\
71.35	0\\
71.36	0\\
71.37	0\\
71.38	0\\
71.39	0\\
71.4	0\\
71.41	0\\
71.42	0\\
71.43	0\\
71.44	0\\
71.45	0\\
71.46	0\\
71.47	0\\
71.48	0\\
71.49	0\\
71.5	0\\
71.51	0\\
71.52	0\\
71.53	0\\
71.54	0\\
71.55	0\\
71.56	0\\
71.57	0\\
71.58	0\\
71.59	0\\
71.6	0\\
71.61	0\\
71.62	0\\
71.63	0\\
71.64	0\\
71.65	0\\
71.66	0\\
71.67	0\\
71.68	0\\
71.69	0\\
71.7	0\\
71.71	0\\
71.72	0\\
71.73	0\\
71.74	0\\
71.75	0\\
71.76	0\\
71.77	0\\
71.78	0\\
71.79	0\\
71.8	0\\
71.81	0\\
71.82	0\\
71.83	0\\
71.84	0\\
71.85	0\\
71.86	0\\
71.87	0\\
71.88	0\\
71.89	0\\
71.9	0\\
71.91	0\\
71.92	0\\
71.93	0\\
71.94	0\\
71.95	0\\
71.96	0\\
71.97	0\\
71.98	0\\
71.99	0\\
72	0\\
72.01	0\\
72.02	0\\
72.03	0\\
72.04	0\\
72.05	0\\
72.06	0\\
72.07	0\\
72.08	0\\
72.09	0\\
72.1	0\\
72.11	0\\
72.12	0\\
72.13	0\\
72.14	0\\
72.15	0\\
72.16	0\\
72.17	0\\
72.18	0\\
72.19	0\\
72.2	0\\
72.21	0\\
72.22	0\\
72.23	0\\
72.24	0\\
72.25	0\\
72.26	0\\
72.27	0\\
72.28	0\\
72.29	0\\
72.3	0\\
72.31	0\\
72.32	0\\
72.33	0\\
72.34	0\\
72.35	0\\
72.36	0\\
72.37	0\\
72.38	0\\
72.39	0\\
72.4	0\\
72.41	0\\
72.42	0\\
72.43	0\\
72.44	0\\
72.45	0\\
72.46	0\\
72.47	0\\
72.48	0\\
72.49	0\\
72.5	0\\
72.51	0\\
72.52	0\\
72.53	0\\
72.54	0\\
72.55	0\\
72.56	0\\
72.57	0\\
72.58	0\\
72.59	0\\
72.6	0\\
72.61	0\\
72.62	0\\
72.63	0\\
72.64	0\\
72.65	0\\
72.66	0\\
72.67	0\\
72.68	0\\
72.69	0\\
72.7	0\\
72.71	0\\
72.72	0\\
72.73	0\\
72.74	0\\
72.75	0\\
72.76	0\\
72.77	0\\
72.78	0\\
72.79	0\\
72.8	0\\
72.81	0\\
72.82	0\\
72.83	0\\
72.84	0\\
72.85	0\\
72.86	0\\
72.87	0\\
72.88	0\\
72.89	0\\
72.9	0\\
72.91	0\\
72.92	0\\
72.93	0\\
72.94	0\\
72.95	0\\
72.96	0\\
72.97	0\\
72.98	0\\
72.99	0\\
73	0\\
73.01	0\\
73.02	0\\
73.03	0\\
73.04	0\\
73.05	0\\
73.06	0\\
73.07	0\\
73.08	0\\
73.09	0\\
73.1	0\\
73.11	0\\
73.12	0\\
73.13	0\\
73.14	0\\
73.15	0\\
73.16	0\\
73.17	0\\
73.18	0\\
73.19	0\\
73.2	0\\
73.21	0\\
73.22	0\\
73.23	0\\
73.24	0\\
73.25	0\\
73.26	0\\
73.27	0\\
73.28	0\\
73.29	0\\
73.3	0\\
73.31	0\\
73.32	0\\
73.33	0\\
73.34	0\\
73.35	0\\
73.36	0\\
73.37	0\\
73.38	0\\
73.39	0\\
73.4	0\\
73.41	0\\
73.42	0\\
73.43	0\\
73.44	0\\
73.45	0\\
73.46	0\\
73.47	0\\
73.48	0\\
73.49	0\\
73.5	0\\
73.51	0\\
73.52	0\\
73.53	0\\
73.54	0\\
73.55	0\\
73.56	0\\
73.57	0\\
73.58	0\\
73.59	0\\
73.6	0\\
73.61	0\\
73.62	0\\
73.63	0\\
73.64	0\\
73.65	0\\
73.66	0\\
73.67	0\\
73.68	0\\
73.69	0\\
73.7	0\\
73.71	0\\
73.72	0\\
73.73	0\\
73.74	0\\
73.75	0\\
73.76	0\\
73.77	0\\
73.78	0\\
73.79	0\\
73.8	0\\
73.81	0\\
73.82	0\\
73.83	0\\
73.84	0\\
73.85	0\\
73.86	0\\
73.87	0\\
73.88	0\\
73.89	0\\
73.9	0\\
73.91	0\\
73.92	0\\
73.93	0\\
73.94	0\\
73.95	0\\
73.96	0\\
73.97	0\\
73.98	0\\
73.99	0\\
74	0\\
74.01	0\\
74.02	0\\
74.03	0\\
74.04	0\\
74.05	0\\
74.06	0\\
74.07	0\\
74.08	0\\
74.09	0\\
74.1	0\\
74.11	0\\
74.12	0\\
74.13	0\\
74.14	0\\
74.15	0\\
74.16	0\\
74.17	0\\
74.18	0\\
74.19	0\\
74.2	0\\
74.21	0\\
74.22	0\\
74.23	0\\
74.24	0\\
74.25	0\\
74.26	0\\
74.27	0\\
74.28	0\\
74.29	0\\
74.3	0\\
74.31	0\\
74.32	0\\
74.33	0\\
74.34	0\\
74.35	0\\
74.36	0\\
74.37	0\\
74.38	0\\
74.39	0\\
74.4	0\\
74.41	0\\
74.42	0\\
74.43	0\\
74.44	0\\
74.45	0\\
74.46	0\\
74.47	0\\
74.48	0\\
74.49	0\\
74.5	0\\
74.51	0\\
74.52	0\\
74.53	0\\
74.54	0\\
74.55	0\\
74.56	0\\
74.57	0\\
74.58	0\\
74.59	0\\
74.6	0\\
74.61	0\\
74.62	0\\
74.63	0\\
74.64	0\\
74.65	0\\
74.66	0\\
74.67	0\\
74.68	0\\
74.69	0\\
74.7	0\\
74.71	0\\
74.72	0\\
74.73	0\\
74.74	0\\
74.75	0\\
74.76	0\\
74.77	0\\
74.78	0\\
74.79	0\\
74.8	0\\
74.81	0\\
74.82	0\\
74.83	0\\
74.84	0\\
74.85	0\\
74.86	0\\
74.87	0\\
74.88	0\\
74.89	0\\
74.9	0\\
74.91	0\\
74.92	0\\
74.93	0\\
74.94	0\\
74.95	0\\
74.96	0\\
74.97	0\\
74.98	0\\
74.99	0\\
75	0\\
75.01	0\\
75.02	0\\
75.03	0\\
75.04	0\\
75.05	0\\
75.06	0\\
75.07	0\\
75.08	0\\
75.09	0\\
75.1	0\\
75.11	0\\
75.12	0\\
75.13	0\\
75.14	0\\
75.15	0\\
75.16	0\\
75.17	0\\
75.18	0\\
75.19	0\\
75.2	0\\
75.21	0\\
75.22	0\\
75.23	0\\
75.24	0\\
75.25	0\\
75.26	0\\
75.27	0\\
75.28	0\\
75.29	0\\
75.3	0\\
75.31	0\\
75.32	0\\
75.33	0\\
75.34	0\\
75.35	0\\
75.36	0\\
75.37	0\\
75.38	0\\
75.39	0\\
75.4	0\\
75.41	0\\
75.42	0\\
75.43	0\\
75.44	0\\
75.45	0\\
75.46	0\\
75.47	0\\
75.48	0\\
75.49	0\\
75.5	0\\
75.51	0\\
75.52	0\\
75.53	0\\
75.54	0\\
75.55	0\\
75.56	0\\
75.57	0\\
75.58	0\\
75.59	0\\
75.6	0\\
75.61	0\\
75.62	0\\
75.63	0\\
75.64	0\\
75.65	0\\
75.66	0\\
75.67	0\\
75.68	0\\
75.69	0\\
75.7	0\\
75.71	0\\
75.72	0\\
75.73	0\\
75.74	0\\
75.75	0\\
75.76	0\\
75.77	0\\
75.78	0\\
75.79	0\\
75.8	0\\
75.81	0\\
75.82	0\\
75.83	0\\
75.84	0\\
75.85	0\\
75.86	0\\
75.87	0\\
75.88	0\\
75.89	0\\
75.9	0\\
75.91	0\\
75.92	0\\
75.93	0\\
75.94	0\\
75.95	0\\
75.96	0\\
75.97	0\\
75.98	0\\
75.99	0\\
76	0\\
76.01	0\\
76.02	0\\
76.03	0\\
76.04	0\\
76.05	0\\
76.06	0\\
76.07	0\\
76.08	0\\
76.09	0\\
76.1	0\\
76.11	0\\
76.12	0\\
76.13	0\\
76.14	0\\
76.15	0\\
76.16	0\\
76.17	0\\
76.18	0\\
76.19	0\\
76.2	0\\
76.21	0\\
76.22	0\\
76.23	0\\
76.24	0\\
76.25	0\\
76.26	0\\
76.27	0\\
76.28	0\\
76.29	0\\
76.3	0\\
76.31	0\\
76.32	0\\
76.33	0\\
76.34	0\\
76.35	0\\
76.36	0\\
76.37	0\\
76.38	0\\
76.39	0\\
76.4	0\\
76.41	0\\
76.42	0\\
76.43	0\\
76.44	0\\
76.45	0\\
76.46	0\\
76.47	0\\
76.48	0\\
76.49	0\\
76.5	0\\
76.51	0\\
76.52	0\\
76.53	0\\
76.54	0\\
76.55	0\\
76.56	0\\
76.57	0\\
76.58	0\\
76.59	0\\
76.6	0\\
76.61	0\\
76.62	0\\
76.63	0\\
76.64	0\\
76.65	0\\
76.66	0\\
76.67	0\\
76.68	0\\
76.69	0\\
76.7	0\\
76.71	0\\
76.72	0\\
76.73	0\\
76.74	0\\
76.75	0\\
76.76	0\\
76.77	0\\
76.78	0\\
76.79	0\\
76.8	0\\
76.81	0\\
76.82	0\\
76.83	0\\
76.84	0\\
76.85	0\\
76.86	0\\
76.87	0\\
76.88	0\\
76.89	0\\
76.9	0\\
76.91	0\\
76.92	0\\
76.93	0\\
76.94	0\\
76.95	0\\
76.96	0\\
76.97	0\\
76.98	0\\
76.99	0\\
77	0\\
77.01	0\\
77.02	0\\
77.03	0\\
77.04	0\\
77.05	0\\
77.06	0\\
77.07	0\\
77.08	0\\
77.09	0\\
77.1	0\\
77.11	0\\
77.12	0\\
77.13	0\\
77.14	0\\
77.15	0\\
77.16	0\\
77.17	0\\
77.18	0\\
77.19	0\\
77.2	0\\
77.21	0\\
77.22	0\\
77.23	0\\
77.24	0\\
77.25	0\\
77.26	0\\
77.27	0\\
77.28	0\\
77.29	0\\
77.3	0\\
77.31	0\\
77.32	0\\
77.33	0\\
77.34	0\\
77.35	0\\
77.36	0\\
77.37	0\\
77.38	0\\
77.39	0\\
77.4	0\\
77.41	0\\
77.42	0\\
77.43	0\\
77.44	0\\
77.45	0\\
77.46	0\\
77.47	0\\
77.48	0\\
77.49	0\\
77.5	0\\
77.51	0\\
77.52	0\\
77.53	0\\
77.54	0\\
77.55	0\\
77.56	0\\
77.57	0\\
77.58	0\\
77.59	0\\
77.6	0\\
77.61	0\\
77.62	0\\
77.63	0\\
77.64	0\\
77.65	0\\
77.66	0\\
77.67	0\\
77.68	0\\
77.69	0\\
77.7	0\\
77.71	0\\
77.72	0\\
77.73	0\\
77.74	0\\
77.75	0\\
77.76	0\\
77.77	0\\
77.78	0\\
77.79	0\\
77.8	0\\
77.81	0\\
77.82	0\\
77.83	0\\
77.84	0\\
77.85	0\\
77.86	0\\
77.87	0\\
77.88	0\\
77.89	0\\
77.9	0\\
77.91	0\\
77.92	0\\
77.93	0\\
77.94	0\\
77.95	0\\
77.96	0\\
77.97	0\\
77.98	0\\
77.99	0\\
78	0\\
78.01	0\\
78.02	0\\
78.03	0\\
78.04	0\\
78.05	0\\
78.06	0\\
78.07	0\\
78.08	0\\
78.09	0\\
78.1	0\\
78.11	0\\
78.12	0\\
78.13	0\\
78.14	0\\
78.15	0\\
78.16	0\\
78.17	0\\
78.18	0\\
78.19	0\\
78.2	0\\
78.21	0\\
78.22	0\\
78.23	0\\
78.24	0\\
78.25	0\\
78.26	0\\
78.27	0\\
78.28	0\\
78.29	0\\
78.3	0\\
78.31	0\\
78.32	0\\
78.33	0\\
78.34	0\\
78.35	0\\
78.36	0\\
78.37	0\\
78.38	0\\
78.39	0\\
78.4	0\\
78.41	0\\
78.42	0\\
78.43	0\\
78.44	0\\
78.45	0\\
78.46	0\\
78.47	0\\
78.48	0\\
78.49	0\\
78.5	0\\
78.51	0\\
78.52	0\\
78.53	0\\
78.54	0\\
78.55	0\\
78.56	0\\
78.57	0\\
78.58	0\\
78.59	0\\
78.6	0\\
78.61	0\\
78.62	0\\
78.63	0\\
78.64	0\\
78.65	0\\
78.66	0\\
78.67	0\\
78.68	0\\
78.69	0\\
78.7	0\\
78.71	0\\
78.72	0\\
78.73	0\\
78.74	0\\
78.75	0\\
78.76	0\\
78.77	0\\
78.78	0\\
78.79	0\\
78.8	0\\
78.81	0\\
78.82	0\\
78.83	0\\
78.84	0\\
78.85	0\\
78.86	0\\
78.87	0\\
78.88	0\\
78.89	0\\
78.9	0\\
78.91	0\\
78.92	0\\
78.93	0\\
78.94	0\\
78.95	0\\
78.96	0\\
78.97	0\\
78.98	0\\
78.99	0\\
79	0\\
79.01	0\\
79.02	0\\
79.03	0\\
79.04	0\\
79.05	0\\
79.06	0\\
79.07	0\\
79.08	0\\
79.09	0\\
79.1	0\\
79.11	0\\
79.12	0\\
79.13	0\\
79.14	0\\
79.15	0\\
79.16	0\\
79.17	0\\
79.18	0\\
79.19	0\\
79.2	0\\
79.21	0\\
79.22	0\\
79.23	0\\
79.24	0\\
79.25	0\\
79.26	0\\
79.27	0\\
79.28	0\\
79.29	0\\
79.3	0\\
79.31	0\\
79.32	0\\
79.33	0\\
79.34	0\\
79.35	0\\
79.36	0\\
79.37	0\\
79.38	0\\
79.39	0\\
79.4	0\\
79.41	0\\
79.42	0\\
79.43	0\\
79.44	0\\
79.45	0\\
79.46	0\\
79.47	0\\
79.48	0\\
79.49	0\\
79.5	0\\
79.51	0\\
79.52	0\\
79.53	0\\
79.54	0\\
79.55	0\\
79.56	0\\
79.57	0\\
79.58	0\\
79.59	0\\
79.6	0\\
79.61	0\\
79.62	0\\
79.63	0\\
79.64	0\\
79.65	0\\
79.66	0\\
79.67	0\\
79.68	0\\
79.69	0\\
79.7	0\\
79.71	0\\
79.72	0\\
79.73	0\\
79.74	0\\
79.75	0\\
79.76	0\\
79.77	0\\
79.78	0\\
79.79	0\\
79.8	0\\
79.81	0\\
79.82	0\\
79.83	0\\
79.84	0\\
79.85	0\\
79.86	0\\
79.87	0\\
79.88	0\\
79.89	0\\
79.9	0\\
79.91	0\\
79.92	0\\
79.93	0\\
79.94	0\\
79.95	0\\
79.96	0\\
79.97	0\\
79.98	0\\
79.99	0\\
80	0\\
80.01	0\\
};
\addplot [color=mycolor1,solid]
  table[row sep=crcr]{%
80.01	0\\
80.02	0\\
80.03	0\\
80.04	0\\
80.05	0\\
80.06	0\\
80.07	0\\
80.08	0\\
80.09	0\\
80.1	0\\
80.11	0\\
80.12	0\\
80.13	0\\
80.14	0\\
80.15	0\\
80.16	0\\
80.17	0\\
80.18	0\\
80.19	0\\
80.2	0\\
80.21	0\\
80.22	0\\
80.23	0\\
80.24	0\\
80.25	0\\
80.26	0\\
80.27	0\\
80.28	0\\
80.29	0\\
80.3	0\\
80.31	0\\
80.32	0\\
80.33	0\\
80.34	0\\
80.35	0\\
80.36	0\\
80.37	0\\
80.38	0\\
80.39	0\\
80.4	0\\
80.41	0\\
80.42	0\\
80.43	0\\
80.44	0\\
80.45	0\\
80.46	0\\
80.47	0\\
80.48	0\\
80.49	0\\
80.5	0\\
80.51	0\\
80.52	0\\
80.53	0\\
80.54	0\\
80.55	0\\
80.56	0\\
80.57	0\\
80.58	0\\
80.59	0\\
80.6	0\\
80.61	0\\
80.62	0\\
80.63	0\\
80.64	0\\
80.65	0\\
80.66	0\\
80.67	0\\
80.68	0\\
80.69	0\\
80.7	0\\
80.71	0\\
80.72	0\\
80.73	0\\
80.74	0\\
80.75	0\\
80.76	0\\
80.77	0\\
80.78	0\\
80.79	0\\
80.8	0\\
80.81	0\\
80.82	0\\
80.83	0\\
80.84	0\\
80.85	0\\
80.86	0\\
80.87	0\\
80.88	0\\
80.89	0\\
80.9	0\\
80.91	0\\
80.92	0\\
80.93	0\\
80.94	0\\
80.95	0\\
80.96	0\\
80.97	0\\
80.98	0\\
80.99	0\\
81	0\\
81.01	0\\
81.02	0\\
81.03	0\\
81.04	0\\
81.05	0\\
81.06	0\\
81.07	0\\
81.08	0\\
81.09	0\\
81.1	0\\
81.11	0\\
81.12	0\\
81.13	0\\
81.14	0\\
81.15	0\\
81.16	0\\
81.17	0\\
81.18	0\\
81.19	0\\
81.2	0\\
81.21	0\\
81.22	0\\
81.23	0\\
81.24	0\\
81.25	0\\
81.26	0\\
81.27	0\\
81.28	0\\
81.29	0\\
81.3	0\\
81.31	0\\
81.32	0\\
81.33	0\\
81.34	0\\
81.35	0\\
81.36	0\\
81.37	0\\
81.38	0\\
81.39	0\\
81.4	0\\
81.41	0\\
81.42	0\\
81.43	0\\
81.44	0\\
81.45	0\\
81.46	0\\
81.47	0\\
81.48	0\\
81.49	0\\
81.5	0\\
81.51	0\\
81.52	0\\
81.53	0\\
81.54	0\\
81.55	0\\
81.56	0\\
81.57	0\\
81.58	0\\
81.59	0\\
81.6	0\\
81.61	0\\
81.62	0\\
81.63	0\\
81.64	0\\
81.65	0\\
81.66	0\\
81.67	0\\
81.68	0\\
81.69	0\\
81.7	0\\
81.71	0\\
81.72	0\\
81.73	0\\
81.74	0\\
81.75	0\\
81.76	0\\
81.77	0\\
81.78	0\\
81.79	0\\
81.8	0\\
81.81	0\\
81.82	0\\
81.83	0\\
81.84	0\\
81.85	0\\
81.86	0\\
81.87	0\\
81.88	0\\
81.89	0\\
81.9	0\\
81.91	0\\
81.92	0\\
81.93	0\\
81.94	0\\
81.95	0\\
81.96	0\\
81.97	0\\
81.98	0\\
81.99	0\\
82	0\\
82.01	0\\
82.02	0\\
82.03	0\\
82.04	0\\
82.05	0\\
82.06	0\\
82.07	0\\
82.08	0\\
82.09	0\\
82.1	0\\
82.11	0\\
82.12	0\\
82.13	0\\
82.14	0\\
82.15	0\\
82.16	0\\
82.17	0\\
82.18	0\\
82.19	0\\
82.2	0\\
82.21	0\\
82.22	0\\
82.23	0\\
82.24	0\\
82.25	0\\
82.26	0\\
82.27	0\\
82.28	0\\
82.29	0\\
82.3	0\\
82.31	0\\
82.32	0\\
82.33	0\\
82.34	0\\
82.35	0\\
82.36	0\\
82.37	0\\
82.38	0\\
82.39	0\\
82.4	0\\
82.41	0\\
82.42	0\\
82.43	0\\
82.44	0\\
82.45	0\\
82.46	0\\
82.47	0\\
82.48	0\\
82.49	0\\
82.5	0\\
82.51	0\\
82.52	0\\
82.53	0\\
82.54	0\\
82.55	0\\
82.56	0\\
82.57	0\\
82.58	0\\
82.59	0\\
82.6	0\\
82.61	0\\
82.62	0\\
82.63	0\\
82.64	0\\
82.65	0\\
82.66	0\\
82.67	0\\
82.68	0\\
82.69	0\\
82.7	0\\
82.71	0\\
82.72	0\\
82.73	0\\
82.74	0\\
82.75	0\\
82.76	0\\
82.77	0\\
82.78	0\\
82.79	0\\
82.8	0\\
82.81	0\\
82.82	0\\
82.83	0\\
82.84	0\\
82.85	0\\
82.86	0\\
82.87	0\\
82.88	0\\
82.89	0\\
82.9	0\\
82.91	0\\
82.92	0\\
82.93	0\\
82.94	0\\
82.95	0\\
82.96	0\\
82.97	0\\
82.98	0\\
82.99	0\\
83	0\\
83.01	0\\
83.02	0\\
83.03	0\\
83.04	0\\
83.05	0\\
83.06	0\\
83.07	0\\
83.08	0\\
83.09	0\\
83.1	0\\
83.11	0\\
83.12	0\\
83.13	0\\
83.14	0\\
83.15	0\\
83.16	0\\
83.17	0\\
83.18	0\\
83.19	0\\
83.2	0\\
83.21	0\\
83.22	0\\
83.23	0\\
83.24	0\\
83.25	0\\
83.26	0\\
83.27	0\\
83.28	0\\
83.29	0\\
83.3	0\\
83.31	0\\
83.32	0\\
83.33	0\\
83.34	0\\
83.35	0\\
83.36	0\\
83.37	0\\
83.38	0\\
83.39	0\\
83.4	0\\
83.41	0\\
83.42	0\\
83.43	0\\
83.44	0\\
83.45	0\\
83.46	0\\
83.47	0\\
83.48	0\\
83.49	0\\
83.5	0\\
83.51	0\\
83.52	0\\
83.53	0\\
83.54	0\\
83.55	0\\
83.56	0\\
83.57	0\\
83.58	0\\
83.59	0\\
83.6	0\\
83.61	0\\
83.62	0\\
83.63	0\\
83.64	0\\
83.65	0\\
83.66	0\\
83.67	0\\
83.68	0\\
83.69	0\\
83.7	0\\
83.71	0\\
83.72	0\\
83.73	0\\
83.74	0\\
83.75	0\\
83.76	0\\
83.77	0\\
83.78	0\\
83.79	0\\
83.8	0\\
83.81	0\\
83.82	0\\
83.83	0\\
83.84	0\\
83.85	0\\
83.86	0\\
83.87	0\\
83.88	0\\
83.89	0\\
83.9	0\\
83.91	0\\
83.92	0\\
83.93	0\\
83.94	0\\
83.95	0\\
83.96	0\\
83.97	0\\
83.98	0\\
83.99	0\\
84	0\\
84.01	0\\
84.02	0\\
84.03	0\\
84.04	0\\
84.05	0\\
84.06	0\\
84.07	0\\
84.08	0\\
84.09	0\\
84.1	0\\
84.11	0\\
84.12	0\\
84.13	0\\
84.14	0\\
84.15	0\\
84.16	0\\
84.17	0\\
84.18	0\\
84.19	0\\
84.2	0\\
84.21	0\\
84.22	0\\
84.23	0\\
84.24	0\\
84.25	0\\
84.26	0\\
84.27	0\\
84.28	0\\
84.29	0\\
84.3	0\\
84.31	0\\
84.32	0\\
84.33	0\\
84.34	0\\
84.35	0\\
84.36	0\\
84.37	0\\
84.38	0\\
84.39	0\\
84.4	0\\
84.41	0\\
84.42	0\\
84.43	0\\
84.44	0\\
84.45	0\\
84.46	0\\
84.47	0\\
84.48	0\\
84.49	0\\
84.5	0\\
84.51	0\\
84.52	0\\
84.53	0\\
84.54	0\\
84.55	0\\
84.56	0\\
84.57	0\\
84.58	0\\
84.59	0\\
84.6	0\\
84.61	0\\
84.62	0\\
84.63	0\\
84.64	0\\
84.65	0\\
84.66	0\\
84.67	0\\
84.68	0\\
84.69	0\\
84.7	0\\
84.71	0\\
84.72	0\\
84.73	0\\
84.74	0\\
84.75	0\\
84.76	0\\
84.77	0\\
84.78	0\\
84.79	0\\
84.8	0\\
84.81	0\\
84.82	0\\
84.83	0\\
84.84	0\\
84.85	0\\
84.86	0\\
84.87	0\\
84.88	0\\
84.89	0\\
84.9	0\\
84.91	0\\
84.92	0\\
84.93	0\\
84.94	0\\
84.95	0\\
84.96	0\\
84.97	0\\
84.98	0\\
84.99	0\\
85	0\\
85.01	0\\
85.02	0\\
85.03	0\\
85.04	0\\
85.05	0\\
85.06	0\\
85.07	0\\
85.08	0\\
85.09	0\\
85.1	0\\
85.11	0\\
85.12	0\\
85.13	0\\
85.14	0\\
85.15	0\\
85.16	0\\
85.17	0\\
85.18	0\\
85.19	0\\
85.2	0\\
85.21	0\\
85.22	0\\
85.23	0\\
85.24	0\\
85.25	0\\
85.26	0\\
85.27	0\\
85.28	0\\
85.29	0\\
85.3	0\\
85.31	0\\
85.32	0\\
85.33	0\\
85.34	0\\
85.35	0\\
85.36	0\\
85.37	0\\
85.38	0\\
85.39	0\\
85.4	0\\
85.41	0\\
85.42	0\\
85.43	0\\
85.44	0\\
85.45	0\\
85.46	0\\
85.47	0\\
85.48	0\\
85.49	0\\
85.5	0\\
85.51	0\\
85.52	0\\
85.53	0\\
85.54	0\\
85.55	0\\
85.56	0\\
85.57	0\\
85.58	0\\
85.59	0\\
85.6	0\\
85.61	0\\
85.62	0\\
85.63	0\\
85.64	0\\
85.65	0\\
85.66	0\\
85.67	0\\
85.68	0\\
85.69	0\\
85.7	0\\
85.71	0\\
85.72	0\\
85.73	0\\
85.74	0\\
85.75	0\\
85.76	0\\
85.77	0\\
85.78	0\\
85.79	0\\
85.8	0\\
85.81	0\\
85.82	0\\
85.83	0\\
85.84	0\\
85.85	0\\
85.86	0\\
85.87	0\\
85.88	0\\
85.89	0\\
85.9	0\\
85.91	0\\
85.92	0\\
85.93	0\\
85.94	0\\
85.95	0\\
85.96	0\\
85.97	0\\
85.98	0\\
85.99	0\\
86	0\\
86.01	0\\
86.02	0\\
86.03	0\\
86.04	0\\
86.05	0\\
86.06	0\\
86.07	0\\
86.08	0\\
86.09	0\\
86.1	0\\
86.11	0\\
86.12	0\\
86.13	0\\
86.14	0\\
86.15	0\\
86.16	0\\
86.17	0\\
86.18	0\\
86.19	0\\
86.2	0\\
86.21	0\\
86.22	0\\
86.23	0\\
86.24	0\\
86.25	0\\
86.26	0\\
86.27	0\\
86.28	0\\
86.29	0\\
86.3	0\\
86.31	0\\
86.32	0\\
86.33	0\\
86.34	0\\
86.35	0\\
86.36	0\\
86.37	0\\
86.38	0\\
86.39	0\\
86.4	0\\
86.41	0\\
86.42	0\\
86.43	0\\
86.44	0\\
86.45	0\\
86.46	0\\
86.47	0\\
86.48	0\\
86.49	0\\
86.5	0\\
86.51	0\\
86.52	0\\
86.53	0\\
86.54	0\\
86.55	0\\
86.56	0\\
86.57	0\\
86.58	0\\
86.59	0\\
86.6	0\\
86.61	0\\
86.62	0\\
86.63	0\\
86.64	0\\
86.65	0\\
86.66	0\\
86.67	0\\
86.68	0\\
86.69	0\\
86.7	0\\
86.71	0\\
86.72	0\\
86.73	0\\
86.74	0\\
86.75	0\\
86.76	0\\
86.77	0\\
86.78	0\\
86.79	0\\
86.8	0\\
86.81	0\\
86.82	0\\
86.83	0\\
86.84	0\\
86.85	0\\
86.86	0\\
86.87	0\\
86.88	0\\
86.89	0\\
86.9	0\\
86.91	0\\
86.92	0\\
86.93	0\\
86.94	0\\
86.95	0\\
86.96	0\\
86.97	0\\
86.98	0\\
86.99	0\\
87	0\\
87.01	0\\
87.02	0\\
87.03	0\\
87.04	0\\
87.05	0\\
87.06	0\\
87.07	0\\
87.08	0\\
87.09	0\\
87.1	0\\
87.11	0\\
87.12	0\\
87.13	0\\
87.14	0\\
87.15	0\\
87.16	0\\
87.17	0\\
87.18	0\\
87.19	0\\
87.2	0\\
87.21	0\\
87.22	0\\
87.23	0\\
87.24	0\\
87.25	0\\
87.26	0\\
87.27	0\\
87.28	0\\
87.29	0\\
87.3	0\\
87.31	0\\
87.32	0\\
87.33	0\\
87.34	0\\
87.35	0\\
87.36	0\\
87.37	0\\
87.38	0\\
87.39	0\\
87.4	0\\
87.41	0\\
87.42	0\\
87.43	0\\
87.44	0\\
87.45	0\\
87.46	0\\
87.47	0\\
87.48	0\\
87.49	0\\
87.5	0\\
87.51	0\\
87.52	0\\
87.53	0\\
87.54	0\\
87.55	0\\
87.56	0\\
87.57	0\\
87.58	0\\
87.59	0\\
87.6	0\\
87.61	0\\
87.62	0\\
87.63	0\\
87.64	0\\
87.65	0\\
87.66	0\\
87.67	0\\
87.68	0\\
87.69	0\\
87.7	0\\
87.71	0\\
87.72	0\\
87.73	0\\
87.74	0\\
87.75	0\\
87.76	0\\
87.77	0\\
87.78	0\\
87.79	0\\
87.8	0\\
87.81	0\\
87.82	0\\
87.83	0\\
87.84	0\\
87.85	0\\
87.86	0\\
87.87	0\\
87.88	0\\
87.89	0\\
87.9	0\\
87.91	0\\
87.92	0\\
87.93	0\\
87.94	0\\
87.95	0\\
87.96	0\\
87.97	0\\
87.98	0\\
87.99	0\\
88	0\\
88.01	0\\
88.02	0\\
88.03	0\\
88.04	0\\
88.05	0\\
88.06	0\\
88.07	0\\
88.08	0\\
88.09	0\\
88.1	0\\
88.11	0\\
88.12	0\\
88.13	0\\
88.14	0\\
88.15	0\\
88.16	0\\
88.17	0\\
88.18	0\\
88.19	0\\
88.2	0\\
88.21	0\\
88.22	0\\
88.23	0\\
88.24	0\\
88.25	0\\
88.26	0\\
88.27	0\\
88.28	0\\
88.29	0\\
88.3	0\\
88.31	0\\
88.32	0\\
88.33	0\\
88.34	0\\
88.35	0\\
88.36	0\\
88.37	0\\
88.38	0\\
88.39	0\\
88.4	0\\
88.41	0\\
88.42	0\\
88.43	0\\
88.44	0\\
88.45	0\\
88.46	0\\
88.47	0\\
88.48	0\\
88.49	0\\
88.5	0\\
88.51	0\\
88.52	0\\
88.53	0\\
88.54	0\\
88.55	0\\
88.56	0\\
88.57	0\\
88.58	0\\
88.59	0\\
88.6	0\\
88.61	0\\
88.62	0\\
88.63	0\\
88.64	0\\
88.65	0\\
88.66	0\\
88.67	0\\
88.68	0\\
88.69	0\\
88.7	0\\
88.71	0\\
88.72	0\\
88.73	0\\
88.74	0\\
88.75	0\\
88.76	0\\
88.77	0\\
88.78	0\\
88.79	0\\
88.8	0\\
88.81	0\\
88.82	0\\
88.83	0\\
88.84	0\\
88.85	0\\
88.86	0\\
88.87	0\\
88.88	0\\
88.89	0\\
88.9	0\\
88.91	0\\
88.92	0\\
88.93	0\\
88.94	0\\
88.95	0\\
88.96	0\\
88.97	0\\
88.98	0\\
88.99	0\\
89	0\\
89.01	0\\
89.02	0\\
89.03	0\\
89.04	0\\
89.05	0\\
89.06	0\\
89.07	0\\
89.08	0\\
89.09	0\\
89.1	0\\
89.11	0\\
89.12	0\\
89.13	0\\
89.14	0\\
89.15	0\\
89.16	0\\
89.17	0\\
89.18	0\\
89.19	0\\
89.2	0\\
89.21	0\\
89.22	0\\
89.23	0\\
89.24	0\\
89.25	0\\
89.26	0\\
89.27	0\\
89.28	0\\
89.29	0\\
89.3	0\\
89.31	0\\
89.32	0\\
89.33	0\\
89.34	0\\
89.35	0\\
89.36	0\\
89.37	0\\
89.38	0\\
89.39	0\\
89.4	0\\
89.41	0\\
89.42	0\\
89.43	0\\
89.44	0\\
89.45	0\\
89.46	0\\
89.47	0\\
89.48	0\\
89.49	0\\
89.5	0\\
89.51	0\\
89.52	0\\
89.53	0\\
89.54	0\\
89.55	0\\
89.56	0\\
89.57	0\\
89.58	0\\
89.59	0\\
89.6	0\\
89.61	0\\
89.62	0\\
89.63	0\\
89.64	0\\
89.65	0\\
89.66	0\\
89.67	0\\
89.68	0\\
89.69	0\\
89.7	0\\
89.71	0\\
89.72	0\\
89.73	0\\
89.74	0\\
89.75	0\\
89.76	0\\
89.77	0\\
89.78	0\\
89.79	0\\
89.8	0\\
89.81	0\\
89.82	0\\
89.83	0\\
89.84	0\\
89.85	0\\
89.86	0\\
89.87	0\\
89.88	0\\
89.89	0\\
89.9	0\\
89.91	0\\
89.92	0\\
89.93	0\\
89.94	0\\
89.95	0\\
89.96	0\\
89.97	0\\
89.98	0\\
89.99	0\\
90	0\\
90.01	0\\
90.02	0\\
90.03	0\\
90.04	0\\
90.05	0\\
90.06	0\\
90.07	0\\
90.08	0\\
90.09	0\\
90.1	0\\
90.11	0\\
90.12	0\\
90.13	0\\
90.14	0\\
90.15	0\\
90.16	0\\
90.17	0\\
90.18	0\\
90.19	0\\
90.2	0\\
90.21	0\\
90.22	0\\
90.23	0\\
90.24	0\\
90.25	0\\
90.26	0\\
90.27	0\\
90.28	0\\
90.29	0\\
90.3	0\\
90.31	0\\
90.32	0\\
90.33	0\\
90.34	0\\
90.35	0\\
90.36	0\\
90.37	0\\
90.38	0\\
90.39	0\\
90.4	0\\
90.41	0\\
90.42	0\\
90.43	0\\
90.44	0\\
90.45	0\\
90.46	0\\
90.47	0\\
90.48	0\\
90.49	0\\
90.5	0\\
90.51	0\\
90.52	0\\
90.53	0\\
90.54	0\\
90.55	0\\
90.56	0\\
90.57	0\\
90.58	0\\
90.59	0\\
90.6	0\\
90.61	0\\
90.62	0\\
90.63	0\\
90.64	0\\
90.65	0\\
90.66	0\\
90.67	0\\
90.68	0\\
90.69	0\\
90.7	0\\
90.71	0\\
90.72	0\\
90.73	0\\
90.74	0\\
90.75	0\\
90.76	0\\
90.77	0\\
90.78	0\\
90.79	0\\
90.8	0\\
90.81	0\\
90.82	0\\
90.83	0\\
90.84	0\\
90.85	0\\
90.86	0\\
90.87	0\\
90.88	0\\
90.89	0\\
90.9	0\\
90.91	0\\
90.92	0\\
90.93	0\\
90.94	0\\
90.95	0\\
90.96	0\\
90.97	0\\
90.98	0\\
90.99	0\\
91	0\\
91.01	0\\
91.02	0\\
91.03	0\\
91.04	0\\
91.05	0\\
91.06	0\\
91.07	0\\
91.08	0\\
91.09	0\\
91.1	0\\
91.11	0\\
91.12	0\\
91.13	0\\
91.14	0\\
91.15	0\\
91.16	0\\
91.17	0\\
91.18	0\\
91.19	0\\
91.2	0\\
91.21	0\\
91.22	0\\
91.23	0\\
91.24	0\\
91.25	0\\
91.26	0\\
91.27	0\\
91.28	0\\
91.29	0\\
91.3	0\\
91.31	0\\
91.32	0\\
91.33	0\\
91.34	0\\
91.35	0\\
91.36	0\\
91.37	0\\
91.38	0\\
91.39	0\\
91.4	0\\
91.41	0\\
91.42	0\\
91.43	0\\
91.44	0\\
91.45	0\\
91.46	0\\
91.47	0\\
91.48	0\\
91.49	0\\
91.5	0\\
91.51	0\\
91.52	0\\
91.53	0\\
91.54	0\\
91.55	0\\
91.56	0\\
91.57	0\\
91.58	0\\
91.59	0\\
91.6	0\\
91.61	0\\
91.62	0\\
91.63	0\\
91.64	0\\
91.65	0\\
91.66	0\\
91.67	0\\
91.68	0\\
91.69	0\\
91.7	0\\
91.71	0\\
91.72	0\\
91.73	0\\
91.74	0\\
91.75	0\\
91.76	0\\
91.77	0\\
91.78	0\\
91.79	0\\
91.8	0\\
91.81	0\\
91.82	0\\
91.83	0\\
91.84	0\\
91.85	0\\
91.86	0\\
91.87	0\\
91.88	0\\
91.89	0\\
91.9	0\\
91.91	0\\
91.92	0\\
91.93	0\\
91.94	0\\
91.95	0\\
91.96	0\\
91.97	0\\
91.98	0\\
91.99	0\\
92	0\\
92.01	0\\
92.02	0\\
92.03	0\\
92.04	0\\
92.05	0\\
92.06	0\\
92.07	0\\
92.08	0\\
92.09	0\\
92.1	0\\
92.11	0\\
92.12	0\\
92.13	0\\
92.14	0\\
92.15	0\\
92.16	0\\
92.17	0\\
92.18	0\\
92.19	0\\
92.2	0\\
92.21	0\\
92.22	0\\
92.23	0\\
92.24	0\\
92.25	0\\
92.26	0\\
92.27	0\\
92.28	0\\
92.29	0\\
92.3	0\\
92.31	0\\
92.32	0\\
92.33	0\\
92.34	0\\
92.35	0\\
92.36	0\\
92.37	0\\
92.38	0\\
92.39	0\\
92.4	0\\
92.41	0\\
92.42	0\\
92.43	0\\
92.44	0\\
92.45	0\\
92.46	0\\
92.47	0\\
92.48	0\\
92.49	0\\
92.5	0\\
92.51	0\\
92.52	0\\
92.53	0\\
92.54	0\\
92.55	0\\
92.56	0\\
92.57	0\\
92.58	0\\
92.59	0\\
92.6	0\\
92.61	0\\
92.62	0\\
92.63	0\\
92.64	0\\
92.65	0\\
92.66	0\\
92.67	0\\
92.68	0\\
92.69	0\\
92.7	0\\
92.71	0\\
92.72	0\\
92.73	0\\
92.74	0\\
92.75	0\\
92.76	0\\
92.77	0\\
92.78	0\\
92.79	0\\
92.8	0\\
92.81	0\\
92.82	0\\
92.83	0\\
92.84	0\\
92.85	0\\
92.86	0\\
92.87	0\\
92.88	0\\
92.89	0\\
92.9	0\\
92.91	0\\
92.92	0\\
92.93	0\\
92.94	0\\
92.95	0\\
92.96	0\\
92.97	0\\
92.98	0\\
92.99	0\\
93	0\\
93.01	0\\
93.02	0\\
93.03	0\\
93.04	0\\
93.05	0\\
93.06	0\\
93.07	0\\
93.08	0\\
93.09	0\\
93.1	0\\
93.11	0\\
93.12	0\\
93.13	0\\
93.14	0\\
93.15	0\\
93.16	0\\
93.17	0\\
93.18	0\\
93.19	0\\
93.2	0\\
93.21	0\\
93.22	0\\
93.23	0\\
93.24	0\\
93.25	0\\
93.26	0\\
93.27	0\\
93.28	0\\
93.29	0\\
93.3	0\\
93.31	0\\
93.32	0\\
93.33	0\\
93.34	0\\
93.35	0\\
93.36	0\\
93.37	0\\
93.38	0\\
93.39	0\\
93.4	0\\
93.41	0\\
93.42	0\\
93.43	0\\
93.44	0\\
93.45	0\\
93.46	0\\
93.47	0\\
93.48	0\\
93.49	0\\
93.5	0\\
93.51	0\\
93.52	0\\
93.53	0\\
93.54	0\\
93.55	0\\
93.56	0\\
93.57	0\\
93.58	0\\
93.59	0\\
93.6	0\\
93.61	0\\
93.62	0\\
93.63	0\\
93.64	0\\
93.65	0\\
93.66	0\\
93.67	0\\
93.68	0\\
93.69	0\\
93.7	0\\
93.71	0\\
93.72	0\\
93.73	0\\
93.74	0\\
93.75	0\\
93.76	0\\
93.77	0\\
93.78	0\\
93.79	0\\
93.8	0\\
93.81	0\\
93.82	0\\
93.83	0\\
93.84	0\\
93.85	0\\
93.86	0\\
93.87	0\\
93.88	0\\
93.89	0\\
93.9	0\\
93.91	0\\
93.92	0\\
93.93	0\\
93.94	0\\
93.95	0\\
93.96	0\\
93.97	0\\
93.98	0\\
93.99	0\\
94	0\\
94.01	0\\
94.02	0\\
94.03	0\\
94.04	0\\
94.05	0\\
94.06	0\\
94.07	0\\
94.08	0\\
94.09	0\\
94.1	0\\
94.11	0\\
94.12	0\\
94.13	0\\
94.14	0\\
94.15	0\\
94.16	0\\
94.17	0\\
94.18	0\\
94.19	0\\
94.2	0\\
94.21	0\\
94.22	0\\
94.23	0\\
94.24	0\\
94.25	0\\
94.26	0\\
94.27	0\\
94.28	0\\
94.29	0\\
94.3	0\\
94.31	0\\
94.32	0\\
94.33	0\\
94.34	0\\
94.35	0\\
94.36	0\\
94.37	0\\
94.38	0\\
94.39	0\\
94.4	0\\
94.41	0\\
94.42	0\\
94.43	0\\
94.44	0\\
94.45	0\\
94.46	0\\
94.47	0\\
94.48	0\\
94.49	0\\
94.5	0\\
94.51	0\\
94.52	0\\
94.53	0\\
94.54	0\\
94.55	0\\
94.56	0\\
94.57	0\\
94.58	0\\
94.59	0\\
94.6	0\\
94.61	0\\
94.62	0\\
94.63	0\\
94.64	0\\
94.65	0\\
94.66	0\\
94.67	0\\
94.68	0\\
94.69	0\\
94.7	0\\
94.71	0\\
94.72	0\\
94.73	0\\
94.74	0\\
94.75	0\\
94.76	0\\
94.77	0\\
94.78	0\\
94.79	0\\
94.8	0\\
94.81	0\\
94.82	0\\
94.83	0\\
94.84	0\\
94.85	0\\
94.86	0\\
94.87	0\\
94.88	0\\
94.89	0\\
94.9	0\\
94.91	0\\
94.92	0\\
94.93	0\\
94.94	0\\
94.95	0\\
94.96	0\\
94.97	0\\
94.98	0\\
94.99	0\\
95	0\\
95.01	0\\
95.02	0\\
95.03	0\\
95.04	0\\
95.05	0\\
95.06	0\\
95.07	0\\
95.08	0\\
95.09	0\\
95.1	0\\
95.11	0\\
95.12	0\\
95.13	0\\
95.14	0\\
95.15	0\\
95.16	0\\
95.17	0\\
95.18	0\\
95.19	0\\
95.2	0\\
95.21	0\\
95.22	0\\
95.23	0\\
95.24	0\\
95.25	0\\
95.26	0\\
95.27	0\\
95.28	0\\
95.29	0\\
95.3	0\\
95.31	0\\
95.32	0\\
95.33	0\\
95.34	0\\
95.35	0\\
95.36	0\\
95.37	0\\
95.38	0\\
95.39	0\\
95.4	0\\
95.41	0\\
95.42	0\\
95.43	0\\
95.44	0\\
95.45	0\\
95.46	0\\
95.47	0\\
95.48	0\\
95.49	0\\
95.5	0\\
95.51	0\\
95.52	0\\
95.53	0\\
95.54	0\\
95.55	0\\
95.56	0\\
95.57	0\\
95.58	0\\
95.59	0\\
95.6	0\\
95.61	0\\
95.62	0\\
95.63	0\\
95.64	0\\
95.65	0\\
95.66	0\\
95.67	0\\
95.68	0\\
95.69	0\\
95.7	0\\
95.71	0\\
95.72	0\\
95.73	0\\
95.74	0\\
95.75	0\\
95.76	0\\
95.77	0\\
95.78	0\\
95.79	0\\
95.8	0\\
95.81	0\\
95.82	0\\
95.83	0\\
95.84	0\\
95.85	0\\
95.86	0\\
95.87	0\\
95.88	0\\
95.89	0\\
95.9	0\\
95.91	0\\
95.92	0\\
95.93	0\\
95.94	0\\
95.95	0\\
95.96	0\\
95.97	0\\
95.98	0\\
95.99	0\\
96	0\\
96.01	0\\
96.02	0\\
96.03	0\\
96.04	0\\
96.05	0\\
96.06	0\\
96.07	0\\
96.08	0\\
96.09	0\\
96.1	0\\
96.11	0\\
96.12	0\\
96.13	0\\
96.14	0\\
96.15	0\\
96.16	0\\
96.17	0\\
96.18	0\\
96.19	0\\
96.2	0\\
96.21	0\\
96.22	0\\
96.23	0\\
96.24	0\\
96.25	0\\
96.26	0\\
96.27	0\\
96.28	0\\
96.29	0\\
96.3	0\\
96.31	0\\
96.32	0\\
96.33	0\\
96.34	0\\
96.35	0\\
96.36	0\\
96.37	0\\
96.38	0\\
96.39	0\\
96.4	0\\
96.41	0\\
96.42	0\\
96.43	0\\
96.44	0\\
96.45	0\\
96.46	0\\
96.47	0\\
96.48	0\\
96.49	0\\
96.5	0\\
96.51	0\\
96.52	0\\
96.53	0\\
96.54	0\\
96.55	0\\
96.56	0\\
96.57	0\\
96.58	0\\
96.59	0\\
96.6	0\\
96.61	0\\
96.62	0\\
96.63	0\\
96.64	0\\
96.65	0\\
96.66	0\\
96.67	0\\
96.68	0\\
96.69	0\\
96.7	0\\
96.71	0\\
96.72	0\\
96.73	0\\
96.74	0\\
96.75	0\\
96.76	0\\
96.77	0\\
96.78	0\\
96.79	0\\
96.8	0\\
96.81	0\\
96.82	0\\
96.83	0\\
96.84	0\\
96.85	0\\
96.86	0\\
96.87	0\\
96.88	0\\
96.89	0\\
96.9	0\\
96.91	0\\
96.92	0\\
96.93	0\\
96.94	0\\
96.95	0\\
96.96	0\\
96.97	0\\
96.98	0\\
96.99	0\\
97	0\\
97.01	0\\
97.02	0\\
97.03	0\\
97.04	0\\
97.05	0\\
97.06	0\\
97.07	0\\
97.08	0\\
97.09	0\\
97.1	0\\
97.11	0\\
97.12	0\\
97.13	0\\
97.14	0\\
97.15	0\\
97.16	0\\
97.17	0\\
97.18	0\\
97.19	0\\
97.2	0\\
97.21	0\\
97.22	0\\
97.23	0\\
97.24	0\\
97.25	0\\
97.26	0\\
97.27	0\\
97.28	0\\
97.29	0\\
97.3	0\\
97.31	0\\
97.32	0\\
97.33	0\\
97.34	0\\
97.35	0\\
97.36	0\\
97.37	0\\
97.38	0\\
97.39	0\\
97.4	0\\
97.41	0\\
97.42	0\\
97.43	0\\
97.44	0\\
97.45	0\\
97.46	0\\
97.47	0\\
97.48	0\\
97.49	0\\
97.5	0\\
97.51	0\\
97.52	0\\
97.53	0\\
97.54	0\\
97.55	0\\
97.56	0\\
97.57	0\\
97.58	0\\
97.59	0\\
97.6	0\\
97.61	0\\
97.62	0\\
97.63	0\\
97.64	0\\
97.65	0\\
97.66	0\\
97.67	0\\
97.68	0\\
97.69	0\\
97.7	0\\
97.71	0\\
97.72	0\\
97.73	0\\
97.74	0\\
97.75	0\\
97.76	0\\
97.77	0\\
97.78	0\\
97.79	0\\
97.8	0\\
97.81	0\\
97.82	0\\
97.83	0\\
97.84	0\\
97.85	0\\
97.86	0\\
97.87	0\\
97.88	0\\
97.89	0\\
97.9	0\\
97.91	0\\
97.92	0\\
97.93	0\\
97.94	0\\
97.95	0\\
97.96	0\\
97.97	0\\
97.98	0\\
97.99	0\\
98	0\\
98.01	0\\
98.02	0\\
98.03	0\\
98.04	0\\
98.05	0\\
98.06	0\\
98.07	3.32142083815171e-05\\
98.08	0.000117592884186023\\
98.09	0.000202637737334657\\
98.1	0.000288355406196129\\
98.11	0.000374752604359582\\
98.12	0.000461836121707214\\
98.13	0.000549612825508049\\
98.14	0.000638089661533441\\
98.15	0.000727273655194927\\
98.16	0.000817171912705021\\
98.17	0.000907791622261615\\
98.18	0.000999140055256649\\
98.19	0.00103249012756317\\
98.2	0.00105230944648016\\
98.21	0.00107227526507901\\
98.22	0.0010924000963525\\
98.23	0.00111268506241714\\
98.24	0.00113313128697955\\
98.25	0.00115373989512096\\
98.26	0.00117451201307303\\
98.27	0.00119544876798465\\
98.28	0.00121655128767936\\
98.29	0.00123782070040311\\
98.3	0.00125925813243572\\
98.31	0.00128086470942302\\
98.32	0.00130264155650186\\
98.33	0.00132458979780865\\
98.34	0.00134671055428548\\
98.35	0.00136900494534457\\
98.36	0.00139147408853117\\
98.37	0.00141414498729552\\
98.38	0.00143702049814452\\
98.39	0.00146010259331748\\
98.4	0.00148339326447661\\
98.41	0.00150689452291022\\
98.42	0.00153060839973829\\
98.43	0.00155453693442337\\
98.44	0.00157868218253603\\
98.45	0.00160304621947003\\
98.46	0.00162763114063745\\
98.47	0.00165243906166219\\
98.48	0.00167747211858005\\
98.49	0.00170273246804082\\
98.5	0.00172822228751266\\
98.51	0.00175394377548859\\
98.52	0.0017798991516953\\
98.53	0.00180609065730417\\
98.54	0.00183252055531797\\
98.55	0.00185919113216576\\
98.56	0.00188610469640969\\
98.57	0.00191326357896613\\
98.58	0.00194067013332923\\
98.59	0.00196832673579699\\
98.6	0.00199623578569987\\
98.61	0.00202439970563186\\
98.62	0.00205282094168419\\
98.63	0.00208150196368166\\
98.64	0.00211044526542164\\
98.65	0.00213965336491573\\
98.66	0.00216912880354172\\
98.67	0.00219887414697771\\
98.68	0.0022288919858204\\
98.69	0.00225918492775749\\
98.7	0.00228975560288559\\
98.71	0.00232060666640964\\
98.72	0.00235174079888165\\
98.73	0.00238316070644164\\
98.74	0.0024148691210655\\
98.75	0.00244686880081364\\
98.76	0.00247916253007931\\
98.77	0.00251175311983919\\
98.78	0.00254464340790631\\
98.79	0.00257783625919663\\
98.8	0.00261133456599068\\
98.81	0.00264514124819473\\
98.82	0.00267925925360437\\
98.83	0.00271369155817346\\
98.84	0.00274844114466484\\
98.85	0.0027835109979611\\
98.86	0.00281890413102645\\
98.87	0.00285462358517075\\
98.88	0.00289067243031597\\
98.89	0.00292705376526521\\
98.9	0.00296377071797415\\
98.91	0.0030008264458252\\
98.92	0.00303822413590406\\
98.93	0.00307596700527906\\
98.94	0.0031140583012829\\
98.95	0.00315250130179727\\
98.96	0.00319129931554\\
98.97	0.00323045568235492\\
98.98	0.00326997377350452\\
98.99	0.00330985699196527\\
99	0.00335010877272577\\
99.01	0.00339073258308769\\
99.02	0.00343173192296956\\
99.03	0.00347311032521334\\
99.04	0.00351487135589395\\
99.05	0.0035570186146317\\
99.06	0.00359955573490761\\
99.07	0.00364248638438174\\
99.08	0.00368581426521446\\
99.09	0.00372954311439079\\
99.1	0.00377367670404771\\
99.11	0.00381821884180458\\
99.12	0.0038631733710967\\
99.13	0.00390854417151193\\
99.14	0.00395433515913047\\
99.15	0.00400055028686791\\
99.16	0.00404719354482142\\
99.17	0.00409426896061924\\
99.18	0.00414178059977342\\
99.19	0.00418973256603592\\
99.2	0.00423812900175794\\
99.21	0.00428697408825284\\
99.22	0.0043362720461622\\
99.23	0.00438602713582556\\
99.24	0.00443624365765346\\
99.25	0.00448692595250408\\
99.26	0.00453807840206335\\
99.27	0.00458970542922874\\
99.28	0.0046418114984965\\
99.29	0.00469440111635265\\
99.3	0.00474747883166762\\
99.31	0.00480104923609458\\
99.32	0.00485511696447148\\
99.33	0.00490968669522694\\
99.34	0.0049647631507899\\
99.35	0.00502035109800311\\
99.36	0.00507645534854054\\
99.37	0.0051330807593287\\
99.38	0.00519023223297186\\
99.39	0.00524791471818133\\
99.4	0.00530613321020878\\
99.41	0.00536489275128353\\
99.42	0.00542419843105407\\
99.43	0.00548405538703363\\
99.44	0.00554446880504997\\
99.45	0.00560544391969941\\
99.46	0.00566698601480507\\
99.47	0.00572910042387948\\
99.48	0.00579179253059146\\
99.49	0.00585506776923741\\
99.5	0.00591893162521708\\
99.51	0.00598338963551369\\
99.52	0.00604844738917865\\
99.53	0.00611411052782078\\
99.54	0.00618038474610015\\
99.55	0.00624727579222653\\
99.56	0.0063147894684625\\
99.57	0.00638293163163133\\
99.58	0.00645170819362957\\
99.59	0.0065211251219445\\
99.6	0.00659118844017634\\
99.61	0.00666190420524143\\
99.62	0.00673327852119868\\
99.63	0.00680531754918415\\
99.64	0.00687802750794424\\
99.65	0.00695141467437383\\
99.66	0.00702548538405953\\
99.67	0.00710024603182786\\
99.68	0.00717570307229865\\
99.69	0.00725186302044349\\
99.7	0.00732873245214951\\
99.71	0.00740631800478823\\
99.72	0.0074846263777898\\
99.73	0.00756366433322253\\
99.74	0.00764343869637786\\
99.75	0.00772395635636067\\
99.76	0.00780522426668515\\
99.77	0.00788724944587618\\
99.78	0.00797003897807629\\
99.79	0.0080536000136582\\
99.8	0.00813793976984314\\
99.81	0.00822306553132476\\
99.82	0.00830898465089893\\
99.83	0.00839570455009931\\
99.84	0.00848323271983882\\
99.85	0.00857157672105694\\
99.86	0.00866074418537317\\
99.87	0.00875074281574629\\
99.88	0.00884158038713987\\
99.89	0.0089332647471938\\
99.9	0.009025803816902\\
99.91	0.00911920559129637\\
99.92	0.00921347814013702\\
99.93	0.00930862960860877\\
99.94	0.00940466821802401\\
99.95	0.00950160226653202\\
99.96	0.0095994401298347\\
99.97	0.00969819026190877\\
99.98	0.0097978611957346\\
99.99	0.00989846154403157\\
100	0.01\\
};
\addlegendentry{$q=3$};

\addplot [color=green,solid,forget plot]
  table[row sep=crcr]{%
0.01	0\\
0.02	0\\
0.03	0\\
0.04	0\\
0.05	0\\
0.06	0\\
0.07	0\\
0.08	0\\
0.09	0\\
0.1	0\\
0.11	0\\
0.12	0\\
0.13	0\\
0.14	0\\
0.15	0\\
0.16	0\\
0.17	0\\
0.18	0\\
0.19	0\\
0.2	0\\
0.21	0\\
0.22	0\\
0.23	0\\
0.24	0\\
0.25	0\\
0.26	0\\
0.27	0\\
0.28	0\\
0.29	0\\
0.3	0\\
0.31	0\\
0.32	0\\
0.33	0\\
0.34	0\\
0.35	0\\
0.36	0\\
0.37	0\\
0.38	0\\
0.39	0\\
0.4	0\\
0.41	0\\
0.42	0\\
0.43	0\\
0.44	0\\
0.45	0\\
0.46	0\\
0.47	0\\
0.48	0\\
0.49	0\\
0.5	0\\
0.51	0\\
0.52	0\\
0.53	0\\
0.54	0\\
0.55	0\\
0.56	0\\
0.57	0\\
0.58	0\\
0.59	0\\
0.6	0\\
0.61	0\\
0.62	0\\
0.63	0\\
0.64	0\\
0.65	0\\
0.66	0\\
0.67	0\\
0.68	0\\
0.69	0\\
0.7	0\\
0.71	0\\
0.72	0\\
0.73	0\\
0.74	0\\
0.75	0\\
0.76	0\\
0.77	0\\
0.78	0\\
0.79	0\\
0.8	0\\
0.81	0\\
0.82	0\\
0.83	0\\
0.84	0\\
0.85	0\\
0.86	0\\
0.87	0\\
0.88	0\\
0.89	0\\
0.9	0\\
0.91	0\\
0.92	0\\
0.93	0\\
0.94	0\\
0.95	0\\
0.96	0\\
0.97	0\\
0.98	0\\
0.99	0\\
1	0\\
1.01	0\\
1.02	0\\
1.03	0\\
1.04	0\\
1.05	0\\
1.06	0\\
1.07	0\\
1.08	0\\
1.09	0\\
1.1	0\\
1.11	0\\
1.12	0\\
1.13	0\\
1.14	0\\
1.15	0\\
1.16	0\\
1.17	0\\
1.18	0\\
1.19	0\\
1.2	0\\
1.21	0\\
1.22	0\\
1.23	0\\
1.24	0\\
1.25	0\\
1.26	0\\
1.27	0\\
1.28	0\\
1.29	0\\
1.3	0\\
1.31	0\\
1.32	0\\
1.33	0\\
1.34	0\\
1.35	0\\
1.36	0\\
1.37	0\\
1.38	0\\
1.39	0\\
1.4	0\\
1.41	0\\
1.42	0\\
1.43	0\\
1.44	0\\
1.45	0\\
1.46	0\\
1.47	0\\
1.48	0\\
1.49	0\\
1.5	0\\
1.51	0\\
1.52	0\\
1.53	0\\
1.54	0\\
1.55	0\\
1.56	0\\
1.57	0\\
1.58	0\\
1.59	0\\
1.6	0\\
1.61	0\\
1.62	0\\
1.63	0\\
1.64	0\\
1.65	0\\
1.66	0\\
1.67	0\\
1.68	0\\
1.69	0\\
1.7	0\\
1.71	0\\
1.72	0\\
1.73	0\\
1.74	0\\
1.75	0\\
1.76	0\\
1.77	0\\
1.78	0\\
1.79	0\\
1.8	0\\
1.81	0\\
1.82	0\\
1.83	0\\
1.84	0\\
1.85	0\\
1.86	0\\
1.87	0\\
1.88	0\\
1.89	0\\
1.9	0\\
1.91	0\\
1.92	0\\
1.93	0\\
1.94	0\\
1.95	0\\
1.96	0\\
1.97	0\\
1.98	0\\
1.99	0\\
2	0\\
2.01	0\\
2.02	0\\
2.03	0\\
2.04	0\\
2.05	0\\
2.06	0\\
2.07	0\\
2.08	0\\
2.09	0\\
2.1	0\\
2.11	0\\
2.12	0\\
2.13	0\\
2.14	0\\
2.15	0\\
2.16	0\\
2.17	0\\
2.18	0\\
2.19	0\\
2.2	0\\
2.21	0\\
2.22	0\\
2.23	0\\
2.24	0\\
2.25	0\\
2.26	0\\
2.27	0\\
2.28	0\\
2.29	0\\
2.3	0\\
2.31	0\\
2.32	0\\
2.33	0\\
2.34	0\\
2.35	0\\
2.36	0\\
2.37	0\\
2.38	0\\
2.39	0\\
2.4	0\\
2.41	0\\
2.42	0\\
2.43	0\\
2.44	0\\
2.45	0\\
2.46	0\\
2.47	0\\
2.48	0\\
2.49	0\\
2.5	0\\
2.51	0\\
2.52	0\\
2.53	0\\
2.54	0\\
2.55	0\\
2.56	0\\
2.57	0\\
2.58	0\\
2.59	0\\
2.6	0\\
2.61	0\\
2.62	0\\
2.63	0\\
2.64	0\\
2.65	0\\
2.66	0\\
2.67	0\\
2.68	0\\
2.69	0\\
2.7	0\\
2.71	0\\
2.72	0\\
2.73	0\\
2.74	0\\
2.75	0\\
2.76	0\\
2.77	0\\
2.78	0\\
2.79	0\\
2.8	0\\
2.81	0\\
2.82	0\\
2.83	0\\
2.84	0\\
2.85	0\\
2.86	0\\
2.87	0\\
2.88	0\\
2.89	0\\
2.9	0\\
2.91	0\\
2.92	0\\
2.93	0\\
2.94	0\\
2.95	0\\
2.96	0\\
2.97	0\\
2.98	0\\
2.99	0\\
3	0\\
3.01	0\\
3.02	0\\
3.03	0\\
3.04	0\\
3.05	0\\
3.06	0\\
3.07	0\\
3.08	0\\
3.09	0\\
3.1	0\\
3.11	0\\
3.12	0\\
3.13	0\\
3.14	0\\
3.15	0\\
3.16	0\\
3.17	0\\
3.18	0\\
3.19	0\\
3.2	0\\
3.21	0\\
3.22	0\\
3.23	0\\
3.24	0\\
3.25	0\\
3.26	0\\
3.27	0\\
3.28	0\\
3.29	0\\
3.3	0\\
3.31	0\\
3.32	0\\
3.33	0\\
3.34	0\\
3.35	0\\
3.36	0\\
3.37	0\\
3.38	0\\
3.39	0\\
3.4	0\\
3.41	0\\
3.42	0\\
3.43	0\\
3.44	0\\
3.45	0\\
3.46	0\\
3.47	0\\
3.48	0\\
3.49	0\\
3.5	0\\
3.51	0\\
3.52	0\\
3.53	0\\
3.54	0\\
3.55	0\\
3.56	0\\
3.57	0\\
3.58	0\\
3.59	0\\
3.6	0\\
3.61	0\\
3.62	0\\
3.63	0\\
3.64	0\\
3.65	0\\
3.66	0\\
3.67	0\\
3.68	0\\
3.69	0\\
3.7	0\\
3.71	0\\
3.72	0\\
3.73	0\\
3.74	0\\
3.75	0\\
3.76	0\\
3.77	0\\
3.78	0\\
3.79	0\\
3.8	0\\
3.81	0\\
3.82	0\\
3.83	0\\
3.84	0\\
3.85	0\\
3.86	0\\
3.87	0\\
3.88	0\\
3.89	0\\
3.9	0\\
3.91	0\\
3.92	0\\
3.93	0\\
3.94	0\\
3.95	0\\
3.96	0\\
3.97	0\\
3.98	0\\
3.99	0\\
4	0\\
4.01	0\\
4.02	0\\
4.03	0\\
4.04	0\\
4.05	0\\
4.06	0\\
4.07	0\\
4.08	0\\
4.09	0\\
4.1	0\\
4.11	0\\
4.12	0\\
4.13	0\\
4.14	0\\
4.15	0\\
4.16	0\\
4.17	0\\
4.18	0\\
4.19	0\\
4.2	0\\
4.21	0\\
4.22	0\\
4.23	0\\
4.24	0\\
4.25	0\\
4.26	0\\
4.27	0\\
4.28	0\\
4.29	0\\
4.3	0\\
4.31	0\\
4.32	0\\
4.33	0\\
4.34	0\\
4.35	0\\
4.36	0\\
4.37	0\\
4.38	0\\
4.39	0\\
4.4	0\\
4.41	0\\
4.42	0\\
4.43	0\\
4.44	0\\
4.45	0\\
4.46	0\\
4.47	0\\
4.48	0\\
4.49	0\\
4.5	0\\
4.51	0\\
4.52	0\\
4.53	0\\
4.54	0\\
4.55	0\\
4.56	0\\
4.57	0\\
4.58	0\\
4.59	0\\
4.6	0\\
4.61	0\\
4.62	0\\
4.63	0\\
4.64	0\\
4.65	0\\
4.66	0\\
4.67	0\\
4.68	0\\
4.69	0\\
4.7	0\\
4.71	0\\
4.72	0\\
4.73	0\\
4.74	0\\
4.75	0\\
4.76	0\\
4.77	0\\
4.78	0\\
4.79	0\\
4.8	0\\
4.81	0\\
4.82	0\\
4.83	0\\
4.84	0\\
4.85	0\\
4.86	0\\
4.87	0\\
4.88	0\\
4.89	0\\
4.9	0\\
4.91	0\\
4.92	0\\
4.93	0\\
4.94	0\\
4.95	0\\
4.96	0\\
4.97	0\\
4.98	0\\
4.99	0\\
5	0\\
5.01	0\\
5.02	0\\
5.03	0\\
5.04	0\\
5.05	0\\
5.06	0\\
5.07	0\\
5.08	0\\
5.09	0\\
5.1	0\\
5.11	0\\
5.12	0\\
5.13	0\\
5.14	0\\
5.15	0\\
5.16	0\\
5.17	0\\
5.18	0\\
5.19	0\\
5.2	0\\
5.21	0\\
5.22	0\\
5.23	0\\
5.24	0\\
5.25	0\\
5.26	0\\
5.27	0\\
5.28	0\\
5.29	0\\
5.3	0\\
5.31	0\\
5.32	0\\
5.33	0\\
5.34	0\\
5.35	0\\
5.36	0\\
5.37	0\\
5.38	0\\
5.39	0\\
5.4	0\\
5.41	0\\
5.42	0\\
5.43	0\\
5.44	0\\
5.45	0\\
5.46	0\\
5.47	0\\
5.48	0\\
5.49	0\\
5.5	0\\
5.51	0\\
5.52	0\\
5.53	0\\
5.54	0\\
5.55	0\\
5.56	0\\
5.57	0\\
5.58	0\\
5.59	0\\
5.6	0\\
5.61	0\\
5.62	0\\
5.63	0\\
5.64	0\\
5.65	0\\
5.66	0\\
5.67	0\\
5.68	0\\
5.69	0\\
5.7	0\\
5.71	0\\
5.72	0\\
5.73	0\\
5.74	0\\
5.75	0\\
5.76	0\\
5.77	0\\
5.78	0\\
5.79	0\\
5.8	0\\
5.81	0\\
5.82	0\\
5.83	0\\
5.84	0\\
5.85	0\\
5.86	0\\
5.87	0\\
5.88	0\\
5.89	0\\
5.9	0\\
5.91	0\\
5.92	0\\
5.93	0\\
5.94	0\\
5.95	0\\
5.96	0\\
5.97	0\\
5.98	0\\
5.99	0\\
6	0\\
6.01	0\\
6.02	0\\
6.03	0\\
6.04	0\\
6.05	0\\
6.06	0\\
6.07	0\\
6.08	0\\
6.09	0\\
6.1	0\\
6.11	0\\
6.12	0\\
6.13	0\\
6.14	0\\
6.15	0\\
6.16	0\\
6.17	0\\
6.18	0\\
6.19	0\\
6.2	0\\
6.21	0\\
6.22	0\\
6.23	0\\
6.24	0\\
6.25	0\\
6.26	0\\
6.27	0\\
6.28	0\\
6.29	0\\
6.3	0\\
6.31	0\\
6.32	0\\
6.33	0\\
6.34	0\\
6.35	0\\
6.36	0\\
6.37	0\\
6.38	0\\
6.39	0\\
6.4	0\\
6.41	0\\
6.42	0\\
6.43	0\\
6.44	0\\
6.45	0\\
6.46	0\\
6.47	0\\
6.48	0\\
6.49	0\\
6.5	0\\
6.51	0\\
6.52	0\\
6.53	0\\
6.54	0\\
6.55	0\\
6.56	0\\
6.57	0\\
6.58	0\\
6.59	0\\
6.6	0\\
6.61	0\\
6.62	0\\
6.63	0\\
6.64	0\\
6.65	0\\
6.66	0\\
6.67	0\\
6.68	0\\
6.69	0\\
6.7	0\\
6.71	0\\
6.72	0\\
6.73	0\\
6.74	0\\
6.75	0\\
6.76	0\\
6.77	0\\
6.78	0\\
6.79	0\\
6.8	0\\
6.81	0\\
6.82	0\\
6.83	0\\
6.84	0\\
6.85	0\\
6.86	0\\
6.87	0\\
6.88	0\\
6.89	0\\
6.9	0\\
6.91	0\\
6.92	0\\
6.93	0\\
6.94	0\\
6.95	0\\
6.96	0\\
6.97	0\\
6.98	0\\
6.99	0\\
7	0\\
7.01	0\\
7.02	0\\
7.03	0\\
7.04	0\\
7.05	0\\
7.06	0\\
7.07	0\\
7.08	0\\
7.09	0\\
7.1	0\\
7.11	0\\
7.12	0\\
7.13	0\\
7.14	0\\
7.15	0\\
7.16	0\\
7.17	0\\
7.18	0\\
7.19	0\\
7.2	0\\
7.21	0\\
7.22	0\\
7.23	0\\
7.24	0\\
7.25	0\\
7.26	0\\
7.27	0\\
7.28	0\\
7.29	0\\
7.3	0\\
7.31	0\\
7.32	0\\
7.33	0\\
7.34	0\\
7.35	0\\
7.36	0\\
7.37	0\\
7.38	0\\
7.39	0\\
7.4	0\\
7.41	0\\
7.42	0\\
7.43	0\\
7.44	0\\
7.45	0\\
7.46	0\\
7.47	0\\
7.48	0\\
7.49	0\\
7.5	0\\
7.51	0\\
7.52	0\\
7.53	0\\
7.54	0\\
7.55	0\\
7.56	0\\
7.57	0\\
7.58	0\\
7.59	0\\
7.6	0\\
7.61	0\\
7.62	0\\
7.63	0\\
7.64	0\\
7.65	0\\
7.66	0\\
7.67	0\\
7.68	0\\
7.69	0\\
7.7	0\\
7.71	0\\
7.72	0\\
7.73	0\\
7.74	0\\
7.75	0\\
7.76	0\\
7.77	0\\
7.78	0\\
7.79	0\\
7.8	0\\
7.81	0\\
7.82	0\\
7.83	0\\
7.84	0\\
7.85	0\\
7.86	0\\
7.87	0\\
7.88	0\\
7.89	0\\
7.9	0\\
7.91	0\\
7.92	0\\
7.93	0\\
7.94	0\\
7.95	0\\
7.96	0\\
7.97	0\\
7.98	0\\
7.99	0\\
8	0\\
8.01	0\\
8.02	0\\
8.03	0\\
8.04	0\\
8.05	0\\
8.06	0\\
8.07	0\\
8.08	0\\
8.09	0\\
8.1	0\\
8.11	0\\
8.12	0\\
8.13	0\\
8.14	0\\
8.15	0\\
8.16	0\\
8.17	0\\
8.18	0\\
8.19	0\\
8.2	0\\
8.21	0\\
8.22	0\\
8.23	0\\
8.24	0\\
8.25	0\\
8.26	0\\
8.27	0\\
8.28	0\\
8.29	0\\
8.3	0\\
8.31	0\\
8.32	0\\
8.33	0\\
8.34	0\\
8.35	0\\
8.36	0\\
8.37	0\\
8.38	0\\
8.39	0\\
8.4	0\\
8.41	0\\
8.42	0\\
8.43	0\\
8.44	0\\
8.45	0\\
8.46	0\\
8.47	0\\
8.48	0\\
8.49	0\\
8.5	0\\
8.51	0\\
8.52	0\\
8.53	0\\
8.54	0\\
8.55	0\\
8.56	0\\
8.57	0\\
8.58	0\\
8.59	0\\
8.6	0\\
8.61	0\\
8.62	0\\
8.63	0\\
8.64	0\\
8.65	0\\
8.66	0\\
8.67	0\\
8.68	0\\
8.69	0\\
8.7	0\\
8.71	0\\
8.72	0\\
8.73	0\\
8.74	0\\
8.75	0\\
8.76	0\\
8.77	0\\
8.78	0\\
8.79	0\\
8.8	0\\
8.81	0\\
8.82	0\\
8.83	0\\
8.84	0\\
8.85	0\\
8.86	0\\
8.87	0\\
8.88	0\\
8.89	0\\
8.9	0\\
8.91	0\\
8.92	0\\
8.93	0\\
8.94	0\\
8.95	0\\
8.96	0\\
8.97	0\\
8.98	0\\
8.99	0\\
9	0\\
9.01	0\\
9.02	0\\
9.03	0\\
9.04	0\\
9.05	0\\
9.06	0\\
9.07	0\\
9.08	0\\
9.09	0\\
9.1	0\\
9.11	0\\
9.12	0\\
9.13	0\\
9.14	0\\
9.15	0\\
9.16	0\\
9.17	0\\
9.18	0\\
9.19	0\\
9.2	0\\
9.21	0\\
9.22	0\\
9.23	0\\
9.24	0\\
9.25	0\\
9.26	0\\
9.27	0\\
9.28	0\\
9.29	0\\
9.3	0\\
9.31	0\\
9.32	0\\
9.33	0\\
9.34	0\\
9.35	0\\
9.36	0\\
9.37	0\\
9.38	0\\
9.39	0\\
9.4	0\\
9.41	0\\
9.42	0\\
9.43	0\\
9.44	0\\
9.45	0\\
9.46	0\\
9.47	0\\
9.48	0\\
9.49	0\\
9.5	0\\
9.51	0\\
9.52	0\\
9.53	0\\
9.54	0\\
9.55	0\\
9.56	0\\
9.57	0\\
9.58	0\\
9.59	0\\
9.6	0\\
9.61	0\\
9.62	0\\
9.63	0\\
9.64	0\\
9.65	0\\
9.66	0\\
9.67	0\\
9.68	0\\
9.69	0\\
9.7	0\\
9.71	0\\
9.72	0\\
9.73	0\\
9.74	0\\
9.75	0\\
9.76	0\\
9.77	0\\
9.78	0\\
9.79	0\\
9.8	0\\
9.81	0\\
9.82	0\\
9.83	0\\
9.84	0\\
9.85	0\\
9.86	0\\
9.87	0\\
9.88	0\\
9.89	0\\
9.9	0\\
9.91	0\\
9.92	0\\
9.93	0\\
9.94	0\\
9.95	0\\
9.96	0\\
9.97	0\\
9.98	0\\
9.99	0\\
10	0\\
10.01	0\\
10.02	0\\
10.03	0\\
10.04	0\\
10.05	0\\
10.06	0\\
10.07	0\\
10.08	0\\
10.09	0\\
10.1	0\\
10.11	0\\
10.12	0\\
10.13	0\\
10.14	0\\
10.15	0\\
10.16	0\\
10.17	0\\
10.18	0\\
10.19	0\\
10.2	0\\
10.21	0\\
10.22	0\\
10.23	0\\
10.24	0\\
10.25	0\\
10.26	0\\
10.27	0\\
10.28	0\\
10.29	0\\
10.3	0\\
10.31	0\\
10.32	0\\
10.33	0\\
10.34	0\\
10.35	0\\
10.36	0\\
10.37	0\\
10.38	0\\
10.39	0\\
10.4	0\\
10.41	0\\
10.42	0\\
10.43	0\\
10.44	0\\
10.45	0\\
10.46	0\\
10.47	0\\
10.48	0\\
10.49	0\\
10.5	0\\
10.51	0\\
10.52	0\\
10.53	0\\
10.54	0\\
10.55	0\\
10.56	0\\
10.57	0\\
10.58	0\\
10.59	0\\
10.6	0\\
10.61	0\\
10.62	0\\
10.63	0\\
10.64	0\\
10.65	0\\
10.66	0\\
10.67	0\\
10.68	0\\
10.69	0\\
10.7	0\\
10.71	0\\
10.72	0\\
10.73	0\\
10.74	0\\
10.75	0\\
10.76	0\\
10.77	0\\
10.78	0\\
10.79	0\\
10.8	0\\
10.81	0\\
10.82	0\\
10.83	0\\
10.84	0\\
10.85	0\\
10.86	0\\
10.87	0\\
10.88	0\\
10.89	0\\
10.9	0\\
10.91	0\\
10.92	0\\
10.93	0\\
10.94	0\\
10.95	0\\
10.96	0\\
10.97	0\\
10.98	0\\
10.99	0\\
11	0\\
11.01	0\\
11.02	0\\
11.03	0\\
11.04	0\\
11.05	0\\
11.06	0\\
11.07	0\\
11.08	0\\
11.09	0\\
11.1	0\\
11.11	0\\
11.12	0\\
11.13	0\\
11.14	0\\
11.15	0\\
11.16	0\\
11.17	0\\
11.18	0\\
11.19	0\\
11.2	0\\
11.21	0\\
11.22	0\\
11.23	0\\
11.24	0\\
11.25	0\\
11.26	0\\
11.27	0\\
11.28	0\\
11.29	0\\
11.3	0\\
11.31	0\\
11.32	0\\
11.33	0\\
11.34	0\\
11.35	0\\
11.36	0\\
11.37	0\\
11.38	0\\
11.39	0\\
11.4	0\\
11.41	0\\
11.42	0\\
11.43	0\\
11.44	0\\
11.45	0\\
11.46	0\\
11.47	0\\
11.48	0\\
11.49	0\\
11.5	0\\
11.51	0\\
11.52	0\\
11.53	0\\
11.54	0\\
11.55	0\\
11.56	0\\
11.57	0\\
11.58	0\\
11.59	0\\
11.6	0\\
11.61	0\\
11.62	0\\
11.63	0\\
11.64	0\\
11.65	0\\
11.66	0\\
11.67	0\\
11.68	0\\
11.69	0\\
11.7	0\\
11.71	0\\
11.72	0\\
11.73	0\\
11.74	0\\
11.75	0\\
11.76	0\\
11.77	0\\
11.78	0\\
11.79	0\\
11.8	0\\
11.81	0\\
11.82	0\\
11.83	0\\
11.84	0\\
11.85	0\\
11.86	0\\
11.87	0\\
11.88	0\\
11.89	0\\
11.9	0\\
11.91	0\\
11.92	0\\
11.93	0\\
11.94	0\\
11.95	0\\
11.96	0\\
11.97	0\\
11.98	0\\
11.99	0\\
12	0\\
12.01	0\\
12.02	0\\
12.03	0\\
12.04	0\\
12.05	0\\
12.06	0\\
12.07	0\\
12.08	0\\
12.09	0\\
12.1	0\\
12.11	0\\
12.12	0\\
12.13	0\\
12.14	0\\
12.15	0\\
12.16	0\\
12.17	0\\
12.18	0\\
12.19	0\\
12.2	0\\
12.21	0\\
12.22	0\\
12.23	0\\
12.24	0\\
12.25	0\\
12.26	0\\
12.27	0\\
12.28	0\\
12.29	0\\
12.3	0\\
12.31	0\\
12.32	0\\
12.33	0\\
12.34	0\\
12.35	0\\
12.36	0\\
12.37	0\\
12.38	0\\
12.39	0\\
12.4	0\\
12.41	0\\
12.42	0\\
12.43	0\\
12.44	0\\
12.45	0\\
12.46	0\\
12.47	0\\
12.48	0\\
12.49	0\\
12.5	0\\
12.51	0\\
12.52	0\\
12.53	0\\
12.54	0\\
12.55	0\\
12.56	0\\
12.57	0\\
12.58	0\\
12.59	0\\
12.6	0\\
12.61	0\\
12.62	0\\
12.63	0\\
12.64	0\\
12.65	0\\
12.66	0\\
12.67	0\\
12.68	0\\
12.69	0\\
12.7	0\\
12.71	0\\
12.72	0\\
12.73	0\\
12.74	0\\
12.75	0\\
12.76	0\\
12.77	0\\
12.78	0\\
12.79	0\\
12.8	0\\
12.81	0\\
12.82	0\\
12.83	0\\
12.84	0\\
12.85	0\\
12.86	0\\
12.87	0\\
12.88	0\\
12.89	0\\
12.9	0\\
12.91	0\\
12.92	0\\
12.93	0\\
12.94	0\\
12.95	0\\
12.96	0\\
12.97	0\\
12.98	0\\
12.99	0\\
13	0\\
13.01	0\\
13.02	0\\
13.03	0\\
13.04	0\\
13.05	0\\
13.06	0\\
13.07	0\\
13.08	0\\
13.09	0\\
13.1	0\\
13.11	0\\
13.12	0\\
13.13	0\\
13.14	0\\
13.15	0\\
13.16	0\\
13.17	0\\
13.18	0\\
13.19	0\\
13.2	0\\
13.21	0\\
13.22	0\\
13.23	0\\
13.24	0\\
13.25	0\\
13.26	0\\
13.27	0\\
13.28	0\\
13.29	0\\
13.3	0\\
13.31	0\\
13.32	0\\
13.33	0\\
13.34	0\\
13.35	0\\
13.36	0\\
13.37	0\\
13.38	0\\
13.39	0\\
13.4	0\\
13.41	0\\
13.42	0\\
13.43	0\\
13.44	0\\
13.45	0\\
13.46	0\\
13.47	0\\
13.48	0\\
13.49	0\\
13.5	0\\
13.51	0\\
13.52	0\\
13.53	0\\
13.54	0\\
13.55	0\\
13.56	0\\
13.57	0\\
13.58	0\\
13.59	0\\
13.6	0\\
13.61	0\\
13.62	0\\
13.63	0\\
13.64	0\\
13.65	0\\
13.66	0\\
13.67	0\\
13.68	0\\
13.69	0\\
13.7	0\\
13.71	0\\
13.72	0\\
13.73	0\\
13.74	0\\
13.75	0\\
13.76	0\\
13.77	0\\
13.78	0\\
13.79	0\\
13.8	0\\
13.81	0\\
13.82	0\\
13.83	0\\
13.84	0\\
13.85	0\\
13.86	0\\
13.87	0\\
13.88	0\\
13.89	0\\
13.9	0\\
13.91	0\\
13.92	0\\
13.93	0\\
13.94	0\\
13.95	0\\
13.96	0\\
13.97	0\\
13.98	0\\
13.99	0\\
14	0\\
14.01	0\\
14.02	0\\
14.03	0\\
14.04	0\\
14.05	0\\
14.06	0\\
14.07	0\\
14.08	0\\
14.09	0\\
14.1	0\\
14.11	0\\
14.12	0\\
14.13	0\\
14.14	0\\
14.15	0\\
14.16	0\\
14.17	0\\
14.18	0\\
14.19	0\\
14.2	0\\
14.21	0\\
14.22	0\\
14.23	0\\
14.24	0\\
14.25	0\\
14.26	0\\
14.27	0\\
14.28	0\\
14.29	0\\
14.3	0\\
14.31	0\\
14.32	0\\
14.33	0\\
14.34	0\\
14.35	0\\
14.36	0\\
14.37	0\\
14.38	0\\
14.39	0\\
14.4	0\\
14.41	0\\
14.42	0\\
14.43	0\\
14.44	0\\
14.45	0\\
14.46	0\\
14.47	0\\
14.48	0\\
14.49	0\\
14.5	0\\
14.51	0\\
14.52	0\\
14.53	0\\
14.54	0\\
14.55	0\\
14.56	0\\
14.57	0\\
14.58	0\\
14.59	0\\
14.6	0\\
14.61	0\\
14.62	0\\
14.63	0\\
14.64	0\\
14.65	0\\
14.66	0\\
14.67	0\\
14.68	0\\
14.69	0\\
14.7	0\\
14.71	0\\
14.72	0\\
14.73	0\\
14.74	0\\
14.75	0\\
14.76	0\\
14.77	0\\
14.78	0\\
14.79	0\\
14.8	0\\
14.81	0\\
14.82	0\\
14.83	0\\
14.84	0\\
14.85	0\\
14.86	0\\
14.87	0\\
14.88	0\\
14.89	0\\
14.9	0\\
14.91	0\\
14.92	0\\
14.93	0\\
14.94	0\\
14.95	0\\
14.96	0\\
14.97	0\\
14.98	0\\
14.99	0\\
15	0\\
15.01	0\\
15.02	0\\
15.03	0\\
15.04	0\\
15.05	0\\
15.06	0\\
15.07	0\\
15.08	0\\
15.09	0\\
15.1	0\\
15.11	0\\
15.12	0\\
15.13	0\\
15.14	0\\
15.15	0\\
15.16	0\\
15.17	0\\
15.18	0\\
15.19	0\\
15.2	0\\
15.21	0\\
15.22	0\\
15.23	0\\
15.24	0\\
15.25	0\\
15.26	0\\
15.27	0\\
15.28	0\\
15.29	0\\
15.3	0\\
15.31	0\\
15.32	0\\
15.33	0\\
15.34	0\\
15.35	0\\
15.36	0\\
15.37	0\\
15.38	0\\
15.39	0\\
15.4	0\\
15.41	0\\
15.42	0\\
15.43	0\\
15.44	0\\
15.45	0\\
15.46	0\\
15.47	0\\
15.48	0\\
15.49	0\\
15.5	0\\
15.51	0\\
15.52	0\\
15.53	0\\
15.54	0\\
15.55	0\\
15.56	0\\
15.57	0\\
15.58	0\\
15.59	0\\
15.6	0\\
15.61	0\\
15.62	0\\
15.63	0\\
15.64	0\\
15.65	0\\
15.66	0\\
15.67	0\\
15.68	0\\
15.69	0\\
15.7	0\\
15.71	0\\
15.72	0\\
15.73	0\\
15.74	0\\
15.75	0\\
15.76	0\\
15.77	0\\
15.78	0\\
15.79	0\\
15.8	0\\
15.81	0\\
15.82	0\\
15.83	0\\
15.84	0\\
15.85	0\\
15.86	0\\
15.87	0\\
15.88	0\\
15.89	0\\
15.9	0\\
15.91	0\\
15.92	0\\
15.93	0\\
15.94	0\\
15.95	0\\
15.96	0\\
15.97	0\\
15.98	0\\
15.99	0\\
16	0\\
16.01	0\\
16.02	0\\
16.03	0\\
16.04	0\\
16.05	0\\
16.06	0\\
16.07	0\\
16.08	0\\
16.09	0\\
16.1	0\\
16.11	0\\
16.12	0\\
16.13	0\\
16.14	0\\
16.15	0\\
16.16	0\\
16.17	0\\
16.18	0\\
16.19	0\\
16.2	0\\
16.21	0\\
16.22	0\\
16.23	0\\
16.24	0\\
16.25	0\\
16.26	0\\
16.27	0\\
16.28	0\\
16.29	0\\
16.3	0\\
16.31	0\\
16.32	0\\
16.33	0\\
16.34	0\\
16.35	0\\
16.36	0\\
16.37	0\\
16.38	0\\
16.39	0\\
16.4	0\\
16.41	0\\
16.42	0\\
16.43	0\\
16.44	0\\
16.45	0\\
16.46	0\\
16.47	0\\
16.48	0\\
16.49	0\\
16.5	0\\
16.51	0\\
16.52	0\\
16.53	0\\
16.54	0\\
16.55	0\\
16.56	0\\
16.57	0\\
16.58	0\\
16.59	0\\
16.6	0\\
16.61	0\\
16.62	0\\
16.63	0\\
16.64	0\\
16.65	0\\
16.66	0\\
16.67	0\\
16.68	0\\
16.69	0\\
16.7	0\\
16.71	0\\
16.72	0\\
16.73	0\\
16.74	0\\
16.75	0\\
16.76	0\\
16.77	0\\
16.78	0\\
16.79	0\\
16.8	0\\
16.81	0\\
16.82	0\\
16.83	0\\
16.84	0\\
16.85	0\\
16.86	0\\
16.87	0\\
16.88	0\\
16.89	0\\
16.9	0\\
16.91	0\\
16.92	0\\
16.93	0\\
16.94	0\\
16.95	0\\
16.96	0\\
16.97	0\\
16.98	0\\
16.99	0\\
17	0\\
17.01	0\\
17.02	0\\
17.03	0\\
17.04	0\\
17.05	0\\
17.06	0\\
17.07	0\\
17.08	0\\
17.09	0\\
17.1	0\\
17.11	0\\
17.12	0\\
17.13	0\\
17.14	0\\
17.15	0\\
17.16	0\\
17.17	0\\
17.18	0\\
17.19	0\\
17.2	0\\
17.21	0\\
17.22	0\\
17.23	0\\
17.24	0\\
17.25	0\\
17.26	0\\
17.27	0\\
17.28	0\\
17.29	0\\
17.3	0\\
17.31	0\\
17.32	0\\
17.33	0\\
17.34	0\\
17.35	0\\
17.36	0\\
17.37	0\\
17.38	0\\
17.39	0\\
17.4	0\\
17.41	0\\
17.42	0\\
17.43	0\\
17.44	0\\
17.45	0\\
17.46	0\\
17.47	0\\
17.48	0\\
17.49	0\\
17.5	0\\
17.51	0\\
17.52	0\\
17.53	0\\
17.54	0\\
17.55	0\\
17.56	0\\
17.57	0\\
17.58	0\\
17.59	0\\
17.6	0\\
17.61	0\\
17.62	0\\
17.63	0\\
17.64	0\\
17.65	0\\
17.66	0\\
17.67	0\\
17.68	0\\
17.69	0\\
17.7	0\\
17.71	0\\
17.72	0\\
17.73	0\\
17.74	0\\
17.75	0\\
17.76	0\\
17.77	0\\
17.78	0\\
17.79	0\\
17.8	0\\
17.81	0\\
17.82	0\\
17.83	0\\
17.84	0\\
17.85	0\\
17.86	0\\
17.87	0\\
17.88	0\\
17.89	0\\
17.9	0\\
17.91	0\\
17.92	0\\
17.93	0\\
17.94	0\\
17.95	0\\
17.96	0\\
17.97	0\\
17.98	0\\
17.99	0\\
18	0\\
18.01	0\\
18.02	0\\
18.03	0\\
18.04	0\\
18.05	0\\
18.06	0\\
18.07	0\\
18.08	0\\
18.09	0\\
18.1	0\\
18.11	0\\
18.12	0\\
18.13	0\\
18.14	0\\
18.15	0\\
18.16	0\\
18.17	0\\
18.18	0\\
18.19	0\\
18.2	0\\
18.21	0\\
18.22	0\\
18.23	0\\
18.24	0\\
18.25	0\\
18.26	0\\
18.27	0\\
18.28	0\\
18.29	0\\
18.3	0\\
18.31	0\\
18.32	0\\
18.33	0\\
18.34	0\\
18.35	0\\
18.36	0\\
18.37	0\\
18.38	0\\
18.39	0\\
18.4	0\\
18.41	0\\
18.42	0\\
18.43	0\\
18.44	0\\
18.45	0\\
18.46	0\\
18.47	0\\
18.48	0\\
18.49	0\\
18.5	0\\
18.51	0\\
18.52	0\\
18.53	0\\
18.54	0\\
18.55	0\\
18.56	0\\
18.57	0\\
18.58	0\\
18.59	0\\
18.6	0\\
18.61	0\\
18.62	0\\
18.63	0\\
18.64	0\\
18.65	0\\
18.66	0\\
18.67	0\\
18.68	0\\
18.69	0\\
18.7	0\\
18.71	0\\
18.72	0\\
18.73	0\\
18.74	0\\
18.75	0\\
18.76	0\\
18.77	0\\
18.78	0\\
18.79	0\\
18.8	0\\
18.81	0\\
18.82	0\\
18.83	0\\
18.84	0\\
18.85	0\\
18.86	0\\
18.87	0\\
18.88	0\\
18.89	0\\
18.9	0\\
18.91	0\\
18.92	0\\
18.93	0\\
18.94	0\\
18.95	0\\
18.96	0\\
18.97	0\\
18.98	0\\
18.99	0\\
19	0\\
19.01	0\\
19.02	0\\
19.03	0\\
19.04	0\\
19.05	0\\
19.06	0\\
19.07	0\\
19.08	0\\
19.09	0\\
19.1	0\\
19.11	0\\
19.12	0\\
19.13	0\\
19.14	0\\
19.15	0\\
19.16	0\\
19.17	0\\
19.18	0\\
19.19	0\\
19.2	0\\
19.21	0\\
19.22	0\\
19.23	0\\
19.24	0\\
19.25	0\\
19.26	0\\
19.27	0\\
19.28	0\\
19.29	0\\
19.3	0\\
19.31	0\\
19.32	0\\
19.33	0\\
19.34	0\\
19.35	0\\
19.36	0\\
19.37	0\\
19.38	0\\
19.39	0\\
19.4	0\\
19.41	0\\
19.42	0\\
19.43	0\\
19.44	0\\
19.45	0\\
19.46	0\\
19.47	0\\
19.48	0\\
19.49	0\\
19.5	0\\
19.51	0\\
19.52	0\\
19.53	0\\
19.54	0\\
19.55	0\\
19.56	0\\
19.57	0\\
19.58	0\\
19.59	0\\
19.6	0\\
19.61	0\\
19.62	0\\
19.63	0\\
19.64	0\\
19.65	0\\
19.66	0\\
19.67	0\\
19.68	0\\
19.69	0\\
19.7	0\\
19.71	0\\
19.72	0\\
19.73	0\\
19.74	0\\
19.75	0\\
19.76	0\\
19.77	0\\
19.78	0\\
19.79	0\\
19.8	0\\
19.81	0\\
19.82	0\\
19.83	0\\
19.84	0\\
19.85	0\\
19.86	0\\
19.87	0\\
19.88	0\\
19.89	0\\
19.9	0\\
19.91	0\\
19.92	0\\
19.93	0\\
19.94	0\\
19.95	0\\
19.96	0\\
19.97	0\\
19.98	0\\
19.99	0\\
20	0\\
20.01	0\\
20.02	0\\
20.03	0\\
20.04	0\\
20.05	0\\
20.06	0\\
20.07	0\\
20.08	0\\
20.09	0\\
20.1	0\\
20.11	0\\
20.12	0\\
20.13	0\\
20.14	0\\
20.15	0\\
20.16	0\\
20.17	0\\
20.18	0\\
20.19	0\\
20.2	0\\
20.21	0\\
20.22	0\\
20.23	0\\
20.24	0\\
20.25	0\\
20.26	0\\
20.27	0\\
20.28	0\\
20.29	0\\
20.3	0\\
20.31	0\\
20.32	0\\
20.33	0\\
20.34	0\\
20.35	0\\
20.36	0\\
20.37	0\\
20.38	0\\
20.39	0\\
20.4	0\\
20.41	0\\
20.42	0\\
20.43	0\\
20.44	0\\
20.45	0\\
20.46	0\\
20.47	0\\
20.48	0\\
20.49	0\\
20.5	0\\
20.51	0\\
20.52	0\\
20.53	0\\
20.54	0\\
20.55	0\\
20.56	0\\
20.57	0\\
20.58	0\\
20.59	0\\
20.6	0\\
20.61	0\\
20.62	0\\
20.63	0\\
20.64	0\\
20.65	0\\
20.66	0\\
20.67	0\\
20.68	0\\
20.69	0\\
20.7	0\\
20.71	0\\
20.72	0\\
20.73	0\\
20.74	0\\
20.75	0\\
20.76	0\\
20.77	0\\
20.78	0\\
20.79	0\\
20.8	0\\
20.81	0\\
20.82	0\\
20.83	0\\
20.84	0\\
20.85	0\\
20.86	0\\
20.87	0\\
20.88	0\\
20.89	0\\
20.9	0\\
20.91	0\\
20.92	0\\
20.93	0\\
20.94	0\\
20.95	0\\
20.96	0\\
20.97	0\\
20.98	0\\
20.99	0\\
21	0\\
21.01	0\\
21.02	0\\
21.03	0\\
21.04	0\\
21.05	0\\
21.06	0\\
21.07	0\\
21.08	0\\
21.09	0\\
21.1	0\\
21.11	0\\
21.12	0\\
21.13	0\\
21.14	0\\
21.15	0\\
21.16	0\\
21.17	0\\
21.18	0\\
21.19	0\\
21.2	0\\
21.21	0\\
21.22	0\\
21.23	0\\
21.24	0\\
21.25	0\\
21.26	0\\
21.27	0\\
21.28	0\\
21.29	0\\
21.3	0\\
21.31	0\\
21.32	0\\
21.33	0\\
21.34	0\\
21.35	0\\
21.36	0\\
21.37	0\\
21.38	0\\
21.39	0\\
21.4	0\\
21.41	0\\
21.42	0\\
21.43	0\\
21.44	0\\
21.45	0\\
21.46	0\\
21.47	0\\
21.48	0\\
21.49	0\\
21.5	0\\
21.51	0\\
21.52	0\\
21.53	0\\
21.54	0\\
21.55	0\\
21.56	0\\
21.57	0\\
21.58	0\\
21.59	0\\
21.6	0\\
21.61	0\\
21.62	0\\
21.63	0\\
21.64	0\\
21.65	0\\
21.66	0\\
21.67	0\\
21.68	0\\
21.69	0\\
21.7	0\\
21.71	0\\
21.72	0\\
21.73	0\\
21.74	0\\
21.75	0\\
21.76	0\\
21.77	0\\
21.78	0\\
21.79	0\\
21.8	0\\
21.81	0\\
21.82	0\\
21.83	0\\
21.84	0\\
21.85	0\\
21.86	0\\
21.87	0\\
21.88	0\\
21.89	0\\
21.9	0\\
21.91	0\\
21.92	0\\
21.93	0\\
21.94	0\\
21.95	0\\
21.96	0\\
21.97	0\\
21.98	0\\
21.99	0\\
22	0\\
22.01	0\\
22.02	0\\
22.03	0\\
22.04	0\\
22.05	0\\
22.06	0\\
22.07	0\\
22.08	0\\
22.09	0\\
22.1	0\\
22.11	0\\
22.12	0\\
22.13	0\\
22.14	0\\
22.15	0\\
22.16	0\\
22.17	0\\
22.18	0\\
22.19	0\\
22.2	0\\
22.21	0\\
22.22	0\\
22.23	0\\
22.24	0\\
22.25	0\\
22.26	0\\
22.27	0\\
22.28	0\\
22.29	0\\
22.3	0\\
22.31	0\\
22.32	0\\
22.33	0\\
22.34	0\\
22.35	0\\
22.36	0\\
22.37	0\\
22.38	0\\
22.39	0\\
22.4	0\\
22.41	0\\
22.42	0\\
22.43	0\\
22.44	0\\
22.45	0\\
22.46	0\\
22.47	0\\
22.48	0\\
22.49	0\\
22.5	0\\
22.51	0\\
22.52	0\\
22.53	0\\
22.54	0\\
22.55	0\\
22.56	0\\
22.57	0\\
22.58	0\\
22.59	0\\
22.6	0\\
22.61	0\\
22.62	0\\
22.63	0\\
22.64	0\\
22.65	0\\
22.66	0\\
22.67	0\\
22.68	0\\
22.69	0\\
22.7	0\\
22.71	0\\
22.72	0\\
22.73	0\\
22.74	0\\
22.75	0\\
22.76	0\\
22.77	0\\
22.78	0\\
22.79	0\\
22.8	0\\
22.81	0\\
22.82	0\\
22.83	0\\
22.84	0\\
22.85	0\\
22.86	0\\
22.87	0\\
22.88	0\\
22.89	0\\
22.9	0\\
22.91	0\\
22.92	0\\
22.93	0\\
22.94	0\\
22.95	0\\
22.96	0\\
22.97	0\\
22.98	0\\
22.99	0\\
23	0\\
23.01	0\\
23.02	0\\
23.03	0\\
23.04	0\\
23.05	0\\
23.06	0\\
23.07	0\\
23.08	0\\
23.09	0\\
23.1	0\\
23.11	0\\
23.12	0\\
23.13	0\\
23.14	0\\
23.15	0\\
23.16	0\\
23.17	0\\
23.18	0\\
23.19	0\\
23.2	0\\
23.21	0\\
23.22	0\\
23.23	0\\
23.24	0\\
23.25	0\\
23.26	0\\
23.27	0\\
23.28	0\\
23.29	0\\
23.3	0\\
23.31	0\\
23.32	0\\
23.33	0\\
23.34	0\\
23.35	0\\
23.36	0\\
23.37	0\\
23.38	0\\
23.39	0\\
23.4	0\\
23.41	0\\
23.42	0\\
23.43	0\\
23.44	0\\
23.45	0\\
23.46	0\\
23.47	0\\
23.48	0\\
23.49	0\\
23.5	0\\
23.51	0\\
23.52	0\\
23.53	0\\
23.54	0\\
23.55	0\\
23.56	0\\
23.57	0\\
23.58	0\\
23.59	0\\
23.6	0\\
23.61	0\\
23.62	0\\
23.63	0\\
23.64	0\\
23.65	0\\
23.66	0\\
23.67	0\\
23.68	0\\
23.69	0\\
23.7	0\\
23.71	0\\
23.72	0\\
23.73	0\\
23.74	0\\
23.75	0\\
23.76	0\\
23.77	0\\
23.78	0\\
23.79	0\\
23.8	0\\
23.81	0\\
23.82	0\\
23.83	0\\
23.84	0\\
23.85	0\\
23.86	0\\
23.87	0\\
23.88	0\\
23.89	0\\
23.9	0\\
23.91	0\\
23.92	0\\
23.93	0\\
23.94	0\\
23.95	0\\
23.96	0\\
23.97	0\\
23.98	0\\
23.99	0\\
24	0\\
24.01	0\\
24.02	0\\
24.03	0\\
24.04	0\\
24.05	0\\
24.06	0\\
24.07	0\\
24.08	0\\
24.09	0\\
24.1	0\\
24.11	0\\
24.12	0\\
24.13	0\\
24.14	0\\
24.15	0\\
24.16	0\\
24.17	0\\
24.18	0\\
24.19	0\\
24.2	0\\
24.21	0\\
24.22	0\\
24.23	0\\
24.24	0\\
24.25	0\\
24.26	0\\
24.27	0\\
24.28	0\\
24.29	0\\
24.3	0\\
24.31	0\\
24.32	0\\
24.33	0\\
24.34	0\\
24.35	0\\
24.36	0\\
24.37	0\\
24.38	0\\
24.39	0\\
24.4	0\\
24.41	0\\
24.42	0\\
24.43	0\\
24.44	0\\
24.45	0\\
24.46	0\\
24.47	0\\
24.48	0\\
24.49	0\\
24.5	0\\
24.51	0\\
24.52	0\\
24.53	0\\
24.54	0\\
24.55	0\\
24.56	0\\
24.57	0\\
24.58	0\\
24.59	0\\
24.6	0\\
24.61	0\\
24.62	0\\
24.63	0\\
24.64	0\\
24.65	0\\
24.66	0\\
24.67	0\\
24.68	0\\
24.69	0\\
24.7	0\\
24.71	0\\
24.72	0\\
24.73	0\\
24.74	0\\
24.75	0\\
24.76	0\\
24.77	0\\
24.78	0\\
24.79	0\\
24.8	0\\
24.81	0\\
24.82	0\\
24.83	0\\
24.84	0\\
24.85	0\\
24.86	0\\
24.87	0\\
24.88	0\\
24.89	0\\
24.9	0\\
24.91	0\\
24.92	0\\
24.93	0\\
24.94	0\\
24.95	0\\
24.96	0\\
24.97	0\\
24.98	0\\
24.99	0\\
25	0\\
25.01	0\\
25.02	0\\
25.03	0\\
25.04	0\\
25.05	0\\
25.06	0\\
25.07	0\\
25.08	0\\
25.09	0\\
25.1	0\\
25.11	0\\
25.12	0\\
25.13	0\\
25.14	0\\
25.15	0\\
25.16	0\\
25.17	0\\
25.18	0\\
25.19	0\\
25.2	0\\
25.21	0\\
25.22	0\\
25.23	0\\
25.24	0\\
25.25	0\\
25.26	0\\
25.27	0\\
25.28	0\\
25.29	0\\
25.3	0\\
25.31	0\\
25.32	0\\
25.33	0\\
25.34	0\\
25.35	0\\
25.36	0\\
25.37	0\\
25.38	0\\
25.39	0\\
25.4	0\\
25.41	0\\
25.42	0\\
25.43	0\\
25.44	0\\
25.45	0\\
25.46	0\\
25.47	0\\
25.48	0\\
25.49	0\\
25.5	0\\
25.51	0\\
25.52	0\\
25.53	0\\
25.54	0\\
25.55	0\\
25.56	0\\
25.57	0\\
25.58	0\\
25.59	0\\
25.6	0\\
25.61	0\\
25.62	0\\
25.63	0\\
25.64	0\\
25.65	0\\
25.66	0\\
25.67	0\\
25.68	0\\
25.69	0\\
25.7	0\\
25.71	0\\
25.72	0\\
25.73	0\\
25.74	0\\
25.75	0\\
25.76	0\\
25.77	0\\
25.78	0\\
25.79	0\\
25.8	0\\
25.81	0\\
25.82	0\\
25.83	0\\
25.84	0\\
25.85	0\\
25.86	0\\
25.87	0\\
25.88	0\\
25.89	0\\
25.9	0\\
25.91	0\\
25.92	0\\
25.93	0\\
25.94	0\\
25.95	0\\
25.96	0\\
25.97	0\\
25.98	0\\
25.99	0\\
26	0\\
26.01	0\\
26.02	0\\
26.03	0\\
26.04	0\\
26.05	0\\
26.06	0\\
26.07	0\\
26.08	0\\
26.09	0\\
26.1	0\\
26.11	0\\
26.12	0\\
26.13	0\\
26.14	0\\
26.15	0\\
26.16	0\\
26.17	0\\
26.18	0\\
26.19	0\\
26.2	0\\
26.21	0\\
26.22	0\\
26.23	0\\
26.24	0\\
26.25	0\\
26.26	0\\
26.27	0\\
26.28	0\\
26.29	0\\
26.3	0\\
26.31	0\\
26.32	0\\
26.33	0\\
26.34	0\\
26.35	0\\
26.36	0\\
26.37	0\\
26.38	0\\
26.39	0\\
26.4	0\\
26.41	0\\
26.42	0\\
26.43	0\\
26.44	0\\
26.45	0\\
26.46	0\\
26.47	0\\
26.48	0\\
26.49	0\\
26.5	0\\
26.51	0\\
26.52	0\\
26.53	0\\
26.54	0\\
26.55	0\\
26.56	0\\
26.57	0\\
26.58	0\\
26.59	0\\
26.6	0\\
26.61	0\\
26.62	0\\
26.63	0\\
26.64	0\\
26.65	0\\
26.66	0\\
26.67	0\\
26.68	0\\
26.69	0\\
26.7	0\\
26.71	0\\
26.72	0\\
26.73	0\\
26.74	0\\
26.75	0\\
26.76	0\\
26.77	0\\
26.78	0\\
26.79	0\\
26.8	0\\
26.81	0\\
26.82	0\\
26.83	0\\
26.84	0\\
26.85	0\\
26.86	0\\
26.87	0\\
26.88	0\\
26.89	0\\
26.9	0\\
26.91	0\\
26.92	0\\
26.93	0\\
26.94	0\\
26.95	0\\
26.96	0\\
26.97	0\\
26.98	0\\
26.99	0\\
27	0\\
27.01	0\\
27.02	0\\
27.03	0\\
27.04	0\\
27.05	0\\
27.06	0\\
27.07	0\\
27.08	0\\
27.09	0\\
27.1	0\\
27.11	0\\
27.12	0\\
27.13	0\\
27.14	0\\
27.15	0\\
27.16	0\\
27.17	0\\
27.18	0\\
27.19	0\\
27.2	0\\
27.21	0\\
27.22	0\\
27.23	0\\
27.24	0\\
27.25	0\\
27.26	0\\
27.27	0\\
27.28	0\\
27.29	0\\
27.3	0\\
27.31	0\\
27.32	0\\
27.33	0\\
27.34	0\\
27.35	0\\
27.36	0\\
27.37	0\\
27.38	0\\
27.39	0\\
27.4	0\\
27.41	0\\
27.42	0\\
27.43	0\\
27.44	0\\
27.45	0\\
27.46	0\\
27.47	0\\
27.48	0\\
27.49	0\\
27.5	0\\
27.51	0\\
27.52	0\\
27.53	0\\
27.54	0\\
27.55	0\\
27.56	0\\
27.57	0\\
27.58	0\\
27.59	0\\
27.6	0\\
27.61	0\\
27.62	0\\
27.63	0\\
27.64	0\\
27.65	0\\
27.66	0\\
27.67	0\\
27.68	0\\
27.69	0\\
27.7	0\\
27.71	0\\
27.72	0\\
27.73	0\\
27.74	0\\
27.75	0\\
27.76	0\\
27.77	0\\
27.78	0\\
27.79	0\\
27.8	0\\
27.81	0\\
27.82	0\\
27.83	0\\
27.84	0\\
27.85	0\\
27.86	0\\
27.87	0\\
27.88	0\\
27.89	0\\
27.9	0\\
27.91	0\\
27.92	0\\
27.93	0\\
27.94	0\\
27.95	0\\
27.96	0\\
27.97	0\\
27.98	0\\
27.99	0\\
28	0\\
28.01	0\\
28.02	0\\
28.03	0\\
28.04	0\\
28.05	0\\
28.06	0\\
28.07	0\\
28.08	0\\
28.09	0\\
28.1	0\\
28.11	0\\
28.12	0\\
28.13	0\\
28.14	0\\
28.15	0\\
28.16	0\\
28.17	0\\
28.18	0\\
28.19	0\\
28.2	0\\
28.21	0\\
28.22	0\\
28.23	0\\
28.24	0\\
28.25	0\\
28.26	0\\
28.27	0\\
28.28	0\\
28.29	0\\
28.3	0\\
28.31	0\\
28.32	0\\
28.33	0\\
28.34	0\\
28.35	0\\
28.36	0\\
28.37	0\\
28.38	0\\
28.39	0\\
28.4	0\\
28.41	0\\
28.42	0\\
28.43	0\\
28.44	0\\
28.45	0\\
28.46	0\\
28.47	0\\
28.48	0\\
28.49	0\\
28.5	0\\
28.51	0\\
28.52	0\\
28.53	0\\
28.54	0\\
28.55	0\\
28.56	0\\
28.57	0\\
28.58	0\\
28.59	0\\
28.6	0\\
28.61	0\\
28.62	0\\
28.63	0\\
28.64	0\\
28.65	0\\
28.66	0\\
28.67	0\\
28.68	0\\
28.69	0\\
28.7	0\\
28.71	0\\
28.72	0\\
28.73	0\\
28.74	0\\
28.75	0\\
28.76	0\\
28.77	0\\
28.78	0\\
28.79	0\\
28.8	0\\
28.81	0\\
28.82	0\\
28.83	0\\
28.84	0\\
28.85	0\\
28.86	0\\
28.87	0\\
28.88	0\\
28.89	0\\
28.9	0\\
28.91	0\\
28.92	0\\
28.93	0\\
28.94	0\\
28.95	0\\
28.96	0\\
28.97	0\\
28.98	0\\
28.99	0\\
29	0\\
29.01	0\\
29.02	0\\
29.03	0\\
29.04	0\\
29.05	0\\
29.06	0\\
29.07	0\\
29.08	0\\
29.09	0\\
29.1	0\\
29.11	0\\
29.12	0\\
29.13	0\\
29.14	0\\
29.15	0\\
29.16	0\\
29.17	0\\
29.18	0\\
29.19	0\\
29.2	0\\
29.21	0\\
29.22	0\\
29.23	0\\
29.24	0\\
29.25	0\\
29.26	0\\
29.27	0\\
29.28	0\\
29.29	0\\
29.3	0\\
29.31	0\\
29.32	0\\
29.33	0\\
29.34	0\\
29.35	0\\
29.36	0\\
29.37	0\\
29.38	0\\
29.39	0\\
29.4	0\\
29.41	0\\
29.42	0\\
29.43	0\\
29.44	0\\
29.45	0\\
29.46	0\\
29.47	0\\
29.48	0\\
29.49	0\\
29.5	0\\
29.51	0\\
29.52	0\\
29.53	0\\
29.54	0\\
29.55	0\\
29.56	0\\
29.57	0\\
29.58	0\\
29.59	0\\
29.6	0\\
29.61	0\\
29.62	0\\
29.63	0\\
29.64	0\\
29.65	0\\
29.66	0\\
29.67	0\\
29.68	0\\
29.69	0\\
29.7	0\\
29.71	0\\
29.72	0\\
29.73	0\\
29.74	0\\
29.75	0\\
29.76	0\\
29.77	0\\
29.78	0\\
29.79	0\\
29.8	0\\
29.81	0\\
29.82	0\\
29.83	0\\
29.84	0\\
29.85	0\\
29.86	0\\
29.87	0\\
29.88	0\\
29.89	0\\
29.9	0\\
29.91	0\\
29.92	0\\
29.93	0\\
29.94	0\\
29.95	0\\
29.96	0\\
29.97	0\\
29.98	0\\
29.99	0\\
30	0\\
30.01	0\\
30.02	0\\
30.03	0\\
30.04	0\\
30.05	0\\
30.06	0\\
30.07	0\\
30.08	0\\
30.09	0\\
30.1	0\\
30.11	0\\
30.12	0\\
30.13	0\\
30.14	0\\
30.15	0\\
30.16	0\\
30.17	0\\
30.18	0\\
30.19	0\\
30.2	0\\
30.21	0\\
30.22	0\\
30.23	0\\
30.24	0\\
30.25	0\\
30.26	0\\
30.27	0\\
30.28	0\\
30.29	0\\
30.3	0\\
30.31	0\\
30.32	0\\
30.33	0\\
30.34	0\\
30.35	0\\
30.36	0\\
30.37	0\\
30.38	0\\
30.39	0\\
30.4	0\\
30.41	0\\
30.42	0\\
30.43	0\\
30.44	0\\
30.45	0\\
30.46	0\\
30.47	0\\
30.48	0\\
30.49	0\\
30.5	0\\
30.51	0\\
30.52	0\\
30.53	0\\
30.54	0\\
30.55	0\\
30.56	0\\
30.57	0\\
30.58	0\\
30.59	0\\
30.6	0\\
30.61	0\\
30.62	0\\
30.63	0\\
30.64	0\\
30.65	0\\
30.66	0\\
30.67	0\\
30.68	0\\
30.69	0\\
30.7	0\\
30.71	0\\
30.72	0\\
30.73	0\\
30.74	0\\
30.75	0\\
30.76	0\\
30.77	0\\
30.78	0\\
30.79	0\\
30.8	0\\
30.81	0\\
30.82	0\\
30.83	0\\
30.84	0\\
30.85	0\\
30.86	0\\
30.87	0\\
30.88	0\\
30.89	0\\
30.9	0\\
30.91	0\\
30.92	0\\
30.93	0\\
30.94	0\\
30.95	0\\
30.96	0\\
30.97	0\\
30.98	0\\
30.99	0\\
31	0\\
31.01	0\\
31.02	0\\
31.03	0\\
31.04	0\\
31.05	0\\
31.06	0\\
31.07	0\\
31.08	0\\
31.09	0\\
31.1	0\\
31.11	0\\
31.12	0\\
31.13	0\\
31.14	0\\
31.15	0\\
31.16	0\\
31.17	0\\
31.18	0\\
31.19	0\\
31.2	0\\
31.21	0\\
31.22	0\\
31.23	0\\
31.24	0\\
31.25	0\\
31.26	0\\
31.27	0\\
31.28	0\\
31.29	0\\
31.3	0\\
31.31	0\\
31.32	0\\
31.33	0\\
31.34	0\\
31.35	0\\
31.36	0\\
31.37	0\\
31.38	0\\
31.39	0\\
31.4	0\\
31.41	0\\
31.42	0\\
31.43	0\\
31.44	0\\
31.45	0\\
31.46	0\\
31.47	0\\
31.48	0\\
31.49	0\\
31.5	0\\
31.51	0\\
31.52	0\\
31.53	0\\
31.54	0\\
31.55	0\\
31.56	0\\
31.57	0\\
31.58	0\\
31.59	0\\
31.6	0\\
31.61	0\\
31.62	0\\
31.63	0\\
31.64	0\\
31.65	0\\
31.66	0\\
31.67	0\\
31.68	0\\
31.69	0\\
31.7	0\\
31.71	0\\
31.72	0\\
31.73	0\\
31.74	0\\
31.75	0\\
31.76	0\\
31.77	0\\
31.78	0\\
31.79	0\\
31.8	0\\
31.81	0\\
31.82	0\\
31.83	0\\
31.84	0\\
31.85	0\\
31.86	0\\
31.87	0\\
31.88	0\\
31.89	0\\
31.9	0\\
31.91	0\\
31.92	0\\
31.93	0\\
31.94	0\\
31.95	0\\
31.96	0\\
31.97	0\\
31.98	0\\
31.99	0\\
32	0\\
32.01	0\\
32.02	0\\
32.03	0\\
32.04	0\\
32.05	0\\
32.06	0\\
32.07	0\\
32.08	0\\
32.09	0\\
32.1	0\\
32.11	0\\
32.12	0\\
32.13	0\\
32.14	0\\
32.15	0\\
32.16	0\\
32.17	0\\
32.18	0\\
32.19	0\\
32.2	0\\
32.21	0\\
32.22	0\\
32.23	0\\
32.24	0\\
32.25	0\\
32.26	0\\
32.27	0\\
32.28	0\\
32.29	0\\
32.3	0\\
32.31	0\\
32.32	0\\
32.33	0\\
32.34	0\\
32.35	0\\
32.36	0\\
32.37	0\\
32.38	0\\
32.39	0\\
32.4	0\\
32.41	0\\
32.42	0\\
32.43	0\\
32.44	0\\
32.45	0\\
32.46	0\\
32.47	0\\
32.48	0\\
32.49	0\\
32.5	0\\
32.51	0\\
32.52	0\\
32.53	0\\
32.54	0\\
32.55	0\\
32.56	0\\
32.57	0\\
32.58	0\\
32.59	0\\
32.6	0\\
32.61	0\\
32.62	0\\
32.63	0\\
32.64	0\\
32.65	0\\
32.66	0\\
32.67	0\\
32.68	0\\
32.69	0\\
32.7	0\\
32.71	0\\
32.72	0\\
32.73	0\\
32.74	0\\
32.75	0\\
32.76	0\\
32.77	0\\
32.78	0\\
32.79	0\\
32.8	0\\
32.81	0\\
32.82	0\\
32.83	0\\
32.84	0\\
32.85	0\\
32.86	0\\
32.87	0\\
32.88	0\\
32.89	0\\
32.9	0\\
32.91	0\\
32.92	0\\
32.93	0\\
32.94	0\\
32.95	0\\
32.96	0\\
32.97	0\\
32.98	0\\
32.99	0\\
33	0\\
33.01	0\\
33.02	0\\
33.03	0\\
33.04	0\\
33.05	0\\
33.06	0\\
33.07	0\\
33.08	0\\
33.09	0\\
33.1	0\\
33.11	0\\
33.12	0\\
33.13	0\\
33.14	0\\
33.15	0\\
33.16	0\\
33.17	0\\
33.18	0\\
33.19	0\\
33.2	0\\
33.21	0\\
33.22	0\\
33.23	0\\
33.24	0\\
33.25	0\\
33.26	0\\
33.27	0\\
33.28	0\\
33.29	0\\
33.3	0\\
33.31	0\\
33.32	0\\
33.33	0\\
33.34	0\\
33.35	0\\
33.36	0\\
33.37	0\\
33.38	0\\
33.39	0\\
33.4	0\\
33.41	0\\
33.42	0\\
33.43	0\\
33.44	0\\
33.45	0\\
33.46	0\\
33.47	0\\
33.48	0\\
33.49	0\\
33.5	0\\
33.51	0\\
33.52	0\\
33.53	0\\
33.54	0\\
33.55	0\\
33.56	0\\
33.57	0\\
33.58	0\\
33.59	0\\
33.6	0\\
33.61	0\\
33.62	0\\
33.63	0\\
33.64	0\\
33.65	0\\
33.66	0\\
33.67	0\\
33.68	0\\
33.69	0\\
33.7	0\\
33.71	0\\
33.72	0\\
33.73	0\\
33.74	0\\
33.75	0\\
33.76	0\\
33.77	0\\
33.78	0\\
33.79	0\\
33.8	0\\
33.81	0\\
33.82	0\\
33.83	0\\
33.84	0\\
33.85	0\\
33.86	0\\
33.87	0\\
33.88	0\\
33.89	0\\
33.9	0\\
33.91	0\\
33.92	0\\
33.93	0\\
33.94	0\\
33.95	0\\
33.96	0\\
33.97	0\\
33.98	0\\
33.99	0\\
34	0\\
34.01	0\\
34.02	0\\
34.03	0\\
34.04	0\\
34.05	0\\
34.06	0\\
34.07	0\\
34.08	0\\
34.09	0\\
34.1	0\\
34.11	0\\
34.12	0\\
34.13	0\\
34.14	0\\
34.15	0\\
34.16	0\\
34.17	0\\
34.18	0\\
34.19	0\\
34.2	0\\
34.21	0\\
34.22	0\\
34.23	0\\
34.24	0\\
34.25	0\\
34.26	0\\
34.27	0\\
34.28	0\\
34.29	0\\
34.3	0\\
34.31	0\\
34.32	0\\
34.33	0\\
34.34	0\\
34.35	0\\
34.36	0\\
34.37	0\\
34.38	0\\
34.39	0\\
34.4	0\\
34.41	0\\
34.42	0\\
34.43	0\\
34.44	0\\
34.45	0\\
34.46	0\\
34.47	0\\
34.48	0\\
34.49	0\\
34.5	0\\
34.51	0\\
34.52	0\\
34.53	0\\
34.54	0\\
34.55	0\\
34.56	0\\
34.57	0\\
34.58	0\\
34.59	0\\
34.6	0\\
34.61	0\\
34.62	0\\
34.63	0\\
34.64	0\\
34.65	0\\
34.66	0\\
34.67	0\\
34.68	0\\
34.69	0\\
34.7	0\\
34.71	0\\
34.72	0\\
34.73	0\\
34.74	0\\
34.75	0\\
34.76	0\\
34.77	0\\
34.78	0\\
34.79	0\\
34.8	0\\
34.81	0\\
34.82	0\\
34.83	0\\
34.84	0\\
34.85	0\\
34.86	0\\
34.87	0\\
34.88	0\\
34.89	0\\
34.9	0\\
34.91	0\\
34.92	0\\
34.93	0\\
34.94	0\\
34.95	0\\
34.96	0\\
34.97	0\\
34.98	0\\
34.99	0\\
35	0\\
35.01	0\\
35.02	0\\
35.03	0\\
35.04	0\\
35.05	0\\
35.06	0\\
35.07	0\\
35.08	0\\
35.09	0\\
35.1	0\\
35.11	0\\
35.12	0\\
35.13	0\\
35.14	0\\
35.15	0\\
35.16	0\\
35.17	0\\
35.18	0\\
35.19	0\\
35.2	0\\
35.21	0\\
35.22	0\\
35.23	0\\
35.24	0\\
35.25	0\\
35.26	0\\
35.27	0\\
35.28	0\\
35.29	0\\
35.3	0\\
35.31	0\\
35.32	0\\
35.33	0\\
35.34	0\\
35.35	0\\
35.36	0\\
35.37	0\\
35.38	0\\
35.39	0\\
35.4	0\\
35.41	0\\
35.42	0\\
35.43	0\\
35.44	0\\
35.45	0\\
35.46	0\\
35.47	0\\
35.48	0\\
35.49	0\\
35.5	0\\
35.51	0\\
35.52	0\\
35.53	0\\
35.54	0\\
35.55	0\\
35.56	0\\
35.57	0\\
35.58	0\\
35.59	0\\
35.6	0\\
35.61	0\\
35.62	0\\
35.63	0\\
35.64	0\\
35.65	0\\
35.66	0\\
35.67	0\\
35.68	0\\
35.69	0\\
35.7	0\\
35.71	0\\
35.72	0\\
35.73	0\\
35.74	0\\
35.75	0\\
35.76	0\\
35.77	0\\
35.78	0\\
35.79	0\\
35.8	0\\
35.81	0\\
35.82	0\\
35.83	0\\
35.84	0\\
35.85	0\\
35.86	0\\
35.87	0\\
35.88	0\\
35.89	0\\
35.9	0\\
35.91	0\\
35.92	0\\
35.93	0\\
35.94	0\\
35.95	0\\
35.96	0\\
35.97	0\\
35.98	0\\
35.99	0\\
36	0\\
36.01	0\\
36.02	0\\
36.03	0\\
36.04	0\\
36.05	0\\
36.06	0\\
36.07	0\\
36.08	0\\
36.09	0\\
36.1	0\\
36.11	0\\
36.12	0\\
36.13	0\\
36.14	0\\
36.15	0\\
36.16	0\\
36.17	0\\
36.18	0\\
36.19	0\\
36.2	0\\
36.21	0\\
36.22	0\\
36.23	0\\
36.24	0\\
36.25	0\\
36.26	0\\
36.27	0\\
36.28	0\\
36.29	0\\
36.3	0\\
36.31	0\\
36.32	0\\
36.33	0\\
36.34	0\\
36.35	0\\
36.36	0\\
36.37	0\\
36.38	0\\
36.39	0\\
36.4	0\\
36.41	0\\
36.42	0\\
36.43	0\\
36.44	0\\
36.45	0\\
36.46	0\\
36.47	0\\
36.48	0\\
36.49	0\\
36.5	0\\
36.51	0\\
36.52	0\\
36.53	0\\
36.54	0\\
36.55	0\\
36.56	0\\
36.57	0\\
36.58	0\\
36.59	0\\
36.6	0\\
36.61	0\\
36.62	0\\
36.63	0\\
36.64	0\\
36.65	0\\
36.66	0\\
36.67	0\\
36.68	0\\
36.69	0\\
36.7	0\\
36.71	0\\
36.72	0\\
36.73	0\\
36.74	0\\
36.75	0\\
36.76	0\\
36.77	0\\
36.78	0\\
36.79	0\\
36.8	0\\
36.81	0\\
36.82	0\\
36.83	0\\
36.84	0\\
36.85	0\\
36.86	0\\
36.87	0\\
36.88	0\\
36.89	0\\
36.9	0\\
36.91	0\\
36.92	0\\
36.93	0\\
36.94	0\\
36.95	0\\
36.96	0\\
36.97	0\\
36.98	0\\
36.99	0\\
37	0\\
37.01	0\\
37.02	0\\
37.03	0\\
37.04	0\\
37.05	0\\
37.06	0\\
37.07	0\\
37.08	0\\
37.09	0\\
37.1	0\\
37.11	0\\
37.12	0\\
37.13	0\\
37.14	0\\
37.15	0\\
37.16	0\\
37.17	0\\
37.18	0\\
37.19	0\\
37.2	0\\
37.21	0\\
37.22	0\\
37.23	0\\
37.24	0\\
37.25	0\\
37.26	0\\
37.27	0\\
37.28	0\\
37.29	0\\
37.3	0\\
37.31	0\\
37.32	0\\
37.33	0\\
37.34	0\\
37.35	0\\
37.36	0\\
37.37	0\\
37.38	0\\
37.39	0\\
37.4	0\\
37.41	0\\
37.42	0\\
37.43	0\\
37.44	0\\
37.45	0\\
37.46	0\\
37.47	0\\
37.48	0\\
37.49	0\\
37.5	0\\
37.51	0\\
37.52	0\\
37.53	0\\
37.54	0\\
37.55	0\\
37.56	0\\
37.57	0\\
37.58	0\\
37.59	0\\
37.6	0\\
37.61	0\\
37.62	0\\
37.63	0\\
37.64	0\\
37.65	0\\
37.66	0\\
37.67	0\\
37.68	0\\
37.69	0\\
37.7	0\\
37.71	0\\
37.72	0\\
37.73	0\\
37.74	0\\
37.75	0\\
37.76	0\\
37.77	0\\
37.78	0\\
37.79	0\\
37.8	0\\
37.81	0\\
37.82	0\\
37.83	0\\
37.84	0\\
37.85	0\\
37.86	0\\
37.87	0\\
37.88	0\\
37.89	0\\
37.9	0\\
37.91	0\\
37.92	0\\
37.93	0\\
37.94	0\\
37.95	0\\
37.96	0\\
37.97	0\\
37.98	0\\
37.99	0\\
38	0\\
38.01	0\\
38.02	0\\
38.03	0\\
38.04	0\\
38.05	0\\
38.06	0\\
38.07	0\\
38.08	0\\
38.09	0\\
38.1	0\\
38.11	0\\
38.12	0\\
38.13	0\\
38.14	0\\
38.15	0\\
38.16	0\\
38.17	0\\
38.18	0\\
38.19	0\\
38.2	0\\
38.21	0\\
38.22	0\\
38.23	0\\
38.24	0\\
38.25	0\\
38.26	0\\
38.27	0\\
38.28	0\\
38.29	0\\
38.3	0\\
38.31	0\\
38.32	0\\
38.33	0\\
38.34	0\\
38.35	0\\
38.36	0\\
38.37	0\\
38.38	0\\
38.39	0\\
38.4	0\\
38.41	0\\
38.42	0\\
38.43	0\\
38.44	0\\
38.45	0\\
38.46	0\\
38.47	0\\
38.48	0\\
38.49	0\\
38.5	0\\
38.51	0\\
38.52	0\\
38.53	0\\
38.54	0\\
38.55	0\\
38.56	0\\
38.57	0\\
38.58	0\\
38.59	0\\
38.6	0\\
38.61	0\\
38.62	0\\
38.63	0\\
38.64	0\\
38.65	0\\
38.66	0\\
38.67	0\\
38.68	0\\
38.69	0\\
38.7	0\\
38.71	0\\
38.72	0\\
38.73	0\\
38.74	0\\
38.75	0\\
38.76	0\\
38.77	0\\
38.78	0\\
38.79	0\\
38.8	0\\
38.81	0\\
38.82	0\\
38.83	0\\
38.84	0\\
38.85	0\\
38.86	0\\
38.87	0\\
38.88	0\\
38.89	0\\
38.9	0\\
38.91	0\\
38.92	0\\
38.93	0\\
38.94	0\\
38.95	0\\
38.96	0\\
38.97	0\\
38.98	0\\
38.99	0\\
39	0\\
39.01	0\\
39.02	0\\
39.03	0\\
39.04	0\\
39.05	0\\
39.06	0\\
39.07	0\\
39.08	0\\
39.09	0\\
39.1	0\\
39.11	0\\
39.12	0\\
39.13	0\\
39.14	0\\
39.15	0\\
39.16	0\\
39.17	0\\
39.18	0\\
39.19	0\\
39.2	0\\
39.21	0\\
39.22	0\\
39.23	0\\
39.24	0\\
39.25	0\\
39.26	0\\
39.27	0\\
39.28	0\\
39.29	0\\
39.3	0\\
39.31	0\\
39.32	0\\
39.33	0\\
39.34	0\\
39.35	0\\
39.36	0\\
39.37	0\\
39.38	0\\
39.39	0\\
39.4	0\\
39.41	0\\
39.42	0\\
39.43	0\\
39.44	0\\
39.45	0\\
39.46	0\\
39.47	0\\
39.48	0\\
39.49	0\\
39.5	0\\
39.51	0\\
39.52	0\\
39.53	0\\
39.54	0\\
39.55	0\\
39.56	0\\
39.57	0\\
39.58	0\\
39.59	0\\
39.6	0\\
39.61	0\\
39.62	0\\
39.63	0\\
39.64	0\\
39.65	0\\
39.66	0\\
39.67	0\\
39.68	0\\
39.69	0\\
39.7	0\\
39.71	0\\
39.72	0\\
39.73	0\\
39.74	0\\
39.75	0\\
39.76	0\\
39.77	0\\
39.78	0\\
39.79	0\\
39.8	0\\
39.81	0\\
39.82	0\\
39.83	0\\
39.84	0\\
39.85	0\\
39.86	0\\
39.87	0\\
39.88	0\\
39.89	0\\
39.9	0\\
39.91	0\\
39.92	0\\
39.93	0\\
39.94	0\\
39.95	0\\
39.96	0\\
39.97	0\\
39.98	0\\
39.99	0\\
40	0\\
40.01	0\\
};
\addplot [color=green,solid,forget plot]
  table[row sep=crcr]{%
40.01	0\\
40.02	0\\
40.03	0\\
40.04	0\\
40.05	0\\
40.06	0\\
40.07	0\\
40.08	0\\
40.09	0\\
40.1	0\\
40.11	0\\
40.12	0\\
40.13	0\\
40.14	0\\
40.15	0\\
40.16	0\\
40.17	0\\
40.18	0\\
40.19	0\\
40.2	0\\
40.21	0\\
40.22	0\\
40.23	0\\
40.24	0\\
40.25	0\\
40.26	0\\
40.27	0\\
40.28	0\\
40.29	0\\
40.3	0\\
40.31	0\\
40.32	0\\
40.33	0\\
40.34	0\\
40.35	0\\
40.36	0\\
40.37	0\\
40.38	0\\
40.39	0\\
40.4	0\\
40.41	0\\
40.42	0\\
40.43	0\\
40.44	0\\
40.45	0\\
40.46	0\\
40.47	0\\
40.48	0\\
40.49	0\\
40.5	0\\
40.51	0\\
40.52	0\\
40.53	0\\
40.54	0\\
40.55	0\\
40.56	0\\
40.57	0\\
40.58	0\\
40.59	0\\
40.6	0\\
40.61	0\\
40.62	0\\
40.63	0\\
40.64	0\\
40.65	0\\
40.66	0\\
40.67	0\\
40.68	0\\
40.69	0\\
40.7	0\\
40.71	0\\
40.72	0\\
40.73	0\\
40.74	0\\
40.75	0\\
40.76	0\\
40.77	0\\
40.78	0\\
40.79	0\\
40.8	0\\
40.81	0\\
40.82	0\\
40.83	0\\
40.84	0\\
40.85	0\\
40.86	0\\
40.87	0\\
40.88	0\\
40.89	0\\
40.9	0\\
40.91	0\\
40.92	0\\
40.93	0\\
40.94	0\\
40.95	0\\
40.96	0\\
40.97	0\\
40.98	0\\
40.99	0\\
41	0\\
41.01	0\\
41.02	0\\
41.03	0\\
41.04	0\\
41.05	0\\
41.06	0\\
41.07	0\\
41.08	0\\
41.09	0\\
41.1	0\\
41.11	0\\
41.12	0\\
41.13	0\\
41.14	0\\
41.15	0\\
41.16	0\\
41.17	0\\
41.18	0\\
41.19	0\\
41.2	0\\
41.21	0\\
41.22	0\\
41.23	0\\
41.24	0\\
41.25	0\\
41.26	0\\
41.27	0\\
41.28	0\\
41.29	0\\
41.3	0\\
41.31	0\\
41.32	0\\
41.33	0\\
41.34	0\\
41.35	0\\
41.36	0\\
41.37	0\\
41.38	0\\
41.39	0\\
41.4	0\\
41.41	0\\
41.42	0\\
41.43	0\\
41.44	0\\
41.45	0\\
41.46	0\\
41.47	0\\
41.48	0\\
41.49	0\\
41.5	0\\
41.51	0\\
41.52	0\\
41.53	0\\
41.54	0\\
41.55	0\\
41.56	0\\
41.57	0\\
41.58	0\\
41.59	0\\
41.6	0\\
41.61	0\\
41.62	0\\
41.63	0\\
41.64	0\\
41.65	0\\
41.66	0\\
41.67	0\\
41.68	0\\
41.69	0\\
41.7	0\\
41.71	0\\
41.72	0\\
41.73	0\\
41.74	0\\
41.75	0\\
41.76	0\\
41.77	0\\
41.78	0\\
41.79	0\\
41.8	0\\
41.81	0\\
41.82	0\\
41.83	0\\
41.84	0\\
41.85	0\\
41.86	0\\
41.87	0\\
41.88	0\\
41.89	0\\
41.9	0\\
41.91	0\\
41.92	0\\
41.93	0\\
41.94	0\\
41.95	0\\
41.96	0\\
41.97	0\\
41.98	0\\
41.99	0\\
42	0\\
42.01	0\\
42.02	0\\
42.03	0\\
42.04	0\\
42.05	0\\
42.06	0\\
42.07	0\\
42.08	0\\
42.09	0\\
42.1	0\\
42.11	0\\
42.12	0\\
42.13	0\\
42.14	0\\
42.15	0\\
42.16	0\\
42.17	0\\
42.18	0\\
42.19	0\\
42.2	0\\
42.21	0\\
42.22	0\\
42.23	0\\
42.24	0\\
42.25	0\\
42.26	0\\
42.27	0\\
42.28	0\\
42.29	0\\
42.3	0\\
42.31	0\\
42.32	0\\
42.33	0\\
42.34	0\\
42.35	0\\
42.36	0\\
42.37	0\\
42.38	0\\
42.39	0\\
42.4	0\\
42.41	0\\
42.42	0\\
42.43	0\\
42.44	0\\
42.45	0\\
42.46	0\\
42.47	0\\
42.48	0\\
42.49	0\\
42.5	0\\
42.51	0\\
42.52	0\\
42.53	0\\
42.54	0\\
42.55	0\\
42.56	0\\
42.57	0\\
42.58	0\\
42.59	0\\
42.6	0\\
42.61	0\\
42.62	0\\
42.63	0\\
42.64	0\\
42.65	0\\
42.66	0\\
42.67	0\\
42.68	0\\
42.69	0\\
42.7	0\\
42.71	0\\
42.72	0\\
42.73	0\\
42.74	0\\
42.75	0\\
42.76	0\\
42.77	0\\
42.78	0\\
42.79	0\\
42.8	0\\
42.81	0\\
42.82	0\\
42.83	0\\
42.84	0\\
42.85	0\\
42.86	0\\
42.87	0\\
42.88	0\\
42.89	0\\
42.9	0\\
42.91	0\\
42.92	0\\
42.93	0\\
42.94	0\\
42.95	0\\
42.96	0\\
42.97	0\\
42.98	0\\
42.99	0\\
43	0\\
43.01	0\\
43.02	0\\
43.03	0\\
43.04	0\\
43.05	0\\
43.06	0\\
43.07	0\\
43.08	0\\
43.09	0\\
43.1	0\\
43.11	0\\
43.12	0\\
43.13	0\\
43.14	0\\
43.15	0\\
43.16	0\\
43.17	0\\
43.18	0\\
43.19	0\\
43.2	0\\
43.21	0\\
43.22	0\\
43.23	0\\
43.24	0\\
43.25	0\\
43.26	0\\
43.27	0\\
43.28	0\\
43.29	0\\
43.3	0\\
43.31	0\\
43.32	0\\
43.33	0\\
43.34	0\\
43.35	0\\
43.36	0\\
43.37	0\\
43.38	0\\
43.39	0\\
43.4	0\\
43.41	0\\
43.42	0\\
43.43	0\\
43.44	0\\
43.45	0\\
43.46	0\\
43.47	0\\
43.48	0\\
43.49	0\\
43.5	0\\
43.51	0\\
43.52	0\\
43.53	0\\
43.54	0\\
43.55	0\\
43.56	0\\
43.57	0\\
43.58	0\\
43.59	0\\
43.6	0\\
43.61	0\\
43.62	0\\
43.63	0\\
43.64	0\\
43.65	0\\
43.66	0\\
43.67	0\\
43.68	0\\
43.69	0\\
43.7	0\\
43.71	0\\
43.72	0\\
43.73	0\\
43.74	0\\
43.75	0\\
43.76	0\\
43.77	0\\
43.78	0\\
43.79	0\\
43.8	0\\
43.81	0\\
43.82	0\\
43.83	0\\
43.84	0\\
43.85	0\\
43.86	0\\
43.87	0\\
43.88	0\\
43.89	0\\
43.9	0\\
43.91	0\\
43.92	0\\
43.93	0\\
43.94	0\\
43.95	0\\
43.96	0\\
43.97	0\\
43.98	0\\
43.99	0\\
44	0\\
44.01	0\\
44.02	0\\
44.03	0\\
44.04	0\\
44.05	0\\
44.06	0\\
44.07	0\\
44.08	0\\
44.09	0\\
44.1	0\\
44.11	0\\
44.12	0\\
44.13	0\\
44.14	0\\
44.15	0\\
44.16	0\\
44.17	0\\
44.18	0\\
44.19	0\\
44.2	0\\
44.21	0\\
44.22	0\\
44.23	0\\
44.24	0\\
44.25	0\\
44.26	0\\
44.27	0\\
44.28	0\\
44.29	0\\
44.3	0\\
44.31	0\\
44.32	0\\
44.33	0\\
44.34	0\\
44.35	0\\
44.36	0\\
44.37	0\\
44.38	0\\
44.39	0\\
44.4	0\\
44.41	0\\
44.42	0\\
44.43	0\\
44.44	0\\
44.45	0\\
44.46	0\\
44.47	0\\
44.48	0\\
44.49	0\\
44.5	0\\
44.51	0\\
44.52	0\\
44.53	0\\
44.54	0\\
44.55	0\\
44.56	0\\
44.57	0\\
44.58	0\\
44.59	0\\
44.6	0\\
44.61	0\\
44.62	0\\
44.63	0\\
44.64	0\\
44.65	0\\
44.66	0\\
44.67	0\\
44.68	0\\
44.69	0\\
44.7	0\\
44.71	0\\
44.72	0\\
44.73	0\\
44.74	0\\
44.75	0\\
44.76	0\\
44.77	0\\
44.78	0\\
44.79	0\\
44.8	0\\
44.81	0\\
44.82	0\\
44.83	0\\
44.84	0\\
44.85	0\\
44.86	0\\
44.87	0\\
44.88	0\\
44.89	0\\
44.9	0\\
44.91	0\\
44.92	0\\
44.93	0\\
44.94	0\\
44.95	0\\
44.96	0\\
44.97	0\\
44.98	0\\
44.99	0\\
45	0\\
45.01	0\\
45.02	0\\
45.03	0\\
45.04	0\\
45.05	0\\
45.06	0\\
45.07	0\\
45.08	0\\
45.09	0\\
45.1	0\\
45.11	0\\
45.12	0\\
45.13	0\\
45.14	0\\
45.15	0\\
45.16	0\\
45.17	0\\
45.18	0\\
45.19	0\\
45.2	0\\
45.21	0\\
45.22	0\\
45.23	0\\
45.24	0\\
45.25	0\\
45.26	0\\
45.27	0\\
45.28	0\\
45.29	0\\
45.3	0\\
45.31	0\\
45.32	0\\
45.33	0\\
45.34	0\\
45.35	0\\
45.36	0\\
45.37	0\\
45.38	0\\
45.39	0\\
45.4	0\\
45.41	0\\
45.42	0\\
45.43	0\\
45.44	0\\
45.45	0\\
45.46	0\\
45.47	0\\
45.48	0\\
45.49	0\\
45.5	0\\
45.51	0\\
45.52	0\\
45.53	0\\
45.54	0\\
45.55	0\\
45.56	0\\
45.57	0\\
45.58	0\\
45.59	0\\
45.6	0\\
45.61	0\\
45.62	0\\
45.63	0\\
45.64	0\\
45.65	0\\
45.66	0\\
45.67	0\\
45.68	0\\
45.69	0\\
45.7	0\\
45.71	0\\
45.72	0\\
45.73	0\\
45.74	0\\
45.75	0\\
45.76	0\\
45.77	0\\
45.78	0\\
45.79	0\\
45.8	0\\
45.81	0\\
45.82	0\\
45.83	0\\
45.84	0\\
45.85	0\\
45.86	0\\
45.87	0\\
45.88	0\\
45.89	0\\
45.9	0\\
45.91	0\\
45.92	0\\
45.93	0\\
45.94	0\\
45.95	0\\
45.96	0\\
45.97	0\\
45.98	0\\
45.99	0\\
46	0\\
46.01	0\\
46.02	0\\
46.03	0\\
46.04	0\\
46.05	0\\
46.06	0\\
46.07	0\\
46.08	0\\
46.09	0\\
46.1	0\\
46.11	0\\
46.12	0\\
46.13	0\\
46.14	0\\
46.15	0\\
46.16	0\\
46.17	0\\
46.18	0\\
46.19	0\\
46.2	0\\
46.21	0\\
46.22	0\\
46.23	0\\
46.24	0\\
46.25	0\\
46.26	0\\
46.27	0\\
46.28	0\\
46.29	0\\
46.3	0\\
46.31	0\\
46.32	0\\
46.33	0\\
46.34	0\\
46.35	0\\
46.36	0\\
46.37	0\\
46.38	0\\
46.39	0\\
46.4	0\\
46.41	0\\
46.42	0\\
46.43	0\\
46.44	0\\
46.45	0\\
46.46	0\\
46.47	0\\
46.48	0\\
46.49	0\\
46.5	0\\
46.51	0\\
46.52	0\\
46.53	0\\
46.54	0\\
46.55	0\\
46.56	0\\
46.57	0\\
46.58	0\\
46.59	0\\
46.6	0\\
46.61	0\\
46.62	0\\
46.63	0\\
46.64	0\\
46.65	0\\
46.66	0\\
46.67	0\\
46.68	0\\
46.69	0\\
46.7	0\\
46.71	0\\
46.72	0\\
46.73	0\\
46.74	0\\
46.75	0\\
46.76	0\\
46.77	0\\
46.78	0\\
46.79	0\\
46.8	0\\
46.81	0\\
46.82	0\\
46.83	0\\
46.84	0\\
46.85	0\\
46.86	0\\
46.87	0\\
46.88	0\\
46.89	0\\
46.9	0\\
46.91	0\\
46.92	0\\
46.93	0\\
46.94	0\\
46.95	0\\
46.96	0\\
46.97	0\\
46.98	0\\
46.99	0\\
47	0\\
47.01	0\\
47.02	0\\
47.03	0\\
47.04	0\\
47.05	0\\
47.06	0\\
47.07	0\\
47.08	0\\
47.09	0\\
47.1	0\\
47.11	0\\
47.12	0\\
47.13	0\\
47.14	0\\
47.15	0\\
47.16	0\\
47.17	0\\
47.18	0\\
47.19	0\\
47.2	0\\
47.21	0\\
47.22	0\\
47.23	0\\
47.24	0\\
47.25	0\\
47.26	0\\
47.27	0\\
47.28	0\\
47.29	0\\
47.3	0\\
47.31	0\\
47.32	0\\
47.33	0\\
47.34	0\\
47.35	0\\
47.36	0\\
47.37	0\\
47.38	0\\
47.39	0\\
47.4	0\\
47.41	0\\
47.42	0\\
47.43	0\\
47.44	0\\
47.45	0\\
47.46	0\\
47.47	0\\
47.48	0\\
47.49	0\\
47.5	0\\
47.51	0\\
47.52	0\\
47.53	0\\
47.54	0\\
47.55	0\\
47.56	0\\
47.57	0\\
47.58	0\\
47.59	0\\
47.6	0\\
47.61	0\\
47.62	0\\
47.63	0\\
47.64	0\\
47.65	0\\
47.66	0\\
47.67	0\\
47.68	0\\
47.69	0\\
47.7	0\\
47.71	0\\
47.72	0\\
47.73	0\\
47.74	0\\
47.75	0\\
47.76	0\\
47.77	0\\
47.78	0\\
47.79	0\\
47.8	0\\
47.81	0\\
47.82	0\\
47.83	0\\
47.84	0\\
47.85	0\\
47.86	0\\
47.87	0\\
47.88	0\\
47.89	0\\
47.9	0\\
47.91	0\\
47.92	0\\
47.93	0\\
47.94	0\\
47.95	0\\
47.96	0\\
47.97	0\\
47.98	0\\
47.99	0\\
48	0\\
48.01	0\\
48.02	0\\
48.03	0\\
48.04	0\\
48.05	0\\
48.06	0\\
48.07	0\\
48.08	0\\
48.09	0\\
48.1	0\\
48.11	0\\
48.12	0\\
48.13	0\\
48.14	0\\
48.15	0\\
48.16	0\\
48.17	0\\
48.18	0\\
48.19	0\\
48.2	0\\
48.21	0\\
48.22	0\\
48.23	0\\
48.24	0\\
48.25	0\\
48.26	0\\
48.27	0\\
48.28	0\\
48.29	0\\
48.3	0\\
48.31	0\\
48.32	0\\
48.33	0\\
48.34	0\\
48.35	0\\
48.36	0\\
48.37	0\\
48.38	0\\
48.39	0\\
48.4	0\\
48.41	0\\
48.42	0\\
48.43	0\\
48.44	0\\
48.45	0\\
48.46	0\\
48.47	0\\
48.48	0\\
48.49	0\\
48.5	0\\
48.51	0\\
48.52	0\\
48.53	0\\
48.54	0\\
48.55	0\\
48.56	0\\
48.57	0\\
48.58	0\\
48.59	0\\
48.6	0\\
48.61	0\\
48.62	0\\
48.63	0\\
48.64	0\\
48.65	0\\
48.66	0\\
48.67	0\\
48.68	0\\
48.69	0\\
48.7	0\\
48.71	0\\
48.72	0\\
48.73	0\\
48.74	0\\
48.75	0\\
48.76	0\\
48.77	0\\
48.78	0\\
48.79	0\\
48.8	0\\
48.81	0\\
48.82	0\\
48.83	0\\
48.84	0\\
48.85	0\\
48.86	0\\
48.87	0\\
48.88	0\\
48.89	0\\
48.9	0\\
48.91	0\\
48.92	0\\
48.93	0\\
48.94	0\\
48.95	0\\
48.96	0\\
48.97	0\\
48.98	0\\
48.99	0\\
49	0\\
49.01	0\\
49.02	0\\
49.03	0\\
49.04	0\\
49.05	0\\
49.06	0\\
49.07	0\\
49.08	0\\
49.09	0\\
49.1	0\\
49.11	0\\
49.12	0\\
49.13	0\\
49.14	0\\
49.15	0\\
49.16	0\\
49.17	0\\
49.18	0\\
49.19	0\\
49.2	0\\
49.21	0\\
49.22	0\\
49.23	0\\
49.24	0\\
49.25	0\\
49.26	0\\
49.27	0\\
49.28	0\\
49.29	0\\
49.3	0\\
49.31	0\\
49.32	0\\
49.33	0\\
49.34	0\\
49.35	0\\
49.36	0\\
49.37	0\\
49.38	0\\
49.39	0\\
49.4	0\\
49.41	0\\
49.42	0\\
49.43	0\\
49.44	0\\
49.45	0\\
49.46	0\\
49.47	0\\
49.48	0\\
49.49	0\\
49.5	0\\
49.51	0\\
49.52	0\\
49.53	0\\
49.54	0\\
49.55	0\\
49.56	0\\
49.57	0\\
49.58	0\\
49.59	0\\
49.6	0\\
49.61	0\\
49.62	0\\
49.63	0\\
49.64	0\\
49.65	0\\
49.66	0\\
49.67	0\\
49.68	0\\
49.69	0\\
49.7	0\\
49.71	0\\
49.72	0\\
49.73	0\\
49.74	0\\
49.75	0\\
49.76	0\\
49.77	0\\
49.78	0\\
49.79	0\\
49.8	0\\
49.81	0\\
49.82	0\\
49.83	0\\
49.84	0\\
49.85	0\\
49.86	0\\
49.87	0\\
49.88	0\\
49.89	0\\
49.9	0\\
49.91	0\\
49.92	0\\
49.93	0\\
49.94	0\\
49.95	0\\
49.96	0\\
49.97	0\\
49.98	0\\
49.99	0\\
50	0\\
50.01	0\\
50.02	0\\
50.03	0\\
50.04	0\\
50.05	0\\
50.06	0\\
50.07	0\\
50.08	0\\
50.09	0\\
50.1	0\\
50.11	0\\
50.12	0\\
50.13	0\\
50.14	0\\
50.15	0\\
50.16	0\\
50.17	0\\
50.18	0\\
50.19	0\\
50.2	0\\
50.21	0\\
50.22	0\\
50.23	0\\
50.24	0\\
50.25	0\\
50.26	0\\
50.27	0\\
50.28	0\\
50.29	0\\
50.3	0\\
50.31	0\\
50.32	0\\
50.33	0\\
50.34	0\\
50.35	0\\
50.36	0\\
50.37	0\\
50.38	0\\
50.39	0\\
50.4	0\\
50.41	0\\
50.42	0\\
50.43	0\\
50.44	0\\
50.45	0\\
50.46	0\\
50.47	0\\
50.48	0\\
50.49	0\\
50.5	0\\
50.51	0\\
50.52	0\\
50.53	0\\
50.54	0\\
50.55	0\\
50.56	0\\
50.57	0\\
50.58	0\\
50.59	0\\
50.6	0\\
50.61	0\\
50.62	0\\
50.63	0\\
50.64	0\\
50.65	0\\
50.66	0\\
50.67	0\\
50.68	0\\
50.69	0\\
50.7	0\\
50.71	0\\
50.72	0\\
50.73	0\\
50.74	0\\
50.75	0\\
50.76	0\\
50.77	0\\
50.78	0\\
50.79	0\\
50.8	0\\
50.81	0\\
50.82	0\\
50.83	0\\
50.84	0\\
50.85	0\\
50.86	0\\
50.87	0\\
50.88	0\\
50.89	0\\
50.9	0\\
50.91	0\\
50.92	0\\
50.93	0\\
50.94	0\\
50.95	0\\
50.96	0\\
50.97	0\\
50.98	0\\
50.99	0\\
51	0\\
51.01	0\\
51.02	0\\
51.03	0\\
51.04	0\\
51.05	0\\
51.06	0\\
51.07	0\\
51.08	0\\
51.09	0\\
51.1	0\\
51.11	0\\
51.12	0\\
51.13	0\\
51.14	0\\
51.15	0\\
51.16	0\\
51.17	0\\
51.18	0\\
51.19	0\\
51.2	0\\
51.21	0\\
51.22	0\\
51.23	0\\
51.24	0\\
51.25	0\\
51.26	0\\
51.27	0\\
51.28	0\\
51.29	0\\
51.3	0\\
51.31	0\\
51.32	0\\
51.33	0\\
51.34	0\\
51.35	0\\
51.36	0\\
51.37	0\\
51.38	0\\
51.39	0\\
51.4	0\\
51.41	0\\
51.42	0\\
51.43	0\\
51.44	0\\
51.45	0\\
51.46	0\\
51.47	0\\
51.48	0\\
51.49	0\\
51.5	0\\
51.51	0\\
51.52	0\\
51.53	0\\
51.54	0\\
51.55	0\\
51.56	0\\
51.57	0\\
51.58	0\\
51.59	0\\
51.6	0\\
51.61	0\\
51.62	0\\
51.63	0\\
51.64	0\\
51.65	0\\
51.66	0\\
51.67	0\\
51.68	0\\
51.69	0\\
51.7	0\\
51.71	0\\
51.72	0\\
51.73	0\\
51.74	0\\
51.75	0\\
51.76	0\\
51.77	0\\
51.78	0\\
51.79	0\\
51.8	0\\
51.81	0\\
51.82	0\\
51.83	0\\
51.84	0\\
51.85	0\\
51.86	0\\
51.87	0\\
51.88	0\\
51.89	0\\
51.9	0\\
51.91	0\\
51.92	0\\
51.93	0\\
51.94	0\\
51.95	0\\
51.96	0\\
51.97	0\\
51.98	0\\
51.99	0\\
52	0\\
52.01	0\\
52.02	0\\
52.03	0\\
52.04	0\\
52.05	0\\
52.06	0\\
52.07	0\\
52.08	0\\
52.09	0\\
52.1	0\\
52.11	0\\
52.12	0\\
52.13	0\\
52.14	0\\
52.15	0\\
52.16	0\\
52.17	0\\
52.18	0\\
52.19	0\\
52.2	0\\
52.21	0\\
52.22	0\\
52.23	0\\
52.24	0\\
52.25	0\\
52.26	0\\
52.27	0\\
52.28	0\\
52.29	0\\
52.3	0\\
52.31	0\\
52.32	0\\
52.33	0\\
52.34	0\\
52.35	0\\
52.36	0\\
52.37	0\\
52.38	0\\
52.39	0\\
52.4	0\\
52.41	0\\
52.42	0\\
52.43	0\\
52.44	0\\
52.45	0\\
52.46	0\\
52.47	0\\
52.48	0\\
52.49	0\\
52.5	0\\
52.51	0\\
52.52	0\\
52.53	0\\
52.54	0\\
52.55	0\\
52.56	0\\
52.57	0\\
52.58	0\\
52.59	0\\
52.6	0\\
52.61	0\\
52.62	0\\
52.63	0\\
52.64	0\\
52.65	0\\
52.66	0\\
52.67	0\\
52.68	0\\
52.69	0\\
52.7	0\\
52.71	0\\
52.72	0\\
52.73	0\\
52.74	0\\
52.75	0\\
52.76	0\\
52.77	0\\
52.78	0\\
52.79	0\\
52.8	0\\
52.81	0\\
52.82	0\\
52.83	0\\
52.84	0\\
52.85	0\\
52.86	0\\
52.87	0\\
52.88	0\\
52.89	0\\
52.9	0\\
52.91	0\\
52.92	0\\
52.93	0\\
52.94	0\\
52.95	0\\
52.96	0\\
52.97	0\\
52.98	0\\
52.99	0\\
53	0\\
53.01	0\\
53.02	0\\
53.03	0\\
53.04	0\\
53.05	0\\
53.06	0\\
53.07	0\\
53.08	0\\
53.09	0\\
53.1	0\\
53.11	0\\
53.12	0\\
53.13	0\\
53.14	0\\
53.15	0\\
53.16	0\\
53.17	0\\
53.18	0\\
53.19	0\\
53.2	0\\
53.21	0\\
53.22	0\\
53.23	0\\
53.24	0\\
53.25	0\\
53.26	0\\
53.27	0\\
53.28	0\\
53.29	0\\
53.3	0\\
53.31	0\\
53.32	0\\
53.33	0\\
53.34	0\\
53.35	0\\
53.36	0\\
53.37	0\\
53.38	0\\
53.39	0\\
53.4	0\\
53.41	0\\
53.42	0\\
53.43	0\\
53.44	0\\
53.45	0\\
53.46	0\\
53.47	0\\
53.48	0\\
53.49	0\\
53.5	0\\
53.51	0\\
53.52	0\\
53.53	0\\
53.54	0\\
53.55	0\\
53.56	0\\
53.57	0\\
53.58	0\\
53.59	0\\
53.6	0\\
53.61	0\\
53.62	0\\
53.63	0\\
53.64	0\\
53.65	0\\
53.66	0\\
53.67	0\\
53.68	0\\
53.69	0\\
53.7	0\\
53.71	0\\
53.72	0\\
53.73	0\\
53.74	0\\
53.75	0\\
53.76	0\\
53.77	0\\
53.78	0\\
53.79	0\\
53.8	0\\
53.81	0\\
53.82	0\\
53.83	0\\
53.84	0\\
53.85	0\\
53.86	0\\
53.87	0\\
53.88	0\\
53.89	0\\
53.9	0\\
53.91	0\\
53.92	0\\
53.93	0\\
53.94	0\\
53.95	0\\
53.96	0\\
53.97	0\\
53.98	0\\
53.99	0\\
54	0\\
54.01	0\\
54.02	0\\
54.03	0\\
54.04	0\\
54.05	0\\
54.06	0\\
54.07	0\\
54.08	0\\
54.09	0\\
54.1	0\\
54.11	0\\
54.12	0\\
54.13	0\\
54.14	0\\
54.15	0\\
54.16	0\\
54.17	0\\
54.18	0\\
54.19	0\\
54.2	0\\
54.21	0\\
54.22	0\\
54.23	0\\
54.24	0\\
54.25	0\\
54.26	0\\
54.27	0\\
54.28	0\\
54.29	0\\
54.3	0\\
54.31	0\\
54.32	0\\
54.33	0\\
54.34	0\\
54.35	0\\
54.36	0\\
54.37	0\\
54.38	0\\
54.39	0\\
54.4	0\\
54.41	0\\
54.42	0\\
54.43	0\\
54.44	0\\
54.45	0\\
54.46	0\\
54.47	0\\
54.48	0\\
54.49	0\\
54.5	0\\
54.51	0\\
54.52	0\\
54.53	0\\
54.54	0\\
54.55	0\\
54.56	0\\
54.57	0\\
54.58	0\\
54.59	0\\
54.6	0\\
54.61	0\\
54.62	0\\
54.63	0\\
54.64	0\\
54.65	0\\
54.66	0\\
54.67	0\\
54.68	0\\
54.69	0\\
54.7	0\\
54.71	0\\
54.72	0\\
54.73	0\\
54.74	0\\
54.75	0\\
54.76	0\\
54.77	0\\
54.78	0\\
54.79	0\\
54.8	0\\
54.81	0\\
54.82	0\\
54.83	0\\
54.84	0\\
54.85	0\\
54.86	0\\
54.87	0\\
54.88	0\\
54.89	0\\
54.9	0\\
54.91	0\\
54.92	0\\
54.93	0\\
54.94	0\\
54.95	0\\
54.96	0\\
54.97	0\\
54.98	0\\
54.99	0\\
55	0\\
55.01	0\\
55.02	0\\
55.03	0\\
55.04	0\\
55.05	0\\
55.06	0\\
55.07	0\\
55.08	0\\
55.09	0\\
55.1	0\\
55.11	0\\
55.12	0\\
55.13	0\\
55.14	0\\
55.15	0\\
55.16	0\\
55.17	0\\
55.18	0\\
55.19	0\\
55.2	0\\
55.21	0\\
55.22	0\\
55.23	0\\
55.24	0\\
55.25	0\\
55.26	0\\
55.27	0\\
55.28	0\\
55.29	0\\
55.3	0\\
55.31	0\\
55.32	0\\
55.33	0\\
55.34	0\\
55.35	0\\
55.36	0\\
55.37	0\\
55.38	0\\
55.39	0\\
55.4	0\\
55.41	0\\
55.42	0\\
55.43	0\\
55.44	0\\
55.45	0\\
55.46	0\\
55.47	0\\
55.48	0\\
55.49	0\\
55.5	0\\
55.51	0\\
55.52	0\\
55.53	0\\
55.54	0\\
55.55	0\\
55.56	0\\
55.57	0\\
55.58	0\\
55.59	0\\
55.6	0\\
55.61	0\\
55.62	0\\
55.63	0\\
55.64	0\\
55.65	0\\
55.66	0\\
55.67	0\\
55.68	0\\
55.69	0\\
55.7	0\\
55.71	0\\
55.72	0\\
55.73	0\\
55.74	0\\
55.75	0\\
55.76	0\\
55.77	0\\
55.78	0\\
55.79	0\\
55.8	0\\
55.81	0\\
55.82	0\\
55.83	0\\
55.84	0\\
55.85	0\\
55.86	0\\
55.87	0\\
55.88	0\\
55.89	0\\
55.9	0\\
55.91	0\\
55.92	0\\
55.93	0\\
55.94	0\\
55.95	0\\
55.96	0\\
55.97	0\\
55.98	0\\
55.99	0\\
56	0\\
56.01	0\\
56.02	0\\
56.03	0\\
56.04	0\\
56.05	0\\
56.06	0\\
56.07	0\\
56.08	0\\
56.09	0\\
56.1	0\\
56.11	0\\
56.12	0\\
56.13	0\\
56.14	0\\
56.15	0\\
56.16	0\\
56.17	0\\
56.18	0\\
56.19	0\\
56.2	0\\
56.21	0\\
56.22	0\\
56.23	0\\
56.24	0\\
56.25	0\\
56.26	0\\
56.27	0\\
56.28	0\\
56.29	0\\
56.3	0\\
56.31	0\\
56.32	0\\
56.33	0\\
56.34	0\\
56.35	0\\
56.36	0\\
56.37	0\\
56.38	0\\
56.39	0\\
56.4	0\\
56.41	0\\
56.42	0\\
56.43	0\\
56.44	0\\
56.45	0\\
56.46	0\\
56.47	0\\
56.48	0\\
56.49	0\\
56.5	0\\
56.51	0\\
56.52	0\\
56.53	0\\
56.54	0\\
56.55	0\\
56.56	0\\
56.57	0\\
56.58	0\\
56.59	0\\
56.6	0\\
56.61	0\\
56.62	0\\
56.63	0\\
56.64	0\\
56.65	0\\
56.66	0\\
56.67	0\\
56.68	0\\
56.69	0\\
56.7	0\\
56.71	0\\
56.72	0\\
56.73	0\\
56.74	0\\
56.75	0\\
56.76	0\\
56.77	0\\
56.78	0\\
56.79	0\\
56.8	0\\
56.81	0\\
56.82	0\\
56.83	0\\
56.84	0\\
56.85	0\\
56.86	0\\
56.87	0\\
56.88	0\\
56.89	0\\
56.9	0\\
56.91	0\\
56.92	0\\
56.93	0\\
56.94	0\\
56.95	0\\
56.96	0\\
56.97	0\\
56.98	0\\
56.99	0\\
57	0\\
57.01	0\\
57.02	0\\
57.03	0\\
57.04	0\\
57.05	0\\
57.06	0\\
57.07	0\\
57.08	0\\
57.09	0\\
57.1	0\\
57.11	0\\
57.12	0\\
57.13	0\\
57.14	0\\
57.15	0\\
57.16	0\\
57.17	0\\
57.18	0\\
57.19	0\\
57.2	0\\
57.21	0\\
57.22	0\\
57.23	0\\
57.24	0\\
57.25	0\\
57.26	0\\
57.27	0\\
57.28	0\\
57.29	0\\
57.3	0\\
57.31	0\\
57.32	0\\
57.33	0\\
57.34	0\\
57.35	0\\
57.36	0\\
57.37	0\\
57.38	0\\
57.39	0\\
57.4	0\\
57.41	0\\
57.42	0\\
57.43	0\\
57.44	0\\
57.45	0\\
57.46	0\\
57.47	0\\
57.48	0\\
57.49	0\\
57.5	0\\
57.51	0\\
57.52	0\\
57.53	0\\
57.54	0\\
57.55	0\\
57.56	0\\
57.57	0\\
57.58	0\\
57.59	0\\
57.6	0\\
57.61	0\\
57.62	0\\
57.63	0\\
57.64	0\\
57.65	0\\
57.66	0\\
57.67	0\\
57.68	0\\
57.69	0\\
57.7	0\\
57.71	0\\
57.72	0\\
57.73	0\\
57.74	0\\
57.75	0\\
57.76	0\\
57.77	0\\
57.78	0\\
57.79	0\\
57.8	0\\
57.81	0\\
57.82	0\\
57.83	0\\
57.84	0\\
57.85	0\\
57.86	0\\
57.87	0\\
57.88	0\\
57.89	0\\
57.9	0\\
57.91	0\\
57.92	0\\
57.93	0\\
57.94	0\\
57.95	0\\
57.96	0\\
57.97	0\\
57.98	0\\
57.99	0\\
58	0\\
58.01	0\\
58.02	0\\
58.03	0\\
58.04	0\\
58.05	0\\
58.06	0\\
58.07	0\\
58.08	0\\
58.09	0\\
58.1	0\\
58.11	0\\
58.12	0\\
58.13	0\\
58.14	0\\
58.15	0\\
58.16	0\\
58.17	0\\
58.18	0\\
58.19	0\\
58.2	0\\
58.21	0\\
58.22	0\\
58.23	0\\
58.24	0\\
58.25	0\\
58.26	0\\
58.27	0\\
58.28	0\\
58.29	0\\
58.3	0\\
58.31	0\\
58.32	0\\
58.33	0\\
58.34	0\\
58.35	0\\
58.36	0\\
58.37	0\\
58.38	0\\
58.39	0\\
58.4	0\\
58.41	0\\
58.42	0\\
58.43	0\\
58.44	0\\
58.45	0\\
58.46	0\\
58.47	0\\
58.48	0\\
58.49	0\\
58.5	0\\
58.51	0\\
58.52	0\\
58.53	0\\
58.54	0\\
58.55	0\\
58.56	0\\
58.57	0\\
58.58	0\\
58.59	0\\
58.6	0\\
58.61	0\\
58.62	0\\
58.63	0\\
58.64	0\\
58.65	0\\
58.66	0\\
58.67	0\\
58.68	0\\
58.69	0\\
58.7	0\\
58.71	0\\
58.72	0\\
58.73	0\\
58.74	0\\
58.75	0\\
58.76	0\\
58.77	0\\
58.78	0\\
58.79	0\\
58.8	0\\
58.81	0\\
58.82	0\\
58.83	0\\
58.84	0\\
58.85	0\\
58.86	0\\
58.87	0\\
58.88	0\\
58.89	0\\
58.9	0\\
58.91	0\\
58.92	0\\
58.93	0\\
58.94	0\\
58.95	0\\
58.96	0\\
58.97	0\\
58.98	0\\
58.99	0\\
59	0\\
59.01	0\\
59.02	0\\
59.03	0\\
59.04	0\\
59.05	0\\
59.06	0\\
59.07	0\\
59.08	0\\
59.09	0\\
59.1	0\\
59.11	0\\
59.12	0\\
59.13	0\\
59.14	0\\
59.15	0\\
59.16	0\\
59.17	0\\
59.18	0\\
59.19	0\\
59.2	0\\
59.21	0\\
59.22	0\\
59.23	0\\
59.24	0\\
59.25	0\\
59.26	0\\
59.27	0\\
59.28	0\\
59.29	0\\
59.3	0\\
59.31	0\\
59.32	0\\
59.33	0\\
59.34	0\\
59.35	0\\
59.36	0\\
59.37	0\\
59.38	0\\
59.39	0\\
59.4	0\\
59.41	0\\
59.42	0\\
59.43	0\\
59.44	0\\
59.45	0\\
59.46	0\\
59.47	0\\
59.48	0\\
59.49	0\\
59.5	0\\
59.51	0\\
59.52	0\\
59.53	0\\
59.54	0\\
59.55	0\\
59.56	0\\
59.57	0\\
59.58	0\\
59.59	0\\
59.6	0\\
59.61	0\\
59.62	0\\
59.63	0\\
59.64	0\\
59.65	0\\
59.66	0\\
59.67	0\\
59.68	0\\
59.69	0\\
59.7	0\\
59.71	0\\
59.72	0\\
59.73	0\\
59.74	0\\
59.75	0\\
59.76	0\\
59.77	0\\
59.78	0\\
59.79	0\\
59.8	0\\
59.81	0\\
59.82	0\\
59.83	0\\
59.84	0\\
59.85	0\\
59.86	0\\
59.87	0\\
59.88	0\\
59.89	0\\
59.9	0\\
59.91	0\\
59.92	0\\
59.93	0\\
59.94	0\\
59.95	0\\
59.96	0\\
59.97	0\\
59.98	0\\
59.99	0\\
60	0\\
60.01	0\\
60.02	0\\
60.03	0\\
60.04	0\\
60.05	0\\
60.06	0\\
60.07	0\\
60.08	0\\
60.09	0\\
60.1	0\\
60.11	0\\
60.12	0\\
60.13	0\\
60.14	0\\
60.15	0\\
60.16	0\\
60.17	0\\
60.18	0\\
60.19	0\\
60.2	0\\
60.21	0\\
60.22	0\\
60.23	0\\
60.24	0\\
60.25	0\\
60.26	0\\
60.27	0\\
60.28	0\\
60.29	0\\
60.3	0\\
60.31	0\\
60.32	0\\
60.33	0\\
60.34	0\\
60.35	0\\
60.36	0\\
60.37	0\\
60.38	0\\
60.39	0\\
60.4	0\\
60.41	0\\
60.42	0\\
60.43	0\\
60.44	0\\
60.45	0\\
60.46	0\\
60.47	0\\
60.48	0\\
60.49	0\\
60.5	0\\
60.51	0\\
60.52	0\\
60.53	0\\
60.54	0\\
60.55	0\\
60.56	0\\
60.57	0\\
60.58	0\\
60.59	0\\
60.6	0\\
60.61	0\\
60.62	0\\
60.63	0\\
60.64	0\\
60.65	0\\
60.66	0\\
60.67	0\\
60.68	0\\
60.69	0\\
60.7	0\\
60.71	0\\
60.72	0\\
60.73	0\\
60.74	0\\
60.75	0\\
60.76	0\\
60.77	0\\
60.78	0\\
60.79	0\\
60.8	0\\
60.81	0\\
60.82	0\\
60.83	0\\
60.84	0\\
60.85	0\\
60.86	0\\
60.87	0\\
60.88	0\\
60.89	0\\
60.9	0\\
60.91	0\\
60.92	0\\
60.93	0\\
60.94	0\\
60.95	0\\
60.96	0\\
60.97	0\\
60.98	0\\
60.99	0\\
61	0\\
61.01	0\\
61.02	0\\
61.03	0\\
61.04	0\\
61.05	0\\
61.06	0\\
61.07	0\\
61.08	0\\
61.09	0\\
61.1	0\\
61.11	0\\
61.12	0\\
61.13	0\\
61.14	0\\
61.15	0\\
61.16	0\\
61.17	0\\
61.18	0\\
61.19	0\\
61.2	0\\
61.21	0\\
61.22	0\\
61.23	0\\
61.24	0\\
61.25	0\\
61.26	0\\
61.27	0\\
61.28	0\\
61.29	0\\
61.3	0\\
61.31	0\\
61.32	0\\
61.33	0\\
61.34	0\\
61.35	0\\
61.36	0\\
61.37	0\\
61.38	0\\
61.39	0\\
61.4	0\\
61.41	0\\
61.42	0\\
61.43	0\\
61.44	0\\
61.45	0\\
61.46	0\\
61.47	0\\
61.48	0\\
61.49	0\\
61.5	0\\
61.51	0\\
61.52	0\\
61.53	0\\
61.54	0\\
61.55	0\\
61.56	0\\
61.57	0\\
61.58	0\\
61.59	0\\
61.6	0\\
61.61	0\\
61.62	0\\
61.63	0\\
61.64	0\\
61.65	0\\
61.66	0\\
61.67	0\\
61.68	0\\
61.69	0\\
61.7	0\\
61.71	0\\
61.72	0\\
61.73	0\\
61.74	0\\
61.75	0\\
61.76	0\\
61.77	0\\
61.78	0\\
61.79	0\\
61.8	0\\
61.81	0\\
61.82	0\\
61.83	0\\
61.84	0\\
61.85	0\\
61.86	0\\
61.87	0\\
61.88	0\\
61.89	0\\
61.9	0\\
61.91	0\\
61.92	0\\
61.93	0\\
61.94	0\\
61.95	0\\
61.96	0\\
61.97	0\\
61.98	0\\
61.99	0\\
62	0\\
62.01	0\\
62.02	0\\
62.03	0\\
62.04	0\\
62.05	0\\
62.06	0\\
62.07	0\\
62.08	0\\
62.09	0\\
62.1	0\\
62.11	0\\
62.12	0\\
62.13	0\\
62.14	0\\
62.15	0\\
62.16	0\\
62.17	0\\
62.18	0\\
62.19	0\\
62.2	0\\
62.21	0\\
62.22	0\\
62.23	0\\
62.24	0\\
62.25	0\\
62.26	0\\
62.27	0\\
62.28	0\\
62.29	0\\
62.3	0\\
62.31	0\\
62.32	0\\
62.33	0\\
62.34	0\\
62.35	0\\
62.36	0\\
62.37	0\\
62.38	0\\
62.39	0\\
62.4	0\\
62.41	0\\
62.42	0\\
62.43	0\\
62.44	0\\
62.45	0\\
62.46	0\\
62.47	0\\
62.48	0\\
62.49	0\\
62.5	0\\
62.51	0\\
62.52	0\\
62.53	0\\
62.54	0\\
62.55	0\\
62.56	0\\
62.57	0\\
62.58	0\\
62.59	0\\
62.6	0\\
62.61	0\\
62.62	0\\
62.63	0\\
62.64	0\\
62.65	0\\
62.66	0\\
62.67	0\\
62.68	0\\
62.69	0\\
62.7	0\\
62.71	0\\
62.72	0\\
62.73	0\\
62.74	0\\
62.75	0\\
62.76	0\\
62.77	0\\
62.78	0\\
62.79	0\\
62.8	0\\
62.81	0\\
62.82	0\\
62.83	0\\
62.84	0\\
62.85	0\\
62.86	0\\
62.87	0\\
62.88	0\\
62.89	0\\
62.9	0\\
62.91	0\\
62.92	0\\
62.93	0\\
62.94	0\\
62.95	0\\
62.96	0\\
62.97	0\\
62.98	0\\
62.99	0\\
63	0\\
63.01	0\\
63.02	0\\
63.03	0\\
63.04	0\\
63.05	0\\
63.06	0\\
63.07	0\\
63.08	0\\
63.09	0\\
63.1	0\\
63.11	0\\
63.12	0\\
63.13	0\\
63.14	0\\
63.15	0\\
63.16	0\\
63.17	0\\
63.18	0\\
63.19	0\\
63.2	0\\
63.21	0\\
63.22	0\\
63.23	0\\
63.24	0\\
63.25	0\\
63.26	0\\
63.27	0\\
63.28	0\\
63.29	0\\
63.3	0\\
63.31	0\\
63.32	0\\
63.33	0\\
63.34	0\\
63.35	0\\
63.36	0\\
63.37	0\\
63.38	0\\
63.39	0\\
63.4	0\\
63.41	0\\
63.42	0\\
63.43	0\\
63.44	0\\
63.45	0\\
63.46	0\\
63.47	0\\
63.48	0\\
63.49	0\\
63.5	0\\
63.51	0\\
63.52	0\\
63.53	0\\
63.54	0\\
63.55	0\\
63.56	0\\
63.57	0\\
63.58	0\\
63.59	0\\
63.6	0\\
63.61	0\\
63.62	0\\
63.63	0\\
63.64	0\\
63.65	0\\
63.66	0\\
63.67	0\\
63.68	0\\
63.69	0\\
63.7	0\\
63.71	0\\
63.72	0\\
63.73	0\\
63.74	0\\
63.75	0\\
63.76	0\\
63.77	0\\
63.78	0\\
63.79	0\\
63.8	0\\
63.81	0\\
63.82	0\\
63.83	0\\
63.84	0\\
63.85	0\\
63.86	0\\
63.87	0\\
63.88	0\\
63.89	0\\
63.9	0\\
63.91	0\\
63.92	0\\
63.93	0\\
63.94	0\\
63.95	0\\
63.96	0\\
63.97	0\\
63.98	0\\
63.99	0\\
64	0\\
64.01	0\\
64.02	0\\
64.03	0\\
64.04	0\\
64.05	0\\
64.06	0\\
64.07	0\\
64.08	0\\
64.09	0\\
64.1	0\\
64.11	0\\
64.12	0\\
64.13	0\\
64.14	0\\
64.15	0\\
64.16	0\\
64.17	0\\
64.18	0\\
64.19	0\\
64.2	0\\
64.21	0\\
64.22	0\\
64.23	0\\
64.24	0\\
64.25	0\\
64.26	0\\
64.27	0\\
64.28	0\\
64.29	0\\
64.3	0\\
64.31	0\\
64.32	0\\
64.33	0\\
64.34	0\\
64.35	0\\
64.36	0\\
64.37	0\\
64.38	0\\
64.39	0\\
64.4	0\\
64.41	0\\
64.42	0\\
64.43	0\\
64.44	0\\
64.45	0\\
64.46	0\\
64.47	0\\
64.48	0\\
64.49	0\\
64.5	0\\
64.51	0\\
64.52	0\\
64.53	0\\
64.54	0\\
64.55	0\\
64.56	0\\
64.57	0\\
64.58	0\\
64.59	0\\
64.6	0\\
64.61	0\\
64.62	0\\
64.63	0\\
64.64	0\\
64.65	0\\
64.66	0\\
64.67	0\\
64.68	0\\
64.69	0\\
64.7	0\\
64.71	0\\
64.72	0\\
64.73	0\\
64.74	0\\
64.75	0\\
64.76	0\\
64.77	0\\
64.78	0\\
64.79	0\\
64.8	0\\
64.81	0\\
64.82	0\\
64.83	0\\
64.84	0\\
64.85	0\\
64.86	0\\
64.87	0\\
64.88	0\\
64.89	0\\
64.9	0\\
64.91	0\\
64.92	0\\
64.93	0\\
64.94	0\\
64.95	0\\
64.96	0\\
64.97	0\\
64.98	0\\
64.99	0\\
65	0\\
65.01	0\\
65.02	0\\
65.03	0\\
65.04	0\\
65.05	0\\
65.06	0\\
65.07	0\\
65.08	0\\
65.09	0\\
65.1	0\\
65.11	0\\
65.12	0\\
65.13	0\\
65.14	0\\
65.15	0\\
65.16	0\\
65.17	0\\
65.18	0\\
65.19	0\\
65.2	0\\
65.21	0\\
65.22	0\\
65.23	0\\
65.24	0\\
65.25	0\\
65.26	0\\
65.27	0\\
65.28	0\\
65.29	0\\
65.3	0\\
65.31	0\\
65.32	0\\
65.33	0\\
65.34	0\\
65.35	0\\
65.36	0\\
65.37	0\\
65.38	0\\
65.39	0\\
65.4	0\\
65.41	0\\
65.42	0\\
65.43	0\\
65.44	0\\
65.45	0\\
65.46	0\\
65.47	0\\
65.48	0\\
65.49	0\\
65.5	0\\
65.51	0\\
65.52	0\\
65.53	0\\
65.54	0\\
65.55	0\\
65.56	0\\
65.57	0\\
65.58	0\\
65.59	0\\
65.6	0\\
65.61	0\\
65.62	0\\
65.63	0\\
65.64	0\\
65.65	0\\
65.66	0\\
65.67	0\\
65.68	0\\
65.69	0\\
65.7	0\\
65.71	0\\
65.72	0\\
65.73	0\\
65.74	0\\
65.75	0\\
65.76	0\\
65.77	0\\
65.78	0\\
65.79	0\\
65.8	0\\
65.81	0\\
65.82	0\\
65.83	0\\
65.84	0\\
65.85	0\\
65.86	0\\
65.87	0\\
65.88	0\\
65.89	0\\
65.9	0\\
65.91	0\\
65.92	0\\
65.93	0\\
65.94	0\\
65.95	0\\
65.96	0\\
65.97	0\\
65.98	0\\
65.99	0\\
66	0\\
66.01	0\\
66.02	0\\
66.03	0\\
66.04	0\\
66.05	0\\
66.06	0\\
66.07	0\\
66.08	0\\
66.09	0\\
66.1	0\\
66.11	0\\
66.12	0\\
66.13	0\\
66.14	0\\
66.15	0\\
66.16	0\\
66.17	0\\
66.18	0\\
66.19	0\\
66.2	0\\
66.21	0\\
66.22	0\\
66.23	0\\
66.24	0\\
66.25	0\\
66.26	0\\
66.27	0\\
66.28	0\\
66.29	0\\
66.3	0\\
66.31	0\\
66.32	0\\
66.33	0\\
66.34	0\\
66.35	0\\
66.36	0\\
66.37	0\\
66.38	0\\
66.39	0\\
66.4	0\\
66.41	0\\
66.42	0\\
66.43	0\\
66.44	0\\
66.45	0\\
66.46	0\\
66.47	0\\
66.48	0\\
66.49	0\\
66.5	0\\
66.51	0\\
66.52	0\\
66.53	0\\
66.54	0\\
66.55	0\\
66.56	0\\
66.57	0\\
66.58	0\\
66.59	0\\
66.6	0\\
66.61	0\\
66.62	0\\
66.63	0\\
66.64	0\\
66.65	0\\
66.66	0\\
66.67	0\\
66.68	0\\
66.69	0\\
66.7	0\\
66.71	0\\
66.72	0\\
66.73	0\\
66.74	0\\
66.75	0\\
66.76	0\\
66.77	0\\
66.78	0\\
66.79	0\\
66.8	0\\
66.81	0\\
66.82	0\\
66.83	0\\
66.84	0\\
66.85	0\\
66.86	0\\
66.87	0\\
66.88	0\\
66.89	0\\
66.9	0\\
66.91	0\\
66.92	0\\
66.93	0\\
66.94	0\\
66.95	0\\
66.96	0\\
66.97	0\\
66.98	0\\
66.99	0\\
67	0\\
67.01	0\\
67.02	0\\
67.03	0\\
67.04	0\\
67.05	0\\
67.06	0\\
67.07	0\\
67.08	0\\
67.09	0\\
67.1	0\\
67.11	0\\
67.12	0\\
67.13	0\\
67.14	0\\
67.15	0\\
67.16	0\\
67.17	0\\
67.18	0\\
67.19	0\\
67.2	0\\
67.21	0\\
67.22	0\\
67.23	0\\
67.24	0\\
67.25	0\\
67.26	0\\
67.27	0\\
67.28	0\\
67.29	0\\
67.3	0\\
67.31	0\\
67.32	0\\
67.33	0\\
67.34	0\\
67.35	0\\
67.36	0\\
67.37	0\\
67.38	0\\
67.39	0\\
67.4	0\\
67.41	0\\
67.42	0\\
67.43	0\\
67.44	0\\
67.45	0\\
67.46	0\\
67.47	0\\
67.48	0\\
67.49	0\\
67.5	0\\
67.51	0\\
67.52	0\\
67.53	0\\
67.54	0\\
67.55	0\\
67.56	0\\
67.57	0\\
67.58	0\\
67.59	0\\
67.6	0\\
67.61	0\\
67.62	0\\
67.63	0\\
67.64	0\\
67.65	0\\
67.66	0\\
67.67	0\\
67.68	0\\
67.69	0\\
67.7	0\\
67.71	0\\
67.72	0\\
67.73	0\\
67.74	0\\
67.75	0\\
67.76	0\\
67.77	0\\
67.78	0\\
67.79	0\\
67.8	0\\
67.81	0\\
67.82	0\\
67.83	0\\
67.84	0\\
67.85	0\\
67.86	0\\
67.87	0\\
67.88	0\\
67.89	0\\
67.9	0\\
67.91	0\\
67.92	0\\
67.93	0\\
67.94	0\\
67.95	0\\
67.96	0\\
67.97	0\\
67.98	0\\
67.99	0\\
68	0\\
68.01	0\\
68.02	0\\
68.03	0\\
68.04	0\\
68.05	0\\
68.06	0\\
68.07	0\\
68.08	0\\
68.09	0\\
68.1	0\\
68.11	0\\
68.12	0\\
68.13	0\\
68.14	0\\
68.15	0\\
68.16	0\\
68.17	0\\
68.18	0\\
68.19	0\\
68.2	0\\
68.21	0\\
68.22	0\\
68.23	0\\
68.24	0\\
68.25	0\\
68.26	0\\
68.27	0\\
68.28	0\\
68.29	0\\
68.3	0\\
68.31	0\\
68.32	0\\
68.33	0\\
68.34	0\\
68.35	0\\
68.36	0\\
68.37	0\\
68.38	0\\
68.39	0\\
68.4	0\\
68.41	0\\
68.42	0\\
68.43	0\\
68.44	0\\
68.45	0\\
68.46	0\\
68.47	0\\
68.48	0\\
68.49	0\\
68.5	0\\
68.51	0\\
68.52	0\\
68.53	0\\
68.54	0\\
68.55	0\\
68.56	0\\
68.57	0\\
68.58	0\\
68.59	0\\
68.6	0\\
68.61	0\\
68.62	0\\
68.63	0\\
68.64	0\\
68.65	0\\
68.66	0\\
68.67	0\\
68.68	0\\
68.69	0\\
68.7	0\\
68.71	0\\
68.72	0\\
68.73	0\\
68.74	0\\
68.75	0\\
68.76	0\\
68.77	0\\
68.78	0\\
68.79	0\\
68.8	0\\
68.81	0\\
68.82	0\\
68.83	0\\
68.84	0\\
68.85	0\\
68.86	0\\
68.87	0\\
68.88	0\\
68.89	0\\
68.9	0\\
68.91	0\\
68.92	0\\
68.93	0\\
68.94	0\\
68.95	0\\
68.96	0\\
68.97	0\\
68.98	0\\
68.99	0\\
69	0\\
69.01	0\\
69.02	0\\
69.03	0\\
69.04	0\\
69.05	0\\
69.06	0\\
69.07	0\\
69.08	0\\
69.09	0\\
69.1	0\\
69.11	0\\
69.12	0\\
69.13	0\\
69.14	0\\
69.15	0\\
69.16	0\\
69.17	0\\
69.18	0\\
69.19	0\\
69.2	0\\
69.21	0\\
69.22	0\\
69.23	0\\
69.24	0\\
69.25	0\\
69.26	0\\
69.27	0\\
69.28	0\\
69.29	0\\
69.3	0\\
69.31	0\\
69.32	0\\
69.33	0\\
69.34	0\\
69.35	0\\
69.36	0\\
69.37	0\\
69.38	0\\
69.39	0\\
69.4	0\\
69.41	0\\
69.42	0\\
69.43	0\\
69.44	0\\
69.45	0\\
69.46	0\\
69.47	0\\
69.48	0\\
69.49	0\\
69.5	0\\
69.51	0\\
69.52	0\\
69.53	0\\
69.54	0\\
69.55	0\\
69.56	0\\
69.57	0\\
69.58	0\\
69.59	0\\
69.6	0\\
69.61	0\\
69.62	0\\
69.63	0\\
69.64	0\\
69.65	0\\
69.66	0\\
69.67	0\\
69.68	0\\
69.69	0\\
69.7	0\\
69.71	0\\
69.72	0\\
69.73	0\\
69.74	0\\
69.75	0\\
69.76	0\\
69.77	0\\
69.78	0\\
69.79	0\\
69.8	0\\
69.81	0\\
69.82	0\\
69.83	0\\
69.84	0\\
69.85	0\\
69.86	0\\
69.87	0\\
69.88	0\\
69.89	0\\
69.9	0\\
69.91	0\\
69.92	0\\
69.93	0\\
69.94	0\\
69.95	0\\
69.96	0\\
69.97	0\\
69.98	0\\
69.99	0\\
70	0\\
70.01	0\\
70.02	0\\
70.03	0\\
70.04	0\\
70.05	0\\
70.06	0\\
70.07	0\\
70.08	0\\
70.09	0\\
70.1	0\\
70.11	0\\
70.12	0\\
70.13	0\\
70.14	0\\
70.15	0\\
70.16	0\\
70.17	0\\
70.18	0\\
70.19	0\\
70.2	0\\
70.21	0\\
70.22	0\\
70.23	0\\
70.24	0\\
70.25	0\\
70.26	0\\
70.27	0\\
70.28	0\\
70.29	0\\
70.3	0\\
70.31	0\\
70.32	0\\
70.33	0\\
70.34	0\\
70.35	0\\
70.36	0\\
70.37	0\\
70.38	0\\
70.39	0\\
70.4	0\\
70.41	0\\
70.42	0\\
70.43	0\\
70.44	0\\
70.45	0\\
70.46	0\\
70.47	0\\
70.48	0\\
70.49	0\\
70.5	0\\
70.51	0\\
70.52	0\\
70.53	0\\
70.54	0\\
70.55	0\\
70.56	0\\
70.57	0\\
70.58	0\\
70.59	0\\
70.6	0\\
70.61	0\\
70.62	0\\
70.63	0\\
70.64	0\\
70.65	0\\
70.66	0\\
70.67	0\\
70.68	0\\
70.69	0\\
70.7	0\\
70.71	0\\
70.72	0\\
70.73	0\\
70.74	0\\
70.75	0\\
70.76	0\\
70.77	0\\
70.78	0\\
70.79	0\\
70.8	0\\
70.81	0\\
70.82	0\\
70.83	0\\
70.84	0\\
70.85	0\\
70.86	0\\
70.87	0\\
70.88	0\\
70.89	0\\
70.9	0\\
70.91	0\\
70.92	0\\
70.93	0\\
70.94	0\\
70.95	0\\
70.96	0\\
70.97	0\\
70.98	0\\
70.99	0\\
71	0\\
71.01	0\\
71.02	0\\
71.03	0\\
71.04	0\\
71.05	0\\
71.06	0\\
71.07	0\\
71.08	0\\
71.09	0\\
71.1	0\\
71.11	0\\
71.12	0\\
71.13	0\\
71.14	0\\
71.15	0\\
71.16	0\\
71.17	0\\
71.18	0\\
71.19	0\\
71.2	0\\
71.21	0\\
71.22	0\\
71.23	0\\
71.24	0\\
71.25	0\\
71.26	0\\
71.27	0\\
71.28	0\\
71.29	0\\
71.3	0\\
71.31	0\\
71.32	0\\
71.33	0\\
71.34	0\\
71.35	0\\
71.36	0\\
71.37	0\\
71.38	0\\
71.39	0\\
71.4	0\\
71.41	0\\
71.42	0\\
71.43	0\\
71.44	0\\
71.45	0\\
71.46	0\\
71.47	0\\
71.48	0\\
71.49	0\\
71.5	0\\
71.51	0\\
71.52	0\\
71.53	0\\
71.54	0\\
71.55	0\\
71.56	0\\
71.57	0\\
71.58	0\\
71.59	0\\
71.6	0\\
71.61	0\\
71.62	0\\
71.63	0\\
71.64	0\\
71.65	0\\
71.66	0\\
71.67	0\\
71.68	0\\
71.69	0\\
71.7	0\\
71.71	0\\
71.72	0\\
71.73	0\\
71.74	0\\
71.75	0\\
71.76	0\\
71.77	0\\
71.78	0\\
71.79	0\\
71.8	0\\
71.81	0\\
71.82	0\\
71.83	0\\
71.84	0\\
71.85	0\\
71.86	0\\
71.87	0\\
71.88	0\\
71.89	0\\
71.9	0\\
71.91	0\\
71.92	0\\
71.93	0\\
71.94	0\\
71.95	0\\
71.96	0\\
71.97	0\\
71.98	0\\
71.99	0\\
72	0\\
72.01	0\\
72.02	0\\
72.03	0\\
72.04	0\\
72.05	0\\
72.06	0\\
72.07	0\\
72.08	0\\
72.09	0\\
72.1	0\\
72.11	0\\
72.12	0\\
72.13	0\\
72.14	0\\
72.15	0\\
72.16	0\\
72.17	0\\
72.18	0\\
72.19	0\\
72.2	0\\
72.21	0\\
72.22	0\\
72.23	0\\
72.24	0\\
72.25	0\\
72.26	0\\
72.27	0\\
72.28	0\\
72.29	0\\
72.3	0\\
72.31	0\\
72.32	0\\
72.33	0\\
72.34	0\\
72.35	0\\
72.36	0\\
72.37	0\\
72.38	0\\
72.39	0\\
72.4	0\\
72.41	0\\
72.42	0\\
72.43	0\\
72.44	0\\
72.45	0\\
72.46	0\\
72.47	0\\
72.48	0\\
72.49	0\\
72.5	0\\
72.51	0\\
72.52	0\\
72.53	0\\
72.54	0\\
72.55	0\\
72.56	0\\
72.57	0\\
72.58	0\\
72.59	0\\
72.6	0\\
72.61	0\\
72.62	0\\
72.63	0\\
72.64	0\\
72.65	0\\
72.66	0\\
72.67	0\\
72.68	0\\
72.69	0\\
72.7	0\\
72.71	0\\
72.72	0\\
72.73	0\\
72.74	0\\
72.75	0\\
72.76	0\\
72.77	0\\
72.78	0\\
72.79	0\\
72.8	0\\
72.81	0\\
72.82	0\\
72.83	0\\
72.84	0\\
72.85	0\\
72.86	0\\
72.87	0\\
72.88	0\\
72.89	0\\
72.9	0\\
72.91	0\\
72.92	0\\
72.93	0\\
72.94	0\\
72.95	0\\
72.96	0\\
72.97	0\\
72.98	0\\
72.99	0\\
73	0\\
73.01	0\\
73.02	0\\
73.03	0\\
73.04	0\\
73.05	0\\
73.06	0\\
73.07	0\\
73.08	0\\
73.09	0\\
73.1	0\\
73.11	0\\
73.12	0\\
73.13	0\\
73.14	0\\
73.15	0\\
73.16	0\\
73.17	0\\
73.18	0\\
73.19	0\\
73.2	0\\
73.21	0\\
73.22	0\\
73.23	0\\
73.24	0\\
73.25	0\\
73.26	0\\
73.27	0\\
73.28	0\\
73.29	0\\
73.3	0\\
73.31	0\\
73.32	0\\
73.33	0\\
73.34	0\\
73.35	0\\
73.36	0\\
73.37	0\\
73.38	0\\
73.39	0\\
73.4	0\\
73.41	0\\
73.42	0\\
73.43	0\\
73.44	0\\
73.45	0\\
73.46	0\\
73.47	0\\
73.48	0\\
73.49	0\\
73.5	0\\
73.51	0\\
73.52	0\\
73.53	0\\
73.54	0\\
73.55	0\\
73.56	0\\
73.57	0\\
73.58	0\\
73.59	0\\
73.6	0\\
73.61	0\\
73.62	0\\
73.63	0\\
73.64	0\\
73.65	0\\
73.66	0\\
73.67	0\\
73.68	0\\
73.69	0\\
73.7	0\\
73.71	0\\
73.72	0\\
73.73	0\\
73.74	0\\
73.75	0\\
73.76	0\\
73.77	0\\
73.78	0\\
73.79	0\\
73.8	0\\
73.81	0\\
73.82	0\\
73.83	0\\
73.84	0\\
73.85	0\\
73.86	0\\
73.87	0\\
73.88	0\\
73.89	0\\
73.9	0\\
73.91	0\\
73.92	0\\
73.93	0\\
73.94	0\\
73.95	0\\
73.96	0\\
73.97	0\\
73.98	0\\
73.99	0\\
74	0\\
74.01	0\\
74.02	0\\
74.03	0\\
74.04	0\\
74.05	0\\
74.06	0\\
74.07	0\\
74.08	0\\
74.09	0\\
74.1	0\\
74.11	0\\
74.12	0\\
74.13	0\\
74.14	0\\
74.15	0\\
74.16	0\\
74.17	0\\
74.18	0\\
74.19	0\\
74.2	0\\
74.21	0\\
74.22	0\\
74.23	0\\
74.24	0\\
74.25	0\\
74.26	0\\
74.27	0\\
74.28	0\\
74.29	0\\
74.3	0\\
74.31	0\\
74.32	0\\
74.33	0\\
74.34	0\\
74.35	0\\
74.36	0\\
74.37	0\\
74.38	0\\
74.39	0\\
74.4	0\\
74.41	0\\
74.42	0\\
74.43	0\\
74.44	0\\
74.45	0\\
74.46	0\\
74.47	0\\
74.48	0\\
74.49	0\\
74.5	0\\
74.51	0\\
74.52	0\\
74.53	0\\
74.54	0\\
74.55	0\\
74.56	0\\
74.57	0\\
74.58	0\\
74.59	0\\
74.6	0\\
74.61	0\\
74.62	0\\
74.63	0\\
74.64	0\\
74.65	0\\
74.66	0\\
74.67	0\\
74.68	0\\
74.69	0\\
74.7	0\\
74.71	0\\
74.72	0\\
74.73	0\\
74.74	0\\
74.75	0\\
74.76	0\\
74.77	0\\
74.78	0\\
74.79	0\\
74.8	0\\
74.81	0\\
74.82	0\\
74.83	0\\
74.84	0\\
74.85	0\\
74.86	0\\
74.87	0\\
74.88	0\\
74.89	0\\
74.9	0\\
74.91	0\\
74.92	0\\
74.93	0\\
74.94	0\\
74.95	0\\
74.96	0\\
74.97	0\\
74.98	0\\
74.99	0\\
75	0\\
75.01	0\\
75.02	0\\
75.03	0\\
75.04	0\\
75.05	0\\
75.06	0\\
75.07	0\\
75.08	0\\
75.09	0\\
75.1	0\\
75.11	0\\
75.12	0\\
75.13	0\\
75.14	0\\
75.15	0\\
75.16	0\\
75.17	0\\
75.18	0\\
75.19	0\\
75.2	0\\
75.21	0\\
75.22	0\\
75.23	0\\
75.24	0\\
75.25	0\\
75.26	0\\
75.27	0\\
75.28	0\\
75.29	0\\
75.3	0\\
75.31	0\\
75.32	0\\
75.33	0\\
75.34	0\\
75.35	0\\
75.36	0\\
75.37	0\\
75.38	0\\
75.39	0\\
75.4	0\\
75.41	0\\
75.42	0\\
75.43	0\\
75.44	0\\
75.45	0\\
75.46	0\\
75.47	0\\
75.48	0\\
75.49	0\\
75.5	0\\
75.51	0\\
75.52	0\\
75.53	0\\
75.54	0\\
75.55	0\\
75.56	0\\
75.57	0\\
75.58	0\\
75.59	0\\
75.6	0\\
75.61	0\\
75.62	0\\
75.63	0\\
75.64	0\\
75.65	0\\
75.66	0\\
75.67	0\\
75.68	0\\
75.69	0\\
75.7	0\\
75.71	0\\
75.72	0\\
75.73	0\\
75.74	0\\
75.75	0\\
75.76	0\\
75.77	0\\
75.78	0\\
75.79	0\\
75.8	0\\
75.81	0\\
75.82	0\\
75.83	0\\
75.84	0\\
75.85	0\\
75.86	0\\
75.87	0\\
75.88	0\\
75.89	0\\
75.9	0\\
75.91	0\\
75.92	0\\
75.93	0\\
75.94	0\\
75.95	0\\
75.96	0\\
75.97	0\\
75.98	0\\
75.99	0\\
76	0\\
76.01	0\\
76.02	0\\
76.03	0\\
76.04	0\\
76.05	0\\
76.06	0\\
76.07	0\\
76.08	0\\
76.09	0\\
76.1	0\\
76.11	0\\
76.12	0\\
76.13	0\\
76.14	0\\
76.15	0\\
76.16	0\\
76.17	0\\
76.18	0\\
76.19	0\\
76.2	0\\
76.21	0\\
76.22	0\\
76.23	0\\
76.24	0\\
76.25	0\\
76.26	0\\
76.27	0\\
76.28	0\\
76.29	0\\
76.3	0\\
76.31	0\\
76.32	0\\
76.33	0\\
76.34	0\\
76.35	0\\
76.36	0\\
76.37	0\\
76.38	0\\
76.39	0\\
76.4	0\\
76.41	0\\
76.42	0\\
76.43	0\\
76.44	0\\
76.45	0\\
76.46	0\\
76.47	0\\
76.48	0\\
76.49	0\\
76.5	0\\
76.51	0\\
76.52	0\\
76.53	0\\
76.54	0\\
76.55	0\\
76.56	0\\
76.57	0\\
76.58	0\\
76.59	0\\
76.6	0\\
76.61	0\\
76.62	0\\
76.63	0\\
76.64	0\\
76.65	0\\
76.66	0\\
76.67	0\\
76.68	0\\
76.69	0\\
76.7	0\\
76.71	0\\
76.72	0\\
76.73	0\\
76.74	0\\
76.75	0\\
76.76	0\\
76.77	0\\
76.78	0\\
76.79	0\\
76.8	0\\
76.81	0\\
76.82	0\\
76.83	0\\
76.84	0\\
76.85	0\\
76.86	0\\
76.87	0\\
76.88	0\\
76.89	0\\
76.9	0\\
76.91	0\\
76.92	0\\
76.93	0\\
76.94	0\\
76.95	0\\
76.96	0\\
76.97	0\\
76.98	0\\
76.99	0\\
77	0\\
77.01	0\\
77.02	0\\
77.03	0\\
77.04	0\\
77.05	0\\
77.06	0\\
77.07	0\\
77.08	0\\
77.09	0\\
77.1	0\\
77.11	0\\
77.12	0\\
77.13	0\\
77.14	0\\
77.15	0\\
77.16	0\\
77.17	0\\
77.18	0\\
77.19	0\\
77.2	0\\
77.21	0\\
77.22	0\\
77.23	0\\
77.24	0\\
77.25	0\\
77.26	0\\
77.27	0\\
77.28	0\\
77.29	0\\
77.3	0\\
77.31	0\\
77.32	0\\
77.33	0\\
77.34	0\\
77.35	0\\
77.36	0\\
77.37	0\\
77.38	0\\
77.39	0\\
77.4	0\\
77.41	0\\
77.42	0\\
77.43	0\\
77.44	0\\
77.45	0\\
77.46	0\\
77.47	0\\
77.48	0\\
77.49	0\\
77.5	0\\
77.51	0\\
77.52	0\\
77.53	0\\
77.54	0\\
77.55	0\\
77.56	0\\
77.57	0\\
77.58	0\\
77.59	0\\
77.6	0\\
77.61	0\\
77.62	0\\
77.63	0\\
77.64	0\\
77.65	0\\
77.66	0\\
77.67	0\\
77.68	0\\
77.69	0\\
77.7	0\\
77.71	0\\
77.72	0\\
77.73	0\\
77.74	0\\
77.75	0\\
77.76	0\\
77.77	0\\
77.78	0\\
77.79	0\\
77.8	0\\
77.81	0\\
77.82	0\\
77.83	0\\
77.84	0\\
77.85	0\\
77.86	0\\
77.87	0\\
77.88	0\\
77.89	0\\
77.9	0\\
77.91	0\\
77.92	0\\
77.93	0\\
77.94	0\\
77.95	0\\
77.96	0\\
77.97	0\\
77.98	0\\
77.99	0\\
78	0\\
78.01	0\\
78.02	0\\
78.03	0\\
78.04	0\\
78.05	0\\
78.06	0\\
78.07	0\\
78.08	0\\
78.09	0\\
78.1	0\\
78.11	0\\
78.12	0\\
78.13	0\\
78.14	0\\
78.15	0\\
78.16	0\\
78.17	0\\
78.18	0\\
78.19	0\\
78.2	0\\
78.21	0\\
78.22	0\\
78.23	0\\
78.24	0\\
78.25	0\\
78.26	0\\
78.27	0\\
78.28	0\\
78.29	0\\
78.3	0\\
78.31	0\\
78.32	0\\
78.33	0\\
78.34	0\\
78.35	0\\
78.36	0\\
78.37	0\\
78.38	0\\
78.39	0\\
78.4	0\\
78.41	0\\
78.42	0\\
78.43	0\\
78.44	0\\
78.45	0\\
78.46	0\\
78.47	0\\
78.48	0\\
78.49	0\\
78.5	0\\
78.51	0\\
78.52	0\\
78.53	0\\
78.54	0\\
78.55	0\\
78.56	0\\
78.57	0\\
78.58	0\\
78.59	0\\
78.6	0\\
78.61	0\\
78.62	0\\
78.63	0\\
78.64	0\\
78.65	0\\
78.66	0\\
78.67	0\\
78.68	0\\
78.69	0\\
78.7	0\\
78.71	0\\
78.72	0\\
78.73	0\\
78.74	0\\
78.75	0\\
78.76	0\\
78.77	0\\
78.78	0\\
78.79	0\\
78.8	0\\
78.81	0\\
78.82	0\\
78.83	0\\
78.84	0\\
78.85	0\\
78.86	0\\
78.87	0\\
78.88	0\\
78.89	0\\
78.9	0\\
78.91	0\\
78.92	0\\
78.93	0\\
78.94	0\\
78.95	0\\
78.96	0\\
78.97	0\\
78.98	0\\
78.99	0\\
79	0\\
79.01	0\\
79.02	0\\
79.03	0\\
79.04	0\\
79.05	0\\
79.06	0\\
79.07	0\\
79.08	0\\
79.09	0\\
79.1	0\\
79.11	0\\
79.12	0\\
79.13	0\\
79.14	0\\
79.15	0\\
79.16	0\\
79.17	0\\
79.18	0\\
79.19	0\\
79.2	0\\
79.21	0\\
79.22	0\\
79.23	0\\
79.24	0\\
79.25	0\\
79.26	0\\
79.27	0\\
79.28	0\\
79.29	0\\
79.3	0\\
79.31	0\\
79.32	0\\
79.33	0\\
79.34	0\\
79.35	0\\
79.36	0\\
79.37	0\\
79.38	0\\
79.39	0\\
79.4	0\\
79.41	0\\
79.42	0\\
79.43	0\\
79.44	0\\
79.45	0\\
79.46	0\\
79.47	0\\
79.48	0\\
79.49	0\\
79.5	0\\
79.51	0\\
79.52	0\\
79.53	0\\
79.54	0\\
79.55	0\\
79.56	0\\
79.57	0\\
79.58	0\\
79.59	0\\
79.6	0\\
79.61	0\\
79.62	0\\
79.63	0\\
79.64	0\\
79.65	0\\
79.66	0\\
79.67	0\\
79.68	0\\
79.69	0\\
79.7	0\\
79.71	0\\
79.72	0\\
79.73	0\\
79.74	0\\
79.75	0\\
79.76	0\\
79.77	0\\
79.78	0\\
79.79	0\\
79.8	0\\
79.81	0\\
79.82	0\\
79.83	0\\
79.84	0\\
79.85	0\\
79.86	0\\
79.87	0\\
79.88	0\\
79.89	0\\
79.9	0\\
79.91	0\\
79.92	0\\
79.93	0\\
79.94	0\\
79.95	0\\
79.96	0\\
79.97	0\\
79.98	0\\
79.99	0\\
80	0\\
80.01	0\\
};
\addplot [color=green,solid]
  table[row sep=crcr]{%
80.01	0\\
80.02	0\\
80.03	0\\
80.04	0\\
80.05	0\\
80.06	0\\
80.07	0\\
80.08	0\\
80.09	0\\
80.1	0\\
80.11	0\\
80.12	0\\
80.13	0\\
80.14	0\\
80.15	0\\
80.16	0\\
80.17	0\\
80.18	0\\
80.19	0\\
80.2	0\\
80.21	0\\
80.22	0\\
80.23	0\\
80.24	0\\
80.25	0\\
80.26	0\\
80.27	0\\
80.28	0\\
80.29	0\\
80.3	0\\
80.31	0\\
80.32	0\\
80.33	0\\
80.34	0\\
80.35	0\\
80.36	0\\
80.37	0\\
80.38	0\\
80.39	0\\
80.4	0\\
80.41	0\\
80.42	0\\
80.43	0\\
80.44	0\\
80.45	0\\
80.46	0\\
80.47	0\\
80.48	0\\
80.49	0\\
80.5	0\\
80.51	0\\
80.52	0\\
80.53	0\\
80.54	0\\
80.55	0\\
80.56	0\\
80.57	0\\
80.58	0\\
80.59	0\\
80.6	0\\
80.61	0\\
80.62	0\\
80.63	0\\
80.64	0\\
80.65	0\\
80.66	0\\
80.67	0\\
80.68	0\\
80.69	0\\
80.7	0\\
80.71	0\\
80.72	0\\
80.73	0\\
80.74	0\\
80.75	0\\
80.76	0\\
80.77	0\\
80.78	0\\
80.79	0\\
80.8	0\\
80.81	0\\
80.82	0\\
80.83	0\\
80.84	0\\
80.85	0\\
80.86	0\\
80.87	0\\
80.88	0\\
80.89	0\\
80.9	0\\
80.91	0\\
80.92	0\\
80.93	0\\
80.94	0\\
80.95	0\\
80.96	0\\
80.97	0\\
80.98	0\\
80.99	0\\
81	0\\
81.01	0\\
81.02	0\\
81.03	0\\
81.04	0\\
81.05	0\\
81.06	0\\
81.07	0\\
81.08	0\\
81.09	0\\
81.1	0\\
81.11	0\\
81.12	0\\
81.13	0\\
81.14	0\\
81.15	0\\
81.16	0\\
81.17	0\\
81.18	0\\
81.19	0\\
81.2	0\\
81.21	0\\
81.22	0\\
81.23	0\\
81.24	0\\
81.25	0\\
81.26	0\\
81.27	0\\
81.28	0\\
81.29	0\\
81.3	0\\
81.31	0\\
81.32	0\\
81.33	0\\
81.34	0\\
81.35	0\\
81.36	0\\
81.37	0\\
81.38	0\\
81.39	0\\
81.4	0\\
81.41	0\\
81.42	0\\
81.43	0\\
81.44	0\\
81.45	0\\
81.46	0\\
81.47	0\\
81.48	0\\
81.49	0\\
81.5	0\\
81.51	0\\
81.52	0\\
81.53	0\\
81.54	0\\
81.55	0\\
81.56	0\\
81.57	0\\
81.58	0\\
81.59	0\\
81.6	0\\
81.61	0\\
81.62	0\\
81.63	0\\
81.64	0\\
81.65	0\\
81.66	0\\
81.67	0\\
81.68	0\\
81.69	0\\
81.7	0\\
81.71	0\\
81.72	0\\
81.73	0\\
81.74	0\\
81.75	0\\
81.76	0\\
81.77	0\\
81.78	0\\
81.79	0\\
81.8	0\\
81.81	0\\
81.82	0\\
81.83	0\\
81.84	0\\
81.85	0\\
81.86	0\\
81.87	0\\
81.88	0\\
81.89	0\\
81.9	0\\
81.91	0\\
81.92	0\\
81.93	0\\
81.94	0\\
81.95	0\\
81.96	0\\
81.97	0\\
81.98	0\\
81.99	0\\
82	0\\
82.01	0\\
82.02	0\\
82.03	0\\
82.04	0\\
82.05	0\\
82.06	0\\
82.07	0\\
82.08	0\\
82.09	0\\
82.1	0\\
82.11	0\\
82.12	0\\
82.13	0\\
82.14	0\\
82.15	0\\
82.16	0\\
82.17	0\\
82.18	0\\
82.19	0\\
82.2	0\\
82.21	0\\
82.22	0\\
82.23	0\\
82.24	0\\
82.25	0\\
82.26	0\\
82.27	0\\
82.28	0\\
82.29	0\\
82.3	0\\
82.31	0\\
82.32	0\\
82.33	0\\
82.34	0\\
82.35	0\\
82.36	0\\
82.37	0\\
82.38	0\\
82.39	0\\
82.4	0\\
82.41	0\\
82.42	0\\
82.43	0\\
82.44	0\\
82.45	0\\
82.46	0\\
82.47	0\\
82.48	0\\
82.49	0\\
82.5	0\\
82.51	0\\
82.52	0\\
82.53	0\\
82.54	0\\
82.55	0\\
82.56	0\\
82.57	0\\
82.58	0\\
82.59	0\\
82.6	0\\
82.61	0\\
82.62	0\\
82.63	0\\
82.64	0\\
82.65	0\\
82.66	0\\
82.67	0\\
82.68	0\\
82.69	0\\
82.7	0\\
82.71	0\\
82.72	0\\
82.73	0\\
82.74	0\\
82.75	0\\
82.76	0\\
82.77	0\\
82.78	0\\
82.79	0\\
82.8	0\\
82.81	0\\
82.82	0\\
82.83	0\\
82.84	0\\
82.85	0\\
82.86	0\\
82.87	0\\
82.88	0\\
82.89	0\\
82.9	0\\
82.91	0\\
82.92	0\\
82.93	0\\
82.94	0\\
82.95	0\\
82.96	0\\
82.97	0\\
82.98	0\\
82.99	0\\
83	0\\
83.01	0\\
83.02	0\\
83.03	0\\
83.04	0\\
83.05	0\\
83.06	0\\
83.07	0\\
83.08	0\\
83.09	0\\
83.1	0\\
83.11	0\\
83.12	0\\
83.13	0\\
83.14	0\\
83.15	0\\
83.16	0\\
83.17	0\\
83.18	0\\
83.19	0\\
83.2	0\\
83.21	0\\
83.22	0\\
83.23	0\\
83.24	0\\
83.25	0\\
83.26	0\\
83.27	0\\
83.28	0\\
83.29	0\\
83.3	0\\
83.31	0\\
83.32	0\\
83.33	0\\
83.34	0\\
83.35	0\\
83.36	0\\
83.37	0\\
83.38	0\\
83.39	0\\
83.4	0\\
83.41	0\\
83.42	0\\
83.43	0\\
83.44	0\\
83.45	0\\
83.46	0\\
83.47	0\\
83.48	0\\
83.49	0\\
83.5	0\\
83.51	0\\
83.52	0\\
83.53	0\\
83.54	0\\
83.55	0\\
83.56	0\\
83.57	0\\
83.58	0\\
83.59	0\\
83.6	0\\
83.61	0\\
83.62	0\\
83.63	0\\
83.64	0\\
83.65	0\\
83.66	0\\
83.67	0\\
83.68	0\\
83.69	0\\
83.7	0\\
83.71	0\\
83.72	0\\
83.73	0\\
83.74	0\\
83.75	0\\
83.76	0\\
83.77	0\\
83.78	0\\
83.79	0\\
83.8	0\\
83.81	0\\
83.82	0\\
83.83	0\\
83.84	0\\
83.85	0\\
83.86	0\\
83.87	0\\
83.88	0\\
83.89	0\\
83.9	0\\
83.91	0\\
83.92	0\\
83.93	0\\
83.94	0\\
83.95	0\\
83.96	0\\
83.97	0\\
83.98	0\\
83.99	0\\
84	0\\
84.01	0\\
84.02	0\\
84.03	0\\
84.04	0\\
84.05	0\\
84.06	0\\
84.07	0\\
84.08	0\\
84.09	0\\
84.1	0\\
84.11	0\\
84.12	0\\
84.13	0\\
84.14	0\\
84.15	0\\
84.16	0\\
84.17	0\\
84.18	0\\
84.19	0\\
84.2	0\\
84.21	0\\
84.22	0\\
84.23	0\\
84.24	0\\
84.25	0\\
84.26	0\\
84.27	0\\
84.28	0\\
84.29	0\\
84.3	0\\
84.31	0\\
84.32	0\\
84.33	0\\
84.34	0\\
84.35	0\\
84.36	0\\
84.37	0\\
84.38	0\\
84.39	0\\
84.4	0\\
84.41	0\\
84.42	0\\
84.43	0\\
84.44	0\\
84.45	0\\
84.46	0\\
84.47	0\\
84.48	0\\
84.49	0\\
84.5	0\\
84.51	0\\
84.52	0\\
84.53	0\\
84.54	0\\
84.55	0\\
84.56	0\\
84.57	0\\
84.58	0\\
84.59	0\\
84.6	0\\
84.61	0\\
84.62	0\\
84.63	0\\
84.64	0\\
84.65	0\\
84.66	0\\
84.67	0\\
84.68	0\\
84.69	0\\
84.7	0\\
84.71	0\\
84.72	0\\
84.73	0\\
84.74	0\\
84.75	0\\
84.76	0\\
84.77	0\\
84.78	0\\
84.79	0\\
84.8	0\\
84.81	0\\
84.82	0\\
84.83	0\\
84.84	0\\
84.85	0\\
84.86	0\\
84.87	0\\
84.88	0\\
84.89	0\\
84.9	0\\
84.91	0\\
84.92	0\\
84.93	0\\
84.94	0\\
84.95	0\\
84.96	0\\
84.97	0\\
84.98	0\\
84.99	0\\
85	0\\
85.01	0\\
85.02	0\\
85.03	0\\
85.04	0\\
85.05	0\\
85.06	0\\
85.07	0\\
85.08	0\\
85.09	0\\
85.1	0\\
85.11	0\\
85.12	0\\
85.13	0\\
85.14	0\\
85.15	0\\
85.16	0\\
85.17	0\\
85.18	0\\
85.19	0\\
85.2	0\\
85.21	0\\
85.22	0\\
85.23	0\\
85.24	0\\
85.25	0\\
85.26	0\\
85.27	0\\
85.28	0\\
85.29	0\\
85.3	0\\
85.31	0\\
85.32	0\\
85.33	0\\
85.34	0\\
85.35	0\\
85.36	0\\
85.37	0\\
85.38	0\\
85.39	0\\
85.4	0\\
85.41	0\\
85.42	0\\
85.43	0\\
85.44	0\\
85.45	0\\
85.46	0\\
85.47	0\\
85.48	0\\
85.49	0\\
85.5	0\\
85.51	0\\
85.52	0\\
85.53	0\\
85.54	0\\
85.55	0\\
85.56	0\\
85.57	0\\
85.58	0\\
85.59	0\\
85.6	0\\
85.61	0\\
85.62	0\\
85.63	0\\
85.64	0\\
85.65	0\\
85.66	0\\
85.67	0\\
85.68	0\\
85.69	0\\
85.7	0\\
85.71	0\\
85.72	0\\
85.73	0\\
85.74	0\\
85.75	0\\
85.76	0\\
85.77	0\\
85.78	0\\
85.79	0\\
85.8	0\\
85.81	0\\
85.82	0\\
85.83	0\\
85.84	0\\
85.85	0\\
85.86	0\\
85.87	0\\
85.88	0\\
85.89	0\\
85.9	0\\
85.91	0\\
85.92	0\\
85.93	0\\
85.94	0\\
85.95	0\\
85.96	0\\
85.97	0\\
85.98	0\\
85.99	0\\
86	0\\
86.01	0\\
86.02	0\\
86.03	0\\
86.04	0\\
86.05	0\\
86.06	0\\
86.07	0\\
86.08	0\\
86.09	0\\
86.1	0\\
86.11	0\\
86.12	0\\
86.13	0\\
86.14	0\\
86.15	0\\
86.16	0\\
86.17	0\\
86.18	0\\
86.19	0\\
86.2	0\\
86.21	0\\
86.22	0\\
86.23	0\\
86.24	0\\
86.25	0\\
86.26	0\\
86.27	0\\
86.28	0\\
86.29	0\\
86.3	0\\
86.31	0\\
86.32	0\\
86.33	0\\
86.34	0\\
86.35	0\\
86.36	0\\
86.37	0\\
86.38	0\\
86.39	0\\
86.4	0\\
86.41	0\\
86.42	0\\
86.43	0\\
86.44	0\\
86.45	0\\
86.46	0\\
86.47	0\\
86.48	0\\
86.49	0\\
86.5	0\\
86.51	0\\
86.52	0\\
86.53	0\\
86.54	0\\
86.55	0\\
86.56	0\\
86.57	0\\
86.58	0\\
86.59	0\\
86.6	0\\
86.61	0\\
86.62	0\\
86.63	0\\
86.64	0\\
86.65	0\\
86.66	0\\
86.67	0\\
86.68	0\\
86.69	0\\
86.7	0\\
86.71	0\\
86.72	0\\
86.73	0\\
86.74	0\\
86.75	0\\
86.76	0\\
86.77	0\\
86.78	0\\
86.79	0\\
86.8	0\\
86.81	0\\
86.82	0\\
86.83	0\\
86.84	0\\
86.85	0\\
86.86	0\\
86.87	0\\
86.88	0\\
86.89	0\\
86.9	0\\
86.91	0\\
86.92	0\\
86.93	0\\
86.94	0\\
86.95	0\\
86.96	0\\
86.97	0\\
86.98	0\\
86.99	0\\
87	0\\
87.01	0\\
87.02	0\\
87.03	0\\
87.04	0\\
87.05	0\\
87.06	0\\
87.07	0\\
87.08	0\\
87.09	0\\
87.1	0\\
87.11	0\\
87.12	0\\
87.13	0\\
87.14	0\\
87.15	0\\
87.16	0\\
87.17	0\\
87.18	0\\
87.19	0\\
87.2	0\\
87.21	0\\
87.22	0\\
87.23	0\\
87.24	0\\
87.25	0\\
87.26	0\\
87.27	0\\
87.28	0\\
87.29	0\\
87.3	0\\
87.31	0\\
87.32	0\\
87.33	0\\
87.34	0\\
87.35	0\\
87.36	0\\
87.37	0\\
87.38	0\\
87.39	0\\
87.4	0\\
87.41	0\\
87.42	0\\
87.43	0\\
87.44	0\\
87.45	0\\
87.46	0\\
87.47	0\\
87.48	0\\
87.49	0\\
87.5	0\\
87.51	0\\
87.52	0\\
87.53	0\\
87.54	0\\
87.55	0\\
87.56	0\\
87.57	0\\
87.58	0\\
87.59	0\\
87.6	0\\
87.61	0\\
87.62	0\\
87.63	0\\
87.64	0\\
87.65	0\\
87.66	0\\
87.67	0\\
87.68	0\\
87.69	0\\
87.7	0\\
87.71	0\\
87.72	0\\
87.73	0\\
87.74	0\\
87.75	0\\
87.76	0\\
87.77	0\\
87.78	0\\
87.79	0\\
87.8	0\\
87.81	0\\
87.82	0\\
87.83	0\\
87.84	0\\
87.85	0\\
87.86	0\\
87.87	0\\
87.88	0\\
87.89	0\\
87.9	0\\
87.91	0\\
87.92	0\\
87.93	0\\
87.94	0\\
87.95	0\\
87.96	0\\
87.97	0\\
87.98	0\\
87.99	0\\
88	0\\
88.01	0\\
88.02	0\\
88.03	0\\
88.04	0\\
88.05	0\\
88.06	0\\
88.07	0\\
88.08	0\\
88.09	0\\
88.1	0\\
88.11	0\\
88.12	0\\
88.13	0\\
88.14	0\\
88.15	0\\
88.16	0\\
88.17	0\\
88.18	0\\
88.19	0\\
88.2	0\\
88.21	0\\
88.22	0\\
88.23	0\\
88.24	0\\
88.25	0\\
88.26	0\\
88.27	0\\
88.28	0\\
88.29	0\\
88.3	0\\
88.31	0\\
88.32	0\\
88.33	0\\
88.34	0\\
88.35	0\\
88.36	0\\
88.37	0\\
88.38	0\\
88.39	0\\
88.4	0\\
88.41	0\\
88.42	0\\
88.43	0\\
88.44	0\\
88.45	0\\
88.46	0\\
88.47	0\\
88.48	0\\
88.49	0\\
88.5	0\\
88.51	0\\
88.52	0\\
88.53	0\\
88.54	0\\
88.55	0\\
88.56	0\\
88.57	0\\
88.58	0\\
88.59	0\\
88.6	0\\
88.61	0\\
88.62	0\\
88.63	0\\
88.64	0\\
88.65	0\\
88.66	0\\
88.67	0\\
88.68	0\\
88.69	0\\
88.7	0\\
88.71	0\\
88.72	0\\
88.73	0\\
88.74	0\\
88.75	0\\
88.76	0\\
88.77	0\\
88.78	0\\
88.79	0\\
88.8	0\\
88.81	0\\
88.82	0\\
88.83	0\\
88.84	0\\
88.85	0\\
88.86	0\\
88.87	0\\
88.88	0\\
88.89	0\\
88.9	0\\
88.91	0\\
88.92	0\\
88.93	0\\
88.94	0\\
88.95	0\\
88.96	0\\
88.97	0\\
88.98	0\\
88.99	0\\
89	0\\
89.01	0\\
89.02	0\\
89.03	0\\
89.04	0\\
89.05	0\\
89.06	0\\
89.07	0\\
89.08	0\\
89.09	0\\
89.1	0\\
89.11	0\\
89.12	0\\
89.13	0\\
89.14	0\\
89.15	0\\
89.16	0\\
89.17	0\\
89.18	0\\
89.19	0\\
89.2	0\\
89.21	0\\
89.22	0\\
89.23	0\\
89.24	0\\
89.25	0\\
89.26	0\\
89.27	0\\
89.28	0\\
89.29	0\\
89.3	0\\
89.31	0\\
89.32	0\\
89.33	0\\
89.34	0\\
89.35	0\\
89.36	0\\
89.37	0\\
89.38	0\\
89.39	0\\
89.4	0\\
89.41	0\\
89.42	0\\
89.43	0\\
89.44	0\\
89.45	0\\
89.46	0\\
89.47	0\\
89.48	0\\
89.49	0\\
89.5	0\\
89.51	0\\
89.52	0\\
89.53	0\\
89.54	0\\
89.55	0\\
89.56	0\\
89.57	0\\
89.58	0\\
89.59	0\\
89.6	0\\
89.61	0\\
89.62	0\\
89.63	0\\
89.64	0\\
89.65	0\\
89.66	0\\
89.67	0\\
89.68	0\\
89.69	0\\
89.7	0\\
89.71	0\\
89.72	0\\
89.73	0\\
89.74	0\\
89.75	0\\
89.76	0\\
89.77	0\\
89.78	0\\
89.79	0\\
89.8	0\\
89.81	0\\
89.82	0\\
89.83	0\\
89.84	0\\
89.85	0\\
89.86	0\\
89.87	0\\
89.88	0\\
89.89	0\\
89.9	0\\
89.91	0\\
89.92	0\\
89.93	0\\
89.94	0\\
89.95	0\\
89.96	0\\
89.97	0\\
89.98	0\\
89.99	0\\
90	0\\
90.01	0\\
90.02	0\\
90.03	0\\
90.04	0\\
90.05	0\\
90.06	0\\
90.07	0\\
90.08	0\\
90.09	0\\
90.1	0\\
90.11	0\\
90.12	0\\
90.13	0\\
90.14	0\\
90.15	0\\
90.16	0\\
90.17	0\\
90.18	0\\
90.19	0\\
90.2	0\\
90.21	0\\
90.22	0\\
90.23	0\\
90.24	0\\
90.25	0\\
90.26	0\\
90.27	0\\
90.28	0\\
90.29	0\\
90.3	0\\
90.31	0\\
90.32	0\\
90.33	0\\
90.34	0\\
90.35	0\\
90.36	0\\
90.37	0\\
90.38	0\\
90.39	0\\
90.4	0\\
90.41	0\\
90.42	0\\
90.43	0\\
90.44	0\\
90.45	0\\
90.46	0\\
90.47	0\\
90.48	0\\
90.49	0\\
90.5	0\\
90.51	0\\
90.52	0\\
90.53	0\\
90.54	0\\
90.55	0\\
90.56	0\\
90.57	0\\
90.58	0\\
90.59	0\\
90.6	0\\
90.61	0\\
90.62	0\\
90.63	0\\
90.64	0\\
90.65	0\\
90.66	0\\
90.67	0\\
90.68	0\\
90.69	0\\
90.7	0\\
90.71	0\\
90.72	0\\
90.73	0\\
90.74	0\\
90.75	0\\
90.76	0\\
90.77	0\\
90.78	0\\
90.79	0\\
90.8	0\\
90.81	0\\
90.82	0\\
90.83	0\\
90.84	0\\
90.85	0\\
90.86	0\\
90.87	0\\
90.88	0\\
90.89	0\\
90.9	0\\
90.91	0\\
90.92	0\\
90.93	0\\
90.94	0\\
90.95	0\\
90.96	0\\
90.97	0\\
90.98	0\\
90.99	0\\
91	0\\
91.01	0\\
91.02	0\\
91.03	0\\
91.04	0\\
91.05	0\\
91.06	0\\
91.07	0\\
91.08	0\\
91.09	0\\
91.1	0\\
91.11	0\\
91.12	0\\
91.13	0\\
91.14	0\\
91.15	0\\
91.16	0\\
91.17	0\\
91.18	0\\
91.19	0\\
91.2	0\\
91.21	0\\
91.22	0\\
91.23	0\\
91.24	0\\
91.25	0\\
91.26	0\\
91.27	0\\
91.28	0\\
91.29	0\\
91.3	0\\
91.31	0\\
91.32	0\\
91.33	0\\
91.34	0\\
91.35	0\\
91.36	0\\
91.37	0\\
91.38	0\\
91.39	0\\
91.4	0\\
91.41	0\\
91.42	0\\
91.43	0\\
91.44	0\\
91.45	0\\
91.46	0\\
91.47	0\\
91.48	0\\
91.49	0\\
91.5	0\\
91.51	0\\
91.52	0\\
91.53	0\\
91.54	0\\
91.55	0\\
91.56	0\\
91.57	0\\
91.58	0\\
91.59	0\\
91.6	0\\
91.61	0\\
91.62	0\\
91.63	0\\
91.64	0\\
91.65	0\\
91.66	0\\
91.67	0\\
91.68	0\\
91.69	0\\
91.7	0\\
91.71	0\\
91.72	0\\
91.73	0\\
91.74	0\\
91.75	0\\
91.76	0\\
91.77	0\\
91.78	0\\
91.79	0\\
91.8	0\\
91.81	0\\
91.82	0\\
91.83	0\\
91.84	0\\
91.85	0\\
91.86	0\\
91.87	0\\
91.88	0\\
91.89	0\\
91.9	0\\
91.91	0\\
91.92	0\\
91.93	0\\
91.94	0\\
91.95	0\\
91.96	0\\
91.97	0\\
91.98	0\\
91.99	0\\
92	0\\
92.01	0\\
92.02	0\\
92.03	0\\
92.04	0\\
92.05	0\\
92.06	0\\
92.07	0\\
92.08	0\\
92.09	0\\
92.1	0\\
92.11	0\\
92.12	0\\
92.13	0\\
92.14	0\\
92.15	0\\
92.16	0\\
92.17	0\\
92.18	0\\
92.19	0\\
92.2	0\\
92.21	0\\
92.22	0\\
92.23	0\\
92.24	0\\
92.25	0\\
92.26	0\\
92.27	0\\
92.28	0\\
92.29	0\\
92.3	0\\
92.31	0\\
92.32	0\\
92.33	0\\
92.34	0\\
92.35	0\\
92.36	0\\
92.37	0\\
92.38	0\\
92.39	0\\
92.4	0\\
92.41	0\\
92.42	0\\
92.43	0\\
92.44	0\\
92.45	0\\
92.46	0\\
92.47	0\\
92.48	0\\
92.49	0\\
92.5	0\\
92.51	0\\
92.52	0\\
92.53	0\\
92.54	0\\
92.55	0\\
92.56	0\\
92.57	0\\
92.58	0\\
92.59	0\\
92.6	0\\
92.61	0\\
92.62	0\\
92.63	0\\
92.64	0\\
92.65	0\\
92.66	0\\
92.67	0\\
92.68	0\\
92.69	0\\
92.7	0\\
92.71	0\\
92.72	0\\
92.73	0\\
92.74	0\\
92.75	0\\
92.76	0\\
92.77	0\\
92.78	0\\
92.79	0\\
92.8	0\\
92.81	0\\
92.82	0\\
92.83	0\\
92.84	0\\
92.85	0\\
92.86	0\\
92.87	0\\
92.88	0\\
92.89	0\\
92.9	0\\
92.91	0\\
92.92	0\\
92.93	0\\
92.94	0\\
92.95	0\\
92.96	0\\
92.97	0\\
92.98	0\\
92.99	0\\
93	0\\
93.01	0\\
93.02	0\\
93.03	0\\
93.04	0\\
93.05	0\\
93.06	0\\
93.07	0\\
93.08	0\\
93.09	0\\
93.1	0\\
93.11	0\\
93.12	0\\
93.13	0\\
93.14	0\\
93.15	0\\
93.16	0\\
93.17	0\\
93.18	0\\
93.19	0\\
93.2	0\\
93.21	0\\
93.22	0\\
93.23	0\\
93.24	0\\
93.25	0\\
93.26	0\\
93.27	0\\
93.28	0\\
93.29	0\\
93.3	0\\
93.31	0\\
93.32	0\\
93.33	0\\
93.34	0\\
93.35	0\\
93.36	0\\
93.37	0\\
93.38	0\\
93.39	0\\
93.4	0\\
93.41	0\\
93.42	0\\
93.43	0\\
93.44	0\\
93.45	0\\
93.46	0\\
93.47	0\\
93.48	0\\
93.49	0\\
93.5	0\\
93.51	0\\
93.52	0\\
93.53	0\\
93.54	0\\
93.55	0\\
93.56	0\\
93.57	0\\
93.58	0\\
93.59	0\\
93.6	0\\
93.61	0\\
93.62	0\\
93.63	0\\
93.64	0\\
93.65	0\\
93.66	0\\
93.67	0\\
93.68	0\\
93.69	0\\
93.7	0\\
93.71	0\\
93.72	0\\
93.73	0\\
93.74	0\\
93.75	0\\
93.76	0\\
93.77	0\\
93.78	0\\
93.79	0\\
93.8	0\\
93.81	0\\
93.82	0\\
93.83	0\\
93.84	0\\
93.85	0\\
93.86	0\\
93.87	0\\
93.88	0\\
93.89	0\\
93.9	0\\
93.91	0\\
93.92	0\\
93.93	0\\
93.94	0\\
93.95	0\\
93.96	0\\
93.97	0\\
93.98	0\\
93.99	0\\
94	0\\
94.01	0\\
94.02	0\\
94.03	0\\
94.04	0\\
94.05	0\\
94.06	0\\
94.07	0\\
94.08	0\\
94.09	0\\
94.1	0\\
94.11	0\\
94.12	0\\
94.13	0\\
94.14	0\\
94.15	0\\
94.16	0\\
94.17	0\\
94.18	0\\
94.19	0\\
94.2	0\\
94.21	0\\
94.22	0\\
94.23	0\\
94.24	0\\
94.25	0\\
94.26	0\\
94.27	0\\
94.28	0\\
94.29	0\\
94.3	0\\
94.31	0\\
94.32	0\\
94.33	0\\
94.34	0\\
94.35	0\\
94.36	0\\
94.37	0\\
94.38	0\\
94.39	0\\
94.4	0\\
94.41	0\\
94.42	0\\
94.43	0\\
94.44	0\\
94.45	0\\
94.46	0\\
94.47	0\\
94.48	0\\
94.49	0\\
94.5	0\\
94.51	0\\
94.52	0\\
94.53	0\\
94.54	0\\
94.55	0\\
94.56	0\\
94.57	0\\
94.58	0\\
94.59	0\\
94.6	0\\
94.61	0\\
94.62	0\\
94.63	0\\
94.64	0\\
94.65	0\\
94.66	0\\
94.67	0\\
94.68	0\\
94.69	0\\
94.7	0\\
94.71	0\\
94.72	0\\
94.73	0\\
94.74	0\\
94.75	0\\
94.76	0\\
94.77	0\\
94.78	0\\
94.79	0\\
94.8	0\\
94.81	0\\
94.82	0\\
94.83	0\\
94.84	0\\
94.85	0\\
94.86	0\\
94.87	0\\
94.88	0\\
94.89	0\\
94.9	0\\
94.91	0\\
94.92	0\\
94.93	0\\
94.94	0\\
94.95	0\\
94.96	0\\
94.97	0\\
94.98	0\\
94.99	0\\
95	0\\
95.01	0\\
95.02	0\\
95.03	0\\
95.04	0\\
95.05	0\\
95.06	0\\
95.07	0\\
95.08	0\\
95.09	0\\
95.1	0\\
95.11	0\\
95.12	0\\
95.13	0\\
95.14	0\\
95.15	0\\
95.16	0\\
95.17	0\\
95.18	0\\
95.19	0\\
95.2	0\\
95.21	0\\
95.22	0\\
95.23	0\\
95.24	0\\
95.25	0\\
95.26	0\\
95.27	0\\
95.28	0\\
95.29	0\\
95.3	0\\
95.31	0\\
95.32	0\\
95.33	0\\
95.34	0\\
95.35	0\\
95.36	0\\
95.37	0\\
95.38	0\\
95.39	0\\
95.4	0\\
95.41	0\\
95.42	0\\
95.43	0\\
95.44	0\\
95.45	0\\
95.46	0\\
95.47	0\\
95.48	0\\
95.49	0\\
95.5	0\\
95.51	0\\
95.52	0\\
95.53	0\\
95.54	0\\
95.55	0\\
95.56	0\\
95.57	0\\
95.58	0\\
95.59	0\\
95.6	0\\
95.61	0\\
95.62	0\\
95.63	0\\
95.64	0\\
95.65	0\\
95.66	0\\
95.67	0\\
95.68	0\\
95.69	0\\
95.7	0\\
95.71	0\\
95.72	0\\
95.73	0\\
95.74	0\\
95.75	0\\
95.76	0\\
95.77	0\\
95.78	0\\
95.79	0\\
95.8	0\\
95.81	0\\
95.82	0\\
95.83	0\\
95.84	0\\
95.85	0\\
95.86	0\\
95.87	0\\
95.88	0\\
95.89	0\\
95.9	0\\
95.91	0\\
95.92	0\\
95.93	0\\
95.94	0\\
95.95	0\\
95.96	0\\
95.97	0\\
95.98	0\\
95.99	0\\
96	0\\
96.01	0\\
96.02	0\\
96.03	0\\
96.04	0\\
96.05	0\\
96.06	0\\
96.07	0\\
96.08	0\\
96.09	0\\
96.1	0\\
96.11	0\\
96.12	0\\
96.13	0\\
96.14	0\\
96.15	0\\
96.16	0\\
96.17	0\\
96.18	0\\
96.19	0\\
96.2	0\\
96.21	0\\
96.22	0\\
96.23	0\\
96.24	0\\
96.25	0\\
96.26	0\\
96.27	0\\
96.28	0\\
96.29	0\\
96.3	0\\
96.31	0\\
96.32	0\\
96.33	0\\
96.34	0\\
96.35	0\\
96.36	0\\
96.37	0\\
96.38	0\\
96.39	0\\
96.4	0\\
96.41	0\\
96.42	0\\
96.43	0\\
96.44	0\\
96.45	0\\
96.46	0\\
96.47	0\\
96.48	0\\
96.49	0\\
96.5	0\\
96.51	0\\
96.52	0\\
96.53	0\\
96.54	0\\
96.55	0\\
96.56	0\\
96.57	0\\
96.58	0\\
96.59	0\\
96.6	0\\
96.61	0\\
96.62	0\\
96.63	0\\
96.64	0\\
96.65	0\\
96.66	0\\
96.67	0\\
96.68	0\\
96.69	0\\
96.7	0\\
96.71	0\\
96.72	0\\
96.73	0\\
96.74	0\\
96.75	0\\
96.76	0\\
96.77	0\\
96.78	0\\
96.79	0\\
96.8	0\\
96.81	0\\
96.82	0\\
96.83	0\\
96.84	0\\
96.85	0\\
96.86	0\\
96.87	0\\
96.88	0\\
96.89	0\\
96.9	0\\
96.91	0\\
96.92	0\\
96.93	0\\
96.94	0\\
96.95	0\\
96.96	0\\
96.97	0\\
96.98	0\\
96.99	0\\
97	0\\
97.01	0\\
97.02	0\\
97.03	0\\
97.04	0\\
97.05	0\\
97.06	0\\
97.07	0\\
97.08	0\\
97.09	0\\
97.1	0\\
97.11	0\\
97.12	0\\
97.13	0\\
97.14	0\\
97.15	0\\
97.16	0\\
97.17	0\\
97.18	0\\
97.19	0\\
97.2	0\\
97.21	0\\
97.22	0\\
97.23	0\\
97.24	0\\
97.25	0\\
97.26	0\\
97.27	0\\
97.28	0\\
97.29	0\\
97.3	0\\
97.31	0\\
97.32	0\\
97.33	0\\
97.34	0\\
97.35	0\\
97.36	0\\
97.37	0\\
97.38	0\\
97.39	0\\
97.4	0\\
97.41	0\\
97.42	0\\
97.43	0\\
97.44	0\\
97.45	0\\
97.46	0\\
97.47	0\\
97.48	0\\
97.49	0\\
97.5	0\\
97.51	0\\
97.52	0\\
97.53	0\\
97.54	0\\
97.55	0\\
97.56	0\\
97.57	0\\
97.58	0\\
97.59	0\\
97.6	0\\
97.61	0\\
97.62	0\\
97.63	0\\
97.64	0\\
97.65	0\\
97.66	0\\
97.67	0\\
97.68	0\\
97.69	0\\
97.7	0\\
97.71	0\\
97.72	0\\
97.73	0\\
97.74	0\\
97.75	0\\
97.76	0\\
97.77	0\\
97.78	0\\
97.79	0\\
97.8	0\\
97.81	0\\
97.82	0\\
97.83	0\\
97.84	0\\
97.85	0\\
97.86	0\\
97.87	0\\
97.88	0\\
97.89	0\\
97.9	0\\
97.91	0\\
97.92	0\\
97.93	0\\
97.94	0\\
97.95	0\\
97.96	0\\
97.97	0\\
97.98	0\\
97.99	5.82904807695447e-05\\
98	0.00015409799186631\\
98.01	0.000250676523664825\\
98.02	0.000348033875184062\\
98.03	0.00044617793610998\\
98.04	0.00054511668815909\\
98.05	0.000644858206471345\\
98.06	0.000745410661033299\\
98.07	0.000813554165158825\\
98.08	0.000831324866783396\\
98.09	0.000849228356317379\\
98.1	0.000867265364232553\\
98.11	0.000885436614948054\\
98.12	0.000903742826422992\\
98.13	0.000922184709734451\\
98.14	0.000940762968640367\\
98.15	0.000959478299126647\\
98.16	0.000978331388938075\\
98.17	0.000997322917092383\\
98.18	0.00101645355337684\\
98.19	0.00103574920928199\\
98.2	0.0010552173123434\\
98.21	0.00107485953100661\\
98.22	0.00109467753629575\\
98.23	0.00111467301549213\\
98.24	0.0011348476722993\\
98.25	0.00115520322700986\\
98.26	0.00117574141667427\\
98.27	0.0011964639952716\\
98.28	0.00121737273388221\\
98.29	0.00123846942086252\\
98.3	0.00125975586202464\\
98.31	0.00128123388081602\\
98.32	0.00130290531850058\\
98.33	0.00132477203434225\\
98.34	0.00134683590579299\\
98.35	0.00136909882868069\\
98.36	0.00139156271739919\\
98.37	0.0014142294935727\\
98.38	0.00143710109677581\\
98.39	0.00146017948510352\\
98.4	0.00148346663534629\\
98.41	0.00150696454316288\\
98.42	0.00153067522325962\\
98.43	0.00155460070958288\\
98.44	0.00157874305550597\\
98.45	0.0016031043340145\\
98.46	0.00162768663821044\\
98.47	0.00165249208284321\\
98.48	0.00167752280288907\\
98.49	0.00170278095374452\\
98.5	0.00172826871142152\\
98.51	0.00175398827274457\\
98.52	0.00177994185554966\\
98.53	0.00180613169888516\\
98.54	0.00183256006304103\\
98.55	0.00185922922837249\\
98.56	0.0018861414970178\\
98.57	0.00191329919310589\\
98.58	0.00194070466296608\\
98.59	0.00196836027533968\\
98.6	0.00199626842159369\\
98.61	0.00202443151593641\\
98.62	0.00205285199563524\\
98.63	0.00208153232123635\\
98.64	0.00211047497678664\\
98.65	0.00213968247005763\\
98.66	0.00216915733277574\\
98.67	0.00219890212085219\\
98.68	0.0022289194146136\\
98.69	0.0022592118190388\\
98.7	0.00228978196399537\\
98.71	0.00232063250447734\\
98.72	0.00235176612084517\\
98.73	0.00238318551906794\\
98.74	0.00241489343096797\\
98.75	0.00244689261446764\\
98.76	0.00247918585383865\\
98.77	0.00251177595995369\\
98.78	0.00254466577054039\\
98.79	0.00257785815043781\\
98.8	0.00261135599185532\\
98.81	0.00264516221463393\\
98.82	0.00267927976651017\\
98.83	0.00271371162338246\\
98.84	0.0027484607678989\\
98.85	0.00278353018490016\\
98.86	0.00281892288730982\\
98.87	0.00285464191639842\\
98.88	0.00289069034204985\\
98.89	0.00292707126303038\\
98.9	0.00296378780726018\\
98.91	0.00300084313208736\\
98.92	0.00303824042456455\\
98.93	0.0030759829017282\\
98.94	0.00311407381088036\\
98.95	0.00315251642987323\\
98.96	0.00319131406739629\\
98.97	0.00323047006326616\\
98.98	0.0032699877887192\\
98.99	0.00330987064670684\\
99	0.00335012207219369\\
99.01	0.00339074553245839\\
99.02	0.00343174452739743\\
99.03	0.00347312258983165\\
99.04	0.00351488328581575\\
99.05	0.00355703021495063\\
99.06	0.00359956701069868\\
99.07	0.00364249734070207\\
99.08	0.00368582490710399\\
99.09	0.00372955344687287\\
99.1	0.00377368673212974\\
99.11	0.0038182285704786\\
99.12	0.00386318280533987\\
99.13	0.00390855331628703\\
99.14	0.00395434401938642\\
99.15	0.00400055886754013\\
99.16	0.00404720185083227\\
99.17	0.00409427699687832\\
99.18	0.00414178837117793\\
99.19	0.00418974007747085\\
99.2	0.00423813625809642\\
99.21	0.00428698109435617\\
99.22	0.00433627880688009\\
99.23	0.00438603365599616\\
99.24	0.00443624994210338\\
99.25	0.0044869320060484\\
99.26	0.00453808422950553\\
99.27	0.00458971103536046\\
99.28	0.00464181688809748\\
99.29	0.00469440629419043\\
99.3	0.0047474838024972\\
99.31	0.00480105400465803\\
99.32	0.00485512153549756\\
99.33	0.00490969107343051\\
99.34	0.00496476734087139\\
99.35	0.00502035510464784\\
99.36	0.00507645917641801\\
99.37	0.00513308441309174\\
99.38	0.00519023571725581\\
99.39	0.00524791803760305\\
99.4	0.0053061363693656\\
99.41	0.00536489575475216\\
99.42	0.00542420128338936\\
99.43	0.0054840580927673\\
99.44	0.00554447136868923\\
99.45	0.00560544634572549\\
99.46	0.00566698830767168\\
99.47	0.00572910258801115\\
99.48	0.0057917945703818\\
99.49	0.00585506968904733\\
99.5	0.00591893342937281\\
99.51	0.00598339132830483\\
99.52	0.00604844897485604\\
99.53	0.00611411201059433\\
99.54	0.00618038613013654\\
99.55	0.00624727708164684\\
99.56	0.00631479066733982\\
99.57	0.00638293274398813\\
99.58	0.00645170922343514\\
99.59	0.00652112607311221\\
99.6	0.00659118931656093\\
99.61	0.00666190501063383\\
99.62	0.00673327925932718\\
99.63	0.00680531822371098\\
99.64	0.00687802812246212\\
99.65	0.00695141523240246\\
99.66	0.007025485889042\\
99.67	0.007100246487127\\
99.68	0.0071757034811933\\
99.69	0.00725186338612484\\
99.7	0.00732873277771724\\
99.71	0.00740631829324684\\
99.72	0.00748462663204484\\
99.73	0.00756366455607695\\
99.74	0.00764343889052842\\
99.75	0.00772395652439445\\
99.76	0.0078052244110762\\
99.77	0.0078872495689823\\
99.78	0.00797003908213601\\
99.79	0.00805360010078805\\
99.8	0.00813793984203508\\
99.81	0.00822306559044401\\
99.82	0.00830898469868217\\
99.83	0.00839570458815322\\
99.84	0.00848323274963918\\
99.85	0.00857157674394827\\
99.86	0.00866074420256893\\
99.87	0.00875074282832986\\
99.88	0.00884158039606625\\
99.89	0.00893326475329224\\
99.9	0.00902580382087961\\
99.91	0.00911920559374282\\
99.92	0.00921347814153047\\
99.93	0.00930862960932319\\
99.94	0.00940466821833801\\
99.95	0.00950160226663938\\
99.96	0.00959944012985672\\
99.97	0.00969819026190876\\
99.98	0.0097978611957346\\
99.99	0.00989846154403157\\
100	0.01\\
};
\addlegendentry{$q=4$};

\end{axis}
\end{tikzpicture}% 
  \caption{Continuous Time w/ nFPC}
\end{subfigure}%
\hfill%
\begin{subfigure}{.45\linewidth}
  \centering
  \setlength\figureheight{\linewidth} 
  \setlength\figurewidth{\linewidth}
  \tikzsetnextfilename{dp_dscr_nFPC_z1}
  % This file was created by matlab2tikz.
%
%The latest updates can be retrieved from
%  http://www.mathworks.com/matlabcentral/fileexchange/22022-matlab2tikz-matlab2tikz
%where you can also make suggestions and rate matlab2tikz.
%
\definecolor{mycolor1}{rgb}{1.00000,0.00000,1.00000}%
%
\begin{tikzpicture}[trim axis left, trim axis right]

\begin{axis}[%
width=\figurewidth,
height=\figureheight,
at={(0\figurewidth,0\figureheight)},
scale only axis,
every outer x axis line/.append style={black},
every x tick label/.append style={font=\color{black}},
xmin=0,
xmax=100,
%xlabel={Time},
every outer y axis line/.append style={black},
every y tick label/.append style={font=\color{black}},
ymin=0,
ymax=0.015,
%ylabel={Depth $\delta^+$},
axis background/.style={fill=white},
axis x line*=bottom,
axis y line*=left,
yticklabel style={
        /pgf/number format/fixed,
        /pgf/number format/precision=3
},
scaled y ticks=false,
legend style={legend cell align=left,align=left,draw=black,font=\footnotesize, at={(0.98,0.02)},anchor=south east},
every axis legend/.code={\renewcommand\addlegendentry[2][]{}}  %ignore legend locally
]
\addplot [color=green,dashed]
  table[row sep=crcr]{%
1	0.0133744051204268\\
2	0.0133732367419305\\
3	0.013372012621659\\
4	0.0133707296376077\\
5	0.0133693844416798\\
6	0.0133679734340833\\
7	0.0133664927327869\\
8	0.0133649381367293\\
9	0.0133633050810592\\
10	0.0133615885820692\\
11	0.0133597831684727\\
12	0.0133578827938262\\
13	0.0133558807212804\\
14	0.013353769364907\\
15	0.0133515400614031\\
16	0.0133491827478389\\
17	0.0133466856452729\\
18	0.0133440353793949\\
19	0.0133412158592881\\
20	0.0133382063050905\\
21	0.0133349788013253\\
22	0.0133314710416767\\
23	0.0133256307720385\\
24	0.0133187268005979\\
25	0.0133115014701138\\
26	0.0133039373776573\\
27	0.0132960163863587\\
28	0.0132877198362999\\
29	0.0132790285640418\\
30	0.0132699147596394\\
31	0.0132603500365029\\
32	0.0132503085665101\\
33	0.013239763717221\\
34	0.013228688850036\\
35	0.0132170595327755\\
36	0.0132048604286939\\
37	0.0131921086965202\\
38	0.0131797678054959\\
39	0.0131679889039777\\
40	0.0131555356694942\\
41	0.0131421149743497\\
42	0.0131258966933831\\
43	0.0131089245905023\\
44	0.0130911359964698\\
45	0.0130725022575923\\
46	0.0130529714764019\\
47	0.0130325185289378\\
48	0.0130109566518488\\
49	0.0129881140253643\\
50	0.0129638608645403\\
51	0.0129380481864874\\
52	0.0129105243784227\\
53	0.0128811922969727\\
54	0.0128598883764592\\
55	0.0128306068684299\\
56	0.0127753578860927\\
57	0.0127172376519649\\
58	0.0126560324730213\\
59	0.0125914737189533\\
60	0.0125230922061851\\
61	0.0124499912528344\\
62	0.0123547517648388\\
63	0.0122515452911363\\
64	0.0121426092756864\\
65	0.0120270049315931\\
66	0.011904636521389\\
67	0.0118179901678075\\
68	0.0115567838500847\\
69	0.0112828861431057\\
70	0.0109997672384112\\
71	0.0107067748385756\\
72	0.0104015375333005\\
73	0.0100837067018434\\
74	0.00992625215172344\\
75	0.00980546347874705\\
76	0.0096934048262946\\
77	0.00958319731487932\\
78	0.00947687363401329\\
79	0.00937730501186515\\
80	0.00928370229659566\\
81	0.00918927822085656\\
82	0.00909405347503245\\
83	0.00899741276828745\\
84	0.00889997132590081\\
85	0.00880131112469073\\
86	0.00870132857036005\\
87	0.00856351667487943\\
88	0.00841531445529877\\
89	0.00789039112455869\\
90	0.0072929319095366\\
91	0.00701078656667871\\
92	0.00685454740616991\\
93	0.00670384513780405\\
94	0.00653293313820345\\
95	0.00630986941934306\\
96	0.00595502673688305\\
97	0.00526543684781529\\
98	0.00372263221541378\\
99	0\\
100	0\\
};
\addlegendentry{$q=-4$};

\addplot [color=mycolor1,dashed]
  table[row sep=crcr]{%
1	0.0134701341523511\\
2	0.0134698838074164\\
3	0.0134696212177822\\
4	0.0134693456751211\\
5	0.0134690564181801\\
6	0.0134687526272248\\
7	0.0134684334176678\\
8	0.0134680978327024\\
9	0.0134677448346896\\
10	0.0134673732949176\\
11	0.0134669819811305\\
12	0.0134665695418827\\
13	0.0134661344863399\\
14	0.0134656751580115\\
15	0.0134651897020547\\
16	0.0134646760277116\\
17	0.0134641317446514\\
18	0.0134635539709976\\
19	0.0134629389695026\\
20	0.0134622813431573\\
21	0.0134615719323341\\
22	0.0134607952077426\\
23	0.0134597060197198\\
24	0.013458462219672\\
25	0.0134571525380902\\
26	0.0134557717215507\\
27	0.0134543141405631\\
28	0.0134527745610136\\
29	0.0134511515065982\\
30	0.0134496365673057\\
31	0.0134481644305644\\
32	0.0134466071367077\\
33	0.0134449552087326\\
34	0.0134431966828015\\
35	0.0134413152914504\\
36	0.0134392850437182\\
37	0.013437051065151\\
38	0.0134337372524226\\
39	0.0134291853479914\\
40	0.0134244200364384\\
41	0.0134194295918729\\
42	0.0134142082959354\\
43	0.0134087367591064\\
44	0.01340299405703\\
45	0.0133969592026743\\
46	0.0133906050693801\\
47	0.0133838975917666\\
48	0.0133767800324706\\
49	0.0133691814885951\\
50	0.0133610198062789\\
51	0.0133522072898686\\
52	0.0133426081295019\\
53	0.0133319378461673\\
54	0.0133112179468777\\
55	0.0132882827281586\\
56	0.013261208971274\\
57	0.0132327388726622\\
58	0.0132027370966733\\
59	0.0131710394515216\\
60	0.0131374476562211\\
61	0.0131017447763351\\
62	0.013063726819675\\
63	0.013023206509652\\
64	0.0129799585299569\\
65	0.0129330499650752\\
66	0.0128834617458436\\
67	0.0127935938377722\\
68	0.0126768342462107\\
69	0.0125531936284494\\
70	0.0124225798726219\\
71	0.0123021342240074\\
72	0.0121732761797577\\
73	0.0120320260918441\\
74	0.0117352022919444\\
75	0.0113987368258269\\
76	0.0110505695981884\\
77	0.010744369337832\\
78	0.0104242052773378\\
79	0.010087955721042\\
80	0.00980491057455578\\
81	0.00965453383350001\\
82	0.00950948033803455\\
83	0.00936543906702952\\
84	0.00922342535285324\\
85	0.009086406347184\\
86	0.00895613361887325\\
87	0.0088308107021435\\
88	0.00870221359062729\\
89	0.00853718400456589\\
90	0.00836484218109529\\
91	0.00789013764499167\\
92	0.00728526236144176\\
93	0.00678359870846127\\
94	0.00656524169251083\\
95	0.00631925648355607\\
96	0.00595502673688305\\
97	0.00526543684781529\\
98	0.00372263221541378\\
99	0\\
100	0\\
};
\addlegendentry{$q=-3$};

\addplot [color=red,dashed]
  table[row sep=crcr]{%
1	0.0135343238337338\\
2	0.0135343031881343\\
3	0.0135342815207839\\
4	0.0135342587715904\\
5	0.0135342348759616\\
6	0.0135342097643806\\
7	0.0135341833619359\\
8	0.0135341555877958\\
9	0.013534126354607\\
10	0.0135340955677864\\
11	0.0135340631246484\\
12	0.0135340289132894\\
13	0.0135339928111224\\
14	0.0135339546828592\\
15	0.0135339143771539\\
16	0.0135338717187688\\
17	0.0135338264893583\\
18	0.0135337783913148\\
19	0.0135337269846451\\
20	0.0135336716069749\\
21	0.0135336114424889\\
22	0.013533546147137\\
23	0.0135334767115275\\
24	0.0135334032420029\\
25	0.0135333252834157\\
26	0.0135332422065085\\
27	0.0135331529493006\\
28	0.0135330551495803\\
29	0.0135329420596284\\
30	0.0135326395983859\\
31	0.0135322006858445\\
32	0.0135317298818973\\
33	0.0135312245018903\\
34	0.0135306814296354\\
35	0.0135300964502921\\
36	0.0135294627018532\\
37	0.0135287685457008\\
38	0.0135279141300613\\
39	0.0135268924274728\\
40	0.0135258288117781\\
41	0.0135247210277558\\
42	0.0135235661358127\\
43	0.013522360922056\\
44	0.0135211017585016\\
45	0.0135197841377087\\
46	0.0135184023061363\\
47	0.0135169486771325\\
48	0.0135154138758027\\
49	0.0135137856729811\\
50	0.0135120498571668\\
51	0.0135101868306629\\
52	0.013508162323659\\
53	0.013505918626762\\
54	0.0135023997061433\\
55	0.0134986600639254\\
56	0.013494721941114\\
57	0.0134905629719759\\
58	0.0134861564652107\\
59	0.013481470725003\\
60	0.0134764688245645\\
61	0.0134711077681729\\
62	0.013465334094982\\
63	0.0134590822794882\\
64	0.0134522220916364\\
65	0.0134444536701181\\
66	0.0134335987208681\\
67	0.0134132997832804\\
68	0.0133915739776063\\
69	0.0133682069536464\\
70	0.013342634583863\\
71	0.0132990061590845\\
72	0.0132518006087606\\
73	0.0132003193723743\\
74	0.0131268930102856\\
75	0.0130450097086445\\
76	0.0129562647097305\\
77	0.0128108664130651\\
78	0.0126565085755826\\
79	0.012492602525595\\
80	0.0122592283844567\\
81	0.0118966548453714\\
82	0.0115172997662786\\
83	0.0111236241951391\\
84	0.0107356419355688\\
85	0.0103354256377107\\
86	0.00993918903635679\\
87	0.00959922542729119\\
88	0.00940838952857641\\
89	0.00921620342488413\\
90	0.00901950584871832\\
91	0.00879044486617429\\
92	0.00854778579999546\\
93	0.00817351423989826\\
94	0.00747809308056406\\
95	0.00666550905412058\\
96	0.00606998573569038\\
97	0.00526543684781529\\
98	0.00372263221541378\\
99	0\\
100	0\\
};
\addlegendentry{$q=-2$};

\addplot [color=blue,dashed]
  table[row sep=crcr]{%
1	0.0136879538880162\\
2	0.0136879532488779\\
3	0.0136879525779846\\
4	0.013687951873426\\
5	0.0136879511331346\\
6	0.0136879503548642\\
7	0.0136879495361634\\
8	0.0136879486743404\\
9	0.0136879477664158\\
10	0.0136879468090551\\
11	0.0136879457984725\\
12	0.0136879447302841\\
13	0.0136879435992629\\
14	0.0136879423988728\\
15	0.0136879411203181\\
16	0.0136879397507781\\
17	0.013687938270854\\
18	0.0136879366518943\\
19	0.0136879348564106\\
20	0.0136879328509975\\
21	0.0136879306392626\\
22	0.0136879282825246\\
23	0.0136879257882397\\
24	0.0136879231269934\\
25	0.0136879202409398\\
26	0.0136879170061582\\
27	0.0136879131477299\\
28	0.0136879080619715\\
29	0.0136879005023335\\
30	0.0136878696178941\\
31	0.0136878223730618\\
32	0.013687771622661\\
33	0.0136877170204215\\
34	0.0136876581134322\\
35	0.0136875942975557\\
36	0.0136875249024927\\
37	0.0136874497442301\\
38	0.0136873701750985\\
39	0.0136872875218237\\
40	0.0136872016321279\\
41	0.0136871123218062\\
42	0.0136870193922346\\
43	0.0136869226176787\\
44	0.0136868217165809\\
45	0.0136867163266599\\
46	0.0136866059707012\\
47	0.0136864900270535\\
48	0.0136863676198777\\
49	0.0136862376111408\\
50	0.0136860985730614\\
51	0.0136859483592565\\
52	0.0136857843863199\\
53	0.013685605759584\\
54	0.0136854174125948\\
55	0.013685218964755\\
56	0.0136850091454994\\
57	0.0136847864183317\\
58	0.0136845489189158\\
59	0.0136842943742768\\
60	0.0136840199107826\\
61	0.0136837213724977\\
62	0.0136833917845932\\
63	0.0136830170366707\\
64	0.0136825704135559\\
65	0.013682009593348\\
66	0.0136810910233284\\
67	0.0136796485810653\\
68	0.0136780824538643\\
69	0.0136763435950804\\
70	0.0136743481687335\\
71	0.0136702200007839\\
72	0.0136657586250234\\
73	0.0136609249111855\\
74	0.0136556928823626\\
75	0.0136498127711416\\
76	0.0136427952026973\\
77	0.0136289106894291\\
78	0.013613508218495\\
79	0.0135940830735077\\
80	0.0135655108209789\\
81	0.0135202758193455\\
82	0.0134719887720633\\
83	0.0134158625421176\\
84	0.0133340471324284\\
85	0.013236598858992\\
86	0.0131036152376525\\
87	0.0128978590642612\\
88	0.0125466906969927\\
89	0.0121802255208229\\
90	0.0117961179867617\\
91	0.0113917221775055\\
92	0.0109639819858097\\
93	0.0104985873627227\\
94	0.00997772646175004\\
95	0.009409852668374\\
96	0.0083273620799101\\
97	0.00667948821753289\\
98	0.00372263221541378\\
99	0\\
100	0\\
};
\addlegendentry{$q=-1$};

\addplot [color=black,solid]
  table[row sep=crcr]{%
1	0.00575426534597786\\
2	0.00575427712397154\\
3	0.00575428941830398\\
4	0.00575430226788118\\
5	0.00575431572062637\\
6	0.00575432983959859\\
7	0.00575434471210626\\
8	0.00575436046555295\\
9	0.0057543772894913\\
10	0.00575439545556883\\
11	0.00575441530725482\\
12	0.00575443716041685\\
13	0.0057544610569919\\
14	0.0057544865049941\\
15	0.00575451291759309\\
16	0.00575454058526589\\
17	0.00575457014517644\\
18	0.00575460300814864\\
19	0.00575464223581203\\
20	0.00575469417316846\\
21	0.00575477096272872\\
22	0.00575489277030884\\
23	0.00575508421830758\\
24	0.00575535102563675\\
25	0.00575563174449665\\
26	0.00575592019313621\\
27	0.00575621670372833\\
28	0.00575652161415754\\
29	0.00575683524586936\\
30	0.00575715796110782\\
31	0.00575748994627936\\
32	0.00575783115194502\\
33	0.00575818174225828\\
34	0.00575854250784988\\
35	0.00575891514749755\\
36	0.00575930142410048\\
37	0.00575970226635192\\
38	0.00576011873645547\\
39	0.00576055201622807\\
40	0.00576100358077005\\
41	0.0057614754293691\\
42	0.00576197052180693\\
43	0.00576249352917535\\
44	0.00576305190885719\\
45	0.0057636567912641\\
46	0.00576432157743081\\
47	0.00576505316688385\\
48	0.0057658319171062\\
49	0.00576662813007013\\
50	0.00576744284506347\\
51	0.00576828083451999\\
52	0.00576915522815688\\
53	0.00577010100550168\\
54	0.00577121050329697\\
55	0.00577272356675477\\
56	0.0057752337607018\\
57	0.00577983812218373\\
58	0.00578779052079711\\
59	0.00579697759927671\\
60	0.00580648992571076\\
61	0.00581635920251381\\
62	0.00582662714043398\\
63	0.00583734787811777\\
64	0.0058485837602797\\
65	0.00586047959393243\\
66	0.00587331682384705\\
67	0.00588753370203374\\
68	0.00590349855825579\\
69	0.00591978275470636\\
70	0.0059362553444193\\
71	0.0059528480472759\\
72	0.00596935970958119\\
73	0.00598534130282059\\
74	0.00600003305315891\\
75	0.00601509020883893\\
76	0.0060327308514258\\
77	0.00605632144821932\\
78	0.00609543669896921\\
79	0.00617595177027325\\
80	0.00637904248331634\\
81	0.00660012577470798\\
82	0.00684186381926437\\
83	0.0071192556232119\\
84	0.00745372701190976\\
85	0.00780802271269009\\
86	0.00819449115781853\\
87	0.00859420524903984\\
88	0.00899516773316457\\
89	0.00942483938052141\\
90	0.009880663115063\\
91	0.0103520113689121\\
92	0.0108143885853798\\
93	0.0112307011044688\\
94	0.0115299065647041\\
95	0.0118453772828423\\
96	0.0121711431019151\\
97	0.0125871619692242\\
98	0.0131599673541368\\
99	0\\
100	0\\
};
\addlegendentry{$q=0$};

\addplot [color=blue,solid]
  table[row sep=crcr]{%
1	0.00611195743992147\\
2	0.00611210127744183\\
3	0.00611225049062715\\
4	0.00611240611526335\\
5	0.00611256858988664\\
6	0.00611273842077946\\
7	0.00611291622794206\\
8	0.00611310280488133\\
9	0.0061132992563266\\
10	0.00611350726200461\\
11	0.00611372950913175\\
12	0.00611397019046641\\
13	0.00611423489675197\\
14	0.00611452790128542\\
15	0.00611484414413289\\
16	0.00611517048364539\\
17	0.0061155080513918\\
18	0.00611585924259172\\
19	0.00611622985442528\\
20	0.00611663477537872\\
21	0.00611711258970825\\
22	0.00611776069710894\\
23	0.00611880917260985\\
24	0.00612072631493327\\
25	0.00612410424085045\\
26	0.00612771283262292\\
27	0.00613142105590445\\
28	0.00613523341110056\\
29	0.00613915495187843\\
30	0.0061431912025746\\
31	0.00614734755506389\\
32	0.00615162737795241\\
33	0.00615602863792215\\
34	0.0061605488507114\\
35	0.006165191724798\\
36	0.00616997982339883\\
37	0.00617494386640447\\
38	0.00618009472629881\\
39	0.00618544441124828\\
40	0.00619100634842221\\
41	0.00619679593343876\\
42	0.00620283173448316\\
43	0.00620913826743095\\
44	0.00621575228835088\\
45	0.00622273639201471\\
46	0.00623020544272944\\
47	0.00623836346882385\\
48	0.0062474859011764\\
49	0.00625741208837532\\
50	0.00626755911935922\\
51	0.00627791631623346\\
52	0.00628848474721297\\
53	0.00629927068719561\\
54	0.00631028700434949\\
55	0.00632161216158944\\
56	0.00633368587905997\\
57	0.00635053227887613\\
58	0.006385735778517\\
59	0.00648331820795726\\
60	0.00660615967268541\\
61	0.006734745832982\\
62	0.00686961317024147\\
63	0.00701144230654091\\
64	0.00716118762100751\\
65	0.00731941956784894\\
66	0.00748741533415371\\
67	0.00766813855059602\\
68	0.00786880137596036\\
69	0.00811211653436205\\
70	0.00836752968578314\\
71	0.00863320622186124\\
72	0.00890958995293047\\
73	0.00919607014322709\\
74	0.00948783313999121\\
75	0.00974280190693537\\
76	0.00998260618842222\\
77	0.0102408042912283\\
78	0.0105124813587247\\
79	0.0107992798224028\\
80	0.0109815129779682\\
81	0.0111687040420122\\
82	0.0113668692677404\\
83	0.0115619946015561\\
84	0.0117158656319735\\
85	0.011872362666853\\
86	0.01201615239816\\
87	0.0121571154521932\\
88	0.0122993275565428\\
89	0.0124305953346559\\
90	0.0125648143575899\\
91	0.0127091240524227\\
92	0.0128618523447465\\
93	0.0130087734130237\\
94	0.0131401900614982\\
95	0.013274348821181\\
96	0.0134137237378342\\
97	0.0135415125049663\\
98	0.0137226322154138\\
99	0\\
100	0\\
};
\addlegendentry{$q=1$};

\addplot [color=red,solid]
  table[row sep=crcr]{%
1	0.00632895508506573\\
2	0.00633084145153905\\
3	0.0063327898353037\\
4	0.00633480574088444\\
5	0.00633690749891728\\
6	0.00633910078889653\\
7	0.00634139195517411\\
8	0.00634378841593545\\
9	0.00634629897419818\\
10	0.00634893477738024\\
11	0.00635171156688937\\
12	0.00635465462117146\\
13	0.00635780852672469\\
14	0.00636125159689001\\
15	0.00636509480871216\\
16	0.00636932404119802\\
17	0.00637368409169078\\
18	0.0063781818127259\\
19	0.00638282692695676\\
20	0.00638763790603132\\
21	0.0063926620419921\\
22	0.00639804469230451\\
23	0.0064042700138639\\
24	0.00641300324110871\\
25	0.00643005399373983\\
26	0.00647586057939858\\
27	0.00652601122097035\\
28	0.00657779307943962\\
29	0.00663129019834863\\
30	0.00668659735846028\\
31	0.00674382286306238\\
32	0.00680308934229339\\
33	0.00686451733197731\\
34	0.00692811829741432\\
35	0.00699384934535099\\
36	0.00706167063042404\\
37	0.00713190383396789\\
38	0.00720520453156834\\
39	0.00728179311926532\\
40	0.00736191397046987\\
41	0.00744583920216459\\
42	0.00753387381806631\\
43	0.0076263640728364\\
44	0.00772371474744643\\
45	0.00782643087124711\\
46	0.00793522919980522\\
47	0.0080513624552378\\
48	0.00817759912090745\\
49	0.00832182390107149\\
50	0.00848547917420062\\
51	0.00865556754041888\\
52	0.00883213841468531\\
53	0.00901545543800836\\
54	0.00920581918513799\\
55	0.00940305290874223\\
56	0.00960530765063492\\
57	0.00978760451252322\\
58	0.00993894072617349\\
59	0.0100459757563916\\
60	0.0101380258113839\\
61	0.0102340390513143\\
62	0.010334289505668\\
63	0.0104391142575754\\
64	0.0105490224994677\\
65	0.010675222322629\\
66	0.0108104263538037\\
67	0.0109487546910201\\
68	0.01109016555593\\
69	0.011204328632062\\
70	0.0113187943178877\\
71	0.011436122972429\\
72	0.0115570879763876\\
73	0.0116816137216073\\
74	0.0118146720086514\\
75	0.0119529324635412\\
76	0.0120722881679121\\
77	0.0121854604739172\\
78	0.0122965444770228\\
79	0.0124042048077948\\
80	0.0124983168303273\\
81	0.012586216201087\\
82	0.012671290989184\\
83	0.0127516837990635\\
84	0.0128238839241057\\
85	0.0128985199103956\\
86	0.0129683087063421\\
87	0.0130325766610747\\
88	0.0130961099206039\\
89	0.0131616966999991\\
90	0.0132309829015184\\
91	0.0132954532730901\\
92	0.0133510563634734\\
93	0.0134012434829163\\
94	0.0134449411682291\\
95	0.0134857845974406\\
96	0.013531098659188\\
97	0.0135945905484418\\
98	0.0137226322154138\\
99	0\\
100	0\\
};
\addlegendentry{$q=2$};

\addplot [color=mycolor1,solid]
  table[row sep=crcr]{%
1	0.00774242316317075\\
2	0.00777118496557083\\
3	0.00780100853035913\\
4	0.00783187552525477\\
5	0.00786378531598214\\
6	0.00789714489015466\\
7	0.00793205595309538\\
8	0.00796863099533649\\
9	0.00800699855745782\\
10	0.00804730429781573\\
11	0.00808971652244768\\
12	0.00813444044633872\\
13	0.00818175699480529\\
14	0.00823213582747507\\
15	0.00828657346284589\\
16	0.00834781258156791\\
17	0.00841762993618121\\
18	0.0084900885891944\\
19	0.00856533307147854\\
20	0.00864351013304426\\
21	0.00872474624884345\\
22	0.00880907110406114\\
23	0.00889616214417913\\
24	0.00898449243628432\\
25	0.00906847528986695\\
26	0.00912878115334637\\
27	0.00918840211423965\\
28	0.00925000266547761\\
29	0.0093136284707402\\
30	0.00937932565915868\\
31	0.00944715471556349\\
32	0.00951723756748327\\
33	0.00958990814091621\\
34	0.00966620147850951\\
35	0.00974420748530797\\
36	0.00981982208491264\\
37	0.00988499969562246\\
38	0.00993879301745489\\
39	0.00999439685913653\\
40	0.0100518746904371\\
41	0.0101112905981897\\
42	0.0101727087860881\\
43	0.0102361925886073\\
44	0.0103018009795629\\
45	0.0103695883173195\\
46	0.0104396182728121\\
47	0.0105119760223757\\
48	0.0105867235870087\\
49	0.0106571035795354\\
50	0.0107182318800902\\
51	0.0107828099100451\\
52	0.0108590729347671\\
53	0.0109408774427858\\
54	0.0110246576997461\\
55	0.0111102306182809\\
56	0.0111972095240246\\
57	0.0112850804328061\\
58	0.0113694380269561\\
59	0.0114487988228078\\
60	0.0115271342816182\\
61	0.011606879986735\\
62	0.011687876169651\\
63	0.0117698951856426\\
64	0.0118525851662363\\
65	0.0119255653130305\\
66	0.0119962295426003\\
67	0.012073892260441\\
68	0.0121504328153382\\
69	0.0122232501316895\\
70	0.0122952662059049\\
71	0.0123662888005724\\
72	0.0124357127132211\\
73	0.0125033780283189\\
74	0.0125632347167128\\
75	0.012617490066832\\
76	0.0126686278090238\\
77	0.0127179740493617\\
78	0.0127718015915979\\
79	0.012831249623831\\
80	0.0128925563906612\\
81	0.0129506770975516\\
82	0.0130004956983336\\
83	0.0130495479118441\\
84	0.0130985645731492\\
85	0.0131413761405173\\
86	0.0131849800381144\\
87	0.0132280361637285\\
88	0.013270860866495\\
89	0.0133101615301819\\
90	0.0133463040606936\\
91	0.013382197074282\\
92	0.0134133556601411\\
93	0.0134406278163612\\
94	0.0134672391046149\\
95	0.0134961142982946\\
96	0.0135332854097662\\
97	0.0135945905484418\\
98	0.0137226322154138\\
99	0\\
100	0\\
};
\addlegendentry{$q=3$};

\addplot [color=green,solid]
  table[row sep=crcr]{%
1	0.00978131038814559\\
2	0.00981637783802372\\
3	0.0098519579948201\\
4	0.00988725161043769\\
5	0.00992001432137705\\
6	0.00994439781645423\\
7	0.00996954322639673\\
8	0.00999543947896755\\
9	0.0100220435934237\\
10	0.010049343870241\\
11	0.0100773304180641\\
12	0.0101059866537501\\
13	0.0101352886140559\\
14	0.0101652053125276\\
15	0.0101956993675745\\
16	0.010224732170877\\
17	0.0102489349038127\\
18	0.0102738240953897\\
19	0.0102994033395041\\
20	0.0103256719693691\\
21	0.0103526222164455\\
22	0.0103802312855305\\
23	0.0104084365821289\\
24	0.0104370541520036\\
25	0.010465504381932\\
26	0.0104925425304635\\
27	0.0105204010627644\\
28	0.010549401409227\\
29	0.0105796321045768\\
30	0.0106111923906807\\
31	0.0106442033340564\\
32	0.0106788205221327\\
33	0.0107152763193217\\
34	0.010753946298309\\
35	0.0108003937513058\\
36	0.0108498679795693\\
37	0.0108992106234798\\
38	0.0109483004090331\\
39	0.0109985185663909\\
40	0.0110498343615476\\
41	0.0111022043669381\\
42	0.0111555706211653\\
43	0.011209860248281\\
44	0.011264991773661\\
45	0.0113207914424315\\
46	0.0113768696379083\\
47	0.0114328769696175\\
48	0.0114891674566955\\
49	0.011544946092546\\
50	0.0115998378016495\\
51	0.0116550428630709\\
52	0.0117029045664042\\
53	0.0117486353736254\\
54	0.011795319388958\\
55	0.0118429933566652\\
56	0.0118917177130443\\
57	0.011941610640322\\
58	0.011994539716604\\
59	0.0120529720239776\\
60	0.0121112138643652\\
61	0.0121690899593177\\
62	0.0122263408119997\\
63	0.0122826633947296\\
64	0.0123376954152201\\
65	0.012390153396452\\
66	0.012439404022673\\
67	0.0124814243912259\\
68	0.0125229229511473\\
69	0.0125639119594753\\
70	0.0126042857995569\\
71	0.0126438992303004\\
72	0.012682653040563\\
73	0.01272945580353\\
74	0.0127809022671538\\
75	0.0128345975476407\\
76	0.0128857948058951\\
77	0.0129355069189137\\
78	0.0129793137219755\\
79	0.0130147563130651\\
80	0.013044942059718\\
81	0.0130750041776383\\
82	0.0131068595515389\\
83	0.0131452874277376\\
84	0.0131834419894261\\
85	0.013220708759408\\
86	0.0132500695219427\\
87	0.0132795242628055\\
88	0.0133096673494054\\
89	0.0133412986932403\\
90	0.0133695116718789\\
91	0.0133948042804017\\
92	0.0134190923181857\\
93	0.013443060749888\\
94	0.0134680597695799\\
95	0.013496272944263\\
96	0.0135332854097662\\
97	0.0135945905484418\\
98	0.0137226322154138\\
99	0\\
100	0\\
};
\addlegendentry{$q=4$};

\end{axis}
\end{tikzpicture}%
 
  \caption{Discrete Time w/ nFPC}
\end{subfigure}\\

\leavevmode\smash{\makebox[0pt]{\hspace{-7em}% HORIZONTAL POSITION           
  \rotatebox[origin=l]{90}{\hspace{20em}% VERTICAL POSITION
    Depth $\delta^+$}%
}}\hspace{0pt plus 1filll}\null

Time (s)

\vspace{1cm}
\begin{subfigure}{\linewidth}
  \centering
  \tikzsetnextfilename{deltalegend}
  \documentclass{article}
\usepackage{pgfplots}
\usetikzlibrary{backgrounds}
\pgfplotsset{compat=newest}  
\newlength\figureheight 
\newlength\figurewidth 

\begin{document}
%
%\begin{figure}
%  \centering
%  \setlength\figureheight{\linewidth} 
%  \setlength\figurewidth{\linewidth}
%  \input{/home/anton/Documents/masc/ml/thesis/tikz/ORCL_comp4stoch.tikz}
%  \caption{Backtest strategy comparison}
%  \label{fig:insample}
%\end{figure}
\definecolor{mycolor1}{rgb}{1.00000,0.00000,1.00000}%
\begin{tikzpicture}[framed]
    \begingroup
    % inits/clears the lists (which might be populated from previous
    % axes):
    \csname pgfplots@init@cleared@structures\endcsname
    \pgfplotsset{legend style={at={(0,1)},anchor=north west},legend columns=-1,legend style={draw=none,column sep=1ex},legend entries={$q=-4$,$q=-3$,$q=-2$,$q=-1$}}%
    
    \csname pgfplots@addlegendimage\endcsname{thick,green,dashed,sharp plot}
    \csname pgfplots@addlegendimage\endcsname{thick,mycolor1,dashed,sharp plot}
    \csname pgfplots@addlegendimage\endcsname{thick,red,dashed,sharp plot}
    \csname pgfplots@addlegendimage\endcsname{thick,blue,dashed,sharp plot}

    % draws the legend:
    \csname pgfplots@createlegend\endcsname
    \endgroup

    \begingroup
    % inits/clears the lists (which might be populated from previous
    % axes):
    \csname pgfplots@init@cleared@structures\endcsname
    \pgfplotsset{legend style={at={(3.45,0.5)},anchor=north west},legend columns=-1,legend style={draw=none,column sep=1ex},legend entries={$q=0$}}%

    \csname pgfplots@addlegendimage\endcsname{thick,black,sharp plot}

    % draws the legend:
    \csname pgfplots@createlegend\endcsname
    \endgroup

    \begingroup
    % inits/clears the lists (which might be populated from previous
    % axes):
    \csname pgfplots@init@cleared@structures\endcsname
    \pgfplotsset{legend style={at={(0,0)},anchor=north west},legend columns=-1,legend style={draw=none,column sep=1ex},legend entries={$q=+4$,$q=+3$,$q=+2$,$q=+1$}}%
    
    \csname pgfplots@addlegendimage\endcsname{thick,green,sharp plot}
    \csname pgfplots@addlegendimage\endcsname{thick,mycolor1,sharp plot}
    \csname pgfplots@addlegendimage\endcsname{thick,red,sharp plot}
    \csname pgfplots@addlegendimage\endcsname{thick,blue,sharp plot}

    % draws the legend:
    \csname pgfplots@createlegend\endcsname
    \endgroup
\end{tikzpicture}

\end{document} 
\end{subfigure}%
  \caption{Optimal buy depths $\delta^{+}$ for Markov state $Z=(\rho = -1, \Delta S = -1)$, implying heavy imbalance in favor of sell pressure, and having previously seen a downward price change. We expect the midprice to fall.}
  \label{fig:comp_dp_z1}
\end{figure}

\begin{figure}
\centering
\begin{subfigure}{.45\linewidth}
  \centering
  \setlength\figureheight{\linewidth} 
  \setlength\figurewidth{\linewidth}
  \tikzsetnextfilename{dp_cts_z8}
  % This file was created by matlab2tikz.
%
%The latest updates can be retrieved from
%  http://www.mathworks.com/matlabcentral/fileexchange/22022-matlab2tikz-matlab2tikz
%where you can also make suggestions and rate matlab2tikz.
%
\definecolor{mycolor1}{rgb}{1.00000,0.00000,1.00000}%
%
\begin{tikzpicture}[trim axis left, trim axis right]

\begin{axis}[%
width=\figurewidth,
height=\figureheight,
at={(0\figurewidth,0\figureheight)},
scale only axis,
every outer x axis line/.append style={black},
every x tick label/.append style={font=\color{black}},
xmin=0,
xmax=100,
%xlabel={Time},
every outer y axis line/.append style={black},
every y tick label/.append style={font=\color{black}},
ymin=0,
ymax=0.015,
%ylabel={Depth $\delta^+$},
axis background/.style={fill=white},
axis x line*=bottom,
axis y line*=left,
yticklabel style={
        /pgf/number format/fixed,
        /pgf/number format/precision=3
},
scaled y ticks=false,
legend style={legend cell align=left,align=left,draw=black,font=\footnotesize, at={(0.98,0.02)},anchor=south east},
every axis legend/.code={\renewcommand\addlegendentry[2][]{}}  %ignore legend locally
]
\addplot [color=green,dashed]
  table[row sep=crcr]{%
0.01	0.00574337684382281\\
1.01	0.00573714257097391\\
2.01	0.00573060439045445\\
3.01	0.00572374696472981\\
4.01	0.00571655398577405\\
5.01	0.00570900807581416\\
6.01	0.00570109068452653\\
7.01	0.00569278199238071\\
8.01	0.00568406083785614\\
9.01	0.00567490469696307\\
10.01	0.00566528970019458\\
11.01	0.0056551905933952\\
12.01	0.00564458066781824\\
13.01	0.00563343168900105\\
14.01	0.00562171382645682\\
15.01	0.00560939558542931\\
16.01	0.00559644374225158\\
17.01	0.00558282328518515\\
18.01	0.00556849736298323\\
19.01	0.00555342724381153\\
20.01	0.00553757228754437\\
21.01	0.00552088993479424\\
22.01	0.00550333571625747\\
23.01	0.00548486328596039\\
24.01	0.00546542448160292\\
25.01	0.00544496941416772\\
26.01	0.00542344658691833\\
27.01	0.00540080304029837\\
28.01	0.00537698451325962\\
29.01	0.00535193560202306\\
30.01	0.00532559988252652\\
31.01	0.00529791994042451\\
32.01	0.00526883721902203\\
33.01	0.005238291546086\\
34.01	0.00520622012872673\\
35.01	0.0051725557061639\\
36.01	0.00513722339215991\\
37.01	0.00510013541902272\\
38.01	0.00506118484307974\\
39.01	0.00502025109539953\\
40.01	0.00497720620061289\\
41.01	0.00493191585711632\\
42.01	0.00488424099449774\\
43.01	0.00483404033902436\\
44.01	0.00478117440670781\\
45.01	0.00472551152196101\\
46.01	0.00466693694772569\\
47.01	0.00460536706918557\\
48.01	0.00454076724158745\\
49.01	0.00447317647854652\\
50.01	0.00440274318473162\\
51.01	0.00432977495393119\\
52.01	0.00425480766396775\\
53.01	0.00417875845854279\\
54.01	0.0041025512313847\\
55.01	0.00402649167987937\\
56.01	0.00395085846478194\\
57.01	0.0038759226263916\\
58.01	0.00380192181583351\\
59.01	0.00372902083263049\\
60.01	0.00365727270584941\\
61.01	0.00358678068417524\\
62.01	0.00351765955583304\\
63.01	0.003449865257216\\
64.01	0.0033830836898928\\
65.01	0.00331649676251478\\
66.01	0.00324921533936638\\
67.01	0.00318119667142671\\
68.01	0.00311243053015165\\
69.01	0.00304287603760009\\
70.01	0.00297245348148554\\
71.01	0.00290103585103694\\
72.01	0.00282844183185719\\
73.01	0.00275445855944312\\
74.01	0.00267886954662985\\
75.01	0.00260146359874958\\
76.01	0.00252208658155456\\
77.01	0.00244064040944975\\
78.01	0.00235702519206653\\
79.01	0.0022711388415078\\
80.01	0.00218288815641267\\
81.01	0.0020921920300596\\
82.01	0.0019989736781259\\
83.01	0.00190316156414087\\
84.01	0.00180469257861682\\
85.01	0.00170351474884668\\
86.01	0.00159958740458458\\
87.01	0.00149287866220315\\
88.01	0.00138336624866992\\
89.01	0.00127104122721427\\
90.01	0.00115591289641989\\
91.01	0.00103801510632268\\
92.01	0.000917414847049264\\
93.01	0.000794223772923664\\
94.01	0.000668613273120433\\
95.01	0.000540834094057197\\
96.01	0.000411241866149138\\
97.01	0.000280330070258007\\
98.01	0.000148772403259365\\
99.01	2.53717161561775e-05\\
99.02	2.46546563816945e-05\\
99.03	2.39601159352475e-05\\
99.04	2.32882766082222e-05\\
99.05	2.26246795302226e-05\\
99.06	2.1969188376313e-05\\
99.07	2.1321850006863e-05\\
99.08	2.06827099941027e-05\\
99.09	2.00518125263397e-05\\
99.1	1.94292049667238e-05\\
99.11	1.88149349641432e-05\\
99.12	1.82090488000525e-05\\
99.13	1.76115912866801e-05\\
99.14	1.70226057255845e-05\\
99.15	1.64421345168805e-05\\
99.16	1.58702182876998e-05\\
99.17	1.53068957691096e-05\\
99.18	1.47522130564508e-05\\
99.19	1.42062267174219e-05\\
99.2	1.36689923055655e-05\\
99.21	1.31405642745074e-05\\
99.22	1.26210036445577e-05\\
99.23	1.21103976940592e-05\\
99.24	1.16088345337613e-05\\
99.25	1.11164031182942e-05\\
99.26	1.06331932580306e-05\\
99.27	1.01592956314039e-05\\
99.28	9.6948017976814e-06\\
99.29	9.23980421022562e-06\\
99.3	8.79439623028763e-06\\
99.31	8.35867214133909e-06\\
99.32	7.93272716396315e-06\\
99.33	7.51665747139585e-06\\
99.34	7.11056020567293e-06\\
99.35	6.71453349447744e-06\\
99.36	6.32867646870733e-06\\
99.37	5.95308927810352e-06\\
99.38	5.58787308975114e-06\\
99.39	5.23313010598533e-06\\
99.4	4.88896356993541e-06\\
99.41	4.55547778330928e-06\\
99.42	4.23277812370594e-06\\
99.43	3.92097106434626e-06\\
99.44	3.62016419468321e-06\\
99.45	3.33046620878789e-06\\
99.46	3.05198690398953e-06\\
99.47	2.78483719876564e-06\\
99.48	2.52912911125232e-06\\
99.49	2.284975724548e-06\\
99.5	2.05249119692753e-06\\
99.51	1.83179077218112e-06\\
99.52	1.62299079003135e-06\\
99.53	1.42620869666987e-06\\
99.54	1.24156305541731e-06\\
99.55	1.06917355746118e-06\\
99.56	9.09161032741287e-07\\
99.57	7.61647460904499e-07\\
99.58	6.2675598241739e-07\\
99.59	5.04610909760408e-07\\
99.6	3.95337738760826e-07\\
99.61	2.99063160008953e-07\\
99.62	2.1591507044609e-07\\
99.63	1.46022585009731e-07\\
99.64	8.95160484556956e-08\\
99.65	4.65270472548712e-08\\
99.66	1.71884216391277e-08\\
99.67	1.63427776520009e-09\\
99.68	0\\
99.69	0\\
99.7	0\\
99.71	0\\
99.72	0\\
99.73	0\\
99.74	0\\
99.75	0\\
99.76	0\\
99.77	0\\
99.78	0\\
99.79	0\\
99.8	0\\
99.81	0\\
99.82	0\\
99.83	0\\
99.84	0\\
99.85	0\\
99.86	0\\
99.87	0\\
99.88	0\\
99.89	0\\
99.9	0\\
99.91	0\\
99.92	0\\
99.93	0\\
99.94	0\\
99.95	0\\
99.96	0\\
99.97	0\\
99.98	0\\
99.99	0\\
100	0\\
};
\addlegendentry{$q=-4$};

\addplot [color=mycolor1,dashed]
  table[row sep=crcr]{%
0.01	0.0070186501501409\\
1.01	0.00701068385965739\\
2.01	0.00700232779984816\\
3.01	0.006993562461304\\
4.01	0.00698436732943601\\
5.01	0.00697472083459523\\
6.01	0.00696460030094368\\
7.01	0.0069539818940995\\
8.01	0.00694284056726258\\
9.01	0.0069311500048925\\
10.01	0.00691888256247217\\
11.01	0.00690600920241112\\
12.01	0.00689249942714721\\
13.01	0.00687832120965122\\
14.01	0.00686344092132698\\
15.01	0.00684782325730622\\
16.01	0.00683143115915048\\
17.01	0.00681422573498196\\
18.01	0.0067961661770761\\
19.01	0.00677720967695822\\
20.01	0.00675731133805484\\
21.01	0.00673642408595915\\
22.01	0.00671449857637999\\
23.01	0.00669148310086328\\
24.01	0.00666732349040707\\
25.01	0.00664196301715866\\
26.01	0.00661534229450709\\
27.01	0.00658739917611283\\
28.01	0.00655806865481995\\
29.01	0.00652728276308499\\
30.01	0.00649497047770828\\
31.01	0.00646105763353036\\
32.01	0.0064254668537741\\
33.01	0.00638811750948853\\
34.01	0.00634892572806825\\
35.01	0.00630780448239685\\
36.01	0.00626466380683602\\
37.01	0.00621941120759202\\
38.01	0.00617195238109056\\
39.01	0.00612219215013244\\
40.01	0.00607003542773831\\
41.01	0.00601538837875547\\
42.01	0.00595815987264445\\
43.01	0.00589826325585382\\
44.01	0.00583561822829223\\
45.01	0.00577014956628841\\
46.01	0.00570174280475895\\
47.01	0.0056302266636106\\
48.01	0.00555554040879982\\
49.01	0.00547765856717402\\
50.01	0.00539658492007209\\
51.01	0.00531235803293239\\
52.01	0.00522504130473293\\
53.01	0.00513472583247616\\
54.01	0.00504163298858386\\
55.01	0.00494611778859616\\
56.01	0.00484871241380105\\
57.01	0.00475018608849895\\
58.01	0.00465162432451925\\
59.01	0.00455453349731493\\
60.01	0.00445886540992659\\
61.01	0.0043591992129649\\
62.01	0.00425526116606771\\
63.01	0.00414746353248364\\
64.01	0.00403653085849659\\
65.01	0.00392370339757333\\
66.01	0.00381027559922914\\
67.01	0.00369676154302832\\
68.01	0.00358371865321317\\
69.01	0.00347179165468971\\
70.01	0.00336171774942204\\
71.01	0.00325432753083376\\
72.01	0.00315028558852393\\
73.01	0.0030483274889385\\
74.01	0.00294799391403962\\
75.01	0.0028487361812953\\
76.01	0.00274930729110097\\
77.01	0.00264932492597471\\
78.01	0.00254867975966727\\
79.01	0.0024470542028248\\
80.01	0.00234391771000994\\
81.01	0.00223902237671757\\
82.01	0.00213225833422466\\
83.01	0.00202350108567132\\
84.01	0.00191261613460473\\
85.01	0.00179948973689244\\
86.01	0.00168408845874976\\
87.01	0.00156645972418092\\
88.01	0.00144666674474913\\
89.01	0.00132478455082746\\
90.01	0.00120091154714239\\
91.01	0.0010751732144873\\
92.01	0.000947717112789589\\
93.01	0.000818715889138384\\
94.01	0.000688377198913548\\
95.01	0.000556952354343623\\
96.01	0.000424750104796532\\
97.01	0.00029215908729637\\
98.01	0.000159680719535645\\
99.01	2.96570498975329e-05\\
99.02	2.85693280234946e-05\\
99.03	2.75032330208091e-05\\
99.04	2.64589315094269e-05\\
99.05	2.54499131151879e-05\\
99.06	2.44715387902536e-05\\
99.07	2.35240465486578e-05\\
99.08	2.2607677821718e-05\\
99.09	2.17226775639217e-05\\
99.1	2.08671890345388e-05\\
99.11	2.00407137099859e-05\\
99.12	1.92434815356798e-05\\
99.13	1.84757259127401e-05\\
99.14	1.7737653948394e-05\\
99.15	1.70291623237742e-05\\
99.16	1.63505144308868e-05\\
99.17	1.57019777590004e-05\\
99.18	1.50801239827813e-05\\
99.19	1.44804907248487e-05\\
99.2	1.39032743025295e-05\\
99.21	1.33486735977156e-05\\
99.22	1.28139271072334e-05\\
99.23	1.22890135552806e-05\\
99.24	1.17740031484286e-05\\
99.25	1.12689654332906e-05\\
99.26	1.07739692289735e-05\\
99.27	1.02890825562797e-05\\
99.28	9.81437256350297e-06\\
99.29	9.34990544862631e-06\\
99.3	8.89574637778952e-06\\
99.31	8.45195939980052e-06\\
99.32	8.01860735655247e-06\\
99.33	7.59575178907977e-06\\
99.34	7.18345283908656e-06\\
99.35	6.78176913591205e-06\\
99.36	6.39075768814855e-06\\
99.37	6.01047517948851e-06\\
99.38	5.64098714129178e-06\\
99.39	5.28235774042984e-06\\
99.4	4.93465549333653e-06\\
99.41	4.59794786000257e-06\\
99.42	4.27230170910176e-06\\
99.43	3.95778245825679e-06\\
99.44	3.65445395983724e-06\\
99.45	3.36239097627374e-06\\
99.46	3.08167491083755e-06\\
99.47	2.81238642052886e-06\\
99.48	2.55461999874342e-06\\
99.49	2.30848818367178e-06\\
99.5	2.07410454113217e-06\\
99.51	1.85158367354936e-06\\
99.52	1.64104122908626e-06\\
99.53	1.44259391091431e-06\\
99.54	1.25635948660871e-06\\
99.55	1.08245679775187e-06\\
99.56	9.21005769639188e-07\\
99.57	7.72127421229396e-07\\
99.58	6.35943875214665e-07\\
99.59	5.1257836833872e-07\\
99.6	4.02155261886722e-07\\
99.61	3.04800052421464e-07\\
99.62	2.20639382726012e-07\\
99.63	1.49801053020426e-07\\
99.64	9.24140324074646e-08\\
99.65	4.86084706270706e-08\\
99.66	1.85157100797417e-08\\
99.67	2.26829817602525e-09\\
99.68	0\\
99.69	0\\
99.7	0\\
99.71	0\\
99.72	0\\
99.73	0\\
99.74	0\\
99.75	0\\
99.76	0\\
99.77	0\\
99.78	0\\
99.79	0\\
99.8	0\\
99.81	0\\
99.82	0\\
99.83	0\\
99.84	0\\
99.85	0\\
99.86	0\\
99.87	0\\
99.88	0\\
99.89	0\\
99.9	0\\
99.91	0\\
99.92	0\\
99.93	0\\
99.94	0\\
99.95	0\\
99.96	0\\
99.97	0\\
99.98	0\\
99.99	0\\
100	0\\
};
\addlegendentry{$q=-3$};

\addplot [color=red,dashed]
  table[row sep=crcr]{%
0.01	0.00976935093545758\\
1.01	0.00976421015741647\\
2.01	0.0097588215886309\\
3.01	0.00975317302442196\\
4.01	0.00974725162725591\\
5.01	0.00974104388983768\\
6.01	0.00973453559548323\\
7.01	0.00972771177549045\\
8.01	0.00972055666319548\\
9.01	0.00971305364437342\\
10.01	0.00970518520362485\\
11.01	0.0096969328663622\\
12.01	0.00968827713594452\\
13.01	0.00967919742543643\\
14.01	0.00966967198339799\\
15.01	0.00965967781303261\\
16.01	0.00964919058392876\\
17.01	0.00963818453552645\\
18.01	0.00962663237131716\\
19.01	0.00961450514264473\\
20.01	0.0096017721208111\\
21.01	0.0095884006559997\\
22.01	0.00957435602130762\\
23.01	0.00955960123991775\\
24.01	0.00954409689313871\\
25.01	0.00952780090668435\\
26.01	0.00951066831214459\\
27.01	0.00949265098010407\\
28.01	0.00947369732077701\\
29.01	0.00945375194732381\\
30.01	0.00943275529617063\\
31.01	0.00941064319762986\\
32.01	0.0093873463888656\\
33.01	0.00936278995970001\\
34.01	0.00933689271983622\\
35.01	0.00930956647373386\\
36.01	0.00928071518658991\\
37.01	0.00925023401925399\\
38.01	0.00921800819881499\\
39.01	0.00918391170311225\\
40.01	0.00914780571916437\\
41.01	0.00910953682961546\\
42.01	0.00906893487004464\\
43.01	0.00902581036899637\\
44.01	0.00897995137612937\\
45.01	0.00893111902912432\\
46.01	0.00887904107466826\\
47.01	0.00882341174230693\\
48.01	0.00876388838872182\\
49.01	0.00870008027497436\\
50.01	0.00863153938499928\\
51.01	0.00855774912691809\\
52.01	0.00847810958725209\\
53.01	0.00839192360633966\\
54.01	0.00829837906931216\\
55.01	0.00819652145168077\\
56.01	0.00808522218423392\\
57.01	0.00796313916492984\\
58.01	0.00782866744072175\\
59.01	0.00767988038378758\\
60.01	0.00751656545295311\\
61.01	0.00734383749457317\\
62.01	0.00716163614187467\\
63.01	0.00696939652633905\\
64.01	0.0067665180071746\\
65.01	0.00655237162172994\\
66.01	0.00632629300831234\\
67.01	0.00608748509419098\\
68.01	0.0058350537733886\\
69.01	0.00556801218100254\\
70.01	0.00528534038425278\\
71.01	0.00498602044039237\\
72.01	0.00472967698385689\\
73.01	0.00454160306231707\\
74.01	0.0043497707016179\\
75.01	0.00415631635582206\\
76.01	0.0039643806323788\\
77.01	0.0037765873398881\\
78.01	0.00359581560290177\\
79.01	0.00342581610811684\\
80.01	0.0032641071483379\\
81.01	0.00310160122367578\\
82.01	0.00293840060970903\\
83.01	0.0027746067283078\\
84.01	0.00261009244484884\\
85.01	0.00244433973219168\\
86.01	0.00227619013466552\\
87.01	0.00210554807654228\\
88.01	0.00193279979337896\\
89.01	0.00175819184561082\\
90.01	0.00158171088518645\\
91.01	0.00140375913670724\\
92.01	0.00122509437063391\\
93.01	0.00104655139364088\\
94.01	0.000869090499221242\\
95.01	0.00069399932017805\\
96.01	0.000522799385888874\\
97.01	0.000357262206438121\\
98.01	0.000199424406982189\\
99.01	5.15714273097625e-05\\
99.02	5.01513467955263e-05\\
99.03	4.87324870754452e-05\\
99.04	4.73148484967465e-05\\
99.05	4.58997202600082e-05\\
99.06	4.44921994131667e-05\\
99.07	4.30923253333e-05\\
99.08	4.17001378667011e-05\\
99.09	4.03156773406757e-05\\
99.1	3.89410853922963e-05\\
99.11	3.75771478830951e-05\\
99.12	3.62239262758513e-05\\
99.13	3.48814827721682e-05\\
99.14	3.35499100302342e-05\\
99.15	3.22296239078142e-05\\
99.16	3.09206882554737e-05\\
99.17	2.96231677817966e-05\\
99.18	2.83408197653007e-05\\
99.19	2.70784264258875e-05\\
99.2	2.58361154154719e-05\\
99.21	2.4614016062012e-05\\
99.22	2.34152131222843e-05\\
99.23	2.22500227116482e-05\\
99.24	2.11186718924554e-05\\
99.25	2.00213908415521e-05\\
99.26	1.89584129430577e-05\\
99.27	1.79299748844716e-05\\
99.28	1.69363167562345e-05\\
99.29	1.59776821549128e-05\\
99.3	1.50543182901221e-05\\
99.31	1.4166476095397e-05\\
99.32	1.33144103431227e-05\\
99.33	1.24983797637635e-05\\
99.34	1.17186471695323e-05\\
99.35	1.09755131177357e-05\\
99.36	1.02692828861141e-05\\
99.37	9.59960440220786e-06\\
99.38	8.96184452921049e-06\\
99.39	8.35626969543889e-06\\
99.4	7.78080695248116e-06\\
99.41	7.2356194598059e-06\\
99.42	6.72064744043632e-06\\
99.43	6.23615102932421e-06\\
99.44	5.78239536754878e-06\\
99.45	5.35523345221965e-06\\
99.46	4.9522228167849e-06\\
99.47	4.57357835185676e-06\\
99.48	4.21454314757791e-06\\
99.49	3.86924434524726e-06\\
99.5	3.53780285034661e-06\\
99.51	3.22034032447863e-06\\
99.52	2.91697917155324e-06\\
99.53	2.62784252295402e-06\\
99.54	2.35305422169839e-06\\
99.55	2.09273880545299e-06\\
99.56	1.84702148846305e-06\\
99.57	1.61602814221368e-06\\
99.58	1.39988527489768e-06\\
99.59	1.19872000950774e-06\\
99.6	1.01266006058941e-06\\
99.61	8.41833709516043e-07\\
99.62	6.86369778247695e-07\\
99.63	5.46397601500909e-07\\
99.64	4.22046997228948e-07\\
99.65	3.13448235346458e-07\\
99.66	2.20732004603194e-07\\
99.67	1.44029377518312e-07\\
99.68	8.3471773252089e-08\\
99.69	3.91920583538846e-08\\
99.7	1.13235991017896e-08\\
99.71	0\\
99.72	0\\
99.73	0\\
99.74	0\\
99.75	0\\
99.76	0\\
99.77	0\\
99.78	0\\
99.79	0\\
99.8	0\\
99.81	0\\
99.82	0\\
99.83	0\\
99.84	0\\
99.85	0\\
99.86	0\\
99.87	0\\
99.88	0\\
99.89	0\\
99.9	0\\
99.91	0\\
99.92	0\\
99.93	0\\
99.94	0\\
99.95	0\\
99.96	0\\
99.97	0\\
99.98	0\\
99.99	0\\
100	0\\
};
\addlegendentry{$q=-2$};

\addplot [color=blue,dashed]
  table[row sep=crcr]{%
0.01	0.01\\
1.01	0.01\\
2.01	0.01\\
3.01	0.01\\
4.01	0.01\\
5.01	0.01\\
6.01	0.01\\
7.01	0.01\\
8.01	0.01\\
9.01	0.01\\
10.01	0.01\\
11.01	0.01\\
12.01	0.01\\
13.01	0.01\\
14.01	0.01\\
15.01	0.01\\
16.01	0.01\\
17.01	0.01\\
18.01	0.01\\
19.01	0.01\\
20.01	0.01\\
21.01	0.01\\
22.01	0.01\\
23.01	0.01\\
24.01	0.01\\
25.01	0.01\\
26.01	0.01\\
27.01	0.01\\
28.01	0.01\\
29.01	0.01\\
30.01	0.01\\
31.01	0.01\\
32.01	0.01\\
33.01	0.01\\
34.01	0.01\\
35.01	0.01\\
36.01	0.01\\
37.01	0.01\\
38.01	0.01\\
39.01	0.01\\
40.01	0.01\\
41.01	0.01\\
42.01	0.01\\
43.01	0.01\\
44.01	0.01\\
45.01	0.01\\
46.01	0.01\\
47.01	0.01\\
48.01	0.01\\
49.01	0.01\\
50.01	0.01\\
51.01	0.01\\
52.01	0.01\\
53.01	0.01\\
54.01	0.01\\
55.01	0.01\\
56.01	0.01\\
57.01	0.01\\
58.01	0.01\\
59.01	0.01\\
60.01	0.01\\
61.01	0.01\\
62.01	0.01\\
63.01	0.01\\
64.01	0.01\\
65.01	0.01\\
66.01	0.01\\
67.01	0.01\\
68.01	0.01\\
69.01	0.01\\
70.01	0.01\\
71.01	0.01\\
72.01	0.00993962270142486\\
73.01	0.0097947536949604\\
74.01	0.00963774812468075\\
75.01	0.00946688035474884\\
76.01	0.00928006638968593\\
77.01	0.00907478347136383\\
78.01	0.00884798695145539\\
79.01	0.00859600179908861\\
80.01	0.00832161547787609\\
81.01	0.00803388501022849\\
82.01	0.00773254790799095\\
83.01	0.00741738634259809\\
84.01	0.00708844456302999\\
85.01	0.00674616946310022\\
86.01	0.00639159211134382\\
87.01	0.00602461642880567\\
88.01	0.00564468650640655\\
89.01	0.00525122367465925\\
90.01	0.00484365353674161\\
91.01	0.00442144174622584\\
92.01	0.00398409351132958\\
93.01	0.00353116349211519\\
94.01	0.00306230474189923\\
95.01	0.00257731619586271\\
96.01	0.00207620154151122\\
97.01	0.00155927909521823\\
98.01	0.00102748243795499\\
99.01	0.000484002402631955\\
99.02	0.000478539426080221\\
99.03	0.000473076568376028\\
99.04	0.000467613860715534\\
99.05	0.000462151334831507\\
99.06	0.000456689022643036\\
99.07	0.000451226956759679\\
99.08	0.0004457651704979\\
99.09	0.000440303697897944\\
99.1	0.000434842573538527\\
99.11	0.000429381832685427\\
99.12	0.000423921511379667\\
99.13	0.000418461646456252\\
99.14	0.00041300227556062\\
99.15	0.000407543437137919\\
99.16	0.000402085170487329\\
99.17	0.00039662751578293\\
99.18	0.000391170513739127\\
99.19	0.00038571420553437\\
99.2	0.000380258633282306\\
99.21	0.000374803840055017\\
99.22	0.000369349869622499\\
99.23	0.000363896765781177\\
99.24	0.000358444573350736\\
99.25	0.000352993338199973\\
99.26	0.000347543107273404\\
99.27	0.000342093928618651\\
99.28	0.000336645851414607\\
99.29	0.000331198926000449\\
99.3	0.000325753203905484\\
99.31	0.000320308737879896\\
99.32	0.000314865581926406\\
99.33	0.000309423791332876\\
99.34	0.000303983422705911\\
99.35	0.000298544533995989\\
99.36	0.000293107184533013\\
99.37	0.000287672093871667\\
99.38	0.000282244243706301\\
99.39	0.000276823722011698\\
99.4	0.000271412954483807\\
99.41	0.000266012136295681\\
99.42	0.000260621683634504\\
99.43	0.000255241698840376\\
99.44	0.000249872285511276\\
99.45	0.000244517949873808\\
99.46	0.000239181489765374\\
99.47	0.000233863051317179\\
99.48	0.000228567739209068\\
99.49	0.000223301755739001\\
99.5	0.000218065313555817\\
99.51	0.000212858628749365\\
99.52	0.000207681920948452\\
99.53	0.000202535413421905\\
99.54	0.000197419333182835\\
99.55	0.000192333911096272\\
99.56	0.000187279381990244\\
99.57	0.000182255984770466\\
99.58	0.000177263962538754\\
99.59	0.00017230356271533\\
99.6	0.000167375037165131\\
99.61	0.000162478642328327\\
99.62	0.000157614639355172\\
99.63	0.000152783294245387\\
99.64	0.000147984877992261\\
99.65	0.000143219666731637\\
99.66	0.000138487941896021\\
99.67	0.000133789990373995\\
99.68	0.000129126104675206\\
99.69	0.000124496583100996\\
99.7	0.000119901729921192\\
99.71	0.000115341855557199\\
99.72	0.000110817277287002\\
99.73	0.000106328320030035\\
99.74	0.000101875315463656\\
99.75	9.74586022346059e-05\\
99.76	9.30785261783317e-05\\
99.77	8.87354405466265e-05\\
99.78	8.44297062438867e-05\\
99.79	8.01616920725161e-05\\
99.8	7.59317749878478e-05\\
99.81	7.17403403631186e-05\\
99.82	6.75877822649829e-05\\
99.83	6.34745037401234e-05\\
99.84	5.94009171135564e-05\\
99.85	5.53674442992213e-05\\
99.86	5.13745171235643e-05\\
99.87	4.74225776628031e-05\\
99.88	4.35120785946402e-05\\
99.89	3.96434835652562e-05\\
99.9	3.58172675724491e-05\\
99.91	3.20339186857307e-05\\
99.92	2.82939389100248e-05\\
99.93	2.45978428235222e-05\\
99.94	2.09461580345793e-05\\
99.95	1.73394256602066e-05\\
99.96	1.37782008275116e-05\\
99.97	1.02630531995645e-05\\
99.98	6.79456752728652e-06\\
99.99	3.37334422906808e-06\\
100	0\\
};
\addlegendentry{$q=-1$};

\addplot [color=black,solid]
  table[row sep=crcr]{%
0.01	0\\
1.01	0\\
2.01	0\\
3.01	0\\
4.01	0\\
5.01	0\\
6.01	0\\
7.01	0\\
8.01	0\\
9.01	0\\
10.01	0\\
11.01	0\\
12.01	0\\
13.01	0\\
14.01	0\\
15.01	0\\
16.01	0\\
17.01	0\\
18.01	0\\
19.01	0\\
20.01	0\\
21.01	0\\
22.01	0\\
23.01	0\\
24.01	0\\
25.01	0\\
26.01	0\\
27.01	0\\
28.01	0\\
29.01	0\\
30.01	0\\
31.01	0\\
32.01	0\\
33.01	0\\
34.01	0\\
35.01	0\\
36.01	0\\
37.01	0\\
38.01	0\\
39.01	0\\
40.01	0\\
41.01	0\\
42.01	0\\
43.01	0\\
44.01	0\\
45.01	0\\
46.01	0\\
47.01	0\\
48.01	0\\
49.01	0\\
50.01	0\\
51.01	0\\
52.01	0\\
53.01	0\\
54.01	0\\
55.01	0\\
56.01	0\\
57.01	0\\
58.01	0\\
59.01	0\\
60.01	0\\
61.01	0\\
62.01	0\\
63.01	0\\
64.01	0\\
65.01	0\\
66.01	0\\
67.01	0\\
68.01	0\\
69.01	0\\
70.01	0\\
71.01	0\\
72.01	0\\
73.01	0\\
74.01	0\\
75.01	0\\
76.01	0\\
77.01	0\\
78.01	9.04769124136298e-05\\
79.01	0.000333505182494108\\
80.01	0.000594652620758983\\
81.01	0.000876535109286314\\
82.01	0.00118230198421931\\
83.01	0.00151576720682256\\
84.01	0.00187945949896115\\
85.01	0.00226196939622701\\
86.01	0.00266118797422353\\
87.01	0.00307770745573711\\
88.01	0.00351192518159225\\
89.01	0.00396378885283169\\
90.01	0.00443228837130114\\
91.01	0.00491777491496314\\
92.01	0.00542085737592643\\
93.01	0.00594205496899957\\
94.01	0.00648174434477769\\
95.01	0.00704007988672598\\
96.01	0.00761688659932939\\
97.01	0.00821145768662725\\
98.01	0.00882212435172335\\
99.01	0.00944376771779301\\
99.02	0.00944999042923145\\
99.03	0.0094562121593535\\
99.04	0.00946243285656591\\
99.05	0.00946865246809523\\
99.06	0.0094748709399575\\
99.07	0.00948108821692697\\
99.08	0.00948730424250401\\
99.09	0.009493518958882\\
99.1	0.00949973230715658\\
99.11	0.00950594422709603\\
99.12	0.00951215465705893\\
99.13	0.00951836353395632\\
99.14	0.0095245707932135\\
99.15	0.00953077636873065\\
99.16	0.00953698019292056\\
99.17	0.00954318219709688\\
99.18	0.00954938231134147\\
99.19	0.0095555804640455\\
99.2	0.00956177658186387\\
99.21	0.00956797058966803\\
99.22	0.00957416241049743\\
99.23	0.00958035196550946\\
99.24	0.00958653917392775\\
99.25	0.00959272395298881\\
99.26	0.00959890621788724\\
99.27	0.00960508588171903\\
99.28	0.00961126285542295\\
99.29	0.00961743704771994\\
99.3	0.00962360836505055\\
99.31	0.00962977270185325\\
99.32	0.0096359270057073\\
99.33	0.00964207113242455\\
99.34	0.00964820493489091\\
99.35	0.00965432826298612\\
99.36	0.00966044096350087\\
99.37	0.00966654288005117\\
99.38	0.00967263241561898\\
99.39	0.00967870861727818\\
99.4	0.0096847713046778\\
99.41	0.0096908202937782\\
99.42	0.0096968554007452\\
99.43	0.00970287370872296\\
99.44	0.00970887190131976\\
99.45	0.00971484789971759\\
99.46	0.0097208014747299\\
99.47	0.00972673239257158\\
99.48	0.00973264041472378\\
99.49	0.00973852529779398\\
99.5	0.00974438679337096\\
99.51	0.0097502246478746\\
99.52	0.00975603860240025\\
99.53	0.00976182839255727\\
99.54	0.00976759374830175\\
99.55	0.00977333439376287\\
99.56	0.00977905004706274\\
99.57	0.00978474042012944\\
99.58	0.00979040521850278\\
99.59	0.00979604414113263\\
99.6	0.00980165688016925\\
99.61	0.00980724312074539\\
99.62	0.00981280254074964\\
99.63	0.00981833481059062\\
99.64	0.00982383959295151\\
99.65	0.00982931654253445\\
99.66	0.00983476530579426\\
99.67	0.00984018552066089\\
99.68	0.00984557681624994\\
99.69	0.00985093881256365\\
99.7	0.00985627112017661\\
99.71	0.00986157333990662\\
99.72	0.00986684506247061\\
99.73	0.009872085868125\\
99.74	0.00987729532628936\\
99.75	0.00988247299515249\\
99.76	0.00988761842125981\\
99.77	0.00989273113908086\\
99.78	0.00989781067055575\\
99.79	0.00990285652461904\\
99.8	0.00990786819669979\\
99.81	0.00991284516819603\\
99.82	0.00991778690592203\\
99.83	0.00992269286152642\\
99.84	0.00992756247087925\\
99.85	0.00993239515342571\\
99.86	0.00993719031150409\\
99.87	0.00994194732962551\\
99.88	0.00994666557371232\\
99.89	0.00995134439029238\\
99.9	0.00995598310564544\\
99.91	0.00996058102489816\\
99.92	0.00996513743106351\\
99.93	0.00996965158402005\\
99.94	0.00997412271942604\\
99.95	0.00997855004756298\\
99.96	0.00998293275210229\\
99.97	0.00998726998878853\\
99.98	0.00999156088403149\\
99.99	0.00999580453339885\\
100	0.01\\
};
\addlegendentry{$q=0$};

\addplot [color=blue,solid]
  table[row sep=crcr]{%
0.01	0\\
1.01	0\\
2.01	0\\
3.01	0\\
4.01	0\\
5.01	0\\
6.01	0\\
7.01	0\\
8.01	0\\
9.01	0\\
10.01	0\\
11.01	0\\
12.01	0\\
13.01	0\\
14.01	0\\
15.01	0\\
16.01	0\\
17.01	0\\
18.01	0\\
19.01	0\\
20.01	0\\
21.01	0\\
22.01	0\\
23.01	0\\
24.01	0\\
25.01	0\\
26.01	0\\
27.01	0\\
28.01	0\\
29.01	0\\
30.01	0\\
31.01	0\\
32.01	0\\
33.01	0\\
34.01	0\\
35.01	0\\
36.01	0\\
37.01	0\\
38.01	0\\
39.01	0\\
40.01	0\\
41.01	0\\
42.01	0\\
43.01	0\\
44.01	0\\
45.01	0\\
46.01	0\\
47.01	0\\
48.01	0\\
49.01	0\\
50.01	0\\
51.01	0\\
52.01	0\\
53.01	0\\
54.01	0\\
55.01	0\\
56.01	0\\
57.01	9.60420732154521e-06\\
58.01	0.000135229208699593\\
59.01	0.000267619698513074\\
60.01	0.000407377997348555\\
61.01	0.000555189437212708\\
62.01	0.000711837486274149\\
63.01	0.000878222137925787\\
64.01	0.00105538212964338\\
65.01	0.00124452076074874\\
66.01	0.00144703101524147\\
67.01	0.00166450581611247\\
68.01	0.00189879061589581\\
69.01	0.00215216061275922\\
70.01	0.00242734934410092\\
71.01	0.0027276214739467\\
72.01	0.0030553224765646\\
73.01	0.00340170047820055\\
74.01	0.00376536584380919\\
75.01	0.00414751595215838\\
76.01	0.00454941555165311\\
77.01	0.00497258007181894\\
78.01	0.00532787110404776\\
79.01	0.00555176416685463\\
80.01	0.00577864355419245\\
81.01	0.006006186134476\\
82.01	0.00623086932827717\\
83.01	0.00644742715515898\\
84.01	0.00665294230431784\\
85.01	0.00685876022147589\\
86.01	0.00706732051874468\\
87.01	0.00727830213096558\\
88.01	0.00749147483156968\\
89.01	0.00770693951350276\\
90.01	0.00792572199941495\\
91.01	0.00814741745533618\\
92.01	0.00837114551457187\\
93.01	0.00859584024134704\\
94.01	0.00882030457488775\\
95.01	0.00904314111888931\\
96.01	0.00926292729381968\\
97.01	0.00947785139077141\\
98.01	0.00968577898191207\\
99.01	0.00988448798209627\\
99.02	0.00988638323582952\\
99.03	0.00988826925410972\\
99.04	0.00989014597140742\\
99.05	0.00989201332144849\\
99.06	0.00989387123719858\\
99.07	0.00989571965084699\\
99.08	0.00989755849379011\\
99.09	0.00989938769661441\\
99.1	0.00990120495110082\\
99.11	0.00990300973914664\\
99.12	0.00990480196679558\\
99.13	0.00990658154135368\\
99.14	0.00990834836890682\\
99.15	0.00991010235429178\\
99.16	0.00991184267987425\\
99.17	0.00991356456016402\\
99.18	0.00991526401895467\\
99.19	0.00991694089850678\\
99.2	0.00991859503883935\\
99.21	0.00992022627766874\\
99.22	0.0099218344503455\\
99.23	0.00992341938978893\\
99.24	0.00992498092641946\\
99.25	0.00992651888808863\\
99.26	0.00992803309951846\\
99.27	0.00992952338219693\\
99.28	0.00993098955483227\\
99.29	0.00993243143327257\\
99.3	0.00993384883042234\\
99.31	0.00993524557604533\\
99.32	0.00993662444919607\\
99.33	0.00993798531576456\\
99.34	0.00993932803949575\\
99.35	0.0099406524819256\\
99.36	0.00994195850231464\\
99.37	0.00994324595757886\\
99.38	0.00994451614662083\\
99.39	0.00994576972259023\\
99.4	0.00994700656056504\\
99.41	0.00994822653365905\\
99.42	0.00994942951093435\\
99.43	0.00995061809883822\\
99.44	0.00995179531099282\\
99.45	0.00995296292686011\\
99.46	0.00995412087134393\\
99.47	0.00995526906889685\\
99.48	0.00995640744352824\\
99.49	0.00995753591881294\\
99.5	0.00995865441790052\\
99.51	0.0099597628635252\\
99.52	0.00996086117801654\\
99.53	0.00996194928331074\\
99.54	0.00996302710096279\\
99.55	0.00996409455215938\\
99.56	0.00996515155773268\\
99.57	0.00996619803817498\\
99.58	0.00996723391365427\\
99.59	0.00996825910403079\\
99.6	0.00996927352887465\\
99.61	0.00997027710748449\\
99.62	0.00997126975890726\\
99.63	0.00997225140195929\\
99.64	0.00997322195524853\\
99.65	0.0099741813371982\\
99.66	0.00997512946607508\\
99.67	0.00997606626001613\\
99.68	0.0099769916370565\\
99.69	0.00997790550391879\\
99.7	0.00997880776335109\\
99.71	0.00997969831793289\\
99.72	0.00998057707010968\\
99.73	0.00998144392222969\\
99.74	0.00998229877658283\\
99.75	0.00998314153544192\\
99.76	0.00998397210110645\\
99.77	0.00998479037594896\\
99.78	0.0099855962624641\\
99.79	0.00998638966332074\\
99.8	0.0099871704814171\\
99.81	0.00998793861993922\\
99.82	0.00998869398242292\\
99.83	0.00998943647281947\\
99.84	0.00999016599556515\\
99.85	0.0099908824556551\\
99.86	0.00999158575872152\\
99.87	0.00999227581111662\\
99.88	0.00999295252000064\\
99.89	0.00999361579343513\\
99.9	0.00999426554048208\\
99.91	0.00999490167130895\\
99.92	0.00999552409730028\\
99.93	0.00999613273117618\\
99.94	0.0099967274871182\\
99.95	0.00999730828090301\\
99.96	0.00999787503004458\\
99.97	0.00999842765394528\\
99.98	0.00999896607405659\\
99.99	0.0099994902140502\\
100	0.01\\
};
\addlegendentry{$q=1$};

\addplot [color=red,solid]
  table[row sep=crcr]{%
0.01	0\\
1.01	4.33087425613307e-06\\
2.01	2.05020285200704e-05\\
3.01	3.73173082565334e-05\\
4.01	5.48052763838687e-05\\
5.01	7.29960398933305e-05\\
6.01	9.19213584828459e-05\\
7.01	0.000111614762468255\\
8.01	0.000132111680702728\\
9.01	0.000153449579223666\\
10.01	0.000175668111308157\\
11.01	0.000198809279582709\\
12.01	0.000222917611182281\\
13.01	0.000248040351063565\\
14.01	0.000274227716331572\\
15.01	0.000301533540658074\\
16.01	0.000330016222390317\\
17.01	0.000359735894048408\\
18.01	0.000390755143170381\\
19.01	0.000423139955802711\\
20.01	0.000456959714356601\\
21.01	0.000492287091090202\\
22.01	0.000529197792254164\\
23.01	0.000567770088567635\\
24.01	0.00060808404356465\\
25.01	0.000650220318232357\\
26.01	0.000694258383589514\\
27.01	0.000740273899091044\\
28.01	0.00078833485226942\\
29.01	0.00083849544581216\\
30.01	0.000890785390867597\\
31.01	0.000945225797411579\\
32.01	0.00100188288591958\\
33.01	0.00106087247601426\\
34.01	0.00112232279011504\\
35.01	0.00118637312521186\\
36.01	0.00125317528135806\\
37.01	0.00132289522822918\\
38.01	0.001395715057432\\
39.01	0.00147183528004889\\
40.01	0.00155147754711322\\
41.01	0.00163488791603481\\
42.01	0.00172234103603849\\
43.01	0.00181414818239593\\
44.01	0.00191068769183668\\
45.01	0.00201238274759013\\
46.01	0.00211968589415343\\
47.01	0.00223311410368929\\
48.01	0.0023532618653983\\
49.01	0.00248081704155053\\
50.01	0.00261658124103317\\
51.01	0.00276149646863232\\
52.01	0.00291667055429176\\
53.01	0.00308341547440325\\
54.01	0.00326330360926503\\
55.01	0.00345827769959624\\
56.01	0.0036707692512595\\
57.01	0.00389108189510773\\
58.01	0.00400673064997862\\
59.01	0.00412657090339202\\
60.01	0.00425060839734533\\
61.01	0.00437880024151678\\
62.01	0.00451103965405149\\
63.01	0.00464713645209664\\
64.01	0.0047867920876785\\
65.01	0.004929567695006\\
66.01	0.00507484344691566\\
67.01	0.00522176948722644\\
68.01	0.00536920427941955\\
69.01	0.00551559391410557\\
70.01	0.00565883078017027\\
71.01	0.00579607275242026\\
72.01	0.00592505417474953\\
73.01	0.00605463702034386\\
74.01	0.00618674940302178\\
75.01	0.00631968360057137\\
76.01	0.00645164677137297\\
77.01	0.00658161168452686\\
78.01	0.00670883506929929\\
79.01	0.00683535875044596\\
80.01	0.00696168976889543\\
81.01	0.0070879286995167\\
82.01	0.00721477791450747\\
83.01	0.00734403669133708\\
84.01	0.00747646503390251\\
85.01	0.00761253205366901\\
86.01	0.00775234821559366\\
87.01	0.00789601354808364\\
88.01	0.00804363171240482\\
89.01	0.00819528512828527\\
90.01	0.00835096456057964\\
91.01	0.00851056933911956\\
92.01	0.00867396499933508\\
93.01	0.00884098409914399\\
94.01	0.00901142201835746\\
95.01	0.00918503015845018\\
96.01	0.00936150594239421\\
97.01	0.00954047723941063\\
98.01	0.00972148782114007\\
99.01	0.00990006106960213\\
99.02	0.00990157576503505\\
99.03	0.00990307369943541\\
99.04	0.00990455474622719\\
99.05	0.00990601877606795\\
99.06	0.00990746565753837\\
99.07	0.0099088952572433\\
99.08	0.00991030743975208\\
99.09	0.00991170206753619\\
99.1	0.0099130812465334\\
99.11	0.00991444529138604\\
99.12	0.00991579409007456\\
99.13	0.00991712752702963\\
99.14	0.00991844548509619\\
99.15	0.00991974784548744\\
99.16	0.00992103521124347\\
99.17	0.00992231216481521\\
99.18	0.00992358248643046\\
99.19	0.00992484613615709\\
99.2	0.00992610307459429\\
99.21	0.00992735326291633\\
99.22	0.00992859666291846\\
99.23	0.00992983323706484\\
99.24	0.00993106294853881\\
99.25	0.00993228576129553\\
99.26	0.00993350164011714\\
99.27	0.00993471055067064\\
99.28	0.00993591245956828\\
99.29	0.00993710733443088\\
99.3	0.0099382951439541\\
99.31	0.00993947584960176\\
99.32	0.00994064940747986\\
99.33	0.00994181577428223\\
99.34	0.00994297490734193\\
99.35	0.00994412676468527\\
99.36	0.00994527130508878\\
99.37	0.00994640848813914\\
99.38	0.00994753826862462\\
99.39	0.00994866059884041\\
99.4	0.00994977543158854\\
99.41	0.00995088272022702\\
99.42	0.00995198241872811\\
99.43	0.00995307447422267\\
99.44	0.00995415882356421\\
99.45	0.00995523539630366\\
99.46	0.00995630412136029\\
99.47	0.00995736492701623\\
99.48	0.00995841774091096\\
99.49	0.00995946249003573\\
99.5	0.00996049910072788\\
99.51	0.00996152749866509\\
99.52	0.00996254760885948\\
99.53	0.00996355935565166\\
99.54	0.00996456266270464\\
99.55	0.00996555745299768\\
99.56	0.00996654364881991\\
99.57	0.00996752117176395\\
99.58	0.00996848994271928\\
99.59	0.00996944988186555\\
99.6	0.00997040090866567\\
99.61	0.00997134294185881\\
99.62	0.00997227589945309\\
99.63	0.00997319969871828\\
99.64	0.00997411425617811\\
99.65	0.00997501948760265\\
99.66	0.00997591529512609\\
99.67	0.00997680157986543\\
99.68	0.00997767824200409\\
99.69	0.00997854518078455\\
99.7	0.00997940229449872\\
99.71	0.00998024948047705\\
99.72	0.0099810866350772\\
99.73	0.00998191365367248\\
99.74	0.00998273043063972\\
99.75	0.00998353685934682\\
99.76	0.00998433283213985\\
99.77	0.00998511824032956\\
99.78	0.00998589297417746\\
99.79	0.00998665692288131\\
99.8	0.00998740997456001\\
99.81	0.00998815201623783\\
99.82	0.00998888293382796\\
99.83	0.00998960261211535\\
99.84	0.00999031093473873\\
99.85	0.0099910077841718\\
99.86	0.00999169304170346\\
99.87	0.00999236658741715\\
99.88	0.00999302830016909\\
99.89	0.0099936780575654\\
99.9	0.00999431573593803\\
99.91	0.00999494121031937\\
99.92	0.00999555435441553\\
99.93	0.00999615504057801\\
99.94	0.0099967431397739\\
99.95	0.00999731852155427\\
99.96	0.00999788105402068\\
99.97	0.0099984306037898\\
99.98	0.00999896703595574\\
99.99	0.0099994902140502\\
100	0.01\\
};
\addlegendentry{$q=2$};

\addplot [color=mycolor1,solid]
  table[row sep=crcr]{%
0.01	0.00284497808044895\\
1.01	0.00286358053885133\\
2.01	0.00287129926155062\\
3.01	0.00287932108641342\\
4.01	0.00288766064546655\\
5.01	0.00289633368763364\\
6.01	0.00290535722157912\\
7.01	0.00291474968346587\\
8.01	0.00292453113476467\\
9.01	0.00293472349645438\\
10.01	0.0029453508274643\\
11.01	0.00295643965718092\\
12.01	0.00296801938486431\\
13.01	0.00298012276870255\\
14.01	0.00299278663179456\\
15.01	0.00300605504441553\\
16.01	0.00302000842328301\\
17.01	0.00303473814541508\\
18.01	0.00305032251621418\\
19.01	0.00306685293628916\\
20.01	0.00308443711580778\\
21.01	0.00310320283411347\\
22.01	0.0031233026949752\\
23.01	0.00314492016665325\\
24.01	0.00316827728551013\\
25.01	0.00319364452034619\\
26.01	0.00322135345120413\\
27.01	0.00325181312046181\\
28.01	0.00328553114811884\\
29.01	0.00332314053250383\\
30.01	0.00336502271506735\\
31.01	0.00340897577803881\\
32.01	0.0034545794415943\\
33.01	0.00350189297771827\\
34.01	0.00355097426539939\\
35.01	0.00360188083358857\\
36.01	0.0036546691968992\\
37.01	0.00370939401008315\\
38.01	0.00376610699494643\\
39.01	0.00382485558054767\\
40.01	0.00388568118090434\\
41.01	0.00394861701374206\\
42.01	0.0040136853418117\\
43.01	0.00408089396233934\\
44.01	0.00415023111434892\\
45.01	0.0042216585007195\\
46.01	0.00429510471647451\\
47.01	0.00437045591469198\\
48.01	0.00444754328204564\\
49.01	0.00452612729180118\\
50.01	0.00460588139141532\\
51.01	0.00468642205166135\\
52.01	0.00476725028192061\\
53.01	0.00484755663600465\\
54.01	0.00492625267494352\\
55.01	0.00500189379822647\\
56.01	0.00507256430410409\\
57.01	0.00513858890404529\\
58.01	0.00520639375284322\\
59.01	0.00527701442725958\\
60.01	0.00535052033884895\\
61.01	0.00542695153494704\\
62.01	0.00550630393992901\\
63.01	0.00558850915363026\\
64.01	0.00567340666506521\\
65.01	0.00576070572521015\\
66.01	0.00584993310939179\\
67.01	0.00594013192365608\\
68.01	0.00603005572768021\\
69.01	0.00611941606762184\\
70.01	0.00620806487750143\\
71.01	0.00629602283762802\\
72.01	0.00638358270879835\\
73.01	0.00647110577514324\\
74.01	0.00655889653940176\\
75.01	0.00664805220961663\\
76.01	0.00673960173277042\\
77.01	0.00683371260408346\\
78.01	0.00693060269721136\\
79.01	0.00703052089712495\\
80.01	0.00713372573133154\\
81.01	0.00724050405463523\\
82.01	0.00735114671509235\\
83.01	0.00746587608382042\\
84.01	0.00758485207437221\\
85.01	0.00770821135400273\\
86.01	0.00783607605931906\\
87.01	0.0079685553154049\\
88.01	0.00810573927843018\\
89.01	0.00824769268040544\\
90.01	0.00839444993526902\\
91.01	0.00854601215628517\\
92.01	0.00870233983786851\\
93.01	0.00886334196067576\\
94.01	0.00902886233730587\\
95.01	0.00919866259882805\\
96.01	0.00937240105551164\\
97.01	0.00954960660956304\\
98.01	0.00972964659424814\\
99.01	0.00990155492092998\\
99.02	0.00990290647092698\\
99.03	0.00990425247739058\\
99.04	0.00990559291149017\\
99.05	0.00990692774516513\\
99.06	0.00990825695118009\\
99.07	0.00990958050318291\\
99.08	0.00991089837576564\\
99.09	0.00991221054452871\\
99.1	0.00991351697823863\\
99.11	0.00991481764483132\\
99.12	0.00991611251291097\\
99.13	0.00991740155180701\\
99.14	0.00991868473162849\\
99.15	0.00991996202332142\\
99.16	0.00992123339633484\\
99.17	0.00992249880599432\\
99.18	0.00992375819814401\\
99.19	0.00992501151811266\\
99.2	0.00992625871070577\\
99.21	0.00992749972019735\\
99.22	0.00992873449032155\\
99.23	0.00992996296426372\\
99.24	0.00993118508465128\\
99.25	0.00993240079354419\\
99.26	0.00993361003242496\\
99.27	0.00993481274218827\\
99.28	0.00993600886313022\\
99.29	0.00993719833493701\\
99.3	0.00993838109667319\\
99.31	0.00993955708678515\\
99.32	0.00994072624310134\\
99.33	0.00994188850282124\\
99.34	0.00994304380250382\\
99.35	0.00994419207805562\\
99.36	0.00994533326471818\\
99.37	0.00994646729705494\\
99.38	0.00994759410895921\\
99.39	0.00994871363365381\\
99.4	0.00994982580367944\\
99.41	0.00995093055088268\\
99.42	0.00995202780640331\\
99.43	0.00995311750068041\\
99.44	0.0099541995634753\\
99.45	0.00995527392388831\\
99.46	0.00995634051035226\\
99.47	0.00995739925062604\\
99.48	0.00995845007178791\\
99.49	0.00995949290022894\\
99.5	0.00996052766164622\\
99.51	0.00996155428103607\\
99.52	0.00996257268268718\\
99.53	0.00996358279017364\\
99.54	0.00996458452634792\\
99.55	0.00996557781333376\\
99.56	0.00996656257251899\\
99.57	0.00996753872454828\\
99.58	0.00996850618931578\\
99.59	0.00996946488595774\\
99.6	0.00997041473284496\\
99.61	0.00997135564757522\\
99.62	0.00997228754696565\\
99.63	0.00997321034704495\\
99.64	0.00997412396304554\\
99.65	0.00997502830878446\\
99.66	0.00997592328496542\\
99.67	0.00997680879132069\\
99.68	0.00997768472660141\\
99.69	0.00997855098856782\\
99.7	0.00997940747397935\\
99.71	0.00998025407858468\\
99.72	0.00998109069711163\\
99.73	0.00998191722325695\\
99.74	0.00998273354967606\\
99.75	0.00998353956797263\\
99.76	0.00998433516868813\\
99.77	0.00998512024129114\\
99.78	0.00998589467416675\\
99.79	0.00998665835460569\\
99.8	0.00998741116879346\\
99.81	0.00998815300179934\\
99.82	0.00998888373756528\\
99.83	0.00998960325889476\\
99.84	0.00999031144744146\\
99.85	0.00999100818369797\\
99.86	0.00999169334698432\\
99.87	0.00999236681543648\\
99.88	0.00999302846599478\\
99.89	0.00999367817439227\\
99.9	0.00999431581514299\\
99.91	0.00999494126153029\\
99.92	0.00999555438559496\\
99.93	0.00999615505812347\\
99.94	0.00999674314863612\\
99.95	0.00999731852537521\\
99.96	0.00999788105529318\\
99.97	0.00999843060404087\\
99.98	0.00999896703595574\\
99.99	0.0099994902140502\\
100	0.01\\
};
\addlegendentry{$q=3$};

\addplot [color=green,solid]
  table[row sep=crcr]{%
0.01	0.00370052694073906\\
1.01	0.00371489680974314\\
2.01	0.003729733266011\\
3.01	0.00374510263360971\\
4.01	0.00376102247641767\\
5.01	0.00377751055051478\\
6.01	0.00379458471472968\\
7.01	0.00381226281754934\\
8.01	0.00383056255504445\\
9.01	0.00384950129320284\\
10.01	0.00386909584647858\\
11.01	0.00388936220235964\\
12.01	0.00391031517917854\\
13.01	0.00393196800074026\\
14.01	0.00395433176257054\\
15.01	0.00397741469037295\\
16.01	0.00400122010200923\\
17.01	0.00402574272849096\\
18.01	0.00405096843538435\\
19.01	0.00407687252174995\\
20.01	0.00410341674042718\\
21.01	0.00413054548883419\\
22.01	0.00415818092575786\\
23.01	0.00418621669149512\\
24.01	0.004214509805859\\
25.01	0.00424287018143341\\
26.01	0.00427104700640877\\
27.01	0.00429871100749294\\
28.01	0.00432543128517476\\
29.01	0.00435064506194685\\
30.01	0.00437402415580537\\
31.01	0.00439784060195186\\
32.01	0.00442262889131399\\
33.01	0.00444842897187869\\
34.01	0.00447528152896721\\
35.01	0.00450322776987547\\
36.01	0.00453230915337759\\
37.01	0.00456256706961049\\
38.01	0.00459404247426034\\
39.01	0.00462677548880118\\
40.01	0.00466080499106544\\
41.01	0.00469616823991596\\
42.01	0.00473290060754\\
43.01	0.00477103553798772\\
44.01	0.00481060492381193\\
45.01	0.00485164020926095\\
46.01	0.00489417462481386\\
47.01	0.00493824795988176\\
48.01	0.00498391133913707\\
49.01	0.00503121817438382\\
50.01	0.00508022057616017\\
51.01	0.00513096797342706\\
52.01	0.00518350491064937\\
53.01	0.00523787114789524\\
54.01	0.00529410167668842\\
55.01	0.00535222627147672\\
56.01	0.00541226976103684\\
57.01	0.00547422262344684\\
58.01	0.00553760852552886\\
59.01	0.00560149855493645\\
60.01	0.00566565758699211\\
61.01	0.00572990498322443\\
62.01	0.00579406766460756\\
63.01	0.00585799988988189\\
64.01	0.00592161366790277\\
65.01	0.0059849244496217\\
66.01	0.00604811864370441\\
67.01	0.00611188378999621\\
68.01	0.00617726638717013\\
69.01	0.00624444233280379\\
70.01	0.00631357221222928\\
71.01	0.00638484897151334\\
72.01	0.0064584920748115\\
73.01	0.00653473707203045\\
74.01	0.0066138304420014\\
75.01	0.00669600074530322\\
76.01	0.00678141400499415\\
77.01	0.00687022464081496\\
78.01	0.0069625913449022\\
79.01	0.00705867454162927\\
80.01	0.00715863382217472\\
81.01	0.00726262479044979\\
82.01	0.00737079598694848\\
83.01	0.00748328811973331\\
84.01	0.00760023549044121\\
85.01	0.00772176484420287\\
86.01	0.0078479928303368\\
87.01	0.00797902251869752\\
88.01	0.00811493918439277\\
89.01	0.00825580530776454\\
90.01	0.00840165453957058\\
91.01	0.00855248419770224\\
92.01	0.00870824597294096\\
93.01	0.00886883461799277\\
94.01	0.0090340742261738\\
95.01	0.00920370159746236\\
96.01	0.00937734608214109\\
97.01	0.00955450515810773\\
98.01	0.00973451482958923\\
99.01	0.00990158038381976\\
99.02	0.00990292910222302\\
99.03	0.00990427259288974\\
99.04	0.0099056108080794\\
99.05	0.00990694369956448\\
99.06	0.00990827121862094\\
99.07	0.00990959331601817\\
99.08	0.00991090994200862\\
99.09	0.00991222104631686\\
99.1	0.00991352657815434\\
99.11	0.009914826486215\\
99.12	0.00991612071866583\\
99.13	0.00991740922313704\\
99.14	0.00991869194671183\\
99.15	0.00991996883591571\\
99.16	0.00992123983671282\\
99.17	0.00992250489453981\\
99.18	0.00992376395432008\\
99.19	0.00992501696045874\\
99.2	0.00992626385683757\\
99.21	0.00992750458680995\\
99.22	0.00992873909319568\\
99.23	0.00992996731827578\\
99.24	0.00993118920378736\\
99.25	0.00993240469091825\\
99.26	0.00993361372030175\\
99.27	0.00993481623201131\\
99.28	0.0099360121655551\\
99.29	0.00993720145987065\\
99.3	0.00993838405331939\\
99.31	0.00993955988368111\\
99.32	0.00994072888814847\\
99.33	0.00994189100332133\\
99.34	0.00994304616520118\\
99.35	0.00994419430918545\\
99.36	0.00994533537006185\\
99.37	0.00994646928200261\\
99.38	0.00994759597855872\\
99.39	0.00994871539265403\\
99.4	0.00994982745657927\\
99.41	0.00995093210198614\\
99.42	0.0099520292598813\\
99.43	0.0099531188606203\\
99.44	0.00995420083390134\\
99.45	0.00995527510875895\\
99.46	0.0099563416135575\\
99.47	0.00995740027598479\\
99.48	0.00995845102304542\\
99.49	0.00995949378105424\\
99.5	0.00996052847562965\\
99.51	0.00996155503168684\\
99.52	0.00996257337343097\\
99.53	0.00996358342435036\\
99.54	0.00996458510720945\\
99.55	0.00996557834404188\\
99.56	0.00996656305614335\\
99.57	0.0099675391640645\\
99.58	0.00996850658760366\\
99.59	0.00996946524579963\\
99.6	0.00997041505692424\\
99.61	0.00997135593847498\\
99.62	0.00997228780716746\\
99.63	0.00997321057892787\\
99.64	0.00997412416888529\\
99.65	0.00997502849073425\\
99.66	0.00997592344507596\\
99.67	0.00997680893153984\\
99.68	0.00997768484877415\\
99.69	0.00997855109443635\\
99.7	0.00997940756518358\\
99.71	0.00998025415666283\\
99.72	0.00998109076350121\\
99.73	0.00998191727929598\\
99.74	0.00998273359660456\\
99.75	0.00998353960693446\\
99.76	0.00998433520073301\\
99.77	0.00998512026737716\\
99.78	0.00998589469516297\\
99.79	0.00998665837129522\\
99.8	0.00998741118187676\\
99.81	0.00998815301189781\\
99.82	0.00998888374522519\\
99.83	0.0099896032645914\\
99.84	0.00999031145158363\\
99.85	0.00999100818663261\\
99.86	0.00999169334900144\\
99.87	0.00999236681677426\\
99.88	0.00999302846684477\\
99.89	0.00999367817490475\\
99.9	0.00999431581543237\\
99.91	0.00999494126168042\\
99.92	0.00999555438566448\\
99.93	0.00999615505815084\\
99.94	0.00999674314864449\\
99.95	0.0099973185253768\\
99.96	0.00999788105529324\\
99.97	0.00999843060404087\\
99.98	0.00999896703595574\\
99.99	0.0099994902140502\\
100	0.01\\
};
\addlegendentry{$q=4$};

\end{axis}
\end{tikzpicture}%

  \caption{Continuous Time}
\end{subfigure}%
\hfill%
\begin{subfigure}{.45\linewidth}
  \centering
  \setlength\figureheight{\linewidth} 
  \setlength\figurewidth{\linewidth}
  \tikzsetnextfilename{dp_dscr_z8}
  % This file was created by matlab2tikz.
%
%The latest updates can be retrieved from
%  http://www.mathworks.com/matlabcentral/fileexchange/22022-matlab2tikz-matlab2tikz
%where you can also make suggestions and rate matlab2tikz.
%
\definecolor{mycolor1}{rgb}{0.00000,1.00000,0.14286}%
\definecolor{mycolor2}{rgb}{0.00000,1.00000,0.28571}%
\definecolor{mycolor3}{rgb}{0.00000,1.00000,0.42857}%
\definecolor{mycolor4}{rgb}{0.00000,1.00000,0.57143}%
\definecolor{mycolor5}{rgb}{0.00000,1.00000,0.71429}%
\definecolor{mycolor6}{rgb}{0.00000,1.00000,0.85714}%
\definecolor{mycolor7}{rgb}{0.00000,1.00000,1.00000}%
\definecolor{mycolor8}{rgb}{0.00000,0.87500,1.00000}%
\definecolor{mycolor9}{rgb}{0.00000,0.62500,1.00000}%
\definecolor{mycolor10}{rgb}{0.12500,0.00000,1.00000}%
\definecolor{mycolor11}{rgb}{0.25000,0.00000,1.00000}%
\definecolor{mycolor12}{rgb}{0.37500,0.00000,1.00000}%
\definecolor{mycolor13}{rgb}{0.50000,0.00000,1.00000}%
\definecolor{mycolor14}{rgb}{0.62500,0.00000,1.00000}%
\definecolor{mycolor15}{rgb}{0.75000,0.00000,1.00000}%
\definecolor{mycolor16}{rgb}{0.87500,0.00000,1.00000}%
\definecolor{mycolor17}{rgb}{1.00000,0.00000,1.00000}%
\definecolor{mycolor18}{rgb}{1.00000,0.00000,0.87500}%
\definecolor{mycolor19}{rgb}{1.00000,0.00000,0.62500}%
\definecolor{mycolor20}{rgb}{0.85714,0.00000,0.00000}%
\definecolor{mycolor21}{rgb}{0.71429,0.00000,0.00000}%
%
\begin{tikzpicture}

\begin{axis}[%
width=4.1in,
height=3.803in,
at={(0.809in,0.513in)},
scale only axis,
point meta min=0,
point meta max=1,
every outer x axis line/.append style={black},
every x tick label/.append style={font=\color{black}},
xmin=0,
xmax=600,
every outer y axis line/.append style={black},
every y tick label/.append style={font=\color{black}},
ymin=0,
ymax=0.01,
axis background/.style={fill=white},
axis x line*=bottom,
axis y line*=left,
colormap={mymap}{[1pt] rgb(0pt)=(0,1,0); rgb(7pt)=(0,1,1); rgb(15pt)=(0,0,1); rgb(23pt)=(1,0,1); rgb(31pt)=(1,0,0); rgb(38pt)=(0,0,0)},
colorbar,
colorbar style={separate axis lines,every outer x axis line/.append style={black},every x tick label/.append style={font=\color{black}},every outer y axis line/.append style={black},every y tick label/.append style={font=\color{black}},yticklabels={{-19},{-17},{-15},{-13},{-11},{-9},{-7},{-5},{-3},{-1},{1},{3},{5},{7},{9},{11},{13},{15},{17},{19}}}
]
\addplot [color=green,solid,forget plot]
  table[row sep=crcr]{%
1	0.00547780019338487\\
2	0.00547779836203247\\
3	0.00547779649665558\\
4	0.00547779459661717\\
5	0.00547779266126814\\
6	0.00547779068994721\\
7	0.00547778868198065\\
8	0.00547778663668193\\
9	0.00547778455335156\\
10	0.00547778243127691\\
11	0.00547778026973184\\
12	0.00547777806797653\\
13	0.00547777582525721\\
14	0.0054777735408058\\
15	0.00547777121383979\\
16	0.00547776884356175\\
17	0.00547776642915928\\
18	0.00547776396980463\\
19	0.00547776146465437\\
20	0.00547775891284912\\
21	0.00547775631351337\\
22	0.00547775366575492\\
23	0.00547775096866472\\
24	0.00547774822131646\\
25	0.00547774542276647\\
26	0.00547774257205312\\
27	0.00547773966819658\\
28	0.00547773671019848\\
29	0.00547773369704161\\
30	0.00547773062768945\\
31	0.00547772750108587\\
32	0.00547772431615465\\
33	0.00547772107179931\\
34	0.00547771776690257\\
35	0.00547771440032607\\
36	0.00547771097090967\\
37	0.00547770747747142\\
38	0.00547770391880689\\
39	0.00547770029368877\\
40	0.00547769660086674\\
41	0.00547769283906642\\
42	0.00547768900698951\\
43	0.00547768510331311\\
44	0.005477681126689\\
45	0.00547767707574356\\
46	0.00547767294907713\\
47	0.00547766874526338\\
48	0.00547766446284898\\
49	0.005477660100353\\
50	0.00547765565626646\\
51	0.00547765112905163\\
52	0.00547764651714156\\
53	0.00547764181893974\\
54	0.00547763703281917\\
55	0.00547763215712194\\
56	0.00547762719015876\\
57	0.00547762213020816\\
58	0.00547761697551608\\
59	0.00547761172429486\\
60	0.00547760637472321\\
61	0.00547760092494489\\
62	0.00547759537306858\\
63	0.00547758971716693\\
64	0.00547758395527595\\
65	0.00547757808539432\\
66	0.00547757210548267\\
67	0.00547756601346271\\
68	0.0054775598072167\\
69	0.00547755348458673\\
70	0.00547754704337352\\
71	0.00547754048133619\\
72	0.00547753379619117\\
73	0.00547752698561129\\
74	0.00547752004722511\\
75	0.0054775129786161\\
76	0.00547750577732178\\
77	0.0054774984408325\\
78	0.00547749096659111\\
79	0.00547748335199146\\
80	0.00547747559437789\\
81	0.00547746769104432\\
82	0.00547745963923266\\
83	0.00547745143613261\\
84	0.00547744307888013\\
85	0.00547743456455674\\
86	0.00547742589018804\\
87	0.00547741705274312\\
88	0.00547740804913314\\
89	0.00547739887621048\\
90	0.00547738953076738\\
91	0.00547738000953478\\
92	0.00547737030918148\\
93	0.00547736042631254\\
94	0.00547735035746826\\
95	0.00547734009912296\\
96	0.00547732964768352\\
97	0.00547731899948825\\
98	0.00547730815080578\\
99	0.00547729709783312\\
100	0.00547728583669486\\
101	0.0054772743634416\\
102	0.00547726267404826\\
103	0.0054772507644131\\
104	0.00547723863035594\\
105	0.00547722626761666\\
106	0.00547721367185374\\
107	0.00547720083864259\\
108	0.00547718776347421\\
109	0.00547717444175333\\
110	0.00547716086879675\\
111	0.00547714703983176\\
112	0.00547713294999438\\
113	0.00547711859432751\\
114	0.00547710396777942\\
115	0.00547708906520158\\
116	0.00547707388134694\\
117	0.00547705841086813\\
118	0.00547704264831543\\
119	0.00547702658813468\\
120	0.00547701022466556\\
121	0.00547699355213936\\
122	0.00547697656467691\\
123	0.00547695925628634\\
124	0.00547694162086135\\
125	0.00547692365217836\\
126	0.00547690534389472\\
127	0.00547688668954638\\
128	0.00547686768254531\\
129	0.00547684831617732\\
130	0.00547682858359955\\
131	0.00547680847783804\\
132	0.00547678799178508\\
133	0.00547676711819685\\
134	0.00547674584969067\\
135	0.00547672417874215\\
136	0.00547670209768282\\
137	0.00547667959869716\\
138	0.00547665667381978\\
139	0.00547663331493242\\
140	0.00547660951376113\\
141	0.00547658526187342\\
142	0.00547656055067495\\
143	0.00547653537140648\\
144	0.00547650971514078\\
145	0.00547648357277933\\
146	0.0054764569350489\\
147	0.00547642979249845\\
148	0.00547640213549555\\
149	0.00547637395422284\\
150	0.00547634523867466\\
151	0.00547631597865313\\
152	0.00547628616376466\\
153	0.0054762557834163\\
154	0.00547622482681152\\
155	0.00547619328294673\\
156	0.00547616114060697\\
157	0.00547612838836183\\
158	0.0054760950145617\\
159	0.00547606100733317\\
160	0.00547602635457487\\
161	0.00547599104395312\\
162	0.00547595506289734\\
163	0.0054759183985957\\
164	0.00547588103799036\\
165	0.0054758429677727\\
166	0.00547580417437866\\
167	0.0054757646439838\\
168	0.00547572436249831\\
169	0.00547568331556171\\
170	0.00547564148853824\\
171	0.00547559886651095\\
172	0.00547555543427678\\
173	0.00547551117634084\\
174	0.00547546607691097\\
175	0.00547542011989205\\
176	0.00547537328888027\\
177	0.00547532556715714\\
178	0.00547527693768367\\
179	0.00547522738309414\\
180	0.00547517688568993\\
181	0.00547512542743321\\
182	0.00547507298994053\\
183	0.00547501955447631\\
184	0.00547496510194596\\
185	0.00547490961288938\\
186	0.00547485306747379\\
187	0.00547479544548693\\
188	0.00547473672632967\\
189	0.00547467688900864\\
190	0.00547461591212905\\
191	0.00547455377388681\\
192	0.0054744904520609\\
193	0.00547442592400535\\
194	0.00547436016664137\\
195	0.00547429315644886\\
196	0.00547422486945841\\
197	0.00547415528124248\\
198	0.0054740843669055\\
199	0.00547401210107523\\
200	0.00547393845789294\\
201	0.00547386341100487\\
202	0.00547378693355224\\
203	0.00547370899816192\\
204	0.00547362957693638\\
205	0.00547354864144361\\
206	0.00547346616270701\\
207	0.00547338211119487\\
208	0.0054732964568095\\
209	0.00547320916887667\\
210	0.00547312021613425\\
211	0.0054730295667209\\
212	0.00547293718816452\\
213	0.0054728430473704\\
214	0.00547274711060898\\
215	0.00547264934350355\\
216	0.00547254971101772\\
217	0.00547244817744224\\
218	0.00547234470638199\\
219	0.0054722392607422\\
220	0.00547213180271464\\
221	0.00547202229376333\\
222	0.00547191069460997\\
223	0.00547179696521914\\
224	0.0054716810647825\\
225	0.00547156295170338\\
226	0.0054714425835802\\
227	0.00547131991719004\\
228	0.00547119490847116\\
229	0.00547106751250553\\
230	0.00547093768350016\\
231	0.00547080537476862\\
232	0.00547067053871114\\
233	0.00547053312679474\\
234	0.00547039308953201\\
235	0.00547025037645972\\
236	0.00547010493611618\\
237	0.00546995671601781\\
238	0.00546980566263517\\
239	0.00546965172136738\\
240	0.00546949483651604\\
241	0.0054693349512578\\
242	0.00546917200761569\\
243	0.00546900594642939\\
244	0.00546883670732378\\
245	0.0054686642286767\\
246	0.00546848844758473\\
247	0.00546830929982718\\
248	0.0054681267198294\\
249	0.00546794064062302\\
250	0.00546775099380574\\
251	0.00546755770949818\\
252	0.00546736071629964\\
253	0.00546715994124117\\
254	0.00546695530973744\\
255	0.00546674674553633\\
256	0.00546653417066647\\
257	0.00546631750538349\\
258	0.00546609666811435\\
259	0.00546587157540045\\
260	0.00546564214183956\\
261	0.00546540828002758\\
262	0.00546516990049957\\
263	0.00546492691167163\\
264	0.00546467921978485\\
265	0.00546442672885071\\
266	0.00546416934060098\\
267	0.00546390695444281\\
268	0.00546363946742059\\
269	0.00546336677418728\\
270	0.0054630887669867\\
271	0.00546280533565021\\
272	0.00546251636760866\\
273	0.00546222174792165\\
274	0.00546192135932096\\
275	0.00546161508225876\\
276	0.0054613027949394\\
277	0.00546098437332235\\
278	0.00546065969126471\\
279	0.00546032862209576\\
280	0.00545999103929571\\
281	0.00545964681381874\\
282	0.00545929581404179\\
283	0.00545893790571285\\
284	0.00545857295189752\\
285	0.00545820081292533\\
286	0.005457821346334\\
287	0.00545743440681342\\
288	0.00545703984614783\\
289	0.00545663751315714\\
290	0.00545622725363689\\
291	0.00545580891029723\\
292	0.00545538232270017\\
293	0.005454947327196\\
294	0.005454503756858\\
295	0.00545405144141609\\
296	0.00545359020718901\\
297	0.0054531198770149\\
298	0.00545264027018071\\
299	0.0054521512023498\\
300	0.00545165248548837\\
301	0.00545114392779012\\
302	0.00545062533359921\\
303	0.00545009650333209\\
304	0.00544955723339684\\
305	0.00544900731611164\\
306	0.00544844653962099\\
307	0.00544787468781012\\
308	0.00544729154021798\\
309	0.00544669687194783\\
310	0.0054460904535762\\
311	0.00544547205105978\\
312	0.00544484142564023\\
313	0.0054441983337471\\
314	0.00544354252689835\\
315	0.00544287375159881\\
316	0.00544219174923633\\
317	0.00544149625597578\\
318	0.00544078700265033\\
319	0.00544006371465047\\
320	0.0054393261118108\\
321	0.00543857390829369\\
322	0.0054378068124707\\
323	0.00543702452680095\\
324	0.00543622674770699\\
325	0.00543541316544767\\
326	0.00543458346398793\\
327	0.00543373732086547\\
328	0.0054328744070547\\
329	0.00543199438682682\\
330	0.00543109691760711\\
331	0.0054301816498285\\
332	0.00542924822678164\\
333	0.00542829628446135\\
334	0.00542732545140942\\
335	0.00542633534855325\\
336	0.00542532558904098\\
337	0.00542429577807218\\
338	0.00542324551272441\\
339	0.00542217438177564\\
340	0.00542108196552196\\
341	0.00541996783559111\\
342	0.00541883155475096\\
343	0.00541767267671351\\
344	0.00541649074593372\\
345	0.00541528529740356\\
346	0.00541405585644107\\
347	0.00541280193847348\\
348	0.00541152304881607\\
349	0.00541021868244466\\
350	0.00540888832376318\\
351	0.00540753144636543\\
352	0.00540614751279144\\
353	0.00540473597427764\\
354	0.00540329627050202\\
355	0.00540182782932307\\
356	0.00540033006651343\\
357	0.00539880238548735\\
358	0.00539724417702318\\
359	0.00539565481897998\\
360	0.00539403367600851\\
361	0.00539238009925782\\
362	0.00539069342607582\\
363	0.00538897297970559\\
364	0.00538721806897695\\
365	0.00538542798799348\\
366	0.00538360201581567\\
367	0.00538173941614002\\
368	0.00537983943697468\\
369	0.00537790131031196\\
370	0.00537592425179702\\
371	0.00537390746039434\\
372	0.00537185011805059\\
373	0.00536975138935397\\
374	0.0053676104211907\\
375	0.00536542634239583\\
376	0.0053631982633991\\
377	0.00536092527586302\\
378	0.00535860645231078\\
379	0.0053562408457403\\
380	0.00535382748921942\\
381	0.00535136539545445\\
382	0.00534885355632346\\
383	0.00534629094236728\\
384	0.00534367650224867\\
385	0.00534100916223877\\
386	0.00533828782582163\\
387	0.00533551137291279\\
388	0.00533267865854607\\
389	0.00532978851220988\\
390	0.00532683973715953\\
391	0.00532383110970253\\
392	0.00532076137845693\\
393	0.00531762926357949\\
394	0.0053144334559635\\
395	0.00531117261640312\\
396	0.00530784537472329\\
397	0.00530445032887193\\
398	0.00530098604397334\\
399	0.00529745105133906\\
400	0.0052938438474343\\
401	0.00529016289279669\\
402	0.00528640661090413\\
403	0.00528257338698845\\
404	0.00527866156679076\\
405	0.00527466945525515\\
406	0.00527059531515573\\
407	0.00526643736565234\\
408	0.0052621937807702\\
409	0.00525786268779764\\
410	0.00525344216559556\\
411	0.00524893024281307\\
412	0.00524432489600113\\
413	0.00523962404761742\\
414	0.00523482556391416\\
415	0.0052299272526993\\
416	0.00522492686096276\\
417	0.00521982207235701\\
418	0.00521461050452102\\
419	0.00520928970623665\\
420	0.00520385715440529\\
421	0.00519831025083372\\
422	0.00519264631881968\\
423	0.00518686259952791\\
424	0.00518095624813875\\
425	0.00517492432971706\\
426	0.00516876381481588\\
427	0.00516247157484287\\
428	0.00515604437712275\\
429	0.00514947887963317\\
430	0.00514277162539182\\
431	0.00513591903646833\\
432	0.00512891740759394\\
433	0.00512176289934059\\
434	0.00511445153083856\\
435	0.005106979171999\\
436	0.00509934153520695\\
437	0.00509153416644783\\
438	0.00508355243582635\\
439	0.00507539152743719\\
440	0.0050670464285423\\
441	0.00505851191800767\\
442	0.00504978255395004\\
443	0.00504085266054141\\
444	0.00503171631391671\\
445	0.00502236732712819\\
446	0.00501279923408703\\
447	0.00500300527243186\\
448	0.00499297836526077\\
449	0.00498271110166135\\
450	0.00497219571597312\\
451	0.00496142406572608\\
452	0.00495038760818735\\
453	0.0049390773754565\\
454	0.00492748394805629\\
455	0.004915597426971\\
456	0.00490340740409602\\
457	0.00489090293107581\\
458	0.00487807248652442\\
459	0.0048649039416482\\
460	0.00485138452431932\\
461	0.00483750078168705\\
462	0.00482323854146321\\
463	0.00480858287207625\\
464	0.0047935180419613\\
465	0.00477802747833561\\
466	0.00476209372589299\\
467	0.00474569840590054\\
468	0.00472882217609919\\
469	0.00471144469131011\\
470	0.00469354456297209\\
471	0.00467509931101878\\
472	0.00465608528883763\\
473	0.0046364775315969\\
474	0.00461624941539552\\
475	0.00459537193149305\\
476	0.00457381089568\\
477	0.00455151205921459\\
478	0.00452843715885329\\
479	0.00450455081417934\\
480	0.00447981684303206\\
481	0.00445419803128292\\
482	0.00442765630872568\\
483	0.00440015300515289\\
484	0.00437164919286085\\
485	0.00434210613156405\\
486	0.00431148583748201\\
487	0.00427975180317468\\
488	0.00424686989968002\\
489	0.00421280949829499\\
490	0.00417754485596785\\
491	0.00414105681581911\\
492	0.00410333488287071\\
493	0.00406437974484788\\
494	0.00402420631952435\\
495	0.00398284742522583\\
496	0.00394035819421055\\
497	0.00389682138917866\\
498	0.00385235385275624\\
499	0.00380711438400887\\
500	0.00376131301138053\\
501	0.00371521852825999\\
502	0.00366915875495774\\
503	0.00362356040215765\\
504	0.00357912772682816\\
505	0.0035387837955858\\
506	0.00350347152895402\\
507	0.00347337562112852\\
508	0.0034477007156403\\
509	0.00342321915374529\\
510	0.0033995558606626\\
511	0.00337624383061699\\
512	0.00335267912856479\\
513	0.00332876510735656\\
514	0.00330442430717686\\
515	0.00327960483711814\\
516	0.00325427310653981\\
517	0.00322840785924007\\
518	0.00320198625777614\\
519	0.00317498308822855\\
520	0.00314738455198443\\
521	0.00311917684391942\\
522	0.00309034579133459\\
523	0.00306087673592382\\
524	0.00303075439888956\\
525	0.00299996271838372\\
526	0.00296848470340479\\
527	0.00293630226281886\\
528	0.00290339600729777\\
529	0.00286974502351109\\
530	0.00283532660238758\\
531	0.00280011576890602\\
532	0.00276406939748543\\
533	0.00272712121191253\\
534	0.0026911965166374\\
535	0.00265655106163712\\
536	0.00262214033891497\\
537	0.00258722058161971\\
538	0.00255170106648381\\
539	0.00251557462242589\\
540	0.00247883838157633\\
541	0.00244149232061441\\
542	0.00240353787403877\\
543	0.00236497792929109\\
544	0.00232581699903547\\
545	0.00228606137899959\\
546	0.00224571921278943\\
547	0.0022048002200509\\
548	0.00216331439627576\\
549	0.00212127465109859\\
550	0.0020787029667516\\
551	0.00203562649819421\\
552	0.00199207845654765\\
553	0.00194934483622062\\
554	0.0019076605974282\\
555	0.00186553102545023\\
556	0.00182295585535523\\
557	0.00177995162674704\\
558	0.00173653654153092\\
559	0.00169273037654859\\
560	0.00164855438448207\\
561	0.0016040311335613\\
562	0.00155918428001908\\
563	0.00151403828525429\\
564	0.0014686181370549\\
565	0.0014229492435685\\
566	0.00137720469839254\\
567	0.00133166599318628\\
568	0.0012858376589304\\
569	0.00123974504677254\\
570	0.00119341585869796\\
571	0.00114688025839211\\
572	0.00110017097212519\\
573	0.00105332337464662\\
574	0.00100637555373387\\
575	0.000959368345375567\\
576	0.00091234532951171\\
577	0.000865352773713771\\
578	0.000818439509057376\\
579	0.000771656718587056\\
580	0.000725057614040402\\
581	0.000678696970709246\\
582	0.000632630483283924\\
583	0.000586913897120322\\
584	0.000541601859672152\\
585	0.000496746426604043\\
586	0.000452395149029594\\
587	0.000408588670365815\\
588	0.000365357795666475\\
589	0.00032272012337526\\
590	0.000280676711614416\\
591	0.000239312817487405\\
592	0.000198788644961841\\
593	0.000159293651685586\\
594	0.000121079888727804\\
595	8.45520570083908e-05\\
596	5.05092148680373e-05\\
597	2.07908715710836e-05\\
598	0\\
599	0\\
600	0\\
};
\addplot [color=mycolor1,solid,forget plot]
  table[row sep=crcr]{%
1	0.00548095199615839\\
2	0.00548094992070687\\
3	0.00548094780715951\\
4	0.00548094565481181\\
5	0.00548094346294612\\
6	0.00548094123083137\\
7	0.00548093895772277\\
8	0.00548093664286176\\
9	0.00548093428547568\\
10	0.00548093188477729\\
11	0.00548092943996483\\
12	0.00548092695022149\\
13	0.00548092441471523\\
14	0.00548092183259856\\
15	0.00548091920300802\\
16	0.00548091652506419\\
17	0.00548091379787118\\
18	0.00548091102051634\\
19	0.00548090819207006\\
20	0.00548090531158525\\
21	0.00548090237809721\\
22	0.00548089939062314\\
23	0.005480896348162\\
24	0.00548089324969399\\
25	0.00548089009418024\\
26	0.00548088688056246\\
27	0.00548088360776255\\
28	0.00548088027468232\\
29	0.00548087688020299\\
30	0.00548087342318489\\
31	0.0054808699024671\\
32	0.00548086631686685\\
33	0.00548086266517931\\
34	0.00548085894617706\\
35	0.00548085515860977\\
36	0.0054808513012037\\
37	0.0054808473726612\\
38	0.00548084337166045\\
39	0.00548083929685487\\
40	0.00548083514687246\\
41	0.00548083092031579\\
42	0.00548082661576104\\
43	0.00548082223175787\\
44	0.00548081776682869\\
45	0.00548081321946827\\
46	0.00548080858814314\\
47	0.00548080387129106\\
48	0.00548079906732057\\
49	0.00548079417461039\\
50	0.00548078919150867\\
51	0.00548078411633274\\
52	0.00548077894736843\\
53	0.00548077368286925\\
54	0.00548076832105606\\
55	0.0054807628601163\\
56	0.00548075729820337\\
57	0.00548075163343604\\
58	0.00548074586389776\\
59	0.00548073998763612\\
60	0.00548073400266186\\
61	0.00548072790694861\\
62	0.00548072169843166\\
63	0.00548071537500777\\
64	0.00548070893453403\\
65	0.00548070237482737\\
66	0.00548069569366378\\
67	0.00548068888877726\\
68	0.00548068195785944\\
69	0.00548067489855835\\
70	0.00548066770847799\\
71	0.00548066038517733\\
72	0.00548065292616929\\
73	0.00548064532892022\\
74	0.00548063759084875\\
75	0.00548062970932491\\
76	0.00548062168166931\\
77	0.00548061350515221\\
78	0.00548060517699245\\
79	0.00548059669435663\\
80	0.00548058805435795\\
81	0.00548057925405533\\
82	0.00548057029045237\\
83	0.00548056116049621\\
84	0.00548055186107659\\
85	0.00548054238902466\\
86	0.00548053274111202\\
87	0.00548052291404934\\
88	0.0054805129044855\\
89	0.00548050270900618\\
90	0.00548049232413272\\
91	0.00548048174632115\\
92	0.00548047097196047\\
93	0.00548045999737188\\
94	0.00548044881880714\\
95	0.00548043743244753\\
96	0.00548042583440221\\
97	0.00548041402070723\\
98	0.00548040198732382\\
99	0.00548038973013709\\
100	0.0054803772449546\\
101	0.00548036452750497\\
102	0.00548035157343624\\
103	0.00548033837831449\\
104	0.00548032493762212\\
105	0.00548031124675644\\
106	0.00548029730102794\\
107	0.00548028309565878\\
108	0.00548026862578087\\
109	0.00548025388643441\\
110	0.00548023887256618\\
111	0.00548022357902757\\
112	0.0054802080005728\\
113	0.00548019213185738\\
114	0.00548017596743578\\
115	0.00548015950175985\\
116	0.00548014272917685\\
117	0.00548012564392743\\
118	0.00548010824014352\\
119	0.00548009051184649\\
120	0.00548007245294482\\
121	0.00548005405723204\\
122	0.00548003531838467\\
123	0.00548001622996009\\
124	0.00547999678539385\\
125	0.00547997697799801\\
126	0.00547995680095824\\
127	0.00547993624733152\\
128	0.00547991531004419\\
129	0.00547989398188898\\
130	0.00547987225552268\\
131	0.00547985012346352\\
132	0.00547982757808871\\
133	0.00547980461163155\\
134	0.00547978121617879\\
135	0.00547975738366809\\
136	0.00547973310588491\\
137	0.00547970837445984\\
138	0.00547968318086549\\
139	0.00547965751641371\\
140	0.00547963137225254\\
141	0.00547960473936312\\
142	0.00547957760855628\\
143	0.0054795499704698\\
144	0.00547952181556483\\
145	0.00547949313412261\\
146	0.00547946391624125\\
147	0.00547943415183217\\
148	0.00547940383061655\\
149	0.00547937294212185\\
150	0.00547934147567809\\
151	0.00547930942041437\\
152	0.00547927676525471\\
153	0.00547924349891452\\
154	0.00547920960989667\\
155	0.00547917508648734\\
156	0.00547913991675218\\
157	0.00547910408853202\\
158	0.00547906758943872\\
159	0.00547903040685087\\
160	0.0054789925279096\\
161	0.0054789539395139\\
162	0.00547891462831629\\
163	0.00547887458071825\\
164	0.00547883378286538\\
165	0.00547879222064283\\
166	0.0054787498796703\\
167	0.00547870674529745\\
168	0.00547866280259838\\
169	0.00547861803636722\\
170	0.00547857243111221\\
171	0.00547852597105114\\
172	0.0054784786401056\\
173	0.00547843042189563\\
174	0.00547838129973434\\
175	0.0054783312566222\\
176	0.00547828027524145\\
177	0.00547822833795021\\
178	0.00547817542677671\\
179	0.00547812152341338\\
180	0.00547806660921088\\
181	0.0054780106651719\\
182	0.00547795367194498\\
183	0.00547789560981841\\
184	0.00547783645871397\\
185	0.00547777619817999\\
186	0.00547771480738572\\
187	0.00547765226511421\\
188	0.00547758854975598\\
189	0.00547752363930238\\
190	0.00547745751133886\\
191	0.00547739014303824\\
192	0.00547732151115371\\
193	0.00547725159201235\\
194	0.00547718036150766\\
195	0.00547710779509384\\
196	0.00547703386778093\\
197	0.00547695855413523\\
198	0.00547688182828359\\
199	0.00547680366388242\\
200	0.00547672403410897\\
201	0.00547664291165308\\
202	0.00547656026870942\\
203	0.00547647607696921\\
204	0.00547639030761201\\
205	0.00547630293129755\\
206	0.00547621391815725\\
207	0.00547612323778595\\
208	0.00547603085923324\\
209	0.00547593675099499\\
210	0.00547584088100468\\
211	0.00547574321662483\\
212	0.00547564372463806\\
213	0.0054755423712384\\
214	0.00547543912202255\\
215	0.00547533394198096\\
216	0.00547522679548875\\
217	0.00547511764629692\\
218	0.00547500645752335\\
219	0.00547489319164394\\
220	0.00547477781048339\\
221	0.00547466027520632\\
222	0.00547454054630832\\
223	0.00547441858360674\\
224	0.00547429434623182\\
225	0.00547416779261774\\
226	0.00547403888049354\\
227	0.00547390756687405\\
228	0.00547377380805125\\
229	0.00547363755958502\\
230	0.00547349877629423\\
231	0.00547335741224781\\
232	0.005473213420756\\
233	0.00547306675436082\\
234	0.00547291736482763\\
235	0.0054727652031355\\
236	0.00547261021946829\\
237	0.00547245236320512\\
238	0.0054722915829106\\
239	0.00547212782632512\\
240	0.00547196104035452\\
241	0.00547179117105935\\
242	0.00547161816364347\\
243	0.00547144196244214\\
244	0.00547126251090899\\
245	0.00547107975160189\\
246	0.00547089362616763\\
247	0.00547070407532488\\
248	0.00547051103884505\\
249	0.00547031445553121\\
250	0.00547011426319354\\
251	0.00546991039862183\\
252	0.00546970279755407\\
253	0.00546949139464002\\
254	0.00546927612339942\\
255	0.00546905691617357\\
256	0.0054688337040694\\
257	0.00546860641689464\\
258	0.00546837498308231\\
259	0.00546813932960382\\
260	0.00546789938186764\\
261	0.00546765506360236\\
262	0.00546740629672147\\
263	0.00546715300116732\\
264	0.00546689509473118\\
265	0.00546663249284685\\
266	0.00546636510835429\\
267	0.00546609285122961\\
268	0.00546581562827825\\
269	0.00546553334278746\\
270	0.00546524589413507\\
271	0.00546495317735124\\
272	0.00546465508262966\\
273	0.0054643514947835\\
274	0.00546404229263197\\
275	0.00546372734826881\\
276	0.00546340652603965\\
277	0.0054630796806094\\
278	0.00546274665192147\\
279	0.00546240724915325\\
280	0.00546206128609614\\
281	0.00546170863564718\\
282	0.00546134916823953\\
283	0.00546098275179508\\
284	0.00546060925167596\\
285	0.00546022853063473\\
286	0.00545984044876465\\
287	0.00545944486344781\\
288	0.00545904162930304\\
289	0.00545863059813262\\
290	0.00545821161886802\\
291	0.00545778453751446\\
292	0.00545734919709457\\
293	0.00545690543759074\\
294	0.00545645309588682\\
295	0.00545599200570826\\
296	0.00545552199756122\\
297	0.00545504289867078\\
298	0.0054545545329177\\
299	0.0054540567207738\\
300	0.00545354927923665\\
301	0.00545303202176261\\
302	0.00545250475819857\\
303	0.00545196729471259\\
304	0.00545141943372308\\
305	0.00545086097382662\\
306	0.00545029170972414\\
307	0.00544971143214647\\
308	0.00544911992777732\\
309	0.00544851697917555\\
310	0.00544790236469576\\
311	0.00544727585840716\\
312	0.0054466372300107\\
313	0.00544598624475502\\
314	0.00544532266335037\\
315	0.00544464624188075\\
316	0.00544395673171442\\
317	0.00544325387941254\\
318	0.00544253742663585\\
319	0.00544180711004963\\
320	0.00544106266122603\\
321	0.00544030380654516\\
322	0.00543953026709337\\
323	0.00543874175855965\\
324	0.00543793799112967\\
325	0.00543711866937707\\
326	0.0054362834921529\\
327	0.00543543215247208\\
328	0.005434564337397\\
329	0.00543367972791881\\
330	0.00543277799883512\\
331	0.00543185881862527\\
332	0.0054309218493221\\
333	0.00542996674638014\\
334	0.00542899315854046\\
335	0.00542800072769201\\
336	0.00542698908872839\\
337	0.00542595786940121\\
338	0.0054249066901685\\
339	0.00542383516403881\\
340	0.00542274289641018\\
341	0.0054216294849042\\
342	0.00542049451919431\\
343	0.00541933758082834\\
344	0.00541815824304497\\
345	0.00541695607058338\\
346	0.00541573061948593\\
347	0.00541448143689398\\
348	0.00541320806083489\\
349	0.00541191002000126\\
350	0.00541058683352134\\
351	0.00540923801071954\\
352	0.00540786305086729\\
353	0.00540646144292367\\
354	0.00540503266526402\\
355	0.00540357618539771\\
356	0.00540209145967306\\
357	0.00540057793296957\\
358	0.0053990350383768\\
359	0.00539746219685888\\
360	0.00539585881690556\\
361	0.00539422429416723\\
362	0.00539255801107627\\
363	0.00539085933645289\\
364	0.00538912762509577\\
365	0.0053873622173592\\
366	0.00538556243871585\\
367	0.00538372759930749\\
368	0.00538185699348492\\
369	0.0053799498993387\\
370	0.00537800557822563\\
371	0.00537602327429234\\
372	0.0053740022140025\\
373	0.00537194160567334\\
374	0.00536984063902765\\
375	0.00536769848477218\\
376	0.00536551429421052\\
377	0.00536328719890381\\
378	0.00536101631039253\\
379	0.00535870071999319\\
380	0.00535633949868242\\
381	0.00535393169707442\\
382	0.00535147634547554\\
383	0.00534897245395576\\
384	0.00534641901230657\\
385	0.00534381498980157\\
386	0.00534115933575377\\
387	0.00533845098860062\\
388	0.00533568888069135\\
389	0.00533287192107836\\
390	0.00532999899488533\\
391	0.00532706896265217\\
392	0.00532408065965528\\
393	0.00532103289520269\\
394	0.00531792445190184\\
395	0.00531475408490083\\
396	0.00531152052110063\\
397	0.00530822245833811\\
398	0.00530485856453716\\
399	0.00530142747682666\\
400	0.00529792780062444\\
401	0.00529435810868445\\
402	0.00529071694010634\\
403	0.00528700279930482\\
404	0.00528321415493732\\
405	0.00527934943878715\\
406	0.00527540704460026\\
407	0.00527138532687276\\
408	0.00526728259958638\\
409	0.00526309713488923\\
410	0.00525882716171782\\
411	0.00525447086435603\\
412	0.0052500263809267\\
413	0.00524549180180995\\
414	0.00524086516798206\\
415	0.00523614446926672\\
416	0.00523132764248807\\
417	0.00522641256951514\\
418	0.00522139707518676\\
419	0.00521627892509709\\
420	0.00521105582321913\\
421	0.00520572540934116\\
422	0.00520028525629376\\
423	0.00519473286698471\\
424	0.0051890656713553\\
425	0.00518328102341414\\
426	0.00517737619710569\\
427	0.00517134838138553\\
428	0.0051651946761483\\
429	0.00515891208789345\\
430	0.00515249752510843\\
431	0.00514594779334966\\
432	0.00513925959000025\\
433	0.00513242949867982\\
434	0.00512545398328218\\
435	0.00511832938161421\\
436	0.00511105189860727\\
437	0.00510361759907029\\
438	0.00509602239995316\\
439	0.00508826206208469\\
440	0.00508033218134922\\
441	0.00507222817926275\\
442	0.00506394529290762\\
443	0.00505547856418199\\
444	0.005046822828319\\
445	0.00503797270162797\\
446	0.00502892256840949\\
447	0.00501966656699718\\
448	0.00501019857488201\\
449	0.00500051219286966\\
450	0.00499060072818567\\
451	0.0049804571764038\\
452	0.00497007420233899\\
453	0.00495944411973041\\
454	0.00494855886965451\\
455	0.00493740999761189\\
456	0.00492598862923287\\
457	0.00491428544455216\\
458	0.00490229065080857\\
459	0.00488999395373417\\
460	0.00487738452730974\\
461	0.00486445098197741\\
462	0.00485118133132173\\
463	0.00483756295725451\\
464	0.00482358257377012\\
465	0.00480922618937484\\
466	0.00479447906833696\\
467	0.00477932569094731\\
468	0.00476374971300784\\
469	0.00474773392474334\\
470	0.00473126020916734\\
471	0.00471430949936953\\
472	0.00469686173269478\\
473	0.00467889579597215\\
474	0.00466038944590183\\
475	0.0046413191586662\\
476	0.00462165983497246\\
477	0.0046013846366147\\
478	0.00458046289322046\\
479	0.00455885105141006\\
480	0.00453650300946854\\
481	0.00451338434121667\\
482	0.00448945977358611\\
483	0.00446469277837288\\
484	0.0044390455330161\\
485	0.00441247901676345\\
486	0.00438495317490672\\
487	0.00435642714567625\\
488	0.00432685956560797\\
489	0.00429620897197185\\
490	0.00426443432577805\\
491	0.00423149568450701\\
492	0.00419735506082228\\
493	0.00416197751252801\\
494	0.00412533252053638\\
495	0.00408739572641924\\
496	0.00404815112021496\\
497	0.00400759379329441\\
498	0.00396573339945054\\
499	0.00392259849595016\\
500	0.00387824197248343\\
501	0.00383274799840133\\
502	0.00378624154179356\\
503	0.00373890021273193\\
504	0.00369096392134772\\
505	0.00364273454391488\\
506	0.00359457071864176\\
507	0.00354702042024983\\
508	0.00350131809110213\\
509	0.00346061049916525\\
510	0.00342502183026241\\
511	0.00339439861452269\\
512	0.00336708322036617\\
513	0.00334086691234794\\
514	0.00331534730435857\\
515	0.00329008181911496\\
516	0.00326469884107946\\
517	0.00323892834194342\\
518	0.00321269198886212\\
519	0.00318593354574616\\
520	0.00315860607848959\\
521	0.00313068753007561\\
522	0.00310216113399003\\
523	0.00307301064759331\\
524	0.00304322031745423\\
525	0.00301277483353352\\
526	0.00298165833580482\\
527	0.00294985424466898\\
528	0.00291734506982389\\
529	0.00288411216856215\\
530	0.00285013547640853\\
531	0.00281539321418112\\
532	0.00277986195570541\\
533	0.00274351740193557\\
534	0.00270633319570252\\
535	0.00266822518666434\\
536	0.00263037485873978\\
537	0.0025936689644631\\
538	0.00255802540455848\\
539	0.00252190230147721\\
540	0.00248520925445124\\
541	0.00244790403567086\\
542	0.00240998217519735\\
543	0.00237144483485123\\
544	0.00233229481254299\\
545	0.00229253661599072\\
546	0.0022521764177109\\
547	0.00221122188822311\\
548	0.00216968132057519\\
549	0.00212756246687727\\
550	0.00208488317727384\\
551	0.00204166550220163\\
552	0.00199793669477371\\
553	0.0019537309880139\\
554	0.00190965922034648\\
555	0.00186720834870374\\
556	0.00182456308654889\\
557	0.00178148227067175\\
558	0.0017379831919425\\
559	0.0016940856717289\\
560	0.00164981119638615\\
561	0.00160518279977741\\
562	0.00156022488653992\\
563	0.00151496298230589\\
564	0.00146942342078429\\
565	0.00142363302680362\\
566	0.00137761899412085\\
567	0.00133166599321896\\
568	0.00128583765893305\\
569	0.0012397450467737\\
570	0.00119341585869848\\
571	0.00114688025839233\\
572	0.00110017097212528\\
573	0.00105332337464666\\
574	0.00100637555373387\\
575	0.000959368345375562\\
576	0.000912345329511703\\
577	0.000865352773713766\\
578	0.000818439509057374\\
579	0.00077165671858706\\
580	0.00072505761404041\\
581	0.000678696970709248\\
582	0.000632630483283932\\
583	0.000586913897120327\\
584	0.000541601859672158\\
585	0.00049674642660404\\
586	0.000452395149029591\\
587	0.000408588670365814\\
588	0.000365357795666474\\
589	0.000322720123375262\\
590	0.000280676711614417\\
591	0.000239312817487403\\
592	0.000198788644961841\\
593	0.000159293651685586\\
594	0.000121079888727805\\
595	8.45520570083909e-05\\
596	5.05092148680371e-05\\
597	2.07908715710836e-05\\
598	0\\
599	0\\
600	0\\
};
\addplot [color=mycolor2,solid,forget plot]
  table[row sep=crcr]{%
1	0.0054891968694182\\
2	0.00548919435537706\\
3	0.00548919179592685\\
4	0.00548918919024169\\
5	0.00548918653748061\\
6	0.00548918383678725\\
7	0.00548918108728961\\
8	0.00548917828809958\\
9	0.00548917543831285\\
10	0.00548917253700857\\
11	0.00548916958324894\\
12	0.00548916657607902\\
13	0.00548916351452631\\
14	0.00548916039760048\\
15	0.00548915722429305\\
16	0.00548915399357712\\
17	0.00548915070440682\\
18	0.00548914735571718\\
19	0.00548914394642359\\
20	0.0054891404754216\\
21	0.00548913694158649\\
22	0.00548913334377289\\
23	0.00548912968081434\\
24	0.00548912595152302\\
25	0.00548912215468925\\
26	0.00548911828908118\\
27	0.00548911435344429\\
28	0.00548911034650102\\
29	0.0054891062669503\\
30	0.00548910211346729\\
31	0.00548909788470254\\
32	0.005489093579282\\
33	0.00548908919580631\\
34	0.00548908473285036\\
35	0.0054890801889628\\
36	0.00548907556266564\\
37	0.00548907085245371\\
38	0.00548906605679403\\
39	0.00548906117412548\\
40	0.0054890562028583\\
41	0.00548905114137334\\
42	0.00548904598802173\\
43	0.00548904074112418\\
44	0.00548903539897048\\
45	0.00548902995981902\\
46	0.00548902442189594\\
47	0.00548901878339485\\
48	0.00548901304247607\\
49	0.00548900719726596\\
50	0.00548900124585643\\
51	0.00548899518630413\\
52	0.00548898901662991\\
53	0.0054889827348182\\
54	0.00548897633881622\\
55	0.00548896982653331\\
56	0.00548896319584039\\
57	0.00548895644456892\\
58	0.00548894957051057\\
59	0.005488942571416\\
60	0.00548893544499462\\
61	0.00548892818891339\\
62	0.00548892080079643\\
63	0.00548891327822374\\
64	0.00548890561873086\\
65	0.00548889781980776\\
66	0.00548888987889793\\
67	0.00548888179339792\\
68	0.00548887356065595\\
69	0.00548886517797149\\
70	0.00548885664259395\\
71	0.00548884795172195\\
72	0.0054888391025024\\
73	0.00548883009202948\\
74	0.00548882091734358\\
75	0.00548881157543053\\
76	0.0054888020632204\\
77	0.00548879237758641\\
78	0.00548878251534416\\
79	0.00548877247325024\\
80	0.00548876224800123\\
81	0.00548875183623264\\
82	0.00548874123451786\\
83	0.0054887304393668\\
84	0.00548871944722482\\
85	0.00548870825447156\\
86	0.00548869685741956\\
87	0.00548868525231328\\
88	0.00548867343532759\\
89	0.00548866140256658\\
90	0.00548864915006227\\
91	0.00548863667377318\\
92	0.00548862396958306\\
93	0.00548861103329949\\
94	0.00548859786065239\\
95	0.00548858444729253\\
96	0.00548857078879037\\
97	0.00548855688063409\\
98	0.00548854271822849\\
99	0.00548852829689328\\
100	0.0054885136118614\\
101	0.00548849865827747\\
102	0.00548848343119637\\
103	0.00548846792558115\\
104	0.00548845213630171\\
105	0.00548843605813288\\
106	0.0054884196857527\\
107	0.00548840301374054\\
108	0.00548838603657553\\
109	0.0054883687486344\\
110	0.00548835114418964\\
111	0.00548833321740775\\
112	0.00548831496234715\\
113	0.00548829637295601\\
114	0.00548827744307067\\
115	0.00548825816641306\\
116	0.00548823853658886\\
117	0.00548821854708523\\
118	0.00548819819126881\\
119	0.00548817746238323\\
120	0.00548815635354697\\
121	0.00548813485775122\\
122	0.00548811296785711\\
123	0.0054880906765936\\
124	0.00548806797655505\\
125	0.00548804486019857\\
126	0.00548802131984158\\
127	0.00548799734765942\\
128	0.00548797293568226\\
129	0.00548794807579285\\
130	0.00548792275972361\\
131	0.00548789697905397\\
132	0.00548787072520739\\
133	0.00548784398944858\\
134	0.00548781676288066\\
135	0.00548778903644202\\
136	0.00548776080090342\\
137	0.00548773204686483\\
138	0.00548770276475235\\
139	0.00548767294481489\\
140	0.00548764257712121\\
141	0.00548761165155607\\
142	0.00548758015781746\\
143	0.00548754808541279\\
144	0.00548751542365553\\
145	0.00548748216166155\\
146	0.00548744828834564\\
147	0.00548741379241772\\
148	0.00548737866237903\\
149	0.00548734288651851\\
150	0.00548730645290866\\
151	0.00548726934940173\\
152	0.00548723156362574\\
153	0.00548719308298007\\
154	0.00548715389463163\\
155	0.00548711398551041\\
156	0.00548707334230508\\
157	0.00548703195145868\\
158	0.00548698979916422\\
159	0.0054869468713599\\
160	0.00548690315372438\\
161	0.00548685863167226\\
162	0.0054868132903491\\
163	0.00548676711462645\\
164	0.00548672008909692\\
165	0.00548667219806894\\
166	0.00548662342556174\\
167	0.00548657375529968\\
168	0.00548652317070723\\
169	0.00548647165490311\\
170	0.00548641919069484\\
171	0.00548636576057296\\
172	0.00548631134670515\\
173	0.00548625593093042\\
174	0.00548619949475278\\
175	0.00548614201933525\\
176	0.00548608348549331\\
177	0.00548602387368881\\
178	0.00548596316402304\\
179	0.00548590133623029\\
180	0.00548583836967079\\
181	0.00548577424332394\\
182	0.00548570893578119\\
183	0.00548564242523852\\
184	0.00548557468948928\\
185	0.00548550570591667\\
186	0.00548543545148583\\
187	0.00548536390273596\\
188	0.00548529103577244\\
189	0.00548521682625827\\
190	0.00548514124940548\\
191	0.0054850642799659\\
192	0.00548498589222079\\
193	0.00548490605996795\\
194	0.00548482475650521\\
195	0.00548474195460491\\
196	0.00548465762647844\\
197	0.0054845717437398\\
198	0.00548448427744685\\
199	0.00548439519858112\\
200	0.0054843044776251\\
201	0.00548421208453498\\
202	0.00548411798873232\\
203	0.00548402215909528\\
204	0.00548392456395014\\
205	0.00548382517106224\\
206	0.00548372394762747\\
207	0.00548362086026317\\
208	0.0054835158749993\\
209	0.00548340895726937\\
210	0.00548330007190167\\
211	0.00548318918310984\\
212	0.0054830762544842\\
213	0.00548296124898257\\
214	0.00548284412892102\\
215	0.00548272485596526\\
216	0.00548260339112148\\
217	0.00548247969472741\\
218	0.0054823537264438\\
219	0.00548222544524543\\
220	0.00548209480941283\\
221	0.00548196177652399\\
222	0.00548182630344604\\
223	0.00548168834632761\\
224	0.00548154786059131\\
225	0.00548140480092645\\
226	0.00548125912128235\\
227	0.00548111077486217\\
228	0.00548095971411708\\
229	0.00548080589074135\\
230	0.00548064925566799\\
231	0.00548048975906534\\
232	0.00548032735033443\\
233	0.00548016197810786\\
234	0.00547999359024942\\
235	0.00547982213385538\\
236	0.00547964755525738\\
237	0.00547946980002722\\
238	0.00547928881298338\\
239	0.00547910453820034\\
240	0.00547891691901986\\
241	0.00547872589806595\\
242	0.00547853141726252\\
243	0.00547833341785528\\
244	0.00547813184043755\\
245	0.00547792662498099\\
246	0.00547771771087146\\
247	0.00547750503695109\\
248	0.00547728854156713\\
249	0.00547706816262813\\
250	0.00547684383766866\\
251	0.00547661550392378\\
252	0.00547638309841347\\
253	0.00547614655803969\\
254	0.00547590581969587\\
255	0.00547566082039141\\
256	0.0054754114973923\\
257	0.0054751577883796\\
258	0.00547489963162761\\
259	0.00547463696620386\\
260	0.00547436973219279\\
261	0.00547409787094485\\
262	0.00547382132535371\\
263	0.00547354004016329\\
264	0.00547325396230583\\
265	0.00547296304127366\\
266	0.00547266722952463\\
267	0.00547236648292219\\
268	0.00547206076120906\\
269	0.00547175002851286\\
270	0.00547143425387946\\
271	0.00547111341182841\\
272	0.00547078748292199\\
273	0.00547045645433792\\
274	0.00547012032043892\\
275	0.00546977908334958\\
276	0.00546943275361923\\
277	0.00546908135127342\\
278	0.00546872490828844\\
279	0.00546836347592409\\
280	0.00546799639134473\\
281	0.00546762240864155\\
282	0.00546724139884568\\
283	0.00546685323060331\\
284	0.00546645777013235\\
285	0.00546605488117901\\
286	0.00546564442497286\\
287	0.00546522626018168\\
288	0.00546480024286546\\
289	0.00546436622642938\\
290	0.00546392406157631\\
291	0.00546347359625828\\
292	0.00546301467562746\\
293	0.00546254714198608\\
294	0.0054620708347355\\
295	0.00546158559032467\\
296	0.00546109124219773\\
297	0.00546058762074055\\
298	0.00546007455322658\\
299	0.00545955186376227\\
300	0.00545901937323075\\
301	0.00545847689923528\\
302	0.00545792425604187\\
303	0.00545736125452067\\
304	0.00545678770208683\\
305	0.00545620340264023\\
306	0.00545560815650464\\
307	0.00545500176036559\\
308	0.0054543840072078\\
309	0.00545375468625155\\
310	0.0054531135828881\\
311	0.00545246047861439\\
312	0.00545179515096701\\
313	0.00545111737345481\\
314	0.00545042691549136\\
315	0.00544972354232614\\
316	0.00544900701497484\\
317	0.0054482770901493\\
318	0.00544753352018626\\
319	0.00544677605297551\\
320	0.00544600443188722\\
321	0.00544521839569857\\
322	0.00544441767851983\\
323	0.00544360200971929\\
324	0.00544277111384802\\
325	0.00544192471056379\\
326	0.00544106251455431\\
327	0.00544018423545993\\
328	0.00543928957779588\\
329	0.00543837824087373\\
330	0.00543744991872266\\
331	0.00543650430001013\\
332	0.00543554106796177\\
333	0.0054345599002812\\
334	0.00543356046906949\\
335	0.00543254244074374\\
336	0.00543150547595554\\
337	0.00543044922950901\\
338	0.00542937335027815\\
339	0.00542827748112374\\
340	0.00542716125880973\\
341	0.00542602431391878\\
342	0.00542486627076723\\
343	0.00542368674731854\\
344	0.00542248535509619\\
345	0.00542126169909461\\
346	0.00542001537768867\\
347	0.00541874598254035\\
348	0.00541745309850346\\
349	0.00541613630352446\\
350	0.00541479516853939\\
351	0.00541342925736604\\
352	0.00541203812658974\\
353	0.00541062132544212\\
354	0.00540917839567121\\
355	0.00540770887140096\\
356	0.00540621227897795\\
357	0.00540468813680368\\
358	0.00540313595514848\\
359	0.00540155523594473\\
360	0.00539994547255471\\
361	0.00539830614950943\\
362	0.00539663674221262\\
363	0.00539493671660397\\
364	0.00539320552877565\\
365	0.00539144262453276\\
366	0.00538964743889032\\
367	0.00538781939549552\\
368	0.0053859579059644\\
369	0.00538406236912007\\
370	0.00538213217011732\\
371	0.00538016667943867\\
372	0.00537816525174382\\
373	0.00537612722455306\\
374	0.00537405191674505\\
375	0.00537193862684648\\
376	0.00536978663109177\\
377	0.00536759518123094\\
378	0.00536536350206276\\
379	0.00536309078867397\\
380	0.00536077620336287\\
381	0.00535841887221443\\
382	0.00535601788124313\\
383	0.00535357227181194\\
384	0.00535108103429341\\
385	0.0053485430962889\\
386	0.00534595729230035\\
387	0.00534332226783449\\
388	0.00534063667519442\\
389	0.00533789953008658\\
390	0.00533510982741454\\
391	0.0053322665406314\\
392	0.00532936862105515\\
393	0.00532641499714516\\
394	0.00532340457373629\\
395	0.00532033623122087\\
396	0.00531720882469156\\
397	0.00531402118304865\\
398	0.00531077210806497\\
399	0.00530746037340651\\
400	0.00530408472360693\\
401	0.00530064387299334\\
402	0.0052971365045622\\
403	0.00529356126880294\\
404	0.00528991678246438\\
405	0.00528620162727035\\
406	0.00528241434857942\\
407	0.00527855345397782\\
408	0.00527461741181547\\
409	0.0052706046496879\\
410	0.00526651355286785\\
411	0.00526234246269237\\
412	0.00525808967491305\\
413	0.00525375343801949\\
414	0.00524933195154946\\
415	0.00524482336440346\\
416	0.00524022577318039\\
417	0.00523553722053481\\
418	0.00523075569352813\\
419	0.00522587912211942\\
420	0.00522090537774521\\
421	0.00521583227193822\\
422	0.00521065755485279\\
423	0.00520537891343096\\
424	0.00519999396908422\\
425	0.00519450027717447\\
426	0.00518889534523272\\
427	0.00518317663173477\\
428	0.00517734151593173\\
429	0.00517138729411216\\
430	0.00516531117563831\\
431	0.00515911027874116\\
432	0.00515278162605946\\
433	0.00514632213990524\\
434	0.00513972863723734\\
435	0.00513299782432283\\
436	0.00512612629106462\\
437	0.0051191105049721\\
438	0.00511194680474882\\
439	0.00510463139346928\\
440	0.00509716033131396\\
441	0.0050895295278284\\
442	0.00508173473366834\\
443	0.00507377153178841\\
444	0.00506563532802464\\
445	0.00505732134101474\\
446	0.00504882459138982\\
447	0.00504013989016983\\
448	0.00503126182632091\\
449	0.00502218475354018\\
450	0.00501290277652464\\
451	0.00500340973646137\\
452	0.0049936991921593\\
453	0.00498376440288637\\
454	0.00497359831009461\\
455	0.00496319351796841\\
456	0.00495254227273068\\
457	0.00494163644064148\\
458	0.00493046748462266\\
459	0.0049190264394442\\
460	0.00490730388540931\\
461	0.00489528992047967\\
462	0.00488297413078798\\
463	0.00487034555949376\\
464	0.00485739267395021\\
465	0.004844103331166\\
466	0.00483046474156964\\
467	0.00481646343111338\\
468	0.00480208520178408\\
469	0.00478731509058795\\
470	0.00477213732701297\\
471	0.00475653528923159\\
472	0.00474049145901888\\
473	0.00472398737499751\\
474	0.00470700358311135\\
475	0.00468951958190851\\
476	0.00467151375681195\\
477	0.00465296326746879\\
478	0.00463384386379001\\
479	0.00461412987626572\\
480	0.00459379379728429\\
481	0.00457280301484697\\
482	0.00455110433656644\\
483	0.00452865953063785\\
484	0.00450543286302288\\
485	0.00448138762584243\\
486	0.0044564855239955\\
487	0.00443068671594453\\
488	0.00440394987873579\\
489	0.00437623233356901\\
490	0.00434749022961062\\
491	0.00431767880121355\\
492	0.00428675271652006\\
493	0.00425466653997079\\
494	0.00422137533634864\\
495	0.00418683545021244\\
496	0.00415100550209224\\
497	0.00411384765185985\\
498	0.00407532919063057\\
499	0.00403542453652763\\
500	0.00399411772952133\\
501	0.00395140554229041\\
502	0.00390730132316084\\
503	0.00386183971978261\\
504	0.00381508261869486\\
505	0.00376712721209702\\
506	0.00371811743437225\\
507	0.0036682546917953\\
508	0.00361780571383835\\
509	0.00356710002107475\\
510	0.00351655779440234\\
511	0.00346685990014391\\
512	0.00342018933600302\\
513	0.00337865550221436\\
514	0.00334233331140309\\
515	0.00331092069354877\\
516	0.00328172789297185\\
517	0.00325356226126486\\
518	0.00322599612173834\\
519	0.00319856266008969\\
520	0.00317113359201397\\
521	0.00314330028321649\\
522	0.00311494865759424\\
523	0.00308604010602848\\
524	0.00305653561648579\\
525	0.00302639862942524\\
526	0.00299561110095096\\
527	0.00296415546625934\\
528	0.00293201443441139\\
529	0.00289917111057011\\
530	0.00286560835076743\\
531	0.00283130815884521\\
532	0.00279625143981421\\
533	0.0027604176819128\\
534	0.00272378461722965\\
535	0.00268632921308865\\
536	0.00264802742428661\\
537	0.00260881862072326\\
538	0.00256896387323768\\
539	0.00253012203665061\\
540	0.00249255684260618\\
541	0.00245517426589838\\
542	0.00241727949581605\\
543	0.00237877084369842\\
544	0.00233964426046532\\
545	0.00229989957797116\\
546	0.00225954118691839\\
547	0.00221857506323328\\
548	0.00217700799179227\\
549	0.00213484522472573\\
550	0.00209210022016971\\
551	0.00204879104734812\\
552	0.00200493976528336\\
553	0.00196057319704059\\
554	0.00191572431648802\\
555	0.00187042054078887\\
556	0.00182648374393835\\
557	0.00178330361132209\\
558	0.00173971373924418\\
559	0.00169571747444497\\
560	0.00165133624243625\\
561	0.00160659337207\\
562	0.00156151382896341\\
563	0.00151612401849032\\
564	0.00147045151570613\\
565	0.00142452473202305\\
566	0.00137837259118908\\
567	0.00133202429968801\\
568	0.00128583765932925\\
569	0.00123974504679409\\
570	0.00119341585870728\\
571	0.00114688025839642\\
572	0.0011001709721271\\
573	0.00105332337464741\\
574	0.00100637555373416\\
575	0.000959368345375665\\
576	0.000912345329511744\\
577	0.000865352773713785\\
578	0.000818439509057384\\
579	0.00077165671858706\\
580	0.000725057614040401\\
581	0.000678696970709242\\
582	0.000632630483283921\\
583	0.000586913897120319\\
584	0.000541601859672149\\
585	0.00049674642660404\\
586	0.000452395149029591\\
587	0.000408588670365815\\
588	0.000365357795666476\\
589	0.00032272012337526\\
590	0.000280676711614416\\
591	0.000239312817487405\\
592	0.000198788644961842\\
593	0.000159293651685586\\
594	0.000121079888727804\\
595	8.45520570083913e-05\\
596	5.05092148680373e-05\\
597	2.07908715710836e-05\\
598	0\\
599	0\\
600	0\\
};
\addplot [color=mycolor3,solid,forget plot]
  table[row sep=crcr]{%
1	0.00550477682330948\\
2	0.00550477425642561\\
3	0.0055047716434672\\
4	0.00550476898360203\\
5	0.00550476627598275\\
6	0.00550476351974651\\
7	0.00550476071401469\\
8	0.00550475785789298\\
9	0.00550475495047054\\
10	0.00550475199082008\\
11	0.00550474897799734\\
12	0.00550474591104091\\
13	0.00550474278897191\\
14	0.00550473961079362\\
15	0.0055047363754912\\
16	0.00550473308203119\\
17	0.0055047297293615\\
18	0.00550472631641071\\
19	0.00550472284208796\\
20	0.00550471930528255\\
21	0.00550471570486342\\
22	0.00550471203967905\\
23	0.00550470830855674\\
24	0.00550470451030261\\
25	0.00550470064370086\\
26	0.0055046967075136\\
27	0.00550469270048034\\
28	0.00550468862131753\\
29	0.00550468446871832\\
30	0.00550468024135183\\
31	0.00550467593786316\\
32	0.00550467155687248\\
33	0.00550466709697481\\
34	0.00550466255673955\\
35	0.00550465793470994\\
36	0.00550465322940267\\
37	0.00550464843930729\\
38	0.00550464356288576\\
39	0.00550463859857201\\
40	0.00550463354477119\\
41	0.00550462839985948\\
42	0.00550462316218334\\
43	0.00550461783005889\\
44	0.00550461240177163\\
45	0.00550460687557555\\
46	0.00550460124969282\\
47	0.00550459552231303\\
48	0.00550458969159265\\
49	0.00550458375565442\\
50	0.0055045777125868\\
51	0.0055045715604432\\
52	0.00550456529724142\\
53	0.005504558920963\\
54	0.00550455242955239\\
55	0.00550454582091652\\
56	0.00550453909292396\\
57	0.00550453224340419\\
58	0.00550452527014696\\
59	0.0055045181709016\\
60	0.00550451094337607\\
61	0.00550450358523637\\
62	0.00550449609410563\\
63	0.00550448846756352\\
64	0.0055044807031453\\
65	0.00550447279834096\\
66	0.00550446475059457\\
67	0.00550445655730317\\
68	0.00550444821581615\\
69	0.00550443972343412\\
70	0.00550443107740831\\
71	0.00550442227493934\\
72	0.00550441331317651\\
73	0.00550440418921677\\
74	0.00550439490010382\\
75	0.0055043854428269\\
76	0.00550437581432005\\
77	0.00550436601146093\\
78	0.00550435603106987\\
79	0.00550434586990864\\
80	0.00550433552467965\\
81	0.00550432499202459\\
82	0.00550431426852338\\
83	0.00550430335069311\\
84	0.00550429223498672\\
85	0.00550428091779194\\
86	0.00550426939543011\\
87	0.00550425766415471\\
88	0.00550424572015037\\
89	0.00550423355953152\\
90	0.00550422117834099\\
91	0.0055042085725488\\
92	0.00550419573805069\\
93	0.00550418267066679\\
94	0.00550416936614034\\
95	0.00550415582013608\\
96	0.0055041420282389\\
97	0.00550412798595239\\
98	0.00550411368869705\\
99	0.00550409913180917\\
100	0.00550408431053899\\
101	0.00550406922004924\\
102	0.0055040538554133\\
103	0.00550403821161376\\
104	0.00550402228354073\\
105	0.00550400606598996\\
106	0.00550398955366117\\
107	0.00550397274115646\\
108	0.00550395562297808\\
109	0.00550393819352694\\
110	0.00550392044710062\\
111	0.00550390237789137\\
112	0.0055038839799842\\
113	0.00550386524735506\\
114	0.00550384617386849\\
115	0.00550382675327583\\
116	0.00550380697921296\\
117	0.00550378684519831\\
118	0.0055037663446305\\
119	0.00550374547078629\\
120	0.00550372421681824\\
121	0.00550370257575223\\
122	0.00550368054048561\\
123	0.00550365810378426\\
124	0.00550363525828055\\
125	0.00550361199647067\\
126	0.00550358831071218\\
127	0.0055035641932214\\
128	0.00550353963607094\\
129	0.00550351463118682\\
130	0.005503489170346\\
131	0.00550346324517338\\
132	0.00550343684713932\\
133	0.00550340996755649\\
134	0.00550338259757716\\
135	0.00550335472819009\\
136	0.00550332635021771\\
137	0.00550329745431279\\
138	0.0055032680309557\\
139	0.00550323807045099\\
140	0.00550320756292402\\
141	0.00550317649831816\\
142	0.005503144866391\\
143	0.00550311265671116\\
144	0.00550307985865477\\
145	0.00550304646140206\\
146	0.00550301245393358\\
147	0.00550297782502669\\
148	0.00550294256325193\\
149	0.00550290665696907\\
150	0.00550287009432344\\
151	0.00550283286324184\\
152	0.0055027949514287\\
153	0.00550275634636214\\
154	0.00550271703528952\\
155	0.00550267700522361\\
156	0.00550263624293808\\
157	0.00550259473496338\\
158	0.00550255246758201\\
159	0.00550250942682436\\
160	0.00550246559846392\\
161	0.00550242096801271\\
162	0.00550237552071655\\
163	0.00550232924155014\\
164	0.00550228211521213\\
165	0.00550223412612033\\
166	0.00550218525840629\\
167	0.00550213549591032\\
168	0.00550208482217605\\
169	0.00550203322044512\\
170	0.00550198067365169\\
171	0.00550192716441669\\
172	0.00550187267504231\\
173	0.0055018171875059\\
174	0.00550176068345432\\
175	0.00550170314419761\\
176	0.00550164455070308\\
177	0.00550158488358875\\
178	0.00550152412311714\\
179	0.00550146224918863\\
180	0.00550139924133485\\
181	0.00550133507871183\\
182	0.00550126974009321\\
183	0.00550120320386315\\
184	0.00550113544800915\\
185	0.00550106645011499\\
186	0.00550099618735303\\
187	0.00550092463647722\\
188	0.00550085177381503\\
189	0.00550077757526\\
190	0.00550070201626403\\
191	0.00550062507182938\\
192	0.00550054671650088\\
193	0.00550046692435791\\
194	0.00550038566900639\\
195	0.00550030292357135\\
196	0.00550021866068941\\
197	0.00550013285250246\\
198	0.00550004547065237\\
199	0.00549995648627275\\
200	0.00549986586996504\\
201	0.00549977359178749\\
202	0.00549967962124463\\
203	0.00549958392727645\\
204	0.00549948647824725\\
205	0.00549938724193426\\
206	0.005499286185516\\
207	0.00549918327556026\\
208	0.00549907847801172\\
209	0.00549897175817939\\
210	0.00549886308072349\\
211	0.0054987524096421\\
212	0.0054986397082573\\
213	0.00549852493920084\\
214	0.0054984080643997\\
215	0.00549828904506062\\
216	0.00549816784165441\\
217	0.00549804441390017\\
218	0.00549791872074772\\
219	0.00549779072036074\\
220	0.00549766037009842\\
221	0.00549752762649653\\
222	0.00549739244524797\\
223	0.00549725478118225\\
224	0.00549711458824402\\
225	0.00549697181947082\\
226	0.00549682642696964\\
227	0.00549667836189246\\
228	0.00549652757441025\\
229	0.00549637401368581\\
230	0.00549621762784532\\
231	0.00549605836394773\\
232	0.00549589616795276\\
233	0.0054957309846868\\
234	0.00549556275780674\\
235	0.00549539142976125\\
236	0.00549521694174974\\
237	0.00549503923367794\\
238	0.00549485824411096\\
239	0.00549467391022222\\
240	0.00549448616773887\\
241	0.00549429495088275\\
242	0.00549410019230668\\
243	0.0054939018230254\\
244	0.00549369977234063\\
245	0.00549349396775976\\
246	0.00549328433490728\\
247	0.00549307079742838\\
248	0.00549285327688355\\
249	0.00549263169263335\\
250	0.00549240596171277\\
251	0.00549217599869314\\
252	0.00549194171553131\\
253	0.00549170302140385\\
254	0.00549145982252539\\
255	0.00549121202194953\\
256	0.0054909595193502\\
257	0.0054907022107821\\
258	0.00549043998841833\\
259	0.00549017274026306\\
260	0.00548990034983755\\
261	0.00548962269583773\\
262	0.00548933965176146\\
263	0.00548905108550404\\
264	0.00548875685892081\\
265	0.00548845682735624\\
266	0.0054881508391397\\
267	0.00548783873504886\\
268	0.00548752034774364\\
269	0.00548719550117455\\
270	0.00548686400997268\\
271	0.00548652567883073\\
272	0.0054861803018893\\
273	0.00548582766214855\\
274	0.00548546753093547\\
275	0.00548509966748068\\
276	0.00548472381871943\\
277	0.00548433971962728\\
278	0.00548394709507049\\
279	0.00548354566654718\\
280	0.00548313585038502\\
281	0.00548271859907244\\
282	0.00548229377981785\\
283	0.0054818612575201\\
284	0.00548142089472977\\
285	0.00548097255161\\
286	0.00548051608589664\\
287	0.00548005135285767\\
288	0.0054795782052522\\
289	0.00547909649328861\\
290	0.00547860606458215\\
291	0.00547810676411174\\
292	0.00547759843417632\\
293	0.00547708091435022\\
294	0.005476554041438\\
295	0.00547601764942866\\
296	0.00547547156944895\\
297	0.00547491562971612\\
298	0.00547434965548969\\
299	0.00547377346902287\\
300	0.00547318688951282\\
301	0.00547258973305047\\
302	0.00547198181256938\\
303	0.00547136293779389\\
304	0.00547073291518661\\
305	0.005470091547895\\
306	0.00546943863569711\\
307	0.00546877397494682\\
308	0.00546809735851799\\
309	0.00546740857574802\\
310	0.00546670741238051\\
311	0.00546599365050735\\
312	0.00546526706850985\\
313	0.00546452744099923\\
314	0.00546377453875629\\
315	0.00546300812867045\\
316	0.00546222797367823\\
317	0.00546143383270067\\
318	0.00546062546058052\\
319	0.00545980260801873\\
320	0.00545896502151018\\
321	0.00545811244327932\\
322	0.00545724461121499\\
323	0.00545636125880529\\
324	0.00545546211507171\\
325	0.00545454690450366\\
326	0.00545361534699242\\
327	0.0054526671577656\\
328	0.00545170204732146\\
329	0.00545071972136402\\
330	0.00544971988073853\\
331	0.00544870222136763\\
332	0.00544766643418897\\
333	0.00544661220509387\\
334	0.00544553921486785\\
335	0.00544444713913311\\
336	0.00544333564829348\\
337	0.00544220440748248\\
338	0.0054410530765146\\
339	0.00543988130984097\\
340	0.0054386887565096\\
341	0.00543747506013127\\
342	0.0054362398588518\\
343	0.00543498278533194\\
344	0.00543370346673545\\
345	0.00543240152472745\\
346	0.00543107657548345\\
347	0.00542972822971189\\
348	0.00542835609269072\\
349	0.00542695976432092\\
350	0.00542553883919896\\
351	0.00542409290671077\\
352	0.00542262155115011\\
353	0.00542112435186449\\
354	0.00541960088343242\\
355	0.00541805071587583\\
356	0.00541647341491245\\
357	0.00541486854225276\\
358	0.00541323565594737\\
359	0.00541157431079114\\
360	0.00540988405879056\\
361	0.0054081644497021\\
362	0.00540641503164977\\
363	0.00540463535183105\\
364	0.0054028249573207\\
365	0.00540098339598347\\
366	0.00539911021750652\\
367	0.00539720497456394\\
368	0.00539526722412573\\
369	0.00539329652892395\\
370	0.00539129245908901\\
371	0.00538925459396853\\
372	0.00538718252414036\\
373	0.00538507585362938\\
374	0.00538293420233504\\
375	0.00538075720867236\\
376	0.00537854453242384\\
377	0.00537629585779\\
378	0.00537401089661826\\
379	0.00537168939177402\\
380	0.00536933112060602\\
381	0.00536693589844766\\
382	0.00536450358211477\\
383	0.00536203407346954\\
384	0.00535952732353251\\
385	0.00535698333899241\\
386	0.00535440219739541\\
387	0.00535178409178646\\
388	0.00534912514988319\\
389	0.00534641761903743\\
390	0.00534366061596057\\
391	0.00534085324265133\\
392	0.00533799458610111\\
393	0.00533508371795533\\
394	0.00533211969415246\\
395	0.00532910155454309\\
396	0.00532602832211091\\
397	0.00532289900203223\\
398	0.00531971258080888\\
399	0.00531646802530294\\
400	0.00531316428168546\\
401	0.00530980027430732\\
402	0.00530637490449251\\
403	0.00530288704928074\\
404	0.00529933556009359\\
405	0.00529571926117679\\
406	0.00529203694800453\\
407	0.00528828738570286\\
408	0.00528446930703295\\
409	0.00528058141012162\\
410	0.00527662235590618\\
411	0.00527259076525703\\
412	0.00526848521573851\\
413	0.00526430423796721\\
414	0.00526004631153165\\
415	0.00525570986046398\\
416	0.00525129324831951\\
417	0.00524679477302037\\
418	0.00524221266141137\\
419	0.00523754506071366\\
420	0.00523279003052249\\
421	0.00522794553410434\\
422	0.0052230094281619\\
423	0.00521797944891619\\
424	0.00521285318686776\\
425	0.00520762802306626\\
426	0.00520230092936529\\
427	0.00519686899937804\\
428	0.00519132985949917\\
429	0.00518568105926974\\
430	0.00517992006754721\\
431	0.00517404426844573\\
432	0.00516805095702335\\
433	0.00516193733472785\\
434	0.00515570050460202\\
435	0.00514933746622776\\
436	0.00514284511040285\\
437	0.00513622021354724\\
438	0.00512945943183804\\
439	0.0051225592950757\\
440	0.00511551620027982\\
441	0.00510832640502191\\
442	0.00510098602050871\\
443	0.00509349100442846\\
444	0.00508583715357047\\
445	0.00507802009621341\\
446	0.00507003528423538\\
447	0.00506187798479317\\
448	0.00505354327122278\\
449	0.00504502601268046\\
450	0.00503632086333021\\
451	0.00502742226201914\\
452	0.00501832447056875\\
453	0.00500902149371753\\
454	0.00499950706396044\\
455	0.00498977462543773\\
456	0.00497981731679798\\
457	0.00496962795297015\\
458	0.00495919900577868\\
459	0.00494852258333314\\
460	0.00493759040812209\\
461	0.00492639379373232\\
462	0.00491492362011725\\
463	0.00490317030733283\\
464	0.00489112378765591\\
465	0.00487877347599587\\
466	0.00486610823851244\\
467	0.00485311635938098\\
468	0.00483978550576249\\
469	0.0048261026913033\\
470	0.00481205423849246\\
471	0.00479762573586072\\
472	0.00478280199444234\\
473	0.00476756700393987\\
474	0.00475190388728252\\
475	0.00473579485348131\\
476	0.00471922114823902\\
477	0.0047021630016523\\
478	0.00468459957115042\\
479	0.00466650885784214\\
480	0.0046478675515963\\
481	0.0046286508102488\\
482	0.00460883243742465\\
483	0.00458838400496198\\
484	0.00456726999448731\\
485	0.00454543469310754\\
486	0.00452284311214705\\
487	0.00449945871076504\\
488	0.00447524373166766\\
489	0.00445015868630056\\
490	0.00442416236753183\\
491	0.00439721188603949\\
492	0.00436926275841055\\
493	0.00434026904545963\\
494	0.00431018355398794\\
495	0.00427895811764693\\
496	0.00424654397658071\\
497	0.00421289228003259\\
498	0.00417795474171066\\
499	0.00414168448465006\\
500	0.0041040371207246\\
501	0.0040649721200045\\
502	0.00402445453844905\\
503	0.00398245719006864\\
504	0.00393896337156799\\
505	0.00389397025230831\\
506	0.00384749302944713\\
507	0.00379957011758446\\
508	0.00375026981833159\\
509	0.00369969955863891\\
510	0.0036480175997573\\
511	0.00359544297495342\\
512	0.0035422627542797\\
513	0.00348883084154451\\
514	0.0034356354815106\\
515	0.00338346436248934\\
516	0.00333529941397275\\
517	0.00329239584335947\\
518	0.00325479594604639\\
519	0.00322212145566726\\
520	0.00319081861398368\\
521	0.0031604552962268\\
522	0.00313060409131133\\
523	0.0031007791385642\\
524	0.00307090240615044\\
525	0.00304081945040074\\
526	0.00301018748031608\\
527	0.00297896731824318\\
528	0.0029471211890936\\
529	0.00291460839813878\\
530	0.00288140050348125\\
531	0.00284747823876823\\
532	0.00281282247445835\\
533	0.00277741396691225\\
534	0.00274123362774897\\
535	0.0027042612251743\\
536	0.00266647498145895\\
537	0.00262785203732332\\
538	0.00258836871834692\\
539	0.00254799039987889\\
540	0.00250663194236034\\
541	0.00246552657985651\\
542	0.00242551775773468\\
543	0.00238679174078739\\
544	0.00234758968952664\\
545	0.00230783756843862\\
546	0.00226746688232517\\
547	0.00222647633461372\\
548	0.00218486861377273\\
549	0.00214264741712259\\
550	0.00209982152218929\\
551	0.00205640582806659\\
552	0.00201241871448934\\
553	0.00196788261038474\\
554	0.001922824759329\\
555	0.00187727861552058\\
556	0.00183128530984319\\
557	0.00178571858457715\\
558	0.00174175370183844\\
559	0.00169765189645918\\
560	0.00165315928882859\\
561	0.00160829620918484\\
562	0.0015630878421284\\
563	0.00151756120705137\\
564	0.00147174485938914\\
565	0.00142566860063263\\
566	0.00137936312148276\\
567	0.00133285966646703\\
568	0.00128618971348929\\
569	0.00123974505182\\
570	0.00119341585886474\\
571	0.00114688025846023\\
572	0.00110017097215749\\
573	0.00105332337466137\\
574	0.00100637555374019\\
575	0.000959368345378089\\
576	0.00091234532951263\\
577	0.00086535277371407\\
578	0.00081843950905746\\
579	0.000771656718587084\\
580	0.000725057614040413\\
581	0.000678696970709247\\
582	0.000632630483283924\\
583	0.00058691389712032\\
584	0.000541601859672149\\
585	0.000496746426604037\\
586	0.000452395149029591\\
587	0.000408588670365813\\
588	0.000365357795666473\\
589	0.000322720123375262\\
590	0.000280676711614414\\
591	0.000239312817487402\\
592	0.000198788644961839\\
593	0.000159293651685585\\
594	0.000121079888727804\\
595	8.45520570083908e-05\\
596	5.05092148680372e-05\\
597	2.07908715710836e-05\\
598	0\\
599	0\\
600	0\\
};
\addplot [color=mycolor4,solid,forget plot]
  table[row sep=crcr]{%
1	0.00552236449378588\\
2	0.00552236161741465\\
3	0.00552235869004543\\
4	0.00552235571076934\\
5	0.00552235267866136\\
6	0.00552234959277986\\
7	0.00552234645216632\\
8	0.00552234325584501\\
9	0.0055223400028227\\
10	0.00552233669208844\\
11	0.0055223333226131\\
12	0.00552232989334907\\
13	0.00552232640323002\\
14	0.00552232285117041\\
15	0.00552231923606535\\
16	0.00552231555678995\\
17	0.00552231181219927\\
18	0.0055223080011278\\
19	0.0055223041223891\\
20	0.00552230017477544\\
21	0.0055222961570574\\
22	0.00552229206798352\\
23	0.00552228790627984\\
24	0.00552228367064954\\
25	0.00552227935977247\\
26	0.00552227497230478\\
27	0.00552227050687855\\
28	0.00552226596210119\\
29	0.00552226133655511\\
30	0.00552225662879733\\
31	0.00552225183735878\\
32	0.0055222469607441\\
33	0.00552224199743105\\
34	0.00552223694586996\\
35	0.00552223180448328\\
36	0.00552222657166512\\
37	0.00552222124578073\\
38	0.00552221582516592\\
39	0.00552221030812658\\
40	0.00552220469293806\\
41	0.00552219897784472\\
42	0.0055221931610593\\
43	0.00552218724076237\\
44	0.00552218121510173\\
45	0.00552217508219179\\
46	0.00552216884011311\\
47	0.00552216248691159\\
48	0.00552215602059788\\
49	0.00552214943914689\\
50	0.00552214274049698\\
51	0.00552213592254938\\
52	0.00552212898316735\\
53	0.00552212192017577\\
54	0.00552211473136027\\
55	0.00552210741446656\\
56	0.00552209996719957\\
57	0.00552209238722291\\
58	0.00552208467215801\\
59	0.00552207681958337\\
60	0.00552206882703378\\
61	0.0055220606919995\\
62	0.00552205241192548\\
63	0.00552204398421052\\
64	0.00552203540620642\\
65	0.00552202667521712\\
66	0.00552201778849787\\
67	0.00552200874325425\\
68	0.00552199953664139\\
69	0.00552199016576291\\
70	0.00552198062767006\\
71	0.00552197091936075\\
72	0.00552196103777858\\
73	0.00552195097981178\\
74	0.00552194074229234\\
75	0.00552193032199479\\
76	0.00552191971563531\\
77	0.00552190891987057\\
78	0.00552189793129664\\
79	0.00552188674644797\\
80	0.00552187536179611\\
81	0.00552186377374865\\
82	0.00552185197864809\\
83	0.00552183997277049\\
84	0.00552182775232437\\
85	0.00552181531344947\\
86	0.00552180265221537\\
87	0.00552178976462037\\
88	0.00552177664659001\\
89	0.00552176329397577\\
90	0.00552174970255381\\
91	0.00552173586802346\\
92	0.00552172178600583\\
93	0.00552170745204245\\
94	0.00552169286159359\\
95	0.00552167801003702\\
96	0.00552166289266629\\
97	0.0055216475046892\\
98	0.00552163184122631\\
99	0.00552161589730922\\
100	0.00552159966787898\\
101	0.00552158314778434\\
102	0.00552156633178012\\
103	0.0055215492145254\\
104	0.00552153179058173\\
105	0.00552151405441141\\
106	0.00552149600037559\\
107	0.0055214776227323\\
108	0.00552145891563467\\
109	0.00552143987312886\\
110	0.00552142048915216\\
111	0.00552140075753084\\
112	0.00552138067197822\\
113	0.00552136022609242\\
114	0.00552133941335427\\
115	0.00552131822712517\\
116	0.00552129666064476\\
117	0.00552127470702868\\
118	0.00552125235926623\\
119	0.00552122961021801\\
120	0.00552120645261356\\
121	0.00552118287904884\\
122	0.0055211588819837\\
123	0.00552113445373936\\
124	0.00552110958649586\\
125	0.00552108427228929\\
126	0.00552105850300921\\
127	0.00552103227039581\\
128	0.00552100556603709\\
129	0.00552097838136615\\
130	0.00552095070765808\\
131	0.00552092253602712\\
132	0.00552089385742359\\
133	0.0055208646626308\\
134	0.00552083494226198\\
135	0.00552080468675704\\
136	0.00552077388637924\\
137	0.00552074253121211\\
138	0.00552071061115585\\
139	0.00552067811592396\\
140	0.00552064503503989\\
141	0.00552061135783322\\
142	0.00552057707343636\\
143	0.00552054217078058\\
144	0.0055205066385925\\
145	0.00552047046539015\\
146	0.00552043363947916\\
147	0.00552039614894883\\
148	0.00552035798166811\\
149	0.00552031912528151\\
150	0.00552027956720514\\
151	0.00552023929462236\\
152	0.00552019829447949\\
153	0.00552015655348165\\
154	0.00552011405808831\\
155	0.00552007079450878\\
156	0.00552002674869782\\
157	0.00551998190635094\\
158	0.00551993625289978\\
159	0.00551988977350748\\
160	0.00551984245306379\\
161	0.00551979427618033\\
162	0.00551974522718557\\
163	0.00551969529011997\\
164	0.0055196444487308\\
165	0.00551959268646715\\
166	0.00551953998647461\\
167	0.00551948633159015\\
168	0.0055194317043366\\
169	0.0055193760869174\\
170	0.00551931946121092\\
171	0.00551926180876504\\
172	0.00551920311079126\\
173	0.00551914334815907\\
174	0.00551908250138998\\
175	0.00551902055065147\\
176	0.00551895747575083\\
177	0.00551889325612899\\
178	0.00551882787085396\\
179	0.0055187612986142\\
180	0.00551869351771199\\
181	0.00551862450605631\\
182	0.00551855424115581\\
183	0.00551848270011119\\
184	0.00551840985960772\\
185	0.00551833569590721\\
186	0.00551826018483992\\
187	0.00551818330179588\\
188	0.00551810502171628\\
189	0.00551802531908424\\
190	0.00551794416791541\\
191	0.00551786154174813\\
192	0.00551777741363352\\
193	0.00551769175612491\\
194	0.00551760454126753\\
195	0.00551751574058748\\
196	0.00551742532508085\\
197	0.00551733326520237\\
198	0.00551723953085419\\
199	0.00551714409137402\\
200	0.00551704691552371\\
201	0.00551694797147738\\
202	0.00551684722680929\\
203	0.00551674464848142\\
204	0.00551664020283073\\
205	0.00551653385555592\\
206	0.00551642557170416\\
207	0.0055163153156571\\
208	0.00551620305111673\\
209	0.00551608874109072\\
210	0.00551597234787748\\
211	0.00551585383305065\\
212	0.00551573315744316\\
213	0.00551561028113103\\
214	0.00551548516341644\\
215	0.00551535776281035\\
216	0.00551522803701485\\
217	0.00551509594290453\\
218	0.00551496143650786\\
219	0.00551482447298749\\
220	0.0055146850066202\\
221	0.00551454299077638\\
222	0.00551439837789852\\
223	0.00551425111947928\\
224	0.00551410116603904\\
225	0.00551394846710242\\
226	0.00551379297117448\\
227	0.00551363462571579\\
228	0.00551347337711732\\
229	0.00551330917067408\\
230	0.00551314195055836\\
231	0.00551297165979225\\
232	0.0055127982402193\\
233	0.00551262163247557\\
234	0.00551244177596003\\
235	0.00551225860880428\\
236	0.00551207206784177\\
237	0.00551188208857648\\
238	0.00551168860515112\\
239	0.00551149155031508\\
240	0.00551129085539215\\
241	0.00551108645024829\\
242	0.00551087826325943\\
243	0.00551066622127962\\
244	0.00551045024961022\\
245	0.00551023027196952\\
246	0.00551000621046425\\
247	0.0055097779855625\\
248	0.00550954551606927\\
249	0.00550930871910496\\
250	0.00550906751008725\\
251	0.00550882180271794\\
252	0.0055085715089747\\
253	0.00550831653910962\\
254	0.00550805680165543\\
255	0.00550779220344061\\
256	0.00550752264961524\\
257	0.00550724804368904\\
258	0.00550696828758372\\
259	0.00550668328170141\\
260	0.00550639292501186\\
261	0.00550609711516044\\
262	0.00550579574859972\\
263	0.00550548872074722\\
264	0.0055051759261724\\
265	0.00550485725881497\\
266	0.00550453261223783\\
267	0.00550420187991627\\
268	0.00550386495556571\\
269	0.00550352173350871\\
270	0.00550317210908153\\
271	0.00550281597907872\\
272	0.00550245324223288\\
273	0.00550208379972421\\
274	0.00550170755571362\\
275	0.00550132441789157\\
276	0.00550093429803821\\
277	0.00550053711259858\\
278	0.00550013278327558\\
279	0.0054997212375169\\
280	0.00549930240779239\\
281	0.00549887620663984\\
282	0.00549844250829231\\
283	0.00549800118495234\\
284	0.00549755210676195\\
285	0.00549709514177206\\
286	0.00549663015591132\\
287	0.00549615701295485\\
288	0.00549567557449205\\
289	0.00549518569989425\\
290	0.00549468724628167\\
291	0.00549418006848997\\
292	0.00549366401903615\\
293	0.00549313894808387\\
294	0.00549260470340838\\
295	0.00549206113036053\\
296	0.00549150807183033\\
297	0.00549094536820982\\
298	0.00549037285735526\\
299	0.00548979037454844\\
300	0.00548919775245742\\
301	0.00548859482109638\\
302	0.00548798140778461\\
303	0.00548735733710453\\
304	0.00548672243085905\\
305	0.00548607650802752\\
306	0.00548541938472111\\
307	0.00548475087413662\\
308	0.00548407078650947\\
309	0.00548337892906525\\
310	0.00548267510597001\\
311	0.00548195911827908\\
312	0.0054812307638845\\
313	0.0054804898374607\\
314	0.00547973613040869\\
315	0.00547896943079811\\
316	0.00547818952330758\\
317	0.0054773961891628\\
318	0.00547658920607254\\
319	0.00547576834816195\\
320	0.00547493338590362\\
321	0.00547408408604557\\
322	0.00547322021153632\\
323	0.00547234152144663\\
324	0.00547144777088796\\
325	0.00547053871092677\\
326	0.00546961408849504\\
327	0.00546867364629624\\
328	0.00546771712270641\\
329	0.00546674425167016\\
330	0.00546575476259082\\
331	0.00546474838021479\\
332	0.0054637248245088\\
333	0.0054626838105302\\
334	0.00546162504828915\\
335	0.00546054824260222\\
336	0.00545945309293683\\
337	0.00545833929324494\\
338	0.00545720653178605\\
339	0.00545605449093745\\
340	0.00545488284699143\\
341	0.00545369126993761\\
342	0.00545247942322917\\
343	0.00545124696353159\\
344	0.00544999354045185\\
345	0.00544871879624645\\
346	0.00544742236550625\\
347	0.00544610387481529\\
348	0.00544476294238176\\
349	0.00544339917763805\\
350	0.00544201218080623\\
351	0.0054406015424265\\
352	0.00543916684284406\\
353	0.00543770765165072\\
354	0.00543622352707626\\
355	0.00543471401532517\\
356	0.0054331786498525\\
357	0.00543161695057351\\
358	0.00543002842300023\\
359	0.00542841255729794\\
360	0.0054267688272538\\
361	0.00542509668914959\\
362	0.00542339558052954\\
363	0.00542166491885402\\
364	0.00541990410002906\\
365	0.00541811249680166\\
366	0.00541628945701016\\
367	0.00541443430167937\\
368	0.00541254632294991\\
369	0.00541062478183212\\
370	0.00540866890577628\\
371	0.00540667788605188\\
372	0.00540465087493226\\
373	0.00540258698268411\\
374	0.00540048527436669\\
375	0.00539834476645274\\
376	0.00539616442329166\\
377	0.00539394315344812\\
378	0.00539167980596377\\
379	0.00538937316660993\\
380	0.00538702195422511\\
381	0.00538462481726917\\
382	0.00538218033078822\\
383	0.00537968699412042\\
384	0.00537714323003581\\
385	0.00537454738714601\\
386	0.00537189775132708\\
387	0.00536919258590505\\
388	0.00536643411745236\\
389	0.00536362814878825\\
390	0.00536077380558616\\
391	0.00535787020184325\\
392	0.00535491644144934\\
393	0.00535191161956328\\
394	0.00534885482403642\\
395	0.0053457451376249\\
396	0.00534258164487129\\
397	0.00533936343356682\\
398	0.00533608959274483\\
399	0.00533275921349833\\
400	0.00532937138914247\\
401	0.00532592521468404\\
402	0.00532241978561147\\
403	0.00531885419614399\\
404	0.0053152275385854\\
405	0.0053115389049386\\
406	0.00530778738568527\\
407	0.00530397206919219\\
408	0.00530009204781425\\
409	0.00529614642026316\\
410	0.00529213429423897\\
411	0.00528805478932205\\
412	0.00528390704009951\\
413	0.00527969019945088\\
414	0.00527540344183438\\
415	0.00527104596630378\\
416	0.00526661699899993\\
417	0.00526211579574787\\
418	0.00525754165017734\\
419	0.00525289393257562\\
420	0.00524817207014708\\
421	0.00524337554790461\\
422	0.00523850391366673\\
423	0.00523355678417246\\
424	0.0052285338562323\\
425	0.00522343493609123\\
426	0.00521826003033973\\
427	0.00521299981013176\\
428	0.00520764154984799\\
429	0.00520218316507152\\
430	0.00519662250553933\\
431	0.00519095735143663\\
432	0.00518518540943985\\
433	0.00517930430786009\\
434	0.00517331159122543\\
435	0.00516720471464439\\
436	0.00516098103757801\\
437	0.00515463781694451\\
438	0.00514817219947595\\
439	0.0051415812132719\\
440	0.00513486175855821\\
441	0.00512801059742921\\
442	0.0051210243424309\\
443	0.00511389944394695\\
444	0.00510663217628548\\
445	0.00509921862234511\\
446	0.00509165465665163\\
447	0.00508393592620462\\
448	0.00507605782727588\\
449	0.00506801547165309\\
450	0.0050598036192859\\
451	0.00505141649510841\\
452	0.0050428475396761\\
453	0.00503409144145339\\
454	0.0050251426509999\\
455	0.00501599536655582\\
456	0.00500664351914084\\
457	0.00499708075683417\\
458	0.00498730042815378\\
459	0.00497729556449069\\
460	0.0049670588615872\\
461	0.00495658266011656\\
462	0.00494585892524127\\
463	0.0049348792251699\\
464	0.00492363470868343\\
465	0.00491211608156425\\
466	0.00490031358175842\\
467	0.00488821695288661\\
468	0.00487581541544252\\
469	0.00486309763559506\\
470	0.00485005169845369\\
471	0.0048366651328826\\
472	0.00482292483719772\\
473	0.00480881700809003\\
474	0.00479432709932754\\
475	0.00477943977858097\\
476	0.0047641388823978\\
477	0.00474840736927814\\
478	0.00473222727046199\\
479	0.00471557963788382\\
480	0.00469844448912965\\
481	0.00468080074667346\\
482	0.0046626261375146\\
483	0.00464389700735168\\
484	0.00462458811794882\\
485	0.0046046727571455\\
486	0.00458412118468098\\
487	0.00456289381669408\\
488	0.00454093863433154\\
489	0.00451821998828621\\
490	0.00449470064864536\\
491	0.00447034196370148\\
492	0.00444510340289893\\
493	0.00441894254081404\\
494	0.00439181506085209\\
495	0.00436367480329994\\
496	0.0043344738555302\\
497	0.00430416269614973\\
498	0.00427269040674317\\
499	0.00424000496854713\\
500	0.00420605366530414\\
501	0.00417078361847794\\
502	0.00413414248746009\\
503	0.00409607937443961\\
504	0.00405654598266416\\
505	0.00401549808831925\\
506	0.00397289740224313\\
507	0.00392871391491754\\
508	0.00388292883819276\\
509	0.00383553825499621\\
510	0.00378655763043521\\
511	0.00373602750432496\\
512	0.00368402082627437\\
513	0.00363065307189473\\
514	0.00357609408201811\\
515	0.00352058662871403\\
516	0.00346443837690163\\
517	0.00340801376574661\\
518	0.00335185318789953\\
519	0.00329683024981812\\
520	0.00324653985142069\\
521	0.00320161003127924\\
522	0.00316207202960114\\
523	0.00312749854380489\\
524	0.00309395292311824\\
525	0.00306108207870875\\
526	0.00302867931543196\\
527	0.00299627525953052\\
528	0.00296378678854382\\
529	0.00293115317034936\\
530	0.00289809271373717\\
531	0.00286441613276463\\
532	0.00283008516840576\\
533	0.00279506205124351\\
534	0.00275930312882306\\
535	0.00272278664427699\\
536	0.00268549207316304\\
537	0.00264739866875622\\
538	0.00260848525570127\\
539	0.00256873024649069\\
540	0.00252811165960377\\
541	0.00248660681497579\\
542	0.00244415586141974\\
543	0.00240076947053404\\
544	0.00235831976722801\\
545	0.00231705559205897\\
546	0.00227650183143643\\
547	0.00223546034490059\\
548	0.00219384736307803\\
549	0.00215161604661321\\
550	0.00210876510575541\\
551	0.00206530472400923\\
552	0.00202124965549764\\
553	0.00197661822767365\\
554	0.00193143270313793\\
555	0.00188571981291925\\
556	0.00183951158664442\\
557	0.00179284685067574\\
558	0.00174575367523509\\
559	0.00169999208744112\\
560	0.00165532117678849\\
561	0.00161033080979638\\
562	0.00156498746152975\\
563	0.00151931648608894\\
564	0.00147334703390769\\
565	0.00142710998207863\\
566	0.00138063755587401\\
567	0.00133396294340406\\
568	0.00128711993625959\\
569	0.00124014252333356\\
570	0.00119341592573713\\
571	0.00114688025971738\\
572	0.00110017097260895\\
573	0.0010533233748797\\
574	0.00100637555384331\\
575	0.000959368345424142\\
576	0.000912345329531775\\
577	0.000865352773721365\\
578	0.000818439509059961\\
579	0.000771656718587823\\
580	0.000725057614040593\\
581	0.000678696970709286\\
582	0.000632630483283935\\
583	0.000586913897120328\\
584	0.000541601859672155\\
585	0.000496746426604042\\
586	0.000452395149029594\\
587	0.000408588670365815\\
588	0.000365357795666473\\
589	0.000322720123375259\\
590	0.000280676711614414\\
591	0.000239312817487403\\
592	0.000198788644961842\\
593	0.000159293651685587\\
594	0.000121079888727805\\
595	8.45520570083912e-05\\
596	5.05092148680371e-05\\
597	2.07908715710836e-05\\
598	0\\
599	0\\
600	0\\
};
\addplot [color=mycolor5,solid,forget plot]
  table[row sep=crcr]{%
1	0.0055565662029982\\
2	0.0055565625234433\\
3	0.00555655877973134\\
4	0.00555655497074048\\
5	0.00555655109532921\\
6	0.00555654715233594\\
7	0.00555654314057877\\
8	0.00555653905885504\\
9	0.00555653490594107\\
10	0.0055565306805916\\
11	0.00555652638153959\\
12	0.00555652200749573\\
13	0.00555651755714814\\
14	0.00555651302916188\\
15	0.00555650842217858\\
16	0.0055565037348161\\
17	0.00555649896566798\\
18	0.00555649411330313\\
19	0.00555648917626525\\
20	0.00555648415307261\\
21	0.00555647904221737\\
22	0.00555647384216524\\
23	0.00555646855135501\\
24	0.00555646316819805\\
25	0.00555645769107788\\
26	0.00555645211834957\\
27	0.00555644644833935\\
28	0.00555644067934405\\
29	0.0055564348096306\\
30	0.00555642883743542\\
31	0.00555642276096407\\
32	0.00555641657839049\\
33	0.00555641028785659\\
34	0.00555640388747161\\
35	0.00555639737531164\\
36	0.00555639074941891\\
37	0.00555638400780123\\
38	0.00555637714843142\\
39	0.00555637016924665\\
40	0.00555636306814791\\
41	0.00555635584299926\\
42	0.00555634849162714\\
43	0.00555634101181982\\
44	0.00555633340132675\\
45	0.00555632565785775\\
46	0.00555631777908234\\
47	0.0055563097626291\\
48	0.00555630160608489\\
49	0.00555629330699411\\
50	0.00555628486285798\\
51	0.00555627627113374\\
52	0.00555626752923394\\
53	0.00555625863452551\\
54	0.00555624958432911\\
55	0.00555624037591822\\
56	0.00555623100651832\\
57	0.00555622147330605\\
58	0.00555621177340834\\
59	0.00555620190390153\\
60	0.00555619186181047\\
61	0.00555618164410758\\
62	0.005556171247712\\
63	0.00555616066948858\\
64	0.0055561499062469\\
65	0.00555613895474034\\
66	0.00555612781166499\\
67	0.00555611647365882\\
68	0.00555610493730038\\
69	0.00555609319910802\\
70	0.00555608125553862\\
71	0.00555606910298657\\
72	0.00555605673778266\\
73	0.00555604415619292\\
74	0.00555603135441743\\
75	0.00555601832858933\\
76	0.00555600507477335\\
77	0.0055559915889648\\
78	0.00555597786708824\\
79	0.00555596390499621\\
80	0.00555594969846801\\
81	0.00555593524320828\\
82	0.00555592053484576\\
83	0.00555590556893193\\
84	0.00555589034093952\\
85	0.00555587484626118\\
86	0.00555585908020805\\
87	0.00555584303800822\\
88	0.00555582671480538\\
89	0.00555581010565713\\
90	0.00555579320553354\\
91	0.00555577600931549\\
92	0.00555575851179316\\
93	0.00555574070766432\\
94	0.00555572259153268\\
95	0.00555570415790615\\
96	0.00555568540119518\\
97	0.00555566631571092\\
98	0.00555564689566349\\
99	0.00555562713516005\\
100	0.00555560702820297\\
101	0.00555558656868798\\
102	0.00555556575040211\\
103	0.0055555445670218\\
104	0.00555552301211084\\
105	0.00555550107911823\\
106	0.00555547876137623\\
107	0.00555545605209811\\
108	0.00555543294437596\\
109	0.00555540943117851\\
110	0.00555538550534878\\
111	0.00555536115960185\\
112	0.00555533638652235\\
113	0.00555531117856221\\
114	0.0055552855280381\\
115	0.00555525942712889\\
116	0.00555523286787309\\
117	0.00555520584216623\\
118	0.00555517834175824\\
119	0.00555515035825068\\
120	0.00555512188309391\\
121	0.00555509290758422\\
122	0.00555506342286103\\
123	0.00555503341990391\\
124	0.00555500288952936\\
125	0.0055549718223879\\
126	0.00555494020896089\\
127	0.00555490803955718\\
128	0.00555487530430996\\
129	0.00555484199317324\\
130	0.0055548080959185\\
131	0.00555477360213113\\
132	0.00555473850120685\\
133	0.005554702782348\\
134	0.00555466643455983\\
135	0.00555462944664663\\
136	0.00555459180720787\\
137	0.005554553504634\\
138	0.00555451452710263\\
139	0.00555447486257415\\
140	0.00555443449878744\\
141	0.00555439342325572\\
142	0.00555435162326185\\
143	0.00555430908585392\\
144	0.00555426579784046\\
145	0.00555422174578586\\
146	0.00555417691600541\\
147	0.00555413129456037\\
148	0.00555408486725289\\
149	0.00555403761962097\\
150	0.00555398953693303\\
151	0.0055539406041827\\
152	0.00555389080608334\\
153	0.00555384012706248\\
154	0.00555378855125606\\
155	0.00555373606250292\\
156	0.00555368264433869\\
157	0.0055536282799901\\
158	0.00555357295236878\\
159	0.00555351664406516\\
160	0.00555345933734259\\
161	0.00555340101413062\\
162	0.00555334165601911\\
163	0.00555328124425156\\
164	0.00555321975971893\\
165	0.00555315718295301\\
166	0.00555309349412002\\
167	0.00555302867301414\\
168	0.00555296269905097\\
169	0.00555289555126099\\
170	0.00555282720828328\\
171	0.00555275764835887\\
172	0.00555268684932458\\
173	0.00555261478860662\\
174	0.00555254144321438\\
175	0.00555246678973436\\
176	0.00555239080432414\\
177	0.00555231346270656\\
178	0.00555223474016394\\
179	0.00555215461153247\\
180	0.00555207305119675\\
181	0.00555199003308438\\
182	0.00555190553066068\\
183	0.00555181951692356\\
184	0.0055517319643983\\
185	0.00555164284513227\\
186	0.00555155213068955\\
187	0.00555145979214541\\
188	0.00555136580008011\\
189	0.00555127012457244\\
190	0.00555117273519229\\
191	0.00555107360099227\\
192	0.00555097269049799\\
193	0.00555086997169687\\
194	0.00555076541202491\\
195	0.00555065897835187\\
196	0.00555055063696405\\
197	0.00555044035354526\\
198	0.00555032809315718\\
199	0.00555021382022778\\
200	0.00555009749853958\\
201	0.00554997909121792\\
202	0.00554985856071892\\
203	0.00554973586881732\\
204	0.00554961097659401\\
205	0.00554948384442358\\
206	0.00554935443196139\\
207	0.00554922269813061\\
208	0.00554908860110916\\
209	0.00554895209831617\\
210	0.0055488131463986\\
211	0.00554867170121735\\
212	0.00554852771783352\\
213	0.00554838115049422\\
214	0.00554823195261837\\
215	0.0055480800767825\\
216	0.00554792547470603\\
217	0.00554776809723678\\
218	0.00554760789433616\\
219	0.00554744481506445\\
220	0.00554727880756594\\
221	0.00554710981905395\\
222	0.00554693779579595\\
223	0.0055467626830986\\
224	0.00554658442529286\\
225	0.00554640296571936\\
226	0.00554621824671349\\
227	0.00554603020959108\\
228	0.00554583879463392\\
229	0.00554564394107592\\
230	0.00554544558708915\\
231	0.0055452436697707\\
232	0.00554503812512965\\
233	0.00554482888807474\\
234	0.00554461589240254\\
235	0.00554439907078644\\
236	0.00554417835476615\\
237	0.00554395367473844\\
238	0.00554372495994837\\
239	0.00554349213848204\\
240	0.00554325513726026\\
241	0.0055430138820335\\
242	0.0055427682973783\\
243	0.00554251830669526\\
244	0.00554226383220837\\
245	0.00554200479496647\\
246	0.00554174111484607\\
247	0.0055414727105564\\
248	0.00554119949964632\\
249	0.00554092139851309\\
250	0.0055406383224134\\
251	0.00554035018547619\\
252	0.00554005690071743\\
253	0.00553975838005681\\
254	0.00553945453433581\\
255	0.00553914527333715\\
256	0.00553883050580524\\
257	0.00553851013946685\\
258	0.00553818408105177\\
259	0.00553785223631224\\
260	0.00553751451004054\\
261	0.00553717080608326\\
262	0.00553682102735139\\
263	0.005536465075824\\
264	0.00553610285254412\\
265	0.00553573425760475\\
266	0.00553535919012243\\
267	0.0055349775481963\\
268	0.00553458922884979\\
269	0.00553419412795238\\
270	0.00553379214011849\\
271	0.00553338315858182\\
272	0.00553296707504204\\
273	0.00553254377948372\\
274	0.00553211315996623\\
275	0.00553167510238606\\
276	0.00553122949021451\\
277	0.00553077620421378\\
278	0.00553031512213377\\
279	0.00552984611838764\\
280	0.0055293690637428\\
281	0.00552888382583964\\
282	0.00552839027033731\\
283	0.0055278882608906\\
284	0.0055273776591266\\
285	0.00552685832462144\\
286	0.00552633011487688\\
287	0.005525792885297\\
288	0.0055252464891646\\
289	0.00552469077761789\\
290	0.00552412559962685\\
291	0.00552355080196984\\
292	0.00552296622920996\\
293	0.00552237172367168\\
294	0.00552176712541721\\
295	0.00552115227222303\\
296	0.00552052699955636\\
297	0.00551989114055175\\
298	0.00551924452598748\\
299	0.00551858698426229\\
300	0.00551791834137189\\
301	0.00551723842088557\\
302	0.00551654704392292\\
303	0.0055158440291304\\
304	0.00551512919265814\\
305	0.00551440234813669\\
306	0.0055136633066536\\
307	0.00551291187673046\\
308	0.00551214786429946\\
309	0.00551137107268013\\
310	0.00551058130255629\\
311	0.0055097783519526\\
312	0.00550896201621118\\
313	0.00550813208796837\\
314	0.00550728835713096\\
315	0.00550643061085276\\
316	0.00550555863351065\\
317	0.00550467220668062\\
318	0.00550377110911339\\
319	0.00550285511670998\\
320	0.00550192400249677\\
321	0.00550097753660009\\
322	0.00550001548622049\\
323	0.00549903761560637\\
324	0.00549804368602693\\
325	0.00549703345574455\\
326	0.00549600667998613\\
327	0.00549496311091374\\
328	0.00549390249759393\\
329	0.00549282458596611\\
330	0.00549172911880951\\
331	0.0054906158357086\\
332	0.00548948447301702\\
333	0.00548833476381949\\
334	0.00548716643789195\\
335	0.00548597922165947\\
336	0.00548477283815144\\
337	0.00548354700695471\\
338	0.00548230144416329\\
339	0.00548103586232541\\
340	0.00547974997038695\\
341	0.0054784434736314\\
342	0.00547711607361598\\
343	0.00547576746810338\\
344	0.0054743973509896\\
345	0.00547300541222673\\
346	0.005471591337741\\
347	0.00547015480934577\\
348	0.00546869550464924\\
349	0.00546721309695638\\
350	0.00546570725516585\\
351	0.0054641776436606\\
352	0.00546262392219336\\
353	0.00546104574576619\\
354	0.00545944276450474\\
355	0.0054578146235275\\
356	0.00545616096281012\\
357	0.00545448141704604\\
358	0.00545277561550345\\
359	0.00545104318188011\\
360	0.00544928373415703\\
361	0.00544749688445256\\
362	0.00544568223887847\\
363	0.00544383939739993\\
364	0.00544196795370227\\
365	0.00544006749506626\\
366	0.00543813760225537\\
367	0.0054361778494178\\
368	0.00543418780400641\\
369	0.00543216702671998\\
370	0.00543011507146868\\
371	0.00542803148536672\\
372	0.00542591580875399\\
373	0.00542376757524806\\
374	0.00542158631182604\\
375	0.00541937153893374\\
376	0.00541712277061681\\
377	0.0054148395146646\\
378	0.00541252127275313\\
379	0.00541016754056764\\
380	0.00540777780787916\\
381	0.00540535155854355\\
382	0.00540288827038902\\
383	0.00540038741496609\\
384	0.00539784845716456\\
385	0.00539527085476927\\
386	0.00539265405805357\\
387	0.00538999750881576\\
388	0.00538730063262149\\
389	0.00538456269424022\\
390	0.00538178267450218\\
391	0.00537895948912136\\
392	0.00537609198820947\\
393	0.00537317895757003\\
394	0.00537021912229371\\
395	0.00536721115319308\\
396	0.00536415367653561\\
397	0.00536104528726815\\
398	0.00535788456594081\\
399	0.00535467009890918\\
400	0.00535140050032142\\
401	0.00534807443324338\\
402	0.00534469062506345\\
403	0.00534124786628992\\
404	0.00533774496506\\
405	0.00533418068466903\\
406	0.00533055374014991\\
407	0.00532686279465179\\
408	0.00532310645558164\\
409	0.00531928327028568\\
410	0.0053153917213833\\
411	0.0053114302217447\\
412	0.00530739710910915\\
413	0.00530329064035132\\
414	0.00529910898541757\\
415	0.00529485022097572\\
416	0.00529051232385152\\
417	0.00528609316436\\
418	0.00528159049964099\\
419	0.00527700196694627\\
420	0.00527232507615669\\
421	0.00526755720374208\\
422	0.00526269558811525\\
423	0.00525773732758168\\
424	0.0052526793845131\\
425	0.00524751860727432\\
426	0.00524225180987519\\
427	0.00523688478169057\\
428	0.00523142625771105\\
429	0.00522587457125705\\
430	0.00522022801988917\\
431	0.00521448486504337\\
432	0.00520864333315186\\
433	0.0052027016247223\\
434	0.00519665791541612\\
435	0.00519051035238131\\
436	0.00518425705670917\\
437	0.00517789612644616\\
438	0.005171425640055\\
439	0.00516484365994742\\
440	0.00515814823576862\\
441	0.00515133741063655\\
442	0.00514440922838081\\
443	0.00513736174019266\\
444	0.00513019301163666\\
445	0.00512290112979097\\
446	0.00511548421024411\\
447	0.00510794040379761\\
448	0.00510026790340343\\
449	0.00509246495436486\\
450	0.00508452987913126\\
451	0.00507646115502242\\
452	0.00506825492819732\\
453	0.00505988116691834\\
454	0.00505133523997187\\
455	0.00504261231016473\\
456	0.00503370731012738\\
457	0.00502461492443078\\
458	0.00501532957139311\\
459	0.00500584538340648\\
460	0.0049961561850006\\
461	0.00498625546799501\\
462	0.00497613636666769\\
463	0.0049657916299069\\
464	0.004955213591057\\
465	0.00494439413520791\\
466	0.00493332466344369\\
467	0.00492199605246461\\
468	0.00491039860409967\\
469	0.00489852196541588\\
470	0.00488635495113334\\
471	0.00487388502444875\\
472	0.00486110016872916\\
473	0.00484798846091057\\
474	0.00483453733968041\\
475	0.00482073356715673\\
476	0.00480656318852167\\
477	0.00479201148947316\\
478	0.00477706295186214\\
479	0.0047617012074535\\
480	0.00474590898969505\\
481	0.0047296680830029\\
482	0.00471295926900949\\
483	0.00469576226956938\\
484	0.00467805567959101\\
485	0.00465981685182614\\
486	0.00464102171153467\\
487	0.00462164470289859\\
488	0.00460165873227493\\
489	0.00458103299112202\\
490	0.00455972502689214\\
491	0.0045376851652642\\
492	0.00451487713810451\\
493	0.00449126302561625\\
494	0.00446680334652431\\
495	0.00444145660496162\\
496	0.00441517925139808\\
497	0.00438792565232686\\
498	0.00435964810079153\\
499	0.00433029686131218\\
500	0.0042998202602825\\
501	0.00426816483262271\\
502	0.00423527553509954\\
503	0.00420109605439452\\
504	0.0041655692281557\\
505	0.00412863760719055\\
506	0.00409024419353347\\
507	0.00405033339713849\\
508	0.00400885226367352\\
509	0.00396575203901821\\
510	0.0039209901516933\\
511	0.00387453271320261\\
512	0.00382635766291394\\
513	0.00377645868634219\\
514	0.00372485008048032\\
515	0.00367157241905168\\
516	0.0036167007092409\\
517	0.00356035948260369\\
518	0.00350273687447986\\
519	0.00344408652000935\\
520	0.00338472437097065\\
521	0.00332502096721129\\
522	0.00326555364965393\\
523	0.00320726580269645\\
524	0.00315410754475045\\
525	0.00310637335838345\\
526	0.00306411528959599\\
527	0.00302688819689068\\
528	0.00299070228713309\\
529	0.00295512081968544\\
530	0.00291990283029932\\
531	0.00288470889343574\\
532	0.00284941466422512\\
533	0.00281394805285373\\
534	0.00277826688557104\\
535	0.00274196152703712\\
536	0.00270497313438378\\
537	0.00266726288515937\\
538	0.0026287901198124\\
539	0.00258951542707469\\
540	0.0025494161112003\\
541	0.00250846903797176\\
542	0.00246665102411426\\
543	0.00242393945402507\\
544	0.00238030871636102\\
545	0.00233567897679858\\
546	0.00229069692309558\\
547	0.00224669736708306\\
548	0.00220394841017211\\
549	0.00216151286075968\\
550	0.00211858016627274\\
551	0.00207506500891316\\
552	0.00203094114340961\\
553	0.00198621792500414\\
554	0.00194091184632872\\
555	0.00189504487825128\\
556	0.00184864360533295\\
557	0.00180173986240179\\
558	0.00175437208562482\\
559	0.00170658633949408\\
560	0.00165892004054196\\
561	0.00161280548965377\\
562	0.00156727794385113\\
563	0.00152145163057755\\
564	0.00147531828604501\\
565	0.0014289079242206\\
566	0.00138225391875487\\
567	0.00133539122235288\\
568	0.00128835585617032\\
569	0.00124118449461087\\
570	0.00119391396189252\\
571	0.00114688118876282\\
572	0.00110017098334024\\
573	0.00105332337801152\\
574	0.00100637555536751\\
575	0.000959368346163701\\
576	0.000912345329873068\\
577	0.000865352773868407\\
578	0.00081843950911817\\
579	0.000771656718608598\\
580	0.000725057614047103\\
581	0.000678696970711006\\
582	0.000632630483284291\\
583	0.000586913897120376\\
584	0.000541601859672158\\
585	0.000496746426604039\\
586	0.000452395149029589\\
587	0.000408588670365811\\
588	0.000365357795666471\\
589	0.000322720123375259\\
590	0.000280676711614414\\
591	0.000239312817487403\\
592	0.00019878864496184\\
593	0.000159293651685586\\
594	0.000121079888727804\\
595	8.45520570083909e-05\\
596	5.05092148680371e-05\\
597	2.07908715710836e-05\\
598	0\\
599	0\\
600	0\\
};
\addplot [color=mycolor6,solid,forget plot]
  table[row sep=crcr]{%
1	0.00564414169708746\\
2	0.00564413625517687\\
3	0.00564413071987849\\
4	0.00564412508958927\\
5	0.00564411936267863\\
6	0.00564411353748807\\
7	0.00564410761233051\\
8	0.00564410158549004\\
9	0.00564409545522124\\
10	0.00564408921974873\\
11	0.00564408287726674\\
12	0.00564407642593847\\
13	0.0056440698638956\\
14	0.00564406318923781\\
15	0.00564405640003209\\
16	0.00564404949431236\\
17	0.00564404247007876\\
18	0.00564403532529714\\
19	0.00564402805789849\\
20	0.00564402066577821\\
21	0.00564401314679571\\
22	0.00564400549877365\\
23	0.00564399771949732\\
24	0.00564398980671406\\
25	0.00564398175813259\\
26	0.00564397357142226\\
27	0.00564396524421251\\
28	0.00564395677409218\\
29	0.00564394815860864\\
30	0.00564393939526732\\
31	0.00564393048153081\\
32	0.00564392141481821\\
33	0.00564391219250438\\
34	0.00564390281191916\\
35	0.00564389327034665\\
36	0.00564388356502433\\
37	0.00564387369314236\\
38	0.00564386365184279\\
39	0.00564385343821856\\
40	0.00564384304931287\\
41	0.00564383248211816\\
42	0.00564382173357545\\
43	0.00564381080057326\\
44	0.00564379967994678\\
45	0.00564378836847701\\
46	0.00564377686288969\\
47	0.00564376515985457\\
48	0.00564375325598425\\
49	0.00564374114783338\\
50	0.00564372883189747\\
51	0.00564371630461208\\
52	0.00564370356235165\\
53	0.00564369060142858\\
54	0.00564367741809204\\
55	0.0056436640085269\\
56	0.0056436503688528\\
57	0.00564363649512285\\
58	0.00564362238332254\\
59	0.00564360802936866\\
60	0.00564359342910799\\
61	0.00564357857831631\\
62	0.00564356347269692\\
63	0.0056435481078796\\
64	0.00564353247941936\\
65	0.00564351658279498\\
66	0.00564350041340793\\
67	0.00564348396658084\\
68	0.00564346723755637\\
69	0.00564345022149561\\
70	0.00564343291347689\\
71	0.00564341530849428\\
72	0.00564339740145613\\
73	0.00564337918718361\\
74	0.00564336066040927\\
75	0.00564334181577543\\
76	0.00564332264783279\\
77	0.0056433031510387\\
78	0.00564328331975563\\
79	0.0056432631482496\\
80	0.00564324263068842\\
81	0.00564322176114014\\
82	0.00564320053357121\\
83	0.0056431789418448\\
84	0.00564315697971908\\
85	0.00564313464084535\\
86	0.00564311191876622\\
87	0.00564308880691379\\
88	0.00564306529860771\\
89	0.00564304138705329\\
90	0.00564301706533944\\
91	0.00564299232643684\\
92	0.00564296716319575\\
93	0.00564294156834404\\
94	0.00564291553448503\\
95	0.00564288905409542\\
96	0.005642862119523\\
97	0.00564283472298454\\
98	0.00564280685656343\\
99	0.00564277851220748\\
100	0.00564274968172658\\
101	0.00564272035679014\\
102	0.00564269052892486\\
103	0.00564266018951221\\
104	0.00564262932978591\\
105	0.00564259794082932\\
106	0.00564256601357286\\
107	0.00564253353879135\\
108	0.00564250050710135\\
109	0.00564246690895835\\
110	0.00564243273465392\\
111	0.0056423979743129\\
112	0.00564236261789058\\
113	0.00564232665516947\\
114	0.00564229007575648\\
115	0.00564225286907973\\
116	0.00564221502438551\\
117	0.00564217653073491\\
118	0.00564213737700059\\
119	0.00564209755186348\\
120	0.00564205704380929\\
121	0.00564201584112511\\
122	0.00564197393189578\\
123	0.00564193130400025\\
124	0.0056418879451079\\
125	0.00564184384267478\\
126	0.00564179898393962\\
127	0.00564175335591998\\
128	0.0056417069454081\\
129	0.00564165973896684\\
130	0.00564161172292541\\
131	0.00564156288337497\\
132	0.00564151320616429\\
133	0.00564146267689518\\
134	0.00564141128091767\\
135	0.00564135900332546\\
136	0.00564130582895089\\
137	0.00564125174235987\\
138	0.00564119672784683\\
139	0.00564114076942943\\
140	0.00564108385084303\\
141	0.00564102595553514\\
142	0.00564096706665966\\
143	0.005640907167071\\
144	0.00564084623931801\\
145	0.00564078426563768\\
146	0.00564072122794868\\
147	0.00564065710784477\\
148	0.00564059188658799\\
149	0.00564052554510143\\
150	0.0056404580639621\\
151	0.0056403894233934\\
152	0.00564031960325727\\
153	0.00564024858304622\\
154	0.00564017634187505\\
155	0.00564010285847223\\
156	0.00564002811117116\\
157	0.00563995207790084\\
158	0.00563987473617662\\
159	0.00563979606309024\\
160	0.00563971603529977\\
161	0.00563963462901919\\
162	0.0056395518200075\\
163	0.00563946758355766\\
164	0.00563938189448501\\
165	0.00563929472711542\\
166	0.00563920605527308\\
167	0.00563911585226782\\
168	0.00563902409088234\\
169	0.00563893074335885\\
170	0.00563883578138554\\
171	0.00563873917608285\\
172	0.00563864089798935\\
173	0.00563854091704757\\
174	0.00563843920258979\\
175	0.00563833572332355\\
176	0.00563823044731771\\
177	0.0056381233419882\\
178	0.00563801437408465\\
179	0.00563790350967728\\
180	0.00563779071414467\\
181	0.00563767595216268\\
182	0.00563755918769461\\
183	0.00563744038398316\\
184	0.00563731950354441\\
185	0.00563719650816463\\
186	0.00563707135889997\\
187	0.00563694401607988\\
188	0.00563681443931492\\
189	0.00563668258750926\\
190	0.00563654841887897\\
191	0.0056364118909759\\
192	0.00563627296071807\\
193	0.00563613158442602\\
194	0.00563598771786444\\
195	0.00563584131628652\\
196	0.00563569233447492\\
197	0.00563554072676165\\
198	0.00563538644697248\\
199	0.00563522944814338\\
200	0.00563506968249567\\
201	0.00563490710142288\\
202	0.00563474165547733\\
203	0.00563457329435654\\
204	0.00563440196688957\\
205	0.00563422762102303\\
206	0.00563405020380734\\
207	0.00563386966138245\\
208	0.00563368593896365\\
209	0.00563349898082722\\
210	0.00563330873029588\\
211	0.00563311512972439\\
212	0.0056329181204847\\
213	0.00563271764295124\\
214	0.00563251363648603\\
215	0.00563230603942387\\
216	0.0056320947890571\\
217	0.00563187982162077\\
218	0.00563166107227743\\
219	0.00563143847510205\\
220	0.00563121196306666\\
221	0.00563098146802532\\
222	0.00563074692069891\\
223	0.00563050825065989\\
224	0.00563026538631715\\
225	0.00563001825490073\\
226	0.00562976678244683\\
227	0.0056295108937827\\
228	0.00562925051251158\\
229	0.00562898556099765\\
230	0.00562871596035131\\
231	0.0056284416304141\\
232	0.00562816248974406\\
233	0.00562787845560098\\
234	0.00562758944393172\\
235	0.0056272953693556\\
236	0.00562699614514988\\
237	0.00562669168323515\\
238	0.00562638189416079\\
239	0.0056260666870904\\
240	0.00562574596978704\\
241	0.00562541964859858\\
242	0.00562508762844261\\
243	0.0056247498127913\\
244	0.00562440610365575\\
245	0.00562405640157011\\
246	0.00562370060557519\\
247	0.00562333861320116\\
248	0.00562297032044985\\
249	0.00562259562177602\\
250	0.00562221441006732\\
251	0.00562182657662356\\
252	0.00562143201113391\\
253	0.00562103060165317\\
254	0.00562062223457563\\
255	0.00562020679460738\\
256	0.00561978416473611\\
257	0.00561935422619877\\
258	0.00561891685844633\\
259	0.0056184719391061\\
260	0.00561801934394091\\
261	0.00561755894680571\\
262	0.00561709061960082\\
263	0.00561661423222263\\
264	0.00561612965251152\\
265	0.00561563674619742\\
266	0.00561513537684337\\
267	0.00561462540578792\\
268	0.00561410669208684\\
269	0.00561357909245545\\
270	0.00561304246121241\\
271	0.00561249665022646\\
272	0.005611941508868\\
273	0.00561137688396609\\
274	0.00561080261977376\\
275	0.00561021855794219\\
276	0.00560962453750519\\
277	0.00560902039487498\\
278	0.00560840596384905\\
279	0.00560778107562823\\
280	0.00560714555884466\\
281	0.00560649923957224\\
282	0.00560584194129897\\
283	0.00560517348489965\\
284	0.00560449368860877\\
285	0.00560380236799386\\
286	0.00560309933592902\\
287	0.00560238440256894\\
288	0.0056016573753236\\
289	0.00560091805883311\\
290	0.00560016625494342\\
291	0.00559940176268257\\
292	0.00559862437823763\\
293	0.00559783389493254\\
294	0.00559703010320651\\
295	0.00559621279059369\\
296	0.0055953817417038\\
297	0.00559453673820359\\
298	0.00559367755879988\\
299	0.00559280397922369\\
300	0.00559191577221573\\
301	0.00559101270751374\\
302	0.00559009455184119\\
303	0.00558916106889795\\
304	0.00558821201935285\\
305	0.00558724716083838\\
306	0.00558626624794766\\
307	0.00558526903223386\\
308	0.00558425526221214\\
309	0.00558322468336484\\
310	0.00558217703814907\\
311	0.00558111206600827\\
312	0.00558002950338697\\
313	0.00557892908374936\\
314	0.00557781053760209\\
315	0.00557667359252137\\
316	0.00557551797318496\\
317	0.00557434340140912\\
318	0.00557314959619112\\
319	0.00557193627375752\\
320	0.00557070314761891\\
321	0.00556944992863151\\
322	0.00556817632506571\\
323	0.00556688204268277\\
324	0.0055655667848199\\
325	0.0055642302524841\\
326	0.00556287214445611\\
327	0.00556149215740428\\
328	0.00556008998601007\\
329	0.00555866532310511\\
330	0.0055572178598215\\
331	0.00555574728575555\\
332	0.00555425328914656\\
333	0.00555273555707142\\
334	0.00555119377565612\\
335	0.00554962763030557\\
336	0.00554803680595303\\
337	0.00554642098733027\\
338	0.00554477985926037\\
339	0.00554311310697427\\
340	0.00554142041645326\\
341	0.00553970147479857\\
342	0.00553795597063039\\
343	0.00553618359451826\\
344	0.00553438403944448\\
345	0.0055325570013031\\
346	0.00553070217943665\\
347	0.00552881927721283\\
348	0.00552690800264369\\
349	0.00552496806904948\\
350	0.00552299919577002\\
351	0.00552100110892575\\
352	0.00551897354223103\\
353	0.00551691623786212\\
354	0.00551482894738188\\
355	0.00551271143272327\\
356	0.00551056346723349\\
357	0.00550838483677973\\
358	0.00550617534091777\\
359	0.00550393479412282\\
360	0.00550166302708257\\
361	0.00549935988804942\\
362	0.00549702524424956\\
363	0.00549465898334295\\
364	0.00549226101492717\\
365	0.00548983127207504\\
366	0.00548736971289305\\
367	0.00548487632208314\\
368	0.00548235111248626\\
369	0.00547979412658058\\
370	0.00547720543789952\\
371	0.00547458515232831\\
372	0.00547193340922715\\
373	0.00546925038231827\\
374	0.0054665362802615\\
375	0.00546379134682676\\
376	0.00546101586055519\\
377	0.00545821013377916\\
378	0.00545537451084841\\
379	0.00545250936538243\\
380	0.00544961509633896\\
381	0.00544669212265489\\
382	0.00544374087617907\\
383	0.00544076179257658\\
384	0.0054377552998406\\
385	0.0054347218039998\\
386	0.0054316616715437\\
387	0.00542857520801726\\
388	0.00542546263241958\\
389	0.00542232405135473\\
390	0.00541915943173047\\
391	0.00541596856053574\\
392	0.00541275099852629\\
393	0.00540950602802968\\
394	0.00540623259578367\\
395	0.00540292925287511\\
396	0.00539959409575478\\
397	0.00539622471454812\\
398	0.00539281815841277\\
399	0.00538937093322477\\
400	0.00538587905482985\\
401	0.00538233819374411\\
402	0.00537874396997767\\
403	0.00537509250059577\\
404	0.00537138136169903\\
405	0.00536760963869151\\
406	0.00536377640670455\\
407	0.00535988073120857\\
408	0.00535592166875451\\
409	0.00535189826786303\\
410	0.00534780957007553\\
411	0.00534365461117891\\
412	0.0053394324226149\\
413	0.00533514203308011\\
414	0.00533078247031827\\
415	0.0053263527630979\\
416	0.00532185194335856\\
417	0.0053172790484941\\
418	0.0053126331237232\\
419	0.00530791322447705\\
420	0.00530311841874507\\
421	0.00529824778923973\\
422	0.00529330043525395\\
423	0.00528827547412149\\
424	0.00528317204227609\\
425	0.00527798929587917\\
426	0.00527272640940693\\
427	0.00526738255837269\\
428	0.00526195660210042\\
429	0.00525644697598015\\
430	0.00525085202010332\\
431	0.00524516996856736\\
432	0.00523939893776977\\
433	0.00523353691359344\\
434	0.00522758173720634\\
435	0.00522153108974944\\
436	0.00521538247611547\\
437	0.00520913320773026\\
438	0.00520278038449278\\
439	0.0051963208761216\\
440	0.00518975130328123\\
441	0.0051830680190074\\
442	0.00517626709101539\\
443	0.00516934428574013\\
444	0.00516229505523642\\
445	0.00515511452830036\\
446	0.00514779750747866\\
447	0.00514033847401999\\
448	0.00513273160350065\\
449	0.00512497079664492\\
450	0.00511704973592452\\
451	0.00510896200076135\\
452	0.00510070378862286\\
453	0.00509229594263602\\
454	0.00508373464752929\\
455	0.00507501624632246\\
456	0.00506613720024991\\
457	0.00505709388351448\\
458	0.00504788255170718\\
459	0.00503849933846256\\
460	0.00502894026034941\\
461	0.00501920122093163\\
462	0.00500927797509956\\
463	0.00499916613887364\\
464	0.00498886118326885\\
465	0.0049783584277492\\
466	0.0049676530324458\\
467	0.00495673999020768\\
468	0.00494561412153549\\
469	0.00493427008262226\\
470	0.00492270241860566\\
471	0.00491090576179196\\
472	0.00489884744474724\\
473	0.00488650547011184\\
474	0.00487386914060477\\
475	0.0048609271690671\\
476	0.00484766764446001\\
477	0.00483407799596508\\
478	0.00482014494740563\\
479	0.004805854470123\\
480	0.00479119173146963\\
481	0.00477614103689769\\
482	0.00476068576490874\\
483	0.00474480829222686\\
484	0.00472848990354356\\
485	0.00471171066811643\\
486	0.00469444922446338\\
487	0.00467668224998897\\
488	0.00465838538681553\\
489	0.00463953438165116\\
490	0.00462010347189625\\
491	0.00460006522725015\\
492	0.00457938821124362\\
493	0.00455802863850385\\
494	0.00453593713953987\\
495	0.00451307676840705\\
496	0.00448940875107804\\
497	0.00446489274774366\\
498	0.00443948628249789\\
499	0.00441314468988377\\
500	0.00438582105601528\\
501	0.00435746622138655\\
502	0.0043280288572434\\
503	0.00429745539131718\\
504	0.00426569009173831\\
505	0.00423267521069181\\
506	0.00419835120334439\\
507	0.00416265704191008\\
508	0.00412553064833676\\
509	0.00408690947531639\\
510	0.00404673127247722\\
511	0.0040049350833286\\
512	0.00396146252888379\\
513	0.00391625944696082\\
514	0.00386927797080916\\
515	0.00382047918193405\\
516	0.00376983645483958\\
517	0.00371733938793335\\
518	0.00366299856508242\\
519	0.0036068522353349\\
520	0.00354898254480159\\
521	0.00348952212899389\\
522	0.00342866087146759\\
523	0.00336665476859641\\
524	0.00330382152996528\\
525	0.00324053311458391\\
526	0.0031773849182677\\
527	0.00311536749133956\\
528	0.00305847479549083\\
529	0.00300703652234188\\
530	0.0029611558647825\\
531	0.00292045584447661\\
532	0.00288119041155718\\
533	0.00284263737455762\\
534	0.00280433019479108\\
535	0.00276613286193941\\
536	0.00272783883980301\\
537	0.00268936526453775\\
538	0.00265065306074691\\
539	0.00261155905477273\\
540	0.00257176095863499\\
541	0.0025312176445629\\
542	0.0024898888746985\\
543	0.00244772979898243\\
544	0.00240470570191919\\
545	0.00236079274626476\\
546	0.00231596758188079\\
547	0.0022701870295366\\
548	0.00222336772233065\\
549	0.0021766009736321\\
550	0.00213084111050196\\
551	0.00208636429435196\\
552	0.00204199580507713\\
553	0.00199714708603788\\
554	0.00195173779208016\\
555	0.00190574754438444\\
556	0.00185919235620512\\
557	0.0018120959308981\\
558	0.00176448917511599\\
559	0.00171640898605244\\
560	0.00166789981490248\\
561	0.00161900323277768\\
562	0.00157077130801954\\
563	0.00152412909216968\\
564	0.00147778367616564\\
565	0.00143118179454771\\
566	0.00138432726713373\\
567	0.00133725430873315\\
568	0.00129000081580688\\
569	0.00124260613446559\\
570	0.00119511043658036\\
571	0.0011475541064327\\
572	0.00110018398429605\\
573	0.00105332347969338\\
574	0.00100637557683123\\
575	0.000959368356499549\\
576	0.000912345335012991\\
577	0.000865352776319937\\
578	0.000818439510213676\\
579	0.000771656719059582\\
580	0.000725057614214845\\
581	0.000678696970765909\\
582	0.000632630483299512\\
583	0.000586913897123733\\
584	0.00054160185967268\\
585	0.000496746426604084\\
586	0.000452395149029592\\
587	0.000408588670365816\\
588	0.000365357795666476\\
589	0.000322720123375263\\
590	0.000280676711614417\\
591	0.000239312817487403\\
592	0.00019878864496184\\
593	0.000159293651685585\\
594	0.000121079888727804\\
595	8.45520570083909e-05\\
596	5.05092148680374e-05\\
597	2.07908715710837e-05\\
598	0\\
599	0\\
600	0\\
};
\addplot [color=mycolor7,solid,forget plot]
  table[row sep=crcr]{%
1	0.00588219123805133\\
2	0.00588218208726686\\
3	0.00588217278113544\\
4	0.00588216331702408\\
5	0.00588215369225527\\
6	0.00588214390410623\\
7	0.00588213394980824\\
8	0.00588212382654572\\
9	0.00588211353145555\\
10	0.00588210306162625\\
11	0.00588209241409704\\
12	0.00588208158585732\\
13	0.00588207057384544\\
14	0.00588205937494817\\
15	0.00588204798599965\\
16	0.00588203640378059\\
17	0.00588202462501728\\
18	0.00588201264638077\\
19	0.00588200046448591\\
20	0.00588198807589044\\
21	0.00588197547709392\\
22	0.00588196266453694\\
23	0.00588194963459998\\
24	0.00588193638360238\\
25	0.00588192290780144\\
26	0.00588190920339136\\
27	0.00588189526650207\\
28	0.00588188109319825\\
29	0.00588186667947823\\
30	0.00588185202127291\\
31	0.00588183711444447\\
32	0.0058818219547854\\
33	0.00588180653801729\\
34	0.00588179085978959\\
35	0.00588177491567842\\
36	0.00588175870118537\\
37	0.00588174221173632\\
38	0.00588172544267996\\
39	0.00588170838928676\\
40	0.00588169104674747\\
41	0.00588167341017188\\
42	0.00588165547458745\\
43	0.00588163723493796\\
44	0.00588161868608199\\
45	0.00588159982279169\\
46	0.00588158063975125\\
47	0.00588156113155537\\
48	0.00588154129270789\\
49	0.00588152111762016\\
50	0.00588150060060958\\
51	0.00588147973589798\\
52	0.00588145851761012\\
53	0.00588143693977191\\
54	0.00588141499630895\\
55	0.00588139268104478\\
56	0.00588136998769915\\
57	0.00588134690988632\\
58	0.0058813234411134\\
59	0.00588129957477834\\
60	0.00588127530416842\\
61	0.00588125062245813\\
62	0.00588122552270752\\
63	0.00588119999786017\\
64	0.00588117404074132\\
65	0.00588114764405584\\
66	0.00588112080038635\\
67	0.00588109350219113\\
68	0.00588106574180199\\
69	0.0058810375114224\\
70	0.00588100880312507\\
71	0.00588097960885001\\
72	0.00588094992040228\\
73	0.00588091972944976\\
74	0.00588088902752087\\
75	0.00588085780600227\\
76	0.00588082605613652\\
77	0.00588079376901975\\
78	0.00588076093559925\\
79	0.00588072754667089\\
80	0.00588069359287686\\
81	0.00588065906470292\\
82	0.00588062395247593\\
83	0.00588058824636128\\
84	0.00588055193636014\\
85	0.00588051501230688\\
86	0.00588047746386625\\
87	0.00588043928053062\\
88	0.00588040045161711\\
89	0.00588036096626489\\
90	0.00588032081343208\\
91	0.00588027998189295\\
92	0.00588023846023473\\
93	0.00588019623685472\\
94	0.00588015329995716\\
95	0.00588010963755001\\
96	0.00588006523744187\\
97	0.00588002008723857\\
98	0.00587997417434008\\
99	0.005879927485937\\
100	0.00587988000900718\\
101	0.0058798317303124\\
102	0.00587978263639473\\
103	0.00587973271357297\\
104	0.00587968194793907\\
105	0.00587963032535454\\
106	0.0058795778314466\\
107	0.00587952445160447\\
108	0.00587947017097546\\
109	0.00587941497446109\\
110	0.00587935884671325\\
111	0.00587930177212999\\
112	0.00587924373485153\\
113	0.00587918471875616\\
114	0.00587912470745592\\
115	0.00587906368429241\\
116	0.0058790016323324\\
117	0.00587893853436342\\
118	0.0058788743728894\\
119	0.00587880913012591\\
120	0.00587874278799578\\
121	0.00587867532812425\\
122	0.00587860673183428\\
123	0.00587853698014171\\
124	0.00587846605375035\\
125	0.00587839393304693\\
126	0.00587832059809621\\
127	0.00587824602863568\\
128	0.00587817020407035\\
129	0.0058780931034675\\
130	0.00587801470555129\\
131	0.0058779349886972\\
132	0.00587785393092651\\
133	0.00587777150990062\\
134	0.00587768770291526\\
135	0.00587760248689474\\
136	0.00587751583838573\\
137	0.00587742773355146\\
138	0.00587733814816538\\
139	0.00587724705760487\\
140	0.00587715443684479\\
141	0.00587706026045104\\
142	0.00587696450257372\\
143	0.00587686713694038\\
144	0.00587676813684901\\
145	0.00587666747516092\\
146	0.00587656512429337\\
147	0.00587646105621216\\
148	0.00587635524242384\\
149	0.00587624765396788\\
150	0.00587613826140854\\
151	0.00587602703482644\\
152	0.00587591394381001\\
153	0.00587579895744645\\
154	0.00587568204431256\\
155	0.00587556317246514\\
156	0.00587544230943093\\
157	0.00587531942219636\\
158	0.00587519447719651\\
159	0.00587506744030385\\
160	0.00587493827681619\\
161	0.00587480695144415\\
162	0.0058746734282978\\
163	0.0058745376708726\\
164	0.00587439964203436\\
165	0.00587425930400317\\
166	0.00587411661833645\\
167	0.00587397154591045\\
168	0.00587382404690054\\
169	0.00587367408075979\\
170	0.00587352160619589\\
171	0.00587336658114587\\
172	0.00587320896274888\\
173	0.00587304870731608\\
174	0.00587288577029793\\
175	0.00587272010624825\\
176	0.0058725516687845\\
177	0.0058723804105444\\
178	0.0058722062831377\\
179	0.00587202923709321\\
180	0.00587184922180032\\
181	0.00587166618544422\\
182	0.00587148007493482\\
183	0.005871290835828\\
184	0.0058710984122396\\
185	0.00587090274675077\\
186	0.00587070378030529\\
187	0.00587050145209808\\
188	0.00587029569945591\\
189	0.00587008645771111\\
190	0.0058698736600702\\
191	0.00586965723748165\\
192	0.00586943711850872\\
193	0.0058692132292203\\
194	0.00586898549312512\\
195	0.0058687538312082\\
196	0.00586851816221758\\
197	0.00586827840359564\\
198	0.00586803447412705\\
199	0.00586778630120221\\
200	0.00586753381139697\\
201	0.00586727693003613\\
202	0.0058670155811731\\
203	0.00586674968756943\\
204	0.00586647917067383\\
205	0.00586620395060098\\
206	0.00586592394611\\
207	0.00586563907458275\\
208	0.00586534925200168\\
209	0.00586505439292744\\
210	0.00586475441047631\\
211	0.00586444921629699\\
212	0.00586413872054743\\
213	0.00586382283187114\\
214	0.00586350145737323\\
215	0.00586317450259602\\
216	0.0058628418714946\\
217	0.00586250346641158\\
218	0.00586215918805206\\
219	0.00586180893545772\\
220	0.00586145260598089\\
221	0.00586109009525804\\
222	0.00586072129718302\\
223	0.00586034610387987\\
224	0.00585996440567515\\
225	0.00585957609106991\\
226	0.00585918104671132\\
227	0.00585877915736358\\
228	0.00585837030587866\\
229	0.00585795437316637\\
230	0.00585753123816403\\
231	0.00585710077780552\\
232	0.0058566628669898\\
233	0.00585621737854895\\
234	0.00585576418321546\\
235	0.005855303149589\\
236	0.00585483414410254\\
237	0.00585435703098766\\
238	0.00585387167223928\\
239	0.00585337792757956\\
240	0.00585287565442103\\
241	0.00585236470782888\\
242	0.00585184494048238\\
243	0.00585131620263546\\
244	0.00585077834207637\\
245	0.00585023120408629\\
246	0.00584967463139692\\
247	0.00584910846414732\\
248	0.00584853253983935\\
249	0.0058479466932923\\
250	0.00584735075659636\\
251	0.00584674455906493\\
252	0.00584612792718618\\
253	0.005845500684573\\
254	0.00584486265191258\\
255	0.0058442136469145\\
256	0.0058435534842581\\
257	0.00584288197553906\\
258	0.00584219892921499\\
259	0.00584150415055066\\
260	0.00584079744156258\\
261	0.00584007860096303\\
262	0.00583934742410421\\
263	0.00583860370292198\\
264	0.00583784722588\\
265	0.0058370777779139\\
266	0.00583629514037624\\
267	0.00583549909098182\\
268	0.00583468940375375\\
269	0.00583386584897045\\
270	0.00583302819311345\\
271	0.00583217619881607\\
272	0.00583130962481249\\
273	0.00583042822588784\\
274	0.00582953175282786\\
275	0.00582861995236874\\
276	0.00582769256714601\\
277	0.00582674933564253\\
278	0.00582578999213452\\
279	0.00582481426663568\\
280	0.00582382188483897\\
281	0.00582281256805669\\
282	0.00582178603316027\\
283	0.00582074199251977\\
284	0.00581968015394315\\
285	0.0058186002206152\\
286	0.00581750189103652\\
287	0.00581638485896217\\
288	0.00581524881334033\\
289	0.00581409343825085\\
290	0.0058129184128441\\
291	0.00581172341127963\\
292	0.00581050810266517\\
293	0.00580927215099572\\
294	0.00580801521509329\\
295	0.0058067369485468\\
296	0.00580543699965241\\
297	0.00580411501135488\\
298	0.00580277062118932\\
299	0.0058014034612239\\
300	0.00580001315800364\\
301	0.00579859933249527\\
302	0.00579716160003328\\
303	0.00579569957026767\\
304	0.00579421284711314\\
305	0.00579270102870023\\
306	0.00579116370732854\\
307	0.00578960046942194\\
308	0.00578801089548685\\
309	0.00578639456007275\\
310	0.00578475103173618\\
311	0.00578307987300787\\
312	0.00578138064036386\\
313	0.00577965288420035\\
314	0.00577789614881355\\
315	0.00577610997238415\\
316	0.00577429388696708\\
317	0.0057724474184875\\
318	0.00577057008674308\\
319	0.00576866140541335\\
320	0.00576672088207668\\
321	0.00576474801823573\\
322	0.00576274230935219\\
323	0.00576070324489125\\
324	0.00575863030837713\\
325	0.00575652297746055\\
326	0.00575438072399893\\
327	0.00575220301415095\\
328	0.00574998930848624\\
329	0.00574773906211215\\
330	0.00574545172481836\\
331	0.00574312674124184\\
332	0.00574076355105323\\
333	0.00573836158916711\\
334	0.00573592028597793\\
335	0.00573343906762434\\
336	0.00573091735628411\\
337	0.00572835457050286\\
338	0.00572575012555948\\
339	0.00572310343387171\\
340	0.00572041390544555\\
341	0.00571768094837294\\
342	0.00571490396938198\\
343	0.00571208237444478\\
344	0.00570921556944856\\
345	0.00570630296093632\\
346	0.00570334395692322\\
347	0.00570033796779702\\
348	0.00569728440731028\\
349	0.00569418269367389\\
350	0.00569103225076202\\
351	0.00568783250943989\\
352	0.00568458290902718\\
353	0.00568128289891095\\
354	0.00567793194032404\\
355	0.00567452950830646\\
356	0.00567107509386924\\
357	0.00566756820638286\\
358	0.00566400837621448\\
359	0.00566039515764186\\
360	0.00565672813207408\\
361	0.00565300691161411\\
362	0.00564923114300126\\
363	0.00564540051197693\\
364	0.00564151474812193\\
365	0.00563757363021977\\
366	0.00563357699220581\\
367	0.00562952472977064\\
368	0.00562541680769257\\
369	0.00562125326798283\\
370	0.00561703423893659\\
371	0.00561275994519176\\
372	0.0056084307189075\\
373	0.00560404701218465\\
374	0.00559960941085824\\
375	0.00559511864980179\\
376	0.00559057562988734\\
377	0.00558598143674747\\
378	0.00558133736148083\\
379	0.00557664492342907\\
380	0.00557190589512597\\
381	0.00556712232947278\\
382	0.00556229658911968\\
383	0.00555743137792129\\
384	0.00555252977416729\\
385	0.00554759526504962\\
386	0.00554263178149162\\
387	0.00553764373200467\\
388	0.00553263603363741\\
389	0.00552761413726744\\
390	0.00552258404403517\\
391	0.00551755230432516\\
392	0.00551252599174899\\
393	0.00550751264045885\\
394	0.00550252012886844\\
395	0.00549755648547518\\
396	0.0054926295808739\\
397	0.00548774667063801\\
398	0.00548291373303975\\
399	0.00547813451476601\\
400	0.00547340916938308\\
401	0.00546873232430192\\
402	0.00546409034107289\\
403	0.00545945746060361\\
404	0.00545479065452195\\
405	0.00545005260128847\\
406	0.00544524264586333\\
407	0.00544036015771497\\
408	0.00543540453319037\\
409	0.00543037519796206\\
410	0.00542527160953924\\
411	0.00542009325982682\\
412	0.00541483967771263\\
413	0.0054095104316578\\
414	0.00540410513226049\\
415	0.00539862343475789\\
416	0.0053930650414277\\
417	0.00538742970385511\\
418	0.00538171722506037\\
419	0.00537592746158062\\
420	0.00537006032535792\\
421	0.00536411578537901\\
422	0.00535809386903167\\
423	0.00535199466314406\\
424	0.00534581831466087\\
425	0.0053395650308661\\
426	0.00533323507899149\\
427	0.00532682878551841\\
428	0.00532034654460782\\
429	0.00531378882878005\\
430	0.00530715618751142\\
431	0.00530044924399701\\
432	0.00529366868970234\\
433	0.00528681527658234\\
434	0.00527988980632675\\
435	0.00527289311561445\\
436	0.00526582605648935\\
437	0.00525868947083078\\
438	0.00525148415773978\\
439	0.00524421083250835\\
440	0.0052368700757049\\
441	0.00522946227082878\\
442	0.00522198752898299\\
443	0.00521444559914102\\
444	0.00520683576294551\\
445	0.00519915671370076\\
446	0.00519140642050471\\
447	0.0051835819805921\\
448	0.00517567946634349\\
449	0.00516769377863308\\
450	0.00515961852592683\\
451	0.005151445958989\\
452	0.00514316699999214\\
453	0.0051347713274009\\
454	0.00512624707099367\\
455	0.00511758257154256\\
456	0.00510876815839775\\
457	0.00509979921378732\\
458	0.0050906715587827\\
459	0.00508138073314177\\
460	0.00507192196861216\\
461	0.00506229016033015\\
462	0.00505247983613435\\
463	0.00504248512515146\\
464	0.00503229972450469\\
465	0.00502191686506552\\
466	0.00501132927698835\\
467	0.00500052915644913\\
468	0.00498950813675828\\
469	0.00497825727254909\\
470	0.0049667670647291\\
471	0.00495502762238985\\
472	0.0049430541617577\\
473	0.00493085021822894\\
474	0.00491840729180021\\
475	0.00490571632057168\\
476	0.00489276766813617\\
477	0.00487955113536917\\
478	0.00486605606432649\\
479	0.00485227131562047\\
480	0.00483818530089091\\
481	0.00482378604549152\\
482	0.00480906125528886\\
483	0.00479399839257722\\
484	0.00477858473533605\\
485	0.00476280739141535\\
486	0.00474665320711469\\
487	0.00473010861408686\\
488	0.00471314021482741\\
489	0.00469570132323347\\
490	0.00467777053771616\\
491	0.00465932506538153\\
492	0.00464034054077538\\
493	0.00462079104619254\\
494	0.00460064903615265\\
495	0.00457988331032009\\
496	0.00455845203088011\\
497	0.00453630139290828\\
498	0.00451339335556852\\
499	0.00448968753890518\\
500	0.00446514170582074\\
501	0.00443971084097204\\
502	0.0044133469625095\\
503	0.0043860022132521\\
504	0.00435762631213689\\
505	0.00432816648945773\\
506	0.00429756746269201\\
507	0.00426577144806584\\
508	0.00423271823782212\\
509	0.0041983453407199\\
510	0.00416258819980435\\
511	0.00412538050824811\\
512	0.00408665464676958\\
513	0.0040463422770657\\
514	0.00400437516965353\\
515	0.00396068628906017\\
516	0.00391521098887378\\
517	0.00386788870705819\\
518	0.00381866515000677\\
519	0.00376749505137355\\
520	0.00371434520876362\\
521	0.0036591985190329\\
522	0.00360205971707007\\
523	0.00354297258640321\\
524	0.00348201672566389\\
525	0.00341931897600911\\
526	0.00335506276957203\\
527	0.00328949729187063\\
528	0.00322293267212333\\
529	0.00315573339194042\\
530	0.00308848525482472\\
531	0.00302213661827374\\
532	0.00296050629345285\\
533	0.00290430468982034\\
534	0.00285370670007441\\
535	0.00280842771725674\\
536	0.00276558846072333\\
537	0.00272363262543577\\
538	0.00268211738834541\\
539	0.00264064638487863\\
540	0.00259910961814965\\
541	0.00255741339571434\\
542	0.00251548061943308\\
543	0.00247326495309667\\
544	0.00243050165550552\\
545	0.00238698086892067\\
546	0.00234265994943839\\
547	0.00229749658658826\\
548	0.00225144245025375\\
549	0.0022044690067269\\
550	0.00215651952593421\\
551	0.00210750156462815\\
552	0.00205873951659417\\
553	0.0020109931608278\\
554	0.00196454992977493\\
555	0.00191823942837985\\
556	0.00187148664771676\\
557	0.00182422356319178\\
558	0.00177642326600128\\
559	0.0017281096045976\\
560	0.00167931427601956\\
561	0.00163007921572574\\
562	0.00158045429977808\\
563	0.00153047646411673\\
564	0.00148151315013013\\
565	0.00143416985390059\\
566	0.00138705937387004\\
567	0.00133975888995393\\
568	0.00129226939301679\\
569	0.0012446284113432\\
570	0.00119687824718637\\
571	0.0011490625211639\\
572	0.00110122520101772\\
573	0.0010534821435598\\
574	0.00100637667104439\\
575	0.000959368504257833\\
576	0.000912345403221071\\
577	0.000865352810900999\\
578	0.000818439527258546\\
579	0.000771656726968869\\
580	0.00072505761760655\\
581	0.000678696972083882\\
582	0.000632630483751538\\
583	0.000586913897255451\\
584	0.000541601859703357\\
585	0.000496746426609176\\
586	0.000452395149030046\\
587	0.000408588670365811\\
588	0.000365357795666471\\
589	0.000322720123375259\\
590	0.000280676711614415\\
591	0.000239312817487403\\
592	0.000198788644961841\\
593	0.000159293651685586\\
594	0.000121079888727804\\
595	8.45520570083908e-05\\
596	5.05092148680372e-05\\
597	2.07908715710836e-05\\
598	0\\
599	0\\
600	0\\
};
\addplot [color=mycolor8,solid,forget plot]
  table[row sep=crcr]{%
1	0.00647624004771694\\
2	0.00647622387146877\\
3	0.00647620742204131\\
4	0.00647619069483262\\
5	0.00647617368516355\\
6	0.00647615638827646\\
7	0.0064761387993338\\
8	0.0064761209134169\\
9	0.00647610272552456\\
10	0.00647608423057166\\
11	0.00647606542338788\\
12	0.00647604629871608\\
13	0.00647602685121104\\
14	0.00647600707543784\\
15	0.0064759869658705\\
16	0.00647596651689038\\
17	0.00647594572278466\\
18	0.00647592457774483\\
19	0.00647590307586496\\
20	0.0064758812111402\\
21	0.00647585897746516\\
22	0.00647583636863205\\
23	0.0064758133783292\\
24	0.00647579000013921\\
25	0.00647576622753725\\
26	0.00647574205388915\\
27	0.00647571747244984\\
28	0.00647569247636121\\
29	0.00647566705865046\\
30	0.00647564121222805\\
31	0.00647561492988589\\
32	0.00647558820429525\\
33	0.00647556102800488\\
34	0.00647553339343885\\
35	0.0064755052928946\\
36	0.00647547671854082\\
37	0.00647544766241516\\
38	0.00647541811642238\\
39	0.00647538807233187\\
40	0.00647535752177552\\
41	0.00647532645624541\\
42	0.00647529486709158\\
43	0.00647526274551955\\
44	0.0064752300825881\\
45	0.00647519686920668\\
46	0.00647516309613301\\
47	0.00647512875397063\\
48	0.00647509383316633\\
49	0.00647505832400752\\
50	0.0064750222166196\\
51	0.00647498550096336\\
52	0.00647494816683225\\
53	0.00647491020384958\\
54	0.00647487160146572\\
55	0.00647483234895532\\
56	0.00647479243541441\\
57	0.00647475184975742\\
58	0.00647471058071415\\
59	0.0064746686168269\\
60	0.00647462594644722\\
61	0.00647458255773297\\
62	0.00647453843864493\\
63	0.00647449357694374\\
64	0.0064744479601865\\
65	0.00647440157572358\\
66	0.00647435441069509\\
67	0.00647430645202757\\
68	0.00647425768643044\\
69	0.00647420810039235\\
70	0.00647415768017784\\
71	0.00647410641182344\\
72	0.00647405428113408\\
73	0.00647400127367924\\
74	0.00647394737478918\\
75	0.00647389256955111\\
76	0.00647383684280508\\
77	0.00647378017914003\\
78	0.00647372256288978\\
79	0.00647366397812885\\
80	0.00647360440866827\\
81	0.00647354383805118\\
82	0.00647348224954876\\
83	0.00647341962615564\\
84	0.00647335595058546\\
85	0.00647329120526639\\
86	0.00647322537233646\\
87	0.00647315843363897\\
88	0.00647309037071768\\
89	0.006473021164812\\
90	0.00647295079685217\\
91	0.0064728792474542\\
92	0.00647280649691494\\
93	0.00647273252520689\\
94	0.00647265731197301\\
95	0.00647258083652153\\
96	0.00647250307782057\\
97	0.00647242401449275\\
98	0.00647234362480961\\
99	0.00647226188668611\\
100	0.00647217877767497\\
101	0.0064720942749609\\
102	0.00647200835535479\\
103	0.00647192099528782\\
104	0.0064718321708054\\
105	0.00647174185756121\\
106	0.006471650030811\\
107	0.00647155666540625\\
108	0.00647146173578796\\
109	0.0064713652159802\\
110	0.00647126707958353\\
111	0.00647116729976853\\
112	0.00647106584926904\\
113	0.0064709627003754\\
114	0.0064708578249275\\
115	0.00647075119430803\\
116	0.00647064277943524\\
117	0.00647053255075598\\
118	0.00647042047823828\\
119	0.00647030653136422\\
120	0.00647019067912238\\
121	0.00647007289000054\\
122	0.00646995313197789\\
123	0.00646983137251749\\
124	0.00646970757855844\\
125	0.0064695817165082\\
126	0.00646945375223438\\
127	0.00646932365105696\\
128	0.00646919137774018\\
129	0.00646905689648426\\
130	0.00646892017091722\\
131	0.00646878116408651\\
132	0.00646863983845066\\
133	0.00646849615587075\\
134	0.00646835007760198\\
135	0.00646820156428492\\
136	0.00646805057593706\\
137	0.0064678970719439\\
138	0.00646774101105035\\
139	0.00646758235135188\\
140	0.00646742105028574\\
141	0.00646725706462213\\
142	0.00646709035045533\\
143	0.0064669208631949\\
144	0.00646674855755685\\
145	0.00646657338755481\\
146	0.00646639530649136\\
147	0.00646621426694923\\
148	0.00646603022078284\\
149	0.00646584311910966\\
150	0.00646565291230193\\
151	0.00646545954997842\\
152	0.00646526298099641\\
153	0.00646506315344396\\
154	0.00646486001463229\\
155	0.00646465351108868\\
156	0.0064644435885495\\
157	0.00646423019195383\\
158	0.00646401326543751\\
159	0.00646379275232764\\
160	0.00646356859513789\\
161	0.00646334073556431\\
162	0.00646310911448212\\
163	0.00646287367194317\\
164	0.00646263434717474\\
165	0.00646239107857922\\
166	0.00646214380373511\\
167	0.00646189245939954\\
168	0.00646163698151231\\
169	0.00646137730520183\\
170	0.00646111336479263\\
171	0.00646084509381557\\
172	0.00646057242501987\\
173	0.00646029529038804\\
174	0.00646001362115334\\
175	0.00645972734782021\\
176	0.00645943640018777\\
177	0.00645914070737652\\
178	0.00645884019785802\\
179	0.00645853479948794\\
180	0.00645822443954161\\
181	0.00645790904475235\\
182	0.00645758854135083\\
183	0.00645726285510531\\
184	0.00645693191136017\\
185	0.00645659563507051\\
186	0.006456253950829\\
187	0.00645590678287998\\
188	0.00645555405511283\\
189	0.00645519569102329\\
190	0.00645483161362372\\
191	0.00645446174526723\\
192	0.00645408600731315\\
193	0.00645370431946377\\
194	0.0064533165983436\\
195	0.0064529227541831\\
196	0.00645252268249495\\
197	0.00645211624207562\\
198	0.00645170319487064\\
199	0.00645128303787878\\
200	0.00645085563272166\\
201	0.0064504208551466\\
202	0.00644997857879551\\
203	0.00644952867516996\\
204	0.00644907101359553\\
205	0.00644860546118575\\
206	0.00644813188280542\\
207	0.00644765014103328\\
208	0.00644716009612396\\
209	0.00644666160596958\\
210	0.00644615452606039\\
211	0.00644563870944506\\
212	0.00644511400669005\\
213	0.0064445802658385\\
214	0.00644403733236834\\
215	0.00644348504914973\\
216	0.00644292325640165\\
217	0.00644235179164807\\
218	0.00644177048967304\\
219	0.00644117918247519\\
220	0.0064405776992215\\
221	0.00643996586620005\\
222	0.00643934350677232\\
223	0.0064387104413243\\
224	0.00643806648721708\\
225	0.00643741145873626\\
226	0.00643674516704073\\
227	0.00643606742011057\\
228	0.00643537802269375\\
229	0.00643467677625234\\
230	0.00643396347890729\\
231	0.00643323792538263\\
232	0.00643249990694855\\
233	0.00643174921136343\\
234	0.00643098562281489\\
235	0.00643020892185986\\
236	0.00642941888536348\\
237	0.00642861528643711\\
238	0.00642779789437499\\
239	0.00642696647459007\\
240	0.00642612078854862\\
241	0.00642526059370366\\
242	0.00642438564342734\\
243	0.00642349568694218\\
244	0.00642259046925103\\
245	0.00642166973106612\\
246	0.00642073320873672\\
247	0.00641978063417571\\
248	0.00641881173478507\\
249	0.00641782623338007\\
250	0.00641682384811246\\
251	0.00641580429239246\\
252	0.00641476727480949\\
253	0.00641371249905202\\
254	0.00641263966382602\\
255	0.00641154846277254\\
256	0.00641043858438417\\
257	0.00640930971192017\\
258	0.00640816152332094\\
259	0.00640699369112094\\
260	0.00640580588236107\\
261	0.00640459775849952\\
262	0.00640336897532169\\
263	0.00640211918284917\\
264	0.00640084802524742\\
265	0.00639955514073233\\
266	0.00639824016147552\\
267	0.00639690271350842\\
268	0.00639554241662507\\
269	0.00639415888428342\\
270	0.00639275172350507\\
271	0.00639132053477362\\
272	0.00638986491193117\\
273	0.00638838444207315\\
274	0.00638687870544136\\
275	0.00638534727531509\\
276	0.00638378971790053\\
277	0.00638220559221794\\
278	0.00638059444998737\\
279	0.00637895583551238\\
280	0.00637728928556208\\
281	0.00637559432925129\\
282	0.00637387048791925\\
283	0.00637211727500619\\
284	0.00637033419592853\\
285	0.00636852074795187\\
286	0.00636667642006241\\
287	0.00636480069283646\\
288	0.00636289303830807\\
289	0.0063609529198348\\
290	0.0063589797919614\\
291	0.00635697310028176\\
292	0.00635493228129887\\
293	0.00635285676228271\\
294	0.00635074596112604\\
295	0.00634859928619845\\
296	0.00634641613619815\\
297	0.00634419590000159\\
298	0.00634193795651116\\
299	0.00633964167450065\\
300	0.00633730641245856\\
301	0.00633493151842919\\
302	0.00633251632985156\\
303	0.00633006017339608\\
304	0.00632756236479882\\
305	0.00632502220869358\\
306	0.00632243899844158\\
307	0.00631981201595874\\
308	0.00631714053154054\\
309	0.00631442380368444\\
310	0.00631166107890984\\
311	0.00630885159157541\\
312	0.00630599456369383\\
313	0.00630308920474416\\
314	0.00630013471148108\\
315	0.00629713026774185\\
316	0.00629407504425036\\
317	0.00629096819841817\\
318	0.00628780887414304\\
319	0.00628459620160424\\
320	0.006281329297055\\
321	0.00627800726261194\\
322	0.00627462918604145\\
323	0.00627119414054285\\
324	0.00626770118452838\\
325	0.00626414936139994\\
326	0.00626053769932257\\
327	0.00625686521099455\\
328	0.00625313089341405\\
329	0.00624933372764253\\
330	0.00624547267856455\\
331	0.00624154669464419\\
332	0.00623755470767811\\
333	0.00623349563254495\\
334	0.00622936836695184\\
335	0.00622517179117711\\
336	0.00622090476781047\\
337	0.0062165661414898\\
338	0.00621215473863561\\
339	0.00620766936718299\\
340	0.00620310881631192\\
341	0.00619847185617595\\
342	0.00619375723763056\\
343	0.00618896369196163\\
344	0.00618408993061516\\
345	0.00617913464492978\\
346	0.00617409650587355\\
347	0.00616897416378686\\
348	0.00616376624813409\\
349	0.00615847136726692\\
350	0.00615308810820268\\
351	0.00614761503642237\\
352	0.00614205069569307\\
353	0.00613639360792149\\
354	0.0061306422730459\\
355	0.0061247951689757\\
356	0.00611885075158952\\
357	0.00611280745480499\\
358	0.00610666369073642\\
359	0.00610041784995903\\
360	0.00609406830190338\\
361	0.00608761339540712\\
362	0.00608105145945854\\
363	0.00607438080417158\\
364	0.00606759972204199\\
365	0.00606070648954351\\
366	0.00605369936913638\\
367	0.00604657661177525\\
368	0.00603933646002237\\
369	0.00603197715189596\\
370	0.00602449692561097\\
371	0.00601689402540503\\
372	0.00600916670868642\\
373	0.0060013132547934\\
374	0.00599333197572258\\
375	0.00598522122926663\\
376	0.00597697943510633\\
377	0.00596860509453244\\
378	0.0059600968146372\\
379	0.00595145333802245\\
380	0.0059426735793323\\
381	0.00593375667024947\\
382	0.00592470201501394\\
383	0.00591550935905417\\
384	0.00590617887399451\\
385	0.00589671126314147\\
386	0.00588710789256086\\
387	0.0058773709539363\\
388	0.00586750366609288\\
389	0.00585751052076736\\
390	0.00584739756964675\\
391	0.00583717286767827\\
392	0.00582684699909627\\
393	0.00581643373854259\\
394	0.00580595090300746\\
395	0.00579542146859583\\
396	0.00578487508013719\\
397	0.00577434973701437\\
398	0.00576389390029359\\
399	0.00575356936161976\\
400	0.00574345493461472\\
401	0.00573365119175744\\
402	0.00572428646084609\\
403	0.00571552405381711\\
404	0.00570756948526384\\
405	0.00570020151858036\\
406	0.00569271599057066\\
407	0.00568511155554668\\
408	0.00567738689906436\\
409	0.00566954074321656\\
410	0.00566157185251424\\
411	0.00565347904043678\\
412	0.00564526117674768\\
413	0.00563691719568569\\
414	0.00562844610515465\\
415	0.00561984699703459\\
416	0.00561111905869675\\
417	0.00560226158565433\\
418	0.00559327399484952\\
419	0.00558415583698435\\
420	0.00557490681356992\\
421	0.00556552679656875\\
422	0.00555601585045792\\
423	0.00554637425700223\\
424	0.0055366025430567\\
425	0.00552670151176883\\
426	0.00551667227767325\\
427	0.0055065163065078\\
428	0.00549623546132956\\
429	0.00548583205295491\\
430	0.0054753088941555\\
431	0.00546466935862506\\
432	0.0054539174425402\\
433	0.00544305782176041\\
434	0.00543209591680835\\
435	0.00542103797090364\\
436	0.0054098911331517\\
437	0.00539866354596811\\
438	0.00538736443515098\\
439	0.00537600420007305\\
440	0.00536459450004577\\
441	0.0053531483307078\\
442	0.00534168008187833\\
443	0.00533020556475153\\
444	0.00531874199109124\\
445	0.00530730788008513\\
446	0.00529592285891626\\
447	0.00528460730995834\\
448	0.00527338179949224\\
449	0.0052622661981414\\
450	0.00525127836917647\\
451	0.0052404322533988\\
452	0.0052297351121516\\
453	0.00521918359548272\\
454	0.00520875817253184\\
455	0.00519841518453535\\
456	0.00518807572061038\\
457	0.00517761143626688\\
458	0.0051670056819433\\
459	0.00515625832843616\\
460	0.00514536935435214\\
461	0.00513433882988236\\
462	0.00512316690726056\\
463	0.00511185380605984\\
464	0.00510039979199753\\
465	0.00508880514764454\\
466	0.00507707013315591\\
467	0.00506519493487367\\
468	0.00505317959946346\\
469	0.00504102395116322\\
470	0.00502872748930082\\
471	0.00501628925885373\\
472	0.00500370765351237\\
473	0.0049909793991094\\
474	0.00497809968549204\\
475	0.00496506227722912\\
476	0.00495185923263095\\
477	0.00493848060039775\\
478	0.00492491410648339\\
479	0.00491114485115891\\
480	0.00489715505779419\\
481	0.00488292393090441\\
482	0.00486842771334284\\
483	0.00485364008282515\\
484	0.00483853311333963\\
485	0.00482307919412716\\
486	0.00480725458589019\\
487	0.00479104299851726\\
488	0.00477444558520379\\
489	0.00475747358007265\\
490	0.00474011146481785\\
491	0.00472234262990966\\
492	0.00470414929833351\\
493	0.00468551243506386\\
494	0.00466641161320032\\
495	0.00464682485774645\\
496	0.00462672867982591\\
497	0.00460609842400275\\
498	0.00458490755158626\\
499	0.00456312446528458\\
500	0.00454069397053237\\
501	0.0045175760745551\\
502	0.00449372477696406\\
503	0.00446905426462991\\
504	0.00444351900407773\\
505	0.00441707207804545\\
506	0.00438966452483288\\
507	0.00436124514501922\\
508	0.00433175993730267\\
509	0.00430115198250383\\
510	0.00426936140119424\\
511	0.0042363253111851\\
512	0.00420197783560677\\
513	0.00416625008663321\\
514	0.00412906987623692\\
515	0.00409036252097511\\
516	0.00405005428318096\\
517	0.00400807037188265\\
518	0.00396433572973611\\
519	0.00391877605958476\\
520	0.00387131917511654\\
521	0.00382189674995734\\
522	0.00377044651661206\\
523	0.00371691446152693\\
524	0.00366125824878971\\
525	0.00360345349321673\\
526	0.00354350693164753\\
527	0.0034814518998612\\
528	0.00341735621285175\\
529	0.00335133319318523\\
530	0.00328355130589948\\
531	0.00321424345000754\\
532	0.00314370305522421\\
533	0.00307227580608966\\
534	0.00300050952196754\\
535	0.00292929211302619\\
536	0.00286178178697941\\
537	0.00279960223019647\\
538	0.00274303581167043\\
539	0.00269196434323486\\
540	0.00264490546750042\\
541	0.00259895181245598\\
542	0.00255370632164094\\
543	0.0025086849952371\\
544	0.00246359209360616\\
545	0.0024183959307389\\
546	0.00237300335466916\\
547	0.00232734162646493\\
548	0.00228137378200997\\
549	0.00223471812268854\\
550	0.00218726473464073\\
551	0.00213896836534945\\
552	0.00208977713704287\\
553	0.00203960532090274\\
554	0.00198836693274954\\
555	0.00193738925710515\\
556	0.00188742405447135\\
557	0.00183875487424147\\
558	0.00179049315303823\\
559	0.00174185184830918\\
560	0.00169278433015013\\
561	0.00164324280767594\\
562	0.0015932603719237\\
563	0.00154287867702145\\
564	0.00149215207571757\\
565	0.00144111678732867\\
566	0.00139123663602677\\
567	0.00134300099050608\\
568	0.00129517846879981\\
569	0.00124726049348067\\
570	0.00119922945498401\\
571	0.00115112545879614\\
572	0.00110299534370064\\
573	0.00105488697841206\\
574	0.00100684692524753\\
575	0.000959381521014027\\
576	0.000912346458482514\\
577	0.000865353251381383\\
578	0.000818439752416602\\
579	0.000771656841517049\\
580	0.000725057672851425\\
581	0.000678696996801308\\
582	0.000632630493807236\\
583	0.000586913900878002\\
584	0.000541601860815896\\
585	0.000496746426883291\\
586	0.000452395149078408\\
587	0.000408588670370421\\
588	0.000365357795666472\\
589	0.00032272012337526\\
590	0.000280676711614416\\
591	0.000239312817487404\\
592	0.000198788644961842\\
593	0.000159293651685588\\
594	0.000121079888727806\\
595	8.45520570083917e-05\\
596	5.05092148680373e-05\\
597	2.07908715710836e-05\\
598	0\\
599	0\\
600	0\\
};
\addplot [color=blue!25!mycolor7,solid,forget plot]
  table[row sep=crcr]{%
1	0.00682152599929207\\
2	0.00682151808048622\\
3	0.00682151002797638\\
4	0.00682150183951254\\
5	0.006821493512807\\
6	0.00682148504553382\\
7	0.00682147643532816\\
8	0.00682146767978563\\
9	0.00682145877646161\\
10	0.00682144972287067\\
11	0.00682144051648577\\
12	0.00682143115473767\\
13	0.00682142163501422\\
14	0.00682141195465958\\
15	0.00682140211097362\\
16	0.00682139210121112\\
17	0.00682138192258097\\
18	0.00682137157224552\\
19	0.00682136104731981\\
20	0.00682135034487067\\
21	0.00682133946191601\\
22	0.00682132839542409\\
23	0.00682131714231252\\
24	0.00682130569944763\\
25	0.00682129406364346\\
26	0.00682128223166108\\
27	0.00682127020020752\\
28	0.00682125796593495\\
29	0.00682124552543993\\
30	0.00682123287526229\\
31	0.00682122001188433\\
32	0.00682120693172984\\
33	0.0068211936311631\\
34	0.00682118010648805\\
35	0.00682116635394711\\
36	0.00682115236972032\\
37	0.00682113814992434\\
38	0.00682112369061126\\
39	0.00682110898776762\\
40	0.00682109403731346\\
41	0.00682107883510114\\
42	0.00682106337691419\\
43	0.00682104765846632\\
44	0.00682103167540016\\
45	0.00682101542328615\\
46	0.00682099889762152\\
47	0.00682098209382884\\
48	0.00682096500725499\\
49	0.00682094763316988\\
50	0.00682092996676526\\
51	0.00682091200315343\\
52	0.00682089373736587\\
53	0.00682087516435213\\
54	0.00682085627897833\\
55	0.00682083707602595\\
56	0.00682081755019037\\
57	0.00682079769607956\\
58	0.00682077750821269\\
59	0.00682075698101862\\
60	0.00682073610883456\\
61	0.00682071488590459\\
62	0.0068206933063781\\
63	0.00682067136430838\\
64	0.00682064905365105\\
65	0.00682062636826243\\
66	0.00682060330189814\\
67	0.00682057984821136\\
68	0.00682055600075125\\
69	0.00682053175296142\\
70	0.00682050709817799\\
71	0.00682048202962821\\
72	0.00682045654042859\\
73	0.00682043062358315\\
74	0.00682040427198172\\
75	0.00682037747839815\\
76	0.00682035023548839\\
77	0.00682032253578886\\
78	0.00682029437171437\\
79	0.00682026573555639\\
80	0.006820236619481\\
81	0.00682020701552719\\
82	0.00682017691560461\\
83	0.00682014631149174\\
84	0.0068201151948338\\
85	0.00682008355714087\\
86	0.00682005138978555\\
87	0.00682001868400113\\
88	0.00681998543087931\\
89	0.00681995162136805\\
90	0.00681991724626948\\
91	0.00681988229623762\\
92	0.00681984676177617\\
93	0.00681981063323628\\
94	0.00681977390081422\\
95	0.00681973655454912\\
96	0.0068196985843205\\
97	0.00681965997984607\\
98	0.00681962073067918\\
99	0.00681958082620655\\
100	0.0068195402556457\\
101	0.00681949900804251\\
102	0.00681945707226868\\
103	0.00681941443701926\\
104	0.00681937109081009\\
105	0.0068193270219751\\
106	0.00681928221866379\\
107	0.00681923666883861\\
108	0.00681919036027223\\
109	0.0068191432805449\\
110	0.00681909541704164\\
111	0.00681904675694962\\
112	0.00681899728725532\\
113	0.00681894699474166\\
114	0.00681889586598545\\
115	0.00681884388735419\\
116	0.00681879104500348\\
117	0.00681873732487395\\
118	0.00681868271268856\\
119	0.00681862719394947\\
120	0.00681857075393517\\
121	0.00681851337769744\\
122	0.00681845505005851\\
123	0.0068183957556079\\
124	0.00681833547869944\\
125	0.0068182742034482\\
126	0.00681821191372749\\
127	0.00681814859316567\\
128	0.00681808422514313\\
129	0.00681801879278911\\
130	0.00681795227897859\\
131	0.00681788466632918\\
132	0.00681781593719792\\
133	0.00681774607367807\\
134	0.00681767505759591\\
135	0.00681760287050757\\
136	0.00681752949369577\\
137	0.0068174549081666\\
138	0.00681737909464619\\
139	0.00681730203357736\\
140	0.00681722370511652\\
141	0.00681714408913013\\
142	0.00681706316519127\\
143	0.00681698091257642\\
144	0.00681689731026185\\
145	0.00681681233692014\\
146	0.00681672597091659\\
147	0.00681663819030566\\
148	0.00681654897282697\\
149	0.00681645829590186\\
150	0.00681636613662908\\
151	0.00681627247178097\\
152	0.00681617727779895\\
153	0.0068160805307893\\
154	0.00681598220651826\\
155	0.00681588228040731\\
156	0.00681578072752773\\
157	0.00681567752259502\\
158	0.00681557263996284\\
159	0.00681546605361657\\
160	0.00681535773716594\\
161	0.00681524766383737\\
162	0.00681513580646522\\
163	0.00681502213748237\\
164	0.00681490662890945\\
165	0.00681478925234313\\
166	0.00681466997894275\\
167	0.00681454877941545\\
168	0.00681442562399916\\
169	0.0068143004824438\\
170	0.00681417332398954\\
171	0.00681404411734251\\
172	0.00681391283064702\\
173	0.00681377943145406\\
174	0.00681364388668554\\
175	0.00681350616259372\\
176	0.00681336622471505\\
177	0.00681322403781779\\
178	0.00681307956584296\\
179	0.00681293277183706\\
180	0.00681278361787651\\
181	0.00681263206498246\\
182	0.00681247807302526\\
183	0.0068123216006177\\
184	0.00681216260499597\\
185	0.00681200104188856\\
186	0.00681183686537187\\
187	0.00681167002771319\\
188	0.00681150047920165\\
189	0.00681132816796832\\
190	0.00681115303979744\\
191	0.00681097503793067\\
192	0.00681079410286415\\
193	0.00681061017213128\\
194	0.0068104231800471\\
195	0.00681023305736157\\
196	0.00681003973076177\\
197	0.0068098431224084\\
198	0.00680964315129954\\
199	0.00680943974594813\\
200	0.00680923284747863\\
201	0.00680902239660873\\
202	0.00680880833307165\\
203	0.00680859059560025\\
204	0.00680836912191124\\
205	0.00680814384868866\\
206	0.00680791471156739\\
207	0.00680768164511646\\
208	0.00680744458282179\\
209	0.00680720345706904\\
210	0.00680695819912599\\
211	0.00680670873912447\\
212	0.00680645500604247\\
213	0.0068061969276855\\
214	0.00680593443066793\\
215	0.0068056674403939\\
216	0.0068053958810381\\
217	0.00680511967552597\\
218	0.00680483874551379\\
219	0.00680455301136853\\
220	0.00680426239214711\\
221	0.00680396680557559\\
222	0.00680366616802786\\
223	0.0068033603945041\\
224	0.00680304939860875\\
225	0.00680273309252847\\
226	0.00680241138700929\\
227	0.00680208419133379\\
228	0.00680175141329776\\
229	0.00680141295918634\\
230	0.00680106873375031\\
231	0.00680071864018141\\
232	0.0068003625800876\\
233	0.00680000045346802\\
234	0.0067996321586873\\
235	0.00679925759244981\\
236	0.00679887664977322\\
237	0.00679848922396195\\
238	0.00679809520658008\\
239	0.006797694487424\\
240	0.00679728695449443\\
241	0.00679687249396846\\
242	0.00679645099017089\\
243	0.00679602232554523\\
244	0.00679558638062453\\
245	0.00679514303400161\\
246	0.006794692162299\\
247	0.00679423364013861\\
248	0.00679376734011086\\
249	0.00679329313274358\\
250	0.00679281088647048\\
251	0.00679232046759924\\
252	0.00679182174027928\\
253	0.00679131456646923\\
254	0.00679079880590386\\
255	0.00679027431606076\\
256	0.00678974095212658\\
257	0.00678919856696295\\
258	0.00678864701107205\\
259	0.00678808613256156\\
260	0.00678751577710937\\
261	0.00678693578792792\\
262	0.00678634600572787\\
263	0.00678574626868148\\
264	0.0067851364123854\\
265	0.00678451626982306\\
266	0.00678388567132639\\
267	0.00678324444453717\\
268	0.00678259241436765\\
269	0.00678192940296065\\
270	0.00678125522964914\\
271	0.00678056971091495\\
272	0.00677987266034734\\
273	0.00677916388860037\\
274	0.00677844320335\\
275	0.00677771040925065\\
276	0.00677696530789062\\
277	0.00677620769774744\\
278	0.00677543737414224\\
279	0.00677465412919359\\
280	0.00677385775177052\\
281	0.00677304802744507\\
282	0.00677222473844398\\
283	0.00677138766359968\\
284	0.0067705365783005\\
285	0.00676967125444025\\
286	0.00676879146036689\\
287	0.00676789696083024\\
288	0.00676698751692908\\
289	0.00676606288605736\\
290	0.00676512282184922\\
291	0.00676416707412344\\
292	0.00676319538882644\\
293	0.00676220750797481\\
294	0.00676120316959627\\
295	0.00676018210766969\\
296	0.00675914405206401\\
297	0.00675808872847585\\
298	0.00675701585836574\\
299	0.00675592515889312\\
300	0.00675481634284983\\
301	0.00675368911859217\\
302	0.00675254318997126\\
303	0.00675137825626169\\
304	0.00675019401208859\\
305	0.00674899014735249\\
306	0.00674776634715239\\
307	0.00674652229170682\\
308	0.00674525765627203\\
309	0.00674397211105854\\
310	0.00674266532114452\\
311	0.00674133694638666\\
312	0.00673998664132796\\
313	0.00673861405510251\\
314	0.00673721883133672\\
315	0.00673580060804694\\
316	0.00673435901753325\\
317	0.00673289368626896\\
318	0.00673140423478562\\
319	0.0067298902775532\\
320	0.0067283514228549\\
321	0.00672678727265644\\
322	0.006725197422469\\
323	0.00672358146120589\\
324	0.00672193897103177\\
325	0.00672026952720406\\
326	0.00671857269790636\\
327	0.00671684804407221\\
328	0.00671509511919944\\
329	0.00671331346915329\\
330	0.00671150263195827\\
331	0.00670966213757684\\
332	0.00670779150767453\\
333	0.00670589025536974\\
334	0.00670395788496691\\
335	0.00670199389167166\\
336	0.00669999776128625\\
337	0.00669796896988296\\
338	0.00669590698345397\\
339	0.00669381125753483\\
340	0.00669168123679933\\
341	0.00668951635462255\\
342	0.00668731603260905\\
343	0.00668507968008221\\
344	0.0066828066935313\\
345	0.00668049645601082\\
346	0.0066781483364877\\
347	0.00667576168913012\\
348	0.00667333585253143\\
349	0.00667087014886187\\
350	0.00666836388293925\\
351	0.00666581634120933\\
352	0.00666322679062493\\
353	0.00666059447741077\\
354	0.0066579186257001\\
355	0.00665519843602668\\
356	0.00665243308365263\\
357	0.00664962171671082\\
358	0.00664676345413608\\
359	0.00664385738335635\\
360	0.00664090255770907\\
361	0.00663789799354379\\
362	0.006634842666964\\
363	0.00663173551015453\\
364	0.00662857540723024\\
365	0.00662536118953125\\
366	0.00662209163027598\\
367	0.00661876543846735\\
368	0.00661538125192753\\
369	0.00661193762931299\\
370	0.00660843304093303\\
371	0.00660486585816018\\
372	0.00660123434117727\\
373	0.00659753662475583\\
374	0.00659377070169413\\
375	0.00658993440346675\\
376	0.00658602537753923\\
377	0.0065820410606818\\
378	0.00657797864746696\\
379	0.00657383505295076\\
380	0.00656960686830603\\
381	0.00656529030789022\\
382	0.00656088114587788\\
383	0.00655637464017199\\
384	0.006551765440869\\
385	0.00654704748027803\\
386	0.00654221384207036\\
387	0.00653725661091762\\
388	0.00653216671896052\\
389	0.0065269338587564\\
390	0.00652154671652445\\
391	0.00651599095147768\\
392	0.00651024932391503\\
393	0.00650430118488542\\
394	0.00649812145499418\\
395	0.00649167893127105\\
396	0.00648493299469873\\
397	0.00647783680310319\\
398	0.00647033688183818\\
399	0.00646236779727463\\
400	0.00645384908975378\\
401	0.00644468139013384\\
402	0.00643474159157995\\
403	0.00642387719790623\\
404	0.00641190104780991\\
405	0.00639903657200411\\
406	0.00638596078425052\\
407	0.00637267027765513\\
408	0.00635916159825583\\
409	0.00634543124535949\\
410	0.00633147567191987\\
411	0.0063172912849489\\
412	0.00630287444596532\\
413	0.00628822147152613\\
414	0.00627332863398959\\
415	0.0062581921629295\\
416	0.00624280824829613\\
417	0.00622717304810649\\
418	0.00621128270763228\\
419	0.0061951334073186\\
420	0.00617872130624063\\
421	0.00616204253850911\\
422	0.00614509322036607\\
423	0.00612786945910534\\
424	0.0061103673640725\\
425	0.00609258305964295\\
426	0.00607451269871913\\
427	0.00605615247067725\\
428	0.00603749858282038\\
429	0.00601854731459214\\
430	0.00599929511417933\\
431	0.00597973868856861\\
432	0.00595987516664406\\
433	0.00593970243760798\\
434	0.00591921918218595\\
435	0.00589842481180852\\
436	0.0058773196493913\\
437	0.00585590515351625\\
438	0.00583418419697106\\
439	0.00581216141364589\\
440	0.00578984363321967\\
441	0.00576724043148619\\
442	0.00574436482077435\\
443	0.00572123411658393\\
444	0.00569787103223775\\
445	0.00567430506515952\\
446	0.00565057425709745\\
447	0.00562672743507575\\
448	0.00560282707190043\\
449	0.00557895294706662\\
450	0.00555520684395962\\
451	0.00553171859080939\\
452	0.00550865384334736\\
453	0.00548622411073188\\
454	0.00546469960050526\\
455	0.00544442531408781\\
456	0.0054258396714864\\
457	0.00540948983742446\\
458	0.00539323260504457\\
459	0.00537676311249133\\
460	0.00536008469878835\\
461	0.00534320198965536\\
462	0.00532612059192507\\
463	0.00530884720770051\\
464	0.00529138975884361\\
465	0.00527375752160495\\
466	0.00525596127074597\\
467	0.00523801343165346\\
468	0.00521992823775862\\
469	0.00520172188888132\\
470	0.00518341270376497\\
471	0.00516502125709558\\
472	0.0051465704860549\\
473	0.00512808576765448\\
474	0.0051095948999598\\
475	0.00509112794187953\\
476	0.00507271685943956\\
477	0.00505439488960137\\
478	0.00503619549586247\\
479	0.00501815076360548\\
480	0.0050002889997187\\
481	0.00498263121570112\\
482	0.00496518604990143\\
483	0.0049479424870087\\
484	0.00493085946334262\\
485	0.00491385109885935\\
486	0.00489676634709877\\
487	0.00487941046645583\\
488	0.00486177900139084\\
489	0.00484386633386189\\
490	0.00482566519217253\\
491	0.00480716726418363\\
492	0.00478836297873783\\
493	0.00476924124746257\\
494	0.00474978916359166\\
495	0.00472999165560429\\
496	0.00470983108666331\\
497	0.00468928677138484\\
498	0.0046683344028814\\
499	0.00464694545929378\\
500	0.00462508734092348\\
501	0.00460272350259559\\
502	0.00457982238022318\\
503	0.00455638069580096\\
504	0.00453233403861286\\
505	0.00450760479277492\\
506	0.00448213612080047\\
507	0.00445587381878527\\
508	0.00442877360691682\\
509	0.00440078970698855\\
510	0.00437187371198864\\
511	0.00434197524437358\\
512	0.00431104092335962\\
513	0.00427901432034292\\
514	0.00424583630966117\\
515	0.0042114290035927\\
516	0.00417567760871252\\
517	0.00413850706469992\\
518	0.00409983938562309\\
519	0.00405959407839367\\
520	0.00401768872169933\\
521	0.0039740397359373\\
522	0.00392856330490904\\
523	0.00388117643313738\\
524	0.00383179797078545\\
525	0.00378034940225419\\
526	0.0037267559301864\\
527	0.0036709496971085\\
528	0.00361287876217814\\
529	0.00355251646179829\\
530	0.00348985500560026\\
531	0.00342491096846352\\
532	0.00335773241447154\\
533	0.00328840965250917\\
534	0.00321708462727485\\
535	0.00314395991947484\\
536	0.00306929731610841\\
537	0.00299340958820004\\
538	0.0029167674290013\\
539	0.00284013624520599\\
540	0.00276555084315231\\
541	0.0026960778195408\\
542	0.00263212286836089\\
543	0.00257378503688891\\
544	0.00252085379215805\\
545	0.00247009708931051\\
546	0.00242037269702207\\
547	0.00237125755908936\\
548	0.00232224243204798\\
549	0.00227316453779023\\
550	0.00222398371594873\\
551	0.00217460623235058\\
552	0.00212495896211347\\
553	0.00207497488170052\\
554	0.00202421243490735\\
555	0.00197262615657274\\
556	0.0019201375163468\\
557	0.00186664188289284\\
558	0.00181318150960817\\
559	0.00176071773665684\\
560	0.00170951692853472\\
561	0.00165928064744219\\
562	0.00160875603182242\\
563	0.00155792315883206\\
564	0.00150671918854937\\
565	0.00145517384907741\\
566	0.00140334093225183\\
567	0.00135125632661799\\
568	0.00130024976398429\\
569	0.00125090910566476\\
570	0.00120241934549694\\
571	0.00115396094926996\\
572	0.00110548107662132\\
573	0.00105702283547423\\
574	0.00100863570447616\\
575	0.000960371942245695\\
576	0.000912495014597589\\
577	0.000865361483000015\\
578	0.000818442575994003\\
579	0.000771658262692563\\
580	0.000725058416501592\\
581	0.000678697369762489\\
582	0.000632630668187501\\
583	0.000586913975324799\\
584	0.000541601889067355\\
585	0.000496746436058626\\
586	0.000452395151479575\\
587	0.000408588670822602\\
588	0.000365357795712724\\
589	0.000322720123375259\\
590	0.000280676711614414\\
591	0.000239312817487402\\
592	0.000198788644961838\\
593	0.000159293651685583\\
594	0.000121079888727803\\
595	8.45520570083905e-05\\
596	5.05092148680371e-05\\
597	2.07908715710836e-05\\
598	0\\
599	0\\
600	0\\
};
\addplot [color=mycolor9,solid,forget plot]
  table[row sep=crcr]{%
1	0.00701212116330883\\
2	0.0070121158585532\\
3	0.00701211046433695\\
4	0.00701210497915749\\
5	0.00701209940148715\\
6	0.0070120937297728\\
7	0.00701208796243541\\
8	0.00701208209786965\\
9	0.00701207613444345\\
10	0.00701207007049751\\
11	0.00701206390434495\\
12	0.00701205763427078\\
13	0.00701205125853145\\
14	0.00701204477535449\\
15	0.00701203818293786\\
16	0.00701203147944953\\
17	0.00701202466302706\\
18	0.00701201773177704\\
19	0.00701201068377461\\
20	0.00701200351706295\\
21	0.00701199622965271\\
22	0.00701198881952148\\
23	0.00701198128461338\\
24	0.00701197362283837\\
25	0.00701196583207182\\
26	0.0070119579101537\\
27	0.00701194985488829\\
28	0.00701194166404353\\
29	0.00701193333535033\\
30	0.00701192486650201\\
31	0.00701191625515377\\
32	0.007011907498922\\
33	0.00701189859538364\\
34	0.00701188954207558\\
35	0.00701188033649402\\
36	0.00701187097609382\\
37	0.00701186145828771\\
38	0.00701185178044581\\
39	0.00701184193989485\\
40	0.00701183193391744\\
41	0.00701182175975135\\
42	0.00701181141458891\\
43	0.00701180089557619\\
44	0.00701179019981224\\
45	0.00701177932434842\\
46	0.00701176826618755\\
47	0.00701175702228318\\
48	0.00701174558953885\\
49	0.00701173396480721\\
50	0.00701172214488927\\
51	0.00701171012653347\\
52	0.0070116979064351\\
53	0.00701168548123514\\
54	0.00701167284751962\\
55	0.00701166000181868\\
56	0.00701164694060568\\
57	0.00701163366029628\\
58	0.00701162015724765\\
59	0.00701160642775744\\
60	0.00701159246806282\\
61	0.00701157827433967\\
62	0.00701156384270145\\
63	0.00701154916919833\\
64	0.00701153424981618\\
65	0.00701151908047565\\
66	0.00701150365703097\\
67	0.00701148797526907\\
68	0.00701147203090845\\
69	0.00701145581959819\\
70	0.00701143933691689\\
71	0.0070114225783714\\
72	0.00701140553939595\\
73	0.00701138821535091\\
74	0.00701137060152166\\
75	0.0070113526931174\\
76	0.00701133448527019\\
77	0.00701131597303348\\
78	0.00701129715138104\\
79	0.0070112780152059\\
80	0.00701125855931881\\
81	0.00701123877844722\\
82	0.00701121866723396\\
83	0.00701119822023592\\
84	0.00701117743192289\\
85	0.00701115629667599\\
86	0.00701113480878663\\
87	0.00701111296245492\\
88	0.00701109075178841\\
89	0.00701106817080076\\
90	0.00701104521341026\\
91	0.00701102187343835\\
92	0.00701099814460837\\
93	0.00701097402054388\\
94	0.00701094949476734\\
95	0.00701092456069852\\
96	0.00701089921165302\\
97	0.00701087344084075\\
98	0.00701084724136438\\
99	0.00701082060621768\\
100	0.00701079352828397\\
101	0.00701076600033452\\
102	0.00701073801502681\\
103	0.00701070956490305\\
104	0.00701068064238827\\
105	0.00701065123978876\\
106	0.00701062134929025\\
107	0.00701059096295618\\
108	0.00701056007272589\\
109	0.0070105286704128\\
110	0.0070104967477027\\
111	0.00701046429615158\\
112	0.00701043130718397\\
113	0.00701039777209098\\
114	0.00701036368202821\\
115	0.00701032902801386\\
116	0.00701029380092672\\
117	0.00701025799150396\\
118	0.0070102215903391\\
119	0.00701018458787989\\
120	0.00701014697442606\\
121	0.0070101087401271\\
122	0.00701006987497999\\
123	0.00701003036882673\\
124	0.00700999021135221\\
125	0.0070099493920815\\
126	0.00700990790037743\\
127	0.00700986572543801\\
128	0.00700982285629382\\
129	0.00700977928180526\\
130	0.00700973499065963\\
131	0.00700968997136834\\
132	0.00700964421226397\\
133	0.00700959770149705\\
134	0.00700955042703295\\
135	0.00700950237664855\\
136	0.00700945353792868\\
137	0.00700940389826273\\
138	0.00700935344484077\\
139	0.00700930216464976\\
140	0.00700925004446939\\
141	0.00700919707086783\\
142	0.00700914323019746\\
143	0.00700908850858996\\
144	0.00700903289195151\\
145	0.00700897636595761\\
146	0.00700891891604771\\
147	0.00700886052741928\\
148	0.00700880118502204\\
149	0.00700874087355126\\
150	0.00700867957744124\\
151	0.00700861728085784\\
152	0.0070085539676912\\
153	0.00700848962154737\\
154	0.00700842422573996\\
155	0.00700835776328083\\
156	0.0070082902168706\\
157	0.00700822156888834\\
158	0.00700815180138065\\
159	0.00700808089604997\\
160	0.00700800883424242\\
161	0.00700793559693475\\
162	0.00700786116472039\\
163	0.00700778551779494\\
164	0.00700770863594062\\
165	0.00700763049850998\\
166	0.00700755108440853\\
167	0.00700747037207685\\
168	0.00700738833947156\\
169	0.00700730496404541\\
170	0.00700722022272688\\
171	0.00700713409189867\\
172	0.0070070465473757\\
173	0.00700695756438277\\
174	0.0070068671175316\\
175	0.00700677518079791\\
176	0.0070066817274985\\
177	0.00700658673026886\\
178	0.00700649016104153\\
179	0.00700639199102593\\
180	0.00700629219068988\\
181	0.00700619072974375\\
182	0.00700608757712771\\
183	0.00700598270100331\\
184	0.00700587606875001\\
185	0.00700576764696783\\
186	0.00700565740148734\\
187	0.00700554529738814\\
188	0.007005431299027\\
189	0.00700531537007689\\
190	0.00700519747357758\\
191	0.00700507757199851\\
192	0.00700495562731353\\
193	0.00700483160108602\\
194	0.00700470545456245\\
195	0.00700457714877309\\
196	0.00700444664464774\\
197	0.0070043139031749\\
198	0.00700417888562068\\
199	0.00700404155337168\\
200	0.00700390186720293\\
201	0.00700375978723681\\
202	0.00700361527293231\\
203	0.00700346828307445\\
204	0.00700331877576297\\
205	0.00700316670840151\\
206	0.00700301203768601\\
207	0.00700285471959327\\
208	0.00700269470936933\\
209	0.0070025319615174\\
210	0.0070023664297858\\
211	0.00700219806715577\\
212	0.00700202682582877\\
213	0.00700185265721398\\
214	0.00700167551191515\\
215	0.0070014953397177\\
216	0.00700131208957524\\
217	0.00700112570959605\\
218	0.0070009361470293\\
219	0.00700074334825093\\
220	0.00700054725874964\\
221	0.00700034782311216\\
222	0.00700014498500878\\
223	0.0069999386871782\\
224	0.00699972887141246\\
225	0.00699951547854149\\
226	0.00699929844841746\\
227	0.00699907771989871\\
228	0.00699885323083372\\
229	0.00699862491804464\\
230	0.0069983927173104\\
231	0.0069981565633499\\
232	0.00699791638980476\\
233	0.00699767212922168\\
234	0.0069974237130348\\
235	0.00699717107154743\\
236	0.00699691413391379\\
237	0.00699665282812034\\
238	0.0069963870809668\\
239	0.00699611681804689\\
240	0.00699584196372884\\
241	0.0069955624411354\\
242	0.00699527817212372\\
243	0.0069949890772649\\
244	0.00699469507582308\\
245	0.00699439608573423\\
246	0.00699409202358484\\
247	0.00699378280458979\\
248	0.00699346834257033\\
249	0.00699314854993146\\
250	0.00699282333763886\\
251	0.00699249261519567\\
252	0.00699215629061865\\
253	0.006991814270414\\
254	0.00699146645955284\\
255	0.00699111276144603\\
256	0.00699075307791886\\
257	0.00699038730918494\\
258	0.00699001535381969\\
259	0.00698963710873354\\
260	0.00698925246914425\\
261	0.00698886132854892\\
262	0.00698846357869551\\
263	0.00698805910955341\\
264	0.0069876478092838\\
265	0.0069872295642091\\
266	0.00698680425878206\\
267	0.00698637177555383\\
268	0.00698593199514149\\
269	0.00698548479619506\\
270	0.00698503005536338\\
271	0.00698456764725966\\
272	0.00698409744442576\\
273	0.00698361931729606\\
274	0.00698313313416024\\
275	0.00698263876112519\\
276	0.00698213606207617\\
277	0.00698162489863678\\
278	0.00698110513012813\\
279	0.00698057661352673\\
280	0.00698003920342169\\
281	0.00697949275197032\\
282	0.00697893710885287\\
283	0.00697837212122592\\
284	0.00697779763367445\\
285	0.00697721348816269\\
286	0.00697661952398332\\
287	0.00697601557770547\\
288	0.00697540148312096\\
289	0.00697477707118888\\
290	0.00697414216997879\\
291	0.00697349660461161\\
292	0.00697284019719909\\
293	0.00697217276678098\\
294	0.00697149412926034\\
295	0.0069708040973366\\
296	0.00697010248043605\\
297	0.00696938908464039\\
298	0.00696866371261231\\
299	0.00696792616351857\\
300	0.00696717623295016\\
301	0.00696641371283952\\
302	0.00696563839137457\\
303	0.00696485005290942\\
304	0.00696404847787139\\
305	0.0069632334426646\\
306	0.00696240471956947\\
307	0.00696156207663779\\
308	0.006960705277584\\
309	0.00695983408167112\\
310	0.00695894824359223\\
311	0.00695804751334643\\
312	0.00695713163610941\\
313	0.00695620035209805\\
314	0.00695525339642884\\
315	0.00695429049896952\\
316	0.00695331138418405\\
317	0.00695231577096971\\
318	0.00695130337248641\\
319	0.00695027389597768\\
320	0.00694922704258255\\
321	0.00694816250713802\\
322	0.00694707997797123\\
323	0.00694597913668085\\
324	0.00694485965790696\\
325	0.00694372120908883\\
326	0.00694256345020909\\
327	0.00694138603352449\\
328	0.00694018860328137\\
329	0.0069389707954152\\
330	0.0069377322372331\\
331	0.00693647254707809\\
332	0.00693519133397384\\
333	0.0069338881972483\\
334	0.00693256272613508\\
335	0.00693121449935059\\
336	0.00692984308464523\\
337	0.00692844803832713\\
338	0.00692702890475551\\
339	0.00692558521580226\\
340	0.00692411649027866\\
341	0.00692262223332462\\
342	0.00692110193575778\\
343	0.00691955507337899\\
344	0.00691798110623043\\
345	0.00691637947780302\\
346	0.00691474961418832\\
347	0.00691309092317046\\
348	0.00691140279325312\\
349	0.00690968459261531\\
350	0.00690793566799061\\
351	0.00690615534346191\\
352	0.00690434291916476\\
353	0.00690249766989062\\
354	0.00690061884358072\\
355	0.00689870565969982\\
356	0.00689675730747893\\
357	0.00689477294401382\\
358	0.00689275169220508\\
359	0.00689069263852404\\
360	0.00688859483058697\\
361	0.00688645727451776\\
362	0.0068842789320776\\
363	0.00688205871753681\\
364	0.00687979549426217\\
365	0.00687748807098947\\
366	0.00687513519774751\\
367	0.00687273556139631\\
368	0.00687028778073843\\
369	0.006867790401157\\
370	0.00686524188873027\\
371	0.00686264062376682\\
372	0.00685998489370048\\
373	0.00685727288527866\\
374	0.00685450267597319\\
375	0.00685167222453705\\
376	0.00684877936062798\\
377	0.00684582177341692\\
378	0.0068427969991017\\
379	0.00683970240724956\\
380	0.0068365351859054\\
381	0.0068332923254236\\
382	0.00682997060101846\\
383	0.0068265665540891\\
384	0.00682307647247982\\
385	0.00681949637001059\\
386	0.00681582196590169\\
387	0.00681204866509814\\
388	0.00680817154044249\\
389	0.00680418531386084\\
390	0.00680008431010625\\
391	0.00679586243472893\\
392	0.00679151321162811\\
393	0.00678702981014269\\
394	0.00678240509034287\\
395	0.00677763170285273\\
396	0.00677270236258949\\
397	0.00676761019656155\\
398	0.00676234896814914\\
399	0.00675691344267549\\
400	0.00675130005788015\\
401	0.00674550787862871\\
402	0.00673953995275049\\
403	0.00673340522722306\\
404	0.00672712117799294\\
405	0.00672071625804746\\
406	0.00671421349619451\\
407	0.00670761146295683\\
408	0.00670090870046873\\
409	0.00669410372057442\\
410	0.00668719500258211\\
411	0.00668018099058624\\
412	0.00667306009025135\\
413	0.00666583066493078\\
414	0.0066584910309769\\
415	0.00665103945208354\\
416	0.00664347413248083\\
417	0.00663579320869548\\
418	0.00662799473913034\\
419	0.00662007668933002\\
420	0.00661203691626788\\
421	0.00660387315302666\\
422	0.00659558299162864\\
423	0.00658716386409457\\
424	0.00657861302397764\\
425	0.00656992753692295\\
426	0.00656110430995516\\
427	0.00655214026025386\\
428	0.00654303296443749\\
429	0.00653377909572568\\
430	0.00652437365427828\\
431	0.00651481037914996\\
432	0.00650508044441175\\
433	0.00649516897538018\\
434	0.00648506248822431\\
435	0.00647475124985164\\
436	0.0064642242365699\\
437	0.00645346889958397\\
438	0.00644247087869849\\
439	0.00643121364884752\\
440	0.00641967806504588\\
441	0.00640784181270222\\
442	0.00639567874715611\\
443	0.00638315807704225\\
444	0.00637024335089058\\
445	0.00635689119486041\\
446	0.00634304973467485\\
447	0.00632865661554167\\
448	0.00631363650875949\\
449	0.00629789796108431\\
450	0.00628132940073313\\
451	0.00626379406023011\\
452	0.00624512351136802\\
453	0.00622510944104158\\
454	0.00620349327436457\\
455	0.00617995345149095\\
456	0.00615409125076077\\
457	0.00612542075730385\\
458	0.00609601646693627\\
459	0.00606616241725746\\
460	0.00603584092572867\\
461	0.00600502899017431\\
462	0.00597372174869766\\
463	0.00594191505454013\\
464	0.00590960567751194\\
465	0.00587679155557354\\
466	0.0058434721069465\\
467	0.00580964861900482\\
468	0.00577532473337828\\
469	0.00574050705142841\\
470	0.00570520588925315\\
471	0.00566943621445993\\
472	0.0056332188542572\\
473	0.0055965820362056\\
474	0.00555956331100378\\
475	0.00552221199590847\\
476	0.00548459229280957\\
477	0.00544678728978496\\
478	0.00540890413927234\\
479	0.00537108040097315\\
480	0.00533349233870938\\
481	0.00529636564331368\\
482	0.00525998922738064\\
483	0.00522473313857186\\
484	0.00519107140812409\\
485	0.00515961030441914\\
486	0.00513111942658377\\
487	0.00510583050143825\\
488	0.00508025809679198\\
489	0.00505441758935908\\
490	0.00502832694705126\\
491	0.00500200695922378\\
492	0.00497548144923012\\
493	0.00494877744918957\\
494	0.0049219253078496\\
495	0.00489495868990997\\
496	0.00486791440817347\\
497	0.0048408320076661\\
498	0.00481375299114852\\
499	0.00478671953457748\\
500	0.00475977245909883\\
501	0.00473294813251668\\
502	0.00470627391279249\\
503	0.00467976119434337\\
504	0.00465339454787544\\
505	0.00462711890340711\\
506	0.00460081918406089\\
507	0.00457429104784732\\
508	0.00454720443578428\\
509	0.00451949920381706\\
510	0.00449112857439873\\
511	0.00446202316154678\\
512	0.00443214253187289\\
513	0.00440144139335974\\
514	0.00436986892632396\\
515	0.00433738351339268\\
516	0.00430396765942999\\
517	0.00426955412733077\\
518	0.00423406767181877\\
519	0.00419742583052031\\
520	0.00415953875905356\\
521	0.00412030917420201\\
522	0.00407963455826062\\
523	0.00403741149826289\\
524	0.00399354342326266\\
525	0.00394794794373237\\
526	0.00390054155404816\\
527	0.00385122407791419\\
528	0.0037998565790712\\
529	0.00374634929783748\\
530	0.00369061540493826\\
531	0.00363257431252558\\
532	0.00357216559047094\\
533	0.00350934272525963\\
534	0.00344407564145591\\
535	0.00337635592798696\\
536	0.00330620316315285\\
537	0.00323367434338437\\
538	0.00315887169747382\\
539	0.00308195591005351\\
540	0.00300314669175381\\
541	0.00292271428666918\\
542	0.00284102108688311\\
543	0.00275866609982855\\
544	0.00267662723432266\\
545	0.00259850455214081\\
546	0.00252561474725806\\
547	0.00245828060502498\\
548	0.00239657932141828\\
549	0.00233997269311372\\
550	0.00228472809978635\\
551	0.00223052266454992\\
552	0.00217693151171248\\
553	0.00212346363709513\\
554	0.00207004191030077\\
555	0.00201655731486821\\
556	0.00196290412541734\\
557	0.00190901167585809\\
558	0.00185475671516\\
559	0.00179972684135543\\
560	0.00174381965237555\\
561	0.00168751872449357\\
562	0.00163222103135658\\
563	0.00157815070953452\\
564	0.00152562817316902\\
565	0.00147318530183207\\
566	0.00142057936312707\\
567	0.00136778956338732\\
568	0.00131479124306664\\
569	0.0012616240709895\\
570	0.00120923369631625\\
571	0.00115852724686328\\
572	0.00110937816652107\\
573	0.00106041064718068\\
574	0.00101157345479646\\
575	0.000962869504412621\\
576	0.000914347675446341\\
577	0.000866058972727197\\
578	0.000818515558008548\\
579	0.000771676801487267\\
580	0.000725067167815385\\
581	0.000678702038941237\\
582	0.000632633102614749\\
583	0.000586915166466385\\
584	0.000541602424099719\\
585	0.000496746650654584\\
586	0.000452395225472513\\
587	0.000408588691486123\\
588	0.000365357799889246\\
589	0.000322720123837111\\
590	0.000280676711614414\\
591	0.000239312817487403\\
592	0.000198788644961841\\
593	0.000159293651685586\\
594	0.000121079888727805\\
595	8.45520570083913e-05\\
596	5.05092148680373e-05\\
597	2.07908715710836e-05\\
598	0\\
599	0\\
600	0\\
};
\addplot [color=blue!50!mycolor7,solid,forget plot]
  table[row sep=crcr]{%
1	0.0074068299566467\\
2	0.00740682441785876\\
3	0.00740681878582221\\
4	0.00740681305897183\\
5	0.00740680723571628\\
6	0.00740680131443758\\
7	0.00740679529349066\\
8	0.00740678917120309\\
9	0.00740678294587443\\
10	0.00740677661577581\\
11	0.00740677017914957\\
12	0.00740676363420853\\
13	0.0074067569791359\\
14	0.00740675021208434\\
15	0.00740674333117578\\
16	0.00740673633450079\\
17	0.00740672922011808\\
18	0.00740672198605392\\
19	0.00740671463030165\\
20	0.00740670715082115\\
21	0.00740669954553828\\
22	0.00740669181234433\\
23	0.00740668394909534\\
24	0.00740667595361162\\
25	0.00740666782367715\\
26	0.00740665955703898\\
27	0.00740665115140663\\
28	0.00740664260445139\\
29	0.00740663391380575\\
30	0.00740662507706283\\
31	0.00740661609177554\\
32	0.00740660695545612\\
33	0.00740659766557532\\
34	0.00740658821956183\\
35	0.00740657861480151\\
36	0.00740656884863661\\
37	0.00740655891836529\\
38	0.00740654882124067\\
39	0.00740653855447019\\
40	0.0074065281152148\\
41	0.0074065175005883\\
42	0.00740650670765637\\
43	0.007406495733436\\
44	0.00740648457489452\\
45	0.00740647322894892\\
46	0.00740646169246489\\
47	0.00740644996225608\\
48	0.00740643803508317\\
49	0.00740642590765303\\
50	0.00740641357661781\\
51	0.00740640103857418\\
52	0.00740638829006216\\
53	0.00740637532756442\\
54	0.00740636214750523\\
55	0.00740634874624953\\
56	0.00740633512010191\\
57	0.00740632126530572\\
58	0.00740630717804195\\
59	0.00740629285442831\\
60	0.00740627829051807\\
61	0.00740626348229912\\
62	0.00740624842569288\\
63	0.00740623311655316\\
64	0.00740621755066502\\
65	0.00740620172374375\\
66	0.00740618563143371\\
67	0.00740616926930701\\
68	0.00740615263286254\\
69	0.00740613571752465\\
70	0.00740611851864188\\
71	0.00740610103148595\\
72	0.00740608325125019\\
73	0.00740606517304846\\
74	0.00740604679191379\\
75	0.00740602810279702\\
76	0.00740600910056551\\
77	0.00740598978000174\\
78	0.00740597013580194\\
79	0.00740595016257458\\
80	0.00740592985483905\\
81	0.00740590920702419\\
82	0.00740588821346668\\
83	0.00740586686840964\\
84	0.00740584516600103\\
85	0.00740582310029207\\
86	0.00740580066523575\\
87	0.00740577785468505\\
88	0.00740575466239144\\
89	0.00740573108200305\\
90	0.0074057071070631\\
91	0.00740568273100809\\
92	0.00740565794716598\\
93	0.00740563274875449\\
94	0.00740560712887922\\
95	0.00740558108053179\\
96	0.00740555459658784\\
97	0.00740552766980521\\
98	0.00740550029282195\\
99	0.00740547245815421\\
100	0.00740544415819427\\
101	0.00740541538520844\\
102	0.00740538613133496\\
103	0.00740535638858165\\
104	0.00740532614882389\\
105	0.00740529540380227\\
106	0.00740526414512026\\
107	0.00740523236424184\\
108	0.00740520005248921\\
109	0.00740516720104009\\
110	0.00740513380092542\\
111	0.00740509984302671\\
112	0.00740506531807335\\
113	0.00740503021664007\\
114	0.00740499452914394\\
115	0.00740495824584187\\
116	0.00740492135682741\\
117	0.00740488385202798\\
118	0.00740484572120179\\
119	0.00740480695393475\\
120	0.00740476753963729\\
121	0.007404727467541\\
122	0.00740468672669541\\
123	0.00740464530596449\\
124	0.00740460319402304\\
125	0.00740456037935315\\
126	0.0074045168502405\\
127	0.00740447259477036\\
128	0.00740442760082377\\
129	0.00740438185607332\\
130	0.00740433534797922\\
131	0.00740428806378474\\
132	0.00740423999051178\\
133	0.00740419111495641\\
134	0.007404141423684\\
135	0.0074040909030245\\
136	0.00740403953906722\\
137	0.00740398731765573\\
138	0.00740393422438254\\
139	0.00740388024458343\\
140	0.00740382536333186\\
141	0.007403769565433\\
142	0.00740371283541765\\
143	0.00740365515753591\\
144	0.00740359651575079\\
145	0.00740353689373137\\
146	0.00740347627484593\\
147	0.00740341464215491\\
148	0.00740335197840341\\
149	0.00740328826601369\\
150	0.00740322348707738\\
151	0.00740315762334745\\
152	0.0074030906562299\\
153	0.00740302256677547\\
154	0.0074029533356708\\
155	0.00740288294322967\\
156	0.00740281136938397\\
157	0.00740273859367428\\
158	0.00740266459524067\\
159	0.00740258935281312\\
160	0.00740251284470192\\
161	0.00740243504878779\\
162	0.00740235594251239\\
163	0.00740227550286832\\
164	0.00740219370638955\\
165	0.00740211052914166\\
166	0.0074020259467126\\
167	0.00740193993420315\\
168	0.00740185246621825\\
169	0.00740176351685822\\
170	0.0074016730597107\\
171	0.00740158106784308\\
172	0.00740148751379556\\
173	0.00740139236957476\\
174	0.0074012956066485\\
175	0.0074011971959411\\
176	0.00740109710782996\\
177	0.00740099531214307\\
178	0.00740089177815762\\
179	0.00740078647459987\\
180	0.00740067936964604\\
181	0.00740057043092449\\
182	0.00740045962551893\\
183	0.00740034691997241\\
184	0.00740023228029238\\
185	0.00740011567195596\\
186	0.00739999705991545\\
187	0.0073998764086036\\
188	0.00739975368193796\\
189	0.00739962884332372\\
190	0.00739950185565448\\
191	0.00739937268130986\\
192	0.00739924128214912\\
193	0.00739910761950039\\
194	0.00739897165414423\\
195	0.0073988333462918\\
196	0.00739869265555768\\
197	0.00739854954092654\\
198	0.00739840396070977\\
199	0.00739825587250141\\
200	0.00739810523316437\\
201	0.00739795199881759\\
202	0.00739779612482297\\
203	0.0073976375657721\\
204	0.00739747627547275\\
205	0.00739731220693504\\
206	0.0073971453123574\\
207	0.00739697554311232\\
208	0.00739680284973175\\
209	0.00739662718189224\\
210	0.00739644848839992\\
211	0.007396266717175\\
212	0.00739608181523616\\
213	0.00739589372868453\\
214	0.0073957024026875\\
215	0.00739550778146202\\
216	0.00739530980825783\\
217	0.00739510842534015\\
218	0.00739490357397224\\
219	0.00739469519439751\\
220	0.00739448322582124\\
221	0.00739426760639218\\
222	0.00739404827318354\\
223	0.00739382516217373\\
224	0.00739359820822686\\
225	0.00739336734507254\\
226	0.00739313250528566\\
227	0.00739289362026561\\
228	0.00739265062021495\\
229	0.00739240343411795\\
230	0.00739215198971861\\
231	0.00739189621349811\\
232	0.007391636030652\\
233	0.00739137136506691\\
234	0.00739110213929665\\
235	0.00739082827453812\\
236	0.00739054969060642\\
237	0.00739026630590975\\
238	0.00738997803742359\\
239	0.00738968480066449\\
240	0.0073893865096632\\
241	0.00738908307693744\\
242	0.007388774413464\\
243	0.00738846042865016\\
244	0.00738814103030472\\
245	0.00738781612460839\\
246	0.00738748561608332\\
247	0.00738714940756228\\
248	0.00738680740015708\\
249	0.00738645949322613\\
250	0.00738610558434173\\
251	0.0073857455692561\\
252	0.00738537934186714\\
253	0.00738500679418307\\
254	0.00738462781628656\\
255	0.0073842422962979\\
256	0.0073838501203373\\
257	0.00738345117248648\\
258	0.00738304533474929\\
259	0.00738263248701129\\
260	0.00738221250699872\\
261	0.0073817852702362\\
262	0.00738135065000343\\
263	0.00738090851729121\\
264	0.00738045874075587\\
265	0.00738000118667315\\
266	0.0073795357188904\\
267	0.00737906219877818\\
268	0.00737858048518024\\
269	0.00737809043436237\\
270	0.00737759189996016\\
271	0.00737708473292503\\
272	0.00737656878146932\\
273	0.0073760438910098\\
274	0.00737550990410958\\
275	0.00737496666041871\\
276	0.00737441399661314\\
277	0.00737385174633213\\
278	0.00737327974011379\\
279	0.00737269780532916\\
280	0.00737210576611426\\
281	0.00737150344330047\\
282	0.00737089065434291\\
283	0.00737026721324675\\
284	0.00736963293049165\\
285	0.00736898761295384\\
286	0.00736833106382621\\
287	0.00736766308253593\\
288	0.00736698346465981\\
289	0.0073662920018371\\
290	0.00736558848167974\\
291	0.0073648726876801\\
292	0.00736414439911577\\
293	0.00736340339095163\\
294	0.00736264943373898\\
295	0.00736188229351134\\
296	0.00736110173167753\\
297	0.00736030750491094\\
298	0.00735949936503572\\
299	0.00735867705890934\\
300	0.00735784032830132\\
301	0.00735698890976823\\
302	0.00735612253452471\\
303	0.00735524092831032\\
304	0.00735434381125222\\
305	0.00735343089772323\\
306	0.00735250189619534\\
307	0.0073515565090885\\
308	0.00735059443261443\\
309	0.00734961535661524\\
310	0.00734861896439659\\
311	0.00734760493255559\\
312	0.00734657293080274\\
313	0.00734552262177798\\
314	0.00734445366086052\\
315	0.00734336569597255\\
316	0.00734225836737576\\
317	0.00734113130746157\\
318	0.00733998414053407\\
319	0.00733881648258547\\
320	0.00733762794106414\\
321	0.00733641811463464\\
322	0.00733518659292982\\
323	0.0073339329562943\\
324	0.00733265677551952\\
325	0.00733135761156964\\
326	0.00733003501529837\\
327	0.00732868852715621\\
328	0.00732731767688786\\
329	0.0073259219832197\\
330	0.00732450095353673\\
331	0.00732305408354902\\
332	0.00732158085694693\\
333	0.00732008074504546\\
334	0.00731855320641677\\
335	0.00731699768651116\\
336	0.00731541361726578\\
337	0.00731380041670108\\
338	0.00731215748850482\\
339	0.00731048422160312\\
340	0.00730877998971845\\
341	0.00730704415091459\\
342	0.00730527604712798\\
343	0.00730347500368556\\
344	0.00730164032880894\\
345	0.00729977131310471\\
346	0.00729786722904094\\
347	0.00729592733040972\\
348	0.00729395085177594\\
349	0.00729193700791217\\
350	0.00728988499322004\\
351	0.00728779398113814\\
352	0.00728566312353681\\
353	0.00728349155010023\\
354	0.00728127836769626\\
355	0.00727902265973467\\
356	0.00727672348551477\\
357	0.00727437987956289\\
358	0.00727199085096141\\
359	0.00726955538267042\\
360	0.00726707243084388\\
361	0.00726454092414224\\
362	0.007261959763044\\
363	0.00725932781915937\\
364	0.00725664393454936\\
365	0.00725390692105469\\
366	0.00725111555964022\\
367	0.00724826859976054\\
368	0.00724536475875515\\
369	0.00724240272128197\\
370	0.00723938113880114\\
371	0.00723629862912245\\
372	0.00723315377603356\\
373	0.0072299451290294\\
374	0.00722667120316761\\
375	0.00722333047908035\\
376	0.00721992140317944\\
377	0.00721644238809939\\
378	0.0072128918134324\\
379	0.00720926802682098\\
380	0.00720556934548708\\
381	0.0072017940582922\\
382	0.0071979404284423\\
383	0.00719400669697306\\
384	0.00718999108717504\\
385	0.00718589181014539\\
386	0.00718170707167325\\
387	0.00717743508066228\\
388	0.00717307405923942\\
389	0.00716862225468668\\
390	0.00716407795460443\\
391	0.00715943950628516\\
392	0.00715470533720593\\
393	0.00714987397793557\\
394	0.00714494408796095\\
395	0.00713991448424189\\
396	0.0071347841672599\\
397	0.00712955233752372\\
398	0.00712421840955929\\
399	0.00711878202108357\\
400	0.00711324302348959\\
401	0.00710760144058647\\
402	0.00710185737522207\\
403	0.00709601083121614\\
404	0.0070900613992831\\
405	0.00708400778027166\\
406	0.00707784782427909\\
407	0.00707157929726241\\
408	0.00706519989025344\\
409	0.00705870721664789\\
410	0.00705209880977829\\
411	0.00704537212084409\\
412	0.00703852451725753\\
413	0.00703155328144083\\
414	0.00702445561010178\\
415	0.00701722861410645\\
416	0.00700986931951129\\
417	0.00700237467173069\\
418	0.00699474154818311\\
419	0.00698696679845117\\
420	0.00697904714820445\\
421	0.00697097917369077\\
422	0.00696275928721133\\
423	0.00695438372097599\\
424	0.00694584850944636\\
425	0.0069371494704971\\
426	0.0069282821853252\\
427	0.00691924197185974\\
428	0.00691002381607948\\
429	0.00690062229619514\\
430	0.00689103157279853\\
431	0.0068812453785263\\
432	0.00687125700243555\\
433	0.00686105947035154\\
434	0.00685064572278754\\
435	0.00684000842741724\\
436	0.00682913979427269\\
437	0.00681803157463512\\
438	0.00680667509094791\\
439	0.00679506134536192\\
440	0.00678318167415435\\
441	0.00677102761914462\\
442	0.00675859057904313\\
443	0.00674586191458853\\
444	0.00673283310171506\\
445	0.00671949595028459\\
446	0.00670584291204887\\
447	0.00669186750977668\\
448	0.00667756493066131\\
449	0.00666293284232163\\
450	0.00664797251059849\\
451	0.00663269032774426\\
452	0.0066170999029803\\
453	0.00660122493712663\\
454	0.00658510323055887\\
455	0.00656879244459335\\
456	0.00655237888856633\\
457	0.00653598164334163\\
458	0.00651975971387527\\
459	0.00650381850604284\\
460	0.00648827189993001\\
461	0.00647325382423801\\
462	0.00645794555767362\\
463	0.00644233367387723\\
464	0.00642640309573541\\
465	0.00641013679445086\\
466	0.00639351544169336\\
467	0.00637651698950613\\
468	0.00635911616107912\\
469	0.00634128383119084\\
470	0.00632298626971447\\
471	0.0063041842149933\\
472	0.00628483173380949\\
473	0.00626487481204921\\
474	0.00624424960687539\\
475	0.00622288027233064\\
476	0.00620067624437277\\
477	0.00617752883802573\\
478	0.00615330696433514\\
479	0.00612785173133297\\
480	0.00610096961666709\\
481	0.00607242379360694\\
482	0.00604192306370216\\
483	0.00600910805190212\\
484	0.00597353391254416\\
485	0.00593464965252985\\
486	0.00589177709624734\\
487	0.0058447826614934\\
488	0.00579706219262404\\
489	0.00574862086705563\\
490	0.00569946786929883\\
491	0.00564961737971868\\
492	0.00559908980549789\\
493	0.00554791331535837\\
494	0.00549612575580097\\
495	0.00544377704720198\\
496	0.00539093218421757\\
497	0.00533767499801058\\
498	0.00528411287952206\\
499	0.00523038271546018\\
500	0.00517665835607147\\
501	0.0051231600238064\\
502	0.0050701661987078\\
503	0.00501802872194789\\
504	0.00496721929228177\\
505	0.00491837790758348\\
506	0.0048723283960852\\
507	0.0048301257606557\\
508	0.00479309668846519\\
509	0.00475625955239109\\
510	0.00471898918517486\\
511	0.00468131797440956\\
512	0.0046432840722822\\
513	0.00460493031991499\\
514	0.004566303879972\\
515	0.00452745570207619\\
516	0.00448844008980965\\
517	0.0044493097997433\\
518	0.00441008687171728\\
519	0.00437074144775284\\
520	0.00433122959196758\\
521	0.00429149784774394\\
522	0.00425142897253515\\
523	0.00421081080778946\\
524	0.00416929313380805\\
525	0.00412643381319458\\
526	0.00408215596336811\\
527	0.00403639275687767\\
528	0.00398910860078957\\
529	0.00394021841245417\\
530	0.00388963168405301\\
531	0.00383725090119883\\
532	0.00378297102842801\\
533	0.00372667994895513\\
534	0.00366826052558962\\
535	0.00360759499538104\\
536	0.00354458408089595\\
537	0.00347914048606964\\
538	0.00341120344786853\\
539	0.00334068582296219\\
540	0.00326757016964767\\
541	0.00319187772760867\\
542	0.0031136686903834\\
543	0.00303305075136004\\
544	0.00295018712023015\\
545	0.0028652920901408\\
546	0.00277862134303654\\
547	0.00269059274570993\\
548	0.00260190152765235\\
549	0.00251390181794879\\
550	0.00243061929734025\\
551	0.00235257346475307\\
552	0.00228003875241692\\
553	0.00221317257157155\\
554	0.00215112287025263\\
555	0.00209052503954314\\
556	0.00203108611377265\\
557	0.00197240575620483\\
558	0.00191405466630919\\
559	0.0018558391868882\\
560	0.001797667721791\\
561	0.00173945129258813\\
562	0.00168110421909231\\
563	0.00162243118677364\\
564	0.00156298187657357\\
565	0.00150437000468756\\
566	0.00144703158843678\\
567	0.00139122030448227\\
568	0.00133673591860897\\
569	0.00128228716789082\\
570	0.00122788747298109\\
571	0.00117350595556176\\
572	0.00111939032316793\\
573	0.00106700721614886\\
574	0.00101651208218248\\
575	0.000967001075013799\\
576	0.000917838202573339\\
577	0.000868968269894002\\
578	0.000820423355365856\\
579	0.000772347639745716\\
580	0.000725198081162504\\
581	0.000678755482625295\\
582	0.000632661594736207\\
583	0.000586930577994372\\
584	0.000541610321954773\\
585	0.00049675039409852\\
586	0.000452396819030694\\
587	0.000408589277753164\\
588	0.000365357975688249\\
589	0.000322720162279343\\
590	0.000280676716267942\\
591	0.000239312817487404\\
592	0.000198788644961841\\
593	0.000159293651685586\\
594	0.000121079888727805\\
595	8.45520570083912e-05\\
596	5.05092148680374e-05\\
597	2.07908715710836e-05\\
598	0\\
599	0\\
600	0\\
};
\addplot [color=blue!40!mycolor9,solid,forget plot]
  table[row sep=crcr]{%
1	0.0096074075755447\\
2	0.00960739244151761\\
3	0.00960737705294697\\
4	0.00960736140555528\\
5	0.00960734549499303\\
6	0.0096073293168376\\
7	0.00960731286659196\\
8	0.00960729613968336\\
9	0.00960727913146209\\
10	0.00960726183720023\\
11	0.00960724425209022\\
12	0.00960722637124369\\
13	0.00960720818968978\\
14	0.00960718970237404\\
15	0.0096071709041569\\
16	0.00960715178981218\\
17	0.00960713235402568\\
18	0.0096071125913937\\
19	0.00960709249642146\\
20	0.00960707206352154\\
21	0.00960705128701242\\
22	0.00960703016111673\\
23	0.00960700867995977\\
24	0.00960698683756772\\
25	0.00960696462786599\\
26	0.00960694204467757\\
27	0.00960691908172121\\
28	0.00960689573260967\\
29	0.00960687199084789\\
30	0.0096068478498312\\
31	0.00960682330284341\\
32	0.00960679834305482\\
33	0.00960677296352051\\
34	0.00960674715717811\\
35	0.00960672091684596\\
36	0.00960669423522108\\
37	0.00960666710487688\\
38	0.0096066395182613\\
39	0.00960661146769448\\
40	0.0096065829453667\\
41	0.00960655394333601\\
42	0.00960652445352608\\
43	0.00960649446772381\\
44	0.00960646397757705\\
45	0.0096064329745921\\
46	0.00960640145013138\\
47	0.00960636939541089\\
48	0.0096063368014977\\
49	0.00960630365930738\\
50	0.00960626995960139\\
51	0.00960623569298438\\
52	0.00960620084990154\\
53	0.00960616542063577\\
54	0.00960612939530495\\
55	0.00960609276385898\\
56	0.00960605551607695\\
57	0.00960601764156417\\
58	0.00960597912974904\\
59	0.0096059399698801\\
60	0.00960590015102288\\
61	0.00960585966205662\\
62	0.00960581849167114\\
63	0.00960577662836343\\
64	0.00960573406043442\\
65	0.00960569077598536\\
66	0.00960564676291445\\
67	0.00960560200891331\\
68	0.00960555650146327\\
69	0.00960551022783172\\
70	0.00960546317506837\\
71	0.00960541533000136\\
72	0.00960536667923342\\
73	0.00960531720913784\\
74	0.00960526690585453\\
75	0.00960521575528577\\
76	0.00960516374309205\\
77	0.00960511085468785\\
78	0.00960505707523722\\
79	0.00960500238964931\\
80	0.00960494678257397\\
81	0.00960489023839697\\
82	0.00960483274123542\\
83	0.00960477427493294\\
84	0.00960471482305481\\
85	0.00960465436888295\\
86	0.00960459289541087\\
87	0.00960453038533852\\
88	0.00960446682106704\\
89	0.00960440218469337\\
90	0.00960433645800474\\
91	0.0096042696224732\\
92	0.00960420165924989\\
93	0.00960413254915918\\
94	0.00960406227269294\\
95	0.0096039908100043\\
96	0.00960391814090174\\
97	0.00960384424484273\\
98	0.00960376910092729\\
99	0.00960369268789166\\
100	0.00960361498410159\\
101	0.00960353596754553\\
102	0.00960345561582784\\
103	0.00960337390616179\\
104	0.00960329081536231\\
105	0.00960320631983879\\
106	0.00960312039558761\\
107	0.00960303301818457\\
108	0.00960294416277709\\
109	0.00960285380407652\\
110	0.00960276191634992\\
111	0.00960266847341198\\
112	0.00960257344861666\\
113	0.00960247681484864\\
114	0.00960237854451474\\
115	0.00960227860953499\\
116	0.00960217698133369\\
117	0.00960207363083024\\
118	0.00960196852842973\\
119	0.00960186164401339\\
120	0.009601752946929\\
121	0.00960164240598088\\
122	0.00960152998941985\\
123	0.00960141566493296\\
124	0.00960129939963303\\
125	0.00960118116004798\\
126	0.00960106091210998\\
127	0.00960093862114444\\
128	0.00960081425185868\\
129	0.00960068776833064\\
130	0.009600559133997\\
131	0.00960042831164155\\
132	0.00960029526338298\\
133	0.00960015995066272\\
134	0.00960002233423241\\
135	0.00959988237414115\\
136	0.00959974002972281\\
137	0.00959959525958278\\
138	0.0095994480215847\\
139	0.00959929827283704\\
140	0.00959914596967927\\
141	0.00959899106766809\\
142	0.00959883352156315\\
143	0.00959867328531301\\
144	0.00959851031204037\\
145	0.00959834455402753\\
146	0.00959817596270155\\
147	0.0095980044886191\\
148	0.00959783008145125\\
149	0.00959765268996817\\
150	0.00959747226202343\\
151	0.00959728874453834\\
152	0.00959710208348609\\
153	0.0095969122238756\\
154	0.00959671910973556\\
155	0.00959652268409789\\
156	0.00959632288898141\\
157	0.00959611966537533\\
158	0.00959591295322249\\
159	0.00959570269140253\\
160	0.00959548881771502\\
161	0.00959527126886251\\
162	0.00959504998043319\\
163	0.00959482488688383\\
164	0.00959459592152232\\
165	0.00959436301649017\\
166	0.00959412610274484\\
167	0.00959388511004209\\
168	0.00959363996691783\\
169	0.00959339060067005\\
170	0.00959313693734044\\
171	0.00959287890169572\\
172	0.00959261641720862\\
173	0.00959234940603876\\
174	0.00959207778901272\\
175	0.00959180148560398\\
176	0.00959152041391221\\
177	0.00959123449064189\\
178	0.00959094363108033\\
179	0.00959064774907484\\
180	0.00959034675700905\\
181	0.0095900405657784\\
182	0.00958972908476431\\
183	0.00958941222180756\\
184	0.00958908988318017\\
185	0.00958876197355619\\
186	0.00958842839598119\\
187	0.00958808905184028\\
188	0.00958774384082494\\
189	0.00958739266089855\\
190	0.00958703540826065\\
191	0.00958667197731019\\
192	0.00958630226060791\\
193	0.00958592614883796\\
194	0.00958554353076913\\
195	0.00958515429321589\\
196	0.00958475832099956\\
197	0.00958435549690961\\
198	0.00958394570166566\\
199	0.00958352881387976\\
200	0.00958310471001807\\
201	0.00958267326436181\\
202	0.00958223434896743\\
203	0.00958178783362597\\
204	0.00958133358582178\\
205	0.00958087147069034\\
206	0.00958040135097536\\
207	0.00957992308698488\\
208	0.00957943653654678\\
209	0.00957894155496329\\
210	0.00957843799496454\\
211	0.00957792570666142\\
212	0.00957740453749741\\
213	0.00957687433219943\\
214	0.00957633493272788\\
215	0.00957578617822562\\
216	0.00957522790496593\\
217	0.00957465994629961\\
218	0.00957408213260086\\
219	0.00957349429121224\\
220	0.0095728962463885\\
221	0.00957228781923943\\
222	0.00957166882767131\\
223	0.00957103908632757\\
224	0.00957039840652806\\
225	0.00956974659620724\\
226	0.00956908345985101\\
227	0.00956840879843246\\
228	0.00956772240934639\\
229	0.00956702408634227\\
230	0.00956631361945617\\
231	0.00956559079494125\\
232	0.00956485539519678\\
233	0.00956410719869591\\
234	0.00956334597991188\\
235	0.00956257150924284\\
236	0.00956178355293522\\
237	0.0095609818730054\\
238	0.00956016622716001\\
239	0.00955933636871454\\
240	0.00955849204651045\\
241	0.00955763300483037\\
242	0.00955675898331186\\
243	0.0095558697168594\\
244	0.00955496493555446\\
245	0.00955404436456391\\
246	0.0095531077240466\\
247	0.00955215472905803\\
248	0.009551185089453\\
249	0.00955019850978655\\
250	0.00954919468921262\\
251	0.00954817332138089\\
252	0.00954713409433151\\
253	0.00954607669038755\\
254	0.00954500078604549\\
255	0.00954390605186338\\
256	0.00954279215234682\\
257	0.00954165874583254\\
258	0.00954050548436971\\
259	0.00953933201359898\\
260	0.00953813797262877\\
261	0.00953692299390935\\
262	0.00953568670310429\\
263	0.00953442871895927\\
264	0.00953314865316835\\
265	0.00953184611023756\\
266	0.00953052068734558\\
267	0.0095291719742018\\
268	0.00952779955290148\\
269	0.00952640299777804\\
270	0.00952498187525226\\
271	0.00952353574367869\\
272	0.0095220641531888\\
273	0.00952056664553099\\
274	0.00951904275390769\\
275	0.00951749200280888\\
276	0.00951591390784251\\
277	0.00951430797556143\\
278	0.00951267370328713\\
279	0.00951101057892972\\
280	0.00950931808080451\\
281	0.00950759567744505\\
282	0.00950584282741231\\
283	0.00950405897910038\\
284	0.00950224357053821\\
285	0.00950039602918757\\
286	0.00949851577173702\\
287	0.00949660220389208\\
288	0.00949465472016117\\
289	0.00949267270363763\\
290	0.00949065552577744\\
291	0.0094886025461728\\
292	0.00948651311232144\\
293	0.00948438655939151\\
294	0.00948222220998218\\
295	0.00948001937387975\\
296	0.00947777734780916\\
297	0.00947549541518103\\
298	0.00947317284583406\\
299	0.00947080889577266\\
300	0.00946840280689992\\
301	0.00946595380674576\\
302	0.00946346110819021\\
303	0.00946092390918188\\
304	0.00945834139245133\\
305	0.00945571272521959\\
306	0.00945303705890157\\
307	0.00945031352880442\\
308	0.00944754125382065\\
309	0.00944471933611625\\
310	0.0094418468608135\\
311	0.00943892289566851\\
312	0.00943594649074358\\
313	0.00943291667807412\\
314	0.0094298324713303\\
315	0.00942669286547325\\
316	0.00942349683640599\\
317	0.00942024334061878\\
318	0.00941693131482899\\
319	0.00941355967561562\\
320	0.00941012731904821\\
321	0.00940663312031021\\
322	0.00940307593331675\\
323	0.00939945459032695\\
324	0.00939576790155048\\
325	0.00939201465474842\\
326	0.00938819361482865\\
327	0.00938430352343525\\
328	0.0093803430985324\\
329	0.00937631103398228\\
330	0.00937220599911714\\
331	0.00936802663830553\\
332	0.00936377157051248\\
333	0.00935943938885354\\
334	0.00935502866014277\\
335	0.00935053792443428\\
336	0.00934596569455752\\
337	0.00934131045564591\\
338	0.00933657066465869\\
339	0.00933174474989592\\
340	0.00932683111050627\\
341	0.00932182811598741\\
342	0.00931673410567867\\
343	0.00931154738824574\\
344	0.00930626624115684\\
345	0.00930088891015026\\
346	0.00929541360869242\\
347	0.00928983851742644\\
348	0.00928416178361004\\
349	0.00927838152054283\\
350	0.00927249580698165\\
351	0.0092665026865438\\
352	0.00926040016709684\\
353	0.00925418622013451\\
354	0.00924785878013752\\
355	0.00924141574391836\\
356	0.00923485496994897\\
357	0.00922817427767003\\
358	0.00922137144678084\\
359	0.00921444421650832\\
360	0.00920739028485377\\
361	0.00920020730781615\\
362	0.00919289289859011\\
363	0.00918544462673757\\
364	0.00917786001733097\\
365	0.00917013655006702\\
366	0.0091622716583488\\
367	0.00915426272833516\\
368	0.00914610709795526\\
369	0.00913780205588702\\
370	0.00912934484049751\\
371	0.00912073263874376\\
372	0.00911196258503221\\
373	0.00910303176003477\\
374	0.00909393718945951\\
375	0.00908467584277375\\
376	0.00907524463187639\\
377	0.00906564040971642\\
378	0.00905585996885271\\
379	0.00904590003994982\\
380	0.00903575729020174\\
381	0.0090254283216738\\
382	0.00901490966954928\\
383	0.00900419780026284\\
384	0.00899328910949718\\
385	0.0089821799200117\\
386	0.00897086647926193\\
387	0.00895934495675723\\
388	0.0089476114410925\\
389	0.00893566193658549\\
390	0.00892349235939025\\
391	0.00891109853290295\\
392	0.00889847618237991\\
393	0.00888562092862454\\
394	0.00887252828055634\\
395	0.00885919362642484\\
396	0.0088456122235615\\
397	0.00883177918680417\\
398	0.00881768947542669\\
399	0.00880333787845024\\
400	0.00878871899876059\\
401	0.00877382723711889\\
402	0.00875865677849773\\
403	0.00874320158609038\\
404	0.00872745541512584\\
405	0.00871141187025567\\
406	0.00869506458123908\\
407	0.00867840695760704\\
408	0.00866143169002759\\
409	0.00864413122629998\\
410	0.00862649776265295\\
411	0.00860852323537416\\
412	0.00859019931259046\\
413	0.00857151738521099\\
414	0.00855246855384727\\
415	0.00853304360289888\\
416	0.0085132329392359\\
417	0.00849302644087801\\
418	0.00847241309164893\\
419	0.00845137969852336\\
420	0.00842991615910179\\
421	0.00840801263579948\\
422	0.00838565906355578\\
423	0.00836284514951068\\
424	0.00833956037385398\\
425	0.00831579399204311\\
426	0.00829153503852154\\
427	0.00826677233202398\\
428	0.00824149448407581\\
429	0.00821568991226852\\
430	0.00818934685571328\\
431	0.00816245339275658\\
432	0.00813499746363264\\
433	0.00810696689224114\\
434	0.00807834940349541\\
435	0.00804913265733362\\
436	0.00801930431272337\\
437	0.00798885212316119\\
438	0.00795776403013172\\
439	0.00792602794353914\\
440	0.00789361383292988\\
441	0.0078604925639844\\
442	0.00782664425161163\\
443	0.00779204802704287\\
444	0.00775668192794685\\
445	0.00772052276554722\\
446	0.00768354596203074\\
447	0.00764572534928105\\
448	0.00760703291693255\\
449	0.00756743849364348\\
450	0.0075269093399834\\
451	0.00748540962396113\\
452	0.00744289974033823\\
453	0.00739933542136051\\
454	0.00735466656663158\\
455	0.0073088356848356\\
456	0.00726177573290427\\
457	0.00721340736359842\\
458	0.00716363577891605\\
459	0.00711235074722052\\
460	0.00705942633796879\\
461	0.00700506357462199\\
462	0.0069897888830494\\
463	0.00697415177545719\\
464	0.00695813781822587\\
465	0.00694173199708212\\
466	0.0069249183178362\\
467	0.00690767973509524\\
468	0.0068899980807137\\
469	0.00687185399387683\\
470	0.00685322685424857\\
471	0.00683409471452101\\
472	0.00681443428282061\\
473	0.00679422095417364\\
474	0.0067734288573931\\
475	0.00675203096749532\\
476	0.00672999931676739\\
477	0.00670730535859358\\
478	0.00668392059335595\\
479	0.00665981715701593\\
480	0.00663496905067333\\
481	0.00660935439457848\\
482	0.00658296001510484\\
483	0.00655577770972172\\
484	0.00652780995818072\\
485	0.00649907640909796\\
486	0.00646961933516901\\
487	0.00643950390821288\\
488	0.00640880524000931\\
489	0.00637750451331813\\
490	0.00634558129860116\\
491	0.00631301334562971\\
492	0.00627977635446164\\
493	0.00624584372821354\\
494	0.00621118631284592\\
495	0.00617577213330074\\
496	0.00613956614119091\\
497	0.00610252999712457\\
498	0.00606462192026863\\
499	0.00602579664634754\\
500	0.00598600553441714\\
501	0.00594519683091755\\
502	0.00590331599908454\\
503	0.0058603058449187\\
504	0.00581513748167444\\
505	0.0057661407166878\\
506	0.00571256209853997\\
507	0.00565347260740609\\
508	0.00558774672291453\\
509	0.00552027502761627\\
510	0.00545171558099382\\
511	0.00538210725278225\\
512	0.00531150662435504\\
513	0.00523999271506372\\
514	0.00516767325099425\\
515	0.00509469267449204\\
516	0.00502124230610663\\
517	0.00494757317034403\\
518	0.004874013063796\\
519	0.00480099012756452\\
520	0.00472906273088725\\
521	0.00465895572787164\\
522	0.00459160707580159\\
523	0.00452822708277632\\
524	0.00447036487618838\\
525	0.0044184375290259\\
526	0.00436587397357472\\
527	0.0043126885610599\\
528	0.00425885476479472\\
529	0.00420427192507372\\
530	0.00414892816415185\\
531	0.00409285714575569\\
532	0.00403608410604617\\
533	0.00397861679527939\\
534	0.00392043183348241\\
535	0.0038614544566571\\
536	0.00380152811035017\\
537	0.00374037288718971\\
538	0.00367753023728315\\
539	0.00361247081341678\\
540	0.00354512799959542\\
541	0.00347543947168619\\
542	0.00340334641225652\\
543	0.0033287955821686\\
544	0.00325174240432155\\
545	0.00317215630959154\\
546	0.00309002880541974\\
547	0.00300538262148774\\
548	0.00291828382082769\\
549	0.00282884679169599\\
550	0.00273717375698181\\
551	0.00264351117793987\\
552	0.0025483095260621\\
553	0.00245231860330577\\
554	0.00235722079604287\\
555	0.00226672926843008\\
556	0.00218134599997664\\
557	0.00210146237082903\\
558	0.00202732470075\\
559	0.00195859464071399\\
560	0.00189147544706247\\
561	0.0018257439348815\\
562	0.00176104830254656\\
563	0.00169700040817372\\
564	0.0016334359172862\\
565	0.00157013355595559\\
566	0.0015069742191685\\
567	0.00144382934319225\\
568	0.00138100221627712\\
569	0.00131954993805358\\
570	0.0012596888460467\\
571	0.00120169720130309\\
572	0.00114504677832975\\
573	0.00108874011920619\\
574	0.00103277243214585\\
575	0.00097811364792387\\
576	0.000925437814226222\\
577	0.0008749039059792\\
578	0.000825214412511029\\
579	0.000776118395576129\\
580	0.000727556856375548\\
581	0.000679739350145835\\
582	0.00063299572419274\\
583	0.000587101587693234\\
584	0.000541705757972209\\
585	0.000496801587980551\\
586	0.000452422511210272\\
587	0.000408600951836586\\
588	0.000365362590765301\\
589	0.000322721662279121\\
590	0.000280677076068218\\
591	0.000239312865627478\\
592	0.00019878864496184\\
593	0.000159293651685586\\
594	0.000121079888727804\\
595	8.45520570083909e-05\\
596	5.0509214868037e-05\\
597	2.07908715710836e-05\\
598	0\\
599	0\\
600	0\\
};
\addplot [color=blue!75!mycolor7,solid,forget plot]
  table[row sep=crcr]{%
1	0.00983598579693763\\
2	0.00983598411159307\\
3	0.00983598239790594\\
4	0.00983598065539993\\
5	0.00983597888359073\\
6	0.00983597708198586\\
7	0.00983597525008457\\
8	0.00983597338737765\\
9	0.00983597149334735\\
10	0.00983596956746719\\
11	0.00983596760920181\\
12	0.00983596561800683\\
13	0.00983596359332872\\
14	0.00983596153460462\\
15	0.00983595944126217\\
16	0.00983595731271935\\
17	0.00983595514838439\\
18	0.00983595294765546\\
19	0.00983595070992064\\
20	0.00983594843455768\\
21	0.00983594612093382\\
22	0.00983594376840563\\
23	0.00983594137631882\\
24	0.00983593894400807\\
25	0.00983593647079682\\
26	0.00983593395599707\\
27	0.00983593139890923\\
28	0.00983592879882189\\
29	0.0098359261550116\\
30	0.00983592346674269\\
31	0.00983592073326708\\
32	0.00983591795382402\\
33	0.0098359151276399\\
34	0.00983591225392803\\
35	0.0098359093318884\\
36	0.00983590636070748\\
37	0.00983590333955796\\
38	0.00983590026759852\\
39	0.00983589714397359\\
40	0.00983589396781312\\
41	0.00983589073823231\\
42	0.00983588745433135\\
43	0.0098358841151952\\
44	0.00983588071989328\\
45	0.00983587726747922\\
46	0.00983587375699062\\
47	0.0098358701874487\\
48	0.0098358665578581\\
49	0.00983586286720652\\
50	0.00983585911446448\\
51	0.00983585529858498\\
52	0.00983585141850324\\
53	0.00983584747313635\\
54	0.00983584346138297\\
55	0.00983583938212302\\
56	0.00983583523421735\\
57	0.00983583101650739\\
58	0.00983582672781482\\
59	0.00983582236694124\\
60	0.0098358179326678\\
61	0.00983581342375487\\
62	0.00983580883894161\\
63	0.00983580417694569\\
64	0.00983579943646286\\
65	0.00983579461616655\\
66	0.00983578971470752\\
67	0.00983578473071342\\
68	0.00983577966278844\\
69	0.00983577450951282\\
70	0.00983576926944251\\
71	0.00983576394110868\\
72	0.00983575852301727\\
73	0.00983575301364863\\
74	0.00983574741145698\\
75	0.00983574171486998\\
76	0.00983573592228828\\
77	0.00983573003208498\\
78	0.0098357240426052\\
79	0.00983571795216557\\
80	0.00983571175905367\\
81	0.00983570546152761\\
82	0.00983569905781539\\
83	0.00983569254611445\\
84	0.00983568592459109\\
85	0.0098356791913799\\
86	0.00983567234458321\\
87	0.00983566538227052\\
88	0.00983565830247785\\
89	0.00983565110320724\\
90	0.00983564378242602\\
91	0.00983563633806631\\
92	0.00983562876802426\\
93	0.00983562107015949\\
94	0.00983561324229438\\
95	0.00983560528221342\\
96	0.00983559718766253\\
97	0.00983558895634832\\
98	0.00983558058593741\\
99	0.00983557207405573\\
100	0.00983556341828769\\
101	0.00983555461617552\\
102	0.00983554566521845\\
103	0.00983553656287193\\
104	0.00983552730654685\\
105	0.00983551789360869\\
106	0.00983550832137675\\
107	0.00983549858712324\\
108	0.00983548868807246\\
109	0.00983547862139991\\
110	0.0098354683842314\\
111	0.00983545797364213\\
112	0.00983544738665577\\
113	0.00983543662024351\\
114	0.00983542567132311\\
115	0.00983541453675792\\
116	0.00983540321335585\\
117	0.0098353916978684\\
118	0.00983537998698958\\
119	0.00983536807735489\\
120	0.00983535596554026\\
121	0.00983534364806088\\
122	0.0098353311213702\\
123	0.00983531838185871\\
124	0.00983530542585284\\
125	0.00983529224961377\\
126	0.00983527884933623\\
127	0.00983526522114734\\
128	0.00983525136110529\\
129	0.00983523726519819\\
130	0.00983522292934274\\
131	0.00983520834938293\\
132	0.00983519352108877\\
133	0.00983517844015494\\
134	0.0098351631021994\\
135	0.00983514750276208\\
136	0.00983513163730345\\
137	0.00983511550120311\\
138	0.00983509908975833\\
139	0.00983508239818266\\
140	0.00983506542160439\\
141	0.00983504815506511\\
142	0.00983503059351812\\
143	0.009835012731827\\
144	0.00983499456476397\\
145	0.00983497608700835\\
146	0.009834957293145\\
147	0.00983493817766266\\
148	0.00983491873495236\\
149	0.00983489895930575\\
150	0.0098348788449135\\
151	0.00983485838586352\\
152	0.00983483757613938\\
153	0.00983481640961852\\
154	0.00983479488007056\\
155	0.00983477298115552\\
156	0.00983475070642213\\
157	0.00983472804930598\\
158	0.00983470500312777\\
159	0.00983468156109151\\
160	0.00983465771628264\\
161	0.00983463346166626\\
162	0.00983460879008519\\
163	0.00983458369425816\\
164	0.00983455816677784\\
165	0.00983453220010894\\
166	0.00983450578658627\\
167	0.00983447891841271\\
168	0.00983445158765724\\
169	0.00983442378625285\\
170	0.00983439550599447\\
171	0.00983436673853685\\
172	0.00983433747539234\\
173	0.00983430770792866\\
174	0.00983427742736665\\
175	0.00983424662477785\\
176	0.00983421529108211\\
177	0.00983418341704506\\
178	0.00983415099327551\\
179	0.00983411801022278\\
180	0.00983408445817392\\
181	0.00983405032725079\\
182	0.0098340156074071\\
183	0.00983398028842531\\
184	0.00983394435991335\\
185	0.00983390781130135\\
186	0.00983387063183816\\
187	0.00983383281058781\\
188	0.00983379433642582\\
189	0.00983375519803545\\
190	0.00983371538390387\\
191	0.00983367488231827\\
192	0.00983363368136185\\
193	0.00983359176890989\\
194	0.00983354913262567\\
195	0.00983350575995649\\
196	0.00983346163812967\\
197	0.00983341675414844\\
198	0.00983337109478805\\
199	0.00983332464659158\\
200	0.0098332773958658\\
201	0.00983322932867695\\
202	0.00983318043084641\\
203	0.00983313068794629\\
204	0.00983308008529495\\
205	0.00983302860795242\\
206	0.00983297624071581\\
207	0.00983292296811446\\
208	0.00983286877440525\\
209	0.00983281364356758\\
210	0.00983275755929841\\
211	0.00983270050500715\\
212	0.00983264246381048\\
213	0.00983258341852703\\
214	0.00983252335167203\\
215	0.00983246224545177\\
216	0.00983240008175807\\
217	0.00983233684216253\\
218	0.00983227250791074\\
219	0.00983220705991638\\
220	0.00983214047875521\\
221	0.0098320727446589\\
222	0.00983200383750881\\
223	0.00983193373682962\\
224	0.00983186242178282\\
225	0.00983178987116015\\
226	0.00983171606337684\\
227	0.00983164097646476\\
228	0.00983156458806543\\
229	0.00983148687542292\\
230	0.00983140781537661\\
231	0.00983132738435378\\
232	0.0098312455583621\\
233	0.009831162312982\\
234	0.00983107762335881\\
235	0.00983099146419485\\
236	0.00983090380974132\\
237	0.00983081463379006\\
238	0.0098307239096651\\
239	0.00983063161021417\\
240	0.00983053770779993\\
241	0.00983044217429109\\
242	0.00983034498105338\\
243	0.00983024609894031\\
244	0.0098301454982838\\
245	0.0098300431488846\\
246	0.00982993902000258\\
247	0.00982983308034675\\
248	0.00982972529806523\\
249	0.00982961564073488\\
250	0.00982950407535088\\
251	0.00982939056831603\\
252	0.00982927508542986\\
253	0.00982915759187762\\
254	0.00982903805221893\\
255	0.00982891643037636\\
256	0.00982879268962371\\
257	0.00982866679257412\\
258	0.00982853870116795\\
259	0.00982840837666048\\
260	0.00982827577960929\\
261	0.00982814086986154\\
262	0.00982800360654092\\
263	0.00982786394803442\\
264	0.00982772185197886\\
265	0.00982757727524719\\
266	0.0098274301739345\\
267	0.00982728050334386\\
268	0.00982712821797182\\
269	0.00982697327149379\\
270	0.00982681561674903\\
271	0.00982665520572546\\
272	0.00982649198954422\\
273	0.00982632591844391\\
274	0.00982615694176462\\
275	0.00982598500793166\\
276	0.00982581006443908\\
277	0.0098256320578328\\
278	0.00982545093369359\\
279	0.00982526663661972\\
280	0.0098250791102093\\
281	0.00982488829704241\\
282	0.00982469413866293\\
283	0.00982449657556001\\
284	0.00982429554714937\\
285	0.00982409099175425\\
286	0.00982388284658605\\
287	0.00982367104772479\\
288	0.00982345553009914\\
289	0.0098232362274663\\
290	0.00982301307239145\\
291	0.00982278599622708\\
292	0.00982255492909187\\
293	0.00982231979984942\\
294	0.00982208053608657\\
295	0.00982183706409158\\
296	0.00982158930883191\\
297	0.00982133719393176\\
298	0.00982108064164936\\
299	0.00982081957285399\\
300	0.00982055390700268\\
301	0.00982028356211668\\
302	0.0098200084547577\\
303	0.00981972850000382\\
304	0.00981944361142525\\
305	0.00981915370105972\\
306	0.00981885867938778\\
307	0.00981855845530773\\
308	0.0098182529361104\\
309	0.00981794202745373\\
310	0.00981762563333706\\
311	0.00981730365607531\\
312	0.00981697599627289\\
313	0.0098166425527975\\
314	0.00981630322275366\\
315	0.00981595790145617\\
316	0.00981560648240336\\
317	0.00981524885725017\\
318	0.00981488491578119\\
319	0.00981451454588345\\
320	0.00981413763351916\\
321	0.00981375406269833\\
322	0.0098133637154513\\
323	0.00981296647180112\\
324	0.00981256220973592\\
325	0.00981215080518117\\
326	0.00981173213197181\\
327	0.00981130606182442\\
328	0.0098108724643092\\
329	0.00981043120682202\\
330	0.00980998215455628\\
331	0.00980952517047473\\
332	0.00980906011528128\\
333	0.00980858684739264\\
334	0.0098081052229099\\
335	0.00980761509558998\\
336	0.00980711631681693\\
337	0.00980660873557315\\
338	0.00980609219841025\\
339	0.00980556654941983\\
340	0.00980503163020391\\
341	0.00980448727984503\\
342	0.00980393333487598\\
343	0.00980336962924906\\
344	0.00980279599430481\\
345	0.0098022122587401\\
346	0.00980161824857549\\
347	0.00980101378712182\\
348	0.00980039869494577\\
349	0.00979977278983434\\
350	0.00979913588675813\\
351	0.00979848779783319\\
352	0.00979782833228123\\
353	0.00979715729638818\\
354	0.00979647449346058\\
355	0.00979577972377991\\
356	0.00979507278455434\\
357	0.00979435346986779\\
358	0.00979362157062594\\
359	0.009792876874499\\
360	0.00979211916586069\\
361	0.00979134822572343\\
362	0.00979056383166907\\
363	0.00978976575777495\\
364	0.00978895377453493\\
365	0.00978812764877493\\
366	0.00978728714356253\\
367	0.00978643201811032\\
368	0.0097855620276725\\
369	0.00978467692343428\\
370	0.00978377645239367\\
371	0.00978286035723521\\
372	0.00978192837619511\\
373	0.00978098024291748\\
374	0.00978001568630101\\
375	0.00977903443033571\\
376	0.00977803619392928\\
377	0.00977702069072246\\
378	0.00977598762889288\\
379	0.00977493671094685\\
380	0.00977386763349839\\
381	0.00977278008703469\\
382	0.00977167375566731\\
383	0.00977054831686783\\
384	0.0097694034411868\\
385	0.00976823879195447\\
386	0.00976705402496097\\
387	0.00976584878811385\\
388	0.00976462272106999\\
389	0.00976337545483856\\
390	0.00976210661134679\\
391	0.00976081580296672\\
392	0.00975950263199832\\
393	0.00975816669010216\\
394	0.00975680755767175\\
395	0.00975542480313566\\
396	0.009754017982191\\
397	0.00975258663695195\\
398	0.00975113029498769\\
399	0.0097496484682359\\
400	0.00974814065177603\\
401	0.0097466063224438\\
402	0.0097450449372837\\
403	0.00974345593202622\\
404	0.00974183872085087\\
405	0.00974019270298737\\
406	0.00973851726202865\\
407	0.00973681173897762\\
408	0.00973507542762574\\
409	0.00973330759298671\\
410	0.00973150746984965\\
411	0.00972967426148173\\
412	0.00972780713863298\\
413	0.00972590523909215\\
414	0.00972396766814842\\
415	0.0097219935001699\\
416	0.00971998177979321\\
417	0.00971793151220656\\
418	0.00971584159280819\\
419	0.00971371092465262\\
420	0.009711538524179\\
421	0.009709323390324\\
422	0.00970706448206493\\
423	0.00970476071462334\\
424	0.00970241095497063\\
425	0.00970001401646347\\
426	0.00969756865238058\\
427	0.00969507354811213\\
428	0.0096925273117392\\
429	0.00968992846244233\\
430	0.00968727541617079\\
431	0.00968456646808999\\
432	0.00968179977106074\\
433	0.00967897330958293\\
434	0.00967608487168244\\
435	0.00967313202846759\\
436	0.00967011215823784\\
437	0.00966702265224127\\
438	0.00966386179836363\\
439	0.00966063213953145\\
440	0.0096581438166461\\
441	0.00965618471398823\\
442	0.00965419083958201\\
443	0.0096521612180493\\
444	0.0096500947584985\\
445	0.00964799022959298\\
446	0.00964584622868649\\
447	0.00964366114348987\\
448	0.00964143310430262\\
449	0.00963915992428517\\
450	0.00963683902452495\\
451	0.00963446733971605\\
452	0.00963204119906393\\
453	0.00962955617537492\\
454	0.00962700689239675\\
455	0.00962438677223191\\
456	0.00962168769901288\\
457	0.00961889955818363\\
458	0.00961600960843108\\
459	0.00961300116654827\\
460	0.00960985036340646\\
461	0.0096061970184462\\
462	0.00956347884637286\\
463	0.00951990473209833\\
464	0.00947545229190016\\
465	0.00943009062697348\\
466	0.00938379558812048\\
467	0.00933654208400312\\
468	0.00928830404964439\\
469	0.00923905444750061\\
470	0.00918876536193582\\
471	0.00913740851543912\\
472	0.00908495400076351\\
473	0.00903136941236922\\
474	0.00897662092565435\\
475	0.00892067309233651\\
476	0.00886348839234452\\
477	0.00880502621294514\\
478	0.00874523943928828\\
479	0.00868408740739547\\
480	0.00862152976468595\\
481	0.00855752423148799\\
482	0.00849202612298176\\
483	0.00842498784328819\\
484	0.00835635834929183\\
485	0.00828608224596591\\
486	0.00821409847920065\\
487	0.0081403390213053\\
488	0.0080647285432332\\
489	0.00798718641478394\\
490	0.00790762535960176\\
491	0.0078259505817493\\
492	0.00774205873614955\\
493	0.0076558367082527\\
494	0.00756716015957302\\
495	0.00747589178508248\\
496	0.00738187921625173\\
497	0.00728495249213396\\
498	0.00718492101848958\\
499	0.00708156996690433\\
500	0.00697465620733147\\
501	0.00686390433814216\\
502	0.0067490048182002\\
503	0.00662962124945524\\
504	0.00654282147243754\\
505	0.00650416080218482\\
506	0.00646417271853324\\
507	0.00642286895962343\\
508	0.006380290798672\\
509	0.00633649164171396\\
510	0.00629140003101015\\
511	0.00624491050299828\\
512	0.00619690040501686\\
513	0.00614722597860324\\
514	0.00609571778460128\\
515	0.00604217500531807\\
516	0.0059863583248483\\
517	0.00592798101679594\\
518	0.00586669774697961\\
519	0.00580209039815476\\
520	0.00573365010006878\\
521	0.00566075463826947\\
522	0.0055826405375136\\
523	0.00549837040510512\\
524	0.00540680211042718\\
525	0.00530801300573983\\
526	0.00520877949944434\\
527	0.00510955840050037\\
528	0.0050109498653376\\
529	0.00491374327105498\\
530	0.00481743395479156\\
531	0.00472041401216693\\
532	0.0046230730277967\\
533	0.00452592907234172\\
534	0.00442966844522136\\
535	0.00433519904004857\\
536	0.00424372441285738\\
537	0.00415679393801028\\
538	0.00407632456672\\
539	0.00400303780149818\\
540	0.00392833741499244\\
541	0.00385228073333916\\
542	0.0037749313033303\\
543	0.00369635469524612\\
544	0.00361661131706036\\
545	0.00353574439772463\\
546	0.00345377407636575\\
547	0.0033706573271108\\
548	0.00328623817180589\\
549	0.00320020881435913\\
550	0.00311226168746383\\
551	0.00302174557633384\\
552	0.0029285721704536\\
553	0.00283284845112247\\
554	0.00273472165432939\\
555	0.0026343666970143\\
556	0.00253200294034252\\
557	0.00242802872893287\\
558	0.00232311566314357\\
559	0.00221851657094213\\
560	0.00211831907397233\\
561	0.00202299700719578\\
562	0.00193303203249592\\
563	0.00184882022133786\\
564	0.00177042797505062\\
565	0.00169520130901441\\
566	0.00162172309513774\\
567	0.00154972673714696\\
568	0.00147890019079677\\
569	0.00140915704425968\\
570	0.00133996649982656\\
571	0.00127121916543033\\
572	0.00120361310685485\\
573	0.00113828975820201\\
574	0.0010750352614193\\
575	0.00101406573453819\\
576	0.00095501450658787\\
577	0.000896782871369601\\
578	0.000840716170556225\\
579	0.0007870993751659\\
580	0.000735978059148706\\
581	0.000686067843775887\\
582	0.000637128970025702\\
583	0.000589268610066713\\
584	0.000542736806562125\\
585	0.000497389550407257\\
586	0.000452751633483821\\
587	0.000408776510568709\\
588	0.000365448460457315\\
589	0.000322758538660347\\
590	0.000280690270321202\\
591	0.000239316379010231\\
592	0.000198789174864552\\
593	0.000159293651685586\\
594	0.000121079888727805\\
595	8.45520570083912e-05\\
596	5.05092148680373e-05\\
597	2.07908715710836e-05\\
598	0\\
599	0\\
600	0\\
};
\addplot [color=blue!80!mycolor9,solid,forget plot]
  table[row sep=crcr]{%
1	0.00995024965342724\\
2	0.0099502494926092\\
3	0.00995024932908664\\
4	0.00995024916281409\\
5	0.0099502489937453\\
6	0.00995024882183327\\
7	0.00995024864703018\\
8	0.00995024846928742\\
9	0.00995024828855554\\
10	0.00995024810478429\\
11	0.00995024791792253\\
12	0.00995024772791829\\
13	0.00995024753471872\\
14	0.00995024733827005\\
15	0.00995024713851763\\
16	0.00995024693540588\\
17	0.00995024672887828\\
18	0.00995024651887735\\
19	0.00995024630534464\\
20	0.00995024608822072\\
21	0.00995024586744516\\
22	0.00995024564295647\\
23	0.00995024541469219\\
24	0.00995024518258872\\
25	0.00995024494658145\\
26	0.00995024470660464\\
27	0.00995024446259145\\
28	0.00995024421447391\\
29	0.0099502439621829\\
30	0.00995024370564812\\
31	0.00995024344479807\\
32	0.00995024317956006\\
33	0.00995024290986016\\
34	0.00995024263562317\\
35	0.00995024235677264\\
36	0.00995024207323081\\
37	0.00995024178491859\\
38	0.00995024149175555\\
39	0.0099502411936599\\
40	0.00995024089054847\\
41	0.00995024058233666\\
42	0.00995024026893843\\
43	0.00995023995026628\\
44	0.00995023962623122\\
45	0.00995023929674274\\
46	0.00995023896170881\\
47	0.00995023862103581\\
48	0.00995023827462851\\
49	0.0099502379223901\\
50	0.00995023756422208\\
51	0.00995023720002429\\
52	0.00995023682969484\\
53	0.00995023645313012\\
54	0.00995023607022472\\
55	0.00995023568087147\\
56	0.00995023528496134\\
57	0.00995023488238342\\
58	0.00995023447302493\\
59	0.00995023405677114\\
60	0.00995023363350536\\
61	0.0099502332031089\\
62	0.00995023276546102\\
63	0.00995023232043893\\
64	0.00995023186791771\\
65	0.00995023140777031\\
66	0.00995023093986748\\
67	0.00995023046407775\\
68	0.00995022998026739\\
69	0.00995022948830038\\
70	0.00995022898803832\\
71	0.00995022847934045\\
72	0.0099502279620636\\
73	0.00995022743606207\\
74	0.0099502269011877\\
75	0.00995022635728974\\
76	0.00995022580421483\\
77	0.00995022524180698\\
78	0.00995022466990747\\
79	0.00995022408835485\\
80	0.00995022349698486\\
81	0.00995022289563039\\
82	0.00995022228412143\\
83	0.00995022166228502\\
84	0.00995022102994519\\
85	0.00995022038692289\\
86	0.00995021973303601\\
87	0.00995021906809919\\
88	0.00995021839192389\\
89	0.00995021770431828\\
90	0.00995021700508715\\
91	0.0099502162940319\\
92	0.00995021557095047\\
93	0.00995021483563724\\
94	0.00995021408788301\\
95	0.0099502133274749\\
96	0.00995021255419632\\
97	0.00995021176782685\\
98	0.00995021096814223\\
99	0.00995021015491425\\
100	0.00995020932791068\\
101	0.00995020848689522\\
102	0.0099502076316274\\
103	0.00995020676186253\\
104	0.00995020587735159\\
105	0.00995020497784118\\
106	0.00995020406307343\\
107	0.00995020313278591\\
108	0.00995020218671155\\
109	0.00995020122457859\\
110	0.00995020024611043\\
111	0.00995019925102558\\
112	0.00995019823903758\\
113	0.00995019720985489\\
114	0.00995019616318081\\
115	0.00995019509871337\\
116	0.00995019401614525\\
117	0.00995019291516367\\
118	0.00995019179545031\\
119	0.00995019065668119\\
120	0.00995018949852657\\
121	0.00995018832065087\\
122	0.00995018712271253\\
123	0.00995018590436392\\
124	0.00995018466525123\\
125	0.00995018340501437\\
126	0.00995018212328684\\
127	0.00995018081969562\\
128	0.00995017949386106\\
129	0.00995017814539676\\
130	0.00995017677390944\\
131	0.00995017537899884\\
132	0.00995017396025759\\
133	0.00995017251727106\\
134	0.00995017104961728\\
135	0.00995016955686677\\
136	0.00995016803858243\\
137	0.00995016649431942\\
138	0.00995016492362499\\
139	0.00995016332603836\\
140	0.00995016170109063\\
141	0.00995016004830457\\
142	0.00995015836719451\\
143	0.00995015665726621\\
144	0.00995015491801671\\
145	0.00995015314893415\\
146	0.0099501513494977\\
147	0.00995014951917732\\
148	0.00995014765743366\\
149	0.00995014576371792\\
150	0.00995014383747166\\
151	0.00995014187812665\\
152	0.00995013988510474\\
153	0.00995013785781765\\
154	0.00995013579566686\\
155	0.00995013369804341\\
156	0.00995013156432773\\
157	0.00995012939388952\\
158	0.0099501271860875\\
159	0.00995012494026931\\
160	0.00995012265577127\\
161	0.00995012033191825\\
162	0.00995011796802346\\
163	0.00995011556338825\\
164	0.00995011311730196\\
165	0.00995011062904168\\
166	0.00995010809787208\\
167	0.00995010552304521\\
168	0.00995010290380026\\
169	0.00995010023936337\\
170	0.00995009752894742\\
171	0.00995009477175178\\
172	0.00995009196696209\\
173	0.00995008911375003\\
174	0.00995008621127306\\
175	0.00995008325867418\\
176	0.00995008025508167\\
177	0.00995007719960882\\
178	0.00995007409135364\\
179	0.00995007092939862\\
180	0.00995006771281036\\
181	0.00995006444063938\\
182	0.00995006111191971\\
183	0.00995005772566862\\
184	0.0099500542808863\\
185	0.00995005077655552\\
186	0.00995004721164129\\
187	0.00995004358509052\\
188	0.00995003989583166\\
189	0.00995003614277436\\
190	0.00995003232480913\\
191	0.00995002844080692\\
192	0.0099500244896188\\
193	0.0099500204700756\\
194	0.0099500163809875\\
195	0.00995001222114367\\
196	0.00995000798931191\\
197	0.00995000368423822\\
198	0.00994999930464644\\
199	0.00994999484923785\\
200	0.0099499903166907\\
201	0.00994998570565989\\
202	0.00994998101477646\\
203	0.00994997624264718\\
204	0.00994997138785415\\
205	0.00994996644895427\\
206	0.00994996142447887\\
207	0.00994995631293317\\
208	0.00994995111279585\\
209	0.00994994582251851\\
210	0.00994994044052526\\
211	0.00994993496521215\\
212	0.00994992939494666\\
213	0.0099499237280672\\
214	0.00994991796288258\\
215	0.00994991209767143\\
216	0.0099499061306817\\
217	0.00994990006013004\\
218	0.00994989388420129\\
219	0.0099498876010478\\
220	0.00994988120878893\\
221	0.0099498747055104\\
222	0.00994986808926366\\
223	0.00994986135806527\\
224	0.00994985450989627\\
225	0.00994984754270148\\
226	0.00994984045438891\\
227	0.00994983324282898\\
228	0.00994982590585392\\
229	0.00994981844125699\\
230	0.00994981084679182\\
231	0.00994980312017161\\
232	0.00994979525906845\\
233	0.00994978726111251\\
234	0.00994977912389129\\
235	0.00994977084494881\\
236	0.00994976242178481\\
237	0.00994975385185393\\
238	0.00994974513256489\\
239	0.00994973626127961\\
240	0.00994972723531233\\
241	0.0099497180519288\\
242	0.00994970870834526\\
243	0.00994969920172764\\
244	0.00994968952919054\\
245	0.00994967968779633\\
246	0.00994966967455414\\
247	0.00994965948641888\\
248	0.00994964912029027\\
249	0.00994963857301175\\
250	0.00994962784136949\\
251	0.00994961692209132\\
252	0.00994960581184558\\
253	0.00994959450724013\\
254	0.00994958300482111\\
255	0.00994957130107188\\
256	0.00994955939241183\\
257	0.00994954727519518\\
258	0.00994953494570976\\
259	0.00994952240017586\\
260	0.00994950963474487\\
261	0.00994949664549808\\
262	0.00994948342844539\\
263	0.00994946997952393\\
264	0.0099494562945968\\
265	0.00994944236945163\\
266	0.00994942819979924\\
267	0.00994941378127223\\
268	0.00994939910942352\\
269	0.0099493841797249\\
270	0.00994936898756556\\
271	0.00994935352825057\\
272	0.00994933779699935\\
273	0.00994932178894409\\
274	0.00994930549912823\\
275	0.00994928892250477\\
276	0.0099492720539347\\
277	0.00994925488818531\\
278	0.00994923741992849\\
279	0.00994921964373906\\
280	0.00994920155409301\\
281	0.00994918314536572\\
282	0.00994916441183021\\
283	0.00994914534765528\\
284	0.00994912594690369\\
285	0.0099491062035303\\
286	0.00994908611138015\\
287	0.00994906566418656\\
288	0.00994904485556918\\
289	0.00994902367903201\\
290	0.00994900212796138\\
291	0.00994898019562399\\
292	0.00994895787516478\\
293	0.0099489351596049\\
294	0.00994891204183962\\
295	0.00994888851463616\\
296	0.00994886457063157\\
297	0.00994884020233053\\
298	0.00994881540210321\\
299	0.00994879016218295\\
300	0.00994876447466411\\
301	0.00994873833149973\\
302	0.00994871172449927\\
303	0.00994868464532629\\
304	0.0099486570854961\\
305	0.00994862903637343\\
306	0.00994860048917002\\
307	0.00994857143494221\\
308	0.00994854186458856\\
309	0.00994851176884737\\
310	0.00994848113829424\\
311	0.00994844996333959\\
312	0.0099484182342261\\
313	0.00994838594102629\\
314	0.00994835307363989\\
315	0.00994831962179133\\
316	0.00994828557502712\\
317	0.0099482509227133\\
318	0.00994821565403277\\
319	0.00994817975798267\\
320	0.00994814322337174\\
321	0.00994810603881756\\
322	0.00994806819274393\\
323	0.00994802967337804\\
324	0.0099479904687478\\
325	0.00994795056667894\\
326	0.00994790995479227\\
327	0.00994786862050074\\
328	0.00994782655100661\\
329	0.00994778373329841\\
330	0.00994774015414801\\
331	0.00994769580010754\\
332	0.00994765065750626\\
333	0.0099476047124474\\
334	0.00994755795080488\\
335	0.00994751035821997\\
336	0.00994746192009786\\
337	0.00994741262160411\\
338	0.00994736244766097\\
339	0.00994731138294361\\
340	0.00994725941187616\\
341	0.00994720651862764\\
342	0.00994715268710765\\
343	0.00994709790096189\\
344	0.00994704214356749\\
345	0.00994698539802803\\
346	0.0099469276471684\\
347	0.00994686887352922\\
348	0.00994680905936108\\
349	0.00994674818661835\\
350	0.00994668623695262\\
351	0.00994662319170575\\
352	0.00994655903190239\\
353	0.00994649373824212\\
354	0.00994642729109096\\
355	0.00994635967047233\\
356	0.00994629085605737\\
357	0.00994622082715467\\
358	0.00994614956269912\\
359	0.00994607704124017\\
360	0.0099460032409291\\
361	0.00994592813950552\\
362	0.00994585171428288\\
363	0.00994577394213301\\
364	0.00994569479946957\\
365	0.00994561426223041\\
366	0.00994553230585879\\
367	0.00994544890528327\\
368	0.00994536403489642\\
369	0.00994527766853207\\
370	0.0099451897794412\\
371	0.00994510034026634\\
372	0.00994500932301441\\
373	0.0099449166990279\\
374	0.00994482243895451\\
375	0.00994472651271482\\
376	0.00994462888946838\\
377	0.00994452953757768\\
378	0.0099444284245702\\
379	0.00994432551709836\\
380	0.00994422078089722\\
381	0.00994411418073977\\
382	0.00994400568038974\\
383	0.00994389524255166\\
384	0.00994378282881784\\
385	0.00994366839961223\\
386	0.00994355191413059\\
387	0.00994343333027662\\
388	0.00994331260459367\\
389	0.00994318969219126\\
390	0.0099430645466659\\
391	0.00994293712001525\\
392	0.00994280736254448\\
393	0.00994267522276367\\
394	0.0099425406472743\\
395	0.00994240358064276\\
396	0.00994226396525818\\
397	0.00994212174117097\\
398	0.00994197684590766\\
399	0.00994182921425599\\
400	0.00994167877801249\\
401	0.00994152546568373\\
402	0.00994136920213928\\
403	0.00994120990824239\\
404	0.00994104750053987\\
405	0.00994088189096524\\
406	0.00994071298604871\\
407	0.00994054068635598\\
408	0.00994036488696237\\
409	0.00994018547708384\\
410	0.0099400023397103\\
411	0.00993981535125655\\
412	0.00993962438125417\\
413	0.00993942929211299\\
414	0.00993922993896757\\
415	0.0099390261695445\\
416	0.00993881782376216\\
417	0.00993860473254393\\
418	0.00993838671728155\\
419	0.00993816359336768\\
420	0.00993793516339354\\
421	0.00993770121470024\\
422	0.00993746151751359\\
423	0.00993721582279131\\
424	0.00993696385973073\\
425	0.00993670533287484\\
426	0.00993643991874117\\
427	0.00993616726188308\\
428	0.00993588697026948\\
429	0.00993559860983968\\
430	0.00993530169804049\\
431	0.00993499569604213\\
432	0.00993467999905065\\
433	0.00993435392334768\\
434	0.0099340166861986\\
435	0.00993366736666794\\
436	0.00993330480829898\\
437	0.00993292733222369\\
438	0.00993253180907329\\
439	0.00993211051292307\\
440	0.00993090909648561\\
441	0.00992913499893833\\
442	0.00992731780245534\\
443	0.00992545602322119\\
444	0.0099235481080655\\
445	0.00992159243265715\\
446	0.00991958730045363\\
447	0.00991753094274116\\
448	0.00991542152021376\\
449	0.00991325712668699\\
450	0.00991103579573368\\
451	0.00990875551127558\\
452	0.00990641422346764\\
453	0.00990400987154726\\
454	0.00990154041558171\\
455	0.0098990038789983\\
456	0.00989639840298146\\
457	0.00989372230993868\\
458	0.00989097416419154\\
459	0.00988815280269732\\
460	0.00988525731382974\\
461	0.00988227107081666\\
462	0.00987726586402313\\
463	0.00987217644234873\\
464	0.00986701232349152\\
465	0.00986207582144072\\
466	0.00985706019359379\\
467	0.0098519638048553\\
468	0.00984678493520374\\
469	0.00984152177448637\\
470	0.00983617244261739\\
471	0.00983073497250655\\
472	0.00982520722452214\\
473	0.00981958689007658\\
474	0.00981387154405269\\
475	0.00980805864121547\\
476	0.00980214548286862\\
477	0.00979612902455971\\
478	0.00979000614332297\\
479	0.0097837739541957\\
480	0.00977742945990819\\
481	0.00977096943423777\\
482	0.00976439038006545\\
483	0.00975768849293456\\
484	0.00975085960756883\\
485	0.00974389912357521\\
486	0.00973680192042395\\
487	0.00972956229998946\\
488	0.00972217402843945\\
489	0.00971463020511259\\
490	0.0097069231534983\\
491	0.00969904429054335\\
492	0.00969098396942824\\
493	0.00968273128975784\\
494	0.00967427386756375\\
495	0.00966559755544607\\
496	0.00965668610019438\\
497	0.00964752072027568\\
498	0.00963807957556147\\
499	0.00962833707685462\\
500	0.00961826291146477\\
501	0.00960782043746308\\
502	0.00959696336322506\\
503	0.00958562636004133\\
504	0.00953824077459935\\
505	0.0094428667253592\\
506	0.00934773776944827\\
507	0.00925021459251212\\
508	0.00915018834629194\\
509	0.00904753857441189\\
510	0.00894213426567151\\
511	0.00883382755758078\\
512	0.0087224673322712\\
513	0.00860790122505926\\
514	0.00848996516568321\\
515	0.00836848236237575\\
516	0.0082432622479476\\
517	0.00811409941774295\\
518	0.00798077260987465\\
519	0.00784304381493639\\
520	0.00770065766270751\\
521	0.00755334132295126\\
522	0.00740080524889721\\
523	0.00724274490892458\\
524	0.00707883679535215\\
525	0.00690873609418503\\
526	0.00673202972081162\\
527	0.00654799664263\\
528	0.00635577039827619\\
529	0.00615428802740013\\
530	0.00599822874844659\\
531	0.0059229785927858\\
532	0.00584393196698481\\
533	0.0057605063025082\\
534	0.00567197500609394\\
535	0.00557742595482749\\
536	0.00547570712171949\\
537	0.00536527523908269\\
538	0.00524427765188107\\
539	0.00511240149795213\\
540	0.00497826976213729\\
541	0.00484217428987451\\
542	0.00470452003423817\\
543	0.00456585911209548\\
544	0.00442693566006206\\
545	0.00428874727498958\\
546	0.00415262655058662\\
547	0.00402035366075448\\
548	0.00389417407788781\\
549	0.00377651184121573\\
550	0.00366528682586713\\
551	0.00356355257474437\\
552	0.00346211700310143\\
553	0.00335854913922197\\
554	0.00325299748522605\\
555	0.00314563583395675\\
556	0.00303666170785842\\
557	0.00292628271907932\\
558	0.00281469808724814\\
559	0.00270207149648971\\
560	0.00258855827107614\\
561	0.00247403834706246\\
562	0.00235818448456371\\
563	0.00224107956118409\\
564	0.00212364440435233\\
565	0.00200907458233774\\
566	0.00189926355595736\\
567	0.00179479801060058\\
568	0.00169626418907342\\
569	0.00160400293221254\\
570	0.00151783759986445\\
571	0.00143390318578877\\
572	0.00135203502647469\\
573	0.00127200640621873\\
574	0.00119387836966914\\
575	0.00111724688276997\\
576	0.0010422787258352\\
577	0.000970248500825921\\
578	0.000901376380147175\\
579	0.000835676581260966\\
580	0.000773155683964323\\
581	0.000713512969864073\\
582	0.000657039252267262\\
583	0.000603820761156831\\
584	0.000552908173124348\\
585	0.000503686775016135\\
586	0.0004564261874486\\
587	0.000410922180362966\\
588	0.000366669791782508\\
589	0.000323409636595769\\
590	0.000280998940489726\\
591	0.000239439664031353\\
592	0.000198826448456301\\
593	0.000159300178734201\\
594	0.000121079888727804\\
595	8.45520570083909e-05\\
596	5.05092148680373e-05\\
597	2.07908715710836e-05\\
598	0\\
599	0\\
600	0\\
};
\addplot [color=blue,solid,forget plot]
  table[row sep=crcr]{%
1	0.00996996602909277\\
2	0.00996996602247315\\
3	0.0099699660157422\\
4	0.00996996600889807\\
5	0.00996996600193883\\
6	0.00996996599486257\\
7	0.00996996598766729\\
8	0.00996996598035102\\
9	0.00996996597291171\\
10	0.00996996596534729\\
11	0.00996996595765566\\
12	0.00996996594983469\\
13	0.00996996594188218\\
14	0.00996996593379593\\
15	0.00996996592557369\\
16	0.00996996591721317\\
17	0.00996996590871205\\
18	0.00996996590006795\\
19	0.00996996589127847\\
20	0.00996996588234117\\
21	0.00996996587325356\\
22	0.0099699658640131\\
23	0.00996996585461723\\
24	0.00996996584506332\\
25	0.00996996583534872\\
26	0.00996996582547072\\
27	0.00996996581542656\\
28	0.00996996580521345\\
29	0.00996996579482855\\
30	0.00996996578426895\\
31	0.00996996577353172\\
32	0.00996996576261385\\
33	0.00996996575151232\\
34	0.00996996574022401\\
35	0.00996996572874579\\
36	0.00996996571707446\\
37	0.00996996570520674\\
38	0.00996996569313934\\
39	0.00996996568086888\\
40	0.00996996566839194\\
41	0.00996996565570504\\
42	0.00996996564280464\\
43	0.00996996562968712\\
44	0.00996996561634883\\
45	0.00996996560278604\\
46	0.00996996558899495\\
47	0.00996996557497172\\
48	0.00996996556071242\\
49	0.00996996554621307\\
50	0.0099699655314696\\
51	0.0099699655164779\\
52	0.00996996550123376\\
53	0.00996996548573293\\
54	0.00996996546997105\\
55	0.00996996545394371\\
56	0.00996996543764643\\
57	0.00996996542107463\\
58	0.00996996540422368\\
59	0.00996996538708884\\
60	0.00996996536966531\\
61	0.00996996535194819\\
62	0.00996996533393252\\
63	0.00996996531561324\\
64	0.0099699652969852\\
65	0.00996996527804317\\
66	0.00996996525878183\\
67	0.00996996523919575\\
68	0.00996996521927943\\
69	0.00996996519902727\\
70	0.00996996517843356\\
71	0.0099699651574925\\
72	0.0099699651361982\\
73	0.00996996511454465\\
74	0.00996996509252575\\
75	0.00996996507013529\\
76	0.00996996504736695\\
77	0.0099699650242143\\
78	0.0099699650006708\\
79	0.0099699649767298\\
80	0.00996996495238452\\
81	0.0099699649276281\\
82	0.00996996490245351\\
83	0.00996996487685363\\
84	0.00996996485082121\\
85	0.00996996482434888\\
86	0.00996996479742911\\
87	0.00996996477005428\\
88	0.00996996474221662\\
89	0.00996996471390822\\
90	0.00996996468512103\\
91	0.00996996465584687\\
92	0.00996996462607742\\
93	0.00996996459580418\\
94	0.00996996456501855\\
95	0.00996996453371174\\
96	0.00996996450187483\\
97	0.00996996446949874\\
98	0.00996996443657422\\
99	0.00996996440309185\\
100	0.00996996436904208\\
101	0.00996996433441515\\
102	0.00996996429920115\\
103	0.00996996426339\\
104	0.00996996422697143\\
105	0.009969964189935\\
106	0.00996996415227006\\
107	0.0099699641139658\\
108	0.00996996407501121\\
109	0.0099699640353951\\
110	0.00996996399510604\\
111	0.00996996395413243\\
112	0.00996996391246248\\
113	0.00996996387008414\\
114	0.0099699638269852\\
115	0.0099699637831532\\
116	0.00996996373857546\\
117	0.00996996369323909\\
118	0.00996996364713098\\
119	0.00996996360023775\\
120	0.00996996355254582\\
121	0.00996996350404134\\
122	0.00996996345471023\\
123	0.00996996340453816\\
124	0.00996996335351054\\
125	0.00996996330161252\\
126	0.009969963248829\\
127	0.00996996319514459\\
128	0.00996996314054364\\
129	0.00996996308501022\\
130	0.00996996302852812\\
131	0.00996996297108083\\
132	0.00996996291265156\\
133	0.00996996285322321\\
134	0.00996996279277839\\
135	0.00996996273129941\\
136	0.00996996266876823\\
137	0.00996996260516652\\
138	0.00996996254047562\\
139	0.00996996247467654\\
140	0.00996996240774995\\
141	0.00996996233967618\\
142	0.00996996227043522\\
143	0.0099699622000067\\
144	0.0099699621283699\\
145	0.00996996205550372\\
146	0.00996996198138669\\
147	0.00996996190599699\\
148	0.00996996182931239\\
149	0.00996996175131028\\
150	0.00996996167196766\\
151	0.00996996159126112\\
152	0.00996996150916683\\
153	0.00996996142566058\\
154	0.00996996134071771\\
155	0.00996996125431315\\
156	0.00996996116642136\\
157	0.00996996107701641\\
158	0.00996996098607187\\
159	0.0099699608935609\\
160	0.00996996079945615\\
161	0.00996996070372984\\
162	0.00996996060635368\\
163	0.00996996050729892\\
164	0.00996996040653629\\
165	0.00996996030403604\\
166	0.00996996019976788\\
167	0.00996996009370104\\
168	0.00996995998580419\\
169	0.00996995987604547\\
170	0.00996995976439248\\
171	0.00996995965081226\\
172	0.00996995953527128\\
173	0.00996995941773546\\
174	0.00996995929817009\\
175	0.0099699591765399\\
176	0.009969959052809\\
177	0.00996995892694088\\
178	0.0099699587988984\\
179	0.00996995866864378\\
180	0.00996995853613859\\
181	0.00996995840134372\\
182	0.00996995826421941\\
183	0.00996995812472517\\
184	0.00996995798281983\\
185	0.00996995783846149\\
186	0.00996995769160753\\
187	0.00996995754221457\\
188	0.00996995739023846\\
189	0.00996995723563431\\
190	0.00996995707835639\\
191	0.00996995691835821\\
192	0.00996995675559244\\
193	0.0099699565900109\\
194	0.0099699564215646\\
195	0.00996995625020365\\
196	0.00996995607587729\\
197	0.00996995589853388\\
198	0.00996995571812082\\
199	0.00996995553458463\\
200	0.00996995534787086\\
201	0.0099699551579241\\
202	0.00996995496468796\\
203	0.00996995476810505\\
204	0.00996995456811695\\
205	0.00996995436466423\\
206	0.00996995415768639\\
207	0.00996995394712184\\
208	0.00996995373290793\\
209	0.00996995351498086\\
210	0.00996995329327574\\
211	0.00996995306772648\\
212	0.00996995283826584\\
213	0.00996995260482538\\
214	0.00996995236733544\\
215	0.0099699521257251\\
216	0.00996995187992221\\
217	0.00996995162985329\\
218	0.0099699513754436\\
219	0.00996995111661702\\
220	0.00996995085329609\\
221	0.00996995058540196\\
222	0.00996995031285437\\
223	0.00996995003557163\\
224	0.00996994975347059\\
225	0.00996994946646659\\
226	0.00996994917447347\\
227	0.00996994887740352\\
228	0.00996994857516746\\
229	0.00996994826767441\\
230	0.00996994795483184\\
231	0.00996994763654557\\
232	0.00996994731271975\\
233	0.00996994698325675\\
234	0.00996994664805725\\
235	0.00996994630702009\\
236	0.0099699459600423\\
237	0.00996994560701908\\
238	0.00996994524784372\\
239	0.00996994488240757\\
240	0.00996994451060006\\
241	0.00996994413230859\\
242	0.00996994374741854\\
243	0.00996994335581322\\
244	0.00996994295737383\\
245	0.00996994255197941\\
246	0.00996994213950683\\
247	0.00996994171983073\\
248	0.00996994129282346\\
249	0.00996994085835508\\
250	0.0099699404162933\\
251	0.00996993996650341\\
252	0.00996993950884827\\
253	0.00996993904318826\\
254	0.00996993856938121\\
255	0.00996993808728239\\
256	0.00996993759674444\\
257	0.00996993709761731\\
258	0.00996993658974824\\
259	0.00996993607298169\\
260	0.00996993554715929\\
261	0.00996993501211981\\
262	0.00996993446769908\\
263	0.00996993391372993\\
264	0.00996993335004219\\
265	0.00996993277646257\\
266	0.00996993219281463\\
267	0.00996993159891875\\
268	0.009969930994592\\
269	0.00996993037964818\\
270	0.00996992975389767\\
271	0.00996992911714741\\
272	0.00996992846920084\\
273	0.00996992780985784\\
274	0.00996992713891465\\
275	0.00996992645616379\\
276	0.00996992576139404\\
277	0.00996992505439036\\
278	0.00996992433493378\\
279	0.00996992360280138\\
280	0.0099699228577662\\
281	0.00996992209959719\\
282	0.00996992132805907\\
283	0.00996992054291237\\
284	0.00996991974391325\\
285	0.00996991893081348\\
286	0.00996991810336036\\
287	0.00996991726129664\\
288	0.00996991640436042\\
289	0.00996991553228512\\
290	0.00996991464479934\\
291	0.00996991374162682\\
292	0.00996991282248637\\
293	0.00996991188709173\\
294	0.00996991093515156\\
295	0.0099699099663693\\
296	0.00996990898044311\\
297	0.00996990797706577\\
298	0.00996990695592461\\
299	0.00996990591670143\\
300	0.00996990485907237\\
301	0.00996990378270787\\
302	0.00996990268727253\\
303	0.00996990157242509\\
304	0.00996990043781826\\
305	0.00996989928309867\\
306	0.00996989810790679\\
307	0.0099698969118768\\
308	0.00996989569463652\\
309	0.00996989445580732\\
310	0.009969893195004\\
311	0.00996989191183472\\
312	0.00996989060590088\\
313	0.00996988927679704\\
314	0.00996988792411083\\
315	0.00996988654742281\\
316	0.00996988514630644\\
317	0.00996988372032789\\
318	0.00996988226904603\\
319	0.00996988079201227\\
320	0.00996987928877048\\
321	0.00996987775885688\\
322	0.00996987620179994\\
323	0.00996987461712028\\
324	0.00996987300433055\\
325	0.00996987136293531\\
326	0.009969869692431\\
327	0.0099698679923057\\
328	0.00996986626203913\\
329	0.0099698645011025\\
330	0.00996986270895836\\
331	0.00996986088506052\\
332	0.00996985902885392\\
333	0.0099698571397745\\
334	0.00996985521724908\\
335	0.0099698532606952\\
336	0.00996985126952101\\
337	0.00996984924312513\\
338	0.00996984718089648\\
339	0.00996984508221414\\
340	0.00996984294644718\\
341	0.0099698407729545\\
342	0.00996983856108466\\
343	0.00996983631017565\\
344	0.00996983401955474\\
345	0.00996983168853827\\
346	0.00996982931643136\\
347	0.00996982690252778\\
348	0.00996982444610959\\
349	0.00996982194644695\\
350	0.00996981940279781\\
351	0.00996981681440757\\
352	0.00996981418050881\\
353	0.0099698115003209\\
354	0.00996980877304962\\
355	0.00996980599788678\\
356	0.00996980317400977\\
357	0.00996980030058113\\
358	0.00996979737674798\\
359	0.00996979440164157\\
360	0.00996979137437667\\
361	0.00996978829405096\\
362	0.00996978515974438\\
363	0.00996978197051844\\
364	0.00996977872541548\\
365	0.00996977542345787\\
366	0.00996977206364713\\
367	0.00996976864496313\\
368	0.00996976516636301\\
369	0.00996976162678025\\
370	0.00996975802512356\\
371	0.00996975436027575\\
372	0.00996975063109251\\
373	0.00996974683640113\\
374	0.0099697429749992\\
375	0.00996973904565309\\
376	0.00996973504709655\\
377	0.00996973097802906\\
378	0.00996972683711417\\
379	0.00996972262297775\\
380	0.00996971833420612\\
381	0.00996971396934407\\
382	0.00996970952689278\\
383	0.00996970500530764\\
384	0.00996970040299582\\
385	0.00996969571831386\\
386	0.0099696909495649\\
387	0.00996968609499584\\
388	0.0099696811527942\\
389	0.00996967612108474\\
390	0.0099696709979258\\
391	0.00996966578130518\\
392	0.00996966046913576\\
393	0.00996965505925043\\
394	0.00996964954939657\\
395	0.00996964393722969\\
396	0.00996963822030625\\
397	0.00996963239607535\\
398	0.00996962646186914\\
399	0.00996962041489166\\
400	0.00996961425220605\\
401	0.00996960797072057\\
402	0.00996960156717452\\
403	0.00996959503812491\\
404	0.00996958837992792\\
405	0.00996958158870555\\
406	0.00996957466032922\\
407	0.00996956759043277\\
408	0.00996956037439588\\
409	0.00996955300732745\\
410	0.00996954548404943\\
411	0.00996953779908189\\
412	0.00996952994662935\\
413	0.0099695219205671\\
414	0.00996951371442238\\
415	0.00996950532134258\\
416	0.00996949673405511\\
417	0.00996948794487979\\
418	0.00996947894579393\\
419	0.00996946972817297\\
420	0.00996946028268595\\
421	0.00996945059920828\\
422	0.00996944066672143\\
423	0.00996943047319725\\
424	0.0099694200054641\\
425	0.00996940924905151\\
426	0.00996939818800903\\
427	0.00996938680469331\\
428	0.00996937507951417\\
429	0.00996936299062253\\
430	0.00996935051350271\\
431	0.00996933762037826\\
432	0.00996932427919697\\
433	0.00996931045158034\\
434	0.00996929608813639\\
435	0.00996928111709898\\
436	0.00996926541677184\\
437	0.00996924875236517\\
438	0.00996923065227521\\
439	0.00996921027240247\\
440	0.00996914722470959\\
441	0.00996905323791148\\
442	0.00996895700353201\\
443	0.00996885844626655\\
444	0.00996875748756669\\
445	0.00996865404562082\\
446	0.00996854803539017\\
447	0.00996843936872157\\
448	0.00996832795456464\\
449	0.0099682136993294\\
450	0.00996809650743038\\
451	0.00996797628207344\\
452	0.0099678529263493\\
453	0.00996772634469452\\
454	0.00996759644474977\\
455	0.00996746313955286\\
456	0.0099673263497839\\
457	0.00996718600542639\\
458	0.00996704204584876\\
459	0.00996689441745486\\
460	0.00996674306799945\\
461	0.0099665879168168\\
462	0.00996642858931903\\
463	0.00996626287390056\\
464	0.00996608086039723\\
465	0.00996560070924794\\
466	0.00996511141802948\\
467	0.00996461271605404\\
468	0.00996410431613691\\
469	0.00996358591377835\\
470	0.00996305718637086\\
471	0.00996251779086151\\
472	0.00996196736363268\\
473	0.00996140552254573\\
474	0.00996083186701568\\
475	0.0099602459774223\\
476	0.00995964741248944\\
477	0.0099590357087881\\
478	0.00995841039055374\\
479	0.00995777095424221\\
480	0.00995711686219497\\
481	0.00995644753924431\\
482	0.00995576236905725\\
483	0.00995506068982423\\
484	0.00995434178917141\\
485	0.00995360489833252\\
486	0.0099528491857619\\
487	0.00995207374943311\\
488	0.00995127760706633\\
489	0.00995045968546667\\
490	0.00994961880800696\\
491	0.00994875367986187\\
492	0.00994786287049847\\
493	0.00994694479278383\\
494	0.00994599767784186\\
495	0.00994501954436793\\
496	0.00994400816019456\\
497	0.00994296099166491\\
498	0.00994187513049822\\
499	0.00994074717187218\\
500	0.00993957297357205\\
501	0.00993834710585245\\
502	0.00993706147747492\\
503	0.00993570179724759\\
504	0.00993248398813303\\
505	0.00992458450554222\\
506	0.00991413653284674\\
507	0.00990352052705566\\
508	0.00989272929216558\\
509	0.00988175512175933\\
510	0.00987058941679399\\
511	0.0098592228462381\\
512	0.00984764596295923\\
513	0.00983584888235029\\
514	0.00982382077557695\\
515	0.00981154976657298\\
516	0.00979902281971108\\
517	0.00978622561948705\\
518	0.00977314244811886\\
519	0.00975975608224493\\
520	0.0097460477821495\\
521	0.00973199762945084\\
522	0.00971758611197099\\
523	0.00970280013215884\\
524	0.0096879112403344\\
525	0.00967331433680058\\
526	0.00965827364290924\\
527	0.0096427331866361\\
528	0.00962661829896217\\
529	0.00960982055197028\\
530	0.00953928920580111\\
531	0.009381629884998\\
532	0.00921843775153357\\
533	0.00904940266084434\\
534	0.00887420844912711\\
535	0.00869254825095905\\
536	0.00850414847764258\\
537	0.00831267373605276\\
538	0.00811927026652865\\
539	0.00791867309489294\\
540	0.00771042776501912\\
541	0.00749378093925573\\
542	0.00726785763693795\\
543	0.00703163999036617\\
544	0.00678393139256419\\
545	0.00652326853684657\\
546	0.00624786368936832\\
547	0.00595554554291631\\
548	0.00564387510515887\\
549	0.00532233053630014\\
550	0.00517288644773442\\
551	0.00500843477872852\\
552	0.00483831416493209\\
553	0.00466529191159948\\
554	0.00448978070172018\\
555	0.00431233796658526\\
556	0.00413367673565331\\
557	0.00395483603638486\\
558	0.00377714582892456\\
559	0.00360215714364037\\
560	0.00343164512815057\\
561	0.00326867563951183\\
562	0.00311755320915099\\
563	0.00297470414778135\\
564	0.00283388527610961\\
565	0.0026973744658845\\
566	0.00255897925763545\\
567	0.00241906878442113\\
568	0.002278315210554\\
569	0.00213766925746306\\
570	0.00199855167123961\\
571	0.00186534421375662\\
572	0.00173829019138898\\
573	0.00161722417034892\\
574	0.00150255682998357\\
575	0.00139486353128288\\
576	0.00129406798244129\\
577	0.0011966074677817\\
578	0.001102369942485\\
579	0.00101118175468484\\
580	0.00092283933127548\\
581	0.000838824294993086\\
582	0.000758927142791253\\
583	0.000683310297993498\\
584	0.00061327566441274\\
585	0.000548715386105192\\
586	0.000488688640616261\\
587	0.00043319358073894\\
588	0.000381096578785599\\
589	0.000332240260312541\\
590	0.000286210565888725\\
591	0.000242202348687757\\
592	0.000200085409992\\
593	0.000159745073714005\\
594	0.000121173935986304\\
595	8.45520570083912e-05\\
596	5.05092148680373e-05\\
597	2.07908715710836e-05\\
598	0\\
599	0\\
600	0\\
};
\addplot [color=mycolor10,solid,forget plot]
  table[row sep=crcr]{%
1	0.00997084548698741\\
2	0.00997084548691379\\
3	0.00997084548683892\\
4	0.0099708454867628\\
5	0.00997084548668539\\
6	0.00997084548660669\\
7	0.00997084548652666\\
8	0.00997084548644529\\
9	0.00997084548636254\\
10	0.00997084548627841\\
11	0.00997084548619286\\
12	0.00997084548610587\\
13	0.00997084548601742\\
14	0.00997084548592748\\
15	0.00997084548583603\\
16	0.00997084548574304\\
17	0.00997084548564849\\
18	0.00997084548555234\\
19	0.00997084548545458\\
20	0.00997084548535518\\
21	0.0099708454852541\\
22	0.00997084548515133\\
23	0.00997084548504682\\
24	0.00997084548494056\\
25	0.0099708454848325\\
26	0.00997084548472264\\
27	0.00997084548461093\\
28	0.00997084548449733\\
29	0.00997084548438182\\
30	0.00997084548426437\\
31	0.00997084548414495\\
32	0.00997084548402352\\
33	0.00997084548390004\\
34	0.00997084548377448\\
35	0.00997084548364682\\
36	0.009970845483517\\
37	0.00997084548338501\\
38	0.00997084548325079\\
39	0.00997084548311431\\
40	0.00997084548297553\\
41	0.00997084548283442\\
42	0.00997084548269094\\
43	0.00997084548254504\\
44	0.00997084548239668\\
45	0.00997084548224583\\
46	0.00997084548209243\\
47	0.00997084548193646\\
48	0.00997084548177786\\
49	0.00997084548161659\\
50	0.0099708454814526\\
51	0.00997084548128586\\
52	0.0099708454811163\\
53	0.00997084548094389\\
54	0.00997084548076858\\
55	0.00997084548059031\\
56	0.00997084548040904\\
57	0.00997084548022471\\
58	0.00997084548003728\\
59	0.00997084547984669\\
60	0.00997084547965289\\
61	0.00997084547945583\\
62	0.00997084547925545\\
63	0.00997084547905168\\
64	0.00997084547884449\\
65	0.00997084547863379\\
66	0.00997084547841955\\
67	0.0099708454782017\\
68	0.00997084547798016\\
69	0.0099708454777549\\
70	0.00997084547752584\\
71	0.00997084547729291\\
72	0.00997084547705604\\
73	0.00997084547681519\\
74	0.00997084547657027\\
75	0.00997084547632122\\
76	0.00997084547606796\\
77	0.00997084547581043\\
78	0.00997084547554855\\
79	0.00997084547528224\\
80	0.00997084547501144\\
81	0.00997084547473606\\
82	0.00997084547445604\\
83	0.00997084547417127\\
84	0.0099708454738817\\
85	0.00997084547358724\\
86	0.00997084547328779\\
87	0.00997084547298328\\
88	0.00997084547267362\\
89	0.00997084547235873\\
90	0.00997084547203851\\
91	0.00997084547171287\\
92	0.00997084547138171\\
93	0.00997084547104496\\
94	0.0099708454707025\\
95	0.00997084547035424\\
96	0.00997084547000008\\
97	0.00997084546963992\\
98	0.00997084546927367\\
99	0.0099708454689012\\
100	0.00997084546852242\\
101	0.00997084546813722\\
102	0.00997084546774548\\
103	0.0099708454673471\\
104	0.00997084546694196\\
105	0.00997084546652994\\
106	0.00997084546611093\\
107	0.00997084546568481\\
108	0.00997084546525145\\
109	0.00997084546481072\\
110	0.00997084546436251\\
111	0.00997084546390668\\
112	0.0099708454634431\\
113	0.00997084546297163\\
114	0.00997084546249215\\
115	0.00997084546200451\\
116	0.00997084546150856\\
117	0.00997084546100418\\
118	0.0099708454604912\\
119	0.00997084545996949\\
120	0.00997084545943889\\
121	0.00997084545889924\\
122	0.00997084545835039\\
123	0.00997084545779219\\
124	0.00997084545722446\\
125	0.00997084545664704\\
126	0.00997084545605977\\
127	0.00997084545546246\\
128	0.00997084545485496\\
129	0.00997084545423708\\
130	0.00997084545360864\\
131	0.00997084545296945\\
132	0.00997084545231933\\
133	0.0099708454516581\\
134	0.00997084545098555\\
135	0.00997084545030149\\
136	0.00997084544960571\\
137	0.00997084544889802\\
138	0.0099708454481782\\
139	0.00997084544744605\\
140	0.00997084544670135\\
141	0.00997084544594388\\
142	0.00997084544517342\\
143	0.00997084544438973\\
144	0.0099708454435926\\
145	0.00997084544278178\\
146	0.00997084544195704\\
147	0.00997084544111813\\
148	0.00997084544026481\\
149	0.00997084543939683\\
150	0.00997084543851392\\
151	0.00997084543761583\\
152	0.00997084543670229\\
153	0.00997084543577304\\
154	0.00997084543482779\\
155	0.00997084543386628\\
156	0.00997084543288821\\
157	0.0099708454318933\\
158	0.00997084543088126\\
159	0.00997084542985178\\
160	0.00997084542880455\\
161	0.00997084542773929\\
162	0.00997084542665566\\
163	0.00997084542555335\\
164	0.00997084542443202\\
165	0.00997084542329137\\
166	0.00997084542213103\\
167	0.00997084542095068\\
168	0.00997084541974996\\
169	0.00997084541852853\\
170	0.00997084541728601\\
171	0.00997084541602204\\
172	0.00997084541473626\\
173	0.00997084541342828\\
174	0.00997084541209771\\
175	0.00997084541074417\\
176	0.00997084540936725\\
177	0.00997084540796655\\
178	0.00997084540654165\\
179	0.00997084540509214\\
180	0.00997084540361759\\
181	0.00997084540211755\\
182	0.00997084540059161\\
183	0.00997084539903929\\
184	0.00997084539746014\\
185	0.0099708453958537\\
186	0.00997084539421949\\
187	0.00997084539255703\\
188	0.00997084539086584\\
189	0.00997084538914539\\
190	0.0099708453873952\\
191	0.00997084538561475\\
192	0.0099708453838035\\
193	0.00997084538196092\\
194	0.00997084538008647\\
195	0.00997084537817959\\
196	0.00997084537623971\\
197	0.00997084537426627\\
198	0.00997084537225868\\
199	0.00997084537021633\\
200	0.00997084536813864\\
201	0.00997084536602497\\
202	0.00997084536387471\\
203	0.00997084536168721\\
204	0.00997084535946183\\
205	0.0099708453571979\\
206	0.00997084535489475\\
207	0.0099708453525517\\
208	0.00997084535016805\\
209	0.00997084534774309\\
210	0.00997084534527609\\
211	0.00997084534276634\\
212	0.00997084534021306\\
213	0.00997084533761551\\
214	0.00997084533497291\\
215	0.00997084533228447\\
216	0.00997084532954939\\
217	0.00997084532676685\\
218	0.00997084532393602\\
219	0.00997084532105606\\
220	0.00997084531812609\\
221	0.00997084531514526\\
222	0.00997084531211265\\
223	0.00997084530902738\\
224	0.0099708453058885\\
225	0.00997084530269509\\
226	0.00997084529944617\\
227	0.00997084529614078\\
228	0.00997084529277792\\
229	0.00997084528935659\\
230	0.00997084528587576\\
231	0.00997084528233437\\
232	0.00997084527873137\\
233	0.00997084527506566\\
234	0.00997084527133614\\
235	0.0099708452675417\\
236	0.00997084526368117\\
237	0.00997084525975341\\
238	0.00997084525575722\\
239	0.0099708452516914\\
240	0.00997084524755472\\
241	0.00997084524334592\\
242	0.00997084523906373\\
243	0.00997084523470686\\
244	0.00997084523027398\\
245	0.00997084522576376\\
246	0.00997084522117482\\
247	0.00997084521650577\\
248	0.00997084521175519\\
249	0.00997084520692164\\
250	0.00997084520200365\\
251	0.00997084519699972\\
252	0.00997084519190834\\
253	0.00997084518672794\\
254	0.00997084518145695\\
255	0.00997084517609377\\
256	0.00997084517063675\\
257	0.00997084516508423\\
258	0.00997084515943451\\
259	0.00997084515368588\\
260	0.00997084514783656\\
261	0.00997084514188476\\
262	0.00997084513582868\\
263	0.00997084512966644\\
264	0.00997084512339617\\
265	0.00997084511701593\\
266	0.00997084511052378\\
267	0.00997084510391771\\
268	0.0099708450971957\\
269	0.00997084509035568\\
270	0.00997084508339555\\
271	0.00997084507631316\\
272	0.00997084506910634\\
273	0.00997084506177287\\
274	0.00997084505431048\\
275	0.00997084504671688\\
276	0.00997084503898972\\
277	0.00997084503112663\\
278	0.00997084502312516\\
279	0.00997084501498286\\
280	0.0099708450066972\\
281	0.00997084499826563\\
282	0.00997084498968554\\
283	0.00997084498095427\\
284	0.00997084497206913\\
285	0.00997084496302736\\
286	0.00997084495382617\\
287	0.00997084494446271\\
288	0.00997084493493408\\
289	0.00997084492523733\\
290	0.00997084491536945\\
291	0.00997084490532739\\
292	0.00997084489510803\\
293	0.00997084488470822\\
294	0.00997084487412471\\
295	0.00997084486335425\\
296	0.00997084485239349\\
297	0.00997084484123903\\
298	0.00997084482988741\\
299	0.00997084481833513\\
300	0.0099708448065786\\
301	0.00997084479461418\\
302	0.00997084478243817\\
303	0.00997084477004681\\
304	0.00997084475743624\\
305	0.00997084474460259\\
306	0.00997084473154187\\
307	0.00997084471825006\\
308	0.00997084470472304\\
309	0.00997084469095665\\
310	0.00997084467694664\\
311	0.00997084466268868\\
312	0.00997084464817839\\
313	0.00997084463341131\\
314	0.00997084461838288\\
315	0.00997084460308849\\
316	0.00997084458752345\\
317	0.00997084457168299\\
318	0.00997084455556225\\
319	0.00997084453915629\\
320	0.00997084452246011\\
321	0.0099708445054686\\
322	0.00997084448817659\\
323	0.00997084447057881\\
324	0.0099708444526699\\
325	0.00997084443444444\\
326	0.00997084441589689\\
327	0.00997084439702164\\
328	0.00997084437781298\\
329	0.00997084435826512\\
330	0.00997084433837215\\
331	0.00997084431812812\\
332	0.00997084429752692\\
333	0.00997084427656238\\
334	0.00997084425522824\\
335	0.00997084423351811\\
336	0.00997084421142552\\
337	0.00997084418894389\\
338	0.00997084416606655\\
339	0.00997084414278671\\
340	0.00997084411909746\\
341	0.00997084409499181\\
342	0.00997084407046265\\
343	0.00997084404550273\\
344	0.00997084402010473\\
345	0.00997084399426118\\
346	0.00997084396796449\\
347	0.00997084394120695\\
348	0.00997084391398074\\
349	0.0099708438862779\\
350	0.00997084385809033\\
351	0.0099708438294098\\
352	0.00997084380022794\\
353	0.00997084377053624\\
354	0.00997084374032603\\
355	0.00997084370958851\\
356	0.0099708436783147\\
357	0.00997084364649547\\
358	0.0099708436141215\\
359	0.00997084358118333\\
360	0.0099708435476713\\
361	0.00997084351357554\\
362	0.00997084347888602\\
363	0.00997084344359248\\
364	0.00997084340768447\\
365	0.0099708433711513\\
366	0.00997084333398205\\
367	0.00997084329616555\\
368	0.00997084325769042\\
369	0.00997084321854495\\
370	0.0099708431787172\\
371	0.00997084313819494\\
372	0.0099708430969656\\
373	0.00997084305501633\\
374	0.00997084301233393\\
375	0.00997084296890484\\
376	0.00997084292471516\\
377	0.00997084287975059\\
378	0.00997084283399642\\
379	0.00997084278743753\\
380	0.00997084274005834\\
381	0.00997084269184282\\
382	0.00997084264277444\\
383	0.00997084259283615\\
384	0.00997084254201035\\
385	0.00997084249027889\\
386	0.00997084243762299\\
387	0.00997084238402323\\
388	0.00997084232945953\\
389	0.00997084227391108\\
390	0.00997084221735631\\
391	0.00997084215977281\\
392	0.00997084210113733\\
393	0.00997084204142566\\
394	0.00997084198061258\\
395	0.00997084191867175\\
396	0.00997084185557563\\
397	0.00997084179129535\\
398	0.00997084172580054\\
399	0.0099708416590592\\
400	0.0099708415910376\\
401	0.00997084152170017\\
402	0.00997084145100952\\
403	0.00997084137892621\\
404	0.00997084130540826\\
405	0.00997084123041089\\
406	0.0099708411538868\\
407	0.00997084107578596\\
408	0.00997084099605557\\
409	0.00997084091463992\\
410	0.00997084083148032\\
411	0.00997084074651508\\
412	0.0099708406596792\\
413	0.00997084057090389\\
414	0.00997084048011545\\
415	0.00997084038723444\\
416	0.00997084029217644\\
417	0.00997084019485359\\
418	0.00997084009517133\\
419	0.00997083999302708\\
420	0.00997083988830929\\
421	0.0099708397808962\\
422	0.00997083967065452\\
423	0.00997083955743784\\
424	0.00997083944108465\\
425	0.00997083932141605\\
426	0.00997083919823265\\
427	0.00997083907131013\\
428	0.00997083894039184\\
429	0.0099708388051747\\
430	0.00997083866527911\\
431	0.00997083852018113\\
432	0.00997083836905726\\
433	0.0099708382104362\\
434	0.00997083804145571\\
435	0.00997083785641904\\
436	0.00997083764446549\\
437	0.0099708373873598\\
438	0.00997083706262313\\
439	0.00997083666448949\\
440	0.00997083623128341\\
441	0.00997083578821133\\
442	0.00997083533497104\\
443	0.00997083487124972\\
444	0.00997083439672429\\
445	0.00997083391106196\\
446	0.00997083341392111\\
447	0.0099708329049525\\
448	0.00997083238380088\\
449	0.0099708318501067\\
450	0.00997083130350745\\
451	0.00997083074363683\\
452	0.00997083017011757\\
453	0.00997082958253852\\
454	0.00997082898039505\\
455	0.00997082836294783\\
456	0.00997082772890378\\
457	0.00997082707570273\\
458	0.00997082639786802\\
459	0.00997082568288892\\
460	0.0099708249000621\\
461	0.00997082396978854\\
462	0.00997082269080592\\
463	0.00997082061716523\\
464	0.00997081682871155\\
465	0.00997079688302175\\
466	0.00997077655963949\\
467	0.00997075584701171\\
468	0.00997073473290042\\
469	0.00997071320433725\\
470	0.00997069124754055\\
471	0.00997066884793057\\
472	0.00997064599020059\\
473	0.00997062265830179\\
474	0.00997059883539591\\
475	0.00997057450379251\\
476	0.00997054964505322\\
477	0.00997052424024599\\
478	0.00997049826936458\\
479	0.00997047171110278\\
480	0.00997044454273607\\
481	0.00997041673999051\\
482	0.00997038827689002\\
483	0.0099703591255796\\
484	0.0099703292561263\\
485	0.00997029863629874\\
486	0.00997026723130749\\
487	0.00997023500349532\\
488	0.00997020191199549\\
489	0.00997016791233528\\
490	0.0099701329559731\\
491	0.00997009698975231\\
492	0.00997005995524691\\
493	0.0099700217879548\\
494	0.00996998241625147\\
495	0.00996994175991042\\
496	0.00996989972773478\\
497	0.00996985621320054\\
498	0.00996981108546361\\
499	0.00996976416950573\\
500	0.00996971520147044\\
501	0.0099696637305697\\
502	0.00996960891877847\\
503	0.00996954919283904\\
504	0.00996948186685571\\
505	0.00996929001369385\\
506	0.00996896204731497\\
507	0.0099686269049131\\
508	0.00996828423231784\\
509	0.00996793364018603\\
510	0.0099675747004568\\
511	0.00996720696758362\\
512	0.00996682997231482\\
513	0.00996644320053974\\
514	0.00996604608795879\\
515	0.00996563801379749\\
516	0.00996521829280044\\
517	0.00996478616328325\\
518	0.00996434076440996\\
519	0.00996388108095988\\
520	0.0099634057846906\\
521	0.00996291273618887\\
522	0.00996239734618991\\
523	0.0099618470340882\\
524	0.00996098584201838\\
525	0.00995942607081101\\
526	0.00995781208910305\\
527	0.00995613837293259\\
528	0.00995439742492012\\
529	0.00995257893419314\\
530	0.00994806151716236\\
531	0.00993936864397242\\
532	0.00993061374323875\\
533	0.00992179613327257\\
534	0.00991291338686796\\
535	0.00990395593179488\\
536	0.0098948917788269\\
537	0.00988209889720862\\
538	0.00986432247477629\\
539	0.00984622759948875\\
540	0.00982778327047255\\
541	0.00980895423599394\\
542	0.00978970060060886\\
543	0.00976997744200252\\
544	0.00974973268212212\\
545	0.00972890408895865\\
546	0.00970741598062481\\
547	0.00968517836419248\\
548	0.00966207204382259\\
549	0.00962617862383391\\
550	0.00940147602621652\\
551	0.00916775487436897\\
552	0.00892443714625995\\
553	0.0086706446418252\\
554	0.00840532921897203\\
555	0.00812729915250492\\
556	0.00783671321482161\\
557	0.00753063773729663\\
558	0.0072069709116426\\
559	0.00686361645109179\\
560	0.00650839030900701\\
561	0.00613192561547077\\
562	0.00572661969500788\\
563	0.0052961591784016\\
564	0.00484649400971892\\
565	0.00437548564659083\\
566	0.00415383614871475\\
567	0.00393383894687057\\
568	0.00371213006982874\\
569	0.00349020085871587\\
570	0.00326984259083614\\
571	0.00305370983431872\\
572	0.00284548539780111\\
573	0.00265016852435452\\
574	0.00245848661210091\\
575	0.00226662545412475\\
576	0.00207713061461257\\
577	0.00189587208535604\\
578	0.00172554605519975\\
579	0.00156853045100365\\
580	0.00142575417691301\\
581	0.00129100157684195\\
582	0.00116453672973894\\
583	0.00104320185244136\\
584	0.000925111851658451\\
585	0.000811581194832413\\
586	0.0007041690715661\\
587	0.000603390290013832\\
588	0.000510932493051772\\
589	0.00042719620397733\\
590	0.00035239839743971\\
591	0.000285859088514837\\
592	0.00022641344495182\\
593	0.000173745915387586\\
594	0.00012714455469262\\
595	8.61467099905492e-05\\
596	5.05092148680371e-05\\
597	2.07908715710836e-05\\
598	0\\
599	0\\
600	0\\
};
\addplot [color=mycolor11,solid,forget plot]
  table[row sep=crcr]{%
1	0.00997158250207167\\
2	0.00997158250206864\\
3	0.00997158250206557\\
4	0.00997158250206244\\
5	0.00997158250205925\\
6	0.00997158250205602\\
7	0.00997158250205273\\
8	0.00997158250204938\\
9	0.00997158250204598\\
10	0.00997158250204253\\
11	0.00997158250203901\\
12	0.00997158250203543\\
13	0.0099715825020318\\
14	0.0099715825020281\\
15	0.00997158250202434\\
16	0.00997158250202052\\
17	0.00997158250201663\\
18	0.00997158250201268\\
19	0.00997158250200866\\
20	0.00997158250200458\\
21	0.00997158250200042\\
22	0.0099715825019962\\
23	0.0099715825019919\\
24	0.00997158250198753\\
25	0.00997158250198309\\
26	0.00997158250197857\\
27	0.00997158250197398\\
28	0.00997158250196931\\
29	0.00997158250196457\\
30	0.00997158250195974\\
31	0.00997158250195483\\
32	0.00997158250194984\\
33	0.00997158250194476\\
34	0.0099715825019396\\
35	0.00997158250193435\\
36	0.00997158250192902\\
37	0.00997158250192359\\
38	0.00997158250191807\\
39	0.00997158250191246\\
40	0.00997158250190676\\
41	0.00997158250190096\\
42	0.00997158250189506\\
43	0.00997158250188906\\
44	0.00997158250188296\\
45	0.00997158250187676\\
46	0.00997158250187046\\
47	0.00997158250186404\\
48	0.00997158250185753\\
49	0.0099715825018509\\
50	0.00997158250184415\\
51	0.0099715825018373\\
52	0.00997158250183033\\
53	0.00997158250182324\\
54	0.00997158250181604\\
55	0.00997158250180871\\
56	0.00997158250180125\\
57	0.00997158250179368\\
58	0.00997158250178598\\
59	0.00997158250177814\\
60	0.00997158250177017\\
61	0.00997158250176207\\
62	0.00997158250175384\\
63	0.00997158250174546\\
64	0.00997158250173694\\
65	0.00997158250172828\\
66	0.00997158250171947\\
67	0.00997158250171052\\
68	0.00997158250170141\\
69	0.00997158250169215\\
70	0.00997158250168273\\
71	0.00997158250167316\\
72	0.00997158250166342\\
73	0.00997158250165352\\
74	0.00997158250164345\\
75	0.00997158250163321\\
76	0.0099715825016228\\
77	0.00997158250161221\\
78	0.00997158250160145\\
79	0.0099715825015905\\
80	0.00997158250157936\\
81	0.00997158250156805\\
82	0.00997158250155653\\
83	0.00997158250154483\\
84	0.00997158250153292\\
85	0.00997158250152082\\
86	0.0099715825015085\\
87	0.00997158250149599\\
88	0.00997158250148326\\
89	0.00997158250147031\\
90	0.00997158250145714\\
91	0.00997158250144376\\
92	0.00997158250143014\\
93	0.0099715825014163\\
94	0.00997158250140222\\
95	0.0099715825013879\\
96	0.00997158250137334\\
97	0.00997158250135853\\
98	0.00997158250134347\\
99	0.00997158250132816\\
100	0.00997158250131259\\
101	0.00997158250129675\\
102	0.00997158250128064\\
103	0.00997158250126427\\
104	0.00997158250124761\\
105	0.00997158250123067\\
106	0.00997158250121344\\
107	0.00997158250119592\\
108	0.0099715825011781\\
109	0.00997158250115998\\
110	0.00997158250114155\\
111	0.00997158250112281\\
112	0.00997158250110375\\
113	0.00997158250108436\\
114	0.00997158250106465\\
115	0.0099715825010446\\
116	0.0099715825010242\\
117	0.00997158250100346\\
118	0.00997158250098237\\
119	0.00997158250096092\\
120	0.0099715825009391\\
121	0.00997158250091691\\
122	0.00997158250089434\\
123	0.00997158250087139\\
124	0.00997158250084804\\
125	0.0099715825008243\\
126	0.00997158250080015\\
127	0.00997158250077559\\
128	0.0099715825007506\\
129	0.0099715825007252\\
130	0.00997158250069935\\
131	0.00997158250067307\\
132	0.00997158250064633\\
133	0.00997158250061914\\
134	0.00997158250059148\\
135	0.00997158250056335\\
136	0.00997158250053474\\
137	0.00997158250050563\\
138	0.00997158250047603\\
139	0.00997158250044592\\
140	0.0099715825004153\\
141	0.00997158250038414\\
142	0.00997158250035246\\
143	0.00997158250032023\\
144	0.00997158250028744\\
145	0.0099715825002541\\
146	0.00997158250022018\\
147	0.00997158250018568\\
148	0.00997158250015058\\
149	0.00997158250011488\\
150	0.00997158250007857\\
151	0.00997158250004163\\
152	0.00997158250000406\\
153	0.00997158249996584\\
154	0.00997158249992696\\
155	0.00997158249988742\\
156	0.00997158249984719\\
157	0.00997158249980627\\
158	0.00997158249976464\\
159	0.0099715824997223\\
160	0.00997158249967923\\
161	0.00997158249963541\\
162	0.00997158249959084\\
163	0.0099715824995455\\
164	0.00997158249949938\\
165	0.00997158249945247\\
166	0.00997158249940474\\
167	0.00997158249935619\\
168	0.0099715824993068\\
169	0.00997158249925656\\
170	0.00997158249920546\\
171	0.00997158249915346\\
172	0.00997158249910057\\
173	0.00997158249904678\\
174	0.00997158249899205\\
175	0.00997158249893637\\
176	0.00997158249887974\\
177	0.00997158249882212\\
178	0.00997158249876351\\
179	0.00997158249870389\\
180	0.00997158249864324\\
181	0.00997158249858154\\
182	0.00997158249851877\\
183	0.00997158249845492\\
184	0.00997158249838997\\
185	0.00997158249832389\\
186	0.00997158249825667\\
187	0.00997158249818829\\
188	0.00997158249811872\\
189	0.00997158249804795\\
190	0.00997158249797596\\
191	0.00997158249790273\\
192	0.00997158249782822\\
193	0.00997158249775243\\
194	0.00997158249767533\\
195	0.00997158249759689\\
196	0.00997158249751709\\
197	0.00997158249743592\\
198	0.00997158249735334\\
199	0.00997158249726933\\
200	0.00997158249718386\\
201	0.00997158249709692\\
202	0.00997158249700847\\
203	0.00997158249691849\\
204	0.00997158249682695\\
205	0.00997158249673382\\
206	0.00997158249663908\\
207	0.0099715824965427\\
208	0.00997158249644465\\
209	0.00997158249634489\\
210	0.00997158249624341\\
211	0.00997158249614017\\
212	0.00997158249603514\\
213	0.00997158249592829\\
214	0.00997158249581958\\
215	0.009971582495709\\
216	0.00997158249559648\\
217	0.00997158249548202\\
218	0.00997158249536557\\
219	0.0099715824952471\\
220	0.00997158249512657\\
221	0.00997158249500395\\
222	0.0099715824948792\\
223	0.00997158249475227\\
224	0.00997158249462315\\
225	0.00997158249449178\\
226	0.00997158249435813\\
227	0.00997158249422215\\
228	0.00997158249408381\\
229	0.00997158249394306\\
230	0.00997158249379987\\
231	0.00997158249365418\\
232	0.00997158249350596\\
233	0.00997158249335515\\
234	0.00997158249320173\\
235	0.00997158249304563\\
236	0.00997158249288681\\
237	0.00997158249272522\\
238	0.00997158249256082\\
239	0.00997158249239355\\
240	0.00997158249222337\\
241	0.00997158249205022\\
242	0.00997158249187405\\
243	0.00997158249169481\\
244	0.00997158249151243\\
245	0.00997158249132688\\
246	0.00997158249113808\\
247	0.00997158249094599\\
248	0.00997158249075055\\
249	0.00997158249055169\\
250	0.00997158249034935\\
251	0.00997158249014348\\
252	0.00997158248993401\\
253	0.00997158248972088\\
254	0.00997158248950401\\
255	0.00997158248928336\\
256	0.00997158248905884\\
257	0.00997158248883039\\
258	0.00997158248859794\\
259	0.00997158248836142\\
260	0.00997158248812076\\
261	0.00997158248787588\\
262	0.00997158248762671\\
263	0.00997158248737317\\
264	0.00997158248711518\\
265	0.00997158248685267\\
266	0.00997158248658555\\
267	0.00997158248631374\\
268	0.00997158248603716\\
269	0.00997158248575573\\
270	0.00997158248546934\\
271	0.00997158248517794\\
272	0.0099715824848814\\
273	0.00997158248457966\\
274	0.00997158248427261\\
275	0.00997158248396016\\
276	0.00997158248364221\\
277	0.00997158248331867\\
278	0.00997158248298943\\
279	0.0099715824826544\\
280	0.00997158248231347\\
281	0.00997158248196653\\
282	0.00997158248161348\\
283	0.00997158248125421\\
284	0.0099715824808886\\
285	0.00997158248051655\\
286	0.00997158248013794\\
287	0.00997158247975265\\
288	0.00997158247936056\\
289	0.00997158247896156\\
290	0.00997158247855551\\
291	0.0099715824781423\\
292	0.00997158247772179\\
293	0.00997158247729384\\
294	0.00997158247685835\\
295	0.00997158247641516\\
296	0.00997158247596413\\
297	0.00997158247550514\\
298	0.00997158247503804\\
299	0.00997158247456268\\
300	0.00997158247407891\\
301	0.00997158247358659\\
302	0.00997158247308556\\
303	0.00997158247257567\\
304	0.00997158247205676\\
305	0.00997158247152868\\
306	0.00997158247099126\\
307	0.00997158247044432\\
308	0.00997158246988771\\
309	0.00997158246932125\\
310	0.00997158246874477\\
311	0.00997158246815809\\
312	0.00997158246756103\\
313	0.00997158246695341\\
314	0.00997158246633504\\
315	0.00997158246570573\\
316	0.00997158246506528\\
317	0.00997158246441351\\
318	0.00997158246375021\\
319	0.00997158246307519\\
320	0.00997158246238822\\
321	0.00997158246168912\\
322	0.00997158246097766\\
323	0.00997158246025362\\
324	0.0099715824595168\\
325	0.00997158245876695\\
326	0.00997158245800387\\
327	0.00997158245722731\\
328	0.00997158245643705\\
329	0.00997158245563285\\
330	0.00997158245481445\\
331	0.00997158245398163\\
332	0.00997158245313413\\
333	0.0099715824522717\\
334	0.00997158245139408\\
335	0.00997158245050101\\
336	0.00997158244959222\\
337	0.00997158244866744\\
338	0.00997158244772641\\
339	0.00997158244676884\\
340	0.00997158244579445\\
341	0.00997158244480295\\
342	0.00997158244379405\\
343	0.00997158244276747\\
344	0.00997158244172288\\
345	0.00997158244066\\
346	0.0099715824395785\\
347	0.00997158243847807\\
348	0.0099715824373584\\
349	0.00997158243621915\\
350	0.00997158243505999\\
351	0.00997158243388059\\
352	0.0099715824326806\\
353	0.00997158243145967\\
354	0.00997158243021745\\
355	0.00997158242895357\\
356	0.00997158242766766\\
357	0.00997158242635936\\
358	0.00997158242502828\\
359	0.00997158242367402\\
360	0.00997158242229619\\
361	0.00997158242089439\\
362	0.00997158241946819\\
363	0.00997158241801718\\
364	0.00997158241654094\\
365	0.009971582415039\\
366	0.00997158241351092\\
367	0.00997158241195625\\
368	0.0099715824103745\\
369	0.00997158240876521\\
370	0.00997158240712786\\
371	0.00997158240546195\\
372	0.00997158240376696\\
373	0.00997158240204237\\
374	0.00997158240028761\\
375	0.00997158239850212\\
376	0.00997158239668534\\
377	0.00997158239483664\\
378	0.00997158239295544\\
379	0.00997158239104109\\
380	0.00997158238909294\\
381	0.00997158238711033\\
382	0.00997158238509255\\
383	0.00997158238303891\\
384	0.00997158238094865\\
385	0.00997158237882102\\
386	0.00997158237665522\\
387	0.00997158237445045\\
388	0.00997158237220584\\
389	0.00997158236992053\\
390	0.00997158236759361\\
391	0.00997158236522412\\
392	0.00997158236281107\\
393	0.00997158236035344\\
394	0.00997158235785015\\
395	0.00997158235530007\\
396	0.00997158235270203\\
397	0.00997158235005476\\
398	0.00997158234735696\\
399	0.00997158234460724\\
400	0.00997158234180415\\
401	0.00997158233894615\\
402	0.00997158233603162\\
403	0.00997158233305882\\
404	0.00997158233002592\\
405	0.00997158232693097\\
406	0.00997158232377192\\
407	0.00997158232054661\\
408	0.00997158231725276\\
409	0.00997158231388795\\
410	0.00997158231044967\\
411	0.00997158230693523\\
412	0.0099715823033418\\
413	0.00997158229966632\\
414	0.0099715822959055\\
415	0.0099715822920559\\
416	0.00997158228811384\\
417	0.00997158228407539\\
418	0.00997158227993627\\
419	0.00997158227569181\\
420	0.0099715822713369\\
421	0.00997158226686594\\
422	0.00997158226227277\\
423	0.00997158225755055\\
424	0.00997158225269167\\
425	0.00997158224768756\\
426	0.00997158224252837\\
427	0.00997158223720239\\
428	0.00997158223169493\\
429	0.00997158222598576\\
430	0.00997158222004395\\
431	0.00997158221381703\\
432	0.00997158220721015\\
433	0.00997158220004925\\
434	0.00997158219202694\\
435	0.0099715821826513\\
436	0.00997158217127756\\
437	0.00997158215739005\\
438	0.00997158214122857\\
439	0.00997158212401503\\
440	0.00997158210640971\\
441	0.00997158208840071\\
442	0.00997158206997567\\
443	0.00997158205112189\\
444	0.00997158203182631\\
445	0.00997158201207553\\
446	0.00997158199185592\\
447	0.00997158197115361\\
448	0.00997158194995454\\
449	0.00997158192824437\\
450	0.00997158190600811\\
451	0.00997158188322912\\
452	0.00997158185988651\\
453	0.00997158183594896\\
454	0.0099715818113606\\
455	0.00997158178600892\\
456	0.00997158175965185\\
457	0.00997158173175193\\
458	0.0099715817011006\\
459	0.00997158166499934\\
460	0.00997158161765242\\
461	0.00997158154774152\\
462	0.00997158143699461\\
463	0.00997158126622755\\
464	0.00997158104189281\\
465	0.00997158081343181\\
466	0.00997158058071974\\
467	0.00997158034362549\\
468	0.00997158010201132\\
469	0.00997157985573162\\
470	0.00997157960463319\\
471	0.00997157934855614\\
472	0.00997157908733262\\
473	0.0099715788207843\\
474	0.00997157854872028\\
475	0.0099715782709403\\
476	0.0099715779872406\\
477	0.00997157769740609\\
478	0.00997157740120733\\
479	0.00997157709839908\\
480	0.00997157678871876\\
481	0.00997157647188453\\
482	0.00997157614759318\\
483	0.00997157581551775\\
484	0.0099715754753051\\
485	0.00997157512657271\\
486	0.00997157476890483\\
487	0.00997157440184818\\
488	0.00997157402490682\\
489	0.00997157363753587\\
490	0.00997157323913343\\
491	0.00997157282902933\\
492	0.00997157240646753\\
493	0.00997157197057501\\
494	0.00997157152030013\\
495	0.00997157105428204\\
496	0.00997157057056529\\
497	0.00997157006597804\\
498	0.00997156953481639\\
499	0.00997156896621494\\
500	0.00997156833939104\\
501	0.00997156761655168\\
502	0.00997156673684214\\
503	0.00997156562584849\\
504	0.00997156425665817\\
505	0.00997156274604111\\
506	0.0099715612080005\\
507	0.00997155964131668\\
508	0.00997155804437759\\
509	0.00997155641517961\\
510	0.00997155475184757\\
511	0.00997155305262294\\
512	0.00997155131554636\\
513	0.00997154953838315\\
514	0.00997154771845011\\
515	0.00997154585217563\\
516	0.00997154393393788\\
517	0.00997154195295965\\
518	0.00997153988504464\\
519	0.00997153767096723\\
520	0.00997153516186568\\
521	0.00997153198982832\\
522	0.00997152729962883\\
523	0.00997151937369225\\
524	0.00997149318604075\\
525	0.00997142978152294\\
526	0.00997136434858027\\
527	0.00997129665617338\\
528	0.00997122642403624\\
529	0.00997115341161725\\
530	0.00997107758492663\\
531	0.00997099889826753\\
532	0.00997091691634657\\
533	0.00997083090031376\\
534	0.00997073943226779\\
535	0.00997063970133814\\
536	0.00997052669160612\\
537	0.0099702082159012\\
538	0.00996962328454001\\
539	0.00996902078936174\\
540	0.00996839922675705\\
541	0.00996775689666558\\
542	0.00996709187429563\\
543	0.00996640193383221\\
544	0.00996568437542882\\
545	0.00996493579503008\\
546	0.00996415169729546\\
547	0.00996332579684075\\
548	0.00996244843394521\\
549	0.00996092208856274\\
550	0.00995025740715425\\
551	0.00993968336057744\\
552	0.00992920641838787\\
553	0.00991882893957265\\
554	0.00990854606370902\\
555	0.00989832445380637\\
556	0.00988667401057027\\
557	0.00987483205160817\\
558	0.00986299957738175\\
559	0.00985111088440256\\
560	0.00982940086753967\\
561	0.00980373496254183\\
562	0.00977743457905058\\
563	0.00975058793269213\\
564	0.00972322653700024\\
565	0.00969510362383588\\
566	0.00940580285354155\\
567	0.00909740060809503\\
568	0.00877252040548564\\
569	0.00842896046466404\\
570	0.00806413126271723\\
571	0.00767497738991288\\
572	0.0072578893771046\\
573	0.00680862291380522\\
574	0.00633647419335779\\
575	0.00584531199636737\\
576	0.00533342520441997\\
577	0.00479868293791651\\
578	0.00423933523304903\\
579	0.00367550052531207\\
580	0.00317524304856927\\
581	0.00291649390778095\\
582	0.00266264787724582\\
583	0.0024212958895209\\
584	0.00219950504465807\\
585	0.00198101316635129\\
586	0.0017652741331012\\
587	0.00155340735392807\\
588	0.0013468096195542\\
589	0.00114664925084055\\
590	0.000951372357338311\\
591	0.000764343087165558\\
592	0.00058935041161929\\
593	0.000429546140184966\\
594	0.000289033664663948\\
595	0.000172252650723403\\
596	8.12550264159866e-05\\
597	2.07908715710836e-05\\
598	0\\
599	0\\
600	0\\
};
\addplot [color=mycolor12,solid,forget plot]
  table[row sep=crcr]{%
1	0.00998476612601138\\
2	0.00998476612601125\\
3	0.00998476612601112\\
4	0.00998476612601099\\
5	0.00998476612601085\\
6	0.00998476612601071\\
7	0.00998476612601057\\
8	0.00998476612601043\\
9	0.00998476612601028\\
10	0.00998476612601013\\
11	0.00998476612600998\\
12	0.00998476612600983\\
13	0.00998476612600967\\
14	0.00998476612600952\\
15	0.00998476612600935\\
16	0.00998476612600919\\
17	0.00998476612600903\\
18	0.00998476612600886\\
19	0.00998476612600869\\
20	0.00998476612600851\\
21	0.00998476612600833\\
22	0.00998476612600815\\
23	0.00998476612600797\\
24	0.00998476612600778\\
25	0.00998476612600759\\
26	0.0099847661260074\\
27	0.0099847661260072\\
28	0.009984766126007\\
29	0.0099847661260068\\
30	0.00998476612600659\\
31	0.00998476612600638\\
32	0.00998476612600617\\
33	0.00998476612600595\\
34	0.00998476612600573\\
35	0.00998476612600551\\
36	0.00998476612600528\\
37	0.00998476612600505\\
38	0.00998476612600481\\
39	0.00998476612600457\\
40	0.00998476612600433\\
41	0.00998476612600408\\
42	0.00998476612600383\\
43	0.00998476612600357\\
44	0.00998476612600331\\
45	0.00998476612600305\\
46	0.00998476612600278\\
47	0.0099847661260025\\
48	0.00998476612600223\\
49	0.00998476612600194\\
50	0.00998476612600165\\
51	0.00998476612600136\\
52	0.00998476612600106\\
53	0.00998476612600076\\
54	0.00998476612600045\\
55	0.00998476612600014\\
56	0.00998476612599982\\
57	0.0099847661259995\\
58	0.00998476612599917\\
59	0.00998476612599883\\
60	0.00998476612599849\\
61	0.00998476612599815\\
62	0.00998476612599779\\
63	0.00998476612599744\\
64	0.00998476612599707\\
65	0.0099847661259967\\
66	0.00998476612599633\\
67	0.00998476612599594\\
68	0.00998476612599555\\
69	0.00998476612599516\\
70	0.00998476612599476\\
71	0.00998476612599435\\
72	0.00998476612599393\\
73	0.00998476612599351\\
74	0.00998476612599307\\
75	0.00998476612599264\\
76	0.00998476612599219\\
77	0.00998476612599174\\
78	0.00998476612599128\\
79	0.00998476612599081\\
80	0.00998476612599034\\
81	0.00998476612598985\\
82	0.00998476612598936\\
83	0.00998476612598886\\
84	0.00998476612598835\\
85	0.00998476612598783\\
86	0.00998476612598731\\
87	0.00998476612598677\\
88	0.00998476612598623\\
89	0.00998476612598567\\
90	0.00998476612598511\\
91	0.00998476612598454\\
92	0.00998476612598396\\
93	0.00998476612598337\\
94	0.00998476612598276\\
95	0.00998476612598215\\
96	0.00998476612598153\\
97	0.00998476612598089\\
98	0.00998476612598025\\
99	0.0099847661259796\\
100	0.00998476612597893\\
101	0.00998476612597826\\
102	0.00998476612597757\\
103	0.00998476612597687\\
104	0.00998476612597616\\
105	0.00998476612597543\\
106	0.00998476612597469\\
107	0.00998476612597395\\
108	0.00998476612597318\\
109	0.00998476612597241\\
110	0.00998476612597162\\
111	0.00998476612597082\\
112	0.00998476612597\\
113	0.00998476612596918\\
114	0.00998476612596833\\
115	0.00998476612596747\\
116	0.0099847661259666\\
117	0.00998476612596572\\
118	0.00998476612596482\\
119	0.00998476612596389\\
120	0.00998476612596296\\
121	0.00998476612596201\\
122	0.00998476612596105\\
123	0.00998476612596007\\
124	0.00998476612595907\\
125	0.00998476612595805\\
126	0.00998476612595702\\
127	0.00998476612595597\\
128	0.0099847661259549\\
129	0.00998476612595382\\
130	0.00998476612595271\\
131	0.00998476612595159\\
132	0.00998476612595045\\
133	0.00998476612594928\\
134	0.0099847661259481\\
135	0.0099847661259469\\
136	0.00998476612594567\\
137	0.00998476612594443\\
138	0.00998476612594316\\
139	0.00998476612594187\\
140	0.00998476612594057\\
141	0.00998476612593923\\
142	0.00998476612593788\\
143	0.0099847661259365\\
144	0.0099847661259351\\
145	0.00998476612593367\\
146	0.00998476612593222\\
147	0.00998476612593075\\
148	0.00998476612592925\\
149	0.00998476612592772\\
150	0.00998476612592616\\
151	0.00998476612592459\\
152	0.00998476612592298\\
153	0.00998476612592135\\
154	0.00998476612591968\\
155	0.00998476612591799\\
156	0.00998476612591627\\
157	0.00998476612591452\\
158	0.00998476612591274\\
159	0.00998476612591093\\
160	0.00998476612590909\\
161	0.00998476612590721\\
162	0.00998476612590531\\
163	0.00998476612590336\\
164	0.0099847661259014\\
165	0.00998476612589939\\
166	0.00998476612589735\\
167	0.00998476612589527\\
168	0.00998476612589316\\
169	0.00998476612589101\\
170	0.00998476612588882\\
171	0.0099847661258866\\
172	0.00998476612588434\\
173	0.00998476612588204\\
174	0.0099847661258797\\
175	0.00998476612587731\\
176	0.00998476612587489\\
177	0.00998476612587243\\
178	0.00998476612586992\\
179	0.00998476612586737\\
180	0.00998476612586478\\
181	0.00998476612586214\\
182	0.00998476612585945\\
183	0.00998476612585672\\
184	0.00998476612585395\\
185	0.00998476612585112\\
186	0.00998476612584824\\
187	0.00998476612584532\\
188	0.00998476612584234\\
189	0.00998476612583932\\
190	0.00998476612583624\\
191	0.00998476612583311\\
192	0.00998476612582992\\
193	0.00998476612582668\\
194	0.00998476612582338\\
195	0.00998476612582003\\
196	0.00998476612581661\\
197	0.00998476612581314\\
198	0.00998476612580961\\
199	0.00998476612580602\\
200	0.00998476612580236\\
201	0.00998476612579864\\
202	0.00998476612579486\\
203	0.00998476612579101\\
204	0.00998476612578709\\
205	0.00998476612578311\\
206	0.00998476612577906\\
207	0.00998476612577493\\
208	0.00998476612577074\\
209	0.00998476612576647\\
210	0.00998476612576213\\
211	0.00998476612575772\\
212	0.00998476612575322\\
213	0.00998476612574865\\
214	0.009984766125744\\
215	0.00998476612573928\\
216	0.00998476612573446\\
217	0.00998476612572956\\
218	0.00998476612572458\\
219	0.00998476612571952\\
220	0.00998476612571436\\
221	0.00998476612570912\\
222	0.00998476612570378\\
223	0.00998476612569835\\
224	0.00998476612569283\\
225	0.00998476612568721\\
226	0.00998476612568149\\
227	0.00998476612567567\\
228	0.00998476612566975\\
229	0.00998476612566373\\
230	0.00998476612565761\\
231	0.00998476612565138\\
232	0.00998476612564504\\
233	0.00998476612563859\\
234	0.00998476612563202\\
235	0.00998476612562534\\
236	0.00998476612561855\\
237	0.00998476612561164\\
238	0.0099847661256046\\
239	0.00998476612559745\\
240	0.00998476612559017\\
241	0.00998476612558276\\
242	0.00998476612557522\\
243	0.00998476612556756\\
244	0.00998476612555975\\
245	0.00998476612555182\\
246	0.00998476612554374\\
247	0.00998476612553552\\
248	0.00998476612552716\\
249	0.00998476612551865\\
250	0.00998476612551\\
251	0.00998476612550119\\
252	0.00998476612549223\\
253	0.00998476612548311\\
254	0.00998476612547383\\
255	0.00998476612546439\\
256	0.00998476612545479\\
257	0.00998476612544501\\
258	0.00998476612543507\\
259	0.00998476612542495\\
260	0.00998476612541465\\
261	0.00998476612540418\\
262	0.00998476612539351\\
263	0.00998476612538267\\
264	0.00998476612537163\\
265	0.0099847661253604\\
266	0.00998476612534897\\
267	0.00998476612533734\\
268	0.00998476612532551\\
269	0.00998476612531347\\
270	0.00998476612530121\\
271	0.00998476612528874\\
272	0.00998476612527606\\
273	0.00998476612526314\\
274	0.00998476612525001\\
275	0.00998476612523664\\
276	0.00998476612522303\\
277	0.00998476612520919\\
278	0.00998476612519511\\
279	0.00998476612518077\\
280	0.00998476612516618\\
281	0.00998476612515134\\
282	0.00998476612513623\\
283	0.00998476612512086\\
284	0.00998476612510522\\
285	0.00998476612508929\\
286	0.0099847661250731\\
287	0.00998476612505661\\
288	0.00998476612503983\\
289	0.00998476612502276\\
290	0.00998476612500538\\
291	0.0099847661249877\\
292	0.00998476612496971\\
293	0.0099847661249514\\
294	0.00998476612493276\\
295	0.0099847661249138\\
296	0.0099847661248945\\
297	0.00998476612487486\\
298	0.00998476612485487\\
299	0.00998476612483453\\
300	0.00998476612481383\\
301	0.00998476612479277\\
302	0.00998476612477133\\
303	0.00998476612474951\\
304	0.00998476612472731\\
305	0.00998476612470471\\
306	0.00998476612468171\\
307	0.00998476612465831\\
308	0.00998476612463449\\
309	0.00998476612461025\\
310	0.00998476612458559\\
311	0.00998476612456048\\
312	0.00998476612453494\\
313	0.00998476612450894\\
314	0.00998476612448248\\
315	0.00998476612445555\\
316	0.00998476612442815\\
317	0.00998476612440026\\
318	0.00998476612437188\\
319	0.009984766124343\\
320	0.00998476612431361\\
321	0.0099847661242837\\
322	0.00998476612425326\\
323	0.00998476612422228\\
324	0.00998476612419075\\
325	0.00998476612415867\\
326	0.00998476612412602\\
327	0.00998476612409281\\
328	0.009984766124059\\
329	0.00998476612402459\\
330	0.00998476612398958\\
331	0.00998476612395395\\
332	0.0099847661239177\\
333	0.0099847661238808\\
334	0.00998476612384326\\
335	0.00998476612380506\\
336	0.00998476612376618\\
337	0.00998476612372663\\
338	0.00998476612368637\\
339	0.00998476612364541\\
340	0.00998476612360373\\
341	0.00998476612356133\\
342	0.00998476612351818\\
343	0.00998476612347427\\
344	0.00998476612342959\\
345	0.00998476612338413\\
346	0.00998476612333788\\
347	0.00998476612329082\\
348	0.00998476612324294\\
349	0.00998476612319422\\
350	0.00998476612314465\\
351	0.00998476612309421\\
352	0.0099847661230429\\
353	0.00998476612299069\\
354	0.00998476612293757\\
355	0.00998476612288353\\
356	0.00998476612282855\\
357	0.00998476612277261\\
358	0.00998476612271569\\
359	0.00998476612265779\\
360	0.00998476612259888\\
361	0.00998476612253895\\
362	0.00998476612247797\\
363	0.00998476612241593\\
364	0.00998476612235281\\
365	0.0099847661222886\\
366	0.00998476612222327\\
367	0.0099847661221568\\
368	0.00998476612208917\\
369	0.00998476612202037\\
370	0.00998476612195036\\
371	0.00998476612187914\\
372	0.00998476612180667\\
373	0.00998476612173293\\
374	0.0099847661216579\\
375	0.00998476612158155\\
376	0.00998476612150387\\
377	0.00998476612142482\\
378	0.00998476612134437\\
379	0.0099847661212625\\
380	0.00998476612117919\\
381	0.00998476612109439\\
382	0.00998476612100809\\
383	0.00998476612092025\\
384	0.00998476612083083\\
385	0.0099847661207398\\
386	0.00998476612064715\\
387	0.00998476612055281\\
388	0.00998476612045675\\
389	0.00998476612035895\\
390	0.00998476612025935\\
391	0.00998476612015792\\
392	0.00998476612005461\\
393	0.00998476611994938\\
394	0.00998476611984217\\
395	0.00998476611973294\\
396	0.00998476611962164\\
397	0.0099847661195082\\
398	0.00998476611939257\\
399	0.00998476611927468\\
400	0.00998476611915448\\
401	0.00998476611903188\\
402	0.00998476611890682\\
403	0.00998476611877922\\
404	0.00998476611864899\\
405	0.00998476611851605\\
406	0.00998476611838029\\
407	0.00998476611824164\\
408	0.00998476611809998\\
409	0.0099847661179552\\
410	0.0099847661178072\\
411	0.00998476611765584\\
412	0.00998476611750099\\
413	0.00998476611734252\\
414	0.00998476611718027\\
415	0.00998476611701409\\
416	0.0099847661168438\\
417	0.00998476611666922\\
418	0.00998476611649014\\
419	0.00998476611630634\\
420	0.00998476611611757\\
421	0.00998476611592355\\
422	0.00998476611572399\\
423	0.00998476611551854\\
424	0.0099847661153068\\
425	0.00998476611508829\\
426	0.00998476611486241\\
427	0.00998476611462834\\
428	0.00998476611438484\\
429	0.0099847661141299\\
430	0.0099847661138601\\
431	0.00998476611356959\\
432	0.00998476611324869\\
433	0.0099847661128826\\
434	0.00998476611245178\\
435	0.00998476611193696\\
436	0.00998476611133118\\
437	0.00998476611065367\\
438	0.00998476610994262\\
439	0.00998476610921538\\
440	0.00998476610847146\\
441	0.00998476610771034\\
442	0.00998476610693152\\
443	0.00998476610613445\\
444	0.00998476610531858\\
445	0.00998476610448337\\
446	0.00998476610362824\\
447	0.00998476610275261\\
448	0.00998476610185584\\
449	0.0099847661009372\\
450	0.00998476609999576\\
451	0.00998476609903\\
452	0.00998476609803725\\
453	0.00998476609701211\\
454	0.00998476609594338\\
455	0.00998476609480752\\
456	0.00998476609355561\\
457	0.00998476609208882\\
458	0.00998476609021732\\
459	0.00998476608760555\\
460	0.00998476608373925\\
461	0.00998476607801881\\
462	0.00998476607014061\\
463	0.00998476606073503\\
464	0.00998476605115606\\
465	0.00998476604139841\\
466	0.00998476603145654\\
467	0.00998476602132461\\
468	0.00998476601099649\\
469	0.00998476600046571\\
470	0.00998476598972551\\
471	0.00998476597876875\\
472	0.00998476596758783\\
473	0.00998476595617469\\
474	0.00998476594452093\\
475	0.00998476593261786\\
476	0.00998476592045632\\
477	0.0099847659080265\\
478	0.00998476589531791\\
479	0.00998476588231931\\
480	0.00998476586901861\\
481	0.0099847658554028\\
482	0.00998476584145781\\
483	0.00998476582716846\\
484	0.00998476581251822\\
485	0.00998476579748913\\
486	0.00998476578206153\\
487	0.00998476576621385\\
488	0.00998476574992225\\
489	0.00998476573316013\\
490	0.00998476571589736\\
491	0.0099847656980989\\
492	0.00998476567972207\\
493	0.00998476566071112\\
494	0.00998476564098614\\
495	0.00998476562042082\\
496	0.00998476559879951\\
497	0.0099847655757395\\
498	0.00998476555056474\\
499	0.00998476552213733\\
500	0.00998476548872632\\
501	0.00998476544816116\\
502	0.00998476539873531\\
503	0.00998476534112569\\
504	0.00998476527941104\\
505	0.00998476521657973\\
506	0.00998476515257641\\
507	0.00998476508733086\\
508	0.00998476502076274\\
509	0.00998476495279559\\
510	0.00998476488335207\\
511	0.00998476481234044\\
512	0.00998476473964256\\
513	0.00998476466508539\\
514	0.00998476458837248\\
515	0.0099847645089216\\
516	0.00998476442548878\\
517	0.00998476433532989\\
518	0.00998476423243209\\
519	0.00998476410410157\\
520	0.00998476392536108\\
521	0.00998476365277688\\
522	0.00998476322724478\\
523	0.00998476261074155\\
524	0.00998476185929397\\
525	0.00998476107951758\\
526	0.00998476026608737\\
527	0.00998475941368839\\
528	0.00998475851976169\\
529	0.00998475758590979\\
530	0.00998475660914964\\
531	0.00998475557133194\\
532	0.00998475443618455\\
533	0.00998475313530915\\
534	0.00998475155561873\\
535	0.00998474955614854\\
536	0.00998474707681101\\
537	0.00998474431322528\\
538	0.00998474148463942\\
539	0.00998473858561031\\
540	0.00998473560989778\\
541	0.00998473254958401\\
542	0.00998472939203223\\
543	0.0099847261133409\\
544	0.00998472266845293\\
545	0.00998471897603467\\
546	0.00998471489917038\\
547	0.00998471023443736\\
548	0.00998470478673389\\
549	0.009984698629892\\
550	0.0099846922355815\\
551	0.00998468539895788\\
552	0.00998467761634069\\
553	0.00998466766639784\\
554	0.00998465284881902\\
555	0.00998462839384316\\
556	0.00998451835371763\\
557	0.00998438713828728\\
558	0.00998424091014161\\
559	0.00998407048417542\\
560	0.00998337245231295\\
561	0.0099824607661268\\
562	0.00998150752370361\\
563	0.00998050497714601\\
564	0.00997944065250886\\
565	0.00997829668708648\\
566	0.00996445051854514\\
567	0.00995056165887344\\
568	0.00993686172377713\\
569	0.00992336079454392\\
570	0.00991007448540605\\
571	0.0098970254209339\\
572	0.0098842472847136\\
573	0.00987180223381704\\
574	0.00986010348666058\\
575	0.00984931761907658\\
576	0.00983942691884809\\
577	0.00983036670881545\\
578	0.00982193005125355\\
579	0.0097931048823283\\
580	0.00967661252408177\\
581	0.00930277543452092\\
582	0.00890533237643708\\
583	0.00848053657077329\\
584	0.00802397360009435\\
585	0.00755205906109305\\
586	0.00706559415643771\\
587	0.00656389548252311\\
588	0.00604646101757425\\
589	0.00551317570106527\\
590	0.00496674996040426\\
591	0.00440647859325226\\
592	0.00383129818654383\\
593	0.00323992338914098\\
594	0.0026306372037739\\
595	0.00200104994316786\\
596	0.00135174293718153\\
597	0.000683662792068022\\
598	0\\
599	0\\
600	0\\
};
\addplot [color=mycolor13,solid,forget plot]
  table[row sep=crcr]{%
1	0.000484240466903288\\
2	0.000484240466903338\\
3	0.000484240466903389\\
4	0.000484240466903441\\
5	0.000484240466903493\\
6	0.000484240466903547\\
7	0.000484240466903602\\
8	0.000484240466903656\\
9	0.000484240466903713\\
10	0.000484240466903771\\
11	0.000484240466903829\\
12	0.000484240466903889\\
13	0.00048424046690395\\
14	0.000484240466904013\\
15	0.000484240466904075\\
16	0.000484240466904139\\
17	0.000484240466904205\\
18	0.000484240466904272\\
19	0.000484240466904339\\
20	0.000484240466904408\\
21	0.000484240466904478\\
22	0.000484240466904549\\
23	0.000484240466904622\\
24	0.000484240466904695\\
25	0.000484240466904771\\
26	0.000484240466904848\\
27	0.000484240466904927\\
28	0.000484240466905006\\
29	0.000484240466905086\\
30	0.000484240466905168\\
31	0.000484240466905252\\
32	0.000484240466905337\\
33	0.000484240466905425\\
34	0.000484240466905513\\
35	0.000484240466905602\\
36	0.000484240466905693\\
37	0.000484240466905788\\
38	0.000484240466905882\\
39	0.000484240466905979\\
40	0.000484240466906078\\
41	0.000484240466906178\\
42	0.000484240466906279\\
43	0.000484240466906384\\
44	0.000484240466906488\\
45	0.000484240466906596\\
46	0.000484240466906706\\
47	0.000484240466906817\\
48	0.000484240466906931\\
49	0.000484240466907046\\
50	0.000484240466907165\\
51	0.000484240466907284\\
52	0.000484240466907405\\
53	0.000484240466907529\\
54	0.000484240466907655\\
55	0.000484240466907784\\
56	0.000484240466907914\\
57	0.000484240466908047\\
58	0.000484240466908183\\
59	0.000484240466908321\\
60	0.000484240466908462\\
61	0.000484240466908604\\
62	0.000484240466908751\\
63	0.000484240466908898\\
64	0.000484240466909049\\
65	0.000484240466909203\\
66	0.000484240466909359\\
67	0.000484240466909518\\
68	0.00048424046690968\\
69	0.000484240466909845\\
70	0.000484240466910012\\
71	0.000484240466910183\\
72	0.000484240466910357\\
73	0.000484240466910534\\
74	0.000484240466910714\\
75	0.000484240466910899\\
76	0.000484240466911085\\
77	0.000484240466911275\\
78	0.000484240466911469\\
79	0.000484240466911665\\
80	0.000484240466911866\\
81	0.00048424046691207\\
82	0.000484240466912277\\
83	0.000484240466912489\\
84	0.000484240466912703\\
85	0.000484240466912923\\
86	0.000484240466913147\\
87	0.000484240466913373\\
88	0.000484240466913604\\
89	0.00048424046691384\\
90	0.000484240466914079\\
91	0.000484240466914324\\
92	0.000484240466914572\\
93	0.000484240466914824\\
94	0.000484240466915081\\
95	0.000484240466915342\\
96	0.00048424046691561\\
97	0.00048424046691588\\
98	0.000484240466916157\\
99	0.000484240466916438\\
100	0.000484240466916723\\
101	0.000484240466917015\\
102	0.000484240466917311\\
103	0.000484240466917613\\
104	0.00048424046691792\\
105	0.000484240466918233\\
106	0.000484240466918552\\
107	0.000484240466918875\\
108	0.000484240466919205\\
109	0.00048424046691954\\
110	0.000484240466919883\\
111	0.000484240466920231\\
112	0.000484240466920584\\
113	0.000484240466920946\\
114	0.000484240466921311\\
115	0.000484240466921684\\
116	0.000484240466922066\\
117	0.000484240466922452\\
118	0.000484240466922847\\
119	0.000484240466923247\\
120	0.000484240466923655\\
121	0.000484240466924071\\
122	0.000484240466924494\\
123	0.000484240466924925\\
124	0.000484240466925363\\
125	0.000484240466925809\\
126	0.000484240466926262\\
127	0.000484240466926725\\
128	0.000484240466927195\\
129	0.000484240466927674\\
130	0.000484240466928163\\
131	0.000484240466928658\\
132	0.000484240466929163\\
133	0.000484240466929677\\
134	0.000484240466930202\\
135	0.000484240466930734\\
136	0.000484240466931277\\
137	0.000484240466931828\\
138	0.000484240466932391\\
139	0.000484240466932962\\
140	0.000484240466933544\\
141	0.000484240466934137\\
142	0.00048424046693474\\
143	0.000484240466935354\\
144	0.00048424046693598\\
145	0.000484240466936615\\
146	0.000484240466937264\\
147	0.000484240466937923\\
148	0.000484240466938593\\
149	0.000484240466939276\\
150	0.000484240466939972\\
151	0.00048424046694068\\
152	0.000484240466941401\\
153	0.000484240466942133\\
154	0.00048424046694288\\
155	0.000484240466943641\\
156	0.000484240466944414\\
157	0.000484240466945202\\
158	0.000484240466946003\\
159	0.000484240466946819\\
160	0.000484240466947649\\
161	0.000484240466948494\\
162	0.000484240466949355\\
163	0.000484240466950231\\
164	0.000484240466951122\\
165	0.00048424046695203\\
166	0.000484240466952954\\
167	0.000484240466953896\\
168	0.000484240466954853\\
169	0.000484240466955828\\
170	0.000484240466956819\\
171	0.000484240466957829\\
172	0.000484240466958857\\
173	0.000484240466959902\\
174	0.000484240466960968\\
175	0.000484240466962053\\
176	0.000484240466963156\\
177	0.000484240466964281\\
178	0.000484240466965423\\
179	0.000484240466966587\\
180	0.000484240466967772\\
181	0.000484240466968979\\
182	0.000484240466970208\\
183	0.000484240466971458\\
184	0.00048424046697273\\
185	0.000484240466974026\\
186	0.000484240466975344\\
187	0.000484240466976687\\
188	0.000484240466978053\\
189	0.000484240466979444\\
190	0.00048424046698086\\
191	0.000484240466982302\\
192	0.000484240466983769\\
193	0.000484240466985264\\
194	0.000484240466986784\\
195	0.000484240466988332\\
196	0.000484240466989907\\
197	0.000484240466991513\\
198	0.000484240466993146\\
199	0.000484240466994807\\
200	0.000484240466996501\\
201	0.000484240466998223\\
202	0.000484240466999977\\
203	0.000484240467001762\\
204	0.00048424046700358\\
205	0.00048424046700543\\
206	0.000484240467007315\\
207	0.000484240467009232\\
208	0.000484240467011185\\
209	0.000484240467013172\\
210	0.000484240467015197\\
211	0.000484240467017256\\
212	0.000484240467019352\\
213	0.000484240467021488\\
214	0.000484240467023662\\
215	0.000484240467025873\\
216	0.000484240467028126\\
217	0.00048424046703042\\
218	0.000484240467032754\\
219	0.000484240467035132\\
220	0.000484240467037552\\
221	0.000484240467040016\\
222	0.000484240467042524\\
223	0.000484240467045077\\
224	0.000484240467047678\\
225	0.000484240467050325\\
226	0.00048424046705302\\
227	0.000484240467055764\\
228	0.000484240467058556\\
229	0.0004842404670614\\
230	0.000484240467064296\\
231	0.000484240467067244\\
232	0.000484240467070247\\
233	0.000484240467073302\\
234	0.000484240467076414\\
235	0.000484240467079583\\
236	0.000484240467082807\\
237	0.000484240467086091\\
238	0.000484240467089436\\
239	0.00048424046709284\\
240	0.000484240467096307\\
241	0.000484240467099838\\
242	0.000484240467103431\\
243	0.000484240467107091\\
244	0.000484240467110817\\
245	0.000484240467114611\\
246	0.000484240467118475\\
247	0.000484240467122408\\
248	0.000484240467126413\\
249	0.000484240467130492\\
250	0.000484240467134645\\
251	0.000484240467138874\\
252	0.00048424046714318\\
253	0.000484240467147566\\
254	0.00048424046715203\\
255	0.000484240467156577\\
256	0.000484240467161208\\
257	0.000484240467165922\\
258	0.000484240467170724\\
259	0.000484240467175612\\
260	0.000484240467180592\\
261	0.000484240467185661\\
262	0.000484240467190824\\
263	0.000484240467196083\\
264	0.000484240467201438\\
265	0.00048424046720689\\
266	0.000484240467212444\\
267	0.000484240467218099\\
268	0.000484240467223858\\
269	0.000484240467229725\\
270	0.000484240467235697\\
271	0.000484240467241782\\
272	0.000484240467247977\\
273	0.000484240467254287\\
274	0.000484240467260715\\
275	0.000484240467267261\\
276	0.000484240467273926\\
277	0.000484240467280716\\
278	0.000484240467287631\\
279	0.000484240467294674\\
280	0.000484240467301846\\
281	0.000484240467309154\\
282	0.000484240467316596\\
283	0.000484240467324175\\
284	0.000484240467331895\\
285	0.000484240467339758\\
286	0.000484240467347768\\
287	0.000484240467355925\\
288	0.000484240467364235\\
289	0.000484240467372699\\
290	0.000484240467381321\\
291	0.000484240467390103\\
292	0.000484240467399049\\
293	0.000484240467408161\\
294	0.000484240467417442\\
295	0.000484240467426897\\
296	0.000484240467436528\\
297	0.000484240467446339\\
298	0.000484240467456333\\
299	0.000484240467466513\\
300	0.000484240467476884\\
301	0.000484240467487449\\
302	0.00048424046749821\\
303	0.000484240467509173\\
304	0.00048424046752034\\
305	0.000484240467531717\\
306	0.000484240467543307\\
307	0.000484240467555115\\
308	0.000484240467567143\\
309	0.000484240467579396\\
310	0.000484240467591879\\
311	0.000484240467604594\\
312	0.000484240467617549\\
313	0.000484240467630747\\
314	0.000484240467644192\\
315	0.00048424046765789\\
316	0.000484240467671845\\
317	0.00048424046768606\\
318	0.000484240467700543\\
319	0.000484240467715297\\
320	0.000484240467730329\\
321	0.000484240467745641\\
322	0.000484240467761242\\
323	0.000484240467777135\\
324	0.000484240467793326\\
325	0.000484240467809823\\
326	0.000484240467826628\\
327	0.000484240467843747\\
328	0.000484240467861189\\
329	0.000484240467878957\\
330	0.000484240467897058\\
331	0.0004842404679155\\
332	0.000484240467934286\\
333	0.000484240467953425\\
334	0.000484240467972922\\
335	0.000484240467992787\\
336	0.000484240468013021\\
337	0.000484240468033635\\
338	0.000484240468054636\\
339	0.000484240468076028\\
340	0.000484240468097822\\
341	0.000484240468120023\\
342	0.000484240468142638\\
343	0.000484240468165678\\
344	0.000484240468189149\\
345	0.000484240468213057\\
346	0.000484240468237412\\
347	0.000484240468262221\\
348	0.000484240468287495\\
349	0.000484240468313239\\
350	0.000484240468339464\\
351	0.000484240468366179\\
352	0.000484240468393391\\
353	0.000484240468421111\\
354	0.000484240468449347\\
355	0.00048424046847811\\
356	0.000484240468507407\\
357	0.000484240468537252\\
358	0.000484240468567653\\
359	0.000484240468598618\\
360	0.000484240468630161\\
361	0.000484240468662293\\
362	0.000484240468695021\\
363	0.000484240468728362\\
364	0.000484240468762321\\
365	0.000484240468796915\\
366	0.000484240468832154\\
367	0.00048424046886805\\
368	0.000484240468904617\\
369	0.000484240468941867\\
370	0.000484240468979814\\
371	0.000484240469018471\\
372	0.000484240469057852\\
373	0.000484240469097974\\
374	0.000484240469138849\\
375	0.000484240469180494\\
376	0.000484240469222925\\
377	0.00048424046926616\\
378	0.000484240469310213\\
379	0.000484240469355105\\
380	0.000484240469400852\\
381	0.000484240469447475\\
382	0.000484240469494994\\
383	0.000484240469543428\\
384	0.000484240469592801\\
385	0.000484240469643136\\
386	0.000484240469694458\\
387	0.000484240469746791\\
388	0.000484240469800161\\
389	0.000484240469854599\\
390	0.000484240469910134\\
391	0.000484240469966797\\
392	0.00048424047002462\\
393	0.00048424047008364\\
394	0.000484240470143896\\
395	0.000484240470205425\\
396	0.000484240470268271\\
397	0.000484240470332477\\
398	0.000484240470398088\\
399	0.000484240470465156\\
400	0.000484240470533727\\
401	0.000484240470603858\\
402	0.000484240470675606\\
403	0.00048424047074903\\
404	0.000484240470824197\\
405	0.000484240470901179\\
406	0.000484240470980048\\
407	0.000484240471060881\\
408	0.000484240471143761\\
409	0.000484240471228782\\
410	0.000484240471316043\\
411	0.000484240471405652\\
412	0.000484240471497727\\
413	0.000484240471592404\\
414	0.000484240471689826\\
415	0.000484240471790159\\
416	0.000484240471893589\\
417	0.000484240472000334\\
418	0.00048424047211065\\
419	0.000484240472224868\\
420	0.00048424047234344\\
421	0.000484240472467045\\
422	0.000484240472596777\\
423	0.000484240472734483\\
424	0.000484240472883296\\
425	0.000484240473048316\\
426	0.000484240473237145\\
427	0.000484240473459371\\
428	0.000484240473723402\\
429	0.000484240474029596\\
430	0.000484240474363645\\
431	0.000484240474704589\\
432	0.000484240475052085\\
433	0.000484240475405585\\
434	0.000484240475764482\\
435	0.000484240476128623\\
436	0.000484240476499017\\
437	0.000484240476877848\\
438	0.000484240477266483\\
439	0.000484240477665547\\
440	0.000484240478076063\\
441	0.000484240478499937\\
442	0.000484240478941073\\
443	0.000484240479407776\\
444	0.00048424047991768\\
445	0.000484240480507274\\
446	0.000484240481248529\\
447	0.000484240482273253\\
448	0.000484240483796312\\
449	0.000484240486103035\\
450	0.0004842404894243\\
451	0.000484240493635419\\
452	0.000484240498062313\\
453	0.000484240502573705\\
454	0.000484240507169916\\
455	0.000484240511850491\\
456	0.000484240516613272\\
457	0.000484240521452686\\
458	0.000484240526357104\\
459	0.000484240531306371\\
460	0.000484240536273769\\
461	0.000484240541240519\\
462	0.000484240546228037\\
463	0.000484240551311235\\
464	0.000484240556492951\\
465	0.000484240561776135\\
466	0.000484240567163889\\
467	0.000484240572659478\\
468	0.000484240578266315\\
469	0.000484240583988024\\
470	0.000484240589828514\\
471	0.000484240595792018\\
472	0.000484240601882934\\
473	0.000484240608105622\\
474	0.000484240614464737\\
475	0.000484240620965255\\
476	0.000484240627612496\\
477	0.000484240634412163\\
478	0.000484240641370363\\
479	0.000484240648493667\\
480	0.000484240655789144\\
481	0.000484240663264423\\
482	0.000484240670927752\\
483	0.000484240678788071\\
484	0.000484240686855094\\
485	0.00048424069513941\\
486	0.000484240703652612\\
487	0.000484240712407421\\
488	0.000484240721417891\\
489	0.000484240730699638\\
490	0.000484240740270217\\
491	0.000484240750149749\\
492	0.00048424076036209\\
493	0.000484240770937182\\
494	0.000484240781915921\\
495	0.00048424079336023\\
496	0.000484240805373368\\
497	0.000484240818138671\\
498	0.000484240831986899\\
499	0.000484240847495088\\
500	0.000484240865582196\\
501	0.000484240887462403\\
502	0.00048424091413874\\
503	0.000484240945128151\\
504	0.000484240977435586\\
505	0.000484241010360868\\
506	0.000484241043948233\\
507	0.000484241078241602\\
508	0.000484241113276388\\
509	0.000484241149094674\\
510	0.000484241185749511\\
511	0.000484241223317212\\
512	0.000484241261926852\\
513	0.000484241301830343\\
514	0.000484241343566185\\
515	0.000484241388329228\\
516	0.000484241438764512\\
517	0.000484241500542035\\
518	0.000484241585077113\\
519	0.000484241712968724\\
520	0.000484241914290421\\
521	0.000484242213187589\\
522	0.000484242580237095\\
523	0.000484242952253886\\
524	0.000484243336263891\\
525	0.000484243733975993\\
526	0.000484244147872262\\
527	0.000484244581210334\\
528	0.000484245037161741\\
529	0.000484245516419022\\
530	0.000484246016790978\\
531	0.000484246545701493\\
532	0.000484247119218224\\
533	0.000484247767406508\\
534	0.000484248541306853\\
535	0.000484249512882124\\
536	0.000484250731440822\\
537	0.000484252078883766\\
538	0.000484253458944545\\
539	0.000484254874471593\\
540	0.000484256328862317\\
541	0.000484257826378404\\
542	0.000484259373029604\\
543	0.000484260979213294\\
544	0.00048426266514276\\
545	0.000484264468980179\\
546	0.000484266455953145\\
547	0.000484268725433269\\
548	0.00048427138412627\\
549	0.000484274421921086\\
550	0.000484277630733646\\
551	0.000484281192722443\\
552	0.000484285569045597\\
553	0.000484291838485129\\
554	0.000484302055719857\\
555	0.000484318648444575\\
556	0.00048433717149597\\
557	0.000484358312049961\\
558	0.000484384433837813\\
559	0.000484419741214524\\
560	0.000484467972308084\\
561	0.000484518384319466\\
562	0.000484571833803974\\
563	0.00048463043821816\\
564	0.000484698204292027\\
565	0.000484780473870265\\
566	0.000484872232915441\\
567	0.000484965833244364\\
568	0.000485062002730881\\
569	0.00048516191722124\\
570	0.000485267787354841\\
571	0.000485383204690684\\
572	0.000485511922586414\\
573	0.000485642971766619\\
574	0.000485775812999213\\
575	0.000485914470174083\\
576	0.000486084229404065\\
577	0.000486346843596949\\
578	0.000486874060292007\\
579	0.000499332124294703\\
580	0.000549606817473229\\
581	0.00088240546943094\\
582	0.00124000461749179\\
583	0.00162646793862394\\
584	0.0020440485140991\\
585	0.00247787686417299\\
586	0.00292892988002479\\
587	0.00339798959748976\\
588	0.00388566154600113\\
589	0.00439214851677062\\
590	0.00491474520087469\\
591	0.00545443621534336\\
592	0.00601233132542992\\
593	0.00658977217077654\\
594	0.00718858827196951\\
595	0.00781118959413535\\
596	0.00845712341615153\\
597	0.00912599392411371\\
598	0.00981478197180822\\
599	0\\
600	0\\
};
\addplot [color=mycolor14,solid,forget plot]
  table[row sep=crcr]{%
1	2.808615381101e-05\\
2	2.80861538121573e-05\\
3	2.8086153813325e-05\\
4	2.80861538145132e-05\\
5	2.80861538157218e-05\\
6	2.80861538169526e-05\\
7	2.80861538182056e-05\\
8	2.80861538194807e-05\\
9	2.80861538207797e-05\\
10	2.80861538221009e-05\\
11	2.80861538234459e-05\\
12	2.8086153824813e-05\\
13	2.80861538262075e-05\\
14	2.80861538276258e-05\\
15	2.80861538290697e-05\\
16	2.80861538305392e-05\\
17	2.80861538320342e-05\\
18	2.80861538335565e-05\\
19	2.80861538351061e-05\\
20	2.80861538366846e-05\\
21	2.80861538382905e-05\\
22	2.80861538399236e-05\\
23	2.80861538415874e-05\\
24	2.80861538432801e-05\\
25	2.80861538450036e-05\\
26	2.80861538467577e-05\\
27	2.80861538485443e-05\\
28	2.80861538503615e-05\\
29	2.80861538522111e-05\\
30	2.80861538540948e-05\\
31	2.80861538560109e-05\\
32	2.80861538579611e-05\\
33	2.80861538599471e-05\\
34	2.80861538619689e-05\\
35	2.80861538640265e-05\\
36	2.80861538661198e-05\\
37	2.80861538682524e-05\\
38	2.80861538704208e-05\\
39	2.80861538726301e-05\\
40	2.80861538748769e-05\\
41	2.80861538771646e-05\\
42	2.80861538794933e-05\\
43	2.80861538818645e-05\\
44	2.80861538842767e-05\\
45	2.80861538867332e-05\\
46	2.80861538892323e-05\\
47	2.80861538917774e-05\\
48	2.80861538943669e-05\\
49	2.80861538970023e-05\\
50	2.80861538996855e-05\\
51	2.80861539024165e-05\\
52	2.80861539051969e-05\\
53	2.8086153908025e-05\\
54	2.80861539109059e-05\\
55	2.80861539138363e-05\\
56	2.80861539168213e-05\\
57	2.80861539198574e-05\\
58	2.8086153922948e-05\\
59	2.80861539260949e-05\\
60	2.80861539292981e-05\\
61	2.80861539325574e-05\\
62	2.80861539358748e-05\\
63	2.80861539392518e-05\\
64	2.80861539426902e-05\\
65	2.80861539461883e-05\\
66	2.80861539497494e-05\\
67	2.80861539533754e-05\\
68	2.80861539570643e-05\\
69	2.80861539608198e-05\\
70	2.80861539646418e-05\\
71	2.80861539685319e-05\\
72	2.80861539724919e-05\\
73	2.80861539765236e-05\\
74	2.80861539806268e-05\\
75	2.80861539848016e-05\\
76	2.80861539890532e-05\\
77	2.80861539933798e-05\\
78	2.8086153997783e-05\\
79	2.80861540022647e-05\\
80	2.80861540068265e-05\\
81	2.80861540114701e-05\\
82	2.80861540161973e-05\\
83	2.8086154021008e-05\\
84	2.80861540259039e-05\\
85	2.80861540308885e-05\\
86	2.80861540359617e-05\\
87	2.80861540411252e-05\\
88	2.80861540463809e-05\\
89	2.80861540517302e-05\\
90	2.80861540571751e-05\\
91	2.80861540627171e-05\\
92	2.8086154068358e-05\\
93	2.80861540740994e-05\\
94	2.80861540799432e-05\\
95	2.80861540858909e-05\\
96	2.8086154091946e-05\\
97	2.80861540981085e-05\\
98	2.80861541043802e-05\\
99	2.80861541107643e-05\\
100	2.80861541172626e-05\\
101	2.80861541238769e-05\\
102	2.80861541306088e-05\\
103	2.80861541374617e-05\\
104	2.80861541444357e-05\\
105	2.80861541515341e-05\\
106	2.80861541587604e-05\\
107	2.80861541661145e-05\\
108	2.80861541735999e-05\\
109	2.80861541812199e-05\\
110	2.80861541889746e-05\\
111	2.80861541968674e-05\\
112	2.80861542049017e-05\\
113	2.80861542130792e-05\\
114	2.80861542214033e-05\\
115	2.80861542298757e-05\\
116	2.80861542384981e-05\\
117	2.80861542472757e-05\\
118	2.80861542562083e-05\\
119	2.8086154265303e-05\\
120	2.80861542745578e-05\\
121	2.80861542839781e-05\\
122	2.80861542935653e-05\\
123	2.80861543033248e-05\\
124	2.80861543132598e-05\\
125	2.80861543233705e-05\\
126	2.80861543336618e-05\\
127	2.80861543441355e-05\\
128	2.80861543547967e-05\\
129	2.80861543656489e-05\\
130	2.80861543766955e-05\\
131	2.8086154387938e-05\\
132	2.808615439938e-05\\
133	2.80861544110283e-05\\
134	2.80861544228828e-05\\
135	2.80861544349505e-05\\
136	2.80861544472329e-05\\
137	2.80861544597335e-05\\
138	2.80861544724575e-05\\
139	2.80861544854098e-05\\
140	2.80861544985924e-05\\
141	2.80861545120101e-05\\
142	2.80861545256665e-05\\
143	2.80861545395684e-05\\
144	2.80861545537175e-05\\
145	2.80861545681189e-05\\
146	2.80861545827777e-05\\
147	2.80861545976973e-05\\
148	2.80861546128846e-05\\
149	2.80861546283429e-05\\
150	2.80861546440756e-05\\
151	2.80861546600896e-05\\
152	2.80861546763918e-05\\
153	2.8086154692982e-05\\
154	2.80861547098706e-05\\
155	2.80861547270591e-05\\
156	2.80861547445563e-05\\
157	2.80861547623654e-05\\
158	2.80861547804915e-05\\
159	2.80861547989416e-05\\
160	2.80861548177224e-05\\
161	2.80861548368372e-05\\
162	2.80861548562948e-05\\
163	2.80861548761001e-05\\
164	2.80861548962583e-05\\
165	2.80861549167762e-05\\
166	2.80861549376605e-05\\
167	2.808615495892e-05\\
168	2.80861549805578e-05\\
169	2.80861550025827e-05\\
170	2.80861550250013e-05\\
171	2.80861550478205e-05\\
172	2.80861550710472e-05\\
173	2.80861550946898e-05\\
174	2.80861551187552e-05\\
175	2.80861551432501e-05\\
176	2.80861551681832e-05\\
177	2.80861551935629e-05\\
178	2.80861552193961e-05\\
179	2.80861552456912e-05\\
180	2.80861552724568e-05\\
181	2.80861552997015e-05\\
182	2.80861553274337e-05\\
183	2.8086155355662e-05\\
184	2.80861553843966e-05\\
185	2.80861554136442e-05\\
186	2.80861554434153e-05\\
187	2.80861554737199e-05\\
188	2.80861555045666e-05\\
189	2.80861555359673e-05\\
190	2.80861555679289e-05\\
191	2.80861556004633e-05\\
192	2.80861556335807e-05\\
193	2.80861556672913e-05\\
194	2.80861557016054e-05\\
195	2.80861557365348e-05\\
196	2.80861557720917e-05\\
197	2.80861558082843e-05\\
198	2.80861558451265e-05\\
199	2.80861558826301e-05\\
200	2.80861559208071e-05\\
201	2.80861559596676e-05\\
202	2.80861559992254e-05\\
203	2.80861560394923e-05\\
204	2.80861560804838e-05\\
205	2.80861561222099e-05\\
206	2.80861561646862e-05\\
207	2.80861562079244e-05\\
208	2.808615625194e-05\\
209	2.80861562967465e-05\\
210	2.80861563423577e-05\\
211	2.80861563887889e-05\\
212	2.80861564360554e-05\\
213	2.80861564841725e-05\\
214	2.80861565331522e-05\\
215	2.80861565830149e-05\\
216	2.80861566337761e-05\\
217	2.80861566854492e-05\\
218	2.80861567380514e-05\\
219	2.80861567916031e-05\\
220	2.8086156846118e-05\\
221	2.80861569016166e-05\\
222	2.8086156958114e-05\\
223	2.80861570156292e-05\\
224	2.80861570741826e-05\\
225	2.80861571337911e-05\\
226	2.80861571944754e-05\\
227	2.8086157256254e-05\\
228	2.80861573191475e-05\\
229	2.80861573831781e-05\\
230	2.80861574483645e-05\\
231	2.80861575147271e-05\\
232	2.80861575822899e-05\\
233	2.80861576510749e-05\\
234	2.80861577211026e-05\\
235	2.8086157792397e-05\\
236	2.80861578649818e-05\\
237	2.80861579388793e-05\\
238	2.80861580141149e-05\\
239	2.80861580907143e-05\\
240	2.80861581686996e-05\\
241	2.80861582480999e-05\\
242	2.80861583289389e-05\\
243	2.8086158411244e-05\\
244	2.80861584950424e-05\\
245	2.80861585803614e-05\\
246	2.808615866723e-05\\
247	2.80861587556754e-05\\
248	2.80861588457284e-05\\
249	2.80861589374196e-05\\
250	2.80861590307763e-05\\
251	2.80861591258309e-05\\
252	2.80861592226158e-05\\
253	2.80861593211634e-05\\
254	2.80861594215043e-05\\
255	2.80861595236743e-05\\
256	2.80861596277059e-05\\
257	2.80861597336349e-05\\
258	2.80861598414953e-05\\
259	2.80861599513246e-05\\
260	2.80861600631604e-05\\
261	2.80861601770367e-05\\
262	2.80861602929962e-05\\
263	2.80861604110746e-05\\
264	2.80861605313111e-05\\
265	2.80861606537502e-05\\
266	2.80861607784293e-05\\
267	2.8086160905391e-05\\
268	2.80861610346796e-05\\
269	2.80861611663394e-05\\
270	2.80861613004132e-05\\
271	2.8086161436945e-05\\
272	2.80861615759846e-05\\
273	2.80861617175778e-05\\
274	2.80861618617707e-05\\
275	2.80861620086161e-05\\
276	2.80861621581618e-05\\
277	2.80861623104571e-05\\
278	2.80861624655585e-05\\
279	2.80861626235135e-05\\
280	2.80861627843802e-05\\
281	2.8086162948213e-05\\
282	2.80861631150649e-05\\
283	2.80861632849972e-05\\
284	2.80861634580662e-05\\
285	2.80861636343298e-05\\
286	2.80861638138511e-05\\
287	2.80861639966915e-05\\
288	2.80861641829124e-05\\
289	2.80861643725784e-05\\
290	2.80861645657562e-05\\
291	2.80861647625104e-05\\
292	2.80861649629093e-05\\
293	2.80861651670228e-05\\
294	2.80861653749207e-05\\
295	2.80861655866747e-05\\
296	2.80861658023598e-05\\
297	2.80861660220475e-05\\
298	2.80861662458163e-05\\
299	2.80861664737411e-05\\
300	2.80861667059039e-05\\
301	2.80861669423831e-05\\
302	2.80861671832603e-05\\
303	2.8086167428621e-05\\
304	2.80861676785469e-05\\
305	2.80861679331266e-05\\
306	2.80861681924488e-05\\
307	2.80861684566022e-05\\
308	2.80861687256771e-05\\
309	2.80861689997707e-05\\
310	2.80861692789733e-05\\
311	2.8086169563382e-05\\
312	2.80861698530975e-05\\
313	2.80861701482204e-05\\
314	2.80861704488495e-05\\
315	2.80861707550905e-05\\
316	2.80861710670491e-05\\
317	2.80861713848309e-05\\
318	2.80861717085486e-05\\
319	2.80861720383129e-05\\
320	2.80861723742346e-05\\
321	2.8086172716433e-05\\
322	2.80861730650224e-05\\
323	2.80861734201238e-05\\
324	2.80861737818582e-05\\
325	2.80861741503501e-05\\
326	2.80861745257256e-05\\
327	2.80861749081126e-05\\
328	2.80861752976423e-05\\
329	2.80861756944477e-05\\
330	2.808617609866e-05\\
331	2.80861765104226e-05\\
332	2.80861769298699e-05\\
333	2.8086177357147e-05\\
334	2.80861777923987e-05\\
335	2.80861782357716e-05\\
336	2.80861786874158e-05\\
337	2.8086179147483e-05\\
338	2.80861796161282e-05\\
339	2.80861800935101e-05\\
340	2.80861805797872e-05\\
341	2.80861810751247e-05\\
342	2.80861815796899e-05\\
343	2.8086182093648e-05\\
344	2.80861826171747e-05\\
345	2.80861831504437e-05\\
346	2.80861836936324e-05\\
347	2.8086184246925e-05\\
348	2.80861848105038e-05\\
349	2.80861853845562e-05\\
350	2.80861859692751e-05\\
351	2.80861865648563e-05\\
352	2.8086187171496e-05\\
353	2.80861877893987e-05\\
354	2.8086188418769e-05\\
355	2.80861890598182e-05\\
356	2.80861897127595e-05\\
357	2.80861903778127e-05\\
358	2.80861910551978e-05\\
359	2.80861917451431e-05\\
360	2.80861924478824e-05\\
361	2.80861931636507e-05\\
362	2.80861938926884e-05\\
363	2.80861946352462e-05\\
364	2.80861953915747e-05\\
365	2.80861961619311e-05\\
366	2.80861969465816e-05\\
367	2.80861977457971e-05\\
368	2.80861985598556e-05\\
369	2.80861993890399e-05\\
370	2.80862002336433e-05\\
371	2.80862010939642e-05\\
372	2.80862019703127e-05\\
373	2.80862028630044e-05\\
374	2.80862037723647e-05\\
375	2.80862046987312e-05\\
376	2.808620564245e-05\\
377	2.80862066038774e-05\\
378	2.80862075833849e-05\\
379	2.80862085813546e-05\\
380	2.80862095981818e-05\\
381	2.80862106342774e-05\\
382	2.8086211690071e-05\\
383	2.80862127660042e-05\\
384	2.80862138625407e-05\\
385	2.80862149801645e-05\\
386	2.80862161193821e-05\\
387	2.8086217280727e-05\\
388	2.80862184647584e-05\\
389	2.80862196720662e-05\\
390	2.80862209032726e-05\\
391	2.80862221590288e-05\\
392	2.80862234400219e-05\\
393	2.80862247469712e-05\\
394	2.80862260806475e-05\\
395	2.80862274418707e-05\\
396	2.80862288315067e-05\\
397	2.80862302504661e-05\\
398	2.80862316997153e-05\\
399	2.80862331802619e-05\\
400	2.80862346931698e-05\\
401	2.80862362395503e-05\\
402	2.80862378205697e-05\\
403	2.80862394374588e-05\\
404	2.8086241091532e-05\\
405	2.80862427842111e-05\\
406	2.80862445170288e-05\\
407	2.80862462915791e-05\\
408	2.80862481095242e-05\\
409	2.80862499726541e-05\\
410	2.80862518829092e-05\\
411	2.80862538424e-05\\
412	2.80862558534264e-05\\
413	2.8086257918505e-05\\
414	2.80862600403959e-05\\
415	2.80862622221511e-05\\
416	2.80862644671482e-05\\
417	2.80862667791621e-05\\
418	2.80862691624525e-05\\
419	2.80862716219223e-05\\
420	2.80862741634101e-05\\
421	2.80862767942748e-05\\
422	2.80862795246284e-05\\
423	2.80862823699214e-05\\
424	2.8086285356212e-05\\
425	2.8086288530267e-05\\
426	2.80862919770322e-05\\
427	2.80862958444663e-05\\
428	2.80863003639551e-05\\
429	2.80863058241207e-05\\
430	2.80863124129024e-05\\
431	2.80863198919898e-05\\
432	2.80863275459902e-05\\
433	2.80863353806416e-05\\
434	2.80863434019857e-05\\
435	2.80863516163779e-05\\
436	2.80863600305164e-05\\
437	2.80863686514935e-05\\
438	2.80863774868853e-05\\
439	2.8086386544978e-05\\
440	2.80863958353203e-05\\
441	2.80864053701281e-05\\
442	2.80864151678357e-05\\
443	2.80864252619863e-05\\
444	2.80864357231286e-05\\
445	2.8086446711237e-05\\
446	2.80864585968064e-05\\
447	2.80864722254694e-05\\
448	2.80864894474804e-05\\
449	2.80865140184409e-05\\
450	2.80865526116246e-05\\
451	2.80866141754087e-05\\
452	2.80867023239014e-05\\
453	2.80867965835178e-05\\
454	2.80868926758453e-05\\
455	2.80869906251254e-05\\
456	2.80870904593219e-05\\
457	2.80871922118918e-05\\
458	2.8087295922761e-05\\
459	2.80874016360984e-05\\
460	2.8087509391849e-05\\
461	2.80876192130461e-05\\
462	2.808773110846e-05\\
463	2.80878451184784e-05\\
464	2.80879613047944e-05\\
465	2.80880797319544e-05\\
466	2.80882004670643e-05\\
467	2.80883235804694e-05\\
468	2.80884491466583e-05\\
469	2.80885772429902e-05\\
470	2.80887079505498e-05\\
471	2.80888413562767e-05\\
472	2.80889775553677e-05\\
473	2.80891166493581e-05\\
474	2.80892587363316e-05\\
475	2.80894039207413e-05\\
476	2.80895523139946e-05\\
477	2.80897040350871e-05\\
478	2.8089859211319e-05\\
479	2.80900179791046e-05\\
480	2.80901804848662e-05\\
481	2.80903468860667e-05\\
482	2.80905173523679e-05\\
483	2.8090692066955e-05\\
484	2.80908712280712e-05\\
485	2.80910550508182e-05\\
486	2.80912437692606e-05\\
487	2.80914376388795e-05\\
488	2.80916369395449e-05\\
489	2.80918419790192e-05\\
490	2.80920530972932e-05\\
491	2.80922706722094e-05\\
492	2.80924951278461e-05\\
493	2.80927269487464e-05\\
494	2.80929667067045e-05\\
495	2.80932151172259e-05\\
496	2.80934731650202e-05\\
497	2.80937423869787e-05\\
498	2.80940254996891e-05\\
499	2.80943277284858e-05\\
500	2.80946593912327e-05\\
501	2.8095040150302e-05\\
502	2.80955034844796e-05\\
503	2.80960927022307e-05\\
504	2.80968231164462e-05\\
505	2.80975931042638e-05\\
506	2.80983776764648e-05\\
507	2.80991778365475e-05\\
508	2.80999946916829e-05\\
509	2.81008290390136e-05\\
510	2.81016817629114e-05\\
511	2.81025538510951e-05\\
512	2.8103446463201e-05\\
513	2.81043611010734e-05\\
514	2.81053000457361e-05\\
515	2.81062675292016e-05\\
516	2.81072728711379e-05\\
517	2.81083387599474e-05\\
518	2.81095225156801e-05\\
519	2.81109679733623e-05\\
520	2.81130200084217e-05\\
521	2.81164217137343e-05\\
522	2.8122372809078e-05\\
523	2.81307328451471e-05\\
524	2.81393374669202e-05\\
525	2.8148210089866e-05\\
526	2.81573853040879e-05\\
527	2.81669176615914e-05\\
528	2.81768901519144e-05\\
529	2.81874046473829e-05\\
530	2.81985071666067e-05\\
531	2.82100322058695e-05\\
532	2.82220592867267e-05\\
533	2.82348190650448e-05\\
534	2.82488249099873e-05\\
535	2.82651675017933e-05\\
536	2.82860473335031e-05\\
537	2.83145323942167e-05\\
538	2.83472442182761e-05\\
539	2.8380749358613e-05\\
540	2.84151133427996e-05\\
541	2.84504107378315e-05\\
542	2.8486726090546e-05\\
543	2.85241574117138e-05\\
544	2.85628504144372e-05\\
545	2.8603106863314e-05\\
546	2.86455858784958e-05\\
547	2.86916048197475e-05\\
548	2.87437943798303e-05\\
549	2.88064993319595e-05\\
550	2.88810108054738e-05\\
551	2.89574258384307e-05\\
552	2.90366136841843e-05\\
553	2.9121818143783e-05\\
554	2.92262971949298e-05\\
555	2.93947309519714e-05\\
556	3.11587393730253e-05\\
557	3.31624962695448e-05\\
558	3.52871923224828e-05\\
559	3.75823663834745e-05\\
560	4.1712988773412e-05\\
561	5.80622218823199e-05\\
562	7.5201694526181e-05\\
563	9.32340626589157e-05\\
564	0.000112306196431779\\
565	0.000132645131255909\\
566	0.00030164397021752\\
567	0.000607768803823587\\
568	0.000930822380175926\\
569	0.00127311620720435\\
570	0.00163736940748546\\
571	0.00202680492261195\\
572	0.00244524712103414\\
573	0.00289722057015775\\
574	0.0033697985617141\\
575	0.00386159633509131\\
576	0.00437431846692315\\
577	0.00491016184443514\\
578	0.00547110656601669\\
579	0.00604483533067061\\
580	0.00659986535839362\\
581	0.00686411121575563\\
582	0.00712289222269977\\
583	0.007367743343057\\
584	0.00759482125548611\\
585	0.00782073949169679\\
586	0.00804446638930281\\
587	0.00826483417025616\\
588	0.00848064034958728\\
589	0.00869044844652259\\
590	0.00889569740473372\\
591	0.00909380142313226\\
592	0.00928132122564964\\
593	0.00945516715155288\\
594	0.00961172798547802\\
595	0.00974741425137617\\
596	0.00986072654667908\\
597	0.00994757959173143\\
598	0.0099999191923403\\
599	0\\
600	0\\
};
\addplot [color=mycolor15,solid,forget plot]
  table[row sep=crcr]{%
1	2.90931793024638e-05\\
2	2.9093179327587e-05\\
3	2.90931793531601e-05\\
4	2.90931793791901e-05\\
5	2.90931794056854e-05\\
6	2.90931794326563e-05\\
7	2.90931794601061e-05\\
8	2.90931794880486e-05\\
9	2.90931795164905e-05\\
10	2.90931795454421e-05\\
11	2.90931795749101e-05\\
12	2.90931796049031e-05\\
13	2.90931796354348e-05\\
14	2.90931796665119e-05\\
15	2.9093179698143e-05\\
16	2.909317973034e-05\\
17	2.90931797631131e-05\\
18	2.90931797964726e-05\\
19	2.90931798304269e-05\\
20	2.90931798649881e-05\\
21	2.90931799001681e-05\\
22	2.90931799359752e-05\\
23	2.90931799724233e-05\\
24	2.90931800095226e-05\\
25	2.90931800472849e-05\\
26	2.90931800857222e-05\\
27	2.90931801248464e-05\\
28	2.90931801646695e-05\\
29	2.90931802052034e-05\\
30	2.90931802464617e-05\\
31	2.9093180288458e-05\\
32	2.90931803312044e-05\\
33	2.90931803747144e-05\\
34	2.90931804190017e-05\\
35	2.90931804640798e-05\\
36	2.90931805099642e-05\\
37	2.90931805566668e-05\\
38	2.90931806042046e-05\\
39	2.90931806525913e-05\\
40	2.90931807018439e-05\\
41	2.90931807519743e-05\\
42	2.90931808030013e-05\\
43	2.90931808549385e-05\\
44	2.90931809078047e-05\\
45	2.90931809616153e-05\\
46	2.90931810163854e-05\\
47	2.90931810721357e-05\\
48	2.90931811288797e-05\\
49	2.90931811866379e-05\\
50	2.90931812454274e-05\\
51	2.90931813052686e-05\\
52	2.90931813661768e-05\\
53	2.90931814281724e-05\\
54	2.90931814912744e-05\\
55	2.90931815555047e-05\\
56	2.90931816208822e-05\\
57	2.90931816874256e-05\\
58	2.9093181755157e-05\\
59	2.90931818240987e-05\\
60	2.90931818942728e-05\\
61	2.9093181965698e-05\\
62	2.90931820383982e-05\\
63	2.90931821123955e-05\\
64	2.90931821877155e-05\\
65	2.90931822643805e-05\\
66	2.90931823424124e-05\\
67	2.90931824218387e-05\\
68	2.90931825026832e-05\\
69	2.90931825849697e-05\\
70	2.90931826687255e-05\\
71	2.90931827539762e-05\\
72	2.9093182840749e-05\\
73	2.90931829290696e-05\\
74	2.90931830189685e-05\\
75	2.90931831104697e-05\\
76	2.90931832036056e-05\\
77	2.90931832984034e-05\\
78	2.90931833948938e-05\\
79	2.90931834931058e-05\\
80	2.909318359307e-05\\
81	2.90931836948189e-05\\
82	2.90931837983831e-05\\
83	2.90931839037968e-05\\
84	2.90931840110905e-05\\
85	2.90931841202984e-05\\
86	2.90931842314564e-05\\
87	2.90931843445984e-05\\
88	2.90931844597586e-05\\
89	2.90931845769727e-05\\
90	2.909318469628e-05\\
91	2.90931848177163e-05\\
92	2.90931849413191e-05\\
93	2.90931850671258e-05\\
94	2.90931851951791e-05\\
95	2.90931853255164e-05\\
96	2.90931854581805e-05\\
97	2.90931855932105e-05\\
98	2.90931857306489e-05\\
99	2.9093185870542e-05\\
100	2.90931860129287e-05\\
101	2.90931861578569e-05\\
102	2.90931863053709e-05\\
103	2.90931864555167e-05\\
104	2.90931866083403e-05\\
105	2.90931867638929e-05\\
106	2.90931869222187e-05\\
107	2.9093187083369e-05\\
108	2.90931872473948e-05\\
109	2.90931874143473e-05\\
110	2.90931875842777e-05\\
111	2.90931877572404e-05\\
112	2.909318793329e-05\\
113	2.90931881124794e-05\\
114	2.90931882948648e-05\\
115	2.90931884805041e-05\\
116	2.90931886694554e-05\\
117	2.90931888617782e-05\\
118	2.90931890575323e-05\\
119	2.9093189256779e-05\\
120	2.90931894595796e-05\\
121	2.9093189665999e-05\\
122	2.90931898761001e-05\\
123	2.90931900899495e-05\\
124	2.90931903076154e-05\\
125	2.90931905291642e-05\\
126	2.90931907546642e-05\\
127	2.90931909841902e-05\\
128	2.90931912178088e-05\\
129	2.90931914555968e-05\\
130	2.90931916976273e-05\\
131	2.90931919439771e-05\\
132	2.90931921947213e-05\\
133	2.90931924499398e-05\\
134	2.90931927097111e-05\\
135	2.90931929741188e-05\\
136	2.90931932432446e-05\\
137	2.90931935171721e-05\\
138	2.90931937959883e-05\\
139	2.90931940797799e-05\\
140	2.90931943686358e-05\\
141	2.90931946626461e-05\\
142	2.90931949619031e-05\\
143	2.9093195266502e-05\\
144	2.90931955765367e-05\\
145	2.90931958921044e-05\\
146	2.90931962133055e-05\\
147	2.9093196540239e-05\\
148	2.90931968730071e-05\\
149	2.90931972117172e-05\\
150	2.90931975564716e-05\\
151	2.90931979073828e-05\\
152	2.90931982645565e-05\\
153	2.90931986281085e-05\\
154	2.90931989981497e-05\\
155	2.90931993747994e-05\\
156	2.90931997581735e-05\\
157	2.9093200148393e-05\\
158	2.90932005455807e-05\\
159	2.90932009498626e-05\\
160	2.9093201361365e-05\\
161	2.90932017802156e-05\\
162	2.90932022065474e-05\\
163	2.90932026404951e-05\\
164	2.9093203082195e-05\\
165	2.90932035317853e-05\\
166	2.9093203989409e-05\\
167	2.90932044552076e-05\\
168	2.90932049293313e-05\\
169	2.90932054119247e-05\\
170	2.90932059031448e-05\\
171	2.90932064031433e-05\\
172	2.90932069120787e-05\\
173	2.90932074301129e-05\\
174	2.90932079574061e-05\\
175	2.90932084941271e-05\\
176	2.90932090404447e-05\\
177	2.90932095965309e-05\\
178	2.90932101625632e-05\\
179	2.90932107387169e-05\\
180	2.9093211325178e-05\\
181	2.90932119221289e-05\\
182	2.90932125297603e-05\\
183	2.90932131482633e-05\\
184	2.90932137778338e-05\\
185	2.90932144186731e-05\\
186	2.90932150709804e-05\\
187	2.90932157349655e-05\\
188	2.90932164108381e-05\\
189	2.90932170988112e-05\\
190	2.90932177991047e-05\\
191	2.90932185119401e-05\\
192	2.90932192375442e-05\\
193	2.90932199761454e-05\\
194	2.90932207279824e-05\\
195	2.90932214932903e-05\\
196	2.90932222723164e-05\\
197	2.90932230653042e-05\\
198	2.90932238725096e-05\\
199	2.90932246941883e-05\\
200	2.90932255306027e-05\\
201	2.90932263820186e-05\\
202	2.9093227248709e-05\\
203	2.90932281309498e-05\\
204	2.90932290290242e-05\\
205	2.90932299432148e-05\\
206	2.90932308738202e-05\\
207	2.90932318211317e-05\\
208	2.90932327854579e-05\\
209	2.9093233767104e-05\\
210	2.90932347663886e-05\\
211	2.9093235783627e-05\\
212	2.909323681915e-05\\
213	2.90932378732866e-05\\
214	2.90932389463793e-05\\
215	2.90932400387691e-05\\
216	2.90932411508105e-05\\
217	2.90932422828615e-05\\
218	2.90932434352834e-05\\
219	2.90932446084512e-05\\
220	2.90932458027433e-05\\
221	2.90932470185433e-05\\
222	2.90932482562432e-05\\
223	2.90932495162469e-05\\
224	2.90932507989602e-05\\
225	2.90932521047973e-05\\
226	2.90932534341825e-05\\
227	2.90932547875471e-05\\
228	2.90932561653293e-05\\
229	2.90932575679772e-05\\
230	2.90932589959478e-05\\
231	2.90932604497046e-05\\
232	2.90932619297197e-05\\
233	2.90932634364791e-05\\
234	2.90932649704702e-05\\
235	2.90932665321958e-05\\
236	2.90932681221656e-05\\
237	2.90932697409011e-05\\
238	2.90932713889309e-05\\
239	2.90932730667934e-05\\
240	2.9093274775041e-05\\
241	2.90932765142328e-05\\
242	2.90932782849433e-05\\
243	2.90932800877501e-05\\
244	2.90932819232483e-05\\
245	2.90932837920446e-05\\
246	2.90932856947527e-05\\
247	2.90932876320032e-05\\
248	2.90932896044337e-05\\
249	2.9093291612697e-05\\
250	2.90932936574579e-05\\
251	2.90932957393947e-05\\
252	2.90932978591995e-05\\
253	2.90933000175729e-05\\
254	2.90933022152342e-05\\
255	2.9093304452913e-05\\
256	2.90933067313574e-05\\
257	2.90933090513244e-05\\
258	2.90933114135895e-05\\
259	2.90933138189419e-05\\
260	2.90933162681879e-05\\
261	2.90933187621439e-05\\
262	2.90933213016469e-05\\
263	2.9093323887549e-05\\
264	2.90933265207198e-05\\
265	2.90933292020454e-05\\
266	2.90933319324242e-05\\
267	2.90933347127801e-05\\
268	2.90933375440488e-05\\
269	2.90933404271866e-05\\
270	2.90933433631668e-05\\
271	2.90933463529849e-05\\
272	2.90933493976515e-05\\
273	2.90933524981998e-05\\
274	2.90933556556813e-05\\
275	2.90933588711683e-05\\
276	2.9093362145755e-05\\
277	2.90933654805545e-05\\
278	2.90933688767037e-05\\
279	2.90933723353618e-05\\
280	2.90933758577065e-05\\
281	2.90933794449447e-05\\
282	2.90933830983003e-05\\
283	2.90933868190243e-05\\
284	2.90933906083918e-05\\
285	2.90933944677015e-05\\
286	2.90933983982781e-05\\
287	2.90934024014714e-05\\
288	2.90934064786569e-05\\
289	2.90934106312359e-05\\
290	2.90934148606385e-05\\
291	2.90934191683202e-05\\
292	2.90934235557693e-05\\
293	2.90934280244941e-05\\
294	2.90934325760407e-05\\
295	2.90934372119791e-05\\
296	2.90934419339097e-05\\
297	2.90934467434655e-05\\
298	2.90934516423083e-05\\
299	2.90934566321344e-05\\
300	2.9093461714672e-05\\
301	2.90934668916767e-05\\
302	2.90934721649436e-05\\
303	2.90934775362964e-05\\
304	2.90934830075999e-05\\
305	2.90934885807462e-05\\
306	2.90934942576664e-05\\
307	2.90935000403294e-05\\
308	2.90935059307362e-05\\
309	2.90935119309289e-05\\
310	2.90935180429871e-05\\
311	2.90935242690242e-05\\
312	2.90935306111966e-05\\
313	2.90935370717015e-05\\
314	2.90935436527717e-05\\
315	2.90935503566846e-05\\
316	2.90935571857566e-05\\
317	2.90935641423467e-05\\
318	2.90935712288566e-05\\
319	2.90935784477308e-05\\
320	2.90935858014594e-05\\
321	2.90935932925737e-05\\
322	2.90936009236545e-05\\
323	2.90936086973232e-05\\
324	2.90936166162525e-05\\
325	2.90936246831596e-05\\
326	2.90936329008091e-05\\
327	2.90936412720169e-05\\
328	2.90936497996484e-05\\
329	2.90936584866132e-05\\
330	2.90936673358806e-05\\
331	2.90936763504659e-05\\
332	2.90936855334373e-05\\
333	2.90936948879192e-05\\
334	2.90937044170888e-05\\
335	2.90937141241779e-05\\
336	2.90937240124763e-05\\
337	2.909373408533e-05\\
338	2.9093744346143e-05\\
339	2.90937547983787e-05\\
340	2.9093765445564e-05\\
341	2.90937762912835e-05\\
342	2.90937873391848e-05\\
343	2.90937985929839e-05\\
344	2.90938100564562e-05\\
345	2.9093821733449e-05\\
346	2.9093833627874e-05\\
347	2.90938457437149e-05\\
348	2.90938580850267e-05\\
349	2.90938706559326e-05\\
350	2.90938834606379e-05\\
351	2.90938965034156e-05\\
352	2.9093909788626e-05\\
353	2.90939233207027e-05\\
354	2.9093937104166e-05\\
355	2.90939511436199e-05\\
356	2.90939654437536e-05\\
357	2.90939800093517e-05\\
358	2.90939948452909e-05\\
359	2.909400995654e-05\\
360	2.90940253481715e-05\\
361	2.9094041025364e-05\\
362	2.90940569933997e-05\\
363	2.90940732576754e-05\\
364	2.90940898237051e-05\\
365	2.90941066971224e-05\\
366	2.90941238836903e-05\\
367	2.90941413893049e-05\\
368	2.90941592200037e-05\\
369	2.90941773819708e-05\\
370	2.90941958815436e-05\\
371	2.90942147252217e-05\\
372	2.90942339196784e-05\\
373	2.90942534717711e-05\\
374	2.90942733885413e-05\\
375	2.90942936772436e-05\\
376	2.90943143453424e-05\\
377	2.90943354005319e-05\\
378	2.90943568507555e-05\\
379	2.90943787042138e-05\\
380	2.90944009693871e-05\\
381	2.90944236550557e-05\\
382	2.90944467703202e-05\\
383	2.90944703246275e-05\\
384	2.9094494327796e-05\\
385	2.90945187900515e-05\\
386	2.9094543722063e-05\\
387	2.90945691349873e-05\\
388	2.90945950405195e-05\\
389	2.90946214509568e-05\\
390	2.9094648379261e-05\\
391	2.9094675839103e-05\\
392	2.90947038448594e-05\\
393	2.90947324115871e-05\\
394	2.9094761555059e-05\\
395	2.90947912921322e-05\\
396	2.9094821640816e-05\\
397	2.90948526201269e-05\\
398	2.90948842501369e-05\\
399	2.90949165520221e-05\\
400	2.90949495480855e-05\\
401	2.9094983261762e-05\\
402	2.90950177176112e-05\\
403	2.90950529413199e-05\\
404	2.9095088959847e-05\\
405	2.90951258017984e-05\\
406	2.90951634980899e-05\\
407	2.90952020822798e-05\\
408	2.90952415895099e-05\\
409	2.90952820557793e-05\\
410	2.90953235198916e-05\\
411	2.90953660238076e-05\\
412	2.9095409613039e-05\\
413	2.90954543371108e-05\\
414	2.90955002500981e-05\\
415	2.90955474112519e-05\\
416	2.90955958857304e-05\\
417	2.90956457454964e-05\\
418	2.90956970703915e-05\\
419	2.90957499495869e-05\\
420	2.90958044836598e-05\\
421	2.90958607880629e-05\\
422	2.90959189997387e-05\\
423	2.90959792911682e-05\\
424	2.90960419019107e-05\\
425	2.9096107210368e-05\\
426	2.90961758938051e-05\\
427	2.90962492668808e-05\\
428	2.90963299303535e-05\\
429	2.90964227872562e-05\\
430	2.90965358853274e-05\\
431	2.90966784828055e-05\\
432	2.90968501766455e-05\\
433	2.90970258792738e-05\\
434	2.90972057215506e-05\\
435	2.90973898409771e-05\\
436	2.90975783821051e-05\\
437	2.90977714969531e-05\\
438	2.90979693455173e-05\\
439	2.9098172096481e-05\\
440	2.90983799285325e-05\\
441	2.90985930334534e-05\\
442	2.90988116241307e-05\\
443	2.90990359564183e-05\\
444	2.9099266389285e-05\\
445	2.9099503550225e-05\\
446	2.90997487864562e-05\\
447	2.91000053791739e-05\\
448	2.91002817408142e-05\\
449	2.91005995366793e-05\\
450	2.91010130430726e-05\\
451	2.9101649662234e-05\\
452	2.91027682805687e-05\\
453	2.91047020870107e-05\\
454	2.91068190737854e-05\\
455	2.91089775935504e-05\\
456	2.91111781752481e-05\\
457	2.91134214173341e-05\\
458	2.91157080402251e-05\\
459	2.9118038945718e-05\\
460	2.91204152310904e-05\\
461	2.91228380245918e-05\\
462	2.91253079910998e-05\\
463	2.91278249038759e-05\\
464	2.91303894186689e-05\\
465	2.91330029185693e-05\\
466	2.91356668570307e-05\\
467	2.91383827440669e-05\\
468	2.91411521584032e-05\\
469	2.91439767852071e-05\\
470	2.91468583702275e-05\\
471	2.91497987253e-05\\
472	2.91527997675997e-05\\
473	2.91558636058589e-05\\
474	2.91589925727011e-05\\
475	2.91621888539417e-05\\
476	2.91654547756575e-05\\
477	2.91687928170124e-05\\
478	2.91722056241405e-05\\
479	2.91756960256899e-05\\
480	2.91792670502333e-05\\
481	2.91829219459032e-05\\
482	2.9186664202492e-05\\
483	2.91904975764392e-05\\
484	2.91944261191551e-05\\
485	2.91984542096033e-05\\
486	2.92025865924738e-05\\
487	2.92068284229079e-05\\
488	2.92111853178327e-05\\
489	2.92156634186546e-05\\
490	2.92202694634806e-05\\
491	2.92250108729358e-05\\
492	2.92298958493688e-05\\
493	2.92349335061599e-05\\
494	2.9240134058643e-05\\
495	2.92455091202278e-05\\
496	2.92510722606742e-05\\
497	2.92568402214188e-05\\
498	2.92628358171017e-05\\
499	2.92690951603317e-05\\
500	2.92756857450048e-05\\
501	2.92827506305816e-05\\
502	2.92906101766628e-05\\
503	2.92999676784904e-05\\
504	2.93121927531563e-05\\
505	2.93289709976244e-05\\
506	2.9346987131976e-05\\
507	2.93653466747014e-05\\
508	2.93840716575043e-05\\
509	2.94031900535272e-05\\
510	2.94227202101169e-05\\
511	2.94426823864227e-05\\
512	2.94630985001884e-05\\
513	2.9483992550852e-05\\
514	2.95053915277041e-05\\
515	2.95273272121389e-05\\
516	2.95498412891083e-05\\
517	2.95730008941682e-05\\
518	2.95969477534526e-05\\
519	2.96220575361834e-05\\
520	2.96494673391088e-05\\
521	2.96828537393044e-05\\
522	2.97345261078044e-05\\
523	3.01021053709156e-05\\
524	3.10823745284106e-05\\
525	3.20921802992323e-05\\
526	3.31334756897174e-05\\
527	3.42085687022956e-05\\
528	3.53203790258592e-05\\
529	3.64729006401969e-05\\
530	3.76717780476027e-05\\
531	3.89235042501472e-05\\
532	4.02246349365916e-05\\
533	4.15798328242147e-05\\
534	4.29965025234212e-05\\
535	4.44859984380255e-05\\
536	4.60685486942687e-05\\
537	4.77956073358063e-05\\
538	5.46164785367609e-05\\
539	6.53176031175611e-05\\
540	7.63862330248849e-05\\
541	8.78552200880326e-05\\
542	9.9762079827913e-05\\
543	0.000112149796931475\\
544	0.000125067565715275\\
545	0.000138572067831741\\
546	0.00015273059037222\\
547	0.000167626377004082\\
548	0.000183363584696547\\
549	0.000200089867442313\\
550	0.000259094817245577\\
551	0.000488015275408033\\
552	0.000726542020146932\\
553	0.000975629408224585\\
554	0.00123636589032539\\
555	0.00150999973544108\\
556	0.00179648290113427\\
557	0.00209887107636958\\
558	0.00241929322761919\\
559	0.00276003182189189\\
560	0.00312210450741178\\
561	0.00349818716082299\\
562	0.00390375972898349\\
563	0.00433457550024097\\
564	0.00478459859119485\\
565	0.00525596815664969\\
566	0.00560062013606284\\
567	0.00582740021304894\\
568	0.00605605899399815\\
569	0.00628510015539956\\
570	0.00651276544529824\\
571	0.00673634212359634\\
572	0.0069520380325136\\
573	0.0071547202173806\\
574	0.00735717499191341\\
575	0.00756064332953579\\
576	0.0077619994239701\\
577	0.00795658404551624\\
578	0.00814198173030071\\
579	0.00831546642819351\\
580	0.00847403078801572\\
581	0.00861655664322103\\
582	0.00875071443575307\\
583	0.00887968108874371\\
584	0.00900452361312009\\
585	0.00912444387945188\\
586	0.0092381478141785\\
587	0.00934511367152276\\
588	0.009443897894158\\
589	0.0095340619381977\\
590	0.00961580002830814\\
591	0.00968902107613021\\
592	0.00975458929242902\\
593	0.00981291410532457\\
594	0.00986442025097488\\
595	0.00990933460668555\\
596	0.00994779104554705\\
597	0.00997906286423442\\
598	0.0099999191923403\\
599	0\\
600	0\\
};
\addplot [color=mycolor16,solid,forget plot]
  table[row sep=crcr]{%
1	2.93043087117404e-05\\
2	2.93043092703793e-05\\
3	2.93043098390084e-05\\
4	2.93043104178049e-05\\
5	2.93043110069512e-05\\
6	2.93043116066297e-05\\
7	2.93043122170296e-05\\
8	2.93043128383435e-05\\
9	2.9304313470764e-05\\
10	2.93043141144905e-05\\
11	2.93043147697259e-05\\
12	2.93043154366748e-05\\
13	2.93043161155449e-05\\
14	2.93043168065513e-05\\
15	2.93043175099102e-05\\
16	2.93043182258399e-05\\
17	2.93043189545671e-05\\
18	2.93043196963202e-05\\
19	2.93043204513311e-05\\
20	2.93043212198348e-05\\
21	2.93043220020752e-05\\
22	2.93043227982943e-05\\
23	2.93043236087444e-05\\
24	2.9304324433676e-05\\
25	2.93043252733517e-05\\
26	2.93043261280323e-05\\
27	2.93043269979852e-05\\
28	2.93043278834833e-05\\
29	2.93043287848044e-05\\
30	2.93043297022315e-05\\
31	2.93043306360509e-05\\
32	2.9304331586554e-05\\
33	2.93043325540409e-05\\
34	2.93043335388133e-05\\
35	2.93043345411797e-05\\
36	2.93043355614537e-05\\
37	2.93043365999557e-05\\
38	2.93043376570097e-05\\
39	2.9304338732948e-05\\
40	2.93043398281064e-05\\
41	2.93043409428274e-05\\
42	2.93043420774606e-05\\
43	2.93043432323622e-05\\
44	2.93043444078918e-05\\
45	2.93043456044193e-05\\
46	2.93043468223162e-05\\
47	2.93043480619679e-05\\
48	2.93043493237578e-05\\
49	2.93043506080848e-05\\
50	2.93043519153494e-05\\
51	2.9304353245959e-05\\
52	2.93043546003329e-05\\
53	2.93043559788922e-05\\
54	2.93043573820698e-05\\
55	2.93043588103037e-05\\
56	2.93043602640422e-05\\
57	2.93043617437405e-05\\
58	2.93043632498588e-05\\
59	2.93043647828691e-05\\
60	2.93043663432522e-05\\
61	2.93043679314956e-05\\
62	2.9304369548097e-05\\
63	2.93043711935609e-05\\
64	2.93043728684021e-05\\
65	2.93043745731439e-05\\
66	2.93043763083197e-05\\
67	2.93043780744734e-05\\
68	2.93043798721572e-05\\
69	2.93043817019318e-05\\
70	2.930438356437e-05\\
71	2.93043854600548e-05\\
72	2.93043873895775e-05\\
73	2.93043893535434e-05\\
74	2.93043913525642e-05\\
75	2.93043933872674e-05\\
76	2.93043954582886e-05\\
77	2.93043975662756e-05\\
78	2.93043997118847e-05\\
79	2.9304401895789e-05\\
80	2.93044041186721e-05\\
81	2.93044063812278e-05\\
82	2.93044086841617e-05\\
83	2.93044110281965e-05\\
84	2.93044134140635e-05\\
85	2.93044158425092e-05\\
86	2.93044183142939e-05\\
87	2.93044208301863e-05\\
88	2.93044233909772e-05\\
89	2.93044259974645e-05\\
90	2.93044286504645e-05\\
91	2.9304431350804e-05\\
92	2.930443409933e-05\\
93	2.93044368969e-05\\
94	2.93044397443884e-05\\
95	2.93044426426867e-05\\
96	2.93044455926998e-05\\
97	2.93044485953501e-05\\
98	2.93044516515749e-05\\
99	2.93044547623304e-05\\
100	2.93044579285901e-05\\
101	2.93044611513407e-05\\
102	2.93044644315931e-05\\
103	2.93044677703717e-05\\
104	2.93044711687196e-05\\
105	2.93044746276987e-05\\
106	2.93044781483897e-05\\
107	2.93044817318955e-05\\
108	2.93044853793341e-05\\
109	2.93044890918477e-05\\
110	2.93044928705969e-05\\
111	2.93044967167614e-05\\
112	2.93045006315444e-05\\
113	2.93045046161699e-05\\
114	2.93045086718838e-05\\
115	2.93045127999542e-05\\
116	2.93045170016716e-05\\
117	2.93045212783518e-05\\
118	2.93045256313294e-05\\
119	2.93045300619681e-05\\
120	2.93045345716518e-05\\
121	2.9304539161792e-05\\
122	2.93045438338238e-05\\
123	2.93045485892098e-05\\
124	2.93045534294344e-05\\
125	2.93045583560166e-05\\
126	2.93045633704937e-05\\
127	2.93045684744355e-05\\
128	2.9304573669441e-05\\
129	2.93045789571328e-05\\
130	2.93045843391694e-05\\
131	2.93045898172315e-05\\
132	2.93045953930355e-05\\
133	2.93046010683268e-05\\
134	2.93046068448816e-05\\
135	2.93046127245083e-05\\
136	2.93046187090495e-05\\
137	2.93046248003767e-05\\
138	2.9304631000399e-05\\
139	2.93046373110579e-05\\
140	2.93046437343306e-05\\
141	2.93046502722301e-05\\
142	2.93046569268019e-05\\
143	2.93046637001356e-05\\
144	2.930467059435e-05\\
145	2.93046776116064e-05\\
146	2.9304684754107e-05\\
147	2.93046920240916e-05\\
148	2.93046994238357e-05\\
149	2.93047069556625e-05\\
150	2.93047146219363e-05\\
151	2.93047224250586e-05\\
152	2.93047303674788e-05\\
153	2.93047384516907e-05\\
154	2.93047466802305e-05\\
155	2.93047550556807e-05\\
156	2.93047635806729e-05\\
157	2.93047722578833e-05\\
158	2.93047810900391e-05\\
159	2.93047900799118e-05\\
160	2.93047992303276e-05\\
161	2.93048085441635e-05\\
162	2.93048180243463e-05\\
163	2.93048276738572e-05\\
164	2.93048374957319e-05\\
165	2.93048474930573e-05\\
166	2.93048576689833e-05\\
167	2.93048680267076e-05\\
168	2.93048785694943e-05\\
169	2.93048893006622e-05\\
170	2.93049002235912e-05\\
171	2.93049113417229e-05\\
172	2.93049226585599e-05\\
173	2.93049341776698e-05\\
174	2.93049459026849e-05\\
175	2.93049578373041e-05\\
176	2.93049699852926e-05\\
177	2.93049823504821e-05\\
178	2.93049949367794e-05\\
179	2.93050077481577e-05\\
180	2.93050207886653e-05\\
181	2.93050340624202e-05\\
182	2.93050475736206e-05\\
183	2.93050613265381e-05\\
184	2.93050753255261e-05\\
185	2.9305089575011e-05\\
186	2.93051040795048e-05\\
187	2.93051188436029e-05\\
188	2.93051338719807e-05\\
189	2.93051491694023e-05\\
190	2.93051647407171e-05\\
191	2.93051805908648e-05\\
192	2.93051967248771e-05\\
193	2.93052131478743e-05\\
194	2.93052298650723e-05\\
195	2.93052468817857e-05\\
196	2.93052642034196e-05\\
197	2.93052818354881e-05\\
198	2.93052997835972e-05\\
199	2.9305318053464e-05\\
200	2.93053366509077e-05\\
201	2.93053555818531e-05\\
202	2.93053748523359e-05\\
203	2.93053944685061e-05\\
204	2.93054144366209e-05\\
205	2.93054347630603e-05\\
206	2.93054554543167e-05\\
207	2.93054765170072e-05\\
208	2.93054979578662e-05\\
209	2.93055197837596e-05\\
210	2.93055420016758e-05\\
211	2.93055646187347e-05\\
212	2.93055876421871e-05\\
213	2.93056110794205e-05\\
214	2.93056349379604e-05\\
215	2.93056592254701e-05\\
216	2.93056839497564e-05\\
217	2.93057091187743e-05\\
218	2.93057347406236e-05\\
219	2.93057608235576e-05\\
220	2.93057873759846e-05\\
221	2.93058144064681e-05\\
222	2.93058419237318e-05\\
223	2.93058699366683e-05\\
224	2.93058984543284e-05\\
225	2.93059274859404e-05\\
226	2.9305957040903e-05\\
227	2.93059871287905e-05\\
228	2.93060177593578e-05\\
229	2.93060489425456e-05\\
230	2.93060806884822e-05\\
231	2.93061130074815e-05\\
232	2.93061459100605e-05\\
233	2.93061794069286e-05\\
234	2.93062135089981e-05\\
235	2.93062482273944e-05\\
236	2.93062835734457e-05\\
237	2.93063195586984e-05\\
238	2.93063561949187e-05\\
239	2.93063934940963e-05\\
240	2.93064314684473e-05\\
241	2.93064701304198e-05\\
242	2.93065094927023e-05\\
243	2.93065495682236e-05\\
244	2.9306590370156e-05\\
245	2.93066319119261e-05\\
246	2.93066742072177e-05\\
247	2.93067172699755e-05\\
248	2.930676111441e-05\\
249	2.93068057550044e-05\\
250	2.930685120652e-05\\
251	2.93068974840007e-05\\
252	2.93069446027787e-05\\
253	2.93069925784809e-05\\
254	2.93070414270343e-05\\
255	2.93070911646745e-05\\
256	2.93071418079455e-05\\
257	2.93071933737134e-05\\
258	2.93072458791667e-05\\
259	2.93072993418294e-05\\
260	2.9307353779558e-05\\
261	2.93074092105568e-05\\
262	2.93074656533828e-05\\
263	2.93075231269477e-05\\
264	2.93075816505345e-05\\
265	2.93076412437948e-05\\
266	2.93077019267599e-05\\
267	2.93077637198516e-05\\
268	2.93078266438872e-05\\
269	2.93078907200863e-05\\
270	2.93079559700773e-05\\
271	2.93080224159134e-05\\
272	2.93080900800703e-05\\
273	2.93081589854635e-05\\
274	2.93082291554465e-05\\
275	2.93083006138332e-05\\
276	2.93083733848958e-05\\
277	2.93084474933771e-05\\
278	2.93085229644989e-05\\
279	2.93085998239753e-05\\
280	2.93086780980152e-05\\
281	2.93087578133384e-05\\
282	2.93088389971796e-05\\
283	2.93089216773039e-05\\
284	2.93090058820111e-05\\
285	2.9309091640152e-05\\
286	2.93091789811311e-05\\
287	2.93092679349241e-05\\
288	2.93093585320843e-05\\
289	2.93094508037568e-05\\
290	2.93095447816828e-05\\
291	2.93096404982194e-05\\
292	2.93097379863436e-05\\
293	2.93098372796669e-05\\
294	2.93099384124448e-05\\
295	2.93100414195927e-05\\
296	2.93101463366906e-05\\
297	2.93102532000003e-05\\
298	2.93103620464757e-05\\
299	2.93104729137747e-05\\
300	2.93105858402726e-05\\
301	2.93107008650693e-05\\
302	2.93108180280113e-05\\
303	2.93109373696914e-05\\
304	2.93110589314767e-05\\
305	2.93111827555043e-05\\
306	2.93113088847092e-05\\
307	2.93114373628293e-05\\
308	2.93115682344172e-05\\
309	2.93117015448571e-05\\
310	2.93118373403806e-05\\
311	2.93119756680697e-05\\
312	2.93121165758844e-05\\
313	2.93122601126605e-05\\
314	2.93124063281393e-05\\
315	2.93125552729684e-05\\
316	2.93127069987215e-05\\
317	2.93128615579127e-05\\
318	2.93130190040092e-05\\
319	2.93131793914446e-05\\
320	2.93133427756377e-05\\
321	2.93135092130009e-05\\
322	2.93136787609609e-05\\
323	2.93138514779671e-05\\
324	2.93140274235173e-05\\
325	2.93142066581623e-05\\
326	2.93143892435288e-05\\
327	2.93145752423309e-05\\
328	2.93147647183921e-05\\
329	2.93149577366578e-05\\
330	2.9315154363217e-05\\
331	2.93153546653141e-05\\
332	2.93155587113752e-05\\
333	2.93157665710192e-05\\
334	2.9315978315082e-05\\
335	2.93161940156423e-05\\
336	2.93164137460263e-05\\
337	2.93166375808508e-05\\
338	2.93168655960277e-05\\
339	2.93170978687985e-05\\
340	2.93173344777514e-05\\
341	2.93175755028535e-05\\
342	2.93178210254676e-05\\
343	2.93180711283923e-05\\
344	2.93183258958727e-05\\
345	2.93185854136492e-05\\
346	2.93188497689687e-05\\
347	2.93191190506294e-05\\
348	2.93193933490078e-05\\
349	2.93196727560963e-05\\
350	2.93199573655373e-05\\
351	2.9320247272664e-05\\
352	2.93205425745397e-05\\
353	2.93208433700023e-05\\
354	2.93211497597115e-05\\
355	2.93214618461989e-05\\
356	2.93217797339203e-05\\
357	2.93221035293106e-05\\
358	2.93224333408535e-05\\
359	2.93227692791394e-05\\
360	2.93231114569356e-05\\
361	2.93234599892746e-05\\
362	2.93238149935226e-05\\
363	2.93241765894779e-05\\
364	2.93245448994635e-05\\
365	2.93249200484307e-05\\
366	2.93253021640719e-05\\
367	2.93256913769481e-05\\
368	2.93260878206101e-05\\
369	2.93264916317505e-05\\
370	2.9326902950348e-05\\
371	2.93273219198469e-05\\
372	2.9327748687329e-05\\
373	2.93281834037115e-05\\
374	2.93286262239667e-05\\
375	2.93290773073439e-05\\
376	2.93295368176299e-05\\
377	2.93300049234186e-05\\
378	2.93304817984106e-05\\
379	2.9330967621743e-05\\
380	2.93314625783413e-05\\
381	2.93319668593139e-05\\
382	2.93324806623867e-05\\
383	2.93330041923679e-05\\
384	2.93335376616909e-05\\
385	2.93340812909987e-05\\
386	2.93346353098155e-05\\
387	2.93351999573295e-05\\
388	2.93357754832963e-05\\
389	2.93363621491538e-05\\
390	2.9336960229354e-05\\
391	2.9337570012906e-05\\
392	2.93381918048516e-05\\
393	2.93388259268588e-05\\
394	2.9339472715923e-05\\
395	2.93401325229076e-05\\
396	2.93408057222732e-05\\
397	2.93414927158237e-05\\
398	2.93421939278154e-05\\
399	2.93429098063191e-05\\
400	2.9343640824428e-05\\
401	2.93443874811739e-05\\
402	2.93451503019373e-05\\
403	2.9345929838172e-05\\
404	2.93467266666941e-05\\
405	2.93475413900327e-05\\
406	2.93483746418173e-05\\
407	2.93492271025257e-05\\
408	2.93500995193379e-05\\
409	2.93509926816469e-05\\
410	2.93519073836872e-05\\
411	2.93528444802147e-05\\
412	2.93538048938463e-05\\
413	2.93547896235088e-05\\
414	2.93557997541904e-05\\
415	2.93568364682357e-05\\
416	2.93579010584453e-05\\
417	2.9358994943368e-05\\
418	2.93601196852197e-05\\
419	2.93612770111704e-05\\
420	2.93624688391245e-05\\
421	2.93636973104335e-05\\
422	2.93649648350278e-05\\
423	2.93662741629849e-05\\
424	2.93676285195353e-05\\
425	2.93690319025862e-05\\
426	2.93704898071103e-05\\
427	2.9372011068031e-05\\
428	2.93736125641227e-05\\
429	2.93753308720823e-05\\
430	2.93772491091177e-05\\
431	2.937954906549e-05\\
432	2.93825649121727e-05\\
433	2.93865647993176e-05\\
434	2.93906580715017e-05\\
435	2.93948477653989e-05\\
436	2.93991370711979e-05\\
437	2.94035293415676e-05\\
438	2.94080281007108e-05\\
439	2.94126370534127e-05\\
440	2.94173600941866e-05\\
441	2.94222013171684e-05\\
442	2.94271650294862e-05\\
443	2.94322557769839e-05\\
444	2.94374784106811e-05\\
445	2.94428382829006e-05\\
446	2.94483418539612e-05\\
447	2.94539986059908e-05\\
448	2.94598271691304e-05\\
449	2.9465875226017e-05\\
450	2.9472285212237e-05\\
451	2.94795147135854e-05\\
452	2.94890877018201e-05\\
453	2.95061946780885e-05\\
454	2.97137528650851e-05\\
455	2.99643177527541e-05\\
456	3.02198907130476e-05\\
457	3.04805329288714e-05\\
458	3.07463097120086e-05\\
459	3.10172934108331e-05\\
460	3.12935673858202e-05\\
461	3.15752306426986e-05\\
462	3.18623999270261e-05\\
463	3.21551991485585e-05\\
464	3.24537260405704e-05\\
465	3.2758119079863e-05\\
466	3.3068553554098e-05\\
467	3.33852152092756e-05\\
468	3.37083005507875e-05\\
469	3.40380176623664e-05\\
470	3.43745878109453e-05\\
471	3.47182432694875e-05\\
472	3.50692271046537e-05\\
473	3.54277931937303e-05\\
474	3.57942075380966e-05\\
475	3.61687510396599e-05\\
476	3.65517064958356e-05\\
477	3.69433747872977e-05\\
478	3.73440764709209e-05\\
479	3.77541535261776e-05\\
480	3.817397129973e-05\\
481	3.86039206799129e-05\\
482	3.90444205384928e-05\\
483	3.9495920483335e-05\\
484	3.99589039722751e-05\\
485	4.04338918478452e-05\\
486	4.09214463707476e-05\\
487	4.14221758634367e-05\\
488	4.19367400822611e-05\\
489	4.24658563771034e-05\\
490	4.30103069572825e-05\\
491	4.3570947296266e-05\\
492	4.41487160335469e-05\\
493	4.47446463784222e-05\\
494	4.53598795319635e-05\\
495	4.59956810291058e-05\\
496	4.66534600498183e-05\\
497	4.73347933366168e-05\\
498	4.80414565567155e-05\\
499	4.87754703249452e-05\\
500	4.9539181367542e-05\\
501	5.033544070285e-05\\
502	5.11680726456097e-05\\
503	5.20432562481787e-05\\
504	5.29738500783278e-05\\
505	5.39938867634016e-05\\
506	5.87673186764253e-05\\
507	6.4606815017149e-05\\
508	7.0590274701469e-05\\
509	7.67252049983633e-05\\
510	8.3019714326634e-05\\
511	8.9481993182314e-05\\
512	9.61209472369967e-05\\
513	0.000102946251576546\\
514	0.000109968443091134\\
515	0.000117199021355419\\
516	0.000124650541318546\\
517	0.000132336703579475\\
518	0.000140272419398281\\
519	0.000148473773838568\\
520	0.000156957625963504\\
521	0.000165739941223944\\
522	0.000174829676150907\\
523	0.000183950961874965\\
524	0.000192825752632742\\
525	0.000202059447372597\\
526	0.000211688618436699\\
527	0.000221755970778601\\
528	0.000232312204801424\\
529	0.000243420847026636\\
530	0.000255169619523371\\
531	0.000267702889893964\\
532	0.000412746538030032\\
533	0.000576195888250494\\
534	0.000745765199194563\\
535	0.000921788564508206\\
536	0.00110457820815963\\
537	0.0012943489400593\\
538	0.00148633721617432\\
539	0.00168191902455894\\
540	0.00188511074741496\\
541	0.00209666253199608\\
542	0.00231744598211593\\
543	0.00254847920196742\\
544	0.00279095733672176\\
545	0.00304628750794034\\
546	0.00331615021568388\\
547	0.0036026079092931\\
548	0.00390820819911889\\
549	0.00423590450669407\\
550	0.00454732922748713\\
551	0.00471530322788151\\
552	0.00489204389082255\\
553	0.00507209484026282\\
554	0.00525506600206778\\
555	0.00544042211683775\\
556	0.00562744957463521\\
557	0.00581514029179541\\
558	0.0060021940002762\\
559	0.00618708573928175\\
560	0.00636777611433338\\
561	0.00654145229614157\\
562	0.00670414805441769\\
563	0.00686023270116887\\
564	0.00701594208881994\\
565	0.00716944104891064\\
566	0.00731902377334951\\
567	0.00746707784675243\\
568	0.00761602253010009\\
569	0.00776477719611707\\
570	0.00791099048273329\\
571	0.00805137960956685\\
572	0.0081854601127217\\
573	0.00831318779882302\\
574	0.0084344350629761\\
575	0.00854874215821512\\
576	0.00865665372359101\\
577	0.00876045240599831\\
578	0.0088600806646344\\
579	0.00895569983895539\\
580	0.00904732770420615\\
581	0.00913419155627037\\
582	0.00921661474496478\\
583	0.00929486551170983\\
584	0.00936833250999321\\
585	0.00943643823988877\\
586	0.00949976142125213\\
587	0.00955824735515365\\
588	0.00961252025824673\\
589	0.00966324058718606\\
590	0.00971085603739681\\
591	0.00975602317557434\\
592	0.00979895351145169\\
593	0.00983980399718307\\
594	0.00987864202528269\\
595	0.00991535020799166\\
596	0.00994937493726701\\
597	0.00997906286423442\\
598	0.0099999191923403\\
599	0\\
600	0\\
};
\addplot [color=mycolor17,solid,forget plot]
  table[row sep=crcr]{%
1	4.85195228051361e-05\\
2	4.85195905472055e-05\\
3	4.85196595005982e-05\\
4	4.85197296869432e-05\\
5	4.85198011282528e-05\\
6	4.85198738469381e-05\\
7	4.85199478658032e-05\\
8	4.85200232080614e-05\\
9	4.85200998973416e-05\\
10	4.8520177957688e-05\\
11	4.85202574135813e-05\\
12	4.85203382899344e-05\\
13	4.85204206121018e-05\\
14	4.85205044058957e-05\\
15	4.85205896975854e-05\\
16	4.85206765139064e-05\\
17	4.8520764882073e-05\\
18	4.85208548297851e-05\\
19	4.85209463852382e-05\\
20	4.85210395771233e-05\\
21	4.85211344346477e-05\\
22	4.85212309875416e-05\\
23	4.85213292660615e-05\\
24	4.85214293010004e-05\\
25	4.85215311237068e-05\\
26	4.85216347660825e-05\\
27	4.85217402606018e-05\\
28	4.8521847640313e-05\\
29	4.85219569388589e-05\\
30	4.85220681904767e-05\\
31	4.85221814300167e-05\\
32	4.85222966929513e-05\\
33	4.85224140153793e-05\\
34	4.85225334340487e-05\\
35	4.85226549863615e-05\\
36	4.85227787103824e-05\\
37	4.85229046448591e-05\\
38	4.85230328292256e-05\\
39	4.85231633036214e-05\\
40	4.85232961088994e-05\\
41	4.85234312866416e-05\\
42	4.85235688791694e-05\\
43	4.85237089295637e-05\\
44	4.85238514816636e-05\\
45	4.85239965800966e-05\\
46	4.85241442702842e-05\\
47	4.85242945984534e-05\\
48	4.85244476116607e-05\\
49	4.85246033577975e-05\\
50	4.85247618856084e-05\\
51	4.85249232447049e-05\\
52	4.85250874855878e-05\\
53	4.85252546596556e-05\\
54	4.85254248192179e-05\\
55	4.85255980175213e-05\\
56	4.85257743087629e-05\\
57	4.85259537480985e-05\\
58	4.85261363916739e-05\\
59	4.85263222966311e-05\\
60	4.85265115211309e-05\\
61	4.85267041243712e-05\\
62	4.85269001666046e-05\\
63	4.85270997091599e-05\\
64	4.85273028144527e-05\\
65	4.85275095460144e-05\\
66	4.85277199685125e-05\\
67	4.85279341477591e-05\\
68	4.8528152150745e-05\\
69	4.85283740456517e-05\\
70	4.85285999018786e-05\\
71	4.852882979006e-05\\
72	4.85290637820906e-05\\
73	4.85293019511444e-05\\
74	4.85295443717018e-05\\
75	4.85297911195685e-05\\
76	4.85300422719043e-05\\
77	4.85302979072419e-05\\
78	4.85305581055176e-05\\
79	4.85308229480864e-05\\
80	4.85310925177616e-05\\
81	4.85313668988247e-05\\
82	4.85316461770697e-05\\
83	4.85319304398101e-05\\
84	4.85322197759229e-05\\
85	4.85325142758694e-05\\
86	4.85328140317205e-05\\
87	4.85331191371924e-05\\
88	4.85334296876743e-05\\
89	4.85337457802515e-05\\
90	4.85340675137454e-05\\
91	4.85343949887367e-05\\
92	4.85347283076049e-05\\
93	4.85350675745539e-05\\
94	4.8535412895644e-05\\
95	4.85357643788334e-05\\
96	4.85361221340029e-05\\
97	4.85364862729943e-05\\
98	4.85368569096487e-05\\
99	4.85372341598347e-05\\
100	4.85376181414916e-05\\
101	4.85380089746661e-05\\
102	4.85384067815443e-05\\
103	4.85388116864975e-05\\
104	4.85392238161132e-05\\
105	4.85396432992459e-05\\
106	4.85400702670465e-05\\
107	4.8540504853011e-05\\
108	4.85409471930187e-05\\
109	4.85413974253778e-05\\
110	4.85418556908663e-05\\
111	4.85423221327831e-05\\
112	4.85427968969806e-05\\
113	4.85432801319221e-05\\
114	4.85437719887268e-05\\
115	4.85442726212134e-05\\
116	4.854478218595e-05\\
117	4.85453008423083e-05\\
118	4.8545828752508e-05\\
119	4.85463660816713e-05\\
120	4.85469129978757e-05\\
121	4.85474696722051e-05\\
122	4.85480362788079e-05\\
123	4.85486129949458e-05\\
124	4.85492000010577e-05\\
125	4.85497974808099e-05\\
126	4.85504056211634e-05\\
127	4.85510246124225e-05\\
128	4.85516546483036e-05\\
129	4.85522959259959e-05\\
130	4.85529486462182e-05\\
131	4.85536130132935e-05\\
132	4.85542892352019e-05\\
133	4.8554977523657e-05\\
134	4.85556780941695e-05\\
135	4.85563911661148e-05\\
136	4.85571169628064e-05\\
137	4.85578557115676e-05\\
138	4.85586076437961e-05\\
139	4.85593729950541e-05\\
140	4.85601520051265e-05\\
141	4.85609449181111e-05\\
142	4.85617519824902e-05\\
143	4.85625734512119e-05\\
144	4.85634095817727e-05\\
145	4.85642606363002e-05\\
146	4.85651268816352e-05\\
147	4.85660085894219e-05\\
148	4.8566906036195e-05\\
149	4.85678195034628e-05\\
150	4.85687492778094e-05\\
151	4.85696956509753e-05\\
152	4.85706589199607e-05\\
153	4.85716393871228e-05\\
154	4.85726373602646e-05\\
155	4.85736531527433e-05\\
156	4.85746870835716e-05\\
157	4.85757394775177e-05\\
158	4.85768106652127e-05\\
159	4.85779009832611e-05\\
160	4.85790107743469e-05\\
161	4.8580140387351e-05\\
162	4.85812901774596e-05\\
163	4.85824605062826e-05\\
164	4.85836517419741e-05\\
165	4.85848642593533e-05\\
166	4.85860984400237e-05\\
167	4.85873546725047e-05\\
168	4.85886333523554e-05\\
169	4.8589934882306e-05\\
170	4.85912596723923e-05\\
171	4.85926081400904e-05\\
172	4.85939807104564e-05\\
173	4.85953778162622e-05\\
174	4.85967998981494e-05\\
175	4.85982474047637e-05\\
176	4.85997207929129e-05\\
177	4.86012205277192e-05\\
178	4.86027470827684e-05\\
179	4.86043009402722e-05\\
180	4.86058825912316e-05\\
181	4.86074925355984e-05\\
182	4.86091312824445e-05\\
183	4.86107993501316e-05\\
184	4.86124972664885e-05\\
185	4.86142255689852e-05\\
186	4.86159848049198e-05\\
187	4.86177755315993e-05\\
188	4.86195983165268e-05\\
189	4.86214537376046e-05\\
190	4.86233423833176e-05\\
191	4.86252648529382e-05\\
192	4.8627221756737e-05\\
193	4.8629213716181e-05\\
194	4.86312413641528e-05\\
195	4.8633305345169e-05\\
196	4.86354063155928e-05\\
197	4.86375449438675e-05\\
198	4.8639721910745e-05\\
199	4.86419379095203e-05\\
200	4.8644193646269e-05\\
201	4.86464898400934e-05\\
202	4.86488272233737e-05\\
203	4.86512065420214e-05\\
204	4.86536285557313e-05\\
205	4.86560940382629e-05\\
206	4.86586037776941e-05\\
207	4.86611585767051e-05\\
208	4.86637592528588e-05\\
209	4.86664066388893e-05\\
210	4.86691015829889e-05\\
211	4.86718449491142e-05\\
212	4.86746376172859e-05\\
213	4.8677480483906e-05\\
214	4.86803744620675e-05\\
215	4.86833204818831e-05\\
216	4.86863194908192e-05\\
217	4.86893724540281e-05\\
218	4.86924803546937e-05\\
219	4.86956441943839e-05\\
220	4.86988649934139e-05\\
221	4.87021437912051e-05\\
222	4.87054816466617e-05\\
223	4.87088796385576e-05\\
224	4.87123388659159e-05\\
225	4.87158604484162e-05\\
226	4.87194455267965e-05\\
227	4.87230952632638e-05\\
228	4.87268108419249e-05\\
229	4.87305934692125e-05\\
230	4.87344443743226e-05\\
231	4.87383648096746e-05\\
232	4.87423560513576e-05\\
233	4.87464193996125e-05\\
234	4.87505561792985e-05\\
235	4.87547677403838e-05\\
236	4.87590554584498e-05\\
237	4.87634207351953e-05\\
238	4.87678649989508e-05\\
239	4.87723897052186e-05\\
240	4.87769963372058e-05\\
241	4.87816864063762e-05\\
242	4.87864614530179e-05\\
243	4.87913230468166e-05\\
244	4.87962727874388e-05\\
245	4.88013123051327e-05\\
246	4.88064432613451e-05\\
247	4.88116673493306e-05\\
248	4.88169862948016e-05\\
249	4.88224018565728e-05\\
250	4.8827915827221e-05\\
251	4.88335300337648e-05\\
252	4.88392463383471e-05\\
253	4.8845066638951e-05\\
254	4.88509928701049e-05\\
255	4.88570270036215e-05\\
256	4.88631710493406e-05\\
257	4.88694270558925e-05\\
258	4.88757971114794e-05\\
259	4.88822833446537e-05\\
260	4.88888879251433e-05\\
261	4.88956130646577e-05\\
262	4.89024610177392e-05\\
263	4.89094340826172e-05\\
264	4.89165346020737e-05\\
265	4.89237649643376e-05\\
266	4.89311276039891e-05\\
267	4.89386250028848e-05\\
268	4.89462596910926e-05\\
269	4.89540342478559e-05\\
270	4.89619513025609e-05\\
271	4.89700135357422e-05\\
272	4.89782236800782e-05\\
273	4.89865845214382e-05\\
274	4.89950988999169e-05\\
275	4.90037697109065e-05\\
276	4.90125999061813e-05\\
277	4.90215924950037e-05\\
278	4.90307505452409e-05\\
279	4.90400771845188e-05\\
280	4.9049575601366e-05\\
281	4.90592490464081e-05\\
282	4.90691008335636e-05\\
283	4.9079134341253e-05\\
284	4.90893530136492e-05\\
285	4.90997603619277e-05\\
286	4.9110359965539e-05\\
287	4.91211554735117e-05\\
288	4.91321506057572e-05\\
289	4.91433491544102e-05\\
290	4.91547549851743e-05\\
291	4.91663720386933e-05\\
292	4.9178204331943e-05\\
293	4.91902559596341e-05\\
294	4.9202531095637e-05\\
295	4.92150339944272e-05\\
296	4.9227768992544e-05\\
297	4.92407405100684e-05\\
298	4.92539530521209e-05\\
299	4.92674112103702e-05\\
300	4.92811196645611e-05\\
301	4.92950831840634e-05\\
302	4.93093066294312e-05\\
303	4.93237949539777e-05\\
304	4.93385532053724e-05\\
305	4.93535865272458e-05\\
306	4.93689001608142e-05\\
307	4.93844994465169e-05\\
308	4.94003898256681e-05\\
309	4.941657684213e-05\\
310	4.94330661439883e-05\\
311	4.9449863485256e-05\\
312	4.94669747275817e-05\\
313	4.94844058419817e-05\\
314	4.95021629105794e-05\\
315	4.95202521283679e-05\\
316	4.9538679804987e-05\\
317	4.9557452366518e-05\\
318	4.95765763572965e-05\\
319	4.95960584417466e-05\\
320	4.96159054062375e-05\\
321	4.96361241609624e-05\\
322	4.9656721741848e-05\\
323	4.96777053124757e-05\\
324	4.96990821660607e-05\\
325	4.97208597274403e-05\\
326	4.97430455551105e-05\\
327	4.9765647343306e-05\\
328	4.97886729241193e-05\\
329	4.98121302696654e-05\\
330	4.98360274943216e-05\\
331	4.98603728569967e-05\\
332	4.98851747634897e-05\\
333	4.99104417689053e-05\\
334	4.99361825801509e-05\\
335	4.99624060585073e-05\\
336	4.9989121222297e-05\\
337	5.00163372496359e-05\\
338	5.00440634812877e-05\\
339	5.00723094236053e-05\\
340	5.01010847516042e-05\\
341	5.01303993121218e-05\\
342	5.01602631270978e-05\\
343	5.01906863969831e-05\\
344	5.02216795042634e-05\\
345	5.02532530171252e-05\\
346	5.02854176932574e-05\\
347	5.03181844838271e-05\\
348	5.03515645375995e-05\\
349	5.03855692052767e-05\\
350	5.04202100440215e-05\\
351	5.04554988222424e-05\\
352	5.04914475246349e-05\\
353	5.05280683575287e-05\\
354	5.05653737545819e-05\\
355	5.06033763828397e-05\\
356	5.06420891492121e-05\\
357	5.06815252073879e-05\\
358	5.07216979652491e-05\\
359	5.07626210928243e-05\\
360	5.08043085308115e-05\\
361	5.08467744997615e-05\\
362	5.08900335099431e-05\\
363	5.09341003719842e-05\\
364	5.09789902083251e-05\\
365	5.10247184655912e-05\\
366	5.10713009279394e-05\\
367	5.11187537314744e-05\\
368	5.11670933798314e-05\\
369	5.12163367610239e-05\\
370	5.12665011656658e-05\\
371	5.13176043066897e-05\\
372	5.13696643406997e-05\\
373	5.14226998910996e-05\\
374	5.14767300731561e-05\\
375	5.15317745211822e-05\\
376	5.15878534180362e-05\\
377	5.1644987527163e-05\\
378	5.17031982274314e-05\\
379	5.17625075510554e-05\\
380	5.18229382249418e-05\\
381	5.18845137158304e-05\\
382	5.19472582797052e-05\\
383	5.20111970159642e-05\\
384	5.20763559270271e-05\\
385	5.2142761984092e-05\\
386	5.22104431999679e-05\\
387	5.22794287101352e-05\\
388	5.23497488634706e-05\\
389	5.24214353245409e-05\\
390	5.24945211900758e-05\\
391	5.25690411231054e-05\\
392	5.26450315089373e-05\\
393	5.27225306351912e-05\\
394	5.28015788873556e-05\\
395	5.28822189220311e-05\\
396	5.29644957750493e-05\\
397	5.30484573727744e-05\\
398	5.31341547007723e-05\\
399	5.3221641628035e-05\\
400	5.33109751041869e-05\\
401	5.34022153600259e-05\\
402	5.34954261059868e-05\\
403	5.35906747198619e-05\\
404	5.36880324110592e-05\\
405	5.37875743466039e-05\\
406	5.38893797348472e-05\\
407	5.39935319156555e-05\\
408	5.41001186388268e-05\\
409	5.42092327617706e-05\\
410	5.4320971769912e-05\\
411	5.44354371495247e-05\\
412	5.45527380881619e-05\\
413	5.46729923971985e-05\\
414	5.47963275720986e-05\\
415	5.49228820134563e-05\\
416	5.50528064366033e-05\\
417	5.51862655029529e-05\\
418	5.53234397127437e-05\\
419	5.54645276068821e-05\\
420	5.56097483355481e-05\\
421	5.57593446642293e-05\\
422	5.5913586505215e-05\\
423	5.60727750903251e-05\\
424	5.6237247952476e-05\\
425	5.6407385006861e-05\\
426	5.65836163645633e-05\\
427	5.67664335585839e-05\\
428	5.69564092393166e-05\\
429	5.71542416273935e-05\\
430	5.73608781483743e-05\\
431	5.75779043230569e-05\\
432	5.78088449023424e-05\\
433	5.81528195365713e-05\\
434	5.944600411787e-05\\
435	6.07708016525669e-05\\
436	6.21282928284642e-05\\
437	6.35196126365089e-05\\
438	6.4945952967713e-05\\
439	6.64085650019074e-05\\
440	6.7908761246117e-05\\
441	6.94479170315175e-05\\
442	7.10274712130791e-05\\
443	7.26489257282991e-05\\
444	7.4313843546243e-05\\
445	7.60238443407499e-05\\
446	7.77805968532678e-05\\
447	7.95858060671009e-05\\
448	8.14411910082956e-05\\
449	8.33484419114667e-05\\
450	8.53091223231798e-05\\
451	8.7324403025868e-05\\
452	8.93942409883346e-05\\
453	9.15146481566733e-05\\
454	9.35035181938711e-05\\
455	9.55063189739746e-05\\
456	9.75619116404759e-05\\
457	9.96717437119328e-05\\
458	0.000101837158122837\\
459	0.000104059334890113\\
460	0.000106339220398147\\
461	0.000108677450218538\\
462	0.000111074278278899\\
463	0.000113529500974515\\
464	0.000116042112437598\\
465	0.000118607941916725\\
466	0.000121225878689891\\
467	0.000123897386685402\\
468	0.000126624013333797\\
469	0.000129407381553128\\
470	0.000132249194157849\\
471	0.000135151309566943\\
472	0.000138115684211088\\
473	0.000141144369207877\\
474	0.000144239505589599\\
475	0.000147403349520146\\
476	0.000150638493100372\\
477	0.000153947246360344\\
478	0.000157332063585502\\
479	0.000160795558711385\\
480	0.000164340522107563\\
481	0.000167969939549219\\
482	0.000171687013726675\\
483	0.00017549518871605\\
484	0.000179398177915557\\
485	0.000183399996029068\\
486	0.000187504995758643\\
487	0.000191717910133522\\
488	0.000196043902545426\\
489	0.00020048862747069\\
490	0.000205058300007985\\
491	0.000209759786026714\\
492	0.000214600706085218\\
493	0.000219589567095355\\
494	0.000224735913396954\\
495	0.000230050508790892\\
496	0.000235545583825447\\
497	0.000241235118228053\\
498	0.000247135186952318\\
499	0.00025326438473167\\
500	0.00025964434531677\\
501	0.000266300365657899\\
502	0.000273262109020108\\
503	0.000280564206036033\\
504	0.000288245964501537\\
505	0.000296341669277401\\
506	0.000308549128404662\\
507	0.000401301320030103\\
508	0.000496495979642583\\
509	0.00059424841886353\\
510	0.000694694016538126\\
511	0.000797978055842614\\
512	0.000904240215439014\\
513	0.00101363016259423\\
514	0.00112630667846547\\
515	0.00124243940537317\\
516	0.00136221071779773\\
517	0.00148581664666537\\
518	0.00161346774579206\\
519	0.0017453898123079\\
520	0.00188182432228271\\
521	0.00202302843785282\\
522	0.00216927474552162\\
523	0.00232085412302543\\
524	0.0024780776132213\\
525	0.00264125873975925\\
526	0.00281073686921825\\
527	0.00298714697226363\\
528	0.00317122809954563\\
529	0.00336384916330986\\
530	0.00356606087372385\\
531	0.00377911073125441\\
532	0.00387101444814595\\
533	0.00395496570181997\\
534	0.00404411861485679\\
535	0.0041393541562261\\
536	0.00424177554848268\\
537	0.0043527725456736\\
538	0.00447412533780575\\
539	0.00460796840269658\\
540	0.00474871272796527\\
541	0.00489195363004134\\
542	0.00503735684616973\\
543	0.00518445717205239\\
544	0.00533261860337901\\
545	0.00548098202337772\\
546	0.00562839635531924\\
547	0.00577331960216846\\
548	0.00591379893350599\\
549	0.00604742138754942\\
550	0.00617093021530702\\
551	0.00628179603343684\\
552	0.00639010919318062\\
553	0.00650075390556414\\
554	0.00661355826910363\\
555	0.00672831871640315\\
556	0.00684480418700696\\
557	0.00696276543345935\\
558	0.00708195602380279\\
559	0.00720220504478184\\
560	0.007323419518955\\
561	0.00744556801198611\\
562	0.00756874620780185\\
563	0.00769285619681815\\
564	0.00781677314052897\\
565	0.00793628932779018\\
566	0.00805088736053623\\
567	0.00816001629554882\\
568	0.00826313047911087\\
569	0.00835993460475097\\
570	0.00845129433017152\\
571	0.00853958906049813\\
572	0.00862495123378327\\
573	0.00870767807453262\\
574	0.00878809121736665\\
575	0.00886661500378345\\
576	0.00894286737083422\\
577	0.00901626113058295\\
578	0.0090866482592975\\
579	0.00915386945533433\\
580	0.00921808471957653\\
581	0.00927992511337708\\
582	0.00933845256431125\\
583	0.00939359549483182\\
584	0.0094456284309534\\
585	0.00949555685866655\\
586	0.00954333369779573\\
587	0.00958916916241128\\
588	0.00963370188398624\\
589	0.00967711629413547\\
590	0.00971958295016231\\
591	0.00976110413435579\\
592	0.00980161112056955\\
593	0.00984099058333643\\
594	0.00987905095617487\\
595	0.00991543381058343\\
596	0.00994937493726701\\
597	0.00997906286423442\\
598	0.0099999191923403\\
599	0\\
600	0\\
};
\addplot [color=mycolor18,solid,forget plot]
  table[row sep=crcr]{%
1	0.000169140775378202\\
2	0.000169141360870221\\
3	0.000169141956833748\\
4	0.000169142563455799\\
5	0.000169143180926731\\
6	0.000169143809440294\\
7	0.000169144449193693\\
8	0.00016914510038765\\
9	0.000169145763226465\\
10	0.000169146437918081\\
11	0.000169147124674152\\
12	0.0001691478237101\\
13	0.000169148535245194\\
14	0.000169149259502606\\
15	0.000169149996709493\\
16	0.000169150747097061\\
17	0.00016915151090063\\
18	0.000169152288359729\\
19	0.000169153079718147\\
20	0.000169153885224021\\
21	0.000169154705129912\\
22	0.000169155539692885\\
23	0.000169156389174587\\
24	0.000169157253841328\\
25	0.000169158133964165\\
26	0.000169159029818986\\
27	0.000169159941686601\\
28	0.00016916086985282\\
29	0.00016916181460855\\
30	0.000169162776249881\\
31	0.000169163755078186\\
32	0.000169164751400206\\
33	0.000169165765528146\\
34	0.000169166797779781\\
35	0.000169167848478545\\
36	0.000169168917953642\\
37	0.000169170006540139\\
38	0.000169171114579074\\
39	0.000169172242417566\\
40	0.000169173390408922\\
41	0.000169174558912741\\
42	0.000169175748295031\\
43	0.000169176958928325\\
44	0.000169178191191797\\
45	0.000169179445471374\\
46	0.000169180722159865\\
47	0.000169182021657074\\
48	0.000169183344369936\\
49	0.000169184690712634\\
50	0.000169186061106733\\
51	0.000169187455981313\\
52	0.000169188875773099\\
53	0.000169190320926595\\
54	0.000169191791894236\\
55	0.00016919328913651\\
56	0.00016919481312212\\
57	0.000169196364328115\\
58	0.00016919794324005\\
59	0.000169199550352132\\
60	0.000169201186167376\\
61	0.00016920285119776\\
62	0.00016920454596439\\
63	0.000169206270997654\\
64	0.000169208026837402\\
65	0.000169209814033093\\
66	0.000169211633143991\\
67	0.000169213484739319\\
68	0.000169215369398452\\
69	0.000169217287711087\\
70	0.000169219240277436\\
71	0.000169221227708407\\
72	0.000169223250625795\\
73	0.000169225309662487\\
74	0.000169227405462643\\
75	0.00016922953868191\\
76	0.000169231709987623\\
77	0.000169233920059015\\
78	0.000169236169587427\\
79	0.000169238459276524\\
80	0.00016924078984252\\
81	0.000169243162014397\\
82	0.000169245576534136\\
83	0.000169248034156946\\
84	0.000169250535651508\\
85	0.000169253081800205\\
86	0.000169255673399376\\
87	0.000169258311259563\\
88	0.000169260996205759\\
89	0.000169263729077678\\
90	0.000169266510730009\\
91	0.000169269342032687\\
92	0.000169272223871166\\
93	0.000169275157146695\\
94	0.000169278142776606\\
95	0.000169281181694597\\
96	0.000169284274851023\\
97	0.000169287423213203\\
98	0.000169290627765713\\
99	0.000169293889510708\\
100	0.000169297209468223\\
101	0.000169300588676507\\
102	0.000169304028192337\\
103	0.000169307529091361\\
104	0.000169311092468429\\
105	0.000169314719437943\\
106	0.000169318411134204\\
107	0.00016932216871177\\
108	0.000169325993345824\\
109	0.000169329886232535\\
110	0.000169333848589447\\
111	0.000169337881655853\\
112	0.000169341986693192\\
113	0.000169346164985444\\
114	0.000169350417839538\\
115	0.000169354746585762\\
116	0.000169359152578186\\
117	0.000169363637195088\\
118	0.000169368201839394\\
119	0.000169372847939113\\
120	0.000169377576947802\\
121	0.000169382390345014\\
122	0.000169387289636777\\
123	0.000169392276356063\\
124	0.000169397352063277\\
125	0.000169402518346758\\
126	0.000169407776823276\\
127	0.000169413129138547\\
128	0.00016941857696776\\
129	0.000169424122016103\\
130	0.000169429766019315\\
131	0.000169435510744231\\
132	0.000169441357989345\\
133	0.000169447309585389\\
134	0.000169453367395912\\
135	0.000169459533317877\\
136	0.000169465809282265\\
137	0.000169472197254695\\
138	0.000169478699236051\\
139	0.000169485317263119\\
140	0.000169492053409246\\
141	0.000169498909785002\\
142	0.000169505888538853\\
143	0.000169512991857853\\
144	0.000169520221968349\\
145	0.000169527581136695\\
146	0.000169535071669976\\
147	0.000169542695916759\\
148	0.000169550456267846\\
149	0.000169558355157043\\
150	0.000169566395061947\\
151	0.000169574578504747\\
152	0.000169582908053038\\
153	0.000169591386320653\\
154	0.000169600015968509\\
155	0.000169608799705471\\
156	0.000169617740289225\\
157	0.000169626840527181\\
158	0.00016963610327738\\
159	0.000169645531449434\\
160	0.000169655128005455\\
161	0.000169664895961041\\
162	0.000169674838386244\\
163	0.000169684958406587\\
164	0.000169695259204072\\
165	0.000169705744018236\\
166	0.000169716416147201\\
167	0.000169727278948765\\
168	0.000169738335841502\\
169	0.000169749590305881\\
170	0.000169761045885419\\
171	0.000169772706187853\\
172	0.000169784574886315\\
173	0.000169796655720564\\
174	0.000169808952498212\\
175	0.000169821469095988\\
176	0.00016983420946103\\
177	0.000169847177612184\\
178	0.000169860377641356\\
179	0.000169873813714861\\
180	0.000169887490074822\\
181	0.000169901411040579\\
182	0.000169915581010145\\
183	0.000169930004461665\\
184	0.000169944685954929\\
185	0.000169959630132898\\
186	0.00016997484172327\\
187	0.000169990325540069\\
188	0.000170006086485271\\
189	0.000170022129550458\\
190	0.000170038459818513\\
191	0.000170055082465341\\
192	0.000170072002761624\\
193	0.000170089226074614\\
194	0.000170106757869964\\
195	0.000170124603713592\\
196	0.000170142769273583\\
197	0.000170161260322129\\
198	0.00017018008273751\\
199	0.000170199242506112\\
200	0.000170218745724491\\
201	0.000170238598601468\\
202	0.000170258807460283\\
203	0.000170279378740776\\
204	0.000170300319001619\\
205	0.0001703216349226\\
206	0.00017034333330694\\
207	0.000170365421083671\\
208	0.00017038790531005\\
209	0.000170410793174038\\
210	0.000170434091996806\\
211	0.000170457809235325\\
212	0.000170481952484971\\
213	0.00017050652948222\\
214	0.000170531548107377\\
215	0.000170557016387361\\
216	0.000170582942498561\\
217	0.000170609334769738\\
218	0.000170636201684998\\
219	0.000170663551886816\\
220	0.000170691394179128\\
221	0.00017071973753049\\
222	0.000170748591077301\\
223	0.000170777964127089\\
224	0.000170807866161872\\
225	0.000170838306841585\\
226	0.00017086929600758\\
227	0.0001709008436862\\
228	0.000170932960092429\\
229	0.000170965655633617\\
230	0.000170998940913282\\
231	0.000171032826734991\\
232	0.000171067324106335\\
233	0.00017110244424297\\
234	0.000171138198572752\\
235	0.000171174598739968\\
236	0.000171211656609632\\
237	0.000171249384271903\\
238	0.000171287794046572\\
239	0.000171326898487654\\
240	0.000171366710388079\\
241	0.000171407242784477\\
242	0.00017144850896207\\
243	0.000171490522459664\\
244	0.000171533297074748\\
245	0.000171576846868694\\
246	0.000171621186172091\\
247	0.000171666329590157\\
248	0.000171712292008301\\
249	0.00017175908859777\\
250	0.000171806734821448\\
251	0.00017185524643975\\
252	0.000171904639516657\\
253	0.000171954930425878\\
254	0.000172006135857134\\
255	0.000172058272822577\\
256	0.000172111358663353\\
257	0.000172165411056294\\
258	0.000172220448020755\\
259	0.000172276487925589\\
260	0.000172333549496276\\
261	0.000172391651822199\\
262	0.000172450814364063\\
263	0.00017251105696149\\
264	0.000172572399840738\\
265	0.000172634863622617\\
266	0.000172698469330543\\
267	0.000172763238398767\\
268	0.000172829192680768\\
269	0.000172896354457827\\
270	0.000172964746447769\\
271	0.000173034391813874\\
272	0.00017310531417399\\
273	0.000173177537609807\\
274	0.000173251086676319\\
275	0.000173325986411495\\
276	0.000173402262346116\\
277	0.00017347994051381\\
278	0.000173559047461299\\
279	0.000173639610258825\\
280	0.000173721656510783\\
281	0.000173805214366563\\
282	0.000173890312531585\\
283	0.000173976980278556\\
284	0.000174065247458914\\
285	0.000174155144514509\\
286	0.000174246702489475\\
287	0.000174339953042325\\
288	0.000174434928458261\\
289	0.000174531661661694\\
290	0.000174630186228997\\
291	0.000174730536401446\\
292	0.000174832747098415\\
293	0.000174936853930769\\
294	0.000175042893214468\\
295	0.000175150901984418\\
296	0.000175260918008512\\
297	0.000175372979801904\\
298	0.0001754871266415\\
299	0.000175603398580655\\
300	0.000175721836464091\\
301	0.000175842481943035\\
302	0.000175965377490548\\
303	0.00017609056641709\\
304	0.000176218092886261\\
305	0.000176348001930775\\
306	0.000176480339468613\\
307	0.000176615152319388\\
308	0.000176752488220905\\
309	0.000176892395845903\\
310	0.000177034924819003\\
311	0.000177180125733827\\
312	0.000177328050170325\\
313	0.000177478750712254\\
314	0.000177632280964872\\
315	0.00017778869557278\\
316	0.000177948050237974\\
317	0.000178110401738062\\
318	0.000178275807944658\\
319	0.00017844432784198\\
320	0.000178616021545623\\
321	0.000178790950321544\\
322	0.000178969176605242\\
323	0.000179150764021163\\
324	0.000179335777402335\\
325	0.000179524282810263\\
326	0.000179716347555077\\
327	0.000179912040215995\\
328	0.00018011143066211\\
329	0.000180314590073516\\
330	0.000180521590962852\\
331	0.000180732507197275\\
332	0.000180947414020906\\
333	0.000181166388077831\\
334	0.000181389507435652\\
335	0.000181616851609706\\
336	0.000181848501587966\\
337	0.000182084539856667\\
338	0.000182325050426761\\
339	0.000182570118861175\\
340	0.00018281983230296\\
341	0.000183074279504315\\
342	0.000183333550856513\\
343	0.000183597738420683\\
344	0.000183866935959448\\
345	0.000184141238969377\\
346	0.000184420744714248\\
347	0.000184705552259115\\
348	0.000184995762505134\\
349	0.000185291478225168\\
350	0.000185592804100197\\
351	0.000185899846756542\\
352	0.000186212714804042\\
353	0.000186531518875363\\
354	0.000186856371666636\\
355	0.000187187387979784\\
356	0.000187524684766725\\
357	0.000187868381175616\\
358	0.000188218598599337\\
359	0.000188575460726427\\
360	0.000188939093594694\\
361	0.000189309625647759\\
362	0.000189687187794814\\
363	0.00019007191347385\\
364	0.000190463938718729\\
365	0.000190863402230402\\
366	0.000191270445452653\\
367	0.000191685212652803\\
368	0.000192107851007744\\
369	0.000192538510695839\\
370	0.000192977344995141\\
371	0.000193424510388485\\
372	0.000193880166676053\\
373	0.000194344477096035\\
374	0.000194817608454052\\
375	0.00019529973126214\\
376	0.000195791019888057\\
377	0.000196291652715871\\
378	0.0001968018123188\\
379	0.000197321685645456\\
380	0.000197851464220781\\
381	0.000198391344363156\\
382	0.000198941527419401\\
383	0.000199502220019748\\
384	0.000200073634355187\\
385	0.000200655988480237\\
386	0.000201249506644805\\
387	0.000201854419659851\\
388	0.000202470965303061\\
389	0.000203099388773038\\
390	0.000203739943204837\\
391	0.000204392890268078\\
392	0.000205058500885782\\
393	0.000205737056143686\\
394	0.000206428848497399\\
395	0.000207134183307596\\
396	0.000207853379823844\\
397	0.000208586765521004\\
398	0.00020933468860833\\
399	0.000210097524724835\\
400	0.000210875667369174\\
401	0.000211669529028189\\
402	0.000212479542287857\\
403	0.000213306160867271\\
404	0.000214149860468807\\
405	0.000215011139248253\\
406	0.000215890517559442\\
407	0.000216788536482469\\
408	0.000217705755060487\\
409	0.000218642749987742\\
410	0.000219600141048629\\
411	0.000220578570894087\\
412	0.000221578619178366\\
413	0.000222600889698149\\
414	0.00022364601233429\\
415	0.00022471464532483\\
416	0.000225807477920701\\
417	0.000226925233513964\\
418	0.000228068673351751\\
419	0.000229238600978757\\
420	0.000230435867590063\\
421	0.000231661378522649\\
422	0.000232916101182086\\
423	0.000234201074767163\\
424	0.000235517422258308\\
425	0.000236866365179842\\
426	0.000238249241555478\\
427	0.000239667526799369\\
428	0.000241122854473137\\
429	0.000242617023424316\\
430	0.000244151940366315\\
431	0.000245729312811729\\
432	0.000247349424765246\\
433	0.000248917206812989\\
434	0.000249596544315368\\
435	0.000250289392039495\\
436	0.000250995965316014\\
437	0.000251716480921113\\
438	0.00025245115865765\\
439	0.000253200223695467\\
440	0.00025396390992625\\
441	0.000254742464665859\\
442	0.000255536155136975\\
443	0.00025634527729162\\
444	0.000257170167690267\\
445	0.000258011219336887\\
446	0.000258868902549362\\
447	0.000259743792028783\\
448	0.000260636601032502\\
449	0.000261548222420013\\
450	0.000262479773609716\\
451	0.00026343263996428\\
452	0.000264408529489691\\
453	0.000265409712920401\\
454	0.000266440324185177\\
455	0.000267507390817175\\
456	0.000268615613600356\\
457	0.000269769795927175\\
458	0.00027097573957645\\
459	0.000272240500972844\\
460	0.00027357274626769\\
461	0.000274983305147449\\
462	0.000276486214505931\\
463	0.000278101230395547\\
464	0.000279861166060325\\
465	0.000307524637827358\\
466	0.000349969645631211\\
467	0.000393328689223382\\
468	0.000437627384730763\\
469	0.000482892213005026\\
470	0.000529150187553064\\
471	0.000576429124412446\\
472	0.0006247596323477\\
473	0.000674173667795424\\
474	0.000724704758816065\\
475	0.00077638852461577\\
476	0.000829263833541199\\
477	0.000883376879913921\\
478	0.000938764033621671\\
479	0.000995463115227499\\
480	0.00105351352101643\\
481	0.00111295636342444\\
482	0.00117383465337102\\
483	0.0012361935426771\\
484	0.00130008065241501\\
485	0.00136554652389328\\
486	0.00143264524472401\\
487	0.00150143532629143\\
488	0.00157198094668955\\
489	0.00164435372963181\\
490	0.00171863436274021\\
491	0.00179490999916231\\
492	0.00187327516151905\\
493	0.00195383234317257\\
494	0.00203669287158261\\
495	0.00212197728380178\\
496	0.00220981618775743\\
497	0.0023003531839373\\
498	0.00239374683710717\\
499	0.00249017303845386\\
500	0.00258982783074779\\
501	0.00269293074595561\\
502	0.00279972860438639\\
503	0.00291049941708504\\
504	0.00302555520514576\\
505	0.00314524552958512\\
506	0.00326263227805496\\
507	0.00330343466190086\\
508	0.00334563719849741\\
509	0.00338919372248778\\
510	0.00343401933015739\\
511	0.00348021493093214\\
512	0.00352790103705132\\
513	0.00357722775633646\\
514	0.0036283968779377\\
515	0.00368161901404434\\
516	0.00373712168333583\\
517	0.00379517532398462\\
518	0.00385610310005531\\
519	0.00392029325315597\\
520	0.0039882148043472\\
521	0.00406043735756756\\
522	0.00413765605181866\\
523	0.00422072260381219\\
524	0.00431068243548318\\
525	0.00440881296717743\\
526	0.00451553428790778\\
527	0.00462318184507296\\
528	0.00473134252956375\\
529	0.00483946652908471\\
530	0.00494682918447378\\
531	0.00505248079026667\\
532	0.00515562250187475\\
533	0.00525876642613349\\
534	0.00536161268074051\\
535	0.00546336034902462\\
536	0.00556294739333485\\
537	0.00565897620022263\\
538	0.00574971498516013\\
539	0.00583295559956373\\
540	0.00591379869531271\\
541	0.00599621185148525\\
542	0.00608011835818136\\
543	0.00616543130002583\\
544	0.00625205770730007\\
545	0.00633990646259004\\
546	0.00642888543393893\\
547	0.00651894091697697\\
548	0.00661009698930814\\
549	0.00670250472605515\\
550	0.0067965135636925\\
551	0.00689273545573893\\
552	0.00699161192885386\\
553	0.00709320002231934\\
554	0.00719740780199425\\
555	0.00730401456018862\\
556	0.0074126633149158\\
557	0.00752281634470193\\
558	0.00763362999187877\\
559	0.00774222432184838\\
560	0.00784624175801257\\
561	0.0079451173684526\\
562	0.00803836138267193\\
563	0.00812565253839744\\
564	0.00820701294573663\\
565	0.00828582198432094\\
566	0.00836215658040431\\
567	0.00843620586934715\\
568	0.00850832721502562\\
569	0.00857901013151536\\
570	0.00864906440997031\\
571	0.00871885032529967\\
572	0.00878725608766623\\
573	0.00885362539487147\\
574	0.00891781359239266\\
575	0.00897959328019709\\
576	0.00903975342437911\\
577	0.00909898426363027\\
578	0.0091567304087957\\
579	0.00921190939223783\\
580	0.00926446940119685\\
581	0.00931498012486593\\
582	0.00936431107260272\\
583	0.00941219245211477\\
584	0.00945866243584944\\
585	0.00950405575984369\\
586	0.00954865718510014\\
587	0.0095925304780904\\
588	0.00963572111751646\\
589	0.00967825788510043\\
590	0.00972017907935068\\
591	0.00976137872239597\\
592	0.00980171668827597\\
593	0.00984102099863122\\
594	0.00987905595166784\\
595	0.00991543381058343\\
596	0.00994937493726701\\
597	0.00997906286423442\\
598	0.0099999191923403\\
599	0\\
600	0\\
};
\addplot [color=red!25!mycolor17,solid,forget plot]
  table[row sep=crcr]{%
1	0.000375855361567015\\
2	0.000375864735362743\\
3	0.000375874276850767\\
4	0.000375883989026931\\
5	0.000375893874940512\\
6	0.000375903937695174\\
7	0.000375914180449942\\
8	0.000375924606420201\\
9	0.000375935218878671\\
10	0.000375946021156463\\
11	0.0003759570166441\\
12	0.000375968208792553\\
13	0.000375979601114359\\
14	0.000375991197184693\\
15	0.000376003000642513\\
16	0.000376015015191641\\
17	0.000376027244601968\\
18	0.000376039692710604\\
19	0.00037605236342309\\
20	0.000376065260714613\\
21	0.000376078388631249\\
22	0.000376091751291189\\
23	0.000376105352886097\\
24	0.000376119197682364\\
25	0.000376133290022439\\
26	0.000376147634326219\\
27	0.000376162235092381\\
28	0.000376177096899843\\
29	0.000376192224409123\\
30	0.000376207622363863\\
31	0.000376223295592248\\
32	0.000376239249008547\\
33	0.000376255487614668\\
34	0.000376272016501651\\
35	0.000376288840851316\\
36	0.000376305965937858\\
37	0.000376323397129494\\
38	0.000376341139890124\\
39	0.000376359199781053\\
40	0.000376377582462738\\
41	0.000376396293696532\\
42	0.000376415339346471\\
43	0.000376434725381129\\
44	0.000376454457875476\\
45	0.000376474543012747\\
46	0.000376494987086415\\
47	0.000376515796502134\\
48	0.000376536977779709\\
49	0.000376558537555178\\
50	0.00037658048258284\\
51	0.000376602819737422\\
52	0.000376625556016146\\
53	0.000376648698540977\\
54	0.000376672254560806\\
55	0.000376696231453725\\
56	0.000376720636729335\\
57	0.000376745478031095\\
58	0.000376770763138686\\
59	0.000376796499970441\\
60	0.000376822696585871\\
61	0.000376849361188093\\
62	0.000376876502126454\\
63	0.000376904127899108\\
64	0.000376932247155693\\
65	0.000376960868699994\\
66	0.000376990001492742\\
67	0.000377019654654379\\
68	0.000377049837467919\\
69	0.00037708055938181\\
70	0.000377111830012952\\
71	0.000377143659149655\\
72	0.000377176056754711\\
73	0.0003772090329685\\
74	0.000377242598112152\\
75	0.000377276762690801\\
76	0.000377311537396846\\
77	0.000377346933113272\\
78	0.000377382960917113\\
79	0.000377419632082857\\
80	0.000377456958085995\\
81	0.000377494950606607\\
82	0.000377533621533028\\
83	0.000377572982965524\\
84	0.000377613047220143\\
85	0.000377653826832503\\
86	0.000377695334561765\\
87	0.0003777375833946\\
88	0.000377780586549253\\
89	0.000377824357479701\\
90	0.00037786890987985\\
91	0.0003779142576878\\
92	0.000377960415090283\\
93	0.000378007396527034\\
94	0.000378055216695353\\
95	0.000378103890554693\\
96	0.000378153433331373\\
97	0.00037820386052334\\
98	0.000378255187905027\\
99	0.000378307431532289\\
100	0.00037836060774748\\
101	0.000378414733184553\\
102	0.00037846982477425\\
103	0.000378525899749498\\
104	0.000378582975650768\\
105	0.000378641070331584\\
106	0.000378700201964145\\
107	0.000378760389045049\\
108	0.000378821650401074\\
109	0.000378884005195147\\
110	0.000378947472932331\\
111	0.000379012073465995\\
112	0.000379077827004037\\
113	0.000379144754115263\\
114	0.000379212875735858\\
115	0.000379282213175986\\
116	0.000379352788126547\\
117	0.000379424622665918\\
118	0.000379497739267039\\
119	0.000379572160804407\\
120	0.000379647910561355\\
121	0.000379725012237393\\
122	0.000379803489955662\\
123	0.00037988336827062\\
124	0.000379964672175753\\
125	0.000380047427111495\\
126	0.000380131658973323\\
127	0.000380217394119883\\
128	0.000380304659381396\\
129	0.000380393482068136\\
130	0.000380483889979131\\
131	0.000380575911410879\\
132	0.000380669575166452\\
133	0.000380764910564579\\
134	0.000380861947448975\\
135	0.000380960716197793\\
136	0.000381061247733372\\
137	0.000381163573531995\\
138	0.00038126772563395\\
139	0.000381373736653725\\
140	0.000381481639790436\\
141	0.000381591468838373\\
142	0.000381703258197825\\
143	0.000381817042886077\\
144	0.00038193285854853\\
145	0.000382050741470206\\
146	0.000382170728587262\\
147	0.000382292857498925\\
148	0.000382417166479432\\
149	0.000382543694490394\\
150	0.000382672481193246\\
151	0.00038280356696203\\
152	0.000382936992896346\\
153	0.000383072800834601\\
154	0.000383211033367421\\
155	0.000383351733851479\\
156	0.00038349494642337\\
157	0.000383640716013898\\
158	0.000383789088362624\\
159	0.000383940110032628\\
160	0.000384093828425572\\
161	0.000384250291797077\\
162	0.00038440954927231\\
163	0.000384571650862024\\
164	0.000384736647478714\\
165	0.000384904590953235\\
166	0.000385075534051627\\
167	0.000385249530492313\\
168	0.000385426634963654\\
169	0.000385606903141773\\
170	0.000385790391708767\\
171	0.00038597715837124\\
172	0.000386167261879223\\
173	0.000386360762045411\\
174	0.000386557719764833\\
175	0.000386758197034831\\
176	0.00038696225697549\\
177	0.000387169963850431\\
178	0.000387381383087991\\
179	0.000387596581302871\\
180	0.000387815626318109\\
181	0.000388038587187575\\
182	0.000388265534218859\\
183	0.000388496538996624\\
184	0.000388731674406369\\
185	0.000388971014658706\\
186	0.000389214635314107\\
187	0.000389462613308146\\
188	0.000389715026977176\\
189	0.000389971956084584\\
190	0.000390233481847511\\
191	0.000390499686964138\\
192	0.00039077065564149\\
193	0.0003910464736238\\
194	0.000391327228221434\\
195	0.000391613008340391\\
196	0.000391903904512368\\
197	0.000392200008925458\\
198	0.000392501415455497\\
199	0.000392808219697883\\
200	0.000393120519000252\\
201	0.000393438412495624\\
202	0.000393762001136319\\
203	0.000394091387728522\\
204	0.000394426676967558\\
205	0.000394767975473868\\
206	0.000395115391829715\\
207	0.000395469036616637\\
208	0.000395829022453655\\
209	0.000396195464036341\\
210	0.000396568478176487\\
211	0.000396948183842786\\
212	0.000397334702202259\\
213	0.000397728156662532\\
214	0.000398128672915005\\
215	0.000398536378978847\\
216	0.000398951405246032\\
217	0.00039937388452712\\
218	0.000399803952098157\\
219	0.000400241745748425\\
220	0.000400687405829319\\
221	0.000401141075304038\\
222	0.000401602899798527\\
223	0.000402073027653355\\
224	0.000402551609976692\\
225	0.000403038800698443\\
226	0.000403534756625491\\
227	0.0004040396374981\\
228	0.000404553606047494\\
229	0.000405076828054671\\
230	0.00040560947241045\\
231	0.000406151711176854\\
232	0.000406703719649673\\
233	0.000407265676422422\\
234	0.000407837763451732\\
235	0.000408420166124007\\
236	0.000409013073323558\\
237	0.000409616677502251\\
238	0.000410231174750549\\
239	0.000410856764870171\\
240	0.000411493651448245\\
241	0.000412142041933094\\
242	0.000412802147711648\\
243	0.000413474184188537\\
244	0.000414158370866868\\
245	0.000414854931430798\\
246	0.000415564093829837\\
247	0.000416286090365026\\
248	0.000417021157777013\\
249	0.000417769537335968\\
250	0.000418531474933557\\
251	0.000419307221176814\\
252	0.000420097031484151\\
253	0.000420901166183438\\
254	0.000421719890612251\\
255	0.000422553475220229\\
256	0.000423402195673846\\
257	0.000424266332963319\\
258	0.000425146173511911\\
259	0.000426042009287706\\
260	0.000426954137917746\\
261	0.000427882862804754\\
262	0.000428828493246336\\
263	0.000429791344556945\\
264	0.000430771738192403\\
265	0.000431770001877172\\
266	0.000432786469734556\\
267	0.000433821482419583\\
268	0.000434875387254931\\
269	0.000435948538369868\\
270	0.000437041296842104\\
271	0.000438154030842877\\
272	0.000439287115785182\\
273	0.000440440934475291\\
274	0.000441615877267434\\
275	0.000442812342222102\\
276	0.00044403073526753\\
277	0.000445271470364928\\
278	0.000446534969677154\\
279	0.000447821663741041\\
280	0.000449131991643555\\
281	0.000450466401201604\\
282	0.000451825349145801\\
283	0.000453209301308104\\
284	0.000454618732813421\\
285	0.000456054128275351\\
286	0.00045751598199604\\
287	0.000459004798170102\\
288	0.000460521091092953\\
289	0.000462065385373391\\
290	0.000463638216150475\\
291	0.00046524012931502\\
292	0.000466871681735374\\
293	0.000468533441487851\\
294	0.000470225988091772\\
295	0.000471949912749025\\
296	0.000473705818588461\\
297	0.000475494320914845\\
298	0.000477316047462662\\
299	0.000479171638654685\\
300	0.000481061747865299\\
301	0.000482987041688683\\
302	0.000484948200211826\\
303	0.000486945917292301\\
304	0.000488980900840993\\
305	0.000491053873109538\\
306	0.000493165570982603\\
307	0.000495316746274937\\
308	0.000497508166033209\\
309	0.000499740612842413\\
310	0.000502014885137115\\
311	0.000504331797517111\\
312	0.000506692181067717\\
313	0.000509096883684502\\
314	0.000511546770402322\\
315	0.000514042723728773\\
316	0.000516585643981767\\
317	0.000519176449631222\\
318	0.00052181607764491\\
319	0.000524505483838112\\
320	0.00052724564322733\\
321	0.000530037550387755\\
322	0.00053288221981458\\
323	0.000535780686288141\\
324	0.000538734005242898\\
325	0.000541743253140319\\
326	0.00054480952784587\\
327	0.000547933949010139\\
328	0.000551117658454588\\
329	0.000554361820562086\\
330	0.000557667622672797\\
331	0.000561036275485951\\
332	0.000564469013468215\\
333	0.000567967095269487\\
334	0.000571531804147124\\
335	0.000575164448399749\\
336	0.000578866361811827\\
337	0.000582638904110663\\
338	0.000586483461437091\\
339	0.000590401446831726\\
340	0.000594394300738327\\
341	0.000598463491525742\\
342	0.000602610516029878\\
343	0.000606836900116855\\
344	0.000611144199267906\\
345	0.000615533999183235\\
346	0.000620007916404401\\
347	0.00062456759895429\\
348	0.0006292147269931\\
349	0.000633951013488447\\
350	0.000638778204896862\\
351	0.00064369808185397\\
352	0.000648712459870477\\
353	0.000653823190031695\\
354	0.00065903215969827\\
355	0.000664341293204633\\
356	0.000669752552572866\\
357	0.000675267938244582\\
358	0.000680889489831862\\
359	0.000686619286888755\\
360	0.000692459449704616\\
361	0.000698412140121118\\
362	0.000704479562374511\\
363	0.000710663963965251\\
364	0.000716967636556976\\
365	0.000723392916907245\\
366	0.000729942187832629\\
367	0.000736617879210691\\
368	0.000743422469022132\\
369	0.00075035848443618\\
370	0.000757428502942744\\
371	0.000764635153535262\\
372	0.000771981117948212\\
373	0.000779469131953892\\
374	0.000787101986723156\\
375	0.000794882530255428\\
376	0.000802813668883572\\
377	0.000810898368859764\\
378	0.000819139658029086\\
379	0.000827540627598236\\
380	0.000836104434007321\\
381	0.000844834300913916\\
382	0.000853733521299567\\
383	0.000862805459710351\\
384	0.000872053554644853\\
385	0.000881481321105537\\
386	0.000891092353332046\\
387	0.000900890327738918\\
388	0.000910879006084138\\
389	0.000921062238898888\\
390	0.000931443969208682\\
391	0.000942028236562785\\
392	0.000952819181333663\\
393	0.000963821049074246\\
394	0.000975038194216569\\
395	0.000986475081000435\\
396	0.00099813627578383\\
397	0.00101002641521094\\
398	0.00102215005977609\\
399	0.00103451205535606\\
400	0.00104711769728795\\
401	0.00105997245280715\\
402	0.00107308197031745\\
403	0.00108645208931192\\
404	0.00110008885116143\\
405	0.00111399851140646\\
406	0.00112818755533907\\
407	0.00114266272165957\\
408	0.00115743104644096\\
409	0.00117249995723704\\
410	0.00118787748593639\\
411	0.00120357290884931\\
412	0.00121959580838612\\
413	0.00123595364884107\\
414	0.00125265406994354\\
415	0.00126970487381734\\
416	0.0012871140242872\\
417	0.00130488964536087\\
418	0.00132304001870451\\
419	0.00134157357987641\\
420	0.00136049891310906\\
421	0.00137982474455152\\
422	0.00139955993346553\\
423	0.00141971346091733\\
424	0.00144029441416247\\
425	0.00146131196750383\\
426	0.00148277535892334\\
427	0.00150469386122702\\
428	0.00152707674509668\\
429	0.00154993322871793\\
430	0.00157327240515157\\
431	0.00159710314396872\\
432	0.00162143403281746\\
433	0.00164627419956294\\
434	0.00167164021505231\\
435	0.00169756119205294\\
436	0.00172405154377306\\
437	0.00175112621257999\\
438	0.00177880070542449\\
439	0.00180709113343772\\
440	0.00183601425639072\\
441	0.00186558753285368\\
442	0.0018958291770815\\
443	0.00192675822389243\\
444	0.00195839460311546\\
445	0.00199075922557841\\
446	0.00202387408312123\\
447	0.0020577623657961\\
448	0.002092448600366\\
449	0.00212795881569221\\
450	0.00216432074310062\\
451	0.00220156406288743\\
452	0.00223972071917512\\
453	0.0022788253353444\\
454	0.0023189156084635\\
455	0.00236003293586921\\
456	0.00240222330492044\\
457	0.00244553836795654\\
458	0.00249003681886893\\
459	0.00253578618256766\\
460	0.00258286512841324\\
461	0.00263136639903981\\
462	0.00268140027958883\\
463	0.00273309391864708\\
464	0.00278658892957238\\
465	0.0028160858344251\\
466	0.00283229364123304\\
467	0.00284891162233504\\
468	0.00286595707549019\\
469	0.00288344845123734\\
470	0.00290140543312626\\
471	0.00291984904478575\\
472	0.00293880188699185\\
473	0.00295828798435947\\
474	0.00297833269726963\\
475	0.00299896215397996\\
476	0.00302020306229267\\
477	0.00304208316074912\\
478	0.00306463106400083\\
479	0.00308787617611319\\
480	0.0031118482766772\\
481	0.0031365768954682\\
482	0.00316209039491253\\
483	0.00318841464993482\\
484	0.00321557117399093\\
485	0.00324357448262726\\
486	0.00327242840310902\\
487	0.00330212091969892\\
488	0.00333261699691251\\
489	0.00336384880815839\\
490	0.00339572914874768\\
491	0.00342827959547576\\
492	0.00346152324687747\\
493	0.00349549303595669\\
494	0.00353023096274955\\
495	0.00356579818865329\\
496	0.00360226589595257\\
497	0.00363968661879014\\
498	0.00367811966898888\\
499	0.00371763218444223\\
500	0.00375830031168834\\
501	0.00380021052783213\\
502	0.00384346111023267\\
503	0.00388816378050593\\
504	0.00393444559791164\\
505	0.00398245123033693\\
506	0.00403235920962256\\
507	0.00408472396073087\\
508	0.00414228410426581\\
509	0.00420609627069447\\
510	0.00427692741442519\\
511	0.00434901585665425\\
512	0.00442233189612585\\
513	0.00449682806344451\\
514	0.0045724331827926\\
515	0.00464904554710942\\
516	0.00472652549801599\\
517	0.00480468550780751\\
518	0.0048832768550998\\
519	0.00496197116355689\\
520	0.00504033495355125\\
521	0.00511780127935372\\
522	0.00519363221594222\\
523	0.00526686946768327\\
524	0.00533627214472673\\
525	0.00540024436774731\\
526	0.00545788363371357\\
527	0.00551627335090793\\
528	0.00557539167135386\\
529	0.00563527859893137\\
530	0.00569594513862986\\
531	0.00575739841589084\\
532	0.00581967989167963\\
533	0.00588276119286136\\
534	0.00594662385371704\\
535	0.0060112831455332\\
536	0.00607681286106642\\
537	0.00614337892796407\\
538	0.00621128615237045\\
539	0.00628104207503862\\
540	0.00635312254556396\\
541	0.0064277223001837\\
542	0.0065048984881829\\
543	0.00658470064598824\\
544	0.00666716561147777\\
545	0.00675231227196042\\
546	0.00684007090564941\\
547	0.00693038605348422\\
548	0.00702318517624303\\
549	0.00711841038944991\\
550	0.00721590978540083\\
551	0.00731534307738251\\
552	0.00741610746377254\\
553	0.00751723386195233\\
554	0.007615025649278\\
555	0.00770805129765681\\
556	0.00779580222598891\\
557	0.00787788560954736\\
558	0.00795385979339356\\
559	0.00802573665129025\\
560	0.00809530374867697\\
561	0.00816276262131606\\
562	0.00822843087997214\\
563	0.00829273829880889\\
564	0.00835656690377135\\
565	0.0084203077058799\\
566	0.0084841452589513\\
567	0.00854828119591742\\
568	0.00861187435009632\\
569	0.00867400404508977\\
570	0.00873448446449719\\
571	0.00879303057564046\\
572	0.00885059500516906\\
573	0.00890770640378388\\
574	0.00896437197501948\\
575	0.00902054698487434\\
576	0.00907467179264195\\
577	0.00912662893845292\\
578	0.00917681195253149\\
579	0.00922620839379051\\
580	0.00927487032649829\\
581	0.00932247740245366\\
582	0.0093690234246846\\
583	0.00941489058069364\\
584	0.00946020548959901\\
585	0.00950497683508797\\
586	0.009549192402668\\
587	0.00959283081692013\\
588	0.00963587774208467\\
589	0.0096783320190804\\
590	0.0097202097033969\\
591	0.00976138909231767\\
592	0.00980171925872612\\
593	0.00984102135019415\\
594	0.00987905595166784\\
595	0.00991543381058343\\
596	0.00994937493726701\\
597	0.00997906286423442\\
598	0.0099999191923403\\
599	0\\
600	0\\
};
\addplot [color=mycolor19,solid,forget plot]
  table[row sep=crcr]{%
1	0.00243322575185515\\
2	0.00243322940690112\\
3	0.002433233127433\\
4	0.00243323691462267\\
5	0.00243324076966301\\
6	0.00243324469376817\\
7	0.00243324868817403\\
8	0.00243325275413852\\
9	0.00243325689294207\\
10	0.00243326110588798\\
11	0.00243326539430283\\
12	0.00243326975953691\\
13	0.00243327420296463\\
14	0.00243327872598494\\
15	0.0024332833300218\\
16	0.0024332880165246\\
17	0.0024332927869686\\
18	0.00243329764285543\\
19	0.00243330258571356\\
20	0.00243330761709872\\
21	0.00243331273859443\\
22	0.00243331795181255\\
23	0.00243332325839364\\
24	0.00243332866000762\\
25	0.00243333415835421\\
26	0.00243333975516345\\
27	0.00243334545219633\\
28	0.00243335125124526\\
29	0.00243335715413463\\
30	0.00243336316272144\\
31	0.00243336927889581\\
32	0.00243337550458164\\
33	0.00243338184173715\\
34	0.00243338829235553\\
35	0.00243339485846553\\
36	0.00243340154213214\\
37	0.0024334083454572\\
38	0.00243341527058006\\
39	0.0024334223196783\\
40	0.00243342949496824\\
41	0.0024334367987059\\
42	0.00243344423318748\\
43	0.00243345180075018\\
44	0.0024334595037729\\
45	0.00243346734467702\\
46	0.00243347532592709\\
47	0.00243348345003162\\
48	0.00243349171954392\\
49	0.00243350013706283\\
50	0.00243350870523359\\
51	0.00243351742674857\\
52	0.00243352630434822\\
53	0.00243353534082182\\
54	0.00243354453900849\\
55	0.00243355390179791\\
56	0.00243356343213137\\
57	0.00243357313300256\\
58	0.00243358300745861\\
59	0.00243359305860097\\
60	0.00243360328958637\\
61	0.00243361370362788\\
62	0.00243362430399586\\
63	0.00243363509401895\\
64	0.00243364607708517\\
65	0.00243365725664295\\
66	0.0024336686362022\\
67	0.0024336802193354\\
68	0.00243369200967867\\
69	0.00243370401093303\\
70	0.00243371622686545\\
71	0.00243372866131001\\
72	0.00243374131816915\\
73	0.00243375420141487\\
74	0.00243376731508999\\
75	0.00243378066330931\\
76	0.00243379425026102\\
77	0.00243380808020794\\
78	0.0024338221574888\\
79	0.00243383648651969\\
80	0.00243385107179539\\
81	0.00243386591789073\\
82	0.00243388102946204\\
83	0.00243389641124867\\
84	0.00243391206807432\\
85	0.00243392800484867\\
86	0.00243394422656882\\
87	0.00243396073832091\\
88	0.00243397754528165\\
89	0.00243399465271997\\
90	0.00243401206599862\\
91	0.00243402979057587\\
92	0.00243404783200714\\
93	0.00243406619594686\\
94	0.00243408488815007\\
95	0.00243410391447433\\
96	0.00243412328088146\\
97	0.00243414299343945\\
98	0.00243416305832428\\
99	0.00243418348182193\\
100	0.00243420427033023\\
101	0.00243422543036089\\
102	0.00243424696854157\\
103	0.00243426889161785\\
104	0.00243429120645541\\
105	0.00243431392004206\\
106	0.00243433703949003\\
107	0.00243436057203805\\
108	0.00243438452505372\\
109	0.00243440890603569\\
110	0.00243443372261605\\
111	0.00243445898256268\\
112	0.00243448469378165\\
113	0.00243451086431968\\
114	0.00243453750236664\\
115	0.00243456461625812\\
116	0.00243459221447791\\
117	0.00243462030566075\\
118	0.00243464889859492\\
119	0.00243467800222503\\
120	0.00243470762565476\\
121	0.00243473777814967\\
122	0.00243476846914011\\
123	0.00243479970822408\\
124	0.00243483150517026\\
125	0.00243486386992106\\
126	0.0024348968125956\\
127	0.00243493034349291\\
128	0.00243496447309514\\
129	0.00243499921207075\\
130	0.00243503457127785\\
131	0.00243507056176759\\
132	0.00243510719478751\\
133	0.00243514448178507\\
134	0.00243518243441118\\
135	0.00243522106452384\\
136	0.00243526038419174\\
137	0.00243530040569808\\
138	0.00243534114154432\\
139	0.00243538260445403\\
140	0.00243542480737691\\
141	0.00243546776349272\\
142	0.00243551148621539\\
143	0.00243555598919714\\
144	0.0024356012863328\\
145	0.00243564739176398\\
146	0.00243569431988355\\
147	0.00243574208534001\\
148	0.00243579070304211\\
149	0.00243584018816341\\
150	0.00243589055614697\\
151	0.00243594182271013\\
152	0.00243599400384942\\
153	0.00243604711584545\\
154	0.00243610117526801\\
155	0.00243615619898108\\
156	0.00243621220414821\\
157	0.00243626920823775\\
158	0.00243632722902821\\
159	0.00243638628461381\\
160	0.0024364463934101\\
161	0.00243650757415964\\
162	0.00243656984593782\\
163	0.00243663322815867\\
164	0.002436697740581\\
165	0.00243676340331443\\
166	0.00243683023682566\\
167	0.00243689826194482\\
168	0.00243696749987184\\
169	0.00243703797218308\\
170	0.00243710970083805\\
171	0.00243718270818609\\
172	0.00243725701697337\\
173	0.00243733265035\\
174	0.00243740963187707\\
175	0.00243748798553408\\
176	0.00243756773572627\\
177	0.00243764890729223\\
178	0.00243773152551165\\
179	0.00243781561611305\\
180	0.0024379012052819\\
181	0.00243798831966862\\
182	0.00243807698639691\\
183	0.00243816723307218\\
184	0.00243825908779009\\
185	0.00243835257914535\\
186	0.00243844773624056\\
187	0.00243854458869515\\
188	0.00243864316665478\\
189	0.00243874350080061\\
190	0.00243884562235886\\
191	0.00243894956311053\\
192	0.00243905535540133\\
193	0.00243916303215172\\
194	0.00243927262686716\\
195	0.0024393841736486\\
196	0.00243949770720314\\
197	0.00243961326285478\\
198	0.00243973087655549\\
199	0.0024398505848965\\
200	0.00243997242511965\\
201	0.00244009643512912\\
202	0.00244022265350323\\
203	0.0024403511195066\\
204	0.00244048187310239\\
205	0.00244061495496486\\
206	0.00244075040649215\\
207	0.0024408882698193\\
208	0.00244102858783148\\
209	0.00244117140417743\\
210	0.00244131676328333\\
211	0.00244146471036672\\
212	0.0024416152914508\\
213	0.00244176855337893\\
214	0.00244192454382946\\
215	0.00244208331133091\\
216	0.00244224490527716\\
217	0.0024424093759433\\
218	0.0024425767745015\\
219	0.00244274715303737\\
220	0.00244292056456639\\
221	0.002443097063051\\
222	0.00244327670341765\\
223	0.00244345954157455\\
224	0.00244364563442941\\
225	0.00244383503990779\\
226	0.00244402781697168\\
227	0.00244422402563843\\
228	0.00244442372700015\\
229	0.00244462698324345\\
230	0.00244483385766953\\
231	0.00244504441471467\\
232	0.00244525871997119\\
233	0.00244547684020884\\
234	0.00244569884339648\\
235	0.00244592479872436\\
236	0.00244615477662682\\
237	0.00244638884880532\\
238	0.0024466270882522\\
239	0.00244686956927463\\
240	0.00244711636751926\\
241	0.00244736755999737\\
242	0.00244762322511045\\
243	0.00244788344267635\\
244	0.00244814829395612\\
245	0.00244841786168116\\
246	0.00244869223008125\\
247	0.00244897148491293\\
248	0.00244925571348868\\
249	0.00244954500470662\\
250	0.00244983944908089\\
251	0.0024501391387728\\
252	0.00245044416762254\\
253	0.0024507546311817\\
254	0.00245107062674643\\
255	0.00245139225339151\\
256	0.00245171961200502\\
257	0.00245205280532398\\
258	0.00245239193797073\\
259	0.00245273711649021\\
260	0.00245308844938809\\
261	0.00245344604716983\\
262	0.0024538100223808\\
263	0.00245418048964712\\
264	0.00245455756571778\\
265	0.00245494136950771\\
266	0.00245533202214183\\
267	0.00245572964700042\\
268	0.00245613436976553\\
269	0.00245654631846854\\
270	0.00245696562353919\\
271	0.00245739241785566\\
272	0.00245782683679614\\
273	0.00245826901829169\\
274	0.00245871910288073\\
275	0.00245917723376485\\
276	0.00245964355686627\\
277	0.00246011822088686\\
278	0.00246060137736899\\
279	0.002461093180758\\
280	0.00246159378846649\\
281	0.0024621033609407\\
282	0.00246262206172861\\
283	0.00246315005755033\\
284	0.00246368751837049\\
285	0.00246423461747296\\
286	0.00246479153153784\\
287	0.00246535844072099\\
288	0.00246593552873593\\
289	0.00246652298293855\\
290	0.00246712099441454\\
291	0.00246772975806956\\
292	0.00246834947272263\\
293	0.00246898034120261\\
294	0.0024696225704477\\
295	0.00247027637160877\\
296	0.00247094196015583\\
297	0.00247161955598855\\
298	0.00247230938355034\\
299	0.00247301167194677\\
300	0.00247372665506791\\
301	0.00247445457171518\\
302	0.00247519566573273\\
303	0.0024759501861435\\
304	0.00247671838729021\\
305	0.00247750052898146\\
306	0.00247829687664303\\
307	0.00247910770147479\\
308	0.00247993328061304\\
309	0.00248077389729889\\
310	0.00248162984105248\\
311	0.00248250140785344\\
312	0.00248338890032766\\
313	0.00248429262794039\\
314	0.00248521290719616\\
315	0.00248615006184504\\
316	0.00248710442309585\\
317	0.00248807632983603\\
318	0.00248906612885809\\
319	0.00249007417509303\\
320	0.0024911008318499\\
321	0.00249214647106188\\
322	0.00249321147353832\\
323	0.00249429622922244\\
324	0.0024954011374541\\
325	0.00249652660723728\\
326	0.00249767305751127\\
327	0.00249884091742482\\
328	0.00250003062661243\\
329	0.00250124263547115\\
330	0.00250247740543719\\
331	0.00250373540926026\\
332	0.00250501713127448\\
333	0.00250632306766386\\
334	0.00250765372672073\\
335	0.00250900962909522\\
336	0.0025103913080341\\
337	0.00251179930960745\\
338	0.00251323419292223\\
339	0.00251469653032239\\
340	0.00251618690757661\\
341	0.00251770592405601\\
342	0.00251925419290494\\
343	0.00252083234120518\\
344	0.0025224410101529\\
345	0.00252408085532325\\
346	0.00252575254696737\\
347	0.00252745677035158\\
348	0.00252919422614939\\
349	0.00253096563089699\\
350	0.00253277171752206\\
351	0.00253461323595275\\
352	0.00253649095380898\\
353	0.00253840565717714\\
354	0.00254035815148238\\
355	0.00254234926247657\\
356	0.00254437983683534\\
357	0.00254645074273674\\
358	0.00254856287045189\\
359	0.00255071713294689\\
360	0.00255291446649582\\
361	0.00255515583130394\\
362	0.00255744221214044\\
363	0.0025597746189799\\
364	0.00256215408765162\\
365	0.00256458168049537\\
366	0.00256705848702252\\
367	0.0025695856245811\\
368	0.00257216423902266\\
369	0.00257479550536971\\
370	0.0025774806284808\\
371	0.00258022084371129\\
372	0.00258301741756675\\
373	0.00258587164834578\\
374	0.00258878486676855\\
375	0.00259175843658691\\
376	0.00259479375517109\\
377	0.00259789225406755\\
378	0.00260105539952129\\
379	0.00260428469295505\\
380	0.00260758167139656\\
381	0.00261094790784304\\
382	0.00261438501155048\\
383	0.00261789462823249\\
384	0.00262147844015029\\
385	0.0026251381660719\\
386	0.00262887556107286\\
387	0.00263269241614548\\
388	0.00263659055757467\\
389	0.00264057184602923\\
390	0.0026446381753039\\
391	0.0026487914706317\\
392	0.00265303368646489\\
393	0.00265736680359865\\
394	0.00266179282548564\\
395	0.00266631377357758\\
396	0.00267093168157545\\
397	0.00267564858866405\\
398	0.00268046653230534\\
399	0.00268538753867355\\
400	0.00269041360363406\\
401	0.00269554667067543\\
402	0.00270078861131059\\
403	0.00270614120321678\\
404	0.00271160610671546\\
405	0.00271718484080559\\
406	0.00272287876090144\\
407	0.00272868904181218\\
408	0.00273461667146946\\
409	0.00274066246357437\\
410	0.00274682710084091\\
411	0.00275311122544019\\
412	0.00275951561264378\\
413	0.00276604151076391\\
414	0.0027726912044118\\
415	0.00277946746198407\\
416	0.00278637314424618\\
417	0.00279341120986781\\
418	0.00280058472191354\\
419	0.00280789685554711\\
420	0.00281535090508996\\
421	0.00282295028744672\\
422	0.00283069855175062\\
423	0.00283859939608647\\
424	0.00284665671753488\\
425	0.0028548746330132\\
426	0.00286325750122126\\
427	0.00287180994486809\\
428	0.00288053687188287\\
429	0.00288944349364547\\
430	0.00289853533701717\\
431	0.00290781824351838\\
432	0.00291729833865878\\
433	0.00292698192475248\\
434	0.00293687522341037\\
435	0.00294698438892954\\
436	0.00295731584817982\\
437	0.00296787631184095\\
438	0.00297867278412394\\
439	0.00298971257008401\\
440	0.00300100327931023\\
441	0.00301255282434944\\
442	0.00302436941164805\\
443	0.00303646152202934\\
444	0.00304883787669455\\
445	0.00306150738335215\\
446	0.00307447905521328\\
447	0.00308776189309536\\
448	0.00310136471771375\\
449	0.00311529593617701\\
450	0.00312956322845109\\
451	0.00314417316807397\\
452	0.0031591306794131\\
453	0.00317443806932127\\
454	0.00319009365038383\\
455	0.00320608971468185\\
456	0.00322241117922144\\
457	0.00323903284133796\\
458	0.00325591568031399\\
459	0.00327300186627504\\
460	0.00329020802013293\\
461	0.00330741618752921\\
462	0.00332446235535646\\
463	0.0033412318864999\\
464	0.00335758742685534\\
465	0.00337338760836127\\
466	0.00338918042814833\\
467	0.00340530655874068\\
468	0.00342178366260316\\
469	0.00343863177671142\\
470	0.00345587363475759\\
471	0.00347353476768834\\
472	0.0034916426578092\\
473	0.00351023430286914\\
474	0.00352935638325661\\
475	0.00354907225021517\\
476	0.00356945357161289\\
477	0.00359057112268743\\
478	0.00361250894201974\\
479	0.00363536731737758\\
480	0.00365926652319814\\
481	0.00368435151566849\\
482	0.00371079785126699\\
483	0.00373881917193254\\
484	0.00376867669766305\\
485	0.00380069128279977\\
486	0.0038352586996068\\
487	0.00387286879664353\\
488	0.0039141285910975\\
489	0.00395978649863041\\
490	0.00400999994379843\\
491	0.00406104221269004\\
492	0.00411290889303081\\
493	0.00416559058337188\\
494	0.00421907124212074\\
495	0.00427332600109312\\
496	0.00432831847470865\\
497	0.00438399836381636\\
498	0.0044402986351139\\
499	0.00449713119941296\\
500	0.00455438145906978\\
501	0.00461190142112322\\
502	0.00466950098744828\\
503	0.00472693692844685\\
504	0.00478389889952752\\
505	0.00483999165498312\\
506	0.00489471185943743\\
507	0.00494740805108051\\
508	0.00499718385826722\\
509	0.00504289057825752\\
510	0.00508359944924854\\
511	0.00512486917326703\\
512	0.0051666669189476\\
513	0.00520895418874585\\
514	0.00525168675647541\\
515	0.00529481481413926\\
516	0.00533828304376811\\
517	0.00538202974207427\\
518	0.00542599384801781\\
519	0.00547014462970504\\
520	0.00551450866905051\\
521	0.00555909764201829\\
522	0.00560393937996247\\
523	0.00564911930132595\\
524	0.00569481048431436\\
525	0.00574131614048781\\
526	0.0057890876497553\\
527	0.00583843272153866\\
528	0.00588943896141111\\
529	0.00594220010200307\\
530	0.00599681393333867\\
531	0.00605338332978803\\
532	0.00611201645377706\\
533	0.00617282830798182\\
534	0.00623593928816131\\
535	0.00630147018111279\\
536	0.00636949234990505\\
537	0.00644006104353074\\
538	0.00651325385171948\\
539	0.0065891142941634\\
540	0.00666763728431354\\
541	0.00674877479497815\\
542	0.00683243735517887\\
543	0.00691848790243981\\
544	0.00700678142595555\\
545	0.00709710460458226\\
546	0.00718914909923042\\
547	0.00728233011489576\\
548	0.00737567760376649\\
549	0.00746556707184357\\
550	0.00755055199800488\\
551	0.00763014193807071\\
552	0.00770374944381809\\
553	0.00777122664892783\\
554	0.00783558854786247\\
555	0.00789783794406934\\
556	0.00795825232183846\\
557	0.0080172196760148\\
558	0.00807556562364243\\
559	0.00813380648727646\\
560	0.00819215789604154\\
561	0.00825081058869847\\
562	0.00830995726729622\\
563	0.00836978176436801\\
564	0.00843009923217756\\
565	0.00848930666147047\\
566	0.00854719485207219\\
567	0.00860348792501928\\
568	0.00865888706570304\\
569	0.0087141507280509\\
570	0.00876926249803578\\
571	0.00882426452232611\\
572	0.00887920168507198\\
573	0.00893313527028819\\
574	0.00898517449966207\\
575	0.00903522942231062\\
576	0.00908474529495557\\
577	0.00913376135612379\\
578	0.00918233514225318\\
579	0.00923013735242427\\
580	0.00927705205247275\\
581	0.00932349070899567\\
582	0.00936953410212466\\
583	0.00941516998195261\\
584	0.00946036084630153\\
585	0.00950506346562379\\
586	0.00954923819722844\\
587	0.0095928530922261\\
588	0.00963588745565751\\
589	0.00967833566889083\\
590	0.0097202108171807\\
591	0.00976138933709939\\
592	0.00980171928775089\\
593	0.00984102135019415\\
594	0.00987905595166784\\
595	0.00991543381058343\\
596	0.00994937493726701\\
597	0.00997906286423442\\
598	0.0099999191923403\\
599	0\\
600	0\\
};
\addplot [color=red!50!mycolor17,solid,forget plot]
  table[row sep=crcr]{%
1	0.00283870092861365\\
2	0.00283870433655056\\
3	0.00283870780564254\\
4	0.00283871133698597\\
5	0.00283871493169682\\
6	0.0028387185909111\\
7	0.00283872231578512\\
8	0.00283872610749591\\
9	0.00283872996724156\\
10	0.00283873389624159\\
11	0.00283873789573735\\
12	0.00283874196699244\\
13	0.00283874611129304\\
14	0.00283875032994839\\
15	0.00283875462429111\\
16	0.00283875899567769\\
17	0.00283876344548891\\
18	0.00283876797513025\\
19	0.00283877258603231\\
20	0.00283877727965132\\
21	0.00283878205746957\\
22	0.00283878692099581\\
23	0.00283879187176586\\
24	0.00283879691134296\\
25	0.00283880204131834\\
26	0.0028388072633117\\
27	0.00283881257897167\\
28	0.00283881798997637\\
29	0.00283882349803399\\
30	0.00283882910488322\\
31	0.00283883481229385\\
32	0.00283884062206733\\
33	0.00283884653603731\\
34	0.00283885255607024\\
35	0.0028388586840659\\
36	0.0028388649219581\\
37	0.00283887127171516\\
38	0.00283887773534063\\
39	0.00283888431487382\\
40	0.00283889101239058\\
41	0.00283889783000374\\
42	0.00283890476986399\\
43	0.00283891183416043\\
44	0.00283891902512122\\
45	0.00283892634501442\\
46	0.00283893379614855\\
47	0.00283894138087341\\
48	0.00283894910158078\\
49	0.00283895696070514\\
50	0.00283896496072446\\
51	0.00283897310416097\\
52	0.00283898139358199\\
53	0.00283898983160067\\
54	0.00283899842087676\\
55	0.00283900716411757\\
56	0.00283901606407876\\
57	0.00283902512356513\\
58	0.00283903434543156\\
59	0.00283904373258396\\
60	0.00283905328798007\\
61	0.00283906301463037\\
62	0.00283907291559913\\
63	0.00283908299400531\\
64	0.00283909325302345\\
65	0.00283910369588481\\
66	0.00283911432587824\\
67	0.0028391251463513\\
68	0.00283913616071125\\
69	0.00283914737242611\\
70	0.00283915878502576\\
71	0.00283917040210304\\
72	0.00283918222731482\\
73	0.0028391942643832\\
74	0.0028392065170966\\
75	0.00283921898931104\\
76	0.00283923168495119\\
77	0.00283924460801173\\
78	0.00283925776255847\\
79	0.00283927115272973\\
80	0.00283928478273752\\
81	0.00283929865686889\\
82	0.00283931277948727\\
83	0.00283932715503378\\
84	0.00283934178802866\\
85	0.00283935668307262\\
86	0.0028393718448483\\
87	0.0028393872781217\\
88	0.00283940298774367\\
89	0.00283941897865134\\
90	0.00283943525586976\\
91	0.0028394518245134\\
92	0.00283946868978772\\
93	0.00283948585699075\\
94	0.00283950333151479\\
95	0.00283952111884801\\
96	0.0028395392245762\\
97	0.00283955765438446\\
98	0.00283957641405893\\
99	0.00283959550948862\\
100	0.00283961494666721\\
101	0.0028396347316949\\
102	0.00283965487078025\\
103	0.00283967537024211\\
104	0.00283969623651155\\
105	0.00283971747613387\\
106	0.0028397390957706\\
107	0.00283976110220152\\
108	0.00283978350232674\\
109	0.00283980630316888\\
110	0.00283982951187513\\
111	0.00283985313571945\\
112	0.00283987718210492\\
113	0.00283990165856587\\
114	0.00283992657277022\\
115	0.00283995193252184\\
116	0.0028399777457629\\
117	0.00284000402057639\\
118	0.00284003076518851\\
119	0.00284005798797113\\
120	0.00284008569744448\\
121	0.00284011390227963\\
122	0.00284014261130115\\
123	0.00284017183348989\\
124	0.00284020157798559\\
125	0.00284023185408969\\
126	0.00284026267126823\\
127	0.00284029403915462\\
128	0.00284032596755265\\
129	0.0028403584664394\\
130	0.00284039154596828\\
131	0.00284042521647216\\
132	0.00284045948846641\\
133	0.00284049437265214\\
134	0.00284052987991943\\
135	0.00284056602135055\\
136	0.00284060280822343\\
137	0.00284064025201492\\
138	0.00284067836440439\\
139	0.00284071715727713\\
140	0.00284075664272799\\
141	0.002840796833065\\
142	0.00284083774081305\\
143	0.00284087937871771\\
144	0.00284092175974898\\
145	0.00284096489710522\\
146	0.00284100880421708\\
147	0.00284105349475156\\
148	0.00284109898261603\\
149	0.00284114528196243\\
150	0.0028411924071915\\
151	0.00284124037295709\\
152	0.00284128919417041\\
153	0.00284133888600459\\
154	0.00284138946389919\\
155	0.00284144094356474\\
156	0.00284149334098739\\
157	0.00284154667243366\\
158	0.00284160095445534\\
159	0.00284165620389425\\
160	0.00284171243788735\\
161	0.00284176967387167\\
162	0.00284182792958957\\
163	0.00284188722309395\\
164	0.0028419475727535\\
165	0.00284200899725816\\
166	0.00284207151562455\\
167	0.00284213514720164\\
168	0.00284219991167635\\
169	0.00284226582907936\\
170	0.00284233291979086\\
171	0.00284240120454669\\
172	0.00284247070444422\\
173	0.00284254144094859\\
174	0.00284261343589891\\
175	0.00284268671151462\\
176	0.00284276129040196\\
177	0.0028428371955605\\
178	0.00284291445038981\\
179	0.00284299307869625\\
180	0.00284307310469975\\
181	0.00284315455304094\\
182	0.00284323744878815\\
183	0.00284332181744459\\
184	0.00284340768495583\\
185	0.00284349507771704\\
186	0.00284358402258066\\
187	0.00284367454686415\\
188	0.00284376667835761\\
189	0.00284386044533187\\
190	0.00284395587654643\\
191	0.00284405300125773\\
192	0.00284415184922738\\
193	0.00284425245073059\\
194	0.00284435483656481\\
195	0.0028444590380584\\
196	0.00284456508707942\\
197	0.00284467301604467\\
198	0.00284478285792881\\
199	0.00284489464627353\\
200	0.00284500841519708\\
201	0.00284512419940369\\
202	0.00284524203419337\\
203	0.00284536195547169\\
204	0.00284548399975977\\
205	0.00284560820420451\\
206	0.00284573460658881\\
207	0.00284586324534205\\
208	0.00284599415955079\\
209	0.00284612738896945\\
210	0.00284626297403133\\
211	0.00284640095585972\\
212	0.00284654137627915\\
213	0.00284668427782694\\
214	0.00284682970376474\\
215	0.00284697769809036\\
216	0.00284712830554981\\
217	0.00284728157164946\\
218	0.0028474375426683\\
219	0.00284759626567062\\
220	0.00284775778851866\\
221	0.00284792215988554\\
222	0.00284808942926841\\
223	0.00284825964700172\\
224	0.00284843286427077\\
225	0.00284860913312545\\
226	0.00284878850649403\\
227	0.00284897103819749\\
228	0.00284915678296373\\
229	0.00284934579644215\\
230	0.00284953813521847\\
231	0.00284973385682968\\
232	0.00284993301977929\\
233	0.00285013568355279\\
234	0.00285034190863333\\
235	0.00285055175651761\\
236	0.00285076528973209\\
237	0.00285098257184941\\
238	0.0028512036675049\\
239	0.00285142864241372\\
240	0.00285165756338779\\
241	0.00285189049835336\\
242	0.00285212751636863\\
243	0.00285236868764177\\
244	0.00285261408354902\\
245	0.00285286377665339\\
246	0.00285311784072332\\
247	0.00285337635075185\\
248	0.00285363938297592\\
249	0.00285390701489617\\
250	0.0028541793252969\\
251	0.00285445639426639\\
252	0.00285473830321754\\
253	0.00285502513490887\\
254	0.00285531697346588\\
255	0.00285561390440267\\
256	0.00285591601464397\\
257	0.0028562233925476\\
258	0.00285653612792724\\
259	0.00285685431207555\\
260	0.00285717803778778\\
261	0.00285750739938577\\
262	0.00285784249274233\\
263	0.00285818341530614\\
264	0.00285853026612701\\
265	0.00285888314588166\\
266	0.00285924215690008\\
267	0.00285960740319219\\
268	0.00285997899047519\\
269	0.00286035702620138\\
270	0.00286074161958659\\
271	0.0028611328816392\\
272	0.00286153092518973\\
273	0.00286193586492109\\
274	0.00286234781739955\\
275	0.00286276690110634\\
276	0.00286319323647004\\
277	0.00286362694589979\\
278	0.00286406815381909\\
279	0.00286451698670082\\
280	0.0028649735731028\\
281	0.0028654380437045\\
282	0.00286591053134476\\
283	0.00286639117106043\\
284	0.00286688010012632\\
285	0.00286737745809612\\
286	0.00286788338684473\\
287	0.0028683980306119\\
288	0.00286892153604727\\
289	0.00286945405225691\\
290	0.00286999573085146\\
291	0.00287054672599602\\
292	0.00287110719446185\\
293	0.00287167729567994\\
294	0.0028722571917969\\
295	0.00287284704773271\\
296	0.00287344703124124\\
297	0.00287405731297303\\
298	0.00287467806654102\\
299	0.00287530946858908\\
300	0.00287595169886375\\
301	0.00287660494028946\\
302	0.0028772693790472\\
303	0.00287794520465731\\
304	0.00287863261006642\\
305	0.00287933179173891\\
306	0.00288004294975335\\
307	0.00288076628790415\\
308	0.00288150201380895\\
309	0.00288225033902219\\
310	0.0028830114791553\\
311	0.002883785654004\\
312	0.00288457308768344\\
313	0.00288537400877162\\
314	0.00288618865046179\\
315	0.00288701725072479\\
316	0.00288786005248165\\
317	0.00288871730378771\\
318	0.00288958925802896\\
319	0.00289047617413145\\
320	0.0028913783167851\\
321	0.00289229595668272\\
322	0.00289322937077541\\
323	0.0028941788425457\\
324	0.00289514466229953\\
325	0.00289612712747834\\
326	0.0028971265429927\\
327	0.00289814322157858\\
328	0.00289917748417769\\
329	0.00290022966034285\\
330	0.00290130008866944\\
331	0.00290238911725369\\
332	0.00290349710417811\\
333	0.00290462441802381\\
334	0.00290577143840861\\
335	0.00290693855654915\\
336	0.00290812617584297\\
337	0.00290933471246505\\
338	0.00291056459596982\\
339	0.00291181626988715\\
340	0.00291309019229849\\
341	0.00291438683638664\\
342	0.0029157066909734\\
343	0.00291705026105032\\
344	0.00291841806776943\\
345	0.00291981064685911\\
346	0.0029212285484882\\
347	0.00292267233700952\\
348	0.00292414259058094\\
349	0.00292563990067231\\
350	0.00292716487147907\\
351	0.002928718119271\\
352	0.00293030027168571\\
353	0.00293191196688974\\
354	0.00293355385245548\\
355	0.0029352265849003\\
356	0.00293693084065037\\
357	0.00293866731783082\\
358	0.00294043673738679\\
359	0.0029422398442847\\
360	0.00294407740880065\\
361	0.00294595022790378\\
362	0.00294785912674309\\
363	0.00294980496024692\\
364	0.00295178861484537\\
365	0.00295381101032717\\
366	0.00295587310184337\\
367	0.00295797588207172\\
368	0.00296012038355706\\
369	0.00296230768124428\\
370	0.00296453889522294\\
371	0.00296681519370352\\
372	0.00296913779624821\\
373	0.00297150797728118\\
374	0.00297392706990552\\
375	0.0029763964700572\\
376	0.00297891764102881\\
377	0.00298149211839925\\
378	0.00298412151540846\\
379	0.00298680752881971\\
380	0.00298955194531476\\
381	0.00299235664847091\\
382	0.0029952236263704\\
383	0.00299815497989572\\
384	0.0030011529317642\\
385	0.00300421983635425\\
386	0.00300735819037249\\
387	0.00301057064440294\\
388	0.003013860015367\\
389	0.00301722929990288\\
390	0.00302068168864234\\
391	0.00302422058131805\\
392	0.00302784960257062\\
393	0.00303157261823363\\
394	0.00303539375174873\\
395	0.00303931740018923\\
396	0.00304334824912741\\
397	0.00304749128523309\\
398	0.00305175180497379\\
399	0.00305613541720487\\
400	0.00306064803684511\\
401	0.00306529586542185\\
402	0.00307008535243514\\
403	0.00307502312943827\\
404	0.00308011590570261\\
405	0.00308537031020121\\
406	0.00309079265898533\\
407	0.00309638861924857\\
408	0.0031021627305885\\
409	0.00310811772883947\\
410	0.00311425359633487\\
411	0.00312056623196952\\
412	0.00312704559457391\\
413	0.00313367314836349\\
414	0.00314041859936968\\
415	0.00314727139920554\\
416	0.00315423369614006\\
417	0.00316130776674751\\
418	0.0031684960276916\\
419	0.0031758010557658\\
420	0.00318322565622549\\
421	0.00319077296565129\\
422	0.00319844631403471\\
423	0.0032062490418238\\
424	0.003214183896715\\
425	0.00322225377701107\\
426	0.00323046174710021\\
427	0.00323881105488027\\
428	0.00324730515158386\\
429	0.00325594771461001\\
430	0.00326474267408322\\
431	0.00327369424394737\\
432	0.00328280695904933\\
433	0.00329208572344072\\
434	0.00330153588294981\\
435	0.00331116331548502\\
436	0.00332097451830581\\
437	0.00333097671219466\\
438	0.00334117796620837\\
439	0.00335158734756848\\
440	0.00336221510235997\\
441	0.00337307287410827\\
442	0.00338417396907542\\
443	0.00339553367935669\\
444	0.00340716967768397\\
445	0.0034191025013505\\
446	0.00343135614685067\\
447	0.0034439588011638\\
448	0.0034569437378115\\
449	0.00347035039750513\\
450	0.00348422562530908\\
451	0.00349862484951891\\
452	0.00351361518620998\\
453	0.00352928220505041\\
454	0.00354573489592987\\
455	0.0035631115355953\\
456	0.0035815654237004\\
457	0.00360128540371144\\
458	0.00362250512451155\\
459	0.00364551475566122\\
460	0.00367067542333025\\
461	0.00369843548734482\\
462	0.00372934297113895\\
463	0.00376095621363585\\
464	0.00379309105106504\\
465	0.00382577263716753\\
466	0.00385901484731637\\
467	0.00389282359395008\\
468	0.00392720376105065\\
469	0.00396215890131907\\
470	0.00399769086849669\\
471	0.00403379936012597\\
472	0.00407048133580928\\
473	0.00410773020443479\\
474	0.00414553463487091\\
475	0.00418387698318528\\
476	0.00422273139401959\\
477	0.00426206186955287\\
478	0.00430182020803441\\
479	0.00434194273055568\\
480	0.0043823461022077\\
481	0.00442292198487946\\
482	0.00446353017886259\\
483	0.00450398980586727\\
484	0.00454406795040861\\
485	0.00458346500642036\\
486	0.00462179578739872\\
487	0.00465856535452245\\
488	0.00469313896240186\\
489	0.00472470844453819\\
490	0.0047530155955093\\
491	0.00478167072697865\\
492	0.00481065613380342\\
493	0.00483995091478616\\
494	0.00486953070053233\\
495	0.0048993674315073\\
496	0.00492942922729178\\
497	0.00495968036195307\\
498	0.00499008137895352\\
499	0.00502058946022384\\
500	0.00505115916793818\\
501	0.00508174373103509\\
502	0.00511229716796209\\
503	0.00514277748512175\\
504	0.00517315143935459\\
505	0.00520340138261903\\
506	0.00523353332555396\\
507	0.00526359232558662\\
508	0.005293685216728\\
509	0.00532401058150215\\
510	0.00535487809816213\\
511	0.00538650320169562\\
512	0.00541896601711208\\
513	0.0054523077533292\\
514	0.00548657443382623\\
515	0.00552181785933801\\
516	0.00555809814807075\\
517	0.00559548305717446\\
518	0.00563404906956839\\
519	0.00567388228072107\\
520	0.00571507818832832\\
521	0.00575774030131697\\
522	0.00580198076408421\\
523	0.00584791935643933\\
524	0.00589567988748809\\
525	0.00594538341309121\\
526	0.00599709568295925\\
527	0.00605087853310064\\
528	0.00610682270070741\\
529	0.00616501674035226\\
530	0.00622554406078374\\
531	0.00628847479267532\\
532	0.00635385821429033\\
533	0.00642172861966806\\
534	0.00649209994009733\\
535	0.00656495654067599\\
536	0.006640281558746\\
537	0.00671802543277983\\
538	0.00679805038301329\\
539	0.00688017893851313\\
540	0.00696422364181383\\
541	0.00704985310511505\\
542	0.00713651715111697\\
543	0.0072233324844939\\
544	0.00730733886327977\\
545	0.00738638960302661\\
546	0.00745982539142606\\
547	0.00752713644055451\\
548	0.00758836051141061\\
549	0.00764653857858205\\
550	0.00770271529057265\\
551	0.00775720579591524\\
552	0.00781071095053047\\
553	0.00786393070041828\\
554	0.00791723492564974\\
555	0.00797081310434071\\
556	0.0080248179508191\\
557	0.00807939150536912\\
558	0.00813468278734686\\
559	0.00819081823622241\\
560	0.0082479087802371\\
561	0.00830477765193133\\
562	0.00836059325980605\\
563	0.00841510282450088\\
564	0.00846832031068026\\
565	0.00852161182516532\\
566	0.00857497379044075\\
567	0.00862842855953914\\
568	0.00868200623738066\\
569	0.00873571853626056\\
570	0.00878953117589831\\
571	0.0088418987105902\\
572	0.00889242540907162\\
573	0.00894187213284206\\
574	0.00899097596863721\\
575	0.00903979684709103\\
576	0.0090883736771921\\
577	0.00913644094746051\\
578	0.00918373619157009\\
579	0.00923062808335025\\
580	0.0092772493210446\\
581	0.00932358230216704\\
582	0.00936958177756908\\
583	0.00941519618323153\\
584	0.00946037485366822\\
585	0.00950507047593196\\
586	0.00954924140305753\\
587	0.00959285439625311\\
588	0.00963588790901514\\
589	0.0096783357954337\\
590	0.00972021084245027\\
591	0.00976138933978722\\
592	0.00980171928775089\\
593	0.00984102135019415\\
594	0.00987905595166784\\
595	0.00991543381058343\\
596	0.00994937493726701\\
597	0.00997906286423442\\
598	0.0099999191923403\\
599	0\\
600	0\\
};
\addplot [color=red!40!mycolor19,solid,forget plot]
  table[row sep=crcr]{%
1	0.0030112771484627\\
2	0.00301128212363723\\
3	0.00301128718813487\\
4	0.00301129234355798\\
5	0.00301129759153769\\
6	0.00301130293373427\\
7	0.0030113083718378\\
8	0.00301131390756862\\
9	0.00301131954267788\\
10	0.00301132527894816\\
11	0.00301133111819388\\
12	0.00301133706226203\\
13	0.00301134311303272\\
14	0.00301134927241964\\
15	0.00301135554237081\\
16	0.00301136192486916\\
17	0.00301136842193308\\
18	0.00301137503561719\\
19	0.00301138176801284\\
20	0.00301138862124889\\
21	0.00301139559749225\\
22	0.00301140269894872\\
23	0.00301140992786355\\
24	0.00301141728652219\\
25	0.00301142477725101\\
26	0.00301143240241807\\
27	0.00301144016443379\\
28	0.00301144806575179\\
29	0.00301145610886957\\
30	0.00301146429632934\\
31	0.00301147263071887\\
32	0.00301148111467217\\
33	0.00301148975087046\\
34	0.00301149854204294\\
35	0.00301150749096766\\
36	0.00301151660047236\\
37	0.00301152587343541\\
38	0.00301153531278668\\
39	0.00301154492150848\\
40	0.00301155470263645\\
41	0.00301156465926061\\
42	0.0030115747945262\\
43	0.00301158511163475\\
44	0.00301159561384511\\
45	0.00301160630447436\\
46	0.00301161718689893\\
47	0.00301162826455568\\
48	0.0030116395409429\\
49	0.0030116510196215\\
50	0.00301166270421601\\
51	0.00301167459841581\\
52	0.00301168670597626\\
53	0.00301169903071984\\
54	0.00301171157653739\\
55	0.00301172434738933\\
56	0.00301173734730686\\
57	0.00301175058039324\\
58	0.0030117640508251\\
59	0.00301177776285367\\
60	0.00301179172080623\\
61	0.00301180592908736\\
62	0.00301182039218036\\
63	0.0030118351146486\\
64	0.00301185010113706\\
65	0.00301186535637363\\
66	0.00301188088517075\\
67	0.00301189669242676\\
68	0.00301191278312753\\
69	0.00301192916234797\\
70	0.00301194583525363\\
71	0.0030119628071023\\
72	0.0030119800832457\\
73	0.00301199766913105\\
74	0.00301201557030285\\
75	0.00301203379240457\\
76	0.00301205234118046\\
77	0.00301207122247726\\
78	0.00301209044224609\\
79	0.00301211000654424\\
80	0.00301212992153708\\
81	0.00301215019350007\\
82	0.00301217082882056\\
83	0.00301219183399986\\
84	0.00301221321565528\\
85	0.00301223498052212\\
86	0.00301225713545579\\
87	0.003012279687434\\
88	0.00301230264355882\\
89	0.00301232601105897\\
90	0.00301234979729207\\
91	0.00301237400974681\\
92	0.0030123986560454\\
93	0.00301242374394588\\
94	0.0030124492813445\\
95	0.00301247527627825\\
96	0.00301250173692723\\
97	0.00301252867161732\\
98	0.00301255608882265\\
99	0.00301258399716825\\
100	0.00301261240543275\\
101	0.00301264132255103\\
102	0.00301267075761706\\
103	0.00301270071988666\\
104	0.00301273121878037\\
105	0.00301276226388639\\
106	0.00301279386496348\\
107	0.00301282603194397\\
108	0.00301285877493694\\
109	0.00301289210423115\\
110	0.00301292603029838\\
111	0.00301296056379656\\
112	0.00301299571557307\\
113	0.00301303149666809\\
114	0.00301306791831796\\
115	0.0030131049919587\\
116	0.00301314272922949\\
117	0.0030131811419762\\
118	0.00301322024225507\\
119	0.00301326004233647\\
120	0.00301330055470852\\
121	0.00301334179208103\\
122	0.00301338376738934\\
123	0.00301342649379836\\
124	0.00301346998470643\\
125	0.00301351425374962\\
126	0.00301355931480574\\
127	0.00301360518199869\\
128	0.00301365186970269\\
129	0.00301369939254675\\
130	0.00301374776541901\\
131	0.00301379700347138\\
132	0.00301384712212412\\
133	0.00301389813707056\\
134	0.0030139500642818\\
135	0.00301400292001168\\
136	0.00301405672080159\\
137	0.00301411148348558\\
138	0.00301416722519542\\
139	0.00301422396336584\\
140	0.00301428171573976\\
141	0.00301434050037368\\
142	0.00301440033564319\\
143	0.00301446124024844\\
144	0.00301452323321978\\
145	0.00301458633392366\\
146	0.00301465056206828\\
147	0.00301471593770963\\
148	0.0030147824812575\\
149	0.00301485021348162\\
150	0.00301491915551792\\
151	0.00301498932887487\\
152	0.00301506075543991\\
153	0.00301513345748602\\
154	0.00301520745767843\\
155	0.00301528277908132\\
156	0.00301535944516481\\
157	0.00301543747981191\\
158	0.00301551690732563\\
159	0.00301559775243631\\
160	0.00301568004030886\\
161	0.00301576379655038\\
162	0.00301584904721765\\
163	0.00301593581882494\\
164	0.00301602413835183\\
165	0.00301611403325125\\
166	0.00301620553145757\\
167	0.00301629866139481\\
168	0.00301639345198515\\
169	0.00301648993265734\\
170	0.00301658813335542\\
171	0.00301668808454757\\
172	0.00301678981723495\\
173	0.00301689336296089\\
174	0.00301699875382012\\
175	0.00301710602246815\\
176	0.00301721520213082\\
177	0.00301732632661396\\
178	0.00301743943031334\\
179	0.00301755454822457\\
180	0.00301767171595336\\
181	0.00301779096972585\\
182	0.00301791234639907\\
183	0.00301803588347162\\
184	0.00301816161909453\\
185	0.00301828959208228\\
186	0.00301841984192393\\
187	0.00301855240879453\\
188	0.00301868733356669\\
189	0.00301882465782218\\
190	0.00301896442386401\\
191	0.00301910667472834\\
192	0.00301925145419692\\
193	0.0030193988068095\\
194	0.00301954877787644\\
195	0.00301970141349166\\
196	0.00301985676054566\\
197	0.00302001486673878\\
198	0.0030201757805947\\
199	0.00302033955147407\\
200	0.00302050622958839\\
201	0.00302067586601417\\
202	0.00302084851270721\\
203	0.00302102422251701\\
204	0.00302120304920175\\
205	0.00302138504744302\\
206	0.00302157027286114\\
207	0.00302175878203055\\
208	0.00302195063249546\\
209	0.00302214588278573\\
210	0.00302234459243299\\
211	0.00302254682198701\\
212	0.00302275263303224\\
213	0.00302296208820471\\
214	0.0030231752512091\\
215	0.00302339218683602\\
216	0.0030236129609797\\
217	0.00302383764065569\\
218	0.00302406629401913\\
219	0.00302429899038295\\
220	0.00302453580023652\\
221	0.00302477679526461\\
222	0.00302502204836646\\
223	0.00302527163367521\\
224	0.0030255256265776\\
225	0.00302578410373393\\
226	0.00302604714309833\\
227	0.00302631482393923\\
228	0.00302658722686019\\
229	0.00302686443382102\\
230	0.00302714652815906\\
231	0.00302743359461096\\
232	0.00302772571933458\\
233	0.00302802298993118\\
234	0.00302832549546807\\
235	0.00302863332650143\\
236	0.0030289465750994\\
237	0.00302926533486556\\
238	0.00302958970096271\\
239	0.00302991977013685\\
240	0.00303025564074161\\
241	0.00303059741276294\\
242	0.00303094518784404\\
243	0.00303129906931069\\
244	0.00303165916219689\\
245	0.00303202557327074\\
246	0.0030323984110607\\
247	0.00303277778588221\\
248	0.00303316380986452\\
249	0.00303355659697799\\
250	0.00303395626306149\\
251	0.00303436292585038\\
252	0.0030347767050046\\
253	0.00303519772213726\\
254	0.0030356261008434\\
255	0.0030360619667292\\
256	0.00303650544744144\\
257	0.0030369566726973\\
258	0.00303741577431452\\
259	0.00303788288624183\\
260	0.00303835814458976\\
261	0.00303884168766177\\
262	0.00303933365598564\\
263	0.00303983419234523\\
264	0.00304034344181264\\
265	0.00304086155178052\\
266	0.00304138867199482\\
267	0.00304192495458787\\
268	0.00304247055411169\\
269	0.00304302562757174\\
270	0.00304359033446077\\
271	0.00304416483679328\\
272	0.00304474929914\\
273	0.00304534388866292\\
274	0.00304594877515046\\
275	0.00304656413105293\\
276	0.00304719013151849\\
277	0.00304782695442915\\
278	0.0030484747804373\\
279	0.0030491337930023\\
280	0.00304980417842765\\
281	0.00305048612589808\\
282	0.00305117982751735\\
283	0.00305188547834593\\
284	0.00305260327643924\\
285	0.00305333342288612\\
286	0.00305407612184748\\
287	0.00305483158059534\\
288	0.00305560000955211\\
289	0.00305638162233016\\
290	0.00305717663577175\\
291	0.00305798526998912\\
292	0.00305880774840501\\
293	0.00305964429779349\\
294	0.00306049514832087\\
295	0.00306136053358738\\
296	0.0030622406906688\\
297	0.0030631358601587\\
298	0.00306404628621093\\
299	0.00306497221658262\\
300	0.00306591390267762\\
301	0.00306687159959031\\
302	0.00306784556615013\\
303	0.00306883606496656\\
304	0.00306984336247477\\
305	0.00307086772898195\\
306	0.00307190943871459\\
307	0.00307296876986641\\
308	0.00307404600464744\\
309	0.0030751414293342\\
310	0.00307625533432112\\
311	0.00307738801417344\\
312	0.00307853976768165\\
313	0.00307971089791798\\
314	0.00308090171229491\\
315	0.00308211252262612\\
316	0.00308334364519038\\
317	0.00308459540079854\\
318	0.00308586811486459\\
319	0.00308716211748075\\
320	0.00308847774349799\\
321	0.00308981533261227\\
322	0.00309117522945808\\
323	0.00309255778370995\\
324	0.00309396335019373\\
325	0.00309539228900961\\
326	0.00309684496566861\\
327	0.00309832175124565\\
328	0.00309982302255223\\
329	0.00310134916233282\\
330	0.00310290055949007\\
331	0.00310447760934495\\
332	0.00310608071393943\\
333	0.00310771028239149\\
334	0.00310936673131448\\
335	0.0031110504853147\\
336	0.00311276197758505\\
337	0.00311450165061221\\
338	0.00311626995701265\\
339	0.00311806736049661\\
340	0.00311989433692556\\
341	0.00312175137539228\\
342	0.00312363897943671\\
343	0.003125557670164\\
344	0.0031275080020705\\
345	0.00312949059349016\\
346	0.00313150609232436\\
347	0.00313355517864743\\
348	0.00313563856738285\\
349	0.0031377570110202\\
350	0.00313991130239194\\
351	0.00314210227763664\\
352	0.00314433081954981\\
353	0.00314659786080227\\
354	0.00314890438113493\\
355	0.00315125136603214\\
356	0.00315363958019717\\
357	0.00315606978981859\\
358	0.00315854278111835\\
359	0.00316105936144412\\
360	0.00316362036046411\\
361	0.00316622663147663\\
362	0.00316887905284827\\
363	0.00317157852959675\\
364	0.00317432599513677\\
365	0.00317712241320975\\
366	0.00317996878002174\\
367	0.00318286612661715\\
368	0.00318581552152086\\
369	0.00318881807368557\\
370	0.00319187493578809\\
371	0.0031949873079246\\
372	0.00319815644176414\\
373	0.00320138364522874\\
374	0.0032046702877813\\
375	0.00320801780641599\\
376	0.00321142771246325\\
377	0.00321490159934192\\
378	0.00321844115141562\\
379	0.0032220481541401\\
380	0.00322572450572489\\
381	0.00322947223057536\\
382	0.003233293494836\\
383	0.00323719062441954\\
384	0.00324116612598733\\
385	0.0032452227114445\\
386	0.00324936332663442\\
387	0.00325359118506684\\
388	0.00325790980770063\\
389	0.00326232307003289\\
390	0.00326683525803616\\
391	0.00327145113484695\\
392	0.00327617602056441\\
393	0.00328101588809075\\
394	0.00328597747866928\\
395	0.00329106844169038\\
396	0.00329629750449335\\
397	0.00330167467934666\\
398	0.00330721151660432\\
399	0.00331292141582399\\
400	0.0033188200096813\\
401	0.00332492563927259\\
402	0.00333125994479976\\
403	0.00333784860241919\\
404	0.00334472224685625\\
405	0.00335191763085759\\
406	0.00335947908746879\\
407	0.00336746038041965\\
408	0.00337592705245904\\
409	0.00338495941124689\\
410	0.00339465632310242\\
411	0.00340513999615007\\
412	0.0034165618451223\\
413	0.00342910904764097\\
414	0.00344300944672684\\
415	0.00345755801444051\\
416	0.00347234515899185\\
417	0.00348737409996265\\
418	0.00350264800731524\\
419	0.00351816984114086\\
420	0.0035339418067238\\
421	0.00354996568105268\\
422	0.00356624732542498\\
423	0.00358279709431792\\
424	0.00359963373631932\\
425	0.00361676185217926\\
426	0.00363418607612658\\
427	0.00365191106702642\\
428	0.00366994149738172\\
429	0.00368828203972282\\
430	0.00370693734984093\\
431	0.00372591204624422\\
432	0.00374521068514317\\
433	0.00376483772985987\\
434	0.0037847975113347\\
435	0.00380509417336685\\
436	0.00382573160994369\\
437	0.00384671338743754\\
438	0.00386804264795385\\
439	0.00388972198915353\\
440	0.00391175331465244\\
441	0.0039341376475423\\
442	0.00395687489758588\\
443	0.00397996357008652\\
444	0.00400340040115312\\
445	0.00402717989986627\\
446	0.00405129377242333\\
447	0.00407573019637187\\
448	0.0041004729041679\\
449	0.00412550002427073\\
450	0.00415078261515364\\
451	0.00417628281527402\\
452	0.00420195147286436\\
453	0.00422772500064694\\
454	0.00425352122695221\\
455	0.00427923405601563\\
456	0.00430472705272688\\
457	0.00432982539830094\\
458	0.00435430509663297\\
459	0.00437787912462104\\
460	0.00440018001228618\\
461	0.00442073945979756\\
462	0.00443897091856598\\
463	0.00445725960360944\\
464	0.00447580055336444\\
465	0.00449458924557552\\
466	0.00451361983613378\\
467	0.00453288517536736\\
468	0.00455237665129548\\
469	0.00457208401945484\\
470	0.00459199521965731\\
471	0.00461209618082414\\
472	0.00463237061681782\\
473	0.00465279982181396\\
474	0.00467336248284928\\
475	0.00469403452956161\\
476	0.0047147890479902\\
477	0.00473559627477015\\
478	0.00475642369917354\\
479	0.00477723634335092\\
480	0.00479799731346455\\
481	0.00481866874103088\\
482	0.00483921327983798\\
483	0.00485959638845276\\
484	0.00487978971827209\\
485	0.00489977605615691\\
486	0.00491955646208567\\
487	0.00493916053055873\\
488	0.00495866108989554\\
489	0.00497819474011869\\
490	0.00499796145418809\\
491	0.0050180776608813\\
492	0.00503855184396394\\
493	0.00505939348956257\\
494	0.0050806133032376\\
495	0.00510222347856746\\
496	0.00512423801255947\\
497	0.00514667307048559\\
498	0.00516954748368211\\
499	0.00519288328983599\\
500	0.00521670633080219\\
501	0.00524104668724238\\
502	0.00526593805116734\\
503	0.00529141966021656\\
504	0.00531753775955137\\
505	0.00534435061159844\\
506	0.00537195477964907\\
507	0.0054004263733106\\
508	0.00542983977475685\\
509	0.00546026936773654\\
510	0.00549178321330305\\
511	0.00552444340442616\\
512	0.00555831555240889\\
513	0.00559346761621968\\
514	0.00562997120760527\\
515	0.00566789562440533\\
516	0.00570725572049871\\
517	0.00574812831483058\\
518	0.0057905930740293\\
519	0.0058347319712251\\
520	0.00588062854546756\\
521	0.0059283669909499\\
522	0.00597803091455968\\
523	0.00602970178255804\\
524	0.0060834574252944\\
525	0.00613936985774463\\
526	0.00619754555422584\\
527	0.00625807035375208\\
528	0.00632097800128714\\
529	0.00638628421195292\\
530	0.00645398122847025\\
531	0.00652402827496695\\
532	0.00659634488034116\\
533	0.00667080369120464\\
534	0.00674722577632755\\
535	0.00682543605429951\\
536	0.00690517067081394\\
537	0.006985979879753\\
538	0.0070671353684765\\
539	0.00714694473789928\\
540	0.00722171707236833\\
541	0.00729072305803338\\
542	0.00735364306074771\\
543	0.00741047186832885\\
544	0.00746351245621795\\
545	0.00751458096933631\\
546	0.0075642019222949\\
547	0.00761304925249949\\
548	0.00766168557552592\\
549	0.00771048316522664\\
550	0.00775963560327215\\
551	0.00780930056370565\\
552	0.00785961270762311\\
553	0.00791065670204328\\
554	0.00796248760120481\\
555	0.00801519227973258\\
556	0.00806886380586354\\
557	0.00812360765985678\\
558	0.0081776104431567\\
559	0.00823054447228591\\
560	0.00828212481938252\\
561	0.00833330018496336\\
562	0.00838468146777848\\
563	0.00843628943921433\\
564	0.0084881762700586\\
565	0.00854036375010845\\
566	0.00859280326717225\\
567	0.00864547347042305\\
568	0.00869831647947506\\
569	0.00874951291989383\\
570	0.0087989032234281\\
571	0.00884783920707862\\
572	0.00889659845102266\\
573	0.00894523117842181\\
574	0.0089937388115468\\
575	0.00904204227672353\\
576	0.00908965434686499\\
577	0.00913682916240663\\
578	0.0091838363449705\\
579	0.00923066223904429\\
580	0.00927726451044059\\
581	0.00932359040395203\\
582	0.00936958612525249\\
583	0.00941519840888499\\
584	0.00946037591227052\\
585	0.00950507093368084\\
586	0.00954924157818994\\
587	0.00959285445321423\\
588	0.00963588792379213\\
589	0.00967833579815023\\
590	0.00972021084271538\\
591	0.00976138933978722\\
592	0.00980171928775088\\
593	0.00984102135019415\\
594	0.00987905595166784\\
595	0.00991543381058343\\
596	0.00994937493726701\\
597	0.00997906286423442\\
598	0.0099999191923403\\
599	0\\
600	0\\
};
\addplot [color=red!75!mycolor17,solid,forget plot]
  table[row sep=crcr]{%
1	0.00330282272054889\\
2	0.00330283342163309\\
3	0.00330284431478508\\
4	0.00330285540344943\\
5	0.00330286669113244\\
6	0.00330287818140321\\
7	0.00330288987789482\\
8	0.0033029017843054\\
9	0.00330291390439932\\
10	0.00330292624200841\\
11	0.00330293880103316\\
12	0.00330295158544386\\
13	0.00330296459928195\\
14	0.00330297784666129\\
15	0.00330299133176942\\
16	0.00330300505886883\\
17	0.00330301903229842\\
18	0.00330303325647477\\
19	0.00330304773589353\\
20	0.00330306247513091\\
21	0.00330307747884505\\
22	0.00330309275177754\\
23	0.0033031082987548\\
24	0.00330312412468978\\
25	0.00330314023458331\\
26	0.00330315663352584\\
27	0.00330317332669894\\
28	0.00330319031937693\\
29	0.00330320761692861\\
30	0.00330322522481887\\
31	0.00330324314861042\\
32	0.00330326139396562\\
33	0.00330327996664815\\
34	0.0033032988725249\\
35	0.00330331811756777\\
36	0.00330333770785555\\
37	0.00330335764957587\\
38	0.00330337794902708\\
39	0.00330339861262027\\
40	0.0033034196468813\\
41	0.00330344105845283\\
42	0.00330346285409639\\
43	0.00330348504069449\\
44	0.00330350762525285\\
45	0.00330353061490255\\
46	0.00330355401690227\\
47	0.00330357783864058\\
48	0.00330360208763825\\
49	0.00330362677155063\\
50	0.00330365189817003\\
51	0.0033036774754282\\
52	0.00330370351139872\\
53	0.0033037300142997\\
54	0.00330375699249623\\
55	0.00330378445450304\\
56	0.00330381240898714\\
57	0.00330384086477062\\
58	0.00330386983083331\\
59	0.00330389931631573\\
60	0.0033039293305218\\
61	0.00330395988292187\\
62	0.00330399098315565\\
63	0.00330402264103528\\
64	0.00330405486654828\\
65	0.00330408766986083\\
66	0.00330412106132083\\
67	0.00330415505146126\\
68	0.00330418965100335\\
69	0.00330422487086005\\
70	0.0033042607221394\\
71	0.00330429721614801\\
72	0.00330433436439457\\
73	0.00330437217859355\\
74	0.00330441067066876\\
75	0.00330444985275714\\
76	0.00330448973721252\\
77	0.00330453033660952\\
78	0.00330457166374751\\
79	0.0033046137316545\\
80	0.00330465655359143\\
81	0.003304700143056\\
82	0.00330474451378721\\
83	0.00330478967976943\\
84	0.00330483565523691\\
85	0.00330488245467818\\
86	0.00330493009284055\\
87	0.00330497858473473\\
88	0.00330502794563957\\
89	0.00330507819110677\\
90	0.00330512933696575\\
91	0.00330518139932865\\
92	0.00330523439459526\\
93	0.00330528833945823\\
94	0.00330534325090826\\
95	0.00330539914623935\\
96	0.00330545604305425\\
97	0.00330551395926991\\
98	0.00330557291312314\\
99	0.00330563292317622\\
100	0.00330569400832267\\
101	0.00330575618779331\\
102	0.003305819481162\\
103	0.00330588390835195\\
104	0.00330594948964182\\
105	0.00330601624567204\\
106	0.0033060841974513\\
107	0.00330615336636303\\
108	0.00330622377417209\\
109	0.00330629544303151\\
110	0.00330636839548943\\
111	0.00330644265449607\\
112	0.00330651824341091\\
113	0.00330659518600989\\
114	0.00330667350649289\\
115	0.00330675322949116\\
116	0.00330683438007506\\
117	0.00330691698376181\\
118	0.00330700106652339\\
119	0.00330708665479459\\
120	0.00330717377548134\\
121	0.00330726245596888\\
122	0.00330735272413047\\
123	0.0033074446083358\\
124	0.00330753813745996\\
125	0.00330763334089236\\
126	0.0033077302485458\\
127	0.00330782889086571\\
128	0.00330792929883973\\
129	0.00330803150400712\\
130	0.00330813553846866\\
131	0.00330824143489649\\
132	0.00330834922654435\\
133	0.00330845894725767\\
134	0.00330857063148426\\
135	0.00330868431428477\\
136	0.00330880003134363\\
137	0.00330891781898009\\
138	0.00330903771415932\\
139	0.00330915975450399\\
140	0.00330928397830574\\
141	0.00330941042453704\\
142	0.00330953913286325\\
143	0.00330967014365478\\
144	0.00330980349799959\\
145	0.00330993923771584\\
146	0.00331007740536477\\
147	0.00331021804426372\\
148	0.00331036119849963\\
149	0.00331050691294246\\
150	0.00331065523325904\\
151	0.0033108062059271\\
152	0.00331095987824959\\
153	0.00331111629836915\\
154	0.00331127551528291\\
155	0.00331143757885755\\
156	0.00331160253984459\\
157	0.00331177044989592\\
158	0.00331194136157969\\
159	0.00331211532839636\\
160	0.0033122924047951\\
161	0.00331247264619046\\
162	0.00331265610897938\\
163	0.00331284285055829\\
164	0.00331303292934082\\
165	0.00331322640477553\\
166	0.00331342333736411\\
167	0.00331362378867986\\
168	0.00331382782138642\\
169	0.00331403549925693\\
170	0.00331424688719343\\
171	0.0033144620512466\\
172	0.00331468105863596\\
173	0.00331490397777025\\
174	0.00331513087826821\\
175	0.00331536183097977\\
176	0.00331559690800757\\
177	0.00331583618272889\\
178	0.00331607972981778\\
179	0.00331632762526781\\
180	0.0033165799464151\\
181	0.00331683677196163\\
182	0.00331709818199918\\
183	0.00331736425803347\\
184	0.00331763508300882\\
185	0.00331791074133321\\
186	0.00331819131890377\\
187	0.00331847690313258\\
188	0.00331876758297316\\
189	0.0033190634489472\\
190	0.00331936459317178\\
191	0.00331967110938712\\
192	0.00331998309298471\\
193	0.00332030064103599\\
194	0.00332062385232146\\
195	0.00332095282736034\\
196	0.00332128766844062\\
197	0.00332162847964977\\
198	0.0033219753669058\\
199	0.00332232843798895\\
200	0.00332268780257393\\
201	0.00332305357226255\\
202	0.00332342586061704\\
203	0.00332380478319396\\
204	0.00332419045757847\\
205	0.00332458300341937\\
206	0.00332498254246467\\
207	0.00332538919859769\\
208	0.00332580309787378\\
209	0.00332622436855776\\
210	0.00332665314116178\\
211	0.003327089548484\\
212	0.00332753372564779\\
213	0.00332798581014162\\
214	0.00332844594185967\\
215	0.0033289142631429\\
216	0.00332939091882108\\
217	0.0033298760562552\\
218	0.00333036982538093\\
219	0.00333087237875247\\
220	0.00333138387158732\\
221	0.00333190446181167\\
222	0.0033324343101066\\
223	0.00333297357995497\\
224	0.00333352243768914\\
225	0.00333408105253948\\
226	0.00333464959668346\\
227	0.00333522824529588\\
228	0.00333581717659961\\
229	0.00333641657191728\\
230	0.00333702661572385\\
231	0.0033376474956999\\
232	0.00333827940278585\\
233	0.0033389225312371\\
234	0.00333957707867991\\
235	0.00334024324616826\\
236	0.00334092123824171\\
237	0.00334161126298399\\
238	0.00334231353208262\\
239	0.00334302826088954\\
240	0.00334375566848259\\
241	0.00334449597772797\\
242	0.00334524941534377\\
243	0.0033460162119644\\
244	0.00334679660220614\\
245	0.0033475908247336\\
246	0.00334839912232735\\
247	0.00334922174195242\\
248	0.00335005893482809\\
249	0.00335091095649854\\
250	0.00335177806690484\\
251	0.00335266053045776\\
252	0.00335355861611198\\
253	0.00335447259744131\\
254	0.00335540275271499\\
255	0.00335634936497527\\
256	0.00335731272211621\\
257	0.0033582931169635\\
258	0.00335929084735563\\
259	0.00336030621622627\\
260	0.00336133953168787\\
261	0.00336239110711638\\
262	0.0033634612612376\\
263	0.00336455031821432\\
264	0.00336565860773516\\
265	0.0033667864651045\\
266	0.0033679342313338\\
267	0.00336910225323429\\
268	0.00337029088351086\\
269	0.00337150048085747\\
270	0.00337273141005392\\
271	0.00337398404206385\\
272	0.00337525875413432\\
273	0.00337655592989667\\
274	0.00337787595946882\\
275	0.00337921923955908\\
276	0.00338058617357127\\
277	0.00338197717171143\\
278	0.00338339265109577\\
279	0.00338483303586036\\
280	0.00338629875727196\\
281	0.00338779025384068\\
282	0.00338930797143373\\
283	0.00339085236339104\\
284	0.00339242389064208\\
285	0.00339402302182419\\
286	0.0033956502334026\\
287	0.00339730600979159\\
288	0.00339899084347746\\
289	0.00340070523514277\\
290	0.00340244969379195\\
291	0.00340422473687872\\
292	0.00340603089043444\\
293	0.00340786868919834\\
294	0.00340973867674884\\
295	0.0034116414056363\\
296	0.00341357743751731\\
297	0.00341554734328991\\
298	0.00341755170323048\\
299	0.00341959110713156\\
300	0.00342166615444105\\
301	0.00342377745440246\\
302	0.00342592562619619\\
303	0.0034281112990819\\
304	0.00343033511254174\\
305	0.00343259771642438\\
306	0.00343489977108982\\
307	0.00343724194755478\\
308	0.00343962492763856\\
309	0.00344204940410927\\
310	0.00344451608083008\\
311	0.00344702567290555\\
312	0.00344957890682761\\
313	0.00345217652062094\\
314	0.00345481926398756\\
315	0.00345750789845017\\
316	0.00346024319749367\\
317	0.00346302594670484\\
318	0.00346585694390895\\
319	0.00346873699930295\\
320	0.00347166693558459\\
321	0.00347464758807583\\
322	0.00347767980484001\\
323	0.00348076444679074\\
324	0.00348390238779121\\
325	0.00348709451474115\\
326	0.00349034172764914\\
327	0.00349364493968681\\
328	0.0034970050772207\\
329	0.0035004230798166\\
330	0.00350389990020994\\
331	0.00350743650423407\\
332	0.00351103387069592\\
333	0.00351469299118633\\
334	0.00351841486980706\\
335	0.00352220052279181\\
336	0.00352605097798613\\
337	0.00352996727412774\\
338	0.00353395045980457\\
339	0.00353800159178545\\
340	0.00354212173185045\\
341	0.00354631193944075\\
342	0.00355057325157021\\
343	0.00355490662209339\\
344	0.00355931272808235\\
345	0.00356379189166622\\
346	0.00356834515469216\\
347	0.00357297356679901\\
348	0.00357767818776288\\
349	0.00358246009188216\\
350	0.00358732037653211\\
351	0.00359226018023511\\
352	0.00359728072439287\\
353	0.00360238341730273\\
354	0.00360757012792604\\
355	0.00361284393236545\\
356	0.00361821119603823\\
357	0.00362367390982745\\
358	0.00362923375132823\\
359	0.00363489242754245\\
360	0.00364065167534496\\
361	0.00364651326193577\\
362	0.00365247898527317\\
363	0.00365855067448048\\
364	0.00366473019021816\\
365	0.00367101942501133\\
366	0.0036774203035199\\
367	0.00368393478273699\\
368	0.00369056485209686\\
369	0.003697312533471\\
370	0.00370417988102549\\
371	0.00371116898090746\\
372	0.00371828195072195\\
373	0.00372552093875149\\
374	0.00373288812286142\\
375	0.00374038570902078\\
376	0.00374801592935452\\
377	0.00375578103962323\\
378	0.00376368331600495\\
379	0.00377172505102493\\
380	0.00377990854844547\\
381	0.00378823611688516\\
382	0.0037967100618844\\
383	0.00380533267606928\\
384	0.00381410622698426\\
385	0.00382303294206365\\
386	0.00383211499008518\\
387	0.00384135445829135\\
388	0.00385075332416488\\
389	0.00386031342059559\\
390	0.00387003639285955\\
391	0.00387992364543375\\
392	0.0038899762761649\\
393	0.0039001949946742\\
394	0.00391058002108038\\
395	0.00392113096013745\\
396	0.00393184664472379\\
397	0.00394272494144054\\
398	0.00395376251048509\\
399	0.00396495449586735\\
400	0.00397629413257958\\
401	0.00398777225813496\\
402	0.00399937669498494\\
403	0.00401109146763229\\
404	0.00402289580756077\\
405	0.00403476288509919\\
406	0.00404665818896517\\
407	0.00405853745010709\\
408	0.00407034397486529\\
409	0.00408200521173058\\
410	0.0040934283265161\\
411	0.00410449451422482\\
412	0.00411505185245532\\
413	0.00412490688083491\\
414	0.00413381680457718\\
415	0.00414246359764626\\
416	0.00415125134065076\\
417	0.00416018176530774\\
418	0.00416925659035271\\
419	0.00417847751893866\\
420	0.00418784624035994\\
421	0.00419736444267878\\
422	0.00420703380198752\\
423	0.00421685584956833\\
424	0.00422683181202162\\
425	0.00423696240349508\\
426	0.00424724815917203\\
427	0.00425768941180764\\
428	0.00426828626551413\\
429	0.00427903856647297\\
430	0.00428994587021759\\
431	0.00430100740509318\\
432	0.00431222203145255\\
433	0.00432358819609439\\
434	0.00433510388148189\\
435	0.00434676654945613\\
436	0.00435857307901268\\
437	0.00437051969760218\\
438	0.00438260190559301\\
439	0.00439481439365558\\
440	0.00440715095302605\\
441	0.00441960437891222\\
442	0.00443216636776108\\
443	0.00444482740977402\\
444	0.00445757667901259\\
445	0.00447040192479705\\
446	0.00448328937001098\\
447	0.00449622362459359\\
448	0.00450918762620207\\
449	0.00452216262508911\\
450	0.00453512823714873\\
451	0.00454806259823822\\
452	0.00456094266652204\\
453	0.00457374474231479\\
454	0.00458644530390559\\
455	0.00459902229070082\\
456	0.00461145699868897\\
457	0.00462373681301679\\
458	0.00463585913954865\\
459	0.00464783703643459\\
460	0.00465970727157742\\
461	0.00467154172123509\\
462	0.00468346221395803\\
463	0.00469554963119461\\
464	0.00470781029851564\\
465	0.00472024469819738\\
466	0.00473285328017813\\
467	0.00474563649024712\\
468	0.00475859480719211\\
469	0.00477172879083086\\
470	0.00478503914318833\\
471	0.00479852678546889\\
472	0.00481219295390058\\
473	0.0048260393178048\\
474	0.00484006812319904\\
475	0.00485428236509176\\
476	0.0048686859910726\\
477	0.00488328413879259\\
478	0.00489808340948448\\
479	0.00491309217733719\\
480	0.00492832090057559\\
481	0.0049437824638965\\
482	0.00495949254182109\\
483	0.0049754699354219\\
484	0.00499173683140299\\
485	0.00500831890355531\\
486	0.00502524513177611\\
487	0.00504254713582935\\
488	0.00506025766691289\\
489	0.0050784076222585\\
490	0.00509702183527401\\
491	0.00511612026777347\\
492	0.00513572427144415\\
493	0.00515585814725386\\
494	0.00517654827885316\\
495	0.005197823043026\\
496	0.0052197129940454\\
497	0.00524225100489788\\
498	0.00526547092274689\\
499	0.00528940925192003\\
500	0.00531410682208839\\
501	0.00533961427475088\\
502	0.00536600662699488\\
503	0.00539333449201868\\
504	0.00542164384722157\\
505	0.00545093317709244\\
506	0.00548125584165916\\
507	0.00551266631417104\\
508	0.00554522086648375\\
509	0.00557897776063199\\
510	0.00561399747215752\\
511	0.00565034295582015\\
512	0.00568807961924574\\
513	0.00572727514130953\\
514	0.0057679989351695\\
515	0.00581032699449293\\
516	0.00585439423092452\\
517	0.00590027932527563\\
518	0.00594806052659907\\
519	0.00599781429473797\\
520	0.00604961359373404\\
521	0.00610352620502252\\
522	0.00615961228747102\\
523	0.00621792029876914\\
524	0.0062784771031797\\
525	0.0063412833477263\\
526	0.00640631260834304\\
527	0.00647349981735246\\
528	0.00654273540940123\\
529	0.00661386712290187\\
530	0.0066866853889961\\
531	0.00676099618491231\\
532	0.00683645871329546\\
533	0.00691251557173578\\
534	0.00698820960614814\\
535	0.0070600184570712\\
536	0.00712619182055303\\
537	0.00718635518837399\\
538	0.00724038537103614\\
539	0.0072891926067398\\
540	0.00733611246284593\\
541	0.00738175075357763\\
542	0.00742656905747836\\
543	0.00747111430771151\\
544	0.00751581455378255\\
545	0.00756089746622922\\
546	0.00760651824603813\\
547	0.00765279954637135\\
548	0.00769981995261393\\
549	0.00774763076508289\\
550	0.00779627233522115\\
551	0.00784578270503273\\
552	0.00789623511963474\\
553	0.00794771179525126\\
554	0.00800011671826545\\
555	0.00805165496701063\\
556	0.00810209881618632\\
557	0.0081511534873175\\
558	0.00820039187507906\\
559	0.00824993373640694\\
560	0.00829982916153586\\
561	0.00835013282549448\\
562	0.00840084161324492\\
563	0.00845191747075763\\
564	0.00850331152133759\\
565	0.0085549989209369\\
566	0.00860690126516819\\
567	0.00865718321445381\\
568	0.00870571147056739\\
569	0.00875412101410349\\
570	0.00880247830324436\\
571	0.00885082904818314\\
572	0.00889914614462142\\
573	0.00894739710985454\\
574	0.00899525183449353\\
575	0.00904253590053291\\
576	0.00908973605534923\\
577	0.0091368486117437\\
578	0.00918384152338781\\
579	0.00923066477496741\\
580	0.00927726586124061\\
581	0.00932359110588857\\
582	0.00936958646951639\\
583	0.0094151985649821\\
584	0.00946037597636478\\
585	0.00950507095687452\\
586	0.00954924158529672\\
587	0.0095928544549437\\
588	0.00963588792408897\\
589	0.00967833579817709\\
590	0.00972021084271539\\
591	0.00976138933978722\\
592	0.00980171928775089\\
593	0.00984102135019415\\
594	0.00987905595166785\\
595	0.00991543381058343\\
596	0.00994937493726701\\
597	0.00997906286423442\\
598	0.0099999191923403\\
599	0\\
600	0\\
};
\addplot [color=red!80!mycolor19,solid,forget plot]
  table[row sep=crcr]{%
1	0.00391309393964753\\
2	0.00391310035145049\\
3	0.00391310687863073\\
4	0.00391311352326403\\
5	0.0039131202874635\\
6	0.00391312717338031\\
7	0.00391313418320423\\
8	0.00391314131916451\\
9	0.00391314858353044\\
10	0.00391315597861217\\
11	0.00391316350676137\\
12	0.00391317117037203\\
13	0.00391317897188121\\
14	0.00391318691376976\\
15	0.00391319499856321\\
16	0.00391320322883247\\
17	0.00391321160719473\\
18	0.00391322013631422\\
19	0.0039132288189031\\
20	0.00391323765772234\\
21	0.00391324665558253\\
22	0.00391325581534482\\
23	0.00391326513992186\\
24	0.00391327463227862\\
25	0.00391328429543347\\
26	0.003913294132459\\
27	0.00391330414648312\\
28	0.00391331434068997\\
29	0.00391332471832095\\
30	0.00391333528267575\\
31	0.00391334603711346\\
32	0.00391335698505355\\
33	0.00391336812997696\\
34	0.00391337947542727\\
35	0.00391339102501179\\
36	0.00391340278240267\\
37	0.00391341475133813\\
38	0.00391342693562365\\
39	0.00391343933913308\\
40	0.00391345196580999\\
41	0.00391346481966886\\
42	0.00391347790479635\\
43	0.00391349122535264\\
44	0.00391350478557267\\
45	0.00391351858976759\\
46	0.0039135326423261\\
47	0.00391354694771575\\
48	0.00391356151048447\\
49	0.00391357633526195\\
50	0.00391359142676115\\
51	0.0039136067897798\\
52	0.00391362242920189\\
53	0.0039136383499992\\
54	0.00391365455723295\\
55	0.00391367105605535\\
56	0.00391368785171129\\
57	0.00391370494953988\\
58	0.00391372235497636\\
59	0.00391374007355357\\
60	0.00391375811090393\\
61	0.00391377647276107\\
62	0.00391379516496177\\
63	0.00391381419344772\\
64	0.00391383356426739\\
65	0.00391385328357808\\
66	0.00391387335764775\\
67	0.00391389379285705\\
68	0.00391391459570137\\
69	0.00391393577279287\\
70	0.00391395733086256\\
71	0.00391397927676248\\
72	0.00391400161746784\\
73	0.00391402436007925\\
74	0.00391404751182495\\
75	0.00391407108006316\\
76	0.00391409507228431\\
77	0.00391411949611354\\
78	0.00391414435931302\\
79	0.00391416966978445\\
80	0.00391419543557158\\
81	0.00391422166486273\\
82	0.00391424836599338\\
83	0.00391427554744888\\
84	0.00391430321786703\\
85	0.00391433138604094\\
86	0.00391436006092166\\
87	0.00391438925162124\\
88	0.00391441896741536\\
89	0.00391444921774651\\
90	0.00391448001222679\\
91	0.00391451136064106\\
92	0.00391454327295003\\
93	0.00391457575929336\\
94	0.00391460882999292\\
95	0.00391464249555611\\
96	0.00391467676667905\\
97	0.00391471165425006\\
98	0.0039147471693531\\
99	0.00391478332327129\\
100	0.00391482012749046\\
101	0.00391485759370267\\
102	0.00391489573381023\\
103	0.00391493455992908\\
104	0.00391497408439289\\
105	0.00391501431975687\\
106	0.00391505527880174\\
107	0.0039150969745378\\
108	0.00391513942020896\\
109	0.00391518262929707\\
110	0.00391522661552606\\
111	0.00391527139286634\\
112	0.00391531697553919\\
113	0.00391536337802129\\
114	0.00391541061504923\\
115	0.00391545870162422\\
116	0.00391550765301681\\
117	0.00391555748477168\\
118	0.00391560821271267\\
119	0.00391565985294758\\
120	0.00391571242187345\\
121	0.00391576593618165\\
122	0.00391582041286306\\
123	0.00391587586921361\\
124	0.0039159323228396\\
125	0.00391598979166334\\
126	0.00391604829392865\\
127	0.00391610784820688\\
128	0.00391616847340245\\
129	0.00391623018875904\\
130	0.00391629301386557\\
131	0.00391635696866237\\
132	0.00391642207344744\\
133	0.00391648834888292\\
134	0.0039165558160015\\
135	0.00391662449621307\\
136	0.0039166944113115\\
137	0.00391676558348142\\
138	0.0039168380353053\\
139	0.00391691178977044\\
140	0.00391698687027623\\
141	0.00391706330064156\\
142	0.0039171411051122\\
143	0.00391722030836855\\
144	0.00391730093553329\\
145	0.0039173830121792\\
146	0.00391746656433737\\
147	0.00391755161850521\\
148	0.00391763820165484\\
149	0.00391772634124151\\
150	0.00391781606521217\\
151	0.00391790740201436\\
152	0.00391800038060491\\
153	0.00391809503045919\\
154	0.00391819138158023\\
155	0.00391828946450815\\
156	0.00391838931032967\\
157	0.00391849095068785\\
158	0.00391859441779201\\
159	0.00391869974442765\\
160	0.00391880696396692\\
161	0.0039189161103788\\
162	0.0039190272182398\\
163	0.0039191403227448\\
164	0.00391925545971783\\
165	0.00391937266562344\\
166	0.0039194919775779\\
167	0.0039196134333608\\
168	0.00391973707142682\\
169	0.00391986293091758\\
170	0.003919991051674\\
171	0.00392012147424852\\
172	0.00392025423991759\\
173	0.0039203893906947\\
174	0.00392052696934325\\
175	0.00392066701938984\\
176	0.00392080958513773\\
177	0.00392095471168059\\
178	0.00392110244491633\\
179	0.00392125283156142\\
180	0.00392140591916523\\
181	0.00392156175612471\\
182	0.00392172039169928\\
183	0.00392188187602615\\
184	0.00392204626013556\\
185	0.00392221359596661\\
186	0.00392238393638313\\
187	0.00392255733519008\\
188	0.00392273384714987\\
189	0.00392291352799926\\
190	0.00392309643446646\\
191	0.0039232826242884\\
192	0.00392347215622846\\
193	0.00392366509009446\\
194	0.00392386148675686\\
195	0.00392406140816729\\
196	0.00392426491737753\\
197	0.00392447207855869\\
198	0.00392468295702064\\
199	0.00392489761923196\\
200	0.00392511613284003\\
201	0.00392533856669158\\
202	0.00392556499085358\\
203	0.00392579547663438\\
204	0.00392603009660527\\
205	0.00392626892462241\\
206	0.00392651203584904\\
207	0.00392675950677819\\
208	0.00392701141525566\\
209	0.00392726784050333\\
210	0.00392752886314305\\
211	0.00392779456522076\\
212	0.00392806503023093\\
213	0.00392834034314166\\
214	0.00392862059041988\\
215	0.00392890586005726\\
216	0.00392919624159628\\
217	0.00392949182615693\\
218	0.00392979270646361\\
219	0.00393009897687269\\
220	0.00393041073340037\\
221	0.00393072807375105\\
222	0.00393105109734605\\
223	0.00393137990535299\\
224	0.00393171460071542\\
225	0.00393205528818294\\
226	0.00393240207434204\\
227	0.00393275506764708\\
228	0.00393311437845195\\
229	0.00393348011904223\\
230	0.00393385240366776\\
231	0.00393423134857578\\
232	0.00393461707204458\\
233	0.00393500969441757\\
234	0.00393540933813807\\
235	0.00393581612778442\\
236	0.0039362301901057\\
237	0.0039366516540581\\
238	0.00393708065084169\\
239	0.00393751731393771\\
240	0.00393796177914665\\
241	0.0039384141846267\\
242	0.00393887467093278\\
243	0.00393934338105621\\
244	0.00393982046046497\\
245	0.00394030605714449\\
246	0.00394080032163905\\
247	0.00394130340709382\\
248	0.00394181546929747\\
249	0.00394233666672541\\
250	0.00394286716058364\\
251	0.00394340711485321\\
252	0.00394395669633537\\
253	0.00394451607469714\\
254	0.00394508542251787\\
255	0.00394566491533617\\
256	0.00394625473169753\\
257	0.00394685505320269\\
258	0.00394746606455656\\
259	0.00394808795361791\\
260	0.00394872091144956\\
261	0.00394936513236943\\
262	0.00395002081400208\\
263	0.00395068815733114\\
264	0.00395136736675211\\
265	0.0039520586501262\\
266	0.00395276221883456\\
267	0.00395347828783326\\
268	0.00395420707570917\\
269	0.0039549488047362\\
270	0.00395570370093242\\
271	0.00395647199411784\\
272	0.00395725391797283\\
273	0.0039580497100973\\
274	0.00395885961207044\\
275	0.00395968386951128\\
276	0.00396052273213981\\
277	0.00396137645383877\\
278	0.00396224529271628\\
279	0.00396312951116883\\
280	0.00396402937594523\\
281	0.00396494515821093\\
282	0.00396587713361319\\
283	0.00396682558234666\\
284	0.00396779078921978\\
285	0.00396877304372167\\
286	0.00396977264008943\\
287	0.00397078987737639\\
288	0.00397182505952052\\
289	0.00397287849541347\\
290	0.00397395049897024\\
291	0.00397504138919915\\
292	0.00397615149027226\\
293	0.00397728113159628\\
294	0.0039784306478837\\
295	0.00397960037922457\\
296	0.00398079067115801\\
297	0.00398200187474465\\
298	0.00398323434663867\\
299	0.00398448844916045\\
300	0.00398576455036895\\
301	0.00398706302413444\\
302	0.003988384250211\\
303	0.00398972861430908\\
304	0.00399109650816784\\
305	0.00399248832962729\\
306	0.0039939044827002\\
307	0.00399534537764357\\
308	0.00399681143102979\\
309	0.00399830306581716\\
310	0.00399982071141991\\
311	0.0040013648037776\\
312	0.00400293578542367\\
313	0.0040045341055533\\
314	0.00400616022009033\\
315	0.00400781459175329\\
316	0.00400949769012048\\
317	0.00401120999169397\\
318	0.00401295197996263\\
319	0.00401472414546422\\
320	0.00401652698584636\\
321	0.00401836100592681\\
322	0.00402022671775282\\
323	0.00402212464066014\\
324	0.00402405530133158\\
325	0.00402601923385591\\
326	0.00402801697978755\\
327	0.00403004908820754\\
328	0.00403211611578721\\
329	0.00403421862685557\\
330	0.00403635719347254\\
331	0.00403853239550991\\
332	0.00404074482074375\\
333	0.00404299506496178\\
334	0.00404528373209185\\
335	0.00404761143435832\\
336	0.00404997879247634\\
337	0.00405238643589678\\
338	0.00405483500311943\\
339	0.00405732514209878\\
340	0.00405985751078104\\
341	0.00406243277784526\\
342	0.00406505162382405\\
343	0.00406771474309292\\
344	0.00407042284814956\\
345	0.0040731766793176\\
346	0.00407597701430165\\
347	0.00407882465138177\\
348	0.00408172041067304\\
349	0.00408466513531158\\
350	0.00408765969244745\\
351	0.00409070497383421\\
352	0.00409380189560034\\
353	0.00409695139626494\\
354	0.00410015443066837\\
355	0.00410341195380954\\
356	0.00410672487967\\
357	0.00411009400125772\\
358	0.00411352010226292\\
359	0.00411700396460171\\
360	0.00412054636708436\\
361	0.00412414808393454\\
362	0.004127809883143\\
363	0.00413153252463749\\
364	0.0041353167582489\\
365	0.00413916332145065\\
366	0.0041430729368464\\
367	0.00414704630937751\\
368	0.0041510841232191\\
369	0.00415518703832901\\
370	0.00415935568661078\\
371	0.00416359066764624\\
372	0.00416789254394891\\
373	0.00417226183568325\\
374	0.00417669901478839\\
375	0.00418120449843865\\
376	0.00418577864176438\\
377	0.00419042172974946\\
378	0.00419513396821128\\
379	0.00419991547376057\\
380	0.00420476626262653\\
381	0.00420968623822302\\
382	0.00421467517731951\\
383	0.00421973271466958\\
384	0.00422485832593976\\
385	0.0042300513087723\\
386	0.0042353107618099\\
387	0.00424063556150867\\
388	0.00424602433657202\\
389	0.00425147543985365\\
390	0.00425698691761027\\
391	0.00426255647603806\\
392	0.00426818144511206\\
393	0.00427385873987608\\
394	0.00427958481952026\\
395	0.00428535564485324\\
396	0.00429116663515453\\
397	0.00429701262590292\\
398	0.00430288782952713\\
399	0.00430878580264344\\
400	0.00431469942497536\\
401	0.00432062089684156\\
402	0.00432654176479418\\
403	0.00433245298898998\\
404	0.00433834507115593\\
405	0.00434420826923577\\
406	0.00435003293470858\\
407	0.00435581002220494\\
408	0.00436153184002012\\
409	0.00436719313706548\\
410	0.00437279266132182\\
411	0.00437833538472884\\
412	0.00438383567545592\\
413	0.00438932177878079\\
414	0.00439484173526621\\
415	0.00440043693996867\\
416	0.00440611959258895\\
417	0.00441189048639354\\
418	0.00441775037617439\\
419	0.00442369997286494\\
420	0.00442973993741127\\
421	0.00443587087335271\\
422	0.00444209331871319\\
423	0.00444840774111909\\
424	0.00445481453771689\\
425	0.0044613140414695\\
426	0.00446790651697064\\
427	0.0044745921563337\\
428	0.00448137107521975\\
429	0.0044882433090829\\
430	0.00449520880974954\\
431	0.00450226744247634\\
432	0.00450941898366786\\
433	0.00451666311947818\\
434	0.00452399944557325\\
435	0.00453142746838399\\
436	0.00453894660824531\\
437	0.00454655620490727\\
438	0.00455425552600115\\
439	0.00456204377916066\\
440	0.00456992012862854\\
441	0.00457788371733072\\
442	0.00458593369556622\\
443	0.00459406925764097\\
444	0.00460228968796137\\
445	0.00461059441828207\\
446	0.00461898309795326\\
447	0.00462745567909243\\
448	0.00463601251855382\\
449	0.00464465449905457\\
450	0.00465338317019809\\
451	0.00466220090898406\\
452	0.00467111109790705\\
453	0.00468011831560007\\
454	0.00468922852994733\\
455	0.00469844927614577\\
456	0.00470778979203129\\
457	0.0047172610686496\\
458	0.00472687575038563\\
459	0.00473664778038361\\
460	0.00474659161162588\\
461	0.00475672065104738\\
462	0.00476704443458528\\
463	0.00477756856208024\\
464	0.00478829864180216\\
465	0.00479924063984959\\
466	0.00481040087052541\\
467	0.00482178602364282\\
468	0.00483340319515331\\
469	0.00484525992006689\\
470	0.00485736420771812\\
471	0.004869724579392\\
472	0.00488235010823822\\
473	0.00489525046131628\\
474	0.00490843594349121\\
475	0.00492191754271023\\
476	0.00493570697592564\\
477	0.00494981673455072\\
478	0.00496426012906006\\
479	0.00497905133959719\\
480	0.00499420641614639\\
481	0.00500974301389922\\
482	0.00502567987587271\\
483	0.00504203681804972\\
484	0.00505883469393796\\
485	0.00507609533954526\\
486	0.00509384150344459\\
487	0.0051120967723867\\
488	0.00513088551504203\\
489	0.00515023291574502\\
490	0.00517016521560303\\
491	0.00519070995746875\\
492	0.00521189574655456\\
493	0.00523370568230926\\
494	0.00525615797345135\\
495	0.00527928085048286\\
496	0.00530310526724121\\
497	0.00532766854668698\\
498	0.00535303735342734\\
499	0.00537925241866236\\
500	0.0054063556515977\\
501	0.00543439127309023\\
502	0.00546340557950195\\
503	0.00549344603574776\\
504	0.00552456889504518\\
505	0.00555688280214329\\
506	0.00559044812484436\\
507	0.00562532780659876\\
508	0.00566158726887005\\
509	0.00569929423683336\\
510	0.00573851846062438\\
511	0.00577933131156622\\
512	0.00582180523855337\\
513	0.0058660130791357\\
514	0.00591202733005102\\
515	0.00595991936941512\\
516	0.00600975518857396\\
517	0.00606159220849871\\
518	0.00611547879735186\\
519	0.00617145111727593\\
520	0.00622952827222556\\
521	0.00628970070576879\\
522	0.00635192430572301\\
523	0.00641612600430055\\
524	0.00648218785696528\\
525	0.00654993917002101\\
526	0.0066191621183003\\
527	0.00668963241587316\\
528	0.00676094353650896\\
529	0.00683211407028415\\
530	0.00690211112710542\\
531	0.00696671445942419\\
532	0.00702546965301922\\
533	0.00707812424877317\\
534	0.00712481587446659\\
535	0.00716849059922027\\
536	0.00721077696089307\\
537	0.00725210040069124\\
538	0.00729298041879735\\
539	0.00733394341899069\\
540	0.00737527202531301\\
541	0.00741711243910963\\
542	0.00745958050016931\\
543	0.00750275904121561\\
544	0.00754670141587454\\
545	0.00759144767135124\\
546	0.00763703054234583\\
547	0.00768347622976199\\
548	0.00773080823398957\\
549	0.00777909452938988\\
550	0.00782841534682607\\
551	0.00787862213957222\\
552	0.00792795218358609\\
553	0.00797616861463886\\
554	0.00802315876012582\\
555	0.00807047849253736\\
556	0.00811816536939713\\
557	0.0081662917589018\\
558	0.00821491588109375\\
559	0.00826402041714991\\
560	0.00831357679383578\\
561	0.00836354672564337\\
562	0.00841388288403202\\
563	0.00846455795573843\\
564	0.00851554903587025\\
565	0.00856510622084549\\
566	0.00861296704327435\\
567	0.00866079323060213\\
568	0.0087086623118578\\
569	0.00875662026486645\\
570	0.00880463250812337\\
571	0.0088526615516325\\
572	0.00890067654903162\\
573	0.00894813872224893\\
574	0.00899535238494104\\
575	0.00904255767963824\\
576	0.00908973840022921\\
577	0.00913684942824384\\
578	0.00918384194182773\\
579	0.00923066499370698\\
580	0.00927726597097645\\
581	0.00932359115751704\\
582	0.00936958649189013\\
583	0.00941519857373444\\
584	0.00946037597937255\\
585	0.00950507095774694\\
586	0.00954924158549703\\
587	0.00959285445497603\\
588	0.00963588792409171\\
589	0.00967833579817709\\
590	0.00972021084271539\\
591	0.00976138933978722\\
592	0.00980171928775089\\
593	0.00984102135019415\\
594	0.00987905595166784\\
595	0.00991543381058343\\
596	0.00994937493726701\\
597	0.00997906286423442\\
598	0.0099999191923403\\
599	0\\
600	0\\
};
\addplot [color=red,solid,forget plot]
  table[row sep=crcr]{%
1	0.00417141729594836\\
2	0.0041714212375612\\
3	0.00417142525044982\\
4	0.00417142933590415\\
5	0.00417143349523751\\
6	0.00417143772978697\\
7	0.00417144204091389\\
8	0.00417144643000426\\
9	0.00417145089846921\\
10	0.00417145544774538\\
11	0.00417146007929553\\
12	0.00417146479460882\\
13	0.00417146959520149\\
14	0.00417147448261719\\
15	0.00417147945842758\\
16	0.00417148452423281\\
17	0.004171489681662\\
18	0.00417149493237377\\
19	0.00417150027805687\\
20	0.0041715057204306\\
21	0.00417151126124547\\
22	0.00417151690228366\\
23	0.00417152264535971\\
24	0.00417152849232102\\
25	0.00417153444504848\\
26	0.00417154050545709\\
27	0.00417154667549652\\
28	0.0041715529571518\\
29	0.00417155935244397\\
30	0.00417156586343074\\
31	0.00417157249220702\\
32	0.00417157924090577\\
33	0.00417158611169864\\
34	0.00417159310679659\\
35	0.0041716002284507\\
36	0.0041716074789529\\
37	0.00417161486063662\\
38	0.00417162237587761\\
39	0.00417163002709472\\
40	0.00417163781675066\\
41	0.00417164574735278\\
42	0.00417165382145392\\
43	0.00417166204165321\\
44	0.00417167041059691\\
45	0.0041716789309793\\
46	0.00417168760554341\\
47	0.00417169643708219\\
48	0.00417170542843912\\
49	0.00417171458250933\\
50	0.00417172390224041\\
51	0.00417173339063343\\
52	0.00417174305074389\\
53	0.00417175288568275\\
54	0.00417176289861736\\
55	0.00417177309277255\\
56	0.00417178347143162\\
57	0.00417179403793749\\
58	0.00417180479569368\\
59	0.00417181574816551\\
60	0.00417182689888108\\
61	0.0041718382514326\\
62	0.00417184980947735\\
63	0.00417186157673903\\
64	0.00417187355700894\\
65	0.00417188575414711\\
66	0.00417189817208365\\
67	0.00417191081481996\\
68	0.00417192368643009\\
69	0.00417193679106193\\
70	0.00417195013293873\\
71	0.00417196371636035\\
72	0.0041719775457047\\
73	0.00417199162542913\\
74	0.00417200596007185\\
75	0.00417202055425349\\
76	0.00417203541267852\\
77	0.0041720505401368\\
78	0.00417206594150512\\
79	0.00417208162174879\\
80	0.00417209758592323\\
81	0.00417211383917564\\
82	0.00417213038674667\\
83	0.00417214723397208\\
84	0.00417216438628448\\
85	0.00417218184921508\\
86	0.00417219962839556\\
87	0.00417221772955976\\
88	0.00417223615854567\\
89	0.00417225492129723\\
90	0.00417227402386632\\
91	0.00417229347241466\\
92	0.00417231327321582\\
93	0.00417233343265732\\
94	0.00417235395724267\\
95	0.00417237485359332\\
96	0.00417239612845109\\
97	0.00417241778868015\\
98	0.00417243984126929\\
99	0.00417246229333417\\
100	0.00417248515211969\\
101	0.00417250842500232\\
102	0.00417253211949238\\
103	0.00417255624323666\\
104	0.00417258080402071\\
105	0.00417260580977149\\
106	0.0041726312685599\\
107	0.00417265718860336\\
108	0.00417268357826855\\
109	0.00417271044607403\\
110	0.00417273780069307\\
111	0.00417276565095637\\
112	0.00417279400585506\\
113	0.00417282287454349\\
114	0.00417285226634222\\
115	0.00417288219074112\\
116	0.00417291265740226\\
117	0.00417294367616329\\
118	0.00417297525704035\\
119	0.00417300741023158\\
120	0.00417304014612021\\
121	0.004173073475278\\
122	0.00417310740846868\\
123	0.00417314195665143\\
124	0.00417317713098439\\
125	0.00417321294282827\\
126	0.00417324940375013\\
127	0.00417328652552689\\
128	0.0041733243201494\\
129	0.00417336279982615\\
130	0.0041734019769873\\
131	0.00417344186428863\\
132	0.00417348247461566\\
133	0.00417352382108783\\
134	0.00417356591706271\\
135	0.00417360877614035\\
136	0.00417365241216765\\
137	0.00417369683924282\\
138	0.00417374207171994\\
139	0.00417378812421355\\
140	0.00417383501160351\\
141	0.00417388274903964\\
142	0.00417393135194666\\
143	0.00417398083602919\\
144	0.00417403121727675\\
145	0.00417408251196902\\
146	0.00417413473668098\\
147	0.00417418790828828\\
148	0.00417424204397271\\
149	0.00417429716122767\\
150	0.00417435327786387\\
151	0.00417441041201495\\
152	0.00417446858214343\\
153	0.0041745278070466\\
154	0.00417458810586256\\
155	0.00417464949807632\\
156	0.00417471200352611\\
157	0.00417477564240979\\
158	0.00417484043529119\\
159	0.00417490640310686\\
160	0.00417497356717269\\
161	0.00417504194919076\\
162	0.00417511157125627\\
163	0.00417518245586462\\
164	0.00417525462591869\\
165	0.00417532810473602\\
166	0.00417540291605636\\
167	0.00417547908404922\\
168	0.0041755566333216\\
169	0.00417563558892583\\
170	0.00417571597636753\\
171	0.00417579782161376\\
172	0.00417588115110138\\
173	0.00417596599174526\\
174	0.00417605237094702\\
175	0.00417614031660364\\
176	0.00417622985711639\\
177	0.00417632102139971\\
178	0.00417641383889057\\
179	0.00417650833955761\\
180	0.00417660455391075\\
181	0.00417670251301075\\
182	0.00417680224847913\\
183	0.004176903792508\\
184	0.00417700717787039\\
185	0.00417711243793043\\
186	0.00417721960665394\\
187	0.00417732871861905\\
188	0.00417743980902709\\
189	0.00417755291371368\\
190	0.00417766806915986\\
191	0.0041777853125036\\
192	0.00417790468155139\\
193	0.00417802621479001\\
194	0.00417814995139857\\
195	0.00417827593126074\\
196	0.00417840419497708\\
197	0.00417853478387773\\
198	0.00417866774003521\\
199	0.00417880310627744\\
200	0.00417894092620098\\
201	0.00417908124418459\\
202	0.00417922410540279\\
203	0.00417936955583983\\
204	0.00417951764230381\\
205	0.00417966841244109\\
206	0.00417982191475086\\
207	0.00417997819859995\\
208	0.00418013731423793\\
209	0.00418029931281241\\
210	0.00418046424638457\\
211	0.00418063216794502\\
212	0.00418080313142981\\
213	0.00418097719173671\\
214	0.00418115440474184\\
215	0.00418133482731642\\
216	0.00418151851734385\\
217	0.00418170553373708\\
218	0.00418189593645622\\
219	0.00418208978652635\\
220	0.00418228714605576\\
221	0.0041824880782543\\
222	0.0041826926474521\\
223	0.00418290091911855\\
224	0.00418311295988156\\
225	0.00418332883754708\\
226	0.00418354862111898\\
227	0.00418377238081907\\
228	0.00418400018810767\\
229	0.00418423211570418\\
230	0.00418446823760813\\
231	0.00418470862912047\\
232	0.00418495336686518\\
233	0.0041852025288112\\
234	0.00418545619429455\\
235	0.00418571444404094\\
236	0.00418597736018854\\
237	0.00418624502631107\\
238	0.00418651752744131\\
239	0.00418679495009489\\
240	0.00418707738229421\\
241	0.0041873649135929\\
242	0.00418765763510055\\
243	0.00418795563950764\\
244	0.00418825902111096\\
245	0.00418856787583913\\
246	0.00418888230127873\\
247	0.00418920239670044\\
248	0.0041895282630857\\
249	0.0041898600031536\\
250	0.00419019772138816\\
251	0.00419054152406586\\
252	0.0041908915192835\\
253	0.00419124781698645\\
254	0.00419161052899709\\
255	0.00419197976904371\\
256	0.00419235565278956\\
257	0.00419273829786239\\
258	0.00419312782388419\\
259	0.00419352435250128\\
260	0.00419392800741462\\
261	0.00419433891441072\\
262	0.00419475720139237\\
263	0.00419518299841011\\
264	0.00419561643769389\\
265	0.00419605765368487\\
266	0.00419650678306768\\
267	0.00419696396480297\\
268	0.00419742934016015\\
269	0.00419790305275047\\
270	0.0041983852485605\\
271	0.00419887607598572\\
272	0.00419937568586451\\
273	0.00419988423151231\\
274	0.00420040186875631\\
275	0.00420092875596998\\
276	0.00420146505410838\\
277	0.00420201092674335\\
278	0.00420256654009912\\
279	0.00420313206308821\\
280	0.00420370766734756\\
281	0.00420429352727493\\
282	0.00420488982006546\\
283	0.00420549672574871\\
284	0.00420611442722573\\
285	0.00420674311030644\\
286	0.00420738296374742\\
287	0.00420803417928973\\
288	0.00420869695169706\\
289	0.00420937147879415\\
290	0.00421005796150549\\
291	0.00421075660389411\\
292	0.00421146761320083\\
293	0.00421219119988342\\
294	0.00421292757765646\\
295	0.00421367696353093\\
296	0.00421443957785442\\
297	0.00421521564435124\\
298	0.00421600539016304\\
299	0.00421680904588942\\
300	0.00421762684562882\\
301	0.00421845902701961\\
302	0.00421930583128124\\
303	0.00422016750325577\\
304	0.0042210442914492\\
305	0.00422193644807332\\
306	0.00422284422908731\\
307	0.00422376789423958\\
308	0.00422470770710954\\
309	0.00422566393514946\\
310	0.00422663684972616\\
311	0.00422762672616268\\
312	0.00422863384377971\\
313	0.00422965848593693\\
314	0.0042307009400738\\
315	0.00423176149775028\\
316	0.00423284045468683\\
317	0.00423393811080408\\
318	0.00423505477026159\\
319	0.00423619074149609\\
320	0.00423734633725874\\
321	0.0042385218746514\\
322	0.00423971767516177\\
323	0.0042409340646973\\
324	0.00424217137361771\\
325	0.00424342993676598\\
326	0.00424471009349763\\
327	0.00424601218770806\\
328	0.00424733656785803\\
329	0.00424868358699681\\
330	0.00425005360278272\\
331	0.0042514469775011\\
332	0.00425286407807892\\
333	0.00425430527609568\\
334	0.00425577094779001\\
335	0.00425726147406045\\
336	0.0042587772404593\\
337	0.00426031863717721\\
338	0.00426188605901504\\
339	0.0042634799053389\\
340	0.00426510058001185\\
341	0.00426674849129432\\
342	0.00426842405170062\\
343	0.00427012767778342\\
344	0.00427185978975912\\
345	0.00427362081076799\\
346	0.00427541116576958\\
347	0.00427723128085833\\
348	0.00427908158246891\\
349	0.00428096249646347\\
350	0.0042828744470972\\
351	0.00428481785586465\\
352	0.00428679314023817\\
353	0.00428880071232879\\
354	0.00429084097756436\\
355	0.00429291433371471\\
356	0.0042950211712544\\
357	0.00429716187620316\\
358	0.00429933682967659\\
359	0.00430154640713625\\
360	0.00430379097758998\\
361	0.00430607090273958\\
362	0.00430838653607358\\
363	0.00431073822190244\\
364	0.00431312629433371\\
365	0.00431555107618505\\
366	0.00431801287783277\\
367	0.0043205119959941\\
368	0.00432304871244172\\
369	0.00432562329264953\\
370	0.00432823598436941\\
371	0.00433088701613951\\
372	0.00433357659572557\\
373	0.00433630490849874\\
374	0.00433907211575428\\
375	0.00434187835297895\\
376	0.0043447237280762\\
377	0.00434760831956328\\
378	0.0043505321747576\\
379	0.00435349530797551\\
380	0.00435649769877283\\
381	0.00435953929026478\\
382	0.00436261998757205\\
383	0.00436573965645217\\
384	0.00436889812218877\\
385	0.00437209516882886\\
386	0.00437533053887853\\
387	0.00437860393359149\\
388	0.00438191501401419\\
389	0.00438526340298479\\
390	0.0043886486883227\\
391	0.00439207042749119\\
392	0.00439552815406753\\
393	0.0043990213864128\\
394	0.00440254963899731\\
395	0.00440611243690426\\
396	0.00440970933410266\\
397	0.00441333993614392\\
398	0.00441700392799374\\
399	0.00442070110772882\\
400	0.00442443142677545\\
401	0.00442819503726806\\
402	0.00443199234687599\\
403	0.00443582408101143\\
404	0.00443969135159781\\
405	0.00444359573040631\\
406	0.00444753932316088\\
407	0.0044515248379043\\
408	0.00445555563713975\\
409	0.00445963575746238\\
410	0.0044637698717059\\
411	0.00446796315433237\\
412	0.00447222098358322\\
413	0.00447654835910707\\
414	0.00448094884530292\\
415	0.00448542420369106\\
416	0.00448997560015105\\
417	0.00449460422115946\\
418	0.00449931127504153\\
419	0.00450409799335963\\
420	0.00450896563245742\\
421	0.00451391547521329\\
422	0.00451894883305504\\
423	0.00452406704818085\\
424	0.00452927149581646\\
425	0.00453456358633038\\
426	0.00453994476757048\\
427	0.00454541652748043\\
428	0.00455098039742705\\
429	0.0045566379564885\\
430	0.00456239083633436\\
431	0.00456824072677151\\
432	0.00457418938203479\\
433	0.00458023862790652\\
434	0.00458639036975255\\
435	0.00459264660156371\\
436	0.00459900941609112\\
437	0.00460548101615734\\
438	0.00461206372721509\\
439	0.00461876001120466\\
440	0.00462557248173187\\
441	0.00463250392054477\\
442	0.00463955729522559\\
443	0.00464673577793208\\
444	0.00465404276491351\\
445	0.00466148189639597\\
446	0.0046690570762898\\
447	0.00467677249104362\\
448	0.00468463262666304\\
449	0.00469264226005057\\
450	0.00470080645466231\\
451	0.00470913056139528\\
452	0.00471762021278094\\
453	0.00472628130882881\\
454	0.00473511999293577\\
455	0.00474414261666887\\
456	0.00475335569311759\\
457	0.00476276583994915\\
458	0.00477237971546353\\
459	0.00478220395394506\\
460	0.00479224511277774\\
461	0.00480250966573844\\
462	0.0048130041410784\\
463	0.00482373531637173\\
464	0.00483471025564672\\
465	0.00484593634938176\\
466	0.00485742262810758\\
467	0.00486917868065496\\
468	0.0048812146512215\\
469	0.0048935412725203\\
470	0.00490616989892637\\
471	0.00491911253884136\\
472	0.00493238188599012\\
473	0.00494599134833053\\
474	0.00495995507289613\\
475	0.00497428796445237\\
476	0.00498900569361151\\
477	0.00500412468545673\\
478	0.00501966204991506\\
479	0.00503563534260202\\
480	0.00505203203850175\\
481	0.00506885035320858\\
482	0.00508610647732088\\
483	0.00510381737352443\\
484	0.00512200081946964\\
485	0.00514067545717838\\
486	0.00515986085050313\\
487	0.0051795775543532\\
488	0.0051998472524993\\
489	0.00522069291948455\\
490	0.00524213893576288\\
491	0.00526421036697117\\
492	0.0052869337710113\\
493	0.00531038708338049\\
494	0.00533463065977775\\
495	0.00535972170666477\\
496	0.00538570357433848\\
497	0.00541262217930969\\
498	0.00544052593988315\\
499	0.00546946488375426\\
500	0.0054994915755224\\
501	0.0055306611806234\\
502	0.00556303151660603\\
503	0.00559666327700001\\
504	0.00563162031738346\\
505	0.00566796721508761\\
506	0.00570576840772046\\
507	0.00574508942737241\\
508	0.00578599628828108\\
509	0.00582855467622107\\
510	0.00587282886678455\\
511	0.00591888013540978\\
512	0.00596676496370957\\
513	0.0060165327491921\\
514	0.00606822295732727\\
515	0.00612186193541779\\
516	0.00617745923946957\\
517	0.00623501086570649\\
518	0.0062944782534695\\
519	0.00635577778146541\\
520	0.0064187788043303\\
521	0.00648329513000825\\
522	0.00654909551745368\\
523	0.00661556809570386\\
524	0.00668225778840485\\
525	0.00674845423256209\\
526	0.00681265527495289\\
527	0.00687114274886836\\
528	0.00692366259412663\\
529	0.00697041755680642\\
530	0.00701168306378662\\
531	0.00705128720921838\\
532	0.00708973073153022\\
533	0.00712748667855347\\
534	0.00716510576662421\\
535	0.00720298425452938\\
536	0.00724131062521522\\
537	0.00728020567755491\\
538	0.0073197621304377\\
539	0.00736003760201659\\
540	0.00740107026007002\\
541	0.00744289189206673\\
542	0.00748552922793119\\
543	0.00752900542539635\\
544	0.00757334165858363\\
545	0.00761855794325576\\
546	0.0076647133488703\\
547	0.0077118861491722\\
548	0.00776011644032672\\
549	0.00780747688911366\\
550	0.00785373177371214\\
551	0.00789882269053892\\
552	0.00794427584846579\\
553	0.00799013221423805\\
554	0.00803647165748248\\
555	0.00808335531207435\\
556	0.00813077466079379\\
557	0.0081787090079221\\
558	0.00822712878473145\\
559	0.00827599893382302\\
560	0.00832527752852308\\
561	0.00837493010777871\\
562	0.00842494111645729\\
563	0.00847397024240702\\
564	0.00852129441202751\\
565	0.00856849529392683\\
566	0.00861580505555088\\
567	0.00866327323530527\\
568	0.00871086822438768\\
569	0.00875855220172137\\
570	0.0088062920664252\\
571	0.00885392879390287\\
572	0.00890103813867015\\
573	0.00894817183198183\\
574	0.00899535443329664\\
575	0.00904255798659048\\
576	0.00908973853069441\\
577	0.00913684949557439\\
578	0.0091838419761039\\
579	0.009230665010298\\
580	0.00927726597847558\\
581	0.00932359116063004\\
582	0.00936958649305349\\
583	0.00941519857411533\\
584	0.00946037597947751\\
585	0.00950507095776978\\
586	0.00954924158550052\\
587	0.00959285445497631\\
588	0.00963588792409171\\
589	0.00967833579817709\\
590	0.00972021084271539\\
591	0.00976138933978722\\
592	0.00980171928775089\\
593	0.00984102135019415\\
594	0.00987905595166784\\
595	0.00991543381058343\\
596	0.00994937493726701\\
597	0.00997906286423442\\
598	0.0099999191923403\\
599	0\\
600	0\\
};
\addplot [color=mycolor20,solid,forget plot]
  table[row sep=crcr]{%
1	0.00426663331528449\\
2	0.00426663605446198\\
3	0.00426663884350637\\
4	0.00426664168332735\\
5	0.00426664457485123\\
6	0.0042666475190213\\
7	0.00426665051679803\\
8	0.00426665356915952\\
9	0.00426665667710172\\
10	0.00426665984163881\\
11	0.00426666306380356\\
12	0.00426666634464759\\
13	0.00426666968524176\\
14	0.00426667308667659\\
15	0.00426667655006247\\
16	0.00426668007653015\\
17	0.00426668366723108\\
18	0.0042666873233378\\
19	0.00426669104604424\\
20	0.00426669483656624\\
21	0.00426669869614191\\
22	0.004266702626032\\
23	0.00426670662752032\\
24	0.0042667107019142\\
25	0.00426671485054495\\
26	0.00426671907476817\\
27	0.00426672337596439\\
28	0.00426672775553932\\
29	0.00426673221492451\\
30	0.00426673675557764\\
31	0.00426674137898315\\
32	0.00426674608665263\\
33	0.00426675088012538\\
34	0.00426675576096894\\
35	0.0042667607307795\\
36	0.00426676579118251\\
37	0.00426677094383328\\
38	0.00426677619041735\\
39	0.00426678153265121\\
40	0.00426678697228281\\
41	0.00426679251109208\\
42	0.00426679815089161\\
43	0.00426680389352723\\
44	0.0042668097408786\\
45	0.00426681569485981\\
46	0.00426682175742013\\
47	0.00426682793054448\\
48	0.00426683421625422\\
49	0.00426684061660772\\
50	0.00426684713370119\\
51	0.00426685376966924\\
52	0.00426686052668561\\
53	0.00426686740696394\\
54	0.00426687441275847\\
55	0.0042668815463648\\
56	0.00426688881012064\\
57	0.00426689620640662\\
58	0.004266903737647\\
59	0.00426691140631057\\
60	0.0042669192149114\\
61	0.00426692716600971\\
62	0.00426693526221271\\
63	0.00426694350617553\\
64	0.00426695190060191\\
65	0.00426696044824531\\
66	0.00426696915190976\\
67	0.00426697801445072\\
68	0.00426698703877609\\
69	0.00426699622784719\\
70	0.00426700558467973\\
71	0.00426701511234478\\
72	0.00426702481396984\\
73	0.00426703469273984\\
74	0.00426704475189825\\
75	0.00426705499474805\\
76	0.00426706542465298\\
77	0.00426707604503858\\
78	0.0042670868593933\\
79	0.0042670978712697\\
80	0.00426710908428565\\
81	0.00426712050212553\\
82	0.00426713212854137\\
83	0.00426714396735423\\
84	0.00426715602245535\\
85	0.00426716829780761\\
86	0.00426718079744667\\
87	0.00426719352548243\\
88	0.00426720648610035\\
89	0.00426721968356286\\
90	0.00426723312221078\\
91	0.00426724680646477\\
92	0.00426726074082681\\
93	0.00426727492988168\\
94	0.0042672893782985\\
95	0.00426730409083235\\
96	0.0042673190723257\\
97	0.00426733432771018\\
98	0.0042673498620082\\
99	0.00426736568033454\\
100	0.00426738178789813\\
101	0.00426739819000379\\
102	0.00426741489205398\\
103	0.00426743189955058\\
104	0.0042674492180968\\
105	0.00426746685339901\\
106	0.00426748481126861\\
107	0.00426750309762402\\
108	0.00426752171849266\\
109	0.00426754068001293\\
110	0.00426755998843633\\
111	0.00426757965012944\\
112	0.00426759967157611\\
113	0.00426762005937972\\
114	0.00426764082026519\\
115	0.00426766196108137\\
116	0.00426768348880335\\
117	0.00426770541053467\\
118	0.00426772773350981\\
119	0.00426775046509656\\
120	0.00426777361279847\\
121	0.0042677971842574\\
122	0.00426782118725611\\
123	0.00426784562972072\\
124	0.00426787051972356\\
125	0.00426789586548568\\
126	0.00426792167537974\\
127	0.00426794795793283\\
128	0.00426797472182915\\
129	0.00426800197591313\\
130	0.00426802972919219\\
131	0.00426805799083993\\
132	0.00426808677019914\\
133	0.00426811607678491\\
134	0.00426814592028779\\
135	0.00426817631057712\\
136	0.00426820725770426\\
137	0.00426823877190602\\
138	0.004268270863608\\
139	0.00426830354342821\\
140	0.0042683368221805\\
141	0.00426837071087827\\
142	0.00426840522073809\\
143	0.00426844036318351\\
144	0.00426847614984887\\
145	0.00426851259258323\\
146	0.00426854970345422\\
147	0.00426858749475229\\
148	0.00426862597899463\\
149	0.00426866516892941\\
150	0.00426870507754012\\
151	0.00426874571804986\\
152	0.00426878710392581\\
153	0.00426882924888365\\
154	0.00426887216689225\\
155	0.00426891587217831\\
156	0.00426896037923111\\
157	0.00426900570280735\\
158	0.00426905185793613\\
159	0.00426909885992393\\
160	0.00426914672435976\\
161	0.00426919546712036\\
162	0.00426924510437553\\
163	0.00426929565259355\\
164	0.00426934712854658\\
165	0.0042693995493164\\
166	0.00426945293230002\\
167	0.00426950729521554\\
168	0.00426956265610805\\
169	0.00426961903335571\\
170	0.00426967644567576\\
171	0.00426973491213084\\
172	0.00426979445213538\\
173	0.00426985508546202\\
174	0.00426991683224824\\
175	0.00426997971300302\\
176	0.00427004374861366\\
177	0.00427010896035288\\
178	0.00427017536988565\\
179	0.00427024299927665\\
180	0.00427031187099741\\
181	0.00427038200793391\\
182	0.00427045343339406\\
183	0.00427052617111558\\
184	0.00427060024527374\\
185	0.00427067568048948\\
186	0.00427075250183746\\
187	0.00427083073485445\\
188	0.0042709104055478\\
189	0.00427099154040393\\
190	0.00427107416639718\\
191	0.00427115831099867\\
192	0.00427124400218538\\
193	0.00427133126844935\\
194	0.00427142013880714\\
195	0.00427151064280929\\
196	0.00427160281055008\\
197	0.0042716966726774\\
198	0.00427179226040283\\
199	0.00427188960551184\\
200	0.00427198874037426\\
201	0.00427208969795479\\
202	0.00427219251182382\\
203	0.00427229721616838\\
204	0.00427240384580332\\
205	0.00427251243618258\\
206	0.00427262302341075\\
207	0.0042727356442548\\
208	0.004272850336156\\
209	0.00427296713724205\\
210	0.00427308608633941\\
211	0.00427320722298579\\
212	0.00427333058744295\\
213	0.00427345622070965\\
214	0.00427358416453478\\
215	0.00427371446143077\\
216	0.00427384715468722\\
217	0.00427398228838469\\
218	0.0042741199074088\\
219	0.00427426005746445\\
220	0.0042744027850904\\
221	0.00427454813767394\\
222	0.00427469616346598\\
223	0.00427484691159612\\
224	0.00427500043208823\\
225	0.00427515677587616\\
226	0.00427531599481954\\
227	0.00427547814172013\\
228	0.00427564327033817\\
229	0.00427581143540913\\
230	0.00427598269266059\\
231	0.00427615709882952\\
232	0.00427633471167965\\
233	0.00427651559001924\\
234	0.00427669979371907\\
235	0.00427688738373068\\
236	0.00427707842210479\\
237	0.00427727297201019\\
238	0.00427747109775273\\
239	0.00427767286479459\\
240	0.00427787833977384\\
241	0.00427808759052437\\
242	0.00427830068609593\\
243	0.00427851769677457\\
244	0.00427873869410317\\
245	0.00427896375090256\\
246	0.00427919294129249\\
247	0.00427942634071331\\
248	0.00427966402594756\\
249	0.00427990607514204\\
250	0.00428015256783011\\
251	0.00428040358495417\\
252	0.00428065920888856\\
253	0.00428091952346254\\
254	0.00428118461398376\\
255	0.00428145456726173\\
256	0.00428172947163182\\
257	0.00428200941697935\\
258	0.00428229449476391\\
259	0.00428258479804408\\
260	0.00428288042150233\\
261	0.00428318146147007\\
262	0.00428348801595315\\
263	0.00428380018465741\\
264	0.00428411806901457\\
265	0.00428444177220832\\
266	0.00428477139920069\\
267	0.00428510705675852\\
268	0.00428544885348034\\
269	0.00428579689982334\\
270	0.00428615130813052\\
271	0.00428651219265822\\
272	0.00428687966960365\\
273	0.00428725385713282\\
274	0.00428763487540848\\
275	0.00428802284661846\\
276	0.00428841789500398\\
277	0.00428882014688836\\
278	0.00428922973070582\\
279	0.00428964677703041\\
280	0.00429007141860525\\
281	0.00429050379037187\\
282	0.00429094402949989\\
283	0.00429139227541663\\
284	0.00429184866983709\\
285	0.00429231335679421\\
286	0.0042927864826692\\
287	0.00429326819622204\\
288	0.00429375864862242\\
289	0.00429425799348073\\
290	0.00429476638687946\\
291	0.00429528398740469\\
292	0.00429581095617808\\
293	0.00429634745688905\\
294	0.00429689365582736\\
295	0.00429744972191603\\
296	0.00429801582674476\\
297	0.00429859214460362\\
298	0.00429917885251751\\
299	0.00429977613028076\\
300	0.00430038416049273\\
301	0.00430100312859371\\
302	0.00430163322290172\\
303	0.00430227463464979\\
304	0.00430292755802435\\
305	0.00430359219020419\\
306	0.00430426873140046\\
307	0.0043049573848976\\
308	0.00430565835709522\\
309	0.00430637185755113\\
310	0.00430709809902548\\
311	0.00430783729752581\\
312	0.0043085896723536\\
313	0.00430935544615174\\
314	0.00431013484495313\\
315	0.00431092809823047\\
316	0.00431173543894718\\
317	0.00431255710360897\\
318	0.00431339333231663\\
319	0.00431424436881902\\
320	0.00431511046056672\\
321	0.00431599185876574\\
322	0.00431688881843094\\
323	0.00431780159843884\\
324	0.00431873046157911\\
325	0.00431967567460451\\
326	0.00432063750827823\\
327	0.0043216162374183\\
328	0.00432261214093769\\
329	0.00432362550187977\\
330	0.00432465660744775\\
331	0.00432570574902713\\
332	0.00432677322220024\\
333	0.00432785932675161\\
334	0.00432896436666306\\
335	0.00433008865009771\\
336	0.00433123248937182\\
337	0.00433239620091378\\
338	0.0043335801052101\\
339	0.00433478452673791\\
340	0.00433600979388494\\
341	0.00433725623885781\\
342	0.00433852419758035\\
343	0.00433981400958369\\
344	0.00434112601789247\\
345	0.00434246056892186\\
346	0.00434381801239764\\
347	0.00434519870128648\\
348	0.00434660299173398\\
349	0.00434803124301478\\
350	0.00434948381749918\\
351	0.00435096108064044\\
352	0.00435246340098642\\
353	0.00435399115021984\\
354	0.00435554470323253\\
355	0.00435712443823616\\
356	0.00435873073685524\\
357	0.00436036398412274\\
358	0.00436202456847492\\
359	0.00436371288175468\\
360	0.00436542931922488\\
361	0.00436717427959442\\
362	0.00436894816505918\\
363	0.00437075138136093\\
364	0.00437258433786724\\
365	0.00437444744767627\\
366	0.00437634112775016\\
367	0.00437826579908203\\
368	0.00438022188690141\\
369	0.00438220982092401\\
370	0.00438423003565191\\
371	0.00438628297073149\\
372	0.00438836907137671\\
373	0.00439048878886601\\
374	0.00439264258112255\\
375	0.00439483091338743\\
376	0.0043970542589972\\
377	0.00439931310027703\\
378	0.00440160792956237\\
379	0.00440393925036222\\
380	0.00440630757867798\\
381	0.00440871344449215\\
382	0.00441115739344192\\
383	0.00441363998869198\\
384	0.00441616181302129\\
385	0.00441872347113729\\
386	0.00442132559222974\\
387	0.00442396883277432\\
388	0.00442665387959276\\
389	0.00442938145317225\\
390	0.00443215231124075\\
391	0.00443496725258767\\
392	0.00443782712110927\\
393	0.0044407328100463\\
394	0.00444368526636549\\
395	0.00444668549521858\\
396	0.00444973456438823\\
397	0.00445283360860144\\
398	0.00445598383354885\\
399	0.00445918651955122\\
400	0.00446244302487293\\
401	0.0044657547879608\\
402	0.00446912332816642\\
403	0.00447255024442683\\
404	0.00447603721130479\\
405	0.00447958597174322\\
406	0.00448319832590475\\
407	0.004486876115589\\
408	0.0044906212040159\\
409	0.00449443545129221\\
410	0.00449832068667244\\
411	0.00450227867983526\\
412	0.00450631111553174\\
413	0.00451041958314296\\
414	0.00451460561420282\\
415	0.00451887075326674\\
416	0.00452321658618162\\
417	0.0045276447430362\\
418	0.0045321569014872\\
419	0.00453675479048509\\
420	0.00454144019441523\\
421	0.00454621495765489\\
422	0.00455108098952513\\
423	0.00455604026958924\\
424	0.00456109485322701\\
425	0.00456624687739753\\
426	0.00457149856649096\\
427	0.00457685223810688\\
428	0.00458231029759972\\
429	0.004587875226996\\
430	0.0045935495869901\\
431	0.00459933601879077\\
432	0.0046052372457871\\
433	0.00461125607500289\\
434	0.00461739539831013\\
435	0.00462365819337468\\
436	0.00463004752431581\\
437	0.00463656654207527\\
438	0.00464321848450911\\
439	0.0046500066762436\\
440	0.00465693452837336\\
441	0.00466400553812897\\
442	0.00467122328870246\\
443	0.00467859144948759\\
444	0.00468611377705478\\
445	0.00469379411721445\\
446	0.00470163640858135\\
447	0.00470964468879945\\
448	0.00471782311025046\\
449	0.00472617665871337\\
450	0.00473471089905645\\
451	0.00474343165936944\\
452	0.00475234504327954\\
453	0.00476145744292708\\
454	0.00477077555288365\\
455	0.00478030638543103\\
456	0.0047900572865475\\
457	0.00480003595306472\\
458	0.00481025045074162\\
459	0.00482070923265264\\
460	0.00483142115711013\\
461	0.00484239550330722\\
462	0.00485364197314648\\
463	0.00486517066339799\\
464	0.0048769919772247\\
465	0.00488911634640063\\
466	0.00490151423351223\\
467	0.00491419268996534\\
468	0.00492716026276567\\
469	0.0049404258593231\\
470	0.00495399877491506\\
471	0.00496788872794736\\
472	0.00498210588125886\\
473	0.00499666087416632\\
474	0.00501156486292523\\
475	0.005026829556054\\
476	0.00504246724598849\\
477	0.00505849076745665\\
478	0.00507491360227769\\
479	0.00509174979338502\\
480	0.00510904461151792\\
481	0.00512683513835379\\
482	0.00514514294413699\\
483	0.00516399086837915\\
484	0.00518340310039239\\
485	0.00520340532154351\\
486	0.00522402486784549\\
487	0.00524529086619148\\
488	0.00526723353178775\\
489	0.00528988555479908\\
490	0.00531328551978419\\
491	0.00533749217269366\\
492	0.00536255852004745\\
493	0.00538852951181075\\
494	0.00541545080092527\\
495	0.00544336957331949\\
496	0.00547233467708075\\
497	0.00550239710763532\\
498	0.00553360994472241\\
499	0.00556602827457043\\
500	0.00559970902539875\\
501	0.00563471072458662\\
502	0.00567109315732001\\
503	0.00570891688837819\\
504	0.00574824251891552\\
505	0.00578912967368312\\
506	0.00583163601149896\\
507	0.00587581585992625\\
508	0.0059217185016282\\
509	0.00596938601077104\\
510	0.00601885147197273\\
511	0.00607014279388628\\
512	0.00612327173424099\\
513	0.00617822896877258\\
514	0.00623497651937254\\
515	0.00629343208266863\\
516	0.00635345162141576\\
517	0.00641450018738264\\
518	0.00647630306501683\\
519	0.00653856759307269\\
520	0.00660083177351316\\
521	0.0066623734845899\\
522	0.00672148584669011\\
523	0.00677514287299271\\
524	0.00682314710642925\\
525	0.00686550063474744\\
526	0.00690303785427555\\
527	0.006939183141758\\
528	0.0069743533931855\\
529	0.00700904633276673\\
530	0.00704380336489818\\
531	0.00707891357147736\\
532	0.00711450782547904\\
533	0.00715069050118957\\
534	0.00718753256758812\\
535	0.00722507731585913\\
536	0.00726335660395916\\
537	0.00730239682932776\\
538	0.00734222033833203\\
539	0.00738284789789318\\
540	0.00742430017052845\\
541	0.00746659733277799\\
542	0.00750975936517875\\
543	0.00755383226995456\\
544	0.00759889240011881\\
545	0.00764503248014944\\
546	0.00769063100310164\\
547	0.00773515726105684\\
548	0.00777837421757474\\
549	0.00782197210231666\\
550	0.00786599669861675\\
551	0.00791053055057518\\
552	0.0079556338739196\\
553	0.00800129915646446\\
554	0.00804751442201892\\
555	0.00809426408080562\\
556	0.00814152538695643\\
557	0.00818926789268914\\
558	0.00823745553953724\\
559	0.00828604517572598\\
560	0.00833502635030947\\
561	0.0083836844121862\\
562	0.00843068273194334\\
563	0.00847720045237796\\
564	0.00852387037291765\\
565	0.0085707516393333\\
566	0.00861782349227564\\
567	0.00866505204116376\\
568	0.00871240258683382\\
569	0.00875984726743983\\
570	0.00880711445205011\\
571	0.00885403140777324\\
572	0.00890105211970034\\
573	0.00894817202601561\\
574	0.00899535447720905\\
575	0.00904255800719692\\
576	0.00908973854120521\\
577	0.00913684950076924\\
578	0.00918384197853101\\
579	0.00923066501135355\\
580	0.00927726597889613\\
581	0.00932359116078053\\
582	0.00936958649310055\\
583	0.00941519857412769\\
584	0.00946037597948007\\
585	0.00950507095777015\\
586	0.00954924158550054\\
587	0.00959285445497631\\
588	0.00963588792409171\\
589	0.00967833579817709\\
590	0.00972021084271539\\
591	0.00976138933978722\\
592	0.00980171928775089\\
593	0.00984102135019415\\
594	0.00987905595166784\\
595	0.00991543381058343\\
596	0.00994937493726701\\
597	0.00997906286423442\\
598	0.0099999191923403\\
599	0\\
600	0\\
};
\addplot [color=mycolor21,solid,forget plot]
  table[row sep=crcr]{%
1	0.00430148357374624\\
2	0.00430148576350371\\
3	0.00430148799339185\\
4	0.00430149026414835\\
5	0.00430149257652449\\
6	0.00430149493128541\\
7	0.00430149732921043\\
8	0.00430149977109314\\
9	0.00430150225774187\\
10	0.00430150478997984\\
11	0.00430150736864544\\
12	0.00430150999459258\\
13	0.00430151266869094\\
14	0.00430151539182618\\
15	0.00430151816490036\\
16	0.00430152098883223\\
17	0.00430152386455745\\
18	0.00430152679302899\\
19	0.00430152977521747\\
20	0.00430153281211134\\
21	0.00430153590471736\\
22	0.00430153905406089\\
23	0.00430154226118625\\
24	0.00430154552715704\\
25	0.0043015488530565\\
26	0.00430155223998792\\
27	0.00430155568907497\\
28	0.00430155920146212\\
29	0.00430156277831491\\
30	0.00430156642082051\\
31	0.00430157013018804\\
32	0.00430157390764895\\
33	0.0043015777544575\\
34	0.00430158167189106\\
35	0.00430158566125074\\
36	0.00430158972386165\\
37	0.0043015938610734\\
38	0.00430159807426065\\
39	0.00430160236482345\\
40	0.0043016067341877\\
41	0.00430161118380577\\
42	0.00430161571515693\\
43	0.00430162032974776\\
44	0.00430162502911277\\
45	0.00430162981481488\\
46	0.00430163468844592\\
47	0.00430163965162722\\
48	0.00430164470601016\\
49	0.00430164985327671\\
50	0.00430165509513993\\
51	0.00430166043334467\\
52	0.00430166586966809\\
53	0.00430167140592028\\
54	0.00430167704394488\\
55	0.00430168278561971\\
56	0.00430168863285737\\
57	0.0043016945876059\\
58	0.00430170065184949\\
59	0.00430170682760911\\
60	0.00430171311694324\\
61	0.00430171952194844\\
62	0.00430172604476025\\
63	0.00430173268755374\\
64	0.00430173945254442\\
65	0.00430174634198882\\
66	0.00430175335818536\\
67	0.00430176050347511\\
68	0.00430176778024267\\
69	0.00430177519091678\\
70	0.00430178273797132\\
71	0.00430179042392609\\
72	0.00430179825134768\\
73	0.00430180622285032\\
74	0.0043018143410968\\
75	0.00430182260879937\\
76	0.0043018310287207\\
77	0.0043018396036747\\
78	0.00430184833652761\\
79	0.00430185723019895\\
80	0.00430186628766246\\
81	0.00430187551194713\\
82	0.00430188490613833\\
83	0.00430189447337872\\
84	0.00430190421686942\\
85	0.00430191413987106\\
86	0.00430192424570493\\
87	0.00430193453775406\\
88	0.00430194501946442\\
89	0.00430195569434613\\
90	0.00430196656597458\\
91	0.00430197763799169\\
92	0.00430198891410719\\
93	0.00430200039809985\\
94	0.00430201209381875\\
95	0.00430202400518465\\
96	0.00430203613619134\\
97	0.00430204849090697\\
98	0.0043020610734754\\
99	0.00430207388811781\\
100	0.00430208693913393\\
101	0.00430210023090359\\
102	0.00430211376788829\\
103	0.00430212755463265\\
104	0.00430214159576601\\
105	0.00430215589600398\\
106	0.00430217046015014\\
107	0.00430218529309764\\
108	0.00430220039983083\\
109	0.0043022157854271\\
110	0.00430223145505849\\
111	0.00430224741399359\\
112	0.00430226366759927\\
113	0.0043022802213426\\
114	0.00430229708079265\\
115	0.00430231425162249\\
116	0.00430233173961106\\
117	0.00430234955064526\\
118	0.00430236769072188\\
119	0.00430238616594977\\
120	0.00430240498255189\\
121	0.00430242414686748\\
122	0.00430244366535419\\
123	0.00430246354459042\\
124	0.00430248379127747\\
125	0.00430250441224198\\
126	0.0043025254144382\\
127	0.00430254680495045\\
128	0.00430256859099555\\
129	0.00430259077992528\\
130	0.004302613379229\\
131	0.00430263639653623\\
132	0.00430265983961922\\
133	0.00430268371639573\\
134	0.00430270803493175\\
135	0.00430273280344428\\
136	0.00430275803030421\\
137	0.00430278372403919\\
138	0.00430280989333666\\
139	0.00430283654704676\\
140	0.00430286369418549\\
141	0.00430289134393785\\
142	0.00430291950566099\\
143	0.00430294818888749\\
144	0.00430297740332864\\
145	0.00430300715887788\\
146	0.00430303746561421\\
147	0.00430306833380574\\
148	0.00430309977391316\\
149	0.00430313179659353\\
150	0.00430316441270393\\
151	0.00430319763330525\\
152	0.00430323146966604\\
153	0.00430326593326646\\
154	0.00430330103580227\\
155	0.00430333678918899\\
156	0.00430337320556595\\
157	0.00430341029730062\\
158	0.00430344807699291\\
159	0.00430348655747963\\
160	0.00430352575183885\\
161	0.00430356567339461\\
162	0.00430360633572154\\
163	0.00430364775264963\\
164	0.00430368993826906\\
165	0.00430373290693513\\
166	0.00430377667327339\\
167	0.00430382125218463\\
168	0.00430386665885023\\
169	0.0043039129087374\\
170	0.00430396001760469\\
171	0.00430400800150752\\
172	0.00430405687680379\\
173	0.00430410666015964\\
174	0.0043041573685553\\
175	0.00430420901929114\\
176	0.00430426162999371\\
177	0.00430431521862189\\
178	0.00430436980347337\\
179	0.00430442540319092\\
180	0.00430448203676915\\
181	0.00430453972356102\\
182	0.00430459848328482\\
183	0.00430465833603102\\
184	0.00430471930226947\\
185	0.00430478140285652\\
186	0.00430484465904247\\
187	0.0043049090924791\\
188	0.00430497472522717\\
189	0.00430504157976439\\
190	0.00430510967899332\\
191	0.00430517904624945\\
192	0.00430524970530943\\
193	0.00430532168039956\\
194	0.00430539499620432\\
195	0.00430546967787514\\
196	0.00430554575103932\\
197	0.00430562324180901\\
198	0.00430570217679061\\
199	0.00430578258309413\\
200	0.00430586448834281\\
201	0.00430594792068292\\
202	0.00430603290879372\\
203	0.00430611948189766\\
204	0.00430620766977075\\
205	0.00430629750275308\\
206	0.00430638901175968\\
207	0.00430648222829146\\
208	0.00430657718444632\\
209	0.0043066739129306\\
210	0.00430677244707076\\
211	0.00430687282082509\\
212	0.0043069750687958\\
213	0.00430707922624141\\
214	0.00430718532908909\\
215	0.00430729341394761\\
216	0.00430740351812016\\
217	0.00430751567961771\\
218	0.00430762993717242\\
219	0.00430774633025148\\
220	0.004307864899071\\
221	0.00430798568461041\\
222	0.00430810872862685\\
223	0.00430823407367006\\
224	0.00430836176309745\\
225	0.00430849184108941\\
226	0.00430862435266503\\
227	0.00430875934369804\\
228	0.00430889686093293\\
229	0.00430903695200158\\
230	0.00430917966544005\\
231	0.00430932505070579\\
232	0.00430947315819503\\
233	0.00430962403926053\\
234	0.00430977774622978\\
235	0.00430993433242325\\
236	0.0043100938521734\\
237	0.00431025636084354\\
238	0.00431042191484736\\
239	0.0043105905716687\\
240	0.00431076238988176\\
241	0.00431093742917142\\
242	0.00431111575035422\\
243	0.00431129741539953\\
244	0.00431148248745111\\
245	0.00431167103084907\\
246	0.00431186311115226\\
247	0.00431205879516082\\
248	0.00431225815093942\\
249	0.00431246124784067\\
250	0.00431266815652895\\
251	0.00431287894900475\\
252	0.00431309369862925\\
253	0.00431331248014943\\
254	0.00431353536972344\\
255	0.00431376244494658\\
256	0.00431399378487743\\
257	0.00431422947006457\\
258	0.00431446958257367\\
259	0.00431471420601489\\
260	0.00431496342557085\\
261	0.00431521732802484\\
262	0.00431547600178957\\
263	0.00431573953693623\\
264	0.00431600802522396\\
265	0.00431628156012981\\
266	0.00431656023687893\\
267	0.00431684415247528\\
268	0.00431713340573277\\
269	0.00431742809730652\\
270	0.00431772832972479\\
271	0.00431803420742105\\
272	0.00431834583676664\\
273	0.00431866332610351\\
274	0.00431898678577741\\
275	0.00431931632817146\\
276	0.00431965206773999\\
277	0.00431999412104263\\
278	0.00432034260677865\\
279	0.00432069764582177\\
280	0.00432105936125497\\
281	0.00432142787840569\\
282	0.00432180332488109\\
283	0.00432218583060372\\
284	0.00432257552784727\\
285	0.00432297255127231\\
286	0.00432337703796235\\
287	0.00432378912746006\\
288	0.00432420896180337\\
289	0.00432463668556183\\
290	0.00432507244587283\\
291	0.0043255163924781\\
292	0.00432596867776004\\
293	0.00432642945677815\\
294	0.00432689888730534\\
295	0.00432737712986429\\
296	0.00432786434776395\\
297	0.00432836070713575\\
298	0.00432886637696996\\
299	0.00432938152915217\\
300	0.00432990633849959\\
301	0.00433044098279764\\
302	0.00433098564283659\\
303	0.00433154050244854\\
304	0.00433210574854453\\
305	0.00433268157115222\\
306	0.00433326816345408\\
307	0.00433386572182634\\
308	0.00433447444587875\\
309	0.00433509453849552\\
310	0.00433572620587763\\
311	0.00433636965758684\\
312	0.00433702510659156\\
313	0.00433769276931526\\
314	0.00433837286568757\\
315	0.00433906561919869\\
316	0.00433977125695748\\
317	0.00434049000975394\\
318	0.00434122211212661\\
319	0.00434196780243559\\
320	0.00434272732294176\\
321	0.004343500919893\\
322	0.00434428884361807\\
323	0.004345091348629\\
324	0.00434590869373256\\
325	0.00434674114215151\\
326	0.00434758896165619\\
327	0.00434845242470694\\
328	0.00434933180860762\\
329	0.00435022739567029\\
330	0.00435113947339081\\
331	0.00435206833463485\\
332	0.00435301427783332\\
333	0.00435397760718565\\
334	0.00435495863286911\\
335	0.00435595767125125\\
336	0.00435697504510243\\
337	0.00435801108380426\\
338	0.00435906612354941\\
339	0.00436014050752808\\
340	0.0043612345860951\\
341	0.0043623487169129\\
342	0.00436348326506536\\
343	0.00436463860313877\\
344	0.004365815111269\\
345	0.00436701317715537\\
346	0.00436823319604323\\
347	0.00436947557074711\\
348	0.0043707407118326\\
349	0.0043720290378046\\
350	0.00437334097530099\\
351	0.00437467695929117\\
352	0.00437603743327923\\
353	0.00437742284951051\\
354	0.00437883366918161\\
355	0.00438027036265104\\
356	0.00438173340965126\\
357	0.00438322329950548\\
358	0.00438474053134917\\
359	0.00438628561435644\\
360	0.00438785906797147\\
361	0.0043894614221449\\
362	0.00439109321757576\\
363	0.00439275500595847\\
364	0.00439444735023603\\
365	0.00439617082485916\\
366	0.00439792601605232\\
367	0.00439971352208733\\
368	0.00440153395356529\\
369	0.00440338793370829\\
370	0.00440527609866231\\
371	0.00440719909781317\\
372	0.0044091575941179\\
373	0.00441115226445455\\
374	0.00441318379999359\\
375	0.00441525290659536\\
376	0.00441736030523858\\
377	0.00441950673248561\\
378	0.00442169294099152\\
379	0.00442391970006501\\
380	0.00442618779629041\\
381	0.00442849803422117\\
382	0.00443085123715657\\
383	0.0044332482480148\\
384	0.00443568993031624\\
385	0.00443817716929183\\
386	0.004440710873132\\
387	0.00444329197439091\\
388	0.00444592143156016\\
389	0.00444860023082302\\
390	0.00445132938799669\\
391	0.00445410995066293\\
392	0.00445694300047828\\
393	0.00445982965564104\\
394	0.00446277107347559\\
395	0.00446576845307248\\
396	0.00446882303790564\\
397	0.004471936118343\\
398	0.00447510903394544\\
399	0.00447834317079437\\
400	0.00448163995012209\\
401	0.00448500082751085\\
402	0.00448842729181676\\
403	0.0044919208638358\\
404	0.00449548309475358\\
405	0.00449911556445268\\
406	0.00450281987978895\\
407	0.0045065976729882\\
408	0.00451045060034677\\
409	0.00451438034142587\\
410	0.00451838859889704\\
411	0.00452247709914652\\
412	0.00452664759381917\\
413	0.00453090186251676\\
414	0.0045352417147148\\
415	0.00453966898894157\\
416	0.0045441855506476\\
417	0.00454879328977864\\
418	0.0045534941180901\\
419	0.00455828996627101\\
420	0.00456318278098549\\
421	0.00456817452198913\\
422	0.00457326715952721\\
423	0.00457846267227382\\
424	0.00458376304609542\\
425	0.00458917027399014\\
426	0.00459468635791866\\
427	0.00460031331623116\\
428	0.00460605352979721\\
429	0.00461190990897305\\
430	0.00461788548949603\\
431	0.00462398344079992\\
432	0.00463020707486431\\
433	0.00463655985556917\\
434	0.0046430454085166\\
435	0.00464966753135385\\
436	0.00465643020445243\\
437	0.00466333760180588\\
438	0.00467039410194801\\
439	0.00467760429861155\\
440	0.00468497301074153\\
441	0.00469250529133314\\
442	0.00470020643436151\\
443	0.00470808197874832\\
444	0.00471613770768533\\
445	0.0047243796400989\\
446	0.00473281400643386\\
447	0.0047414471853645\\
448	0.0047502855217763\\
449	0.00475931359508267\\
450	0.00476852503652838\\
451	0.00477792411818879\\
452	0.00478751522745984\\
453	0.00479730286871586\\
454	0.00480729166239887\\
455	0.00481748633836206\\
456	0.00482789175655993\\
457	0.00483851291571316\\
458	0.00484935496444078\\
459	0.00486042321558921\\
460	0.00487172316447817\\
461	0.00488326051116053\\
462	0.00489504118541369\\
463	0.0049070713661606\\
464	0.0049193574708969\\
465	0.00493190604161025\\
466	0.00494476407015433\\
467	0.00495794401795656\\
468	0.00497145787261548\\
469	0.00498531831362211\\
470	0.00499953876316261\\
471	0.00501413343991552\\
472	0.00502911741621035\\
473	0.00504450667898729\\
474	0.00506031819431716\\
475	0.00507656997661637\\
476	0.00509328116762204\\
477	0.0051104721473742\\
478	0.0051281647361623\\
479	0.00514638269199349\\
480	0.00516515080193658\\
481	0.00518449389127852\\
482	0.0052044376264225\\
483	0.0052250091872611\\
484	0.00524623737495304\\
485	0.00526815187863189\\
486	0.00529078468564697\\
487	0.00531417357557784\\
488	0.00533837735097006\\
489	0.00536344414407026\\
490	0.0053894165439534\\
491	0.00541633916344982\\
492	0.00544425797132098\\
493	0.00547322049379098\\
494	0.00550327613013326\\
495	0.00553447607120864\\
496	0.00556687314828419\\
497	0.00560052158990553\\
498	0.00563547657178693\\
499	0.00567179360157377\\
500	0.00570952787723091\\
501	0.0057487334698313\\
502	0.00578946229353781\\
503	0.00583176282309453\\
504	0.00587568061430283\\
505	0.00592126098008407\\
506	0.00596854041180243\\
507	0.0060175431613355\\
508	0.00606827693345952\\
509	0.0061207277550409\\
510	0.0061748191704826\\
511	0.00623017318207606\\
512	0.00628662745140264\\
513	0.00634395484991725\\
514	0.00640187056429931\\
515	0.00646008917106267\\
516	0.00651816770353974\\
517	0.00657574688726379\\
518	0.00663128465556018\\
519	0.00668140030949382\\
520	0.0067258985133532\\
521	0.00676486664598173\\
522	0.00679941350434478\\
523	0.00683269133876112\\
524	0.00686511250829205\\
525	0.00689716691348905\\
526	0.00692935080768741\\
527	0.00696191476452744\\
528	0.00699497463284185\\
529	0.00702862070688934\\
530	0.00706290768641132\\
531	0.00709786900375971\\
532	0.00713353198173733\\
533	0.00716991911972827\\
534	0.00720705005775949\\
535	0.00724494357205851\\
536	0.00728361853919417\\
537	0.00732309453011616\\
538	0.00736339161534406\\
539	0.00740453024079764\\
540	0.00744653870738161\\
541	0.00748948972389208\\
542	0.00753347142049639\\
543	0.00757747381845585\\
544	0.00762044744386721\\
545	0.00766211550727995\\
546	0.00770387386088141\\
547	0.00774606208827088\\
548	0.00778876715738478\\
549	0.00783205956081914\\
550	0.00787593614332746\\
551	0.00792038862872709\\
552	0.00796540239658982\\
553	0.00801096018996022\\
554	0.00805704284211701\\
555	0.00810363228870382\\
556	0.00815069909831113\\
557	0.00819820719505273\\
558	0.00824613258604494\\
559	0.00829446918237396\\
560	0.00834131866989406\\
561	0.00838711839027356\\
562	0.00843309183235914\\
563	0.00847930989270675\\
564	0.00852577430147598\\
565	0.00857245512281394\\
566	0.00861932104340902\\
567	0.0086663403000851\\
568	0.00871348883275888\\
569	0.00876041921360522\\
570	0.00880715257596726\\
571	0.00885403364170714\\
572	0.00890105214066917\\
573	0.0089481720325389\\
574	0.00899535448039004\\
575	0.0090425580087869\\
576	0.00908973854196738\\
577	0.00913684950111323\\
578	0.00918384197867516\\
579	0.00923066501140877\\
580	0.0092772659789151\\
581	0.00932359116078622\\
582	0.00936958649310198\\
583	0.00941519857412797\\
584	0.00946037597948012\\
585	0.00950507095777015\\
586	0.00954924158550055\\
587	0.00959285445497632\\
588	0.00963588792409171\\
589	0.0096783357981771\\
590	0.00972021084271539\\
591	0.00976138933978722\\
592	0.00980171928775089\\
593	0.00984102135019415\\
594	0.00987905595166784\\
595	0.00991543381058343\\
596	0.00994937493726701\\
597	0.00997906286423442\\
598	0.0099999191923403\\
599	0\\
600	0\\
};
\addplot [color=black!20!mycolor21,solid,forget plot]
  table[row sep=crcr]{%
1	0.00431499694832056\\
2	0.00431499894077788\\
3	0.00431500096991465\\
4	0.00431500303640849\\
5	0.00431500514094959\\
6	0.00431500728424092\\
7	0.00431500946699855\\
8	0.00431501168995181\\
9	0.00431501395384356\\
10	0.00431501625943045\\
11	0.00431501860748317\\
12	0.00431502099878675\\
13	0.00431502343414068\\
14	0.00431502591435945\\
15	0.00431502844027261\\
16	0.00431503101272507\\
17	0.00431503363257744\\
18	0.00431503630070642\\
19	0.00431503901800483\\
20	0.00431504178538223\\
21	0.00431504460376497\\
22	0.00431504747409669\\
23	0.00431505039733851\\
24	0.00431505337446941\\
25	0.0043150564064866\\
26	0.00431505949440579\\
27	0.00431506263926162\\
28	0.0043150658421079\\
29	0.00431506910401811\\
30	0.00431507242608562\\
31	0.00431507580942415\\
32	0.0043150792551682\\
33	0.0043150827644733\\
34	0.0043150863385165\\
35	0.00431508997849671\\
36	0.00431509368563522\\
37	0.00431509746117602\\
38	0.0043151013063862\\
39	0.00431510522255646\\
40	0.0043151092110016\\
41	0.00431511327306083\\
42	0.00431511741009825\\
43	0.00431512162350343\\
44	0.00431512591469183\\
45	0.0043151302851052\\
46	0.0043151347362122\\
47	0.00431513926950884\\
48	0.00431514388651896\\
49	0.00431514858879484\\
50	0.0043151533779177\\
51	0.0043151582554982\\
52	0.00431516322317698\\
53	0.00431516828262534\\
54	0.00431517343554571\\
55	0.00431517868367225\\
56	0.00431518402877144\\
57	0.00431518947264272\\
58	0.00431519501711916\\
59	0.00431520066406793\\
60	0.00431520641539104\\
61	0.00431521227302606\\
62	0.00431521823894665\\
63	0.00431522431516331\\
64	0.00431523050372412\\
65	0.00431523680671531\\
66	0.00431524322626214\\
67	0.00431524976452953\\
68	0.00431525642372277\\
69	0.00431526320608839\\
70	0.00431527011391491\\
71	0.00431527714953353\\
72	0.00431528431531907\\
73	0.00431529161369067\\
74	0.00431529904711277\\
75	0.00431530661809576\\
76	0.00431531432919702\\
77	0.00431532218302178\\
78	0.00431533018222386\\
79	0.00431533832950687\\
80	0.00431534662762482\\
81	0.00431535507938334\\
82	0.00431536368764055\\
83	0.00431537245530801\\
84	0.00431538138535179\\
85	0.00431539048079349\\
86	0.00431539974471124\\
87	0.00431540918024076\\
88	0.00431541879057662\\
89	0.00431542857897303\\
90	0.00431543854874529\\
91	0.00431544870327075\\
92	0.00431545904599003\\
93	0.00431546958040823\\
94	0.00431548031009616\\
95	0.00431549123869157\\
96	0.00431550236990035\\
97	0.0043155137074979\\
98	0.00431552525533045\\
99	0.00431553701731626\\
100	0.00431554899744721\\
101	0.00431556119979001\\
102	0.0043155736284877\\
103	0.00431558628776109\\
104	0.00431559918191016\\
105	0.00431561231531569\\
106	0.00431562569244063\\
107	0.00431563931783176\\
108	0.00431565319612135\\
109	0.0043156673320286\\
110	0.00431568173036141\\
111	0.004315696396018\\
112	0.00431571133398872\\
113	0.00431572654935767\\
114	0.00431574204730453\\
115	0.00431575783310647\\
116	0.00431577391213986\\
117	0.00431579028988224\\
118	0.00431580697191418\\
119	0.00431582396392129\\
120	0.00431584127169621\\
121	0.00431585890114065\\
122	0.00431587685826738\\
123	0.0043158951492025\\
124	0.00431591378018745\\
125	0.0043159327575813\\
126	0.00431595208786295\\
127	0.0043159717776334\\
128	0.00431599183361809\\
129	0.00431601226266936\\
130	0.00431603307176863\\
131	0.00431605426802919\\
132	0.00431607585869843\\
133	0.00431609785116058\\
134	0.00431612025293926\\
135	0.00431614307170008\\
136	0.00431616631525346\\
137	0.00431618999155733\\
138	0.00431621410872\\
139	0.00431623867500296\\
140	0.00431626369882397\\
141	0.00431628918875977\\
142	0.00431631515354941\\
143	0.00431634160209725\\
144	0.00431636854347602\\
145	0.00431639598693029\\
146	0.00431642394187952\\
147	0.00431645241792153\\
148	0.00431648142483594\\
149	0.00431651097258764\\
150	0.00431654107133032\\
151	0.00431657173141015\\
152	0.00431660296336936\\
153	0.00431663477795024\\
154	0.00431666718609875\\
155	0.00431670019896855\\
156	0.00431673382792501\\
157	0.00431676808454926\\
158	0.00431680298064243\\
159	0.00431683852822968\\
160	0.00431687473956477\\
161	0.00431691162713437\\
162	0.0043169492036625\\
163	0.0043169874821152\\
164	0.00431702647570518\\
165	0.00431706619789658\\
166	0.00431710666240985\\
167	0.00431714788322675\\
168	0.00431718987459531\\
169	0.00431723265103512\\
170	0.00431727622734244\\
171	0.00431732061859577\\
172	0.00431736584016113\\
173	0.00431741190769776\\
174	0.00431745883716381\\
175	0.00431750664482215\\
176	0.00431755534724629\\
177	0.00431760496132637\\
178	0.00431765550427544\\
179	0.00431770699363569\\
180	0.00431775944728483\\
181	0.00431781288344283\\
182	0.00431786732067833\\
183	0.00431792277791573\\
184	0.00431797927444189\\
185	0.00431803682991343\\
186	0.00431809546436388\\
187	0.00431815519821102\\
188	0.00431821605226454\\
189	0.00431827804773369\\
190	0.00431834120623502\\
191	0.00431840554980058\\
192	0.00431847110088596\\
193	0.00431853788237864\\
194	0.0043186059176065\\
195	0.00431867523034657\\
196	0.00431874584483381\\
197	0.00431881778577029\\
198	0.00431889107833416\\
199	0.00431896574818935\\
200	0.004319041821495\\
201	0.0043191193249154\\
202	0.00431919828562992\\
203	0.00431927873134326\\
204	0.0043193606902959\\
205	0.0043194441912748\\
206	0.00431952926362411\\
207	0.00431961593725647\\
208	0.0043197042426642\\
209	0.00431979421093099\\
210	0.00431988587374356\\
211	0.00431997926340383\\
212	0.00432007441284126\\
213	0.00432017135562532\\
214	0.00432027012597845\\
215	0.00432037075878894\\
216	0.00432047328962461\\
217	0.00432057775474629\\
218	0.00432068419112179\\
219	0.00432079263644017\\
220	0.00432090312912635\\
221	0.00432101570835582\\
222	0.00432113041406999\\
223	0.00432124728699163\\
224	0.00432136636864066\\
225	0.00432148770135029\\
226	0.00432161132828377\\
227	0.00432173729345093\\
228	0.00432186564172562\\
229	0.00432199641886331\\
230	0.00432212967151902\\
231	0.00432226544726559\\
232	0.00432240379461269\\
233	0.00432254476302579\\
234	0.00432268840294584\\
235	0.00432283476580929\\
236	0.00432298390406842\\
237	0.0043231358712123\\
238	0.00432329072178821\\
239	0.00432344851142332\\
240	0.004323609296847\\
241	0.00432377313591361\\
242	0.00432394008762582\\
243	0.00432411021215819\\
244	0.00432428357088166\\
245	0.00432446022638823\\
246	0.00432464024251637\\
247	0.00432482368437691\\
248	0.0043250106183795\\
249	0.00432520111225972\\
250	0.00432539523510669\\
251	0.00432559305739126\\
252	0.0043257946509949\\
253	0.00432600008923925\\
254	0.00432620944691617\\
255	0.00432642280031853\\
256	0.00432664022727168\\
257	0.00432686180716552\\
258	0.00432708762098739\\
259	0.00432731775135558\\
260	0.00432755228255349\\
261	0.0043277913005647\\
262	0.00432803489310863\\
263	0.00432828314967698\\
264	0.00432853616157106\\
265	0.00432879402193971\\
266	0.00432905682581817\\
267	0.00432932467016762\\
268	0.00432959765391561\\
269	0.00432987587799733\\
270	0.00433015944539764\\
271	0.00433044846119397\\
272	0.00433074303260004\\
273	0.00433104326901036\\
274	0.00433134928204578\\
275	0.00433166118559969\\
276	0.00433197909588508\\
277	0.00433230313148254\\
278	0.00433263341338901\\
279	0.00433297006506743\\
280	0.00433331321249714\\
281	0.00433366298422504\\
282	0.00433401951141767\\
283	0.00433438292791386\\
284	0.00433475337027831\\
285	0.00433513097785564\\
286	0.00433551589282521\\
287	0.00433590826025656\\
288	0.00433630822816532\\
289	0.00433671594756963\\
290	0.00433713157254702\\
291	0.00433755526029147\\
292	0.00433798717117093\\
293	0.00433842746878474\\
294	0.00433887632002143\\
295	0.00433933389511596\\
296	0.00433980036770704\\
297	0.00434027591489401\\
298	0.00434076071729329\\
299	0.00434125495909364\\
300	0.00434175882811116\\
301	0.00434227251584266\\
302	0.00434279621751782\\
303	0.00434333013214985\\
304	0.00434387446258422\\
305	0.00434442941554523\\
306	0.0043449952016803\\
307	0.00434557203560147\\
308	0.00434616013592399\\
309	0.00434675972530172\\
310	0.00434737103045892\\
311	0.00434799428221822\\
312	0.00434862971552461\\
313	0.00434927756946509\\
314	0.00434993808728401\\
315	0.00435061151639384\\
316	0.00435129810838146\\
317	0.00435199811901004\\
318	0.00435271180821678\\
319	0.0043534394401069\\
320	0.00435418128294456\\
321	0.00435493760914164\\
322	0.00435570869524555\\
323	0.00435649482192759\\
324	0.0043572962739739\\
325	0.00435811334028154\\
326	0.00435894631386279\\
327	0.00435979549186134\\
328	0.00436066117558451\\
329	0.00436154367055733\\
330	0.0043624432866041\\
331	0.00436336033796421\\
332	0.00436429514345049\\
333	0.00436524802665853\\
334	0.00436621931623619\\
335	0.00436720934622373\\
336	0.00436821845647436\\
337	0.00436924699316567\\
338	0.00437029530941072\\
339	0.00437136376597633\\
340	0.00437245273211302\\
341	0.00437356258649627\\
342	0.00437469371827179\\
343	0.00437584652819098\\
344	0.00437702142981376\\
345	0.00437821885075589\\
346	0.00437943923394699\\
347	0.00438068303683586\\
348	0.0043819507271359\\
349	0.00438324278311871\\
350	0.00438455969391738\\
351	0.00438590195984028\\
352	0.00438727009269485\\
353	0.00438866461612229\\
354	0.00439008606594292\\
355	0.00439153499051145\\
356	0.00439301195107609\\
357	0.00439451752214569\\
358	0.00439605229186454\\
359	0.00439761686239407\\
360	0.00439921185030041\\
361	0.00440083788694632\\
362	0.00440249561888556\\
363	0.00440418570825825\\
364	0.00440590883318406\\
365	0.00440766568815095\\
366	0.00440945698439527\\
367	0.0044112834502697\\
368	0.00441314583159376\\
369	0.00441504489198145\\
370	0.00441698141313957\\
371	0.00441895619512892\\
372	0.00442097005658\\
373	0.00442302383485335\\
374	0.0044251183861332\\
375	0.00442725458544242\\
376	0.00442943332656388\\
377	0.00443165552185372\\
378	0.00443392210192886\\
379	0.00443623401521027\\
380	0.00443859222730249\\
381	0.00444099772018842\\
382	0.00444345149121776\\
383	0.00444595455186778\\
384	0.00444850792625582\\
385	0.00445111264938523\\
386	0.00445376976511048\\
387	0.00445648032381303\\
388	0.00445924537978965\\
389	0.00446206598836737\\
390	0.00446494320277788\\
391	0.00446787807084819\\
392	0.00447087163159433\\
393	0.00447392491184105\\
394	0.00447703892302631\\
395	0.00448021465837707\\
396	0.00448345309066581\\
397	0.00448675517093273\\
398	0.00449012182999356\\
399	0.00449355412269602\\
400	0.00449705347939348\\
401	0.00450062137353033\\
402	0.00450425932415012\\
403	0.00450796889867491\\
404	0.00451175171598503\\
405	0.00451560944982605\\
406	0.00451954383256431\\
407	0.00452355665930099\\
408	0.00452764979233868\\
409	0.00453182516598714\\
410	0.0045360847917572\\
411	0.00454043076433954\\
412	0.00454486526831602\\
413	0.00454939058452903\\
414	0.00455400909656978\\
415	0.00455872329750913\\
416	0.00456353579682076\\
417	0.00456844932732751\\
418	0.00457346675197657\\
419	0.00457859107016703\\
420	0.00458382542321563\\
421	0.00458917309840705\\
422	0.00459463753106349\\
423	0.00460022230356219\\
424	0.0046059311395983\\
425	0.00461176788800474\\
426	0.00461773648292344\\
427	0.00462384083473687\\
428	0.00463007454269391\\
429	0.00463642669478293\\
430	0.00464289947138564\\
431	0.00464949507804951\\
432	0.00465621574373993\\
433	0.00466306371942736\\
434	0.00467004127672492\\
435	0.00467715070359887\\
436	0.00468439430175754\\
437	0.00469177438409394\\
438	0.00469929327236011\\
439	0.0047069532953228\\
440	0.00471475678774227\\
441	0.00472270609062633\\
442	0.00473080355332531\\
443	0.00473905153808924\\
444	0.00474745242750651\\
445	0.00475600863417552\\
446	0.00476472260824388\\
447	0.00477359682704871\\
448	0.00478263371851145\\
449	0.00479185742060597\\
450	0.00480128388406291\\
451	0.00481091912788268\\
452	0.0048207694746903\\
453	0.00483084157395689\\
454	0.00484114242736373\\
455	0.00485167941703031\\
456	0.00486246033618055\\
457	0.00487349342185919\\
458	0.00488478739021973\\
459	0.00489635147453042\\
460	0.00490819546631836\\
461	0.00492032976123099\\
462	0.00493276541531882\\
463	0.00494551423176829\\
464	0.00495858894656423\\
465	0.00497200374370233\\
466	0.00498577276785185\\
467	0.00499990968924556\\
468	0.00501442887447785\\
469	0.00502934546358139\\
470	0.00504467541338836\\
471	0.00506043554226604\\
472	0.00507664357606649\\
473	0.00509331819510764\\
474	0.00511047908213691\\
475	0.00512814697166886\\
476	0.00514634370104921\\
477	0.00516509226166558\\
478	0.00518441684913845\\
479	0.00520434292710943\\
480	0.00522489733605872\\
481	0.00524610843829957\\
482	0.00526800545903931\\
483	0.00529061981117917\\
484	0.00531398865034318\\
485	0.0053381701578214\\
486	0.00536321129115987\\
487	0.00538915358114079\\
488	0.00541604042217931\\
489	0.0054439163752652\\
490	0.00547282747592509\\
491	0.00550282136238553\\
492	0.00553394694540943\\
493	0.0055662540850593\\
494	0.00559979322677082\\
495	0.00563461493277553\\
496	0.00567076928958796\\
497	0.00570830517638911\\
498	0.00574727309521228\\
499	0.00578772535756451\\
500	0.00582971021638085\\
501	0.00587326972706042\\
502	0.00591843697540275\\
503	0.00596523248465533\\
504	0.00601358359371006\\
505	0.00606322632772836\\
506	0.00611408959572197\\
507	0.00616606641138297\\
508	0.00621900369888828\\
509	0.00627268736679497\\
510	0.00632686701159178\\
511	0.00638156012032947\\
512	0.00643639486293027\\
513	0.00649077754730627\\
514	0.00654356250451543\\
515	0.00659100231749131\\
516	0.00663288736294835\\
517	0.00666934946135635\\
518	0.00670138719394604\\
519	0.00673220793885969\\
520	0.00676222399287684\\
521	0.00679192047183419\\
522	0.0068217634860095\\
523	0.00685198358162248\\
524	0.00688268742381229\\
525	0.00691395526121181\\
526	0.00694583429170407\\
527	0.0069783535437772\\
528	0.00701153670833231\\
529	0.00704540320024276\\
530	0.00707997042946935\\
531	0.00711525540894092\\
532	0.0071512752192859\\
533	0.0071880474870327\\
534	0.00722559083761262\\
535	0.0072639251102361\\
536	0.00730307067581672\\
537	0.00734304842914756\\
538	0.00738391399526787\\
539	0.00742574890892478\\
540	0.00746833575598918\\
541	0.00750995398553259\\
542	0.00755035018874908\\
543	0.00759032441223851\\
544	0.00763072324758412\\
545	0.00767162903141855\\
546	0.00771311969698628\\
547	0.00775520894382689\\
548	0.00779789355612366\\
549	0.00784116403536065\\
550	0.007885008292511\\
551	0.00792941148083906\\
552	0.00797435600912795\\
553	0.00801982170725804\\
554	0.00806578702538616\\
555	0.00811222764331581\\
556	0.0081591063939708\\
557	0.00820641693101756\\
558	0.00825328101622345\\
559	0.00829844873438657\\
560	0.00834367510390144\\
561	0.00838916204614102\\
562	0.00843494259992949\\
563	0.00848099134388042\\
564	0.00852727997864265\\
565	0.00857377962259063\\
566	0.008620461017887\\
567	0.00866730268255435\\
568	0.00871391830588996\\
569	0.00876044306491334\\
570	0.00880715274894682\\
571	0.00885403364426453\\
572	0.00890105214164188\\
573	0.00894817203301684\\
574	0.00899535448062322\\
575	0.00904255800889528\\
576	0.00908973854201466\\
577	0.00913684950113236\\
578	0.00918384197868222\\
579	0.0092306650114111\\
580	0.00927726597891577\\
581	0.00932359116078638\\
582	0.00936958649310202\\
583	0.00941519857412799\\
584	0.00946037597948012\\
585	0.00950507095777015\\
586	0.00954924158550055\\
587	0.00959285445497631\\
588	0.00963588792409171\\
589	0.00967833579817709\\
590	0.00972021084271539\\
591	0.00976138933978722\\
592	0.00980171928775089\\
593	0.00984102135019415\\
594	0.00987905595166784\\
595	0.00991543381058343\\
596	0.00994937493726701\\
597	0.00997906286423442\\
598	0.0099999191923403\\
599	0\\
600	0\\
};
\addplot [color=black!50!mycolor20,solid,forget plot]
  table[row sep=crcr]{%
1	0.00432156273776794\\
2	0.0043215647387563\\
3	0.00432156677664413\\
4	0.00432156885211437\\
5	0.00432157096586263\\
6	0.00432157311859742\\
7	0.00432157531104039\\
8	0.00432157754392669\\
9	0.00432157981800506\\
10	0.00432158213403821\\
11	0.00432158449280303\\
12	0.00432158689509076\\
13	0.00432158934170756\\
14	0.00432159183347438\\
15	0.0043215943712275\\
16	0.00432159695581882\\
17	0.00432159958811599\\
18	0.00432160226900288\\
19	0.00432160499937982\\
20	0.00432160778016382\\
21	0.00432161061228898\\
22	0.00432161349670679\\
23	0.00432161643438645\\
24	0.00432161942631524\\
25	0.00432162247349878\\
26	0.00432162557696142\\
27	0.00432162873774662\\
28	0.00432163195691715\\
29	0.00432163523555574\\
30	0.00432163857476522\\
31	0.00432164197566891\\
32	0.0043216454394111\\
33	0.00432164896715743\\
34	0.00432165256009517\\
35	0.00432165621943383\\
36	0.00432165994640539\\
37	0.00432166374226482\\
38	0.00432166760829047\\
39	0.00432167154578451\\
40	0.00432167555607341\\
41	0.00432167964050835\\
42	0.00432168380046576\\
43	0.00432168803734771\\
44	0.00432169235258235\\
45	0.0043216967476246\\
46	0.00432170122395643\\
47	0.00432170578308751\\
48	0.00432171042655564\\
49	0.0043217151559273\\
50	0.00432171997279828\\
51	0.00432172487879413\\
52	0.0043217298755707\\
53	0.00432173496481483\\
54	0.00432174014824476\\
55	0.00432174542761091\\
56	0.00432175080469635\\
57	0.00432175628131747\\
58	0.00432176185932452\\
59	0.00432176754060238\\
60	0.00432177332707118\\
61	0.00432177922068684\\
62	0.00432178522344185\\
63	0.00432179133736603\\
64	0.00432179756452703\\
65	0.00432180390703123\\
66	0.00432181036702437\\
67	0.00432181694669225\\
68	0.00432182364826164\\
69	0.00432183047400093\\
70	0.00432183742622088\\
71	0.0043218445072756\\
72	0.00432185171956308\\
73	0.00432185906552633\\
74	0.00432186654765398\\
75	0.00432187416848134\\
76	0.00432188193059106\\
77	0.00432188983661416\\
78	0.00432189788923104\\
79	0.00432190609117213\\
80	0.0043219144452191\\
81	0.00432192295420573\\
82	0.00432193162101884\\
83	0.00432194044859928\\
84	0.00432194943994308\\
85	0.00432195859810238\\
86	0.00432196792618659\\
87	0.00432197742736336\\
88	0.00432198710485967\\
89	0.00432199696196312\\
90	0.00432200700202283\\
91	0.00432201722845085\\
92	0.00432202764472316\\
93	0.00432203825438095\\
94	0.0043220490610319\\
95	0.00432206006835131\\
96	0.00432207128008351\\
97	0.00432208270004313\\
98	0.00432209433211635\\
99	0.00432210618026241\\
100	0.0043221182485148\\
101	0.00432213054098283\\
102	0.00432214306185296\\
103	0.00432215581539031\\
104	0.00432216880594014\\
105	0.00432218203792936\\
106	0.00432219551586814\\
107	0.00432220924435133\\
108	0.00432222322806028\\
109	0.00432223747176431\\
110	0.00432225198032239\\
111	0.00432226675868503\\
112	0.00432228181189574\\
113	0.00432229714509297\\
114	0.00432231276351192\\
115	0.00432232867248623\\
116	0.00432234487745005\\
117	0.0043223613839398\\
118	0.00432237819759618\\
119	0.00432239532416614\\
120	0.00432241276950487\\
121	0.00432243053957785\\
122	0.00432244864046308\\
123	0.00432246707835305\\
124	0.00432248585955702\\
125	0.00432250499050324\\
126	0.00432252447774112\\
127	0.00432254432794366\\
128	0.00432256454790976\\
129	0.00432258514456658\\
130	0.0043226061249721\\
131	0.00432262749631745\\
132	0.00432264926592956\\
133	0.00432267144127379\\
134	0.00432269402995643\\
135	0.0043227170397275\\
136	0.00432274047848354\\
137	0.00432276435427027\\
138	0.00432278867528554\\
139	0.00432281344988224\\
140	0.0043228386865711\\
141	0.00432286439402405\\
142	0.00432289058107703\\
143	0.00432291725673317\\
144	0.0043229444301661\\
145	0.00432297211072314\\
146	0.00432300030792858\\
147	0.00432302903148723\\
148	0.00432305829128777\\
149	0.00432308809740627\\
150	0.00432311846010989\\
151	0.00432314938986035\\
152	0.00432318089731803\\
153	0.00432321299334535\\
154	0.00432324568901105\\
155	0.00432327899559389\\
156	0.00432331292458686\\
157	0.00432334748770122\\
158	0.00432338269687067\\
159	0.00432341856425585\\
160	0.00432345510224846\\
161	0.00432349232347589\\
162	0.00432353024080571\\
163	0.0043235688673503\\
164	0.00432360821647165\\
165	0.00432364830178617\\
166	0.00432368913716954\\
167	0.0043237307367618\\
168	0.00432377311497254\\
169	0.00432381628648593\\
170	0.00432386026626634\\
171	0.00432390506956349\\
172	0.00432395071191814\\
173	0.00432399720916782\\
174	0.00432404457745243\\
175	0.00432409283322027\\
176	0.00432414199323388\\
177	0.00432419207457645\\
178	0.00432424309465783\\
179	0.00432429507122102\\
180	0.00432434802234867\\
181	0.00432440196646958\\
182	0.00432445692236575\\
183	0.00432451290917902\\
184	0.00432456994641837\\
185	0.00432462805396691\\
186	0.00432468725208936\\
187	0.00432474756143953\\
188	0.00432480900306796\\
189	0.00432487159842961\\
190	0.00432493536939208\\
191	0.0043250003382435\\
192	0.00432506652770101\\
193	0.00432513396091912\\
194	0.00432520266149843\\
195	0.00432527265349443\\
196	0.00432534396142653\\
197	0.00432541661028717\\
198	0.00432549062555138\\
199	0.00432556603318634\\
200	0.00432564285966097\\
201	0.00432572113195616\\
202	0.00432580087757486\\
203	0.00432588212455263\\
204	0.00432596490146808\\
205	0.00432604923745394\\
206	0.00432613516220803\\
207	0.00432622270600464\\
208	0.00432631189970619\\
209	0.00432640277477489\\
210	0.00432649536328497\\
211	0.00432658969793499\\
212	0.0043266858120604\\
213	0.00432678373964642\\
214	0.00432688351534124\\
215	0.00432698517446952\\
216	0.00432708875304603\\
217	0.00432719428778965\\
218	0.00432730181613792\\
219	0.00432741137626148\\
220	0.00432752300707917\\
221	0.00432763674827337\\
222	0.00432775264030556\\
223	0.00432787072443241\\
224	0.00432799104272207\\
225	0.00432811363807102\\
226	0.00432823855422097\\
227	0.00432836583577658\\
228	0.00432849552822319\\
229	0.00432862767794518\\
230	0.00432876233224459\\
231	0.00432889953936042\\
232	0.00432903934848804\\
233	0.00432918180979924\\
234	0.00432932697446275\\
235	0.00432947489466525\\
236	0.0043296256236327\\
237	0.0043297792156524\\
238	0.00432993572609534\\
239	0.00433009521143934\\
240	0.0043302577292924\\
241	0.004330423338417\\
242	0.00433059209875454\\
243	0.00433076407145074\\
244	0.00433093931888154\\
245	0.00433111790467953\\
246	0.00433129989376101\\
247	0.00433148535235388\\
248	0.00433167434802614\\
249	0.00433186694971496\\
250	0.00433206322775662\\
251	0.00433226325391725\\
252	0.00433246710142406\\
253	0.00433267484499761\\
254	0.00433288656088486\\
255	0.00433310232689295\\
256	0.00433332222242387\\
257	0.00433354632851018\\
258	0.00433377472785137\\
259	0.00433400750485138\\
260	0.00433424474565704\\
261	0.00433448653819752\\
262	0.00433473297222474\\
263	0.00433498413935495\\
264	0.00433524013311138\\
265	0.004335501048968\\
266	0.00433576698439459\\
267	0.00433603803890286\\
268	0.00433631431409392\\
269	0.00433659591370706\\
270	0.0043368829436698\\
271	0.00433717551214946\\
272	0.00433747372960603\\
273	0.00433777770884665\\
274	0.00433808756508153\\
275	0.00433840341598155\\
276	0.0043387253817374\\
277	0.00433905358512051\\
278	0.00433938815154574\\
279	0.00433972920913584\\
280	0.00434007688878776\\
281	0.00434043132424101\\
282	0.00434079265214799\\
283	0.00434116101214642\\
284	0.00434153654693376\\
285	0.004341919402344\\
286	0.00434230972742683\\
287	0.00434270767452879\\
288	0.00434311339937728\\
289	0.00434352706116669\\
290	0.00434394882264741\\
291	0.00434437885021715\\
292	0.00434481731401516\\
293	0.00434526438801898\\
294	0.00434572025014423\\
295	0.00434618508234692\\
296	0.00434665907072889\\
297	0.00434714240564587\\
298	0.00434763528181837\\
299	0.00434813789844575\\
300	0.00434865045932256\\
301	0.00434917317295794\\
302	0.00434970625269746\\
303	0.00435024991684725\\
304	0.00435080438880053\\
305	0.00435136989716599\\
306	0.00435194667589759\\
307	0.00435253496442565\\
308	0.00435313500778857\\
309	0.00435374705676422\\
310	0.00435437136800087\\
311	0.00435500820414609\\
312	0.00435565783397282\\
313	0.0043563205325014\\
314	0.0043569965811155\\
315	0.00435768626767073\\
316	0.00435838988659347\\
317	0.00435910773896732\\
318	0.0043598401326048\\
319	0.00436058738210031\\
320	0.0043613498088614\\
321	0.00436212774111298\\
322	0.00436292151387087\\
323	0.0043637314688778\\
324	0.00436455795449694\\
325	0.00436540132555528\\
326	0.00436626194312957\\
327	0.00436714017426647\\
328	0.00436803639162843\\
329	0.00436895097305518\\
330	0.00436988430103141\\
331	0.0043708367620511\\
332	0.00437180874586785\\
333	0.00437280064462305\\
334	0.00437381285184419\\
335	0.00437484576130809\\
336	0.0043758997657681\\
337	0.00437697525554858\\
338	0.00437807261701865\\
339	0.00437919223096592\\
340	0.00438033447090391\\
341	0.00438149970136107\\
342	0.00438268827621516\\
343	0.00438390053714678\\
344	0.00438513681229731\\
345	0.00438639741528997\\
346	0.00438768264538279\\
347	0.00438899284910283\\
348	0.00439032852186834\\
349	0.00439169016963117\\
350	0.00439307830916022\\
351	0.00439449346833624\\
352	0.00439593618645757\\
353	0.00439740701455486\\
354	0.00439890651571774\\
355	0.004400435265466\\
356	0.00440199385231873\\
357	0.00440358287824578\\
358	0.00440520295913732\\
359	0.00440685472530561\\
360	0.00440853882202268\\
361	0.00441025591009763\\
362	0.00441200666649775\\
363	0.0044137917850179\\
364	0.0044156119770036\\
365	0.00441746797213329\\
366	0.00441936051926643\\
367	0.00442129038736393\\
368	0.00442325836648904\\
369	0.00442526526889701\\
370	0.00442731193022267\\
371	0.0044293992107759\\
372	0.00443152799695614\\
373	0.00443369920279728\\
374	0.00443591377165537\\
375	0.00443817267805222\\
376	0.00444047692968832\\
377	0.0044428275696388\\
378	0.00444522567874573\\
379	0.00444767237822\\
380	0.00445016883246353\\
381	0.00445271625212137\\
382	0.00445531589736821\\
383	0.00445796908142958\\
384	0.00446067717432911\\
385	0.00446344160684308\\
386	0.00446626387462823\\
387	0.0044691455424694\\
388	0.00447208824856691\\
389	0.00447509370874885\\
390	0.00447816372044754\\
391	0.00448130016621814\\
392	0.00448450501649252\\
393	0.00448778033113468\\
394	0.00449112825913685\\
395	0.00449455103528687\\
396	0.00449805097120155\\
397	0.00450163043343578\\
398	0.00450529178476443\\
399	0.00450903312499081\\
400	0.00451284518155388\\
401	0.00451672918428431\\
402	0.00452068637426448\\
403	0.00452471800279135\\
404	0.0045288253302109\\
405	0.0045330096246486\\
406	0.00453727216070178\\
407	0.00454161421822174\\
408	0.00454603708133855\\
409	0.00455054203755669\\
410	0.00455513037477349\\
411	0.00455980336567712\\
412	0.0045645622530175\\
413	0.00456940826439893\\
414	0.00457434261038014\\
415	0.00457936647922931\\
416	0.00458448103142263\\
417	0.00458968739511137\\
418	0.00459498666169048\\
419	0.00460037988191109\\
420	0.0046058680639046\\
421	0.0046114521734614\\
422	0.00461713313071034\\
423	0.00462291180709291\\
424	0.00462878901320134\\
425	0.00463476551515811\\
426	0.00464084202686863\\
427	0.00464701916605228\\
428	0.00465330764650406\\
429	0.00465972294715053\\
430	0.00466626796949668\\
431	0.0046729457221835\\
432	0.00467975932905515\\
433	0.00468671203806247\\
434	0.00469380723103485\\
435	0.0047010484344903\\
436	0.0047084393316286\\
437	0.00471598377555301\\
438	0.00472368580381427\\
439	0.00473154965436415\\
440	0.004739579782992\\
441	0.00474778088230933\\
442	0.00475615790236735\\
443	0.00476471607315437\\
444	0.00477346092989589\\
445	0.00478239834456105\\
446	0.00479153457560125\\
447	0.00480087637727662\\
448	0.00481043130862172\\
449	0.00482020706581279\\
450	0.00483021084487188\\
451	0.00484044985114416\\
452	0.00485093163581722\\
453	0.00486166411637998\\
454	0.00487265559813321\\
455	0.00488391479673149\\
456	0.00489545086173582\\
457	0.00490727340119204\\
458	0.00491939250723813\\
459	0.00493181878277906\\
460	0.00494456336936779\\
461	0.00495763797663287\\
462	0.00497105491383159\\
463	0.0049848271235522\\
464	0.0049989682118395\\
465	0.0050134924396604\\
466	0.00502841475798565\\
467	0.00504375091022794\\
468	0.00505951747774275\\
469	0.00507573192480233\\
470	0.0050924126447223\\
471	0.00510957900694087\\
472	0.00512725140490503\\
473	0.00514545130482393\\
474	0.0051642012957185\\
475	0.0051835251411441\\
476	0.00520344785397583\\
477	0.00522399583307634\\
478	0.0052451969499694\\
479	0.00526708001994137\\
480	0.00528967582130181\\
481	0.00531302073050156\\
482	0.00533717128049394\\
483	0.00536217440954135\\
484	0.00538807060812187\\
485	0.00541490203822802\\
486	0.00544271156614922\\
487	0.00547154299835975\\
488	0.00550144111267412\\
489	0.00553245141674796\\
490	0.00556461982902692\\
491	0.00559799228696207\\
492	0.00563262021729175\\
493	0.00566855676452508\\
494	0.00570585375095233\\
495	0.00574456031946986\\
496	0.00578472113107927\\
497	0.00582637393627206\\
498	0.00586941332332261\\
499	0.00591367412563156\\
500	0.0059591312442232\\
501	0.00600573960009374\\
502	0.00605342869976447\\
503	0.00610209427571644\\
504	0.00615166353661207\\
505	0.00620218464799494\\
506	0.0062534348899334\\
507	0.00630517093441952\\
508	0.00635706597096826\\
509	0.00640860003944755\\
510	0.00645893456739127\\
511	0.0065044826325323\\
512	0.00654456380460978\\
513	0.00657922957752016\\
514	0.00660912384878171\\
515	0.00663781310127959\\
516	0.00666570035195224\\
517	0.00669325982984133\\
518	0.00672095204731527\\
519	0.0067490035419314\\
520	0.00677751497187324\\
521	0.00680656037357772\\
522	0.00683618266850091\\
523	0.00686640821868181\\
524	0.00689725841090671\\
525	0.0069287508654358\\
526	0.00696090155737811\\
527	0.00699372616758012\\
528	0.00702724047294954\\
529	0.00706146075026851\\
530	0.00709640408608181\\
531	0.00713208865138898\\
532	0.00716853402710913\\
533	0.00720576089941698\\
534	0.00724379067528484\\
535	0.00728265318316113\\
536	0.00732242355127311\\
537	0.00736319290804136\\
538	0.00740362713540518\\
539	0.00744294329917313\\
540	0.00748116645121112\\
541	0.00751979739156501\\
542	0.00755890721054429\\
543	0.00759857987564944\\
544	0.00763885345226277\\
545	0.00767972887409422\\
546	0.00772120151702209\\
547	0.00776326398074099\\
548	0.00780590660744546\\
549	0.00784911740722832\\
550	0.00789288179370365\\
551	0.00793718233108313\\
552	0.0079819985880279\\
553	0.00802730742623898\\
554	0.00807308398607837\\
555	0.00811930438610058\\
556	0.00816597124869282\\
557	0.00821131168944101\\
558	0.00825575379156772\\
559	0.00830045453103746\\
560	0.00834548332965024\\
561	0.00839082502862416\\
562	0.00843645437478884\\
563	0.00848234516615932\\
564	0.00852847072095564\\
565	0.00857480405939613\\
566	0.00862132528999987\\
567	0.00866764572460421\\
568	0.00871393573045342\\
569	0.00876044307747747\\
570	0.00880715274928313\\
571	0.0088540336444077\\
572	0.00890105214171166\\
573	0.00894817203305002\\
574	0.00899535448063817\\
575	0.00904255800890162\\
576	0.00908973854201715\\
577	0.00913684950113324\\
578	0.0091838419786825\\
579	0.00923066501141119\\
580	0.00927726597891579\\
581	0.00932359116078638\\
582	0.00936958649310201\\
583	0.00941519857412797\\
584	0.00946037597948012\\
585	0.00950507095777015\\
586	0.00954924158550055\\
587	0.00959285445497631\\
588	0.00963588792409171\\
589	0.00967833579817709\\
590	0.00972021084271538\\
591	0.00976138933978722\\
592	0.00980171928775089\\
593	0.00984102135019415\\
594	0.00987905595166784\\
595	0.00991543381058343\\
596	0.00994937493726701\\
597	0.00997906286423442\\
598	0.0099999191923403\\
599	0\\
600	0\\
};
\addplot [color=black!60!mycolor21,solid,forget plot]
  table[row sep=crcr]{%
1	0.00432695855327979\\
2	0.00432696069719906\\
3	0.00432696288063723\\
4	0.00432696510432514\\
5	0.00432696736900736\\
6	0.00432696967544223\\
7	0.0043269720244022\\
8	0.00432697441667397\\
9	0.004326976853059\\
10	0.00432697933437348\\
11	0.0043269818614489\\
12	0.00432698443513213\\
13	0.00432698705628577\\
14	0.00432698972578843\\
15	0.00432699244453511\\
16	0.0043269952134374\\
17	0.0043269980334239\\
18	0.00432700090544033\\
19	0.00432700383045009\\
20	0.00432700680943442\\
21	0.00432700984339293\\
22	0.00432701293334372\\
23	0.00432701608032385\\
24	0.00432701928538961\\
25	0.004327022549617\\
26	0.004327025874102\\
27	0.00432702925996095\\
28	0.00432703270833108\\
29	0.00432703622037062\\
30	0.0043270397972594\\
31	0.00432704344019931\\
32	0.0043270471504144\\
33	0.00432705092915171\\
34	0.00432705477768136\\
35	0.00432705869729714\\
36	0.00432706268931694\\
37	0.00432706675508314\\
38	0.00432707089596317\\
39	0.00432707511334992\\
40	0.00432707940866214\\
41	0.00432708378334511\\
42	0.00432708823887093\\
43	0.00432709277673917\\
44	0.00432709739847733\\
45	0.00432710210564138\\
46	0.00432710689981621\\
47	0.0043271117826163\\
48	0.00432711675568621\\
49	0.00432712182070122\\
50	0.00432712697936762\\
51	0.0043271322334237\\
52	0.00432713758464017\\
53	0.00432714303482059\\
54	0.00432714858580225\\
55	0.00432715423945664\\
56	0.00432715999769022\\
57	0.004327165862445\\
58	0.00432717183569916\\
59	0.00432717791946777\\
60	0.00432718411580356\\
61	0.0043271904267975\\
62	0.00432719685457959\\
63	0.00432720340131958\\
64	0.00432721006922769\\
65	0.00432721686055543\\
66	0.00432722377759633\\
67	0.0043272308226867\\
68	0.00432723799820648\\
69	0.00432724530658002\\
70	0.00432725275027702\\
71	0.00432726033181314\\
72	0.00432726805375125\\
73	0.00432727591870186\\
74	0.00432728392932442\\
75	0.00432729208832793\\
76	0.00432730039847212\\
77	0.0043273088625682\\
78	0.00432731748347992\\
79	0.0043273262641246\\
80	0.00432733520747408\\
81	0.00432734431655567\\
82	0.00432735359445329\\
83	0.00432736304430868\\
84	0.00432737266932216\\
85	0.00432738247275397\\
86	0.00432739245792525\\
87	0.00432740262821932\\
88	0.00432741298708276\\
89	0.00432742353802666\\
90	0.00432743428462778\\
91	0.00432744523052981\\
92	0.00432745637944468\\
93	0.00432746773515366\\
94	0.00432747930150892\\
95	0.00432749108243471\\
96	0.00432750308192877\\
97	0.00432751530406363\\
98	0.00432752775298816\\
99	0.00432754043292886\\
100	0.00432755334819144\\
101	0.00432756650316226\\
102	0.00432757990230986\\
103	0.00432759355018648\\
104	0.00432760745142971\\
105	0.00432762161076401\\
106	0.00432763603300237\\
107	0.00432765072304815\\
108	0.00432766568589645\\
109	0.00432768092663611\\
110	0.0043276964504514\\
111	0.00432771226262376\\
112	0.0043277283685338\\
113	0.00432774477366301\\
114	0.00432776148359568\\
115	0.00432777850402097\\
116	0.00432779584073471\\
117	0.00432781349964156\\
118	0.00432783148675699\\
119	0.00432784980820943\\
120	0.00432786847024234\\
121	0.00432788747921642\\
122	0.00432790684161185\\
123	0.00432792656403053\\
124	0.00432794665319832\\
125	0.00432796711596752\\
126	0.00432798795931918\\
127	0.00432800919036546\\
128	0.00432803081635228\\
129	0.00432805284466176\\
130	0.00432807528281481\\
131	0.00432809813847368\\
132	0.00432812141944487\\
133	0.00432814513368153\\
134	0.00432816928928657\\
135	0.00432819389451525\\
136	0.00432821895777821\\
137	0.00432824448764443\\
138	0.00432827049284404\\
139	0.00432829698227167\\
140	0.0043283239649894\\
141	0.00432835145022986\\
142	0.00432837944739962\\
143	0.00432840796608247\\
144	0.00432843701604282\\
145	0.00432846660722887\\
146	0.00432849674977659\\
147	0.00432852745401276\\
148	0.00432855873045901\\
149	0.0043285905898353\\
150	0.00432862304306379\\
151	0.00432865610127265\\
152	0.00432868977579992\\
153	0.00432872407819774\\
154	0.00432875902023616\\
155	0.00432879461390737\\
156	0.00432883087143009\\
157	0.00432886780525368\\
158	0.00432890542806266\\
159	0.00432894375278118\\
160	0.0043289827925775\\
161	0.0043290225608689\\
162	0.00432906307132606\\
163	0.00432910433787831\\
164	0.00432914637471825\\
165	0.00432918919630696\\
166	0.00432923281737907\\
167	0.00432927725294803\\
168	0.00432932251831144\\
169	0.0043293686290565\\
170	0.00432941560106546\\
171	0.0043294634505214\\
172	0.00432951219391403\\
173	0.00432956184804536\\
174	0.00432961243003595\\
175	0.00432966395733081\\
176	0.00432971644770586\\
177	0.00432976991927409\\
178	0.00432982439049204\\
179	0.00432987988016658\\
180	0.00432993640746147\\
181	0.00432999399190443\\
182	0.00433005265339393\\
183	0.00433011241220651\\
184	0.0043301732890039\\
185	0.00433023530484067\\
186	0.00433029848117162\\
187	0.00433036283985964\\
188	0.00433042840318345\\
189	0.00433049519384591\\
190	0.00433056323498191\\
191	0.00433063255016688\\
192	0.00433070316342541\\
193	0.00433077509923983\\
194	0.00433084838255915\\
195	0.00433092303880804\\
196	0.00433099909389616\\
197	0.0043310765742276\\
198	0.00433115550671049\\
199	0.00433123591876661\\
200	0.00433131783834173\\
201	0.00433140129391559\\
202	0.00433148631451246\\
203	0.00433157292971159\\
204	0.00433166116965828\\
205	0.00433175106507474\\
206	0.00433184264727161\\
207	0.00433193594815924\\
208	0.00433203100025962\\
209	0.00433212783671843\\
210	0.00433222649131709\\
211	0.00433232699848539\\
212	0.00433242939331433\\
213	0.00433253371156901\\
214	0.00433263998970192\\
215	0.00433274826486667\\
216	0.00433285857493167\\
217	0.00433297095849443\\
218	0.0043330854548959\\
219	0.00433320210423526\\
220	0.00433332094738493\\
221	0.00433344202600606\\
222	0.00433356538256405\\
223	0.00433369106034479\\
224	0.00433381910347088\\
225	0.00433394955691843\\
226	0.00433408246653407\\
227	0.00433421787905251\\
228	0.00433435584211419\\
229	0.00433449640428373\\
230	0.00433463961506829\\
231	0.00433478552493675\\
232	0.00433493418533899\\
233	0.00433508564872588\\
234	0.00433523996856943\\
235	0.00433539719938359\\
236	0.00433555739674543\\
237	0.00433572061731673\\
238	0.0043358869188662\\
239	0.00433605636029206\\
240	0.00433622900164514\\
241	0.00433640490415262\\
242	0.00433658413024219\\
243	0.00433676674356681\\
244	0.00433695280902993\\
245	0.00433714239281142\\
246	0.0043373355623941\\
247	0.00433753238659085\\
248	0.0043377329355722\\
249	0.00433793728089482\\
250	0.00433814549553049\\
251	0.00433835765389585\\
252	0.00433857383188282\\
253	0.00433879410688977\\
254	0.00433901855785339\\
255	0.00433924726528139\\
256	0.00433948031128586\\
257	0.00433971777961768\\
258	0.00433995975570167\\
259	0.00434020632667236\\
260	0.00434045758141111\\
261	0.0043407136105838\\
262	0.00434097450667962\\
263	0.00434124036405092\\
264	0.00434151127895385\\
265	0.00434178734959036\\
266	0.00434206867615101\\
267	0.00434235536085907\\
268	0.00434264750801582\\
269	0.00434294522404696\\
270	0.00434324861755036\\
271	0.00434355779934511\\
272	0.00434387288252188\\
273	0.00434419398249479\\
274	0.00434452121705471\\
275	0.00434485470642407\\
276	0.00434519457331332\\
277	0.00434554094297909\\
278	0.00434589394328401\\
279	0.00434625370475853\\
280	0.0043466203606644\\
281	0.00434699404706052\\
282	0.00434737490287053\\
283	0.00434776306995298\\
284	0.00434815869317363\\
285	0.00434856192048034\\
286	0.0043489729029806\\
287	0.00434939179502176\\
288	0.00434981875427422\\
289	0.00435025394181805\\
290	0.00435069752223251\\
291	0.00435114966368958\\
292	0.00435161053805094\\
293	0.00435208032096964\\
294	0.00435255919199538\\
295	0.00435304733468546\\
296	0.00435354493672011\\
297	0.00435405219002397\\
298	0.00435456929089321\\
299	0.00435509644012949\\
300	0.0043556338431809\\
301	0.00435618171029078\\
302	0.0043567402566551\\
303	0.00435730970258917\\
304	0.00435789027370451\\
305	0.0043584822010971\\
306	0.00435908572154804\\
307	0.00435970107773785\\
308	0.00436032851847588\\
309	0.0043609682989465\\
310	0.0043616206809735\\
311	0.00436228593330542\\
312	0.00436296433192296\\
313	0.00436365616037193\\
314	0.00436436171012386\\
315	0.00436508128096751\\
316	0.00436581518143451\\
317	0.00436656372926307\\
318	0.00436732725190336\\
319	0.00436810608706939\\
320	0.00436890058334199\\
321	0.00436971110082804\\
322	0.00437053801188148\\
323	0.0043713817018922\\
324	0.00437224257014867\\
325	0.0043731210307803\\
326	0.00437401751378662\\
327	0.00437493246615868\\
328	0.00437586635309825\\
329	0.0043768196593399\\
330	0.00437779289057866\\
331	0.00437878657500424\\
332	0.00437980126494006\\
333	0.00438083753858002\\
334	0.0043818960018099\\
335	0.00438297729009362\\
336	0.00438408207039105\\
337	0.0043852110430619\\
338	0.00438636494368967\\
339	0.00438754454473416\\
340	0.00438875065688568\\
341	0.00438998412993984\\
342	0.00439124585291623\\
343	0.00439253675292674\\
344	0.00439385779169215\\
345	0.00439520995665367\\
346	0.00439659423677513\\
347	0.00439800980650354\\
348	0.00439945288719747\\
349	0.00440092400728502\\
350	0.00440242370465192\\
351	0.00440395252680818\\
352	0.00440551103111611\\
353	0.00440709978509535\\
354	0.00440871936667319\\
355	0.00441037036353843\\
356	0.00441205336796628\\
357	0.00441376898130935\\
358	0.00441551781433659\\
359	0.00441730048719107\\
360	0.00441911762932339\\
361	0.00442096987939735\\
362	0.00442285788516469\\
363	0.00442478230330447\\
364	0.00442674379922321\\
365	0.0044287430468102\\
366	0.00443078072814295\\
367	0.00443285753313633\\
368	0.00443497415912887\\
369	0.00443713131039822\\
370	0.00443932969759767\\
371	0.00444157003710466\\
372	0.00444385305027066\\
373	0.00444617946256172\\
374	0.00444855000257832\\
375	0.00445096540094059\\
376	0.00445342638902633\\
377	0.00445593369754693\\
378	0.00445848805494721\\
379	0.00446109018561414\\
380	0.00446374080788017\\
381	0.00446644063180854\\
382	0.00446919035674984\\
383	0.00447199066866269\\
384	0.00447484223719717\\
385	0.00447774571254766\\
386	0.0044807017220934\\
387	0.00448371086686047\\
388	0.00448677371786047\\
389	0.00448989081239065\\
390	0.00449306265041508\\
391	0.00449628969119351\\
392	0.0044995723503721\\
393	0.00450291099778542\\
394	0.00450630595618257\\
395	0.00450975750081975\\
396	0.00451326585891238\\
397	0.00451683120515006\\
398	0.00452045364209266\\
399	0.00452413735820681\\
400	0.00452789394284761\\
401	0.00453172483387663\\
402	0.00453563150123215\\
403	0.00453961544838681\\
404	0.00454367821396767\\
405	0.00454782137356271\\
406	0.00455204654173728\\
407	0.00455635537427974\\
408	0.00456074957067161\\
409	0.00456523087672764\\
410	0.00456980108733492\\
411	0.0045744620497789\\
412	0.00457921566841893\\
413	0.00458406390985492\\
414	0.00458900880804196\\
415	0.00459405246996129\\
416	0.00459919708201783\\
417	0.00460444491720469\\
418	0.00460979834306404\\
419	0.00461525983049784\\
420	0.00462083196341752\\
421	0.00462651744911842\\
422	0.0046323191295337\\
423	0.00463823999399964\\
424	0.00464428319606497\\
425	0.00465045208049502\\
426	0.00465675024305133\\
427	0.00466318170211106\\
428	0.00466975068776097\\
429	0.00467646102269409\\
430	0.00468331626401164\\
431	0.00469032009822216\\
432	0.00469747634809923\\
433	0.00470478897990422\\
434	0.00471226211098159\\
435	0.00471990001772997\\
436	0.00472770714394326\\
437	0.00473568810951569\\
438	0.00474384771949964\\
439	0.00475219097350291\\
440	0.00476072307541075\\
441	0.0047694494434255\\
442	0.00477837572044345\\
443	0.00478750778485379\\
444	0.00479685176198382\\
445	0.0048064140365962\\
446	0.00481620126662683\\
447	0.00482622039523245\\
448	0.00483647864149911\\
449	0.004846983503795\\
450	0.00485774280977831\\
451	0.00486876475354069\\
452	0.00488005791751922\\
453	0.00489163129574941\\
454	0.00490349431852351\\
455	0.00491565687852088\\
456	0.00492812935847706\\
457	0.00494092266045717\\
458	0.00495404823679719\\
459	0.00496751812277545\\
460	0.00498134497106543\\
461	0.00499554208798379\\
462	0.00501012347143418\\
463	0.00502510385021868\\
464	0.00504049872440022\\
465	0.00505632440914279\\
466	0.00507259808091958\\
467	0.00508933782211824\\
468	0.0051065626661562\\
469	0.00512429264292472\\
470	0.00514254882450478\\
471	0.00516135337167488\\
472	0.00518072958309387\\
473	0.00520070195230599\\
474	0.00522129624785663\\
475	0.00524253966141403\\
476	0.00526446073178525\\
477	0.00528708936517332\\
478	0.00531046069907047\\
479	0.00533462727696618\\
480	0.00535963747295621\\
481	0.00538552963238027\\
482	0.00541234318243217\\
483	0.0054401178576874\\
484	0.00546889387848687\\
485	0.00549871271132115\\
486	0.00552962358923932\\
487	0.00556167658536844\\
488	0.00559492199409357\\
489	0.00562940946645967\\
490	0.00566518680759279\\
491	0.00570229826537744\\
492	0.00574057270663483\\
493	0.00577994483980594\\
494	0.00582041438361801\\
495	0.00586196977977204\\
496	0.00590458482482532\\
497	0.00594821370429102\\
498	0.0059929190006349\\
499	0.00603878586440403\\
500	0.00608571187789154\\
501	0.00613354908577968\\
502	0.00618209679919275\\
503	0.00623111975679493\\
504	0.0062803460795765\\
505	0.00632932117173365\\
506	0.00637730219787598\\
507	0.00642155840085762\\
508	0.00646045553143953\\
509	0.00649394778892431\\
510	0.00652243322939103\\
511	0.00654927054506817\\
512	0.00657526604504233\\
513	0.00660087923655145\\
514	0.00662658810284878\\
515	0.00665262797492567\\
516	0.00667909600829342\\
517	0.00670606353836837\\
518	0.00673357132505209\\
519	0.00676164372812885\\
520	0.0067903003794335\\
521	0.00681955743687604\\
522	0.00684942956662143\\
523	0.00687993114722263\\
524	0.00691107661826311\\
525	0.00694288081781127\\
526	0.00697535922344092\\
527	0.00700852813728366\\
528	0.00704240489215253\\
529	0.0070770080849942\\
530	0.00711235785255778\\
531	0.00714847517504492\\
532	0.0071853817036795\\
533	0.00722312753643244\\
534	0.00726179417464418\\
535	0.00730115804025518\\
536	0.00733950427705983\\
537	0.00737657424641395\\
538	0.007413491098651\\
539	0.00745084701001097\\
540	0.00748873240113763\\
541	0.00752721478980981\\
542	0.00756629878659511\\
543	0.00760598438697385\\
544	0.00764626809925852\\
545	0.00768714464315555\\
546	0.0077286069021184\\
547	0.00777064572670249\\
548	0.00781324967273179\\
549	0.00785640469595659\\
550	0.00790009381669526\\
551	0.00794429678697347\\
552	0.00798898986048007\\
553	0.00803414520393946\\
554	0.00807975463982383\\
555	0.00812542668900067\\
556	0.00816939338006928\\
557	0.00821329420027352\\
558	0.00825751363257453\\
559	0.00830208407653316\\
560	0.00834698368736043\\
561	0.00839218901342985\\
562	0.00843767581465922\\
563	0.00848341941476406\\
564	0.0085293949138881\\
565	0.00857558303702803\\
566	0.00862162837759405\\
567	0.00866765938100907\\
568	0.00871393573140663\\
569	0.00876044307752332\\
570	0.00880715274930379\\
571	0.00885403364441763\\
572	0.00890105214171626\\
573	0.00894817203305202\\
574	0.00899535448063901\\
575	0.00904255800890193\\
576	0.00908973854201726\\
577	0.00913684950113328\\
578	0.00918384197868252\\
579	0.00923066501141119\\
580	0.00927726597891579\\
581	0.00932359116078638\\
582	0.00936958649310202\\
583	0.00941519857412797\\
584	0.00946037597948011\\
585	0.00950507095777015\\
586	0.00954924158550055\\
587	0.00959285445497631\\
588	0.00963588792409171\\
589	0.00967833579817709\\
590	0.00972021084271538\\
591	0.00976138933978722\\
592	0.00980171928775089\\
593	0.00984102135019415\\
594	0.00987905595166784\\
595	0.00991543381058343\\
596	0.00994937493726701\\
597	0.00997906286423442\\
598	0.0099999191923403\\
599	0\\
600	0\\
};
\addplot [color=black!80!mycolor21,solid,forget plot]
  table[row sep=crcr]{%
1	0.00433439942612947\\
2	0.00433440181287078\\
3	0.00433440424353772\\
4	0.00433440671894131\\
5	0.00433440923990749\\
6	0.0043344118072775\\
7	0.00433441442190828\\
8	0.00433441708467256\\
9	0.00433441979645923\\
10	0.00433442255817373\\
11	0.00433442537073827\\
12	0.00433442823509211\\
13	0.00433443115219192\\
14	0.00433443412301212\\
15	0.0043344371485452\\
16	0.00433444022980203\\
17	0.00433444336781228\\
18	0.00433444656362468\\
19	0.00433444981830739\\
20	0.00433445313294842\\
21	0.00433445650865589\\
22	0.00433445994655854\\
23	0.00433446344780604\\
24	0.0043344670135694\\
25	0.0043344706450413\\
26	0.00433447434343657\\
27	0.00433447810999261\\
28	0.00433448194596979\\
29	0.00433448585265188\\
30	0.00433448983134642\\
31	0.0043344938833852\\
32	0.00433449801012487\\
33	0.00433450221294707\\
34	0.00433450649325921\\
35	0.00433451085249485\\
36	0.00433451529211415\\
37	0.00433451981360434\\
38	0.00433452441848027\\
39	0.00433452910828499\\
40	0.00433453388459013\\
41	0.00433453874899659\\
42	0.00433454370313495\\
43	0.00433454874866611\\
44	0.00433455388728183\\
45	0.00433455912070522\\
46	0.00433456445069154\\
47	0.00433456987902862\\
48	0.00433457540753754\\
49	0.00433458103807315\\
50	0.00433458677252486\\
51	0.00433459261281712\\
52	0.00433459856091015\\
53	0.0043346046188007\\
54	0.00433461078852263\\
55	0.00433461707214761\\
56	0.00433462347178577\\
57	0.00433462998958654\\
58	0.00433463662773934\\
59	0.0043346433884743\\
60	0.00433465027406297\\
61	0.00433465728681925\\
62	0.00433466442910008\\
63	0.00433467170330617\\
64	0.00433467911188301\\
65	0.00433468665732144\\
66	0.00433469434215876\\
67	0.00433470216897956\\
68	0.00433471014041645\\
69	0.00433471825915103\\
70	0.00433472652791482\\
71	0.00433473494949035\\
72	0.00433474352671167\\
73	0.00433475226246583\\
74	0.0043347611596935\\
75	0.00433477022139027\\
76	0.00433477945060735\\
77	0.00433478885045292\\
78	0.00433479842409299\\
79	0.00433480817475246\\
80	0.00433481810571649\\
81	0.00433482822033137\\
82	0.00433483852200576\\
83	0.00433484901421175\\
84	0.00433485970048622\\
85	0.00433487058443192\\
86	0.00433488166971885\\
87	0.00433489296008528\\
88	0.00433490445933922\\
89	0.00433491617135971\\
90	0.00433492810009815\\
91	0.00433494024957949\\
92	0.00433495262390377\\
93	0.00433496522724763\\
94	0.0043349780638655\\
95	0.00433499113809108\\
96	0.00433500445433906\\
97	0.00433501801710641\\
98	0.00433503183097401\\
99	0.00433504590060823\\
100	0.00433506023076251\\
101	0.00433507482627893\\
102	0.00433508969209002\\
103	0.0043351048332203\\
104	0.00433512025478811\\
105	0.00433513596200738\\
106	0.00433515196018933\\
107	0.00433516825474427\\
108	0.00433518485118366\\
109	0.00433520175512177\\
110	0.00433521897227784\\
111	0.00433523650847781\\
112	0.00433525436965644\\
113	0.00433527256185936\\
114	0.00433529109124517\\
115	0.00433530996408747\\
116	0.00433532918677717\\
117	0.00433534876582449\\
118	0.00433536870786138\\
119	0.00433538901964363\\
120	0.00433540970805338\\
121	0.00433543078010148\\
122	0.00433545224292965\\
123	0.00433547410381325\\
124	0.0043354963701636\\
125	0.00433551904953057\\
126	0.00433554214960529\\
127	0.00433556567822269\\
128	0.00433558964336422\\
129	0.00433561405316057\\
130	0.00433563891589457\\
131	0.00433566424000396\\
132	0.00433569003408434\\
133	0.00433571630689219\\
134	0.00433574306734774\\
135	0.00433577032453824\\
136	0.00433579808772094\\
137	0.00433582636632634\\
138	0.00433585516996155\\
139	0.00433588450841341\\
140	0.00433591439165201\\
141	0.00433594482983415\\
142	0.00433597583330677\\
143	0.00433600741261058\\
144	0.00433603957848358\\
145	0.00433607234186507\\
146	0.0043361057138991\\
147	0.00433613970593858\\
148	0.00433617432954894\\
149	0.00433620959651243\\
150	0.00433624551883205\\
151	0.00433628210873568\\
152	0.00433631937868039\\
153	0.00433635734135661\\
154	0.00433639600969271\\
155	0.0043364353968595\\
156	0.00433647551627451\\
157	0.0043365163816069\\
158	0.00433655800678218\\
159	0.00433660040598687\\
160	0.00433664359367367\\
161	0.00433668758456619\\
162	0.00433673239366435\\
163	0.00433677803624933\\
164	0.00433682452788904\\
165	0.00433687188444338\\
166	0.00433692012206996\\
167	0.00433696925722949\\
168	0.00433701930669161\\
169	0.00433707028754067\\
170	0.00433712221718179\\
171	0.0043371751133467\\
172	0.00433722899410006\\
173	0.00433728387784567\\
174	0.00433733978333297\\
175	0.00433739672966335\\
176	0.00433745473629706\\
177	0.00433751382305975\\
178	0.00433757401014956\\
179	0.00433763531814403\\
180	0.00433769776800725\\
181	0.00433776138109724\\
182	0.00433782617917334\\
183	0.00433789218440382\\
184	0.00433795941937356\\
185	0.00433802790709201\\
186	0.00433809767100114\\
187	0.00433816873498355\\
188	0.00433824112337095\\
189	0.00433831486095252\\
190	0.00433838997298366\\
191	0.00433846648519474\\
192	0.00433854442380009\\
193	0.00433862381550717\\
194	0.00433870468752586\\
195	0.00433878706757803\\
196	0.00433887098390719\\
197	0.00433895646528827\\
198	0.00433904354103794\\
199	0.00433913224102455\\
200	0.0043392225956787\\
201	0.00433931463600408\\
202	0.00433940839358797\\
203	0.0043395039006125\\
204	0.00433960118986595\\
205	0.00433970029475408\\
206	0.00433980124931195\\
207	0.00433990408821587\\
208	0.00434000884679544\\
209	0.004340115561046\\
210	0.00434022426764125\\
211	0.00434033500394618\\
212	0.00434044780802998\\
213	0.0043405627186797\\
214	0.00434067977541363\\
215	0.00434079901849514\\
216	0.00434092048894711\\
217	0.00434104422856615\\
218	0.00434117027993722\\
219	0.00434129868644876\\
220	0.00434142949230795\\
221	0.00434156274255614\\
222	0.00434169848308484\\
223	0.00434183676065168\\
224	0.00434197762289716\\
225	0.00434212111836126\\
226	0.00434226729650053\\
227	0.0043424162077056\\
228	0.00434256790331901\\
229	0.00434272243565309\\
230	0.00434287985800876\\
231	0.00434304022469402\\
232	0.00434320359104341\\
233	0.00434337001343736\\
234	0.00434353954932225\\
235	0.00434371225723068\\
236	0.00434388819680213\\
237	0.00434406742880416\\
238	0.00434425001515388\\
239	0.00434443601893982\\
240	0.00434462550444436\\
241	0.00434481853716655\\
242	0.00434501518384526\\
243	0.00434521551248277\\
244	0.00434541959236909\\
245	0.00434562749410651\\
246	0.00434583928963452\\
247	0.00434605505225551\\
248	0.00434627485666074\\
249	0.00434649877895702\\
250	0.00434672689669355\\
251	0.00434695928888968\\
252	0.0043471960360629\\
253	0.00434743722025764\\
254	0.00434768292507426\\
255	0.00434793323569891\\
256	0.004348188238934\\
257	0.00434844802322883\\
258	0.00434871267871118\\
259	0.00434898229721944\\
260	0.00434925697233537\\
261	0.00434953679941713\\
262	0.00434982187563352\\
263	0.00435011229999835\\
264	0.00435040817340566\\
265	0.00435070959866558\\
266	0.00435101668054085\\
267	0.00435132952578396\\
268	0.00435164824317491\\
269	0.00435197294355986\\
270	0.0043523037398901\\
271	0.00435264074726211\\
272	0.00435298408295787\\
273	0.00435333386648616\\
274	0.00435369021962445\\
275	0.00435405326646142\\
276	0.00435442313344009\\
277	0.00435479994940184\\
278	0.00435518384563063\\
279	0.00435557495589837\\
280	0.00435597341651037\\
281	0.00435637936635177\\
282	0.00435679294693444\\
283	0.00435721430244384\\
284	0.00435764357978722\\
285	0.00435808092864142\\
286	0.00435852650150141\\
287	0.00435898045372909\\
288	0.004359442943602\\
289	0.00435991413236213\\
290	0.00436039418426491\\
291	0.00436088326662764\\
292	0.0043613815498779\\
293	0.00436188920760098\\
294	0.00436240641658698\\
295	0.00436293335687661\\
296	0.0043634702118058\\
297	0.00436401716804851\\
298	0.00436457441565798\\
299	0.00436514214810526\\
300	0.00436572056231491\\
301	0.00436630985869759\\
302	0.00436691024117822\\
303	0.00436752191721976\\
304	0.00436814509784179\\
305	0.00436877999763203\\
306	0.00436942683475128\\
307	0.00437008583092947\\
308	0.00437075721145214\\
309	0.00437144120513579\\
310	0.00437213804429049\\
311	0.00437284796466764\\
312	0.0043735712053918\\
313	0.00437430800887281\\
314	0.00437505862069719\\
315	0.00437582328949509\\
316	0.00437660226677961\\
317	0.00437739580675559\\
318	0.00437820416609351\\
319	0.00437902760366431\\
320	0.00437986638023082\\
321	0.00438072075809017\\
322	0.0043815910006622\\
323	0.00438247737201782\\
324	0.00438338013634087\\
325	0.00438429955731751\\
326	0.00438523589744583\\
327	0.00438618941725928\\
328	0.00438716037445751\\
329	0.0043881490229384\\
330	0.0043891556117265\\
331	0.00439018038379473\\
332	0.00439122357477812\\
333	0.00439228541158204\\
334	0.00439336611089349\\
335	0.00439446587760887\\
336	0.00439558490320252\\
337	0.00439672336407297\\
338	0.00439788141991926\\
339	0.00439905921222226\\
340	0.00440025686292988\\
341	0.0044014744734705\\
342	0.00440271212422077\\
343	0.00440396987448398\\
344	0.00440524776275998\\
345	0.00440654580531069\\
346	0.00440786398489007\\
347	0.00440920402385365\\
348	0.00441057060935592\\
349	0.00441196426536201\\
350	0.00441338552577781\\
351	0.00441483493460724\\
352	0.00441631304610621\\
353	0.00441782042492295\\
354	0.0044193576462028\\
355	0.00442092529565036\\
356	0.00442252396978652\\
357	0.00442415427627943\\
358	0.00442581683414257\\
359	0.00442751227392775\\
360	0.00442924123792482\\
361	0.00443100438036846\\
362	0.00443280236765423\\
363	0.00443463587856507\\
364	0.00443650560451036\\
365	0.00443841224977974\\
366	0.00444035653181446\\
367	0.00444233918149978\\
368	0.00444436094348195\\
369	0.00444642257651406\\
370	0.00444852485383648\\
371	0.00445066856359745\\
372	0.00445285450932144\\
373	0.00445508351043292\\
374	0.00445735640284518\\
375	0.00445967403962522\\
376	0.00446203729174694\\
377	0.00446444704894753\\
378	0.00446690422070181\\
379	0.00446940973733432\\
380	0.00447196455128907\\
381	0.00447456963857968\\
382	0.00447722600044551\\
383	0.00447993466524188\\
384	0.0044826966905934\\
385	0.00448551316584321\\
386	0.0044883852148302\\
387	0.00449131399902813\\
388	0.00449430072107765\\
389	0.00449734662873975\\
390	0.00450045301929358\\
391	0.00450362124439519\\
392	0.00450685271541414\\
393	0.00451014890930527\\
394	0.00451351137525375\\
395	0.00451694174302952\\
396	0.0045204417364464\\
397	0.00452401320375389\\
398	0.00452765820524174\\
399	0.0045313789886601\\
400	0.00453517753936264\\
401	0.00453905557873385\\
402	0.0045430148741424\\
403	0.00454705724075805\\
404	0.00455118454346668\\
405	0.00455539869888527\\
406	0.00455970167748078\\
407	0.00456409550579306\\
408	0.00456858226876349\\
409	0.00457316411217185\\
410	0.00457784324519672\\
411	0.00458262194310348\\
412	0.00458750255001831\\
413	0.00459248748176176\\
414	0.00459757922875969\\
415	0.00460278035902185\\
416	0.00460809352116916\\
417	0.00461352144748655\\
418	0.00461906695697717\\
419	0.00462473295838905\\
420	0.00463052245318885\\
421	0.00463643853847783\\
422	0.0046424844098883\\
423	0.00464866336458066\\
424	0.00465497880455615\\
425	0.00466143424044806\\
426	0.00466803329447707\\
427	0.00467477969251494\\
428	0.00468167725361806\\
429	0.0046887299093657\\
430	0.00469594173033701\\
431	0.00470331693310338\\
432	0.00471085988765404\\
433	0.00471857512528687\\
434	0.00472646734699674\\
435	0.00473454143239654\\
436	0.00474280244921074\\
437	0.00475125566338333\\
438	0.00475990654984844\\
439	0.00476876080401437\\
440	0.00477782435402021\\
441	0.00478710337383024\\
442	0.00479660429724071\\
443	0.0048063338328793\\
444	0.00481629898027238\\
445	0.00482650704700604\\
446	0.00483696566688477\\
447	0.0048476828189708\\
448	0.0048586668488317\\
449	0.0048699264922354\\
450	0.00488147089918492\\
451	0.00489330965865071\\
452	0.00490545282471894\\
453	0.00491791094418666\\
454	0.00493069508562214\\
455	0.0049438168698928\\
456	0.00495728850214065\\
457	0.00497112280515963\\
458	0.00498533325409378\\
459	0.0049999340123333\\
460	0.00501493996843038\\
461	0.0050303667737904\\
462	0.00504623088081715\\
463	0.00506254958112901\\
464	0.00507934104342271\\
465	0.00509662435029209\\
466	0.0051144195332361\\
467	0.00513274760527605\\
468	0.00515163059078765\\
469	0.00517109155320189\\
470	0.00519115462429839\\
471	0.00521184504677305\\
472	0.00523318926021573\\
473	0.00525521509555144\\
474	0.00527795217275313\\
475	0.00530143265222818\\
476	0.00532569914663003\\
477	0.00535080864374956\\
478	0.00537679809147539\\
479	0.00540370560503476\\
480	0.00543157784115154\\
481	0.00546046220442261\\
482	0.00549040667370869\\
483	0.00552145904822377\\
484	0.00555366580937486\\
485	0.00558704762300578\\
486	0.00562138769391787\\
487	0.0056566986408806\\
488	0.00569298744985147\\
489	0.00573025396980975\\
490	0.00576848877383418\\
491	0.00580766931668448\\
492	0.00584796783373438\\
493	0.00588943448996365\\
494	0.0059320294334912\\
495	0.00597568959241816\\
496	0.00602032355118579\\
497	0.00606580644661143\\
498	0.00611196917249998\\
499	0.0061585824520033\\
500	0.00620541096103591\\
501	0.0062520924216836\\
502	0.00629802034333257\\
503	0.00634147450349103\\
504	0.00637971149189926\\
505	0.00641262592420366\\
506	0.00644048029382127\\
507	0.00646574725732371\\
508	0.00649010650081638\\
509	0.00651399622170918\\
510	0.00653789763179205\\
511	0.0065620801181893\\
512	0.00658665536465146\\
513	0.00661169510902583\\
514	0.006637239803336\\
515	0.00666331249094905\\
516	0.00668993163021992\\
517	0.00671711228438945\\
518	0.00674486803670351\\
519	0.00677321216013866\\
520	0.0068021579289145\\
521	0.00683171890503599\\
522	0.00686190912800818\\
523	0.00689274324833507\\
524	0.00692423666951791\\
525	0.00695640570141792\\
526	0.00698926773593825\\
527	0.00702284145500003\\
528	0.0070571470756616\\
529	0.00709220664141625\\
530	0.00712804440069717\\
531	0.00716472854050527\\
532	0.00720234880295647\\
533	0.00723985893619002\\
534	0.00727623466988971\\
535	0.00731151311131741\\
536	0.00734719353934899\\
537	0.0073833556106261\\
538	0.00742008300313997\\
539	0.00745740535755315\\
540	0.00749532669484875\\
541	0.00753384645106589\\
542	0.00757296262261381\\
543	0.00761267172446651\\
544	0.00765296869609351\\
545	0.00769384670138497\\
546	0.00773529689100766\\
547	0.00777730812572076\\
548	0.00781986665437991\\
549	0.00786295573820756\\
550	0.00790655521515773\\
551	0.00795064102098\\
552	0.00799518479699153\\
553	0.0080401859272667\\
554	0.00808464783813621\\
555	0.00812773859917388\\
556	0.00817111836314038\\
557	0.00821486580995575\\
558	0.00825897791658296\\
559	0.00830343402391313\\
560	0.00834821243243883\\
561	0.00839329070202066\\
562	0.00843864597211353\\
563	0.00848425526709758\\
564	0.00853009920226463\\
565	0.00857589049842109\\
566	0.00862164036385014\\
567	0.0086676593810865\\
568	0.00871393573141297\\
569	0.00876044307752625\\
570	0.00880715274930517\\
571	0.00885403364441827\\
572	0.00890105214171656\\
573	0.00894817203305216\\
574	0.00899535448063906\\
575	0.00904255800890195\\
576	0.00908973854201727\\
577	0.00913684950113329\\
578	0.00918384197868252\\
579	0.00923066501141119\\
580	0.00927726597891579\\
581	0.00932359116078638\\
582	0.00936958649310202\\
583	0.00941519857412798\\
584	0.00946037597948012\\
585	0.00950507095777016\\
586	0.00954924158550055\\
587	0.00959285445497631\\
588	0.00963588792409171\\
589	0.00967833579817709\\
590	0.00972021084271539\\
591	0.00976138933978722\\
592	0.00980171928775089\\
593	0.00984102135019415\\
594	0.00987905595166784\\
595	0.00991543381058343\\
596	0.00994937493726701\\
597	0.00997906286423442\\
598	0.0099999191923403\\
599	0\\
600	0\\
};
\addplot [color=black,solid,forget plot]
  table[row sep=crcr]{%
1	0.00434222595919898\\
2	0.00434222821860981\\
3	0.00434223051966135\\
4	0.00434223286312353\\
5	0.00434223524978059\\
6	0.00434223768043138\\
7	0.00434224015588943\\
8	0.00434224267698356\\
9	0.00434224524455786\\
10	0.00434224785947209\\
11	0.00434225052260194\\
12	0.00434225323483943\\
13	0.00434225599709309\\
14	0.00434225881028843\\
15	0.00434226167536802\\
16	0.00434226459329199\\
17	0.00434226756503826\\
18	0.00434227059160297\\
19	0.00434227367400076\\
20	0.00434227681326514\\
21	0.00434228001044872\\
22	0.00434228326662375\\
23	0.00434228658288232\\
24	0.00434228996033694\\
25	0.00434229340012076\\
26	0.00434229690338792\\
27	0.00434230047131408\\
28	0.0043423041050967\\
29	0.00434230780595548\\
30	0.0043423115751329\\
31	0.00434231541389449\\
32	0.0043423193235292\\
33	0.00434232330535016\\
34	0.00434232736069477\\
35	0.00434233149092533\\
36	0.00434233569742945\\
37	0.00434233998162055\\
38	0.00434234434493852\\
39	0.00434234878884976\\
40	0.00434235331484817\\
41	0.00434235792445522\\
42	0.00434236261922094\\
43	0.00434236740072407\\
44	0.00434237227057269\\
45	0.00434237723040493\\
46	0.00434238228188933\\
47	0.00434238742672547\\
48	0.00434239266664463\\
49	0.00434239800341021\\
50	0.00434240343881856\\
51	0.00434240897469947\\
52	0.00434241461291672\\
53	0.00434242035536886\\
54	0.00434242620398974\\
55	0.00434243216074926\\
56	0.00434243822765407\\
57	0.00434244440674809\\
58	0.00434245070011332\\
59	0.00434245710987066\\
60	0.00434246363818046\\
61	0.00434247028724327\\
62	0.00434247705930069\\
63	0.00434248395663615\\
64	0.00434249098157545\\
65	0.00434249813648803\\
66	0.00434250542378723\\
67	0.00434251284593148\\
68	0.00434252040542502\\
69	0.00434252810481881\\
70	0.00434253594671141\\
71	0.00434254393374964\\
72	0.00434255206862989\\
73	0.00434256035409873\\
74	0.00434256879295399\\
75	0.00434257738804563\\
76	0.00434258614227683\\
77	0.0043425950586049\\
78	0.00434260414004231\\
79	0.00434261338965775\\
80	0.00434262281057715\\
81	0.00434263240598475\\
82	0.00434264217912425\\
83	0.00434265213329983\\
84	0.00434266227187737\\
85	0.00434267259828563\\
86	0.00434268311601725\\
87	0.00434269382863009\\
88	0.00434270473974859\\
89	0.00434271585306465\\
90	0.00434272717233917\\
91	0.00434273870140327\\
92	0.00434275044415967\\
93	0.00434276240458394\\
94	0.00434277458672585\\
95	0.00434278699471084\\
96	0.00434279963274138\\
97	0.00434281250509854\\
98	0.00434282561614324\\
99	0.00434283897031795\\
100	0.00434285257214809\\
101	0.00434286642624372\\
102	0.00434288053730087\\
103	0.00434289491010356\\
104	0.00434290954952511\\
105	0.00434292446052985\\
106	0.00434293964817495\\
107	0.00434295511761211\\
108	0.00434297087408942\\
109	0.00434298692295295\\
110	0.00434300326964874\\
111	0.00434301991972472\\
112	0.00434303687883253\\
113	0.00434305415272947\\
114	0.00434307174728058\\
115	0.00434308966846045\\
116	0.00434310792235553\\
117	0.00434312651516604\\
118	0.00434314545320822\\
119	0.00434316474291647\\
120	0.00434318439084554\\
121	0.00434320440367275\\
122	0.00434322478820046\\
123	0.00434324555135824\\
124	0.00434326670020539\\
125	0.00434328824193327\\
126	0.00434331018386777\\
127	0.00434333253347188\\
128	0.00434335529834836\\
129	0.00434337848624228\\
130	0.0043434021050436\\
131	0.00434342616279007\\
132	0.00434345066766977\\
133	0.00434347562802419\\
134	0.00434350105235097\\
135	0.0043435269493068\\
136	0.00434355332771052\\
137	0.00434358019654606\\
138	0.0043436075649656\\
139	0.00434363544229271\\
140	0.0043436638380256\\
141	0.00434369276184034\\
142	0.00434372222359422\\
143	0.00434375223332918\\
144	0.00434378280127535\\
145	0.00434381393785427\\
146	0.00434384565368292\\
147	0.00434387795957702\\
148	0.00434391086655505\\
149	0.00434394438584184\\
150	0.00434397852887255\\
151	0.00434401330729659\\
152	0.00434404873298164\\
153	0.00434408481801783\\
154	0.00434412157472182\\
155	0.00434415901564095\\
156	0.00434419715355784\\
157	0.00434423600149462\\
158	0.00434427557271744\\
159	0.00434431588074112\\
160	0.00434435693933373\\
161	0.00434439876252149\\
162	0.00434444136459342\\
163	0.00434448476010633\\
164	0.00434452896389\\
165	0.00434457399105212\\
166	0.00434461985698352\\
167	0.00434466657736367\\
168	0.00434471416816581\\
169	0.00434476264566271\\
170	0.00434481202643204\\
171	0.0043448623273624\\
172	0.00434491356565887\\
173	0.00434496575884917\\
174	0.00434501892478946\\
175	0.00434507308167075\\
176	0.00434512824802498\\
177	0.00434518444273155\\
178	0.00434524168502374\\
179	0.00434529999449536\\
180	0.00434535939110761\\
181	0.00434541989519574\\
182	0.00434548152747634\\
183	0.00434554430905421\\
184	0.00434560826142982\\
185	0.00434567340650657\\
186	0.00434573976659853\\
187	0.00434580736443798\\
188	0.00434587622318318\\
189	0.00434594636642652\\
190	0.00434601781820248\\
191	0.00434609060299602\\
192	0.00434616474575091\\
193	0.00434624027187832\\
194	0.00434631720726564\\
195	0.00434639557828524\\
196	0.00434647541180373\\
197	0.00434655673519102\\
198	0.00434663957632959\\
199	0.00434672396362446\\
200	0.00434680992601267\\
201	0.00434689749297314\\
202	0.00434698669453694\\
203	0.00434707756129756\\
204	0.00434717012442129\\
205	0.00434726441565799\\
206	0.004347360467352\\
207	0.00434745831245296\\
208	0.00434755798452739\\
209	0.00434765951777001\\
210	0.00434776294701537\\
211	0.00434786830774987\\
212	0.00434797563612374\\
213	0.0043480849689634\\
214	0.00434819634378422\\
215	0.00434830979880302\\
216	0.00434842537295115\\
217	0.00434854310588775\\
218	0.00434866303801332\\
219	0.00434878521048335\\
220	0.0043489096652223\\
221	0.00434903644493781\\
222	0.00434916559313522\\
223	0.00434929715413229\\
224	0.00434943117307416\\
225	0.00434956769594852\\
226	0.00434970676960141\\
227	0.00434984844175275\\
228	0.0043499927610126\\
229	0.00435013977689738\\
230	0.0043502895398467\\
231	0.00435044210124015\\
232	0.00435059751341458\\
233	0.00435075582968171\\
234	0.00435091710434584\\
235	0.00435108139272219\\
236	0.00435124875115522\\
237	0.00435141923703743\\
238	0.00435159290882859\\
239	0.00435176982607499\\
240	0.00435195004942941\\
241	0.00435213364067103\\
242	0.00435232066272596\\
243	0.00435251117968812\\
244	0.0043527052568402\\
245	0.00435290296067521\\
246	0.00435310435891838\\
247	0.00435330952054942\\
248	0.00435351851582486\\
249	0.00435373141630114\\
250	0.00435394829485812\\
251	0.00435416922572236\\
252	0.0043543942844916\\
253	0.0043546235481589\\
254	0.00435485709513777\\
255	0.00435509500528721\\
256	0.00435533735993752\\
257	0.0043555842419162\\
258	0.00435583573557455\\
259	0.0043560919268145\\
260	0.00435635290311574\\
261	0.00435661875356367\\
262	0.00435688956887727\\
263	0.00435716544143785\\
264	0.0043574464653178\\
265	0.00435773273631016\\
266	0.00435802435195835\\
267	0.00435832141158653\\
268	0.00435862401633014\\
269	0.00435893226916712\\
270	0.00435924627494958\\
271	0.00435956614043557\\
272	0.00435989197432189\\
273	0.00436022388727688\\
274	0.00436056199197369\\
275	0.00436090640312437\\
276	0.00436125723751402\\
277	0.00436161461403568\\
278	0.00436197865372554\\
279	0.00436234947979873\\
280	0.00436272721768567\\
281	0.00436311199506868\\
282	0.00436350394191926\\
283	0.00436390319053608\\
284	0.00436430987558299\\
285	0.00436472413412836\\
286	0.00436514610568424\\
287	0.00436557593224635\\
288	0.00436601375833513\\
289	0.00436645973103691\\
290	0.00436691400004596\\
291	0.00436737671770755\\
292	0.00436784803906134\\
293	0.00436832812188603\\
294	0.0043688171267447\\
295	0.00436931521703131\\
296	0.00436982255901841\\
297	0.00437033932190591\\
298	0.00437086567787115\\
299	0.00437140180212078\\
300	0.00437194787294426\\
301	0.00437250407176894\\
302	0.00437307058321792\\
303	0.00437364759516981\\
304	0.00437423529882131\\
305	0.00437483388875333\\
306	0.0043754435630001\\
307	0.00437606452312268\\
308	0.00437669697428681\\
309	0.00437734112534621\\
310	0.00437799718893176\\
311	0.00437866538154727\\
312	0.0043793459236733\\
313	0.00438003903987961\\
314	0.00438074495894785\\
315	0.00438146391400548\\
316	0.00438219614267317\\
317	0.0043829418872268\\
318	0.00438370139477653\\
319	0.00438447491746537\\
320	0.00438526271268933\\
321	0.00438606504334291\\
322	0.00438688217809268\\
323	0.00438771439168233\\
324	0.0043885619652747\\
325	0.00438942518683345\\
326	0.00439030435155112\\
327	0.004391199762328\\
328	0.00439211173030835\\
329	0.00439304057548008\\
330	0.00439398662734469\\
331	0.00439495022566533\\
332	0.00439593172129862\\
333	0.00439693147711934\\
334	0.00439794986904292\\
335	0.00439898728715272\\
336	0.00440004413693626\\
337	0.00440112084063253\\
338	0.00440221783869206\\
339	0.00440333559135105\\
340	0.0044044745803329\\
341	0.00440563531073312\\
342	0.00440681831329135\\
343	0.00440802414769152\\
344	0.00440925340879721\\
345	0.00441050674127023\\
346	0.00441178487748705\\
347	0.00441308861159611\\
348	0.00441441859668612\\
349	0.00441577537239013\\
350	0.00441715949007091\\
351	0.00441857151312482\\
352	0.00442001201729709\\
353	0.00442148159100886\\
354	0.00442298083569901\\
355	0.00442451036618554\\
356	0.00442607081104137\\
357	0.00442766281297806\\
358	0.00442928702924466\\
359	0.00443094413204339\\
360	0.00443263480896246\\
361	0.00443435976342743\\
362	0.00443611971517158\\
363	0.00443791540072674\\
364	0.00443974757393486\\
365	0.00444161700648187\\
366	0.00444352448845459\\
367	0.00444547082892134\\
368	0.0044474568565372\\
369	0.0044494834201751\\
370	0.00445155138958286\\
371	0.00445366165606704\\
372	0.00445581513320393\\
373	0.00445801275757793\\
374	0.00446025548954678\\
375	0.00446254431403404\\
376	0.00446488024134691\\
377	0.00446726430801856\\
378	0.00446969757767323\\
379	0.00447218114191075\\
380	0.00447471612120721\\
381	0.00447730366582724\\
382	0.00447994495674243\\
383	0.0044826412065479\\
384	0.00448539366036974\\
385	0.00448820359675243\\
386	0.0044910723285143\\
387	0.00449400120355786\\
388	0.00449699160561887\\
389	0.00450004495493866\\
390	0.00450316270884218\\
391	0.00450634636220972\\
392	0.00450959744784129\\
393	0.00451291753673766\\
394	0.00451630823837839\\
395	0.00451977120114798\\
396	0.00452330811301538\\
397	0.00452692070170608\\
398	0.00453061072922442\\
399	0.00453437998069945\\
400	0.00453823027231605\\
401	0.00454216346841403\\
402	0.00454618148311681\\
403	0.00455028628203201\\
404	0.00455447988402687\\
405	0.00455876436308343\\
406	0.00456314185023713\\
407	0.004567614535605\\
408	0.00457218467050823\\
409	0.00457685456969654\\
410	0.00458162661367982\\
411	0.00458650325117443\\
412	0.0045914870016723\\
413	0.00459658045814428\\
414	0.00460178628988913\\
415	0.0046071072455413\\
416	0.00461254615625144\\
417	0.00461810593905845\\
418	0.00462378960047169\\
419	0.0046296002402883\\
420	0.00463554105567392\\
421	0.00464161534554071\\
422	0.0046478265152618\\
423	0.00465417808175562\\
424	0.00466067367894573\\
425	0.00466731706352769\\
426	0.00467411212094043\\
427	0.00468106287215399\\
428	0.00468817348186214\\
429	0.00469544826687676\\
430	0.00470289170423734\\
431	0.00471050843987783\\
432	0.00471830329789414\\
433	0.00472628129045722\\
434	0.00473444762842076\\
435	0.00474280773267571\\
436	0.00475136724630788\\
437	0.00476013204761824\\
438	0.00476910826406953\\
439	0.00477830228722649\\
440	0.0047877207887615\\
441	0.00479737073760044\\
442	0.00480725941828665\\
443	0.00481739445064306\\
444	0.00482778381081094\\
445	0.00483843585374646\\
446	0.00484935933727037\\
447	0.00486056344780547\\
448	0.00487205782786032\\
449	0.00488385260528553\\
450	0.00489595842442378\\
451	0.00490838647926189\\
452	0.00492114854865773\\
453	0.00493425703369915\\
454	0.00494772499723305\\
455	0.00496156620557198\\
456	0.00497579517234968\\
457	0.00499042720444371\\
458	0.00500547844981828\\
459	0.00502096594705182\\
460	0.005036907676203\\
461	0.00505332261052375\\
462	0.00507023076834851\\
463	0.00508765326425881\\
464	0.00510561235833505\\
465	0.00512413150199129\\
466	0.00514323537859757\\
467	0.0051629499370455\\
468	0.00518330241737538\\
469	0.00520432137194165\\
470	0.00522603670084335\\
471	0.0052484797701117\\
472	0.00527168384025851\\
473	0.00529568554045985\\
474	0.00532052975210367\\
475	0.00534628267470183\\
476	0.00537298517324241\\
477	0.00540067823671829\\
478	0.005429401877641\\
479	0.00545919466390937\\
480	0.00548981733005028\\
481	0.00552129955298564\\
482	0.00555367068432858\\
483	0.00558695979453451\\
484	0.00562119644582296\\
485	0.00565643318442142\\
486	0.00569293098925691\\
487	0.00573071750642869\\
488	0.00576981032505155\\
489	0.00581021341315327\\
490	0.00585191298560103\\
491	0.00589487354705255\\
492	0.00593902316472585\\
493	0.00598423743188394\\
494	0.00603033170296704\\
495	0.00607704681967084\\
496	0.00612402716256964\\
497	0.0061707902296417\\
498	0.00621667981092483\\
499	0.00626077931899491\\
500	0.00629986903605491\\
501	0.00633321627267928\\
502	0.00636114201732931\\
503	0.00638521509377619\\
504	0.00640821620985371\\
505	0.00643059378635421\\
506	0.00645283186455397\\
507	0.00647526889657669\\
508	0.00649805406722564\\
509	0.0065212637067607\\
510	0.00654494242954996\\
511	0.00656911485166403\\
512	0.00659379925237844\\
513	0.00661901035700388\\
514	0.00664476133373063\\
515	0.0066710650495984\\
516	0.00669793437851615\\
517	0.0067253824797126\\
518	0.00675342297778298\\
519	0.00678207008594943\\
520	0.00681133873692882\\
521	0.00684124472623264\\
522	0.00687180488058706\\
523	0.00690303726415001\\
524	0.00693496143280232\\
525	0.00696759874906755\\
526	0.00700097277136572\\
527	0.00703510972972449\\
528	0.00707003908897031\\
529	0.00710579415587524\\
530	0.00714241253002996\\
531	0.00717823387096147\\
532	0.00721273677147728\\
533	0.00724682109888564\\
534	0.00728132396461994\\
535	0.00731634289462122\\
536	0.00735194707806507\\
537	0.00738814406559645\\
538	0.00742493702546428\\
539	0.00746232678336369\\
540	0.00750031295800742\\
541	0.00753889396747831\\
542	0.00757806687614832\\
543	0.00761782721299181\\
544	0.00765816875730519\\
545	0.00769908328917873\\
546	0.00774056029745439\\
547	0.00778258663495929\\
548	0.00782514610529904\\
549	0.00786821895268939\\
550	0.00791178119219577\\
551	0.00795580361922691\\
552	0.00800025003764972\\
553	0.00804363618546738\\
554	0.00808620366337529\\
555	0.00812910677813416\\
556	0.00817240401823542\\
557	0.00821607813701697\\
558	0.00826010969053816\\
559	0.00830447822831884\\
560	0.0083491625665264\\
561	0.0083941411104164\\
562	0.00843939212066833\\
563	0.00848489356497313\\
564	0.00853045416626427\\
565	0.00857590279196591\\
566	0.0086216403638569\\
567	0.00866765938108735\\
568	0.00871393573141338\\
569	0.00876044307752645\\
570	0.00880715274930527\\
571	0.00885403364441831\\
572	0.00890105214171657\\
573	0.00894817203305217\\
574	0.00899535448063906\\
575	0.00904255800890197\\
576	0.00908973854201728\\
577	0.00913684950113329\\
578	0.00918384197868253\\
579	0.00923066501141121\\
580	0.00927726597891581\\
581	0.00932359116078639\\
582	0.00936958649310202\\
583	0.00941519857412798\\
584	0.00946037597948012\\
585	0.00950507095777015\\
586	0.00954924158550055\\
587	0.00959285445497631\\
588	0.00963588792409171\\
589	0.00967833579817709\\
590	0.00972021084271539\\
591	0.00976138933978722\\
592	0.00980171928775089\\
593	0.00984102135019415\\
594	0.00987905595166784\\
595	0.00991543381058343\\
596	0.00994937493726701\\
597	0.00997906286423442\\
598	0.0099999191923403\\
599	0\\
600	0\\
};
\end{axis}
\end{tikzpicture}% 
  \caption{Discrete Time}
\end{subfigure}\\
\vspace{1cm}
\begin{subfigure}{.45\linewidth}
  \centering
  \setlength\figureheight{\linewidth} 
  \setlength\figurewidth{\linewidth}
  \tikzsetnextfilename{dp_cts_nFPC_z8}
  % This file was created by matlab2tikz.
%
%The latest updates can be retrieved from
%  http://www.mathworks.com/matlabcentral/fileexchange/22022-matlab2tikz-matlab2tikz
%where you can also make suggestions and rate matlab2tikz.
%
\definecolor{mycolor1}{rgb}{1.00000,0.00000,1.00000}%
%
\begin{tikzpicture}

\begin{axis}[%
width=4.564in,
height=3.803in,
at={(1.067in,0.513in)},
scale only axis,
every outer x axis line/.append style={black},
every x tick label/.append style={font=\color{black}},
xmin=0,
xmax=100,
xlabel={Time},
every outer y axis line/.append style={black},
every y tick label/.append style={font=\color{black}},
ymin=0,
ymax=0.01,
ylabel={Depth $\delta$},
axis background/.style={fill=white},
title={Z=8},
axis x line*=bottom,
axis y line*=left,
legend style={legend cell align=left,align=left,draw=black}
]
\addplot [color=green,dashed,forget plot]
  table[row sep=crcr]{%
0.01	0.00879904401680774\\
0.02	0.00879894761016513\\
0.03	0.0087988511565914\\
0.04	0.00879875465606243\\
0.05	0.00879865810855404\\
0.06	0.00879856151404209\\
0.07	0.00879846487250238\\
0.08	0.00879836818391074\\
0.09	0.00879827144824294\\
0.1	0.00879817466547478\\
0.11	0.00879807783558202\\
0.12	0.00879798095854041\\
0.13	0.0087978840343257\\
0.14	0.00879778706291362\\
0.15	0.00879769004427989\\
0.16	0.0087975929784002\\
0.17	0.00879749586525025\\
0.18	0.00879739870480571\\
0.19	0.00879730149704226\\
0.2	0.00879720424193552\\
0.21	0.00879710693946116\\
0.22	0.00879700958959477\\
0.23	0.00879691219231199\\
0.24	0.00879681474758841\\
0.25	0.00879671725539961\\
0.26	0.00879661971572116\\
0.27	0.00879652212852861\\
0.28	0.00879642449379753\\
0.29	0.00879632681150343\\
0.3	0.00879622908162183\\
0.31	0.00879613130412824\\
0.32	0.00879603347899816\\
0.33	0.00879593560620705\\
0.34	0.00879583768573038\\
0.35	0.00879573971754362\\
0.36	0.00879564170162219\\
0.37	0.00879554363794152\\
0.38	0.00879544552647702\\
0.39	0.0087953473672041\\
0.4	0.00879524916009813\\
0.41	0.00879515090513448\\
0.42	0.00879505260228853\\
0.43	0.00879495425153561\\
0.44	0.00879485585285105\\
0.45	0.00879475740621017\\
0.46	0.00879465891158828\\
0.47	0.00879456036896068\\
0.48	0.00879446177830263\\
0.49	0.0087943631395894\\
0.5	0.00879426445279625\\
0.51	0.00879416571789841\\
0.52	0.00879406693487111\\
0.53	0.00879396810368956\\
0.54	0.00879386922432896\\
0.55	0.00879377029676448\\
0.56	0.00879367132097131\\
0.57	0.0087935722969246\\
0.58	0.00879347322459949\\
0.59	0.00879337410397112\\
0.6	0.0087932749350146\\
0.61	0.00879317571770504\\
0.62	0.00879307645201752\\
0.63	0.00879297713792712\\
0.64	0.00879287777540891\\
0.65	0.00879277836443793\\
0.66	0.00879267890498922\\
0.67	0.0087925793970378\\
0.68	0.00879247984055869\\
0.69	0.00879238023552687\\
0.7	0.00879228058191732\\
0.71	0.00879218087970502\\
0.72	0.00879208112886493\\
0.73	0.00879198132937197\\
0.74	0.00879188148120108\\
0.75	0.00879178158432718\\
0.76	0.00879168163872515\\
0.77	0.00879158164436989\\
0.78	0.00879148160123627\\
0.79	0.00879138150929915\\
0.8	0.00879128136853338\\
0.81	0.00879118117891378\\
0.82	0.00879108094041517\\
0.83	0.00879098065301236\\
0.84	0.00879088031668014\\
0.85	0.00879077993139329\\
0.86	0.00879067949712657\\
0.87	0.00879057901385472\\
0.88	0.00879047848155248\\
0.89	0.00879037790019458\\
0.9	0.00879027726975573\\
0.91	0.00879017659021061\\
0.92	0.00879007586153391\\
0.93	0.00878997508370029\\
0.94	0.00878987425668441\\
0.95	0.00878977338046091\\
0.96	0.00878967245500441\\
0.97	0.00878957148028952\\
0.98	0.00878947045629085\\
0.99	0.00878936938298296\\
1	0.00878926826034045\\
1.01	0.00878916708833785\\
1.02	0.00878906586694972\\
1.03	0.00878896459615058\\
1.04	0.00878886327591495\\
1.05	0.00878876190621732\\
1.06	0.00878866048703218\\
1.07	0.00878855901833402\\
1.08	0.00878845750009728\\
1.09	0.00878835593229641\\
1.1	0.00878825431490583\\
1.11	0.00878815264789998\\
1.12	0.00878805093125325\\
1.13	0.00878794916494002\\
1.14	0.00878784734893469\\
1.15	0.0087877454832116\\
1.16	0.0087876435677451\\
1.17	0.00878754160250953\\
1.18	0.00878743958747921\\
1.19	0.00878733752262843\\
1.2	0.0087872354079315\\
1.21	0.00878713324336269\\
1.22	0.00878703102889626\\
1.23	0.00878692876450645\\
1.24	0.00878682645016752\\
1.25	0.00878672408585367\\
1.26	0.00878662167153911\\
1.27	0.00878651920719804\\
1.28	0.00878641669280464\\
1.29	0.00878631412833306\\
1.3	0.00878621151375746\\
1.31	0.00878610884905198\\
1.32	0.00878600613419073\\
1.33	0.00878590336914782\\
1.34	0.00878580055389736\\
1.35	0.00878569768841341\\
1.36	0.00878559477267004\\
1.37	0.00878549180664131\\
1.38	0.00878538879030125\\
1.39	0.00878528572362388\\
1.4	0.00878518260658322\\
1.41	0.00878507943915325\\
1.42	0.00878497622130796\\
1.43	0.00878487295302131\\
1.44	0.00878476963426725\\
1.45	0.00878466626501972\\
1.46	0.00878456284525264\\
1.47	0.00878445937493993\\
1.48	0.00878435585405547\\
1.49	0.00878425228257315\\
1.5	0.00878414866046683\\
1.51	0.00878404498771036\\
1.52	0.00878394126427758\\
1.53	0.00878383749014231\\
1.54	0.00878373366527837\\
1.55	0.00878362978965954\\
1.56	0.0087835258632596\\
1.57	0.00878342188605233\\
1.58	0.00878331785801147\\
1.59	0.00878321377911075\\
1.6	0.0087831096493239\\
1.61	0.00878300546862462\\
1.62	0.00878290123698662\\
1.63	0.00878279695438357\\
1.64	0.00878269262078912\\
1.65	0.00878258823617694\\
1.66	0.00878248380052066\\
1.67	0.0087823793137939\\
1.68	0.00878227477597025\\
1.69	0.00878217018702333\\
1.7	0.00878206554692671\\
1.71	0.00878196085565394\\
1.72	0.00878185611317857\\
1.73	0.00878175131947414\\
1.74	0.00878164647451417\\
1.75	0.00878154157827216\\
1.76	0.0087814366307216\\
1.77	0.00878133163183597\\
1.78	0.00878122658158873\\
1.79	0.00878112147995332\\
1.8	0.00878101632690319\\
1.81	0.00878091112241173\\
1.82	0.00878080586645236\\
1.83	0.00878070055899846\\
1.84	0.00878059520002341\\
1.85	0.00878048978950057\\
1.86	0.00878038432740327\\
1.87	0.00878027881370485\\
1.88	0.00878017324837863\\
1.89	0.0087800676313979\\
1.9	0.00877996196273594\\
1.91	0.00877985624236603\\
1.92	0.00877975047026142\\
1.93	0.00877964464639535\\
1.94	0.00877953877074106\\
1.95	0.00877943284327174\\
1.96	0.0087793268639606\\
1.97	0.00877922083278082\\
1.98	0.00877911474970557\\
1.99	0.00877900861470799\\
2	0.00877890242776123\\
2.01	0.0087787961888384\\
2.02	0.00877868989791262\\
2.03	0.00877858355495698\\
2.04	0.00877847715994455\\
2.05	0.00877837071284839\\
2.06	0.00877826421364157\\
2.07	0.00877815766229711\\
2.08	0.00877805105878802\\
2.09	0.00877794440308731\\
2.1	0.00877783769516798\\
2.11	0.00877773093500298\\
2.12	0.00877762412256529\\
2.13	0.00877751725782784\\
2.14	0.00877741034076357\\
2.15	0.00877730337134539\\
2.16	0.00877719634954619\\
2.17	0.00877708927533886\\
2.18	0.00877698214869627\\
2.19	0.00877687496959127\\
2.2	0.00877676773799671\\
2.21	0.0087766604538854\\
2.22	0.00877655311723016\\
2.23	0.00877644572800377\\
2.24	0.00877633828617902\\
2.25	0.00877623079172867\\
2.26	0.00877612324462546\\
2.27	0.00877601564484214\\
2.28	0.00877590799235141\\
2.29	0.00877580028712598\\
2.3	0.00877569252913855\\
2.31	0.00877558471836178\\
2.32	0.00877547685476832\\
2.33	0.00877536893833084\\
2.34	0.00877526096902194\\
2.35	0.00877515294681424\\
2.36	0.00877504487168034\\
2.37	0.00877493674359283\\
2.38	0.00877482856252425\\
2.39	0.00877472032844717\\
2.4	0.00877461204133413\\
2.41	0.00877450370115764\\
2.42	0.00877439530789021\\
2.43	0.00877428686150433\\
2.44	0.00877417836197248\\
2.45	0.00877406980926711\\
2.46	0.00877396120336067\\
2.47	0.00877385254422558\\
2.48	0.00877374383183426\\
2.49	0.00877363506615912\\
2.5	0.00877352624717252\\
2.51	0.00877341737484684\\
2.52	0.00877330844915444\\
2.53	0.00877319947006764\\
2.54	0.00877309043755877\\
2.55	0.00877298135160014\\
2.56	0.00877287221216403\\
2.57	0.00877276301922273\\
2.58	0.00877265377274848\\
2.59	0.00877254447271354\\
2.6	0.00877243511909013\\
2.61	0.00877232571185047\\
2.62	0.00877221625096675\\
2.63	0.00877210673641116\\
2.64	0.00877199716815586\\
2.65	0.008771887546173\\
2.66	0.00877177787043473\\
2.67	0.00877166814091315\\
2.68	0.00877155835758038\\
2.69	0.0087714485204085\\
2.7	0.0087713386293696\\
2.71	0.00877122868443571\\
2.72	0.0087711186855789\\
2.73	0.00877100863277117\\
2.74	0.00877089852598456\\
2.75	0.00877078836519103\\
2.76	0.0087706781503626\\
2.77	0.0087705678814712\\
2.78	0.0087704575584888\\
2.79	0.00877034718138732\\
2.8	0.00877023675013868\\
2.81	0.00877012626471479\\
2.82	0.00877001572508753\\
2.83	0.00876990513122876\\
2.84	0.00876979448311034\\
2.85	0.00876968378070412\\
2.86	0.00876957302398191\\
2.87	0.00876946221291552\\
2.88	0.00876935134747674\\
2.89	0.00876924042763735\\
2.9	0.0087691294533691\\
2.91	0.00876901842464374\\
2.92	0.008768907341433\\
2.93	0.00876879620370859\\
2.94	0.0087686850114422\\
2.95	0.00876857376460552\\
2.96	0.00876846246317021\\
2.97	0.00876835110710792\\
2.98	0.00876823969639027\\
2.99	0.00876812823098889\\
3	0.00876801671087538\\
3.01	0.00876790513602132\\
3.02	0.00876779350639828\\
3.03	0.00876768182197782\\
3.04	0.00876757008273146\\
3.05	0.00876745828863074\\
3.06	0.00876734643964715\\
3.07	0.00876723453575219\\
3.08	0.00876712257691732\\
3.09	0.008767010563114\\
3.1	0.00876689849431368\\
3.11	0.00876678637048777\\
3.12	0.0087666741916077\\
3.13	0.00876656195764485\\
3.14	0.00876644966857059\\
3.15	0.00876633732435628\\
3.16	0.00876622492497328\\
3.17	0.0087661124703929\\
3.18	0.00876599996058646\\
3.19	0.00876588739552526\\
3.2	0.00876577477518057\\
3.21	0.00876566209952366\\
3.22	0.00876554936852578\\
3.23	0.00876543658215814\\
3.24	0.00876532374039198\\
3.25	0.00876521084319849\\
3.26	0.00876509789054885\\
3.27	0.00876498488241422\\
3.28	0.00876487181876576\\
3.29	0.0087647586995746\\
3.3	0.00876464552481185\\
3.31	0.00876453229444862\\
3.32	0.00876441900845599\\
3.33	0.00876430566680503\\
3.34	0.00876419226946679\\
3.35	0.0087640788164123\\
3.36	0.00876396530761259\\
3.37	0.00876385174303865\\
3.38	0.00876373812266147\\
3.39	0.00876362444645203\\
3.4	0.00876351071438127\\
3.41	0.00876339692642013\\
3.42	0.00876328308253954\\
3.43	0.00876316918271038\\
3.44	0.00876305522690357\\
3.45	0.00876294121508995\\
3.46	0.0087628271472404\\
3.47	0.00876271302332574\\
3.48	0.0087625988433168\\
3.49	0.00876248460718437\\
3.5	0.00876237031489927\\
3.51	0.00876225596643224\\
3.52	0.00876214156175405\\
3.53	0.00876202710083544\\
3.54	0.00876191258364712\\
3.55	0.00876179801015981\\
3.56	0.00876168338034418\\
3.57	0.00876156869417092\\
3.58	0.00876145395161068\\
3.59	0.00876133915263409\\
3.6	0.00876122429721178\\
3.61	0.00876110938531435\\
3.62	0.00876099441691239\\
3.63	0.00876087939197648\\
3.64	0.00876076431047716\\
3.65	0.00876064917238499\\
3.66	0.00876053397767047\\
3.67	0.00876041872630411\\
3.68	0.0087603034182564\\
3.69	0.00876018805349781\\
3.7	0.0087600726319988\\
3.71	0.00875995715372981\\
3.72	0.00875984161866124\\
3.73	0.00875972602676351\\
3.74	0.00875961037800701\\
3.75	0.0087594946723621\\
3.76	0.00875937890979913\\
3.77	0.00875926309028845\\
3.78	0.00875914721380037\\
3.79	0.00875903128030519\\
3.8	0.0087589152897732\\
3.81	0.00875879924217467\\
3.82	0.00875868313747984\\
3.83	0.00875856697565896\\
3.84	0.00875845075668224\\
3.85	0.00875833448051988\\
3.86	0.00875821814714206\\
3.87	0.00875810175651894\\
3.88	0.00875798530862069\\
3.89	0.00875786880341743\\
3.9	0.00875775224087926\\
3.91	0.00875763562097631\\
3.92	0.00875751894367863\\
3.93	0.0087574022089563\\
3.94	0.00875728541677936\\
3.95	0.00875716856711784\\
3.96	0.00875705165994176\\
3.97	0.0087569346952211\\
3.98	0.00875681767292585\\
3.99	0.00875670059302596\\
4	0.00875658345549138\\
4.01	0.00875646626029204\\
4.02	0.00875634900739784\\
4.03	0.00875623169677867\\
4.04	0.0087561143284044\\
4.05	0.00875599690224491\\
4.06	0.00875587941827001\\
4.07	0.00875576187644954\\
4.08	0.0087556442767533\\
4.09	0.00875552661915108\\
4.1	0.00875540890361264\\
4.11	0.00875529113010774\\
4.12	0.00875517329860612\\
4.13	0.00875505540907749\\
4.14	0.00875493746149155\\
4.15	0.00875481945581798\\
4.16	0.00875470139202645\\
4.17	0.00875458327008661\\
4.18	0.00875446508996809\\
4.19	0.00875434685164049\\
4.2	0.00875422855507342\\
4.21	0.00875411020023646\\
4.22	0.00875399178709915\\
4.23	0.00875387331563105\\
4.24	0.00875375478580168\\
4.25	0.00875363619758055\\
4.26	0.00875351755093715\\
4.27	0.00875339884584094\\
4.28	0.00875328008226138\\
4.29	0.00875316126016791\\
4.3	0.00875304237952995\\
4.31	0.00875292344031691\\
4.32	0.00875280444249815\\
4.33	0.00875268538604305\\
4.34	0.00875256627092096\\
4.35	0.0087524470971012\\
4.36	0.00875232786455309\\
4.37	0.00875220857324594\\
4.38	0.008752089223149\\
4.39	0.00875196981423154\\
4.4	0.00875185034646281\\
4.41	0.00875173081981202\\
4.42	0.00875161123424838\\
4.43	0.00875149158974108\\
4.44	0.00875137188625928\\
4.45	0.00875125212377215\\
4.46	0.00875113230224881\\
4.47	0.00875101242165839\\
4.48	0.00875089248196997\\
4.49	0.00875077248315263\\
4.5	0.00875065242517546\\
4.51	0.00875053230800747\\
4.52	0.00875041213161771\\
4.53	0.00875029189597518\\
4.54	0.00875017160104886\\
4.55	0.00875005124680773\\
4.56	0.00874993083322075\\
4.57	0.00874981036025684\\
4.58	0.00874968982788494\\
4.59	0.00874956923607394\\
4.6	0.00874944858479272\\
4.61	0.00874932787401013\\
4.62	0.00874920710369504\\
4.63	0.00874908627381625\\
4.64	0.00874896538434259\\
4.65	0.00874884443524284\\
4.66	0.00874872342648578\\
4.67	0.00874860235804016\\
4.68	0.00874848122987471\\
4.69	0.00874836004195815\\
4.7	0.00874823879425919\\
4.71	0.0087481174867465\\
4.72	0.00874799611938874\\
4.73	0.00874787469215457\\
4.74	0.0087477532050126\\
4.75	0.00874763165793143\\
4.76	0.00874751005087967\\
4.77	0.00874738838382588\\
4.78	0.00874726665673861\\
4.79	0.0087471448695864\\
4.8	0.00874702302233776\\
4.81	0.00874690111496118\\
4.82	0.00874677914742515\\
4.83	0.00874665711969813\\
4.84	0.00874653503174856\\
4.85	0.00874641288354485\\
4.86	0.00874629067505542\\
4.87	0.00874616840624865\\
4.88	0.0087460460770929\\
4.89	0.00874592368755652\\
4.9	0.00874580123760785\\
4.91	0.00874567872721519\\
4.92	0.00874555615634684\\
4.93	0.00874543352497107\\
4.94	0.00874531083305613\\
4.95	0.00874518808057025\\
4.96	0.00874506526748167\\
4.97	0.00874494239375858\\
4.98	0.00874481945936915\\
4.99	0.00874469646428155\\
5	0.00874457340846392\\
5.01	0.00874445029188437\\
5.02	0.00874432711451103\\
5.03	0.00874420387631197\\
5.04	0.00874408057725526\\
5.05	0.00874395721730895\\
5.06	0.00874383379644106\\
5.07	0.00874371031461961\\
5.08	0.00874358677181259\\
5.09	0.00874346316798797\\
5.1	0.0087433395031137\\
5.11	0.00874321577715772\\
5.12	0.00874309199008795\\
5.13	0.00874296814187227\\
5.14	0.00874284423247857\\
5.15	0.00874272026187469\\
5.16	0.0087425962300285\\
5.17	0.00874247213690779\\
5.18	0.00874234798248038\\
5.19	0.00874222376671405\\
5.2	0.00874209948957655\\
5.21	0.00874197515103564\\
5.22	0.00874185075105902\\
5.23	0.00874172628961442\\
5.24	0.00874160176666952\\
5.25	0.00874147718219198\\
5.26	0.00874135253614944\\
5.27	0.00874122782850954\\
5.28	0.00874110305923989\\
5.29	0.00874097822830808\\
5.3	0.00874085333568167\\
5.31	0.00874072838132822\\
5.32	0.00874060336521525\\
5.33	0.00874047828731028\\
5.34	0.0087403531475808\\
5.35	0.00874022794599429\\
5.36	0.0087401026825182\\
5.37	0.00873997735711995\\
5.38	0.00873985196976697\\
5.39	0.00873972652042665\\
5.4	0.00873960100906637\\
5.41	0.00873947543565349\\
5.42	0.00873934980015533\\
5.43	0.00873922410253923\\
5.44	0.00873909834277246\\
5.45	0.00873897252082232\\
5.46	0.00873884663665605\\
5.47	0.00873872069024091\\
5.48	0.0087385946815441\\
5.49	0.00873846861053283\\
5.5	0.00873834247717428\\
5.51	0.0087382162814356\\
5.52	0.00873809002328394\\
5.53	0.00873796370268641\\
5.54	0.00873783731961012\\
5.55	0.00873771087402214\\
5.56	0.00873758436588955\\
5.57	0.00873745779517937\\
5.58	0.00873733116185863\\
5.59	0.00873720446589434\\
5.6	0.00873707770725346\\
5.61	0.00873695088590297\\
5.62	0.0087368240018098\\
5.63	0.00873669705494088\\
5.64	0.00873657004526311\\
5.65	0.00873644297274336\\
5.66	0.00873631583734851\\
5.67	0.00873618863904539\\
5.68	0.00873606137780083\\
5.69	0.00873593405358162\\
5.7	0.00873580666635455\\
5.71	0.00873567921608638\\
5.72	0.00873555170274385\\
5.73	0.00873542412629367\\
5.74	0.00873529648670256\\
5.75	0.00873516878393719\\
5.76	0.00873504101796422\\
5.77	0.0087349131887503\\
5.78	0.00873478529626204\\
5.79	0.00873465734046604\\
5.8	0.00873452932132888\\
5.81	0.00873440123881712\\
5.82	0.0087342730928973\\
5.83	0.00873414488353594\\
5.84	0.00873401661069953\\
5.85	0.00873388827435456\\
5.86	0.00873375987446747\\
5.87	0.00873363141100472\\
5.88	0.0087335028839327\\
5.89	0.00873337429321783\\
5.9	0.00873324563882647\\
5.91	0.00873311692072498\\
5.92	0.0087329881388797\\
5.93	0.00873285929325693\\
5.94	0.00873273038382298\\
5.95	0.00873260141054411\\
5.96	0.00873247237338657\\
5.97	0.00873234327231661\\
5.98	0.00873221410730042\\
5.99	0.0087320848783042\\
6	0.00873195558529411\\
6.01	0.00873182622823632\\
6.02	0.00873169680709694\\
6.03	0.00873156732184208\\
6.04	0.00873143777243783\\
6.05	0.00873130815885025\\
6.06	0.00873117848104538\\
6.07	0.00873104873898927\\
6.08	0.00873091893264789\\
6.09	0.00873078906198724\\
6.1	0.00873065912697328\\
6.11	0.00873052912757194\\
6.12	0.00873039906374916\\
6.13	0.00873026893547083\\
6.14	0.00873013874270282\\
6.15	0.00873000848541098\\
6.16	0.00872987816356116\\
6.17	0.00872974777711918\\
6.18	0.00872961732605082\\
6.19	0.00872948681032185\\
6.2	0.00872935622989803\\
6.21	0.00872922558474508\\
6.22	0.00872909487482872\\
6.23	0.00872896410011464\\
6.24	0.0087288332605685\\
6.25	0.00872870235615594\\
6.26	0.00872857138684258\\
6.27	0.00872844035259404\\
6.28	0.00872830925337589\\
6.29	0.0087281780891537\\
6.3	0.00872804685989299\\
6.31	0.00872791556555929\\
6.32	0.0087277842061181\\
6.33	0.00872765278153488\\
6.34	0.0087275212917751\\
6.35	0.00872738973680418\\
6.36	0.00872725811658753\\
6.37	0.00872712643109055\\
6.38	0.00872699468027859\\
6.39	0.00872686286411701\\
6.4	0.00872673098257113\\
6.41	0.00872659903560624\\
6.42	0.00872646702318764\\
6.43	0.00872633494528059\\
6.44	0.00872620280185031\\
6.45	0.00872607059286202\\
6.46	0.00872593831828093\\
6.47	0.00872580597807219\\
6.48	0.00872567357220097\\
6.49	0.00872554110063238\\
6.5	0.00872540856333155\\
6.51	0.00872527596026355\\
6.52	0.00872514329139344\\
6.53	0.00872501055668628\\
6.54	0.00872487775610707\\
6.55	0.00872474488962082\\
6.56	0.0087246119571925\\
6.57	0.00872447895878707\\
6.58	0.00872434589436945\\
6.59	0.00872421276390456\\
6.6	0.00872407956735729\\
6.61	0.0087239463046925\\
6.62	0.00872381297587503\\
6.63	0.00872367958086971\\
6.64	0.00872354611964134\\
6.65	0.00872341259215469\\
6.66	0.00872327899837452\\
6.67	0.00872314533826556\\
6.68	0.00872301161179252\\
6.69	0.0087228778189201\\
6.7	0.00872274395961296\\
6.71	0.00872261003383574\\
6.72	0.00872247604155306\\
6.73	0.00872234198272953\\
6.74	0.00872220785732971\\
6.75	0.00872207366531817\\
6.76	0.00872193940665944\\
6.77	0.00872180508131803\\
6.78	0.00872167068925842\\
6.79	0.00872153623044509\\
6.8	0.00872140170484247\\
6.81	0.00872126711241497\\
6.82	0.00872113245312701\\
6.83	0.00872099772694295\\
6.84	0.00872086293382715\\
6.85	0.00872072807374393\\
6.86	0.00872059314665761\\
6.87	0.00872045815253246\\
6.88	0.00872032309133275\\
6.89	0.00872018796302272\\
6.9	0.00872005276756659\\
6.91	0.00871991750492854\\
6.92	0.00871978217507275\\
6.93	0.00871964677796335\\
6.94	0.00871951131356449\\
6.95	0.00871937578184026\\
6.96	0.00871924018275474\\
6.97	0.00871910451627198\\
6.98	0.00871896878235603\\
6.99	0.00871883298097089\\
7	0.00871869711208053\\
7.01	0.00871856117564894\\
7.02	0.00871842517164005\\
7.03	0.00871828910001778\\
7.04	0.00871815296074603\\
7.05	0.00871801675378865\\
7.06	0.00871788047910951\\
7.07	0.00871774413667243\\
7.08	0.00871760772644121\\
7.09	0.00871747124837963\\
7.1	0.00871733470245145\\
7.11	0.00871719808862039\\
7.12	0.00871706140685016\\
7.13	0.00871692465710446\\
7.14	0.00871678783934695\\
7.15	0.00871665095354126\\
7.16	0.008716513999651\\
7.17	0.00871637697763978\\
7.18	0.00871623988747115\\
7.19	0.00871610272910867\\
7.2	0.00871596550251586\\
7.21	0.0087158282076562\\
7.22	0.00871569084449318\\
7.23	0.00871555341299026\\
7.24	0.00871541591311084\\
7.25	0.00871527834481835\\
7.26	0.00871514070807615\\
7.27	0.0087150030028476\\
7.28	0.00871486522909604\\
7.29	0.00871472738678478\\
7.3	0.0087145894758771\\
7.31	0.00871445149633626\\
7.32	0.00871431344812549\\
7.33	0.00871417533120801\\
7.34	0.00871403714554701\\
7.35	0.00871389889110566\\
7.36	0.00871376056784709\\
7.37	0.00871362217573443\\
7.38	0.00871348371473076\\
7.39	0.00871334518479915\\
7.4	0.00871320658590265\\
7.41	0.00871306791800428\\
7.42	0.00871292918106705\\
7.43	0.0087127903750539\\
7.44	0.00871265149992781\\
7.45	0.00871251255565169\\
7.46	0.00871237354218844\\
7.47	0.00871223445950095\\
7.48	0.00871209530755205\\
7.49	0.00871195608630458\\
7.5	0.00871181679572134\\
7.51	0.00871167743576511\\
7.52	0.00871153800639865\\
7.53	0.00871139850758467\\
7.54	0.00871125893928589\\
7.55	0.00871111930146498\\
7.56	0.00871097959408462\\
7.57	0.00871083981710741\\
7.58	0.00871069997049598\\
7.59	0.0087105600542129\\
7.6	0.00871042006822073\\
7.61	0.00871028001248201\\
7.62	0.00871013988695924\\
7.63	0.0087099996916149\\
7.64	0.00870985942641146\\
7.65	0.00870971909131134\\
7.66	0.00870957868627696\\
7.67	0.00870943821127071\\
7.68	0.00870929766625493\\
7.69	0.00870915705119197\\
7.7	0.00870901636604413\\
7.71	0.00870887561077371\\
7.72	0.00870873478534295\\
7.73	0.0087085938897141\\
7.74	0.00870845292384935\\
7.75	0.0087083118877109\\
7.76	0.00870817078126091\\
7.77	0.00870802960446151\\
7.78	0.00870788835727481\\
7.79	0.00870774703966289\\
7.8	0.00870760565158781\\
7.81	0.0087074641930116\\
7.82	0.00870732266389628\\
7.83	0.00870718106420382\\
7.84	0.00870703939389618\\
7.85	0.0087068976529353\\
7.86	0.00870675584128307\\
7.87	0.00870661395890138\\
7.88	0.00870647200575209\\
7.89	0.00870632998179702\\
7.9	0.00870618788699798\\
7.91	0.00870604572131675\\
7.92	0.00870590348471508\\
7.93	0.0087057611771547\\
7.94	0.00870561879859731\\
7.95	0.00870547634900459\\
7.96	0.00870533382833818\\
7.97	0.00870519123655971\\
7.98	0.00870504857363078\\
7.99	0.00870490583951297\\
8	0.00870476303416781\\
8.01	0.00870462015755683\\
8.02	0.00870447720964153\\
8.03	0.00870433419038337\\
8.04	0.00870419109974381\\
8.05	0.00870404793768424\\
8.06	0.00870390470416608\\
8.07	0.00870376139915068\\
8.08	0.00870361802259937\\
8.09	0.00870347457447348\\
8.1	0.00870333105473429\\
8.11	0.00870318746334306\\
8.12	0.00870304380026103\\
8.13	0.0087029000654494\\
8.14	0.00870275625886935\\
8.15	0.00870261238048204\\
8.16	0.0087024684302486\\
8.17	0.00870232440813013\\
8.18	0.00870218031408771\\
8.19	0.00870203614808238\\
8.2	0.00870189191007517\\
8.21	0.00870174760002707\\
8.22	0.00870160321789906\\
8.23	0.00870145876365208\\
8.24	0.00870131423724704\\
8.25	0.00870116963864484\\
8.26	0.00870102496780634\\
8.27	0.00870088022469237\\
8.28	0.00870073540926374\\
8.29	0.00870059052148124\\
8.3	0.00870044556130563\\
8.31	0.00870030052869762\\
8.32	0.00870015542361794\\
8.33	0.00870001024602724\\
8.34	0.00869986499588618\\
8.35	0.00869971967315538\\
8.36	0.00869957427779542\\
8.37	0.00869942880976689\\
8.38	0.00869928326903032\\
8.39	0.00869913765554622\\
8.4	0.00869899196927508\\
8.41	0.00869884621017736\\
8.42	0.00869870037821349\\
8.43	0.00869855447334386\\
8.44	0.00869840849552887\\
8.45	0.00869826244472885\\
8.46	0.00869811632090414\\
8.47	0.00869797012401501\\
8.48	0.00869782385402175\\
8.49	0.00869767751088459\\
8.5	0.00869753109456373\\
8.51	0.00869738460501938\\
8.52	0.00869723804221168\\
8.53	0.00869709140610076\\
8.54	0.00869694469664672\\
8.55	0.00869679791380964\\
8.56	0.00869665105754956\\
8.57	0.00869650412782651\\
8.58	0.00869635712460047\\
8.59	0.00869621004783139\\
8.6	0.00869606289747922\\
8.61	0.00869591567350387\\
8.62	0.00869576837586522\\
8.63	0.0086956210045231\\
8.64	0.00869547355943735\\
8.65	0.00869532604056776\\
8.66	0.00869517844787409\\
8.67	0.00869503078131609\\
8.68	0.00869488304085346\\
8.69	0.00869473522644589\\
8.7	0.00869458733805304\\
8.71	0.00869443937563451\\
8.72	0.00869429133914992\\
8.73	0.00869414322855883\\
8.74	0.00869399504382078\\
8.75	0.00869384678489528\\
8.76	0.00869369845174182\\
8.77	0.00869355004431984\\
8.78	0.00869340156258879\\
8.79	0.00869325300650804\\
8.8	0.00869310437603698\\
8.81	0.00869295567113494\\
8.82	0.00869280689176123\\
8.83	0.00869265803787514\\
8.84	0.00869250910943592\\
8.85	0.00869236010640279\\
8.86	0.00869221102873496\\
8.87	0.00869206187639158\\
8.88	0.0086919126493318\\
8.89	0.00869176334751473\\
8.9	0.00869161397089945\\
8.91	0.008691464519445\\
8.92	0.00869131499311041\\
8.93	0.00869116539185468\\
8.94	0.00869101571563677\\
8.95	0.00869086596441561\\
8.96	0.00869071613815011\\
8.97	0.00869056623679913\\
8.98	0.00869041626032154\\
8.99	0.00869026620867615\\
9	0.00869011608182174\\
9.01	0.00868996587971709\\
9.02	0.0086898156023209\\
9.03	0.00868966524959189\\
9.04	0.00868951482148872\\
9.05	0.00868936431797004\\
9.06	0.00868921373899445\\
9.07	0.00868906308452054\\
9.08	0.00868891235450686\\
9.09	0.00868876154891192\\
9.1	0.00868861066769423\\
9.11	0.00868845971081224\\
9.12	0.00868830867822439\\
9.13	0.00868815756988907\\
9.14	0.00868800638576466\\
9.15	0.00868785512580951\\
9.16	0.00868770378998192\\
9.17	0.00868755237824018\\
9.18	0.00868740089054254\\
9.19	0.00868724932684721\\
9.2	0.0086870976871124\\
9.21	0.00868694597129626\\
9.22	0.00868679417935692\\
9.23	0.00868664231125249\\
9.24	0.00868649036694104\\
9.25	0.0086863383463806\\
9.26	0.00868618624952919\\
9.27	0.00868603407634478\\
9.28	0.00868588182678532\\
9.29	0.00868572950080873\\
9.3	0.0086855770983729\\
9.31	0.00868542461943568\\
9.32	0.00868527206395491\\
9.33	0.00868511943188837\\
9.34	0.00868496672319382\\
9.35	0.00868481393782901\\
9.36	0.00868466107575163\\
9.37	0.00868450813691937\\
9.38	0.00868435512128985\\
9.39	0.00868420202882068\\
9.4	0.00868404885946945\\
9.41	0.0086838956131937\\
9.42	0.00868374228995096\\
9.43	0.0086835888896987\\
9.44	0.00868343541239439\\
9.45	0.00868328185799543\\
9.46	0.00868312822645923\\
9.47	0.00868297451774314\\
9.48	0.00868282073180449\\
9.49	0.00868266686860058\\
9.5	0.00868251292808867\\
9.51	0.00868235891022601\\
9.52	0.00868220481496979\\
9.53	0.00868205064227718\\
9.54	0.00868189639210532\\
9.55	0.00868174206441133\\
9.56	0.00868158765915227\\
9.57	0.0086814331762852\\
9.58	0.00868127861576711\\
9.59	0.008681123977555\\
9.6	0.00868096926160581\\
9.61	0.00868081446787646\\
9.62	0.00868065959632383\\
9.63	0.00868050464690478\\
9.64	0.00868034961957613\\
9.65	0.00868019451429465\\
9.66	0.00868003933101712\\
9.67	0.00867988406970025\\
9.68	0.00867972873030074\\
9.69	0.00867957331277525\\
9.7	0.00867941781708039\\
9.71	0.00867926224317278\\
9.72	0.00867910659100896\\
9.73	0.00867895086054547\\
9.74	0.0086787950517388\\
9.75	0.00867863916454542\\
9.76	0.00867848319892176\\
9.77	0.00867832715482422\\
9.78	0.00867817103220917\\
9.79	0.00867801483103294\\
9.8	0.00867785855125182\\
9.81	0.00867770219282209\\
9.82	0.00867754575569998\\
9.83	0.0086773892398417\\
9.84	0.00867723264520341\\
9.85	0.00867707597174124\\
9.86	0.00867691921941131\\
9.87	0.00867676238816968\\
9.88	0.00867660547797238\\
9.89	0.00867644848877541\\
9.9	0.00867629142053476\\
9.91	0.00867613427320636\\
9.92	0.0086759770467461\\
9.93	0.00867581974110985\\
9.94	0.00867566235625346\\
9.95	0.00867550489213272\\
9.96	0.00867534734870341\\
9.97	0.00867518972592126\\
9.98	0.00867503202374196\\
9.99	0.00867487424212119\\
10	0.00867471638101459\\
10.01	0.00867455844037774\\
10.02	0.00867440042016622\\
10.03	0.00867424232033556\\
10.04	0.00867408414084126\\
10.05	0.00867392588163878\\
10.06	0.00867376754268356\\
10.07	0.00867360912393098\\
10.08	0.00867345062533641\\
10.09	0.00867329204685517\\
10.1	0.00867313338844257\\
10.11	0.00867297465005386\\
10.12	0.00867281583164426\\
10.13	0.00867265693316897\\
10.14	0.00867249795458313\\
10.15	0.00867233889584188\\
10.16	0.00867217975690029\\
10.17	0.00867202053771342\\
10.18	0.00867186123823628\\
10.19	0.00867170185842387\\
10.2	0.00867154239823111\\
10.21	0.00867138285761294\\
10.22	0.00867122323652422\\
10.23	0.00867106353491981\\
10.24	0.0086709037527545\\
10.25	0.00867074388998307\\
10.26	0.00867058394656026\\
10.27	0.00867042392244077\\
10.28	0.00867026381757927\\
10.29	0.00867010363193039\\
10.3	0.00866994336544873\\
10.31	0.00866978301808885\\
10.32	0.00866962258980527\\
10.33	0.00866946208055249\\
10.34	0.00866930149028496\\
10.35	0.00866914081895711\\
10.36	0.00866898006652331\\
10.37	0.00866881923293791\\
10.38	0.00866865831815524\\
10.39	0.00866849732212955\\
10.4	0.00866833624481511\\
10.41	0.0086681750861661\\
10.42	0.0086680138461367\\
10.43	0.00866785252468105\\
10.44	0.00866769112175323\\
10.45	0.00866752963730732\\
10.46	0.00866736807129733\\
10.47	0.00866720642367727\\
10.48	0.00866704469440106\\
10.49	0.00866688288342265\\
10.5	0.0086667209906959\\
10.51	0.00866655901617466\\
10.52	0.00866639695981273\\
10.53	0.00866623482156389\\
10.54	0.00866607260138187\\
10.55	0.00866591029922036\\
10.56	0.00866574791503304\\
10.57	0.00866558544877352\\
10.58	0.00866542290039538\\
10.59	0.00866526026985219\\
10.6	0.00866509755709746\\
10.61	0.00866493476208465\\
10.62	0.00866477188476721\\
10.63	0.00866460892509855\\
10.64	0.00866444588303202\\
10.65	0.00866428275852097\\
10.66	0.00866411955151866\\
10.67	0.00866395626197837\\
10.68	0.00866379288985331\\
10.69	0.00866362943509665\\
10.7	0.00866346589766153\\
10.71	0.00866330227750107\\
10.72	0.00866313857456833\\
10.73	0.00866297478881633\\
10.74	0.00866281092019807\\
10.75	0.00866264696866651\\
10.76	0.00866248293417454\\
10.77	0.00866231881667506\\
10.78	0.00866215461612091\\
10.79	0.00866199033246489\\
10.8	0.00866182596565975\\
10.81	0.00866166151565823\\
10.82	0.00866149698241301\\
10.83	0.00866133236587675\\
10.84	0.00866116766600204\\
10.85	0.00866100288274148\\
10.86	0.00866083801604759\\
10.87	0.00866067306587286\\
10.88	0.00866050803216976\\
10.89	0.0086603429148907\\
10.9	0.00866017771398807\\
10.91	0.0086600124294142\\
10.92	0.0086598470611214\\
10.93	0.00865968160906194\\
10.94	0.00865951607318803\\
10.95	0.00865935045345187\\
10.96	0.0086591847498056\\
10.97	0.00865901896220134\\
10.98	0.00865885309059114\\
10.99	0.00865868713492705\\
11	0.00865852109516105\\
11.01	0.00865835497124509\\
11.02	0.00865818876313109\\
11.03	0.00865802247077093\\
11.04	0.00865785609411642\\
11.05	0.00865768963311938\\
11.06	0.00865752308773155\\
11.07	0.00865735645790465\\
11.08	0.00865718974359035\\
11.09	0.0086570229447403\\
11.1	0.00865685606130608\\
11.11	0.00865668909323925\\
11.12	0.00865652204049133\\
11.13	0.0086563549030138\\
11.14	0.00865618768075809\\
11.15	0.0086560203736756\\
11.16	0.00865585298171768\\
11.17	0.00865568550483565\\
11.18	0.00865551794298079\\
11.19	0.00865535029610432\\
11.2	0.00865518256415746\\
11.21	0.00865501474709134\\
11.22	0.00865484684485708\\
11.23	0.00865467885740576\\
11.24	0.0086545107846884\\
11.25	0.00865434262665601\\
11.26	0.00865417438325953\\
11.27	0.00865400605444987\\
11.28	0.0086538376401779\\
11.29	0.00865366914039446\\
11.3	0.00865350055505031\\
11.31	0.00865333188409623\\
11.32	0.0086531631274829\\
11.33	0.00865299428516099\\
11.34	0.00865282535708113\\
11.35	0.00865265634319389\\
11.36	0.00865248724344981\\
11.37	0.0086523180577994\\
11.38	0.0086521487861931\\
11.39	0.00865197942858135\\
11.4	0.0086518099849145\\
11.41	0.0086516404551429\\
11.42	0.00865147083921682\\
11.43	0.00865130113708652\\
11.44	0.00865113134870221\\
11.45	0.00865096147401405\\
11.46	0.00865079151297216\\
11.47	0.00865062146552662\\
11.48	0.00865045133162747\\
11.49	0.0086502811112247\\
11.5	0.00865011080426827\\
11.51	0.0086499404107081\\
11.52	0.00864976993049403\\
11.53	0.00864959936357592\\
11.54	0.00864942870990353\\
11.55	0.0086492579694266\\
11.56	0.00864908714209484\\
11.57	0.00864891622785791\\
11.58	0.0086487452266654\\
11.59	0.00864857413846691\\
11.6	0.00864840296321194\\
11.61	0.00864823170084998\\
11.62	0.00864806035133047\\
11.63	0.00864788891460282\\
11.64	0.00864771739061637\\
11.65	0.00864754577932043\\
11.66	0.00864737408066427\\
11.67	0.00864720229459713\\
11.68	0.00864703042106816\\
11.69	0.00864685846002652\\
11.7	0.0086466864114213\\
11.71	0.00864651427520154\\
11.72	0.00864634205131625\\
11.73	0.0086461697397144\\
11.74	0.0086459973403449\\
11.75	0.00864582485315661\\
11.76	0.00864565227809838\\
11.77	0.008645479615119\\
11.78	0.00864530686416719\\
11.79	0.00864513402519166\\
11.8	0.00864496109814106\\
11.81	0.008644788082964\\
11.82	0.00864461497960904\\
11.83	0.00864444178802471\\
11.84	0.00864426850815946\\
11.85	0.00864409513996175\\
11.86	0.00864392168337995\\
11.87	0.00864374813836241\\
11.88	0.00864357450485741\\
11.89	0.00864340078281321\\
11.9	0.00864322697217801\\
11.91	0.00864305307289998\\
11.92	0.00864287908492723\\
11.93	0.00864270500820783\\
11.94	0.0086425308426898\\
11.95	0.00864235658832113\\
11.96	0.00864218224504975\\
11.97	0.00864200781282355\\
11.98	0.00864183329159036\\
11.99	0.008641658681298\\
12	0.0086414839818942\\
12.01	0.00864130919332668\\
12.02	0.00864113431554308\\
12.03	0.00864095934849104\\
12.04	0.00864078429211811\\
12.05	0.00864060914637181\\
12.06	0.00864043391119963\\
12.07	0.008640258586549\\
12.08	0.00864008317236728\\
12.09	0.00863990766860182\\
12.1	0.00863973207519991\\
12.11	0.00863955639210881\\
12.12	0.00863938061927568\\
12.13	0.0086392047566477\\
12.14	0.00863902880417197\\
12.15	0.00863885276179553\\
12.16	0.0086386766294654\\
12.17	0.00863850040712854\\
12.18	0.00863832409473187\\
12.19	0.00863814769222227\\
12.2	0.00863797119954654\\
12.21	0.00863779461665146\\
12.22	0.00863761794348376\\
12.23	0.00863744117999012\\
12.24	0.00863726432611716\\
12.25	0.00863708738181148\\
12.26	0.00863691034701961\\
12.27	0.00863673322168805\\
12.28	0.00863655600576321\\
12.29	0.00863637869919152\\
12.3	0.0086362013019193\\
12.31	0.00863602381389286\\
12.32	0.00863584623505844\\
12.33	0.00863566856536224\\
12.34	0.00863549080475043\\
12.35	0.0086353129531691\\
12.36	0.0086351350105643\\
12.37	0.00863495697688206\\
12.38	0.00863477885206831\\
12.39	0.00863460063606898\\
12.4	0.00863442232882992\\
12.41	0.00863424393029695\\
12.42	0.00863406544041584\\
12.43	0.0086338868591323\\
12.44	0.00863370818639198\\
12.45	0.00863352942214052\\
12.46	0.00863335056632347\\
12.47	0.00863317161888636\\
12.48	0.00863299257977466\\
12.49	0.00863281344893378\\
12.5	0.0086326342263091\\
12.51	0.00863245491184593\\
12.52	0.00863227550548955\\
12.53	0.00863209600718518\\
12.54	0.00863191641687798\\
12.55	0.00863173673451309\\
12.56	0.00863155696003557\\
12.57	0.00863137709339045\\
12.58	0.00863119713452269\\
12.59	0.00863101708337721\\
12.6	0.0086308369398989\\
12.61	0.00863065670403256\\
12.62	0.00863047637572298\\
12.63	0.00863029595491486\\
12.64	0.00863011544155288\\
12.65	0.00862993483558166\\
12.66	0.00862975413694577\\
12.67	0.00862957334558971\\
12.68	0.00862939246145796\\
12.69	0.00862921148449494\\
12.7	0.008629030414645\\
12.71	0.00862884925185246\\
12.72	0.00862866799606158\\
12.73	0.00862848664721658\\
12.74	0.0086283052052616\\
12.75	0.00862812367014077\\
12.76	0.00862794204179814\\
12.77	0.0086277603201777\\
12.78	0.00862757850522341\\
12.79	0.00862739659687918\\
12.8	0.00862721459508884\\
12.81	0.00862703249979621\\
12.82	0.00862685031094501\\
12.83	0.00862666802847896\\
12.84	0.00862648565234169\\
12.85	0.00862630318247677\\
12.86	0.00862612061882776\\
12.87	0.00862593796133814\\
12.88	0.00862575520995134\\
12.89	0.00862557236461073\\
12.9	0.00862538942525965\\
12.91	0.00862520639184136\\
12.92	0.00862502326429908\\
12.93	0.00862484004257599\\
12.94	0.00862465672661519\\
12.95	0.00862447331635975\\
12.96	0.00862428981175267\\
12.97	0.00862410621273692\\
12.98	0.00862392251925539\\
12.99	0.00862373873125093\\
13	0.00862355484866633\\
13.01	0.00862337087144434\\
13.02	0.00862318679952764\\
13.03	0.00862300263285886\\
13.04	0.00862281837138059\\
13.05	0.00862263401503535\\
13.06	0.00862244956376562\\
13.07	0.0086222650175138\\
13.08	0.00862208037622227\\
13.09	0.00862189563983332\\
13.1	0.00862171080828923\\
13.11	0.00862152588153218\\
13.12	0.00862134085950433\\
13.13	0.00862115574214776\\
13.14	0.00862097052940451\\
13.15	0.00862078522121656\\
13.16	0.00862059981752584\\
13.17	0.00862041431827422\\
13.18	0.00862022872340352\\
13.19	0.0086200430328555\\
13.2	0.00861985724657186\\
13.21	0.00861967136449425\\
13.22	0.00861948538656428\\
13.23	0.00861929931272348\\
13.24	0.00861911314291333\\
13.25	0.00861892687707527\\
13.26	0.00861874051515067\\
13.27	0.00861855405708085\\
13.28	0.00861836750280706\\
13.29	0.00861818085227052\\
13.3	0.00861799410541238\\
13.31	0.00861780726217372\\
13.32	0.00861762032249559\\
13.33	0.00861743328631896\\
13.34	0.00861724615358475\\
13.35	0.00861705892423385\\
13.36	0.00861687159820705\\
13.37	0.00861668417544511\\
13.38	0.00861649665588874\\
13.39	0.00861630903947856\\
13.4	0.00861612132615517\\
13.41	0.00861593351585909\\
13.42	0.0086157456085308\\
13.43	0.0086155576041107\\
13.44	0.00861536950253915\\
13.45	0.00861518130375644\\
13.46	0.00861499300770283\\
13.47	0.00861480461431848\\
13.48	0.00861461612354354\\
13.49	0.00861442753531805\\
13.5	0.00861423884958204\\
13.51	0.00861405006627546\\
13.52	0.00861386118533819\\
13.53	0.00861367220671008\\
13.54	0.0086134831303309\\
13.55	0.00861329395614037\\
13.56	0.00861310468407815\\
13.57	0.00861291531408385\\
13.58	0.008612725846097\\
13.59	0.00861253628005709\\
13.6	0.00861234661590356\\
13.61	0.00861215685357576\\
13.62	0.00861196699301301\\
13.63	0.00861177703415456\\
13.64	0.00861158697693958\\
13.65	0.00861139682130723\\
13.66	0.00861120656719656\\
13.67	0.0086110162145466\\
13.68	0.00861082576329629\\
13.69	0.00861063521338453\\
13.7	0.00861044456475016\\
13.71	0.00861025381733195\\
13.72	0.0086100629710686\\
13.73	0.00860987202589878\\
13.74	0.00860968098176109\\
13.75	0.00860948983859405\\
13.76	0.00860929859633614\\
13.77	0.00860910725492578\\
13.78	0.00860891581430132\\
13.79	0.00860872427440105\\
13.8	0.00860853263516321\\
13.81	0.00860834089652597\\
13.82	0.00860814905842743\\
13.83	0.00860795712080566\\
13.84	0.00860776508359863\\
13.85	0.00860757294674429\\
13.86	0.00860738071018049\\
13.87	0.00860718837384505\\
13.88	0.0086069959376757\\
13.89	0.00860680340161014\\
13.9	0.00860661076558599\\
13.91	0.0086064180295408\\
13.92	0.00860622519341207\\
13.93	0.00860603225713726\\
13.94	0.00860583922065372\\
13.95	0.00860564608389877\\
13.96	0.00860545284680967\\
13.97	0.0086052595093236\\
13.98	0.00860506607137769\\
13.99	0.00860487253290901\\
14	0.00860467889385456\\
14.01	0.00860448515415128\\
14.02	0.00860429131373605\\
14.03	0.00860409737254568\\
14.04	0.00860390333051693\\
14.05	0.00860370918758648\\
14.06	0.00860351494369097\\
14.07	0.00860332059876696\\
14.08	0.00860312615275094\\
14.09	0.00860293160557935\\
14.1	0.00860273695718857\\
14.11	0.00860254220751491\\
14.12	0.00860234735649462\\
14.13	0.00860215240406387\\
14.14	0.00860195735015879\\
14.15	0.00860176219471544\\
14.16	0.0086015669376698\\
14.17	0.00860137157895781\\
14.18	0.00860117611851533\\
14.19	0.00860098055627815\\
14.2	0.00860078489218202\\
14.21	0.0086005891261626\\
14.22	0.00860039325815551\\
14.23	0.00860019728809628\\
14.24	0.0086000012159204\\
14.25	0.00859980504156327\\
14.26	0.00859960876496024\\
14.27	0.00859941238604661\\
14.28	0.00859921590475758\\
14.29	0.0085990193210283\\
14.3	0.00859882263479388\\
14.31	0.00859862584598932\\
14.32	0.00859842895454959\\
14.33	0.00859823196040958\\
14.34	0.00859803486350412\\
14.35	0.00859783766376796\\
14.36	0.00859764036113581\\
14.37	0.00859744295554228\\
14.38	0.00859724544692195\\
14.39	0.00859704783520931\\
14.4	0.00859685012033879\\
14.41	0.00859665230224476\\
14.42	0.00859645438086151\\
14.43	0.00859625635612328\\
14.44	0.00859605822796422\\
14.45	0.00859585999631845\\
14.46	0.00859566166111999\\
14.47	0.00859546322230281\\
14.48	0.00859526467980081\\
14.49	0.00859506603354782\\
14.5	0.0085948672834776\\
14.51	0.00859466842952384\\
14.52	0.00859446947162019\\
14.53	0.0085942704097002\\
14.54	0.00859407124369736\\
14.55	0.00859387197354512\\
14.56	0.00859367259917682\\
14.57	0.00859347312052576\\
14.58	0.00859327353752515\\
14.59	0.00859307385010817\\
14.6	0.00859287405820789\\
14.61	0.00859267416175733\\
14.62	0.00859247416068946\\
14.63	0.00859227405493714\\
14.64	0.00859207384443321\\
14.65	0.0085918735291104\\
14.66	0.00859167310890139\\
14.67	0.00859147258373879\\
14.68	0.00859127195355514\\
14.69	0.00859107121828293\\
14.7	0.00859087037785453\\
14.71	0.00859066943220231\\
14.72	0.0085904683812585\\
14.73	0.00859026722495531\\
14.74	0.00859006596322487\\
14.75	0.00858986459599923\\
14.76	0.00858966312321039\\
14.77	0.00858946154479025\\
14.78	0.00858925986067066\\
14.79	0.00858905807078341\\
14.8	0.0085888561750602\\
14.81	0.00858865417343267\\
14.82	0.00858845206583239\\
14.83	0.00858824985219085\\
14.84	0.00858804753243948\\
14.85	0.00858784510650963\\
14.86	0.0085876425743326\\
14.87	0.0085874399358396\\
14.88	0.00858723719096178\\
14.89	0.0085870343396302\\
14.9	0.00858683138177587\\
14.91	0.00858662831732973\\
14.92	0.00858642514622263\\
14.93	0.00858622186838537\\
14.94	0.00858601848374866\\
14.95	0.00858581499224315\\
14.96	0.00858561139379943\\
14.97	0.00858540768834798\\
14.98	0.00858520387581925\\
14.99	0.0085849999561436\\
15	0.00858479592925131\\
15.01	0.0085845917950726\\
15.02	0.00858438755353763\\
15.03	0.00858418320457645\\
15.04	0.00858397874811908\\
15.05	0.00858377418409544\\
15.06	0.00858356951243539\\
15.07	0.00858336473306871\\
15.08	0.00858315984592512\\
15.09	0.00858295485093425\\
15.1	0.00858274974802567\\
15.11	0.00858254453712888\\
15.12	0.00858233921817328\\
15.13	0.00858213379108824\\
15.14	0.00858192825580302\\
15.15	0.00858172261224682\\
15.16	0.00858151686034877\\
15.17	0.00858131100003793\\
15.18	0.00858110503124328\\
15.19	0.00858089895389372\\
15.2	0.00858069276791809\\
15.21	0.00858048647324515\\
15.22	0.00858028006980358\\
15.23	0.008580073557522\\
15.24	0.00857986693632893\\
15.25	0.00857966020615286\\
15.26	0.00857945336692215\\
15.27	0.00857924641856514\\
15.28	0.00857903936101006\\
15.29	0.00857883219418508\\
15.3	0.00857862491801829\\
15.31	0.0085784175324377\\
15.32	0.00857821003737126\\
15.33	0.00857800243274683\\
15.34	0.00857779471849221\\
15.35	0.00857758689453512\\
15.36	0.00857737896080319\\
15.37	0.00857717091722399\\
15.38	0.00857696276372502\\
15.39	0.00857675450023369\\
15.4	0.00857654612667733\\
15.41	0.00857633764298323\\
15.42	0.00857612904907856\\
15.43	0.00857592034489044\\
15.44	0.0085757115303459\\
15.45	0.00857550260537191\\
15.46	0.00857529356989536\\
15.47	0.00857508442384304\\
15.48	0.0085748751671417\\
15.49	0.008574665799718\\
15.5	0.0085744563214985\\
15.51	0.00857424673240973\\
15.52	0.00857403703237809\\
15.53	0.00857382722132996\\
15.54	0.00857361729919159\\
15.55	0.00857340726588919\\
15.56	0.00857319712134887\\
15.57	0.00857298686549668\\
15.58	0.00857277649825859\\
15.59	0.00857256601956048\\
15.6	0.00857235542932816\\
15.61	0.00857214472748738\\
15.62	0.00857193391396379\\
15.63	0.00857172298868295\\
15.64	0.00857151195157039\\
15.65	0.00857130080255152\\
15.66	0.00857108954155168\\
15.67	0.00857087816849615\\
15.68	0.00857066668331012\\
15.69	0.00857045508591869\\
15.7	0.0085702433762469\\
15.71	0.00857003155421971\\
15.72	0.008569819619762\\
15.73	0.00856960757279856\\
15.74	0.00856939541325411\\
15.75	0.0085691831410533\\
15.76	0.0085689707561207\\
15.77	0.00856875825838077\\
15.78	0.00856854564775794\\
15.79	0.00856833292417652\\
15.8	0.00856812008756076\\
15.81	0.00856790713783485\\
15.82	0.00856769407492286\\
15.83	0.00856748089874879\\
15.84	0.0085672676092366\\
15.85	0.00856705420631012\\
15.86	0.00856684068989313\\
15.87	0.00856662705990932\\
15.88	0.00856641331628231\\
15.89	0.00856619945893562\\
15.9	0.00856598548779272\\
15.91	0.00856577140277697\\
15.92	0.00856555720381167\\
15.93	0.00856534289082003\\
15.94	0.0085651284637252\\
15.95	0.00856491392245021\\
15.96	0.00856469926691805\\
15.97	0.0085644844970516\\
15.98	0.00856426961277368\\
15.99	0.00856405461400702\\
16	0.00856383950067427\\
16.01	0.008563624272698\\
16.02	0.0085634089300007\\
16.03	0.00856319347250479\\
16.04	0.00856297790013259\\
16.05	0.00856276221280634\\
16.06	0.00856254641044821\\
16.07	0.00856233049298029\\
16.08	0.00856211446032458\\
16.09	0.00856189831240301\\
16.1	0.00856168204913741\\
16.11	0.00856146567044954\\
16.12	0.00856124917626108\\
16.13	0.00856103256649363\\
16.14	0.00856081584106871\\
16.15	0.00856059899990774\\
16.16	0.00856038204293209\\
16.17	0.00856016497006301\\
16.18	0.00855994778122171\\
16.19	0.00855973047632927\\
16.2	0.00855951305530673\\
16.21	0.00855929551807503\\
16.22	0.00855907786455502\\
16.23	0.00855886009466749\\
16.24	0.00855864220833313\\
16.25	0.00855842420547255\\
16.26	0.00855820608600629\\
16.27	0.00855798784985477\\
16.28	0.00855776949693839\\
16.29	0.0085575510271774\\
16.3	0.00855733244049203\\
16.31	0.00855711373680238\\
16.32	0.00855689491602847\\
16.33	0.00855667597809028\\
16.34	0.00855645692290765\\
16.35	0.00855623775040038\\
16.36	0.00855601846048818\\
16.37	0.00855579905309065\\
16.38	0.00855557952812733\\
16.39	0.00855535988551767\\
16.4	0.00855514012518105\\
16.41	0.00855492024703674\\
16.42	0.00855470025100395\\
16.43	0.00855448013700179\\
16.44	0.0085542599049493\\
16.45	0.00855403955476543\\
16.46	0.00855381908636904\\
16.47	0.00855359849967893\\
16.48	0.00855337779461377\\
16.49	0.0085531569710922\\
16.5	0.00855293602903274\\
16.51	0.00855271496835384\\
16.52	0.00855249378897386\\
16.53	0.00855227249081108\\
16.54	0.00855205107378369\\
16.55	0.00855182953780981\\
16.56	0.00855160788280746\\
16.57	0.00855138610869457\\
16.58	0.00855116421538901\\
16.59	0.00855094220280856\\
16.6	0.00855072007087089\\
16.61	0.0085504978194936\\
16.62	0.00855027544859424\\
16.63	0.00855005295809021\\
16.64	0.00854983034789887\\
16.65	0.0085496076179375\\
16.66	0.00854938476812325\\
16.67	0.00854916179837323\\
16.68	0.00854893870860445\\
16.69	0.00854871549873384\\
16.7	0.00854849216867823\\
16.71	0.00854826871835438\\
16.72	0.00854804514767896\\
16.73	0.00854782145656855\\
16.74	0.00854759764493964\\
16.75	0.00854737371270866\\
16.76	0.00854714965979194\\
16.77	0.0085469254861057\\
16.78	0.00854670119156612\\
16.79	0.00854647677608927\\
16.8	0.00854625223959112\\
16.81	0.00854602758198759\\
16.82	0.00854580280319448\\
16.83	0.00854557790312753\\
16.84	0.00854535288170239\\
16.85	0.0085451277388346\\
16.86	0.00854490247443964\\
16.87	0.00854467708843291\\
16.88	0.0085444515807297\\
16.89	0.00854422595124523\\
16.9	0.00854400019989462\\
16.91	0.00854377432659293\\
16.92	0.0085435483312551\\
16.93	0.00854332221379602\\
16.94	0.00854309597413046\\
16.95	0.00854286961217313\\
16.96	0.00854264312783864\\
16.97	0.00854241652104152\\
16.98	0.00854218979169622\\
16.99	0.00854196293971708\\
17	0.00854173596501837\\
17.01	0.00854150886751429\\
17.02	0.00854128164711892\\
17.03	0.00854105430374629\\
17.04	0.0085408268373103\\
17.05	0.00854059924772481\\
17.06	0.00854037153490357\\
17.07	0.00854014369876023\\
17.08	0.00853991573920838\\
17.09	0.00853968765616151\\
17.1	0.00853945944953304\\
17.11	0.00853923111923627\\
17.12	0.00853900266518445\\
17.13	0.00853877408729072\\
17.14	0.00853854538546814\\
17.15	0.0085383165596297\\
17.16	0.00853808760968827\\
17.17	0.00853785853555666\\
17.18	0.00853762933714759\\
17.19	0.00853740001437368\\
17.2	0.00853717056714747\\
17.21	0.00853694099538143\\
17.22	0.00853671129898792\\
17.23	0.00853648147787922\\
17.24	0.00853625153196754\\
17.25	0.00853602146116497\\
17.26	0.00853579126538354\\
17.27	0.0085355609445352\\
17.28	0.00853533049853179\\
17.29	0.00853509992728507\\
17.3	0.00853486923070672\\
17.31	0.00853463840870834\\
17.32	0.00853440746120142\\
17.33	0.00853417638809739\\
17.34	0.00853394518930757\\
17.35	0.00853371386474321\\
17.36	0.00853348241431547\\
17.37	0.00853325083793542\\
17.38	0.00853301913551404\\
17.39	0.00853278730696224\\
17.4	0.00853255535219081\\
17.41	0.0085323232711105\\
17.42	0.00853209106363193\\
17.43	0.00853185872966567\\
17.44	0.00853162626912217\\
17.45	0.00853139368191182\\
17.46	0.00853116096794491\\
17.47	0.00853092812713165\\
17.48	0.00853069515938215\\
17.49	0.00853046206460646\\
17.5	0.00853022884271451\\
17.51	0.00852999549361618\\
17.52	0.00852976201722122\\
17.53	0.00852952841343935\\
17.54	0.00852929468218015\\
17.55	0.00852906082335314\\
17.56	0.00852882683686776\\
17.57	0.00852859272263335\\
17.58	0.00852835848055916\\
17.59	0.0085281241105544\\
17.6	0.00852788961252815\\
17.61	0.00852765498638945\\
17.62	0.00852742023204721\\
17.63	0.00852718534941031\\
17.64	0.00852695033838751\\
17.65	0.0085267151988875\\
17.66	0.00852647993081889\\
17.67	0.00852624453409021\\
17.68	0.00852600900860991\\
17.69	0.00852577335428634\\
17.7	0.00852553757102778\\
17.71	0.00852530165874244\\
17.72	0.00852506561733842\\
17.73	0.00852482944672375\\
17.74	0.00852459314680639\\
17.75	0.0085243567174942\\
17.76	0.00852412015869496\\
17.77	0.00852388347031638\\
17.78	0.00852364665226607\\
17.79	0.00852340970445155\\
17.8	0.00852317262678029\\
17.81	0.00852293541915966\\
17.82	0.00852269808149692\\
17.83	0.00852246061369928\\
17.84	0.00852222301567387\\
17.85	0.00852198528732771\\
17.86	0.00852174742856776\\
17.87	0.00852150943930088\\
17.88	0.00852127131943385\\
17.89	0.00852103306887336\\
17.9	0.00852079468752604\\
17.91	0.00852055617529841\\
17.92	0.00852031753209693\\
17.93	0.00852007875782795\\
17.94	0.00851983985239775\\
17.95	0.00851960081571252\\
17.96	0.00851936164767838\\
17.97	0.00851912234820135\\
17.98	0.00851888291718736\\
17.99	0.00851864335454228\\
18	0.00851840366017187\\
18.01	0.00851816383398183\\
18.02	0.00851792387587776\\
18.03	0.00851768378576516\\
18.04	0.00851744356354948\\
18.05	0.00851720320913606\\
18.06	0.00851696272243016\\
18.07	0.00851672210333696\\
18.08	0.00851648135176155\\
18.09	0.00851624046760894\\
18.1	0.00851599945078405\\
18.11	0.00851575830119171\\
18.12	0.00851551701873668\\
18.13	0.00851527560332362\\
18.14	0.0085150340548571\\
18.15	0.00851479237324163\\
18.16	0.00851455055838161\\
18.17	0.00851430861018136\\
18.18	0.00851406652854511\\
18.19	0.00851382431337703\\
18.2	0.00851358196458117\\
18.21	0.00851333948206151\\
18.22	0.00851309686572194\\
18.23	0.00851285411546626\\
18.24	0.00851261123119821\\
18.25	0.00851236821282141\\
18.26	0.0085121250602394\\
18.27	0.00851188177335565\\
18.28	0.00851163835207353\\
18.29	0.00851139479629632\\
18.3	0.00851115110592723\\
18.31	0.00851090728086937\\
18.32	0.00851066332102576\\
18.33	0.00851041922629935\\
18.34	0.00851017499659299\\
18.35	0.00850993063180945\\
18.36	0.00850968613185139\\
18.37	0.00850944149662142\\
18.38	0.00850919672602203\\
18.39	0.00850895181995565\\
18.4	0.00850870677832461\\
18.41	0.00850846160103113\\
18.42	0.00850821628797739\\
18.43	0.00850797083906543\\
18.44	0.00850772525419726\\
18.45	0.00850747953327474\\
18.46	0.00850723367619971\\
18.47	0.00850698768287385\\
18.48	0.00850674155319881\\
18.49	0.00850649528707611\\
18.5	0.00850624888440722\\
18.51	0.00850600234509349\\
18.52	0.00850575566903621\\
18.53	0.00850550885613655\\
18.54	0.00850526190629562\\
18.55	0.00850501481941443\\
18.56	0.0085047675953939\\
18.57	0.00850452023413485\\
18.58	0.00850427273553804\\
18.59	0.00850402509950413\\
18.6	0.00850377732593367\\
18.61	0.00850352941472715\\
18.62	0.00850328136578495\\
18.63	0.00850303317900738\\
18.64	0.00850278485429465\\
18.65	0.00850253639154688\\
18.66	0.0085022877906641\\
18.67	0.00850203905154625\\
18.68	0.0085017901740932\\
18.69	0.00850154115820471\\
18.7	0.00850129200378045\\
18.71	0.00850104271072001\\
18.72	0.00850079327892288\\
18.73	0.00850054370828848\\
18.74	0.00850029399871612\\
18.75	0.00850004415010503\\
18.76	0.00849979416235434\\
18.77	0.00849954403536311\\
18.78	0.00849929376903029\\
18.79	0.00849904336325476\\
18.8	0.00849879281793528\\
18.81	0.00849854213297056\\
18.82	0.00849829130825917\\
18.83	0.00849804034369965\\
18.84	0.0084977892391904\\
18.85	0.00849753799462974\\
18.86	0.00849728660991592\\
18.87	0.00849703508494708\\
18.88	0.00849678341962128\\
18.89	0.00849653161383647\\
18.9	0.00849627966749054\\
18.91	0.00849602758048127\\
18.92	0.00849577535270636\\
18.93	0.00849552298406339\\
18.94	0.00849527047444988\\
18.95	0.00849501782376326\\
18.96	0.00849476503190085\\
18.97	0.00849451209875989\\
18.98	0.00849425902423753\\
18.99	0.00849400580823081\\
19	0.0084937524506367\\
19.01	0.00849349895135208\\
19.02	0.00849324531027374\\
19.03	0.00849299152729834\\
19.04	0.0084927376023225\\
19.05	0.00849248353524272\\
19.06	0.00849222932595541\\
19.07	0.00849197497435689\\
19.08	0.00849172048034341\\
19.09	0.00849146584381109\\
19.1	0.00849121106465598\\
19.11	0.00849095614277404\\
19.12	0.00849070107806113\\
19.13	0.00849044587041301\\
19.14	0.00849019051972538\\
19.15	0.00848993502589382\\
19.16	0.00848967938881382\\
19.17	0.00848942360838077\\
19.18	0.00848916768449\\
19.19	0.00848891161703671\\
19.2	0.00848865540591603\\
19.21	0.00848839905102299\\
19.22	0.00848814255225253\\
19.23	0.00848788590949951\\
19.24	0.00848762912265865\\
19.25	0.00848737219162464\\
19.26	0.00848711511629204\\
19.27	0.00848685789655531\\
19.28	0.00848660053230885\\
19.29	0.00848634302344694\\
19.3	0.00848608536986378\\
19.31	0.00848582757145346\\
19.32	0.00848556962811001\\
19.33	0.00848531153972732\\
19.34	0.00848505330619923\\
19.35	0.00848479492741947\\
19.36	0.00848453640328166\\
19.37	0.00848427773367935\\
19.38	0.00848401891850599\\
19.39	0.00848375995765493\\
19.4	0.00848350085101944\\
19.41	0.00848324159849267\\
19.42	0.0084829821999677\\
19.43	0.00848272265533752\\
19.44	0.008482462964495\\
19.45	0.00848220312733294\\
19.46	0.00848194314374403\\
19.47	0.00848168301362088\\
19.48	0.008481422736856\\
19.49	0.00848116231334179\\
19.5	0.00848090174297059\\
19.51	0.00848064102563461\\
19.52	0.00848038016122599\\
19.53	0.00848011914963676\\
19.54	0.00847985799075888\\
19.55	0.00847959668448418\\
19.56	0.00847933523070441\\
19.57	0.00847907362931125\\
19.58	0.00847881188019624\\
19.59	0.00847854998325087\\
19.6	0.0084782879383665\\
19.61	0.00847802574543442\\
19.62	0.0084777634043458\\
19.63	0.00847750091499174\\
19.64	0.00847723827726324\\
19.65	0.00847697549105119\\
19.66	0.0084767125562464\\
19.67	0.00847644947273958\\
19.68	0.00847618624042133\\
19.69	0.00847592285918219\\
19.7	0.00847565932891257\\
19.71	0.0084753956495028\\
19.72	0.00847513182084311\\
19.73	0.00847486784282364\\
19.74	0.00847460371533443\\
19.75	0.00847433943826543\\
19.76	0.00847407501150649\\
19.77	0.00847381043494736\\
19.78	0.00847354570847771\\
19.79	0.00847328083198708\\
19.8	0.00847301580536496\\
19.81	0.00847275062850071\\
19.82	0.00847248530128361\\
19.83	0.00847221982360284\\
19.84	0.00847195419534747\\
19.85	0.00847168841640651\\
19.86	0.00847142248666883\\
19.87	0.00847115640602324\\
19.88	0.00847089017435843\\
19.89	0.008470623791563\\
19.9	0.00847035725752547\\
19.91	0.00847009057213424\\
19.92	0.00846982373527762\\
19.93	0.00846955674684383\\
19.94	0.008469289606721\\
19.95	0.00846902231479714\\
19.96	0.00846875487096019\\
19.97	0.00846848727509797\\
19.98	0.00846821952709822\\
19.99	0.00846795162684857\\
20	0.00846768357423657\\
20.01	0.00846741536914966\\
20.02	0.00846714701147518\\
20.03	0.00846687850110039\\
20.04	0.00846660983791244\\
20.05	0.00846634102179838\\
20.06	0.00846607205264518\\
20.07	0.00846580293033969\\
20.08	0.00846553365476868\\
20.09	0.00846526422581882\\
20.1	0.00846499464337668\\
20.11	0.00846472490732873\\
20.12	0.00846445501756136\\
20.13	0.00846418497396083\\
20.14	0.00846391477641333\\
20.15	0.00846364442480494\\
20.16	0.00846337391902164\\
20.17	0.00846310325894934\\
20.18	0.00846283244447381\\
20.19	0.00846256147548075\\
20.2	0.00846229035185577\\
20.21	0.00846201907348435\\
20.22	0.00846174764025191\\
20.23	0.00846147605204373\\
20.24	0.00846120430874503\\
20.25	0.00846093241024092\\
20.26	0.0084606603564164\\
20.27	0.00846038814715639\\
20.28	0.0084601157823457\\
20.29	0.00845984326186905\\
20.3	0.00845957058561105\\
20.31	0.00845929775345623\\
20.32	0.00845902476528902\\
20.33	0.00845875162099372\\
20.34	0.00845847832045458\\
20.35	0.00845820486355571\\
20.36	0.00845793125018115\\
20.37	0.00845765748021483\\
20.38	0.00845738355354058\\
20.39	0.00845710947004213\\
20.4	0.00845683522960313\\
20.41	0.00845656083210712\\
20.42	0.00845628627743752\\
20.43	0.00845601156547769\\
20.44	0.00845573669611087\\
20.45	0.00845546166922021\\
20.46	0.00845518648468874\\
20.47	0.00845491114239942\\
20.48	0.0084546356422351\\
20.49	0.00845435998407852\\
20.5	0.00845408416781234\\
20.51	0.00845380819331912\\
20.52	0.0084535320604813\\
20.53	0.00845325576918125\\
20.54	0.00845297931930122\\
20.55	0.00845270271072337\\
20.56	0.00845242594332976\\
20.57	0.00845214901700235\\
20.58	0.00845187193162301\\
20.59	0.00845159468707349\\
20.6	0.00845131728323546\\
20.61	0.0084510397199905\\
20.62	0.00845076199722006\\
20.63	0.00845048411480551\\
20.64	0.00845020607262812\\
20.65	0.00844992787056906\\
20.66	0.0084496495085094\\
20.67	0.00844937098633011\\
20.68	0.00844909230391208\\
20.69	0.00844881346113606\\
20.7	0.00844853445788274\\
20.71	0.00844825529403269\\
20.72	0.00844797596946639\\
20.73	0.0084476964840642\\
20.74	0.00844741683770643\\
20.75	0.00844713703027323\\
20.76	0.00844685706164469\\
20.77	0.0084465769317008\\
20.78	0.00844629664032142\\
20.79	0.00844601618738635\\
20.8	0.00844573557277527\\
20.81	0.00844545479636776\\
20.82	0.00844517385804329\\
20.83	0.00844489275768127\\
20.84	0.00844461149516097\\
20.85	0.00844433007036157\\
20.86	0.00844404848316217\\
20.87	0.00844376673344176\\
20.88	0.00844348482107921\\
20.89	0.00844320274595333\\
20.9	0.00844292050794278\\
20.91	0.00844263810692617\\
20.92	0.00844235554278199\\
20.93	0.00844207281538863\\
20.94	0.00844178992462437\\
20.95	0.00844150687036741\\
20.96	0.00844122365249583\\
20.97	0.00844094027088764\\
20.98	0.00844065672542072\\
20.99	0.00844037301597287\\
21	0.00844008914242178\\
21.01	0.00843980510464505\\
21.02	0.00843952090252017\\
21.03	0.00843923653592453\\
21.04	0.00843895200473544\\
21.05	0.00843866730883008\\
21.06	0.00843838244808555\\
21.07	0.00843809742237885\\
21.08	0.00843781223158689\\
21.09	0.00843752687558644\\
21.1	0.00843724135425422\\
21.11	0.00843695566746681\\
21.12	0.00843666981510072\\
21.13	0.00843638379703235\\
21.14	0.008436097613138\\
21.15	0.00843581126329386\\
21.16	0.00843552474737604\\
21.17	0.00843523806526055\\
21.18	0.00843495121682327\\
21.19	0.00843466420194\\
21.2	0.00843437702048646\\
21.21	0.00843408967233824\\
21.22	0.00843380215737085\\
21.23	0.00843351447545969\\
21.24	0.00843322662648006\\
21.25	0.00843293861030717\\
21.26	0.00843265042681612\\
21.27	0.00843236207588191\\
21.28	0.00843207355737946\\
21.29	0.00843178487118356\\
21.3	0.00843149601716892\\
21.31	0.00843120699521015\\
21.32	0.00843091780518175\\
21.33	0.00843062844695814\\
21.34	0.00843033892041362\\
21.35	0.00843004922542239\\
21.36	0.00842975936185857\\
21.37	0.00842946932959617\\
21.38	0.0084291791285091\\
21.39	0.00842888875847116\\
21.4	0.00842859821935607\\
21.41	0.00842830751103743\\
21.42	0.00842801663338877\\
21.43	0.00842772558628349\\
21.44	0.0084274343695949\\
21.45	0.00842714298319622\\
21.46	0.00842685142696056\\
21.47	0.00842655970076095\\
21.48	0.00842626780447029\\
21.49	0.0084259757379614\\
21.5	0.008425683501107\\
21.51	0.00842539109377971\\
21.52	0.00842509851585204\\
21.53	0.00842480576719641\\
21.54	0.00842451284768514\\
21.55	0.00842421975719046\\
21.56	0.00842392649558449\\
21.57	0.00842363306273925\\
21.58	0.00842333945852666\\
21.59	0.00842304568281854\\
21.6	0.00842275173548663\\
21.61	0.00842245761640255\\
21.62	0.00842216332543783\\
21.63	0.00842186886246389\\
21.64	0.00842157422735206\\
21.65	0.00842127941997358\\
21.66	0.00842098444019957\\
21.67	0.00842068928790107\\
21.68	0.00842039396294901\\
21.69	0.00842009846521423\\
21.7	0.00841980279456747\\
21.71	0.00841950695087935\\
21.72	0.00841921093402042\\
21.73	0.00841891474386113\\
21.74	0.0084186183802718\\
21.75	0.00841832184312268\\
21.76	0.00841802513228393\\
21.77	0.00841772824762558\\
21.78	0.00841743118901758\\
21.79	0.00841713395632978\\
21.8	0.00841683654943193\\
21.81	0.00841653896819368\\
21.82	0.00841624121248459\\
21.83	0.00841594328217411\\
21.84	0.0084156451771316\\
21.85	0.00841534689722632\\
21.86	0.00841504844232743\\
21.87	0.00841474981230398\\
21.88	0.00841445100702496\\
21.89	0.00841415202635921\\
21.9	0.00841385287017552\\
21.91	0.00841355353834254\\
21.92	0.00841325403072886\\
21.93	0.00841295434720295\\
21.94	0.00841265448763318\\
21.95	0.00841235445188785\\
21.96	0.00841205423983511\\
21.97	0.00841175385134308\\
21.98	0.00841145328627972\\
21.99	0.00841115254451292\\
22	0.00841085162591049\\
22.01	0.00841055053034011\\
22.02	0.00841024925766938\\
22.03	0.00840994780776579\\
22.04	0.00840964618049676\\
22.05	0.00840934437572959\\
22.06	0.00840904239333149\\
22.07	0.00840874023316955\\
22.08	0.00840843789511081\\
22.09	0.00840813537902217\\
22.1	0.00840783268477046\\
22.11	0.0084075298122224\\
22.12	0.00840722676124461\\
22.13	0.00840692353170364\\
22.14	0.0084066201234659\\
22.15	0.00840631653639774\\
22.16	0.0084060127703654\\
22.17	0.00840570882523502\\
22.18	0.00840540470087264\\
22.19	0.00840510039714423\\
22.2	0.00840479591391564\\
22.21	0.00840449125105262\\
22.22	0.00840418640842083\\
22.23	0.00840388138588585\\
22.24	0.00840357618331315\\
22.25	0.00840327080056809\\
22.26	0.00840296523751597\\
22.27	0.00840265949402196\\
22.28	0.00840235356995115\\
22.29	0.00840204746516854\\
22.3	0.00840174117953902\\
22.31	0.0084014347129274\\
22.32	0.00840112806519838\\
22.33	0.00840082123621656\\
22.34	0.00840051422584649\\
22.35	0.00840020703395256\\
22.36	0.00839989966039911\\
22.37	0.00839959210505037\\
22.38	0.00839928436777048\\
22.39	0.00839897644842348\\
22.4	0.00839866834687331\\
22.41	0.00839836006298384\\
22.42	0.00839805159661882\\
22.43	0.00839774294764192\\
22.44	0.0083974341159167\\
22.45	0.00839712510130664\\
22.46	0.00839681590367512\\
22.47	0.00839650652288545\\
22.48	0.00839619695880081\\
22.49	0.00839588721128429\\
22.5	0.00839557728019891\\
22.51	0.00839526716540758\\
22.52	0.00839495686677313\\
22.53	0.00839464638415828\\
22.54	0.00839433571742567\\
22.55	0.00839402486643783\\
22.56	0.00839371383105721\\
22.57	0.00839340261114619\\
22.58	0.008393091206567\\
22.59	0.00839277961718183\\
22.6	0.00839246784285276\\
22.61	0.00839215588344176\\
22.62	0.00839184373881073\\
22.63	0.00839153140882147\\
22.64	0.00839121889333569\\
22.65	0.00839090619221501\\
22.66	0.00839059330532096\\
22.67	0.00839028023251496\\
22.68	0.00838996697365836\\
22.69	0.0083896535286124\\
22.7	0.00838933989723826\\
22.71	0.00838902607939699\\
22.72	0.00838871207494958\\
22.73	0.00838839788375689\\
22.74	0.00838808350567975\\
22.75	0.00838776894057884\\
22.76	0.00838745418831479\\
22.77	0.00838713924874812\\
22.78	0.00838682412173925\\
22.79	0.00838650880714853\\
22.8	0.00838619330483622\\
22.81	0.00838587761466248\\
22.82	0.00838556173648737\\
22.83	0.00838524567017089\\
22.84	0.00838492941557293\\
22.85	0.0083846129725533\\
22.86	0.0083842963409717\\
22.87	0.00838397952068778\\
22.88	0.00838366251156106\\
22.89	0.00838334531345099\\
22.9	0.00838302792621694\\
22.91	0.00838271034971817\\
22.92	0.00838239258381388\\
22.93	0.00838207462836314\\
22.94	0.00838175648322499\\
22.95	0.00838143814825833\\
22.96	0.00838111962332199\\
22.97	0.00838080090827473\\
22.98	0.00838048200297519\\
22.99	0.00838016290728196\\
23	0.0083798436210535\\
23.01	0.00837952414414822\\
23.02	0.00837920447642443\\
23.03	0.00837888461774035\\
23.04	0.00837856456795411\\
23.05	0.00837824432692377\\
23.06	0.00837792389450729\\
23.07	0.00837760327056255\\
23.08	0.00837728245494734\\
23.09	0.00837696144751937\\
23.1	0.00837664024813625\\
23.11	0.00837631885665554\\
23.12	0.00837599727293468\\
23.13	0.00837567549683104\\
23.14	0.00837535352820189\\
23.15	0.00837503136690444\\
23.16	0.00837470901279579\\
23.17	0.00837438646573299\\
23.18	0.00837406372557297\\
23.19	0.0083737407921726\\
23.2	0.00837341766538865\\
23.21	0.00837309434507782\\
23.22	0.00837277083109672\\
23.23	0.00837244712330188\\
23.24	0.00837212322154974\\
23.25	0.00837179912569668\\
23.26	0.00837147483559896\\
23.27	0.00837115035111279\\
23.28	0.00837082567209429\\
23.29	0.00837050079839949\\
23.3	0.00837017572988435\\
23.31	0.00836985046640473\\
23.32	0.00836952500781644\\
23.33	0.00836919935397517\\
23.34	0.00836887350473657\\
23.35	0.00836854745995617\\
23.36	0.00836822121948945\\
23.37	0.0083678947831918\\
23.38	0.00836756815091852\\
23.39	0.00836724132252485\\
23.4	0.00836691429786593\\
23.41	0.00836658707679684\\
23.42	0.00836625965917257\\
23.43	0.00836593204484803\\
23.44	0.00836560423367805\\
23.45	0.0083652762255174\\
23.46	0.00836494802022074\\
23.47	0.00836461961764269\\
23.48	0.00836429101763776\\
23.49	0.0083639622200604\\
23.5	0.00836363322476497\\
23.51	0.00836330403160578\\
23.52	0.00836297464043703\\
23.53	0.00836264505111287\\
23.54	0.00836231526348735\\
23.55	0.00836198527741446\\
23.56	0.00836165509274811\\
23.57	0.00836132470934215\\
23.58	0.00836099412705032\\
23.59	0.00836066334572631\\
23.6	0.00836033236522375\\
23.61	0.00836000118539615\\
23.62	0.00835966980609699\\
23.63	0.00835933822717965\\
23.64	0.00835900644849746\\
23.65	0.00835867446990364\\
23.66	0.00835834229125138\\
23.67	0.00835800991239375\\
23.68	0.0083576773331838\\
23.69	0.00835734455347446\\
23.7	0.00835701157311862\\
23.71	0.00835667839196909\\
23.72	0.0083563450098786\\
23.73	0.00835601142669982\\
23.74	0.00835567764228534\\
23.75	0.00835534365648768\\
23.76	0.0083550094691593\\
23.77	0.00835467508015258\\
23.78	0.00835434048931984\\
23.79	0.00835400569651331\\
23.8	0.00835367070158518\\
23.81	0.00835333550438755\\
23.82	0.00835300010477246\\
23.83	0.00835266450259187\\
23.84	0.00835232869769769\\
23.85	0.00835199268994175\\
23.86	0.00835165647917582\\
23.87	0.00835132006525159\\
23.88	0.00835098344802071\\
23.89	0.00835064662733474\\
23.9	0.00835030960304517\\
23.91	0.00834997237500344\\
23.92	0.00834963494306093\\
23.93	0.00834929730706893\\
23.94	0.00834895946687869\\
23.95	0.00834862142234137\\
23.96	0.0083482831733081\\
23.97	0.00834794471962991\\
23.98	0.00834760606115779\\
23.99	0.00834726719774266\\
24	0.00834692812923538\\
24.01	0.00834658885548674\\
24.02	0.00834624937634748\\
24.03	0.00834590969166826\\
24.04	0.00834556980129969\\
24.05	0.00834522970509233\\
24.06	0.00834488940289666\\
24.07	0.00834454889456311\\
24.08	0.00834420817994205\\
24.09	0.00834386725888378\\
24.1	0.00834352613123855\\
24.11	0.00834318479685655\\
24.12	0.00834284325558792\\
24.13	0.00834250150728271\\
24.14	0.00834215955179096\\
24.15	0.00834181738896261\\
24.16	0.00834147501864757\\
24.17	0.00834113244069567\\
24.18	0.0083407896549567\\
24.19	0.0083404466612804\\
24.2	0.00834010345951644\\
24.21	0.00833976004951444\\
24.22	0.00833941643112397\\
24.23	0.00833907260419453\\
24.24	0.00833872856857557\\
24.25	0.00833838432411652\\
24.26	0.00833803987066671\\
24.27	0.00833769520807543\\
24.28	0.00833735033619194\\
24.29	0.00833700525486543\\
24.3	0.00833665996394503\\
24.31	0.00833631446327984\\
24.32	0.0083359687527189\\
24.33	0.00833562283211119\\
24.34	0.00833527670130564\\
24.35	0.00833493036015116\\
24.36	0.00833458380849656\\
24.37	0.00833423704619066\\
24.38	0.00833389007308218\\
24.39	0.00833354288901983\\
24.4	0.00833319549385225\\
24.41	0.00833284788742803\\
24.42	0.00833250006959573\\
24.43	0.00833215204020386\\
24.44	0.00833180379910088\\
24.45	0.0083314553461352\\
24.46	0.0083311066811552\\
24.47	0.0083307578040092\\
24.48	0.00833040871454549\\
24.49	0.00833005941261231\\
24.5	0.00832970989805784\\
24.51	0.00832936017073025\\
24.52	0.00832901023047765\\
24.53	0.0083286600771481\\
24.54	0.00832830971058964\\
24.55	0.00832795913065026\\
24.56	0.0083276083371779\\
24.57	0.00832725733002047\\
24.58	0.00832690610902584\\
24.59	0.00832655467404184\\
24.6	0.00832620302491628\\
24.61	0.00832585116149688\\
24.62	0.00832549908363139\\
24.63	0.00832514679116747\\
24.64	0.00832479428395278\\
24.65	0.00832444156183492\\
24.66	0.00832408862466146\\
24.67	0.00832373547227994\\
24.68	0.00832338210453788\\
24.69	0.00832302852128273\\
24.7	0.00832267472236194\\
24.71	0.00832232070762292\\
24.72	0.00832196647691304\\
24.73	0.00832161203007963\\
24.74	0.00832125736697002\\
24.75	0.00832090248743147\\
24.76	0.00832054739131124\\
24.77	0.00832019207845655\\
24.78	0.0083198365487146\\
24.79	0.00831948080193254\\
24.8	0.00831912483795751\\
24.81	0.00831876865663663\\
24.82	0.00831841225781696\\
24.83	0.00831805564134557\\
24.84	0.00831769880706949\\
24.85	0.00831734175483573\\
24.86	0.00831698448449126\\
24.87	0.00831662699588305\\
24.88	0.00831626928885802\\
24.89	0.00831591136326308\\
24.9	0.00831555321894514\\
24.91	0.00831519485575105\\
24.92	0.00831483627351357\\
24.93	0.00831447747203023\\
24.94	0.00831411845109824\\
24.95	0.00831375921051452\\
24.96	0.0083133997500757\\
24.97	0.00831304006957812\\
24.98	0.0083126801688178\\
24.99	0.00831232004759048\\
25	0.0083119597056916\\
25.01	0.00831159914291629\\
25.02	0.00831123835905939\\
25.03	0.00831087735391545\\
25.04	0.00831051612727869\\
25.05	0.00831015467894306\\
25.06	0.00830979300870219\\
25.07	0.00830943111634941\\
25.08	0.00830906900167775\\
25.09	0.00830870666447992\\
25.1	0.00830834410454836\\
25.11	0.00830798132167517\\
25.12	0.00830761831565217\\
25.13	0.00830725508627085\\
25.14	0.00830689163332241\\
25.15	0.00830652795659774\\
25.16	0.00830616405588743\\
25.17	0.00830579993098173\\
25.18	0.00830543558167063\\
25.19	0.00830507100774376\\
25.2	0.00830470620899048\\
25.21	0.00830434118519981\\
25.22	0.00830397593616047\\
25.23	0.00830361046166087\\
25.24	0.0083032447614891\\
25.25	0.00830287883543295\\
25.26	0.00830251268327989\\
25.27	0.00830214630481706\\
25.28	0.0083017796998313\\
25.29	0.00830141286810914\\
25.3	0.00830104580943677\\
25.31	0.00830067852360009\\
25.32	0.00830031101038466\\
25.33	0.00829994326957574\\
25.34	0.00829957530095825\\
25.35	0.00829920710431681\\
25.36	0.00829883867943571\\
25.37	0.00829847002609892\\
25.38	0.00829810114409009\\
25.39	0.00829773203319255\\
25.4	0.0082973626931893\\
25.41	0.00829699312386302\\
25.42	0.00829662332499607\\
25.43	0.00829625329637048\\
25.44	0.00829588303776796\\
25.45	0.00829551254896989\\
25.46	0.00829514182975732\\
25.47	0.00829477087991097\\
25.48	0.00829439969921125\\
25.49	0.00829402828743823\\
25.5	0.00829365664437164\\
25.51	0.0082932847697909\\
25.52	0.0082929126634751\\
25.53	0.00829254032520296\\
25.54	0.00829216775475292\\
25.55	0.00829179495190306\\
25.56	0.00829142191643113\\
25.57	0.00829104864811455\\
25.58	0.0082906751467304\\
25.59	0.00829030141205543\\
25.6	0.00828992744386605\\
25.61	0.00828955324193834\\
25.62	0.00828917880604804\\
25.63	0.00828880413597054\\
25.64	0.00828842923148092\\
25.65	0.00828805409235389\\
25.66	0.00828767871836385\\
25.67	0.00828730310928483\\
25.68	0.00828692726489053\\
25.69	0.00828655118495433\\
25.7	0.00828617486924923\\
25.71	0.00828579831754791\\
25.72	0.00828542152962271\\
25.73	0.00828504450524561\\
25.74	0.00828466724418825\\
25.75	0.00828428974622193\\
25.76	0.00828391201111761\\
25.77	0.00828353403864588\\
25.78	0.00828315582857701\\
25.79	0.00828277738068089\\
25.8	0.00828239869472709\\
25.81	0.00828201977048481\\
25.82	0.00828164060772292\\
25.83	0.00828126120620991\\
25.84	0.00828088156571395\\
25.85	0.00828050168600283\\
25.86	0.00828012156684401\\
25.87	0.00827974120800457\\
25.88	0.00827936060925127\\
25.89	0.00827897977035049\\
25.9	0.00827859869106824\\
25.91	0.00827821737117022\\
25.92	0.00827783581042173\\
25.93	0.00827745400858773\\
25.94	0.00827707196543281\\
25.95	0.00827668968072122\\
25.96	0.00827630715421684\\
25.97	0.00827592438568318\\
25.98	0.0082755413748834\\
25.99	0.00827515812158075\\
26	0.00827477462554393\\
26.01	0.00827439088654129\\
26.02	0.00827400690434086\\
26.03	0.0082736226787103\\
26.04	0.00827323820941694\\
26.05	0.00827285349622775\\
26.06	0.00827246853890938\\
26.07	0.0082720833372281\\
26.08	0.00827169789094987\\
26.09	0.00827131219984025\\
26.1	0.00827092626366451\\
26.11	0.00827054008218754\\
26.12	0.00827015365517386\\
26.13	0.00826976698238768\\
26.14	0.00826938006359284\\
26.15	0.00826899289855281\\
26.16	0.00826860548703074\\
26.17	0.00826821782878939\\
26.18	0.00826782992359119\\
26.19	0.00826744177119821\\
26.2	0.00826705337137217\\
26.21	0.00826666472387441\\
26.22	0.00826627582846593\\
26.23	0.00826588668490736\\
26.24	0.008265497292959\\
26.25	0.00826510765238075\\
26.26	0.00826471776293218\\
26.27	0.00826432762437247\\
26.28	0.00826393723646048\\
26.29	0.00826354659895466\\
26.3	0.00826315571161313\\
26.31	0.00826276457419362\\
26.32	0.00826237318645352\\
26.33	0.00826198154814983\\
26.34	0.00826158965903921\\
26.35	0.00826119751887793\\
26.36	0.0082608051274219\\
26.37	0.00826041248442666\\
26.38	0.00826001958964737\\
26.39	0.00825962644283884\\
26.4	0.00825923304375551\\
26.41	0.00825883939215141\\
26.42	0.00825844548778024\\
26.43	0.00825805133039531\\
26.44	0.00825765691974954\\
26.45	0.00825726225559552\\
26.46	0.00825686733768541\\
26.47	0.00825647216577102\\
26.48	0.00825607673960379\\
26.49	0.00825568105893476\\
26.5	0.00825528512351461\\
26.51	0.00825488893309363\\
26.52	0.00825449248742174\\
26.53	0.00825409578624845\\
26.54	0.00825369882932293\\
26.55	0.00825330161639393\\
26.56	0.00825290414720983\\
26.57	0.00825250642151863\\
26.58	0.00825210843906794\\
26.59	0.008251710199605\\
26.6	0.00825131170287662\\
26.61	0.00825091294862926\\
26.62	0.00825051393660899\\
26.63	0.00825011466656146\\
26.64	0.00824971513823197\\
26.65	0.0082493153513654\\
26.66	0.00824891530570626\\
26.67	0.00824851500099863\\
26.68	0.00824811443698624\\
26.69	0.0082477136134124\\
26.7	0.00824731253002002\\
26.71	0.00824691118655165\\
26.72	0.00824650958274939\\
26.73	0.00824610771835498\\
26.74	0.00824570559310976\\
26.75	0.00824530320675465\\
26.76	0.00824490055903019\\
26.77	0.00824449764967651\\
26.78	0.00824409447843332\\
26.79	0.00824369104503996\\
26.8	0.00824328734923535\\
26.81	0.00824288339075801\\
26.82	0.00824247916934604\\
26.83	0.00824207468473715\\
26.84	0.00824166993666864\\
26.85	0.0082412649248774\\
26.86	0.0082408596490999\\
26.87	0.00824045410907223\\
26.88	0.00824004830453003\\
26.89	0.00823964223520856\\
26.9	0.00823923590084266\\
26.91	0.00823882930116674\\
26.92	0.00823842243591482\\
26.93	0.00823801530482048\\
26.94	0.00823760790761691\\
26.95	0.00823720024403688\\
26.96	0.00823679231381271\\
26.97	0.00823638411667634\\
26.98	0.00823597565235927\\
26.99	0.00823556692059258\\
27	0.00823515792110695\\
27.01	0.00823474865363261\\
27.02	0.00823433911789938\\
27.03	0.00823392931363666\\
27.04	0.00823351924057342\\
27.05	0.0082331088984382\\
27.06	0.00823269828695911\\
27.07	0.00823228740586385\\
27.08	0.00823187625487967\\
27.09	0.00823146483373342\\
27.1	0.00823105314215149\\
27.11	0.00823064117985985\\
27.12	0.00823022894658404\\
27.13	0.00822981644204917\\
27.14	0.00822940366597991\\
27.15	0.00822899061810049\\
27.16	0.00822857729813471\\
27.17	0.00822816370580593\\
27.18	0.00822774984083709\\
27.19	0.00822733570295067\\
27.2	0.00822692129186871\\
27.21	0.00822650660731283\\
27.22	0.00822609164900418\\
27.23	0.0082256764166635\\
27.24	0.00822526091001106\\
27.25	0.00822484512876669\\
27.26	0.0082244290726498\\
27.27	0.0082240127413793\\
27.28	0.00822359613467372\\
27.29	0.00822317925225108\\
27.3	0.008222762093829\\
27.31	0.00822234465912461\\
27.32	0.00822192694785462\\
27.33	0.00822150895973527\\
27.34	0.00822109069448236\\
27.35	0.00822067215181122\\
27.36	0.00822025333143672\\
27.37	0.00821983423307331\\
27.38	0.00821941485643496\\
27.39	0.00821899520123517\\
27.4	0.00821857526718699\\
27.41	0.00821815505400302\\
27.42	0.00821773456139539\\
27.43	0.00821731378907577\\
27.44	0.00821689273675536\\
27.45	0.00821647140414492\\
27.46	0.00821604979095471\\
27.47	0.00821562789689455\\
27.48	0.00821520572167378\\
27.49	0.00821478326500128\\
27.5	0.00821436052658546\\
27.51	0.00821393750613426\\
27.52	0.00821351420335514\\
27.53	0.00821309061795509\\
27.54	0.00821266674964064\\
27.55	0.00821224259811783\\
27.56	0.00821181816309224\\
27.57	0.00821139344426895\\
27.58	0.00821096844135259\\
27.59	0.00821054315404731\\
27.6	0.00821011758205674\\
27.61	0.00820969172508408\\
27.62	0.00820926558283203\\
27.63	0.00820883915500281\\
27.64	0.00820841244129813\\
27.65	0.00820798544141925\\
27.66	0.00820755815506693\\
27.67	0.00820713058194145\\
27.68	0.00820670272174259\\
27.69	0.00820627457416965\\
27.7	0.00820584613892144\\
27.71	0.00820541741569628\\
27.72	0.00820498840419198\\
27.73	0.00820455910410589\\
27.74	0.00820412951513484\\
27.75	0.00820369963697516\\
27.76	0.00820326946932271\\
27.77	0.00820283901187283\\
27.78	0.00820240826432037\\
27.79	0.00820197722635969\\
27.8	0.00820154589768462\\
27.81	0.00820111427798851\\
27.82	0.00820068236696421\\
27.83	0.00820025016430406\\
27.84	0.00819981766969989\\
27.85	0.00819938488284302\\
27.86	0.00819895180342428\\
27.87	0.00819851843113397\\
27.88	0.0081980847656619\\
27.89	0.00819765080669736\\
27.9	0.00819721655392913\\
27.91	0.00819678200704546\\
27.92	0.00819634716573411\\
27.93	0.00819591202968231\\
27.94	0.00819547659857679\\
27.95	0.00819504087210375\\
27.96	0.00819460484994885\\
27.97	0.00819416853179727\\
27.98	0.00819373191733364\\
27.99	0.0081932950062421\\
28	0.00819285779820622\\
28.01	0.00819242029290907\\
28.02	0.00819198249003321\\
28.03	0.00819154438926064\\
28.04	0.00819110599027286\\
28.05	0.00819066729275082\\
28.06	0.00819022829637496\\
28.07	0.00818978900082516\\
28.08	0.00818934940578081\\
28.09	0.0081889095109207\\
28.1	0.00818846931592316\\
28.11	0.00818802882046593\\
28.12	0.00818758802422624\\
28.13	0.00818714692688076\\
28.14	0.00818670552810564\\
28.15	0.00818626382757647\\
28.16	0.00818582182496832\\
28.17	0.00818537951995569\\
28.18	0.00818493691221255\\
28.19	0.00818449400141234\\
28.2	0.00818405078722791\\
28.21	0.0081836072693316\\
28.22	0.00818316344739518\\
28.23	0.00818271932108988\\
28.24	0.00818227489008637\\
28.25	0.00818183015405477\\
28.26	0.00818138511266464\\
28.27	0.008180939765585\\
28.28	0.00818049411248428\\
28.29	0.0081800481530304\\
28.3	0.00817960188689067\\
28.31	0.00817915531373187\\
28.32	0.0081787084332202\\
28.33	0.00817826124502132\\
28.34	0.00817781374880031\\
28.35	0.00817736594422167\\
28.36	0.00817691783094935\\
28.37	0.00817646940864673\\
28.38	0.00817602067697663\\
28.39	0.00817557163560127\\
28.4	0.00817512228418232\\
28.41	0.00817467262238087\\
28.42	0.00817422264985744\\
28.43	0.00817377236627195\\
28.44	0.00817332177128379\\
28.45	0.00817287086455172\\
28.46	0.00817241964573395\\
28.47	0.0081719681144881\\
28.48	0.00817151627047121\\
28.49	0.00817106411333971\\
28.5	0.00817061164274949\\
28.51	0.00817015885835582\\
28.52	0.00816970575981339\\
28.53	0.0081692523467763\\
28.54	0.00816879861889806\\
28.55	0.0081683445758316\\
28.56	0.00816789021722923\\
28.57	0.00816743554274269\\
28.58	0.00816698055202311\\
28.59	0.00816652524472102\\
28.6	0.00816606962048636\\
28.61	0.00816561367896846\\
28.62	0.00816515741981606\\
28.63	0.00816470084267728\\
28.64	0.00816424394719966\\
28.65	0.0081637867330301\\
28.66	0.00816332919981493\\
28.67	0.00816287134719985\\
28.68	0.00816241317482994\\
28.69	0.0081619546823497\\
28.7	0.00816149586940298\\
28.71	0.00816103673563305\\
28.72	0.00816057728068254\\
28.73	0.00816011750419347\\
28.74	0.00815965740580724\\
28.75	0.00815919698516465\\
28.76	0.00815873624190584\\
28.77	0.00815827517567035\\
28.78	0.00815781378609711\\
28.79	0.0081573520728244\\
28.8	0.00815689003548988\\
28.81	0.00815642767373058\\
28.82	0.00815596498718291\\
28.83	0.00815550197548263\\
28.84	0.00815503863826488\\
28.85	0.00815457497516418\\
28.86	0.00815411098581438\\
28.87	0.00815364666984871\\
28.88	0.00815318202689976\\
28.89	0.0081527170565995\\
28.9	0.00815225175857922\\
28.91	0.00815178613246959\\
28.92	0.00815132017790064\\
28.93	0.00815085389450174\\
28.94	0.00815038728190162\\
28.95	0.00814992033972836\\
28.96	0.00814945306760937\\
28.97	0.00814898546517145\\
28.98	0.00814851753204071\\
28.99	0.00814804926784261\\
29	0.00814758067220196\\
29.01	0.00814711174474293\\
29.02	0.00814664248508899\\
29.03	0.00814617289286299\\
29.04	0.00814570296768709\\
29.05	0.00814523270918279\\
29.06	0.00814476211697092\\
29.07	0.00814429119067167\\
29.08	0.00814381992990452\\
29.09	0.00814334833428831\\
29.1	0.0081428764034412\\
29.11	0.00814240413698067\\
29.12	0.00814193153452353\\
29.13	0.00814145859568591\\
29.14	0.00814098532008326\\
29.15	0.00814051170733037\\
29.16	0.00814003775704132\\
29.17	0.00813956346882952\\
29.18	0.0081390888423077\\
29.19	0.00813861387708789\\
29.2	0.00813813857278145\\
29.21	0.00813766292899902\\
29.22	0.00813718694535059\\
29.23	0.00813671062144543\\
29.24	0.00813623395689212\\
29.25	0.00813575695129855\\
29.26	0.0081352796042719\\
29.27	0.00813480191541867\\
29.28	0.00813432388434464\\
29.29	0.0081338455106549\\
29.3	0.00813336679395382\\
29.31	0.0081328877338451\\
29.32	0.00813240832993167\\
29.33	0.00813192858181583\\
29.34	0.0081314484890991\\
29.35	0.00813096805138233\\
29.36	0.00813048726826563\\
29.37	0.0081300061393484\\
29.38	0.00812952466422935\\
29.39	0.00812904284250643\\
29.4	0.00812856067377689\\
29.41	0.00812807815763726\\
29.42	0.00812759529368334\\
29.43	0.0081271120815102\\
29.44	0.00812662852071219\\
29.45	0.00812614461088293\\
29.46	0.00812566035161529\\
29.47	0.00812517574250144\\
29.48	0.00812469078313279\\
29.49	0.00812420547310003\\
29.5	0.00812371981199309\\
29.51	0.00812323379940119\\
29.52	0.00812274743491279\\
29.53	0.0081222607181156\\
29.54	0.00812177364859659\\
29.55	0.008121286225942\\
29.56	0.0081207984497373\\
29.57	0.00812031031956723\\
29.58	0.00811982183501575\\
29.59	0.00811933299566609\\
29.6	0.00811884380110071\\
29.61	0.00811835425090132\\
29.62	0.00811786434464888\\
29.63	0.00811737408192356\\
29.64	0.00811688346230479\\
29.65	0.00811639248537124\\
29.66	0.0081159011507008\\
29.67	0.00811540945787059\\
29.68	0.00811491740645696\\
29.69	0.0081144249960355\\
29.7	0.00811393222618102\\
29.71	0.00811343909646754\\
29.72	0.00811294560646834\\
29.73	0.00811245175575588\\
29.74	0.00811195754390184\\
29.75	0.00811146297047716\\
29.76	0.00811096803505195\\
29.77	0.00811047273719554\\
29.78	0.0081099770764765\\
29.79	0.00810948105246259\\
29.8	0.00810898466472075\\
29.81	0.00810848791281718\\
29.82	0.00810799079631725\\
29.83	0.00810749331478552\\
29.84	0.00810699546778578\\
29.85	0.00810649725488102\\
29.86	0.00810599867563339\\
29.87	0.00810549972960426\\
29.88	0.0081050004163542\\
29.89	0.00810450073544294\\
29.9	0.00810400068642942\\
29.91	0.00810350026887177\\
29.92	0.0081029994823273\\
29.93	0.00810249832635248\\
29.94	0.00810199680050299\\
29.95	0.00810149490433368\\
29.96	0.00810099263739857\\
29.97	0.00810048999925085\\
29.98	0.00809998698944291\\
29.99	0.00809948360752627\\
30	0.00809897985305166\\
30.01	0.00809847572556894\\
30.02	0.00809797122462715\\
30.03	0.00809746634977451\\
30.04	0.00809696110055837\\
30.05	0.00809645547652525\\
30.06	0.00809594947722084\\
30.07	0.00809544310218996\\
30.08	0.0080949363509766\\
30.09	0.0080944292231239\\
30.1	0.00809392171817413\\
30.11	0.00809341383566873\\
30.12	0.00809290557514827\\
30.13	0.00809239693615246\\
30.14	0.00809188791822017\\
30.15	0.00809137852088937\\
30.16	0.00809086874369721\\
30.17	0.00809035858617995\\
30.18	0.00808984804787297\\
30.19	0.0080893371283108\\
30.2	0.0080888258270271\\
30.21	0.00808831414355463\\
30.22	0.0080878020774253\\
30.23	0.00808728962817012\\
30.24	0.00808677679531923\\
30.25	0.00808626357840189\\
30.26	0.00808574997694648\\
30.27	0.00808523599048047\\
30.28	0.00808472161853044\\
30.29	0.00808420686062211\\
30.3	0.00808369171628027\\
30.31	0.00808317618502885\\
30.32	0.00808266026639084\\
30.33	0.00808214395988836\\
30.34	0.00808162726504262\\
30.35	0.00808111018137391\\
30.36	0.00808059270840164\\
30.37	0.00808007484564429\\
30.38	0.00807955659261942\\
30.39	0.00807903794884371\\
30.4	0.00807851891383289\\
30.41	0.00807799948710179\\
30.42	0.00807747966816432\\
30.43	0.00807695945653345\\
30.44	0.00807643885172124\\
30.45	0.00807591785323882\\
30.46	0.00807539646059639\\
30.47	0.0080748746733032\\
30.48	0.00807435249086761\\
30.49	0.00807382991279698\\
30.5	0.0080733069385978\\
30.51	0.00807278356777557\\
30.52	0.00807225979983487\\
30.53	0.0080717356342793\\
30.54	0.00807121107061156\\
30.55	0.00807068610833338\\
30.56	0.00807016074694552\\
30.57	0.0080696349859478\\
30.58	0.00806910882483909\\
30.59	0.00806858226311729\\
30.6	0.00806805530027934\\
30.61	0.0080675279358212\\
30.62	0.00806700016923788\\
30.63	0.00806647200002343\\
30.64	0.0080659434276709\\
30.65	0.0080654144516724\\
30.66	0.00806488507151902\\
30.67	0.00806435528670092\\
30.68	0.00806382509670724\\
30.69	0.00806329450102615\\
30.7	0.00806276349914484\\
30.71	0.0080622320905495\\
30.72	0.00806170027472534\\
30.73	0.00806116805115656\\
30.74	0.00806063541932638\\
30.75	0.00806010237871701\\
30.76	0.00805956892880967\\
30.77	0.00805903506908456\\
30.78	0.00805850079902089\\
30.79	0.00805796611809686\\
30.8	0.00805743102578965\\
30.81	0.00805689552157542\\
30.82	0.00805635960492935\\
30.83	0.00805582327532556\\
30.84	0.00805528653223716\\
30.85	0.00805474937513625\\
30.86	0.00805421180349389\\
30.87	0.00805367381678011\\
30.88	0.00805313541446393\\
30.89	0.00805259659601331\\
30.9	0.00805205736089518\\
30.91	0.00805151770857544\\
30.92	0.00805097763851895\\
30.93	0.0080504371501895\\
30.94	0.00804989624304987\\
30.95	0.00804935491656175\\
30.96	0.00804881317018582\\
30.97	0.00804827100338167\\
30.98	0.00804772841560787\\
30.99	0.00804718540632188\\
31	0.00804664197498014\\
31.01	0.008046098121038\\
31.02	0.00804555384394976\\
31.03	0.00804500914316864\\
31.04	0.00804446401814678\\
31.05	0.00804391846833525\\
31.06	0.00804337249318405\\
31.07	0.00804282609214208\\
31.08	0.00804227926465717\\
31.09	0.00804173201017607\\
31.1	0.0080411843281444\\
31.11	0.00804063621800674\\
31.12	0.00804008767920655\\
31.13	0.00803953871118619\\
31.14	0.00803898931338692\\
31.15	0.0080384394852489\\
31.16	0.00803788922621118\\
31.17	0.00803733853571171\\
31.18	0.00803678741318732\\
31.19	0.00803623585807373\\
31.2	0.00803568386980553\\
31.21	0.0080351314478162\\
31.22	0.00803457859153811\\
31.23	0.00803402530040248\\
31.24	0.0080334715738394\\
31.25	0.00803291741127786\\
31.26	0.00803236281214569\\
31.27	0.00803180777586958\\
31.28	0.0080312523018751\\
31.29	0.00803069638958665\\
31.3	0.00803014003842751\\
31.31	0.00802958324781979\\
31.32	0.00802902601718447\\
31.33	0.00802846834594134\\
31.34	0.00802791023350908\\
31.35	0.00802735167930517\\
31.36	0.00802679268274595\\
31.37	0.00802623324324657\\
31.38	0.00802567336022104\\
31.39	0.00802511303308217\\
31.4	0.00802455226124162\\
31.41	0.00802399104410984\\
31.42	0.00802342938109613\\
31.43	0.00802286727160859\\
31.44	0.00802230471505414\\
31.45	0.00802174171083849\\
31.46	0.00802117825836619\\
31.47	0.00802061435704055\\
31.48	0.00802005000626373\\
31.49	0.00801948520543664\\
31.5	0.00801891995395902\\
31.51	0.00801835425122938\\
31.52	0.00801778809664504\\
31.53	0.00801722148960207\\
31.54	0.00801665442949536\\
31.55	0.00801608691571854\\
31.56	0.00801551894766406\\
31.57	0.00801495052472311\\
31.58	0.00801438164628565\\
31.59	0.00801381231174043\\
31.6	0.00801324252047494\\
31.61	0.00801267227187544\\
31.62	0.00801210156532694\\
31.63	0.00801153040021322\\
31.64	0.00801095877591679\\
31.65	0.00801038669181892\\
31.66	0.00800981414729961\\
31.67	0.00800924114173762\\
31.68	0.00800866767451044\\
31.69	0.00800809374499428\\
31.7	0.00800751935256411\\
31.71	0.0080069444965936\\
31.72	0.00800636917645516\\
31.73	0.00800579339151991\\
31.74	0.00800521714115771\\
31.75	0.00800464042473709\\
31.76	0.00800406324162535\\
31.77	0.00800348559118846\\
31.78	0.0080029074727911\\
31.79	0.00800232888579665\\
31.8	0.0080017498295672\\
31.81	0.00800117030346351\\
31.82	0.00800059030684507\\
31.83	0.00800000983907003\\
31.84	0.00799942889949523\\
31.85	0.00799884748747618\\
31.86	0.00799826560236709\\
31.87	0.00799768324352084\\
31.88	0.00799710041028896\\
31.89	0.00799651710202167\\
31.9	0.00799593331806783\\
31.91	0.007995349057775\\
31.92	0.00799476432048935\\
31.93	0.00799417910555574\\
31.94	0.00799359341231767\\
31.95	0.00799300724011726\\
31.96	0.00799242058829532\\
31.97	0.00799183345619126\\
31.98	0.00799124584314316\\
31.99	0.00799065774848769\\
32	0.00799006917156018\\
32.01	0.00798948011169459\\
32.02	0.00798889056822347\\
32.03	0.00798830054047802\\
32.04	0.00798771002778803\\
32.05	0.00798711902948193\\
32.06	0.00798652754488673\\
32.07	0.00798593557332806\\
32.08	0.00798534311413014\\
32.09	0.00798475016661578\\
32.1	0.00798415673010641\\
32.11	0.00798356280392203\\
32.12	0.00798296838738122\\
32.13	0.00798237347980115\\
32.14	0.00798177808049757\\
32.15	0.0079811821887848\\
32.16	0.00798058580397573\\
32.17	0.00797998892538183\\
32.18	0.00797939155231311\\
32.19	0.00797879368407815\\
32.2	0.00797819531998409\\
32.21	0.00797759645933662\\
32.22	0.00797699710143998\\
32.23	0.00797639724559695\\
32.24	0.00797579689110884\\
32.25	0.00797519603727552\\
32.26	0.00797459468339538\\
32.27	0.00797399282876533\\
32.28	0.00797339047268082\\
32.29	0.0079727876144358\\
32.3	0.00797218425332278\\
32.31	0.00797158038863274\\
32.32	0.00797097601965518\\
32.33	0.00797037114567812\\
32.34	0.00796976576598807\\
32.35	0.00796915987987004\\
32.36	0.00796855348660754\\
32.37	0.00796794658548254\\
32.38	0.00796733917577555\\
32.39	0.0079667312567655\\
32.4	0.00796612282772986\\
32.41	0.00796551388794451\\
32.42	0.00796490443668386\\
32.43	0.00796429447322074\\
32.44	0.00796368399682646\\
32.45	0.0079630730067708\\
32.46	0.00796246150232196\\
32.47	0.00796184948274663\\
32.48	0.00796123694730991\\
32.49	0.00796062389527536\\
32.5	0.00796001032590498\\
32.51	0.00795939623845919\\
32.52	0.00795878163219684\\
32.53	0.00795816650637522\\
32.54	0.00795755086025003\\
32.55	0.00795693469307537\\
32.56	0.00795631800410379\\
32.57	0.0079557007925862\\
32.58	0.00795508305777196\\
32.59	0.00795446479890881\\
32.6	0.00795384601524288\\
32.61	0.00795322670601869\\
32.62	0.00795260687047916\\
32.63	0.00795198650786557\\
32.64	0.00795136561741761\\
32.65	0.00795074419837331\\
32.66	0.00795012224996909\\
32.67	0.00794949977143975\\
32.68	0.0079488767620184\\
32.69	0.00794825322093656\\
32.7	0.00794762914742407\\
32.71	0.00794700454070912\\
32.72	0.00794637940001826\\
32.73	0.00794575372457637\\
32.74	0.00794512751360666\\
32.75	0.00794450076633067\\
32.76	0.00794387348196828\\
32.77	0.00794324565973766\\
32.78	0.00794261729885534\\
32.79	0.00794198839853611\\
32.8	0.00794135895799312\\
32.81	0.00794072897643778\\
32.82	0.00794009845307982\\
32.83	0.00793946738712728\\
32.84	0.00793883577778644\\
32.85	0.00793820362426191\\
32.86	0.00793757092575657\\
32.87	0.00793693768147156\\
32.88	0.0079363038906063\\
32.89	0.00793566955235849\\
32.9	0.00793503466592406\\
32.91	0.00793439923049723\\
32.92	0.00793376324527045\\
32.93	0.00793312670943442\\
32.94	0.0079324896221781\\
32.95	0.00793185198268867\\
32.96	0.00793121379015156\\
32.97	0.0079305750437504\\
32.98	0.00792993574266707\\
32.99	0.00792929588608166\\
33	0.00792865547317247\\
33.01	0.00792801450311603\\
33.02	0.00792737297508706\\
33.03	0.00792673088825846\\
33.04	0.00792608824180136\\
33.05	0.00792544503488505\\
33.06	0.00792480126667703\\
33.07	0.00792415693634297\\
33.08	0.00792351204304671\\
33.09	0.00792286658595028\\
33.1	0.00792222056421384\\
33.11	0.00792157397699574\\
33.12	0.00792092682345248\\
33.13	0.0079202791027387\\
33.14	0.00791963081400721\\
33.15	0.00791898195640893\\
33.16	0.00791833252909294\\
33.17	0.00791768253120642\\
33.18	0.00791703196189472\\
33.19	0.00791638082030127\\
33.2	0.00791572910556764\\
33.21	0.0079150768168335\\
33.22	0.00791442395323662\\
33.23	0.00791377051391288\\
33.24	0.00791311649799625\\
33.25	0.0079124619046188\\
33.26	0.00791180673291066\\
33.27	0.00791115098200007\\
33.28	0.00791049465101332\\
33.29	0.00790983773907477\\
33.3	0.00790918024530686\\
33.31	0.00790852216883008\\
33.32	0.00790786350876296\\
33.33	0.00790720426422209\\
33.34	0.00790654443432212\\
33.35	0.00790588401817569\\
33.36	0.00790522301489353\\
33.37	0.00790456142358434\\
33.38	0.00790389924335488\\
33.39	0.00790323647330991\\
33.4	0.0079025731125522\\
33.41	0.00790190916018253\\
33.42	0.00790124461529968\\
33.43	0.00790057947700041\\
33.44	0.00789991374437948\\
33.45	0.00789924741652964\\
33.46	0.0078985804925416\\
33.47	0.00789791297150405\\
33.48	0.00789724485250364\\
33.49	0.007896576134625\\
33.5	0.00789590681695069\\
33.51	0.00789523689856123\\
33.52	0.00789456637853509\\
33.53	0.00789389525594867\\
33.54	0.0078932235298763\\
33.55	0.00789255119939025\\
33.56	0.00789187826356069\\
33.57	0.00789120472145574\\
33.58	0.00789053057214138\\
33.59	0.00788985581468153\\
33.6	0.00788918044813801\\
33.61	0.00788850447157051\\
33.62	0.00788782788403662\\
33.63	0.00788715068459182\\
33.64	0.00788647287228944\\
33.65	0.00788579444618069\\
33.66	0.00788511540531466\\
33.67	0.00788443574873828\\
33.68	0.00788375547549633\\
33.69	0.00788307458463144\\
33.7	0.00788239307518408\\
33.71	0.00788171094619256\\
33.72	0.007881028196693\\
33.73	0.00788034482571935\\
33.74	0.0078796608323034\\
33.75	0.00787897621547471\\
33.76	0.00787829097426066\\
33.77	0.00787760510768643\\
33.78	0.007876918614775\\
33.79	0.00787623149454712\\
33.8	0.00787554374602131\\
33.81	0.0078748553682139\\
33.82	0.00787416636013894\\
33.83	0.00787347672080829\\
33.84	0.00787278644923151\\
33.85	0.00787209554441596\\
33.86	0.00787140400536671\\
33.87	0.00787071183108657\\
33.88	0.00787001902057609\\
33.89	0.00786932557283354\\
33.9	0.00786863148685489\\
33.91	0.00786793676163385\\
33.92	0.00786724139616181\\
33.93	0.00786654538942787\\
33.94	0.00786584874041881\\
33.95	0.0078651514481191\\
33.96	0.0078644535115109\\
33.97	0.00786375492957402\\
33.98	0.00786305570128597\\
33.99	0.00786235582562187\\
34	0.00786165530155454\\
34.01	0.00786095412805441\\
34.02	0.00786025230408957\\
34.03	0.00785954982862573\\
34.04	0.00785884670062625\\
34.05	0.00785814291905207\\
34.06	0.00785743848286178\\
34.07	0.00785673339101156\\
34.08	0.00785602764245519\\
34.09	0.00785532123614404\\
34.1	0.00785461417102708\\
34.11	0.00785390644605084\\
34.12	0.00785319806015943\\
34.13	0.00785248901229454\\
34.14	0.00785177930139539\\
34.15	0.00785106892639878\\
34.16	0.00785035788623904\\
34.17	0.00784964617984805\\
34.18	0.0078489338061552\\
34.19	0.00784822076408743\\
34.2	0.00784750705256918\\
34.21	0.00784679267052242\\
34.22	0.00784607761686658\\
34.23	0.00784536189051864\\
34.24	0.00784464549039304\\
34.25	0.00784392841540172\\
34.26	0.00784321066445407\\
34.27	0.00784249223645697\\
34.28	0.00784177313031476\\
34.29	0.00784105334492921\\
34.3	0.00784033287919957\\
34.31	0.00783961173202252\\
34.32	0.00783888990229216\\
34.33	0.00783816738890003\\
34.34	0.00783744419073506\\
34.35	0.00783672030668364\\
34.36	0.00783599573562952\\
34.37	0.00783527047645386\\
34.38	0.00783454452803521\\
34.39	0.00783381788924952\\
34.4	0.00783309055897007\\
34.41	0.00783236253606756\\
34.42	0.00783163381941\\
34.43	0.00783090440786278\\
34.44	0.00783017430028864\\
34.45	0.00782944349554763\\
34.46	0.00782871199249715\\
34.47	0.00782797978999192\\
34.48	0.00782724688688396\\
34.49	0.00782651328202262\\
34.5	0.00782577897425452\\
34.51	0.00782504396242359\\
34.52	0.00782430824537105\\
34.53	0.00782357182193538\\
34.54	0.00782283469095233\\
34.55	0.00782209685125492\\
34.56	0.00782135830167341\\
34.57	0.00782061904103532\\
34.58	0.00781987906816539\\
34.59	0.0078191383818856\\
34.6	0.00781839698101515\\
34.61	0.00781765486437046\\
34.62	0.00781691203076514\\
34.63	0.00781616847901\\
34.64	0.00781542420791305\\
34.65	0.00781467921627947\\
34.66	0.00781393350291163\\
34.67	0.00781318706660905\\
34.68	0.00781243990616841\\
34.69	0.00781169202038354\\
34.7	0.00781094340804541\\
34.71	0.00781019406794213\\
34.72	0.00780944399885893\\
34.73	0.00780869319957816\\
34.74	0.00780794166887927\\
34.75	0.0078071894055388\\
34.76	0.00780643640833042\\
34.77	0.00780568267602484\\
34.78	0.00780492820738989\\
34.79	0.00780417300119042\\
34.8	0.00780341705618835\\
34.81	0.00780266037114269\\
34.82	0.00780190294480945\\
34.83	0.00780114477594167\\
34.84	0.00780038586328944\\
34.85	0.00779962620559986\\
34.86	0.00779886580161703\\
34.87	0.00779810465008204\\
34.88	0.00779734274973298\\
34.89	0.00779658009930495\\
34.9	0.00779581669752996\\
34.91	0.00779505254313705\\
34.92	0.00779428763485216\\
34.93	0.00779352197139822\\
34.94	0.00779275555149507\\
34.95	0.00779198837385949\\
34.96	0.00779122043720518\\
34.97	0.00779045174024275\\
34.98	0.00778968228167973\\
34.99	0.00778891206022051\\
35	0.00778814107456638\\
35.01	0.00778736932341551\\
35.02	0.00778659680546295\\
35.03	0.00778582351940057\\
35.04	0.00778504946391712\\
35.05	0.00778427463769819\\
35.06	0.00778349903942618\\
35.07	0.00778272266778032\\
35.08	0.00778194552143667\\
35.09	0.00778116759906806\\
35.1	0.00778038889934413\\
35.11	0.00777960942093133\\
35.12	0.00777882916249283\\
35.13	0.00777804812268861\\
35.14	0.00777726630017539\\
35.15	0.00777648369360665\\
35.16	0.00777570030163258\\
35.17	0.00777491612290013\\
35.18	0.00777413115605295\\
35.19	0.0077733453997314\\
35.2	0.00777255885257254\\
35.21	0.00777177151321013\\
35.22	0.00777098338027461\\
35.23	0.00777019445239306\\
35.24	0.00776940472818928\\
35.25	0.00776861420628365\\
35.26	0.00776782288529326\\
35.27	0.00776703076383179\\
35.28	0.00776623784050954\\
35.29	0.00776544411393345\\
35.3	0.00776464958270705\\
35.31	0.00776385424543046\\
35.32	0.00776305810070037\\
35.33	0.00776226114711007\\
35.34	0.00776146338324939\\
35.35	0.00776066480770472\\
35.36	0.00775986541905902\\
35.37	0.00775906521589173\\
35.38	0.00775826419677884\\
35.39	0.00775746236029287\\
35.4	0.00775665970500281\\
35.41	0.00775585622947416\\
35.42	0.0077550519322689\\
35.43	0.00775424681194548\\
35.44	0.00775344086705882\\
35.45	0.00775263409616026\\
35.46	0.00775182649779763\\
35.47	0.00775101807051514\\
35.48	0.00775020881285345\\
35.49	0.00774939872334963\\
35.5	0.00774858780053713\\
35.51	0.00774777604294579\\
35.52	0.00774696344910186\\
35.53	0.00774615001752791\\
35.54	0.0077453357467429\\
35.55	0.00774452063526213\\
35.56	0.00774370468159723\\
35.57	0.00774288788425614\\
35.58	0.00774207024174313\\
35.59	0.00774125175255878\\
35.6	0.00774043241519994\\
35.61	0.00773961222815976\\
35.62	0.00773879118992765\\
35.63	0.00773796929898928\\
35.64	0.00773714655382658\\
35.65	0.00773632295291768\\
35.66	0.00773549849473698\\
35.67	0.00773467317775507\\
35.68	0.00773384700043876\\
35.69	0.00773301996125103\\
35.7	0.00773219205865105\\
35.71	0.00773136329109418\\
35.72	0.0077305336570319\\
35.73	0.00772970315491187\\
35.74	0.00772887178317788\\
35.75	0.00772803954026984\\
35.76	0.00772720642462375\\
35.77	0.00772637243467176\\
35.78	0.00772553756884207\\
35.79	0.00772470182555898\\
35.8	0.00772386520324284\\
35.81	0.00772302770031006\\
35.82	0.00772218931517312\\
35.83	0.00772135004624049\\
35.84	0.00772050989191669\\
35.85	0.00771966885060224\\
35.86	0.00771882692069365\\
35.87	0.00771798410058342\\
35.88	0.00771714038866002\\
35.89	0.00771629578330789\\
35.9	0.00771545028290741\\
35.91	0.00771460388583491\\
35.92	0.00771375659046261\\
35.93	0.00771290839515869\\
35.94	0.0077120592982872\\
35.95	0.00771120929820807\\
35.96	0.00771035839327716\\
35.97	0.00770950658184612\\
35.98	0.00770865386226251\\
35.99	0.00770780054274214\\
36	0.00770694676580507\\
36.01	0.00770609253118292\\
36.02	0.00770523783860718\\
36.03	0.00770438268780913\\
36.04	0.00770352707851992\\
36.05	0.00770267101047053\\
36.06	0.00770181448339179\\
36.07	0.00770095749701434\\
36.08	0.00770010005106869\\
36.09	0.00769924214528518\\
36.1	0.00769838377939396\\
36.11	0.00769752495312506\\
36.12	0.00769666566620831\\
36.13	0.00769580591837341\\
36.14	0.00769494570934986\\
36.15	0.00769408503886703\\
36.16	0.00769322390665412\\
36.17	0.00769236231244015\\
36.18	0.00769150025595399\\
36.19	0.00769063773692434\\
36.2	0.00768977475507973\\
36.21	0.00768891131014856\\
36.22	0.00768804740185902\\
36.23	0.00768718302993916\\
36.24	0.00768631819411687\\
36.25	0.00768545289411985\\
36.26	0.00768458712967566\\
36.27	0.00768372090051169\\
36.28	0.00768285420635514\\
36.29	0.00768198704693308\\
36.3	0.0076811194219724\\
36.31	0.00768025133119981\\
36.32	0.00767938277434188\\
36.33	0.00767851375112498\\
36.34	0.00767764426127535\\
36.35	0.00767677430451904\\
36.36	0.00767590388058194\\
36.37	0.00767503298918976\\
36.38	0.00767416163006807\\
36.39	0.00767328980294225\\
36.4	0.00767241750753751\\
36.41	0.00767154474357892\\
36.42	0.00767067151079134\\
36.43	0.00766979780889949\\
36.44	0.00766892363762792\\
36.45	0.007668048996701\\
36.46	0.00766717388584293\\
36.47	0.00766629830477777\\
36.48	0.00766542225322937\\
36.49	0.00766454573092143\\
36.5	0.00766366873757749\\
36.51	0.00766279127292089\\
36.52	0.00766191333667483\\
36.53	0.00766103492856234\\
36.54	0.00766015604830624\\
36.55	0.00765927669562923\\
36.56	0.0076583968702538\\
36.57	0.00765751657190228\\
36.58	0.00765663580029684\\
36.59	0.00765575455515947\\
36.6	0.00765487283621198\\
36.61	0.00765399064317603\\
36.62	0.00765310797577308\\
36.63	0.00765222483372444\\
36.64	0.00765134121675123\\
36.65	0.00765045712457441\\
36.66	0.00764957255691475\\
36.67	0.00764868751349287\\
36.68	0.0076478019940292\\
36.69	0.00764691599824399\\
36.7	0.00764602952585734\\
36.71	0.00764514257658916\\
36.72	0.00764425515015917\\
36.73	0.00764336724628695\\
36.74	0.00764247886469187\\
36.75	0.00764159000509315\\
36.76	0.00764070066720982\\
36.77	0.00763981085076074\\
36.78	0.00763892055546459\\
36.79	0.00763802978103988\\
36.8	0.00763713852720494\\
36.81	0.00763624679367792\\
36.82	0.00763535458017679\\
36.83	0.00763446188641937\\
36.84	0.00763356871212325\\
36.85	0.00763267505700589\\
36.86	0.00763178092078455\\
36.87	0.00763088630317632\\
36.88	0.0076299912038981\\
36.89	0.00762909562266663\\
36.9	0.00762819955919845\\
36.91	0.00762730301320993\\
36.92	0.00762640598441726\\
36.93	0.00762550847253645\\
36.94	0.00762461047728334\\
36.95	0.00762371199837356\\
36.96	0.00762281303552261\\
36.97	0.00762191358844575\\
36.98	0.00762101365685809\\
36.99	0.00762011324047457\\
37	0.00761921233900993\\
37.01	0.00761831095217872\\
37.02	0.00761740907969533\\
37.03	0.00761650672127395\\
37.04	0.0076156038766286\\
37.05	0.00761470054547311\\
37.06	0.00761379672752113\\
37.07	0.00761289242248611\\
37.08	0.00761198763008134\\
37.09	0.00761108235001992\\
37.1	0.00761017658201475\\
37.11	0.00760927032577857\\
37.12	0.0076083635810239\\
37.13	0.00760745634746312\\
37.14	0.00760654862480839\\
37.15	0.00760564041277168\\
37.16	0.00760473171106482\\
37.17	0.00760382251939939\\
37.18	0.00760291283748683\\
37.19	0.00760200266503838\\
37.2	0.00760109200176508\\
37.21	0.0076001808473778\\
37.22	0.00759926920158721\\
37.23	0.0075983570641038\\
37.24	0.00759744443463786\\
37.25	0.00759653131289952\\
37.26	0.00759561769859867\\
37.27	0.00759470359144506\\
37.28	0.00759378899114822\\
37.29	0.00759287389741751\\
37.3	0.00759195830996208\\
37.31	0.0075910422284909\\
37.32	0.00759012565271275\\
37.33	0.00758920858233622\\
37.34	0.00758829101706969\\
37.35	0.00758737295662138\\
37.36	0.00758645440069929\\
37.37	0.00758553534901124\\
37.38	0.00758461580126486\\
37.39	0.00758369575716757\\
37.4	0.00758277521642661\\
37.41	0.00758185417874904\\
37.42	0.00758093264384168\\
37.43	0.00758001061141121\\
37.44	0.00757908808116407\\
37.45	0.00757816505280654\\
37.46	0.00757724152604467\\
37.47	0.00757631750058435\\
37.48	0.00757539297613125\\
37.49	0.00757446795239086\\
37.5	0.00757354242906844\\
37.51	0.00757261640586909\\
37.52	0.00757168988249769\\
37.53	0.00757076285865893\\
37.54	0.00756983533405731\\
37.55	0.00756890730839711\\
37.56	0.00756797878138243\\
37.57	0.00756704975271716\\
37.58	0.00756612022210499\\
37.59	0.00756519018924942\\
37.6	0.00756425965385374\\
37.61	0.00756332861562104\\
37.62	0.00756239707425421\\
37.63	0.00756146502945594\\
37.64	0.00756053248092872\\
37.65	0.00755959942837482\\
37.66	0.00755866587149634\\
37.67	0.00755773180999514\\
37.68	0.0075567972435729\\
37.69	0.0075558621719311\\
37.7	0.00755492659477099\\
37.71	0.00755399051179365\\
37.72	0.00755305392269992\\
37.73	0.00755211682719045\\
37.74	0.00755117922496569\\
37.75	0.00755024111572589\\
37.76	0.00754930249917106\\
37.77	0.00754836337500104\\
37.78	0.00754742374291544\\
37.79	0.00754648360261368\\
37.8	0.00754554295379494\\
37.81	0.00754460179615822\\
37.82	0.00754366012940231\\
37.83	0.00754271795322578\\
37.84	0.007541775267327\\
37.85	0.00754083207140411\\
37.86	0.00753988836515506\\
37.87	0.00753894414827759\\
37.88	0.0075379994204692\\
37.89	0.00753705418142721\\
37.9	0.00753610843084871\\
37.91	0.00753516216843059\\
37.92	0.00753421539386951\\
37.93	0.00753326810686193\\
37.94	0.00753232030710408\\
37.95	0.00753137199429201\\
37.96	0.00753042316812151\\
37.97	0.00752947382828818\\
37.98	0.0075285239744874\\
37.99	0.00752757360641435\\
38	0.00752662272376396\\
38.01	0.00752567132623096\\
38.02	0.00752471941350987\\
38.03	0.00752376698529497\\
38.04	0.00752281404128034\\
38.05	0.00752186058115985\\
38.06	0.00752090660462711\\
38.07	0.00751995211137556\\
38.08	0.00751899710109838\\
38.09	0.00751804157348855\\
38.1	0.00751708552823883\\
38.11	0.00751612896504174\\
38.12	0.00751517188358959\\
38.13	0.00751421428357448\\
38.14	0.00751325616468825\\
38.15	0.00751229752662256\\
38.16	0.00751133836906881\\
38.17	0.0075103786917182\\
38.18	0.00750941849426168\\
38.19	0.00750845777639\\
38.2	0.00750749653779367\\
38.21	0.00750653477816298\\
38.22	0.00750557249718798\\
38.23	0.0075046096945585\\
38.24	0.00750364636996415\\
38.25	0.0075026825230943\\
38.26	0.00750171815363808\\
38.27	0.00750075326128443\\
38.28	0.00749978784572201\\
38.29	0.00749882190663928\\
38.3	0.00749785544372447\\
38.31	0.00749688845666555\\
38.32	0.0074959209451503\\
38.33	0.00749495290886623\\
38.34	0.00749398434750064\\
38.35	0.00749301526074057\\
38.36	0.00749204564827285\\
38.37	0.00749107550978407\\
38.38	0.00749010484496057\\
38.39	0.00748913365348848\\
38.4	0.00748816193505367\\
38.41	0.00748718968934177\\
38.42	0.0074862169160382\\
38.43	0.00748524361482811\\
38.44	0.00748426978539643\\
38.45	0.00748329542742785\\
38.46	0.00748232054060681\\
38.47	0.00748134512461751\\
38.48	0.00748036917914392\\
38.49	0.00747939270386976\\
38.5	0.00747841569847851\\
38.51	0.0074774381626534\\
38.52	0.00747646009607742\\
38.53	0.00747548149843332\\
38.54	0.00747450236940361\\
38.55	0.00747352270867054\\
38.56	0.00747254251591613\\
38.57	0.00747156179082213\\
38.58	0.00747058053307007\\
38.59	0.00746959874234121\\
38.6	0.00746861641831658\\
38.61	0.00746763356067695\\
38.62	0.00746665016910283\\
38.63	0.0074656662432745\\
38.64	0.00746468178287199\\
38.65	0.00746369678757506\\
38.66	0.00746271125706323\\
38.67	0.00746172519101576\\
38.68	0.00746073858911167\\
38.69	0.00745975145102972\\
38.7	0.00745876377644839\\
38.71	0.00745777556504595\\
38.72	0.00745678681650038\\
38.73	0.00745579753048941\\
38.74	0.00745480770669053\\
38.75	0.00745381734478094\\
38.76	0.00745282644443762\\
38.77	0.00745183500533725\\
38.78	0.00745084302715628\\
38.79	0.0074498505095709\\
38.8	0.00744885745225702\\
38.81	0.00744786385489029\\
38.82	0.00744686971714611\\
38.83	0.00744587503869961\\
38.84	0.00744487981922566\\
38.85	0.00744388405839885\\
38.86	0.00744288775589353\\
38.87	0.00744189091138375\\
38.88	0.00744089352454333\\
38.89	0.0074398955950458\\
38.9	0.00743889712256443\\
38.91	0.0074378981067722\\
38.92	0.00743689854734186\\
38.93	0.00743589844394585\\
38.94	0.00743489779625636\\
38.95	0.00743389660394531\\
38.96	0.00743289486668435\\
38.97	0.00743189258414482\\
38.98	0.00743088975599783\\
38.99	0.0074298863819142\\
39	0.00742888246156447\\
39.01	0.00742787799461891\\
39.02	0.0074268729807475\\
39.03	0.00742586741961995\\
39.04	0.00742486131090571\\
39.05	0.00742385465427392\\
39.06	0.00742284744939344\\
39.07	0.00742183969593289\\
39.08	0.00742083139356056\\
39.09	0.00741982254194448\\
39.1	0.00741881314075239\\
39.11	0.00741780318965176\\
39.12	0.00741679268830976\\
39.13	0.00741578163639327\\
39.14	0.00741477003356889\\
39.15	0.00741375787950295\\
39.16	0.00741274517386147\\
39.17	0.00741173191631017\\
39.18	0.00741071810651452\\
39.19	0.00740970374413966\\
39.2	0.00740868882885047\\
39.21	0.0074076733603115\\
39.22	0.00740665733818705\\
39.23	0.00740564076214109\\
39.24	0.00740462363183731\\
39.25	0.00740360594693912\\
39.26	0.0074025877071096\\
39.27	0.00740156891201156\\
39.28	0.00740054956130749\\
39.29	0.00739952965465961\\
39.3	0.0073985091917298\\
39.31	0.00739748817217967\\
39.32	0.00739646659567052\\
39.33	0.00739544446186334\\
39.34	0.00739442177041883\\
39.35	0.00739339852099737\\
39.36	0.00739237471325905\\
39.37	0.00739135034686364\\
39.38	0.00739032542147061\\
39.39	0.00738929993673911\\
39.4	0.00738827389232801\\
39.41	0.00738724728789583\\
39.42	0.00738622012310081\\
39.43	0.00738519239760087\\
39.44	0.00738416411105361\\
39.45	0.00738313526311632\\
39.46	0.00738210585344598\\
39.47	0.00738107588169924\\
39.48	0.00738004534753245\\
39.49	0.00737901425060163\\
39.5	0.0073779825905625\\
39.51	0.00737695036707044\\
39.52	0.00737591757978051\\
39.53	0.00737488422834746\\
39.54	0.00737385031242571\\
39.55	0.00737281583166936\\
39.56	0.00737178078573219\\
39.57	0.00737074517426763\\
39.58	0.00736970899692882\\
39.59	0.00736867225336855\\
39.6	0.00736763494323928\\
39.61	0.00736659706619315\\
39.62	0.00736555862188195\\
39.63	0.00736451960995717\\
39.64	0.00736348003006995\\
39.65	0.00736243988187108\\
39.66	0.00736139916501105\\
39.67	0.00736035787913998\\
39.68	0.00735931602390767\\
39.69	0.00735827359896359\\
39.7	0.00735723060395686\\
39.71	0.00735618703853625\\
39.72	0.0073551429023502\\
39.73	0.00735409819504681\\
39.74	0.00735305291627384\\
39.75	0.00735200706567868\\
39.76	0.00735096064290841\\
39.77	0.00734991364760973\\
39.78	0.00734886607942902\\
39.79	0.00734781793801229\\
39.8	0.00734676922300522\\
39.81	0.00734571993405311\\
39.82	0.00734467007080093\\
39.83	0.0073436196328933\\
39.84	0.00734256861997446\\
39.85	0.00734151703168832\\
39.86	0.00734046486767841\\
39.87	0.00733941212758793\\
39.88	0.0073383588110597\\
39.89	0.00733730491773617\\
39.9	0.00733625044725946\\
39.91	0.00733519539927129\\
39.92	0.00733413977341304\\
39.93	0.00733308356932572\\
39.94	0.00733202678664998\\
39.95	0.00733096942502609\\
39.96	0.00732991148409393\\
39.97	0.00732885296349307\\
39.98	0.00732779386286265\\
39.99	0.00732673418184146\\
40	0.00732567392006792\\
40.01	0.00732461307718008\\
};
\addplot [color=green,dashed,forget plot]
  table[row sep=crcr]{%
40.01	0.00732461307718008\\
40.02	0.00732355165281558\\
40.03	0.00732248964661172\\
40.04	0.0073214270582054\\
40.05	0.00732036388723315\\
40.06	0.0073193001333311\\
40.07	0.00731823579613502\\
40.08	0.00731717087528029\\
40.09	0.00731610537040189\\
40.1	0.00731503928113443\\
40.11	0.00731397260711211\\
40.12	0.00731290534796878\\
40.13	0.00731183750333785\\
40.14	0.00731076907285237\\
40.15	0.00730970005614499\\
40.16	0.00730863045284795\\
40.17	0.00730756026259312\\
40.18	0.00730648948501195\\
40.19	0.0073054181197355\\
40.2	0.00730434616639442\\
40.21	0.00730327362461898\\
40.22	0.00730220049403902\\
40.23	0.007301126774284\\
40.24	0.00730005246498294\\
40.25	0.0072989775657645\\
40.26	0.00729790207625688\\
40.27	0.0072968259960879\\
40.28	0.00729574932488497\\
40.29	0.00729467206227508\\
40.3	0.00729359420788479\\
40.31	0.00729251576134026\\
40.32	0.00729143672226724\\
40.33	0.00729035709029104\\
40.34	0.00728927686503656\\
40.35	0.00728819604612827\\
40.36	0.00728711463319024\\
40.37	0.00728603262584608\\
40.38	0.007284950023719\\
40.39	0.00728386682643177\\
40.4	0.00728278303360674\\
40.41	0.0072816986448658\\
40.42	0.00728061365983045\\
40.43	0.00727952807812173\\
40.44	0.00727844189936023\\
40.45	0.00727735512316614\\
40.46	0.00727626774915918\\
40.47	0.00727517977695864\\
40.48	0.00727409120618336\\
40.49	0.00727300203645175\\
40.5	0.00727191226738178\\
40.51	0.00727082189859094\\
40.52	0.0072697309296963\\
40.53	0.00726863936031447\\
40.54	0.00726754719006161\\
40.55	0.00726645441855341\\
40.56	0.00726536104540514\\
40.57	0.00726426707023159\\
40.58	0.00726317249264708\\
40.59	0.00726207731226549\\
40.6	0.00726098152870023\\
40.61	0.00725988514156424\\
40.62	0.00725878815047001\\
40.63	0.00725769055502954\\
40.64	0.00725659235485439\\
40.65	0.00725549354955562\\
40.66	0.00725439413874383\\
40.67	0.00725329412202917\\
40.68	0.00725219349902126\\
40.69	0.00725109226932929\\
40.7	0.00724999043256194\\
40.71	0.00724888798832744\\
40.72	0.0072477849362335\\
40.73	0.00724668127588737\\
40.74	0.0072455770068958\\
40.75	0.00724447212886507\\
40.76	0.00724336664140095\\
40.77	0.00724226054410872\\
40.78	0.00724115383659318\\
40.79	0.00724004651845861\\
40.8	0.00723893858930882\\
40.81	0.0072378300487471\\
40.82	0.00723672089637625\\
40.83	0.00723561113179855\\
40.84	0.00723450075461579\\
40.85	0.00723338976442926\\
40.86	0.00723227816083971\\
40.87	0.0072311659434474\\
40.88	0.00723005311185209\\
40.89	0.00722893966565299\\
40.9	0.00722782560444882\\
40.91	0.00722671092783777\\
40.92	0.00722559563541751\\
40.93	0.0072244797267852\\
40.94	0.00722336320153745\\
40.95	0.00722224605927036\\
40.96	0.00722112829957949\\
40.97	0.00722000992205988\\
40.98	0.00721889092630603\\
40.99	0.00721777131191191\\
41	0.00721665107847095\\
41.01	0.00721553022557604\\
41.02	0.00721440875281952\\
41.03	0.00721328665979319\\
41.04	0.00721216394608833\\
41.05	0.00721104061129563\\
41.06	0.00720991665500526\\
41.07	0.00720879207680682\\
41.08	0.00720766687628937\\
41.09	0.00720654105304141\\
41.1	0.00720541460665088\\
41.11	0.00720428753670514\\
41.12	0.00720315984279103\\
41.13	0.00720203152449479\\
41.14	0.00720090258140209\\
41.15	0.00719977301309806\\
41.16	0.00719864281916723\\
41.17	0.00719751199919357\\
41.18	0.00719638055276047\\
41.19	0.00719524847945073\\
41.2	0.0071941157788466\\
41.21	0.00719298245052971\\
41.22	0.00719184849408113\\
41.23	0.00719071390908133\\
41.24	0.0071895786951102\\
41.25	0.00718844285174703\\
41.26	0.00718730637857049\\
41.27	0.00718616927515871\\
41.28	0.00718503154108918\\
41.29	0.00718389317593878\\
41.3	0.00718275417928383\\
41.31	0.00718161455069999\\
41.32	0.00718047428976235\\
41.33	0.00717933339604538\\
41.34	0.00717819186912292\\
41.35	0.0071770497085682\\
41.36	0.00717590691395385\\
41.37	0.00717476348485184\\
41.38	0.00717361942083356\\
41.39	0.00717247472146974\\
41.4	0.00717132938633049\\
41.41	0.00717018341498532\\
41.42	0.00716903680700304\\
41.43	0.00716788956195188\\
41.44	0.00716674167939941\\
41.45	0.00716559315891256\\
41.46	0.00716444400005762\\
41.47	0.00716329420240022\\
41.48	0.00716214376550534\\
41.49	0.00716099268893734\\
41.5	0.00715984097225989\\
41.51	0.00715868861503601\\
41.52	0.00715753561682807\\
41.53	0.00715638197719777\\
41.54	0.00715522769570613\\
41.55	0.00715407277191354\\
41.56	0.00715291720537968\\
41.57	0.00715176099566358\\
41.58	0.00715060414232358\\
41.59	0.00714944664491735\\
41.6	0.00714828850300187\\
41.61	0.00714712971613343\\
41.62	0.00714597028386767\\
41.63	0.00714481020575948\\
41.64	0.00714364948136311\\
41.65	0.00714248811023207\\
41.66	0.00714132609191921\\
41.67	0.00714016342597666\\
41.68	0.00713900011195583\\
41.69	0.00713783614940745\\
41.7	0.00713667153788153\\
41.71	0.00713550627692736\\
41.72	0.00713434036609351\\
41.73	0.00713317380492785\\
41.74	0.00713200659297751\\
41.75	0.0071308387297889\\
41.76	0.0071296702149077\\
41.77	0.00712850104787886\\
41.78	0.0071273312282466\\
41.79	0.00712616075555439\\
41.8	0.00712498962934498\\
41.81	0.00712381784916035\\
41.82	0.00712264541454176\\
41.83	0.0071214723250297\\
41.84	0.00712029858016392\\
41.85	0.00711912417948341\\
41.86	0.00711794912252639\\
41.87	0.00711677340883033\\
41.88	0.00711559703793193\\
41.89	0.00711442000936713\\
41.9	0.00711324232267108\\
41.91	0.00711206397737819\\
41.92	0.00711088497302204\\
41.93	0.00710970530913546\\
41.94	0.00710852498525051\\
41.95	0.00710734400089842\\
41.96	0.00710616235560965\\
41.97	0.00710498004891389\\
41.98	0.00710379708033997\\
41.99	0.00710261344941599\\
42	0.00710142915566919\\
42.01	0.00710024419862604\\
42.02	0.00709905857781217\\
42.03	0.0070978722927524\\
42.04	0.00709668534297074\\
42.05	0.00709549772799038\\
42.06	0.00709430944733368\\
42.07	0.00709312050052216\\
42.08	0.00709193088707653\\
42.09	0.00709074060651664\\
42.1	0.00708954965836153\\
42.11	0.00708835804212936\\
42.12	0.00708716575733747\\
42.13	0.00708597280350235\\
42.14	0.00708477918013961\\
42.15	0.00708358488676402\\
42.16	0.00708238992288951\\
42.17	0.00708119428802912\\
42.18	0.007079997981695\\
42.19	0.00707880100339849\\
42.2	0.00707760335264999\\
42.21	0.00707640502895906\\
42.22	0.00707520603183436\\
42.23	0.00707400636078368\\
42.24	0.00707280601531388\\
42.25	0.00707160499493099\\
42.26	0.00707040329914007\\
42.27	0.00706920092744533\\
42.28	0.00706799787935004\\
42.29	0.00706679415435658\\
42.3	0.00706558975196641\\
42.31	0.00706438467168007\\
42.32	0.0070631789129972\\
42.33	0.00706197247541646\\
42.34	0.00706076535843565\\
42.35	0.00705955756155157\\
42.36	0.00705834908426013\\
42.37	0.00705713992605628\\
42.38	0.00705593008643402\\
42.39	0.0070547195648864\\
42.4	0.00705350836090554\\
42.41	0.00705229647398256\\
42.42	0.00705108390360766\\
42.43	0.00704987064927004\\
42.44	0.00704865671045795\\
42.45	0.00704744208665866\\
42.46	0.00704622677735846\\
42.47	0.00704501078204265\\
42.48	0.00704379410019555\\
42.49	0.0070425767313005\\
42.5	0.00704135867483981\\
42.51	0.00704013993029482\\
42.52	0.00703892049714586\\
42.53	0.00703770037487225\\
42.54	0.0070364795629523\\
42.55	0.00703525806086327\\
42.56	0.00703403586808144\\
42.57	0.00703281298408205\\
42.58	0.0070315894083393\\
42.59	0.00703036514032636\\
42.6	0.00702914017951536\\
42.61	0.00702791452537739\\
42.62	0.00702668817738248\\
42.63	0.00702546113499962\\
42.64	0.00702423339769673\\
42.65	0.00702300496494067\\
42.66	0.00702177583619724\\
42.67	0.00702054601093116\\
42.68	0.00701931548860607\\
42.69	0.00701808426868454\\
42.7	0.00701685235062805\\
42.71	0.00701561973389697\\
42.72	0.00701438641795061\\
42.73	0.00701315240224715\\
42.74	0.00701191768624367\\
42.75	0.00701068226939616\\
42.76	0.00700944615115946\\
42.77	0.00700820933098732\\
42.78	0.00700697180833236\\
42.79	0.00700573358264606\\
42.8	0.00700449465337876\\
42.81	0.00700325501997969\\
42.82	0.0070020146818969\\
42.83	0.00700077363857731\\
42.84	0.00699953188946667\\
42.85	0.00699828943400961\\
42.86	0.00699704627164954\\
42.87	0.00699580240182873\\
42.88	0.00699455782398829\\
42.89	0.00699331253756812\\
42.9	0.00699206654200694\\
42.91	0.00699081983674229\\
42.92	0.00698957242121051\\
42.93	0.00698832429484674\\
42.94	0.00698707545708491\\
42.95	0.00698582590735773\\
42.96	0.00698457564509671\\
42.97	0.00698332466973214\\
42.98	0.00698207298069306\\
42.99	0.00698082057740728\\
43	0.0069795674593014\\
43.01	0.00697831362580073\\
43.02	0.00697705907632938\\
43.03	0.00697580381031017\\
43.04	0.00697454782716467\\
43.05	0.00697329112631318\\
43.06	0.00697203370717475\\
43.07	0.00697077556916711\\
43.08	0.00696951671170674\\
43.09	0.00696825713420882\\
43.1	0.00696699683608724\\
43.11	0.00696573581675458\\
43.12	0.00696447407562213\\
43.13	0.00696321161209986\\
43.14	0.00696194842559642\\
43.15	0.00696068451551912\\
43.16	0.00695941988127397\\
43.17	0.00695815452226563\\
43.18	0.00695688843789743\\
43.19	0.00695562162757134\\
43.2	0.00695435409068798\\
43.21	0.0069530858266466\\
43.22	0.00695181683484512\\
43.23	0.00695054711468005\\
43.24	0.00694927666554653\\
43.25	0.00694800548683834\\
43.26	0.00694673357794785\\
43.27	0.00694546093826604\\
43.28	0.00694418756718247\\
43.29	0.00694291346408534\\
43.3	0.00694163862836137\\
43.31	0.00694036305939592\\
43.32	0.00693908675657288\\
43.33	0.00693780971927473\\
43.34	0.00693653194688249\\
43.35	0.00693525343877576\\
43.36	0.00693397419433267\\
43.37	0.00693269421292988\\
43.38	0.00693141349394262\\
43.39	0.0069301320367446\\
43.4	0.0069288498407081\\
43.41	0.00692756690520387\\
43.42	0.00692628322960121\\
43.43	0.00692499881326789\\
43.44	0.00692371365557017\\
43.45	0.00692242775587284\\
43.46	0.00692114111353912\\
43.47	0.00691985372793074\\
43.48	0.00691856559840788\\
43.49	0.00691727672432919\\
43.5	0.00691598710505176\\
43.51	0.00691469673993114\\
43.52	0.00691340562832132\\
43.53	0.00691211376957471\\
43.54	0.00691082116304216\\
43.55	0.00690952780807292\\
43.56	0.00690823370401468\\
43.57	0.00690693885021351\\
43.58	0.00690564324601388\\
43.59	0.00690434689075867\\
43.6	0.00690304978378911\\
43.61	0.00690175192444485\\
43.62	0.00690045331206386\\
43.63	0.0068991539459825\\
43.64	0.00689785382553548\\
43.65	0.00689655295005586\\
43.66	0.00689525131885908\\
43.67	0.00689394893125086\\
43.68	0.00689264578653586\\
43.69	0.00689134188401765\\
43.7	0.00689003722299874\\
43.71	0.00688873180278053\\
43.72	0.00688742562266336\\
43.73	0.00688611868194645\\
43.74	0.00688481097992796\\
43.75	0.00688350251590494\\
43.76	0.00688219328917336\\
43.77	0.00688088329902807\\
43.78	0.00687957254476283\\
43.79	0.00687826102567032\\
43.8	0.00687694874104206\\
43.81	0.00687563569016852\\
43.82	0.00687432187233902\\
43.83	0.00687300728684177\\
43.84	0.00687169193296389\\
43.85	0.00687037580999136\\
43.86	0.00686905891720903\\
43.87	0.00686774125390064\\
43.88	0.00686642281934881\\
43.89	0.00686510361283501\\
43.9	0.00686378363363959\\
43.91	0.00686246288104177\\
43.92	0.00686114135431961\\
43.93	0.00685981905275006\\
43.94	0.00685849597560891\\
43.95	0.0068571721221708\\
43.96	0.00685584749170923\\
43.97	0.00685452208349655\\
43.98	0.00685319589680396\\
43.99	0.00685186893090148\\
44	0.00685054118505799\\
44.01	0.00684921265854122\\
44.02	0.0068478833506177\\
44.03	0.00684655326055282\\
44.04	0.00684522238761078\\
44.05	0.00684389073105464\\
44.06	0.00684255829014625\\
44.07	0.00684122506414629\\
44.08	0.00683989105231426\\
44.09	0.00683855625390848\\
44.1	0.00683722066818608\\
44.11	0.006835884294403\\
44.12	0.00683454713181397\\
44.13	0.00683320917967256\\
44.14	0.00683187043723111\\
44.15	0.00683053090374077\\
44.16	0.00682919057845147\\
44.17	0.00682784946061197\\
44.18	0.00682650754946978\\
44.19	0.00682516484427122\\
44.2	0.00682382134426138\\
44.21	0.00682247704868414\\
44.22	0.00682113195678214\\
44.23	0.00681978606779683\\
44.24	0.00681843938096841\\
44.25	0.00681709189553583\\
44.26	0.00681574361073684\\
44.27	0.00681439452580794\\
44.28	0.00681304463998438\\
44.29	0.00681169395250019\\
44.3	0.00681034246258813\\
44.31	0.00680899016947971\\
44.32	0.00680763707240521\\
44.33	0.00680628317059364\\
44.34	0.00680492846327276\\
44.35	0.00680357294966905\\
44.36	0.00680221662900774\\
44.37	0.0068008595005128\\
44.38	0.00679950156340691\\
44.39	0.0067981428169115\\
44.4	0.00679678326024669\\
44.41	0.00679542289263136\\
44.42	0.00679406171328308\\
44.43	0.00679269972141813\\
44.44	0.00679133691625152\\
44.45	0.00678997329699695\\
44.46	0.00678860886286684\\
44.47	0.00678724361307229\\
44.48	0.00678587754682313\\
44.49	0.00678451066332786\\
44.5	0.00678314296179367\\
44.51	0.00678177444142644\\
44.52	0.00678040510143076\\
44.53	0.00677903494100987\\
44.54	0.0067776639593657\\
44.55	0.00677629215569885\\
44.56	0.00677491952920861\\
44.57	0.00677354607909293\\
44.58	0.0067721718045484\\
44.59	0.00677079670477031\\
44.6	0.00676942077895258\\
44.61	0.00676804402628781\\
44.62	0.00676666644596724\\
44.63	0.00676528803718075\\
44.64	0.00676390879911688\\
44.65	0.00676252873096281\\
44.66	0.00676114783190434\\
44.67	0.00675976610112593\\
44.68	0.00675838353781065\\
44.69	0.00675700014114023\\
44.7	0.00675561591029498\\
44.71	0.00675423084445387\\
44.72	0.00675284494279448\\
44.73	0.00675145820449298\\
44.74	0.00675007062872419\\
44.75	0.0067486822146615\\
44.76	0.00674729296147694\\
44.77	0.00674590286834111\\
44.78	0.00674451193442323\\
44.79	0.00674312015889109\\
44.8	0.0067417275409111\\
44.81	0.00674033407964822\\
44.82	0.00673893977426603\\
44.83	0.00673754462392667\\
44.84	0.00673614862779086\\
44.85	0.00673475178501788\\
44.86	0.0067333540947656\\
44.87	0.00673195555619043\\
44.88	0.00673055616844738\\
44.89	0.00672915593068998\\
44.9	0.00672775484207033\\
44.91	0.00672635290173907\\
44.92	0.00672495010884541\\
44.93	0.00672354646253709\\
44.94	0.00672214196196038\\
44.95	0.00672073660626008\\
44.96	0.00671933039457957\\
44.97	0.00671792332606069\\
44.98	0.00671651539984387\\
44.99	0.006715106615068\\
45	0.00671369697087054\\
45.01	0.00671228646638744\\
45.02	0.00671087510075314\\
45.03	0.00670946287310063\\
45.04	0.00670804978256136\\
45.05	0.0067066358282653\\
45.06	0.00670522100934092\\
45.07	0.00670380532491515\\
45.08	0.00670238877411345\\
45.09	0.00670097135605972\\
45.1	0.00669955306987637\\
45.11	0.00669813391468426\\
45.12	0.00669671388960275\\
45.13	0.00669529299374964\\
45.14	0.00669387122624121\\
45.15	0.00669244858619218\\
45.16	0.00669102507271576\\
45.17	0.00668960068492358\\
45.18	0.00668817542192572\\
45.19	0.00668674928283072\\
45.2	0.00668532226674555\\
45.21	0.00668389437277561\\
45.22	0.00668246560002472\\
45.23	0.00668103594759517\\
45.24	0.00667960541458763\\
45.25	0.00667817400010121\\
45.26	0.00667674170323344\\
45.27	0.00667530852308022\\
45.28	0.00667387445873592\\
45.29	0.00667243950929326\\
45.3	0.00667100367384339\\
45.31	0.00666956695147583\\
45.32	0.0066681293412785\\
45.33	0.00666669084233773\\
45.34	0.00666525145373818\\
45.35	0.00666381117456293\\
45.36	0.00666237000389343\\
45.37	0.00666092794080948\\
45.38	0.00665948498438925\\
45.39	0.00665804113370928\\
45.4	0.00665659638784447\\
45.41	0.00665515074586804\\
45.42	0.0066537042068516\\
45.43	0.00665225676986506\\
45.44	0.00665080843397673\\
45.45	0.00664935919825318\\
45.46	0.00664790906175936\\
45.47	0.00664645802355854\\
45.48	0.00664500608271229\\
45.49	0.00664355323828053\\
45.5	0.00664209948932144\\
45.51	0.00664064483489158\\
45.52	0.00663918927404576\\
45.53	0.00663773280583709\\
45.54	0.00663627542931701\\
45.55	0.00663481714353522\\
45.56	0.00663335794753972\\
45.57	0.00663189784037679\\
45.58	0.00663043682109097\\
45.59	0.0066289748887251\\
45.6	0.00662751204232026\\
45.61	0.00662604828091582\\
45.62	0.00662458360354938\\
45.63	0.00662311800925681\\
45.64	0.00662165149707224\\
45.65	0.00662018406602802\\
45.66	0.00661871571515475\\
45.67	0.00661724644348128\\
45.68	0.00661577625003467\\
45.69	0.0066143051338402\\
45.7	0.0066128330939214\\
45.71	0.00661136012929999\\
45.72	0.00660988623899592\\
45.73	0.00660841142202734\\
45.74	0.0066069356774106\\
45.75	0.00660545900416024\\
45.76	0.006603981401289\\
45.77	0.00660250286780781\\
45.78	0.00660102340272578\\
45.79	0.00659954300505021\\
45.8	0.00659806167378655\\
45.81	0.00659657940793843\\
45.82	0.00659509620650763\\
45.83	0.00659361206849413\\
45.84	0.00659212699289601\\
45.85	0.00659064097870953\\
45.86	0.00658915402492908\\
45.87	0.00658766613054721\\
45.88	0.00658617729455456\\
45.89	0.00658468751593995\\
45.9	0.00658319679369029\\
45.91	0.00658170512679061\\
45.92	0.00658021251422408\\
45.93	0.00657871895497193\\
45.94	0.00657722444801355\\
45.95	0.00657572899232639\\
45.96	0.00657423258688599\\
45.97	0.00657273523066599\\
45.98	0.00657123692263812\\
45.99	0.00656973766177218\\
46	0.00656823744703603\\
46.01	0.00656673627739562\\
46.02	0.00656523415181494\\
46.03	0.00656373106925605\\
46.04	0.00656222702867905\\
46.05	0.00656072202904209\\
46.06	0.00655921606930138\\
46.07	0.00655770914841112\\
46.08	0.00655620126532358\\
46.09	0.00655469241898904\\
46.1	0.0065531826083558\\
46.11	0.00655167183237016\\
46.12	0.00655016008997646\\
46.13	0.00654864738011702\\
46.14	0.00654713370173215\\
46.15	0.00654561905376017\\
46.16	0.00654410343513737\\
46.17	0.00654258684479803\\
46.18	0.0065410692816744\\
46.19	0.00653955074469672\\
46.2	0.00653803123279317\\
46.21	0.00653651074488989\\
46.22	0.00653498927991098\\
46.23	0.00653346683677849\\
46.24	0.0065319434144124\\
46.25	0.00653041901173063\\
46.26	0.00652889362764905\\
46.27	0.00652736726108141\\
46.28	0.00652583991093944\\
46.29	0.00652431157613272\\
46.3	0.00652278225556878\\
46.31	0.00652125194815304\\
46.32	0.0065197206527888\\
46.33	0.00651818836837727\\
46.34	0.00651665509381754\\
46.35	0.00651512082800657\\
46.36	0.00651358556983921\\
46.37	0.00651204931820814\\
46.38	0.00651051207200394\\
46.39	0.00650897383011503\\
46.4	0.00650743459142767\\
46.41	0.00650589435482597\\
46.42	0.00650435311919188\\
46.43	0.00650281088340518\\
46.44	0.00650126764634346\\
46.45	0.00649972340688215\\
46.46	0.00649817816389447\\
46.47	0.00649663191625147\\
46.48	0.00649508466282198\\
46.49	0.00649353640247263\\
46.5	0.00649198713406785\\
46.51	0.00649043685646984\\
46.52	0.00648888556853856\\
46.53	0.00648733326913179\\
46.54	0.00648577995710501\\
46.55	0.00648422563131149\\
46.56	0.00648267029060225\\
46.57	0.00648111393382605\\
46.58	0.00647955655982939\\
46.59	0.0064779981674565\\
46.6	0.00647643875554933\\
46.61	0.00647487832294756\\
46.62	0.00647331686848857\\
46.63	0.00647175439100746\\
46.64	0.00647019088933701\\
46.65	0.0064686263623077\\
46.66	0.00646706080874771\\
46.67	0.00646549422748289\\
46.68	0.00646392661733675\\
46.69	0.0064623579771305\\
46.7	0.00646078830568297\\
46.71	0.00645921760181067\\
46.72	0.00645764586432775\\
46.73	0.006456073092046\\
46.74	0.00645449928377484\\
46.75	0.00645292443832133\\
46.76	0.00645134855449013\\
46.77	0.00644977163108353\\
46.78	0.00644819366690142\\
46.79	0.00644661466074128\\
46.8	0.0064450346113982\\
46.81	0.00644345351766485\\
46.82	0.00644187137833148\\
46.83	0.0064402881921859\\
46.84	0.00643870395801351\\
46.85	0.00643711867459724\\
46.86	0.00643553234071759\\
46.87	0.0064339449551526\\
46.88	0.00643235651667785\\
46.89	0.00643076702406644\\
46.9	0.00642917647608901\\
46.91	0.00642758487151371\\
46.92	0.00642599220910618\\
46.93	0.0064243984876296\\
46.94	0.0064228037058446\\
46.95	0.00642120786250935\\
46.96	0.00641961095637944\\
46.97	0.00641801298620799\\
46.98	0.00641641395074554\\
46.99	0.00641481384874013\\
47	0.00641321267893722\\
47.01	0.00641161044007972\\
47.02	0.006410007130908\\
47.03	0.00640840275015982\\
47.04	0.00640679729657039\\
47.05	0.00640519076887233\\
47.06	0.00640358316579567\\
47.07	0.00640197448606782\\
47.08	0.00640036472841361\\
47.09	0.00639875389155523\\
47.1	0.00639714197421227\\
47.11	0.00639552897510167\\
47.12	0.00639391489293773\\
47.13	0.00639229972643211\\
47.14	0.00639068347429384\\
47.15	0.00638906613522926\\
47.16	0.00638744770794204\\
47.17	0.00638582819113318\\
47.18	0.00638420758350102\\
47.19	0.00638258588374116\\
47.2	0.00638096309054654\\
47.21	0.00637933920260737\\
47.22	0.00637771421861116\\
47.23	0.00637608813724267\\
47.24	0.00637446095718396\\
47.25	0.00637283267711434\\
47.26	0.00637120329571035\\
47.27	0.00636957281164581\\
47.28	0.00636794122359175\\
47.29	0.00636630853021644\\
47.3	0.00636467473018537\\
47.31	0.00636303982216122\\
47.32	0.00636140380480392\\
47.33	0.00635976667677055\\
47.34	0.00635812843671541\\
47.35	0.00635648908328995\\
47.36	0.0063548486151428\\
47.37	0.00635320703091978\\
47.38	0.00635156432926383\\
47.39	0.00634992050881505\\
47.4	0.00634827556821068\\
47.41	0.00634662950608507\\
47.42	0.00634498232106972\\
47.43	0.00634333401179322\\
47.44	0.00634168457688127\\
47.45	0.00634003401495666\\
47.46	0.00633838232463929\\
47.47	0.00633672950454611\\
47.48	0.00633507555329115\\
47.49	0.00633342046948549\\
47.5	0.00633176425173729\\
47.51	0.00633010689865172\\
47.52	0.00632844840883101\\
47.53	0.00632678878087439\\
47.54	0.00632512801337814\\
47.55	0.00632346610493552\\
47.56	0.0063218030541368\\
47.57	0.00632013885956924\\
47.58	0.00631847351981708\\
47.59	0.00631680703346153\\
47.6	0.00631513939908077\\
47.61	0.00631347061524992\\
47.62	0.00631180068054108\\
47.63	0.00631012959352323\\
47.64	0.00630845735276233\\
47.65	0.00630678395682123\\
47.66	0.0063051094042597\\
47.67	0.0063034336936344\\
47.68	0.00630175682349888\\
47.69	0.00630007879240358\\
47.7	0.00629839959889581\\
47.71	0.00629671924151974\\
47.72	0.00629503771881638\\
47.73	0.0062933550293236\\
47.74	0.00629167117157611\\
47.75	0.00628998614410542\\
47.76	0.00628829994543988\\
47.77	0.00628661257410463\\
47.78	0.00628492402862161\\
47.79	0.00628323430750955\\
47.8	0.00628154340928394\\
47.81	0.00627985133245707\\
47.82	0.00627815807553794\\
47.83	0.00627646363703235\\
47.84	0.00627476801544279\\
47.85	0.00627307120926852\\
47.86	0.00627137321700548\\
47.87	0.00626967403714635\\
47.88	0.00626797366818047\\
47.89	0.00626627210859392\\
47.9	0.0062645693568694\\
47.91	0.00626286541148633\\
47.92	0.00626116027092076\\
47.93	0.00625945393364539\\
47.94	0.00625774639812956\\
47.95	0.00625603766283924\\
47.96	0.00625432772623702\\
47.97	0.00625261658678209\\
47.98	0.00625090424293023\\
47.99	0.00624919069313384\\
48	0.00624747593584185\\
48.01	0.00624575996949978\\
48.02	0.00624404279254971\\
48.03	0.00624232440343026\\
48.04	0.00624060480057658\\
48.05	0.00623888398242035\\
48.06	0.00623716194738974\\
48.07	0.00623543869390946\\
48.08	0.00623371422040069\\
48.09	0.0062319885252811\\
48.1	0.00623026160696481\\
48.11	0.00622853346386243\\
48.12	0.00622680409438099\\
48.13	0.00622507349692399\\
48.14	0.00622334166989132\\
48.15	0.00622160861167931\\
48.16	0.00621987432068069\\
48.17	0.00621813879528459\\
48.18	0.00621640203387652\\
48.19	0.00621466403483835\\
48.2	0.00621292479654832\\
48.21	0.00621118431738103\\
48.22	0.0062094425957074\\
48.23	0.00620769962989468\\
48.24	0.00620595541830646\\
48.25	0.00620420995930261\\
48.26	0.00620246325123929\\
48.27	0.00620071529246896\\
48.28	0.00619896608134034\\
48.29	0.00619721561619842\\
48.3	0.00619546389538442\\
48.31	0.00619371091723581\\
48.32	0.00619195668008627\\
48.33	0.00619020118226571\\
48.34	0.00618844442210023\\
48.35	0.00618668639791213\\
48.36	0.00618492710801986\\
48.37	0.00618316655073806\\
48.38	0.00618140472437754\\
48.39	0.00617964162724521\\
48.4	0.00617787725764413\\
48.41	0.00617611161387348\\
48.42	0.00617434469422855\\
48.43	0.00617257649700071\\
48.44	0.00617080702047743\\
48.45	0.00616903626294222\\
48.46	0.00616726422267468\\
48.47	0.00616549089795044\\
48.48	0.00616371628704116\\
48.49	0.00616194038821453\\
48.5	0.00616016319973423\\
48.51	0.00615838471985995\\
48.52	0.00615660494684736\\
48.53	0.00615487450476265\\
48.54	0.00615395281727438\\
48.55	0.00615303064111654\\
48.56	0.00615210797600223\\
48.57	0.00615118482164451\\
48.58	0.00615026117775647\\
48.59	0.00614933704405116\\
48.6	0.00614841242024162\\
48.61	0.00614748730604091\\
48.62	0.00614656170116206\\
48.63	0.0061456356053181\\
48.64	0.00614470901822207\\
48.65	0.00614378193958699\\
48.66	0.00614285436912589\\
48.67	0.00614192630655178\\
48.68	0.00614099775157769\\
48.69	0.00614006870391664\\
48.7	0.00613913916328166\\
48.71	0.00613820912938578\\
48.72	0.00613727860194202\\
48.73	0.00613634758066343\\
48.74	0.00613541606526304\\
48.75	0.00613448405545392\\
48.76	0.00613355155094911\\
48.77	0.00613261855146168\\
48.78	0.0061316850567047\\
48.79	0.00613075106639128\\
48.8	0.0061298165802345\\
48.81	0.00612888159794747\\
48.82	0.00612794611924334\\
48.83	0.00612701014383523\\
48.84	0.00612607367143632\\
48.85	0.00612513670175977\\
48.86	0.00612419923451877\\
48.87	0.00612326126942656\\
48.88	0.00612232280619635\\
48.89	0.00612138384454142\\
48.9	0.00612044438417503\\
48.91	0.0061195044248105\\
48.92	0.00611856396616115\\
48.93	0.00611762300794034\\
48.94	0.00611668154986144\\
48.95	0.00611573959163788\\
48.96	0.00611479713298309\\
48.97	0.00611385417361055\\
48.98	0.00611291071323376\\
48.99	0.00611196675156626\\
49	0.00611102228832162\\
49.01	0.00611007732321344\\
49.02	0.00610913185595536\\
49.03	0.00610818588626109\\
49.04	0.00610723941384432\\
49.05	0.00610629243841881\\
49.06	0.00610534495969838\\
49.07	0.00610439697739686\\
49.08	0.00610344849122814\\
49.09	0.00610249950090616\\
49.1	0.00610155000614489\\
49.11	0.00610060000665836\\
49.12	0.00609964950216065\\
49.13	0.00609869849236587\\
49.14	0.0060977469769882\\
49.15	0.00609679495574188\\
49.16	0.00609584242834117\\
49.17	0.00609488939450041\\
49.18	0.006093935853934\\
49.19	0.00609298180635639\\
49.2	0.00609202725148208\\
49.21	0.00609107218902564\\
49.22	0.00609011661870169\\
49.23	0.00608916054022494\\
49.24	0.00608820395331012\\
49.25	0.00608724685767208\\
49.26	0.00608628925302568\\
49.27	0.00608533113908588\\
49.28	0.0060843725155677\\
49.29	0.00608341338218624\\
49.3	0.00608245373865666\\
49.31	0.00608149358469419\\
49.32	0.00608053292001416\\
49.33	0.00607957174433193\\
49.34	0.00607861005736299\\
49.35	0.00607764785882287\\
49.36	0.00607668514842718\\
49.37	0.00607572192589164\\
49.38	0.00607475819093202\\
49.39	0.00607379394326421\\
49.4	0.00607282918260414\\
49.41	0.00607186390866787\\
49.42	0.00607089812117152\\
49.43	0.00606993181983131\\
49.44	0.00606896500436355\\
49.45	0.00606799767448464\\
49.46	0.00606702982991107\\
49.47	0.00606606147035943\\
49.48	0.00606509259554642\\
49.49	0.00606412320518882\\
49.5	0.00606315329900351\\
49.51	0.00606218287670748\\
49.52	0.0060612119380178\\
49.53	0.0060602404826517\\
49.54	0.00605926851032644\\
49.55	0.00605829602075945\\
49.56	0.00605732301366822\\
49.57	0.00605634948877039\\
49.58	0.00605537544578368\\
49.59	0.00605440088442596\\
49.6	0.00605342580441517\\
49.61	0.00605245020546939\\
49.62	0.00605147408730682\\
49.63	0.00605049744964578\\
49.64	0.00604952029220469\\
49.65	0.00604854261470212\\
49.66	0.00604756441685675\\
49.67	0.00604658569838738\\
49.68	0.00604560645901294\\
49.69	0.0060446266984525\\
49.7	0.00604364641642526\\
49.71	0.00604266561265053\\
49.72	0.00604168428684777\\
49.73	0.00604070243873657\\
49.74	0.00603972006803667\\
49.75	0.00603873717446793\\
49.76	0.00603775375775035\\
49.77	0.00603676981760409\\
49.78	0.00603578535374944\\
49.79	0.00603480036590684\\
49.8	0.00603381485379686\\
49.81	0.00603282881714025\\
49.82	0.00603184225565788\\
49.83	0.00603085516907079\\
49.84	0.00602986755710015\\
49.85	0.00602887941946732\\
49.86	0.00602789075589379\\
49.87	0.00602690156610122\\
49.88	0.00602591184981144\\
49.89	0.0060249216067464\\
49.9	0.00602393083662827\\
49.91	0.00602293953917934\\
49.92	0.0060219477141221\\
49.93	0.00602095536117919\\
49.94	0.00601996248007342\\
49.95	0.00601896907052779\\
49.96	0.00601797513226545\\
49.97	0.00601698066500974\\
49.98	0.00601598566848419\\
49.99	0.00601499014241248\\
50	0.0060139940865185\\
50.01	0.00601299750052631\\
50.02	0.00601200038416016\\
50.03	0.00601100273714448\\
50.04	0.00601000455920389\\
50.05	0.00600900585006321\\
50.06	0.00600800660944745\\
50.07	0.00600700683708181\\
50.08	0.00600600653269169\\
50.09	0.00600500569600268\\
50.1	0.00600400432674059\\
50.11	0.00600300242463141\\
50.12	0.00600199998940135\\
50.13	0.00600099702077682\\
50.14	0.00599999351848445\\
50.15	0.00599898948225105\\
50.16	0.00599798491180368\\
50.17	0.00599697980686958\\
50.18	0.00599597416717623\\
50.19	0.00599496799245131\\
50.2	0.00599396128242275\\
50.21	0.00599295403681866\\
50.22	0.00599194625536741\\
50.23	0.00599093793779758\\
50.24	0.00598992908383798\\
50.25	0.00598891969321765\\
50.26	0.00598790976566586\\
50.27	0.00598689930091213\\
50.28	0.00598588829868618\\
50.29	0.00598487675871802\\
50.3	0.00598386468073786\\
50.31	0.00598285206447616\\
50.32	0.00598183890966364\\
50.33	0.00598082521603126\\
50.34	0.00597981098331021\\
50.35	0.00597879621123197\\
50.36	0.00597778089952824\\
50.37	0.00597676504793098\\
50.38	0.00597574865617242\\
50.39	0.00597473172398505\\
50.4	0.00597371425110161\\
50.41	0.00597269623725512\\
50.42	0.00597167768217885\\
50.43	0.00597065858560635\\
50.44	0.00596963894727142\\
50.45	0.00596861876690817\\
50.46	0.00596759804425096\\
50.47	0.00596657677903443\\
50.48	0.0059655549709935\\
50.49	0.00596453261986338\\
50.5	0.00596350972537956\\
50.51	0.0059624862872778\\
50.52	0.0059614623052942\\
50.53	0.00596043777916509\\
50.54	0.00595941270862712\\
50.55	0.00595838709341725\\
50.56	0.00595736093327272\\
50.57	0.00595633422793107\\
50.58	0.00595530697713015\\
50.59	0.00595427918060813\\
50.6	0.00595325083810345\\
50.61	0.00595222194935489\\
50.62	0.00595119251410154\\
50.63	0.00595016253208279\\
50.64	0.00594913200303837\\
50.65	0.00594810092670831\\
50.66	0.00594706930283298\\
50.67	0.00594603713115306\\
50.68	0.00594500441140955\\
50.69	0.00594397114334381\\
50.7	0.0059429373266975\\
50.71	0.00594190296121264\\
50.72	0.00594086804663157\\
50.73	0.00593983258269698\\
50.74	0.0059387965691519\\
50.75	0.00593776000573971\\
50.76	0.00593672289220411\\
50.77	0.00593568522828919\\
50.78	0.00593464701373937\\
50.79	0.00593360824829943\\
50.8	0.00593256893171452\\
50.81	0.00593152906373011\\
50.82	0.00593048864409208\\
50.83	0.00592944767254665\\
50.84	0.00592840614884043\\
50.85	0.00592736407272036\\
50.86	0.0059263214439338\\
50.87	0.00592527826222846\\
50.88	0.00592423452735243\\
50.89	0.00592319023905421\\
50.9	0.00592214539708263\\
50.91	0.00592110000118696\\
50.92	0.00592005405111683\\
50.93	0.00591900754662228\\
50.94	0.00591796048745374\\
50.95	0.00591691287336203\\
50.96	0.00591586470409837\\
50.97	0.0059148159794144\\
50.98	0.00591376669906215\\
50.99	0.00591271686279408\\
51	0.00591166647036303\\
51.01	0.00591061552152229\\
51.02	0.00590956401602555\\
51.03	0.00590851195362693\\
51.04	0.00590745933408095\\
51.05	0.00590640615714258\\
51.06	0.00590535242256723\\
51.07	0.00590429813011071\\
51.08	0.00590324327952928\\
51.09	0.00590218787057966\\
51.1	0.00590113190301897\\
51.11	0.00590007537660481\\
51.12	0.00589901829109521\\
51.13	0.00589796064624865\\
51.14	0.00589690244182408\\
51.15	0.00589584367758088\\
51.16	0.00589478435327892\\
51.17	0.00589372446867849\\
51.18	0.0058926640235404\\
51.19	0.00589160301762588\\
51.2	0.00589054145069667\\
51.21	0.00588947932251495\\
51.22	0.0058884166328434\\
51.23	0.00588735338144519\\
51.24	0.00588628956808394\\
51.25	0.00588522519252379\\
51.26	0.00588416025452936\\
51.27	0.00588309475386575\\
51.28	0.00588202869029858\\
51.29	0.00588096206359397\\
51.3	0.00587989487351851\\
51.31	0.00587882711983933\\
51.32	0.00587775880232406\\
51.33	0.00587668992074084\\
51.34	0.00587562047485834\\
51.35	0.00587455046444572\\
51.36	0.0058734798892727\\
51.37	0.0058724087491095\\
51.38	0.00587133704372689\\
51.39	0.00587026477289614\\
51.4	0.00586919193638909\\
51.41	0.00586811853397811\\
51.42	0.0058670445654361\\
51.43	0.00586597003053653\\
51.44	0.0058648949290534\\
51.45	0.00586381926076128\\
51.46	0.00586274302543528\\
51.47	0.00586166622285107\\
51.48	0.00586058885278491\\
51.49	0.00585951091501361\\
51.5	0.00585843240931454\\
51.51	0.00585735333546565\\
51.52	0.00585627369324549\\
51.53	0.00585519348243316\\
51.54	0.00585411270280838\\
51.55	0.00585303135415141\\
51.56	0.00585194943624316\\
51.57	0.00585086694886508\\
51.58	0.00584978389179926\\
51.59	0.00584870026482838\\
51.6	0.00584761606773573\\
51.61	0.00584653130030519\\
51.62	0.00584544596232129\\
51.63	0.00584436005356915\\
51.64	0.00584327357383453\\
51.65	0.00584218652290379\\
51.66	0.00584109890056395\\
51.67	0.00584001070660264\\
51.68	0.00583892194080814\\
51.69	0.00583783260296936\\
51.7	0.00583674269287586\\
51.71	0.00583565221031786\\
51.72	0.00583456115508621\\
51.73	0.00583346952697243\\
51.74	0.00583237732576869\\
51.75	0.00583128455126784\\
51.76	0.00583019120326338\\
51.77	0.00582909728154949\\
51.78	0.00582800278592103\\
51.79	0.00582690771617352\\
51.8	0.00582581207210318\\
51.81	0.00582471585350693\\
51.82	0.00582361906018236\\
51.83	0.00582252169192775\\
51.84	0.00582142374854211\\
51.85	0.00582032522982512\\
51.86	0.00581922613557719\\
51.87	0.00581812646559944\\
51.88	0.00581702621969368\\
51.89	0.00581592539766248\\
51.9	0.00581482399930911\\
51.91	0.00581372202443758\\
51.92	0.00581261947285261\\
51.93	0.00581151634435969\\
51.94	0.00581041263876502\\
51.95	0.00580930835587557\\
51.96	0.00580820349549904\\
51.97	0.0058070980574439\\
51.98	0.00580599204151936\\
51.99	0.00580488544753541\\
52	0.00580377827530281\\
52.01	0.00580267052463306\\
52.02	0.00580156219533847\\
52.03	0.00580045328723211\\
52.04	0.00579934380012785\\
52.05	0.00579823373384033\\
52.06	0.005797123088185\\
52.07	0.0057960118629781\\
52.08	0.00579490005803666\\
52.09	0.00579378767317853\\
52.1	0.00579267470822239\\
52.11	0.0057915611629877\\
52.12	0.00579044703729475\\
52.13	0.00578933233096468\\
52.14	0.00578821704381942\\
52.15	0.00578710117568176\\
52.16	0.00578598472637533\\
52.17	0.00578486769572459\\
52.18	0.00578375008355486\\
52.19	0.00578263188969229\\
52.2	0.00578151311396392\\
52.21	0.00578039375619763\\
52.22	0.00577927381622217\\
52.23	0.00577815329386717\\
52.24	0.00577703218896312\\
52.25	0.0057759105013414\\
52.26	0.0057747882308343\\
52.27	0.00577366537727496\\
52.28	0.00577254194049744\\
52.29	0.00577141792033668\\
52.3	0.00577029331662857\\
52.31	0.00576916812920985\\
52.32	0.00576804235791823\\
52.33	0.0057669160025923\\
52.34	0.0057657890630716\\
52.35	0.00576466153919659\\
52.36	0.00576353343080866\\
52.37	0.00576240473775015\\
52.38	0.00576127545986436\\
52.39	0.0057601455969955\\
52.4	0.00575901514898878\\
52.41	0.00575788411569034\\
52.42	0.00575675249694729\\
52.43	0.00575562029260774\\
52.44	0.00575448750252073\\
52.45	0.00575335412653632\\
52.46	0.00575222016450554\\
52.47	0.00575108561628041\\
52.48	0.00574995048171396\\
52.49	0.0057488147606602\\
52.5	0.00574767845297418\\
52.51	0.00574654155851193\\
52.52	0.00574540407713053\\
52.53	0.00574426600868807\\
52.54	0.00574312735304365\\
52.55	0.00574198811005744\\
52.56	0.00574084827959064\\
52.57	0.00573970786150547\\
52.58	0.00573856685566524\\
52.59	0.00573742526193429\\
52.6	0.00573628308017803\\
52.61	0.00573514031026295\\
52.62	0.0057339969520566\\
52.63	0.0057328530054276\\
52.64	0.00573170847024569\\
52.65	0.00573056334638166\\
52.66	0.00572941763370743\\
52.67	0.00572827133209599\\
52.68	0.00572712444142147\\
52.69	0.0057259769615591\\
52.7	0.00572482889238522\\
52.71	0.00572368023377731\\
52.72	0.00572253098561396\\
52.73	0.00572138114777493\\
52.74	0.00572023072014109\\
52.75	0.00571907970259449\\
52.76	0.0057179280950183\\
52.77	0.00571677589729688\\
52.78	0.00571562310931574\\
52.79	0.00571446973096157\\
52.8	0.00571331576212224\\
52.81	0.00571216120268679\\
52.82	0.00571100605254547\\
52.83	0.00570985031158972\\
52.84	0.00570869397971218\\
52.85	0.0057075370568067\\
52.86	0.00570637954276834\\
52.87	0.00570522143749341\\
52.88	0.00570406274087941\\
52.89	0.00570290345282508\\
52.9	0.00570174357323044\\
52.91	0.0057005831019967\\
52.92	0.00569942203902635\\
52.93	0.00569826038422316\\
52.94	0.00569709813749214\\
52.95	0.00569593529873956\\
52.96	0.005694771867873\\
52.97	0.00569360784480131\\
52.98	0.00569244322943463\\
52.99	0.00569127802168441\\
53	0.00569011222146338\\
53.01	0.00568894582868561\\
53.02	0.00568777884326646\\
53.03	0.00568661126512265\\
53.04	0.00568544309417219\\
53.05	0.00568427433033446\\
53.06	0.00568310497353017\\
53.07	0.00568193502368138\\
53.08	0.00568076448071152\\
53.09	0.00567959334454537\\
53.1	0.00567842161510909\\
53.11	0.00567724929233022\\
53.12	0.00567607637613767\\
53.13	0.00567490286646177\\
53.14	0.00567372876323423\\
53.15	0.00567255406638816\\
53.16	0.0056713787758581\\
53.17	0.00567020289158001\\
53.18	0.00566902641349125\\
53.19	0.00566784934153066\\
53.2	0.00566667167563847\\
53.21	0.00566549341575641\\
53.22	0.00566431456182762\\
53.23	0.00566313511379673\\
53.24	0.00566195507160984\\
53.25	0.00566077443521451\\
53.26	0.0056595932045598\\
53.27	0.00565841137959625\\
53.28	0.00565722896027593\\
53.29	0.00565604594655237\\
53.3	0.00565486233838064\\
53.31	0.00565367813571734\\
53.32	0.00565249333852057\\
53.33	0.00565130794675\\
53.34	0.00565012196036683\\
53.35	0.00564893537933379\\
53.36	0.0056477482036152\\
53.37	0.00564656043317695\\
53.38	0.00564537206798647\\
53.39	0.0056441831080128\\
53.4	0.00564299355322656\\
53.41	0.00564180340359997\\
53.42	0.00564061265910685\\
53.43	0.00563942131972264\\
53.44	0.00563822938542441\\
53.45	0.00563703685619084\\
53.46	0.00563584373200224\\
53.47	0.0056346500128406\\
53.48	0.00563345569868953\\
53.49	0.00563226078953432\\
53.5	0.00563106528536191\\
53.51	0.00562986918616093\\
53.52	0.00562867249192169\\
53.53	0.0056274752026362\\
53.54	0.00562627731829817\\
53.55	0.005625078838903\\
53.56	0.00562387976444783\\
53.57	0.00562268009493153\\
53.58	0.00562147983035468\\
53.59	0.00562027897071961\\
53.6	0.00561907751603041\\
53.61	0.00561787546629293\\
53.62	0.00561667282151478\\
53.63	0.00561546958170535\\
53.64	0.00561426574687581\\
53.65	0.00561306131703911\\
53.66	0.00561185629221005\\
53.67	0.00561065067240518\\
53.68	0.0056094444576429\\
53.69	0.00560823764794345\\
53.7	0.00560703024332887\\
53.71	0.00560582224382307\\
53.72	0.00560461364945182\\
53.73	0.00560340446024273\\
53.74	0.0056021946762253\\
53.75	0.0056009842974309\\
53.76	0.00559977332389278\\
53.77	0.00559856175564613\\
53.78	0.00559734959272798\\
53.79	0.00559613683517735\\
53.8	0.00559492348303512\\
53.81	0.00559370953634415\\
53.82	0.00559249499514922\\
53.83	0.00559127985949707\\
53.84	0.00559006412943641\\
53.85	0.0055888478050179\\
53.86	0.0055876308862942\\
53.87	0.00558641337331995\\
53.88	0.00558519526615179\\
53.89	0.00558397656484838\\
53.9	0.00558275726947038\\
53.91	0.00558153738008049\\
53.92	0.00558031689674344\\
53.93	0.00557909581952601\\
53.94	0.00557787414849703\\
53.95	0.00557665188372741\\
53.96	0.00557542902529014\\
53.97	0.00557420557326026\\
53.98	0.00557298152771493\\
53.99	0.00557175688873342\\
54	0.0055705316563971\\
54.01	0.00556930583078947\\
54.02	0.00556807941199618\\
54.03	0.00556685240010499\\
54.04	0.00556562479520584\\
54.05	0.00556439659739083\\
54.06	0.00556316780675424\\
54.07	0.00556193842339251\\
54.08	0.0055607084474043\\
54.09	0.00555947787889048\\
54.1	0.00555824671795409\\
54.11	0.00555701496470046\\
54.12	0.0055557826192371\\
54.13	0.00555454968167379\\
54.14	0.00555331615212258\\
54.15	0.00555208203069777\\
54.16	0.00555084731751592\\
54.17	0.00554961201269591\\
54.18	0.00554837611635891\\
54.19	0.00554713962862838\\
54.2	0.00554590254963013\\
54.21	0.00554466487949228\\
54.22	0.0055434266183453\\
54.23	0.005542187766322\\
54.24	0.00554094832355757\\
54.25	0.00553970829018955\\
54.26	0.00553846766635789\\
54.27	0.00553722645220493\\
54.28	0.0055359846478754\\
54.29	0.00553474225351647\\
54.3	0.00553349926927771\\
54.31	0.00553225569531117\\
54.32	0.00553101153177132\\
54.33	0.0055297667788151\\
54.34	0.00552852143660192\\
54.35	0.00552727550529369\\
54.36	0.00552602898505481\\
54.37	0.00552478187605216\\
54.38	0.00552353417845519\\
54.39	0.00552228589243583\\
54.4	0.00552103701816859\\
54.41	0.00551978755583052\\
54.42	0.00551853750560123\\
54.43	0.00551728686766292\\
54.44	0.00551603564220036\\
54.45	0.00551478382940094\\
54.46	0.00551353142945465\\
54.47	0.00551227844255411\\
54.48	0.00551102486889458\\
54.49	0.00550977070867396\\
54.5	0.0055085159620928\\
54.51	0.00550726062935436\\
54.52	0.00550600471066454\\
54.53	0.00550474820623198\\
54.54	0.00550349111626798\\
54.55	0.00550223344098661\\
54.56	0.00550097518060464\\
54.57	0.0054997163353416\\
54.58	0.00549845690541978\\
54.59	0.00549719689106422\\
54.6	0.00549593629250277\\
54.61	0.00549467510996607\\
54.62	0.00549341334368756\\
54.63	0.00549215099390351\\
54.64	0.00549088806085301\\
54.65	0.00548962454477801\\
54.66	0.00548836044592331\\
54.67	0.0054870957645366\\
54.68	0.00548583050086843\\
54.69	0.00548456465517227\\
54.7	0.00548329822770449\\
54.71	0.00548203121872439\\
54.72	0.00548076362849421\\
54.73	0.00547949545727912\\
54.74	0.00547822670534728\\
54.75	0.00547695737296981\\
54.76	0.00547568746042084\\
54.77	0.00547441696797747\\
54.78	0.00547314589591986\\
54.79	0.00547187424453118\\
54.8	0.00547060201409763\\
54.81	0.00546932920490849\\
54.82	0.00546805581725612\\
54.83	0.00546678185143593\\
54.84	0.00546550730774646\\
54.85	0.00546423218648935\\
54.86	0.00546295648796937\\
54.87	0.00546168021249444\\
54.88	0.00546040336037562\\
54.89	0.00545912593192714\\
54.9	0.00545784792746644\\
54.91	0.00545656934731412\\
54.92	0.00545529019179402\\
54.93	0.00545401046123318\\
54.94	0.00545273015596192\\
54.95	0.00545144927631376\\
54.96	0.00545016782262554\\
54.97	0.00544888579523737\\
54.98	0.00544760319449263\\
54.99	0.00544632002073805\\
55	0.00544503627432367\\
55.01	0.00544375195560288\\
55.02	0.00544246706493241\\
55.03	0.00544118160267239\\
55.04	0.00543989556918632\\
55.05	0.0054386089648411\\
55.06	0.00543732179000705\\
55.07	0.00543603404505793\\
55.08	0.00543474573037094\\
55.09	0.00543345684632674\\
55.1	0.00543216739330948\\
55.11	0.00543087737170679\\
55.12	0.00542958678190982\\
55.13	0.00542829562431324\\
55.14	0.00542700389931526\\
55.15	0.00542571160731765\\
55.16	0.00542441874872573\\
55.17	0.00542312532394844\\
55.18	0.0054218313333983\\
55.19	0.00542053677749145\\
55.2	0.00541924165664768\\
55.21	0.0054179459712904\\
55.22	0.00541664972184674\\
55.23	0.00541535290874745\\
55.24	0.00541405553242703\\
55.25	0.00541275759332365\\
55.26	0.00541145909187926\\
55.27	0.00541016002853952\\
55.28	0.00540886040375387\\
55.29	0.00540756021797553\\
55.3	0.00540625947166151\\
55.31	0.00540495816527265\\
55.32	0.00540365629927361\\
55.33	0.0054023538741329\\
55.34	0.00540105089032289\\
55.35	0.00539974734831983\\
55.36	0.00539844324860387\\
55.37	0.0053971385916591\\
55.38	0.0053958333779735\\
55.39	0.00539452760803903\\
55.4	0.0053932212823516\\
55.41	0.00539191440141112\\
55.42	0.0053906069657215\\
55.43	0.00538929897579065\\
55.44	0.00538799043213054\\
55.45	0.00538668133525718\\
55.46	0.00538537168569066\\
55.47	0.00538406148395514\\
55.48	0.00538275073057893\\
55.49	0.00538143942609442\\
55.5	0.00538012757103818\\
55.51	0.0053788151659509\\
55.52	0.0053775022113775\\
55.53	0.00537618870786706\\
55.54	0.00537487465597288\\
55.55	0.00537356005625252\\
55.56	0.00537224490926777\\
55.57	0.00537092921558469\\
55.58	0.00536961297577364\\
55.59	0.00536829619040929\\
55.6	0.00536697886007063\\
55.61	0.00536566098534099\\
55.62	0.00536434256680808\\
55.63	0.00536302360506399\\
55.64	0.0053617041007052\\
55.65	0.00536038405433263\\
55.66	0.00535906346655163\\
55.67	0.005357742337972\\
55.68	0.00535642066920805\\
55.69	0.00535509846087856\\
55.7	0.00535377571360685\\
55.71	0.00535245242802076\\
55.72	0.00535112860475269\\
55.73	0.00534980424443962\\
55.74	0.00534847934772314\\
55.75	0.00534715391524944\\
55.76	0.00534582794766936\\
55.77	0.00534450144563838\\
55.78	0.00534317440981667\\
55.79	0.00534184684086909\\
55.8	0.00534051873946522\\
55.81	0.00533919010627937\\
55.82	0.00533786094199063\\
55.83	0.00533653124728286\\
55.84	0.0053352010228447\\
55.85	0.00533387026936963\\
55.86	0.00533253898755596\\
55.87	0.00533120717810688\\
55.88	0.00532987484173043\\
55.89	0.00532854197913958\\
55.9	0.00532720859105222\\
55.91	0.00532587467819117\\
55.92	0.00532454024128424\\
55.93	0.00532320528106421\\
55.94	0.00532186979826887\\
55.95	0.00532053379364107\\
55.96	0.00531919726792867\\
55.97	0.00531786022188463\\
55.98	0.005316522656267\\
55.99	0.00531518457183896\\
56	0.00531384596936881\\
56.01	0.00531250684963002\\
56.02	0.00531116721340124\\
56.03	0.00530982706146634\\
56.04	0.00530848639461441\\
56.05	0.00530714521363979\\
56.06	0.00530580351934209\\
56.07	0.00530446131252623\\
56.08	0.00530311859400242\\
56.09	0.00530177536458625\\
56.1	0.00530043162509865\\
56.11	0.00529908737636595\\
56.12	0.00529774261921987\\
56.13	0.00529639735449758\\
56.14	0.00529505158304171\\
56.15	0.00529370530570035\\
56.16	0.00529235852332712\\
56.17	0.00529101123678115\\
56.18	0.00528966344692712\\
56.19	0.00528831515463528\\
56.2	0.0052869663607815\\
56.21	0.00528561706624724\\
56.22	0.00528426727191964\\
56.23	0.00528291697869149\\
56.24	0.00528156618746126\\
56.25	0.00528021489913318\\
56.26	0.00527886311461717\\
56.27	0.00527751083482897\\
56.28	0.00527615806069007\\
56.29	0.00527480479312778\\
56.3	0.00527345103307529\\
56.31	0.00527209678147161\\
56.32	0.00527074203926167\\
56.33	0.00526938680739629\\
56.34	0.00526803108683226\\
56.35	0.00526667487853231\\
56.36	0.00526531818346518\\
56.37	0.00526396100260562\\
56.38	0.00526260333693442\\
56.39	0.00526124518743843\\
56.4	0.00525988655511063\\
56.41	0.00525852744095007\\
56.42	0.00525716784596198\\
56.43	0.00525580777115774\\
56.44	0.00525444721755496\\
56.45	0.00525308618617743\\
56.46	0.00525172467805523\\
56.47	0.00525036269422469\\
56.48	0.00524900023572847\\
56.49	0.00524763730361552\\
56.5	0.00524627389894119\\
56.51	0.00524491002276719\\
56.52	0.00524354567616165\\
56.53	0.00524218086019912\\
56.54	0.00524081557596065\\
56.55	0.00523944982453375\\
56.56	0.00523808360701247\\
56.57	0.0052367169244974\\
56.58	0.00523534977809569\\
56.59	0.00523398216892113\\
56.6	0.00523261409809412\\
56.61	0.0052312455667417\\
56.62	0.00522987657599764\\
56.63	0.00522850712700238\\
56.64	0.00522713722090314\\
56.65	0.0052257668588539\\
56.66	0.00522439604201543\\
56.67	0.00522302477155533\\
56.68	0.00522165304864808\\
56.69	0.00522028087447504\\
56.7	0.00521890825022448\\
56.71	0.00521753517709159\\
56.72	0.00521616165627858\\
56.73	0.00521478768899465\\
56.74	0.00521341327645601\\
56.75	0.00521203841988596\\
56.76	0.00521066312051489\\
56.77	0.0052092873795803\\
56.78	0.00520791119832685\\
56.79	0.00520653457800638\\
56.8	0.00520515751987795\\
56.81	0.00520378002520786\\
56.82	0.00520240209526967\\
56.83	0.00520102373134427\\
56.84	0.00519964493471987\\
56.85	0.00519826570669205\\
56.86	0.00519688604856378\\
56.87	0.00519550596164547\\
56.88	0.00519412544725497\\
56.89	0.00519274450671765\\
56.9	0.00519136314136637\\
56.91	0.00518998135254156\\
56.92	0.00518859914159124\\
56.93	0.00518721650987103\\
56.94	0.00518583345874423\\
56.95	0.00518444998958179\\
56.96	0.00518306610376238\\
56.97	0.00518168180267242\\
56.98	0.00518029708770612\\
56.99	0.00517891196026548\\
57	0.00517752642176035\\
57.01	0.00517614047360847\\
57.02	0.00517475411723547\\
57.03	0.00517336735407493\\
57.04	0.0051719801855684\\
57.05	0.00517059261316546\\
57.06	0.00516920463832372\\
57.07	0.00516781626250886\\
57.08	0.00516642748719466\\
57.09	0.00516503831386308\\
57.1	0.00516364874400423\\
57.11	0.00516225877911645\\
57.12	0.00516086842070629\\
57.13	0.00515947767028863\\
57.14	0.00515808652938664\\
57.15	0.00515669499953184\\
57.16	0.00515530308226414\\
57.17	0.00515391077913187\\
57.18	0.00515251809169181\\
57.19	0.00515112502150925\\
57.2	0.00514973157015799\\
57.21	0.00514833773922039\\
57.22	0.00514694353028741\\
57.23	0.00514554894495865\\
57.24	0.00514415398484238\\
57.25	0.00514275865155558\\
57.26	0.00514136294672394\\
57.27	0.00513996687198198\\
57.28	0.005138570428973\\
57.29	0.00513717361934916\\
57.3	0.00513577644477151\\
57.31	0.00513437890691003\\
57.32	0.00513298100744365\\
57.33	0.00513158274806031\\
57.34	0.005130184130457\\
57.35	0.00512878515633977\\
57.36	0.00512738582742378\\
57.37	0.00512598614543335\\
57.38	0.005124586112102\\
57.39	0.00512318572917246\\
57.4	0.00512178499839673\\
57.41	0.00512038392153612\\
57.42	0.00511898250036128\\
57.43	0.00511758073665225\\
57.44	0.00511617863219849\\
57.45	0.00511477618879891\\
57.46	0.00511337340826192\\
57.47	0.00511197029240548\\
57.48	0.00511056684305713\\
57.49	0.00510916306205403\\
57.5	0.00510775895124298\\
57.51	0.00510635451248049\\
57.52	0.00510494974763284\\
57.53	0.00510354465857604\\
57.54	0.00510213924719596\\
57.55	0.00510073351538829\\
57.56	0.00509932746505866\\
57.57	0.00509792109812264\\
57.58	0.00509651441650577\\
57.59	0.00509510742214361\\
57.6	0.00509370011698181\\
57.61	0.00509229250297613\\
57.62	0.00509088458209246\\
57.63	0.00508947635630689\\
57.64	0.00508806782760575\\
57.65	0.00508665899798567\\
57.66	0.00508524986945356\\
57.67	0.00508384044402673\\
57.68	0.00508243072373286\\
57.69	0.00508102071061013\\
57.7	0.00507961040670716\\
57.71	0.00507819981408314\\
57.72	0.00507678893480784\\
57.73	0.00507537777096165\\
57.74	0.00507396632463562\\
57.75	0.00507255459793152\\
57.76	0.00507114259296187\\
57.77	0.00506973031185\\
57.78	0.00506831775673008\\
57.79	0.00506690492974718\\
57.8	0.0050654918330573\\
57.81	0.00506407846882742\\
57.82	0.00506266483923556\\
57.83	0.00506125094647079\\
57.84	0.00505983679273331\\
57.85	0.00505842238023449\\
57.86	0.00505700771119691\\
57.87	0.00505559278785441\\
57.88	0.00505417761245211\\
57.89	0.00505276218724651\\
57.9	0.00505134651450548\\
57.91	0.00504993059650836\\
57.92	0.00504851443554597\\
57.93	0.00504709803392067\\
57.94	0.00504568139394642\\
57.95	0.00504426451794878\\
57.96	0.00504284740826503\\
57.97	0.00504143006724417\\
57.98	0.00504001249724696\\
57.99	0.00503859470064602\\
58	0.00503717667982582\\
58.01	0.00503575843718278\\
58.02	0.00503433997512528\\
58.03	0.00503292129607373\\
58.04	0.00503150240246061\\
58.05	0.00503008329673052\\
58.06	0.00502866398134026\\
58.07	0.00502724445875884\\
58.08	0.00502582473146753\\
58.09	0.00502440480195994\\
58.1	0.00502298467274206\\
58.11	0.0050215643463323\\
58.12	0.00502014382526156\\
58.13	0.00501872311207326\\
58.14	0.0050173022093234\\
58.15	0.00501588111958062\\
58.16	0.00501445984542626\\
58.17	0.00501303838945438\\
58.18	0.00501161675427182\\
58.19	0.0050101949424983\\
58.2	0.00500877295676641\\
58.21	0.00500735079972169\\
58.22	0.0050059284740227\\
58.23	0.00500450598234104\\
58.24	0.00500308332736144\\
58.25	0.00500166051178177\\
58.26	0.00500023753831313\\
58.27	0.0049988144096799\\
58.28	0.00499739112861978\\
58.29	0.00499596769788386\\
58.3	0.00499454412023667\\
58.31	0.00499312039845622\\
58.32	0.0049916966234396\\
58.33	0.00499027280133458\\
58.34	0.00498884893394189\\
58.35	0.00498742502306905\\
58.36	0.0049860010705304\\
58.37	0.00498457707814716\\
58.38	0.00498315304774736\\
58.39	0.00498172898116597\\
58.4	0.00498030488024486\\
58.41	0.00497888074683282\\
58.42	0.00497745658278563\\
58.43	0.00497603238996604\\
58.44	0.00497460817024382\\
58.45	0.00497318392549577\\
58.46	0.00497175965760576\\
58.47	0.00497033536846473\\
58.48	0.00496891105997072\\
58.49	0.00496748673402894\\
58.5	0.00496606239255172\\
58.51	0.00496463803745859\\
58.52	0.00496321367067628\\
58.53	0.00496178929413876\\
58.54	0.00496036490978726\\
58.55	0.00495894051957026\\
58.56	0.00495751612544359\\
58.57	0.00495609172937039\\
58.58	0.00495466733332116\\
58.59	0.00495324293927378\\
58.6	0.00495181854921355\\
58.61	0.00495039416513318\\
58.62	0.00494896978903288\\
58.63	0.0049475454229203\\
58.64	0.00494612106881063\\
58.65	0.0049446967287266\\
58.66	0.00494327240469849\\
58.67	0.00494184809876416\\
58.68	0.00494042381296911\\
58.69	0.00493899954936648\\
58.7	0.00493757531001706\\
58.71	0.00493615109698933\\
58.72	0.00493472691235954\\
58.73	0.00493330275821163\\
58.74	0.00493187863663735\\
58.75	0.00493045454973624\\
58.76	0.00492903049961567\\
58.77	0.00492760648839087\\
58.78	0.00492618251818495\\
58.79	0.00492475859112893\\
58.8	0.00492333470936177\\
58.81	0.00492191087503041\\
58.82	0.00492048709028974\\
58.83	0.00491906335730272\\
58.84	0.00491763967824033\\
58.85	0.00491621605528164\\
58.86	0.00491479249061381\\
58.87	0.00491336898643215\\
58.88	0.00491194554494011\\
58.89	0.00491052216834935\\
58.9	0.00490909885887974\\
58.91	0.00490767561875938\\
58.92	0.00490625245022467\\
58.93	0.00490482935552029\\
58.94	0.00490340633689927\\
58.95	0.00490198339662299\\
58.96	0.00490056053696123\\
58.97	0.00489913776019217\\
58.98	0.00489771506860246\\
58.99	0.0048962924644872\\
59	0.00489486995015003\\
59.01	0.0048934475279031\\
59.02	0.00489202520006714\\
59.03	0.00489060296897146\\
59.04	0.00488918083695403\\
59.05	0.00488775880636142\\
59.06	0.00488633687954893\\
59.07	0.00488491505888056\\
59.08	0.00488349334672905\\
59.09	0.00488207174547594\\
59.1	0.00488065025751154\\
59.11	0.00487922888523503\\
59.12	0.00487780763105444\\
59.13	0.0048763864973867\\
59.14	0.00487496548665767\\
59.15	0.00487354460130218\\
59.16	0.00487212384376402\\
59.17	0.00487070321649606\\
59.18	0.00486928272196017\\
59.19	0.00486786236262733\\
59.2	0.00486644214097764\\
59.21	0.00486502205950032\\
59.22	0.00486360212069382\\
59.23	0.00486218232706577\\
59.24	0.00486076268113303\\
59.25	0.00485934318542178\\
59.26	0.00485792384246748\\
59.27	0.00485650465481492\\
59.28	0.0048550856250183\\
59.29	0.0048536667556412\\
59.3	0.00485224804925666\\
59.31	0.00485082950844715\\
59.32	0.0048494111358047\\
59.33	0.00484799293393085\\
59.34	0.0048465749054367\\
59.35	0.00484515705294298\\
59.36	0.00484373937908004\\
59.37	0.0048423218864879\\
59.38	0.00484090457781631\\
59.39	0.00483948745572473\\
59.4	0.00483807052288241\\
59.41	0.00483665378196839\\
59.42	0.00483523723567158\\
59.43	0.00483382088669074\\
59.44	0.00483240473773455\\
59.45	0.00483098879152164\\
59.46	0.0048295730507806\\
59.47	0.00482815751825005\\
59.48	0.00482674219667866\\
59.49	0.00482532708882517\\
59.5	0.00482391219745846\\
59.51	0.00482249752535755\\
59.52	0.00482108307531164\\
59.53	0.00481966885012017\\
59.54	0.00481825485259285\\
59.55	0.00481684108554967\\
59.56	0.00481542755182095\\
59.57	0.00481401425424739\\
59.58	0.00481260119568009\\
59.59	0.0048111883789806\\
59.6	0.00480977580702093\\
59.61	0.00480836348268362\\
59.62	0.00480695140886175\\
59.63	0.004805539588459\\
59.64	0.00480412802438966\\
59.65	0.00480271671957869\\
59.66	0.00480130567696176\\
59.67	0.00479989489948525\\
59.68	0.00479848439010634\\
59.69	0.00479707415179301\\
59.7	0.00479566418752408\\
59.71	0.00479425450028928\\
59.72	0.00479284509308924\\
59.73	0.00479143596893557\\
59.74	0.00479002713085089\\
59.75	0.00478861858186883\\
59.76	0.00478721032503414\\
59.77	0.00478580236340266\\
59.78	0.00478439470004138\\
59.79	0.0047829873380285\\
59.8	0.00478158028045346\\
59.81	0.00478017353041697\\
59.82	0.00477876709103104\\
59.83	0.00477736096541905\\
59.84	0.00477595515671577\\
59.85	0.00477454966806739\\
59.86	0.00477314450263159\\
59.87	0.00477173966357756\\
59.88	0.00477033515408602\\
59.89	0.00476893097734933\\
59.9	0.00476752713657144\\
59.91	0.004766123634968\\
59.92	0.00476472047576637\\
59.93	0.00476331766220565\\
59.94	0.00476191519753677\\
59.95	0.00476051308502248\\
59.96	0.00475911132793742\\
59.97	0.00475770992956814\\
59.98	0.00475630889321317\\
59.99	0.00475490822218303\\
60	0.00475350791980031\\
60.01	0.00475210798939968\\
60.02	0.00475070843432793\\
60.03	0.00474930925794404\\
60.04	0.00474791046361922\\
60.05	0.00474651205473691\\
60.06	0.00474511403469289\\
60.07	0.00474371640689527\\
60.08	0.00474231917476455\\
60.09	0.00474092234173366\\
60.1	0.00473952591124803\\
60.11	0.00473812988676558\\
60.12	0.00473673427175683\\
60.13	0.00473533906970489\\
60.14	0.00473394428410552\\
60.15	0.0047325499184672\\
60.16	0.00473115597631113\\
60.17	0.00472976246117132\\
60.18	0.0047283693765946\\
60.19	0.00472697672614067\\
60.2	0.00472558451338219\\
60.21	0.00472419274190474\\
60.22	0.00472280141530696\\
60.23	0.00472141053720052\\
60.24	0.0047200201112102\\
60.25	0.00471863014097395\\
60.26	0.0047172406301429\\
60.27	0.00471585158238144\\
60.28	0.00471446300136724\\
60.29	0.00471307489079132\\
60.3	0.00471168725435807\\
60.31	0.00471030009578534\\
60.32	0.00470891341880442\\
60.33	0.00470752722716017\\
60.34	0.00470614152461099\\
60.35	0.00470475631492893\\
60.36	0.00470337160189969\\
60.37	0.00470198738932271\\
60.38	0.00470060368101117\\
60.39	0.00469922048079209\\
60.4	0.00469783779250636\\
60.41	0.00469645562000875\\
60.42	0.00469507396716802\\
60.43	0.00469369283786693\\
60.44	0.00469231223600232\\
60.45	0.0046909321654851\\
60.46	0.0046895526302404\\
60.47	0.00468817363420751\\
60.48	0.00468679518133999\\
60.49	0.00468541727560574\\
60.5	0.00468403992098699\\
60.51	0.00468266312148038\\
60.52	0.00468128688109705\\
60.53	0.00467991120386262\\
60.54	0.00467853609381728\\
60.55	0.00467716155501585\\
60.56	0.0046757875915278\\
60.57	0.00467441420743735\\
60.58	0.00467304140684345\\
60.59	0.0046716691938599\\
60.6	0.00467029757261539\\
60.61	0.0046689265472535\\
60.62	0.00466755612193285\\
60.63	0.00466618630082702\\
60.64	0.00466481708812475\\
60.65	0.00466344848802987\\
60.66	0.00466208050476143\\
60.67	0.00466071314255372\\
60.68	0.00465934640565635\\
60.69	0.00465798029833425\\
60.7	0.0046566148248678\\
60.71	0.00465524998955283\\
60.72	0.00465388579670069\\
60.73	0.00465252225063829\\
60.74	0.0046511593557082\\
60.75	0.00464979711626866\\
60.76	0.00464843553669365\\
60.77	0.00464707462137295\\
60.78	0.0046457143747122\\
60.79	0.00464435480113293\\
60.8	0.00464299590507265\\
60.81	0.0046416376909849\\
60.82	0.00464028016333927\\
60.83	0.00463892332662152\\
60.84	0.00463756718533359\\
60.85	0.00463621174399364\\
60.86	0.00463485700713619\\
60.87	0.00463350297931209\\
60.88	0.00463214966508863\\
60.89	0.00463079706904958\\
60.9	0.00462944519579523\\
60.91	0.0046280940499425\\
60.92	0.00462674363612495\\
60.93	0.00462539395899287\\
60.94	0.0046240450232133\\
60.95	0.00462269683347015\\
60.96	0.00462134939446421\\
60.97	0.00462000271091321\\
60.98	0.00461865678755193\\
60.99	0.00461731162913219\\
61	0.00461596724042297\\
61.01	0.00461462362621046\\
61.02	0.00461328079129807\\
61.03	0.00461193874050655\\
61.04	0.00461059747867405\\
61.05	0.00460925701065614\\
61.06	0.0046079173413259\\
61.07	0.00460657847557399\\
61.08	0.00460524041830868\\
61.09	0.00460390317445594\\
61.1	0.00460256674895952\\
61.11	0.00460123114678095\\
61.12	0.00459989637289967\\
61.13	0.00459856243231306\\
61.14	0.0045972293300365\\
61.15	0.00459589707110345\\
61.16	0.00459456566056551\\
61.17	0.00459323510349248\\
61.18	0.00459190540497245\\
61.19	0.00459057657011179\\
61.2	0.00458924860403533\\
61.21	0.00458792151188631\\
61.22	0.00458659529882654\\
61.23	0.0045852699700364\\
61.24	0.00458394553071495\\
61.25	0.00458262198607996\\
61.26	0.004581299341368\\
61.27	0.0045799776018345\\
61.28	0.00457865677275384\\
61.29	0.00457733685941937\\
61.3	0.00457601786714351\\
61.31	0.00457469980125782\\
61.32	0.00457338266711305\\
61.33	0.00457206647007924\\
61.34	0.00457075121554574\\
61.35	0.00456943690892133\\
61.36	0.00456812355563424\\
61.37	0.00456681116113228\\
61.38	0.00456549973088283\\
61.39	0.004564189270373\\
61.4	0.00456287978510963\\
61.41	0.0045615712806194\\
61.42	0.00456026376244886\\
61.43	0.00455895723616457\\
61.44	0.00455765170735309\\
61.45	0.00455634718162111\\
61.46	0.0045550436645955\\
61.47	0.00455374116192339\\
61.48	0.00455243967927222\\
61.49	0.00455113922232985\\
61.5	0.0045498397968046\\
61.51	0.00454854140842534\\
61.52	0.00454724406294155\\
61.53	0.00454594776612343\\
61.54	0.00454465252376192\\
61.55	0.00454335834166881\\
61.56	0.0045420652256768\\
61.57	0.0045407731816396\\
61.58	0.00453948221543198\\
61.59	0.00453819233294982\\
61.6	0.00453690354011026\\
61.61	0.00453561584285172\\
61.62	0.00453432924713397\\
61.63	0.00453304375893825\\
61.64	0.0045317593842673\\
61.65	0.00453047612914549\\
61.66	0.00452919399961885\\
61.67	0.00452791300175516\\
61.68	0.00452663314164403\\
61.69	0.004525354425397\\
61.7	0.00452407685914758\\
61.71	0.00452280044905136\\
61.72	0.00452152520128609\\
61.73	0.00452025112205171\\
61.74	0.0045189782175705\\
61.75	0.00451770649408711\\
61.76	0.00451643595786866\\
61.77	0.00451516661520483\\
61.78	0.00451389847240793\\
61.79	0.00451263153581294\\
61.8	0.00451136581177768\\
61.81	0.00451010035659968\\
61.82	0.00450883463279916\\
61.83	0.00450756864084738\\
61.84	0.00450630238121686\\
61.85	0.0045050358543814\\
61.86	0.00450376906081604\\
61.87	0.00450250200099712\\
61.88	0.00450123467540221\\
61.89	0.00449996708451019\\
61.9	0.00449869922880116\\
61.91	0.00449743110875655\\
61.92	0.00449616272485904\\
61.93	0.00449489407759257\\
61.94	0.00449362516744238\\
61.95	0.00449235599489498\\
61.96	0.00449108656043816\\
61.97	0.00448981686456101\\
61.98	0.00448854690775388\\
61.99	0.00448727669050843\\
62	0.00448600621331759\\
62.01	0.00448473547667558\\
62.02	0.00448346448107793\\
62.03	0.00448219322702144\\
62.04	0.00448092171500421\\
62.05	0.00447964994552565\\
62.06	0.00447837791908646\\
62.07	0.00447710563618862\\
62.08	0.00447583309733545\\
62.09	0.00447456030303154\\
62.1	0.00447328725378279\\
62.11	0.00447201395009641\\
62.12	0.00447074039248091\\
62.13	0.00446946658144614\\
62.14	0.00446819251750323\\
62.15	0.00446691820116461\\
62.16	0.00446564363294405\\
62.17	0.00446436881335664\\
62.18	0.00446309374291876\\
62.19	0.00446181842214812\\
62.2	0.00446054285156377\\
62.21	0.00445926703168604\\
62.22	0.00445799096303662\\
62.23	0.0044567146461385\\
62.24	0.00445543808151602\\
62.25	0.00445416126969483\\
62.26	0.00445288421120191\\
62.27	0.00445160690656557\\
62.28	0.00445032935631547\\
62.29	0.00444905156098258\\
62.3	0.00444777352109922\\
62.31	0.00444649523719905\\
62.32	0.00444521670981704\\
62.33	0.00444393793948954\\
62.34	0.0044426589267542\\
62.35	0.00444137967215006\\
62.36	0.00444010017621746\\
62.37	0.00443882043949812\\
62.38	0.00443754046253509\\
62.39	0.00443626024587276\\
62.4	0.00443497979005689\\
62.41	0.00443369909563458\\
62.42	0.00443241816315429\\
62.43	0.00443113699316584\\
62.44	0.00442985558622039\\
62.45	0.00442857394287046\\
62.46	0.00442729206366994\\
62.47	0.00442600994917407\\
62.48	0.00442472759993947\\
62.49	0.00442344501652411\\
62.5	0.00442216219948732\\
62.51	0.0044208791493898\\
62.52	0.00441959586679363\\
62.53	0.00441831235226225\\
62.54	0.00441702860636047\\
62.55	0.00441574462965446\\
62.56	0.0044144604227118\\
62.57	0.00441317598610141\\
62.58	0.0044118913203936\\
62.59	0.00441060642616005\\
62.6	0.00440932130397383\\
62.61	0.00440803595440938\\
62.62	0.00440675037804254\\
62.63	0.00440546457545051\\
62.64	0.00440417854721189\\
62.65	0.00440289229390667\\
62.66	0.0044016058161162\\
62.67	0.00440031911442326\\
62.68	0.00439903218941198\\
62.69	0.00439774504166792\\
62.7	0.004396457671778\\
62.71	0.00439517008033055\\
62.72	0.00439388226791529\\
62.73	0.00439259423512334\\
62.74	0.00439130598254722\\
62.75	0.00439001751078085\\
62.76	0.00438872882041955\\
62.77	0.00438743991206003\\
62.78	0.00438615078630042\\
62.79	0.00438486144374025\\
62.8	0.00438357188498044\\
62.81	0.00438228211062336\\
62.82	0.00438099212127274\\
62.83	0.00437970191753373\\
62.84	0.00437841150001293\\
62.85	0.0043771208693183\\
62.86	0.00437583002605925\\
62.87	0.00437453897084658\\
62.88	0.00437324770429253\\
62.89	0.00437195622701073\\
62.9	0.00437066453961626\\
62.91	0.00436937264272559\\
62.92	0.00436808053695663\\
62.93	0.0043667882229287\\
62.94	0.00436549570126255\\
62.95	0.00436420297258036\\
62.96	0.00436291003750573\\
62.97	0.00436161689666367\\
62.98	0.00436032355068065\\
62.99	0.00435903000018455\\
63	0.00435773624580468\\
63.01	0.00435644228817179\\
63.02	0.00435514812791805\\
63.03	0.00435385376567707\\
63.04	0.0043525592020839\\
63.05	0.00435126443777502\\
63.06	0.00434996947338835\\
63.07	0.00434867430956324\\
63.08	0.0043473789469405\\
63.09	0.00434608338616235\\
63.1	0.00434478762787248\\
63.11	0.00434349167271599\\
63.12	0.00434219552133947\\
63.13	0.00434089917439092\\
63.14	0.00433960263251978\\
63.15	0.00433830589637695\\
63.16	0.00433700896661479\\
63.17	0.00433571184388708\\
63.18	0.00433441452884908\\
63.19	0.00433311702215748\\
63.2	0.00433181932447043\\
63.21	0.00433052143644752\\
63.22	0.00432922335874982\\
63.23	0.00432792509203983\\
63.24	0.00432662663698151\\
63.25	0.00432532799424028\\
63.26	0.00432402916448302\\
63.27	0.00432273014837808\\
63.28	0.00432143094659523\\
63.29	0.00432013155980573\\
63.3	0.0043188319886823\\
63.31	0.00431753223389913\\
63.32	0.00431623229613184\\
63.33	0.00431493217605755\\
63.34	0.00431363187435482\\
63.35	0.00431233139170369\\
63.36	0.00431103072878566\\
63.37	0.00430972988628368\\
63.38	0.0043084288648822\\
63.39	0.00430712766526713\\
63.4	0.00430582628812582\\
63.41	0.00430452473414714\\
63.42	0.00430322300402139\\
63.43	0.00430192109844037\\
63.44	0.00430061901809732\\
63.45	0.00429931676368698\\
63.46	0.00429801433590555\\
63.47	0.00429671173545072\\
63.48	0.00429540896302165\\
63.49	0.00429410601931896\\
63.5	0.00429280290504478\\
63.51	0.00429149962090267\\
63.52	0.00429019616759771\\
63.53	0.00428889254583645\\
63.54	0.0042875887563269\\
63.55	0.00428628479977858\\
63.56	0.00428498067690247\\
63.57	0.00428367638841103\\
63.58	0.00428237193501822\\
63.59	0.00428106731743946\\
63.6	0.00427976253639167\\
63.61	0.00427845759259326\\
63.62	0.00427715248676409\\
63.63	0.00427584721962555\\
63.64	0.00427454179190048\\
63.65	0.00427323620431324\\
63.66	0.00427193045758963\\
63.67	0.00427062455245698\\
63.68	0.00426931848964409\\
63.69	0.00426801226988125\\
63.7	0.00426670589390023\\
63.71	0.00426539936243431\\
63.72	0.00426409267621823\\
63.73	0.00426278583598826\\
63.74	0.00426147884248211\\
63.75	0.00426017169643903\\
63.76	0.00425886439859972\\
63.77	0.0042575569497064\\
63.78	0.00425624935050277\\
63.79	0.00425494160173403\\
63.8	0.00425363370414685\\
63.81	0.00425232565848943\\
63.82	0.00425101746551142\\
63.83	0.00424970912596401\\
63.84	0.00424840064059985\\
63.85	0.00424709201017308\\
63.86	0.00424578323543936\\
63.87	0.00424447431715584\\
63.88	0.00424316525608114\\
63.89	0.0042418560529754\\
63.9	0.00424054670860025\\
63.91	0.00423923722371881\\
63.92	0.0042379275990957\\
63.93	0.00423661783549704\\
63.94	0.00423530793369042\\
63.95	0.00423399789444496\\
63.96	0.00423268771853126\\
63.97	0.00423137740672142\\
63.98	0.00423006695978903\\
63.99	0.00422875637850917\\
64	0.00422744566365845\\
64.01	0.00422613481601493\\
64.02	0.00422482383635821\\
64.03	0.00422351272546934\\
64.04	0.00422220148413092\\
64.05	0.004220890113127\\
64.06	0.00421957861324315\\
64.07	0.00421826698526643\\
64.08	0.0042169552299854\\
64.09	0.00421564334819011\\
64.1	0.00421433134067211\\
64.11	0.00421301920822445\\
64.12	0.00421170695164166\\
64.13	0.00421039457171979\\
64.14	0.00420908206925636\\
64.15	0.0042077694450504\\
64.16	0.00420645669990245\\
64.17	0.0042051438346145\\
64.18	0.00420383084999008\\
64.19	0.00420251774683419\\
64.2	0.00420120452595332\\
64.21	0.00419989118815548\\
64.22	0.00419857773425014\\
64.23	0.0041972641650483\\
64.24	0.00419595048136242\\
64.25	0.00419463668400646\\
64.26	0.00419332277379589\\
64.27	0.00419200875154765\\
64.28	0.00419069461808019\\
64.29	0.00418938037421342\\
64.3	0.00418806602076878\\
64.31	0.00418675155856918\\
64.32	0.00418543698843902\\
64.33	0.00418412231120418\\
64.34	0.00418280752769203\\
64.35	0.00418149263873146\\
64.36	0.00418017764515279\\
64.37	0.00417886254778789\\
64.38	0.00417754734747007\\
64.39	0.00417623204503413\\
64.4	0.00417491664131637\\
64.41	0.00417360113715458\\
64.42	0.00417228553338801\\
64.43	0.00417096983085741\\
64.44	0.004169654030405\\
64.45	0.0041683381328745\\
64.46	0.00416702213911108\\
64.47	0.00416570604996141\\
64.48	0.00416438986627365\\
64.49	0.00416307358889741\\
64.5	0.00416175721868379\\
64.51	0.00416044075648537\\
64.52	0.00415912420315621\\
64.53	0.00415780755955182\\
64.54	0.00415649082652921\\
64.55	0.00415517400494685\\
64.56	0.00415385709566468\\
64.57	0.00415254009954413\\
64.58	0.00415122301744806\\
64.59	0.00414990585024084\\
64.6	0.00414858859878828\\
64.61	0.00414727126395766\\
64.62	0.00414595384661775\\
64.63	0.00414463634763874\\
64.64	0.00414331876789233\\
64.65	0.00414200110825164\\
64.66	0.00414068336959127\\
64.67	0.00413936555278729\\
64.68	0.00413804765871721\\
64.69	0.00413672968825999\\
64.7	0.00413541164229609\\
64.71	0.00413409352170736\\
64.72	0.00413277532737714\\
64.73	0.00413145706019023\\
64.74	0.00413013872103285\\
64.75	0.00412882031079269\\
64.76	0.00412750183035889\\
64.77	0.00412618328062201\\
64.78	0.00412486466247407\\
64.79	0.00412354597680854\\
64.8	0.00412222722452032\\
64.81	0.00412090840650575\\
64.82	0.00411958952366261\\
64.83	0.00411827057689012\\
64.84	0.00411695156708894\\
64.85	0.00411563249516113\\
64.86	0.00411431336201021\\
64.87	0.00411299416854114\\
64.88	0.00411167491566027\\
64.89	0.0041103556042754\\
64.9	0.00410903623529576\\
64.91	0.00410771680963199\\
64.92	0.00410639732819614\\
64.93	0.0041050777919017\\
64.94	0.00410375820166357\\
64.95	0.00410243855839805\\
64.96	0.00410111886302286\\
64.97	0.00409979911645715\\
64.98	0.00409847931962146\\
64.99	0.00409715947343772\\
65	0.0040958395788293\\
65.01	0.00409451963672095\\
65.02	0.00409319964803882\\
65.03	0.00409187961371045\\
65.04	0.00409055953466481\\
65.05	0.00408923941183223\\
65.06	0.00408791924614444\\
65.07	0.00408659903853456\\
65.08	0.00408527878993708\\
65.09	0.00408395850128791\\
65.1	0.00408263817352432\\
65.11	0.00408131780758495\\
65.12	0.00407999740440983\\
65.13	0.00407867696494035\\
65.14	0.00407735649011931\\
65.15	0.00407603598089083\\
65.16	0.00407471543820041\\
65.17	0.00407339486299495\\
65.18	0.00407207425622266\\
65.19	0.00407075361883313\\
65.2	0.00406943295177732\\
65.21	0.00406811225600752\\
65.22	0.00406679153247738\\
65.23	0.0040654707821419\\
65.24	0.00406415000595742\\
65.25	0.00406282920488163\\
65.26	0.00406150837987355\\
65.27	0.00406018753189352\\
65.28	0.00405886666190326\\
65.29	0.00405754577086577\\
65.3	0.00405622485974541\\
65.31	0.00405490392950784\\
65.32	0.00405358298112005\\
65.33	0.00405226201555036\\
65.34	0.00405094103376838\\
65.35	0.00404962003674507\\
65.36	0.00404829902545265\\
65.37	0.00404697800086466\\
65.38	0.00404565696395597\\
65.39	0.00404433591570271\\
65.4	0.00404301485708232\\
65.41	0.00404169378907355\\
65.42	0.00404037271265638\\
65.43	0.00403905162881214\\
65.44	0.00403773053852341\\
65.45	0.00403640944277404\\
65.46	0.00403508834254916\\
65.47	0.00403376723883518\\
65.48	0.00403244613261977\\
65.49	0.00403112502489184\\
65.5	0.00402980391664159\\
65.51	0.00402848280886047\\
65.52	0.00402716170254115\\
65.53	0.00402584059867759\\
65.54	0.00402451949826495\\
65.55	0.00402319840229967\\
65.56	0.00402187731177939\\
65.57	0.004020556227703\\
65.58	0.00401923515107061\\
65.59	0.00401791408288356\\
65.6	0.00401659302414439\\
65.61	0.00401527197585688\\
65.62	0.004013950939026\\
65.63	0.00401262991465792\\
65.64	0.00401130890376004\\
65.65	0.00400998790734092\\
65.66	0.00400866692641034\\
65.67	0.00400734596197926\\
65.68	0.00400602501505981\\
65.69	0.00400470408666533\\
65.7	0.0040033831778103\\
65.71	0.00400206228951039\\
65.72	0.00400074142278242\\
65.73	0.00399942057864438\\
65.74	0.00399809975811542\\
65.75	0.00399677896221584\\
65.76	0.00399545819196705\\
65.77	0.00399413744839165\\
65.78	0.00399281673251334\\
65.79	0.00399149604535698\\
65.8	0.00399017538794855\\
65.81	0.00398885476131511\\
65.82	0.00398753416648488\\
65.83	0.00398621360448719\\
65.84	0.00398489307635245\\
65.85	0.00398357258311218\\
65.86	0.003982252125799\\
65.87	0.0039809317054466\\
65.88	0.00397961132308978\\
65.89	0.00397829097976439\\
65.9	0.00397697067650738\\
65.91	0.00397565041435675\\
65.92	0.00397433019435157\\
65.93	0.00397301001753194\\
65.94	0.00397168988493904\\
65.95	0.00397036979761509\\
65.96	0.00396904975660334\\
65.97	0.00396772976294806\\
65.98	0.00396640981769457\\
65.99	0.0039650899218892\\
66	0.0039637700765793\\
66.01	0.00396245028281321\\
66.02	0.00396113054164029\\
66.03	0.00395981085411089\\
66.04	0.00395849122127635\\
66.05	0.00395717164418898\\
66.06	0.0039558521239021\\
66.07	0.00395453266146997\\
66.08	0.00395321325794783\\
66.09	0.00395189391439185\\
66.1	0.0039505746318592\\
66.11	0.00394925541140796\\
66.12	0.00394793625409714\\
66.13	0.00394661716098671\\
66.14	0.00394529813313755\\
66.15	0.00394397917161144\\
66.16	0.0039426602774711\\
66.17	0.00394134145178014\\
66.18	0.00394002269560306\\
66.19	0.00393870401000527\\
66.2	0.00393738539605304\\
66.21	0.00393606685481352\\
66.22	0.00393474838735472\\
66.23	0.00393342999474555\\
66.24	0.00393211167805572\\
66.25	0.00393079343835582\\
66.26	0.00392947527671727\\
66.27	0.00392815719421231\\
66.28	0.00392683919191402\\
66.29	0.00392552127089629\\
66.3	0.0039242034322338\\
66.31	0.00392288567700205\\
66.32	0.00392156800627734\\
66.33	0.00392025042113673\\
66.34	0.00391893292265806\\
66.35	0.00391761551191994\\
66.36	0.00391629819000178\\
66.37	0.00391498095798366\\
66.38	0.00391366381694648\\
66.39	0.00391234676797182\\
66.4	0.00391102981214204\\
66.41	0.00390971295054018\\
66.42	0.00390839618424998\\
66.43	0.00390707951435594\\
66.44	0.00390576294194317\\
66.45	0.00390444646809754\\
66.46	0.00390313009390556\\
66.47	0.0039018138204544\\
66.48	0.0039004976488319\\
66.49	0.00389918158012655\\
66.5	0.00389786561542747\\
66.51	0.00389654975582443\\
66.52	0.00389523400240779\\
66.53	0.00389391835626855\\
66.54	0.0038926028184983\\
66.55	0.00389128739018924\\
66.56	0.00388997207243412\\
66.57	0.00388865686632629\\
66.58	0.00388734177295966\\
66.59	0.00388602679342871\\
66.6	0.00388471192882844\\
66.61	0.00388339718025438\\
66.62	0.00388208254880263\\
66.63	0.00388076803556976\\
66.64	0.00387945364165287\\
66.65	0.00387813936814955\\
66.66	0.00387682521615788\\
66.67	0.0038755111867764\\
66.68	0.00387419728110413\\
66.69	0.00387288350024055\\
66.7	0.00387156984528557\\
66.71	0.00387025631733953\\
66.72	0.0038689429175032\\
66.73	0.00386762964687777\\
66.74	0.00386631650656483\\
66.75	0.00386500349766635\\
66.76	0.00386369062128468\\
66.77	0.00386237787852256\\
66.78	0.00386106527048305\\
66.79	0.0038597527982696\\
66.8	0.00385844046298595\\
66.81	0.00385712826573621\\
66.82	0.00385581620762477\\
66.83	0.00385450428975632\\
66.84	0.00385319251323586\\
66.85	0.00385188087916865\\
66.86	0.00385056938866024\\
66.87	0.00384925804281639\\
66.88	0.00384794684274316\\
66.89	0.00384663578954677\\
66.9	0.00384532488433373\\
66.91	0.00384401412821071\\
66.92	0.00384270352228459\\
66.93	0.00384139306766242\\
66.94	0.00384008276545143\\
66.95	0.00383877261675901\\
66.96	0.00383746262269268\\
66.97	0.00383615278436008\\
66.98	0.00383484310286901\\
66.99	0.00383353357932733\\
67	0.00383222421484303\\
67.01	0.00383091501052414\\
67.02	0.0038296059674788\\
67.03	0.00382829708681516\\
67.04	0.00382698836964144\\
67.05	0.00382567981706588\\
67.06	0.00382437143019671\\
67.07	0.0038230632101422\\
67.08	0.00382175515801055\\
67.09	0.00382044727490999\\
67.1	0.00381913956194867\\
67.11	0.00381783202023469\\
67.12	0.0038165246508761\\
67.13	0.00381521745498082\\
67.14	0.00381391043365673\\
67.15	0.00381260358801154\\
67.16	0.00381129691915287\\
67.17	0.00380999042818818\\
67.18	0.00380868411622478\\
67.19	0.00380737798436979\\
67.2	0.00380607203373019\\
67.21	0.00380476626541269\\
67.22	0.00380346068052385\\
67.23	0.00380215528016994\\
67.24	0.00380085006545702\\
67.25	0.00379954503749088\\
67.26	0.00379824019737702\\
67.27	0.00379693554622067\\
67.28	0.00379563108512671\\
67.29	0.00379432681519973\\
67.3	0.00379302273754396\\
67.31	0.00379171885326329\\
67.32	0.00379041516346122\\
67.33	0.00378911166924085\\
67.34	0.00378780837170491\\
67.35	0.00378650527195568\\
67.36	0.00378520237109499\\
67.37	0.00378389967022426\\
67.38	0.00378259717044439\\
67.39	0.0037812948728558\\
67.4	0.00377999277855842\\
67.41	0.00377869088865165\\
67.42	0.00377738920423435\\
67.43	0.0037760877264048\\
67.44	0.00377478645626073\\
67.45	0.00377348539489926\\
67.46	0.00377218454341693\\
67.47	0.0037708839029096\\
67.48	0.00376958347447252\\
67.49	0.00376828325920026\\
67.5	0.00376698325818672\\
67.51	0.00376568347252508\\
67.52	0.00376438390330782\\
67.53	0.00376308455162666\\
67.54	0.00376178541857257\\
67.55	0.00376048650523576\\
67.56	0.00375918781270562\\
67.57	0.00375788934207075\\
67.58	0.00375659109441889\\
67.59	0.00375529307083697\\
67.6	0.00375399527241098\\
67.61	0.00375269770022609\\
67.62	0.00375140035536651\\
67.63	0.00375010323891555\\
67.64	0.00374880635195554\\
67.65	0.00374750969556787\\
67.66	0.00374621327083292\\
67.67	0.00374491707883005\\
67.68	0.00374362112063762\\
67.69	0.00374232539733289\\
67.7	0.0037410299099921\\
67.71	0.00373973465969037\\
67.72	0.0037384396475017\\
67.73	0.00373714487449894\\
67.74	0.00373585034175384\\
67.75	0.0037345560503369\\
67.76	0.00373326200131747\\
67.77	0.00373196819576366\\
67.78	0.00373067463474234\\
67.79	0.0037293813193191\\
67.8	0.00372808825055826\\
67.81	0.00372679542952283\\
67.82	0.00372550285727448\\
67.83	0.00372421053487352\\
67.84	0.00372291846337889\\
67.85	0.00372162664384815\\
67.86	0.00372033507733738\\
67.87	0.00371904376490127\\
67.88	0.00371775270759303\\
67.89	0.00371646190646434\\
67.9	0.00371517136256541\\
67.91	0.00371388107694488\\
67.92	0.00371259105064985\\
67.93	0.0037113012847258\\
67.94	0.00371001178021664\\
67.95	0.00370872253816459\\
67.96	0.00370743355961026\\
67.97	0.00370614484559254\\
67.98	0.00370485639714863\\
67.99	0.00370356821531398\\
68	0.00370228030112229\\
68.01	0.00370099265560547\\
68.02	0.00369970527979364\\
68.03	0.00369841817471503\\
68.04	0.00369713134139607\\
68.05	0.00369584478086125\\
68.06	0.00369455849413319\\
68.07	0.00369327248223254\\
68.08	0.00369198674617799\\
68.09	0.00369070128698625\\
68.1	0.00368941610567199\\
68.11	0.00368813120324785\\
68.12	0.00368684658072438\\
68.13	0.00368556223911004\\
68.14	0.00368427817941116\\
68.15	0.00368299440263191\\
68.16	0.00368171090977429\\
68.17	0.00368042770183807\\
68.18	0.00367914477982078\\
68.19	0.00367786214471771\\
68.2	0.00367657979752182\\
68.21	0.00367529773922376\\
68.22	0.00367401597081182\\
68.23	0.00367273449327193\\
68.24	0.00367145330758758\\
68.25	0.00367017241473983\\
68.26	0.00366889181570725\\
68.27	0.00366761151146597\\
68.28	0.00366633150298951\\
68.29	0.00366505179124888\\
68.3	0.00366377237721249\\
68.31	0.00366249326184613\\
68.32	0.00366121444611292\\
68.33	0.00365993593097331\\
68.34	0.00365865771738505\\
68.35	0.00365737980630313\\
68.36	0.00365610219867977\\
68.37	0.00365482489546437\\
68.38	0.0036535478976035\\
68.39	0.00365227120604087\\
68.4	0.00365099482171726\\
68.41	0.00364971874557053\\
68.42	0.00364844297853558\\
68.43	0.00364716752154429\\
68.44	0.00364589237552552\\
68.45	0.00364461754140504\\
68.46	0.00364334302010555\\
68.47	0.00364206881254659\\
68.48	0.00364079491964454\\
68.49	0.00363952134231258\\
68.5	0.00363824808146066\\
68.51	0.00363697513799544\\
68.52	0.00363570251282028\\
68.53	0.00363443020683521\\
68.54	0.00363315822093687\\
68.55	0.00363188655601851\\
68.56	0.0036306152129699\\
68.57	0.00362934419267736\\
68.58	0.00362807349602366\\
68.59	0.00362680312388806\\
68.6	0.00362553307714617\\
68.61	0.00362426335667001\\
68.62	0.00362299396332794\\
68.63	0.0036217248979846\\
68.64	0.00362045616150089\\
68.65	0.00361918775473394\\
68.66	0.00361791967853708\\
68.67	0.00361665193375975\\
68.68	0.00361538452124754\\
68.69	0.00361411744184208\\
68.7	0.00361285069638105\\
68.71	0.0036115842856981\\
68.72	0.00361031821062287\\
68.73	0.00360905247198089\\
68.74	0.00360778707059356\\
68.75	0.00360652200727811\\
68.76	0.00360525728284761\\
68.77	0.00360399289811082\\
68.78	0.00360272885387225\\
68.79	0.00360146515093208\\
68.8	0.00360020179008611\\
68.81	0.00359893877212573\\
68.82	0.0035976760978379\\
68.83	0.00359641376800505\\
68.84	0.00359515178340509\\
68.85	0.00359389014481136\\
68.86	0.00359262885299257\\
68.87	0.00359136790871276\\
68.88	0.00359010731273127\\
68.89	0.00358884706580268\\
68.9	0.00358758716867678\\
68.91	0.00358632762209852\\
68.92	0.00358506842680796\\
68.93	0.00358380958354025\\
68.94	0.00358255109302553\\
68.95	0.00358129295598896\\
68.96	0.00358003517315061\\
68.97	0.00357877774522545\\
68.98	0.00357752067292329\\
68.99	0.00357626395694874\\
69	0.00357500759800117\\
69.01	0.00357375159677462\\
69.02	0.0035724959540176\\
69.03	0.00357124067091168\\
69.04	0.00356998574863691\\
69.05	0.00356873118837174\\
69.06	0.0035674769912931\\
69.07	0.00356622315857623\\
69.08	0.0035649696913948\\
69.09	0.00356371659092076\\
69.1	0.00356246385832442\\
69.11	0.00356121149477436\\
69.12	0.00355995950143741\\
69.13	0.00355870787947865\\
69.14	0.00355745663006138\\
69.15	0.00355620575434706\\
69.16	0.00355495525349534\\
69.17	0.00355370512866397\\
69.18	0.00355245538100886\\
69.19	0.00355120601168394\\
69.2	0.00354995702184123\\
69.21	0.00354870841263077\\
69.22	0.00354746018520061\\
69.23	0.00354621234069674\\
69.24	0.00354496488026315\\
69.25	0.0035437178050417\\
69.26	0.00354247111617218\\
69.27	0.0035412248147922\\
69.28	0.00353997890203727\\
69.29	0.00353873337904064\\
69.3	0.00353748824693339\\
69.31	0.00353624350684431\\
69.32	0.00353499915989996\\
69.33	0.00353375520722456\\
69.34	0.00353251164994\\
69.35	0.00353126848916581\\
69.36	0.00353002572601914\\
69.37	0.00352878336161471\\
69.38	0.00352754139706476\\
69.39	0.0035262998334791\\
69.4	0.00352505867196498\\
69.41	0.00352381791362714\\
69.42	0.00352257743880314\\
69.43	0.00352133724319583\\
69.44	0.00352009732923924\\
69.45	0.00351885769937576\\
69.46	0.00351761835605609\\
69.47	0.00351637930173932\\
69.48	0.0035151405388929\\
69.49	0.00351390206999276\\
69.5	0.00351266389752323\\
69.51	0.00351142602397715\\
69.52	0.0035101884518559\\
69.53	0.00350895118366936\\
69.54	0.00350771422193601\\
69.55	0.00350647756918293\\
69.56	0.00350524122794585\\
69.57	0.00350400520076913\\
69.58	0.00350276949020587\\
69.59	0.00350153409881785\\
69.6	0.00350029902917564\\
69.61	0.0034990642838586\\
69.62	0.00349782986545488\\
69.63	0.00349659577656149\\
69.64	0.00349536201978434\\
69.65	0.00349412859773824\\
69.66	0.00349289551304693\\
69.67	0.00349166276834314\\
69.68	0.0034904303662686\\
69.69	0.00348919830947408\\
69.7	0.00348796660061941\\
69.71	0.00348673524237354\\
69.72	0.00348550423741454\\
69.73	0.00348427358842963\\
69.74	0.00348304329811527\\
69.75	0.00348181336917711\\
69.76	0.00348058380433008\\
69.77	0.0034793546062984\\
69.78	0.00347812577781562\\
69.79	0.00347689732162466\\
69.8	0.00347566924047781\\
69.81	0.0034744415371368\\
69.82	0.00347321421437284\\
69.83	0.0034719872749666\\
69.84	0.00347076072170829\\
69.85	0.00346953455739767\\
69.86	0.00346830878484414\\
69.87	0.00346708340686665\\
69.88	0.00346585842629388\\
69.89	0.00346463384596417\\
69.9	0.00346340966872561\\
69.91	0.00346218589743604\\
69.92	0.00346096253496309\\
69.93	0.00345973958418424\\
69.94	0.00345851704798685\\
69.95	0.00345729492926817\\
69.96	0.00345607323093537\\
69.97	0.00345485195590561\\
69.98	0.00345363110710606\\
69.99	0.00345241068747394\\
70	0.00345119069995654\\
70.01	0.00344997114751124\\
70.02	0.00344875203310563\\
70.03	0.00344753335971744\\
70.04	0.00344631513033461\\
70.05	0.00344509734795541\\
70.06	0.00344388001558832\\
70.07	0.00344266313625221\\
70.08	0.00344144671297628\\
70.09	0.00344023074880018\\
70.1	0.00343901524677394\\
70.11	0.00343780020995814\\
70.12	0.0034365856414238\\
70.13	0.00343537154425256\\
70.14	0.00343415792153661\\
70.15	0.00343294477637878\\
70.16	0.00343173211189257\\
70.17	0.00343051993120217\\
70.18	0.00342930823744252\\
70.19	0.00342809703375934\\
70.2	0.00342688632330915\\
70.21	0.00342567610925937\\
70.22	0.00342446639478826\\
70.23	0.00342325718308504\\
70.24	0.00342204847734992\\
70.25	0.00342084028079408\\
70.26	0.00341963259663977\\
70.27	0.00341842542812033\\
70.28	0.00341721877848023\\
70.29	0.00341601265097511\\
70.3	0.0034148070488718\\
70.31	0.00341360197544839\\
70.32	0.00341239743399425\\
70.33	0.00341119342781009\\
70.34	0.00340998996020798\\
70.35	0.00340878703451139\\
70.36	0.00340758465405524\\
70.37	0.00340638282218596\\
70.38	0.00340518154226147\\
70.39	0.0034039808176513\\
70.4	0.00340278065173657\\
70.41	0.00340158104791005\\
70.42	0.00340038200957622\\
70.43	0.00339918354015128\\
70.44	0.00339798564306321\\
70.45	0.00339678832175183\\
70.46	0.00339559157966878\\
70.47	0.00339439542027765\\
70.48	0.00339319984705394\\
70.49	0.00339200486348516\\
70.5	0.00339081047307083\\
70.51	0.00338961667932256\\
70.52	0.00338842348576408\\
70.53	0.00338723089593126\\
70.54	0.00338603891337219\\
70.55	0.00338484754164719\\
70.56	0.00338365678432889\\
70.57	0.00338246664500223\\
70.58	0.00338127712726455\\
70.59	0.00338008823472559\\
70.6	0.00337889997100757\\
70.61	0.0033777123397452\\
70.62	0.00337652534458578\\
70.63	0.00337533898918918\\
70.64	0.00337415327722792\\
70.65	0.00337296821238719\\
70.66	0.00337178379836496\\
70.67	0.00337060003887192\\
70.68	0.00336941693763164\\
70.69	0.00336823449838051\\
70.7	0.00336705272486788\\
70.71	0.00336587162085603\\
70.72	0.00336469119012025\\
70.73	0.00336351143644891\\
70.74	0.00336233236364344\\
70.75	0.00336115397551844\\
70.76	0.0033599762759017\\
70.77	0.00335879926863426\\
70.78	0.00335762295757043\\
70.79	0.00335644734657786\\
70.8	0.00335527243953758\\
70.81	0.00335409824034406\\
70.82	0.00335292475290524\\
70.83	0.00335175198114257\\
70.84	0.0033505799289911\\
70.85	0.00334940860039948\\
70.86	0.00334823799933003\\
70.87	0.00334706812975882\\
70.88	0.00334589899567564\\
70.89	0.00334473060108411\\
70.9	0.00334356295000174\\
70.91	0.00334239604645992\\
70.92	0.00334122989450402\\
70.93	0.00334006449819343\\
70.94	0.00333889986160158\\
70.95	0.00333773598881604\\
70.96	0.00333657288393851\\
70.97	0.00333541055108493\\
70.98	0.00333424899438549\\
70.99	0.0033330882179847\\
71	0.00333192822604142\\
71.01	0.00333076902272894\\
71.02	0.00332961061223501\\
71.03	0.0033284529987619\\
71.04	0.00332729618652645\\
71.05	0.00332614017976011\\
71.06	0.00332498498270902\\
71.07	0.00332383059963403\\
71.08	0.00332267703481077\\
71.09	0.0033215242925297\\
71.1	0.00332037237709616\\
71.11	0.00331922129283043\\
71.12	0.00331807104406775\\
71.13	0.00331692163515845\\
71.14	0.0033157730704679\\
71.15	0.00331462535437664\\
71.16	0.00331347849128042\\
71.17	0.00331233248559022\\
71.18	0.00331118734173234\\
71.19	0.00331004306414845\\
71.2	0.00330889965729562\\
71.21	0.00330775712564638\\
71.22	0.00330661547368882\\
71.23	0.00330547470592657\\
71.24	0.00330433482687892\\
71.25	0.00330319584108085\\
71.26	0.00330205775308306\\
71.27	0.00330092056745206\\
71.28	0.00329978428877024\\
71.29	0.00329864892163588\\
71.3	0.00329751447066321\\
71.31	0.00329638094048252\\
71.32	0.00329524833574015\\
71.33	0.00329411666109859\\
71.34	0.00329298592123653\\
71.35	0.00329185612084889\\
71.36	0.00329072726464692\\
71.37	0.0032895993573582\\
71.38	0.00328847240372678\\
71.39	0.00328734640851314\\
71.4	0.00328622137649433\\
71.41	0.00328509731246398\\
71.42	0.00328397422123238\\
71.43	0.00328285210762653\\
71.44	0.00328173097649018\\
71.45	0.00328061083268395\\
71.46	0.0032794916810853\\
71.47	0.00327837352658868\\
71.48	0.0032772563741055\\
71.49	0.00327614022856428\\
71.5	0.00327502509491063\\
71.51	0.00327391097810736\\
71.52	0.00327279788313451\\
71.53	0.00327168581498945\\
71.54	0.00327057477868688\\
71.55	0.00326946477925895\\
71.56	0.00326835582175529\\
71.57	0.00326724791124308\\
71.58	0.00326614105280708\\
71.59	0.00326503525154974\\
71.6	0.00326393051259125\\
71.61	0.00326282684106956\\
71.62	0.0032617242421405\\
71.63	0.00326062269007743\\
71.64	0.0032595211927667\\
71.65	0.00325841975071102\\
71.66	0.00325731836441211\\
71.67	0.00325621703437073\\
71.68	0.00325511576108665\\
71.69	0.00325401454505862\\
71.7	0.00325291338678441\\
71.71	0.00325181228676074\\
71.72	0.0032507112454833\\
71.73	0.00324961026344674\\
71.74	0.00324850934114464\\
71.75	0.00324740847906952\\
71.76	0.0032463076777128\\
71.77	0.00324520693756483\\
71.78	0.00324410625911483\\
71.79	0.00324300564285091\\
71.8	0.00324190508926003\\
71.81	0.00324080459882805\\
71.82	0.00323970417203962\\
71.83	0.00323860380937825\\
71.84	0.00323750351132626\\
71.85	0.00323640327836478\\
71.86	0.00323530311097372\\
71.87	0.00323420300963179\\
71.88	0.00323310297481644\\
71.89	0.00323200300700391\\
71.9	0.00323090310666915\\
71.91	0.00322980327428585\\
71.92	0.00322870351032641\\
71.93	0.00322760381526193\\
71.94	0.00322650418956221\\
71.95	0.00322540463369571\\
71.96	0.00322430514812955\\
71.97	0.00322320573332952\\
71.98	0.00322210638976002\\
71.99	0.00322100711788408\\
72	0.00321990791816335\\
72.01	0.00321880879105804\\
72.02	0.00321770973702696\\
72.03	0.0032166107565275\\
72.04	0.00321551185001558\\
72.05	0.00321441301794564\\
72.06	0.00321331426077069\\
72.07	0.00321221557894221\\
72.08	0.00321111697291018\\
72.09	0.00321001844312308\\
72.1	0.00320891999002782\\
72.11	0.0032078216140698\\
72.12	0.00320672331569283\\
72.13	0.00320562509533913\\
72.14	0.00320452695344937\\
72.15	0.00320342889046257\\
72.16	0.00320233090681614\\
72.17	0.00320123300294587\\
72.18	0.00320013517928584\\
72.19	0.00319903743626854\\
72.2	0.0031979397743247\\
72.21	0.0031968421938834\\
72.22	0.00319574469537198\\
72.23	0.00319464727921606\\
72.24	0.00319354994583951\\
72.25	0.00319245269566444\\
72.26	0.00319135552911117\\
72.27	0.00319025844659824\\
72.28	0.00318916144854236\\
72.29	0.00318806453535845\\
72.3	0.00318696770745954\\
72.31	0.00318587096525685\\
72.32	0.00318477430915966\\
72.33	0.00318367773957543\\
72.34	0.00318258125690968\\
72.35	0.00318148486156597\\
72.36	0.00318038855394599\\
72.37	0.00317929233444941\\
72.38	0.00317819620347396\\
72.39	0.00317710016141534\\
72.4	0.0031760042086673\\
72.41	0.0031749083456215\\
72.42	0.00317381257266759\\
72.43	0.00317271689019317\\
72.44	0.00317162129858372\\
72.45	0.00317052579822265\\
72.46	0.00316943038949127\\
72.47	0.00316833507276872\\
72.48	0.00316723984843202\\
72.49	0.003166144716856\\
72.5	0.00316504967841333\\
72.51	0.00316395473347446\\
72.52	0.00316285988240761\\
72.53	0.00316176512557878\\
72.54	0.00316067046335169\\
72.55	0.0031595758960878\\
72.56	0.00315848142414627\\
72.57	0.00315738704788393\\
72.58	0.00315629276765528\\
72.59	0.00315519858381249\\
72.6	0.00315410449670534\\
72.61	0.00315301050668122\\
72.62	0.00315191661408509\\
72.63	0.00315082281925954\\
72.64	0.00314972912254464\\
72.65	0.00314863552427804\\
72.66	0.00314754202479487\\
72.67	0.00314644862442778\\
72.68	0.00314535532350687\\
72.69	0.0031442621223597\\
72.7	0.00314316902131126\\
72.71	0.00314207602068396\\
72.72	0.00314098312079759\\
72.73	0.00313989032196932\\
72.74	0.00313879762451366\\
72.75	0.00313770502874246\\
72.76	0.00313661253496488\\
72.77	0.00313552014348734\\
72.78	0.00313442785461358\\
72.79	0.00313333566864453\\
72.8	0.00313224358587838\\
72.81	0.00313115160661052\\
72.82	0.00313005973113349\\
72.83	0.00312896795973704\\
72.84	0.00312787629270803\\
72.85	0.00312678473033044\\
72.86	0.00312569327288533\\
72.87	0.00312460192065087\\
72.88	0.00312351067390225\\
72.89	0.00312241953291169\\
72.9	0.00312132849794844\\
72.91	0.0031202375692787\\
72.92	0.00311914674716566\\
72.93	0.00311805603186943\\
72.94	0.00311696542364705\\
72.95	0.00311587492275244\\
72.96	0.0031147845294364\\
72.97	0.00311369424394656\\
72.98	0.00311260406652738\\
72.99	0.00311151399742013\\
73	0.00311042403686285\\
73.01	0.00310933418509031\\
73.02	0.00310824444233405\\
73.03	0.00310715480882227\\
73.04	0.00310606528477989\\
73.05	0.00310497587042845\\
73.06	0.00310388656598613\\
73.07	0.00310279737166775\\
73.08	0.00310170828768466\\
73.09	0.0031006193142448\\
73.1	0.00309953045155264\\
73.11	0.00309844169980915\\
73.12	0.00309735305921177\\
73.13	0.00309626452995442\\
73.14	0.00309517611222745\\
73.15	0.00309408780621759\\
73.16	0.00309299961210797\\
73.17	0.00309191153007808\\
73.18	0.00309082356030372\\
73.19	0.003089735702957\\
73.2	0.00308864795820631\\
73.21	0.00308756032621628\\
73.22	0.00308647280714777\\
73.23	0.00308538540115782\\
73.24	0.00308429810839967\\
73.25	0.00308321092902268\\
73.26	0.00308212386317231\\
73.27	0.00308103691099014\\
73.28	0.00307995007261379\\
73.29	0.00307886334817692\\
73.3	0.0030777767378092\\
73.31	0.00307669024163625\\
73.32	0.00307560385977966\\
73.33	0.00307451759235695\\
73.34	0.00307343143948152\\
73.35	0.00307234540126261\\
73.36	0.00307125947780534\\
73.37	0.00307017366921059\\
73.38	0.00306908797557508\\
73.39	0.0030680023969912\\
73.4	0.00306691693354711\\
73.41	0.00306583158532665\\
73.42	0.00306474635240932\\
73.43	0.00306366123487023\\
73.44	0.00306257623278012\\
73.45	0.00306149134620527\\
73.46	0.00306040657520753\\
73.47	0.00305932191984425\\
73.48	0.00305823738016823\\
73.49	0.00305715295622775\\
73.5	0.0030560686480665\\
73.51	0.00305498445572355\\
73.52	0.00305390037923334\\
73.53	0.0030528164186256\\
73.54	0.00305173257392538\\
73.55	0.00305064884515299\\
73.56	0.00304956523232394\\
73.57	0.00304848173544898\\
73.58	0.00304739835453399\\
73.59	0.00304631508957997\\
73.6	0.00304523194058308\\
73.61	0.00304414890753446\\
73.62	0.00304306599042035\\
73.63	0.00304198318922196\\
73.64	0.00304090050391548\\
73.65	0.00303981793447201\\
73.66	0.00303873548085757\\
73.67	0.00303765314303304\\
73.68	0.00303657092095412\\
73.69	0.00303548881457132\\
73.7	0.00303440682382992\\
73.71	0.00303332494866989\\
73.72	0.00303224318902593\\
73.73	0.00303116154482739\\
73.74	0.00303008001599822\\
73.75	0.00302899860245699\\
73.76	0.00302791730411679\\
73.77	0.00302683612088526\\
73.78	0.00302575505266448\\
73.79	0.003024674099351\\
73.8	0.00302359326083577\\
73.81	0.00302251253700411\\
73.82	0.00302143192773568\\
73.83	0.00302035143290441\\
73.84	0.00301927105237853\\
73.85	0.00301819078602046\\
73.86	0.00301711063368683\\
73.87	0.00301603059522837\\
73.88	0.00301495067048998\\
73.89	0.00301387085931058\\
73.9	0.00301279116152316\\
73.91	0.00301171157695466\\
73.92	0.00301063210542602\\
73.93	0.00300955274675207\\
73.94	0.0030084735007415\\
73.95	0.00300739436719687\\
73.96	0.00300631534591452\\
73.97	0.00300523643668456\\
73.98	0.00300415763929078\\
73.99	0.00300307895351067\\
74	0.00300200037911537\\
74.01	0.00300092191586959\\
74.02	0.00299984356353162\\
74.03	0.00299876532185323\\
74.04	0.00299768719057971\\
74.05	0.00299660916944975\\
74.06	0.00299553125819545\\
74.07	0.00299445345654227\\
74.08	0.00299337576420897\\
74.09	0.0029922981809076\\
74.1	0.00299122070634344\\
74.11	0.00299014334021492\\
74.12	0.00298906608221368\\
74.13	0.00298798893202443\\
74.14	0.00298691188932494\\
74.15	0.00298583495378603\\
74.16	0.00298475812507148\\
74.17	0.002983681402838\\
74.18	0.00298260478673521\\
74.19	0.00298152827640557\\
74.2	0.00298045187148437\\
74.21	0.00297937557159963\\
74.22	0.0029782993763721\\
74.23	0.00297722328541523\\
74.24	0.00297614729833508\\
74.25	0.00297507141473029\\
74.26	0.00297399563419207\\
74.27	0.0029729199563041\\
74.28	0.00297184438064255\\
74.29	0.00297076890677594\\
74.3	0.0029696935342652\\
74.31	0.00296861826266357\\
74.32	0.00296754309151654\\
74.33	0.00296646802036184\\
74.34	0.00296539304872937\\
74.35	0.00296431817614117\\
74.36	0.00296324340211135\\
74.37	0.00296216872614608\\
74.38	0.00296109414774349\\
74.39	0.00296001966639368\\
74.4	0.00295894528157861\\
74.41	0.00295787099277213\\
74.42	0.00295679679943986\\
74.43	0.00295572270103916\\
74.44	0.00295464869701913\\
74.45	0.00295357478682048\\
74.46	0.00295250096987556\\
74.47	0.00295142724560825\\
74.48	0.00295035361343394\\
74.49	0.0029492800727595\\
74.5	0.00294820662298316\\
74.51	0.00294713326349455\\
74.52	0.00294605999367459\\
74.53	0.00294498681289546\\
74.54	0.00294391372052052\\
74.55	0.00294284071590431\\
74.56	0.00294176779839247\\
74.57	0.00294069496732168\\
74.58	0.00293962222201962\\
74.59	0.00293854956180493\\
74.6	0.00293747698598713\\
74.61	0.00293640449386659\\
74.62	0.00293533208473446\\
74.63	0.00293425975787264\\
74.64	0.00293318751255369\\
74.65	0.00293211534804084\\
74.66	0.00293104326358786\\
74.67	0.00292997125843905\\
74.68	0.00292889933182918\\
74.69	0.00292782748298343\\
74.7	0.00292675571111735\\
74.71	0.00292568401543678\\
74.72	0.00292461239513782\\
74.73	0.00292354084940673\\
74.74	0.00292246937741994\\
74.75	0.00292139797834395\\
74.76	0.00292032665133528\\
74.77	0.00291925539554042\\
74.78	0.00291818421009578\\
74.79	0.00291711309412759\\
74.8	0.00291604204675191\\
74.81	0.00291497106707452\\
74.82	0.00291390015419088\\
74.83	0.00291282930718608\\
74.84	0.00291175852513477\\
74.85	0.00291068780710109\\
74.86	0.00290961715213864\\
74.87	0.00290854655929041\\
74.88	0.00290747602758868\\
74.89	0.00290640555605504\\
74.9	0.00290533514370025\\
74.91	0.00290426478952425\\
74.92	0.00290319449251602\\
74.93	0.00290212425165359\\
74.94	0.00290105406590396\\
74.95	0.00289998393422301\\
74.96	0.00289891385555545\\
74.97	0.00289784382883479\\
74.98	0.00289677385298324\\
74.99	0.00289570392691166\\
75	0.00289463404951948\\
75.01	0.00289356421969468\\
75.02	0.00289249443631369\\
75.03	0.00289142469824132\\
75.04	0.00289035500433074\\
75.05	0.00288928535342335\\
75.06	0.00288821574434879\\
75.07	0.00288714617592479\\
75.08	0.00288607664695719\\
75.09	0.00288500715623981\\
75.1	0.00288393770255441\\
75.11	0.00288286828467064\\
75.12	0.00288179890134594\\
75.13	0.00288072955132548\\
75.14	0.0028796602333421\\
75.15	0.00287859094611627\\
75.16	0.00287752168835595\\
75.17	0.00287645245875661\\
75.18	0.00287538325600107\\
75.19	0.00287431407875952\\
75.2	0.00287324492568937\\
75.21	0.00287217579543526\\
75.22	0.00287110668662891\\
75.23	0.0028700375978891\\
75.24	0.00286896852782159\\
75.25	0.00286789947501905\\
75.26	0.00286683043806097\\
75.27	0.00286576141551362\\
75.28	0.00286469240592996\\
75.29	0.00286362340784953\\
75.3	0.00286255441979848\\
75.31	0.00286148544028938\\
75.32	0.00286041646782122\\
75.33	0.0028593475008793\\
75.34	0.00285827853793519\\
75.35	0.00285720957744664\\
75.36	0.00285614061785748\\
75.37	0.00285507165759757\\
75.38	0.00285400269508274\\
75.39	0.00285293372871468\\
75.4	0.00285186475688087\\
75.41	0.00285079577795454\\
75.42	0.00284972679029454\\
75.43	0.0028486577922453\\
75.44	0.00284758878213673\\
75.45	0.00284651975828417\\
75.46	0.00284545071898827\\
75.47	0.00284438166253496\\
75.48	0.00284331258719533\\
75.49	0.00284224349122557\\
75.5	0.00284117437286689\\
75.51	0.00284010523034544\\
75.52	0.00283903606187221\\
75.53	0.00283796686564297\\
75.54	0.00283689763983821\\
75.55	0.00283582838262299\\
75.56	0.00283475909214693\\
75.57	0.00283368976654408\\
75.58	0.00283262040393286\\
75.59	0.00283155100241599\\
75.6	0.00283048156008034\\
75.61	0.00282941207499695\\
75.62	0.00282834254522085\\
75.63	0.00282727296879103\\
75.64	0.00282620334373033\\
75.65	0.00282513366804536\\
75.66	0.00282406393972645\\
75.67	0.00282299415674748\\
75.68	0.00282192431706589\\
75.69	0.00282085441862251\\
75.7	0.00281978445934153\\
75.71	0.00281871443713041\\
75.72	0.00281764434987971\\
75.73	0.00281657419546314\\
75.74	0.00281550397173734\\
75.75	0.00281443367654186\\
75.76	0.00281336330769907\\
75.77	0.00281229286301404\\
75.78	0.00281122234027447\\
75.79	0.00281015173725058\\
75.8	0.00280908105169505\\
75.81	0.00280801028134288\\
75.82	0.00280693942391137\\
75.83	0.00280586847709993\\
75.84	0.00280479743859008\\
75.85	0.00280372630604529\\
75.86	0.00280265507711094\\
75.87	0.00280158374941417\\
75.88	0.0028005123205638\\
75.89	0.00279944078815029\\
75.9	0.00279836914974556\\
75.91	0.00279729740290296\\
75.92	0.00279622554515712\\
75.93	0.00279515357402389\\
75.94	0.00279408148700024\\
75.95	0.00279300928394473\\
75.96	0.00279193696484594\\
75.97	0.00279086452969186\\
75.98	0.00278979197846995\\
75.99	0.00278871931116706\\
76	0.00278764652776951\\
76.01	0.00278657362826304\\
76.02	0.00278550061263281\\
76.03	0.00278442748086342\\
76.04	0.00278335423293894\\
76.05	0.00278228086884283\\
76.06	0.00278120738855803\\
76.07	0.00278013379206689\\
76.08	0.00277906007935124\\
76.09	0.0027779862503923\\
76.1	0.00277691230517079\\
76.11	0.00277583824366684\\
76.12	0.00277476406586006\\
76.13	0.00277368977172948\\
76.14	0.00277261536125359\\
76.15	0.00277154083441034\\
76.16	0.00277046619117712\\
76.17	0.00276939143153081\\
76.18	0.0027683165554477\\
76.19	0.00276724156290358\\
76.2	0.00276616645387367\\
76.21	0.00276509122833268\\
76.22	0.00276401588625476\\
76.23	0.00276294042761356\\
76.24	0.00276186485238216\\
76.25	0.00276078916053313\\
76.26	0.00275971335203852\\
76.27	0.00275863742686983\\
76.28	0.00275756138499806\\
76.29	0.00275648522639368\\
76.3	0.00275540895102663\\
76.31	0.00275433255886636\\
76.32	0.00275325604988176\\
76.33	0.00275217942404126\\
76.34	0.00275110268131273\\
76.35	0.00275002582166358\\
76.36	0.00274894884506066\\
76.37	0.00274787175147036\\
76.38	0.00274679454085854\\
76.39	0.00274571721319058\\
76.4	0.00274463976843136\\
76.41	0.00274356220654524\\
76.42	0.00274248452749613\\
76.43	0.00274140673124742\\
76.44	0.00274032881776203\\
76.45	0.00273925078700237\\
76.46	0.0027381726389304\\
76.47	0.00273709437350759\\
76.48	0.00273601599069493\\
76.49	0.00273493749045294\\
76.5	0.00273385887274167\\
76.51	0.0027327801375207\\
76.52	0.00273170128474914\\
76.53	0.00273062231438565\\
76.54	0.00272954322638843\\
76.55	0.00272846402071522\\
76.56	0.0027273846973233\\
76.57	0.00272630525616952\\
76.58	0.00272522569721027\\
76.59	0.0027241460204015\\
76.6	0.00272306622569872\\
76.61	0.00272198631305701\\
76.62	0.002720906282431\\
76.63	0.0027198261337749\\
76.64	0.00271874586704251\\
76.65	0.00271766548218719\\
76.66	0.00271658497916188\\
76.67	0.0027155043579191\\
76.68	0.00271442361841098\\
76.69	0.00271334276058923\\
76.7	0.00271226178440513\\
76.71	0.00271118068980961\\
76.72	0.00271009947675316\\
76.73	0.0027090181451859\\
76.74	0.00270793669505756\\
76.75	0.00270685512631747\\
76.76	0.00270577343891457\\
76.77	0.00270469163279748\\
76.78	0.00270360970791437\\
76.79	0.00270252766421311\\
76.8	0.00270144550164114\\
76.81	0.0027003632201456\\
76.82	0.00269928081967324\\
76.83	0.00269819830017046\\
76.84	0.00269711566158332\\
76.85	0.00269603290385754\\
76.86	0.0026949500269385\\
76.87	0.00269386703077123\\
76.88	0.00269278391530046\\
76.89	0.00269170068047058\\
76.9	0.00269061732622565\\
76.91	0.00268953385250943\\
76.92	0.00268845025926537\\
76.93	0.0026873665464366\\
76.94	0.00268628271396598\\
76.95	0.00268519876179603\\
76.96	0.00268411468986903\\
76.97	0.00268303049812693\\
76.98	0.00268194618651143\\
76.99	0.00268086175496395\\
77	0.00267977720342564\\
77.01	0.00267869253183736\\
77.02	0.00267760774013976\\
77.03	0.00267652282827321\\
77.04	0.00267543779617782\\
77.05	0.00267435264379348\\
77.06	0.00267326737105984\\
77.07	0.00267218197791631\\
77.08	0.00267109646430209\\
77.09	0.00267001083015614\\
77.1	0.00266892507541722\\
77.11	0.0026678392000239\\
77.12	0.00266675320391452\\
77.13	0.00266566708702724\\
77.14	0.00266458084930002\\
77.15	0.00266349449067066\\
77.16	0.00266240801107676\\
77.17	0.00266132141045577\\
77.18	0.00266023468874495\\
77.19	0.00265914784588143\\
77.2	0.00265806088180219\\
77.21	0.00265697379644403\\
77.22	0.00265588658974367\\
77.23	0.00265479926163765\\
77.24	0.0026537118120624\\
77.25	0.00265262424095424\\
77.26	0.0026515365482494\\
77.27	0.00265044873388396\\
77.28	0.00264936079779393\\
77.29	0.00264827273991526\\
77.3	0.00264718456018377\\
77.31	0.00264609625853523\\
77.32	0.00264500783490534\\
77.33	0.00264391928922974\\
77.34	0.00264283062144403\\
77.35	0.00264174183148375\\
77.36	0.00264065291928441\\
77.37	0.00263956388478149\\
77.38	0.00263847472791045\\
77.39	0.00263738544860675\\
77.4	0.00263629604680582\\
77.41	0.00263520652244311\\
77.42	0.00263411687545409\\
77.43	0.00263302710577422\\
77.44	0.00263193721333902\\
77.45	0.00263084719808402\\
77.46	0.00262975705994481\\
77.47	0.00262866679885704\\
77.48	0.00262757641475641\\
77.49	0.00262648590757867\\
77.5	0.0026253952772597\\
77.51	0.00262430452373543\\
77.52	0.00262321364694188\\
77.53	0.00262212264681521\\
77.54	0.00262103152329168\\
77.55	0.00261994027630766\\
77.56	0.00261884890579966\\
77.57	0.00261775741170435\\
77.58	0.00261666579395854\\
77.59	0.0026155740524992\\
77.6	0.00261448218726347\\
77.61	0.00261339019818867\\
77.62	0.00261229808521232\\
77.63	0.00261120584827213\\
77.64	0.00261011348730604\\
77.65	0.00260902100225218\\
77.66	0.00260792839304894\\
77.67	0.00260683565963492\\
77.68	0.002605742801949\\
77.69	0.00260464981993031\\
77.7	0.00260355671351825\\
77.71	0.00260246348265251\\
77.72	0.00260137012727306\\
77.73	0.00260027664732019\\
77.74	0.00259918304273449\\
77.75	0.0025980893134569\\
77.76	0.00259699545942865\\
77.77	0.00259590148059138\\
77.78	0.00259480737688702\\
77.79	0.00259371314825795\\
77.8	0.00259261879464685\\
77.81	0.00259152431599687\\
77.82	0.0025904297122515\\
77.83	0.00258933498335468\\
77.84	0.00258824012925079\\
77.85	0.00258714514988461\\
77.86	0.00258605004520142\\
77.87	0.00258495481514693\\
77.88	0.00258385945966734\\
77.89	0.00258276397870933\\
77.9	0.00258166837222011\\
77.91	0.00258057264014736\\
77.92	0.00257947678243933\\
77.93	0.00257838079904478\\
77.94	0.00257728468991304\\
77.95	0.002576188454994\\
77.96	0.00257509209423812\\
77.97	0.00257399560759646\\
77.98	0.00257289899502069\\
77.99	0.00257180225646308\\
78	0.00257070539187656\\
78.01	0.00256960840121469\\
78.02	0.00256851128443168\\
78.03	0.00256741404148242\\
78.04	0.00256631667232251\\
78.05	0.00256521917690821\\
78.06	0.00256412155519652\\
78.07	0.00256302380714518\\
78.08	0.00256192593271265\\
78.09	0.00256082793185816\\
78.1	0.0025597298045417\\
78.11	0.00255863155072408\\
78.12	0.0025575331703669\\
78.13	0.00255643466343254\\
78.14	0.00255533602988427\\
78.15	0.00255423726968617\\
78.16	0.00255313838280321\\
78.17	0.00255203936920121\\
78.18	0.00255094022884691\\
78.19	0.00254984096170795\\
78.2	0.0025487415677529\\
78.21	0.00254764204695127\\
78.22	0.00254654239927352\\
78.23	0.0025454426246911\\
78.24	0.00254434272317644\\
78.25	0.002543242694703\\
78.26	0.00254214253924522\\
78.27	0.00254104225677862\\
78.28	0.00253994184727977\\
78.29	0.00253884131072631\\
78.3	0.00253774064709698\\
78.31	0.0025366398563716\\
78.32	0.00253553893853115\\
78.33	0.00253443789355775\\
78.34	0.00253333672143467\\
78.35	0.00253223542214638\\
78.36	0.0025311339956785\\
78.37	0.00253003244201794\\
78.38	0.00252893076115277\\
78.39	0.00252782895307236\\
78.4	0.00252672701776734\\
78.41	0.00252562495522963\\
78.42	0.00252452276545245\\
78.43	0.00252342044843035\\
78.44	0.00252231800415925\\
78.45	0.00252121543263639\\
78.46	0.00252011273386044\\
78.47	0.00251900990783145\\
78.48	0.00251790695455091\\
78.49	0.00251680387402174\\
78.5	0.00251570066624831\\
78.51	0.00251459733123653\\
78.52	0.00251349386899375\\
78.53	0.00251239027952889\\
78.54	0.0025112865628524\\
78.55	0.00251018271897629\\
78.56	0.00250907874791417\\
78.57	0.00250797464968125\\
78.58	0.00250687042429437\\
78.59	0.00250576607177203\\
78.6	0.00250466159213439\\
78.61	0.00250355698540331\\
78.62	0.00250245225160238\\
78.63	0.0025013473907569\\
78.64	0.00250024240289394\\
78.65	0.00249913728804238\\
78.66	0.00249803204623286\\
78.67	0.00249692667749788\\
78.68	0.00249582118187179\\
78.69	0.00249471555939078\\
78.7	0.00249360981009298\\
78.71	0.00249250393401842\\
78.72	0.00249139793120907\\
78.73	0.00249029180170888\\
78.74	0.00248918554556378\\
78.75	0.00248807916282171\\
78.76	0.00248697265353265\\
78.77	0.00248586601774868\\
78.78	0.00248475925552392\\
78.79	0.00248365236691463\\
78.8	0.00248254535197919\\
78.81	0.00248143821077817\\
78.82	0.00248033094337429\\
78.83	0.00247922354983252\\
78.84	0.00247811603022005\\
78.85	0.00247700838460633\\
78.86	0.00247590061306313\\
78.87	0.00247479271566449\\
78.88	0.00247368469248685\\
78.89	0.00247257654360897\\
78.9	0.00247146826911203\\
78.91	0.00247035986907964\\
78.92	0.00246925134362118\\
78.93	0.00246814269299274\\
78.94	0.0024670339174503\\
78.95	0.0024659250172498\\
78.96	0.0024648159926471\\
78.97	0.00246370684389801\\
78.98	0.00246259757125824\\
78.99	0.00246148817498344\\
79	0.00246037865532918\\
79.01	0.00245926901255094\\
79.02	0.00245815924690416\\
79.03	0.00245704935864414\\
79.04	0.00245593934802612\\
79.05	0.00245482921530527\\
79.06	0.00245371896073665\\
79.07	0.0024526085845752\\
79.08	0.00245149808707583\\
79.09	0.00245038746849329\\
79.1	0.00244927672908227\\
79.11	0.00244816586909734\\
79.12	0.00244705488879298\\
79.13	0.00244594378842354\\
79.14	0.00244483256824327\\
79.15	0.00244372122850634\\
79.16	0.00244260976946675\\
79.17	0.00244149819137844\\
79.18	0.00244038649449518\\
79.19	0.00243927467907066\\
79.2	0.00243816274535843\\
79.21	0.00243705069361191\\
79.22	0.00243593852408439\\
79.23	0.00243482623702905\\
79.24	0.00243371383269891\\
79.25	0.00243260131134687\\
79.26	0.00243148867322569\\
79.27	0.00243037591858799\\
79.28	0.00242926304768625\\
79.29	0.00242815006077279\\
79.3	0.0024270369580998\\
79.31	0.0024259237399193\\
79.32	0.00242481040648317\\
79.33	0.00242369695804314\\
79.34	0.00242258339485076\\
79.35	0.00242146971715745\\
79.36	0.00242035592521442\\
79.37	0.00241924201927278\\
79.38	0.0024181279995834\\
79.39	0.00241701386639703\\
79.4	0.00241589961996422\\
79.41	0.00241478526053535\\
79.42	0.00241367078836063\\
79.43	0.00241255620369007\\
79.44	0.00241144150677351\\
79.45	0.00241032669786059\\
79.46	0.00240921177720078\\
79.47	0.00240809674504334\\
79.48	0.00240698160163733\\
79.49	0.00240586634723163\\
79.5	0.0024047509820749\\
79.51	0.0024036355064156\\
79.52	0.00240251992050201\\
79.53	0.00240140422458215\\
79.54	0.00240028841890388\\
79.55	0.0023991725037148\\
79.56	0.00239805647926232\\
79.57	0.00239694034579362\\
79.58	0.00239582410355567\\
79.59	0.00239470775279518\\
79.6	0.00239359129375868\\
79.61	0.0023924747266924\\
79.62	0.00239135805184241\\
79.63	0.00239024126945448\\
79.64	0.00238912437977419\\
79.65	0.00238800738304682\\
79.66	0.00238689027951744\\
79.67	0.00238577306943087\\
79.68	0.00238465575303167\\
79.69	0.00238353833056414\\
79.7	0.00238242080227233\\
79.71	0.00238130316840001\\
79.72	0.00238018542919069\\
79.73	0.00237906758488764\\
79.74	0.00237794963573382\\
79.75	0.00237683158197194\\
79.76	0.00237571342384442\\
79.77	0.00237459516159341\\
79.78	0.00237347679546077\\
79.79	0.00237235832568806\\
79.8	0.00237123975251658\\
79.81	0.00237012107618731\\
79.82	0.00236900229694093\\
79.83	0.00236788341501786\\
79.84	0.00236676443065818\\
79.85	0.00236564534410168\\
79.86	0.00236452615558781\\
79.87	0.00236340686535576\\
79.88	0.00236228747364437\\
79.89	0.00236116798069216\\
79.9	0.00236004838673735\\
79.91	0.0023589286920178\\
79.92	0.00235780889677108\\
79.93	0.00235668900123439\\
79.94	0.00235556900564463\\
79.95	0.00235444891023833\\
79.96	0.0023533287152517\\
79.97	0.00235220842092058\\
79.98	0.00235108802748049\\
79.99	0.00234996753516659\\
80	0.00234884694421366\\
80.01	0.00234772625485614\\
};
\addplot [color=green,dashed]
  table[row sep=crcr]{%
80.01	0.00234772625485614\\
80.02	0.00234660546732811\\
80.03	0.00234548458186328\\
80.04	0.00234436359869498\\
80.05	0.00234324251805617\\
80.06	0.00234212134017945\\
80.07	0.00234100006529703\\
80.08	0.00233987869364071\\
80.09	0.00233875722544194\\
80.1	0.00233763566093177\\
80.11	0.00233651400034083\\
80.12	0.00233539224389938\\
80.13	0.00233427039183727\\
80.14	0.00233314844438394\\
80.15	0.00233202640176842\\
80.16	0.00233090426421934\\
80.17	0.00232978203196488\\
80.18	0.00232865970523285\\
80.19	0.00232753728425058\\
80.2	0.00232641476924501\\
80.21	0.00232529216044265\\
80.22	0.00232416945806955\\
80.23	0.00232304666235132\\
80.24	0.00232192377351317\\
80.25	0.00232080079177981\\
80.26	0.00231967771737552\\
80.27	0.00231855455052415\\
80.28	0.00231743129144905\\
80.29	0.00231630794037312\\
80.3	0.00231518449751881\\
80.31	0.00231406096310809\\
80.32	0.00231293733736246\\
80.33	0.00231181362050293\\
80.34	0.00231068981275003\\
80.35	0.00230956591432381\\
80.36	0.00230844192544384\\
80.37	0.00230731784632918\\
80.38	0.00230619367719838\\
80.39	0.00230506941826954\\
80.4	0.00230394506976017\\
80.41	0.00230282063188736\\
80.42	0.00230169610486762\\
80.43	0.00230057148891697\\
80.44	0.00229944678425091\\
80.45	0.00229832199108439\\
80.46	0.00229719710963185\\
80.47	0.00229607214010719\\
80.48	0.00229494708272377\\
80.49	0.0022938219376944\\
80.5	0.00229269670523134\\
80.51	0.00229157138554632\\
80.52	0.0022904459788505\\
80.53	0.00228932048535447\\
80.54	0.00228819490526825\\
80.55	0.00228706923880132\\
80.56	0.00228594348616257\\
80.57	0.00228481764756031\\
80.58	0.00228369172320225\\
80.59	0.00228256571329555\\
80.6	0.00228143961804676\\
80.61	0.00228031343766183\\
80.62	0.00227918717234611\\
80.63	0.00227806082230436\\
80.64	0.0022769343877407\\
80.65	0.00227580786885868\\
80.66	0.0022746812658612\\
80.67	0.00227355457895053\\
80.68	0.00227242780832835\\
80.69	0.00227130095419567\\
80.7	0.00227017401675287\\
80.71	0.00226904699619973\\
80.72	0.00226791989273531\\
80.73	0.0022667927065581\\
80.74	0.00226566543786586\\
80.75	0.00226453808685576\\
80.76	0.00226341065372425\\
80.77	0.00226228313866713\\
80.78	0.00226115554187953\\
80.79	0.0022600278635559\\
80.8	0.00225890010389001\\
80.81	0.00225777226307493\\
80.82	0.00225664434130303\\
80.83	0.00225551633876602\\
80.84	0.00225438825565485\\
80.85	0.00225326009215981\\
80.86	0.00225213184847046\\
80.87	0.00225100352477563\\
80.88	0.00224987512126345\\
80.89	0.0022487466381213\\
80.9	0.00224761807553585\\
80.91	0.00224648943369299\\
80.92	0.00224536071277793\\
80.93	0.00224423191297509\\
80.94	0.00224310303446812\\
80.95	0.00224197407743996\\
80.96	0.00224084504207275\\
80.97	0.00223971592854789\\
80.98	0.00223858673704598\\
80.99	0.00223745746774684\\
81	0.00223632812082953\\
81.01	0.00223519869647231\\
81.02	0.00223406919485264\\
81.03	0.00223293961614717\\
81.04	0.00223180996053177\\
81.05	0.00223068022818149\\
81.06	0.00222955041927055\\
81.07	0.00222842053397237\\
81.08	0.00222729057245952\\
81.09	0.00222616053490376\\
81.1	0.00222503042147601\\
81.11	0.00222390023234633\\
81.12	0.00222276996768394\\
81.13	0.0022216396276572\\
81.14	0.00222050921243364\\
81.15	0.00221937872217989\\
81.16	0.00221824815706173\\
81.17	0.00221711751724404\\
81.18	0.00221598680289086\\
81.19	0.00221485601416529\\
81.2	0.00221372515122959\\
81.21	0.00221259421424507\\
81.22	0.00221146320337218\\
81.23	0.00221033211877043\\
81.24	0.00220920096059843\\
81.25	0.00220806972901385\\
81.26	0.00220693842417347\\
81.27	0.0022058070462331\\
81.28	0.00220467559534761\\
81.29	0.00220354407167095\\
81.3	0.00220241247535611\\
81.31	0.00220128080655511\\
81.32	0.00220014906541903\\
81.33	0.00219901725209796\\
81.34	0.00219788536674103\\
81.35	0.00219675340949637\\
81.36	0.00219562138051115\\
81.37	0.00219448927993154\\
81.38	0.00219335710790268\\
81.39	0.00219222486456874\\
81.4	0.00219109255007288\\
81.41	0.00218996016455723\\
81.42	0.00218882770816289\\
81.43	0.00218769518102993\\
81.44	0.00218656258329741\\
81.45	0.00218542991510332\\
81.46	0.00218429717658461\\
81.47	0.00218316436787719\\
81.48	0.00218203148911588\\
81.49	0.00218089854043446\\
81.5	0.00217976552196563\\
81.51	0.00217863243384098\\
81.52	0.00217749927619106\\
81.53	0.0021763660491453\\
81.54	0.00217523275283204\\
81.55	0.00217409938737849\\
81.56	0.0021729659529108\\
81.57	0.00217183244955394\\
81.58	0.00217069887743179\\
81.59	0.00216956523666709\\
81.6	0.00216843152738144\\
81.61	0.00216729774969531\\
81.62	0.00216616390372798\\
81.63	0.0021650299895976\\
81.64	0.00216389600742115\\
81.65	0.00216276195731444\\
81.66	0.0021616278393921\\
81.67	0.00216049365376755\\
81.68	0.00215935940055305\\
81.69	0.00215822507985966\\
81.7	0.00215709069179719\\
81.71	0.00215595623647429\\
81.72	0.00215482171399836\\
81.73	0.00215368712447557\\
81.74	0.00215255246801087\\
81.75	0.00215141774470796\\
81.76	0.00215028295466929\\
81.77	0.00214914809799607\\
81.78	0.00214801317478821\\
81.79	0.0021468781851444\\
81.8	0.00214574312916201\\
81.81	0.00214460800693714\\
81.82	0.00214347281856461\\
81.83	0.00214233756413793\\
81.84	0.00214120224374928\\
81.85	0.00214006685748959\\
81.86	0.0021389314054484\\
81.87	0.00213779588771396\\
81.88	0.00213666030437317\\
81.89	0.0021355246555116\\
81.9	0.00213438894121346\\
81.91	0.00213325316156159\\
81.92	0.0021321173166375\\
81.93	0.00213098140652128\\
81.94	0.00212984543129167\\
81.95	0.00212870939102603\\
81.96	0.00212757328580029\\
81.97	0.00212643711568899\\
81.98	0.00212530088076527\\
81.99	0.00212416458110084\\
82	0.002123028216766\\
82.01	0.00212189178782957\\
82.02	0.00212075529435899\\
82.03	0.0021196187364202\\
82.04	0.0021184821140777\\
82.05	0.00211734542739453\\
82.06	0.00211620867643224\\
82.07	0.0021150718612509\\
82.08	0.00211393498190912\\
82.09	0.00211279803846396\\
82.1	0.00211166103097101\\
82.11	0.00211052395948434\\
82.12	0.00210938682405649\\
82.13	0.00210824962473847\\
82.14	0.00210711236157976\\
82.15	0.00210597503462828\\
82.16	0.0021048376439304\\
82.17	0.00210370018953091\\
82.18	0.00210256267147308\\
82.19	0.00210142508979854\\
82.2	0.00210028744454735\\
82.21	0.00209914973575799\\
82.22	0.00209801196346731\\
82.23	0.00209687412771057\\
82.24	0.00209573622852138\\
82.25	0.00209459826593174\\
82.26	0.002093460239972\\
82.27	0.00209232215067087\\
82.28	0.00209118399805538\\
82.29	0.00209004578215092\\
82.3	0.0020889075029812\\
82.31	0.00208776916056823\\
82.32	0.00208663075493235\\
82.33	0.00208549228609219\\
82.34	0.00208435375406465\\
82.35	0.00208321515886495\\
82.36	0.00208207650050655\\
82.37	0.00208093777900119\\
82.38	0.00207979899435886\\
82.39	0.00207866014658778\\
82.4	0.00207752123569444\\
82.41	0.00207638226168354\\
82.42	0.00207524322455797\\
82.43	0.00207410412431886\\
82.44	0.00207296496096556\\
82.45	0.00207182573449555\\
82.46	0.00207068644490455\\
82.47	0.00206954709218641\\
82.48	0.00206840767633317\\
82.49	0.002067268197335\\
82.5	0.00206612865518024\\
82.51	0.00206498904985533\\
82.52	0.00206384938134487\\
82.53	0.00206270964963155\\
82.54	0.00206156985469618\\
82.55	0.00206042999651766\\
82.56	0.00205929007507297\\
82.57	0.00205815009033718\\
82.58	0.00205701004228342\\
82.59	0.00205586993088287\\
82.6	0.00205472975610477\\
82.61	0.00205358951791639\\
82.62	0.00205244921628302\\
82.63	0.002051308851168\\
82.64	0.00205016842253264\\
82.65	0.00204902793033626\\
82.66	0.00204788737453618\\
82.67	0.00204674675508765\\
82.68	0.00204560607194384\\
82.69	0.00204446532505582\\
82.7	0.00204332451437256\\
82.71	0.00204218363984093\\
82.72	0.00204104270140564\\
82.73	0.0020399016990093\\
82.74	0.00203876063259235\\
82.75	0.00203761950209309\\
82.76	0.00203647830744762\\
82.77	0.00203533704858991\\
82.78	0.00203419572545171\\
82.79	0.00203305433796256\\
82.8	0.00203191288604984\\
82.81	0.00203077136963865\\
82.82	0.00202962978865189\\
82.83	0.00202848814301022\\
82.84	0.00202734643263204\\
82.85	0.00202620465743346\\
82.86	0.00202506281732836\\
82.87	0.00202392091222831\\
82.88	0.00202277894204258\\
82.89	0.00202163690667813\\
82.9	0.00202049480603962\\
82.91	0.00201935264002935\\
82.92	0.0020182104085473\\
82.93	0.00201706811149109\\
82.94	0.00201592574875597\\
82.95	0.00201478332023482\\
82.96	0.00201364082581813\\
82.97	0.00201249826539401\\
82.98	0.00201135563884811\\
82.99	0.00201021294606371\\
83	0.00200907018692164\\
83.01	0.00200792736130027\\
83.02	0.00200678446907552\\
83.03	0.00200564151012087\\
83.04	0.00200449848430728\\
83.05	0.00200335539150324\\
83.06	0.00200221223157474\\
83.07	0.00200106900438523\\
83.08	0.00199992570979565\\
83.09	0.00199878234766442\\
83.1	0.00199763891784737\\
83.11	0.0019964954201978\\
83.12	0.0019953518545664\\
83.13	0.00199420822080131\\
83.14	0.00199306451874805\\
83.15	0.00199192074824954\\
83.16	0.00199077690914606\\
83.17	0.00198963300127526\\
83.18	0.00198848902447216\\
83.19	0.00198734497856908\\
83.2	0.00198620086339571\\
83.21	0.00198505667877903\\
83.22	0.00198391242454332\\
83.23	0.00198276810051016\\
83.24	0.0019816237064984\\
83.25	0.00198047924232417\\
83.26	0.00197933470780083\\
83.27	0.00197819010273898\\
83.28	0.00197704542694647\\
83.29	0.00197590068022834\\
83.3	0.00197475586238684\\
83.31	0.00197361097322139\\
83.32	0.00197246601252861\\
83.33	0.00197132098010228\\
83.34	0.0019701758757333\\
83.35	0.00196903069920974\\
83.36	0.00196788545031678\\
83.37	0.00196674012883671\\
83.38	0.00196559473454889\\
83.39	0.0019644492672298\\
83.4	0.00196330372665298\\
83.41	0.00196215811258901\\
83.42	0.00196101242480554\\
83.43	0.00195986666306722\\
83.44	0.00195872082713573\\
83.45	0.00195757491676976\\
83.46	0.00195642893172497\\
83.47	0.00195528287175401\\
83.48	0.00195413673660648\\
83.49	0.00195299052602894\\
83.5	0.00195184423976488\\
83.51	0.0019506978775547\\
83.52	0.00194955143913571\\
83.53	0.00194840492424213\\
83.54	0.00194725833260504\\
83.55	0.00194611166395238\\
83.56	0.00194496491800895\\
83.57	0.00194381809449638\\
83.58	0.00194267119313314\\
83.59	0.00194152421363449\\
83.6	0.00194037715571248\\
83.61	0.00193923001907594\\
83.62	0.0019380828034305\\
83.63	0.00193693550847848\\
83.64	0.00193578813391897\\
83.65	0.00193464067944779\\
83.66	0.00193349314475745\\
83.67	0.00193234552953715\\
83.68	0.00193119783347279\\
83.69	0.0019300500562469\\
83.7	0.00192890219753867\\
83.71	0.00192775425702395\\
83.72	0.00192660623437518\\
83.73	0.00192545812926139\\
83.74	0.00192430994134823\\
83.75	0.00192316167029791\\
83.76	0.0019220133157692\\
83.77	0.00192086487741741\\
83.78	0.00191971635489438\\
83.79	0.00191856774784847\\
83.8	0.00191741905592452\\
83.81	0.00191627027876386\\
83.82	0.0019151214160043\\
83.83	0.00191397246728008\\
83.84	0.00191282343222189\\
83.85	0.00191167431045684\\
83.86	0.00191052510160843\\
83.87	0.00190937580529657\\
83.88	0.00190822642113752\\
83.89	0.00190707694874392\\
83.9	0.00190592738772473\\
83.91	0.00190477773768527\\
83.92	0.00190362799822712\\
83.93	0.00190247816894819\\
83.94	0.00190132824944267\\
83.95	0.00190017823930099\\
83.96	0.00189902813810983\\
83.97	0.00189787794545213\\
83.98	0.00189672766090699\\
83.99	0.00189557728404975\\
84	0.00189442681445193\\
84.01	0.00189327625168117\\
84.02	0.00189212559530129\\
84.03	0.00189097484487227\\
84.04	0.00188982399995015\\
84.05	0.00188867306008709\\
84.06	0.00188752202483135\\
84.07	0.00188637089372721\\
84.08	0.00188521966631505\\
84.09	0.00188406834213124\\
84.1	0.00188291692070818\\
84.11	0.00188176540157428\\
84.12	0.0018806137842539\\
84.13	0.00187946206826739\\
84.14	0.00187831025313102\\
84.15	0.00187715833835701\\
84.16	0.00187600632345349\\
84.17	0.00187485420792447\\
84.18	0.00187370199126984\\
84.19	0.00187254967298537\\
84.2	0.00187139725256264\\
84.21	0.00187024472948907\\
84.22	0.00186909210324789\\
84.23	0.00186793937331811\\
84.24	0.00186678653917451\\
84.25	0.00186563360028764\\
84.26	0.00186448055612376\\
84.27	0.00186332740614487\\
84.28	0.00186217414980866\\
84.29	0.00186102078656849\\
84.3	0.0018598673158734\\
84.31	0.00185871373716808\\
84.32	0.00185756004989283\\
84.33	0.00185640625348357\\
84.34	0.0018552523473718\\
84.35	0.0018540983309846\\
84.36	0.0018529442037446\\
84.37	0.00185178996506997\\
84.38	0.0018506356143744\\
84.39	0.00184948115106707\\
84.4	0.00184832657455263\\
84.41	0.00184717188423121\\
84.42	0.00184601707949838\\
84.43	0.00184486215974511\\
84.44	0.00184370712435781\\
84.45	0.00184255197271824\\
84.46	0.00184139670420355\\
84.47	0.00184024131818621\\
84.48	0.00183908581403407\\
84.49	0.00183793019111022\\
84.5	0.0018367744487731\\
84.51	0.00183561858637637\\
84.52	0.00183446260326897\\
84.53	0.00183330649879508\\
84.54	0.00183215027229406\\
84.55	0.00183099392310048\\
84.56	0.00182983745054408\\
84.57	0.00182868085394975\\
84.58	0.00182752413263751\\
84.59	0.0018263672859225\\
84.6	0.00182521031311495\\
84.61	0.00182405321352016\\
84.62	0.00182289598643849\\
84.63	0.00182173863116532\\
84.64	0.00182058114699105\\
84.65	0.00181942353320108\\
84.66	0.00181826578907575\\
84.67	0.0018171079138904\\
84.68	0.00181594990691526\\
84.69	0.00181479176741549\\
84.7	0.00181363349465114\\
84.71	0.00181247508787712\\
84.72	0.00181131654634318\\
84.73	0.00181015786929394\\
84.74	0.00180899905596878\\
84.75	0.00180784010560188\\
84.76	0.0018066810174222\\
84.77	0.00180552179065342\\
84.78	0.00180436242451397\\
84.79	0.00180320291821695\\
84.8	0.00180204327097018\\
84.81	0.0018008834819761\\
84.82	0.00179972355043181\\
84.83	0.00179856347552902\\
84.84	0.00179740325645404\\
84.85	0.00179624289238775\\
84.86	0.00179508238250558\\
84.87	0.00179392172597749\\
84.88	0.00179276092196795\\
84.89	0.00179159996963592\\
84.9	0.00179043886813481\\
84.91	0.00178927761661249\\
84.92	0.00178811621421125\\
84.93	0.00178695466006777\\
84.94	0.0017857929533131\\
84.95	0.00178463109307265\\
84.96	0.00178346907846618\\
84.97	0.00178230690860774\\
84.98	0.00178114458260565\\
84.99	0.00177998209956254\\
85	0.00177881945857525\\
85.01	0.00177765665873484\\
85.02	0.00177649369912656\\
85.03	0.00177533057882987\\
85.04	0.00177416729691835\\
85.05	0.00177300385245971\\
85.06	0.00177184024451577\\
85.07	0.00177067647214242\\
85.08	0.00176951253438964\\
85.09	0.00176834843030141\\
85.1	0.00176718415891574\\
85.11	0.00176601971926463\\
85.12	0.00176485511037404\\
85.13	0.00176369033126387\\
85.14	0.00176252538094794\\
85.15	0.00176136025843398\\
85.16	0.00176019496272355\\
85.17	0.00175902949281211\\
85.18	0.0017578638476889\\
85.19	0.00175669802633699\\
85.2	0.00175553202773321\\
85.21	0.00175436585084814\\
85.22	0.00175319949464609\\
85.23	0.00175203295808509\\
85.24	0.00175086624011681\\
85.25	0.00174969933968662\\
85.26	0.00174853225573349\\
85.27	0.00174736498719\\
85.28	0.00174619753298231\\
85.29	0.00174502989203015\\
85.3	0.00174386206324678\\
85.31	0.00174269404553895\\
85.32	0.00174152583780691\\
85.33	0.00174035743894435\\
85.34	0.00173918884783841\\
85.35	0.00173802006336963\\
85.36	0.00173685108441192\\
85.37	0.00173568190983258\\
85.38	0.0017345125384922\\
85.39	0.0017333429692447\\
85.4	0.00173217320093729\\
85.41	0.00173100323241041\\
85.42	0.00172983306249775\\
85.43	0.0017286626900262\\
85.44	0.00172749211381581\\
85.45	0.00172632133267981\\
85.46	0.00172515034542454\\
85.47	0.00172397915084945\\
85.48	0.00172280774774707\\
85.49	0.00172163613490295\\
85.5	0.0017204643110957\\
85.51	0.00171929227509689\\
85.52	0.00171812002567111\\
85.53	0.00171694756157584\\
85.54	0.00171577488189922\\
85.55	0.00171460198644673\\
85.56	0.00171342887502622\\
85.57	0.00171225554744797\\
85.58	0.00171108200352467\\
85.59	0.00170990824307149\\
85.6	0.00170873426590615\\
85.61	0.00170756007184885\\
85.62	0.0017063856607224\\
85.63	0.00170521103235054\\
85.64	0.00170403618655775\\
85.65	0.00170286112316926\\
85.66	0.00170168584201108\\
85.67	0.00170051034291\\
85.68	0.00169933462569362\\
85.69	0.00169815869019037\\
85.7	0.00169698253622949\\
85.71	0.00169580616364111\\
85.72	0.00169462957225618\\
85.73	0.00169345276190657\\
85.74	0.00169227573242504\\
85.75	0.00169109848364525\\
85.76	0.00168992101540181\\
85.77	0.00168874332753027\\
85.78	0.00168756541986714\\
85.79	0.00168638729224994\\
85.8	0.00168520894451715\\
85.81	0.00168403037650829\\
85.82	0.00168285158806391\\
85.83	0.0016816725790256\\
85.84	0.00168049334923603\\
85.85	0.00167931389853894\\
85.86	0.0016781342267792\\
85.87	0.00167695433380278\\
85.88	0.00167577421945679\\
85.89	0.0016745938835895\\
85.9	0.00167341332605036\\
85.91	0.00167223254669\\
85.92	0.00167105154536028\\
85.93	0.00166987032191429\\
85.94	0.00166868887620634\\
85.95	0.00166750720809205\\
85.96	0.0016663253174283\\
85.97	0.0016651432040733\\
85.98	0.00166396086788655\\
85.99	0.00166277830872892\\
86	0.00166159552646266\\
86.01	0.00166041252095138\\
86.02	0.0016592292920601\\
86.03	0.00165804583965527\\
86.04	0.00165686216360478\\
86.05	0.001655678263778\\
86.06	0.00165449414004576\\
86.07	0.00165330979228044\\
86.08	0.0016521252203559\\
86.09	0.00165094042414758\\
86.1	0.00164975540353248\\
86.11	0.00164857015838918\\
86.12	0.00164738468859791\\
86.13	0.00164619899404049\\
86.14	0.00164501307460043\\
86.15	0.00164382693016289\\
86.16	0.00164264056061476\\
86.17	0.00164145396584463\\
86.18	0.00164026714574284\\
86.19	0.0016390801002015\\
86.2	0.00163789282911452\\
86.21	0.00163670533237761\\
86.22	0.00163551760988832\\
86.23	0.00163432966154605\\
86.24	0.0016331414872521\\
86.25	0.00163195308690968\\
86.26	0.00163076446042391\\
86.27	0.00162957560770187\\
86.28	0.00162838652865262\\
86.29	0.00162719722318724\\
86.3	0.0016260076912188\\
86.31	0.00162481793266246\\
86.32	0.00162362794743543\\
86.33	0.00162243773545704\\
86.34	0.00162124729664872\\
86.35	0.00162005663093408\\
86.36	0.00161886573823888\\
86.37	0.00161767461849112\\
86.38	0.00161648327162099\\
86.39	0.00161529169756095\\
86.4	0.00161409989624576\\
86.41	0.00161290786761245\\
86.42	0.0016117156116004\\
86.43	0.00161052312815136\\
86.44	0.00160933041720946\\
86.45	0.00160813747872124\\
86.46	0.00160694431263569\\
86.47	0.00160575091890426\\
86.48	0.00160455729748091\\
86.49	0.00160336344832211\\
86.5	0.00160216937138689\\
86.51	0.00160097506663685\\
86.52	0.00159978053403624\\
86.53	0.00159858577355188\\
86.54	0.00159739078515333\\
86.55	0.0015961955688128\\
86.56	0.00159500012450522\\
86.57	0.00159380445220831\\
86.58	0.00159260855190256\\
86.59	0.00159141242357126\\
86.6	0.00159021606720056\\
86.61	0.00158901948277947\\
86.62	0.00158782267029994\\
86.63	0.0015866256297568\\
86.64	0.00158542836114789\\
86.65	0.00158423086447405\\
86.66	0.00158303313973911\\
86.67	0.00158183518694999\\
86.68	0.00158063700611671\\
86.69	0.0015794385972524\\
86.7	0.00157823996037334\\
86.71	0.00157704109549901\\
86.72	0.00157584200265211\\
86.73	0.00157464268185859\\
86.74	0.00157344313314769\\
86.75	0.00157224335655197\\
86.76	0.00157104335210735\\
86.77	0.00156984311985313\\
86.78	0.00156864265983203\\
86.79	0.00156744197209023\\
86.8	0.00156624105667741\\
86.81	0.00156503991364674\\
86.82	0.001563838543055\\
86.83	0.00156263694496253\\
86.84	0.00156143511943331\\
86.85	0.00156023306653499\\
86.86	0.00155903078633889\\
86.87	0.00155782827892013\\
86.88	0.00155662554435754\\
86.89	0.0015554225827338\\
86.9	0.00155421939413541\\
86.91	0.00155301597865279\\
86.92	0.00155181233638025\\
86.93	0.00155060846741605\\
86.94	0.00154940437185914\\
86.95	0.00154820004980893\\
86.96	0.00154699550136527\\
86.97	0.00154579072662851\\
86.98	0.00154458572569944\\
86.99	0.00154338049867934\\
87	0.00154217504566994\\
87.01	0.00154096936677342\\
87.02	0.00153976346209245\\
87.03	0.00153855733173015\\
87.04	0.0015373509757901\\
87.05	0.00153614439437634\\
87.06	0.00153493758759337\\
87.07	0.00153373055554615\\
87.08	0.00153252329834009\\
87.09	0.00153131581608108\\
87.1	0.00153010810887544\\
87.11	0.00152890017682994\\
87.12	0.00152769202005184\\
87.13	0.00152648363864881\\
87.14	0.001525275032729\\
87.15	0.001524066202401\\
87.16	0.00152285714777385\\
87.17	0.00152164786895704\\
87.18	0.00152043836606049\\
87.19	0.00151922863919457\\
87.2	0.00151801868847013\\
87.21	0.0015168085139984\\
87.22	0.00151559811589111\\
87.23	0.00151438749426037\\
87.24	0.00151317664921879\\
87.25	0.00151196558087936\\
87.26	0.00151075428935554\\
87.27	0.00150954277476121\\
87.28	0.00150833103721068\\
87.29	0.0015071190768187\\
87.3	0.00150590689370042\\
87.31	0.00150469448797145\\
87.32	0.00150348185974779\\
87.33	0.0015022690091459\\
87.34	0.00150105593628263\\
87.35	0.00149984264127525\\
87.36	0.00149862912424146\\
87.37	0.00149741538529936\\
87.38	0.00149620142456747\\
87.39	0.00149498724216472\\
87.4	0.00149377283821043\\
87.41	0.00149255821282436\\
87.42	0.00149134336612665\\
87.43	0.00149012829823782\\
87.44	0.00148891300927883\\
87.45	0.00148769749937101\\
87.46	0.00148648176863609\\
87.47	0.00148526581719618\\
87.48	0.00148404964517379\\
87.49	0.00148283325269183\\
87.5	0.00148161663987354\\
87.51	0.00148039980684261\\
87.52	0.00147918275372305\\
87.53	0.00147796548063927\\
87.54	0.00147674798771605\\
87.55	0.00147553027507853\\
87.56	0.00147431234285223\\
87.57	0.00147309419116301\\
87.58	0.00147187582013712\\
87.59	0.00147065722990113\\
87.6	0.00146943842058199\\
87.61	0.001468219392307\\
87.62	0.00146700014520379\\
87.63	0.00146578067940035\\
87.64	0.00146456099502498\\
87.65	0.00146334109220636\\
87.66	0.00146212097107346\\
87.67	0.00146090063175562\\
87.68	0.00145968007438247\\
87.69	0.00145845929908398\\
87.7	0.00145723830599044\\
87.71	0.00145601709523245\\
87.72	0.0014547956669409\\
87.73	0.00145357402124703\\
87.74	0.00145235215828215\\
87.75	0.0014511300781777\\
87.76	0.00144990778106522\\
87.77	0.00144868526707635\\
87.78	0.00144746253634284\\
87.79	0.00144623958899654\\
87.8	0.00144501642516941\\
87.81	0.00144379304499352\\
87.82	0.00144256944860104\\
87.83	0.00144134563612422\\
87.84	0.00144012160769547\\
87.85	0.00143889736344724\\
87.86	0.00143767290351214\\
87.87	0.00143644822802284\\
87.88	0.00143522333711216\\
87.89	0.00143399823091297\\
87.9	0.00143277290955829\\
87.91	0.00143154737318121\\
87.92	0.00143032162191496\\
87.93	0.00142909565589283\\
87.94	0.00142786947524824\\
87.95	0.00142664308011472\\
87.96	0.00142541647062588\\
87.97	0.00142418964691545\\
87.98	0.00142296260911725\\
87.99	0.00142173535736522\\
88	0.00142050789179339\\
88.01	0.00141928021253589\\
88.02	0.00141805231972695\\
88.03	0.00141682421350093\\
88.04	0.00141559589399225\\
88.05	0.00141436736133546\\
88.06	0.00141313861566521\\
88.07	0.00141190965711625\\
88.08	0.00141068048582342\\
88.09	0.00140945110192167\\
88.1	0.00140822150554605\\
88.11	0.00140699169683171\\
88.12	0.00140576167591392\\
88.13	0.00140453144292802\\
88.14	0.00140330099800947\\
88.15	0.00140207034129382\\
88.16	0.00140083947291673\\
88.17	0.00139960839301397\\
88.18	0.00139837710172139\\
88.19	0.00139714559917494\\
88.2	0.00139591388551069\\
88.21	0.00139468196086479\\
88.22	0.0013934498253735\\
88.23	0.00139221747917319\\
88.24	0.00139098492240031\\
88.25	0.00138975215519141\\
88.26	0.00138851917768316\\
88.27	0.00138728599001232\\
88.28	0.00138605259231574\\
88.29	0.00138481898473038\\
88.3	0.00138358516739328\\
88.31	0.00138235114044162\\
88.32	0.00138111690401263\\
88.33	0.00137988245824368\\
88.34	0.0013786478032722\\
88.35	0.00137741293923576\\
88.36	0.001376177866272\\
88.37	0.00137494258451866\\
88.38	0.00137370709411359\\
88.39	0.00137247139519474\\
88.4	0.00137123548790014\\
88.41	0.00136999937236792\\
88.42	0.00136876304873635\\
88.43	0.00136752651714373\\
88.44	0.0013662897777285\\
88.45	0.00136505283062921\\
88.46	0.00136381567598447\\
88.47	0.001362578313933\\
88.48	0.00136134074461364\\
88.49	0.00136010296816529\\
88.5	0.00135886498472698\\
88.51	0.00135762679443781\\
88.52	0.00135638839743701\\
88.53	0.00135514979386388\\
88.54	0.00135391098385781\\
88.55	0.00135267196755831\\
88.56	0.00135143274510498\\
88.57	0.00135019331663751\\
88.58	0.0013489536822957\\
88.59	0.00134771384221941\\
88.6	0.00134647379654864\\
88.61	0.00134523354542347\\
88.62	0.00134399308898408\\
88.63	0.00134275242737072\\
88.64	0.00134151156072377\\
88.65	0.00134027048918369\\
88.66	0.00133902921289103\\
88.67	0.00133778773198646\\
88.68	0.00133654604661071\\
88.69	0.00133530415690462\\
88.7	0.00133406206300915\\
88.71	0.00133281976506533\\
88.72	0.00133157726321427\\
88.73	0.00133033455759721\\
88.74	0.00132909164835547\\
88.75	0.00132784853563046\\
88.76	0.00132660521956368\\
88.77	0.00132536170029676\\
88.78	0.00132411797797137\\
88.79	0.00132287405272931\\
88.8	0.00132162992471247\\
88.81	0.00132038559406284\\
88.82	0.00131914106092249\\
88.83	0.00131789632543358\\
88.84	0.00131665138773839\\
88.85	0.00131540624797927\\
88.86	0.00131416090629867\\
88.87	0.00131291536283915\\
88.88	0.00131166961774333\\
88.89	0.00131042367115396\\
88.9	0.00130917752321386\\
88.91	0.00130793117406596\\
88.92	0.00130668462385328\\
88.93	0.00130543787271891\\
88.94	0.00130419092080606\\
88.95	0.00130294376825803\\
88.96	0.00130169641521821\\
88.97	0.00130044886183007\\
88.98	0.0012992011082372\\
88.99	0.00129795315458327\\
89	0.00129670500101203\\
89.01	0.00129545664766734\\
89.02	0.00129420809469315\\
89.03	0.0012929593422335\\
89.04	0.00129171039043252\\
89.05	0.00129046123943445\\
89.06	0.00128921188938359\\
89.07	0.00128796234042437\\
89.08	0.00128671259270128\\
89.09	0.00128546264635892\\
89.1	0.00128421250154199\\
89.11	0.00128296215839527\\
89.12	0.00128171161706363\\
89.13	0.00128046087769205\\
89.14	0.00127920994042557\\
89.15	0.00127795880540935\\
89.16	0.00127670747278864\\
89.17	0.00127545594270877\\
89.18	0.00127420421531518\\
89.19	0.00127295229075338\\
89.2	0.00127170016916898\\
89.21	0.0012704478507077\\
89.22	0.00126919533551533\\
89.23	0.00126794262373776\\
89.24	0.00126668971552097\\
89.25	0.00126543661101103\\
89.26	0.00126418331035412\\
89.27	0.00126292981369648\\
89.28	0.00126167612118447\\
89.29	0.00126042223296453\\
89.3	0.00125916814918318\\
89.31	0.00125791386998707\\
89.32	0.00125665939552289\\
89.33	0.00125540472593747\\
89.34	0.0012541498613777\\
89.35	0.00125289480199057\\
89.36	0.00125163954792316\\
89.37	0.00125038409932265\\
89.38	0.00124912845633632\\
89.39	0.0012478726191115\\
89.4	0.00124661658779567\\
89.41	0.00124536036253635\\
89.42	0.00124410394348118\\
89.43	0.00124284733077788\\
89.44	0.00124159052457428\\
89.45	0.00124033352501828\\
89.46	0.00123907633225787\\
89.47	0.00123781894644116\\
89.48	0.00123656136771632\\
89.49	0.00123530359623162\\
89.5	0.00123404563213544\\
89.51	0.00123278747557623\\
89.52	0.00123152912670254\\
89.53	0.001230270585663\\
89.54	0.00122901185260636\\
89.55	0.00122775292768144\\
89.56	0.00122649381103715\\
89.57	0.00122523450282249\\
89.58	0.00122397500318657\\
89.59	0.00122271531227858\\
89.6	0.0012214554302478\\
89.61	0.0012201953572436\\
89.62	0.00121893509341545\\
89.63	0.0012176746389129\\
89.64	0.00121641399388561\\
89.65	0.0012151531584833\\
89.66	0.00121389213285583\\
89.67	0.0012126309171531\\
89.68	0.00121136951152515\\
89.69	0.00121010791612206\\
89.7	0.00120884613109405\\
89.71	0.00120758415659141\\
89.72	0.00120632199276452\\
89.73	0.00120505963976385\\
89.74	0.00120379709773997\\
89.75	0.00120253436684355\\
89.76	0.00120127144722533\\
89.77	0.00120000833903616\\
89.78	0.00119874504242698\\
89.79	0.00119748155754881\\
89.8	0.00119621788455278\\
89.81	0.0011949540235901\\
89.82	0.00119368997481206\\
89.83	0.00119242573837009\\
89.84	0.00119116131441565\\
89.85	0.00118989670310035\\
89.86	0.00118863190457585\\
89.87	0.00118736691899392\\
89.88	0.00118610174650643\\
89.89	0.00118483638726532\\
89.9	0.00118357084142266\\
89.91	0.00118230510913057\\
89.92	0.00118103919054129\\
89.93	0.00117977308580715\\
89.94	0.00117850679508056\\
89.95	0.00117724031851406\\
89.96	0.00117597365626022\\
89.97	0.00117470680847177\\
89.98	0.00117343977530149\\
89.99	0.00117217255690227\\
90	0.00117090515342709\\
90.01	0.00116963756502902\\
90.02	0.00116836979186125\\
90.03	0.00116710183407702\\
90.04	0.0011658336918297\\
90.05	0.00116456536527273\\
90.06	0.00116329685455967\\
90.07	0.00116202815984415\\
90.08	0.00116075928127991\\
90.09	0.00115949021902077\\
90.1	0.00115822097322067\\
90.11	0.00115695154403362\\
90.12	0.00115568193161373\\
90.13	0.00115441213611522\\
90.14	0.00115314215769238\\
90.15	0.00115187199649963\\
90.16	0.00115060165269145\\
90.17	0.00114933112642243\\
90.18	0.00114806041784726\\
90.19	0.00114678952712073\\
90.2	0.0011455184543977\\
90.21	0.00114424719983317\\
90.22	0.00114297576358219\\
90.23	0.00114170414579993\\
90.24	0.00114043234664167\\
90.25	0.00113916036626276\\
90.26	0.00113788820481865\\
90.27	0.0011366158624649\\
90.28	0.00113534333935717\\
90.29	0.0011340706356512\\
90.3	0.00113279775150284\\
90.31	0.00113152468706803\\
90.32	0.00113025144250282\\
90.33	0.00112897801796334\\
90.34	0.00112770441360584\\
90.35	0.00112643062958665\\
90.36	0.00112515666606221\\
90.37	0.00112388252318904\\
90.38	0.00112260820112379\\
90.39	0.00112133370002318\\
90.4	0.00112005902004405\\
90.41	0.00111878416134333\\
90.42	0.00111750912407804\\
90.43	0.00111623390840531\\
90.44	0.00111495851448239\\
90.45	0.00111368294246659\\
90.46	0.00111240719251534\\
90.47	0.0011111312647862\\
90.48	0.00110985515943676\\
90.49	0.00110857887662478\\
90.5	0.00110730241650809\\
90.51	0.00110602577924463\\
90.52	0.00110474896499243\\
90.53	0.00110347197390963\\
90.54	0.00110219480615448\\
90.55	0.00110091746188531\\
90.56	0.00109963994126058\\
90.57	0.00109836224443884\\
90.58	0.00109708437157873\\
90.59	0.00109580632283902\\
90.6	0.00109452809837856\\
90.61	0.00109324969835633\\
90.62	0.00109197112293137\\
90.63	0.00109069237226288\\
90.64	0.00108941344651013\\
90.65	0.00108813434583249\\
90.66	0.00108685507038945\\
90.67	0.00108557562034061\\
90.68	0.00108429599584567\\
90.69	0.00108301619706442\\
90.7	0.00108173622415679\\
90.71	0.00108045607728278\\
90.72	0.00107917575660252\\
90.73	0.00107789526227624\\
90.74	0.00107661459446428\\
90.75	0.00107533375332708\\
90.76	0.0010740527390252\\
90.77	0.0010727715517193\\
90.78	0.00107149019157015\\
90.79	0.00107020865873862\\
90.8	0.00106892695338572\\
90.81	0.00106764507567254\\
90.82	0.00106636302576028\\
90.83	0.00106508080381026\\
90.84	0.00106379840998392\\
90.85	0.00106251584444279\\
90.86	0.00106123310734852\\
90.87	0.00105995019886288\\
90.88	0.00105866711914773\\
90.89	0.00105738386836508\\
90.9	0.001056100446677\\
90.91	0.00105481685424573\\
90.92	0.00105353309123357\\
90.93	0.00105224915780298\\
90.94	0.0010509650541165\\
90.95	0.0010496807803368\\
90.96	0.00104839633662666\\
90.97	0.00104711172314897\\
90.98	0.00104582694006677\\
90.99	0.00104454198754316\\
91	0.00104325686574139\\
91.01	0.00104197157482483\\
91.02	0.00104068611495695\\
91.03	0.00103940048630135\\
91.04	0.00103811468902174\\
91.05	0.00103682872328196\\
91.06	0.00103554258924595\\
91.07	0.00103425628707778\\
91.08	0.00103296981694164\\
91.09	0.00103168317900185\\
91.1	0.00103039637342283\\
91.11	0.00102910940036912\\
91.12	0.00102782226000541\\
91.13	0.00102653495249648\\
91.14	0.00102524747800724\\
91.15	0.00102395983670275\\
91.16	0.00102267202874815\\
91.17	0.00102138405430873\\
91.18	0.0010200959135499\\
91.19	0.00101880760663718\\
91.2	0.00101751913373625\\
91.21	0.00101623049501288\\
91.22	0.00101494169063298\\
91.23	0.00101365272076257\\
91.24	0.00101236358556784\\
91.25	0.00101107428521505\\
91.26	0.00100978481987064\\
91.27	0.00100849518970114\\
91.28	0.00100720539487322\\
91.29	0.0010059154355537\\
91.3	0.0010046253119095\\
91.31	0.00100333502410769\\
91.32	0.00100204457231546\\
91.33	0.00100075395670013\\
91.34	0.000999463177429175\\
91.35	0.000998172234670167\\
91.36	0.000996881128590837\\
91.37	0.00099558985935904\\
91.38	0.000994298427142765\\
91.39	0.000993006832110141\\
91.4	0.000991715074429425\\
91.41	0.000990423154269015\\
91.42	0.000989131071797449\\
91.43	0.000987838827183387\\
91.44	0.000986546420595646\\
91.45	0.000985253852203166\\
91.46	0.000983961122175032\\
91.47	0.000982668230680474\\
91.48	0.000981375177888847\\
91.49	0.000980081963969662\\
91.5	0.000978788589092562\\
91.51	0.000977495053427337\\
91.52	0.000976201357143913\\
91.53	0.000974907500412368\\
91.54	0.000973613483402913\\
91.55	0.000972319306285914\\
91.56	0.000971024969231878\\
91.57	0.000969730472411454\\
91.58	0.000968435815995445\\
91.59	0.000967141000154791\\
91.6	0.000965846025060587\\
91.61	0.000964550890884071\\
91.62	0.000963255597796638\\
91.63	0.000961960145969831\\
91.64	0.000960664535575334\\
91.65	0.000959368766784994\\
91.66	0.000958072839770804\\
91.67	0.000956776754704906\\
91.68	0.000955480511759602\\
91.69	0.000954184111107344\\
91.7	0.000952887552920745\\
91.71	0.000951590837372564\\
91.72	0.000950293964635722\\
91.73	0.000948996934883295\\
91.74	0.000947699748288518\\
91.75	0.000946402405024785\\
91.76	0.000945104905265644\\
91.77	0.000943807249184811\\
91.78	0.000942509436956158\\
91.79	0.000941211468753717\\
91.8	0.000939913344751686\\
91.81	0.000938615065124427\\
91.82	0.000937316630046462\\
91.83	0.000936018039692477\\
91.84	0.000934719294237328\\
91.85	0.000933420393856035\\
91.86	0.000932121338723784\\
91.87	0.000930822129015935\\
91.88	0.000929522764908002\\
91.89	0.000928223246575691\\
91.9	0.000926923574194861\\
91.91	0.000925623747941544\\
91.92	0.000924323767991956\\
91.93	0.000923023634522471\\
91.94	0.000921723347709649\\
91.95	0.000920422907730213\\
91.96	0.00091912231476108\\
91.97	0.000917821568979326\\
91.98	0.000916520670562209\\
91.99	0.000915219619687173\\
92	0.000913918416531835\\
92.01	0.000912617061273989\\
92.02	0.000911315554091616\\
92.03	0.000910013895162883\\
92.04	0.000908712084666129\\
92.05	0.000907410122779886\\
92.06	0.000906108009682867\\
92.07	0.000904805745553973\\
92.08	0.00090350333057229\\
92.09	0.000902200764917096\\
92.1	0.000900898048767852\\
92.11	0.000899595182304218\\
92.12	0.000898292165706029\\
92.13	0.000896988999153327\\
92.14	0.000895685682826344\\
92.15	0.000894382216905501\\
92.16	0.00089307860157142\\
92.17	0.000891774837004915\\
92.18	0.000890470923386994\\
92.19	0.000889166860898875\\
92.2	0.000887862649721963\\
92.21	0.000886558290037868\\
92.22	0.000885253782028399\\
92.23	0.000883949125875579\\
92.24	0.000882644321761617\\
92.25	0.000881339369868934\\
92.26	0.000880034270380163\\
92.27	0.000878729023478134\\
92.28	0.000877423629345899\\
92.29	0.000876118088166695\\
92.3	0.000874812400123997\\
92.31	0.000873506565401472\\
92.32	0.000872200584183008\\
92.33	0.000870894456652707\\
92.34	0.000869588182994875\\
92.35	0.000868281763394051\\
92.36	0.000866975198034976\\
92.37	0.000865668487102624\\
92.38	0.000864361630782176\\
92.39	0.000863054629259039\\
92.4	0.000861747482718844\\
92.41	0.00086044019134744\\
92.42	0.000859132755330907\\
92.43	0.000857825174855546\\
92.44	0.000856517450107888\\
92.45	0.000855209581274687\\
92.46	0.000853901568542934\\
92.47	0.000852593412099844\\
92.48	0.000851285112132873\\
92.49	0.000849976668829697\\
92.5	0.000848668082378238\\
92.51	0.000847359352966651\\
92.52	0.000846050480783327\\
92.53	0.000844741466016896\\
92.54	0.00084343230885623\\
92.55	0.000842123009490438\\
92.56	0.000840813568108879\\
92.57	0.000839503984901149\\
92.58	0.000838194260057094\\
92.59	0.000836884393766802\\
92.6	0.000835574386220613\\
92.61	0.000834264237609116\\
92.62	0.000832953948123152\\
92.63	0.000831643517953813\\
92.64	0.000830332947292442\\
92.65	0.000829022236330643\\
92.66	0.000827711385260275\\
92.67	0.000826400394273454\\
92.68	0.000825089263562557\\
92.69	0.000823777993320217\\
92.7	0.000822466583739341\\
92.71	0.000821155035013092\\
92.72	0.000819843347334894\\
92.73	0.00081853152089845\\
92.74	0.000817219555897728\\
92.75	0.000815907452526956\\
92.76	0.000814595210980647\\
92.77	0.00081328283145358\\
92.78	0.000811970314140816\\
92.79	0.000810657659237683\\
92.8	0.000809344866939791\\
92.81	0.000808031937443032\\
92.82	0.000806718870943576\\
92.83	0.000805405667637883\\
92.84	0.000804092327722689\\
92.85	0.000802778851395012\\
92.86	0.000801465238852179\\
92.87	0.00080015149029178\\
92.88	0.000798837605911715\\
92.89	0.000797523585910171\\
92.9	0.000796209430485625\\
92.91	0.000794895139836851\\
92.92	0.000793580714162934\\
92.93	0.00079226615366324\\
92.94	0.000790951458537446\\
92.95	0.00078963662898553\\
92.96	0.000788321665207785\\
92.97	0.00078700656740479\\
92.98	0.000785691335777449\\
92.99	0.000784375970526972\\
93	0.000783060471854872\\
93.01	0.000781744839962994\\
93.02	0.000780429075053482\\
93.03	0.000779113177328802\\
93.04	0.000777797146991744\\
93.05	0.000776480984245419\\
93.06	0.000775164689293249\\
93.07	0.000773848262339\\
93.08	0.000772531703586743\\
93.09	0.0007712150132409\\
93.1	0.000769898191506203\\
93.11	0.000768581238587734\\
93.12	0.000767264154690896\\
93.13	0.000765946940021434\\
93.14	0.000764629594785435\\
93.15	0.000763312119189315\\
93.16	0.000761994513439846\\
93.17	0.000760676777744137\\
93.18	0.000759358912309643\\
93.19	0.000758040917344167\\
93.2	0.000756722793055864\\
93.21	0.00075540453965324\\
93.22	0.000754086157345157\\
93.23	0.000752767646340833\\
93.24	0.000751449006849841\\
93.25	0.000750130239082124\\
93.26	0.000748811343247972\\
93.27	0.000747492319558056\\
93.28	0.000746173168223407\\
93.29	0.000744853889455426\\
93.3	0.000743534483465881\\
93.31	0.000742214950466917\\
93.32	0.000740895290671053\\
93.33	0.000739575504291193\\
93.34	0.000738255591540614\\
93.35	0.00073693555263297\\
93.36	0.00073561538778231\\
93.37	0.00073429509720307\\
93.38	0.000732974681110063\\
93.39	0.000731654139718506\\
93.4	0.000730333473244003\\
93.41	0.000729012681902548\\
93.42	0.000727691765910549\\
93.43	0.000726370725484803\\
93.44	0.000725049560842512\\
93.45	0.00072372827220128\\
93.46	0.000722406859779126\\
93.47	0.000721085323794472\\
93.48	0.000719763664466159\\
93.49	0.00071844188201343\\
93.5	0.000717119976655961\\
93.51	0.000715797948613836\\
93.52	0.00071447579810757\\
93.53	0.000713153525358097\\
93.54	0.000711831130586768\\
93.55	0.000710508614015387\\
93.56	0.000709185975866168\\
93.57	0.000707863216361774\\
93.58	0.000706540335725292\\
93.59	0.000705217334180263\\
93.6	0.000703894211950655\\
93.61	0.000702570969260895\\
93.62	0.00070124760633584\\
93.63	0.00069992412340082\\
93.64	0.000698600520681599\\
93.65	0.0006972767984044\\
93.66	0.000695952956795908\\
93.67	0.000694628996083262\\
93.68	0.000693304916494069\\
93.69	0.000691980718256402\\
93.7	0.000690656401598806\\
93.71	0.000689331966750284\\
93.72	0.00068800741394032\\
93.73	0.000686682743398887\\
93.74	0.000685357955356418\\
93.75	0.000684033050043839\\
93.76	0.000682708027692555\\
93.77	0.000681382888534471\\
93.78	0.000680057632801967\\
93.79	0.000678732260727928\\
93.8	0.000677406772545729\\
93.81	0.000676081168489244\\
93.82	0.000674755448792858\\
93.83	0.000673429613691452\\
93.84	0.000672103663420418\\
93.85	0.00067077759821566\\
93.86	0.000669451418313595\\
93.87	0.000668125123951155\\
93.88	0.000666798715365792\\
93.89	0.000665472192795488\\
93.9	0.000664145556478741\\
93.91	0.000662818806654588\\
93.92	0.000661491943562588\\
93.93	0.000660164967442841\\
93.94	0.000658837878535984\\
93.95	0.000657510677083197\\
93.96	0.0006561833633262\\
93.97	0.000654855937507267\\
93.98	0.000653528399869216\\
93.99	0.000652200750655424\\
94	0.000650872990109819\\
94.01	0.0006495451184769\\
94.02	0.00064821713600172\\
94.03	0.000646889042929906\\
94.04	0.000645560839507642\\
94.05	0.000644232525981704\\
94.06	0.000642904102599427\\
94.07	0.000641575569608739\\
94.08	0.000640246927258144\\
94.09	0.000638918175796737\\
94.1	0.0006375893154742\\
94.11	0.000636260346540808\\
94.12	0.000634931269247437\\
94.13	0.000633602083845561\\
94.14	0.000632272790587253\\
94.15	0.000630943389725197\\
94.16	0.000629613881512694\\
94.17	0.000628284266203651\\
94.18	0.000626954544052594\\
94.19	0.000625624715314669\\
94.2	0.000624294780245649\\
94.21	0.000622964739101941\\
94.22	0.00062163459214057\\
94.23	0.000620304339619208\\
94.24	0.000618973981796165\\
94.25	0.000617643518930388\\
94.26	0.000616312951281478\\
94.27	0.000614982279109678\\
94.28	0.000613651502675889\\
94.29	0.000612320622241673\\
94.3	0.000610989638069244\\
94.31	0.000609658550421497\\
94.32	0.000608327359561978\\
94.33	0.000606996065754919\\
94.34	0.000605664669265217\\
94.35	0.000604333170358464\\
94.36	0.000603001569300923\\
94.37	0.000601669866359547\\
94.38	0.000600338061801995\\
94.39	0.0005990061558966\\
94.4	0.000597674148912415\\
94.41	0.000596342041119182\\
94.42	0.000595009832787362\\
94.43	0.000593677524188123\\
94.44	0.000592345115593347\\
94.45	0.000591012607275633\\
94.46	0.000589679999508324\\
94.47	0.000588347292565463\\
94.48	0.000587014486721845\\
94.49	0.000585681582252995\\
94.5	0.000584348579435179\\
94.51	0.000583015478545406\\
94.52	0.000581682279861434\\
94.53	0.000580348983661784\\
94.54	0.000579015590225725\\
94.55	0.00057768209983328\\
94.56	0.000576348512765256\\
94.57	0.000575014829303217\\
94.58	0.00057368104972951\\
94.59	0.00057234717432726\\
94.6	0.000571013203380363\\
94.61	0.000569679137173517\\
94.62	0.000568344975992211\\
94.63	0.00056701072012272\\
94.64	0.000565676369852138\\
94.65	0.000564341925468342\\
94.66	0.000563007387260037\\
94.67	0.000561672755516739\\
94.68	0.000560338030528773\\
94.69	0.000559003212587299\\
94.7	0.000557668301984303\\
94.71	0.0005563332990126\\
94.72	0.000554998203965843\\
94.73	0.000553663017138535\\
94.74	0.000552327738826014\\
94.75	0.000550992369324479\\
94.76	0.000549656908930976\\
94.77	0.000548321357943424\\
94.78	0.0005469857166606\\
94.79	0.000545649985382151\\
94.8	0.000544314164408604\\
94.81	0.000542978254041359\\
94.82	0.000541642254582711\\
94.83	0.000540306166335837\\
94.84	0.000538969989604813\\
94.85	0.000537633724694614\\
94.86	0.000536297371911116\\
94.87	0.000534960931561114\\
94.88	0.000533624403952311\\
94.89	0.000532287789393323\\
94.9	0.000530951088193704\\
94.91	0.000529614300663938\\
94.92	0.000528277427115429\\
94.93	0.000526940467860537\\
94.94	0.000525603423212563\\
94.95	0.000524266293485754\\
94.96	0.000522929078995312\\
94.97	0.000521591780057409\\
94.98	0.000520254396989174\\
94.99	0.000518916930108715\\
95	0.000517579379735112\\
95.01	0.000516241746188426\\
95.02	0.000514904029789714\\
95.03	0.000513566230861009\\
95.04	0.000512228349725358\\
95.05	0.000510890386706807\\
95.06	0.000509552342130412\\
95.07	0.000508214216322243\\
95.08	0.000506876009609384\\
95.09	0.000505537722319951\\
95.1	0.000504199354783092\\
95.11	0.000502860907328989\\
95.12	0.000501522380288867\\
95.13	0.000500183773995001\\
95.14	0.000498845088780715\\
95.15	0.000497506324980397\\
95.16	0.000496167482929499\\
95.17	0.000494828562964539\\
95.18	0.00049348956542312\\
95.19	0.000492150490643923\\
95.2	0.000490811338966715\\
95.21	0.000489472110732359\\
95.22	0.000488132806282822\\
95.23	0.000486793425961167\\
95.24	0.000485453970111583\\
95.25	0.000484114439079357\\
95.26	0.000482774833210922\\
95.27	0.000481435152853826\\
95.28	0.000480095398356755\\
95.29	0.000478755570069538\\
95.3	0.000477415668343155\\
95.31	0.000476075693529735\\
95.32	0.000474735645982568\\
95.33	0.000473395526056119\\
95.34	0.000472055334106007\\
95.35	0.00047071507048905\\
95.36	0.000469374735563237\\
95.37	0.000468034329687757\\
95.38	0.000466693853222988\\
95.39	0.000465353306530523\\
95.4	0.000464012689973155\\
95.41	0.000462672003914902\\
95.42	0.000461331248720994\\
95.43	0.000459990424757906\\
95.44	0.000458649532393338\\
95.45	0.000457308571996233\\
95.46	0.000455967543936788\\
95.47	0.000454626448586454\\
95.48	0.000453285286317943\\
95.49	0.000451944057505241\\
95.5	0.000450602762523603\\
95.51	0.000449261401749569\\
95.52	0.000447919975560969\\
95.53	0.000446578484336929\\
95.54	0.00044523692845788\\
95.55	0.000443895308305549\\
95.56	0.000442553624263002\\
95.57	0.000441211876714613\\
95.58	0.000439870066046085\\
95.59	0.000438528192644469\\
95.6	0.000437186256898146\\
95.61	0.000435844259196861\\
95.62	0.000434502199931708\\
95.63	0.000433160079495157\\
95.64	0.000431817898281036\\
95.65	0.000430475656684562\\
95.66	0.000429133355102336\\
95.67	0.000427790993932352\\
95.68	0.000426448573574007\\
95.69	0.000425106094428104\\
95.7	0.000423763556896866\\
95.71	0.000422420961383929\\
95.72	0.000421078308294373\\
95.73	0.000419735598034699\\
95.74	0.00041839283101287\\
95.75	0.000417050007638286\\
95.76	0.000415707128321821\\
95.77	0.000414364193475807\\
95.78	0.000413021203514054\\
95.79	0.000411678158851852\\
95.8	0.000410335059905986\\
95.81	0.000408991907094735\\
95.82	0.000407648700837883\\
95.83	0.00040630544155673\\
95.84	0.000404962129674094\\
95.85	0.000403618765614325\\
95.86	0.000402275349803307\\
95.87	0.000400931882668465\\
95.88	0.000399588364638782\\
95.89	0.000398244796144802\\
95.9	0.000396901177618633\\
95.91	0.000395557509493957\\
95.92	0.000394213792206046\\
95.93	0.000392870026191759\\
95.94	0.000391526211889559\\
95.95	0.000390182349739516\\
95.96	0.000388838440183312\\
95.97	0.000387494483664262\\
95.98	0.000386150480627305\\
95.99	0.00038480643151903\\
96	0.000383462336787667\\
96.01	0.000382118196883106\\
96.02	0.000380774012256909\\
96.03	0.000379429783362304\\
96.04	0.000378085510654208\\
96.05	0.000376741194589223\\
96.06	0.000375396835625657\\
96.07	0.000374052434223527\\
96.08	0.000372707990844556\\
96.09	0.000371363505952205\\
96.1	0.000370018980011667\\
96.11	0.000368674413489872\\
96.12	0.000367329806855502\\
96.13	0.000365985160579009\\
96.14	0.000364640475132601\\
96.15	0.000363295750990275\\
96.16	0.000361950988627804\\
96.17	0.000360606188522768\\
96.18	0.000359261351154544\\
96.19	0.00035791647700432\\
96.2	0.00035657156655512\\
96.21	0.000355226620291785\\
96.22	0.000353881638701002\\
96.23	0.000352536622271313\\
96.24	0.00035119157149311\\
96.25	0.00034984648685866\\
96.26	0.000348501368862102\\
96.27	0.000347156217999467\\
96.28	0.000345811034768681\\
96.29	0.000344465819669569\\
96.3	0.000343120573203879\\
96.31	0.00034177529587528\\
96.32	0.000340429988189372\\
96.33	0.000339084650653703\\
96.34	0.000337739283777772\\
96.35	0.000336393888073038\\
96.36	0.000335048464052934\\
96.37	0.00033370301223287\\
96.38	0.000332357533130257\\
96.39	0.0003310120272645\\
96.4	0.00032966649515701\\
96.41	0.000328320937331228\\
96.42	0.000326975354312623\\
96.43	0.000325629746628702\\
96.44	0.000324284114809024\\
96.45	0.000322938459385207\\
96.46	0.000321592780890936\\
96.47	0.000320247079861986\\
96.48	0.00031890135683621\\
96.49	0.00031755561235357\\
96.5	0.000316209846956137\\
96.51	0.000314864061188106\\
96.52	0.000313518255595792\\
96.53	0.000312172430727663\\
96.54	0.000310826587134334\\
96.55	0.000309480725368584\\
96.56	0.000308134845985361\\
96.57	0.0003067889495418\\
96.58	0.000305443036597226\\
96.59	0.000304097107713171\\
96.6	0.000302751163453385\\
96.61	0.000301405204383832\\
96.62	0.00030005923107272\\
96.63	0.000298713244090509\\
96.64	0.000297367244009902\\
96.65	0.000296021231405883\\
96.66	0.000294675206855713\\
96.67	0.000293329170938933\\
96.68	0.000291983124237399\\
96.69	0.000290637067335267\\
96.7	0.000289291000819024\\
96.71	0.00028794492527749\\
96.72	0.000286598841301824\\
96.73	0.000285252749485545\\
96.74	0.000283906650424539\\
96.75	0.000282560544717075\\
96.76	0.000281214432963807\\
96.77	0.000279868315767789\\
96.78	0.000278522193734488\\
96.79	0.000277176067471801\\
96.8	0.000275829937590051\\
96.81	0.000274483804702014\\
96.82	0.000273137669422926\\
96.83	0.000271791532370487\\
96.84	0.000270445394164881\\
96.85	0.000269099255428786\\
96.86	0.000267753116787383\\
96.87	0.000266406978868376\\
96.88	0.000265060842301986\\
96.89	0.000263714707720988\\
96.9	0.000262368575760698\\
96.91	0.000261022447059002\\
96.92	0.000259676322256359\\
96.93	0.000258330201995819\\
96.94	0.000256984086923031\\
96.95	0.000255637977686256\\
96.96	0.000254291874936382\\
96.97	0.000252945779326927\\
96.98	0.000251599691514066\\
96.99	0.000250253612156632\\
97	0.000248907541916128\\
97.01	0.00024756148145675\\
97.02	0.000246215431445389\\
97.03	0.000244869392551644\\
97.04	0.000243523365447843\\
97.05	0.000242177350809045\\
97.06	0.000240831349313063\\
97.07	0.000239485361640467\\
97.08	0.000238139388474602\\
97.09	0.0002367934305016\\
97.1	0.000235447488410397\\
97.11	0.000234101562892733\\
97.12	0.000232755654643178\\
97.13	0.000231409764359144\\
97.14	0.000230063892740889\\
97.15	0.000228718040491536\\
97.16	0.000227372208317089\\
97.17	0.000226026396926439\\
97.18	0.000224680607031381\\
97.19	0.000223334839346633\\
97.2	0.000221989094589835\\
97.21	0.000220643373481575\\
97.22	0.000219297676745401\\
97.23	0.000217952005107826\\
97.24	0.000216606359298351\\
97.25	0.000215260740049471\\
97.26	0.000213915148096695\\
97.27	0.000212569584178561\\
97.28	0.000211224049036635\\
97.29	0.000209878543415545\\
97.3	0.000208533068062981\\
97.31	0.00020718762372971\\
97.32	0.000205842211169599\\
97.33	0.000204496831139619\\
97.34	0.000203151484399863\\
97.35	0.000201806171713561\\
97.36	0.000200460893847093\\
97.37	0.000199115651569999\\
97.38	0.000197770445655004\\
97.39	0.000196425276878019\\
97.4	0.000195080146018163\\
97.41	0.00019373505385778\\
97.42	0.000192390001182447\\
97.43	0.000191044988780987\\
97.44	0.000189700017445494\\
97.45	0.000188355087971336\\
97.46	0.000187010201157177\\
97.47	0.000185665357804989\\
97.48	0.000184320558720063\\
97.49	0.000182975804711034\\
97.5	0.000181631096589883\\
97.51	0.000180286435171964\\
97.52	0.000178941821276007\\
97.53	0.000177597255724143\\
97.54	0.000176252739341919\\
97.55	0.000174908272958298\\
97.56	0.000173563857405695\\
97.57	0.000172219493519978\\
97.58	0.000170875182140489\\
97.59	0.000169530924110059\\
97.6	0.000168186720275017\\
97.61	0.000166842571485219\\
97.62	0.000165498478594045\\
97.63	0.000164154442458434\\
97.64	0.000162810463938883\\
97.65	0.000161466543899471\\
97.66	0.000160122683207876\\
97.67	0.000158778882735385\\
97.68	0.000157435143356914\\
97.69	0.000156091465951018\\
97.7	0.000154747851400036\\
97.71	0.000153404300590123\\
97.72	0.000152060814411268\\
97.73	0.000150717393757304\\
97.74	0.000149374039525923\\
97.75	0.000148030752618688\\
97.76	0.000146687533941052\\
97.77	0.000145344384402352\\
97.78	0.000144001304915846\\
97.79	0.000142658296398705\\
97.8	0.000141315359479272\\
97.81	0.000139972494572746\\
97.82	0.000138629702101208\\
97.83	0.000137286982493707\\
97.84	0.000135944336186347\\
97.85	0.000134601763622377\\
97.86	0.000133259265252277\\
97.87	0.000131916841533856\\
97.88	0.000130574492932344\\
97.89	0.000129232219776799\\
97.9	0.000127890020957828\\
97.91	0.000126547893017973\\
97.92	0.000125205832460567\\
97.93	0.000123864910660196\\
97.94	0.00012252581358818\\
97.95	0.000121188545859499\\
97.96	0.000119853112104362\\
97.97	0.000118519516967454\\
97.98	0.000117187765057444\\
97.99	0.000115857861002487\\
98	0.000114529809450571\\
98.01	0.000113204229188223\\
98.02	0.000111886377497135\\
98.03	0.000110583684521723\\
98.04	0.000109296289321601\\
98.05	0.000108024410716714\\
98.06	0.000106768291156864\\
98.07	0.000105528070912258\\
98.08	0.000104303891723391\\
98.09	0.000103095896819935\\
98.1	0.000101903528003319\\
98.11	0.000100726032038069\\
98.12	9.95635478075879e-05\\
98.13	9.84162156899785e-05\\
98.14	9.72841775775096e-05\\
98.15	9.6167576896438e-05\\
98.16	9.50664536770018e-05\\
98.17	9.39803675914178e-05\\
98.18	9.29055674737875e-05\\
98.19	9.18347434802728e-05\\
98.2	9.07679165073217e-05\\
98.21	8.97051061932285e-05\\
98.22	8.86463321764525e-05\\
98.23	8.75916140914493e-05\\
98.24	8.65409715643823e-05\\
98.25	8.54944242086788e-05\\
98.26	8.44519916204187e-05\\
98.27	8.3413693373573e-05\\
98.28	8.23795490150665e-05\\
98.29	8.13495780596866e-05\\
98.3	8.03237999847991e-05\\
98.31	7.93022342248807e-05\\
98.32	7.82849001658825e-05\\
98.33	7.72718172501775e-05\\
98.34	7.62630048716673e-05\\
98.35	7.52584855767992e-05\\
98.36	7.42582830585113e-05\\
98.37	7.32624210060418e-05\\
98.38	7.22709230996654e-05\\
98.39	7.1283813005198e-05\\
98.4	7.0301114875302e-05\\
98.41	6.93228548944904e-05\\
98.42	6.83490723662316e-05\\
98.43	6.73798069644026e-05\\
98.44	6.64150987370205e-05\\
98.45	6.54549881100221e-05\\
98.46	6.44995158910876e-05\\
98.47	6.35487232735054e-05\\
98.48	6.26026518400821e-05\\
98.49	6.16613435671133e-05\\
98.5	6.07248408283894e-05\\
98.51	5.97931863992492e-05\\
98.52	5.88664234606965e-05\\
98.53	5.79445956035533e-05\\
98.54	5.70277468326767e-05\\
98.55	5.61159215712106e-05\\
98.56	5.52091646643139e-05\\
98.57	5.43075213789105e-05\\
98.58	5.34110374078942e-05\\
98.59	5.25197588743774e-05\\
98.6	5.16337323359928e-05\\
98.61	5.07530047892269e-05\\
98.62	4.98776236738472e-05\\
98.63	4.9007636877315e-05\\
98.64	4.81430927392525e-05\\
98.65	4.72840400559827e-05\\
98.66	4.64305280851538e-05\\
98.67	4.55826065503939e-05\\
98.68	4.4740325646048e-05\\
98.69	4.39037360419389e-05\\
98.7	4.30728888882181e-05\\
98.71	4.22478358202572e-05\\
98.72	4.142862896361e-05\\
98.73	4.06153209390244e-05\\
98.74	3.98079648666269e-05\\
98.75	3.90066143700751e-05\\
98.76	3.82113235817064e-05\\
98.77	3.74221471477563e-05\\
98.78	3.66391402333039e-05\\
98.79	3.58623585261005e-05\\
98.8	3.50918582272593e-05\\
98.81	3.43276960560156e-05\\
98.82	3.35699292545559e-05\\
98.83	3.28186155928757e-05\\
98.84	3.20738133736918e-05\\
98.85	3.13355814373934e-05\\
98.86	3.06039791670466e-05\\
98.87	2.9879066493434e-05\\
98.88	2.91609039001581e-05\\
98.89	2.84495524287675e-05\\
98.9	2.77450736839522e-05\\
98.91	2.70475298387793e-05\\
98.92	2.63569836399854e-05\\
98.93	2.56734984133162e-05\\
98.94	2.49971380689353e-05\\
98.95	2.43279671068505e-05\\
98.96	2.36660506224232e-05\\
98.97	2.30114543119109e-05\\
98.98	2.23642444780736e-05\\
98.99	2.17244880358122e-05\\
99	2.10922525178785e-05\\
99.01	2.04676060806437e-05\\
99.02	1.98506175098923e-05\\
99.03	1.92413562266989e-05\\
99.04	1.86398922933494e-05\\
99.05	1.804629641931e-05\\
99.06	1.74606399672653e-05\\
99.07	1.68829949592029e-05\\
99.08	1.6313434082562e-05\\
99.09	1.57520306964323e-05\\
99.1	1.51988588378213e-05\\
99.11	1.46539932279633e-05\\
99.12	1.41175092787053e-05\\
99.13	1.35894830989459e-05\\
99.14	1.30699915011235e-05\\
99.15	1.25591120077907e-05\\
99.16	1.20569228582185e-05\\
99.17	1.15635030150888e-05\\
99.18	1.10789321712389e-05\\
99.19	1.06032981187169e-05\\
99.2	1.01366899886816e-05\\
99.21	9.67919775133295e-06\\
99.22	9.23091222374516e-06\\
99.23	8.79192507778105e-06\\
99.24	8.36232884808032e-06\\
99.25	7.94221694011034e-06\\
99.26	7.53168363830208e-06\\
99.27	7.13082411427438e-06\\
99.28	6.73973443509994e-06\\
99.29	6.3585115716875e-06\\
99.3	5.98725340720911e-06\\
99.31	5.62605874563499e-06\\
99.32	5.27502732032906e-06\\
99.33	4.93425980272778e-06\\
99.34	4.60385781110052e-06\\
99.35	4.28392391940528e-06\\
99.36	3.97456166620347e-06\\
99.37	3.67587556367177e-06\\
99.38	3.38797110669906e-06\\
99.39	3.11095478206652e-06\\
99.4	2.84493407771459e-06\\
99.41	2.59001749209654e-06\\
99.42	2.34631454362755e-06\\
99.43	2.11393578021697e-06\\
99.44	1.89299278887875e-06\\
99.45	1.68359820545798e-06\\
99.46	1.48586572441996e-06\\
99.47	1.29991010874159e-06\\
99.48	1.12584719991031e-06\\
99.49	9.63793927985512e-07\\
99.5	8.13868321781347e-07\\
99.51	6.76189519131093e-07\\
99.52	5.50877777248659e-07\\
99.53	4.38054483194172e-07\\
99.54	3.37842164419358e-07\\
99.55	2.50364499436093e-07\\
99.56	1.7574632856128e-07\\
99.57	1.14113664783158e-07\\
99.58	6.55937047039368e-08\\
99.59	3.03148396090663e-08\\
99.6	8.40666662844936e-09\\
99.61	0\\
99.62	0\\
99.63	0\\
99.64	0\\
99.65	0\\
99.66	0\\
99.67	0\\
99.68	0\\
99.69	0\\
99.7	0\\
99.71	0\\
99.72	0\\
99.73	0\\
99.74	0\\
99.75	0\\
99.76	0\\
99.77	0\\
99.78	0\\
99.79	0\\
99.8	0\\
99.81	0\\
99.82	0\\
99.83	0\\
99.84	0\\
99.85	0\\
99.86	0\\
99.87	0\\
99.88	0\\
99.89	0\\
99.9	0\\
99.91	0\\
99.92	0\\
99.93	0\\
99.94	0\\
99.95	0\\
99.96	0\\
99.97	0\\
99.98	0\\
99.99	0\\
100	0\\
};
\addlegendentry{$q=-4$};

\addplot [color=mycolor1,dashed,forget plot]
  table[row sep=crcr]{%
0.01	0.01\\
0.02	0.01\\
0.03	0.01\\
0.04	0.01\\
0.05	0.01\\
0.06	0.01\\
0.07	0.01\\
0.08	0.01\\
0.09	0.01\\
0.1	0.01\\
0.11	0.01\\
0.12	0.01\\
0.13	0.01\\
0.14	0.01\\
0.15	0.01\\
0.16	0.01\\
0.17	0.01\\
0.18	0.01\\
0.19	0.01\\
0.2	0.01\\
0.21	0.01\\
0.22	0.01\\
0.23	0.01\\
0.24	0.01\\
0.25	0.01\\
0.26	0.01\\
0.27	0.01\\
0.28	0.01\\
0.29	0.01\\
0.3	0.01\\
0.31	0.01\\
0.32	0.01\\
0.33	0.01\\
0.34	0.01\\
0.35	0.01\\
0.36	0.01\\
0.37	0.01\\
0.38	0.01\\
0.39	0.01\\
0.4	0.01\\
0.41	0.01\\
0.42	0.01\\
0.43	0.01\\
0.44	0.01\\
0.45	0.01\\
0.46	0.01\\
0.47	0.01\\
0.48	0.01\\
0.49	0.01\\
0.5	0.01\\
0.51	0.01\\
0.52	0.01\\
0.53	0.01\\
0.54	0.01\\
0.55	0.01\\
0.56	0.01\\
0.57	0.01\\
0.58	0.01\\
0.59	0.01\\
0.6	0.01\\
0.61	0.01\\
0.62	0.01\\
0.63	0.01\\
0.64	0.01\\
0.65	0.01\\
0.66	0.01\\
0.67	0.01\\
0.68	0.01\\
0.69	0.01\\
0.7	0.01\\
0.71	0.01\\
0.72	0.01\\
0.73	0.01\\
0.74	0.01\\
0.75	0.01\\
0.76	0.01\\
0.77	0.01\\
0.78	0.01\\
0.79	0.01\\
0.8	0.01\\
0.81	0.01\\
0.82	0.01\\
0.83	0.01\\
0.84	0.01\\
0.85	0.01\\
0.86	0.01\\
0.87	0.01\\
0.88	0.01\\
0.89	0.01\\
0.9	0.01\\
0.91	0.01\\
0.92	0.01\\
0.93	0.01\\
0.94	0.01\\
0.95	0.01\\
0.96	0.01\\
0.97	0.01\\
0.98	0.01\\
0.99	0.01\\
1	0.01\\
1.01	0.01\\
1.02	0.01\\
1.03	0.01\\
1.04	0.01\\
1.05	0.01\\
1.06	0.01\\
1.07	0.01\\
1.08	0.01\\
1.09	0.01\\
1.1	0.01\\
1.11	0.01\\
1.12	0.01\\
1.13	0.01\\
1.14	0.01\\
1.15	0.01\\
1.16	0.01\\
1.17	0.01\\
1.18	0.01\\
1.19	0.01\\
1.2	0.01\\
1.21	0.01\\
1.22	0.01\\
1.23	0.01\\
1.24	0.01\\
1.25	0.01\\
1.26	0.01\\
1.27	0.01\\
1.28	0.01\\
1.29	0.01\\
1.3	0.01\\
1.31	0.01\\
1.32	0.01\\
1.33	0.01\\
1.34	0.01\\
1.35	0.01\\
1.36	0.01\\
1.37	0.01\\
1.38	0.01\\
1.39	0.01\\
1.4	0.01\\
1.41	0.01\\
1.42	0.01\\
1.43	0.01\\
1.44	0.01\\
1.45	0.01\\
1.46	0.01\\
1.47	0.01\\
1.48	0.01\\
1.49	0.01\\
1.5	0.01\\
1.51	0.01\\
1.52	0.01\\
1.53	0.01\\
1.54	0.01\\
1.55	0.01\\
1.56	0.01\\
1.57	0.01\\
1.58	0.01\\
1.59	0.01\\
1.6	0.01\\
1.61	0.01\\
1.62	0.01\\
1.63	0.01\\
1.64	0.01\\
1.65	0.01\\
1.66	0.01\\
1.67	0.01\\
1.68	0.01\\
1.69	0.01\\
1.7	0.01\\
1.71	0.01\\
1.72	0.01\\
1.73	0.01\\
1.74	0.01\\
1.75	0.01\\
1.76	0.01\\
1.77	0.01\\
1.78	0.01\\
1.79	0.01\\
1.8	0.01\\
1.81	0.01\\
1.82	0.01\\
1.83	0.01\\
1.84	0.01\\
1.85	0.01\\
1.86	0.01\\
1.87	0.01\\
1.88	0.01\\
1.89	0.01\\
1.9	0.01\\
1.91	0.01\\
1.92	0.01\\
1.93	0.01\\
1.94	0.01\\
1.95	0.01\\
1.96	0.01\\
1.97	0.01\\
1.98	0.01\\
1.99	0.01\\
2	0.01\\
2.01	0.01\\
2.02	0.01\\
2.03	0.01\\
2.04	0.01\\
2.05	0.01\\
2.06	0.01\\
2.07	0.01\\
2.08	0.01\\
2.09	0.01\\
2.1	0.01\\
2.11	0.01\\
2.12	0.01\\
2.13	0.01\\
2.14	0.01\\
2.15	0.01\\
2.16	0.01\\
2.17	0.01\\
2.18	0.01\\
2.19	0.01\\
2.2	0.01\\
2.21	0.01\\
2.22	0.01\\
2.23	0.01\\
2.24	0.01\\
2.25	0.01\\
2.26	0.01\\
2.27	0.01\\
2.28	0.01\\
2.29	0.01\\
2.3	0.01\\
2.31	0.01\\
2.32	0.01\\
2.33	0.01\\
2.34	0.01\\
2.35	0.01\\
2.36	0.01\\
2.37	0.01\\
2.38	0.01\\
2.39	0.01\\
2.4	0.01\\
2.41	0.01\\
2.42	0.01\\
2.43	0.01\\
2.44	0.01\\
2.45	0.01\\
2.46	0.01\\
2.47	0.01\\
2.48	0.01\\
2.49	0.01\\
2.5	0.01\\
2.51	0.01\\
2.52	0.01\\
2.53	0.01\\
2.54	0.01\\
2.55	0.01\\
2.56	0.01\\
2.57	0.01\\
2.58	0.01\\
2.59	0.01\\
2.6	0.01\\
2.61	0.01\\
2.62	0.01\\
2.63	0.01\\
2.64	0.01\\
2.65	0.01\\
2.66	0.01\\
2.67	0.01\\
2.68	0.01\\
2.69	0.01\\
2.7	0.01\\
2.71	0.01\\
2.72	0.01\\
2.73	0.01\\
2.74	0.01\\
2.75	0.01\\
2.76	0.01\\
2.77	0.01\\
2.78	0.01\\
2.79	0.01\\
2.8	0.01\\
2.81	0.01\\
2.82	0.01\\
2.83	0.01\\
2.84	0.01\\
2.85	0.01\\
2.86	0.01\\
2.87	0.01\\
2.88	0.01\\
2.89	0.01\\
2.9	0.01\\
2.91	0.01\\
2.92	0.01\\
2.93	0.01\\
2.94	0.01\\
2.95	0.01\\
2.96	0.01\\
2.97	0.01\\
2.98	0.01\\
2.99	0.01\\
3	0.01\\
3.01	0.01\\
3.02	0.01\\
3.03	0.01\\
3.04	0.01\\
3.05	0.01\\
3.06	0.01\\
3.07	0.01\\
3.08	0.01\\
3.09	0.01\\
3.1	0.01\\
3.11	0.01\\
3.12	0.01\\
3.13	0.01\\
3.14	0.01\\
3.15	0.01\\
3.16	0.01\\
3.17	0.01\\
3.18	0.01\\
3.19	0.01\\
3.2	0.01\\
3.21	0.01\\
3.22	0.01\\
3.23	0.01\\
3.24	0.01\\
3.25	0.01\\
3.26	0.01\\
3.27	0.01\\
3.28	0.01\\
3.29	0.01\\
3.3	0.01\\
3.31	0.01\\
3.32	0.01\\
3.33	0.01\\
3.34	0.01\\
3.35	0.01\\
3.36	0.01\\
3.37	0.01\\
3.38	0.01\\
3.39	0.01\\
3.4	0.01\\
3.41	0.01\\
3.42	0.01\\
3.43	0.01\\
3.44	0.01\\
3.45	0.01\\
3.46	0.01\\
3.47	0.01\\
3.48	0.01\\
3.49	0.01\\
3.5	0.01\\
3.51	0.01\\
3.52	0.01\\
3.53	0.01\\
3.54	0.01\\
3.55	0.01\\
3.56	0.01\\
3.57	0.01\\
3.58	0.01\\
3.59	0.01\\
3.6	0.01\\
3.61	0.01\\
3.62	0.01\\
3.63	0.01\\
3.64	0.01\\
3.65	0.01\\
3.66	0.01\\
3.67	0.01\\
3.68	0.01\\
3.69	0.01\\
3.7	0.01\\
3.71	0.01\\
3.72	0.01\\
3.73	0.01\\
3.74	0.01\\
3.75	0.01\\
3.76	0.01\\
3.77	0.01\\
3.78	0.01\\
3.79	0.01\\
3.8	0.01\\
3.81	0.01\\
3.82	0.01\\
3.83	0.01\\
3.84	0.01\\
3.85	0.01\\
3.86	0.01\\
3.87	0.01\\
3.88	0.01\\
3.89	0.01\\
3.9	0.01\\
3.91	0.01\\
3.92	0.01\\
3.93	0.01\\
3.94	0.01\\
3.95	0.01\\
3.96	0.01\\
3.97	0.01\\
3.98	0.01\\
3.99	0.01\\
4	0.01\\
4.01	0.01\\
4.02	0.01\\
4.03	0.01\\
4.04	0.01\\
4.05	0.01\\
4.06	0.01\\
4.07	0.01\\
4.08	0.01\\
4.09	0.01\\
4.1	0.01\\
4.11	0.01\\
4.12	0.01\\
4.13	0.01\\
4.14	0.01\\
4.15	0.01\\
4.16	0.01\\
4.17	0.01\\
4.18	0.01\\
4.19	0.01\\
4.2	0.01\\
4.21	0.01\\
4.22	0.01\\
4.23	0.01\\
4.24	0.01\\
4.25	0.01\\
4.26	0.01\\
4.27	0.01\\
4.28	0.01\\
4.29	0.01\\
4.3	0.01\\
4.31	0.01\\
4.32	0.01\\
4.33	0.01\\
4.34	0.01\\
4.35	0.01\\
4.36	0.01\\
4.37	0.01\\
4.38	0.01\\
4.39	0.01\\
4.4	0.01\\
4.41	0.01\\
4.42	0.01\\
4.43	0.01\\
4.44	0.01\\
4.45	0.01\\
4.46	0.01\\
4.47	0.01\\
4.48	0.01\\
4.49	0.01\\
4.5	0.01\\
4.51	0.01\\
4.52	0.01\\
4.53	0.01\\
4.54	0.01\\
4.55	0.01\\
4.56	0.01\\
4.57	0.01\\
4.58	0.01\\
4.59	0.01\\
4.6	0.01\\
4.61	0.01\\
4.62	0.01\\
4.63	0.01\\
4.64	0.01\\
4.65	0.01\\
4.66	0.01\\
4.67	0.01\\
4.68	0.01\\
4.69	0.01\\
4.7	0.01\\
4.71	0.01\\
4.72	0.01\\
4.73	0.01\\
4.74	0.01\\
4.75	0.01\\
4.76	0.01\\
4.77	0.01\\
4.78	0.01\\
4.79	0.01\\
4.8	0.01\\
4.81	0.01\\
4.82	0.01\\
4.83	0.01\\
4.84	0.01\\
4.85	0.01\\
4.86	0.01\\
4.87	0.01\\
4.88	0.01\\
4.89	0.01\\
4.9	0.01\\
4.91	0.01\\
4.92	0.01\\
4.93	0.01\\
4.94	0.01\\
4.95	0.01\\
4.96	0.01\\
4.97	0.01\\
4.98	0.01\\
4.99	0.01\\
5	0.01\\
5.01	0.01\\
5.02	0.01\\
5.03	0.01\\
5.04	0.01\\
5.05	0.01\\
5.06	0.01\\
5.07	0.01\\
5.08	0.01\\
5.09	0.01\\
5.1	0.01\\
5.11	0.01\\
5.12	0.01\\
5.13	0.01\\
5.14	0.01\\
5.15	0.01\\
5.16	0.01\\
5.17	0.01\\
5.18	0.01\\
5.19	0.01\\
5.2	0.01\\
5.21	0.01\\
5.22	0.01\\
5.23	0.01\\
5.24	0.01\\
5.25	0.01\\
5.26	0.01\\
5.27	0.01\\
5.28	0.01\\
5.29	0.01\\
5.3	0.01\\
5.31	0.01\\
5.32	0.01\\
5.33	0.01\\
5.34	0.01\\
5.35	0.01\\
5.36	0.01\\
5.37	0.01\\
5.38	0.01\\
5.39	0.01\\
5.4	0.01\\
5.41	0.01\\
5.42	0.01\\
5.43	0.01\\
5.44	0.01\\
5.45	0.01\\
5.46	0.01\\
5.47	0.01\\
5.48	0.01\\
5.49	0.01\\
5.5	0.01\\
5.51	0.01\\
5.52	0.01\\
5.53	0.01\\
5.54	0.01\\
5.55	0.01\\
5.56	0.01\\
5.57	0.01\\
5.58	0.01\\
5.59	0.01\\
5.6	0.01\\
5.61	0.01\\
5.62	0.01\\
5.63	0.01\\
5.64	0.01\\
5.65	0.01\\
5.66	0.01\\
5.67	0.01\\
5.68	0.01\\
5.69	0.01\\
5.7	0.01\\
5.71	0.01\\
5.72	0.01\\
5.73	0.01\\
5.74	0.01\\
5.75	0.01\\
5.76	0.01\\
5.77	0.01\\
5.78	0.01\\
5.79	0.01\\
5.8	0.01\\
5.81	0.01\\
5.82	0.01\\
5.83	0.01\\
5.84	0.01\\
5.85	0.01\\
5.86	0.01\\
5.87	0.01\\
5.88	0.01\\
5.89	0.01\\
5.9	0.01\\
5.91	0.01\\
5.92	0.01\\
5.93	0.01\\
5.94	0.01\\
5.95	0.01\\
5.96	0.01\\
5.97	0.01\\
5.98	0.01\\
5.99	0.01\\
6	0.01\\
6.01	0.01\\
6.02	0.01\\
6.03	0.01\\
6.04	0.01\\
6.05	0.01\\
6.06	0.01\\
6.07	0.01\\
6.08	0.01\\
6.09	0.01\\
6.1	0.01\\
6.11	0.01\\
6.12	0.01\\
6.13	0.01\\
6.14	0.01\\
6.15	0.01\\
6.16	0.01\\
6.17	0.01\\
6.18	0.01\\
6.19	0.01\\
6.2	0.01\\
6.21	0.01\\
6.22	0.01\\
6.23	0.01\\
6.24	0.01\\
6.25	0.01\\
6.26	0.01\\
6.27	0.01\\
6.28	0.01\\
6.29	0.01\\
6.3	0.01\\
6.31	0.01\\
6.32	0.01\\
6.33	0.01\\
6.34	0.01\\
6.35	0.01\\
6.36	0.01\\
6.37	0.01\\
6.38	0.01\\
6.39	0.01\\
6.4	0.01\\
6.41	0.01\\
6.42	0.01\\
6.43	0.01\\
6.44	0.01\\
6.45	0.01\\
6.46	0.01\\
6.47	0.01\\
6.48	0.01\\
6.49	0.01\\
6.5	0.01\\
6.51	0.01\\
6.52	0.01\\
6.53	0.01\\
6.54	0.01\\
6.55	0.01\\
6.56	0.01\\
6.57	0.01\\
6.58	0.01\\
6.59	0.01\\
6.6	0.01\\
6.61	0.01\\
6.62	0.01\\
6.63	0.01\\
6.64	0.01\\
6.65	0.01\\
6.66	0.01\\
6.67	0.01\\
6.68	0.01\\
6.69	0.01\\
6.7	0.01\\
6.71	0.01\\
6.72	0.01\\
6.73	0.01\\
6.74	0.01\\
6.75	0.01\\
6.76	0.01\\
6.77	0.01\\
6.78	0.01\\
6.79	0.01\\
6.8	0.01\\
6.81	0.01\\
6.82	0.01\\
6.83	0.01\\
6.84	0.01\\
6.85	0.01\\
6.86	0.01\\
6.87	0.01\\
6.88	0.01\\
6.89	0.01\\
6.9	0.01\\
6.91	0.01\\
6.92	0.01\\
6.93	0.01\\
6.94	0.01\\
6.95	0.01\\
6.96	0.01\\
6.97	0.01\\
6.98	0.01\\
6.99	0.01\\
7	0.01\\
7.01	0.01\\
7.02	0.01\\
7.03	0.01\\
7.04	0.01\\
7.05	0.01\\
7.06	0.01\\
7.07	0.01\\
7.08	0.01\\
7.09	0.01\\
7.1	0.01\\
7.11	0.01\\
7.12	0.01\\
7.13	0.01\\
7.14	0.01\\
7.15	0.01\\
7.16	0.01\\
7.17	0.01\\
7.18	0.01\\
7.19	0.01\\
7.2	0.01\\
7.21	0.01\\
7.22	0.01\\
7.23	0.01\\
7.24	0.01\\
7.25	0.01\\
7.26	0.01\\
7.27	0.01\\
7.28	0.01\\
7.29	0.01\\
7.3	0.01\\
7.31	0.01\\
7.32	0.01\\
7.33	0.01\\
7.34	0.01\\
7.35	0.01\\
7.36	0.01\\
7.37	0.01\\
7.38	0.01\\
7.39	0.01\\
7.4	0.01\\
7.41	0.01\\
7.42	0.01\\
7.43	0.01\\
7.44	0.01\\
7.45	0.01\\
7.46	0.01\\
7.47	0.01\\
7.48	0.01\\
7.49	0.01\\
7.5	0.01\\
7.51	0.01\\
7.52	0.01\\
7.53	0.01\\
7.54	0.01\\
7.55	0.01\\
7.56	0.01\\
7.57	0.01\\
7.58	0.01\\
7.59	0.01\\
7.6	0.01\\
7.61	0.01\\
7.62	0.01\\
7.63	0.01\\
7.64	0.01\\
7.65	0.01\\
7.66	0.01\\
7.67	0.01\\
7.68	0.01\\
7.69	0.01\\
7.7	0.01\\
7.71	0.01\\
7.72	0.01\\
7.73	0.01\\
7.74	0.01\\
7.75	0.01\\
7.76	0.01\\
7.77	0.01\\
7.78	0.01\\
7.79	0.01\\
7.8	0.01\\
7.81	0.01\\
7.82	0.01\\
7.83	0.01\\
7.84	0.01\\
7.85	0.01\\
7.86	0.01\\
7.87	0.01\\
7.88	0.01\\
7.89	0.01\\
7.9	0.01\\
7.91	0.01\\
7.92	0.01\\
7.93	0.01\\
7.94	0.01\\
7.95	0.01\\
7.96	0.01\\
7.97	0.01\\
7.98	0.01\\
7.99	0.01\\
8	0.01\\
8.01	0.01\\
8.02	0.01\\
8.03	0.01\\
8.04	0.01\\
8.05	0.01\\
8.06	0.01\\
8.07	0.01\\
8.08	0.01\\
8.09	0.01\\
8.1	0.01\\
8.11	0.01\\
8.12	0.01\\
8.13	0.01\\
8.14	0.01\\
8.15	0.01\\
8.16	0.01\\
8.17	0.01\\
8.18	0.01\\
8.19	0.01\\
8.2	0.01\\
8.21	0.01\\
8.22	0.01\\
8.23	0.01\\
8.24	0.01\\
8.25	0.01\\
8.26	0.01\\
8.27	0.01\\
8.28	0.01\\
8.29	0.01\\
8.3	0.01\\
8.31	0.01\\
8.32	0.01\\
8.33	0.01\\
8.34	0.01\\
8.35	0.01\\
8.36	0.01\\
8.37	0.01\\
8.38	0.01\\
8.39	0.01\\
8.4	0.01\\
8.41	0.01\\
8.42	0.01\\
8.43	0.01\\
8.44	0.01\\
8.45	0.01\\
8.46	0.01\\
8.47	0.01\\
8.48	0.01\\
8.49	0.01\\
8.5	0.01\\
8.51	0.01\\
8.52	0.01\\
8.53	0.01\\
8.54	0.01\\
8.55	0.01\\
8.56	0.01\\
8.57	0.01\\
8.58	0.01\\
8.59	0.01\\
8.6	0.01\\
8.61	0.01\\
8.62	0.01\\
8.63	0.01\\
8.64	0.01\\
8.65	0.01\\
8.66	0.01\\
8.67	0.01\\
8.68	0.01\\
8.69	0.01\\
8.7	0.01\\
8.71	0.01\\
8.72	0.01\\
8.73	0.01\\
8.74	0.01\\
8.75	0.01\\
8.76	0.01\\
8.77	0.01\\
8.78	0.01\\
8.79	0.01\\
8.8	0.01\\
8.81	0.01\\
8.82	0.01\\
8.83	0.01\\
8.84	0.01\\
8.85	0.01\\
8.86	0.01\\
8.87	0.01\\
8.88	0.01\\
8.89	0.01\\
8.9	0.01\\
8.91	0.01\\
8.92	0.01\\
8.93	0.01\\
8.94	0.01\\
8.95	0.01\\
8.96	0.01\\
8.97	0.01\\
8.98	0.01\\
8.99	0.01\\
9	0.01\\
9.01	0.01\\
9.02	0.01\\
9.03	0.01\\
9.04	0.01\\
9.05	0.01\\
9.06	0.01\\
9.07	0.01\\
9.08	0.01\\
9.09	0.01\\
9.1	0.01\\
9.11	0.01\\
9.12	0.01\\
9.13	0.01\\
9.14	0.01\\
9.15	0.01\\
9.16	0.01\\
9.17	0.01\\
9.18	0.01\\
9.19	0.01\\
9.2	0.01\\
9.21	0.01\\
9.22	0.01\\
9.23	0.01\\
9.24	0.01\\
9.25	0.01\\
9.26	0.01\\
9.27	0.01\\
9.28	0.01\\
9.29	0.01\\
9.3	0.01\\
9.31	0.01\\
9.32	0.01\\
9.33	0.01\\
9.34	0.01\\
9.35	0.01\\
9.36	0.01\\
9.37	0.01\\
9.38	0.01\\
9.39	0.01\\
9.4	0.01\\
9.41	0.01\\
9.42	0.01\\
9.43	0.01\\
9.44	0.01\\
9.45	0.01\\
9.46	0.01\\
9.47	0.01\\
9.48	0.01\\
9.49	0.01\\
9.5	0.01\\
9.51	0.01\\
9.52	0.01\\
9.53	0.01\\
9.54	0.01\\
9.55	0.01\\
9.56	0.01\\
9.57	0.01\\
9.58	0.01\\
9.59	0.01\\
9.6	0.01\\
9.61	0.01\\
9.62	0.01\\
9.63	0.01\\
9.64	0.01\\
9.65	0.01\\
9.66	0.01\\
9.67	0.01\\
9.68	0.01\\
9.69	0.01\\
9.7	0.01\\
9.71	0.01\\
9.72	0.01\\
9.73	0.01\\
9.74	0.01\\
9.75	0.01\\
9.76	0.01\\
9.77	0.01\\
9.78	0.01\\
9.79	0.01\\
9.8	0.01\\
9.81	0.01\\
9.82	0.01\\
9.83	0.01\\
9.84	0.01\\
9.85	0.01\\
9.86	0.01\\
9.87	0.01\\
9.88	0.01\\
9.89	0.01\\
9.9	0.01\\
9.91	0.01\\
9.92	0.01\\
9.93	0.01\\
9.94	0.01\\
9.95	0.01\\
9.96	0.01\\
9.97	0.01\\
9.98	0.01\\
9.99	0.01\\
10	0.01\\
10.01	0.01\\
10.02	0.01\\
10.03	0.01\\
10.04	0.01\\
10.05	0.01\\
10.06	0.01\\
10.07	0.01\\
10.08	0.01\\
10.09	0.01\\
10.1	0.01\\
10.11	0.01\\
10.12	0.01\\
10.13	0.01\\
10.14	0.01\\
10.15	0.01\\
10.16	0.01\\
10.17	0.01\\
10.18	0.01\\
10.19	0.01\\
10.2	0.01\\
10.21	0.01\\
10.22	0.01\\
10.23	0.01\\
10.24	0.01\\
10.25	0.01\\
10.26	0.01\\
10.27	0.01\\
10.28	0.01\\
10.29	0.01\\
10.3	0.01\\
10.31	0.01\\
10.32	0.01\\
10.33	0.01\\
10.34	0.01\\
10.35	0.01\\
10.36	0.01\\
10.37	0.01\\
10.38	0.01\\
10.39	0.01\\
10.4	0.01\\
10.41	0.01\\
10.42	0.01\\
10.43	0.01\\
10.44	0.01\\
10.45	0.01\\
10.46	0.01\\
10.47	0.01\\
10.48	0.01\\
10.49	0.01\\
10.5	0.01\\
10.51	0.01\\
10.52	0.01\\
10.53	0.01\\
10.54	0.01\\
10.55	0.01\\
10.56	0.01\\
10.57	0.01\\
10.58	0.01\\
10.59	0.01\\
10.6	0.01\\
10.61	0.01\\
10.62	0.01\\
10.63	0.01\\
10.64	0.01\\
10.65	0.01\\
10.66	0.01\\
10.67	0.01\\
10.68	0.01\\
10.69	0.01\\
10.7	0.01\\
10.71	0.01\\
10.72	0.01\\
10.73	0.01\\
10.74	0.01\\
10.75	0.01\\
10.76	0.01\\
10.77	0.01\\
10.78	0.01\\
10.79	0.01\\
10.8	0.01\\
10.81	0.01\\
10.82	0.01\\
10.83	0.01\\
10.84	0.01\\
10.85	0.01\\
10.86	0.01\\
10.87	0.01\\
10.88	0.01\\
10.89	0.01\\
10.9	0.01\\
10.91	0.01\\
10.92	0.01\\
10.93	0.01\\
10.94	0.01\\
10.95	0.01\\
10.96	0.01\\
10.97	0.01\\
10.98	0.01\\
10.99	0.01\\
11	0.01\\
11.01	0.01\\
11.02	0.01\\
11.03	0.01\\
11.04	0.01\\
11.05	0.01\\
11.06	0.01\\
11.07	0.01\\
11.08	0.01\\
11.09	0.01\\
11.1	0.01\\
11.11	0.01\\
11.12	0.01\\
11.13	0.01\\
11.14	0.01\\
11.15	0.01\\
11.16	0.01\\
11.17	0.01\\
11.18	0.01\\
11.19	0.01\\
11.2	0.01\\
11.21	0.01\\
11.22	0.01\\
11.23	0.01\\
11.24	0.01\\
11.25	0.01\\
11.26	0.01\\
11.27	0.01\\
11.28	0.01\\
11.29	0.01\\
11.3	0.01\\
11.31	0.01\\
11.32	0.01\\
11.33	0.01\\
11.34	0.01\\
11.35	0.01\\
11.36	0.01\\
11.37	0.01\\
11.38	0.01\\
11.39	0.01\\
11.4	0.01\\
11.41	0.01\\
11.42	0.01\\
11.43	0.01\\
11.44	0.01\\
11.45	0.01\\
11.46	0.01\\
11.47	0.01\\
11.48	0.01\\
11.49	0.01\\
11.5	0.01\\
11.51	0.01\\
11.52	0.01\\
11.53	0.01\\
11.54	0.01\\
11.55	0.01\\
11.56	0.01\\
11.57	0.01\\
11.58	0.01\\
11.59	0.01\\
11.6	0.01\\
11.61	0.01\\
11.62	0.01\\
11.63	0.01\\
11.64	0.01\\
11.65	0.01\\
11.66	0.01\\
11.67	0.01\\
11.68	0.01\\
11.69	0.01\\
11.7	0.01\\
11.71	0.01\\
11.72	0.01\\
11.73	0.01\\
11.74	0.01\\
11.75	0.01\\
11.76	0.01\\
11.77	0.01\\
11.78	0.01\\
11.79	0.01\\
11.8	0.01\\
11.81	0.01\\
11.82	0.01\\
11.83	0.01\\
11.84	0.01\\
11.85	0.01\\
11.86	0.01\\
11.87	0.01\\
11.88	0.01\\
11.89	0.01\\
11.9	0.01\\
11.91	0.01\\
11.92	0.01\\
11.93	0.01\\
11.94	0.01\\
11.95	0.01\\
11.96	0.01\\
11.97	0.01\\
11.98	0.01\\
11.99	0.01\\
12	0.01\\
12.01	0.01\\
12.02	0.01\\
12.03	0.01\\
12.04	0.01\\
12.05	0.01\\
12.06	0.01\\
12.07	0.01\\
12.08	0.01\\
12.09	0.01\\
12.1	0.01\\
12.11	0.01\\
12.12	0.01\\
12.13	0.01\\
12.14	0.01\\
12.15	0.01\\
12.16	0.01\\
12.17	0.01\\
12.18	0.01\\
12.19	0.01\\
12.2	0.01\\
12.21	0.01\\
12.22	0.01\\
12.23	0.01\\
12.24	0.01\\
12.25	0.01\\
12.26	0.01\\
12.27	0.01\\
12.28	0.01\\
12.29	0.01\\
12.3	0.01\\
12.31	0.01\\
12.32	0.01\\
12.33	0.01\\
12.34	0.01\\
12.35	0.01\\
12.36	0.01\\
12.37	0.01\\
12.38	0.01\\
12.39	0.01\\
12.4	0.01\\
12.41	0.01\\
12.42	0.01\\
12.43	0.01\\
12.44	0.01\\
12.45	0.01\\
12.46	0.01\\
12.47	0.01\\
12.48	0.01\\
12.49	0.01\\
12.5	0.01\\
12.51	0.01\\
12.52	0.01\\
12.53	0.01\\
12.54	0.01\\
12.55	0.01\\
12.56	0.01\\
12.57	0.01\\
12.58	0.01\\
12.59	0.01\\
12.6	0.01\\
12.61	0.01\\
12.62	0.01\\
12.63	0.01\\
12.64	0.01\\
12.65	0.01\\
12.66	0.01\\
12.67	0.01\\
12.68	0.01\\
12.69	0.01\\
12.7	0.01\\
12.71	0.01\\
12.72	0.01\\
12.73	0.01\\
12.74	0.01\\
12.75	0.01\\
12.76	0.01\\
12.77	0.01\\
12.78	0.01\\
12.79	0.01\\
12.8	0.01\\
12.81	0.01\\
12.82	0.01\\
12.83	0.01\\
12.84	0.01\\
12.85	0.01\\
12.86	0.01\\
12.87	0.01\\
12.88	0.01\\
12.89	0.01\\
12.9	0.01\\
12.91	0.01\\
12.92	0.01\\
12.93	0.01\\
12.94	0.01\\
12.95	0.01\\
12.96	0.01\\
12.97	0.01\\
12.98	0.01\\
12.99	0.01\\
13	0.01\\
13.01	0.01\\
13.02	0.01\\
13.03	0.01\\
13.04	0.01\\
13.05	0.01\\
13.06	0.01\\
13.07	0.01\\
13.08	0.01\\
13.09	0.01\\
13.1	0.01\\
13.11	0.01\\
13.12	0.01\\
13.13	0.01\\
13.14	0.01\\
13.15	0.01\\
13.16	0.01\\
13.17	0.01\\
13.18	0.01\\
13.19	0.01\\
13.2	0.01\\
13.21	0.01\\
13.22	0.01\\
13.23	0.01\\
13.24	0.01\\
13.25	0.01\\
13.26	0.01\\
13.27	0.01\\
13.28	0.01\\
13.29	0.01\\
13.3	0.01\\
13.31	0.01\\
13.32	0.01\\
13.33	0.01\\
13.34	0.01\\
13.35	0.01\\
13.36	0.01\\
13.37	0.01\\
13.38	0.01\\
13.39	0.01\\
13.4	0.01\\
13.41	0.01\\
13.42	0.01\\
13.43	0.01\\
13.44	0.01\\
13.45	0.01\\
13.46	0.01\\
13.47	0.01\\
13.48	0.01\\
13.49	0.01\\
13.5	0.01\\
13.51	0.01\\
13.52	0.01\\
13.53	0.01\\
13.54	0.01\\
13.55	0.01\\
13.56	0.01\\
13.57	0.01\\
13.58	0.01\\
13.59	0.01\\
13.6	0.01\\
13.61	0.01\\
13.62	0.01\\
13.63	0.01\\
13.64	0.01\\
13.65	0.01\\
13.66	0.01\\
13.67	0.01\\
13.68	0.01\\
13.69	0.01\\
13.7	0.01\\
13.71	0.01\\
13.72	0.01\\
13.73	0.01\\
13.74	0.01\\
13.75	0.01\\
13.76	0.01\\
13.77	0.01\\
13.78	0.01\\
13.79	0.01\\
13.8	0.01\\
13.81	0.01\\
13.82	0.01\\
13.83	0.01\\
13.84	0.01\\
13.85	0.01\\
13.86	0.01\\
13.87	0.01\\
13.88	0.01\\
13.89	0.01\\
13.9	0.01\\
13.91	0.01\\
13.92	0.01\\
13.93	0.01\\
13.94	0.01\\
13.95	0.01\\
13.96	0.01\\
13.97	0.01\\
13.98	0.01\\
13.99	0.01\\
14	0.01\\
14.01	0.01\\
14.02	0.01\\
14.03	0.01\\
14.04	0.01\\
14.05	0.01\\
14.06	0.01\\
14.07	0.01\\
14.08	0.01\\
14.09	0.01\\
14.1	0.01\\
14.11	0.01\\
14.12	0.01\\
14.13	0.01\\
14.14	0.01\\
14.15	0.01\\
14.16	0.01\\
14.17	0.01\\
14.18	0.01\\
14.19	0.01\\
14.2	0.01\\
14.21	0.01\\
14.22	0.01\\
14.23	0.01\\
14.24	0.01\\
14.25	0.01\\
14.26	0.01\\
14.27	0.01\\
14.28	0.01\\
14.29	0.01\\
14.3	0.01\\
14.31	0.01\\
14.32	0.01\\
14.33	0.01\\
14.34	0.01\\
14.35	0.01\\
14.36	0.01\\
14.37	0.01\\
14.38	0.01\\
14.39	0.01\\
14.4	0.01\\
14.41	0.01\\
14.42	0.01\\
14.43	0.01\\
14.44	0.01\\
14.45	0.01\\
14.46	0.01\\
14.47	0.01\\
14.48	0.01\\
14.49	0.01\\
14.5	0.01\\
14.51	0.01\\
14.52	0.01\\
14.53	0.01\\
14.54	0.01\\
14.55	0.01\\
14.56	0.01\\
14.57	0.01\\
14.58	0.01\\
14.59	0.01\\
14.6	0.01\\
14.61	0.01\\
14.62	0.01\\
14.63	0.01\\
14.64	0.01\\
14.65	0.01\\
14.66	0.01\\
14.67	0.01\\
14.68	0.01\\
14.69	0.01\\
14.7	0.01\\
14.71	0.01\\
14.72	0.01\\
14.73	0.01\\
14.74	0.01\\
14.75	0.01\\
14.76	0.01\\
14.77	0.01\\
14.78	0.01\\
14.79	0.01\\
14.8	0.01\\
14.81	0.01\\
14.82	0.01\\
14.83	0.01\\
14.84	0.01\\
14.85	0.01\\
14.86	0.01\\
14.87	0.01\\
14.88	0.01\\
14.89	0.01\\
14.9	0.01\\
14.91	0.01\\
14.92	0.01\\
14.93	0.01\\
14.94	0.01\\
14.95	0.01\\
14.96	0.01\\
14.97	0.01\\
14.98	0.01\\
14.99	0.01\\
15	0.01\\
15.01	0.01\\
15.02	0.01\\
15.03	0.01\\
15.04	0.01\\
15.05	0.01\\
15.06	0.01\\
15.07	0.01\\
15.08	0.01\\
15.09	0.01\\
15.1	0.01\\
15.11	0.01\\
15.12	0.01\\
15.13	0.01\\
15.14	0.01\\
15.15	0.01\\
15.16	0.01\\
15.17	0.01\\
15.18	0.01\\
15.19	0.01\\
15.2	0.01\\
15.21	0.01\\
15.22	0.01\\
15.23	0.01\\
15.24	0.01\\
15.25	0.01\\
15.26	0.01\\
15.27	0.01\\
15.28	0.01\\
15.29	0.01\\
15.3	0.01\\
15.31	0.01\\
15.32	0.01\\
15.33	0.01\\
15.34	0.01\\
15.35	0.01\\
15.36	0.01\\
15.37	0.01\\
15.38	0.01\\
15.39	0.01\\
15.4	0.01\\
15.41	0.01\\
15.42	0.01\\
15.43	0.01\\
15.44	0.01\\
15.45	0.01\\
15.46	0.01\\
15.47	0.01\\
15.48	0.01\\
15.49	0.01\\
15.5	0.01\\
15.51	0.01\\
15.52	0.01\\
15.53	0.01\\
15.54	0.01\\
15.55	0.01\\
15.56	0.01\\
15.57	0.01\\
15.58	0.01\\
15.59	0.01\\
15.6	0.01\\
15.61	0.01\\
15.62	0.01\\
15.63	0.01\\
15.64	0.01\\
15.65	0.01\\
15.66	0.01\\
15.67	0.01\\
15.68	0.01\\
15.69	0.01\\
15.7	0.01\\
15.71	0.01\\
15.72	0.01\\
15.73	0.01\\
15.74	0.01\\
15.75	0.01\\
15.76	0.01\\
15.77	0.01\\
15.78	0.01\\
15.79	0.01\\
15.8	0.01\\
15.81	0.01\\
15.82	0.01\\
15.83	0.01\\
15.84	0.01\\
15.85	0.01\\
15.86	0.01\\
15.87	0.01\\
15.88	0.01\\
15.89	0.01\\
15.9	0.01\\
15.91	0.01\\
15.92	0.01\\
15.93	0.01\\
15.94	0.01\\
15.95	0.01\\
15.96	0.01\\
15.97	0.01\\
15.98	0.01\\
15.99	0.01\\
16	0.01\\
16.01	0.01\\
16.02	0.01\\
16.03	0.01\\
16.04	0.01\\
16.05	0.01\\
16.06	0.01\\
16.07	0.01\\
16.08	0.01\\
16.09	0.01\\
16.1	0.01\\
16.11	0.01\\
16.12	0.01\\
16.13	0.01\\
16.14	0.01\\
16.15	0.01\\
16.16	0.01\\
16.17	0.01\\
16.18	0.01\\
16.19	0.01\\
16.2	0.01\\
16.21	0.01\\
16.22	0.01\\
16.23	0.01\\
16.24	0.01\\
16.25	0.01\\
16.26	0.01\\
16.27	0.01\\
16.28	0.01\\
16.29	0.01\\
16.3	0.01\\
16.31	0.01\\
16.32	0.01\\
16.33	0.01\\
16.34	0.01\\
16.35	0.01\\
16.36	0.01\\
16.37	0.01\\
16.38	0.01\\
16.39	0.01\\
16.4	0.01\\
16.41	0.01\\
16.42	0.01\\
16.43	0.01\\
16.44	0.01\\
16.45	0.01\\
16.46	0.01\\
16.47	0.01\\
16.48	0.01\\
16.49	0.01\\
16.5	0.01\\
16.51	0.01\\
16.52	0.01\\
16.53	0.01\\
16.54	0.01\\
16.55	0.01\\
16.56	0.01\\
16.57	0.01\\
16.58	0.01\\
16.59	0.01\\
16.6	0.01\\
16.61	0.01\\
16.62	0.01\\
16.63	0.01\\
16.64	0.01\\
16.65	0.01\\
16.66	0.01\\
16.67	0.01\\
16.68	0.01\\
16.69	0.01\\
16.7	0.01\\
16.71	0.01\\
16.72	0.01\\
16.73	0.01\\
16.74	0.01\\
16.75	0.01\\
16.76	0.01\\
16.77	0.01\\
16.78	0.01\\
16.79	0.01\\
16.8	0.01\\
16.81	0.01\\
16.82	0.01\\
16.83	0.01\\
16.84	0.01\\
16.85	0.01\\
16.86	0.01\\
16.87	0.01\\
16.88	0.01\\
16.89	0.01\\
16.9	0.01\\
16.91	0.01\\
16.92	0.01\\
16.93	0.01\\
16.94	0.01\\
16.95	0.01\\
16.96	0.01\\
16.97	0.01\\
16.98	0.01\\
16.99	0.01\\
17	0.01\\
17.01	0.01\\
17.02	0.01\\
17.03	0.01\\
17.04	0.01\\
17.05	0.01\\
17.06	0.01\\
17.07	0.01\\
17.08	0.01\\
17.09	0.01\\
17.1	0.01\\
17.11	0.01\\
17.12	0.01\\
17.13	0.01\\
17.14	0.01\\
17.15	0.01\\
17.16	0.01\\
17.17	0.01\\
17.18	0.01\\
17.19	0.01\\
17.2	0.01\\
17.21	0.01\\
17.22	0.01\\
17.23	0.01\\
17.24	0.01\\
17.25	0.01\\
17.26	0.01\\
17.27	0.01\\
17.28	0.01\\
17.29	0.01\\
17.3	0.01\\
17.31	0.01\\
17.32	0.01\\
17.33	0.01\\
17.34	0.01\\
17.35	0.01\\
17.36	0.01\\
17.37	0.01\\
17.38	0.01\\
17.39	0.01\\
17.4	0.01\\
17.41	0.01\\
17.42	0.01\\
17.43	0.01\\
17.44	0.01\\
17.45	0.01\\
17.46	0.01\\
17.47	0.01\\
17.48	0.01\\
17.49	0.01\\
17.5	0.01\\
17.51	0.01\\
17.52	0.01\\
17.53	0.01\\
17.54	0.01\\
17.55	0.01\\
17.56	0.01\\
17.57	0.01\\
17.58	0.01\\
17.59	0.01\\
17.6	0.01\\
17.61	0.01\\
17.62	0.01\\
17.63	0.01\\
17.64	0.01\\
17.65	0.01\\
17.66	0.01\\
17.67	0.01\\
17.68	0.01\\
17.69	0.01\\
17.7	0.01\\
17.71	0.01\\
17.72	0.01\\
17.73	0.01\\
17.74	0.01\\
17.75	0.01\\
17.76	0.01\\
17.77	0.01\\
17.78	0.01\\
17.79	0.01\\
17.8	0.01\\
17.81	0.01\\
17.82	0.01\\
17.83	0.01\\
17.84	0.01\\
17.85	0.01\\
17.86	0.01\\
17.87	0.01\\
17.88	0.01\\
17.89	0.01\\
17.9	0.01\\
17.91	0.01\\
17.92	0.01\\
17.93	0.01\\
17.94	0.01\\
17.95	0.01\\
17.96	0.01\\
17.97	0.01\\
17.98	0.01\\
17.99	0.01\\
18	0.01\\
18.01	0.01\\
18.02	0.01\\
18.03	0.01\\
18.04	0.01\\
18.05	0.01\\
18.06	0.01\\
18.07	0.01\\
18.08	0.01\\
18.09	0.01\\
18.1	0.01\\
18.11	0.01\\
18.12	0.01\\
18.13	0.01\\
18.14	0.01\\
18.15	0.01\\
18.16	0.01\\
18.17	0.01\\
18.18	0.01\\
18.19	0.01\\
18.2	0.01\\
18.21	0.01\\
18.22	0.01\\
18.23	0.01\\
18.24	0.01\\
18.25	0.01\\
18.26	0.01\\
18.27	0.01\\
18.28	0.01\\
18.29	0.01\\
18.3	0.01\\
18.31	0.01\\
18.32	0.01\\
18.33	0.01\\
18.34	0.01\\
18.35	0.01\\
18.36	0.01\\
18.37	0.01\\
18.38	0.01\\
18.39	0.01\\
18.4	0.01\\
18.41	0.01\\
18.42	0.01\\
18.43	0.01\\
18.44	0.01\\
18.45	0.01\\
18.46	0.01\\
18.47	0.01\\
18.48	0.01\\
18.49	0.01\\
18.5	0.01\\
18.51	0.01\\
18.52	0.01\\
18.53	0.01\\
18.54	0.01\\
18.55	0.01\\
18.56	0.01\\
18.57	0.01\\
18.58	0.01\\
18.59	0.01\\
18.6	0.01\\
18.61	0.01\\
18.62	0.01\\
18.63	0.01\\
18.64	0.01\\
18.65	0.01\\
18.66	0.01\\
18.67	0.01\\
18.68	0.01\\
18.69	0.01\\
18.7	0.01\\
18.71	0.01\\
18.72	0.01\\
18.73	0.01\\
18.74	0.01\\
18.75	0.01\\
18.76	0.01\\
18.77	0.01\\
18.78	0.01\\
18.79	0.01\\
18.8	0.01\\
18.81	0.01\\
18.82	0.01\\
18.83	0.01\\
18.84	0.01\\
18.85	0.01\\
18.86	0.01\\
18.87	0.01\\
18.88	0.01\\
18.89	0.01\\
18.9	0.01\\
18.91	0.01\\
18.92	0.01\\
18.93	0.01\\
18.94	0.01\\
18.95	0.01\\
18.96	0.01\\
18.97	0.01\\
18.98	0.01\\
18.99	0.01\\
19	0.01\\
19.01	0.01\\
19.02	0.01\\
19.03	0.01\\
19.04	0.01\\
19.05	0.01\\
19.06	0.01\\
19.07	0.01\\
19.08	0.01\\
19.09	0.01\\
19.1	0.01\\
19.11	0.01\\
19.12	0.01\\
19.13	0.01\\
19.14	0.01\\
19.15	0.01\\
19.16	0.01\\
19.17	0.01\\
19.18	0.01\\
19.19	0.01\\
19.2	0.01\\
19.21	0.01\\
19.22	0.01\\
19.23	0.01\\
19.24	0.01\\
19.25	0.01\\
19.26	0.01\\
19.27	0.01\\
19.28	0.01\\
19.29	0.01\\
19.3	0.01\\
19.31	0.01\\
19.32	0.01\\
19.33	0.01\\
19.34	0.01\\
19.35	0.01\\
19.36	0.01\\
19.37	0.01\\
19.38	0.01\\
19.39	0.01\\
19.4	0.01\\
19.41	0.01\\
19.42	0.01\\
19.43	0.01\\
19.44	0.01\\
19.45	0.01\\
19.46	0.01\\
19.47	0.01\\
19.48	0.01\\
19.49	0.01\\
19.5	0.01\\
19.51	0.01\\
19.52	0.01\\
19.53	0.01\\
19.54	0.01\\
19.55	0.01\\
19.56	0.01\\
19.57	0.01\\
19.58	0.01\\
19.59	0.01\\
19.6	0.01\\
19.61	0.01\\
19.62	0.01\\
19.63	0.01\\
19.64	0.01\\
19.65	0.01\\
19.66	0.01\\
19.67	0.01\\
19.68	0.01\\
19.69	0.01\\
19.7	0.01\\
19.71	0.01\\
19.72	0.01\\
19.73	0.01\\
19.74	0.01\\
19.75	0.01\\
19.76	0.01\\
19.77	0.01\\
19.78	0.01\\
19.79	0.01\\
19.8	0.01\\
19.81	0.01\\
19.82	0.01\\
19.83	0.01\\
19.84	0.01\\
19.85	0.01\\
19.86	0.01\\
19.87	0.01\\
19.88	0.01\\
19.89	0.01\\
19.9	0.01\\
19.91	0.01\\
19.92	0.01\\
19.93	0.01\\
19.94	0.01\\
19.95	0.01\\
19.96	0.01\\
19.97	0.01\\
19.98	0.01\\
19.99	0.01\\
20	0.01\\
20.01	0.01\\
20.02	0.01\\
20.03	0.01\\
20.04	0.01\\
20.05	0.01\\
20.06	0.01\\
20.07	0.01\\
20.08	0.01\\
20.09	0.01\\
20.1	0.01\\
20.11	0.01\\
20.12	0.01\\
20.13	0.01\\
20.14	0.01\\
20.15	0.01\\
20.16	0.01\\
20.17	0.01\\
20.18	0.01\\
20.19	0.01\\
20.2	0.01\\
20.21	0.01\\
20.22	0.01\\
20.23	0.01\\
20.24	0.01\\
20.25	0.01\\
20.26	0.01\\
20.27	0.01\\
20.28	0.01\\
20.29	0.01\\
20.3	0.01\\
20.31	0.01\\
20.32	0.01\\
20.33	0.01\\
20.34	0.01\\
20.35	0.01\\
20.36	0.01\\
20.37	0.01\\
20.38	0.01\\
20.39	0.01\\
20.4	0.01\\
20.41	0.01\\
20.42	0.01\\
20.43	0.01\\
20.44	0.01\\
20.45	0.01\\
20.46	0.01\\
20.47	0.01\\
20.48	0.01\\
20.49	0.01\\
20.5	0.01\\
20.51	0.01\\
20.52	0.01\\
20.53	0.01\\
20.54	0.01\\
20.55	0.01\\
20.56	0.01\\
20.57	0.01\\
20.58	0.01\\
20.59	0.01\\
20.6	0.01\\
20.61	0.01\\
20.62	0.01\\
20.63	0.01\\
20.64	0.01\\
20.65	0.01\\
20.66	0.01\\
20.67	0.01\\
20.68	0.01\\
20.69	0.01\\
20.7	0.01\\
20.71	0.01\\
20.72	0.01\\
20.73	0.01\\
20.74	0.01\\
20.75	0.01\\
20.76	0.01\\
20.77	0.01\\
20.78	0.01\\
20.79	0.01\\
20.8	0.01\\
20.81	0.01\\
20.82	0.01\\
20.83	0.01\\
20.84	0.01\\
20.85	0.01\\
20.86	0.01\\
20.87	0.01\\
20.88	0.01\\
20.89	0.01\\
20.9	0.01\\
20.91	0.01\\
20.92	0.01\\
20.93	0.01\\
20.94	0.01\\
20.95	0.01\\
20.96	0.01\\
20.97	0.01\\
20.98	0.01\\
20.99	0.01\\
21	0.01\\
21.01	0.01\\
21.02	0.01\\
21.03	0.01\\
21.04	0.01\\
21.05	0.01\\
21.06	0.01\\
21.07	0.01\\
21.08	0.01\\
21.09	0.01\\
21.1	0.01\\
21.11	0.01\\
21.12	0.01\\
21.13	0.01\\
21.14	0.01\\
21.15	0.01\\
21.16	0.01\\
21.17	0.01\\
21.18	0.01\\
21.19	0.01\\
21.2	0.01\\
21.21	0.01\\
21.22	0.01\\
21.23	0.01\\
21.24	0.01\\
21.25	0.01\\
21.26	0.01\\
21.27	0.01\\
21.28	0.01\\
21.29	0.01\\
21.3	0.01\\
21.31	0.01\\
21.32	0.01\\
21.33	0.01\\
21.34	0.01\\
21.35	0.01\\
21.36	0.01\\
21.37	0.01\\
21.38	0.01\\
21.39	0.01\\
21.4	0.01\\
21.41	0.01\\
21.42	0.01\\
21.43	0.01\\
21.44	0.01\\
21.45	0.01\\
21.46	0.01\\
21.47	0.01\\
21.48	0.01\\
21.49	0.01\\
21.5	0.01\\
21.51	0.01\\
21.52	0.01\\
21.53	0.01\\
21.54	0.01\\
21.55	0.01\\
21.56	0.01\\
21.57	0.01\\
21.58	0.01\\
21.59	0.01\\
21.6	0.01\\
21.61	0.01\\
21.62	0.01\\
21.63	0.01\\
21.64	0.01\\
21.65	0.01\\
21.66	0.01\\
21.67	0.01\\
21.68	0.01\\
21.69	0.01\\
21.7	0.01\\
21.71	0.01\\
21.72	0.01\\
21.73	0.01\\
21.74	0.01\\
21.75	0.01\\
21.76	0.01\\
21.77	0.01\\
21.78	0.01\\
21.79	0.01\\
21.8	0.01\\
21.81	0.01\\
21.82	0.01\\
21.83	0.01\\
21.84	0.01\\
21.85	0.01\\
21.86	0.01\\
21.87	0.01\\
21.88	0.01\\
21.89	0.01\\
21.9	0.01\\
21.91	0.01\\
21.92	0.01\\
21.93	0.01\\
21.94	0.01\\
21.95	0.01\\
21.96	0.01\\
21.97	0.01\\
21.98	0.01\\
21.99	0.01\\
22	0.01\\
22.01	0.01\\
22.02	0.01\\
22.03	0.01\\
22.04	0.01\\
22.05	0.01\\
22.06	0.01\\
22.07	0.01\\
22.08	0.01\\
22.09	0.01\\
22.1	0.01\\
22.11	0.01\\
22.12	0.01\\
22.13	0.01\\
22.14	0.01\\
22.15	0.01\\
22.16	0.01\\
22.17	0.01\\
22.18	0.01\\
22.19	0.01\\
22.2	0.01\\
22.21	0.01\\
22.22	0.01\\
22.23	0.01\\
22.24	0.01\\
22.25	0.01\\
22.26	0.01\\
22.27	0.01\\
22.28	0.01\\
22.29	0.01\\
22.3	0.01\\
22.31	0.01\\
22.32	0.01\\
22.33	0.01\\
22.34	0.01\\
22.35	0.01\\
22.36	0.01\\
22.37	0.01\\
22.38	0.01\\
22.39	0.01\\
22.4	0.01\\
22.41	0.01\\
22.42	0.01\\
22.43	0.01\\
22.44	0.01\\
22.45	0.01\\
22.46	0.01\\
22.47	0.01\\
22.48	0.01\\
22.49	0.01\\
22.5	0.01\\
22.51	0.01\\
22.52	0.01\\
22.53	0.01\\
22.54	0.01\\
22.55	0.01\\
22.56	0.01\\
22.57	0.01\\
22.58	0.01\\
22.59	0.01\\
22.6	0.01\\
22.61	0.01\\
22.62	0.01\\
22.63	0.01\\
22.64	0.01\\
22.65	0.01\\
22.66	0.01\\
22.67	0.01\\
22.68	0.01\\
22.69	0.01\\
22.7	0.01\\
22.71	0.01\\
22.72	0.01\\
22.73	0.01\\
22.74	0.01\\
22.75	0.01\\
22.76	0.01\\
22.77	0.01\\
22.78	0.01\\
22.79	0.01\\
22.8	0.01\\
22.81	0.01\\
22.82	0.01\\
22.83	0.01\\
22.84	0.01\\
22.85	0.01\\
22.86	0.01\\
22.87	0.01\\
22.88	0.01\\
22.89	0.01\\
22.9	0.01\\
22.91	0.01\\
22.92	0.01\\
22.93	0.01\\
22.94	0.01\\
22.95	0.01\\
22.96	0.01\\
22.97	0.01\\
22.98	0.01\\
22.99	0.01\\
23	0.01\\
23.01	0.01\\
23.02	0.01\\
23.03	0.01\\
23.04	0.01\\
23.05	0.01\\
23.06	0.01\\
23.07	0.01\\
23.08	0.01\\
23.09	0.01\\
23.1	0.01\\
23.11	0.01\\
23.12	0.01\\
23.13	0.01\\
23.14	0.01\\
23.15	0.01\\
23.16	0.01\\
23.17	0.01\\
23.18	0.01\\
23.19	0.01\\
23.2	0.01\\
23.21	0.01\\
23.22	0.01\\
23.23	0.01\\
23.24	0.01\\
23.25	0.01\\
23.26	0.01\\
23.27	0.01\\
23.28	0.01\\
23.29	0.01\\
23.3	0.01\\
23.31	0.01\\
23.32	0.01\\
23.33	0.01\\
23.34	0.01\\
23.35	0.01\\
23.36	0.01\\
23.37	0.01\\
23.38	0.01\\
23.39	0.01\\
23.4	0.01\\
23.41	0.01\\
23.42	0.01\\
23.43	0.01\\
23.44	0.01\\
23.45	0.01\\
23.46	0.01\\
23.47	0.01\\
23.48	0.01\\
23.49	0.01\\
23.5	0.01\\
23.51	0.01\\
23.52	0.01\\
23.53	0.01\\
23.54	0.01\\
23.55	0.01\\
23.56	0.01\\
23.57	0.01\\
23.58	0.01\\
23.59	0.01\\
23.6	0.01\\
23.61	0.01\\
23.62	0.01\\
23.63	0.01\\
23.64	0.01\\
23.65	0.01\\
23.66	0.01\\
23.67	0.01\\
23.68	0.01\\
23.69	0.01\\
23.7	0.01\\
23.71	0.01\\
23.72	0.01\\
23.73	0.01\\
23.74	0.01\\
23.75	0.01\\
23.76	0.01\\
23.77	0.01\\
23.78	0.01\\
23.79	0.01\\
23.8	0.01\\
23.81	0.01\\
23.82	0.01\\
23.83	0.01\\
23.84	0.01\\
23.85	0.01\\
23.86	0.01\\
23.87	0.01\\
23.88	0.01\\
23.89	0.01\\
23.9	0.01\\
23.91	0.01\\
23.92	0.01\\
23.93	0.01\\
23.94	0.01\\
23.95	0.01\\
23.96	0.01\\
23.97	0.01\\
23.98	0.01\\
23.99	0.01\\
24	0.01\\
24.01	0.01\\
24.02	0.01\\
24.03	0.01\\
24.04	0.01\\
24.05	0.01\\
24.06	0.01\\
24.07	0.01\\
24.08	0.01\\
24.09	0.01\\
24.1	0.01\\
24.11	0.01\\
24.12	0.01\\
24.13	0.01\\
24.14	0.01\\
24.15	0.01\\
24.16	0.01\\
24.17	0.01\\
24.18	0.01\\
24.19	0.01\\
24.2	0.01\\
24.21	0.01\\
24.22	0.01\\
24.23	0.01\\
24.24	0.01\\
24.25	0.01\\
24.26	0.01\\
24.27	0.01\\
24.28	0.01\\
24.29	0.01\\
24.3	0.01\\
24.31	0.01\\
24.32	0.01\\
24.33	0.01\\
24.34	0.01\\
24.35	0.01\\
24.36	0.01\\
24.37	0.01\\
24.38	0.01\\
24.39	0.01\\
24.4	0.01\\
24.41	0.01\\
24.42	0.01\\
24.43	0.01\\
24.44	0.01\\
24.45	0.01\\
24.46	0.01\\
24.47	0.01\\
24.48	0.01\\
24.49	0.01\\
24.5	0.01\\
24.51	0.01\\
24.52	0.01\\
24.53	0.01\\
24.54	0.01\\
24.55	0.01\\
24.56	0.01\\
24.57	0.01\\
24.58	0.01\\
24.59	0.01\\
24.6	0.01\\
24.61	0.01\\
24.62	0.01\\
24.63	0.01\\
24.64	0.01\\
24.65	0.01\\
24.66	0.01\\
24.67	0.01\\
24.68	0.01\\
24.69	0.01\\
24.7	0.01\\
24.71	0.01\\
24.72	0.01\\
24.73	0.01\\
24.74	0.01\\
24.75	0.01\\
24.76	0.01\\
24.77	0.01\\
24.78	0.01\\
24.79	0.01\\
24.8	0.01\\
24.81	0.01\\
24.82	0.01\\
24.83	0.01\\
24.84	0.01\\
24.85	0.01\\
24.86	0.01\\
24.87	0.01\\
24.88	0.01\\
24.89	0.01\\
24.9	0.01\\
24.91	0.01\\
24.92	0.01\\
24.93	0.01\\
24.94	0.01\\
24.95	0.01\\
24.96	0.01\\
24.97	0.01\\
24.98	0.01\\
24.99	0.01\\
25	0.01\\
25.01	0.01\\
25.02	0.01\\
25.03	0.01\\
25.04	0.01\\
25.05	0.01\\
25.06	0.01\\
25.07	0.01\\
25.08	0.01\\
25.09	0.01\\
25.1	0.01\\
25.11	0.01\\
25.12	0.01\\
25.13	0.01\\
25.14	0.01\\
25.15	0.01\\
25.16	0.01\\
25.17	0.01\\
25.18	0.01\\
25.19	0.01\\
25.2	0.01\\
25.21	0.01\\
25.22	0.01\\
25.23	0.01\\
25.24	0.01\\
25.25	0.01\\
25.26	0.01\\
25.27	0.01\\
25.28	0.01\\
25.29	0.01\\
25.3	0.01\\
25.31	0.01\\
25.32	0.01\\
25.33	0.01\\
25.34	0.01\\
25.35	0.01\\
25.36	0.01\\
25.37	0.01\\
25.38	0.01\\
25.39	0.01\\
25.4	0.01\\
25.41	0.01\\
25.42	0.01\\
25.43	0.01\\
25.44	0.01\\
25.45	0.01\\
25.46	0.01\\
25.47	0.01\\
25.48	0.01\\
25.49	0.01\\
25.5	0.01\\
25.51	0.01\\
25.52	0.01\\
25.53	0.01\\
25.54	0.01\\
25.55	0.01\\
25.56	0.01\\
25.57	0.01\\
25.58	0.01\\
25.59	0.01\\
25.6	0.01\\
25.61	0.01\\
25.62	0.01\\
25.63	0.01\\
25.64	0.01\\
25.65	0.01\\
25.66	0.01\\
25.67	0.01\\
25.68	0.01\\
25.69	0.01\\
25.7	0.01\\
25.71	0.01\\
25.72	0.01\\
25.73	0.01\\
25.74	0.01\\
25.75	0.01\\
25.76	0.01\\
25.77	0.01\\
25.78	0.01\\
25.79	0.01\\
25.8	0.01\\
25.81	0.01\\
25.82	0.01\\
25.83	0.01\\
25.84	0.01\\
25.85	0.01\\
25.86	0.01\\
25.87	0.01\\
25.88	0.01\\
25.89	0.01\\
25.9	0.01\\
25.91	0.01\\
25.92	0.01\\
25.93	0.01\\
25.94	0.01\\
25.95	0.01\\
25.96	0.01\\
25.97	0.01\\
25.98	0.01\\
25.99	0.01\\
26	0.01\\
26.01	0.01\\
26.02	0.01\\
26.03	0.01\\
26.04	0.01\\
26.05	0.01\\
26.06	0.01\\
26.07	0.01\\
26.08	0.01\\
26.09	0.01\\
26.1	0.01\\
26.11	0.01\\
26.12	0.01\\
26.13	0.01\\
26.14	0.01\\
26.15	0.01\\
26.16	0.01\\
26.17	0.01\\
26.18	0.01\\
26.19	0.01\\
26.2	0.01\\
26.21	0.01\\
26.22	0.01\\
26.23	0.01\\
26.24	0.01\\
26.25	0.01\\
26.26	0.01\\
26.27	0.01\\
26.28	0.01\\
26.29	0.01\\
26.3	0.01\\
26.31	0.01\\
26.32	0.01\\
26.33	0.01\\
26.34	0.01\\
26.35	0.01\\
26.36	0.01\\
26.37	0.01\\
26.38	0.01\\
26.39	0.01\\
26.4	0.01\\
26.41	0.01\\
26.42	0.01\\
26.43	0.01\\
26.44	0.01\\
26.45	0.01\\
26.46	0.01\\
26.47	0.01\\
26.48	0.01\\
26.49	0.01\\
26.5	0.01\\
26.51	0.01\\
26.52	0.01\\
26.53	0.01\\
26.54	0.01\\
26.55	0.01\\
26.56	0.01\\
26.57	0.01\\
26.58	0.01\\
26.59	0.01\\
26.6	0.01\\
26.61	0.01\\
26.62	0.01\\
26.63	0.01\\
26.64	0.01\\
26.65	0.01\\
26.66	0.01\\
26.67	0.01\\
26.68	0.01\\
26.69	0.01\\
26.7	0.01\\
26.71	0.01\\
26.72	0.01\\
26.73	0.01\\
26.74	0.01\\
26.75	0.01\\
26.76	0.01\\
26.77	0.01\\
26.78	0.01\\
26.79	0.01\\
26.8	0.01\\
26.81	0.01\\
26.82	0.01\\
26.83	0.01\\
26.84	0.01\\
26.85	0.01\\
26.86	0.01\\
26.87	0.01\\
26.88	0.01\\
26.89	0.01\\
26.9	0.01\\
26.91	0.01\\
26.92	0.01\\
26.93	0.01\\
26.94	0.01\\
26.95	0.01\\
26.96	0.01\\
26.97	0.01\\
26.98	0.01\\
26.99	0.01\\
27	0.01\\
27.01	0.01\\
27.02	0.01\\
27.03	0.01\\
27.04	0.01\\
27.05	0.01\\
27.06	0.01\\
27.07	0.01\\
27.08	0.01\\
27.09	0.01\\
27.1	0.01\\
27.11	0.01\\
27.12	0.01\\
27.13	0.01\\
27.14	0.01\\
27.15	0.01\\
27.16	0.01\\
27.17	0.01\\
27.18	0.01\\
27.19	0.01\\
27.2	0.01\\
27.21	0.01\\
27.22	0.01\\
27.23	0.01\\
27.24	0.01\\
27.25	0.01\\
27.26	0.01\\
27.27	0.01\\
27.28	0.01\\
27.29	0.01\\
27.3	0.01\\
27.31	0.01\\
27.32	0.01\\
27.33	0.01\\
27.34	0.01\\
27.35	0.01\\
27.36	0.01\\
27.37	0.01\\
27.38	0.01\\
27.39	0.01\\
27.4	0.01\\
27.41	0.01\\
27.42	0.01\\
27.43	0.01\\
27.44	0.01\\
27.45	0.01\\
27.46	0.01\\
27.47	0.01\\
27.48	0.01\\
27.49	0.01\\
27.5	0.01\\
27.51	0.01\\
27.52	0.01\\
27.53	0.01\\
27.54	0.01\\
27.55	0.01\\
27.56	0.01\\
27.57	0.01\\
27.58	0.01\\
27.59	0.01\\
27.6	0.01\\
27.61	0.01\\
27.62	0.01\\
27.63	0.01\\
27.64	0.01\\
27.65	0.01\\
27.66	0.01\\
27.67	0.01\\
27.68	0.01\\
27.69	0.01\\
27.7	0.01\\
27.71	0.01\\
27.72	0.01\\
27.73	0.01\\
27.74	0.01\\
27.75	0.01\\
27.76	0.01\\
27.77	0.01\\
27.78	0.01\\
27.79	0.01\\
27.8	0.01\\
27.81	0.01\\
27.82	0.01\\
27.83	0.01\\
27.84	0.01\\
27.85	0.01\\
27.86	0.01\\
27.87	0.01\\
27.88	0.01\\
27.89	0.01\\
27.9	0.01\\
27.91	0.01\\
27.92	0.01\\
27.93	0.01\\
27.94	0.01\\
27.95	0.01\\
27.96	0.01\\
27.97	0.01\\
27.98	0.01\\
27.99	0.01\\
28	0.01\\
28.01	0.01\\
28.02	0.01\\
28.03	0.01\\
28.04	0.01\\
28.05	0.01\\
28.06	0.01\\
28.07	0.01\\
28.08	0.01\\
28.09	0.01\\
28.1	0.01\\
28.11	0.01\\
28.12	0.01\\
28.13	0.01\\
28.14	0.01\\
28.15	0.01\\
28.16	0.01\\
28.17	0.01\\
28.18	0.01\\
28.19	0.01\\
28.2	0.01\\
28.21	0.01\\
28.22	0.01\\
28.23	0.01\\
28.24	0.01\\
28.25	0.01\\
28.26	0.01\\
28.27	0.01\\
28.28	0.01\\
28.29	0.01\\
28.3	0.01\\
28.31	0.01\\
28.32	0.01\\
28.33	0.01\\
28.34	0.01\\
28.35	0.01\\
28.36	0.01\\
28.37	0.01\\
28.38	0.01\\
28.39	0.01\\
28.4	0.01\\
28.41	0.01\\
28.42	0.01\\
28.43	0.01\\
28.44	0.01\\
28.45	0.01\\
28.46	0.01\\
28.47	0.01\\
28.48	0.01\\
28.49	0.01\\
28.5	0.01\\
28.51	0.01\\
28.52	0.01\\
28.53	0.01\\
28.54	0.01\\
28.55	0.01\\
28.56	0.01\\
28.57	0.01\\
28.58	0.01\\
28.59	0.01\\
28.6	0.01\\
28.61	0.01\\
28.62	0.01\\
28.63	0.01\\
28.64	0.01\\
28.65	0.01\\
28.66	0.01\\
28.67	0.01\\
28.68	0.01\\
28.69	0.01\\
28.7	0.01\\
28.71	0.01\\
28.72	0.01\\
28.73	0.01\\
28.74	0.01\\
28.75	0.01\\
28.76	0.01\\
28.77	0.01\\
28.78	0.01\\
28.79	0.01\\
28.8	0.01\\
28.81	0.01\\
28.82	0.01\\
28.83	0.01\\
28.84	0.01\\
28.85	0.01\\
28.86	0.01\\
28.87	0.01\\
28.88	0.01\\
28.89	0.01\\
28.9	0.01\\
28.91	0.01\\
28.92	0.01\\
28.93	0.01\\
28.94	0.01\\
28.95	0.01\\
28.96	0.01\\
28.97	0.01\\
28.98	0.01\\
28.99	0.01\\
29	0.01\\
29.01	0.01\\
29.02	0.01\\
29.03	0.01\\
29.04	0.01\\
29.05	0.01\\
29.06	0.01\\
29.07	0.01\\
29.08	0.01\\
29.09	0.01\\
29.1	0.01\\
29.11	0.01\\
29.12	0.01\\
29.13	0.01\\
29.14	0.01\\
29.15	0.01\\
29.16	0.01\\
29.17	0.01\\
29.18	0.01\\
29.19	0.01\\
29.2	0.01\\
29.21	0.01\\
29.22	0.01\\
29.23	0.01\\
29.24	0.01\\
29.25	0.01\\
29.26	0.01\\
29.27	0.01\\
29.28	0.01\\
29.29	0.01\\
29.3	0.01\\
29.31	0.01\\
29.32	0.01\\
29.33	0.01\\
29.34	0.01\\
29.35	0.01\\
29.36	0.01\\
29.37	0.01\\
29.38	0.01\\
29.39	0.01\\
29.4	0.01\\
29.41	0.01\\
29.42	0.01\\
29.43	0.01\\
29.44	0.01\\
29.45	0.01\\
29.46	0.01\\
29.47	0.01\\
29.48	0.01\\
29.49	0.01\\
29.5	0.01\\
29.51	0.01\\
29.52	0.01\\
29.53	0.01\\
29.54	0.01\\
29.55	0.01\\
29.56	0.01\\
29.57	0.01\\
29.58	0.01\\
29.59	0.01\\
29.6	0.01\\
29.61	0.01\\
29.62	0.01\\
29.63	0.01\\
29.64	0.01\\
29.65	0.01\\
29.66	0.01\\
29.67	0.01\\
29.68	0.01\\
29.69	0.01\\
29.7	0.01\\
29.71	0.01\\
29.72	0.01\\
29.73	0.01\\
29.74	0.01\\
29.75	0.01\\
29.76	0.01\\
29.77	0.01\\
29.78	0.01\\
29.79	0.01\\
29.8	0.01\\
29.81	0.01\\
29.82	0.01\\
29.83	0.01\\
29.84	0.01\\
29.85	0.01\\
29.86	0.01\\
29.87	0.01\\
29.88	0.01\\
29.89	0.01\\
29.9	0.01\\
29.91	0.01\\
29.92	0.01\\
29.93	0.01\\
29.94	0.01\\
29.95	0.01\\
29.96	0.01\\
29.97	0.01\\
29.98	0.01\\
29.99	0.01\\
30	0.01\\
30.01	0.01\\
30.02	0.01\\
30.03	0.01\\
30.04	0.01\\
30.05	0.01\\
30.06	0.01\\
30.07	0.01\\
30.08	0.01\\
30.09	0.01\\
30.1	0.01\\
30.11	0.01\\
30.12	0.01\\
30.13	0.01\\
30.14	0.01\\
30.15	0.01\\
30.16	0.01\\
30.17	0.01\\
30.18	0.01\\
30.19	0.01\\
30.2	0.01\\
30.21	0.01\\
30.22	0.01\\
30.23	0.01\\
30.24	0.01\\
30.25	0.01\\
30.26	0.01\\
30.27	0.01\\
30.28	0.01\\
30.29	0.01\\
30.3	0.01\\
30.31	0.01\\
30.32	0.01\\
30.33	0.01\\
30.34	0.01\\
30.35	0.01\\
30.36	0.01\\
30.37	0.01\\
30.38	0.01\\
30.39	0.01\\
30.4	0.01\\
30.41	0.01\\
30.42	0.01\\
30.43	0.01\\
30.44	0.01\\
30.45	0.01\\
30.46	0.01\\
30.47	0.01\\
30.48	0.01\\
30.49	0.01\\
30.5	0.01\\
30.51	0.01\\
30.52	0.01\\
30.53	0.01\\
30.54	0.01\\
30.55	0.01\\
30.56	0.01\\
30.57	0.01\\
30.58	0.01\\
30.59	0.01\\
30.6	0.01\\
30.61	0.01\\
30.62	0.01\\
30.63	0.01\\
30.64	0.01\\
30.65	0.01\\
30.66	0.01\\
30.67	0.01\\
30.68	0.01\\
30.69	0.01\\
30.7	0.01\\
30.71	0.01\\
30.72	0.01\\
30.73	0.01\\
30.74	0.01\\
30.75	0.01\\
30.76	0.01\\
30.77	0.01\\
30.78	0.01\\
30.79	0.01\\
30.8	0.01\\
30.81	0.01\\
30.82	0.01\\
30.83	0.01\\
30.84	0.01\\
30.85	0.01\\
30.86	0.01\\
30.87	0.01\\
30.88	0.01\\
30.89	0.01\\
30.9	0.01\\
30.91	0.01\\
30.92	0.01\\
30.93	0.01\\
30.94	0.01\\
30.95	0.01\\
30.96	0.01\\
30.97	0.01\\
30.98	0.01\\
30.99	0.01\\
31	0.01\\
31.01	0.01\\
31.02	0.01\\
31.03	0.01\\
31.04	0.01\\
31.05	0.01\\
31.06	0.01\\
31.07	0.01\\
31.08	0.01\\
31.09	0.01\\
31.1	0.01\\
31.11	0.01\\
31.12	0.01\\
31.13	0.01\\
31.14	0.01\\
31.15	0.01\\
31.16	0.01\\
31.17	0.01\\
31.18	0.01\\
31.19	0.01\\
31.2	0.01\\
31.21	0.01\\
31.22	0.01\\
31.23	0.01\\
31.24	0.01\\
31.25	0.01\\
31.26	0.01\\
31.27	0.01\\
31.28	0.01\\
31.29	0.01\\
31.3	0.01\\
31.31	0.01\\
31.32	0.01\\
31.33	0.01\\
31.34	0.01\\
31.35	0.01\\
31.36	0.01\\
31.37	0.01\\
31.38	0.01\\
31.39	0.01\\
31.4	0.01\\
31.41	0.01\\
31.42	0.01\\
31.43	0.01\\
31.44	0.01\\
31.45	0.01\\
31.46	0.01\\
31.47	0.01\\
31.48	0.01\\
31.49	0.01\\
31.5	0.01\\
31.51	0.01\\
31.52	0.01\\
31.53	0.01\\
31.54	0.01\\
31.55	0.01\\
31.56	0.01\\
31.57	0.01\\
31.58	0.01\\
31.59	0.01\\
31.6	0.01\\
31.61	0.01\\
31.62	0.01\\
31.63	0.01\\
31.64	0.01\\
31.65	0.01\\
31.66	0.01\\
31.67	0.01\\
31.68	0.01\\
31.69	0.01\\
31.7	0.01\\
31.71	0.01\\
31.72	0.01\\
31.73	0.01\\
31.74	0.01\\
31.75	0.01\\
31.76	0.01\\
31.77	0.01\\
31.78	0.01\\
31.79	0.01\\
31.8	0.01\\
31.81	0.01\\
31.82	0.01\\
31.83	0.01\\
31.84	0.01\\
31.85	0.01\\
31.86	0.01\\
31.87	0.01\\
31.88	0.01\\
31.89	0.01\\
31.9	0.01\\
31.91	0.01\\
31.92	0.01\\
31.93	0.01\\
31.94	0.01\\
31.95	0.01\\
31.96	0.01\\
31.97	0.01\\
31.98	0.01\\
31.99	0.01\\
32	0.01\\
32.01	0.01\\
32.02	0.01\\
32.03	0.01\\
32.04	0.01\\
32.05	0.01\\
32.06	0.01\\
32.07	0.01\\
32.08	0.01\\
32.09	0.01\\
32.1	0.01\\
32.11	0.01\\
32.12	0.01\\
32.13	0.01\\
32.14	0.01\\
32.15	0.01\\
32.16	0.01\\
32.17	0.01\\
32.18	0.01\\
32.19	0.01\\
32.2	0.01\\
32.21	0.01\\
32.22	0.01\\
32.23	0.01\\
32.24	0.01\\
32.25	0.01\\
32.26	0.01\\
32.27	0.01\\
32.28	0.01\\
32.29	0.01\\
32.3	0.01\\
32.31	0.01\\
32.32	0.01\\
32.33	0.01\\
32.34	0.01\\
32.35	0.01\\
32.36	0.01\\
32.37	0.01\\
32.38	0.01\\
32.39	0.01\\
32.4	0.01\\
32.41	0.01\\
32.42	0.01\\
32.43	0.01\\
32.44	0.01\\
32.45	0.01\\
32.46	0.01\\
32.47	0.01\\
32.48	0.01\\
32.49	0.01\\
32.5	0.01\\
32.51	0.01\\
32.52	0.01\\
32.53	0.01\\
32.54	0.01\\
32.55	0.01\\
32.56	0.01\\
32.57	0.01\\
32.58	0.01\\
32.59	0.01\\
32.6	0.01\\
32.61	0.01\\
32.62	0.01\\
32.63	0.01\\
32.64	0.01\\
32.65	0.01\\
32.66	0.01\\
32.67	0.01\\
32.68	0.01\\
32.69	0.01\\
32.7	0.01\\
32.71	0.01\\
32.72	0.01\\
32.73	0.01\\
32.74	0.01\\
32.75	0.01\\
32.76	0.01\\
32.77	0.01\\
32.78	0.01\\
32.79	0.01\\
32.8	0.01\\
32.81	0.01\\
32.82	0.01\\
32.83	0.01\\
32.84	0.01\\
32.85	0.01\\
32.86	0.01\\
32.87	0.01\\
32.88	0.01\\
32.89	0.01\\
32.9	0.01\\
32.91	0.01\\
32.92	0.01\\
32.93	0.01\\
32.94	0.01\\
32.95	0.01\\
32.96	0.01\\
32.97	0.01\\
32.98	0.01\\
32.99	0.01\\
33	0.01\\
33.01	0.01\\
33.02	0.01\\
33.03	0.01\\
33.04	0.01\\
33.05	0.01\\
33.06	0.01\\
33.07	0.01\\
33.08	0.01\\
33.09	0.01\\
33.1	0.01\\
33.11	0.01\\
33.12	0.01\\
33.13	0.01\\
33.14	0.01\\
33.15	0.01\\
33.16	0.01\\
33.17	0.01\\
33.18	0.01\\
33.19	0.01\\
33.2	0.01\\
33.21	0.01\\
33.22	0.01\\
33.23	0.01\\
33.24	0.01\\
33.25	0.01\\
33.26	0.01\\
33.27	0.01\\
33.28	0.01\\
33.29	0.01\\
33.3	0.01\\
33.31	0.01\\
33.32	0.01\\
33.33	0.01\\
33.34	0.01\\
33.35	0.01\\
33.36	0.01\\
33.37	0.01\\
33.38	0.01\\
33.39	0.01\\
33.4	0.01\\
33.41	0.01\\
33.42	0.01\\
33.43	0.01\\
33.44	0.01\\
33.45	0.01\\
33.46	0.01\\
33.47	0.01\\
33.48	0.01\\
33.49	0.01\\
33.5	0.01\\
33.51	0.01\\
33.52	0.01\\
33.53	0.01\\
33.54	0.01\\
33.55	0.01\\
33.56	0.01\\
33.57	0.01\\
33.58	0.01\\
33.59	0.01\\
33.6	0.01\\
33.61	0.01\\
33.62	0.01\\
33.63	0.01\\
33.64	0.01\\
33.65	0.01\\
33.66	0.01\\
33.67	0.01\\
33.68	0.01\\
33.69	0.01\\
33.7	0.01\\
33.71	0.01\\
33.72	0.01\\
33.73	0.01\\
33.74	0.01\\
33.75	0.01\\
33.76	0.01\\
33.77	0.01\\
33.78	0.01\\
33.79	0.01\\
33.8	0.01\\
33.81	0.01\\
33.82	0.01\\
33.83	0.01\\
33.84	0.01\\
33.85	0.01\\
33.86	0.01\\
33.87	0.01\\
33.88	0.01\\
33.89	0.01\\
33.9	0.01\\
33.91	0.01\\
33.92	0.01\\
33.93	0.01\\
33.94	0.01\\
33.95	0.01\\
33.96	0.01\\
33.97	0.01\\
33.98	0.01\\
33.99	0.01\\
34	0.01\\
34.01	0.01\\
34.02	0.01\\
34.03	0.01\\
34.04	0.01\\
34.05	0.01\\
34.06	0.01\\
34.07	0.01\\
34.08	0.01\\
34.09	0.01\\
34.1	0.01\\
34.11	0.01\\
34.12	0.01\\
34.13	0.01\\
34.14	0.01\\
34.15	0.01\\
34.16	0.01\\
34.17	0.01\\
34.18	0.01\\
34.19	0.01\\
34.2	0.01\\
34.21	0.01\\
34.22	0.01\\
34.23	0.01\\
34.24	0.01\\
34.25	0.01\\
34.26	0.01\\
34.27	0.01\\
34.28	0.01\\
34.29	0.01\\
34.3	0.01\\
34.31	0.01\\
34.32	0.01\\
34.33	0.01\\
34.34	0.01\\
34.35	0.01\\
34.36	0.01\\
34.37	0.01\\
34.38	0.01\\
34.39	0.01\\
34.4	0.01\\
34.41	0.01\\
34.42	0.01\\
34.43	0.01\\
34.44	0.01\\
34.45	0.01\\
34.46	0.01\\
34.47	0.01\\
34.48	0.01\\
34.49	0.01\\
34.5	0.01\\
34.51	0.01\\
34.52	0.01\\
34.53	0.01\\
34.54	0.01\\
34.55	0.01\\
34.56	0.01\\
34.57	0.01\\
34.58	0.01\\
34.59	0.01\\
34.6	0.01\\
34.61	0.01\\
34.62	0.01\\
34.63	0.01\\
34.64	0.01\\
34.65	0.01\\
34.66	0.01\\
34.67	0.01\\
34.68	0.01\\
34.69	0.01\\
34.7	0.01\\
34.71	0.01\\
34.72	0.01\\
34.73	0.01\\
34.74	0.01\\
34.75	0.01\\
34.76	0.01\\
34.77	0.01\\
34.78	0.01\\
34.79	0.01\\
34.8	0.01\\
34.81	0.01\\
34.82	0.01\\
34.83	0.01\\
34.84	0.01\\
34.85	0.01\\
34.86	0.01\\
34.87	0.01\\
34.88	0.01\\
34.89	0.01\\
34.9	0.01\\
34.91	0.01\\
34.92	0.01\\
34.93	0.01\\
34.94	0.01\\
34.95	0.01\\
34.96	0.01\\
34.97	0.01\\
34.98	0.01\\
34.99	0.01\\
35	0.01\\
35.01	0.01\\
35.02	0.01\\
35.03	0.01\\
35.04	0.01\\
35.05	0.01\\
35.06	0.01\\
35.07	0.01\\
35.08	0.01\\
35.09	0.01\\
35.1	0.01\\
35.11	0.01\\
35.12	0.01\\
35.13	0.01\\
35.14	0.01\\
35.15	0.01\\
35.16	0.01\\
35.17	0.01\\
35.18	0.01\\
35.19	0.01\\
35.2	0.01\\
35.21	0.01\\
35.22	0.01\\
35.23	0.01\\
35.24	0.01\\
35.25	0.01\\
35.26	0.01\\
35.27	0.01\\
35.28	0.01\\
35.29	0.01\\
35.3	0.01\\
35.31	0.01\\
35.32	0.01\\
35.33	0.01\\
35.34	0.01\\
35.35	0.01\\
35.36	0.01\\
35.37	0.01\\
35.38	0.01\\
35.39	0.01\\
35.4	0.01\\
35.41	0.01\\
35.42	0.01\\
35.43	0.01\\
35.44	0.01\\
35.45	0.01\\
35.46	0.01\\
35.47	0.01\\
35.48	0.01\\
35.49	0.01\\
35.5	0.01\\
35.51	0.01\\
35.52	0.01\\
35.53	0.01\\
35.54	0.01\\
35.55	0.01\\
35.56	0.01\\
35.57	0.01\\
35.58	0.01\\
35.59	0.01\\
35.6	0.01\\
35.61	0.01\\
35.62	0.01\\
35.63	0.01\\
35.64	0.01\\
35.65	0.01\\
35.66	0.01\\
35.67	0.01\\
35.68	0.01\\
35.69	0.01\\
35.7	0.01\\
35.71	0.01\\
35.72	0.01\\
35.73	0.01\\
35.74	0.01\\
35.75	0.01\\
35.76	0.01\\
35.77	0.01\\
35.78	0.01\\
35.79	0.01\\
35.8	0.01\\
35.81	0.01\\
35.82	0.01\\
35.83	0.01\\
35.84	0.01\\
35.85	0.01\\
35.86	0.01\\
35.87	0.01\\
35.88	0.01\\
35.89	0.01\\
35.9	0.01\\
35.91	0.01\\
35.92	0.01\\
35.93	0.01\\
35.94	0.01\\
35.95	0.01\\
35.96	0.01\\
35.97	0.01\\
35.98	0.01\\
35.99	0.01\\
36	0.01\\
36.01	0.01\\
36.02	0.01\\
36.03	0.01\\
36.04	0.01\\
36.05	0.01\\
36.06	0.01\\
36.07	0.01\\
36.08	0.01\\
36.09	0.01\\
36.1	0.01\\
36.11	0.01\\
36.12	0.01\\
36.13	0.01\\
36.14	0.01\\
36.15	0.01\\
36.16	0.01\\
36.17	0.01\\
36.18	0.01\\
36.19	0.01\\
36.2	0.01\\
36.21	0.01\\
36.22	0.01\\
36.23	0.01\\
36.24	0.01\\
36.25	0.01\\
36.26	0.01\\
36.27	0.01\\
36.28	0.01\\
36.29	0.01\\
36.3	0.01\\
36.31	0.01\\
36.32	0.01\\
36.33	0.01\\
36.34	0.01\\
36.35	0.01\\
36.36	0.01\\
36.37	0.01\\
36.38	0.01\\
36.39	0.01\\
36.4	0.01\\
36.41	0.01\\
36.42	0.01\\
36.43	0.01\\
36.44	0.01\\
36.45	0.01\\
36.46	0.01\\
36.47	0.01\\
36.48	0.01\\
36.49	0.01\\
36.5	0.01\\
36.51	0.01\\
36.52	0.01\\
36.53	0.01\\
36.54	0.01\\
36.55	0.01\\
36.56	0.01\\
36.57	0.01\\
36.58	0.01\\
36.59	0.01\\
36.6	0.01\\
36.61	0.01\\
36.62	0.01\\
36.63	0.01\\
36.64	0.01\\
36.65	0.01\\
36.66	0.01\\
36.67	0.01\\
36.68	0.01\\
36.69	0.01\\
36.7	0.01\\
36.71	0.01\\
36.72	0.01\\
36.73	0.01\\
36.74	0.01\\
36.75	0.01\\
36.76	0.01\\
36.77	0.01\\
36.78	0.01\\
36.79	0.01\\
36.8	0.01\\
36.81	0.01\\
36.82	0.01\\
36.83	0.01\\
36.84	0.01\\
36.85	0.01\\
36.86	0.01\\
36.87	0.01\\
36.88	0.01\\
36.89	0.01\\
36.9	0.01\\
36.91	0.01\\
36.92	0.01\\
36.93	0.01\\
36.94	0.01\\
36.95	0.01\\
36.96	0.01\\
36.97	0.01\\
36.98	0.01\\
36.99	0.01\\
37	0.01\\
37.01	0.01\\
37.02	0.01\\
37.03	0.01\\
37.04	0.01\\
37.05	0.01\\
37.06	0.01\\
37.07	0.01\\
37.08	0.01\\
37.09	0.01\\
37.1	0.01\\
37.11	0.01\\
37.12	0.01\\
37.13	0.01\\
37.14	0.01\\
37.15	0.01\\
37.16	0.01\\
37.17	0.01\\
37.18	0.01\\
37.19	0.01\\
37.2	0.01\\
37.21	0.01\\
37.22	0.01\\
37.23	0.01\\
37.24	0.01\\
37.25	0.01\\
37.26	0.01\\
37.27	0.01\\
37.28	0.01\\
37.29	0.01\\
37.3	0.01\\
37.31	0.01\\
37.32	0.01\\
37.33	0.01\\
37.34	0.01\\
37.35	0.01\\
37.36	0.01\\
37.37	0.01\\
37.38	0.01\\
37.39	0.01\\
37.4	0.01\\
37.41	0.01\\
37.42	0.01\\
37.43	0.01\\
37.44	0.01\\
37.45	0.01\\
37.46	0.01\\
37.47	0.01\\
37.48	0.01\\
37.49	0.01\\
37.5	0.01\\
37.51	0.01\\
37.52	0.01\\
37.53	0.01\\
37.54	0.01\\
37.55	0.01\\
37.56	0.01\\
37.57	0.01\\
37.58	0.01\\
37.59	0.01\\
37.6	0.01\\
37.61	0.01\\
37.62	0.01\\
37.63	0.01\\
37.64	0.01\\
37.65	0.01\\
37.66	0.01\\
37.67	0.01\\
37.68	0.01\\
37.69	0.01\\
37.7	0.01\\
37.71	0.01\\
37.72	0.01\\
37.73	0.01\\
37.74	0.01\\
37.75	0.01\\
37.76	0.01\\
37.77	0.01\\
37.78	0.01\\
37.79	0.01\\
37.8	0.01\\
37.81	0.01\\
37.82	0.01\\
37.83	0.01\\
37.84	0.01\\
37.85	0.01\\
37.86	0.01\\
37.87	0.01\\
37.88	0.01\\
37.89	0.01\\
37.9	0.01\\
37.91	0.01\\
37.92	0.01\\
37.93	0.01\\
37.94	0.01\\
37.95	0.01\\
37.96	0.01\\
37.97	0.01\\
37.98	0.01\\
37.99	0.01\\
38	0.01\\
38.01	0.01\\
38.02	0.01\\
38.03	0.01\\
38.04	0.01\\
38.05	0.01\\
38.06	0.01\\
38.07	0.01\\
38.08	0.01\\
38.09	0.01\\
38.1	0.01\\
38.11	0.01\\
38.12	0.01\\
38.13	0.01\\
38.14	0.01\\
38.15	0.01\\
38.16	0.01\\
38.17	0.01\\
38.18	0.01\\
38.19	0.01\\
38.2	0.01\\
38.21	0.01\\
38.22	0.01\\
38.23	0.01\\
38.24	0.01\\
38.25	0.01\\
38.26	0.01\\
38.27	0.01\\
38.28	0.01\\
38.29	0.01\\
38.3	0.01\\
38.31	0.01\\
38.32	0.01\\
38.33	0.01\\
38.34	0.01\\
38.35	0.01\\
38.36	0.01\\
38.37	0.01\\
38.38	0.01\\
38.39	0.01\\
38.4	0.01\\
38.41	0.01\\
38.42	0.01\\
38.43	0.01\\
38.44	0.01\\
38.45	0.01\\
38.46	0.01\\
38.47	0.01\\
38.48	0.01\\
38.49	0.01\\
38.5	0.01\\
38.51	0.01\\
38.52	0.01\\
38.53	0.01\\
38.54	0.01\\
38.55	0.01\\
38.56	0.01\\
38.57	0.01\\
38.58	0.01\\
38.59	0.01\\
38.6	0.01\\
38.61	0.01\\
38.62	0.01\\
38.63	0.01\\
38.64	0.01\\
38.65	0.01\\
38.66	0.01\\
38.67	0.01\\
38.68	0.01\\
38.69	0.01\\
38.7	0.01\\
38.71	0.01\\
38.72	0.01\\
38.73	0.01\\
38.74	0.01\\
38.75	0.01\\
38.76	0.01\\
38.77	0.01\\
38.78	0.01\\
38.79	0.01\\
38.8	0.01\\
38.81	0.01\\
38.82	0.01\\
38.83	0.01\\
38.84	0.01\\
38.85	0.01\\
38.86	0.01\\
38.87	0.01\\
38.88	0.01\\
38.89	0.01\\
38.9	0.01\\
38.91	0.01\\
38.92	0.01\\
38.93	0.01\\
38.94	0.01\\
38.95	0.01\\
38.96	0.01\\
38.97	0.01\\
38.98	0.01\\
38.99	0.01\\
39	0.01\\
39.01	0.01\\
39.02	0.01\\
39.03	0.01\\
39.04	0.01\\
39.05	0.01\\
39.06	0.01\\
39.07	0.01\\
39.08	0.01\\
39.09	0.01\\
39.1	0.01\\
39.11	0.01\\
39.12	0.01\\
39.13	0.01\\
39.14	0.01\\
39.15	0.01\\
39.16	0.01\\
39.17	0.01\\
39.18	0.01\\
39.19	0.01\\
39.2	0.01\\
39.21	0.01\\
39.22	0.01\\
39.23	0.01\\
39.24	0.01\\
39.25	0.01\\
39.26	0.01\\
39.27	0.01\\
39.28	0.01\\
39.29	0.01\\
39.3	0.01\\
39.31	0.01\\
39.32	0.01\\
39.33	0.01\\
39.34	0.01\\
39.35	0.01\\
39.36	0.01\\
39.37	0.01\\
39.38	0.01\\
39.39	0.01\\
39.4	0.01\\
39.41	0.01\\
39.42	0.01\\
39.43	0.01\\
39.44	0.01\\
39.45	0.01\\
39.46	0.01\\
39.47	0.01\\
39.48	0.01\\
39.49	0.01\\
39.5	0.01\\
39.51	0.01\\
39.52	0.01\\
39.53	0.01\\
39.54	0.01\\
39.55	0.01\\
39.56	0.01\\
39.57	0.01\\
39.58	0.01\\
39.59	0.01\\
39.6	0.01\\
39.61	0.01\\
39.62	0.01\\
39.63	0.01\\
39.64	0.01\\
39.65	0.01\\
39.66	0.01\\
39.67	0.01\\
39.68	0.01\\
39.69	0.01\\
39.7	0.01\\
39.71	0.01\\
39.72	0.01\\
39.73	0.01\\
39.74	0.01\\
39.75	0.01\\
39.76	0.01\\
39.77	0.01\\
39.78	0.01\\
39.79	0.01\\
39.8	0.01\\
39.81	0.01\\
39.82	0.01\\
39.83	0.01\\
39.84	0.01\\
39.85	0.01\\
39.86	0.01\\
39.87	0.01\\
39.88	0.01\\
39.89	0.01\\
39.9	0.01\\
39.91	0.01\\
39.92	0.01\\
39.93	0.01\\
39.94	0.01\\
39.95	0.01\\
39.96	0.01\\
39.97	0.01\\
39.98	0.01\\
39.99	0.01\\
40	0.01\\
40.01	0.01\\
};
\addplot [color=mycolor1,dashed,forget plot]
  table[row sep=crcr]{%
40.01	0.01\\
40.02	0.01\\
40.03	0.01\\
40.04	0.01\\
40.05	0.01\\
40.06	0.01\\
40.07	0.01\\
40.08	0.01\\
40.09	0.01\\
40.1	0.01\\
40.11	0.01\\
40.12	0.01\\
40.13	0.01\\
40.14	0.01\\
40.15	0.01\\
40.16	0.01\\
40.17	0.01\\
40.18	0.01\\
40.19	0.01\\
40.2	0.01\\
40.21	0.01\\
40.22	0.01\\
40.23	0.01\\
40.24	0.01\\
40.25	0.01\\
40.26	0.01\\
40.27	0.01\\
40.28	0.01\\
40.29	0.01\\
40.3	0.01\\
40.31	0.01\\
40.32	0.01\\
40.33	0.01\\
40.34	0.01\\
40.35	0.01\\
40.36	0.01\\
40.37	0.01\\
40.38	0.01\\
40.39	0.01\\
40.4	0.01\\
40.41	0.01\\
40.42	0.01\\
40.43	0.01\\
40.44	0.01\\
40.45	0.01\\
40.46	0.01\\
40.47	0.01\\
40.48	0.01\\
40.49	0.01\\
40.5	0.01\\
40.51	0.01\\
40.52	0.01\\
40.53	0.01\\
40.54	0.01\\
40.55	0.01\\
40.56	0.01\\
40.57	0.01\\
40.58	0.01\\
40.59	0.01\\
40.6	0.01\\
40.61	0.01\\
40.62	0.01\\
40.63	0.01\\
40.64	0.01\\
40.65	0.01\\
40.66	0.01\\
40.67	0.01\\
40.68	0.01\\
40.69	0.01\\
40.7	0.01\\
40.71	0.01\\
40.72	0.01\\
40.73	0.01\\
40.74	0.01\\
40.75	0.01\\
40.76	0.01\\
40.77	0.01\\
40.78	0.01\\
40.79	0.01\\
40.8	0.01\\
40.81	0.01\\
40.82	0.01\\
40.83	0.01\\
40.84	0.01\\
40.85	0.01\\
40.86	0.01\\
40.87	0.01\\
40.88	0.01\\
40.89	0.01\\
40.9	0.01\\
40.91	0.01\\
40.92	0.01\\
40.93	0.01\\
40.94	0.01\\
40.95	0.01\\
40.96	0.01\\
40.97	0.01\\
40.98	0.01\\
40.99	0.01\\
41	0.01\\
41.01	0.01\\
41.02	0.01\\
41.03	0.01\\
41.04	0.01\\
41.05	0.01\\
41.06	0.01\\
41.07	0.01\\
41.08	0.01\\
41.09	0.01\\
41.1	0.01\\
41.11	0.01\\
41.12	0.01\\
41.13	0.01\\
41.14	0.01\\
41.15	0.01\\
41.16	0.01\\
41.17	0.01\\
41.18	0.01\\
41.19	0.01\\
41.2	0.01\\
41.21	0.01\\
41.22	0.01\\
41.23	0.01\\
41.24	0.01\\
41.25	0.01\\
41.26	0.01\\
41.27	0.01\\
41.28	0.01\\
41.29	0.01\\
41.3	0.01\\
41.31	0.01\\
41.32	0.01\\
41.33	0.01\\
41.34	0.01\\
41.35	0.01\\
41.36	0.01\\
41.37	0.01\\
41.38	0.01\\
41.39	0.01\\
41.4	0.01\\
41.41	0.01\\
41.42	0.01\\
41.43	0.01\\
41.44	0.01\\
41.45	0.01\\
41.46	0.01\\
41.47	0.01\\
41.48	0.01\\
41.49	0.01\\
41.5	0.01\\
41.51	0.01\\
41.52	0.01\\
41.53	0.01\\
41.54	0.01\\
41.55	0.01\\
41.56	0.01\\
41.57	0.01\\
41.58	0.01\\
41.59	0.01\\
41.6	0.01\\
41.61	0.01\\
41.62	0.01\\
41.63	0.01\\
41.64	0.01\\
41.65	0.01\\
41.66	0.01\\
41.67	0.01\\
41.68	0.01\\
41.69	0.01\\
41.7	0.01\\
41.71	0.01\\
41.72	0.01\\
41.73	0.01\\
41.74	0.01\\
41.75	0.01\\
41.76	0.01\\
41.77	0.01\\
41.78	0.01\\
41.79	0.01\\
41.8	0.01\\
41.81	0.01\\
41.82	0.01\\
41.83	0.01\\
41.84	0.01\\
41.85	0.01\\
41.86	0.01\\
41.87	0.01\\
41.88	0.01\\
41.89	0.01\\
41.9	0.01\\
41.91	0.01\\
41.92	0.01\\
41.93	0.01\\
41.94	0.01\\
41.95	0.01\\
41.96	0.01\\
41.97	0.01\\
41.98	0.01\\
41.99	0.01\\
42	0.01\\
42.01	0.01\\
42.02	0.01\\
42.03	0.01\\
42.04	0.01\\
42.05	0.01\\
42.06	0.01\\
42.07	0.01\\
42.08	0.01\\
42.09	0.01\\
42.1	0.01\\
42.11	0.01\\
42.12	0.01\\
42.13	0.01\\
42.14	0.01\\
42.15	0.01\\
42.16	0.01\\
42.17	0.01\\
42.18	0.01\\
42.19	0.01\\
42.2	0.01\\
42.21	0.01\\
42.22	0.01\\
42.23	0.01\\
42.24	0.01\\
42.25	0.01\\
42.26	0.01\\
42.27	0.01\\
42.28	0.01\\
42.29	0.01\\
42.3	0.01\\
42.31	0.01\\
42.32	0.01\\
42.33	0.01\\
42.34	0.01\\
42.35	0.01\\
42.36	0.01\\
42.37	0.01\\
42.38	0.01\\
42.39	0.01\\
42.4	0.01\\
42.41	0.01\\
42.42	0.01\\
42.43	0.01\\
42.44	0.01\\
42.45	0.01\\
42.46	0.01\\
42.47	0.01\\
42.48	0.01\\
42.49	0.01\\
42.5	0.01\\
42.51	0.01\\
42.52	0.01\\
42.53	0.01\\
42.54	0.01\\
42.55	0.01\\
42.56	0.01\\
42.57	0.01\\
42.58	0.01\\
42.59	0.01\\
42.6	0.01\\
42.61	0.01\\
42.62	0.01\\
42.63	0.01\\
42.64	0.01\\
42.65	0.01\\
42.66	0.01\\
42.67	0.01\\
42.68	0.01\\
42.69	0.01\\
42.7	0.01\\
42.71	0.01\\
42.72	0.01\\
42.73	0.01\\
42.74	0.01\\
42.75	0.01\\
42.76	0.01\\
42.77	0.01\\
42.78	0.01\\
42.79	0.01\\
42.8	0.01\\
42.81	0.01\\
42.82	0.01\\
42.83	0.01\\
42.84	0.01\\
42.85	0.01\\
42.86	0.01\\
42.87	0.01\\
42.88	0.01\\
42.89	0.01\\
42.9	0.01\\
42.91	0.01\\
42.92	0.01\\
42.93	0.01\\
42.94	0.01\\
42.95	0.01\\
42.96	0.01\\
42.97	0.01\\
42.98	0.01\\
42.99	0.01\\
43	0.01\\
43.01	0.01\\
43.02	0.01\\
43.03	0.01\\
43.04	0.01\\
43.05	0.01\\
43.06	0.01\\
43.07	0.01\\
43.08	0.01\\
43.09	0.01\\
43.1	0.01\\
43.11	0.01\\
43.12	0.01\\
43.13	0.01\\
43.14	0.01\\
43.15	0.01\\
43.16	0.01\\
43.17	0.01\\
43.18	0.01\\
43.19	0.01\\
43.2	0.01\\
43.21	0.01\\
43.22	0.01\\
43.23	0.01\\
43.24	0.01\\
43.25	0.01\\
43.26	0.01\\
43.27	0.01\\
43.28	0.01\\
43.29	0.01\\
43.3	0.01\\
43.31	0.01\\
43.32	0.01\\
43.33	0.01\\
43.34	0.01\\
43.35	0.01\\
43.36	0.01\\
43.37	0.01\\
43.38	0.01\\
43.39	0.01\\
43.4	0.01\\
43.41	0.01\\
43.42	0.01\\
43.43	0.01\\
43.44	0.01\\
43.45	0.01\\
43.46	0.01\\
43.47	0.01\\
43.48	0.01\\
43.49	0.01\\
43.5	0.01\\
43.51	0.01\\
43.52	0.01\\
43.53	0.01\\
43.54	0.01\\
43.55	0.01\\
43.56	0.01\\
43.57	0.01\\
43.58	0.01\\
43.59	0.01\\
43.6	0.01\\
43.61	0.01\\
43.62	0.01\\
43.63	0.01\\
43.64	0.01\\
43.65	0.01\\
43.66	0.01\\
43.67	0.01\\
43.68	0.01\\
43.69	0.01\\
43.7	0.01\\
43.71	0.01\\
43.72	0.01\\
43.73	0.01\\
43.74	0.01\\
43.75	0.01\\
43.76	0.01\\
43.77	0.01\\
43.78	0.01\\
43.79	0.01\\
43.8	0.01\\
43.81	0.01\\
43.82	0.01\\
43.83	0.01\\
43.84	0.01\\
43.85	0.01\\
43.86	0.01\\
43.87	0.01\\
43.88	0.01\\
43.89	0.01\\
43.9	0.01\\
43.91	0.01\\
43.92	0.01\\
43.93	0.01\\
43.94	0.01\\
43.95	0.01\\
43.96	0.01\\
43.97	0.01\\
43.98	0.01\\
43.99	0.01\\
44	0.01\\
44.01	0.01\\
44.02	0.01\\
44.03	0.01\\
44.04	0.01\\
44.05	0.01\\
44.06	0.01\\
44.07	0.01\\
44.08	0.01\\
44.09	0.01\\
44.1	0.01\\
44.11	0.01\\
44.12	0.01\\
44.13	0.01\\
44.14	0.01\\
44.15	0.01\\
44.16	0.01\\
44.17	0.01\\
44.18	0.01\\
44.19	0.01\\
44.2	0.01\\
44.21	0.01\\
44.22	0.01\\
44.23	0.01\\
44.24	0.01\\
44.25	0.01\\
44.26	0.01\\
44.27	0.01\\
44.28	0.01\\
44.29	0.01\\
44.3	0.01\\
44.31	0.01\\
44.32	0.01\\
44.33	0.01\\
44.34	0.01\\
44.35	0.01\\
44.36	0.01\\
44.37	0.01\\
44.38	0.01\\
44.39	0.01\\
44.4	0.01\\
44.41	0.01\\
44.42	0.01\\
44.43	0.01\\
44.44	0.01\\
44.45	0.01\\
44.46	0.01\\
44.47	0.01\\
44.48	0.01\\
44.49	0.01\\
44.5	0.01\\
44.51	0.01\\
44.52	0.01\\
44.53	0.01\\
44.54	0.01\\
44.55	0.01\\
44.56	0.01\\
44.57	0.01\\
44.58	0.01\\
44.59	0.01\\
44.6	0.01\\
44.61	0.01\\
44.62	0.01\\
44.63	0.01\\
44.64	0.01\\
44.65	0.01\\
44.66	0.01\\
44.67	0.01\\
44.68	0.01\\
44.69	0.01\\
44.7	0.01\\
44.71	0.01\\
44.72	0.01\\
44.73	0.01\\
44.74	0.01\\
44.75	0.01\\
44.76	0.01\\
44.77	0.01\\
44.78	0.01\\
44.79	0.01\\
44.8	0.01\\
44.81	0.01\\
44.82	0.01\\
44.83	0.01\\
44.84	0.01\\
44.85	0.01\\
44.86	0.01\\
44.87	0.01\\
44.88	0.01\\
44.89	0.01\\
44.9	0.01\\
44.91	0.01\\
44.92	0.01\\
44.93	0.01\\
44.94	0.01\\
44.95	0.01\\
44.96	0.01\\
44.97	0.01\\
44.98	0.01\\
44.99	0.01\\
45	0.01\\
45.01	0.01\\
45.02	0.01\\
45.03	0.01\\
45.04	0.01\\
45.05	0.01\\
45.06	0.01\\
45.07	0.01\\
45.08	0.01\\
45.09	0.01\\
45.1	0.01\\
45.11	0.01\\
45.12	0.01\\
45.13	0.01\\
45.14	0.01\\
45.15	0.01\\
45.16	0.01\\
45.17	0.01\\
45.18	0.01\\
45.19	0.01\\
45.2	0.01\\
45.21	0.01\\
45.22	0.01\\
45.23	0.01\\
45.24	0.01\\
45.25	0.01\\
45.26	0.01\\
45.27	0.01\\
45.28	0.01\\
45.29	0.01\\
45.3	0.01\\
45.31	0.01\\
45.32	0.01\\
45.33	0.01\\
45.34	0.01\\
45.35	0.01\\
45.36	0.01\\
45.37	0.01\\
45.38	0.01\\
45.39	0.01\\
45.4	0.01\\
45.41	0.01\\
45.42	0.01\\
45.43	0.01\\
45.44	0.01\\
45.45	0.01\\
45.46	0.01\\
45.47	0.01\\
45.48	0.01\\
45.49	0.01\\
45.5	0.01\\
45.51	0.01\\
45.52	0.01\\
45.53	0.01\\
45.54	0.01\\
45.55	0.01\\
45.56	0.01\\
45.57	0.01\\
45.58	0.01\\
45.59	0.01\\
45.6	0.01\\
45.61	0.01\\
45.62	0.01\\
45.63	0.01\\
45.64	0.01\\
45.65	0.01\\
45.66	0.01\\
45.67	0.01\\
45.68	0.01\\
45.69	0.01\\
45.7	0.01\\
45.71	0.01\\
45.72	0.01\\
45.73	0.01\\
45.74	0.01\\
45.75	0.01\\
45.76	0.01\\
45.77	0.01\\
45.78	0.01\\
45.79	0.01\\
45.8	0.01\\
45.81	0.01\\
45.82	0.01\\
45.83	0.01\\
45.84	0.01\\
45.85	0.01\\
45.86	0.01\\
45.87	0.01\\
45.88	0.01\\
45.89	0.01\\
45.9	0.01\\
45.91	0.01\\
45.92	0.01\\
45.93	0.01\\
45.94	0.01\\
45.95	0.01\\
45.96	0.01\\
45.97	0.01\\
45.98	0.01\\
45.99	0.01\\
46	0.01\\
46.01	0.01\\
46.02	0.01\\
46.03	0.01\\
46.04	0.01\\
46.05	0.01\\
46.06	0.01\\
46.07	0.01\\
46.08	0.01\\
46.09	0.01\\
46.1	0.01\\
46.11	0.01\\
46.12	0.01\\
46.13	0.01\\
46.14	0.01\\
46.15	0.01\\
46.16	0.01\\
46.17	0.01\\
46.18	0.01\\
46.19	0.01\\
46.2	0.01\\
46.21	0.01\\
46.22	0.01\\
46.23	0.01\\
46.24	0.01\\
46.25	0.01\\
46.26	0.01\\
46.27	0.01\\
46.28	0.01\\
46.29	0.01\\
46.3	0.01\\
46.31	0.01\\
46.32	0.01\\
46.33	0.01\\
46.34	0.01\\
46.35	0.01\\
46.36	0.01\\
46.37	0.01\\
46.38	0.01\\
46.39	0.01\\
46.4	0.01\\
46.41	0.01\\
46.42	0.01\\
46.43	0.01\\
46.44	0.01\\
46.45	0.01\\
46.46	0.01\\
46.47	0.01\\
46.48	0.01\\
46.49	0.01\\
46.5	0.01\\
46.51	0.01\\
46.52	0.01\\
46.53	0.01\\
46.54	0.01\\
46.55	0.01\\
46.56	0.01\\
46.57	0.01\\
46.58	0.01\\
46.59	0.01\\
46.6	0.01\\
46.61	0.01\\
46.62	0.01\\
46.63	0.01\\
46.64	0.01\\
46.65	0.01\\
46.66	0.01\\
46.67	0.01\\
46.68	0.01\\
46.69	0.01\\
46.7	0.01\\
46.71	0.01\\
46.72	0.01\\
46.73	0.01\\
46.74	0.01\\
46.75	0.01\\
46.76	0.01\\
46.77	0.01\\
46.78	0.01\\
46.79	0.01\\
46.8	0.01\\
46.81	0.01\\
46.82	0.01\\
46.83	0.01\\
46.84	0.01\\
46.85	0.01\\
46.86	0.01\\
46.87	0.01\\
46.88	0.01\\
46.89	0.01\\
46.9	0.01\\
46.91	0.01\\
46.92	0.01\\
46.93	0.01\\
46.94	0.01\\
46.95	0.01\\
46.96	0.01\\
46.97	0.01\\
46.98	0.01\\
46.99	0.01\\
47	0.01\\
47.01	0.01\\
47.02	0.01\\
47.03	0.01\\
47.04	0.01\\
47.05	0.01\\
47.06	0.01\\
47.07	0.01\\
47.08	0.01\\
47.09	0.01\\
47.1	0.01\\
47.11	0.01\\
47.12	0.01\\
47.13	0.01\\
47.14	0.01\\
47.15	0.01\\
47.16	0.01\\
47.17	0.01\\
47.18	0.01\\
47.19	0.01\\
47.2	0.01\\
47.21	0.01\\
47.22	0.01\\
47.23	0.01\\
47.24	0.01\\
47.25	0.01\\
47.26	0.01\\
47.27	0.01\\
47.28	0.01\\
47.29	0.01\\
47.3	0.01\\
47.31	0.01\\
47.32	0.01\\
47.33	0.01\\
47.34	0.01\\
47.35	0.01\\
47.36	0.01\\
47.37	0.01\\
47.38	0.01\\
47.39	0.01\\
47.4	0.01\\
47.41	0.01\\
47.42	0.01\\
47.43	0.01\\
47.44	0.01\\
47.45	0.01\\
47.46	0.01\\
47.47	0.01\\
47.48	0.01\\
47.49	0.01\\
47.5	0.01\\
47.51	0.01\\
47.52	0.01\\
47.53	0.01\\
47.54	0.01\\
47.55	0.01\\
47.56	0.01\\
47.57	0.01\\
47.58	0.01\\
47.59	0.01\\
47.6	0.01\\
47.61	0.01\\
47.62	0.01\\
47.63	0.01\\
47.64	0.01\\
47.65	0.01\\
47.66	0.01\\
47.67	0.01\\
47.68	0.01\\
47.69	0.01\\
47.7	0.01\\
47.71	0.01\\
47.72	0.01\\
47.73	0.01\\
47.74	0.01\\
47.75	0.01\\
47.76	0.01\\
47.77	0.01\\
47.78	0.01\\
47.79	0.01\\
47.8	0.01\\
47.81	0.01\\
47.82	0.01\\
47.83	0.01\\
47.84	0.01\\
47.85	0.01\\
47.86	0.01\\
47.87	0.01\\
47.88	0.01\\
47.89	0.01\\
47.9	0.01\\
47.91	0.01\\
47.92	0.01\\
47.93	0.01\\
47.94	0.01\\
47.95	0.01\\
47.96	0.01\\
47.97	0.01\\
47.98	0.01\\
47.99	0.01\\
48	0.01\\
48.01	0.01\\
48.02	0.01\\
48.03	0.01\\
48.04	0.01\\
48.05	0.01\\
48.06	0.01\\
48.07	0.01\\
48.08	0.01\\
48.09	0.01\\
48.1	0.01\\
48.11	0.01\\
48.12	0.01\\
48.13	0.01\\
48.14	0.01\\
48.15	0.01\\
48.16	0.01\\
48.17	0.01\\
48.18	0.01\\
48.19	0.01\\
48.2	0.01\\
48.21	0.01\\
48.22	0.01\\
48.23	0.01\\
48.24	0.01\\
48.25	0.01\\
48.26	0.01\\
48.27	0.01\\
48.28	0.01\\
48.29	0.01\\
48.3	0.01\\
48.31	0.01\\
48.32	0.01\\
48.33	0.01\\
48.34	0.01\\
48.35	0.01\\
48.36	0.01\\
48.37	0.01\\
48.38	0.01\\
48.39	0.01\\
48.4	0.01\\
48.41	0.01\\
48.42	0.01\\
48.43	0.01\\
48.44	0.01\\
48.45	0.01\\
48.46	0.01\\
48.47	0.01\\
48.48	0.01\\
48.49	0.01\\
48.5	0.01\\
48.51	0.01\\
48.52	0.01\\
48.53	0.00999994939160281\\
48.54	0.00999908902815979\\
48.55	0.00999822815220618\\
48.56	0.00999736676329754\\
48.57	0.00999650486098884\\
48.58	0.00999564244483448\\
48.59	0.00999477951438828\\
48.6	0.00999391606920347\\
48.61	0.00999305210883271\\
48.62	0.00999218763282806\\
48.63	0.009991322640741\\
48.64	0.00999045713212242\\
48.65	0.00998959110652264\\
48.66	0.00998872456349135\\
48.67	0.00998785750257772\\
48.68	0.00998698992333025\\
48.69	0.0099861218252969\\
48.7	0.00998525320802502\\
48.71	0.00998438407106136\\
48.72	0.0099835144139521\\
48.73	0.00998264423624279\\
48.74	0.00998177353747841\\
48.75	0.00998090231720331\\
48.76	0.00998003057496127\\
48.77	0.00997915831029547\\
48.78	0.00997828552274846\\
48.79	0.0099774122118622\\
48.8	0.00997653837717807\\
48.81	0.00997566401823681\\
48.82	0.00997478913457856\\
48.83	0.00997391372574287\\
48.84	0.00997303779126867\\
48.85	0.00997216133069428\\
48.86	0.00997128434355739\\
48.87	0.00997040682939513\\
48.88	0.00996952878774396\\
48.89	0.00996865021813976\\
48.9	0.00996777112011778\\
48.91	0.00996689149321265\\
48.92	0.00996601133695839\\
48.93	0.00996513065088841\\
48.94	0.00996424943453549\\
48.95	0.00996336768743177\\
48.96	0.0099624854091088\\
48.97	0.00996160259909748\\
48.98	0.0099607192569281\\
48.99	0.00995983538213033\\
49	0.00995895097423318\\
49.01	0.00995806603276508\\
49.02	0.00995718055725377\\
49.03	0.00995629454722643\\
49.04	0.00995540800220955\\
49.05	0.00995452092172901\\
49.06	0.00995363330531005\\
49.07	0.00995274515247729\\
49.08	0.00995185646275469\\
49.09	0.00995096723566558\\
49.1	0.00995007747073266\\
49.11	0.00994918716747798\\
49.12	0.00994829632542295\\
49.13	0.00994740494408833\\
49.14	0.00994651302299425\\
49.15	0.0099456205616602\\
49.16	0.00994472755960499\\
49.17	0.00994383401634682\\
49.18	0.00994293993140321\\
49.19	0.00994204530429105\\
49.2	0.00994115013452656\\
49.21	0.00994025442162535\\
49.22	0.00993935816510231\\
49.23	0.00993846136447174\\
49.24	0.00993756401924723\\
49.25	0.00993666612894175\\
49.26	0.00993576769306758\\
49.27	0.00993486871113637\\
49.28	0.0099339691826591\\
49.29	0.00993306910714606\\
49.3	0.0099321684841069\\
49.31	0.00993126731305063\\
49.32	0.00993036559348553\\
49.33	0.00992946332491926\\
49.34	0.0099285605068588\\
49.35	0.00992765713881045\\
49.36	0.00992675322027986\\
49.37	0.00992584875077197\\
49.38	0.00992494372979109\\
49.39	0.00992403815684082\\
49.4	0.00992313203142409\\
49.41	0.00992222535304316\\
49.42	0.00992131812119962\\
49.43	0.00992041033539434\\
49.44	0.00991950199512755\\
49.45	0.00991859309989877\\
49.46	0.00991768364920686\\
49.47	0.00991677364254996\\
49.48	0.00991586307942555\\
49.49	0.00991495195933041\\
49.5	0.00991404028176063\\
49.51	0.00991312804621162\\
49.52	0.00991221525217807\\
49.53	0.009911301899154\\
49.54	0.00991038798663274\\
49.55	0.00990947351410688\\
49.56	0.00990855848106837\\
49.57	0.00990764288700842\\
49.58	0.00990672673141755\\
49.59	0.00990581001378558\\
49.6	0.00990489273360162\\
49.61	0.00990397489035409\\
49.62	0.00990305648353068\\
49.63	0.0099021375126184\\
49.64	0.00990121797710351\\
49.65	0.00990029787647161\\
49.66	0.00989937721020756\\
49.67	0.0098984559777955\\
49.68	0.00989753417871886\\
49.69	0.00989661181246037\\
49.7	0.00989568887850202\\
49.71	0.00989476537632509\\
49.72	0.00989384130541016\\
49.73	0.00989291666523705\\
49.74	0.00989199145528488\\
49.75	0.00989106567503205\\
49.76	0.0098901393239562\\
49.77	0.00988921240153429\\
49.78	0.00988828490724252\\
49.79	0.00988735684055637\\
49.8	0.00988642820095057\\
49.81	0.00988549898789915\\
49.82	0.00988456920087538\\
49.83	0.0098836388393518\\
49.84	0.00988270790280022\\
49.85	0.00988177639069169\\
49.86	0.00988084430249655\\
49.87	0.00987991163768437\\
49.88	0.00987897839572399\\
49.89	0.0098780445760835\\
49.9	0.00987711017823025\\
49.91	0.00987617520163084\\
49.92	0.00987523964575111\\
49.93	0.00987430351005616\\
49.94	0.00987336679401034\\
49.95	0.00987242949707724\\
49.96	0.00987149161871969\\
49.97	0.00987055315839977\\
49.98	0.00986961411557881\\
49.99	0.00986867448971735\\
50	0.00986773428027521\\
50.01	0.00986679348671141\\
50.02	0.00986585210848423\\
50.03	0.00986491014505117\\
50.04	0.00986396759586897\\
50.05	0.00986302446039359\\
50.06	0.00986208073808025\\
50.07	0.00986113642838335\\
50.08	0.00986019153075655\\
50.09	0.00985924604465273\\
50.1	0.00985829996952399\\
50.11	0.00985735330482164\\
50.12	0.00985640604999624\\
50.13	0.00985545820449753\\
50.14	0.0098545097677745\\
50.15	0.00985356073927533\\
50.16	0.00985261111844743\\
50.17	0.00985166090473743\\
50.18	0.00985071009759114\\
50.19	0.0098497586964536\\
50.2	0.00984880670076907\\
50.21	0.00984785410998098\\
50.22	0.00984690092353199\\
50.23	0.00984594714086397\\
50.24	0.00984499276141795\\
50.25	0.00984403778463422\\
50.26	0.00984308220995221\\
50.27	0.00984212603681059\\
50.28	0.00984116926464719\\
50.29	0.00984021189289905\\
50.3	0.00983925392100241\\
50.31	0.00983829534839268\\
50.32	0.00983733617450448\\
50.33	0.00983637639877158\\
50.34	0.00983541602062698\\
50.35	0.00983445503950284\\
50.36	0.00983349345483048\\
50.37	0.00983253126604045\\
50.38	0.00983156847256243\\
50.39	0.0098306050738253\\
50.4	0.00982964106925712\\
50.41	0.0098286764582851\\
50.42	0.00982771124033566\\
50.43	0.00982674541483433\\
50.44	0.00982577898120587\\
50.45	0.00982481193887416\\
50.46	0.00982384428726227\\
50.47	0.00982287602579244\\
50.48	0.00982190715388604\\
50.49	0.00982093767096362\\
50.5	0.00981996757644488\\
50.51	0.00981899686974869\\
50.52	0.00981802555029305\\
50.53	0.00981705361749514\\
50.54	0.00981608107077127\\
50.55	0.00981510790953691\\
50.56	0.00981413413320668\\
50.57	0.00981315974119431\\
50.58	0.00981218473291273\\
50.59	0.00981120910777398\\
50.6	0.00981023286518923\\
50.61	0.00980925600456881\\
50.62	0.00980827852532218\\
50.63	0.00980730042685794\\
50.64	0.00980632170858381\\
50.65	0.00980534236990665\\
50.66	0.00980436241023245\\
50.67	0.00980338182896632\\
50.68	0.0098024006255125\\
50.69	0.00980141879927435\\
50.7	0.00980043634965438\\
50.71	0.00979945327605418\\
50.72	0.00979846957787447\\
50.73	0.00979748525451511\\
50.74	0.00979650030537504\\
50.75	0.00979551472985235\\
50.76	0.00979452852734421\\
50.77	0.00979354169724691\\
50.78	0.00979255423895586\\
50.79	0.00979156615186555\\
50.8	0.00979057743536959\\
50.81	0.00978958808886071\\
50.82	0.00978859811173069\\
50.83	0.00978760750337047\\
50.84	0.00978661626317003\\
50.85	0.00978562439051848\\
50.86	0.00978463188480401\\
50.87	0.00978363874541389\\
50.88	0.00978264497173452\\
50.89	0.00978165056315133\\
50.9	0.00978065551904887\\
50.91	0.00977965983881078\\
50.92	0.00977866352181975\\
50.93	0.00977766656745756\\
50.94	0.00977666897510509\\
50.95	0.00977567074414228\\
50.96	0.00977467187394813\\
50.97	0.00977367236390072\\
50.98	0.00977267221337721\\
50.99	0.00977167142175382\\
51	0.00977066998840583\\
51.01	0.0097696679127076\\
51.02	0.00976866519403252\\
51.03	0.00976766183175308\\
51.04	0.00976665782524079\\
51.05	0.00976565317386625\\
51.06	0.00976464787699908\\
51.07	0.00976364193400798\\
51.08	0.00976263534426068\\
51.09	0.00976162810712397\\
51.1	0.00976062022196368\\
51.11	0.00975961168814468\\
51.12	0.00975860250503089\\
51.13	0.00975759267198526\\
51.14	0.00975658218836979\\
51.15	0.0097555710535455\\
51.16	0.00975455926687246\\
51.17	0.00975354682770976\\
51.18	0.00975253373541552\\
51.19	0.00975151998934689\\
51.2	0.00975050558886003\\
51.21	0.00974949053331016\\
51.22	0.00974847482205149\\
51.23	0.00974745845443725\\
51.24	0.00974644142981969\\
51.25	0.0097454237475501\\
51.26	0.00974440540697874\\
51.27	0.0097433864074549\\
51.28	0.00974236674832689\\
51.29	0.00974134642894202\\
51.3	0.00974032544864658\\
51.31	0.0097393038067859\\
51.32	0.00973828150270428\\
51.33	0.00973725853574504\\
51.34	0.00973623490525047\\
51.35	0.00973521061056188\\
51.36	0.00973418565101955\\
51.37	0.00973316002596276\\
51.38	0.00973213373472977\\
51.39	0.00973110677665784\\
51.4	0.0097300791510832\\
51.41	0.00972905085734106\\
51.42	0.00972802189476561\\
51.43	0.00972699226269001\\
51.44	0.00972596196044641\\
51.45	0.0097249309873659\\
51.46	0.00972389934277859\\
51.47	0.00972286702601351\\
51.48	0.00972183403639867\\
51.49	0.00972080037326104\\
51.5	0.00971976603592657\\
51.51	0.00971873102372015\\
51.52	0.00971769533596562\\
51.53	0.00971665897198578\\
51.54	0.0097156219311024\\
51.55	0.00971458421263617\\
51.56	0.00971354581590673\\
51.57	0.00971250674023269\\
51.58	0.00971146698493157\\
51.59	0.00971042654931987\\
51.6	0.00970938543271297\\
51.61	0.00970834363442523\\
51.62	0.00970730115376994\\
51.63	0.0097062579900593\\
51.64	0.00970521414260446\\
51.65	0.00970416961071547\\
51.66	0.00970312439370132\\
51.67	0.00970207849086993\\
51.68	0.00970103190152813\\
51.69	0.00969998462498164\\
51.7	0.00969893666053515\\
51.71	0.00969788800749222\\
51.72	0.00969683866515532\\
51.73	0.00969578863282584\\
51.74	0.00969473790980407\\
51.75	0.00969368649538921\\
51.76	0.00969263438887935\\
51.77	0.00969158158957147\\
51.78	0.00969052809676147\\
51.79	0.00968947390974412\\
51.8	0.00968841902781309\\
51.81	0.00968736345026093\\
51.82	0.00968630717637907\\
51.83	0.00968525020545786\\
51.84	0.00968419253678648\\
51.85	0.00968313416965302\\
51.86	0.00968207510334442\\
51.87	0.00968101533714652\\
51.88	0.00967995487034402\\
51.89	0.00967889370222048\\
51.9	0.00967783183205834\\
51.91	0.00967676925913888\\
51.92	0.00967570598274227\\
51.93	0.0096746420021475\\
51.94	0.00967357731663246\\
51.95	0.00967251192547385\\
51.96	0.00967144582794725\\
51.97	0.00967037902332708\\
51.98	0.00966931151088659\\
51.99	0.00966824328989789\\
52	0.00966717435963193\\
52.01	0.00966610471935848\\
52.02	0.00966503436834615\\
52.03	0.00966396330586241\\
52.04	0.00966289153117351\\
52.05	0.00966181904354458\\
52.06	0.00966074584223953\\
52.07	0.00965967192652112\\
52.08	0.0096585972956509\\
52.09	0.00965752194888928\\
52.1	0.00965644588549545\\
52.11	0.0096553691047274\\
52.12	0.00965429160584197\\
52.13	0.00965321338809477\\
52.14	0.00965213445074023\\
52.15	0.00965105479303159\\
52.16	0.00964997441422085\\
52.17	0.00964889331355884\\
52.18	0.00964781149029517\\
52.19	0.00964672894367824\\
52.2	0.00964564567295523\\
52.21	0.00964456167737212\\
52.22	0.00964347695617367\\
52.23	0.00964239150860339\\
52.24	0.00964130533390361\\
52.25	0.00964021843131539\\
52.26	0.00963913080007858\\
52.27	0.00963804243943181\\
52.28	0.00963695334861245\\
52.29	0.00963586352685666\\
52.3	0.00963477297339932\\
52.31	0.0096336816874741\\
52.32	0.00963258966831341\\
52.33	0.00963149691514842\\
52.34	0.00963040342720904\\
52.35	0.00962930920372392\\
52.36	0.00962821424392046\\
52.37	0.0096271185470248\\
52.38	0.00962602211226181\\
52.39	0.0096249249388551\\
52.4	0.00962382702602699\\
52.41	0.00962272837299858\\
52.42	0.00962162897898963\\
52.43	0.00962052884321866\\
52.44	0.00961942796490291\\
52.45	0.00961832634325833\\
52.46	0.00961722397749957\\
52.47	0.00961612086683999\\
52.48	0.00961501701049169\\
52.49	0.00961391240766545\\
52.5	0.00961280705757074\\
52.51	0.00961170095941575\\
52.52	0.00961059411240736\\
52.53	0.00960948651575113\\
52.54	0.00960837816865134\\
52.55	0.00960726907031092\\
52.56	0.0096061592199315\\
52.57	0.00960504861671339\\
52.58	0.00960393725985559\\
52.59	0.00960282514855574\\
52.6	0.0096017122820102\\
52.61	0.00960059865941394\\
52.62	0.00959948427996065\\
52.63	0.00959836914284265\\
52.64	0.00959725324725093\\
52.65	0.00959613659237513\\
52.66	0.00959501917740355\\
52.67	0.00959390100152314\\
52.68	0.00959278206391947\\
52.69	0.00959166236377681\\
52.7	0.00959054190027801\\
52.71	0.0095894206726046\\
52.72	0.00958829867993672\\
52.73	0.00958717592145315\\
52.74	0.0095860523963313\\
52.75	0.00958492810374719\\
52.76	0.00958380304287549\\
52.77	0.00958267721288945\\
52.78	0.00958155061296097\\
52.79	0.00958042324226055\\
52.8	0.00957929509995728\\
52.81	0.00957816618521889\\
52.82	0.00957703649721167\\
52.83	0.00957590603510055\\
52.84	0.00957477479804904\\
52.85	0.00957364278521924\\
52.86	0.00957250999577182\\
52.87	0.00957137642886606\\
52.88	0.00957024208365983\\
52.89	0.00956910695930955\\
52.9	0.00956797105497025\\
52.91	0.0095668343697955\\
52.92	0.00956569690293746\\
52.93	0.00956455865354684\\
52.94	0.00956341962077293\\
52.95	0.00956227980376357\\
52.96	0.00956113920166514\\
52.97	0.00955999781362261\\
52.98	0.00955885563877947\\
52.99	0.00955771267627776\\
53	0.00955656892525806\\
53.01	0.0095554243848595\\
53.02	0.00955427905421974\\
53.03	0.00955313293247497\\
53.04	0.0095519860187599\\
53.05	0.00955083831220778\\
53.06	0.00954968981195037\\
53.07	0.00954854051711795\\
53.08	0.00954739042683933\\
53.09	0.00954623954024181\\
53.1	0.00954508785645119\\
53.11	0.00954393537459182\\
53.12	0.0095427820937865\\
53.13	0.00954162801315654\\
53.14	0.00954047313182176\\
53.15	0.00953931744890046\\
53.16	0.00953816096350941\\
53.17	0.00953700367476389\\
53.18	0.00953584558177765\\
53.19	0.0095346866836629\\
53.2	0.00953352697953033\\
53.21	0.00953236646848911\\
53.22	0.00953120514964688\\
53.23	0.0095300430221097\\
53.24	0.00952888008498211\\
53.25	0.00952771633736713\\
53.26	0.0095265517783662\\
53.27	0.0095253864070792\\
53.28	0.00952422022260447\\
53.29	0.00952305322403878\\
53.3	0.00952188541047735\\
53.31	0.00952071678101381\\
53.32	0.00951954733474022\\
53.33	0.00951837707074709\\
53.34	0.0095172059881233\\
53.35	0.00951603408595621\\
53.36	0.00951486136333154\\
53.37	0.00951368781933343\\
53.38	0.00951251345304444\\
53.39	0.00951133826354553\\
53.4	0.00951016224991603\\
53.41	0.0095089854112337\\
53.42	0.00950780774657467\\
53.43	0.00950662925501345\\
53.44	0.00950544993562294\\
53.45	0.00950426978747441\\
53.46	0.00950308880963753\\
53.47	0.00950190700118031\\
53.48	0.00950072436116913\\
53.49	0.00949954088866874\\
53.5	0.00949835658274225\\
53.51	0.00949717144245112\\
53.52	0.00949598546685516\\
53.53	0.00949479865501253\\
53.54	0.00949361100597972\\
53.55	0.00949242251881158\\
53.56	0.00949123319256126\\
53.57	0.00949004302628028\\
53.58	0.00948885201901846\\
53.59	0.00948766016982395\\
53.6	0.00948646747774321\\
53.61	0.00948527394182103\\
53.62	0.0094840795611005\\
53.63	0.00948288433462301\\
53.64	0.00948168826142827\\
53.65	0.00948049134055426\\
53.66	0.00947929357103727\\
53.67	0.00947809495191189\\
53.68	0.00947689548221097\\
53.69	0.00947569516096566\\
53.7	0.00947449398720538\\
53.71	0.00947329195995781\\
53.72	0.00947208907824894\\
53.73	0.00947088534110296\\
53.74	0.00946968074754237\\
53.75	0.00946847529658792\\
53.76	0.00946726898725859\\
53.77	0.00946606181857161\\
53.78	0.00946485378954247\\
53.79	0.00946364489918488\\
53.8	0.00946243514651081\\
53.81	0.00946122453053043\\
53.82	0.00946001305025214\\
53.83	0.00945880070468259\\
53.84	0.00945758749282661\\
53.85	0.00945637341368727\\
53.86	0.00945515846626582\\
53.87	0.00945394264956174\\
53.88	0.0094527259625727\\
53.89	0.00945150840429454\\
53.9	0.00945028997372135\\
53.91	0.00944907066984535\\
53.92	0.00944785049165695\\
53.93	0.00944662943814477\\
53.94	0.00944540750829556\\
53.95	0.00944418470109427\\
53.96	0.00944296101552398\\
53.97	0.00944173645056597\\
53.98	0.00944051100519965\\
53.99	0.00943928467840257\\
54	0.00943805746915045\\
54.01	0.00943682937641714\\
54.02	0.00943560039917462\\
54.03	0.009434370536393\\
54.04	0.00943313978704051\\
54.05	0.00943190815008355\\
54.06	0.00943067562448657\\
54.07	0.00942944220921218\\
54.08	0.00942820790322107\\
54.09	0.00942697270547206\\
54.1	0.00942573661492203\\
54.11	0.009424499630526\\
54.12	0.00942326175123704\\
54.13	0.00942202297600633\\
54.14	0.00942078330378312\\
54.15	0.00941954273351472\\
54.16	0.00941830126414654\\
54.17	0.00941705889462203\\
54.18	0.00941581562388272\\
54.19	0.00941457145086818\\
54.2	0.00941332637451603\\
54.21	0.00941208039376196\\
54.22	0.00941083350753965\\
54.23	0.00940958571478087\\
54.24	0.0094083370144154\\
54.25	0.00940708740537104\\
54.26	0.00940583688657361\\
54.27	0.00940458545694696\\
54.28	0.00940333311541295\\
54.29	0.00940207986089142\\
54.3	0.00940082569230024\\
54.31	0.00939957060855528\\
54.32	0.00939831460857037\\
54.33	0.00939705769125735\\
54.34	0.00939579985552603\\
54.35	0.0093945411002842\\
54.36	0.00939328142443764\\
54.37	0.00939202082689006\\
54.38	0.00939075930654314\\
54.39	0.00938949686229654\\
54.4	0.00938823349304785\\
54.41	0.0093869691976926\\
54.42	0.00938570397512427\\
54.43	0.00938443782423429\\
54.44	0.00938317074391197\\
54.45	0.0093819027330446\\
54.46	0.00938063379051736\\
54.47	0.00937936391521334\\
54.48	0.00937809310601357\\
54.49	0.00937682136179693\\
54.5	0.00937554868144025\\
54.51	0.00937427506381822\\
54.52	0.00937300050780344\\
54.53	0.00937172501226635\\
54.54	0.00937044857607533\\
54.55	0.00936917119809657\\
54.56	0.00936789287719417\\
54.57	0.00936661361223005\\
54.58	0.00936533340206403\\
54.59	0.00936405224555375\\
54.6	0.00936277014155468\\
54.61	0.00936148708892017\\
54.62	0.00936020308650137\\
54.63	0.00935891813314726\\
54.64	0.00935763222770466\\
54.65	0.0093563453690182\\
54.66	0.00935505755593029\\
54.67	0.00935376878728119\\
54.68	0.00935247906190893\\
54.69	0.00935118837864934\\
54.7	0.00934989673633605\\
54.71	0.00934860413380045\\
54.72	0.00934731056987172\\
54.73	0.00934601604337682\\
54.74	0.00934472055314046\\
54.75	0.0093434240979851\\
54.76	0.00934212667673098\\
54.77	0.00934082828819607\\
54.78	0.0093395289311961\\
54.79	0.0093382286045445\\
54.8	0.00933692730705247\\
54.81	0.00933562503752891\\
54.82	0.00933432179478045\\
54.83	0.00933301757761144\\
54.84	0.0093317123848239\\
54.85	0.00933040621521759\\
54.86	0.00932909906758996\\
54.87	0.00932779094073614\\
54.88	0.00932648183344892\\
54.89	0.00932517174451882\\
54.9	0.00932386067273399\\
54.91	0.00932254861688025\\
54.92	0.0093212355757411\\
54.93	0.00931992154809766\\
54.94	0.00931860653272874\\
54.95	0.00931729052841076\\
54.96	0.00931597353391776\\
54.97	0.00931465554802144\\
54.98	0.00931333656949113\\
54.99	0.00931201659709373\\
55	0.0093106956295938\\
55.01	0.00930937366575347\\
55.02	0.00930805070433247\\
55.03	0.00930672674408814\\
55.04	0.0093054017837754\\
55.05	0.00930407582214674\\
55.06	0.00930274885795221\\
55.07	0.00930142088993946\\
55.08	0.00930009191685367\\
55.09	0.00929876193743759\\
55.1	0.0092974309504315\\
55.11	0.00929609895457325\\
55.12	0.00929476594859819\\
55.13	0.00929343193123922\\
55.14	0.00929209690122674\\
55.15	0.0092907608572887\\
55.16	0.00928942379815052\\
55.17	0.00928808572253514\\
55.18	0.009286746629163\\
55.19	0.00928540651675201\\
55.2	0.00928406538401758\\
55.21	0.00928272322967259\\
55.22	0.00928138005242738\\
55.23	0.00928003585098976\\
55.24	0.00927869062406498\\
55.25	0.00927734437035578\\
55.26	0.00927599708856229\\
55.27	0.0092746487773821\\
55.28	0.00927329943551024\\
55.29	0.00927194906163914\\
55.3	0.00927059765445865\\
55.31	0.00926924521265603\\
55.32	0.00926789173491595\\
55.33	0.00926653721992045\\
55.34	0.00926518166634898\\
55.35	0.00926382507287836\\
55.36	0.0092624674381828\\
55.37	0.00926110876093385\\
55.38	0.00925974903980043\\
55.39	0.00925838827344882\\
55.4	0.00925702646054263\\
55.41	0.00925566359974283\\
55.42	0.0092542996897077\\
55.43	0.00925293472909286\\
55.44	0.00925156871655123\\
55.45	0.00925020165073306\\
55.46	0.00924883353028589\\
55.47	0.00924746435385457\\
55.48	0.0092460941200812\\
55.49	0.00924472282760521\\
55.5	0.00924335047506328\\
55.51	0.00924197706108935\\
55.52	0.00924060258431462\\
55.53	0.00923922704336758\\
55.54	0.00923785043687391\\
55.55	0.00923647276345656\\
55.56	0.0092350940217357\\
55.57	0.00923371421032872\\
55.58	0.00923233332785025\\
55.59	0.00923095137291208\\
55.6	0.00922956834412324\\
55.61	0.00922818424008995\\
55.62	0.00922679905941559\\
55.63	0.00922541280070074\\
55.64	0.00922402546254314\\
55.65	0.0092226370435377\\
55.66	0.00922124754227647\\
55.67	0.00921985695734868\\
55.68	0.00921846528734065\\
55.69	0.00921707253083588\\
55.7	0.00921567868641496\\
55.71	0.00921428375265561\\
55.72	0.00921288772813267\\
55.73	0.00921149061141806\\
55.74	0.00921009240108079\\
55.75	0.00920869309568699\\
55.76	0.00920729269379983\\
55.77	0.00920589119397956\\
55.78	0.00920448859478351\\
55.79	0.00920308489476603\\
55.8	0.00920168009247855\\
55.81	0.00920027418646952\\
55.82	0.00919886717528441\\
55.83	0.00919745905746573\\
55.84	0.009196049831553\\
55.85	0.00919463949608273\\
55.86	0.00919322804958844\\
55.87	0.00919181549060065\\
55.88	0.00919040181764685\\
55.89	0.00918898702925148\\
55.9	0.00918757112393598\\
55.91	0.00918615410021873\\
55.92	0.00918473595661506\\
55.93	0.00918331669163725\\
55.94	0.00918189630379448\\
55.95	0.00918047479159289\\
55.96	0.00917905215353552\\
55.97	0.00917762838812231\\
55.98	0.0091762034938501\\
55.99	0.00917477746921262\\
56	0.00917335031270049\\
56.01	0.00917192202280119\\
56.02	0.00917049259799907\\
56.03	0.00916906203677533\\
56.04	0.00916763033760803\\
56.05	0.00916619749897204\\
56.06	0.0091647635193391\\
56.07	0.00916332839717775\\
56.08	0.00916189213095331\\
56.09	0.00916045471912796\\
56.1	0.00915901616016064\\
56.11	0.00915757645250708\\
56.12	0.0091561355946198\\
56.13	0.00915469358494807\\
56.14	0.00915325042193792\\
56.15	0.00915180610403215\\
56.16	0.00915036062967028\\
56.17	0.00914891399728857\\
56.18	0.00914746620532002\\
56.19	0.0091460172521943\\
56.2	0.00914456713633784\\
56.21	0.00914311585617373\\
56.22	0.00914166341012175\\
56.23	0.00914020979659837\\
56.24	0.00913875501401672\\
56.25	0.00913729906078658\\
56.26	0.00913584193531443\\
56.27	0.00913438363600331\\
56.28	0.00913292416125296\\
56.29	0.00913146350945972\\
56.3	0.00913000167901652\\
56.31	0.00912853866831294\\
56.32	0.0091270744757351\\
56.33	0.00912560909966575\\
56.34	0.0091241425384842\\
56.35	0.00912267479056632\\
56.36	0.00912120585428454\\
56.37	0.00911973572800784\\
56.38	0.00911826441010173\\
56.39	0.00911679189892826\\
56.4	0.00911531819284597\\
56.41	0.00911384329020995\\
56.42	0.00911236718937176\\
56.43	0.00911088988867946\\
56.44	0.00910941138647757\\
56.45	0.00910793168110711\\
56.46	0.00910645077090552\\
56.47	0.00910496865420673\\
56.48	0.00910348532934109\\
56.49	0.00910200079463538\\
56.5	0.00910051504841278\\
56.51	0.00909902808899293\\
56.52	0.00909753991469183\\
56.53	0.00909605052382187\\
56.54	0.00909455991469184\\
56.55	0.00909306808560687\\
56.56	0.00909157503486849\\
56.57	0.00909008076077453\\
56.58	0.00908858526161921\\
56.59	0.00908708853569303\\
56.6	0.00908559058128284\\
56.61	0.00908409139667179\\
56.62	0.00908259098013932\\
56.63	0.00908108932996116\\
56.64	0.00907958644440933\\
56.65	0.00907808232175209\\
56.66	0.00907657696025398\\
56.67	0.00907507035817577\\
56.68	0.00907356251377447\\
56.69	0.00907205342530332\\
56.7	0.00907054309101176\\
56.71	0.00906903150914544\\
56.72	0.00906751867794619\\
56.73	0.00906600459565205\\
56.74	0.00906448926049719\\
56.75	0.00906297267071199\\
56.76	0.00906145482452292\\
56.77	0.00905993572015264\\
56.78	0.00905841535581989\\
56.79	0.00905689372973956\\
56.8	0.00905537084012265\\
56.81	0.00905384668517622\\
56.82	0.00905232126310343\\
56.83	0.00905079457210352\\
56.84	0.00904926661037177\\
56.85	0.00904773737609953\\
56.86	0.00904620686747419\\
56.87	0.00904467508267913\\
56.88	0.0090431420198938\\
56.89	0.0090416076772936\\
56.9	0.00904007205304996\\
56.91	0.00903853514533029\\
56.92	0.00903699695229794\\
56.93	0.00903545747211224\\
56.94	0.00903391670292848\\
56.95	0.00903237464289785\\
56.96	0.00903083129016751\\
56.97	0.00902928664288048\\
56.98	0.00902774069917571\\
56.99	0.00902619345718805\\
57	0.0090246449150482\\
57.01	0.00902309507088274\\
57.02	0.00902154392281409\\
57.03	0.00901999146896053\\
57.04	0.00901843770743617\\
57.05	0.00901688263635092\\
57.06	0.0090153262538105\\
57.07	0.00901376855791644\\
57.08	0.00901220954676603\\
57.09	0.00901064921845233\\
57.1	0.00900908757106419\\
57.11	0.00900752460268617\\
57.12	0.00900596031139857\\
57.13	0.00900439469527742\\
57.14	0.00900282775239444\\
57.15	0.00900125948081708\\
57.16	0.00899968987860845\\
57.17	0.00899811894382732\\
57.18	0.00899654667452814\\
57.19	0.00899497306876101\\
57.2	0.00899339812457163\\
57.21	0.00899182184000136\\
57.22	0.00899024421308714\\
57.23	0.00898866524186154\\
57.24	0.00898708492435266\\
57.25	0.00898550325858422\\
57.26	0.00898392024257548\\
57.27	0.00898233587434124\\
57.28	0.00898075015189182\\
57.29	0.00897916307323309\\
57.3	0.0089775746363664\\
57.31	0.00897598483928862\\
57.32	0.00897439367999205\\
57.33	0.00897280115646452\\
57.34	0.00897120726668925\\
57.35	0.00896961200864496\\
57.36	0.00896801538030575\\
57.37	0.00896641737964116\\
57.38	0.00896481800461612\\
57.39	0.00896321725319096\\
57.4	0.00896161512332136\\
57.41	0.00896001161295837\\
57.42	0.0089584067200484\\
57.43	0.00895680044253318\\
57.44	0.00895519277834976\\
57.45	0.0089535837254305\\
57.46	0.00895197328170305\\
57.47	0.00895036144509034\\
57.48	0.00894874821351057\\
57.49	0.00894713358487716\\
57.5	0.00894551755709882\\
57.51	0.00894390012807943\\
57.52	0.00894228129571812\\
57.53	0.00894066105790919\\
57.54	0.00893903941254212\\
57.55	0.00893741635750158\\
57.56	0.00893579189066737\\
57.57	0.00893416600991444\\
57.58	0.00893253871311286\\
57.59	0.0089309099981278\\
57.6	0.00892927986281956\\
57.61	0.00892764830504346\\
57.62	0.00892601532264996\\
57.63	0.00892438091348452\\
57.64	0.00892274507538765\\
57.65	0.00892110780619489\\
57.66	0.00891946910373679\\
57.67	0.00891782896583888\\
57.68	0.00891618739032168\\
57.69	0.00891454437500066\\
57.7	0.00891289991768627\\
57.71	0.00891125401618388\\
57.72	0.00890960666829375\\
57.73	0.00890795787181108\\
57.74	0.00890630762452595\\
57.75	0.00890465592422332\\
57.76	0.00890300276868298\\
57.77	0.00890134815567961\\
57.78	0.00889969208298268\\
57.79	0.00889803454835649\\
57.8	0.00889637554956014\\
57.81	0.00889471508434751\\
57.82	0.00889305315046724\\
57.83	0.00889138974566272\\
57.84	0.00888972486767211\\
57.85	0.00888805851422823\\
57.86	0.00888639068305866\\
57.87	0.00888472137188565\\
57.88	0.00888305057842611\\
57.89	0.00888137830039163\\
57.9	0.00887970453548842\\
57.91	0.00887802928141733\\
57.92	0.00887635253587381\\
57.93	0.00887467429654792\\
57.94	0.00887299456112428\\
57.95	0.00887131332728208\\
57.96	0.00886963059269505\\
57.97	0.00886794635503146\\
57.98	0.00886626061195407\\
57.99	0.00886457336112016\\
58	0.00886288460018149\\
58.01	0.00886119432678425\\
58.02	0.00885950253856912\\
58.03	0.00885780923317118\\
58.04	0.00885611440821995\\
58.05	0.0088544180613393\\
58.06	0.00885272019014754\\
58.07	0.0088510207922573\\
58.08	0.00884931986527557\\
58.09	0.00884761740680367\\
58.1	0.00884591341443722\\
58.11	0.00884420788576617\\
58.12	0.00884250081837471\\
58.13	0.0088407922098413\\
58.14	0.00883908205773866\\
58.15	0.00883737035963371\\
58.16	0.00883565711308761\\
58.17	0.00883394231565568\\
58.18	0.00883222596488745\\
58.19	0.00883050805832656\\
58.2	0.00882878859351083\\
58.21	0.00882706756797219\\
58.22	0.00882534497923665\\
58.23	0.00882362082482435\\
58.24	0.00882189510224945\\
58.25	0.00882016780902019\\
58.26	0.00881843894263885\\
58.27	0.00881670850060168\\
58.28	0.00881497648039898\\
58.29	0.00881324287951499\\
58.3	0.00881150769542792\\
58.31	0.00880977092560993\\
58.32	0.00880803256748838\\
58.33	0.00880629261848303\\
58.34	0.00880455107600795\\
58.35	0.00880280793747143\\
58.36	0.00880106320027607\\
58.37	0.00879931686181868\\
58.38	0.00879756891949031\\
58.39	0.00879581937067621\\
58.4	0.00879406821275585\\
58.41	0.00879231544310285\\
58.42	0.00879056105908503\\
58.43	0.00878880505806433\\
58.44	0.00878704743739684\\
58.45	0.00878528819443278\\
58.46	0.00878352732651645\\
58.47	0.00878176483098627\\
58.48	0.0087800007051747\\
58.49	0.00877823494640829\\
58.5	0.0087764675520076\\
58.51	0.00877469851928725\\
58.52	0.00877292784555584\\
58.53	0.008771155528116\\
58.54	0.0087693815642643\\
58.55	0.0087676059512913\\
58.56	0.00876582868648151\\
58.57	0.00876404976711335\\
58.58	0.00876226919045917\\
58.59	0.00876048695378522\\
58.6	0.00875870305435163\\
58.61	0.0087569174894124\\
58.62	0.00875513025621537\\
58.63	0.00875334135200224\\
58.64	0.00875155077400851\\
58.65	0.00874975851946347\\
58.66	0.00874796458559022\\
58.67	0.00874616896960561\\
58.68	0.00874437166872027\\
58.69	0.00874257268013854\\
58.7	0.00874077200105849\\
58.71	0.00873896962867189\\
58.72	0.0087371655601642\\
58.73	0.00873535979271454\\
58.74	0.00873355232349569\\
58.75	0.00873174314967408\\
58.76	0.00872993226840973\\
58.77	0.0087281196768563\\
58.78	0.00872630537216098\\
58.79	0.00872448935146459\\
58.8	0.00872267161190145\\
58.81	0.00872085215059945\\
58.82	0.00871903096467997\\
58.83	0.00871720805125791\\
58.84	0.00871538340744163\\
58.85	0.00871355703033299\\
58.86	0.00871172891702725\\
58.87	0.00870989906461315\\
58.88	0.00870806747017279\\
58.89	0.00870623413078172\\
58.9	0.00870439904350881\\
58.91	0.00870256220541635\\
58.92	0.0087007236135599\\
58.93	0.00869888326498842\\
58.94	0.00869704115674413\\
58.95	0.00869519728586253\\
58.96	0.00869335164937242\\
58.97	0.00869150424429585\\
58.98	0.00868965506764808\\
58.99	0.00868780411643761\\
59	0.00868595138766612\\
59.01	0.00868409687832849\\
59.02	0.00868224058541274\\
59.03	0.00868038250590004\\
59.04	0.00867852263676469\\
59.05	0.0086766609749741\\
59.06	0.00867479751748876\\
59.07	0.00867293226126221\\
59.08	0.00867106520324109\\
59.09	0.00866919634036503\\
59.1	0.00866732566956669\\
59.11	0.00866545318777171\\
59.12	0.00866357889189872\\
59.13	0.00866170277885931\\
59.14	0.00865982484555799\\
59.15	0.00865794508889221\\
59.16	0.0086560635057523\\
59.17	0.00865418009302149\\
59.18	0.00865229484757585\\
59.19	0.00865040776628431\\
59.2	0.00864851884600861\\
59.21	0.00864662808360331\\
59.22	0.00864473547591573\\
59.23	0.00864284101978598\\
59.24	0.00864094471204689\\
59.25	0.00863904654952404\\
59.26	0.0086371465290357\\
59.27	0.00863524464739283\\
59.28	0.00863334090139906\\
59.29	0.00863143528785066\\
59.3	0.00862952780353653\\
59.31	0.00862761844523818\\
59.32	0.0086257072097297\\
59.33	0.00862379409377775\\
59.34	0.00862187909414154\\
59.35	0.0086199622075728\\
59.36	0.00861804343081576\\
59.37	0.00861612276060716\\
59.38	0.00861420019367618\\
59.39	0.00861227572674446\\
59.4	0.00861034935652605\\
59.41	0.00860842107972741\\
59.42	0.0086064908930474\\
59.43	0.00860455879317721\\
59.44	0.00860262477680041\\
59.45	0.00860068884059286\\
59.46	0.00859875098122272\\
59.47	0.00859681119535046\\
59.48	0.00859486947962876\\
59.49	0.00859292583070259\\
59.5	0.0085909802452091\\
59.51	0.00858903271977763\\
59.52	0.00858708325102973\\
59.53	0.00858513183557907\\
59.54	0.00858317847003145\\
59.55	0.0085812231509848\\
59.56	0.00857926587502912\\
59.57	0.00857730663874648\\
59.58	0.00857534543871101\\
59.59	0.00857338227148883\\
59.6	0.00857141713363808\\
59.61	0.0085694500217089\\
59.62	0.00856748093224333\\
59.63	0.00856550986177541\\
59.64	0.00856353680683104\\
59.65	0.00856156176392805\\
59.66	0.0085595847295761\\
59.67	0.00855760570027673\\
59.68	0.0085556246725233\\
59.69	0.00855364164280094\\
59.7	0.00855165660758658\\
59.71	0.00854966956334892\\
59.72	0.00854768050654839\\
59.73	0.00854568943363709\\
59.74	0.00854369634105886\\
59.75	0.00854170122524918\\
59.76	0.00853970408263517\\
59.77	0.00853770490963558\\
59.78	0.00853570370266075\\
59.79	0.0085337004581126\\
59.8	0.00853169517238459\\
59.81	0.00852968784186172\\
59.82	0.00852767846292047\\
59.83	0.00852566703192883\\
59.84	0.00852365354524621\\
59.85	0.00852163799922349\\
59.86	0.00851962039020295\\
59.87	0.00851760071451823\\
59.88	0.00851557896849436\\
59.89	0.0085135551484477\\
59.9	0.00851152925068591\\
59.91	0.00850950127150796\\
59.92	0.00850747120720409\\
59.93	0.00850543905405574\\
59.94	0.00850340480833562\\
59.95	0.0085013684663076\\
59.96	0.00849933002422674\\
59.97	0.00849728947833921\\
59.98	0.00849524682488235\\
59.99	0.00849320206008454\\
60	0.00849115518016527\\
60.01	0.00848910618133506\\
60.02	0.00848705505979546\\
60.03	0.00848500181173901\\
60.04	0.00848294643334921\\
60.05	0.00848088892080053\\
60.06	0.00847882927025834\\
60.07	0.00847676747787891\\
60.08	0.00847470353980938\\
60.09	0.00847263745218775\\
60.1	0.00847056921114281\\
60.11	0.00846849881279418\\
60.12	0.0084664262532522\\
60.13	0.008464351528618\\
60.14	0.0084622746349834\\
60.15	0.00846019556843092\\
60.16	0.00845811432503373\\
60.17	0.00845603090085566\\
60.18	0.00845394529195113\\
60.19	0.00845185749436518\\
60.2	0.00844976750413337\\
60.21	0.00844767531728182\\
60.22	0.00844558092982714\\
60.23	0.00844348433777644\\
60.24	0.00844138553712726\\
60.25	0.00843928452386759\\
60.26	0.00843718129397581\\
60.27	0.00843507584342067\\
60.28	0.00843296816816127\\
60.29	0.00843085826414703\\
60.3	0.00842874612731766\\
60.31	0.00842663175360315\\
60.32	0.0084245151389237\\
60.33	0.00842239627918974\\
60.34	0.00842027517030188\\
60.35	0.00841815180815088\\
60.36	0.00841602618861764\\
60.37	0.00841389830757314\\
60.38	0.00841176816087845\\
60.39	0.00840963574438468\\
60.4	0.00840750105393294\\
60.41	0.00840536408535435\\
60.42	0.00840322483446999\\
60.43	0.00840108329709084\\
60.44	0.00839893946901783\\
60.45	0.00839679334604173\\
60.46	0.00839464492394318\\
60.47	0.0083924941984926\\
60.48	0.00839034116545025\\
60.49	0.00838818582056611\\
60.5	0.0083860281595799\\
60.51	0.00838386817822106\\
60.52	0.00838170587220867\\
60.53	0.0083795412372515\\
60.54	0.00837737426904789\\
60.55	0.00837520496328579\\
60.56	0.00837303331564269\\
60.57	0.00837085932178563\\
60.58	0.00836868297737112\\
60.59	0.00836650427804515\\
60.6	0.00836432321944315\\
60.61	0.00836213979718995\\
60.62	0.00835995400689975\\
60.63	0.00835776584417611\\
60.64	0.00835557530461191\\
60.65	0.00835338238378929\\
60.66	0.00835118707727968\\
60.67	0.00834898938064372\\
60.68	0.00834678928943123\\
60.69	0.00834458679918121\\
60.7	0.00834238190542179\\
60.71	0.00834017460367021\\
60.72	0.00833796488943277\\
60.73	0.0083357527582048\\
60.74	0.00833353820547067\\
60.75	0.00833132122670369\\
60.76	0.00832910181736615\\
60.77	0.00832687997290923\\
60.78	0.00832465568877301\\
60.79	0.0083224289603864\\
60.8	0.00832019978316715\\
60.81	0.00831796815252179\\
60.82	0.0083157340638456\\
60.83	0.00831349751252259\\
60.84	0.00831125849392546\\
60.85	0.00830901700341555\\
60.86	0.00830677303634287\\
60.87	0.00830452658804598\\
60.88	0.00830227765385202\\
60.89	0.00830002622907665\\
60.9	0.00829777230902404\\
60.91	0.0082955158889868\\
60.92	0.00829325696424599\\
60.93	0.00829099553007105\\
60.94	0.0082887315817198\\
60.95	0.00828646511443837\\
60.96	0.00828419612346119\\
60.97	0.00828192460401097\\
60.98	0.00827965055129863\\
60.99	0.00827737396052328\\
61	0.00827509482687221\\
61.01	0.00827281314552082\\
61.02	0.00827052891163261\\
61.03	0.00826824212035913\\
61.04	0.00826595276683996\\
61.05	0.00826366084620268\\
61.06	0.00826136635356281\\
61.07	0.00825906928402378\\
61.08	0.00825676963267692\\
61.09	0.00825446739460141\\
61.1	0.00825216256486424\\
61.11	0.00824985513852019\\
61.12	0.00824754511061176\\
61.13	0.00824523247616918\\
61.14	0.00824291723021034\\
61.15	0.00824059936774078\\
61.16	0.00823827888375365\\
61.17	0.00823595577322962\\
61.18	0.00823363003113695\\
61.19	0.00823130165243135\\
61.2	0.00822897063205601\\
61.21	0.00822663696494153\\
61.22	0.0082243006460059\\
61.23	0.00822196167015446\\
61.24	0.00821962003227985\\
61.25	0.00821727572726199\\
61.26	0.00821492874996805\\
61.27	0.00821257909525239\\
61.28	0.00821022675795652\\
61.29	0.00820787173290909\\
61.3	0.00820551401492585\\
61.31	0.00820315359880958\\
61.32	0.00820079047935009\\
61.33	0.00819842465132413\\
61.34	0.00819605610949544\\
61.35	0.00819368484861462\\
61.36	0.00819131086341915\\
61.37	0.00818893414863332\\
61.38	0.00818655469896821\\
61.39	0.00818417250912166\\
61.4	0.00818178757377818\\
61.41	0.00817939988760899\\
61.42	0.00817700944527191\\
61.43	0.00817461624141136\\
61.44	0.00817222027065831\\
61.45	0.00816982152763024\\
61.46	0.00816742000693112\\
61.47	0.00816501570315132\\
61.48	0.00816260861086761\\
61.49	0.00816019872464315\\
61.5	0.00815778603902736\\
61.51	0.00815537054855596\\
61.52	0.00815295224775091\\
61.53	0.00815053113112034\\
61.54	0.00814810719315855\\
61.55	0.00814568042834594\\
61.56	0.00814325083114897\\
61.57	0.00814081839602015\\
61.58	0.00813838311739798\\
61.59	0.00813594498970687\\
61.6	0.00813350400735718\\
61.61	0.0081310601647451\\
61.62	0.00812861345625267\\
61.63	0.00812616387624768\\
61.64	0.00812371141908369\\
61.65	0.00812125607909993\\
61.66	0.00811879785062128\\
61.67	0.00811633672795828\\
61.68	0.00811387270540698\\
61.69	0.00811140577724899\\
61.7	0.00810893593775139\\
61.71	0.00810646318116671\\
61.72	0.00810398750173287\\
61.73	0.00810150889367315\\
61.74	0.00809902735119615\\
61.75	0.00809654286849571\\
61.76	0.00809405543975092\\
61.77	0.00809156505912604\\
61.78	0.00808907172077047\\
61.79	0.00808657541881871\\
61.8	0.00808407614739028\\
61.81	0.00808157484985784\\
61.82	0.00807907206291085\\
61.83	0.0080765677852878\\
61.84	0.00807406201572598\\
61.85	0.00807155475296149\\
61.86	0.00806904599572924\\
61.87	0.00806653574276295\\
61.88	0.00806402399279513\\
61.89	0.00806151074455713\\
61.9	0.00805899599677908\\
61.91	0.00805647974818992\\
61.92	0.00805396199751742\\
61.93	0.00805144274348812\\
61.94	0.00804892198482739\\
61.95	0.00804639972025939\\
61.96	0.00804387594850711\\
61.97	0.00804135066829231\\
61.98	0.00803882387833559\\
61.99	0.00803629557735632\\
62	0.0080337657640727\\
62.01	0.00803123443720172\\
62.02	0.00802870159545917\\
62.03	0.00802616723755966\\
62.04	0.00802363136221658\\
62.05	0.00802109396814215\\
62.06	0.00801855505404737\\
62.07	0.00801601461864203\\
62.08	0.00801347266063477\\
62.09	0.00801092917873297\\
62.1	0.00800838417164287\\
62.11	0.00800583763806945\\
62.12	0.00800328957671655\\
62.13	0.00800073998628677\\
62.14	0.00799818886548153\\
62.15	0.00799563621300103\\
62.16	0.0079930820275443\\
62.17	0.00799052630780914\\
62.18	0.00798796905249215\\
62.19	0.00798541026028877\\
62.2	0.00798284992989317\\
62.21	0.0079802880599984\\
62.22	0.00797772464929622\\
62.23	0.00797515969647727\\
62.24	0.00797259320023094\\
62.25	0.00797002515924542\\
62.26	0.00796745557220772\\
62.27	0.00796488443780362\\
62.28	0.00796231175471773\\
62.29	0.00795973752163343\\
62.3	0.00795716173723291\\
62.31	0.00795458440019715\\
62.32	0.00795200550920592\\
62.33	0.00794942506293782\\
62.34	0.00794684306007021\\
62.35	0.00794425949927926\\
62.36	0.00794167437923994\\
62.37	0.00793908769862601\\
62.38	0.00793649945611004\\
62.39	0.00793390965036337\\
62.4	0.00793131828005616\\
62.41	0.00792872534385736\\
62.42	0.0079261308404347\\
62.43	0.00792353476845474\\
62.44	0.0079209371265828\\
62.45	0.00791833791348301\\
62.46	0.00791573712781832\\
62.47	0.00791313476825042\\
62.48	0.00791053083343984\\
62.49	0.00790792532204591\\
62.5	0.00790531823272672\\
62.51	0.00790270956413918\\
62.52	0.007900099314939\\
62.53	0.00789748748378066\\
62.54	0.00789487406931746\\
62.55	0.0078922590702015\\
62.56	0.00788964248508365\\
62.57	0.0078870243126136\\
62.58	0.00788440455143981\\
62.59	0.00788178320020957\\
62.6	0.00787916025756893\\
62.61	0.00787653572216276\\
62.62	0.00787390959263473\\
62.63	0.00787128186762728\\
62.64	0.00786865254578167\\
62.65	0.00786602162573795\\
62.66	0.00786338910613495\\
62.67	0.00786075498561033\\
62.68	0.00785811926280053\\
62.69	0.00785548193634076\\
62.7	0.00785284300486507\\
62.71	0.00785020246700629\\
62.72	0.00784756032139604\\
62.73	0.00784491656666474\\
62.74	0.00784227120144162\\
62.75	0.00783962422435469\\
62.76	0.00783697563403077\\
62.77	0.00783432542909548\\
62.78	0.00783167360817322\\
62.79	0.00782902016988721\\
62.8	0.00782636511285946\\
62.81	0.00782370843571078\\
62.82	0.00782105013706078\\
62.83	0.00781839021552785\\
62.84	0.00781572866972922\\
62.85	0.00781306549828087\\
62.86	0.00781040069979764\\
62.87	0.00780773427289311\\
62.88	0.00780506621617969\\
62.89	0.0078023965282686\\
62.9	0.00779972520776985\\
62.91	0.00779705225329224\\
62.92	0.00779437766344339\\
62.93	0.00779170143682972\\
62.94	0.00778902357205644\\
62.95	0.00778634406772757\\
62.96	0.00778366292244594\\
62.97	0.00778098013481317\\
62.98	0.0077782957034297\\
62.99	0.00777560962689475\\
63	0.00777292190380638\\
63.01	0.00777023253276142\\
63.02	0.00776754151235552\\
63.03	0.00776484884118315\\
63.04	0.00776215451783755\\
63.05	0.00775945854091081\\
63.06	0.0077567609089938\\
63.07	0.00775406162067619\\
63.08	0.00775136067454649\\
63.09	0.00774865806919198\\
63.1	0.00774595380319878\\
63.11	0.00774324787515182\\
63.12	0.00774054028363481\\
63.13	0.00773783102723029\\
63.14	0.0077351201045196\\
63.15	0.00773240751408293\\
63.16	0.00772969325449924\\
63.17	0.00772697732434629\\
63.18	0.00772425972220071\\
63.19	0.0077215404466379\\
63.2	0.00771881949623208\\
63.21	0.00771609686955629\\
63.22	0.00771337256518239\\
63.23	0.00771064658168105\\
63.24	0.00770791891762176\\
63.25	0.00770518957157282\\
63.26	0.00770245854210136\\
63.27	0.00769972582777331\\
63.28	0.00769699142715344\\
63.29	0.00769425533880533\\
63.3	0.00769151756129137\\
63.31	0.0076887780931728\\
63.32	0.00768603693300964\\
63.33	0.00768329407936077\\
63.34	0.00768054953078388\\
63.35	0.00767780328583546\\
63.36	0.00767505534307088\\
63.37	0.00767230570104428\\
63.38	0.00766955435830865\\
63.39	0.00766680131341582\\
63.4	0.00766404656491643\\
63.41	0.00766129011135993\\
63.42	0.00765853195129464\\
63.43	0.00765577208326769\\
63.44	0.00765301050582504\\
63.45	0.00765024721751149\\
63.46	0.00764748221687065\\
63.47	0.007644715502445\\
63.48	0.00764194707277581\\
63.49	0.00763917692640323\\
63.5	0.00763640506186622\\
63.51	0.00763363147770259\\
63.52	0.00763085617244896\\
63.53	0.00762807914464083\\
63.54	0.0076253003928125\\
63.55	0.00762251991549716\\
63.56	0.00761973771122679\\
63.57	0.00761695377853223\\
63.58	0.00761416811594318\\
63.59	0.00761138072198818\\
63.6	0.00760859159519459\\
63.61	0.00760580073408865\\
63.62	0.00760300813719544\\
63.63	0.00760021380303886\\
63.64	0.0075974177301417\\
63.65	0.00759461991702558\\
63.66	0.00759182036221098\\
63.67	0.00758901906421722\\
63.68	0.0075862160215625\\
63.69	0.00758341123276384\\
63.7	0.00758060469633714\\
63.71	0.00757779641079717\\
63.72	0.00757498637465753\\
63.73	0.00757217458643069\\
63.74	0.007569361044628\\
63.75	0.00756654574775964\\
63.76	0.00756372869433468\\
63.77	0.00756090988286104\\
63.78	0.00755808931184553\\
63.79	0.0075552669797938\\
63.8	0.00755244288521037\\
63.81	0.00754961702659866\\
63.82	0.00754678940246092\\
63.83	0.00754396001129832\\
63.84	0.00754112885161085\\
63.85	0.00753829592189743\\
63.86	0.0075354612206558\\
63.87	0.00753262474638265\\
63.88	0.00752978649757348\\
63.89	0.00752694647272272\\
63.9	0.00752410467032365\\
63.91	0.00752126108886845\\
63.92	0.00751841572684819\\
63.93	0.00751556858275282\\
63.94	0.00751271965507117\\
63.95	0.00750986894229098\\
63.96	0.00750701644289885\\
63.97	0.00750416215538033\\
63.98	0.0075013060782198\\
63.99	0.00749844820990057\\
64	0.00749558854890485\\
64.01	0.00749272709371373\\
64.02	0.00748986384280723\\
64.03	0.00748699879466425\\
64.04	0.0074841319477626\\
64.05	0.00748126330057899\\
64.06	0.00747839285158905\\
64.07	0.00747552059926732\\
64.08	0.00747264654208724\\
64.09	0.00746977067852117\\
64.1	0.00746689300704038\\
64.11	0.00746401352611505\\
64.12	0.00746113223421431\\
64.13	0.00745824912980617\\
64.14	0.00745536421135759\\
64.15	0.00745247747733443\\
64.16	0.00744958892620151\\
64.17	0.00744669855642253\\
64.18	0.00744380636646017\\
64.19	0.00744091235477601\\
64.2	0.00743801651983056\\
64.21	0.00743511886008329\\
64.22	0.00743221937399258\\
64.23	0.00742931806001577\\
64.24	0.00742641491660913\\
64.25	0.00742350994222789\\
64.26	0.00742060313532618\\
64.27	0.00741769449435713\\
64.28	0.0074147840177728\\
64.29	0.00741187170402419\\
64.3	0.00740895755156127\\
64.31	0.00740604155883295\\
64.32	0.00740312372428711\\
64.33	0.00740020404637059\\
64.34	0.0073972825235292\\
64.35	0.00739435915420767\\
64.36	0.00739143393684976\\
64.37	0.00738850686989815\\
64.38	0.00738557795179452\\
64.39	0.00738264718097951\\
64.4	0.00737971455589274\\
64.41	0.0073767800749728\\
64.42	0.00737384373665728\\
64.43	0.00737090553938273\\
64.44	0.0073679654815847\\
64.45	0.00736502356169772\\
64.46	0.00736207977815532\\
64.47	0.00735913412939001\\
64.48	0.0073561866138333\\
64.49	0.00735323722991571\\
64.5	0.00735028597606673\\
64.51	0.00734733285071488\\
64.52	0.00734437785228768\\
64.53	0.00734142097921166\\
64.54	0.00733846222991233\\
64.55	0.00733550160281427\\
64.56	0.00733253909634102\\
64.57	0.00732957470891516\\
64.58	0.0073266084389583\\
64.59	0.00732364028489107\\
64.6	0.00732067024513312\\
64.61	0.00731769831810313\\
64.62	0.00731472450221881\\
64.63	0.00731174879589692\\
64.64	0.00730877119755323\\
64.65	0.00730579170560259\\
64.66	0.00730281031845885\\
64.67	0.00729982703453494\\
64.68	0.00729684185224281\\
64.69	0.0072938547699935\\
64.7	0.00729086578619707\\
64.71	0.00728787489926266\\
64.72	0.00728488210759846\\
64.73	0.00728188740961173\\
64.74	0.00727889080370878\\
64.75	0.00727589228829502\\
64.76	0.00727289186177493\\
64.77	0.00726988952255204\\
64.78	0.00726688526902897\\
64.79	0.00726387909960745\\
64.8	0.00726087101268825\\
64.81	0.00725786100667129\\
64.82	0.00725484907995552\\
64.83	0.00725183523093903\\
64.84	0.00724881945801899\\
64.85	0.00724580175959167\\
64.86	0.00724278213405247\\
64.87	0.00723976057979587\\
64.88	0.00723673709521548\\
64.89	0.00723371167870403\\
64.9	0.00723068432865335\\
64.91	0.00722765504345443\\
64.92	0.00722462382149735\\
64.93	0.00722159066117135\\
64.94	0.00721855556086477\\
64.95	0.00721551851896512\\
64.96	0.00721247953385905\\
64.97	0.00720943860393234\\
64.98	0.00720639572756992\\
64.99	0.00720335090315588\\
65	0.00720030412907347\\
65.01	0.0071972554037051\\
65.02	0.00719420472543233\\
65.03	0.00719115209263592\\
65.04	0.00718809750369575\\
65.05	0.00718504095699095\\
65.06	0.00718198245089975\\
65.07	0.00717892198379963\\
65.08	0.00717585955406721\\
65.09	0.00717279516007834\\
65.1	0.00716972880020804\\
65.11	0.00716666047283056\\
65.12	0.0071635901763193\\
65.13	0.00716051790904694\\
65.14	0.00715744366938531\\
65.15	0.0071543674557055\\
65.16	0.00715128926637781\\
65.17	0.00714820909977175\\
65.18	0.00714512695425608\\
65.19	0.00714204282819879\\
65.2	0.00713895671996711\\
65.21	0.0071358686279275\\
65.22	0.00713277855044569\\
65.23	0.00712968648588664\\
65.24	0.0071265924326146\\
65.25	0.00712349638899304\\
65.26	0.00712039835338472\\
65.27	0.00711729832415168\\
65.28	0.00711419629965522\\
65.29	0.00711109227825592\\
65.3	0.00710798625831366\\
65.31	0.00710487823818759\\
65.32	0.00710176821623618\\
65.33	0.00709865619081718\\
65.34	0.00709554216028765\\
65.35	0.00709242612300395\\
65.36	0.00708930807732179\\
65.37	0.00708618802159615\\
65.38	0.00708306595418138\\
65.39	0.00707994187343113\\
65.4	0.0070768157776984\\
65.41	0.00707368766533552\\
65.42	0.00707055753469419\\
65.43	0.00706742538412543\\
65.44	0.00706429121197962\\
65.45	0.00706115501660653\\
65.46	0.00705801679635526\\
65.47	0.00705487654957432\\
65.48	0.00705173427461156\\
65.49	0.00704858996981424\\
65.5	0.007045443633529\\
65.51	0.00704229526410187\\
65.52	0.0070391448598783\\
65.53	0.00703599241920311\\
65.54	0.00703283794042058\\
65.55	0.00702968142187436\\
65.56	0.00702652286190754\\
65.57	0.00702336225886266\\
65.58	0.00702019961108167\\
65.59	0.00701703491690597\\
65.6	0.0070138681746764\\
65.61	0.00701069938273327\\
65.62	0.00700752853941632\\
65.63	0.00700435564306479\\
65.64	0.00700118069201736\\
65.65	0.00699800368461222\\
65.66	0.006994824619187\\
65.67	0.00699164349407887\\
65.68	0.00698846030762446\\
65.69	0.00698527505815992\\
65.7	0.0069820877440209\\
65.71	0.00697889836354257\\
65.72	0.00697570691505962\\
65.73	0.00697251339690628\\
65.74	0.00696931780741632\\
65.75	0.00696612014492301\\
65.76	0.00696292040775922\\
65.77	0.00695971859425735\\
65.78	0.00695651470274936\\
65.79	0.0069533087315668\\
65.8	0.00695010067904079\\
65.81	0.00694689054350201\\
65.82	0.00694367832328076\\
65.83	0.00694046401670693\\
65.84	0.00693724762210999\\
65.85	0.00693402913781908\\
65.86	0.0069308085621629\\
65.87	0.0069275858934698\\
65.88	0.00692436113006779\\
65.89	0.00692113427028446\\
65.9	0.00691790531244712\\
65.91	0.00691467425488268\\
65.92	0.00691144109591775\\
65.93	0.0069082058338786\\
65.94	0.00690496846709117\\
65.95	0.00690172899388112\\
65.96	0.00689848741257376\\
65.97	0.00689524372149414\\
65.98	0.00689199791896699\\
65.99	0.00688875000331681\\
66	0.00688549997286777\\
66.01	0.0068822478259438\\
66.02	0.00687899356086859\\
66.03	0.00687573717596555\\
66.04	0.00687247866955789\\
66.05	0.00686921803996854\\
66.06	0.00686595528552026\\
66.07	0.00686269040453555\\
66.08	0.00685942339533673\\
66.09	0.00685615425624593\\
66.1	0.00685288298558506\\
66.11	0.00684960958167589\\
66.12	0.00684633404283998\\
66.13	0.00684305636739876\\
66.14	0.0068397765536735\\
66.15	0.00683649459998532\\
66.16	0.0068332105046552\\
66.17	0.00682992426600402\\
66.18	0.00682663588235252\\
66.19	0.00682334535202134\\
66.2	0.00682005267333102\\
66.21	0.00681675784460203\\
66.22	0.00681346086415474\\
66.23	0.00681016173030947\\
66.24	0.00680686044138645\\
66.25	0.00680355699570591\\
66.26	0.006800251391588\\
66.27	0.00679694362735287\\
66.28	0.00679363370132063\\
66.29	0.00679032161181137\\
66.3	0.00678700735714523\\
66.31	0.00678369093564231\\
66.32	0.00678037234562276\\
66.33	0.00677705158540675\\
66.34	0.0067737286533145\\
66.35	0.0067704035476663\\
66.36	0.00676707626678246\\
66.37	0.00676374680898341\\
66.38	0.00676041517258963\\
66.39	0.00675708135592172\\
66.4	0.00675374535730038\\
66.41	0.00675040717504643\\
66.42	0.00674706680748081\\
66.43	0.00674372425292463\\
66.44	0.00674037950969911\\
66.45	0.00673703257612566\\
66.46	0.00673368345052588\\
66.47	0.00673033213122151\\
66.48	0.00672697861653453\\
66.49	0.00672362290478712\\
66.5	0.00672026499430167\\
66.51	0.00671690488340083\\
66.52	0.00671354257040746\\
66.53	0.0067101780536447\\
66.54	0.00670681133143597\\
66.55	0.00670344240210497\\
66.56	0.00670007126397566\\
66.57	0.00669669791537236\\
66.58	0.00669332235461967\\
66.59	0.00668994458004256\\
66.6	0.0066865645899663\\
66.61	0.00668318238271657\\
66.62	0.00667979795661939\\
66.63	0.00667641131000116\\
66.64	0.00667302244118872\\
66.65	0.00666963134850927\\
66.66	0.00666623803029048\\
66.67	0.00666284248486043\\
66.68	0.00665944471054766\\
66.69	0.00665604470568119\\
66.7	0.00665264246859051\\
66.71	0.0066492379976056\\
66.72	0.00664583129105695\\
66.73	0.00664242234727559\\
66.74	0.00663901116459307\\
66.75	0.00663559774134151\\
66.76	0.00663218207585357\\
66.77	0.0066287641664625\\
66.78	0.00662534401150218\\
66.79	0.00662192160930704\\
66.8	0.0066184969582122\\
66.81	0.00661507005655338\\
66.82	0.00661164090266697\\
66.83	0.00660820949489003\\
66.84	0.00660477583156031\\
66.85	0.00660133991101627\\
66.86	0.00659790173159708\\
66.87	0.00659446129164264\\
66.88	0.00659101858949362\\
66.89	0.00658757362349146\\
66.9	0.00658412639197835\\
66.91	0.00658067689329733\\
66.92	0.00657722512579221\\
66.93	0.00657377108780767\\
66.94	0.00657031477768922\\
66.95	0.00656685619378325\\
66.96	0.00656339533443705\\
66.97	0.00655993219799877\\
66.98	0.00655646678281751\\
66.99	0.00655299908724331\\
67	0.00654952910962717\\
67.01	0.00654605684832105\\
67.02	0.00654258230167789\\
67.03	0.00653910546805168\\
67.04	0.00653562634579741\\
67.05	0.00653214493327112\\
67.06	0.00652866122882992\\
67.07	0.006525175230832\\
67.08	0.00652168693763667\\
67.09	0.00651819634760435\\
67.1	0.00651470345909661\\
67.11	0.00651120827047617\\
67.12	0.00650771078010694\\
67.13	0.00650421098635403\\
67.14	0.00650070888758378\\
67.15	0.00649720448216376\\
67.16	0.00649369776846283\\
67.17	0.00649018874485108\\
67.18	0.00648667740969994\\
67.19	0.00648316376138218\\
67.2	0.0064796477982719\\
67.21	0.00647612951874454\\
67.22	0.00647260892117696\\
67.23	0.00646908600394743\\
67.24	0.00646556076543565\\
67.25	0.00646203320402275\\
67.26	0.00645850331809138\\
67.27	0.00645497110602566\\
67.28	0.00645143656621123\\
67.29	0.00644789969703527\\
67.3	0.00644436049688657\\
67.31	0.00644081896415546\\
67.32	0.00643727509723392\\
67.33	0.00643372889451554\\
67.34	0.00643018035439559\\
67.35	0.00642662947527101\\
67.36	0.00642307625554048\\
67.37	0.00641952069360439\\
67.38	0.00641596278786489\\
67.39	0.00641240253672592\\
67.4	0.00640883993859324\\
67.41	0.00640527499187442\\
67.42	0.00640170769497892\\
67.43	0.00639813804631806\\
67.44	0.0063945660443051\\
67.45	0.0063909916873552\\
67.46	0.00638741497388553\\
67.47	0.00638383590231523\\
67.48	0.00638025447106545\\
67.49	0.00637667067855941\\
67.5	0.00637308452322239\\
67.51	0.00636949600348176\\
67.52	0.00636590511776706\\
67.53	0.00636231186450995\\
67.54	0.00635871624214429\\
67.55	0.00635511824910616\\
67.56	0.00635151788383387\\
67.57	0.00634791514476804\\
67.58	0.00634431003035155\\
67.59	0.00634070253902964\\
67.6	0.00633709266924992\\
67.61	0.00633348041946238\\
67.62	0.00632986578811943\\
67.63	0.00632624877367595\\
67.64	0.00632262937458931\\
67.65	0.00631900758931938\\
67.66	0.00631538341632861\\
67.67	0.00631175685408202\\
67.68	0.00630812790104724\\
67.69	0.00630449655569455\\
67.7	0.00630086281649694\\
67.71	0.00629722668193006\\
67.72	0.00629358815047237\\
67.73	0.00628994722060508\\
67.74	0.00628630389081222\\
67.75	0.00628265815958069\\
67.76	0.00627901002540026\\
67.77	0.00627535948676363\\
67.78	0.00627170654216646\\
67.79	0.00626805119010741\\
67.8	0.00626439342908817\\
67.81	0.00626073325761349\\
67.82	0.00625707067419124\\
67.83	0.00625340567733242\\
67.84	0.00624973826555122\\
67.85	0.00624606843736506\\
67.86	0.0062423961912946\\
67.87	0.0062387215258638\\
67.88	0.00623504443959997\\
67.89	0.00623136493103378\\
67.9	0.00622768299869933\\
67.91	0.00622399864113417\\
67.92	0.00622031185687936\\
67.93	0.00621662264447949\\
67.94	0.00621293100248274\\
67.95	0.00620923692944089\\
67.96	0.00620554042390942\\
67.97	0.00620184148444751\\
67.98	0.00619814010961807\\
67.99	0.00619443629798783\\
68	0.00619073004812737\\
68.01	0.00618702135861113\\
68.02	0.0061833102280175\\
68.03	0.00617959665492885\\
68.04	0.00617588063793155\\
68.05	0.00617216217561606\\
68.06	0.00616844126657697\\
68.07	0.006164717909413\\
68.08	0.00616099210272711\\
68.09	0.0061572638451265\\
68.1	0.00615353313522272\\
68.11	0.00614979997163163\\
68.12	0.00614606435297353\\
68.13	0.00614232627787318\\
68.14	0.00613858574495984\\
68.15	0.00613484275286734\\
68.16	0.00613109730023412\\
68.17	0.00612734938570331\\
68.18	0.00612359900792274\\
68.19	0.00611984616554502\\
68.2	0.00611609085722758\\
68.21	0.00611233308163276\\
68.22	0.00610857283742781\\
68.23	0.006104810123285\\
68.24	0.00610104493788165\\
68.25	0.00609727727990017\\
68.26	0.00609350714802816\\
68.27	0.00608973454095842\\
68.28	0.00608595945738908\\
68.29	0.00608218189602357\\
68.3	0.00607840185557073\\
68.31	0.00607461933474491\\
68.32	0.00607083433226594\\
68.33	0.00606704684685925\\
68.34	0.00606325687725594\\
68.35	0.0060594644221928\\
68.36	0.00605566948041242\\
68.37	0.00605187205066325\\
68.38	0.0060480721316996\\
68.39	0.00604426972228181\\
68.4	0.00604046482117625\\
68.41	0.00603665742715537\\
68.42	0.00603284753899783\\
68.43	0.00602903515548855\\
68.44	0.00602522027541874\\
68.45	0.00602140289758603\\
68.46	0.00601758302079448\\
68.47	0.00601376064385471\\
68.48	0.00600993576558394\\
68.49	0.00600610838480606\\
68.5	0.00600227850035173\\
68.51	0.00599844611105844\\
68.52	0.00599461121577056\\
68.53	0.00599077381333949\\
68.54	0.00598693390262365\\
68.55	0.00598309148248863\\
68.56	0.00597924655180724\\
68.57	0.00597539910945955\\
68.58	0.00597154915433307\\
68.59	0.00596769668532274\\
68.6	0.00596384170133105\\
68.61	0.00595998420126813\\
68.62	0.00595612418405184\\
68.63	0.00595226164860782\\
68.64	0.0059483965938696\\
68.65	0.00594452901877871\\
68.66	0.00594065892228473\\
68.67	0.00593678630334542\\
68.68	0.00593291116092673\\
68.69	0.00592903349400304\\
68.7	0.00592515330155708\\
68.71	0.00592127058258015\\
68.72	0.00591738533607217\\
68.73	0.00591349756104176\\
68.74	0.00590960725650639\\
68.75	0.0059057144214924\\
68.76	0.00590181905503518\\
68.77	0.00589792115617921\\
68.78	0.00589402072397822\\
68.79	0.0058901177574952\\
68.8	0.00588621225580262\\
68.81	0.00588230421798246\\
68.82	0.00587839364312633\\
68.83	0.00587448053033558\\
68.84	0.00587056487872142\\
68.85	0.00586664668740502\\
68.86	0.00586272595551762\\
68.87	0.00585880268220065\\
68.88	0.00585487686660584\\
68.89	0.00585094850789534\\
68.9	0.00584701760524182\\
68.91	0.00584308415782859\\
68.92	0.00583914816484979\\
68.93	0.00583520962551036\\
68.94	0.00583126853902634\\
68.95	0.00582732490462485\\
68.96	0.0058233787215443\\
68.97	0.00581942998903446\\
68.98	0.00581547870635666\\
68.99	0.00581152487278383\\
69	0.00580756848760071\\
69.01	0.00580360955010393\\
69.02	0.00579964805948012\\
69.03	0.00579568401403238\\
69.04	0.00579171741206535\\
69.05	0.00578774825188526\\
69.06	0.00578377653179986\\
69.07	0.00577980225011854\\
69.08	0.0057758254051523\\
69.09	0.00577184599521378\\
69.1	0.00576786401861729\\
69.11	0.0057638794736788\\
69.12	0.00575989235871601\\
69.13	0.00575590267204832\\
69.14	0.0057519104119969\\
69.15	0.00574791557688468\\
69.16	0.00574391816503638\\
69.17	0.00573991817477853\\
69.18	0.00573591560443951\\
69.19	0.00573191045234954\\
69.2	0.00572790271684074\\
69.21	0.00572389239624712\\
69.22	0.00571987948890463\\
69.23	0.00571586399315115\\
69.24	0.00571184590732657\\
69.25	0.00570782522977273\\
69.26	0.00570380195883353\\
69.27	0.0056997760928549\\
69.28	0.00569574763018485\\
69.29	0.00569171656917347\\
69.3	0.00568768290817298\\
69.31	0.00568364664553775\\
69.32	0.00567960777962429\\
69.33	0.00567556630879134\\
69.34	0.00567152223139984\\
69.35	0.00566747554581297\\
69.36	0.00566342625039621\\
69.37	0.00565937434351729\\
69.38	0.0056553198235463\\
69.39	0.00565126268885568\\
69.4	0.00564720293782021\\
69.41	0.00564314056881713\\
69.42	0.00563907570087051\\
69.43	0.00563500833763439\\
69.44	0.00563093847603353\\
69.45	0.00562686611298631\\
69.46	0.00562279124540478\\
69.47	0.00561871387019461\\
69.48	0.00561463398425509\\
69.49	0.00561055158447911\\
69.5	0.00560646666775315\\
69.51	0.0056023792309573\\
69.52	0.00559828927096517\\
69.53	0.00559419678464398\\
69.54	0.00559010176885448\\
69.55	0.00558600422045094\\
69.56	0.00558190413628117\\
69.57	0.00557780151318651\\
69.58	0.00557369634800177\\
69.59	0.00556958863755528\\
69.6	0.00556547837866883\\
69.61	0.0055613655681577\\
69.62	0.00555725020283062\\
69.63	0.00555313227948976\\
69.64	0.00554901179493074\\
69.65	0.0055448887459426\\
69.66	0.00554076312930781\\
69.67	0.00553663494180224\\
69.68	0.00553250418019514\\
69.69	0.00552837084124918\\
69.7	0.00552423492172037\\
69.71	0.00552009641835812\\
69.72	0.00551595532790517\\
69.73	0.00551181164709762\\
69.74	0.00550766537266491\\
69.75	0.0055035165013298\\
69.76	0.00549936502980836\\
69.77	0.00549521095480999\\
69.78	0.00549105427303738\\
69.79	0.00548689498118652\\
69.8	0.00548273307594665\\
69.81	0.00547856855400032\\
69.82	0.00547440141202332\\
69.83	0.00547023164668472\\
69.84	0.00546605925464681\\
69.85	0.00546188423256514\\
69.86	0.00545770657708847\\
69.87	0.00545352628485882\\
69.88	0.00544934335251137\\
69.89	0.00544515777667457\\
69.9	0.00544096955397002\\
69.91	0.00543677868101254\\
69.92	0.00543258515441012\\
69.93	0.00542838897076393\\
69.94	0.00542419012666834\\
69.95	0.00541998861871082\\
69.96	0.00541578444347208\\
69.97	0.00541157759752592\\
69.98	0.00540736807743931\\
69.99	0.00540315587977235\\
70	0.00539894100107829\\
70.01	0.00539472343790348\\
70.02	0.00539050318678742\\
70.03	0.00538628024426271\\
70.04	0.00538205460685507\\
70.05	0.00537782627108332\\
70.06	0.0053735952334594\\
70.07	0.00536936149048832\\
70.08	0.0053651250386682\\
70.09	0.00536088587449026\\
70.1	0.00535664399443878\\
70.11	0.00535239939499114\\
70.12	0.0053481520726178\\
70.13	0.00534390202378228\\
70.14	0.00533964924494119\\
70.15	0.00533539373254419\\
70.16	0.00533113548303403\\
70.17	0.00532687449284651\\
70.18	0.00532261075841051\\
70.19	0.00531834427614795\\
70.2	0.00531407504247383\\
70.21	0.00530980305379619\\
70.22	0.00530552830651615\\
70.23	0.00530125079702786\\
70.24	0.00529697052171856\\
70.25	0.00529268747696852\\
70.26	0.0052884016591511\\
70.27	0.00528411306463267\\
70.28	0.00527982168977271\\
70.29	0.00527552753092373\\
70.3	0.00527123058443131\\
70.31	0.0052669308466341\\
70.32	0.0052626283138638\\
70.33	0.0052583229824452\\
70.34	0.00525401484869614\\
70.35	0.00524970390892756\\
70.36	0.00524539015944345\\
70.37	0.0052410735965409\\
70.38	0.00523675421651009\\
70.39	0.00523243201563427\\
70.4	0.0052281069901898\\
70.41	0.00522377913644615\\
70.42	0.00521944845066587\\
70.43	0.00521511492910466\\
70.44	0.00521077856801131\\
70.45	0.00520643936362774\\
70.46	0.00520209731218902\\
70.47	0.00519775240992334\\
70.48	0.00519340465305207\\
70.49	0.00518905403778971\\
70.5	0.00518470056034393\\
70.51	0.0051803442169156\\
70.52	0.00517598500369875\\
70.53	0.00517162291688062\\
70.54	0.00516725795264167\\
70.55	0.00516289010715556\\
70.56	0.00515851937658918\\
70.57	0.0051541457571027\\
70.58	0.00514976924484949\\
70.59	0.00514538983597625\\
70.6	0.00514100752662291\\
70.61	0.00513662231292274\\
70.62	0.00513223419100231\\
70.63	0.0051278431569815\\
70.64	0.00512344920697357\\
70.65	0.00511905233708512\\
70.66	0.00511465254341613\\
70.67	0.00511024982205997\\
70.68	0.00510584416910346\\
70.69	0.0051014355806268\\
70.7	0.0050970240527037\\
70.71	0.00509260958140131\\
70.72	0.00508819216278027\\
70.73	0.00508377179289476\\
70.74	0.00507934846779247\\
70.75	0.0050749221835147\\
70.76	0.00507049293609628\\
70.77	0.00506606072156568\\
70.78	0.00506162553594498\\
70.79	0.00505718737524995\\
70.8	0.00505274623549003\\
70.81	0.00504830211266836\\
70.82	0.00504385500278185\\
70.83	0.00503940490182115\\
70.84	0.00503495180577072\\
70.85	0.00503049571060884\\
70.86	0.00502603661230767\\
70.87	0.00502157450683323\\
70.88	0.0050171093901455\\
70.89	0.00501264125819837\\
70.9	0.00500817010693976\\
70.91	0.00500369593231159\\
70.92	0.00499921873024986\\
70.93	0.00499473849668464\\
70.94	0.00499025522754017\\
70.95	0.00498576891873483\\
70.96	0.00498127956618124\\
70.97	0.00497678716578625\\
70.98	0.00497229171345102\\
70.99	0.00496779320507105\\
71	0.00496329163653619\\
71.01	0.00495878700373076\\
71.02	0.0049542793025335\\
71.03	0.0049497685288177\\
71.04	0.00494525467845121\\
71.05	0.00494073774729647\\
71.06	0.0049362177312106\\
71.07	0.00493169462604542\\
71.08	0.00492716842764751\\
71.09	0.00492263913185829\\
71.1	0.00491810673451401\\
71.11	0.00491357123144588\\
71.12	0.00490903261848006\\
71.13	0.00490449089143779\\
71.14	0.00489994604613535\\
71.15	0.00489539807838421\\
71.16	0.00489084698399106\\
71.17	0.00488629275875786\\
71.18	0.00488173539848191\\
71.19	0.00487717489895593\\
71.2	0.00487261125596811\\
71.21	0.00486804446530217\\
71.22	0.00486347452273746\\
71.23	0.004858901424049\\
71.24	0.00485432516500757\\
71.25	0.00484974574137978\\
71.26	0.00484516314892812\\
71.27	0.00484057738341108\\
71.28	0.00483598844058321\\
71.29	0.00483139631619518\\
71.3	0.00482680100599388\\
71.31	0.00482220250572249\\
71.32	0.00481760081112059\\
71.33	0.00481299591792421\\
71.34	0.00480838782186597\\
71.35	0.00480377651867507\\
71.36	0.00479916200407751\\
71.37	0.00479454427379608\\
71.38	0.0047899233235505\\
71.39	0.00478529914905751\\
71.4	0.00478067174603098\\
71.41	0.00477604111018195\\
71.42	0.00477140723721882\\
71.43	0.00476677012284739\\
71.44	0.00476212976277098\\
71.45	0.00475748615269057\\
71.46	0.00475283928830483\\
71.47	0.00474818916531033\\
71.48	0.0047435357794016\\
71.49	0.00473887912627121\\
71.5	0.00473421920160997\\
71.51	0.00472955600110699\\
71.52	0.00472488952044982\\
71.53	0.00472021975532458\\
71.54	0.00471554670141608\\
71.55	0.00471087035440792\\
71.56	0.00470619070998268\\
71.57	0.00470150776382202\\
71.58	0.00469682151160679\\
71.59	0.00469213194901722\\
71.6	0.00468743907173303\\
71.61	0.00468274287543357\\
71.62	0.00467804335579797\\
71.63	0.00467341792590884\\
71.64	0.00467128668778365\\
71.65	0.00466915529691223\\
71.66	0.00466702375563885\\
71.67	0.00466489206631688\\
71.68	0.00466276023130885\\
71.69	0.00466062825298646\\
71.7	0.00465849613373064\\
71.71	0.00465636387593153\\
71.72	0.00465423148198858\\
71.73	0.00465209895431056\\
71.74	0.00464996629531557\\
71.75	0.00464783350743109\\
71.76	0.00464570059309402\\
71.77	0.0046435675547507\\
71.78	0.00464143439485698\\
71.79	0.00463930111587822\\
71.8	0.00463716772028931\\
71.81	0.00463503421057475\\
71.82	0.00463290058922867\\
71.83	0.00463076685875483\\
71.84	0.0046286330216667\\
71.85	0.00462649908048751\\
71.86	0.00462436503775019\\
71.87	0.00462223089599753\\
71.88	0.00462009665778211\\
71.89	0.00461796232566642\\
71.9	0.00461582790222284\\
71.91	0.0046136933900337\\
71.92	0.0046115587916913\\
71.93	0.00460942410979798\\
71.94	0.00460728934696612\\
71.95	0.00460515450581818\\
71.96	0.00460301958898678\\
71.97	0.00460088459911469\\
71.98	0.00459874953885488\\
71.99	0.00459661441087056\\
72	0.00459447921783524\\
72.01	0.00459234396243271\\
72.02	0.00459020864735715\\
72.03	0.00458807327531313\\
72.04	0.00458593784901561\\
72.05	0.0045838023711901\\
72.06	0.00458166684457253\\
72.07	0.00457953127190945\\
72.08	0.00457739565595794\\
72.09	0.00457525999948576\\
72.1	0.0045731243052713\\
72.11	0.00457098857610366\\
72.12	0.00456885281478271\\
72.13	0.00456671702411907\\
72.14	0.00456458120693422\\
72.15	0.00456244536606047\\
72.16	0.00456030950434106\\
72.17	0.00455817362463018\\
72.18	0.00455603772979301\\
72.19	0.00455390182270572\\
72.2	0.0045517659062556\\
72.21	0.00454962998334104\\
72.22	0.00454749405687155\\
72.23	0.00454535812976788\\
72.24	0.00454322220496199\\
72.25	0.00454108628539712\\
72.26	0.00453895037402785\\
72.27	0.0045368144738201\\
72.28	0.00453467858775122\\
72.29	0.00453254271881\\
72.3	0.00453040686999672\\
72.31	0.00452827104432318\\
72.32	0.00452613524481279\\
72.33	0.00452399947450059\\
72.34	0.00452186373643324\\
72.35	0.00451972803366915\\
72.36	0.00451759236927849\\
72.37	0.0045154567463432\\
72.38	0.00451332116795708\\
72.39	0.00451118563722584\\
72.4	0.00450905015726709\\
72.41	0.00450691473121043\\
72.42	0.00450477936219753\\
72.43	0.00450264405338206\\
72.44	0.00450050880792986\\
72.45	0.00449837362901892\\
72.46	0.00449623851983943\\
72.47	0.00449410348359385\\
72.48	0.00449196852349695\\
72.49	0.00448983364277582\\
72.5	0.00448769884466999\\
72.51	0.00448556413243142\\
72.52	0.00448342950932453\\
72.53	0.00448129497862634\\
72.54	0.00447916054362643\\
72.55	0.00447702620762701\\
72.56	0.004474891973943\\
72.57	0.00447275784590204\\
72.58	0.00447062382684455\\
72.59	0.00446848992012382\\
72.6	0.00446635612910598\\
72.61	0.00446422245717012\\
72.62	0.0044620889077083\\
72.63	0.00445995548412564\\
72.64	0.00445782218984031\\
72.65	0.00445568902828364\\
72.66	0.00445355600290015\\
72.67	0.00445142311714759\\
72.68	0.004449290374497\\
72.69	0.00444715777843277\\
72.7	0.00444502533245266\\
72.71	0.00444289304006792\\
72.72	0.00444076090480326\\
72.73	0.00443862893019696\\
72.74	0.0044364971198009\\
72.75	0.00443436547718064\\
72.76	0.00443223400591542\\
72.77	0.00443010270959826\\
72.78	0.00442797159183601\\
72.79	0.00442584065624939\\
72.8	0.00442370990647304\\
72.81	0.00442157934615561\\
72.82	0.00441944897895976\\
72.83	0.00441731880856228\\
72.84	0.00441518883865408\\
72.85	0.00441305907294031\\
72.86	0.00441092951514035\\
72.87	0.00440880016898793\\
72.88	0.00440667103823116\\
72.89	0.00440454212663257\\
72.9	0.00440241343796919\\
72.91	0.00440028497603263\\
72.92	0.00439815674462904\\
72.93	0.00439602874757933\\
72.94	0.00439390098871906\\
72.95	0.00439177347189864\\
72.96	0.00438964620098327\\
72.97	0.00438751917985312\\
72.98	0.00438539241240327\\
72.99	0.00438326590254385\\
73	0.0043811396542001\\
73.01	0.00437901367131237\\
73.02	0.00437688795783625\\
73.03	0.00437476251774259\\
73.04	0.00437263735501759\\
73.05	0.00437051247366284\\
73.06	0.00436838787769538\\
73.07	0.0043662635711478\\
73.08	0.00436413955806823\\
73.09	0.00436201584252052\\
73.1	0.00435989242858418\\
73.11	0.00435776932035452\\
73.12	0.0043556465219427\\
73.13	0.00435352403747579\\
73.14	0.00435140187109682\\
73.15	0.00434928002696488\\
73.16	0.00434715850925517\\
73.17	0.00434503732215906\\
73.18	0.00434291646988416\\
73.19	0.00434079595665438\\
73.2	0.00433867578671005\\
73.21	0.0043365559643079\\
73.22	0.0043344364937212\\
73.23	0.0043323173792398\\
73.24	0.00433019862517021\\
73.25	0.00432808023583565\\
73.26	0.00432596221557614\\
73.27	0.00432384456874859\\
73.28	0.0043217272997268\\
73.29	0.00431961041290162\\
73.3	0.00431749391268095\\
73.31	0.00431537780348987\\
73.32	0.00431326208977066\\
73.33	0.00431114677598291\\
73.34	0.00430903186660359\\
73.35	0.0043069173661271\\
73.36	0.00430480327906538\\
73.37	0.00430268960994795\\
73.38	0.004300576363322\\
73.39	0.00429846354375249\\
73.4	0.00429635115582218\\
73.41	0.00429423920413174\\
73.42	0.00429212769329981\\
73.43	0.00429001662796311\\
73.44	0.00428790601277647\\
73.45	0.00428579585241294\\
73.46	0.00428368615156386\\
73.47	0.00428157691493896\\
73.48	0.00427946814726639\\
73.49	0.00427735985329286\\
73.5	0.00427525203778367\\
73.51	0.00427314470552285\\
73.52	0.00427103786131317\\
73.53	0.00426893150997629\\
73.54	0.00426682565635281\\
73.55	0.00426472030530233\\
73.56	0.0042626154617036\\
73.57	0.00426051113045454\\
73.58	0.00425840731647237\\
73.59	0.00425630402469367\\
73.6	0.00425420126007448\\
73.61	0.00425209902759037\\
73.62	0.00424999733223656\\
73.63	0.00424789617902797\\
73.64	0.00424579557299933\\
73.65	0.00424369551920528\\
73.66	0.00424159602272043\\
73.67	0.00423949708863947\\
73.68	0.00423739872207726\\
73.69	0.00423530092816894\\
73.7	0.00423320371206996\\
73.71	0.00423110707895625\\
73.72	0.00422901103402426\\
73.73	0.00422691558249108\\
73.74	0.00422482072959452\\
73.75	0.00422272648059324\\
73.76	0.00422063284076678\\
73.77	0.00421853981541572\\
73.78	0.00421644740986177\\
73.79	0.0042143556294478\\
73.8	0.00421226447953803\\
73.81	0.0042101739655181\\
73.82	0.00420808409279512\\
73.83	0.00420599486679785\\
73.84	0.00420390629297674\\
73.85	0.00420181837680406\\
73.86	0.004199731123774\\
73.87	0.00419764453940277\\
73.88	0.00419555862922871\\
73.89	0.0041934733988124\\
73.9	0.00419138885373673\\
73.91	0.00418930499960705\\
73.92	0.00418722184205127\\
73.93	0.00418513938671994\\
73.94	0.00418305763928641\\
73.95	0.00418097660544688\\
73.96	0.00417889629092055\\
73.97	0.00417681670144971\\
73.98	0.00417473784279988\\
73.99	0.0041726597207599\\
74	0.00417058234114195\\
74.01	0.00416850570978012\\
74.02	0.00416642983253049\\
74.03	0.00416435471527117\\
74.04	0.00416228036390246\\
74.05	0.00416020678434687\\
74.06	0.00415813398254925\\
74.07	0.00415606196447683\\
74.08	0.00415399073611938\\
74.09	0.0041519203034892\\
74.1	0.00414985067262129\\
74.11	0.00414778184957342\\
74.12	0.00414571384042615\\
74.13	0.00414364665128303\\
74.14	0.00414158028827061\\
74.15	0.00413951475753853\\
74.16	0.00413745006525966\\
74.17	0.00413538621763017\\
74.18	0.00413332322086957\\
74.19	0.00413126108122088\\
74.2	0.00412919980495067\\
74.21	0.00412713939834917\\
74.22	0.00412507986773036\\
74.23	0.00412302121943206\\
74.24	0.00412096345981602\\
74.25	0.00411890659526803\\
74.26	0.004116850632198\\
74.27	0.00411479557704006\\
74.28	0.00411274143625263\\
74.29	0.00411068821631858\\
74.3	0.00410863592374524\\
74.31	0.00410658456506459\\
74.32	0.00410453414683328\\
74.33	0.00410248467563273\\
74.34	0.0041004361580693\\
74.35	0.00409838860077434\\
74.36	0.00409634201040425\\
74.37	0.00409429639364065\\
74.38	0.0040922517571904\\
74.39	0.00409020810778572\\
74.4	0.00408816545218431\\
74.41	0.00408612379716928\\
74.42	0.00408408314954874\\
74.43	0.00408204351615586\\
74.44	0.004080004903849\\
74.45	0.00407796731951173\\
74.46	0.004075930770053\\
74.47	0.00407389526240719\\
74.48	0.00407186080353419\\
74.49	0.00406982740041954\\
74.5	0.00406779506007448\\
74.51	0.00406576378953605\\
74.52	0.0040637335958672\\
74.53	0.00406170448615688\\
74.54	0.00405967646752011\\
74.55	0.00405764954709812\\
74.56	0.00405562373205838\\
74.57	0.00405359902959477\\
74.58	0.00405157544692763\\
74.59	0.00404955299130384\\
74.6	0.00404753166999697\\
74.61	0.00404551149030732\\
74.62	0.00404349245956209\\
74.63	0.00404147458511538\\
74.64	0.00403945787434837\\
74.65	0.00403744233466937\\
74.66	0.00403542797351396\\
74.67	0.00403341479834504\\
74.68	0.00403140281665297\\
74.69	0.00402939203595564\\
74.7	0.0040273824637986\\
74.71	0.00402537410775514\\
74.72	0.00402336697542637\\
74.73	0.00402136107444139\\
74.74	0.00401935641245732\\
74.75	0.00401735299715942\\
74.76	0.00401535083626123\\
74.77	0.00401334993750462\\
74.78	0.00401135030865992\\
74.79	0.00400935195752604\\
74.8	0.00400735489193054\\
74.81	0.00400535911972974\\
74.82	0.00400336464880885\\
74.83	0.00400137148708205\\
74.84	0.00399937964249261\\
74.85	0.00399738912301296\\
74.86	0.00399539993664488\\
74.87	0.00399341209141951\\
74.88	0.0039914255953975\\
74.89	0.00398944045666915\\
74.9	0.00398745668335444\\
74.91	0.00398547428360322\\
74.92	0.00398349326559525\\
74.93	0.00398151363754038\\
74.94	0.00397953540767857\\
74.95	0.00397755858428008\\
74.96	0.00397558317564556\\
74.97	0.00397360919010613\\
74.98	0.00397163663602351\\
74.99	0.00396966552179014\\
75	0.0039676958558293\\
75.01	0.00396572764659518\\
75.02	0.00396376090257304\\
75.03	0.00396179563227929\\
75.04	0.00395983184426163\\
75.05	0.00395786954709915\\
75.06	0.00395590874940243\\
75.07	0.00395394945981369\\
75.08	0.00395199168700689\\
75.09	0.00395003543968782\\
75.1	0.00394808072659426\\
75.11	0.00394612755649605\\
75.12	0.00394417593819526\\
75.13	0.00394222588052626\\
75.14	0.00394027739235588\\
75.15	0.00393833048258348\\
75.16	0.00393638516014111\\
75.17	0.00393444143399362\\
75.18	0.00393249931313876\\
75.19	0.00393055880660734\\
75.2	0.00392861992346328\\
75.21	0.00392668267280385\\
75.22	0.00392474706375964\\
75.23	0.00392281310549481\\
75.24	0.00392088080720715\\
75.25	0.00391895017812823\\
75.26	0.00391702122752348\\
75.27	0.00391509396469238\\
75.28	0.00391316839896853\\
75.29	0.00391124453971979\\
75.3	0.00390932239634843\\
75.31	0.0039074019782912\\
75.32	0.00390548329501953\\
75.33	0.00390356635603961\\
75.34	0.00390165117089251\\
75.35	0.00389973774915433\\
75.36	0.00389782610043634\\
75.37	0.00389591623438506\\
75.38	0.00389400816068246\\
75.39	0.00389210188904601\\
75.4	0.00389019742922889\\
75.41	0.00388829479102005\\
75.42	0.00388639398424439\\
75.43	0.00388449501876288\\
75.44	0.00388259790447266\\
75.45	0.00388070265130724\\
75.46	0.00387880926923656\\
75.47	0.00387691776826717\\
75.48	0.00387502815844237\\
75.49	0.00387314044984229\\
75.5	0.00387125465258409\\
75.51	0.00386937077682206\\
75.52	0.00386748883274776\\
75.53	0.00386560883059018\\
75.54	0.00386373078061583\\
75.55	0.00386185469312894\\
75.56	0.00385998057847155\\
75.57	0.00385810844702364\\
75.58	0.00385623830920335\\
75.59	0.00385437017546702\\
75.6	0.00385250405630938\\
75.61	0.00385063996226371\\
75.62	0.00384877790390193\\
75.63	0.00384691789183481\\
75.64	0.00384505993671202\\
75.65	0.00384320404922239\\
75.66	0.00384135024009392\\
75.67	0.00383949852009406\\
75.68	0.00383764890002976\\
75.69	0.00383580139074765\\
75.7	0.00383395600313417\\
75.71	0.00383211274811575\\
75.72	0.00383027163665894\\
75.73	0.00382843267977053\\
75.74	0.00382659588849774\\
75.75	0.00382476127392835\\
75.76	0.00382292884719084\\
75.77	0.00382109861945459\\
75.78	0.00381927060192993\\
75.79	0.00381744480586841\\
75.8	0.00381562124256286\\
75.81	0.00381379992334762\\
75.82	0.00381198085959859\\
75.83	0.0038101640627335\\
75.84	0.00380834954421198\\
75.85	0.00380653731553576\\
75.86	0.00380472738824879\\
75.87	0.00380291977393745\\
75.88	0.00380111448423063\\
75.89	0.00379931153079996\\
75.9	0.00379751092535992\\
75.91	0.00379571267966803\\
75.92	0.00379391680552499\\
75.93	0.00379212331477484\\
75.94	0.00379033221930514\\
75.95	0.00378854099534318\\
75.96	0.0037867495040187\\
75.97	0.00378495774595762\\
75.98	0.00378316572178552\\
75.99	0.00378137343212763\\
76	0.00377958087760879\\
76.01	0.00377778805885348\\
76.02	0.00377599497648576\\
76.03	0.0037742016311293\\
76.04	0.00377240802340734\\
76.05	0.00377061415394268\\
76.06	0.00376882002335768\\
76.07	0.00376702563227425\\
76.08	0.00376523098131381\\
76.09	0.00376343607109731\\
76.1	0.0037616409022452\\
76.11	0.00375984547537741\\
76.12	0.00375804979111336\\
76.13	0.00375625385007193\\
76.14	0.00375445765287143\\
76.15	0.00375266120012966\\
76.16	0.00375086449246378\\
76.17	0.00374906753049038\\
76.18	0.00374727031482548\\
76.19	0.00374547284608445\\
76.2	0.00374367512488202\\
76.21	0.00374187715183231\\
76.22	0.00374007892754875\\
76.23	0.00373828045264411\\
76.24	0.00373648172773048\\
76.25	0.00373468275341924\\
76.26	0.00373288353032105\\
76.27	0.00373108405904584\\
76.28	0.00372928434020281\\
76.29	0.0037274843744004\\
76.3	0.00372568416224626\\
76.31	0.00372388370434726\\
76.32	0.00372208300130947\\
76.33	0.00372028205373814\\
76.34	0.00371848086223769\\
76.35	0.00371667942741169\\
76.36	0.00371487774986284\\
76.37	0.00371307583019296\\
76.38	0.00371127366900301\\
76.39	0.00370947126689298\\
76.4	0.003707668624462\\
76.41	0.0037058657423082\\
76.42	0.0037040626210288\\
76.43	0.00370225926122003\\
76.44	0.00370045566347711\\
76.45	0.0036986518283943\\
76.46	0.00369684775656481\\
76.47	0.00369504344858081\\
76.48	0.00369323890503343\\
76.49	0.00369143412651273\\
76.5	0.00368962911360768\\
76.51	0.00368782386690614\\
76.52	0.00368601838699485\\
76.53	0.00368421267445944\\
76.54	0.00368240672988436\\
76.55	0.00368060055385289\\
76.56	0.00367879414694713\\
76.57	0.00367698750974797\\
76.58	0.00367518064283509\\
76.59	0.00367337354678691\\
76.6	0.00367156622218061\\
76.61	0.00366975866959209\\
76.62	0.00366795088959595\\
76.63	0.00366614288276548\\
76.64	0.00366433464967264\\
76.65	0.00366252619088806\\
76.66	0.00366071750698099\\
76.67	0.00365890859851928\\
76.68	0.00365709946606941\\
76.69	0.00365529011019642\\
76.7	0.00365348053146391\\
76.71	0.00365167073043403\\
76.72	0.00364986070766743\\
76.73	0.00364805046372331\\
76.74	0.00364623999915931\\
76.75	0.00364442931453155\\
76.76	0.0036426184103946\\
76.77	0.00364080728730143\\
76.78	0.00363899594580347\\
76.79	0.0036371843864505\\
76.8	0.00363537260979066\\
76.81	0.00363356061637045\\
76.82	0.0036317484067347\\
76.83	0.00362993598142654\\
76.84	0.0036281233409874\\
76.85	0.00362631048595695\\
76.86	0.00362449741687313\\
76.87	0.00362268413427209\\
76.88	0.00362087063868819\\
76.89	0.00361905693065397\\
76.9	0.00361724301070013\\
76.91	0.00361542887935552\\
76.92	0.0036136145371471\\
76.93	0.00361179998459993\\
76.94	0.00360998522223715\\
76.95	0.00360817025057995\\
76.96	0.00360635507014754\\
76.97	0.00360453968145718\\
76.98	0.00360272408502408\\
76.99	0.00360090828136143\\
77	0.00359909227098037\\
77.01	0.00359727605438995\\
77.02	0.00359545963209713\\
77.03	0.00359364300460673\\
77.04	0.00359182617242146\\
77.05	0.0035900091360418\\
77.06	0.00358819189596611\\
77.07	0.00358637445269047\\
77.08	0.00358455680670875\\
77.09	0.00358273895851258\\
77.1	0.00358092090859125\\
77.11	0.00357910265743177\\
77.12	0.00357728420551884\\
77.13	0.00357546555333475\\
77.14	0.00357364670135944\\
77.15	0.00357182765007045\\
77.16	0.00357000839994284\\
77.17	0.00356818895144929\\
77.18	0.00356636930505993\\
77.19	0.00356454946124242\\
77.2	0.00356272942046188\\
77.21	0.00356090918318087\\
77.22	0.00355908874985939\\
77.23	0.0035572681209548\\
77.24	0.00355544729692185\\
77.25	0.00355362627821263\\
77.26	0.00355180506527653\\
77.27	0.00354998365856024\\
77.28	0.00354816205850773\\
77.29	0.00354634026556016\\
77.3	0.00354451828015597\\
77.31	0.00354269610273071\\
77.32	0.00354087373371713\\
77.33	0.00353905117354512\\
77.34	0.00353722842264162\\
77.35	0.0035354054814307\\
77.36	0.00353358235033346\\
77.37	0.003531759029768\\
77.38	0.00352993552014943\\
77.39	0.00352811182188984\\
77.4	0.00352628793539821\\
77.41	0.00352446386108048\\
77.42	0.00352263959933944\\
77.43	0.00352081515057473\\
77.44	0.00351899051518282\\
77.45	0.00351716569355698\\
77.46	0.00351534068608724\\
77.47	0.00351351549316035\\
77.48	0.00351169011515979\\
77.49	0.00350986455246568\\
77.5	0.00350803880545484\\
77.51	0.00350621287450066\\
77.52	0.00350438675997312\\
77.53	0.00350256046223878\\
77.54	0.00350073398166072\\
77.55	0.00349890731859849\\
77.56	0.00349708047340814\\
77.57	0.00349525344644213\\
77.58	0.00349342623804932\\
77.59	0.00349159884857496\\
77.6	0.00348977127836063\\
77.61	0.0034879435277442\\
77.62	0.00348611559705985\\
77.63	0.00348428748663798\\
77.64	0.00348245919680521\\
77.65	0.00348063072788433\\
77.66	0.00347880208019428\\
77.67	0.00347697325405014\\
77.68	0.00347514424976304\\
77.69	0.00347331506764016\\
77.7	0.00347148570798471\\
77.71	0.00346965617109588\\
77.72	0.0034678264572688\\
77.73	0.00346599656679452\\
77.74	0.00346416649995996\\
77.75	0.00346233625704791\\
77.76	0.00346050583833693\\
77.77	0.0034586752441014\\
77.78	0.00345684447461143\\
77.79	0.00345501353013282\\
77.8	0.00345318241092706\\
77.81	0.00345135111725126\\
77.82	0.00344951964935815\\
77.83	0.00344768800749603\\
77.84	0.0034458561919087\\
77.85	0.00344402420283549\\
77.86	0.00344219204051115\\
77.87	0.00344035970516587\\
77.88	0.00343852719702523\\
77.89	0.00343669451631016\\
77.9	0.00343486166323686\\
77.91	0.00343302863801686\\
77.92	0.00343119544085689\\
77.93	0.00342936207195888\\
77.94	0.00342752853151993\\
77.95	0.00342569481973224\\
77.96	0.00342386093678313\\
77.97	0.00342202688285492\\
77.98	0.00342019265812498\\
77.99	0.0034183582627656\\
78	0.00341652369694404\\
78.01	0.00341468896082242\\
78.02	0.00341285405455771\\
78.03	0.0034110189783017\\
78.04	0.00340918373220094\\
78.05	0.00340734831639672\\
78.06	0.00340551273102499\\
78.07	0.00340367697621637\\
78.08	0.0034018410520961\\
78.09	0.00340000495878394\\
78.1	0.0033981686963942\\
78.11	0.00339633226503567\\
78.12	0.00339449566481159\\
78.13	0.00339265889581956\\
78.14	0.00339082195815158\\
78.15	0.00338898485189393\\
78.16	0.00338714757712718\\
78.17	0.00338531013392611\\
78.18	0.0033834725223597\\
78.19	0.00338163474249106\\
78.2	0.00337979679437739\\
78.21	0.00337795867806996\\
78.22	0.00337612039361402\\
78.23	0.00337428194104883\\
78.24	0.00337244332040752\\
78.25	0.00337060453171712\\
78.26	0.00336876557499848\\
78.27	0.00336692645026624\\
78.28	0.00336508715752878\\
78.29	0.00336324769678816\\
78.3	0.00336140806804009\\
78.31	0.00335956827127388\\
78.32	0.00335772830647241\\
78.33	0.00335588817361203\\
78.34	0.00335404787266259\\
78.35	0.00335220740358731\\
78.36	0.00335036676634281\\
78.37	0.003348525960879\\
78.38	0.00334668498713908\\
78.39	0.00334484384505945\\
78.4	0.00334300253456968\\
78.41	0.00334116105559248\\
78.42	0.00333931940804361\\
78.43	0.00333747759183189\\
78.44	0.00333563560685906\\
78.45	0.00333379345301982\\
78.46	0.00333195113020174\\
78.47	0.0033301086382852\\
78.48	0.00332826597714336\\
78.49	0.0033264231466421\\
78.5	0.00332458014663997\\
78.51	0.00332273697698814\\
78.52	0.00332089363753033\\
78.53	0.00331905012810279\\
78.54	0.00331720644853424\\
78.55	0.00331536259864578\\
78.56	0.00331351857825087\\
78.57	0.00331167438715531\\
78.58	0.0033098300251571\\
78.59	0.00330798549204647\\
78.6	0.00330614078760577\\
78.61	0.00330429591160944\\
78.62	0.00330245086382396\\
78.63	0.00330060564400777\\
78.64	0.00329876025191126\\
78.65	0.00329691468727665\\
78.66	0.00329506894983799\\
78.67	0.00329322303932109\\
78.68	0.00329137695544344\\
78.69	0.00328953069791418\\
78.7	0.00328768426643403\\
78.71	0.00328583766069524\\
78.72	0.00328399088038152\\
78.73	0.003282143925168\\
78.74	0.00328029679472114\\
78.75	0.00327844948869875\\
78.76	0.00327660200674979\\
78.77	0.00327475434851447\\
78.78	0.00327290651362406\\
78.79	0.00327105850170093\\
78.8	0.00326921031235843\\
78.81	0.00326736194520083\\
78.82	0.00326551339982328\\
78.83	0.00326366467581177\\
78.84	0.003261815772743\\
78.85	0.00325996669018439\\
78.86	0.00325811742769398\\
78.87	0.00325626798482038\\
78.88	0.00325441836110268\\
78.89	0.00325256855607045\\
78.9	0.00325071856924359\\
78.91	0.00324886840013234\\
78.92	0.00324701802581425\\
78.93	0.0032451673072342\\
78.94	0.00324331624555393\\
78.95	0.00324146484193895\\
78.96	0.00323961309755856\\
78.97	0.00323776101358589\\
78.98	0.00323590859119788\\
78.99	0.00323405583157529\\
79	0.00323220273590273\\
79.01	0.00323034930536865\\
79.02	0.00322849554116538\\
79.03	0.00322664144448908\\
79.04	0.00322478701653983\\
79.05	0.00322293225852158\\
79.06	0.00322107717164219\\
79.07	0.00321922175711343\\
79.08	0.00321736601615098\\
79.09	0.00321550994997446\\
79.1	0.00321365355980746\\
79.11	0.00321179684687747\\
79.12	0.00320993981241599\\
79.13	0.00320808245765847\\
79.14	0.00320622478384436\\
79.15	0.0032043667922171\\
79.16	0.00320250848402414\\
79.17	0.00320064986051694\\
79.18	0.00319879092295099\\
79.19	0.00319693167258584\\
79.2	0.00319507211068505\\
79.21	0.00319321223851629\\
79.22	0.00319135205735127\\
79.23	0.00318949156846578\\
79.24	0.00318763077313972\\
79.25	0.00318576967265709\\
79.26	0.00318390826830602\\
79.27	0.00318204656137874\\
79.28	0.00318018455317163\\
79.29	0.00317832224498523\\
79.3	0.00317645963812422\\
79.31	0.00317459673389747\\
79.32	0.00317273353361804\\
79.33	0.00317087003860315\\
79.34	0.00316900625017426\\
79.35	0.00316714216965702\\
79.36	0.00316527779838135\\
79.37	0.00316341313768135\\
79.38	0.00316154818889544\\
79.39	0.00315968295336623\\
79.4	0.00315781743244065\\
79.41	0.00315595162746993\\
79.42	0.00315408553980954\\
79.43	0.00315221917081931\\
79.44	0.00315035252186337\\
79.45	0.00314848559431019\\
79.46	0.00314661838953256\\
79.47	0.00314475090890767\\
79.48	0.00314288315381703\\
79.49	0.00314101512564655\\
79.5	0.00313914682578656\\
79.51	0.00313727825563173\\
79.52	0.0031354094165812\\
79.53	0.0031335403100385\\
79.54	0.00313167093741162\\
79.55	0.00312980130011299\\
79.56	0.00312793139955953\\
79.57	0.00312606123717258\\
79.58	0.00312419081437801\\
79.59	0.00312232013260618\\
79.6	0.00312044919329195\\
79.61	0.00311857799787472\\
79.62	0.00311670654779842\\
79.63	0.00311483484451151\\
79.64	0.00311296288946703\\
79.65	0.00311109068412259\\
79.66	0.00310921822994038\\
79.67	0.00310734552838718\\
79.68	0.0031054725809344\\
79.69	0.00310359938905805\\
79.7	0.00310172595423879\\
79.71	0.00309985227796192\\
79.72	0.00309797836171739\\
79.73	0.00309610420699985\\
79.74	0.0030942298153086\\
79.75	0.00309235518814768\\
79.76	0.0030904803270258\\
79.77	0.00308860523345641\\
79.78	0.0030867299089577\\
79.79	0.0030848543550526\\
79.8	0.00308297857326881\\
79.81	0.00308110256513878\\
79.82	0.0030792263321998\\
79.83	0.00307734987599391\\
79.84	0.00307547319806798\\
79.85	0.00307359629997371\\
79.86	0.00307171918326764\\
79.87	0.00306984184951116\\
79.88	0.00306796430027053\\
79.89	0.00306608653711688\\
79.9	0.00306420856162624\\
79.91	0.00306233037537954\\
79.92	0.00306045197996264\\
79.93	0.00305857337696632\\
79.94	0.00305669456798631\\
79.95	0.00305481555462331\\
79.96	0.00305293633848297\\
79.97	0.00305105692117595\\
79.98	0.00304917730431789\\
79.99	0.00304729748952945\\
80	0.00304541747843634\\
80.01	0.00304353727266926\\
};
\addplot [color=mycolor1,dashed]
  table[row sep=crcr]{%
80.01	0.00304353727266926\\
80.02	0.00304165687386401\\
80.03	0.00303977628366145\\
80.04	0.00303789550370749\\
80.05	0.00303601453565319\\
80.06	0.00303413338115466\\
80.07	0.00303225204187319\\
80.08	0.00303037051947516\\
80.09	0.00302848881563213\\
80.1	0.00302660693202082\\
80.11	0.00302472487032312\\
80.12	0.00302284263222613\\
80.13	0.00302096021942213\\
80.14	0.00301907763360865\\
80.15	0.00301719487648845\\
80.16	0.00301531194976953\\
80.17	0.00301342885516517\\
80.18	0.00301154559439391\\
80.19	0.0030096621691796\\
80.2	0.00300777858125138\\
80.21	0.00300589483234375\\
80.22	0.00300401092419649\\
80.23	0.00300212685855479\\
80.24	0.00300024263716915\\
80.25	0.00299835826179551\\
80.26	0.00299647373419514\\
80.27	0.00299458905613478\\
80.28	0.00299270422938656\\
80.29	0.00299081925572806\\
80.3	0.00298893413694232\\
80.31	0.00298704887481783\\
80.32	0.00298516347114858\\
80.33	0.00298327792773406\\
80.34	0.00298139224637927\\
80.35	0.00297950642889473\\
80.36	0.00297762047709653\\
80.37	0.0029757343928063\\
80.38	0.00297384817785124\\
80.39	0.00297196183406415\\
80.4	0.00297007536328345\\
80.41	0.00296818876735313\\
80.42	0.00296630204812289\\
80.43	0.00296441520744802\\
80.44	0.00296252824718949\\
80.45	0.00296064116921398\\
80.46	0.00295875397539385\\
80.47	0.00295686666760715\\
80.48	0.00295497924773771\\
80.49	0.00295309171767505\\
80.5	0.00295120407931448\\
80.51	0.00294931633455711\\
80.52	0.00294742848530978\\
80.53	0.00294554053348519\\
80.54	0.00294365248100185\\
80.55	0.0029417643297841\\
80.56	0.00293987608176214\\
80.57	0.00293798773887204\\
80.58	0.00293609930305577\\
80.59	0.0029342107762612\\
80.6	0.0029323221604421\\
80.61	0.00293043345755821\\
80.62	0.00292854466957518\\
80.63	0.00292665579846467\\
80.64	0.00292476684620432\\
80.65	0.00292287781477774\\
80.66	0.00292098870617459\\
80.67	0.00291909952239054\\
80.68	0.00291721026542733\\
80.69	0.00291532093729277\\
80.7	0.00291343154000073\\
80.71	0.00291154207557123\\
80.72	0.00290965254603034\\
80.73	0.00290776295341032\\
80.74	0.00290587329974955\\
80.75	0.00290398358709259\\
80.76	0.0029020938174902\\
80.77	0.00290020399299931\\
80.78	0.00289831411568309\\
80.79	0.00289642418761094\\
80.8	0.00289453421085853\\
80.81	0.00289264418750777\\
80.82	0.00289075411964688\\
80.83	0.00288886400937039\\
80.84	0.00288697385877912\\
80.85	0.00288508366998026\\
80.86	0.00288319344508736\\
80.87	0.00288130318622032\\
80.88	0.00287941289550544\\
80.89	0.00287752257507544\\
80.9	0.00287563222706947\\
80.91	0.00287374185363311\\
80.92	0.0028718514569184\\
80.93	0.00286996103908389\\
80.94	0.00286807060229459\\
80.95	0.00286618014872204\\
80.96	0.00286428968054433\\
80.97	0.00286239919994609\\
80.98	0.00286050870911851\\
80.99	0.00285861821025938\\
81	0.00285672770557309\\
81.01	0.00285483719727066\\
81.02	0.00285294668756975\\
81.03	0.00285105617869468\\
81.04	0.00284916567287645\\
81.05	0.00284727517235275\\
81.06	0.002845384679368\\
81.07	0.00284349419617334\\
81.08	0.00284160372502667\\
81.09	0.00283971326819266\\
81.1	0.00283782282794278\\
81.11	0.00283593240655528\\
81.12	0.00283404200631526\\
81.13	0.00283215162951468\\
81.14	0.00283026127845233\\
81.15	0.0028283709554339\\
81.16	0.00282648066277199\\
81.17	0.00282459040278611\\
81.18	0.00282270017780273\\
81.19	0.00282080999015524\\
81.2	0.00281891984218405\\
81.21	0.00281702973623655\\
81.22	0.00281513967466715\\
81.23	0.0028132496598373\\
81.24	0.0028113596941155\\
81.25	0.00280946977987733\\
81.26	0.00280757991950546\\
81.27	0.00280569011538968\\
81.28	0.00280380036992691\\
81.29	0.00280191068552124\\
81.3	0.0028000210645839\\
81.31	0.00279813150953334\\
81.32	0.00279624202279521\\
81.33	0.00279435260680241\\
81.34	0.00279246326399507\\
81.35	0.00279057399682061\\
81.36	0.00278868480773372\\
81.37	0.00278679569919644\\
81.38	0.00278490667367812\\
81.39	0.00278301773365544\\
81.4	0.0027811288816125\\
81.41	0.00277924012004077\\
81.42	0.00277735145143912\\
81.43	0.00277546287831388\\
81.44	0.00277357440317882\\
81.45	0.0027716860285552\\
81.46	0.00276979775697175\\
81.47	0.00276790959096474\\
81.48	0.00276602153307797\\
81.49	0.00276413358586279\\
81.5	0.00276224575187813\\
81.51	0.00276035803369053\\
81.52	0.00275847043387414\\
81.53	0.00275658295501077\\
81.54	0.00275469559968986\\
81.55	0.00275280837050856\\
81.56	0.00275092127007172\\
81.57	0.00274903430099191\\
81.58	0.00274714746588943\\
81.59	0.00274526076739239\\
81.6	0.00274337420813666\\
81.61	0.00274148779076592\\
81.62	0.00273960151793169\\
81.63	0.00273771539229336\\
81.64	0.00273582941651816\\
81.65	0.00273394359328124\\
81.66	0.00273205792526566\\
81.67	0.00273017241516245\\
81.68	0.00272828706567055\\
81.69	0.00272640187949693\\
81.7	0.00272451685935655\\
81.71	0.00272263200797238\\
81.72	0.00272074732807547\\
81.73	0.00271886282240493\\
81.74	0.00271697849370796\\
81.75	0.00271509434473988\\
81.76	0.00271321037826415\\
81.77	0.00271132659705241\\
81.78	0.00270944300388443\\
81.79	0.00270755960154826\\
81.8	0.00270567639284011\\
81.81	0.00270379338056448\\
81.82	0.00270191056753414\\
81.83	0.00270002795657016\\
81.84	0.0026981455505019\\
81.85	0.0026962633521671\\
81.86	0.00269438136441184\\
81.87	0.00269249959009061\\
81.88	0.00269061803206628\\
81.89	0.00268873669321017\\
81.9	0.00268685557640208\\
81.91	0.00268497468453025\\
81.92	0.00268309402049145\\
81.93	0.00268121358719096\\
81.94	0.00267933338754263\\
81.95	0.00267745342446887\\
81.96	0.00267557370090067\\
81.97	0.00267369421977768\\
81.98	0.00267181498404818\\
81.99	0.00266993599666908\\
82	0.00266805726060605\\
82.01	0.00266617877883341\\
82.02	0.00266430055433425\\
82.03	0.00266242259010042\\
82.04	0.00266054488913255\\
82.05	0.0026586674544401\\
82.06	0.00265679028904133\\
82.07	0.0026549133959634\\
82.08	0.00265303677824231\\
82.09	0.002651160438923\\
82.1	0.00264928438105933\\
82.11	0.00264740860771411\\
82.12	0.00264553312195914\\
82.13	0.00264365792687521\\
82.14	0.00264178302555216\\
82.15	0.00263990842108888\\
82.16	0.00263803411659331\\
82.17	0.00263616011518252\\
82.18	0.00263428641998271\\
82.19	0.00263241303412921\\
82.2	0.00263053996076654\\
82.21	0.00262866720304844\\
82.22	0.00262679476413785\\
82.23	0.00262492264720697\\
82.24	0.00262305085543727\\
82.25	0.00262117939201955\\
82.26	0.00261930826015391\\
82.27	0.00261743746304982\\
82.28	0.00261556700392611\\
82.29	0.00261369688601102\\
82.3	0.00261182711254224\\
82.31	0.00260995768676689\\
82.32	0.00260808861194157\\
82.33	0.00260621989133241\\
82.34	0.00260435152821504\\
82.35	0.00260248352587466\\
82.36	0.00260061588760607\\
82.37	0.00259874861671366\\
82.38	0.00259688171651145\\
82.39	0.00259501519032313\\
82.4	0.00259314904148209\\
82.41	0.00259128327333141\\
82.42	0.00258941788922391\\
82.43	0.0025875528925222\\
82.44	0.00258568828659864\\
82.45	0.00258382407483545\\
82.46	0.00258196026062466\\
82.47	0.00258009684736821\\
82.48	0.00257823383847789\\
82.49	0.00257637123737543\\
82.5	0.00257450904749254\\
82.51	0.00257264727227086\\
82.52	0.00257078591516208\\
82.53	0.00256892497962787\\
82.54	0.00256706446914\\
82.55	0.00256520438718029\\
82.56	0.0025633447372407\\
82.57	0.0025614855228233\\
82.58	0.00255962674744034\\
82.59	0.00255776841461427\\
82.6	0.00255591052787775\\
82.61	0.00255405309077367\\
82.62	0.0025521961068552\\
82.63	0.00255033957968584\\
82.64	0.00254848351283938\\
82.65	0.00254662790989999\\
82.66	0.0025447727744622\\
82.67	0.00254291811017059\\
82.68	0.0025410639206978\\
82.69	0.00253921020972722\\
82.7	0.00253735698095301\\
82.71	0.00253550423808015\\
82.72	0.00253365198482446\\
82.73	0.00253180022491263\\
82.74	0.00252994896208225\\
82.75	0.00252809820008186\\
82.76	0.00252624794267094\\
82.77	0.00252439819361996\\
82.78	0.00252254895671042\\
82.79	0.00252070023573489\\
82.8	0.00251885203449698\\
82.81	0.00251700435681145\\
82.82	0.0025151572065042\\
82.83	0.00251331058741228\\
82.84	0.00251146450338398\\
82.85	0.0025096189582788\\
82.86	0.00250777395596752\\
82.87	0.00250592950033223\\
82.88	0.00250408559526632\\
82.89	0.00250224224467455\\
82.9	0.0025003994524731\\
82.91	0.00249855722258955\\
82.92	0.00249671555896294\\
82.93	0.00249487446554377\\
82.94	0.00249303394629412\\
82.95	0.00249119400518756\\
82.96	0.00248935464620928\\
82.97	0.00248751587335607\\
82.98	0.00248567769063636\\
82.99	0.00248384010207026\\
83	0.00248200311168961\\
83.01	0.00248016672353796\\
83.02	0.00247833094167067\\
83.03	0.00247649577015488\\
83.04	0.00247466121306958\\
83.05	0.00247282727450564\\
83.06	0.00247099395856581\\
83.07	0.00246916126936481\\
83.08	0.00246732921102931\\
83.09	0.00246549778769798\\
83.1	0.00246366700352155\\
83.11	0.0024618368626628\\
83.12	0.00246000736929661\\
83.13	0.00245817852761002\\
83.14	0.00245635034180221\\
83.15	0.00245452281608459\\
83.16	0.00245269595468078\\
83.17	0.0024508697618267\\
83.18	0.00244904424177054\\
83.19	0.00244721939877286\\
83.2	0.00244539523710658\\
83.21	0.00244357176105701\\
83.22	0.00244174897492192\\
83.23	0.00243992688301155\\
83.24	0.00243810548964865\\
83.25	0.0024362847991685\\
83.26	0.00243446481591898\\
83.27	0.00243264554426054\\
83.28	0.00243082698856633\\
83.29	0.00242900915322214\\
83.3	0.00242719204262649\\
83.31	0.00242537566119067\\
83.32	0.00242356001333871\\
83.33	0.0024217451035075\\
83.34	0.00241993093614677\\
83.35	0.00241811751571915\\
83.36	0.00241630484670017\\
83.37	0.00241449293357837\\
83.38	0.00241268178085524\\
83.39	0.00241087139304533\\
83.4	0.00240906177467625\\
83.41	0.00240725293028871\\
83.42	0.00240544486443657\\
83.43	0.00240363758168687\\
83.44	0.00240183108661983\\
83.45	0.00240002538382897\\
83.46	0.00239822047792105\\
83.47	0.00239641637351617\\
83.48	0.00239461307524779\\
83.49	0.00239281058776275\\
83.5	0.00239100891572134\\
83.51	0.00238920806379731\\
83.52	0.0023874080366779\\
83.53	0.00238560883906391\\
83.54	0.00238381047566971\\
83.55	0.00238201295122327\\
83.56	0.00238021627046625\\
83.57	0.00237842043815396\\
83.58	0.00237662545905546\\
83.59	0.00237483133795356\\
83.6	0.00237303807964488\\
83.61	0.00237124568893989\\
83.62	0.0023694541706629\\
83.63	0.00236766352965218\\
83.64	0.00236587377075993\\
83.65	0.00236408489885232\\
83.66	0.0023622969188096\\
83.67	0.00236050983552604\\
83.68	0.00235872365391003\\
83.69	0.00235693837888411\\
83.7	0.002355154015385\\
83.71	0.00235337056836363\\
83.72	0.00235158804278521\\
83.73	0.00234980644362922\\
83.74	0.00234802577588949\\
83.75	0.00234624604457424\\
83.76	0.00234446725470607\\
83.77	0.00234268941132205\\
83.78	0.00234091251947377\\
83.79	0.0023391365842273\\
83.8	0.00233736161066333\\
83.81	0.00233558760387712\\
83.82	0.0023338145689786\\
83.83	0.0023320425110924\\
83.84	0.00233027143535785\\
83.85	0.00232850134692908\\
83.86	0.00232673225097501\\
83.87	0.0023249641526794\\
83.88	0.00232319705724093\\
83.89	0.00232143096987319\\
83.9	0.00231966589580473\\
83.91	0.00231790184027913\\
83.92	0.002316138808555\\
83.93	0.00231437680590606\\
83.94	0.00231261583762115\\
83.95	0.00231085590900428\\
83.96	0.00230909702537469\\
83.97	0.00230733919206685\\
83.98	0.00230558241443053\\
83.99	0.00230382669783086\\
84	0.0023020720476483\\
84.01	0.00230031846927878\\
84.02	0.00229856596813366\\
84.03	0.0022968145496398\\
84.04	0.0022950642192396\\
84.05	0.00229331498239107\\
84.06	0.00229156684456781\\
84.07	0.00228981981125912\\
84.08	0.00228807388796998\\
84.09	0.00228632908022113\\
84.1	0.00228458539354913\\
84.11	0.00228284283350632\\
84.12	0.00228110140566098\\
84.13	0.00227936111559726\\
84.14	0.0022776219689153\\
84.15	0.00227588397123123\\
84.16	0.00227414712817724\\
84.17	0.00227241144540159\\
84.18	0.00227067692856869\\
84.19	0.00226894358335912\\
84.2	0.00226721141546968\\
84.21	0.00226548043061342\\
84.22	0.0022637506345197\\
84.23	0.00226202203293424\\
84.24	0.00226029463161913\\
84.25	0.0022585684363529\\
84.26	0.00225684345293057\\
84.27	0.00225511968716365\\
84.28	0.00225339714488026\\
84.29	0.00225167583192508\\
84.3	0.00224995575415948\\
84.31	0.00224823691746149\\
84.32	0.0022465193277259\\
84.33	0.00224480299086428\\
84.34	0.00224308791280503\\
84.35	0.00224137409949341\\
84.36	0.0022396615568916\\
84.37	0.00223795029097873\\
84.38	0.00223624030775097\\
84.39	0.0022345316132215\\
84.4	0.0022328242134206\\
84.41	0.00223111811439569\\
84.42	0.00222941332221137\\
84.43	0.00222770984294947\\
84.44	0.00222600768270911\\
84.45	0.00222430684760668\\
84.46	0.00222260734377597\\
84.47	0.00222090917736815\\
84.48	0.00221921235455187\\
84.49	0.00221751688151324\\
84.5	0.00221582276445595\\
84.51	0.00221413000960125\\
84.52	0.00221243862318802\\
84.53	0.00221074861147283\\
84.54	0.00220905998072997\\
84.55	0.0022073727372515\\
84.56	0.00220568688734729\\
84.57	0.00220400243734506\\
84.58	0.00220231939359044\\
84.59	0.00220063776244705\\
84.6	0.00219895755029643\\
84.61	0.00219727876353823\\
84.62	0.00219560140859017\\
84.63	0.00219392549188808\\
84.64	0.00219225101988599\\
84.65	0.00219057799905618\\
84.66	0.00218890643588915\\
84.67	0.00218723633689376\\
84.68	0.00218556770859724\\
84.69	0.0021839005575452\\
84.7	0.00218223489030173\\
84.71	0.00218057071344942\\
84.72	0.00217890803358942\\
84.73	0.00217724685734146\\
84.74	0.00217558719134394\\
84.75	0.00217392904225392\\
84.76	0.00217227241674722\\
84.77	0.00217061732151845\\
84.78	0.00216896376328104\\
84.79	0.00216731174876729\\
84.8	0.00216566128472845\\
84.81	0.00216401237793472\\
84.82	0.00216236503517533\\
84.83	0.00216071926325858\\
84.84	0.00215907506901188\\
84.85	0.0021574324592818\\
84.86	0.00215579144093411\\
84.87	0.00215415202085385\\
84.88	0.00215251420594534\\
84.89	0.00215087800313229\\
84.9	0.00214924341935776\\
84.91	0.00214761046158429\\
84.92	0.00214597913679389\\
84.93	0.00214434945198812\\
84.94	0.00214272141418812\\
84.95	0.00214109503043466\\
84.96	0.0021394703077882\\
84.97	0.00213784725332892\\
84.98	0.00213622587415679\\
84.99	0.00213460617739158\\
85	0.00213298817017296\\
85.01	0.00213137185966049\\
85.02	0.00212975725303372\\
85.03	0.00212814435749219\\
85.04	0.00212653318025552\\
85.05	0.00212492372856342\\
85.06	0.00212331600967577\\
85.07	0.00212171003087266\\
85.08	0.00212010579945441\\
85.09	0.00211850332274166\\
85.1	0.00211690260807536\\
85.11	0.00211530366281691\\
85.12	0.0021137064943481\\
85.13	0.00211211111007123\\
85.14	0.00211051751740912\\
85.15	0.00210892572380521\\
85.16	0.00210733573672353\\
85.17	0.00210574756364883\\
85.18	0.00210416121208652\\
85.19	0.00210257668956286\\
85.2	0.00210099400362488\\
85.21	0.00209941316184049\\
85.22	0.00209783417179852\\
85.23	0.00209625704110876\\
85.24	0.00209468177740201\\
85.25	0.00209310838833011\\
85.26	0.00209153688156605\\
85.27	0.00208996726480391\\
85.28	0.00208839954575901\\
85.29	0.00208683373216791\\
85.3	0.00208526983178843\\
85.31	0.00208370785239978\\
85.32	0.00208214780180252\\
85.33	0.00208058968781865\\
85.34	0.00207903351829165\\
85.35	0.00207747930108653\\
85.36	0.00207592704408986\\
85.37	0.00207437675520985\\
85.38	0.00207282844237635\\
85.39	0.00207128211354095\\
85.4	0.00206973777667697\\
85.41	0.00206819543977955\\
85.42	0.00206665511086567\\
85.43	0.00206511679797422\\
85.44	0.00206358050916603\\
85.45	0.0020620462525239\\
85.46	0.00206051403615268\\
85.47	0.0020589838681793\\
85.48	0.00205745575675281\\
85.49	0.00205592971004443\\
85.5	0.00205440573624759\\
85.51	0.00205288384357798\\
85.52	0.00205136404027362\\
85.53	0.00204984633459484\\
85.54	0.00204833008408384\\
85.55	0.0020468139033471\\
85.56	0.00204529778880708\\
85.57	0.00204378173685461\\
85.58	0.00204226574384867\\
85.59	0.00204074980611624\\
85.6	0.00203923391995215\\
85.61	0.00203771808161887\\
85.62	0.00203620228734638\\
85.63	0.00203468653565361\\
85.64	0.00203317082532171\\
85.65	0.00203165515511737\\
85.66	0.00203013952379279\\
85.67	0.00202862393008553\\
85.68	0.00202710837271848\\
85.69	0.00202559285039976\\
85.7	0.00202407736182261\\
85.71	0.00202256190566536\\
85.72	0.00202104648059128\\
85.73	0.00201953108524853\\
85.74	0.0020180157182701\\
85.75	0.00201650037827366\\
85.76	0.00201498506386151\\
85.77	0.00201346977362051\\
85.78	0.00201195450612195\\
85.79	0.00201043925992148\\
85.8	0.00200892403355904\\
85.81	0.00200740882555875\\
85.82	0.00200589363442883\\
85.83	0.00200437845866147\\
85.84	0.00200286329673284\\
85.85	0.00200134814710286\\
85.86	0.00199983300821525\\
85.87	0.00199831787849733\\
85.88	0.00199680275635998\\
85.89	0.00199528764019755\\
85.9	0.00199377252838773\\
85.91	0.0019922574192915\\
85.92	0.00199074231125302\\
85.93	0.00198922720259953\\
85.94	0.00198771209164123\\
85.95	0.00198619697667126\\
85.96	0.00198468185596554\\
85.97	0.00198316672778267\\
85.98	0.0019816515903639\\
85.99	0.00198013644193294\\
86	0.00197862128069595\\
86.01	0.0019771061048414\\
86.02	0.00197559091253995\\
86.03	0.00197407570194441\\
86.04	0.0019725604711896\\
86.05	0.00197104521839224\\
86.06	0.00196952994165089\\
86.07	0.00196801463904582\\
86.08	0.00196649930863893\\
86.09	0.00196498394847363\\
86.1	0.00196346855657473\\
86.11	0.00196195313094838\\
86.12	0.00196043766958193\\
86.13	0.00195892217044384\\
86.14	0.00195740663148355\\
86.15	0.00195589105063144\\
86.16	0.00195437542579864\\
86.17	0.00195285975487702\\
86.18	0.001951344035739\\
86.19	0.00194982826623749\\
86.2	0.00194831244420576\\
86.21	0.00194679656745737\\
86.22	0.00194528063378601\\
86.23	0.00194376464096545\\
86.24	0.00194224858674937\\
86.25	0.00194073246887131\\
86.26	0.00193921628504451\\
86.27	0.00193770003296185\\
86.28	0.00193618371029569\\
86.29	0.0019346673146978\\
86.3	0.00193315084379922\\
86.31	0.00193163429521015\\
86.32	0.00193011766651988\\
86.33	0.0019286009552966\\
86.34	0.00192708415908738\\
86.35	0.00192556727541796\\
86.36	0.00192405030179271\\
86.37	0.0019225332356945\\
86.38	0.00192101607458454\\
86.39	0.00191949881590231\\
86.4	0.00191798145706544\\
86.41	0.00191646399546956\\
86.42	0.00191494642848825\\
86.43	0.00191342875347283\\
86.44	0.00191191096775232\\
86.45	0.00191039306863327\\
86.46	0.00190887505339968\\
86.47	0.00190735691931285\\
86.48	0.00190583866361128\\
86.49	0.00190432028351053\\
86.5	0.0019028017762031\\
86.51	0.00190128313885835\\
86.52	0.00189976436862231\\
86.53	0.00189824546261761\\
86.54	0.00189672641794331\\
86.55	0.00189520723167483\\
86.56	0.00189368790086379\\
86.57	0.00189216842253788\\
86.58	0.00189064879370077\\
86.59	0.00188912901133193\\
86.6	0.00188760907238656\\
86.61	0.00188608897379541\\
86.62	0.00188456871246469\\
86.63	0.00188304828527594\\
86.64	0.00188152768908586\\
86.65	0.00188000692072623\\
86.66	0.00187848597700376\\
86.67	0.00187696485469994\\
86.68	0.00187544355057096\\
86.69	0.00187392206134751\\
86.7	0.00187240038373469\\
86.71	0.00187087851441188\\
86.72	0.00186935645003259\\
86.73	0.00186783418722434\\
86.74	0.00186631172258848\\
86.75	0.00186478905270014\\
86.76	0.00186326617410802\\
86.77	0.00186174308333427\\
86.78	0.00186021977687437\\
86.79	0.00185869625119699\\
86.8	0.00185717250274383\\
86.81	0.00185564852792951\\
86.82	0.0018541243231414\\
86.83	0.00185259988473952\\
86.84	0.00185107520905636\\
86.85	0.00184955029239674\\
86.86	0.00184802513103772\\
86.87	0.00184649972122839\\
86.88	0.00184497405918975\\
86.89	0.00184344814111461\\
86.9	0.00184192196316737\\
86.91	0.00184039552148395\\
86.92	0.00183886881217158\\
86.93	0.00183734183130869\\
86.94	0.00183581457875494\\
86.95	0.00183428705458884\\
86.96	0.00183275925889077\\
86.97	0.00183123119174291\\
86.98	0.00182970285322933\\
86.99	0.00182817424343597\\
87	0.00182664536245068\\
87.01	0.00182511621036325\\
87.02	0.0018235867872654\\
87.03	0.00182205709325086\\
87.04	0.00182052712841532\\
87.05	0.00181899689285651\\
87.06	0.00181746638667421\\
87.07	0.00181593560997025\\
87.08	0.00181440456284857\\
87.09	0.00181287324541521\\
87.1	0.00181134165777834\\
87.11	0.00180980980004833\\
87.12	0.00180827767233768\\
87.13	0.00180674527476116\\
87.14	0.00180521260743573\\
87.15	0.00180367967048063\\
87.16	0.00180214646401738\\
87.17	0.00180061298816982\\
87.18	0.00179907924306411\\
87.19	0.00179754522882879\\
87.2	0.00179601094559476\\
87.21	0.00179447639349536\\
87.22	0.00179294157266635\\
87.23	0.00179140648324595\\
87.24	0.0017898711253749\\
87.25	0.00178833549919643\\
87.26	0.00178679960485631\\
87.27	0.00178526344250292\\
87.28	0.00178372701228719\\
87.29	0.00178219031436271\\
87.3	0.00178065334888571\\
87.31	0.0017791161160151\\
87.32	0.0017775786159125\\
87.33	0.00177604084874228\\
87.34	0.00177450281467155\\
87.35	0.00177296451387023\\
87.36	0.00177142594651107\\
87.37	0.00176988711276967\\
87.38	0.00176834801282448\\
87.39	0.00176680864685691\\
87.4	0.00176526901505126\\
87.41	0.00176372911759485\\
87.42	0.00176218895467795\\
87.43	0.0017606485264939\\
87.44	0.00175910783323908\\
87.45	0.00175756687511297\\
87.46	0.00175602565231815\\
87.47	0.00175448416506038\\
87.48	0.0017529424135486\\
87.49	0.00175140039799495\\
87.5	0.00174985811861484\\
87.51	0.00174831557562693\\
87.52	0.00174677276925322\\
87.53	0.00174522969971903\\
87.54	0.00174368636725308\\
87.55	0.00174214277208748\\
87.56	0.0017405989144578\\
87.57	0.00173905479460306\\
87.58	0.00173751041276582\\
87.59	0.00173596576919216\\
87.6	0.00173442086413175\\
87.61	0.00173287569783786\\
87.62	0.00173133027056742\\
87.63	0.00172978458258103\\
87.64	0.00172823863414301\\
87.65	0.00172669242552142\\
87.66	0.00172514595698811\\
87.67	0.00172359922881878\\
87.68	0.00172205224129295\\
87.69	0.00172050499469403\\
87.7	0.00171895748930941\\
87.71	0.00171740972543039\\
87.72	0.00171586170335231\\
87.73	0.00171431342337453\\
87.74	0.00171276488600161\\
87.75	0.00171121609174509\\
87.76	0.0017096670411175\\
87.77	0.00170811773463236\\
87.78	0.00170656817280413\\
87.79	0.0017050183561483\\
87.8	0.00170346828518129\\
87.81	0.00170191796042053\\
87.82	0.00170036738238442\\
87.83	0.00169881655159234\\
87.84	0.00169726546856465\\
87.85	0.00169571413382271\\
87.86	0.00169416254788885\\
87.87	0.00169261071128638\\
87.88	0.0016910586245396\\
87.89	0.00168950628817381\\
87.9	0.00168795370271527\\
87.91	0.00168640086869125\\
87.92	0.00168484778662998\\
87.93	0.00168329445706073\\
87.94	0.0016817408805137\\
87.95	0.00168018705752012\\
87.96	0.0016786329886122\\
87.97	0.00167707867432313\\
87.98	0.00167552411518711\\
87.99	0.00167396931173932\\
88	0.00167241426451595\\
88.01	0.00167085897405416\\
88.02	0.00166930344089214\\
88.03	0.00166774766556904\\
88.04	0.00166619164862501\\
88.05	0.00166463539060124\\
88.06	0.00166307889203987\\
88.07	0.00166152215348405\\
88.08	0.00165996517547795\\
88.09	0.00165840795856672\\
88.1	0.00165685050329652\\
88.11	0.00165529281021449\\
88.12	0.00165373487986882\\
88.13	0.00165217671280865\\
88.14	0.00165061830958415\\
88.15	0.0016490596707465\\
88.16	0.00164750079684787\\
88.17	0.00164594168844144\\
88.18	0.0016443823460814\\
88.19	0.00164282277032295\\
88.2	0.00164126296172228\\
88.21	0.00163970292083661\\
88.22	0.00163814264822416\\
88.23	0.00163658214444415\\
88.24	0.00163502141005682\\
88.25	0.00163346044562343\\
88.26	0.00163189925170623\\
88.27	0.0016303378288685\\
88.28	0.00162877617767453\\
88.29	0.00162721429868961\\
88.3	0.00162565219248005\\
88.31	0.00162408985961319\\
88.32	0.00162252730065737\\
88.33	0.00162096451618195\\
88.34	0.00161940150675731\\
88.35	0.00161783827295483\\
88.36	0.00161627481534692\\
88.37	0.00161471113450704\\
88.38	0.0016131472310096\\
88.39	0.00161158310543008\\
88.4	0.00161001875834498\\
88.41	0.0016084541903318\\
88.42	0.00160688940196906\\
88.43	0.00160532439383633\\
88.44	0.00160375916651418\\
88.45	0.00160219372058421\\
88.46	0.00160062805662903\\
88.47	0.00159906217523229\\
88.48	0.00159749607697868\\
88.49	0.00159592976245387\\
88.5	0.0015943632322446\\
88.51	0.0015927964869386\\
88.52	0.00159122952712467\\
88.53	0.00158966235339261\\
88.54	0.00158809496633324\\
88.55	0.00158652736653842\\
88.56	0.00158495955460107\\
88.57	0.00158339153111507\\
88.58	0.0015818232966754\\
88.59	0.00158025485187803\\
88.6	0.00157868619731998\\
88.61	0.0015771173335993\\
88.62	0.00157554826131506\\
88.63	0.00157397898106738\\
88.64	0.00157240949345739\\
88.65	0.00157083979908729\\
88.66	0.00156926989856029\\
88.67	0.00156769979248064\\
88.68	0.00156612948145362\\
88.69	0.00156455896608555\\
88.7	0.0015629882469838\\
88.71	0.00156141732475676\\
88.72	0.00155984620001387\\
88.73	0.0015582748733656\\
88.74	0.00155670334542346\\
88.75	0.0015551316168\\
88.76	0.00155355968810881\\
88.77	0.00155198755996452\\
88.78	0.00155041523298281\\
88.79	0.00154884270778039\\
88.8	0.001547269984975\\
88.81	0.00154569706518546\\
88.82	0.00154412394903159\\
88.83	0.00154255063713428\\
88.84	0.00154097713011545\\
88.85	0.00153940342859808\\
88.86	0.00153782953320617\\
88.87	0.0015362554445648\\
88.88	0.00153468116330005\\
88.89	0.00153310669003908\\
88.9	0.00153153202541008\\
88.91	0.00152995717004231\\
88.92	0.00152838212456604\\
88.93	0.00152680688961261\\
88.94	0.00152523146581441\\
88.95	0.00152365585380487\\
88.96	0.00152208005421847\\
88.97	0.00152050406769074\\
88.98	0.00151892789485826\\
88.99	0.00151735153635866\\
89	0.00151577499283061\\
89.01	0.00151419826491386\\
89.02	0.00151262135324917\\
89.03	0.00151104425847838\\
89.04	0.00150946698124437\\
89.05	0.00150788952219108\\
89.06	0.0015063118819635\\
89.07	0.00150473406120766\\
89.08	0.00150315606057067\\
89.09	0.00150157788070066\\
89.1	0.00149999952224684\\
89.11	0.00149842098585946\\
89.12	0.00149684227218983\\
89.13	0.00149526338189033\\
89.14	0.00149368431561435\\
89.15	0.00149210507401638\\
89.16	0.00149052565775195\\
89.17	0.00148894606747764\\
89.18	0.00148736630385111\\
89.19	0.00148578636753103\\
89.2	0.00148420625917718\\
89.21	0.00148262597945036\\
89.22	0.00148104552901245\\
89.23	0.00147946490852637\\
89.24	0.00147788411865611\\
89.25	0.00147630316006671\\
89.26	0.00147472203342429\\
89.27	0.00147314073939598\\
89.28	0.00147155927865002\\
89.29	0.0014699776518557\\
89.3	0.00146839585968334\\
89.31	0.00146681390280435\\
89.32	0.00146523178189118\\
89.33	0.00146364949761737\\
89.34	0.00146206705065748\\
89.35	0.00146048444168716\\
89.36	0.00145890167138312\\
89.37	0.0014573187404231\\
89.38	0.00145573564948594\\
89.39	0.00145415239925152\\
89.4	0.0014525689904008\\
89.41	0.00145098542361577\\
89.42	0.00144940169957952\\
89.43	0.00144781781897617\\
89.44	0.00144623378249092\\
89.45	0.00144464959081004\\
89.46	0.00144306524462083\\
89.47	0.0014414807446117\\
89.48	0.00143989609147208\\
89.49	0.00143831128589248\\
89.5	0.00143672632856448\\
89.51	0.00143514122018072\\
89.52	0.00143355596143489\\
89.53	0.00143197055302177\\
89.54	0.00143038499563718\\
89.55	0.00142879928997802\\
89.56	0.00142721343674223\\
89.57	0.00142562743662885\\
89.58	0.00142404129033795\\
89.59	0.00142245499857069\\
89.6	0.00142086856202929\\
89.61	0.00141928198141701\\
89.62	0.0014176952574382\\
89.63	0.00141610839079828\\
89.64	0.0014145213822037\\
89.65	0.00141293423236202\\
89.66	0.00141134694198182\\
89.67	0.00140975951177278\\
89.68	0.00140817194244564\\
89.69	0.00140658423471218\\
89.7	0.00140499638928526\\
89.71	0.00140340840687882\\
89.72	0.00140182028820784\\
89.73	0.00140023203398838\\
89.74	0.00139864364493757\\
89.75	0.00139705512177357\\
89.76	0.00139546646521566\\
89.77	0.00139387767598415\\
89.78	0.0013922887548004\\
89.79	0.00139069970238687\\
89.8	0.00138911051946708\\
89.81	0.00138752120676558\\
89.82	0.00138593176500804\\
89.83	0.00138434219492114\\
89.84	0.00138275249723265\\
89.85	0.00138116267267141\\
89.86	0.00137957272196733\\
89.87	0.00137798264585134\\
89.88	0.00137639244505549\\
89.89	0.00137480212031287\\
89.9	0.00137321167235763\\
89.91	0.00137162110192497\\
89.92	0.0013700304097512\\
89.93	0.00136843959657365\\
89.94	0.00136684866313073\\
89.95	0.00136525761016191\\
89.96	0.00136366643840773\\
89.97	0.00136207514860979\\
89.98	0.00136048374151075\\
89.99	0.00135889221785432\\
90	0.0013573005783853\\
90.01	0.00135570882384954\\
90.02	0.00135411695499394\\
90.03	0.00135252497256649\\
90.04	0.00135093287731621\\
90.05	0.00134934066999319\\
90.06	0.0013477483513486\\
90.07	0.00134615592213466\\
90.08	0.00134456338310465\\
90.09	0.00134297073501289\\
90.1	0.0013413779786148\\
90.11	0.00133978511466683\\
90.12	0.00133819214392651\\
90.13	0.00133659906715241\\
90.14	0.00133500588510417\\
90.15	0.00133341259854249\\
90.16	0.00133181920822912\\
90.17	0.00133022571492689\\
90.18	0.00132863211939965\\
90.19	0.00132703842241236\\
90.2	0.00132544462473098\\
90.21	0.00132385072712257\\
90.22	0.00132225673035522\\
90.23	0.0013206626351981\\
90.24	0.00131906844242142\\
90.25	0.00131747415279646\\
90.26	0.00131587976709552\\
90.27	0.001314285286092\\
90.28	0.00131269071056033\\
90.29	0.001311096041276\\
90.3	0.00130950127901554\\
90.31	0.00130790642455656\\
90.32	0.0013063114786777\\
90.33	0.00130471644215867\\
90.34	0.00130312131578022\\
90.35	0.00130152610032414\\
90.36	0.0012999307965733\\
90.37	0.00129833540531161\\
90.38	0.00129673992732401\\
90.39	0.00129514436339652\\
90.4	0.00129354871431618\\
90.41	0.00129195298087111\\
90.42	0.00129035716385045\\
90.43	0.00128876126404441\\
90.44	0.00128716528224422\\
90.45	0.00128556921924219\\
90.46	0.00128397307583165\\
90.47	0.001282376852807\\
90.48	0.00128078055096364\\
90.49	0.00127918417109807\\
90.5	0.00127758771400779\\
90.51	0.00127599118049137\\
90.52	0.00127439457134841\\
90.53	0.00127279788737956\\
90.54	0.00127120112938651\\
90.55	0.00126960429817198\\
90.56	0.00126800739453974\\
90.57	0.00126641041929459\\
90.58	0.00126481337324239\\
90.59	0.00126321625719001\\
90.6	0.00126161907194538\\
90.61	0.00126002181831746\\
90.62	0.00125842449711624\\
90.63	0.00125682710915275\\
90.64	0.00125522965523905\\
90.65	0.00125363213618825\\
90.66	0.00125203455281447\\
90.67	0.00125043690593289\\
90.68	0.00124883919635969\\
90.69	0.0012472414249121\\
90.7	0.00124564359240839\\
90.71	0.00124404569966783\\
90.72	0.00124244774751074\\
90.73	0.00124084973675848\\
90.74	0.0012392516682334\\
90.75	0.0012376535427589\\
90.76	0.00123605536115941\\
90.77	0.00123445712426037\\
90.78	0.00123285883288826\\
90.79	0.00123126048787057\\
90.8	0.00122966209003581\\
90.81	0.00122806364021352\\
90.82	0.00122646513923426\\
90.83	0.0012248665879296\\
90.84	0.00122326798713214\\
90.85	0.00122166933767549\\
90.86	0.00122007064039428\\
90.87	0.00121847189612416\\
90.88	0.00121687310570179\\
90.89	0.00121527426996483\\
90.9	0.00121367538975197\\
90.91	0.00121207646590292\\
90.92	0.00121047749925838\\
90.93	0.00120887849066007\\
90.94	0.00120727944095072\\
90.95	0.00120568035097406\\
90.96	0.00120408122157483\\
90.97	0.00120248205359879\\
90.98	0.00120088284789268\\
90.99	0.00119928360530426\\
91	0.0011976843266823\\
91.01	0.00119608501287654\\
91.02	0.00119448566473775\\
91.03	0.00119288628311769\\
91.04	0.00119128686886911\\
91.05	0.00118968742284577\\
91.06	0.00118808794590243\\
91.07	0.00118648843889482\\
91.08	0.00118488890267968\\
91.09	0.00118328933811475\\
91.1	0.00118168974605874\\
91.11	0.00118009012737137\\
91.12	0.00117849048291334\\
91.13	0.00117689081354632\\
91.14	0.00117529112013301\\
91.15	0.00117369140353706\\
91.16	0.00117209166462309\\
91.17	0.00117049190425676\\
91.18	0.00116889212330465\\
91.19	0.00116729232263435\\
91.2	0.00116569250311443\\
91.21	0.00116409266561444\\
91.22	0.00116249281100487\\
91.23	0.00116089294015724\\
91.24	0.00115929305394399\\
91.25	0.00115769315323857\\
91.26	0.00115609323891539\\
91.27	0.00115449331184981\\
91.28	0.00115289337291818\\
91.29	0.00115129342299782\\
91.3	0.00114969346296698\\
91.31	0.00114809349370492\\
91.32	0.00114649351609183\\
91.33	0.00114489353100886\\
91.34	0.00114329353933814\\
91.35	0.00114169354196274\\
91.36	0.00114009353976669\\
91.37	0.00113849353363498\\
91.38	0.00113689352445353\\
91.39	0.00113529351310925\\
91.4	0.00113369350048996\\
91.41	0.00113209348748446\\
91.42	0.00113049347498248\\
91.43	0.00112889346387469\\
91.44	0.00112729345505272\\
91.45	0.00112569344940911\\
91.46	0.00112409344783739\\
91.47	0.00112249345123199\\
91.48	0.00112089346048828\\
91.49	0.00111929347650257\\
91.5	0.00111769350017211\\
91.51	0.00111609353239508\\
91.52	0.00111449357407057\\
91.53	0.00111289362609863\\
91.54	0.00111129368938021\\
91.55	0.0011096937648172\\
91.56	0.0011080938533124\\
91.57	0.00110649395576954\\
91.58	0.00110489407309326\\
91.59	0.00110329420618914\\
91.6	0.00110169435596364\\
91.61	0.00110009452332416\\
91.62	0.001098494709179\\
91.63	0.00109689491443738\\
91.64	0.00109529514000942\\
91.65	0.00109369538680614\\
91.66	0.00109209565573948\\
91.67	0.00109049594772227\\
91.68	0.00108889626366824\\
91.69	0.00108729660449202\\
91.7	0.00108569697110914\\
91.71	0.00108409736443602\\
91.72	0.00108249778538996\\
91.73	0.00108089823488919\\
91.74	0.00107929871385278\\
91.75	0.00107769922320072\\
91.76	0.00107609976385385\\
91.77	0.00107450033673393\\
91.78	0.00107290094276358\\
91.79	0.0010713015828663\\
91.8	0.00106970225796645\\
91.81	0.00106810296898929\\
91.82	0.00106650371686094\\
91.83	0.00106490450250838\\
91.84	0.00106330532685946\\
91.85	0.00106170619084291\\
91.86	0.0010601070953883\\
91.87	0.00105850804142607\\
91.88	0.00105690902988753\\
91.89	0.00105531006170481\\
91.9	0.00105371113781093\\
91.91	0.00105211225913975\\
91.92	0.00105051342662596\\
91.93	0.00104891464120512\\
91.94	0.00104731590381361\\
91.95	0.00104571721538868\\
91.96	0.00104411857686841\\
91.97	0.00104251998919168\\
91.98	0.00104092145329825\\
91.99	0.0010393229701287\\
92	0.00103772454062443\\
92.01	0.00103612616572766\\
92.02	0.00103452784638146\\
92.03	0.0010329295835297\\
92.04	0.00103133137811706\\
92.05	0.00102973323108907\\
92.06	0.00102813514339205\\
92.07	0.00102653711597312\\
92.08	0.00102493914978024\\
92.09	0.00102334124576216\\
92.1	0.00102174340486843\\
92.11	0.0010201456280494\\
92.12	0.00101854791625624\\
92.13	0.00101695027044089\\
92.14	0.00101535269155609\\
92.15	0.00101375518055538\\
92.16	0.00101215773839308\\
92.17	0.0010105603660243\\
92.18	0.00100896306440491\\
92.19	0.00100736583449161\\
92.2	0.00100576867724181\\
92.21	0.00100417159361375\\
92.22	0.00100257458456642\\
92.23	0.00100097765105957\\
92.24	0.000999380794053732\\
92.25	0.00099778401451019\\
92.26	0.000996187313390989\\
92.27	0.000994590691658935\\
92.28	0.000992994150277586\\
92.29	0.000991397690211244\\
92.3	0.000989801312424973\\
92.31	0.00098820501788457\\
92.32	0.000986608807556582\\
92.33	0.000985012682408293\\
92.34	0.000983416643407715\\
92.35	0.000981820691523611\\
92.36	0.000980224827725454\\
92.37	0.000978629052983454\\
92.38	0.00097703336826855\\
92.39	0.000975437774552383\\
92.4	0.000973842272807333\\
92.41	0.000972246864006482\\
92.42	0.000970651549123623\\
92.43	0.00096905632913326\\
92.44	0.000967461205010603\\
92.45	0.000965866177731556\\
92.46	0.00096427124827273\\
92.47	0.000962676417611429\\
92.48	0.000961081686725644\\
92.49	0.000959487056594053\\
92.5	0.000957892528196029\\
92.51	0.000956298102511608\\
92.52	0.000954703780521522\\
92.53	0.000953109563207167\\
92.54	0.000951515451550618\\
92.55	0.000949921446534607\\
92.56	0.000948327549142534\\
92.57	0.000946733760358467\\
92.58	0.00094514008116712\\
92.59	0.000943546512553867\\
92.6	0.000941953055504732\\
92.61	0.000940359711006379\\
92.62	0.000938766480046115\\
92.63	0.000937173363611899\\
92.64	0.000935580362692307\\
92.65	0.000933987478276561\\
92.66	0.000932394711354494\\
92.67	0.000930802062916588\\
92.68	0.000929209533953918\\
92.69	0.000927617125458196\\
92.7	0.000926024838421731\\
92.71	0.000924432673837452\\
92.72	0.000922840632698893\\
92.73	0.000921248716000179\\
92.74	0.000919656924736039\\
92.75	0.000918065259901799\\
92.76	0.00091647372249336\\
92.77	0.000914882313507228\\
92.78	0.000913291033940467\\
92.79	0.000911699884790741\\
92.8	0.000910108867056275\\
92.81	0.000908517981735866\\
92.82	0.000906927229828868\\
92.83	0.000905336612335201\\
92.84	0.000903746130255354\\
92.85	0.000902155784590344\\
92.86	0.00090056557634176\\
92.87	0.000898975506511719\\
92.88	0.000897385576102883\\
92.89	0.000895795786118448\\
92.9	0.000894206137562139\\
92.91	0.000892616631438218\\
92.92	0.000891027268751461\\
92.93	0.000889438050507155\\
92.94	0.000887848977711119\\
92.95	0.000886260051369664\\
92.96	0.000884671272489613\\
92.97	0.000883082642078287\\
92.98	0.000881494161143507\\
92.99	0.000879905830693582\\
93	0.000878317651737305\\
93.01	0.000876729625283948\\
93.02	0.000875141752343276\\
93.03	0.000873554033925509\\
93.04	0.000871966471041343\\
93.05	0.000870379064701934\\
93.06	0.000868791815918897\\
93.07	0.000867204725704307\\
93.08	0.000865617795070675\\
93.09	0.00086403102503097\\
93.1	0.000862444416598589\\
93.11	0.000860857970787372\\
93.12	0.000859271688611581\\
93.13	0.000857685571085901\\
93.14	0.000856099619225449\\
93.15	0.000854513834045743\\
93.16	0.00085292821656271\\
93.17	0.000851342767792699\\
93.18	0.000849757488752432\\
93.19	0.000848172380459046\\
93.2	0.000846587443930062\\
93.21	0.000845002680183374\\
93.22	0.000843418090237271\\
93.23	0.0008418336751104\\
93.24	0.000840249435821784\\
93.25	0.000838665373390814\\
93.26	0.000837081488837222\\
93.27	0.000835497783181113\\
93.28	0.000833914257442918\\
93.29	0.00083233091264342\\
93.3	0.000830747749803734\\
93.31	0.000829164769945316\\
93.32	0.000827581974089933\\
93.33	0.000825999363259671\\
93.34	0.000824416938476938\\
93.35	0.00082283470076445\\
93.36	0.000821252651145217\\
93.37	0.000819670790642548\\
93.38	0.000818089120280053\\
93.39	0.000816507641081611\\
93.4	0.000814926354071389\\
93.41	0.000813345260273829\\
93.42	0.000811764360713637\\
93.43	0.000810183656415782\\
93.44	0.000808603148405487\\
93.45	0.000807022837708228\\
93.46	0.000805442725349723\\
93.47	0.000803862812355923\\
93.48	0.000802283099753024\\
93.49	0.000800703588567436\\
93.5	0.000799124279825784\\
93.51	0.000797545174554923\\
93.52	0.000795966273781904\\
93.53	0.00079438757853398\\
93.54	0.000792809089838606\\
93.55	0.000791230808723405\\
93.56	0.00078965273621621\\
93.57	0.000788074873345005\\
93.58	0.000786497221137957\\
93.59	0.000784919780623399\\
93.6	0.000783342552829806\\
93.61	0.000781765538785813\\
93.62	0.000780188739520194\\
93.63	0.000778612156061865\\
93.64	0.000777035789439872\\
93.65	0.000775459640683382\\
93.66	0.000773883710821689\\
93.67	0.000772308000884184\\
93.68	0.000770732511900388\\
93.69	0.00076915724489989\\
93.7	0.000767582200912402\\
93.71	0.000766007380967706\\
93.72	0.000764432786095663\\
93.73	0.000762858417326219\\
93.74	0.00076128427568937\\
93.75	0.000759710362215187\\
93.76	0.000758136677933785\\
93.77	0.000756563223875331\\
93.78	0.000754990001070028\\
93.79	0.000753417010548115\\
93.8	0.000751844253339854\\
93.81	0.000750271730475524\\
93.82	0.000748699442985428\\
93.83	0.000747127391899855\\
93.84	0.000745555578249108\\
93.85	0.000743984003063481\\
93.86	0.000742412667373236\\
93.87	0.000740841572208628\\
93.88	0.000739270718599874\\
93.89	0.00073770010757716\\
93.9	0.00073612974017062\\
93.91	0.000734559617410334\\
93.92	0.000732989740326331\\
93.93	0.000731420109948568\\
93.94	0.000729850727306931\\
93.95	0.000728281593431213\\
93.96	0.000726712709351133\\
93.97	0.000725144076096303\\
93.98	0.00072357569469623\\
93.99	0.00072200756618032\\
94	0.000720439691577838\\
94.01	0.000718872071917943\\
94.02	0.000717304708229644\\
94.03	0.000715737601541812\\
94.04	0.000714170752883164\\
94.05	0.000712604163282258\\
94.06	0.000711037833767489\\
94.07	0.00070947176536707\\
94.08	0.000707905959109035\\
94.09	0.000706340416021219\\
94.1	0.000704775137131269\\
94.11	0.000703210123466611\\
94.12	0.000701645376054464\\
94.13	0.000700080895921816\\
94.14	0.000698516684095421\\
94.15	0.000696952741601796\\
94.16	0.000695389069467209\\
94.17	0.000693825668717658\\
94.18	0.000692262540378885\\
94.19	0.000690699685476347\\
94.2	0.000689137105035225\\
94.21	0.000687574800080399\\
94.22	0.000686012771636447\\
94.23	0.000684451020727639\\
94.24	0.000682889548377924\\
94.25	0.000681328355610912\\
94.26	0.000679767443449886\\
94.27	0.000678206812917777\\
94.28	0.000676646465037156\\
94.29	0.000675086400830237\\
94.3	0.000673526621318846\\
94.31	0.000671967127524426\\
94.32	0.000670407920468036\\
94.33	0.000668849001170325\\
94.34	0.000667290370651523\\
94.35	0.000665732029931441\\
94.36	0.000664173980029459\\
94.37	0.000662616221964516\\
94.38	0.00066105875675509\\
94.39	0.000659501585419201\\
94.4	0.000657944708974402\\
94.41	0.000656388128437759\\
94.42	0.000654831844825846\\
94.43	0.000653275859154747\\
94.44	0.000651720172440032\\
94.45	0.000650164785696752\\
94.46	0.000648609699939446\\
94.47	0.000647054916182102\\
94.48	0.000645500435438179\\
94.49	0.000643946258720574\\
94.5	0.000642392387041635\\
94.51	0.00064083882141313\\
94.52	0.000639285562846252\\
94.53	0.000637732612351607\\
94.54	0.000636179970939203\\
94.55	0.00063462763961844\\
94.56	0.000633075619398106\\
94.57	0.000631523911286366\\
94.58	0.000629972516290747\\
94.59	0.000628421435418134\\
94.6	0.000626870669674763\\
94.61	0.000625320220066198\\
94.62	0.000623770087597354\\
94.63	0.000622220273272437\\
94.64	0.000620670778094985\\
94.65	0.000619121603067829\\
94.66	0.000617572749193088\\
94.67	0.000616024217472168\\
94.68	0.000614476008905745\\
94.69	0.000612928124493756\\
94.7	0.000611380565235393\\
94.71	0.000609833332129087\\
94.72	0.000608286426172512\\
94.73	0.000606739848362554\\
94.74	0.00060519359969532\\
94.75	0.000603647681166121\\
94.76	0.000602102093769459\\
94.77	0.000600556838499026\\
94.78	0.000599011916347685\\
94.79	0.000597467328307464\\
94.8	0.000595923075369546\\
94.81	0.000594379158524264\\
94.82	0.000592835578761077\\
94.83	0.000591292337068576\\
94.84	0.000589749434434466\\
94.85	0.000588206871845553\\
94.86	0.000586664650287739\\
94.87	0.000585122770746016\\
94.88	0.000583581234204442\\
94.89	0.000582040041646142\\
94.9	0.000580499194053298\\
94.91	0.000578958692407129\\
94.92	0.000577418537687894\\
94.93	0.000575878730874864\\
94.94	0.000574339272946338\\
94.95	0.000572800164879605\\
94.96	0.000571261407650943\\
94.97	0.000569723002235616\\
94.98	0.000568184949607862\\
94.99	0.000566647250740867\\
95	0.000565109906606775\\
95.01	0.000563572918176667\\
95.02	0.000562036286420544\\
95.03	0.000560500012307329\\
95.04	0.000558964096804856\\
95.05	0.000557428540879845\\
95.06	0.000555893345497901\\
95.07	0.000554358511623511\\
95.08	0.000552824040220007\\
95.09	0.000551289932249587\\
95.1	0.000549756188673286\\
95.11	0.000548222810450959\\
95.12	0.0005466897985413\\
95.13	0.000545157153901783\\
95.14	0.000543624877488698\\
95.15	0.000542092970257107\\
95.16	0.000540561433160851\\
95.17	0.00053903026715253\\
95.18	0.000537499473183505\\
95.19	0.000535969052203858\\
95.2	0.000534439005162413\\
95.21	0.000532909333006707\\
95.22	0.000531380036682979\\
95.23	0.00052985111713616\\
95.24	0.000528322575309869\\
95.25	0.000526794412146389\\
95.26	0.000525266628586669\\
95.27	0.000523739225570299\\
95.28	0.000522212204035503\\
95.29	0.000520685564919136\\
95.3	0.000519159309156655\\
95.31	0.000517633437682128\\
95.32	0.000516107951428204\\
95.33	0.000514582851326105\\
95.34	0.000513058138305626\\
95.35	0.000511533813295105\\
95.36	0.000510009877221435\\
95.37	0.000508486331010022\\
95.38	0.000506963175584792\\
95.39	0.000505440411868183\\
95.4	0.000503918040781115\\
95.41	0.000502396063242995\\
95.42	0.000500874480171692\\
95.43	0.000499353292483537\\
95.44	0.000497832501093297\\
95.45	0.000496312106914174\\
95.46	0.000494792110857787\\
95.47	0.00049327251383416\\
95.48	0.000491753316751718\\
95.49	0.000490234520517253\\
95.5	0.000488716126035937\\
95.51	0.000487198134211294\\
95.52	0.000485680545945189\\
95.53	0.000484163362137824\\
95.54	0.000482646583687714\\
95.55	0.000481130211491676\\
95.56	0.000479614246444831\\
95.57	0.00047809868944057\\
95.58	0.00047658354137055\\
95.59	0.000475068803124692\\
95.6	0.000473554475591146\\
95.61	0.000472040559656297\\
95.62	0.000470527056204746\\
95.63	0.000469013966119293\\
95.64	0.000467501290280932\\
95.65	0.000465989029568825\\
95.66	0.000464477184860302\\
95.67	0.000462965757030845\\
95.68	0.00046145474695407\\
95.69	0.000459944155501711\\
95.7	0.000458433983543621\\
95.71	0.000456924231947747\\
95.72	0.000455414901580114\\
95.73	0.000453905993304821\\
95.74	0.000452397507984027\\
95.75	0.000450889446477927\\
95.76	0.000449381809644749\\
95.77	0.000447874598340739\\
95.78	0.000446367813420142\\
95.79	0.000444861455735189\\
95.8	0.000443355526136094\\
95.81	0.000441850025471027\\
95.82	0.000440344954586105\\
95.83	0.000438840314325386\\
95.84	0.000437336105530832\\
95.85	0.000435832329042332\\
95.86	0.000434328985697651\\
95.87	0.000432826076332436\\
95.88	0.000431323601780206\\
95.89	0.00042982156287232\\
95.9	0.000428319960437981\\
95.91	0.000426818795304209\\
95.92	0.000425318068295833\\
95.93	0.000423817780235479\\
95.94	0.00042231793194355\\
95.95	0.000420818524238213\\
95.96	0.000419319557935392\\
95.97	0.00041782103384874\\
95.98	0.000416322952789641\\
95.99	0.000414825315567179\\
96	0.000413328122988136\\
96.01	0.000411831375856974\\
96.02	0.000410335074975815\\
96.03	0.000408839221144439\\
96.04	0.000407343815160255\\
96.05	0.000405848857818291\\
96.06	0.000404354349911191\\
96.07	0.000402860292229185\\
96.08	0.000401366685560076\\
96.09	0.000399873530689238\\
96.1	0.00039838082839958\\
96.11	0.000396888579471557\\
96.12	0.000395396784683139\\
96.13	0.000393905444809788\\
96.14	0.000392414560624462\\
96.15	0.000390924132897597\\
96.16	0.000389434162397074\\
96.17	0.000387944649888229\\
96.18	0.000386455596133817\\
96.19	0.000384967001894007\\
96.2	0.000383478867926374\\
96.21	0.000381991194985857\\
96.22	0.000380503983824785\\
96.23	0.000379017235192821\\
96.24	0.000377530949836967\\
96.25	0.000376045128501553\\
96.26	0.000374559771928209\\
96.27	0.000373074880855856\\
96.28	0.000371590456020693\\
96.29	0.000370106498156169\\
96.3	0.000368623007992988\\
96.31	0.000367139986259077\\
96.32	0.000365657433679575\\
96.33	0.000364175350976817\\
96.34	0.000362693738870318\\
96.35	0.000361212598076765\\
96.36	0.000359731929309989\\
96.37	0.000358251733280952\\
96.38	0.000356772010697746\\
96.39	0.000355292762265555\\
96.4	0.00035381398868665\\
96.41	0.000352335690660373\\
96.42	0.000350857868883123\\
96.43	0.000349380524048339\\
96.44	0.000347903656846478\\
96.45	0.000346427267965002\\
96.46	0.000344951358088371\\
96.47	0.000343475927898011\\
96.48	0.000342000978072307\\
96.49	0.000340526509286595\\
96.5	0.000339052522213125\\
96.51	0.000337579017521062\\
96.52	0.000336105995876461\\
96.53	0.000334633457942256\\
96.54	0.000333161404378243\\
96.55	0.000331689835841054\\
96.56	0.000330218752984163\\
96.57	0.000328748156457837\\
96.58	0.000327278046909149\\
96.59	0.000325808424981946\\
96.6	0.000324339291316839\\
96.61	0.000322870646551185\\
96.62	0.00032140249131906\\
96.63	0.000319934826251261\\
96.64	0.000318467651975275\\
96.65	0.00031700096911527\\
96.66	0.000315534778292072\\
96.67	0.000314069080123155\\
96.68	0.000312603875222615\\
96.69	0.000311139164201165\\
96.7	0.000309674947666105\\
96.71	0.00030821122622132\\
96.72	0.00030674800046725\\
96.73	0.000305285271000879\\
96.74	0.000303823038415717\\
96.75	0.000302361303301779\\
96.76	0.000300900066245578\\
96.77	0.000299439327830102\\
96.78	0.000297979088634789\\
96.79	0.000296519349235524\\
96.8	0.000295060110204614\\
96.81	0.000293601372110774\\
96.82	0.000292143135519099\\
96.83	0.000290685400991064\\
96.84	0.000289228169084501\\
96.85	0.000287771440353564\\
96.86	0.000286315215348741\\
96.87	0.000284859494616816\\
96.88	0.00028340427870086\\
96.89	0.000281949568140209\\
96.9	0.000280495363470445\\
96.91	0.000279041665223393\\
96.92	0.000277588473927079\\
96.93	0.000276135790105735\\
96.94	0.000274683614279772\\
96.95	0.000273231946965756\\
96.96	0.000271780788676408\\
96.97	0.000270330139920567\\
96.98	0.000268880001203181\\
96.99	0.00026743037302529\\
97	0.000265981255884012\\
97.01	0.000264532650272513\\
97.02	0.000263084556680001\\
97.03	0.000261636975591705\\
97.04	0.000260189907488851\\
97.05	0.00025874335284865\\
97.06	0.000257297312144286\\
97.07	0.000255851785844882\\
97.08	0.000254406774415497\\
97.09	0.000252962278317099\\
97.1	0.000251518298006551\\
97.11	0.000250074833936595\\
97.12	0.000248631886555828\\
97.13	0.000247189456308685\\
97.14	0.000245747543635431\\
97.15	0.000244306148972124\\
97.16	0.000242865272750616\\
97.17	0.000241424915398523\\
97.18	0.000239985077339211\\
97.19	0.000238545758991779\\
97.2	0.000237106960771033\\
97.21	0.00023566868308748\\
97.22	0.000234230926347299\\
97.23	0.000232793690952332\\
97.24	0.000231356977300054\\
97.25	0.000229920785783565\\
97.26	0.000228485116791568\\
97.27	0.000227049970708354\\
97.28	0.000225615347913774\\
97.29	0.000224181248783229\\
97.3	0.000222747673687651\\
97.31	0.000221314622993478\\
97.32	0.000219882097062652\\
97.33	0.000218450096252574\\
97.34	0.000217018620916113\\
97.35	0.000215587671401569\\
97.36	0.00021415724805266\\
97.37	0.000212727351208506\\
97.38	0.000211297981203605\\
97.39	0.000209869138367827\\
97.4	0.000208440823026375\\
97.41	0.000207013035499783\\
97.42	0.000205585776103889\\
97.43	0.000204159045149825\\
97.44	0.00020273284294399\\
97.45	0.00020130716978803\\
97.46	0.000199882025978827\\
97.47	0.000198457411808479\\
97.48	0.000197033327564275\\
97.49	0.000195609773528678\\
97.5	0.000194186749979319\\
97.51	0.000192764257188956\\
97.52	0.000191342295425471\\
97.53	0.000189920864951851\\
97.54	0.000188499966026162\\
97.55	0.000187079598901536\\
97.56	0.000185659763826144\\
97.57	0.000184240461043194\\
97.58	0.000182821690790889\\
97.59	0.00018140345330243\\
97.6	0.000179985748805987\\
97.61	0.000178568577524673\\
97.62	0.000177151939676542\\
97.63	0.000175735835474559\\
97.64	0.000174320265126579\\
97.65	0.000172905228835341\\
97.66	0.000171490726798432\\
97.67	0.000170076759208283\\
97.68	0.000168663326252141\\
97.69	0.000167250428112055\\
97.7	0.000165838064964854\\
97.71	0.000164426236982132\\
97.72	0.000163014944330226\\
97.73	0.00016160418717019\\
97.74	0.0001601939656578\\
97.75	0.000158784279943503\\
97.76	0.000157375130172418\\
97.77	0.000155966516484321\\
97.78	0.000154558439013609\\
97.79	0.000153150897889297\\
97.8	0.000151743893235065\\
97.81	0.000150337425169305\\
97.82	0.000148931493805094\\
97.83	0.000147526099250173\\
97.84	0.000146121241606926\\
97.85	0.000144716920972355\\
97.86	0.000143313137438069\\
97.87	0.000141909891090248\\
97.88	0.000140507182009619\\
97.89	0.000139105010271498\\
97.9	0.000137703375946119\\
97.91	0.000136302279099288\\
97.92	0.000134901719792349\\
97.93	0.000133501697794601\\
97.94	0.000132102212688756\\
97.95	0.000130703264054717\\
97.96	0.000129304851469616\\
97.97	0.000127906974507885\\
97.98	0.000126509632741294\\
97.99	0.000125112825739012\\
98	0.000123716553067654\\
98.01	0.000122320814127108\\
98.02	0.000120925606898312\\
98.03	0.000119530927353886\\
98.04	0.000118136771421792\\
98.05	0.000116743134964492\\
98.06	0.000115350013773462\\
98.07	0.00011395740359667\\
98.08	0.000112565300138151\\
98.09	0.000111173699057597\\
98.1	0.000109783298154441\\
98.11	0.000108394990769643\\
98.12	0.000107008779600988\\
98.13	0.000105624667346543\\
98.14	0.000104242656704808\\
98.15	0.000102862750374865\\
98.16	0.000101485055938421\\
98.17	0.000100110162105732\\
98.18	9.87419653347837e-05\\
98.19	9.73879133488714e-05\\
98.2	9.60481236720965e-05\\
98.21	9.47227150263439e-05\\
98.22	9.34118073461897e-05\\
98.23	9.21155217940544e-05\\
98.24	9.08339807756172e-05\\
98.25	8.95673079554585e-05\\
98.26	8.83156282729721e-05\\
98.27	8.70790679585402e-05\\
98.28	8.58577545499572e-05\\
98.29	8.46518169091651e-05\\
98.3	8.34613852392189e-05\\
98.31	8.22865911015783e-05\\
98.32	8.11275674336698e-05\\
98.33	7.99843285393433e-05\\
98.34	7.88569989661515e-05\\
98.35	7.77443087035621e-05\\
98.36	7.66463885000131e-05\\
98.37	7.55633711730444e-05\\
98.38	7.44953909389782e-05\\
98.39	7.34425834309806e-05\\
98.4	7.240479691842e-05\\
98.41	7.13808484974505e-05\\
98.42	7.03621857074555e-05\\
98.43	6.93488396395318e-05\\
98.44	6.83408415566283e-05\\
98.45	6.73382228926957e-05\\
98.46	6.63410152517627e-05\\
98.47	6.53492504069479e-05\\
98.48	6.43629602993835e-05\\
98.49	6.33821770370628e-05\\
98.5	6.24069328936123e-05\\
98.51	6.14372603069801e-05\\
98.52	6.04731918780309e-05\\
98.53	5.9514760369056e-05\\
98.54	5.85619987021856e-05\\
98.55	5.76149399577237e-05\\
98.56	5.66736175449424e-05\\
98.57	5.57380663403235e-05\\
98.58	5.4808321408719e-05\\
98.59	5.38844180020277e-05\\
98.6	5.29663915577877e-05\\
98.61	5.20542776976576e-05\\
98.62	5.11481122257972e-05\\
98.63	5.02479311374291e-05\\
98.64	4.93537706873076e-05\\
98.65	4.84656673069971e-05\\
98.66	4.75836576027955e-05\\
98.67	4.67077783535674e-05\\
98.68	4.58380665084041e-05\\
98.69	4.49745591842039e-05\\
98.7	4.41172936624448e-05\\
98.71	4.32663073863318e-05\\
98.72	4.24216379579217e-05\\
98.73	4.15833231351189e-05\\
98.74	4.07514016388874e-05\\
98.75	3.9925913233603e-05\\
98.76	3.91068978568884e-05\\
98.77	3.82943956165217e-05\\
98.78	3.74884470218211e-05\\
98.79	3.66890938656855e-05\\
98.8	3.58963901455397e-05\\
98.81	3.51103903651184e-05\\
98.82	3.43311495393447e-05\\
98.83	3.35587231993055e-05\\
98.84	3.27931673972433e-05\\
98.85	3.2034538711612e-05\\
98.86	3.12828942521919e-05\\
98.87	3.05382916652355e-05\\
98.88	2.98007891386867e-05\\
98.89	2.90704454074479e-05\\
98.9	2.83473197586898e-05\\
98.91	2.76314720372376e-05\\
98.92	2.69229626470337e-05\\
98.93	2.62218525559875e-05\\
98.94	2.5528203301298e-05\\
98.95	2.48420769948503e-05\\
98.96	2.41635363286679e-05\\
98.97	2.34926445803856e-05\\
98.98	2.28294656188094e-05\\
98.99	2.21740639095094e-05\\
99	2.15265045204697e-05\\
99.01	2.08868531277918e-05\\
99.02	2.02551760214512e-05\\
99.03	1.96315401111164e-05\\
99.04	1.90160129319994e-05\\
99.05	1.84086626507998e-05\\
99.06	1.78095580716676e-05\\
99.07	1.7218768642243e-05\\
99.08	1.66363644597527e-05\\
99.09	1.60624162771558e-05\\
99.1	1.54969955093523e-05\\
99.11	1.49401742394594e-05\\
99.12	1.43920252251226e-05\\
99.13	1.38526219049216e-05\\
99.14	1.33220384048085e-05\\
99.15	1.28003495446158e-05\\
99.16	1.22876308446349e-05\\
99.17	1.1783958532248e-05\\
99.18	1.12894095486218e-05\\
99.19	1.08040673361641e-05\\
99.2	1.03280183761673e-05\\
99.21	9.86134997679744e-06\\
99.22	9.40415028095366e-06\\
99.23	8.95650827419971e-06\\
99.24	8.51851379275563e-06\\
99.25	8.09025753160421e-06\\
99.26	7.67183105262158e-06\\
99.27	7.26332679283626e-06\\
99.28	6.86483807273673e-06\\
99.29	6.47645910466059e-06\\
99.3	6.09828500128598e-06\\
99.31	5.73041178416993e-06\\
99.32	5.37293639239766e-06\\
99.33	5.02595669130655e-06\\
99.34	4.68957148128807e-06\\
99.35	4.36388050667827e-06\\
99.36	4.04898446473498e-06\\
99.37	3.74498501470345e-06\\
99.38	3.45198478696705e-06\\
99.39	3.17008739210423e-06\\
99.4	2.89939742239079e-06\\
99.41	2.64002046114487e-06\\
99.42	2.39206309212044e-06\\
99.43	2.15563290903095e-06\\
99.44	1.93083852514583e-06\\
99.45	1.71778958298931e-06\\
99.46	1.51659676413639e-06\\
99.47	1.32737179910601e-06\\
99.48	1.15022747734263e-06\\
99.49	9.85277657314029e-07\\
99.5	8.32637276701118e-07\\
99.51	6.92422362690015e-07\\
99.52	5.64750042368264e-07\\
99.53	4.49738553228579e-07\\
99.54	3.47507253785351e-07\\
99.55	2.58176634277893e-07\\
99.56	1.81868327514198e-07\\
99.57	1.18705119791021e-07\\
99.58	6.88109619700894e-08\\
99.59	3.23109806184968e-08\\
99.6	9.33148930348793e-09\\
99.61	0\\
99.62	0\\
99.63	0\\
99.64	0\\
99.65	0\\
99.66	0\\
99.67	0\\
99.68	0\\
99.69	0\\
99.7	0\\
99.71	0\\
99.72	0\\
99.73	0\\
99.74	0\\
99.75	0\\
99.76	0\\
99.77	0\\
99.78	0\\
99.79	0\\
99.8	0\\
99.81	0\\
99.82	0\\
99.83	0\\
99.84	0\\
99.85	0\\
99.86	0\\
99.87	0\\
99.88	0\\
99.89	0\\
99.9	0\\
99.91	0\\
99.92	0\\
99.93	0\\
99.94	0\\
99.95	0\\
99.96	0\\
99.97	0\\
99.98	0\\
99.99	0\\
100	0\\
};
\addlegendentry{$q=-3$};

\addplot [color=red,dashed,forget plot]
  table[row sep=crcr]{%
0.01	0.01\\
0.02	0.01\\
0.03	0.01\\
0.04	0.01\\
0.05	0.01\\
0.06	0.01\\
0.07	0.01\\
0.08	0.01\\
0.09	0.01\\
0.1	0.01\\
0.11	0.01\\
0.12	0.01\\
0.13	0.01\\
0.14	0.01\\
0.15	0.01\\
0.16	0.01\\
0.17	0.01\\
0.18	0.01\\
0.19	0.01\\
0.2	0.01\\
0.21	0.01\\
0.22	0.01\\
0.23	0.01\\
0.24	0.01\\
0.25	0.01\\
0.26	0.01\\
0.27	0.01\\
0.28	0.01\\
0.29	0.01\\
0.3	0.01\\
0.31	0.01\\
0.32	0.01\\
0.33	0.01\\
0.34	0.01\\
0.35	0.01\\
0.36	0.01\\
0.37	0.01\\
0.38	0.01\\
0.39	0.01\\
0.4	0.01\\
0.41	0.01\\
0.42	0.01\\
0.43	0.01\\
0.44	0.01\\
0.45	0.01\\
0.46	0.01\\
0.47	0.01\\
0.48	0.01\\
0.49	0.01\\
0.5	0.01\\
0.51	0.01\\
0.52	0.01\\
0.53	0.01\\
0.54	0.01\\
0.55	0.01\\
0.56	0.01\\
0.57	0.01\\
0.58	0.01\\
0.59	0.01\\
0.6	0.01\\
0.61	0.01\\
0.62	0.01\\
0.63	0.01\\
0.64	0.01\\
0.65	0.01\\
0.66	0.01\\
0.67	0.01\\
0.68	0.01\\
0.69	0.01\\
0.7	0.01\\
0.71	0.01\\
0.72	0.01\\
0.73	0.01\\
0.74	0.01\\
0.75	0.01\\
0.76	0.01\\
0.77	0.01\\
0.78	0.01\\
0.79	0.01\\
0.8	0.01\\
0.81	0.01\\
0.82	0.01\\
0.83	0.01\\
0.84	0.01\\
0.85	0.01\\
0.86	0.01\\
0.87	0.01\\
0.88	0.01\\
0.89	0.01\\
0.9	0.01\\
0.91	0.01\\
0.92	0.01\\
0.93	0.01\\
0.94	0.01\\
0.95	0.01\\
0.96	0.01\\
0.97	0.01\\
0.98	0.01\\
0.99	0.01\\
1	0.01\\
1.01	0.01\\
1.02	0.01\\
1.03	0.01\\
1.04	0.01\\
1.05	0.01\\
1.06	0.01\\
1.07	0.01\\
1.08	0.01\\
1.09	0.01\\
1.1	0.01\\
1.11	0.01\\
1.12	0.01\\
1.13	0.01\\
1.14	0.01\\
1.15	0.01\\
1.16	0.01\\
1.17	0.01\\
1.18	0.01\\
1.19	0.01\\
1.2	0.01\\
1.21	0.01\\
1.22	0.01\\
1.23	0.01\\
1.24	0.01\\
1.25	0.01\\
1.26	0.01\\
1.27	0.01\\
1.28	0.01\\
1.29	0.01\\
1.3	0.01\\
1.31	0.01\\
1.32	0.01\\
1.33	0.01\\
1.34	0.01\\
1.35	0.01\\
1.36	0.01\\
1.37	0.01\\
1.38	0.01\\
1.39	0.01\\
1.4	0.01\\
1.41	0.01\\
1.42	0.01\\
1.43	0.01\\
1.44	0.01\\
1.45	0.01\\
1.46	0.01\\
1.47	0.01\\
1.48	0.01\\
1.49	0.01\\
1.5	0.01\\
1.51	0.01\\
1.52	0.01\\
1.53	0.01\\
1.54	0.01\\
1.55	0.01\\
1.56	0.01\\
1.57	0.01\\
1.58	0.01\\
1.59	0.01\\
1.6	0.01\\
1.61	0.01\\
1.62	0.01\\
1.63	0.01\\
1.64	0.01\\
1.65	0.01\\
1.66	0.01\\
1.67	0.01\\
1.68	0.01\\
1.69	0.01\\
1.7	0.01\\
1.71	0.01\\
1.72	0.01\\
1.73	0.01\\
1.74	0.01\\
1.75	0.01\\
1.76	0.01\\
1.77	0.01\\
1.78	0.01\\
1.79	0.01\\
1.8	0.01\\
1.81	0.01\\
1.82	0.01\\
1.83	0.01\\
1.84	0.01\\
1.85	0.01\\
1.86	0.01\\
1.87	0.01\\
1.88	0.01\\
1.89	0.01\\
1.9	0.01\\
1.91	0.01\\
1.92	0.01\\
1.93	0.01\\
1.94	0.01\\
1.95	0.01\\
1.96	0.01\\
1.97	0.01\\
1.98	0.01\\
1.99	0.01\\
2	0.01\\
2.01	0.01\\
2.02	0.01\\
2.03	0.01\\
2.04	0.01\\
2.05	0.01\\
2.06	0.01\\
2.07	0.01\\
2.08	0.01\\
2.09	0.01\\
2.1	0.01\\
2.11	0.01\\
2.12	0.01\\
2.13	0.01\\
2.14	0.01\\
2.15	0.01\\
2.16	0.01\\
2.17	0.01\\
2.18	0.01\\
2.19	0.01\\
2.2	0.01\\
2.21	0.01\\
2.22	0.01\\
2.23	0.01\\
2.24	0.01\\
2.25	0.01\\
2.26	0.01\\
2.27	0.01\\
2.28	0.01\\
2.29	0.01\\
2.3	0.01\\
2.31	0.01\\
2.32	0.01\\
2.33	0.01\\
2.34	0.01\\
2.35	0.01\\
2.36	0.01\\
2.37	0.01\\
2.38	0.01\\
2.39	0.01\\
2.4	0.01\\
2.41	0.01\\
2.42	0.01\\
2.43	0.01\\
2.44	0.01\\
2.45	0.01\\
2.46	0.01\\
2.47	0.01\\
2.48	0.01\\
2.49	0.01\\
2.5	0.01\\
2.51	0.01\\
2.52	0.01\\
2.53	0.01\\
2.54	0.01\\
2.55	0.01\\
2.56	0.01\\
2.57	0.01\\
2.58	0.01\\
2.59	0.01\\
2.6	0.01\\
2.61	0.01\\
2.62	0.01\\
2.63	0.01\\
2.64	0.01\\
2.65	0.01\\
2.66	0.01\\
2.67	0.01\\
2.68	0.01\\
2.69	0.01\\
2.7	0.01\\
2.71	0.01\\
2.72	0.01\\
2.73	0.01\\
2.74	0.01\\
2.75	0.01\\
2.76	0.01\\
2.77	0.01\\
2.78	0.01\\
2.79	0.01\\
2.8	0.01\\
2.81	0.01\\
2.82	0.01\\
2.83	0.01\\
2.84	0.01\\
2.85	0.01\\
2.86	0.01\\
2.87	0.01\\
2.88	0.01\\
2.89	0.01\\
2.9	0.01\\
2.91	0.01\\
2.92	0.01\\
2.93	0.01\\
2.94	0.01\\
2.95	0.01\\
2.96	0.01\\
2.97	0.01\\
2.98	0.01\\
2.99	0.01\\
3	0.01\\
3.01	0.01\\
3.02	0.01\\
3.03	0.01\\
3.04	0.01\\
3.05	0.01\\
3.06	0.01\\
3.07	0.01\\
3.08	0.01\\
3.09	0.01\\
3.1	0.01\\
3.11	0.01\\
3.12	0.01\\
3.13	0.01\\
3.14	0.01\\
3.15	0.01\\
3.16	0.01\\
3.17	0.01\\
3.18	0.01\\
3.19	0.01\\
3.2	0.01\\
3.21	0.01\\
3.22	0.01\\
3.23	0.01\\
3.24	0.01\\
3.25	0.01\\
3.26	0.01\\
3.27	0.01\\
3.28	0.01\\
3.29	0.01\\
3.3	0.01\\
3.31	0.01\\
3.32	0.01\\
3.33	0.01\\
3.34	0.01\\
3.35	0.01\\
3.36	0.01\\
3.37	0.01\\
3.38	0.01\\
3.39	0.01\\
3.4	0.01\\
3.41	0.01\\
3.42	0.01\\
3.43	0.01\\
3.44	0.01\\
3.45	0.01\\
3.46	0.01\\
3.47	0.01\\
3.48	0.01\\
3.49	0.01\\
3.5	0.01\\
3.51	0.01\\
3.52	0.01\\
3.53	0.01\\
3.54	0.01\\
3.55	0.01\\
3.56	0.01\\
3.57	0.01\\
3.58	0.01\\
3.59	0.01\\
3.6	0.01\\
3.61	0.01\\
3.62	0.01\\
3.63	0.01\\
3.64	0.01\\
3.65	0.01\\
3.66	0.01\\
3.67	0.01\\
3.68	0.01\\
3.69	0.01\\
3.7	0.01\\
3.71	0.01\\
3.72	0.01\\
3.73	0.01\\
3.74	0.01\\
3.75	0.01\\
3.76	0.01\\
3.77	0.01\\
3.78	0.01\\
3.79	0.01\\
3.8	0.01\\
3.81	0.01\\
3.82	0.01\\
3.83	0.01\\
3.84	0.01\\
3.85	0.01\\
3.86	0.01\\
3.87	0.01\\
3.88	0.01\\
3.89	0.01\\
3.9	0.01\\
3.91	0.01\\
3.92	0.01\\
3.93	0.01\\
3.94	0.01\\
3.95	0.01\\
3.96	0.01\\
3.97	0.01\\
3.98	0.01\\
3.99	0.01\\
4	0.01\\
4.01	0.01\\
4.02	0.01\\
4.03	0.01\\
4.04	0.01\\
4.05	0.01\\
4.06	0.01\\
4.07	0.01\\
4.08	0.01\\
4.09	0.01\\
4.1	0.01\\
4.11	0.01\\
4.12	0.01\\
4.13	0.01\\
4.14	0.01\\
4.15	0.01\\
4.16	0.01\\
4.17	0.01\\
4.18	0.01\\
4.19	0.01\\
4.2	0.01\\
4.21	0.01\\
4.22	0.01\\
4.23	0.01\\
4.24	0.01\\
4.25	0.01\\
4.26	0.01\\
4.27	0.01\\
4.28	0.01\\
4.29	0.01\\
4.3	0.01\\
4.31	0.01\\
4.32	0.01\\
4.33	0.01\\
4.34	0.01\\
4.35	0.01\\
4.36	0.01\\
4.37	0.01\\
4.38	0.01\\
4.39	0.01\\
4.4	0.01\\
4.41	0.01\\
4.42	0.01\\
4.43	0.01\\
4.44	0.01\\
4.45	0.01\\
4.46	0.01\\
4.47	0.01\\
4.48	0.01\\
4.49	0.01\\
4.5	0.01\\
4.51	0.01\\
4.52	0.01\\
4.53	0.01\\
4.54	0.01\\
4.55	0.01\\
4.56	0.01\\
4.57	0.01\\
4.58	0.01\\
4.59	0.01\\
4.6	0.01\\
4.61	0.01\\
4.62	0.01\\
4.63	0.01\\
4.64	0.01\\
4.65	0.01\\
4.66	0.01\\
4.67	0.01\\
4.68	0.01\\
4.69	0.01\\
4.7	0.01\\
4.71	0.01\\
4.72	0.01\\
4.73	0.01\\
4.74	0.01\\
4.75	0.01\\
4.76	0.01\\
4.77	0.01\\
4.78	0.01\\
4.79	0.01\\
4.8	0.01\\
4.81	0.01\\
4.82	0.01\\
4.83	0.01\\
4.84	0.01\\
4.85	0.01\\
4.86	0.01\\
4.87	0.01\\
4.88	0.01\\
4.89	0.01\\
4.9	0.01\\
4.91	0.01\\
4.92	0.01\\
4.93	0.01\\
4.94	0.01\\
4.95	0.01\\
4.96	0.01\\
4.97	0.01\\
4.98	0.01\\
4.99	0.01\\
5	0.01\\
5.01	0.01\\
5.02	0.01\\
5.03	0.01\\
5.04	0.01\\
5.05	0.01\\
5.06	0.01\\
5.07	0.01\\
5.08	0.01\\
5.09	0.01\\
5.1	0.01\\
5.11	0.01\\
5.12	0.01\\
5.13	0.01\\
5.14	0.01\\
5.15	0.01\\
5.16	0.01\\
5.17	0.01\\
5.18	0.01\\
5.19	0.01\\
5.2	0.01\\
5.21	0.01\\
5.22	0.01\\
5.23	0.01\\
5.24	0.01\\
5.25	0.01\\
5.26	0.01\\
5.27	0.01\\
5.28	0.01\\
5.29	0.01\\
5.3	0.01\\
5.31	0.01\\
5.32	0.01\\
5.33	0.01\\
5.34	0.01\\
5.35	0.01\\
5.36	0.01\\
5.37	0.01\\
5.38	0.01\\
5.39	0.01\\
5.4	0.01\\
5.41	0.01\\
5.42	0.01\\
5.43	0.01\\
5.44	0.01\\
5.45	0.01\\
5.46	0.01\\
5.47	0.01\\
5.48	0.01\\
5.49	0.01\\
5.5	0.01\\
5.51	0.01\\
5.52	0.01\\
5.53	0.01\\
5.54	0.01\\
5.55	0.01\\
5.56	0.01\\
5.57	0.01\\
5.58	0.01\\
5.59	0.01\\
5.6	0.01\\
5.61	0.01\\
5.62	0.01\\
5.63	0.01\\
5.64	0.01\\
5.65	0.01\\
5.66	0.01\\
5.67	0.01\\
5.68	0.01\\
5.69	0.01\\
5.7	0.01\\
5.71	0.01\\
5.72	0.01\\
5.73	0.01\\
5.74	0.01\\
5.75	0.01\\
5.76	0.01\\
5.77	0.01\\
5.78	0.01\\
5.79	0.01\\
5.8	0.01\\
5.81	0.01\\
5.82	0.01\\
5.83	0.01\\
5.84	0.01\\
5.85	0.01\\
5.86	0.01\\
5.87	0.01\\
5.88	0.01\\
5.89	0.01\\
5.9	0.01\\
5.91	0.01\\
5.92	0.01\\
5.93	0.01\\
5.94	0.01\\
5.95	0.01\\
5.96	0.01\\
5.97	0.01\\
5.98	0.01\\
5.99	0.01\\
6	0.01\\
6.01	0.01\\
6.02	0.01\\
6.03	0.01\\
6.04	0.01\\
6.05	0.01\\
6.06	0.01\\
6.07	0.01\\
6.08	0.01\\
6.09	0.01\\
6.1	0.01\\
6.11	0.01\\
6.12	0.01\\
6.13	0.01\\
6.14	0.01\\
6.15	0.01\\
6.16	0.01\\
6.17	0.01\\
6.18	0.01\\
6.19	0.01\\
6.2	0.01\\
6.21	0.01\\
6.22	0.01\\
6.23	0.01\\
6.24	0.01\\
6.25	0.01\\
6.26	0.01\\
6.27	0.01\\
6.28	0.01\\
6.29	0.01\\
6.3	0.01\\
6.31	0.01\\
6.32	0.01\\
6.33	0.01\\
6.34	0.01\\
6.35	0.01\\
6.36	0.01\\
6.37	0.01\\
6.38	0.01\\
6.39	0.01\\
6.4	0.01\\
6.41	0.01\\
6.42	0.01\\
6.43	0.01\\
6.44	0.01\\
6.45	0.01\\
6.46	0.01\\
6.47	0.01\\
6.48	0.01\\
6.49	0.01\\
6.5	0.01\\
6.51	0.01\\
6.52	0.01\\
6.53	0.01\\
6.54	0.01\\
6.55	0.01\\
6.56	0.01\\
6.57	0.01\\
6.58	0.01\\
6.59	0.01\\
6.6	0.01\\
6.61	0.01\\
6.62	0.01\\
6.63	0.01\\
6.64	0.01\\
6.65	0.01\\
6.66	0.01\\
6.67	0.01\\
6.68	0.01\\
6.69	0.01\\
6.7	0.01\\
6.71	0.01\\
6.72	0.01\\
6.73	0.01\\
6.74	0.01\\
6.75	0.01\\
6.76	0.01\\
6.77	0.01\\
6.78	0.01\\
6.79	0.01\\
6.8	0.01\\
6.81	0.01\\
6.82	0.01\\
6.83	0.01\\
6.84	0.01\\
6.85	0.01\\
6.86	0.01\\
6.87	0.01\\
6.88	0.01\\
6.89	0.01\\
6.9	0.01\\
6.91	0.01\\
6.92	0.01\\
6.93	0.01\\
6.94	0.01\\
6.95	0.01\\
6.96	0.01\\
6.97	0.01\\
6.98	0.01\\
6.99	0.01\\
7	0.01\\
7.01	0.01\\
7.02	0.01\\
7.03	0.01\\
7.04	0.01\\
7.05	0.01\\
7.06	0.01\\
7.07	0.01\\
7.08	0.01\\
7.09	0.01\\
7.1	0.01\\
7.11	0.01\\
7.12	0.01\\
7.13	0.01\\
7.14	0.01\\
7.15	0.01\\
7.16	0.01\\
7.17	0.01\\
7.18	0.01\\
7.19	0.01\\
7.2	0.01\\
7.21	0.01\\
7.22	0.01\\
7.23	0.01\\
7.24	0.01\\
7.25	0.01\\
7.26	0.01\\
7.27	0.01\\
7.28	0.01\\
7.29	0.01\\
7.3	0.01\\
7.31	0.01\\
7.32	0.01\\
7.33	0.01\\
7.34	0.01\\
7.35	0.01\\
7.36	0.01\\
7.37	0.01\\
7.38	0.01\\
7.39	0.01\\
7.4	0.01\\
7.41	0.01\\
7.42	0.01\\
7.43	0.01\\
7.44	0.01\\
7.45	0.01\\
7.46	0.01\\
7.47	0.01\\
7.48	0.01\\
7.49	0.01\\
7.5	0.01\\
7.51	0.01\\
7.52	0.01\\
7.53	0.01\\
7.54	0.01\\
7.55	0.01\\
7.56	0.01\\
7.57	0.01\\
7.58	0.01\\
7.59	0.01\\
7.6	0.01\\
7.61	0.01\\
7.62	0.01\\
7.63	0.01\\
7.64	0.01\\
7.65	0.01\\
7.66	0.01\\
7.67	0.01\\
7.68	0.01\\
7.69	0.01\\
7.7	0.01\\
7.71	0.01\\
7.72	0.01\\
7.73	0.01\\
7.74	0.01\\
7.75	0.01\\
7.76	0.01\\
7.77	0.01\\
7.78	0.01\\
7.79	0.01\\
7.8	0.01\\
7.81	0.01\\
7.82	0.01\\
7.83	0.01\\
7.84	0.01\\
7.85	0.01\\
7.86	0.01\\
7.87	0.01\\
7.88	0.01\\
7.89	0.01\\
7.9	0.01\\
7.91	0.01\\
7.92	0.01\\
7.93	0.01\\
7.94	0.01\\
7.95	0.01\\
7.96	0.01\\
7.97	0.01\\
7.98	0.01\\
7.99	0.01\\
8	0.01\\
8.01	0.01\\
8.02	0.01\\
8.03	0.01\\
8.04	0.01\\
8.05	0.01\\
8.06	0.01\\
8.07	0.01\\
8.08	0.01\\
8.09	0.01\\
8.1	0.01\\
8.11	0.01\\
8.12	0.01\\
8.13	0.01\\
8.14	0.01\\
8.15	0.01\\
8.16	0.01\\
8.17	0.01\\
8.18	0.01\\
8.19	0.01\\
8.2	0.01\\
8.21	0.01\\
8.22	0.01\\
8.23	0.01\\
8.24	0.01\\
8.25	0.01\\
8.26	0.01\\
8.27	0.01\\
8.28	0.01\\
8.29	0.01\\
8.3	0.01\\
8.31	0.01\\
8.32	0.01\\
8.33	0.01\\
8.34	0.01\\
8.35	0.01\\
8.36	0.01\\
8.37	0.01\\
8.38	0.01\\
8.39	0.01\\
8.4	0.01\\
8.41	0.01\\
8.42	0.01\\
8.43	0.01\\
8.44	0.01\\
8.45	0.01\\
8.46	0.01\\
8.47	0.01\\
8.48	0.01\\
8.49	0.01\\
8.5	0.01\\
8.51	0.01\\
8.52	0.01\\
8.53	0.01\\
8.54	0.01\\
8.55	0.01\\
8.56	0.01\\
8.57	0.01\\
8.58	0.01\\
8.59	0.01\\
8.6	0.01\\
8.61	0.01\\
8.62	0.01\\
8.63	0.01\\
8.64	0.01\\
8.65	0.01\\
8.66	0.01\\
8.67	0.01\\
8.68	0.01\\
8.69	0.01\\
8.7	0.01\\
8.71	0.01\\
8.72	0.01\\
8.73	0.01\\
8.74	0.01\\
8.75	0.01\\
8.76	0.01\\
8.77	0.01\\
8.78	0.01\\
8.79	0.01\\
8.8	0.01\\
8.81	0.01\\
8.82	0.01\\
8.83	0.01\\
8.84	0.01\\
8.85	0.01\\
8.86	0.01\\
8.87	0.01\\
8.88	0.01\\
8.89	0.01\\
8.9	0.01\\
8.91	0.01\\
8.92	0.01\\
8.93	0.01\\
8.94	0.01\\
8.95	0.01\\
8.96	0.01\\
8.97	0.01\\
8.98	0.01\\
8.99	0.01\\
9	0.01\\
9.01	0.01\\
9.02	0.01\\
9.03	0.01\\
9.04	0.01\\
9.05	0.01\\
9.06	0.01\\
9.07	0.01\\
9.08	0.01\\
9.09	0.01\\
9.1	0.01\\
9.11	0.01\\
9.12	0.01\\
9.13	0.01\\
9.14	0.01\\
9.15	0.01\\
9.16	0.01\\
9.17	0.01\\
9.18	0.01\\
9.19	0.01\\
9.2	0.01\\
9.21	0.01\\
9.22	0.01\\
9.23	0.01\\
9.24	0.01\\
9.25	0.01\\
9.26	0.01\\
9.27	0.01\\
9.28	0.01\\
9.29	0.01\\
9.3	0.01\\
9.31	0.01\\
9.32	0.01\\
9.33	0.01\\
9.34	0.01\\
9.35	0.01\\
9.36	0.01\\
9.37	0.01\\
9.38	0.01\\
9.39	0.01\\
9.4	0.01\\
9.41	0.01\\
9.42	0.01\\
9.43	0.01\\
9.44	0.01\\
9.45	0.01\\
9.46	0.01\\
9.47	0.01\\
9.48	0.01\\
9.49	0.01\\
9.5	0.01\\
9.51	0.01\\
9.52	0.01\\
9.53	0.01\\
9.54	0.01\\
9.55	0.01\\
9.56	0.01\\
9.57	0.01\\
9.58	0.01\\
9.59	0.01\\
9.6	0.01\\
9.61	0.01\\
9.62	0.01\\
9.63	0.01\\
9.64	0.01\\
9.65	0.01\\
9.66	0.01\\
9.67	0.01\\
9.68	0.01\\
9.69	0.01\\
9.7	0.01\\
9.71	0.01\\
9.72	0.01\\
9.73	0.01\\
9.74	0.01\\
9.75	0.01\\
9.76	0.01\\
9.77	0.01\\
9.78	0.01\\
9.79	0.01\\
9.8	0.01\\
9.81	0.01\\
9.82	0.01\\
9.83	0.01\\
9.84	0.01\\
9.85	0.01\\
9.86	0.01\\
9.87	0.01\\
9.88	0.01\\
9.89	0.01\\
9.9	0.01\\
9.91	0.01\\
9.92	0.01\\
9.93	0.01\\
9.94	0.01\\
9.95	0.01\\
9.96	0.01\\
9.97	0.01\\
9.98	0.01\\
9.99	0.01\\
10	0.01\\
10.01	0.01\\
10.02	0.01\\
10.03	0.01\\
10.04	0.01\\
10.05	0.01\\
10.06	0.01\\
10.07	0.01\\
10.08	0.01\\
10.09	0.01\\
10.1	0.01\\
10.11	0.01\\
10.12	0.01\\
10.13	0.01\\
10.14	0.01\\
10.15	0.01\\
10.16	0.01\\
10.17	0.01\\
10.18	0.01\\
10.19	0.01\\
10.2	0.01\\
10.21	0.01\\
10.22	0.01\\
10.23	0.01\\
10.24	0.01\\
10.25	0.01\\
10.26	0.01\\
10.27	0.01\\
10.28	0.01\\
10.29	0.01\\
10.3	0.01\\
10.31	0.01\\
10.32	0.01\\
10.33	0.01\\
10.34	0.01\\
10.35	0.01\\
10.36	0.01\\
10.37	0.01\\
10.38	0.01\\
10.39	0.01\\
10.4	0.01\\
10.41	0.01\\
10.42	0.01\\
10.43	0.01\\
10.44	0.01\\
10.45	0.01\\
10.46	0.01\\
10.47	0.01\\
10.48	0.01\\
10.49	0.01\\
10.5	0.01\\
10.51	0.01\\
10.52	0.01\\
10.53	0.01\\
10.54	0.01\\
10.55	0.01\\
10.56	0.01\\
10.57	0.01\\
10.58	0.01\\
10.59	0.01\\
10.6	0.01\\
10.61	0.01\\
10.62	0.01\\
10.63	0.01\\
10.64	0.01\\
10.65	0.01\\
10.66	0.01\\
10.67	0.01\\
10.68	0.01\\
10.69	0.01\\
10.7	0.01\\
10.71	0.01\\
10.72	0.01\\
10.73	0.01\\
10.74	0.01\\
10.75	0.01\\
10.76	0.01\\
10.77	0.01\\
10.78	0.01\\
10.79	0.01\\
10.8	0.01\\
10.81	0.01\\
10.82	0.01\\
10.83	0.01\\
10.84	0.01\\
10.85	0.01\\
10.86	0.01\\
10.87	0.01\\
10.88	0.01\\
10.89	0.01\\
10.9	0.01\\
10.91	0.01\\
10.92	0.01\\
10.93	0.01\\
10.94	0.01\\
10.95	0.01\\
10.96	0.01\\
10.97	0.01\\
10.98	0.01\\
10.99	0.01\\
11	0.01\\
11.01	0.01\\
11.02	0.01\\
11.03	0.01\\
11.04	0.01\\
11.05	0.01\\
11.06	0.01\\
11.07	0.01\\
11.08	0.01\\
11.09	0.01\\
11.1	0.01\\
11.11	0.01\\
11.12	0.01\\
11.13	0.01\\
11.14	0.01\\
11.15	0.01\\
11.16	0.01\\
11.17	0.01\\
11.18	0.01\\
11.19	0.01\\
11.2	0.01\\
11.21	0.01\\
11.22	0.01\\
11.23	0.01\\
11.24	0.01\\
11.25	0.01\\
11.26	0.01\\
11.27	0.01\\
11.28	0.01\\
11.29	0.01\\
11.3	0.01\\
11.31	0.01\\
11.32	0.01\\
11.33	0.01\\
11.34	0.01\\
11.35	0.01\\
11.36	0.01\\
11.37	0.01\\
11.38	0.01\\
11.39	0.01\\
11.4	0.01\\
11.41	0.01\\
11.42	0.01\\
11.43	0.01\\
11.44	0.01\\
11.45	0.01\\
11.46	0.01\\
11.47	0.01\\
11.48	0.01\\
11.49	0.01\\
11.5	0.01\\
11.51	0.01\\
11.52	0.01\\
11.53	0.01\\
11.54	0.01\\
11.55	0.01\\
11.56	0.01\\
11.57	0.01\\
11.58	0.01\\
11.59	0.01\\
11.6	0.01\\
11.61	0.01\\
11.62	0.01\\
11.63	0.01\\
11.64	0.01\\
11.65	0.01\\
11.66	0.01\\
11.67	0.01\\
11.68	0.01\\
11.69	0.01\\
11.7	0.01\\
11.71	0.01\\
11.72	0.01\\
11.73	0.01\\
11.74	0.01\\
11.75	0.01\\
11.76	0.01\\
11.77	0.01\\
11.78	0.01\\
11.79	0.01\\
11.8	0.01\\
11.81	0.01\\
11.82	0.01\\
11.83	0.01\\
11.84	0.01\\
11.85	0.01\\
11.86	0.01\\
11.87	0.01\\
11.88	0.01\\
11.89	0.01\\
11.9	0.01\\
11.91	0.01\\
11.92	0.01\\
11.93	0.01\\
11.94	0.01\\
11.95	0.01\\
11.96	0.01\\
11.97	0.01\\
11.98	0.01\\
11.99	0.01\\
12	0.01\\
12.01	0.01\\
12.02	0.01\\
12.03	0.01\\
12.04	0.01\\
12.05	0.01\\
12.06	0.01\\
12.07	0.01\\
12.08	0.01\\
12.09	0.01\\
12.1	0.01\\
12.11	0.01\\
12.12	0.01\\
12.13	0.01\\
12.14	0.01\\
12.15	0.01\\
12.16	0.01\\
12.17	0.01\\
12.18	0.01\\
12.19	0.01\\
12.2	0.01\\
12.21	0.01\\
12.22	0.01\\
12.23	0.01\\
12.24	0.01\\
12.25	0.01\\
12.26	0.01\\
12.27	0.01\\
12.28	0.01\\
12.29	0.01\\
12.3	0.01\\
12.31	0.01\\
12.32	0.01\\
12.33	0.01\\
12.34	0.01\\
12.35	0.01\\
12.36	0.01\\
12.37	0.01\\
12.38	0.01\\
12.39	0.01\\
12.4	0.01\\
12.41	0.01\\
12.42	0.01\\
12.43	0.01\\
12.44	0.01\\
12.45	0.01\\
12.46	0.01\\
12.47	0.01\\
12.48	0.01\\
12.49	0.01\\
12.5	0.01\\
12.51	0.01\\
12.52	0.01\\
12.53	0.01\\
12.54	0.01\\
12.55	0.01\\
12.56	0.01\\
12.57	0.01\\
12.58	0.01\\
12.59	0.01\\
12.6	0.01\\
12.61	0.01\\
12.62	0.01\\
12.63	0.01\\
12.64	0.01\\
12.65	0.01\\
12.66	0.01\\
12.67	0.01\\
12.68	0.01\\
12.69	0.01\\
12.7	0.01\\
12.71	0.01\\
12.72	0.01\\
12.73	0.01\\
12.74	0.01\\
12.75	0.01\\
12.76	0.01\\
12.77	0.01\\
12.78	0.01\\
12.79	0.01\\
12.8	0.01\\
12.81	0.01\\
12.82	0.01\\
12.83	0.01\\
12.84	0.01\\
12.85	0.01\\
12.86	0.01\\
12.87	0.01\\
12.88	0.01\\
12.89	0.01\\
12.9	0.01\\
12.91	0.01\\
12.92	0.01\\
12.93	0.01\\
12.94	0.01\\
12.95	0.01\\
12.96	0.01\\
12.97	0.01\\
12.98	0.01\\
12.99	0.01\\
13	0.01\\
13.01	0.01\\
13.02	0.01\\
13.03	0.01\\
13.04	0.01\\
13.05	0.01\\
13.06	0.01\\
13.07	0.01\\
13.08	0.01\\
13.09	0.01\\
13.1	0.01\\
13.11	0.01\\
13.12	0.01\\
13.13	0.01\\
13.14	0.01\\
13.15	0.01\\
13.16	0.01\\
13.17	0.01\\
13.18	0.01\\
13.19	0.01\\
13.2	0.01\\
13.21	0.01\\
13.22	0.01\\
13.23	0.01\\
13.24	0.01\\
13.25	0.01\\
13.26	0.01\\
13.27	0.01\\
13.28	0.01\\
13.29	0.01\\
13.3	0.01\\
13.31	0.01\\
13.32	0.01\\
13.33	0.01\\
13.34	0.01\\
13.35	0.01\\
13.36	0.01\\
13.37	0.01\\
13.38	0.01\\
13.39	0.01\\
13.4	0.01\\
13.41	0.01\\
13.42	0.01\\
13.43	0.01\\
13.44	0.01\\
13.45	0.01\\
13.46	0.01\\
13.47	0.01\\
13.48	0.01\\
13.49	0.01\\
13.5	0.01\\
13.51	0.01\\
13.52	0.01\\
13.53	0.01\\
13.54	0.01\\
13.55	0.01\\
13.56	0.01\\
13.57	0.01\\
13.58	0.01\\
13.59	0.01\\
13.6	0.01\\
13.61	0.01\\
13.62	0.01\\
13.63	0.01\\
13.64	0.01\\
13.65	0.01\\
13.66	0.01\\
13.67	0.01\\
13.68	0.01\\
13.69	0.01\\
13.7	0.01\\
13.71	0.01\\
13.72	0.01\\
13.73	0.01\\
13.74	0.01\\
13.75	0.01\\
13.76	0.01\\
13.77	0.01\\
13.78	0.01\\
13.79	0.01\\
13.8	0.01\\
13.81	0.01\\
13.82	0.01\\
13.83	0.01\\
13.84	0.01\\
13.85	0.01\\
13.86	0.01\\
13.87	0.01\\
13.88	0.01\\
13.89	0.01\\
13.9	0.01\\
13.91	0.01\\
13.92	0.01\\
13.93	0.01\\
13.94	0.01\\
13.95	0.01\\
13.96	0.01\\
13.97	0.01\\
13.98	0.01\\
13.99	0.01\\
14	0.01\\
14.01	0.01\\
14.02	0.01\\
14.03	0.01\\
14.04	0.01\\
14.05	0.01\\
14.06	0.01\\
14.07	0.01\\
14.08	0.01\\
14.09	0.01\\
14.1	0.01\\
14.11	0.01\\
14.12	0.01\\
14.13	0.01\\
14.14	0.01\\
14.15	0.01\\
14.16	0.01\\
14.17	0.01\\
14.18	0.01\\
14.19	0.01\\
14.2	0.01\\
14.21	0.01\\
14.22	0.01\\
14.23	0.01\\
14.24	0.01\\
14.25	0.01\\
14.26	0.01\\
14.27	0.01\\
14.28	0.01\\
14.29	0.01\\
14.3	0.01\\
14.31	0.01\\
14.32	0.01\\
14.33	0.01\\
14.34	0.01\\
14.35	0.01\\
14.36	0.01\\
14.37	0.01\\
14.38	0.01\\
14.39	0.01\\
14.4	0.01\\
14.41	0.01\\
14.42	0.01\\
14.43	0.01\\
14.44	0.01\\
14.45	0.01\\
14.46	0.01\\
14.47	0.01\\
14.48	0.01\\
14.49	0.01\\
14.5	0.01\\
14.51	0.01\\
14.52	0.01\\
14.53	0.01\\
14.54	0.01\\
14.55	0.01\\
14.56	0.01\\
14.57	0.01\\
14.58	0.01\\
14.59	0.01\\
14.6	0.01\\
14.61	0.01\\
14.62	0.01\\
14.63	0.01\\
14.64	0.01\\
14.65	0.01\\
14.66	0.01\\
14.67	0.01\\
14.68	0.01\\
14.69	0.01\\
14.7	0.01\\
14.71	0.01\\
14.72	0.01\\
14.73	0.01\\
14.74	0.01\\
14.75	0.01\\
14.76	0.01\\
14.77	0.01\\
14.78	0.01\\
14.79	0.01\\
14.8	0.01\\
14.81	0.01\\
14.82	0.01\\
14.83	0.01\\
14.84	0.01\\
14.85	0.01\\
14.86	0.01\\
14.87	0.01\\
14.88	0.01\\
14.89	0.01\\
14.9	0.01\\
14.91	0.01\\
14.92	0.01\\
14.93	0.01\\
14.94	0.01\\
14.95	0.01\\
14.96	0.01\\
14.97	0.01\\
14.98	0.01\\
14.99	0.01\\
15	0.01\\
15.01	0.01\\
15.02	0.01\\
15.03	0.01\\
15.04	0.01\\
15.05	0.01\\
15.06	0.01\\
15.07	0.01\\
15.08	0.01\\
15.09	0.01\\
15.1	0.01\\
15.11	0.01\\
15.12	0.01\\
15.13	0.01\\
15.14	0.01\\
15.15	0.01\\
15.16	0.01\\
15.17	0.01\\
15.18	0.01\\
15.19	0.01\\
15.2	0.01\\
15.21	0.01\\
15.22	0.01\\
15.23	0.01\\
15.24	0.01\\
15.25	0.01\\
15.26	0.01\\
15.27	0.01\\
15.28	0.01\\
15.29	0.01\\
15.3	0.01\\
15.31	0.01\\
15.32	0.01\\
15.33	0.01\\
15.34	0.01\\
15.35	0.01\\
15.36	0.01\\
15.37	0.01\\
15.38	0.01\\
15.39	0.01\\
15.4	0.01\\
15.41	0.01\\
15.42	0.01\\
15.43	0.01\\
15.44	0.01\\
15.45	0.01\\
15.46	0.01\\
15.47	0.01\\
15.48	0.01\\
15.49	0.01\\
15.5	0.01\\
15.51	0.01\\
15.52	0.01\\
15.53	0.01\\
15.54	0.01\\
15.55	0.01\\
15.56	0.01\\
15.57	0.01\\
15.58	0.01\\
15.59	0.01\\
15.6	0.01\\
15.61	0.01\\
15.62	0.01\\
15.63	0.01\\
15.64	0.01\\
15.65	0.01\\
15.66	0.01\\
15.67	0.01\\
15.68	0.01\\
15.69	0.01\\
15.7	0.01\\
15.71	0.01\\
15.72	0.01\\
15.73	0.01\\
15.74	0.01\\
15.75	0.01\\
15.76	0.01\\
15.77	0.01\\
15.78	0.01\\
15.79	0.01\\
15.8	0.01\\
15.81	0.01\\
15.82	0.01\\
15.83	0.01\\
15.84	0.01\\
15.85	0.01\\
15.86	0.01\\
15.87	0.01\\
15.88	0.01\\
15.89	0.01\\
15.9	0.01\\
15.91	0.01\\
15.92	0.01\\
15.93	0.01\\
15.94	0.01\\
15.95	0.01\\
15.96	0.01\\
15.97	0.01\\
15.98	0.01\\
15.99	0.01\\
16	0.01\\
16.01	0.01\\
16.02	0.01\\
16.03	0.01\\
16.04	0.01\\
16.05	0.01\\
16.06	0.01\\
16.07	0.01\\
16.08	0.01\\
16.09	0.01\\
16.1	0.01\\
16.11	0.01\\
16.12	0.01\\
16.13	0.01\\
16.14	0.01\\
16.15	0.01\\
16.16	0.01\\
16.17	0.01\\
16.18	0.01\\
16.19	0.01\\
16.2	0.01\\
16.21	0.01\\
16.22	0.01\\
16.23	0.01\\
16.24	0.01\\
16.25	0.01\\
16.26	0.01\\
16.27	0.01\\
16.28	0.01\\
16.29	0.01\\
16.3	0.01\\
16.31	0.01\\
16.32	0.01\\
16.33	0.01\\
16.34	0.01\\
16.35	0.01\\
16.36	0.01\\
16.37	0.01\\
16.38	0.01\\
16.39	0.01\\
16.4	0.01\\
16.41	0.01\\
16.42	0.01\\
16.43	0.01\\
16.44	0.01\\
16.45	0.01\\
16.46	0.01\\
16.47	0.01\\
16.48	0.01\\
16.49	0.01\\
16.5	0.01\\
16.51	0.01\\
16.52	0.01\\
16.53	0.01\\
16.54	0.01\\
16.55	0.01\\
16.56	0.01\\
16.57	0.01\\
16.58	0.01\\
16.59	0.01\\
16.6	0.01\\
16.61	0.01\\
16.62	0.01\\
16.63	0.01\\
16.64	0.01\\
16.65	0.01\\
16.66	0.01\\
16.67	0.01\\
16.68	0.01\\
16.69	0.01\\
16.7	0.01\\
16.71	0.01\\
16.72	0.01\\
16.73	0.01\\
16.74	0.01\\
16.75	0.01\\
16.76	0.01\\
16.77	0.01\\
16.78	0.01\\
16.79	0.01\\
16.8	0.01\\
16.81	0.01\\
16.82	0.01\\
16.83	0.01\\
16.84	0.01\\
16.85	0.01\\
16.86	0.01\\
16.87	0.01\\
16.88	0.01\\
16.89	0.01\\
16.9	0.01\\
16.91	0.01\\
16.92	0.01\\
16.93	0.01\\
16.94	0.01\\
16.95	0.01\\
16.96	0.01\\
16.97	0.01\\
16.98	0.01\\
16.99	0.01\\
17	0.01\\
17.01	0.01\\
17.02	0.01\\
17.03	0.01\\
17.04	0.01\\
17.05	0.01\\
17.06	0.01\\
17.07	0.01\\
17.08	0.01\\
17.09	0.01\\
17.1	0.01\\
17.11	0.01\\
17.12	0.01\\
17.13	0.01\\
17.14	0.01\\
17.15	0.01\\
17.16	0.01\\
17.17	0.01\\
17.18	0.01\\
17.19	0.01\\
17.2	0.01\\
17.21	0.01\\
17.22	0.01\\
17.23	0.01\\
17.24	0.01\\
17.25	0.01\\
17.26	0.01\\
17.27	0.01\\
17.28	0.01\\
17.29	0.01\\
17.3	0.01\\
17.31	0.01\\
17.32	0.01\\
17.33	0.01\\
17.34	0.01\\
17.35	0.01\\
17.36	0.01\\
17.37	0.01\\
17.38	0.01\\
17.39	0.01\\
17.4	0.01\\
17.41	0.01\\
17.42	0.01\\
17.43	0.01\\
17.44	0.01\\
17.45	0.01\\
17.46	0.01\\
17.47	0.01\\
17.48	0.01\\
17.49	0.01\\
17.5	0.01\\
17.51	0.01\\
17.52	0.01\\
17.53	0.01\\
17.54	0.01\\
17.55	0.01\\
17.56	0.01\\
17.57	0.01\\
17.58	0.01\\
17.59	0.01\\
17.6	0.01\\
17.61	0.01\\
17.62	0.01\\
17.63	0.01\\
17.64	0.01\\
17.65	0.01\\
17.66	0.01\\
17.67	0.01\\
17.68	0.01\\
17.69	0.01\\
17.7	0.01\\
17.71	0.01\\
17.72	0.01\\
17.73	0.01\\
17.74	0.01\\
17.75	0.01\\
17.76	0.01\\
17.77	0.01\\
17.78	0.01\\
17.79	0.01\\
17.8	0.01\\
17.81	0.01\\
17.82	0.01\\
17.83	0.01\\
17.84	0.01\\
17.85	0.01\\
17.86	0.01\\
17.87	0.01\\
17.88	0.01\\
17.89	0.01\\
17.9	0.01\\
17.91	0.01\\
17.92	0.01\\
17.93	0.01\\
17.94	0.01\\
17.95	0.01\\
17.96	0.01\\
17.97	0.01\\
17.98	0.01\\
17.99	0.01\\
18	0.01\\
18.01	0.01\\
18.02	0.01\\
18.03	0.01\\
18.04	0.01\\
18.05	0.01\\
18.06	0.01\\
18.07	0.01\\
18.08	0.01\\
18.09	0.01\\
18.1	0.01\\
18.11	0.01\\
18.12	0.01\\
18.13	0.01\\
18.14	0.01\\
18.15	0.01\\
18.16	0.01\\
18.17	0.01\\
18.18	0.01\\
18.19	0.01\\
18.2	0.01\\
18.21	0.01\\
18.22	0.01\\
18.23	0.01\\
18.24	0.01\\
18.25	0.01\\
18.26	0.01\\
18.27	0.01\\
18.28	0.01\\
18.29	0.01\\
18.3	0.01\\
18.31	0.01\\
18.32	0.01\\
18.33	0.01\\
18.34	0.01\\
18.35	0.01\\
18.36	0.01\\
18.37	0.01\\
18.38	0.01\\
18.39	0.01\\
18.4	0.01\\
18.41	0.01\\
18.42	0.01\\
18.43	0.01\\
18.44	0.01\\
18.45	0.01\\
18.46	0.01\\
18.47	0.01\\
18.48	0.01\\
18.49	0.01\\
18.5	0.01\\
18.51	0.01\\
18.52	0.01\\
18.53	0.01\\
18.54	0.01\\
18.55	0.01\\
18.56	0.01\\
18.57	0.01\\
18.58	0.01\\
18.59	0.01\\
18.6	0.01\\
18.61	0.01\\
18.62	0.01\\
18.63	0.01\\
18.64	0.01\\
18.65	0.01\\
18.66	0.01\\
18.67	0.01\\
18.68	0.01\\
18.69	0.01\\
18.7	0.01\\
18.71	0.01\\
18.72	0.01\\
18.73	0.01\\
18.74	0.01\\
18.75	0.01\\
18.76	0.01\\
18.77	0.01\\
18.78	0.01\\
18.79	0.01\\
18.8	0.01\\
18.81	0.01\\
18.82	0.01\\
18.83	0.01\\
18.84	0.01\\
18.85	0.01\\
18.86	0.01\\
18.87	0.01\\
18.88	0.01\\
18.89	0.01\\
18.9	0.01\\
18.91	0.01\\
18.92	0.01\\
18.93	0.01\\
18.94	0.01\\
18.95	0.01\\
18.96	0.01\\
18.97	0.01\\
18.98	0.01\\
18.99	0.01\\
19	0.01\\
19.01	0.01\\
19.02	0.01\\
19.03	0.01\\
19.04	0.01\\
19.05	0.01\\
19.06	0.01\\
19.07	0.01\\
19.08	0.01\\
19.09	0.01\\
19.1	0.01\\
19.11	0.01\\
19.12	0.01\\
19.13	0.01\\
19.14	0.01\\
19.15	0.01\\
19.16	0.01\\
19.17	0.01\\
19.18	0.01\\
19.19	0.01\\
19.2	0.01\\
19.21	0.01\\
19.22	0.01\\
19.23	0.01\\
19.24	0.01\\
19.25	0.01\\
19.26	0.01\\
19.27	0.01\\
19.28	0.01\\
19.29	0.01\\
19.3	0.01\\
19.31	0.01\\
19.32	0.01\\
19.33	0.01\\
19.34	0.01\\
19.35	0.01\\
19.36	0.01\\
19.37	0.01\\
19.38	0.01\\
19.39	0.01\\
19.4	0.01\\
19.41	0.01\\
19.42	0.01\\
19.43	0.01\\
19.44	0.01\\
19.45	0.01\\
19.46	0.01\\
19.47	0.01\\
19.48	0.01\\
19.49	0.01\\
19.5	0.01\\
19.51	0.01\\
19.52	0.01\\
19.53	0.01\\
19.54	0.01\\
19.55	0.01\\
19.56	0.01\\
19.57	0.01\\
19.58	0.01\\
19.59	0.01\\
19.6	0.01\\
19.61	0.01\\
19.62	0.01\\
19.63	0.01\\
19.64	0.01\\
19.65	0.01\\
19.66	0.01\\
19.67	0.01\\
19.68	0.01\\
19.69	0.01\\
19.7	0.01\\
19.71	0.01\\
19.72	0.01\\
19.73	0.01\\
19.74	0.01\\
19.75	0.01\\
19.76	0.01\\
19.77	0.01\\
19.78	0.01\\
19.79	0.01\\
19.8	0.01\\
19.81	0.01\\
19.82	0.01\\
19.83	0.01\\
19.84	0.01\\
19.85	0.01\\
19.86	0.01\\
19.87	0.01\\
19.88	0.01\\
19.89	0.01\\
19.9	0.01\\
19.91	0.01\\
19.92	0.01\\
19.93	0.01\\
19.94	0.01\\
19.95	0.01\\
19.96	0.01\\
19.97	0.01\\
19.98	0.01\\
19.99	0.01\\
20	0.01\\
20.01	0.01\\
20.02	0.01\\
20.03	0.01\\
20.04	0.01\\
20.05	0.01\\
20.06	0.01\\
20.07	0.01\\
20.08	0.01\\
20.09	0.01\\
20.1	0.01\\
20.11	0.01\\
20.12	0.01\\
20.13	0.01\\
20.14	0.01\\
20.15	0.01\\
20.16	0.01\\
20.17	0.01\\
20.18	0.01\\
20.19	0.01\\
20.2	0.01\\
20.21	0.01\\
20.22	0.01\\
20.23	0.01\\
20.24	0.01\\
20.25	0.01\\
20.26	0.01\\
20.27	0.01\\
20.28	0.01\\
20.29	0.01\\
20.3	0.01\\
20.31	0.01\\
20.32	0.01\\
20.33	0.01\\
20.34	0.01\\
20.35	0.01\\
20.36	0.01\\
20.37	0.01\\
20.38	0.01\\
20.39	0.01\\
20.4	0.01\\
20.41	0.01\\
20.42	0.01\\
20.43	0.01\\
20.44	0.01\\
20.45	0.01\\
20.46	0.01\\
20.47	0.01\\
20.48	0.01\\
20.49	0.01\\
20.5	0.01\\
20.51	0.01\\
20.52	0.01\\
20.53	0.01\\
20.54	0.01\\
20.55	0.01\\
20.56	0.01\\
20.57	0.01\\
20.58	0.01\\
20.59	0.01\\
20.6	0.01\\
20.61	0.01\\
20.62	0.01\\
20.63	0.01\\
20.64	0.01\\
20.65	0.01\\
20.66	0.01\\
20.67	0.01\\
20.68	0.01\\
20.69	0.01\\
20.7	0.01\\
20.71	0.01\\
20.72	0.01\\
20.73	0.01\\
20.74	0.01\\
20.75	0.01\\
20.76	0.01\\
20.77	0.01\\
20.78	0.01\\
20.79	0.01\\
20.8	0.01\\
20.81	0.01\\
20.82	0.01\\
20.83	0.01\\
20.84	0.01\\
20.85	0.01\\
20.86	0.01\\
20.87	0.01\\
20.88	0.01\\
20.89	0.01\\
20.9	0.01\\
20.91	0.01\\
20.92	0.01\\
20.93	0.01\\
20.94	0.01\\
20.95	0.01\\
20.96	0.01\\
20.97	0.01\\
20.98	0.01\\
20.99	0.01\\
21	0.01\\
21.01	0.01\\
21.02	0.01\\
21.03	0.01\\
21.04	0.01\\
21.05	0.01\\
21.06	0.01\\
21.07	0.01\\
21.08	0.01\\
21.09	0.01\\
21.1	0.01\\
21.11	0.01\\
21.12	0.01\\
21.13	0.01\\
21.14	0.01\\
21.15	0.01\\
21.16	0.01\\
21.17	0.01\\
21.18	0.01\\
21.19	0.01\\
21.2	0.01\\
21.21	0.01\\
21.22	0.01\\
21.23	0.01\\
21.24	0.01\\
21.25	0.01\\
21.26	0.01\\
21.27	0.01\\
21.28	0.01\\
21.29	0.01\\
21.3	0.01\\
21.31	0.01\\
21.32	0.01\\
21.33	0.01\\
21.34	0.01\\
21.35	0.01\\
21.36	0.01\\
21.37	0.01\\
21.38	0.01\\
21.39	0.01\\
21.4	0.01\\
21.41	0.01\\
21.42	0.01\\
21.43	0.01\\
21.44	0.01\\
21.45	0.01\\
21.46	0.01\\
21.47	0.01\\
21.48	0.01\\
21.49	0.01\\
21.5	0.01\\
21.51	0.01\\
21.52	0.01\\
21.53	0.01\\
21.54	0.01\\
21.55	0.01\\
21.56	0.01\\
21.57	0.01\\
21.58	0.01\\
21.59	0.01\\
21.6	0.01\\
21.61	0.01\\
21.62	0.01\\
21.63	0.01\\
21.64	0.01\\
21.65	0.01\\
21.66	0.01\\
21.67	0.01\\
21.68	0.01\\
21.69	0.01\\
21.7	0.01\\
21.71	0.01\\
21.72	0.01\\
21.73	0.01\\
21.74	0.01\\
21.75	0.01\\
21.76	0.01\\
21.77	0.01\\
21.78	0.01\\
21.79	0.01\\
21.8	0.01\\
21.81	0.01\\
21.82	0.01\\
21.83	0.01\\
21.84	0.01\\
21.85	0.01\\
21.86	0.01\\
21.87	0.01\\
21.88	0.01\\
21.89	0.01\\
21.9	0.01\\
21.91	0.01\\
21.92	0.01\\
21.93	0.01\\
21.94	0.01\\
21.95	0.01\\
21.96	0.01\\
21.97	0.01\\
21.98	0.01\\
21.99	0.01\\
22	0.01\\
22.01	0.01\\
22.02	0.01\\
22.03	0.01\\
22.04	0.01\\
22.05	0.01\\
22.06	0.01\\
22.07	0.01\\
22.08	0.01\\
22.09	0.01\\
22.1	0.01\\
22.11	0.01\\
22.12	0.01\\
22.13	0.01\\
22.14	0.01\\
22.15	0.01\\
22.16	0.01\\
22.17	0.01\\
22.18	0.01\\
22.19	0.01\\
22.2	0.01\\
22.21	0.01\\
22.22	0.01\\
22.23	0.01\\
22.24	0.01\\
22.25	0.01\\
22.26	0.01\\
22.27	0.01\\
22.28	0.01\\
22.29	0.01\\
22.3	0.01\\
22.31	0.01\\
22.32	0.01\\
22.33	0.01\\
22.34	0.01\\
22.35	0.01\\
22.36	0.01\\
22.37	0.01\\
22.38	0.01\\
22.39	0.01\\
22.4	0.01\\
22.41	0.01\\
22.42	0.01\\
22.43	0.01\\
22.44	0.01\\
22.45	0.01\\
22.46	0.01\\
22.47	0.01\\
22.48	0.01\\
22.49	0.01\\
22.5	0.01\\
22.51	0.01\\
22.52	0.01\\
22.53	0.01\\
22.54	0.01\\
22.55	0.01\\
22.56	0.01\\
22.57	0.01\\
22.58	0.01\\
22.59	0.01\\
22.6	0.01\\
22.61	0.01\\
22.62	0.01\\
22.63	0.01\\
22.64	0.01\\
22.65	0.01\\
22.66	0.01\\
22.67	0.01\\
22.68	0.01\\
22.69	0.01\\
22.7	0.01\\
22.71	0.01\\
22.72	0.01\\
22.73	0.01\\
22.74	0.01\\
22.75	0.01\\
22.76	0.01\\
22.77	0.01\\
22.78	0.01\\
22.79	0.01\\
22.8	0.01\\
22.81	0.01\\
22.82	0.01\\
22.83	0.01\\
22.84	0.01\\
22.85	0.01\\
22.86	0.01\\
22.87	0.01\\
22.88	0.01\\
22.89	0.01\\
22.9	0.01\\
22.91	0.01\\
22.92	0.01\\
22.93	0.01\\
22.94	0.01\\
22.95	0.01\\
22.96	0.01\\
22.97	0.01\\
22.98	0.01\\
22.99	0.01\\
23	0.01\\
23.01	0.01\\
23.02	0.01\\
23.03	0.01\\
23.04	0.01\\
23.05	0.01\\
23.06	0.01\\
23.07	0.01\\
23.08	0.01\\
23.09	0.01\\
23.1	0.01\\
23.11	0.01\\
23.12	0.01\\
23.13	0.01\\
23.14	0.01\\
23.15	0.01\\
23.16	0.01\\
23.17	0.01\\
23.18	0.01\\
23.19	0.01\\
23.2	0.01\\
23.21	0.01\\
23.22	0.01\\
23.23	0.01\\
23.24	0.01\\
23.25	0.01\\
23.26	0.01\\
23.27	0.01\\
23.28	0.01\\
23.29	0.01\\
23.3	0.01\\
23.31	0.01\\
23.32	0.01\\
23.33	0.01\\
23.34	0.01\\
23.35	0.01\\
23.36	0.01\\
23.37	0.01\\
23.38	0.01\\
23.39	0.01\\
23.4	0.01\\
23.41	0.01\\
23.42	0.01\\
23.43	0.01\\
23.44	0.01\\
23.45	0.01\\
23.46	0.01\\
23.47	0.01\\
23.48	0.01\\
23.49	0.01\\
23.5	0.01\\
23.51	0.01\\
23.52	0.01\\
23.53	0.01\\
23.54	0.01\\
23.55	0.01\\
23.56	0.01\\
23.57	0.01\\
23.58	0.01\\
23.59	0.01\\
23.6	0.01\\
23.61	0.01\\
23.62	0.01\\
23.63	0.01\\
23.64	0.01\\
23.65	0.01\\
23.66	0.01\\
23.67	0.01\\
23.68	0.01\\
23.69	0.01\\
23.7	0.01\\
23.71	0.01\\
23.72	0.01\\
23.73	0.01\\
23.74	0.01\\
23.75	0.01\\
23.76	0.01\\
23.77	0.01\\
23.78	0.01\\
23.79	0.01\\
23.8	0.01\\
23.81	0.01\\
23.82	0.01\\
23.83	0.01\\
23.84	0.01\\
23.85	0.01\\
23.86	0.01\\
23.87	0.01\\
23.88	0.01\\
23.89	0.01\\
23.9	0.01\\
23.91	0.01\\
23.92	0.01\\
23.93	0.01\\
23.94	0.01\\
23.95	0.01\\
23.96	0.01\\
23.97	0.01\\
23.98	0.01\\
23.99	0.01\\
24	0.01\\
24.01	0.01\\
24.02	0.01\\
24.03	0.01\\
24.04	0.01\\
24.05	0.01\\
24.06	0.01\\
24.07	0.01\\
24.08	0.01\\
24.09	0.01\\
24.1	0.01\\
24.11	0.01\\
24.12	0.01\\
24.13	0.01\\
24.14	0.01\\
24.15	0.01\\
24.16	0.01\\
24.17	0.01\\
24.18	0.01\\
24.19	0.01\\
24.2	0.01\\
24.21	0.01\\
24.22	0.01\\
24.23	0.01\\
24.24	0.01\\
24.25	0.01\\
24.26	0.01\\
24.27	0.01\\
24.28	0.01\\
24.29	0.01\\
24.3	0.01\\
24.31	0.01\\
24.32	0.01\\
24.33	0.01\\
24.34	0.01\\
24.35	0.01\\
24.36	0.01\\
24.37	0.01\\
24.38	0.01\\
24.39	0.01\\
24.4	0.01\\
24.41	0.01\\
24.42	0.01\\
24.43	0.01\\
24.44	0.01\\
24.45	0.01\\
24.46	0.01\\
24.47	0.01\\
24.48	0.01\\
24.49	0.01\\
24.5	0.01\\
24.51	0.01\\
24.52	0.01\\
24.53	0.01\\
24.54	0.01\\
24.55	0.01\\
24.56	0.01\\
24.57	0.01\\
24.58	0.01\\
24.59	0.01\\
24.6	0.01\\
24.61	0.01\\
24.62	0.01\\
24.63	0.01\\
24.64	0.01\\
24.65	0.01\\
24.66	0.01\\
24.67	0.01\\
24.68	0.01\\
24.69	0.01\\
24.7	0.01\\
24.71	0.01\\
24.72	0.01\\
24.73	0.01\\
24.74	0.01\\
24.75	0.01\\
24.76	0.01\\
24.77	0.01\\
24.78	0.01\\
24.79	0.01\\
24.8	0.01\\
24.81	0.01\\
24.82	0.01\\
24.83	0.01\\
24.84	0.01\\
24.85	0.01\\
24.86	0.01\\
24.87	0.01\\
24.88	0.01\\
24.89	0.01\\
24.9	0.01\\
24.91	0.01\\
24.92	0.01\\
24.93	0.01\\
24.94	0.01\\
24.95	0.01\\
24.96	0.01\\
24.97	0.01\\
24.98	0.01\\
24.99	0.01\\
25	0.01\\
25.01	0.01\\
25.02	0.01\\
25.03	0.01\\
25.04	0.01\\
25.05	0.01\\
25.06	0.01\\
25.07	0.01\\
25.08	0.01\\
25.09	0.01\\
25.1	0.01\\
25.11	0.01\\
25.12	0.01\\
25.13	0.01\\
25.14	0.01\\
25.15	0.01\\
25.16	0.01\\
25.17	0.01\\
25.18	0.01\\
25.19	0.01\\
25.2	0.01\\
25.21	0.01\\
25.22	0.01\\
25.23	0.01\\
25.24	0.01\\
25.25	0.01\\
25.26	0.01\\
25.27	0.01\\
25.28	0.01\\
25.29	0.01\\
25.3	0.01\\
25.31	0.01\\
25.32	0.01\\
25.33	0.01\\
25.34	0.01\\
25.35	0.01\\
25.36	0.01\\
25.37	0.01\\
25.38	0.01\\
25.39	0.01\\
25.4	0.01\\
25.41	0.01\\
25.42	0.01\\
25.43	0.01\\
25.44	0.01\\
25.45	0.01\\
25.46	0.01\\
25.47	0.01\\
25.48	0.01\\
25.49	0.01\\
25.5	0.01\\
25.51	0.01\\
25.52	0.01\\
25.53	0.01\\
25.54	0.01\\
25.55	0.01\\
25.56	0.01\\
25.57	0.01\\
25.58	0.01\\
25.59	0.01\\
25.6	0.01\\
25.61	0.01\\
25.62	0.01\\
25.63	0.01\\
25.64	0.01\\
25.65	0.01\\
25.66	0.01\\
25.67	0.01\\
25.68	0.01\\
25.69	0.01\\
25.7	0.01\\
25.71	0.01\\
25.72	0.01\\
25.73	0.01\\
25.74	0.01\\
25.75	0.01\\
25.76	0.01\\
25.77	0.01\\
25.78	0.01\\
25.79	0.01\\
25.8	0.01\\
25.81	0.01\\
25.82	0.01\\
25.83	0.01\\
25.84	0.01\\
25.85	0.01\\
25.86	0.01\\
25.87	0.01\\
25.88	0.01\\
25.89	0.01\\
25.9	0.01\\
25.91	0.01\\
25.92	0.01\\
25.93	0.01\\
25.94	0.01\\
25.95	0.01\\
25.96	0.01\\
25.97	0.01\\
25.98	0.01\\
25.99	0.01\\
26	0.01\\
26.01	0.01\\
26.02	0.01\\
26.03	0.01\\
26.04	0.01\\
26.05	0.01\\
26.06	0.01\\
26.07	0.01\\
26.08	0.01\\
26.09	0.01\\
26.1	0.01\\
26.11	0.01\\
26.12	0.01\\
26.13	0.01\\
26.14	0.01\\
26.15	0.01\\
26.16	0.01\\
26.17	0.01\\
26.18	0.01\\
26.19	0.01\\
26.2	0.01\\
26.21	0.01\\
26.22	0.01\\
26.23	0.01\\
26.24	0.01\\
26.25	0.01\\
26.26	0.01\\
26.27	0.01\\
26.28	0.01\\
26.29	0.01\\
26.3	0.01\\
26.31	0.01\\
26.32	0.01\\
26.33	0.01\\
26.34	0.01\\
26.35	0.01\\
26.36	0.01\\
26.37	0.01\\
26.38	0.01\\
26.39	0.01\\
26.4	0.01\\
26.41	0.01\\
26.42	0.01\\
26.43	0.01\\
26.44	0.01\\
26.45	0.01\\
26.46	0.01\\
26.47	0.01\\
26.48	0.01\\
26.49	0.01\\
26.5	0.01\\
26.51	0.01\\
26.52	0.01\\
26.53	0.01\\
26.54	0.01\\
26.55	0.01\\
26.56	0.01\\
26.57	0.01\\
26.58	0.01\\
26.59	0.01\\
26.6	0.01\\
26.61	0.01\\
26.62	0.01\\
26.63	0.01\\
26.64	0.01\\
26.65	0.01\\
26.66	0.01\\
26.67	0.01\\
26.68	0.01\\
26.69	0.01\\
26.7	0.01\\
26.71	0.01\\
26.72	0.01\\
26.73	0.01\\
26.74	0.01\\
26.75	0.01\\
26.76	0.01\\
26.77	0.01\\
26.78	0.01\\
26.79	0.01\\
26.8	0.01\\
26.81	0.01\\
26.82	0.01\\
26.83	0.01\\
26.84	0.01\\
26.85	0.01\\
26.86	0.01\\
26.87	0.01\\
26.88	0.01\\
26.89	0.01\\
26.9	0.01\\
26.91	0.01\\
26.92	0.01\\
26.93	0.01\\
26.94	0.01\\
26.95	0.01\\
26.96	0.01\\
26.97	0.01\\
26.98	0.01\\
26.99	0.01\\
27	0.01\\
27.01	0.01\\
27.02	0.01\\
27.03	0.01\\
27.04	0.01\\
27.05	0.01\\
27.06	0.01\\
27.07	0.01\\
27.08	0.01\\
27.09	0.01\\
27.1	0.01\\
27.11	0.01\\
27.12	0.01\\
27.13	0.01\\
27.14	0.01\\
27.15	0.01\\
27.16	0.01\\
27.17	0.01\\
27.18	0.01\\
27.19	0.01\\
27.2	0.01\\
27.21	0.01\\
27.22	0.01\\
27.23	0.01\\
27.24	0.01\\
27.25	0.01\\
27.26	0.01\\
27.27	0.01\\
27.28	0.01\\
27.29	0.01\\
27.3	0.01\\
27.31	0.01\\
27.32	0.01\\
27.33	0.01\\
27.34	0.01\\
27.35	0.01\\
27.36	0.01\\
27.37	0.01\\
27.38	0.01\\
27.39	0.01\\
27.4	0.01\\
27.41	0.01\\
27.42	0.01\\
27.43	0.01\\
27.44	0.01\\
27.45	0.01\\
27.46	0.01\\
27.47	0.01\\
27.48	0.01\\
27.49	0.01\\
27.5	0.01\\
27.51	0.01\\
27.52	0.01\\
27.53	0.01\\
27.54	0.01\\
27.55	0.01\\
27.56	0.01\\
27.57	0.01\\
27.58	0.01\\
27.59	0.01\\
27.6	0.01\\
27.61	0.01\\
27.62	0.01\\
27.63	0.01\\
27.64	0.01\\
27.65	0.01\\
27.66	0.01\\
27.67	0.01\\
27.68	0.01\\
27.69	0.01\\
27.7	0.01\\
27.71	0.01\\
27.72	0.01\\
27.73	0.01\\
27.74	0.01\\
27.75	0.01\\
27.76	0.01\\
27.77	0.01\\
27.78	0.01\\
27.79	0.01\\
27.8	0.01\\
27.81	0.01\\
27.82	0.01\\
27.83	0.01\\
27.84	0.01\\
27.85	0.01\\
27.86	0.01\\
27.87	0.01\\
27.88	0.01\\
27.89	0.01\\
27.9	0.01\\
27.91	0.01\\
27.92	0.01\\
27.93	0.01\\
27.94	0.01\\
27.95	0.01\\
27.96	0.01\\
27.97	0.01\\
27.98	0.01\\
27.99	0.01\\
28	0.01\\
28.01	0.01\\
28.02	0.01\\
28.03	0.01\\
28.04	0.01\\
28.05	0.01\\
28.06	0.01\\
28.07	0.01\\
28.08	0.01\\
28.09	0.01\\
28.1	0.01\\
28.11	0.01\\
28.12	0.01\\
28.13	0.01\\
28.14	0.01\\
28.15	0.01\\
28.16	0.01\\
28.17	0.01\\
28.18	0.01\\
28.19	0.01\\
28.2	0.01\\
28.21	0.01\\
28.22	0.01\\
28.23	0.01\\
28.24	0.01\\
28.25	0.01\\
28.26	0.01\\
28.27	0.01\\
28.28	0.01\\
28.29	0.01\\
28.3	0.01\\
28.31	0.01\\
28.32	0.01\\
28.33	0.01\\
28.34	0.01\\
28.35	0.01\\
28.36	0.01\\
28.37	0.01\\
28.38	0.01\\
28.39	0.01\\
28.4	0.01\\
28.41	0.01\\
28.42	0.01\\
28.43	0.01\\
28.44	0.01\\
28.45	0.01\\
28.46	0.01\\
28.47	0.01\\
28.48	0.01\\
28.49	0.01\\
28.5	0.01\\
28.51	0.01\\
28.52	0.01\\
28.53	0.01\\
28.54	0.01\\
28.55	0.01\\
28.56	0.01\\
28.57	0.01\\
28.58	0.01\\
28.59	0.01\\
28.6	0.01\\
28.61	0.01\\
28.62	0.01\\
28.63	0.01\\
28.64	0.01\\
28.65	0.01\\
28.66	0.01\\
28.67	0.01\\
28.68	0.01\\
28.69	0.01\\
28.7	0.01\\
28.71	0.01\\
28.72	0.01\\
28.73	0.01\\
28.74	0.01\\
28.75	0.01\\
28.76	0.01\\
28.77	0.01\\
28.78	0.01\\
28.79	0.01\\
28.8	0.01\\
28.81	0.01\\
28.82	0.01\\
28.83	0.01\\
28.84	0.01\\
28.85	0.01\\
28.86	0.01\\
28.87	0.01\\
28.88	0.01\\
28.89	0.01\\
28.9	0.01\\
28.91	0.01\\
28.92	0.01\\
28.93	0.01\\
28.94	0.01\\
28.95	0.01\\
28.96	0.01\\
28.97	0.01\\
28.98	0.01\\
28.99	0.01\\
29	0.01\\
29.01	0.01\\
29.02	0.01\\
29.03	0.01\\
29.04	0.01\\
29.05	0.01\\
29.06	0.01\\
29.07	0.01\\
29.08	0.01\\
29.09	0.01\\
29.1	0.01\\
29.11	0.01\\
29.12	0.01\\
29.13	0.01\\
29.14	0.01\\
29.15	0.01\\
29.16	0.01\\
29.17	0.01\\
29.18	0.01\\
29.19	0.01\\
29.2	0.01\\
29.21	0.01\\
29.22	0.01\\
29.23	0.01\\
29.24	0.01\\
29.25	0.01\\
29.26	0.01\\
29.27	0.01\\
29.28	0.01\\
29.29	0.01\\
29.3	0.01\\
29.31	0.01\\
29.32	0.01\\
29.33	0.01\\
29.34	0.01\\
29.35	0.01\\
29.36	0.01\\
29.37	0.01\\
29.38	0.01\\
29.39	0.01\\
29.4	0.01\\
29.41	0.01\\
29.42	0.01\\
29.43	0.01\\
29.44	0.01\\
29.45	0.01\\
29.46	0.01\\
29.47	0.01\\
29.48	0.01\\
29.49	0.01\\
29.5	0.01\\
29.51	0.01\\
29.52	0.01\\
29.53	0.01\\
29.54	0.01\\
29.55	0.01\\
29.56	0.01\\
29.57	0.01\\
29.58	0.01\\
29.59	0.01\\
29.6	0.01\\
29.61	0.01\\
29.62	0.01\\
29.63	0.01\\
29.64	0.01\\
29.65	0.01\\
29.66	0.01\\
29.67	0.01\\
29.68	0.01\\
29.69	0.01\\
29.7	0.01\\
29.71	0.01\\
29.72	0.01\\
29.73	0.01\\
29.74	0.01\\
29.75	0.01\\
29.76	0.01\\
29.77	0.01\\
29.78	0.01\\
29.79	0.01\\
29.8	0.01\\
29.81	0.01\\
29.82	0.01\\
29.83	0.01\\
29.84	0.01\\
29.85	0.01\\
29.86	0.01\\
29.87	0.01\\
29.88	0.01\\
29.89	0.01\\
29.9	0.01\\
29.91	0.01\\
29.92	0.01\\
29.93	0.01\\
29.94	0.01\\
29.95	0.01\\
29.96	0.01\\
29.97	0.01\\
29.98	0.01\\
29.99	0.01\\
30	0.01\\
30.01	0.01\\
30.02	0.01\\
30.03	0.01\\
30.04	0.01\\
30.05	0.01\\
30.06	0.01\\
30.07	0.01\\
30.08	0.01\\
30.09	0.01\\
30.1	0.01\\
30.11	0.01\\
30.12	0.01\\
30.13	0.01\\
30.14	0.01\\
30.15	0.01\\
30.16	0.01\\
30.17	0.01\\
30.18	0.01\\
30.19	0.01\\
30.2	0.01\\
30.21	0.01\\
30.22	0.01\\
30.23	0.01\\
30.24	0.01\\
30.25	0.01\\
30.26	0.01\\
30.27	0.01\\
30.28	0.01\\
30.29	0.01\\
30.3	0.01\\
30.31	0.01\\
30.32	0.01\\
30.33	0.01\\
30.34	0.01\\
30.35	0.01\\
30.36	0.01\\
30.37	0.01\\
30.38	0.01\\
30.39	0.01\\
30.4	0.01\\
30.41	0.01\\
30.42	0.01\\
30.43	0.01\\
30.44	0.01\\
30.45	0.01\\
30.46	0.01\\
30.47	0.01\\
30.48	0.01\\
30.49	0.01\\
30.5	0.01\\
30.51	0.01\\
30.52	0.01\\
30.53	0.01\\
30.54	0.01\\
30.55	0.01\\
30.56	0.01\\
30.57	0.01\\
30.58	0.01\\
30.59	0.01\\
30.6	0.01\\
30.61	0.01\\
30.62	0.01\\
30.63	0.01\\
30.64	0.01\\
30.65	0.01\\
30.66	0.01\\
30.67	0.01\\
30.68	0.01\\
30.69	0.01\\
30.7	0.01\\
30.71	0.01\\
30.72	0.01\\
30.73	0.01\\
30.74	0.01\\
30.75	0.01\\
30.76	0.01\\
30.77	0.01\\
30.78	0.01\\
30.79	0.01\\
30.8	0.01\\
30.81	0.01\\
30.82	0.01\\
30.83	0.01\\
30.84	0.01\\
30.85	0.01\\
30.86	0.01\\
30.87	0.01\\
30.88	0.01\\
30.89	0.01\\
30.9	0.01\\
30.91	0.01\\
30.92	0.01\\
30.93	0.01\\
30.94	0.01\\
30.95	0.01\\
30.96	0.01\\
30.97	0.01\\
30.98	0.01\\
30.99	0.01\\
31	0.01\\
31.01	0.01\\
31.02	0.01\\
31.03	0.01\\
31.04	0.01\\
31.05	0.01\\
31.06	0.01\\
31.07	0.01\\
31.08	0.01\\
31.09	0.01\\
31.1	0.01\\
31.11	0.01\\
31.12	0.01\\
31.13	0.01\\
31.14	0.01\\
31.15	0.01\\
31.16	0.01\\
31.17	0.01\\
31.18	0.01\\
31.19	0.01\\
31.2	0.01\\
31.21	0.01\\
31.22	0.01\\
31.23	0.01\\
31.24	0.01\\
31.25	0.01\\
31.26	0.01\\
31.27	0.01\\
31.28	0.01\\
31.29	0.01\\
31.3	0.01\\
31.31	0.01\\
31.32	0.01\\
31.33	0.01\\
31.34	0.01\\
31.35	0.01\\
31.36	0.01\\
31.37	0.01\\
31.38	0.01\\
31.39	0.01\\
31.4	0.01\\
31.41	0.01\\
31.42	0.01\\
31.43	0.01\\
31.44	0.01\\
31.45	0.01\\
31.46	0.01\\
31.47	0.01\\
31.48	0.01\\
31.49	0.01\\
31.5	0.01\\
31.51	0.01\\
31.52	0.01\\
31.53	0.01\\
31.54	0.01\\
31.55	0.01\\
31.56	0.01\\
31.57	0.01\\
31.58	0.01\\
31.59	0.01\\
31.6	0.01\\
31.61	0.01\\
31.62	0.01\\
31.63	0.01\\
31.64	0.01\\
31.65	0.01\\
31.66	0.01\\
31.67	0.01\\
31.68	0.01\\
31.69	0.01\\
31.7	0.01\\
31.71	0.01\\
31.72	0.01\\
31.73	0.01\\
31.74	0.01\\
31.75	0.01\\
31.76	0.01\\
31.77	0.01\\
31.78	0.01\\
31.79	0.01\\
31.8	0.01\\
31.81	0.01\\
31.82	0.01\\
31.83	0.01\\
31.84	0.01\\
31.85	0.01\\
31.86	0.01\\
31.87	0.01\\
31.88	0.01\\
31.89	0.01\\
31.9	0.01\\
31.91	0.01\\
31.92	0.01\\
31.93	0.01\\
31.94	0.01\\
31.95	0.01\\
31.96	0.01\\
31.97	0.01\\
31.98	0.01\\
31.99	0.01\\
32	0.01\\
32.01	0.01\\
32.02	0.01\\
32.03	0.01\\
32.04	0.01\\
32.05	0.01\\
32.06	0.01\\
32.07	0.01\\
32.08	0.01\\
32.09	0.01\\
32.1	0.01\\
32.11	0.01\\
32.12	0.01\\
32.13	0.01\\
32.14	0.01\\
32.15	0.01\\
32.16	0.01\\
32.17	0.01\\
32.18	0.01\\
32.19	0.01\\
32.2	0.01\\
32.21	0.01\\
32.22	0.01\\
32.23	0.01\\
32.24	0.01\\
32.25	0.01\\
32.26	0.01\\
32.27	0.01\\
32.28	0.01\\
32.29	0.01\\
32.3	0.01\\
32.31	0.01\\
32.32	0.01\\
32.33	0.01\\
32.34	0.01\\
32.35	0.01\\
32.36	0.01\\
32.37	0.01\\
32.38	0.01\\
32.39	0.01\\
32.4	0.01\\
32.41	0.01\\
32.42	0.01\\
32.43	0.01\\
32.44	0.01\\
32.45	0.01\\
32.46	0.01\\
32.47	0.01\\
32.48	0.01\\
32.49	0.01\\
32.5	0.01\\
32.51	0.01\\
32.52	0.01\\
32.53	0.01\\
32.54	0.01\\
32.55	0.01\\
32.56	0.01\\
32.57	0.01\\
32.58	0.01\\
32.59	0.01\\
32.6	0.01\\
32.61	0.01\\
32.62	0.01\\
32.63	0.01\\
32.64	0.01\\
32.65	0.01\\
32.66	0.01\\
32.67	0.01\\
32.68	0.01\\
32.69	0.01\\
32.7	0.01\\
32.71	0.01\\
32.72	0.01\\
32.73	0.01\\
32.74	0.01\\
32.75	0.01\\
32.76	0.01\\
32.77	0.01\\
32.78	0.01\\
32.79	0.01\\
32.8	0.01\\
32.81	0.01\\
32.82	0.01\\
32.83	0.01\\
32.84	0.01\\
32.85	0.01\\
32.86	0.01\\
32.87	0.01\\
32.88	0.01\\
32.89	0.01\\
32.9	0.01\\
32.91	0.01\\
32.92	0.01\\
32.93	0.01\\
32.94	0.01\\
32.95	0.01\\
32.96	0.01\\
32.97	0.01\\
32.98	0.01\\
32.99	0.01\\
33	0.01\\
33.01	0.01\\
33.02	0.01\\
33.03	0.01\\
33.04	0.01\\
33.05	0.01\\
33.06	0.01\\
33.07	0.01\\
33.08	0.01\\
33.09	0.01\\
33.1	0.01\\
33.11	0.01\\
33.12	0.01\\
33.13	0.01\\
33.14	0.01\\
33.15	0.01\\
33.16	0.01\\
33.17	0.01\\
33.18	0.01\\
33.19	0.01\\
33.2	0.01\\
33.21	0.01\\
33.22	0.01\\
33.23	0.01\\
33.24	0.01\\
33.25	0.01\\
33.26	0.01\\
33.27	0.01\\
33.28	0.01\\
33.29	0.01\\
33.3	0.01\\
33.31	0.01\\
33.32	0.01\\
33.33	0.01\\
33.34	0.01\\
33.35	0.01\\
33.36	0.01\\
33.37	0.01\\
33.38	0.01\\
33.39	0.01\\
33.4	0.01\\
33.41	0.01\\
33.42	0.01\\
33.43	0.01\\
33.44	0.01\\
33.45	0.01\\
33.46	0.01\\
33.47	0.01\\
33.48	0.01\\
33.49	0.01\\
33.5	0.01\\
33.51	0.01\\
33.52	0.01\\
33.53	0.01\\
33.54	0.01\\
33.55	0.01\\
33.56	0.01\\
33.57	0.01\\
33.58	0.01\\
33.59	0.01\\
33.6	0.01\\
33.61	0.01\\
33.62	0.01\\
33.63	0.01\\
33.64	0.01\\
33.65	0.01\\
33.66	0.01\\
33.67	0.01\\
33.68	0.01\\
33.69	0.01\\
33.7	0.01\\
33.71	0.01\\
33.72	0.01\\
33.73	0.01\\
33.74	0.01\\
33.75	0.01\\
33.76	0.01\\
33.77	0.01\\
33.78	0.01\\
33.79	0.01\\
33.8	0.01\\
33.81	0.01\\
33.82	0.01\\
33.83	0.01\\
33.84	0.01\\
33.85	0.01\\
33.86	0.01\\
33.87	0.01\\
33.88	0.01\\
33.89	0.01\\
33.9	0.01\\
33.91	0.01\\
33.92	0.01\\
33.93	0.01\\
33.94	0.01\\
33.95	0.01\\
33.96	0.01\\
33.97	0.01\\
33.98	0.01\\
33.99	0.01\\
34	0.01\\
34.01	0.01\\
34.02	0.01\\
34.03	0.01\\
34.04	0.01\\
34.05	0.01\\
34.06	0.01\\
34.07	0.01\\
34.08	0.01\\
34.09	0.01\\
34.1	0.01\\
34.11	0.01\\
34.12	0.01\\
34.13	0.01\\
34.14	0.01\\
34.15	0.01\\
34.16	0.01\\
34.17	0.01\\
34.18	0.01\\
34.19	0.01\\
34.2	0.01\\
34.21	0.01\\
34.22	0.01\\
34.23	0.01\\
34.24	0.01\\
34.25	0.01\\
34.26	0.01\\
34.27	0.01\\
34.28	0.01\\
34.29	0.01\\
34.3	0.01\\
34.31	0.01\\
34.32	0.01\\
34.33	0.01\\
34.34	0.01\\
34.35	0.01\\
34.36	0.01\\
34.37	0.01\\
34.38	0.01\\
34.39	0.01\\
34.4	0.01\\
34.41	0.01\\
34.42	0.01\\
34.43	0.01\\
34.44	0.01\\
34.45	0.01\\
34.46	0.01\\
34.47	0.01\\
34.48	0.01\\
34.49	0.01\\
34.5	0.01\\
34.51	0.01\\
34.52	0.01\\
34.53	0.01\\
34.54	0.01\\
34.55	0.01\\
34.56	0.01\\
34.57	0.01\\
34.58	0.01\\
34.59	0.01\\
34.6	0.01\\
34.61	0.01\\
34.62	0.01\\
34.63	0.01\\
34.64	0.01\\
34.65	0.01\\
34.66	0.01\\
34.67	0.01\\
34.68	0.01\\
34.69	0.01\\
34.7	0.01\\
34.71	0.01\\
34.72	0.01\\
34.73	0.01\\
34.74	0.01\\
34.75	0.01\\
34.76	0.01\\
34.77	0.01\\
34.78	0.01\\
34.79	0.01\\
34.8	0.01\\
34.81	0.01\\
34.82	0.01\\
34.83	0.01\\
34.84	0.01\\
34.85	0.01\\
34.86	0.01\\
34.87	0.01\\
34.88	0.01\\
34.89	0.01\\
34.9	0.01\\
34.91	0.01\\
34.92	0.01\\
34.93	0.01\\
34.94	0.01\\
34.95	0.01\\
34.96	0.01\\
34.97	0.01\\
34.98	0.01\\
34.99	0.01\\
35	0.01\\
35.01	0.01\\
35.02	0.01\\
35.03	0.01\\
35.04	0.01\\
35.05	0.01\\
35.06	0.01\\
35.07	0.01\\
35.08	0.01\\
35.09	0.01\\
35.1	0.01\\
35.11	0.01\\
35.12	0.01\\
35.13	0.01\\
35.14	0.01\\
35.15	0.01\\
35.16	0.01\\
35.17	0.01\\
35.18	0.01\\
35.19	0.01\\
35.2	0.01\\
35.21	0.01\\
35.22	0.01\\
35.23	0.01\\
35.24	0.01\\
35.25	0.01\\
35.26	0.01\\
35.27	0.01\\
35.28	0.01\\
35.29	0.01\\
35.3	0.01\\
35.31	0.01\\
35.32	0.01\\
35.33	0.01\\
35.34	0.01\\
35.35	0.01\\
35.36	0.01\\
35.37	0.01\\
35.38	0.01\\
35.39	0.01\\
35.4	0.01\\
35.41	0.01\\
35.42	0.01\\
35.43	0.01\\
35.44	0.01\\
35.45	0.01\\
35.46	0.01\\
35.47	0.01\\
35.48	0.01\\
35.49	0.01\\
35.5	0.01\\
35.51	0.01\\
35.52	0.01\\
35.53	0.01\\
35.54	0.01\\
35.55	0.01\\
35.56	0.01\\
35.57	0.01\\
35.58	0.01\\
35.59	0.01\\
35.6	0.01\\
35.61	0.01\\
35.62	0.01\\
35.63	0.01\\
35.64	0.01\\
35.65	0.01\\
35.66	0.01\\
35.67	0.01\\
35.68	0.01\\
35.69	0.01\\
35.7	0.01\\
35.71	0.01\\
35.72	0.01\\
35.73	0.01\\
35.74	0.01\\
35.75	0.01\\
35.76	0.01\\
35.77	0.01\\
35.78	0.01\\
35.79	0.01\\
35.8	0.01\\
35.81	0.01\\
35.82	0.01\\
35.83	0.01\\
35.84	0.01\\
35.85	0.01\\
35.86	0.01\\
35.87	0.01\\
35.88	0.01\\
35.89	0.01\\
35.9	0.01\\
35.91	0.01\\
35.92	0.01\\
35.93	0.01\\
35.94	0.01\\
35.95	0.01\\
35.96	0.01\\
35.97	0.01\\
35.98	0.01\\
35.99	0.01\\
36	0.01\\
36.01	0.01\\
36.02	0.01\\
36.03	0.01\\
36.04	0.01\\
36.05	0.01\\
36.06	0.01\\
36.07	0.01\\
36.08	0.01\\
36.09	0.01\\
36.1	0.01\\
36.11	0.01\\
36.12	0.01\\
36.13	0.01\\
36.14	0.01\\
36.15	0.01\\
36.16	0.01\\
36.17	0.01\\
36.18	0.01\\
36.19	0.01\\
36.2	0.01\\
36.21	0.01\\
36.22	0.01\\
36.23	0.01\\
36.24	0.01\\
36.25	0.01\\
36.26	0.01\\
36.27	0.01\\
36.28	0.01\\
36.29	0.01\\
36.3	0.01\\
36.31	0.01\\
36.32	0.01\\
36.33	0.01\\
36.34	0.01\\
36.35	0.01\\
36.36	0.01\\
36.37	0.01\\
36.38	0.01\\
36.39	0.01\\
36.4	0.01\\
36.41	0.01\\
36.42	0.01\\
36.43	0.01\\
36.44	0.01\\
36.45	0.01\\
36.46	0.01\\
36.47	0.01\\
36.48	0.01\\
36.49	0.01\\
36.5	0.01\\
36.51	0.01\\
36.52	0.01\\
36.53	0.01\\
36.54	0.01\\
36.55	0.01\\
36.56	0.01\\
36.57	0.01\\
36.58	0.01\\
36.59	0.01\\
36.6	0.01\\
36.61	0.01\\
36.62	0.01\\
36.63	0.01\\
36.64	0.01\\
36.65	0.01\\
36.66	0.01\\
36.67	0.01\\
36.68	0.01\\
36.69	0.01\\
36.7	0.01\\
36.71	0.01\\
36.72	0.01\\
36.73	0.01\\
36.74	0.01\\
36.75	0.01\\
36.76	0.01\\
36.77	0.01\\
36.78	0.01\\
36.79	0.01\\
36.8	0.01\\
36.81	0.01\\
36.82	0.01\\
36.83	0.01\\
36.84	0.01\\
36.85	0.01\\
36.86	0.01\\
36.87	0.01\\
36.88	0.01\\
36.89	0.01\\
36.9	0.01\\
36.91	0.01\\
36.92	0.01\\
36.93	0.01\\
36.94	0.01\\
36.95	0.01\\
36.96	0.01\\
36.97	0.01\\
36.98	0.01\\
36.99	0.01\\
37	0.01\\
37.01	0.01\\
37.02	0.01\\
37.03	0.01\\
37.04	0.01\\
37.05	0.01\\
37.06	0.01\\
37.07	0.01\\
37.08	0.01\\
37.09	0.01\\
37.1	0.01\\
37.11	0.01\\
37.12	0.01\\
37.13	0.01\\
37.14	0.01\\
37.15	0.01\\
37.16	0.01\\
37.17	0.01\\
37.18	0.01\\
37.19	0.01\\
37.2	0.01\\
37.21	0.01\\
37.22	0.01\\
37.23	0.01\\
37.24	0.01\\
37.25	0.01\\
37.26	0.01\\
37.27	0.01\\
37.28	0.01\\
37.29	0.01\\
37.3	0.01\\
37.31	0.01\\
37.32	0.01\\
37.33	0.01\\
37.34	0.01\\
37.35	0.01\\
37.36	0.01\\
37.37	0.01\\
37.38	0.01\\
37.39	0.01\\
37.4	0.01\\
37.41	0.01\\
37.42	0.01\\
37.43	0.01\\
37.44	0.01\\
37.45	0.01\\
37.46	0.01\\
37.47	0.01\\
37.48	0.01\\
37.49	0.01\\
37.5	0.01\\
37.51	0.01\\
37.52	0.01\\
37.53	0.01\\
37.54	0.01\\
37.55	0.01\\
37.56	0.01\\
37.57	0.01\\
37.58	0.01\\
37.59	0.01\\
37.6	0.01\\
37.61	0.01\\
37.62	0.01\\
37.63	0.01\\
37.64	0.01\\
37.65	0.01\\
37.66	0.01\\
37.67	0.01\\
37.68	0.01\\
37.69	0.01\\
37.7	0.01\\
37.71	0.01\\
37.72	0.01\\
37.73	0.01\\
37.74	0.01\\
37.75	0.01\\
37.76	0.01\\
37.77	0.01\\
37.78	0.01\\
37.79	0.01\\
37.8	0.01\\
37.81	0.01\\
37.82	0.01\\
37.83	0.01\\
37.84	0.01\\
37.85	0.01\\
37.86	0.01\\
37.87	0.01\\
37.88	0.01\\
37.89	0.01\\
37.9	0.01\\
37.91	0.01\\
37.92	0.01\\
37.93	0.01\\
37.94	0.01\\
37.95	0.01\\
37.96	0.01\\
37.97	0.01\\
37.98	0.01\\
37.99	0.01\\
38	0.01\\
38.01	0.01\\
38.02	0.01\\
38.03	0.01\\
38.04	0.01\\
38.05	0.01\\
38.06	0.01\\
38.07	0.01\\
38.08	0.01\\
38.09	0.01\\
38.1	0.01\\
38.11	0.01\\
38.12	0.01\\
38.13	0.01\\
38.14	0.01\\
38.15	0.01\\
38.16	0.01\\
38.17	0.01\\
38.18	0.01\\
38.19	0.01\\
38.2	0.01\\
38.21	0.01\\
38.22	0.01\\
38.23	0.01\\
38.24	0.01\\
38.25	0.01\\
38.26	0.01\\
38.27	0.01\\
38.28	0.01\\
38.29	0.01\\
38.3	0.01\\
38.31	0.01\\
38.32	0.01\\
38.33	0.01\\
38.34	0.01\\
38.35	0.01\\
38.36	0.01\\
38.37	0.01\\
38.38	0.01\\
38.39	0.01\\
38.4	0.01\\
38.41	0.01\\
38.42	0.01\\
38.43	0.01\\
38.44	0.01\\
38.45	0.01\\
38.46	0.01\\
38.47	0.01\\
38.48	0.01\\
38.49	0.01\\
38.5	0.01\\
38.51	0.01\\
38.52	0.01\\
38.53	0.01\\
38.54	0.01\\
38.55	0.01\\
38.56	0.01\\
38.57	0.01\\
38.58	0.01\\
38.59	0.01\\
38.6	0.01\\
38.61	0.01\\
38.62	0.01\\
38.63	0.01\\
38.64	0.01\\
38.65	0.01\\
38.66	0.01\\
38.67	0.01\\
38.68	0.01\\
38.69	0.01\\
38.7	0.01\\
38.71	0.01\\
38.72	0.01\\
38.73	0.01\\
38.74	0.01\\
38.75	0.01\\
38.76	0.01\\
38.77	0.01\\
38.78	0.01\\
38.79	0.01\\
38.8	0.01\\
38.81	0.01\\
38.82	0.01\\
38.83	0.01\\
38.84	0.01\\
38.85	0.01\\
38.86	0.01\\
38.87	0.01\\
38.88	0.01\\
38.89	0.01\\
38.9	0.01\\
38.91	0.01\\
38.92	0.01\\
38.93	0.01\\
38.94	0.01\\
38.95	0.01\\
38.96	0.01\\
38.97	0.01\\
38.98	0.01\\
38.99	0.01\\
39	0.01\\
39.01	0.01\\
39.02	0.01\\
39.03	0.01\\
39.04	0.01\\
39.05	0.01\\
39.06	0.01\\
39.07	0.01\\
39.08	0.01\\
39.09	0.01\\
39.1	0.01\\
39.11	0.01\\
39.12	0.01\\
39.13	0.01\\
39.14	0.01\\
39.15	0.01\\
39.16	0.01\\
39.17	0.01\\
39.18	0.01\\
39.19	0.01\\
39.2	0.01\\
39.21	0.01\\
39.22	0.01\\
39.23	0.01\\
39.24	0.01\\
39.25	0.01\\
39.26	0.01\\
39.27	0.01\\
39.28	0.01\\
39.29	0.01\\
39.3	0.01\\
39.31	0.01\\
39.32	0.01\\
39.33	0.01\\
39.34	0.01\\
39.35	0.01\\
39.36	0.01\\
39.37	0.01\\
39.38	0.01\\
39.39	0.01\\
39.4	0.01\\
39.41	0.01\\
39.42	0.01\\
39.43	0.01\\
39.44	0.01\\
39.45	0.01\\
39.46	0.01\\
39.47	0.01\\
39.48	0.01\\
39.49	0.01\\
39.5	0.01\\
39.51	0.01\\
39.52	0.01\\
39.53	0.01\\
39.54	0.01\\
39.55	0.01\\
39.56	0.01\\
39.57	0.01\\
39.58	0.01\\
39.59	0.01\\
39.6	0.01\\
39.61	0.01\\
39.62	0.01\\
39.63	0.01\\
39.64	0.01\\
39.65	0.01\\
39.66	0.01\\
39.67	0.01\\
39.68	0.01\\
39.69	0.01\\
39.7	0.01\\
39.71	0.01\\
39.72	0.01\\
39.73	0.01\\
39.74	0.01\\
39.75	0.01\\
39.76	0.01\\
39.77	0.01\\
39.78	0.01\\
39.79	0.01\\
39.8	0.01\\
39.81	0.01\\
39.82	0.01\\
39.83	0.01\\
39.84	0.01\\
39.85	0.01\\
39.86	0.01\\
39.87	0.01\\
39.88	0.01\\
39.89	0.01\\
39.9	0.01\\
39.91	0.01\\
39.92	0.01\\
39.93	0.01\\
39.94	0.01\\
39.95	0.01\\
39.96	0.01\\
39.97	0.01\\
39.98	0.01\\
39.99	0.01\\
40	0.01\\
40.01	0.01\\
};
\addplot [color=red,dashed,forget plot]
  table[row sep=crcr]{%
40.01	0.01\\
40.02	0.01\\
40.03	0.01\\
40.04	0.01\\
40.05	0.01\\
40.06	0.01\\
40.07	0.01\\
40.08	0.01\\
40.09	0.01\\
40.1	0.01\\
40.11	0.01\\
40.12	0.01\\
40.13	0.01\\
40.14	0.01\\
40.15	0.01\\
40.16	0.01\\
40.17	0.01\\
40.18	0.01\\
40.19	0.01\\
40.2	0.01\\
40.21	0.01\\
40.22	0.01\\
40.23	0.01\\
40.24	0.01\\
40.25	0.01\\
40.26	0.01\\
40.27	0.01\\
40.28	0.01\\
40.29	0.01\\
40.3	0.01\\
40.31	0.01\\
40.32	0.01\\
40.33	0.01\\
40.34	0.01\\
40.35	0.01\\
40.36	0.01\\
40.37	0.01\\
40.38	0.01\\
40.39	0.01\\
40.4	0.01\\
40.41	0.01\\
40.42	0.01\\
40.43	0.01\\
40.44	0.01\\
40.45	0.01\\
40.46	0.01\\
40.47	0.01\\
40.48	0.01\\
40.49	0.01\\
40.5	0.01\\
40.51	0.01\\
40.52	0.01\\
40.53	0.01\\
40.54	0.01\\
40.55	0.01\\
40.56	0.01\\
40.57	0.01\\
40.58	0.01\\
40.59	0.01\\
40.6	0.01\\
40.61	0.01\\
40.62	0.01\\
40.63	0.01\\
40.64	0.01\\
40.65	0.01\\
40.66	0.01\\
40.67	0.01\\
40.68	0.01\\
40.69	0.01\\
40.7	0.01\\
40.71	0.01\\
40.72	0.01\\
40.73	0.01\\
40.74	0.01\\
40.75	0.01\\
40.76	0.01\\
40.77	0.01\\
40.78	0.01\\
40.79	0.01\\
40.8	0.01\\
40.81	0.01\\
40.82	0.01\\
40.83	0.01\\
40.84	0.01\\
40.85	0.01\\
40.86	0.01\\
40.87	0.01\\
40.88	0.01\\
40.89	0.01\\
40.9	0.01\\
40.91	0.01\\
40.92	0.01\\
40.93	0.01\\
40.94	0.01\\
40.95	0.01\\
40.96	0.01\\
40.97	0.01\\
40.98	0.01\\
40.99	0.01\\
41	0.01\\
41.01	0.01\\
41.02	0.01\\
41.03	0.01\\
41.04	0.01\\
41.05	0.01\\
41.06	0.01\\
41.07	0.01\\
41.08	0.01\\
41.09	0.01\\
41.1	0.01\\
41.11	0.01\\
41.12	0.01\\
41.13	0.01\\
41.14	0.01\\
41.15	0.01\\
41.16	0.01\\
41.17	0.01\\
41.18	0.01\\
41.19	0.01\\
41.2	0.01\\
41.21	0.01\\
41.22	0.01\\
41.23	0.01\\
41.24	0.01\\
41.25	0.01\\
41.26	0.01\\
41.27	0.01\\
41.28	0.01\\
41.29	0.01\\
41.3	0.01\\
41.31	0.01\\
41.32	0.01\\
41.33	0.01\\
41.34	0.01\\
41.35	0.01\\
41.36	0.01\\
41.37	0.01\\
41.38	0.01\\
41.39	0.01\\
41.4	0.01\\
41.41	0.01\\
41.42	0.01\\
41.43	0.01\\
41.44	0.01\\
41.45	0.01\\
41.46	0.01\\
41.47	0.01\\
41.48	0.01\\
41.49	0.01\\
41.5	0.01\\
41.51	0.01\\
41.52	0.01\\
41.53	0.01\\
41.54	0.01\\
41.55	0.01\\
41.56	0.01\\
41.57	0.01\\
41.58	0.01\\
41.59	0.01\\
41.6	0.01\\
41.61	0.01\\
41.62	0.01\\
41.63	0.01\\
41.64	0.01\\
41.65	0.01\\
41.66	0.01\\
41.67	0.01\\
41.68	0.01\\
41.69	0.01\\
41.7	0.01\\
41.71	0.01\\
41.72	0.01\\
41.73	0.01\\
41.74	0.01\\
41.75	0.01\\
41.76	0.01\\
41.77	0.01\\
41.78	0.01\\
41.79	0.01\\
41.8	0.01\\
41.81	0.01\\
41.82	0.01\\
41.83	0.01\\
41.84	0.01\\
41.85	0.01\\
41.86	0.01\\
41.87	0.01\\
41.88	0.01\\
41.89	0.01\\
41.9	0.01\\
41.91	0.01\\
41.92	0.01\\
41.93	0.01\\
41.94	0.01\\
41.95	0.01\\
41.96	0.01\\
41.97	0.01\\
41.98	0.01\\
41.99	0.01\\
42	0.01\\
42.01	0.01\\
42.02	0.01\\
42.03	0.01\\
42.04	0.01\\
42.05	0.01\\
42.06	0.01\\
42.07	0.01\\
42.08	0.01\\
42.09	0.01\\
42.1	0.01\\
42.11	0.01\\
42.12	0.01\\
42.13	0.01\\
42.14	0.01\\
42.15	0.01\\
42.16	0.01\\
42.17	0.01\\
42.18	0.01\\
42.19	0.01\\
42.2	0.01\\
42.21	0.01\\
42.22	0.01\\
42.23	0.01\\
42.24	0.01\\
42.25	0.01\\
42.26	0.01\\
42.27	0.01\\
42.28	0.01\\
42.29	0.01\\
42.3	0.01\\
42.31	0.01\\
42.32	0.01\\
42.33	0.01\\
42.34	0.01\\
42.35	0.01\\
42.36	0.01\\
42.37	0.01\\
42.38	0.01\\
42.39	0.01\\
42.4	0.01\\
42.41	0.01\\
42.42	0.01\\
42.43	0.01\\
42.44	0.01\\
42.45	0.01\\
42.46	0.01\\
42.47	0.01\\
42.48	0.01\\
42.49	0.01\\
42.5	0.01\\
42.51	0.01\\
42.52	0.01\\
42.53	0.01\\
42.54	0.01\\
42.55	0.01\\
42.56	0.01\\
42.57	0.01\\
42.58	0.01\\
42.59	0.01\\
42.6	0.01\\
42.61	0.01\\
42.62	0.01\\
42.63	0.01\\
42.64	0.01\\
42.65	0.01\\
42.66	0.01\\
42.67	0.01\\
42.68	0.01\\
42.69	0.01\\
42.7	0.01\\
42.71	0.01\\
42.72	0.01\\
42.73	0.01\\
42.74	0.01\\
42.75	0.01\\
42.76	0.01\\
42.77	0.01\\
42.78	0.01\\
42.79	0.01\\
42.8	0.01\\
42.81	0.01\\
42.82	0.01\\
42.83	0.01\\
42.84	0.01\\
42.85	0.01\\
42.86	0.01\\
42.87	0.01\\
42.88	0.01\\
42.89	0.01\\
42.9	0.01\\
42.91	0.01\\
42.92	0.01\\
42.93	0.01\\
42.94	0.01\\
42.95	0.01\\
42.96	0.01\\
42.97	0.01\\
42.98	0.01\\
42.99	0.01\\
43	0.01\\
43.01	0.01\\
43.02	0.01\\
43.03	0.01\\
43.04	0.01\\
43.05	0.01\\
43.06	0.01\\
43.07	0.01\\
43.08	0.01\\
43.09	0.01\\
43.1	0.01\\
43.11	0.01\\
43.12	0.01\\
43.13	0.01\\
43.14	0.01\\
43.15	0.01\\
43.16	0.01\\
43.17	0.01\\
43.18	0.01\\
43.19	0.01\\
43.2	0.01\\
43.21	0.01\\
43.22	0.01\\
43.23	0.01\\
43.24	0.01\\
43.25	0.01\\
43.26	0.01\\
43.27	0.01\\
43.28	0.01\\
43.29	0.01\\
43.3	0.01\\
43.31	0.01\\
43.32	0.01\\
43.33	0.01\\
43.34	0.01\\
43.35	0.01\\
43.36	0.01\\
43.37	0.01\\
43.38	0.01\\
43.39	0.01\\
43.4	0.01\\
43.41	0.01\\
43.42	0.01\\
43.43	0.01\\
43.44	0.01\\
43.45	0.01\\
43.46	0.01\\
43.47	0.01\\
43.48	0.01\\
43.49	0.01\\
43.5	0.01\\
43.51	0.01\\
43.52	0.01\\
43.53	0.01\\
43.54	0.01\\
43.55	0.01\\
43.56	0.01\\
43.57	0.01\\
43.58	0.01\\
43.59	0.01\\
43.6	0.01\\
43.61	0.01\\
43.62	0.01\\
43.63	0.01\\
43.64	0.01\\
43.65	0.01\\
43.66	0.01\\
43.67	0.01\\
43.68	0.01\\
43.69	0.01\\
43.7	0.01\\
43.71	0.01\\
43.72	0.01\\
43.73	0.01\\
43.74	0.01\\
43.75	0.01\\
43.76	0.01\\
43.77	0.01\\
43.78	0.01\\
43.79	0.01\\
43.8	0.01\\
43.81	0.01\\
43.82	0.01\\
43.83	0.01\\
43.84	0.01\\
43.85	0.01\\
43.86	0.01\\
43.87	0.01\\
43.88	0.01\\
43.89	0.01\\
43.9	0.01\\
43.91	0.01\\
43.92	0.01\\
43.93	0.01\\
43.94	0.01\\
43.95	0.01\\
43.96	0.01\\
43.97	0.01\\
43.98	0.01\\
43.99	0.01\\
44	0.01\\
44.01	0.01\\
44.02	0.01\\
44.03	0.01\\
44.04	0.01\\
44.05	0.01\\
44.06	0.01\\
44.07	0.01\\
44.08	0.01\\
44.09	0.01\\
44.1	0.01\\
44.11	0.01\\
44.12	0.01\\
44.13	0.01\\
44.14	0.01\\
44.15	0.01\\
44.16	0.01\\
44.17	0.01\\
44.18	0.01\\
44.19	0.01\\
44.2	0.01\\
44.21	0.01\\
44.22	0.01\\
44.23	0.01\\
44.24	0.01\\
44.25	0.01\\
44.26	0.01\\
44.27	0.01\\
44.28	0.01\\
44.29	0.01\\
44.3	0.01\\
44.31	0.01\\
44.32	0.01\\
44.33	0.01\\
44.34	0.01\\
44.35	0.01\\
44.36	0.01\\
44.37	0.01\\
44.38	0.01\\
44.39	0.01\\
44.4	0.01\\
44.41	0.01\\
44.42	0.01\\
44.43	0.01\\
44.44	0.01\\
44.45	0.01\\
44.46	0.01\\
44.47	0.01\\
44.48	0.01\\
44.49	0.01\\
44.5	0.01\\
44.51	0.01\\
44.52	0.01\\
44.53	0.01\\
44.54	0.01\\
44.55	0.01\\
44.56	0.01\\
44.57	0.01\\
44.58	0.01\\
44.59	0.01\\
44.6	0.01\\
44.61	0.01\\
44.62	0.01\\
44.63	0.01\\
44.64	0.01\\
44.65	0.01\\
44.66	0.01\\
44.67	0.01\\
44.68	0.01\\
44.69	0.01\\
44.7	0.01\\
44.71	0.01\\
44.72	0.01\\
44.73	0.01\\
44.74	0.01\\
44.75	0.01\\
44.76	0.01\\
44.77	0.01\\
44.78	0.01\\
44.79	0.01\\
44.8	0.01\\
44.81	0.01\\
44.82	0.01\\
44.83	0.01\\
44.84	0.01\\
44.85	0.01\\
44.86	0.01\\
44.87	0.01\\
44.88	0.01\\
44.89	0.01\\
44.9	0.01\\
44.91	0.01\\
44.92	0.01\\
44.93	0.01\\
44.94	0.01\\
44.95	0.01\\
44.96	0.01\\
44.97	0.01\\
44.98	0.01\\
44.99	0.01\\
45	0.01\\
45.01	0.01\\
45.02	0.01\\
45.03	0.01\\
45.04	0.01\\
45.05	0.01\\
45.06	0.01\\
45.07	0.01\\
45.08	0.01\\
45.09	0.01\\
45.1	0.01\\
45.11	0.01\\
45.12	0.01\\
45.13	0.01\\
45.14	0.01\\
45.15	0.01\\
45.16	0.01\\
45.17	0.01\\
45.18	0.01\\
45.19	0.01\\
45.2	0.01\\
45.21	0.01\\
45.22	0.01\\
45.23	0.01\\
45.24	0.01\\
45.25	0.01\\
45.26	0.01\\
45.27	0.01\\
45.28	0.01\\
45.29	0.01\\
45.3	0.01\\
45.31	0.01\\
45.32	0.01\\
45.33	0.01\\
45.34	0.01\\
45.35	0.01\\
45.36	0.01\\
45.37	0.01\\
45.38	0.01\\
45.39	0.01\\
45.4	0.01\\
45.41	0.01\\
45.42	0.01\\
45.43	0.01\\
45.44	0.01\\
45.45	0.01\\
45.46	0.01\\
45.47	0.01\\
45.48	0.01\\
45.49	0.01\\
45.5	0.01\\
45.51	0.01\\
45.52	0.01\\
45.53	0.01\\
45.54	0.01\\
45.55	0.01\\
45.56	0.01\\
45.57	0.01\\
45.58	0.01\\
45.59	0.01\\
45.6	0.01\\
45.61	0.01\\
45.62	0.01\\
45.63	0.01\\
45.64	0.01\\
45.65	0.01\\
45.66	0.01\\
45.67	0.01\\
45.68	0.01\\
45.69	0.01\\
45.7	0.01\\
45.71	0.01\\
45.72	0.01\\
45.73	0.01\\
45.74	0.01\\
45.75	0.01\\
45.76	0.01\\
45.77	0.01\\
45.78	0.01\\
45.79	0.01\\
45.8	0.01\\
45.81	0.01\\
45.82	0.01\\
45.83	0.01\\
45.84	0.01\\
45.85	0.01\\
45.86	0.01\\
45.87	0.01\\
45.88	0.01\\
45.89	0.01\\
45.9	0.01\\
45.91	0.01\\
45.92	0.01\\
45.93	0.01\\
45.94	0.01\\
45.95	0.01\\
45.96	0.01\\
45.97	0.01\\
45.98	0.01\\
45.99	0.01\\
46	0.01\\
46.01	0.01\\
46.02	0.01\\
46.03	0.01\\
46.04	0.01\\
46.05	0.01\\
46.06	0.01\\
46.07	0.01\\
46.08	0.01\\
46.09	0.01\\
46.1	0.01\\
46.11	0.01\\
46.12	0.01\\
46.13	0.01\\
46.14	0.01\\
46.15	0.01\\
46.16	0.01\\
46.17	0.01\\
46.18	0.01\\
46.19	0.01\\
46.2	0.01\\
46.21	0.01\\
46.22	0.01\\
46.23	0.01\\
46.24	0.01\\
46.25	0.01\\
46.26	0.01\\
46.27	0.01\\
46.28	0.01\\
46.29	0.01\\
46.3	0.01\\
46.31	0.01\\
46.32	0.01\\
46.33	0.01\\
46.34	0.01\\
46.35	0.01\\
46.36	0.01\\
46.37	0.01\\
46.38	0.01\\
46.39	0.01\\
46.4	0.01\\
46.41	0.01\\
46.42	0.01\\
46.43	0.01\\
46.44	0.01\\
46.45	0.01\\
46.46	0.01\\
46.47	0.01\\
46.48	0.01\\
46.49	0.01\\
46.5	0.01\\
46.51	0.01\\
46.52	0.01\\
46.53	0.01\\
46.54	0.01\\
46.55	0.01\\
46.56	0.01\\
46.57	0.01\\
46.58	0.01\\
46.59	0.01\\
46.6	0.01\\
46.61	0.01\\
46.62	0.01\\
46.63	0.01\\
46.64	0.01\\
46.65	0.01\\
46.66	0.01\\
46.67	0.01\\
46.68	0.01\\
46.69	0.01\\
46.7	0.01\\
46.71	0.01\\
46.72	0.01\\
46.73	0.01\\
46.74	0.01\\
46.75	0.01\\
46.76	0.01\\
46.77	0.01\\
46.78	0.01\\
46.79	0.01\\
46.8	0.01\\
46.81	0.01\\
46.82	0.01\\
46.83	0.01\\
46.84	0.01\\
46.85	0.01\\
46.86	0.01\\
46.87	0.01\\
46.88	0.01\\
46.89	0.01\\
46.9	0.01\\
46.91	0.01\\
46.92	0.01\\
46.93	0.01\\
46.94	0.01\\
46.95	0.01\\
46.96	0.01\\
46.97	0.01\\
46.98	0.01\\
46.99	0.01\\
47	0.01\\
47.01	0.01\\
47.02	0.01\\
47.03	0.01\\
47.04	0.01\\
47.05	0.01\\
47.06	0.01\\
47.07	0.01\\
47.08	0.01\\
47.09	0.01\\
47.1	0.01\\
47.11	0.01\\
47.12	0.01\\
47.13	0.01\\
47.14	0.01\\
47.15	0.01\\
47.16	0.01\\
47.17	0.01\\
47.18	0.01\\
47.19	0.01\\
47.2	0.01\\
47.21	0.01\\
47.22	0.01\\
47.23	0.01\\
47.24	0.01\\
47.25	0.01\\
47.26	0.01\\
47.27	0.01\\
47.28	0.01\\
47.29	0.01\\
47.3	0.01\\
47.31	0.01\\
47.32	0.01\\
47.33	0.01\\
47.34	0.01\\
47.35	0.01\\
47.36	0.01\\
47.37	0.01\\
47.38	0.01\\
47.39	0.01\\
47.4	0.01\\
47.41	0.01\\
47.42	0.01\\
47.43	0.01\\
47.44	0.01\\
47.45	0.01\\
47.46	0.01\\
47.47	0.01\\
47.48	0.01\\
47.49	0.01\\
47.5	0.01\\
47.51	0.01\\
47.52	0.01\\
47.53	0.01\\
47.54	0.01\\
47.55	0.01\\
47.56	0.01\\
47.57	0.01\\
47.58	0.01\\
47.59	0.01\\
47.6	0.01\\
47.61	0.01\\
47.62	0.01\\
47.63	0.01\\
47.64	0.01\\
47.65	0.01\\
47.66	0.01\\
47.67	0.01\\
47.68	0.01\\
47.69	0.01\\
47.7	0.01\\
47.71	0.01\\
47.72	0.01\\
47.73	0.01\\
47.74	0.01\\
47.75	0.01\\
47.76	0.01\\
47.77	0.01\\
47.78	0.01\\
47.79	0.01\\
47.8	0.01\\
47.81	0.01\\
47.82	0.01\\
47.83	0.01\\
47.84	0.01\\
47.85	0.01\\
47.86	0.01\\
47.87	0.01\\
47.88	0.01\\
47.89	0.01\\
47.9	0.01\\
47.91	0.01\\
47.92	0.01\\
47.93	0.01\\
47.94	0.01\\
47.95	0.01\\
47.96	0.01\\
47.97	0.01\\
47.98	0.01\\
47.99	0.01\\
48	0.01\\
48.01	0.01\\
48.02	0.01\\
48.03	0.01\\
48.04	0.01\\
48.05	0.01\\
48.06	0.01\\
48.07	0.01\\
48.08	0.01\\
48.09	0.01\\
48.1	0.01\\
48.11	0.01\\
48.12	0.01\\
48.13	0.01\\
48.14	0.01\\
48.15	0.01\\
48.16	0.01\\
48.17	0.01\\
48.18	0.01\\
48.19	0.01\\
48.2	0.01\\
48.21	0.01\\
48.22	0.01\\
48.23	0.01\\
48.24	0.01\\
48.25	0.01\\
48.26	0.01\\
48.27	0.01\\
48.28	0.01\\
48.29	0.01\\
48.3	0.01\\
48.31	0.01\\
48.32	0.01\\
48.33	0.01\\
48.34	0.01\\
48.35	0.01\\
48.36	0.01\\
48.37	0.01\\
48.38	0.01\\
48.39	0.01\\
48.4	0.01\\
48.41	0.01\\
48.42	0.01\\
48.43	0.01\\
48.44	0.01\\
48.45	0.01\\
48.46	0.01\\
48.47	0.01\\
48.48	0.01\\
48.49	0.01\\
48.5	0.01\\
48.51	0.01\\
48.52	0.01\\
48.53	0.01\\
48.54	0.01\\
48.55	0.01\\
48.56	0.01\\
48.57	0.01\\
48.58	0.01\\
48.59	0.01\\
48.6	0.01\\
48.61	0.01\\
48.62	0.01\\
48.63	0.01\\
48.64	0.01\\
48.65	0.01\\
48.66	0.01\\
48.67	0.01\\
48.68	0.01\\
48.69	0.01\\
48.7	0.01\\
48.71	0.01\\
48.72	0.01\\
48.73	0.01\\
48.74	0.01\\
48.75	0.01\\
48.76	0.01\\
48.77	0.01\\
48.78	0.01\\
48.79	0.01\\
48.8	0.01\\
48.81	0.01\\
48.82	0.01\\
48.83	0.01\\
48.84	0.01\\
48.85	0.01\\
48.86	0.01\\
48.87	0.01\\
48.88	0.01\\
48.89	0.01\\
48.9	0.01\\
48.91	0.01\\
48.92	0.01\\
48.93	0.01\\
48.94	0.01\\
48.95	0.01\\
48.96	0.01\\
48.97	0.01\\
48.98	0.01\\
48.99	0.01\\
49	0.01\\
49.01	0.01\\
49.02	0.01\\
49.03	0.01\\
49.04	0.01\\
49.05	0.01\\
49.06	0.01\\
49.07	0.01\\
49.08	0.01\\
49.09	0.01\\
49.1	0.01\\
49.11	0.01\\
49.12	0.01\\
49.13	0.01\\
49.14	0.01\\
49.15	0.01\\
49.16	0.01\\
49.17	0.01\\
49.18	0.01\\
49.19	0.01\\
49.2	0.01\\
49.21	0.01\\
49.22	0.01\\
49.23	0.01\\
49.24	0.01\\
49.25	0.01\\
49.26	0.01\\
49.27	0.01\\
49.28	0.01\\
49.29	0.01\\
49.3	0.01\\
49.31	0.01\\
49.32	0.01\\
49.33	0.01\\
49.34	0.01\\
49.35	0.01\\
49.36	0.01\\
49.37	0.01\\
49.38	0.01\\
49.39	0.01\\
49.4	0.01\\
49.41	0.01\\
49.42	0.01\\
49.43	0.01\\
49.44	0.01\\
49.45	0.01\\
49.46	0.01\\
49.47	0.01\\
49.48	0.01\\
49.49	0.01\\
49.5	0.01\\
49.51	0.01\\
49.52	0.01\\
49.53	0.01\\
49.54	0.01\\
49.55	0.01\\
49.56	0.01\\
49.57	0.01\\
49.58	0.01\\
49.59	0.01\\
49.6	0.01\\
49.61	0.01\\
49.62	0.01\\
49.63	0.01\\
49.64	0.01\\
49.65	0.01\\
49.66	0.01\\
49.67	0.01\\
49.68	0.01\\
49.69	0.01\\
49.7	0.01\\
49.71	0.01\\
49.72	0.01\\
49.73	0.01\\
49.74	0.01\\
49.75	0.01\\
49.76	0.01\\
49.77	0.01\\
49.78	0.01\\
49.79	0.01\\
49.8	0.01\\
49.81	0.01\\
49.82	0.01\\
49.83	0.01\\
49.84	0.01\\
49.85	0.01\\
49.86	0.01\\
49.87	0.01\\
49.88	0.01\\
49.89	0.01\\
49.9	0.01\\
49.91	0.01\\
49.92	0.01\\
49.93	0.01\\
49.94	0.01\\
49.95	0.01\\
49.96	0.01\\
49.97	0.01\\
49.98	0.01\\
49.99	0.01\\
50	0.01\\
50.01	0.01\\
50.02	0.01\\
50.03	0.01\\
50.04	0.01\\
50.05	0.01\\
50.06	0.01\\
50.07	0.01\\
50.08	0.01\\
50.09	0.01\\
50.1	0.01\\
50.11	0.01\\
50.12	0.01\\
50.13	0.01\\
50.14	0.01\\
50.15	0.01\\
50.16	0.01\\
50.17	0.01\\
50.18	0.01\\
50.19	0.01\\
50.2	0.01\\
50.21	0.01\\
50.22	0.01\\
50.23	0.01\\
50.24	0.01\\
50.25	0.01\\
50.26	0.01\\
50.27	0.01\\
50.28	0.01\\
50.29	0.01\\
50.3	0.01\\
50.31	0.01\\
50.32	0.01\\
50.33	0.01\\
50.34	0.01\\
50.35	0.01\\
50.36	0.01\\
50.37	0.01\\
50.38	0.01\\
50.39	0.01\\
50.4	0.01\\
50.41	0.01\\
50.42	0.01\\
50.43	0.01\\
50.44	0.01\\
50.45	0.01\\
50.46	0.01\\
50.47	0.01\\
50.48	0.01\\
50.49	0.01\\
50.5	0.01\\
50.51	0.01\\
50.52	0.01\\
50.53	0.01\\
50.54	0.01\\
50.55	0.01\\
50.56	0.01\\
50.57	0.01\\
50.58	0.01\\
50.59	0.01\\
50.6	0.01\\
50.61	0.01\\
50.62	0.01\\
50.63	0.01\\
50.64	0.01\\
50.65	0.01\\
50.66	0.01\\
50.67	0.01\\
50.68	0.01\\
50.69	0.01\\
50.7	0.01\\
50.71	0.01\\
50.72	0.01\\
50.73	0.01\\
50.74	0.01\\
50.75	0.01\\
50.76	0.01\\
50.77	0.01\\
50.78	0.01\\
50.79	0.01\\
50.8	0.01\\
50.81	0.01\\
50.82	0.01\\
50.83	0.01\\
50.84	0.01\\
50.85	0.01\\
50.86	0.01\\
50.87	0.01\\
50.88	0.01\\
50.89	0.01\\
50.9	0.01\\
50.91	0.01\\
50.92	0.01\\
50.93	0.01\\
50.94	0.01\\
50.95	0.01\\
50.96	0.01\\
50.97	0.01\\
50.98	0.01\\
50.99	0.01\\
51	0.01\\
51.01	0.01\\
51.02	0.01\\
51.03	0.01\\
51.04	0.01\\
51.05	0.01\\
51.06	0.01\\
51.07	0.01\\
51.08	0.01\\
51.09	0.01\\
51.1	0.01\\
51.11	0.01\\
51.12	0.01\\
51.13	0.01\\
51.14	0.01\\
51.15	0.01\\
51.16	0.01\\
51.17	0.01\\
51.18	0.01\\
51.19	0.01\\
51.2	0.01\\
51.21	0.01\\
51.22	0.01\\
51.23	0.01\\
51.24	0.01\\
51.25	0.01\\
51.26	0.01\\
51.27	0.01\\
51.28	0.01\\
51.29	0.01\\
51.3	0.01\\
51.31	0.01\\
51.32	0.01\\
51.33	0.01\\
51.34	0.01\\
51.35	0.01\\
51.36	0.01\\
51.37	0.01\\
51.38	0.01\\
51.39	0.01\\
51.4	0.01\\
51.41	0.01\\
51.42	0.01\\
51.43	0.01\\
51.44	0.01\\
51.45	0.01\\
51.46	0.01\\
51.47	0.01\\
51.48	0.01\\
51.49	0.01\\
51.5	0.01\\
51.51	0.01\\
51.52	0.01\\
51.53	0.01\\
51.54	0.01\\
51.55	0.01\\
51.56	0.01\\
51.57	0.01\\
51.58	0.01\\
51.59	0.01\\
51.6	0.01\\
51.61	0.01\\
51.62	0.01\\
51.63	0.01\\
51.64	0.01\\
51.65	0.01\\
51.66	0.01\\
51.67	0.01\\
51.68	0.01\\
51.69	0.01\\
51.7	0.01\\
51.71	0.01\\
51.72	0.01\\
51.73	0.01\\
51.74	0.01\\
51.75	0.01\\
51.76	0.01\\
51.77	0.01\\
51.78	0.01\\
51.79	0.01\\
51.8	0.01\\
51.81	0.01\\
51.82	0.01\\
51.83	0.01\\
51.84	0.01\\
51.85	0.01\\
51.86	0.01\\
51.87	0.01\\
51.88	0.01\\
51.89	0.01\\
51.9	0.01\\
51.91	0.01\\
51.92	0.01\\
51.93	0.01\\
51.94	0.01\\
51.95	0.01\\
51.96	0.01\\
51.97	0.01\\
51.98	0.01\\
51.99	0.01\\
52	0.01\\
52.01	0.01\\
52.02	0.01\\
52.03	0.01\\
52.04	0.01\\
52.05	0.01\\
52.06	0.01\\
52.07	0.01\\
52.08	0.01\\
52.09	0.01\\
52.1	0.01\\
52.11	0.01\\
52.12	0.01\\
52.13	0.01\\
52.14	0.01\\
52.15	0.01\\
52.16	0.01\\
52.17	0.01\\
52.18	0.01\\
52.19	0.01\\
52.2	0.01\\
52.21	0.01\\
52.22	0.01\\
52.23	0.01\\
52.24	0.01\\
52.25	0.01\\
52.26	0.01\\
52.27	0.01\\
52.28	0.01\\
52.29	0.01\\
52.3	0.01\\
52.31	0.01\\
52.32	0.01\\
52.33	0.01\\
52.34	0.01\\
52.35	0.01\\
52.36	0.01\\
52.37	0.01\\
52.38	0.01\\
52.39	0.01\\
52.4	0.01\\
52.41	0.01\\
52.42	0.01\\
52.43	0.01\\
52.44	0.01\\
52.45	0.01\\
52.46	0.01\\
52.47	0.01\\
52.48	0.01\\
52.49	0.01\\
52.5	0.01\\
52.51	0.01\\
52.52	0.01\\
52.53	0.01\\
52.54	0.01\\
52.55	0.01\\
52.56	0.01\\
52.57	0.01\\
52.58	0.01\\
52.59	0.01\\
52.6	0.01\\
52.61	0.01\\
52.62	0.01\\
52.63	0.01\\
52.64	0.01\\
52.65	0.01\\
52.66	0.01\\
52.67	0.01\\
52.68	0.01\\
52.69	0.01\\
52.7	0.01\\
52.71	0.01\\
52.72	0.01\\
52.73	0.01\\
52.74	0.01\\
52.75	0.01\\
52.76	0.01\\
52.77	0.01\\
52.78	0.01\\
52.79	0.01\\
52.8	0.01\\
52.81	0.01\\
52.82	0.01\\
52.83	0.01\\
52.84	0.01\\
52.85	0.01\\
52.86	0.01\\
52.87	0.01\\
52.88	0.01\\
52.89	0.01\\
52.9	0.01\\
52.91	0.01\\
52.92	0.01\\
52.93	0.01\\
52.94	0.01\\
52.95	0.01\\
52.96	0.01\\
52.97	0.01\\
52.98	0.01\\
52.99	0.01\\
53	0.01\\
53.01	0.01\\
53.02	0.01\\
53.03	0.01\\
53.04	0.01\\
53.05	0.01\\
53.06	0.01\\
53.07	0.01\\
53.08	0.01\\
53.09	0.01\\
53.1	0.01\\
53.11	0.01\\
53.12	0.01\\
53.13	0.01\\
53.14	0.01\\
53.15	0.01\\
53.16	0.01\\
53.17	0.01\\
53.18	0.01\\
53.19	0.01\\
53.2	0.01\\
53.21	0.01\\
53.22	0.01\\
53.23	0.01\\
53.24	0.01\\
53.25	0.01\\
53.26	0.01\\
53.27	0.01\\
53.28	0.01\\
53.29	0.01\\
53.3	0.01\\
53.31	0.01\\
53.32	0.01\\
53.33	0.01\\
53.34	0.01\\
53.35	0.01\\
53.36	0.01\\
53.37	0.01\\
53.38	0.01\\
53.39	0.01\\
53.4	0.01\\
53.41	0.01\\
53.42	0.01\\
53.43	0.01\\
53.44	0.01\\
53.45	0.01\\
53.46	0.01\\
53.47	0.01\\
53.48	0.01\\
53.49	0.01\\
53.5	0.01\\
53.51	0.01\\
53.52	0.01\\
53.53	0.01\\
53.54	0.01\\
53.55	0.01\\
53.56	0.01\\
53.57	0.01\\
53.58	0.01\\
53.59	0.01\\
53.6	0.01\\
53.61	0.01\\
53.62	0.01\\
53.63	0.01\\
53.64	0.01\\
53.65	0.01\\
53.66	0.01\\
53.67	0.01\\
53.68	0.01\\
53.69	0.01\\
53.7	0.01\\
53.71	0.01\\
53.72	0.01\\
53.73	0.01\\
53.74	0.01\\
53.75	0.01\\
53.76	0.01\\
53.77	0.01\\
53.78	0.01\\
53.79	0.01\\
53.8	0.01\\
53.81	0.01\\
53.82	0.01\\
53.83	0.01\\
53.84	0.01\\
53.85	0.01\\
53.86	0.01\\
53.87	0.01\\
53.88	0.01\\
53.89	0.01\\
53.9	0.01\\
53.91	0.01\\
53.92	0.01\\
53.93	0.01\\
53.94	0.01\\
53.95	0.01\\
53.96	0.01\\
53.97	0.01\\
53.98	0.01\\
53.99	0.01\\
54	0.01\\
54.01	0.01\\
54.02	0.01\\
54.03	0.01\\
54.04	0.01\\
54.05	0.01\\
54.06	0.01\\
54.07	0.01\\
54.08	0.01\\
54.09	0.01\\
54.1	0.01\\
54.11	0.01\\
54.12	0.01\\
54.13	0.01\\
54.14	0.01\\
54.15	0.01\\
54.16	0.01\\
54.17	0.01\\
54.18	0.01\\
54.19	0.01\\
54.2	0.01\\
54.21	0.01\\
54.22	0.01\\
54.23	0.01\\
54.24	0.01\\
54.25	0.01\\
54.26	0.01\\
54.27	0.01\\
54.28	0.01\\
54.29	0.01\\
54.3	0.01\\
54.31	0.01\\
54.32	0.01\\
54.33	0.01\\
54.34	0.01\\
54.35	0.01\\
54.36	0.01\\
54.37	0.01\\
54.38	0.01\\
54.39	0.01\\
54.4	0.01\\
54.41	0.01\\
54.42	0.01\\
54.43	0.01\\
54.44	0.01\\
54.45	0.01\\
54.46	0.01\\
54.47	0.01\\
54.48	0.01\\
54.49	0.01\\
54.5	0.01\\
54.51	0.01\\
54.52	0.01\\
54.53	0.01\\
54.54	0.01\\
54.55	0.01\\
54.56	0.01\\
54.57	0.01\\
54.58	0.01\\
54.59	0.01\\
54.6	0.01\\
54.61	0.01\\
54.62	0.01\\
54.63	0.01\\
54.64	0.01\\
54.65	0.01\\
54.66	0.01\\
54.67	0.01\\
54.68	0.01\\
54.69	0.01\\
54.7	0.01\\
54.71	0.01\\
54.72	0.01\\
54.73	0.01\\
54.74	0.01\\
54.75	0.01\\
54.76	0.01\\
54.77	0.01\\
54.78	0.01\\
54.79	0.01\\
54.8	0.01\\
54.81	0.01\\
54.82	0.01\\
54.83	0.01\\
54.84	0.01\\
54.85	0.01\\
54.86	0.01\\
54.87	0.01\\
54.88	0.01\\
54.89	0.01\\
54.9	0.01\\
54.91	0.01\\
54.92	0.01\\
54.93	0.01\\
54.94	0.01\\
54.95	0.01\\
54.96	0.01\\
54.97	0.01\\
54.98	0.01\\
54.99	0.01\\
55	0.01\\
55.01	0.01\\
55.02	0.01\\
55.03	0.01\\
55.04	0.01\\
55.05	0.01\\
55.06	0.01\\
55.07	0.01\\
55.08	0.01\\
55.09	0.01\\
55.1	0.01\\
55.11	0.01\\
55.12	0.01\\
55.13	0.01\\
55.14	0.01\\
55.15	0.01\\
55.16	0.01\\
55.17	0.01\\
55.18	0.01\\
55.19	0.01\\
55.2	0.01\\
55.21	0.01\\
55.22	0.01\\
55.23	0.01\\
55.24	0.01\\
55.25	0.01\\
55.26	0.01\\
55.27	0.01\\
55.28	0.01\\
55.29	0.01\\
55.3	0.01\\
55.31	0.01\\
55.32	0.01\\
55.33	0.01\\
55.34	0.01\\
55.35	0.01\\
55.36	0.01\\
55.37	0.01\\
55.38	0.01\\
55.39	0.01\\
55.4	0.01\\
55.41	0.01\\
55.42	0.01\\
55.43	0.01\\
55.44	0.01\\
55.45	0.01\\
55.46	0.01\\
55.47	0.01\\
55.48	0.01\\
55.49	0.01\\
55.5	0.01\\
55.51	0.01\\
55.52	0.01\\
55.53	0.01\\
55.54	0.01\\
55.55	0.01\\
55.56	0.01\\
55.57	0.01\\
55.58	0.01\\
55.59	0.01\\
55.6	0.01\\
55.61	0.01\\
55.62	0.01\\
55.63	0.01\\
55.64	0.01\\
55.65	0.01\\
55.66	0.01\\
55.67	0.01\\
55.68	0.01\\
55.69	0.01\\
55.7	0.01\\
55.71	0.01\\
55.72	0.01\\
55.73	0.01\\
55.74	0.01\\
55.75	0.01\\
55.76	0.01\\
55.77	0.01\\
55.78	0.01\\
55.79	0.01\\
55.8	0.01\\
55.81	0.01\\
55.82	0.01\\
55.83	0.01\\
55.84	0.01\\
55.85	0.01\\
55.86	0.01\\
55.87	0.01\\
55.88	0.01\\
55.89	0.01\\
55.9	0.01\\
55.91	0.01\\
55.92	0.01\\
55.93	0.01\\
55.94	0.01\\
55.95	0.01\\
55.96	0.01\\
55.97	0.01\\
55.98	0.01\\
55.99	0.01\\
56	0.01\\
56.01	0.01\\
56.02	0.01\\
56.03	0.01\\
56.04	0.01\\
56.05	0.01\\
56.06	0.01\\
56.07	0.01\\
56.08	0.01\\
56.09	0.01\\
56.1	0.01\\
56.11	0.01\\
56.12	0.01\\
56.13	0.01\\
56.14	0.01\\
56.15	0.01\\
56.16	0.01\\
56.17	0.01\\
56.18	0.01\\
56.19	0.01\\
56.2	0.01\\
56.21	0.01\\
56.22	0.01\\
56.23	0.01\\
56.24	0.01\\
56.25	0.01\\
56.26	0.01\\
56.27	0.01\\
56.28	0.01\\
56.29	0.01\\
56.3	0.01\\
56.31	0.01\\
56.32	0.01\\
56.33	0.01\\
56.34	0.01\\
56.35	0.01\\
56.36	0.01\\
56.37	0.01\\
56.38	0.01\\
56.39	0.01\\
56.4	0.01\\
56.41	0.01\\
56.42	0.01\\
56.43	0.01\\
56.44	0.01\\
56.45	0.01\\
56.46	0.01\\
56.47	0.01\\
56.48	0.01\\
56.49	0.01\\
56.5	0.01\\
56.51	0.01\\
56.52	0.01\\
56.53	0.01\\
56.54	0.01\\
56.55	0.01\\
56.56	0.01\\
56.57	0.01\\
56.58	0.01\\
56.59	0.01\\
56.6	0.01\\
56.61	0.01\\
56.62	0.01\\
56.63	0.01\\
56.64	0.01\\
56.65	0.01\\
56.66	0.01\\
56.67	0.01\\
56.68	0.01\\
56.69	0.01\\
56.7	0.01\\
56.71	0.01\\
56.72	0.01\\
56.73	0.01\\
56.74	0.01\\
56.75	0.01\\
56.76	0.01\\
56.77	0.01\\
56.78	0.01\\
56.79	0.01\\
56.8	0.01\\
56.81	0.01\\
56.82	0.01\\
56.83	0.01\\
56.84	0.01\\
56.85	0.01\\
56.86	0.01\\
56.87	0.01\\
56.88	0.01\\
56.89	0.01\\
56.9	0.01\\
56.91	0.01\\
56.92	0.01\\
56.93	0.01\\
56.94	0.01\\
56.95	0.01\\
56.96	0.01\\
56.97	0.01\\
56.98	0.01\\
56.99	0.01\\
57	0.01\\
57.01	0.01\\
57.02	0.01\\
57.03	0.01\\
57.04	0.01\\
57.05	0.01\\
57.06	0.01\\
57.07	0.01\\
57.08	0.01\\
57.09	0.01\\
57.1	0.01\\
57.11	0.01\\
57.12	0.01\\
57.13	0.01\\
57.14	0.01\\
57.15	0.01\\
57.16	0.01\\
57.17	0.01\\
57.18	0.01\\
57.19	0.01\\
57.2	0.01\\
57.21	0.01\\
57.22	0.01\\
57.23	0.01\\
57.24	0.01\\
57.25	0.01\\
57.26	0.01\\
57.27	0.01\\
57.28	0.01\\
57.29	0.01\\
57.3	0.01\\
57.31	0.01\\
57.32	0.01\\
57.33	0.01\\
57.34	0.01\\
57.35	0.01\\
57.36	0.01\\
57.37	0.01\\
57.38	0.01\\
57.39	0.01\\
57.4	0.01\\
57.41	0.01\\
57.42	0.01\\
57.43	0.01\\
57.44	0.01\\
57.45	0.01\\
57.46	0.01\\
57.47	0.01\\
57.48	0.01\\
57.49	0.01\\
57.5	0.01\\
57.51	0.01\\
57.52	0.01\\
57.53	0.01\\
57.54	0.01\\
57.55	0.01\\
57.56	0.01\\
57.57	0.01\\
57.58	0.01\\
57.59	0.01\\
57.6	0.01\\
57.61	0.01\\
57.62	0.01\\
57.63	0.01\\
57.64	0.01\\
57.65	0.01\\
57.66	0.01\\
57.67	0.01\\
57.68	0.01\\
57.69	0.01\\
57.7	0.01\\
57.71	0.01\\
57.72	0.01\\
57.73	0.01\\
57.74	0.01\\
57.75	0.01\\
57.76	0.01\\
57.77	0.01\\
57.78	0.01\\
57.79	0.01\\
57.8	0.01\\
57.81	0.01\\
57.82	0.01\\
57.83	0.01\\
57.84	0.01\\
57.85	0.01\\
57.86	0.01\\
57.87	0.01\\
57.88	0.01\\
57.89	0.01\\
57.9	0.01\\
57.91	0.01\\
57.92	0.01\\
57.93	0.01\\
57.94	0.01\\
57.95	0.01\\
57.96	0.01\\
57.97	0.01\\
57.98	0.01\\
57.99	0.01\\
58	0.01\\
58.01	0.01\\
58.02	0.01\\
58.03	0.01\\
58.04	0.01\\
58.05	0.01\\
58.06	0.01\\
58.07	0.01\\
58.08	0.01\\
58.09	0.01\\
58.1	0.01\\
58.11	0.01\\
58.12	0.01\\
58.13	0.01\\
58.14	0.01\\
58.15	0.01\\
58.16	0.01\\
58.17	0.01\\
58.18	0.01\\
58.19	0.01\\
58.2	0.01\\
58.21	0.01\\
58.22	0.01\\
58.23	0.01\\
58.24	0.01\\
58.25	0.01\\
58.26	0.01\\
58.27	0.01\\
58.28	0.01\\
58.29	0.01\\
58.3	0.01\\
58.31	0.01\\
58.32	0.01\\
58.33	0.01\\
58.34	0.01\\
58.35	0.01\\
58.36	0.01\\
58.37	0.01\\
58.38	0.01\\
58.39	0.01\\
58.4	0.01\\
58.41	0.01\\
58.42	0.01\\
58.43	0.01\\
58.44	0.01\\
58.45	0.01\\
58.46	0.01\\
58.47	0.01\\
58.48	0.01\\
58.49	0.01\\
58.5	0.01\\
58.51	0.01\\
58.52	0.01\\
58.53	0.01\\
58.54	0.01\\
58.55	0.01\\
58.56	0.01\\
58.57	0.01\\
58.58	0.01\\
58.59	0.01\\
58.6	0.01\\
58.61	0.01\\
58.62	0.01\\
58.63	0.01\\
58.64	0.01\\
58.65	0.01\\
58.66	0.01\\
58.67	0.01\\
58.68	0.01\\
58.69	0.01\\
58.7	0.01\\
58.71	0.01\\
58.72	0.01\\
58.73	0.01\\
58.74	0.01\\
58.75	0.01\\
58.76	0.01\\
58.77	0.01\\
58.78	0.01\\
58.79	0.01\\
58.8	0.01\\
58.81	0.01\\
58.82	0.01\\
58.83	0.01\\
58.84	0.01\\
58.85	0.01\\
58.86	0.01\\
58.87	0.01\\
58.88	0.01\\
58.89	0.01\\
58.9	0.01\\
58.91	0.01\\
58.92	0.01\\
58.93	0.01\\
58.94	0.01\\
58.95	0.01\\
58.96	0.01\\
58.97	0.01\\
58.98	0.01\\
58.99	0.01\\
59	0.01\\
59.01	0.01\\
59.02	0.01\\
59.03	0.01\\
59.04	0.01\\
59.05	0.01\\
59.06	0.01\\
59.07	0.01\\
59.08	0.01\\
59.09	0.01\\
59.1	0.01\\
59.11	0.01\\
59.12	0.01\\
59.13	0.01\\
59.14	0.01\\
59.15	0.01\\
59.16	0.01\\
59.17	0.01\\
59.18	0.01\\
59.19	0.01\\
59.2	0.01\\
59.21	0.01\\
59.22	0.01\\
59.23	0.01\\
59.24	0.01\\
59.25	0.01\\
59.26	0.01\\
59.27	0.01\\
59.28	0.01\\
59.29	0.01\\
59.3	0.01\\
59.31	0.01\\
59.32	0.01\\
59.33	0.01\\
59.34	0.01\\
59.35	0.01\\
59.36	0.01\\
59.37	0.01\\
59.38	0.01\\
59.39	0.01\\
59.4	0.01\\
59.41	0.01\\
59.42	0.01\\
59.43	0.01\\
59.44	0.01\\
59.45	0.01\\
59.46	0.01\\
59.47	0.01\\
59.48	0.01\\
59.49	0.01\\
59.5	0.01\\
59.51	0.01\\
59.52	0.01\\
59.53	0.01\\
59.54	0.01\\
59.55	0.01\\
59.56	0.01\\
59.57	0.01\\
59.58	0.01\\
59.59	0.01\\
59.6	0.01\\
59.61	0.01\\
59.62	0.01\\
59.63	0.01\\
59.64	0.01\\
59.65	0.01\\
59.66	0.01\\
59.67	0.01\\
59.68	0.01\\
59.69	0.01\\
59.7	0.01\\
59.71	0.01\\
59.72	0.01\\
59.73	0.01\\
59.74	0.01\\
59.75	0.01\\
59.76	0.01\\
59.77	0.01\\
59.78	0.01\\
59.79	0.01\\
59.8	0.01\\
59.81	0.01\\
59.82	0.01\\
59.83	0.01\\
59.84	0.01\\
59.85	0.01\\
59.86	0.01\\
59.87	0.01\\
59.88	0.01\\
59.89	0.01\\
59.9	0.01\\
59.91	0.01\\
59.92	0.01\\
59.93	0.01\\
59.94	0.01\\
59.95	0.01\\
59.96	0.01\\
59.97	0.01\\
59.98	0.01\\
59.99	0.01\\
60	0.01\\
60.01	0.01\\
60.02	0.01\\
60.03	0.01\\
60.04	0.01\\
60.05	0.01\\
60.06	0.01\\
60.07	0.01\\
60.08	0.01\\
60.09	0.01\\
60.1	0.01\\
60.11	0.01\\
60.12	0.01\\
60.13	0.01\\
60.14	0.01\\
60.15	0.01\\
60.16	0.01\\
60.17	0.01\\
60.18	0.01\\
60.19	0.01\\
60.2	0.01\\
60.21	0.01\\
60.22	0.01\\
60.23	0.01\\
60.24	0.01\\
60.25	0.01\\
60.26	0.01\\
60.27	0.01\\
60.28	0.01\\
60.29	0.01\\
60.3	0.01\\
60.31	0.01\\
60.32	0.01\\
60.33	0.01\\
60.34	0.01\\
60.35	0.01\\
60.36	0.01\\
60.37	0.01\\
60.38	0.01\\
60.39	0.01\\
60.4	0.01\\
60.41	0.01\\
60.42	0.01\\
60.43	0.01\\
60.44	0.01\\
60.45	0.01\\
60.46	0.01\\
60.47	0.01\\
60.48	0.01\\
60.49	0.01\\
60.5	0.01\\
60.51	0.01\\
60.52	0.01\\
60.53	0.01\\
60.54	0.01\\
60.55	0.01\\
60.56	0.01\\
60.57	0.01\\
60.58	0.01\\
60.59	0.01\\
60.6	0.01\\
60.61	0.01\\
60.62	0.01\\
60.63	0.01\\
60.64	0.01\\
60.65	0.01\\
60.66	0.01\\
60.67	0.01\\
60.68	0.01\\
60.69	0.01\\
60.7	0.01\\
60.71	0.01\\
60.72	0.01\\
60.73	0.01\\
60.74	0.01\\
60.75	0.01\\
60.76	0.01\\
60.77	0.01\\
60.78	0.01\\
60.79	0.01\\
60.8	0.01\\
60.81	0.01\\
60.82	0.01\\
60.83	0.01\\
60.84	0.01\\
60.85	0.01\\
60.86	0.01\\
60.87	0.01\\
60.88	0.01\\
60.89	0.01\\
60.9	0.01\\
60.91	0.01\\
60.92	0.01\\
60.93	0.01\\
60.94	0.01\\
60.95	0.01\\
60.96	0.01\\
60.97	0.01\\
60.98	0.01\\
60.99	0.01\\
61	0.01\\
61.01	0.01\\
61.02	0.01\\
61.03	0.01\\
61.04	0.01\\
61.05	0.01\\
61.06	0.01\\
61.07	0.01\\
61.08	0.01\\
61.09	0.01\\
61.1	0.01\\
61.11	0.01\\
61.12	0.01\\
61.13	0.01\\
61.14	0.01\\
61.15	0.01\\
61.16	0.01\\
61.17	0.01\\
61.18	0.01\\
61.19	0.01\\
61.2	0.01\\
61.21	0.01\\
61.22	0.01\\
61.23	0.01\\
61.24	0.01\\
61.25	0.01\\
61.26	0.01\\
61.27	0.01\\
61.28	0.01\\
61.29	0.01\\
61.3	0.01\\
61.31	0.01\\
61.32	0.01\\
61.33	0.01\\
61.34	0.01\\
61.35	0.01\\
61.36	0.01\\
61.37	0.01\\
61.38	0.01\\
61.39	0.01\\
61.4	0.01\\
61.41	0.01\\
61.42	0.01\\
61.43	0.01\\
61.44	0.01\\
61.45	0.01\\
61.46	0.01\\
61.47	0.01\\
61.48	0.01\\
61.49	0.01\\
61.5	0.01\\
61.51	0.01\\
61.52	0.01\\
61.53	0.01\\
61.54	0.01\\
61.55	0.01\\
61.56	0.01\\
61.57	0.01\\
61.58	0.01\\
61.59	0.01\\
61.6	0.01\\
61.61	0.01\\
61.62	0.01\\
61.63	0.01\\
61.64	0.01\\
61.65	0.01\\
61.66	0.01\\
61.67	0.01\\
61.68	0.01\\
61.69	0.01\\
61.7	0.01\\
61.71	0.01\\
61.72	0.01\\
61.73	0.01\\
61.74	0.01\\
61.75	0.01\\
61.76	0.01\\
61.77	0.01\\
61.78	0.01\\
61.79	0.01\\
61.8	0.01\\
61.81	0.01\\
61.82	0.01\\
61.83	0.01\\
61.84	0.01\\
61.85	0.01\\
61.86	0.01\\
61.87	0.01\\
61.88	0.01\\
61.89	0.01\\
61.9	0.01\\
61.91	0.01\\
61.92	0.01\\
61.93	0.01\\
61.94	0.01\\
61.95	0.01\\
61.96	0.01\\
61.97	0.01\\
61.98	0.01\\
61.99	0.01\\
62	0.01\\
62.01	0.01\\
62.02	0.01\\
62.03	0.01\\
62.04	0.01\\
62.05	0.01\\
62.06	0.01\\
62.07	0.01\\
62.08	0.01\\
62.09	0.01\\
62.1	0.01\\
62.11	0.01\\
62.12	0.01\\
62.13	0.01\\
62.14	0.01\\
62.15	0.01\\
62.16	0.01\\
62.17	0.01\\
62.18	0.01\\
62.19	0.01\\
62.2	0.01\\
62.21	0.01\\
62.22	0.01\\
62.23	0.01\\
62.24	0.01\\
62.25	0.01\\
62.26	0.01\\
62.27	0.01\\
62.28	0.01\\
62.29	0.01\\
62.3	0.01\\
62.31	0.01\\
62.32	0.01\\
62.33	0.01\\
62.34	0.01\\
62.35	0.01\\
62.36	0.01\\
62.37	0.01\\
62.38	0.01\\
62.39	0.01\\
62.4	0.01\\
62.41	0.01\\
62.42	0.01\\
62.43	0.01\\
62.44	0.01\\
62.45	0.01\\
62.46	0.01\\
62.47	0.01\\
62.48	0.01\\
62.49	0.01\\
62.5	0.01\\
62.51	0.01\\
62.52	0.01\\
62.53	0.01\\
62.54	0.01\\
62.55	0.01\\
62.56	0.01\\
62.57	0.01\\
62.58	0.01\\
62.59	0.01\\
62.6	0.01\\
62.61	0.01\\
62.62	0.01\\
62.63	0.01\\
62.64	0.01\\
62.65	0.01\\
62.66	0.01\\
62.67	0.01\\
62.68	0.01\\
62.69	0.01\\
62.7	0.01\\
62.71	0.01\\
62.72	0.01\\
62.73	0.01\\
62.74	0.01\\
62.75	0.01\\
62.76	0.01\\
62.77	0.01\\
62.78	0.01\\
62.79	0.01\\
62.8	0.01\\
62.81	0.01\\
62.82	0.01\\
62.83	0.01\\
62.84	0.01\\
62.85	0.01\\
62.86	0.01\\
62.87	0.01\\
62.88	0.01\\
62.89	0.01\\
62.9	0.01\\
62.91	0.01\\
62.92	0.01\\
62.93	0.01\\
62.94	0.01\\
62.95	0.01\\
62.96	0.01\\
62.97	0.01\\
62.98	0.01\\
62.99	0.01\\
63	0.01\\
63.01	0.01\\
63.02	0.01\\
63.03	0.01\\
63.04	0.01\\
63.05	0.01\\
63.06	0.01\\
63.07	0.01\\
63.08	0.01\\
63.09	0.01\\
63.1	0.01\\
63.11	0.01\\
63.12	0.01\\
63.13	0.01\\
63.14	0.01\\
63.15	0.01\\
63.16	0.01\\
63.17	0.01\\
63.18	0.01\\
63.19	0.01\\
63.2	0.01\\
63.21	0.01\\
63.22	0.01\\
63.23	0.01\\
63.24	0.01\\
63.25	0.01\\
63.26	0.01\\
63.27	0.01\\
63.28	0.01\\
63.29	0.01\\
63.3	0.01\\
63.31	0.01\\
63.32	0.01\\
63.33	0.01\\
63.34	0.01\\
63.35	0.01\\
63.36	0.01\\
63.37	0.01\\
63.38	0.01\\
63.39	0.01\\
63.4	0.01\\
63.41	0.01\\
63.42	0.01\\
63.43	0.01\\
63.44	0.01\\
63.45	0.01\\
63.46	0.01\\
63.47	0.01\\
63.48	0.01\\
63.49	0.01\\
63.5	0.01\\
63.51	0.01\\
63.52	0.01\\
63.53	0.01\\
63.54	0.01\\
63.55	0.01\\
63.56	0.01\\
63.57	0.01\\
63.58	0.01\\
63.59	0.01\\
63.6	0.01\\
63.61	0.01\\
63.62	0.01\\
63.63	0.01\\
63.64	0.01\\
63.65	0.01\\
63.66	0.01\\
63.67	0.01\\
63.68	0.01\\
63.69	0.01\\
63.7	0.01\\
63.71	0.01\\
63.72	0.01\\
63.73	0.01\\
63.74	0.01\\
63.75	0.01\\
63.76	0.01\\
63.77	0.01\\
63.78	0.01\\
63.79	0.01\\
63.8	0.01\\
63.81	0.01\\
63.82	0.01\\
63.83	0.01\\
63.84	0.01\\
63.85	0.01\\
63.86	0.01\\
63.87	0.01\\
63.88	0.01\\
63.89	0.01\\
63.9	0.01\\
63.91	0.01\\
63.92	0.01\\
63.93	0.01\\
63.94	0.01\\
63.95	0.01\\
63.96	0.01\\
63.97	0.01\\
63.98	0.01\\
63.99	0.01\\
64	0.01\\
64.01	0.01\\
64.02	0.01\\
64.03	0.01\\
64.04	0.01\\
64.05	0.01\\
64.06	0.01\\
64.07	0.01\\
64.08	0.01\\
64.09	0.01\\
64.1	0.01\\
64.11	0.01\\
64.12	0.01\\
64.13	0.01\\
64.14	0.01\\
64.15	0.01\\
64.16	0.01\\
64.17	0.01\\
64.18	0.01\\
64.19	0.01\\
64.2	0.01\\
64.21	0.01\\
64.22	0.01\\
64.23	0.01\\
64.24	0.01\\
64.25	0.01\\
64.26	0.01\\
64.27	0.01\\
64.28	0.01\\
64.29	0.01\\
64.3	0.01\\
64.31	0.01\\
64.32	0.01\\
64.33	0.01\\
64.34	0.01\\
64.35	0.01\\
64.36	0.01\\
64.37	0.01\\
64.38	0.01\\
64.39	0.01\\
64.4	0.01\\
64.41	0.01\\
64.42	0.01\\
64.43	0.01\\
64.44	0.01\\
64.45	0.01\\
64.46	0.01\\
64.47	0.01\\
64.48	0.01\\
64.49	0.01\\
64.5	0.01\\
64.51	0.01\\
64.52	0.01\\
64.53	0.01\\
64.54	0.01\\
64.55	0.01\\
64.56	0.01\\
64.57	0.01\\
64.58	0.01\\
64.59	0.01\\
64.6	0.01\\
64.61	0.01\\
64.62	0.01\\
64.63	0.01\\
64.64	0.01\\
64.65	0.01\\
64.66	0.01\\
64.67	0.01\\
64.68	0.01\\
64.69	0.01\\
64.7	0.01\\
64.71	0.01\\
64.72	0.01\\
64.73	0.01\\
64.74	0.01\\
64.75	0.01\\
64.76	0.01\\
64.77	0.01\\
64.78	0.01\\
64.79	0.01\\
64.8	0.01\\
64.81	0.01\\
64.82	0.01\\
64.83	0.01\\
64.84	0.01\\
64.85	0.01\\
64.86	0.01\\
64.87	0.01\\
64.88	0.01\\
64.89	0.01\\
64.9	0.01\\
64.91	0.01\\
64.92	0.01\\
64.93	0.01\\
64.94	0.01\\
64.95	0.01\\
64.96	0.01\\
64.97	0.01\\
64.98	0.01\\
64.99	0.01\\
65	0.01\\
65.01	0.01\\
65.02	0.01\\
65.03	0.01\\
65.04	0.01\\
65.05	0.01\\
65.06	0.01\\
65.07	0.01\\
65.08	0.01\\
65.09	0.01\\
65.1	0.01\\
65.11	0.01\\
65.12	0.01\\
65.13	0.01\\
65.14	0.01\\
65.15	0.01\\
65.16	0.01\\
65.17	0.01\\
65.18	0.01\\
65.19	0.01\\
65.2	0.01\\
65.21	0.01\\
65.22	0.01\\
65.23	0.01\\
65.24	0.01\\
65.25	0.01\\
65.26	0.01\\
65.27	0.01\\
65.28	0.01\\
65.29	0.01\\
65.3	0.01\\
65.31	0.01\\
65.32	0.01\\
65.33	0.01\\
65.34	0.01\\
65.35	0.01\\
65.36	0.01\\
65.37	0.01\\
65.38	0.01\\
65.39	0.01\\
65.4	0.01\\
65.41	0.01\\
65.42	0.01\\
65.43	0.01\\
65.44	0.01\\
65.45	0.01\\
65.46	0.01\\
65.47	0.01\\
65.48	0.01\\
65.49	0.01\\
65.5	0.01\\
65.51	0.01\\
65.52	0.01\\
65.53	0.01\\
65.54	0.01\\
65.55	0.01\\
65.56	0.01\\
65.57	0.01\\
65.58	0.01\\
65.59	0.01\\
65.6	0.01\\
65.61	0.01\\
65.62	0.01\\
65.63	0.01\\
65.64	0.01\\
65.65	0.01\\
65.66	0.01\\
65.67	0.01\\
65.68	0.01\\
65.69	0.01\\
65.7	0.01\\
65.71	0.01\\
65.72	0.01\\
65.73	0.01\\
65.74	0.01\\
65.75	0.01\\
65.76	0.01\\
65.77	0.01\\
65.78	0.01\\
65.79	0.01\\
65.8	0.01\\
65.81	0.01\\
65.82	0.01\\
65.83	0.01\\
65.84	0.01\\
65.85	0.01\\
65.86	0.01\\
65.87	0.01\\
65.88	0.01\\
65.89	0.01\\
65.9	0.01\\
65.91	0.01\\
65.92	0.01\\
65.93	0.01\\
65.94	0.01\\
65.95	0.01\\
65.96	0.01\\
65.97	0.01\\
65.98	0.01\\
65.99	0.01\\
66	0.01\\
66.01	0.01\\
66.02	0.01\\
66.03	0.01\\
66.04	0.01\\
66.05	0.01\\
66.06	0.01\\
66.07	0.01\\
66.08	0.01\\
66.09	0.01\\
66.1	0.01\\
66.11	0.01\\
66.12	0.01\\
66.13	0.01\\
66.14	0.01\\
66.15	0.01\\
66.16	0.01\\
66.17	0.01\\
66.18	0.01\\
66.19	0.01\\
66.2	0.01\\
66.21	0.01\\
66.22	0.01\\
66.23	0.01\\
66.24	0.01\\
66.25	0.01\\
66.26	0.01\\
66.27	0.01\\
66.28	0.01\\
66.29	0.01\\
66.3	0.01\\
66.31	0.01\\
66.32	0.01\\
66.33	0.01\\
66.34	0.01\\
66.35	0.01\\
66.36	0.01\\
66.37	0.01\\
66.38	0.01\\
66.39	0.01\\
66.4	0.01\\
66.41	0.01\\
66.42	0.01\\
66.43	0.01\\
66.44	0.01\\
66.45	0.01\\
66.46	0.01\\
66.47	0.01\\
66.48	0.01\\
66.49	0.01\\
66.5	0.01\\
66.51	0.01\\
66.52	0.01\\
66.53	0.01\\
66.54	0.01\\
66.55	0.01\\
66.56	0.01\\
66.57	0.01\\
66.58	0.01\\
66.59	0.01\\
66.6	0.01\\
66.61	0.01\\
66.62	0.01\\
66.63	0.01\\
66.64	0.01\\
66.65	0.01\\
66.66	0.01\\
66.67	0.01\\
66.68	0.01\\
66.69	0.01\\
66.7	0.01\\
66.71	0.01\\
66.72	0.01\\
66.73	0.01\\
66.74	0.01\\
66.75	0.01\\
66.76	0.01\\
66.77	0.01\\
66.78	0.01\\
66.79	0.01\\
66.8	0.01\\
66.81	0.01\\
66.82	0.01\\
66.83	0.01\\
66.84	0.01\\
66.85	0.01\\
66.86	0.01\\
66.87	0.01\\
66.88	0.01\\
66.89	0.01\\
66.9	0.01\\
66.91	0.01\\
66.92	0.01\\
66.93	0.01\\
66.94	0.01\\
66.95	0.01\\
66.96	0.01\\
66.97	0.01\\
66.98	0.01\\
66.99	0.01\\
67	0.01\\
67.01	0.01\\
67.02	0.01\\
67.03	0.01\\
67.04	0.01\\
67.05	0.01\\
67.06	0.01\\
67.07	0.01\\
67.08	0.01\\
67.09	0.01\\
67.1	0.01\\
67.11	0.01\\
67.12	0.01\\
67.13	0.01\\
67.14	0.01\\
67.15	0.01\\
67.16	0.01\\
67.17	0.01\\
67.18	0.01\\
67.19	0.01\\
67.2	0.01\\
67.21	0.01\\
67.22	0.01\\
67.23	0.01\\
67.24	0.01\\
67.25	0.01\\
67.26	0.01\\
67.27	0.01\\
67.28	0.01\\
67.29	0.01\\
67.3	0.01\\
67.31	0.01\\
67.32	0.01\\
67.33	0.01\\
67.34	0.01\\
67.35	0.01\\
67.36	0.01\\
67.37	0.01\\
67.38	0.01\\
67.39	0.01\\
67.4	0.01\\
67.41	0.01\\
67.42	0.01\\
67.43	0.01\\
67.44	0.01\\
67.45	0.01\\
67.46	0.01\\
67.47	0.01\\
67.48	0.01\\
67.49	0.01\\
67.5	0.01\\
67.51	0.01\\
67.52	0.01\\
67.53	0.01\\
67.54	0.01\\
67.55	0.01\\
67.56	0.01\\
67.57	0.01\\
67.58	0.01\\
67.59	0.01\\
67.6	0.01\\
67.61	0.01\\
67.62	0.01\\
67.63	0.01\\
67.64	0.01\\
67.65	0.01\\
67.66	0.01\\
67.67	0.01\\
67.68	0.01\\
67.69	0.01\\
67.7	0.01\\
67.71	0.01\\
67.72	0.01\\
67.73	0.01\\
67.74	0.01\\
67.75	0.01\\
67.76	0.01\\
67.77	0.01\\
67.78	0.01\\
67.79	0.01\\
67.8	0.01\\
67.81	0.01\\
67.82	0.01\\
67.83	0.01\\
67.84	0.01\\
67.85	0.01\\
67.86	0.01\\
67.87	0.01\\
67.88	0.01\\
67.89	0.01\\
67.9	0.01\\
67.91	0.01\\
67.92	0.01\\
67.93	0.01\\
67.94	0.01\\
67.95	0.01\\
67.96	0.01\\
67.97	0.01\\
67.98	0.01\\
67.99	0.01\\
68	0.01\\
68.01	0.01\\
68.02	0.01\\
68.03	0.01\\
68.04	0.01\\
68.05	0.01\\
68.06	0.01\\
68.07	0.01\\
68.08	0.01\\
68.09	0.01\\
68.1	0.01\\
68.11	0.01\\
68.12	0.01\\
68.13	0.01\\
68.14	0.01\\
68.15	0.01\\
68.16	0.01\\
68.17	0.01\\
68.18	0.01\\
68.19	0.01\\
68.2	0.01\\
68.21	0.01\\
68.22	0.01\\
68.23	0.01\\
68.24	0.01\\
68.25	0.01\\
68.26	0.01\\
68.27	0.01\\
68.28	0.01\\
68.29	0.01\\
68.3	0.01\\
68.31	0.01\\
68.32	0.01\\
68.33	0.01\\
68.34	0.01\\
68.35	0.01\\
68.36	0.01\\
68.37	0.01\\
68.38	0.01\\
68.39	0.01\\
68.4	0.01\\
68.41	0.01\\
68.42	0.01\\
68.43	0.01\\
68.44	0.01\\
68.45	0.01\\
68.46	0.01\\
68.47	0.01\\
68.48	0.01\\
68.49	0.01\\
68.5	0.01\\
68.51	0.01\\
68.52	0.01\\
68.53	0.01\\
68.54	0.01\\
68.55	0.01\\
68.56	0.01\\
68.57	0.01\\
68.58	0.01\\
68.59	0.01\\
68.6	0.01\\
68.61	0.01\\
68.62	0.01\\
68.63	0.01\\
68.64	0.01\\
68.65	0.01\\
68.66	0.01\\
68.67	0.01\\
68.68	0.01\\
68.69	0.01\\
68.7	0.01\\
68.71	0.01\\
68.72	0.01\\
68.73	0.01\\
68.74	0.01\\
68.75	0.01\\
68.76	0.01\\
68.77	0.01\\
68.78	0.01\\
68.79	0.01\\
68.8	0.01\\
68.81	0.01\\
68.82	0.01\\
68.83	0.01\\
68.84	0.01\\
68.85	0.01\\
68.86	0.01\\
68.87	0.01\\
68.88	0.01\\
68.89	0.01\\
68.9	0.01\\
68.91	0.01\\
68.92	0.01\\
68.93	0.01\\
68.94	0.01\\
68.95	0.01\\
68.96	0.01\\
68.97	0.01\\
68.98	0.01\\
68.99	0.01\\
69	0.01\\
69.01	0.01\\
69.02	0.01\\
69.03	0.01\\
69.04	0.01\\
69.05	0.01\\
69.06	0.01\\
69.07	0.01\\
69.08	0.01\\
69.09	0.01\\
69.1	0.01\\
69.11	0.01\\
69.12	0.01\\
69.13	0.01\\
69.14	0.01\\
69.15	0.01\\
69.16	0.01\\
69.17	0.01\\
69.18	0.01\\
69.19	0.01\\
69.2	0.01\\
69.21	0.01\\
69.22	0.01\\
69.23	0.01\\
69.24	0.01\\
69.25	0.01\\
69.26	0.01\\
69.27	0.01\\
69.28	0.01\\
69.29	0.01\\
69.3	0.01\\
69.31	0.01\\
69.32	0.01\\
69.33	0.01\\
69.34	0.01\\
69.35	0.01\\
69.36	0.01\\
69.37	0.01\\
69.38	0.01\\
69.39	0.01\\
69.4	0.01\\
69.41	0.01\\
69.42	0.01\\
69.43	0.01\\
69.44	0.01\\
69.45	0.01\\
69.46	0.01\\
69.47	0.01\\
69.48	0.01\\
69.49	0.01\\
69.5	0.01\\
69.51	0.01\\
69.52	0.01\\
69.53	0.01\\
69.54	0.01\\
69.55	0.01\\
69.56	0.01\\
69.57	0.01\\
69.58	0.01\\
69.59	0.01\\
69.6	0.01\\
69.61	0.01\\
69.62	0.01\\
69.63	0.01\\
69.64	0.01\\
69.65	0.01\\
69.66	0.01\\
69.67	0.01\\
69.68	0.01\\
69.69	0.01\\
69.7	0.01\\
69.71	0.01\\
69.72	0.01\\
69.73	0.01\\
69.74	0.01\\
69.75	0.01\\
69.76	0.01\\
69.77	0.01\\
69.78	0.01\\
69.79	0.01\\
69.8	0.01\\
69.81	0.01\\
69.82	0.01\\
69.83	0.01\\
69.84	0.01\\
69.85	0.01\\
69.86	0.01\\
69.87	0.01\\
69.88	0.01\\
69.89	0.01\\
69.9	0.01\\
69.91	0.01\\
69.92	0.01\\
69.93	0.01\\
69.94	0.01\\
69.95	0.01\\
69.96	0.01\\
69.97	0.01\\
69.98	0.01\\
69.99	0.01\\
70	0.01\\
70.01	0.01\\
70.02	0.01\\
70.03	0.01\\
70.04	0.01\\
70.05	0.01\\
70.06	0.01\\
70.07	0.01\\
70.08	0.01\\
70.09	0.01\\
70.1	0.01\\
70.11	0.01\\
70.12	0.01\\
70.13	0.01\\
70.14	0.01\\
70.15	0.01\\
70.16	0.01\\
70.17	0.01\\
70.18	0.01\\
70.19	0.01\\
70.2	0.01\\
70.21	0.01\\
70.22	0.01\\
70.23	0.01\\
70.24	0.01\\
70.25	0.01\\
70.26	0.01\\
70.27	0.01\\
70.28	0.01\\
70.29	0.01\\
70.3	0.01\\
70.31	0.01\\
70.32	0.01\\
70.33	0.01\\
70.34	0.01\\
70.35	0.01\\
70.36	0.01\\
70.37	0.01\\
70.38	0.01\\
70.39	0.01\\
70.4	0.01\\
70.41	0.01\\
70.42	0.01\\
70.43	0.01\\
70.44	0.01\\
70.45	0.01\\
70.46	0.01\\
70.47	0.01\\
70.48	0.01\\
70.49	0.01\\
70.5	0.01\\
70.51	0.01\\
70.52	0.01\\
70.53	0.01\\
70.54	0.01\\
70.55	0.01\\
70.56	0.01\\
70.57	0.01\\
70.58	0.01\\
70.59	0.01\\
70.6	0.01\\
70.61	0.01\\
70.62	0.01\\
70.63	0.01\\
70.64	0.01\\
70.65	0.01\\
70.66	0.01\\
70.67	0.01\\
70.68	0.01\\
70.69	0.01\\
70.7	0.01\\
70.71	0.01\\
70.72	0.01\\
70.73	0.01\\
70.74	0.01\\
70.75	0.01\\
70.76	0.01\\
70.77	0.01\\
70.78	0.01\\
70.79	0.01\\
70.8	0.01\\
70.81	0.01\\
70.82	0.01\\
70.83	0.01\\
70.84	0.01\\
70.85	0.01\\
70.86	0.01\\
70.87	0.01\\
70.88	0.01\\
70.89	0.01\\
70.9	0.01\\
70.91	0.01\\
70.92	0.01\\
70.93	0.01\\
70.94	0.01\\
70.95	0.01\\
70.96	0.01\\
70.97	0.01\\
70.98	0.01\\
70.99	0.01\\
71	0.01\\
71.01	0.01\\
71.02	0.01\\
71.03	0.01\\
71.04	0.01\\
71.05	0.01\\
71.06	0.01\\
71.07	0.01\\
71.08	0.01\\
71.09	0.01\\
71.1	0.01\\
71.11	0.01\\
71.12	0.01\\
71.13	0.01\\
71.14	0.01\\
71.15	0.01\\
71.16	0.01\\
71.17	0.01\\
71.18	0.01\\
71.19	0.01\\
71.2	0.01\\
71.21	0.01\\
71.22	0.01\\
71.23	0.01\\
71.24	0.01\\
71.25	0.01\\
71.26	0.01\\
71.27	0.01\\
71.28	0.01\\
71.29	0.01\\
71.3	0.01\\
71.31	0.01\\
71.32	0.01\\
71.33	0.01\\
71.34	0.01\\
71.35	0.01\\
71.36	0.01\\
71.37	0.01\\
71.38	0.01\\
71.39	0.01\\
71.4	0.01\\
71.41	0.01\\
71.42	0.01\\
71.43	0.01\\
71.44	0.01\\
71.45	0.01\\
71.46	0.01\\
71.47	0.01\\
71.48	0.01\\
71.49	0.01\\
71.5	0.01\\
71.51	0.01\\
71.52	0.01\\
71.53	0.01\\
71.54	0.01\\
71.55	0.01\\
71.56	0.01\\
71.57	0.01\\
71.58	0.01\\
71.59	0.01\\
71.6	0.01\\
71.61	0.01\\
71.62	0.01\\
71.63	0.00999992261345002\\
71.64	0.00999734872975903\\
71.65	0.00999477269224675\\
71.66	0.00999219449737882\\
71.67	0.00998961414161215\\
71.68	0.00998703162139485\\
71.69	0.00998444693316621\\
71.7	0.00998186007335671\\
71.71	0.0099792710383879\\
71.72	0.00997667982467244\\
71.73	0.00997408642861399\\
71.74	0.00997149084660726\\
71.75	0.00996889307503789\\
71.76	0.00996629311028248\\
71.77	0.0099636909487085\\
71.78	0.00996108658667428\\
71.79	0.00995848002052897\\
71.8	0.0099558712466125\\
71.81	0.00995326026125554\\
71.82	0.00995064706077944\\
71.83	0.00994803164149624\\
71.84	0.00994541399970859\\
71.85	0.00994279413170971\\
71.86	0.00994017203378336\\
71.87	0.00993754770220384\\
71.88	0.00993492113323585\\
71.89	0.00993229232313455\\
71.9	0.00992966126814547\\
71.91	0.00992702796450446\\
71.92	0.00992439240843766\\
71.93	0.00992175459616149\\
71.94	0.00991911452388256\\
71.95	0.00991647218779762\\
71.96	0.00991382758409356\\
71.97	0.00991118070894736\\
71.98	0.00990853155852601\\
71.99	0.0099058801289865\\
72	0.00990322641647573\\
72.01	0.00990057041713054\\
72.02	0.00989791212707759\\
72.03	0.00989525154243336\\
72.04	0.00989258865930407\\
72.05	0.00988992347378565\\
72.06	0.00988725598196371\\
72.07	0.00988458617991347\\
72.08	0.00988191406369969\\
72.09	0.00987923962937668\\
72.1	0.0098765628729882\\
72.11	0.00987388379056743\\
72.12	0.00987120237813691\\
72.13	0.00986851863170853\\
72.14	0.00986583254728339\\
72.15	0.00986314412085186\\
72.16	0.00986045334839344\\
72.17	0.00985776022587675\\
72.18	0.00985506474925948\\
72.19	0.00985236691448831\\
72.2	0.00984966671749887\\
72.21	0.00984696415421569\\
72.22	0.00984425922055216\\
72.23	0.00984155191241042\\
72.24	0.00983884222568136\\
72.25	0.00983613015624457\\
72.26	0.00983341569996822\\
72.27	0.00983069885270906\\
72.28	0.00982797961031234\\
72.29	0.00982525796861174\\
72.3	0.00982253392342936\\
72.31	0.00981980747057559\\
72.32	0.00981707860584911\\
72.33	0.0098143473250368\\
72.34	0.00981161362391368\\
72.35	0.00980887749824287\\
72.36	0.00980613894377549\\
72.37	0.00980339795625062\\
72.38	0.00980065453139525\\
72.39	0.00979790866492419\\
72.4	0.00979516035254002\\
72.41	0.00979240958993303\\
72.42	0.00978965637278111\\
72.43	0.00978690069674976\\
72.44	0.00978414255749197\\
72.45	0.00978138195064814\\
72.46	0.00977861887184607\\
72.47	0.00977585331670083\\
72.48	0.00977308528081473\\
72.49	0.00977031475977722\\
72.5	0.00976754174916486\\
72.51	0.0097647662445412\\
72.52	0.00976198824145673\\
72.53	0.00975920773544882\\
72.54	0.00975642472204163\\
72.55	0.00975363919674603\\
72.56	0.00975085115505953\\
72.57	0.00974806059246622\\
72.58	0.00974526750443666\\
72.59	0.00974247188642784\\
72.6	0.00973967373388307\\
72.61	0.00973687304223191\\
72.62	0.00973406980689012\\
72.63	0.0097312640232595\\
72.64	0.00972845568672791\\
72.65	0.0097256447926691\\
72.66	0.0097228313364427\\
72.67	0.00972001531339407\\
72.68	0.00971719671885425\\
72.69	0.00971437554813986\\
72.7	0.00971155179655305\\
72.71	0.00970872545938135\\
72.72	0.00970589653189762\\
72.73	0.00970306500935998\\
72.74	0.00970023088701165\\
72.75	0.00969739416008093\\
72.76	0.00969455482378108\\
72.77	0.00969171287331022\\
72.78	0.00968886830385121\\
72.79	0.00968602111057165\\
72.8	0.00968317128862364\\
72.81	0.00968031883314383\\
72.82	0.00967746373925321\\
72.83	0.00967460600205706\\
72.84	0.00967174561664485\\
72.85	0.00966888257809012\\
72.86	0.00966601688145038\\
72.87	0.00966314852176703\\
72.88	0.00966027749406522\\
72.89	0.00965740379335378\\
72.9	0.00965452741462506\\
72.91	0.00965164835285488\\
72.92	0.0096487666030024\\
72.93	0.00964588216000997\\
72.94	0.00964299501880307\\
72.95	0.00964010517429019\\
72.96	0.00963721262136268\\
72.97	0.00963431735489466\\
72.98	0.00963141936974291\\
72.99	0.00962851866074672\\
73	0.00962561522272781\\
73.01	0.00962270905049018\\
73.02	0.00961980013881999\\
73.03	0.00961688848248545\\
73.04	0.00961397407623669\\
73.05	0.00961105691480561\\
73.06	0.0096081369929058\\
73.07	0.00960521430523234\\
73.08	0.00960228884646176\\
73.09	0.00959936061125183\\
73.1	0.00959642959424145\\
73.11	0.00959349579005052\\
73.12	0.00959055919327981\\
73.13	0.00958761979851081\\
73.14	0.00958467760030558\\
73.15	0.00958173259320663\\
73.16	0.00957878477173677\\
73.17	0.00957583413039894\\
73.18	0.00957288066367611\\
73.19	0.00956992436603109\\
73.2	0.0095669652319064\\
73.21	0.00956400325572409\\
73.22	0.00956103843188565\\
73.23	0.00955807075477178\\
73.24	0.00955510021874229\\
73.25	0.0095521268181359\\
73.26	0.00954915054727009\\
73.27	0.00954617140044097\\
73.28	0.00954318937192305\\
73.29	0.00954020445596915\\
73.3	0.00953721664681017\\
73.31	0.00953422593865494\\
73.32	0.00953123232569006\\
73.33	0.00952823580207971\\
73.34	0.00952523636196547\\
73.35	0.00952223399946616\\
73.36	0.00951922870867762\\
73.37	0.00951622048367261\\
73.38	0.00951320931850051\\
73.39	0.00951019520718722\\
73.4	0.00950717814373494\\
73.41	0.00950415812212197\\
73.42	0.00950113513630253\\
73.43	0.00949810918020657\\
73.44	0.00949508024773953\\
73.45	0.0094920483327822\\
73.46	0.00948901342919046\\
73.47	0.0094859755307951\\
73.48	0.00948293463140163\\
73.49	0.00947989072479002\\
73.5	0.00947684380471453\\
73.51	0.00947379386490348\\
73.52	0.009470740899059\\
73.53	0.00946768490085687\\
73.54	0.00946462586394625\\
73.55	0.00946156378194946\\
73.56	0.00945849864846175\\
73.57	0.00945543045705109\\
73.58	0.00945235920125789\\
73.59	0.00944928487459482\\
73.6	0.0094462074705465\\
73.61	0.00944312698256932\\
73.62	0.00944004340409116\\
73.63	0.00943695672851112\\
73.64	0.00943386694919934\\
73.65	0.00943077405949663\\
73.66	0.00942767805271433\\
73.67	0.00942457892213396\\
73.68	0.009421476661007\\
73.69	0.00941837126255459\\
73.7	0.00941526271996728\\
73.71	0.00941215102640476\\
73.72	0.00940903617499554\\
73.73	0.00940591815883672\\
73.74	0.00940279697099366\\
73.75	0.00939967260449971\\
73.76	0.0093965450523559\\
73.77	0.00939341430753069\\
73.78	0.00939028036295958\\
73.79	0.00938714321154492\\
73.8	0.00938400284615549\\
73.81	0.00938085925962626\\
73.82	0.00937771244475804\\
73.83	0.00937456239431718\\
73.84	0.00937140910103523\\
73.85	0.00936825255760859\\
73.86	0.00936509275669826\\
73.87	0.00936192969092938\\
73.88	0.00935876335289101\\
73.89	0.00935559373513568\\
73.9	0.00935242083017913\\
73.91	0.00934924463049988\\
73.92	0.00934606512853892\\
73.93	0.00934288231669933\\
73.94	0.00933969618734592\\
73.95	0.00933650673280483\\
73.96	0.00933331394536319\\
73.97	0.0093301178172687\\
73.98	0.00932691834072927\\
73.99	0.00932371550791263\\
74	0.00932050931096307\\
74.01	0.00931729974228617\\
74.02	0.00931408679425123\\
74.03	0.00931087045919091\\
74.04	0.00930765072940085\\
74.05	0.00930442759713939\\
74.06	0.00930120105462711\\
74.07	0.00929797109404659\\
74.08	0.00929473770754195\\
74.09	0.00929150088721853\\
74.1	0.00928826062514249\\
74.11	0.00928501691334047\\
74.12	0.00928176974379915\\
74.13	0.00927851910846491\\
74.14	0.0092752649992434\\
74.15	0.00927200740799919\\
74.16	0.00926874632655531\\
74.17	0.00926548174669287\\
74.18	0.00926221366015065\\
74.19	0.00925894205862468\\
74.2	0.00925566693376778\\
74.21	0.00925238827718919\\
74.22	0.00924910608045409\\
74.23	0.00924582033508317\\
74.24	0.00924253103255218\\
74.25	0.00923923816429148\\
74.26	0.00923594172168561\\
74.27	0.00923264169607275\\
74.28	0.00922933807874434\\
74.29	0.00922603086094453\\
74.3	0.00922272003386974\\
74.31	0.00921940558866816\\
74.32	0.00921608751643925\\
74.33	0.00921276580823322\\
74.34	0.00920944045505058\\
74.35	0.00920611144784156\\
74.36	0.00920277877750562\\
74.37	0.0091994424349401\\
74.38	0.00919610241106353\\
74.39	0.00919275869674249\\
74.4	0.00918941128279104\\
74.41	0.00918606016039872\\
74.42	0.00918270532253261\\
74.43	0.00917934676214259\\
74.44	0.00917598447216132\\
74.45	0.00917261844550419\\
74.46	0.00916924867506927\\
74.47	0.00916587515373725\\
74.48	0.0091624978743714\\
74.49	0.00915911682981751\\
74.5	0.00915573201290387\\
74.51	0.00915234341644115\\
74.52	0.00914895103322246\\
74.53	0.00914555485602319\\
74.54	0.00914215487760102\\
74.55	0.00913875109069584\\
74.56	0.00913534348802975\\
74.57	0.00913193206230691\\
74.58	0.00912851680621362\\
74.59	0.00912509771241815\\
74.6	0.00912167477357076\\
74.61	0.00911824798230361\\
74.62	0.00911481733123074\\
74.63	0.00911138281294797\\
74.64	0.00910794442003292\\
74.65	0.00910450214504488\\
74.66	0.00910105598052482\\
74.67	0.00909760591899529\\
74.68	0.00909415195296039\\
74.69	0.00909069407490571\\
74.7	0.0090872322772983\\
74.71	0.00908376655258657\\
74.72	0.00908029689320027\\
74.73	0.00907682329155045\\
74.74	0.00907334574002934\\
74.75	0.00906986423101038\\
74.76	0.00906637875684811\\
74.77	0.00906288930987812\\
74.78	0.00905939588241701\\
74.79	0.00905589846676235\\
74.8	0.00905239705519257\\
74.81	0.00904889163996696\\
74.82	0.00904538221332559\\
74.83	0.00904186876748927\\
74.84	0.00903835129465945\\
74.85	0.00903482978701822\\
74.86	0.00903130423672822\\
74.87	0.0090277746359326\\
74.88	0.00902424097675494\\
74.89	0.00902070325129922\\
74.9	0.00901716145164974\\
74.91	0.00901361556987109\\
74.92	0.00901006559800807\\
74.93	0.00900651152808563\\
74.94	0.00900295335210881\\
74.95	0.00899939106206273\\
74.96	0.00899582464991244\\
74.97	0.00899225410760298\\
74.98	0.00898867942705919\\
74.99	0.00898510060018577\\
75	0.00898151761886713\\
75.01	0.0089779304749674\\
75.02	0.0089743391603303\\
75.03	0.00897074366677916\\
75.04	0.0089671439861168\\
75.05	0.00896354011012549\\
75.06	0.00895993203056689\\
75.07	0.00895631973918198\\
75.08	0.00895270322769102\\
75.09	0.00894908248779347\\
75.1	0.00894545751116794\\
75.11	0.00894182828947211\\
75.12	0.00893819481434271\\
75.13	0.0089345570773954\\
75.14	0.00893091507022476\\
75.15	0.00892726878440418\\
75.16	0.00892361821148585\\
75.17	0.00891996334300067\\
75.18	0.00891630417045817\\
75.19	0.00891264068534647\\
75.2	0.00890897287913222\\
75.21	0.00890530074326053\\
75.22	0.00890162426915489\\
75.23	0.00889794344821713\\
75.24	0.00889425827182734\\
75.25	0.00889056873134383\\
75.26	0.00888687481810301\\
75.27	0.00888317652341941\\
75.28	0.00887947383858553\\
75.29	0.00887576675487183\\
75.3	0.00887205526352663\\
75.31	0.00886833935577608\\
75.32	0.00886461902282407\\
75.33	0.00886089425585215\\
75.34	0.00885716504601952\\
75.35	0.00885343138446287\\
75.36	0.00884969326229644\\
75.37	0.00884595067061181\\
75.38	0.00884220360047795\\
75.39	0.00883845204294109\\
75.4	0.00883469598902468\\
75.41	0.00883093542972929\\
75.42	0.00882717035603258\\
75.43	0.00882340075888921\\
75.44	0.00881962662923078\\
75.45	0.00881584795796574\\
75.46	0.00881206473597934\\
75.47	0.00880827695413359\\
75.48	0.00880448460326711\\
75.49	0.00880068767419514\\
75.5	0.00879688615770943\\
75.51	0.00879308004457818\\
75.52	0.00878926932554595\\
75.53	0.00878545399133362\\
75.54	0.00878163403263831\\
75.55	0.0087778094401333\\
75.56	0.00877398020446794\\
75.57	0.00877014631626763\\
75.58	0.00876630776613371\\
75.59	0.00876246454464338\\
75.6	0.00875861664234966\\
75.61	0.00875476404978128\\
75.62	0.00875090675744265\\
75.63	0.00874704475581374\\
75.64	0.00874317803535005\\
75.65	0.00873930658648251\\
75.66	0.00873543039961738\\
75.67	0.00873154946513628\\
75.68	0.00872766377339595\\
75.69	0.00872377331472834\\
75.7	0.00871987807944043\\
75.71	0.00871597805781418\\
75.72	0.00871207324010649\\
75.73	0.00870816361654906\\
75.74	0.00870424917734837\\
75.75	0.00870032991268557\\
75.76	0.00869640581271644\\
75.77	0.00869247686757125\\
75.78	0.00868854306735475\\
75.79	0.00868460440214604\\
75.8	0.00868066086199854\\
75.81	0.00867671243693986\\
75.82	0.00867275911697178\\
75.83	0.0086688008920701\\
75.84	0.00866483775218463\\
75.85	0.00866086968723907\\
75.86	0.00865689668713096\\
75.87	0.00865291874173154\\
75.88	0.00864893584088576\\
75.89	0.00864494797441212\\
75.9	0.00864095513210262\\
75.91	0.0086369573037227\\
75.92	0.00863295447901113\\
75.93	0.00862894664767992\\
75.94	0.00862493379941427\\
75.95	0.00862091845719767\\
75.96	0.00861690075898092\\
75.97	0.00861288070321891\\
75.98	0.00860885828836722\\
75.99	0.00860483351288209\\
76	0.00860080637522044\\
76.01	0.00859677687383986\\
76.02	0.00859274500719869\\
76.03	0.00858871077375595\\
76.04	0.0085846741719714\\
76.05	0.00858063520030555\\
76.06	0.0085765938572196\\
76.07	0.00857255014117558\\
76.08	0.00856850405063624\\
76.09	0.00856445558406512\\
76.1	0.00856040473992655\\
76.11	0.00855635151668567\\
76.12	0.00855229591280843\\
76.13	0.00854823792676159\\
76.14	0.00854417755701276\\
76.15	0.00854011480203038\\
76.16	0.00853604966028376\\
76.17	0.00853198213024308\\
76.18	0.00852791221037938\\
76.19	0.00852383989916462\\
76.2	0.00851976519507163\\
76.21	0.0085156880965742\\
76.22	0.00851160860214698\\
76.23	0.00850752671026563\\
76.24	0.0085034424194067\\
76.25	0.00849935572804774\\
76.26	0.00849526663466725\\
76.27	0.00849117513774472\\
76.28	0.00848708123576066\\
76.29	0.00848298492719655\\
76.3	0.00847888621053492\\
76.31	0.00847478508425932\\
76.32	0.00847068154685435\\
76.33	0.00846657559680567\\
76.34	0.0084624672326\\
76.35	0.00845835645272515\\
76.36	0.00845424325567004\\
76.37	0.00845012763992465\\
76.38	0.00844600960398014\\
76.39	0.00844188914632876\\
76.4	0.00843776626546394\\
76.41	0.00843364095988022\\
76.42	0.00842951322807335\\
76.43	0.00842538306854025\\
76.44	0.00842125047977905\\
76.45	0.00841711546028907\\
76.46	0.00841297800857086\\
76.47	0.00840883812312621\\
76.48	0.00840469580245816\\
76.49	0.00840055104507101\\
76.5	0.00839640384947033\\
76.51	0.008392254214163\\
76.52	0.00838810213765717\\
76.53	0.00838394761846232\\
76.54	0.00837979065508929\\
76.55	0.00837563124605021\\
76.56	0.0083714693898586\\
76.57	0.00836730508502937\\
76.58	0.00836313833007876\\
76.59	0.00835896912352445\\
76.6	0.00835479746388552\\
76.61	0.00835062334968248\\
76.62	0.00834644677943728\\
76.63	0.00834226775167333\\
76.64	0.0083380862649155\\
76.65	0.00833390231769014\\
76.66	0.00832971590852511\\
76.67	0.00832552703594979\\
76.68	0.00832133569849506\\
76.69	0.00831714189469338\\
76.7	0.00831294562307872\\
76.71	0.00830874688218666\\
76.72	0.00830454567055435\\
76.73	0.00830034198672054\\
76.74	0.0082961358292256\\
76.75	0.00829192719661152\\
76.76	0.00828771608742197\\
76.77	0.00828350250020224\\
76.78	0.00827928643349932\\
76.79	0.00827506788586188\\
76.8	0.00827084685584031\\
76.81	0.00826662334198671\\
76.82	0.00826239734285492\\
76.83	0.00825816885700054\\
76.84	0.00825393788298096\\
76.85	0.0082497044193553\\
76.86	0.00824546846468453\\
76.87	0.00824123001753143\\
76.88	0.00823698907646059\\
76.89	0.00823274564003847\\
76.9	0.00822849970683339\\
76.91	0.00822425127541556\\
76.92	0.00822000034435706\\
76.93	0.00821574691223192\\
76.94	0.00821149097761608\\
76.95	0.00820723253908743\\
76.96	0.00820297159522583\\
76.97	0.00819870814461311\\
76.98	0.00819444218583311\\
76.99	0.00819017371747168\\
77	0.0081859027381167\\
77.01	0.00818162924635808\\
77.02	0.00817735324078785\\
77.03	0.00817307472000006\\
77.04	0.0081687936825909\\
77.05	0.00816451012715865\\
77.06	0.00816022405230375\\
77.07	0.00815593545662878\\
77.08	0.0081516443387385\\
77.09	0.00814735069723982\\
77.1	0.00814305453074191\\
77.11	0.00813875583785613\\
77.12	0.00813445461719609\\
77.13	0.00813015086737765\\
77.14	0.00812584458701896\\
77.15	0.00812153577474045\\
77.16	0.0081172244291649\\
77.17	0.00811291054891737\\
77.18	0.0081085941326253\\
77.19	0.0081042751789185\\
77.2	0.00809995368642917\\
77.21	0.0080956296537919\\
77.22	0.00809130307964372\\
77.23	0.00808697396262409\\
77.24	0.00808264230137495\\
77.25	0.00807830809454072\\
77.26	0.0080739713407683\\
77.27	0.00806963203870714\\
77.28	0.00806529018700921\\
77.29	0.00806094578432905\\
77.3	0.00805659882932378\\
77.31	0.00805224932065309\\
77.32	0.00804789725697934\\
77.33	0.00804354263696748\\
77.34	0.00803918545928515\\
77.35	0.00803482572260265\\
77.36	0.00803046342559298\\
77.37	0.00802609856693187\\
77.38	0.00802173114529778\\
77.39	0.00801736115937193\\
77.4	0.00801298860783831\\
77.41	0.00800861348938373\\
77.42	0.0080042358026978\\
77.43	0.00799985554647299\\
77.44	0.00799547271940462\\
77.45	0.0079910873201909\\
77.46	0.00798669934753294\\
77.47	0.00798230880013477\\
77.48	0.00797791567670338\\
77.49	0.00797351997594872\\
77.5	0.00796912169658374\\
77.51	0.00796472083732437\\
77.52	0.00796031739688961\\
77.53	0.0079559113740015\\
77.54	0.00795150276738516\\
77.55	0.00794709157576878\\
77.56	0.00794267779788373\\
77.57	0.00793826143246446\\
77.58	0.00793384247824862\\
77.59	0.00792942093397704\\
77.6	0.00792499679839377\\
77.61	0.00792057007024606\\
77.62	0.00791614074828446\\
77.63	0.00791170883126275\\
77.64	0.00790727431793806\\
77.65	0.0079028372070708\\
77.66	0.00789839749742475\\
77.67	0.00789395518776707\\
77.68	0.00788951027686828\\
77.69	0.00788506276350235\\
77.7	0.00788061264644668\\
77.71	0.00787615992448213\\
77.72	0.00787170459639303\\
77.73	0.00786724666096727\\
77.74	0.00786278611699623\\
77.75	0.00785832296327487\\
77.76	0.00785385719860175\\
77.77	0.00784938882177899\\
77.78	0.0078449178316124\\
77.79	0.0078404442269114\\
77.8	0.00783596800648914\\
77.81	0.00783148916916243\\
77.82	0.00782700771375185\\
77.83	0.0078225236390817\\
77.84	0.00781803694398011\\
77.85	0.00781354762727897\\
77.86	0.00780905568781403\\
77.87	0.0078045611244249\\
77.88	0.00780006393595505\\
77.89	0.00779556412125189\\
77.9	0.00779106167916675\\
77.91	0.00778655660855494\\
77.92	0.00778204890827571\\
77.93	0.00777753857719238\\
77.94	0.00777302561417229\\
77.95	0.00776851001808682\\
77.96	0.0077639917878115\\
77.97	0.00775947092222593\\
77.98	0.00775494742021388\\
77.99	0.00775042128066329\\
78	0.00774589250246631\\
78.01	0.00774136108451931\\
78.02	0.0077368270257229\\
78.03	0.007732290324982\\
78.04	0.00772775098120585\\
78.05	0.00772320899330798\\
78.06	0.00771866436020634\\
78.07	0.00771411708082325\\
78.08	0.00770956715408543\\
78.09	0.00770501457892411\\
78.1	0.00770045935427496\\
78.11	0.00769590147907814\\
78.12	0.00769134095227841\\
78.13	0.00768677777282503\\
78.14	0.00768221193967188\\
78.15	0.00767764345177748\\
78.16	0.00767307230810498\\
78.17	0.00766849850762223\\
78.18	0.00766392204930178\\
78.19	0.00765934293212093\\
78.2	0.00765476115506173\\
78.21	0.00765017671711107\\
78.22	0.00764558961726062\\
78.23	0.00764099985450696\\
78.24	0.00763640742785155\\
78.25	0.00763181233630073\\
78.26	0.00762721457886587\\
78.27	0.00762261415456325\\
78.28	0.00761801106241419\\
78.29	0.0076134053014451\\
78.3	0.00760879687068739\\
78.31	0.00760418576917762\\
78.32	0.0075995719959575\\
78.33	0.00759495555007389\\
78.34	0.00759033643057886\\
78.35	0.00758571463652971\\
78.36	0.00758109016698901\\
78.37	0.00757646302102464\\
78.38	0.00757183319770979\\
78.39	0.00756720069612304\\
78.4	0.00756256551534835\\
78.41	0.00755792765447511\\
78.42	0.00755328711259817\\
78.43	0.00754864388881788\\
78.44	0.00754399798224014\\
78.45	0.00753934939197638\\
78.46	0.00753469811714364\\
78.47	0.0075300441568646\\
78.48	0.00752538751026757\\
78.49	0.0075207281764866\\
78.5	0.00751606615466145\\
78.51	0.00751140144393766\\
78.52	0.00750673404346655\\
78.53	0.00750206395240528\\
78.54	0.0074973911699169\\
78.55	0.00749271569517036\\
78.56	0.00748803752734054\\
78.57	0.00748335666560829\\
78.58	0.0074786731091605\\
78.59	0.00747398685719009\\
78.6	0.00746929790889606\\
78.61	0.00746460626348354\\
78.62	0.00745991192016381\\
78.63	0.00745521487815435\\
78.64	0.00745051513667885\\
78.65	0.00744581269496729\\
78.66	0.00744110755225593\\
78.67	0.00743639970778739\\
78.68	0.00743168916081066\\
78.69	0.00742697591058113\\
78.7	0.00742225995636067\\
78.71	0.00741754129741761\\
78.72	0.00741281993302682\\
78.73	0.00740809586246975\\
78.74	0.00740336908503443\\
78.75	0.00739863960001556\\
78.76	0.00739390740671448\\
78.77	0.0073891725044393\\
78.78	0.00738443489250486\\
78.79	0.00737969457023279\\
78.8	0.00737495153695158\\
78.81	0.00737020579199659\\
78.82	0.00736545733471009\\
78.83	0.00736070616444132\\
78.84	0.00735595228054652\\
78.85	0.00735119568238894\\
78.86	0.00734643636933893\\
78.87	0.00734167434077395\\
78.88	0.00733690959607865\\
78.89	0.00733214213464483\\
78.9	0.00732737195587157\\
78.91	0.00732259905916523\\
78.92	0.0073178234663391\\
78.93	0.00731304531519263\\
78.94	0.00730826460330868\\
78.95	0.00730348132826657\\
78.96	0.00729869548764205\\
78.97	0.00729390707900735\\
78.98	0.00728911609993109\\
78.99	0.00728432254797836\\
79	0.00727952642071066\\
79.01	0.00727472771568591\\
79.02	0.00726992643045844\\
79.03	0.00726512256257898\\
79.04	0.00726031610959469\\
79.05	0.00725550706904908\\
79.06	0.00725069543848207\\
79.07	0.00724588121542998\\
79.08	0.00724106439742547\\
79.09	0.00723624498199759\\
79.1	0.00723142296667175\\
79.11	0.00722659834896971\\
79.12	0.00722177112640958\\
79.13	0.00721694129650582\\
79.14	0.00721210885676923\\
79.15	0.00720727380470693\\
79.16	0.00720243613782237\\
79.17	0.00719759585361533\\
79.18	0.00719275294958189\\
79.19	0.00718790742321443\\
79.2	0.00718305927200164\\
79.21	0.00717820849342851\\
79.22	0.0071733550849763\\
79.23	0.00716849904412258\\
79.24	0.00716364036834115\\
79.25	0.00715877905510212\\
79.26	0.00715391510187183\\
79.27	0.00714904850611291\\
79.28	0.0071441792652842\\
79.29	0.00713930737684082\\
79.3	0.00713443283823411\\
79.31	0.00712955564691163\\
79.32	0.00712467580031719\\
79.33	0.00711979329589079\\
79.34	0.00711490813106867\\
79.35	0.00711002030328324\\
79.36	0.00710512980996314\\
79.37	0.0071002366485332\\
79.38	0.00709534081641442\\
79.39	0.00709044231102397\\
79.4	0.00708554112977525\\
79.41	0.00708063727007776\\
79.42	0.00707573072933721\\
79.43	0.00707082150495542\\
79.44	0.0070659095943304\\
79.45	0.00706099499485629\\
79.46	0.00705607770392335\\
79.47	0.00705115771891798\\
79.48	0.00704623503722272\\
79.49	0.00704130965621619\\
79.5	0.00703638157327315\\
79.51	0.00703145078576446\\
79.52	0.00702651729105706\\
79.53	0.00702158108651401\\
79.54	0.00701664216949444\\
79.55	0.00701170053735355\\
79.56	0.00700675618744262\\
79.57	0.00700180911710901\\
79.58	0.00699685932369613\\
79.59	0.00699190680454343\\
79.6	0.00698695155698642\\
79.61	0.00698199357835667\\
79.62	0.00697703286598175\\
79.63	0.00697206941718529\\
79.64	0.00696710322928691\\
79.65	0.00696213429960228\\
79.66	0.00695716262544306\\
79.67	0.00695218820411692\\
79.68	0.00694721103292753\\
79.69	0.00694223110917453\\
79.7	0.00693724843015359\\
79.71	0.00693226299315631\\
79.72	0.0069272747954703\\
79.73	0.00692228383437912\\
79.74	0.00691729010716229\\
79.75	0.00691229361109527\\
79.76	0.00690729434344951\\
79.77	0.00690229230149237\\
79.78	0.00689728748248715\\
79.79	0.00689227988369308\\
79.8	0.00688726950236531\\
79.81	0.00688225633575493\\
79.82	0.00687724038110892\\
79.83	0.00687222163567015\\
79.84	0.00686720009667744\\
79.85	0.00686217576136545\\
79.86	0.00685714862696476\\
79.87	0.00685211869070182\\
79.88	0.00684708594979895\\
79.89	0.00684205040147436\\
79.9	0.00683701204294209\\
79.91	0.00683197087141206\\
79.92	0.00682692688409003\\
79.93	0.00682188007817763\\
79.94	0.00681683045087229\\
79.95	0.00681177799936731\\
79.96	0.00680672272085179\\
79.97	0.00680166461251067\\
79.98	0.00679660367152469\\
79.99	0.00679153989507041\\
80	0.0067864732803202\\
80.01	0.00678140382444221\\
};
\addplot [color=red,dashed]
  table[row sep=crcr]{%
80.01	0.00678140382444221\\
80.02	0.0067763315246004\\
80.03	0.00677125637795449\\
80.04	0.00676617838166003\\
80.05	0.00676109753286829\\
80.06	0.00675601382872634\\
80.07	0.00675092726637702\\
80.08	0.0067458378429589\\
80.09	0.00674074555560633\\
80.1	0.00673565040144938\\
80.11	0.00673055237761389\\
80.12	0.00672545148122141\\
80.13	0.00672034770938924\\
80.14	0.00671524105923039\\
80.15	0.0067101315278536\\
80.16	0.00670501911236331\\
80.17	0.0066999038098597\\
80.18	0.00669478561743859\\
80.19	0.00668966453219157\\
80.2	0.00668454055120587\\
80.21	0.00667941367156443\\
80.22	0.00667428389034586\\
80.23	0.00666915120462446\\
80.24	0.0066640156114702\\
80.25	0.00665887710794869\\
80.26	0.00665373569112123\\
80.27	0.00664859135804476\\
80.28	0.00664344410577187\\
80.29	0.0066382939313508\\
80.3	0.00663314083182544\\
80.31	0.00662798480423529\\
80.32	0.00662282584561549\\
80.33	0.00661766395299683\\
80.34	0.00661249912340568\\
80.35	0.00660733135386406\\
80.36	0.00660216064138956\\
80.37	0.00659698698299543\\
80.38	0.00659181037569048\\
80.39	0.00658663081647911\\
80.4	0.00658144830236136\\
80.41	0.0065762628303328\\
80.42	0.00657107439738463\\
80.43	0.0065658830005036\\
80.44	0.00656068863667204\\
80.45	0.00655549130286784\\
80.46	0.00655029099606448\\
80.47	0.00654508771323098\\
80.48	0.00653988145133192\\
80.49	0.00653467220732743\\
80.5	0.00652945997817319\\
80.51	0.00652424476082043\\
80.52	0.00651902655221592\\
80.53	0.00651380534930194\\
80.54	0.00650858114901634\\
80.55	0.00650335394829246\\
80.56	0.00649812374405919\\
80.57	0.00649289053324095\\
80.58	0.00648765431275763\\
80.59	0.00648241507952467\\
80.6	0.00647717283045302\\
80.61	0.00647192756244911\\
80.62	0.0064666792724149\\
80.63	0.00646142795724783\\
80.64	0.00645617361384084\\
80.65	0.00645091623908236\\
80.66	0.00644565582985631\\
80.67	0.0064403923830421\\
80.68	0.00643512589551462\\
80.69	0.00642985636414423\\
80.7	0.00642458378579678\\
80.71	0.00641930815733358\\
80.72	0.00641402947561143\\
80.73	0.00640874773748257\\
80.74	0.00640346293979474\\
80.75	0.0063981750793911\\
80.76	0.00639288415311031\\
80.77	0.00638759015778646\\
80.78	0.00638229309024911\\
80.79	0.00637699294732328\\
80.8	0.00637168972582941\\
80.81	0.00636638342258343\\
80.82	0.00636107403439668\\
80.83	0.00635576155807596\\
80.84	0.00635044599042353\\
80.85	0.00634512732823705\\
80.86	0.00633980556830966\\
80.87	0.00633448070742992\\
80.88	0.00632915274238183\\
80.89	0.00632382166994482\\
80.9	0.00631848748689376\\
80.91	0.00631315018999895\\
80.92	0.00630780977602613\\
80.93	0.00630246624173645\\
80.94	0.00629711958388651\\
80.95	0.00629176979922833\\
80.96	0.00628641688450936\\
80.97	0.00628106083647248\\
80.98	0.006275701651856\\
80.99	0.00627033932739365\\
81	0.00626497385981459\\
81.01	0.0062596052458434\\
81.02	0.00625423348220011\\
81.03	0.00624885856560015\\
81.04	0.0062434804927544\\
81.05	0.00623809926036913\\
81.06	0.0062327148651461\\
81.07	0.00622732730378243\\
81.08	0.00622193657297073\\
81.09	0.00621654266939899\\
81.1	0.00621114558975067\\
81.11	0.00620574533070464\\
81.12	0.00620034188893523\\
81.13	0.00619493526111217\\
81.14	0.00618952544390064\\
81.15	0.00618411243396128\\
81.16	0.00617869622795014\\
81.17	0.00617327682251874\\
81.18	0.00616785421431402\\
81.19	0.00616242839997839\\
81.2	0.00615699937614969\\
81.21	0.00615156713946121\\
81.22	0.00614613168654173\\
81.23	0.00614069301401545\\
81.24	0.00613525111850204\\
81.25	0.00612980599661666\\
81.26	0.0061243576449699\\
81.27	0.00611890606016783\\
81.28	0.00611345123881202\\
81.29	0.0061079931774995\\
81.3	0.00610253187282277\\
81.31	0.00609706732136985\\
81.32	0.00609159951972423\\
81.33	0.00608612846446491\\
81.34	0.00608065415216637\\
81.35	0.00607517657939862\\
81.36	0.00606969574272716\\
81.37	0.00606421163871303\\
81.38	0.00605872426391278\\
81.39	0.00605323361487849\\
81.4	0.00604773968815777\\
81.41	0.00604224248029378\\
81.42	0.00603674198782522\\
81.43	0.00603123820728636\\
81.44	0.00602573113520701\\
81.45	0.00602022076811256\\
81.46	0.00601470710252397\\
81.47	0.00600919013495779\\
81.48	0.00600366986192615\\
81.49	0.00599814627993678\\
81.5	0.00599261938549303\\
81.51	0.00598708917509386\\
81.52	0.00598155564523385\\
81.53	0.00597601879240321\\
81.54	0.00597047861308779\\
81.55	0.00596493510376912\\
81.56	0.00595938826092435\\
81.57	0.00595383808102633\\
81.58	0.00594828456054358\\
81.59	0.00594272769594033\\
81.6	0.00593716748367648\\
81.61	0.00593160392020766\\
81.62	0.00592603700198525\\
81.63	0.00592046672545632\\
81.64	0.00591489308706374\\
81.65	0.00590931608324609\\
81.66	0.00590373571043776\\
81.67	0.00589815196506891\\
81.68	0.00589256484356551\\
81.69	0.00588697434234934\\
81.7	0.00588138045783799\\
81.71	0.00587578318644491\\
81.72	0.00587018252457939\\
81.73	0.00586457846864661\\
81.74	0.00585897101504761\\
81.75	0.00585336016017935\\
81.76	0.00584774590043468\\
81.77	0.00584212823220241\\
81.78	0.00583650715186729\\
81.79	0.00583088265581001\\
81.8	0.00582525474040727\\
81.81	0.00581962340203176\\
81.82	0.00581398863705218\\
81.83	0.00580835044183326\\
81.84	0.00580270881273579\\
81.85	0.00579706374611665\\
81.86	0.00579141523832878\\
81.87	0.00578576328572124\\
81.88	0.00578010788463923\\
81.89	0.0057744490314241\\
81.9	0.00576878672241335\\
81.91	0.0057631209539407\\
81.92	0.00575745172233606\\
81.93	0.0057517790239256\\
81.94	0.00574610285503175\\
81.95	0.00574042321197319\\
81.96	0.00573474009106494\\
81.97	0.00572905348861832\\
81.98	0.00572336340094103\\
81.99	0.00571766982433714\\
82	0.00571197275510711\\
82.01	0.00570627218954784\\
82.02	0.00570056812395268\\
82.03	0.00569486055461147\\
82.04	0.00568914947781056\\
82.05	0.00568343488983282\\
82.06	0.00567771678695771\\
82.07	0.00567199516546125\\
82.08	0.00566627002161611\\
82.09	0.00566054135169159\\
82.1	0.0056548091519537\\
82.11	0.00564907341866514\\
82.12	0.00564333414808535\\
82.13	0.00563759133647057\\
82.14	0.00563184498007381\\
82.15	0.00562609507514498\\
82.16	0.0056203416179308\\
82.17	0.00561458460467494\\
82.18	0.00560882403161799\\
82.19	0.00560305989499754\\
82.2	0.00559729219104817\\
82.21	0.00559152091600153\\
82.22	0.00558574606608636\\
82.23	0.0055799676375285\\
82.24	0.00557418562655098\\
82.25	0.00556840002937401\\
82.26	0.00556261084221508\\
82.27	0.00555681806128891\\
82.28	0.00555102168280759\\
82.29	0.00554522170298054\\
82.3	0.00553941811801463\\
82.31	0.00553361092411413\\
82.32	0.00552780011748084\\
82.33	0.00552198569431409\\
82.34	0.00551616765081078\\
82.35	0.00551034598316547\\
82.36	0.00550452068757037\\
82.37	0.00549869176021542\\
82.38	0.00549285919728834\\
82.39	0.00548702299497467\\
82.4	0.00548118314945781\\
82.41	0.0054753396569191\\
82.42	0.00546949251353783\\
82.43	0.00546364171549133\\
82.44	0.005457787258955\\
82.45	0.00545192914010239\\
82.46	0.00544606735510521\\
82.47	0.00544020190013344\\
82.48	0.00543433277135532\\
82.49	0.00542845996493749\\
82.5	0.00542258347704499\\
82.51	0.0054167033038413\\
82.52	0.00541081944148847\\
82.53	0.00540493188614714\\
82.54	0.00539904063397659\\
82.55	0.00539314568113484\\
82.56	0.00538724702377867\\
82.57	0.00538134465806372\\
82.58	0.00537543858014454\\
82.59	0.00536952878617465\\
82.6	0.00536361527230664\\
82.61	0.0053576980346922\\
82.62	0.00535177706948222\\
82.63	0.00534585237282682\\
82.64	0.00533992394087547\\
82.65	0.00533399176977705\\
82.66	0.0053280558556799\\
82.67	0.00532211619438163\\
82.68	0.00531617278152109\\
82.69	0.00531022561273023\\
82.7	0.0053042746836342\\
82.71	0.00529831998985126\\
82.72	0.00529236152699282\\
82.73	0.00528639929066343\\
82.74	0.00528043327646079\\
82.75	0.00527446347997571\\
82.76	0.00526848989679214\\
82.77	0.00526251252248718\\
82.78	0.00525653135263102\\
82.79	0.00525054638278702\\
82.8	0.00524455760851162\\
82.81	0.00523856502535443\\
82.82	0.00523256862885814\\
82.83	0.00522656841455859\\
82.84	0.00522056437798473\\
82.85	0.00521455651465863\\
82.86	0.00520854482009547\\
82.87	0.00520252928980357\\
82.88	0.00519650991928436\\
82.89	0.00519048670403237\\
82.9	0.00518445963953528\\
82.91	0.00517842872127387\\
82.92	0.00517239394472205\\
82.93	0.00516635530534685\\
82.94	0.00516031279860841\\
82.95	0.00515426641996\\
82.96	0.00514821616484803\\
82.97	0.00514216202871202\\
82.98	0.00513610400698462\\
82.99	0.00513004209509163\\
83	0.00512397628845194\\
83.01	0.00511790658247763\\
83.02	0.00511183297257387\\
83.03	0.00510575545413901\\
83.04	0.00509967402256452\\
83.05	0.00509358867323502\\
83.06	0.0050874994015283\\
83.07	0.00508140620281529\\
83.08	0.00507530907246009\\
83.09	0.00506920800581995\\
83.1	0.0050631029982453\\
83.11	0.00505699404507976\\
83.12	0.00505088114166009\\
83.13	0.00504476428331628\\
83.14	0.00503864346537147\\
83.15	0.00503251868314204\\
83.16	0.00502638993193754\\
83.17	0.00502025720706075\\
83.18	0.00501412050380765\\
83.19	0.00500797981746745\\
83.2	0.00500183514332262\\
83.21	0.00499568647664883\\
83.22	0.00498953381271502\\
83.23	0.00498337714678338\\
83.24	0.00497721647410938\\
83.25	0.00497105178994173\\
83.26	0.00496488308952247\\
83.27	0.0049587103680869\\
83.28	0.00495253362086363\\
83.29	0.00494635284307459\\
83.3	0.00494016802993505\\
83.31	0.00493397917665358\\
83.32	0.00492778627843213\\
83.33	0.004921589330466\\
83.34	0.00491538832794384\\
83.35	0.00490918326604773\\
83.36	0.00490297413995311\\
83.37	0.00489676094482885\\
83.38	0.00489054367583722\\
83.39	0.00488432232813396\\
83.4	0.00487809689686824\\
83.41	0.00487186737718271\\
83.42	0.00486563376421349\\
83.43	0.00485939605309022\\
83.44	0.00485315423893604\\
83.45	0.00484690831686761\\
83.46	0.00484065828199517\\
83.47	0.0048344041294225\\
83.48	0.00482814585424696\\
83.49	0.00482188345155955\\
83.5	0.00481561691644484\\
83.51	0.00480934624398107\\
83.52	0.00480307142924015\\
83.53	0.00479679246728763\\
83.54	0.0047905093531828\\
83.55	0.00478422208197864\\
83.56	0.00477793064872191\\
83.57	0.00477163504845308\\
83.58	0.00476533527620647\\
83.59	0.00475903132701018\\
83.6	0.00475272319588612\\
83.61	0.00474641087785011\\
83.62	0.00474009436791182\\
83.63	0.00473377366107483\\
83.64	0.00472744875233667\\
83.65	0.00472111963668882\\
83.66	0.00471478630911673\\
83.67	0.00470844876459992\\
83.68	0.00470210699811188\\
83.69	0.00469576100462023\\
83.7	0.00468941077908666\\
83.71	0.00468305631646701\\
83.72	0.00467669761171126\\
83.73	0.0046703346597636\\
83.74	0.00466396745556243\\
83.75	0.00465759599404043\\
83.76	0.00465122027012454\\
83.77	0.00464484027873604\\
83.78	0.00463845601479056\\
83.79	0.00463206747319813\\
83.8	0.00462567464886321\\
83.81	0.00461927753668471\\
83.82	0.00461287613155604\\
83.83	0.00460647042836517\\
83.84	0.00460006042199464\\
83.85	0.0045936461073216\\
83.86	0.00458722747921785\\
83.87	0.0045808045325499\\
83.88	0.00457437726217901\\
83.89	0.00456794566296119\\
83.9	0.0045615097297473\\
83.91	0.00455506945738305\\
83.92	0.00454862484070908\\
83.93	0.00454217587456096\\
83.94	0.00453572255376931\\
83.95	0.00452926487315976\\
83.96	0.00452280282755306\\
83.97	0.00451633641176509\\
83.98	0.00450986562060694\\
83.99	0.00450339044888495\\
84	0.00449691089140076\\
84.01	0.00449042694295134\\
84.02	0.00448393859832911\\
84.03	0.0044774458523219\\
84.04	0.00447094869971308\\
84.05	0.00446444713528158\\
84.06	0.00445794115380198\\
84.07	0.00445143075004451\\
84.08	0.00444491591877517\\
84.09	0.00443839665475575\\
84.1	0.00443187295274391\\
84.11	0.00442534480749325\\
84.12	0.00441881221375334\\
84.13	0.0044122751662698\\
84.14	0.0044057336597844\\
84.15	0.00439918768903507\\
84.16	0.004392637248756\\
84.17	0.00438608233367769\\
84.18	0.00437952293852707\\
84.19	0.00437295905802748\\
84.2	0.00436639068689884\\
84.21	0.00435981781985764\\
84.22	0.00435324045161707\\
84.23	0.00434665857688709\\
84.24	0.00434007219037446\\
84.25	0.00433348128678288\\
84.26	0.00432688586081302\\
84.27	0.00432028590716265\\
84.28	0.00431368142052666\\
84.29	0.0043070723955972\\
84.3	0.00430045882706373\\
84.31	0.00429384070961311\\
84.32	0.00428721803792973\\
84.33	0.00428059080669551\\
84.34	0.00427395901059008\\
84.35	0.00426732264429082\\
84.36	0.00426068170247295\\
84.37	0.00425403617980969\\
84.38	0.00424738607097225\\
84.39	0.00424073137063001\\
84.4	0.00423407207345061\\
84.41	0.00422740817409998\\
84.42	0.00422073966724255\\
84.43	0.00421406654754126\\
84.44	0.00420738880965773\\
84.45	0.00420070644825231\\
84.46	0.00419401945798424\\
84.47	0.00418732783351173\\
84.48	0.00418063156949208\\
84.49	0.00417393066058179\\
84.5	0.00416722510143666\\
84.51	0.00416051488671196\\
84.52	0.00415380001106248\\
84.53	0.0041470804691427\\
84.54	0.00414035625560688\\
84.55	0.00413362736510922\\
84.56	0.00412689379230395\\
84.57	0.00412015553184546\\
84.58	0.00411341257838846\\
84.59	0.00410666492658809\\
84.6	0.00409991257110005\\
84.61	0.00409315550658074\\
84.62	0.00408639372768741\\
84.63	0.00407962722907829\\
84.64	0.00407285600541271\\
84.65	0.00406608005135128\\
84.66	0.00405929936155603\\
84.67	0.00405251393069052\\
84.68	0.00404572375342004\\
84.69	0.00403892882441174\\
84.7	0.00403212913833478\\
84.71	0.00402532468986048\\
84.72	0.00401851547366252\\
84.73	0.00401170148441705\\
84.74	0.00400488271680288\\
84.75	0.00399805916550166\\
84.76	0.003991230825198\\
84.77	0.00398439769057969\\
84.78	0.00397755975633789\\
84.79	0.00397071701716721\\
84.8	0.00396386946776598\\
84.81	0.00395701710283643\\
84.82	0.00395015991708481\\
84.83	0.00394329790522163\\
84.84	0.00393643106196181\\
84.85	0.00392955938202494\\
84.86	0.00392268286013537\\
84.87	0.00391580149102254\\
84.88	0.00390891526942103\\
84.89	0.00390202419007089\\
84.9	0.00389512824771779\\
84.91	0.00388822743711323\\
84.92	0.00388132175301474\\
84.93	0.00387441119018616\\
84.94	0.00386749574339776\\
84.95	0.00386057540742655\\
84.96	0.00385365017705644\\
84.97	0.00384672004707852\\
84.98	0.00383978501229125\\
84.99	0.00383284506750072\\
85	0.00382590020752085\\
85.01	0.00381895042717371\\
85.02	0.00381199572128966\\
85.03	0.00380503608470768\\
85.04	0.00379807151227559\\
85.05	0.00379110199885029\\
85.06	0.00378412753929805\\
85.07	0.00377714812849476\\
85.08	0.00377016376132618\\
85.09	0.00376317443268822\\
85.1	0.00375618013748722\\
85.11	0.00374918087064023\\
85.12	0.00374217662707528\\
85.13	0.00373516740173165\\
85.14	0.00372815318956021\\
85.15	0.00372113398552364\\
85.16	0.0037141097845968\\
85.17	0.00370708058176697\\
85.18	0.00370004637203422\\
85.19	0.00369300715041164\\
85.2	0.00368596291192571\\
85.21	0.00367891365161663\\
85.22	0.00367185936453859\\
85.23	0.00366480004576013\\
85.24	0.00365773569036448\\
85.25	0.00365066629344985\\
85.26	0.00364359185012984\\
85.27	0.00363651235553373\\
85.28	0.00362942780480686\\
85.29	0.00362233819311097\\
85.3	0.00361524351562456\\
85.31	0.00360814376754328\\
85.32	0.00360103894408026\\
85.33	0.00359392904046652\\
85.34	0.00358681405195133\\
85.35	0.00357969397380262\\
85.36	0.00357256880130733\\
85.37	0.00356543852977187\\
85.38	0.00355830315452243\\
85.39	0.0035511626709055\\
85.4	0.00354401707428819\\
85.41	0.00353686636005869\\
85.42	0.00352971052362669\\
85.43	0.00352254956042382\\
85.44	0.00351538346590407\\
85.45	0.00350821223554424\\
85.46	0.00350103586484438\\
85.47	0.00349385434932828\\
85.48	0.00348666768454386\\
85.49	0.00347947586606373\\
85.5	0.00347227888948558\\
85.51	0.00346507675043272\\
85.52	0.00345786944455453\\
85.53	0.00345065696752697\\
85.54	0.00344488106945515\\
85.55	0.00344218099965369\\
85.56	0.0034394855100355\\
85.57	0.00343679462172208\\
85.58	0.00343410835590736\\
85.59	0.00343142673385796\\
85.6	0.00342874977691339\\
85.61	0.00342607750648627\\
85.62	0.00342340994406259\\
85.63	0.00342074436059817\\
85.64	0.00341808043756742\\
85.65	0.00341541818512362\\
85.66	0.00341275761345693\\
85.67	0.00341009873279459\\
85.68	0.00340744155340098\\
85.69	0.00340478608557778\\
85.7	0.00340213233966408\\
85.71	0.0033994803260365\\
85.72	0.00339683005510936\\
85.73	0.00339418153733473\\
85.74	0.00339153478320262\\
85.75	0.00338888980324109\\
85.76	0.00338624660801637\\
85.77	0.00338360520813297\\
85.78	0.00338096561423386\\
85.79	0.00337832783700055\\
85.8	0.00337569188715323\\
85.81	0.00337305777545092\\
85.82	0.00337042551269156\\
85.83	0.00336779510971218\\
85.84	0.00336516657738901\\
85.85	0.00336253992663759\\
85.86	0.00335991516841297\\
85.87	0.00335729231370973\\
85.88	0.00335467137356222\\
85.89	0.00335205235904463\\
85.9	0.00334943528127114\\
85.91	0.00334682015139603\\
85.92	0.00334420698061387\\
85.93	0.00334159578015956\\
85.94	0.00333898656130856\\
85.95	0.00333637933537696\\
85.96	0.00333377411372163\\
85.97	0.00333117090774036\\
85.98	0.00332856972887198\\
85.99	0.00332597058859652\\
86	0.00332337349843531\\
86.01	0.00332077846995116\\
86.02	0.00331818551474843\\
86.03	0.00331559464447323\\
86.04	0.00331300587081352\\
86.05	0.00331041920549925\\
86.06	0.00330783466030252\\
86.07	0.00330525224703768\\
86.08	0.00330267197756149\\
86.09	0.00330009386377323\\
86.1	0.0032975179176149\\
86.11	0.00329494415107128\\
86.12	0.00329237257617012\\
86.13	0.00328980320498225\\
86.14	0.00328723604962175\\
86.15	0.00328467112224605\\
86.16	0.00328210843505609\\
86.17	0.00327954800029646\\
86.18	0.00327698983025554\\
86.19	0.00327443393726562\\
86.2	0.00327188033370307\\
86.21	0.00326932903198846\\
86.22	0.00326678004458671\\
86.23	0.00326423338400721\\
86.24	0.003261689062804\\
86.25	0.00325914709357586\\
86.26	0.00325660748896651\\
86.27	0.0032540702616647\\
86.28	0.00325153542440436\\
86.29	0.00324900298996479\\
86.3	0.00324647297117074\\
86.31	0.00324394538089257\\
86.32	0.00324142023204643\\
86.33	0.00323889753759435\\
86.34	0.00323637731054441\\
86.35	0.00323385956395089\\
86.36	0.0032313443109144\\
86.37	0.003228831564582\\
86.38	0.0032263213381474\\
86.39	0.00322381364485108\\
86.4	0.00322130849798039\\
86.41	0.00321880591086976\\
86.42	0.00321630589690082\\
86.43	0.00321380846950252\\
86.44	0.0032113136421513\\
86.45	0.00320882142837124\\
86.46	0.00320633184173419\\
86.47	0.00320384489585993\\
86.48	0.00320136060441628\\
86.49	0.00319887898111929\\
86.5	0.00319640003973337\\
86.51	0.00319392379407142\\
86.52	0.00319145025799499\\
86.53	0.00318897944541442\\
86.54	0.00318651137028901\\
86.55	0.00318404604662712\\
86.56	0.00318158348848636\\
86.57	0.0031791237099737\\
86.58	0.00317666672524565\\
86.59	0.00317421254850839\\
86.6	0.00317176119401791\\
86.61	0.00316931267608017\\
86.62	0.00316686700905124\\
86.63	0.00316442420733744\\
86.64	0.00316198428539551\\
86.65	0.00315954725773271\\
86.66	0.00315711313890705\\
86.67	0.00315468194352733\\
86.68	0.00315225368625337\\
86.69	0.00314982838179614\\
86.7	0.00314740604491786\\
86.71	0.00314498669043221\\
86.72	0.00314257033320445\\
86.73	0.00314015698815154\\
86.74	0.00313774667024235\\
86.75	0.00313533939449774\\
86.76	0.00313293517599075\\
86.77	0.00313053402984673\\
86.78	0.0031281359712435\\
86.79	0.00312574101541148\\
86.8	0.00312334917763384\\
86.81	0.00312096047324665\\
86.82	0.00311857491763905\\
86.83	0.00311619252625333\\
86.84	0.00311381331458515\\
86.85	0.00311143729818366\\
86.86	0.00310906449265163\\
86.87	0.0031066949136456\\
86.88	0.00310432857687605\\
86.89	0.00310196549810751\\
86.9	0.00309960569315875\\
86.91	0.00309724917790287\\
86.92	0.00309489596826751\\
86.93	0.00309254608023492\\
86.94	0.00309019580124057\\
86.95	0.00308784491742243\\
86.96	0.00308549342798923\\
86.97	0.00308314133214113\\
86.98	0.0030807886290697\\
86.99	0.00307843531795785\\
87	0.00307608139797979\\
87.01	0.00307372686830099\\
87.02	0.00307137172807808\\
87.03	0.00306901597645881\\
87.04	0.00306665961258203\\
87.05	0.0030643026355776\\
87.06	0.00306194504456634\\
87.07	0.00305958683865997\\
87.08	0.00305722801696107\\
87.09	0.003054868578563\\
87.1	0.00305250852254986\\
87.11	0.00305014784799643\\
87.12	0.0030477865539681\\
87.13	0.00304542463952084\\
87.14	0.00304306210370109\\
87.15	0.00304069894554578\\
87.16	0.00303833516408217\\
87.17	0.00303597075832789\\
87.18	0.00303360572729081\\
87.19	0.00303124006996902\\
87.2	0.00302887378535076\\
87.21	0.00302650687241433\\
87.22	0.00302413933012808\\
87.23	0.00302177115745031\\
87.24	0.00301940235332923\\
87.25	0.00301703291670287\\
87.26	0.00301466284649908\\
87.27	0.00301229214163538\\
87.28	0.00300992080101896\\
87.29	0.00300754882354661\\
87.3	0.00300517620810464\\
87.31	0.00300280295356881\\
87.32	0.0030004290588043\\
87.33	0.0029980545226656\\
87.34	0.00299567934399651\\
87.35	0.00299330352162999\\
87.36	0.00299092705438816\\
87.37	0.00298854994108222\\
87.38	0.00298617218051239\\
87.39	0.00298379377146778\\
87.4	0.00298141471272644\\
87.41	0.00297903500305519\\
87.42	0.0029766546412096\\
87.43	0.00297427362593393\\
87.44	0.00297189195596101\\
87.45	0.00296950963001225\\
87.46	0.0029671266467975\\
87.47	0.00296474300501502\\
87.48	0.00296235870335141\\
87.49	0.00295997374048152\\
87.5	0.0029575881150684\\
87.51	0.00295520182576321\\
87.52	0.00295281487120518\\
87.53	0.0029504272500215\\
87.54	0.00294803896082729\\
87.55	0.00294565000222548\\
87.56	0.00294326037280679\\
87.57	0.00294087007114963\\
87.58	0.00293847909582001\\
87.59	0.00293608744537151\\
87.6	0.00293369511834517\\
87.61	0.00293130211326944\\
87.62	0.00292890842866009\\
87.63	0.00292651406302013\\
87.64	0.00292411901483978\\
87.65	0.00292172328259632\\
87.66	0.00291932686475408\\
87.67	0.00291692975976433\\
87.68	0.00291453196606522\\
87.69	0.0029121334820817\\
87.7	0.00290973430622542\\
87.71	0.0029073344368947\\
87.72	0.00290493387247439\\
87.73	0.00290253261133586\\
87.74	0.00290013046945741\\
87.75	0.00289772744173341\\
87.76	0.00289532352850149\\
87.77	0.00289291873010137\\
87.78	0.00289051304687493\\
87.79	0.00288810647916613\\
87.8	0.0028856990273211\\
87.81	0.00288329069168808\\
87.82	0.00288088147261747\\
87.83	0.00287847137046183\\
87.84	0.00287606038557585\\
87.85	0.0028736485183164\\
87.86	0.00287123576904251\\
87.87	0.00286882213811539\\
87.88	0.00286640762589842\\
87.89	0.00286399223275719\\
87.9	0.00286157595905945\\
87.91	0.00285915880517517\\
87.92	0.00285674077147652\\
87.93	0.00285432185833788\\
87.94	0.00285190206613584\\
87.95	0.00284948139524924\\
87.96	0.00284705984605911\\
87.97	0.00284463741894874\\
87.98	0.00284221411430366\\
87.99	0.00283978993251165\\
88	0.00283736487396273\\
88.01	0.00283493893904919\\
88.02	0.0028325121281656\\
88.03	0.00283008444170878\\
88.04	0.00282765588007784\\
88.05	0.00282522644367417\\
88.06	0.00282279613290147\\
88.07	0.00282036494816573\\
88.08	0.00281793288987523\\
88.09	0.00281549995844058\\
88.1	0.0028130661542747\\
88.11	0.00281063147779284\\
88.12	0.00280819592941258\\
88.13	0.00280575950955381\\
88.14	0.00280332221863882\\
88.15	0.0028008840570922\\
88.16	0.00279844502534093\\
88.17	0.00279600512381433\\
88.18	0.00279356435294409\\
88.19	0.0027911227131643\\
88.2	0.00278868020491142\\
88.21	0.00278623682862429\\
88.22	0.00278379258474416\\
88.23	0.00278134747371469\\
88.24	0.00277890149598192\\
88.25	0.00277645465199434\\
88.26	0.00277400694220284\\
88.27	0.00277155836706075\\
88.28	0.00276910892702384\\
88.29	0.00276665862255032\\
88.3	0.00276420745410085\\
88.31	0.00276175542213854\\
88.32	0.00275930252712898\\
88.33	0.00275684876954021\\
88.34	0.00275439414984277\\
88.35	0.00275193866850968\\
88.36	0.00274948232601643\\
88.37	0.00274702512284103\\
88.38	0.00274456705946399\\
88.39	0.00274210813636834\\
88.4	0.0027396483540396\\
88.41	0.00273718771296586\\
88.42	0.00273472621363771\\
88.43	0.00273226385654829\\
88.44	0.00272980064219328\\
88.45	0.00272733657107094\\
88.46	0.00272487164368207\\
88.47	0.00272240586053003\\
88.48	0.00271993922212077\\
88.49	0.00271747172896284\\
88.5	0.00271500338156733\\
88.51	0.00271253418044799\\
88.52	0.00271006412612112\\
88.53	0.00270759321910566\\
88.54	0.00270512145992317\\
88.55	0.00270264884909782\\
88.56	0.00270017538715642\\
88.57	0.00269770107462843\\
88.58	0.00269522591204595\\
88.59	0.00269274989994373\\
88.6	0.0026902730388592\\
88.61	0.00268779532933243\\
88.62	0.0026853167719062\\
88.63	0.00268283736712596\\
88.64	0.00268035711553984\\
88.65	0.0026778760176987\\
88.66	0.00267539407415609\\
88.67	0.00267291128546825\\
88.68	0.00267042765219419\\
88.69	0.00266794317489562\\
88.7	0.00266545785413699\\
88.71	0.0026629716904855\\
88.72	0.00266048468451109\\
88.73	0.00265799683678649\\
88.74	0.00265550814788717\\
88.75	0.00265301861839137\\
88.76	0.00265052824888015\\
88.77	0.00264803703993732\\
88.78	0.0026455449921495\\
88.79	0.00264305210610614\\
88.8	0.00264055838239947\\
88.81	0.00263806382162456\\
88.82	0.00263556842437931\\
88.83	0.00263307219126445\\
88.84	0.00263057512288356\\
88.85	0.00262807721984307\\
88.86	0.00262557848275227\\
88.87	0.00262307891222332\\
88.88	0.00262057850887126\\
88.89	0.00261807727331401\\
88.9	0.00261557520617238\\
88.91	0.0026130723080701\\
88.92	0.00261056857963378\\
88.93	0.00260806402149296\\
88.94	0.00260555863428011\\
88.95	0.00260305241863062\\
88.96	0.00260054537518284\\
88.97	0.00259803750457804\\
88.98	0.00259552880746047\\
88.99	0.00259301928447734\\
89	0.00259050893627883\\
89.01	0.00258799776351812\\
89.02	0.00258548576685134\\
89.03	0.00258297294693765\\
89.04	0.00258045930443921\\
89.05	0.00257794484002119\\
89.06	0.00257542955435179\\
89.07	0.00257291344810223\\
89.08	0.00257039652194678\\
89.09	0.00256787877656276\\
89.1	0.00256536021263052\\
89.11	0.00256284083083351\\
89.12	0.00256032063185824\\
89.13	0.0025577996163943\\
89.14	0.00255527778513437\\
89.15	0.00255275513877422\\
89.16	0.00255023167801274\\
89.17	0.00254770740355194\\
89.18	0.00254518231609694\\
89.19	0.002542656416356\\
89.2	0.00254012970504053\\
89.21	0.00253760218286507\\
89.22	0.00253507385054734\\
89.23	0.00253254470880821\\
89.24	0.00253001475837174\\
89.25	0.00252748399996517\\
89.26	0.00252495243431894\\
89.27	0.00252242006216667\\
89.28	0.00251988688424523\\
89.29	0.00251735290129467\\
89.3	0.00251481811405831\\
89.31	0.00251228252328266\\
89.32	0.00250974612971753\\
89.33	0.00250720893411594\\
89.34	0.00250467093723421\\
89.35	0.00250213213983191\\
89.36	0.0024995925426719\\
89.37	0.00249705214652034\\
89.38	0.00249451095214668\\
89.39	0.00249196896032369\\
89.4	0.00248942617182746\\
89.41	0.00248688258743739\\
89.42	0.00248433820793625\\
89.43	0.00248179303411012\\
89.44	0.00247924706674846\\
89.45	0.0024767003066441\\
89.46	0.00247415275459322\\
89.47	0.00247160441139541\\
89.48	0.00246905527785363\\
89.49	0.00246650535477426\\
89.5	0.00246395464296709\\
89.51	0.00246140314324531\\
89.52	0.00245885085642559\\
89.53	0.00245629778332798\\
89.54	0.00245374392477601\\
89.55	0.00245118928159668\\
89.56	0.00244863385462044\\
89.57	0.00244607764468123\\
89.58	0.00244352065261646\\
89.59	0.00244096287926706\\
89.6	0.00243840432547745\\
89.61	0.00243584499209556\\
89.62	0.00243328487997289\\
89.63	0.00243072398996442\\
89.64	0.00242816232292872\\
89.65	0.00242559987972789\\
89.66	0.00242303666122759\\
89.67	0.00242047266829708\\
89.68	0.0024179079018092\\
89.69	0.00241534236264036\\
89.7	0.0024127760516706\\
89.71	0.00241020896978357\\
89.72	0.00240764111786654\\
89.73	0.00240507249681041\\
89.74	0.00240250310750975\\
89.75	0.00239993295086274\\
89.76	0.00239736202777127\\
89.77	0.00239479033914088\\
89.78	0.0023922178858808\\
89.79	0.00238964466890395\\
89.8	0.00238707068912697\\
89.81	0.00238449594747021\\
89.82	0.00238192044485773\\
89.83	0.00237934418221735\\
89.84	0.00237676716048061\\
89.85	0.00237418938058284\\
89.86	0.0023716108434631\\
89.87	0.00236903155006426\\
89.88	0.00236645150133295\\
89.89	0.00236387069821963\\
89.9	0.00236128914167854\\
89.91	0.00235870683266774\\
89.92	0.00235612377214914\\
89.93	0.00235353996108849\\
89.94	0.00235095540045536\\
89.95	0.0023483700912232\\
89.96	0.00234578403436936\\
89.97	0.00234319723087501\\
89.98	0.00234060968172529\\
89.99	0.00233802138790917\\
90	0.00233543235041959\\
90.01	0.00233284257025338\\
90.02	0.00233025204841133\\
90.03	0.00232766078589817\\
90.04	0.00232506878372258\\
90.05	0.00232247604289722\\
90.06	0.00231988256443873\\
90.07	0.00231728834936773\\
90.08	0.00231469339870885\\
90.09	0.00231209771349074\\
90.1	0.00230950129474606\\
90.11	0.00230690414351151\\
90.12	0.00230430626082784\\
90.13	0.00230170764773986\\
90.14	0.00229910830529645\\
90.15	0.00229650823455056\\
90.16	0.00229390743655922\\
90.17	0.0022913059123836\\
90.18	0.00228870366308896\\
90.19	0.00228610068974468\\
90.2	0.00228349699342429\\
90.21	0.00228089257520546\\
90.22	0.00227828743617002\\
90.23	0.00227568157740398\\
90.24	0.00227307499999752\\
90.25	0.00227046770504503\\
90.26	0.00226785969364509\\
90.27	0.0022652509669005\\
90.28	0.00226264152591828\\
90.29	0.00226003137180973\\
90.3	0.00225742050569035\\
90.31	0.00225480892867994\\
90.32	0.00225219664190257\\
90.33	0.00224958364648658\\
90.34	0.00224696994356463\\
90.35	0.00224435553427369\\
90.36	0.00224174041975503\\
90.37	0.00223912460115428\\
90.38	0.00223650807962141\\
90.39	0.00223389085631075\\
90.4	0.00223127293238101\\
90.41	0.00222865430899525\\
90.42	0.00222603498732096\\
90.43	0.00222341496853004\\
90.44	0.00222079425379879\\
90.45	0.00221817284430794\\
90.46	0.00221555074124268\\
90.47	0.00221292794579265\\
90.48	0.00221030445915197\\
90.49	0.00220768028251922\\
90.5	0.00220505541709749\\
90.51	0.00220242986409437\\
90.52	0.00219980362472197\\
90.53	0.00219717670019694\\
90.54	0.00219454909174046\\
90.55	0.00219192080057828\\
90.56	0.00218929182794071\\
90.57	0.00218666217506264\\
90.58	0.00218403184318356\\
90.59	0.00218140083354757\\
90.6	0.00217876914740338\\
90.61	0.00217613678600436\\
90.62	0.00217350375060848\\
90.63	0.00217087004247842\\
90.64	0.00216823566288148\\
90.65	0.0021656006130897\\
90.66	0.00216296489437977\\
90.67	0.00216032850803311\\
90.68	0.00215769145533588\\
90.69	0.00215505373757894\\
90.7	0.00215241535605794\\
90.71	0.00214977631207326\\
90.72	0.00214713660693009\\
90.73	0.00214449624193838\\
90.74	0.0021418552184129\\
90.75	0.00213921353767322\\
90.76	0.00213657120104377\\
90.77	0.0021339282098538\\
90.78	0.00213128456543742\\
90.79	0.00212864026913361\\
90.8	0.00212599532228625\\
90.81	0.0021233497262441\\
90.82	0.00212070348236085\\
90.83	0.00211805659199508\\
90.84	0.00211540905651035\\
90.85	0.00211276087727515\\
90.86	0.00211011205566294\\
90.87	0.00210746259305217\\
90.88	0.00210481249082627\\
90.89	0.0021021617503737\\
90.9	0.0020995103730879\\
90.91	0.00209685836036741\\
90.92	0.00209420571361576\\
90.93	0.00209155243424158\\
90.94	0.00208889852365856\\
90.95	0.00208624398328551\\
90.96	0.00208358881454632\\
90.97	0.00208093301887002\\
90.98	0.00207827659769075\\
90.99	0.00207561955244784\\
91	0.00207296188458575\\
91.01	0.00207030359555413\\
91.02	0.00206764468680785\\
91.03	0.00206498515980695\\
91.04	0.0020623250160167\\
91.05	0.00205966425690764\\
91.06	0.00205700288395552\\
91.07	0.0020543408986414\\
91.08	0.00205167830245158\\
91.09	0.00204901509687769\\
91.1	0.00204635128341665\\
91.11	0.00204368686357072\\
91.12	0.00204102183884751\\
91.13	0.00203835621075996\\
91.14	0.00203568998082642\\
91.15	0.00203302315057058\\
91.16	0.00203035572152157\\
91.17	0.00202768769521393\\
91.18	0.00202501907318763\\
91.19	0.00202234985698809\\
91.2	0.00201968004816619\\
91.21	0.00201700964827829\\
91.22	0.00201433865888625\\
91.23	0.00201166708155744\\
91.24	0.00200899491786475\\
91.25	0.00200632216938662\\
91.26	0.00200364883770704\\
91.27	0.00200097492441558\\
91.28	0.00199830043110739\\
91.29	0.00199562535938324\\
91.3	0.00199294971084952\\
91.31	0.00199027348711823\\
91.32	0.00198759668980705\\
91.33	0.00198491932053933\\
91.34	0.00198224138094409\\
91.35	0.00197956287265606\\
91.36	0.00197688379731568\\
91.37	0.00197420415656915\\
91.38	0.00197152395206837\\
91.39	0.00196884318547106\\
91.4	0.0019661618584407\\
91.41	0.00196347997264657\\
91.42	0.00196079752976376\\
91.43	0.0019581145314732\\
91.44	0.00195543097946168\\
91.45	0.00195274687542184\\
91.46	0.00195006222105221\\
91.47	0.00194737701805722\\
91.48	0.00194469126814722\\
91.49	0.00194200497303849\\
91.5	0.00193931813445327\\
91.51	0.00193663075411976\\
91.52	0.00193394283377214\\
91.53	0.0019312543751506\\
91.54	0.00192856538000135\\
91.55	0.00192587585007665\\
91.56	0.00192318578713479\\
91.57	0.00192049519294015\\
91.58	0.00191780406926319\\
91.59	0.00191511241788048\\
91.6	0.00191242024057473\\
91.61	0.00190972753913477\\
91.62	0.00190703431535561\\
91.63	0.00190434057103841\\
91.64	0.00190164630799057\\
91.65	0.00189895152802567\\
91.66	0.00189625623296353\\
91.67	0.00189356042463024\\
91.68	0.00189086410485813\\
91.69	0.00188816727548583\\
91.7	0.00188546993835828\\
91.71	0.00188277209532673\\
91.72	0.0018800737482488\\
91.73	0.00187737489898844\\
91.74	0.001874675549416\\
91.75	0.00187197570140821\\
91.76	0.00186927535684824\\
91.77	0.00186657451762568\\
91.78	0.00186387318563658\\
91.79	0.00186117136278346\\
91.8	0.00185846905097533\\
91.81	0.00185576625212773\\
91.82	0.0018530629681627\\
91.83	0.00185035920100886\\
91.84	0.00184765495260138\\
91.85	0.00184495022488203\\
91.86	0.00184224501979918\\
91.87	0.00183953933930784\\
91.88	0.00183683318536965\\
91.89	0.00183412655995294\\
91.9	0.0018314194650327\\
91.91	0.00182871190259065\\
91.92	0.00182600387461523\\
91.93	0.00182329538310162\\
91.94	0.00182058643005176\\
91.95	0.00181787701747442\\
91.96	0.00181516714738512\\
91.97	0.00181245682180624\\
91.98	0.00180974604276701\\
91.99	0.00180703481230352\\
92	0.00180432313245874\\
92.01	0.00180161100528257\\
92.02	0.00179889843283183\\
92.03	0.00179618541717022\\
92.04	0.00179347196036831\\
92.05	0.0017907580645035\\
92.06	0.00178804373166012\\
92.07	0.00178532896392937\\
92.08	0.00178261376340941\\
92.09	0.00177989813220532\\
92.1	0.00177718207242918\\
92.11	0.00177446558620003\\
92.12	0.00177174867564394\\
92.13	0.00176903134289401\\
92.14	0.00176631359009039\\
92.15	0.00176359541938031\\
92.16	0.00176087683291809\\
92.17	0.00175815783286517\\
92.18	0.00175543842139012\\
92.19	0.00175271860066868\\
92.2	0.00174999837288376\\
92.21	0.00174727774022549\\
92.22	0.00174455670489121\\
92.23	0.00174183526908552\\
92.24	0.00173911343502026\\
92.25	0.0017363912049146\\
92.26	0.001733668580995\\
92.27	0.00173094556549524\\
92.28	0.00172822216065649\\
92.29	0.00172549836872728\\
92.3	0.00172277419196354\\
92.31	0.00172004963262863\\
92.32	0.00171732469299335\\
92.33	0.00171459937533599\\
92.34	0.00171187368194231\\
92.35	0.0017091476151056\\
92.36	0.00170642117712667\\
92.37	0.00170369437031391\\
92.38	0.0017009671969833\\
92.39	0.0016982396594584\\
92.4	0.00169551176007044\\
92.41	0.00169278350115826\\
92.42	0.00169005488506843\\
92.43	0.00168732591415519\\
92.44	0.0016845965907805\\
92.45	0.00168186691731411\\
92.46	0.0016791368961335\\
92.47	0.00167640652962399\\
92.48	0.00167367582017869\\
92.49	0.00167094477019858\\
92.5	0.0016682133820925\\
92.51	0.00166548165827719\\
92.52	0.00166274960117732\\
92.53	0.00166001721322552\\
92.54	0.00165728449686234\\
92.55	0.00165455145453638\\
92.56	0.00165181808870424\\
92.57	0.00164908440183059\\
92.58	0.00164635039638813\\
92.59	0.0016436160748577\\
92.6	0.00164088143972826\\
92.61	0.00163814649349689\\
92.62	0.00163541123866889\\
92.63	0.00163267567775773\\
92.64	0.00162993981328513\\
92.65	0.00162720364778104\\
92.66	0.00162446718378373\\
92.67	0.00162173042383975\\
92.68	0.001618993370504\\
92.69	0.00161625602633973\\
92.7	0.00161351839391859\\
92.71	0.00161078047582065\\
92.72	0.00160804227463441\\
92.73	0.00160530379295686\\
92.74	0.00160256503339348\\
92.75	0.00159982599855827\\
92.76	0.0015970866910738\\
92.77	0.00159434711357122\\
92.78	0.00159160726869029\\
92.79	0.00158886715907939\\
92.8	0.00158612678739561\\
92.81	0.0015833861563047\\
92.82	0.00158064526848115\\
92.83	0.00157790412660821\\
92.84	0.00157516273337792\\
92.85	0.0015724210914911\\
92.86	0.00156967920365744\\
92.87	0.00156693707259551\\
92.88	0.00156419470103276\\
92.89	0.00156145209170558\\
92.9	0.00155870924735933\\
92.91	0.00155596617074834\\
92.92	0.001553222864636\\
92.93	0.00155047933179472\\
92.94	0.001547735575006\\
92.95	0.00154499159706047\\
92.96	0.00154224740075789\\
92.97	0.00153950298890721\\
92.98	0.00153675836432658\\
92.99	0.0015340135298434\\
93	0.00153126848829432\\
93.01	0.00152852324252531\\
93.02	0.0015257777953917\\
93.03	0.00152303214975814\\
93.04	0.00152028630849872\\
93.05	0.00151754027449695\\
93.06	0.00151479405064579\\
93.07	0.00151204763984774\\
93.08	0.0015093010450148\\
93.09	0.00150655426906855\\
93.1	0.00150380731494017\\
93.11	0.00150106018557047\\
93.12	0.00149831288390993\\
93.13	0.00149556541291874\\
93.14	0.00149281777556683\\
93.15	0.00149006997483388\\
93.16	0.0014873220137094\\
93.17	0.00148457389519275\\
93.18	0.00148182562229313\\
93.19	0.00147907719802969\\
93.2	0.00147632862543152\\
93.21	0.00147357990753768\\
93.22	0.00147083104739728\\
93.23	0.00146808204806947\\
93.24	0.00146533291262349\\
93.25	0.00146258364413873\\
93.26	0.00145983424570474\\
93.27	0.00145708472042127\\
93.28	0.00145433507139834\\
93.29	0.00145158530175621\\
93.3	0.0014488354146255\\
93.31	0.00144608541314718\\
93.32	0.0014433353004726\\
93.33	0.00144058507976358\\
93.34	0.00143783475419238\\
93.35	0.0014350843269418\\
93.36	0.00143233380120518\\
93.37	0.00142958318018649\\
93.38	0.00142683246710028\\
93.39	0.00142408166517183\\
93.4	0.0014213307776371\\
93.41	0.00141857980774284\\
93.42	0.00141582875874656\\
93.43	0.00141307763391666\\
93.44	0.00141032643653238\\
93.45	0.00140757516988392\\
93.46	0.00140482383727242\\
93.47	0.00140207244201006\\
93.48	0.00139932098742005\\
93.49	0.00139656947683671\\
93.5	0.00139381791360551\\
93.51	0.00139106630108308\\
93.52	0.00138831464263731\\
93.53	0.00138556294164735\\
93.54	0.00138281120150368\\
93.55	0.00138005942560812\\
93.56	0.00137730761737395\\
93.57	0.00137455578022586\\
93.58	0.00137180391760006\\
93.59	0.00136905203294432\\
93.6	0.00136630012971801\\
93.61	0.00136354821139167\\
93.62	0.00136079628144688\\
93.63	0.00135804434337625\\
93.64	0.00135529240068351\\
93.65	0.00135254045688353\\
93.66	0.00134978851550232\\
93.67	0.00134703658007717\\
93.68	0.00134428465415661\\
93.69	0.00134153274130048\\
93.7	0.00133878084507997\\
93.71	0.00133602896907769\\
93.72	0.00133327711688766\\
93.73	0.00133052529211541\\
93.74	0.00132777349837799\\
93.75	0.00132502173930403\\
93.76	0.00132227001853379\\
93.77	0.00131951833971918\\
93.78	0.00131676670652384\\
93.79	0.00131401512262316\\
93.8	0.00131126359170436\\
93.81	0.0013085121174665\\
93.82	0.00130576070362053\\
93.83	0.00130300935388939\\
93.84	0.00130025807200799\\
93.85	0.0012975068617233\\
93.86	0.00129475572679439\\
93.87	0.0012920046709925\\
93.88	0.00128925369810102\\
93.89	0.00128650281191566\\
93.9	0.00128375201624438\\
93.91	0.00128100131490752\\
93.92	0.00127825071173781\\
93.93	0.00127550021058047\\
93.94	0.00127274981529319\\
93.95	0.00126999952974627\\
93.96	0.0012672493578226\\
93.97	0.00126449930341777\\
93.98	0.00126174937044008\\
93.99	0.00125899956281063\\
94	0.00125624988446335\\
94.01	0.00125350033934508\\
94.02	0.00125075093141561\\
94.03	0.00124800166464777\\
94.04	0.00124525254302741\\
94.05	0.00124250357055356\\
94.06	0.00123975475123841\\
94.07	0.00123700608910741\\
94.08	0.00123425758819934\\
94.09	0.00123150925256632\\
94.1	0.00122876108627392\\
94.11	0.0012260130934012\\
94.12	0.00122326527804079\\
94.13	0.00122051764429893\\
94.14	0.00121777019629553\\
94.15	0.0012150229381643\\
94.16	0.00121227587405271\\
94.17	0.00120952900812214\\
94.18	0.00120678234454792\\
94.19	0.00120403588751938\\
94.2	0.00120128964123994\\
94.21	0.00119854360992717\\
94.22	0.00119579779781287\\
94.23	0.00119305220914311\\
94.24	0.00119030684817835\\
94.25	0.00118756171919345\\
94.26	0.00118481682647779\\
94.27	0.00118207217433532\\
94.28	0.00117932776708466\\
94.29	0.00117658360905912\\
94.3	0.00117383970460685\\
94.31	0.00117109605809083\\
94.32	0.00116835267388903\\
94.33	0.00116560955639443\\
94.34	0.00116286671001511\\
94.35	0.00116012413917436\\
94.36	0.00115738184831071\\
94.37	0.00115463984187806\\
94.38	0.00115189812434571\\
94.39	0.00114915670019849\\
94.4	0.00114641557393682\\
94.41	0.00114367475007458\\
94.42	0.00114093423313898\\
94.43	0.00113819402767061\\
94.44	0.00113545413822342\\
94.45	0.00113271456936483\\
94.46	0.00112997532567568\\
94.47	0.00112723641175035\\
94.48	0.00112449783219672\\
94.49	0.00112175959163623\\
94.5	0.00111902169470394\\
94.51	0.00111628414604853\\
94.52	0.00111354695033234\\
94.53	0.00111081011223142\\
94.54	0.00110807363643554\\
94.55	0.00110533752764826\\
94.56	0.00110260179058692\\
94.57	0.00109986642998269\\
94.58	0.00109713145058066\\
94.59	0.00109439685713976\\
94.6	0.0010916626544329\\
94.61	0.00108892884724695\\
94.62	0.0010861954403828\\
94.63	0.00108346243865537\\
94.64	0.00108072984689367\\
94.65	0.0010779976699408\\
94.66	0.00107526591265404\\
94.67	0.00107253457990483\\
94.68	0.00106980367657883\\
94.69	0.00106707320757587\\
94.7	0.00106434317780998\\
94.71	0.00106161359220937\\
94.72	0.00105888445571649\\
94.73	0.00105615577328809\\
94.74	0.00105342754989522\\
94.75	0.00105069979052326\\
94.76	0.001047972500172\\
94.77	0.00104524568385565\\
94.78	0.00104251934660285\\
94.79	0.00103979349345675\\
94.8	0.00103706812947502\\
94.81	0.00103434325972991\\
94.82	0.00103161888930824\\
94.83	0.00102889502331148\\
94.84	0.00102617166685578\\
94.85	0.00102344882507198\\
94.86	0.00102072650310568\\
94.87	0.00101800470611725\\
94.88	0.00101528343928187\\
94.89	0.0010125627077896\\
94.9	0.00100984251684536\\
94.91	0.00100712287166901\\
94.92	0.00100440377749539\\
94.93	0.0010016852395743\\
94.94	0.000998967263170622\\
94.95	0.000996249853564285\\
94.96	0.000993533016050334\\
94.97	0.000990816755938971\\
94.98	0.000988101078555579\\
94.99	0.000985385989240762\\
95	0.000982671493350394\\
95.01	0.000979957596255637\\
95.02	0.000977244303342993\\
95.03	0.000974531620014343\\
95.04	0.000971819551686971\\
95.05	0.000969108103793619\\
95.06	0.00096639728178251\\
95.07	0.000963687091117394\\
95.08	0.000960977537277582\\
95.09	0.000958268625757994\\
95.1	0.00095556036206918\\
95.11	0.000952852751737374\\
95.12	0.000950145800304529\\
95.13	0.000947439513328341\\
95.14	0.00094473389638231\\
95.15	0.000942028955055766\\
95.16	0.000939324694953909\\
95.17	0.000936621121697843\\
95.18	0.000933918240924628\\
95.19	0.000931216058287306\\
95.2	0.000928514579454946\\
95.21	0.000925813810112685\\
95.22	0.00092311375596176\\
95.23	0.00092041442271955\\
95.24	0.000917715816119621\\
95.25	0.000915017941911763\\
95.26	0.000912320805862022\\
95.27	0.000909624413752747\\
95.28	0.000906928771382627\\
95.29	0.000904233884566735\\
95.3	0.000901539759136556\\
95.31	0.000898846400940045\\
95.32	0.000896153815841648\\
95.33	0.000893462009722358\\
95.34	0.000890770988479742\\
95.35	0.000888080758027987\\
95.36	0.000885391324297947\\
95.37	0.00088270269323717\\
95.38	0.00088001487080995\\
95.39	0.000877327862997357\\
95.4	0.000874641675797283\\
95.41	0.000871956315224489\\
95.42	0.000869271787310633\\
95.43	0.00086658809810432\\
95.44	0.000863905253671138\\
95.45	0.0008612232600937\\
95.46	0.00085854212347169\\
95.47	0.000855861849921896\\
95.48	0.000853182445578258\\
95.49	0.0008505039165919\\
95.5	0.000847826269131187\\
95.51	0.000845149509381747\\
95.52	0.00084247364354653\\
95.53	0.000839798677845841\\
95.54	0.000837124618517374\\
95.55	0.000834451471816275\\
95.56	0.000831779244015163\\
95.57	0.000829107941404183\\
95.58	0.000826437570291041\\
95.59	0.000823768137001054\\
95.6	0.000821099647877186\\
95.61	0.000818432109280093\\
95.62	0.000815765527588164\\
95.63	0.000813099909197562\\
95.64	0.00081043526052227\\
95.65	0.000807771587994133\\
95.66	0.000805108898062895\\
95.67	0.000802447197196247\\
95.68	0.000799786491879872\\
95.69	0.000797126788617481\\
95.7	0.000794468093930854\\
95.71	0.000791810414359899\\
95.72	0.000789153756462678\\
95.73	0.000786498126815454\\
95.74	0.00078384353201274\\
95.75	0.000781189978667338\\
95.76	0.000778537473410381\\
95.77	0.000775886022891378\\
95.78	0.00077323563377826\\
95.79	0.000770586312757419\\
95.8	0.000767938066533757\\
95.81	0.000765290901830723\\
95.82	0.00076264482539036\\
95.83	0.000759999843973357\\
95.84	0.000757355964359073\\
95.85	0.000754713193345599\\
95.86	0.000752071537749801\\
95.87	0.000749431004407351\\
95.88	0.000746791600172785\\
95.89	0.00074415333191954\\
95.9	0.000741516206539998\\
95.91	0.000738880230945538\\
95.92	0.000736245412066575\\
95.93	0.000733611756852595\\
95.94	0.000730979272272224\\
95.95	0.000728347965313243\\
95.96	0.000725717842982665\\
95.97	0.000723088912306749\\
95.98	0.000720461180331066\\
95.99	0.000717834654120534\\
96	0.000715209340759472\\
96.01	0.000712585247351627\\
96.02	0.00070996238102025\\
96.03	0.000707340748908105\\
96.04	0.00070472035817754\\
96.05	0.000702101216010531\\
96.06	0.000699483329608713\\
96.07	0.000696866706193436\\
96.08	0.000694251353005814\\
96.09	0.000691637277306754\\
96.1	0.000689024486377027\\
96.11	0.000686412987517294\\
96.12	0.00068380278804816\\
96.13	0.000681193895310213\\
96.14	0.000678586316664087\\
96.15	0.000675980059490487\\
96.16	0.000673375131190248\\
96.17	0.00067077153918438\\
96.18	0.000668169290914115\\
96.19	0.000665568393840946\\
96.2	0.000662968855446687\\
96.21	0.000660370683233501\\
96.22	0.000657773884723968\\
96.23	0.000655178467461116\\
96.24	0.000652584439008477\\
96.25	0.000649991806950122\\
96.26	0.000647400578890726\\
96.27	0.0006448107624556\\
96.28	0.000642222365290745\\
96.29	0.000639635395062895\\
96.3	0.000637049859459568\\
96.31	0.00063446576618912\\
96.32	0.000631883122980769\\
96.33	0.000629301937584674\\
96.34	0.000626722217771958\\
96.35	0.000624143971334762\\
96.36	0.000621567206086302\\
96.37	0.000618991929860912\\
96.38	0.000616418150514076\\
96.39	0.000613845875922501\\
96.4	0.000611275113984153\\
96.41	0.000608705872618302\\
96.42	0.000606138159765573\\
96.43	0.000603571983388\\
96.44	0.000601007351469063\\
96.45	0.000598444272013747\\
96.46	0.000595882753048581\\
96.47	0.000593322802621693\\
96.48	0.000590764428802862\\
96.49	0.000588207639683555\\
96.5	0.00058565244337698\\
96.51	0.000583098848018141\\
96.52	0.00058054686176388\\
96.53	0.00057799649279293\\
96.54	0.000575447749305961\\
96.55	0.00057290063952563\\
96.56	0.000570355171696625\\
96.57	0.000567811354085727\\
96.58	0.000565269194981845\\
96.59	0.000562728702696073\\
96.6	0.000560189885561739\\
96.61	0.000557652751934449\\
96.62	0.000555117310192142\\
96.63	0.000552583568735139\\
96.64	0.000550051535986187\\
96.65	0.000547521220390515\\
96.66	0.000544992630415884\\
96.67	0.000542465774552628\\
96.68	0.00053994066131371\\
96.69	0.000537417299234776\\
96.7	0.000534895696874198\\
96.71	0.000532375862813123\\
96.72	0.000529857805655527\\
96.73	0.000527341534028265\\
96.74	0.00052482705658112\\
96.75	0.000522314381986855\\
96.76	0.000519803518941254\\
96.77	0.000517294476163187\\
96.78	0.00051478726239465\\
96.79	0.000512281886400819\\
96.8	0.0005097783569701\\
96.81	0.000507276682914179\\
96.82	0.00050477687306808\\
96.83	0.000502278936290202\\
96.84	0.000499782881462378\\
96.85	0.000497288717489929\\
96.86	0.000494796453301713\\
96.87	0.000492306097850169\\
96.88	0.000489817660111372\\
96.89	0.000487331149085096\\
96.9	0.000484846573794842\\
96.91	0.000482363943287909\\
96.92	0.000479883266635438\\
96.93	0.000477404552932464\\
96.94	0.000474927811297963\\
96.95	0.000472453050874911\\
96.96	0.00046998028083033\\
96.97	0.000467509510355341\\
96.98	0.000465040748665222\\
96.99	0.000462574004999444\\
97	0.00046010928862174\\
97.01	0.000457646608820146\\
97.02	0.000455185974907058\\
97.03	0.000452727396219278\\
97.04	0.000450270882118072\\
97.05	0.000447816441989221\\
97.06	0.000445364085243073\\
97.07	0.000442913821314585\\
97.08	0.000440465659663393\\
97.09	0.000438019609773853\\
97.1	0.000435575681155089\\
97.11	0.000433133883341057\\
97.12	0.000430694225890593\\
97.13	0.000428256718387459\\
97.14	0.000425821370440399\\
97.15	0.000423388191683201\\
97.16	0.000420957191774733\\
97.17	0.000418528380399004\\
97.18	0.000416101767265219\\
97.19	0.000413677362107824\\
97.2	0.000411255174686572\\
97.21	0.000408835214786556\\
97.22	0.000406417492218277\\
97.23	0.000404002016817695\\
97.24	0.000401588798446271\\
97.25	0.000399177846991035\\
97.26	0.000396769172364624\\
97.27	0.000394362784505352\\
97.28	0.000391958693377241\\
97.29	0.000389556908970096\\
97.3	0.000387157441299543\\
97.31	0.000384760300407086\\
97.32	0.000382365496360162\\
97.33	0.000379973039252194\\
97.34	0.000377582939202642\\
97.35	0.000375195206357055\\
97.36	0.00037280985088713\\
97.37	0.000370426882990755\\
97.38	0.000368046312892077\\
97.39	0.00036566815084154\\
97.4	0.000363292407115945\\
97.41	0.000360919092018503\\
97.42	0.000358548215878895\\
97.43	0.000356179789053309\\
97.44	0.000353813821924509\\
97.45	0.000351450324901881\\
97.46	0.000349089308421487\\
97.47	0.000346730782946118\\
97.48	0.000344374758965349\\
97.49	0.000342021246995595\\
97.5	0.000339670257580157\\
97.51	0.000337321801289281\\
97.52	0.000334975888720212\\
97.53	0.000332632530497244\\
97.54	0.000330291737271775\\
97.55	0.00032795351972236\\
97.56	0.000325617888554766\\
97.57	0.000323284854502028\\
97.58	0.000320954428324496\\
97.59	0.00031862662080989\\
97.6	0.000316301442773361\\
97.61	0.000313978905057534\\
97.62	0.00031165901853257\\
97.63	0.000309341794096212\\
97.64	0.000307027242673851\\
97.65	0.000304715375218562\\
97.66	0.000302406202711172\\
97.67	0.00030009973616031\\
97.68	0.000297795986602458\\
97.69	0.000295494965102002\\
97.7	0.000293196682751296\\
97.71	0.000290901150670705\\
97.72	0.000288608380008664\\
97.73	0.000286318381941731\\
97.74	0.000284031167674642\\
97.75	0.000281746748440359\\
97.76	0.000279465135500128\\
97.77	0.000277186340143534\\
97.78	0.00027491037368855\\
97.79	0.000272637247481593\\
97.8	0.000270366972897583\\
97.81	0.000268099561339984\\
97.82	0.000265835024240869\\
97.83	0.000263573373060967\\
97.84	0.000261314619289719\\
97.85	0.000259058774445335\\
97.86	0.000256805850074837\\
97.87	0.000254555857754126\\
97.88	0.000252308809088024\\
97.89	0.000250064715710331\\
97.9	0.000247823589283885\\
97.91	0.000245585441500601\\
97.92	0.000243350284081543\\
97.93	0.000241118128777033\\
97.94	0.000238888987366764\\
97.95	0.000236662871659852\\
97.96	0.00023443979349488\\
97.97	0.000232219764739949\\
97.98	0.00023000279729273\\
97.99	0.000227788903080513\\
98	0.000225578094060265\\
98.01	0.00022337038221871\\
98.02	0.000221165779572775\\
98.03	0.000218964298170168\\
98.04	0.000216765950089454\\
98.05	0.000214570747440119\\
98.06	0.000212378702362634\\
98.07	0.000210189827028522\\
98.08	0.000208004133640411\\
98.09	0.000205821634432111\\
98.1	0.000203642341481064\\
98.11	0.000201466266657718\\
98.12	0.000199293421865464\\
98.13	0.000197123819040758\\
98.14	0.000194957470153214\\
98.15	0.000192794387205738\\
98.16	0.000190634582207314\\
98.17	0.000188478067045202\\
98.18	0.000186324852593203\\
98.19	0.00018417494775237\\
98.2	0.000182028361422695\\
98.21	0.000179885102502837\\
98.22	0.000177745179889843\\
98.23	0.000175608602478861\\
98.24	0.000173475379162853\\
98.25	0.0001713455188323\\
98.26	0.0001692190303749\\
98.27	0.000167095922675274\\
98.28	0.000164976204614647\\
98.29	0.000162859885070541\\
98.3	0.000160746972916457\\
98.31	0.000158637477021552\\
98.32	0.000156531406250314\\
98.33	0.000154428889410309\\
98.34	0.000152329946462073\\
98.35	0.000150235990568359\\
98.36	0.000148147035042449\\
98.37	0.000146063093212499\\
98.38	0.000143984178421541\\
98.39	0.000141910304027489\\
98.4	0.000139841771773151\\
98.41	0.00013777991434433\\
98.42	0.000135733424685895\\
98.43	0.000133702413337858\\
98.44	0.000131686991825733\\
98.45	0.000129687272671966\\
98.46	0.00012770336940757\\
98.47	0.000125735396583913\\
98.48	0.000123783469784707\\
98.49	0.000121847705638169\\
98.5	0.000119928221829378\\
98.51	0.000118025137112818\\
98.52	0.000116138571325128\\
98.53	0.000114268645398031\\
98.54	0.000112415481371494\\
98.55	0.000110579202407077\\
98.56	0.000108759932629461\\
98.57	0.000106957796004338\\
98.58	0.000105172917629471\\
98.59	0.000103405423748352\\
98.6	0.000101655441764102\\
98.61	9.99231002535763e-05\\
98.62	9.82085289817029e-05\\
98.63	9.6511848141816e-05\\
98.64	9.48331057032663e-05\\
98.65	9.31724359905829e-05\\
98.66	9.15299745840991e-05\\
98.67	8.99058583355688e-05\\
98.68	8.83002253840651e-05\\
98.69	8.67132151721045e-05\\
98.7	8.51449694648683e-05\\
98.71	8.35956315950563e-05\\
98.72	8.20653462532666e-05\\
98.73	8.05542595052854e-05\\
98.74	7.90619610721213e-05\\
98.75	7.75879928146608e-05\\
98.76	7.61324997723516e-05\\
98.77	7.4695628389158e-05\\
98.78	7.32774004500343e-05\\
98.79	7.18773164142782e-05\\
98.8	7.04879875739993e-05\\
98.81	6.91094829359008e-05\\
98.82	6.77418720203167e-05\\
98.83	6.63852248652241e-05\\
98.84	6.50396120302644e-05\\
98.85	6.37051046008009e-05\\
98.86	6.23817741919903e-05\\
98.87	6.10696929528778e-05\\
98.88	5.97689335705091e-05\\
98.89	5.8479569274076e-05\\
98.9	5.72016738390625e-05\\
98.91	5.59353215914375e-05\\
98.92	5.46805884606635e-05\\
98.93	5.34375510873788e-05\\
98.94	5.22062867035281e-05\\
98.95	5.09868731372199e-05\\
98.96	4.97793888175994e-05\\
98.97	4.85839127797802e-05\\
98.98	4.74005246697794e-05\\
98.99	4.62293047495052e-05\\
99	4.50703339017599e-05\\
99.01	4.39236936352807e-05\\
99.02	4.27894660898227e-05\\
99.03	4.16677340412432e-05\\
99.04	4.05585809066453e-05\\
99.05	3.94620907495404e-05\\
99.06	3.83783482850279e-05\\
99.07	3.73074388850345e-05\\
99.08	3.62494485835439e-05\\
99.09	3.52044640818897e-05\\
99.1	3.41725727540567e-05\\
99.11	3.31538626520134e-05\\
99.12	3.21484225110687e-05\\
99.13	3.11563417552743e-05\\
99.14	3.01777105028277e-05\\
99.15	2.92126195715094e-05\\
99.16	2.82611604841623e-05\\
99.17	2.73234254741638e-05\\
99.18	2.63995074909502e-05\\
99.19	2.54895002014359e-05\\
99.2	2.45934979939289e-05\\
99.21	2.37115959836436e-05\\
99.22	2.28438900182325e-05\\
99.23	2.19904766833416e-05\\
99.24	2.11514533081832e-05\\
99.25	2.03269179711221e-05\\
99.26	1.95169695052736e-05\\
99.27	1.87217075041362e-05\\
99.28	1.79412323272118e-05\\
99.29	1.71756451056593e-05\\
99.3	1.64250477479484e-05\\
99.31	1.56895429455264e-05\\
99.32	1.49692341784944e-05\\
99.33	1.42642257212871e-05\\
99.34	1.35746226483605e-05\\
99.35	1.29005308398839e-05\\
99.36	1.22420569874312e-05\\
99.37	1.15993085996729e-05\\
99.38	1.09723940080689e-05\\
99.39	1.03614226602163e-05\\
99.4	9.76651714497574e-06\\
99.41	9.18780102560947e-06\\
99.42	8.62539884664143e-06\\
99.43	8.07943614066463e-06\\
99.44	7.55003943518281e-06\\
99.45	7.03733625940541e-06\\
99.46	6.5414551510528e-06\\
99.47	6.06252566313741e-06\\
99.48	5.60067837072735e-06\\
99.49	5.15604487768065e-06\\
99.5	4.72875782336381e-06\\
99.51	4.3189508893253e-06\\
99.52	3.92675880595342e-06\\
99.53	3.5523173590752e-06\\
99.54	3.19576339652752e-06\\
99.55	2.85723483467448e-06\\
99.56	2.53687066486923e-06\\
99.57	2.23481095987417e-06\\
99.58	1.95119688019101e-06\\
99.59	1.68617068034213e-06\\
99.6	1.4398757150879e-06\\
99.61	1.2124564455207e-06\\
99.62	1.00405963195105e-06\\
99.63	8.14835682361875e-07\\
99.64	6.449361741271e-07\\
99.65	4.94513860235801e-07\\
99.66	3.63722675407116e-07\\
99.67	2.52717742073305e-07\\
99.68	1.61655376246586e-07\\
99.69	9.069309325066e-08\\
99.7	3.99896132857042e-08\\
99.71	9.70486685111793e-09\\
99.72	0\\
99.73	0\\
99.74	0\\
99.75	0\\
99.76	0\\
99.77	0\\
99.78	0\\
99.79	0\\
99.8	0\\
99.81	0\\
99.82	0\\
99.83	0\\
99.84	0\\
99.85	0\\
99.86	0\\
99.87	0\\
99.88	0\\
99.89	0\\
99.9	0\\
99.91	0\\
99.92	0\\
99.93	0\\
99.94	0\\
99.95	0\\
99.96	0\\
99.97	0\\
99.98	0\\
99.99	0\\
100	0\\
};
\addlegendentry{$q=-2$};

\addplot [color=blue,dashed,forget plot]
  table[row sep=crcr]{%
0.01	0.01\\
0.02	0.01\\
0.03	0.01\\
0.04	0.01\\
0.05	0.01\\
0.06	0.01\\
0.07	0.01\\
0.08	0.01\\
0.09	0.01\\
0.1	0.01\\
0.11	0.01\\
0.12	0.01\\
0.13	0.01\\
0.14	0.01\\
0.15	0.01\\
0.16	0.01\\
0.17	0.01\\
0.18	0.01\\
0.19	0.01\\
0.2	0.01\\
0.21	0.01\\
0.22	0.01\\
0.23	0.01\\
0.24	0.01\\
0.25	0.01\\
0.26	0.01\\
0.27	0.01\\
0.28	0.01\\
0.29	0.01\\
0.3	0.01\\
0.31	0.01\\
0.32	0.01\\
0.33	0.01\\
0.34	0.01\\
0.35	0.01\\
0.36	0.01\\
0.37	0.01\\
0.38	0.01\\
0.39	0.01\\
0.4	0.01\\
0.41	0.01\\
0.42	0.01\\
0.43	0.01\\
0.44	0.01\\
0.45	0.01\\
0.46	0.01\\
0.47	0.01\\
0.48	0.01\\
0.49	0.01\\
0.5	0.01\\
0.51	0.01\\
0.52	0.01\\
0.53	0.01\\
0.54	0.01\\
0.55	0.01\\
0.56	0.01\\
0.57	0.01\\
0.58	0.01\\
0.59	0.01\\
0.6	0.01\\
0.61	0.01\\
0.62	0.01\\
0.63	0.01\\
0.64	0.01\\
0.65	0.01\\
0.66	0.01\\
0.67	0.01\\
0.68	0.01\\
0.69	0.01\\
0.7	0.01\\
0.71	0.01\\
0.72	0.01\\
0.73	0.01\\
0.74	0.01\\
0.75	0.01\\
0.76	0.01\\
0.77	0.01\\
0.78	0.01\\
0.79	0.01\\
0.8	0.01\\
0.81	0.01\\
0.82	0.01\\
0.83	0.01\\
0.84	0.01\\
0.85	0.01\\
0.86	0.01\\
0.87	0.01\\
0.88	0.01\\
0.89	0.01\\
0.9	0.01\\
0.91	0.01\\
0.92	0.01\\
0.93	0.01\\
0.94	0.01\\
0.95	0.01\\
0.96	0.01\\
0.97	0.01\\
0.98	0.01\\
0.99	0.01\\
1	0.01\\
1.01	0.01\\
1.02	0.01\\
1.03	0.01\\
1.04	0.01\\
1.05	0.01\\
1.06	0.01\\
1.07	0.01\\
1.08	0.01\\
1.09	0.01\\
1.1	0.01\\
1.11	0.01\\
1.12	0.01\\
1.13	0.01\\
1.14	0.01\\
1.15	0.01\\
1.16	0.01\\
1.17	0.01\\
1.18	0.01\\
1.19	0.01\\
1.2	0.01\\
1.21	0.01\\
1.22	0.01\\
1.23	0.01\\
1.24	0.01\\
1.25	0.01\\
1.26	0.01\\
1.27	0.01\\
1.28	0.01\\
1.29	0.01\\
1.3	0.01\\
1.31	0.01\\
1.32	0.01\\
1.33	0.01\\
1.34	0.01\\
1.35	0.01\\
1.36	0.01\\
1.37	0.01\\
1.38	0.01\\
1.39	0.01\\
1.4	0.01\\
1.41	0.01\\
1.42	0.01\\
1.43	0.01\\
1.44	0.01\\
1.45	0.01\\
1.46	0.01\\
1.47	0.01\\
1.48	0.01\\
1.49	0.01\\
1.5	0.01\\
1.51	0.01\\
1.52	0.01\\
1.53	0.01\\
1.54	0.01\\
1.55	0.01\\
1.56	0.01\\
1.57	0.01\\
1.58	0.01\\
1.59	0.01\\
1.6	0.01\\
1.61	0.01\\
1.62	0.01\\
1.63	0.01\\
1.64	0.01\\
1.65	0.01\\
1.66	0.01\\
1.67	0.01\\
1.68	0.01\\
1.69	0.01\\
1.7	0.01\\
1.71	0.01\\
1.72	0.01\\
1.73	0.01\\
1.74	0.01\\
1.75	0.01\\
1.76	0.01\\
1.77	0.01\\
1.78	0.01\\
1.79	0.01\\
1.8	0.01\\
1.81	0.01\\
1.82	0.01\\
1.83	0.01\\
1.84	0.01\\
1.85	0.01\\
1.86	0.01\\
1.87	0.01\\
1.88	0.01\\
1.89	0.01\\
1.9	0.01\\
1.91	0.01\\
1.92	0.01\\
1.93	0.01\\
1.94	0.01\\
1.95	0.01\\
1.96	0.01\\
1.97	0.01\\
1.98	0.01\\
1.99	0.01\\
2	0.01\\
2.01	0.01\\
2.02	0.01\\
2.03	0.01\\
2.04	0.01\\
2.05	0.01\\
2.06	0.01\\
2.07	0.01\\
2.08	0.01\\
2.09	0.01\\
2.1	0.01\\
2.11	0.01\\
2.12	0.01\\
2.13	0.01\\
2.14	0.01\\
2.15	0.01\\
2.16	0.01\\
2.17	0.01\\
2.18	0.01\\
2.19	0.01\\
2.2	0.01\\
2.21	0.01\\
2.22	0.01\\
2.23	0.01\\
2.24	0.01\\
2.25	0.01\\
2.26	0.01\\
2.27	0.01\\
2.28	0.01\\
2.29	0.01\\
2.3	0.01\\
2.31	0.01\\
2.32	0.01\\
2.33	0.01\\
2.34	0.01\\
2.35	0.01\\
2.36	0.01\\
2.37	0.01\\
2.38	0.01\\
2.39	0.01\\
2.4	0.01\\
2.41	0.01\\
2.42	0.01\\
2.43	0.01\\
2.44	0.01\\
2.45	0.01\\
2.46	0.01\\
2.47	0.01\\
2.48	0.01\\
2.49	0.01\\
2.5	0.01\\
2.51	0.01\\
2.52	0.01\\
2.53	0.01\\
2.54	0.01\\
2.55	0.01\\
2.56	0.01\\
2.57	0.01\\
2.58	0.01\\
2.59	0.01\\
2.6	0.01\\
2.61	0.01\\
2.62	0.01\\
2.63	0.01\\
2.64	0.01\\
2.65	0.01\\
2.66	0.01\\
2.67	0.01\\
2.68	0.01\\
2.69	0.01\\
2.7	0.01\\
2.71	0.01\\
2.72	0.01\\
2.73	0.01\\
2.74	0.01\\
2.75	0.01\\
2.76	0.01\\
2.77	0.01\\
2.78	0.01\\
2.79	0.01\\
2.8	0.01\\
2.81	0.01\\
2.82	0.01\\
2.83	0.01\\
2.84	0.01\\
2.85	0.01\\
2.86	0.01\\
2.87	0.01\\
2.88	0.01\\
2.89	0.01\\
2.9	0.01\\
2.91	0.01\\
2.92	0.01\\
2.93	0.01\\
2.94	0.01\\
2.95	0.01\\
2.96	0.01\\
2.97	0.01\\
2.98	0.01\\
2.99	0.01\\
3	0.01\\
3.01	0.01\\
3.02	0.01\\
3.03	0.01\\
3.04	0.01\\
3.05	0.01\\
3.06	0.01\\
3.07	0.01\\
3.08	0.01\\
3.09	0.01\\
3.1	0.01\\
3.11	0.01\\
3.12	0.01\\
3.13	0.01\\
3.14	0.01\\
3.15	0.01\\
3.16	0.01\\
3.17	0.01\\
3.18	0.01\\
3.19	0.01\\
3.2	0.01\\
3.21	0.01\\
3.22	0.01\\
3.23	0.01\\
3.24	0.01\\
3.25	0.01\\
3.26	0.01\\
3.27	0.01\\
3.28	0.01\\
3.29	0.01\\
3.3	0.01\\
3.31	0.01\\
3.32	0.01\\
3.33	0.01\\
3.34	0.01\\
3.35	0.01\\
3.36	0.01\\
3.37	0.01\\
3.38	0.01\\
3.39	0.01\\
3.4	0.01\\
3.41	0.01\\
3.42	0.01\\
3.43	0.01\\
3.44	0.01\\
3.45	0.01\\
3.46	0.01\\
3.47	0.01\\
3.48	0.01\\
3.49	0.01\\
3.5	0.01\\
3.51	0.01\\
3.52	0.01\\
3.53	0.01\\
3.54	0.01\\
3.55	0.01\\
3.56	0.01\\
3.57	0.01\\
3.58	0.01\\
3.59	0.01\\
3.6	0.01\\
3.61	0.01\\
3.62	0.01\\
3.63	0.01\\
3.64	0.01\\
3.65	0.01\\
3.66	0.01\\
3.67	0.01\\
3.68	0.01\\
3.69	0.01\\
3.7	0.01\\
3.71	0.01\\
3.72	0.01\\
3.73	0.01\\
3.74	0.01\\
3.75	0.01\\
3.76	0.01\\
3.77	0.01\\
3.78	0.01\\
3.79	0.01\\
3.8	0.01\\
3.81	0.01\\
3.82	0.01\\
3.83	0.01\\
3.84	0.01\\
3.85	0.01\\
3.86	0.01\\
3.87	0.01\\
3.88	0.01\\
3.89	0.01\\
3.9	0.01\\
3.91	0.01\\
3.92	0.01\\
3.93	0.01\\
3.94	0.01\\
3.95	0.01\\
3.96	0.01\\
3.97	0.01\\
3.98	0.01\\
3.99	0.01\\
4	0.01\\
4.01	0.01\\
4.02	0.01\\
4.03	0.01\\
4.04	0.01\\
4.05	0.01\\
4.06	0.01\\
4.07	0.01\\
4.08	0.01\\
4.09	0.01\\
4.1	0.01\\
4.11	0.01\\
4.12	0.01\\
4.13	0.01\\
4.14	0.01\\
4.15	0.01\\
4.16	0.01\\
4.17	0.01\\
4.18	0.01\\
4.19	0.01\\
4.2	0.01\\
4.21	0.01\\
4.22	0.01\\
4.23	0.01\\
4.24	0.01\\
4.25	0.01\\
4.26	0.01\\
4.27	0.01\\
4.28	0.01\\
4.29	0.01\\
4.3	0.01\\
4.31	0.01\\
4.32	0.01\\
4.33	0.01\\
4.34	0.01\\
4.35	0.01\\
4.36	0.01\\
4.37	0.01\\
4.38	0.01\\
4.39	0.01\\
4.4	0.01\\
4.41	0.01\\
4.42	0.01\\
4.43	0.01\\
4.44	0.01\\
4.45	0.01\\
4.46	0.01\\
4.47	0.01\\
4.48	0.01\\
4.49	0.01\\
4.5	0.01\\
4.51	0.01\\
4.52	0.01\\
4.53	0.01\\
4.54	0.01\\
4.55	0.01\\
4.56	0.01\\
4.57	0.01\\
4.58	0.01\\
4.59	0.01\\
4.6	0.01\\
4.61	0.01\\
4.62	0.01\\
4.63	0.01\\
4.64	0.01\\
4.65	0.01\\
4.66	0.01\\
4.67	0.01\\
4.68	0.01\\
4.69	0.01\\
4.7	0.01\\
4.71	0.01\\
4.72	0.01\\
4.73	0.01\\
4.74	0.01\\
4.75	0.01\\
4.76	0.01\\
4.77	0.01\\
4.78	0.01\\
4.79	0.01\\
4.8	0.01\\
4.81	0.01\\
4.82	0.01\\
4.83	0.01\\
4.84	0.01\\
4.85	0.01\\
4.86	0.01\\
4.87	0.01\\
4.88	0.01\\
4.89	0.01\\
4.9	0.01\\
4.91	0.01\\
4.92	0.01\\
4.93	0.01\\
4.94	0.01\\
4.95	0.01\\
4.96	0.01\\
4.97	0.01\\
4.98	0.01\\
4.99	0.01\\
5	0.01\\
5.01	0.01\\
5.02	0.01\\
5.03	0.01\\
5.04	0.01\\
5.05	0.01\\
5.06	0.01\\
5.07	0.01\\
5.08	0.01\\
5.09	0.01\\
5.1	0.01\\
5.11	0.01\\
5.12	0.01\\
5.13	0.01\\
5.14	0.01\\
5.15	0.01\\
5.16	0.01\\
5.17	0.01\\
5.18	0.01\\
5.19	0.01\\
5.2	0.01\\
5.21	0.01\\
5.22	0.01\\
5.23	0.01\\
5.24	0.01\\
5.25	0.01\\
5.26	0.01\\
5.27	0.01\\
5.28	0.01\\
5.29	0.01\\
5.3	0.01\\
5.31	0.01\\
5.32	0.01\\
5.33	0.01\\
5.34	0.01\\
5.35	0.01\\
5.36	0.01\\
5.37	0.01\\
5.38	0.01\\
5.39	0.01\\
5.4	0.01\\
5.41	0.01\\
5.42	0.01\\
5.43	0.01\\
5.44	0.01\\
5.45	0.01\\
5.46	0.01\\
5.47	0.01\\
5.48	0.01\\
5.49	0.01\\
5.5	0.01\\
5.51	0.01\\
5.52	0.01\\
5.53	0.01\\
5.54	0.01\\
5.55	0.01\\
5.56	0.01\\
5.57	0.01\\
5.58	0.01\\
5.59	0.01\\
5.6	0.01\\
5.61	0.01\\
5.62	0.01\\
5.63	0.01\\
5.64	0.01\\
5.65	0.01\\
5.66	0.01\\
5.67	0.01\\
5.68	0.01\\
5.69	0.01\\
5.7	0.01\\
5.71	0.01\\
5.72	0.01\\
5.73	0.01\\
5.74	0.01\\
5.75	0.01\\
5.76	0.01\\
5.77	0.01\\
5.78	0.01\\
5.79	0.01\\
5.8	0.01\\
5.81	0.01\\
5.82	0.01\\
5.83	0.01\\
5.84	0.01\\
5.85	0.01\\
5.86	0.01\\
5.87	0.01\\
5.88	0.01\\
5.89	0.01\\
5.9	0.01\\
5.91	0.01\\
5.92	0.01\\
5.93	0.01\\
5.94	0.01\\
5.95	0.01\\
5.96	0.01\\
5.97	0.01\\
5.98	0.01\\
5.99	0.01\\
6	0.01\\
6.01	0.01\\
6.02	0.01\\
6.03	0.01\\
6.04	0.01\\
6.05	0.01\\
6.06	0.01\\
6.07	0.01\\
6.08	0.01\\
6.09	0.01\\
6.1	0.01\\
6.11	0.01\\
6.12	0.01\\
6.13	0.01\\
6.14	0.01\\
6.15	0.01\\
6.16	0.01\\
6.17	0.01\\
6.18	0.01\\
6.19	0.01\\
6.2	0.01\\
6.21	0.01\\
6.22	0.01\\
6.23	0.01\\
6.24	0.01\\
6.25	0.01\\
6.26	0.01\\
6.27	0.01\\
6.28	0.01\\
6.29	0.01\\
6.3	0.01\\
6.31	0.01\\
6.32	0.01\\
6.33	0.01\\
6.34	0.01\\
6.35	0.01\\
6.36	0.01\\
6.37	0.01\\
6.38	0.01\\
6.39	0.01\\
6.4	0.01\\
6.41	0.01\\
6.42	0.01\\
6.43	0.01\\
6.44	0.01\\
6.45	0.01\\
6.46	0.01\\
6.47	0.01\\
6.48	0.01\\
6.49	0.01\\
6.5	0.01\\
6.51	0.01\\
6.52	0.01\\
6.53	0.01\\
6.54	0.01\\
6.55	0.01\\
6.56	0.01\\
6.57	0.01\\
6.58	0.01\\
6.59	0.01\\
6.6	0.01\\
6.61	0.01\\
6.62	0.01\\
6.63	0.01\\
6.64	0.01\\
6.65	0.01\\
6.66	0.01\\
6.67	0.01\\
6.68	0.01\\
6.69	0.01\\
6.7	0.01\\
6.71	0.01\\
6.72	0.01\\
6.73	0.01\\
6.74	0.01\\
6.75	0.01\\
6.76	0.01\\
6.77	0.01\\
6.78	0.01\\
6.79	0.01\\
6.8	0.01\\
6.81	0.01\\
6.82	0.01\\
6.83	0.01\\
6.84	0.01\\
6.85	0.01\\
6.86	0.01\\
6.87	0.01\\
6.88	0.01\\
6.89	0.01\\
6.9	0.01\\
6.91	0.01\\
6.92	0.01\\
6.93	0.01\\
6.94	0.01\\
6.95	0.01\\
6.96	0.01\\
6.97	0.01\\
6.98	0.01\\
6.99	0.01\\
7	0.01\\
7.01	0.01\\
7.02	0.01\\
7.03	0.01\\
7.04	0.01\\
7.05	0.01\\
7.06	0.01\\
7.07	0.01\\
7.08	0.01\\
7.09	0.01\\
7.1	0.01\\
7.11	0.01\\
7.12	0.01\\
7.13	0.01\\
7.14	0.01\\
7.15	0.01\\
7.16	0.01\\
7.17	0.01\\
7.18	0.01\\
7.19	0.01\\
7.2	0.01\\
7.21	0.01\\
7.22	0.01\\
7.23	0.01\\
7.24	0.01\\
7.25	0.01\\
7.26	0.01\\
7.27	0.01\\
7.28	0.01\\
7.29	0.01\\
7.3	0.01\\
7.31	0.01\\
7.32	0.01\\
7.33	0.01\\
7.34	0.01\\
7.35	0.01\\
7.36	0.01\\
7.37	0.01\\
7.38	0.01\\
7.39	0.01\\
7.4	0.01\\
7.41	0.01\\
7.42	0.01\\
7.43	0.01\\
7.44	0.01\\
7.45	0.01\\
7.46	0.01\\
7.47	0.01\\
7.48	0.01\\
7.49	0.01\\
7.5	0.01\\
7.51	0.01\\
7.52	0.01\\
7.53	0.01\\
7.54	0.01\\
7.55	0.01\\
7.56	0.01\\
7.57	0.01\\
7.58	0.01\\
7.59	0.01\\
7.6	0.01\\
7.61	0.01\\
7.62	0.01\\
7.63	0.01\\
7.64	0.01\\
7.65	0.01\\
7.66	0.01\\
7.67	0.01\\
7.68	0.01\\
7.69	0.01\\
7.7	0.01\\
7.71	0.01\\
7.72	0.01\\
7.73	0.01\\
7.74	0.01\\
7.75	0.01\\
7.76	0.01\\
7.77	0.01\\
7.78	0.01\\
7.79	0.01\\
7.8	0.01\\
7.81	0.01\\
7.82	0.01\\
7.83	0.01\\
7.84	0.01\\
7.85	0.01\\
7.86	0.01\\
7.87	0.01\\
7.88	0.01\\
7.89	0.01\\
7.9	0.01\\
7.91	0.01\\
7.92	0.01\\
7.93	0.01\\
7.94	0.01\\
7.95	0.01\\
7.96	0.01\\
7.97	0.01\\
7.98	0.01\\
7.99	0.01\\
8	0.01\\
8.01	0.01\\
8.02	0.01\\
8.03	0.01\\
8.04	0.01\\
8.05	0.01\\
8.06	0.01\\
8.07	0.01\\
8.08	0.01\\
8.09	0.01\\
8.1	0.01\\
8.11	0.01\\
8.12	0.01\\
8.13	0.01\\
8.14	0.01\\
8.15	0.01\\
8.16	0.01\\
8.17	0.01\\
8.18	0.01\\
8.19	0.01\\
8.2	0.01\\
8.21	0.01\\
8.22	0.01\\
8.23	0.01\\
8.24	0.01\\
8.25	0.01\\
8.26	0.01\\
8.27	0.01\\
8.28	0.01\\
8.29	0.01\\
8.3	0.01\\
8.31	0.01\\
8.32	0.01\\
8.33	0.01\\
8.34	0.01\\
8.35	0.01\\
8.36	0.01\\
8.37	0.01\\
8.38	0.01\\
8.39	0.01\\
8.4	0.01\\
8.41	0.01\\
8.42	0.01\\
8.43	0.01\\
8.44	0.01\\
8.45	0.01\\
8.46	0.01\\
8.47	0.01\\
8.48	0.01\\
8.49	0.01\\
8.5	0.01\\
8.51	0.01\\
8.52	0.01\\
8.53	0.01\\
8.54	0.01\\
8.55	0.01\\
8.56	0.01\\
8.57	0.01\\
8.58	0.01\\
8.59	0.01\\
8.6	0.01\\
8.61	0.01\\
8.62	0.01\\
8.63	0.01\\
8.64	0.01\\
8.65	0.01\\
8.66	0.01\\
8.67	0.01\\
8.68	0.01\\
8.69	0.01\\
8.7	0.01\\
8.71	0.01\\
8.72	0.01\\
8.73	0.01\\
8.74	0.01\\
8.75	0.01\\
8.76	0.01\\
8.77	0.01\\
8.78	0.01\\
8.79	0.01\\
8.8	0.01\\
8.81	0.01\\
8.82	0.01\\
8.83	0.01\\
8.84	0.01\\
8.85	0.01\\
8.86	0.01\\
8.87	0.01\\
8.88	0.01\\
8.89	0.01\\
8.9	0.01\\
8.91	0.01\\
8.92	0.01\\
8.93	0.01\\
8.94	0.01\\
8.95	0.01\\
8.96	0.01\\
8.97	0.01\\
8.98	0.01\\
8.99	0.01\\
9	0.01\\
9.01	0.01\\
9.02	0.01\\
9.03	0.01\\
9.04	0.01\\
9.05	0.01\\
9.06	0.01\\
9.07	0.01\\
9.08	0.01\\
9.09	0.01\\
9.1	0.01\\
9.11	0.01\\
9.12	0.01\\
9.13	0.01\\
9.14	0.01\\
9.15	0.01\\
9.16	0.01\\
9.17	0.01\\
9.18	0.01\\
9.19	0.01\\
9.2	0.01\\
9.21	0.01\\
9.22	0.01\\
9.23	0.01\\
9.24	0.01\\
9.25	0.01\\
9.26	0.01\\
9.27	0.01\\
9.28	0.01\\
9.29	0.01\\
9.3	0.01\\
9.31	0.01\\
9.32	0.01\\
9.33	0.01\\
9.34	0.01\\
9.35	0.01\\
9.36	0.01\\
9.37	0.01\\
9.38	0.01\\
9.39	0.01\\
9.4	0.01\\
9.41	0.01\\
9.42	0.01\\
9.43	0.01\\
9.44	0.01\\
9.45	0.01\\
9.46	0.01\\
9.47	0.01\\
9.48	0.01\\
9.49	0.01\\
9.5	0.01\\
9.51	0.01\\
9.52	0.01\\
9.53	0.01\\
9.54	0.01\\
9.55	0.01\\
9.56	0.01\\
9.57	0.01\\
9.58	0.01\\
9.59	0.01\\
9.6	0.01\\
9.61	0.01\\
9.62	0.01\\
9.63	0.01\\
9.64	0.01\\
9.65	0.01\\
9.66	0.01\\
9.67	0.01\\
9.68	0.01\\
9.69	0.01\\
9.7	0.01\\
9.71	0.01\\
9.72	0.01\\
9.73	0.01\\
9.74	0.01\\
9.75	0.01\\
9.76	0.01\\
9.77	0.01\\
9.78	0.01\\
9.79	0.01\\
9.8	0.01\\
9.81	0.01\\
9.82	0.01\\
9.83	0.01\\
9.84	0.01\\
9.85	0.01\\
9.86	0.01\\
9.87	0.01\\
9.88	0.01\\
9.89	0.01\\
9.9	0.01\\
9.91	0.01\\
9.92	0.01\\
9.93	0.01\\
9.94	0.01\\
9.95	0.01\\
9.96	0.01\\
9.97	0.01\\
9.98	0.01\\
9.99	0.01\\
10	0.01\\
10.01	0.01\\
10.02	0.01\\
10.03	0.01\\
10.04	0.01\\
10.05	0.01\\
10.06	0.01\\
10.07	0.01\\
10.08	0.01\\
10.09	0.01\\
10.1	0.01\\
10.11	0.01\\
10.12	0.01\\
10.13	0.01\\
10.14	0.01\\
10.15	0.01\\
10.16	0.01\\
10.17	0.01\\
10.18	0.01\\
10.19	0.01\\
10.2	0.01\\
10.21	0.01\\
10.22	0.01\\
10.23	0.01\\
10.24	0.01\\
10.25	0.01\\
10.26	0.01\\
10.27	0.01\\
10.28	0.01\\
10.29	0.01\\
10.3	0.01\\
10.31	0.01\\
10.32	0.01\\
10.33	0.01\\
10.34	0.01\\
10.35	0.01\\
10.36	0.01\\
10.37	0.01\\
10.38	0.01\\
10.39	0.01\\
10.4	0.01\\
10.41	0.01\\
10.42	0.01\\
10.43	0.01\\
10.44	0.01\\
10.45	0.01\\
10.46	0.01\\
10.47	0.01\\
10.48	0.01\\
10.49	0.01\\
10.5	0.01\\
10.51	0.01\\
10.52	0.01\\
10.53	0.01\\
10.54	0.01\\
10.55	0.01\\
10.56	0.01\\
10.57	0.01\\
10.58	0.01\\
10.59	0.01\\
10.6	0.01\\
10.61	0.01\\
10.62	0.01\\
10.63	0.01\\
10.64	0.01\\
10.65	0.01\\
10.66	0.01\\
10.67	0.01\\
10.68	0.01\\
10.69	0.01\\
10.7	0.01\\
10.71	0.01\\
10.72	0.01\\
10.73	0.01\\
10.74	0.01\\
10.75	0.01\\
10.76	0.01\\
10.77	0.01\\
10.78	0.01\\
10.79	0.01\\
10.8	0.01\\
10.81	0.01\\
10.82	0.01\\
10.83	0.01\\
10.84	0.01\\
10.85	0.01\\
10.86	0.01\\
10.87	0.01\\
10.88	0.01\\
10.89	0.01\\
10.9	0.01\\
10.91	0.01\\
10.92	0.01\\
10.93	0.01\\
10.94	0.01\\
10.95	0.01\\
10.96	0.01\\
10.97	0.01\\
10.98	0.01\\
10.99	0.01\\
11	0.01\\
11.01	0.01\\
11.02	0.01\\
11.03	0.01\\
11.04	0.01\\
11.05	0.01\\
11.06	0.01\\
11.07	0.01\\
11.08	0.01\\
11.09	0.01\\
11.1	0.01\\
11.11	0.01\\
11.12	0.01\\
11.13	0.01\\
11.14	0.01\\
11.15	0.01\\
11.16	0.01\\
11.17	0.01\\
11.18	0.01\\
11.19	0.01\\
11.2	0.01\\
11.21	0.01\\
11.22	0.01\\
11.23	0.01\\
11.24	0.01\\
11.25	0.01\\
11.26	0.01\\
11.27	0.01\\
11.28	0.01\\
11.29	0.01\\
11.3	0.01\\
11.31	0.01\\
11.32	0.01\\
11.33	0.01\\
11.34	0.01\\
11.35	0.01\\
11.36	0.01\\
11.37	0.01\\
11.38	0.01\\
11.39	0.01\\
11.4	0.01\\
11.41	0.01\\
11.42	0.01\\
11.43	0.01\\
11.44	0.01\\
11.45	0.01\\
11.46	0.01\\
11.47	0.01\\
11.48	0.01\\
11.49	0.01\\
11.5	0.01\\
11.51	0.01\\
11.52	0.01\\
11.53	0.01\\
11.54	0.01\\
11.55	0.01\\
11.56	0.01\\
11.57	0.01\\
11.58	0.01\\
11.59	0.01\\
11.6	0.01\\
11.61	0.01\\
11.62	0.01\\
11.63	0.01\\
11.64	0.01\\
11.65	0.01\\
11.66	0.01\\
11.67	0.01\\
11.68	0.01\\
11.69	0.01\\
11.7	0.01\\
11.71	0.01\\
11.72	0.01\\
11.73	0.01\\
11.74	0.01\\
11.75	0.01\\
11.76	0.01\\
11.77	0.01\\
11.78	0.01\\
11.79	0.01\\
11.8	0.01\\
11.81	0.01\\
11.82	0.01\\
11.83	0.01\\
11.84	0.01\\
11.85	0.01\\
11.86	0.01\\
11.87	0.01\\
11.88	0.01\\
11.89	0.01\\
11.9	0.01\\
11.91	0.01\\
11.92	0.01\\
11.93	0.01\\
11.94	0.01\\
11.95	0.01\\
11.96	0.01\\
11.97	0.01\\
11.98	0.01\\
11.99	0.01\\
12	0.01\\
12.01	0.01\\
12.02	0.01\\
12.03	0.01\\
12.04	0.01\\
12.05	0.01\\
12.06	0.01\\
12.07	0.01\\
12.08	0.01\\
12.09	0.01\\
12.1	0.01\\
12.11	0.01\\
12.12	0.01\\
12.13	0.01\\
12.14	0.01\\
12.15	0.01\\
12.16	0.01\\
12.17	0.01\\
12.18	0.01\\
12.19	0.01\\
12.2	0.01\\
12.21	0.01\\
12.22	0.01\\
12.23	0.01\\
12.24	0.01\\
12.25	0.01\\
12.26	0.01\\
12.27	0.01\\
12.28	0.01\\
12.29	0.01\\
12.3	0.01\\
12.31	0.01\\
12.32	0.01\\
12.33	0.01\\
12.34	0.01\\
12.35	0.01\\
12.36	0.01\\
12.37	0.01\\
12.38	0.01\\
12.39	0.01\\
12.4	0.01\\
12.41	0.01\\
12.42	0.01\\
12.43	0.01\\
12.44	0.01\\
12.45	0.01\\
12.46	0.01\\
12.47	0.01\\
12.48	0.01\\
12.49	0.01\\
12.5	0.01\\
12.51	0.01\\
12.52	0.01\\
12.53	0.01\\
12.54	0.01\\
12.55	0.01\\
12.56	0.01\\
12.57	0.01\\
12.58	0.01\\
12.59	0.01\\
12.6	0.01\\
12.61	0.01\\
12.62	0.01\\
12.63	0.01\\
12.64	0.01\\
12.65	0.01\\
12.66	0.01\\
12.67	0.01\\
12.68	0.01\\
12.69	0.01\\
12.7	0.01\\
12.71	0.01\\
12.72	0.01\\
12.73	0.01\\
12.74	0.01\\
12.75	0.01\\
12.76	0.01\\
12.77	0.01\\
12.78	0.01\\
12.79	0.01\\
12.8	0.01\\
12.81	0.01\\
12.82	0.01\\
12.83	0.01\\
12.84	0.01\\
12.85	0.01\\
12.86	0.01\\
12.87	0.01\\
12.88	0.01\\
12.89	0.01\\
12.9	0.01\\
12.91	0.01\\
12.92	0.01\\
12.93	0.01\\
12.94	0.01\\
12.95	0.01\\
12.96	0.01\\
12.97	0.01\\
12.98	0.01\\
12.99	0.01\\
13	0.01\\
13.01	0.01\\
13.02	0.01\\
13.03	0.01\\
13.04	0.01\\
13.05	0.01\\
13.06	0.01\\
13.07	0.01\\
13.08	0.01\\
13.09	0.01\\
13.1	0.01\\
13.11	0.01\\
13.12	0.01\\
13.13	0.01\\
13.14	0.01\\
13.15	0.01\\
13.16	0.01\\
13.17	0.01\\
13.18	0.01\\
13.19	0.01\\
13.2	0.01\\
13.21	0.01\\
13.22	0.01\\
13.23	0.01\\
13.24	0.01\\
13.25	0.01\\
13.26	0.01\\
13.27	0.01\\
13.28	0.01\\
13.29	0.01\\
13.3	0.01\\
13.31	0.01\\
13.32	0.01\\
13.33	0.01\\
13.34	0.01\\
13.35	0.01\\
13.36	0.01\\
13.37	0.01\\
13.38	0.01\\
13.39	0.01\\
13.4	0.01\\
13.41	0.01\\
13.42	0.01\\
13.43	0.01\\
13.44	0.01\\
13.45	0.01\\
13.46	0.01\\
13.47	0.01\\
13.48	0.01\\
13.49	0.01\\
13.5	0.01\\
13.51	0.01\\
13.52	0.01\\
13.53	0.01\\
13.54	0.01\\
13.55	0.01\\
13.56	0.01\\
13.57	0.01\\
13.58	0.01\\
13.59	0.01\\
13.6	0.01\\
13.61	0.01\\
13.62	0.01\\
13.63	0.01\\
13.64	0.01\\
13.65	0.01\\
13.66	0.01\\
13.67	0.01\\
13.68	0.01\\
13.69	0.01\\
13.7	0.01\\
13.71	0.01\\
13.72	0.01\\
13.73	0.01\\
13.74	0.01\\
13.75	0.01\\
13.76	0.01\\
13.77	0.01\\
13.78	0.01\\
13.79	0.01\\
13.8	0.01\\
13.81	0.01\\
13.82	0.01\\
13.83	0.01\\
13.84	0.01\\
13.85	0.01\\
13.86	0.01\\
13.87	0.01\\
13.88	0.01\\
13.89	0.01\\
13.9	0.01\\
13.91	0.01\\
13.92	0.01\\
13.93	0.01\\
13.94	0.01\\
13.95	0.01\\
13.96	0.01\\
13.97	0.01\\
13.98	0.01\\
13.99	0.01\\
14	0.01\\
14.01	0.01\\
14.02	0.01\\
14.03	0.01\\
14.04	0.01\\
14.05	0.01\\
14.06	0.01\\
14.07	0.01\\
14.08	0.01\\
14.09	0.01\\
14.1	0.01\\
14.11	0.01\\
14.12	0.01\\
14.13	0.01\\
14.14	0.01\\
14.15	0.01\\
14.16	0.01\\
14.17	0.01\\
14.18	0.01\\
14.19	0.01\\
14.2	0.01\\
14.21	0.01\\
14.22	0.01\\
14.23	0.01\\
14.24	0.01\\
14.25	0.01\\
14.26	0.01\\
14.27	0.01\\
14.28	0.01\\
14.29	0.01\\
14.3	0.01\\
14.31	0.01\\
14.32	0.01\\
14.33	0.01\\
14.34	0.01\\
14.35	0.01\\
14.36	0.01\\
14.37	0.01\\
14.38	0.01\\
14.39	0.01\\
14.4	0.01\\
14.41	0.01\\
14.42	0.01\\
14.43	0.01\\
14.44	0.01\\
14.45	0.01\\
14.46	0.01\\
14.47	0.01\\
14.48	0.01\\
14.49	0.01\\
14.5	0.01\\
14.51	0.01\\
14.52	0.01\\
14.53	0.01\\
14.54	0.01\\
14.55	0.01\\
14.56	0.01\\
14.57	0.01\\
14.58	0.01\\
14.59	0.01\\
14.6	0.01\\
14.61	0.01\\
14.62	0.01\\
14.63	0.01\\
14.64	0.01\\
14.65	0.01\\
14.66	0.01\\
14.67	0.01\\
14.68	0.01\\
14.69	0.01\\
14.7	0.01\\
14.71	0.01\\
14.72	0.01\\
14.73	0.01\\
14.74	0.01\\
14.75	0.01\\
14.76	0.01\\
14.77	0.01\\
14.78	0.01\\
14.79	0.01\\
14.8	0.01\\
14.81	0.01\\
14.82	0.01\\
14.83	0.01\\
14.84	0.01\\
14.85	0.01\\
14.86	0.01\\
14.87	0.01\\
14.88	0.01\\
14.89	0.01\\
14.9	0.01\\
14.91	0.01\\
14.92	0.01\\
14.93	0.01\\
14.94	0.01\\
14.95	0.01\\
14.96	0.01\\
14.97	0.01\\
14.98	0.01\\
14.99	0.01\\
15	0.01\\
15.01	0.01\\
15.02	0.01\\
15.03	0.01\\
15.04	0.01\\
15.05	0.01\\
15.06	0.01\\
15.07	0.01\\
15.08	0.01\\
15.09	0.01\\
15.1	0.01\\
15.11	0.01\\
15.12	0.01\\
15.13	0.01\\
15.14	0.01\\
15.15	0.01\\
15.16	0.01\\
15.17	0.01\\
15.18	0.01\\
15.19	0.01\\
15.2	0.01\\
15.21	0.01\\
15.22	0.01\\
15.23	0.01\\
15.24	0.01\\
15.25	0.01\\
15.26	0.01\\
15.27	0.01\\
15.28	0.01\\
15.29	0.01\\
15.3	0.01\\
15.31	0.01\\
15.32	0.01\\
15.33	0.01\\
15.34	0.01\\
15.35	0.01\\
15.36	0.01\\
15.37	0.01\\
15.38	0.01\\
15.39	0.01\\
15.4	0.01\\
15.41	0.01\\
15.42	0.01\\
15.43	0.01\\
15.44	0.01\\
15.45	0.01\\
15.46	0.01\\
15.47	0.01\\
15.48	0.01\\
15.49	0.01\\
15.5	0.01\\
15.51	0.01\\
15.52	0.01\\
15.53	0.01\\
15.54	0.01\\
15.55	0.01\\
15.56	0.01\\
15.57	0.01\\
15.58	0.01\\
15.59	0.01\\
15.6	0.01\\
15.61	0.01\\
15.62	0.01\\
15.63	0.01\\
15.64	0.01\\
15.65	0.01\\
15.66	0.01\\
15.67	0.01\\
15.68	0.01\\
15.69	0.01\\
15.7	0.01\\
15.71	0.01\\
15.72	0.01\\
15.73	0.01\\
15.74	0.01\\
15.75	0.01\\
15.76	0.01\\
15.77	0.01\\
15.78	0.01\\
15.79	0.01\\
15.8	0.01\\
15.81	0.01\\
15.82	0.01\\
15.83	0.01\\
15.84	0.01\\
15.85	0.01\\
15.86	0.01\\
15.87	0.01\\
15.88	0.01\\
15.89	0.01\\
15.9	0.01\\
15.91	0.01\\
15.92	0.01\\
15.93	0.01\\
15.94	0.01\\
15.95	0.01\\
15.96	0.01\\
15.97	0.01\\
15.98	0.01\\
15.99	0.01\\
16	0.01\\
16.01	0.01\\
16.02	0.01\\
16.03	0.01\\
16.04	0.01\\
16.05	0.01\\
16.06	0.01\\
16.07	0.01\\
16.08	0.01\\
16.09	0.01\\
16.1	0.01\\
16.11	0.01\\
16.12	0.01\\
16.13	0.01\\
16.14	0.01\\
16.15	0.01\\
16.16	0.01\\
16.17	0.01\\
16.18	0.01\\
16.19	0.01\\
16.2	0.01\\
16.21	0.01\\
16.22	0.01\\
16.23	0.01\\
16.24	0.01\\
16.25	0.01\\
16.26	0.01\\
16.27	0.01\\
16.28	0.01\\
16.29	0.01\\
16.3	0.01\\
16.31	0.01\\
16.32	0.01\\
16.33	0.01\\
16.34	0.01\\
16.35	0.01\\
16.36	0.01\\
16.37	0.01\\
16.38	0.01\\
16.39	0.01\\
16.4	0.01\\
16.41	0.01\\
16.42	0.01\\
16.43	0.01\\
16.44	0.01\\
16.45	0.01\\
16.46	0.01\\
16.47	0.01\\
16.48	0.01\\
16.49	0.01\\
16.5	0.01\\
16.51	0.01\\
16.52	0.01\\
16.53	0.01\\
16.54	0.01\\
16.55	0.01\\
16.56	0.01\\
16.57	0.01\\
16.58	0.01\\
16.59	0.01\\
16.6	0.01\\
16.61	0.01\\
16.62	0.01\\
16.63	0.01\\
16.64	0.01\\
16.65	0.01\\
16.66	0.01\\
16.67	0.01\\
16.68	0.01\\
16.69	0.01\\
16.7	0.01\\
16.71	0.01\\
16.72	0.01\\
16.73	0.01\\
16.74	0.01\\
16.75	0.01\\
16.76	0.01\\
16.77	0.01\\
16.78	0.01\\
16.79	0.01\\
16.8	0.01\\
16.81	0.01\\
16.82	0.01\\
16.83	0.01\\
16.84	0.01\\
16.85	0.01\\
16.86	0.01\\
16.87	0.01\\
16.88	0.01\\
16.89	0.01\\
16.9	0.01\\
16.91	0.01\\
16.92	0.01\\
16.93	0.01\\
16.94	0.01\\
16.95	0.01\\
16.96	0.01\\
16.97	0.01\\
16.98	0.01\\
16.99	0.01\\
17	0.01\\
17.01	0.01\\
17.02	0.01\\
17.03	0.01\\
17.04	0.01\\
17.05	0.01\\
17.06	0.01\\
17.07	0.01\\
17.08	0.01\\
17.09	0.01\\
17.1	0.01\\
17.11	0.01\\
17.12	0.01\\
17.13	0.01\\
17.14	0.01\\
17.15	0.01\\
17.16	0.01\\
17.17	0.01\\
17.18	0.01\\
17.19	0.01\\
17.2	0.01\\
17.21	0.01\\
17.22	0.01\\
17.23	0.01\\
17.24	0.01\\
17.25	0.01\\
17.26	0.01\\
17.27	0.01\\
17.28	0.01\\
17.29	0.01\\
17.3	0.01\\
17.31	0.01\\
17.32	0.01\\
17.33	0.01\\
17.34	0.01\\
17.35	0.01\\
17.36	0.01\\
17.37	0.01\\
17.38	0.01\\
17.39	0.01\\
17.4	0.01\\
17.41	0.01\\
17.42	0.01\\
17.43	0.01\\
17.44	0.01\\
17.45	0.01\\
17.46	0.01\\
17.47	0.01\\
17.48	0.01\\
17.49	0.01\\
17.5	0.01\\
17.51	0.01\\
17.52	0.01\\
17.53	0.01\\
17.54	0.01\\
17.55	0.01\\
17.56	0.01\\
17.57	0.01\\
17.58	0.01\\
17.59	0.01\\
17.6	0.01\\
17.61	0.01\\
17.62	0.01\\
17.63	0.01\\
17.64	0.01\\
17.65	0.01\\
17.66	0.01\\
17.67	0.01\\
17.68	0.01\\
17.69	0.01\\
17.7	0.01\\
17.71	0.01\\
17.72	0.01\\
17.73	0.01\\
17.74	0.01\\
17.75	0.01\\
17.76	0.01\\
17.77	0.01\\
17.78	0.01\\
17.79	0.01\\
17.8	0.01\\
17.81	0.01\\
17.82	0.01\\
17.83	0.01\\
17.84	0.01\\
17.85	0.01\\
17.86	0.01\\
17.87	0.01\\
17.88	0.01\\
17.89	0.01\\
17.9	0.01\\
17.91	0.01\\
17.92	0.01\\
17.93	0.01\\
17.94	0.01\\
17.95	0.01\\
17.96	0.01\\
17.97	0.01\\
17.98	0.01\\
17.99	0.01\\
18	0.01\\
18.01	0.01\\
18.02	0.01\\
18.03	0.01\\
18.04	0.01\\
18.05	0.01\\
18.06	0.01\\
18.07	0.01\\
18.08	0.01\\
18.09	0.01\\
18.1	0.01\\
18.11	0.01\\
18.12	0.01\\
18.13	0.01\\
18.14	0.01\\
18.15	0.01\\
18.16	0.01\\
18.17	0.01\\
18.18	0.01\\
18.19	0.01\\
18.2	0.01\\
18.21	0.01\\
18.22	0.01\\
18.23	0.01\\
18.24	0.01\\
18.25	0.01\\
18.26	0.01\\
18.27	0.01\\
18.28	0.01\\
18.29	0.01\\
18.3	0.01\\
18.31	0.01\\
18.32	0.01\\
18.33	0.01\\
18.34	0.01\\
18.35	0.01\\
18.36	0.01\\
18.37	0.01\\
18.38	0.01\\
18.39	0.01\\
18.4	0.01\\
18.41	0.01\\
18.42	0.01\\
18.43	0.01\\
18.44	0.01\\
18.45	0.01\\
18.46	0.01\\
18.47	0.01\\
18.48	0.01\\
18.49	0.01\\
18.5	0.01\\
18.51	0.01\\
18.52	0.01\\
18.53	0.01\\
18.54	0.01\\
18.55	0.01\\
18.56	0.01\\
18.57	0.01\\
18.58	0.01\\
18.59	0.01\\
18.6	0.01\\
18.61	0.01\\
18.62	0.01\\
18.63	0.01\\
18.64	0.01\\
18.65	0.01\\
18.66	0.01\\
18.67	0.01\\
18.68	0.01\\
18.69	0.01\\
18.7	0.01\\
18.71	0.01\\
18.72	0.01\\
18.73	0.01\\
18.74	0.01\\
18.75	0.01\\
18.76	0.01\\
18.77	0.01\\
18.78	0.01\\
18.79	0.01\\
18.8	0.01\\
18.81	0.01\\
18.82	0.01\\
18.83	0.01\\
18.84	0.01\\
18.85	0.01\\
18.86	0.01\\
18.87	0.01\\
18.88	0.01\\
18.89	0.01\\
18.9	0.01\\
18.91	0.01\\
18.92	0.01\\
18.93	0.01\\
18.94	0.01\\
18.95	0.01\\
18.96	0.01\\
18.97	0.01\\
18.98	0.01\\
18.99	0.01\\
19	0.01\\
19.01	0.01\\
19.02	0.01\\
19.03	0.01\\
19.04	0.01\\
19.05	0.01\\
19.06	0.01\\
19.07	0.01\\
19.08	0.01\\
19.09	0.01\\
19.1	0.01\\
19.11	0.01\\
19.12	0.01\\
19.13	0.01\\
19.14	0.01\\
19.15	0.01\\
19.16	0.01\\
19.17	0.01\\
19.18	0.01\\
19.19	0.01\\
19.2	0.01\\
19.21	0.01\\
19.22	0.01\\
19.23	0.01\\
19.24	0.01\\
19.25	0.01\\
19.26	0.01\\
19.27	0.01\\
19.28	0.01\\
19.29	0.01\\
19.3	0.01\\
19.31	0.01\\
19.32	0.01\\
19.33	0.01\\
19.34	0.01\\
19.35	0.01\\
19.36	0.01\\
19.37	0.01\\
19.38	0.01\\
19.39	0.01\\
19.4	0.01\\
19.41	0.01\\
19.42	0.01\\
19.43	0.01\\
19.44	0.01\\
19.45	0.01\\
19.46	0.01\\
19.47	0.01\\
19.48	0.01\\
19.49	0.01\\
19.5	0.01\\
19.51	0.01\\
19.52	0.01\\
19.53	0.01\\
19.54	0.01\\
19.55	0.01\\
19.56	0.01\\
19.57	0.01\\
19.58	0.01\\
19.59	0.01\\
19.6	0.01\\
19.61	0.01\\
19.62	0.01\\
19.63	0.01\\
19.64	0.01\\
19.65	0.01\\
19.66	0.01\\
19.67	0.01\\
19.68	0.01\\
19.69	0.01\\
19.7	0.01\\
19.71	0.01\\
19.72	0.01\\
19.73	0.01\\
19.74	0.01\\
19.75	0.01\\
19.76	0.01\\
19.77	0.01\\
19.78	0.01\\
19.79	0.01\\
19.8	0.01\\
19.81	0.01\\
19.82	0.01\\
19.83	0.01\\
19.84	0.01\\
19.85	0.01\\
19.86	0.01\\
19.87	0.01\\
19.88	0.01\\
19.89	0.01\\
19.9	0.01\\
19.91	0.01\\
19.92	0.01\\
19.93	0.01\\
19.94	0.01\\
19.95	0.01\\
19.96	0.01\\
19.97	0.01\\
19.98	0.01\\
19.99	0.01\\
20	0.01\\
20.01	0.01\\
20.02	0.01\\
20.03	0.01\\
20.04	0.01\\
20.05	0.01\\
20.06	0.01\\
20.07	0.01\\
20.08	0.01\\
20.09	0.01\\
20.1	0.01\\
20.11	0.01\\
20.12	0.01\\
20.13	0.01\\
20.14	0.01\\
20.15	0.01\\
20.16	0.01\\
20.17	0.01\\
20.18	0.01\\
20.19	0.01\\
20.2	0.01\\
20.21	0.01\\
20.22	0.01\\
20.23	0.01\\
20.24	0.01\\
20.25	0.01\\
20.26	0.01\\
20.27	0.01\\
20.28	0.01\\
20.29	0.01\\
20.3	0.01\\
20.31	0.01\\
20.32	0.01\\
20.33	0.01\\
20.34	0.01\\
20.35	0.01\\
20.36	0.01\\
20.37	0.01\\
20.38	0.01\\
20.39	0.01\\
20.4	0.01\\
20.41	0.01\\
20.42	0.01\\
20.43	0.01\\
20.44	0.01\\
20.45	0.01\\
20.46	0.01\\
20.47	0.01\\
20.48	0.01\\
20.49	0.01\\
20.5	0.01\\
20.51	0.01\\
20.52	0.01\\
20.53	0.01\\
20.54	0.01\\
20.55	0.01\\
20.56	0.01\\
20.57	0.01\\
20.58	0.01\\
20.59	0.01\\
20.6	0.01\\
20.61	0.01\\
20.62	0.01\\
20.63	0.01\\
20.64	0.01\\
20.65	0.01\\
20.66	0.01\\
20.67	0.01\\
20.68	0.01\\
20.69	0.01\\
20.7	0.01\\
20.71	0.01\\
20.72	0.01\\
20.73	0.01\\
20.74	0.01\\
20.75	0.01\\
20.76	0.01\\
20.77	0.01\\
20.78	0.01\\
20.79	0.01\\
20.8	0.01\\
20.81	0.01\\
20.82	0.01\\
20.83	0.01\\
20.84	0.01\\
20.85	0.01\\
20.86	0.01\\
20.87	0.01\\
20.88	0.01\\
20.89	0.01\\
20.9	0.01\\
20.91	0.01\\
20.92	0.01\\
20.93	0.01\\
20.94	0.01\\
20.95	0.01\\
20.96	0.01\\
20.97	0.01\\
20.98	0.01\\
20.99	0.01\\
21	0.01\\
21.01	0.01\\
21.02	0.01\\
21.03	0.01\\
21.04	0.01\\
21.05	0.01\\
21.06	0.01\\
21.07	0.01\\
21.08	0.01\\
21.09	0.01\\
21.1	0.01\\
21.11	0.01\\
21.12	0.01\\
21.13	0.01\\
21.14	0.01\\
21.15	0.01\\
21.16	0.01\\
21.17	0.01\\
21.18	0.01\\
21.19	0.01\\
21.2	0.01\\
21.21	0.01\\
21.22	0.01\\
21.23	0.01\\
21.24	0.01\\
21.25	0.01\\
21.26	0.01\\
21.27	0.01\\
21.28	0.01\\
21.29	0.01\\
21.3	0.01\\
21.31	0.01\\
21.32	0.01\\
21.33	0.01\\
21.34	0.01\\
21.35	0.01\\
21.36	0.01\\
21.37	0.01\\
21.38	0.01\\
21.39	0.01\\
21.4	0.01\\
21.41	0.01\\
21.42	0.01\\
21.43	0.01\\
21.44	0.01\\
21.45	0.01\\
21.46	0.01\\
21.47	0.01\\
21.48	0.01\\
21.49	0.01\\
21.5	0.01\\
21.51	0.01\\
21.52	0.01\\
21.53	0.01\\
21.54	0.01\\
21.55	0.01\\
21.56	0.01\\
21.57	0.01\\
21.58	0.01\\
21.59	0.01\\
21.6	0.01\\
21.61	0.01\\
21.62	0.01\\
21.63	0.01\\
21.64	0.01\\
21.65	0.01\\
21.66	0.01\\
21.67	0.01\\
21.68	0.01\\
21.69	0.01\\
21.7	0.01\\
21.71	0.01\\
21.72	0.01\\
21.73	0.01\\
21.74	0.01\\
21.75	0.01\\
21.76	0.01\\
21.77	0.01\\
21.78	0.01\\
21.79	0.01\\
21.8	0.01\\
21.81	0.01\\
21.82	0.01\\
21.83	0.01\\
21.84	0.01\\
21.85	0.01\\
21.86	0.01\\
21.87	0.01\\
21.88	0.01\\
21.89	0.01\\
21.9	0.01\\
21.91	0.01\\
21.92	0.01\\
21.93	0.01\\
21.94	0.01\\
21.95	0.01\\
21.96	0.01\\
21.97	0.01\\
21.98	0.01\\
21.99	0.01\\
22	0.01\\
22.01	0.01\\
22.02	0.01\\
22.03	0.01\\
22.04	0.01\\
22.05	0.01\\
22.06	0.01\\
22.07	0.01\\
22.08	0.01\\
22.09	0.01\\
22.1	0.01\\
22.11	0.01\\
22.12	0.01\\
22.13	0.01\\
22.14	0.01\\
22.15	0.01\\
22.16	0.01\\
22.17	0.01\\
22.18	0.01\\
22.19	0.01\\
22.2	0.01\\
22.21	0.01\\
22.22	0.01\\
22.23	0.01\\
22.24	0.01\\
22.25	0.01\\
22.26	0.01\\
22.27	0.01\\
22.28	0.01\\
22.29	0.01\\
22.3	0.01\\
22.31	0.01\\
22.32	0.01\\
22.33	0.01\\
22.34	0.01\\
22.35	0.01\\
22.36	0.01\\
22.37	0.01\\
22.38	0.01\\
22.39	0.01\\
22.4	0.01\\
22.41	0.01\\
22.42	0.01\\
22.43	0.01\\
22.44	0.01\\
22.45	0.01\\
22.46	0.01\\
22.47	0.01\\
22.48	0.01\\
22.49	0.01\\
22.5	0.01\\
22.51	0.01\\
22.52	0.01\\
22.53	0.01\\
22.54	0.01\\
22.55	0.01\\
22.56	0.01\\
22.57	0.01\\
22.58	0.01\\
22.59	0.01\\
22.6	0.01\\
22.61	0.01\\
22.62	0.01\\
22.63	0.01\\
22.64	0.01\\
22.65	0.01\\
22.66	0.01\\
22.67	0.01\\
22.68	0.01\\
22.69	0.01\\
22.7	0.01\\
22.71	0.01\\
22.72	0.01\\
22.73	0.01\\
22.74	0.01\\
22.75	0.01\\
22.76	0.01\\
22.77	0.01\\
22.78	0.01\\
22.79	0.01\\
22.8	0.01\\
22.81	0.01\\
22.82	0.01\\
22.83	0.01\\
22.84	0.01\\
22.85	0.01\\
22.86	0.01\\
22.87	0.01\\
22.88	0.01\\
22.89	0.01\\
22.9	0.01\\
22.91	0.01\\
22.92	0.01\\
22.93	0.01\\
22.94	0.01\\
22.95	0.01\\
22.96	0.01\\
22.97	0.01\\
22.98	0.01\\
22.99	0.01\\
23	0.01\\
23.01	0.01\\
23.02	0.01\\
23.03	0.01\\
23.04	0.01\\
23.05	0.01\\
23.06	0.01\\
23.07	0.01\\
23.08	0.01\\
23.09	0.01\\
23.1	0.01\\
23.11	0.01\\
23.12	0.01\\
23.13	0.01\\
23.14	0.01\\
23.15	0.01\\
23.16	0.01\\
23.17	0.01\\
23.18	0.01\\
23.19	0.01\\
23.2	0.01\\
23.21	0.01\\
23.22	0.01\\
23.23	0.01\\
23.24	0.01\\
23.25	0.01\\
23.26	0.01\\
23.27	0.01\\
23.28	0.01\\
23.29	0.01\\
23.3	0.01\\
23.31	0.01\\
23.32	0.01\\
23.33	0.01\\
23.34	0.01\\
23.35	0.01\\
23.36	0.01\\
23.37	0.01\\
23.38	0.01\\
23.39	0.01\\
23.4	0.01\\
23.41	0.01\\
23.42	0.01\\
23.43	0.01\\
23.44	0.01\\
23.45	0.01\\
23.46	0.01\\
23.47	0.01\\
23.48	0.01\\
23.49	0.01\\
23.5	0.01\\
23.51	0.01\\
23.52	0.01\\
23.53	0.01\\
23.54	0.01\\
23.55	0.01\\
23.56	0.01\\
23.57	0.01\\
23.58	0.01\\
23.59	0.01\\
23.6	0.01\\
23.61	0.01\\
23.62	0.01\\
23.63	0.01\\
23.64	0.01\\
23.65	0.01\\
23.66	0.01\\
23.67	0.01\\
23.68	0.01\\
23.69	0.01\\
23.7	0.01\\
23.71	0.01\\
23.72	0.01\\
23.73	0.01\\
23.74	0.01\\
23.75	0.01\\
23.76	0.01\\
23.77	0.01\\
23.78	0.01\\
23.79	0.01\\
23.8	0.01\\
23.81	0.01\\
23.82	0.01\\
23.83	0.01\\
23.84	0.01\\
23.85	0.01\\
23.86	0.01\\
23.87	0.01\\
23.88	0.01\\
23.89	0.01\\
23.9	0.01\\
23.91	0.01\\
23.92	0.01\\
23.93	0.01\\
23.94	0.01\\
23.95	0.01\\
23.96	0.01\\
23.97	0.01\\
23.98	0.01\\
23.99	0.01\\
24	0.01\\
24.01	0.01\\
24.02	0.01\\
24.03	0.01\\
24.04	0.01\\
24.05	0.01\\
24.06	0.01\\
24.07	0.01\\
24.08	0.01\\
24.09	0.01\\
24.1	0.01\\
24.11	0.01\\
24.12	0.01\\
24.13	0.01\\
24.14	0.01\\
24.15	0.01\\
24.16	0.01\\
24.17	0.01\\
24.18	0.01\\
24.19	0.01\\
24.2	0.01\\
24.21	0.01\\
24.22	0.01\\
24.23	0.01\\
24.24	0.01\\
24.25	0.01\\
24.26	0.01\\
24.27	0.01\\
24.28	0.01\\
24.29	0.01\\
24.3	0.01\\
24.31	0.01\\
24.32	0.01\\
24.33	0.01\\
24.34	0.01\\
24.35	0.01\\
24.36	0.01\\
24.37	0.01\\
24.38	0.01\\
24.39	0.01\\
24.4	0.01\\
24.41	0.01\\
24.42	0.01\\
24.43	0.01\\
24.44	0.01\\
24.45	0.01\\
24.46	0.01\\
24.47	0.01\\
24.48	0.01\\
24.49	0.01\\
24.5	0.01\\
24.51	0.01\\
24.52	0.01\\
24.53	0.01\\
24.54	0.01\\
24.55	0.01\\
24.56	0.01\\
24.57	0.01\\
24.58	0.01\\
24.59	0.01\\
24.6	0.01\\
24.61	0.01\\
24.62	0.01\\
24.63	0.01\\
24.64	0.01\\
24.65	0.01\\
24.66	0.01\\
24.67	0.01\\
24.68	0.01\\
24.69	0.01\\
24.7	0.01\\
24.71	0.01\\
24.72	0.01\\
24.73	0.01\\
24.74	0.01\\
24.75	0.01\\
24.76	0.01\\
24.77	0.01\\
24.78	0.01\\
24.79	0.01\\
24.8	0.01\\
24.81	0.01\\
24.82	0.01\\
24.83	0.01\\
24.84	0.01\\
24.85	0.01\\
24.86	0.01\\
24.87	0.01\\
24.88	0.01\\
24.89	0.01\\
24.9	0.01\\
24.91	0.01\\
24.92	0.01\\
24.93	0.01\\
24.94	0.01\\
24.95	0.01\\
24.96	0.01\\
24.97	0.01\\
24.98	0.01\\
24.99	0.01\\
25	0.01\\
25.01	0.01\\
25.02	0.01\\
25.03	0.01\\
25.04	0.01\\
25.05	0.01\\
25.06	0.01\\
25.07	0.01\\
25.08	0.01\\
25.09	0.01\\
25.1	0.01\\
25.11	0.01\\
25.12	0.01\\
25.13	0.01\\
25.14	0.01\\
25.15	0.01\\
25.16	0.01\\
25.17	0.01\\
25.18	0.01\\
25.19	0.01\\
25.2	0.01\\
25.21	0.01\\
25.22	0.01\\
25.23	0.01\\
25.24	0.01\\
25.25	0.01\\
25.26	0.01\\
25.27	0.01\\
25.28	0.01\\
25.29	0.01\\
25.3	0.01\\
25.31	0.01\\
25.32	0.01\\
25.33	0.01\\
25.34	0.01\\
25.35	0.01\\
25.36	0.01\\
25.37	0.01\\
25.38	0.01\\
25.39	0.01\\
25.4	0.01\\
25.41	0.01\\
25.42	0.01\\
25.43	0.01\\
25.44	0.01\\
25.45	0.01\\
25.46	0.01\\
25.47	0.01\\
25.48	0.01\\
25.49	0.01\\
25.5	0.01\\
25.51	0.01\\
25.52	0.01\\
25.53	0.01\\
25.54	0.01\\
25.55	0.01\\
25.56	0.01\\
25.57	0.01\\
25.58	0.01\\
25.59	0.01\\
25.6	0.01\\
25.61	0.01\\
25.62	0.01\\
25.63	0.01\\
25.64	0.01\\
25.65	0.01\\
25.66	0.01\\
25.67	0.01\\
25.68	0.01\\
25.69	0.01\\
25.7	0.01\\
25.71	0.01\\
25.72	0.01\\
25.73	0.01\\
25.74	0.01\\
25.75	0.01\\
25.76	0.01\\
25.77	0.01\\
25.78	0.01\\
25.79	0.01\\
25.8	0.01\\
25.81	0.01\\
25.82	0.01\\
25.83	0.01\\
25.84	0.01\\
25.85	0.01\\
25.86	0.01\\
25.87	0.01\\
25.88	0.01\\
25.89	0.01\\
25.9	0.01\\
25.91	0.01\\
25.92	0.01\\
25.93	0.01\\
25.94	0.01\\
25.95	0.01\\
25.96	0.01\\
25.97	0.01\\
25.98	0.01\\
25.99	0.01\\
26	0.01\\
26.01	0.01\\
26.02	0.01\\
26.03	0.01\\
26.04	0.01\\
26.05	0.01\\
26.06	0.01\\
26.07	0.01\\
26.08	0.01\\
26.09	0.01\\
26.1	0.01\\
26.11	0.01\\
26.12	0.01\\
26.13	0.01\\
26.14	0.01\\
26.15	0.01\\
26.16	0.01\\
26.17	0.01\\
26.18	0.01\\
26.19	0.01\\
26.2	0.01\\
26.21	0.01\\
26.22	0.01\\
26.23	0.01\\
26.24	0.01\\
26.25	0.01\\
26.26	0.01\\
26.27	0.01\\
26.28	0.01\\
26.29	0.01\\
26.3	0.01\\
26.31	0.01\\
26.32	0.01\\
26.33	0.01\\
26.34	0.01\\
26.35	0.01\\
26.36	0.01\\
26.37	0.01\\
26.38	0.01\\
26.39	0.01\\
26.4	0.01\\
26.41	0.01\\
26.42	0.01\\
26.43	0.01\\
26.44	0.01\\
26.45	0.01\\
26.46	0.01\\
26.47	0.01\\
26.48	0.01\\
26.49	0.01\\
26.5	0.01\\
26.51	0.01\\
26.52	0.01\\
26.53	0.01\\
26.54	0.01\\
26.55	0.01\\
26.56	0.01\\
26.57	0.01\\
26.58	0.01\\
26.59	0.01\\
26.6	0.01\\
26.61	0.01\\
26.62	0.01\\
26.63	0.01\\
26.64	0.01\\
26.65	0.01\\
26.66	0.01\\
26.67	0.01\\
26.68	0.01\\
26.69	0.01\\
26.7	0.01\\
26.71	0.01\\
26.72	0.01\\
26.73	0.01\\
26.74	0.01\\
26.75	0.01\\
26.76	0.01\\
26.77	0.01\\
26.78	0.01\\
26.79	0.01\\
26.8	0.01\\
26.81	0.01\\
26.82	0.01\\
26.83	0.01\\
26.84	0.01\\
26.85	0.01\\
26.86	0.01\\
26.87	0.01\\
26.88	0.01\\
26.89	0.01\\
26.9	0.01\\
26.91	0.01\\
26.92	0.01\\
26.93	0.01\\
26.94	0.01\\
26.95	0.01\\
26.96	0.01\\
26.97	0.01\\
26.98	0.01\\
26.99	0.01\\
27	0.01\\
27.01	0.01\\
27.02	0.01\\
27.03	0.01\\
27.04	0.01\\
27.05	0.01\\
27.06	0.01\\
27.07	0.01\\
27.08	0.01\\
27.09	0.01\\
27.1	0.01\\
27.11	0.01\\
27.12	0.01\\
27.13	0.01\\
27.14	0.01\\
27.15	0.01\\
27.16	0.01\\
27.17	0.01\\
27.18	0.01\\
27.19	0.01\\
27.2	0.01\\
27.21	0.01\\
27.22	0.01\\
27.23	0.01\\
27.24	0.01\\
27.25	0.01\\
27.26	0.01\\
27.27	0.01\\
27.28	0.01\\
27.29	0.01\\
27.3	0.01\\
27.31	0.01\\
27.32	0.01\\
27.33	0.01\\
27.34	0.01\\
27.35	0.01\\
27.36	0.01\\
27.37	0.01\\
27.38	0.01\\
27.39	0.01\\
27.4	0.01\\
27.41	0.01\\
27.42	0.01\\
27.43	0.01\\
27.44	0.01\\
27.45	0.01\\
27.46	0.01\\
27.47	0.01\\
27.48	0.01\\
27.49	0.01\\
27.5	0.01\\
27.51	0.01\\
27.52	0.01\\
27.53	0.01\\
27.54	0.01\\
27.55	0.01\\
27.56	0.01\\
27.57	0.01\\
27.58	0.01\\
27.59	0.01\\
27.6	0.01\\
27.61	0.01\\
27.62	0.01\\
27.63	0.01\\
27.64	0.01\\
27.65	0.01\\
27.66	0.01\\
27.67	0.01\\
27.68	0.01\\
27.69	0.01\\
27.7	0.01\\
27.71	0.01\\
27.72	0.01\\
27.73	0.01\\
27.74	0.01\\
27.75	0.01\\
27.76	0.01\\
27.77	0.01\\
27.78	0.01\\
27.79	0.01\\
27.8	0.01\\
27.81	0.01\\
27.82	0.01\\
27.83	0.01\\
27.84	0.01\\
27.85	0.01\\
27.86	0.01\\
27.87	0.01\\
27.88	0.01\\
27.89	0.01\\
27.9	0.01\\
27.91	0.01\\
27.92	0.01\\
27.93	0.01\\
27.94	0.01\\
27.95	0.01\\
27.96	0.01\\
27.97	0.01\\
27.98	0.01\\
27.99	0.01\\
28	0.01\\
28.01	0.01\\
28.02	0.01\\
28.03	0.01\\
28.04	0.01\\
28.05	0.01\\
28.06	0.01\\
28.07	0.01\\
28.08	0.01\\
28.09	0.01\\
28.1	0.01\\
28.11	0.01\\
28.12	0.01\\
28.13	0.01\\
28.14	0.01\\
28.15	0.01\\
28.16	0.01\\
28.17	0.01\\
28.18	0.01\\
28.19	0.01\\
28.2	0.01\\
28.21	0.01\\
28.22	0.01\\
28.23	0.01\\
28.24	0.01\\
28.25	0.01\\
28.26	0.01\\
28.27	0.01\\
28.28	0.01\\
28.29	0.01\\
28.3	0.01\\
28.31	0.01\\
28.32	0.01\\
28.33	0.01\\
28.34	0.01\\
28.35	0.01\\
28.36	0.01\\
28.37	0.01\\
28.38	0.01\\
28.39	0.01\\
28.4	0.01\\
28.41	0.01\\
28.42	0.01\\
28.43	0.01\\
28.44	0.01\\
28.45	0.01\\
28.46	0.01\\
28.47	0.01\\
28.48	0.01\\
28.49	0.01\\
28.5	0.01\\
28.51	0.01\\
28.52	0.01\\
28.53	0.01\\
28.54	0.01\\
28.55	0.01\\
28.56	0.01\\
28.57	0.01\\
28.58	0.01\\
28.59	0.01\\
28.6	0.01\\
28.61	0.01\\
28.62	0.01\\
28.63	0.01\\
28.64	0.01\\
28.65	0.01\\
28.66	0.01\\
28.67	0.01\\
28.68	0.01\\
28.69	0.01\\
28.7	0.01\\
28.71	0.01\\
28.72	0.01\\
28.73	0.01\\
28.74	0.01\\
28.75	0.01\\
28.76	0.01\\
28.77	0.01\\
28.78	0.01\\
28.79	0.01\\
28.8	0.01\\
28.81	0.01\\
28.82	0.01\\
28.83	0.01\\
28.84	0.01\\
28.85	0.01\\
28.86	0.01\\
28.87	0.01\\
28.88	0.01\\
28.89	0.01\\
28.9	0.01\\
28.91	0.01\\
28.92	0.01\\
28.93	0.01\\
28.94	0.01\\
28.95	0.01\\
28.96	0.01\\
28.97	0.01\\
28.98	0.01\\
28.99	0.01\\
29	0.01\\
29.01	0.01\\
29.02	0.01\\
29.03	0.01\\
29.04	0.01\\
29.05	0.01\\
29.06	0.01\\
29.07	0.01\\
29.08	0.01\\
29.09	0.01\\
29.1	0.01\\
29.11	0.01\\
29.12	0.01\\
29.13	0.01\\
29.14	0.01\\
29.15	0.01\\
29.16	0.01\\
29.17	0.01\\
29.18	0.01\\
29.19	0.01\\
29.2	0.01\\
29.21	0.01\\
29.22	0.01\\
29.23	0.01\\
29.24	0.01\\
29.25	0.01\\
29.26	0.01\\
29.27	0.01\\
29.28	0.01\\
29.29	0.01\\
29.3	0.01\\
29.31	0.01\\
29.32	0.01\\
29.33	0.01\\
29.34	0.01\\
29.35	0.01\\
29.36	0.01\\
29.37	0.01\\
29.38	0.01\\
29.39	0.01\\
29.4	0.01\\
29.41	0.01\\
29.42	0.01\\
29.43	0.01\\
29.44	0.01\\
29.45	0.01\\
29.46	0.01\\
29.47	0.01\\
29.48	0.01\\
29.49	0.01\\
29.5	0.01\\
29.51	0.01\\
29.52	0.01\\
29.53	0.01\\
29.54	0.01\\
29.55	0.01\\
29.56	0.01\\
29.57	0.01\\
29.58	0.01\\
29.59	0.01\\
29.6	0.01\\
29.61	0.01\\
29.62	0.01\\
29.63	0.01\\
29.64	0.01\\
29.65	0.01\\
29.66	0.01\\
29.67	0.01\\
29.68	0.01\\
29.69	0.01\\
29.7	0.01\\
29.71	0.01\\
29.72	0.01\\
29.73	0.01\\
29.74	0.01\\
29.75	0.01\\
29.76	0.01\\
29.77	0.01\\
29.78	0.01\\
29.79	0.01\\
29.8	0.01\\
29.81	0.01\\
29.82	0.01\\
29.83	0.01\\
29.84	0.01\\
29.85	0.01\\
29.86	0.01\\
29.87	0.01\\
29.88	0.01\\
29.89	0.01\\
29.9	0.01\\
29.91	0.01\\
29.92	0.01\\
29.93	0.01\\
29.94	0.01\\
29.95	0.01\\
29.96	0.01\\
29.97	0.01\\
29.98	0.01\\
29.99	0.01\\
30	0.01\\
30.01	0.01\\
30.02	0.01\\
30.03	0.01\\
30.04	0.01\\
30.05	0.01\\
30.06	0.01\\
30.07	0.01\\
30.08	0.01\\
30.09	0.01\\
30.1	0.01\\
30.11	0.01\\
30.12	0.01\\
30.13	0.01\\
30.14	0.01\\
30.15	0.01\\
30.16	0.01\\
30.17	0.01\\
30.18	0.01\\
30.19	0.01\\
30.2	0.01\\
30.21	0.01\\
30.22	0.01\\
30.23	0.01\\
30.24	0.01\\
30.25	0.01\\
30.26	0.01\\
30.27	0.01\\
30.28	0.01\\
30.29	0.01\\
30.3	0.01\\
30.31	0.01\\
30.32	0.01\\
30.33	0.01\\
30.34	0.01\\
30.35	0.01\\
30.36	0.01\\
30.37	0.01\\
30.38	0.01\\
30.39	0.01\\
30.4	0.01\\
30.41	0.01\\
30.42	0.01\\
30.43	0.01\\
30.44	0.01\\
30.45	0.01\\
30.46	0.01\\
30.47	0.01\\
30.48	0.01\\
30.49	0.01\\
30.5	0.01\\
30.51	0.01\\
30.52	0.01\\
30.53	0.01\\
30.54	0.01\\
30.55	0.01\\
30.56	0.01\\
30.57	0.01\\
30.58	0.01\\
30.59	0.01\\
30.6	0.01\\
30.61	0.01\\
30.62	0.01\\
30.63	0.01\\
30.64	0.01\\
30.65	0.01\\
30.66	0.01\\
30.67	0.01\\
30.68	0.01\\
30.69	0.01\\
30.7	0.01\\
30.71	0.01\\
30.72	0.01\\
30.73	0.01\\
30.74	0.01\\
30.75	0.01\\
30.76	0.01\\
30.77	0.01\\
30.78	0.01\\
30.79	0.01\\
30.8	0.01\\
30.81	0.01\\
30.82	0.01\\
30.83	0.01\\
30.84	0.01\\
30.85	0.01\\
30.86	0.01\\
30.87	0.01\\
30.88	0.01\\
30.89	0.01\\
30.9	0.01\\
30.91	0.01\\
30.92	0.01\\
30.93	0.01\\
30.94	0.01\\
30.95	0.01\\
30.96	0.01\\
30.97	0.01\\
30.98	0.01\\
30.99	0.01\\
31	0.01\\
31.01	0.01\\
31.02	0.01\\
31.03	0.01\\
31.04	0.01\\
31.05	0.01\\
31.06	0.01\\
31.07	0.01\\
31.08	0.01\\
31.09	0.01\\
31.1	0.01\\
31.11	0.01\\
31.12	0.01\\
31.13	0.01\\
31.14	0.01\\
31.15	0.01\\
31.16	0.01\\
31.17	0.01\\
31.18	0.01\\
31.19	0.01\\
31.2	0.01\\
31.21	0.01\\
31.22	0.01\\
31.23	0.01\\
31.24	0.01\\
31.25	0.01\\
31.26	0.01\\
31.27	0.01\\
31.28	0.01\\
31.29	0.01\\
31.3	0.01\\
31.31	0.01\\
31.32	0.01\\
31.33	0.01\\
31.34	0.01\\
31.35	0.01\\
31.36	0.01\\
31.37	0.01\\
31.38	0.01\\
31.39	0.01\\
31.4	0.01\\
31.41	0.01\\
31.42	0.01\\
31.43	0.01\\
31.44	0.01\\
31.45	0.01\\
31.46	0.01\\
31.47	0.01\\
31.48	0.01\\
31.49	0.01\\
31.5	0.01\\
31.51	0.01\\
31.52	0.01\\
31.53	0.01\\
31.54	0.01\\
31.55	0.01\\
31.56	0.01\\
31.57	0.01\\
31.58	0.01\\
31.59	0.01\\
31.6	0.01\\
31.61	0.01\\
31.62	0.01\\
31.63	0.01\\
31.64	0.01\\
31.65	0.01\\
31.66	0.01\\
31.67	0.01\\
31.68	0.01\\
31.69	0.01\\
31.7	0.01\\
31.71	0.01\\
31.72	0.01\\
31.73	0.01\\
31.74	0.01\\
31.75	0.01\\
31.76	0.01\\
31.77	0.01\\
31.78	0.01\\
31.79	0.01\\
31.8	0.01\\
31.81	0.01\\
31.82	0.01\\
31.83	0.01\\
31.84	0.01\\
31.85	0.01\\
31.86	0.01\\
31.87	0.01\\
31.88	0.01\\
31.89	0.01\\
31.9	0.01\\
31.91	0.01\\
31.92	0.01\\
31.93	0.01\\
31.94	0.01\\
31.95	0.01\\
31.96	0.01\\
31.97	0.01\\
31.98	0.01\\
31.99	0.01\\
32	0.01\\
32.01	0.01\\
32.02	0.01\\
32.03	0.01\\
32.04	0.01\\
32.05	0.01\\
32.06	0.01\\
32.07	0.01\\
32.08	0.01\\
32.09	0.01\\
32.1	0.01\\
32.11	0.01\\
32.12	0.01\\
32.13	0.01\\
32.14	0.01\\
32.15	0.01\\
32.16	0.01\\
32.17	0.01\\
32.18	0.01\\
32.19	0.01\\
32.2	0.01\\
32.21	0.01\\
32.22	0.01\\
32.23	0.01\\
32.24	0.01\\
32.25	0.01\\
32.26	0.01\\
32.27	0.01\\
32.28	0.01\\
32.29	0.01\\
32.3	0.01\\
32.31	0.01\\
32.32	0.01\\
32.33	0.01\\
32.34	0.01\\
32.35	0.01\\
32.36	0.01\\
32.37	0.01\\
32.38	0.01\\
32.39	0.01\\
32.4	0.01\\
32.41	0.01\\
32.42	0.01\\
32.43	0.01\\
32.44	0.01\\
32.45	0.01\\
32.46	0.01\\
32.47	0.01\\
32.48	0.01\\
32.49	0.01\\
32.5	0.01\\
32.51	0.01\\
32.52	0.01\\
32.53	0.01\\
32.54	0.01\\
32.55	0.01\\
32.56	0.01\\
32.57	0.01\\
32.58	0.01\\
32.59	0.01\\
32.6	0.01\\
32.61	0.01\\
32.62	0.01\\
32.63	0.01\\
32.64	0.01\\
32.65	0.01\\
32.66	0.01\\
32.67	0.01\\
32.68	0.01\\
32.69	0.01\\
32.7	0.01\\
32.71	0.01\\
32.72	0.01\\
32.73	0.01\\
32.74	0.01\\
32.75	0.01\\
32.76	0.01\\
32.77	0.01\\
32.78	0.01\\
32.79	0.01\\
32.8	0.01\\
32.81	0.01\\
32.82	0.01\\
32.83	0.01\\
32.84	0.01\\
32.85	0.01\\
32.86	0.01\\
32.87	0.01\\
32.88	0.01\\
32.89	0.01\\
32.9	0.01\\
32.91	0.01\\
32.92	0.01\\
32.93	0.01\\
32.94	0.01\\
32.95	0.01\\
32.96	0.01\\
32.97	0.01\\
32.98	0.01\\
32.99	0.01\\
33	0.01\\
33.01	0.01\\
33.02	0.01\\
33.03	0.01\\
33.04	0.01\\
33.05	0.01\\
33.06	0.01\\
33.07	0.01\\
33.08	0.01\\
33.09	0.01\\
33.1	0.01\\
33.11	0.01\\
33.12	0.01\\
33.13	0.01\\
33.14	0.01\\
33.15	0.01\\
33.16	0.01\\
33.17	0.01\\
33.18	0.01\\
33.19	0.01\\
33.2	0.01\\
33.21	0.01\\
33.22	0.01\\
33.23	0.01\\
33.24	0.01\\
33.25	0.01\\
33.26	0.01\\
33.27	0.01\\
33.28	0.01\\
33.29	0.01\\
33.3	0.01\\
33.31	0.01\\
33.32	0.01\\
33.33	0.01\\
33.34	0.01\\
33.35	0.01\\
33.36	0.01\\
33.37	0.01\\
33.38	0.01\\
33.39	0.01\\
33.4	0.01\\
33.41	0.01\\
33.42	0.01\\
33.43	0.01\\
33.44	0.01\\
33.45	0.01\\
33.46	0.01\\
33.47	0.01\\
33.48	0.01\\
33.49	0.01\\
33.5	0.01\\
33.51	0.01\\
33.52	0.01\\
33.53	0.01\\
33.54	0.01\\
33.55	0.01\\
33.56	0.01\\
33.57	0.01\\
33.58	0.01\\
33.59	0.01\\
33.6	0.01\\
33.61	0.01\\
33.62	0.01\\
33.63	0.01\\
33.64	0.01\\
33.65	0.01\\
33.66	0.01\\
33.67	0.01\\
33.68	0.01\\
33.69	0.01\\
33.7	0.01\\
33.71	0.01\\
33.72	0.01\\
33.73	0.01\\
33.74	0.01\\
33.75	0.01\\
33.76	0.01\\
33.77	0.01\\
33.78	0.01\\
33.79	0.01\\
33.8	0.01\\
33.81	0.01\\
33.82	0.01\\
33.83	0.01\\
33.84	0.01\\
33.85	0.01\\
33.86	0.01\\
33.87	0.01\\
33.88	0.01\\
33.89	0.01\\
33.9	0.01\\
33.91	0.01\\
33.92	0.01\\
33.93	0.01\\
33.94	0.01\\
33.95	0.01\\
33.96	0.01\\
33.97	0.01\\
33.98	0.01\\
33.99	0.01\\
34	0.01\\
34.01	0.01\\
34.02	0.01\\
34.03	0.01\\
34.04	0.01\\
34.05	0.01\\
34.06	0.01\\
34.07	0.01\\
34.08	0.01\\
34.09	0.01\\
34.1	0.01\\
34.11	0.01\\
34.12	0.01\\
34.13	0.01\\
34.14	0.01\\
34.15	0.01\\
34.16	0.01\\
34.17	0.01\\
34.18	0.01\\
34.19	0.01\\
34.2	0.01\\
34.21	0.01\\
34.22	0.01\\
34.23	0.01\\
34.24	0.01\\
34.25	0.01\\
34.26	0.01\\
34.27	0.01\\
34.28	0.01\\
34.29	0.01\\
34.3	0.01\\
34.31	0.01\\
34.32	0.01\\
34.33	0.01\\
34.34	0.01\\
34.35	0.01\\
34.36	0.01\\
34.37	0.01\\
34.38	0.01\\
34.39	0.01\\
34.4	0.01\\
34.41	0.01\\
34.42	0.01\\
34.43	0.01\\
34.44	0.01\\
34.45	0.01\\
34.46	0.01\\
34.47	0.01\\
34.48	0.01\\
34.49	0.01\\
34.5	0.01\\
34.51	0.01\\
34.52	0.01\\
34.53	0.01\\
34.54	0.01\\
34.55	0.01\\
34.56	0.01\\
34.57	0.01\\
34.58	0.01\\
34.59	0.01\\
34.6	0.01\\
34.61	0.01\\
34.62	0.01\\
34.63	0.01\\
34.64	0.01\\
34.65	0.01\\
34.66	0.01\\
34.67	0.01\\
34.68	0.01\\
34.69	0.01\\
34.7	0.01\\
34.71	0.01\\
34.72	0.01\\
34.73	0.01\\
34.74	0.01\\
34.75	0.01\\
34.76	0.01\\
34.77	0.01\\
34.78	0.01\\
34.79	0.01\\
34.8	0.01\\
34.81	0.01\\
34.82	0.01\\
34.83	0.01\\
34.84	0.01\\
34.85	0.01\\
34.86	0.01\\
34.87	0.01\\
34.88	0.01\\
34.89	0.01\\
34.9	0.01\\
34.91	0.01\\
34.92	0.01\\
34.93	0.01\\
34.94	0.01\\
34.95	0.01\\
34.96	0.01\\
34.97	0.01\\
34.98	0.01\\
34.99	0.01\\
35	0.01\\
35.01	0.01\\
35.02	0.01\\
35.03	0.01\\
35.04	0.01\\
35.05	0.01\\
35.06	0.01\\
35.07	0.01\\
35.08	0.01\\
35.09	0.01\\
35.1	0.01\\
35.11	0.01\\
35.12	0.01\\
35.13	0.01\\
35.14	0.01\\
35.15	0.01\\
35.16	0.01\\
35.17	0.01\\
35.18	0.01\\
35.19	0.01\\
35.2	0.01\\
35.21	0.01\\
35.22	0.01\\
35.23	0.01\\
35.24	0.01\\
35.25	0.01\\
35.26	0.01\\
35.27	0.01\\
35.28	0.01\\
35.29	0.01\\
35.3	0.01\\
35.31	0.01\\
35.32	0.01\\
35.33	0.01\\
35.34	0.01\\
35.35	0.01\\
35.36	0.01\\
35.37	0.01\\
35.38	0.01\\
35.39	0.01\\
35.4	0.01\\
35.41	0.01\\
35.42	0.01\\
35.43	0.01\\
35.44	0.01\\
35.45	0.01\\
35.46	0.01\\
35.47	0.01\\
35.48	0.01\\
35.49	0.01\\
35.5	0.01\\
35.51	0.01\\
35.52	0.01\\
35.53	0.01\\
35.54	0.01\\
35.55	0.01\\
35.56	0.01\\
35.57	0.01\\
35.58	0.01\\
35.59	0.01\\
35.6	0.01\\
35.61	0.01\\
35.62	0.01\\
35.63	0.01\\
35.64	0.01\\
35.65	0.01\\
35.66	0.01\\
35.67	0.01\\
35.68	0.01\\
35.69	0.01\\
35.7	0.01\\
35.71	0.01\\
35.72	0.01\\
35.73	0.01\\
35.74	0.01\\
35.75	0.01\\
35.76	0.01\\
35.77	0.01\\
35.78	0.01\\
35.79	0.01\\
35.8	0.01\\
35.81	0.01\\
35.82	0.01\\
35.83	0.01\\
35.84	0.01\\
35.85	0.01\\
35.86	0.01\\
35.87	0.01\\
35.88	0.01\\
35.89	0.01\\
35.9	0.01\\
35.91	0.01\\
35.92	0.01\\
35.93	0.01\\
35.94	0.01\\
35.95	0.01\\
35.96	0.01\\
35.97	0.01\\
35.98	0.01\\
35.99	0.01\\
36	0.01\\
36.01	0.01\\
36.02	0.01\\
36.03	0.01\\
36.04	0.01\\
36.05	0.01\\
36.06	0.01\\
36.07	0.01\\
36.08	0.01\\
36.09	0.01\\
36.1	0.01\\
36.11	0.01\\
36.12	0.01\\
36.13	0.01\\
36.14	0.01\\
36.15	0.01\\
36.16	0.01\\
36.17	0.01\\
36.18	0.01\\
36.19	0.01\\
36.2	0.01\\
36.21	0.01\\
36.22	0.01\\
36.23	0.01\\
36.24	0.01\\
36.25	0.01\\
36.26	0.01\\
36.27	0.01\\
36.28	0.01\\
36.29	0.01\\
36.3	0.01\\
36.31	0.01\\
36.32	0.01\\
36.33	0.01\\
36.34	0.01\\
36.35	0.01\\
36.36	0.01\\
36.37	0.01\\
36.38	0.01\\
36.39	0.01\\
36.4	0.01\\
36.41	0.01\\
36.42	0.01\\
36.43	0.01\\
36.44	0.01\\
36.45	0.01\\
36.46	0.01\\
36.47	0.01\\
36.48	0.01\\
36.49	0.01\\
36.5	0.01\\
36.51	0.01\\
36.52	0.01\\
36.53	0.01\\
36.54	0.01\\
36.55	0.01\\
36.56	0.01\\
36.57	0.01\\
36.58	0.01\\
36.59	0.01\\
36.6	0.01\\
36.61	0.01\\
36.62	0.01\\
36.63	0.01\\
36.64	0.01\\
36.65	0.01\\
36.66	0.01\\
36.67	0.01\\
36.68	0.01\\
36.69	0.01\\
36.7	0.01\\
36.71	0.01\\
36.72	0.01\\
36.73	0.01\\
36.74	0.01\\
36.75	0.01\\
36.76	0.01\\
36.77	0.01\\
36.78	0.01\\
36.79	0.01\\
36.8	0.01\\
36.81	0.01\\
36.82	0.01\\
36.83	0.01\\
36.84	0.01\\
36.85	0.01\\
36.86	0.01\\
36.87	0.01\\
36.88	0.01\\
36.89	0.01\\
36.9	0.01\\
36.91	0.01\\
36.92	0.01\\
36.93	0.01\\
36.94	0.01\\
36.95	0.01\\
36.96	0.01\\
36.97	0.01\\
36.98	0.01\\
36.99	0.01\\
37	0.01\\
37.01	0.01\\
37.02	0.01\\
37.03	0.01\\
37.04	0.01\\
37.05	0.01\\
37.06	0.01\\
37.07	0.01\\
37.08	0.01\\
37.09	0.01\\
37.1	0.01\\
37.11	0.01\\
37.12	0.01\\
37.13	0.01\\
37.14	0.01\\
37.15	0.01\\
37.16	0.01\\
37.17	0.01\\
37.18	0.01\\
37.19	0.01\\
37.2	0.01\\
37.21	0.01\\
37.22	0.01\\
37.23	0.01\\
37.24	0.01\\
37.25	0.01\\
37.26	0.01\\
37.27	0.01\\
37.28	0.01\\
37.29	0.01\\
37.3	0.01\\
37.31	0.01\\
37.32	0.01\\
37.33	0.01\\
37.34	0.01\\
37.35	0.01\\
37.36	0.01\\
37.37	0.01\\
37.38	0.01\\
37.39	0.01\\
37.4	0.01\\
37.41	0.01\\
37.42	0.01\\
37.43	0.01\\
37.44	0.01\\
37.45	0.01\\
37.46	0.01\\
37.47	0.01\\
37.48	0.01\\
37.49	0.01\\
37.5	0.01\\
37.51	0.01\\
37.52	0.01\\
37.53	0.01\\
37.54	0.01\\
37.55	0.01\\
37.56	0.01\\
37.57	0.01\\
37.58	0.01\\
37.59	0.01\\
37.6	0.01\\
37.61	0.01\\
37.62	0.01\\
37.63	0.01\\
37.64	0.01\\
37.65	0.01\\
37.66	0.01\\
37.67	0.01\\
37.68	0.01\\
37.69	0.01\\
37.7	0.01\\
37.71	0.01\\
37.72	0.01\\
37.73	0.01\\
37.74	0.01\\
37.75	0.01\\
37.76	0.01\\
37.77	0.01\\
37.78	0.01\\
37.79	0.01\\
37.8	0.01\\
37.81	0.01\\
37.82	0.01\\
37.83	0.01\\
37.84	0.01\\
37.85	0.01\\
37.86	0.01\\
37.87	0.01\\
37.88	0.01\\
37.89	0.01\\
37.9	0.01\\
37.91	0.01\\
37.92	0.01\\
37.93	0.01\\
37.94	0.01\\
37.95	0.01\\
37.96	0.01\\
37.97	0.01\\
37.98	0.01\\
37.99	0.01\\
38	0.01\\
38.01	0.01\\
38.02	0.01\\
38.03	0.01\\
38.04	0.01\\
38.05	0.01\\
38.06	0.01\\
38.07	0.01\\
38.08	0.01\\
38.09	0.01\\
38.1	0.01\\
38.11	0.01\\
38.12	0.01\\
38.13	0.01\\
38.14	0.01\\
38.15	0.01\\
38.16	0.01\\
38.17	0.01\\
38.18	0.01\\
38.19	0.01\\
38.2	0.01\\
38.21	0.01\\
38.22	0.01\\
38.23	0.01\\
38.24	0.01\\
38.25	0.01\\
38.26	0.01\\
38.27	0.01\\
38.28	0.01\\
38.29	0.01\\
38.3	0.01\\
38.31	0.01\\
38.32	0.01\\
38.33	0.01\\
38.34	0.01\\
38.35	0.01\\
38.36	0.01\\
38.37	0.01\\
38.38	0.01\\
38.39	0.01\\
38.4	0.01\\
38.41	0.01\\
38.42	0.01\\
38.43	0.01\\
38.44	0.01\\
38.45	0.01\\
38.46	0.01\\
38.47	0.01\\
38.48	0.01\\
38.49	0.01\\
38.5	0.01\\
38.51	0.01\\
38.52	0.01\\
38.53	0.01\\
38.54	0.01\\
38.55	0.01\\
38.56	0.01\\
38.57	0.01\\
38.58	0.01\\
38.59	0.01\\
38.6	0.01\\
38.61	0.01\\
38.62	0.01\\
38.63	0.01\\
38.64	0.01\\
38.65	0.01\\
38.66	0.01\\
38.67	0.01\\
38.68	0.01\\
38.69	0.01\\
38.7	0.01\\
38.71	0.01\\
38.72	0.01\\
38.73	0.01\\
38.74	0.01\\
38.75	0.01\\
38.76	0.01\\
38.77	0.01\\
38.78	0.01\\
38.79	0.01\\
38.8	0.01\\
38.81	0.01\\
38.82	0.01\\
38.83	0.01\\
38.84	0.01\\
38.85	0.01\\
38.86	0.01\\
38.87	0.01\\
38.88	0.01\\
38.89	0.01\\
38.9	0.01\\
38.91	0.01\\
38.92	0.01\\
38.93	0.01\\
38.94	0.01\\
38.95	0.01\\
38.96	0.01\\
38.97	0.01\\
38.98	0.01\\
38.99	0.01\\
39	0.01\\
39.01	0.01\\
39.02	0.01\\
39.03	0.01\\
39.04	0.01\\
39.05	0.01\\
39.06	0.01\\
39.07	0.01\\
39.08	0.01\\
39.09	0.01\\
39.1	0.01\\
39.11	0.01\\
39.12	0.01\\
39.13	0.01\\
39.14	0.01\\
39.15	0.01\\
39.16	0.01\\
39.17	0.01\\
39.18	0.01\\
39.19	0.01\\
39.2	0.01\\
39.21	0.01\\
39.22	0.01\\
39.23	0.01\\
39.24	0.01\\
39.25	0.01\\
39.26	0.01\\
39.27	0.01\\
39.28	0.01\\
39.29	0.01\\
39.3	0.01\\
39.31	0.01\\
39.32	0.01\\
39.33	0.01\\
39.34	0.01\\
39.35	0.01\\
39.36	0.01\\
39.37	0.01\\
39.38	0.01\\
39.39	0.01\\
39.4	0.01\\
39.41	0.01\\
39.42	0.01\\
39.43	0.01\\
39.44	0.01\\
39.45	0.01\\
39.46	0.01\\
39.47	0.01\\
39.48	0.01\\
39.49	0.01\\
39.5	0.01\\
39.51	0.01\\
39.52	0.01\\
39.53	0.01\\
39.54	0.01\\
39.55	0.01\\
39.56	0.01\\
39.57	0.01\\
39.58	0.01\\
39.59	0.01\\
39.6	0.01\\
39.61	0.01\\
39.62	0.01\\
39.63	0.01\\
39.64	0.01\\
39.65	0.01\\
39.66	0.01\\
39.67	0.01\\
39.68	0.01\\
39.69	0.01\\
39.7	0.01\\
39.71	0.01\\
39.72	0.01\\
39.73	0.01\\
39.74	0.01\\
39.75	0.01\\
39.76	0.01\\
39.77	0.01\\
39.78	0.01\\
39.79	0.01\\
39.8	0.01\\
39.81	0.01\\
39.82	0.01\\
39.83	0.01\\
39.84	0.01\\
39.85	0.01\\
39.86	0.01\\
39.87	0.01\\
39.88	0.01\\
39.89	0.01\\
39.9	0.01\\
39.91	0.01\\
39.92	0.01\\
39.93	0.01\\
39.94	0.01\\
39.95	0.01\\
39.96	0.01\\
39.97	0.01\\
39.98	0.01\\
39.99	0.01\\
40	0.01\\
40.01	0.01\\
};
\addplot [color=blue,dashed,forget plot]
  table[row sep=crcr]{%
40.01	0.01\\
40.02	0.01\\
40.03	0.01\\
40.04	0.01\\
40.05	0.01\\
40.06	0.01\\
40.07	0.01\\
40.08	0.01\\
40.09	0.01\\
40.1	0.01\\
40.11	0.01\\
40.12	0.01\\
40.13	0.01\\
40.14	0.01\\
40.15	0.01\\
40.16	0.01\\
40.17	0.01\\
40.18	0.01\\
40.19	0.01\\
40.2	0.01\\
40.21	0.01\\
40.22	0.01\\
40.23	0.01\\
40.24	0.01\\
40.25	0.01\\
40.26	0.01\\
40.27	0.01\\
40.28	0.01\\
40.29	0.01\\
40.3	0.01\\
40.31	0.01\\
40.32	0.01\\
40.33	0.01\\
40.34	0.01\\
40.35	0.01\\
40.36	0.01\\
40.37	0.01\\
40.38	0.01\\
40.39	0.01\\
40.4	0.01\\
40.41	0.01\\
40.42	0.01\\
40.43	0.01\\
40.44	0.01\\
40.45	0.01\\
40.46	0.01\\
40.47	0.01\\
40.48	0.01\\
40.49	0.01\\
40.5	0.01\\
40.51	0.01\\
40.52	0.01\\
40.53	0.01\\
40.54	0.01\\
40.55	0.01\\
40.56	0.01\\
40.57	0.01\\
40.58	0.01\\
40.59	0.01\\
40.6	0.01\\
40.61	0.01\\
40.62	0.01\\
40.63	0.01\\
40.64	0.01\\
40.65	0.01\\
40.66	0.01\\
40.67	0.01\\
40.68	0.01\\
40.69	0.01\\
40.7	0.01\\
40.71	0.01\\
40.72	0.01\\
40.73	0.01\\
40.74	0.01\\
40.75	0.01\\
40.76	0.01\\
40.77	0.01\\
40.78	0.01\\
40.79	0.01\\
40.8	0.01\\
40.81	0.01\\
40.82	0.01\\
40.83	0.01\\
40.84	0.01\\
40.85	0.01\\
40.86	0.01\\
40.87	0.01\\
40.88	0.01\\
40.89	0.01\\
40.9	0.01\\
40.91	0.01\\
40.92	0.01\\
40.93	0.01\\
40.94	0.01\\
40.95	0.01\\
40.96	0.01\\
40.97	0.01\\
40.98	0.01\\
40.99	0.01\\
41	0.01\\
41.01	0.01\\
41.02	0.01\\
41.03	0.01\\
41.04	0.01\\
41.05	0.01\\
41.06	0.01\\
41.07	0.01\\
41.08	0.01\\
41.09	0.01\\
41.1	0.01\\
41.11	0.01\\
41.12	0.01\\
41.13	0.01\\
41.14	0.01\\
41.15	0.01\\
41.16	0.01\\
41.17	0.01\\
41.18	0.01\\
41.19	0.01\\
41.2	0.01\\
41.21	0.01\\
41.22	0.01\\
41.23	0.01\\
41.24	0.01\\
41.25	0.01\\
41.26	0.01\\
41.27	0.01\\
41.28	0.01\\
41.29	0.01\\
41.3	0.01\\
41.31	0.01\\
41.32	0.01\\
41.33	0.01\\
41.34	0.01\\
41.35	0.01\\
41.36	0.01\\
41.37	0.01\\
41.38	0.01\\
41.39	0.01\\
41.4	0.01\\
41.41	0.01\\
41.42	0.01\\
41.43	0.01\\
41.44	0.01\\
41.45	0.01\\
41.46	0.01\\
41.47	0.01\\
41.48	0.01\\
41.49	0.01\\
41.5	0.01\\
41.51	0.01\\
41.52	0.01\\
41.53	0.01\\
41.54	0.01\\
41.55	0.01\\
41.56	0.01\\
41.57	0.01\\
41.58	0.01\\
41.59	0.01\\
41.6	0.01\\
41.61	0.01\\
41.62	0.01\\
41.63	0.01\\
41.64	0.01\\
41.65	0.01\\
41.66	0.01\\
41.67	0.01\\
41.68	0.01\\
41.69	0.01\\
41.7	0.01\\
41.71	0.01\\
41.72	0.01\\
41.73	0.01\\
41.74	0.01\\
41.75	0.01\\
41.76	0.01\\
41.77	0.01\\
41.78	0.01\\
41.79	0.01\\
41.8	0.01\\
41.81	0.01\\
41.82	0.01\\
41.83	0.01\\
41.84	0.01\\
41.85	0.01\\
41.86	0.01\\
41.87	0.01\\
41.88	0.01\\
41.89	0.01\\
41.9	0.01\\
41.91	0.01\\
41.92	0.01\\
41.93	0.01\\
41.94	0.01\\
41.95	0.01\\
41.96	0.01\\
41.97	0.01\\
41.98	0.01\\
41.99	0.01\\
42	0.01\\
42.01	0.01\\
42.02	0.01\\
42.03	0.01\\
42.04	0.01\\
42.05	0.01\\
42.06	0.01\\
42.07	0.01\\
42.08	0.01\\
42.09	0.01\\
42.1	0.01\\
42.11	0.01\\
42.12	0.01\\
42.13	0.01\\
42.14	0.01\\
42.15	0.01\\
42.16	0.01\\
42.17	0.01\\
42.18	0.01\\
42.19	0.01\\
42.2	0.01\\
42.21	0.01\\
42.22	0.01\\
42.23	0.01\\
42.24	0.01\\
42.25	0.01\\
42.26	0.01\\
42.27	0.01\\
42.28	0.01\\
42.29	0.01\\
42.3	0.01\\
42.31	0.01\\
42.32	0.01\\
42.33	0.01\\
42.34	0.01\\
42.35	0.01\\
42.36	0.01\\
42.37	0.01\\
42.38	0.01\\
42.39	0.01\\
42.4	0.01\\
42.41	0.01\\
42.42	0.01\\
42.43	0.01\\
42.44	0.01\\
42.45	0.01\\
42.46	0.01\\
42.47	0.01\\
42.48	0.01\\
42.49	0.01\\
42.5	0.01\\
42.51	0.01\\
42.52	0.01\\
42.53	0.01\\
42.54	0.01\\
42.55	0.01\\
42.56	0.01\\
42.57	0.01\\
42.58	0.01\\
42.59	0.01\\
42.6	0.01\\
42.61	0.01\\
42.62	0.01\\
42.63	0.01\\
42.64	0.01\\
42.65	0.01\\
42.66	0.01\\
42.67	0.01\\
42.68	0.01\\
42.69	0.01\\
42.7	0.01\\
42.71	0.01\\
42.72	0.01\\
42.73	0.01\\
42.74	0.01\\
42.75	0.01\\
42.76	0.01\\
42.77	0.01\\
42.78	0.01\\
42.79	0.01\\
42.8	0.01\\
42.81	0.01\\
42.82	0.01\\
42.83	0.01\\
42.84	0.01\\
42.85	0.01\\
42.86	0.01\\
42.87	0.01\\
42.88	0.01\\
42.89	0.01\\
42.9	0.01\\
42.91	0.01\\
42.92	0.01\\
42.93	0.01\\
42.94	0.01\\
42.95	0.01\\
42.96	0.01\\
42.97	0.01\\
42.98	0.01\\
42.99	0.01\\
43	0.01\\
43.01	0.01\\
43.02	0.01\\
43.03	0.01\\
43.04	0.01\\
43.05	0.01\\
43.06	0.01\\
43.07	0.01\\
43.08	0.01\\
43.09	0.01\\
43.1	0.01\\
43.11	0.01\\
43.12	0.01\\
43.13	0.01\\
43.14	0.01\\
43.15	0.01\\
43.16	0.01\\
43.17	0.01\\
43.18	0.01\\
43.19	0.01\\
43.2	0.01\\
43.21	0.01\\
43.22	0.01\\
43.23	0.01\\
43.24	0.01\\
43.25	0.01\\
43.26	0.01\\
43.27	0.01\\
43.28	0.01\\
43.29	0.01\\
43.3	0.01\\
43.31	0.01\\
43.32	0.01\\
43.33	0.01\\
43.34	0.01\\
43.35	0.01\\
43.36	0.01\\
43.37	0.01\\
43.38	0.01\\
43.39	0.01\\
43.4	0.01\\
43.41	0.01\\
43.42	0.01\\
43.43	0.01\\
43.44	0.01\\
43.45	0.01\\
43.46	0.01\\
43.47	0.01\\
43.48	0.01\\
43.49	0.01\\
43.5	0.01\\
43.51	0.01\\
43.52	0.01\\
43.53	0.01\\
43.54	0.01\\
43.55	0.01\\
43.56	0.01\\
43.57	0.01\\
43.58	0.01\\
43.59	0.01\\
43.6	0.01\\
43.61	0.01\\
43.62	0.01\\
43.63	0.01\\
43.64	0.01\\
43.65	0.01\\
43.66	0.01\\
43.67	0.01\\
43.68	0.01\\
43.69	0.01\\
43.7	0.01\\
43.71	0.01\\
43.72	0.01\\
43.73	0.01\\
43.74	0.01\\
43.75	0.01\\
43.76	0.01\\
43.77	0.01\\
43.78	0.01\\
43.79	0.01\\
43.8	0.01\\
43.81	0.01\\
43.82	0.01\\
43.83	0.01\\
43.84	0.01\\
43.85	0.01\\
43.86	0.01\\
43.87	0.01\\
43.88	0.01\\
43.89	0.01\\
43.9	0.01\\
43.91	0.01\\
43.92	0.01\\
43.93	0.01\\
43.94	0.01\\
43.95	0.01\\
43.96	0.01\\
43.97	0.01\\
43.98	0.01\\
43.99	0.01\\
44	0.01\\
44.01	0.01\\
44.02	0.01\\
44.03	0.01\\
44.04	0.01\\
44.05	0.01\\
44.06	0.01\\
44.07	0.01\\
44.08	0.01\\
44.09	0.01\\
44.1	0.01\\
44.11	0.01\\
44.12	0.01\\
44.13	0.01\\
44.14	0.01\\
44.15	0.01\\
44.16	0.01\\
44.17	0.01\\
44.18	0.01\\
44.19	0.01\\
44.2	0.01\\
44.21	0.01\\
44.22	0.01\\
44.23	0.01\\
44.24	0.01\\
44.25	0.01\\
44.26	0.01\\
44.27	0.01\\
44.28	0.01\\
44.29	0.01\\
44.3	0.01\\
44.31	0.01\\
44.32	0.01\\
44.33	0.01\\
44.34	0.01\\
44.35	0.01\\
44.36	0.01\\
44.37	0.01\\
44.38	0.01\\
44.39	0.01\\
44.4	0.01\\
44.41	0.01\\
44.42	0.01\\
44.43	0.01\\
44.44	0.01\\
44.45	0.01\\
44.46	0.01\\
44.47	0.01\\
44.48	0.01\\
44.49	0.01\\
44.5	0.01\\
44.51	0.01\\
44.52	0.01\\
44.53	0.01\\
44.54	0.01\\
44.55	0.01\\
44.56	0.01\\
44.57	0.01\\
44.58	0.01\\
44.59	0.01\\
44.6	0.01\\
44.61	0.01\\
44.62	0.01\\
44.63	0.01\\
44.64	0.01\\
44.65	0.01\\
44.66	0.01\\
44.67	0.01\\
44.68	0.01\\
44.69	0.01\\
44.7	0.01\\
44.71	0.01\\
44.72	0.01\\
44.73	0.01\\
44.74	0.01\\
44.75	0.01\\
44.76	0.01\\
44.77	0.01\\
44.78	0.01\\
44.79	0.01\\
44.8	0.01\\
44.81	0.01\\
44.82	0.01\\
44.83	0.01\\
44.84	0.01\\
44.85	0.01\\
44.86	0.01\\
44.87	0.01\\
44.88	0.01\\
44.89	0.01\\
44.9	0.01\\
44.91	0.01\\
44.92	0.01\\
44.93	0.01\\
44.94	0.01\\
44.95	0.01\\
44.96	0.01\\
44.97	0.01\\
44.98	0.01\\
44.99	0.01\\
45	0.01\\
45.01	0.01\\
45.02	0.01\\
45.03	0.01\\
45.04	0.01\\
45.05	0.01\\
45.06	0.01\\
45.07	0.01\\
45.08	0.01\\
45.09	0.01\\
45.1	0.01\\
45.11	0.01\\
45.12	0.01\\
45.13	0.01\\
45.14	0.01\\
45.15	0.01\\
45.16	0.01\\
45.17	0.01\\
45.18	0.01\\
45.19	0.01\\
45.2	0.01\\
45.21	0.01\\
45.22	0.01\\
45.23	0.01\\
45.24	0.01\\
45.25	0.01\\
45.26	0.01\\
45.27	0.01\\
45.28	0.01\\
45.29	0.01\\
45.3	0.01\\
45.31	0.01\\
45.32	0.01\\
45.33	0.01\\
45.34	0.01\\
45.35	0.01\\
45.36	0.01\\
45.37	0.01\\
45.38	0.01\\
45.39	0.01\\
45.4	0.01\\
45.41	0.01\\
45.42	0.01\\
45.43	0.01\\
45.44	0.01\\
45.45	0.01\\
45.46	0.01\\
45.47	0.01\\
45.48	0.01\\
45.49	0.01\\
45.5	0.01\\
45.51	0.01\\
45.52	0.01\\
45.53	0.01\\
45.54	0.01\\
45.55	0.01\\
45.56	0.01\\
45.57	0.01\\
45.58	0.01\\
45.59	0.01\\
45.6	0.01\\
45.61	0.01\\
45.62	0.01\\
45.63	0.01\\
45.64	0.01\\
45.65	0.01\\
45.66	0.01\\
45.67	0.01\\
45.68	0.01\\
45.69	0.01\\
45.7	0.01\\
45.71	0.01\\
45.72	0.01\\
45.73	0.01\\
45.74	0.01\\
45.75	0.01\\
45.76	0.01\\
45.77	0.01\\
45.78	0.01\\
45.79	0.01\\
45.8	0.01\\
45.81	0.01\\
45.82	0.01\\
45.83	0.01\\
45.84	0.01\\
45.85	0.01\\
45.86	0.01\\
45.87	0.01\\
45.88	0.01\\
45.89	0.01\\
45.9	0.01\\
45.91	0.01\\
45.92	0.01\\
45.93	0.01\\
45.94	0.01\\
45.95	0.01\\
45.96	0.01\\
45.97	0.01\\
45.98	0.01\\
45.99	0.01\\
46	0.01\\
46.01	0.01\\
46.02	0.01\\
46.03	0.01\\
46.04	0.01\\
46.05	0.01\\
46.06	0.01\\
46.07	0.01\\
46.08	0.01\\
46.09	0.01\\
46.1	0.01\\
46.11	0.01\\
46.12	0.01\\
46.13	0.01\\
46.14	0.01\\
46.15	0.01\\
46.16	0.01\\
46.17	0.01\\
46.18	0.01\\
46.19	0.01\\
46.2	0.01\\
46.21	0.01\\
46.22	0.01\\
46.23	0.01\\
46.24	0.01\\
46.25	0.01\\
46.26	0.01\\
46.27	0.01\\
46.28	0.01\\
46.29	0.01\\
46.3	0.01\\
46.31	0.01\\
46.32	0.01\\
46.33	0.01\\
46.34	0.01\\
46.35	0.01\\
46.36	0.01\\
46.37	0.01\\
46.38	0.01\\
46.39	0.01\\
46.4	0.01\\
46.41	0.01\\
46.42	0.01\\
46.43	0.01\\
46.44	0.01\\
46.45	0.01\\
46.46	0.01\\
46.47	0.01\\
46.48	0.01\\
46.49	0.01\\
46.5	0.01\\
46.51	0.01\\
46.52	0.01\\
46.53	0.01\\
46.54	0.01\\
46.55	0.01\\
46.56	0.01\\
46.57	0.01\\
46.58	0.01\\
46.59	0.01\\
46.6	0.01\\
46.61	0.01\\
46.62	0.01\\
46.63	0.01\\
46.64	0.01\\
46.65	0.01\\
46.66	0.01\\
46.67	0.01\\
46.68	0.01\\
46.69	0.01\\
46.7	0.01\\
46.71	0.01\\
46.72	0.01\\
46.73	0.01\\
46.74	0.01\\
46.75	0.01\\
46.76	0.01\\
46.77	0.01\\
46.78	0.01\\
46.79	0.01\\
46.8	0.01\\
46.81	0.01\\
46.82	0.01\\
46.83	0.01\\
46.84	0.01\\
46.85	0.01\\
46.86	0.01\\
46.87	0.01\\
46.88	0.01\\
46.89	0.01\\
46.9	0.01\\
46.91	0.01\\
46.92	0.01\\
46.93	0.01\\
46.94	0.01\\
46.95	0.01\\
46.96	0.01\\
46.97	0.01\\
46.98	0.01\\
46.99	0.01\\
47	0.01\\
47.01	0.01\\
47.02	0.01\\
47.03	0.01\\
47.04	0.01\\
47.05	0.01\\
47.06	0.01\\
47.07	0.01\\
47.08	0.01\\
47.09	0.01\\
47.1	0.01\\
47.11	0.01\\
47.12	0.01\\
47.13	0.01\\
47.14	0.01\\
47.15	0.01\\
47.16	0.01\\
47.17	0.01\\
47.18	0.01\\
47.19	0.01\\
47.2	0.01\\
47.21	0.01\\
47.22	0.01\\
47.23	0.01\\
47.24	0.01\\
47.25	0.01\\
47.26	0.01\\
47.27	0.01\\
47.28	0.01\\
47.29	0.01\\
47.3	0.01\\
47.31	0.01\\
47.32	0.01\\
47.33	0.01\\
47.34	0.01\\
47.35	0.01\\
47.36	0.01\\
47.37	0.01\\
47.38	0.01\\
47.39	0.01\\
47.4	0.01\\
47.41	0.01\\
47.42	0.01\\
47.43	0.01\\
47.44	0.01\\
47.45	0.01\\
47.46	0.01\\
47.47	0.01\\
47.48	0.01\\
47.49	0.01\\
47.5	0.01\\
47.51	0.01\\
47.52	0.01\\
47.53	0.01\\
47.54	0.01\\
47.55	0.01\\
47.56	0.01\\
47.57	0.01\\
47.58	0.01\\
47.59	0.01\\
47.6	0.01\\
47.61	0.01\\
47.62	0.01\\
47.63	0.01\\
47.64	0.01\\
47.65	0.01\\
47.66	0.01\\
47.67	0.01\\
47.68	0.01\\
47.69	0.01\\
47.7	0.01\\
47.71	0.01\\
47.72	0.01\\
47.73	0.01\\
47.74	0.01\\
47.75	0.01\\
47.76	0.01\\
47.77	0.01\\
47.78	0.01\\
47.79	0.01\\
47.8	0.01\\
47.81	0.01\\
47.82	0.01\\
47.83	0.01\\
47.84	0.01\\
47.85	0.01\\
47.86	0.01\\
47.87	0.01\\
47.88	0.01\\
47.89	0.01\\
47.9	0.01\\
47.91	0.01\\
47.92	0.01\\
47.93	0.01\\
47.94	0.01\\
47.95	0.01\\
47.96	0.01\\
47.97	0.01\\
47.98	0.01\\
47.99	0.01\\
48	0.01\\
48.01	0.01\\
48.02	0.01\\
48.03	0.01\\
48.04	0.01\\
48.05	0.01\\
48.06	0.01\\
48.07	0.01\\
48.08	0.01\\
48.09	0.01\\
48.1	0.01\\
48.11	0.01\\
48.12	0.01\\
48.13	0.01\\
48.14	0.01\\
48.15	0.01\\
48.16	0.01\\
48.17	0.01\\
48.18	0.01\\
48.19	0.01\\
48.2	0.01\\
48.21	0.01\\
48.22	0.01\\
48.23	0.01\\
48.24	0.01\\
48.25	0.01\\
48.26	0.01\\
48.27	0.01\\
48.28	0.01\\
48.29	0.01\\
48.3	0.01\\
48.31	0.01\\
48.32	0.01\\
48.33	0.01\\
48.34	0.01\\
48.35	0.01\\
48.36	0.01\\
48.37	0.01\\
48.38	0.01\\
48.39	0.01\\
48.4	0.01\\
48.41	0.01\\
48.42	0.01\\
48.43	0.01\\
48.44	0.01\\
48.45	0.01\\
48.46	0.01\\
48.47	0.01\\
48.48	0.01\\
48.49	0.01\\
48.5	0.01\\
48.51	0.01\\
48.52	0.01\\
48.53	0.01\\
48.54	0.01\\
48.55	0.01\\
48.56	0.01\\
48.57	0.01\\
48.58	0.01\\
48.59	0.01\\
48.6	0.01\\
48.61	0.01\\
48.62	0.01\\
48.63	0.01\\
48.64	0.01\\
48.65	0.01\\
48.66	0.01\\
48.67	0.01\\
48.68	0.01\\
48.69	0.01\\
48.7	0.01\\
48.71	0.01\\
48.72	0.01\\
48.73	0.01\\
48.74	0.01\\
48.75	0.01\\
48.76	0.01\\
48.77	0.01\\
48.78	0.01\\
48.79	0.01\\
48.8	0.01\\
48.81	0.01\\
48.82	0.01\\
48.83	0.01\\
48.84	0.01\\
48.85	0.01\\
48.86	0.01\\
48.87	0.01\\
48.88	0.01\\
48.89	0.01\\
48.9	0.01\\
48.91	0.01\\
48.92	0.01\\
48.93	0.01\\
48.94	0.01\\
48.95	0.01\\
48.96	0.01\\
48.97	0.01\\
48.98	0.01\\
48.99	0.01\\
49	0.01\\
49.01	0.01\\
49.02	0.01\\
49.03	0.01\\
49.04	0.01\\
49.05	0.01\\
49.06	0.01\\
49.07	0.01\\
49.08	0.01\\
49.09	0.01\\
49.1	0.01\\
49.11	0.01\\
49.12	0.01\\
49.13	0.01\\
49.14	0.01\\
49.15	0.01\\
49.16	0.01\\
49.17	0.01\\
49.18	0.01\\
49.19	0.01\\
49.2	0.01\\
49.21	0.01\\
49.22	0.01\\
49.23	0.01\\
49.24	0.01\\
49.25	0.01\\
49.26	0.01\\
49.27	0.01\\
49.28	0.01\\
49.29	0.01\\
49.3	0.01\\
49.31	0.01\\
49.32	0.01\\
49.33	0.01\\
49.34	0.01\\
49.35	0.01\\
49.36	0.01\\
49.37	0.01\\
49.38	0.01\\
49.39	0.01\\
49.4	0.01\\
49.41	0.01\\
49.42	0.01\\
49.43	0.01\\
49.44	0.01\\
49.45	0.01\\
49.46	0.01\\
49.47	0.01\\
49.48	0.01\\
49.49	0.01\\
49.5	0.01\\
49.51	0.01\\
49.52	0.01\\
49.53	0.01\\
49.54	0.01\\
49.55	0.01\\
49.56	0.01\\
49.57	0.01\\
49.58	0.01\\
49.59	0.01\\
49.6	0.01\\
49.61	0.01\\
49.62	0.01\\
49.63	0.01\\
49.64	0.01\\
49.65	0.01\\
49.66	0.01\\
49.67	0.01\\
49.68	0.01\\
49.69	0.01\\
49.7	0.01\\
49.71	0.01\\
49.72	0.01\\
49.73	0.01\\
49.74	0.01\\
49.75	0.01\\
49.76	0.01\\
49.77	0.01\\
49.78	0.01\\
49.79	0.01\\
49.8	0.01\\
49.81	0.01\\
49.82	0.01\\
49.83	0.01\\
49.84	0.01\\
49.85	0.01\\
49.86	0.01\\
49.87	0.01\\
49.88	0.01\\
49.89	0.01\\
49.9	0.01\\
49.91	0.01\\
49.92	0.01\\
49.93	0.01\\
49.94	0.01\\
49.95	0.01\\
49.96	0.01\\
49.97	0.01\\
49.98	0.01\\
49.99	0.01\\
50	0.01\\
50.01	0.01\\
50.02	0.01\\
50.03	0.01\\
50.04	0.01\\
50.05	0.01\\
50.06	0.01\\
50.07	0.01\\
50.08	0.01\\
50.09	0.01\\
50.1	0.01\\
50.11	0.01\\
50.12	0.01\\
50.13	0.01\\
50.14	0.01\\
50.15	0.01\\
50.16	0.01\\
50.17	0.01\\
50.18	0.01\\
50.19	0.01\\
50.2	0.01\\
50.21	0.01\\
50.22	0.01\\
50.23	0.01\\
50.24	0.01\\
50.25	0.01\\
50.26	0.01\\
50.27	0.01\\
50.28	0.01\\
50.29	0.01\\
50.3	0.01\\
50.31	0.01\\
50.32	0.01\\
50.33	0.01\\
50.34	0.01\\
50.35	0.01\\
50.36	0.01\\
50.37	0.01\\
50.38	0.01\\
50.39	0.01\\
50.4	0.01\\
50.41	0.01\\
50.42	0.01\\
50.43	0.01\\
50.44	0.01\\
50.45	0.01\\
50.46	0.01\\
50.47	0.01\\
50.48	0.01\\
50.49	0.01\\
50.5	0.01\\
50.51	0.01\\
50.52	0.01\\
50.53	0.01\\
50.54	0.01\\
50.55	0.01\\
50.56	0.01\\
50.57	0.01\\
50.58	0.01\\
50.59	0.01\\
50.6	0.01\\
50.61	0.01\\
50.62	0.01\\
50.63	0.01\\
50.64	0.01\\
50.65	0.01\\
50.66	0.01\\
50.67	0.01\\
50.68	0.01\\
50.69	0.01\\
50.7	0.01\\
50.71	0.01\\
50.72	0.01\\
50.73	0.01\\
50.74	0.01\\
50.75	0.01\\
50.76	0.01\\
50.77	0.01\\
50.78	0.01\\
50.79	0.01\\
50.8	0.01\\
50.81	0.01\\
50.82	0.01\\
50.83	0.01\\
50.84	0.01\\
50.85	0.01\\
50.86	0.01\\
50.87	0.01\\
50.88	0.01\\
50.89	0.01\\
50.9	0.01\\
50.91	0.01\\
50.92	0.01\\
50.93	0.01\\
50.94	0.01\\
50.95	0.01\\
50.96	0.01\\
50.97	0.01\\
50.98	0.01\\
50.99	0.01\\
51	0.01\\
51.01	0.01\\
51.02	0.01\\
51.03	0.01\\
51.04	0.01\\
51.05	0.01\\
51.06	0.01\\
51.07	0.01\\
51.08	0.01\\
51.09	0.01\\
51.1	0.01\\
51.11	0.01\\
51.12	0.01\\
51.13	0.01\\
51.14	0.01\\
51.15	0.01\\
51.16	0.01\\
51.17	0.01\\
51.18	0.01\\
51.19	0.01\\
51.2	0.01\\
51.21	0.01\\
51.22	0.01\\
51.23	0.01\\
51.24	0.01\\
51.25	0.01\\
51.26	0.01\\
51.27	0.01\\
51.28	0.01\\
51.29	0.01\\
51.3	0.01\\
51.31	0.01\\
51.32	0.01\\
51.33	0.01\\
51.34	0.01\\
51.35	0.01\\
51.36	0.01\\
51.37	0.01\\
51.38	0.01\\
51.39	0.01\\
51.4	0.01\\
51.41	0.01\\
51.42	0.01\\
51.43	0.01\\
51.44	0.01\\
51.45	0.01\\
51.46	0.01\\
51.47	0.01\\
51.48	0.01\\
51.49	0.01\\
51.5	0.01\\
51.51	0.01\\
51.52	0.01\\
51.53	0.01\\
51.54	0.01\\
51.55	0.01\\
51.56	0.01\\
51.57	0.01\\
51.58	0.01\\
51.59	0.01\\
51.6	0.01\\
51.61	0.01\\
51.62	0.01\\
51.63	0.01\\
51.64	0.01\\
51.65	0.01\\
51.66	0.01\\
51.67	0.01\\
51.68	0.01\\
51.69	0.01\\
51.7	0.01\\
51.71	0.01\\
51.72	0.01\\
51.73	0.01\\
51.74	0.01\\
51.75	0.01\\
51.76	0.01\\
51.77	0.01\\
51.78	0.01\\
51.79	0.01\\
51.8	0.01\\
51.81	0.01\\
51.82	0.01\\
51.83	0.01\\
51.84	0.01\\
51.85	0.01\\
51.86	0.01\\
51.87	0.01\\
51.88	0.01\\
51.89	0.01\\
51.9	0.01\\
51.91	0.01\\
51.92	0.01\\
51.93	0.01\\
51.94	0.01\\
51.95	0.01\\
51.96	0.01\\
51.97	0.01\\
51.98	0.01\\
51.99	0.01\\
52	0.01\\
52.01	0.01\\
52.02	0.01\\
52.03	0.01\\
52.04	0.01\\
52.05	0.01\\
52.06	0.01\\
52.07	0.01\\
52.08	0.01\\
52.09	0.01\\
52.1	0.01\\
52.11	0.01\\
52.12	0.01\\
52.13	0.01\\
52.14	0.01\\
52.15	0.01\\
52.16	0.01\\
52.17	0.01\\
52.18	0.01\\
52.19	0.01\\
52.2	0.01\\
52.21	0.01\\
52.22	0.01\\
52.23	0.01\\
52.24	0.01\\
52.25	0.01\\
52.26	0.01\\
52.27	0.01\\
52.28	0.01\\
52.29	0.01\\
52.3	0.01\\
52.31	0.01\\
52.32	0.01\\
52.33	0.01\\
52.34	0.01\\
52.35	0.01\\
52.36	0.01\\
52.37	0.01\\
52.38	0.01\\
52.39	0.01\\
52.4	0.01\\
52.41	0.01\\
52.42	0.01\\
52.43	0.01\\
52.44	0.01\\
52.45	0.01\\
52.46	0.01\\
52.47	0.01\\
52.48	0.01\\
52.49	0.01\\
52.5	0.01\\
52.51	0.01\\
52.52	0.01\\
52.53	0.01\\
52.54	0.01\\
52.55	0.01\\
52.56	0.01\\
52.57	0.01\\
52.58	0.01\\
52.59	0.01\\
52.6	0.01\\
52.61	0.01\\
52.62	0.01\\
52.63	0.01\\
52.64	0.01\\
52.65	0.01\\
52.66	0.01\\
52.67	0.01\\
52.68	0.01\\
52.69	0.01\\
52.7	0.01\\
52.71	0.01\\
52.72	0.01\\
52.73	0.01\\
52.74	0.01\\
52.75	0.01\\
52.76	0.01\\
52.77	0.01\\
52.78	0.01\\
52.79	0.01\\
52.8	0.01\\
52.81	0.01\\
52.82	0.01\\
52.83	0.01\\
52.84	0.01\\
52.85	0.01\\
52.86	0.01\\
52.87	0.01\\
52.88	0.01\\
52.89	0.01\\
52.9	0.01\\
52.91	0.01\\
52.92	0.01\\
52.93	0.01\\
52.94	0.01\\
52.95	0.01\\
52.96	0.01\\
52.97	0.01\\
52.98	0.01\\
52.99	0.01\\
53	0.01\\
53.01	0.01\\
53.02	0.01\\
53.03	0.01\\
53.04	0.01\\
53.05	0.01\\
53.06	0.01\\
53.07	0.01\\
53.08	0.01\\
53.09	0.01\\
53.1	0.01\\
53.11	0.01\\
53.12	0.01\\
53.13	0.01\\
53.14	0.01\\
53.15	0.01\\
53.16	0.01\\
53.17	0.01\\
53.18	0.01\\
53.19	0.01\\
53.2	0.01\\
53.21	0.01\\
53.22	0.01\\
53.23	0.01\\
53.24	0.01\\
53.25	0.01\\
53.26	0.01\\
53.27	0.01\\
53.28	0.01\\
53.29	0.01\\
53.3	0.01\\
53.31	0.01\\
53.32	0.01\\
53.33	0.01\\
53.34	0.01\\
53.35	0.01\\
53.36	0.01\\
53.37	0.01\\
53.38	0.01\\
53.39	0.01\\
53.4	0.01\\
53.41	0.01\\
53.42	0.01\\
53.43	0.01\\
53.44	0.01\\
53.45	0.01\\
53.46	0.01\\
53.47	0.01\\
53.48	0.01\\
53.49	0.01\\
53.5	0.01\\
53.51	0.01\\
53.52	0.01\\
53.53	0.01\\
53.54	0.01\\
53.55	0.01\\
53.56	0.01\\
53.57	0.01\\
53.58	0.01\\
53.59	0.01\\
53.6	0.01\\
53.61	0.01\\
53.62	0.01\\
53.63	0.01\\
53.64	0.01\\
53.65	0.01\\
53.66	0.01\\
53.67	0.01\\
53.68	0.01\\
53.69	0.01\\
53.7	0.01\\
53.71	0.01\\
53.72	0.01\\
53.73	0.01\\
53.74	0.01\\
53.75	0.01\\
53.76	0.01\\
53.77	0.01\\
53.78	0.01\\
53.79	0.01\\
53.8	0.01\\
53.81	0.01\\
53.82	0.01\\
53.83	0.01\\
53.84	0.01\\
53.85	0.01\\
53.86	0.01\\
53.87	0.01\\
53.88	0.01\\
53.89	0.01\\
53.9	0.01\\
53.91	0.01\\
53.92	0.01\\
53.93	0.01\\
53.94	0.01\\
53.95	0.01\\
53.96	0.01\\
53.97	0.01\\
53.98	0.01\\
53.99	0.01\\
54	0.01\\
54.01	0.01\\
54.02	0.01\\
54.03	0.01\\
54.04	0.01\\
54.05	0.01\\
54.06	0.01\\
54.07	0.01\\
54.08	0.01\\
54.09	0.01\\
54.1	0.01\\
54.11	0.01\\
54.12	0.01\\
54.13	0.01\\
54.14	0.01\\
54.15	0.01\\
54.16	0.01\\
54.17	0.01\\
54.18	0.01\\
54.19	0.01\\
54.2	0.01\\
54.21	0.01\\
54.22	0.01\\
54.23	0.01\\
54.24	0.01\\
54.25	0.01\\
54.26	0.01\\
54.27	0.01\\
54.28	0.01\\
54.29	0.01\\
54.3	0.01\\
54.31	0.01\\
54.32	0.01\\
54.33	0.01\\
54.34	0.01\\
54.35	0.01\\
54.36	0.01\\
54.37	0.01\\
54.38	0.01\\
54.39	0.01\\
54.4	0.01\\
54.41	0.01\\
54.42	0.01\\
54.43	0.01\\
54.44	0.01\\
54.45	0.01\\
54.46	0.01\\
54.47	0.01\\
54.48	0.01\\
54.49	0.01\\
54.5	0.01\\
54.51	0.01\\
54.52	0.01\\
54.53	0.01\\
54.54	0.01\\
54.55	0.01\\
54.56	0.01\\
54.57	0.01\\
54.58	0.01\\
54.59	0.01\\
54.6	0.01\\
54.61	0.01\\
54.62	0.01\\
54.63	0.01\\
54.64	0.01\\
54.65	0.01\\
54.66	0.01\\
54.67	0.01\\
54.68	0.01\\
54.69	0.01\\
54.7	0.01\\
54.71	0.01\\
54.72	0.01\\
54.73	0.01\\
54.74	0.01\\
54.75	0.01\\
54.76	0.01\\
54.77	0.01\\
54.78	0.01\\
54.79	0.01\\
54.8	0.01\\
54.81	0.01\\
54.82	0.01\\
54.83	0.01\\
54.84	0.01\\
54.85	0.01\\
54.86	0.01\\
54.87	0.01\\
54.88	0.01\\
54.89	0.01\\
54.9	0.01\\
54.91	0.01\\
54.92	0.01\\
54.93	0.01\\
54.94	0.01\\
54.95	0.01\\
54.96	0.01\\
54.97	0.01\\
54.98	0.01\\
54.99	0.01\\
55	0.01\\
55.01	0.01\\
55.02	0.01\\
55.03	0.01\\
55.04	0.01\\
55.05	0.01\\
55.06	0.01\\
55.07	0.01\\
55.08	0.01\\
55.09	0.01\\
55.1	0.01\\
55.11	0.01\\
55.12	0.01\\
55.13	0.01\\
55.14	0.01\\
55.15	0.01\\
55.16	0.01\\
55.17	0.01\\
55.18	0.01\\
55.19	0.01\\
55.2	0.01\\
55.21	0.01\\
55.22	0.01\\
55.23	0.01\\
55.24	0.01\\
55.25	0.01\\
55.26	0.01\\
55.27	0.01\\
55.28	0.01\\
55.29	0.01\\
55.3	0.01\\
55.31	0.01\\
55.32	0.01\\
55.33	0.01\\
55.34	0.01\\
55.35	0.01\\
55.36	0.01\\
55.37	0.01\\
55.38	0.01\\
55.39	0.01\\
55.4	0.01\\
55.41	0.01\\
55.42	0.01\\
55.43	0.01\\
55.44	0.01\\
55.45	0.01\\
55.46	0.01\\
55.47	0.01\\
55.48	0.01\\
55.49	0.01\\
55.5	0.01\\
55.51	0.01\\
55.52	0.01\\
55.53	0.01\\
55.54	0.01\\
55.55	0.01\\
55.56	0.01\\
55.57	0.01\\
55.58	0.01\\
55.59	0.01\\
55.6	0.01\\
55.61	0.01\\
55.62	0.01\\
55.63	0.01\\
55.64	0.01\\
55.65	0.01\\
55.66	0.01\\
55.67	0.01\\
55.68	0.01\\
55.69	0.01\\
55.7	0.01\\
55.71	0.01\\
55.72	0.01\\
55.73	0.01\\
55.74	0.01\\
55.75	0.01\\
55.76	0.01\\
55.77	0.01\\
55.78	0.01\\
55.79	0.01\\
55.8	0.01\\
55.81	0.01\\
55.82	0.01\\
55.83	0.01\\
55.84	0.01\\
55.85	0.01\\
55.86	0.01\\
55.87	0.01\\
55.88	0.01\\
55.89	0.01\\
55.9	0.01\\
55.91	0.01\\
55.92	0.01\\
55.93	0.01\\
55.94	0.01\\
55.95	0.01\\
55.96	0.01\\
55.97	0.01\\
55.98	0.01\\
55.99	0.01\\
56	0.01\\
56.01	0.01\\
56.02	0.01\\
56.03	0.01\\
56.04	0.01\\
56.05	0.01\\
56.06	0.01\\
56.07	0.01\\
56.08	0.01\\
56.09	0.01\\
56.1	0.01\\
56.11	0.01\\
56.12	0.01\\
56.13	0.01\\
56.14	0.01\\
56.15	0.01\\
56.16	0.01\\
56.17	0.01\\
56.18	0.01\\
56.19	0.01\\
56.2	0.01\\
56.21	0.01\\
56.22	0.01\\
56.23	0.01\\
56.24	0.01\\
56.25	0.01\\
56.26	0.01\\
56.27	0.01\\
56.28	0.01\\
56.29	0.01\\
56.3	0.01\\
56.31	0.01\\
56.32	0.01\\
56.33	0.01\\
56.34	0.01\\
56.35	0.01\\
56.36	0.01\\
56.37	0.01\\
56.38	0.01\\
56.39	0.01\\
56.4	0.01\\
56.41	0.01\\
56.42	0.01\\
56.43	0.01\\
56.44	0.01\\
56.45	0.01\\
56.46	0.01\\
56.47	0.01\\
56.48	0.01\\
56.49	0.01\\
56.5	0.01\\
56.51	0.01\\
56.52	0.01\\
56.53	0.01\\
56.54	0.01\\
56.55	0.01\\
56.56	0.01\\
56.57	0.01\\
56.58	0.01\\
56.59	0.01\\
56.6	0.01\\
56.61	0.01\\
56.62	0.01\\
56.63	0.01\\
56.64	0.01\\
56.65	0.01\\
56.66	0.01\\
56.67	0.01\\
56.68	0.01\\
56.69	0.01\\
56.7	0.01\\
56.71	0.01\\
56.72	0.01\\
56.73	0.01\\
56.74	0.01\\
56.75	0.01\\
56.76	0.01\\
56.77	0.01\\
56.78	0.01\\
56.79	0.01\\
56.8	0.01\\
56.81	0.01\\
56.82	0.01\\
56.83	0.01\\
56.84	0.01\\
56.85	0.01\\
56.86	0.01\\
56.87	0.01\\
56.88	0.01\\
56.89	0.01\\
56.9	0.01\\
56.91	0.01\\
56.92	0.01\\
56.93	0.01\\
56.94	0.01\\
56.95	0.01\\
56.96	0.01\\
56.97	0.01\\
56.98	0.01\\
56.99	0.01\\
57	0.01\\
57.01	0.01\\
57.02	0.01\\
57.03	0.01\\
57.04	0.01\\
57.05	0.01\\
57.06	0.01\\
57.07	0.01\\
57.08	0.01\\
57.09	0.01\\
57.1	0.01\\
57.11	0.01\\
57.12	0.01\\
57.13	0.01\\
57.14	0.01\\
57.15	0.01\\
57.16	0.01\\
57.17	0.01\\
57.18	0.01\\
57.19	0.01\\
57.2	0.01\\
57.21	0.01\\
57.22	0.01\\
57.23	0.01\\
57.24	0.01\\
57.25	0.01\\
57.26	0.01\\
57.27	0.01\\
57.28	0.01\\
57.29	0.01\\
57.3	0.01\\
57.31	0.01\\
57.32	0.01\\
57.33	0.01\\
57.34	0.01\\
57.35	0.01\\
57.36	0.01\\
57.37	0.01\\
57.38	0.01\\
57.39	0.01\\
57.4	0.01\\
57.41	0.01\\
57.42	0.01\\
57.43	0.01\\
57.44	0.01\\
57.45	0.01\\
57.46	0.01\\
57.47	0.01\\
57.48	0.01\\
57.49	0.01\\
57.5	0.01\\
57.51	0.01\\
57.52	0.01\\
57.53	0.01\\
57.54	0.01\\
57.55	0.01\\
57.56	0.01\\
57.57	0.01\\
57.58	0.01\\
57.59	0.01\\
57.6	0.01\\
57.61	0.01\\
57.62	0.01\\
57.63	0.01\\
57.64	0.01\\
57.65	0.01\\
57.66	0.01\\
57.67	0.01\\
57.68	0.01\\
57.69	0.01\\
57.7	0.01\\
57.71	0.01\\
57.72	0.01\\
57.73	0.01\\
57.74	0.01\\
57.75	0.01\\
57.76	0.01\\
57.77	0.01\\
57.78	0.01\\
57.79	0.01\\
57.8	0.01\\
57.81	0.01\\
57.82	0.01\\
57.83	0.01\\
57.84	0.01\\
57.85	0.01\\
57.86	0.01\\
57.87	0.01\\
57.88	0.01\\
57.89	0.01\\
57.9	0.01\\
57.91	0.01\\
57.92	0.01\\
57.93	0.01\\
57.94	0.01\\
57.95	0.01\\
57.96	0.01\\
57.97	0.01\\
57.98	0.01\\
57.99	0.01\\
58	0.01\\
58.01	0.01\\
58.02	0.01\\
58.03	0.01\\
58.04	0.01\\
58.05	0.01\\
58.06	0.01\\
58.07	0.01\\
58.08	0.01\\
58.09	0.01\\
58.1	0.01\\
58.11	0.01\\
58.12	0.01\\
58.13	0.01\\
58.14	0.01\\
58.15	0.01\\
58.16	0.01\\
58.17	0.01\\
58.18	0.01\\
58.19	0.01\\
58.2	0.01\\
58.21	0.01\\
58.22	0.01\\
58.23	0.01\\
58.24	0.01\\
58.25	0.01\\
58.26	0.01\\
58.27	0.01\\
58.28	0.01\\
58.29	0.01\\
58.3	0.01\\
58.31	0.01\\
58.32	0.01\\
58.33	0.01\\
58.34	0.01\\
58.35	0.01\\
58.36	0.01\\
58.37	0.01\\
58.38	0.01\\
58.39	0.01\\
58.4	0.01\\
58.41	0.01\\
58.42	0.01\\
58.43	0.01\\
58.44	0.01\\
58.45	0.01\\
58.46	0.01\\
58.47	0.01\\
58.48	0.01\\
58.49	0.01\\
58.5	0.01\\
58.51	0.01\\
58.52	0.01\\
58.53	0.01\\
58.54	0.01\\
58.55	0.01\\
58.56	0.01\\
58.57	0.01\\
58.58	0.01\\
58.59	0.01\\
58.6	0.01\\
58.61	0.01\\
58.62	0.01\\
58.63	0.01\\
58.64	0.01\\
58.65	0.01\\
58.66	0.01\\
58.67	0.01\\
58.68	0.01\\
58.69	0.01\\
58.7	0.01\\
58.71	0.01\\
58.72	0.01\\
58.73	0.01\\
58.74	0.01\\
58.75	0.01\\
58.76	0.01\\
58.77	0.01\\
58.78	0.01\\
58.79	0.01\\
58.8	0.01\\
58.81	0.01\\
58.82	0.01\\
58.83	0.01\\
58.84	0.01\\
58.85	0.01\\
58.86	0.01\\
58.87	0.01\\
58.88	0.01\\
58.89	0.01\\
58.9	0.01\\
58.91	0.01\\
58.92	0.01\\
58.93	0.01\\
58.94	0.01\\
58.95	0.01\\
58.96	0.01\\
58.97	0.01\\
58.98	0.01\\
58.99	0.01\\
59	0.01\\
59.01	0.01\\
59.02	0.01\\
59.03	0.01\\
59.04	0.01\\
59.05	0.01\\
59.06	0.01\\
59.07	0.01\\
59.08	0.01\\
59.09	0.01\\
59.1	0.01\\
59.11	0.01\\
59.12	0.01\\
59.13	0.01\\
59.14	0.01\\
59.15	0.01\\
59.16	0.01\\
59.17	0.01\\
59.18	0.01\\
59.19	0.01\\
59.2	0.01\\
59.21	0.01\\
59.22	0.01\\
59.23	0.01\\
59.24	0.01\\
59.25	0.01\\
59.26	0.01\\
59.27	0.01\\
59.28	0.01\\
59.29	0.01\\
59.3	0.01\\
59.31	0.01\\
59.32	0.01\\
59.33	0.01\\
59.34	0.01\\
59.35	0.01\\
59.36	0.01\\
59.37	0.01\\
59.38	0.01\\
59.39	0.01\\
59.4	0.01\\
59.41	0.01\\
59.42	0.01\\
59.43	0.01\\
59.44	0.01\\
59.45	0.01\\
59.46	0.01\\
59.47	0.01\\
59.48	0.01\\
59.49	0.01\\
59.5	0.01\\
59.51	0.01\\
59.52	0.01\\
59.53	0.01\\
59.54	0.01\\
59.55	0.01\\
59.56	0.01\\
59.57	0.01\\
59.58	0.01\\
59.59	0.01\\
59.6	0.01\\
59.61	0.01\\
59.62	0.01\\
59.63	0.01\\
59.64	0.01\\
59.65	0.01\\
59.66	0.01\\
59.67	0.01\\
59.68	0.01\\
59.69	0.01\\
59.7	0.01\\
59.71	0.01\\
59.72	0.01\\
59.73	0.01\\
59.74	0.01\\
59.75	0.01\\
59.76	0.01\\
59.77	0.01\\
59.78	0.01\\
59.79	0.01\\
59.8	0.01\\
59.81	0.01\\
59.82	0.01\\
59.83	0.01\\
59.84	0.01\\
59.85	0.01\\
59.86	0.01\\
59.87	0.01\\
59.88	0.01\\
59.89	0.01\\
59.9	0.01\\
59.91	0.01\\
59.92	0.01\\
59.93	0.01\\
59.94	0.01\\
59.95	0.01\\
59.96	0.01\\
59.97	0.01\\
59.98	0.01\\
59.99	0.01\\
60	0.01\\
60.01	0.01\\
60.02	0.01\\
60.03	0.01\\
60.04	0.01\\
60.05	0.01\\
60.06	0.01\\
60.07	0.01\\
60.08	0.01\\
60.09	0.01\\
60.1	0.01\\
60.11	0.01\\
60.12	0.01\\
60.13	0.01\\
60.14	0.01\\
60.15	0.01\\
60.16	0.01\\
60.17	0.01\\
60.18	0.01\\
60.19	0.01\\
60.2	0.01\\
60.21	0.01\\
60.22	0.01\\
60.23	0.01\\
60.24	0.01\\
60.25	0.01\\
60.26	0.01\\
60.27	0.01\\
60.28	0.01\\
60.29	0.01\\
60.3	0.01\\
60.31	0.01\\
60.32	0.01\\
60.33	0.01\\
60.34	0.01\\
60.35	0.01\\
60.36	0.01\\
60.37	0.01\\
60.38	0.01\\
60.39	0.01\\
60.4	0.01\\
60.41	0.01\\
60.42	0.01\\
60.43	0.01\\
60.44	0.01\\
60.45	0.01\\
60.46	0.01\\
60.47	0.01\\
60.48	0.01\\
60.49	0.01\\
60.5	0.01\\
60.51	0.01\\
60.52	0.01\\
60.53	0.01\\
60.54	0.01\\
60.55	0.01\\
60.56	0.01\\
60.57	0.01\\
60.58	0.01\\
60.59	0.01\\
60.6	0.01\\
60.61	0.01\\
60.62	0.01\\
60.63	0.01\\
60.64	0.01\\
60.65	0.01\\
60.66	0.01\\
60.67	0.01\\
60.68	0.01\\
60.69	0.01\\
60.7	0.01\\
60.71	0.01\\
60.72	0.01\\
60.73	0.01\\
60.74	0.01\\
60.75	0.01\\
60.76	0.01\\
60.77	0.01\\
60.78	0.01\\
60.79	0.01\\
60.8	0.01\\
60.81	0.01\\
60.82	0.01\\
60.83	0.01\\
60.84	0.01\\
60.85	0.01\\
60.86	0.01\\
60.87	0.01\\
60.88	0.01\\
60.89	0.01\\
60.9	0.01\\
60.91	0.01\\
60.92	0.01\\
60.93	0.01\\
60.94	0.01\\
60.95	0.01\\
60.96	0.01\\
60.97	0.01\\
60.98	0.01\\
60.99	0.01\\
61	0.01\\
61.01	0.01\\
61.02	0.01\\
61.03	0.01\\
61.04	0.01\\
61.05	0.01\\
61.06	0.01\\
61.07	0.01\\
61.08	0.01\\
61.09	0.01\\
61.1	0.01\\
61.11	0.01\\
61.12	0.01\\
61.13	0.01\\
61.14	0.01\\
61.15	0.01\\
61.16	0.01\\
61.17	0.01\\
61.18	0.01\\
61.19	0.01\\
61.2	0.01\\
61.21	0.01\\
61.22	0.01\\
61.23	0.01\\
61.24	0.01\\
61.25	0.01\\
61.26	0.01\\
61.27	0.01\\
61.28	0.01\\
61.29	0.01\\
61.3	0.01\\
61.31	0.01\\
61.32	0.01\\
61.33	0.01\\
61.34	0.01\\
61.35	0.01\\
61.36	0.01\\
61.37	0.01\\
61.38	0.01\\
61.39	0.01\\
61.4	0.01\\
61.41	0.01\\
61.42	0.01\\
61.43	0.01\\
61.44	0.01\\
61.45	0.01\\
61.46	0.01\\
61.47	0.01\\
61.48	0.01\\
61.49	0.01\\
61.5	0.01\\
61.51	0.01\\
61.52	0.01\\
61.53	0.01\\
61.54	0.01\\
61.55	0.01\\
61.56	0.01\\
61.57	0.01\\
61.58	0.01\\
61.59	0.01\\
61.6	0.01\\
61.61	0.01\\
61.62	0.01\\
61.63	0.01\\
61.64	0.01\\
61.65	0.01\\
61.66	0.01\\
61.67	0.01\\
61.68	0.01\\
61.69	0.01\\
61.7	0.01\\
61.71	0.01\\
61.72	0.01\\
61.73	0.01\\
61.74	0.01\\
61.75	0.01\\
61.76	0.01\\
61.77	0.01\\
61.78	0.01\\
61.79	0.01\\
61.8	0.01\\
61.81	0.01\\
61.82	0.01\\
61.83	0.01\\
61.84	0.01\\
61.85	0.01\\
61.86	0.01\\
61.87	0.01\\
61.88	0.01\\
61.89	0.01\\
61.9	0.01\\
61.91	0.01\\
61.92	0.01\\
61.93	0.01\\
61.94	0.01\\
61.95	0.01\\
61.96	0.01\\
61.97	0.01\\
61.98	0.01\\
61.99	0.01\\
62	0.01\\
62.01	0.01\\
62.02	0.01\\
62.03	0.01\\
62.04	0.01\\
62.05	0.01\\
62.06	0.01\\
62.07	0.01\\
62.08	0.01\\
62.09	0.01\\
62.1	0.01\\
62.11	0.01\\
62.12	0.01\\
62.13	0.01\\
62.14	0.01\\
62.15	0.01\\
62.16	0.01\\
62.17	0.01\\
62.18	0.01\\
62.19	0.01\\
62.2	0.01\\
62.21	0.01\\
62.22	0.01\\
62.23	0.01\\
62.24	0.01\\
62.25	0.01\\
62.26	0.01\\
62.27	0.01\\
62.28	0.01\\
62.29	0.01\\
62.3	0.01\\
62.31	0.01\\
62.32	0.01\\
62.33	0.01\\
62.34	0.01\\
62.35	0.01\\
62.36	0.01\\
62.37	0.01\\
62.38	0.01\\
62.39	0.01\\
62.4	0.01\\
62.41	0.01\\
62.42	0.01\\
62.43	0.01\\
62.44	0.01\\
62.45	0.01\\
62.46	0.01\\
62.47	0.01\\
62.48	0.01\\
62.49	0.01\\
62.5	0.01\\
62.51	0.01\\
62.52	0.01\\
62.53	0.01\\
62.54	0.01\\
62.55	0.01\\
62.56	0.01\\
62.57	0.01\\
62.58	0.01\\
62.59	0.01\\
62.6	0.01\\
62.61	0.01\\
62.62	0.01\\
62.63	0.01\\
62.64	0.01\\
62.65	0.01\\
62.66	0.01\\
62.67	0.01\\
62.68	0.01\\
62.69	0.01\\
62.7	0.01\\
62.71	0.01\\
62.72	0.01\\
62.73	0.01\\
62.74	0.01\\
62.75	0.01\\
62.76	0.01\\
62.77	0.01\\
62.78	0.01\\
62.79	0.01\\
62.8	0.01\\
62.81	0.01\\
62.82	0.01\\
62.83	0.01\\
62.84	0.01\\
62.85	0.01\\
62.86	0.01\\
62.87	0.01\\
62.88	0.01\\
62.89	0.01\\
62.9	0.01\\
62.91	0.01\\
62.92	0.01\\
62.93	0.01\\
62.94	0.01\\
62.95	0.01\\
62.96	0.01\\
62.97	0.01\\
62.98	0.01\\
62.99	0.01\\
63	0.01\\
63.01	0.01\\
63.02	0.01\\
63.03	0.01\\
63.04	0.01\\
63.05	0.01\\
63.06	0.01\\
63.07	0.01\\
63.08	0.01\\
63.09	0.01\\
63.1	0.01\\
63.11	0.01\\
63.12	0.01\\
63.13	0.01\\
63.14	0.01\\
63.15	0.01\\
63.16	0.01\\
63.17	0.01\\
63.18	0.01\\
63.19	0.01\\
63.2	0.01\\
63.21	0.01\\
63.22	0.01\\
63.23	0.01\\
63.24	0.01\\
63.25	0.01\\
63.26	0.01\\
63.27	0.01\\
63.28	0.01\\
63.29	0.01\\
63.3	0.01\\
63.31	0.01\\
63.32	0.01\\
63.33	0.01\\
63.34	0.01\\
63.35	0.01\\
63.36	0.01\\
63.37	0.01\\
63.38	0.01\\
63.39	0.01\\
63.4	0.01\\
63.41	0.01\\
63.42	0.01\\
63.43	0.01\\
63.44	0.01\\
63.45	0.01\\
63.46	0.01\\
63.47	0.01\\
63.48	0.01\\
63.49	0.01\\
63.5	0.01\\
63.51	0.01\\
63.52	0.01\\
63.53	0.01\\
63.54	0.01\\
63.55	0.01\\
63.56	0.01\\
63.57	0.01\\
63.58	0.01\\
63.59	0.01\\
63.6	0.01\\
63.61	0.01\\
63.62	0.01\\
63.63	0.01\\
63.64	0.01\\
63.65	0.01\\
63.66	0.01\\
63.67	0.01\\
63.68	0.01\\
63.69	0.01\\
63.7	0.01\\
63.71	0.01\\
63.72	0.01\\
63.73	0.01\\
63.74	0.01\\
63.75	0.01\\
63.76	0.01\\
63.77	0.01\\
63.78	0.01\\
63.79	0.01\\
63.8	0.01\\
63.81	0.01\\
63.82	0.01\\
63.83	0.01\\
63.84	0.01\\
63.85	0.01\\
63.86	0.01\\
63.87	0.01\\
63.88	0.01\\
63.89	0.01\\
63.9	0.01\\
63.91	0.01\\
63.92	0.01\\
63.93	0.01\\
63.94	0.01\\
63.95	0.01\\
63.96	0.01\\
63.97	0.01\\
63.98	0.01\\
63.99	0.01\\
64	0.01\\
64.01	0.01\\
64.02	0.01\\
64.03	0.01\\
64.04	0.01\\
64.05	0.01\\
64.06	0.01\\
64.07	0.01\\
64.08	0.01\\
64.09	0.01\\
64.1	0.01\\
64.11	0.01\\
64.12	0.01\\
64.13	0.01\\
64.14	0.01\\
64.15	0.01\\
64.16	0.01\\
64.17	0.01\\
64.18	0.01\\
64.19	0.01\\
64.2	0.01\\
64.21	0.01\\
64.22	0.01\\
64.23	0.01\\
64.24	0.01\\
64.25	0.01\\
64.26	0.01\\
64.27	0.01\\
64.28	0.01\\
64.29	0.01\\
64.3	0.01\\
64.31	0.01\\
64.32	0.01\\
64.33	0.01\\
64.34	0.01\\
64.35	0.01\\
64.36	0.01\\
64.37	0.01\\
64.38	0.01\\
64.39	0.01\\
64.4	0.01\\
64.41	0.01\\
64.42	0.01\\
64.43	0.01\\
64.44	0.01\\
64.45	0.01\\
64.46	0.01\\
64.47	0.01\\
64.48	0.01\\
64.49	0.01\\
64.5	0.01\\
64.51	0.01\\
64.52	0.01\\
64.53	0.01\\
64.54	0.01\\
64.55	0.01\\
64.56	0.01\\
64.57	0.01\\
64.58	0.01\\
64.59	0.01\\
64.6	0.01\\
64.61	0.01\\
64.62	0.01\\
64.63	0.01\\
64.64	0.01\\
64.65	0.01\\
64.66	0.01\\
64.67	0.01\\
64.68	0.01\\
64.69	0.01\\
64.7	0.01\\
64.71	0.01\\
64.72	0.01\\
64.73	0.01\\
64.74	0.01\\
64.75	0.01\\
64.76	0.01\\
64.77	0.01\\
64.78	0.01\\
64.79	0.01\\
64.8	0.01\\
64.81	0.01\\
64.82	0.01\\
64.83	0.01\\
64.84	0.01\\
64.85	0.01\\
64.86	0.01\\
64.87	0.01\\
64.88	0.01\\
64.89	0.01\\
64.9	0.01\\
64.91	0.01\\
64.92	0.01\\
64.93	0.01\\
64.94	0.01\\
64.95	0.01\\
64.96	0.01\\
64.97	0.01\\
64.98	0.01\\
64.99	0.01\\
65	0.01\\
65.01	0.01\\
65.02	0.01\\
65.03	0.01\\
65.04	0.01\\
65.05	0.01\\
65.06	0.01\\
65.07	0.01\\
65.08	0.01\\
65.09	0.01\\
65.1	0.01\\
65.11	0.01\\
65.12	0.01\\
65.13	0.01\\
65.14	0.01\\
65.15	0.01\\
65.16	0.01\\
65.17	0.01\\
65.18	0.01\\
65.19	0.01\\
65.2	0.01\\
65.21	0.01\\
65.22	0.01\\
65.23	0.01\\
65.24	0.01\\
65.25	0.01\\
65.26	0.01\\
65.27	0.01\\
65.28	0.01\\
65.29	0.01\\
65.3	0.01\\
65.31	0.01\\
65.32	0.01\\
65.33	0.01\\
65.34	0.01\\
65.35	0.01\\
65.36	0.01\\
65.37	0.01\\
65.38	0.01\\
65.39	0.01\\
65.4	0.01\\
65.41	0.01\\
65.42	0.01\\
65.43	0.01\\
65.44	0.01\\
65.45	0.01\\
65.46	0.01\\
65.47	0.01\\
65.48	0.01\\
65.49	0.01\\
65.5	0.01\\
65.51	0.01\\
65.52	0.01\\
65.53	0.01\\
65.54	0.01\\
65.55	0.01\\
65.56	0.01\\
65.57	0.01\\
65.58	0.01\\
65.59	0.01\\
65.6	0.01\\
65.61	0.01\\
65.62	0.01\\
65.63	0.01\\
65.64	0.01\\
65.65	0.01\\
65.66	0.01\\
65.67	0.01\\
65.68	0.01\\
65.69	0.01\\
65.7	0.01\\
65.71	0.01\\
65.72	0.01\\
65.73	0.01\\
65.74	0.01\\
65.75	0.01\\
65.76	0.01\\
65.77	0.01\\
65.78	0.01\\
65.79	0.01\\
65.8	0.01\\
65.81	0.01\\
65.82	0.01\\
65.83	0.01\\
65.84	0.01\\
65.85	0.01\\
65.86	0.01\\
65.87	0.01\\
65.88	0.01\\
65.89	0.01\\
65.9	0.01\\
65.91	0.01\\
65.92	0.01\\
65.93	0.01\\
65.94	0.01\\
65.95	0.01\\
65.96	0.01\\
65.97	0.01\\
65.98	0.01\\
65.99	0.01\\
66	0.01\\
66.01	0.01\\
66.02	0.01\\
66.03	0.01\\
66.04	0.01\\
66.05	0.01\\
66.06	0.01\\
66.07	0.01\\
66.08	0.01\\
66.09	0.01\\
66.1	0.01\\
66.11	0.01\\
66.12	0.01\\
66.13	0.01\\
66.14	0.01\\
66.15	0.01\\
66.16	0.01\\
66.17	0.01\\
66.18	0.01\\
66.19	0.01\\
66.2	0.01\\
66.21	0.01\\
66.22	0.01\\
66.23	0.01\\
66.24	0.01\\
66.25	0.01\\
66.26	0.01\\
66.27	0.01\\
66.28	0.01\\
66.29	0.01\\
66.3	0.01\\
66.31	0.01\\
66.32	0.01\\
66.33	0.01\\
66.34	0.01\\
66.35	0.01\\
66.36	0.01\\
66.37	0.01\\
66.38	0.01\\
66.39	0.01\\
66.4	0.01\\
66.41	0.01\\
66.42	0.01\\
66.43	0.01\\
66.44	0.01\\
66.45	0.01\\
66.46	0.01\\
66.47	0.01\\
66.48	0.01\\
66.49	0.01\\
66.5	0.01\\
66.51	0.01\\
66.52	0.01\\
66.53	0.01\\
66.54	0.01\\
66.55	0.01\\
66.56	0.01\\
66.57	0.01\\
66.58	0.01\\
66.59	0.01\\
66.6	0.01\\
66.61	0.01\\
66.62	0.01\\
66.63	0.01\\
66.64	0.01\\
66.65	0.01\\
66.66	0.01\\
66.67	0.01\\
66.68	0.01\\
66.69	0.01\\
66.7	0.01\\
66.71	0.01\\
66.72	0.01\\
66.73	0.01\\
66.74	0.01\\
66.75	0.01\\
66.76	0.01\\
66.77	0.01\\
66.78	0.01\\
66.79	0.01\\
66.8	0.01\\
66.81	0.01\\
66.82	0.01\\
66.83	0.01\\
66.84	0.01\\
66.85	0.01\\
66.86	0.01\\
66.87	0.01\\
66.88	0.01\\
66.89	0.01\\
66.9	0.01\\
66.91	0.01\\
66.92	0.01\\
66.93	0.01\\
66.94	0.01\\
66.95	0.01\\
66.96	0.01\\
66.97	0.01\\
66.98	0.01\\
66.99	0.01\\
67	0.01\\
67.01	0.01\\
67.02	0.01\\
67.03	0.01\\
67.04	0.01\\
67.05	0.01\\
67.06	0.01\\
67.07	0.01\\
67.08	0.01\\
67.09	0.01\\
67.1	0.01\\
67.11	0.01\\
67.12	0.01\\
67.13	0.01\\
67.14	0.01\\
67.15	0.01\\
67.16	0.01\\
67.17	0.01\\
67.18	0.01\\
67.19	0.01\\
67.2	0.01\\
67.21	0.01\\
67.22	0.01\\
67.23	0.01\\
67.24	0.01\\
67.25	0.01\\
67.26	0.01\\
67.27	0.01\\
67.28	0.01\\
67.29	0.01\\
67.3	0.01\\
67.31	0.01\\
67.32	0.01\\
67.33	0.01\\
67.34	0.01\\
67.35	0.01\\
67.36	0.01\\
67.37	0.01\\
67.38	0.01\\
67.39	0.01\\
67.4	0.01\\
67.41	0.01\\
67.42	0.01\\
67.43	0.01\\
67.44	0.01\\
67.45	0.01\\
67.46	0.01\\
67.47	0.01\\
67.48	0.01\\
67.49	0.01\\
67.5	0.01\\
67.51	0.01\\
67.52	0.01\\
67.53	0.01\\
67.54	0.01\\
67.55	0.01\\
67.56	0.01\\
67.57	0.01\\
67.58	0.01\\
67.59	0.01\\
67.6	0.01\\
67.61	0.01\\
67.62	0.01\\
67.63	0.01\\
67.64	0.01\\
67.65	0.01\\
67.66	0.01\\
67.67	0.01\\
67.68	0.01\\
67.69	0.01\\
67.7	0.01\\
67.71	0.01\\
67.72	0.01\\
67.73	0.01\\
67.74	0.01\\
67.75	0.01\\
67.76	0.01\\
67.77	0.01\\
67.78	0.01\\
67.79	0.01\\
67.8	0.01\\
67.81	0.01\\
67.82	0.01\\
67.83	0.01\\
67.84	0.01\\
67.85	0.01\\
67.86	0.01\\
67.87	0.01\\
67.88	0.01\\
67.89	0.01\\
67.9	0.01\\
67.91	0.01\\
67.92	0.01\\
67.93	0.01\\
67.94	0.01\\
67.95	0.01\\
67.96	0.01\\
67.97	0.01\\
67.98	0.01\\
67.99	0.01\\
68	0.01\\
68.01	0.01\\
68.02	0.01\\
68.03	0.01\\
68.04	0.01\\
68.05	0.01\\
68.06	0.01\\
68.07	0.01\\
68.08	0.01\\
68.09	0.01\\
68.1	0.01\\
68.11	0.01\\
68.12	0.01\\
68.13	0.01\\
68.14	0.01\\
68.15	0.01\\
68.16	0.01\\
68.17	0.01\\
68.18	0.01\\
68.19	0.01\\
68.2	0.01\\
68.21	0.01\\
68.22	0.01\\
68.23	0.01\\
68.24	0.01\\
68.25	0.01\\
68.26	0.01\\
68.27	0.01\\
68.28	0.01\\
68.29	0.01\\
68.3	0.01\\
68.31	0.01\\
68.32	0.01\\
68.33	0.01\\
68.34	0.01\\
68.35	0.01\\
68.36	0.01\\
68.37	0.01\\
68.38	0.01\\
68.39	0.01\\
68.4	0.01\\
68.41	0.01\\
68.42	0.01\\
68.43	0.01\\
68.44	0.01\\
68.45	0.01\\
68.46	0.01\\
68.47	0.01\\
68.48	0.01\\
68.49	0.01\\
68.5	0.01\\
68.51	0.01\\
68.52	0.01\\
68.53	0.01\\
68.54	0.01\\
68.55	0.01\\
68.56	0.01\\
68.57	0.01\\
68.58	0.01\\
68.59	0.01\\
68.6	0.01\\
68.61	0.01\\
68.62	0.01\\
68.63	0.01\\
68.64	0.01\\
68.65	0.01\\
68.66	0.01\\
68.67	0.01\\
68.68	0.01\\
68.69	0.01\\
68.7	0.01\\
68.71	0.01\\
68.72	0.01\\
68.73	0.01\\
68.74	0.01\\
68.75	0.01\\
68.76	0.01\\
68.77	0.01\\
68.78	0.01\\
68.79	0.01\\
68.8	0.01\\
68.81	0.01\\
68.82	0.01\\
68.83	0.01\\
68.84	0.01\\
68.85	0.01\\
68.86	0.01\\
68.87	0.01\\
68.88	0.01\\
68.89	0.01\\
68.9	0.01\\
68.91	0.01\\
68.92	0.01\\
68.93	0.01\\
68.94	0.01\\
68.95	0.01\\
68.96	0.01\\
68.97	0.01\\
68.98	0.01\\
68.99	0.01\\
69	0.01\\
69.01	0.01\\
69.02	0.01\\
69.03	0.01\\
69.04	0.01\\
69.05	0.01\\
69.06	0.01\\
69.07	0.01\\
69.08	0.01\\
69.09	0.01\\
69.1	0.01\\
69.11	0.01\\
69.12	0.01\\
69.13	0.01\\
69.14	0.01\\
69.15	0.01\\
69.16	0.01\\
69.17	0.01\\
69.18	0.01\\
69.19	0.01\\
69.2	0.01\\
69.21	0.01\\
69.22	0.01\\
69.23	0.01\\
69.24	0.01\\
69.25	0.01\\
69.26	0.01\\
69.27	0.01\\
69.28	0.01\\
69.29	0.01\\
69.3	0.01\\
69.31	0.01\\
69.32	0.01\\
69.33	0.01\\
69.34	0.01\\
69.35	0.01\\
69.36	0.01\\
69.37	0.01\\
69.38	0.01\\
69.39	0.01\\
69.4	0.01\\
69.41	0.01\\
69.42	0.01\\
69.43	0.01\\
69.44	0.01\\
69.45	0.01\\
69.46	0.01\\
69.47	0.01\\
69.48	0.01\\
69.49	0.01\\
69.5	0.01\\
69.51	0.01\\
69.52	0.01\\
69.53	0.01\\
69.54	0.01\\
69.55	0.01\\
69.56	0.01\\
69.57	0.01\\
69.58	0.01\\
69.59	0.01\\
69.6	0.01\\
69.61	0.01\\
69.62	0.01\\
69.63	0.01\\
69.64	0.01\\
69.65	0.01\\
69.66	0.01\\
69.67	0.01\\
69.68	0.01\\
69.69	0.01\\
69.7	0.01\\
69.71	0.01\\
69.72	0.01\\
69.73	0.01\\
69.74	0.01\\
69.75	0.01\\
69.76	0.01\\
69.77	0.01\\
69.78	0.01\\
69.79	0.01\\
69.8	0.01\\
69.81	0.01\\
69.82	0.01\\
69.83	0.01\\
69.84	0.01\\
69.85	0.01\\
69.86	0.01\\
69.87	0.01\\
69.88	0.01\\
69.89	0.01\\
69.9	0.01\\
69.91	0.01\\
69.92	0.01\\
69.93	0.01\\
69.94	0.01\\
69.95	0.01\\
69.96	0.01\\
69.97	0.01\\
69.98	0.01\\
69.99	0.01\\
70	0.01\\
70.01	0.01\\
70.02	0.01\\
70.03	0.01\\
70.04	0.01\\
70.05	0.01\\
70.06	0.01\\
70.07	0.01\\
70.08	0.01\\
70.09	0.01\\
70.1	0.01\\
70.11	0.01\\
70.12	0.01\\
70.13	0.01\\
70.14	0.01\\
70.15	0.01\\
70.16	0.01\\
70.17	0.01\\
70.18	0.01\\
70.19	0.01\\
70.2	0.01\\
70.21	0.01\\
70.22	0.01\\
70.23	0.01\\
70.24	0.01\\
70.25	0.01\\
70.26	0.01\\
70.27	0.01\\
70.28	0.01\\
70.29	0.01\\
70.3	0.01\\
70.31	0.01\\
70.32	0.01\\
70.33	0.01\\
70.34	0.01\\
70.35	0.01\\
70.36	0.01\\
70.37	0.01\\
70.38	0.01\\
70.39	0.01\\
70.4	0.01\\
70.41	0.01\\
70.42	0.01\\
70.43	0.01\\
70.44	0.01\\
70.45	0.01\\
70.46	0.01\\
70.47	0.01\\
70.48	0.01\\
70.49	0.01\\
70.5	0.01\\
70.51	0.01\\
70.52	0.01\\
70.53	0.01\\
70.54	0.01\\
70.55	0.01\\
70.56	0.01\\
70.57	0.01\\
70.58	0.01\\
70.59	0.01\\
70.6	0.01\\
70.61	0.01\\
70.62	0.01\\
70.63	0.01\\
70.64	0.01\\
70.65	0.01\\
70.66	0.01\\
70.67	0.01\\
70.68	0.01\\
70.69	0.01\\
70.7	0.01\\
70.71	0.01\\
70.72	0.01\\
70.73	0.01\\
70.74	0.01\\
70.75	0.01\\
70.76	0.01\\
70.77	0.01\\
70.78	0.01\\
70.79	0.01\\
70.8	0.01\\
70.81	0.01\\
70.82	0.01\\
70.83	0.01\\
70.84	0.01\\
70.85	0.01\\
70.86	0.01\\
70.87	0.01\\
70.88	0.01\\
70.89	0.01\\
70.9	0.01\\
70.91	0.01\\
70.92	0.01\\
70.93	0.01\\
70.94	0.01\\
70.95	0.01\\
70.96	0.01\\
70.97	0.01\\
70.98	0.01\\
70.99	0.01\\
71	0.01\\
71.01	0.01\\
71.02	0.01\\
71.03	0.01\\
71.04	0.01\\
71.05	0.01\\
71.06	0.01\\
71.07	0.01\\
71.08	0.01\\
71.09	0.01\\
71.1	0.01\\
71.11	0.01\\
71.12	0.01\\
71.13	0.01\\
71.14	0.01\\
71.15	0.01\\
71.16	0.01\\
71.17	0.01\\
71.18	0.01\\
71.19	0.01\\
71.2	0.01\\
71.21	0.01\\
71.22	0.01\\
71.23	0.01\\
71.24	0.01\\
71.25	0.01\\
71.26	0.01\\
71.27	0.01\\
71.28	0.01\\
71.29	0.01\\
71.3	0.01\\
71.31	0.01\\
71.32	0.01\\
71.33	0.01\\
71.34	0.01\\
71.35	0.01\\
71.36	0.01\\
71.37	0.01\\
71.38	0.01\\
71.39	0.01\\
71.4	0.01\\
71.41	0.01\\
71.42	0.01\\
71.43	0.01\\
71.44	0.01\\
71.45	0.01\\
71.46	0.01\\
71.47	0.01\\
71.48	0.01\\
71.49	0.01\\
71.5	0.01\\
71.51	0.01\\
71.52	0.01\\
71.53	0.01\\
71.54	0.01\\
71.55	0.01\\
71.56	0.01\\
71.57	0.01\\
71.58	0.01\\
71.59	0.01\\
71.6	0.01\\
71.61	0.01\\
71.62	0.01\\
71.63	0.01\\
71.64	0.01\\
71.65	0.01\\
71.66	0.01\\
71.67	0.01\\
71.68	0.01\\
71.69	0.01\\
71.7	0.01\\
71.71	0.01\\
71.72	0.01\\
71.73	0.01\\
71.74	0.01\\
71.75	0.01\\
71.76	0.01\\
71.77	0.01\\
71.78	0.01\\
71.79	0.01\\
71.8	0.01\\
71.81	0.01\\
71.82	0.01\\
71.83	0.01\\
71.84	0.01\\
71.85	0.01\\
71.86	0.01\\
71.87	0.01\\
71.88	0.01\\
71.89	0.01\\
71.9	0.01\\
71.91	0.01\\
71.92	0.01\\
71.93	0.01\\
71.94	0.01\\
71.95	0.01\\
71.96	0.01\\
71.97	0.01\\
71.98	0.01\\
71.99	0.01\\
72	0.01\\
72.01	0.01\\
72.02	0.01\\
72.03	0.01\\
72.04	0.01\\
72.05	0.01\\
72.06	0.01\\
72.07	0.01\\
72.08	0.01\\
72.09	0.01\\
72.1	0.01\\
72.11	0.01\\
72.12	0.01\\
72.13	0.01\\
72.14	0.01\\
72.15	0.01\\
72.16	0.01\\
72.17	0.01\\
72.18	0.01\\
72.19	0.01\\
72.2	0.01\\
72.21	0.01\\
72.22	0.01\\
72.23	0.01\\
72.24	0.01\\
72.25	0.01\\
72.26	0.01\\
72.27	0.01\\
72.28	0.01\\
72.29	0.01\\
72.3	0.01\\
72.31	0.01\\
72.32	0.01\\
72.33	0.01\\
72.34	0.01\\
72.35	0.01\\
72.36	0.01\\
72.37	0.01\\
72.38	0.01\\
72.39	0.01\\
72.4	0.01\\
72.41	0.01\\
72.42	0.01\\
72.43	0.01\\
72.44	0.01\\
72.45	0.01\\
72.46	0.01\\
72.47	0.01\\
72.48	0.01\\
72.49	0.01\\
72.5	0.01\\
72.51	0.01\\
72.52	0.01\\
72.53	0.01\\
72.54	0.01\\
72.55	0.01\\
72.56	0.01\\
72.57	0.01\\
72.58	0.01\\
72.59	0.01\\
72.6	0.01\\
72.61	0.01\\
72.62	0.01\\
72.63	0.01\\
72.64	0.01\\
72.65	0.01\\
72.66	0.01\\
72.67	0.01\\
72.68	0.01\\
72.69	0.01\\
72.7	0.01\\
72.71	0.01\\
72.72	0.01\\
72.73	0.01\\
72.74	0.01\\
72.75	0.01\\
72.76	0.01\\
72.77	0.01\\
72.78	0.01\\
72.79	0.01\\
72.8	0.01\\
72.81	0.01\\
72.82	0.01\\
72.83	0.01\\
72.84	0.01\\
72.85	0.01\\
72.86	0.01\\
72.87	0.01\\
72.88	0.01\\
72.89	0.01\\
72.9	0.01\\
72.91	0.01\\
72.92	0.01\\
72.93	0.01\\
72.94	0.01\\
72.95	0.01\\
72.96	0.01\\
72.97	0.01\\
72.98	0.01\\
72.99	0.01\\
73	0.01\\
73.01	0.01\\
73.02	0.01\\
73.03	0.01\\
73.04	0.01\\
73.05	0.01\\
73.06	0.01\\
73.07	0.01\\
73.08	0.01\\
73.09	0.01\\
73.1	0.01\\
73.11	0.01\\
73.12	0.01\\
73.13	0.01\\
73.14	0.01\\
73.15	0.01\\
73.16	0.01\\
73.17	0.01\\
73.18	0.01\\
73.19	0.01\\
73.2	0.01\\
73.21	0.01\\
73.22	0.01\\
73.23	0.01\\
73.24	0.01\\
73.25	0.01\\
73.26	0.01\\
73.27	0.01\\
73.28	0.01\\
73.29	0.01\\
73.3	0.01\\
73.31	0.01\\
73.32	0.01\\
73.33	0.01\\
73.34	0.01\\
73.35	0.01\\
73.36	0.01\\
73.37	0.01\\
73.38	0.01\\
73.39	0.01\\
73.4	0.01\\
73.41	0.01\\
73.42	0.01\\
73.43	0.01\\
73.44	0.01\\
73.45	0.01\\
73.46	0.01\\
73.47	0.01\\
73.48	0.01\\
73.49	0.01\\
73.5	0.01\\
73.51	0.01\\
73.52	0.01\\
73.53	0.01\\
73.54	0.01\\
73.55	0.01\\
73.56	0.01\\
73.57	0.01\\
73.58	0.01\\
73.59	0.01\\
73.6	0.01\\
73.61	0.01\\
73.62	0.01\\
73.63	0.01\\
73.64	0.01\\
73.65	0.01\\
73.66	0.01\\
73.67	0.01\\
73.68	0.01\\
73.69	0.01\\
73.7	0.01\\
73.71	0.01\\
73.72	0.01\\
73.73	0.01\\
73.74	0.01\\
73.75	0.01\\
73.76	0.01\\
73.77	0.01\\
73.78	0.01\\
73.79	0.01\\
73.8	0.01\\
73.81	0.01\\
73.82	0.01\\
73.83	0.01\\
73.84	0.01\\
73.85	0.01\\
73.86	0.01\\
73.87	0.01\\
73.88	0.01\\
73.89	0.01\\
73.9	0.01\\
73.91	0.01\\
73.92	0.01\\
73.93	0.01\\
73.94	0.01\\
73.95	0.01\\
73.96	0.01\\
73.97	0.01\\
73.98	0.01\\
73.99	0.01\\
74	0.01\\
74.01	0.01\\
74.02	0.01\\
74.03	0.01\\
74.04	0.01\\
74.05	0.01\\
74.06	0.01\\
74.07	0.01\\
74.08	0.01\\
74.09	0.01\\
74.1	0.01\\
74.11	0.01\\
74.12	0.01\\
74.13	0.01\\
74.14	0.01\\
74.15	0.01\\
74.16	0.01\\
74.17	0.01\\
74.18	0.01\\
74.19	0.01\\
74.2	0.01\\
74.21	0.01\\
74.22	0.01\\
74.23	0.01\\
74.24	0.01\\
74.25	0.01\\
74.26	0.01\\
74.27	0.01\\
74.28	0.01\\
74.29	0.01\\
74.3	0.01\\
74.31	0.01\\
74.32	0.01\\
74.33	0.01\\
74.34	0.01\\
74.35	0.01\\
74.36	0.01\\
74.37	0.01\\
74.38	0.01\\
74.39	0.01\\
74.4	0.01\\
74.41	0.01\\
74.42	0.01\\
74.43	0.01\\
74.44	0.01\\
74.45	0.01\\
74.46	0.01\\
74.47	0.01\\
74.48	0.01\\
74.49	0.01\\
74.5	0.01\\
74.51	0.01\\
74.52	0.01\\
74.53	0.01\\
74.54	0.01\\
74.55	0.01\\
74.56	0.01\\
74.57	0.01\\
74.58	0.01\\
74.59	0.01\\
74.6	0.01\\
74.61	0.01\\
74.62	0.01\\
74.63	0.01\\
74.64	0.01\\
74.65	0.01\\
74.66	0.01\\
74.67	0.01\\
74.68	0.01\\
74.69	0.01\\
74.7	0.01\\
74.71	0.01\\
74.72	0.01\\
74.73	0.01\\
74.74	0.01\\
74.75	0.01\\
74.76	0.01\\
74.77	0.01\\
74.78	0.01\\
74.79	0.01\\
74.8	0.01\\
74.81	0.01\\
74.82	0.01\\
74.83	0.01\\
74.84	0.01\\
74.85	0.01\\
74.86	0.01\\
74.87	0.01\\
74.88	0.01\\
74.89	0.01\\
74.9	0.01\\
74.91	0.01\\
74.92	0.01\\
74.93	0.01\\
74.94	0.01\\
74.95	0.01\\
74.96	0.01\\
74.97	0.01\\
74.98	0.01\\
74.99	0.01\\
75	0.01\\
75.01	0.01\\
75.02	0.01\\
75.03	0.01\\
75.04	0.01\\
75.05	0.01\\
75.06	0.01\\
75.07	0.01\\
75.08	0.01\\
75.09	0.01\\
75.1	0.01\\
75.11	0.01\\
75.12	0.01\\
75.13	0.01\\
75.14	0.01\\
75.15	0.01\\
75.16	0.01\\
75.17	0.01\\
75.18	0.01\\
75.19	0.01\\
75.2	0.01\\
75.21	0.01\\
75.22	0.01\\
75.23	0.01\\
75.24	0.01\\
75.25	0.01\\
75.26	0.01\\
75.27	0.01\\
75.28	0.01\\
75.29	0.01\\
75.3	0.01\\
75.31	0.01\\
75.32	0.01\\
75.33	0.01\\
75.34	0.01\\
75.35	0.01\\
75.36	0.01\\
75.37	0.01\\
75.38	0.01\\
75.39	0.01\\
75.4	0.01\\
75.41	0.01\\
75.42	0.01\\
75.43	0.01\\
75.44	0.01\\
75.45	0.01\\
75.46	0.01\\
75.47	0.01\\
75.48	0.01\\
75.49	0.01\\
75.5	0.01\\
75.51	0.01\\
75.52	0.01\\
75.53	0.01\\
75.54	0.01\\
75.55	0.01\\
75.56	0.01\\
75.57	0.01\\
75.58	0.01\\
75.59	0.01\\
75.6	0.01\\
75.61	0.01\\
75.62	0.01\\
75.63	0.01\\
75.64	0.01\\
75.65	0.01\\
75.66	0.01\\
75.67	0.01\\
75.68	0.01\\
75.69	0.01\\
75.7	0.01\\
75.71	0.01\\
75.72	0.01\\
75.73	0.01\\
75.74	0.01\\
75.75	0.01\\
75.76	0.01\\
75.77	0.01\\
75.78	0.01\\
75.79	0.01\\
75.8	0.01\\
75.81	0.01\\
75.82	0.01\\
75.83	0.01\\
75.84	0.01\\
75.85	0.01\\
75.86	0.01\\
75.87	0.01\\
75.88	0.01\\
75.89	0.01\\
75.9	0.01\\
75.91	0.01\\
75.92	0.01\\
75.93	0.01\\
75.94	0.01\\
75.95	0.01\\
75.96	0.01\\
75.97	0.01\\
75.98	0.01\\
75.99	0.01\\
76	0.01\\
76.01	0.01\\
76.02	0.01\\
76.03	0.01\\
76.04	0.01\\
76.05	0.01\\
76.06	0.01\\
76.07	0.01\\
76.08	0.01\\
76.09	0.01\\
76.1	0.01\\
76.11	0.01\\
76.12	0.01\\
76.13	0.01\\
76.14	0.01\\
76.15	0.01\\
76.16	0.01\\
76.17	0.01\\
76.18	0.01\\
76.19	0.01\\
76.2	0.01\\
76.21	0.01\\
76.22	0.01\\
76.23	0.01\\
76.24	0.01\\
76.25	0.01\\
76.26	0.01\\
76.27	0.01\\
76.28	0.01\\
76.29	0.01\\
76.3	0.01\\
76.31	0.01\\
76.32	0.01\\
76.33	0.01\\
76.34	0.01\\
76.35	0.01\\
76.36	0.01\\
76.37	0.01\\
76.38	0.01\\
76.39	0.01\\
76.4	0.01\\
76.41	0.01\\
76.42	0.01\\
76.43	0.01\\
76.44	0.01\\
76.45	0.01\\
76.46	0.01\\
76.47	0.01\\
76.48	0.01\\
76.49	0.01\\
76.5	0.01\\
76.51	0.01\\
76.52	0.01\\
76.53	0.01\\
76.54	0.01\\
76.55	0.01\\
76.56	0.01\\
76.57	0.01\\
76.58	0.01\\
76.59	0.01\\
76.6	0.01\\
76.61	0.01\\
76.62	0.01\\
76.63	0.01\\
76.64	0.01\\
76.65	0.01\\
76.66	0.01\\
76.67	0.01\\
76.68	0.01\\
76.69	0.01\\
76.7	0.01\\
76.71	0.01\\
76.72	0.01\\
76.73	0.01\\
76.74	0.01\\
76.75	0.01\\
76.76	0.01\\
76.77	0.01\\
76.78	0.01\\
76.79	0.01\\
76.8	0.01\\
76.81	0.01\\
76.82	0.01\\
76.83	0.01\\
76.84	0.01\\
76.85	0.01\\
76.86	0.01\\
76.87	0.01\\
76.88	0.01\\
76.89	0.01\\
76.9	0.01\\
76.91	0.01\\
76.92	0.01\\
76.93	0.01\\
76.94	0.01\\
76.95	0.01\\
76.96	0.01\\
76.97	0.01\\
76.98	0.01\\
76.99	0.01\\
77	0.01\\
77.01	0.01\\
77.02	0.01\\
77.03	0.01\\
77.04	0.01\\
77.05	0.01\\
77.06	0.01\\
77.07	0.01\\
77.08	0.01\\
77.09	0.01\\
77.1	0.01\\
77.11	0.01\\
77.12	0.01\\
77.13	0.01\\
77.14	0.01\\
77.15	0.01\\
77.16	0.01\\
77.17	0.01\\
77.18	0.01\\
77.19	0.01\\
77.2	0.01\\
77.21	0.01\\
77.22	0.01\\
77.23	0.01\\
77.24	0.01\\
77.25	0.01\\
77.26	0.01\\
77.27	0.01\\
77.28	0.01\\
77.29	0.01\\
77.3	0.01\\
77.31	0.01\\
77.32	0.01\\
77.33	0.01\\
77.34	0.01\\
77.35	0.01\\
77.36	0.01\\
77.37	0.01\\
77.38	0.01\\
77.39	0.01\\
77.4	0.01\\
77.41	0.01\\
77.42	0.01\\
77.43	0.01\\
77.44	0.01\\
77.45	0.01\\
77.46	0.01\\
77.47	0.01\\
77.48	0.01\\
77.49	0.01\\
77.5	0.01\\
77.51	0.01\\
77.52	0.01\\
77.53	0.01\\
77.54	0.01\\
77.55	0.01\\
77.56	0.01\\
77.57	0.01\\
77.58	0.01\\
77.59	0.01\\
77.6	0.01\\
77.61	0.01\\
77.62	0.01\\
77.63	0.01\\
77.64	0.01\\
77.65	0.01\\
77.66	0.01\\
77.67	0.01\\
77.68	0.01\\
77.69	0.01\\
77.7	0.01\\
77.71	0.01\\
77.72	0.01\\
77.73	0.01\\
77.74	0.01\\
77.75	0.01\\
77.76	0.01\\
77.77	0.01\\
77.78	0.01\\
77.79	0.01\\
77.8	0.01\\
77.81	0.01\\
77.82	0.01\\
77.83	0.01\\
77.84	0.01\\
77.85	0.01\\
77.86	0.01\\
77.87	0.01\\
77.88	0.01\\
77.89	0.01\\
77.9	0.01\\
77.91	0.01\\
77.92	0.01\\
77.93	0.01\\
77.94	0.01\\
77.95	0.01\\
77.96	0.01\\
77.97	0.01\\
77.98	0.01\\
77.99	0.01\\
78	0.01\\
78.01	0.01\\
78.02	0.01\\
78.03	0.01\\
78.04	0.01\\
78.05	0.01\\
78.06	0.01\\
78.07	0.01\\
78.08	0.01\\
78.09	0.01\\
78.1	0.01\\
78.11	0.01\\
78.12	0.01\\
78.13	0.01\\
78.14	0.01\\
78.15	0.01\\
78.16	0.01\\
78.17	0.01\\
78.18	0.01\\
78.19	0.01\\
78.2	0.01\\
78.21	0.01\\
78.22	0.01\\
78.23	0.01\\
78.24	0.01\\
78.25	0.01\\
78.26	0.01\\
78.27	0.01\\
78.28	0.01\\
78.29	0.01\\
78.3	0.01\\
78.31	0.01\\
78.32	0.01\\
78.33	0.01\\
78.34	0.01\\
78.35	0.01\\
78.36	0.01\\
78.37	0.01\\
78.38	0.01\\
78.39	0.01\\
78.4	0.01\\
78.41	0.01\\
78.42	0.01\\
78.43	0.01\\
78.44	0.01\\
78.45	0.01\\
78.46	0.01\\
78.47	0.01\\
78.48	0.01\\
78.49	0.01\\
78.5	0.01\\
78.51	0.01\\
78.52	0.01\\
78.53	0.01\\
78.54	0.01\\
78.55	0.01\\
78.56	0.01\\
78.57	0.01\\
78.58	0.01\\
78.59	0.01\\
78.6	0.01\\
78.61	0.01\\
78.62	0.01\\
78.63	0.01\\
78.64	0.01\\
78.65	0.01\\
78.66	0.01\\
78.67	0.01\\
78.68	0.01\\
78.69	0.01\\
78.7	0.01\\
78.71	0.01\\
78.72	0.01\\
78.73	0.01\\
78.74	0.01\\
78.75	0.01\\
78.76	0.01\\
78.77	0.01\\
78.78	0.01\\
78.79	0.01\\
78.8	0.01\\
78.81	0.01\\
78.82	0.01\\
78.83	0.01\\
78.84	0.01\\
78.85	0.01\\
78.86	0.01\\
78.87	0.01\\
78.88	0.01\\
78.89	0.01\\
78.9	0.01\\
78.91	0.01\\
78.92	0.01\\
78.93	0.01\\
78.94	0.01\\
78.95	0.01\\
78.96	0.01\\
78.97	0.01\\
78.98	0.01\\
78.99	0.01\\
79	0.01\\
79.01	0.01\\
79.02	0.01\\
79.03	0.01\\
79.04	0.01\\
79.05	0.01\\
79.06	0.01\\
79.07	0.01\\
79.08	0.01\\
79.09	0.01\\
79.1	0.01\\
79.11	0.01\\
79.12	0.01\\
79.13	0.01\\
79.14	0.01\\
79.15	0.01\\
79.16	0.01\\
79.17	0.01\\
79.18	0.01\\
79.19	0.01\\
79.2	0.01\\
79.21	0.01\\
79.22	0.01\\
79.23	0.01\\
79.24	0.01\\
79.25	0.01\\
79.26	0.01\\
79.27	0.01\\
79.28	0.01\\
79.29	0.01\\
79.3	0.01\\
79.31	0.01\\
79.32	0.01\\
79.33	0.01\\
79.34	0.01\\
79.35	0.01\\
79.36	0.01\\
79.37	0.01\\
79.38	0.01\\
79.39	0.01\\
79.4	0.01\\
79.41	0.01\\
79.42	0.01\\
79.43	0.01\\
79.44	0.01\\
79.45	0.01\\
79.46	0.01\\
79.47	0.01\\
79.48	0.01\\
79.49	0.01\\
79.5	0.01\\
79.51	0.01\\
79.52	0.01\\
79.53	0.01\\
79.54	0.01\\
79.55	0.01\\
79.56	0.01\\
79.57	0.01\\
79.58	0.01\\
79.59	0.01\\
79.6	0.01\\
79.61	0.01\\
79.62	0.01\\
79.63	0.01\\
79.64	0.01\\
79.65	0.01\\
79.66	0.01\\
79.67	0.01\\
79.68	0.01\\
79.69	0.01\\
79.7	0.01\\
79.71	0.01\\
79.72	0.01\\
79.73	0.01\\
79.74	0.01\\
79.75	0.01\\
79.76	0.01\\
79.77	0.01\\
79.78	0.01\\
79.79	0.01\\
79.8	0.01\\
79.81	0.01\\
79.82	0.01\\
79.83	0.01\\
79.84	0.01\\
79.85	0.01\\
79.86	0.01\\
79.87	0.01\\
79.88	0.01\\
79.89	0.01\\
79.9	0.01\\
79.91	0.01\\
79.92	0.01\\
79.93	0.01\\
79.94	0.01\\
79.95	0.01\\
79.96	0.01\\
79.97	0.01\\
79.98	0.01\\
79.99	0.01\\
80	0.01\\
80.01	0.01\\
};
\addplot [color=blue,dashed]
  table[row sep=crcr]{%
80.01	0.01\\
80.02	0.01\\
80.03	0.01\\
80.04	0.01\\
80.05	0.01\\
80.06	0.01\\
80.07	0.01\\
80.08	0.01\\
80.09	0.01\\
80.1	0.01\\
80.11	0.01\\
80.12	0.01\\
80.13	0.01\\
80.14	0.01\\
80.15	0.01\\
80.16	0.01\\
80.17	0.01\\
80.18	0.01\\
80.19	0.01\\
80.2	0.01\\
80.21	0.01\\
80.22	0.01\\
80.23	0.01\\
80.24	0.01\\
80.25	0.01\\
80.26	0.01\\
80.27	0.01\\
80.28	0.01\\
80.29	0.01\\
80.3	0.01\\
80.31	0.01\\
80.32	0.01\\
80.33	0.01\\
80.34	0.01\\
80.35	0.01\\
80.36	0.01\\
80.37	0.01\\
80.38	0.01\\
80.39	0.01\\
80.4	0.01\\
80.41	0.01\\
80.42	0.01\\
80.43	0.01\\
80.44	0.01\\
80.45	0.01\\
80.46	0.01\\
80.47	0.01\\
80.48	0.01\\
80.49	0.01\\
80.5	0.01\\
80.51	0.01\\
80.52	0.01\\
80.53	0.01\\
80.54	0.01\\
80.55	0.01\\
80.56	0.01\\
80.57	0.01\\
80.58	0.01\\
80.59	0.01\\
80.6	0.01\\
80.61	0.01\\
80.62	0.01\\
80.63	0.01\\
80.64	0.01\\
80.65	0.01\\
80.66	0.01\\
80.67	0.01\\
80.68	0.01\\
80.69	0.01\\
80.7	0.01\\
80.71	0.01\\
80.72	0.01\\
80.73	0.01\\
80.74	0.01\\
80.75	0.01\\
80.76	0.01\\
80.77	0.01\\
80.78	0.01\\
80.79	0.01\\
80.8	0.01\\
80.81	0.01\\
80.82	0.01\\
80.83	0.01\\
80.84	0.01\\
80.85	0.01\\
80.86	0.01\\
80.87	0.01\\
80.88	0.01\\
80.89	0.01\\
80.9	0.01\\
80.91	0.01\\
80.92	0.01\\
80.93	0.01\\
80.94	0.01\\
80.95	0.01\\
80.96	0.01\\
80.97	0.01\\
80.98	0.01\\
80.99	0.01\\
81	0.01\\
81.01	0.01\\
81.02	0.01\\
81.03	0.01\\
81.04	0.01\\
81.05	0.01\\
81.06	0.01\\
81.07	0.01\\
81.08	0.01\\
81.09	0.01\\
81.1	0.01\\
81.11	0.01\\
81.12	0.01\\
81.13	0.01\\
81.14	0.01\\
81.15	0.01\\
81.16	0.01\\
81.17	0.01\\
81.18	0.01\\
81.19	0.01\\
81.2	0.01\\
81.21	0.01\\
81.22	0.01\\
81.23	0.01\\
81.24	0.01\\
81.25	0.01\\
81.26	0.01\\
81.27	0.01\\
81.28	0.01\\
81.29	0.01\\
81.3	0.01\\
81.31	0.01\\
81.32	0.01\\
81.33	0.01\\
81.34	0.01\\
81.35	0.01\\
81.36	0.01\\
81.37	0.01\\
81.38	0.01\\
81.39	0.01\\
81.4	0.01\\
81.41	0.01\\
81.42	0.01\\
81.43	0.01\\
81.44	0.01\\
81.45	0.01\\
81.46	0.01\\
81.47	0.01\\
81.48	0.01\\
81.49	0.01\\
81.5	0.01\\
81.51	0.01\\
81.52	0.01\\
81.53	0.01\\
81.54	0.01\\
81.55	0.01\\
81.56	0.01\\
81.57	0.01\\
81.58	0.01\\
81.59	0.01\\
81.6	0.01\\
81.61	0.01\\
81.62	0.01\\
81.63	0.01\\
81.64	0.01\\
81.65	0.01\\
81.66	0.01\\
81.67	0.01\\
81.68	0.01\\
81.69	0.01\\
81.7	0.01\\
81.71	0.01\\
81.72	0.01\\
81.73	0.01\\
81.74	0.01\\
81.75	0.01\\
81.76	0.01\\
81.77	0.01\\
81.78	0.01\\
81.79	0.01\\
81.8	0.01\\
81.81	0.01\\
81.82	0.01\\
81.83	0.01\\
81.84	0.01\\
81.85	0.01\\
81.86	0.01\\
81.87	0.01\\
81.88	0.01\\
81.89	0.01\\
81.9	0.01\\
81.91	0.01\\
81.92	0.01\\
81.93	0.01\\
81.94	0.01\\
81.95	0.01\\
81.96	0.01\\
81.97	0.01\\
81.98	0.01\\
81.99	0.01\\
82	0.01\\
82.01	0.01\\
82.02	0.01\\
82.03	0.01\\
82.04	0.01\\
82.05	0.01\\
82.06	0.01\\
82.07	0.01\\
82.08	0.01\\
82.09	0.01\\
82.1	0.01\\
82.11	0.01\\
82.12	0.01\\
82.13	0.01\\
82.14	0.01\\
82.15	0.01\\
82.16	0.01\\
82.17	0.01\\
82.18	0.01\\
82.19	0.01\\
82.2	0.01\\
82.21	0.01\\
82.22	0.01\\
82.23	0.01\\
82.24	0.01\\
82.25	0.01\\
82.26	0.01\\
82.27	0.01\\
82.28	0.01\\
82.29	0.01\\
82.3	0.01\\
82.31	0.01\\
82.32	0.01\\
82.33	0.01\\
82.34	0.01\\
82.35	0.01\\
82.36	0.01\\
82.37	0.01\\
82.38	0.01\\
82.39	0.01\\
82.4	0.01\\
82.41	0.01\\
82.42	0.01\\
82.43	0.01\\
82.44	0.01\\
82.45	0.01\\
82.46	0.01\\
82.47	0.01\\
82.48	0.01\\
82.49	0.01\\
82.5	0.01\\
82.51	0.01\\
82.52	0.01\\
82.53	0.01\\
82.54	0.01\\
82.55	0.01\\
82.56	0.01\\
82.57	0.01\\
82.58	0.01\\
82.59	0.01\\
82.6	0.01\\
82.61	0.01\\
82.62	0.01\\
82.63	0.01\\
82.64	0.01\\
82.65	0.01\\
82.66	0.01\\
82.67	0.01\\
82.68	0.01\\
82.69	0.01\\
82.7	0.01\\
82.71	0.01\\
82.72	0.01\\
82.73	0.01\\
82.74	0.01\\
82.75	0.01\\
82.76	0.01\\
82.77	0.01\\
82.78	0.01\\
82.79	0.01\\
82.8	0.01\\
82.81	0.01\\
82.82	0.01\\
82.83	0.01\\
82.84	0.01\\
82.85	0.01\\
82.86	0.01\\
82.87	0.01\\
82.88	0.01\\
82.89	0.01\\
82.9	0.01\\
82.91	0.01\\
82.92	0.01\\
82.93	0.01\\
82.94	0.01\\
82.95	0.01\\
82.96	0.01\\
82.97	0.01\\
82.98	0.01\\
82.99	0.01\\
83	0.01\\
83.01	0.01\\
83.02	0.01\\
83.03	0.01\\
83.04	0.01\\
83.05	0.01\\
83.06	0.01\\
83.07	0.01\\
83.08	0.01\\
83.09	0.01\\
83.1	0.01\\
83.11	0.01\\
83.12	0.01\\
83.13	0.01\\
83.14	0.01\\
83.15	0.01\\
83.16	0.01\\
83.17	0.01\\
83.18	0.01\\
83.19	0.01\\
83.2	0.01\\
83.21	0.01\\
83.22	0.01\\
83.23	0.01\\
83.24	0.01\\
83.25	0.01\\
83.26	0.01\\
83.27	0.01\\
83.28	0.01\\
83.29	0.01\\
83.3	0.01\\
83.31	0.01\\
83.32	0.01\\
83.33	0.01\\
83.34	0.01\\
83.35	0.01\\
83.36	0.01\\
83.37	0.01\\
83.38	0.01\\
83.39	0.01\\
83.4	0.01\\
83.41	0.01\\
83.42	0.01\\
83.43	0.01\\
83.44	0.01\\
83.45	0.01\\
83.46	0.01\\
83.47	0.01\\
83.48	0.01\\
83.49	0.01\\
83.5	0.01\\
83.51	0.01\\
83.52	0.01\\
83.53	0.01\\
83.54	0.01\\
83.55	0.01\\
83.56	0.01\\
83.57	0.01\\
83.58	0.01\\
83.59	0.01\\
83.6	0.01\\
83.61	0.01\\
83.62	0.01\\
83.63	0.01\\
83.64	0.01\\
83.65	0.01\\
83.66	0.01\\
83.67	0.01\\
83.68	0.01\\
83.69	0.01\\
83.7	0.01\\
83.71	0.01\\
83.72	0.01\\
83.73	0.01\\
83.74	0.01\\
83.75	0.01\\
83.76	0.01\\
83.77	0.01\\
83.78	0.01\\
83.79	0.01\\
83.8	0.01\\
83.81	0.01\\
83.82	0.01\\
83.83	0.01\\
83.84	0.01\\
83.85	0.01\\
83.86	0.01\\
83.87	0.01\\
83.88	0.01\\
83.89	0.01\\
83.9	0.01\\
83.91	0.01\\
83.92	0.01\\
83.93	0.01\\
83.94	0.01\\
83.95	0.01\\
83.96	0.01\\
83.97	0.01\\
83.98	0.01\\
83.99	0.01\\
84	0.01\\
84.01	0.01\\
84.02	0.01\\
84.03	0.01\\
84.04	0.01\\
84.05	0.01\\
84.06	0.01\\
84.07	0.01\\
84.08	0.01\\
84.09	0.01\\
84.1	0.01\\
84.11	0.01\\
84.12	0.01\\
84.13	0.01\\
84.14	0.01\\
84.15	0.01\\
84.16	0.01\\
84.17	0.01\\
84.18	0.01\\
84.19	0.01\\
84.2	0.01\\
84.21	0.01\\
84.22	0.01\\
84.23	0.01\\
84.24	0.01\\
84.25	0.01\\
84.26	0.01\\
84.27	0.01\\
84.28	0.01\\
84.29	0.01\\
84.3	0.01\\
84.31	0.01\\
84.32	0.01\\
84.33	0.01\\
84.34	0.01\\
84.35	0.01\\
84.36	0.01\\
84.37	0.01\\
84.38	0.01\\
84.39	0.01\\
84.4	0.01\\
84.41	0.01\\
84.42	0.01\\
84.43	0.01\\
84.44	0.01\\
84.45	0.01\\
84.46	0.01\\
84.47	0.01\\
84.48	0.01\\
84.49	0.01\\
84.5	0.01\\
84.51	0.01\\
84.52	0.01\\
84.53	0.01\\
84.54	0.01\\
84.55	0.01\\
84.56	0.01\\
84.57	0.01\\
84.58	0.01\\
84.59	0.01\\
84.6	0.01\\
84.61	0.01\\
84.62	0.01\\
84.63	0.01\\
84.64	0.01\\
84.65	0.01\\
84.66	0.01\\
84.67	0.01\\
84.68	0.01\\
84.69	0.01\\
84.7	0.01\\
84.71	0.01\\
84.72	0.01\\
84.73	0.01\\
84.74	0.01\\
84.75	0.01\\
84.76	0.01\\
84.77	0.01\\
84.78	0.01\\
84.79	0.01\\
84.8	0.01\\
84.81	0.01\\
84.82	0.01\\
84.83	0.01\\
84.84	0.01\\
84.85	0.01\\
84.86	0.01\\
84.87	0.01\\
84.88	0.01\\
84.89	0.01\\
84.9	0.01\\
84.91	0.01\\
84.92	0.01\\
84.93	0.01\\
84.94	0.01\\
84.95	0.01\\
84.96	0.01\\
84.97	0.01\\
84.98	0.01\\
84.99	0.01\\
85	0.01\\
85.01	0.01\\
85.02	0.01\\
85.03	0.01\\
85.04	0.01\\
85.05	0.01\\
85.06	0.01\\
85.07	0.01\\
85.08	0.01\\
85.09	0.01\\
85.1	0.01\\
85.11	0.01\\
85.12	0.01\\
85.13	0.01\\
85.14	0.01\\
85.15	0.01\\
85.16	0.01\\
85.17	0.01\\
85.18	0.01\\
85.19	0.01\\
85.2	0.01\\
85.21	0.01\\
85.22	0.01\\
85.23	0.01\\
85.24	0.01\\
85.25	0.01\\
85.26	0.01\\
85.27	0.01\\
85.28	0.01\\
85.29	0.01\\
85.3	0.01\\
85.31	0.01\\
85.32	0.01\\
85.33	0.01\\
85.34	0.01\\
85.35	0.01\\
85.36	0.01\\
85.37	0.01\\
85.38	0.01\\
85.39	0.01\\
85.4	0.01\\
85.41	0.01\\
85.42	0.01\\
85.43	0.01\\
85.44	0.01\\
85.45	0.01\\
85.46	0.01\\
85.47	0.01\\
85.48	0.01\\
85.49	0.01\\
85.5	0.01\\
85.51	0.01\\
85.52	0.01\\
85.53	0.01\\
85.54	0.00999855889538515\\
85.55	0.00999403882326935\\
85.56	0.00998951103636577\\
85.57	0.00998497551822438\\
85.58	0.00998043225236838\\
85.59	0.00997588122229432\\
85.6	0.00997132241147221\\
85.61	0.0099667558033456\\
85.62	0.0099621813813317\\
85.63	0.00995760187610974\\
85.64	0.00995301760746407\\
85.65	0.00994842856651868\\
85.66	0.00994383474437918\\
85.67	0.00993923613213287\\
85.68	0.00993463272084862\\
85.69	0.00993002450157701\\
85.7	0.00992541146535017\\
85.71	0.0099207936031819\\
85.72	0.00991617090606758\\
85.73	0.00991154336498421\\
85.74	0.00990691097089038\\
85.75	0.00990227371472629\\
85.76	0.00989763158741372\\
85.77	0.00989298457985604\\
85.78	0.00988833268293821\\
85.79	0.00988367588752678\\
85.8	0.00987901418446988\\
85.81	0.00987434756459721\\
85.82	0.00986967601872006\\
85.83	0.00986499953763133\\
85.84	0.00986031811210546\\
85.85	0.0098556317328985\\
85.86	0.0098509403907481\\
85.87	0.00984624407637348\\
85.88	0.00984154278047547\\
85.89	0.00983683649373649\\
85.9	0.00983212520682059\\
85.91	0.00982740891037341\\
85.92	0.00982268759502222\\
85.93	0.00981796125137592\\
85.94	0.00981322987002503\\
85.95	0.00980849344154174\\
85.96	0.00980375195647988\\
85.97	0.00979900540537494\\
85.98	0.00979425377874412\\
85.99	0.00978949706708627\\
86	0.00978473526088197\\
86.01	0.00977996835059351\\
86.02	0.00977519632666492\\
86.03	0.00977041917952197\\
86.04	0.0097656368995722\\
86.05	0.00976084947720494\\
86.06	0.00975605690279131\\
86.07	0.00975125916668428\\
86.08	0.00974645625921862\\
86.09	0.009741648170711\\
86.1	0.00973683489145998\\
86.11	0.00973201641174599\\
86.12	0.00972719272183145\\
86.13	0.00972236381196069\\
86.14	0.00971752967236008\\
86.15	0.00971269029323796\\
86.16	0.00970784566478474\\
86.17	0.0097029957771729\\
86.18	0.00969814062055701\\
86.19	0.00969328018507382\\
86.2	0.00968841446084219\\
86.21	0.00968354343796325\\
86.22	0.00967866710652031\\
86.23	0.009673785456579\\
86.24	0.00966889847818726\\
86.25	0.00966400616137538\\
86.26	0.00965910849615603\\
86.27	0.00965420547252435\\
86.28	0.00964929708045793\\
86.29	0.00964438330991691\\
86.3	0.00963946415084398\\
86.31	0.00963453959316446\\
86.32	0.00962960962678634\\
86.33	0.00962467424160032\\
86.34	0.00961973342747986\\
86.35	0.00961478717428127\\
86.36	0.00960983547184369\\
86.37	0.00960487830998922\\
86.38	0.00959991567852294\\
86.39	0.00959494756723298\\
86.4	0.00958997396589058\\
86.41	0.00958499486425011\\
86.42	0.00958001025204922\\
86.43	0.00957502011900884\\
86.44	0.00957002445483324\\
86.45	0.00956502324921014\\
86.46	0.00956001649181076\\
86.47	0.00955500417228988\\
86.48	0.00954998628028595\\
86.49	0.0095449628054211\\
86.5	0.00953993373730129\\
86.51	0.00953489906551634\\
86.52	0.00952985877964002\\
86.53	0.00952481286923016\\
86.54	0.00951976132382868\\
86.55	0.00951470413296173\\
86.56	0.00950964128613974\\
86.57	0.00950457277285755\\
86.58	0.00949949858259444\\
86.59	0.0094944187048143\\
86.6	0.00948933312896566\\
86.61	0.00948424184448182\\
86.62	0.00947914484078096\\
86.63	0.00947404210726619\\
86.64	0.00946893363332574\\
86.65	0.00946381940833298\\
86.66	0.00945869942164659\\
86.67	0.00945357366261063\\
86.68	0.00944844212055469\\
86.69	0.00944330478479397\\
86.7	0.00943816164462942\\
86.71	0.00943301268934786\\
86.72	0.00942785790822211\\
86.73	0.00942269729051107\\
86.74	0.00941753082545991\\
86.75	0.00941235850230016\\
86.76	0.00940718031024987\\
86.77	0.0094019962385137\\
86.78	0.00939680627628312\\
86.79	0.0093916104127365\\
86.8	0.00938640863703929\\
86.81	0.00938120093834413\\
86.82	0.00937598730579101\\
86.83	0.00937076772850748\\
86.84	0.0093655421956087\\
86.85	0.00936031069619768\\
86.86	0.0093550732193654\\
86.87	0.00934982975419101\\
86.88	0.00934458028974196\\
86.89	0.00933932481507417\\
86.9	0.00933406331923223\\
86.91	0.00932879579124956\\
86.92	0.00932352222014861\\
86.93	0.009318242594941\\
86.94	0.00931296062792505\\
86.95	0.00930767653239375\\
86.96	0.00930239030856869\\
86.97	0.00929710195667923\\
86.98	0.00929181147696262\\
86.99	0.00928651886966399\\
87	0.0092812241350364\\
87.01	0.0092759272733409\\
87.02	0.00927062828484655\\
87.03	0.00926532716983047\\
87.04	0.00926002392857791\\
87.05	0.00925471856138223\\
87.06	0.00924941106854499\\
87.07	0.009244101450376\\
87.08	0.00923878970719332\\
87.09	0.00923347583932335\\
87.1	0.00922815984710081\\
87.11	0.00922284173086888\\
87.12	0.00921752149097914\\
87.13	0.0092121991277917\\
87.14	0.00920687464167518\\
87.15	0.00920154803300678\\
87.16	0.00919621930217235\\
87.17	0.00919088844956639\\
87.18	0.00918555547559213\\
87.19	0.00918022038066155\\
87.2	0.00917488316519543\\
87.21	0.00916954382962343\\
87.22	0.00916420237438408\\
87.23	0.00915885879992488\\
87.24	0.0091535131067023\\
87.25	0.00914816529518186\\
87.26	0.00914281536583817\\
87.27	0.00913746331915497\\
87.28	0.00913210915562518\\
87.29	0.00912675287575093\\
87.3	0.00912139448004367\\
87.31	0.00911603396902413\\
87.32	0.00911067134322244\\
87.33	0.00910530660317814\\
87.34	0.00909993974944025\\
87.35	0.00909457078256731\\
87.36	0.00908919970312743\\
87.37	0.00908382651169832\\
87.38	0.0090784512088674\\
87.39	0.00907307379523177\\
87.4	0.00906769427139833\\
87.41	0.00906231263798379\\
87.42	0.00905692889561473\\
87.43	0.00905154304492766\\
87.44	0.00904615508656906\\
87.45	0.00904076502119545\\
87.46	0.00903537284947342\\
87.47	0.00902997857207971\\
87.48	0.0090245821897012\\
87.49	0.00901918370303508\\
87.5	0.00901378311278876\\
87.51	0.00900838041968005\\
87.52	0.00900297562443712\\
87.53	0.00899756872779863\\
87.54	0.00899215973051371\\
87.55	0.00898674863334209\\
87.56	0.0089813354370541\\
87.57	0.00897592014243073\\
87.58	0.00897050275026372\\
87.59	0.00896508326135558\\
87.6	0.00895966167651969\\
87.61	0.00895423799658029\\
87.62	0.0089488122223726\\
87.63	0.00894338435474283\\
87.64	0.00893795439454829\\
87.65	0.00893252234265739\\
87.66	0.00892708819994972\\
87.67	0.00892165196731616\\
87.68	0.00891621364565883\\
87.69	0.00891077323589125\\
87.7	0.00890533073893836\\
87.71	0.00889988615573656\\
87.72	0.00889443948723381\\
87.73	0.00888899073438965\\
87.74	0.00888354008006392\\
87.75	0.00887808752818598\\
87.76	0.00887263307724207\\
87.77	0.00886717672571643\\
87.78	0.00886171847209137\\
87.79	0.00885625831484717\\
87.8	0.00885079625246215\\
87.81	0.00884533228341264\\
87.82	0.00883986640617296\\
87.83	0.00883439861921546\\
87.84	0.00882892892101046\\
87.85	0.00882345731002631\\
87.86	0.00881798378472933\\
87.87	0.00881250834358382\\
87.88	0.0088070309850521\\
87.89	0.00880155170759444\\
87.9	0.00879607050966912\\
87.91	0.00879058738973237\\
87.92	0.00878510234623841\\
87.93	0.00877961537763941\\
87.94	0.00877412648238552\\
87.95	0.00876863565892484\\
87.96	0.00876314290570345\\
87.97	0.00875764822116535\\
87.98	0.00875215160375254\\
87.99	0.0087466530519049\\
88	0.00874115256406031\\
88.01	0.00873565013865457\\
88.02	0.00873014577412142\\
88.03	0.00872463946889251\\
88.04	0.00871913122139745\\
88.05	0.00871362103006376\\
88.06	0.00870810889331687\\
88.07	0.00870259480958016\\
88.08	0.00869707877727489\\
88.09	0.00869156079482024\\
88.1	0.00868604086063332\\
88.11	0.00868051897312909\\
88.12	0.00867499513072047\\
88.13	0.00866946933181822\\
88.14	0.00866394157483103\\
88.15	0.00865841185816546\\
88.16	0.00865288018022596\\
88.17	0.00864734653941483\\
88.18	0.00864181093413229\\
88.19	0.00863627336277641\\
88.2	0.00863073382374313\\
88.21	0.00862519231542623\\
88.22	0.0086196488362174\\
88.23	0.00861410338450612\\
88.24	0.00860855595867979\\
88.25	0.00860300655712361\\
88.26	0.00859745517822064\\
88.27	0.00859190182035177\\
88.28	0.00858634648189574\\
88.29	0.00858078916122912\\
88.3	0.00857522985672628\\
88.31	0.00856966856675945\\
88.32	0.00856410528969866\\
88.33	0.00855854002391175\\
88.34	0.00855297276776437\\
88.35	0.00854740351962\\
88.36	0.00854183227783989\\
88.37	0.00853625904078312\\
88.38	0.00853068380680653\\
88.39	0.00852510657426477\\
88.4	0.00851952734151027\\
88.41	0.00851394610689324\\
88.42	0.00850836286876167\\
88.43	0.00850277762546132\\
88.44	0.00849719037533572\\
88.45	0.00849160111672615\\
88.46	0.00848600984797166\\
88.47	0.00848041656740906\\
88.48	0.00847482127337289\\
88.49	0.00846922396419547\\
88.5	0.00846362463820682\\
88.51	0.00845802329373473\\
88.52	0.0084524199291047\\
88.53	0.00844681454263997\\
88.54	0.00844120713266149\\
88.55	0.00843559769748794\\
88.56	0.00842998623543571\\
88.57	0.00842437274481891\\
88.58	0.00841875722394933\\
88.59	0.00841313967113647\\
88.6	0.00840752008468755\\
88.61	0.00840189846290743\\
88.62	0.0083962748040987\\
88.63	0.00839064910656161\\
88.64	0.00838502136859409\\
88.65	0.00837939158849175\\
88.66	0.00837375976454785\\
88.67	0.00836812589505332\\
88.68	0.00836248997829673\\
88.69	0.00835685201256435\\
88.7	0.00835121199614003\\
88.71	0.00834556992730531\\
88.72	0.00833992580433935\\
88.73	0.00833427962551895\\
88.74	0.00832863138911852\\
88.75	0.0083229810934101\\
88.76	0.00831732873666335\\
88.77	0.00831167431714555\\
88.78	0.00830601783312155\\
88.79	0.00830035928285384\\
88.8	0.0082946986646025\\
88.81	0.00828903597662517\\
88.82	0.00828337121717711\\
88.83	0.00827770438451114\\
88.84	0.00827203547687768\\
88.85	0.00826636449252466\\
88.86	0.00826069142969765\\
88.87	0.00825501628663972\\
88.88	0.00824933906159153\\
88.89	0.00824365975279126\\
88.9	0.00823797835847464\\
88.91	0.00823229487687495\\
88.92	0.00822660930622299\\
88.93	0.00822092164474708\\
88.94	0.00821523189067307\\
88.95	0.00820954004222431\\
88.96	0.00820384609762167\\
88.97	0.00819815005508352\\
88.98	0.00819245191282572\\
88.99	0.00818675166906163\\
89	0.0081810493220021\\
89.01	0.00817534486985543\\
89.02	0.00816963831082743\\
89.03	0.00816392964312135\\
89.04	0.0081582188649379\\
89.05	0.00815250597447528\\
89.06	0.00814679096992909\\
89.07	0.00814107384949241\\
89.08	0.00813535461135575\\
89.09	0.00812963325370703\\
89.1	0.00812390977473161\\
89.11	0.00811818417261227\\
89.12	0.00811245644552919\\
89.13	0.00810672659165998\\
89.14	0.00810099460917962\\
89.15	0.00809526049626049\\
89.16	0.00808952425107238\\
89.17	0.00808378587178241\\
89.18	0.00807804535655514\\
89.19	0.00807230270355242\\
89.2	0.00806655791093353\\
89.21	0.00806081097685506\\
89.22	0.00805506189947095\\
89.23	0.0080493106769325\\
89.24	0.00804355730738833\\
89.25	0.00803780178898438\\
89.26	0.00803204411986391\\
89.27	0.00802628429816751\\
89.28	0.00802052232203306\\
89.29	0.00801475818959573\\
89.3	0.008008991898988\\
89.31	0.00800322344833962\\
89.32	0.00799745283577762\\
89.33	0.0079916800594263\\
89.34	0.00798590511740722\\
89.35	0.00798012800783919\\
89.36	0.00797434872883829\\
89.37	0.0079685672785178\\
89.38	0.00796278365498827\\
89.39	0.00795699785635746\\
89.4	0.00795120988073034\\
89.41	0.00794541972620908\\
89.42	0.00793962739089309\\
89.43	0.00793383287287894\\
89.44	0.00792803617026038\\
89.45	0.00792223728112838\\
89.46	0.00791643620357103\\
89.47	0.00791063293567361\\
89.48	0.00790482747551855\\
89.49	0.00789901982118543\\
89.5	0.00789320997075095\\
89.51	0.00788739792228895\\
89.52	0.0078815836738704\\
89.53	0.00787576722356336\\
89.54	0.007869948569433\\
89.55	0.00786412770954162\\
89.56	0.00785830464194855\\
89.57	0.00785247936471024\\
89.58	0.00784665187588018\\
89.59	0.00784082217350895\\
89.6	0.00783499025564416\\
89.61	0.00782915612033047\\
89.62	0.00782331976560958\\
89.63	0.00781748118952019\\
89.64	0.00781164039009804\\
89.65	0.00780579736537587\\
89.66	0.00779995211338341\\
89.67	0.00779410463214738\\
89.68	0.00778825491969149\\
89.69	0.0077824029740364\\
89.7	0.00777654879319973\\
89.71	0.00777069237519607\\
89.72	0.00776483371803692\\
89.73	0.00775897281973073\\
89.74	0.00775310967828285\\
89.75	0.00774724429169557\\
89.76	0.00774137665796805\\
89.77	0.00773550677509635\\
89.78	0.00772963464107341\\
89.79	0.00772376025388903\\
89.8	0.00771788361152989\\
89.81	0.00771200471197948\\
89.82	0.00770612355321816\\
89.83	0.00770024013322309\\
89.84	0.00769435444996828\\
89.85	0.00768846650142451\\
89.86	0.00768257628555936\\
89.87	0.0076766838003372\\
89.88	0.00767078904371917\\
89.89	0.00766489201366316\\
89.9	0.00765899270812383\\
89.91	0.00765309112505254\\
89.92	0.00764718726239741\\
89.93	0.00764128111810326\\
89.94	0.0076353726901116\\
89.95	0.00762946197636064\\
89.96	0.00762354897478528\\
89.97	0.00761763368331706\\
89.98	0.00761171609988418\\
89.99	0.00760579622241149\\
90	0.00759987404882046\\
90.01	0.00759394957702919\\
90.02	0.00758802280495235\\
90.03	0.00758209373050124\\
90.04	0.0075761623515837\\
90.05	0.00757022866610416\\
90.06	0.00756429267196359\\
90.07	0.0075583543670595\\
90.08	0.00755241374928592\\
90.09	0.00754647081653339\\
90.1	0.00754052556668895\\
90.11	0.00753457799763612\\
90.12	0.0075286281072549\\
90.13	0.00752267589342172\\
90.14	0.00751672135400947\\
90.15	0.00751076448688747\\
90.16	0.00750480528992144\\
90.17	0.00749884376097349\\
90.18	0.00749287989790213\\
90.19	0.00748691369856224\\
90.2	0.00748094516080501\\
90.21	0.00747497428247803\\
90.22	0.00746900106142516\\
90.23	0.00746302549548659\\
90.24	0.00745704758249881\\
90.25	0.00745106732029455\\
90.26	0.00744508470670283\\
90.27	0.0074390997395489\\
90.28	0.00743311241665424\\
90.29	0.00742712273583653\\
90.3	0.00742113069490966\\
90.31	0.00741513629168369\\
90.32	0.00740913952396482\\
90.33	0.00740314038955543\\
90.34	0.00739713888625398\\
90.35	0.00739113501185508\\
90.36	0.00738512876414941\\
90.37	0.00737912014092371\\
90.38	0.0073731091399608\\
90.39	0.00736709575903951\\
90.4	0.00736107999593472\\
90.41	0.00735506184841727\\
90.42	0.00734904131425401\\
90.43	0.00734301839120773\\
90.44	0.00733699307703719\\
90.45	0.00733096536949704\\
90.46	0.00732493526633786\\
90.47	0.0073189027653061\\
90.48	0.00731286786414407\\
90.49	0.00730683056058994\\
90.5	0.00730079085237768\\
90.51	0.00729474873723708\\
90.52	0.0072887042128937\\
90.53	0.00728265727706888\\
90.54	0.00727660792747968\\
90.55	0.00727055616183888\\
90.56	0.00726450197785497\\
90.57	0.00725844537323211\\
90.58	0.00725238634567008\\
90.59	0.00724632489286435\\
90.6	0.00724026101250595\\
90.61	0.0072341947022815\\
90.62	0.00722812595987321\\
90.63	0.0072220547829588\\
90.64	0.00721598116921151\\
90.65	0.00720990511630007\\
90.66	0.0072038266218887\\
90.67	0.00719774568363701\\
90.68	0.00719166229920008\\
90.69	0.00718557646622837\\
90.7	0.00717948818236767\\
90.71	0.00717339744525917\\
90.72	0.00716730425253933\\
90.73	0.00716120860183992\\
90.74	0.00715511049078797\\
90.75	0.00714900991700576\\
90.76	0.00714290687811076\\
90.77	0.00713680137171562\\
90.78	0.00713069339542819\\
90.79	0.00712458294685138\\
90.8	0.00711847002358325\\
90.81	0.00711235462321693\\
90.82	0.00710623674334056\\
90.83	0.00710011638153732\\
90.84	0.00709399353538538\\
90.85	0.00708786820245783\\
90.86	0.00708174038032273\\
90.87	0.00707561006654301\\
90.88	0.00706947725867645\\
90.89	0.00706334195427569\\
90.9	0.00705720415088816\\
90.91	0.00705106384605606\\
90.92	0.00704492103731632\\
90.93	0.0070387757222006\\
90.94	0.00703262789823519\\
90.95	0.00702647756294107\\
90.96	0.00702032471383378\\
90.97	0.00701416934842345\\
90.98	0.00700801146421477\\
90.99	0.00700185105870688\\
91	0.00699568812939345\\
91.01	0.00698952267376253\\
91.02	0.00698335468929659\\
91.03	0.00697718417347246\\
91.04	0.00697101112376128\\
91.05	0.00696483553762849\\
91.06	0.00695865741253378\\
91.07	0.00695247674593103\\
91.08	0.00694629353526831\\
91.09	0.00694010777798782\\
91.1	0.00693391947152584\\
91.11	0.00692772861331272\\
91.12	0.00692153520077282\\
91.13	0.00691533923132446\\
91.14	0.00690914070237992\\
91.15	0.00690293961134533\\
91.16	0.00689673595562069\\
91.17	0.00689052973259982\\
91.18	0.00688432093967026\\
91.19	0.0068781095742133\\
91.2	0.0068718956336039\\
91.21	0.00686567911521064\\
91.22	0.00685946001639569\\
91.23	0.00685323833451475\\
91.24	0.00684701406691702\\
91.25	0.00684078721094514\\
91.26	0.00683455776393515\\
91.27	0.00682832572321644\\
91.28	0.00682209108611169\\
91.29	0.00681585384993686\\
91.3	0.00680961401200106\\
91.31	0.00680337156960662\\
91.32	0.0067971265200489\\
91.33	0.00679087886061638\\
91.34	0.00678462858859047\\
91.35	0.00677837570124557\\
91.36	0.00677212019584896\\
91.37	0.00676586206966076\\
91.38	0.00675960131993385\\
91.39	0.00675333794391386\\
91.4	0.0067470719388391\\
91.41	0.00674080330194046\\
91.42	0.00673453203044142\\
91.43	0.00672825812155795\\
91.44	0.00672198157249846\\
91.45	0.00671570238046373\\
91.46	0.00670942054264688\\
91.47	0.00670313605623329\\
91.48	0.00669684891840052\\
91.49	0.00669055912631829\\
91.5	0.00668426667714838\\
91.51	0.00667797156804458\\
91.52	0.00667167379615263\\
91.53	0.00666537335861016\\
91.54	0.0066590702525466\\
91.55	0.00665276447508314\\
91.56	0.00664645602333262\\
91.57	0.00664014489439954\\
91.58	0.00663383108537989\\
91.59	0.00662751459336118\\
91.6	0.00662119541542228\\
91.61	0.00661487354863341\\
91.62	0.00660854899005605\\
91.63	0.00660222173674285\\
91.64	0.00659589178573757\\
91.65	0.00658955913407501\\
91.66	0.00658322377878092\\
91.67	0.00657688571687192\\
91.68	0.00657054494535545\\
91.69	0.00656420146122965\\
91.7	0.00655785526148332\\
91.71	0.00655150634309581\\
91.72	0.00654515470303694\\
91.73	0.00653880033826695\\
91.74	0.00653244324573636\\
91.75	0.00652608342238592\\
91.76	0.00651972086514655\\
91.77	0.00651335557093919\\
91.78	0.00650698753667474\\
91.79	0.006500616759254\\
91.8	0.00649424323556753\\
91.81	0.00648786696249559\\
91.82	0.00648148793690804\\
91.83	0.00647510615566426\\
91.84	0.00646872161561299\\
91.85	0.00646233431359235\\
91.86	0.00645594424642962\\
91.87	0.00644955141094123\\
91.88	0.00644315580393261\\
91.89	0.00643675742219812\\
91.9	0.00643035626252093\\
91.91	0.00642395232167291\\
91.92	0.00641754559641455\\
91.93	0.00641113608349483\\
91.94	0.00640472377965113\\
91.95	0.0063983086816091\\
91.96	0.00639189078608258\\
91.97	0.00638547008977347\\
91.98	0.00637904658937159\\
91.99	0.00637262028155464\\
92	0.006366191162988\\
92.01	0.00635975923032467\\
92.02	0.00635332448020514\\
92.03	0.00634688690945161\\
92.04	0.00634044651505435\\
92.05	0.00633400329399381\\
92.06	0.00632755724324056\\
92.07	0.00632110835975515\\
92.08	0.00631465664048807\\
92.09	0.00630820208237962\\
92.1	0.00630174468235988\\
92.11	0.0062952844373485\\
92.12	0.00628882134425473\\
92.13	0.00628235539997724\\
92.14	0.00627588660140406\\
92.15	0.00626941494541246\\
92.16	0.00626294042886886\\
92.17	0.00625646304862873\\
92.18	0.00624998280153649\\
92.19	0.00624349968442537\\
92.2	0.00623701369411737\\
92.21	0.00623052482742308\\
92.22	0.00622403308114163\\
92.23	0.00621753845206054\\
92.24	0.00621104093695563\\
92.25	0.00620454053259091\\
92.26	0.00619803723571843\\
92.27	0.00619153104307821\\
92.28	0.0061850219513981\\
92.29	0.00617850995739366\\
92.3	0.00617199505776803\\
92.31	0.00616547724921186\\
92.32	0.00615895652840309\\
92.33	0.00615243289200693\\
92.34	0.00614590633667566\\
92.35	0.00613937685904852\\
92.36	0.0061328444557516\\
92.37	0.00612630912339769\\
92.38	0.00611977085858614\\
92.39	0.00611322965790273\\
92.4	0.00610668551791956\\
92.41	0.00610013843519487\\
92.42	0.0060935884062729\\
92.43	0.0060870354276838\\
92.44	0.00608047949594341\\
92.45	0.00607392060755319\\
92.46	0.006067358759\\
92.47	0.006060793946756\\
92.48	0.00605422616727848\\
92.49	0.0060476554170097\\
92.5	0.00604108169237674\\
92.51	0.00603450498979135\\
92.52	0.00602792530564978\\
92.53	0.0060213426363326\\
92.54	0.00601475697820458\\
92.55	0.00600816832761447\\
92.56	0.00600157668089489\\
92.57	0.00599498203436208\\
92.58	0.00598838438431581\\
92.59	0.00598178372703916\\
92.6	0.00597518005879834\\
92.61	0.00596857337584251\\
92.62	0.00596196367440361\\
92.63	0.00595535095069619\\
92.64	0.00594873520091718\\
92.65	0.0059421164212457\\
92.66	0.00593549460784294\\
92.67	0.00592886975685187\\
92.68	0.00592224186439711\\
92.69	0.00591561092658467\\
92.7	0.00590897693950182\\
92.71	0.00590233989921682\\
92.72	0.00589569980177874\\
92.73	0.00588905664321726\\
92.74	0.00588241041954241\\
92.75	0.00587576112674441\\
92.76	0.0058691087607934\\
92.77	0.00586245331763927\\
92.78	0.00585579479321139\\
92.79	0.00584913318341837\\
92.8	0.00584246848414789\\
92.81	0.00583580069126642\\
92.82	0.00582912980061897\\
92.83	0.0058224558080289\\
92.84	0.00581577870929762\\
92.85	0.00580909850020437\\
92.86	0.00580241517650598\\
92.87	0.0057957287339366\\
92.88	0.00578903916820742\\
92.89	0.00578234647500646\\
92.9	0.00577565064999828\\
92.91	0.00576895168882367\\
92.92	0.00576224958709948\\
92.93	0.00575554434041822\\
92.94	0.00574883594434789\\
92.95	0.00574212439443163\\
92.96	0.00573540968618745\\
92.97	0.00572869181510798\\
92.98	0.00572197077666009\\
92.99	0.00571524656628467\\
93	0.0057085191793963\\
93.01	0.00570178861138295\\
93.02	0.00569505485760563\\
93.03	0.00568831791339816\\
93.04	0.00568157777406675\\
93.05	0.00567483443488977\\
93.06	0.00566808789111736\\
93.07	0.00566133813797112\\
93.08	0.00565458517064378\\
93.09	0.00564782898429884\\
93.1	0.00564106957407026\\
93.11	0.00563430693506209\\
93.12	0.00562754106234808\\
93.13	0.00562077195097139\\
93.14	0.00561399959594419\\
93.15	0.00560722399224728\\
93.16	0.00560044513482974\\
93.17	0.00559366301860854\\
93.18	0.00558687763846815\\
93.19	0.00558008898926018\\
93.2	0.00557329706580295\\
93.21	0.00556650186288111\\
93.22	0.00555970337524524\\
93.23	0.00555290159761143\\
93.24	0.00554609652466086\\
93.25	0.0055392881510394\\
93.26	0.00553247647135717\\
93.27	0.0055256614801881\\
93.28	0.00551884317206951\\
93.29	0.00551202154150165\\
93.3	0.00550519658294727\\
93.31	0.00549836829083112\\
93.32	0.00549153665953958\\
93.33	0.00548470168342008\\
93.34	0.00547786335678069\\
93.35	0.00547102167388965\\
93.36	0.00546417662897484\\
93.37	0.00545732821622333\\
93.38	0.00545047642978083\\
93.39	0.00544362126375123\\
93.4	0.00543676271219607\\
93.41	0.00542990076913402\\
93.42	0.00542303542854035\\
93.43	0.00541616668434639\\
93.44	0.005409294530439\\
93.45	0.00540241896066003\\
93.46	0.00539553996880575\\
93.47	0.00538865754862628\\
93.48	0.00538177169382503\\
93.49	0.00537488239805814\\
93.5	0.00536798965493386\\
93.51	0.00536109345801197\\
93.52	0.00535419380080319\\
93.53	0.00534729067676855\\
93.54	0.00534038407931881\\
93.55	0.00533347400181378\\
93.56	0.00532656043756174\\
93.57	0.00531964337981873\\
93.58	0.005312722821788\\
93.59	0.00530579875661924\\
93.6	0.00529887117740799\\
93.61	0.00529194007841584\\
93.62	0.00528500545444815\\
93.63	0.00527806730027387\\
93.64	0.00527112561062511\\
93.65	0.00526418038019663\\
93.66	0.00525723160364533\\
93.67	0.00525027927558981\\
93.68	0.00524332339060986\\
93.69	0.00523636394324591\\
93.7	0.00522940092799855\\
93.71	0.00522243433932801\\
93.72	0.00521546417165362\\
93.73	0.0052084904193533\\
93.74	0.00520151307676297\\
93.75	0.00519453213817603\\
93.76	0.00518754759784282\\
93.77	0.00518055944997\\
93.78	0.00517356768872003\\
93.79	0.00516657230821055\\
93.8	0.00515957330251383\\
93.81	0.00515257066565612\\
93.82	0.00514556439161709\\
93.83	0.00513855447432919\\
93.84	0.00513154090767703\\
93.85	0.00512452368549676\\
93.86	0.00511750280157541\\
93.87	0.00511047824965026\\
93.88	0.00510345002340816\\
93.89	0.00509641811648491\\
93.9	0.00508938252246449\\
93.91	0.00508234323487848\\
93.92	0.00507530024720528\\
93.93	0.00506825355286948\\
93.94	0.00506120314524107\\
93.95	0.00505414901763477\\
93.96	0.00504709116330927\\
93.97	0.00504002957546647\\
93.98	0.00503296424725077\\
93.99	0.00502589517174828\\
94	0.00501882234198602\\
94.01	0.00501174575093118\\
94.02	0.00500466539149026\\
94.03	0.00499758125650835\\
94.04	0.0049904933387682\\
94.05	0.0049834016309895\\
94.06	0.00497630612582794\\
94.07	0.00496920681587443\\
94.08	0.00496210369365419\\
94.09	0.00495499675162588\\
94.1	0.00494788598218072\\
94.11	0.00494077137764159\\
94.12	0.00493365293026209\\
94.13	0.00492653063222566\\
94.14	0.00491940447564458\\
94.15	0.00491227445255907\\
94.16	0.00490514055493629\\
94.17	0.00489800277466938\\
94.18	0.00489086110357646\\
94.19	0.00488371553339963\\
94.2	0.00487656605580393\\
94.21	0.00486941266237634\\
94.22	0.00486225534462472\\
94.23	0.00485509409397671\\
94.24	0.00484792890177873\\
94.25	0.00484075975929481\\
94.26	0.00483358665770555\\
94.27	0.00482640958810694\\
94.28	0.00481922854150927\\
94.29	0.00481204350883597\\
94.3	0.00480485448092243\\
94.31	0.0047976614485148\\
94.32	0.00479046440226885\\
94.33	0.00478326333274869\\
94.34	0.00477605823042558\\
94.35	0.00476884908567667\\
94.36	0.00476163588878373\\
94.37	0.00475441862993188\\
94.38	0.00474719729920827\\
94.39	0.0047399718866008\\
94.4	0.00473274238199673\\
94.41	0.00472550878074039\\
94.42	0.0047182710786135\\
94.43	0.00471102927139192\\
94.44	0.00470378335484564\\
94.45	0.0046965333247388\\
94.46	0.00468927917682961\\
94.47	0.00468202090687045\\
94.48	0.00467475851060777\\
94.49	0.00466749198378214\\
94.5	0.00466022132212823\\
94.51	0.00465294652137478\\
94.52	0.00464566757724465\\
94.53	0.00463838448545476\\
94.54	0.00463109724171612\\
94.55	0.00462380584173379\\
94.56	0.00461651028120693\\
94.57	0.00460921055582874\\
94.58	0.00460190666128647\\
94.59	0.00459459859326145\\
94.6	0.00458728634742903\\
94.61	0.00457996991945862\\
94.62	0.00457264930501367\\
94.63	0.00456532449975164\\
94.64	0.00455799549932404\\
94.65	0.00455066229937642\\
94.66	0.00454332489554831\\
94.67	0.0045359832834733\\
94.68	0.00452863745877897\\
94.69	0.00452128741727185\\
94.7	0.00451393315485705\\
94.71	0.00450657466743472\\
94.72	0.00449921195089999\\
94.73	0.00449184500114304\\
94.74	0.00448447381404904\\
94.75	0.00447709838549819\\
94.76	0.00446971871136567\\
94.77	0.00446233478752166\\
94.78	0.00445494660983133\\
94.79	0.00444755417415486\\
94.8	0.0044401574763474\\
94.81	0.00443275651225907\\
94.82	0.00442535127773499\\
94.83	0.00441794176861524\\
94.84	0.00441052798073486\\
94.85	0.00440310990992386\\
94.86	0.00439568755200722\\
94.87	0.00438826090280486\\
94.88	0.00438082995813166\\
94.89	0.00437339471379744\\
94.9	0.00436595516560698\\
94.91	0.00435851130935997\\
94.92	0.00435106314085105\\
94.93	0.00434361065586981\\
94.94	0.00433615385020074\\
94.95	0.00432869271962326\\
94.96	0.00432122725991171\\
94.97	0.00431375746683534\\
94.98	0.00430628333615831\\
94.99	0.00429880486363971\\
95	0.0042913220450335\\
95.01	0.00428383487608855\\
95.02	0.00427634335254864\\
95.03	0.00426884747015242\\
95.04	0.00426134722463344\\
95.05	0.00425384261172012\\
95.06	0.00424633362713577\\
95.07	0.00423882026659858\\
95.08	0.0042313025258216\\
95.09	0.00422378040051275\\
95.1	0.00421625388637482\\
95.11	0.00420872297910546\\
95.12	0.00420118767439716\\
95.13	0.00419364796793729\\
95.14	0.00418610385540806\\
95.15	0.0041785553324865\\
95.16	0.00417100239484452\\
95.17	0.00416344503814884\\
95.18	0.00415588325806103\\
95.19	0.00414831705023748\\
95.2	0.00414074641032943\\
95.21	0.00413317133398291\\
95.22	0.00412559181683879\\
95.23	0.00411800785453276\\
95.24	0.00411041944269531\\
95.25	0.00410282657695175\\
95.26	0.00409522925292219\\
95.27	0.00408762746622156\\
95.28	0.00408002121245956\\
95.29	0.0040724104872407\\
95.3	0.0040647952861643\\
95.31	0.00405717560482444\\
95.32	0.00404955143881\\
95.33	0.00404192278370465\\
95.34	0.00403428963508684\\
95.35	0.00402665198852976\\
95.36	0.00401900983960143\\
95.37	0.00401136318386459\\
95.38	0.00400371201687679\\
95.39	0.0039960563341903\\
95.4	0.00398839613135218\\
95.41	0.00398073140390425\\
95.42	0.00397306214738305\\
95.43	0.00396538835731993\\
95.44	0.00395771002924092\\
95.45	0.00395002715866684\\
95.46	0.00394233974111325\\
95.47	0.00393464777209044\\
95.48	0.00392695124710342\\
95.49	0.00391925016165197\\
95.5	0.00391154451123058\\
95.51	0.00390383429132846\\
95.52	0.00389611949742956\\
95.53	0.00388840012501254\\
95.54	0.00388067616955081\\
95.55	0.00387294762651245\\
95.56	0.0038652144913603\\
95.57	0.00385747675955188\\
95.58	0.00384973442653943\\
95.59	0.00384198748776991\\
95.6	0.00383423593868495\\
95.61	0.00382647977472092\\
95.62	0.00381871899130887\\
95.63	0.00381095358387455\\
95.64	0.0038031835478384\\
95.65	0.00379540887861556\\
95.66	0.00378762957161586\\
95.67	0.0037798456222438\\
95.68	0.00377205702589859\\
95.69	0.00376426377797411\\
95.7	0.00375646587385892\\
95.71	0.00374866330893626\\
95.72	0.00374085607858406\\
95.73	0.0037330441781749\\
95.74	0.00372522760307604\\
95.75	0.00371740634864943\\
95.76	0.00370958041025166\\
95.77	0.00370174978323402\\
95.78	0.00369391446294243\\
95.79	0.00368607444471748\\
95.8	0.00367822972389445\\
95.81	0.00367038029580324\\
95.82	0.00366252615576844\\
95.83	0.00365466729910927\\
95.84	0.00364680372113963\\
95.85	0.00363893541716805\\
95.86	0.00363106238249771\\
95.87	0.00362318461242647\\
95.88	0.0036153021022468\\
95.89	0.00360741484724584\\
95.9	0.00359952284270538\\
95.91	0.00359162608390182\\
95.92	0.00358372456610625\\
95.93	0.00357581828458436\\
95.94	0.00356790723459651\\
95.95	0.00355999141139768\\
95.96	0.00355207081023749\\
95.97	0.00354414542636021\\
95.98	0.00353621525500474\\
95.99	0.0035282802914046\\
96	0.00352034053078797\\
96.01	0.00351239596837763\\
96.02	0.00350444659939104\\
96.03	0.00349649241904023\\
96.04	0.00348853342253192\\
96.05	0.00348056960506742\\
96.06	0.0034726009618427\\
96.07	0.00346462748804832\\
96.08	0.0034566491788695\\
96.09	0.00344866602948609\\
96.1	0.00344067803507254\\
96.11	0.00343268519079795\\
96.12	0.00342468749182604\\
96.13	0.00341668493331517\\
96.14	0.0034086775104183\\
96.15	0.00340066521828302\\
96.16	0.00339264805205159\\
96.17	0.00338462600686083\\
96.18	0.00337659907784224\\
96.19	0.00336856726012192\\
96.2	0.0033605305488206\\
96.21	0.00335248893905364\\
96.22	0.00334444242593103\\
96.23	0.00333639100455739\\
96.24	0.00332833467003195\\
96.25	0.00332027341744859\\
96.26	0.0033122072418958\\
96.27	0.00330413613845672\\
96.28	0.00329606010220911\\
96.29	0.00328797912822536\\
96.3	0.00327989321157249\\
96.31	0.00327180234731215\\
96.32	0.00326370653050064\\
96.33	0.00325560575618888\\
96.34	0.00324750001942242\\
96.35	0.00323938931524148\\
96.36	0.00323127363868087\\
96.37	0.00322315298477007\\
96.38	0.00321502734853321\\
96.39	0.00320689672498903\\
96.4	0.00319876110915094\\
96.41	0.00319062049602698\\
96.42	0.00318247488061984\\
96.43	0.00317432425792688\\
96.44	0.00316616862294007\\
96.45	0.00315800797064607\\
96.46	0.00314984229602617\\
96.47	0.00314167159405633\\
96.48	0.00313349585970716\\
96.49	0.00312531508794394\\
96.5	0.0031171292737266\\
96.51	0.00310893841200975\\
96.52	0.00310074249774266\\
96.53	0.00309254152586927\\
96.54	0.00308433549132819\\
96.55	0.00307612438905271\\
96.56	0.00306790821397081\\
96.57	0.00305968696100513\\
96.58	0.00305146062507301\\
96.59	0.00304322920108647\\
96.6	0.00303499268395221\\
96.61	0.00302675106857165\\
96.62	0.00301850434984089\\
96.63	0.00301025252265073\\
96.64	0.00300199558188669\\
96.65	0.00299373352242897\\
96.66	0.0029854663391525\\
96.67	0.00297719402692693\\
96.68	0.00296891658061661\\
96.69	0.00296063399508063\\
96.7	0.00295234626517281\\
96.71	0.00294405338574168\\
96.72	0.00293575535163052\\
96.73	0.00292745215767735\\
96.74	0.00291914379871494\\
96.75	0.0029108302695708\\
96.76	0.00290251156506715\\
96.77	0.0028941876800207\\
96.78	0.00288585860924267\\
96.79	0.00287752434753879\\
96.8	0.00286918488970926\\
96.81	0.00286084023054882\\
96.82	0.00285249036484668\\
96.83	0.00284413528738658\\
96.84	0.00283577499294675\\
96.85	0.00282740947629995\\
96.86	0.00281903873221342\\
96.87	0.00281066275544894\\
96.88	0.00280228154076279\\
96.89	0.00279389508290576\\
96.9	0.00278550337662318\\
96.91	0.00277710641665487\\
96.92	0.00276870419773519\\
96.93	0.00276029671459302\\
96.94	0.00275188396195176\\
96.95	0.00274346593452935\\
96.96	0.00273504262703824\\
96.97	0.00272661403418543\\
96.98	0.00271818015067244\\
96.99	0.00270974097119535\\
97	0.00270129649044475\\
97.01	0.00269284670310579\\
97.02	0.00268439160385817\\
97.03	0.00267593118737612\\
97.04	0.00266746544832843\\
97.05	0.00265899438137845\\
97.06	0.00265051798118407\\
97.07	0.00264203624239776\\
97.08	0.00263354915966653\\
97.09	0.00262505672763197\\
97.1	0.00261655894093024\\
97.11	0.00260805579419207\\
97.12	0.00259954728204275\\
97.13	0.00259103339910218\\
97.14	0.00258251413998481\\
97.15	0.00257398949929971\\
97.16	0.0025654594716505\\
97.17	0.00255692405163544\\
97.18	0.00254838323384734\\
97.19	0.00253983701287365\\
97.2	0.00253128538329641\\
97.21	0.00252272833969227\\
97.22	0.0025141658766325\\
97.23	0.00250559798868298\\
97.24	0.00249702467040422\\
97.25	0.00248844591635134\\
97.26	0.00247986172107413\\
97.27	0.00247127207911698\\
97.28	0.00246267698501893\\
97.29	0.00245407643331368\\
97.3	0.00244547041852957\\
97.31	0.00243685893518959\\
97.32	0.0024282419778114\\
97.33	0.00241961954090733\\
97.34	0.00241099161898438\\
97.35	0.00240235820654421\\
97.36	0.00239371929808318\\
97.37	0.00238507488809233\\
97.38	0.0023764249710574\\
97.39	0.00236776954145883\\
97.4	0.00235910859377175\\
97.41	0.00235044212246602\\
97.42	0.0023417701220062\\
97.43	0.00233309258685159\\
97.44	0.0023244095114562\\
97.45	0.00231572089026879\\
97.46	0.00230702671773286\\
97.47	0.00229832698828666\\
97.48	0.00228962169636317\\
97.49	0.00228091083639017\\
97.5	0.00227219440279019\\
97.51	0.00226347238998052\\
97.52	0.00225474479237327\\
97.53	0.00224601160437531\\
97.54	0.00223727282038831\\
97.55	0.00222852843480874\\
97.56	0.00221977844202791\\
97.57	0.00221102283643192\\
97.58	0.0022022616124017\\
97.59	0.00219349476431302\\
97.6	0.0021847222865365\\
97.61	0.0021759441734376\\
97.62	0.00216716041937663\\
97.63	0.00215837101870879\\
97.64	0.00214957596578414\\
97.65	0.00214077525494762\\
97.66	0.00213196888053906\\
97.67	0.00212315683689321\\
97.68	0.00211433911833971\\
97.69	0.00210551571920312\\
97.7	0.00209668663380295\\
97.71	0.0020878518564536\\
97.72	0.00207901138146446\\
97.73	0.00207016520313986\\
97.74	0.00206131331577909\\
97.75	0.00205245571367641\\
97.76	0.00204359239112108\\
97.77	0.00203472334239733\\
97.78	0.00202584856178441\\
97.79	0.0020169680435566\\
97.8	0.00200808178198316\\
97.81	0.00199918977132842\\
97.82	0.00199029200585175\\
97.83	0.00198138847980756\\
97.84	0.00197247918744535\\
97.85	0.00196356412300966\\
97.86	0.00195464328074016\\
97.87	0.00194571665487159\\
97.88	0.00193678423963382\\
97.89	0.00192784602925183\\
97.9	0.00191890201794574\\
97.91	0.0019099521999308\\
97.92	0.00190099656941745\\
97.93	0.00189203512061126\\
97.94	0.00188306784771302\\
97.95	0.00187409474491867\\
97.96	0.0018651158064194\\
97.97	0.0018561310264016\\
97.98	0.00184714039904687\\
97.99	0.0018381439185321\\
98	0.0018291415790294\\
98.01	0.00182013337470615\\
98.02	0.00181111929972505\\
98.03	0.00180209934824406\\
98.04	0.00179307351441646\\
98.05	0.00178404179239085\\
98.06	0.00177500417631119\\
98.07	0.00176596066031677\\
98.08	0.00175691123854224\\
98.09	0.00174785590511765\\
98.1	0.00173879465416847\\
98.11	0.00172972747981571\\
98.12	0.0017206543761759\\
98.13	0.00171157533736114\\
98.14	0.00170249035747905\\
98.15	0.00169339943063285\\
98.16	0.00168430255092135\\
98.17	0.00167519971243899\\
98.18	0.00166609090927615\\
98.19	0.00165697613551974\\
98.2	0.00164785538525315\\
98.21	0.00163872865255636\\
98.22	0.00162959593150587\\
98.23	0.00162045721617481\\
98.24	0.00161131250063292\\
98.25	0.00160216177894657\\
98.26	0.0015930050451788\\
98.27	0.00158384229338934\\
98.28	0.00157467351763463\\
98.29	0.00156549871196786\\
98.3	0.00155631787043897\\
98.31	0.0015471309870947\\
98.32	0.0015379380559786\\
98.33	0.00152873907109969\\
98.34	0.00151953402646197\\
98.35	0.00151032291569093\\
98.36	0.00150110573241224\\
98.37	0.00149188247025179\\
98.38	0.00148265312283583\\
98.39	0.00147341768379099\\
98.4	0.00146417614666711\\
98.41	0.00145492850465893\\
98.42	0.00144567474861469\\
98.43	0.00143641486935398\\
98.44	0.00142714885766748\\
98.45	0.00141787670431677\\
98.46	0.00140859840003403\\
98.47	0.00139931393552186\\
98.48	0.00139002330145297\\
98.49	0.00138072648846998\\
98.5	0.00137142348718512\\
98.51	0.00136211428818003\\
98.52	0.00135279888200543\\
98.53	0.00134347725918092\\
98.54	0.00133414941019467\\
98.55	0.00132481532550315\\
98.56	0.00131547499553094\\
98.57	0.00130612841067081\\
98.58	0.00129677556128345\\
98.59	0.00128741643769718\\
98.6	0.00127805103020772\\
98.61	0.00126867932907785\\
98.62	0.00125930132453712\\
98.63	0.0012499170175457\\
98.64	0.00124052648239183\\
98.65	0.00123112970820827\\
98.66	0.00122172668408664\\
98.67	0.00121231739907713\\
98.68	0.00120290184218811\\
98.69	0.00119348000238578\\
98.7	0.00118405186859308\\
98.71	0.0011746174296891\\
98.72	0.00116517667450878\\
98.73	0.00115572959184246\\
98.74	0.00114627672733929\\
98.75	0.00113681867495569\\
98.76	0.00112735542682799\\
98.77	0.00111788697506406\\
98.78	0.00110841343759048\\
98.79	0.0010989354539535\\
98.8	0.00108946054226941\\
98.81	0.00107998876405688\\
98.82	0.00107052018144894\\
98.83	0.00106105485720021\\
98.84	0.00105159285469419\\
98.85	0.0010421342379507\\
98.86	0.00103267907163337\\
98.87	0.00102322742105735\\
98.88	0.00101377935219699\\
98.89	0.00100433493169381\\
98.9	0.000994894226864423\\
98.91	0.000985457305708684\\
98.92	0.000976024235873082\\
98.93	0.000966595085556268\\
98.94	0.000957169923641867\\
98.95	0.000947748819706719\\
98.96	0.000938331844029227\\
98.97	0.000928919067597863\\
98.98	0.000919510562119788\\
98.99	0.000910106400029611\\
99	0.00090070665449829\\
99.01	0.000891311399442173\\
99.02	0.000881920709532172\\
99.03	0.0008725346602031\\
99.04	0.000863153327663143\\
99.05	0.000853776788903485\\
99.06	0.000844405121708091\\
99.07	0.000835038404663648\\
99.08	0.000825676717169661\\
99.09	0.000816320139448719\\
99.1	0.000806968752556916\\
99.11	0.000797622638394458\\
99.12	0.000788281879716437\\
99.13	0.000778946560143766\\
99.14	0.000769616764174317\\
99.15	0.000760292577194233\\
99.16	0.000750974085489424\\
99.17	0.000741661376257259\\
99.18	0.000732354537618451\\
99.19	0.000723053658629137\\
99.2	0.000713758829293178\\
99.21	0.000704470140574644\\
99.22	0.000695187684410533\\
99.23	0.000685911553723684\\
99.24	0.000676641842435923\\
99.25	0.000667378645481434\\
99.26	0.00065812205882035\\
99.27	0.000648872179452586\\
99.28	0.000639629105431912\\
99.29	0.000630392935880257\\
99.3	0.000621163771002279\\
99.31	0.000611941712100177\\
99.32	0.000602726861588772\\
99.33	0.000593519323010844\\
99.34	0.000584319201052751\\
99.35	0.000575126601560311\\
99.36	0.000565941631554984\\
99.37	0.00055676439925033\\
99.38	0.00054759501406877\\
99.39	0.000538433586658444\\
99.4	0.000529280228901823\\
99.41	0.000520135053933318\\
99.42	0.000510998176157183\\
99.43	0.000501869711265774\\
99.44	0.000492749776258126\\
99.45	0.000483638489458892\\
99.46	0.000474535970537619\\
99.47	0.000465442340528399\\
99.48	0.000456357721849881\\
99.49	0.000447282238325672\\
99.5	0.000438216015205108\\
99.51	0.000429159179184449\\
99.52	0.000420111858428454\\
99.53	0.000411074182592397\\
99.54	0.000402046282844485\\
99.55	0.000393028291888746\\
99.56	0.000384020343988333\\
99.57	0.000375022574989311\\
99.58	0.000366035122344904\\
99.59	0.000357058125140244\\
99.6	0.000348091724117582\\
99.61	0.000339136061702052\\
99.62	0.000330191282027605\\
99.63	0.000321257530963142\\
99.64	0.000312334956139852\\
99.65	0.000303423706979116\\
99.66	0.000294523934720993\\
99.67	0.000285635792453304\\
99.68	0.000276759435141329\\
99.69	0.000267895019658133\\
99.7	0.000259042704815544\\
99.71	0.000250202651395793\\
99.72	0.000241375022183844\\
99.73	0.000232559984946454\\
99.74	0.000223757712161201\\
99.75	0.000214968378467508\\
99.76	0.000206192160702477\\
99.77	0.000197429237937504\\
99.78	0.000188679791515737\\
99.79	0.00017994400509036\\
99.8	0.000171222064663754\\
99.81	0.000162514158627553\\
99.82	0.000153820477803632\\
99.83	0.000145141215486051\\
99.84	0.000136476567483968\\
99.85	0.00012782673216559\\
99.86	0.000119191910503162\\
99.87	0.000110572306119041\\
99.88	0.000101968125332889\\
99.89	9.33795772100343e-05\\
99.9	8.4806873611008e-05\\
99.91	7.62502292423421e-05\\
99.92	6.77098617086289e-05\\
99.93	5.91859915659125e-05\\
99.94	5.06788423764483e-05\\
99.95	4.21886407648911e-05\\
99.96	3.37156164759312e-05\\
99.97	2.52600024334866e-05\\
99.98	1.68220348014392e-05\\
99.99	8.40195304604129e-06\\
100	0\\
};
\addlegendentry{$q=-1$};

\addplot [color=black,solid,forget plot]
  table[row sep=crcr]{%
0.01	0\\
0.02	0\\
0.03	0\\
0.04	0\\
0.05	0\\
0.06	0\\
0.07	0\\
0.08	0\\
0.09	0\\
0.1	0\\
0.11	0\\
0.12	0\\
0.13	0\\
0.14	0\\
0.15	0\\
0.16	0\\
0.17	0\\
0.18	0\\
0.19	0\\
0.2	0\\
0.21	0\\
0.22	0\\
0.23	0\\
0.24	0\\
0.25	0\\
0.26	0\\
0.27	0\\
0.28	0\\
0.29	0\\
0.3	0\\
0.31	0\\
0.32	0\\
0.33	0\\
0.34	0\\
0.35	0\\
0.36	0\\
0.37	0\\
0.38	0\\
0.39	0\\
0.4	0\\
0.41	0\\
0.42	0\\
0.43	0\\
0.44	0\\
0.45	0\\
0.46	0\\
0.47	0\\
0.48	0\\
0.49	0\\
0.5	0\\
0.51	0\\
0.52	0\\
0.53	0\\
0.54	0\\
0.55	0\\
0.56	0\\
0.57	0\\
0.58	0\\
0.59	0\\
0.6	0\\
0.61	0\\
0.62	0\\
0.63	0\\
0.64	0\\
0.65	0\\
0.66	0\\
0.67	0\\
0.68	0\\
0.69	0\\
0.7	0\\
0.71	0\\
0.72	0\\
0.73	0\\
0.74	0\\
0.75	0\\
0.76	0\\
0.77	0\\
0.78	0\\
0.79	0\\
0.8	0\\
0.81	0\\
0.82	0\\
0.83	0\\
0.84	0\\
0.85	0\\
0.86	0\\
0.87	0\\
0.88	0\\
0.89	0\\
0.9	0\\
0.91	0\\
0.92	0\\
0.93	0\\
0.94	0\\
0.95	0\\
0.96	0\\
0.97	0\\
0.98	0\\
0.99	0\\
1	0\\
1.01	0\\
1.02	0\\
1.03	0\\
1.04	0\\
1.05	0\\
1.06	0\\
1.07	0\\
1.08	0\\
1.09	0\\
1.1	0\\
1.11	0\\
1.12	0\\
1.13	0\\
1.14	0\\
1.15	0\\
1.16	0\\
1.17	0\\
1.18	0\\
1.19	0\\
1.2	0\\
1.21	0\\
1.22	0\\
1.23	0\\
1.24	0\\
1.25	0\\
1.26	0\\
1.27	0\\
1.28	0\\
1.29	0\\
1.3	0\\
1.31	0\\
1.32	0\\
1.33	0\\
1.34	0\\
1.35	0\\
1.36	0\\
1.37	0\\
1.38	0\\
1.39	0\\
1.4	0\\
1.41	0\\
1.42	0\\
1.43	0\\
1.44	0\\
1.45	0\\
1.46	0\\
1.47	0\\
1.48	0\\
1.49	0\\
1.5	0\\
1.51	0\\
1.52	0\\
1.53	0\\
1.54	0\\
1.55	0\\
1.56	0\\
1.57	0\\
1.58	0\\
1.59	0\\
1.6	0\\
1.61	0\\
1.62	0\\
1.63	0\\
1.64	0\\
1.65	0\\
1.66	0\\
1.67	0\\
1.68	0\\
1.69	0\\
1.7	0\\
1.71	0\\
1.72	0\\
1.73	0\\
1.74	0\\
1.75	0\\
1.76	0\\
1.77	0\\
1.78	0\\
1.79	0\\
1.8	0\\
1.81	0\\
1.82	0\\
1.83	0\\
1.84	0\\
1.85	0\\
1.86	0\\
1.87	0\\
1.88	0\\
1.89	0\\
1.9	0\\
1.91	0\\
1.92	0\\
1.93	0\\
1.94	0\\
1.95	0\\
1.96	0\\
1.97	0\\
1.98	0\\
1.99	0\\
2	0\\
2.01	0\\
2.02	0\\
2.03	0\\
2.04	0\\
2.05	0\\
2.06	0\\
2.07	0\\
2.08	0\\
2.09	0\\
2.1	0\\
2.11	0\\
2.12	0\\
2.13	0\\
2.14	0\\
2.15	0\\
2.16	0\\
2.17	0\\
2.18	0\\
2.19	0\\
2.2	0\\
2.21	0\\
2.22	0\\
2.23	0\\
2.24	0\\
2.25	0\\
2.26	0\\
2.27	0\\
2.28	0\\
2.29	0\\
2.3	0\\
2.31	0\\
2.32	0\\
2.33	0\\
2.34	0\\
2.35	0\\
2.36	0\\
2.37	0\\
2.38	0\\
2.39	0\\
2.4	0\\
2.41	0\\
2.42	0\\
2.43	0\\
2.44	0\\
2.45	0\\
2.46	0\\
2.47	0\\
2.48	0\\
2.49	0\\
2.5	0\\
2.51	0\\
2.52	0\\
2.53	0\\
2.54	0\\
2.55	0\\
2.56	0\\
2.57	0\\
2.58	0\\
2.59	0\\
2.6	0\\
2.61	0\\
2.62	0\\
2.63	0\\
2.64	0\\
2.65	0\\
2.66	0\\
2.67	0\\
2.68	0\\
2.69	0\\
2.7	0\\
2.71	0\\
2.72	0\\
2.73	0\\
2.74	0\\
2.75	0\\
2.76	0\\
2.77	0\\
2.78	0\\
2.79	0\\
2.8	0\\
2.81	0\\
2.82	0\\
2.83	0\\
2.84	0\\
2.85	0\\
2.86	0\\
2.87	0\\
2.88	0\\
2.89	0\\
2.9	0\\
2.91	0\\
2.92	0\\
2.93	0\\
2.94	0\\
2.95	0\\
2.96	0\\
2.97	0\\
2.98	0\\
2.99	0\\
3	0\\
3.01	0\\
3.02	0\\
3.03	0\\
3.04	0\\
3.05	0\\
3.06	0\\
3.07	0\\
3.08	0\\
3.09	0\\
3.1	0\\
3.11	0\\
3.12	0\\
3.13	0\\
3.14	0\\
3.15	0\\
3.16	0\\
3.17	0\\
3.18	0\\
3.19	0\\
3.2	0\\
3.21	0\\
3.22	0\\
3.23	0\\
3.24	0\\
3.25	0\\
3.26	0\\
3.27	0\\
3.28	0\\
3.29	0\\
3.3	0\\
3.31	0\\
3.32	0\\
3.33	0\\
3.34	0\\
3.35	0\\
3.36	0\\
3.37	0\\
3.38	0\\
3.39	0\\
3.4	0\\
3.41	0\\
3.42	0\\
3.43	0\\
3.44	0\\
3.45	0\\
3.46	0\\
3.47	0\\
3.48	0\\
3.49	0\\
3.5	0\\
3.51	0\\
3.52	0\\
3.53	0\\
3.54	0\\
3.55	0\\
3.56	0\\
3.57	0\\
3.58	0\\
3.59	0\\
3.6	0\\
3.61	0\\
3.62	0\\
3.63	0\\
3.64	0\\
3.65	0\\
3.66	0\\
3.67	0\\
3.68	0\\
3.69	0\\
3.7	0\\
3.71	0\\
3.72	0\\
3.73	0\\
3.74	0\\
3.75	0\\
3.76	0\\
3.77	0\\
3.78	0\\
3.79	0\\
3.8	0\\
3.81	0\\
3.82	0\\
3.83	0\\
3.84	0\\
3.85	0\\
3.86	0\\
3.87	0\\
3.88	0\\
3.89	0\\
3.9	0\\
3.91	0\\
3.92	0\\
3.93	0\\
3.94	0\\
3.95	0\\
3.96	0\\
3.97	0\\
3.98	0\\
3.99	0\\
4	0\\
4.01	0\\
4.02	0\\
4.03	0\\
4.04	0\\
4.05	0\\
4.06	0\\
4.07	0\\
4.08	0\\
4.09	0\\
4.1	0\\
4.11	0\\
4.12	0\\
4.13	0\\
4.14	0\\
4.15	0\\
4.16	0\\
4.17	0\\
4.18	0\\
4.19	0\\
4.2	0\\
4.21	0\\
4.22	0\\
4.23	0\\
4.24	0\\
4.25	0\\
4.26	0\\
4.27	0\\
4.28	0\\
4.29	0\\
4.3	0\\
4.31	0\\
4.32	0\\
4.33	0\\
4.34	0\\
4.35	0\\
4.36	0\\
4.37	0\\
4.38	0\\
4.39	0\\
4.4	0\\
4.41	0\\
4.42	0\\
4.43	0\\
4.44	0\\
4.45	0\\
4.46	0\\
4.47	0\\
4.48	0\\
4.49	0\\
4.5	0\\
4.51	0\\
4.52	0\\
4.53	0\\
4.54	0\\
4.55	0\\
4.56	0\\
4.57	0\\
4.58	0\\
4.59	0\\
4.6	0\\
4.61	0\\
4.62	0\\
4.63	0\\
4.64	0\\
4.65	0\\
4.66	0\\
4.67	0\\
4.68	0\\
4.69	0\\
4.7	0\\
4.71	0\\
4.72	0\\
4.73	0\\
4.74	0\\
4.75	0\\
4.76	0\\
4.77	0\\
4.78	0\\
4.79	0\\
4.8	0\\
4.81	0\\
4.82	0\\
4.83	0\\
4.84	0\\
4.85	0\\
4.86	0\\
4.87	0\\
4.88	0\\
4.89	0\\
4.9	0\\
4.91	0\\
4.92	0\\
4.93	0\\
4.94	0\\
4.95	0\\
4.96	0\\
4.97	0\\
4.98	0\\
4.99	0\\
5	0\\
5.01	0\\
5.02	0\\
5.03	0\\
5.04	0\\
5.05	0\\
5.06	0\\
5.07	0\\
5.08	0\\
5.09	0\\
5.1	0\\
5.11	0\\
5.12	0\\
5.13	0\\
5.14	0\\
5.15	0\\
5.16	0\\
5.17	0\\
5.18	0\\
5.19	0\\
5.2	0\\
5.21	0\\
5.22	0\\
5.23	0\\
5.24	0\\
5.25	0\\
5.26	0\\
5.27	0\\
5.28	0\\
5.29	0\\
5.3	0\\
5.31	0\\
5.32	0\\
5.33	0\\
5.34	0\\
5.35	0\\
5.36	0\\
5.37	0\\
5.38	0\\
5.39	0\\
5.4	0\\
5.41	0\\
5.42	0\\
5.43	0\\
5.44	0\\
5.45	0\\
5.46	0\\
5.47	0\\
5.48	0\\
5.49	0\\
5.5	0\\
5.51	0\\
5.52	0\\
5.53	0\\
5.54	0\\
5.55	0\\
5.56	0\\
5.57	0\\
5.58	0\\
5.59	0\\
5.6	0\\
5.61	0\\
5.62	0\\
5.63	0\\
5.64	0\\
5.65	0\\
5.66	0\\
5.67	0\\
5.68	0\\
5.69	0\\
5.7	0\\
5.71	0\\
5.72	0\\
5.73	0\\
5.74	0\\
5.75	0\\
5.76	0\\
5.77	0\\
5.78	0\\
5.79	0\\
5.8	0\\
5.81	0\\
5.82	0\\
5.83	0\\
5.84	0\\
5.85	0\\
5.86	0\\
5.87	0\\
5.88	0\\
5.89	0\\
5.9	0\\
5.91	0\\
5.92	0\\
5.93	0\\
5.94	0\\
5.95	0\\
5.96	0\\
5.97	0\\
5.98	0\\
5.99	0\\
6	0\\
6.01	0\\
6.02	0\\
6.03	0\\
6.04	0\\
6.05	0\\
6.06	0\\
6.07	0\\
6.08	0\\
6.09	0\\
6.1	0\\
6.11	0\\
6.12	0\\
6.13	0\\
6.14	0\\
6.15	0\\
6.16	0\\
6.17	0\\
6.18	0\\
6.19	0\\
6.2	0\\
6.21	0\\
6.22	0\\
6.23	0\\
6.24	0\\
6.25	0\\
6.26	0\\
6.27	0\\
6.28	0\\
6.29	0\\
6.3	0\\
6.31	0\\
6.32	0\\
6.33	0\\
6.34	0\\
6.35	0\\
6.36	0\\
6.37	0\\
6.38	0\\
6.39	0\\
6.4	0\\
6.41	0\\
6.42	0\\
6.43	0\\
6.44	0\\
6.45	0\\
6.46	0\\
6.47	0\\
6.48	0\\
6.49	0\\
6.5	0\\
6.51	0\\
6.52	0\\
6.53	0\\
6.54	0\\
6.55	0\\
6.56	0\\
6.57	0\\
6.58	0\\
6.59	0\\
6.6	0\\
6.61	0\\
6.62	0\\
6.63	0\\
6.64	0\\
6.65	0\\
6.66	0\\
6.67	0\\
6.68	0\\
6.69	0\\
6.7	0\\
6.71	0\\
6.72	0\\
6.73	0\\
6.74	0\\
6.75	0\\
6.76	0\\
6.77	0\\
6.78	0\\
6.79	0\\
6.8	0\\
6.81	0\\
6.82	0\\
6.83	0\\
6.84	0\\
6.85	0\\
6.86	0\\
6.87	0\\
6.88	0\\
6.89	0\\
6.9	0\\
6.91	0\\
6.92	0\\
6.93	0\\
6.94	0\\
6.95	0\\
6.96	0\\
6.97	0\\
6.98	0\\
6.99	0\\
7	0\\
7.01	0\\
7.02	0\\
7.03	0\\
7.04	0\\
7.05	0\\
7.06	0\\
7.07	0\\
7.08	0\\
7.09	0\\
7.1	0\\
7.11	0\\
7.12	0\\
7.13	0\\
7.14	0\\
7.15	0\\
7.16	0\\
7.17	0\\
7.18	0\\
7.19	0\\
7.2	0\\
7.21	0\\
7.22	0\\
7.23	0\\
7.24	0\\
7.25	0\\
7.26	0\\
7.27	0\\
7.28	0\\
7.29	0\\
7.3	0\\
7.31	0\\
7.32	0\\
7.33	0\\
7.34	0\\
7.35	0\\
7.36	0\\
7.37	0\\
7.38	0\\
7.39	0\\
7.4	0\\
7.41	0\\
7.42	0\\
7.43	0\\
7.44	0\\
7.45	0\\
7.46	0\\
7.47	0\\
7.48	0\\
7.49	0\\
7.5	0\\
7.51	0\\
7.52	0\\
7.53	0\\
7.54	0\\
7.55	0\\
7.56	0\\
7.57	0\\
7.58	0\\
7.59	0\\
7.6	0\\
7.61	0\\
7.62	0\\
7.63	0\\
7.64	0\\
7.65	0\\
7.66	0\\
7.67	0\\
7.68	0\\
7.69	0\\
7.7	0\\
7.71	0\\
7.72	0\\
7.73	0\\
7.74	0\\
7.75	0\\
7.76	0\\
7.77	0\\
7.78	0\\
7.79	0\\
7.8	0\\
7.81	0\\
7.82	0\\
7.83	0\\
7.84	0\\
7.85	0\\
7.86	0\\
7.87	0\\
7.88	0\\
7.89	0\\
7.9	0\\
7.91	0\\
7.92	0\\
7.93	0\\
7.94	0\\
7.95	0\\
7.96	0\\
7.97	0\\
7.98	0\\
7.99	0\\
8	0\\
8.01	0\\
8.02	0\\
8.03	0\\
8.04	0\\
8.05	0\\
8.06	0\\
8.07	0\\
8.08	0\\
8.09	0\\
8.1	0\\
8.11	0\\
8.12	0\\
8.13	0\\
8.14	0\\
8.15	0\\
8.16	0\\
8.17	0\\
8.18	0\\
8.19	0\\
8.2	0\\
8.21	0\\
8.22	0\\
8.23	0\\
8.24	0\\
8.25	0\\
8.26	0\\
8.27	0\\
8.28	0\\
8.29	0\\
8.3	0\\
8.31	0\\
8.32	0\\
8.33	0\\
8.34	0\\
8.35	0\\
8.36	0\\
8.37	0\\
8.38	0\\
8.39	0\\
8.4	0\\
8.41	0\\
8.42	0\\
8.43	0\\
8.44	0\\
8.45	0\\
8.46	0\\
8.47	0\\
8.48	0\\
8.49	0\\
8.5	0\\
8.51	0\\
8.52	0\\
8.53	0\\
8.54	0\\
8.55	0\\
8.56	0\\
8.57	0\\
8.58	0\\
8.59	0\\
8.6	0\\
8.61	0\\
8.62	0\\
8.63	0\\
8.64	0\\
8.65	0\\
8.66	0\\
8.67	0\\
8.68	0\\
8.69	0\\
8.7	0\\
8.71	0\\
8.72	0\\
8.73	0\\
8.74	0\\
8.75	0\\
8.76	0\\
8.77	0\\
8.78	0\\
8.79	0\\
8.8	0\\
8.81	0\\
8.82	0\\
8.83	0\\
8.84	0\\
8.85	0\\
8.86	0\\
8.87	0\\
8.88	0\\
8.89	0\\
8.9	0\\
8.91	0\\
8.92	0\\
8.93	0\\
8.94	0\\
8.95	0\\
8.96	0\\
8.97	0\\
8.98	0\\
8.99	0\\
9	0\\
9.01	0\\
9.02	0\\
9.03	0\\
9.04	0\\
9.05	0\\
9.06	0\\
9.07	0\\
9.08	0\\
9.09	0\\
9.1	0\\
9.11	0\\
9.12	0\\
9.13	0\\
9.14	0\\
9.15	0\\
9.16	0\\
9.17	0\\
9.18	0\\
9.19	0\\
9.2	0\\
9.21	0\\
9.22	0\\
9.23	0\\
9.24	0\\
9.25	0\\
9.26	0\\
9.27	0\\
9.28	0\\
9.29	0\\
9.3	0\\
9.31	0\\
9.32	0\\
9.33	0\\
9.34	0\\
9.35	0\\
9.36	0\\
9.37	0\\
9.38	0\\
9.39	0\\
9.4	0\\
9.41	0\\
9.42	0\\
9.43	0\\
9.44	0\\
9.45	0\\
9.46	0\\
9.47	0\\
9.48	0\\
9.49	0\\
9.5	0\\
9.51	0\\
9.52	0\\
9.53	0\\
9.54	0\\
9.55	0\\
9.56	0\\
9.57	0\\
9.58	0\\
9.59	0\\
9.6	0\\
9.61	0\\
9.62	0\\
9.63	0\\
9.64	0\\
9.65	0\\
9.66	0\\
9.67	0\\
9.68	0\\
9.69	0\\
9.7	0\\
9.71	0\\
9.72	0\\
9.73	0\\
9.74	0\\
9.75	0\\
9.76	0\\
9.77	0\\
9.78	0\\
9.79	0\\
9.8	0\\
9.81	0\\
9.82	0\\
9.83	0\\
9.84	0\\
9.85	0\\
9.86	0\\
9.87	0\\
9.88	0\\
9.89	0\\
9.9	0\\
9.91	0\\
9.92	0\\
9.93	0\\
9.94	0\\
9.95	0\\
9.96	0\\
9.97	0\\
9.98	0\\
9.99	0\\
10	0\\
10.01	0\\
10.02	0\\
10.03	0\\
10.04	0\\
10.05	0\\
10.06	0\\
10.07	0\\
10.08	0\\
10.09	0\\
10.1	0\\
10.11	0\\
10.12	0\\
10.13	0\\
10.14	0\\
10.15	0\\
10.16	0\\
10.17	0\\
10.18	0\\
10.19	0\\
10.2	0\\
10.21	0\\
10.22	0\\
10.23	0\\
10.24	0\\
10.25	0\\
10.26	0\\
10.27	0\\
10.28	0\\
10.29	0\\
10.3	0\\
10.31	0\\
10.32	0\\
10.33	0\\
10.34	0\\
10.35	0\\
10.36	0\\
10.37	0\\
10.38	0\\
10.39	0\\
10.4	0\\
10.41	0\\
10.42	0\\
10.43	0\\
10.44	0\\
10.45	0\\
10.46	0\\
10.47	0\\
10.48	0\\
10.49	0\\
10.5	0\\
10.51	0\\
10.52	0\\
10.53	0\\
10.54	0\\
10.55	0\\
10.56	0\\
10.57	0\\
10.58	0\\
10.59	0\\
10.6	0\\
10.61	0\\
10.62	0\\
10.63	0\\
10.64	0\\
10.65	0\\
10.66	0\\
10.67	0\\
10.68	0\\
10.69	0\\
10.7	0\\
10.71	0\\
10.72	0\\
10.73	0\\
10.74	0\\
10.75	0\\
10.76	0\\
10.77	0\\
10.78	0\\
10.79	0\\
10.8	0\\
10.81	0\\
10.82	0\\
10.83	0\\
10.84	0\\
10.85	0\\
10.86	0\\
10.87	0\\
10.88	0\\
10.89	0\\
10.9	0\\
10.91	0\\
10.92	0\\
10.93	0\\
10.94	0\\
10.95	0\\
10.96	0\\
10.97	0\\
10.98	0\\
10.99	0\\
11	0\\
11.01	0\\
11.02	0\\
11.03	0\\
11.04	0\\
11.05	0\\
11.06	0\\
11.07	0\\
11.08	0\\
11.09	0\\
11.1	0\\
11.11	0\\
11.12	0\\
11.13	0\\
11.14	0\\
11.15	0\\
11.16	0\\
11.17	0\\
11.18	0\\
11.19	0\\
11.2	0\\
11.21	0\\
11.22	0\\
11.23	0\\
11.24	0\\
11.25	0\\
11.26	0\\
11.27	0\\
11.28	0\\
11.29	0\\
11.3	0\\
11.31	0\\
11.32	0\\
11.33	0\\
11.34	0\\
11.35	0\\
11.36	0\\
11.37	0\\
11.38	0\\
11.39	0\\
11.4	0\\
11.41	0\\
11.42	0\\
11.43	0\\
11.44	0\\
11.45	0\\
11.46	0\\
11.47	0\\
11.48	0\\
11.49	0\\
11.5	0\\
11.51	0\\
11.52	0\\
11.53	0\\
11.54	0\\
11.55	0\\
11.56	0\\
11.57	0\\
11.58	0\\
11.59	0\\
11.6	0\\
11.61	0\\
11.62	0\\
11.63	0\\
11.64	0\\
11.65	0\\
11.66	0\\
11.67	0\\
11.68	0\\
11.69	0\\
11.7	0\\
11.71	0\\
11.72	0\\
11.73	0\\
11.74	0\\
11.75	0\\
11.76	0\\
11.77	0\\
11.78	0\\
11.79	0\\
11.8	0\\
11.81	0\\
11.82	0\\
11.83	0\\
11.84	0\\
11.85	0\\
11.86	0\\
11.87	0\\
11.88	0\\
11.89	0\\
11.9	0\\
11.91	0\\
11.92	0\\
11.93	0\\
11.94	0\\
11.95	0\\
11.96	0\\
11.97	0\\
11.98	0\\
11.99	0\\
12	0\\
12.01	0\\
12.02	0\\
12.03	0\\
12.04	0\\
12.05	0\\
12.06	0\\
12.07	0\\
12.08	0\\
12.09	0\\
12.1	0\\
12.11	0\\
12.12	0\\
12.13	0\\
12.14	0\\
12.15	0\\
12.16	0\\
12.17	0\\
12.18	0\\
12.19	0\\
12.2	0\\
12.21	0\\
12.22	0\\
12.23	0\\
12.24	0\\
12.25	0\\
12.26	0\\
12.27	0\\
12.28	0\\
12.29	0\\
12.3	0\\
12.31	0\\
12.32	0\\
12.33	0\\
12.34	0\\
12.35	0\\
12.36	0\\
12.37	0\\
12.38	0\\
12.39	0\\
12.4	0\\
12.41	0\\
12.42	0\\
12.43	0\\
12.44	0\\
12.45	0\\
12.46	0\\
12.47	0\\
12.48	0\\
12.49	0\\
12.5	0\\
12.51	0\\
12.52	0\\
12.53	0\\
12.54	0\\
12.55	0\\
12.56	0\\
12.57	0\\
12.58	0\\
12.59	0\\
12.6	0\\
12.61	0\\
12.62	0\\
12.63	0\\
12.64	0\\
12.65	0\\
12.66	0\\
12.67	0\\
12.68	0\\
12.69	0\\
12.7	0\\
12.71	0\\
12.72	0\\
12.73	0\\
12.74	0\\
12.75	0\\
12.76	0\\
12.77	0\\
12.78	0\\
12.79	0\\
12.8	0\\
12.81	0\\
12.82	0\\
12.83	0\\
12.84	0\\
12.85	0\\
12.86	0\\
12.87	0\\
12.88	0\\
12.89	0\\
12.9	0\\
12.91	0\\
12.92	0\\
12.93	0\\
12.94	0\\
12.95	0\\
12.96	0\\
12.97	0\\
12.98	0\\
12.99	0\\
13	0\\
13.01	0\\
13.02	0\\
13.03	0\\
13.04	0\\
13.05	0\\
13.06	0\\
13.07	0\\
13.08	0\\
13.09	0\\
13.1	0\\
13.11	0\\
13.12	0\\
13.13	0\\
13.14	0\\
13.15	0\\
13.16	0\\
13.17	0\\
13.18	0\\
13.19	0\\
13.2	0\\
13.21	0\\
13.22	0\\
13.23	0\\
13.24	0\\
13.25	0\\
13.26	0\\
13.27	0\\
13.28	0\\
13.29	0\\
13.3	0\\
13.31	0\\
13.32	0\\
13.33	0\\
13.34	0\\
13.35	0\\
13.36	0\\
13.37	0\\
13.38	0\\
13.39	0\\
13.4	0\\
13.41	0\\
13.42	0\\
13.43	0\\
13.44	0\\
13.45	0\\
13.46	0\\
13.47	0\\
13.48	0\\
13.49	0\\
13.5	0\\
13.51	0\\
13.52	0\\
13.53	0\\
13.54	0\\
13.55	0\\
13.56	0\\
13.57	0\\
13.58	0\\
13.59	0\\
13.6	0\\
13.61	0\\
13.62	0\\
13.63	0\\
13.64	0\\
13.65	0\\
13.66	0\\
13.67	0\\
13.68	0\\
13.69	0\\
13.7	0\\
13.71	0\\
13.72	0\\
13.73	0\\
13.74	0\\
13.75	0\\
13.76	0\\
13.77	0\\
13.78	0\\
13.79	0\\
13.8	0\\
13.81	0\\
13.82	0\\
13.83	0\\
13.84	0\\
13.85	0\\
13.86	0\\
13.87	0\\
13.88	0\\
13.89	0\\
13.9	0\\
13.91	0\\
13.92	0\\
13.93	0\\
13.94	0\\
13.95	0\\
13.96	0\\
13.97	0\\
13.98	0\\
13.99	0\\
14	0\\
14.01	0\\
14.02	0\\
14.03	0\\
14.04	0\\
14.05	0\\
14.06	0\\
14.07	0\\
14.08	0\\
14.09	0\\
14.1	0\\
14.11	0\\
14.12	0\\
14.13	0\\
14.14	0\\
14.15	0\\
14.16	0\\
14.17	0\\
14.18	0\\
14.19	0\\
14.2	0\\
14.21	0\\
14.22	0\\
14.23	0\\
14.24	0\\
14.25	0\\
14.26	0\\
14.27	0\\
14.28	0\\
14.29	0\\
14.3	0\\
14.31	0\\
14.32	0\\
14.33	0\\
14.34	0\\
14.35	0\\
14.36	0\\
14.37	0\\
14.38	0\\
14.39	0\\
14.4	0\\
14.41	0\\
14.42	0\\
14.43	0\\
14.44	0\\
14.45	0\\
14.46	0\\
14.47	0\\
14.48	0\\
14.49	0\\
14.5	0\\
14.51	0\\
14.52	0\\
14.53	0\\
14.54	0\\
14.55	0\\
14.56	0\\
14.57	0\\
14.58	0\\
14.59	0\\
14.6	0\\
14.61	0\\
14.62	0\\
14.63	0\\
14.64	0\\
14.65	0\\
14.66	0\\
14.67	0\\
14.68	0\\
14.69	0\\
14.7	0\\
14.71	0\\
14.72	0\\
14.73	0\\
14.74	0\\
14.75	0\\
14.76	0\\
14.77	0\\
14.78	0\\
14.79	0\\
14.8	0\\
14.81	0\\
14.82	0\\
14.83	0\\
14.84	0\\
14.85	0\\
14.86	0\\
14.87	0\\
14.88	0\\
14.89	0\\
14.9	0\\
14.91	0\\
14.92	0\\
14.93	0\\
14.94	0\\
14.95	0\\
14.96	0\\
14.97	0\\
14.98	0\\
14.99	0\\
15	0\\
15.01	0\\
15.02	0\\
15.03	0\\
15.04	0\\
15.05	0\\
15.06	0\\
15.07	0\\
15.08	0\\
15.09	0\\
15.1	0\\
15.11	0\\
15.12	0\\
15.13	0\\
15.14	0\\
15.15	0\\
15.16	0\\
15.17	0\\
15.18	0\\
15.19	0\\
15.2	0\\
15.21	0\\
15.22	0\\
15.23	0\\
15.24	0\\
15.25	0\\
15.26	0\\
15.27	0\\
15.28	0\\
15.29	0\\
15.3	0\\
15.31	0\\
15.32	0\\
15.33	0\\
15.34	0\\
15.35	0\\
15.36	0\\
15.37	0\\
15.38	0\\
15.39	0\\
15.4	0\\
15.41	0\\
15.42	0\\
15.43	0\\
15.44	0\\
15.45	0\\
15.46	0\\
15.47	0\\
15.48	0\\
15.49	0\\
15.5	0\\
15.51	0\\
15.52	0\\
15.53	0\\
15.54	0\\
15.55	0\\
15.56	0\\
15.57	0\\
15.58	0\\
15.59	0\\
15.6	0\\
15.61	0\\
15.62	0\\
15.63	0\\
15.64	0\\
15.65	0\\
15.66	0\\
15.67	0\\
15.68	0\\
15.69	0\\
15.7	0\\
15.71	0\\
15.72	0\\
15.73	0\\
15.74	0\\
15.75	0\\
15.76	0\\
15.77	0\\
15.78	0\\
15.79	0\\
15.8	0\\
15.81	0\\
15.82	0\\
15.83	0\\
15.84	0\\
15.85	0\\
15.86	0\\
15.87	0\\
15.88	0\\
15.89	0\\
15.9	0\\
15.91	0\\
15.92	0\\
15.93	0\\
15.94	0\\
15.95	0\\
15.96	0\\
15.97	0\\
15.98	0\\
15.99	0\\
16	0\\
16.01	0\\
16.02	0\\
16.03	0\\
16.04	0\\
16.05	0\\
16.06	0\\
16.07	0\\
16.08	0\\
16.09	0\\
16.1	0\\
16.11	0\\
16.12	0\\
16.13	0\\
16.14	0\\
16.15	0\\
16.16	0\\
16.17	0\\
16.18	0\\
16.19	0\\
16.2	0\\
16.21	0\\
16.22	0\\
16.23	0\\
16.24	0\\
16.25	0\\
16.26	0\\
16.27	0\\
16.28	0\\
16.29	0\\
16.3	0\\
16.31	0\\
16.32	0\\
16.33	0\\
16.34	0\\
16.35	0\\
16.36	0\\
16.37	0\\
16.38	0\\
16.39	0\\
16.4	0\\
16.41	0\\
16.42	0\\
16.43	0\\
16.44	0\\
16.45	0\\
16.46	0\\
16.47	0\\
16.48	0\\
16.49	0\\
16.5	0\\
16.51	0\\
16.52	0\\
16.53	0\\
16.54	0\\
16.55	0\\
16.56	0\\
16.57	0\\
16.58	0\\
16.59	0\\
16.6	0\\
16.61	0\\
16.62	0\\
16.63	0\\
16.64	0\\
16.65	0\\
16.66	0\\
16.67	0\\
16.68	0\\
16.69	0\\
16.7	0\\
16.71	0\\
16.72	0\\
16.73	0\\
16.74	0\\
16.75	0\\
16.76	0\\
16.77	0\\
16.78	0\\
16.79	0\\
16.8	0\\
16.81	0\\
16.82	0\\
16.83	0\\
16.84	0\\
16.85	0\\
16.86	0\\
16.87	0\\
16.88	0\\
16.89	0\\
16.9	0\\
16.91	0\\
16.92	0\\
16.93	0\\
16.94	0\\
16.95	0\\
16.96	0\\
16.97	0\\
16.98	0\\
16.99	0\\
17	0\\
17.01	0\\
17.02	0\\
17.03	0\\
17.04	0\\
17.05	0\\
17.06	0\\
17.07	0\\
17.08	0\\
17.09	0\\
17.1	0\\
17.11	0\\
17.12	0\\
17.13	0\\
17.14	0\\
17.15	0\\
17.16	0\\
17.17	0\\
17.18	0\\
17.19	0\\
17.2	0\\
17.21	0\\
17.22	0\\
17.23	0\\
17.24	0\\
17.25	0\\
17.26	0\\
17.27	0\\
17.28	0\\
17.29	0\\
17.3	0\\
17.31	0\\
17.32	0\\
17.33	0\\
17.34	0\\
17.35	0\\
17.36	0\\
17.37	0\\
17.38	0\\
17.39	0\\
17.4	0\\
17.41	0\\
17.42	0\\
17.43	0\\
17.44	0\\
17.45	0\\
17.46	0\\
17.47	0\\
17.48	0\\
17.49	0\\
17.5	0\\
17.51	0\\
17.52	0\\
17.53	0\\
17.54	0\\
17.55	0\\
17.56	0\\
17.57	0\\
17.58	0\\
17.59	0\\
17.6	0\\
17.61	0\\
17.62	0\\
17.63	0\\
17.64	0\\
17.65	0\\
17.66	0\\
17.67	0\\
17.68	0\\
17.69	0\\
17.7	0\\
17.71	0\\
17.72	0\\
17.73	0\\
17.74	0\\
17.75	0\\
17.76	0\\
17.77	0\\
17.78	0\\
17.79	0\\
17.8	0\\
17.81	0\\
17.82	0\\
17.83	0\\
17.84	0\\
17.85	0\\
17.86	0\\
17.87	0\\
17.88	0\\
17.89	0\\
17.9	0\\
17.91	0\\
17.92	0\\
17.93	0\\
17.94	0\\
17.95	0\\
17.96	0\\
17.97	0\\
17.98	0\\
17.99	0\\
18	0\\
18.01	0\\
18.02	0\\
18.03	0\\
18.04	0\\
18.05	0\\
18.06	0\\
18.07	0\\
18.08	0\\
18.09	0\\
18.1	0\\
18.11	0\\
18.12	0\\
18.13	0\\
18.14	0\\
18.15	0\\
18.16	0\\
18.17	0\\
18.18	0\\
18.19	0\\
18.2	0\\
18.21	0\\
18.22	0\\
18.23	0\\
18.24	0\\
18.25	0\\
18.26	0\\
18.27	0\\
18.28	0\\
18.29	0\\
18.3	0\\
18.31	0\\
18.32	0\\
18.33	0\\
18.34	0\\
18.35	0\\
18.36	0\\
18.37	0\\
18.38	0\\
18.39	0\\
18.4	0\\
18.41	0\\
18.42	0\\
18.43	0\\
18.44	0\\
18.45	0\\
18.46	0\\
18.47	0\\
18.48	0\\
18.49	0\\
18.5	0\\
18.51	0\\
18.52	0\\
18.53	0\\
18.54	0\\
18.55	0\\
18.56	0\\
18.57	0\\
18.58	0\\
18.59	0\\
18.6	0\\
18.61	0\\
18.62	0\\
18.63	0\\
18.64	0\\
18.65	0\\
18.66	0\\
18.67	0\\
18.68	0\\
18.69	0\\
18.7	0\\
18.71	0\\
18.72	0\\
18.73	0\\
18.74	0\\
18.75	0\\
18.76	0\\
18.77	0\\
18.78	0\\
18.79	0\\
18.8	0\\
18.81	0\\
18.82	0\\
18.83	0\\
18.84	0\\
18.85	0\\
18.86	0\\
18.87	0\\
18.88	0\\
18.89	0\\
18.9	0\\
18.91	0\\
18.92	0\\
18.93	0\\
18.94	0\\
18.95	0\\
18.96	0\\
18.97	0\\
18.98	0\\
18.99	0\\
19	0\\
19.01	0\\
19.02	0\\
19.03	0\\
19.04	0\\
19.05	0\\
19.06	0\\
19.07	0\\
19.08	0\\
19.09	0\\
19.1	0\\
19.11	0\\
19.12	0\\
19.13	0\\
19.14	0\\
19.15	0\\
19.16	0\\
19.17	0\\
19.18	0\\
19.19	0\\
19.2	0\\
19.21	0\\
19.22	0\\
19.23	0\\
19.24	0\\
19.25	0\\
19.26	0\\
19.27	0\\
19.28	0\\
19.29	0\\
19.3	0\\
19.31	0\\
19.32	0\\
19.33	0\\
19.34	0\\
19.35	0\\
19.36	0\\
19.37	0\\
19.38	0\\
19.39	0\\
19.4	0\\
19.41	0\\
19.42	0\\
19.43	0\\
19.44	0\\
19.45	0\\
19.46	0\\
19.47	0\\
19.48	0\\
19.49	0\\
19.5	0\\
19.51	0\\
19.52	0\\
19.53	0\\
19.54	0\\
19.55	0\\
19.56	0\\
19.57	0\\
19.58	0\\
19.59	0\\
19.6	0\\
19.61	0\\
19.62	0\\
19.63	0\\
19.64	0\\
19.65	0\\
19.66	0\\
19.67	0\\
19.68	0\\
19.69	0\\
19.7	0\\
19.71	0\\
19.72	0\\
19.73	0\\
19.74	0\\
19.75	0\\
19.76	0\\
19.77	0\\
19.78	0\\
19.79	0\\
19.8	0\\
19.81	0\\
19.82	0\\
19.83	0\\
19.84	0\\
19.85	0\\
19.86	0\\
19.87	0\\
19.88	0\\
19.89	0\\
19.9	0\\
19.91	0\\
19.92	0\\
19.93	0\\
19.94	0\\
19.95	0\\
19.96	0\\
19.97	0\\
19.98	0\\
19.99	0\\
20	0\\
20.01	0\\
20.02	0\\
20.03	0\\
20.04	0\\
20.05	0\\
20.06	0\\
20.07	0\\
20.08	0\\
20.09	0\\
20.1	0\\
20.11	0\\
20.12	0\\
20.13	0\\
20.14	0\\
20.15	0\\
20.16	0\\
20.17	0\\
20.18	0\\
20.19	0\\
20.2	0\\
20.21	0\\
20.22	0\\
20.23	0\\
20.24	0\\
20.25	0\\
20.26	0\\
20.27	0\\
20.28	0\\
20.29	0\\
20.3	0\\
20.31	0\\
20.32	0\\
20.33	0\\
20.34	0\\
20.35	0\\
20.36	0\\
20.37	0\\
20.38	0\\
20.39	0\\
20.4	0\\
20.41	0\\
20.42	0\\
20.43	0\\
20.44	0\\
20.45	0\\
20.46	0\\
20.47	0\\
20.48	0\\
20.49	0\\
20.5	0\\
20.51	0\\
20.52	0\\
20.53	0\\
20.54	0\\
20.55	0\\
20.56	0\\
20.57	0\\
20.58	0\\
20.59	0\\
20.6	0\\
20.61	0\\
20.62	0\\
20.63	0\\
20.64	0\\
20.65	0\\
20.66	0\\
20.67	0\\
20.68	0\\
20.69	0\\
20.7	0\\
20.71	0\\
20.72	0\\
20.73	0\\
20.74	0\\
20.75	0\\
20.76	0\\
20.77	0\\
20.78	0\\
20.79	0\\
20.8	0\\
20.81	0\\
20.82	0\\
20.83	0\\
20.84	0\\
20.85	0\\
20.86	0\\
20.87	0\\
20.88	0\\
20.89	0\\
20.9	0\\
20.91	0\\
20.92	0\\
20.93	0\\
20.94	0\\
20.95	0\\
20.96	0\\
20.97	0\\
20.98	0\\
20.99	0\\
21	0\\
21.01	0\\
21.02	0\\
21.03	0\\
21.04	0\\
21.05	0\\
21.06	0\\
21.07	0\\
21.08	0\\
21.09	0\\
21.1	0\\
21.11	0\\
21.12	0\\
21.13	0\\
21.14	0\\
21.15	0\\
21.16	0\\
21.17	0\\
21.18	0\\
21.19	0\\
21.2	0\\
21.21	0\\
21.22	0\\
21.23	0\\
21.24	0\\
21.25	0\\
21.26	0\\
21.27	0\\
21.28	0\\
21.29	0\\
21.3	0\\
21.31	0\\
21.32	0\\
21.33	0\\
21.34	0\\
21.35	0\\
21.36	0\\
21.37	0\\
21.38	0\\
21.39	0\\
21.4	0\\
21.41	0\\
21.42	0\\
21.43	0\\
21.44	0\\
21.45	0\\
21.46	0\\
21.47	0\\
21.48	0\\
21.49	0\\
21.5	0\\
21.51	0\\
21.52	0\\
21.53	0\\
21.54	0\\
21.55	0\\
21.56	0\\
21.57	0\\
21.58	0\\
21.59	0\\
21.6	0\\
21.61	0\\
21.62	0\\
21.63	0\\
21.64	0\\
21.65	0\\
21.66	0\\
21.67	0\\
21.68	0\\
21.69	0\\
21.7	0\\
21.71	0\\
21.72	0\\
21.73	0\\
21.74	0\\
21.75	0\\
21.76	0\\
21.77	0\\
21.78	0\\
21.79	0\\
21.8	0\\
21.81	0\\
21.82	0\\
21.83	0\\
21.84	0\\
21.85	0\\
21.86	0\\
21.87	0\\
21.88	0\\
21.89	0\\
21.9	0\\
21.91	0\\
21.92	0\\
21.93	0\\
21.94	0\\
21.95	0\\
21.96	0\\
21.97	0\\
21.98	0\\
21.99	0\\
22	0\\
22.01	0\\
22.02	0\\
22.03	0\\
22.04	0\\
22.05	0\\
22.06	0\\
22.07	0\\
22.08	0\\
22.09	0\\
22.1	0\\
22.11	0\\
22.12	0\\
22.13	0\\
22.14	0\\
22.15	0\\
22.16	0\\
22.17	0\\
22.18	0\\
22.19	0\\
22.2	0\\
22.21	0\\
22.22	0\\
22.23	0\\
22.24	0\\
22.25	0\\
22.26	0\\
22.27	0\\
22.28	0\\
22.29	0\\
22.3	0\\
22.31	0\\
22.32	0\\
22.33	0\\
22.34	0\\
22.35	0\\
22.36	0\\
22.37	0\\
22.38	0\\
22.39	0\\
22.4	0\\
22.41	0\\
22.42	0\\
22.43	0\\
22.44	0\\
22.45	0\\
22.46	0\\
22.47	0\\
22.48	0\\
22.49	0\\
22.5	0\\
22.51	0\\
22.52	0\\
22.53	0\\
22.54	0\\
22.55	0\\
22.56	0\\
22.57	0\\
22.58	0\\
22.59	0\\
22.6	0\\
22.61	0\\
22.62	0\\
22.63	0\\
22.64	0\\
22.65	0\\
22.66	0\\
22.67	0\\
22.68	0\\
22.69	0\\
22.7	0\\
22.71	0\\
22.72	0\\
22.73	0\\
22.74	0\\
22.75	0\\
22.76	0\\
22.77	0\\
22.78	0\\
22.79	0\\
22.8	0\\
22.81	0\\
22.82	0\\
22.83	0\\
22.84	0\\
22.85	0\\
22.86	0\\
22.87	0\\
22.88	0\\
22.89	0\\
22.9	0\\
22.91	0\\
22.92	0\\
22.93	0\\
22.94	0\\
22.95	0\\
22.96	0\\
22.97	0\\
22.98	0\\
22.99	0\\
23	0\\
23.01	0\\
23.02	0\\
23.03	0\\
23.04	0\\
23.05	0\\
23.06	0\\
23.07	0\\
23.08	0\\
23.09	0\\
23.1	0\\
23.11	0\\
23.12	0\\
23.13	0\\
23.14	0\\
23.15	0\\
23.16	0\\
23.17	0\\
23.18	0\\
23.19	0\\
23.2	0\\
23.21	0\\
23.22	0\\
23.23	0\\
23.24	0\\
23.25	0\\
23.26	0\\
23.27	0\\
23.28	0\\
23.29	0\\
23.3	0\\
23.31	0\\
23.32	0\\
23.33	0\\
23.34	0\\
23.35	0\\
23.36	0\\
23.37	0\\
23.38	0\\
23.39	0\\
23.4	0\\
23.41	0\\
23.42	0\\
23.43	0\\
23.44	0\\
23.45	0\\
23.46	0\\
23.47	0\\
23.48	0\\
23.49	0\\
23.5	0\\
23.51	0\\
23.52	0\\
23.53	0\\
23.54	0\\
23.55	0\\
23.56	0\\
23.57	0\\
23.58	0\\
23.59	0\\
23.6	0\\
23.61	0\\
23.62	0\\
23.63	0\\
23.64	0\\
23.65	0\\
23.66	0\\
23.67	0\\
23.68	0\\
23.69	0\\
23.7	0\\
23.71	0\\
23.72	0\\
23.73	0\\
23.74	0\\
23.75	0\\
23.76	0\\
23.77	0\\
23.78	0\\
23.79	0\\
23.8	0\\
23.81	0\\
23.82	0\\
23.83	0\\
23.84	0\\
23.85	0\\
23.86	0\\
23.87	0\\
23.88	0\\
23.89	0\\
23.9	0\\
23.91	0\\
23.92	0\\
23.93	0\\
23.94	0\\
23.95	0\\
23.96	0\\
23.97	0\\
23.98	0\\
23.99	0\\
24	0\\
24.01	0\\
24.02	0\\
24.03	0\\
24.04	0\\
24.05	0\\
24.06	0\\
24.07	0\\
24.08	0\\
24.09	0\\
24.1	0\\
24.11	0\\
24.12	0\\
24.13	0\\
24.14	0\\
24.15	0\\
24.16	0\\
24.17	0\\
24.18	0\\
24.19	0\\
24.2	0\\
24.21	0\\
24.22	0\\
24.23	0\\
24.24	0\\
24.25	0\\
24.26	0\\
24.27	0\\
24.28	0\\
24.29	0\\
24.3	0\\
24.31	0\\
24.32	0\\
24.33	0\\
24.34	0\\
24.35	0\\
24.36	0\\
24.37	0\\
24.38	0\\
24.39	0\\
24.4	0\\
24.41	0\\
24.42	0\\
24.43	0\\
24.44	0\\
24.45	0\\
24.46	0\\
24.47	0\\
24.48	0\\
24.49	0\\
24.5	0\\
24.51	0\\
24.52	0\\
24.53	0\\
24.54	0\\
24.55	0\\
24.56	0\\
24.57	0\\
24.58	0\\
24.59	0\\
24.6	0\\
24.61	0\\
24.62	0\\
24.63	0\\
24.64	0\\
24.65	0\\
24.66	0\\
24.67	0\\
24.68	0\\
24.69	0\\
24.7	0\\
24.71	0\\
24.72	0\\
24.73	0\\
24.74	0\\
24.75	0\\
24.76	0\\
24.77	0\\
24.78	0\\
24.79	0\\
24.8	0\\
24.81	0\\
24.82	0\\
24.83	0\\
24.84	0\\
24.85	0\\
24.86	0\\
24.87	0\\
24.88	0\\
24.89	0\\
24.9	0\\
24.91	0\\
24.92	0\\
24.93	0\\
24.94	0\\
24.95	0\\
24.96	0\\
24.97	0\\
24.98	0\\
24.99	0\\
25	0\\
25.01	0\\
25.02	0\\
25.03	0\\
25.04	0\\
25.05	0\\
25.06	0\\
25.07	0\\
25.08	0\\
25.09	0\\
25.1	0\\
25.11	0\\
25.12	0\\
25.13	0\\
25.14	0\\
25.15	0\\
25.16	0\\
25.17	0\\
25.18	0\\
25.19	0\\
25.2	0\\
25.21	0\\
25.22	0\\
25.23	0\\
25.24	0\\
25.25	0\\
25.26	0\\
25.27	0\\
25.28	0\\
25.29	0\\
25.3	0\\
25.31	0\\
25.32	0\\
25.33	0\\
25.34	0\\
25.35	0\\
25.36	0\\
25.37	0\\
25.38	0\\
25.39	0\\
25.4	0\\
25.41	0\\
25.42	0\\
25.43	0\\
25.44	0\\
25.45	0\\
25.46	0\\
25.47	0\\
25.48	0\\
25.49	0\\
25.5	0\\
25.51	0\\
25.52	0\\
25.53	0\\
25.54	0\\
25.55	0\\
25.56	0\\
25.57	0\\
25.58	0\\
25.59	0\\
25.6	0\\
25.61	0\\
25.62	0\\
25.63	0\\
25.64	0\\
25.65	0\\
25.66	0\\
25.67	0\\
25.68	0\\
25.69	0\\
25.7	0\\
25.71	0\\
25.72	0\\
25.73	0\\
25.74	0\\
25.75	0\\
25.76	0\\
25.77	0\\
25.78	0\\
25.79	0\\
25.8	0\\
25.81	0\\
25.82	0\\
25.83	0\\
25.84	0\\
25.85	0\\
25.86	0\\
25.87	0\\
25.88	0\\
25.89	0\\
25.9	0\\
25.91	0\\
25.92	0\\
25.93	0\\
25.94	0\\
25.95	0\\
25.96	0\\
25.97	0\\
25.98	0\\
25.99	0\\
26	0\\
26.01	0\\
26.02	0\\
26.03	0\\
26.04	0\\
26.05	0\\
26.06	0\\
26.07	0\\
26.08	0\\
26.09	0\\
26.1	0\\
26.11	0\\
26.12	0\\
26.13	0\\
26.14	0\\
26.15	0\\
26.16	0\\
26.17	0\\
26.18	0\\
26.19	0\\
26.2	0\\
26.21	0\\
26.22	0\\
26.23	0\\
26.24	0\\
26.25	0\\
26.26	0\\
26.27	0\\
26.28	0\\
26.29	0\\
26.3	0\\
26.31	0\\
26.32	0\\
26.33	0\\
26.34	0\\
26.35	0\\
26.36	0\\
26.37	0\\
26.38	0\\
26.39	0\\
26.4	0\\
26.41	0\\
26.42	0\\
26.43	0\\
26.44	0\\
26.45	0\\
26.46	0\\
26.47	0\\
26.48	0\\
26.49	0\\
26.5	0\\
26.51	0\\
26.52	0\\
26.53	0\\
26.54	0\\
26.55	0\\
26.56	0\\
26.57	0\\
26.58	0\\
26.59	0\\
26.6	0\\
26.61	0\\
26.62	0\\
26.63	0\\
26.64	0\\
26.65	0\\
26.66	0\\
26.67	0\\
26.68	0\\
26.69	0\\
26.7	0\\
26.71	0\\
26.72	0\\
26.73	0\\
26.74	0\\
26.75	0\\
26.76	0\\
26.77	0\\
26.78	0\\
26.79	0\\
26.8	0\\
26.81	0\\
26.82	0\\
26.83	0\\
26.84	0\\
26.85	0\\
26.86	0\\
26.87	0\\
26.88	0\\
26.89	0\\
26.9	0\\
26.91	0\\
26.92	0\\
26.93	0\\
26.94	0\\
26.95	0\\
26.96	0\\
26.97	0\\
26.98	0\\
26.99	0\\
27	0\\
27.01	0\\
27.02	0\\
27.03	0\\
27.04	0\\
27.05	0\\
27.06	0\\
27.07	0\\
27.08	0\\
27.09	0\\
27.1	0\\
27.11	0\\
27.12	0\\
27.13	0\\
27.14	0\\
27.15	0\\
27.16	0\\
27.17	0\\
27.18	0\\
27.19	0\\
27.2	0\\
27.21	0\\
27.22	0\\
27.23	0\\
27.24	0\\
27.25	0\\
27.26	0\\
27.27	0\\
27.28	0\\
27.29	0\\
27.3	0\\
27.31	0\\
27.32	0\\
27.33	0\\
27.34	0\\
27.35	0\\
27.36	0\\
27.37	0\\
27.38	0\\
27.39	0\\
27.4	0\\
27.41	0\\
27.42	0\\
27.43	0\\
27.44	0\\
27.45	0\\
27.46	0\\
27.47	0\\
27.48	0\\
27.49	0\\
27.5	0\\
27.51	0\\
27.52	0\\
27.53	0\\
27.54	0\\
27.55	0\\
27.56	0\\
27.57	0\\
27.58	0\\
27.59	0\\
27.6	0\\
27.61	0\\
27.62	0\\
27.63	0\\
27.64	0\\
27.65	0\\
27.66	0\\
27.67	0\\
27.68	0\\
27.69	0\\
27.7	0\\
27.71	0\\
27.72	0\\
27.73	0\\
27.74	0\\
27.75	0\\
27.76	0\\
27.77	0\\
27.78	0\\
27.79	0\\
27.8	0\\
27.81	0\\
27.82	0\\
27.83	0\\
27.84	0\\
27.85	0\\
27.86	0\\
27.87	0\\
27.88	0\\
27.89	0\\
27.9	0\\
27.91	0\\
27.92	0\\
27.93	0\\
27.94	0\\
27.95	0\\
27.96	0\\
27.97	0\\
27.98	0\\
27.99	0\\
28	0\\
28.01	0\\
28.02	0\\
28.03	0\\
28.04	0\\
28.05	0\\
28.06	0\\
28.07	0\\
28.08	0\\
28.09	0\\
28.1	0\\
28.11	0\\
28.12	0\\
28.13	0\\
28.14	0\\
28.15	0\\
28.16	0\\
28.17	0\\
28.18	0\\
28.19	0\\
28.2	0\\
28.21	0\\
28.22	0\\
28.23	0\\
28.24	0\\
28.25	0\\
28.26	0\\
28.27	0\\
28.28	0\\
28.29	0\\
28.3	0\\
28.31	0\\
28.32	0\\
28.33	0\\
28.34	0\\
28.35	0\\
28.36	0\\
28.37	0\\
28.38	0\\
28.39	0\\
28.4	0\\
28.41	0\\
28.42	0\\
28.43	0\\
28.44	0\\
28.45	0\\
28.46	0\\
28.47	0\\
28.48	0\\
28.49	0\\
28.5	0\\
28.51	0\\
28.52	0\\
28.53	0\\
28.54	0\\
28.55	0\\
28.56	0\\
28.57	0\\
28.58	0\\
28.59	0\\
28.6	0\\
28.61	0\\
28.62	0\\
28.63	0\\
28.64	0\\
28.65	0\\
28.66	0\\
28.67	0\\
28.68	0\\
28.69	0\\
28.7	0\\
28.71	0\\
28.72	0\\
28.73	0\\
28.74	0\\
28.75	0\\
28.76	0\\
28.77	0\\
28.78	0\\
28.79	0\\
28.8	0\\
28.81	0\\
28.82	0\\
28.83	0\\
28.84	0\\
28.85	0\\
28.86	0\\
28.87	0\\
28.88	0\\
28.89	0\\
28.9	0\\
28.91	0\\
28.92	0\\
28.93	0\\
28.94	0\\
28.95	0\\
28.96	0\\
28.97	0\\
28.98	0\\
28.99	0\\
29	0\\
29.01	0\\
29.02	0\\
29.03	0\\
29.04	0\\
29.05	0\\
29.06	0\\
29.07	0\\
29.08	0\\
29.09	0\\
29.1	0\\
29.11	0\\
29.12	0\\
29.13	0\\
29.14	0\\
29.15	0\\
29.16	0\\
29.17	0\\
29.18	0\\
29.19	0\\
29.2	0\\
29.21	0\\
29.22	0\\
29.23	0\\
29.24	0\\
29.25	0\\
29.26	0\\
29.27	0\\
29.28	0\\
29.29	0\\
29.3	0\\
29.31	0\\
29.32	0\\
29.33	0\\
29.34	0\\
29.35	0\\
29.36	0\\
29.37	0\\
29.38	0\\
29.39	0\\
29.4	0\\
29.41	0\\
29.42	0\\
29.43	0\\
29.44	0\\
29.45	0\\
29.46	0\\
29.47	0\\
29.48	0\\
29.49	0\\
29.5	0\\
29.51	0\\
29.52	0\\
29.53	0\\
29.54	0\\
29.55	0\\
29.56	0\\
29.57	0\\
29.58	0\\
29.59	0\\
29.6	0\\
29.61	0\\
29.62	0\\
29.63	0\\
29.64	0\\
29.65	0\\
29.66	0\\
29.67	0\\
29.68	0\\
29.69	0\\
29.7	0\\
29.71	0\\
29.72	0\\
29.73	0\\
29.74	0\\
29.75	0\\
29.76	0\\
29.77	0\\
29.78	0\\
29.79	0\\
29.8	0\\
29.81	0\\
29.82	0\\
29.83	0\\
29.84	0\\
29.85	0\\
29.86	0\\
29.87	0\\
29.88	0\\
29.89	0\\
29.9	0\\
29.91	0\\
29.92	0\\
29.93	0\\
29.94	0\\
29.95	0\\
29.96	0\\
29.97	0\\
29.98	0\\
29.99	0\\
30	0\\
30.01	0\\
30.02	0\\
30.03	0\\
30.04	0\\
30.05	0\\
30.06	0\\
30.07	0\\
30.08	0\\
30.09	0\\
30.1	0\\
30.11	0\\
30.12	0\\
30.13	0\\
30.14	0\\
30.15	0\\
30.16	0\\
30.17	0\\
30.18	0\\
30.19	0\\
30.2	0\\
30.21	0\\
30.22	0\\
30.23	0\\
30.24	0\\
30.25	0\\
30.26	0\\
30.27	0\\
30.28	0\\
30.29	0\\
30.3	0\\
30.31	0\\
30.32	0\\
30.33	0\\
30.34	0\\
30.35	0\\
30.36	0\\
30.37	0\\
30.38	0\\
30.39	0\\
30.4	0\\
30.41	0\\
30.42	0\\
30.43	0\\
30.44	0\\
30.45	0\\
30.46	0\\
30.47	0\\
30.48	0\\
30.49	0\\
30.5	0\\
30.51	0\\
30.52	0\\
30.53	0\\
30.54	0\\
30.55	0\\
30.56	0\\
30.57	0\\
30.58	0\\
30.59	0\\
30.6	0\\
30.61	0\\
30.62	0\\
30.63	0\\
30.64	0\\
30.65	0\\
30.66	0\\
30.67	0\\
30.68	0\\
30.69	0\\
30.7	0\\
30.71	0\\
30.72	0\\
30.73	0\\
30.74	0\\
30.75	0\\
30.76	0\\
30.77	0\\
30.78	0\\
30.79	0\\
30.8	0\\
30.81	0\\
30.82	0\\
30.83	0\\
30.84	0\\
30.85	0\\
30.86	0\\
30.87	0\\
30.88	0\\
30.89	0\\
30.9	0\\
30.91	0\\
30.92	0\\
30.93	0\\
30.94	0\\
30.95	0\\
30.96	0\\
30.97	0\\
30.98	0\\
30.99	0\\
31	0\\
31.01	0\\
31.02	0\\
31.03	0\\
31.04	0\\
31.05	0\\
31.06	0\\
31.07	0\\
31.08	0\\
31.09	0\\
31.1	0\\
31.11	0\\
31.12	0\\
31.13	0\\
31.14	0\\
31.15	0\\
31.16	0\\
31.17	0\\
31.18	0\\
31.19	0\\
31.2	0\\
31.21	0\\
31.22	0\\
31.23	0\\
31.24	0\\
31.25	0\\
31.26	0\\
31.27	0\\
31.28	0\\
31.29	0\\
31.3	0\\
31.31	0\\
31.32	0\\
31.33	0\\
31.34	0\\
31.35	0\\
31.36	0\\
31.37	0\\
31.38	0\\
31.39	0\\
31.4	0\\
31.41	0\\
31.42	0\\
31.43	0\\
31.44	0\\
31.45	0\\
31.46	0\\
31.47	0\\
31.48	0\\
31.49	0\\
31.5	0\\
31.51	0\\
31.52	0\\
31.53	0\\
31.54	0\\
31.55	0\\
31.56	0\\
31.57	0\\
31.58	0\\
31.59	0\\
31.6	0\\
31.61	0\\
31.62	0\\
31.63	0\\
31.64	0\\
31.65	0\\
31.66	0\\
31.67	0\\
31.68	0\\
31.69	0\\
31.7	0\\
31.71	0\\
31.72	0\\
31.73	0\\
31.74	0\\
31.75	0\\
31.76	0\\
31.77	0\\
31.78	0\\
31.79	0\\
31.8	0\\
31.81	0\\
31.82	0\\
31.83	0\\
31.84	0\\
31.85	0\\
31.86	0\\
31.87	0\\
31.88	0\\
31.89	0\\
31.9	0\\
31.91	0\\
31.92	0\\
31.93	0\\
31.94	0\\
31.95	0\\
31.96	0\\
31.97	0\\
31.98	0\\
31.99	0\\
32	0\\
32.01	0\\
32.02	0\\
32.03	0\\
32.04	0\\
32.05	0\\
32.06	0\\
32.07	0\\
32.08	0\\
32.09	0\\
32.1	0\\
32.11	0\\
32.12	0\\
32.13	0\\
32.14	0\\
32.15	0\\
32.16	0\\
32.17	0\\
32.18	0\\
32.19	0\\
32.2	0\\
32.21	0\\
32.22	0\\
32.23	0\\
32.24	0\\
32.25	0\\
32.26	0\\
32.27	0\\
32.28	0\\
32.29	0\\
32.3	0\\
32.31	0\\
32.32	0\\
32.33	0\\
32.34	0\\
32.35	0\\
32.36	0\\
32.37	0\\
32.38	0\\
32.39	0\\
32.4	0\\
32.41	0\\
32.42	0\\
32.43	0\\
32.44	0\\
32.45	0\\
32.46	0\\
32.47	0\\
32.48	0\\
32.49	0\\
32.5	0\\
32.51	0\\
32.52	0\\
32.53	0\\
32.54	0\\
32.55	0\\
32.56	0\\
32.57	0\\
32.58	0\\
32.59	0\\
32.6	0\\
32.61	0\\
32.62	0\\
32.63	0\\
32.64	0\\
32.65	0\\
32.66	0\\
32.67	0\\
32.68	0\\
32.69	0\\
32.7	0\\
32.71	0\\
32.72	0\\
32.73	0\\
32.74	0\\
32.75	0\\
32.76	0\\
32.77	0\\
32.78	0\\
32.79	0\\
32.8	0\\
32.81	0\\
32.82	0\\
32.83	0\\
32.84	0\\
32.85	0\\
32.86	0\\
32.87	0\\
32.88	0\\
32.89	0\\
32.9	0\\
32.91	0\\
32.92	0\\
32.93	0\\
32.94	0\\
32.95	0\\
32.96	0\\
32.97	0\\
32.98	0\\
32.99	0\\
33	0\\
33.01	0\\
33.02	0\\
33.03	0\\
33.04	0\\
33.05	0\\
33.06	0\\
33.07	0\\
33.08	0\\
33.09	0\\
33.1	0\\
33.11	0\\
33.12	0\\
33.13	0\\
33.14	0\\
33.15	0\\
33.16	0\\
33.17	0\\
33.18	0\\
33.19	0\\
33.2	0\\
33.21	0\\
33.22	0\\
33.23	0\\
33.24	0\\
33.25	0\\
33.26	0\\
33.27	0\\
33.28	0\\
33.29	0\\
33.3	0\\
33.31	0\\
33.32	0\\
33.33	0\\
33.34	0\\
33.35	0\\
33.36	0\\
33.37	0\\
33.38	0\\
33.39	0\\
33.4	0\\
33.41	0\\
33.42	0\\
33.43	0\\
33.44	0\\
33.45	0\\
33.46	0\\
33.47	0\\
33.48	0\\
33.49	0\\
33.5	0\\
33.51	0\\
33.52	0\\
33.53	0\\
33.54	0\\
33.55	0\\
33.56	0\\
33.57	0\\
33.58	0\\
33.59	0\\
33.6	0\\
33.61	0\\
33.62	0\\
33.63	0\\
33.64	0\\
33.65	0\\
33.66	0\\
33.67	0\\
33.68	0\\
33.69	0\\
33.7	0\\
33.71	0\\
33.72	0\\
33.73	0\\
33.74	0\\
33.75	0\\
33.76	0\\
33.77	0\\
33.78	0\\
33.79	0\\
33.8	0\\
33.81	0\\
33.82	0\\
33.83	0\\
33.84	0\\
33.85	0\\
33.86	0\\
33.87	0\\
33.88	0\\
33.89	0\\
33.9	0\\
33.91	0\\
33.92	0\\
33.93	0\\
33.94	0\\
33.95	0\\
33.96	0\\
33.97	0\\
33.98	0\\
33.99	0\\
34	0\\
34.01	0\\
34.02	0\\
34.03	0\\
34.04	0\\
34.05	0\\
34.06	0\\
34.07	0\\
34.08	0\\
34.09	0\\
34.1	0\\
34.11	0\\
34.12	0\\
34.13	0\\
34.14	0\\
34.15	0\\
34.16	0\\
34.17	0\\
34.18	0\\
34.19	0\\
34.2	0\\
34.21	0\\
34.22	0\\
34.23	0\\
34.24	0\\
34.25	0\\
34.26	0\\
34.27	0\\
34.28	0\\
34.29	0\\
34.3	0\\
34.31	0\\
34.32	0\\
34.33	0\\
34.34	0\\
34.35	0\\
34.36	0\\
34.37	0\\
34.38	0\\
34.39	0\\
34.4	0\\
34.41	0\\
34.42	0\\
34.43	0\\
34.44	0\\
34.45	0\\
34.46	0\\
34.47	0\\
34.48	0\\
34.49	0\\
34.5	0\\
34.51	0\\
34.52	0\\
34.53	0\\
34.54	0\\
34.55	0\\
34.56	0\\
34.57	0\\
34.58	0\\
34.59	0\\
34.6	0\\
34.61	0\\
34.62	0\\
34.63	0\\
34.64	0\\
34.65	0\\
34.66	0\\
34.67	0\\
34.68	0\\
34.69	0\\
34.7	0\\
34.71	0\\
34.72	0\\
34.73	0\\
34.74	0\\
34.75	0\\
34.76	0\\
34.77	0\\
34.78	0\\
34.79	0\\
34.8	0\\
34.81	0\\
34.82	0\\
34.83	0\\
34.84	0\\
34.85	0\\
34.86	0\\
34.87	0\\
34.88	0\\
34.89	0\\
34.9	0\\
34.91	0\\
34.92	0\\
34.93	0\\
34.94	0\\
34.95	0\\
34.96	0\\
34.97	0\\
34.98	0\\
34.99	0\\
35	0\\
35.01	0\\
35.02	0\\
35.03	0\\
35.04	0\\
35.05	0\\
35.06	0\\
35.07	0\\
35.08	0\\
35.09	0\\
35.1	0\\
35.11	0\\
35.12	0\\
35.13	0\\
35.14	0\\
35.15	0\\
35.16	0\\
35.17	0\\
35.18	0\\
35.19	0\\
35.2	0\\
35.21	0\\
35.22	0\\
35.23	0\\
35.24	0\\
35.25	0\\
35.26	0\\
35.27	0\\
35.28	0\\
35.29	0\\
35.3	0\\
35.31	0\\
35.32	0\\
35.33	0\\
35.34	0\\
35.35	0\\
35.36	0\\
35.37	0\\
35.38	0\\
35.39	0\\
35.4	0\\
35.41	0\\
35.42	0\\
35.43	0\\
35.44	0\\
35.45	0\\
35.46	0\\
35.47	0\\
35.48	0\\
35.49	0\\
35.5	0\\
35.51	0\\
35.52	0\\
35.53	0\\
35.54	0\\
35.55	0\\
35.56	0\\
35.57	0\\
35.58	0\\
35.59	0\\
35.6	0\\
35.61	0\\
35.62	0\\
35.63	0\\
35.64	0\\
35.65	0\\
35.66	0\\
35.67	0\\
35.68	0\\
35.69	0\\
35.7	0\\
35.71	0\\
35.72	0\\
35.73	0\\
35.74	0\\
35.75	0\\
35.76	0\\
35.77	0\\
35.78	0\\
35.79	0\\
35.8	0\\
35.81	0\\
35.82	0\\
35.83	0\\
35.84	0\\
35.85	0\\
35.86	0\\
35.87	0\\
35.88	0\\
35.89	0\\
35.9	0\\
35.91	0\\
35.92	0\\
35.93	0\\
35.94	0\\
35.95	0\\
35.96	0\\
35.97	0\\
35.98	0\\
35.99	0\\
36	0\\
36.01	0\\
36.02	0\\
36.03	0\\
36.04	0\\
36.05	0\\
36.06	0\\
36.07	0\\
36.08	0\\
36.09	0\\
36.1	0\\
36.11	0\\
36.12	0\\
36.13	0\\
36.14	0\\
36.15	0\\
36.16	0\\
36.17	0\\
36.18	0\\
36.19	0\\
36.2	0\\
36.21	0\\
36.22	0\\
36.23	0\\
36.24	0\\
36.25	0\\
36.26	0\\
36.27	0\\
36.28	0\\
36.29	0\\
36.3	0\\
36.31	0\\
36.32	0\\
36.33	0\\
36.34	0\\
36.35	0\\
36.36	0\\
36.37	0\\
36.38	0\\
36.39	0\\
36.4	0\\
36.41	0\\
36.42	0\\
36.43	0\\
36.44	0\\
36.45	0\\
36.46	0\\
36.47	0\\
36.48	0\\
36.49	0\\
36.5	0\\
36.51	0\\
36.52	0\\
36.53	0\\
36.54	0\\
36.55	0\\
36.56	0\\
36.57	0\\
36.58	0\\
36.59	0\\
36.6	0\\
36.61	0\\
36.62	0\\
36.63	0\\
36.64	0\\
36.65	0\\
36.66	0\\
36.67	0\\
36.68	0\\
36.69	0\\
36.7	0\\
36.71	0\\
36.72	0\\
36.73	0\\
36.74	0\\
36.75	0\\
36.76	0\\
36.77	0\\
36.78	0\\
36.79	0\\
36.8	0\\
36.81	0\\
36.82	0\\
36.83	0\\
36.84	0\\
36.85	0\\
36.86	0\\
36.87	0\\
36.88	0\\
36.89	0\\
36.9	0\\
36.91	0\\
36.92	0\\
36.93	0\\
36.94	0\\
36.95	0\\
36.96	0\\
36.97	0\\
36.98	0\\
36.99	0\\
37	0\\
37.01	0\\
37.02	0\\
37.03	0\\
37.04	0\\
37.05	0\\
37.06	0\\
37.07	0\\
37.08	0\\
37.09	0\\
37.1	0\\
37.11	0\\
37.12	0\\
37.13	0\\
37.14	0\\
37.15	0\\
37.16	0\\
37.17	0\\
37.18	0\\
37.19	0\\
37.2	0\\
37.21	0\\
37.22	0\\
37.23	0\\
37.24	0\\
37.25	0\\
37.26	0\\
37.27	0\\
37.28	0\\
37.29	0\\
37.3	0\\
37.31	0\\
37.32	0\\
37.33	0\\
37.34	0\\
37.35	0\\
37.36	0\\
37.37	0\\
37.38	0\\
37.39	0\\
37.4	0\\
37.41	0\\
37.42	0\\
37.43	0\\
37.44	0\\
37.45	0\\
37.46	0\\
37.47	0\\
37.48	0\\
37.49	0\\
37.5	0\\
37.51	0\\
37.52	0\\
37.53	0\\
37.54	0\\
37.55	0\\
37.56	0\\
37.57	0\\
37.58	0\\
37.59	0\\
37.6	0\\
37.61	0\\
37.62	0\\
37.63	0\\
37.64	0\\
37.65	0\\
37.66	0\\
37.67	0\\
37.68	0\\
37.69	0\\
37.7	0\\
37.71	0\\
37.72	0\\
37.73	0\\
37.74	0\\
37.75	0\\
37.76	0\\
37.77	0\\
37.78	0\\
37.79	0\\
37.8	0\\
37.81	0\\
37.82	0\\
37.83	0\\
37.84	0\\
37.85	0\\
37.86	0\\
37.87	0\\
37.88	0\\
37.89	0\\
37.9	0\\
37.91	0\\
37.92	0\\
37.93	0\\
37.94	0\\
37.95	0\\
37.96	0\\
37.97	0\\
37.98	0\\
37.99	0\\
38	0\\
38.01	0\\
38.02	0\\
38.03	0\\
38.04	0\\
38.05	0\\
38.06	0\\
38.07	0\\
38.08	0\\
38.09	0\\
38.1	0\\
38.11	0\\
38.12	0\\
38.13	0\\
38.14	0\\
38.15	0\\
38.16	0\\
38.17	0\\
38.18	0\\
38.19	0\\
38.2	0\\
38.21	0\\
38.22	0\\
38.23	0\\
38.24	0\\
38.25	0\\
38.26	0\\
38.27	0\\
38.28	0\\
38.29	0\\
38.3	0\\
38.31	0\\
38.32	0\\
38.33	0\\
38.34	0\\
38.35	0\\
38.36	0\\
38.37	0\\
38.38	0\\
38.39	0\\
38.4	0\\
38.41	0\\
38.42	0\\
38.43	0\\
38.44	0\\
38.45	0\\
38.46	0\\
38.47	0\\
38.48	0\\
38.49	0\\
38.5	0\\
38.51	0\\
38.52	0\\
38.53	0\\
38.54	0\\
38.55	0\\
38.56	0\\
38.57	0\\
38.58	0\\
38.59	0\\
38.6	0\\
38.61	0\\
38.62	0\\
38.63	0\\
38.64	0\\
38.65	0\\
38.66	0\\
38.67	0\\
38.68	0\\
38.69	0\\
38.7	0\\
38.71	0\\
38.72	0\\
38.73	0\\
38.74	0\\
38.75	0\\
38.76	0\\
38.77	0\\
38.78	0\\
38.79	0\\
38.8	0\\
38.81	0\\
38.82	0\\
38.83	0\\
38.84	0\\
38.85	0\\
38.86	0\\
38.87	0\\
38.88	0\\
38.89	0\\
38.9	0\\
38.91	0\\
38.92	0\\
38.93	0\\
38.94	0\\
38.95	0\\
38.96	0\\
38.97	0\\
38.98	0\\
38.99	0\\
39	0\\
39.01	0\\
39.02	0\\
39.03	0\\
39.04	0\\
39.05	0\\
39.06	0\\
39.07	0\\
39.08	0\\
39.09	0\\
39.1	0\\
39.11	0\\
39.12	0\\
39.13	0\\
39.14	0\\
39.15	0\\
39.16	0\\
39.17	0\\
39.18	0\\
39.19	0\\
39.2	0\\
39.21	0\\
39.22	0\\
39.23	0\\
39.24	0\\
39.25	0\\
39.26	0\\
39.27	0\\
39.28	0\\
39.29	0\\
39.3	0\\
39.31	0\\
39.32	0\\
39.33	0\\
39.34	0\\
39.35	0\\
39.36	0\\
39.37	0\\
39.38	0\\
39.39	0\\
39.4	0\\
39.41	0\\
39.42	0\\
39.43	0\\
39.44	0\\
39.45	0\\
39.46	0\\
39.47	0\\
39.48	0\\
39.49	0\\
39.5	0\\
39.51	0\\
39.52	0\\
39.53	0\\
39.54	0\\
39.55	0\\
39.56	0\\
39.57	0\\
39.58	0\\
39.59	0\\
39.6	0\\
39.61	0\\
39.62	0\\
39.63	0\\
39.64	0\\
39.65	0\\
39.66	0\\
39.67	0\\
39.68	0\\
39.69	0\\
39.7	0\\
39.71	0\\
39.72	0\\
39.73	0\\
39.74	0\\
39.75	0\\
39.76	0\\
39.77	0\\
39.78	0\\
39.79	0\\
39.8	0\\
39.81	0\\
39.82	0\\
39.83	0\\
39.84	0\\
39.85	0\\
39.86	0\\
39.87	0\\
39.88	0\\
39.89	0\\
39.9	0\\
39.91	0\\
39.92	0\\
39.93	0\\
39.94	0\\
39.95	0\\
39.96	0\\
39.97	0\\
39.98	0\\
39.99	0\\
40	0\\
40.01	0\\
};
\addplot [color=black,solid,forget plot]
  table[row sep=crcr]{%
40.01	0\\
40.02	0\\
40.03	0\\
40.04	0\\
40.05	0\\
40.06	0\\
40.07	0\\
40.08	0\\
40.09	0\\
40.1	0\\
40.11	0\\
40.12	0\\
40.13	0\\
40.14	0\\
40.15	0\\
40.16	0\\
40.17	0\\
40.18	0\\
40.19	0\\
40.2	0\\
40.21	0\\
40.22	0\\
40.23	0\\
40.24	0\\
40.25	0\\
40.26	0\\
40.27	0\\
40.28	0\\
40.29	0\\
40.3	0\\
40.31	0\\
40.32	0\\
40.33	0\\
40.34	0\\
40.35	0\\
40.36	0\\
40.37	0\\
40.38	0\\
40.39	0\\
40.4	0\\
40.41	0\\
40.42	0\\
40.43	0\\
40.44	0\\
40.45	0\\
40.46	0\\
40.47	0\\
40.48	0\\
40.49	0\\
40.5	0\\
40.51	0\\
40.52	0\\
40.53	0\\
40.54	0\\
40.55	0\\
40.56	0\\
40.57	0\\
40.58	0\\
40.59	0\\
40.6	0\\
40.61	0\\
40.62	0\\
40.63	0\\
40.64	0\\
40.65	0\\
40.66	0\\
40.67	0\\
40.68	0\\
40.69	0\\
40.7	0\\
40.71	0\\
40.72	0\\
40.73	0\\
40.74	0\\
40.75	0\\
40.76	0\\
40.77	0\\
40.78	0\\
40.79	0\\
40.8	0\\
40.81	0\\
40.82	0\\
40.83	0\\
40.84	0\\
40.85	0\\
40.86	0\\
40.87	0\\
40.88	0\\
40.89	0\\
40.9	0\\
40.91	0\\
40.92	0\\
40.93	0\\
40.94	0\\
40.95	0\\
40.96	0\\
40.97	0\\
40.98	0\\
40.99	0\\
41	0\\
41.01	0\\
41.02	0\\
41.03	0\\
41.04	0\\
41.05	0\\
41.06	0\\
41.07	0\\
41.08	0\\
41.09	0\\
41.1	0\\
41.11	0\\
41.12	0\\
41.13	0\\
41.14	0\\
41.15	0\\
41.16	0\\
41.17	0\\
41.18	0\\
41.19	0\\
41.2	0\\
41.21	0\\
41.22	0\\
41.23	0\\
41.24	0\\
41.25	0\\
41.26	0\\
41.27	0\\
41.28	0\\
41.29	0\\
41.3	0\\
41.31	0\\
41.32	0\\
41.33	0\\
41.34	0\\
41.35	0\\
41.36	0\\
41.37	0\\
41.38	0\\
41.39	0\\
41.4	0\\
41.41	0\\
41.42	0\\
41.43	0\\
41.44	0\\
41.45	0\\
41.46	0\\
41.47	0\\
41.48	0\\
41.49	0\\
41.5	0\\
41.51	0\\
41.52	0\\
41.53	0\\
41.54	0\\
41.55	0\\
41.56	0\\
41.57	0\\
41.58	0\\
41.59	0\\
41.6	0\\
41.61	0\\
41.62	0\\
41.63	0\\
41.64	0\\
41.65	0\\
41.66	0\\
41.67	0\\
41.68	0\\
41.69	0\\
41.7	0\\
41.71	0\\
41.72	0\\
41.73	0\\
41.74	0\\
41.75	0\\
41.76	0\\
41.77	0\\
41.78	0\\
41.79	0\\
41.8	0\\
41.81	0\\
41.82	0\\
41.83	0\\
41.84	0\\
41.85	0\\
41.86	0\\
41.87	0\\
41.88	0\\
41.89	0\\
41.9	0\\
41.91	0\\
41.92	0\\
41.93	0\\
41.94	0\\
41.95	0\\
41.96	0\\
41.97	0\\
41.98	0\\
41.99	0\\
42	0\\
42.01	0\\
42.02	0\\
42.03	0\\
42.04	0\\
42.05	0\\
42.06	0\\
42.07	0\\
42.08	0\\
42.09	0\\
42.1	0\\
42.11	0\\
42.12	0\\
42.13	0\\
42.14	0\\
42.15	0\\
42.16	0\\
42.17	0\\
42.18	0\\
42.19	0\\
42.2	0\\
42.21	0\\
42.22	0\\
42.23	0\\
42.24	0\\
42.25	0\\
42.26	0\\
42.27	0\\
42.28	0\\
42.29	0\\
42.3	0\\
42.31	0\\
42.32	0\\
42.33	0\\
42.34	0\\
42.35	0\\
42.36	0\\
42.37	0\\
42.38	0\\
42.39	0\\
42.4	0\\
42.41	0\\
42.42	0\\
42.43	0\\
42.44	0\\
42.45	0\\
42.46	0\\
42.47	0\\
42.48	0\\
42.49	0\\
42.5	0\\
42.51	0\\
42.52	0\\
42.53	0\\
42.54	0\\
42.55	0\\
42.56	0\\
42.57	0\\
42.58	0\\
42.59	0\\
42.6	0\\
42.61	0\\
42.62	0\\
42.63	0\\
42.64	0\\
42.65	0\\
42.66	0\\
42.67	0\\
42.68	0\\
42.69	0\\
42.7	0\\
42.71	0\\
42.72	0\\
42.73	0\\
42.74	0\\
42.75	0\\
42.76	0\\
42.77	0\\
42.78	0\\
42.79	0\\
42.8	0\\
42.81	0\\
42.82	0\\
42.83	0\\
42.84	0\\
42.85	0\\
42.86	0\\
42.87	0\\
42.88	0\\
42.89	0\\
42.9	0\\
42.91	0\\
42.92	0\\
42.93	0\\
42.94	0\\
42.95	0\\
42.96	0\\
42.97	0\\
42.98	0\\
42.99	0\\
43	0\\
43.01	0\\
43.02	0\\
43.03	0\\
43.04	0\\
43.05	0\\
43.06	0\\
43.07	0\\
43.08	0\\
43.09	0\\
43.1	0\\
43.11	0\\
43.12	0\\
43.13	0\\
43.14	0\\
43.15	0\\
43.16	0\\
43.17	0\\
43.18	0\\
43.19	0\\
43.2	0\\
43.21	0\\
43.22	0\\
43.23	0\\
43.24	0\\
43.25	0\\
43.26	0\\
43.27	0\\
43.28	0\\
43.29	0\\
43.3	0\\
43.31	0\\
43.32	0\\
43.33	0\\
43.34	0\\
43.35	0\\
43.36	0\\
43.37	0\\
43.38	0\\
43.39	0\\
43.4	0\\
43.41	0\\
43.42	0\\
43.43	0\\
43.44	0\\
43.45	0\\
43.46	0\\
43.47	0\\
43.48	0\\
43.49	0\\
43.5	0\\
43.51	0\\
43.52	0\\
43.53	0\\
43.54	0\\
43.55	0\\
43.56	0\\
43.57	0\\
43.58	0\\
43.59	0\\
43.6	0\\
43.61	0\\
43.62	0\\
43.63	0\\
43.64	0\\
43.65	0\\
43.66	0\\
43.67	0\\
43.68	0\\
43.69	0\\
43.7	0\\
43.71	0\\
43.72	0\\
43.73	0\\
43.74	0\\
43.75	0\\
43.76	0\\
43.77	0\\
43.78	0\\
43.79	0\\
43.8	0\\
43.81	0\\
43.82	0\\
43.83	0\\
43.84	0\\
43.85	0\\
43.86	0\\
43.87	0\\
43.88	0\\
43.89	0\\
43.9	0\\
43.91	0\\
43.92	0\\
43.93	0\\
43.94	0\\
43.95	0\\
43.96	0\\
43.97	0\\
43.98	0\\
43.99	0\\
44	0\\
44.01	0\\
44.02	0\\
44.03	0\\
44.04	0\\
44.05	0\\
44.06	0\\
44.07	0\\
44.08	0\\
44.09	0\\
44.1	0\\
44.11	0\\
44.12	0\\
44.13	0\\
44.14	0\\
44.15	0\\
44.16	0\\
44.17	0\\
44.18	0\\
44.19	0\\
44.2	0\\
44.21	0\\
44.22	0\\
44.23	0\\
44.24	0\\
44.25	0\\
44.26	0\\
44.27	0\\
44.28	0\\
44.29	0\\
44.3	0\\
44.31	0\\
44.32	0\\
44.33	0\\
44.34	0\\
44.35	0\\
44.36	0\\
44.37	0\\
44.38	0\\
44.39	0\\
44.4	0\\
44.41	0\\
44.42	0\\
44.43	0\\
44.44	0\\
44.45	0\\
44.46	0\\
44.47	0\\
44.48	0\\
44.49	0\\
44.5	0\\
44.51	0\\
44.52	0\\
44.53	0\\
44.54	0\\
44.55	0\\
44.56	0\\
44.57	0\\
44.58	0\\
44.59	0\\
44.6	0\\
44.61	0\\
44.62	0\\
44.63	0\\
44.64	0\\
44.65	0\\
44.66	0\\
44.67	0\\
44.68	0\\
44.69	0\\
44.7	0\\
44.71	0\\
44.72	0\\
44.73	0\\
44.74	0\\
44.75	0\\
44.76	0\\
44.77	0\\
44.78	0\\
44.79	0\\
44.8	0\\
44.81	0\\
44.82	0\\
44.83	0\\
44.84	0\\
44.85	0\\
44.86	0\\
44.87	0\\
44.88	0\\
44.89	0\\
44.9	0\\
44.91	0\\
44.92	0\\
44.93	0\\
44.94	0\\
44.95	0\\
44.96	0\\
44.97	0\\
44.98	0\\
44.99	0\\
45	0\\
45.01	0\\
45.02	0\\
45.03	0\\
45.04	0\\
45.05	0\\
45.06	0\\
45.07	0\\
45.08	0\\
45.09	0\\
45.1	0\\
45.11	0\\
45.12	0\\
45.13	0\\
45.14	0\\
45.15	0\\
45.16	0\\
45.17	0\\
45.18	0\\
45.19	0\\
45.2	0\\
45.21	0\\
45.22	0\\
45.23	0\\
45.24	0\\
45.25	0\\
45.26	0\\
45.27	0\\
45.28	0\\
45.29	0\\
45.3	0\\
45.31	0\\
45.32	0\\
45.33	0\\
45.34	0\\
45.35	0\\
45.36	0\\
45.37	0\\
45.38	0\\
45.39	0\\
45.4	0\\
45.41	0\\
45.42	0\\
45.43	0\\
45.44	0\\
45.45	0\\
45.46	0\\
45.47	0\\
45.48	0\\
45.49	0\\
45.5	0\\
45.51	0\\
45.52	0\\
45.53	0\\
45.54	0\\
45.55	0\\
45.56	0\\
45.57	0\\
45.58	0\\
45.59	0\\
45.6	0\\
45.61	0\\
45.62	0\\
45.63	0\\
45.64	0\\
45.65	0\\
45.66	0\\
45.67	0\\
45.68	0\\
45.69	0\\
45.7	0\\
45.71	0\\
45.72	0\\
45.73	0\\
45.74	0\\
45.75	0\\
45.76	0\\
45.77	0\\
45.78	0\\
45.79	0\\
45.8	0\\
45.81	0\\
45.82	0\\
45.83	0\\
45.84	0\\
45.85	0\\
45.86	0\\
45.87	0\\
45.88	0\\
45.89	0\\
45.9	0\\
45.91	0\\
45.92	0\\
45.93	0\\
45.94	0\\
45.95	0\\
45.96	0\\
45.97	0\\
45.98	0\\
45.99	0\\
46	0\\
46.01	0\\
46.02	0\\
46.03	0\\
46.04	0\\
46.05	0\\
46.06	0\\
46.07	0\\
46.08	0\\
46.09	0\\
46.1	0\\
46.11	0\\
46.12	0\\
46.13	0\\
46.14	0\\
46.15	0\\
46.16	0\\
46.17	0\\
46.18	0\\
46.19	0\\
46.2	0\\
46.21	0\\
46.22	0\\
46.23	0\\
46.24	0\\
46.25	0\\
46.26	0\\
46.27	0\\
46.28	0\\
46.29	0\\
46.3	0\\
46.31	0\\
46.32	0\\
46.33	0\\
46.34	0\\
46.35	0\\
46.36	0\\
46.37	0\\
46.38	0\\
46.39	0\\
46.4	0\\
46.41	0\\
46.42	0\\
46.43	0\\
46.44	0\\
46.45	0\\
46.46	0\\
46.47	0\\
46.48	0\\
46.49	0\\
46.5	0\\
46.51	0\\
46.52	0\\
46.53	0\\
46.54	0\\
46.55	0\\
46.56	0\\
46.57	0\\
46.58	0\\
46.59	0\\
46.6	0\\
46.61	0\\
46.62	0\\
46.63	0\\
46.64	0\\
46.65	0\\
46.66	0\\
46.67	0\\
46.68	0\\
46.69	0\\
46.7	0\\
46.71	0\\
46.72	0\\
46.73	0\\
46.74	0\\
46.75	0\\
46.76	0\\
46.77	0\\
46.78	0\\
46.79	0\\
46.8	0\\
46.81	0\\
46.82	0\\
46.83	0\\
46.84	0\\
46.85	0\\
46.86	0\\
46.87	0\\
46.88	0\\
46.89	0\\
46.9	0\\
46.91	0\\
46.92	0\\
46.93	0\\
46.94	0\\
46.95	0\\
46.96	0\\
46.97	0\\
46.98	0\\
46.99	0\\
47	0\\
47.01	0\\
47.02	0\\
47.03	0\\
47.04	0\\
47.05	0\\
47.06	0\\
47.07	0\\
47.08	0\\
47.09	0\\
47.1	0\\
47.11	0\\
47.12	0\\
47.13	0\\
47.14	0\\
47.15	0\\
47.16	0\\
47.17	0\\
47.18	0\\
47.19	0\\
47.2	0\\
47.21	0\\
47.22	0\\
47.23	0\\
47.24	0\\
47.25	0\\
47.26	0\\
47.27	0\\
47.28	0\\
47.29	0\\
47.3	0\\
47.31	0\\
47.32	0\\
47.33	0\\
47.34	0\\
47.35	0\\
47.36	0\\
47.37	0\\
47.38	0\\
47.39	0\\
47.4	0\\
47.41	0\\
47.42	0\\
47.43	0\\
47.44	0\\
47.45	0\\
47.46	0\\
47.47	0\\
47.48	0\\
47.49	0\\
47.5	0\\
47.51	0\\
47.52	0\\
47.53	0\\
47.54	0\\
47.55	0\\
47.56	0\\
47.57	0\\
47.58	0\\
47.59	0\\
47.6	0\\
47.61	0\\
47.62	0\\
47.63	0\\
47.64	0\\
47.65	0\\
47.66	0\\
47.67	0\\
47.68	0\\
47.69	0\\
47.7	0\\
47.71	0\\
47.72	0\\
47.73	0\\
47.74	0\\
47.75	0\\
47.76	0\\
47.77	0\\
47.78	0\\
47.79	0\\
47.8	0\\
47.81	0\\
47.82	0\\
47.83	0\\
47.84	0\\
47.85	0\\
47.86	0\\
47.87	0\\
47.88	0\\
47.89	0\\
47.9	0\\
47.91	0\\
47.92	0\\
47.93	0\\
47.94	0\\
47.95	0\\
47.96	0\\
47.97	0\\
47.98	0\\
47.99	0\\
48	0\\
48.01	0\\
48.02	0\\
48.03	0\\
48.04	0\\
48.05	0\\
48.06	0\\
48.07	0\\
48.08	0\\
48.09	0\\
48.1	0\\
48.11	0\\
48.12	0\\
48.13	0\\
48.14	0\\
48.15	0\\
48.16	0\\
48.17	0\\
48.18	0\\
48.19	0\\
48.2	0\\
48.21	0\\
48.22	0\\
48.23	0\\
48.24	0\\
48.25	0\\
48.26	0\\
48.27	0\\
48.28	0\\
48.29	0\\
48.3	0\\
48.31	0\\
48.32	0\\
48.33	0\\
48.34	0\\
48.35	0\\
48.36	0\\
48.37	0\\
48.38	0\\
48.39	0\\
48.4	0\\
48.41	0\\
48.42	0\\
48.43	0\\
48.44	0\\
48.45	0\\
48.46	0\\
48.47	0\\
48.48	0\\
48.49	0\\
48.5	0\\
48.51	0\\
48.52	0\\
48.53	0\\
48.54	0\\
48.55	0\\
48.56	0\\
48.57	0\\
48.58	0\\
48.59	0\\
48.6	0\\
48.61	0\\
48.62	0\\
48.63	0\\
48.64	0\\
48.65	0\\
48.66	0\\
48.67	0\\
48.68	0\\
48.69	0\\
48.7	0\\
48.71	0\\
48.72	0\\
48.73	0\\
48.74	0\\
48.75	0\\
48.76	0\\
48.77	0\\
48.78	0\\
48.79	0\\
48.8	0\\
48.81	0\\
48.82	0\\
48.83	0\\
48.84	0\\
48.85	0\\
48.86	0\\
48.87	0\\
48.88	0\\
48.89	0\\
48.9	0\\
48.91	0\\
48.92	0\\
48.93	0\\
48.94	0\\
48.95	0\\
48.96	0\\
48.97	0\\
48.98	0\\
48.99	0\\
49	0\\
49.01	0\\
49.02	0\\
49.03	0\\
49.04	0\\
49.05	0\\
49.06	0\\
49.07	0\\
49.08	0\\
49.09	0\\
49.1	0\\
49.11	0\\
49.12	0\\
49.13	0\\
49.14	0\\
49.15	0\\
49.16	0\\
49.17	0\\
49.18	0\\
49.19	0\\
49.2	0\\
49.21	0\\
49.22	0\\
49.23	0\\
49.24	0\\
49.25	0\\
49.26	0\\
49.27	0\\
49.28	0\\
49.29	0\\
49.3	0\\
49.31	0\\
49.32	0\\
49.33	0\\
49.34	0\\
49.35	0\\
49.36	0\\
49.37	0\\
49.38	0\\
49.39	0\\
49.4	0\\
49.41	0\\
49.42	0\\
49.43	0\\
49.44	0\\
49.45	0\\
49.46	0\\
49.47	0\\
49.48	0\\
49.49	0\\
49.5	0\\
49.51	0\\
49.52	0\\
49.53	0\\
49.54	0\\
49.55	0\\
49.56	0\\
49.57	0\\
49.58	0\\
49.59	0\\
49.6	0\\
49.61	0\\
49.62	0\\
49.63	0\\
49.64	0\\
49.65	0\\
49.66	0\\
49.67	0\\
49.68	0\\
49.69	0\\
49.7	0\\
49.71	0\\
49.72	0\\
49.73	0\\
49.74	0\\
49.75	0\\
49.76	0\\
49.77	0\\
49.78	0\\
49.79	0\\
49.8	0\\
49.81	0\\
49.82	0\\
49.83	0\\
49.84	0\\
49.85	0\\
49.86	0\\
49.87	0\\
49.88	0\\
49.89	0\\
49.9	0\\
49.91	0\\
49.92	0\\
49.93	0\\
49.94	0\\
49.95	0\\
49.96	0\\
49.97	0\\
49.98	0\\
49.99	0\\
50	0\\
50.01	0\\
50.02	0\\
50.03	0\\
50.04	0\\
50.05	0\\
50.06	0\\
50.07	0\\
50.08	0\\
50.09	0\\
50.1	0\\
50.11	0\\
50.12	0\\
50.13	0\\
50.14	0\\
50.15	0\\
50.16	0\\
50.17	0\\
50.18	0\\
50.19	0\\
50.2	0\\
50.21	0\\
50.22	0\\
50.23	0\\
50.24	0\\
50.25	0\\
50.26	0\\
50.27	0\\
50.28	0\\
50.29	0\\
50.3	0\\
50.31	0\\
50.32	0\\
50.33	0\\
50.34	0\\
50.35	0\\
50.36	0\\
50.37	0\\
50.38	0\\
50.39	0\\
50.4	0\\
50.41	0\\
50.42	0\\
50.43	0\\
50.44	0\\
50.45	0\\
50.46	0\\
50.47	0\\
50.48	0\\
50.49	0\\
50.5	0\\
50.51	0\\
50.52	0\\
50.53	0\\
50.54	0\\
50.55	0\\
50.56	0\\
50.57	0\\
50.58	0\\
50.59	0\\
50.6	0\\
50.61	0\\
50.62	0\\
50.63	0\\
50.64	0\\
50.65	0\\
50.66	0\\
50.67	0\\
50.68	0\\
50.69	0\\
50.7	0\\
50.71	0\\
50.72	0\\
50.73	0\\
50.74	0\\
50.75	0\\
50.76	0\\
50.77	0\\
50.78	0\\
50.79	0\\
50.8	0\\
50.81	0\\
50.82	0\\
50.83	0\\
50.84	0\\
50.85	0\\
50.86	0\\
50.87	0\\
50.88	0\\
50.89	0\\
50.9	0\\
50.91	0\\
50.92	0\\
50.93	0\\
50.94	0\\
50.95	0\\
50.96	0\\
50.97	0\\
50.98	0\\
50.99	0\\
51	0\\
51.01	0\\
51.02	0\\
51.03	0\\
51.04	0\\
51.05	0\\
51.06	0\\
51.07	0\\
51.08	0\\
51.09	0\\
51.1	0\\
51.11	0\\
51.12	0\\
51.13	0\\
51.14	0\\
51.15	0\\
51.16	0\\
51.17	0\\
51.18	0\\
51.19	0\\
51.2	0\\
51.21	0\\
51.22	0\\
51.23	0\\
51.24	0\\
51.25	0\\
51.26	0\\
51.27	0\\
51.28	0\\
51.29	0\\
51.3	0\\
51.31	0\\
51.32	0\\
51.33	0\\
51.34	0\\
51.35	0\\
51.36	0\\
51.37	0\\
51.38	0\\
51.39	0\\
51.4	0\\
51.41	0\\
51.42	0\\
51.43	0\\
51.44	0\\
51.45	0\\
51.46	0\\
51.47	0\\
51.48	0\\
51.49	0\\
51.5	0\\
51.51	0\\
51.52	0\\
51.53	0\\
51.54	0\\
51.55	0\\
51.56	0\\
51.57	0\\
51.58	0\\
51.59	0\\
51.6	0\\
51.61	0\\
51.62	0\\
51.63	0\\
51.64	0\\
51.65	0\\
51.66	0\\
51.67	0\\
51.68	0\\
51.69	0\\
51.7	0\\
51.71	0\\
51.72	0\\
51.73	0\\
51.74	0\\
51.75	0\\
51.76	0\\
51.77	0\\
51.78	0\\
51.79	0\\
51.8	0\\
51.81	0\\
51.82	0\\
51.83	0\\
51.84	0\\
51.85	0\\
51.86	0\\
51.87	0\\
51.88	0\\
51.89	0\\
51.9	0\\
51.91	0\\
51.92	0\\
51.93	0\\
51.94	0\\
51.95	0\\
51.96	0\\
51.97	0\\
51.98	0\\
51.99	0\\
52	0\\
52.01	0\\
52.02	0\\
52.03	0\\
52.04	0\\
52.05	0\\
52.06	0\\
52.07	0\\
52.08	0\\
52.09	0\\
52.1	0\\
52.11	0\\
52.12	0\\
52.13	0\\
52.14	0\\
52.15	0\\
52.16	0\\
52.17	0\\
52.18	0\\
52.19	0\\
52.2	0\\
52.21	0\\
52.22	0\\
52.23	0\\
52.24	0\\
52.25	0\\
52.26	0\\
52.27	0\\
52.28	0\\
52.29	0\\
52.3	0\\
52.31	0\\
52.32	0\\
52.33	0\\
52.34	0\\
52.35	0\\
52.36	0\\
52.37	0\\
52.38	0\\
52.39	0\\
52.4	0\\
52.41	0\\
52.42	0\\
52.43	0\\
52.44	0\\
52.45	0\\
52.46	0\\
52.47	0\\
52.48	0\\
52.49	0\\
52.5	0\\
52.51	0\\
52.52	0\\
52.53	0\\
52.54	0\\
52.55	0\\
52.56	0\\
52.57	0\\
52.58	0\\
52.59	0\\
52.6	0\\
52.61	0\\
52.62	0\\
52.63	0\\
52.64	0\\
52.65	0\\
52.66	0\\
52.67	0\\
52.68	0\\
52.69	0\\
52.7	0\\
52.71	0\\
52.72	0\\
52.73	0\\
52.74	0\\
52.75	0\\
52.76	0\\
52.77	0\\
52.78	0\\
52.79	0\\
52.8	0\\
52.81	0\\
52.82	0\\
52.83	0\\
52.84	0\\
52.85	0\\
52.86	0\\
52.87	0\\
52.88	0\\
52.89	0\\
52.9	0\\
52.91	0\\
52.92	0\\
52.93	0\\
52.94	0\\
52.95	0\\
52.96	0\\
52.97	0\\
52.98	0\\
52.99	0\\
53	0\\
53.01	0\\
53.02	0\\
53.03	0\\
53.04	0\\
53.05	0\\
53.06	0\\
53.07	0\\
53.08	0\\
53.09	0\\
53.1	0\\
53.11	0\\
53.12	0\\
53.13	0\\
53.14	0\\
53.15	0\\
53.16	0\\
53.17	0\\
53.18	0\\
53.19	0\\
53.2	0\\
53.21	0\\
53.22	0\\
53.23	0\\
53.24	0\\
53.25	0\\
53.26	0\\
53.27	0\\
53.28	0\\
53.29	0\\
53.3	0\\
53.31	0\\
53.32	0\\
53.33	0\\
53.34	0\\
53.35	0\\
53.36	0\\
53.37	0\\
53.38	0\\
53.39	0\\
53.4	0\\
53.41	0\\
53.42	0\\
53.43	0\\
53.44	0\\
53.45	0\\
53.46	0\\
53.47	0\\
53.48	0\\
53.49	0\\
53.5	0\\
53.51	0\\
53.52	0\\
53.53	0\\
53.54	0\\
53.55	0\\
53.56	0\\
53.57	0\\
53.58	0\\
53.59	0\\
53.6	0\\
53.61	0\\
53.62	0\\
53.63	0\\
53.64	0\\
53.65	0\\
53.66	0\\
53.67	0\\
53.68	0\\
53.69	0\\
53.7	0\\
53.71	0\\
53.72	0\\
53.73	0\\
53.74	0\\
53.75	0\\
53.76	0\\
53.77	0\\
53.78	0\\
53.79	0\\
53.8	0\\
53.81	0\\
53.82	0\\
53.83	0\\
53.84	0\\
53.85	0\\
53.86	0\\
53.87	0\\
53.88	0\\
53.89	0\\
53.9	0\\
53.91	0\\
53.92	0\\
53.93	0\\
53.94	0\\
53.95	0\\
53.96	0\\
53.97	0\\
53.98	0\\
53.99	0\\
54	0\\
54.01	0\\
54.02	0\\
54.03	0\\
54.04	0\\
54.05	0\\
54.06	0\\
54.07	0\\
54.08	0\\
54.09	0\\
54.1	0\\
54.11	0\\
54.12	0\\
54.13	0\\
54.14	0\\
54.15	0\\
54.16	0\\
54.17	0\\
54.18	0\\
54.19	0\\
54.2	0\\
54.21	0\\
54.22	0\\
54.23	0\\
54.24	0\\
54.25	0\\
54.26	0\\
54.27	0\\
54.28	0\\
54.29	0\\
54.3	0\\
54.31	0\\
54.32	0\\
54.33	0\\
54.34	0\\
54.35	0\\
54.36	0\\
54.37	0\\
54.38	0\\
54.39	0\\
54.4	0\\
54.41	0\\
54.42	0\\
54.43	0\\
54.44	0\\
54.45	0\\
54.46	0\\
54.47	0\\
54.48	0\\
54.49	0\\
54.5	0\\
54.51	0\\
54.52	0\\
54.53	0\\
54.54	0\\
54.55	0\\
54.56	0\\
54.57	0\\
54.58	0\\
54.59	0\\
54.6	0\\
54.61	0\\
54.62	0\\
54.63	0\\
54.64	0\\
54.65	0\\
54.66	0\\
54.67	0\\
54.68	0\\
54.69	0\\
54.7	0\\
54.71	0\\
54.72	0\\
54.73	0\\
54.74	0\\
54.75	0\\
54.76	0\\
54.77	0\\
54.78	0\\
54.79	0\\
54.8	0\\
54.81	0\\
54.82	0\\
54.83	0\\
54.84	0\\
54.85	0\\
54.86	0\\
54.87	0\\
54.88	0\\
54.89	0\\
54.9	0\\
54.91	0\\
54.92	0\\
54.93	0\\
54.94	0\\
54.95	0\\
54.96	0\\
54.97	0\\
54.98	0\\
54.99	0\\
55	0\\
55.01	0\\
55.02	0\\
55.03	0\\
55.04	0\\
55.05	0\\
55.06	0\\
55.07	0\\
55.08	0\\
55.09	0\\
55.1	0\\
55.11	0\\
55.12	0\\
55.13	0\\
55.14	0\\
55.15	0\\
55.16	0\\
55.17	0\\
55.18	0\\
55.19	0\\
55.2	0\\
55.21	0\\
55.22	0\\
55.23	0\\
55.24	0\\
55.25	0\\
55.26	0\\
55.27	0\\
55.28	0\\
55.29	0\\
55.3	0\\
55.31	0\\
55.32	0\\
55.33	0\\
55.34	0\\
55.35	0\\
55.36	0\\
55.37	0\\
55.38	0\\
55.39	0\\
55.4	0\\
55.41	0\\
55.42	0\\
55.43	0\\
55.44	0\\
55.45	0\\
55.46	0\\
55.47	0\\
55.48	0\\
55.49	0\\
55.5	0\\
55.51	0\\
55.52	0\\
55.53	0\\
55.54	0\\
55.55	0\\
55.56	0\\
55.57	0\\
55.58	0\\
55.59	0\\
55.6	0\\
55.61	0\\
55.62	0\\
55.63	0\\
55.64	0\\
55.65	0\\
55.66	0\\
55.67	0\\
55.68	0\\
55.69	0\\
55.7	0\\
55.71	0\\
55.72	0\\
55.73	0\\
55.74	0\\
55.75	0\\
55.76	0\\
55.77	0\\
55.78	0\\
55.79	0\\
55.8	0\\
55.81	0\\
55.82	0\\
55.83	0\\
55.84	0\\
55.85	0\\
55.86	0\\
55.87	0\\
55.88	0\\
55.89	0\\
55.9	0\\
55.91	0\\
55.92	0\\
55.93	0\\
55.94	0\\
55.95	0\\
55.96	0\\
55.97	0\\
55.98	0\\
55.99	0\\
56	0\\
56.01	0\\
56.02	0\\
56.03	0\\
56.04	0\\
56.05	0\\
56.06	0\\
56.07	0\\
56.08	0\\
56.09	0\\
56.1	0\\
56.11	0\\
56.12	0\\
56.13	0\\
56.14	0\\
56.15	0\\
56.16	0\\
56.17	0\\
56.18	0\\
56.19	0\\
56.2	0\\
56.21	0\\
56.22	0\\
56.23	0\\
56.24	0\\
56.25	0\\
56.26	0\\
56.27	0\\
56.28	0\\
56.29	0\\
56.3	0\\
56.31	0\\
56.32	0\\
56.33	0\\
56.34	0\\
56.35	0\\
56.36	0\\
56.37	0\\
56.38	0\\
56.39	0\\
56.4	0\\
56.41	0\\
56.42	0\\
56.43	0\\
56.44	0\\
56.45	0\\
56.46	0\\
56.47	0\\
56.48	0\\
56.49	0\\
56.5	0\\
56.51	0\\
56.52	0\\
56.53	0\\
56.54	0\\
56.55	0\\
56.56	0\\
56.57	0\\
56.58	0\\
56.59	0\\
56.6	0\\
56.61	0\\
56.62	0\\
56.63	0\\
56.64	0\\
56.65	0\\
56.66	0\\
56.67	0\\
56.68	0\\
56.69	0\\
56.7	0\\
56.71	0\\
56.72	0\\
56.73	0\\
56.74	0\\
56.75	0\\
56.76	0\\
56.77	0\\
56.78	0\\
56.79	0\\
56.8	0\\
56.81	0\\
56.82	0\\
56.83	0\\
56.84	0\\
56.85	0\\
56.86	0\\
56.87	0\\
56.88	0\\
56.89	0\\
56.9	0\\
56.91	0\\
56.92	0\\
56.93	0\\
56.94	0\\
56.95	0\\
56.96	0\\
56.97	0\\
56.98	0\\
56.99	0\\
57	0\\
57.01	0\\
57.02	0\\
57.03	0\\
57.04	0\\
57.05	0\\
57.06	0\\
57.07	0\\
57.08	0\\
57.09	0\\
57.1	0\\
57.11	0\\
57.12	0\\
57.13	0\\
57.14	0\\
57.15	0\\
57.16	0\\
57.17	0\\
57.18	0\\
57.19	0\\
57.2	0\\
57.21	0\\
57.22	0\\
57.23	0\\
57.24	0\\
57.25	0\\
57.26	0\\
57.27	0\\
57.28	0\\
57.29	0\\
57.3	0\\
57.31	0\\
57.32	0\\
57.33	0\\
57.34	0\\
57.35	0\\
57.36	0\\
57.37	0\\
57.38	0\\
57.39	0\\
57.4	0\\
57.41	0\\
57.42	0\\
57.43	0\\
57.44	0\\
57.45	0\\
57.46	0\\
57.47	0\\
57.48	0\\
57.49	0\\
57.5	0\\
57.51	0\\
57.52	0\\
57.53	0\\
57.54	0\\
57.55	0\\
57.56	0\\
57.57	0\\
57.58	0\\
57.59	0\\
57.6	0\\
57.61	0\\
57.62	0\\
57.63	0\\
57.64	0\\
57.65	0\\
57.66	0\\
57.67	0\\
57.68	0\\
57.69	0\\
57.7	0\\
57.71	0\\
57.72	0\\
57.73	0\\
57.74	0\\
57.75	0\\
57.76	0\\
57.77	0\\
57.78	0\\
57.79	0\\
57.8	0\\
57.81	0\\
57.82	0\\
57.83	0\\
57.84	0\\
57.85	0\\
57.86	0\\
57.87	0\\
57.88	0\\
57.89	0\\
57.9	0\\
57.91	0\\
57.92	0\\
57.93	0\\
57.94	0\\
57.95	0\\
57.96	0\\
57.97	0\\
57.98	0\\
57.99	0\\
58	0\\
58.01	0\\
58.02	0\\
58.03	0\\
58.04	0\\
58.05	0\\
58.06	0\\
58.07	0\\
58.08	0\\
58.09	0\\
58.1	0\\
58.11	0\\
58.12	0\\
58.13	0\\
58.14	0\\
58.15	0\\
58.16	0\\
58.17	0\\
58.18	0\\
58.19	0\\
58.2	0\\
58.21	0\\
58.22	0\\
58.23	0\\
58.24	0\\
58.25	0\\
58.26	0\\
58.27	0\\
58.28	0\\
58.29	0\\
58.3	0\\
58.31	0\\
58.32	0\\
58.33	0\\
58.34	0\\
58.35	0\\
58.36	0\\
58.37	0\\
58.38	0\\
58.39	0\\
58.4	0\\
58.41	0\\
58.42	0\\
58.43	0\\
58.44	0\\
58.45	0\\
58.46	0\\
58.47	0\\
58.48	0\\
58.49	0\\
58.5	0\\
58.51	0\\
58.52	0\\
58.53	0\\
58.54	0\\
58.55	0\\
58.56	0\\
58.57	0\\
58.58	0\\
58.59	0\\
58.6	0\\
58.61	0\\
58.62	0\\
58.63	0\\
58.64	0\\
58.65	0\\
58.66	0\\
58.67	0\\
58.68	0\\
58.69	0\\
58.7	0\\
58.71	0\\
58.72	0\\
58.73	0\\
58.74	0\\
58.75	0\\
58.76	0\\
58.77	0\\
58.78	0\\
58.79	0\\
58.8	0\\
58.81	0\\
58.82	0\\
58.83	0\\
58.84	0\\
58.85	0\\
58.86	0\\
58.87	0\\
58.88	0\\
58.89	0\\
58.9	0\\
58.91	0\\
58.92	0\\
58.93	0\\
58.94	0\\
58.95	0\\
58.96	0\\
58.97	0\\
58.98	0\\
58.99	0\\
59	0\\
59.01	0\\
59.02	0\\
59.03	0\\
59.04	0\\
59.05	0\\
59.06	0\\
59.07	0\\
59.08	0\\
59.09	0\\
59.1	0\\
59.11	0\\
59.12	0\\
59.13	0\\
59.14	0\\
59.15	0\\
59.16	0\\
59.17	0\\
59.18	0\\
59.19	0\\
59.2	0\\
59.21	0\\
59.22	0\\
59.23	0\\
59.24	0\\
59.25	0\\
59.26	0\\
59.27	0\\
59.28	0\\
59.29	0\\
59.3	0\\
59.31	0\\
59.32	0\\
59.33	0\\
59.34	0\\
59.35	0\\
59.36	0\\
59.37	0\\
59.38	0\\
59.39	0\\
59.4	0\\
59.41	0\\
59.42	0\\
59.43	0\\
59.44	0\\
59.45	0\\
59.46	0\\
59.47	0\\
59.48	0\\
59.49	0\\
59.5	0\\
59.51	0\\
59.52	0\\
59.53	0\\
59.54	0\\
59.55	0\\
59.56	0\\
59.57	0\\
59.58	0\\
59.59	0\\
59.6	0\\
59.61	0\\
59.62	0\\
59.63	0\\
59.64	0\\
59.65	0\\
59.66	0\\
59.67	0\\
59.68	0\\
59.69	0\\
59.7	0\\
59.71	0\\
59.72	0\\
59.73	0\\
59.74	0\\
59.75	0\\
59.76	0\\
59.77	0\\
59.78	0\\
59.79	0\\
59.8	0\\
59.81	0\\
59.82	0\\
59.83	0\\
59.84	0\\
59.85	0\\
59.86	0\\
59.87	0\\
59.88	0\\
59.89	0\\
59.9	0\\
59.91	0\\
59.92	0\\
59.93	0\\
59.94	0\\
59.95	0\\
59.96	0\\
59.97	0\\
59.98	0\\
59.99	0\\
60	0\\
60.01	0\\
60.02	0\\
60.03	0\\
60.04	0\\
60.05	0\\
60.06	0\\
60.07	0\\
60.08	0\\
60.09	0\\
60.1	0\\
60.11	0\\
60.12	0\\
60.13	0\\
60.14	0\\
60.15	0\\
60.16	0\\
60.17	0\\
60.18	0\\
60.19	0\\
60.2	0\\
60.21	0\\
60.22	0\\
60.23	0\\
60.24	0\\
60.25	0\\
60.26	0\\
60.27	0\\
60.28	0\\
60.29	0\\
60.3	0\\
60.31	0\\
60.32	0\\
60.33	0\\
60.34	0\\
60.35	0\\
60.36	0\\
60.37	0\\
60.38	0\\
60.39	0\\
60.4	0\\
60.41	0\\
60.42	0\\
60.43	0\\
60.44	0\\
60.45	0\\
60.46	0\\
60.47	0\\
60.48	0\\
60.49	0\\
60.5	0\\
60.51	0\\
60.52	0\\
60.53	0\\
60.54	0\\
60.55	0\\
60.56	0\\
60.57	0\\
60.58	0\\
60.59	0\\
60.6	0\\
60.61	0\\
60.62	0\\
60.63	0\\
60.64	0\\
60.65	0\\
60.66	0\\
60.67	0\\
60.68	0\\
60.69	0\\
60.7	0\\
60.71	0\\
60.72	0\\
60.73	0\\
60.74	0\\
60.75	0\\
60.76	0\\
60.77	0\\
60.78	0\\
60.79	0\\
60.8	0\\
60.81	0\\
60.82	0\\
60.83	0\\
60.84	0\\
60.85	0\\
60.86	0\\
60.87	0\\
60.88	0\\
60.89	0\\
60.9	0\\
60.91	0\\
60.92	0\\
60.93	0\\
60.94	0\\
60.95	0\\
60.96	0\\
60.97	0\\
60.98	0\\
60.99	0\\
61	0\\
61.01	0\\
61.02	0\\
61.03	0\\
61.04	0\\
61.05	0\\
61.06	0\\
61.07	0\\
61.08	0\\
61.09	0\\
61.1	0\\
61.11	0\\
61.12	0\\
61.13	0\\
61.14	0\\
61.15	0\\
61.16	0\\
61.17	0\\
61.18	0\\
61.19	0\\
61.2	0\\
61.21	0\\
61.22	0\\
61.23	0\\
61.24	0\\
61.25	0\\
61.26	0\\
61.27	0\\
61.28	0\\
61.29	0\\
61.3	0\\
61.31	0\\
61.32	0\\
61.33	0\\
61.34	0\\
61.35	0\\
61.36	0\\
61.37	0\\
61.38	0\\
61.39	0\\
61.4	0\\
61.41	0\\
61.42	0\\
61.43	0\\
61.44	0\\
61.45	0\\
61.46	0\\
61.47	0\\
61.48	0\\
61.49	0\\
61.5	0\\
61.51	0\\
61.52	0\\
61.53	0\\
61.54	0\\
61.55	0\\
61.56	0\\
61.57	0\\
61.58	0\\
61.59	0\\
61.6	0\\
61.61	0\\
61.62	0\\
61.63	0\\
61.64	0\\
61.65	0\\
61.66	0\\
61.67	0\\
61.68	0\\
61.69	0\\
61.7	0\\
61.71	0\\
61.72	0\\
61.73	0\\
61.74	0\\
61.75	0\\
61.76	0\\
61.77	0\\
61.78	0\\
61.79	0\\
61.8	0\\
61.81	0\\
61.82	0\\
61.83	0\\
61.84	0\\
61.85	0\\
61.86	0\\
61.87	0\\
61.88	0\\
61.89	0\\
61.9	0\\
61.91	0\\
61.92	0\\
61.93	0\\
61.94	0\\
61.95	0\\
61.96	0\\
61.97	0\\
61.98	0\\
61.99	0\\
62	0\\
62.01	0\\
62.02	0\\
62.03	0\\
62.04	0\\
62.05	0\\
62.06	0\\
62.07	0\\
62.08	0\\
62.09	0\\
62.1	0\\
62.11	0\\
62.12	0\\
62.13	0\\
62.14	0\\
62.15	0\\
62.16	0\\
62.17	0\\
62.18	0\\
62.19	0\\
62.2	0\\
62.21	0\\
62.22	0\\
62.23	0\\
62.24	0\\
62.25	0\\
62.26	0\\
62.27	0\\
62.28	0\\
62.29	0\\
62.3	0\\
62.31	0\\
62.32	0\\
62.33	0\\
62.34	0\\
62.35	0\\
62.36	0\\
62.37	0\\
62.38	0\\
62.39	0\\
62.4	0\\
62.41	0\\
62.42	0\\
62.43	0\\
62.44	0\\
62.45	0\\
62.46	0\\
62.47	0\\
62.48	0\\
62.49	0\\
62.5	0\\
62.51	0\\
62.52	0\\
62.53	0\\
62.54	0\\
62.55	0\\
62.56	0\\
62.57	0\\
62.58	0\\
62.59	0\\
62.6	0\\
62.61	0\\
62.62	0\\
62.63	0\\
62.64	0\\
62.65	0\\
62.66	0\\
62.67	0\\
62.68	0\\
62.69	0\\
62.7	0\\
62.71	0\\
62.72	0\\
62.73	0\\
62.74	0\\
62.75	0\\
62.76	0\\
62.77	0\\
62.78	0\\
62.79	0\\
62.8	0\\
62.81	0\\
62.82	0\\
62.83	0\\
62.84	0\\
62.85	0\\
62.86	0\\
62.87	0\\
62.88	0\\
62.89	0\\
62.9	0\\
62.91	0\\
62.92	0\\
62.93	0\\
62.94	0\\
62.95	0\\
62.96	0\\
62.97	0\\
62.98	0\\
62.99	0\\
63	0\\
63.01	0\\
63.02	0\\
63.03	0\\
63.04	0\\
63.05	0\\
63.06	0\\
63.07	0\\
63.08	0\\
63.09	0\\
63.1	0\\
63.11	0\\
63.12	0\\
63.13	0\\
63.14	0\\
63.15	0\\
63.16	0\\
63.17	0\\
63.18	0\\
63.19	0\\
63.2	0\\
63.21	0\\
63.22	0\\
63.23	0\\
63.24	0\\
63.25	0\\
63.26	0\\
63.27	0\\
63.28	0\\
63.29	0\\
63.3	0\\
63.31	0\\
63.32	0\\
63.33	0\\
63.34	0\\
63.35	0\\
63.36	0\\
63.37	0\\
63.38	0\\
63.39	0\\
63.4	0\\
63.41	0\\
63.42	0\\
63.43	0\\
63.44	0\\
63.45	0\\
63.46	0\\
63.47	0\\
63.48	0\\
63.49	0\\
63.5	0\\
63.51	0\\
63.52	0\\
63.53	0\\
63.54	0\\
63.55	0\\
63.56	0\\
63.57	0\\
63.58	0\\
63.59	0\\
63.6	0\\
63.61	0\\
63.62	0\\
63.63	0\\
63.64	0\\
63.65	0\\
63.66	0\\
63.67	0\\
63.68	0\\
63.69	0\\
63.7	0\\
63.71	0\\
63.72	0\\
63.73	0\\
63.74	0\\
63.75	0\\
63.76	0\\
63.77	0\\
63.78	0\\
63.79	0\\
63.8	0\\
63.81	0\\
63.82	0\\
63.83	0\\
63.84	0\\
63.85	0\\
63.86	0\\
63.87	0\\
63.88	0\\
63.89	0\\
63.9	0\\
63.91	0\\
63.92	0\\
63.93	0\\
63.94	0\\
63.95	0\\
63.96	0\\
63.97	0\\
63.98	0\\
63.99	0\\
64	0\\
64.01	0\\
64.02	0\\
64.03	0\\
64.04	0\\
64.05	0\\
64.06	0\\
64.07	0\\
64.08	0\\
64.09	0\\
64.1	0\\
64.11	0\\
64.12	0\\
64.13	0\\
64.14	0\\
64.15	0\\
64.16	0\\
64.17	0\\
64.18	0\\
64.19	0\\
64.2	0\\
64.21	0\\
64.22	0\\
64.23	0\\
64.24	0\\
64.25	0\\
64.26	0\\
64.27	0\\
64.28	0\\
64.29	0\\
64.3	0\\
64.31	0\\
64.32	0\\
64.33	0\\
64.34	0\\
64.35	0\\
64.36	0\\
64.37	0\\
64.38	0\\
64.39	0\\
64.4	0\\
64.41	0\\
64.42	0\\
64.43	0\\
64.44	0\\
64.45	0\\
64.46	0\\
64.47	0\\
64.48	0\\
64.49	0\\
64.5	0\\
64.51	0\\
64.52	0\\
64.53	0\\
64.54	0\\
64.55	0\\
64.56	0\\
64.57	0\\
64.58	0\\
64.59	0\\
64.6	0\\
64.61	0\\
64.62	0\\
64.63	0\\
64.64	0\\
64.65	0\\
64.66	0\\
64.67	0\\
64.68	0\\
64.69	0\\
64.7	0\\
64.71	0\\
64.72	0\\
64.73	0\\
64.74	0\\
64.75	0\\
64.76	0\\
64.77	0\\
64.78	0\\
64.79	0\\
64.8	0\\
64.81	0\\
64.82	0\\
64.83	0\\
64.84	0\\
64.85	0\\
64.86	0\\
64.87	0\\
64.88	0\\
64.89	0\\
64.9	0\\
64.91	0\\
64.92	0\\
64.93	0\\
64.94	0\\
64.95	0\\
64.96	0\\
64.97	0\\
64.98	0\\
64.99	0\\
65	0\\
65.01	0\\
65.02	0\\
65.03	0\\
65.04	0\\
65.05	0\\
65.06	0\\
65.07	0\\
65.08	0\\
65.09	0\\
65.1	0\\
65.11	0\\
65.12	0\\
65.13	0\\
65.14	0\\
65.15	0\\
65.16	0\\
65.17	0\\
65.18	0\\
65.19	0\\
65.2	0\\
65.21	0\\
65.22	0\\
65.23	0\\
65.24	0\\
65.25	0\\
65.26	0\\
65.27	0\\
65.28	0\\
65.29	0\\
65.3	0\\
65.31	0\\
65.32	0\\
65.33	0\\
65.34	0\\
65.35	0\\
65.36	0\\
65.37	0\\
65.38	0\\
65.39	0\\
65.4	0\\
65.41	0\\
65.42	0\\
65.43	0\\
65.44	0\\
65.45	0\\
65.46	0\\
65.47	0\\
65.48	0\\
65.49	0\\
65.5	0\\
65.51	0\\
65.52	0\\
65.53	0\\
65.54	0\\
65.55	0\\
65.56	0\\
65.57	0\\
65.58	0\\
65.59	0\\
65.6	0\\
65.61	0\\
65.62	0\\
65.63	0\\
65.64	0\\
65.65	0\\
65.66	0\\
65.67	0\\
65.68	0\\
65.69	0\\
65.7	0\\
65.71	0\\
65.72	0\\
65.73	0\\
65.74	0\\
65.75	0\\
65.76	0\\
65.77	0\\
65.78	0\\
65.79	0\\
65.8	0\\
65.81	0\\
65.82	0\\
65.83	0\\
65.84	0\\
65.85	0\\
65.86	0\\
65.87	0\\
65.88	0\\
65.89	0\\
65.9	0\\
65.91	0\\
65.92	0\\
65.93	0\\
65.94	0\\
65.95	0\\
65.96	0\\
65.97	0\\
65.98	0\\
65.99	0\\
66	0\\
66.01	0\\
66.02	0\\
66.03	0\\
66.04	0\\
66.05	0\\
66.06	0\\
66.07	0\\
66.08	0\\
66.09	0\\
66.1	0\\
66.11	0\\
66.12	0\\
66.13	0\\
66.14	0\\
66.15	0\\
66.16	0\\
66.17	0\\
66.18	0\\
66.19	0\\
66.2	0\\
66.21	0\\
66.22	0\\
66.23	0\\
66.24	0\\
66.25	0\\
66.26	0\\
66.27	0\\
66.28	0\\
66.29	0\\
66.3	0\\
66.31	0\\
66.32	0\\
66.33	0\\
66.34	0\\
66.35	0\\
66.36	0\\
66.37	0\\
66.38	0\\
66.39	0\\
66.4	0\\
66.41	0\\
66.42	0\\
66.43	0\\
66.44	0\\
66.45	0\\
66.46	0\\
66.47	0\\
66.48	0\\
66.49	0\\
66.5	0\\
66.51	0\\
66.52	0\\
66.53	0\\
66.54	0\\
66.55	0\\
66.56	0\\
66.57	0\\
66.58	0\\
66.59	0\\
66.6	0\\
66.61	0\\
66.62	0\\
66.63	0\\
66.64	0\\
66.65	0\\
66.66	0\\
66.67	0\\
66.68	0\\
66.69	0\\
66.7	0\\
66.71	0\\
66.72	0\\
66.73	0\\
66.74	0\\
66.75	0\\
66.76	0\\
66.77	0\\
66.78	0\\
66.79	0\\
66.8	0\\
66.81	0\\
66.82	0\\
66.83	0\\
66.84	0\\
66.85	0\\
66.86	0\\
66.87	0\\
66.88	0\\
66.89	0\\
66.9	0\\
66.91	0\\
66.92	0\\
66.93	0\\
66.94	0\\
66.95	0\\
66.96	0\\
66.97	0\\
66.98	0\\
66.99	0\\
67	0\\
67.01	0\\
67.02	0\\
67.03	0\\
67.04	0\\
67.05	0\\
67.06	0\\
67.07	0\\
67.08	0\\
67.09	0\\
67.1	0\\
67.11	0\\
67.12	0\\
67.13	0\\
67.14	0\\
67.15	0\\
67.16	0\\
67.17	0\\
67.18	0\\
67.19	0\\
67.2	0\\
67.21	0\\
67.22	0\\
67.23	0\\
67.24	0\\
67.25	0\\
67.26	0\\
67.27	0\\
67.28	0\\
67.29	0\\
67.3	0\\
67.31	0\\
67.32	0\\
67.33	0\\
67.34	0\\
67.35	0\\
67.36	0\\
67.37	0\\
67.38	0\\
67.39	0\\
67.4	0\\
67.41	0\\
67.42	0\\
67.43	0\\
67.44	0\\
67.45	0\\
67.46	0\\
67.47	0\\
67.48	0\\
67.49	0\\
67.5	0\\
67.51	0\\
67.52	0\\
67.53	0\\
67.54	0\\
67.55	0\\
67.56	0\\
67.57	0\\
67.58	0\\
67.59	0\\
67.6	0\\
67.61	0\\
67.62	0\\
67.63	0\\
67.64	0\\
67.65	0\\
67.66	0\\
67.67	0\\
67.68	0\\
67.69	0\\
67.7	0\\
67.71	0\\
67.72	0\\
67.73	0\\
67.74	0\\
67.75	0\\
67.76	0\\
67.77	0\\
67.78	0\\
67.79	0\\
67.8	0\\
67.81	0\\
67.82	0\\
67.83	0\\
67.84	0\\
67.85	0\\
67.86	0\\
67.87	0\\
67.88	0\\
67.89	0\\
67.9	0\\
67.91	0\\
67.92	0\\
67.93	0\\
67.94	0\\
67.95	0\\
67.96	0\\
67.97	0\\
67.98	0\\
67.99	0\\
68	0\\
68.01	0\\
68.02	0\\
68.03	0\\
68.04	0\\
68.05	0\\
68.06	0\\
68.07	0\\
68.08	0\\
68.09	0\\
68.1	0\\
68.11	0\\
68.12	0\\
68.13	0\\
68.14	0\\
68.15	0\\
68.16	0\\
68.17	0\\
68.18	0\\
68.19	0\\
68.2	0\\
68.21	0\\
68.22	0\\
68.23	0\\
68.24	0\\
68.25	0\\
68.26	0\\
68.27	0\\
68.28	0\\
68.29	0\\
68.3	0\\
68.31	0\\
68.32	0\\
68.33	0\\
68.34	0\\
68.35	0\\
68.36	0\\
68.37	0\\
68.38	0\\
68.39	0\\
68.4	0\\
68.41	0\\
68.42	0\\
68.43	0\\
68.44	0\\
68.45	0\\
68.46	0\\
68.47	0\\
68.48	0\\
68.49	0\\
68.5	0\\
68.51	0\\
68.52	0\\
68.53	0\\
68.54	0\\
68.55	0\\
68.56	0\\
68.57	0\\
68.58	0\\
68.59	0\\
68.6	0\\
68.61	0\\
68.62	0\\
68.63	0\\
68.64	0\\
68.65	0\\
68.66	0\\
68.67	0\\
68.68	0\\
68.69	0\\
68.7	0\\
68.71	0\\
68.72	0\\
68.73	0\\
68.74	0\\
68.75	0\\
68.76	0\\
68.77	0\\
68.78	0\\
68.79	0\\
68.8	0\\
68.81	0\\
68.82	0\\
68.83	0\\
68.84	0\\
68.85	0\\
68.86	0\\
68.87	0\\
68.88	0\\
68.89	0\\
68.9	0\\
68.91	0\\
68.92	0\\
68.93	0\\
68.94	0\\
68.95	0\\
68.96	0\\
68.97	0\\
68.98	0\\
68.99	0\\
69	0\\
69.01	0\\
69.02	0\\
69.03	0\\
69.04	0\\
69.05	0\\
69.06	0\\
69.07	0\\
69.08	0\\
69.09	0\\
69.1	0\\
69.11	0\\
69.12	0\\
69.13	0\\
69.14	0\\
69.15	0\\
69.16	0\\
69.17	0\\
69.18	0\\
69.19	0\\
69.2	0\\
69.21	0\\
69.22	0\\
69.23	0\\
69.24	0\\
69.25	0\\
69.26	0\\
69.27	0\\
69.28	0\\
69.29	0\\
69.3	0\\
69.31	0\\
69.32	0\\
69.33	0\\
69.34	0\\
69.35	0\\
69.36	0\\
69.37	0\\
69.38	0\\
69.39	0\\
69.4	0\\
69.41	0\\
69.42	0\\
69.43	0\\
69.44	0\\
69.45	0\\
69.46	0\\
69.47	0\\
69.48	0\\
69.49	0\\
69.5	0\\
69.51	0\\
69.52	0\\
69.53	0\\
69.54	0\\
69.55	0\\
69.56	0\\
69.57	0\\
69.58	0\\
69.59	0\\
69.6	0\\
69.61	0\\
69.62	0\\
69.63	0\\
69.64	0\\
69.65	0\\
69.66	0\\
69.67	0\\
69.68	0\\
69.69	0\\
69.7	0\\
69.71	0\\
69.72	0\\
69.73	0\\
69.74	0\\
69.75	0\\
69.76	0\\
69.77	0\\
69.78	0\\
69.79	0\\
69.8	0\\
69.81	0\\
69.82	0\\
69.83	0\\
69.84	0\\
69.85	0\\
69.86	0\\
69.87	0\\
69.88	0\\
69.89	0\\
69.9	0\\
69.91	0\\
69.92	0\\
69.93	0\\
69.94	0\\
69.95	0\\
69.96	0\\
69.97	0\\
69.98	0\\
69.99	0\\
70	0\\
70.01	0\\
70.02	0\\
70.03	0\\
70.04	0\\
70.05	0\\
70.06	0\\
70.07	0\\
70.08	0\\
70.09	0\\
70.1	0\\
70.11	0\\
70.12	0\\
70.13	0\\
70.14	0\\
70.15	0\\
70.16	0\\
70.17	0\\
70.18	0\\
70.19	0\\
70.2	0\\
70.21	0\\
70.22	0\\
70.23	0\\
70.24	0\\
70.25	0\\
70.26	0\\
70.27	0\\
70.28	0\\
70.29	0\\
70.3	0\\
70.31	0\\
70.32	0\\
70.33	0\\
70.34	0\\
70.35	0\\
70.36	0\\
70.37	0\\
70.38	0\\
70.39	0\\
70.4	0\\
70.41	0\\
70.42	0\\
70.43	0\\
70.44	0\\
70.45	0\\
70.46	0\\
70.47	0\\
70.48	0\\
70.49	0\\
70.5	0\\
70.51	0\\
70.52	0\\
70.53	0\\
70.54	0\\
70.55	0\\
70.56	0\\
70.57	0\\
70.58	0\\
70.59	0\\
70.6	0\\
70.61	0\\
70.62	0\\
70.63	0\\
70.64	0\\
70.65	0\\
70.66	0\\
70.67	0\\
70.68	0\\
70.69	0\\
70.7	0\\
70.71	0\\
70.72	0\\
70.73	0\\
70.74	0\\
70.75	0\\
70.76	0\\
70.77	0\\
70.78	0\\
70.79	0\\
70.8	0\\
70.81	0\\
70.82	0\\
70.83	0\\
70.84	0\\
70.85	0\\
70.86	0\\
70.87	0\\
70.88	0\\
70.89	0\\
70.9	0\\
70.91	0\\
70.92	0\\
70.93	0\\
70.94	0\\
70.95	0\\
70.96	0\\
70.97	0\\
70.98	0\\
70.99	0\\
71	0\\
71.01	0\\
71.02	0\\
71.03	0\\
71.04	0\\
71.05	0\\
71.06	0\\
71.07	0\\
71.08	0\\
71.09	0\\
71.1	0\\
71.11	0\\
71.12	0\\
71.13	0\\
71.14	0\\
71.15	0\\
71.16	0\\
71.17	0\\
71.18	0\\
71.19	0\\
71.2	0\\
71.21	0\\
71.22	0\\
71.23	0\\
71.24	0\\
71.25	0\\
71.26	0\\
71.27	0\\
71.28	0\\
71.29	0\\
71.3	0\\
71.31	0\\
71.32	0\\
71.33	0\\
71.34	0\\
71.35	0\\
71.36	0\\
71.37	0\\
71.38	0\\
71.39	0\\
71.4	0\\
71.41	0\\
71.42	0\\
71.43	0\\
71.44	0\\
71.45	0\\
71.46	0\\
71.47	0\\
71.48	0\\
71.49	0\\
71.5	0\\
71.51	0\\
71.52	0\\
71.53	0\\
71.54	0\\
71.55	0\\
71.56	0\\
71.57	0\\
71.58	0\\
71.59	0\\
71.6	0\\
71.61	0\\
71.62	0\\
71.63	0\\
71.64	0\\
71.65	0\\
71.66	0\\
71.67	0\\
71.68	0\\
71.69	0\\
71.7	0\\
71.71	0\\
71.72	0\\
71.73	0\\
71.74	0\\
71.75	0\\
71.76	0\\
71.77	0\\
71.78	0\\
71.79	0\\
71.8	0\\
71.81	0\\
71.82	0\\
71.83	0\\
71.84	0\\
71.85	0\\
71.86	0\\
71.87	0\\
71.88	0\\
71.89	0\\
71.9	0\\
71.91	0\\
71.92	0\\
71.93	0\\
71.94	0\\
71.95	0\\
71.96	0\\
71.97	0\\
71.98	0\\
71.99	0\\
72	0\\
72.01	0\\
72.02	0\\
72.03	0\\
72.04	0\\
72.05	0\\
72.06	0\\
72.07	0\\
72.08	0\\
72.09	0\\
72.1	0\\
72.11	0\\
72.12	0\\
72.13	0\\
72.14	0\\
72.15	0\\
72.16	0\\
72.17	0\\
72.18	0\\
72.19	0\\
72.2	0\\
72.21	0\\
72.22	0\\
72.23	0\\
72.24	0\\
72.25	0\\
72.26	0\\
72.27	0\\
72.28	0\\
72.29	0\\
72.3	0\\
72.31	0\\
72.32	0\\
72.33	0\\
72.34	0\\
72.35	0\\
72.36	0\\
72.37	0\\
72.38	0\\
72.39	0\\
72.4	0\\
72.41	0\\
72.42	0\\
72.43	0\\
72.44	0\\
72.45	0\\
72.46	0\\
72.47	0\\
72.48	0\\
72.49	0\\
72.5	0\\
72.51	0\\
72.52	0\\
72.53	0\\
72.54	0\\
72.55	0\\
72.56	0\\
72.57	0\\
72.58	0\\
72.59	0\\
72.6	0\\
72.61	0\\
72.62	0\\
72.63	0\\
72.64	0\\
72.65	0\\
72.66	0\\
72.67	0\\
72.68	0\\
72.69	0\\
72.7	0\\
72.71	0\\
72.72	0\\
72.73	0\\
72.74	0\\
72.75	0\\
72.76	0\\
72.77	0\\
72.78	0\\
72.79	0\\
72.8	0\\
72.81	0\\
72.82	0\\
72.83	0\\
72.84	0\\
72.85	0\\
72.86	0\\
72.87	0\\
72.88	0\\
72.89	0\\
72.9	0\\
72.91	0\\
72.92	0\\
72.93	0\\
72.94	0\\
72.95	0\\
72.96	0\\
72.97	0\\
72.98	0\\
72.99	0\\
73	0\\
73.01	0\\
73.02	0\\
73.03	0\\
73.04	0\\
73.05	0\\
73.06	0\\
73.07	0\\
73.08	0\\
73.09	0\\
73.1	0\\
73.11	0\\
73.12	0\\
73.13	0\\
73.14	0\\
73.15	0\\
73.16	0\\
73.17	0\\
73.18	0\\
73.19	0\\
73.2	0\\
73.21	0\\
73.22	0\\
73.23	0\\
73.24	0\\
73.25	0\\
73.26	0\\
73.27	0\\
73.28	0\\
73.29	0\\
73.3	0\\
73.31	0\\
73.32	0\\
73.33	0\\
73.34	0\\
73.35	0\\
73.36	0\\
73.37	0\\
73.38	0\\
73.39	0\\
73.4	0\\
73.41	0\\
73.42	0\\
73.43	0\\
73.44	0\\
73.45	0\\
73.46	0\\
73.47	0\\
73.48	0\\
73.49	0\\
73.5	0\\
73.51	0\\
73.52	0\\
73.53	0\\
73.54	0\\
73.55	0\\
73.56	0\\
73.57	0\\
73.58	0\\
73.59	0\\
73.6	0\\
73.61	0\\
73.62	0\\
73.63	0\\
73.64	0\\
73.65	0\\
73.66	0\\
73.67	0\\
73.68	0\\
73.69	0\\
73.7	0\\
73.71	0\\
73.72	0\\
73.73	0\\
73.74	0\\
73.75	0\\
73.76	0\\
73.77	0\\
73.78	0\\
73.79	0\\
73.8	0\\
73.81	0\\
73.82	0\\
73.83	0\\
73.84	0\\
73.85	0\\
73.86	0\\
73.87	0\\
73.88	0\\
73.89	0\\
73.9	0\\
73.91	0\\
73.92	0\\
73.93	0\\
73.94	0\\
73.95	0\\
73.96	0\\
73.97	0\\
73.98	0\\
73.99	0\\
74	0\\
74.01	0\\
74.02	0\\
74.03	0\\
74.04	0\\
74.05	0\\
74.06	0\\
74.07	0\\
74.08	0\\
74.09	0\\
74.1	0\\
74.11	0\\
74.12	0\\
74.13	0\\
74.14	0\\
74.15	0\\
74.16	0\\
74.17	0\\
74.18	0\\
74.19	0\\
74.2	0\\
74.21	0\\
74.22	0\\
74.23	0\\
74.24	0\\
74.25	0\\
74.26	0\\
74.27	0\\
74.28	0\\
74.29	0\\
74.3	0\\
74.31	0\\
74.32	0\\
74.33	0\\
74.34	0\\
74.35	0\\
74.36	0\\
74.37	0\\
74.38	0\\
74.39	0\\
74.4	0\\
74.41	0\\
74.42	0\\
74.43	0\\
74.44	0\\
74.45	0\\
74.46	0\\
74.47	0\\
74.48	0\\
74.49	0\\
74.5	0\\
74.51	0\\
74.52	0\\
74.53	0\\
74.54	0\\
74.55	0\\
74.56	0\\
74.57	0\\
74.58	0\\
74.59	0\\
74.6	0\\
74.61	0\\
74.62	0\\
74.63	0\\
74.64	0\\
74.65	0\\
74.66	0\\
74.67	0\\
74.68	0\\
74.69	0\\
74.7	0\\
74.71	0\\
74.72	0\\
74.73	0\\
74.74	0\\
74.75	0\\
74.76	0\\
74.77	0\\
74.78	0\\
74.79	0\\
74.8	0\\
74.81	0\\
74.82	0\\
74.83	0\\
74.84	0\\
74.85	0\\
74.86	0\\
74.87	0\\
74.88	0\\
74.89	0\\
74.9	0\\
74.91	0\\
74.92	0\\
74.93	0\\
74.94	0\\
74.95	0\\
74.96	0\\
74.97	0\\
74.98	0\\
74.99	0\\
75	0\\
75.01	0\\
75.02	0\\
75.03	0\\
75.04	0\\
75.05	0\\
75.06	0\\
75.07	0\\
75.08	0\\
75.09	0\\
75.1	0\\
75.11	0\\
75.12	0\\
75.13	0\\
75.14	0\\
75.15	0\\
75.16	0\\
75.17	0\\
75.18	0\\
75.19	0\\
75.2	0\\
75.21	0\\
75.22	0\\
75.23	0\\
75.24	0\\
75.25	0\\
75.26	0\\
75.27	0\\
75.28	0\\
75.29	0\\
75.3	0\\
75.31	0\\
75.32	0\\
75.33	0\\
75.34	0\\
75.35	0\\
75.36	0\\
75.37	0\\
75.38	0\\
75.39	0\\
75.4	0\\
75.41	0\\
75.42	0\\
75.43	0\\
75.44	0\\
75.45	0\\
75.46	0\\
75.47	0\\
75.48	0\\
75.49	0\\
75.5	0\\
75.51	0\\
75.52	0\\
75.53	0\\
75.54	0\\
75.55	0\\
75.56	0\\
75.57	0\\
75.58	0\\
75.59	0\\
75.6	0\\
75.61	0\\
75.62	0\\
75.63	0\\
75.64	0\\
75.65	0\\
75.66	0\\
75.67	0\\
75.68	0\\
75.69	0\\
75.7	0\\
75.71	0\\
75.72	0\\
75.73	0\\
75.74	0\\
75.75	0\\
75.76	0\\
75.77	0\\
75.78	0\\
75.79	0\\
75.8	0\\
75.81	0\\
75.82	0\\
75.83	0\\
75.84	0\\
75.85	0\\
75.86	0\\
75.87	0\\
75.88	0\\
75.89	0\\
75.9	0\\
75.91	0\\
75.92	0\\
75.93	0\\
75.94	0\\
75.95	0\\
75.96	0\\
75.97	0\\
75.98	0\\
75.99	0\\
76	0\\
76.01	0\\
76.02	0\\
76.03	0\\
76.04	0\\
76.05	0\\
76.06	0\\
76.07	0\\
76.08	0\\
76.09	0\\
76.1	0\\
76.11	0\\
76.12	0\\
76.13	0\\
76.14	0\\
76.15	0\\
76.16	0\\
76.17	0\\
76.18	0\\
76.19	0\\
76.2	0\\
76.21	0\\
76.22	0\\
76.23	0\\
76.24	0\\
76.25	0\\
76.26	0\\
76.27	0\\
76.28	0\\
76.29	0\\
76.3	0\\
76.31	0\\
76.32	0\\
76.33	0\\
76.34	0\\
76.35	0\\
76.36	0\\
76.37	0\\
76.38	0\\
76.39	0\\
76.4	0\\
76.41	0\\
76.42	0\\
76.43	0\\
76.44	0\\
76.45	0\\
76.46	0\\
76.47	0\\
76.48	0\\
76.49	0\\
76.5	0\\
76.51	0\\
76.52	0\\
76.53	0\\
76.54	0\\
76.55	0\\
76.56	0\\
76.57	0\\
76.58	0\\
76.59	0\\
76.6	0\\
76.61	0\\
76.62	0\\
76.63	0\\
76.64	0\\
76.65	0\\
76.66	0\\
76.67	0\\
76.68	0\\
76.69	0\\
76.7	0\\
76.71	0\\
76.72	0\\
76.73	0\\
76.74	0\\
76.75	0\\
76.76	0\\
76.77	0\\
76.78	0\\
76.79	0\\
76.8	0\\
76.81	0\\
76.82	0\\
76.83	0\\
76.84	0\\
76.85	0\\
76.86	0\\
76.87	0\\
76.88	0\\
76.89	0\\
76.9	0\\
76.91	0\\
76.92	0\\
76.93	0\\
76.94	0\\
76.95	0\\
76.96	0\\
76.97	0\\
76.98	0\\
76.99	0\\
77	0\\
77.01	0\\
77.02	0\\
77.03	0\\
77.04	0\\
77.05	0\\
77.06	0\\
77.07	0\\
77.08	0\\
77.09	0\\
77.1	0\\
77.11	0\\
77.12	0\\
77.13	0\\
77.14	0\\
77.15	0\\
77.16	0\\
77.17	0\\
77.18	0\\
77.19	0\\
77.2	0\\
77.21	0\\
77.22	0\\
77.23	0\\
77.24	0\\
77.25	0\\
77.26	0\\
77.27	0\\
77.28	0\\
77.29	0\\
77.3	0\\
77.31	0\\
77.32	0\\
77.33	0\\
77.34	0\\
77.35	0\\
77.36	0\\
77.37	0\\
77.38	0\\
77.39	0\\
77.4	0\\
77.41	0\\
77.42	0\\
77.43	0\\
77.44	0\\
77.45	0\\
77.46	0\\
77.47	0\\
77.48	0\\
77.49	0\\
77.5	0\\
77.51	0\\
77.52	0\\
77.53	0\\
77.54	0\\
77.55	0\\
77.56	0\\
77.57	0\\
77.58	0\\
77.59	0\\
77.6	0\\
77.61	0\\
77.62	0\\
77.63	0\\
77.64	0\\
77.65	0\\
77.66	0\\
77.67	0\\
77.68	0\\
77.69	0\\
77.7	0\\
77.71	0\\
77.72	0\\
77.73	0\\
77.74	0\\
77.75	0\\
77.76	0\\
77.77	0\\
77.78	0\\
77.79	0\\
77.8	0\\
77.81	0\\
77.82	0\\
77.83	0\\
77.84	0\\
77.85	0\\
77.86	0\\
77.87	0\\
77.88	0\\
77.89	0\\
77.9	0\\
77.91	0\\
77.92	0\\
77.93	0\\
77.94	0\\
77.95	0\\
77.96	0\\
77.97	0\\
77.98	0\\
77.99	0\\
78	0\\
78.01	0\\
78.02	0\\
78.03	0\\
78.04	0\\
78.05	0\\
78.06	0\\
78.07	0\\
78.08	0\\
78.09	0\\
78.1	0\\
78.11	0\\
78.12	0\\
78.13	0\\
78.14	0\\
78.15	0\\
78.16	0\\
78.17	0\\
78.18	0\\
78.19	0\\
78.2	0\\
78.21	0\\
78.22	0\\
78.23	0\\
78.24	0\\
78.25	0\\
78.26	0\\
78.27	0\\
78.28	0\\
78.29	0\\
78.3	0\\
78.31	0\\
78.32	0\\
78.33	0\\
78.34	0\\
78.35	0\\
78.36	0\\
78.37	0\\
78.38	0\\
78.39	0\\
78.4	0\\
78.41	0\\
78.42	0\\
78.43	0\\
78.44	0\\
78.45	0\\
78.46	0\\
78.47	0\\
78.48	0\\
78.49	0\\
78.5	0\\
78.51	0\\
78.52	0\\
78.53	0\\
78.54	0\\
78.55	0\\
78.56	0\\
78.57	0\\
78.58	0\\
78.59	0\\
78.6	0\\
78.61	0\\
78.62	0\\
78.63	0\\
78.64	0\\
78.65	0\\
78.66	0\\
78.67	0\\
78.68	0\\
78.69	0\\
78.7	0\\
78.71	0\\
78.72	0\\
78.73	0\\
78.74	0\\
78.75	0\\
78.76	0\\
78.77	0\\
78.78	0\\
78.79	0\\
78.8	0\\
78.81	0\\
78.82	0\\
78.83	0\\
78.84	0\\
78.85	0\\
78.86	0\\
78.87	0\\
78.88	0\\
78.89	0\\
78.9	0\\
78.91	0\\
78.92	0\\
78.93	0\\
78.94	0\\
78.95	0\\
78.96	0\\
78.97	0\\
78.98	0\\
78.99	0\\
79	0\\
79.01	0\\
79.02	0\\
79.03	0\\
79.04	0\\
79.05	0\\
79.06	0\\
79.07	0\\
79.08	0\\
79.09	0\\
79.1	0\\
79.11	0\\
79.12	0\\
79.13	0\\
79.14	0\\
79.15	0\\
79.16	0\\
79.17	0\\
79.18	0\\
79.19	0\\
79.2	0\\
79.21	0\\
79.22	0\\
79.23	0\\
79.24	0\\
79.25	0\\
79.26	0\\
79.27	0\\
79.28	0\\
79.29	0\\
79.3	0\\
79.31	0\\
79.32	0\\
79.33	0\\
79.34	0\\
79.35	0\\
79.36	0\\
79.37	0\\
79.38	0\\
79.39	0\\
79.4	0\\
79.41	0\\
79.42	0\\
79.43	0\\
79.44	0\\
79.45	0\\
79.46	0\\
79.47	0\\
79.48	0\\
79.49	0\\
79.5	0\\
79.51	0\\
79.52	0\\
79.53	0\\
79.54	0\\
79.55	0\\
79.56	0\\
79.57	0\\
79.58	0\\
79.59	0\\
79.6	0\\
79.61	0\\
79.62	0\\
79.63	0\\
79.64	0\\
79.65	0\\
79.66	0\\
79.67	0\\
79.68	0\\
79.69	0\\
79.7	0\\
79.71	0\\
79.72	0\\
79.73	0\\
79.74	0\\
79.75	0\\
79.76	0\\
79.77	0\\
79.78	0\\
79.79	0\\
79.8	0\\
79.81	0\\
79.82	0\\
79.83	0\\
79.84	0\\
79.85	0\\
79.86	0\\
79.87	0\\
79.88	0\\
79.89	0\\
79.9	0\\
79.91	0\\
79.92	0\\
79.93	0\\
79.94	0\\
79.95	0\\
79.96	0\\
79.97	0\\
79.98	0\\
79.99	0\\
80	0\\
80.01	0\\
};
\addplot [color=black,solid]
  table[row sep=crcr]{%
80.01	0\\
80.02	0\\
80.03	0\\
80.04	0\\
80.05	0\\
80.06	0\\
80.07	0\\
80.08	0\\
80.09	0\\
80.1	0\\
80.11	0\\
80.12	0\\
80.13	0\\
80.14	0\\
80.15	0\\
80.16	0\\
80.17	0\\
80.18	0\\
80.19	0\\
80.2	0\\
80.21	0\\
80.22	0\\
80.23	0\\
80.24	0\\
80.25	0\\
80.26	0\\
80.27	0\\
80.28	0\\
80.29	0\\
80.3	0\\
80.31	0\\
80.32	0\\
80.33	0\\
80.34	0\\
80.35	0\\
80.36	0\\
80.37	0\\
80.38	0\\
80.39	0\\
80.4	0\\
80.41	0\\
80.42	0\\
80.43	0\\
80.44	0\\
80.45	0\\
80.46	0\\
80.47	0\\
80.48	0\\
80.49	0\\
80.5	0\\
80.51	0\\
80.52	0\\
80.53	0\\
80.54	0\\
80.55	0\\
80.56	0\\
80.57	0\\
80.58	0\\
80.59	0\\
80.6	0\\
80.61	0\\
80.62	0\\
80.63	0\\
80.64	0\\
80.65	0\\
80.66	0\\
80.67	0\\
80.68	0\\
80.69	0\\
80.7	0\\
80.71	0\\
80.72	0\\
80.73	0\\
80.74	0\\
80.75	0\\
80.76	0\\
80.77	0\\
80.78	0\\
80.79	0\\
80.8	0\\
80.81	0\\
80.82	0\\
80.83	0\\
80.84	0\\
80.85	0\\
80.86	0\\
80.87	0\\
80.88	0\\
80.89	0\\
80.9	0\\
80.91	0\\
80.92	0\\
80.93	0\\
80.94	0\\
80.95	0\\
80.96	0\\
80.97	0\\
80.98	0\\
80.99	0\\
81	0\\
81.01	0\\
81.02	0\\
81.03	0\\
81.04	0\\
81.05	0\\
81.06	0\\
81.07	0\\
81.08	0\\
81.09	0\\
81.1	0\\
81.11	0\\
81.12	0\\
81.13	0\\
81.14	0\\
81.15	0\\
81.16	0\\
81.17	0\\
81.18	0\\
81.19	0\\
81.2	0\\
81.21	0\\
81.22	0\\
81.23	0\\
81.24	0\\
81.25	0\\
81.26	0\\
81.27	0\\
81.28	0\\
81.29	0\\
81.3	0\\
81.31	0\\
81.32	0\\
81.33	0\\
81.34	0\\
81.35	0\\
81.36	0\\
81.37	0\\
81.38	0\\
81.39	0\\
81.4	0\\
81.41	0\\
81.42	0\\
81.43	0\\
81.44	0\\
81.45	0\\
81.46	0\\
81.47	0\\
81.48	0\\
81.49	0\\
81.5	0\\
81.51	0\\
81.52	0\\
81.53	0\\
81.54	0\\
81.55	0\\
81.56	0\\
81.57	0\\
81.58	0\\
81.59	0\\
81.6	0\\
81.61	0\\
81.62	0\\
81.63	0\\
81.64	0\\
81.65	0\\
81.66	0\\
81.67	0\\
81.68	0\\
81.69	0\\
81.7	0\\
81.71	0\\
81.72	0\\
81.73	0\\
81.74	0\\
81.75	0\\
81.76	0\\
81.77	0\\
81.78	0\\
81.79	0\\
81.8	0\\
81.81	0\\
81.82	0\\
81.83	0\\
81.84	0\\
81.85	0\\
81.86	0\\
81.87	0\\
81.88	0\\
81.89	0\\
81.9	0\\
81.91	0\\
81.92	0\\
81.93	0\\
81.94	0\\
81.95	0\\
81.96	0\\
81.97	0\\
81.98	0\\
81.99	0\\
82	0\\
82.01	0\\
82.02	0\\
82.03	0\\
82.04	0\\
82.05	0\\
82.06	0\\
82.07	0\\
82.08	0\\
82.09	0\\
82.1	0\\
82.11	0\\
82.12	0\\
82.13	0\\
82.14	0\\
82.15	0\\
82.16	0\\
82.17	0\\
82.18	0\\
82.19	0\\
82.2	0\\
82.21	0\\
82.22	0\\
82.23	0\\
82.24	0\\
82.25	0\\
82.26	0\\
82.27	0\\
82.28	0\\
82.29	0\\
82.3	0\\
82.31	0\\
82.32	0\\
82.33	0\\
82.34	0\\
82.35	0\\
82.36	0\\
82.37	0\\
82.38	0\\
82.39	0\\
82.4	0\\
82.41	0\\
82.42	0\\
82.43	0\\
82.44	0\\
82.45	0\\
82.46	0\\
82.47	0\\
82.48	0\\
82.49	0\\
82.5	0\\
82.51	0\\
82.52	0\\
82.53	0\\
82.54	0\\
82.55	0\\
82.56	0\\
82.57	0\\
82.58	0\\
82.59	0\\
82.6	0\\
82.61	0\\
82.62	0\\
82.63	0\\
82.64	0\\
82.65	0\\
82.66	0\\
82.67	0\\
82.68	0\\
82.69	0\\
82.7	0\\
82.71	0\\
82.72	0\\
82.73	0\\
82.74	0\\
82.75	0\\
82.76	0\\
82.77	0\\
82.78	0\\
82.79	0\\
82.8	0\\
82.81	0\\
82.82	0\\
82.83	0\\
82.84	0\\
82.85	0\\
82.86	0\\
82.87	0\\
82.88	0\\
82.89	0\\
82.9	0\\
82.91	0\\
82.92	0\\
82.93	0\\
82.94	0\\
82.95	0\\
82.96	0\\
82.97	0\\
82.98	0\\
82.99	0\\
83	0\\
83.01	0\\
83.02	0\\
83.03	0\\
83.04	0\\
83.05	0\\
83.06	0\\
83.07	0\\
83.08	0\\
83.09	0\\
83.1	0\\
83.11	0\\
83.12	0\\
83.13	0\\
83.14	0\\
83.15	0\\
83.16	0\\
83.17	0\\
83.18	0\\
83.19	0\\
83.2	0\\
83.21	0\\
83.22	0\\
83.23	0\\
83.24	0\\
83.25	0\\
83.26	0\\
83.27	0\\
83.28	0\\
83.29	0\\
83.3	0\\
83.31	0\\
83.32	0\\
83.33	0\\
83.34	0\\
83.35	0\\
83.36	0\\
83.37	0\\
83.38	0\\
83.39	0\\
83.4	0\\
83.41	0\\
83.42	0\\
83.43	0\\
83.44	0\\
83.45	0\\
83.46	0\\
83.47	0\\
83.48	0\\
83.49	0\\
83.5	0\\
83.51	0\\
83.52	0\\
83.53	0\\
83.54	0\\
83.55	0\\
83.56	0\\
83.57	0\\
83.58	0\\
83.59	0\\
83.6	0\\
83.61	0\\
83.62	0\\
83.63	0\\
83.64	0\\
83.65	0\\
83.66	0\\
83.67	0\\
83.68	0\\
83.69	0\\
83.7	0\\
83.71	0\\
83.72	0\\
83.73	0\\
83.74	0\\
83.75	0\\
83.76	0\\
83.77	0\\
83.78	0\\
83.79	0\\
83.8	0\\
83.81	0\\
83.82	0\\
83.83	0\\
83.84	0\\
83.85	0\\
83.86	0\\
83.87	0\\
83.88	0\\
83.89	0\\
83.9	0\\
83.91	0\\
83.92	0\\
83.93	0\\
83.94	0\\
83.95	0\\
83.96	0\\
83.97	0\\
83.98	0\\
83.99	0\\
84	0\\
84.01	0\\
84.02	0\\
84.03	0\\
84.04	0\\
84.05	0\\
84.06	0\\
84.07	0\\
84.08	0\\
84.09	0\\
84.1	0\\
84.11	0\\
84.12	0\\
84.13	0\\
84.14	0\\
84.15	0\\
84.16	0\\
84.17	0\\
84.18	0\\
84.19	0\\
84.2	0\\
84.21	0\\
84.22	0\\
84.23	0\\
84.24	0\\
84.25	0\\
84.26	0\\
84.27	0\\
84.28	0\\
84.29	0\\
84.3	0\\
84.31	0\\
84.32	0\\
84.33	0\\
84.34	0\\
84.35	0\\
84.36	0\\
84.37	0\\
84.38	0\\
84.39	0\\
84.4	0\\
84.41	0\\
84.42	0\\
84.43	0\\
84.44	0\\
84.45	0\\
84.46	0\\
84.47	0\\
84.48	0\\
84.49	0\\
84.5	0\\
84.51	0\\
84.52	0\\
84.53	0\\
84.54	0\\
84.55	0\\
84.56	0\\
84.57	0\\
84.58	0\\
84.59	0\\
84.6	0\\
84.61	0\\
84.62	0\\
84.63	0\\
84.64	0\\
84.65	0\\
84.66	0\\
84.67	0\\
84.68	0\\
84.69	0\\
84.7	0\\
84.71	0\\
84.72	0\\
84.73	0\\
84.74	0\\
84.75	0\\
84.76	0\\
84.77	0\\
84.78	0\\
84.79	0\\
84.8	0\\
84.81	0\\
84.82	0\\
84.83	0\\
84.84	0\\
84.85	0\\
84.86	0\\
84.87	0\\
84.88	0\\
84.89	0\\
84.9	0\\
84.91	0\\
84.92	0\\
84.93	0\\
84.94	0\\
84.95	0\\
84.96	0\\
84.97	0\\
84.98	0\\
84.99	0\\
85	0\\
85.01	0\\
85.02	0\\
85.03	0\\
85.04	0\\
85.05	0\\
85.06	0\\
85.07	0\\
85.08	0\\
85.09	0\\
85.1	0\\
85.11	0\\
85.12	0\\
85.13	0\\
85.14	0\\
85.15	0\\
85.16	0\\
85.17	0\\
85.18	0\\
85.19	0\\
85.2	0\\
85.21	0\\
85.22	0\\
85.23	0\\
85.24	0\\
85.25	0\\
85.26	0\\
85.27	0\\
85.28	0\\
85.29	0\\
85.3	0\\
85.31	0\\
85.32	0\\
85.33	0\\
85.34	0\\
85.35	0\\
85.36	0\\
85.37	0\\
85.38	0\\
85.39	0\\
85.4	0\\
85.41	0\\
85.42	0\\
85.43	0\\
85.44	0\\
85.45	0\\
85.46	0\\
85.47	0\\
85.48	0\\
85.49	0\\
85.5	0\\
85.51	0\\
85.52	0\\
85.53	0\\
85.54	0\\
85.55	0\\
85.56	0\\
85.57	0\\
85.58	0\\
85.59	0\\
85.6	0\\
85.61	0\\
85.62	0\\
85.63	0\\
85.64	0\\
85.65	0\\
85.66	0\\
85.67	0\\
85.68	0\\
85.69	0\\
85.7	0\\
85.71	0\\
85.72	0\\
85.73	0\\
85.74	0\\
85.75	0\\
85.76	0\\
85.77	0\\
85.78	0\\
85.79	0\\
85.8	0\\
85.81	0\\
85.82	0\\
85.83	0\\
85.84	0\\
85.85	0\\
85.86	0\\
85.87	0\\
85.88	0\\
85.89	0\\
85.9	0\\
85.91	0\\
85.92	0\\
85.93	0\\
85.94	0\\
85.95	0\\
85.96	0\\
85.97	0\\
85.98	0\\
85.99	0\\
86	0\\
86.01	0\\
86.02	0\\
86.03	0\\
86.04	0\\
86.05	0\\
86.06	0\\
86.07	0\\
86.08	0\\
86.09	0\\
86.1	0\\
86.11	0\\
86.12	0\\
86.13	0\\
86.14	0\\
86.15	0\\
86.16	0\\
86.17	0\\
86.18	0\\
86.19	0\\
86.2	0\\
86.21	0\\
86.22	0\\
86.23	0\\
86.24	0\\
86.25	0\\
86.26	0\\
86.27	0\\
86.28	0\\
86.29	0\\
86.3	0\\
86.31	0\\
86.32	0\\
86.33	0\\
86.34	0\\
86.35	0\\
86.36	0\\
86.37	0\\
86.38	0\\
86.39	0\\
86.4	0\\
86.41	0\\
86.42	0\\
86.43	0\\
86.44	0\\
86.45	0\\
86.46	0\\
86.47	0\\
86.48	0\\
86.49	0\\
86.5	0\\
86.51	0\\
86.52	0\\
86.53	0\\
86.54	0\\
86.55	0\\
86.56	0\\
86.57	0\\
86.58	0\\
86.59	0\\
86.6	0\\
86.61	0\\
86.62	0\\
86.63	0\\
86.64	0\\
86.65	0\\
86.66	0\\
86.67	0\\
86.68	0\\
86.69	0\\
86.7	0\\
86.71	0\\
86.72	0\\
86.73	0\\
86.74	0\\
86.75	0\\
86.76	0\\
86.77	0\\
86.78	0\\
86.79	0\\
86.8	0\\
86.81	0\\
86.82	3.27970578975663e-06\\
86.83	7.9811883659086e-06\\
86.84	1.26868625115548e-05\\
86.85	1.73967356682046e-05\\
86.86	2.21108152934513e-05\\
86.87	2.68291088610009e-05\\
86.88	3.15516238606952e-05\\
86.89	3.62783677985336e-05\\
86.9	4.10093481966987e-05\\
86.91	4.5744572593584e-05\\
86.92	5.0484048543813e-05\\
86.93	5.52277836182655e-05\\
86.94	5.9975785404098e-05\\
86.95	6.47280615047718e-05\\
86.96	6.94846195400719e-05\\
86.97	7.42454671461331e-05\\
86.98	7.9010611975459e-05\\
86.99	8.37800616969499e-05\\
87	8.85538239959183e-05\\
87.01	9.33319065741164e-05\\
87.02	9.8114317149756e-05\\
87.03	0.000102901063457532\\
87.04	0.000107692153248637\\
87.05	0.000112487594290792\\
87.06	0.000117287394368261\\
87.07	0.000122091561281873\\
87.08	0.000126900102849044\\
87.09	0.000131713026903795\\
87.1	0.000136530341296773\\
87.11	0.000141352053895271\\
87.12	0.000146178172583247\\
87.13	0.000151008705261342\\
87.14	0.0001558436598469\\
87.15	0.000160683044273989\\
87.16	0.000165526866493413\\
87.17	0.000170375134472734\\
87.18	0.000175227856196291\\
87.19	0.000180085039665216\\
87.2	0.000184946692897448\\
87.21	0.000189812823927755\\
87.22	0.000194683440807747\\
87.23	0.000199558551605892\\
87.24	0.000204438164407532\\
87.25	0.000209322287314902\\
87.26	0.00021421092844714\\
87.27	0.000219104095940302\\
87.28	0.000224001797947381\\
87.29	0.00022890404263832\\
87.3	0.000233810838200019\\
87.31	0.000238722192836355\\
87.32	0.000243638114768192\\
87.33	0.000248558612233397\\
87.34	0.000253483693486848\\
87.35	0.000258413366800447\\
87.36	0.000263347640463129\\
87.37	0.000268286522780881\\
87.38	0.000273230022076741\\
87.39	0.000278178146690818\\
87.4	0.000283130904980295\\
87.41	0.000288088305319439\\
87.42	0.000293050356099615\\
87.43	0.000298017065729286\\
87.44	0.000302988442634029\\
87.45	0.000307964495256535\\
87.46	0.000312945232056619\\
87.47	0.00031793066151123\\
87.48	0.000322920792114451\\
87.49	0.000327915632377506\\
87.5	0.000332915190828766\\
87.51	0.00033791947601375\\
87.52	0.000342928496495136\\
87.53	0.000347942260852755\\
87.54	0.000352960777683599\\
87.55	0.000357984055601824\\
87.56	0.000363012103238747\\
87.57	0.000368044929242852\\
87.58	0.000373082542279787\\
87.59	0.000378124951032363\\
87.6	0.000383172164200555\\
87.61	0.000388224190501503\\
87.62	0.000393281038669506\\
87.63	0.000398342717456018\\
87.64	0.000403409235629648\\
87.65	0.000408480601976159\\
87.66	0.000413556825298454\\
87.67	0.000418637914416578\\
87.68	0.000423723878167713\\
87.69	0.000428814725406164\\
87.7	0.000433910465003354\\
87.71	0.000439011105847827\\
87.72	0.000444116656845218\\
87.73	0.000449227126918264\\
87.74	0.000454342525006779\\
87.75	0.000459462860067653\\
87.76	0.000464588141074833\\
87.77	0.000469718377019313\\
87.78	0.00047485357690912\\
87.79	0.000479993749769304\\
87.8	0.000485138904641913\\
87.81	0.000490289050585988\\
87.82	0.00049544419667754\\
87.83	0.000500604352009531\\
87.84	0.000505769525691859\\
87.85	0.000510939726851341\\
87.86	0.000516114964631685\\
87.87	0.000521295248193474\\
87.88	0.000526480586714144\\
87.89	0.000531670989387957\\
87.9	0.000536866465425983\\
87.91	0.00054206702405607\\
87.92	0.000547272674522819\\
87.93	0.000552483426087559\\
87.94	0.000557699288028317\\
87.95	0.000562920269639786\\
87.96	0.000568146380233302\\
87.97	0.000573377629136807\\
87.98	0.000578614025694821\\
87.99	0.000583855579268405\\
88	0.000589102299235127\\
88.01	0.00059435419498903\\
88.02	0.00059961127594059\\
88.03	0.000604873551516682\\
88.04	0.000610141031160539\\
88.05	0.000615413724331715\\
88.06	0.000620691640506037\\
88.07	0.00062597478917557\\
88.08	0.000631263179848568\\
88.09	0.00063655682204943\\
88.1	0.000641855725318654\\
88.11	0.000647159899212789\\
88.12	0.000652469353304385\\
88.13	0.000657784097181946\\
88.14	0.000663104140449873\\
88.15	0.000668429492728415\\
88.16	0.000673760163653612\\
88.17	0.000679096162877242\\
88.18	0.000684437500066758\\
88.19	0.000689784184905236\\
88.2	0.000695136227091308\\
88.21	0.000700493636339108\\
88.22	0.0007058564223782\\
88.23	0.000711224594953517\\
88.24	0.000716598163825298\\
88.25	0.000721977138769012\\
88.26	0.000727361529575298\\
88.27	0.000732751346049883\\
88.28	0.000738146598013521\\
88.29	0.000743547295301903\\
88.3	0.000748953447765597\\
88.31	0.000754365065269959\\
88.32	0.000759782157695057\\
88.33	0.000765204734935587\\
88.34	0.000770632806900791\\
88.35	0.000776066383514375\\
88.36	0.000781505474714417\\
88.37	0.000786950090453278\\
88.38	0.000792400240697516\\
88.39	0.000797855935427786\\
88.4	0.00080331718463875\\
88.41	0.000808783998338981\\
88.42	0.00081425638655086\\
88.43	0.000819734359310476\\
88.44	0.000825217926667527\\
88.45	0.000830707098685211\\
88.46	0.000836201885440117\\
88.47	0.000841702297022122\\
88.48	0.000847208343534273\\
88.49	0.000852720035092675\\
88.5	0.000858237381826377\\
88.51	0.000863760393877252\\
88.52	0.000869289081399873\\
88.53	0.000874823454561399\\
88.54	0.000880363523541438\\
88.55	0.000885909298531927\\
88.56	0.000891460789736998\\
88.57	0.000897018007372845\\
88.58	0.000902580961667593\\
88.59	0.000908149662861152\\
88.6	0.000913724121205085\\
88.61	0.00091930434696246\\
88.62	0.000924890350407701\\
88.63	0.000930482141826446\\
88.64	0.000936079731515393\\
88.65	0.000941683129782142\\
88.66	0.000947292346945044\\
88.67	0.000952907393333039\\
88.68	0.000958528279285489\\
88.69	0.000964155015152018\\
88.7	0.000969787611292339\\
88.71	0.000975426078076087\\
88.72	0.000981070425882639\\
88.73	0.000986720665100936\\
88.74	0.000992376806129309\\
88.75	0.000998038859375284\\
88.76	0.0010037068352554\\
88.77	0.00100938074419503\\
88.78	0.00101506059662815\\
88.79	0.00102074640299719\\
88.8	0.00102643817375279\\
88.81	0.00103213591935363\\
88.82	0.0010378396502662\\
88.83	0.00104354937696459\\
88.84	0.00104926510993028\\
88.85	0.00105498685965191\\
88.86	0.0010607146366251\\
88.87	0.00106644845135216\\
88.88	0.00107218831434189\\
88.89	0.00107793423610936\\
88.9	0.00108368622717565\\
88.91	0.00108944429806762\\
88.92	0.00109520845931765\\
88.93	0.0011009787214634\\
88.94	0.00110675509504756\\
88.95	0.00111253759061758\\
88.96	0.00111832621872541\\
88.97	0.00112412098992724\\
88.98	0.00112992191478323\\
88.99	0.00113572900385719\\
89	0.00114154226771638\\
89.01	0.00114736171693115\\
89.02	0.00115318736207468\\
89.03	0.00115901921372272\\
89.04	0.0011648572824532\\
89.05	0.00117070157884602\\
89.06	0.00117655211348268\\
89.07	0.00118240889694601\\
89.08	0.00118827193981979\\
89.09	0.0011941412526885\\
89.1	0.00120001684613694\\
89.11	0.00120589873074988\\
89.12	0.00121178691711179\\
89.13	0.00121768141580642\\
89.14	0.00122358223741649\\
89.15	0.00122948939252332\\
89.16	0.00123540289170648\\
89.17	0.0012413227455434\\
89.18	0.00124724896460901\\
89.19	0.00125318029646694\\
89.2	0.00125911453504934\\
89.21	0.00126505168240045\\
89.22	0.00127099174056352\\
89.23	0.00127693471158071\\
89.24	0.00128288059749313\\
89.25	0.00128882940034078\\
89.26	0.00129478112216256\\
89.27	0.00130073576499621\\
89.28	0.00130669333087835\\
89.29	0.00131265382184441\\
89.3	0.00131861723992862\\
89.31	0.00132458358716402\\
89.32	0.00133055286558238\\
89.33	0.00133652507721424\\
89.34	0.00134250022408888\\
89.35	0.00134847830823424\\
89.36	0.00135445933167698\\
89.37	0.0013604432964424\\
89.38	0.00136643020455446\\
89.39	0.00137242005803573\\
89.4	0.00137841285890736\\
89.41	0.00138440860918911\\
89.42	0.00139040731089926\\
89.43	0.00139640896605465\\
89.44	0.00140241357667062\\
89.45	0.00140842114476098\\
89.46	0.00141443167233804\\
89.47	0.00142044516141252\\
89.48	0.00142646161399358\\
89.49	0.00143248103208877\\
89.5	0.00143850341770401\\
89.51	0.00144452877284358\\
89.52	0.00145055709951008\\
89.53	0.00145658839970441\\
89.54	0.00146262267542577\\
89.55	0.00146865992867159\\
89.56	0.00147470016143755\\
89.57	0.00148074337571753\\
89.58	0.0014867895735036\\
89.59	0.00149283875678597\\
89.6	0.00149889092755302\\
89.61	0.0015049460877912\\
89.62	0.00151100423948508\\
89.63	0.00151706538461725\\
89.64	0.00152312952516837\\
89.65	0.00152919666311709\\
89.66	0.00153526680044005\\
89.67	0.00154133993911184\\
89.68	0.00154741608110497\\
89.69	0.00155349522838989\\
89.7	0.00155957738293488\\
89.71	0.00156566254670612\\
89.72	0.00157175072166757\\
89.73	0.00157784190978101\\
89.74	0.00158393611300599\\
89.75	0.00159003333329979\\
89.76	0.00159613357261743\\
89.77	0.00160223683291159\\
89.78	0.00160834311613261\\
89.79	0.00161445242422848\\
89.8	0.00162056475914478\\
89.81	0.00162668012282467\\
89.82	0.00163279851720885\\
89.83	0.00163891994423553\\
89.84	0.00164504440584042\\
89.85	0.00165117190395668\\
89.86	0.00165730244051489\\
89.87	0.00166343601744305\\
89.88	0.00166957263666651\\
89.89	0.00167571230010797\\
89.9	0.00168185500968741\\
89.91	0.00168800076732213\\
89.92	0.00169414957492664\\
89.93	0.00170030143441267\\
89.94	0.00170645634768916\\
89.95	0.00171261431666216\\
89.96	0.00171877534323488\\
89.97	0.00172493942930759\\
89.98	0.00173110657677763\\
89.99	0.00173727678753935\\
90	0.00174345006348412\\
90.01	0.00174962640650023\\
90.02	0.00175580581847291\\
90.03	0.00176198830128429\\
90.04	0.00176817385681336\\
90.05	0.00177436248693591\\
90.06	0.00178055419352454\\
90.07	0.00178674897844859\\
90.08	0.00179294684357414\\
90.09	0.00179914779076395\\
90.1	0.00180535182187742\\
90.11	0.00181155893877057\\
90.12	0.001817769143296\\
90.13	0.00182398243730286\\
90.14	0.00183019882263679\\
90.15	0.00183641830113994\\
90.16	0.00184264087465084\\
90.17	0.00184886654500447\\
90.18	0.00185509531403213\\
90.19	0.00186132718356148\\
90.2	0.00186756215541643\\
90.21	0.00187380023141715\\
90.22	0.00188004141338004\\
90.23	0.00188628570311763\\
90.24	0.00189253310243863\\
90.25	0.00189878361314781\\
90.26	0.00190503723704599\\
90.27	0.00191129397593003\\
90.28	0.00191755383159273\\
90.29	0.00192381680582287\\
90.3	0.00193008290040506\\
90.31	0.00193635211711983\\
90.32	0.00194262445774347\\
90.33	0.00194889992404806\\
90.34	0.0019551785178014\\
90.35	0.00196146024076699\\
90.36	0.00196774509470396\\
90.37	0.00197403308136704\\
90.38	0.00198032420250652\\
90.39	0.00198661845986821\\
90.4	0.00199291585519339\\
90.41	0.00199921639021876\\
90.42	0.00200552006667641\\
90.43	0.00201182688629375\\
90.44	0.0020181368507935\\
90.45	0.00202444996189363\\
90.46	0.0020307662213073\\
90.47	0.00203708563074283\\
90.48	0.00204340819190365\\
90.49	0.00204973390648825\\
90.5	0.00205606277619015\\
90.51	0.00206239480269782\\
90.52	0.00206872998769465\\
90.53	0.00207506833285893\\
90.54	0.00208140983986374\\
90.55	0.00208775451037696\\
90.56	0.00209410234606119\\
90.57	0.00210045334857369\\
90.58	0.00210680751956637\\
90.59	0.00211316486068572\\
90.6	0.00211952537357272\\
90.61	0.00212588905986287\\
90.62	0.00213225592118607\\
90.63	0.00213862595916659\\
90.64	0.00214499917542303\\
90.65	0.00215137557156824\\
90.66	0.00215775514920931\\
90.67	0.00216413790994748\\
90.68	0.00217052385537807\\
90.69	0.00217691298709051\\
90.7	0.00218330530666816\\
90.71	0.00218970081568839\\
90.72	0.0021960995157224\\
90.73	0.00220250140833527\\
90.74	0.00220890649508581\\
90.75	0.00221531477752658\\
90.76	0.0022217262572038\\
90.77	0.00222814079385458\\
90.78	0.00223455818222918\\
90.79	0.00224097842531869\\
90.8	0.0022474015261192\\
90.81	0.00225382748763176\\
90.82	0.00226025631286246\\
90.83	0.00226668800482241\\
90.84	0.00227312256652777\\
90.85	0.00227956000099976\\
90.86	0.0022860003112647\\
90.87	0.002292443500354\\
90.88	0.0022988895713042\\
90.89	0.00230533852715696\\
90.9	0.00231179037095911\\
90.91	0.00231824510576266\\
90.92	0.0023247027346248\\
90.93	0.00233116326060795\\
90.94	0.00233762668677974\\
90.95	0.00234409301621307\\
90.96	0.0023505622519861\\
90.97	0.00235703439718229\\
90.98	0.0023635094548904\\
90.99	0.00236998742820452\\
91	0.00237646832022409\\
91.01	0.00238295213405392\\
91.02	0.00238943887280421\\
91.03	0.00239592853959057\\
91.04	0.00240242113753404\\
91.05	0.0024089166697611\\
91.06	0.00241541513940374\\
91.07	0.00242191654959939\\
91.08	0.00242842090349103\\
91.09	0.00243492820422717\\
91.1	0.00244143845496186\\
91.11	0.00244795165885477\\
91.12	0.00245446781907112\\
91.13	0.00246098693878179\\
91.14	0.0024675090211633\\
91.15	0.00247403406939781\\
91.16	0.00248056208667322\\
91.17	0.00248709307618311\\
91.18	0.00249362704112679\\
91.19	0.00250016398470937\\
91.2	0.0025067039101417\\
91.21	0.00251324682064046\\
91.22	0.00251979271942817\\
91.23	0.00252634160973318\\
91.24	0.00253289349478974\\
91.25	0.002539448377838\\
91.26	0.00254600626212403\\
91.27	0.00255256715089987\\
91.28	0.00255913104742352\\
91.29	0.002565697954959\\
91.3	0.00257226787677635\\
91.31	0.00257884081615167\\
91.32	0.00258541677636715\\
91.33	0.00259199576071106\\
91.34	0.00259857777247784\\
91.35	0.00260516281496807\\
91.36	0.00261175089148851\\
91.37	0.00261834200535216\\
91.38	0.00262493615987823\\
91.39	0.00263153335839224\\
91.4	0.00263813360422596\\
91.41	0.00264473690071753\\
91.42	0.00265134325121141\\
91.43	0.00265795265905846\\
91.44	0.00266456512761595\\
91.45	0.00267118066024758\\
91.46	0.00267779926032352\\
91.47	0.00268442093122045\\
91.48	0.00269104567632157\\
91.49	0.00269767349901664\\
91.5	0.002704304402702\\
91.51	0.00271093839078063\\
91.52	0.00271757546666214\\
91.53	0.00272421563376283\\
91.54	0.0027308588955057\\
91.55	0.00273750525532051\\
91.56	0.00274415471664378\\
91.57	0.00275080728291886\\
91.58	0.00275746295759592\\
91.59	0.00276412174413199\\
91.6	0.00277078364599105\\
91.61	0.00277744866664398\\
91.62	0.00278411680956864\\
91.63	0.0027907880782499\\
91.64	0.00279746247617968\\
91.65	0.00280414000685696\\
91.66	0.00281082067378783\\
91.67	0.00281750448048554\\
91.68	0.00282419143047049\\
91.69	0.00283088152727033\\
91.7	0.00283757477441993\\
91.71	0.00284427117546147\\
91.72	0.00285097073394445\\
91.73	0.00285767345342571\\
91.74	0.00286437933746951\\
91.75	0.00287108838964754\\
91.76	0.00287780061353896\\
91.77	0.00288451601273045\\
91.78	0.00289123459081622\\
91.79	0.00289795635139808\\
91.8	0.00290468129808548\\
91.81	0.00291140943449552\\
91.82	0.00291814076425301\\
91.83	0.0029248752909905\\
91.84	0.00293161301834836\\
91.85	0.00293835394997475\\
91.86	0.00294509808952571\\
91.87	0.00295184544066521\\
91.88	0.00295859600706515\\
91.89	0.00296534979240543\\
91.9	0.00297210680037401\\
91.91	0.0029788670346669\\
91.92	0.00298563049898825\\
91.93	0.00299239719705038\\
91.94	0.00299916713257381\\
91.95	0.00300594030928735\\
91.96	0.00301271673092806\\
91.97	0.0030194964012414\\
91.98	0.00302627932398118\\
91.99	0.00303306550290968\\
92	0.00303985494179765\\
92.01	0.00304664764442438\\
92.02	0.00305344361457773\\
92.03	0.00306024285605419\\
92.04	0.00306704537265895\\
92.05	0.00307385116820589\\
92.06	0.00308066024651768\\
92.07	0.00308747261142581\\
92.08	0.00309428826677064\\
92.09	0.00310110721640146\\
92.1	0.00310792946417653\\
92.11	0.00311475501396314\\
92.12	0.00312158386963764\\
92.13	0.00312841603508553\\
92.14	0.00313525151420147\\
92.15	0.00314209031088937\\
92.16	0.00314893242906241\\
92.17	0.00315577787264314\\
92.18	0.00316262664556347\\
92.19	0.00316947875176478\\
92.2	0.00317633419519836\\
92.21	0.00318319297982619\\
92.22	0.00319005510962103\\
92.23	0.00319692058856655\\
92.24	0.00320378942065734\\
92.25	0.00321066160989907\\
92.26	0.0032175371603085\\
92.27	0.00322441607591367\\
92.28	0.00323129836075389\\
92.29	0.00323818401887989\\
92.3	0.00324507305435391\\
92.31	0.00325196547124976\\
92.32	0.00325886127365294\\
92.33	0.00326576046566075\\
92.34	0.00327266305138236\\
92.35	0.00327956903493891\\
92.36	0.00328647842046362\\
92.37	0.00329339121210189\\
92.38	0.00330030741401141\\
92.39	0.00330722703036223\\
92.4	0.00331415006533692\\
92.41	0.00332107652313061\\
92.42	0.00332800640795116\\
92.43	0.00333493972401921\\
92.44	0.00334187647556835\\
92.45	0.00334881666684519\\
92.46	0.00335576030210946\\
92.47	0.00336270738563417\\
92.48	0.00336965792170572\\
92.49	0.00337661191462395\\
92.5	0.00338356936870236\\
92.51	0.00339053028826814\\
92.52	0.00339749467766235\\
92.53	0.00340446254124001\\
92.54	0.00341143388337023\\
92.55	0.00341840870843637\\
92.56	0.00342538702083612\\
92.57	0.00343236882498164\\
92.58	0.00343935412529971\\
92.59	0.00344634292623185\\
92.6	0.00345333523223447\\
92.61	0.00346033104777896\\
92.62	0.00346733037735189\\
92.63	0.00347433322545509\\
92.64	0.00348133959660584\\
92.65	0.00348834949533698\\
92.66	0.00349536292619707\\
92.67	0.00350237989375053\\
92.68	0.00350940040257779\\
92.69	0.00351642445727544\\
92.7	0.00352345206245638\\
92.71	0.00353048322274998\\
92.72	0.00353751794280222\\
92.73	0.00354455622727588\\
92.74	0.00355159808085066\\
92.75	0.00355864350822336\\
92.76	0.00356569251410805\\
92.77	0.00357274510323624\\
92.78	0.00357980128035701\\
92.79	0.00358686105023724\\
92.8	0.00359392441766172\\
92.81	0.00360099138743337\\
92.82	0.00360806196437341\\
92.83	0.00361513615332151\\
92.84	0.00362221395913599\\
92.85	0.00362929538667364\\
92.86	0.00363638044069388\\
92.87	0.00364346912597721\\
92.88	0.00365056144732537\\
92.89	0.0036576574095616\\
92.9	0.00366475701753088\\
92.91	0.00367186027610016\\
92.92	0.0036789671901586\\
92.93	0.00368607776461783\\
92.94	0.00369319200441218\\
92.95	0.00370030991449897\\
92.96	0.00370743149985873\\
92.97	0.00371455676549545\\
92.98	0.0037216857164369\\
92.99	0.00372881835773484\\
93	0.00373595469446534\\
93.01	0.00374309473172899\\
93.02	0.00375023847465124\\
93.03	0.00375738592838267\\
93.04	0.00376453709809923\\
93.05	0.0037716919890026\\
93.06	0.00377885060632044\\
93.07	0.00378601295530667\\
93.08	0.00379317904124182\\
93.09	0.00380034886943333\\
93.1	0.00380752244521581\\
93.11	0.0038146997739514\\
93.12	0.00382188086103009\\
93.13	0.00382906571187001\\
93.14	0.00383625433191781\\
93.15	0.00384344672664892\\
93.16	0.00385064290156795\\
93.17	0.003857842862209\\
93.18	0.00386504661413603\\
93.19	0.00387225416294316\\
93.2	0.0038794655142551\\
93.21	0.00388668067372743\\
93.22	0.00389389964704702\\
93.23	0.00390112243993241\\
93.24	0.00390834905813412\\
93.25	0.00391557950743509\\
93.26	0.00392281379365105\\
93.27	0.00393005192263089\\
93.28	0.0039372939002571\\
93.29	0.00394453973244611\\
93.3	0.00395178942514877\\
93.31	0.00395904298435068\\
93.32	0.0039663004160727\\
93.33	0.0039735617263713\\
93.34	0.00398082692133902\\
93.35	0.00398809600710491\\
93.36	0.00399536898983496\\
93.37	0.00400264587573256\\
93.38	0.00400992667103894\\
93.39	0.00401721138203364\\
93.4	0.00402450001503498\\
93.41	0.00403179257640052\\
93.42	0.00403908907252755\\
93.43	0.00404638950985358\\
93.44	0.00405369389485682\\
93.45	0.00406100223405669\\
93.46	0.00406831453401432\\
93.47	0.00407563080133307\\
93.48	0.00408295104265903\\
93.49	0.0040902752646816\\
93.5	0.00409760347413396\\
93.51	0.00410493567779366\\
93.52	0.00411227188248312\\
93.53	0.00411961209507027\\
93.54	0.00412695632246903\\
93.55	0.00413430457163991\\
93.56	0.00414165684959064\\
93.57	0.00414901316300528\\
93.58	0.00415637351846089\\
93.59	0.00416373792258096\\
93.6	0.0041711063820361\\
93.61	0.00417847890354457\\
93.62	0.00418585549387297\\
93.63	0.00419323615983681\\
93.64	0.00420062090830116\\
93.65	0.00420800974618131\\
93.66	0.00421540268044339\\
93.67	0.00422279971810507\\
93.68	0.00423020086623619\\
93.69	0.00423760613195947\\
93.7	0.00424501552245117\\
93.71	0.00425242904494182\\
93.72	0.00425984670671691\\
93.73	0.00426726851511759\\
93.74	0.00427469447754145\\
93.75	0.00428212460144319\\
93.76	0.00428955889433544\\
93.77	0.00429699736378944\\
93.78	0.00430444001743589\\
93.79	0.00431188686296565\\
93.8	0.00431933790813061\\
93.81	0.00432679316074442\\
93.82	0.00433425262868335\\
93.83	0.00434171631988708\\
93.84	0.00434918424235958\\
93.85	0.0043566564041699\\
93.86	0.00436413281345308\\
93.87	0.00437161347841098\\
93.88	0.00437909840731319\\
93.89	0.0043865876084979\\
93.9	0.00439408109037284\\
93.91	0.00440157886141618\\
93.92	0.00440908093017744\\
93.93	0.00441658730527849\\
93.94	0.00442409799541444\\
93.95	0.00443161300935467\\
93.96	0.00443913235594379\\
93.97	0.00444665604410264\\
93.98	0.0044541840828293\\
93.99	0.00446171648120012\\
94	0.00446925324837075\\
94.01	0.00447679439357722\\
94.02	0.00448433992613697\\
94.03	0.00449188985544999\\
94.04	0.00449944419099987\\
94.05	0.00450700294235491\\
94.06	0.00451456611916929\\
94.07	0.00452213373118421\\
94.08	0.004529705788229\\
94.09	0.00453728230022235\\
94.1	0.0045448632771735\\
94.11	0.00455244872918341\\
94.12	0.00456003866644602\\
94.13	0.00456763309924948\\
94.14	0.00457523203797742\\
94.15	0.00458283549311022\\
94.16	0.00459044347522631\\
94.17	0.00459805599500346\\
94.18	0.00460567306322015\\
94.19	0.00461329469075688\\
94.2	0.00462092088859757\\
94.21	0.00462855166783092\\
94.22	0.00463618703965184\\
94.23	0.00464382701536284\\
94.24	0.00465147160637551\\
94.25	0.00465912082421196\\
94.26	0.00466677468050632\\
94.27	0.00467443318700623\\
94.28	0.00468209635557437\\
94.29	0.00468976419819001\\
94.3	0.00469743672695055\\
94.31	0.00470511395407316\\
94.32	0.00471279589189633\\
94.33	0.00472048255288156\\
94.34	0.00472817394961495\\
94.35	0.00473587009480893\\
94.36	0.00474357100130393\\
94.37	0.00475127668207011\\
94.38	0.0047589871502091\\
94.39	0.00476670241895579\\
94.4	0.00477442250168007\\
94.41	0.00478214741188873\\
94.42	0.00478987716322721\\
94.43	0.00479761176948155\\
94.44	0.00480535124458021\\
94.45	0.00481309560259604\\
94.46	0.0048208448577482\\
94.47	0.00482859902440414\\
94.48	0.00483635811708157\\
94.49	0.00484412215045054\\
94.5	0.0048518911393354\\
94.51	0.00485966509871699\\
94.52	0.00486744404373465\\
94.53	0.00487522798968839\\
94.54	0.00488301695204107\\
94.55	0.00489081094642058\\
94.56	0.00489860998862204\\
94.57	0.00490641409461008\\
94.58	0.00491422328052111\\
94.59	0.00492203756266562\\
94.6	0.00492985695753054\\
94.61	0.00493768148178161\\
94.62	0.00494551115226578\\
94.63	0.00495334598601365\\
94.64	0.00496118600024193\\
94.65	0.00496903121235598\\
94.66	0.00497688163577178\\
94.67	0.0049847372760841\\
94.68	0.00499259813879543\\
94.69	0.0050004642294144\\
94.7	0.00500833555345587\\
94.71	0.00501621211644086\\
94.72	0.00502409392389658\\
94.73	0.00503198098135646\\
94.74	0.00503987329436007\\
94.75	0.00504777086845321\\
94.76	0.00505567370918784\\
94.77	0.00506358182212211\\
94.78	0.00507149521282038\\
94.79	0.00507941388685317\\
94.8	0.0050873378497972\\
94.81	0.00509526710723537\\
94.82	0.00510320166475675\\
94.83	0.00511114152795662\\
94.84	0.00511908670243642\\
94.85	0.0051270371938038\\
94.86	0.00513499300767255\\
94.87	0.00514295414966267\\
94.88	0.00515092062540034\\
94.89	0.00515889244051789\\
94.9	0.00516686960065387\\
94.91	0.00517485211145295\\
94.92	0.00518283997856602\\
94.93	0.00519083320765012\\
94.94	0.00519883180436847\\
94.95	0.00520683577439044\\
94.96	0.00521484512339161\\
94.97	0.00522285985705367\\
94.98	0.00523087998106452\\
94.99	0.0052389055011182\\
95	0.00524693642291491\\
95.01	0.00525497275216101\\
95.02	0.00526301449456903\\
95.03	0.00527106165585764\\
95.04	0.00527911424175167\\
95.05	0.00528717225798208\\
95.06	0.00529523571028601\\
95.07	0.00530330460440671\\
95.08	0.0053113789460936\\
95.09	0.00531945874110222\\
95.1	0.00532754399519427\\
95.11	0.00533563471413756\\
95.12	0.00534373090370604\\
95.13	0.0053518325696798\\
95.14	0.00535993971784504\\
95.15	0.00536805235399409\\
95.16	0.0053761704839254\\
95.17	0.00538429411344352\\
95.18	0.00539242324835915\\
95.19	0.00540055789448906\\
95.2	0.00540869805765614\\
95.21	0.0054168437436894\\
95.22	0.00542499495842392\\
95.23	0.00543315170770089\\
95.24	0.0054413139973676\\
95.25	0.0054494818332774\\
95.26	0.00545765522128975\\
95.27	0.00546583416727018\\
95.28	0.00547401867709029\\
95.29	0.00548220875662775\\
95.3	0.00549040441176632\\
95.31	0.00549860564839578\\
95.32	0.00550681247241201\\
95.33	0.00551502488971691\\
95.34	0.00552324290621844\\
95.35	0.00553146652783061\\
95.36	0.00553969576047347\\
95.37	0.00554793061007308\\
95.38	0.00555617108256156\\
95.39	0.00556441718387702\\
95.4	0.00557266891996362\\
95.41	0.0055809262967715\\
95.42	0.00558918932025684\\
95.43	0.0055974579963818\\
95.44	0.00560573233111453\\
95.45	0.00561401233042919\\
95.46	0.0056222980003059\\
95.47	0.00563058934673079\\
95.48	0.00563888637569592\\
95.49	0.00564718909319936\\
95.5	0.0056554975052451\\
95.51	0.0056638116178431\\
95.52	0.00567213143700928\\
95.53	0.00568045696876546\\
95.54	0.00568878821913943\\
95.55	0.0056971251941649\\
95.56	0.00570546789988148\\
95.57	0.0057138163423347\\
95.58	0.00572217052757601\\
95.59	0.00573053046166273\\
95.6	0.0057388961506581\\
95.61	0.00574726760063121\\
95.62	0.00575564481765705\\
95.63	0.00576402780781646\\
95.64	0.00577241657719615\\
95.65	0.00578081113188868\\
95.66	0.00578921147799245\\
95.67	0.00579761762161168\\
95.68	0.00580602956885645\\
95.69	0.00581444732584262\\
95.7	0.00582287089869188\\
95.71	0.00583130029353173\\
95.72	0.00583973551649543\\
95.73	0.00584817657372204\\
95.74	0.00585662347135641\\
95.75	0.00586507621554912\\
95.76	0.00587353481245653\\
95.77	0.00588199926824073\\
95.78	0.00589046958906955\\
95.79	0.00589894578111656\\
95.8	0.00590742785056102\\
95.81	0.00591591580358792\\
95.82	0.00592440964638792\\
95.83	0.00593290938515739\\
95.84	0.00594141502609836\\
95.85	0.00594992657541854\\
95.86	0.00595844403933126\\
95.87	0.00596696742405552\\
95.88	0.00597549673581596\\
95.89	0.00598403198084282\\
95.9	0.00599257316537195\\
95.91	0.0060011202956448\\
95.92	0.00600967337790842\\
95.93	0.00601823241841542\\
95.94	0.00602679742342398\\
95.95	0.00603536839919781\\
95.96	0.0060439453520062\\
95.97	0.00605252828812393\\
95.98	0.0060611172138313\\
95.99	0.00606971213541413\\
96	0.0060783130591637\\
96.01	0.00608691999137679\\
96.02	0.00609553293835562\\
96.03	0.00610415190640789\\
96.04	0.00611277690184671\\
96.05	0.00612140793099061\\
96.06	0.00613004500016354\\
96.07	0.00613868811569485\\
96.08	0.00614733728391926\\
96.09	0.00615599251117686\\
96.1	0.00616465380381308\\
96.11	0.00617332116817871\\
96.12	0.00618199461062984\\
96.13	0.00619067413752788\\
96.14	0.00619935975523954\\
96.15	0.00620805147013679\\
96.16	0.00621674928859688\\
96.17	0.00622545321700228\\
96.18	0.00623416326174073\\
96.19	0.00624287942920514\\
96.2	0.00625160172579366\\
96.21	0.00626033015790959\\
96.22	0.00626906473196142\\
96.23	0.00627780545436278\\
96.24	0.00628655233153242\\
96.25	0.00629530536989424\\
96.26	0.00630406457587719\\
96.27	0.00631282995591534\\
96.28	0.00632160151644781\\
96.29	0.00633037926391877\\
96.3	0.00633916320477741\\
96.31	0.00634795334547793\\
96.32	0.00635674969247953\\
96.33	0.00636555225224637\\
96.34	0.00637436103124759\\
96.35	0.00638317603595722\\
96.36	0.00639199727285426\\
96.37	0.00640082474842255\\
96.38	0.00640965846915086\\
96.39	0.00641849844153278\\
96.4	0.00642734467206676\\
96.41	0.00643619716725605\\
96.42	0.0064450559336087\\
96.43	0.00645392097763755\\
96.44	0.00646279230586019\\
96.45	0.00647166992479893\\
96.46	0.0064805538409808\\
96.47	0.00648944406093753\\
96.48	0.00649834059120552\\
96.49	0.00650724343832581\\
96.5	0.00651615260884407\\
96.51	0.00652506810931057\\
96.52	0.00653398994628017\\
96.53	0.00654291812631227\\
96.54	0.00655185265597083\\
96.55	0.00656079354182431\\
96.56	0.00656974079044565\\
96.57	0.00657869440841226\\
96.58	0.00658765440230601\\
96.59	0.00659662077871316\\
96.6	0.00660559354422437\\
96.61	0.00661457270543468\\
96.62	0.00662355826894346\\
96.63	0.0066325502413544\\
96.64	0.00664154862927548\\
96.65	0.00665055343931895\\
96.66	0.0066595646781013\\
96.67	0.00666858235224324\\
96.68	0.00667760646836966\\
96.69	0.00668663703310961\\
96.7	0.00669567405309628\\
96.71	0.00670471753496696\\
96.72	0.00671376748536304\\
96.73	0.00672282391092993\\
96.74	0.00673188681831709\\
96.75	0.00674095621417796\\
96.76	0.00675003210516996\\
96.77	0.00675911449795443\\
96.78	0.00676820339919665\\
96.79	0.00677729881556575\\
96.8	0.00678640075373473\\
96.81	0.00679550922038039\\
96.82	0.00680462422218334\\
96.83	0.00681374576582794\\
96.84	0.00682287385800228\\
96.85	0.00683200850539815\\
96.86	0.00684114971471101\\
96.87	0.00685029749263994\\
96.88	0.00685945184588765\\
96.89	0.00686861278116039\\
96.9	0.00687778030516798\\
96.91	0.0068869544246237\\
96.92	0.00689613514624436\\
96.93	0.00690532247675017\\
96.94	0.00691451642286475\\
96.95	0.00692371699131509\\
96.96	0.00693292418883154\\
96.97	0.00694213802214773\\
96.98	0.00695135849800055\\
96.99	0.00696058562313014\\
97	0.00696981940427983\\
97.01	0.00697905984819611\\
97.02	0.00698830696162859\\
97.03	0.00699756075132997\\
97.04	0.00700682122405601\\
97.05	0.00701608838656546\\
97.06	0.00702536224562006\\
97.07	0.00703464280798449\\
97.08	0.00704393008042632\\
97.09	0.00705322406971599\\
97.1	0.00706252478262675\\
97.11	0.00707183222593463\\
97.12	0.00708114640641842\\
97.13	0.0070904673308596\\
97.14	0.00709979500604231\\
97.15	0.00710912943875332\\
97.16	0.00711847063578195\\
97.17	0.00712781860392011\\
97.18	0.00713717334996215\\
97.19	0.00714653488070492\\
97.2	0.00715590320294765\\
97.21	0.00716527832349196\\
97.22	0.00717466024914177\\
97.23	0.00718404898670331\\
97.24	0.00719344454298502\\
97.25	0.00720284692479755\\
97.26	0.00721225613895369\\
97.27	0.00722167219226832\\
97.28	0.00723109509155841\\
97.29	0.00724052484364291\\
97.3	0.00724996145534274\\
97.31	0.00725940493348073\\
97.32	0.0072688552848816\\
97.33	0.00727831251637188\\
97.34	0.00728777663477986\\
97.35	0.00729724764693557\\
97.36	0.00730672555967071\\
97.37	0.0073162103798186\\
97.38	0.00732570211421415\\
97.39	0.00733520076969379\\
97.4	0.00734470635309541\\
97.41	0.00735421887125835\\
97.42	0.0073637383310233\\
97.43	0.00737326473923226\\
97.44	0.00738279810272854\\
97.45	0.00739233842835663\\
97.46	0.00740188572296218\\
97.47	0.00741143999339196\\
97.48	0.00742100124649378\\
97.49	0.00743056948911647\\
97.5	0.00744014472810977\\
97.51	0.00744972697032433\\
97.52	0.00745931622261162\\
97.53	0.00746891249182389\\
97.54	0.00747851578481409\\
97.55	0.00748812610843585\\
97.56	0.00749774346954338\\
97.57	0.00750736787499146\\
97.58	0.0075169993316353\\
97.59	0.00752663784633059\\
97.6	0.00753628342593335\\
97.61	0.0075459360772999\\
97.62	0.00755559580728682\\
97.63	0.00756526262275084\\
97.64	0.00757493653054882\\
97.65	0.00758461753753769\\
97.66	0.00759430565057435\\
97.67	0.00760400087651561\\
97.68	0.00761370322221818\\
97.69	0.00762341269453854\\
97.7	0.00763312930033291\\
97.71	0.00764285304645716\\
97.72	0.00765258393976677\\
97.73	0.00766232198711677\\
97.74	0.00767206719536161\\
97.75	0.00768181957135516\\
97.76	0.00769157912195062\\
97.77	0.00770134585400044\\
97.78	0.00771111977435625\\
97.79	0.0077209008898688\\
97.8	0.00773068920738788\\
97.81	0.00774048473376227\\
97.82	0.00775028747583962\\
97.83	0.00776009744046643\\
97.84	0.00776991463448794\\
97.85	0.00777973906474809\\
97.86	0.00778957073808939\\
97.87	0.00779940966135291\\
97.88	0.00780925584137815\\
97.89	0.00781910928500301\\
97.9	0.00782896999906365\\
97.91	0.00783883799039448\\
97.92	0.00784871326582805\\
97.93	0.00785859583219497\\
97.94	0.00786848569632381\\
97.95	0.00787838286504107\\
97.96	0.00788828734517107\\
97.97	0.00789819914353585\\
97.98	0.00790811826695511\\
97.99	0.00791804472224614\\
98	0.00792797851622369\\
98.01	0.00793791965569993\\
98.02	0.00794786814748433\\
98.03	0.00795782399838358\\
98.04	0.00796778721520148\\
98.05	0.00797775780473889\\
98.06	0.00798773577379363\\
98.07	0.00799772112916038\\
98.08	0.00800771387763031\\
98.09	0.0080177140259909\\
98.1	0.00802772158102585\\
98.11	0.00803773654951499\\
98.12	0.00804775893823414\\
98.13	0.00805778875395509\\
98.14	0.00806782600344543\\
98.15	0.00807787069346848\\
98.16	0.0080879228307832\\
98.17	0.00809798242214404\\
98.18	0.00810804947430092\\
98.19	0.00811812399399903\\
98.2	0.00812820598797874\\
98.21	0.00813829546297551\\
98.22	0.00814839242571964\\
98.23	0.00815849688293618\\
98.24	0.00816860884134483\\
98.25	0.00817872830765984\\
98.26	0.00818885528858987\\
98.27	0.0081989897908379\\
98.28	0.00820913182110115\\
98.29	0.00821928138607091\\
98.3	0.00822943849248156\\
98.31	0.0082396031470706\\
98.32	0.00824977535656853\\
98.33	0.008259955127815\\
98.34	0.00827014246888159\\
98.35	0.00828033738784738\\
98.36	0.00829053989279898\\
98.37	0.00830074999192238\\
98.38	0.00831096769341426\\
98.39	0.00832119300547873\\
98.4	0.00833142593632734\\
98.41	0.00834166649417914\\
98.42	0.0083519146872607\\
98.43	0.00836217052413896\\
98.44	0.00837243401349226\\
98.45	0.0083827051640066\\
98.46	0.00839298398437564\\
98.47	0.00840327048330072\\
98.48	0.00841356466949091\\
98.49	0.00842386655166301\\
98.5	0.00843417613854159\\
98.51	0.00844449343885901\\
98.52	0.00845481846135549\\
98.53	0.00846515121477906\\
98.54	0.00847549170788566\\
98.55	0.00848583994943916\\
98.56	0.00849619594821136\\
98.57	0.00850655971298206\\
98.58	0.00851693125253908\\
98.59	0.00852731057567829\\
98.6	0.00853769769120368\\
98.61	0.00854809260792734\\
98.62	0.00855849533466957\\
98.63	0.00856890588025885\\
98.64	0.00857932425353194\\
98.65	0.0085897504633339\\
98.66	0.00860018451851813\\
98.67	0.00861062642794642\\
98.68	0.008621076200489\\
98.69	0.00863153361642958\\
98.7	0.00864199861111742\\
98.71	0.00865247119243095\\
98.72	0.00866295136824614\\
98.73	0.00867343914566332\\
98.74	0.00868393372853303\\
98.75	0.00869443110019272\\
98.76	0.00870493122910945\\
98.77	0.00871543408338785\\
98.78	0.00872593963076603\\
98.79	0.0087364467118356\\
98.8	0.00874695496642904\\
98.81	0.00875746435560937\\
98.82	0.00876797484002521\\
98.83	0.00877848637990586\\
98.84	0.00878899893505648\\
98.85	0.00879951246485317\\
98.86	0.00881002692823806\\
98.87	0.00882054228371423\\
98.88	0.00883105848934069\\
98.89	0.00884157550272716\\
98.9	0.00885209328102884\\
98.91	0.0088626117809411\\
98.92	0.00887313095869405\\
98.93	0.00888365077004706\\
98.94	0.00889417117028323\\
98.95	0.00890469211420367\\
98.96	0.00891521355612183\\
98.97	0.00892573544985762\\
98.98	0.00893625774873154\\
98.99	0.00894678040555864\\
99	0.00895730337264244\\
99.01	0.00896782660176875\\
99.02	0.00897835004419935\\
99.03	0.00898887365066564\\
99.04	0.00899939737136215\\
99.05	0.00900992115593992\\
99.06	0.00902044495349986\\
99.07	0.00903096871258595\\
99.08	0.00904149238117846\\
99.09	0.00905201590668712\\
99.1	0.009062539235944\\
99.11	0.00907306231519623\\
99.12	0.0090835850903821\\
99.13	0.00909410750691204\\
99.14	0.00910462950959158\\
99.15	0.00911515104261371\\
99.16	0.00912567204955099\\
99.17	0.00913619247334767\\
99.18	0.00914671225631155\\
99.19	0.00915723134010582\\
99.2	0.00916774966574069\\
99.21	0.00917826717356491\\
99.22	0.00918878380325717\\
99.23	0.00919929949381728\\
99.24	0.00920981418355733\\
99.25	0.00922032781009259\\
99.26	0.0092308403103323\\
99.27	0.00924135162047032\\
99.28	0.00925186167597559\\
99.29	0.00926237041158246\\
99.3	0.00927287776128084\\
99.31	0.00928338365830616\\
99.32	0.00929388803512921\\
99.33	0.00930439082344577\\
99.34	0.00931489195416604\\
99.35	0.00932539135740396\\
99.36	0.00933588896246625\\
99.37	0.00934638469784135\\
99.38	0.00935687849118809\\
99.39	0.00936737026932421\\
99.4	0.00937785995821467\\
99.41	0.00938834748295972\\
99.42	0.00939883276778278\\
99.43	0.00940931573601812\\
99.44	0.00941979631009832\\
99.45	0.00943027441154143\\
99.46	0.00944074996093798\\
99.47	0.00945122287793775\\
99.48	0.00946169308123622\\
99.49	0.00947216048856088\\
99.5	0.0094826250166572\\
99.51	0.00949308658127434\\
99.52	0.00950354509715069\\
99.53	0.00951400047799902\\
99.54	0.00952445263649142\\
99.55	0.00953490148424391\\
99.56	0.00954534693180085\\
99.57	0.00955578888861889\\
99.58	0.00956622726305078\\
99.59	0.00957666196232875\\
99.6	0.00958709289254763\\
99.61	0.0095975199586476\\
99.62	0.00960794306439661\\
99.63	0.00961836211237247\\
99.64	0.00962877700394458\\
99.65	0.00963918763925525\\
99.66	0.00964959391720069\\
99.67	0.00965999573541164\\
99.68	0.00967039299023348\\
99.69	0.0096807855767061\\
99.7	0.00969117338854325\\
99.71	0.00970155631812073\\
99.72	0.0097119342564558\\
99.73	0.00972230709318531\\
99.74	0.00973267471654341\\
99.75	0.00974303701333878\\
99.76	0.00975339386893135\\
99.77	0.00976374516720854\\
99.78	0.00977409079056095\\
99.79	0.00978443061985756\\
99.8	0.0097947645344203\\
99.81	0.00980509241199814\\
99.82	0.00981541412874054\\
99.83	0.00982572955917033\\
99.84	0.00983603857615593\\
99.85	0.00984634105088294\\
99.86	0.00985663685282512\\
99.87	0.00986692584971462\\
99.88	0.00987720790751152\\
99.89	0.0098874828903727\\
99.9	0.00989775066061986\\
99.91	0.00990801107870688\\
99.92	0.00991826400318625\\
99.93	0.00992850929067483\\
99.94	0.00993874679581858\\
99.95	0.00994897637125655\\
99.96	0.00995919786758387\\
99.97	0.0099694111333138\\
99.98	0.00997961601483888\\
99.99	0.0099898123563909\\
100	0.01\\
};
\addlegendentry{$q=0$};

\addplot [color=blue,solid,forget plot]
  table[row sep=crcr]{%
0.01	0\\
0.02	0\\
0.03	0\\
0.04	0\\
0.05	0\\
0.06	0\\
0.07	0\\
0.08	0\\
0.09	0\\
0.1	0\\
0.11	0\\
0.12	0\\
0.13	0\\
0.14	0\\
0.15	0\\
0.16	0\\
0.17	0\\
0.18	0\\
0.19	0\\
0.2	0\\
0.21	0\\
0.22	0\\
0.23	0\\
0.24	0\\
0.25	0\\
0.26	0\\
0.27	0\\
0.28	0\\
0.29	0\\
0.3	0\\
0.31	0\\
0.32	0\\
0.33	0\\
0.34	0\\
0.35	0\\
0.36	0\\
0.37	0\\
0.38	0\\
0.39	0\\
0.4	0\\
0.41	0\\
0.42	0\\
0.43	0\\
0.44	0\\
0.45	0\\
0.46	0\\
0.47	0\\
0.48	0\\
0.49	0\\
0.5	0\\
0.51	0\\
0.52	0\\
0.53	0\\
0.54	0\\
0.55	0\\
0.56	0\\
0.57	0\\
0.58	0\\
0.59	0\\
0.6	0\\
0.61	0\\
0.62	0\\
0.63	0\\
0.64	0\\
0.65	0\\
0.66	0\\
0.67	0\\
0.68	0\\
0.69	0\\
0.7	0\\
0.71	0\\
0.72	0\\
0.73	0\\
0.74	0\\
0.75	0\\
0.76	0\\
0.77	0\\
0.78	0\\
0.79	0\\
0.8	0\\
0.81	0\\
0.82	0\\
0.83	0\\
0.84	0\\
0.85	0\\
0.86	0\\
0.87	0\\
0.88	0\\
0.89	0\\
0.9	0\\
0.91	0\\
0.92	0\\
0.93	0\\
0.94	0\\
0.95	0\\
0.96	0\\
0.97	0\\
0.98	0\\
0.99	0\\
1	0\\
1.01	0\\
1.02	0\\
1.03	0\\
1.04	0\\
1.05	0\\
1.06	0\\
1.07	0\\
1.08	0\\
1.09	0\\
1.1	0\\
1.11	0\\
1.12	0\\
1.13	0\\
1.14	0\\
1.15	0\\
1.16	0\\
1.17	0\\
1.18	0\\
1.19	0\\
1.2	0\\
1.21	0\\
1.22	0\\
1.23	0\\
1.24	0\\
1.25	0\\
1.26	0\\
1.27	0\\
1.28	0\\
1.29	0\\
1.3	0\\
1.31	0\\
1.32	0\\
1.33	0\\
1.34	0\\
1.35	0\\
1.36	0\\
1.37	0\\
1.38	0\\
1.39	0\\
1.4	0\\
1.41	0\\
1.42	0\\
1.43	0\\
1.44	0\\
1.45	0\\
1.46	0\\
1.47	0\\
1.48	0\\
1.49	0\\
1.5	0\\
1.51	0\\
1.52	0\\
1.53	0\\
1.54	0\\
1.55	0\\
1.56	0\\
1.57	0\\
1.58	0\\
1.59	0\\
1.6	0\\
1.61	0\\
1.62	0\\
1.63	0\\
1.64	0\\
1.65	0\\
1.66	0\\
1.67	0\\
1.68	0\\
1.69	0\\
1.7	0\\
1.71	0\\
1.72	0\\
1.73	0\\
1.74	0\\
1.75	0\\
1.76	0\\
1.77	0\\
1.78	0\\
1.79	0\\
1.8	0\\
1.81	0\\
1.82	0\\
1.83	0\\
1.84	0\\
1.85	0\\
1.86	0\\
1.87	0\\
1.88	0\\
1.89	0\\
1.9	0\\
1.91	0\\
1.92	0\\
1.93	0\\
1.94	0\\
1.95	0\\
1.96	0\\
1.97	0\\
1.98	0\\
1.99	0\\
2	0\\
2.01	0\\
2.02	0\\
2.03	0\\
2.04	0\\
2.05	0\\
2.06	0\\
2.07	0\\
2.08	0\\
2.09	0\\
2.1	0\\
2.11	0\\
2.12	0\\
2.13	0\\
2.14	0\\
2.15	0\\
2.16	0\\
2.17	0\\
2.18	0\\
2.19	0\\
2.2	0\\
2.21	0\\
2.22	0\\
2.23	0\\
2.24	0\\
2.25	0\\
2.26	0\\
2.27	0\\
2.28	0\\
2.29	0\\
2.3	0\\
2.31	0\\
2.32	0\\
2.33	0\\
2.34	0\\
2.35	0\\
2.36	0\\
2.37	0\\
2.38	0\\
2.39	0\\
2.4	0\\
2.41	0\\
2.42	0\\
2.43	0\\
2.44	0\\
2.45	0\\
2.46	0\\
2.47	0\\
2.48	0\\
2.49	0\\
2.5	0\\
2.51	0\\
2.52	0\\
2.53	0\\
2.54	0\\
2.55	0\\
2.56	0\\
2.57	0\\
2.58	0\\
2.59	0\\
2.6	0\\
2.61	0\\
2.62	0\\
2.63	0\\
2.64	0\\
2.65	0\\
2.66	0\\
2.67	0\\
2.68	0\\
2.69	0\\
2.7	0\\
2.71	0\\
2.72	0\\
2.73	0\\
2.74	0\\
2.75	0\\
2.76	0\\
2.77	0\\
2.78	0\\
2.79	0\\
2.8	0\\
2.81	0\\
2.82	0\\
2.83	0\\
2.84	0\\
2.85	0\\
2.86	0\\
2.87	0\\
2.88	0\\
2.89	0\\
2.9	0\\
2.91	0\\
2.92	0\\
2.93	0\\
2.94	0\\
2.95	0\\
2.96	0\\
2.97	0\\
2.98	0\\
2.99	0\\
3	0\\
3.01	0\\
3.02	0\\
3.03	0\\
3.04	0\\
3.05	0\\
3.06	0\\
3.07	0\\
3.08	0\\
3.09	0\\
3.1	0\\
3.11	0\\
3.12	0\\
3.13	0\\
3.14	0\\
3.15	0\\
3.16	0\\
3.17	0\\
3.18	0\\
3.19	0\\
3.2	0\\
3.21	0\\
3.22	0\\
3.23	0\\
3.24	0\\
3.25	0\\
3.26	0\\
3.27	0\\
3.28	0\\
3.29	0\\
3.3	0\\
3.31	0\\
3.32	0\\
3.33	0\\
3.34	0\\
3.35	0\\
3.36	0\\
3.37	0\\
3.38	0\\
3.39	0\\
3.4	0\\
3.41	0\\
3.42	0\\
3.43	0\\
3.44	0\\
3.45	0\\
3.46	0\\
3.47	0\\
3.48	0\\
3.49	0\\
3.5	0\\
3.51	0\\
3.52	0\\
3.53	0\\
3.54	0\\
3.55	0\\
3.56	0\\
3.57	0\\
3.58	0\\
3.59	0\\
3.6	0\\
3.61	0\\
3.62	0\\
3.63	0\\
3.64	0\\
3.65	0\\
3.66	0\\
3.67	0\\
3.68	0\\
3.69	0\\
3.7	0\\
3.71	0\\
3.72	0\\
3.73	0\\
3.74	0\\
3.75	0\\
3.76	0\\
3.77	0\\
3.78	0\\
3.79	0\\
3.8	0\\
3.81	0\\
3.82	0\\
3.83	0\\
3.84	0\\
3.85	0\\
3.86	0\\
3.87	0\\
3.88	0\\
3.89	0\\
3.9	0\\
3.91	0\\
3.92	0\\
3.93	0\\
3.94	0\\
3.95	0\\
3.96	0\\
3.97	0\\
3.98	0\\
3.99	0\\
4	0\\
4.01	0\\
4.02	0\\
4.03	0\\
4.04	0\\
4.05	0\\
4.06	0\\
4.07	0\\
4.08	0\\
4.09	0\\
4.1	0\\
4.11	0\\
4.12	0\\
4.13	0\\
4.14	0\\
4.15	0\\
4.16	0\\
4.17	0\\
4.18	0\\
4.19	0\\
4.2	0\\
4.21	0\\
4.22	0\\
4.23	0\\
4.24	0\\
4.25	0\\
4.26	0\\
4.27	0\\
4.28	0\\
4.29	0\\
4.3	0\\
4.31	0\\
4.32	0\\
4.33	0\\
4.34	0\\
4.35	0\\
4.36	0\\
4.37	0\\
4.38	0\\
4.39	0\\
4.4	0\\
4.41	0\\
4.42	0\\
4.43	0\\
4.44	0\\
4.45	0\\
4.46	0\\
4.47	0\\
4.48	0\\
4.49	0\\
4.5	0\\
4.51	0\\
4.52	0\\
4.53	0\\
4.54	0\\
4.55	0\\
4.56	0\\
4.57	0\\
4.58	0\\
4.59	0\\
4.6	0\\
4.61	0\\
4.62	0\\
4.63	0\\
4.64	0\\
4.65	0\\
4.66	0\\
4.67	0\\
4.68	0\\
4.69	0\\
4.7	0\\
4.71	0\\
4.72	0\\
4.73	0\\
4.74	0\\
4.75	0\\
4.76	0\\
4.77	0\\
4.78	0\\
4.79	0\\
4.8	0\\
4.81	0\\
4.82	0\\
4.83	0\\
4.84	0\\
4.85	0\\
4.86	0\\
4.87	0\\
4.88	0\\
4.89	0\\
4.9	0\\
4.91	0\\
4.92	0\\
4.93	0\\
4.94	0\\
4.95	0\\
4.96	0\\
4.97	0\\
4.98	0\\
4.99	0\\
5	0\\
5.01	0\\
5.02	0\\
5.03	0\\
5.04	0\\
5.05	0\\
5.06	0\\
5.07	0\\
5.08	0\\
5.09	0\\
5.1	0\\
5.11	0\\
5.12	0\\
5.13	0\\
5.14	0\\
5.15	0\\
5.16	0\\
5.17	0\\
5.18	0\\
5.19	0\\
5.2	0\\
5.21	0\\
5.22	0\\
5.23	0\\
5.24	0\\
5.25	0\\
5.26	0\\
5.27	0\\
5.28	0\\
5.29	0\\
5.3	0\\
5.31	0\\
5.32	0\\
5.33	0\\
5.34	0\\
5.35	0\\
5.36	0\\
5.37	0\\
5.38	0\\
5.39	0\\
5.4	0\\
5.41	0\\
5.42	0\\
5.43	0\\
5.44	0\\
5.45	0\\
5.46	0\\
5.47	0\\
5.48	0\\
5.49	0\\
5.5	0\\
5.51	0\\
5.52	0\\
5.53	0\\
5.54	0\\
5.55	0\\
5.56	0\\
5.57	0\\
5.58	0\\
5.59	0\\
5.6	0\\
5.61	0\\
5.62	0\\
5.63	0\\
5.64	0\\
5.65	0\\
5.66	0\\
5.67	0\\
5.68	0\\
5.69	0\\
5.7	0\\
5.71	0\\
5.72	0\\
5.73	0\\
5.74	0\\
5.75	0\\
5.76	0\\
5.77	0\\
5.78	0\\
5.79	0\\
5.8	0\\
5.81	0\\
5.82	0\\
5.83	0\\
5.84	0\\
5.85	0\\
5.86	0\\
5.87	0\\
5.88	0\\
5.89	0\\
5.9	0\\
5.91	0\\
5.92	0\\
5.93	0\\
5.94	0\\
5.95	0\\
5.96	0\\
5.97	0\\
5.98	0\\
5.99	0\\
6	0\\
6.01	0\\
6.02	0\\
6.03	0\\
6.04	0\\
6.05	0\\
6.06	0\\
6.07	0\\
6.08	0\\
6.09	0\\
6.1	0\\
6.11	0\\
6.12	0\\
6.13	0\\
6.14	0\\
6.15	0\\
6.16	0\\
6.17	0\\
6.18	0\\
6.19	0\\
6.2	0\\
6.21	0\\
6.22	0\\
6.23	0\\
6.24	0\\
6.25	0\\
6.26	0\\
6.27	0\\
6.28	0\\
6.29	0\\
6.3	0\\
6.31	0\\
6.32	0\\
6.33	0\\
6.34	0\\
6.35	0\\
6.36	0\\
6.37	0\\
6.38	0\\
6.39	0\\
6.4	0\\
6.41	0\\
6.42	0\\
6.43	0\\
6.44	0\\
6.45	0\\
6.46	0\\
6.47	0\\
6.48	0\\
6.49	0\\
6.5	0\\
6.51	0\\
6.52	0\\
6.53	0\\
6.54	0\\
6.55	0\\
6.56	0\\
6.57	0\\
6.58	0\\
6.59	0\\
6.6	0\\
6.61	0\\
6.62	0\\
6.63	0\\
6.64	0\\
6.65	0\\
6.66	0\\
6.67	0\\
6.68	0\\
6.69	0\\
6.7	0\\
6.71	0\\
6.72	0\\
6.73	0\\
6.74	0\\
6.75	0\\
6.76	0\\
6.77	0\\
6.78	0\\
6.79	0\\
6.8	0\\
6.81	0\\
6.82	0\\
6.83	0\\
6.84	0\\
6.85	0\\
6.86	0\\
6.87	0\\
6.88	0\\
6.89	0\\
6.9	0\\
6.91	0\\
6.92	0\\
6.93	0\\
6.94	0\\
6.95	0\\
6.96	0\\
6.97	0\\
6.98	0\\
6.99	0\\
7	0\\
7.01	0\\
7.02	0\\
7.03	0\\
7.04	0\\
7.05	0\\
7.06	0\\
7.07	0\\
7.08	0\\
7.09	0\\
7.1	0\\
7.11	0\\
7.12	0\\
7.13	0\\
7.14	0\\
7.15	0\\
7.16	0\\
7.17	0\\
7.18	0\\
7.19	0\\
7.2	0\\
7.21	0\\
7.22	0\\
7.23	0\\
7.24	0\\
7.25	0\\
7.26	0\\
7.27	0\\
7.28	0\\
7.29	0\\
7.3	0\\
7.31	0\\
7.32	0\\
7.33	0\\
7.34	0\\
7.35	0\\
7.36	0\\
7.37	0\\
7.38	0\\
7.39	0\\
7.4	0\\
7.41	0\\
7.42	0\\
7.43	0\\
7.44	0\\
7.45	0\\
7.46	0\\
7.47	0\\
7.48	0\\
7.49	0\\
7.5	0\\
7.51	0\\
7.52	0\\
7.53	0\\
7.54	0\\
7.55	0\\
7.56	0\\
7.57	0\\
7.58	0\\
7.59	0\\
7.6	0\\
7.61	0\\
7.62	0\\
7.63	0\\
7.64	0\\
7.65	0\\
7.66	0\\
7.67	0\\
7.68	0\\
7.69	0\\
7.7	0\\
7.71	0\\
7.72	0\\
7.73	0\\
7.74	0\\
7.75	0\\
7.76	0\\
7.77	0\\
7.78	0\\
7.79	0\\
7.8	0\\
7.81	0\\
7.82	0\\
7.83	0\\
7.84	0\\
7.85	0\\
7.86	0\\
7.87	0\\
7.88	0\\
7.89	0\\
7.9	0\\
7.91	0\\
7.92	0\\
7.93	0\\
7.94	0\\
7.95	0\\
7.96	0\\
7.97	0\\
7.98	0\\
7.99	0\\
8	0\\
8.01	0\\
8.02	0\\
8.03	0\\
8.04	0\\
8.05	0\\
8.06	0\\
8.07	0\\
8.08	0\\
8.09	0\\
8.1	0\\
8.11	0\\
8.12	0\\
8.13	0\\
8.14	0\\
8.15	0\\
8.16	0\\
8.17	0\\
8.18	0\\
8.19	0\\
8.2	0\\
8.21	0\\
8.22	0\\
8.23	0\\
8.24	0\\
8.25	0\\
8.26	0\\
8.27	0\\
8.28	0\\
8.29	0\\
8.3	0\\
8.31	0\\
8.32	0\\
8.33	0\\
8.34	0\\
8.35	0\\
8.36	0\\
8.37	0\\
8.38	0\\
8.39	0\\
8.4	0\\
8.41	0\\
8.42	0\\
8.43	0\\
8.44	0\\
8.45	0\\
8.46	0\\
8.47	0\\
8.48	0\\
8.49	0\\
8.5	0\\
8.51	0\\
8.52	0\\
8.53	0\\
8.54	0\\
8.55	0\\
8.56	0\\
8.57	0\\
8.58	0\\
8.59	0\\
8.6	0\\
8.61	0\\
8.62	0\\
8.63	0\\
8.64	0\\
8.65	0\\
8.66	0\\
8.67	0\\
8.68	0\\
8.69	0\\
8.7	0\\
8.71	0\\
8.72	0\\
8.73	0\\
8.74	0\\
8.75	0\\
8.76	0\\
8.77	0\\
8.78	0\\
8.79	0\\
8.8	0\\
8.81	0\\
8.82	0\\
8.83	0\\
8.84	0\\
8.85	0\\
8.86	0\\
8.87	0\\
8.88	0\\
8.89	0\\
8.9	0\\
8.91	0\\
8.92	0\\
8.93	0\\
8.94	0\\
8.95	0\\
8.96	0\\
8.97	0\\
8.98	0\\
8.99	0\\
9	0\\
9.01	0\\
9.02	0\\
9.03	0\\
9.04	0\\
9.05	0\\
9.06	0\\
9.07	0\\
9.08	0\\
9.09	0\\
9.1	0\\
9.11	0\\
9.12	0\\
9.13	0\\
9.14	0\\
9.15	0\\
9.16	0\\
9.17	0\\
9.18	0\\
9.19	0\\
9.2	0\\
9.21	0\\
9.22	0\\
9.23	0\\
9.24	0\\
9.25	0\\
9.26	0\\
9.27	0\\
9.28	0\\
9.29	0\\
9.3	0\\
9.31	0\\
9.32	0\\
9.33	0\\
9.34	0\\
9.35	0\\
9.36	0\\
9.37	0\\
9.38	0\\
9.39	0\\
9.4	0\\
9.41	0\\
9.42	0\\
9.43	0\\
9.44	0\\
9.45	0\\
9.46	0\\
9.47	0\\
9.48	0\\
9.49	0\\
9.5	0\\
9.51	0\\
9.52	0\\
9.53	0\\
9.54	0\\
9.55	0\\
9.56	0\\
9.57	0\\
9.58	0\\
9.59	0\\
9.6	0\\
9.61	0\\
9.62	0\\
9.63	0\\
9.64	0\\
9.65	0\\
9.66	0\\
9.67	0\\
9.68	0\\
9.69	0\\
9.7	0\\
9.71	0\\
9.72	0\\
9.73	0\\
9.74	0\\
9.75	0\\
9.76	0\\
9.77	0\\
9.78	0\\
9.79	0\\
9.8	0\\
9.81	0\\
9.82	0\\
9.83	0\\
9.84	0\\
9.85	0\\
9.86	0\\
9.87	0\\
9.88	0\\
9.89	0\\
9.9	0\\
9.91	0\\
9.92	0\\
9.93	0\\
9.94	0\\
9.95	0\\
9.96	0\\
9.97	0\\
9.98	0\\
9.99	0\\
10	0\\
10.01	0\\
10.02	0\\
10.03	0\\
10.04	0\\
10.05	0\\
10.06	0\\
10.07	0\\
10.08	0\\
10.09	0\\
10.1	0\\
10.11	0\\
10.12	0\\
10.13	0\\
10.14	0\\
10.15	0\\
10.16	0\\
10.17	0\\
10.18	0\\
10.19	0\\
10.2	0\\
10.21	0\\
10.22	0\\
10.23	0\\
10.24	0\\
10.25	0\\
10.26	0\\
10.27	0\\
10.28	0\\
10.29	0\\
10.3	0\\
10.31	0\\
10.32	0\\
10.33	0\\
10.34	0\\
10.35	0\\
10.36	0\\
10.37	0\\
10.38	0\\
10.39	0\\
10.4	0\\
10.41	0\\
10.42	0\\
10.43	0\\
10.44	0\\
10.45	0\\
10.46	0\\
10.47	0\\
10.48	0\\
10.49	0\\
10.5	0\\
10.51	0\\
10.52	0\\
10.53	0\\
10.54	0\\
10.55	0\\
10.56	0\\
10.57	0\\
10.58	0\\
10.59	0\\
10.6	0\\
10.61	0\\
10.62	0\\
10.63	0\\
10.64	0\\
10.65	0\\
10.66	0\\
10.67	0\\
10.68	0\\
10.69	0\\
10.7	0\\
10.71	0\\
10.72	0\\
10.73	0\\
10.74	0\\
10.75	0\\
10.76	0\\
10.77	0\\
10.78	0\\
10.79	0\\
10.8	0\\
10.81	0\\
10.82	0\\
10.83	0\\
10.84	0\\
10.85	0\\
10.86	0\\
10.87	0\\
10.88	0\\
10.89	0\\
10.9	0\\
10.91	0\\
10.92	0\\
10.93	0\\
10.94	0\\
10.95	0\\
10.96	0\\
10.97	0\\
10.98	0\\
10.99	0\\
11	0\\
11.01	0\\
11.02	0\\
11.03	0\\
11.04	0\\
11.05	0\\
11.06	0\\
11.07	0\\
11.08	0\\
11.09	0\\
11.1	0\\
11.11	0\\
11.12	0\\
11.13	0\\
11.14	0\\
11.15	0\\
11.16	0\\
11.17	0\\
11.18	0\\
11.19	0\\
11.2	0\\
11.21	0\\
11.22	0\\
11.23	0\\
11.24	0\\
11.25	0\\
11.26	0\\
11.27	0\\
11.28	0\\
11.29	0\\
11.3	0\\
11.31	0\\
11.32	0\\
11.33	0\\
11.34	0\\
11.35	0\\
11.36	0\\
11.37	0\\
11.38	0\\
11.39	0\\
11.4	0\\
11.41	0\\
11.42	0\\
11.43	0\\
11.44	0\\
11.45	0\\
11.46	0\\
11.47	0\\
11.48	0\\
11.49	0\\
11.5	0\\
11.51	0\\
11.52	0\\
11.53	0\\
11.54	0\\
11.55	0\\
11.56	0\\
11.57	0\\
11.58	0\\
11.59	0\\
11.6	0\\
11.61	0\\
11.62	0\\
11.63	0\\
11.64	0\\
11.65	0\\
11.66	0\\
11.67	0\\
11.68	0\\
11.69	0\\
11.7	0\\
11.71	0\\
11.72	0\\
11.73	0\\
11.74	0\\
11.75	0\\
11.76	0\\
11.77	0\\
11.78	0\\
11.79	0\\
11.8	0\\
11.81	0\\
11.82	0\\
11.83	0\\
11.84	0\\
11.85	0\\
11.86	0\\
11.87	0\\
11.88	0\\
11.89	0\\
11.9	0\\
11.91	0\\
11.92	0\\
11.93	0\\
11.94	0\\
11.95	0\\
11.96	0\\
11.97	0\\
11.98	0\\
11.99	0\\
12	0\\
12.01	0\\
12.02	0\\
12.03	0\\
12.04	0\\
12.05	0\\
12.06	0\\
12.07	0\\
12.08	0\\
12.09	0\\
12.1	0\\
12.11	0\\
12.12	0\\
12.13	0\\
12.14	0\\
12.15	0\\
12.16	0\\
12.17	0\\
12.18	0\\
12.19	0\\
12.2	0\\
12.21	0\\
12.22	0\\
12.23	0\\
12.24	0\\
12.25	0\\
12.26	0\\
12.27	0\\
12.28	0\\
12.29	0\\
12.3	0\\
12.31	0\\
12.32	0\\
12.33	0\\
12.34	0\\
12.35	0\\
12.36	0\\
12.37	0\\
12.38	0\\
12.39	0\\
12.4	0\\
12.41	0\\
12.42	0\\
12.43	0\\
12.44	0\\
12.45	0\\
12.46	0\\
12.47	0\\
12.48	0\\
12.49	0\\
12.5	0\\
12.51	0\\
12.52	0\\
12.53	0\\
12.54	0\\
12.55	0\\
12.56	0\\
12.57	0\\
12.58	0\\
12.59	0\\
12.6	0\\
12.61	0\\
12.62	0\\
12.63	0\\
12.64	0\\
12.65	0\\
12.66	0\\
12.67	0\\
12.68	0\\
12.69	0\\
12.7	0\\
12.71	0\\
12.72	0\\
12.73	0\\
12.74	0\\
12.75	0\\
12.76	0\\
12.77	0\\
12.78	0\\
12.79	0\\
12.8	0\\
12.81	0\\
12.82	0\\
12.83	0\\
12.84	0\\
12.85	0\\
12.86	0\\
12.87	0\\
12.88	0\\
12.89	0\\
12.9	0\\
12.91	0\\
12.92	0\\
12.93	0\\
12.94	0\\
12.95	0\\
12.96	0\\
12.97	0\\
12.98	0\\
12.99	0\\
13	0\\
13.01	0\\
13.02	0\\
13.03	0\\
13.04	0\\
13.05	0\\
13.06	0\\
13.07	0\\
13.08	0\\
13.09	0\\
13.1	0\\
13.11	0\\
13.12	0\\
13.13	0\\
13.14	0\\
13.15	0\\
13.16	0\\
13.17	0\\
13.18	0\\
13.19	0\\
13.2	0\\
13.21	0\\
13.22	0\\
13.23	0\\
13.24	0\\
13.25	0\\
13.26	0\\
13.27	0\\
13.28	0\\
13.29	0\\
13.3	0\\
13.31	0\\
13.32	0\\
13.33	0\\
13.34	0\\
13.35	0\\
13.36	0\\
13.37	0\\
13.38	0\\
13.39	0\\
13.4	0\\
13.41	0\\
13.42	0\\
13.43	0\\
13.44	0\\
13.45	0\\
13.46	0\\
13.47	0\\
13.48	0\\
13.49	0\\
13.5	0\\
13.51	0\\
13.52	0\\
13.53	0\\
13.54	0\\
13.55	0\\
13.56	0\\
13.57	0\\
13.58	0\\
13.59	0\\
13.6	0\\
13.61	0\\
13.62	0\\
13.63	0\\
13.64	0\\
13.65	0\\
13.66	0\\
13.67	0\\
13.68	0\\
13.69	0\\
13.7	0\\
13.71	0\\
13.72	0\\
13.73	0\\
13.74	0\\
13.75	0\\
13.76	0\\
13.77	0\\
13.78	0\\
13.79	0\\
13.8	0\\
13.81	0\\
13.82	0\\
13.83	0\\
13.84	0\\
13.85	0\\
13.86	0\\
13.87	0\\
13.88	0\\
13.89	0\\
13.9	0\\
13.91	0\\
13.92	0\\
13.93	0\\
13.94	0\\
13.95	0\\
13.96	0\\
13.97	0\\
13.98	0\\
13.99	0\\
14	0\\
14.01	0\\
14.02	0\\
14.03	0\\
14.04	0\\
14.05	0\\
14.06	0\\
14.07	0\\
14.08	0\\
14.09	0\\
14.1	0\\
14.11	0\\
14.12	0\\
14.13	0\\
14.14	0\\
14.15	0\\
14.16	0\\
14.17	0\\
14.18	0\\
14.19	0\\
14.2	0\\
14.21	0\\
14.22	0\\
14.23	0\\
14.24	0\\
14.25	0\\
14.26	0\\
14.27	0\\
14.28	0\\
14.29	0\\
14.3	0\\
14.31	0\\
14.32	0\\
14.33	0\\
14.34	0\\
14.35	0\\
14.36	0\\
14.37	0\\
14.38	0\\
14.39	0\\
14.4	0\\
14.41	0\\
14.42	0\\
14.43	0\\
14.44	0\\
14.45	0\\
14.46	0\\
14.47	0\\
14.48	0\\
14.49	0\\
14.5	0\\
14.51	0\\
14.52	0\\
14.53	0\\
14.54	0\\
14.55	0\\
14.56	0\\
14.57	0\\
14.58	0\\
14.59	0\\
14.6	0\\
14.61	0\\
14.62	0\\
14.63	0\\
14.64	0\\
14.65	0\\
14.66	0\\
14.67	0\\
14.68	0\\
14.69	0\\
14.7	0\\
14.71	0\\
14.72	0\\
14.73	0\\
14.74	0\\
14.75	0\\
14.76	0\\
14.77	0\\
14.78	0\\
14.79	0\\
14.8	0\\
14.81	0\\
14.82	0\\
14.83	0\\
14.84	0\\
14.85	0\\
14.86	0\\
14.87	0\\
14.88	0\\
14.89	0\\
14.9	0\\
14.91	0\\
14.92	0\\
14.93	0\\
14.94	0\\
14.95	0\\
14.96	0\\
14.97	0\\
14.98	0\\
14.99	0\\
15	0\\
15.01	0\\
15.02	0\\
15.03	0\\
15.04	0\\
15.05	0\\
15.06	0\\
15.07	0\\
15.08	0\\
15.09	0\\
15.1	0\\
15.11	0\\
15.12	0\\
15.13	0\\
15.14	0\\
15.15	0\\
15.16	0\\
15.17	0\\
15.18	0\\
15.19	0\\
15.2	0\\
15.21	0\\
15.22	0\\
15.23	0\\
15.24	0\\
15.25	0\\
15.26	0\\
15.27	0\\
15.28	0\\
15.29	0\\
15.3	0\\
15.31	0\\
15.32	0\\
15.33	0\\
15.34	0\\
15.35	0\\
15.36	0\\
15.37	0\\
15.38	0\\
15.39	0\\
15.4	0\\
15.41	0\\
15.42	0\\
15.43	0\\
15.44	0\\
15.45	0\\
15.46	0\\
15.47	0\\
15.48	0\\
15.49	0\\
15.5	0\\
15.51	0\\
15.52	0\\
15.53	0\\
15.54	0\\
15.55	0\\
15.56	0\\
15.57	0\\
15.58	0\\
15.59	0\\
15.6	0\\
15.61	0\\
15.62	0\\
15.63	0\\
15.64	0\\
15.65	0\\
15.66	0\\
15.67	0\\
15.68	0\\
15.69	0\\
15.7	0\\
15.71	0\\
15.72	0\\
15.73	0\\
15.74	0\\
15.75	0\\
15.76	0\\
15.77	0\\
15.78	0\\
15.79	0\\
15.8	0\\
15.81	0\\
15.82	0\\
15.83	0\\
15.84	0\\
15.85	0\\
15.86	0\\
15.87	0\\
15.88	0\\
15.89	0\\
15.9	0\\
15.91	0\\
15.92	0\\
15.93	0\\
15.94	0\\
15.95	0\\
15.96	0\\
15.97	0\\
15.98	0\\
15.99	0\\
16	0\\
16.01	0\\
16.02	0\\
16.03	0\\
16.04	0\\
16.05	0\\
16.06	0\\
16.07	0\\
16.08	0\\
16.09	0\\
16.1	0\\
16.11	0\\
16.12	0\\
16.13	0\\
16.14	0\\
16.15	0\\
16.16	0\\
16.17	0\\
16.18	0\\
16.19	0\\
16.2	0\\
16.21	0\\
16.22	0\\
16.23	0\\
16.24	0\\
16.25	0\\
16.26	0\\
16.27	0\\
16.28	0\\
16.29	0\\
16.3	0\\
16.31	0\\
16.32	0\\
16.33	0\\
16.34	0\\
16.35	0\\
16.36	0\\
16.37	0\\
16.38	0\\
16.39	0\\
16.4	0\\
16.41	0\\
16.42	0\\
16.43	0\\
16.44	0\\
16.45	0\\
16.46	0\\
16.47	0\\
16.48	0\\
16.49	0\\
16.5	0\\
16.51	0\\
16.52	0\\
16.53	0\\
16.54	0\\
16.55	0\\
16.56	0\\
16.57	0\\
16.58	0\\
16.59	0\\
16.6	0\\
16.61	0\\
16.62	0\\
16.63	0\\
16.64	0\\
16.65	0\\
16.66	0\\
16.67	0\\
16.68	0\\
16.69	0\\
16.7	0\\
16.71	0\\
16.72	0\\
16.73	0\\
16.74	0\\
16.75	0\\
16.76	0\\
16.77	0\\
16.78	0\\
16.79	0\\
16.8	0\\
16.81	0\\
16.82	0\\
16.83	0\\
16.84	0\\
16.85	0\\
16.86	0\\
16.87	0\\
16.88	0\\
16.89	0\\
16.9	0\\
16.91	0\\
16.92	0\\
16.93	0\\
16.94	0\\
16.95	0\\
16.96	0\\
16.97	0\\
16.98	0\\
16.99	0\\
17	0\\
17.01	0\\
17.02	0\\
17.03	0\\
17.04	0\\
17.05	0\\
17.06	0\\
17.07	0\\
17.08	0\\
17.09	0\\
17.1	0\\
17.11	0\\
17.12	0\\
17.13	0\\
17.14	0\\
17.15	0\\
17.16	0\\
17.17	0\\
17.18	0\\
17.19	0\\
17.2	0\\
17.21	0\\
17.22	0\\
17.23	0\\
17.24	0\\
17.25	0\\
17.26	0\\
17.27	0\\
17.28	0\\
17.29	0\\
17.3	0\\
17.31	0\\
17.32	0\\
17.33	0\\
17.34	0\\
17.35	0\\
17.36	0\\
17.37	0\\
17.38	0\\
17.39	0\\
17.4	0\\
17.41	0\\
17.42	0\\
17.43	0\\
17.44	0\\
17.45	0\\
17.46	0\\
17.47	0\\
17.48	0\\
17.49	0\\
17.5	0\\
17.51	0\\
17.52	0\\
17.53	0\\
17.54	0\\
17.55	0\\
17.56	0\\
17.57	0\\
17.58	0\\
17.59	0\\
17.6	0\\
17.61	0\\
17.62	0\\
17.63	0\\
17.64	0\\
17.65	0\\
17.66	0\\
17.67	0\\
17.68	0\\
17.69	0\\
17.7	0\\
17.71	0\\
17.72	0\\
17.73	0\\
17.74	0\\
17.75	0\\
17.76	0\\
17.77	0\\
17.78	0\\
17.79	0\\
17.8	0\\
17.81	0\\
17.82	0\\
17.83	0\\
17.84	0\\
17.85	0\\
17.86	0\\
17.87	0\\
17.88	0\\
17.89	0\\
17.9	0\\
17.91	0\\
17.92	0\\
17.93	0\\
17.94	0\\
17.95	0\\
17.96	0\\
17.97	0\\
17.98	0\\
17.99	0\\
18	0\\
18.01	0\\
18.02	0\\
18.03	0\\
18.04	0\\
18.05	0\\
18.06	0\\
18.07	0\\
18.08	0\\
18.09	0\\
18.1	0\\
18.11	0\\
18.12	0\\
18.13	0\\
18.14	0\\
18.15	0\\
18.16	0\\
18.17	0\\
18.18	0\\
18.19	0\\
18.2	0\\
18.21	0\\
18.22	0\\
18.23	0\\
18.24	0\\
18.25	0\\
18.26	0\\
18.27	0\\
18.28	0\\
18.29	0\\
18.3	0\\
18.31	0\\
18.32	0\\
18.33	0\\
18.34	0\\
18.35	0\\
18.36	0\\
18.37	0\\
18.38	0\\
18.39	0\\
18.4	0\\
18.41	0\\
18.42	0\\
18.43	0\\
18.44	0\\
18.45	0\\
18.46	0\\
18.47	0\\
18.48	0\\
18.49	0\\
18.5	0\\
18.51	0\\
18.52	0\\
18.53	0\\
18.54	0\\
18.55	0\\
18.56	0\\
18.57	0\\
18.58	0\\
18.59	0\\
18.6	0\\
18.61	0\\
18.62	0\\
18.63	0\\
18.64	0\\
18.65	0\\
18.66	0\\
18.67	0\\
18.68	0\\
18.69	0\\
18.7	0\\
18.71	0\\
18.72	0\\
18.73	0\\
18.74	0\\
18.75	0\\
18.76	0\\
18.77	0\\
18.78	0\\
18.79	0\\
18.8	0\\
18.81	0\\
18.82	0\\
18.83	0\\
18.84	0\\
18.85	0\\
18.86	0\\
18.87	0\\
18.88	0\\
18.89	0\\
18.9	0\\
18.91	0\\
18.92	0\\
18.93	0\\
18.94	0\\
18.95	0\\
18.96	0\\
18.97	0\\
18.98	0\\
18.99	0\\
19	0\\
19.01	0\\
19.02	0\\
19.03	0\\
19.04	0\\
19.05	0\\
19.06	0\\
19.07	0\\
19.08	0\\
19.09	0\\
19.1	0\\
19.11	0\\
19.12	0\\
19.13	0\\
19.14	0\\
19.15	0\\
19.16	0\\
19.17	0\\
19.18	0\\
19.19	0\\
19.2	0\\
19.21	0\\
19.22	0\\
19.23	0\\
19.24	0\\
19.25	0\\
19.26	0\\
19.27	0\\
19.28	0\\
19.29	0\\
19.3	0\\
19.31	0\\
19.32	0\\
19.33	0\\
19.34	0\\
19.35	0\\
19.36	0\\
19.37	0\\
19.38	0\\
19.39	0\\
19.4	0\\
19.41	0\\
19.42	0\\
19.43	0\\
19.44	0\\
19.45	0\\
19.46	0\\
19.47	0\\
19.48	0\\
19.49	0\\
19.5	0\\
19.51	0\\
19.52	0\\
19.53	0\\
19.54	0\\
19.55	0\\
19.56	0\\
19.57	0\\
19.58	0\\
19.59	0\\
19.6	0\\
19.61	0\\
19.62	0\\
19.63	0\\
19.64	0\\
19.65	0\\
19.66	0\\
19.67	0\\
19.68	0\\
19.69	0\\
19.7	0\\
19.71	0\\
19.72	0\\
19.73	0\\
19.74	0\\
19.75	0\\
19.76	0\\
19.77	0\\
19.78	0\\
19.79	0\\
19.8	0\\
19.81	0\\
19.82	0\\
19.83	0\\
19.84	0\\
19.85	0\\
19.86	0\\
19.87	0\\
19.88	0\\
19.89	0\\
19.9	0\\
19.91	0\\
19.92	0\\
19.93	0\\
19.94	0\\
19.95	0\\
19.96	0\\
19.97	0\\
19.98	0\\
19.99	0\\
20	0\\
20.01	0\\
20.02	0\\
20.03	0\\
20.04	0\\
20.05	0\\
20.06	0\\
20.07	0\\
20.08	0\\
20.09	0\\
20.1	0\\
20.11	0\\
20.12	0\\
20.13	0\\
20.14	0\\
20.15	0\\
20.16	0\\
20.17	0\\
20.18	0\\
20.19	0\\
20.2	0\\
20.21	0\\
20.22	0\\
20.23	0\\
20.24	0\\
20.25	0\\
20.26	0\\
20.27	0\\
20.28	0\\
20.29	0\\
20.3	0\\
20.31	0\\
20.32	0\\
20.33	0\\
20.34	0\\
20.35	0\\
20.36	0\\
20.37	0\\
20.38	0\\
20.39	0\\
20.4	0\\
20.41	0\\
20.42	0\\
20.43	0\\
20.44	0\\
20.45	0\\
20.46	0\\
20.47	0\\
20.48	0\\
20.49	0\\
20.5	0\\
20.51	0\\
20.52	0\\
20.53	0\\
20.54	0\\
20.55	0\\
20.56	0\\
20.57	0\\
20.58	0\\
20.59	0\\
20.6	0\\
20.61	0\\
20.62	0\\
20.63	0\\
20.64	0\\
20.65	0\\
20.66	0\\
20.67	0\\
20.68	0\\
20.69	0\\
20.7	0\\
20.71	0\\
20.72	0\\
20.73	0\\
20.74	0\\
20.75	0\\
20.76	0\\
20.77	0\\
20.78	0\\
20.79	0\\
20.8	0\\
20.81	0\\
20.82	0\\
20.83	0\\
20.84	0\\
20.85	0\\
20.86	0\\
20.87	0\\
20.88	0\\
20.89	0\\
20.9	0\\
20.91	0\\
20.92	0\\
20.93	0\\
20.94	0\\
20.95	0\\
20.96	0\\
20.97	0\\
20.98	0\\
20.99	0\\
21	0\\
21.01	0\\
21.02	0\\
21.03	0\\
21.04	0\\
21.05	0\\
21.06	0\\
21.07	0\\
21.08	0\\
21.09	0\\
21.1	0\\
21.11	0\\
21.12	0\\
21.13	0\\
21.14	0\\
21.15	0\\
21.16	0\\
21.17	0\\
21.18	0\\
21.19	0\\
21.2	0\\
21.21	0\\
21.22	0\\
21.23	0\\
21.24	0\\
21.25	0\\
21.26	0\\
21.27	0\\
21.28	0\\
21.29	0\\
21.3	0\\
21.31	0\\
21.32	0\\
21.33	0\\
21.34	0\\
21.35	0\\
21.36	0\\
21.37	0\\
21.38	0\\
21.39	0\\
21.4	0\\
21.41	0\\
21.42	0\\
21.43	0\\
21.44	0\\
21.45	0\\
21.46	0\\
21.47	0\\
21.48	0\\
21.49	0\\
21.5	0\\
21.51	0\\
21.52	0\\
21.53	0\\
21.54	0\\
21.55	0\\
21.56	0\\
21.57	0\\
21.58	0\\
21.59	0\\
21.6	0\\
21.61	0\\
21.62	0\\
21.63	0\\
21.64	0\\
21.65	0\\
21.66	0\\
21.67	0\\
21.68	0\\
21.69	0\\
21.7	0\\
21.71	0\\
21.72	0\\
21.73	0\\
21.74	0\\
21.75	0\\
21.76	0\\
21.77	0\\
21.78	0\\
21.79	0\\
21.8	0\\
21.81	0\\
21.82	0\\
21.83	0\\
21.84	0\\
21.85	0\\
21.86	0\\
21.87	0\\
21.88	0\\
21.89	0\\
21.9	0\\
21.91	0\\
21.92	0\\
21.93	0\\
21.94	0\\
21.95	0\\
21.96	0\\
21.97	0\\
21.98	0\\
21.99	0\\
22	0\\
22.01	0\\
22.02	0\\
22.03	0\\
22.04	0\\
22.05	0\\
22.06	0\\
22.07	0\\
22.08	0\\
22.09	0\\
22.1	0\\
22.11	0\\
22.12	0\\
22.13	0\\
22.14	0\\
22.15	0\\
22.16	0\\
22.17	0\\
22.18	0\\
22.19	0\\
22.2	0\\
22.21	0\\
22.22	0\\
22.23	0\\
22.24	0\\
22.25	0\\
22.26	0\\
22.27	0\\
22.28	0\\
22.29	0\\
22.3	0\\
22.31	0\\
22.32	0\\
22.33	0\\
22.34	0\\
22.35	0\\
22.36	0\\
22.37	0\\
22.38	0\\
22.39	0\\
22.4	0\\
22.41	0\\
22.42	0\\
22.43	0\\
22.44	0\\
22.45	0\\
22.46	0\\
22.47	0\\
22.48	0\\
22.49	0\\
22.5	0\\
22.51	0\\
22.52	0\\
22.53	0\\
22.54	0\\
22.55	0\\
22.56	0\\
22.57	0\\
22.58	0\\
22.59	0\\
22.6	0\\
22.61	0\\
22.62	0\\
22.63	0\\
22.64	0\\
22.65	0\\
22.66	0\\
22.67	0\\
22.68	0\\
22.69	0\\
22.7	0\\
22.71	0\\
22.72	0\\
22.73	0\\
22.74	0\\
22.75	0\\
22.76	0\\
22.77	0\\
22.78	0\\
22.79	0\\
22.8	0\\
22.81	0\\
22.82	0\\
22.83	0\\
22.84	0\\
22.85	0\\
22.86	0\\
22.87	0\\
22.88	0\\
22.89	0\\
22.9	0\\
22.91	0\\
22.92	0\\
22.93	0\\
22.94	0\\
22.95	0\\
22.96	0\\
22.97	0\\
22.98	0\\
22.99	0\\
23	0\\
23.01	0\\
23.02	0\\
23.03	0\\
23.04	0\\
23.05	0\\
23.06	0\\
23.07	0\\
23.08	0\\
23.09	0\\
23.1	0\\
23.11	0\\
23.12	0\\
23.13	0\\
23.14	0\\
23.15	0\\
23.16	0\\
23.17	0\\
23.18	0\\
23.19	0\\
23.2	0\\
23.21	0\\
23.22	0\\
23.23	0\\
23.24	0\\
23.25	0\\
23.26	0\\
23.27	0\\
23.28	0\\
23.29	0\\
23.3	0\\
23.31	0\\
23.32	0\\
23.33	0\\
23.34	0\\
23.35	0\\
23.36	0\\
23.37	0\\
23.38	0\\
23.39	0\\
23.4	0\\
23.41	0\\
23.42	0\\
23.43	0\\
23.44	0\\
23.45	0\\
23.46	0\\
23.47	0\\
23.48	0\\
23.49	0\\
23.5	0\\
23.51	0\\
23.52	0\\
23.53	0\\
23.54	0\\
23.55	0\\
23.56	0\\
23.57	0\\
23.58	0\\
23.59	0\\
23.6	0\\
23.61	0\\
23.62	0\\
23.63	0\\
23.64	0\\
23.65	0\\
23.66	0\\
23.67	0\\
23.68	0\\
23.69	0\\
23.7	0\\
23.71	0\\
23.72	0\\
23.73	0\\
23.74	0\\
23.75	0\\
23.76	0\\
23.77	0\\
23.78	0\\
23.79	0\\
23.8	0\\
23.81	0\\
23.82	0\\
23.83	0\\
23.84	0\\
23.85	0\\
23.86	0\\
23.87	0\\
23.88	0\\
23.89	0\\
23.9	0\\
23.91	0\\
23.92	0\\
23.93	0\\
23.94	0\\
23.95	0\\
23.96	0\\
23.97	0\\
23.98	0\\
23.99	0\\
24	0\\
24.01	0\\
24.02	0\\
24.03	0\\
24.04	0\\
24.05	0\\
24.06	0\\
24.07	0\\
24.08	0\\
24.09	0\\
24.1	0\\
24.11	0\\
24.12	0\\
24.13	0\\
24.14	0\\
24.15	0\\
24.16	0\\
24.17	0\\
24.18	0\\
24.19	0\\
24.2	0\\
24.21	0\\
24.22	0\\
24.23	0\\
24.24	0\\
24.25	0\\
24.26	0\\
24.27	0\\
24.28	0\\
24.29	0\\
24.3	0\\
24.31	0\\
24.32	0\\
24.33	0\\
24.34	0\\
24.35	0\\
24.36	0\\
24.37	0\\
24.38	0\\
24.39	0\\
24.4	0\\
24.41	0\\
24.42	0\\
24.43	0\\
24.44	0\\
24.45	0\\
24.46	0\\
24.47	0\\
24.48	0\\
24.49	0\\
24.5	0\\
24.51	0\\
24.52	0\\
24.53	0\\
24.54	0\\
24.55	0\\
24.56	0\\
24.57	0\\
24.58	0\\
24.59	0\\
24.6	0\\
24.61	0\\
24.62	0\\
24.63	0\\
24.64	0\\
24.65	0\\
24.66	0\\
24.67	0\\
24.68	0\\
24.69	0\\
24.7	0\\
24.71	0\\
24.72	0\\
24.73	0\\
24.74	0\\
24.75	0\\
24.76	0\\
24.77	0\\
24.78	0\\
24.79	0\\
24.8	0\\
24.81	0\\
24.82	0\\
24.83	0\\
24.84	0\\
24.85	0\\
24.86	0\\
24.87	0\\
24.88	0\\
24.89	0\\
24.9	0\\
24.91	0\\
24.92	0\\
24.93	0\\
24.94	0\\
24.95	0\\
24.96	0\\
24.97	0\\
24.98	0\\
24.99	0\\
25	0\\
25.01	0\\
25.02	0\\
25.03	0\\
25.04	0\\
25.05	0\\
25.06	0\\
25.07	0\\
25.08	0\\
25.09	0\\
25.1	0\\
25.11	0\\
25.12	0\\
25.13	0\\
25.14	0\\
25.15	0\\
25.16	0\\
25.17	0\\
25.18	0\\
25.19	0\\
25.2	0\\
25.21	0\\
25.22	0\\
25.23	0\\
25.24	0\\
25.25	0\\
25.26	0\\
25.27	0\\
25.28	0\\
25.29	0\\
25.3	0\\
25.31	0\\
25.32	0\\
25.33	0\\
25.34	0\\
25.35	0\\
25.36	0\\
25.37	0\\
25.38	0\\
25.39	0\\
25.4	0\\
25.41	0\\
25.42	0\\
25.43	0\\
25.44	0\\
25.45	0\\
25.46	0\\
25.47	0\\
25.48	0\\
25.49	0\\
25.5	0\\
25.51	0\\
25.52	0\\
25.53	0\\
25.54	0\\
25.55	0\\
25.56	0\\
25.57	0\\
25.58	0\\
25.59	0\\
25.6	0\\
25.61	0\\
25.62	0\\
25.63	0\\
25.64	0\\
25.65	0\\
25.66	0\\
25.67	0\\
25.68	0\\
25.69	0\\
25.7	0\\
25.71	0\\
25.72	0\\
25.73	0\\
25.74	0\\
25.75	0\\
25.76	0\\
25.77	0\\
25.78	0\\
25.79	0\\
25.8	0\\
25.81	0\\
25.82	0\\
25.83	0\\
25.84	0\\
25.85	0\\
25.86	0\\
25.87	0\\
25.88	0\\
25.89	0\\
25.9	0\\
25.91	0\\
25.92	0\\
25.93	0\\
25.94	0\\
25.95	0\\
25.96	0\\
25.97	0\\
25.98	0\\
25.99	0\\
26	0\\
26.01	0\\
26.02	0\\
26.03	0\\
26.04	0\\
26.05	0\\
26.06	0\\
26.07	0\\
26.08	0\\
26.09	0\\
26.1	0\\
26.11	0\\
26.12	0\\
26.13	0\\
26.14	0\\
26.15	0\\
26.16	0\\
26.17	0\\
26.18	0\\
26.19	0\\
26.2	0\\
26.21	0\\
26.22	0\\
26.23	0\\
26.24	0\\
26.25	0\\
26.26	0\\
26.27	0\\
26.28	0\\
26.29	0\\
26.3	0\\
26.31	0\\
26.32	0\\
26.33	0\\
26.34	0\\
26.35	0\\
26.36	0\\
26.37	0\\
26.38	0\\
26.39	0\\
26.4	0\\
26.41	0\\
26.42	0\\
26.43	0\\
26.44	0\\
26.45	0\\
26.46	0\\
26.47	0\\
26.48	0\\
26.49	0\\
26.5	0\\
26.51	0\\
26.52	0\\
26.53	0\\
26.54	0\\
26.55	0\\
26.56	0\\
26.57	0\\
26.58	0\\
26.59	0\\
26.6	0\\
26.61	0\\
26.62	0\\
26.63	0\\
26.64	0\\
26.65	0\\
26.66	0\\
26.67	0\\
26.68	0\\
26.69	0\\
26.7	0\\
26.71	0\\
26.72	0\\
26.73	0\\
26.74	0\\
26.75	0\\
26.76	0\\
26.77	0\\
26.78	0\\
26.79	0\\
26.8	0\\
26.81	0\\
26.82	0\\
26.83	0\\
26.84	0\\
26.85	0\\
26.86	0\\
26.87	0\\
26.88	0\\
26.89	0\\
26.9	0\\
26.91	0\\
26.92	0\\
26.93	0\\
26.94	0\\
26.95	0\\
26.96	0\\
26.97	0\\
26.98	0\\
26.99	0\\
27	0\\
27.01	0\\
27.02	0\\
27.03	0\\
27.04	0\\
27.05	0\\
27.06	0\\
27.07	0\\
27.08	0\\
27.09	0\\
27.1	0\\
27.11	0\\
27.12	0\\
27.13	0\\
27.14	0\\
27.15	0\\
27.16	0\\
27.17	0\\
27.18	0\\
27.19	0\\
27.2	0\\
27.21	0\\
27.22	0\\
27.23	0\\
27.24	0\\
27.25	0\\
27.26	0\\
27.27	0\\
27.28	0\\
27.29	0\\
27.3	0\\
27.31	0\\
27.32	0\\
27.33	0\\
27.34	0\\
27.35	0\\
27.36	0\\
27.37	0\\
27.38	0\\
27.39	0\\
27.4	0\\
27.41	0\\
27.42	0\\
27.43	0\\
27.44	0\\
27.45	0\\
27.46	0\\
27.47	0\\
27.48	0\\
27.49	0\\
27.5	0\\
27.51	0\\
27.52	0\\
27.53	0\\
27.54	0\\
27.55	0\\
27.56	0\\
27.57	0\\
27.58	0\\
27.59	0\\
27.6	0\\
27.61	0\\
27.62	0\\
27.63	0\\
27.64	0\\
27.65	0\\
27.66	0\\
27.67	0\\
27.68	0\\
27.69	0\\
27.7	0\\
27.71	0\\
27.72	0\\
27.73	0\\
27.74	0\\
27.75	0\\
27.76	0\\
27.77	0\\
27.78	0\\
27.79	0\\
27.8	0\\
27.81	0\\
27.82	0\\
27.83	0\\
27.84	0\\
27.85	0\\
27.86	0\\
27.87	0\\
27.88	0\\
27.89	0\\
27.9	0\\
27.91	0\\
27.92	0\\
27.93	0\\
27.94	0\\
27.95	0\\
27.96	0\\
27.97	0\\
27.98	0\\
27.99	0\\
28	0\\
28.01	0\\
28.02	0\\
28.03	0\\
28.04	0\\
28.05	0\\
28.06	0\\
28.07	0\\
28.08	0\\
28.09	0\\
28.1	0\\
28.11	0\\
28.12	0\\
28.13	0\\
28.14	0\\
28.15	0\\
28.16	0\\
28.17	0\\
28.18	0\\
28.19	0\\
28.2	0\\
28.21	0\\
28.22	0\\
28.23	0\\
28.24	0\\
28.25	0\\
28.26	0\\
28.27	0\\
28.28	0\\
28.29	0\\
28.3	0\\
28.31	0\\
28.32	0\\
28.33	0\\
28.34	0\\
28.35	0\\
28.36	0\\
28.37	0\\
28.38	0\\
28.39	0\\
28.4	0\\
28.41	0\\
28.42	0\\
28.43	0\\
28.44	0\\
28.45	0\\
28.46	0\\
28.47	0\\
28.48	0\\
28.49	0\\
28.5	0\\
28.51	0\\
28.52	0\\
28.53	0\\
28.54	0\\
28.55	0\\
28.56	0\\
28.57	0\\
28.58	0\\
28.59	0\\
28.6	0\\
28.61	0\\
28.62	0\\
28.63	0\\
28.64	0\\
28.65	0\\
28.66	0\\
28.67	0\\
28.68	0\\
28.69	0\\
28.7	0\\
28.71	0\\
28.72	0\\
28.73	0\\
28.74	0\\
28.75	0\\
28.76	0\\
28.77	0\\
28.78	0\\
28.79	0\\
28.8	0\\
28.81	0\\
28.82	0\\
28.83	0\\
28.84	0\\
28.85	0\\
28.86	0\\
28.87	0\\
28.88	0\\
28.89	0\\
28.9	0\\
28.91	0\\
28.92	0\\
28.93	0\\
28.94	0\\
28.95	0\\
28.96	0\\
28.97	0\\
28.98	0\\
28.99	0\\
29	0\\
29.01	0\\
29.02	0\\
29.03	0\\
29.04	0\\
29.05	0\\
29.06	0\\
29.07	0\\
29.08	0\\
29.09	0\\
29.1	0\\
29.11	0\\
29.12	0\\
29.13	0\\
29.14	0\\
29.15	0\\
29.16	0\\
29.17	0\\
29.18	0\\
29.19	0\\
29.2	0\\
29.21	0\\
29.22	0\\
29.23	0\\
29.24	0\\
29.25	0\\
29.26	0\\
29.27	0\\
29.28	0\\
29.29	0\\
29.3	0\\
29.31	0\\
29.32	0\\
29.33	0\\
29.34	0\\
29.35	0\\
29.36	0\\
29.37	0\\
29.38	0\\
29.39	0\\
29.4	0\\
29.41	0\\
29.42	0\\
29.43	0\\
29.44	0\\
29.45	0\\
29.46	0\\
29.47	0\\
29.48	0\\
29.49	0\\
29.5	0\\
29.51	0\\
29.52	0\\
29.53	0\\
29.54	0\\
29.55	0\\
29.56	0\\
29.57	0\\
29.58	0\\
29.59	0\\
29.6	0\\
29.61	0\\
29.62	0\\
29.63	0\\
29.64	0\\
29.65	0\\
29.66	0\\
29.67	0\\
29.68	0\\
29.69	0\\
29.7	0\\
29.71	0\\
29.72	0\\
29.73	0\\
29.74	0\\
29.75	0\\
29.76	0\\
29.77	0\\
29.78	0\\
29.79	0\\
29.8	0\\
29.81	0\\
29.82	0\\
29.83	0\\
29.84	0\\
29.85	0\\
29.86	0\\
29.87	0\\
29.88	0\\
29.89	0\\
29.9	0\\
29.91	0\\
29.92	0\\
29.93	0\\
29.94	0\\
29.95	0\\
29.96	0\\
29.97	0\\
29.98	0\\
29.99	0\\
30	0\\
30.01	0\\
30.02	0\\
30.03	0\\
30.04	0\\
30.05	0\\
30.06	0\\
30.07	0\\
30.08	0\\
30.09	0\\
30.1	0\\
30.11	0\\
30.12	0\\
30.13	0\\
30.14	0\\
30.15	0\\
30.16	0\\
30.17	0\\
30.18	0\\
30.19	0\\
30.2	0\\
30.21	0\\
30.22	0\\
30.23	0\\
30.24	0\\
30.25	0\\
30.26	0\\
30.27	0\\
30.28	0\\
30.29	0\\
30.3	0\\
30.31	0\\
30.32	0\\
30.33	0\\
30.34	0\\
30.35	0\\
30.36	0\\
30.37	0\\
30.38	0\\
30.39	0\\
30.4	0\\
30.41	0\\
30.42	0\\
30.43	0\\
30.44	0\\
30.45	0\\
30.46	0\\
30.47	0\\
30.48	0\\
30.49	0\\
30.5	0\\
30.51	0\\
30.52	0\\
30.53	0\\
30.54	0\\
30.55	0\\
30.56	0\\
30.57	0\\
30.58	0\\
30.59	0\\
30.6	0\\
30.61	0\\
30.62	0\\
30.63	0\\
30.64	0\\
30.65	0\\
30.66	0\\
30.67	0\\
30.68	0\\
30.69	0\\
30.7	0\\
30.71	0\\
30.72	0\\
30.73	0\\
30.74	0\\
30.75	0\\
30.76	0\\
30.77	0\\
30.78	0\\
30.79	0\\
30.8	0\\
30.81	0\\
30.82	0\\
30.83	0\\
30.84	0\\
30.85	0\\
30.86	0\\
30.87	0\\
30.88	0\\
30.89	0\\
30.9	0\\
30.91	0\\
30.92	0\\
30.93	0\\
30.94	0\\
30.95	0\\
30.96	0\\
30.97	0\\
30.98	0\\
30.99	0\\
31	0\\
31.01	0\\
31.02	0\\
31.03	0\\
31.04	0\\
31.05	0\\
31.06	0\\
31.07	0\\
31.08	0\\
31.09	0\\
31.1	0\\
31.11	0\\
31.12	0\\
31.13	0\\
31.14	0\\
31.15	0\\
31.16	0\\
31.17	0\\
31.18	0\\
31.19	0\\
31.2	0\\
31.21	0\\
31.22	0\\
31.23	0\\
31.24	0\\
31.25	0\\
31.26	0\\
31.27	0\\
31.28	0\\
31.29	0\\
31.3	0\\
31.31	0\\
31.32	0\\
31.33	0\\
31.34	0\\
31.35	0\\
31.36	0\\
31.37	0\\
31.38	0\\
31.39	0\\
31.4	0\\
31.41	0\\
31.42	0\\
31.43	0\\
31.44	0\\
31.45	0\\
31.46	0\\
31.47	0\\
31.48	0\\
31.49	0\\
31.5	0\\
31.51	0\\
31.52	0\\
31.53	0\\
31.54	0\\
31.55	0\\
31.56	0\\
31.57	0\\
31.58	0\\
31.59	0\\
31.6	0\\
31.61	0\\
31.62	0\\
31.63	0\\
31.64	0\\
31.65	0\\
31.66	0\\
31.67	0\\
31.68	0\\
31.69	0\\
31.7	0\\
31.71	0\\
31.72	0\\
31.73	0\\
31.74	0\\
31.75	0\\
31.76	0\\
31.77	0\\
31.78	0\\
31.79	0\\
31.8	0\\
31.81	0\\
31.82	0\\
31.83	0\\
31.84	0\\
31.85	0\\
31.86	0\\
31.87	0\\
31.88	0\\
31.89	0\\
31.9	0\\
31.91	0\\
31.92	0\\
31.93	0\\
31.94	0\\
31.95	0\\
31.96	0\\
31.97	0\\
31.98	0\\
31.99	0\\
32	0\\
32.01	0\\
32.02	0\\
32.03	0\\
32.04	0\\
32.05	0\\
32.06	0\\
32.07	0\\
32.08	0\\
32.09	0\\
32.1	0\\
32.11	0\\
32.12	0\\
32.13	0\\
32.14	0\\
32.15	0\\
32.16	0\\
32.17	0\\
32.18	0\\
32.19	0\\
32.2	0\\
32.21	0\\
32.22	0\\
32.23	0\\
32.24	0\\
32.25	0\\
32.26	0\\
32.27	0\\
32.28	0\\
32.29	0\\
32.3	0\\
32.31	0\\
32.32	0\\
32.33	0\\
32.34	0\\
32.35	0\\
32.36	0\\
32.37	0\\
32.38	0\\
32.39	0\\
32.4	0\\
32.41	0\\
32.42	0\\
32.43	0\\
32.44	0\\
32.45	0\\
32.46	0\\
32.47	0\\
32.48	0\\
32.49	0\\
32.5	0\\
32.51	0\\
32.52	0\\
32.53	0\\
32.54	0\\
32.55	0\\
32.56	0\\
32.57	0\\
32.58	0\\
32.59	0\\
32.6	0\\
32.61	0\\
32.62	0\\
32.63	0\\
32.64	0\\
32.65	0\\
32.66	0\\
32.67	0\\
32.68	0\\
32.69	0\\
32.7	0\\
32.71	0\\
32.72	0\\
32.73	0\\
32.74	0\\
32.75	0\\
32.76	0\\
32.77	0\\
32.78	0\\
32.79	0\\
32.8	0\\
32.81	0\\
32.82	0\\
32.83	0\\
32.84	0\\
32.85	0\\
32.86	0\\
32.87	0\\
32.88	0\\
32.89	0\\
32.9	0\\
32.91	0\\
32.92	0\\
32.93	0\\
32.94	0\\
32.95	0\\
32.96	0\\
32.97	0\\
32.98	0\\
32.99	0\\
33	0\\
33.01	0\\
33.02	0\\
33.03	0\\
33.04	0\\
33.05	0\\
33.06	0\\
33.07	0\\
33.08	0\\
33.09	0\\
33.1	0\\
33.11	0\\
33.12	0\\
33.13	0\\
33.14	0\\
33.15	0\\
33.16	0\\
33.17	0\\
33.18	0\\
33.19	0\\
33.2	0\\
33.21	0\\
33.22	0\\
33.23	0\\
33.24	0\\
33.25	0\\
33.26	0\\
33.27	0\\
33.28	0\\
33.29	0\\
33.3	0\\
33.31	0\\
33.32	0\\
33.33	0\\
33.34	0\\
33.35	0\\
33.36	0\\
33.37	0\\
33.38	0\\
33.39	0\\
33.4	0\\
33.41	0\\
33.42	0\\
33.43	0\\
33.44	0\\
33.45	0\\
33.46	0\\
33.47	0\\
33.48	0\\
33.49	0\\
33.5	0\\
33.51	0\\
33.52	0\\
33.53	0\\
33.54	0\\
33.55	0\\
33.56	0\\
33.57	0\\
33.58	0\\
33.59	0\\
33.6	0\\
33.61	0\\
33.62	0\\
33.63	0\\
33.64	0\\
33.65	0\\
33.66	0\\
33.67	0\\
33.68	0\\
33.69	0\\
33.7	0\\
33.71	0\\
33.72	0\\
33.73	0\\
33.74	0\\
33.75	0\\
33.76	0\\
33.77	0\\
33.78	0\\
33.79	0\\
33.8	0\\
33.81	0\\
33.82	0\\
33.83	0\\
33.84	0\\
33.85	0\\
33.86	0\\
33.87	0\\
33.88	0\\
33.89	0\\
33.9	0\\
33.91	0\\
33.92	0\\
33.93	0\\
33.94	0\\
33.95	0\\
33.96	0\\
33.97	0\\
33.98	0\\
33.99	0\\
34	0\\
34.01	0\\
34.02	0\\
34.03	0\\
34.04	0\\
34.05	0\\
34.06	0\\
34.07	0\\
34.08	0\\
34.09	0\\
34.1	0\\
34.11	0\\
34.12	0\\
34.13	0\\
34.14	0\\
34.15	0\\
34.16	0\\
34.17	0\\
34.18	0\\
34.19	0\\
34.2	0\\
34.21	0\\
34.22	0\\
34.23	0\\
34.24	0\\
34.25	0\\
34.26	0\\
34.27	0\\
34.28	0\\
34.29	0\\
34.3	0\\
34.31	0\\
34.32	0\\
34.33	0\\
34.34	0\\
34.35	0\\
34.36	0\\
34.37	0\\
34.38	0\\
34.39	0\\
34.4	0\\
34.41	0\\
34.42	0\\
34.43	0\\
34.44	0\\
34.45	0\\
34.46	0\\
34.47	0\\
34.48	0\\
34.49	0\\
34.5	0\\
34.51	0\\
34.52	0\\
34.53	0\\
34.54	0\\
34.55	0\\
34.56	0\\
34.57	0\\
34.58	0\\
34.59	0\\
34.6	0\\
34.61	0\\
34.62	0\\
34.63	0\\
34.64	0\\
34.65	0\\
34.66	0\\
34.67	0\\
34.68	0\\
34.69	0\\
34.7	0\\
34.71	0\\
34.72	0\\
34.73	0\\
34.74	0\\
34.75	0\\
34.76	0\\
34.77	0\\
34.78	0\\
34.79	0\\
34.8	0\\
34.81	0\\
34.82	0\\
34.83	0\\
34.84	0\\
34.85	0\\
34.86	0\\
34.87	0\\
34.88	0\\
34.89	0\\
34.9	0\\
34.91	0\\
34.92	0\\
34.93	0\\
34.94	0\\
34.95	0\\
34.96	0\\
34.97	0\\
34.98	0\\
34.99	0\\
35	0\\
35.01	0\\
35.02	0\\
35.03	0\\
35.04	0\\
35.05	0\\
35.06	0\\
35.07	0\\
35.08	0\\
35.09	0\\
35.1	0\\
35.11	0\\
35.12	0\\
35.13	0\\
35.14	0\\
35.15	0\\
35.16	0\\
35.17	0\\
35.18	0\\
35.19	0\\
35.2	0\\
35.21	0\\
35.22	0\\
35.23	0\\
35.24	0\\
35.25	0\\
35.26	0\\
35.27	0\\
35.28	0\\
35.29	0\\
35.3	0\\
35.31	0\\
35.32	0\\
35.33	0\\
35.34	0\\
35.35	0\\
35.36	0\\
35.37	0\\
35.38	0\\
35.39	0\\
35.4	0\\
35.41	0\\
35.42	0\\
35.43	0\\
35.44	0\\
35.45	0\\
35.46	0\\
35.47	0\\
35.48	0\\
35.49	0\\
35.5	0\\
35.51	0\\
35.52	0\\
35.53	0\\
35.54	0\\
35.55	0\\
35.56	0\\
35.57	0\\
35.58	0\\
35.59	0\\
35.6	0\\
35.61	0\\
35.62	0\\
35.63	0\\
35.64	0\\
35.65	0\\
35.66	0\\
35.67	0\\
35.68	0\\
35.69	0\\
35.7	0\\
35.71	0\\
35.72	0\\
35.73	0\\
35.74	0\\
35.75	0\\
35.76	0\\
35.77	0\\
35.78	0\\
35.79	0\\
35.8	0\\
35.81	0\\
35.82	0\\
35.83	0\\
35.84	0\\
35.85	0\\
35.86	0\\
35.87	0\\
35.88	0\\
35.89	0\\
35.9	0\\
35.91	0\\
35.92	0\\
35.93	0\\
35.94	0\\
35.95	0\\
35.96	0\\
35.97	0\\
35.98	0\\
35.99	0\\
36	0\\
36.01	0\\
36.02	0\\
36.03	0\\
36.04	0\\
36.05	0\\
36.06	0\\
36.07	0\\
36.08	0\\
36.09	0\\
36.1	0\\
36.11	0\\
36.12	0\\
36.13	0\\
36.14	0\\
36.15	0\\
36.16	0\\
36.17	0\\
36.18	0\\
36.19	0\\
36.2	0\\
36.21	0\\
36.22	0\\
36.23	0\\
36.24	0\\
36.25	0\\
36.26	0\\
36.27	0\\
36.28	0\\
36.29	0\\
36.3	0\\
36.31	0\\
36.32	0\\
36.33	0\\
36.34	0\\
36.35	0\\
36.36	0\\
36.37	0\\
36.38	0\\
36.39	0\\
36.4	0\\
36.41	0\\
36.42	0\\
36.43	0\\
36.44	0\\
36.45	0\\
36.46	0\\
36.47	0\\
36.48	0\\
36.49	0\\
36.5	0\\
36.51	0\\
36.52	0\\
36.53	0\\
36.54	0\\
36.55	0\\
36.56	0\\
36.57	0\\
36.58	0\\
36.59	0\\
36.6	0\\
36.61	0\\
36.62	0\\
36.63	0\\
36.64	0\\
36.65	0\\
36.66	0\\
36.67	0\\
36.68	0\\
36.69	0\\
36.7	0\\
36.71	0\\
36.72	0\\
36.73	0\\
36.74	0\\
36.75	0\\
36.76	0\\
36.77	0\\
36.78	0\\
36.79	0\\
36.8	0\\
36.81	0\\
36.82	0\\
36.83	0\\
36.84	0\\
36.85	0\\
36.86	0\\
36.87	0\\
36.88	0\\
36.89	0\\
36.9	0\\
36.91	0\\
36.92	0\\
36.93	0\\
36.94	0\\
36.95	0\\
36.96	0\\
36.97	0\\
36.98	0\\
36.99	0\\
37	0\\
37.01	0\\
37.02	0\\
37.03	0\\
37.04	0\\
37.05	0\\
37.06	0\\
37.07	0\\
37.08	0\\
37.09	0\\
37.1	0\\
37.11	0\\
37.12	0\\
37.13	0\\
37.14	0\\
37.15	0\\
37.16	0\\
37.17	0\\
37.18	0\\
37.19	0\\
37.2	0\\
37.21	0\\
37.22	0\\
37.23	0\\
37.24	0\\
37.25	0\\
37.26	0\\
37.27	0\\
37.28	0\\
37.29	0\\
37.3	0\\
37.31	0\\
37.32	0\\
37.33	0\\
37.34	0\\
37.35	0\\
37.36	0\\
37.37	0\\
37.38	0\\
37.39	0\\
37.4	0\\
37.41	0\\
37.42	0\\
37.43	0\\
37.44	0\\
37.45	0\\
37.46	0\\
37.47	0\\
37.48	0\\
37.49	0\\
37.5	0\\
37.51	0\\
37.52	0\\
37.53	0\\
37.54	0\\
37.55	0\\
37.56	0\\
37.57	0\\
37.58	0\\
37.59	0\\
37.6	0\\
37.61	0\\
37.62	0\\
37.63	0\\
37.64	0\\
37.65	0\\
37.66	0\\
37.67	0\\
37.68	0\\
37.69	0\\
37.7	0\\
37.71	0\\
37.72	0\\
37.73	0\\
37.74	0\\
37.75	0\\
37.76	0\\
37.77	0\\
37.78	0\\
37.79	0\\
37.8	0\\
37.81	0\\
37.82	0\\
37.83	0\\
37.84	0\\
37.85	0\\
37.86	0\\
37.87	0\\
37.88	0\\
37.89	0\\
37.9	0\\
37.91	0\\
37.92	0\\
37.93	0\\
37.94	0\\
37.95	0\\
37.96	0\\
37.97	0\\
37.98	0\\
37.99	0\\
38	0\\
38.01	0\\
38.02	0\\
38.03	0\\
38.04	0\\
38.05	0\\
38.06	0\\
38.07	0\\
38.08	0\\
38.09	0\\
38.1	0\\
38.11	0\\
38.12	0\\
38.13	0\\
38.14	0\\
38.15	0\\
38.16	0\\
38.17	0\\
38.18	0\\
38.19	0\\
38.2	0\\
38.21	0\\
38.22	0\\
38.23	0\\
38.24	0\\
38.25	0\\
38.26	0\\
38.27	0\\
38.28	0\\
38.29	0\\
38.3	0\\
38.31	0\\
38.32	0\\
38.33	0\\
38.34	0\\
38.35	0\\
38.36	0\\
38.37	0\\
38.38	0\\
38.39	0\\
38.4	0\\
38.41	0\\
38.42	0\\
38.43	0\\
38.44	0\\
38.45	0\\
38.46	0\\
38.47	0\\
38.48	0\\
38.49	0\\
38.5	0\\
38.51	0\\
38.52	0\\
38.53	0\\
38.54	0\\
38.55	0\\
38.56	0\\
38.57	0\\
38.58	0\\
38.59	0\\
38.6	0\\
38.61	0\\
38.62	0\\
38.63	0\\
38.64	0\\
38.65	0\\
38.66	0\\
38.67	0\\
38.68	0\\
38.69	0\\
38.7	0\\
38.71	0\\
38.72	0\\
38.73	0\\
38.74	0\\
38.75	0\\
38.76	0\\
38.77	0\\
38.78	0\\
38.79	0\\
38.8	0\\
38.81	0\\
38.82	0\\
38.83	0\\
38.84	0\\
38.85	0\\
38.86	0\\
38.87	0\\
38.88	0\\
38.89	0\\
38.9	0\\
38.91	0\\
38.92	0\\
38.93	0\\
38.94	0\\
38.95	0\\
38.96	0\\
38.97	0\\
38.98	0\\
38.99	0\\
39	0\\
39.01	0\\
39.02	0\\
39.03	0\\
39.04	0\\
39.05	0\\
39.06	0\\
39.07	0\\
39.08	0\\
39.09	0\\
39.1	0\\
39.11	0\\
39.12	0\\
39.13	0\\
39.14	0\\
39.15	0\\
39.16	0\\
39.17	0\\
39.18	0\\
39.19	0\\
39.2	0\\
39.21	0\\
39.22	0\\
39.23	0\\
39.24	0\\
39.25	0\\
39.26	0\\
39.27	0\\
39.28	0\\
39.29	0\\
39.3	0\\
39.31	0\\
39.32	0\\
39.33	0\\
39.34	0\\
39.35	0\\
39.36	0\\
39.37	0\\
39.38	0\\
39.39	0\\
39.4	0\\
39.41	0\\
39.42	0\\
39.43	0\\
39.44	0\\
39.45	0\\
39.46	0\\
39.47	0\\
39.48	0\\
39.49	0\\
39.5	0\\
39.51	0\\
39.52	0\\
39.53	0\\
39.54	0\\
39.55	0\\
39.56	0\\
39.57	0\\
39.58	0\\
39.59	0\\
39.6	0\\
39.61	0\\
39.62	0\\
39.63	0\\
39.64	0\\
39.65	0\\
39.66	0\\
39.67	0\\
39.68	0\\
39.69	0\\
39.7	0\\
39.71	0\\
39.72	0\\
39.73	0\\
39.74	0\\
39.75	0\\
39.76	0\\
39.77	0\\
39.78	0\\
39.79	0\\
39.8	0\\
39.81	0\\
39.82	0\\
39.83	0\\
39.84	0\\
39.85	0\\
39.86	0\\
39.87	0\\
39.88	0\\
39.89	0\\
39.9	0\\
39.91	0\\
39.92	0\\
39.93	0\\
39.94	0\\
39.95	0\\
39.96	0\\
39.97	0\\
39.98	0\\
39.99	0\\
40	0\\
40.01	0\\
};
\addplot [color=blue,solid,forget plot]
  table[row sep=crcr]{%
40.01	0\\
40.02	0\\
40.03	0\\
40.04	0\\
40.05	0\\
40.06	0\\
40.07	0\\
40.08	0\\
40.09	0\\
40.1	0\\
40.11	0\\
40.12	0\\
40.13	0\\
40.14	0\\
40.15	0\\
40.16	0\\
40.17	0\\
40.18	0\\
40.19	0\\
40.2	0\\
40.21	0\\
40.22	0\\
40.23	0\\
40.24	0\\
40.25	0\\
40.26	0\\
40.27	0\\
40.28	0\\
40.29	0\\
40.3	0\\
40.31	0\\
40.32	0\\
40.33	0\\
40.34	0\\
40.35	0\\
40.36	0\\
40.37	0\\
40.38	0\\
40.39	0\\
40.4	0\\
40.41	0\\
40.42	0\\
40.43	0\\
40.44	0\\
40.45	0\\
40.46	0\\
40.47	0\\
40.48	0\\
40.49	0\\
40.5	0\\
40.51	0\\
40.52	0\\
40.53	0\\
40.54	0\\
40.55	0\\
40.56	0\\
40.57	0\\
40.58	0\\
40.59	0\\
40.6	0\\
40.61	0\\
40.62	0\\
40.63	0\\
40.64	0\\
40.65	0\\
40.66	0\\
40.67	0\\
40.68	0\\
40.69	0\\
40.7	0\\
40.71	0\\
40.72	0\\
40.73	0\\
40.74	0\\
40.75	0\\
40.76	0\\
40.77	0\\
40.78	0\\
40.79	0\\
40.8	0\\
40.81	0\\
40.82	0\\
40.83	0\\
40.84	0\\
40.85	0\\
40.86	0\\
40.87	0\\
40.88	0\\
40.89	0\\
40.9	0\\
40.91	0\\
40.92	0\\
40.93	0\\
40.94	0\\
40.95	0\\
40.96	0\\
40.97	0\\
40.98	0\\
40.99	0\\
41	0\\
41.01	0\\
41.02	0\\
41.03	0\\
41.04	0\\
41.05	0\\
41.06	0\\
41.07	0\\
41.08	0\\
41.09	0\\
41.1	0\\
41.11	0\\
41.12	0\\
41.13	0\\
41.14	0\\
41.15	0\\
41.16	0\\
41.17	0\\
41.18	0\\
41.19	0\\
41.2	0\\
41.21	0\\
41.22	0\\
41.23	0\\
41.24	0\\
41.25	0\\
41.26	0\\
41.27	0\\
41.28	0\\
41.29	0\\
41.3	0\\
41.31	0\\
41.32	0\\
41.33	0\\
41.34	0\\
41.35	0\\
41.36	0\\
41.37	0\\
41.38	0\\
41.39	0\\
41.4	0\\
41.41	0\\
41.42	0\\
41.43	0\\
41.44	0\\
41.45	0\\
41.46	0\\
41.47	0\\
41.48	0\\
41.49	0\\
41.5	0\\
41.51	0\\
41.52	0\\
41.53	0\\
41.54	0\\
41.55	0\\
41.56	0\\
41.57	0\\
41.58	0\\
41.59	0\\
41.6	0\\
41.61	0\\
41.62	0\\
41.63	0\\
41.64	0\\
41.65	0\\
41.66	0\\
41.67	0\\
41.68	0\\
41.69	0\\
41.7	0\\
41.71	0\\
41.72	0\\
41.73	0\\
41.74	0\\
41.75	0\\
41.76	0\\
41.77	0\\
41.78	0\\
41.79	0\\
41.8	0\\
41.81	0\\
41.82	0\\
41.83	0\\
41.84	0\\
41.85	0\\
41.86	0\\
41.87	0\\
41.88	0\\
41.89	0\\
41.9	0\\
41.91	0\\
41.92	0\\
41.93	0\\
41.94	0\\
41.95	0\\
41.96	0\\
41.97	0\\
41.98	0\\
41.99	0\\
42	0\\
42.01	0\\
42.02	0\\
42.03	0\\
42.04	0\\
42.05	0\\
42.06	0\\
42.07	0\\
42.08	0\\
42.09	0\\
42.1	0\\
42.11	0\\
42.12	0\\
42.13	0\\
42.14	0\\
42.15	0\\
42.16	0\\
42.17	0\\
42.18	0\\
42.19	0\\
42.2	0\\
42.21	0\\
42.22	0\\
42.23	0\\
42.24	0\\
42.25	0\\
42.26	0\\
42.27	0\\
42.28	0\\
42.29	0\\
42.3	0\\
42.31	0\\
42.32	0\\
42.33	0\\
42.34	0\\
42.35	0\\
42.36	0\\
42.37	0\\
42.38	0\\
42.39	0\\
42.4	0\\
42.41	0\\
42.42	0\\
42.43	0\\
42.44	0\\
42.45	0\\
42.46	0\\
42.47	0\\
42.48	0\\
42.49	0\\
42.5	0\\
42.51	0\\
42.52	0\\
42.53	0\\
42.54	0\\
42.55	0\\
42.56	0\\
42.57	0\\
42.58	0\\
42.59	0\\
42.6	0\\
42.61	0\\
42.62	0\\
42.63	0\\
42.64	0\\
42.65	0\\
42.66	0\\
42.67	0\\
42.68	0\\
42.69	0\\
42.7	0\\
42.71	0\\
42.72	0\\
42.73	0\\
42.74	0\\
42.75	0\\
42.76	0\\
42.77	0\\
42.78	0\\
42.79	0\\
42.8	0\\
42.81	0\\
42.82	0\\
42.83	0\\
42.84	0\\
42.85	0\\
42.86	0\\
42.87	0\\
42.88	0\\
42.89	0\\
42.9	0\\
42.91	0\\
42.92	0\\
42.93	0\\
42.94	0\\
42.95	0\\
42.96	0\\
42.97	0\\
42.98	0\\
42.99	0\\
43	0\\
43.01	0\\
43.02	0\\
43.03	0\\
43.04	0\\
43.05	0\\
43.06	0\\
43.07	0\\
43.08	0\\
43.09	0\\
43.1	0\\
43.11	0\\
43.12	0\\
43.13	0\\
43.14	0\\
43.15	0\\
43.16	0\\
43.17	0\\
43.18	0\\
43.19	0\\
43.2	0\\
43.21	0\\
43.22	0\\
43.23	0\\
43.24	0\\
43.25	0\\
43.26	0\\
43.27	0\\
43.28	0\\
43.29	0\\
43.3	0\\
43.31	0\\
43.32	0\\
43.33	0\\
43.34	0\\
43.35	0\\
43.36	0\\
43.37	0\\
43.38	0\\
43.39	0\\
43.4	0\\
43.41	0\\
43.42	0\\
43.43	0\\
43.44	0\\
43.45	0\\
43.46	0\\
43.47	0\\
43.48	0\\
43.49	0\\
43.5	0\\
43.51	0\\
43.52	0\\
43.53	0\\
43.54	0\\
43.55	0\\
43.56	0\\
43.57	0\\
43.58	0\\
43.59	0\\
43.6	0\\
43.61	0\\
43.62	0\\
43.63	0\\
43.64	0\\
43.65	0\\
43.66	0\\
43.67	0\\
43.68	0\\
43.69	0\\
43.7	0\\
43.71	0\\
43.72	0\\
43.73	0\\
43.74	0\\
43.75	0\\
43.76	0\\
43.77	0\\
43.78	0\\
43.79	0\\
43.8	0\\
43.81	0\\
43.82	0\\
43.83	0\\
43.84	0\\
43.85	0\\
43.86	0\\
43.87	0\\
43.88	0\\
43.89	0\\
43.9	0\\
43.91	0\\
43.92	0\\
43.93	0\\
43.94	0\\
43.95	0\\
43.96	0\\
43.97	0\\
43.98	0\\
43.99	0\\
44	0\\
44.01	0\\
44.02	0\\
44.03	0\\
44.04	0\\
44.05	0\\
44.06	0\\
44.07	0\\
44.08	0\\
44.09	0\\
44.1	0\\
44.11	0\\
44.12	0\\
44.13	0\\
44.14	0\\
44.15	0\\
44.16	0\\
44.17	0\\
44.18	0\\
44.19	0\\
44.2	0\\
44.21	0\\
44.22	0\\
44.23	0\\
44.24	0\\
44.25	0\\
44.26	0\\
44.27	0\\
44.28	0\\
44.29	0\\
44.3	0\\
44.31	0\\
44.32	0\\
44.33	0\\
44.34	0\\
44.35	0\\
44.36	0\\
44.37	0\\
44.38	0\\
44.39	0\\
44.4	0\\
44.41	0\\
44.42	0\\
44.43	0\\
44.44	0\\
44.45	0\\
44.46	0\\
44.47	0\\
44.48	0\\
44.49	0\\
44.5	0\\
44.51	0\\
44.52	0\\
44.53	0\\
44.54	0\\
44.55	0\\
44.56	0\\
44.57	0\\
44.58	0\\
44.59	0\\
44.6	0\\
44.61	0\\
44.62	0\\
44.63	0\\
44.64	0\\
44.65	0\\
44.66	0\\
44.67	0\\
44.68	0\\
44.69	0\\
44.7	0\\
44.71	0\\
44.72	0\\
44.73	0\\
44.74	0\\
44.75	0\\
44.76	0\\
44.77	0\\
44.78	0\\
44.79	0\\
44.8	0\\
44.81	0\\
44.82	0\\
44.83	0\\
44.84	0\\
44.85	0\\
44.86	0\\
44.87	0\\
44.88	0\\
44.89	0\\
44.9	0\\
44.91	0\\
44.92	0\\
44.93	0\\
44.94	0\\
44.95	0\\
44.96	0\\
44.97	0\\
44.98	0\\
44.99	0\\
45	0\\
45.01	0\\
45.02	0\\
45.03	0\\
45.04	0\\
45.05	0\\
45.06	0\\
45.07	0\\
45.08	0\\
45.09	0\\
45.1	0\\
45.11	0\\
45.12	0\\
45.13	0\\
45.14	0\\
45.15	0\\
45.16	0\\
45.17	0\\
45.18	0\\
45.19	0\\
45.2	0\\
45.21	0\\
45.22	0\\
45.23	0\\
45.24	0\\
45.25	0\\
45.26	0\\
45.27	0\\
45.28	0\\
45.29	0\\
45.3	0\\
45.31	0\\
45.32	0\\
45.33	0\\
45.34	0\\
45.35	0\\
45.36	0\\
45.37	0\\
45.38	0\\
45.39	0\\
45.4	0\\
45.41	0\\
45.42	0\\
45.43	0\\
45.44	0\\
45.45	0\\
45.46	0\\
45.47	0\\
45.48	0\\
45.49	0\\
45.5	0\\
45.51	0\\
45.52	0\\
45.53	0\\
45.54	0\\
45.55	0\\
45.56	0\\
45.57	0\\
45.58	0\\
45.59	0\\
45.6	0\\
45.61	0\\
45.62	0\\
45.63	0\\
45.64	0\\
45.65	0\\
45.66	0\\
45.67	0\\
45.68	0\\
45.69	0\\
45.7	0\\
45.71	0\\
45.72	0\\
45.73	0\\
45.74	0\\
45.75	0\\
45.76	0\\
45.77	0\\
45.78	0\\
45.79	0\\
45.8	0\\
45.81	0\\
45.82	0\\
45.83	0\\
45.84	0\\
45.85	0\\
45.86	0\\
45.87	0\\
45.88	0\\
45.89	0\\
45.9	0\\
45.91	0\\
45.92	0\\
45.93	0\\
45.94	0\\
45.95	0\\
45.96	0\\
45.97	0\\
45.98	0\\
45.99	0\\
46	0\\
46.01	0\\
46.02	0\\
46.03	0\\
46.04	0\\
46.05	0\\
46.06	0\\
46.07	0\\
46.08	0\\
46.09	0\\
46.1	0\\
46.11	0\\
46.12	0\\
46.13	0\\
46.14	0\\
46.15	0\\
46.16	0\\
46.17	0\\
46.18	0\\
46.19	0\\
46.2	0\\
46.21	0\\
46.22	0\\
46.23	0\\
46.24	0\\
46.25	0\\
46.26	0\\
46.27	0\\
46.28	0\\
46.29	0\\
46.3	0\\
46.31	0\\
46.32	0\\
46.33	0\\
46.34	0\\
46.35	0\\
46.36	0\\
46.37	0\\
46.38	0\\
46.39	0\\
46.4	0\\
46.41	0\\
46.42	0\\
46.43	0\\
46.44	0\\
46.45	0\\
46.46	0\\
46.47	0\\
46.48	0\\
46.49	0\\
46.5	0\\
46.51	0\\
46.52	0\\
46.53	0\\
46.54	0\\
46.55	0\\
46.56	0\\
46.57	0\\
46.58	0\\
46.59	0\\
46.6	0\\
46.61	0\\
46.62	0\\
46.63	0\\
46.64	0\\
46.65	0\\
46.66	0\\
46.67	0\\
46.68	0\\
46.69	0\\
46.7	0\\
46.71	0\\
46.72	0\\
46.73	0\\
46.74	0\\
46.75	0\\
46.76	0\\
46.77	0\\
46.78	0\\
46.79	0\\
46.8	0\\
46.81	0\\
46.82	0\\
46.83	0\\
46.84	0\\
46.85	0\\
46.86	0\\
46.87	0\\
46.88	0\\
46.89	0\\
46.9	0\\
46.91	0\\
46.92	0\\
46.93	0\\
46.94	0\\
46.95	0\\
46.96	0\\
46.97	0\\
46.98	0\\
46.99	0\\
47	0\\
47.01	0\\
47.02	0\\
47.03	0\\
47.04	0\\
47.05	0\\
47.06	0\\
47.07	0\\
47.08	0\\
47.09	0\\
47.1	0\\
47.11	0\\
47.12	0\\
47.13	0\\
47.14	0\\
47.15	0\\
47.16	0\\
47.17	0\\
47.18	0\\
47.19	0\\
47.2	0\\
47.21	0\\
47.22	0\\
47.23	0\\
47.24	0\\
47.25	0\\
47.26	0\\
47.27	0\\
47.28	0\\
47.29	0\\
47.3	0\\
47.31	0\\
47.32	0\\
47.33	0\\
47.34	0\\
47.35	0\\
47.36	0\\
47.37	0\\
47.38	0\\
47.39	0\\
47.4	0\\
47.41	0\\
47.42	0\\
47.43	0\\
47.44	0\\
47.45	0\\
47.46	0\\
47.47	0\\
47.48	0\\
47.49	0\\
47.5	0\\
47.51	0\\
47.52	0\\
47.53	0\\
47.54	0\\
47.55	0\\
47.56	0\\
47.57	0\\
47.58	0\\
47.59	0\\
47.6	0\\
47.61	0\\
47.62	0\\
47.63	0\\
47.64	0\\
47.65	0\\
47.66	0\\
47.67	0\\
47.68	0\\
47.69	0\\
47.7	0\\
47.71	0\\
47.72	0\\
47.73	0\\
47.74	0\\
47.75	0\\
47.76	0\\
47.77	0\\
47.78	0\\
47.79	0\\
47.8	0\\
47.81	0\\
47.82	0\\
47.83	0\\
47.84	0\\
47.85	0\\
47.86	0\\
47.87	0\\
47.88	0\\
47.89	0\\
47.9	0\\
47.91	0\\
47.92	0\\
47.93	0\\
47.94	0\\
47.95	0\\
47.96	0\\
47.97	0\\
47.98	0\\
47.99	0\\
48	0\\
48.01	0\\
48.02	0\\
48.03	0\\
48.04	0\\
48.05	0\\
48.06	0\\
48.07	0\\
48.08	0\\
48.09	0\\
48.1	0\\
48.11	0\\
48.12	0\\
48.13	0\\
48.14	0\\
48.15	0\\
48.16	0\\
48.17	0\\
48.18	0\\
48.19	0\\
48.2	0\\
48.21	0\\
48.22	0\\
48.23	0\\
48.24	0\\
48.25	0\\
48.26	0\\
48.27	0\\
48.28	0\\
48.29	0\\
48.3	0\\
48.31	0\\
48.32	0\\
48.33	0\\
48.34	0\\
48.35	0\\
48.36	0\\
48.37	0\\
48.38	0\\
48.39	0\\
48.4	0\\
48.41	0\\
48.42	0\\
48.43	0\\
48.44	0\\
48.45	0\\
48.46	0\\
48.47	0\\
48.48	0\\
48.49	0\\
48.5	0\\
48.51	0\\
48.52	0\\
48.53	0\\
48.54	0\\
48.55	0\\
48.56	0\\
48.57	0\\
48.58	0\\
48.59	0\\
48.6	0\\
48.61	0\\
48.62	0\\
48.63	0\\
48.64	0\\
48.65	0\\
48.66	0\\
48.67	0\\
48.68	0\\
48.69	0\\
48.7	0\\
48.71	0\\
48.72	0\\
48.73	0\\
48.74	0\\
48.75	0\\
48.76	0\\
48.77	0\\
48.78	0\\
48.79	0\\
48.8	0\\
48.81	0\\
48.82	0\\
48.83	0\\
48.84	0\\
48.85	0\\
48.86	0\\
48.87	0\\
48.88	0\\
48.89	0\\
48.9	0\\
48.91	0\\
48.92	0\\
48.93	0\\
48.94	0\\
48.95	0\\
48.96	0\\
48.97	0\\
48.98	0\\
48.99	0\\
49	0\\
49.01	0\\
49.02	0\\
49.03	0\\
49.04	0\\
49.05	0\\
49.06	0\\
49.07	0\\
49.08	0\\
49.09	0\\
49.1	0\\
49.11	0\\
49.12	0\\
49.13	0\\
49.14	0\\
49.15	0\\
49.16	0\\
49.17	0\\
49.18	0\\
49.19	0\\
49.2	0\\
49.21	0\\
49.22	0\\
49.23	0\\
49.24	0\\
49.25	0\\
49.26	0\\
49.27	0\\
49.28	0\\
49.29	0\\
49.3	0\\
49.31	0\\
49.32	0\\
49.33	0\\
49.34	0\\
49.35	0\\
49.36	0\\
49.37	0\\
49.38	0\\
49.39	0\\
49.4	0\\
49.41	0\\
49.42	0\\
49.43	0\\
49.44	0\\
49.45	0\\
49.46	0\\
49.47	0\\
49.48	0\\
49.49	0\\
49.5	0\\
49.51	0\\
49.52	0\\
49.53	0\\
49.54	0\\
49.55	0\\
49.56	0\\
49.57	0\\
49.58	0\\
49.59	0\\
49.6	0\\
49.61	0\\
49.62	0\\
49.63	0\\
49.64	0\\
49.65	0\\
49.66	0\\
49.67	0\\
49.68	0\\
49.69	0\\
49.7	0\\
49.71	0\\
49.72	0\\
49.73	0\\
49.74	0\\
49.75	0\\
49.76	0\\
49.77	0\\
49.78	0\\
49.79	0\\
49.8	0\\
49.81	0\\
49.82	0\\
49.83	0\\
49.84	0\\
49.85	0\\
49.86	0\\
49.87	0\\
49.88	0\\
49.89	0\\
49.9	0\\
49.91	0\\
49.92	0\\
49.93	0\\
49.94	0\\
49.95	0\\
49.96	0\\
49.97	0\\
49.98	0\\
49.99	0\\
50	0\\
50.01	0\\
50.02	0\\
50.03	0\\
50.04	0\\
50.05	0\\
50.06	0\\
50.07	0\\
50.08	0\\
50.09	0\\
50.1	0\\
50.11	0\\
50.12	0\\
50.13	0\\
50.14	0\\
50.15	0\\
50.16	0\\
50.17	0\\
50.18	0\\
50.19	0\\
50.2	0\\
50.21	0\\
50.22	0\\
50.23	0\\
50.24	0\\
50.25	0\\
50.26	0\\
50.27	0\\
50.28	0\\
50.29	0\\
50.3	0\\
50.31	0\\
50.32	0\\
50.33	0\\
50.34	0\\
50.35	0\\
50.36	0\\
50.37	0\\
50.38	0\\
50.39	0\\
50.4	0\\
50.41	0\\
50.42	0\\
50.43	0\\
50.44	0\\
50.45	0\\
50.46	0\\
50.47	0\\
50.48	0\\
50.49	0\\
50.5	0\\
50.51	0\\
50.52	0\\
50.53	0\\
50.54	0\\
50.55	0\\
50.56	0\\
50.57	0\\
50.58	0\\
50.59	0\\
50.6	0\\
50.61	0\\
50.62	0\\
50.63	0\\
50.64	0\\
50.65	0\\
50.66	0\\
50.67	0\\
50.68	0\\
50.69	0\\
50.7	0\\
50.71	0\\
50.72	0\\
50.73	0\\
50.74	0\\
50.75	0\\
50.76	0\\
50.77	0\\
50.78	0\\
50.79	0\\
50.8	0\\
50.81	0\\
50.82	0\\
50.83	0\\
50.84	0\\
50.85	0\\
50.86	0\\
50.87	0\\
50.88	0\\
50.89	0\\
50.9	0\\
50.91	0\\
50.92	0\\
50.93	0\\
50.94	0\\
50.95	0\\
50.96	0\\
50.97	0\\
50.98	0\\
50.99	0\\
51	0\\
51.01	0\\
51.02	0\\
51.03	0\\
51.04	0\\
51.05	0\\
51.06	0\\
51.07	0\\
51.08	0\\
51.09	0\\
51.1	0\\
51.11	0\\
51.12	0\\
51.13	0\\
51.14	0\\
51.15	0\\
51.16	0\\
51.17	0\\
51.18	0\\
51.19	0\\
51.2	0\\
51.21	0\\
51.22	0\\
51.23	0\\
51.24	0\\
51.25	0\\
51.26	0\\
51.27	0\\
51.28	0\\
51.29	0\\
51.3	0\\
51.31	0\\
51.32	0\\
51.33	0\\
51.34	0\\
51.35	0\\
51.36	0\\
51.37	0\\
51.38	0\\
51.39	0\\
51.4	0\\
51.41	0\\
51.42	0\\
51.43	0\\
51.44	0\\
51.45	0\\
51.46	0\\
51.47	0\\
51.48	0\\
51.49	0\\
51.5	0\\
51.51	0\\
51.52	0\\
51.53	0\\
51.54	0\\
51.55	0\\
51.56	0\\
51.57	0\\
51.58	0\\
51.59	0\\
51.6	0\\
51.61	0\\
51.62	0\\
51.63	0\\
51.64	0\\
51.65	0\\
51.66	0\\
51.67	0\\
51.68	0\\
51.69	0\\
51.7	0\\
51.71	0\\
51.72	0\\
51.73	0\\
51.74	0\\
51.75	0\\
51.76	0\\
51.77	0\\
51.78	0\\
51.79	0\\
51.8	0\\
51.81	0\\
51.82	0\\
51.83	0\\
51.84	0\\
51.85	0\\
51.86	0\\
51.87	0\\
51.88	0\\
51.89	0\\
51.9	0\\
51.91	0\\
51.92	0\\
51.93	0\\
51.94	0\\
51.95	0\\
51.96	0\\
51.97	0\\
51.98	0\\
51.99	0\\
52	0\\
52.01	0\\
52.02	0\\
52.03	0\\
52.04	0\\
52.05	0\\
52.06	0\\
52.07	0\\
52.08	0\\
52.09	0\\
52.1	0\\
52.11	0\\
52.12	0\\
52.13	0\\
52.14	0\\
52.15	0\\
52.16	0\\
52.17	0\\
52.18	0\\
52.19	0\\
52.2	0\\
52.21	0\\
52.22	0\\
52.23	0\\
52.24	0\\
52.25	0\\
52.26	0\\
52.27	0\\
52.28	0\\
52.29	0\\
52.3	0\\
52.31	0\\
52.32	0\\
52.33	0\\
52.34	0\\
52.35	0\\
52.36	0\\
52.37	0\\
52.38	0\\
52.39	0\\
52.4	0\\
52.41	0\\
52.42	0\\
52.43	0\\
52.44	0\\
52.45	0\\
52.46	0\\
52.47	0\\
52.48	0\\
52.49	0\\
52.5	0\\
52.51	0\\
52.52	0\\
52.53	0\\
52.54	0\\
52.55	0\\
52.56	0\\
52.57	0\\
52.58	0\\
52.59	0\\
52.6	0\\
52.61	0\\
52.62	0\\
52.63	0\\
52.64	0\\
52.65	0\\
52.66	0\\
52.67	0\\
52.68	0\\
52.69	0\\
52.7	0\\
52.71	0\\
52.72	0\\
52.73	0\\
52.74	0\\
52.75	0\\
52.76	0\\
52.77	0\\
52.78	0\\
52.79	0\\
52.8	0\\
52.81	0\\
52.82	0\\
52.83	0\\
52.84	0\\
52.85	0\\
52.86	0\\
52.87	0\\
52.88	0\\
52.89	0\\
52.9	0\\
52.91	0\\
52.92	0\\
52.93	0\\
52.94	0\\
52.95	0\\
52.96	0\\
52.97	0\\
52.98	0\\
52.99	0\\
53	0\\
53.01	0\\
53.02	0\\
53.03	0\\
53.04	0\\
53.05	0\\
53.06	0\\
53.07	0\\
53.08	0\\
53.09	0\\
53.1	0\\
53.11	0\\
53.12	0\\
53.13	0\\
53.14	0\\
53.15	0\\
53.16	0\\
53.17	0\\
53.18	0\\
53.19	0\\
53.2	0\\
53.21	0\\
53.22	0\\
53.23	0\\
53.24	0\\
53.25	0\\
53.26	0\\
53.27	0\\
53.28	0\\
53.29	0\\
53.3	0\\
53.31	0\\
53.32	0\\
53.33	0\\
53.34	0\\
53.35	0\\
53.36	0\\
53.37	0\\
53.38	0\\
53.39	0\\
53.4	0\\
53.41	0\\
53.42	0\\
53.43	0\\
53.44	0\\
53.45	0\\
53.46	0\\
53.47	0\\
53.48	0\\
53.49	0\\
53.5	0\\
53.51	0\\
53.52	0\\
53.53	0\\
53.54	0\\
53.55	0\\
53.56	0\\
53.57	0\\
53.58	0\\
53.59	0\\
53.6	0\\
53.61	0\\
53.62	0\\
53.63	0\\
53.64	0\\
53.65	0\\
53.66	0\\
53.67	0\\
53.68	0\\
53.69	0\\
53.7	0\\
53.71	0\\
53.72	0\\
53.73	0\\
53.74	0\\
53.75	0\\
53.76	0\\
53.77	0\\
53.78	0\\
53.79	0\\
53.8	0\\
53.81	0\\
53.82	0\\
53.83	0\\
53.84	0\\
53.85	0\\
53.86	0\\
53.87	0\\
53.88	0\\
53.89	0\\
53.9	0\\
53.91	0\\
53.92	0\\
53.93	0\\
53.94	0\\
53.95	0\\
53.96	0\\
53.97	0\\
53.98	0\\
53.99	0\\
54	0\\
54.01	0\\
54.02	0\\
54.03	0\\
54.04	0\\
54.05	0\\
54.06	0\\
54.07	0\\
54.08	0\\
54.09	0\\
54.1	0\\
54.11	0\\
54.12	0\\
54.13	0\\
54.14	0\\
54.15	0\\
54.16	0\\
54.17	0\\
54.18	0\\
54.19	0\\
54.2	0\\
54.21	0\\
54.22	0\\
54.23	0\\
54.24	0\\
54.25	0\\
54.26	0\\
54.27	0\\
54.28	0\\
54.29	0\\
54.3	0\\
54.31	0\\
54.32	0\\
54.33	0\\
54.34	0\\
54.35	0\\
54.36	0\\
54.37	0\\
54.38	0\\
54.39	0\\
54.4	0\\
54.41	0\\
54.42	0\\
54.43	0\\
54.44	0\\
54.45	0\\
54.46	0\\
54.47	0\\
54.48	0\\
54.49	0\\
54.5	0\\
54.51	0\\
54.52	0\\
54.53	0\\
54.54	0\\
54.55	0\\
54.56	0\\
54.57	0\\
54.58	0\\
54.59	0\\
54.6	0\\
54.61	0\\
54.62	0\\
54.63	0\\
54.64	0\\
54.65	0\\
54.66	0\\
54.67	0\\
54.68	0\\
54.69	0\\
54.7	0\\
54.71	0\\
54.72	0\\
54.73	0\\
54.74	0\\
54.75	0\\
54.76	0\\
54.77	0\\
54.78	0\\
54.79	0\\
54.8	0\\
54.81	0\\
54.82	0\\
54.83	0\\
54.84	0\\
54.85	0\\
54.86	0\\
54.87	0\\
54.88	0\\
54.89	0\\
54.9	0\\
54.91	0\\
54.92	0\\
54.93	0\\
54.94	0\\
54.95	0\\
54.96	0\\
54.97	0\\
54.98	0\\
54.99	0\\
55	0\\
55.01	0\\
55.02	0\\
55.03	0\\
55.04	0\\
55.05	0\\
55.06	0\\
55.07	0\\
55.08	0\\
55.09	0\\
55.1	0\\
55.11	0\\
55.12	0\\
55.13	0\\
55.14	0\\
55.15	0\\
55.16	0\\
55.17	0\\
55.18	0\\
55.19	0\\
55.2	0\\
55.21	0\\
55.22	0\\
55.23	0\\
55.24	0\\
55.25	0\\
55.26	0\\
55.27	0\\
55.28	0\\
55.29	0\\
55.3	0\\
55.31	0\\
55.32	0\\
55.33	0\\
55.34	0\\
55.35	0\\
55.36	0\\
55.37	0\\
55.38	0\\
55.39	0\\
55.4	0\\
55.41	0\\
55.42	0\\
55.43	0\\
55.44	0\\
55.45	0\\
55.46	0\\
55.47	0\\
55.48	0\\
55.49	0\\
55.5	0\\
55.51	0\\
55.52	0\\
55.53	0\\
55.54	0\\
55.55	0\\
55.56	0\\
55.57	0\\
55.58	0\\
55.59	0\\
55.6	0\\
55.61	0\\
55.62	0\\
55.63	0\\
55.64	0\\
55.65	0\\
55.66	0\\
55.67	0\\
55.68	0\\
55.69	0\\
55.7	0\\
55.71	0\\
55.72	0\\
55.73	0\\
55.74	0\\
55.75	0\\
55.76	0\\
55.77	0\\
55.78	0\\
55.79	0\\
55.8	0\\
55.81	0\\
55.82	0\\
55.83	0\\
55.84	0\\
55.85	0\\
55.86	0\\
55.87	0\\
55.88	0\\
55.89	0\\
55.9	0\\
55.91	0\\
55.92	0\\
55.93	0\\
55.94	0\\
55.95	0\\
55.96	0\\
55.97	0\\
55.98	0\\
55.99	0\\
56	0\\
56.01	0\\
56.02	0\\
56.03	0\\
56.04	0\\
56.05	0\\
56.06	0\\
56.07	0\\
56.08	0\\
56.09	0\\
56.1	0\\
56.11	0\\
56.12	0\\
56.13	0\\
56.14	0\\
56.15	0\\
56.16	0\\
56.17	0\\
56.18	0\\
56.19	0\\
56.2	0\\
56.21	0\\
56.22	0\\
56.23	0\\
56.24	0\\
56.25	0\\
56.26	0\\
56.27	0\\
56.28	0\\
56.29	0\\
56.3	0\\
56.31	0\\
56.32	0\\
56.33	0\\
56.34	0\\
56.35	0\\
56.36	0\\
56.37	0\\
56.38	0\\
56.39	0\\
56.4	0\\
56.41	0\\
56.42	0\\
56.43	0\\
56.44	0\\
56.45	0\\
56.46	0\\
56.47	0\\
56.48	0\\
56.49	0\\
56.5	0\\
56.51	0\\
56.52	0\\
56.53	0\\
56.54	0\\
56.55	0\\
56.56	0\\
56.57	0\\
56.58	0\\
56.59	0\\
56.6	0\\
56.61	0\\
56.62	0\\
56.63	0\\
56.64	0\\
56.65	0\\
56.66	0\\
56.67	0\\
56.68	0\\
56.69	0\\
56.7	0\\
56.71	0\\
56.72	0\\
56.73	0\\
56.74	0\\
56.75	0\\
56.76	0\\
56.77	0\\
56.78	0\\
56.79	0\\
56.8	0\\
56.81	0\\
56.82	0\\
56.83	0\\
56.84	0\\
56.85	0\\
56.86	0\\
56.87	0\\
56.88	0\\
56.89	0\\
56.9	0\\
56.91	0\\
56.92	0\\
56.93	0\\
56.94	0\\
56.95	0\\
56.96	0\\
56.97	0\\
56.98	0\\
56.99	0\\
57	0\\
57.01	0\\
57.02	0\\
57.03	0\\
57.04	0\\
57.05	0\\
57.06	0\\
57.07	0\\
57.08	0\\
57.09	0\\
57.1	0\\
57.11	0\\
57.12	0\\
57.13	0\\
57.14	0\\
57.15	0\\
57.16	0\\
57.17	0\\
57.18	0\\
57.19	0\\
57.2	0\\
57.21	0\\
57.22	0\\
57.23	0\\
57.24	0\\
57.25	0\\
57.26	0\\
57.27	0\\
57.28	0\\
57.29	0\\
57.3	0\\
57.31	0\\
57.32	0\\
57.33	0\\
57.34	0\\
57.35	0\\
57.36	0\\
57.37	0\\
57.38	0\\
57.39	0\\
57.4	0\\
57.41	0\\
57.42	0\\
57.43	0\\
57.44	0\\
57.45	0\\
57.46	0\\
57.47	0\\
57.48	0\\
57.49	0\\
57.5	0\\
57.51	0\\
57.52	0\\
57.53	0\\
57.54	0\\
57.55	0\\
57.56	0\\
57.57	0\\
57.58	0\\
57.59	0\\
57.6	0\\
57.61	0\\
57.62	0\\
57.63	0\\
57.64	0\\
57.65	0\\
57.66	0\\
57.67	0\\
57.68	0\\
57.69	0\\
57.7	0\\
57.71	0\\
57.72	0\\
57.73	0\\
57.74	0\\
57.75	0\\
57.76	0\\
57.77	0\\
57.78	0\\
57.79	0\\
57.8	0\\
57.81	0\\
57.82	0\\
57.83	0\\
57.84	0\\
57.85	0\\
57.86	0\\
57.87	0\\
57.88	0\\
57.89	0\\
57.9	0\\
57.91	0\\
57.92	0\\
57.93	0\\
57.94	0\\
57.95	0\\
57.96	0\\
57.97	0\\
57.98	0\\
57.99	0\\
58	0\\
58.01	0\\
58.02	0\\
58.03	0\\
58.04	0\\
58.05	0\\
58.06	0\\
58.07	0\\
58.08	0\\
58.09	0\\
58.1	0\\
58.11	0\\
58.12	0\\
58.13	0\\
58.14	0\\
58.15	0\\
58.16	0\\
58.17	0\\
58.18	0\\
58.19	0\\
58.2	0\\
58.21	0\\
58.22	0\\
58.23	0\\
58.24	0\\
58.25	0\\
58.26	0\\
58.27	0\\
58.28	0\\
58.29	0\\
58.3	0\\
58.31	0\\
58.32	0\\
58.33	0\\
58.34	0\\
58.35	0\\
58.36	0\\
58.37	0\\
58.38	0\\
58.39	0\\
58.4	0\\
58.41	0\\
58.42	0\\
58.43	0\\
58.44	0\\
58.45	0\\
58.46	0\\
58.47	0\\
58.48	0\\
58.49	0\\
58.5	0\\
58.51	0\\
58.52	0\\
58.53	0\\
58.54	0\\
58.55	0\\
58.56	0\\
58.57	0\\
58.58	0\\
58.59	0\\
58.6	0\\
58.61	0\\
58.62	0\\
58.63	0\\
58.64	0\\
58.65	0\\
58.66	0\\
58.67	0\\
58.68	0\\
58.69	0\\
58.7	0\\
58.71	0\\
58.72	0\\
58.73	0\\
58.74	0\\
58.75	0\\
58.76	0\\
58.77	0\\
58.78	0\\
58.79	0\\
58.8	0\\
58.81	0\\
58.82	0\\
58.83	0\\
58.84	0\\
58.85	0\\
58.86	0\\
58.87	0\\
58.88	0\\
58.89	0\\
58.9	0\\
58.91	0\\
58.92	0\\
58.93	0\\
58.94	0\\
58.95	0\\
58.96	0\\
58.97	0\\
58.98	0\\
58.99	0\\
59	0\\
59.01	0\\
59.02	0\\
59.03	0\\
59.04	0\\
59.05	0\\
59.06	0\\
59.07	0\\
59.08	0\\
59.09	0\\
59.1	0\\
59.11	0\\
59.12	0\\
59.13	0\\
59.14	0\\
59.15	0\\
59.16	0\\
59.17	0\\
59.18	0\\
59.19	0\\
59.2	0\\
59.21	0\\
59.22	0\\
59.23	0\\
59.24	0\\
59.25	0\\
59.26	0\\
59.27	0\\
59.28	0\\
59.29	0\\
59.3	0\\
59.31	0\\
59.32	0\\
59.33	0\\
59.34	0\\
59.35	0\\
59.36	0\\
59.37	0\\
59.38	0\\
59.39	0\\
59.4	0\\
59.41	0\\
59.42	0\\
59.43	0\\
59.44	0\\
59.45	0\\
59.46	0\\
59.47	0\\
59.48	0\\
59.49	0\\
59.5	0\\
59.51	0\\
59.52	0\\
59.53	0\\
59.54	0\\
59.55	0\\
59.56	0\\
59.57	0\\
59.58	0\\
59.59	0\\
59.6	0\\
59.61	0\\
59.62	0\\
59.63	0\\
59.64	0\\
59.65	0\\
59.66	0\\
59.67	0\\
59.68	0\\
59.69	0\\
59.7	0\\
59.71	0\\
59.72	0\\
59.73	0\\
59.74	0\\
59.75	0\\
59.76	0\\
59.77	0\\
59.78	0\\
59.79	0\\
59.8	0\\
59.81	0\\
59.82	0\\
59.83	0\\
59.84	0\\
59.85	0\\
59.86	0\\
59.87	0\\
59.88	0\\
59.89	0\\
59.9	0\\
59.91	0\\
59.92	0\\
59.93	0\\
59.94	0\\
59.95	0\\
59.96	0\\
59.97	0\\
59.98	0\\
59.99	0\\
60	0\\
60.01	0\\
60.02	0\\
60.03	0\\
60.04	0\\
60.05	0\\
60.06	0\\
60.07	0\\
60.08	0\\
60.09	0\\
60.1	0\\
60.11	0\\
60.12	0\\
60.13	0\\
60.14	0\\
60.15	0\\
60.16	0\\
60.17	0\\
60.18	0\\
60.19	0\\
60.2	0\\
60.21	0\\
60.22	0\\
60.23	0\\
60.24	0\\
60.25	0\\
60.26	0\\
60.27	0\\
60.28	0\\
60.29	0\\
60.3	0\\
60.31	0\\
60.32	0\\
60.33	0\\
60.34	0\\
60.35	0\\
60.36	0\\
60.37	0\\
60.38	0\\
60.39	0\\
60.4	0\\
60.41	0\\
60.42	0\\
60.43	0\\
60.44	0\\
60.45	0\\
60.46	0\\
60.47	0\\
60.48	0\\
60.49	0\\
60.5	0\\
60.51	0\\
60.52	0\\
60.53	0\\
60.54	0\\
60.55	0\\
60.56	0\\
60.57	0\\
60.58	0\\
60.59	0\\
60.6	0\\
60.61	0\\
60.62	0\\
60.63	0\\
60.64	0\\
60.65	0\\
60.66	0\\
60.67	0\\
60.68	0\\
60.69	0\\
60.7	0\\
60.71	0\\
60.72	0\\
60.73	0\\
60.74	0\\
60.75	0\\
60.76	0\\
60.77	0\\
60.78	0\\
60.79	0\\
60.8	0\\
60.81	0\\
60.82	0\\
60.83	0\\
60.84	0\\
60.85	0\\
60.86	0\\
60.87	0\\
60.88	0\\
60.89	0\\
60.9	0\\
60.91	0\\
60.92	0\\
60.93	0\\
60.94	0\\
60.95	0\\
60.96	0\\
60.97	0\\
60.98	0\\
60.99	0\\
61	0\\
61.01	0\\
61.02	0\\
61.03	0\\
61.04	0\\
61.05	0\\
61.06	0\\
61.07	0\\
61.08	0\\
61.09	0\\
61.1	0\\
61.11	0\\
61.12	0\\
61.13	0\\
61.14	0\\
61.15	0\\
61.16	0\\
61.17	0\\
61.18	0\\
61.19	0\\
61.2	0\\
61.21	0\\
61.22	0\\
61.23	0\\
61.24	0\\
61.25	0\\
61.26	0\\
61.27	0\\
61.28	0\\
61.29	0\\
61.3	0\\
61.31	0\\
61.32	0\\
61.33	0\\
61.34	0\\
61.35	0\\
61.36	0\\
61.37	0\\
61.38	0\\
61.39	0\\
61.4	0\\
61.41	0\\
61.42	0\\
61.43	0\\
61.44	0\\
61.45	0\\
61.46	0\\
61.47	0\\
61.48	0\\
61.49	0\\
61.5	0\\
61.51	0\\
61.52	0\\
61.53	0\\
61.54	0\\
61.55	0\\
61.56	0\\
61.57	0\\
61.58	0\\
61.59	0\\
61.6	0\\
61.61	0\\
61.62	0\\
61.63	0\\
61.64	0\\
61.65	0\\
61.66	0\\
61.67	0\\
61.68	0\\
61.69	0\\
61.7	0\\
61.71	0\\
61.72	0\\
61.73	0\\
61.74	0\\
61.75	0\\
61.76	0\\
61.77	0\\
61.78	0\\
61.79	0\\
61.8	0\\
61.81	0\\
61.82	0\\
61.83	0\\
61.84	0\\
61.85	0\\
61.86	0\\
61.87	0\\
61.88	0\\
61.89	0\\
61.9	0\\
61.91	0\\
61.92	0\\
61.93	0\\
61.94	0\\
61.95	0\\
61.96	0\\
61.97	0\\
61.98	0\\
61.99	0\\
62	0\\
62.01	0\\
62.02	0\\
62.03	0\\
62.04	0\\
62.05	0\\
62.06	0\\
62.07	0\\
62.08	0\\
62.09	0\\
62.1	0\\
62.11	0\\
62.12	0\\
62.13	0\\
62.14	0\\
62.15	0\\
62.16	0\\
62.17	0\\
62.18	0\\
62.19	0\\
62.2	0\\
62.21	0\\
62.22	0\\
62.23	0\\
62.24	0\\
62.25	0\\
62.26	0\\
62.27	0\\
62.28	0\\
62.29	0\\
62.3	0\\
62.31	0\\
62.32	0\\
62.33	0\\
62.34	0\\
62.35	0\\
62.36	0\\
62.37	0\\
62.38	0\\
62.39	0\\
62.4	0\\
62.41	0\\
62.42	0\\
62.43	0\\
62.44	0\\
62.45	0\\
62.46	0\\
62.47	0\\
62.48	0\\
62.49	0\\
62.5	0\\
62.51	0\\
62.52	0\\
62.53	0\\
62.54	0\\
62.55	0\\
62.56	0\\
62.57	0\\
62.58	0\\
62.59	0\\
62.6	0\\
62.61	0\\
62.62	0\\
62.63	0\\
62.64	0\\
62.65	0\\
62.66	0\\
62.67	0\\
62.68	0\\
62.69	0\\
62.7	0\\
62.71	0\\
62.72	0\\
62.73	0\\
62.74	0\\
62.75	0\\
62.76	0\\
62.77	0\\
62.78	0\\
62.79	0\\
62.8	0\\
62.81	0\\
62.82	0\\
62.83	0\\
62.84	0\\
62.85	0\\
62.86	0\\
62.87	0\\
62.88	0\\
62.89	0\\
62.9	0\\
62.91	0\\
62.92	0\\
62.93	0\\
62.94	0\\
62.95	0\\
62.96	0\\
62.97	0\\
62.98	0\\
62.99	0\\
63	0\\
63.01	0\\
63.02	0\\
63.03	0\\
63.04	0\\
63.05	0\\
63.06	0\\
63.07	0\\
63.08	0\\
63.09	0\\
63.1	0\\
63.11	0\\
63.12	0\\
63.13	0\\
63.14	0\\
63.15	0\\
63.16	0\\
63.17	0\\
63.18	0\\
63.19	0\\
63.2	0\\
63.21	0\\
63.22	0\\
63.23	0\\
63.24	0\\
63.25	0\\
63.26	0\\
63.27	0\\
63.28	0\\
63.29	0\\
63.3	0\\
63.31	0\\
63.32	0\\
63.33	0\\
63.34	0\\
63.35	0\\
63.36	0\\
63.37	0\\
63.38	0\\
63.39	0\\
63.4	0\\
63.41	0\\
63.42	0\\
63.43	0\\
63.44	0\\
63.45	0\\
63.46	0\\
63.47	0\\
63.48	0\\
63.49	0\\
63.5	0\\
63.51	0\\
63.52	0\\
63.53	0\\
63.54	0\\
63.55	0\\
63.56	0\\
63.57	0\\
63.58	0\\
63.59	0\\
63.6	0\\
63.61	0\\
63.62	0\\
63.63	0\\
63.64	0\\
63.65	0\\
63.66	0\\
63.67	0\\
63.68	0\\
63.69	0\\
63.7	0\\
63.71	0\\
63.72	0\\
63.73	0\\
63.74	0\\
63.75	0\\
63.76	0\\
63.77	0\\
63.78	0\\
63.79	0\\
63.8	0\\
63.81	0\\
63.82	0\\
63.83	0\\
63.84	0\\
63.85	0\\
63.86	0\\
63.87	0\\
63.88	0\\
63.89	0\\
63.9	0\\
63.91	0\\
63.92	0\\
63.93	0\\
63.94	0\\
63.95	0\\
63.96	0\\
63.97	0\\
63.98	0\\
63.99	0\\
64	0\\
64.01	0\\
64.02	0\\
64.03	0\\
64.04	0\\
64.05	0\\
64.06	0\\
64.07	0\\
64.08	0\\
64.09	0\\
64.1	0\\
64.11	0\\
64.12	0\\
64.13	0\\
64.14	0\\
64.15	0\\
64.16	0\\
64.17	0\\
64.18	0\\
64.19	0\\
64.2	0\\
64.21	0\\
64.22	0\\
64.23	0\\
64.24	0\\
64.25	0\\
64.26	0\\
64.27	0\\
64.28	0\\
64.29	0\\
64.3	0\\
64.31	0\\
64.32	0\\
64.33	0\\
64.34	0\\
64.35	0\\
64.36	0\\
64.37	0\\
64.38	0\\
64.39	0\\
64.4	0\\
64.41	0\\
64.42	0\\
64.43	0\\
64.44	0\\
64.45	0\\
64.46	0\\
64.47	0\\
64.48	0\\
64.49	0\\
64.5	0\\
64.51	0\\
64.52	0\\
64.53	0\\
64.54	0\\
64.55	0\\
64.56	0\\
64.57	0\\
64.58	0\\
64.59	0\\
64.6	0\\
64.61	0\\
64.62	0\\
64.63	0\\
64.64	0\\
64.65	0\\
64.66	0\\
64.67	0\\
64.68	0\\
64.69	0\\
64.7	0\\
64.71	0\\
64.72	0\\
64.73	0\\
64.74	0\\
64.75	0\\
64.76	0\\
64.77	0\\
64.78	0\\
64.79	0\\
64.8	0\\
64.81	0\\
64.82	0\\
64.83	0\\
64.84	0\\
64.85	0\\
64.86	0\\
64.87	0\\
64.88	0\\
64.89	0\\
64.9	0\\
64.91	0\\
64.92	0\\
64.93	0\\
64.94	0\\
64.95	0\\
64.96	0\\
64.97	0\\
64.98	0\\
64.99	0\\
65	0\\
65.01	0\\
65.02	0\\
65.03	0\\
65.04	0\\
65.05	0\\
65.06	0\\
65.07	0\\
65.08	0\\
65.09	0\\
65.1	0\\
65.11	0\\
65.12	0\\
65.13	0\\
65.14	0\\
65.15	0\\
65.16	0\\
65.17	0\\
65.18	0\\
65.19	0\\
65.2	0\\
65.21	0\\
65.22	0\\
65.23	0\\
65.24	0\\
65.25	0\\
65.26	0\\
65.27	0\\
65.28	0\\
65.29	0\\
65.3	0\\
65.31	0\\
65.32	0\\
65.33	0\\
65.34	0\\
65.35	0\\
65.36	0\\
65.37	0\\
65.38	0\\
65.39	0\\
65.4	0\\
65.41	0\\
65.42	0\\
65.43	0\\
65.44	0\\
65.45	0\\
65.46	0\\
65.47	0\\
65.48	0\\
65.49	0\\
65.5	0\\
65.51	0\\
65.52	0\\
65.53	0\\
65.54	0\\
65.55	0\\
65.56	0\\
65.57	0\\
65.58	0\\
65.59	0\\
65.6	0\\
65.61	0\\
65.62	0\\
65.63	0\\
65.64	0\\
65.65	0\\
65.66	0\\
65.67	0\\
65.68	0\\
65.69	0\\
65.7	0\\
65.71	0\\
65.72	0\\
65.73	0\\
65.74	0\\
65.75	0\\
65.76	0\\
65.77	0\\
65.78	0\\
65.79	0\\
65.8	0\\
65.81	0\\
65.82	0\\
65.83	0\\
65.84	0\\
65.85	0\\
65.86	0\\
65.87	0\\
65.88	0\\
65.89	0\\
65.9	0\\
65.91	0\\
65.92	0\\
65.93	0\\
65.94	0\\
65.95	0\\
65.96	0\\
65.97	0\\
65.98	0\\
65.99	0\\
66	0\\
66.01	0\\
66.02	0\\
66.03	0\\
66.04	0\\
66.05	0\\
66.06	0\\
66.07	0\\
66.08	0\\
66.09	0\\
66.1	0\\
66.11	0\\
66.12	0\\
66.13	0\\
66.14	0\\
66.15	0\\
66.16	0\\
66.17	0\\
66.18	0\\
66.19	0\\
66.2	0\\
66.21	0\\
66.22	0\\
66.23	0\\
66.24	0\\
66.25	0\\
66.26	0\\
66.27	0\\
66.28	0\\
66.29	0\\
66.3	0\\
66.31	0\\
66.32	0\\
66.33	0\\
66.34	0\\
66.35	0\\
66.36	0\\
66.37	0\\
66.38	0\\
66.39	0\\
66.4	0\\
66.41	0\\
66.42	0\\
66.43	0\\
66.44	0\\
66.45	0\\
66.46	0\\
66.47	0\\
66.48	0\\
66.49	0\\
66.5	0\\
66.51	0\\
66.52	0\\
66.53	0\\
66.54	0\\
66.55	0\\
66.56	0\\
66.57	0\\
66.58	0\\
66.59	0\\
66.6	0\\
66.61	0\\
66.62	0\\
66.63	0\\
66.64	0\\
66.65	0\\
66.66	0\\
66.67	0\\
66.68	0\\
66.69	0\\
66.7	0\\
66.71	0\\
66.72	0\\
66.73	0\\
66.74	0\\
66.75	0\\
66.76	0\\
66.77	0\\
66.78	0\\
66.79	0\\
66.8	0\\
66.81	0\\
66.82	0\\
66.83	0\\
66.84	0\\
66.85	0\\
66.86	0\\
66.87	0\\
66.88	0\\
66.89	0\\
66.9	0\\
66.91	0\\
66.92	0\\
66.93	0\\
66.94	0\\
66.95	0\\
66.96	0\\
66.97	0\\
66.98	0\\
66.99	0\\
67	0\\
67.01	0\\
67.02	0\\
67.03	0\\
67.04	0\\
67.05	0\\
67.06	0\\
67.07	0\\
67.08	0\\
67.09	0\\
67.1	0\\
67.11	0\\
67.12	0\\
67.13	0\\
67.14	0\\
67.15	0\\
67.16	0\\
67.17	0\\
67.18	0\\
67.19	0\\
67.2	0\\
67.21	0\\
67.22	0\\
67.23	0\\
67.24	0\\
67.25	0\\
67.26	0\\
67.27	0\\
67.28	0\\
67.29	0\\
67.3	0\\
67.31	0\\
67.32	0\\
67.33	0\\
67.34	0\\
67.35	0\\
67.36	0\\
67.37	0\\
67.38	0\\
67.39	0\\
67.4	0\\
67.41	0\\
67.42	0\\
67.43	0\\
67.44	0\\
67.45	0\\
67.46	0\\
67.47	0\\
67.48	0\\
67.49	0\\
67.5	0\\
67.51	0\\
67.52	0\\
67.53	0\\
67.54	0\\
67.55	0\\
67.56	0\\
67.57	0\\
67.58	0\\
67.59	0\\
67.6	0\\
67.61	0\\
67.62	0\\
67.63	0\\
67.64	0\\
67.65	0\\
67.66	0\\
67.67	0\\
67.68	0\\
67.69	0\\
67.7	0\\
67.71	0\\
67.72	0\\
67.73	0\\
67.74	0\\
67.75	0\\
67.76	0\\
67.77	0\\
67.78	0\\
67.79	0\\
67.8	0\\
67.81	0\\
67.82	0\\
67.83	0\\
67.84	0\\
67.85	0\\
67.86	0\\
67.87	0\\
67.88	0\\
67.89	0\\
67.9	0\\
67.91	0\\
67.92	0\\
67.93	0\\
67.94	0\\
67.95	0\\
67.96	0\\
67.97	0\\
67.98	0\\
67.99	0\\
68	0\\
68.01	0\\
68.02	0\\
68.03	0\\
68.04	0\\
68.05	0\\
68.06	0\\
68.07	0\\
68.08	0\\
68.09	0\\
68.1	0\\
68.11	0\\
68.12	0\\
68.13	0\\
68.14	0\\
68.15	0\\
68.16	0\\
68.17	0\\
68.18	0\\
68.19	0\\
68.2	0\\
68.21	0\\
68.22	0\\
68.23	0\\
68.24	0\\
68.25	0\\
68.26	0\\
68.27	0\\
68.28	0\\
68.29	0\\
68.3	0\\
68.31	0\\
68.32	0\\
68.33	0\\
68.34	0\\
68.35	0\\
68.36	0\\
68.37	0\\
68.38	0\\
68.39	0\\
68.4	0\\
68.41	0\\
68.42	0\\
68.43	0\\
68.44	0\\
68.45	0\\
68.46	0\\
68.47	0\\
68.48	0\\
68.49	0\\
68.5	0\\
68.51	0\\
68.52	0\\
68.53	0\\
68.54	0\\
68.55	0\\
68.56	0\\
68.57	0\\
68.58	0\\
68.59	0\\
68.6	0\\
68.61	0\\
68.62	0\\
68.63	0\\
68.64	0\\
68.65	0\\
68.66	0\\
68.67	0\\
68.68	0\\
68.69	0\\
68.7	0\\
68.71	0\\
68.72	0\\
68.73	0\\
68.74	0\\
68.75	0\\
68.76	0\\
68.77	0\\
68.78	0\\
68.79	0\\
68.8	0\\
68.81	0\\
68.82	0\\
68.83	0\\
68.84	0\\
68.85	0\\
68.86	0\\
68.87	0\\
68.88	0\\
68.89	0\\
68.9	0\\
68.91	0\\
68.92	0\\
68.93	0\\
68.94	0\\
68.95	0\\
68.96	0\\
68.97	0\\
68.98	0\\
68.99	0\\
69	0\\
69.01	0\\
69.02	0\\
69.03	0\\
69.04	0\\
69.05	0\\
69.06	0\\
69.07	0\\
69.08	0\\
69.09	0\\
69.1	0\\
69.11	0\\
69.12	0\\
69.13	0\\
69.14	0\\
69.15	0\\
69.16	0\\
69.17	0\\
69.18	0\\
69.19	0\\
69.2	0\\
69.21	0\\
69.22	0\\
69.23	0\\
69.24	0\\
69.25	0\\
69.26	0\\
69.27	0\\
69.28	0\\
69.29	0\\
69.3	0\\
69.31	0\\
69.32	0\\
69.33	0\\
69.34	0\\
69.35	0\\
69.36	0\\
69.37	0\\
69.38	0\\
69.39	0\\
69.4	0\\
69.41	0\\
69.42	0\\
69.43	0\\
69.44	0\\
69.45	0\\
69.46	0\\
69.47	0\\
69.48	0\\
69.49	0\\
69.5	0\\
69.51	0\\
69.52	0\\
69.53	0\\
69.54	0\\
69.55	0\\
69.56	0\\
69.57	0\\
69.58	0\\
69.59	0\\
69.6	0\\
69.61	0\\
69.62	0\\
69.63	0\\
69.64	0\\
69.65	0\\
69.66	0\\
69.67	0\\
69.68	0\\
69.69	0\\
69.7	0\\
69.71	0\\
69.72	0\\
69.73	0\\
69.74	0\\
69.75	0\\
69.76	0\\
69.77	0\\
69.78	0\\
69.79	0\\
69.8	0\\
69.81	0\\
69.82	0\\
69.83	0\\
69.84	0\\
69.85	0\\
69.86	0\\
69.87	0\\
69.88	0\\
69.89	0\\
69.9	0\\
69.91	0\\
69.92	0\\
69.93	0\\
69.94	0\\
69.95	0\\
69.96	0\\
69.97	0\\
69.98	0\\
69.99	0\\
70	0\\
70.01	0\\
70.02	0\\
70.03	0\\
70.04	0\\
70.05	0\\
70.06	0\\
70.07	0\\
70.08	0\\
70.09	0\\
70.1	0\\
70.11	0\\
70.12	0\\
70.13	0\\
70.14	0\\
70.15	0\\
70.16	0\\
70.17	0\\
70.18	0\\
70.19	0\\
70.2	0\\
70.21	0\\
70.22	0\\
70.23	0\\
70.24	0\\
70.25	0\\
70.26	0\\
70.27	0\\
70.28	0\\
70.29	0\\
70.3	0\\
70.31	0\\
70.32	0\\
70.33	0\\
70.34	0\\
70.35	0\\
70.36	0\\
70.37	0\\
70.38	0\\
70.39	0\\
70.4	0\\
70.41	0\\
70.42	0\\
70.43	0\\
70.44	0\\
70.45	0\\
70.46	0\\
70.47	0\\
70.48	0\\
70.49	0\\
70.5	0\\
70.51	0\\
70.52	0\\
70.53	0\\
70.54	0\\
70.55	0\\
70.56	0\\
70.57	0\\
70.58	0\\
70.59	0\\
70.6	0\\
70.61	0\\
70.62	0\\
70.63	0\\
70.64	0\\
70.65	0\\
70.66	0\\
70.67	0\\
70.68	0\\
70.69	0\\
70.7	0\\
70.71	0\\
70.72	0\\
70.73	0\\
70.74	0\\
70.75	0\\
70.76	0\\
70.77	0\\
70.78	0\\
70.79	0\\
70.8	0\\
70.81	0\\
70.82	0\\
70.83	0\\
70.84	0\\
70.85	0\\
70.86	0\\
70.87	0\\
70.88	0\\
70.89	0\\
70.9	0\\
70.91	0\\
70.92	0\\
70.93	0\\
70.94	0\\
70.95	0\\
70.96	0\\
70.97	0\\
70.98	0\\
70.99	0\\
71	0\\
71.01	0\\
71.02	0\\
71.03	0\\
71.04	0\\
71.05	0\\
71.06	0\\
71.07	0\\
71.08	0\\
71.09	0\\
71.1	0\\
71.11	0\\
71.12	0\\
71.13	0\\
71.14	0\\
71.15	0\\
71.16	0\\
71.17	0\\
71.18	0\\
71.19	0\\
71.2	0\\
71.21	0\\
71.22	0\\
71.23	0\\
71.24	0\\
71.25	0\\
71.26	0\\
71.27	0\\
71.28	0\\
71.29	0\\
71.3	0\\
71.31	0\\
71.32	0\\
71.33	0\\
71.34	0\\
71.35	0\\
71.36	0\\
71.37	0\\
71.38	0\\
71.39	0\\
71.4	0\\
71.41	0\\
71.42	0\\
71.43	0\\
71.44	0\\
71.45	0\\
71.46	0\\
71.47	0\\
71.48	0\\
71.49	0\\
71.5	0\\
71.51	0\\
71.52	0\\
71.53	0\\
71.54	0\\
71.55	0\\
71.56	0\\
71.57	0\\
71.58	0\\
71.59	0\\
71.6	0\\
71.61	0\\
71.62	0\\
71.63	0\\
71.64	0\\
71.65	0\\
71.66	0\\
71.67	0\\
71.68	0\\
71.69	0\\
71.7	0\\
71.71	0\\
71.72	0\\
71.73	0\\
71.74	0\\
71.75	0\\
71.76	0\\
71.77	0\\
71.78	0\\
71.79	0\\
71.8	0\\
71.81	0\\
71.82	0\\
71.83	0\\
71.84	0\\
71.85	0\\
71.86	0\\
71.87	0\\
71.88	0\\
71.89	0\\
71.9	0\\
71.91	0\\
71.92	0\\
71.93	0\\
71.94	0\\
71.95	0\\
71.96	0\\
71.97	0\\
71.98	0\\
71.99	0\\
72	0\\
72.01	0\\
72.02	0\\
72.03	0\\
72.04	0\\
72.05	0\\
72.06	0\\
72.07	0\\
72.08	0\\
72.09	0\\
72.1	0\\
72.11	0\\
72.12	0\\
72.13	0\\
72.14	0\\
72.15	0\\
72.16	0\\
72.17	0\\
72.18	0\\
72.19	0\\
72.2	0\\
72.21	0\\
72.22	0\\
72.23	0\\
72.24	0\\
72.25	0\\
72.26	0\\
72.27	0\\
72.28	0\\
72.29	0\\
72.3	0\\
72.31	0\\
72.32	0\\
72.33	0\\
72.34	0\\
72.35	0\\
72.36	0\\
72.37	0\\
72.38	0\\
72.39	0\\
72.4	0\\
72.41	0\\
72.42	0\\
72.43	0\\
72.44	0\\
72.45	0\\
72.46	0\\
72.47	0\\
72.48	0\\
72.49	0\\
72.5	0\\
72.51	0\\
72.52	0\\
72.53	0\\
72.54	0\\
72.55	0\\
72.56	0\\
72.57	0\\
72.58	0\\
72.59	0\\
72.6	0\\
72.61	0\\
72.62	0\\
72.63	0\\
72.64	0\\
72.65	0\\
72.66	0\\
72.67	0\\
72.68	0\\
72.69	0\\
72.7	0\\
72.71	0\\
72.72	0\\
72.73	0\\
72.74	0\\
72.75	0\\
72.76	0\\
72.77	0\\
72.78	0\\
72.79	0\\
72.8	0\\
72.81	0\\
72.82	0\\
72.83	0\\
72.84	0\\
72.85	0\\
72.86	0\\
72.87	0\\
72.88	0\\
72.89	0\\
72.9	0\\
72.91	0\\
72.92	0\\
72.93	0\\
72.94	0\\
72.95	0\\
72.96	0\\
72.97	0\\
72.98	0\\
72.99	0\\
73	0\\
73.01	0\\
73.02	0\\
73.03	0\\
73.04	0\\
73.05	0\\
73.06	0\\
73.07	0\\
73.08	0\\
73.09	0\\
73.1	0\\
73.11	0\\
73.12	0\\
73.13	0\\
73.14	0\\
73.15	0\\
73.16	0\\
73.17	0\\
73.18	0\\
73.19	0\\
73.2	0\\
73.21	0\\
73.22	0\\
73.23	0\\
73.24	0\\
73.25	0\\
73.26	0\\
73.27	0\\
73.28	0\\
73.29	0\\
73.3	0\\
73.31	0\\
73.32	0\\
73.33	0\\
73.34	0\\
73.35	0\\
73.36	0\\
73.37	0\\
73.38	0\\
73.39	0\\
73.4	0\\
73.41	0\\
73.42	0\\
73.43	0\\
73.44	0\\
73.45	0\\
73.46	0\\
73.47	0\\
73.48	0\\
73.49	0\\
73.5	0\\
73.51	0\\
73.52	0\\
73.53	0\\
73.54	0\\
73.55	0\\
73.56	0\\
73.57	0\\
73.58	0\\
73.59	0\\
73.6	0\\
73.61	0\\
73.62	0\\
73.63	0\\
73.64	0\\
73.65	0\\
73.66	0\\
73.67	0\\
73.68	0\\
73.69	0\\
73.7	0\\
73.71	0\\
73.72	0\\
73.73	0\\
73.74	0\\
73.75	0\\
73.76	0\\
73.77	0\\
73.78	0\\
73.79	0\\
73.8	0\\
73.81	0\\
73.82	0\\
73.83	0\\
73.84	0\\
73.85	0\\
73.86	0\\
73.87	0\\
73.88	0\\
73.89	0\\
73.9	0\\
73.91	0\\
73.92	0\\
73.93	0\\
73.94	0\\
73.95	0\\
73.96	0\\
73.97	0\\
73.98	0\\
73.99	0\\
74	0\\
74.01	0\\
74.02	0\\
74.03	0\\
74.04	0\\
74.05	0\\
74.06	0\\
74.07	0\\
74.08	0\\
74.09	0\\
74.1	0\\
74.11	0\\
74.12	0\\
74.13	0\\
74.14	0\\
74.15	0\\
74.16	0\\
74.17	0\\
74.18	0\\
74.19	0\\
74.2	0\\
74.21	0\\
74.22	0\\
74.23	0\\
74.24	0\\
74.25	0\\
74.26	0\\
74.27	0\\
74.28	0\\
74.29	0\\
74.3	0\\
74.31	0\\
74.32	0\\
74.33	0\\
74.34	0\\
74.35	0\\
74.36	0\\
74.37	0\\
74.38	0\\
74.39	0\\
74.4	0\\
74.41	0\\
74.42	0\\
74.43	0\\
74.44	0\\
74.45	0\\
74.46	0\\
74.47	0\\
74.48	0\\
74.49	0\\
74.5	0\\
74.51	0\\
74.52	0\\
74.53	0\\
74.54	0\\
74.55	0\\
74.56	0\\
74.57	0\\
74.58	0\\
74.59	0\\
74.6	0\\
74.61	0\\
74.62	0\\
74.63	0\\
74.64	0\\
74.65	0\\
74.66	0\\
74.67	0\\
74.68	0\\
74.69	0\\
74.7	0\\
74.71	0\\
74.72	0\\
74.73	0\\
74.74	0\\
74.75	0\\
74.76	0\\
74.77	0\\
74.78	0\\
74.79	0\\
74.8	0\\
74.81	0\\
74.82	0\\
74.83	0\\
74.84	0\\
74.85	0\\
74.86	0\\
74.87	0\\
74.88	0\\
74.89	0\\
74.9	0\\
74.91	0\\
74.92	0\\
74.93	0\\
74.94	0\\
74.95	0\\
74.96	0\\
74.97	0\\
74.98	0\\
74.99	0\\
75	0\\
75.01	0\\
75.02	0\\
75.03	0\\
75.04	0\\
75.05	0\\
75.06	0\\
75.07	0\\
75.08	0\\
75.09	0\\
75.1	0\\
75.11	0\\
75.12	0\\
75.13	0\\
75.14	0\\
75.15	0\\
75.16	0\\
75.17	0\\
75.18	0\\
75.19	0\\
75.2	0\\
75.21	0\\
75.22	0\\
75.23	0\\
75.24	0\\
75.25	0\\
75.26	0\\
75.27	0\\
75.28	0\\
75.29	0\\
75.3	0\\
75.31	0\\
75.32	0\\
75.33	0\\
75.34	0\\
75.35	0\\
75.36	0\\
75.37	0\\
75.38	0\\
75.39	0\\
75.4	0\\
75.41	0\\
75.42	0\\
75.43	0\\
75.44	0\\
75.45	0\\
75.46	0\\
75.47	0\\
75.48	0\\
75.49	0\\
75.5	0\\
75.51	0\\
75.52	1.83244008483338e-06\\
75.53	4.95691426126779e-06\\
75.54	8.08391395710692e-06\\
75.55	1.1213442883462e-05\\
75.56	1.43455047594103e-05\\
75.57	1.74801033120328e-05\\
75.58	2.06172422764286e-05\\
75.59	2.3756925395739e-05\\
75.6	2.68991564211681e-05\\
75.61	3.00439391120143e-05\\
75.62	3.31912772356911e-05\\
75.63	3.63411745677444e-05\\
75.64	3.94936348918837e-05\\
75.65	4.26486620000062e-05\\
75.66	4.58062596922146e-05\\
75.67	4.89664317768514e-05\\
75.68	5.21291820705129e-05\\
75.69	5.52945143980771e-05\\
75.7	5.84624325927346e-05\\
75.71	6.16329404960064e-05\\
75.72	6.48060419577712e-05\\
75.73	6.79817408362864e-05\\
75.74	7.11600409982194e-05\\
75.75	7.43409463186684e-05\\
75.76	7.75244606811901e-05\\
75.77	8.07105879778239e-05\\
75.78	8.38993321091198e-05\\
75.79	8.70906969841559e-05\\
75.8	9.0284686520576e-05\\
75.81	9.34813046446076e-05\\
75.82	9.66805552910892e-05\\
75.83	9.98824424034983e-05\\
75.84	0.000103086969933976\\
75.85	0.000106294141843353\\
75.86	0.00010950396210118\\
75.87	0.000112716434685748\\
75.88	0.00011593156358412\\
75.89	0.000119149352792156\\
75.9	0.00012236980631454\\
75.91	0.00012559292816481\\
75.92	0.000128818722365379\\
75.93	0.000132047192947571\\
75.94	0.000135278343951637\\
75.95	0.000138512179426797\\
75.96	0.000141748703431251\\
75.97	0.000144987920032227\\
75.98	0.000148229833305986\\
75.99	0.000151474447337877\\
76	0.000154721766222339\\
76.01	0.000157971794062944\\
76.02	0.000161224534972429\\
76.03	0.000164479993072713\\
76.04	0.000167738172494932\\
76.05	0.000170999077379469\\
76.06	0.000174262711875987\\
76.07	0.000177529080143447\\
76.08	0.000180798186350143\\
76.09	0.000184070034673741\\
76.1	0.000187344629301298\\
76.11	0.000190621974429289\\
76.12	0.000193902074263647\\
76.13	0.000197184933019789\\
76.14	0.000200470554922651\\
76.15	0.000203758944206708\\
76.16	0.000207050105116013\\
76.17	0.000210344041904228\\
76.18	0.000213640758834652\\
76.19	0.000216940260180255\\
76.2	0.000220242550223706\\
76.21	0.000223547633257409\\
76.22	0.000226855513583531\\
76.23	0.000230166195514039\\
76.24	0.000233479683370726\\
76.25	0.000236795981485245\\
76.26	0.000240115094199144\\
76.27	0.000243437025863897\\
76.28	0.000246761780840939\\
76.29	0.000250089363501696\\
76.3	0.000253419778227619\\
76.31	0.000256753029410219\\
76.32	0.000260089121451101\\
76.33	0.000263428058761995\\
76.34	0.000266769845764788\\
76.35	0.000270114486891571\\
76.36	0.000273461986584653\\
76.37	0.000276812349296617\\
76.38	0.000280165579490337\\
76.39	0.000283521681639021\\
76.4	0.000286880660226248\\
76.41	0.000290242519746006\\
76.42	0.000293607264702712\\
76.43	0.000296974899611266\\
76.44	0.000300345428997078\\
76.45	0.000303718857396103\\
76.46	0.000307095189354886\\
76.47	0.000310474429430586\\
76.48	0.000313856582191026\\
76.49	0.000317241652214719\\
76.5	0.000320629644090912\\
76.51	0.000324020562419626\\
76.52	0.00032741441181168\\
76.53	0.000330811196888749\\
76.54	0.000334210922283387\\
76.55	0.000337613592639067\\
76.56	0.000341019212610229\\
76.57	0.000344427786862313\\
76.58	0.000347839320071795\\
76.59	0.000351253816926227\\
76.6	0.000354671282124287\\
76.61	0.000358091720375806\\
76.62	0.000361515136401813\\
76.63	0.000364941534934583\\
76.64	0.000368370920717662\\
76.65	0.000371803298505922\\
76.66	0.000375238673065596\\
76.67	0.000378677049174318\\
76.68	0.00038211843162117\\
76.69	0.000385562825206721\\
76.7	0.000389010234743067\\
76.71	0.000392460665053877\\
76.72	0.000395914120974439\\
76.73	0.000399370607351693\\
76.74	0.000402830129044286\\
76.75	0.000406292690922606\\
76.76	0.000409758297868833\\
76.77	0.000413226954776982\\
76.78	0.000416698666552941\\
76.79	0.000420173438186528\\
76.8	0.000423651274694466\\
76.81	0.000427132181105395\\
76.82	0.000430616162459919\\
76.83	0.000434103223810649\\
76.84	0.000437593370222245\\
76.85	0.000441086606771467\\
76.86	0.000444582938547211\\
76.87	0.000448082370650561\\
76.88	0.00045158490819483\\
76.89	0.000455090556305614\\
76.9	0.00045859932012083\\
76.91	0.000462111204790762\\
76.92	0.000465626215478115\\
76.93	0.000469144357358061\\
76.94	0.000472665635618277\\
76.95	0.00047619005545901\\
76.96	0.000479717622093107\\
76.97	0.000483248340746079\\
76.98	0.000486782216656139\\
76.99	0.000490319255074256\\
77	0.000493859461264203\\
77.01	0.000497402840502613\\
77.02	0.000500949398079025\\
77.03	0.000504499139295931\\
77.04	0.000508052069468828\\
77.05	0.000511608193926285\\
77.06	0.000515167518009969\\
77.07	0.000518730047074718\\
77.08	0.000522295786488589\\
77.09	0.000525864741632905\\
77.1	0.000529436917902314\\
77.11	0.000533012320704841\\
77.12	0.000536590955461946\\
77.13	0.000540172827608574\\
77.14	0.000543757942593211\\
77.15	0.000547346305877942\\
77.16	0.000550937922938508\\
77.17	0.000554532799264359\\
77.18	0.000558130940358718\\
77.19	0.000561732351738623\\
77.2	0.000565337038935007\\
77.21	0.000568945007492733\\
77.22	0.000572556262970678\\
77.23	0.000576170810941766\\
77.24	0.000579788656993051\\
77.25	0.000583409806725763\\
77.26	0.000587034265755369\\
77.27	0.000590662039711646\\
77.28	0.00059429313423873\\
77.29	0.000597927554995187\\
77.3	0.000601565307654071\\
77.31	0.000605206397902985\\
77.32	0.000608850831444154\\
77.33	0.000612498613994483\\
77.34	0.00061614975128562\\
77.35	0.000619804249064028\\
77.36	0.000623462113091045\\
77.37	0.000627123349142954\\
77.38	0.00063078796301105\\
77.39	0.000634455960501708\\
77.4	0.00063812734743645\\
77.41	0.000641802129652012\\
77.42	0.000645480313000418\\
77.43	0.000649161903349051\\
77.44	0.000652846906580715\\
77.45	0.000656535328593714\\
77.46	0.000660227175301929\\
77.47	0.000663922452634871\\
77.48	0.00066762116653778\\
77.49	0.000671323322971682\\
77.5	0.000675028927913462\\
77.51	0.000678737987355953\\
77.52	0.000682450507307995\\
77.53	0.000686166493794528\\
77.54	0.000689885952856659\\
77.55	0.000693608890551745\\
77.56	0.000697335312953466\\
77.57	0.000701065226151903\\
77.58	0.000704798636253636\\
77.59	0.0007085355493818\\
77.6	0.000712275971676182\\
77.61	0.000716019909293302\\
77.62	0.000719767368406492\\
77.63	0.000723518355205979\\
77.64	0.000727272875898976\\
77.65	0.000731030936709762\\
77.66	0.000734792543879769\\
77.67	0.000738557703667674\\
77.68	0.00074232642234948\\
77.69	0.000746098706218605\\
77.7	0.00074987456158598\\
77.71	0.000753653994780133\\
77.72	0.000757437012147276\\
77.73	0.000761223620051404\\
77.74	0.000765013824874389\\
77.75	0.000768807633016066\\
77.76	0.00077260505089434\\
77.77	0.000776406084945266\\
77.78	0.000780210741623162\\
77.79	0.000784019027400697\\
77.8	0.000787830948768987\\
77.81	0.000791646512237704\\
77.82	0.000795465724335167\\
77.83	0.000799288591608455\\
77.84	0.000803115120623497\\
77.85	0.000806945317965182\\
77.86	0.000810779190237466\\
77.87	0.000814616744063471\\
77.88	0.000818457986085597\\
77.89	0.000822302922965625\\
77.9	0.000826151561384832\\
77.91	0.000830003908044094\\
77.92	0.000833859969664008\\
77.93	0.000837719752984988\\
77.94	0.000841583264767392\\
77.95	0.000845450511791631\\
77.96	0.000849321500858282\\
77.97	0.000853196238788215\\
77.98	0.000857074732422698\\
77.99	0.000860956988623523\\
78	0.000864843014273128\\
78.01	0.000868732816274718\\
78.02	0.000872626401552383\\
78.03	0.000876523777051223\\
78.04	0.000880424949737484\\
78.05	0.000884329926598672\\
78.06	0.000888238714643691\\
78.07	0.00089215132086555\\
78.08	0.000896067751923985\\
78.09	0.000899988014494167\\
78.1	0.000903912115266737\\
78.11	0.000907840060947851\\
78.12	0.000911771858259219\\
78.13	0.000915707513938154\\
78.14	0.000919647034737611\\
78.15	0.000923590427426228\\
78.16	0.000927537698788377\\
78.17	0.000931488855624194\\
78.18	0.000935443904749641\\
78.19	0.00093940285299653\\
78.2	0.000943365707212582\\
78.21	0.000947332474261459\\
78.22	0.000951303161022814\\
78.23	0.000955277774392339\\
78.24	0.000959256321281798\\
78.25	0.000963238808619083\\
78.26	0.000967225243348253\\
78.27	0.000971215632429575\\
78.28	0.000975209982839578\\
78.29	0.000979208301571086\\
78.3	0.000983210595633277\\
78.31	0.000987216872051715\\
78.32	0.000991227137868401\\
78.33	0.000995241400141823\\
78.34	0.000999259665946994\\
78.35	0.0010032819423755\\
78.36	0.00100730823653555\\
78.37	0.00101133855555201\\
78.38	0.00101537290656647\\
78.39	0.00101941129673727\\
78.4	0.00102345373323956\\
78.41	0.00102750022326532\\
78.42	0.00103155077402347\\
78.43	0.00103560539273983\\
78.44	0.00103966408665724\\
78.45	0.00104372686303557\\
78.46	0.00104779372915178\\
78.47	0.00105186469229994\\
78.48	0.00105593975979134\\
78.49	0.00106001893895446\\
78.5	0.00106410223713509\\
78.51	0.00106818966169633\\
78.52	0.00107228122001864\\
78.53	0.00107637691949993\\
78.54	0.00108047676755555\\
78.55	0.00108458077161838\\
78.56	0.00108868893913888\\
78.57	0.00109280127758511\\
78.58	0.00109691779444279\\
78.59	0.00110103849721538\\
78.6	0.00110516339342407\\
78.61	0.00110929249060788\\
78.62	0.00111342579632371\\
78.63	0.00111756331814633\\
78.64	0.00112170506366852\\
78.65	0.00112585104050104\\
78.66	0.00113000125627273\\
78.67	0.00113415571863052\\
78.68	0.00113831443523954\\
78.69	0.0011424774137831\\
78.7	0.0011466446619628\\
78.71	0.00115081618749854\\
78.72	0.00115499199812859\\
78.73	0.00115917210160967\\
78.74	0.00116335650571692\\
78.75	0.00116754521824404\\
78.76	0.00117173824700329\\
78.77	0.00117593559982556\\
78.78	0.00118013728456042\\
78.79	0.00118434330907616\\
78.8	0.00118855368125987\\
78.81	0.00119276840901747\\
78.82	0.00119698750027375\\
78.83	0.00120121096297248\\
78.84	0.00120543880507639\\
78.85	0.00120967103456729\\
78.86	0.00121390765944605\\
78.87	0.00121814868773274\\
78.88	0.00122239412746661\\
78.89	0.00122664398670618\\
78.9	0.0012308982735293\\
78.91	0.00123515699603318\\
78.92	0.00123942016233444\\
78.93	0.00124368778056922\\
78.94	0.00124795985889316\\
78.95	0.00125223640548151\\
78.96	0.00125651742852917\\
78.97	0.00126080293625071\\
78.98	0.00126509293688049\\
78.99	0.00126938743867267\\
79	0.00127368644990128\\
79.01	0.00127798997886026\\
79.02	0.00128229803386356\\
79.03	0.00128661062324513\\
79.04	0.00129092775535904\\
79.05	0.00129524943857949\\
79.06	0.00129957568130092\\
79.07	0.00130390649193799\\
79.08	0.0013082418789257\\
79.09	0.00131258185071945\\
79.1	0.00131692641579503\\
79.11	0.00132127558264875\\
79.12	0.00132562935979749\\
79.13	0.00132998775577869\\
79.14	0.00133435077915051\\
79.15	0.00133871843849179\\
79.16	0.00134309074240218\\
79.17	0.00134746769950218\\
79.18	0.00135184931843317\\
79.19	0.00135623560785751\\
79.2	0.00136062657645857\\
79.21	0.00136502223294079\\
79.22	0.00136942258602978\\
79.23	0.00137382764447231\\
79.24	0.00137823741703646\\
79.25	0.00138265191251158\\
79.26	0.00138707113970843\\
79.27	0.00139149510745919\\
79.28	0.00139592382461757\\
79.29	0.00140035730005882\\
79.3	0.00140479554267982\\
79.31	0.00140923856139914\\
79.32	0.00141368636515708\\
79.33	0.00141813896291576\\
79.34	0.00142259636365918\\
79.35	0.00142705857639326\\
79.36	0.00143152561014591\\
79.37	0.0014359974739671\\
79.38	0.00144047417692892\\
79.39	0.00144495572812566\\
79.4	0.00144944213667383\\
79.41	0.00145393341171225\\
79.42	0.00145842956240213\\
79.43	0.00146293059792709\\
79.44	0.00146743652749328\\
79.45	0.00147194736032938\\
79.46	0.00147646310568673\\
79.47	0.00148098377283933\\
79.48	0.00148550937108395\\
79.49	0.0014900399097402\\
79.5	0.00149457539815053\\
79.51	0.00149911584568039\\
79.52	0.00150366126171821\\
79.53	0.00150821165567551\\
79.54	0.00151276703698697\\
79.55	0.00151732741511047\\
79.56	0.00152189279952718\\
79.57	0.00152646319974159\\
79.58	0.00153103862528164\\
79.59	0.00153561908569871\\
79.6	0.00154020459056775\\
79.61	0.00154479514948732\\
79.62	0.00154939077207965\\
79.63	0.00155399146799072\\
79.64	0.00155859724689033\\
79.65	0.00156320811847215\\
79.66	0.00156782409245381\\
79.67	0.00157244517857697\\
79.68	0.00157707138660734\\
79.69	0.00158170272633483\\
79.7	0.00158633920757354\\
79.71	0.00159098084016189\\
79.72	0.00159562763396264\\
79.73	0.00160027959886299\\
79.74	0.00160493674477465\\
79.75	0.00160959908163389\\
79.76	0.00161426661940164\\
79.77	0.00161893936806351\\
79.78	0.00162361733762992\\
79.79	0.00162830053813614\\
79.8	0.00163298897964236\\
79.81	0.00163768267223374\\
79.82	0.00164238162602055\\
79.83	0.00164708585113818\\
79.84	0.00165179535774721\\
79.85	0.00165651015603352\\
79.86	0.00166123025620835\\
79.87	0.00166595566850836\\
79.88	0.00167068640319569\\
79.89	0.00167542247055807\\
79.9	0.00168016388090887\\
79.91	0.00168491064458717\\
79.92	0.00168966277195785\\
79.93	0.00169442027341166\\
79.94	0.00169918315936526\\
79.95	0.00170395144026135\\
79.96	0.0017087251265687\\
79.97	0.00171350422878224\\
79.98	0.00171828875742314\\
79.99	0.00172307872303887\\
80	0.00172787413620331\\
80.01	0.00173267500751677\\
};
\addplot [color=blue,solid]
  table[row sep=crcr]{%
80.01	0.00173267500751677\\
80.02	0.0017374813476061\\
80.03	0.00174229316712478\\
80.04	0.00174711047675296\\
80.05	0.00175193328719754\\
80.06	0.00175676160919229\\
80.07	0.00176159545349788\\
80.08	0.00176643483090196\\
80.09	0.00177127975221927\\
80.1	0.00177613022829169\\
80.11	0.0017809862699883\\
80.12	0.00178584788820552\\
80.13	0.00179071509386712\\
80.14	0.00179558789792433\\
80.15	0.00180046631135592\\
80.16	0.00180535034516829\\
80.17	0.00181024001039549\\
80.18	0.00181513531809937\\
80.19	0.00182003627936964\\
80.2	0.0018249429053239\\
80.21	0.0018298552071078\\
80.22	0.00183477319589504\\
80.23	0.00183969688288753\\
80.24	0.00184462627931538\\
80.25	0.00184956139643707\\
80.26	0.00185450224553945\\
80.27	0.00185944883793791\\
80.28	0.00186440118497635\\
80.29	0.00186935929802737\\
80.3	0.00187432318849228\\
80.31	0.00187929286780121\\
80.32	0.00188426834741318\\
80.33	0.0018892496388162\\
80.34	0.00189423675352733\\
80.35	0.00189922970309278\\
80.36	0.00190422849908798\\
80.37	0.00190923315311767\\
80.38	0.00191424367681599\\
80.39	0.00191926008184654\\
80.4	0.00192428237990249\\
80.41	0.00192931058270664\\
80.42	0.0019343431849562\\
80.43	0.00193937914030812\\
80.44	0.00194441845210473\\
80.45	0.00194946112369128\\
80.46	0.00195450715841599\\
80.47	0.00195955655963\\
80.48	0.00196460933068741\\
80.49	0.00196966547494521\\
80.5	0.00197472499576333\\
80.51	0.00197978789650457\\
80.52	0.00198485418053465\\
80.53	0.00198992385122214\\
80.54	0.00199499691193851\\
80.55	0.00200007336605808\\
80.56	0.00200515321695801\\
80.57	0.00201023646801831\\
80.58	0.0020153231226218\\
80.59	0.00202041318415414\\
80.6	0.00202550665600379\\
80.61	0.00203060354156197\\
80.62	0.00203570384422274\\
80.63	0.00204080756738287\\
80.64	0.00204591471444192\\
80.65	0.0020510252888022\\
80.66	0.00205613929386872\\
80.67	0.00206125673304925\\
80.68	0.00206637760975423\\
80.69	0.00207150192739683\\
80.7	0.00207662968939287\\
80.71	0.00208176089916084\\
80.72	0.00208689556012191\\
80.73	0.00209203367569986\\
80.74	0.00209717524932111\\
80.75	0.00210232028441468\\
80.76	0.00210746878441221\\
80.77	0.00211262075274789\\
80.78	0.0021177761928585\\
80.79	0.00212293510818336\\
80.8	0.00212809750216434\\
80.81	0.00213326337824581\\
80.82	0.00213843273987467\\
80.83	0.00214360559050028\\
80.84	0.00214878193357449\\
80.85	0.00215396177255161\\
80.86	0.00215914511088839\\
80.87	0.00216433195204398\\
80.88	0.00216952229947996\\
80.89	0.00217471615666029\\
80.9	0.00217991352705129\\
80.91	0.00218511441412165\\
80.92	0.00219031882134237\\
80.93	0.00219552675218679\\
80.94	0.00220073821013054\\
80.95	0.00220595319865151\\
80.96	0.00221117172122986\\
80.97	0.002216393781348\\
80.98	0.00222161938249054\\
80.99	0.0022268485281443\\
81	0.00223208122179828\\
81.01	0.00223731746694362\\
81.02	0.00224255726707362\\
81.03	0.00224780062568368\\
81.04	0.0022530475462713\\
81.05	0.00225829803233605\\
81.06	0.00226355208737955\\
81.07	0.00226880971490545\\
81.08	0.0022740709184194\\
81.09	0.00227933570142903\\
81.1	0.00228460406744393\\
81.11	0.00228987601997565\\
81.12	0.00229515156253761\\
81.13	0.00230043069864514\\
81.14	0.00230571343181543\\
81.15	0.0023109997655675\\
81.16	0.00231628970342218\\
81.17	0.00232158324890211\\
81.18	0.00232688040553165\\
81.19	0.00233218117683692\\
81.2	0.00233748556634576\\
81.21	0.00234279357758764\\
81.22	0.00234810521409375\\
81.23	0.00235342047939684\\
81.24	0.0023587393770313\\
81.25	0.00236406191053308\\
81.26	0.00236938808343965\\
81.27	0.00237471789929002\\
81.28	0.00238005136162465\\
81.29	0.00238538847398546\\
81.3	0.00239072923991582\\
81.31	0.00239607366296044\\
81.32	0.00240142174666542\\
81.33	0.00240677349457818\\
81.34	0.00241212891024743\\
81.35	0.00241748799722314\\
81.36	0.00242285075905653\\
81.37	0.00242821719929999\\
81.38	0.00243358732150707\\
81.39	0.00243896112923248\\
81.4	0.00244433862603199\\
81.41	0.00244971981546244\\
81.42	0.00245510470108169\\
81.43	0.00246049328644861\\
81.44	0.00246588557512298\\
81.45	0.00247128157066552\\
81.46	0.00247668127663782\\
81.47	0.00248208469660231\\
81.48	0.00248749183412222\\
81.49	0.00249290269276153\\
81.5	0.00249831727608496\\
81.51	0.0025037355876579\\
81.52	0.00250915763104638\\
81.53	0.00251458340981704\\
81.54	0.00252001292753707\\
81.55	0.00252544618777418\\
81.56	0.00253088319409657\\
81.57	0.00253632395007284\\
81.58	0.00254176845927201\\
81.59	0.00254721672526342\\
81.6	0.00255266875161674\\
81.61	0.00255812454190187\\
81.62	0.00256358409968892\\
81.63	0.00256904742854817\\
81.64	0.00257451453205002\\
81.65	0.00257998541376493\\
81.66	0.00258546007726339\\
81.67	0.00259093852611584\\
81.68	0.00259642076389268\\
81.69	0.00260190679416414\\
81.7	0.0026073966205003\\
81.71	0.002612890246471\\
81.72	0.00261838767564581\\
81.73	0.00262388891159394\\
81.74	0.00262939395788426\\
81.75	0.00263490281808514\\
81.76	0.00264041549576451\\
81.77	0.00264593199448972\\
81.78	0.00265145231782752\\
81.79	0.00265697646934399\\
81.8	0.00266250445260451\\
81.81	0.00266803627117367\\
81.82	0.00267357192861522\\
81.83	0.00267911142849203\\
81.84	0.002684654774366\\
81.85	0.00269020196979801\\
81.86	0.00269575301834789\\
81.87	0.0027013079235743\\
81.88	0.00270686668903471\\
81.89	0.00271242931828532\\
81.9	0.002717995814881\\
81.91	0.00272356618237522\\
81.92	0.00272914042431999\\
81.93	0.00273471854426579\\
81.94	0.0027403005457615\\
81.95	0.00274588643235433\\
81.96	0.00275147620758974\\
81.97	0.00275706987501142\\
81.98	0.00276266743816115\\
81.99	0.00276826890057877\\
82	0.00277387426580209\\
82.01	0.00277948353736682\\
82.02	0.00278509671880652\\
82.03	0.00279071381365248\\
82.04	0.00279633482543366\\
82.05	0.00280195975767663\\
82.06	0.00280758861390548\\
82.07	0.00281322139764173\\
82.08	0.00281885811240426\\
82.09	0.00282449876170924\\
82.1	0.00283014334907001\\
82.11	0.00283579187799704\\
82.12	0.00284144435199781\\
82.13	0.00284710077457677\\
82.14	0.0028527611492352\\
82.15	0.00285842547947114\\
82.16	0.00286409376877932\\
82.17	0.00286976602065107\\
82.18	0.0028754422385742\\
82.19	0.00288112242603292\\
82.2	0.00288680658650777\\
82.21	0.00289249472347549\\
82.22	0.00289818684040896\\
82.23	0.00290388294077707\\
82.24	0.00290958302804465\\
82.25	0.00291528710567234\\
82.26	0.00292099517711654\\
82.27	0.00292670724582925\\
82.28	0.00293242331525801\\
82.29	0.00293814338884578\\
82.3	0.00294386747003082\\
82.31	0.00294959556224663\\
82.32	0.00295532766892178\\
82.33	0.00296106379347985\\
82.34	0.00296680393933932\\
82.35	0.0029725481099134\\
82.36	0.00297829630861\\
82.37	0.00298404853950593\\
82.38	0.00298980480669384\\
82.39	0.00299556511426892\\
82.4	0.00300132946632891\\
82.41	0.00300709786697403\\
82.42	0.00301287032030701\\
82.43	0.00301864683043302\\
82.44	0.00302442740145972\\
82.45	0.00303021203749715\\
82.46	0.00303600074265779\\
82.47	0.00304179352105649\\
82.48	0.00304759037681047\\
82.49	0.00305339131403929\\
82.5	0.00305919633686482\\
82.51	0.00306500544941125\\
82.52	0.00307081865580504\\
82.53	0.0030766359601749\\
82.54	0.00308245736665176\\
82.55	0.00308828287936878\\
82.56	0.0030941125024613\\
82.57	0.00309994624006681\\
82.58	0.00310578409632495\\
82.59	0.00311162607537746\\
82.6	0.00311747218136819\\
82.61	0.00312332241844303\\
82.62	0.00312917679074992\\
82.63	0.00313503530243884\\
82.64	0.0031408979576617\\
82.65	0.00314676476057243\\
82.66	0.00315263571532686\\
82.67	0.00315851082608274\\
82.68	0.00316439009699972\\
82.69	0.00317027353223928\\
82.7	0.00317616113596473\\
82.71	0.00318205291234121\\
82.72	0.00318794886553559\\
82.73	0.00319384899971653\\
82.74	0.00319975331905436\\
82.75	0.00320566182772112\\
82.76	0.0032115745298905\\
82.77	0.00321749142973784\\
82.78	0.00322341253144003\\
82.79	0.00322933783917557\\
82.8	0.00323526735712447\\
82.81	0.00324120108946825\\
82.82	0.00324713904038991\\
82.83	0.00325308121407388\\
82.84	0.003259027614706\\
82.85	0.00326497824647349\\
82.86	0.00327093311356491\\
82.87	0.00327689222017012\\
82.88	0.00328285557048028\\
82.89	0.00328882316868775\\
82.9	0.00329479501898614\\
82.91	0.0033007711255702\\
82.92	0.00330675149263583\\
82.93	0.00331273612438003\\
82.94	0.00331872502500085\\
82.95	0.00332471819869738\\
82.96	0.00333071564966969\\
82.97	0.00333671738211882\\
82.98	0.0033427234002467\\
82.99	0.00334873370825615\\
83	0.00335474831035082\\
83.01	0.00336076721073516\\
83.02	0.00336679041361439\\
83.03	0.00337281792319442\\
83.04	0.00337884974368186\\
83.05	0.00338488587928394\\
83.06	0.0033909263342085\\
83.07	0.00339697111266392\\
83.08	0.00340302021885908\\
83.09	0.00340907365700336\\
83.1	0.00341513143130652\\
83.11	0.00342119354597872\\
83.12	0.00342726000523046\\
83.13	0.00343333081327251\\
83.14	0.00343940597431591\\
83.15	0.00344548549257186\\
83.16	0.00345156937225174\\
83.17	0.00345765761756701\\
83.18	0.00346375023272921\\
83.19	0.00346984722194987\\
83.2	0.00347594858944048\\
83.21	0.00348205433941243\\
83.22	0.00348816447607699\\
83.23	0.00349427900364522\\
83.24	0.00350039792632793\\
83.25	0.00350652124833567\\
83.26	0.0035126489738786\\
83.27	0.0035187811071665\\
83.28	0.0035249176524087\\
83.29	0.00353105861381401\\
83.3	0.00353720399559068\\
83.31	0.00354335380194635\\
83.32	0.00354950803708797\\
83.33	0.00355566670522177\\
83.34	0.00356182981055318\\
83.35	0.0035679973572868\\
83.36	0.00357416934962632\\
83.37	0.00358034579177445\\
83.38	0.00358652668793288\\
83.39	0.00359271204230224\\
83.4	0.00359890185908198\\
83.41	0.00360509614247035\\
83.42	0.00361129489666434\\
83.43	0.00361749812585958\\
83.44	0.00362370583425033\\
83.45	0.00362991802602937\\
83.46	0.00363613470538794\\
83.47	0.00364235587651569\\
83.48	0.00364858154360062\\
83.49	0.00365481171082897\\
83.5	0.0036610463823852\\
83.51	0.00366728556245188\\
83.52	0.00367352925520967\\
83.53	0.00367977746483717\\
83.54	0.00368603019551094\\
83.55	0.00369228745140535\\
83.56	0.00369854923669256\\
83.57	0.0037048155555424\\
83.58	0.00371108641212235\\
83.59	0.00371736181059739\\
83.6	0.00372364175513001\\
83.61	0.00372992624988005\\
83.62	0.00373621529900467\\
83.63	0.00374250890665827\\
83.64	0.00374880707699237\\
83.65	0.00375510981415559\\
83.66	0.00376141712229349\\
83.67	0.00376772900554856\\
83.68	0.0037740454680601\\
83.69	0.00378036651396413\\
83.7	0.0037866921473933\\
83.71	0.00379302237247685\\
83.72	0.00379935719334045\\
83.73	0.00380569661410616\\
83.74	0.00381204063889233\\
83.75	0.0038183892718135\\
83.76	0.0038247425169803\\
83.77	0.00383110037849939\\
83.78	0.00383746286047331\\
83.79	0.00384382996700045\\
83.8	0.0038502017021749\\
83.81	0.00385657807008638\\
83.82	0.00386295907482011\\
83.83	0.00386934472045676\\
83.84	0.00387573501107231\\
83.85	0.00388212995073794\\
83.86	0.00388852954351997\\
83.87	0.0038949337934797\\
83.88	0.00390134270467335\\
83.89	0.00390775628115192\\
83.9	0.00391417452696111\\
83.91	0.00392059744614118\\
83.92	0.00392702504272686\\
83.93	0.00393345732074723\\
83.94	0.00393989428422559\\
83.95	0.00394633593717938\\
83.96	0.00395278228362005\\
83.97	0.00395923332755291\\
83.98	0.00396568907297705\\
83.99	0.00397214952388521\\
84	0.00397861468426364\\
84.01	0.00398508455809201\\
84.02	0.00399155914934324\\
84.03	0.00399803846198341\\
84.04	0.00400452249997163\\
84.05	0.00401101126725989\\
84.06	0.00401750476779292\\
84.07	0.00402400300550813\\
84.08	0.00403050598433536\\
84.09	0.00403701370819685\\
84.1	0.00404352618100706\\
84.11	0.00405004340667249\\
84.12	0.00405656524916951\\
84.13	0.00406309155378942\\
84.14	0.00406962232515156\\
84.15	0.00407615756787363\\
84.16	0.00408269728657157\\
84.17	0.00408924148585945\\
84.18	0.00409579017034934\\
84.19	0.00410234334465123\\
84.2	0.00410890101337289\\
84.21	0.00411546318111974\\
84.22	0.00412202985249477\\
84.23	0.00412860103209834\\
84.24	0.00413517672452817\\
84.25	0.00414175693437912\\
84.26	0.0041483416662431\\
84.27	0.00415493092470892\\
84.28	0.0041615247143622\\
84.29	0.00416812303978521\\
84.3	0.00417472590555673\\
84.31	0.00418133331625192\\
84.32	0.0041879452764422\\
84.33	0.00419456179069509\\
84.34	0.00420118286357406\\
84.35	0.00420780849963841\\
84.36	0.00421443870344312\\
84.37	0.0042210734795387\\
84.38	0.00422771283247101\\
84.39	0.00423435676678117\\
84.4	0.00424100528700535\\
84.41	0.00424765839767464\\
84.42	0.00425431610331489\\
84.43	0.00426097840844655\\
84.44	0.0042676453175845\\
84.45	0.0042743168352379\\
84.46	0.00428099296591002\\
84.47	0.00428767371409804\\
84.48	0.00429435908429294\\
84.49	0.00430104908097926\\
84.5	0.00430774370863498\\
84.51	0.0043144429717313\\
84.52	0.0043211468747325\\
84.53	0.0043278554220957\\
84.54	0.00433456861827074\\
84.55	0.00434128646769996\\
84.56	0.00434800897481798\\
84.57	0.00435473614405158\\
84.58	0.00436146797981943\\
84.59	0.00436820448653193\\
84.6	0.00437494566859101\\
84.61	0.00438169153038993\\
84.62	0.00438844207631304\\
84.63	0.0043951973107356\\
84.64	0.00440195723802358\\
84.65	0.00440872186253341\\
84.66	0.00441549118861179\\
84.67	0.00442226522059545\\
84.68	0.00442904396281094\\
84.69	0.00443582741957442\\
84.7	0.00444261559519138\\
84.71	0.00444940849395645\\
84.72	0.00445620612015317\\
84.73	0.00446300847805372\\
84.74	0.00446981557191868\\
84.75	0.00447662740599682\\
84.76	0.00448344398452482\\
84.77	0.00449026531172703\\
84.78	0.00449709139181521\\
84.79	0.00450392222898828\\
84.8	0.00451075782743204\\
84.81	0.00451759819131893\\
84.82	0.00452444332480776\\
84.83	0.0045312932320434\\
84.84	0.00453814791715654\\
84.85	0.00454500738426341\\
84.86	0.00455187163746549\\
84.87	0.00455874068084921\\
84.88	0.00456561451848568\\
84.89	0.00457249315443037\\
84.9	0.00457937659272284\\
84.91	0.00458626483738643\\
84.92	0.00459315789507986\\
84.93	0.0046000557728471\\
84.94	0.00460695847774704\\
84.95	0.00461386601685348\\
84.96	0.00462077839725518\\
84.97	0.00462769562605585\\
84.98	0.00463461771037419\\
84.99	0.00464154465734389\\
85	0.00464847647411363\\
85.01	0.00465541316784714\\
85.02	0.00466235474572317\\
85.03	0.00466930121493551\\
85.04	0.00467625258269305\\
85.05	0.00468320885621974\\
85.06	0.00469017004275461\\
85.07	0.00469713614955181\\
85.08	0.0047041071838806\\
85.09	0.00471108315302539\\
85.1	0.00471806406428569\\
85.11	0.0047250499249762\\
85.12	0.00473204074242676\\
85.13	0.00473903652398239\\
85.14	0.00474603727700328\\
85.15	0.00475304300886484\\
85.16	0.00476005372695766\\
85.17	0.00476706943868753\\
85.18	0.00477409015147547\\
85.19	0.00478111587275772\\
85.2	0.00478814660998576\\
85.21	0.0047951823706263\\
85.22	0.00480222316216129\\
85.23	0.00480926899208792\\
85.24	0.00481631986791867\\
85.25	0.00482337579718123\\
85.26	0.00483043678741858\\
85.27	0.00483750284618898\\
85.28	0.00484457398106592\\
85.29	0.00485165019963818\\
85.3	0.00485873150950981\\
85.31	0.00486581791830015\\
85.32	0.00487290943364378\\
85.33	0.00488000606319056\\
85.34	0.00488710781460566\\
85.35	0.00489421469556947\\
85.36	0.00490132671377768\\
85.37	0.00490844387694122\\
85.38	0.00491556619278632\\
85.39	0.00492269366905444\\
85.4	0.00492982631350229\\
85.41	0.00493696413390185\\
85.42	0.00494410713804032\\
85.43	0.00495125533372015\\
85.44	0.00495840872875902\\
85.45	0.00496556733098981\\
85.46	0.00497273114826064\\
85.47	0.00497990018843479\\
85.48	0.00498707445939076\\
85.49	0.00499425396902222\\
85.5	0.00500143872523801\\
85.51	0.00500862873596212\\
85.52	0.00501582400913367\\
85.53	0.00502302455270692\\
85.54	0.00503023037465123\\
85.55	0.00503744148295105\\
85.56	0.00504465788560591\\
85.57	0.0050518795906304\\
85.58	0.00505910660605415\\
85.59	0.0050663389399218\\
85.6	0.00507357660029298\\
85.61	0.00508081959524231\\
85.62	0.00508806793285937\\
85.63	0.00509532162124864\\
85.64	0.00510258066852953\\
85.65	0.00510984508283632\\
85.66	0.00511711487231813\\
85.67	0.00512439004513892\\
85.68	0.00513167060947744\\
85.69	0.0051389565735272\\
85.7	0.00514624794549645\\
85.71	0.00515354473360812\\
85.72	0.00516084694609984\\
85.73	0.00516815459122386\\
85.74	0.00517546767724701\\
85.75	0.00518278621245071\\
85.76	0.00519011020513089\\
85.77	0.00519743966359795\\
85.78	0.00520477459617676\\
85.79	0.00521211501120658\\
85.8	0.00521946091704101\\
85.81	0.00522681232204799\\
85.82	0.00523416923460973\\
85.83	0.00524153166312265\\
85.84	0.00524889961599734\\
85.85	0.00525627310165852\\
85.86	0.00526365212854499\\
85.87	0.00527103670510956\\
85.88	0.00527842683981903\\
85.89	0.00528582254115407\\
85.9	0.00529322381760925\\
85.91	0.00530063067769291\\
85.92	0.00530804312992714\\
85.93	0.0053154611828477\\
85.94	0.00532288484500396\\
85.95	0.00533031412495885\\
85.96	0.00533774903128877\\
85.97	0.00534518957258356\\
85.98	0.00535263575744638\\
85.99	0.00536008759449367\\
86	0.00536754509235508\\
86.01	0.0053750082596734\\
86.02	0.00538247710510444\\
86.03	0.00538995163731701\\
86.04	0.00539743186499281\\
86.05	0.00540491779682635\\
86.06	0.00541240944152486\\
86.07	0.00541990680780824\\
86.08	0.00542740990440893\\
86.09	0.00543491874007184\\
86.1	0.00544243332355425\\
86.11	0.00544995366362573\\
86.12	0.00545747976906806\\
86.13	0.00546501164867508\\
86.14	0.00547254931125263\\
86.15	0.00548009276561845\\
86.16	0.00548764202060207\\
86.17	0.00549519708504469\\
86.18	0.0055027579677991\\
86.19	0.00551032467772953\\
86.2	0.00551789722371158\\
86.21	0.00552547561463209\\
86.22	0.005533059859389\\
86.23	0.00554064996689127\\
86.24	0.00554824594605873\\
86.25	0.00555584780582197\\
86.26	0.00556345555512219\\
86.27	0.00557106920291112\\
86.28	0.00557868875815082\\
86.29	0.00558631422981361\\
86.3	0.00559394562688189\\
86.31	0.00560158295834803\\
86.32	0.00560922623321419\\
86.33	0.00561687546049221\\
86.34	0.00562453064920347\\
86.35	0.00563219180837868\\
86.36	0.00563985894705779\\
86.37	0.00564753207428982\\
86.38	0.00565521119913267\\
86.39	0.00566289633065297\\
86.4	0.00567058747792595\\
86.41	0.00567828465003522\\
86.42	0.00568598785607264\\
86.43	0.00569369710513812\\
86.44	0.00570141240633944\\
86.45	0.00570913376879209\\
86.46	0.00571686120161907\\
86.47	0.00572459471395072\\
86.48	0.00573233431492449\\
86.49	0.00574008001368478\\
86.5	0.00574783181938275\\
86.51	0.00575558974117608\\
86.52	0.00576335378822878\\
86.53	0.00577112396971103\\
86.54	0.00577890029479886\\
86.55	0.00578668277267406\\
86.56	0.00579447141252385\\
86.57	0.00580226622354074\\
86.58	0.00581006721492224\\
86.59	0.00581787439587067\\
86.6	0.00582568777559292\\
86.61	0.00583350736330016\\
86.62	0.00584133316820768\\
86.63	0.00584916519953458\\
86.64	0.00585700346650352\\
86.65	0.00586484797834051\\
86.66	0.00587269874427459\\
86.67	0.00588055577353761\\
86.68	0.00588841907536393\\
86.69	0.00589628865899017\\
86.7	0.00590416453365492\\
86.71	0.00591204670859844\\
86.72	0.00591993519306241\\
86.73	0.00592782999628959\\
86.74	0.00593573112752357\\
86.75	0.00594363859600842\\
86.76	0.00595155241098842\\
86.77	0.00595947258170774\\
86.78	0.00596739911741009\\
86.79	0.00597533202733844\\
86.8	0.00598327132073466\\
86.81	0.00599121700683921\\
86.82	0.00599588779591692\\
86.83	0.00599914093098286\\
86.84	0.00600239399845674\\
86.85	0.00600564699208879\\
86.86	0.00600889990560195\\
86.87	0.00601215273269177\\
86.88	0.00601540546702631\\
86.89	0.00601865810224604\\
86.9	0.00602191063196372\\
86.91	0.0060251630497643\\
86.92	0.00602841534920478\\
86.93	0.00603166752381417\\
86.94	0.00603491956709329\\
86.95	0.00603817147251475\\
86.96	0.00604142323352275\\
86.97	0.00604467484353305\\
86.98	0.00604792629593282\\
86.99	0.0060511775840805\\
87	0.00605442870130575\\
87.01	0.00605767964090931\\
87.02	0.00606093039616285\\
87.03	0.00606418096030893\\
87.04	0.0060674313265608\\
87.05	0.00607068148810239\\
87.06	0.00607393143808807\\
87.07	0.00607718116964265\\
87.08	0.0060804306758612\\
87.09	0.00608367994980895\\
87.1	0.00608692898452116\\
87.11	0.00609017777300305\\
87.12	0.00609342630822962\\
87.13	0.00609667458314557\\
87.14	0.00609992259066517\\
87.15	0.00610317032367214\\
87.16	0.00610641777501955\\
87.17	0.00610966493752969\\
87.18	0.00611291180399392\\
87.19	0.0061161583671726\\
87.2	0.00611940461979493\\
87.21	0.00612265055455885\\
87.22	0.00612589616413091\\
87.23	0.00612914144114617\\
87.24	0.00613238637820802\\
87.25	0.00613563096788813\\
87.26	0.00613887520272627\\
87.27	0.00614211907523022\\
87.28	0.00614536257787563\\
87.29	0.00614860570310588\\
87.3	0.006151848443332\\
87.31	0.00615509079093252\\
87.32	0.00615833273825332\\
87.33	0.00616157427760754\\
87.34	0.00616481540127544\\
87.35	0.00616805610150425\\
87.36	0.0061712963705081\\
87.37	0.00617453620046782\\
87.38	0.00617777558353087\\
87.39	0.00618101451181118\\
87.4	0.00618425297738902\\
87.41	0.00618749097231089\\
87.42	0.00619072848858936\\
87.43	0.00619396551820299\\
87.44	0.00619720205309612\\
87.45	0.00620043808517882\\
87.46	0.0062036736063267\\
87.47	0.0062069086083808\\
87.48	0.00621014308314747\\
87.49	0.0062133770223982\\
87.5	0.00621661041786951\\
87.51	0.00621984326126282\\
87.52	0.00622307554424432\\
87.53	0.00622630725844478\\
87.54	0.00622953839545949\\
87.55	0.00623276894684809\\
87.56	0.0062359989041344\\
87.57	0.00623922825880635\\
87.58	0.00624245700231578\\
87.59	0.00624568512607835\\
87.6	0.00624891262147336\\
87.61	0.00625213947984364\\
87.62	0.0062553656924954\\
87.63	0.00625859125069808\\
87.64	0.00626181614568424\\
87.65	0.00626504036864938\\
87.66	0.00626826391075183\\
87.67	0.00627148676311257\\
87.68	0.00627470891681513\\
87.69	0.00627793036290544\\
87.7	0.00628115109239165\\
87.71	0.00628437109624403\\
87.72	0.00628759036539479\\
87.73	0.00629080889073797\\
87.74	0.00629402666312926\\
87.75	0.00629724367338589\\
87.76	0.00630045991228645\\
87.77	0.00630367537057075\\
87.78	0.00630689003893969\\
87.79	0.00631010390805511\\
87.8	0.00631331696853962\\
87.81	0.00631652921097647\\
87.82	0.0063197406259094\\
87.83	0.00632295120384248\\
87.84	0.00632616093523997\\
87.85	0.00632936981052617\\
87.86	0.00633257782008525\\
87.87	0.00633578495426113\\
87.88	0.00633899120335729\\
87.89	0.00634219655763667\\
87.9	0.00634540100732145\\
87.91	0.00634860454259295\\
87.92	0.00635180715359147\\
87.93	0.00635500883041609\\
87.94	0.00635820956312457\\
87.95	0.00636140934173317\\
87.96	0.00636460815621651\\
87.97	0.00636780599650737\\
87.98	0.00637100285249659\\
87.99	0.00637419871403289\\
88	0.00637739357092268\\
88.01	0.00638058741292995\\
88.02	0.0063837802297761\\
88.03	0.00638697201113976\\
88.04	0.00639016274665664\\
88.05	0.00639335242591938\\
88.06	0.00639654103847739\\
88.07	0.00639972857383666\\
88.08	0.00640291502145964\\
88.09	0.00640610037076506\\
88.1	0.00640928461112773\\
88.11	0.00641246773187846\\
88.12	0.00641564972230383\\
88.13	0.00641883057164602\\
88.14	0.00642201026910271\\
88.15	0.00642518880382686\\
88.16	0.00642836616492655\\
88.17	0.00643154234146484\\
88.18	0.00643471732245958\\
88.19	0.00643789109688326\\
88.2	0.00644106365366283\\
88.21	0.00644423498167954\\
88.22	0.00644740506976877\\
88.23	0.00645057390671985\\
88.24	0.00645374148127591\\
88.25	0.00645690778213371\\
88.26	0.00646007279794346\\
88.27	0.00646323651730864\\
88.28	0.00646639892878586\\
88.29	0.00646956002088465\\
88.3	0.00647271978206732\\
88.31	0.00647587820074878\\
88.32	0.00647903526529637\\
88.33	0.00648219096402967\\
88.34	0.00648534528522033\\
88.35	0.00648849821709193\\
88.36	0.00649164974781975\\
88.37	0.00649479986553066\\
88.38	0.00649794855830287\\
88.39	0.00650109581416582\\
88.4	0.00650424162109999\\
88.41	0.00650738596703668\\
88.42	0.00651052883985788\\
88.43	0.00651367022739609\\
88.44	0.00651681011743412\\
88.45	0.00651994849770491\\
88.46	0.00652308535589139\\
88.47	0.00652622067962626\\
88.48	0.00652935445649182\\
88.49	0.00653248667401981\\
88.5	0.0065356173196912\\
88.51	0.00653874638093606\\
88.52	0.00654187384513329\\
88.53	0.00654499969961054\\
88.54	0.00654812393164395\\
88.55	0.00655124652845803\\
88.56	0.00655436747722543\\
88.57	0.00655748676506677\\
88.58	0.00656060437905047\\
88.59	0.00656372030619255\\
88.6	0.00656683453345645\\
88.61	0.00656994704775288\\
88.62	0.00657305783593957\\
88.63	0.00657616688482113\\
88.64	0.00657927418114885\\
88.65	0.00658237971162054\\
88.66	0.0065854834628803\\
88.67	0.00658858542151836\\
88.68	0.00659168557407091\\
88.69	0.00659478390701985\\
88.7	0.00659788040679269\\
88.71	0.0066009750597623\\
88.72	0.00660406785224672\\
88.73	0.00660715877050902\\
88.74	0.00661024780075708\\
88.75	0.00661333492914338\\
88.76	0.00661642014176486\\
88.77	0.00661950342466269\\
88.78	0.00662258476382209\\
88.79	0.00662566414517218\\
88.8	0.00662874155458571\\
88.81	0.00663181697787893\\
88.82	0.0066348904008114\\
88.83	0.00663796180908576\\
88.84	0.00664103118834756\\
88.85	0.0066440985241851\\
88.86	0.00664716380212917\\
88.87	0.00665022700765292\\
88.88	0.00665328812617162\\
88.89	0.00665634714304252\\
88.9	0.0066594040435646\\
88.91	0.00666245881297843\\
88.92	0.00666551143646592\\
88.93	0.00666856189915018\\
88.94	0.00667161018609532\\
88.95	0.0066746562823062\\
88.96	0.00667770017272831\\
88.97	0.00668074184224753\\
88.98	0.00668378127568996\\
88.99	0.0066868184578217\\
89	0.00668985337334868\\
89.01	0.00669288600691645\\
89.02	0.00669591634311\\
89.03	0.00669894436645356\\
89.04	0.00670197006141038\\
89.05	0.00670499341238258\\
89.06	0.00670801440371093\\
89.07	0.00671103301967463\\
89.08	0.0067140492444912\\
89.09	0.00671706306231615\\
89.1	0.00672007445724293\\
89.11	0.00672308341330262\\
89.12	0.00672608991446381\\
89.13	0.00672909394463236\\
89.14	0.00673209548765124\\
89.15	0.00673509452730028\\
89.16	0.00673809104729605\\
89.17	0.0067410850312916\\
89.18	0.0067440764628763\\
89.19	0.00674706659121478\\
89.2	0.00675005762367183\\
89.21	0.00675304955949813\\
89.22	0.00675604239794348\\
89.23	0.00675903613825679\\
89.24	0.00676203077968608\\
89.25	0.00676502632147851\\
89.26	0.00676802276288041\\
89.27	0.00677102010313725\\
89.28	0.0067740183414937\\
89.29	0.00677701747719362\\
89.3	0.00678001750948009\\
89.31	0.0067830184375954\\
89.32	0.0067860202607811\\
89.33	0.00678902297827799\\
89.34	0.00679202658932615\\
89.35	0.00679503109316494\\
89.36	0.00679803648903304\\
89.37	0.00680104277616843\\
89.38	0.00680404995380847\\
89.39	0.00680705802118982\\
89.4	0.00681006697754856\\
89.41	0.00681307682212013\\
89.42	0.00681608755413939\\
89.43	0.00681909917284061\\
89.44	0.00682211167745752\\
89.45	0.00682512506722328\\
89.46	0.00682813934137056\\
89.47	0.00683115449913149\\
89.48	0.00683417053973772\\
89.49	0.00683718746242045\\
89.5	0.00684020526641039\\
89.51	0.00684322395093785\\
89.52	0.0068462435152327\\
89.53	0.00684926395852442\\
89.54	0.00685228528004213\\
89.55	0.00685530747901456\\
89.56	0.00685833055467013\\
89.57	0.00686135450623691\\
89.58	0.00686437933294268\\
89.59	0.00686740503401495\\
89.6	0.00687043160868097\\
89.61	0.00687345905616772\\
89.62	0.00687648737570199\\
89.63	0.00687951656651036\\
89.64	0.00688254662781922\\
89.65	0.00688557755885482\\
89.66	0.00688860935884325\\
89.67	0.0068916420270105\\
89.68	0.00689467556258246\\
89.69	0.00689770996478495\\
89.7	0.00690074523284373\\
89.71	0.00690378136598452\\
89.72	0.00690681836343307\\
89.73	0.00690985622441509\\
89.74	0.00691289494815639\\
89.75	0.00691593453388278\\
89.76	0.00691897498082018\\
89.77	0.00692201628819462\\
89.78	0.00692505845523225\\
89.79	0.00692810148115938\\
89.8	0.00693114536520249\\
89.81	0.00693419010658824\\
89.82	0.00693723570454356\\
89.83	0.00694028215829559\\
89.84	0.00694332946707176\\
89.85	0.00694637763009978\\
89.86	0.00694942664660771\\
89.87	0.00695247651582393\\
89.88	0.00695552723697722\\
89.89	0.00695857880929672\\
89.9	0.00696163123201203\\
89.91	0.00696468450435319\\
89.92	0.00696773862555071\\
89.93	0.0069707935948356\\
89.94	0.00697384941143942\\
89.95	0.00697690607459425\\
89.96	0.00697996358353279\\
89.97	0.00698302193748833\\
89.98	0.0069860811356948\\
89.99	0.00698914117738681\\
90	0.00699220206179963\\
90.01	0.0069952637881693\\
90.02	0.00699832635573257\\
90.03	0.007001389763727\\
90.04	0.00700445401139091\\
90.05	0.00700751909796352\\
90.06	0.00701058502268487\\
90.07	0.00701365178479591\\
90.08	0.00701671938353854\\
90.09	0.00701978781815557\\
90.1	0.00702285708789082\\
90.11	0.00702592719198915\\
90.12	0.00702899812969643\\
90.13	0.00703206990025962\\
90.14	0.00703514250292681\\
90.15	0.00703821593694721\\
90.16	0.00704129020157122\\
90.17	0.00704436529605042\\
90.18	0.00704744121963766\\
90.19	0.00705051797158705\\
90.2	0.00705359555115399\\
90.21	0.00705667395759524\\
90.22	0.00705975319016891\\
90.23	0.00706283324813453\\
90.24	0.00706591413075306\\
90.25	0.00706899583728694\\
90.26	0.00707207836700011\\
90.27	0.00707516171915804\\
90.28	0.00707824589302781\\
90.29	0.00708133088787809\\
90.3	0.00708441670297919\\
90.31	0.00708750333760313\\
90.32	0.00709059079102363\\
90.33	0.00709367906251617\\
90.34	0.00709676815135802\\
90.35	0.0070998580568283\\
90.36	0.00710294877820799\\
90.37	0.00710604031477995\\
90.38	0.00710913266582901\\
90.39	0.00711222583064199\\
90.4	0.00711531980850769\\
90.41	0.00711841459871702\\
90.42	0.00712151020056295\\
90.43	0.0071246066133406\\
90.44	0.00712770383634728\\
90.45	0.0071308018688825\\
90.46	0.00713390071024803\\
90.47	0.00713700035974795\\
90.48	0.00714010081668868\\
90.49	0.00714320208037901\\
90.5	0.00714630415013016\\
90.51	0.00714940702525581\\
90.52	0.00715251070507215\\
90.53	0.00715561518889793\\
90.54	0.00715872047605448\\
90.55	0.00716182656586576\\
90.56	0.00716493345765844\\
90.57	0.00716804115076189\\
90.58	0.00717114964450825\\
90.59	0.00717425893823248\\
90.6	0.0071773690312724\\
90.61	0.00718047992296871\\
90.62	0.0071835916126651\\
90.63	0.00718670409970823\\
90.64	0.0071898173834478\\
90.65	0.00719293146323661\\
90.66	0.0071960463384306\\
90.67	0.00719916200838886\\
90.68	0.00720227847247376\\
90.69	0.00720539573005092\\
90.7	0.0072085137804893\\
90.71	0.00721163262316121\\
90.72	0.00721475225744244\\
90.73	0.0072178726827122\\
90.74	0.00722099389835327\\
90.75	0.00722411590375198\\
90.76	0.0072272386982983\\
90.77	0.00723036242357136\\
90.78	0.00723348728669508\\
90.79	0.00723661328655174\\
90.8	0.00723974042201773\\
90.81	0.00724286869196362\\
90.82	0.0072459980952541\\
90.83	0.00724912863074794\\
90.84	0.00725226029729803\\
90.85	0.00725539309375131\\
90.86	0.00725852701894878\\
90.87	0.00726166207172546\\
90.88	0.00726479825091039\\
90.89	0.00726793555532661\\
90.9	0.00727107398379113\\
90.91	0.00727421353511491\\
90.92	0.00727735420810285\\
90.93	0.00728049600155377\\
90.94	0.0072836389142604\\
90.95	0.00728678294500932\\
90.96	0.00728992809258101\\
90.97	0.00729307435574975\\
90.98	0.00729622173328368\\
90.99	0.00729937022394472\\
91	0.00730251982648858\\
91.01	0.00730567053966473\\
91.02	0.00730882236221638\\
91.03	0.00731197529288049\\
91.04	0.00731512933038769\\
91.05	0.00731828447346232\\
91.06	0.00732144072082237\\
91.07	0.00732459807117949\\
91.08	0.00732775652323893\\
91.09	0.00733091607569959\\
91.1	0.00733407672725392\\
91.11	0.00733723847658794\\
91.12	0.00734040132238122\\
91.13	0.00734356526330686\\
91.14	0.00734673029803145\\
91.15	0.0073498964252151\\
91.16	0.00735306364351133\\
91.17	0.00735623195156716\\
91.18	0.00735940134802299\\
91.19	0.00736257183151265\\
91.2	0.00736574340066334\\
91.21	0.00736891605409562\\
91.22	0.0073720897904234\\
91.23	0.0073752646082539\\
91.24	0.00737844050618765\\
91.25	0.00738161748281845\\
91.26	0.00738479553673337\\
91.27	0.00738797466651269\\
91.28	0.00739115487072994\\
91.29	0.00739433614795181\\
91.3	0.00739751849673819\\
91.31	0.0074007019156421\\
91.32	0.00740388640320971\\
91.33	0.00740707195798028\\
91.34	0.00741025857848618\\
91.35	0.00741344626325283\\
91.36	0.0074166350107987\\
91.37	0.00741982481963527\\
91.38	0.00742301568826703\\
91.39	0.00742620761519146\\
91.4	0.00742940059889898\\
91.41	0.00743259463787296\\
91.42	0.00743578973058967\\
91.43	0.00743898587551829\\
91.44	0.00744218307112083\\
91.45	0.0074453813158522\\
91.46	0.00744858060816011\\
91.47	0.00745178094648505\\
91.48	0.00745498232926032\\
91.49	0.00745818475491198\\
91.5	0.0074613882218588\\
91.51	0.00746459272851229\\
91.52	0.00746779827327663\\
91.53	0.00747100485454868\\
91.54	0.00747421247071793\\
91.55	0.00747742112016651\\
91.56	0.00748063080126914\\
91.57	0.00748384151239312\\
91.58	0.00748705325189829\\
91.59	0.00749026601813703\\
91.6	0.00749347980945424\\
91.61	0.00749669462418727\\
91.62	0.00749991046066596\\
91.63	0.00750312731721258\\
91.64	0.0075063451921418\\
91.65	0.0075095640837607\\
91.66	0.00751278399036871\\
91.67	0.00751600491025762\\
91.68	0.00751922684171151\\
91.69	0.00752244978300678\\
91.7	0.0075256737324121\\
91.71	0.00752889868818838\\
91.72	0.00753212464858875\\
91.73	0.00753535161185855\\
91.74	0.00753857957623529\\
91.75	0.00754180853994863\\
91.76	0.00754503850122036\\
91.77	0.00754826945826436\\
91.78	0.00755150140928662\\
91.79	0.00755473435248514\\
91.8	0.00755796828604999\\
91.81	0.00756120320816321\\
91.82	0.00756443911699885\\
91.83	0.00756767601072289\\
91.84	0.00757091388749325\\
91.85	0.00757415274545976\\
91.86	0.00757739258276413\\
91.87	0.00758063339753992\\
91.88	0.00758387518791251\\
91.89	0.00758711795199911\\
91.9	0.00759036168790869\\
91.91	0.00759360639374198\\
91.92	0.00759685206759145\\
91.93	0.00760009870754124\\
91.94	0.0076033463116672\\
91.95	0.00760659487803683\\
91.96	0.00760984440470924\\
91.97	0.00761309488973516\\
91.98	0.00761634633115686\\
91.99	0.00761959872700821\\
92	0.00762285207531455\\
92.01	0.00762610637409276\\
92.02	0.00762936162135116\\
92.03	0.00763261781508954\\
92.04	0.00763587495329908\\
92.05	0.00763913303396238\\
92.06	0.00764239205505338\\
92.07	0.00764565201453737\\
92.08	0.00764891291037095\\
92.09	0.00765217474050201\\
92.1	0.00765543750286969\\
92.11	0.00765870119540437\\
92.12	0.00766196581602763\\
92.13	0.00766523136265223\\
92.14	0.00766849783318207\\
92.15	0.00767176522551218\\
92.16	0.00767503353752869\\
92.17	0.00767830276710881\\
92.18	0.00768157291212075\\
92.19	0.00768484397042378\\
92.2	0.00768811593986813\\
92.21	0.007691388818295\\
92.22	0.00769466260353649\\
92.23	0.00769793729341566\\
92.24	0.00770121288574638\\
92.25	0.0077044893783334\\
92.26	0.00770776676897229\\
92.27	0.00771104505544937\\
92.28	0.00771432423554177\\
92.29	0.0077176043070173\\
92.3	0.00772088526763451\\
92.31	0.00772416711514258\\
92.32	0.00772744984728138\\
92.33	0.00773073346178137\\
92.34	0.00773401795636357\\
92.35	0.0077373033287396\\
92.36	0.00774058957661156\\
92.37	0.00774387669767209\\
92.38	0.00774716468960425\\
92.39	0.00775045355008156\\
92.4	0.00775374327676796\\
92.41	0.00775703386731772\\
92.42	0.00776032531937549\\
92.43	0.00776361763057624\\
92.44	0.00776691079854519\\
92.45	0.00777020482089784\\
92.46	0.00777349969523991\\
92.47	0.00777679541916732\\
92.48	0.00778009199026612\\
92.49	0.00778338940611253\\
92.5	0.00778668766427284\\
92.51	0.00778998676230343\\
92.52	0.0077932866977507\\
92.53	0.00779658746815106\\
92.54	0.00779988907103091\\
92.55	0.00780319150390658\\
92.56	0.00780649476428431\\
92.57	0.00780979884966022\\
92.58	0.00781310375752027\\
92.59	0.00781640948534026\\
92.6	0.00781971603058576\\
92.61	0.00782302339071208\\
92.62	0.00782633156316427\\
92.63	0.00782964054537703\\
92.64	0.00783295033477477\\
92.65	0.00783626092877147\\
92.66	0.00783957232477073\\
92.67	0.00784288452016569\\
92.68	0.007846197512339\\
92.69	0.00784951129866283\\
92.7	0.00785282587649879\\
92.71	0.00785614124319791\\
92.72	0.0078594573961006\\
92.73	0.00786277433253665\\
92.74	0.00786609204982515\\
92.75	0.00786941054527448\\
92.76	0.00787272981618229\\
92.77	0.00787604985983542\\
92.78	0.00787937067350992\\
92.79	0.00788269225447098\\
92.8	0.0078860145999729\\
92.81	0.00788933770725907\\
92.82	0.00789266157356193\\
92.83	0.00789598619610291\\
92.84	0.00789931157209245\\
92.85	0.0079026376987299\\
92.86	0.00790596457320359\\
92.87	0.00790929219269076\\
92.88	0.00791262055435754\\
92.89	0.00791594965535889\\
92.9	0.0079192794928386\\
92.91	0.00792261006392919\\
92.92	0.00792594136575193\\
92.93	0.00792927339541678\\
92.94	0.00793260615002235\\
92.95	0.00793593962665586\\
92.96	0.00793927382239311\\
92.97	0.00794260873429842\\
92.98	0.00794594435942461\\
92.99	0.00794928069481295\\
93	0.00795261773749315\\
93.01	0.00795595548448326\\
93.02	0.00795929393278968\\
93.03	0.00796263307940712\\
93.04	0.00796597292131849\\
93.05	0.00796931345549498\\
93.06	0.00797265467889589\\
93.07	0.00797599658846869\\
93.08	0.00797933918114892\\
93.09	0.00798268245386018\\
93.1	0.00798602640351404\\
93.11	0.00798937102701006\\
93.12	0.00799271632123572\\
93.13	0.00799606228306636\\
93.14	0.00799940890936516\\
93.15	0.0080027561969831\\
93.16	0.00800610414275888\\
93.17	0.00800945274351894\\
93.18	0.00801280199607734\\
93.19	0.00801615189723577\\
93.2	0.0080195024437835\\
93.21	0.00802285363249732\\
93.22	0.0080262054601415\\
93.23	0.00802955792346773\\
93.24	0.00803291101921512\\
93.25	0.00803626474411009\\
93.26	0.00803961909486639\\
93.27	0.008042974068185\\
93.28	0.00804632966075412\\
93.29	0.0080496858692491\\
93.3	0.0080530426903324\\
93.31	0.00805640012065355\\
93.32	0.0080597581568491\\
93.33	0.00806311679554257\\
93.34	0.00806647603334439\\
93.35	0.00806983586685186\\
93.36	0.00807319629264914\\
93.37	0.00807655730730711\\
93.38	0.00807991890738341\\
93.39	0.00808328108942237\\
93.4	0.00808664384995491\\
93.41	0.00809000718549855\\
93.42	0.00809337109255732\\
93.43	0.00809673556762173\\
93.44	0.00810010060716873\\
93.45	0.00810346620766163\\
93.46	0.00810683236555004\\
93.47	0.00811019907726988\\
93.48	0.00811356633924325\\
93.49	0.00811693414787842\\
93.5	0.0081203024995698\\
93.51	0.00812367139069782\\
93.52	0.00812704081762892\\
93.53	0.0081304107767155\\
93.54	0.00813378126429584\\
93.55	0.00813715227669408\\
93.56	0.00814052381022013\\
93.57	0.00814389586116978\\
93.58	0.00814726842582469\\
93.59	0.00815064150045237\\
93.6	0.00815401508130608\\
93.61	0.0081573891646248\\
93.62	0.00816076374663317\\
93.63	0.00816413882354141\\
93.64	0.0081675143915453\\
93.65	0.00817089044682608\\
93.66	0.00817426698555043\\
93.67	0.00817764400387037\\
93.68	0.00818102149792323\\
93.69	0.00818439946383158\\
93.7	0.00818777789770317\\
93.71	0.00819115679563085\\
93.72	0.00819453615369255\\
93.73	0.00819791596795117\\
93.74	0.00820129623445456\\
93.75	0.00820467694923541\\
93.76	0.00820805810831122\\
93.77	0.00821143970768425\\
93.78	0.00821482174334139\\
93.79	0.00821820421125416\\
93.8	0.00822158710737863\\
93.81	0.00822497042765531\\
93.82	0.00822835416800914\\
93.83	0.00823173832434939\\
93.84	0.00823512289256961\\
93.85	0.00823850786854752\\
93.86	0.00824189324814501\\
93.87	0.008245279027208\\
93.88	0.0082486652015664\\
93.89	0.00825205176703407\\
93.9	0.00825543871940869\\
93.91	0.00825882605447172\\
93.92	0.00826221376798831\\
93.93	0.00826560185570726\\
93.94	0.00826899031336091\\
93.95	0.00827237913666507\\
93.96	0.00827576832131897\\
93.97	0.00827915786300515\\
93.98	0.0082825477573894\\
93.99	0.0082859380001207\\
94	0.0082893285868311\\
94.01	0.00829271951313568\\
94.02	0.00829611077463243\\
94.03	0.00829950236690223\\
94.04	0.00830289428550869\\
94.05	0.00830628652599815\\
94.06	0.00830967908389952\\
94.07	0.00831307195472427\\
94.08	0.00831646513396628\\
94.09	0.0083198586171018\\
94.1	0.00832325239958935\\
94.11	0.00832664647686962\\
94.12	0.0083300408443654\\
94.13	0.0083334354974815\\
94.14	0.00833683043160463\\
94.15	0.00834022564210333\\
94.16	0.00834362112432788\\
94.17	0.00834701687361022\\
94.18	0.00835041288526381\\
94.19	0.0083538091545836\\
94.2	0.00835720567684589\\
94.21	0.00836060244730826\\
94.22	0.00836399946120945\\
94.23	0.00836739671376928\\
94.24	0.00837079420018855\\
94.25	0.00837419191564896\\
94.26	0.00837758985531296\\
94.27	0.00838098801432369\\
94.28	0.00838438638780487\\
94.29	0.0083877849708607\\
94.3	0.00839118375857574\\
94.31	0.00839458274601481\\
94.32	0.00839798192822291\\
94.33	0.00840138130022508\\
94.34	0.00840478085702629\\
94.35	0.00840818059361137\\
94.36	0.00841158050494486\\
94.37	0.00841498058597091\\
94.38	0.00841838083161318\\
94.39	0.00842178123677471\\
94.4	0.00842518179633781\\
94.41	0.00842858250516394\\
94.42	0.00843198335809362\\
94.43	0.00843538434994626\\
94.44	0.00843878547552008\\
94.45	0.00844218672959198\\
94.46	0.00844558810691742\\
94.47	0.00844898960223028\\
94.48	0.00845239121024275\\
94.49	0.0084557929256452\\
94.5	0.00845919474310606\\
94.51	0.00846259665727166\\
94.52	0.00846599866276615\\
94.53	0.00846940075419132\\
94.54	0.00847280292612649\\
94.55	0.0084762051731284\\
94.56	0.008479607489731\\
94.57	0.00848300987044539\\
94.58	0.00848641230975964\\
94.59	0.00848981480213866\\
94.6	0.00849321734202405\\
94.61	0.00849661992383397\\
94.62	0.00850002254196298\\
94.63	0.0085034251907819\\
94.64	0.00850682786463766\\
94.65	0.00851023055785316\\
94.66	0.00851363326472893\\
94.67	0.00851703597954653\\
94.68	0.00852043869656852\\
94.69	0.00852384141003841\\
94.7	0.00852724411418066\\
94.71	0.00853064680320056\\
94.72	0.00853404947128423\\
94.73	0.00853745211259857\\
94.74	0.00854085472129118\\
94.75	0.00854425729149035\\
94.76	0.00854765981730497\\
94.77	0.00855106229282451\\
94.78	0.00855446471211898\\
94.79	0.00855786706923885\\
94.8	0.00856126935821502\\
94.81	0.00856467157305875\\
94.82	0.00856807370776165\\
94.83	0.00857147575629561\\
94.84	0.00857487771261272\\
94.85	0.00857827957064528\\
94.86	0.0085816813243057\\
94.87	0.00858508296748647\\
94.88	0.00858848449406012\\
94.89	0.00859188589787916\\
94.9	0.00859528717277601\\
94.91	0.008598688312563\\
94.92	0.00860208931103226\\
94.93	0.00860549016195573\\
94.94	0.00860889085908505\\
94.95	0.00861229139615156\\
94.96	0.00861569176686622\\
94.97	0.00861909196491959\\
94.98	0.00862249198398172\\
94.99	0.00862589181770217\\
95	0.00862929145970991\\
95.01	0.0086326909036133\\
95.02	0.00863609014300003\\
95.03	0.00863948917143704\\
95.04	0.00864288798247051\\
95.05	0.00864628656962581\\
95.06	0.0086496849264074\\
95.07	0.00865308304629884\\
95.08	0.0086564809227627\\
95.09	0.00865987854924052\\
95.1	0.00866327591915276\\
95.11	0.00866667302589873\\
95.12	0.0086700698628566\\
95.13	0.00867346642338326\\
95.14	0.00867686270081433\\
95.15	0.00868025868846409\\
95.16	0.00868365437962544\\
95.17	0.00868704976756984\\
95.18	0.00869044484554723\\
95.19	0.00869383960678603\\
95.2	0.00869723404449306\\
95.21	0.00870062815185348\\
95.22	0.00870402192203077\\
95.23	0.00870741534816664\\
95.24	0.00871080842338101\\
95.25	0.00871420114077193\\
95.26	0.00871759349341554\\
95.27	0.00872098547436605\\
95.28	0.00872437707665561\\
95.29	0.00872776829329435\\
95.3	0.00873115911727026\\
95.31	0.00873454954154916\\
95.32	0.00873793955907465\\
95.33	0.00874132916276806\\
95.34	0.0087447183455284\\
95.35	0.00874810710023229\\
95.36	0.00875149541973392\\
95.37	0.00875488329686498\\
95.38	0.00875827072443466\\
95.39	0.00876165769522953\\
95.4	0.00876504420201353\\
95.41	0.00876843023752789\\
95.42	0.00877181579449112\\
95.43	0.0087752008655989\\
95.44	0.00877858544352408\\
95.45	0.00878196952091657\\
95.46	0.00878535309040335\\
95.47	0.00878873614458838\\
95.48	0.00879211867605256\\
95.49	0.00879550067735364\\
95.5	0.00879888214102623\\
95.51	0.00880226305958171\\
95.52	0.00880564342550816\\
95.53	0.00880902323127035\\
95.54	0.00881240246930966\\
95.55	0.00881578113204402\\
95.56	0.00881915921186788\\
95.57	0.00882253670115215\\
95.58	0.00882591359224411\\
95.59	0.00882928987746741\\
95.6	0.00883266554912201\\
95.61	0.00883604059948408\\
95.62	0.00883941502080599\\
95.63	0.00884278880531625\\
95.64	0.00884616194521943\\
95.65	0.00884953443269615\\
95.66	0.00885290625990299\\
95.67	0.00885627741897245\\
95.68	0.00885964790201289\\
95.69	0.00886301770110849\\
95.7	0.00886638680831918\\
95.71	0.00886975521568061\\
95.72	0.00887312291520406\\
95.73	0.00887648989887641\\
95.74	0.0088798561586601\\
95.75	0.00888322168649304\\
95.76	0.00888658647428859\\
95.77	0.00888995051393546\\
95.78	0.00889331379729774\\
95.79	0.00889667631621474\\
95.8	0.00890003806250103\\
95.81	0.00890339902794632\\
95.82	0.00890675920431545\\
95.83	0.00891011858334829\\
95.84	0.00891347715675975\\
95.85	0.00891683491623968\\
95.86	0.0089201918534528\\
95.87	0.0089235479600387\\
95.88	0.00892690322761175\\
95.89	0.00893025764776106\\
95.9	0.00893361121205041\\
95.91	0.00893696391201822\\
95.92	0.00894031573917746\\
95.93	0.00894366668501566\\
95.94	0.00894701674099477\\
95.95	0.00895036589855118\\
95.96	0.00895371414909562\\
95.97	0.00895706148401315\\
95.98	0.00896040789466304\\
95.99	0.0089637533723788\\
96	0.00896709790846804\\
96.01	0.00897044149421249\\
96.02	0.00897378412086791\\
96.03	0.00897712577966402\\
96.04	0.00898046646180448\\
96.05	0.00898380615846684\\
96.06	0.00898714486080244\\
96.07	0.0089904825599364\\
96.08	0.00899381924696756\\
96.09	0.00899715491296841\\
96.1	0.00900048954898505\\
96.11	0.00900382314603711\\
96.12	0.00900715569511776\\
96.13	0.00901048718719357\\
96.14	0.00901381761320453\\
96.15	0.00901714696406396\\
96.16	0.00902047523065846\\
96.17	0.00902380240384787\\
96.18	0.00902712847446519\\
96.19	0.00903045343331657\\
96.2	0.00903377727118121\\
96.21	0.00903709997881133\\
96.22	0.00904042154693213\\
96.23	0.0090437419662417\\
96.24	0.00904706122741101\\
96.25	0.00905037932108382\\
96.26	0.00905369623787665\\
96.27	0.00905701196837871\\
96.28	0.00906032650315189\\
96.29	0.00906363983273062\\
96.3	0.00906695194762191\\
96.31	0.00907026283830525\\
96.32	0.00907357249523258\\
96.33	0.00907688090882819\\
96.34	0.00908018806948873\\
96.35	0.00908349396758312\\
96.36	0.0090867985934525\\
96.37	0.0090901019374102\\
96.38	0.00909340398974166\\
96.39	0.0090967047407044\\
96.4	0.00910000418052793\\
96.41	0.00910330229941377\\
96.42	0.00910659908753532\\
96.43	0.00910989453503787\\
96.44	0.00911318863203848\\
96.45	0.00911648136862602\\
96.46	0.00911977273486103\\
96.47	0.00912306272077572\\
96.48	0.00912635131637391\\
96.49	0.00912963851163097\\
96.5	0.00913292429649376\\
96.51	0.00913620866088061\\
96.52	0.00913949159468125\\
96.53	0.00914277308775675\\
96.54	0.00914605312993947\\
96.55	0.00914933171103306\\
96.56	0.00915260882081232\\
96.57	0.00915588444902323\\
96.58	0.00915915858538284\\
96.59	0.00916243121957929\\
96.6	0.00916570234127167\\
96.61	0.00916897194009006\\
96.62	0.00917224000563539\\
96.63	0.0091755065274795\\
96.64	0.00917877149516496\\
96.65	0.00918203489820516\\
96.66	0.00918529672608412\\
96.67	0.00918855696825658\\
96.68	0.00919181561414782\\
96.69	0.00919507265315372\\
96.7	0.00919832807464064\\
96.71	0.00920158186794539\\
96.72	0.00920483402237523\\
96.73	0.00920808452720773\\
96.74	0.00921133337169079\\
96.75	0.00921458054504259\\
96.76	0.0092178260364515\\
96.77	0.00922106983507607\\
96.78	0.00922431193004498\\
96.79	0.00922755231045697\\
96.8	0.00923079096538081\\
96.81	0.00923402788385525\\
96.82	0.00923726305488898\\
96.83	0.00924049646746057\\
96.84	0.00924372811051844\\
96.85	0.00924695797298078\\
96.86	0.00925018604373554\\
96.87	0.0092534123116404\\
96.88	0.00925663676552265\\
96.89	0.00925985939417923\\
96.9	0.00926308018637662\\
96.91	0.00926629913085083\\
96.92	0.00926951621630735\\
96.93	0.00927273143142111\\
96.94	0.0092759447648364\\
96.95	0.00927915620516689\\
96.96	0.00928236574099551\\
96.97	0.00928557336087448\\
96.98	0.00928877905332521\\
96.99	0.00929198280683828\\
97	0.00929518460987343\\
97.01	0.00929838445085943\\
97.02	0.00930158231819412\\
97.03	0.00930477820024435\\
97.04	0.00930797208534589\\
97.05	0.00931116396180347\\
97.06	0.00931435381789066\\
97.07	0.00931754164184987\\
97.08	0.0093207274218923\\
97.09	0.0093239111461979\\
97.1	0.00932709280291535\\
97.11	0.00933027238016196\\
97.12	0.0093334498660237\\
97.13	0.00933662524855512\\
97.14	0.00933979851577933\\
97.15	0.00934296965568792\\
97.16	0.009346138656241\\
97.17	0.00934930550536707\\
97.18	0.00935247019096306\\
97.19	0.00935563270089424\\
97.2	0.00935879302299419\\
97.21	0.0093619511450648\\
97.22	0.00936510705487618\\
97.23	0.00936826074016668\\
97.24	0.00937141218864279\\
97.25	0.00937456138797916\\
97.26	0.00937770832581853\\
97.27	0.00938085298977171\\
97.28	0.00938399536741754\\
97.29	0.00938713544630286\\
97.3	0.00939027321394247\\
97.31	0.0093934086578191\\
97.32	0.00939654176538336\\
97.33	0.00939967252405375\\
97.34	0.00940280092121657\\
97.35	0.00940592694422593\\
97.36	0.00940905058040371\\
97.37	0.00941217181703951\\
97.38	0.00941529064139062\\
97.39	0.00941840704068204\\
97.4	0.00942152100210638\\
97.41	0.00942463251282384\\
97.42	0.00942774155996225\\
97.43	0.00943084813061695\\
97.44	0.0094339522118508\\
97.45	0.00943705379069419\\
97.46	0.00944015285414494\\
97.47	0.00944324938916831\\
97.48	0.00944634338269697\\
97.49	0.009449434821631\\
97.5	0.0094525236928378\\
97.51	0.00945560998315212\\
97.52	0.009458693679376\\
97.53	0.00946177476827877\\
97.54	0.00946485323659702\\
97.55	0.00946792907103456\\
97.56	0.00947100225826241\\
97.57	0.00947407278491878\\
97.58	0.00947714063760904\\
97.59	0.00948020580290569\\
97.6	0.00948326826734836\\
97.61	0.00948632801744377\\
97.62	0.00948938503966571\\
97.63	0.00949243932045505\\
97.64	0.00949549084621966\\
97.65	0.00949853960333446\\
97.66	0.00950158557814135\\
97.67	0.00950462875694921\\
97.68	0.00950766912603389\\
97.69	0.0095107066716382\\
97.7	0.00951374137997184\\
97.71	0.00951677323721146\\
97.72	0.0095198022295006\\
97.73	0.00952282834294968\\
97.74	0.00952585156363598\\
97.75	0.00952887187760365\\
97.76	0.00953188927086369\\
97.77	0.00953490372939391\\
97.78	0.00953791523913895\\
97.79	0.00954092378601027\\
97.8	0.00954392935588611\\
97.81	0.00954693193461148\\
97.82	0.00954993150799817\\
97.83	0.00955292806182471\\
97.84	0.00955592158183638\\
97.85	0.00955891205374521\\
97.86	0.00956189946322995\\
97.87	0.00956488379593608\\
97.88	0.00956786503747582\\
97.89	0.0095708431734281\\
97.9	0.00957381818933817\\
97.91	0.00957679007071763\\
97.92	0.00957975880304432\\
97.93	0.00958272437176239\\
97.94	0.00958568676228229\\
97.95	0.0095886459599807\\
97.96	0.0095916019502006\\
97.97	0.00959455471825123\\
97.98	0.00959750424940809\\
97.99	0.00960045052891294\\
98	0.00960339354197378\\
98.01	0.0096063332737912\\
98.02	0.00960926970957048\\
98.03	0.00961220283448236\\
98.04	0.00961513263366306\\
98.05	0.00961805909221424\\
98.06	0.00962098219520301\\
98.07	0.00962390192766177\\
98.08	0.00962681827577519\\
98.09	0.00962973122604842\\
98.1	0.00963264076497008\\
98.11	0.00963554687901242\\
98.12	0.00963844955463153\\
98.13	0.00964134877826755\\
98.14	0.00964424453634482\\
98.15	0.00964713681527215\\
98.16	0.00965002560144298\\
98.17	0.0096529108812356\\
98.18	0.00965579264101337\\
98.19	0.00965867086721747\\
98.2	0.00966154554629619\\
98.21	0.00966441666477487\\
98.22	0.009667284209581\\
98.23	0.00967014816762863\\
98.24	0.0096730085258186\\
98.25	0.00967586527103869\\
98.26	0.00967871839016394\\
98.27	0.00968156787005677\\
98.28	0.0096844136975673\\
98.29	0.00968725585953354\\
98.3	0.00969009413741489\\
98.31	0.00969292847736932\\
98.32	0.00969575886699815\\
98.33	0.00969858481230294\\
98.34	0.00970140116599026\\
98.35	0.00970420786165537\\
98.36	0.00970700483236056\\
98.37	0.00970979162671079\\
98.38	0.00971256816126521\\
98.39	0.00971533436570355\\
98.4	0.00971809016913108\\
98.41	0.00972083550007231\\
98.42	0.00972357028646462\\
98.43	0.00972629306455234\\
98.44	0.00972900332028077\\
98.45	0.00973170097090251\\
98.46	0.00973438593299963\\
98.47	0.00973705812247618\\
98.48	0.0097397174545505\\
98.49	0.00974236384374756\\
98.5	0.00974499720389108\\
98.51	0.00974761744809552\\
98.52	0.00975022448875802\\
98.53	0.00975281823755019\\
98.54	0.00975539860540969\\
98.55	0.00975796550253184\\
98.56	0.00976051883836092\\
98.57	0.00976305852158147\\
98.58	0.00976558446010939\\
98.59	0.0097680965610829\\
98.6	0.00977059473085338\\
98.61	0.00977307887497604\\
98.62	0.00977554889820042\\
98.63	0.00977800470446083\\
98.64	0.00978044619686648\\
98.65	0.00978287327769162\\
98.66	0.00978528584836517\\
98.67	0.00978768380946042\\
98.68	0.00979006706068459\\
98.69	0.00979243572973505\\
98.7	0.00979478978845448\\
98.71	0.00979712913527064\\
98.72	0.00979945366771696\\
98.73	0.00980176328319591\\
98.74	0.00980405868274031\\
98.75	0.00980634379333697\\
98.76	0.00980861855607513\\
98.77	0.00981088291163054\\
98.78	0.00981313680026141\\
98.79	0.00981538129038395\\
98.8	0.00981761665092274\\
98.81	0.00981984282880382\\
98.82	0.00982205977059208\\
98.83	0.00982426742275659\\
98.84	0.00982646573159911\\
98.85	0.00982865464305535\\
98.86	0.00983083410269208\\
98.87	0.00983300405570406\\
98.88	0.00983516444691103\\
98.89	0.00983731522075465\\
98.9	0.00983945632129542\\
98.91	0.00984158769220961\\
98.92	0.0098437092767861\\
98.93	0.00984582101792329\\
98.94	0.00984792285812586\\
98.95	0.00985001473950168\\
98.96	0.00985209660375852\\
98.97	0.00985416839220086\\
98.98	0.00985623004572663\\
98.99	0.00985828150482393\\
99	0.00986032270956777\\
99.01	0.0098623535996167\\
99.02	0.00986437411420948\\
99.03	0.00986638419216177\\
99.04	0.0098683837718627\\
99.05	0.00987037279127147\\
99.06	0.00987235118791395\\
99.07	0.00987431889887921\\
99.08	0.00987627586053859\\
99.09	0.00987822200859097\\
99.1	0.00988015727829076\\
99.11	0.00988208160444435\\
99.12	0.00988399492112197\\
99.13	0.00988589716186539\\
99.14	0.00988778825975373\\
99.15	0.00988966814739988\\
99.16	0.00989153675694688\\
99.17	0.00989339402006423\\
99.18	0.00989523986794429\\
99.19	0.00989707423129856\\
99.2	0.00989889704035396\\
99.21	0.00990070822484916\\
99.22	0.00990250771403083\\
99.23	0.00990429543664984\\
99.24	0.00990607132095756\\
99.25	0.00990783529470205\\
99.26	0.00990958728512422\\
99.27	0.00991132721895409\\
99.28	0.0099130550224069\\
99.29	0.00991477062117929\\
99.3	0.00991647394044544\\
99.31	0.00991816490485323\\
99.32	0.00991984343852032\\
99.33	0.00992150946503027\\
99.34	0.00992316290742868\\
99.35	0.00992480368821925\\
99.36	0.00992643172935984\\
99.37	0.00992804695225863\\
99.38	0.00992964927777009\\
99.39	0.00993123862619115\\
99.4	0.00993281491725719\\
99.41	0.00993437807013812\\
99.42	0.00993592800343448\\
99.43	0.00993746463517346\\
99.44	0.00993898788280498\\
99.45	0.00994049766319776\\
99.46	0.00994199389263541\\
99.47	0.00994347648681246\\
99.48	0.00994494536083052\\
99.49	0.00994640042919431\\
99.5	0.00994784160580784\\
99.51	0.00994926880397047\\
99.52	0.00995068193637311\\
99.53	0.00995208091509435\\
99.54	0.00995346565159667\\
99.55	0.00995483605672261\\
99.56	0.00995619204069105\\
99.57	0.00995753351309345\\
99.58	0.00995886038289013\\
99.59	0.00996017255840666\\
99.6	0.00996146994733017\\
99.61	0.00996275245671045\\
99.62	0.00996401999295883\\
99.63	0.00996527246184481\\
99.64	0.00996650976849268\\
99.65	0.00996773181737827\\
99.66	0.0099689385123257\\
99.67	0.00997012975650425\\
99.68	0.00997130545242531\\
99.69	0.00997246550193929\\
99.7	0.00997360980623282\\
99.71	0.009974738248558\\
99.72	0.00997585070992424\\
99.73	0.00997694707046292\\
99.74	0.00997802720942307\\
99.75	0.00997909100516722\\
99.76	0.00998013833516726\\
99.77	0.00998116907600054\\
99.78	0.00998218310334597\\
99.79	0.00998318029198038\\
99.8	0.00998416051577496\\
99.81	0.00998512364769184\\
99.82	0.00998606955978089\\
99.83	0.00998699812317667\\
99.84	0.00998790920809558\\
99.85	0.00998880268383318\\
99.86	0.0099896784187618\\
99.87	0.00999053628032826\\
99.88	0.00999137613505194\\
99.89	0.00999219784852308\\
99.9	0.00999300128540131\\
99.91	0.0099937863094145\\
99.92	0.00999455278335794\\
99.93	0.00999530056909381\\
99.94	0.009996029527551\\
99.95	0.00999673951872531\\
99.96	0.00999743040168006\\
99.97	0.00999810203454702\\
99.98	0.00999875427452794\\
99.99	0.00999938697789635\\
100	0.01\\
};
\addlegendentry{$q=1$};

\addplot [color=red,solid,forget plot]
  table[row sep=crcr]{%
0.01	0\\
0.02	0\\
0.03	0\\
0.04	0\\
0.05	0\\
0.06	0\\
0.07	0\\
0.08	0\\
0.09	0\\
0.1	0\\
0.11	0\\
0.12	0\\
0.13	0\\
0.14	0\\
0.15	0\\
0.16	0\\
0.17	0\\
0.18	0\\
0.19	0\\
0.2	0\\
0.21	0\\
0.22	0\\
0.23	0\\
0.24	0\\
0.25	0\\
0.26	0\\
0.27	0\\
0.28	0\\
0.29	0\\
0.3	0\\
0.31	0\\
0.32	0\\
0.33	0\\
0.34	0\\
0.35	0\\
0.36	0\\
0.37	0\\
0.38	0\\
0.39	0\\
0.4	0\\
0.41	0\\
0.42	0\\
0.43	0\\
0.44	0\\
0.45	0\\
0.46	0\\
0.47	0\\
0.48	0\\
0.49	0\\
0.5	0\\
0.51	0\\
0.52	0\\
0.53	0\\
0.54	0\\
0.55	0\\
0.56	0\\
0.57	0\\
0.58	0\\
0.59	0\\
0.6	0\\
0.61	0\\
0.62	0\\
0.63	0\\
0.64	0\\
0.65	0\\
0.66	0\\
0.67	0\\
0.68	0\\
0.69	0\\
0.7	0\\
0.71	0\\
0.72	0\\
0.73	0\\
0.74	0\\
0.75	0\\
0.76	0\\
0.77	0\\
0.78	0\\
0.79	0\\
0.8	0\\
0.81	0\\
0.82	0\\
0.83	0\\
0.84	0\\
0.85	0\\
0.86	0\\
0.87	0\\
0.88	0\\
0.89	0\\
0.9	0\\
0.91	0\\
0.92	0\\
0.93	0\\
0.94	0\\
0.95	0\\
0.96	0\\
0.97	0\\
0.98	0\\
0.99	0\\
1	0\\
1.01	0\\
1.02	0\\
1.03	0\\
1.04	0\\
1.05	0\\
1.06	0\\
1.07	0\\
1.08	0\\
1.09	0\\
1.1	0\\
1.11	0\\
1.12	0\\
1.13	0\\
1.14	0\\
1.15	0\\
1.16	0\\
1.17	0\\
1.18	0\\
1.19	0\\
1.2	0\\
1.21	0\\
1.22	0\\
1.23	0\\
1.24	0\\
1.25	0\\
1.26	0\\
1.27	0\\
1.28	0\\
1.29	0\\
1.3	0\\
1.31	0\\
1.32	0\\
1.33	0\\
1.34	0\\
1.35	0\\
1.36	0\\
1.37	0\\
1.38	0\\
1.39	0\\
1.4	0\\
1.41	0\\
1.42	0\\
1.43	0\\
1.44	0\\
1.45	0\\
1.46	0\\
1.47	0\\
1.48	0\\
1.49	0\\
1.5	0\\
1.51	0\\
1.52	0\\
1.53	0\\
1.54	0\\
1.55	0\\
1.56	0\\
1.57	0\\
1.58	0\\
1.59	0\\
1.6	0\\
1.61	0\\
1.62	0\\
1.63	0\\
1.64	0\\
1.65	0\\
1.66	0\\
1.67	0\\
1.68	0\\
1.69	0\\
1.7	0\\
1.71	0\\
1.72	0\\
1.73	0\\
1.74	0\\
1.75	0\\
1.76	0\\
1.77	0\\
1.78	0\\
1.79	0\\
1.8	0\\
1.81	0\\
1.82	0\\
1.83	0\\
1.84	0\\
1.85	0\\
1.86	0\\
1.87	0\\
1.88	0\\
1.89	0\\
1.9	0\\
1.91	0\\
1.92	0\\
1.93	0\\
1.94	0\\
1.95	0\\
1.96	0\\
1.97	0\\
1.98	0\\
1.99	0\\
2	0\\
2.01	0\\
2.02	0\\
2.03	0\\
2.04	0\\
2.05	0\\
2.06	0\\
2.07	0\\
2.08	0\\
2.09	0\\
2.1	0\\
2.11	0\\
2.12	0\\
2.13	0\\
2.14	0\\
2.15	0\\
2.16	0\\
2.17	0\\
2.18	0\\
2.19	0\\
2.2	0\\
2.21	0\\
2.22	0\\
2.23	0\\
2.24	0\\
2.25	0\\
2.26	0\\
2.27	0\\
2.28	0\\
2.29	0\\
2.3	0\\
2.31	0\\
2.32	0\\
2.33	0\\
2.34	0\\
2.35	0\\
2.36	0\\
2.37	0\\
2.38	0\\
2.39	0\\
2.4	0\\
2.41	0\\
2.42	0\\
2.43	0\\
2.44	0\\
2.45	0\\
2.46	0\\
2.47	0\\
2.48	0\\
2.49	0\\
2.5	0\\
2.51	0\\
2.52	0\\
2.53	0\\
2.54	0\\
2.55	0\\
2.56	0\\
2.57	0\\
2.58	0\\
2.59	0\\
2.6	0\\
2.61	0\\
2.62	0\\
2.63	0\\
2.64	0\\
2.65	0\\
2.66	0\\
2.67	0\\
2.68	0\\
2.69	0\\
2.7	0\\
2.71	0\\
2.72	0\\
2.73	0\\
2.74	0\\
2.75	0\\
2.76	0\\
2.77	0\\
2.78	0\\
2.79	0\\
2.8	0\\
2.81	0\\
2.82	0\\
2.83	0\\
2.84	0\\
2.85	0\\
2.86	0\\
2.87	0\\
2.88	0\\
2.89	0\\
2.9	0\\
2.91	0\\
2.92	0\\
2.93	0\\
2.94	0\\
2.95	0\\
2.96	0\\
2.97	0\\
2.98	0\\
2.99	0\\
3	0\\
3.01	0\\
3.02	0\\
3.03	0\\
3.04	0\\
3.05	0\\
3.06	0\\
3.07	0\\
3.08	0\\
3.09	0\\
3.1	0\\
3.11	0\\
3.12	0\\
3.13	0\\
3.14	0\\
3.15	0\\
3.16	0\\
3.17	0\\
3.18	0\\
3.19	0\\
3.2	0\\
3.21	0\\
3.22	0\\
3.23	0\\
3.24	0\\
3.25	0\\
3.26	0\\
3.27	0\\
3.28	0\\
3.29	0\\
3.3	0\\
3.31	0\\
3.32	0\\
3.33	0\\
3.34	0\\
3.35	0\\
3.36	0\\
3.37	0\\
3.38	0\\
3.39	0\\
3.4	0\\
3.41	0\\
3.42	0\\
3.43	0\\
3.44	0\\
3.45	0\\
3.46	0\\
3.47	0\\
3.48	0\\
3.49	0\\
3.5	0\\
3.51	0\\
3.52	0\\
3.53	0\\
3.54	0\\
3.55	0\\
3.56	0\\
3.57	0\\
3.58	0\\
3.59	0\\
3.6	0\\
3.61	0\\
3.62	0\\
3.63	0\\
3.64	0\\
3.65	0\\
3.66	0\\
3.67	0\\
3.68	0\\
3.69	0\\
3.7	0\\
3.71	0\\
3.72	0\\
3.73	0\\
3.74	0\\
3.75	0\\
3.76	0\\
3.77	0\\
3.78	0\\
3.79	0\\
3.8	0\\
3.81	0\\
3.82	0\\
3.83	0\\
3.84	0\\
3.85	0\\
3.86	0\\
3.87	0\\
3.88	0\\
3.89	0\\
3.9	0\\
3.91	0\\
3.92	0\\
3.93	0\\
3.94	0\\
3.95	0\\
3.96	0\\
3.97	0\\
3.98	0\\
3.99	0\\
4	0\\
4.01	0\\
4.02	0\\
4.03	0\\
4.04	0\\
4.05	0\\
4.06	0\\
4.07	0\\
4.08	0\\
4.09	0\\
4.1	0\\
4.11	0\\
4.12	0\\
4.13	0\\
4.14	0\\
4.15	0\\
4.16	0\\
4.17	0\\
4.18	0\\
4.19	0\\
4.2	0\\
4.21	0\\
4.22	0\\
4.23	0\\
4.24	0\\
4.25	0\\
4.26	0\\
4.27	0\\
4.28	0\\
4.29	0\\
4.3	0\\
4.31	0\\
4.32	0\\
4.33	0\\
4.34	0\\
4.35	0\\
4.36	0\\
4.37	0\\
4.38	0\\
4.39	0\\
4.4	0\\
4.41	0\\
4.42	0\\
4.43	0\\
4.44	0\\
4.45	0\\
4.46	0\\
4.47	0\\
4.48	0\\
4.49	0\\
4.5	0\\
4.51	0\\
4.52	0\\
4.53	0\\
4.54	0\\
4.55	0\\
4.56	0\\
4.57	0\\
4.58	0\\
4.59	0\\
4.6	0\\
4.61	0\\
4.62	0\\
4.63	0\\
4.64	0\\
4.65	0\\
4.66	0\\
4.67	0\\
4.68	0\\
4.69	0\\
4.7	0\\
4.71	0\\
4.72	0\\
4.73	0\\
4.74	0\\
4.75	0\\
4.76	0\\
4.77	0\\
4.78	0\\
4.79	0\\
4.8	0\\
4.81	0\\
4.82	0\\
4.83	0\\
4.84	0\\
4.85	0\\
4.86	0\\
4.87	0\\
4.88	0\\
4.89	0\\
4.9	0\\
4.91	0\\
4.92	0\\
4.93	0\\
4.94	0\\
4.95	0\\
4.96	0\\
4.97	0\\
4.98	0\\
4.99	0\\
5	0\\
5.01	0\\
5.02	0\\
5.03	0\\
5.04	0\\
5.05	0\\
5.06	0\\
5.07	0\\
5.08	0\\
5.09	0\\
5.1	0\\
5.11	0\\
5.12	0\\
5.13	0\\
5.14	0\\
5.15	0\\
5.16	0\\
5.17	0\\
5.18	0\\
5.19	0\\
5.2	0\\
5.21	0\\
5.22	0\\
5.23	0\\
5.24	0\\
5.25	0\\
5.26	0\\
5.27	0\\
5.28	0\\
5.29	0\\
5.3	0\\
5.31	0\\
5.32	0\\
5.33	0\\
5.34	0\\
5.35	0\\
5.36	0\\
5.37	0\\
5.38	0\\
5.39	0\\
5.4	0\\
5.41	0\\
5.42	0\\
5.43	0\\
5.44	0\\
5.45	0\\
5.46	0\\
5.47	0\\
5.48	0\\
5.49	0\\
5.5	0\\
5.51	0\\
5.52	0\\
5.53	0\\
5.54	0\\
5.55	0\\
5.56	0\\
5.57	0\\
5.58	0\\
5.59	0\\
5.6	0\\
5.61	0\\
5.62	0\\
5.63	0\\
5.64	0\\
5.65	0\\
5.66	0\\
5.67	0\\
5.68	0\\
5.69	0\\
5.7	0\\
5.71	0\\
5.72	0\\
5.73	0\\
5.74	0\\
5.75	0\\
5.76	0\\
5.77	0\\
5.78	0\\
5.79	0\\
5.8	0\\
5.81	0\\
5.82	0\\
5.83	0\\
5.84	0\\
5.85	0\\
5.86	0\\
5.87	0\\
5.88	0\\
5.89	0\\
5.9	0\\
5.91	0\\
5.92	0\\
5.93	0\\
5.94	0\\
5.95	0\\
5.96	0\\
5.97	0\\
5.98	0\\
5.99	0\\
6	0\\
6.01	0\\
6.02	0\\
6.03	0\\
6.04	0\\
6.05	0\\
6.06	0\\
6.07	0\\
6.08	0\\
6.09	0\\
6.1	0\\
6.11	0\\
6.12	0\\
6.13	0\\
6.14	0\\
6.15	0\\
6.16	0\\
6.17	0\\
6.18	0\\
6.19	0\\
6.2	0\\
6.21	0\\
6.22	0\\
6.23	0\\
6.24	0\\
6.25	0\\
6.26	0\\
6.27	0\\
6.28	0\\
6.29	0\\
6.3	0\\
6.31	0\\
6.32	0\\
6.33	0\\
6.34	0\\
6.35	0\\
6.36	0\\
6.37	0\\
6.38	0\\
6.39	0\\
6.4	0\\
6.41	0\\
6.42	0\\
6.43	0\\
6.44	0\\
6.45	0\\
6.46	0\\
6.47	0\\
6.48	0\\
6.49	0\\
6.5	0\\
6.51	0\\
6.52	0\\
6.53	0\\
6.54	0\\
6.55	0\\
6.56	0\\
6.57	0\\
6.58	0\\
6.59	0\\
6.6	0\\
6.61	0\\
6.62	0\\
6.63	0\\
6.64	0\\
6.65	0\\
6.66	0\\
6.67	0\\
6.68	0\\
6.69	0\\
6.7	0\\
6.71	0\\
6.72	0\\
6.73	0\\
6.74	0\\
6.75	0\\
6.76	0\\
6.77	0\\
6.78	0\\
6.79	0\\
6.8	0\\
6.81	0\\
6.82	0\\
6.83	0\\
6.84	0\\
6.85	0\\
6.86	0\\
6.87	0\\
6.88	0\\
6.89	0\\
6.9	0\\
6.91	0\\
6.92	0\\
6.93	0\\
6.94	0\\
6.95	0\\
6.96	0\\
6.97	0\\
6.98	0\\
6.99	0\\
7	0\\
7.01	0\\
7.02	0\\
7.03	0\\
7.04	0\\
7.05	0\\
7.06	0\\
7.07	0\\
7.08	0\\
7.09	0\\
7.1	0\\
7.11	0\\
7.12	0\\
7.13	0\\
7.14	0\\
7.15	0\\
7.16	0\\
7.17	0\\
7.18	0\\
7.19	0\\
7.2	0\\
7.21	0\\
7.22	0\\
7.23	0\\
7.24	0\\
7.25	0\\
7.26	0\\
7.27	0\\
7.28	0\\
7.29	0\\
7.3	0\\
7.31	0\\
7.32	0\\
7.33	0\\
7.34	0\\
7.35	0\\
7.36	0\\
7.37	0\\
7.38	0\\
7.39	0\\
7.4	0\\
7.41	0\\
7.42	0\\
7.43	0\\
7.44	0\\
7.45	0\\
7.46	0\\
7.47	0\\
7.48	0\\
7.49	0\\
7.5	0\\
7.51	0\\
7.52	0\\
7.53	0\\
7.54	0\\
7.55	0\\
7.56	0\\
7.57	0\\
7.58	0\\
7.59	0\\
7.6	0\\
7.61	0\\
7.62	0\\
7.63	0\\
7.64	0\\
7.65	0\\
7.66	0\\
7.67	0\\
7.68	0\\
7.69	0\\
7.7	0\\
7.71	0\\
7.72	0\\
7.73	0\\
7.74	0\\
7.75	0\\
7.76	0\\
7.77	0\\
7.78	0\\
7.79	0\\
7.8	0\\
7.81	0\\
7.82	0\\
7.83	0\\
7.84	0\\
7.85	0\\
7.86	0\\
7.87	0\\
7.88	0\\
7.89	0\\
7.9	0\\
7.91	0\\
7.92	0\\
7.93	0\\
7.94	0\\
7.95	0\\
7.96	0\\
7.97	0\\
7.98	0\\
7.99	0\\
8	0\\
8.01	0\\
8.02	0\\
8.03	0\\
8.04	0\\
8.05	0\\
8.06	0\\
8.07	0\\
8.08	0\\
8.09	0\\
8.1	0\\
8.11	0\\
8.12	0\\
8.13	0\\
8.14	0\\
8.15	0\\
8.16	0\\
8.17	0\\
8.18	0\\
8.19	0\\
8.2	0\\
8.21	0\\
8.22	0\\
8.23	0\\
8.24	0\\
8.25	0\\
8.26	0\\
8.27	0\\
8.28	0\\
8.29	0\\
8.3	0\\
8.31	0\\
8.32	0\\
8.33	0\\
8.34	0\\
8.35	0\\
8.36	0\\
8.37	0\\
8.38	0\\
8.39	0\\
8.4	0\\
8.41	0\\
8.42	0\\
8.43	0\\
8.44	0\\
8.45	0\\
8.46	0\\
8.47	0\\
8.48	0\\
8.49	0\\
8.5	0\\
8.51	0\\
8.52	0\\
8.53	0\\
8.54	0\\
8.55	0\\
8.56	0\\
8.57	0\\
8.58	0\\
8.59	0\\
8.6	0\\
8.61	0\\
8.62	0\\
8.63	0\\
8.64	0\\
8.65	0\\
8.66	0\\
8.67	0\\
8.68	0\\
8.69	0\\
8.7	0\\
8.71	0\\
8.72	0\\
8.73	0\\
8.74	0\\
8.75	0\\
8.76	0\\
8.77	0\\
8.78	0\\
8.79	0\\
8.8	0\\
8.81	0\\
8.82	0\\
8.83	0\\
8.84	0\\
8.85	0\\
8.86	0\\
8.87	0\\
8.88	0\\
8.89	0\\
8.9	0\\
8.91	0\\
8.92	0\\
8.93	0\\
8.94	0\\
8.95	0\\
8.96	0\\
8.97	0\\
8.98	0\\
8.99	0\\
9	0\\
9.01	0\\
9.02	0\\
9.03	0\\
9.04	0\\
9.05	0\\
9.06	0\\
9.07	0\\
9.08	0\\
9.09	0\\
9.1	0\\
9.11	0\\
9.12	0\\
9.13	0\\
9.14	0\\
9.15	0\\
9.16	0\\
9.17	0\\
9.18	0\\
9.19	0\\
9.2	0\\
9.21	0\\
9.22	0\\
9.23	0\\
9.24	0\\
9.25	0\\
9.26	0\\
9.27	0\\
9.28	0\\
9.29	0\\
9.3	0\\
9.31	0\\
9.32	0\\
9.33	0\\
9.34	0\\
9.35	0\\
9.36	0\\
9.37	0\\
9.38	0\\
9.39	0\\
9.4	0\\
9.41	0\\
9.42	0\\
9.43	0\\
9.44	0\\
9.45	0\\
9.46	0\\
9.47	0\\
9.48	0\\
9.49	0\\
9.5	0\\
9.51	0\\
9.52	0\\
9.53	0\\
9.54	0\\
9.55	0\\
9.56	0\\
9.57	0\\
9.58	0\\
9.59	0\\
9.6	0\\
9.61	0\\
9.62	0\\
9.63	0\\
9.64	0\\
9.65	0\\
9.66	0\\
9.67	0\\
9.68	0\\
9.69	0\\
9.7	0\\
9.71	0\\
9.72	0\\
9.73	0\\
9.74	0\\
9.75	0\\
9.76	0\\
9.77	0\\
9.78	0\\
9.79	0\\
9.8	0\\
9.81	0\\
9.82	0\\
9.83	0\\
9.84	0\\
9.85	0\\
9.86	0\\
9.87	0\\
9.88	0\\
9.89	0\\
9.9	0\\
9.91	0\\
9.92	0\\
9.93	0\\
9.94	0\\
9.95	0\\
9.96	0\\
9.97	0\\
9.98	0\\
9.99	0\\
10	0\\
10.01	0\\
10.02	0\\
10.03	0\\
10.04	0\\
10.05	0\\
10.06	0\\
10.07	0\\
10.08	0\\
10.09	0\\
10.1	0\\
10.11	0\\
10.12	0\\
10.13	0\\
10.14	0\\
10.15	0\\
10.16	0\\
10.17	0\\
10.18	0\\
10.19	0\\
10.2	0\\
10.21	0\\
10.22	0\\
10.23	0\\
10.24	0\\
10.25	0\\
10.26	0\\
10.27	0\\
10.28	0\\
10.29	0\\
10.3	0\\
10.31	0\\
10.32	0\\
10.33	0\\
10.34	0\\
10.35	0\\
10.36	0\\
10.37	0\\
10.38	0\\
10.39	0\\
10.4	0\\
10.41	0\\
10.42	0\\
10.43	0\\
10.44	0\\
10.45	0\\
10.46	0\\
10.47	0\\
10.48	0\\
10.49	0\\
10.5	0\\
10.51	0\\
10.52	0\\
10.53	0\\
10.54	0\\
10.55	0\\
10.56	0\\
10.57	0\\
10.58	0\\
10.59	0\\
10.6	0\\
10.61	0\\
10.62	0\\
10.63	0\\
10.64	0\\
10.65	0\\
10.66	0\\
10.67	0\\
10.68	0\\
10.69	0\\
10.7	0\\
10.71	0\\
10.72	0\\
10.73	0\\
10.74	0\\
10.75	0\\
10.76	0\\
10.77	0\\
10.78	0\\
10.79	0\\
10.8	0\\
10.81	0\\
10.82	0\\
10.83	0\\
10.84	0\\
10.85	0\\
10.86	0\\
10.87	0\\
10.88	0\\
10.89	0\\
10.9	0\\
10.91	0\\
10.92	0\\
10.93	0\\
10.94	0\\
10.95	0\\
10.96	0\\
10.97	0\\
10.98	0\\
10.99	0\\
11	0\\
11.01	0\\
11.02	0\\
11.03	0\\
11.04	0\\
11.05	0\\
11.06	0\\
11.07	0\\
11.08	0\\
11.09	0\\
11.1	0\\
11.11	0\\
11.12	0\\
11.13	0\\
11.14	0\\
11.15	0\\
11.16	0\\
11.17	0\\
11.18	0\\
11.19	0\\
11.2	0\\
11.21	0\\
11.22	0\\
11.23	0\\
11.24	0\\
11.25	0\\
11.26	0\\
11.27	0\\
11.28	0\\
11.29	0\\
11.3	0\\
11.31	0\\
11.32	0\\
11.33	0\\
11.34	0\\
11.35	0\\
11.36	0\\
11.37	0\\
11.38	0\\
11.39	0\\
11.4	0\\
11.41	0\\
11.42	0\\
11.43	0\\
11.44	0\\
11.45	0\\
11.46	0\\
11.47	0\\
11.48	0\\
11.49	0\\
11.5	0\\
11.51	0\\
11.52	0\\
11.53	0\\
11.54	0\\
11.55	0\\
11.56	0\\
11.57	0\\
11.58	0\\
11.59	0\\
11.6	0\\
11.61	0\\
11.62	0\\
11.63	0\\
11.64	0\\
11.65	0\\
11.66	0\\
11.67	0\\
11.68	0\\
11.69	0\\
11.7	0\\
11.71	0\\
11.72	0\\
11.73	0\\
11.74	0\\
11.75	0\\
11.76	0\\
11.77	0\\
11.78	0\\
11.79	0\\
11.8	0\\
11.81	0\\
11.82	0\\
11.83	0\\
11.84	0\\
11.85	0\\
11.86	0\\
11.87	0\\
11.88	0\\
11.89	0\\
11.9	0\\
11.91	0\\
11.92	0\\
11.93	0\\
11.94	0\\
11.95	0\\
11.96	0\\
11.97	0\\
11.98	0\\
11.99	0\\
12	0\\
12.01	0\\
12.02	0\\
12.03	0\\
12.04	0\\
12.05	0\\
12.06	0\\
12.07	0\\
12.08	0\\
12.09	0\\
12.1	0\\
12.11	0\\
12.12	0\\
12.13	0\\
12.14	0\\
12.15	0\\
12.16	0\\
12.17	0\\
12.18	0\\
12.19	0\\
12.2	0\\
12.21	0\\
12.22	0\\
12.23	0\\
12.24	0\\
12.25	0\\
12.26	0\\
12.27	0\\
12.28	0\\
12.29	0\\
12.3	0\\
12.31	0\\
12.32	0\\
12.33	0\\
12.34	0\\
12.35	0\\
12.36	0\\
12.37	0\\
12.38	0\\
12.39	0\\
12.4	0\\
12.41	0\\
12.42	0\\
12.43	0\\
12.44	0\\
12.45	0\\
12.46	0\\
12.47	0\\
12.48	0\\
12.49	0\\
12.5	0\\
12.51	0\\
12.52	0\\
12.53	0\\
12.54	0\\
12.55	0\\
12.56	0\\
12.57	0\\
12.58	0\\
12.59	0\\
12.6	0\\
12.61	0\\
12.62	0\\
12.63	0\\
12.64	0\\
12.65	0\\
12.66	0\\
12.67	0\\
12.68	0\\
12.69	0\\
12.7	0\\
12.71	0\\
12.72	0\\
12.73	0\\
12.74	0\\
12.75	0\\
12.76	0\\
12.77	0\\
12.78	0\\
12.79	0\\
12.8	0\\
12.81	0\\
12.82	0\\
12.83	0\\
12.84	0\\
12.85	0\\
12.86	0\\
12.87	0\\
12.88	0\\
12.89	0\\
12.9	0\\
12.91	0\\
12.92	0\\
12.93	0\\
12.94	0\\
12.95	0\\
12.96	0\\
12.97	0\\
12.98	0\\
12.99	0\\
13	0\\
13.01	0\\
13.02	0\\
13.03	0\\
13.04	0\\
13.05	0\\
13.06	0\\
13.07	0\\
13.08	0\\
13.09	0\\
13.1	0\\
13.11	0\\
13.12	0\\
13.13	0\\
13.14	0\\
13.15	0\\
13.16	0\\
13.17	0\\
13.18	0\\
13.19	0\\
13.2	0\\
13.21	0\\
13.22	0\\
13.23	0\\
13.24	0\\
13.25	0\\
13.26	0\\
13.27	0\\
13.28	0\\
13.29	0\\
13.3	0\\
13.31	0\\
13.32	0\\
13.33	0\\
13.34	0\\
13.35	0\\
13.36	0\\
13.37	0\\
13.38	0\\
13.39	0\\
13.4	0\\
13.41	0\\
13.42	0\\
13.43	0\\
13.44	0\\
13.45	0\\
13.46	0\\
13.47	0\\
13.48	0\\
13.49	0\\
13.5	0\\
13.51	0\\
13.52	0\\
13.53	0\\
13.54	0\\
13.55	0\\
13.56	0\\
13.57	0\\
13.58	0\\
13.59	0\\
13.6	0\\
13.61	0\\
13.62	0\\
13.63	0\\
13.64	0\\
13.65	0\\
13.66	0\\
13.67	0\\
13.68	0\\
13.69	0\\
13.7	0\\
13.71	0\\
13.72	0\\
13.73	0\\
13.74	0\\
13.75	0\\
13.76	0\\
13.77	0\\
13.78	0\\
13.79	0\\
13.8	0\\
13.81	0\\
13.82	0\\
13.83	0\\
13.84	0\\
13.85	0\\
13.86	0\\
13.87	0\\
13.88	0\\
13.89	0\\
13.9	0\\
13.91	0\\
13.92	0\\
13.93	0\\
13.94	0\\
13.95	0\\
13.96	0\\
13.97	0\\
13.98	0\\
13.99	0\\
14	0\\
14.01	0\\
14.02	0\\
14.03	0\\
14.04	0\\
14.05	0\\
14.06	0\\
14.07	0\\
14.08	0\\
14.09	0\\
14.1	0\\
14.11	0\\
14.12	0\\
14.13	0\\
14.14	0\\
14.15	0\\
14.16	0\\
14.17	0\\
14.18	0\\
14.19	0\\
14.2	0\\
14.21	0\\
14.22	0\\
14.23	0\\
14.24	0\\
14.25	0\\
14.26	0\\
14.27	0\\
14.28	0\\
14.29	0\\
14.3	0\\
14.31	0\\
14.32	0\\
14.33	0\\
14.34	0\\
14.35	0\\
14.36	0\\
14.37	0\\
14.38	0\\
14.39	0\\
14.4	0\\
14.41	0\\
14.42	0\\
14.43	0\\
14.44	0\\
14.45	0\\
14.46	0\\
14.47	0\\
14.48	0\\
14.49	0\\
14.5	0\\
14.51	0\\
14.52	0\\
14.53	0\\
14.54	0\\
14.55	0\\
14.56	0\\
14.57	0\\
14.58	0\\
14.59	0\\
14.6	0\\
14.61	0\\
14.62	0\\
14.63	0\\
14.64	0\\
14.65	0\\
14.66	0\\
14.67	0\\
14.68	0\\
14.69	0\\
14.7	0\\
14.71	0\\
14.72	0\\
14.73	0\\
14.74	0\\
14.75	0\\
14.76	0\\
14.77	0\\
14.78	0\\
14.79	0\\
14.8	0\\
14.81	0\\
14.82	0\\
14.83	0\\
14.84	0\\
14.85	0\\
14.86	0\\
14.87	0\\
14.88	0\\
14.89	0\\
14.9	0\\
14.91	0\\
14.92	0\\
14.93	0\\
14.94	0\\
14.95	0\\
14.96	0\\
14.97	0\\
14.98	0\\
14.99	0\\
15	0\\
15.01	0\\
15.02	0\\
15.03	0\\
15.04	0\\
15.05	0\\
15.06	0\\
15.07	0\\
15.08	0\\
15.09	0\\
15.1	0\\
15.11	0\\
15.12	0\\
15.13	0\\
15.14	0\\
15.15	0\\
15.16	0\\
15.17	0\\
15.18	0\\
15.19	0\\
15.2	0\\
15.21	0\\
15.22	0\\
15.23	0\\
15.24	0\\
15.25	0\\
15.26	0\\
15.27	0\\
15.28	0\\
15.29	0\\
15.3	0\\
15.31	0\\
15.32	0\\
15.33	0\\
15.34	0\\
15.35	0\\
15.36	0\\
15.37	0\\
15.38	0\\
15.39	0\\
15.4	0\\
15.41	0\\
15.42	0\\
15.43	0\\
15.44	0\\
15.45	0\\
15.46	0\\
15.47	0\\
15.48	0\\
15.49	0\\
15.5	0\\
15.51	0\\
15.52	0\\
15.53	0\\
15.54	0\\
15.55	0\\
15.56	0\\
15.57	0\\
15.58	0\\
15.59	0\\
15.6	0\\
15.61	0\\
15.62	0\\
15.63	0\\
15.64	0\\
15.65	0\\
15.66	0\\
15.67	0\\
15.68	0\\
15.69	0\\
15.7	0\\
15.71	0\\
15.72	0\\
15.73	0\\
15.74	0\\
15.75	0\\
15.76	0\\
15.77	0\\
15.78	0\\
15.79	0\\
15.8	0\\
15.81	0\\
15.82	0\\
15.83	0\\
15.84	0\\
15.85	0\\
15.86	0\\
15.87	0\\
15.88	0\\
15.89	0\\
15.9	0\\
15.91	0\\
15.92	0\\
15.93	0\\
15.94	0\\
15.95	0\\
15.96	0\\
15.97	0\\
15.98	0\\
15.99	0\\
16	0\\
16.01	0\\
16.02	0\\
16.03	0\\
16.04	0\\
16.05	0\\
16.06	0\\
16.07	0\\
16.08	0\\
16.09	0\\
16.1	0\\
16.11	0\\
16.12	0\\
16.13	0\\
16.14	0\\
16.15	0\\
16.16	0\\
16.17	0\\
16.18	0\\
16.19	0\\
16.2	0\\
16.21	0\\
16.22	0\\
16.23	0\\
16.24	0\\
16.25	0\\
16.26	0\\
16.27	0\\
16.28	0\\
16.29	0\\
16.3	0\\
16.31	0\\
16.32	0\\
16.33	0\\
16.34	0\\
16.35	0\\
16.36	0\\
16.37	0\\
16.38	0\\
16.39	0\\
16.4	0\\
16.41	0\\
16.42	0\\
16.43	0\\
16.44	0\\
16.45	0\\
16.46	0\\
16.47	0\\
16.48	0\\
16.49	0\\
16.5	0\\
16.51	0\\
16.52	0\\
16.53	0\\
16.54	0\\
16.55	0\\
16.56	0\\
16.57	0\\
16.58	0\\
16.59	0\\
16.6	0\\
16.61	0\\
16.62	0\\
16.63	0\\
16.64	0\\
16.65	0\\
16.66	0\\
16.67	0\\
16.68	0\\
16.69	0\\
16.7	0\\
16.71	0\\
16.72	0\\
16.73	0\\
16.74	0\\
16.75	0\\
16.76	0\\
16.77	0\\
16.78	0\\
16.79	0\\
16.8	0\\
16.81	0\\
16.82	0\\
16.83	0\\
16.84	0\\
16.85	0\\
16.86	0\\
16.87	0\\
16.88	0\\
16.89	0\\
16.9	0\\
16.91	0\\
16.92	0\\
16.93	0\\
16.94	0\\
16.95	0\\
16.96	0\\
16.97	0\\
16.98	0\\
16.99	0\\
17	0\\
17.01	0\\
17.02	0\\
17.03	0\\
17.04	0\\
17.05	0\\
17.06	0\\
17.07	0\\
17.08	0\\
17.09	0\\
17.1	0\\
17.11	0\\
17.12	0\\
17.13	0\\
17.14	0\\
17.15	0\\
17.16	0\\
17.17	0\\
17.18	0\\
17.19	0\\
17.2	0\\
17.21	0\\
17.22	0\\
17.23	0\\
17.24	0\\
17.25	0\\
17.26	0\\
17.27	0\\
17.28	0\\
17.29	0\\
17.3	0\\
17.31	0\\
17.32	0\\
17.33	0\\
17.34	0\\
17.35	0\\
17.36	0\\
17.37	0\\
17.38	0\\
17.39	0\\
17.4	0\\
17.41	0\\
17.42	0\\
17.43	0\\
17.44	0\\
17.45	0\\
17.46	0\\
17.47	0\\
17.48	0\\
17.49	0\\
17.5	0\\
17.51	0\\
17.52	0\\
17.53	0\\
17.54	0\\
17.55	0\\
17.56	0\\
17.57	0\\
17.58	0\\
17.59	0\\
17.6	0\\
17.61	0\\
17.62	0\\
17.63	0\\
17.64	0\\
17.65	0\\
17.66	0\\
17.67	0\\
17.68	0\\
17.69	0\\
17.7	0\\
17.71	0\\
17.72	0\\
17.73	0\\
17.74	0\\
17.75	0\\
17.76	0\\
17.77	0\\
17.78	0\\
17.79	0\\
17.8	0\\
17.81	0\\
17.82	0\\
17.83	0\\
17.84	0\\
17.85	0\\
17.86	0\\
17.87	0\\
17.88	0\\
17.89	0\\
17.9	0\\
17.91	0\\
17.92	0\\
17.93	0\\
17.94	0\\
17.95	0\\
17.96	0\\
17.97	0\\
17.98	0\\
17.99	0\\
18	0\\
18.01	0\\
18.02	0\\
18.03	0\\
18.04	0\\
18.05	0\\
18.06	0\\
18.07	0\\
18.08	0\\
18.09	0\\
18.1	0\\
18.11	0\\
18.12	0\\
18.13	0\\
18.14	0\\
18.15	0\\
18.16	0\\
18.17	0\\
18.18	0\\
18.19	0\\
18.2	0\\
18.21	0\\
18.22	0\\
18.23	0\\
18.24	0\\
18.25	0\\
18.26	0\\
18.27	0\\
18.28	0\\
18.29	0\\
18.3	0\\
18.31	0\\
18.32	0\\
18.33	0\\
18.34	0\\
18.35	0\\
18.36	0\\
18.37	0\\
18.38	0\\
18.39	0\\
18.4	0\\
18.41	0\\
18.42	0\\
18.43	0\\
18.44	0\\
18.45	0\\
18.46	0\\
18.47	0\\
18.48	0\\
18.49	0\\
18.5	0\\
18.51	0\\
18.52	0\\
18.53	0\\
18.54	0\\
18.55	0\\
18.56	0\\
18.57	0\\
18.58	0\\
18.59	0\\
18.6	0\\
18.61	0\\
18.62	0\\
18.63	0\\
18.64	0\\
18.65	0\\
18.66	0\\
18.67	0\\
18.68	0\\
18.69	0\\
18.7	0\\
18.71	0\\
18.72	0\\
18.73	0\\
18.74	0\\
18.75	0\\
18.76	0\\
18.77	0\\
18.78	0\\
18.79	0\\
18.8	0\\
18.81	0\\
18.82	0\\
18.83	0\\
18.84	0\\
18.85	0\\
18.86	0\\
18.87	0\\
18.88	0\\
18.89	0\\
18.9	0\\
18.91	0\\
18.92	0\\
18.93	0\\
18.94	0\\
18.95	0\\
18.96	0\\
18.97	0\\
18.98	0\\
18.99	0\\
19	0\\
19.01	0\\
19.02	0\\
19.03	0\\
19.04	0\\
19.05	0\\
19.06	0\\
19.07	0\\
19.08	0\\
19.09	0\\
19.1	0\\
19.11	0\\
19.12	0\\
19.13	0\\
19.14	0\\
19.15	0\\
19.16	0\\
19.17	0\\
19.18	0\\
19.19	0\\
19.2	0\\
19.21	0\\
19.22	0\\
19.23	0\\
19.24	0\\
19.25	0\\
19.26	0\\
19.27	0\\
19.28	0\\
19.29	0\\
19.3	0\\
19.31	0\\
19.32	0\\
19.33	0\\
19.34	0\\
19.35	0\\
19.36	0\\
19.37	0\\
19.38	0\\
19.39	0\\
19.4	0\\
19.41	0\\
19.42	0\\
19.43	0\\
19.44	0\\
19.45	0\\
19.46	0\\
19.47	0\\
19.48	0\\
19.49	0\\
19.5	0\\
19.51	0\\
19.52	0\\
19.53	0\\
19.54	0\\
19.55	0\\
19.56	0\\
19.57	0\\
19.58	0\\
19.59	0\\
19.6	0\\
19.61	0\\
19.62	0\\
19.63	0\\
19.64	0\\
19.65	0\\
19.66	0\\
19.67	0\\
19.68	0\\
19.69	0\\
19.7	0\\
19.71	0\\
19.72	0\\
19.73	0\\
19.74	0\\
19.75	0\\
19.76	0\\
19.77	0\\
19.78	0\\
19.79	0\\
19.8	0\\
19.81	0\\
19.82	0\\
19.83	0\\
19.84	0\\
19.85	0\\
19.86	0\\
19.87	0\\
19.88	0\\
19.89	0\\
19.9	0\\
19.91	0\\
19.92	0\\
19.93	0\\
19.94	0\\
19.95	0\\
19.96	0\\
19.97	0\\
19.98	0\\
19.99	0\\
20	0\\
20.01	0\\
20.02	0\\
20.03	0\\
20.04	0\\
20.05	0\\
20.06	0\\
20.07	0\\
20.08	0\\
20.09	0\\
20.1	0\\
20.11	0\\
20.12	0\\
20.13	0\\
20.14	0\\
20.15	0\\
20.16	0\\
20.17	0\\
20.18	0\\
20.19	0\\
20.2	0\\
20.21	0\\
20.22	0\\
20.23	0\\
20.24	0\\
20.25	0\\
20.26	0\\
20.27	0\\
20.28	0\\
20.29	0\\
20.3	0\\
20.31	0\\
20.32	0\\
20.33	0\\
20.34	0\\
20.35	0\\
20.36	0\\
20.37	0\\
20.38	0\\
20.39	0\\
20.4	0\\
20.41	0\\
20.42	0\\
20.43	0\\
20.44	0\\
20.45	0\\
20.46	0\\
20.47	0\\
20.48	0\\
20.49	0\\
20.5	0\\
20.51	0\\
20.52	0\\
20.53	0\\
20.54	0\\
20.55	0\\
20.56	0\\
20.57	0\\
20.58	0\\
20.59	0\\
20.6	0\\
20.61	0\\
20.62	0\\
20.63	0\\
20.64	0\\
20.65	0\\
20.66	0\\
20.67	0\\
20.68	0\\
20.69	0\\
20.7	0\\
20.71	0\\
20.72	0\\
20.73	0\\
20.74	0\\
20.75	0\\
20.76	0\\
20.77	0\\
20.78	0\\
20.79	0\\
20.8	0\\
20.81	0\\
20.82	0\\
20.83	0\\
20.84	0\\
20.85	0\\
20.86	0\\
20.87	0\\
20.88	0\\
20.89	0\\
20.9	0\\
20.91	0\\
20.92	0\\
20.93	0\\
20.94	0\\
20.95	0\\
20.96	0\\
20.97	0\\
20.98	0\\
20.99	0\\
21	0\\
21.01	0\\
21.02	0\\
21.03	0\\
21.04	0\\
21.05	0\\
21.06	0\\
21.07	0\\
21.08	0\\
21.09	0\\
21.1	0\\
21.11	0\\
21.12	0\\
21.13	0\\
21.14	0\\
21.15	0\\
21.16	0\\
21.17	0\\
21.18	0\\
21.19	0\\
21.2	0\\
21.21	0\\
21.22	0\\
21.23	0\\
21.24	0\\
21.25	0\\
21.26	0\\
21.27	0\\
21.28	0\\
21.29	0\\
21.3	0\\
21.31	0\\
21.32	0\\
21.33	0\\
21.34	0\\
21.35	0\\
21.36	0\\
21.37	0\\
21.38	0\\
21.39	0\\
21.4	0\\
21.41	0\\
21.42	0\\
21.43	0\\
21.44	0\\
21.45	0\\
21.46	0\\
21.47	0\\
21.48	0\\
21.49	0\\
21.5	0\\
21.51	0\\
21.52	0\\
21.53	0\\
21.54	0\\
21.55	0\\
21.56	0\\
21.57	0\\
21.58	0\\
21.59	0\\
21.6	0\\
21.61	0\\
21.62	0\\
21.63	0\\
21.64	0\\
21.65	0\\
21.66	0\\
21.67	0\\
21.68	0\\
21.69	0\\
21.7	0\\
21.71	0\\
21.72	0\\
21.73	0\\
21.74	0\\
21.75	0\\
21.76	0\\
21.77	0\\
21.78	0\\
21.79	0\\
21.8	0\\
21.81	0\\
21.82	0\\
21.83	0\\
21.84	0\\
21.85	0\\
21.86	0\\
21.87	0\\
21.88	0\\
21.89	0\\
21.9	0\\
21.91	0\\
21.92	0\\
21.93	0\\
21.94	0\\
21.95	0\\
21.96	0\\
21.97	0\\
21.98	0\\
21.99	0\\
22	0\\
22.01	0\\
22.02	0\\
22.03	0\\
22.04	0\\
22.05	0\\
22.06	0\\
22.07	0\\
22.08	0\\
22.09	0\\
22.1	0\\
22.11	0\\
22.12	0\\
22.13	0\\
22.14	0\\
22.15	0\\
22.16	0\\
22.17	0\\
22.18	0\\
22.19	0\\
22.2	0\\
22.21	0\\
22.22	0\\
22.23	0\\
22.24	0\\
22.25	0\\
22.26	0\\
22.27	0\\
22.28	0\\
22.29	0\\
22.3	0\\
22.31	0\\
22.32	0\\
22.33	0\\
22.34	0\\
22.35	0\\
22.36	0\\
22.37	0\\
22.38	0\\
22.39	0\\
22.4	0\\
22.41	0\\
22.42	0\\
22.43	0\\
22.44	0\\
22.45	0\\
22.46	0\\
22.47	0\\
22.48	0\\
22.49	0\\
22.5	0\\
22.51	0\\
22.52	0\\
22.53	0\\
22.54	0\\
22.55	0\\
22.56	0\\
22.57	0\\
22.58	0\\
22.59	0\\
22.6	0\\
22.61	0\\
22.62	0\\
22.63	0\\
22.64	0\\
22.65	0\\
22.66	0\\
22.67	0\\
22.68	0\\
22.69	0\\
22.7	0\\
22.71	0\\
22.72	0\\
22.73	0\\
22.74	0\\
22.75	0\\
22.76	0\\
22.77	0\\
22.78	0\\
22.79	0\\
22.8	0\\
22.81	0\\
22.82	0\\
22.83	0\\
22.84	0\\
22.85	0\\
22.86	0\\
22.87	0\\
22.88	0\\
22.89	0\\
22.9	0\\
22.91	0\\
22.92	0\\
22.93	0\\
22.94	0\\
22.95	0\\
22.96	0\\
22.97	0\\
22.98	0\\
22.99	0\\
23	0\\
23.01	0\\
23.02	0\\
23.03	0\\
23.04	0\\
23.05	0\\
23.06	0\\
23.07	0\\
23.08	0\\
23.09	0\\
23.1	0\\
23.11	0\\
23.12	0\\
23.13	0\\
23.14	0\\
23.15	0\\
23.16	0\\
23.17	0\\
23.18	0\\
23.19	0\\
23.2	0\\
23.21	0\\
23.22	0\\
23.23	0\\
23.24	0\\
23.25	0\\
23.26	0\\
23.27	0\\
23.28	0\\
23.29	0\\
23.3	0\\
23.31	0\\
23.32	0\\
23.33	0\\
23.34	0\\
23.35	0\\
23.36	0\\
23.37	0\\
23.38	0\\
23.39	0\\
23.4	0\\
23.41	0\\
23.42	0\\
23.43	0\\
23.44	0\\
23.45	0\\
23.46	0\\
23.47	0\\
23.48	0\\
23.49	0\\
23.5	0\\
23.51	0\\
23.52	0\\
23.53	0\\
23.54	0\\
23.55	0\\
23.56	0\\
23.57	0\\
23.58	0\\
23.59	0\\
23.6	0\\
23.61	0\\
23.62	0\\
23.63	0\\
23.64	0\\
23.65	0\\
23.66	0\\
23.67	0\\
23.68	0\\
23.69	0\\
23.7	0\\
23.71	0\\
23.72	0\\
23.73	0\\
23.74	0\\
23.75	0\\
23.76	0\\
23.77	0\\
23.78	0\\
23.79	0\\
23.8	0\\
23.81	0\\
23.82	0\\
23.83	0\\
23.84	0\\
23.85	0\\
23.86	0\\
23.87	0\\
23.88	0\\
23.89	0\\
23.9	0\\
23.91	0\\
23.92	0\\
23.93	0\\
23.94	0\\
23.95	0\\
23.96	0\\
23.97	0\\
23.98	0\\
23.99	0\\
24	0\\
24.01	0\\
24.02	0\\
24.03	0\\
24.04	0\\
24.05	0\\
24.06	0\\
24.07	0\\
24.08	0\\
24.09	0\\
24.1	0\\
24.11	0\\
24.12	0\\
24.13	0\\
24.14	0\\
24.15	0\\
24.16	0\\
24.17	0\\
24.18	0\\
24.19	0\\
24.2	0\\
24.21	0\\
24.22	0\\
24.23	0\\
24.24	0\\
24.25	0\\
24.26	0\\
24.27	0\\
24.28	0\\
24.29	0\\
24.3	0\\
24.31	0\\
24.32	0\\
24.33	0\\
24.34	0\\
24.35	0\\
24.36	0\\
24.37	0\\
24.38	0\\
24.39	0\\
24.4	0\\
24.41	0\\
24.42	0\\
24.43	0\\
24.44	0\\
24.45	0\\
24.46	0\\
24.47	0\\
24.48	0\\
24.49	0\\
24.5	0\\
24.51	0\\
24.52	0\\
24.53	0\\
24.54	0\\
24.55	0\\
24.56	0\\
24.57	0\\
24.58	0\\
24.59	0\\
24.6	0\\
24.61	0\\
24.62	0\\
24.63	0\\
24.64	0\\
24.65	0\\
24.66	0\\
24.67	0\\
24.68	0\\
24.69	0\\
24.7	0\\
24.71	0\\
24.72	0\\
24.73	0\\
24.74	0\\
24.75	0\\
24.76	0\\
24.77	0\\
24.78	0\\
24.79	0\\
24.8	0\\
24.81	0\\
24.82	0\\
24.83	0\\
24.84	0\\
24.85	0\\
24.86	0\\
24.87	0\\
24.88	0\\
24.89	0\\
24.9	0\\
24.91	0\\
24.92	0\\
24.93	0\\
24.94	0\\
24.95	0\\
24.96	0\\
24.97	0\\
24.98	0\\
24.99	0\\
25	0\\
25.01	0\\
25.02	0\\
25.03	0\\
25.04	0\\
25.05	0\\
25.06	0\\
25.07	0\\
25.08	0\\
25.09	0\\
25.1	0\\
25.11	0\\
25.12	0\\
25.13	0\\
25.14	0\\
25.15	0\\
25.16	0\\
25.17	0\\
25.18	0\\
25.19	0\\
25.2	0\\
25.21	0\\
25.22	0\\
25.23	0\\
25.24	0\\
25.25	0\\
25.26	0\\
25.27	0\\
25.28	0\\
25.29	0\\
25.3	0\\
25.31	0\\
25.32	0\\
25.33	0\\
25.34	0\\
25.35	0\\
25.36	0\\
25.37	0\\
25.38	0\\
25.39	0\\
25.4	0\\
25.41	0\\
25.42	0\\
25.43	0\\
25.44	0\\
25.45	0\\
25.46	0\\
25.47	0\\
25.48	0\\
25.49	0\\
25.5	0\\
25.51	0\\
25.52	0\\
25.53	0\\
25.54	0\\
25.55	0\\
25.56	0\\
25.57	0\\
25.58	0\\
25.59	0\\
25.6	0\\
25.61	0\\
25.62	0\\
25.63	0\\
25.64	0\\
25.65	0\\
25.66	0\\
25.67	0\\
25.68	0\\
25.69	0\\
25.7	0\\
25.71	0\\
25.72	0\\
25.73	0\\
25.74	0\\
25.75	0\\
25.76	0\\
25.77	0\\
25.78	0\\
25.79	0\\
25.8	0\\
25.81	0\\
25.82	0\\
25.83	0\\
25.84	0\\
25.85	0\\
25.86	0\\
25.87	0\\
25.88	0\\
25.89	0\\
25.9	0\\
25.91	0\\
25.92	0\\
25.93	0\\
25.94	0\\
25.95	0\\
25.96	0\\
25.97	0\\
25.98	0\\
25.99	0\\
26	0\\
26.01	0\\
26.02	0\\
26.03	0\\
26.04	0\\
26.05	0\\
26.06	0\\
26.07	0\\
26.08	0\\
26.09	0\\
26.1	0\\
26.11	0\\
26.12	0\\
26.13	0\\
26.14	0\\
26.15	0\\
26.16	0\\
26.17	0\\
26.18	0\\
26.19	0\\
26.2	0\\
26.21	0\\
26.22	0\\
26.23	0\\
26.24	0\\
26.25	0\\
26.26	0\\
26.27	0\\
26.28	0\\
26.29	0\\
26.3	0\\
26.31	0\\
26.32	0\\
26.33	0\\
26.34	0\\
26.35	0\\
26.36	0\\
26.37	0\\
26.38	0\\
26.39	0\\
26.4	0\\
26.41	0\\
26.42	0\\
26.43	0\\
26.44	0\\
26.45	0\\
26.46	0\\
26.47	0\\
26.48	0\\
26.49	0\\
26.5	0\\
26.51	0\\
26.52	0\\
26.53	0\\
26.54	0\\
26.55	0\\
26.56	0\\
26.57	0\\
26.58	0\\
26.59	0\\
26.6	0\\
26.61	0\\
26.62	0\\
26.63	0\\
26.64	0\\
26.65	0\\
26.66	0\\
26.67	0\\
26.68	0\\
26.69	0\\
26.7	0\\
26.71	0\\
26.72	0\\
26.73	0\\
26.74	0\\
26.75	0\\
26.76	0\\
26.77	0\\
26.78	0\\
26.79	0\\
26.8	0\\
26.81	0\\
26.82	0\\
26.83	0\\
26.84	0\\
26.85	0\\
26.86	0\\
26.87	0\\
26.88	0\\
26.89	0\\
26.9	0\\
26.91	0\\
26.92	0\\
26.93	0\\
26.94	0\\
26.95	0\\
26.96	0\\
26.97	0\\
26.98	0\\
26.99	0\\
27	0\\
27.01	0\\
27.02	0\\
27.03	0\\
27.04	0\\
27.05	0\\
27.06	0\\
27.07	0\\
27.08	0\\
27.09	0\\
27.1	0\\
27.11	0\\
27.12	0\\
27.13	0\\
27.14	0\\
27.15	0\\
27.16	0\\
27.17	0\\
27.18	0\\
27.19	0\\
27.2	0\\
27.21	0\\
27.22	0\\
27.23	0\\
27.24	0\\
27.25	0\\
27.26	0\\
27.27	0\\
27.28	0\\
27.29	0\\
27.3	0\\
27.31	0\\
27.32	0\\
27.33	0\\
27.34	0\\
27.35	0\\
27.36	0\\
27.37	0\\
27.38	0\\
27.39	0\\
27.4	0\\
27.41	0\\
27.42	0\\
27.43	0\\
27.44	0\\
27.45	0\\
27.46	0\\
27.47	0\\
27.48	0\\
27.49	0\\
27.5	0\\
27.51	0\\
27.52	0\\
27.53	0\\
27.54	0\\
27.55	0\\
27.56	0\\
27.57	0\\
27.58	0\\
27.59	0\\
27.6	0\\
27.61	0\\
27.62	0\\
27.63	0\\
27.64	0\\
27.65	0\\
27.66	0\\
27.67	0\\
27.68	0\\
27.69	0\\
27.7	0\\
27.71	0\\
27.72	0\\
27.73	0\\
27.74	0\\
27.75	0\\
27.76	0\\
27.77	0\\
27.78	0\\
27.79	0\\
27.8	0\\
27.81	0\\
27.82	0\\
27.83	0\\
27.84	0\\
27.85	0\\
27.86	0\\
27.87	0\\
27.88	0\\
27.89	0\\
27.9	0\\
27.91	0\\
27.92	0\\
27.93	0\\
27.94	0\\
27.95	0\\
27.96	0\\
27.97	0\\
27.98	0\\
27.99	0\\
28	0\\
28.01	0\\
28.02	0\\
28.03	0\\
28.04	0\\
28.05	0\\
28.06	0\\
28.07	0\\
28.08	0\\
28.09	0\\
28.1	0\\
28.11	0\\
28.12	0\\
28.13	0\\
28.14	0\\
28.15	0\\
28.16	0\\
28.17	0\\
28.18	0\\
28.19	0\\
28.2	0\\
28.21	0\\
28.22	0\\
28.23	0\\
28.24	0\\
28.25	0\\
28.26	0\\
28.27	0\\
28.28	0\\
28.29	0\\
28.3	0\\
28.31	0\\
28.32	0\\
28.33	0\\
28.34	0\\
28.35	0\\
28.36	0\\
28.37	0\\
28.38	0\\
28.39	0\\
28.4	0\\
28.41	0\\
28.42	0\\
28.43	0\\
28.44	0\\
28.45	0\\
28.46	0\\
28.47	0\\
28.48	0\\
28.49	0\\
28.5	0\\
28.51	0\\
28.52	0\\
28.53	0\\
28.54	0\\
28.55	0\\
28.56	0\\
28.57	0\\
28.58	0\\
28.59	0\\
28.6	0\\
28.61	0\\
28.62	0\\
28.63	0\\
28.64	0\\
28.65	0\\
28.66	0\\
28.67	0\\
28.68	0\\
28.69	0\\
28.7	0\\
28.71	0\\
28.72	0\\
28.73	0\\
28.74	0\\
28.75	0\\
28.76	0\\
28.77	0\\
28.78	0\\
28.79	0\\
28.8	0\\
28.81	0\\
28.82	0\\
28.83	0\\
28.84	0\\
28.85	0\\
28.86	0\\
28.87	0\\
28.88	0\\
28.89	0\\
28.9	0\\
28.91	0\\
28.92	0\\
28.93	0\\
28.94	0\\
28.95	0\\
28.96	0\\
28.97	0\\
28.98	0\\
28.99	0\\
29	0\\
29.01	0\\
29.02	0\\
29.03	0\\
29.04	0\\
29.05	0\\
29.06	0\\
29.07	0\\
29.08	0\\
29.09	0\\
29.1	0\\
29.11	0\\
29.12	0\\
29.13	0\\
29.14	0\\
29.15	0\\
29.16	0\\
29.17	0\\
29.18	0\\
29.19	0\\
29.2	0\\
29.21	0\\
29.22	0\\
29.23	0\\
29.24	0\\
29.25	0\\
29.26	0\\
29.27	0\\
29.28	0\\
29.29	0\\
29.3	0\\
29.31	0\\
29.32	0\\
29.33	0\\
29.34	0\\
29.35	0\\
29.36	0\\
29.37	0\\
29.38	0\\
29.39	0\\
29.4	0\\
29.41	0\\
29.42	0\\
29.43	0\\
29.44	0\\
29.45	0\\
29.46	0\\
29.47	0\\
29.48	0\\
29.49	0\\
29.5	0\\
29.51	0\\
29.52	0\\
29.53	0\\
29.54	0\\
29.55	0\\
29.56	0\\
29.57	0\\
29.58	0\\
29.59	0\\
29.6	0\\
29.61	0\\
29.62	0\\
29.63	0\\
29.64	0\\
29.65	0\\
29.66	0\\
29.67	0\\
29.68	0\\
29.69	0\\
29.7	0\\
29.71	0\\
29.72	0\\
29.73	0\\
29.74	0\\
29.75	0\\
29.76	0\\
29.77	0\\
29.78	0\\
29.79	0\\
29.8	0\\
29.81	0\\
29.82	0\\
29.83	0\\
29.84	0\\
29.85	0\\
29.86	0\\
29.87	0\\
29.88	0\\
29.89	0\\
29.9	0\\
29.91	0\\
29.92	0\\
29.93	0\\
29.94	0\\
29.95	0\\
29.96	0\\
29.97	0\\
29.98	0\\
29.99	0\\
30	0\\
30.01	0\\
30.02	0\\
30.03	0\\
30.04	0\\
30.05	0\\
30.06	0\\
30.07	0\\
30.08	0\\
30.09	0\\
30.1	0\\
30.11	0\\
30.12	0\\
30.13	0\\
30.14	0\\
30.15	0\\
30.16	0\\
30.17	0\\
30.18	0\\
30.19	0\\
30.2	0\\
30.21	0\\
30.22	0\\
30.23	0\\
30.24	0\\
30.25	0\\
30.26	0\\
30.27	0\\
30.28	0\\
30.29	0\\
30.3	0\\
30.31	0\\
30.32	0\\
30.33	0\\
30.34	0\\
30.35	0\\
30.36	0\\
30.37	0\\
30.38	0\\
30.39	0\\
30.4	0\\
30.41	0\\
30.42	0\\
30.43	0\\
30.44	0\\
30.45	0\\
30.46	0\\
30.47	0\\
30.48	0\\
30.49	0\\
30.5	0\\
30.51	0\\
30.52	0\\
30.53	0\\
30.54	0\\
30.55	0\\
30.56	0\\
30.57	0\\
30.58	0\\
30.59	0\\
30.6	0\\
30.61	0\\
30.62	0\\
30.63	0\\
30.64	0\\
30.65	0\\
30.66	0\\
30.67	0\\
30.68	0\\
30.69	0\\
30.7	0\\
30.71	0\\
30.72	0\\
30.73	0\\
30.74	0\\
30.75	0\\
30.76	0\\
30.77	0\\
30.78	0\\
30.79	0\\
30.8	0\\
30.81	0\\
30.82	0\\
30.83	0\\
30.84	0\\
30.85	0\\
30.86	0\\
30.87	0\\
30.88	0\\
30.89	0\\
30.9	0\\
30.91	0\\
30.92	0\\
30.93	0\\
30.94	0\\
30.95	0\\
30.96	0\\
30.97	0\\
30.98	0\\
30.99	0\\
31	0\\
31.01	0\\
31.02	0\\
31.03	0\\
31.04	0\\
31.05	0\\
31.06	0\\
31.07	0\\
31.08	0\\
31.09	0\\
31.1	0\\
31.11	0\\
31.12	0\\
31.13	0\\
31.14	0\\
31.15	0\\
31.16	0\\
31.17	0\\
31.18	0\\
31.19	0\\
31.2	0\\
31.21	0\\
31.22	0\\
31.23	0\\
31.24	0\\
31.25	0\\
31.26	0\\
31.27	0\\
31.28	0\\
31.29	0\\
31.3	0\\
31.31	0\\
31.32	0\\
31.33	0\\
31.34	0\\
31.35	0\\
31.36	0\\
31.37	0\\
31.38	0\\
31.39	0\\
31.4	0\\
31.41	0\\
31.42	0\\
31.43	0\\
31.44	0\\
31.45	0\\
31.46	0\\
31.47	0\\
31.48	0\\
31.49	0\\
31.5	0\\
31.51	0\\
31.52	0\\
31.53	0\\
31.54	0\\
31.55	0\\
31.56	0\\
31.57	0\\
31.58	0\\
31.59	0\\
31.6	0\\
31.61	0\\
31.62	0\\
31.63	0\\
31.64	0\\
31.65	0\\
31.66	0\\
31.67	0\\
31.68	0\\
31.69	0\\
31.7	0\\
31.71	0\\
31.72	0\\
31.73	0\\
31.74	0\\
31.75	0\\
31.76	0\\
31.77	0\\
31.78	0\\
31.79	0\\
31.8	0\\
31.81	0\\
31.82	0\\
31.83	0\\
31.84	0\\
31.85	0\\
31.86	0\\
31.87	0\\
31.88	0\\
31.89	0\\
31.9	0\\
31.91	0\\
31.92	0\\
31.93	0\\
31.94	0\\
31.95	0\\
31.96	0\\
31.97	0\\
31.98	0\\
31.99	0\\
32	0\\
32.01	0\\
32.02	0\\
32.03	0\\
32.04	0\\
32.05	0\\
32.06	0\\
32.07	0\\
32.08	0\\
32.09	0\\
32.1	0\\
32.11	0\\
32.12	0\\
32.13	0\\
32.14	0\\
32.15	0\\
32.16	0\\
32.17	0\\
32.18	0\\
32.19	0\\
32.2	0\\
32.21	0\\
32.22	0\\
32.23	0\\
32.24	0\\
32.25	0\\
32.26	0\\
32.27	0\\
32.28	0\\
32.29	0\\
32.3	0\\
32.31	0\\
32.32	0\\
32.33	0\\
32.34	0\\
32.35	0\\
32.36	0\\
32.37	0\\
32.38	0\\
32.39	0\\
32.4	0\\
32.41	0\\
32.42	0\\
32.43	0\\
32.44	0\\
32.45	0\\
32.46	0\\
32.47	0\\
32.48	0\\
32.49	0\\
32.5	0\\
32.51	0\\
32.52	0\\
32.53	0\\
32.54	0\\
32.55	0\\
32.56	0\\
32.57	0\\
32.58	0\\
32.59	0\\
32.6	0\\
32.61	0\\
32.62	0\\
32.63	0\\
32.64	0\\
32.65	0\\
32.66	0\\
32.67	0\\
32.68	0\\
32.69	0\\
32.7	0\\
32.71	0\\
32.72	0\\
32.73	0\\
32.74	0\\
32.75	0\\
32.76	0\\
32.77	0\\
32.78	0\\
32.79	0\\
32.8	0\\
32.81	0\\
32.82	0\\
32.83	0\\
32.84	0\\
32.85	0\\
32.86	0\\
32.87	0\\
32.88	0\\
32.89	0\\
32.9	0\\
32.91	0\\
32.92	0\\
32.93	0\\
32.94	0\\
32.95	0\\
32.96	0\\
32.97	0\\
32.98	0\\
32.99	0\\
33	0\\
33.01	0\\
33.02	0\\
33.03	0\\
33.04	0\\
33.05	0\\
33.06	0\\
33.07	0\\
33.08	0\\
33.09	0\\
33.1	0\\
33.11	0\\
33.12	0\\
33.13	0\\
33.14	0\\
33.15	0\\
33.16	0\\
33.17	0\\
33.18	0\\
33.19	0\\
33.2	0\\
33.21	0\\
33.22	0\\
33.23	0\\
33.24	0\\
33.25	0\\
33.26	0\\
33.27	0\\
33.28	0\\
33.29	0\\
33.3	0\\
33.31	0\\
33.32	0\\
33.33	0\\
33.34	0\\
33.35	0\\
33.36	0\\
33.37	0\\
33.38	0\\
33.39	0\\
33.4	0\\
33.41	0\\
33.42	0\\
33.43	0\\
33.44	0\\
33.45	0\\
33.46	0\\
33.47	0\\
33.48	0\\
33.49	0\\
33.5	0\\
33.51	0\\
33.52	0\\
33.53	0\\
33.54	0\\
33.55	0\\
33.56	0\\
33.57	0\\
33.58	0\\
33.59	0\\
33.6	0\\
33.61	0\\
33.62	0\\
33.63	0\\
33.64	0\\
33.65	0\\
33.66	0\\
33.67	0\\
33.68	0\\
33.69	0\\
33.7	0\\
33.71	0\\
33.72	0\\
33.73	0\\
33.74	0\\
33.75	0\\
33.76	0\\
33.77	0\\
33.78	0\\
33.79	0\\
33.8	0\\
33.81	0\\
33.82	0\\
33.83	0\\
33.84	0\\
33.85	0\\
33.86	0\\
33.87	0\\
33.88	0\\
33.89	0\\
33.9	0\\
33.91	0\\
33.92	0\\
33.93	0\\
33.94	0\\
33.95	0\\
33.96	0\\
33.97	0\\
33.98	0\\
33.99	0\\
34	0\\
34.01	0\\
34.02	0\\
34.03	0\\
34.04	0\\
34.05	0\\
34.06	0\\
34.07	0\\
34.08	0\\
34.09	0\\
34.1	0\\
34.11	0\\
34.12	0\\
34.13	0\\
34.14	0\\
34.15	0\\
34.16	0\\
34.17	0\\
34.18	0\\
34.19	0\\
34.2	0\\
34.21	0\\
34.22	0\\
34.23	0\\
34.24	0\\
34.25	0\\
34.26	0\\
34.27	0\\
34.28	0\\
34.29	0\\
34.3	0\\
34.31	0\\
34.32	0\\
34.33	0\\
34.34	0\\
34.35	0\\
34.36	0\\
34.37	0\\
34.38	0\\
34.39	0\\
34.4	0\\
34.41	0\\
34.42	0\\
34.43	0\\
34.44	0\\
34.45	0\\
34.46	0\\
34.47	0\\
34.48	0\\
34.49	0\\
34.5	0\\
34.51	0\\
34.52	0\\
34.53	0\\
34.54	0\\
34.55	0\\
34.56	0\\
34.57	0\\
34.58	0\\
34.59	0\\
34.6	0\\
34.61	0\\
34.62	0\\
34.63	0\\
34.64	0\\
34.65	0\\
34.66	0\\
34.67	0\\
34.68	0\\
34.69	0\\
34.7	0\\
34.71	0\\
34.72	0\\
34.73	0\\
34.74	0\\
34.75	0\\
34.76	0\\
34.77	0\\
34.78	0\\
34.79	0\\
34.8	0\\
34.81	0\\
34.82	0\\
34.83	0\\
34.84	0\\
34.85	0\\
34.86	0\\
34.87	0\\
34.88	0\\
34.89	0\\
34.9	0\\
34.91	0\\
34.92	0\\
34.93	0\\
34.94	0\\
34.95	0\\
34.96	0\\
34.97	0\\
34.98	0\\
34.99	0\\
35	0\\
35.01	0\\
35.02	0\\
35.03	0\\
35.04	0\\
35.05	0\\
35.06	0\\
35.07	0\\
35.08	0\\
35.09	0\\
35.1	0\\
35.11	0\\
35.12	0\\
35.13	0\\
35.14	0\\
35.15	0\\
35.16	0\\
35.17	0\\
35.18	0\\
35.19	0\\
35.2	0\\
35.21	0\\
35.22	0\\
35.23	0\\
35.24	0\\
35.25	0\\
35.26	0\\
35.27	0\\
35.28	0\\
35.29	0\\
35.3	0\\
35.31	0\\
35.32	0\\
35.33	0\\
35.34	0\\
35.35	0\\
35.36	0\\
35.37	0\\
35.38	0\\
35.39	0\\
35.4	0\\
35.41	0\\
35.42	0\\
35.43	0\\
35.44	0\\
35.45	0\\
35.46	0\\
35.47	0\\
35.48	0\\
35.49	0\\
35.5	0\\
35.51	0\\
35.52	0\\
35.53	0\\
35.54	0\\
35.55	0\\
35.56	0\\
35.57	0\\
35.58	0\\
35.59	0\\
35.6	0\\
35.61	0\\
35.62	0\\
35.63	0\\
35.64	0\\
35.65	0\\
35.66	0\\
35.67	0\\
35.68	0\\
35.69	0\\
35.7	0\\
35.71	0\\
35.72	0\\
35.73	0\\
35.74	0\\
35.75	0\\
35.76	0\\
35.77	0\\
35.78	0\\
35.79	0\\
35.8	0\\
35.81	0\\
35.82	0\\
35.83	0\\
35.84	0\\
35.85	0\\
35.86	0\\
35.87	0\\
35.88	0\\
35.89	0\\
35.9	0\\
35.91	0\\
35.92	0\\
35.93	0\\
35.94	0\\
35.95	0\\
35.96	0\\
35.97	0\\
35.98	0\\
35.99	0\\
36	0\\
36.01	0\\
36.02	0\\
36.03	0\\
36.04	0\\
36.05	0\\
36.06	0\\
36.07	0\\
36.08	0\\
36.09	0\\
36.1	0\\
36.11	0\\
36.12	0\\
36.13	0\\
36.14	0\\
36.15	0\\
36.16	0\\
36.17	0\\
36.18	0\\
36.19	0\\
36.2	0\\
36.21	0\\
36.22	0\\
36.23	0\\
36.24	0\\
36.25	0\\
36.26	0\\
36.27	0\\
36.28	0\\
36.29	0\\
36.3	0\\
36.31	0\\
36.32	0\\
36.33	0\\
36.34	0\\
36.35	0\\
36.36	0\\
36.37	0\\
36.38	0\\
36.39	0\\
36.4	0\\
36.41	0\\
36.42	0\\
36.43	0\\
36.44	0\\
36.45	0\\
36.46	0\\
36.47	0\\
36.48	0\\
36.49	0\\
36.5	0\\
36.51	0\\
36.52	0\\
36.53	0\\
36.54	0\\
36.55	0\\
36.56	0\\
36.57	0\\
36.58	0\\
36.59	0\\
36.6	0\\
36.61	0\\
36.62	0\\
36.63	0\\
36.64	0\\
36.65	0\\
36.66	0\\
36.67	0\\
36.68	0\\
36.69	0\\
36.7	0\\
36.71	0\\
36.72	0\\
36.73	0\\
36.74	0\\
36.75	0\\
36.76	0\\
36.77	0\\
36.78	0\\
36.79	0\\
36.8	0\\
36.81	0\\
36.82	0\\
36.83	0\\
36.84	0\\
36.85	0\\
36.86	0\\
36.87	0\\
36.88	0\\
36.89	0\\
36.9	0\\
36.91	0\\
36.92	0\\
36.93	0\\
36.94	0\\
36.95	0\\
36.96	0\\
36.97	0\\
36.98	0\\
36.99	0\\
37	0\\
37.01	0\\
37.02	0\\
37.03	0\\
37.04	0\\
37.05	0\\
37.06	0\\
37.07	0\\
37.08	0\\
37.09	0\\
37.1	0\\
37.11	0\\
37.12	0\\
37.13	0\\
37.14	0\\
37.15	0\\
37.16	0\\
37.17	0\\
37.18	0\\
37.19	0\\
37.2	0\\
37.21	0\\
37.22	0\\
37.23	0\\
37.24	0\\
37.25	0\\
37.26	0\\
37.27	0\\
37.28	0\\
37.29	0\\
37.3	0\\
37.31	0\\
37.32	0\\
37.33	0\\
37.34	0\\
37.35	0\\
37.36	0\\
37.37	0\\
37.38	0\\
37.39	0\\
37.4	0\\
37.41	0\\
37.42	0\\
37.43	0\\
37.44	0\\
37.45	0\\
37.46	0\\
37.47	0\\
37.48	0\\
37.49	0\\
37.5	0\\
37.51	0\\
37.52	0\\
37.53	0\\
37.54	0\\
37.55	0\\
37.56	0\\
37.57	0\\
37.58	0\\
37.59	0\\
37.6	0\\
37.61	0\\
37.62	0\\
37.63	0\\
37.64	0\\
37.65	0\\
37.66	0\\
37.67	0\\
37.68	0\\
37.69	0\\
37.7	0\\
37.71	0\\
37.72	0\\
37.73	0\\
37.74	0\\
37.75	0\\
37.76	0\\
37.77	0\\
37.78	0\\
37.79	0\\
37.8	0\\
37.81	0\\
37.82	0\\
37.83	0\\
37.84	0\\
37.85	0\\
37.86	0\\
37.87	0\\
37.88	0\\
37.89	0\\
37.9	0\\
37.91	0\\
37.92	0\\
37.93	0\\
37.94	0\\
37.95	0\\
37.96	0\\
37.97	0\\
37.98	0\\
37.99	0\\
38	0\\
38.01	0\\
38.02	0\\
38.03	0\\
38.04	0\\
38.05	0\\
38.06	0\\
38.07	0\\
38.08	0\\
38.09	0\\
38.1	0\\
38.11	0\\
38.12	0\\
38.13	0\\
38.14	0\\
38.15	0\\
38.16	0\\
38.17	0\\
38.18	0\\
38.19	0\\
38.2	0\\
38.21	0\\
38.22	0\\
38.23	0\\
38.24	0\\
38.25	0\\
38.26	0\\
38.27	0\\
38.28	0\\
38.29	0\\
38.3	0\\
38.31	0\\
38.32	0\\
38.33	0\\
38.34	0\\
38.35	0\\
38.36	0\\
38.37	0\\
38.38	0\\
38.39	0\\
38.4	0\\
38.41	0\\
38.42	0\\
38.43	0\\
38.44	0\\
38.45	0\\
38.46	0\\
38.47	0\\
38.48	0\\
38.49	0\\
38.5	0\\
38.51	0\\
38.52	0\\
38.53	0\\
38.54	0\\
38.55	0\\
38.56	0\\
38.57	0\\
38.58	0\\
38.59	0\\
38.6	0\\
38.61	0\\
38.62	0\\
38.63	0\\
38.64	0\\
38.65	0\\
38.66	0\\
38.67	0\\
38.68	0\\
38.69	0\\
38.7	0\\
38.71	0\\
38.72	0\\
38.73	0\\
38.74	0\\
38.75	0\\
38.76	0\\
38.77	0\\
38.78	0\\
38.79	0\\
38.8	0\\
38.81	0\\
38.82	0\\
38.83	0\\
38.84	0\\
38.85	0\\
38.86	0\\
38.87	0\\
38.88	0\\
38.89	0\\
38.9	0\\
38.91	0\\
38.92	0\\
38.93	0\\
38.94	0\\
38.95	0\\
38.96	0\\
38.97	0\\
38.98	0\\
38.99	0\\
39	0\\
39.01	0\\
39.02	0\\
39.03	0\\
39.04	0\\
39.05	0\\
39.06	0\\
39.07	0\\
39.08	0\\
39.09	0\\
39.1	0\\
39.11	0\\
39.12	0\\
39.13	0\\
39.14	0\\
39.15	0\\
39.16	0\\
39.17	0\\
39.18	0\\
39.19	0\\
39.2	0\\
39.21	0\\
39.22	0\\
39.23	0\\
39.24	0\\
39.25	0\\
39.26	0\\
39.27	0\\
39.28	0\\
39.29	0\\
39.3	0\\
39.31	0\\
39.32	0\\
39.33	0\\
39.34	0\\
39.35	0\\
39.36	0\\
39.37	0\\
39.38	0\\
39.39	0\\
39.4	0\\
39.41	0\\
39.42	0\\
39.43	0\\
39.44	0\\
39.45	0\\
39.46	0\\
39.47	0\\
39.48	0\\
39.49	0\\
39.5	0\\
39.51	0\\
39.52	0\\
39.53	0\\
39.54	0\\
39.55	0\\
39.56	0\\
39.57	0\\
39.58	0\\
39.59	0\\
39.6	0\\
39.61	0\\
39.62	0\\
39.63	0\\
39.64	0\\
39.65	0\\
39.66	0\\
39.67	0\\
39.68	0\\
39.69	0\\
39.7	0\\
39.71	0\\
39.72	0\\
39.73	0\\
39.74	0\\
39.75	0\\
39.76	0\\
39.77	0\\
39.78	0\\
39.79	0\\
39.8	0\\
39.81	0\\
39.82	0\\
39.83	0\\
39.84	0\\
39.85	0\\
39.86	0\\
39.87	0\\
39.88	0\\
39.89	0\\
39.9	0\\
39.91	0\\
39.92	0\\
39.93	0\\
39.94	0\\
39.95	0\\
39.96	0\\
39.97	0\\
39.98	0\\
39.99	0\\
40	0\\
40.01	0\\
};
\addplot [color=red,solid,forget plot]
  table[row sep=crcr]{%
40.01	0\\
40.02	0\\
40.03	0\\
40.04	0\\
40.05	0\\
40.06	0\\
40.07	0\\
40.08	0\\
40.09	0\\
40.1	0\\
40.11	0\\
40.12	0\\
40.13	0\\
40.14	0\\
40.15	0\\
40.16	0\\
40.17	0\\
40.18	0\\
40.19	0\\
40.2	0\\
40.21	0\\
40.22	0\\
40.23	0\\
40.24	0\\
40.25	0\\
40.26	0\\
40.27	0\\
40.28	0\\
40.29	0\\
40.3	0\\
40.31	0\\
40.32	0\\
40.33	0\\
40.34	0\\
40.35	0\\
40.36	0\\
40.37	0\\
40.38	0\\
40.39	0\\
40.4	0\\
40.41	0\\
40.42	0\\
40.43	0\\
40.44	0\\
40.45	0\\
40.46	0\\
40.47	0\\
40.48	0\\
40.49	0\\
40.5	0\\
40.51	0\\
40.52	0\\
40.53	0\\
40.54	0\\
40.55	0\\
40.56	0\\
40.57	0\\
40.58	0\\
40.59	0\\
40.6	0\\
40.61	0\\
40.62	0\\
40.63	0\\
40.64	0\\
40.65	0\\
40.66	0\\
40.67	0\\
40.68	0\\
40.69	0\\
40.7	0\\
40.71	0\\
40.72	0\\
40.73	0\\
40.74	0\\
40.75	0\\
40.76	0\\
40.77	0\\
40.78	0\\
40.79	0\\
40.8	0\\
40.81	0\\
40.82	0\\
40.83	0\\
40.84	0\\
40.85	0\\
40.86	0\\
40.87	0\\
40.88	0\\
40.89	0\\
40.9	0\\
40.91	0\\
40.92	0\\
40.93	0\\
40.94	0\\
40.95	0\\
40.96	0\\
40.97	0\\
40.98	0\\
40.99	0\\
41	0\\
41.01	0\\
41.02	0\\
41.03	0\\
41.04	0\\
41.05	0\\
41.06	0\\
41.07	0\\
41.08	0\\
41.09	0\\
41.1	0\\
41.11	0\\
41.12	0\\
41.13	0\\
41.14	0\\
41.15	0\\
41.16	0\\
41.17	0\\
41.18	0\\
41.19	0\\
41.2	0\\
41.21	0\\
41.22	0\\
41.23	0\\
41.24	0\\
41.25	0\\
41.26	0\\
41.27	0\\
41.28	0\\
41.29	0\\
41.3	0\\
41.31	0\\
41.32	0\\
41.33	0\\
41.34	0\\
41.35	0\\
41.36	0\\
41.37	0\\
41.38	0\\
41.39	0\\
41.4	0\\
41.41	0\\
41.42	0\\
41.43	0\\
41.44	0\\
41.45	0\\
41.46	0\\
41.47	0\\
41.48	0\\
41.49	0\\
41.5	0\\
41.51	0\\
41.52	0\\
41.53	0\\
41.54	0\\
41.55	0\\
41.56	0\\
41.57	0\\
41.58	0\\
41.59	0\\
41.6	0\\
41.61	0\\
41.62	0\\
41.63	0\\
41.64	0\\
41.65	0\\
41.66	0\\
41.67	0\\
41.68	0\\
41.69	0\\
41.7	0\\
41.71	0\\
41.72	0\\
41.73	0\\
41.74	0\\
41.75	0\\
41.76	0\\
41.77	0\\
41.78	0\\
41.79	0\\
41.8	0\\
41.81	0\\
41.82	0\\
41.83	0\\
41.84	0\\
41.85	0\\
41.86	0\\
41.87	0\\
41.88	0\\
41.89	0\\
41.9	0\\
41.91	0\\
41.92	0\\
41.93	0\\
41.94	0\\
41.95	0\\
41.96	0\\
41.97	0\\
41.98	0\\
41.99	0\\
42	0\\
42.01	0\\
42.02	0\\
42.03	0\\
42.04	0\\
42.05	0\\
42.06	0\\
42.07	0\\
42.08	0\\
42.09	0\\
42.1	0\\
42.11	0\\
42.12	0\\
42.13	0\\
42.14	0\\
42.15	0\\
42.16	0\\
42.17	0\\
42.18	0\\
42.19	0\\
42.2	0\\
42.21	0\\
42.22	0\\
42.23	0\\
42.24	0\\
42.25	0\\
42.26	0\\
42.27	0\\
42.28	0\\
42.29	0\\
42.3	0\\
42.31	0\\
42.32	0\\
42.33	0\\
42.34	0\\
42.35	0\\
42.36	0\\
42.37	0\\
42.38	0\\
42.39	0\\
42.4	0\\
42.41	0\\
42.42	0\\
42.43	0\\
42.44	0\\
42.45	0\\
42.46	0\\
42.47	0\\
42.48	0\\
42.49	0\\
42.5	0\\
42.51	0\\
42.52	0\\
42.53	0\\
42.54	0\\
42.55	0\\
42.56	0\\
42.57	0\\
42.58	0\\
42.59	0\\
42.6	0\\
42.61	0\\
42.62	0\\
42.63	0\\
42.64	0\\
42.65	0\\
42.66	0\\
42.67	0\\
42.68	0\\
42.69	0\\
42.7	0\\
42.71	0\\
42.72	0\\
42.73	0\\
42.74	0\\
42.75	0\\
42.76	0\\
42.77	0\\
42.78	0\\
42.79	0\\
42.8	0\\
42.81	0\\
42.82	0\\
42.83	0\\
42.84	0\\
42.85	0\\
42.86	0\\
42.87	0\\
42.88	0\\
42.89	0\\
42.9	0\\
42.91	0\\
42.92	0\\
42.93	0\\
42.94	0\\
42.95	0\\
42.96	0\\
42.97	0\\
42.98	0\\
42.99	0\\
43	0\\
43.01	0\\
43.02	0\\
43.03	0\\
43.04	0\\
43.05	0\\
43.06	0\\
43.07	0\\
43.08	0\\
43.09	0\\
43.1	0\\
43.11	0\\
43.12	0\\
43.13	0\\
43.14	0\\
43.15	0\\
43.16	0\\
43.17	0\\
43.18	0\\
43.19	0\\
43.2	0\\
43.21	0\\
43.22	0\\
43.23	0\\
43.24	0\\
43.25	0\\
43.26	0\\
43.27	0\\
43.28	0\\
43.29	0\\
43.3	0\\
43.31	0\\
43.32	0\\
43.33	0\\
43.34	0\\
43.35	0\\
43.36	0\\
43.37	0\\
43.38	0\\
43.39	0\\
43.4	0\\
43.41	0\\
43.42	0\\
43.43	0\\
43.44	0\\
43.45	0\\
43.46	0\\
43.47	0\\
43.48	0\\
43.49	0\\
43.5	0\\
43.51	0\\
43.52	0\\
43.53	0\\
43.54	0\\
43.55	0\\
43.56	0\\
43.57	0\\
43.58	0\\
43.59	0\\
43.6	0\\
43.61	0\\
43.62	0\\
43.63	0\\
43.64	0\\
43.65	0\\
43.66	0\\
43.67	0\\
43.68	0\\
43.69	0\\
43.7	0\\
43.71	0\\
43.72	0\\
43.73	0\\
43.74	0\\
43.75	0\\
43.76	0\\
43.77	0\\
43.78	0\\
43.79	0\\
43.8	0\\
43.81	0\\
43.82	0\\
43.83	0\\
43.84	0\\
43.85	0\\
43.86	0\\
43.87	0\\
43.88	0\\
43.89	0\\
43.9	0\\
43.91	0\\
43.92	0\\
43.93	0\\
43.94	0\\
43.95	0\\
43.96	0\\
43.97	0\\
43.98	0\\
43.99	0\\
44	0\\
44.01	0\\
44.02	0\\
44.03	0\\
44.04	0\\
44.05	0\\
44.06	0\\
44.07	0\\
44.08	0\\
44.09	0\\
44.1	0\\
44.11	0\\
44.12	0\\
44.13	0\\
44.14	0\\
44.15	0\\
44.16	0\\
44.17	0\\
44.18	0\\
44.19	0\\
44.2	0\\
44.21	0\\
44.22	0\\
44.23	0\\
44.24	0\\
44.25	0\\
44.26	0\\
44.27	0\\
44.28	0\\
44.29	0\\
44.3	0\\
44.31	0\\
44.32	0\\
44.33	0\\
44.34	0\\
44.35	0\\
44.36	0\\
44.37	0\\
44.38	0\\
44.39	0\\
44.4	0\\
44.41	0\\
44.42	0\\
44.43	0\\
44.44	0\\
44.45	0\\
44.46	0\\
44.47	0\\
44.48	0\\
44.49	0\\
44.5	0\\
44.51	0\\
44.52	0\\
44.53	0\\
44.54	0\\
44.55	0\\
44.56	0\\
44.57	0\\
44.58	0\\
44.59	0\\
44.6	0\\
44.61	0\\
44.62	0\\
44.63	0\\
44.64	0\\
44.65	0\\
44.66	0\\
44.67	0\\
44.68	0\\
44.69	0\\
44.7	0\\
44.71	0\\
44.72	0\\
44.73	0\\
44.74	0\\
44.75	0\\
44.76	0\\
44.77	0\\
44.78	0\\
44.79	0\\
44.8	0\\
44.81	0\\
44.82	0\\
44.83	0\\
44.84	0\\
44.85	0\\
44.86	0\\
44.87	0\\
44.88	0\\
44.89	0\\
44.9	0\\
44.91	0\\
44.92	0\\
44.93	0\\
44.94	0\\
44.95	0\\
44.96	0\\
44.97	0\\
44.98	0\\
44.99	0\\
45	0\\
45.01	0\\
45.02	0\\
45.03	0\\
45.04	0\\
45.05	0\\
45.06	0\\
45.07	0\\
45.08	0\\
45.09	0\\
45.1	0\\
45.11	0\\
45.12	0\\
45.13	0\\
45.14	0\\
45.15	0\\
45.16	0\\
45.17	0\\
45.18	0\\
45.19	0\\
45.2	0\\
45.21	0\\
45.22	0\\
45.23	0\\
45.24	0\\
45.25	0\\
45.26	0\\
45.27	0\\
45.28	0\\
45.29	0\\
45.3	0\\
45.31	0\\
45.32	0\\
45.33	0\\
45.34	0\\
45.35	0\\
45.36	0\\
45.37	0\\
45.38	0\\
45.39	0\\
45.4	0\\
45.41	0\\
45.42	0\\
45.43	0\\
45.44	0\\
45.45	0\\
45.46	0\\
45.47	0\\
45.48	0\\
45.49	0\\
45.5	0\\
45.51	0\\
45.52	0\\
45.53	0\\
45.54	0\\
45.55	0\\
45.56	0\\
45.57	0\\
45.58	0\\
45.59	0\\
45.6	0\\
45.61	0\\
45.62	0\\
45.63	0\\
45.64	0\\
45.65	0\\
45.66	0\\
45.67	0\\
45.68	0\\
45.69	0\\
45.7	0\\
45.71	0\\
45.72	0\\
45.73	0\\
45.74	0\\
45.75	0\\
45.76	0\\
45.77	0\\
45.78	0\\
45.79	0\\
45.8	0\\
45.81	0\\
45.82	0\\
45.83	0\\
45.84	0\\
45.85	0\\
45.86	0\\
45.87	0\\
45.88	0\\
45.89	0\\
45.9	0\\
45.91	0\\
45.92	0\\
45.93	0\\
45.94	0\\
45.95	0\\
45.96	0\\
45.97	0\\
45.98	0\\
45.99	0\\
46	0\\
46.01	0\\
46.02	0\\
46.03	0\\
46.04	0\\
46.05	0\\
46.06	0\\
46.07	0\\
46.08	0\\
46.09	0\\
46.1	0\\
46.11	0\\
46.12	0\\
46.13	0\\
46.14	0\\
46.15	0\\
46.16	0\\
46.17	0\\
46.18	0\\
46.19	0\\
46.2	0\\
46.21	0\\
46.22	0\\
46.23	0\\
46.24	0\\
46.25	0\\
46.26	0\\
46.27	0\\
46.28	0\\
46.29	0\\
46.3	0\\
46.31	0\\
46.32	0\\
46.33	0\\
46.34	0\\
46.35	0\\
46.36	0\\
46.37	0\\
46.38	0\\
46.39	0\\
46.4	0\\
46.41	0\\
46.42	0\\
46.43	0\\
46.44	0\\
46.45	0\\
46.46	0\\
46.47	0\\
46.48	0\\
46.49	0\\
46.5	0\\
46.51	0\\
46.52	0\\
46.53	0\\
46.54	0\\
46.55	0\\
46.56	0\\
46.57	0\\
46.58	0\\
46.59	0\\
46.6	0\\
46.61	0\\
46.62	0\\
46.63	0\\
46.64	0\\
46.65	0\\
46.66	0\\
46.67	0\\
46.68	0\\
46.69	0\\
46.7	0\\
46.71	0\\
46.72	0\\
46.73	0\\
46.74	0\\
46.75	0\\
46.76	0\\
46.77	0\\
46.78	0\\
46.79	0\\
46.8	0\\
46.81	0\\
46.82	0\\
46.83	0\\
46.84	0\\
46.85	0\\
46.86	0\\
46.87	0\\
46.88	0\\
46.89	0\\
46.9	0\\
46.91	0\\
46.92	0\\
46.93	0\\
46.94	0\\
46.95	0\\
46.96	0\\
46.97	0\\
46.98	0\\
46.99	0\\
47	0\\
47.01	0\\
47.02	0\\
47.03	0\\
47.04	0\\
47.05	0\\
47.06	0\\
47.07	0\\
47.08	0\\
47.09	0\\
47.1	0\\
47.11	0\\
47.12	0\\
47.13	0\\
47.14	0\\
47.15	0\\
47.16	0\\
47.17	0\\
47.18	0\\
47.19	0\\
47.2	0\\
47.21	0\\
47.22	0\\
47.23	0\\
47.24	0\\
47.25	0\\
47.26	0\\
47.27	0\\
47.28	0\\
47.29	0\\
47.3	0\\
47.31	0\\
47.32	0\\
47.33	0\\
47.34	0\\
47.35	0\\
47.36	0\\
47.37	0\\
47.38	0\\
47.39	0\\
47.4	0\\
47.41	0\\
47.42	0\\
47.43	0\\
47.44	0\\
47.45	0\\
47.46	0\\
47.47	0\\
47.48	0\\
47.49	0\\
47.5	0\\
47.51	0\\
47.52	0\\
47.53	0\\
47.54	0\\
47.55	0\\
47.56	0\\
47.57	0\\
47.58	0\\
47.59	0\\
47.6	0\\
47.61	0\\
47.62	0\\
47.63	0\\
47.64	0\\
47.65	0\\
47.66	0\\
47.67	0\\
47.68	0\\
47.69	0\\
47.7	0\\
47.71	0\\
47.72	0\\
47.73	0\\
47.74	0\\
47.75	0\\
47.76	0\\
47.77	0\\
47.78	0\\
47.79	0\\
47.8	0\\
47.81	0\\
47.82	0\\
47.83	0\\
47.84	0\\
47.85	0\\
47.86	0\\
47.87	0\\
47.88	0\\
47.89	0\\
47.9	0\\
47.91	0\\
47.92	0\\
47.93	0\\
47.94	0\\
47.95	0\\
47.96	0\\
47.97	0\\
47.98	0\\
47.99	0\\
48	0\\
48.01	0\\
48.02	0\\
48.03	0\\
48.04	0\\
48.05	0\\
48.06	0\\
48.07	0\\
48.08	0\\
48.09	0\\
48.1	0\\
48.11	0\\
48.12	0\\
48.13	0\\
48.14	0\\
48.15	0\\
48.16	0\\
48.17	0\\
48.18	0\\
48.19	0\\
48.2	0\\
48.21	0\\
48.22	0\\
48.23	0\\
48.24	0\\
48.25	0\\
48.26	0\\
48.27	0\\
48.28	0\\
48.29	0\\
48.3	0\\
48.31	0\\
48.32	0\\
48.33	0\\
48.34	0\\
48.35	0\\
48.36	0\\
48.37	0\\
48.38	0\\
48.39	0\\
48.4	0\\
48.41	0\\
48.42	0\\
48.43	0\\
48.44	0\\
48.45	0\\
48.46	0\\
48.47	0\\
48.48	0\\
48.49	0\\
48.5	0\\
48.51	0\\
48.52	0\\
48.53	0\\
48.54	0\\
48.55	0\\
48.56	0\\
48.57	0\\
48.58	0\\
48.59	0\\
48.6	0\\
48.61	0\\
48.62	0\\
48.63	0\\
48.64	0\\
48.65	0\\
48.66	0\\
48.67	0\\
48.68	0\\
48.69	0\\
48.7	0\\
48.71	0\\
48.72	0\\
48.73	0\\
48.74	0\\
48.75	0\\
48.76	0\\
48.77	0\\
48.78	0\\
48.79	0\\
48.8	0\\
48.81	0\\
48.82	0\\
48.83	0\\
48.84	0\\
48.85	0\\
48.86	0\\
48.87	0\\
48.88	0\\
48.89	0\\
48.9	0\\
48.91	0\\
48.92	0\\
48.93	0\\
48.94	0\\
48.95	0\\
48.96	0\\
48.97	0\\
48.98	0\\
48.99	0\\
49	0\\
49.01	0\\
49.02	0\\
49.03	0\\
49.04	0\\
49.05	0\\
49.06	0\\
49.07	0\\
49.08	0\\
49.09	0\\
49.1	0\\
49.11	0\\
49.12	0\\
49.13	0\\
49.14	0\\
49.15	0\\
49.16	0\\
49.17	0\\
49.18	0\\
49.19	0\\
49.2	0\\
49.21	0\\
49.22	0\\
49.23	0\\
49.24	0\\
49.25	0\\
49.26	0\\
49.27	0\\
49.28	0\\
49.29	0\\
49.3	0\\
49.31	0\\
49.32	0\\
49.33	0\\
49.34	0\\
49.35	0\\
49.36	0\\
49.37	0\\
49.38	0\\
49.39	0\\
49.4	0\\
49.41	0\\
49.42	0\\
49.43	0\\
49.44	0\\
49.45	0\\
49.46	0\\
49.47	0\\
49.48	0\\
49.49	0\\
49.5	0\\
49.51	0\\
49.52	0\\
49.53	0\\
49.54	0\\
49.55	0\\
49.56	0\\
49.57	0\\
49.58	0\\
49.59	0\\
49.6	0\\
49.61	0\\
49.62	0\\
49.63	0\\
49.64	0\\
49.65	0\\
49.66	0\\
49.67	0\\
49.68	0\\
49.69	0\\
49.7	0\\
49.71	0\\
49.72	0\\
49.73	0\\
49.74	0\\
49.75	0\\
49.76	0\\
49.77	0\\
49.78	0\\
49.79	0\\
49.8	0\\
49.81	0\\
49.82	0\\
49.83	0\\
49.84	0\\
49.85	0\\
49.86	0\\
49.87	0\\
49.88	0\\
49.89	0\\
49.9	0\\
49.91	0\\
49.92	0\\
49.93	0\\
49.94	0\\
49.95	0\\
49.96	0\\
49.97	0\\
49.98	0\\
49.99	0\\
50	0\\
50.01	0\\
50.02	0\\
50.03	0\\
50.04	0\\
50.05	0\\
50.06	0\\
50.07	0\\
50.08	0\\
50.09	0\\
50.1	0\\
50.11	0\\
50.12	0\\
50.13	0\\
50.14	0\\
50.15	0\\
50.16	0\\
50.17	0\\
50.18	0\\
50.19	0\\
50.2	0\\
50.21	0\\
50.22	0\\
50.23	0\\
50.24	0\\
50.25	0\\
50.26	0\\
50.27	0\\
50.28	0\\
50.29	0\\
50.3	0\\
50.31	0\\
50.32	0\\
50.33	0\\
50.34	0\\
50.35	0\\
50.36	0\\
50.37	0\\
50.38	0\\
50.39	0\\
50.4	0\\
50.41	0\\
50.42	0\\
50.43	0\\
50.44	0\\
50.45	0\\
50.46	0\\
50.47	0\\
50.48	0\\
50.49	0\\
50.5	0\\
50.51	0\\
50.52	0\\
50.53	0\\
50.54	0\\
50.55	0\\
50.56	0\\
50.57	0\\
50.58	0\\
50.59	0\\
50.6	0\\
50.61	0\\
50.62	0\\
50.63	0\\
50.64	0\\
50.65	0\\
50.66	0\\
50.67	0\\
50.68	0\\
50.69	0\\
50.7	0\\
50.71	0\\
50.72	0\\
50.73	0\\
50.74	0\\
50.75	0\\
50.76	0\\
50.77	0\\
50.78	0\\
50.79	0\\
50.8	0\\
50.81	0\\
50.82	0\\
50.83	0\\
50.84	0\\
50.85	0\\
50.86	0\\
50.87	0\\
50.88	0\\
50.89	0\\
50.9	0\\
50.91	0\\
50.92	0\\
50.93	0\\
50.94	0\\
50.95	0\\
50.96	0\\
50.97	0\\
50.98	0\\
50.99	0\\
51	0\\
51.01	0\\
51.02	0\\
51.03	0\\
51.04	0\\
51.05	0\\
51.06	0\\
51.07	0\\
51.08	0\\
51.09	0\\
51.1	0\\
51.11	0\\
51.12	0\\
51.13	0\\
51.14	0\\
51.15	0\\
51.16	0\\
51.17	0\\
51.18	0\\
51.19	0\\
51.2	0\\
51.21	0\\
51.22	0\\
51.23	0\\
51.24	0\\
51.25	0\\
51.26	0\\
51.27	0\\
51.28	0\\
51.29	0\\
51.3	0\\
51.31	0\\
51.32	0\\
51.33	0\\
51.34	0\\
51.35	0\\
51.36	0\\
51.37	0\\
51.38	0\\
51.39	0\\
51.4	0\\
51.41	0\\
51.42	0\\
51.43	0\\
51.44	0\\
51.45	0\\
51.46	0\\
51.47	0\\
51.48	0\\
51.49	0\\
51.5	0\\
51.51	0\\
51.52	0\\
51.53	0\\
51.54	0\\
51.55	0\\
51.56	0\\
51.57	0\\
51.58	0\\
51.59	0\\
51.6	0\\
51.61	0\\
51.62	0\\
51.63	0\\
51.64	0\\
51.65	0\\
51.66	0\\
51.67	0\\
51.68	0\\
51.69	0\\
51.7	0\\
51.71	0\\
51.72	0\\
51.73	0\\
51.74	0\\
51.75	0\\
51.76	0\\
51.77	0\\
51.78	0\\
51.79	0\\
51.8	0\\
51.81	0\\
51.82	0\\
51.83	0\\
51.84	0\\
51.85	0\\
51.86	0\\
51.87	0\\
51.88	0\\
51.89	0\\
51.9	0\\
51.91	0\\
51.92	0\\
51.93	0\\
51.94	0\\
51.95	0\\
51.96	0\\
51.97	0\\
51.98	0\\
51.99	0\\
52	0\\
52.01	0\\
52.02	0\\
52.03	0\\
52.04	0\\
52.05	0\\
52.06	0\\
52.07	0\\
52.08	0\\
52.09	0\\
52.1	0\\
52.11	0\\
52.12	0\\
52.13	0\\
52.14	0\\
52.15	0\\
52.16	0\\
52.17	0\\
52.18	0\\
52.19	0\\
52.2	0\\
52.21	0\\
52.22	0\\
52.23	0\\
52.24	0\\
52.25	0\\
52.26	0\\
52.27	0\\
52.28	0\\
52.29	0\\
52.3	0\\
52.31	0\\
52.32	0\\
52.33	0\\
52.34	0\\
52.35	0\\
52.36	0\\
52.37	0\\
52.38	0\\
52.39	0\\
52.4	0\\
52.41	0\\
52.42	0\\
52.43	0\\
52.44	0\\
52.45	0\\
52.46	0\\
52.47	0\\
52.48	0\\
52.49	0\\
52.5	0\\
52.51	0\\
52.52	0\\
52.53	0\\
52.54	0\\
52.55	0\\
52.56	0\\
52.57	0\\
52.58	0\\
52.59	0\\
52.6	0\\
52.61	0\\
52.62	0\\
52.63	0\\
52.64	0\\
52.65	0\\
52.66	0\\
52.67	0\\
52.68	0\\
52.69	0\\
52.7	0\\
52.71	0\\
52.72	0\\
52.73	0\\
52.74	0\\
52.75	0\\
52.76	0\\
52.77	0\\
52.78	0\\
52.79	0\\
52.8	0\\
52.81	0\\
52.82	0\\
52.83	0\\
52.84	0\\
52.85	0\\
52.86	0\\
52.87	0\\
52.88	0\\
52.89	0\\
52.9	0\\
52.91	0\\
52.92	0\\
52.93	0\\
52.94	0\\
52.95	0\\
52.96	0\\
52.97	0\\
52.98	0\\
52.99	0\\
53	0\\
53.01	0\\
53.02	0\\
53.03	0\\
53.04	0\\
53.05	0\\
53.06	0\\
53.07	0\\
53.08	0\\
53.09	0\\
53.1	0\\
53.11	0\\
53.12	0\\
53.13	0\\
53.14	0\\
53.15	0\\
53.16	0\\
53.17	0\\
53.18	0\\
53.19	0\\
53.2	0\\
53.21	0\\
53.22	0\\
53.23	0\\
53.24	0\\
53.25	0\\
53.26	0\\
53.27	0\\
53.28	0\\
53.29	0\\
53.3	0\\
53.31	0\\
53.32	0\\
53.33	0\\
53.34	0\\
53.35	0\\
53.36	0\\
53.37	0\\
53.38	0\\
53.39	0\\
53.4	0\\
53.41	0\\
53.42	0\\
53.43	0\\
53.44	0\\
53.45	0\\
53.46	0\\
53.47	0\\
53.48	0\\
53.49	0\\
53.5	0\\
53.51	0\\
53.52	0\\
53.53	0\\
53.54	0\\
53.55	0\\
53.56	0\\
53.57	0\\
53.58	0\\
53.59	0\\
53.6	0\\
53.61	0\\
53.62	0\\
53.63	0\\
53.64	0\\
53.65	0\\
53.66	0\\
53.67	0\\
53.68	0\\
53.69	0\\
53.7	0\\
53.71	0\\
53.72	0\\
53.73	0\\
53.74	0\\
53.75	0\\
53.76	0\\
53.77	0\\
53.78	0\\
53.79	0\\
53.8	0\\
53.81	0\\
53.82	0\\
53.83	0\\
53.84	0\\
53.85	0\\
53.86	0\\
53.87	0\\
53.88	0\\
53.89	0\\
53.9	0\\
53.91	0\\
53.92	0\\
53.93	0\\
53.94	0\\
53.95	0\\
53.96	0\\
53.97	0\\
53.98	0\\
53.99	0\\
54	0\\
54.01	0\\
54.02	0\\
54.03	0\\
54.04	0\\
54.05	0\\
54.06	0\\
54.07	0\\
54.08	0\\
54.09	0\\
54.1	0\\
54.11	0\\
54.12	0\\
54.13	0\\
54.14	0\\
54.15	0\\
54.16	0\\
54.17	0\\
54.18	0\\
54.19	0\\
54.2	0\\
54.21	0\\
54.22	0\\
54.23	0\\
54.24	0\\
54.25	0\\
54.26	0\\
54.27	0\\
54.28	0\\
54.29	0\\
54.3	0\\
54.31	0\\
54.32	0\\
54.33	0\\
54.34	0\\
54.35	0\\
54.36	0\\
54.37	0\\
54.38	0\\
54.39	0\\
54.4	0\\
54.41	0\\
54.42	0\\
54.43	0\\
54.44	0\\
54.45	0\\
54.46	0\\
54.47	0\\
54.48	0\\
54.49	0\\
54.5	0\\
54.51	0\\
54.52	0\\
54.53	0\\
54.54	0\\
54.55	0\\
54.56	0\\
54.57	0\\
54.58	0\\
54.59	0\\
54.6	0\\
54.61	0\\
54.62	0\\
54.63	0\\
54.64	0\\
54.65	0\\
54.66	0\\
54.67	0\\
54.68	0\\
54.69	0\\
54.7	0\\
54.71	0\\
54.72	0\\
54.73	0\\
54.74	0\\
54.75	0\\
54.76	0\\
54.77	0\\
54.78	0\\
54.79	0\\
54.8	0\\
54.81	0\\
54.82	0\\
54.83	0\\
54.84	0\\
54.85	0\\
54.86	0\\
54.87	0\\
54.88	0\\
54.89	0\\
54.9	0\\
54.91	0\\
54.92	0\\
54.93	0\\
54.94	0\\
54.95	0\\
54.96	0\\
54.97	0\\
54.98	0\\
54.99	0\\
55	0\\
55.01	0\\
55.02	0\\
55.03	0\\
55.04	0\\
55.05	0\\
55.06	0\\
55.07	0\\
55.08	0\\
55.09	0\\
55.1	0\\
55.11	0\\
55.12	0\\
55.13	0\\
55.14	0\\
55.15	0\\
55.16	0\\
55.17	0\\
55.18	0\\
55.19	0\\
55.2	0\\
55.21	0\\
55.22	0\\
55.23	0\\
55.24	0\\
55.25	0\\
55.26	0\\
55.27	0\\
55.28	0\\
55.29	0\\
55.3	0\\
55.31	0\\
55.32	0\\
55.33	0\\
55.34	0\\
55.35	0\\
55.36	0\\
55.37	0\\
55.38	0\\
55.39	0\\
55.4	0\\
55.41	0\\
55.42	0\\
55.43	0\\
55.44	0\\
55.45	0\\
55.46	0\\
55.47	0\\
55.48	0\\
55.49	0\\
55.5	0\\
55.51	0\\
55.52	0\\
55.53	0\\
55.54	0\\
55.55	0\\
55.56	0\\
55.57	0\\
55.58	0\\
55.59	0\\
55.6	0\\
55.61	0\\
55.62	0\\
55.63	0\\
55.64	0\\
55.65	0\\
55.66	0\\
55.67	0\\
55.68	0\\
55.69	0\\
55.7	0\\
55.71	0\\
55.72	0\\
55.73	0\\
55.74	0\\
55.75	0\\
55.76	0\\
55.77	0\\
55.78	0\\
55.79	0\\
55.8	0\\
55.81	0\\
55.82	0\\
55.83	0\\
55.84	0\\
55.85	0\\
55.86	0\\
55.87	0\\
55.88	0\\
55.89	0\\
55.9	0\\
55.91	0\\
55.92	0\\
55.93	0\\
55.94	0\\
55.95	0\\
55.96	0\\
55.97	0\\
55.98	0\\
55.99	0\\
56	0\\
56.01	0\\
56.02	0\\
56.03	0\\
56.04	0\\
56.05	0\\
56.06	0\\
56.07	0\\
56.08	0\\
56.09	0\\
56.1	0\\
56.11	0\\
56.12	0\\
56.13	0\\
56.14	0\\
56.15	0\\
56.16	0\\
56.17	0\\
56.18	0\\
56.19	0\\
56.2	0\\
56.21	0\\
56.22	0\\
56.23	0\\
56.24	0\\
56.25	0\\
56.26	0\\
56.27	0\\
56.28	0\\
56.29	0\\
56.3	0\\
56.31	0\\
56.32	0\\
56.33	0\\
56.34	0\\
56.35	0\\
56.36	0\\
56.37	0\\
56.38	0\\
56.39	0\\
56.4	0\\
56.41	0\\
56.42	0\\
56.43	0\\
56.44	0\\
56.45	0\\
56.46	0\\
56.47	0\\
56.48	0\\
56.49	0\\
56.5	0\\
56.51	0\\
56.52	0\\
56.53	0\\
56.54	0\\
56.55	0\\
56.56	0\\
56.57	0\\
56.58	0\\
56.59	0\\
56.6	0\\
56.61	0\\
56.62	0\\
56.63	0\\
56.64	0\\
56.65	0\\
56.66	0\\
56.67	0\\
56.68	0\\
56.69	0\\
56.7	0\\
56.71	0\\
56.72	0\\
56.73	0\\
56.74	0\\
56.75	0\\
56.76	0\\
56.77	0\\
56.78	0\\
56.79	0\\
56.8	0\\
56.81	0\\
56.82	0\\
56.83	0\\
56.84	0\\
56.85	0\\
56.86	0\\
56.87	0\\
56.88	0\\
56.89	0\\
56.9	0\\
56.91	0\\
56.92	0\\
56.93	0\\
56.94	0\\
56.95	0\\
56.96	0\\
56.97	0\\
56.98	0\\
56.99	0\\
57	0\\
57.01	0\\
57.02	0\\
57.03	0\\
57.04	0\\
57.05	0\\
57.06	0\\
57.07	0\\
57.08	0\\
57.09	0\\
57.1	0\\
57.11	0\\
57.12	0\\
57.13	0\\
57.14	0\\
57.15	0\\
57.16	0\\
57.17	0\\
57.18	0\\
57.19	0\\
57.2	0\\
57.21	0\\
57.22	0\\
57.23	0\\
57.24	0\\
57.25	0\\
57.26	0\\
57.27	0\\
57.28	0\\
57.29	0\\
57.3	0\\
57.31	0\\
57.32	0\\
57.33	0\\
57.34	0\\
57.35	0\\
57.36	0\\
57.37	0\\
57.38	0\\
57.39	0\\
57.4	0\\
57.41	0\\
57.42	0\\
57.43	0\\
57.44	0\\
57.45	0\\
57.46	0\\
57.47	0\\
57.48	0\\
57.49	0\\
57.5	0\\
57.51	0\\
57.52	0\\
57.53	0\\
57.54	0\\
57.55	0\\
57.56	0\\
57.57	0\\
57.58	0\\
57.59	0\\
57.6	0\\
57.61	0\\
57.62	0\\
57.63	0\\
57.64	0\\
57.65	0\\
57.66	0\\
57.67	0\\
57.68	0\\
57.69	0\\
57.7	0\\
57.71	0\\
57.72	0\\
57.73	0\\
57.74	0\\
57.75	0\\
57.76	0\\
57.77	0\\
57.78	0\\
57.79	0\\
57.8	0\\
57.81	0\\
57.82	0\\
57.83	0\\
57.84	0\\
57.85	0\\
57.86	0\\
57.87	0\\
57.88	0\\
57.89	0\\
57.9	0\\
57.91	0\\
57.92	0\\
57.93	0\\
57.94	0\\
57.95	0\\
57.96	0\\
57.97	0\\
57.98	0\\
57.99	0\\
58	0\\
58.01	0\\
58.02	0\\
58.03	0\\
58.04	0\\
58.05	0\\
58.06	0\\
58.07	0\\
58.08	0\\
58.09	0\\
58.1	0\\
58.11	0\\
58.12	0\\
58.13	0\\
58.14	0\\
58.15	0\\
58.16	0\\
58.17	0\\
58.18	0\\
58.19	0\\
58.2	0\\
58.21	0\\
58.22	0\\
58.23	0\\
58.24	0\\
58.25	0\\
58.26	0\\
58.27	0\\
58.28	0\\
58.29	0\\
58.3	0\\
58.31	0\\
58.32	0\\
58.33	0\\
58.34	0\\
58.35	0\\
58.36	0\\
58.37	0\\
58.38	0\\
58.39	0\\
58.4	0\\
58.41	0\\
58.42	0\\
58.43	0\\
58.44	0\\
58.45	0\\
58.46	0\\
58.47	0\\
58.48	0\\
58.49	0\\
58.5	0\\
58.51	0\\
58.52	0\\
58.53	0\\
58.54	0\\
58.55	0\\
58.56	0\\
58.57	0\\
58.58	0\\
58.59	0\\
58.6	0\\
58.61	0\\
58.62	0\\
58.63	0\\
58.64	0\\
58.65	0\\
58.66	0\\
58.67	0\\
58.68	0\\
58.69	0\\
58.7	0\\
58.71	0\\
58.72	0\\
58.73	0\\
58.74	0\\
58.75	0\\
58.76	0\\
58.77	0\\
58.78	0\\
58.79	0\\
58.8	0\\
58.81	0\\
58.82	0\\
58.83	0\\
58.84	0\\
58.85	0\\
58.86	0\\
58.87	0\\
58.88	0\\
58.89	0\\
58.9	0\\
58.91	0\\
58.92	0\\
58.93	0\\
58.94	0\\
58.95	0\\
58.96	0\\
58.97	0\\
58.98	0\\
58.99	0\\
59	0\\
59.01	0\\
59.02	0\\
59.03	0\\
59.04	0\\
59.05	0\\
59.06	0\\
59.07	0\\
59.08	0\\
59.09	0\\
59.1	0\\
59.11	0\\
59.12	0\\
59.13	0\\
59.14	0\\
59.15	0\\
59.16	0\\
59.17	0\\
59.18	0\\
59.19	0\\
59.2	0\\
59.21	0\\
59.22	0\\
59.23	0\\
59.24	0\\
59.25	0\\
59.26	0\\
59.27	0\\
59.28	0\\
59.29	0\\
59.3	0\\
59.31	0\\
59.32	0\\
59.33	0\\
59.34	0\\
59.35	0\\
59.36	0\\
59.37	0\\
59.38	0\\
59.39	0\\
59.4	0\\
59.41	0\\
59.42	0\\
59.43	0\\
59.44	0\\
59.45	0\\
59.46	0\\
59.47	0\\
59.48	0\\
59.49	0\\
59.5	0\\
59.51	0\\
59.52	0\\
59.53	0\\
59.54	0\\
59.55	0\\
59.56	0\\
59.57	0\\
59.58	0\\
59.59	0\\
59.6	0\\
59.61	0\\
59.62	0\\
59.63	0\\
59.64	0\\
59.65	0\\
59.66	0\\
59.67	0\\
59.68	0\\
59.69	0\\
59.7	0\\
59.71	0\\
59.72	0\\
59.73	0\\
59.74	0\\
59.75	0\\
59.76	0\\
59.77	0\\
59.78	0\\
59.79	0\\
59.8	0\\
59.81	0\\
59.82	0\\
59.83	0\\
59.84	0\\
59.85	0\\
59.86	0\\
59.87	0\\
59.88	0\\
59.89	0\\
59.9	0\\
59.91	0\\
59.92	0\\
59.93	0\\
59.94	0\\
59.95	0\\
59.96	0\\
59.97	0\\
59.98	0\\
59.99	0\\
60	0\\
60.01	0\\
60.02	0\\
60.03	0\\
60.04	0\\
60.05	0\\
60.06	0\\
60.07	0\\
60.08	0\\
60.09	0\\
60.1	0\\
60.11	0\\
60.12	0\\
60.13	0\\
60.14	0\\
60.15	0\\
60.16	0\\
60.17	0\\
60.18	0\\
60.19	0\\
60.2	0\\
60.21	0\\
60.22	0\\
60.23	0\\
60.24	0\\
60.25	0\\
60.26	0\\
60.27	0\\
60.28	0\\
60.29	0\\
60.3	0\\
60.31	0\\
60.32	0\\
60.33	0\\
60.34	0\\
60.35	0\\
60.36	0\\
60.37	0\\
60.38	0\\
60.39	0\\
60.4	0\\
60.41	0\\
60.42	0\\
60.43	0\\
60.44	0\\
60.45	0\\
60.46	0\\
60.47	0\\
60.48	0\\
60.49	0\\
60.5	0\\
60.51	0\\
60.52	0\\
60.53	0\\
60.54	0\\
60.55	0\\
60.56	1.57822659636181e-06\\
60.57	3.32362392112609e-06\\
60.58	5.07013025857761e-06\\
60.59	6.81774671597302e-06\\
60.6	8.56647440161326e-06\\
60.61	1.03163144248436e-05\\
60.62	1.20672678960328e-05\\
60.63	1.38193359265799e-05\\
60.64	1.55725196289008e-05\\
60.65	1.73268201164208e-05\\
60.66	1.90822385035712e-05\\
60.67	2.08387759057724e-05\\
60.68	2.25964334394368e-05\\
60.69	2.4355212221959e-05\\
60.7	2.61151133717046e-05\\
60.71	2.7876138007997e-05\\
60.72	2.96382872511242e-05\\
60.73	3.14015622223142e-05\\
60.74	3.31659640437389e-05\\
60.75	3.49314938384963e-05\\
60.76	3.66981527306076e-05\\
60.77	3.84659418450062e-05\\
60.78	4.02348623075241e-05\\
60.79	4.2004915244892e-05\\
60.8	4.37761017847219e-05\\
60.81	4.55484230554998e-05\\
60.82	4.73218801865759e-05\\
60.83	4.90964743081537e-05\\
60.84	5.08722065512834e-05\\
60.85	5.26490780478478e-05\\
60.86	5.44270899305484e-05\\
60.87	5.62062433329055e-05\\
60.88	5.79865393892341e-05\\
60.89	5.97679792346435e-05\\
60.9	6.15505640050168e-05\\
60.91	6.33342948370107e-05\\
60.92	6.51191728680348e-05\\
60.93	6.69051992362377e-05\\
60.94	6.86923750805035e-05\\
60.95	7.04807015404312e-05\\
60.96	7.22701797563341e-05\\
60.97	7.40608108692094e-05\\
60.98	7.5852596020741e-05\\
60.99	7.76455363532755e-05\\
61	7.94396330098188e-05\\
61.01	8.12348871340116e-05\\
61.02	8.30312998701226e-05\\
61.03	8.4828872363038e-05\\
61.04	8.66276057582341e-05\\
61.05	8.84275012017731e-05\\
61.06	9.02285598402865e-05\\
61.07	9.20307828209573e-05\\
61.08	9.38341712915097e-05\\
61.09	9.56387264001886e-05\\
61.1	9.74444492957451e-05\\
61.11	9.92513411274196e-05\\
61.12	0.000101059403044931\\
61.13	0.000102868636198454\\
61.14	0.000104679041738602\\
61.15	0.000106490620816416\\
61.16	0.000108303374583341\\
61.17	0.00011011730419121\\
61.18	0.000111932410792229\\
61.19	0.000113748695538953\\
61.2	0.000115566159584274\\
61.21	0.0001173848040814\\
61.22	0.00011920463018383\\
61.23	0.000121025639045345\\
61.24	0.00012284783181998\\
61.25	0.000124671209662006\\
61.26	0.000126495773725914\\
61.27	0.000128321525166387\\
61.28	0.000130148465138286\\
61.29	0.000131976594796621\\
61.3	0.000133805915296537\\
61.31	0.000135636427793288\\
61.32	0.000137468133442214\\
61.33	0.000139301033398723\\
61.34	0.000141135128818262\\
61.35	0.000142970420856297\\
61.36	0.000144806910668288\\
61.37	0.000146644599409669\\
61.38	0.000148483488235814\\
61.39	0.000150323578302022\\
61.4	0.000152164870763485\\
61.41	0.000154007366775271\\
61.42	0.000155851067492285\\
61.43	0.000157695974069252\\
61.44	0.00015954208766069\\
61.45	0.000161389409420881\\
61.46	0.000163237940503838\\
61.47	0.000165087682063289\\
61.48	0.000166938635252637\\
61.49	0.000168790801224936\\
61.5	0.000170644181132863\\
61.51	0.000172498776128686\\
61.52	0.000174354587364232\\
61.53	0.000176211615990862\\
61.54	0.000178069863159432\\
61.55	0.00017992933002027\\
61.56	0.000181790017723137\\
61.57	0.000183651927417199\\
61.58	0.000185515060250992\\
61.59	0.000187379417372385\\
61.6	0.000189244999928555\\
61.61	0.00019111180906594\\
61.62	0.000192979845930218\\
61.63	0.00019484911166626\\
61.64	0.0001967196074181\\
61.65	0.000198591334328892\\
61.66	0.000200464293540883\\
61.67	0.000202338486195369\\
61.68	0.000204213913432654\\
61.69	0.000206090576392013\\
61.7	0.00020796847621166\\
61.71	0.000209847614028695\\
61.72	0.000211727990979078\\
61.73	0.000213609608197572\\
61.74	0.000215492466817714\\
61.75	0.000217376567971769\\
61.76	0.000219261912790682\\
61.77	0.000221148502404039\\
61.78	0.000223036337940026\\
61.79	0.000224925420525372\\
61.8	0.000226815751285317\\
61.81	0.000228707331343558\\
61.82	0.000230600161822209\\
61.83	0.000232494243841742\\
61.84	0.000234389578520944\\
61.85	0.000236286166976876\\
61.86	0.000238184010324809\\
61.87	0.000240083109678189\\
61.88	0.000241983466148572\\
61.89	0.000243885080845579\\
61.9	0.000245787954876842\\
61.91	0.00024769208968017\\
61.92	0.000249597486707175\\
61.93	0.000251504147411881\\
61.94	0.000253412073250726\\
61.95	0.000255321265682566\\
61.96	0.000257231726168684\\
61.97	0.000259143456172787\\
61.98	0.00026105645716102\\
61.99	0.000262970730601962\\
62	0.000264886277966638\\
62.01	0.000266803100728522\\
62.02	0.000268721200363538\\
62.03	0.000270640578350068\\
62.04	0.000272561236168957\\
62.05	0.000274483175303514\\
62.06	0.000276406397239522\\
62.07	0.000278330903465244\\
62.08	0.000280256695471416\\
62.09	0.000282183774751265\\
62.1	0.000284112142800511\\
62.11	0.000286041801117364\\
62.12	0.000287972751202539\\
62.13	0.000289904994559251\\
62.14	0.000291838532693236\\
62.15	0.00029377336711273\\
62.16	0.0002957094993285\\
62.17	0.000297646930853834\\
62.18	0.000299585663204548\\
62.19	0.000301525697898992\\
62.2	0.000303467036458057\\
62.21	0.000305409680405176\\
62.22	0.000307353631266334\\
62.23	0.000309298890570062\\
62.24	0.000311245459847457\\
62.25	0.000313193340632175\\
62.26	0.000315142534460442\\
62.27	0.000317093042871055\\
62.28	0.000319044867405388\\
62.29	0.000320998009607398\\
62.3	0.000322952471023628\\
62.31	0.000324908253203218\\
62.32	0.000326865357697898\\
62.33	0.000328823786062003\\
62.34	0.000330783539852473\\
62.35	0.000332744620628864\\
62.36	0.000334707029953342\\
62.37	0.000336670769390694\\
62.38	0.000338635840508337\\
62.39	0.000340602244876316\\
62.4	0.000342569984067308\\
62.41	0.000344539059656635\\
62.42	0.000346509473222262\\
62.43	0.000348481226344805\\
62.44	0.00035045432060753\\
62.45	0.000352428757596366\\
62.46	0.000354404538899903\\
62.47	0.000356381666109403\\
62.48	0.0003583601408188\\
62.49	0.000360339964624701\\
62.5	0.000362321139126404\\
62.51	0.000364303665925891\\
62.52	0.000366287546627833\\
62.53	0.000368272782839602\\
62.54	0.000370259376171272\\
62.55	0.000372247328235622\\
62.56	0.000374236640648139\\
62.57	0.000376227315027029\\
62.58	0.000378219352993221\\
62.59	0.000380212756170364\\
62.6	0.000382207526184838\\
62.61	0.000384203664665756\\
62.62	0.000386201173244973\\
62.63	0.000388200053557088\\
62.64	0.000390200307239442\\
62.65	0.000392201935932136\\
62.66	0.000394204941278023\\
62.67	0.000396209324922719\\
62.68	0.000398215088514606\\
62.69	0.000400222233704839\\
62.7	0.000402230762147342\\
62.71	0.000404240675498827\\
62.72	0.000406251975418784\\
62.73	0.000408264663569493\\
62.74	0.000410278741616026\\
62.75	0.000412294211226256\\
62.76	0.000414311074070854\\
62.77	0.000416329331823298\\
62.78	0.000418348986159878\\
62.79	0.000420370038759697\\
62.8	0.000422392491304675\\
62.81	0.000424416345479559\\
62.82	0.000426441602971922\\
62.83	0.000428468265472166\\
62.84	0.000430496334673534\\
62.85	0.000432525812272103\\
62.86	0.000434556699966799\\
62.87	0.000436588999459392\\
62.88	0.000438622712454508\\
62.89	0.000440657840659626\\
62.9	0.000442694385785088\\
62.91	0.000444732349544098\\
62.92	0.000446771733652731\\
62.93	0.000448812539829927\\
62.94	0.000450854769797511\\
62.95	0.000452898425280183\\
62.96	0.000454943508005526\\
62.97	0.00045699001970401\\
62.98	0.000459037962108999\\
62.99	0.00046108733695675\\
63	0.000463138145986419\\
63.01	0.000465190390940062\\
63.02	0.000467244073562641\\
63.03	0.000469299195602027\\
63.04	0.000471355758809005\\
63.05	0.000473413764937272\\
63.06	0.000475473215743448\\
63.07	0.000477534112987073\\
63.08	0.000479596458430615\\
63.09	0.000481660253839469\\
63.1	0.000483725500981962\\
63.11	0.000485792201629358\\
63.12	0.000487860357555856\\
63.13	0.000489929970538601\\
63.14	0.000492001042357679\\
63.15	0.000494073574796122\\
63.16	0.000496147569639914\\
63.17	0.000498223028677993\\
63.18	0.000500299953702248\\
63.19	0.000502378346507532\\
63.2	0.00050445820889165\\
63.21	0.000506539542655379\\
63.22	0.000508622349602458\\
63.23	0.000510706631539591\\
63.24	0.000512792390276455\\
63.25	0.000514879627625697\\
63.26	0.000516968345402945\\
63.27	0.000519058545426797\\
63.28	0.000521150229518829\\
63.29	0.000523243399503603\\
63.3	0.000525338057208658\\
63.31	0.000527434204464518\\
63.32	0.000529531843104697\\
63.33	0.00053163097496569\\
63.34	0.000533731601886984\\
63.35	0.000535833725711057\\
63.36	0.000537937348283373\\
63.37	0.000540042471452395\\
63.38	0.000542149097069577\\
63.39	0.000544257226989368\\
63.4	0.000546366863069209\\
63.41	0.000548478007169544\\
63.42	0.000550590661153809\\
63.43	0.000552704826888437\\
63.44	0.000554820506242861\\
63.45	0.000556937701089513\\
63.46	0.000559056413303824\\
63.47	0.000561176644764221\\
63.48	0.000563298397352131\\
63.49	0.000565421672951979\\
63.5	0.000567546473451187\\
63.51	0.000569672800740179\\
63.52	0.000571800656712373\\
63.53	0.000573930043264188\\
63.54	0.000576060962295032\\
63.55	0.000578193415707313\\
63.56	0.000580327405406592\\
63.57	0.000582462933301751\\
63.58	0.000584600001304995\\
63.59	0.000586738611331863\\
63.6	0.00058887876530124\\
63.61	0.000591020465135348\\
63.62	0.000593163712759772\\
63.63	0.000595308510103455\\
63.64	0.00059745485909871\\
63.65	0.000599602761681219\\
63.66	0.000601752219790055\\
63.67	0.000603903235367673\\
63.68	0.000606055810359923\\
63.69	0.000608209946716062\\
63.7	0.000610365646388753\\
63.71	0.000612522911334082\\
63.72	0.000614681743511553\\
63.73	0.000616842144884104\\
63.74	0.000619004117418109\\
63.75	0.000621167663083393\\
63.76	0.000623332783853223\\
63.77	0.000625499481704334\\
63.78	0.000627667758616928\\
63.79	0.000629837616574678\\
63.8	0.000632009057564734\\
63.81	0.000634182083577742\\
63.82	0.000636356696607839\\
63.83	0.000638532898652663\\
63.84	0.000640710691713366\\
63.85	0.000642890077794615\\
63.86	0.0006450710589046\\
63.87	0.000647253637055048\\
63.88	0.00064943781426122\\
63.89	0.000651623592541926\\
63.9	0.000653810973919526\\
63.91	0.000655999960419946\\
63.92	0.000658190554072676\\
63.93	0.000660382756910785\\
63.94	0.000662576570970921\\
63.95	0.000664771998293328\\
63.96	0.000666969040921846\\
63.97	0.000669167700903917\\
63.98	0.000671367980290601\\
63.99	0.000673569881136576\\
64	0.000675773405500145\\
64.01	0.00067797855544325\\
64.02	0.000680185333031477\\
64.03	0.000682393740334061\\
64.04	0.000684603779423892\\
64.05	0.000686815452377532\\
64.06	0.000689028761275211\\
64.07	0.000691243708200839\\
64.08	0.000693460295242018\\
64.09	0.000695678524490043\\
64.1	0.000697898398039919\\
64.11	0.000700119917990353\\
64.12	0.000702343086443777\\
64.13	0.000704567905506347\\
64.14	0.000706794377287955\\
64.15	0.000709022503902236\\
64.16	0.000711252287466574\\
64.17	0.000713483730102112\\
64.18	0.000715716833933754\\
64.19	0.000717951601090179\\
64.2	0.000720188033703852\\
64.21	0.000722426133911019\\
64.22	0.000724665903851729\\
64.23	0.000726907345669832\\
64.24	0.000729150461512996\\
64.25	0.000731395253532701\\
64.26	0.000733641723884258\\
64.27	0.000735889874726819\\
64.28	0.000738139708223375\\
64.29	0.000740391226540772\\
64.3	0.000742644431849713\\
64.31	0.000744899326324773\\
64.32	0.000747155912144398\\
64.33	0.000749414191490924\\
64.34	0.000751674166550573\\
64.35	0.00075393583951347\\
64.36	0.000756199212573649\\
64.37	0.000758464287929059\\
64.38	0.000760731067781576\\
64.39	0.000762999554337004\\
64.4	0.000765269749805092\\
64.41	0.000767541656399532\\
64.42	0.00076981527633798\\
64.43	0.000772090611842052\\
64.44	0.000774367665137339\\
64.45	0.000776646438453411\\
64.46	0.000778926934023833\\
64.47	0.000781209154086165\\
64.48	0.000783493100881973\\
64.49	0.000785778776656837\\
64.5	0.000788066183660362\\
64.51	0.000790355324146182\\
64.52	0.00079264620037197\\
64.53	0.000794938814599449\\
64.54	0.000797233169094396\\
64.55	0.000799529266126652\\
64.56	0.000801827107970134\\
64.57	0.00080412669690284\\
64.58	0.000806428035206853\\
64.59	0.000808731125168358\\
64.6	0.00081103596907765\\
64.61	0.000813342569229127\\
64.62	0.000815650927921324\\
64.63	0.000817961047456899\\
64.64	0.000820272930142654\\
64.65	0.000822586578289541\\
64.66	0.00082490199421267\\
64.67	0.00082721918023131\\
64.68	0.000829538138668914\\
64.69	0.000831858871853112\\
64.7	0.000834181382115728\\
64.71	0.000836505671792789\\
64.72	0.000838831743224529\\
64.73	0.000841159598755394\\
64.74	0.000843489240734067\\
64.75	0.00084582067151346\\
64.76	0.000848153893450727\\
64.77	0.000850488908907279\\
64.78	0.000852825720248786\\
64.79	0.000855164329845188\\
64.8	0.000857504740070705\\
64.81	0.000859846953303843\\
64.82	0.000862190971927405\\
64.83	0.000864536798328503\\
64.84	0.000866884434898553\\
64.85	0.000869233884033302\\
64.86	0.000871585148132827\\
64.87	0.00087393822960154\\
64.88	0.000876293130848214\\
64.89	0.000878649854285967\\
64.9	0.000881008402332292\\
64.91	0.000883368777409056\\
64.92	0.000885730981942515\\
64.93	0.000888095018363309\\
64.94	0.000890460889106489\\
64.95	0.000892828596611515\\
64.96	0.000895198143322271\\
64.97	0.000897569531687063\\
64.98	0.000899942764158647\\
64.99	0.000902317843194216\\
65	0.000904694771255429\\
65.01	0.000907073550808404\\
65.02	0.000909454184323737\\
65.03	0.000911836674276509\\
65.04	0.00091422102314629\\
65.05	0.000916607233417157\\
65.06	0.000918995307577698\\
65.07	0.000921385248121021\\
65.08	0.000923777057544761\\
65.09	0.000926170738351096\\
65.1	0.000928566293046749\\
65.11	0.000930963724142999\\
65.12	0.000933363034155699\\
65.13	0.000935764225605269\\
65.14	0.000938167301016719\\
65.15	0.000940572262919652\\
65.16	0.000942979113848275\\
65.17	0.000945387856341403\\
65.18	0.000947798492942482\\
65.19	0.00095021102619958\\
65.2	0.000952625458665413\\
65.21	0.000955041792897345\\
65.22	0.000957460031457393\\
65.23	0.000959880176912252\\
65.24	0.000962302231833287\\
65.25	0.000964726198796555\\
65.26	0.00096715208038281\\
65.27	0.00096957987917751\\
65.28	0.000972009597770829\\
65.29	0.000974441238757662\\
65.3	0.000976874804737647\\
65.31	0.000979310298315161\\
65.32	0.00098174772209933\\
65.33	0.000984187078704051\\
65.34	0.000986628370747984\\
65.35	0.000989071600854579\\
65.36	0.000991516771652073\\
65.37	0.000993963885773504\\
65.38	0.000996412945856718\\
65.39	0.000998863954544384\\
65.4	0.001001316914484\\
65.41	0.00100377182832789\\
65.42	0.00100622869873326\\
65.43	0.00100868752836213\\
65.44	0.00101114831988141\\
65.45	0.00101361107596289\\
65.46	0.00101607579928325\\
65.47	0.00101854249252404\\
65.48	0.00102101115837175\\
65.49	0.00102348179951775\\
65.5	0.00102595441865837\\
65.51	0.00102842901849485\\
65.52	0.00103090560173337\\
65.53	0.00103338417108509\\
65.54	0.00103586472926612\\
65.55	0.00103834727899752\\
65.56	0.00104083182300538\\
65.57	0.00104331836402074\\
65.58	0.00104580690477965\\
65.59	0.0010482974480232\\
65.6	0.00105078999649747\\
65.61	0.00105328455295357\\
65.62	0.00105578112014768\\
65.63	0.00105827970084101\\
65.64	0.00106078029779981\\
65.65	0.00106328291379545\\
65.66	0.00106578755160433\\
65.67	0.00106829421400797\\
65.68	0.00107080290379297\\
65.69	0.00107331362375106\\
65.7	0.00107582637667907\\
65.71	0.00107834116537896\\
65.72	0.00108085799265783\\
65.73	0.00108337686132792\\
65.74	0.00108589777420665\\
65.75	0.00108842073411659\\
65.76	0.00109094574388548\\
65.77	0.00109347280634626\\
65.78	0.00109600192433705\\
65.79	0.0010985331007012\\
65.8	0.00110106633828726\\
65.81	0.001103601639949\\
65.82	0.00110613900854543\\
65.83	0.00110867844694081\\
65.84	0.00111121995800464\\
65.85	0.00111376354461171\\
65.86	0.00111630920964206\\
65.87	0.00111885695598102\\
65.88	0.00112140678651922\\
65.89	0.00112395870415258\\
65.9	0.00112651271178234\\
65.91	0.00112906881231507\\
65.92	0.00113162700866266\\
65.93	0.00113418730374234\\
65.94	0.0011367497004767\\
65.95	0.0011393142017937\\
65.96	0.00114188081062664\\
65.97	0.00114444952991423\\
65.98	0.00114702036260055\\
65.99	0.0011495933116351\\
66	0.00115216837997278\\
66.01	0.0011547455705739\\
66.02	0.0011573248864042\\
66.03	0.00115990633043488\\
66.04	0.00116248990564256\\
66.05	0.00116507561500933\\
66.06	0.00116766346152276\\
66.07	0.00117025344817586\\
66.08	0.00117284557796717\\
66.09	0.0011754398539007\\
66.1	0.00117803627898597\\
66.11	0.00118063485623801\\
66.12	0.00118323558867738\\
66.13	0.00118583847933018\\
66.14	0.00118844353122806\\
66.15	0.00119105074740818\\
66.16	0.00119366013091331\\
66.17	0.00119627168479177\\
66.18	0.00119888541209748\\
66.19	0.00120150131588991\\
66.2	0.00120411939923418\\
66.21	0.00120673966520098\\
66.22	0.00120936211686664\\
66.23	0.00121198675731311\\
66.24	0.00121461358962798\\
66.25	0.00121724261690449\\
66.26	0.00121987384224152\\
66.27	0.00122250726874365\\
66.28	0.00122514289952109\\
66.29	0.00122778073768976\\
66.3	0.00123042078637128\\
66.31	0.00123306304869295\\
66.32	0.0012357075277878\\
66.33	0.00123835422679455\\
66.34	0.00124100314885769\\
66.35	0.00124365429712743\\
66.36	0.00124630767475972\\
66.37	0.00124896328491626\\
66.38	0.00125162113076455\\
66.39	0.00125428121547783\\
66.4	0.00125694354223513\\
66.41	0.00125960811422128\\
66.42	0.00126227493462692\\
66.43	0.00126494400664847\\
66.44	0.00126761533348818\\
66.45	0.00127028891835415\\
66.46	0.0012729647644603\\
66.47	0.00127564287502637\\
66.48	0.00127832325327798\\
66.49	0.00128100590244662\\
66.5	0.00128369082576963\\
66.51	0.00128637802649022\\
66.52	0.00128906750785752\\
66.53	0.00129175927312652\\
66.54	0.00129445332555814\\
66.55	0.0012971496684192\\
66.56	0.00129984830498244\\
66.57	0.00130254923852652\\
66.58	0.00130525247233607\\
66.59	0.00130795800970161\\
66.6	0.00131066585391967\\
66.61	0.00131337600829269\\
66.62	0.00131608847612912\\
66.63	0.00131880326074336\\
66.64	0.0013215203654558\\
66.65	0.00132423979359283\\
66.66	0.00132696154848684\\
66.67	0.00132968563347621\\
66.68	0.00133241205190535\\
66.69	0.00133514080712468\\
66.7	0.00133787190249068\\
66.71	0.00134060534136582\\
66.72	0.00134334112711866\\
66.73	0.00134607926312379\\
66.74	0.00134881975276185\\
66.75	0.00135156259941956\\
66.76	0.00135430780648972\\
66.77	0.0013570553773712\\
66.78	0.00135980531546895\\
66.79	0.00136255762419402\\
66.8	0.00136531230696358\\
66.81	0.00136806936720087\\
66.82	0.00137082880833527\\
66.83	0.00137359063380228\\
66.84	0.00137635484704351\\
66.85	0.00137912145150673\\
66.86	0.00138189045064583\\
66.87	0.00138466184792084\\
66.88	0.00138743564679796\\
66.89	0.00139021185074954\\
66.9	0.00139299046325408\\
66.91	0.00139577148779628\\
66.92	0.00139855492786698\\
66.93	0.00140134078696324\\
66.94	0.00140412906858826\\
66.95	0.00140691977625147\\
66.96	0.00140971291346847\\
66.97	0.00141250848376109\\
66.98	0.00141530649065734\\
66.99	0.00141810693769146\\
67	0.0014209098284039\\
67.01	0.00142371516634133\\
67.02	0.00142652295505664\\
67.03	0.00142933319810897\\
67.04	0.00143214589906368\\
67.05	0.00143496106149238\\
67.06	0.00143777868897291\\
67.07	0.00144059878508937\\
67.08	0.00144342135343209\\
67.09	0.0014462463975977\\
67.1	0.00144907392118903\\
67.11	0.00145190392781522\\
67.12	0.00145473642109166\\
67.13	0.00145757140463999\\
67.14	0.00146040888208815\\
67.15	0.00146324885707035\\
67.16	0.00146609133322705\\
67.17	0.00146893631420503\\
67.18	0.00147178380365732\\
67.19	0.00147463380524325\\
67.2	0.00147748632262844\\
67.21	0.0014803413594848\\
67.22	0.0014831989194905\\
67.23	0.00148605900633005\\
67.24	0.00148892162369423\\
67.25	0.0014917867752801\\
67.26	0.00149465446479104\\
67.27	0.00149752469593672\\
67.28	0.0015003974724331\\
67.29	0.00150327279800246\\
67.3	0.00150615067637336\\
67.31	0.00150903111128066\\
67.32	0.00151191410646552\\
67.33	0.0015147996656754\\
67.34	0.00151768779266408\\
67.35	0.0015205784911916\\
67.36	0.00152347176502431\\
67.37	0.00152636761793488\\
67.38	0.00152926605370226\\
67.39	0.00153216707611167\\
67.4	0.00153507068895465\\
67.41	0.00153797689602902\\
67.42	0.0015408857011389\\
67.43	0.00154379710809468\\
67.44	0.00154671112071303\\
67.45	0.00154962774281692\\
67.46	0.00155254697823558\\
67.47	0.00155546883080452\\
67.48	0.00155839330436551\\
67.49	0.00156132040276662\\
67.5	0.00156425012986213\\
67.51	0.00156718248951263\\
67.52	0.00157011748558492\\
67.53	0.00157305512195207\\
67.54	0.00157599540249339\\
67.55	0.00157893833109444\\
67.56	0.00158188391164699\\
67.57	0.00158483214804905\\
67.58	0.00158778304420485\\
67.59	0.00159073660402482\\
67.6	0.00159369283142561\\
67.61	0.00159665173033006\\
67.62	0.00159961330466722\\
67.63	0.00160257755837229\\
67.64	0.00160554449538668\\
67.65	0.00160851411965795\\
67.66	0.00161148643513981\\
67.67	0.00161446144579214\\
67.68	0.00161743915558094\\
67.69	0.00162041956847836\\
67.7	0.00162340268846264\\
67.71	0.00162638851951816\\
67.72	0.00162937706563538\\
67.73	0.00163236833081084\\
67.74	0.00163536231904719\\
67.75	0.00163835903435309\\
67.76	0.00164135848074329\\
67.77	0.00164436066223857\\
67.78	0.00164736558286573\\
67.79	0.00165037324665756\\
67.8	0.00165338365765289\\
67.81	0.0016563968198965\\
67.82	0.00165941273743914\\
67.83	0.00166243141433752\\
67.84	0.00166545285465428\\
67.85	0.00166847706245798\\
67.86	0.00167150404182309\\
67.87	0.00167453379682996\\
67.88	0.0016775663315648\\
67.89	0.00168060165011968\\
67.9	0.0016836397565925\\
67.91	0.00168668065508698\\
67.92	0.00168972434971261\\
67.93	0.00169277084458467\\
67.94	0.00169582014382418\\
67.95	0.00169887225155791\\
67.96	0.00170192717191833\\
67.97	0.00170498490904358\\
67.98	0.00170804546707749\\
67.99	0.00171110885016952\\
68	0.00171417506247474\\
68.01	0.00171724410815382\\
68.02	0.00172031599137299\\
68.03	0.00172339071630405\\
68.04	0.00172646828712427\\
68.05	0.00172954870801645\\
68.06	0.00173263198316884\\
68.07	0.0017357181167751\\
68.08	0.00173880711303434\\
68.09	0.001741898976151\\
68.1	0.00174499371033489\\
68.11	0.00174809131980114\\
68.12	0.00175119180881041\\
68.13	0.0017542951816527\\
68.14	0.00175740144262318\\
68.15	0.00176051059602213\\
68.16	0.00176362264615492\\
68.17	0.001766737597332\\
68.18	0.00176985545386882\\
68.19	0.0017729762200858\\
68.2	0.0017760999003083\\
68.21	0.00177922649886662\\
68.22	0.00178235602009586\\
68.23	0.001785488468336\\
68.24	0.00178862384793176\\
68.25	0.00179176216323261\\
68.26	0.00179490341859271\\
68.27	0.00179804761837086\\
68.28	0.00180119476693048\\
68.29	0.00180434486863953\\
68.3	0.00180749792787049\\
68.31	0.0018106539490003\\
68.32	0.00181381293641031\\
68.33	0.00181697489448623\\
68.34	0.00182013982761811\\
68.35	0.00182330774020022\\
68.36	0.00182647863663106\\
68.37	0.0018296525213133\\
68.38	0.00183282939865368\\
68.39	0.001836009273063\\
68.4	0.00183919214895606\\
68.41	0.00184237803075157\\
68.42	0.00184556692287213\\
68.43	0.00184875882974414\\
68.44	0.00185195375579775\\
68.45	0.00185515170546682\\
68.46	0.00185835268318881\\
68.47	0.00186155669340476\\
68.48	0.00186476374055921\\
68.49	0.00186797382910011\\
68.5	0.00187118696347878\\
68.51	0.00187440314814986\\
68.52	0.00187762238757118\\
68.53	0.00188084468620373\\
68.54	0.00188407004851159\\
68.55	0.00188729847896183\\
68.56	0.00189052998202447\\
68.57	0.00189376456217238\\
68.58	0.00189700222388118\\
68.59	0.00190024297162921\\
68.6	0.00190348680989742\\
68.61	0.0019067337431693\\
68.62	0.00190998377593079\\
68.63	0.00191323691267016\\
68.64	0.00191649315787802\\
68.65	0.00191975251604712\\
68.66	0.00192301499167232\\
68.67	0.00192628058925053\\
68.68	0.00192954931328053\\
68.69	0.00193282116826293\\
68.7	0.0019360961587001\\
68.71	0.00193937428909602\\
68.72	0.00194265556395619\\
68.73	0.00194593998778756\\
68.74	0.00194922756509841\\
68.75	0.00195251830039823\\
68.76	0.00195581219819765\\
68.77	0.0019591092630083\\
68.78	0.0019624094993427\\
68.79	0.00196571291171418\\
68.8	0.00196901950463676\\
68.81	0.00197232928262499\\
68.82	0.00197564225019389\\
68.83	0.00197895841185881\\
68.84	0.00198227777213529\\
68.85	0.00198560033553898\\
68.86	0.00198892610658547\\
68.87	0.00199225508979019\\
68.88	0.00199558728966829\\
68.89	0.00199892271073447\\
68.9	0.00200226135750288\\
68.91	0.00200560323448699\\
68.92	0.00200894834619942\\
68.93	0.00201229669715183\\
68.94	0.00201564829312565\\
68.95	0.00201900314014025\\
68.96	0.00202236124423031\\
68.97	0.00202572261144596\\
68.98	0.00202908724785274\\
68.99	0.00203245515953172\\
69	0.00203582635257949\\
69.01	0.00203920083310826\\
69.02	0.00204257860724584\\
69.03	0.00204595968113575\\
69.04	0.00204934406093724\\
69.05	0.00205273175282534\\
69.06	0.00205612276299092\\
69.07	0.0020595170976407\\
69.08	0.00206291476299735\\
69.09	0.00206631576529951\\
69.1	0.00206972011080183\\
69.11	0.00207312780577504\\
69.12	0.00207653885650599\\
69.13	0.0020799532692977\\
69.14	0.00208337105046941\\
69.15	0.00208679220635663\\
69.16	0.00209021674331116\\
69.17	0.00209364466770122\\
69.18	0.0020970759859114\\
69.19	0.00210051070434277\\
69.2	0.00210394882941294\\
69.21	0.00210739036755606\\
69.22	0.00211083532522291\\
69.23	0.00211428370888094\\
69.24	0.00211773552501433\\
69.25	0.00212119078012401\\
69.26	0.00212464948072775\\
69.27	0.00212811163336019\\
69.28	0.0021315772445729\\
69.29	0.00213504632093443\\
69.3	0.00213851886903035\\
69.31	0.00214199489546333\\
69.32	0.00214547440685316\\
69.33	0.00214895740983683\\
69.34	0.00215244391106856\\
69.35	0.00215593391721987\\
69.36	0.00215942743497963\\
69.37	0.00216292447105411\\
69.38	0.00216642503216703\\
69.39	0.00216992912505962\\
69.4	0.00217343675649067\\
69.41	0.0021769479332366\\
69.42	0.00218046266209148\\
69.43	0.00218398094986712\\
69.44	0.00218750280339311\\
69.45	0.00219102822951687\\
69.46	0.0021945572351037\\
69.47	0.00219808982703687\\
69.48	0.00220162601221764\\
69.49	0.00220516579756531\\
69.5	0.00220870919001733\\
69.51	0.00221225619652928\\
69.52	0.00221580682407499\\
69.53	0.00221936107964656\\
69.54	0.00222291897025444\\
69.55	0.00222648050292747\\
69.56	0.00223004568471293\\
69.57	0.00223361452267663\\
69.58	0.00223718702390295\\
69.59	0.00224076319549487\\
69.6	0.00224434304457408\\
69.61	0.002247926578281\\
69.62	0.00225151380377484\\
69.63	0.0022551047282337\\
69.64	0.00225869935885456\\
69.65	0.00226229770285341\\
69.66	0.00226589976746525\\
69.67	0.0022695055599442\\
69.68	0.0022731150875635\\
69.69	0.00227672835761565\\
69.7	0.00228034537741239\\
69.71	0.00228396615428482\\
69.72	0.0022875906955834\\
69.73	0.0022912190086781\\
69.74	0.00229485110095836\\
69.75	0.00229848697983322\\
69.76	0.00230212665273137\\
69.77	0.00230577012710119\\
69.78	0.00230941741041081\\
69.79	0.00231306851014823\\
69.8	0.00231672343382129\\
69.81	0.00232038218895781\\
69.82	0.00232404478310562\\
69.83	0.00232771122383262\\
69.84	0.00233138151872685\\
69.85	0.00233505567539658\\
69.86	0.00233873370147031\\
69.87	0.0023424156045969\\
69.88	0.00234610139244559\\
69.89	0.00234979107270609\\
69.9	0.00235348465308863\\
69.91	0.00235718214132403\\
69.92	0.00236088354516377\\
69.93	0.00236458887238005\\
69.94	0.00236829752194856\\
69.95	0.00237200838171067\\
69.96	0.00237572145407267\\
69.97	0.0023794367414451\\
69.98	0.00238315424624278\\
69.99	0.0023868739708848\\
70	0.00239059591779454\\
70.01	0.00239432008939969\\
70.02	0.00239804648813223\\
70.03	0.00240177511642847\\
70.04	0.00240550597672905\\
70.05	0.00240923907147897\\
70.06	0.00241297440312756\\
70.07	0.00241671197412853\\
70.08	0.00242045178693995\\
70.09	0.00242419384402431\\
70.1	0.00242793814784846\\
70.11	0.0024316847008837\\
70.12	0.00243543350560571\\
70.13	0.00243918456449465\\
70.14	0.00244293788003507\\
70.15	0.00244669345471603\\
70.16	0.00245045129103103\\
70.17	0.00245421139147805\\
70.18	0.00245797375855956\\
70.19	0.00246173839478254\\
70.2	0.0024655053026585\\
70.21	0.00246927448470344\\
70.22	0.00247304594343793\\
70.23	0.00247681968138707\\
70.24	0.00248059570108053\\
70.25	0.00248437400505257\\
70.26	0.00248815459584201\\
70.27	0.00249193747599229\\
70.28	0.00249572264805145\\
70.29	0.00249951011457216\\
70.3	0.00250329987811171\\
70.31	0.00250709194123206\\
70.32	0.00251088630649982\\
70.33	0.00251468297648627\\
70.34	0.00251848195376738\\
70.35	0.00252228324092383\\
70.36	0.00252608684054098\\
70.37	0.00252989275520894\\
70.38	0.00253370098752255\\
70.39	0.0025375115400814\\
70.4	0.00254132441548984\\
70.41	0.00254513961635699\\
70.42	0.00254895714529677\\
70.43	0.00255277700492789\\
70.44	0.00255659919787389\\
70.45	0.00256042372676312\\
70.46	0.00256425059422878\\
70.47	0.00256807980290892\\
70.48	0.00257191135544648\\
70.49	0.00257574525448925\\
70.5	0.00257958150268992\\
70.51	0.00258342010270612\\
70.52	0.00258726105720036\\
70.53	0.00259110436884012\\
70.54	0.0025949500402978\\
70.55	0.0025987980742508\\
70.56	0.00260264847338147\\
70.57	0.00260650124037716\\
70.58	0.00261035637793022\\
70.59	0.00261421388873805\\
70.6	0.00261807377550304\\
70.61	0.00262193604093265\\
70.62	0.00262580068773943\\
70.63	0.00262966771864096\\
70.64	0.00263353713635995\\
70.65	0.0026374089436242\\
70.66	0.00264128314316664\\
70.67	0.00264515973772533\\
70.68	0.00264903873004348\\
70.69	0.00265292012286948\\
70.7	0.00265680391895689\\
70.71	0.00266069012106448\\
70.72	0.00266457873195622\\
70.73	0.00266846975440132\\
70.74	0.00267236319117421\\
70.75	0.00267625904505462\\
70.76	0.00268015731882751\\
70.77	0.00268405801528315\\
70.78	0.00268796113721714\\
70.79	0.00269186668743035\\
70.8	0.00269577466872904\\
70.81	0.00269968508392479\\
70.82	0.00270359793583457\\
70.83	0.00270751322728071\\
70.84	0.00271143096109099\\
70.85	0.00271535114009855\\
70.86	0.00271927376714201\\
70.87	0.00272319884506544\\
70.88	0.00272712637671836\\
70.89	0.00273105636495578\\
70.9	0.00273498881263822\\
70.91	0.00273892372263173\\
70.92	0.00274286109780788\\
70.93	0.0027468009410438\\
70.94	0.0027507432552222\\
70.95	0.00275468804323136\\
70.96	0.0027586353079652\\
70.97	0.00276258505232324\\
70.98	0.00276653727919371\\
70.99	0.00277049199134839\\
71	0.00277444919156405\\
71.01	0.00277840888262248\\
71.02	0.0027823710673105\\
71.03	0.00278633574841996\\
71.04	0.00279030292874776\\
71.05	0.0027942726110959\\
71.06	0.00279824479827141\\
71.07	0.00280221949308646\\
71.08	0.00280619669835828\\
71.09	0.00281017641690928\\
71.1	0.00281415865156696\\
71.11	0.00281814340516399\\
71.12	0.00282213068053819\\
71.13	0.00282612048053259\\
71.14	0.00283011280799538\\
71.15	0.00283410766577997\\
71.16	0.00283810505674499\\
71.17	0.00284210498375431\\
71.18	0.00284610744967704\\
71.19	0.00285011245738757\\
71.2	0.00285412000976556\\
71.21	0.00285813010969597\\
71.22	0.00286214276006905\\
71.23	0.00286615796378041\\
71.24	0.00287017572373096\\
71.25	0.00287419604282699\\
71.26	0.00287821892398015\\
71.27	0.00288224437010748\\
71.28	0.00288627238413141\\
71.29	0.00289030296897978\\
71.3	0.00289433612758587\\
71.31	0.0028983718628884\\
71.32	0.00290241017783157\\
71.33	0.00290645107536502\\
71.34	0.00291049455844392\\
71.35	0.00291454063002891\\
71.36	0.00291858929308618\\
71.37	0.00292264055058744\\
71.38	0.00292669440550998\\
71.39	0.00293075086083664\\
71.4	0.00293480991955585\\
71.41	0.00293887158466164\\
71.42	0.00294293585915368\\
71.43	0.00294700274603724\\
71.44	0.00295107224832327\\
71.45	0.00295514436902839\\
71.46	0.00295921911117488\\
71.47	0.00296329647779075\\
71.48	0.0029673764719097\\
71.49	0.00297145909657118\\
71.5	0.0029755443548204\\
71.51	0.00297963224970833\\
71.52	0.00298372278429171\\
71.53	0.00298781596163311\\
71.54	0.0029919117848009\\
71.55	0.0029960102568693\\
71.56	0.00300011138091838\\
71.57	0.00300421516003408\\
71.58	0.00300832159730824\\
71.59	0.00301243069583857\\
71.6	0.00301654245872876\\
71.61	0.00302065688908841\\
71.62	0.00302477399003307\\
71.63	0.00302889376468431\\
71.64	0.00303301621616965\\
71.65	0.00303714134762266\\
71.66	0.00304126916218293\\
71.67	0.00304539966299609\\
71.68	0.00304953285321386\\
71.69	0.00305366873599404\\
71.7	0.00305780731450053\\
71.71	0.00306194859190337\\
71.72	0.00306609257137874\\
71.73	0.00307023925610897\\
71.74	0.0030743886492826\\
71.75	0.00307854075409435\\
71.76	0.00308269557374517\\
71.77	0.00308685311144224\\
71.78	0.00309101337039903\\
71.79	0.00309517635383525\\
71.8	0.00309934206497693\\
71.81	0.00310351050705643\\
71.82	0.00310768168331243\\
71.83	0.00311185559698998\\
71.84	0.00311603225134051\\
71.85	0.00312021164962184\\
71.86	0.00312439379509821\\
71.87	0.00312857869104032\\
71.88	0.00313276634072531\\
71.89	0.00313695674743682\\
71.9	0.00314114991446497\\
71.91	0.00314534584510643\\
71.92	0.00314954454266438\\
71.93	0.00315374601044861\\
71.94	0.00315795025177546\\
71.95	0.0031621572699679\\
71.96	0.00316636706835552\\
71.97	0.00317057965027457\\
71.98	0.00317479501906797\\
71.99	0.00317901317808533\\
72	0.00318323413068298\\
72.01	0.003187457880224\\
72.02	0.00319168443007823\\
72.03	0.00319591378362228\\
72.04	0.00320014594423958\\
72.05	0.00320438091532039\\
72.06	0.00320861870026182\\
72.07	0.00321285930246786\\
72.08	0.0032171027253494\\
72.09	0.00322134897232424\\
72.1	0.00322559804681714\\
72.11	0.00322984995225982\\
72.12	0.00323410469209101\\
72.13	0.00323836226975643\\
72.14	0.00324262268870888\\
72.15	0.00324688595240818\\
72.16	0.00325115206432127\\
72.17	0.0032554210279222\\
72.18	0.00325969284669216\\
72.19	0.00326396752411948\\
72.2	0.00326824506369973\\
72.21	0.00327252546893563\\
72.22	0.00327680874333719\\
72.23	0.00328109489042167\\
72.24	0.0032853839137136\\
72.25	0.00328967581674485\\
72.26	0.00329397060305463\\
72.27	0.00329826827618951\\
72.28	0.00330256883970345\\
72.29	0.00330687229715784\\
72.3	0.00331117865212151\\
72.31	0.00331548790817078\\
72.32	0.00331980006888946\\
72.33	0.00332411513786888\\
72.34	0.00332843311870794\\
72.35	0.00333275401501312\\
72.36	0.0033370778303985\\
72.37	0.00334140456848581\\
72.38	0.00334573423290446\\
72.39	0.00335006682729152\\
72.4	0.00335440235529181\\
72.41	0.00335874082055789\\
72.42	0.00336308222675013\\
72.43	0.00336742657753667\\
72.44	0.00337177387659352\\
72.45	0.00337612412760453\\
72.46	0.00338047733426148\\
72.47	0.00338483350026406\\
72.48	0.00338919262931993\\
72.49	0.00339355472514472\\
72.5	0.0033979197914621\\
72.51	0.0034022878320038\\
72.52	0.00340665885050959\\
72.53	0.00341103285072739\\
72.54	0.00341540983641326\\
72.55	0.00341978981133141\\
72.56	0.0034241727792543\\
72.57	0.00342855874396259\\
72.58	0.00343294770924523\\
72.59	0.00343733967889948\\
72.6	0.00344173465673093\\
72.61	0.00344613264655354\\
72.62	0.00345053365218968\\
72.63	0.00345493767747015\\
72.64	0.00345934472623423\\
72.65	0.0034637548023297\\
72.66	0.00346816790961289\\
72.67	0.0034725840519487\\
72.68	0.00347700323321062\\
72.69	0.00348142545728081\\
72.7	0.00348585072805011\\
72.71	0.00349027904941806\\
72.72	0.00349471042529296\\
72.73	0.00349914485959189\\
72.74	0.00350358235624079\\
72.75	0.0035080229191744\\
72.76	0.00351246655233642\\
72.77	0.00351691325967944\\
72.78	0.00352136304516506\\
72.79	0.00352581591276386\\
72.8	0.0035302718664555\\
72.81	0.00353473091022869\\
72.82	0.00353919304808131\\
72.83	0.00354365828402037\\
72.84	0.00354812662206213\\
72.85	0.00355259806623204\\
72.86	0.00355707262056489\\
72.87	0.00356155028910475\\
72.88	0.0035660310759051\\
72.89	0.0035705149850288\\
72.9	0.00357500202054817\\
72.91	0.00357949218654502\\
72.92	0.0035839854871107\\
72.93	0.00358848192634614\\
72.94	0.00359298150836187\\
72.95	0.0035974842372781\\
72.96	0.00360199011722475\\
72.97	0.00360649915234148\\
72.98	0.00361101134677775\\
72.99	0.00361552670469286\\
73	0.003620045230256\\
73.01	0.00362456692764628\\
73.02	0.00362909180105279\\
73.03	0.00363361985467465\\
73.04	0.00363815109272104\\
73.05	0.00364268551941127\\
73.06	0.0036472231389748\\
73.07	0.00365176395565131\\
73.08	0.00365630797369074\\
73.09	0.00366085519735334\\
73.1	0.00366540563090972\\
73.11	0.00366995927864089\\
73.12	0.00367451614483832\\
73.13	0.00367907623380401\\
73.14	0.00368363954985049\\
73.15	0.00368820609730092\\
73.16	0.00369277588048912\\
73.17	0.00369734890375962\\
73.18	0.00370192517146773\\
73.19	0.00370650468797957\\
73.2	0.00371108745767214\\
73.21	0.00371567348493338\\
73.22	0.0037202627741622\\
73.23	0.00372485532976857\\
73.24	0.00372945115617352\\
73.25	0.00373405025780928\\
73.26	0.00373865263911926\\
73.27	0.00374325830455813\\
73.28	0.00374786725859191\\
73.29	0.003752479505698\\
73.3	0.00375709505036521\\
73.31	0.0037617138970939\\
73.32	0.00376633605039596\\
73.33	0.0037709615147949\\
73.34	0.00377559029482595\\
73.35	0.00378022239503604\\
73.36	0.00378485781998394\\
73.37	0.00378949657424028\\
73.38	0.00379413866238764\\
73.39	0.00379878408902057\\
73.4	0.00380343285874572\\
73.41	0.00380808497618184\\
73.42	0.00381274044595991\\
73.43	0.00381739927272313\\
73.44	0.00382206146112707\\
73.45	0.00382672701583968\\
73.46	0.00383139594154139\\
73.47	0.00383606824292515\\
73.48	0.00384074392469653\\
73.49	0.00384542299157378\\
73.5	0.0038501054482879\\
73.51	0.0038547912995827\\
73.52	0.00385948055021491\\
73.53	0.0038641732049542\\
73.54	0.00386886926858332\\
73.55	0.0038735687458981\\
73.56	0.0038782716417076\\
73.57	0.00388297796083414\\
73.58	0.00388768770811337\\
73.59	0.00389240088839441\\
73.6	0.00389711750653985\\
73.61	0.00390183756742589\\
73.62	0.00390656107594238\\
73.63	0.00391128803699295\\
73.64	0.00391601845549503\\
73.65	0.00392075233637999\\
73.66	0.00392548968459318\\
73.67	0.00393023050509407\\
73.68	0.00393497480285627\\
73.69	0.00393972258286766\\
73.7	0.00394447385013046\\
73.71	0.00394922860966134\\
73.72	0.00395398686649149\\
73.73	0.0039587486256667\\
73.74	0.00396351389224749\\
73.75	0.00396828267130917\\
73.76	0.00397305496794195\\
73.77	0.00397783078725101\\
73.78	0.00398261013435665\\
73.79	0.00398739301439431\\
73.8	0.00399217943251474\\
73.81	0.00399696939388405\\
73.82	0.00400176290368383\\
73.83	0.00400655996711126\\
73.84	0.00401136058937918\\
73.85	0.00401616477571623\\
73.86	0.00402097253136693\\
73.87	0.0040257838615918\\
73.88	0.00403059877166745\\
73.89	0.0040354172668867\\
73.9	0.00404023935255869\\
73.91	0.00404506503400898\\
73.92	0.00404989431657969\\
73.93	0.00405472720562955\\
73.94	0.00405956370653409\\
73.95	0.0040644038246857\\
73.96	0.00406924756549377\\
73.97	0.00407409493438481\\
73.98	0.00407894593680256\\
73.99	0.0040838005782081\\
74	0.00408865886408\\
74.01	0.00409352079991442\\
74.02	0.00409838639122525\\
74.03	0.00410325564354422\\
74.04	0.00410812856242104\\
74.05	0.00411300515342351\\
74.06	0.00411788542213769\\
74.07	0.00412276937416798\\
74.08	0.00412765701513731\\
74.09	0.00413254835068721\\
74.1	0.004137443386478\\
74.11	0.00414234212818891\\
74.12	0.00414724458151821\\
74.13	0.00415215075218336\\
74.14	0.00415706064592116\\
74.15	0.00416197426848788\\
74.16	0.00416689162565941\\
74.17	0.00417181272323141\\
74.18	0.00417673756701947\\
74.19	0.00418166616285925\\
74.2	0.00418659851660661\\
74.21	0.00419153463413783\\
74.22	0.00419647452134969\\
74.23	0.00420141818415968\\
74.24	0.00420636562850614\\
74.25	0.00421131686034844\\
74.26	0.00421627188566711\\
74.27	0.00422123071046405\\
74.28	0.00422619334076265\\
74.29	0.00423115978260801\\
74.3	0.00423613004206708\\
74.31	0.00424110412522883\\
74.32	0.00424608203820446\\
74.33	0.00425106378712754\\
74.34	0.00425604937815422\\
74.35	0.0042610388174634\\
74.36	0.0042660321112569\\
74.37	0.00427102926575967\\
74.38	0.00427603028721999\\
74.39	0.00428103518190962\\
74.4	0.00428604395612403\\
74.41	0.00429105661618259\\
74.42	0.00429607316842875\\
74.43	0.00430109361923027\\
74.44	0.00430611797497939\\
74.45	0.00431114624209308\\
74.46	0.00431617842701321\\
74.47	0.00432121453620676\\
74.48	0.00432625457616607\\
74.49	0.00433129855340905\\
74.5	0.00433634647447934\\
74.51	0.00434139834594662\\
74.52	0.00434645417440677\\
74.53	0.00435151396648212\\
74.54	0.0043565777288217\\
74.55	0.00436164546810142\\
74.56	0.00436671719102436\\
74.57	0.00437179290432099\\
74.58	0.0043768726147494\\
74.59	0.00438195632909557\\
74.6	0.00438704405417357\\
74.61	0.00439213579682589\\
74.62	0.00439723156392361\\
74.63	0.00440233136236672\\
74.64	0.00440743519908435\\
74.65	0.00441254308103504\\
74.66	0.00441765501520701\\
74.67	0.00442277100861842\\
74.68	0.00442789106831767\\
74.69	0.00443301520138366\\
74.7	0.00443814341492605\\
74.71	0.00444327571608559\\
74.72	0.00444841211203439\\
74.73	0.00445355260997618\\
74.74	0.00445869721714666\\
74.75	0.00446384594081375\\
74.76	0.00446899878827795\\
74.77	0.00447415576687257\\
74.78	0.00447931688396412\\
74.79	0.00448448214695258\\
74.8	0.00448965156327172\\
74.81	0.00449482514038944\\
74.82	0.00450000288580808\\
74.83	0.00450518480706478\\
74.84	0.00451037091173178\\
74.85	0.00451556120741678\\
74.86	0.0045207557017633\\
74.87	0.00452595440245099\\
74.88	0.004531157317196\\
74.89	0.00453636445375137\\
74.9	0.00454157581990735\\
74.91	0.00454679142349178\\
74.92	0.00455201127237047\\
74.93	0.00455723537444757\\
74.94	0.00456246373766596\\
74.95	0.00456769637000763\\
74.96	0.00457293327949407\\
74.97	0.00457817447418668\\
74.98	0.00458341996218714\\
74.99	0.00458866975163788\\
75	0.00459392385072242\\
75.01	0.00459918226766583\\
75.02	0.00460444501073517\\
75.03	0.00460971208823985\\
75.04	0.00461498350853216\\
75.05	0.00462025928000763\\
75.06	0.00462553941110551\\
75.07	0.00463082391030923\\
75.08	0.00463611278614684\\
75.09	0.0046414060471915\\
75.1	0.00464670370206192\\
75.11	0.00465200575821142\\
75.12	0.00465731222253704\\
75.13	0.00466262310195214\\
75.14	0.00466793840338646\\
75.15	0.00467325813378614\\
75.16	0.00467858230011376\\
75.17	0.00468391090934843\\
75.18	0.00468924396848578\\
75.19	0.00469458148453803\\
75.2	0.00469992346453403\\
75.21	0.00470526991551934\\
75.22	0.00471062084455619\\
75.23	0.00471597625872361\\
75.24	0.00472133616511744\\
75.25	0.00472670057085037\\
75.26	0.004732069483052\\
75.27	0.00473744290886886\\
75.28	0.00474282085546449\\
75.29	0.00474820333001946\\
75.3	0.00475359033973142\\
75.31	0.00475898189181515\\
75.32	0.0047643779935026\\
75.33	0.00476977865204295\\
75.34	0.00477518387470261\\
75.35	0.00478059366876535\\
75.36	0.00478600804153224\\
75.37	0.00479142700032179\\
75.38	0.00479685055246993\\
75.39	0.00480227870533008\\
75.4	0.00480771146627321\\
75.41	0.00481314884268786\\
75.42	0.00481859084198021\\
75.43	0.00482403747157407\\
75.44	0.00482948873891102\\
75.45	0.00483494465145036\\
75.46	0.00484040521666923\\
75.47	0.00484587044206259\\
75.48	0.00485134033514332\\
75.49	0.00485681490344223\\
75.5	0.00486229415450814\\
75.51	0.00486777809590788\\
75.52	0.00487143349896525\\
75.53	0.00487380021309105\\
75.54	0.00487616774895587\\
75.55	0.00487853610494478\\
75.56	0.00488090527943267\\
75.57	0.00488327527078415\\
75.58	0.00488564607735359\\
75.59	0.00488801769748496\\
75.6	0.00489039012951192\\
75.61	0.00489276337175763\\
75.62	0.00489513742253482\\
75.63	0.00489751228014567\\
75.64	0.00489988794288181\\
75.65	0.00490226440902422\\
75.66	0.00490464167684325\\
75.67	0.00490701974459849\\
75.68	0.0049093986105388\\
75.69	0.00491177827290222\\
75.7	0.00491415872991591\\
75.71	0.00491653997979613\\
75.72	0.00491892202074819\\
75.73	0.00492130485096638\\
75.74	0.0049236884686339\\
75.75	0.00492607287192291\\
75.76	0.00492845805899434\\
75.77	0.00493084402799794\\
75.78	0.00493323077707221\\
75.79	0.0049356183043443\\
75.8	0.00493800660793003\\
75.81	0.00494039568593379\\
75.82	0.0049427855364485\\
75.83	0.00494517615755557\\
75.84	0.00494756754732484\\
75.85	0.00494995970381451\\
75.86	0.00495235262507111\\
75.87	0.00495474630912946\\
75.88	0.00495714075401257\\
75.89	0.00495953595773162\\
75.9	0.00496193191828591\\
75.91	0.00496432863366279\\
75.92	0.00496672610183762\\
75.93	0.00496912432077368\\
75.94	0.00497152328842218\\
75.95	0.00497392300272215\\
75.96	0.00497632346160039\\
75.97	0.00497872466297145\\
75.98	0.00498112660473753\\
75.99	0.00498352928478848\\
76	0.00498593270100165\\
76.01	0.00498833685124197\\
76.02	0.00499074173336172\\
76.03	0.00499314734520067\\
76.04	0.00499555368458584\\
76.05	0.00499796074933156\\
76.06	0.00500036853723938\\
76.07	0.00500277704609799\\
76.08	0.00500518627368318\\
76.09	0.0050075962177578\\
76.1	0.00501000687607165\\
76.11	0.00501241824636148\\
76.12	0.00501483032635088\\
76.13	0.00501724311375025\\
76.14	0.00501965660625674\\
76.15	0.00502207080155417\\
76.16	0.00502448569731297\\
76.17	0.00502690129119017\\
76.18	0.00502931758082924\\
76.19	0.00503173456386013\\
76.2	0.00503415223789914\\
76.21	0.00503657060054891\\
76.22	0.0050389896493983\\
76.23	0.00504140938202235\\
76.24	0.00504382979598227\\
76.25	0.00504625088882527\\
76.26	0.0050486726580846\\
76.27	0.00505109510127942\\
76.28	0.00505351821591476\\
76.29	0.00505594199948146\\
76.3	0.0050583664494561\\
76.31	0.0050607915633009\\
76.32	0.00506321733846372\\
76.33	0.00506564377237795\\
76.34	0.00506807086246245\\
76.35	0.00507049860612148\\
76.36	0.00507292700074466\\
76.37	0.00507535604370685\\
76.38	0.00507778573236815\\
76.39	0.00508021606407378\\
76.4	0.00508264703615401\\
76.41	0.00508507864592413\\
76.42	0.00508751089068437\\
76.43	0.00508994376771979\\
76.44	0.00509237727430027\\
76.45	0.00509481140768038\\
76.46	0.00509724616509937\\
76.47	0.00509968154378105\\
76.48	0.00510211754093374\\
76.49	0.0051045541537502\\
76.5	0.00510699137940756\\
76.51	0.00510942921506722\\
76.52	0.00511186765787483\\
76.53	0.00511430670496016\\
76.54	0.00511674635343707\\
76.55	0.0051191866004034\\
76.56	0.00512162744294094\\
76.57	0.00512406887811529\\
76.58	0.00512651090297589\\
76.59	0.00512895351455581\\
76.6	0.00513139670987178\\
76.61	0.00513384048592408\\
76.62	0.00513628483969645\\
76.63	0.00513872976815602\\
76.64	0.00514117526825326\\
76.65	0.00514362133692187\\
76.66	0.00514606797107867\\
76.67	0.00514851516762364\\
76.68	0.00515096292343969\\
76.69	0.00515341123539272\\
76.7	0.00515586010033141\\
76.71	0.00515830951508726\\
76.72	0.00516075947647442\\
76.73	0.00516320998128966\\
76.74	0.00516566102631226\\
76.75	0.00516811260830394\\
76.76	0.0051705647240088\\
76.77	0.00517301737015317\\
76.78	0.00517547054344563\\
76.79	0.00517792405067916\\
76.8	0.00518037784931282\\
76.81	0.00518283193648724\\
76.82	0.00518528630933034\\
76.83	0.0051877409649573\\
76.84	0.00519019590047051\\
76.85	0.00519265111295945\\
76.86	0.00519510659950072\\
76.87	0.00519756235715793\\
76.88	0.00520001838298167\\
76.89	0.00520247467400943\\
76.9	0.00520493122726555\\
76.91	0.00520738803976118\\
76.92	0.0052098451084942\\
76.93	0.00521230243044918\\
76.94	0.0052147600025973\\
76.95	0.00521721782189631\\
76.96	0.00521967588529047\\
76.97	0.00522213418971046\\
76.98	0.00522459273207338\\
76.99	0.00522705150928262\\
77	0.00522951051822788\\
77.01	0.00523196975578501\\
77.02	0.00523442921881604\\
77.03	0.00523688890416905\\
77.04	0.00523934880867817\\
77.05	0.00524180892916347\\
77.06	0.00524426926243091\\
77.07	0.00524672980527229\\
77.08	0.00524919055446516\\
77.09	0.00525165150677279\\
77.1	0.00525411265894409\\
77.11	0.00525657400771352\\
77.12	0.00525903554980107\\
77.13	0.00526149728191217\\
77.14	0.00526395920073762\\
77.15	0.00526642130295356\\
77.16	0.00526888358522132\\
77.17	0.00527134604418746\\
77.18	0.00527380867648363\\
77.19	0.00527627147872652\\
77.2	0.00527873444751778\\
77.21	0.00528119757944401\\
77.22	0.0052836608710766\\
77.23	0.00528612431897174\\
77.24	0.00528858791967029\\
77.25	0.00529105166969776\\
77.26	0.0052935155655642\\
77.27	0.00529597960376415\\
77.28	0.00529844378077657\\
77.29	0.00530090809306474\\
77.3	0.00530337253707623\\
77.31	0.0053058371092428\\
77.32	0.00530830180598032\\
77.33	0.00531076662368871\\
77.34	0.00531323155875187\\
77.35	0.0053156966075376\\
77.36	0.00531816176639751\\
77.37	0.00532062703166697\\
77.38	0.00532309239966501\\
77.39	0.00532555786669424\\
77.4	0.00532802342904081\\
77.41	0.00533048908297431\\
77.42	0.00533295482474766\\
77.43	0.00533542065059708\\
77.44	0.00533788655674198\\
77.45	0.0053403525393849\\
77.46	0.00534281859471143\\
77.47	0.00534528471889007\\
77.48	0.00534775090807226\\
77.49	0.00535021715839219\\
77.5	0.00535268346596677\\
77.51	0.00535514982689556\\
77.52	0.00535761623726064\\
77.53	0.00536008269312656\\
77.54	0.00536254919054024\\
77.55	0.0053650157255309\\
77.56	0.00536748229410994\\
77.57	0.00536994889227091\\
77.58	0.00537241551598936\\
77.59	0.00537488216122279\\
77.6	0.00537734882391055\\
77.61	0.00537981549997377\\
77.62	0.00538228218531522\\
77.63	0.00538474887581929\\
77.64	0.00538721556735182\\
77.65	0.00538968225576009\\
77.66	0.00539214893687266\\
77.67	0.00539461560649932\\
77.68	0.00539708226043097\\
77.69	0.00539954889443953\\
77.7	0.00540201550427787\\
77.71	0.00540448208567968\\
77.72	0.0054069486343594\\
77.73	0.00540941514601209\\
77.74	0.00541188161631338\\
77.75	0.00541434804091932\\
77.76	0.00541681441546634\\
77.77	0.00541928073557107\\
77.78	0.00542174699683032\\
77.79	0.00542421319482093\\
77.8	0.00542667932509968\\
77.81	0.00542914538320318\\
77.82	0.00543161136464781\\
77.83	0.00543407726492953\\
77.84	0.00543654307952386\\
77.85	0.00543900880388572\\
77.86	0.00544147443344935\\
77.87	0.0054439399636282\\
77.88	0.00544640538981478\\
77.89	0.00544887070738064\\
77.9	0.00545133591167617\\
77.91	0.00545380099803052\\
77.92	0.00545626596175151\\
77.93	0.00545873079812549\\
77.94	0.00546119550241725\\
77.95	0.00546366006986986\\
77.96	0.00546612449570463\\
77.97	0.00546858877512091\\
77.98	0.00547105290329603\\
77.99	0.00547351687538518\\
78	0.00547598068652124\\
78.01	0.00547844433181472\\
78.02	0.00548090780635362\\
78.03	0.00548337110520328\\
78.04	0.0054858342234063\\
78.05	0.00548829715598238\\
78.06	0.00549075989792821\\
78.07	0.00549322244421761\\
78.08	0.00549568478980357\\
78.09	0.00549814692961819\\
78.1	0.00550060885857262\\
78.11	0.00550307057155694\\
78.12	0.00550553206344014\\
78.13	0.00550799332906995\\
78.14	0.00551045436327284\\
78.15	0.0055129151608539\\
78.16	0.00551537571659676\\
78.17	0.00551783602526349\\
78.18	0.00552029608159456\\
78.19	0.00552275588030871\\
78.2	0.00552521541610289\\
78.21	0.00552767468365219\\
78.22	0.00553013367760971\\
78.23	0.00553259239260651\\
78.24	0.00553505082325153\\
78.25	0.00553750896413148\\
78.26	0.00553996680981073\\
78.27	0.00554242435483132\\
78.28	0.00554488159371275\\
78.29	0.005547338520952\\
78.3	0.00554979513102334\\
78.31	0.00555225141837837\\
78.32	0.00555470737744578\\
78.33	0.0055571630026314\\
78.34	0.005559618288318\\
78.35	0.00556207322886529\\
78.36	0.00556452781860977\\
78.37	0.00556698205186466\\
78.38	0.00556943592291982\\
78.39	0.00557188942604164\\
78.4	0.00557434255547297\\
78.41	0.00557679530543299\\
78.42	0.00557924767011716\\
78.43	0.00558169964369713\\
78.44	0.00558415122032058\\
78.45	0.00558660239411122\\
78.46	0.00558905315916864\\
78.47	0.0055915035095682\\
78.48	0.00559395343936102\\
78.49	0.00559640294257377\\
78.5	0.00559885201320867\\
78.51	0.00560130064524334\\
78.52	0.00560374883263073\\
78.53	0.00560619656929903\\
78.54	0.00560864384915153\\
78.55	0.00561109066606657\\
78.56	0.00561353701389742\\
78.57	0.0056159828864722\\
78.58	0.00561842827759374\\
78.59	0.00562087318103957\\
78.6	0.00562331759056168\\
78.61	0.00562576149988657\\
78.62	0.00562820490271504\\
78.63	0.00563064779272217\\
78.64	0.00563309016355714\\
78.65	0.00563553200884321\\
78.66	0.00563797332217757\\
78.67	0.00564041409713121\\
78.68	0.00564285432724891\\
78.69	0.00564529400604905\\
78.7	0.00564773312702355\\
78.71	0.00565017168363775\\
78.72	0.00565260966933033\\
78.73	0.00565504707751318\\
78.74	0.00565748390157128\\
78.75	0.00565992013486264\\
78.76	0.00566235577071819\\
78.77	0.00566479080244163\\
78.78	0.00566722522330936\\
78.79	0.00566965902657035\\
78.8	0.00567209220544608\\
78.81	0.00567452475313036\\
78.82	0.0056769566627893\\
78.83	0.00567938792756112\\
78.84	0.00568181854055612\\
78.85	0.00568424849485651\\
78.86	0.00568667778351633\\
78.87	0.00568910639956134\\
78.88	0.00569153433598889\\
78.89	0.00569396158576784\\
78.9	0.00569638814183841\\
78.91	0.0056988139971121\\
78.92	0.00570123914447156\\
78.93	0.00570366357677048\\
78.94	0.00570608728683349\\
78.95	0.00570851026745601\\
78.96	0.00571093251140418\\
78.97	0.00571335401141473\\
78.98	0.00571577476019485\\
78.99	0.00571819475042207\\
79	0.00572061397474418\\
79.01	0.00572303242577908\\
79.02	0.00572545009611468\\
79.03	0.00572786697830876\\
79.04	0.00573028306488888\\
79.05	0.00573269834835227\\
79.06	0.00573511282116563\\
79.07	0.00573752647576513\\
79.08	0.0057399393045562\\
79.09	0.00574235129991345\\
79.1	0.00574476245418053\\
79.11	0.00574717275967002\\
79.12	0.00574958220866331\\
79.13	0.00575199079341046\\
79.14	0.00575439850613009\\
79.15	0.00575680533900927\\
79.16	0.00575921128420337\\
79.17	0.00576161633383594\\
79.18	0.00576402047999861\\
79.19	0.00576642371475091\\
79.2	0.00576882603012022\\
79.21	0.00577122741810159\\
79.22	0.0057736278706576\\
79.23	0.0057760273797183\\
79.24	0.00577842593718101\\
79.25	0.00578082353491023\\
79.26	0.0057832201647375\\
79.27	0.00578561581846126\\
79.28	0.00578801048784677\\
79.29	0.00579040416462589\\
79.3	0.00579279684049704\\
79.31	0.005795188507125\\
79.32	0.00579757915614082\\
79.33	0.00579996877914167\\
79.34	0.0058023573676907\\
79.35	0.00580474491331692\\
79.36	0.00580713140751506\\
79.37	0.00580951684174543\\
79.38	0.0058119012074338\\
79.39	0.00581428449597124\\
79.4	0.005816666698714\\
79.41	0.00581904780698337\\
79.42	0.00582142781206553\\
79.43	0.00582380670521144\\
79.44	0.00582618447763666\\
79.45	0.00582856112052126\\
79.46	0.00583093662500961\\
79.47	0.00583331098221034\\
79.48	0.00583568418319608\\
79.49	0.00583805621900342\\
79.5	0.00584042708063272\\
79.51	0.00584279675904795\\
79.52	0.00584516524517659\\
79.53	0.00584753252990945\\
79.54	0.00584989860410056\\
79.55	0.00585226345856698\\
79.56	0.00585462708408869\\
79.57	0.00585698947140844\\
79.58	0.00585935061123158\\
79.59	0.00586171049422592\\
79.6	0.00586406911102162\\
79.61	0.00586642645221096\\
79.62	0.00586878250834828\\
79.63	0.00587113726994976\\
79.64	0.00587349072749332\\
79.65	0.00587584287141844\\
79.66	0.00587819369212601\\
79.67	0.00588054317997817\\
79.68	0.00588289132529818\\
79.69	0.00588523811837024\\
79.7	0.00588758354943938\\
79.71	0.00588992760871123\\
79.72	0.00589227028635195\\
79.73	0.00589461157248799\\
79.74	0.00589695145720601\\
79.75	0.00589928993055264\\
79.76	0.00590162698253442\\
79.77	0.00590396260311754\\
79.78	0.00590629678222776\\
79.79	0.00590862950975021\\
79.8	0.00591096077552921\\
79.81	0.00591329056936817\\
79.82	0.00591561888102938\\
79.83	0.00591794570023384\\
79.84	0.00592027101666113\\
79.85	0.00592259481994924\\
79.86	0.00592491709969436\\
79.87	0.00592723784545078\\
79.88	0.00592955704673069\\
79.89	0.00593187469300399\\
79.9	0.00593419077369818\\
79.91	0.00593650527819814\\
79.92	0.00593881819584596\\
79.93	0.00594112951594082\\
79.94	0.00594343922773877\\
79.95	0.00594574732045259\\
79.96	0.00594805378325159\\
79.97	0.00595035860526145\\
79.98	0.00595266177556406\\
79.99	0.00595496328319733\\
80	0.00595726311715502\\
80.01	0.00595956126638655\\
};
\addplot [color=red,solid]
  table[row sep=crcr]{%
80.01	0.00595956126638655\\
80.02	0.00596185771979684\\
80.03	0.00596415246624615\\
80.04	0.00596644549454984\\
80.05	0.00596873679347827\\
80.06	0.00597102635175657\\
80.07	0.00597331415806446\\
80.08	0.0059756002010361\\
80.09	0.00597788446925988\\
80.1	0.00598016695127826\\
80.11	0.00598244763558754\\
80.12	0.00598472651063775\\
80.13	0.00598700356483241\\
80.14	0.00598927878652837\\
80.15	0.00599155216403559\\
80.16	0.00599382368561699\\
80.17	0.00599609333948826\\
80.18	0.00599836111381767\\
80.19	0.00600062699672582\\
80.2	0.00600289097628555\\
80.21	0.0060051530405217\\
80.22	0.00600741317741088\\
80.23	0.00600967137488136\\
80.24	0.0060119276208128\\
80.25	0.00601418190303612\\
80.26	0.00601643420933325\\
80.27	0.00601868452743696\\
80.28	0.00602093284503068\\
80.29	0.00602317914974827\\
80.3	0.00602542342917383\\
80.31	0.00602766567084153\\
80.32	0.00602990586223537\\
80.33	0.00603214399078901\\
80.34	0.00603438004388553\\
80.35	0.00603661400885729\\
80.36	0.00603884587298567\\
80.37	0.0060410756235009\\
80.38	0.0060433032475818\\
80.39	0.00604552873235568\\
80.4	0.00604775206489801\\
80.41	0.00604997323223232\\
80.42	0.00605219374020172\\
80.43	0.00605441463694088\\
80.44	0.00605663592089763\\
80.45	0.00605885759051529\\
80.46	0.00606107964423271\\
80.47	0.0060633020804842\\
80.48	0.00606552489769959\\
80.49	0.00606774809430417\\
80.5	0.00606997166871869\\
80.51	0.00607219561935941\\
80.52	0.00607441994463801\\
80.53	0.00607664464296163\\
80.54	0.00607886971273288\\
80.55	0.00608109515234977\\
80.56	0.00608332096020576\\
80.57	0.00608554713468975\\
80.58	0.00608777367418602\\
80.59	0.0060900005770743\\
80.6	0.0060922278417297\\
80.61	0.00609445546652272\\
80.62	0.00609668344981929\\
80.63	0.00609891178998067\\
80.64	0.00610114048536353\\
80.65	0.00610336953431991\\
80.66	0.00610559893519719\\
80.67	0.00610782868633813\\
80.68	0.00611005878608081\\
80.69	0.0061122892327587\\
80.7	0.00611452002470056\\
80.71	0.0061167511602305\\
80.72	0.00611898263766793\\
80.73	0.00612121445532762\\
80.74	0.0061234466115196\\
80.75	0.00612567910454923\\
80.76	0.00612791193271717\\
80.77	0.00613014509431935\\
80.78	0.00613237858764699\\
80.79	0.00613461241098659\\
80.8	0.00613684656261991\\
80.81	0.00613908104082399\\
80.82	0.00614131584387111\\
80.83	0.0061435509700288\\
80.84	0.00614578641755984\\
80.85	0.00614802218472226\\
80.86	0.00615025826976928\\
80.87	0.00615249467094939\\
80.88	0.00615473138650626\\
80.89	0.00615696841467881\\
80.9	0.00615920575370113\\
80.91	0.00616144340180252\\
80.92	0.00616368135720749\\
80.93	0.0061659196181357\\
80.94	0.00616815818280203\\
80.95	0.0061703970494165\\
80.96	0.00617263621618433\\
80.97	0.00617487568130587\\
80.98	0.00617711544297663\\
80.99	0.00617935549938731\\
81	0.00618159584872369\\
81.01	0.00618383648916674\\
81.02	0.00618607741889252\\
81.03	0.00618831863607225\\
81.04	0.00619056013887224\\
81.05	0.00619280192545394\\
81.06	0.00619504399397389\\
81.07	0.00619728634258373\\
81.08	0.00619952896943021\\
81.09	0.00620177187265514\\
81.1	0.00620401505039547\\
81.11	0.00620625850078316\\
81.12	0.00620850222194528\\
81.13	0.00621074621200397\\
81.14	0.00621299046907641\\
81.15	0.00621523499127487\\
81.16	0.00621747977670663\\
81.17	0.00621972482347403\\
81.18	0.00622197012967445\\
81.19	0.00622421569340031\\
81.2	0.00622646151273904\\
81.21	0.00622870758577313\\
81.22	0.00623095391058003\\
81.23	0.00623320048523224\\
81.24	0.00623544730779726\\
81.25	0.00623769437633759\\
81.26	0.00623994168891073\\
81.27	0.00624218924356914\\
81.28	0.00624443703836031\\
81.29	0.00624668507132668\\
81.3	0.00624893334050568\\
81.31	0.00625118184392971\\
81.32	0.00625343057962611\\
81.33	0.00625567954561721\\
81.34	0.00625792873992029\\
81.35	0.00626017816054757\\
81.36	0.0062624278055062\\
81.37	0.00626467767279832\\
81.38	0.00626692776042095\\
81.39	0.00626917806636608\\
81.4	0.00627142858862059\\
81.41	0.00627367932516633\\
81.42	0.00627593027398003\\
81.43	0.00627818143303334\\
81.44	0.00628043280029281\\
81.45	0.00628268437371992\\
81.46	0.00628493615127103\\
81.47	0.00628718813089739\\
81.48	0.00628944031054514\\
81.49	0.00629169268815532\\
81.5	0.00629394526166386\\
81.51	0.00629619802900152\\
81.52	0.00629845098809399\\
81.53	0.00630070413686179\\
81.54	0.00630295747322033\\
81.55	0.00630521099507986\\
81.56	0.00630746470034551\\
81.57	0.00630971858691725\\
81.58	0.00631197265268989\\
81.59	0.00631422689555311\\
81.6	0.00631648131339142\\
81.61	0.00631873590408416\\
81.62	0.00632099066550551\\
81.63	0.0063232455955245\\
81.64	0.00632550069200495\\
81.65	0.00632775595280556\\
81.66	0.00633001137577979\\
81.67	0.00633226695877595\\
81.68	0.00633452269963717\\
81.69	0.00633677859620137\\
81.7	0.00633903464630131\\
81.71	0.00634129084776452\\
81.72	0.00634354719841334\\
81.73	0.00634580369606493\\
81.74	0.00634806033853124\\
81.75	0.00635031712361899\\
81.76	0.00635257404912971\\
81.77	0.00635483111285973\\
81.78	0.00635708831260013\\
81.79	0.00635934564613682\\
81.8	0.00636160311125044\\
81.81	0.00636386070571645\\
81.82	0.00636611842730506\\
81.83	0.00636837627378126\\
81.84	0.00637063424290482\\
81.85	0.00637289233243028\\
81.86	0.00637515054010692\\
81.87	0.00637740886367882\\
81.88	0.00637966730088481\\
81.89	0.00638192584945847\\
81.9	0.00638418450712815\\
81.91	0.00638644327161695\\
81.92	0.00638870214064276\\
81.93	0.00639096111191818\\
81.94	0.00639322018315057\\
81.95	0.00639547935204208\\
81.96	0.00639773861628957\\
81.97	0.00639999797358467\\
81.98	0.00640225742161374\\
81.99	0.00640451695805792\\
82	0.00640677658059306\\
82.01	0.00640903628688978\\
82.02	0.00641129607461344\\
82.03	0.00641355594142414\\
82.04	0.00641581588497672\\
82.05	0.00641807590292076\\
82.06	0.00642033599290062\\
82.07	0.00642259615255533\\
82.08	0.00642485637951875\\
82.09	0.0064271166714194\\
82.1	0.0064293770258806\\
82.11	0.00643163744052039\\
82.12	0.00643389791295155\\
82.13	0.00643615844078159\\
82.14	0.0064384190216128\\
82.15	0.00644067965304218\\
82.16	0.00644294033266148\\
82.17	0.0064452010580572\\
82.18	0.00644746182681059\\
82.19	0.00644972263649764\\
82.2	0.00645198348468908\\
82.21	0.0064542443689504\\
82.22	0.00645650528684182\\
82.23	0.00645876623591834\\
82.24	0.00646102721372968\\
82.25	0.00646328821782034\\
82.26	0.00646554924572955\\
82.27	0.00646781029499131\\
82.28	0.00647007136313439\\
82.29	0.00647233244768229\\
82.3	0.0064745935461533\\
82.31	0.00647685465606046\\
82.32	0.00647911577491158\\
82.33	0.00648137690020924\\
82.34	0.00648363802945079\\
82.35	0.00648589916012836\\
82.36	0.00648816028972883\\
82.37	0.00649042141569527\\
82.38	0.00649268253546475\\
82.39	0.00649494364646921\\
82.4	0.00649720474613542\\
82.41	0.006499465831885\\
82.42	0.00650172690113441\\
82.43	0.00650398795129496\\
82.44	0.0065062489797728\\
82.45	0.00650850998396896\\
82.46	0.00651077096127931\\
82.47	0.0065130319090946\\
82.48	0.00651529282480043\\
82.49	0.00651755370577729\\
82.5	0.00651981454940051\\
82.51	0.00652207535304036\\
82.52	0.00652433611406193\\
82.53	0.00652659682982524\\
82.54	0.00652885749768518\\
82.55	0.00653111811499155\\
82.56	0.00653337867908904\\
82.57	0.00653563918731728\\
82.58	0.00653789963701076\\
82.59	0.00654016002549893\\
82.6	0.00654242035010613\\
82.61	0.00654468060815167\\
82.62	0.00654694079694975\\
82.63	0.00654920091380952\\
82.64	0.00655146095603509\\
82.65	0.0065537209209255\\
82.66	0.00655598080577478\\
82.67	0.00655824060787188\\
82.68	0.00656050032450073\\
82.69	0.00656275995294025\\
82.7	0.00656501949046433\\
82.71	0.00656727893434186\\
82.72	0.0065695382818367\\
82.73	0.00657179753020773\\
82.74	0.00657405667670883\\
82.75	0.0065763157185889\\
82.76	0.00657857465309186\\
82.77	0.00658083347745666\\
82.78	0.00658309218891729\\
82.79	0.00658535078470278\\
82.8	0.00658760926203721\\
82.81	0.00658986761813974\\
82.82	0.00659212585022456\\
82.83	0.00659438395550098\\
82.84	0.00659664193117337\\
82.85	0.00659889977444119\\
82.86	0.00660115748249902\\
82.87	0.00660341505253654\\
82.88	0.00660567248173855\\
82.89	0.00660792976728499\\
82.9	0.00661018690635091\\
82.91	0.00661244389610653\\
82.92	0.00661470073371723\\
82.93	0.00661695741634354\\
82.94	0.00661921394114118\\
82.95	0.00662147030526105\\
82.96	0.00662372650584924\\
82.97	0.00662598254004706\\
82.98	0.00662823840499103\\
82.99	0.00663049409781289\\
83	0.00663274961563963\\
83.01	0.00663500495559349\\
83.02	0.00663726011479195\\
83.03	0.00663951509034779\\
83.04	0.00664176987936906\\
83.05	0.00664402447895909\\
83.06	0.00664627888621653\\
83.07	0.00664853309823536\\
83.08	0.00665078711210485\\
83.09	0.00665304092490967\\
83.1	0.00665529453372979\\
83.11	0.00665754793564058\\
83.12	0.00665980112771278\\
83.13	0.00666205410701251\\
83.14	0.00666430687060132\\
83.15	0.00666655941553616\\
83.16	0.00666881173886942\\
83.17	0.00667106383764894\\
83.18	0.00667331570891801\\
83.19	0.0066755673497154\\
83.2	0.00667781875707537\\
83.21	0.00668006992802767\\
83.22	0.00668232085959758\\
83.23	0.00668457154880591\\
83.24	0.00668682199266901\\
83.25	0.00668907218819879\\
83.26	0.00669132213240277\\
83.27	0.006693571822284\\
83.28	0.0066958212548412\\
83.29	0.00669807042706867\\
83.3	0.00670031933595637\\
83.31	0.00670256797848992\\
83.32	0.00670481635165062\\
83.33	0.00670706445241542\\
83.34	0.00670931227775702\\
83.35	0.00671155982464382\\
83.36	0.00671380709003998\\
83.37	0.0067160540709054\\
83.38	0.00671830076419575\\
83.39	0.00672054716686254\\
83.4	0.00672279327585303\\
83.41	0.00672503908811035\\
83.42	0.00672728460057348\\
83.43	0.00672952981017725\\
83.44	0.00673177471385239\\
83.45	0.00673401930852554\\
83.46	0.00673626359111925\\
83.47	0.00673850755855202\\
83.48	0.00674075120773834\\
83.49	0.00674299453558866\\
83.5	0.00674523753900943\\
83.51	0.00674748021490316\\
83.52	0.00674972256016838\\
83.53	0.0067519645716997\\
83.54	0.00675420624638782\\
83.55	0.00675644758111955\\
83.56	0.00675868857277783\\
83.57	0.00676092921824177\\
83.58	0.00676316951438665\\
83.59	0.00676540945808395\\
83.6	0.00676764904620136\\
83.61	0.00676988827560284\\
83.62	0.00677212714314862\\
83.63	0.0067743656456952\\
83.64	0.00677660378009543\\
83.65	0.00677884154319847\\
83.66	0.00678107893184985\\
83.67	0.00678331594289154\\
83.68	0.00678555257316184\\
83.69	0.00678778881949558\\
83.7	0.00679002467872398\\
83.71	0.0067922601476748\\
83.72	0.00679449522317232\\
83.73	0.00679672990203731\\
83.74	0.00679896418108718\\
83.75	0.0068011980571359\\
83.76	0.00680343152699406\\
83.77	0.00680566458746892\\
83.78	0.00680789723536441\\
83.79	0.00681012946748117\\
83.8	0.00681236128061658\\
83.81	0.00681459267156477\\
83.82	0.00681682363711668\\
83.83	0.00681905417406006\\
83.84	0.00682128427917951\\
83.85	0.00682351394925652\\
83.86	0.00682574318106951\\
83.87	0.00682797197139379\\
83.88	0.00683020031700169\\
83.89	0.00683242821466253\\
83.9	0.00683465566114265\\
83.91	0.00683688265320546\\
83.92	0.00683910918761149\\
83.93	0.00684133526111837\\
83.94	0.00684356087048093\\
83.95	0.00684578601245114\\
83.96	0.00684801068377825\\
83.97	0.00685023488120875\\
83.98	0.00685245860148642\\
83.99	0.0068546818413524\\
84	0.00685690459754515\\
84.01	0.00685912686680056\\
84.02	0.00686134864585195\\
84.03	0.00686356993143011\\
84.04	0.00686579072026334\\
84.05	0.00686801100907745\\
84.06	0.00687023079459588\\
84.07	0.00687245007353965\\
84.08	0.00687466884262744\\
84.09	0.00687688709857564\\
84.1	0.00687910483809833\\
84.11	0.00688132205790739\\
84.12	0.00688353889481202\\
84.13	0.00688575550449723\\
84.14	0.00688797188330864\\
84.15	0.00689018802758196\\
84.16	0.00689240393364306\\
84.17	0.00689461959780786\\
84.18	0.0068968350163824\\
84.19	0.00689905018566276\\
84.2	0.00690126510193508\\
84.21	0.00690347976147553\\
84.22	0.00690569416055031\\
84.23	0.00690790829541559\\
84.24	0.00691012216231755\\
84.25	0.00691233575749234\\
84.26	0.00691454907716604\\
84.27	0.0069167621175547\\
84.28	0.00691897487486426\\
84.29	0.00692118734529059\\
84.3	0.00692339952501943\\
84.31	0.0069256114102264\\
84.32	0.00692782299707699\\
84.33	0.00693003428172653\\
84.34	0.00693224526032016\\
84.35	0.00693445592899286\\
84.36	0.00693666628386938\\
84.37	0.00693887632106428\\
84.38	0.00694108603668185\\
84.39	0.00694329542681618\\
84.4	0.00694550448755105\\
84.41	0.00694771321495999\\
84.42	0.00694992160510623\\
84.43	0.0069521296540427\\
84.44	0.006954337357812\\
84.45	0.00695654471244639\\
84.46	0.00695875171396777\\
84.47	0.00696095835838769\\
84.48	0.00696316464170733\\
84.49	0.00696537055991746\\
84.5	0.00696757610899843\\
84.51	0.00696978128492019\\
84.52	0.00697198608364223\\
84.53	0.00697419050111361\\
84.54	0.00697639453327292\\
84.55	0.00697859817604827\\
84.56	0.00698080142535727\\
84.57	0.00698300427710703\\
84.58	0.00698520672719415\\
84.59	0.00698740877150468\\
84.6	0.00698961040591415\\
84.61	0.0069918116262875\\
84.62	0.00699401242847912\\
84.63	0.00699621280833281\\
84.64	0.00699841276168178\\
84.65	0.00700061228434861\\
84.66	0.00700281137214527\\
84.67	0.00700501002087311\\
84.68	0.0070072082263228\\
84.69	0.00700940598427437\\
84.7	0.00701160329049719\\
84.71	0.00701380014074991\\
84.72	0.00701599653078053\\
84.73	0.0070181924563263\\
84.74	0.00702038791311378\\
84.75	0.00702258289685878\\
84.76	0.0070247774032664\\
84.77	0.00702697142803094\\
84.78	0.00702916496683599\\
84.79	0.00703135801535432\\
84.8	0.00703355056924793\\
84.81	0.00703574262416804\\
84.82	0.00703793417575504\\
84.83	0.00704012521963852\\
84.84	0.00704231575143723\\
84.85	0.0070445057667591\\
84.86	0.00704669526120121\\
84.87	0.00704888423034975\\
84.88	0.0070510726697801\\
84.89	0.00705326057505672\\
84.9	0.00705544794173322\\
84.91	0.00705763476535228\\
84.92	0.00705982104020222\\
84.93	0.00706200676037457\\
84.94	0.00706419191993808\\
84.95	0.00706637651293863\\
84.96	0.00706856053339917\\
84.97	0.00707074397531958\\
84.98	0.00707292683267664\\
84.99	0.00707510909942391\\
85	0.00707729076949167\\
85.01	0.00707947183678679\\
85.02	0.00708165229519269\\
85.03	0.00708383213856922\\
85.04	0.00708601136075258\\
85.05	0.00708818995555525\\
85.06	0.00709036791676585\\
85.07	0.00709254523814913\\
85.08	0.0070947219134458\\
85.09	0.00709689793637249\\
85.1	0.00709907330062166\\
85.11	0.00710124799986148\\
85.12	0.00710342202773574\\
85.13	0.0071055953778638\\
85.14	0.00710776804384045\\
85.15	0.00710994001923587\\
85.16	0.00711211129759548\\
85.17	0.00711428187243987\\
85.18	0.00711645173726472\\
85.19	0.00711862088554072\\
85.2	0.00712078931071343\\
85.21	0.00712295700620322\\
85.22	0.00712512396540515\\
85.23	0.0071272901816889\\
85.24	0.00712945564839868\\
85.25	0.00713162035885311\\
85.26	0.00713378430634512\\
85.27	0.0071359474841419\\
85.28	0.00713810988548474\\
85.29	0.00714027150358898\\
85.3	0.00714243233164389\\
85.31	0.0071445923628126\\
85.32	0.00714675159023195\\
85.33	0.00714891000701246\\
85.34	0.00715106760623816\\
85.35	0.00715322438096654\\
85.36	0.00715538032422843\\
85.37	0.00715753542902792\\
85.38	0.00715968968834223\\
85.39	0.00716184309512164\\
85.4	0.00716399564228935\\
85.41	0.00716614732274143\\
85.42	0.00716829812934667\\
85.43	0.0071704480549465\\
85.44	0.0071725970923549\\
85.45	0.00717474523435827\\
85.46	0.00717689247371535\\
85.47	0.00717903880315708\\
85.48	0.00718118421538656\\
85.49	0.00718332870307887\\
85.5	0.00718547225888103\\
85.51	0.00718761487541185\\
85.52	0.00718975654526185\\
85.53	0.00719189726099315\\
85.54	0.00719403701513933\\
85.55	0.0071961758002054\\
85.56	0.00719831360866759\\
85.57	0.00720045043297334\\
85.58	0.00720258626554113\\
85.59	0.00720472109876039\\
85.6	0.00720685492499141\\
85.61	0.00720898773656517\\
85.62	0.00721111952578332\\
85.63	0.007213250284918\\
85.64	0.00721538000621175\\
85.65	0.0072175086818774\\
85.66	0.00721963630409798\\
85.67	0.00722176286502657\\
85.68	0.0072238883567862\\
85.69	0.00722601277146977\\
85.7	0.00722813610113989\\
85.71	0.00723025833782879\\
85.72	0.00723237947353821\\
85.73	0.00723449950023927\\
85.74	0.00723661840987238\\
85.75	0.0072387361943471\\
85.76	0.00724085284554202\\
85.77	0.0072429683553047\\
85.78	0.00724508271545148\\
85.79	0.0072471959177674\\
85.8	0.00724930795400608\\
85.81	0.00725141881588961\\
85.82	0.00725352849510841\\
85.83	0.00725563698332114\\
85.84	0.00725774427215456\\
85.85	0.00725985035320341\\
85.86	0.00726195521803031\\
85.87	0.00726405885816562\\
85.88	0.00726616126510733\\
85.89	0.00726826243032092\\
85.9	0.00727036234523928\\
85.91	0.00727246100126256\\
85.92	0.00727455838975803\\
85.93	0.00727665450205999\\
85.94	0.00727874932946964\\
85.95	0.00728084286325494\\
85.96	0.00728293509465051\\
85.97	0.00728502601485748\\
85.98	0.00728711561504338\\
85.99	0.00728920388634204\\
86	0.00729129081985339\\
86.01	0.00729337640664342\\
86.02	0.00729546063774399\\
86.03	0.00729754350415274\\
86.04	0.00729962499683294\\
86.05	0.00730170510671339\\
86.06	0.00730378382468825\\
86.07	0.00730586114161695\\
86.08	0.00730793704832404\\
86.09	0.00731001153559907\\
86.1	0.00731208459419644\\
86.11	0.0073141562148353\\
86.12	0.00731622638819939\\
86.13	0.00731829510493696\\
86.14	0.00732036235566054\\
86.15	0.00732242813094691\\
86.16	0.00732449242133691\\
86.17	0.00732655521733532\\
86.18	0.00732861650941074\\
86.19	0.00733067628799542\\
86.2	0.00733273454348518\\
86.21	0.00733479126623921\\
86.22	0.00733684644657998\\
86.23	0.00733890007479311\\
86.24	0.00734095214112718\\
86.25	0.00734300263579366\\
86.26	0.00734505154896671\\
86.27	0.00734709887078311\\
86.28	0.00734914459134205\\
86.29	0.00735118870070503\\
86.3	0.00735323118889573\\
86.31	0.00735527204589984\\
86.32	0.00735731126166496\\
86.33	0.00735934882610041\\
86.34	0.00736138472907712\\
86.35	0.00736341896042749\\
86.36	0.00736545150994524\\
86.37	0.00736748236738524\\
86.38	0.00736951152246343\\
86.39	0.00737153896485663\\
86.4	0.0073735646842024\\
86.41	0.00737558867009889\\
86.42	0.00737761091210475\\
86.43	0.00737963139973889\\
86.44	0.00738165012248043\\
86.45	0.00738366706976849\\
86.46	0.00738568223100204\\
86.47	0.00738769559553983\\
86.48	0.00738970715270015\\
86.49	0.00739171689176072\\
86.5	0.00739372480195855\\
86.51	0.00739573087248978\\
86.52	0.00739773509250955\\
86.53	0.0073997374511318\\
86.54	0.00740173793742918\\
86.55	0.00740373654043286\\
86.56	0.00740573324913238\\
86.57	0.00740772805247553\\
86.58	0.00740972093936816\\
86.59	0.00741171189867405\\
86.6	0.00741370091921474\\
86.61	0.00741568798976939\\
86.62	0.00741767309907463\\
86.63	0.00741965623582438\\
86.64	0.00742163738866974\\
86.65	0.00742361654621878\\
86.66	0.00742559369703641\\
86.67	0.00742756882964423\\
86.68	0.00742954193252038\\
86.69	0.00743151299409935\\
86.7	0.00743348200277185\\
86.71	0.00743544894688464\\
86.72	0.00743741381474038\\
86.73	0.00743937659459745\\
86.74	0.00744133727466982\\
86.75	0.00744329584312687\\
86.76	0.00744525228809324\\
86.77	0.00744720659764865\\
86.78	0.00744915875982776\\
86.79	0.00745110876262001\\
86.8	0.00745305659396942\\
86.81	0.00745500224177448\\
86.82	0.00745694728972756\\
86.83	0.00745889242469601\\
86.84	0.0074608376463522\\
86.85	0.00746278295437578\\
86.86	0.00746472834845371\\
86.87	0.00746667382828035\\
86.88	0.0074686193935575\\
86.89	0.00747056504399447\\
86.9	0.00747251077930815\\
86.91	0.00747445659922308\\
86.92	0.00747640250347148\\
86.93	0.00747834849179335\\
86.94	0.00748029456393652\\
86.95	0.00748224071965672\\
86.96	0.00748418695871764\\
86.97	0.00748613328089099\\
86.98	0.00748807968595658\\
86.99	0.00749002617370239\\
87	0.00749197274392462\\
87.01	0.00749391939642775\\
87.02	0.00749586613102465\\
87.03	0.00749781294753659\\
87.04	0.00749975984579338\\
87.05	0.00750170682563335\\
87.06	0.00750365388690351\\
87.07	0.00750560102945955\\
87.08	0.00750754825316596\\
87.09	0.00750949555789604\\
87.1	0.00751144294353205\\
87.11	0.00751339040996523\\
87.12	0.00751533795709586\\
87.13	0.00751728558483338\\
87.14	0.00751923329309643\\
87.15	0.00752118108181292\\
87.16	0.00752312895092013\\
87.17	0.00752507690036475\\
87.18	0.00752702493010299\\
87.19	0.00752897304010061\\
87.2	0.00753092123033306\\
87.21	0.00753286950078548\\
87.22	0.00753481785145283\\
87.23	0.00753676628233996\\
87.24	0.00753871479346166\\
87.25	0.00754066338484275\\
87.26	0.00754261205651819\\
87.27	0.0075445608085331\\
87.28	0.00754650964094289\\
87.29	0.0075484585538133\\
87.3	0.00755040754722052\\
87.31	0.00755235662125122\\
87.32	0.00755430577600269\\
87.33	0.00755625501158287\\
87.34	0.00755820432811045\\
87.35	0.00756015372571497\\
87.36	0.00756210320453685\\
87.37	0.00756405276472755\\
87.38	0.00756600240644959\\
87.39	0.00756795212987666\\
87.4	0.0075699019351937\\
87.41	0.00757185182259698\\
87.42	0.0075738017922942\\
87.43	0.00757575184450454\\
87.44	0.00757770197945881\\
87.45	0.00757965219739947\\
87.46	0.00758160249858077\\
87.47	0.00758355288326878\\
87.48	0.00758550335174154\\
87.49	0.00758745390428912\\
87.5	0.00758940454121369\\
87.51	0.00759135526282964\\
87.52	0.00759330606946367\\
87.53	0.00759525696145485\\
87.54	0.00759720793915477\\
87.55	0.00759915900292753\\
87.56	0.00760111015314996\\
87.57	0.00760306139021162\\
87.58	0.00760501271451491\\
87.59	0.00760696412647521\\
87.6	0.0076089156265209\\
87.61	0.00761086721509354\\
87.62	0.00761281889264787\\
87.63	0.00761477065965201\\
87.64	0.00761672251658746\\
87.65	0.00761867446394927\\
87.66	0.00762062650224609\\
87.67	0.0076225786320003\\
87.68	0.00762453085374808\\
87.69	0.00762648316803953\\
87.7	0.00762843557543878\\
87.71	0.00763038807652405\\
87.72	0.0076323406718878\\
87.73	0.00763429336213677\\
87.74	0.00763624614789217\\
87.75	0.00763819902978969\\
87.76	0.00764015200847967\\
87.77	0.00764210508462716\\
87.78	0.00764405825891205\\
87.79	0.00764601153202919\\
87.8	0.00764796490468844\\
87.81	0.00764991837761483\\
87.82	0.00765187195154864\\
87.83	0.00765382562724552\\
87.84	0.00765577940547659\\
87.85	0.00765773328702853\\
87.86	0.00765968727270376\\
87.87	0.00766164136332044\\
87.88	0.00766359555971268\\
87.89	0.00766554986273059\\
87.9	0.00766750427324044\\
87.91	0.0076694587921247\\
87.92	0.00767141342028224\\
87.93	0.00767336815862837\\
87.94	0.007675323008095\\
87.95	0.00767727796963074\\
87.96	0.007679233044201\\
87.97	0.00768118823278815\\
87.98	0.00768314353639157\\
87.99	0.00768509895602783\\
88	0.00768705449273078\\
88.01	0.00768901014755165\\
88.02	0.00769096592155922\\
88.03	0.00769292181583989\\
88.04	0.00769487783149782\\
88.05	0.00769683396965505\\
88.06	0.00769879023145164\\
88.07	0.00770074661804575\\
88.08	0.00770270313061381\\
88.09	0.00770465977035062\\
88.1	0.00770661653846946\\
88.11	0.00770857343620224\\
88.12	0.00771053046479962\\
88.13	0.00771248762553114\\
88.14	0.00771444491968533\\
88.15	0.00771640234856985\\
88.16	0.00771835991351163\\
88.17	0.00772031761585696\\
88.18	0.00772227545697169\\
88.19	0.00772423343824126\\
88.2	0.00772619156107095\\
88.21	0.00772814982688591\\
88.22	0.00773010823713133\\
88.23	0.00773206679327262\\
88.24	0.00773402549679545\\
88.25	0.00773598434920597\\
88.26	0.00773794335203091\\
88.27	0.00773990250681769\\
88.28	0.00774186181513462\\
88.29	0.00774382127857099\\
88.3	0.00774578089873722\\
88.31	0.00774774067726499\\
88.32	0.00774970061580742\\
88.33	0.00775166071603915\\
88.34	0.00775362097965655\\
88.35	0.0077555814083778\\
88.36	0.00775754200394306\\
88.37	0.00775950276811462\\
88.38	0.00776146370267704\\
88.39	0.0077634248094373\\
88.4	0.00776538609022492\\
88.41	0.00776734754689214\\
88.42	0.00776930918131405\\
88.43	0.00777127099538876\\
88.44	0.00777323299103749\\
88.45	0.00777519517020482\\
88.46	0.00777715753485873\\
88.47	0.00777912008699083\\
88.48	0.00778108282861651\\
88.49	0.00778304576177502\\
88.5	0.00778500888852972\\
88.51	0.00778697221096816\\
88.52	0.00778893573120229\\
88.53	0.00779089945136859\\
88.54	0.00779286337362819\\
88.55	0.00779482750016713\\
88.56	0.00779679183319641\\
88.57	0.00779875637495222\\
88.58	0.00780072112769608\\
88.59	0.00780268609371497\\
88.6	0.00780465127532158\\
88.61	0.00780661667485435\\
88.62	0.00780858229467775\\
88.63	0.00781054813718236\\
88.64	0.0078125142047851\\
88.65	0.00781448049992934\\
88.66	0.00781644702508511\\
88.67	0.00781841378274927\\
88.68	0.00782038077544561\\
88.69	0.00782234800572514\\
88.7	0.00782431547616616\\
88.71	0.00782628318937444\\
88.72	0.00782825114798348\\
88.73	0.00783021935465458\\
88.74	0.00783218781207706\\
88.75	0.00783415652296846\\
88.76	0.00783612549007466\\
88.77	0.00783809471617011\\
88.78	0.00784006420405797\\
88.79	0.00784203395657032\\
88.8	0.00784400397656831\\
88.81	0.00784597426694237\\
88.82	0.00784794483061236\\
88.83	0.0078499156705278\\
88.84	0.00785188678966801\\
88.85	0.0078538581910423\\
88.86	0.00785582987769019\\
88.87	0.00785780185268154\\
88.88	0.00785977411911682\\
88.89	0.0078617466801272\\
88.9	0.00786371953887482\\
88.91	0.00786569269855294\\
88.92	0.00786766616238615\\
88.93	0.00786963993363053\\
88.94	0.00787161401557389\\
88.95	0.00787358841153594\\
88.96	0.00787556312486849\\
88.97	0.00787753815895564\\
88.98	0.00787951351721397\\
88.99	0.00788148920309276\\
89	0.0078834652200742\\
89.01	0.00788544157167353\\
89.02	0.00788741826143932\\
89.03	0.00788939529295362\\
89.04	0.00789137266983216\\
89.05	0.00789335039572461\\
89.06	0.00789532847431472\\
89.07	0.00789730690932056\\
89.08	0.00789928570449472\\
89.09	0.00790126486362455\\
89.1	0.0079032443905323\\
89.11	0.00790522428907539\\
89.12	0.00790720456314661\\
89.13	0.00790918521667433\\
89.14	0.00791116625362269\\
89.15	0.00791314767799186\\
89.16	0.00791512949381821\\
89.17	0.00791711170517456\\
89.18	0.0079190943161704\\
89.19	0.00792107732930387\\
89.2	0.00792306074420649\\
89.21	0.00792504456050932\\
89.22	0.00792702877784298\\
89.23	0.00792901339583763\\
89.24	0.00793099841412301\\
89.25	0.00793298383232837\\
89.26	0.00793496965008253\\
89.27	0.00793695586701381\\
89.28	0.00793894248275008\\
89.29	0.00794092949691874\\
89.3	0.00794291690914668\\
89.31	0.00794490471906033\\
89.32	0.00794689292628563\\
89.33	0.00794888153044799\\
89.34	0.00795087053117234\\
89.35	0.00795285992808312\\
89.36	0.00795484972080421\\
89.37	0.007956839908959\\
89.38	0.00795883049217036\\
89.39	0.00796082147006061\\
89.4	0.00796281284225154\\
89.41	0.00796480460836439\\
89.42	0.00796679676801986\\
89.43	0.0079687893208381\\
89.44	0.00797078226643868\\
89.45	0.00797277560444062\\
89.46	0.00797476933446234\\
89.47	0.00797676345612171\\
89.48	0.00797875796903599\\
89.49	0.00798075287282186\\
89.5	0.00798274816709538\\
89.51	0.00798474385147203\\
89.52	0.00798673992556665\\
89.53	0.00798873638899347\\
89.54	0.00799073324136609\\
89.55	0.00799273048229747\\
89.56	0.00799472811139993\\
89.57	0.00799672612828515\\
89.58	0.00799872453256412\\
89.59	0.00800072332384721\\
89.6	0.00800272250174408\\
89.61	0.00800472206586372\\
89.62	0.00800672201581444\\
89.63	0.00800872235120384\\
89.64	0.00801072307163883\\
89.65	0.0080127241767256\\
89.66	0.00801472566606961\\
89.67	0.00801672753927561\\
89.68	0.00801872979594759\\
89.69	0.00802073243568883\\
89.7	0.00802273545810181\\
89.71	0.00802473886278827\\
89.72	0.00802674264934919\\
89.73	0.00802874681738475\\
89.74	0.00803075136649435\\
89.75	0.00803275629627658\\
89.76	0.00803476160632923\\
89.77	0.00803676729624929\\
89.78	0.00803877336563289\\
89.79	0.00804077981407533\\
89.8	0.0080427866411711\\
89.81	0.00804479384651378\\
89.82	0.00804680142969614\\
89.83	0.00804880939031002\\
89.84	0.00805081772794641\\
89.85	0.00805282644219541\\
89.86	0.00805483553264618\\
89.87	0.00805684499888699\\
89.88	0.00805885484050518\\
89.89	0.00806086505708714\\
89.9	0.00806287564821834\\
89.91	0.00806488661348325\\
89.92	0.0080668979524654\\
89.93	0.00806890966474732\\
89.94	0.00807092174991057\\
89.95	0.00807293420753568\\
89.96	0.00807494703720219\\
89.97	0.00807696023848859\\
89.98	0.00807897381097235\\
89.99	0.00808098775422988\\
90	0.00808300206783652\\
90.01	0.00808501675136655\\
90.02	0.00808703180439315\\
90.03	0.00808904722648841\\
90.04	0.00809106301722331\\
90.05	0.00809307917616769\\
90.06	0.00809509570289028\\
90.07	0.00809711259695862\\
90.08	0.00809912985793912\\
90.09	0.008101147485397\\
90.1	0.0081031654788963\\
90.11	0.00810518383799984\\
90.12	0.00810720256226923\\
90.13	0.00810922165126486\\
90.14	0.00811124110454586\\
90.15	0.00811326092167012\\
90.16	0.00811528110219422\\
90.17	0.00811730164567349\\
90.18	0.00811932255166195\\
90.19	0.0081213438197123\\
90.2	0.0081233654493759\\
90.21	0.00812538744020277\\
90.22	0.00812740979174158\\
90.23	0.0081294325035396\\
90.24	0.00813145557514274\\
90.25	0.00813347900609547\\
90.26	0.00813550279594086\\
90.27	0.00813752694422053\\
90.28	0.00813955145047464\\
90.29	0.00814157631424189\\
90.3	0.00814360153505949\\
90.31	0.00814562711246315\\
90.32	0.00814765304598703\\
90.33	0.00814967933516381\\
90.34	0.00815170597952455\\
90.35	0.00815373297859879\\
90.36	0.00815576033191444\\
90.37	0.00815778803899784\\
90.38	0.00815981609937368\\
90.39	0.00816184451256501\\
90.4	0.00816387327809323\\
90.41	0.00816590239547807\\
90.42	0.00816793186423753\\
90.43	0.00816996168388792\\
90.44	0.00817199185394381\\
90.45	0.00817402237391801\\
90.46	0.00817605324332157\\
90.47	0.00817808446166374\\
90.48	0.00818011602845195\\
90.49	0.00818214794319182\\
90.5	0.00818418020538711\\
90.51	0.00818621281453969\\
90.52	0.00818824577014956\\
90.53	0.00819027907171482\\
90.54	0.00819231271873159\\
90.55	0.00819434671069409\\
90.56	0.00819638104709454\\
90.57	0.00819841572742316\\
90.58	0.00820045075116816\\
90.59	0.00820248611781571\\
90.6	0.00820452182684992\\
90.61	0.00820655787775283\\
90.62	0.00820859427000433\\
90.63	0.00821063100308224\\
90.64	0.0082126680764622\\
90.65	0.00821470548961767\\
90.66	0.00821674324201993\\
90.67	0.00821878133313804\\
90.68	0.0082208197624388\\
90.69	0.00822285852938675\\
90.7	0.00822489763344416\\
90.71	0.00822693707407095\\
90.72	0.00822897685072472\\
90.73	0.00823101696286071\\
90.74	0.00823305740993175\\
90.75	0.00823509819138827\\
90.76	0.00823713930667828\\
90.77	0.00823918075505629\\
90.78	0.00824122253549515\\
90.79	0.00824326464696576\\
90.8	0.00824530708843713\\
90.81	0.0082473498588763\\
90.82	0.00824939295724842\\
90.83	0.00825143638251671\\
90.84	0.00825348013364246\\
90.85	0.00825552420958505\\
90.86	0.00825756860930192\\
90.87	0.00825961333174863\\
90.88	0.00826165837587878\\
90.89	0.00826370374064407\\
90.9	0.00826574942499429\\
90.91	0.00826779542787729\\
90.92	0.00826984174823902\\
90.93	0.00827188838502354\\
90.94	0.00827393533717295\\
90.95	0.00827598260362745\\
90.96	0.00827803018332536\\
90.97	0.00828007807520305\\
90.98	0.00828212627819499\\
90.99	0.00828417479123376\\
91	0.00828622361325002\\
91.01	0.0082882727431725\\
91.02	0.00829032217992807\\
91.03	0.00829237192244166\\
91.04	0.00829442196963631\\
91.05	0.00829647232043315\\
91.06	0.00829852297375143\\
91.07	0.00830057392850846\\
91.08	0.00830262518361968\\
91.09	0.00830467673799865\\
91.1	0.00830672859055698\\
91.11	0.00830878074020443\\
91.12	0.00831083318584886\\
91.13	0.00831288592639621\\
91.14	0.00831493896075057\\
91.15	0.0083169922878141\\
91.16	0.0083190459064871\\
91.17	0.00832109981566797\\
91.18	0.00832315401425323\\
91.19	0.00832520850113752\\
91.2	0.00832726327521359\\
91.21	0.00832931833537231\\
91.22	0.00833137368050266\\
91.23	0.00833342930949177\\
91.24	0.00833548522122487\\
91.25	0.00833754141458534\\
91.26	0.00833959788845465\\
91.27	0.00834165464171244\\
91.28	0.00834371167323644\\
91.29	0.00834576898190255\\
91.3	0.00834782656658478\\
91.31	0.00834988442615529\\
91.32	0.00835194255948437\\
91.33	0.00835400096544046\\
91.34	0.00835605964289011\\
91.35	0.00835811859069806\\
91.36	0.00836017780772716\\
91.37	0.00836223729283843\\
91.38	0.00836429704489101\\
91.39	0.00836635706274224\\
91.4	0.00836841734524756\\
91.41	0.0083704778912606\\
91.42	0.00837253869963315\\
91.43	0.00837459976921512\\
91.44	0.00837666109885464\\
91.45	0.00837872268739795\\
91.46	0.0083807845336895\\
91.47	0.00838284663657187\\
91.48	0.00838490899488586\\
91.49	0.00838697160747039\\
91.5	0.00838903447316258\\
91.51	0.00839109759079775\\
91.52	0.00839316095920935\\
91.53	0.00839522457722906\\
91.54	0.00839728844368673\\
91.55	0.00839935255741039\\
91.56	0.00840141691722626\\
91.57	0.00840348152195876\\
91.58	0.00840554637043051\\
91.59	0.00840761146146232\\
91.6	0.00840967679387321\\
91.61	0.00841174236648039\\
91.62	0.00841380817809928\\
91.63	0.00841587422754352\\
91.64	0.00841794051362496\\
91.65	0.00842000703515365\\
91.66	0.00842207379093787\\
91.67	0.00842414077978411\\
91.68	0.0084262080004971\\
91.69	0.00842827545187979\\
91.7	0.00843034313273334\\
91.71	0.00843241104185718\\
91.72	0.00843447917804894\\
91.73	0.0084365475401045\\
91.74	0.00843861612681799\\
91.75	0.00844068493698178\\
91.76	0.00844275396938647\\
91.77	0.00844482322282095\\
91.78	0.00844689269607233\\
91.79	0.00844896238792599\\
91.8	0.00845103229716557\\
91.81	0.00845310242257297\\
91.82	0.00845517276292836\\
91.83	0.0084572433170102\\
91.84	0.00845931408359519\\
91.85	0.00846138506145832\\
91.86	0.00846345624937289\\
91.87	0.00846552764611044\\
91.88	0.00846759925044083\\
91.89	0.00846967106113219\\
91.9	0.00847174307695097\\
91.91	0.00847381529666191\\
91.92	0.00847588771902805\\
91.93	0.00847796034281073\\
91.94	0.00848003316676962\\
91.95	0.00848210618966269\\
91.96	0.00848417941024623\\
91.97	0.00848625282727485\\
91.98	0.00848832643950151\\
91.99	0.00849040024567746\\
92	0.00849247424455233\\
92.01	0.00849454843487404\\
92.02	0.00849662281538891\\
92.03	0.00849869738484155\\
92.04	0.00850077214197495\\
92.05	0.00850284708553047\\
92.06	0.00850492221424779\\
92.07	0.00850699752686498\\
92.08	0.00850907302211849\\
92.09	0.00851114869874311\\
92.1	0.00851322455547203\\
92.11	0.00851530059103681\\
92.12	0.00851737680416741\\
92.13	0.00851945319359216\\
92.14	0.00852152975803781\\
92.15	0.00852360649622948\\
92.16	0.00852568340689073\\
92.17	0.00852776048874349\\
92.18	0.00852983774050814\\
92.19	0.00853191516090345\\
92.2	0.00853399274864664\\
92.21	0.00853607050245333\\
92.22	0.0085381484210376\\
92.23	0.00854022650311195\\
92.24	0.00854230474738733\\
92.25	0.00854438315257314\\
92.26	0.00854646171737722\\
92.27	0.00854854044050589\\
92.28	0.00855061932066392\\
92.29	0.00855269835655454\\
92.3	0.00855477754687948\\
92.31	0.00855685689033894\\
92.32	0.00855893638563158\\
92.33	0.00856101603145457\\
92.34	0.00856309582650359\\
92.35	0.0085651757694728\\
92.36	0.00856725585905487\\
92.37	0.00856933609394099\\
92.38	0.00857141647282085\\
92.39	0.00857349699438267\\
92.4	0.00857557765731323\\
92.41	0.00857765846029779\\
92.42	0.00857973940202021\\
92.43	0.00858182048116283\\
92.44	0.0085839016964066\\
92.45	0.008585983046431\\
92.46	0.00858806452991408\\
92.47	0.00859014614553246\\
92.48	0.00859222789196133\\
92.49	0.00859430976787447\\
92.5	0.00859639177194425\\
92.51	0.00859847390284163\\
92.52	0.00860055615923617\\
92.53	0.00860263853979604\\
92.54	0.00860472104318802\\
92.55	0.00860680366807751\\
92.56	0.00860888641312853\\
92.57	0.00861096927700376\\
92.58	0.0086130522583645\\
92.59	0.00861513535587067\\
92.6	0.00861721856818089\\
92.61	0.00861930189395242\\
92.62	0.00862138533184117\\
92.63	0.00862346888050175\\
92.64	0.00862555253858743\\
92.65	0.00862763630475017\\
92.66	0.00862972017764062\\
92.67	0.00863180415590815\\
92.68	0.00863388823820082\\
92.69	0.0086359724231654\\
92.7	0.0086380567094474\\
92.71	0.00864014109569104\\
92.72	0.00864222558053928\\
92.73	0.00864431016263385\\
92.74	0.00864639484061519\\
92.75	0.00864847961312253\\
92.76	0.00865056447879383\\
92.77	0.00865264943626586\\
92.78	0.00865473448417417\\
92.79	0.00865681962115305\\
92.8	0.00865890484583565\\
92.81	0.00866099015685389\\
92.82	0.00866307555283848\\
92.83	0.00866516103241901\\
92.84	0.00866724659422384\\
92.85	0.00866933223688018\\
92.86	0.00867141795901411\\
92.87	0.00867350375925053\\
92.88	0.00867558963621321\\
92.89	0.00867767558852478\\
92.9	0.00867976161480676\\
92.91	0.00868184771367954\\
92.92	0.00868393388376241\\
92.93	0.00868602012367357\\
92.94	0.00868810643203009\\
92.95	0.00869019280744801\\
92.96	0.00869227924854226\\
92.97	0.00869436575392671\\
92.98	0.00869645232221418\\
92.99	0.00869853895201643\\
93	0.00870062564194421\\
93.01	0.00870271239060721\\
93.02	0.0087047991966141\\
93.03	0.00870688605857254\\
93.04	0.0087089729750892\\
93.05	0.00871105994476976\\
93.06	0.00871314696621888\\
93.07	0.00871523403804026\\
93.08	0.00871732115883666\\
93.09	0.00871940832720985\\
93.1	0.00872149554176066\\
93.11	0.00872358280108898\\
93.12	0.00872567010379378\\
93.13	0.00872775744847311\\
93.14	0.0087298448337241\\
93.15	0.00873193225814298\\
93.16	0.0087340197203251\\
93.17	0.00873610721886492\\
93.18	0.00873819475235604\\
93.19	0.0087402823193912\\
93.2	0.00874236991856227\\
93.21	0.00874445754846029\\
93.22	0.00874654520767549\\
93.23	0.00874863289479724\\
93.24	0.00875072060841415\\
93.25	0.00875280834711399\\
93.26	0.00875489610948376\\
93.27	0.00875698389410968\\
93.28	0.00875907169957719\\
93.29	0.008761159524471\\
93.3	0.00876324736737505\\
93.31	0.00876533522687256\\
93.32	0.00876742310154601\\
93.33	0.00876951098997719\\
93.34	0.00877159889074716\\
93.35	0.00877368680243631\\
93.36	0.00877577472362434\\
93.37	0.0087778626528903\\
93.38	0.00877995058881255\\
93.39	0.00878203852996884\\
93.4	0.00878412647493625\\
93.41	0.00878621442229127\\
93.42	0.00878830237060975\\
93.43	0.00879039031846697\\
93.44	0.0087924782644376\\
93.45	0.00879456620709575\\
93.46	0.00879665414501495\\
93.47	0.00879874207676819\\
93.48	0.00880083000092791\\
93.49	0.00880291791606604\\
93.5	0.00880500582075399\\
93.51	0.00880709371356264\\
93.52	0.00880918159306242\\
93.53	0.00881126945782326\\
93.54	0.00881335730641463\\
93.55	0.00881544513740553\\
93.56	0.00881753294936456\\
93.57	0.00881962074085985\\
93.58	0.00882170851045915\\
93.59	0.00882379625672977\\
93.6	0.00882588397823867\\
93.61	0.00882797167355241\\
93.62	0.0088300593412372\\
93.63	0.0088321469798589\\
93.64	0.00883423458798302\\
93.65	0.00883632216417476\\
93.66	0.00883840970699901\\
93.67	0.00884049721502037\\
93.68	0.00884258468680313\\
93.69	0.00884467212091134\\
93.7	0.00884675951590879\\
93.71	0.00884884687035903\\
93.72	0.00885093418282536\\
93.73	0.0088530214518709\\
93.74	0.00885510867605856\\
93.75	0.00885719585395106\\
93.76	0.00885928298411094\\
93.77	0.00886137006510062\\
93.78	0.00886345709548233\\
93.79	0.00886554407381821\\
93.8	0.00886763099867029\\
93.81	0.00886971786860047\\
93.82	0.00887180468217059\\
93.83	0.00887389143794243\\
93.84	0.0088759781344777\\
93.85	0.00887806477033809\\
93.86	0.00888015134408524\\
93.87	0.00888223785428083\\
93.88	0.0088843242994865\\
93.89	0.00888641067826394\\
93.9	0.00888849698917488\\
93.91	0.0088905832307811\\
93.92	0.00889266940164446\\
93.93	0.00889475550032689\\
93.94	0.00889684152539044\\
93.95	0.00889892747539728\\
93.96	0.0089010133489097\\
93.97	0.00890309914449017\\
93.98	0.0089051848607013\\
93.99	0.0089072704961059\\
94	0.00890935604926699\\
94.01	0.00891144151874779\\
94.02	0.00891352690311177\\
94.03	0.00891561220092265\\
94.04	0.00891769741074441\\
94.05	0.00891978253114133\\
94.06	0.00892186756067799\\
94.07	0.00892395249791929\\
94.08	0.00892603734143048\\
94.09	0.00892812208977715\\
94.1	0.00893020674152527\\
94.11	0.00893229129524122\\
94.12	0.00893437574949177\\
94.13	0.00893646010284411\\
94.14	0.00893854435386591\\
94.15	0.00894062850112529\\
94.16	0.00894271254319083\\
94.17	0.00894479647863166\\
94.18	0.00894688030601739\\
94.19	0.00894896402391818\\
94.2	0.00895104763090477\\
94.21	0.00895313112554844\\
94.22	0.00895521450642111\\
94.23	0.00895729777209528\\
94.24	0.0089593809211441\\
94.25	0.00896146395214137\\
94.26	0.00896354686366159\\
94.27	0.00896562965427991\\
94.28	0.00896771232257223\\
94.29	0.00896979486711518\\
94.3	0.00897187728648612\\
94.31	0.00897395957926322\\
94.32	0.00897604174402542\\
94.33	0.00897812377935248\\
94.34	0.00898020568382501\\
94.35	0.00898228745602445\\
94.36	0.00898436909453316\\
94.37	0.00898645059793436\\
94.38	0.00898853196481221\\
94.39	0.00899061319375181\\
94.4	0.00899269428333921\\
94.41	0.00899477523216148\\
94.42	0.00899685603880665\\
94.43	0.00899893670186382\\
94.44	0.00900101721992311\\
94.45	0.00900309759157573\\
94.46	0.009005177815414\\
94.47	0.00900725789003131\\
94.48	0.00900933781402223\\
94.49	0.0090114175859825\\
94.5	0.00901349720450901\\
94.51	0.00901557666819989\\
94.52	0.00901765597565448\\
94.53	0.0090197351254734\\
94.54	0.00902181411625854\\
94.55	0.00902389294661307\\
94.56	0.00902597161514153\\
94.57	0.00902805012044977\\
94.58	0.00903012846114504\\
94.59	0.00903220663583598\\
94.6	0.00903428464313266\\
94.61	0.00903636248164659\\
94.62	0.00903844014999076\\
94.63	0.00904051764677967\\
94.64	0.00904259497062931\\
94.65	0.00904467212015724\\
94.66	0.00904674909398261\\
94.67	0.00904882589072612\\
94.68	0.00905090250901014\\
94.69	0.00905297894745864\\
94.7	0.00905505520469728\\
94.71	0.00905713127935337\\
94.72	0.00905920717005597\\
94.73	0.00906128287543585\\
94.74	0.00906335839412553\\
94.75	0.00906543372475931\\
94.76	0.00906750886597329\\
94.77	0.00906958381640541\\
94.78	0.00907165857469543\\
94.79	0.00907373313948499\\
94.8	0.00907580750941763\\
94.81	0.0090778816831388\\
94.82	0.00907995565929589\\
94.83	0.00908202943653827\\
94.84	0.00908410301351727\\
94.85	0.00908617638888625\\
94.86	0.00908824956130061\\
94.87	0.00909032252941781\\
94.88	0.00909239529189738\\
94.89	0.00909446784740098\\
94.9	0.00909654019459238\\
94.91	0.00909861233213755\\
94.92	0.00910068425870458\\
94.93	0.00910275597296383\\
94.94	0.00910482747358786\\
94.95	0.0091068987592515\\
94.96	0.00910896982863185\\
94.97	0.00911104068040833\\
94.98	0.0091131113132627\\
94.99	0.00911518172587906\\
95	0.0091172519169439\\
95.01	0.00911932188514614\\
95.02	0.00912139162917712\\
95.03	0.00912346114773062\\
95.04	0.00912553043950296\\
95.05	0.00912759950319293\\
95.06	0.00912966833750187\\
95.07	0.0091317369411337\\
95.08	0.00913380531279492\\
95.09	0.00913587345119465\\
95.1	0.00913794135504466\\
95.11	0.00914000902305939\\
95.12	0.00914207645395597\\
95.13	0.00914414364645428\\
95.14	0.00914621059927692\\
95.15	0.0091482773111493\\
95.16	0.00915034378079961\\
95.17	0.0091524100069589\\
95.18	0.00915447598836106\\
95.19	0.00915654172374289\\
95.2	0.00915860721184408\\
95.21	0.00916067245140728\\
95.22	0.00916273744117813\\
95.23	0.00916480217990523\\
95.24	0.00916686666634024\\
95.25	0.00916893089923786\\
95.26	0.00917099487735588\\
95.27	0.0091730585994552\\
95.28	0.00917512206429987\\
95.29	0.00917718527065709\\
95.3	0.00917924821729728\\
95.31	0.00918131090299407\\
95.32	0.00918337332652436\\
95.33	0.00918543548666832\\
95.34	0.00918749738220945\\
95.35	0.00918955901193456\\
95.36	0.00919162037463389\\
95.37	0.00919368146910102\\
95.38	0.009195742294133\\
95.39	0.00919780284853032\\
95.4	0.00919986313109697\\
95.41	0.00920192314064047\\
95.42	0.00920398287597186\\
95.43	0.00920604233590579\\
95.44	0.0092081015192605\\
95.45	0.00921016042485788\\
95.46	0.00921221905152349\\
95.47	0.00921427739808658\\
95.48	0.00921633546338014\\
95.49	0.00921839324624094\\
95.5	0.0092204507455095\\
95.51	0.0092225079600302\\
95.52	0.00922456488865125\\
95.53	0.00922662153022477\\
95.54	0.00922867788360678\\
95.55	0.00923073394765725\\
95.56	0.00923278972124014\\
95.57	0.00923484520322339\\
95.58	0.00923690039247902\\
95.59	0.0092389552878831\\
95.6	0.00924100988831582\\
95.61	0.00924306419266149\\
95.62	0.00924511819980861\\
95.63	0.00924717190864987\\
95.64	0.00924922531808219\\
95.65	0.00925127842700677\\
95.66	0.0092533312343291\\
95.67	0.00925538373895899\\
95.68	0.00925743593981064\\
95.69	0.00925948783580262\\
95.7	0.00926153942585795\\
95.71	0.0092635907089041\\
95.72	0.00926564168387303\\
95.73	0.00926769234970125\\
95.74	0.00926974270532982\\
95.75	0.00927179274970439\\
95.76	0.00927384248177523\\
95.77	0.00927589190049731\\
95.78	0.00927794100483025\\
95.79	0.00927998979373843\\
95.8	0.00928203826619098\\
95.81	0.00928408642116183\\
95.82	0.00928613425762974\\
95.83	0.00928818177457835\\
95.84	0.00929022897099618\\
95.85	0.0092922758458767\\
95.86	0.00929432239821833\\
95.87	0.00929636862702452\\
95.88	0.00929841453130373\\
95.89	0.00930046011006951\\
95.9	0.00930250536234053\\
95.91	0.00930455028714057\\
95.92	0.00930659488349861\\
95.93	0.00930863915044885\\
95.94	0.00931068308703072\\
95.95	0.00931272669228895\\
95.96	0.00931476996527357\\
95.97	0.00931681290503999\\
95.98	0.00931885551064899\\
95.99	0.00932089778116679\\
96	0.00932293971566507\\
96.01	0.00932498131322101\\
96.02	0.00932702257291732\\
96.03	0.00932906349384229\\
96.04	0.0093311040750898\\
96.05	0.0093331443157594\\
96.06	0.00933518421495629\\
96.07	0.00933722377179142\\
96.08	0.00933926298538147\\
96.09	0.00934130185484892\\
96.1	0.00934334037932209\\
96.11	0.00934537855793513\\
96.12	0.00934741638982813\\
96.13	0.00934945387414709\\
96.14	0.00935149101004401\\
96.15	0.0093535277966769\\
96.16	0.0093555642332098\\
96.17	0.00935760031881286\\
96.18	0.00935963605266235\\
96.19	0.00936167143394072\\
96.2	0.00936370646183658\\
96.21	0.00936574113554484\\
96.22	0.00936777545426664\\
96.23	0.00936980941720946\\
96.24	0.00937184302358714\\
96.25	0.00937387627261988\\
96.26	0.00937590916353435\\
96.27	0.00937794169556369\\
96.28	0.00937997386794751\\
96.29	0.00938200567993201\\
96.3	0.00938403713076996\\
96.31	0.00938606821972075\\
96.32	0.00938809894605045\\
96.33	0.00939012930903182\\
96.34	0.00939215930794437\\
96.35	0.0093941889420744\\
96.36	0.009396218210715\\
96.37	0.00939824711316617\\
96.38	0.00940027564873478\\
96.39	0.00940230381673464\\
96.4	0.00940433161648654\\
96.41	0.00940635904731832\\
96.42	0.00940838610856484\\
96.43	0.00941041279956807\\
96.44	0.00941243911967715\\
96.45	0.00941446506824837\\
96.46	0.00941649064464523\\
96.47	0.00941851584823854\\
96.48	0.00942054067840636\\
96.49	0.00942256513453411\\
96.5	0.00942458921601461\\
96.51	0.00942661292224808\\
96.52	0.00942863625264222\\
96.53	0.00943065920661222\\
96.54	0.00943268178358082\\
96.55	0.00943470398297836\\
96.56	0.00943672580424279\\
96.57	0.00943874724681973\\
96.58	0.00944076831016253\\
96.59	0.00944278899373227\\
96.6	0.00944480929699783\\
96.61	0.00944682921943593\\
96.62	0.00944884876053116\\
96.63	0.00945086791977602\\
96.64	0.00945288669667098\\
96.65	0.00945490509072452\\
96.66	0.00945692310145313\\
96.67	0.00945894072838142\\
96.68	0.0094609579710421\\
96.69	0.00946297482897606\\
96.7	0.00946499130173242\\
96.71	0.00946700738886851\\
96.72	0.00946902308994998\\
96.73	0.00947103840455083\\
96.74	0.00947305333225341\\
96.75	0.00947506787264852\\
96.76	0.00947708202533539\\
96.77	0.0094790957899218\\
96.78	0.00948110916602406\\
96.79	0.00948312215326705\\
96.8	0.00948513475128432\\
96.81	0.00948714695971807\\
96.82	0.00948915877821925\\
96.83	0.00949117020644755\\
96.84	0.00949318124407148\\
96.85	0.00949519189076839\\
96.86	0.00949720214622452\\
96.87	0.00949921201013507\\
96.88	0.00950122148220419\\
96.89	0.00950323056214506\\
96.9	0.00950523924967994\\
96.91	0.00950724754454018\\
96.92	0.0095092554464663\\
96.93	0.009511262955208\\
96.94	0.00951327007052423\\
96.95	0.00951527679218322\\
96.96	0.00951728311996252\\
96.97	0.00951928905364905\\
96.98	0.00952129459303916\\
96.99	0.00952329973793864\\
97	0.00952530448816279\\
97.01	0.00952730884353645\\
97.02	0.00952931280389405\\
97.03	0.00953131636907965\\
97.04	0.00953331953894699\\
97.05	0.00953532231335953\\
97.06	0.0095373246921905\\
97.07	0.00953932667532292\\
97.08	0.00954132826264969\\
97.09	0.00954332945407357\\
97.1	0.0095453302495073\\
97.11	0.00954733064887357\\
97.12	0.00954933065210512\\
97.13	0.00955133025914475\\
97.14	0.0095533294699454\\
97.15	0.00955532828447014\\
97.16	0.00955732670269225\\
97.17	0.00955932472459529\\
97.18	0.00956132235017309\\
97.19	0.0095633195794298\\
97.2	0.00956531641238\\
97.21	0.00956731284904864\\
97.22	0.0095693088894712\\
97.23	0.00957130453369362\\
97.24	0.00957329978177244\\
97.25	0.00957529463377479\\
97.26	0.00957728908977844\\
97.27	0.00957928314987185\\
97.28	0.00958127681415425\\
97.29	0.0095832700827356\\
97.3	0.00958526295573674\\
97.31	0.00958725543328932\\
97.32	0.00958924751553597\\
97.33	0.00959123920263022\\
97.34	0.00959323049473664\\
97.35	0.00959522139203083\\
97.36	0.00959721189469948\\
97.37	0.00959920200294044\\
97.38	0.0096011917169627\\
97.39	0.0096031810369865\\
97.4	0.00960516996324335\\
97.41	0.00960715849597607\\
97.42	0.00960914663543882\\
97.43	0.00961113438189718\\
97.44	0.00961312173562817\\
97.45	0.00961510869692029\\
97.46	0.00961709526607359\\
97.47	0.00961908144339969\\
97.48	0.00962106722922182\\
97.49	0.00962305262387489\\
97.5	0.00962503762770551\\
97.51	0.00962702224107204\\
97.52	0.00962900646434466\\
97.53	0.00963099029790537\\
97.54	0.00963297374214805\\
97.55	0.00963495679747852\\
97.56	0.00963693946431458\\
97.57	0.00963892174308601\\
97.58	0.0096409036342347\\
97.59	0.0096428851382146\\
97.6	0.00964486625549183\\
97.61	0.00964684698654469\\
97.62	0.0096488273318637\\
97.63	0.00965080729195169\\
97.64	0.00965278686732378\\
97.65	0.00965476605850744\\
97.66	0.00965674486604252\\
97.67	0.00965872329048132\\
97.68	0.0096607013323886\\
97.69	0.00966267899234164\\
97.7	0.00966465627093029\\
97.71	0.00966663316875698\\
97.72	0.00966860968643683\\
97.73	0.0096705858245976\\
97.74	0.00967256158387982\\
97.75	0.00967453696493679\\
97.76	0.00967651196843432\\
97.77	0.00967848659505067\\
97.78	0.00968046084547652\\
97.79	0.00968243472041504\\
97.8	0.00968440822059293\\
97.81	0.00968638134680899\\
97.82	0.00968835409987373\\
97.83	0.0096903264806095\\
97.84	0.0096922984898504\\
97.85	0.00969427012844236\\
97.86	0.00969624139724315\\
97.87	0.00969821229712239\\
97.88	0.00970018282896157\\
97.89	0.00970215299365407\\
97.9	0.00970412279366507\\
97.91	0.00970609223162047\\
97.92	0.00970806131017858\\
97.93	0.00971003003203033\\
97.94	0.00971199839989956\\
97.95	0.00971396641654319\\
97.96	0.00971593408475148\\
97.97	0.00971790140734819\\
97.98	0.00971986838719079\\
97.99	0.00972183502717071\\
98	0.00972380133021363\\
98.01	0.00972576718884031\\
98.02	0.0097277324437127\\
98.03	0.00972969709868626\\
98.04	0.00973166115766945\\
98.05	0.0097336246246242\\
98.06	0.00973558750356637\\
98.07	0.00973754979903053\\
98.08	0.00973950657959474\\
98.09	0.00974145638196767\\
98.1	0.00974339914775435\\
98.11	0.00974533481796013\\
98.12	0.00974726333298333\\
98.13	0.00974918463260789\\
98.14	0.00975109865599576\\
98.15	0.0097530053416793\\
98.16	0.00975490462755354\\
98.17	0.00975679645086824\\
98.18	0.00975868074821993\\
98.19	0.00976055706691691\\
98.2	0.00976242524915769\\
98.21	0.00976428484382074\\
98.22	0.009766134034081\\
98.23	0.0097679727397121\\
98.24	0.00976980087968108\\
98.25	0.00977161837213827\\
98.26	0.00977342513440691\\
98.27	0.00977522108297272\\
98.28	0.0097770061334732\\
98.29	0.00977878020068684\\
98.3	0.00978054340417745\\
98.31	0.00978229569797715\\
98.32	0.00978403699377235\\
98.33	0.00978576768453989\\
98.34	0.00978749282210007\\
98.35	0.00978921237663149\\
98.36	0.00979092631808516\\
98.37	0.00979263500045184\\
98.38	0.00979433840901014\\
98.39	0.00979603651513999\\
98.4	0.00979772929000989\\
98.41	0.00979941670457537\\
98.42	0.00980109872957738\\
98.43	0.00980277672842595\\
98.44	0.00980445111460809\\
98.45	0.0098061218695262\\
98.46	0.00980778897446873\\
98.47	0.00980945241061039\\
98.48	0.0098111121590124\\
98.49	0.00981276820062277\\
98.5	0.00981442051627664\\
98.51	0.00981606908669666\\
98.52	0.00981771389249343\\
98.53	0.00981935491416593\\
98.54	0.00982099213210204\\
98.55	0.0098226255265788\\
98.56	0.00982425507776307\\
98.57	0.00982588076571214\\
98.58	0.00982750257037453\\
98.59	0.00982912047159071\\
98.6	0.009830734449094\\
98.61	0.0098323444825114\\
98.62	0.00983395055136465\\
98.63	0.00983555263507116\\
98.64	0.00983715071294523\\
98.65	0.0098387447641991\\
98.66	0.00984033476836617\\
98.67	0.00984192070503014\\
98.68	0.00984350255368167\\
98.69	0.00984508029344775\\
98.7	0.00984665390327529\\
98.71	0.00984822336201848\\
98.72	0.0098497886484405\\
98.73	0.00985134974121392\\
98.74	0.00985290661762233\\
98.75	0.00985445924848284\\
98.76	0.00985600760434484\\
98.77	0.00985755165548777\\
98.78	0.00985909137191898\\
98.79	0.0098606267215656\\
98.8	0.00986215767152559\\
98.81	0.00986368418857422\\
98.82	0.0098652062391609\\
98.83	0.00986672378913649\\
98.84	0.00986823680374156\\
98.85	0.00986974524761878\\
98.86	0.00987124908506735\\
98.87	0.00987274828003957\\
98.88	0.00987424279613727\\
98.89	0.00987573259660837\\
98.9	0.00987721764434327\\
98.91	0.00987869790187125\\
98.92	0.00988017333135684\\
98.93	0.00988164389459616\\
98.94	0.00988310955301321\\
98.95	0.00988457026765612\\
98.96	0.00988602599919337\\
98.97	0.00988747670790996\\
98.98	0.00988892235370357\\
98.99	0.00989036289608069\\
99	0.00989179829415259\\
99.01	0.00989322850663147\\
99.02	0.00989465349182635\\
99.03	0.00989607320763906\\
99.04	0.00989748761156016\\
99.05	0.00989889666066474\\
99.06	0.00990030031160832\\
99.07	0.00990169852062259\\
99.08	0.00990309124351133\\
99.09	0.0099044784356463\\
99.1	0.00990586005196288\\
99.11	0.00990723604695566\\
99.12	0.00990860637467526\\
99.13	0.00990997098872422\\
99.14	0.00991132984225253\\
99.15	0.00991268288795305\\
99.16	0.009914030078057\\
99.17	0.00991537136432928\\
99.18	0.00991670669806387\\
99.19	0.00991803603007907\\
99.2	0.00991935931071277\\
99.21	0.00992067648981765\\
99.22	0.0099219875167563\\
99.23	0.00992329234039635\\
99.24	0.00992459090910551\\
99.25	0.00992588317074659\\
99.26	0.00992716907267241\\
99.27	0.00992844856172076\\
99.28	0.0099297215842092\\
99.29	0.0099309880859299\\
99.3	0.00993224801214435\\
99.31	0.0099335013075781\\
99.32	0.00993474791641538\\
99.33	0.00993598778229369\\
99.34	0.00993722084829831\\
99.35	0.00993844705695684\\
99.36	0.00993966635023357\\
99.37	0.00994087866952389\\
99.38	0.00994208395564854\\
99.39	0.00994328214884795\\
99.4	0.00994447318877637\\
99.41	0.00994565701449602\\
99.42	0.00994683356447118\\
99.43	0.00994800277656223\\
99.44	0.00994916458801953\\
99.45	0.00995031893547741\\
99.46	0.00995146575494794\\
99.47	0.00995260498181472\\
99.48	0.00995373655082658\\
99.49	0.00995486039609122\\
99.5	0.0099559764510688\\
99.51	0.00995708464856543\\
99.52	0.00995818492072661\\
99.53	0.00995927719903064\\
99.54	0.00996036141428188\\
99.55	0.00996143749660399\\
99.56	0.00996250537543313\\
99.57	0.00996356497951101\\
99.58	0.00996461623687793\\
99.59	0.00996565907486572\\
99.6	0.00996669342009061\\
99.61	0.00996771918980673\\
99.62	0.00996873629602636\\
99.63	0.00996974464986599\\
99.64	0.00997074416153733\\
99.65	0.00997173474033815\\
99.66	0.00997271629464304\\
99.67	0.00997368873189412\\
99.68	0.00997465195859159\\
99.69	0.00997560588028419\\
99.7	0.00997655040155958\\
99.71	0.00997748542604711\\
99.72	0.00997841085640911\\
99.73	0.00997932659433115\\
99.74	0.00998023254051213\\
99.75	0.00998112859465432\\
99.76	0.0099820146554532\\
99.77	0.00998289062058732\\
99.78	0.00998375638670789\\
99.79	0.00998461184942835\\
99.8	0.00998545690331381\\
99.81	0.00998629144187034\\
99.82	0.00998711535753411\\
99.83	0.00998792854166048\\
99.84	0.00998873088451288\\
99.85	0.00998952227525159\\
99.86	0.00999030260192242\\
99.87	0.00999107175144514\\
99.88	0.00999182960960189\\
99.89	0.00999257606102539\\
99.9	0.00999331098918694\\
99.91	0.00999403427638442\\
99.92	0.00999474580372994\\
99.93	0.0099954454511375\\
99.94	0.00999613309731035\\
99.95	0.0099968086197283\\
99.96	0.00999747189463473\\
99.97	0.00999812279702354\\
99.98	0.00999876120062581\\
99.99	0.00999938697789635\\
100	0.01\\
};
\addlegendentry{$q=2$};

\addplot [color=mycolor1,solid,forget plot]
  table[row sep=crcr]{%
0.01	0\\
0.02	0\\
0.03	0\\
0.04	0\\
0.05	0\\
0.06	0\\
0.07	0\\
0.08	0\\
0.09	0\\
0.1	0\\
0.11	0\\
0.12	0\\
0.13	0\\
0.14	0\\
0.15	0\\
0.16	0\\
0.17	0\\
0.18	0\\
0.19	0\\
0.2	0\\
0.21	0\\
0.22	0\\
0.23	0\\
0.24	0\\
0.25	0\\
0.26	0\\
0.27	0\\
0.28	0\\
0.29	0\\
0.3	0\\
0.31	0\\
0.32	0\\
0.33	0\\
0.34	0\\
0.35	0\\
0.36	0\\
0.37	0\\
0.38	0\\
0.39	0\\
0.4	0\\
0.41	0\\
0.42	0\\
0.43	0\\
0.44	0\\
0.45	0\\
0.46	0\\
0.47	0\\
0.48	0\\
0.49	0\\
0.5	0\\
0.51	0\\
0.52	0\\
0.53	0\\
0.54	0\\
0.55	0\\
0.56	0\\
0.57	0\\
0.58	0\\
0.59	0\\
0.6	0\\
0.61	0\\
0.62	0\\
0.63	0\\
0.64	0\\
0.65	0\\
0.66	0\\
0.67	0\\
0.68	0\\
0.69	0\\
0.7	0\\
0.71	0\\
0.72	0\\
0.73	0\\
0.74	0\\
0.75	0\\
0.76	0\\
0.77	0\\
0.78	0\\
0.79	0\\
0.8	0\\
0.81	0\\
0.82	0\\
0.83	0\\
0.84	0\\
0.85	0\\
0.86	0\\
0.87	0\\
0.88	0\\
0.89	0\\
0.9	0\\
0.91	0\\
0.92	0\\
0.93	0\\
0.94	0\\
0.95	0\\
0.96	0\\
0.97	0\\
0.98	0\\
0.99	0\\
1	0\\
1.01	0\\
1.02	0\\
1.03	0\\
1.04	0\\
1.05	0\\
1.06	0\\
1.07	0\\
1.08	0\\
1.09	0\\
1.1	0\\
1.11	0\\
1.12	0\\
1.13	0\\
1.14	0\\
1.15	0\\
1.16	0\\
1.17	0\\
1.18	0\\
1.19	0\\
1.2	0\\
1.21	0\\
1.22	0\\
1.23	0\\
1.24	0\\
1.25	0\\
1.26	0\\
1.27	0\\
1.28	0\\
1.29	0\\
1.3	0\\
1.31	0\\
1.32	0\\
1.33	0\\
1.34	0\\
1.35	0\\
1.36	0\\
1.37	0\\
1.38	0\\
1.39	0\\
1.4	0\\
1.41	0\\
1.42	0\\
1.43	0\\
1.44	0\\
1.45	0\\
1.46	0\\
1.47	0\\
1.48	0\\
1.49	0\\
1.5	0\\
1.51	0\\
1.52	0\\
1.53	0\\
1.54	0\\
1.55	0\\
1.56	0\\
1.57	0\\
1.58	0\\
1.59	0\\
1.6	0\\
1.61	0\\
1.62	0\\
1.63	0\\
1.64	0\\
1.65	0\\
1.66	0\\
1.67	0\\
1.68	0\\
1.69	0\\
1.7	0\\
1.71	0\\
1.72	0\\
1.73	0\\
1.74	0\\
1.75	0\\
1.76	0\\
1.77	0\\
1.78	0\\
1.79	0\\
1.8	0\\
1.81	0\\
1.82	0\\
1.83	0\\
1.84	0\\
1.85	0\\
1.86	0\\
1.87	0\\
1.88	0\\
1.89	0\\
1.9	0\\
1.91	0\\
1.92	0\\
1.93	0\\
1.94	0\\
1.95	0\\
1.96	0\\
1.97	0\\
1.98	0\\
1.99	0\\
2	0\\
2.01	0\\
2.02	0\\
2.03	0\\
2.04	0\\
2.05	0\\
2.06	0\\
2.07	0\\
2.08	0\\
2.09	0\\
2.1	0\\
2.11	0\\
2.12	0\\
2.13	0\\
2.14	0\\
2.15	0\\
2.16	0\\
2.17	0\\
2.18	0\\
2.19	0\\
2.2	0\\
2.21	0\\
2.22	0\\
2.23	0\\
2.24	0\\
2.25	0\\
2.26	0\\
2.27	0\\
2.28	0\\
2.29	0\\
2.3	0\\
2.31	0\\
2.32	0\\
2.33	0\\
2.34	0\\
2.35	0\\
2.36	0\\
2.37	0\\
2.38	0\\
2.39	0\\
2.4	0\\
2.41	0\\
2.42	0\\
2.43	0\\
2.44	0\\
2.45	0\\
2.46	0\\
2.47	0\\
2.48	0\\
2.49	0\\
2.5	0\\
2.51	0\\
2.52	0\\
2.53	0\\
2.54	0\\
2.55	0\\
2.56	0\\
2.57	0\\
2.58	0\\
2.59	0\\
2.6	0\\
2.61	0\\
2.62	0\\
2.63	0\\
2.64	0\\
2.65	0\\
2.66	0\\
2.67	0\\
2.68	0\\
2.69	0\\
2.7	0\\
2.71	0\\
2.72	0\\
2.73	0\\
2.74	0\\
2.75	0\\
2.76	0\\
2.77	0\\
2.78	0\\
2.79	0\\
2.8	0\\
2.81	0\\
2.82	0\\
2.83	0\\
2.84	0\\
2.85	0\\
2.86	0\\
2.87	0\\
2.88	0\\
2.89	0\\
2.9	0\\
2.91	0\\
2.92	0\\
2.93	0\\
2.94	0\\
2.95	0\\
2.96	0\\
2.97	0\\
2.98	0\\
2.99	0\\
3	0\\
3.01	0\\
3.02	0\\
3.03	0\\
3.04	0\\
3.05	0\\
3.06	0\\
3.07	0\\
3.08	0\\
3.09	0\\
3.1	0\\
3.11	0\\
3.12	0\\
3.13	0\\
3.14	0\\
3.15	0\\
3.16	0\\
3.17	0\\
3.18	0\\
3.19	0\\
3.2	0\\
3.21	0\\
3.22	0\\
3.23	0\\
3.24	0\\
3.25	0\\
3.26	0\\
3.27	0\\
3.28	0\\
3.29	0\\
3.3	0\\
3.31	0\\
3.32	0\\
3.33	0\\
3.34	0\\
3.35	0\\
3.36	0\\
3.37	0\\
3.38	0\\
3.39	0\\
3.4	0\\
3.41	0\\
3.42	0\\
3.43	0\\
3.44	0\\
3.45	0\\
3.46	0\\
3.47	0\\
3.48	0\\
3.49	0\\
3.5	0\\
3.51	0\\
3.52	0\\
3.53	0\\
3.54	0\\
3.55	0\\
3.56	0\\
3.57	0\\
3.58	0\\
3.59	0\\
3.6	0\\
3.61	0\\
3.62	0\\
3.63	0\\
3.64	0\\
3.65	0\\
3.66	0\\
3.67	0\\
3.68	0\\
3.69	0\\
3.7	0\\
3.71	0\\
3.72	0\\
3.73	0\\
3.74	0\\
3.75	0\\
3.76	0\\
3.77	0\\
3.78	0\\
3.79	0\\
3.8	0\\
3.81	0\\
3.82	0\\
3.83	0\\
3.84	0\\
3.85	0\\
3.86	0\\
3.87	0\\
3.88	0\\
3.89	0\\
3.9	0\\
3.91	0\\
3.92	0\\
3.93	0\\
3.94	0\\
3.95	0\\
3.96	0\\
3.97	0\\
3.98	0\\
3.99	0\\
4	0\\
4.01	0\\
4.02	0\\
4.03	0\\
4.04	0\\
4.05	0\\
4.06	0\\
4.07	0\\
4.08	0\\
4.09	0\\
4.1	0\\
4.11	0\\
4.12	0\\
4.13	0\\
4.14	0\\
4.15	0\\
4.16	0\\
4.17	0\\
4.18	0\\
4.19	0\\
4.2	0\\
4.21	0\\
4.22	0\\
4.23	0\\
4.24	0\\
4.25	0\\
4.26	0\\
4.27	0\\
4.28	0\\
4.29	0\\
4.3	0\\
4.31	0\\
4.32	0\\
4.33	0\\
4.34	0\\
4.35	0\\
4.36	0\\
4.37	0\\
4.38	0\\
4.39	0\\
4.4	0\\
4.41	0\\
4.42	0\\
4.43	0\\
4.44	0\\
4.45	0\\
4.46	0\\
4.47	0\\
4.48	0\\
4.49	0\\
4.5	0\\
4.51	0\\
4.52	0\\
4.53	0\\
4.54	0\\
4.55	0\\
4.56	0\\
4.57	0\\
4.58	0\\
4.59	0\\
4.6	0\\
4.61	0\\
4.62	0\\
4.63	0\\
4.64	0\\
4.65	0\\
4.66	0\\
4.67	0\\
4.68	0\\
4.69	0\\
4.7	0\\
4.71	0\\
4.72	0\\
4.73	0\\
4.74	0\\
4.75	0\\
4.76	0\\
4.77	0\\
4.78	0\\
4.79	0\\
4.8	0\\
4.81	0\\
4.82	0\\
4.83	0\\
4.84	0\\
4.85	0\\
4.86	0\\
4.87	0\\
4.88	0\\
4.89	0\\
4.9	0\\
4.91	0\\
4.92	0\\
4.93	0\\
4.94	0\\
4.95	0\\
4.96	0\\
4.97	0\\
4.98	0\\
4.99	0\\
5	0\\
5.01	0\\
5.02	0\\
5.03	0\\
5.04	0\\
5.05	0\\
5.06	0\\
5.07	0\\
5.08	0\\
5.09	0\\
5.1	0\\
5.11	0\\
5.12	0\\
5.13	0\\
5.14	0\\
5.15	0\\
5.16	0\\
5.17	0\\
5.18	0\\
5.19	0\\
5.2	0\\
5.21	0\\
5.22	0\\
5.23	0\\
5.24	0\\
5.25	0\\
5.26	0\\
5.27	0\\
5.28	0\\
5.29	0\\
5.3	0\\
5.31	0\\
5.32	0\\
5.33	0\\
5.34	0\\
5.35	0\\
5.36	0\\
5.37	0\\
5.38	0\\
5.39	0\\
5.4	0\\
5.41	0\\
5.42	0\\
5.43	0\\
5.44	0\\
5.45	0\\
5.46	0\\
5.47	0\\
5.48	0\\
5.49	0\\
5.5	0\\
5.51	0\\
5.52	0\\
5.53	0\\
5.54	0\\
5.55	0\\
5.56	0\\
5.57	0\\
5.58	0\\
5.59	0\\
5.6	0\\
5.61	0\\
5.62	0\\
5.63	0\\
5.64	0\\
5.65	0\\
5.66	0\\
5.67	0\\
5.68	0\\
5.69	0\\
5.7	0\\
5.71	0\\
5.72	0\\
5.73	0\\
5.74	0\\
5.75	0\\
5.76	0\\
5.77	0\\
5.78	0\\
5.79	0\\
5.8	0\\
5.81	0\\
5.82	0\\
5.83	0\\
5.84	0\\
5.85	0\\
5.86	0\\
5.87	0\\
5.88	0\\
5.89	0\\
5.9	0\\
5.91	0\\
5.92	0\\
5.93	0\\
5.94	0\\
5.95	0\\
5.96	0\\
5.97	0\\
5.98	0\\
5.99	0\\
6	0\\
6.01	0\\
6.02	0\\
6.03	0\\
6.04	0\\
6.05	0\\
6.06	0\\
6.07	0\\
6.08	0\\
6.09	0\\
6.1	0\\
6.11	0\\
6.12	0\\
6.13	0\\
6.14	0\\
6.15	0\\
6.16	0\\
6.17	0\\
6.18	0\\
6.19	0\\
6.2	0\\
6.21	0\\
6.22	0\\
6.23	0\\
6.24	0\\
6.25	0\\
6.26	0\\
6.27	0\\
6.28	0\\
6.29	0\\
6.3	0\\
6.31	0\\
6.32	0\\
6.33	0\\
6.34	0\\
6.35	0\\
6.36	0\\
6.37	0\\
6.38	0\\
6.39	0\\
6.4	0\\
6.41	0\\
6.42	0\\
6.43	0\\
6.44	0\\
6.45	0\\
6.46	0\\
6.47	0\\
6.48	0\\
6.49	0\\
6.5	0\\
6.51	0\\
6.52	0\\
6.53	0\\
6.54	0\\
6.55	0\\
6.56	0\\
6.57	0\\
6.58	0\\
6.59	0\\
6.6	0\\
6.61	0\\
6.62	0\\
6.63	0\\
6.64	0\\
6.65	0\\
6.66	0\\
6.67	0\\
6.68	0\\
6.69	0\\
6.7	0\\
6.71	0\\
6.72	0\\
6.73	0\\
6.74	0\\
6.75	0\\
6.76	0\\
6.77	0\\
6.78	0\\
6.79	0\\
6.8	0\\
6.81	0\\
6.82	0\\
6.83	0\\
6.84	0\\
6.85	0\\
6.86	0\\
6.87	0\\
6.88	0\\
6.89	0\\
6.9	0\\
6.91	0\\
6.92	0\\
6.93	0\\
6.94	0\\
6.95	0\\
6.96	0\\
6.97	0\\
6.98	0\\
6.99	0\\
7	0\\
7.01	0\\
7.02	0\\
7.03	0\\
7.04	0\\
7.05	0\\
7.06	0\\
7.07	0\\
7.08	0\\
7.09	0\\
7.1	0\\
7.11	0\\
7.12	0\\
7.13	0\\
7.14	0\\
7.15	0\\
7.16	0\\
7.17	0\\
7.18	0\\
7.19	0\\
7.2	0\\
7.21	0\\
7.22	0\\
7.23	0\\
7.24	0\\
7.25	0\\
7.26	0\\
7.27	0\\
7.28	0\\
7.29	0\\
7.3	0\\
7.31	0\\
7.32	0\\
7.33	0\\
7.34	0\\
7.35	0\\
7.36	0\\
7.37	0\\
7.38	0\\
7.39	0\\
7.4	0\\
7.41	0\\
7.42	0\\
7.43	0\\
7.44	0\\
7.45	0\\
7.46	0\\
7.47	0\\
7.48	0\\
7.49	0\\
7.5	0\\
7.51	0\\
7.52	0\\
7.53	0\\
7.54	0\\
7.55	0\\
7.56	0\\
7.57	0\\
7.58	0\\
7.59	0\\
7.6	0\\
7.61	0\\
7.62	0\\
7.63	0\\
7.64	0\\
7.65	0\\
7.66	0\\
7.67	0\\
7.68	0\\
7.69	0\\
7.7	0\\
7.71	0\\
7.72	0\\
7.73	0\\
7.74	0\\
7.75	0\\
7.76	0\\
7.77	0\\
7.78	0\\
7.79	0\\
7.8	0\\
7.81	0\\
7.82	0\\
7.83	0\\
7.84	0\\
7.85	0\\
7.86	0\\
7.87	0\\
7.88	0\\
7.89	0\\
7.9	0\\
7.91	0\\
7.92	0\\
7.93	0\\
7.94	0\\
7.95	0\\
7.96	0\\
7.97	0\\
7.98	0\\
7.99	0\\
8	0\\
8.01	0\\
8.02	0\\
8.03	0\\
8.04	0\\
8.05	0\\
8.06	0\\
8.07	0\\
8.08	0\\
8.09	0\\
8.1	0\\
8.11	0\\
8.12	0\\
8.13	0\\
8.14	0\\
8.15	0\\
8.16	0\\
8.17	0\\
8.18	0\\
8.19	0\\
8.2	0\\
8.21	0\\
8.22	0\\
8.23	0\\
8.24	0\\
8.25	0\\
8.26	0\\
8.27	0\\
8.28	0\\
8.29	0\\
8.3	0\\
8.31	0\\
8.32	0\\
8.33	0\\
8.34	0\\
8.35	0\\
8.36	0\\
8.37	0\\
8.38	0\\
8.39	0\\
8.4	0\\
8.41	0\\
8.42	0\\
8.43	0\\
8.44	0\\
8.45	0\\
8.46	0\\
8.47	0\\
8.48	0\\
8.49	0\\
8.5	0\\
8.51	0\\
8.52	0\\
8.53	0\\
8.54	0\\
8.55	0\\
8.56	0\\
8.57	0\\
8.58	0\\
8.59	0\\
8.6	0\\
8.61	0\\
8.62	0\\
8.63	0\\
8.64	0\\
8.65	0\\
8.66	0\\
8.67	0\\
8.68	0\\
8.69	0\\
8.7	0\\
8.71	0\\
8.72	0\\
8.73	0\\
8.74	0\\
8.75	0\\
8.76	0\\
8.77	0\\
8.78	0\\
8.79	0\\
8.8	0\\
8.81	0\\
8.82	0\\
8.83	0\\
8.84	0\\
8.85	0\\
8.86	0\\
8.87	0\\
8.88	0\\
8.89	0\\
8.9	0\\
8.91	0\\
8.92	0\\
8.93	0\\
8.94	0\\
8.95	0\\
8.96	0\\
8.97	0\\
8.98	0\\
8.99	0\\
9	0\\
9.01	0\\
9.02	0\\
9.03	0\\
9.04	0\\
9.05	0\\
9.06	0\\
9.07	0\\
9.08	0\\
9.09	0\\
9.1	0\\
9.11	0\\
9.12	0\\
9.13	0\\
9.14	0\\
9.15	0\\
9.16	0\\
9.17	0\\
9.18	0\\
9.19	0\\
9.2	0\\
9.21	0\\
9.22	0\\
9.23	0\\
9.24	0\\
9.25	0\\
9.26	0\\
9.27	0\\
9.28	0\\
9.29	0\\
9.3	0\\
9.31	0\\
9.32	0\\
9.33	0\\
9.34	0\\
9.35	0\\
9.36	0\\
9.37	0\\
9.38	0\\
9.39	0\\
9.4	0\\
9.41	0\\
9.42	0\\
9.43	0\\
9.44	0\\
9.45	0\\
9.46	0\\
9.47	0\\
9.48	0\\
9.49	0\\
9.5	0\\
9.51	0\\
9.52	0\\
9.53	0\\
9.54	0\\
9.55	0\\
9.56	0\\
9.57	0\\
9.58	0\\
9.59	0\\
9.6	0\\
9.61	0\\
9.62	0\\
9.63	0\\
9.64	0\\
9.65	0\\
9.66	0\\
9.67	0\\
9.68	0\\
9.69	0\\
9.7	0\\
9.71	0\\
9.72	0\\
9.73	0\\
9.74	0\\
9.75	0\\
9.76	0\\
9.77	0\\
9.78	0\\
9.79	0\\
9.8	0\\
9.81	0\\
9.82	0\\
9.83	0\\
9.84	0\\
9.85	0\\
9.86	0\\
9.87	0\\
9.88	0\\
9.89	0\\
9.9	0\\
9.91	0\\
9.92	0\\
9.93	0\\
9.94	0\\
9.95	0\\
9.96	0\\
9.97	0\\
9.98	0\\
9.99	0\\
10	0\\
10.01	0\\
10.02	0\\
10.03	0\\
10.04	0\\
10.05	0\\
10.06	0\\
10.07	0\\
10.08	0\\
10.09	0\\
10.1	0\\
10.11	0\\
10.12	0\\
10.13	0\\
10.14	0\\
10.15	0\\
10.16	0\\
10.17	0\\
10.18	0\\
10.19	0\\
10.2	0\\
10.21	0\\
10.22	0\\
10.23	0\\
10.24	0\\
10.25	0\\
10.26	0\\
10.27	0\\
10.28	0\\
10.29	0\\
10.3	0\\
10.31	0\\
10.32	0\\
10.33	0\\
10.34	0\\
10.35	0\\
10.36	0\\
10.37	0\\
10.38	0\\
10.39	0\\
10.4	0\\
10.41	0\\
10.42	0\\
10.43	0\\
10.44	0\\
10.45	0\\
10.46	0\\
10.47	0\\
10.48	0\\
10.49	0\\
10.5	0\\
10.51	0\\
10.52	0\\
10.53	0\\
10.54	0\\
10.55	0\\
10.56	0\\
10.57	0\\
10.58	0\\
10.59	0\\
10.6	0\\
10.61	0\\
10.62	0\\
10.63	0\\
10.64	0\\
10.65	0\\
10.66	0\\
10.67	0\\
10.68	0\\
10.69	0\\
10.7	0\\
10.71	0\\
10.72	0\\
10.73	0\\
10.74	0\\
10.75	0\\
10.76	0\\
10.77	0\\
10.78	0\\
10.79	0\\
10.8	0\\
10.81	0\\
10.82	0\\
10.83	0\\
10.84	0\\
10.85	0\\
10.86	0\\
10.87	0\\
10.88	0\\
10.89	0\\
10.9	0\\
10.91	0\\
10.92	0\\
10.93	0\\
10.94	0\\
10.95	0\\
10.96	0\\
10.97	0\\
10.98	0\\
10.99	0\\
11	0\\
11.01	0\\
11.02	0\\
11.03	0\\
11.04	0\\
11.05	0\\
11.06	0\\
11.07	0\\
11.08	0\\
11.09	0\\
11.1	0\\
11.11	0\\
11.12	0\\
11.13	0\\
11.14	0\\
11.15	0\\
11.16	0\\
11.17	0\\
11.18	0\\
11.19	0\\
11.2	0\\
11.21	0\\
11.22	0\\
11.23	0\\
11.24	0\\
11.25	0\\
11.26	0\\
11.27	0\\
11.28	0\\
11.29	0\\
11.3	0\\
11.31	0\\
11.32	0\\
11.33	0\\
11.34	0\\
11.35	0\\
11.36	0\\
11.37	0\\
11.38	0\\
11.39	0\\
11.4	0\\
11.41	0\\
11.42	0\\
11.43	0\\
11.44	0\\
11.45	0\\
11.46	0\\
11.47	0\\
11.48	0\\
11.49	0\\
11.5	0\\
11.51	0\\
11.52	0\\
11.53	0\\
11.54	0\\
11.55	0\\
11.56	0\\
11.57	0\\
11.58	0\\
11.59	0\\
11.6	0\\
11.61	0\\
11.62	0\\
11.63	0\\
11.64	0\\
11.65	0\\
11.66	0\\
11.67	0\\
11.68	0\\
11.69	0\\
11.7	0\\
11.71	0\\
11.72	0\\
11.73	0\\
11.74	0\\
11.75	0\\
11.76	0\\
11.77	0\\
11.78	0\\
11.79	0\\
11.8	0\\
11.81	0\\
11.82	0\\
11.83	0\\
11.84	0\\
11.85	0\\
11.86	0\\
11.87	0\\
11.88	0\\
11.89	0\\
11.9	0\\
11.91	0\\
11.92	0\\
11.93	0\\
11.94	0\\
11.95	0\\
11.96	0\\
11.97	0\\
11.98	0\\
11.99	0\\
12	0\\
12.01	0\\
12.02	0\\
12.03	0\\
12.04	0\\
12.05	0\\
12.06	0\\
12.07	0\\
12.08	0\\
12.09	0\\
12.1	0\\
12.11	0\\
12.12	0\\
12.13	0\\
12.14	0\\
12.15	0\\
12.16	0\\
12.17	0\\
12.18	0\\
12.19	0\\
12.2	0\\
12.21	0\\
12.22	0\\
12.23	0\\
12.24	0\\
12.25	0\\
12.26	0\\
12.27	0\\
12.28	0\\
12.29	0\\
12.3	0\\
12.31	0\\
12.32	0\\
12.33	0\\
12.34	0\\
12.35	0\\
12.36	0\\
12.37	0\\
12.38	0\\
12.39	0\\
12.4	0\\
12.41	0\\
12.42	0\\
12.43	0\\
12.44	0\\
12.45	0\\
12.46	0\\
12.47	0\\
12.48	0\\
12.49	0\\
12.5	0\\
12.51	0\\
12.52	0\\
12.53	0\\
12.54	0\\
12.55	0\\
12.56	0\\
12.57	0\\
12.58	0\\
12.59	0\\
12.6	0\\
12.61	0\\
12.62	0\\
12.63	0\\
12.64	0\\
12.65	0\\
12.66	0\\
12.67	0\\
12.68	0\\
12.69	0\\
12.7	0\\
12.71	0\\
12.72	0\\
12.73	0\\
12.74	0\\
12.75	0\\
12.76	0\\
12.77	0\\
12.78	0\\
12.79	0\\
12.8	0\\
12.81	0\\
12.82	0\\
12.83	0\\
12.84	0\\
12.85	0\\
12.86	0\\
12.87	0\\
12.88	0\\
12.89	0\\
12.9	0\\
12.91	0\\
12.92	0\\
12.93	0\\
12.94	0\\
12.95	0\\
12.96	0\\
12.97	0\\
12.98	0\\
12.99	0\\
13	0\\
13.01	0\\
13.02	0\\
13.03	0\\
13.04	0\\
13.05	0\\
13.06	0\\
13.07	0\\
13.08	0\\
13.09	0\\
13.1	0\\
13.11	0\\
13.12	0\\
13.13	0\\
13.14	0\\
13.15	0\\
13.16	0\\
13.17	0\\
13.18	0\\
13.19	0\\
13.2	0\\
13.21	0\\
13.22	0\\
13.23	0\\
13.24	0\\
13.25	0\\
13.26	0\\
13.27	0\\
13.28	0\\
13.29	0\\
13.3	0\\
13.31	0\\
13.32	0\\
13.33	0\\
13.34	0\\
13.35	0\\
13.36	0\\
13.37	0\\
13.38	0\\
13.39	0\\
13.4	0\\
13.41	0\\
13.42	0\\
13.43	0\\
13.44	0\\
13.45	0\\
13.46	0\\
13.47	0\\
13.48	0\\
13.49	0\\
13.5	0\\
13.51	0\\
13.52	0\\
13.53	0\\
13.54	0\\
13.55	0\\
13.56	0\\
13.57	0\\
13.58	0\\
13.59	0\\
13.6	0\\
13.61	0\\
13.62	0\\
13.63	0\\
13.64	0\\
13.65	0\\
13.66	0\\
13.67	0\\
13.68	0\\
13.69	0\\
13.7	0\\
13.71	0\\
13.72	0\\
13.73	0\\
13.74	0\\
13.75	0\\
13.76	0\\
13.77	0\\
13.78	0\\
13.79	0\\
13.8	0\\
13.81	0\\
13.82	0\\
13.83	0\\
13.84	0\\
13.85	0\\
13.86	0\\
13.87	0\\
13.88	0\\
13.89	0\\
13.9	0\\
13.91	0\\
13.92	0\\
13.93	0\\
13.94	0\\
13.95	0\\
13.96	0\\
13.97	0\\
13.98	0\\
13.99	0\\
14	0\\
14.01	0\\
14.02	0\\
14.03	0\\
14.04	0\\
14.05	0\\
14.06	0\\
14.07	0\\
14.08	0\\
14.09	0\\
14.1	0\\
14.11	0\\
14.12	0\\
14.13	0\\
14.14	0\\
14.15	0\\
14.16	0\\
14.17	0\\
14.18	0\\
14.19	0\\
14.2	0\\
14.21	0\\
14.22	0\\
14.23	0\\
14.24	0\\
14.25	0\\
14.26	0\\
14.27	0\\
14.28	0\\
14.29	0\\
14.3	0\\
14.31	0\\
14.32	0\\
14.33	0\\
14.34	0\\
14.35	0\\
14.36	0\\
14.37	0\\
14.38	0\\
14.39	0\\
14.4	0\\
14.41	0\\
14.42	0\\
14.43	0\\
14.44	0\\
14.45	0\\
14.46	0\\
14.47	0\\
14.48	0\\
14.49	0\\
14.5	0\\
14.51	0\\
14.52	0\\
14.53	0\\
14.54	0\\
14.55	0\\
14.56	0\\
14.57	0\\
14.58	0\\
14.59	0\\
14.6	0\\
14.61	0\\
14.62	0\\
14.63	0\\
14.64	0\\
14.65	0\\
14.66	0\\
14.67	0\\
14.68	0\\
14.69	0\\
14.7	0\\
14.71	0\\
14.72	0\\
14.73	0\\
14.74	0\\
14.75	0\\
14.76	0\\
14.77	0\\
14.78	0\\
14.79	0\\
14.8	0\\
14.81	0\\
14.82	0\\
14.83	0\\
14.84	0\\
14.85	0\\
14.86	0\\
14.87	0\\
14.88	0\\
14.89	0\\
14.9	0\\
14.91	0\\
14.92	0\\
14.93	0\\
14.94	0\\
14.95	0\\
14.96	0\\
14.97	0\\
14.98	0\\
14.99	0\\
15	0\\
15.01	0\\
15.02	0\\
15.03	0\\
15.04	0\\
15.05	0\\
15.06	0\\
15.07	0\\
15.08	0\\
15.09	0\\
15.1	0\\
15.11	0\\
15.12	0\\
15.13	0\\
15.14	0\\
15.15	0\\
15.16	0\\
15.17	0\\
15.18	0\\
15.19	0\\
15.2	0\\
15.21	0\\
15.22	0\\
15.23	0\\
15.24	0\\
15.25	0\\
15.26	0\\
15.27	0\\
15.28	0\\
15.29	0\\
15.3	0\\
15.31	0\\
15.32	0\\
15.33	0\\
15.34	0\\
15.35	0\\
15.36	0\\
15.37	0\\
15.38	0\\
15.39	0\\
15.4	0\\
15.41	0\\
15.42	0\\
15.43	0\\
15.44	0\\
15.45	0\\
15.46	0\\
15.47	0\\
15.48	0\\
15.49	0\\
15.5	0\\
15.51	0\\
15.52	0\\
15.53	0\\
15.54	0\\
15.55	0\\
15.56	0\\
15.57	0\\
15.58	0\\
15.59	0\\
15.6	0\\
15.61	0\\
15.62	0\\
15.63	0\\
15.64	0\\
15.65	0\\
15.66	0\\
15.67	0\\
15.68	0\\
15.69	0\\
15.7	0\\
15.71	0\\
15.72	0\\
15.73	0\\
15.74	0\\
15.75	0\\
15.76	0\\
15.77	0\\
15.78	0\\
15.79	0\\
15.8	0\\
15.81	0\\
15.82	0\\
15.83	0\\
15.84	0\\
15.85	0\\
15.86	0\\
15.87	0\\
15.88	0\\
15.89	0\\
15.9	0\\
15.91	0\\
15.92	0\\
15.93	0\\
15.94	0\\
15.95	0\\
15.96	0\\
15.97	0\\
15.98	0\\
15.99	0\\
16	0\\
16.01	0\\
16.02	0\\
16.03	0\\
16.04	0\\
16.05	0\\
16.06	0\\
16.07	0\\
16.08	0\\
16.09	0\\
16.1	0\\
16.11	0\\
16.12	0\\
16.13	0\\
16.14	0\\
16.15	0\\
16.16	0\\
16.17	0\\
16.18	0\\
16.19	0\\
16.2	0\\
16.21	0\\
16.22	0\\
16.23	0\\
16.24	0\\
16.25	0\\
16.26	0\\
16.27	0\\
16.28	0\\
16.29	0\\
16.3	0\\
16.31	0\\
16.32	0\\
16.33	0\\
16.34	0\\
16.35	0\\
16.36	0\\
16.37	0\\
16.38	0\\
16.39	0\\
16.4	0\\
16.41	0\\
16.42	0\\
16.43	0\\
16.44	0\\
16.45	0\\
16.46	0\\
16.47	0\\
16.48	0\\
16.49	0\\
16.5	0\\
16.51	0\\
16.52	0\\
16.53	0\\
16.54	0\\
16.55	0\\
16.56	0\\
16.57	0\\
16.58	0\\
16.59	0\\
16.6	0\\
16.61	0\\
16.62	0\\
16.63	0\\
16.64	0\\
16.65	0\\
16.66	0\\
16.67	0\\
16.68	0\\
16.69	0\\
16.7	0\\
16.71	0\\
16.72	0\\
16.73	0\\
16.74	0\\
16.75	0\\
16.76	0\\
16.77	0\\
16.78	0\\
16.79	0\\
16.8	0\\
16.81	0\\
16.82	0\\
16.83	0\\
16.84	0\\
16.85	0\\
16.86	0\\
16.87	0\\
16.88	0\\
16.89	0\\
16.9	0\\
16.91	0\\
16.92	0\\
16.93	0\\
16.94	0\\
16.95	0\\
16.96	0\\
16.97	0\\
16.98	0\\
16.99	0\\
17	0\\
17.01	0\\
17.02	0\\
17.03	0\\
17.04	0\\
17.05	0\\
17.06	0\\
17.07	0\\
17.08	0\\
17.09	0\\
17.1	0\\
17.11	0\\
17.12	0\\
17.13	0\\
17.14	0\\
17.15	0\\
17.16	0\\
17.17	0\\
17.18	0\\
17.19	0\\
17.2	0\\
17.21	0\\
17.22	0\\
17.23	0\\
17.24	0\\
17.25	0\\
17.26	0\\
17.27	0\\
17.28	0\\
17.29	0\\
17.3	0\\
17.31	0\\
17.32	0\\
17.33	0\\
17.34	0\\
17.35	0\\
17.36	0\\
17.37	0\\
17.38	0\\
17.39	0\\
17.4	0\\
17.41	0\\
17.42	0\\
17.43	0\\
17.44	0\\
17.45	0\\
17.46	0\\
17.47	0\\
17.48	0\\
17.49	0\\
17.5	0\\
17.51	0\\
17.52	0\\
17.53	0\\
17.54	0\\
17.55	0\\
17.56	0\\
17.57	0\\
17.58	0\\
17.59	0\\
17.6	0\\
17.61	0\\
17.62	0\\
17.63	0\\
17.64	0\\
17.65	0\\
17.66	0\\
17.67	0\\
17.68	0\\
17.69	0\\
17.7	0\\
17.71	0\\
17.72	0\\
17.73	0\\
17.74	0\\
17.75	0\\
17.76	0\\
17.77	0\\
17.78	0\\
17.79	0\\
17.8	0\\
17.81	0\\
17.82	0\\
17.83	0\\
17.84	0\\
17.85	0\\
17.86	0\\
17.87	0\\
17.88	0\\
17.89	0\\
17.9	0\\
17.91	0\\
17.92	0\\
17.93	0\\
17.94	0\\
17.95	0\\
17.96	0\\
17.97	0\\
17.98	0\\
17.99	0\\
18	0\\
18.01	0\\
18.02	0\\
18.03	0\\
18.04	0\\
18.05	0\\
18.06	0\\
18.07	0\\
18.08	0\\
18.09	0\\
18.1	0\\
18.11	0\\
18.12	0\\
18.13	0\\
18.14	0\\
18.15	0\\
18.16	0\\
18.17	0\\
18.18	0\\
18.19	0\\
18.2	0\\
18.21	0\\
18.22	0\\
18.23	0\\
18.24	0\\
18.25	0\\
18.26	0\\
18.27	0\\
18.28	0\\
18.29	0\\
18.3	0\\
18.31	0\\
18.32	0\\
18.33	0\\
18.34	0\\
18.35	0\\
18.36	0\\
18.37	0\\
18.38	0\\
18.39	0\\
18.4	0\\
18.41	0\\
18.42	0\\
18.43	0\\
18.44	0\\
18.45	0\\
18.46	0\\
18.47	0\\
18.48	0\\
18.49	0\\
18.5	0\\
18.51	0\\
18.52	0\\
18.53	0\\
18.54	0\\
18.55	0\\
18.56	0\\
18.57	0\\
18.58	0\\
18.59	0\\
18.6	0\\
18.61	0\\
18.62	0\\
18.63	0\\
18.64	0\\
18.65	0\\
18.66	0\\
18.67	0\\
18.68	0\\
18.69	0\\
18.7	0\\
18.71	0\\
18.72	0\\
18.73	0\\
18.74	0\\
18.75	0\\
18.76	0\\
18.77	0\\
18.78	0\\
18.79	0\\
18.8	0\\
18.81	0\\
18.82	0\\
18.83	0\\
18.84	0\\
18.85	0\\
18.86	0\\
18.87	0\\
18.88	0\\
18.89	0\\
18.9	0\\
18.91	0\\
18.92	0\\
18.93	0\\
18.94	0\\
18.95	0\\
18.96	0\\
18.97	0\\
18.98	0\\
18.99	0\\
19	0\\
19.01	0\\
19.02	0\\
19.03	0\\
19.04	0\\
19.05	0\\
19.06	0\\
19.07	0\\
19.08	0\\
19.09	0\\
19.1	0\\
19.11	0\\
19.12	0\\
19.13	0\\
19.14	0\\
19.15	0\\
19.16	0\\
19.17	0\\
19.18	0\\
19.19	0\\
19.2	0\\
19.21	0\\
19.22	0\\
19.23	0\\
19.24	0\\
19.25	0\\
19.26	0\\
19.27	0\\
19.28	0\\
19.29	0\\
19.3	0\\
19.31	0\\
19.32	0\\
19.33	0\\
19.34	0\\
19.35	0\\
19.36	0\\
19.37	0\\
19.38	0\\
19.39	0\\
19.4	0\\
19.41	0\\
19.42	0\\
19.43	0\\
19.44	0\\
19.45	0\\
19.46	0\\
19.47	0\\
19.48	0\\
19.49	0\\
19.5	0\\
19.51	0\\
19.52	0\\
19.53	0\\
19.54	0\\
19.55	0\\
19.56	0\\
19.57	0\\
19.58	0\\
19.59	0\\
19.6	0\\
19.61	0\\
19.62	0\\
19.63	0\\
19.64	0\\
19.65	0\\
19.66	0\\
19.67	0\\
19.68	0\\
19.69	0\\
19.7	0\\
19.71	0\\
19.72	0\\
19.73	0\\
19.74	0\\
19.75	0\\
19.76	0\\
19.77	0\\
19.78	0\\
19.79	0\\
19.8	0\\
19.81	0\\
19.82	0\\
19.83	0\\
19.84	0\\
19.85	0\\
19.86	0\\
19.87	0\\
19.88	0\\
19.89	0\\
19.9	0\\
19.91	0\\
19.92	0\\
19.93	0\\
19.94	0\\
19.95	0\\
19.96	0\\
19.97	0\\
19.98	0\\
19.99	0\\
20	0\\
20.01	0\\
20.02	0\\
20.03	0\\
20.04	0\\
20.05	0\\
20.06	0\\
20.07	0\\
20.08	0\\
20.09	0\\
20.1	0\\
20.11	0\\
20.12	0\\
20.13	0\\
20.14	0\\
20.15	0\\
20.16	0\\
20.17	0\\
20.18	0\\
20.19	0\\
20.2	0\\
20.21	0\\
20.22	0\\
20.23	0\\
20.24	0\\
20.25	0\\
20.26	0\\
20.27	0\\
20.28	0\\
20.29	0\\
20.3	0\\
20.31	0\\
20.32	0\\
20.33	0\\
20.34	0\\
20.35	0\\
20.36	0\\
20.37	0\\
20.38	0\\
20.39	0\\
20.4	0\\
20.41	0\\
20.42	0\\
20.43	0\\
20.44	0\\
20.45	0\\
20.46	0\\
20.47	0\\
20.48	0\\
20.49	0\\
20.5	0\\
20.51	0\\
20.52	0\\
20.53	0\\
20.54	0\\
20.55	0\\
20.56	0\\
20.57	0\\
20.58	0\\
20.59	0\\
20.6	0\\
20.61	0\\
20.62	0\\
20.63	0\\
20.64	0\\
20.65	0\\
20.66	0\\
20.67	0\\
20.68	0\\
20.69	0\\
20.7	0\\
20.71	0\\
20.72	0\\
20.73	0\\
20.74	0\\
20.75	0\\
20.76	0\\
20.77	0\\
20.78	0\\
20.79	0\\
20.8	0\\
20.81	0\\
20.82	0\\
20.83	0\\
20.84	0\\
20.85	0\\
20.86	0\\
20.87	0\\
20.88	0\\
20.89	0\\
20.9	0\\
20.91	0\\
20.92	0\\
20.93	0\\
20.94	0\\
20.95	0\\
20.96	0\\
20.97	0\\
20.98	0\\
20.99	0\\
21	0\\
21.01	0\\
21.02	0\\
21.03	0\\
21.04	0\\
21.05	0\\
21.06	0\\
21.07	0\\
21.08	0\\
21.09	0\\
21.1	0\\
21.11	0\\
21.12	0\\
21.13	0\\
21.14	0\\
21.15	0\\
21.16	0\\
21.17	0\\
21.18	0\\
21.19	0\\
21.2	0\\
21.21	0\\
21.22	0\\
21.23	0\\
21.24	0\\
21.25	0\\
21.26	0\\
21.27	0\\
21.28	0\\
21.29	0\\
21.3	0\\
21.31	0\\
21.32	0\\
21.33	0\\
21.34	0\\
21.35	0\\
21.36	0\\
21.37	0\\
21.38	0\\
21.39	0\\
21.4	0\\
21.41	0\\
21.42	0\\
21.43	0\\
21.44	0\\
21.45	0\\
21.46	0\\
21.47	0\\
21.48	0\\
21.49	0\\
21.5	0\\
21.51	0\\
21.52	0\\
21.53	0\\
21.54	0\\
21.55	0\\
21.56	0\\
21.57	0\\
21.58	0\\
21.59	0\\
21.6	0\\
21.61	0\\
21.62	0\\
21.63	0\\
21.64	0\\
21.65	0\\
21.66	0\\
21.67	0\\
21.68	0\\
21.69	0\\
21.7	0\\
21.71	0\\
21.72	0\\
21.73	0\\
21.74	0\\
21.75	0\\
21.76	0\\
21.77	0\\
21.78	0\\
21.79	0\\
21.8	0\\
21.81	0\\
21.82	0\\
21.83	0\\
21.84	0\\
21.85	0\\
21.86	0\\
21.87	0\\
21.88	0\\
21.89	0\\
21.9	0\\
21.91	0\\
21.92	0\\
21.93	0\\
21.94	0\\
21.95	0\\
21.96	0\\
21.97	0\\
21.98	0\\
21.99	0\\
22	0\\
22.01	0\\
22.02	0\\
22.03	0\\
22.04	0\\
22.05	0\\
22.06	0\\
22.07	0\\
22.08	0\\
22.09	0\\
22.1	0\\
22.11	0\\
22.12	0\\
22.13	0\\
22.14	0\\
22.15	0\\
22.16	0\\
22.17	0\\
22.18	0\\
22.19	0\\
22.2	0\\
22.21	0\\
22.22	0\\
22.23	0\\
22.24	0\\
22.25	0\\
22.26	0\\
22.27	0\\
22.28	0\\
22.29	0\\
22.3	0\\
22.31	0\\
22.32	0\\
22.33	0\\
22.34	0\\
22.35	0\\
22.36	0\\
22.37	0\\
22.38	0\\
22.39	0\\
22.4	0\\
22.41	0\\
22.42	0\\
22.43	0\\
22.44	0\\
22.45	0\\
22.46	0\\
22.47	0\\
22.48	0\\
22.49	0\\
22.5	0\\
22.51	0\\
22.52	0\\
22.53	0\\
22.54	0\\
22.55	0\\
22.56	0\\
22.57	0\\
22.58	0\\
22.59	0\\
22.6	0\\
22.61	0\\
22.62	0\\
22.63	0\\
22.64	0\\
22.65	0\\
22.66	0\\
22.67	0\\
22.68	0\\
22.69	0\\
22.7	0\\
22.71	0\\
22.72	0\\
22.73	0\\
22.74	0\\
22.75	0\\
22.76	0\\
22.77	0\\
22.78	0\\
22.79	0\\
22.8	0\\
22.81	0\\
22.82	0\\
22.83	0\\
22.84	0\\
22.85	0\\
22.86	0\\
22.87	0\\
22.88	0\\
22.89	0\\
22.9	0\\
22.91	0\\
22.92	0\\
22.93	0\\
22.94	0\\
22.95	0\\
22.96	0\\
22.97	0\\
22.98	0\\
22.99	0\\
23	0\\
23.01	0\\
23.02	0\\
23.03	0\\
23.04	0\\
23.05	0\\
23.06	0\\
23.07	0\\
23.08	0\\
23.09	0\\
23.1	0\\
23.11	0\\
23.12	0\\
23.13	0\\
23.14	0\\
23.15	0\\
23.16	0\\
23.17	0\\
23.18	0\\
23.19	0\\
23.2	0\\
23.21	0\\
23.22	0\\
23.23	0\\
23.24	0\\
23.25	0\\
23.26	0\\
23.27	0\\
23.28	0\\
23.29	0\\
23.3	0\\
23.31	0\\
23.32	0\\
23.33	0\\
23.34	0\\
23.35	0\\
23.36	0\\
23.37	0\\
23.38	0\\
23.39	0\\
23.4	0\\
23.41	0\\
23.42	0\\
23.43	0\\
23.44	0\\
23.45	0\\
23.46	0\\
23.47	0\\
23.48	0\\
23.49	0\\
23.5	0\\
23.51	0\\
23.52	0\\
23.53	0\\
23.54	0\\
23.55	0\\
23.56	0\\
23.57	0\\
23.58	0\\
23.59	0\\
23.6	0\\
23.61	0\\
23.62	0\\
23.63	0\\
23.64	0\\
23.65	0\\
23.66	0\\
23.67	0\\
23.68	0\\
23.69	0\\
23.7	0\\
23.71	0\\
23.72	0\\
23.73	0\\
23.74	0\\
23.75	0\\
23.76	0\\
23.77	0\\
23.78	0\\
23.79	0\\
23.8	0\\
23.81	0\\
23.82	0\\
23.83	0\\
23.84	0\\
23.85	0\\
23.86	0\\
23.87	0\\
23.88	0\\
23.89	0\\
23.9	0\\
23.91	0\\
23.92	0\\
23.93	0\\
23.94	0\\
23.95	0\\
23.96	0\\
23.97	0\\
23.98	0\\
23.99	0\\
24	0\\
24.01	0\\
24.02	0\\
24.03	0\\
24.04	0\\
24.05	0\\
24.06	0\\
24.07	0\\
24.08	0\\
24.09	0\\
24.1	0\\
24.11	0\\
24.12	0\\
24.13	0\\
24.14	0\\
24.15	0\\
24.16	0\\
24.17	0\\
24.18	0\\
24.19	0\\
24.2	0\\
24.21	0\\
24.22	0\\
24.23	0\\
24.24	0\\
24.25	0\\
24.26	0\\
24.27	0\\
24.28	0\\
24.29	0\\
24.3	0\\
24.31	0\\
24.32	0\\
24.33	0\\
24.34	0\\
24.35	0\\
24.36	0\\
24.37	0\\
24.38	0\\
24.39	0\\
24.4	0\\
24.41	0\\
24.42	0\\
24.43	0\\
24.44	0\\
24.45	0\\
24.46	0\\
24.47	0\\
24.48	0\\
24.49	0\\
24.5	0\\
24.51	0\\
24.52	0\\
24.53	0\\
24.54	0\\
24.55	0\\
24.56	0\\
24.57	0\\
24.58	0\\
24.59	0\\
24.6	0\\
24.61	0\\
24.62	0\\
24.63	0\\
24.64	0\\
24.65	0\\
24.66	0\\
24.67	0\\
24.68	0\\
24.69	0\\
24.7	0\\
24.71	0\\
24.72	0\\
24.73	0\\
24.74	0\\
24.75	0\\
24.76	0\\
24.77	0\\
24.78	0\\
24.79	0\\
24.8	0\\
24.81	0\\
24.82	0\\
24.83	0\\
24.84	0\\
24.85	0\\
24.86	0\\
24.87	0\\
24.88	0\\
24.89	0\\
24.9	0\\
24.91	0\\
24.92	0\\
24.93	0\\
24.94	0\\
24.95	0\\
24.96	0\\
24.97	0\\
24.98	0\\
24.99	0\\
25	0\\
25.01	0\\
25.02	0\\
25.03	0\\
25.04	0\\
25.05	0\\
25.06	0\\
25.07	0\\
25.08	0\\
25.09	0\\
25.1	0\\
25.11	0\\
25.12	0\\
25.13	0\\
25.14	0\\
25.15	0\\
25.16	0\\
25.17	0\\
25.18	0\\
25.19	0\\
25.2	0\\
25.21	0\\
25.22	0\\
25.23	0\\
25.24	0\\
25.25	0\\
25.26	0\\
25.27	0\\
25.28	0\\
25.29	0\\
25.3	0\\
25.31	0\\
25.32	0\\
25.33	0\\
25.34	0\\
25.35	0\\
25.36	0\\
25.37	0\\
25.38	0\\
25.39	0\\
25.4	0\\
25.41	0\\
25.42	0\\
25.43	0\\
25.44	0\\
25.45	0\\
25.46	0\\
25.47	0\\
25.48	0\\
25.49	0\\
25.5	0\\
25.51	0\\
25.52	0\\
25.53	0\\
25.54	0\\
25.55	0\\
25.56	0\\
25.57	0\\
25.58	0\\
25.59	0\\
25.6	0\\
25.61	0\\
25.62	0\\
25.63	0\\
25.64	0\\
25.65	0\\
25.66	0\\
25.67	0\\
25.68	0\\
25.69	0\\
25.7	0\\
25.71	0\\
25.72	0\\
25.73	0\\
25.74	0\\
25.75	0\\
25.76	0\\
25.77	0\\
25.78	0\\
25.79	0\\
25.8	0\\
25.81	0\\
25.82	0\\
25.83	0\\
25.84	0\\
25.85	0\\
25.86	0\\
25.87	0\\
25.88	0\\
25.89	0\\
25.9	0\\
25.91	0\\
25.92	0\\
25.93	0\\
25.94	0\\
25.95	0\\
25.96	0\\
25.97	0\\
25.98	0\\
25.99	0\\
26	0\\
26.01	0\\
26.02	0\\
26.03	0\\
26.04	0\\
26.05	0\\
26.06	0\\
26.07	0\\
26.08	0\\
26.09	0\\
26.1	0\\
26.11	0\\
26.12	0\\
26.13	0\\
26.14	0\\
26.15	0\\
26.16	0\\
26.17	0\\
26.18	0\\
26.19	0\\
26.2	0\\
26.21	0\\
26.22	0\\
26.23	0\\
26.24	0\\
26.25	0\\
26.26	0\\
26.27	0\\
26.28	0\\
26.29	0\\
26.3	0\\
26.31	0\\
26.32	0\\
26.33	0\\
26.34	0\\
26.35	0\\
26.36	0\\
26.37	0\\
26.38	0\\
26.39	0\\
26.4	0\\
26.41	0\\
26.42	0\\
26.43	0\\
26.44	0\\
26.45	0\\
26.46	0\\
26.47	0\\
26.48	0\\
26.49	0\\
26.5	0\\
26.51	0\\
26.52	0\\
26.53	0\\
26.54	0\\
26.55	0\\
26.56	0\\
26.57	0\\
26.58	0\\
26.59	0\\
26.6	0\\
26.61	0\\
26.62	0\\
26.63	0\\
26.64	0\\
26.65	0\\
26.66	0\\
26.67	0\\
26.68	0\\
26.69	0\\
26.7	0\\
26.71	0\\
26.72	0\\
26.73	0\\
26.74	0\\
26.75	0\\
26.76	0\\
26.77	0\\
26.78	0\\
26.79	0\\
26.8	0\\
26.81	0\\
26.82	0\\
26.83	0\\
26.84	0\\
26.85	0\\
26.86	0\\
26.87	0\\
26.88	0\\
26.89	0\\
26.9	0\\
26.91	0\\
26.92	0\\
26.93	0\\
26.94	0\\
26.95	0\\
26.96	0\\
26.97	0\\
26.98	0\\
26.99	0\\
27	0\\
27.01	0\\
27.02	0\\
27.03	0\\
27.04	0\\
27.05	0\\
27.06	0\\
27.07	0\\
27.08	0\\
27.09	0\\
27.1	0\\
27.11	0\\
27.12	0\\
27.13	0\\
27.14	0\\
27.15	0\\
27.16	0\\
27.17	0\\
27.18	0\\
27.19	0\\
27.2	0\\
27.21	0\\
27.22	0\\
27.23	0\\
27.24	0\\
27.25	0\\
27.26	0\\
27.27	0\\
27.28	0\\
27.29	0\\
27.3	0\\
27.31	0\\
27.32	0\\
27.33	0\\
27.34	0\\
27.35	0\\
27.36	0\\
27.37	0\\
27.38	0\\
27.39	0\\
27.4	0\\
27.41	0\\
27.42	0\\
27.43	0\\
27.44	0\\
27.45	0\\
27.46	0\\
27.47	0\\
27.48	0\\
27.49	0\\
27.5	0\\
27.51	0\\
27.52	0\\
27.53	0\\
27.54	0\\
27.55	0\\
27.56	0\\
27.57	0\\
27.58	0\\
27.59	0\\
27.6	0\\
27.61	0\\
27.62	0\\
27.63	0\\
27.64	0\\
27.65	0\\
27.66	0\\
27.67	0\\
27.68	0\\
27.69	0\\
27.7	0\\
27.71	0\\
27.72	0\\
27.73	0\\
27.74	0\\
27.75	0\\
27.76	0\\
27.77	0\\
27.78	0\\
27.79	0\\
27.8	0\\
27.81	0\\
27.82	0\\
27.83	0\\
27.84	0\\
27.85	0\\
27.86	0\\
27.87	0\\
27.88	0\\
27.89	0\\
27.9	0\\
27.91	0\\
27.92	0\\
27.93	0\\
27.94	0\\
27.95	0\\
27.96	0\\
27.97	0\\
27.98	0\\
27.99	0\\
28	0\\
28.01	0\\
28.02	0\\
28.03	0\\
28.04	0\\
28.05	0\\
28.06	0\\
28.07	0\\
28.08	0\\
28.09	0\\
28.1	0\\
28.11	0\\
28.12	0\\
28.13	0\\
28.14	0\\
28.15	0\\
28.16	0\\
28.17	0\\
28.18	0\\
28.19	0\\
28.2	0\\
28.21	0\\
28.22	0\\
28.23	0\\
28.24	0\\
28.25	0\\
28.26	0\\
28.27	0\\
28.28	0\\
28.29	0\\
28.3	0\\
28.31	0\\
28.32	0\\
28.33	0\\
28.34	0\\
28.35	0\\
28.36	0\\
28.37	0\\
28.38	0\\
28.39	0\\
28.4	0\\
28.41	0\\
28.42	0\\
28.43	0\\
28.44	0\\
28.45	0\\
28.46	0\\
28.47	0\\
28.48	0\\
28.49	0\\
28.5	0\\
28.51	0\\
28.52	0\\
28.53	0\\
28.54	0\\
28.55	0\\
28.56	0\\
28.57	0\\
28.58	0\\
28.59	0\\
28.6	0\\
28.61	0\\
28.62	0\\
28.63	0\\
28.64	0\\
28.65	0\\
28.66	0\\
28.67	0\\
28.68	0\\
28.69	0\\
28.7	0\\
28.71	0\\
28.72	0\\
28.73	0\\
28.74	0\\
28.75	0\\
28.76	0\\
28.77	0\\
28.78	0\\
28.79	0\\
28.8	0\\
28.81	0\\
28.82	0\\
28.83	0\\
28.84	0\\
28.85	0\\
28.86	0\\
28.87	0\\
28.88	0\\
28.89	0\\
28.9	0\\
28.91	0\\
28.92	0\\
28.93	0\\
28.94	0\\
28.95	0\\
28.96	0\\
28.97	0\\
28.98	0\\
28.99	0\\
29	0\\
29.01	0\\
29.02	0\\
29.03	0\\
29.04	0\\
29.05	0\\
29.06	0\\
29.07	0\\
29.08	0\\
29.09	0\\
29.1	0\\
29.11	0\\
29.12	0\\
29.13	0\\
29.14	0\\
29.15	0\\
29.16	0\\
29.17	0\\
29.18	0\\
29.19	0\\
29.2	0\\
29.21	0\\
29.22	0\\
29.23	0\\
29.24	0\\
29.25	0\\
29.26	0\\
29.27	0\\
29.28	0\\
29.29	0\\
29.3	0\\
29.31	0\\
29.32	0\\
29.33	0\\
29.34	0\\
29.35	0\\
29.36	0\\
29.37	0\\
29.38	0\\
29.39	0\\
29.4	0\\
29.41	0\\
29.42	0\\
29.43	0\\
29.44	0\\
29.45	0\\
29.46	0\\
29.47	0\\
29.48	0\\
29.49	0\\
29.5	0\\
29.51	0\\
29.52	0\\
29.53	0\\
29.54	0\\
29.55	0\\
29.56	0\\
29.57	0\\
29.58	0\\
29.59	0\\
29.6	0\\
29.61	0\\
29.62	0\\
29.63	0\\
29.64	0\\
29.65	0\\
29.66	0\\
29.67	0\\
29.68	0\\
29.69	0\\
29.7	0\\
29.71	0\\
29.72	0\\
29.73	0\\
29.74	0\\
29.75	0\\
29.76	0\\
29.77	0\\
29.78	0\\
29.79	0\\
29.8	0\\
29.81	0\\
29.82	0\\
29.83	0\\
29.84	0\\
29.85	0\\
29.86	0\\
29.87	0\\
29.88	0\\
29.89	0\\
29.9	0\\
29.91	0\\
29.92	0\\
29.93	0\\
29.94	0\\
29.95	0\\
29.96	0\\
29.97	0\\
29.98	0\\
29.99	0\\
30	0\\
30.01	0\\
30.02	0\\
30.03	0\\
30.04	0\\
30.05	0\\
30.06	0\\
30.07	0\\
30.08	0\\
30.09	0\\
30.1	0\\
30.11	0\\
30.12	0\\
30.13	0\\
30.14	0\\
30.15	0\\
30.16	0\\
30.17	0\\
30.18	0\\
30.19	0\\
30.2	0\\
30.21	0\\
30.22	0\\
30.23	0\\
30.24	0\\
30.25	0\\
30.26	0\\
30.27	0\\
30.28	0\\
30.29	0\\
30.3	0\\
30.31	0\\
30.32	0\\
30.33	0\\
30.34	0\\
30.35	0\\
30.36	0\\
30.37	0\\
30.38	0\\
30.39	0\\
30.4	0\\
30.41	0\\
30.42	0\\
30.43	0\\
30.44	0\\
30.45	0\\
30.46	0\\
30.47	0\\
30.48	0\\
30.49	0\\
30.5	0\\
30.51	0\\
30.52	0\\
30.53	0\\
30.54	0\\
30.55	0\\
30.56	0\\
30.57	0\\
30.58	0\\
30.59	0\\
30.6	0\\
30.61	0\\
30.62	0\\
30.63	0\\
30.64	0\\
30.65	0\\
30.66	0\\
30.67	0\\
30.68	0\\
30.69	0\\
30.7	0\\
30.71	0\\
30.72	0\\
30.73	0\\
30.74	0\\
30.75	0\\
30.76	0\\
30.77	0\\
30.78	0\\
30.79	0\\
30.8	0\\
30.81	0\\
30.82	0\\
30.83	0\\
30.84	0\\
30.85	0\\
30.86	0\\
30.87	0\\
30.88	0\\
30.89	0\\
30.9	0\\
30.91	0\\
30.92	0\\
30.93	0\\
30.94	0\\
30.95	0\\
30.96	0\\
30.97	0\\
30.98	0\\
30.99	0\\
31	0\\
31.01	0\\
31.02	0\\
31.03	0\\
31.04	0\\
31.05	0\\
31.06	0\\
31.07	0\\
31.08	0\\
31.09	0\\
31.1	0\\
31.11	0\\
31.12	0\\
31.13	0\\
31.14	0\\
31.15	0\\
31.16	0\\
31.17	0\\
31.18	0\\
31.19	0\\
31.2	0\\
31.21	0\\
31.22	0\\
31.23	0\\
31.24	0\\
31.25	0\\
31.26	0\\
31.27	0\\
31.28	0\\
31.29	0\\
31.3	0\\
31.31	0\\
31.32	0\\
31.33	0\\
31.34	0\\
31.35	0\\
31.36	0\\
31.37	0\\
31.38	0\\
31.39	0\\
31.4	0\\
31.41	0\\
31.42	0\\
31.43	0\\
31.44	0\\
31.45	0\\
31.46	0\\
31.47	0\\
31.48	0\\
31.49	0\\
31.5	0\\
31.51	0\\
31.52	0\\
31.53	0\\
31.54	0\\
31.55	0\\
31.56	0\\
31.57	0\\
31.58	0\\
31.59	0\\
31.6	0\\
31.61	0\\
31.62	0\\
31.63	0\\
31.64	0\\
31.65	0\\
31.66	0\\
31.67	0\\
31.68	0\\
31.69	0\\
31.7	0\\
31.71	0\\
31.72	0\\
31.73	0\\
31.74	0\\
31.75	0\\
31.76	0\\
31.77	0\\
31.78	0\\
31.79	0\\
31.8	0\\
31.81	0\\
31.82	0\\
31.83	0\\
31.84	0\\
31.85	0\\
31.86	0\\
31.87	0\\
31.88	0\\
31.89	0\\
31.9	0\\
31.91	0\\
31.92	0\\
31.93	0\\
31.94	0\\
31.95	0\\
31.96	0\\
31.97	0\\
31.98	0\\
31.99	0\\
32	0\\
32.01	0\\
32.02	0\\
32.03	0\\
32.04	0\\
32.05	0\\
32.06	0\\
32.07	0\\
32.08	0\\
32.09	0\\
32.1	0\\
32.11	3.53033933628288e-07\\
32.12	9.03672751115248e-07\\
32.13	1.45457018996265e-06\\
32.14	2.00572640271512e-06\\
32.15	2.55714154200753e-06\\
32.16	3.10881576057187e-06\\
32.17	3.66074921123727e-06\\
32.18	4.21294204692307e-06\\
32.19	4.76539442065271e-06\\
32.2	5.31810648553982e-06\\
32.21	5.87107839478823e-06\\
32.22	6.42431030169893e-06\\
32.23	6.97780235967005e-06\\
32.24	7.53155472219685e-06\\
32.25	8.08556754287176e-06\\
32.26	8.63984097537046e-06\\
32.27	9.19437517347271e-06\\
32.28	9.74917029104155e-06\\
32.29	1.0304226482051e-05\\
32.3	1.08595439005585e-05\\
32.31	1.14151227007184e-05\\
32.32	1.19709630367754e-05\\
32.33	1.25270650630782e-05\\
32.34	1.30834289340589e-05\\
32.35	1.36400548042537e-05\\
32.36	1.41969428282818e-05\\
32.37	1.47540931608597e-05\\
32.38	1.53115059568082e-05\\
32.39	1.5869181371031e-05\\
32.4	1.64271195585222e-05\\
32.41	1.698532067438e-05\\
32.42	1.75437848737928e-05\\
32.43	1.81025123120321e-05\\
32.44	1.86615031444806e-05\\
32.45	1.92207575266043e-05\\
32.46	1.97802756139592e-05\\
32.47	2.03400575621987e-05\\
32.48	2.09001035270731e-05\\
32.49	2.14604136644161e-05\\
32.5	2.20209881301656e-05\\
32.51	2.25818270803424e-05\\
32.52	2.31429306710648e-05\\
32.53	2.37042990585482e-05\\
32.54	2.4265932399084e-05\\
32.55	2.48278308490751e-05\\
32.56	2.53899945650074e-05\\
32.57	2.59524237034639e-05\\
32.58	2.6515118421111e-05\\
32.59	2.70780788747121e-05\\
32.6	2.7641305221128e-05\\
32.61	2.82047976173025e-05\\
32.62	2.87685562202766e-05\\
32.63	2.93325811871747e-05\\
32.64	2.98968726752252e-05\\
32.65	3.04614308417467e-05\\
32.66	3.1026255844134e-05\\
32.67	3.15913478398933e-05\\
32.68	3.21567069866066e-05\\
32.69	3.27223334419535e-05\\
32.7	3.32882273637106e-05\\
32.71	3.38543889097306e-05\\
32.72	3.44208182379638e-05\\
32.73	3.49875155064572e-05\\
32.74	3.55544808733413e-05\\
32.75	3.61217144968437e-05\\
32.76	3.66892165352753e-05\\
32.77	3.72569871470371e-05\\
32.78	3.78250264906274e-05\\
32.79	3.83933347246276e-05\\
32.8	3.89619120077095e-05\\
32.81	3.95307584986418e-05\\
32.82	4.00998743562697e-05\\
32.83	4.06692597395425e-05\\
32.84	4.12389148074926e-05\\
32.85	4.18088397192359e-05\\
32.86	4.23790346339922e-05\\
32.87	4.29494997110508e-05\\
32.88	4.35202351098049e-05\\
32.89	4.40912409897382e-05\\
32.9	4.46625175104104e-05\\
32.91	4.52340648314717e-05\\
32.92	4.58058831126762e-05\\
32.93	4.63779725138475e-05\\
32.94	4.69503331949062e-05\\
32.95	4.75229653158632e-05\\
32.96	4.80958690368127e-05\\
32.97	4.86690445179391e-05\\
32.98	4.924249191951e-05\\
32.99	4.98162114018902e-05\\
33	5.03902031255207e-05\\
33.01	5.09644672509399e-05\\
33.02	5.15390039387692e-05\\
33.03	5.21138133497134e-05\\
33.04	5.26888956445676e-05\\
33.05	5.32642509842099e-05\\
33.06	5.38398795296088e-05\\
33.07	5.44157814418231e-05\\
33.08	5.49919568819945e-05\\
33.09	5.55684060113415e-05\\
33.1	5.61451289911863e-05\\
33.11	5.67221259829206e-05\\
33.12	5.72993971480334e-05\\
33.13	5.78769426480968e-05\\
33.14	5.84547626447593e-05\\
33.15	5.90328572997664e-05\\
33.16	5.96112267749402e-05\\
33.17	6.01898712321997e-05\\
33.18	6.07687908335333e-05\\
33.19	6.13479857410199e-05\\
33.2	6.19274561168282e-05\\
33.21	6.25072021232104e-05\\
33.22	6.3087223922495e-05\\
33.23	6.36675216771007e-05\\
33.24	6.42480955495295e-05\\
33.25	6.48289457023665e-05\\
33.26	6.54100722982803e-05\\
33.27	6.59914755000296e-05\\
33.28	6.65731554704424e-05\\
33.29	6.7155112372437e-05\\
33.3	6.77373463690151e-05\\
33.31	6.83198576232613e-05\\
33.32	6.89026462983439e-05\\
33.33	6.94857125575071e-05\\
33.34	7.00690565640857e-05\\
33.35	7.06526784814906e-05\\
33.36	7.12365784732161e-05\\
33.37	7.18207567028395e-05\\
33.38	7.24052133340217e-05\\
33.39	7.29899485304997e-05\\
33.4	7.35749624560939e-05\\
33.41	7.41602552747009e-05\\
33.42	7.47458271503076e-05\\
33.43	7.53316782469773e-05\\
33.44	7.59178087288565e-05\\
33.45	7.65042187601608e-05\\
33.46	7.70909085051966e-05\\
33.47	7.7677878128346e-05\\
33.48	7.82651277940749e-05\\
33.49	7.88526576669252e-05\\
33.5	7.94404679115152e-05\\
33.51	8.00285586925464e-05\\
33.52	8.06169301747969e-05\\
33.53	8.12055825231278e-05\\
33.54	8.17945159024766e-05\\
33.55	8.23837304778502e-05\\
33.56	8.29732264143457e-05\\
33.57	8.35630038771365e-05\\
33.58	8.41530630314655e-05\\
33.59	8.47434040426587e-05\\
33.6	8.53340270761255e-05\\
33.61	8.59249322973377e-05\\
33.62	8.65161198718503e-05\\
33.63	8.71075899653018e-05\\
33.64	8.76993427433997e-05\\
33.65	8.82913783719283e-05\\
33.66	8.88836970167547e-05\\
33.67	8.94762988438089e-05\\
33.68	9.00691840191106e-05\\
33.69	9.06623527087425e-05\\
33.7	9.12558050788631e-05\\
33.71	9.18495412957215e-05\\
33.72	9.24435615256222e-05\\
33.73	9.30378659349529e-05\\
33.74	9.36324546901707e-05\\
33.75	9.4227327957816e-05\\
33.76	9.48224859044916e-05\\
33.77	9.54179286968837e-05\\
33.78	9.60136565017478e-05\\
33.79	9.66096694859087e-05\\
33.8	9.72059678162676e-05\\
33.81	9.78025516598022e-05\\
33.82	9.83994211835523e-05\\
33.83	9.89965765546344e-05\\
33.84	9.9594017940241e-05\\
33.85	0.000100191745507627\\
33.86	0.000100789759424132\\
33.87	0.000101388059857148\\
33.88	0.000101986646974153\\
33.89	0.000102585520942694\\
33.9	0.000103184681930386\\
33.91	0.000103784130104909\\
33.92	0.000104383865634015\\
33.93	0.000104983888685531\\
33.94	0.000105584199427348\\
33.95	0.000106184798027431\\
33.96	0.000106785684653804\\
33.97	0.000107386859474569\\
33.98	0.00010798832265789\\
33.99	0.000108590074372007\\
34	0.000109192114785217\\
34.01	0.000109794444065885\\
34.02	0.000110397062382452\\
34.03	0.000110999969903422\\
34.04	0.000111603166797362\\
34.05	0.000112206653232909\\
34.06	0.000112810429378767\\
34.07	0.000113414495403696\\
34.08	0.000114018851476536\\
34.09	0.000114623497766184\\
34.1	0.000115228434441596\\
34.11	0.000115833661671803\\
34.12	0.0001164391796259\\
34.13	0.000117044988473036\\
34.14	0.00011765108838243\\
34.15	0.000118257479523362\\
34.16	0.000118864162065184\\
34.17	0.000119471136177295\\
34.18	0.00012007840202917\\
34.19	0.00012068595979034\\
34.2	0.000121293809630392\\
34.21	0.000121901951718989\\
34.22	0.000122510386225842\\
34.23	0.000123119113320733\\
34.24	0.000123728133173497\\
34.25	0.000124337445954033\\
34.26	0.000124947051832296\\
34.27	0.000125556950978302\\
34.28	0.000126167143562131\\
34.29	0.000126777629753917\\
34.3	0.000127388409723858\\
34.31	0.000127999483642199\\
34.32	0.000128610851679256\\
34.33	0.000129222514005392\\
34.34	0.000129834470791033\\
34.35	0.000130446722206662\\
34.36	0.000131059268422815\\
34.37	0.000131672109610084\\
34.38	0.000132285245939126\\
34.39	0.000132898677580644\\
34.4	0.000133512404705396\\
34.41	0.000134126427484199\\
34.42	0.000134740746087929\\
34.43	0.000135355360687504\\
34.44	0.000135970271453907\\
34.45	0.000136585478558167\\
34.46	0.000137200982171369\\
34.47	0.000137816782464648\\
34.48	0.000138432879609199\\
34.49	0.000139049273776261\\
34.5	0.000139665965137127\\
34.51	0.000140282953863147\\
34.52	0.00014090024012571\\
34.53	0.000141517824096264\\
34.54	0.000142135705946303\\
34.55	0.000142753885847377\\
34.56	0.000143372363971078\\
34.57	0.000143991140489054\\
34.58	0.000144610215572993\\
34.59	0.000145229589394641\\
34.6	0.000145849262125783\\
34.61	0.000146469233938254\\
34.62	0.000147089505003938\\
34.63	0.000147710075494766\\
34.64	0.000148330945582711\\
34.65	0.000148952115439797\\
34.66	0.000149573585238094\\
34.67	0.000150195355149707\\
34.68	0.000150817425346797\\
34.69	0.000151439796001566\\
34.7	0.00015206246728626\\
34.71	0.000152685439373162\\
34.72	0.000153308712434609\\
34.73	0.000153932286642969\\
34.74	0.000154556162170659\\
34.75	0.000155180339190132\\
34.76	0.000155804817873896\\
34.77	0.000156429598394486\\
34.78	0.00015705468092448\\
34.79	0.000157680065636497\\
34.8	0.000158305752703196\\
34.81	0.00015893174229728\\
34.82	0.00015955803459148\\
34.83	0.000160184629758572\\
34.84	0.000160811527971372\\
34.85	0.000161438729402733\\
34.86	0.000162066234225537\\
34.87	0.000162694042612703\\
34.88	0.000163322154737196\\
34.89	0.000163950570772009\\
34.9	0.000164579290890171\\
34.91	0.000165208315264752\\
34.92	0.000165837644068842\\
34.93	0.000166467277475574\\
34.94	0.000167097215658114\\
34.95	0.000167727458789665\\
34.96	0.00016835800704345\\
34.97	0.000168988860592731\\
34.98	0.000169620019610808\\
34.99	0.000170251484270995\\
35	0.000170883254746651\\
35.01	0.00017151533121116\\
35.02	0.000172147713837931\\
35.03	0.000172780402800408\\
35.04	0.000173413398272058\\
35.05	0.000174046700426381\\
35.06	0.000174680309436896\\
35.07	0.000175314225477161\\
35.08	0.00017594844872075\\
35.09	0.000176582979341269\\
35.1	0.00017721781751235\\
35.11	0.000177852963407638\\
35.12	0.00017848841720082\\
35.13	0.000179124179065591\\
35.14	0.000179760249175674\\
35.15	0.000180396627704818\\
35.16	0.000181033314826794\\
35.17	0.000181670310715386\\
35.18	0.000182307615544414\\
35.19	0.000182945229487703\\
35.2	0.000183583152719109\\
35.21	0.000184221385412499\\
35.22	0.000184859927741768\\
35.23	0.000185498779880826\\
35.24	0.000186137942003589\\
35.25	0.000186777414284009\\
35.26	0.000187417196896043\\
35.27	0.000188057290013664\\
35.28	0.000188697693810871\\
35.29	0.000189338408461664\\
35.3	0.000189979434140064\\
35.31	0.000190620771020104\\
35.32	0.000191262419275834\\
35.33	0.000191904379081315\\
35.34	0.000192546650610616\\
35.35	0.00019318923403782\\
35.36	0.000193832129537018\\
35.37	0.00019447533728232\\
35.38	0.000195118857447843\\
35.39	0.000195762690207706\\
35.4	0.000196406835736039\\
35.41	0.000197051294206982\\
35.42	0.000197696065794679\\
35.43	0.000198341150673284\\
35.44	0.000198986549016962\\
35.45	0.000199632260999873\\
35.46	0.000200278286796182\\
35.47	0.000200924626580064\\
35.48	0.000201571280525698\\
35.49	0.000202218248807264\\
35.5	0.000202865531598942\\
35.51	0.00020351312907492\\
35.52	0.000204161041409377\\
35.53	0.000204809268776501\\
35.54	0.000205457811350472\\
35.55	0.000206106669305477\\
35.56	0.000206755842815695\\
35.57	0.000207405332055308\\
35.58	0.000208055137198487\\
35.59	0.000208705258419407\\
35.6	0.000209355695892241\\
35.61	0.000210006449791147\\
35.62	0.000210657520290279\\
35.63	0.000211308907563795\\
35.64	0.000211960611785834\\
35.65	0.000212612633130535\\
35.66	0.000213264971772029\\
35.67	0.000213917627884433\\
35.68	0.000214570601641852\\
35.69	0.000215223893218389\\
35.7	0.000215877502788127\\
35.71	0.000216531430525142\\
35.72	0.000217185676603504\\
35.73	0.000217840241197254\\
35.74	0.000218495124480432\\
35.75	0.00021915032662706\\
35.76	0.000219805847811137\\
35.77	0.000220461688206657\\
35.78	0.00022111784798759\\
35.79	0.000221774327327888\\
35.8	0.000222431126401489\\
35.81	0.000223088245382308\\
35.82	0.000223745684444242\\
35.83	0.000224403443761172\\
35.84	0.000225061523506945\\
35.85	0.000225719923855394\\
35.86	0.000226378644980332\\
35.87	0.000227037687055544\\
35.88	0.000227697050254792\\
35.89	0.000228356734751807\\
35.9	0.000229016740720303\\
35.91	0.000229677068333961\\
35.92	0.000230337717766434\\
35.93	0.000230998689191353\\
35.94	0.000231659982782316\\
35.95	0.000232321598712885\\
35.96	0.000232983537156603\\
35.97	0.000233645798286969\\
35.98	0.000234308382277462\\
35.99	0.000234971289301514\\
36	0.00023563451953254\\
36.01	0.000236298073143903\\
36.02	0.000236961950308937\\
36.03	0.000237626151200945\\
36.04	0.000238290675993187\\
36.05	0.000238955524858885\\
36.06	0.000239620697971218\\
36.07	0.000240286195503335\\
36.08	0.000240952017628339\\
36.09	0.000241618164519293\\
36.1	0.000242284636349212\\
36.11	0.000242951433291075\\
36.12	0.000243618555517806\\
36.13	0.000244286003202296\\
36.14	0.000244953776517391\\
36.15	0.000245621875635878\\
36.16	0.000246290300730499\\
36.17	0.000246959051973958\\
36.18	0.0002476281295389\\
36.19	0.000248297533597924\\
36.2	0.00024896726432358\\
36.21	0.000249637321888353\\
36.22	0.000250307706464691\\
36.23	0.000250978418224974\\
36.24	0.000251649457341539\\
36.25	0.000252320823986657\\
36.26	0.000252992518332551\\
36.27	0.000253664540551382\\
36.28	0.000254336890815249\\
36.29	0.000255009569296201\\
36.3	0.000255682576166219\\
36.31	0.000256355911597221\\
36.32	0.00025702957576107\\
36.33	0.000257703568829551\\
36.34	0.000258377890974403\\
36.35	0.000259052542367293\\
36.36	0.000259727523179813\\
36.37	0.000260402833583501\\
36.38	0.000261078473749815\\
36.39	0.000261754443850154\\
36.4	0.000262430744055839\\
36.41	0.000263107374538121\\
36.42	0.000263784335468184\\
36.43	0.000264461627017133\\
36.44	0.000265139249356006\\
36.45	0.000265817202655748\\
36.46	0.000266495487087251\\
36.47	0.000267174102821316\\
36.48	0.00026785305002866\\
36.49	0.000268532328879939\\
36.5	0.00026921193954571\\
36.51	0.000269891882196463\\
36.52	0.000270572157002595\\
36.53	0.000271252764134429\\
36.54	0.000271933703762189\\
36.55	0.000272614976056032\\
36.56	0.000273296581186008\\
36.57	0.000273978519322093\\
36.58	0.000274660790634172\\
36.59	0.000275343395292039\\
36.6	0.000276026333465393\\
36.61	0.00027670960532384\\
36.62	0.000277393211036904\\
36.63	0.000278077150773999\\
36.64	0.000278761424704455\\
36.65	0.000279446032997496\\
36.66	0.00028013097582226\\
36.67	0.000280816253347776\\
36.68	0.000281501865742981\\
36.69	0.000282187813176703\\
36.7	0.000282874095817669\\
36.71	0.000283560713834501\\
36.72	0.000284247667395723\\
36.73	0.000284934956669751\\
36.74	0.000285622581824893\\
36.75	0.000286310543029342\\
36.76	0.000286998840451191\\
36.77	0.000287687474258419\\
36.78	0.00028837644461889\\
36.79	0.000289065751700367\\
36.8	0.000289755395670485\\
36.81	0.000290445376696764\\
36.82	0.000291135694946619\\
36.83	0.000291826350587339\\
36.84	0.000292517343786089\\
36.85	0.00029320867470993\\
36.86	0.000293900343525784\\
36.87	0.000294592350400462\\
36.88	0.000295284695500643\\
36.89	0.000295977378992882\\
36.9	0.000296670401043614\\
36.91	0.000297363761819144\\
36.92	0.000298057461485646\\
36.93	0.000298751500209159\\
36.94	0.000299445878155599\\
36.95	0.000300140595490744\\
36.96	0.000300835652380239\\
36.97	0.000301531048989598\\
36.98	0.000302226785484187\\
36.99	0.000302922862029244\\
37	0.000303619278789864\\
37.01	0.000304316035931007\\
37.02	0.000305013133617478\\
37.03	0.000305710572013944\\
37.04	0.00030640835128494\\
37.05	0.000307106471594834\\
37.06	0.000307804933107864\\
37.07	0.000308503735988105\\
37.08	0.000309202880399491\\
37.09	0.000309902366505806\\
37.1	0.000310602194470676\\
37.11	0.000311302364457568\\
37.12	0.000312002876629809\\
37.13	0.000312703731150554\\
37.14	0.000313404928182805\\
37.15	0.000314106467889402\\
37.16	0.000314808350433023\\
37.17	0.000315510575976188\\
37.18	0.000316213144681249\\
37.19	0.000316916056710401\\
37.2	0.000317619312225653\\
37.21	0.000318322911388863\\
37.22	0.000319026854361713\\
37.23	0.000319731141305714\\
37.24	0.000320435772382201\\
37.25	0.000321140747752337\\
37.26	0.000321846067577111\\
37.27	0.000322551732017326\\
37.28	0.000323257741233623\\
37.29	0.000323964095386445\\
37.3	0.000324670794636064\\
37.31	0.000325377839142563\\
37.32	0.000326085229065839\\
37.33	0.000326792964565607\\
37.34	0.000327501045801398\\
37.35	0.000328209472932543\\
37.36	0.000328918246118186\\
37.37	0.000329627365517275\\
37.38	0.000330336831288573\\
37.39	0.000331046643590641\\
37.4	0.000331756802581844\\
37.41	0.000332467308420348\\
37.42	0.00033317816126411\\
37.43	0.000333889361270899\\
37.44	0.00033460090859827\\
37.45	0.000335312803403577\\
37.46	0.000336025045843971\\
37.47	0.000336737636076383\\
37.48	0.00033745057425754\\
37.49	0.000338163860543958\\
37.5	0.000338877495091934\\
37.51	0.000339591478057559\\
37.52	0.000340305809596701\\
37.53	0.000341020489865007\\
37.54	0.000341735519017908\\
37.55	0.000342450897210612\\
37.56	0.000343166624598101\\
37.57	0.000343882701335133\\
37.58	0.000344599127576235\\
37.59	0.000345315903475717\\
37.6	0.000346033029187648\\
37.61	0.000346750504865864\\
37.62	0.000347468330663972\\
37.63	0.000348186506735335\\
37.64	0.000348905033233086\\
37.65	0.000349623910310119\\
37.66	0.000350343138119089\\
37.67	0.000351062716812393\\
37.68	0.000351782646542197\\
37.69	0.000352502927460419\\
37.7	0.000353223559718722\\
37.71	0.000353944543468529\\
37.72	0.000354665878861002\\
37.73	0.000355387566047055\\
37.74	0.000356109605177339\\
37.75	0.000356831996402254\\
37.76	0.000357554739871938\\
37.77	0.000358277835736262\\
37.78	0.000359001284144844\\
37.79	0.00035972508524703\\
37.8	0.000360449239191908\\
37.81	0.000361173746128284\\
37.82	0.000361898606204697\\
37.83	0.000362623819569426\\
37.84	0.000363349386370455\\
37.85	0.000364075306755508\\
37.86	0.000364801580872015\\
37.87	0.000365528208867143\\
37.88	0.000366255190887767\\
37.89	0.000366982527080471\\
37.9	0.00036771021759157\\
37.91	0.000368438262567071\\
37.92	0.000369166662152705\\
37.93	0.000369895416493904\\
37.94	0.000370624525735809\\
37.95	0.000371353990023263\\
37.96	0.000372083809500802\\
37.97	0.00037281398431268\\
37.98	0.000373544514602829\\
37.99	0.000374275400514892\\
38	0.000375006642192192\\
38.01	0.000375738239777754\\
38.02	0.000376470193414281\\
38.03	0.000377202503244176\\
38.04	0.000377935169409517\\
38.05	0.000378668192052074\\
38.06	0.000379401571313283\\
38.07	0.000380135307334273\\
38.08	0.000380869400255841\\
38.09	0.000381603850218459\\
38.1	0.000382338657362272\\
38.11	0.000383073821827097\\
38.12	0.000383809343752421\\
38.13	0.00038454522327739\\
38.14	0.000385281460540815\\
38.15	0.000386018055681163\\
38.16	0.000386755008836573\\
38.17	0.00038749232014483\\
38.18	0.000388229989743381\\
38.19	0.000388968017769317\\
38.2	0.000389706404359383\\
38.21	0.000390445149649969\\
38.22	0.000391184253777113\\
38.23	0.000391923716876491\\
38.24	0.000392663539083432\\
38.25	0.000393403720532891\\
38.26	0.000394144261359461\\
38.27	0.000394885161697375\\
38.28	0.000395626421680492\\
38.29	0.000396368041442295\\
38.3	0.000397110021115907\\
38.31	0.000397852360834068\\
38.32	0.000398595060729139\\
38.33	0.000399338120933103\\
38.34	0.000400081541577557\\
38.35	0.000400825322793709\\
38.36	0.000401569464712391\\
38.37	0.000402313967464034\\
38.38	0.000403058831178686\\
38.39	0.000403804055985994\\
38.4	0.000404549642015202\\
38.41	0.000405295589395165\\
38.42	0.000406041898254328\\
38.43	0.000406788568720735\\
38.44	0.000407535600922021\\
38.45	0.000408282994985409\\
38.46	0.000409030751037709\\
38.47	0.000409778869205318\\
38.48	0.00041052734961422\\
38.49	0.000411276192389974\\
38.5	0.000412025397657709\\
38.51	0.000412774965542144\\
38.52	0.000413524896167555\\
38.53	0.000414275189657795\\
38.54	0.000415025846136285\\
38.55	0.000415776865726004\\
38.56	0.0004165282485495\\
38.57	0.000417279994728875\\
38.58	0.000418032104385783\\
38.59	0.00041878457764144\\
38.6	0.000419537414616609\\
38.61	0.000420290615431605\\
38.62	0.000421044180206276\\
38.63	0.000421798109060026\\
38.64	0.000422552402111795\\
38.65	0.000423307059480051\\
38.66	0.00042406208128281\\
38.67	0.000424817467637617\\
38.68	0.000425573218661532\\
38.69	0.000426329334471162\\
38.7	0.000427085815182623\\
38.71	0.000427842660911552\\
38.72	0.000428599871773107\\
38.73	0.000429357447881959\\
38.74	0.000430115389352295\\
38.75	0.000430873696297802\\
38.76	0.000431632368831682\\
38.77	0.000432391407066635\\
38.78	0.000433150811114863\\
38.79	0.000433910581088069\\
38.8	0.000434670717097442\\
38.81	0.000435431219253671\\
38.82	0.000436192087666924\\
38.83	0.000436953322446865\\
38.84	0.000437714923702627\\
38.85	0.00043847689154284\\
38.86	0.000439239226075597\\
38.87	0.000440001927408465\\
38.88	0.00044076499564849\\
38.89	0.000441528430902179\\
38.9	0.0004422922332755\\
38.91	0.000443056402873891\\
38.92	0.000443820939802241\\
38.93	0.000444585844164902\\
38.94	0.000445351116065668\\
38.95	0.000446116755607788\\
38.96	0.000446882762893952\\
38.97	0.0004476491380263\\
38.98	0.000448415881106404\\
38.99	0.000449182992235272\\
39	0.000449950471513354\\
39.01	0.000450718319040512\\
39.02	0.000451486534916057\\
39.03	0.000452255119238708\\
39.04	0.000453024072106609\\
39.05	0.00045379339361732\\
39.06	0.000454563083867814\\
39.07	0.00045533314295447\\
39.08	0.000456103570973081\\
39.09	0.000456874368018841\\
39.1	0.000457645534186341\\
39.11	0.000458417069569568\\
39.12	0.000459188974261913\\
39.13	0.00045996124835615\\
39.14	0.000460733891944433\\
39.15	0.000461506905118307\\
39.16	0.0004622802879687\\
39.17	0.000463054040585907\\
39.18	0.000463828163059607\\
39.19	0.000464602655478832\\
39.2	0.000465377517931999\\
39.21	0.000466152750506876\\
39.22	0.00046692835329059\\
39.23	0.000467704326369625\\
39.24	0.000468480669829824\\
39.25	0.000469257383756368\\
39.26	0.000470034468233786\\
39.27	0.00047081192334595\\
39.28	0.000471589749176063\\
39.29	0.000472367945806677\\
39.3	0.000473146513319658\\
39.31	0.000473925451796205\\
39.32	0.000474704761316842\\
39.33	0.000475484442007325\\
39.34	0.000476264494117307\\
39.35	0.00047704491789672\\
39.36	0.000477825713595781\\
39.37	0.000478606881464984\\
39.38	0.000479388421755114\\
39.39	0.000480170334717235\\
39.4	0.000480952620602701\\
39.41	0.000481735279663144\\
39.42	0.00048251831215048\\
39.43	0.000483301718316918\\
39.44	0.00048408549841495\\
39.45	0.000484869652697353\\
39.46	0.000485654181417196\\
39.47	0.00048643908482783\\
39.48	0.000487224363182895\\
39.49	0.000488010016736318\\
39.5	0.000488796045742312\\
39.51	0.000489582450455391\\
39.52	0.000490369231130348\\
39.53	0.000491156388022267\\
39.54	0.000491943921386531\\
39.55	0.000492731831478806\\
39.56	0.000493520118555052\\
39.57	0.000494308782871516\\
39.58	0.000495097824684751\\
39.59	0.000495887244251585\\
39.6	0.000496677041829159\\
39.61	0.000497467217674886\\
39.62	0.000498257772046491\\
39.63	0.000499048705201989\\
39.64	0.000499840017399687\\
39.65	0.000500631708898192\\
39.66	0.000501423779956399\\
39.67	0.000502216230833505\\
39.68	0.00050300906178901\\
39.69	0.000503802273082707\\
39.7	0.000504595864974679\\
39.71	0.000505389837725322\\
39.72	0.000506184191595316\\
39.73	0.000506978926845654\\
39.74	0.00050777404373762\\
39.75	0.000508569542532797\\
39.76	0.000509365423493079\\
39.77	0.000510161686880652\\
39.78	0.000510958332958014\\
39.79	0.000511755361987949\\
39.8	0.000512552774233559\\
39.81	0.000513350569958243\\
39.82	0.000514148749425701\\
39.83	0.000514947312899944\\
39.84	0.000515746260645281\\
39.85	0.000516545592926333\\
39.86	0.000517345310008022\\
39.87	0.000518145412155578\\
39.88	0.000518945899634533\\
39.89	0.00051974677271073\\
39.9	0.000520548031650317\\
39.91	0.000521349676719754\\
39.92	0.000522151708185807\\
39.93	0.000522954126315554\\
39.94	0.000523756931376379\\
39.95	0.000524560123635977\\
39.96	0.000525363703362348\\
39.97	0.000526167670823814\\
39.98	0.000526972026288999\\
39.99	0.000527776770026842\\
40	0.000528581902306598\\
40.01	0.00052938742339783\\
};
\addplot [color=mycolor1,solid,forget plot]
  table[row sep=crcr]{%
40.01	0.00052938742339783\\
40.02	0.000530193333570413\\
40.03	0.000530999633094539\\
40.04	0.000531806322240716\\
40.05	0.000532613401279768\\
40.06	0.000533420870482824\\
40.07	0.000534228730121343\\
40.08	0.000535036980467085\\
40.09	0.000535845621792139\\
40.1	0.000536654654368904\\
40.11	0.000537464078470107\\
40.12	0.000538273894368779\\
40.13	0.000539084102338279\\
40.14	0.000539894702652276\\
40.15	0.000540705695584773\\
40.16	0.000541517081410085\\
40.17	0.000542328860402848\\
40.18	0.000543141032838014\\
40.19	0.000543953598990871\\
40.2	0.00054476655913701\\
40.21	0.000545579913552363\\
40.22	0.000546393662513175\\
40.23	0.00054720780629601\\
40.24	0.00054802234517777\\
40.25	0.000548837279435672\\
40.26	0.000549652609347265\\
40.27	0.000550468335190415\\
40.28	0.000551284457243319\\
40.29	0.000552100975784495\\
40.3	0.000552917891092797\\
40.31	0.000553735203447404\\
40.32	0.000554552913127823\\
40.33	0.000555371020413885\\
40.34	0.000556189525585754\\
40.35	0.000557008428923929\\
40.36	0.000557827730709226\\
40.37	0.000558647431222803\\
40.38	0.000559467530746142\\
40.39	0.000560288029561061\\
40.4	0.000561108927949715\\
40.41	0.000561930226194579\\
40.42	0.00056275192457847\\
40.43	0.000563574023384535\\
40.44	0.000564396522896256\\
40.45	0.000565219423397455\\
40.46	0.000566042725172279\\
40.47	0.000566866428505224\\
40.48	0.000567690533681109\\
40.49	0.000568515040985096\\
40.5	0.000569339950702685\\
40.51	0.000570165263119711\\
40.52	0.000570990978522347\\
40.53	0.000571817097197107\\
40.54	0.000572643619430845\\
40.55	0.000573470545510754\\
40.56	0.000574297875724369\\
40.57	0.000575125610359564\\
40.58	0.000575953749704546\\
40.59	0.000576782294047881\\
40.6	0.000577611243678458\\
40.61	0.000578440598885531\\
40.62	0.000579270359958674\\
40.63	0.000580100527187821\\
40.64	0.000580931100863248\\
40.65	0.000581762081275576\\
40.66	0.000582593468715767\\
40.67	0.000583425263475129\\
40.68	0.000584257465845318\\
40.69	0.000585090076118337\\
40.7	0.000585923094586542\\
40.71	0.000586756521542629\\
40.72	0.000587590357279642\\
40.73	0.000588424602090978\\
40.74	0.000589259256270361\\
40.75	0.00059009432011186\\
40.76	0.000590929793909886\\
40.77	0.00059176567795919\\
40.78	0.000592601972554854\\
40.79	0.00059343867799231\\
40.8	0.000594275794567334\\
40.81	0.000595113322576044\\
40.82	0.000595951262314896\\
40.83	0.000596789614080696\\
40.84	0.000597628378170595\\
40.85	0.000598467554882083\\
40.86	0.000599307144512992\\
40.87	0.000600147147361514\\
40.88	0.000600987563726181\\
40.89	0.000601828393905872\\
40.9	0.000602669638199813\\
40.91	0.000603511296907576\\
40.92	0.000604353370329089\\
40.93	0.000605195858764625\\
40.94	0.000606038762514811\\
40.95	0.000606882081880623\\
40.96	0.000607725817163382\\
40.97	0.000608569968664777\\
40.98	0.000609414536686829\\
40.99	0.000610259521531931\\
41	0.000611104923502817\\
41.01	0.000611950742902587\\
41.02	0.000612796980034684\\
41.03	0.000613643635202908\\
41.04	0.000614490708711425\\
41.05	0.000615338200864748\\
41.06	0.000616186111967759\\
41.07	0.000617034442325685\\
41.08	0.000617883192244116\\
41.09	0.000618732362029001\\
41.1	0.000619581951986657\\
41.11	0.000620431962423749\\
41.12	0.00062128239364731\\
41.13	0.000622133245964732\\
41.14	0.000622984519683775\\
41.15	0.000623836215112555\\
41.16	0.000624688332559548\\
41.17	0.000625540872333609\\
41.18	0.000626393834743944\\
41.19	0.00062724722010013\\
41.2	0.000628101028712115\\
41.21	0.000628955260890199\\
41.22	0.000629809916945064\\
41.23	0.000630664997187752\\
41.24	0.000631520501929675\\
41.25	0.000632376431482617\\
41.26	0.000633232786158726\\
41.27	0.000634089566270521\\
41.28	0.000634946772130898\\
41.29	0.000635804404053128\\
41.3	0.000636662462350841\\
41.31	0.000637520947338044\\
41.32	0.000638379859329123\\
41.33	0.000639239198638834\\
41.34	0.000640098965582307\\
41.35	0.000640959160475053\\
41.36	0.000641819783632952\\
41.37	0.000642680835372264\\
41.38	0.000643542316009632\\
41.39	0.000644404225862066\\
41.4	0.000645266565246966\\
41.41	0.000646129334482097\\
41.42	0.000646992533885615\\
41.43	0.000647856163776057\\
41.44	0.000648720224472335\\
41.45	0.000649584716293748\\
41.46	0.000650449639559979\\
41.47	0.00065131499459109\\
41.48	0.00065218078170752\\
41.49	0.00065304700123011\\
41.5	0.000653913653480068\\
41.51	0.000654780738779004\\
41.52	0.000655648257448904\\
41.53	0.000656516209812143\\
41.54	0.000657384596191482\\
41.55	0.00065825341691008\\
41.56	0.000659122672291478\\
41.57	0.000659992362659602\\
41.58	0.000660862488338777\\
41.59	0.000661733049653723\\
41.6	0.000662604046929534\\
41.61	0.000663475480491713\\
41.62	0.000664347350666152\\
41.63	0.000665219657779138\\
41.64	0.000666092402157349\\
41.65	0.000666965584127856\\
41.66	0.000667839204018132\\
41.67	0.000668713262156048\\
41.68	0.000669587758869862\\
41.69	0.000670462694488241\\
41.7	0.000671338069340249\\
41.71	0.000672213883755345\\
41.72	0.000673090138063388\\
41.73	0.000673966832594637\\
41.74	0.000674843967679764\\
41.75	0.000675721543649831\\
41.76	0.000676599560836301\\
41.77	0.000677478019571055\\
41.78	0.000678356920186363\\
41.79	0.000679236263014908\\
41.8	0.00068011604838978\\
41.81	0.000680996276644467\\
41.82	0.000681876948112878\\
41.83	0.000682758063129313\\
41.84	0.000683639622028491\\
41.85	0.00068452162514554\\
41.86	0.000685404072815997\\
41.87	0.000686286965375808\\
41.88	0.00068717030316133\\
41.89	0.000688054086509335\\
41.9	0.000688938315757004\\
41.91	0.000689822991241935\\
41.92	0.000690708113302144\\
41.93	0.000691593682276052\\
41.94	0.000692479698502502\\
41.95	0.000693366162320755\\
41.96	0.000694253074070478\\
41.97	0.000695140434091773\\
41.98	0.000696028242725154\\
41.99	0.000696916500311541\\
42	0.000697805207192294\\
42.01	0.000698694363709187\\
42.02	0.000699583970204413\\
42.03	0.000700474027020587\\
42.04	0.000701364534500745\\
42.05	0.000702255492988357\\
42.06	0.000703146902827308\\
42.07	0.000704038764361911\\
42.08	0.000704931077936906\\
42.09	0.00070582384389746\\
42.1	0.000706717062589166\\
42.11	0.000707610734358051\\
42.12	0.000708504859550561\\
42.13	0.000709399438513579\\
42.14	0.000710294471594416\\
42.15	0.000711189959140816\\
42.16	0.000712085901500958\\
42.17	0.000712982299023446\\
42.18	0.000713879152057326\\
42.19	0.000714776460952068\\
42.2	0.000715674226057589\\
42.21	0.000716572447724238\\
42.22	0.000717471126302799\\
42.23	0.00071837026214449\\
42.24	0.000719269855600974\\
42.25	0.000720169907024351\\
42.26	0.00072107041676716\\
42.27	0.000721971385182374\\
42.28	0.000722872812623426\\
42.29	0.000723774699444173\\
42.3	0.00072467704599892\\
42.31	0.000725579852642426\\
42.32	0.000726483119729879\\
42.33	0.000727386847616918\\
42.34	0.000728291036659633\\
42.35	0.000729195687214558\\
42.36	0.000730100799638679\\
42.37	0.000731006374289418\\
42.38	0.000731912411524663\\
42.39	0.000732818911702737\\
42.4	0.000733725875182417\\
42.41	0.000734633302322943\\
42.42	0.000735541193483999\\
42.43	0.000736449549025721\\
42.44	0.000737358369308702\\
42.45	0.000738267654693987\\
42.46	0.000739177405543086\\
42.47	0.000740087622217951\\
42.48	0.000740998305081009\\
42.49	0.000741909454495128\\
42.5	0.00074282107082365\\
42.51	0.000743733154430368\\
42.52	0.000744645705679531\\
42.53	0.000745558724935862\\
42.54	0.00074647221256454\\
42.55	0.000747386168931212\\
42.56	0.000748300594401979\\
42.57	0.000749215489343417\\
42.58	0.000750130854122566\\
42.59	0.00075104668910693\\
42.6	0.000751962994664478\\
42.61	0.000752879771163652\\
42.62	0.000753797018973365\\
42.63	0.000754714738462989\\
42.64	0.00075563293000238\\
42.65	0.00075655159396186\\
42.66	0.000757470730712223\\
42.67	0.000758390340624734\\
42.68	0.000759310424071145\\
42.69	0.000760230981423665\\
42.7	0.000761152013054989\\
42.71	0.00076207351933829\\
42.72	0.000762995500647214\\
42.73	0.000763917957355885\\
42.74	0.000764840889838914\\
42.75	0.000765764298471389\\
42.76	0.000766688183628877\\
42.77	0.000767612545687427\\
42.78	0.000768537385023568\\
42.79	0.00076946270201432\\
42.8	0.000770388497037186\\
42.81	0.000771314770470148\\
42.82	0.000772241522691682\\
42.83	0.000773168754080747\\
42.84	0.000774096465016798\\
42.85	0.000775024655879766\\
42.86	0.000775953327050084\\
42.87	0.000776882478908669\\
42.88	0.000777812111836938\\
42.89	0.000778742226216787\\
42.9	0.000779672822430619\\
42.91	0.000780603900861321\\
42.92	0.000781535461892281\\
42.93	0.000782467505907387\\
42.94	0.000783400033291019\\
42.95	0.00078433304442805\\
42.96	0.000785266539703865\\
42.97	0.000786200519504343\\
42.98	0.000787134984215856\\
42.99	0.00078806993422529\\
43	0.000789005369920028\\
43.01	0.000789941291687961\\
43.02	0.000790877699917474\\
43.03	0.00079181459499747\\
43.04	0.000792751977317346\\
43.05	0.000793689847267021\\
43.06	0.000794628205236916\\
43.07	0.000795567051617957\\
43.08	0.000796506386801579\\
43.09	0.000797446211179734\\
43.1	0.000798386525144891\\
43.11	0.000799327329090022\\
43.12	0.000800268623408609\\
43.13	0.000801210408494665\\
43.14	0.000802152684742707\\
43.15	0.000803095452547774\\
43.16	0.000804038712305423\\
43.17	0.000804982464411727\\
43.18	0.000805926709263272\\
43.19	0.000806871447257176\\
43.2	0.000807816678791082\\
43.21	0.000808762404263141\\
43.22	0.000809708624072036\\
43.23	0.000810655338616975\\
43.24	0.000811602548297687\\
43.25	0.000812550253514434\\
43.26	0.000813498454668\\
43.27	0.000814447152159697\\
43.28	0.000815396346391377\\
43.29	0.000816346037765404\\
43.3	0.000817296226684694\\
43.31	0.00081824691355268\\
43.32	0.00081919809877333\\
43.33	0.000820149782751155\\
43.34	0.000821101965891191\\
43.35	0.000822054648599017\\
43.36	0.000823007831280753\\
43.37	0.000823961514343052\\
43.38	0.000824915698193103\\
43.39	0.000825870383238642\\
43.4	0.000826825569887939\\
43.41	0.000827781258549813\\
43.42	0.000828737449633625\\
43.43	0.000829694143549282\\
43.44	0.000830651340707235\\
43.45	0.00083160904151848\\
43.46	0.000832567246394557\\
43.47	0.000833525955747566\\
43.48	0.000834485169990143\\
43.49	0.000835444889535485\\
43.5	0.000836405114797337\\
43.51	0.000837365846189993\\
43.52	0.000838327084128308\\
43.53	0.000839288829027687\\
43.54	0.000840251081304096\\
43.55	0.000841213841374047\\
43.56	0.000842177109654618\\
43.57	0.000843140886563447\\
43.58	0.000844105172518726\\
43.59	0.000845069967939212\\
43.6	0.000846035273244228\\
43.61	0.000847001088853654\\
43.62	0.000847967415187931\\
43.63	0.000848934252668077\\
43.64	0.00084990160171567\\
43.65	0.000850869462752854\\
43.66	0.000851837836202338\\
43.67	0.000852806722487408\\
43.68	0.00085377612203192\\
43.69	0.000854746035260305\\
43.7	0.00085571646259755\\
43.71	0.000856687404469238\\
43.72	0.000857658861301513\\
43.73	0.000858630833521098\\
43.74	0.00085960332155529\\
43.75	0.000860576325831974\\
43.76	0.000861549846779608\\
43.77	0.000862523884827229\\
43.78	0.00086349844040446\\
43.79	0.000864473513941505\\
43.8	0.000865449105869151\\
43.81	0.000866425216618769\\
43.82	0.000867401846622318\\
43.83	0.000868378996312341\\
43.84	0.000869356666121972\\
43.85	0.000870334856484939\\
43.86	0.000871313567835548\\
43.87	0.000872292800608708\\
43.88	0.000873272555239918\\
43.89	0.000874252832165273\\
43.9	0.000875233631821452\\
43.91	0.000876214954645745\\
43.92	0.000877196801076029\\
43.93	0.000878179171550787\\
43.94	0.0008791620665091\\
43.95	0.000880145486390642\\
43.96	0.000881129431635694\\
43.97	0.000882113902685139\\
43.98	0.000883098899980471\\
43.99	0.000884084423963781\\
44	0.00088507047507777\\
44.01	0.000886057053765751\\
44.02	0.000887044160471633\\
44.03	0.000888031795639947\\
44.04	0.000889019959715832\\
44.05	0.000890008653145043\\
44.06	0.000890997876373936\\
44.07	0.000891987629849503\\
44.08	0.000892977914019329\\
44.09	0.000893968729331632\\
44.1	0.000894960076235234\\
44.11	0.000895951955179594\\
44.12	0.000896944366614784\\
44.13	0.000897937310991491\\
44.14	0.000898930788761035\\
44.15	0.000899924800375353\\
44.16	0.00090091934628702\\
44.17	0.000901914426949223\\
44.18	0.000902910042815779\\
44.19	0.000903906194341143\\
44.2	0.000904902881980391\\
44.21	0.000905900106189239\\
44.22	0.000906897867424032\\
44.23	0.000907896166141751\\
44.24	0.000908895002800003\\
44.25	0.000909894377857046\\
44.26	0.000910894291771762\\
44.27	0.000911894745003687\\
44.28	0.000912895738012981\\
44.29	0.000913897271260455\\
44.3	0.000914899345207568\\
44.31	0.000915901960316406\\
44.32	0.00091690511704972\\
44.33	0.000917908815870894\\
44.34	0.00091891305724396\\
44.35	0.000919917841633609\\
44.36	0.000920923169505171\\
44.37	0.000921929041324633\\
44.38	0.000922935457558637\\
44.39	0.000923942418674481\\
44.4	0.000924949925140105\\
44.41	0.000925957977424119\\
44.42	0.000926966575995787\\
44.43	0.000927975721325032\\
44.44	0.000928985413882442\\
44.45	0.00092999565413926\\
44.46	0.000931006442567391\\
44.47	0.000932017779639409\\
44.48	0.000933029665828554\\
44.49	0.000934042101608731\\
44.5	0.00093505508745452\\
44.51	0.000936068623841158\\
44.52	0.000937082711244562\\
44.53	0.000938097350141318\\
44.54	0.000939112541008688\\
44.55	0.000940128284324611\\
44.56	0.000941144580567696\\
44.57	0.000942161430217234\\
44.58	0.000943178833753194\\
44.59	0.000944196791656228\\
44.6	0.000945215304407658\\
44.61	0.000946234372489509\\
44.62	0.00094725399638447\\
44.63	0.000948274176575926\\
44.64	0.000949294913547956\\
44.65	0.000950316207785309\\
44.66	0.000951338059773439\\
44.67	0.000952360469998491\\
44.68	0.00095338343894729\\
44.69	0.000954406967107364\\
44.7	0.00095543105496694\\
44.71	0.000956455703014933\\
44.72	0.000957480911740957\\
44.73	0.000958506681635329\\
44.74	0.000959533013189073\\
44.75	0.0009605599068939\\
44.76	0.000961587363242235\\
44.77	0.000962615382727204\\
44.78	0.000963643965842642\\
44.79	0.000964673113083091\\
44.8	0.0009657028249438\\
44.81	0.000966733101920733\\
44.82	0.000967763944510563\\
44.83	0.000968795353210677\\
44.84	0.000969827328519177\\
44.85	0.000970859870934877\\
44.86	0.000971892980957317\\
44.87	0.000972926659086756\\
44.88	0.000973960905824162\\
44.89	0.000974995721671236\\
44.9	0.000976031107130397\\
44.91	0.000977067062704791\\
44.92	0.000978103588898292\\
44.93	0.000979140686215504\\
44.94	0.000980178355161752\\
44.95	0.000981216596243097\\
44.96	0.000982255409966326\\
44.97	0.000983294796838972\\
44.98	0.000984334757369296\\
44.99	0.000985375292066291\\
45	0.000986416401439698\\
45.01	0.000987458085999988\\
45.02	0.000988500346258379\\
45.03	0.000989543182726829\\
45.04	0.000990586595918046\\
45.05	0.000991630586345479\\
45.06	0.000992675154523316\\
45.07	0.000993720300966512\\
45.08	0.000994766026190752\\
45.09	0.000995812330712484\\
45.1	0.000996859215048912\\
45.11	0.00099790667971799\\
45.12	0.000998954725238421\\
45.13	0.00100000335212968\\
45.14	0.00100105256091199\\
45.15	0.00100210235210633\\
45.16	0.00100315272623448\\
45.17	0.00100420368381892\\
45.18	0.00100525522538294\\
45.19	0.0010063073514506\\
45.2	0.00100736006254669\\
45.21	0.00100841335919681\\
45.22	0.00100946724192732\\
45.23	0.00101052171126534\\
45.24	0.00101157676773878\\
45.25	0.00101263241187631\\
45.26	0.00101368864420739\\
45.27	0.00101474546526228\\
45.28	0.00101580287557197\\
45.29	0.00101686087566828\\
45.3	0.00101791946608378\\
45.31	0.00101897864735186\\
45.32	0.00102003842000666\\
45.33	0.00102109878458314\\
45.34	0.00102215974161703\\
45.35	0.00102322129164487\\
45.36	0.00102428343520398\\
45.37	0.00102534617283249\\
45.38	0.0010264095050693\\
45.39	0.00102747343245414\\
45.4	0.00102853795552752\\
45.41	0.00102960307483076\\
45.42	0.00103066879090599\\
45.43	0.00103173510429612\\
45.44	0.00103280201554489\\
45.45	0.00103386952519685\\
45.46	0.00103493763379735\\
45.47	0.00103600634189256\\
45.48	0.00103707565002945\\
45.49	0.00103814555875582\\
45.5	0.00103921606862029\\
45.51	0.00104028718017226\\
45.52	0.00104135889396201\\
45.53	0.0010424312105406\\
45.54	0.00104350413045992\\
45.55	0.0010445776542727\\
45.56	0.00104565178253249\\
45.57	0.00104672651579366\\
45.58	0.00104780185461142\\
45.59	0.00104887779954182\\
45.6	0.00104995435114171\\
45.61	0.00105103150996882\\
45.62	0.00105210927658168\\
45.63	0.00105318765153969\\
45.64	0.00105426663540308\\
45.65	0.0010553462287329\\
45.66	0.00105642643209108\\
45.67	0.00105750724604037\\
45.68	0.00105858867114438\\
45.69	0.00105967070796757\\
45.7	0.00106075335707524\\
45.71	0.00106183661903356\\
45.72	0.00106292049440954\\
45.73	0.00106400498377105\\
45.74	0.00106509008768683\\
45.75	0.00106617580672648\\
45.76	0.00106726214146043\\
45.77	0.00106834909246002\\
45.78	0.00106943666029741\\
45.79	0.00107052484554566\\
45.8	0.0010716136487787\\
45.81	0.00107270307057131\\
45.82	0.00107379311149915\\
45.83	0.00107488377213877\\
45.84	0.00107597505306757\\
45.85	0.00107706695486385\\
45.86	0.00107815947810678\\
45.87	0.00107925262337642\\
45.88	0.0010803463912537\\
45.89	0.00108144078232045\\
45.9	0.00108253579715938\\
45.91	0.00108363143635409\\
45.92	0.00108472770048907\\
45.93	0.00108582459014971\\
45.94	0.0010869221059223\\
45.95	0.00108802024839399\\
45.96	0.00108911901815288\\
45.97	0.00109021841578794\\
45.98	0.00109131844188904\\
45.99	0.00109241909704697\\
46	0.00109352038185342\\
46.01	0.00109462229690098\\
46.02	0.00109572484278317\\
46.03	0.0010968280200944\\
46.04	0.00109793182943001\\
46.05	0.00109903627138624\\
46.06	0.00110014134656027\\
46.07	0.00110124705555018\\
46.08	0.00110235339895497\\
46.09	0.00110346037737459\\
46.1	0.00110456799140989\\
46.11	0.00110567624166264\\
46.12	0.00110678512873558\\
46.13	0.00110789465323233\\
46.14	0.0011090048157575\\
46.15	0.00111011561691658\\
46.16	0.00111122705731603\\
46.17	0.00111233913756325\\
46.18	0.00111345185826656\\
46.19	0.00111456522003524\\
46.2	0.00111567922347951\\
46.21	0.00111679386921055\\
46.22	0.00111790915784046\\
46.23	0.00111902508998232\\
46.24	0.00112014166625016\\
46.25	0.00112125888725894\\
46.26	0.0011223767536246\\
46.27	0.00112349526596404\\
46.28	0.00112461442489512\\
46.29	0.00112573423103666\\
46.3	0.00112685468500844\\
46.31	0.00112797578743121\\
46.32	0.00112909753892669\\
46.33	0.00113021994011758\\
46.34	0.00113134299162756\\
46.35	0.00113246669408126\\
46.36	0.0011335910481043\\
46.37	0.0011347160543233\\
46.38	0.00113584171336582\\
46.39	0.00113696802586045\\
46.4	0.00113809499243673\\
46.41	0.00113922261372521\\
46.42	0.00114035089035743\\
46.43	0.00114147982296591\\
46.44	0.00114260941218417\\
46.45	0.00114373965864674\\
46.46	0.00114487056298913\\
46.47	0.00114600212584785\\
46.48	0.00114713434786044\\
46.49	0.00114826722966541\\
46.5	0.00114940077190231\\
46.51	0.00115053497521168\\
46.52	0.00115166984023508\\
46.53	0.00115280536761508\\
46.54	0.00115394155799528\\
46.55	0.00115507841202027\\
46.56	0.00115621593033569\\
46.57	0.00115735411358818\\
46.58	0.00115849296242542\\
46.59	0.00115963247749612\\
46.6	0.00116077265945001\\
46.61	0.00116191350893784\\
46.62	0.00116305502661144\\
46.63	0.00116419721312362\\
46.64	0.00116534006912827\\
46.65	0.00116648359528029\\
46.66	0.00116762779223564\\
46.67	0.00116877266065134\\
46.68	0.00116991820118542\\
46.69	0.00117106441449699\\
46.7	0.0011722113012462\\
46.71	0.00117335886209426\\
46.72	0.00117450709770343\\
46.73	0.00117565600873704\\
46.74	0.00117680559585947\\
46.75	0.00117795585973616\\
46.76	0.00117910680103363\\
46.77	0.00118025842041946\\
46.78	0.00118141071856231\\
46.79	0.00118256369613189\\
46.8	0.00118371735379901\\
46.81	0.00118487169223555\\
46.82	0.00118602671211446\\
46.83	0.00118718241410979\\
46.84	0.00118833879889665\\
46.85	0.00118949586715127\\
46.86	0.00119065361955092\\
46.87	0.00119181205677402\\
46.88	0.00119297117950005\\
46.89	0.0011941309884096\\
46.9	0.00119529148418433\\
46.91	0.00119645266750704\\
46.92	0.00119761453906162\\
46.93	0.00119877709953305\\
46.94	0.00119994034960744\\
46.95	0.001201104289972\\
46.96	0.00120226892131505\\
46.97	0.00120343424432604\\
46.98	0.00120460025969553\\
46.99	0.0012057669681152\\
47	0.00120693437027784\\
47.01	0.0012081024668774\\
47.02	0.00120927125860892\\
47.03	0.00121044074616861\\
47.04	0.00121161093025376\\
47.05	0.00121278181156286\\
47.06	0.00121395339079548\\
47.07	0.00121512566865237\\
47.08	0.00121629864583541\\
47.09	0.00121747232304761\\
47.1	0.00121864670099316\\
47.11	0.00121982178037736\\
47.12	0.00122099756190671\\
47.13	0.00122217404628883\\
47.14	0.00122335123423252\\
47.15	0.00122452912644772\\
47.16	0.00122570772364555\\
47.17	0.00122688702653828\\
47.18	0.00122806703583938\\
47.19	0.00122924775226346\\
47.2	0.00123042917652631\\
47.21	0.00123161130934491\\
47.22	0.00123279415143741\\
47.23	0.00123397770352313\\
47.24	0.0012351619663226\\
47.25	0.00123634694055752\\
47.26	0.00123753262695078\\
47.27	0.00123871902622645\\
47.28	0.00123990613910983\\
47.29	0.00124109396632739\\
47.3	0.00124228250860681\\
47.31	0.00124347176667695\\
47.32	0.00124466174126792\\
47.33	0.001245852433111\\
47.34	0.0012470438429387\\
47.35	0.00124823597148473\\
47.36	0.00124942881948404\\
47.37	0.00125062238767276\\
47.38	0.00125181667678828\\
47.39	0.00125301168756919\\
47.4	0.00125420742075533\\
47.41	0.00125540387708774\\
47.42	0.00125660105730872\\
47.43	0.00125779896216179\\
47.44	0.00125899759239171\\
47.45	0.00126019694874448\\
47.46	0.00126139703196736\\
47.47	0.00126259784280883\\
47.48	0.00126379938201864\\
47.49	0.00126500165034778\\
47.5	0.00126620464854851\\
47.51	0.00126740837737433\\
47.52	0.00126861283758001\\
47.53	0.00126981802992157\\
47.54	0.00127102395515633\\
47.55	0.00127223061404283\\
47.56	0.00127343800734093\\
47.57	0.00127464613581173\\
47.58	0.00127585500021764\\
47.59	0.00127706460132231\\
47.6	0.00127827493989071\\
47.61	0.00127948601668907\\
47.62	0.00128069783248493\\
47.63	0.00128191038804711\\
47.64	0.00128312368414573\\
47.65	0.0012843377215522\\
47.66	0.00128555250103925\\
47.67	0.00128676802338088\\
47.68	0.00128798428935244\\
47.69	0.00128920129973055\\
47.7	0.00129041905529317\\
47.71	0.00129163755681956\\
47.72	0.00129285680509032\\
47.73	0.00129407680088733\\
47.74	0.00129529754499384\\
47.75	0.00129651903819441\\
47.76	0.00129774128127492\\
47.77	0.00129896427502259\\
47.78	0.001300188020226\\
47.79	0.00130141251767502\\
47.8	0.00130263776816092\\
47.81	0.00130386377247627\\
47.82	0.00130509053141502\\
47.83	0.00130631804577246\\
47.84	0.00130754631634522\\
47.85	0.00130877534393131\\
47.86	0.00131000512933011\\
47.87	0.00131123567334232\\
47.88	0.00131246697677005\\
47.89	0.00131369904041678\\
47.9	0.00131493186508733\\
47.91	0.00131616545158792\\
47.92	0.00131739980072616\\
47.93	0.00131863491331102\\
47.94	0.00131987079015286\\
47.95	0.00132110743206346\\
47.96	0.00132234483985594\\
47.97	0.00132358301434488\\
47.98	0.00132482195634619\\
47.99	0.00132606166667723\\
48	0.00132730214615676\\
48.01	0.00132854339560494\\
48.02	0.00132978541584333\\
48.03	0.00133102820769493\\
48.04	0.00133227177198414\\
48.05	0.0013335161095368\\
48.06	0.00133476122118015\\
48.07	0.00133600710774287\\
48.08	0.00133725377005509\\
48.09	0.00133850120894834\\
48.1	0.00133974942525562\\
48.11	0.00134099841981136\\
48.12	0.00134224819345143\\
48.13	0.00134349874701314\\
48.14	0.00134475008133527\\
48.15	0.00134600219725805\\
48.16	0.00134725509562316\\
48.17	0.00134850877727376\\
48.18	0.00134976324305445\\
48.19	0.00135101849381131\\
48.2	0.0013522745303919\\
48.21	0.00135353135364526\\
48.22	0.00135478896442187\\
48.23	0.00135604736357373\\
48.24	0.00135730655195432\\
48.25	0.00135856653041859\\
48.26	0.001359827299823\\
48.27	0.00136108886102551\\
48.28	0.00136235121488556\\
48.29	0.00136361436226411\\
48.3	0.0013648783040236\\
48.31	0.00136614304102803\\
48.32	0.00136740857414286\\
48.33	0.0013686749042351\\
48.34	0.00136994203217326\\
48.35	0.0013712099588274\\
48.36	0.00137247868506907\\
48.37	0.00137374821177138\\
48.38	0.00137501853980897\\
48.39	0.00137628967005801\\
48.4	0.00137756160339622\\
48.41	0.00137883434070286\\
48.42	0.00138010788285874\\
48.43	0.00138138223074623\\
48.44	0.00138265738524924\\
48.45	0.00138393334725324\\
48.46	0.00138521011764529\\
48.47	0.001386487697314\\
48.48	0.00138776608714954\\
48.49	0.00138904528804366\\
48.5	0.00139032530088971\\
48.51	0.0013916061265826\\
48.52	0.00139288776601883\\
48.53	0.00139417022009648\\
48.54	0.00139545348971525\\
48.55	0.00139673757577641\\
48.56	0.00139802247918284\\
48.57	0.00139930820083902\\
48.58	0.00140059474165106\\
48.59	0.00140188210252665\\
48.6	0.00140317028437511\\
48.61	0.00140445928810738\\
48.62	0.00140574911463603\\
48.63	0.00140703976487525\\
48.64	0.00140833123974086\\
48.65	0.00140962354015031\\
48.66	0.0014109166670227\\
48.67	0.00141221062127876\\
48.68	0.00141350540384089\\
48.69	0.00141480101563311\\
48.7	0.0014160974575811\\
48.71	0.00141739473061223\\
48.72	0.00141869283565549\\
48.73	0.00141999177364156\\
48.74	0.00142129154550278\\
48.75	0.00142259215217318\\
48.76	0.00142389359458844\\
48.77	0.00142519587368595\\
48.78	0.00142649899040477\\
48.79	0.00142780294568565\\
48.8	0.00142910774047104\\
48.81	0.00143041337570508\\
48.82	0.00143171985233362\\
48.83	0.00143302717130422\\
48.84	0.00143433533356613\\
48.85	0.00143564434007033\\
48.86	0.00143695419176953\\
48.87	0.00143826488961812\\
48.88	0.00143957643457227\\
48.89	0.00144088882758984\\
48.9	0.00144220206963044\\
48.91	0.00144351616165542\\
48.92	0.00144483110462786\\
48.93	0.00144614689951262\\
48.94	0.00144746354727627\\
48.95	0.00144878104888715\\
48.96	0.00145009940531538\\
48.97	0.00145141861753282\\
48.98	0.0014527386865131\\
48.99	0.00145405961323163\\
49	0.0014553813986656\\
49.01	0.00145670404379396\\
49.02	0.00145802754959749\\
49.03	0.00145935191705869\\
49.04	0.00146067714716192\\
49.05	0.0014620032408933\\
49.06	0.00146333019924077\\
49.07	0.00146465802319406\\
49.08	0.00146598671374473\\
49.09	0.00146731627188613\\
49.1	0.00146864669861346\\
49.11	0.00146997799492372\\
49.12	0.00147131016181576\\
49.13	0.00147264320029024\\
49.14	0.00147397711134967\\
49.15	0.00147531189599839\\
49.16	0.00147664755524261\\
49.17	0.00147798409009037\\
49.18	0.00147932150155158\\
49.19	0.00148065979063797\\
49.2	0.0014819989583632\\
49.21	0.00148333900574274\\
49.22	0.00148467993379396\\
49.23	0.00148602174353611\\
49.24	0.00148736443599031\\
49.25	0.00148870801217958\\
49.26	0.00149005247312881\\
49.27	0.0014913978198648\\
49.28	0.00149274405341625\\
49.29	0.00149409117481378\\
49.3	0.00149543918508988\\
49.31	0.00149678808527899\\
49.32	0.00149813787641746\\
49.33	0.00149948855954355\\
49.34	0.00150084013569747\\
49.35	0.00150219260592134\\
49.36	0.00150354597125923\\
49.37	0.00150490023275716\\
49.38	0.00150625539146309\\
49.39	0.00150761144842692\\
49.4	0.00150896840470052\\
49.41	0.00151032626133772\\
49.42	0.00151168501939431\\
49.43	0.00151304467992806\\
49.44	0.00151440524399871\\
49.45	0.00151576671266798\\
49.46	0.00151712908699957\\
49.47	0.00151849236805919\\
49.48	0.00151985655691452\\
49.49	0.00152122165463526\\
49.5	0.00152258766229309\\
49.51	0.00152395458096174\\
49.52	0.00152532241171691\\
49.53	0.00152669115563636\\
49.54	0.00152806081379984\\
49.55	0.00152943138728916\\
49.56	0.00153080287718813\\
49.57	0.00153217528458263\\
49.58	0.00153354861056056\\
49.59	0.00153492285621191\\
49.6	0.00153629802262868\\
49.61	0.00153767411090494\\
49.62	0.00153905112213684\\
49.63	0.0015404290574226\\
49.64	0.00154180791786249\\
49.65	0.00154318770455888\\
49.66	0.00154456841861623\\
49.67	0.00154595006114107\\
49.68	0.00154733263324204\\
49.69	0.00154871613602989\\
49.7	0.00155010057061744\\
49.71	0.00155148593811966\\
49.72	0.00155287223965362\\
49.73	0.00155425947633851\\
49.74	0.00155564764929564\\
49.75	0.00155703675964847\\
49.76	0.00155842680852258\\
49.77	0.0015598177970457\\
49.78	0.00156120972634771\\
49.79	0.00156260259756063\\
49.8	0.00156399641181866\\
49.81	0.00156539117025815\\
49.82	0.00156678687401763\\
49.83	0.00156818352423778\\
49.84	0.00156958112206148\\
49.85	0.00157097966863379\\
49.86	0.00157237916510198\\
49.87	0.00157377961261548\\
49.88	0.00157518101232593\\
49.89	0.00157658336538721\\
49.9	0.00157798667295538\\
49.91	0.00157939093618872\\
49.92	0.00158079615624774\\
49.93	0.00158220233429519\\
49.94	0.00158360947149604\\
49.95	0.00158501756901751\\
49.96	0.00158642662802905\\
49.97	0.00158783664970238\\
49.98	0.00158924763521148\\
49.99	0.00159065958573255\\
50	0.00159207250244413\\
50.01	0.00159348638652697\\
50.02	0.00159490123916413\\
50.03	0.00159631706154095\\
50.04	0.00159773385484505\\
50.05	0.00159915162026638\\
50.06	0.00160057035899715\\
50.07	0.00160199007223192\\
50.08	0.00160341076116752\\
50.09	0.00160483242700314\\
50.1	0.00160625507094028\\
50.11	0.00160767869418275\\
50.12	0.00160910329793675\\
50.13	0.00161052888341076\\
50.14	0.00161195545181566\\
50.15	0.00161338300436466\\
50.16	0.00161481154227334\\
50.17	0.00161624106675964\\
50.18	0.00161767157904388\\
50.19	0.00161910308034875\\
50.2	0.00162053557189935\\
50.21	0.00162196905492314\\
50.22	0.00162340353064998\\
50.23	0.00162483900031216\\
50.24	0.00162627546514436\\
50.25	0.00162771292638366\\
50.26	0.0016291513852696\\
50.27	0.00163059084304411\\
50.28	0.00163203130095157\\
50.29	0.00163347276023881\\
50.3	0.00163491522215508\\
50.31	0.00163635868795211\\
50.32	0.00163780315888407\\
50.33	0.0016392486362076\\
50.34	0.00164069512118182\\
50.35	0.0016421426150683\\
50.36	0.00164359111913113\\
50.37	0.00164504063463686\\
50.38	0.00164649116285454\\
50.39	0.00164794270505574\\
50.4	0.00164939526251453\\
50.41	0.00165084883650748\\
50.42	0.00165230342831371\\
50.43	0.00165375903921484\\
50.44	0.00165521567049504\\
50.45	0.00165667332344101\\
50.46	0.00165813199934201\\
50.47	0.00165959169948986\\
50.48	0.00166105242517891\\
50.49	0.0016625141777061\\
50.5	0.00166397695837095\\
50.51	0.00166544076847553\\
50.52	0.00166690560932452\\
50.53	0.0016683714822252\\
50.54	0.00166983838848742\\
50.55	0.00167130632942365\\
50.56	0.001672775306349\\
50.57	0.00167424532058116\\
50.58	0.00167571637344046\\
50.59	0.00167718846624987\\
50.6	0.00167866160033499\\
50.61	0.00168013577702409\\
50.62	0.00168161099764805\\
50.63	0.00168308726354046\\
50.64	0.00168456457603754\\
50.65	0.00168604293647819\\
50.66	0.001687522346204\\
50.67	0.00168900280655925\\
50.68	0.00169048431889091\\
50.69	0.00169196688454865\\
50.7	0.00169345050488485\\
50.71	0.0016949351812546\\
50.72	0.00169642091501571\\
50.73	0.00169790770752874\\
50.74	0.00169939556015696\\
50.75	0.00170088447426641\\
50.76	0.00170237445122585\\
50.77	0.00170386549240683\\
50.78	0.00170535759918364\\
50.79	0.00170685077293335\\
50.8	0.00170834501503582\\
50.81	0.00170984032687367\\
50.82	0.00171133670983234\\
50.83	0.00171283416530005\\
50.84	0.00171433269466786\\
50.85	0.0017158322993296\\
50.86	0.00171733298068196\\
50.87	0.00171883474012444\\
50.88	0.00172033757905938\\
50.89	0.00172184149889197\\
50.9	0.00172334650103025\\
50.91	0.00172485258688511\\
50.92	0.00172635975787032\\
50.93	0.00172786801540252\\
50.94	0.00172937736090123\\
50.95	0.00173088779578886\\
50.96	0.00173239932149071\\
50.97	0.001733911939435\\
50.98	0.00173542565105284\\
50.99	0.00173694045777828\\
51	0.00173845636104828\\
51.01	0.00173997336230275\\
51.02	0.00174149146298452\\
51.03	0.00174301066453941\\
51.04	0.00174453096841614\\
51.05	0.00174605237606645\\
51.06	0.00174757488894502\\
51.07	0.00174909850850952\\
51.08	0.00175062323622061\\
51.09	0.00175214907354194\\
51.1	0.00175367602194017\\
51.11	0.00175520408288498\\
51.12	0.00175673325784905\\
51.13	0.00175826354830811\\
51.14	0.00175979495574091\\
51.15	0.00176132748162924\\
51.16	0.00176286112745795\\
51.17	0.00176439589471496\\
51.18	0.00176593178489124\\
51.19	0.00176746879948084\\
51.2	0.00176900693998091\\
51.21	0.00177054620789166\\
51.22	0.00177208660471643\\
51.23	0.00177362813196165\\
51.24	0.00177517079113687\\
51.25	0.00177671458375477\\
51.26	0.00177825951133117\\
51.27	0.00177980557538501\\
51.28	0.00178135277743838\\
51.29	0.00178290111901656\\
51.3	0.00178445060164797\\
51.31	0.00178600122686419\\
51.32	0.00178755299620002\\
51.33	0.00178910591119342\\
51.34	0.00179065997338556\\
51.35	0.00179221518432083\\
51.36	0.00179377154554682\\
51.37	0.00179532905861433\\
51.38	0.00179688772507745\\
51.39	0.00179844754649344\\
51.4	0.00180000852442287\\
51.41	0.00180157066042954\\
51.42	0.00180313395608051\\
51.43	0.00180469841294614\\
51.44	0.00180626403260007\\
51.45	0.00180783081661921\\
51.46	0.00180939876658381\\
51.47	0.0018109678840774\\
51.48	0.00181253817068684\\
51.49	0.00181410962800233\\
51.5	0.0018156822576174\\
51.51	0.00181725606112891\\
51.52	0.0018188310401371\\
51.53	0.00182040719624557\\
51.54	0.00182198453106128\\
51.55	0.0018235630461946\\
51.56	0.00182514274325925\\
51.57	0.00182672362387238\\
51.58	0.00182830568965454\\
51.59	0.00182988894222972\\
51.6	0.0018314733832253\\
51.61	0.00183305901427212\\
51.62	0.00183464583700446\\
51.63	0.00183623385306007\\
51.64	0.00183782306408013\\
51.65	0.00183941347170934\\
51.66	0.00184100507759584\\
51.67	0.00184259788339129\\
51.68	0.00184419189075084\\
51.69	0.00184578710133316\\
51.7	0.00184738351680042\\
51.71	0.00184898113881835\\
51.72	0.0018505799690562\\
51.73	0.00185218000918677\\
51.74	0.00185378126088642\\
51.75	0.00185538372583508\\
51.76	0.00185698740571626\\
51.77	0.00185859230221706\\
51.78	0.00186019841702816\\
51.79	0.00186180575184385\\
51.8	0.00186341430836206\\
51.81	0.00186502408828432\\
51.82	0.00186663509331582\\
51.83	0.00186824732516536\\
51.84	0.00186986078554542\\
51.85	0.00187147547617215\\
51.86	0.00187309139876537\\
51.87	0.00187470855504857\\
51.88	0.00187632694674894\\
51.89	0.00187794657559739\\
51.9	0.00187956744332855\\
51.91	0.00188118955168073\\
51.92	0.00188281290239603\\
51.93	0.00188443749722026\\
51.94	0.001886063337903\\
51.95	0.00188769042619758\\
51.96	0.00188931876386113\\
51.97	0.00189094835265454\\
51.98	0.00189257919434252\\
51.99	0.00189421129069357\\
52	0.00189584464348001\\
52.01	0.00189747925447799\\
52.02	0.0018991151254675\\
52.03	0.00190075225823237\\
52.04	0.00190239065456029\\
52.05	0.00190403031624282\\
52.06	0.0019056712450754\\
52.07	0.00190731344285736\\
52.08	0.00190895691139193\\
52.09	0.00191060165248625\\
52.1	0.00191224766795139\\
52.11	0.00191389495960233\\
52.12	0.00191554352925801\\
52.13	0.00191719337874133\\
52.14	0.00191884450987915\\
52.15	0.00192049692450229\\
52.16	0.00192215062444557\\
52.17	0.00192380561154781\\
52.18	0.00192546188765182\\
52.19	0.00192711945460446\\
52.2	0.00192877831425658\\
52.21	0.00193043846846311\\
52.22	0.001932099919083\\
52.23	0.0019337626679793\\
52.24	0.00193542671701908\\
52.25	0.00193709206807355\\
52.26	0.00193875872301798\\
52.27	0.00194042668373177\\
52.28	0.00194209595209843\\
52.29	0.00194376653000559\\
52.3	0.00194543841934505\\
52.31	0.00194711162201273\\
52.32	0.00194878613990874\\
52.33	0.00195046197493734\\
52.34	0.00195213912900701\\
52.35	0.0019538176040304\\
52.36	0.00195549740192438\\
52.37	0.00195717852461006\\
52.38	0.00195886097401275\\
52.39	0.00196054475206204\\
52.4	0.00196222986069174\\
52.41	0.00196391630183997\\
52.42	0.00196560407744908\\
52.43	0.00196729318946578\\
52.44	0.00196898363984101\\
52.45	0.00197067543053008\\
52.46	0.00197236856349261\\
52.47	0.00197406304069254\\
52.48	0.0019757588640982\\
52.49	0.00197745603568227\\
52.5	0.00197915455742177\\
52.51	0.00198085443129817\\
52.52	0.00198255565929729\\
52.53	0.0019842582434094\\
52.54	0.00198596218562916\\
52.55	0.00198766748795568\\
52.56	0.00198937415239253\\
52.57	0.00199108218094773\\
52.58	0.00199279157563378\\
52.59	0.00199450233846767\\
52.6	0.00199621447147087\\
52.61	0.00199792797666937\\
52.62	0.00199964285609371\\
52.63	0.00200135911177891\\
52.64	0.0020030767457646\\
52.65	0.00200479576009494\\
52.66	0.00200651615681865\\
52.67	0.00200823793798907\\
52.68	0.00200996110566413\\
52.69	0.00201168566190636\\
52.7	0.00201341160878291\\
52.71	0.00201513894836561\\
52.72	0.00201686768273089\\
52.73	0.00201859781395988\\
52.74	0.00202032934413837\\
52.75	0.00202206227535685\\
52.76	0.00202379660971049\\
52.77	0.00202553234929922\\
52.78	0.00202726949622765\\
52.79	0.00202900805260518\\
52.8	0.00203074802054594\\
52.81	0.00203248940216882\\
52.82	0.00203423219959752\\
52.83	0.00203597641496051\\
52.84	0.0020377220503911\\
52.85	0.00203946910802739\\
52.86	0.00204121759001234\\
52.87	0.00204296749849375\\
52.88	0.00204471883562429\\
52.89	0.00204647160356151\\
52.9	0.00204822580446785\\
52.91	0.00204998144051063\\
52.92	0.00205173851386213\\
52.93	0.00205349702669954\\
52.94	0.00205525698120499\\
52.95	0.00205701837956559\\
52.96	0.00205878122397342\\
52.97	0.00206054551662553\\
52.98	0.00206231125972399\\
52.99	0.00206407845547588\\
53	0.00206584710609332\\
53.01	0.00206761721379348\\
53.02	0.00206938878079855\\
53.03	0.00207116180933584\\
53.04	0.00207293630163772\\
53.05	0.00207471225994168\\
53.06	0.00207648968649031\\
53.07	0.00207826858353134\\
53.08	0.00208004895331765\\
53.09	0.00208183079810728\\
53.1	0.00208361412016343\\
53.11	0.0020853989217545\\
53.12	0.0020871852051541\\
53.13	0.00208897297264105\\
53.14	0.00209076222649939\\
53.15	0.00209255296901845\\
53.16	0.00209434520249278\\
53.17	0.00209613892922223\\
53.18	0.00209793415151195\\
53.19	0.00209973087167238\\
53.2	0.00210152909201929\\
53.21	0.0021033288148738\\
53.22	0.00210513004256237\\
53.23	0.00210693277741682\\
53.24	0.00210873702177439\\
53.25	0.00211054277797769\\
53.26	0.00211235004837475\\
53.27	0.00211415883531903\\
53.28	0.00211596914116944\\
53.29	0.00211778096829035\\
53.3	0.00211959431905161\\
53.31	0.00212140919582857\\
53.32	0.00212322560100205\\
53.33	0.00212504353695846\\
53.34	0.00212686300608968\\
53.35	0.0021286840107932\\
53.36	0.00213050655347205\\
53.37	0.00213233063653486\\
53.38	0.00213415626239586\\
53.39	0.0021359834334749\\
53.4	0.00213781215219747\\
53.41	0.00213964242099472\\
53.42	0.00214147424230343\\
53.43	0.00214330761856611\\
53.44	0.00214514255223094\\
53.45	0.00214697904575184\\
53.46	0.00214881710158844\\
53.47	0.00215065672220614\\
53.48	0.0021524979100761\\
53.49	0.00215434066767525\\
53.5	0.00215618499748634\\
53.51	0.00215803090199793\\
53.52	0.00215987838370439\\
53.53	0.00216172744510598\\
53.54	0.0021635780887088\\
53.55	0.00216543031702483\\
53.56	0.00216728413257197\\
53.57	0.00216913953787403\\
53.58	0.00217099653546075\\
53.59	0.00217285512786781\\
53.6	0.00217471531763689\\
53.61	0.00217657710731563\\
53.62	0.00217844049945768\\
53.63	0.00218030549662272\\
53.64	0.00218217210137646\\
53.65	0.00218404031629066\\
53.66	0.00218591014394316\\
53.67	0.0021877815869179\\
53.68	0.0021896546478049\\
53.69	0.00219152932920034\\
53.7	0.00219340563370652\\
53.71	0.00219528356393191\\
53.72	0.00219716312249117\\
53.73	0.00219904431200513\\
53.74	0.00220092713510086\\
53.75	0.00220281159441165\\
53.76	0.00220469769257706\\
53.77	0.00220658543224288\\
53.78	0.00220847481606124\\
53.79	0.00221036584669054\\
53.8	0.00221225852679552\\
53.81	0.00221415285904724\\
53.82	0.00221604884612316\\
53.83	0.00221794649070708\\
53.84	0.00221984579548922\\
53.85	0.00222174676316623\\
53.86	0.00222364939644116\\
53.87	0.00222555369802355\\
53.88	0.00222745967062938\\
53.89	0.00222936731698115\\
53.9	0.00223127663980786\\
53.91	0.00223318764184505\\
53.92	0.00223510032583479\\
53.93	0.00223701469452572\\
53.94	0.0022389307506731\\
53.95	0.00224084849703875\\
53.96	0.00224276793639116\\
53.97	0.00224468907150543\\
53.98	0.00224661190516335\\
53.99	0.00224853644015338\\
54	0.00225046267927069\\
54.01	0.00225239062531717\\
54.02	0.00225432028110147\\
54.03	0.00225625164943898\\
54.04	0.00225818473315188\\
54.05	0.00226011953506917\\
54.06	0.00226205605802664\\
54.07	0.00226399430486696\\
54.08	0.00226593427843964\\
54.09	0.00226787598160109\\
54.1	0.0022698194172146\\
54.11	0.00227176458815039\\
54.12	0.00227371149728565\\
54.13	0.00227566014750451\\
54.14	0.0022776105416981\\
54.15	0.00227956268276454\\
54.16	0.00228151657360898\\
54.17	0.00228347221714364\\
54.18	0.00228542961628777\\
54.19	0.00228738877396774\\
54.2	0.00228934969311702\\
54.21	0.00229131237667621\\
54.22	0.00229327682759306\\
54.23	0.0022952430488225\\
54.24	0.00229721104332666\\
54.25	0.00229918081407485\\
54.26	0.00230115236404366\\
54.27	0.00230312569621692\\
54.28	0.00230510081358574\\
54.29	0.00230707771914854\\
54.3	0.00230905641591105\\
54.31	0.00231103690688635\\
54.32	0.0023130191950949\\
54.33	0.00231500328356452\\
54.34	0.00231698917533047\\
54.35	0.00231897687343542\\
54.36	0.00232096638092951\\
54.37	0.00232295770087035\\
54.38	0.00232495083632304\\
54.39	0.00232694579036023\\
54.4	0.00232894256606209\\
54.41	0.00233094116651636\\
54.42	0.00233294159481837\\
54.43	0.00233494385407105\\
54.44	0.00233694794738499\\
54.45	0.00233895387787841\\
54.46	0.00234096164867723\\
54.47	0.00234297126291506\\
54.48	0.00234498272373325\\
54.49	0.00234699603428087\\
54.5	0.00234901119771479\\
54.51	0.00235102821719967\\
54.52	0.00235304709590797\\
54.53	0.00235506783702\\
54.54	0.00235709044372397\\
54.55	0.00235911491921591\\
54.56	0.00236114126669983\\
54.57	0.00236316948938763\\
54.58	0.00236519959049918\\
54.59	0.00236723157326236\\
54.6	0.00236926544091302\\
54.61	0.00237130119669506\\
54.62	0.00237333884386043\\
54.63	0.00237537838566917\\
54.64	0.00237741982538943\\
54.65	0.00237946316629745\\
54.66	0.00238150841167767\\
54.67	0.00238355556482267\\
54.68	0.00238560462903326\\
54.69	0.00238765560761844\\
54.7	0.00238970850389552\\
54.71	0.00239176332119003\\
54.72	0.00239382006283582\\
54.73	0.00239587873217509\\
54.74	0.00239793933255835\\
54.75	0.00240000186734451\\
54.76	0.00240206633990089\\
54.77	0.0024041327536032\\
54.78	0.00240620111183566\\
54.79	0.0024082714179909\\
54.8	0.00241034367547012\\
54.81	0.002412417887683\\
54.82	0.00241449405804779\\
54.83	0.00241657218999134\\
54.84	0.00241865228694908\\
54.85	0.0024207343523651\\
54.86	0.00242281838969211\\
54.87	0.00242490440239154\\
54.88	0.00242699239393352\\
54.89	0.00242908236779692\\
54.9	0.00243117432746937\\
54.91	0.00243326827644728\\
54.92	0.00243536421823592\\
54.93	0.00243746215634936\\
54.94	0.00243956209431055\\
54.95	0.00244166403565136\\
54.96	0.00244376798391257\\
54.97	0.00244587394264391\\
54.98	0.00244798191540409\\
54.99	0.00245009190576084\\
55	0.0024522039172909\\
55.01	0.00245431795358011\\
55.02	0.00245643401822335\\
55.03	0.00245855211482467\\
55.04	0.00246067224699723\\
55.05	0.00246279441836336\\
55.06	0.00246491863255462\\
55.07	0.00246704489321178\\
55.08	0.00246917320398487\\
55.09	0.0024713035685332\\
55.1	0.00247343599052541\\
55.11	0.00247557047363946\\
55.12	0.0024777070215627\\
55.13	0.00247984563799187\\
55.14	0.00248198632663314\\
55.15	0.00248412909120215\\
55.16	0.00248627393542401\\
55.17	0.00248842086303334\\
55.18	0.00249056987777432\\
55.19	0.00249272098340069\\
55.2	0.00249487418367582\\
55.21	0.00249702948237267\\
55.22	0.00249918688327391\\
55.23	0.00250134639017187\\
55.24	0.0025035080068686\\
55.25	0.00250567173717591\\
55.26	0.00250783758491541\\
55.27	0.00251000555391848\\
55.28	0.00251217564802639\\
55.29	0.00251434787109023\\
55.3	0.00251652222697105\\
55.31	0.00251869871953977\\
55.32	0.00252087735267734\\
55.33	0.00252305813027464\\
55.34	0.00252524105623262\\
55.35	0.00252742613446227\\
55.36	0.00252961336888466\\
55.37	0.002531802763431\\
55.38	0.00253399432204262\\
55.39	0.00253618804867106\\
55.4	0.00253838394727804\\
55.41	0.00254058202183555\\
55.42	0.00254278227632586\\
55.43	0.00254498471474152\\
55.44	0.00254718934108543\\
55.45	0.00254939615937087\\
55.46	0.00255160517362153\\
55.47	0.0025538163878715\\
55.48	0.00255602980616537\\
55.49	0.00255824543255823\\
55.5	0.00256046327111568\\
55.51	0.0025626833259139\\
55.52	0.00256490560103968\\
55.53	0.00256713010059042\\
55.54	0.00256935682867419\\
55.55	0.00257158578940977\\
55.56	0.00257381698692665\\
55.57	0.0025760504253651\\
55.58	0.00257828610887618\\
55.59	0.00258052404162179\\
55.6	0.00258276422777468\\
55.61	0.00258500667151852\\
55.62	0.00258725137704788\\
55.63	0.00258949834856833\\
55.64	0.00259174759029642\\
55.65	0.00259399842110423\\
55.66	0.00259625055670001\\
55.67	0.00259850399834153\\
55.68	0.00260075874728894\\
55.69	0.00260301480480472\\
55.7	0.00260527217215373\\
55.71	0.00260753085060318\\
55.72	0.00260979084142265\\
55.73	0.00261205214588411\\
55.74	0.00261431476526192\\
55.75	0.00261657870083281\\
55.76	0.00261884395387595\\
55.77	0.00262111052567288\\
55.78	0.00262337841750759\\
55.79	0.00262564763066649\\
55.8	0.00262791816643838\\
55.81	0.00263019002611457\\
55.82	0.00263246321098876\\
55.83	0.00263473772235712\\
55.84	0.00263701356151829\\
55.85	0.00263929072977338\\
55.86	0.00264156922842598\\
55.87	0.00264384905878214\\
55.88	0.00264613022215044\\
55.89	0.00264841271984193\\
55.9	0.00265069655317019\\
55.91	0.0026529817234513\\
55.92	0.00265526823200387\\
55.93	0.00265755608014905\\
55.94	0.00265984526921052\\
55.95	0.0026621358005145\\
55.96	0.0026644276753898\\
55.97	0.00266672089516774\\
55.98	0.00266901546118227\\
55.99	0.00267131137476988\\
56	0.00267360863726965\\
56.01	0.00267590725002329\\
56.02	0.00267820721437508\\
56.03	0.00268050853167191\\
56.04	0.00268281120326332\\
56.05	0.00268511523050146\\
56.06	0.00268742061474111\\
56.07	0.00268972735733971\\
56.08	0.00269203545965735\\
56.09	0.00269434492305678\\
56.1	0.00269665574890342\\
56.11	0.00269896793856538\\
56.12	0.00270128149341344\\
56.13	0.00270359641482108\\
56.14	0.0027059127041645\\
56.15	0.0027082303628226\\
56.16	0.00271054939217699\\
56.17	0.00271286979361204\\
56.18	0.00271519156851483\\
56.19	0.00271751471827521\\
56.2	0.00271983924428578\\
56.21	0.00272216514794189\\
56.22	0.00272449243064169\\
56.23	0.00272682109378609\\
56.24	0.00272915113877881\\
56.25	0.00273148256702635\\
56.26	0.00273381537993804\\
56.27	0.00273614957892602\\
56.28	0.00273848516540525\\
56.29	0.00274082214079354\\
56.3	0.00274316050651154\\
56.31	0.00274550026398276\\
56.32	0.00274784141463356\\
56.33	0.00275018395989318\\
56.34	0.00275252790119376\\
56.35	0.00275487323997029\\
56.36	0.00275721997766071\\
56.37	0.00275956811570583\\
56.38	0.00276191765554942\\
56.39	0.00276426859863813\\
56.4	0.00276662094642158\\
56.41	0.00276897470035235\\
56.42	0.00277132986188595\\
56.43	0.00277368643248085\\
56.44	0.00277604441359854\\
56.45	0.00277840380670345\\
56.46	0.00278076461326302\\
56.47	0.00278312683474772\\
56.48	0.00278549047263101\\
56.49	0.00278785552838937\\
56.5	0.00279022200350234\\
56.51	0.00279258989945248\\
56.52	0.00279495921772543\\
56.53	0.00279732995980986\\
56.54	0.00279970212719755\\
56.55	0.00280207572138334\\
56.56	0.00280445074386517\\
56.57	0.0028068271961441\\
56.58	0.00280920507972428\\
56.59	0.002811584396113\\
56.6	0.00281396514682069\\
56.61	0.00281634733336091\\
56.62	0.00281873095725039\\
56.63	0.00282111602000901\\
56.64	0.00282350252315985\\
56.65	0.00282589046822914\\
56.66	0.00282827985674634\\
56.67	0.00283067069024412\\
56.68	0.00283306297025834\\
56.69	0.0028354566983281\\
56.7	0.00283785187599575\\
56.71	0.00284024850480688\\
56.72	0.00284264658631034\\
56.73	0.00284504612205827\\
56.74	0.00284744711360606\\
56.75	0.00284984956251241\\
56.76	0.00285225347033932\\
56.77	0.00285465883865212\\
56.78	0.00285706566901945\\
56.79	0.00285947396301328\\
56.8	0.00286188372220894\\
56.81	0.00286429494818512\\
56.82	0.00286670764252388\\
56.83	0.00286912180681064\\
56.84	0.00287153744263424\\
56.85	0.00287395455158691\\
56.86	0.00287637313526429\\
56.87	0.00287879319526545\\
56.88	0.00288121473319292\\
56.89	0.00288363775065263\\
56.9	0.00288606224925401\\
56.91	0.00288848823060996\\
56.92	0.00289091569633683\\
56.93	0.0028933446480545\\
56.94	0.00289577508738634\\
56.95	0.00289820701595924\\
56.96	0.00290064043540363\\
56.97	0.00290307534735346\\
56.98	0.00290551175344625\\
56.99	0.00290794965532309\\
57	0.00291038905462863\\
57.01	0.00291282995301112\\
57.02	0.00291527235212241\\
57.03	0.00291771625361796\\
57.04	0.00292016165915686\\
57.05	0.00292260857040182\\
57.06	0.00292505698901924\\
57.07	0.00292750691668037\\
57.08	0.00292995835510231\\
57.09	0.00293241130600555\\
57.1	0.00293486577111402\\
57.11	0.0029373217521551\\
57.12	0.00293977925085961\\
57.13	0.00294223826896185\\
57.14	0.00294469880819957\\
57.15	0.00294716087031401\\
57.16	0.0029496244570499\\
57.17	0.00295208957015547\\
57.18	0.00295455621138246\\
57.19	0.00295702438248612\\
57.2	0.00295949408522525\\
57.21	0.00296196532136218\\
57.22	0.00296443809266277\\
57.23	0.00296691240089647\\
57.24	0.00296938824783628\\
57.25	0.00297186563525879\\
57.26	0.00297434456494416\\
57.27	0.00297682503867617\\
57.28	0.0029793070582422\\
57.29	0.00298179062543324\\
57.3	0.00298427574204394\\
57.31	0.00298676240987256\\
57.32	0.002989250630721\\
57.33	0.00299174040639486\\
57.34	0.00299423173870339\\
57.35	0.00299672462945951\\
57.36	0.00299921908047984\\
57.37	0.00300171509358471\\
57.38	0.00300421267059815\\
57.39	0.00300671181334791\\
57.4	0.00300921252366548\\
57.41	0.00301171480338609\\
57.42	0.00301421865434873\\
57.43	0.00301672407839616\\
57.44	0.00301923107737487\\
57.45	0.00302173965313521\\
57.46	0.00302424980753126\\
57.47	0.00302676154242094\\
57.48	0.00302927485966598\\
57.49	0.00303178976113194\\
57.5	0.00303430624868822\\
57.51	0.00303682432420806\\
57.52	0.00303934398956859\\
57.53	0.00304186524665077\\
57.54	0.00304438809733946\\
57.55	0.00304691254352344\\
57.56	0.00304943858709536\\
57.57	0.0030519662299518\\
57.58	0.00305449547399328\\
57.59	0.00305702632112423\\
57.6	0.00305955877325304\\
57.61	0.00306209283229208\\
57.62	0.00306462850015767\\
57.63	0.00306716577877011\\
57.64	0.00306970467005373\\
57.65	0.00307224517593681\\
57.66	0.00307478729835171\\
57.67	0.00307733103923476\\
57.68	0.00307987640052636\\
57.69	0.00308242338417097\\
57.7	0.0030849719921171\\
57.71	0.00308752222631732\\
57.72	0.00309007408872832\\
57.73	0.00309262758131087\\
57.74	0.00309518270602984\\
57.75	0.00309773946485424\\
57.76	0.0031002978597572\\
57.77	0.00310285789271601\\
57.78	0.00310541956571211\\
57.79	0.00310798288073109\\
57.8	0.00311054783976274\\
57.81	0.00311311444480105\\
57.82	0.0031156826978442\\
57.83	0.00311825260089459\\
57.84	0.00312082415595885\\
57.85	0.00312339736504785\\
57.86	0.00312597223017672\\
57.87	0.00312854875336484\\
57.88	0.00313112693663588\\
57.89	0.0031337067820178\\
57.9	0.00313628829154286\\
57.91	0.00313887146724762\\
57.92	0.003141456311173\\
57.93	0.00314404282536421\\
57.94	0.00314663101187087\\
57.95	0.00314922087274692\\
57.96	0.00315181241005069\\
57.97	0.00315440562584491\\
57.98	0.0031570005221967\\
57.99	0.00315959710117758\\
58	0.00316219536486353\\
58.01	0.00316479531533496\\
58.02	0.00316739695467672\\
58.03	0.00317000028497814\\
58.04	0.00317260530833301\\
58.05	0.00317521202683963\\
58.06	0.0031778204426008\\
58.07	0.00318043055772383\\
58.08	0.00318304237432057\\
58.09	0.00318565589450741\\
58.1	0.00318827112040529\\
58.11	0.00319088805413972\\
58.12	0.00319350669784082\\
58.13	0.00319612705364326\\
58.14	0.00319874912368636\\
58.15	0.00320137291011404\\
58.16	0.00320399841507487\\
58.17	0.00320662564072206\\
58.18	0.00320925458921348\\
58.19	0.00321188526271169\\
58.2	0.00321451766338393\\
58.21	0.00321715179340215\\
58.22	0.00321978765494301\\
58.23	0.00322242525018791\\
58.24	0.00322506458132298\\
58.25	0.00322770565053912\\
58.26	0.00323034846003201\\
58.27	0.0032329930120021\\
58.28	0.00323563930865464\\
58.29	0.00323828735219971\\
58.3	0.00324093714485221\\
58.31	0.00324358868883187\\
58.32	0.00324624198636329\\
58.33	0.00324889703967593\\
58.34	0.00325155385100415\\
58.35	0.0032542124225872\\
58.36	0.00325687275666924\\
58.37	0.00325953485549934\\
58.38	0.00326219872133154\\
58.39	0.00326486435642484\\
58.4	0.00326753176304317\\
58.41	0.00327020094345549\\
58.42	0.00327287189993573\\
58.43	0.00327554463476283\\
58.44	0.00327821915022079\\
58.45	0.00328089544859862\\
58.46	0.0032835735321904\\
58.47	0.00328625340329529\\
58.48	0.00328893506421752\\
58.49	0.00329161851726643\\
58.5	0.00329430376475649\\
58.51	0.00329699080900728\\
58.52	0.00329967965234353\\
58.53	0.00330237029709514\\
58.54	0.00330506274559719\\
58.55	0.00330775700018993\\
58.56	0.00331045306321885\\
58.57	0.00331315093703463\\
58.58	0.00331585062399321\\
58.59	0.00331855212645576\\
58.6	0.00332125544678874\\
58.61	0.00332396058736388\\
58.62	0.00332666755055822\\
58.63	0.0033293763387541\\
58.64	0.00333208695433919\\
58.65	0.00333479939970652\\
58.66	0.00333751367725448\\
58.67	0.00334022978938681\\
58.68	0.00334294773851266\\
58.69	0.0033456675270466\\
58.7	0.0033483891574086\\
58.71	0.0033511126320241\\
58.72	0.00335383795332395\\
58.73	0.00335656512374452\\
58.74	0.00335929414572763\\
58.75	0.00336202502172063\\
58.76	0.00336475775417637\\
58.77	0.00336749234555324\\
58.78	0.00337022879831519\\
58.79	0.00337296711493173\\
58.8	0.00337570729787796\\
58.81	0.00337844934963457\\
58.82	0.00338119327268788\\
58.83	0.00338393906952983\\
58.84	0.00338668674265803\\
58.85	0.00338943629457573\\
58.86	0.00339218772779188\\
58.87	0.00339494104482113\\
58.88	0.00339769624818384\\
58.89	0.0034004533404061\\
58.9	0.00340321232401975\\
58.91	0.00340597320156241\\
58.92	0.00340873597557747\\
58.93	0.00341150064861413\\
58.94	0.00341426722322738\\
58.95	0.00341703570197809\\
58.96	0.00341980608743293\\
58.97	0.00342257838216449\\
58.98	0.00342535258875121\\
58.99	0.00342812870977743\\
59	0.00343090674783344\\
59.01	0.00343368670551544\\
59.02	0.00343646858542559\\
59.03	0.00343925239017203\\
59.04	0.00344203812236889\\
59.05	0.00344482578463629\\
59.06	0.00344761537960039\\
59.07	0.00345040690989339\\
59.08	0.00345320037815354\\
59.09	0.00345599578702518\\
59.1	0.00345879313915875\\
59.11	0.00346159243721077\\
59.12	0.00346439368384394\\
59.13	0.00346719688172705\\
59.14	0.00347000203353511\\
59.15	0.00347280914194929\\
59.16	0.00347561820965697\\
59.17	0.00347842923935172\\
59.18	0.0034812422337334\\
59.19	0.00348405719550809\\
59.2	0.00348687412738816\\
59.21	0.00348969303209227\\
59.22	0.00349251391234539\\
59.23	0.00349533677087882\\
59.24	0.00349816161043021\\
59.25	0.00350098843374358\\
59.26	0.00350381724356933\\
59.27	0.00350664804266427\\
59.28	0.00350948083379163\\
59.29	0.00351231561972108\\
59.3	0.00351515240322874\\
59.31	0.00351799118709724\\
59.32	0.00352083197411567\\
59.33	0.00352367476707968\\
59.34	0.0035265195687914\\
59.35	0.00352936638205957\\
59.36	0.00353221520969946\\
59.37	0.00353506605453297\\
59.38	0.00353791891938858\\
59.39	0.00354077380710142\\
59.4	0.00354363072051326\\
59.41	0.00354648966247255\\
59.42	0.00354935063583442\\
59.43	0.00355221364346071\\
59.44	0.00355507868822\\
59.45	0.00355794577298759\\
59.46	0.00356081490064557\\
59.47	0.00356368607408281\\
59.48	0.00356655929619498\\
59.49	0.00356943456988459\\
59.5	0.00357231189806097\\
59.51	0.00357519128364033\\
59.52	0.00357807272954578\\
59.53	0.00358095623870731\\
59.54	0.00358384181406184\\
59.55	0.00358672945855323\\
59.56	0.00358961917513233\\
59.57	0.00359251096675696\\
59.58	0.00359540483639192\\
59.59	0.00359830078700907\\
59.6	0.0036011988215873\\
59.61	0.00360409894311255\\
59.62	0.00360700115457788\\
59.63	0.00360990545898343\\
59.64	0.00361281185933647\\
59.65	0.00361572035865142\\
59.66	0.00361863095994986\\
59.67	0.00362154366626057\\
59.68	0.00362445848061953\\
59.69	0.00362737540606995\\
59.7	0.00363029444566228\\
59.71	0.00363321560245426\\
59.72	0.00363613887951091\\
59.73	0.00363906427990454\\
59.74	0.00364199180671483\\
59.75	0.00364492146302879\\
59.76	0.00364785325194081\\
59.77	0.00365078717655267\\
59.78	0.00365372323997357\\
59.79	0.00365666144532015\\
59.8	0.00365960179571651\\
59.81	0.00366254429429421\\
59.82	0.00366548894419233\\
59.83	0.00366843574855747\\
59.84	0.00367138471054376\\
59.85	0.00367433583331291\\
59.86	0.00367728912003421\\
59.87	0.00368024457388453\\
59.88	0.00368320219804841\\
59.89	0.00368616199571803\\
59.9	0.00368912397009321\\
59.91	0.0036920881243815\\
59.92	0.00369505446179814\\
59.93	0.00369802298556612\\
59.94	0.00370099369891619\\
59.95	0.00370396660508685\\
59.96	0.00370694170732444\\
59.97	0.00370991900888309\\
59.98	0.00371289851302481\\
59.99	0.00371588022301943\\
60	0.00371886414214471\\
60.01	0.00372185027368629\\
60.02	0.00372483862093776\\
60.03	0.00372782918720066\\
60.04	0.0037308219757845\\
60.05	0.00373381699000679\\
60.06	0.00373681423319306\\
60.07	0.0037398137086769\\
60.08	0.00374281541979993\\
60.09	0.00374581936991188\\
60.1	0.00374882556237058\\
60.11	0.00375183400054199\\
60.12	0.00375484468780024\\
60.13	0.00375785762752761\\
60.14	0.00376087282311459\\
60.15	0.00376389027795989\\
60.16	0.00376690999547046\\
60.17	0.00376993197906152\\
60.18	0.00377295623215658\\
60.19	0.00377598275818744\\
60.2	0.00377901156059425\\
60.21	0.00378204264282551\\
60.22	0.00378507600833811\\
60.23	0.0037881116605973\\
60.24	0.0037911496030768\\
60.25	0.00379418983925876\\
60.26	0.00379723237263377\\
60.27	0.00380027720670094\\
60.28	0.00380332434496788\\
60.29	0.00380637379095076\\
60.3	0.00380942554817427\\
60.31	0.00381247962017171\\
60.32	0.00381553601048496\\
60.33	0.00381859472266455\\
60.34	0.00382165576026965\\
60.35	0.0038247191268681\\
60.36	0.00382778482603643\\
60.37	0.0038308528613599\\
60.38	0.00383392323643251\\
60.39	0.003836995954857\\
60.4	0.00384007102024492\\
60.41	0.00384314843621664\\
60.42	0.00384622820640132\\
60.43	0.00384931033443701\\
60.44	0.00385239482397062\\
60.45	0.00385548167865798\\
60.46	0.00385857090216382\\
60.47	0.00386166249816183\\
60.48	0.00386475647033468\\
60.49	0.00386785282237399\\
60.5	0.00387095155798045\\
60.51	0.00387405268086375\\
60.52	0.00387715619474265\\
60.53	0.00388026210334501\\
60.54	0.00388337041040777\\
60.55	0.00388648111967701\\
60.56	0.00388801538673845\\
60.57	0.00388938420347543\\
60.58	0.0038907536328378\\
60.59	0.00389212367495261\\
60.6	0.0038934943299461\\
60.61	0.00389486559794366\\
60.62	0.00389623747906983\\
60.63	0.00389760997344829\\
60.64	0.00389898308120188\\
60.65	0.00390035680245256\\
60.66	0.00390173113732143\\
60.67	0.00390310608592875\\
60.68	0.00390448164839386\\
60.69	0.00390585782483526\\
60.7	0.00390723461537055\\
60.71	0.00390861202011646\\
60.72	0.0039099900391888\\
60.73	0.00391136867270252\\
60.74	0.00391274792077168\\
60.75	0.00391412778350939\\
60.76	0.00391550826102789\\
60.77	0.00391688935343851\\
60.78	0.00391827106085165\\
60.79	0.00391965338337681\\
60.8	0.00392103632112255\\
60.81	0.00392241987419652\\
60.82	0.0039238040427054\\
60.83	0.00392518882675501\\
60.84	0.00392657422645015\\
60.85	0.00392796024189471\\
60.86	0.00392934687319165\\
60.87	0.00393073412044295\\
60.88	0.00393212198374964\\
60.89	0.00393351046321179\\
60.9	0.00393489955892851\\
60.91	0.00393628927099793\\
60.92	0.00393767959951723\\
60.93	0.00393907054458257\\
60.94	0.00394046210628916\\
60.95	0.00394185428473122\\
60.96	0.00394324708000196\\
60.97	0.00394464049219361\\
60.98	0.0039460345213974\\
60.99	0.00394742916770354\\
61	0.00394882443120124\\
61.01	0.00395022031197872\\
61.02	0.00395161681012313\\
61.03	0.00395301392572064\\
61.04	0.00395441165885637\\
61.05	0.00395581000961442\\
61.06	0.00395720897807787\\
61.07	0.00395860856432872\\
61.08	0.00396000876844797\\
61.09	0.00396140959051551\\
61.1	0.00396281103061024\\
61.11	0.00396421308880997\\
61.12	0.00396561576519145\\
61.13	0.00396701905983036\\
61.14	0.0039684229728013\\
61.15	0.00396982750417783\\
61.16	0.00397123265403238\\
61.17	0.00397263842243633\\
61.18	0.00397404480945995\\
61.19	0.00397545181517243\\
61.2	0.00397685943964184\\
61.21	0.00397826768293517\\
61.22	0.00397967654511828\\
61.23	0.00398108602625593\\
61.24	0.00398249612641176\\
61.25	0.00398390684564828\\
61.26	0.00398531818402687\\
61.27	0.0039867301416078\\
61.28	0.00398814271845017\\
61.29	0.00398955591461198\\
61.3	0.00399096973015005\\
61.31	0.00399238416512006\\
61.32	0.00399379921957653\\
61.33	0.00399521489357284\\
61.34	0.00399663118716118\\
61.35	0.00399804810039259\\
61.36	0.00399946563331691\\
61.37	0.00400088378598284\\
61.38	0.00400230255843785\\
61.39	0.00400372195072827\\
61.4	0.0040051419628992\\
61.41	0.00400656259499456\\
61.42	0.00400798384705706\\
61.43	0.00400940571912822\\
61.44	0.00401082821124832\\
61.45	0.00401225132345643\\
61.46	0.00401367505579043\\
61.47	0.00401509940828693\\
61.48	0.00401652438098134\\
61.49	0.00401794997390784\\
61.5	0.00401937618709932\\
61.51	0.00402080302058747\\
61.52	0.00402223047440273\\
61.53	0.00402365854857427\\
61.54	0.00402508724313\\
61.55	0.00402651655809658\\
61.56	0.00402794649349937\\
61.57	0.0040293770493625\\
61.58	0.00403080822570878\\
61.59	0.00403224002255978\\
61.6	0.00403367243993574\\
61.61	0.00403510547785563\\
61.62	0.00403653913633713\\
61.63	0.00403797341539658\\
61.64	0.00403940831504906\\
61.65	0.0040408438353083\\
61.66	0.00404227997618675\\
61.67	0.00404371673769549\\
61.68	0.00404515411984433\\
61.69	0.00404659212264171\\
61.7	0.00404803074609475\\
61.71	0.00404946999020923\\
61.72	0.00405090985498956\\
61.73	0.00405235034043883\\
61.74	0.00405379144655878\\
61.75	0.00405523317334975\\
61.76	0.00405667552081077\\
61.77	0.00405811848893944\\
61.78	0.00405956207773205\\
61.79	0.00406100628718347\\
61.8	0.0040624511172872\\
61.81	0.00406389656803534\\
61.82	0.00406534263941861\\
61.83	0.00406678933142634\\
61.84	0.00406823664404644\\
61.85	0.00406968457726543\\
61.86	0.00407113313106841\\
61.87	0.00407258230543906\\
61.88	0.00407403210035965\\
61.89	0.004075482515811\\
61.9	0.00407693355177254\\
61.91	0.0040783852082054\\
61.92	0.00407983748506865\\
61.93	0.00408129038231982\\
61.94	0.00408274389991491\\
61.95	0.00408419803780843\\
61.96	0.00408565279595333\\
61.97	0.00408710817430102\\
61.98	0.00408856417280138\\
61.99	0.00409002079140271\\
62	0.00409147803005176\\
62.01	0.00409293588869374\\
62.02	0.00409439436727224\\
62.03	0.0040958534657293\\
62.04	0.00409731318400538\\
62.05	0.00409877352203932\\
62.06	0.00410023447976838\\
62.07	0.00410169605712821\\
62.08	0.00410315825405284\\
62.09	0.00410462107047469\\
62.1	0.00410608450632455\\
62.11	0.00410754856153157\\
62.12	0.00410901323602328\\
62.13	0.00411047852972553\\
62.14	0.00411194444256253\\
62.15	0.00411341097445685\\
62.16	0.00411487812532937\\
62.17	0.00411634589509929\\
62.18	0.00411781428368414\\
62.19	0.00411928329099977\\
62.2	0.00412075291696031\\
62.21	0.00412222316147821\\
62.22	0.0041236940244642\\
62.23	0.00412516550582728\\
62.24	0.00412663760547475\\
62.25	0.00412811032331217\\
62.26	0.00412958365924335\\
62.27	0.00413105761317036\\
62.28	0.00413253218499354\\
62.29	0.00413400737461142\\
62.3	0.00413548318192081\\
62.31	0.00413695960681673\\
62.32	0.0041384366491924\\
62.33	0.00413991430893929\\
62.34	0.00414139258594701\\
62.35	0.00414287148010344\\
62.36	0.00414435099129459\\
62.37	0.00414583111940467\\
62.38	0.00414731186431607\\
62.39	0.00414879322590934\\
62.4	0.0041502752040632\\
62.41	0.00415175779865448\\
62.42	0.00415324100955821\\
62.43	0.0041547248366475\\
62.44	0.00415620927979363\\
62.45	0.00415769433886598\\
62.46	0.00415918001373204\\
62.47	0.00416066630425741\\
62.48	0.00416215321030578\\
62.49	0.00416364073173895\\
62.5	0.00416512886841677\\
62.51	0.00416661762019716\\
62.52	0.00416810698693616\\
62.53	0.00416959696848779\\
62.54	0.00417108756470418\\
62.55	0.00417257877543548\\
62.56	0.00417407060052986\\
62.57	0.00417556303983352\\
62.58	0.0041770560931907\\
62.59	0.00417854976044361\\
62.6	0.0041800440414325\\
62.61	0.00418153893599559\\
62.62	0.00418303444396909\\
62.63	0.00418453056518717\\
62.64	0.00418602729948198\\
62.65	0.00418752464668365\\
62.66	0.00418902260662023\\
62.67	0.00419052117911772\\
62.68	0.00419202036400006\\
62.69	0.00419352016108912\\
62.7	0.00419502057020468\\
62.71	0.00419652159116441\\
62.72	0.00419802322378393\\
62.73	0.00419952546787671\\
62.74	0.00420102832325411\\
62.75	0.00420253178972538\\
62.76	0.00420403586709762\\
62.77	0.0042055405551758\\
62.78	0.00420704585376272\\
62.79	0.00420855176265904\\
62.8	0.00421005828166324\\
62.81	0.00421156541057164\\
62.82	0.00421307314917833\\
62.83	0.00421458149727526\\
62.84	0.00421609045465214\\
62.85	0.00421760002109647\\
62.86	0.00421911019639355\\
62.87	0.00422062098032642\\
62.88	0.00422213237267589\\
62.89	0.00422364437322054\\
62.9	0.00422515698173667\\
62.91	0.00422667019799832\\
62.92	0.00422818402177724\\
62.93	0.00422969845284293\\
62.94	0.00423121349096255\\
62.95	0.00423272913590099\\
62.96	0.00423424538742082\\
62.97	0.00423576224528226\\
62.98	0.00423727970924324\\
62.99	0.00423879777905933\\
63	0.00424031645448374\\
63.01	0.00424183573526733\\
63.02	0.00424335562115859\\
63.03	0.00424487611190363\\
63.04	0.00424639720724616\\
63.05	0.00424791890692752\\
63.06	0.00424944121068662\\
63.07	0.00425096411825995\\
63.08	0.00425248762938159\\
63.09	0.00425401174378316\\
63.1	0.00425553646119387\\
63.11	0.00425706178134042\\
63.12	0.0042585877039471\\
63.13	0.0042601142287357\\
63.14	0.00426164135542551\\
63.15	0.00426316908373332\\
63.16	0.00426469741337347\\
63.17	0.00426622634405772\\
63.18	0.00426775587549533\\
63.19	0.00426928600739304\\
63.2	0.00427081673945502\\
63.21	0.00427234807138289\\
63.22	0.00427388000287571\\
63.23	0.00427541253362995\\
63.24	0.00427694566333952\\
63.25	0.00427847939169571\\
63.26	0.00428001371838721\\
63.27	0.0042815486431001\\
63.28	0.00428308416551781\\
63.29	0.00428462028532116\\
63.3	0.0042861570021883\\
63.31	0.00428769431579474\\
63.32	0.00428923222581331\\
63.33	0.00429077073191415\\
63.34	0.00429230983376472\\
63.35	0.00429384953102979\\
63.36	0.0042953898233714\\
63.37	0.00429693071044888\\
63.38	0.00429847219191883\\
63.39	0.00430001426743507\\
63.4	0.00430155693664872\\
63.41	0.0043031001992081\\
63.42	0.00430464405475876\\
63.43	0.00430618850294345\\
63.44	0.00430773354340216\\
63.45	0.00430927917577205\\
63.46	0.00431082539968743\\
63.47	0.00431237221477982\\
63.48	0.00431391962067789\\
63.49	0.00431546761700744\\
63.5	0.00431701620339143\\
63.51	0.00431856537944992\\
63.52	0.0043201151448001\\
63.53	0.00432166549905626\\
63.54	0.00432321644182976\\
63.55	0.00432476797272908\\
63.56	0.00432632009135972\\
63.57	0.00432787279732428\\
63.58	0.00432942609022238\\
63.59	0.00433097996965068\\
63.6	0.00433253443520287\\
63.61	0.00433408948646961\\
63.62	0.00433564512303861\\
63.63	0.00433720134449454\\
63.64	0.00433875815041904\\
63.65	0.00434031554039071\\
63.66	0.00434187351398511\\
63.67	0.00434343207077474\\
63.68	0.00434499121032901\\
63.69	0.00434655093221425\\
63.7	0.0043481112359937\\
63.71	0.00434967212122747\\
63.72	0.00435123358747255\\
63.73	0.00435279563428281\\
63.74	0.00435435826120896\\
63.75	0.00435592146779853\\
63.76	0.00435748525359592\\
63.77	0.00435904961814229\\
63.78	0.00436061456097565\\
63.79	0.00436218008163076\\
63.8	0.00436374617963918\\
63.81	0.00436531285452922\\
63.82	0.00436688010582596\\
63.83	0.00436844793305117\\
63.84	0.00437001633572339\\
63.85	0.00437158531335786\\
63.86	0.00437315486546652\\
63.87	0.00437472499155796\\
63.88	0.00437629569113749\\
63.89	0.00437786696370706\\
63.9	0.00437943880876526\\
63.91	0.00438101122580731\\
63.92	0.00438258421432506\\
63.93	0.00438415777380696\\
63.94	0.00438573190373805\\
63.95	0.00438730660359996\\
63.96	0.00438888187287086\\
63.97	0.00439045771102549\\
63.98	0.00439203411753512\\
63.99	0.00439361109186757\\
64	0.00439518863348715\\
64.01	0.00439676674185464\\
64.02	0.00439834541642734\\
64.03	0.00439992465665901\\
64.04	0.00440150446199987\\
64.05	0.00440308483189656\\
64.06	0.00440466576579216\\
64.07	0.00440624726312618\\
64.08	0.00440782932333449\\
64.09	0.00440941194584939\\
64.1	0.00441099513009951\\
64.11	0.00441257887550986\\
64.12	0.00441416318150178\\
64.13	0.00441574804749295\\
64.14	0.00441733347289735\\
64.15	0.00441891945712525\\
64.16	0.00442050599958324\\
64.17	0.00442209309967413\\
64.18	0.00442368075679703\\
64.19	0.00442526897034726\\
64.2	0.00442685773971636\\
64.21	0.00442844706429213\\
64.22	0.0044300369434585\\
64.23	0.00443162737659561\\
64.24	0.00443321836307976\\
64.25	0.00443480990228342\\
64.26	0.00443640199357517\\
64.27	0.00443799463631972\\
64.28	0.00443958782987788\\
64.29	0.00444118157360655\\
64.3	0.0044427758668587\\
64.31	0.00444437070898337\\
64.32	0.00444596609932563\\
64.33	0.00444756203722659\\
64.34	0.00444915852202335\\
64.35	0.00445075555304904\\
64.36	0.00445235312963273\\
64.37	0.00445395125109949\\
64.38	0.00445554991677032\\
64.39	0.00445714912596214\\
64.4	0.00445874887798783\\
64.41	0.00446034917215612\\
64.42	0.00446195000777166\\
64.43	0.00446355138413495\\
64.44	0.00446515330054235\\
64.45	0.00446675575628605\\
64.46	0.00446835875065407\\
64.47	0.00446996228293022\\
64.48	0.00447156635239412\\
64.49	0.00447317095832112\\
64.5	0.00447477609998237\\
64.51	0.00447638177664472\\
64.52	0.00447798798757076\\
64.53	0.00447959473201878\\
64.54	0.00448120200924276\\
64.55	0.00448280981849233\\
64.56	0.0044844181590128\\
64.57	0.0044860270300451\\
64.58	0.00448763643082579\\
64.59	0.00448924636058701\\
64.6	0.00449085681855652\\
64.61	0.0044924678039576\\
64.62	0.00449407931600913\\
64.63	0.00449569135392548\\
64.64	0.00449730391691656\\
64.65	0.00449891700418777\\
64.66	0.00450053061493998\\
64.67	0.00450214474836956\\
64.68	0.00450375940366828\\
64.69	0.00450537458002333\\
64.7	0.00450699027661738\\
64.71	0.0045086064926284\\
64.72	0.0045102232272298\\
64.73	0.00451184047959031\\
64.74	0.004513458248874\\
64.75	0.00451507653424028\\
64.76	0.00451669533484383\\
64.77	0.00451831464983463\\
64.78	0.00451993447835791\\
64.79	0.00452155481955417\\
64.8	0.00452317567255912\\
64.81	0.00452479703650367\\
64.82	0.00452641891051391\\
64.83	0.00452804129371113\\
64.84	0.00452966418521176\\
64.85	0.00453128758412735\\
64.86	0.00453291148956458\\
64.87	0.00453453590062521\\
64.88	0.00453616081640609\\
64.89	0.0045377862359991\\
64.9	0.00453941215849118\\
64.91	0.00454103858296429\\
64.92	0.00454266550849535\\
64.93	0.00454429293415631\\
64.94	0.00454592085901406\\
64.95	0.00454754928213041\\
64.96	0.00454917820256209\\
64.97	0.00455080761936077\\
64.98	0.00455243753157294\\
64.99	0.00455406793824001\\
65	0.00455569883839817\\
65.01	0.00455733023107847\\
65.02	0.00455896211530676\\
65.03	0.00456059449010366\\
65.04	0.00456222735448453\\
65.05	0.00456386070745948\\
65.06	0.00456549454803337\\
65.07	0.0045671288752057\\
65.08	0.00456876368797068\\
65.09	0.00457039898531717\\
65.1	0.00457203476622868\\
65.11	0.00457367102968331\\
65.12	0.00457530777465377\\
65.13	0.00457694500010732\\
65.14	0.00457858270500579\\
65.15	0.00458022088830554\\
65.16	0.00458185954895744\\
65.17	0.00458349868590682\\
65.18	0.00458513829809352\\
65.19	0.00458677838445178\\
65.2	0.0045884189439103\\
65.21	0.00459005997539215\\
65.22	0.0045917014778148\\
65.23	0.00459334345009006\\
65.24	0.00459498589112409\\
65.25	0.00459662879981736\\
65.26	0.00459827217506463\\
65.27	0.00459991601575491\\
65.28	0.0046015603207715\\
65.29	0.00460320508899188\\
65.3	0.00460485031928776\\
65.31	0.00460649601052501\\
65.32	0.00460814216156367\\
65.33	0.00460978877125789\\
65.34	0.00461143583845599\\
65.35	0.0046130833620003\\
65.36	0.00461473134072728\\
65.37	0.0046163797734674\\
65.38	0.00461802865904513\\
65.39	0.004619677996279\\
65.4	0.00462132778398144\\
65.41	0.00462297802095889\\
65.42	0.00462462870601166\\
65.43	0.00462627983793401\\
65.44	0.00462793141551405\\
65.45	0.00462958343753375\\
65.46	0.00463123590276892\\
65.47	0.00463288880998915\\
65.48	0.00463454215795787\\
65.49	0.00463619594543219\\
65.5	0.00463785017116302\\
65.51	0.00463950483389495\\
65.52	0.00464115993236626\\
65.53	0.00464281546530888\\
65.54	0.00464447143144842\\
65.55	0.00464612782950403\\
65.56	0.00464778465818853\\
65.57	0.00464944191620822\\
65.58	0.004651099602263\\
65.59	0.00465275771504624\\
65.6	0.00465441625324484\\
65.61	0.00465607521553913\\
65.62	0.00465773460060288\\
65.63	0.00465939440710327\\
65.64	0.00466105463370087\\
65.65	0.00466271527904962\\
65.66	0.00466437634179678\\
65.67	0.00466603782058291\\
65.68	0.00466769971404189\\
65.69	0.0046693620208008\\
65.7	0.00467102473948\\
65.71	0.00467268786869304\\
65.72	0.00467435140704662\\
65.73	0.00467601535314063\\
65.74	0.00467767970556807\\
65.75	0.00467934446291501\\
65.76	0.00468100962376063\\
65.77	0.00468267518667716\\
65.78	0.0046843411502298\\
65.79	0.00468600751297679\\
65.8	0.00468767427346929\\
65.81	0.00468934143025143\\
65.82	0.00469100898186024\\
65.83	0.00469267692682561\\
65.84	0.00469434526367032\\
65.85	0.00469601399090994\\
65.86	0.00469768310705287\\
65.87	0.00469935261060025\\
65.88	0.00470102250004598\\
65.89	0.00470269277387668\\
65.9	0.00470436343057164\\
65.91	0.00470603446860281\\
65.92	0.00470770588643477\\
65.93	0.00470937768252472\\
65.94	0.00471104985532242\\
65.95	0.00471272240327015\\
65.96	0.00471439532480272\\
65.97	0.00471606861834745\\
65.98	0.00471774228232409\\
65.99	0.00471941631514482\\
66	0.0047210907152142\\
66.01	0.00472276548092922\\
66.02	0.00472444061067914\\
66.03	0.00472611610284557\\
66.04	0.00472779195580239\\
66.05	0.00472946816791572\\
66.06	0.00473114473754391\\
66.07	0.00473282166303748\\
66.08	0.00473449894273916\\
66.09	0.00473617657498376\\
66.1	0.00473785455809819\\
66.11	0.00473953289040146\\
66.12	0.0047412115702046\\
66.13	0.00474289059581064\\
66.14	0.0047445699655146\\
66.15	0.00474624967760344\\
66.16	0.00474792973035602\\
66.17	0.00474961012204311\\
66.18	0.00475129085092729\\
66.19	0.00475297191526301\\
66.2	0.00475465331329648\\
66.21	0.00475633504326566\\
66.22	0.00475801710340024\\
66.23	0.00475969949192163\\
66.24	0.00476138220704286\\
66.25	0.00476306524696861\\
66.26	0.00476474860989515\\
66.27	0.00476643229401031\\
66.28	0.00476811629749347\\
66.29	0.00476980061851548\\
66.3	0.00477148525523868\\
66.31	0.00477317020581683\\
66.32	0.00477485546839508\\
66.33	0.00477654104110997\\
66.34	0.00477822692208935\\
66.35	0.00477991310945239\\
66.36	0.00478159960130951\\
66.37	0.00478328639576236\\
66.38	0.0047849734909038\\
66.39	0.00478666088481786\\
66.4	0.00478834857557969\\
66.41	0.00479003656125551\\
66.42	0.00479172483990266\\
66.43	0.00479341340956944\\
66.44	0.00479510226829521\\
66.45	0.00479679141411024\\
66.46	0.00479848084503574\\
66.47	0.00480017055908379\\
66.48	0.00480186055425735\\
66.49	0.00480355082855019\\
66.5	0.00480524137994685\\
66.51	0.00480693220642263\\
66.52	0.00480862330594353\\
66.53	0.00481031467646622\\
66.54	0.00481200631593803\\
66.55	0.00481369822229689\\
66.56	0.00481539039347126\\
66.57	0.00481708282738018\\
66.58	0.00481877552193315\\
66.59	0.00482046847503014\\
66.6	0.00482216168456154\\
66.61	0.00482385514840813\\
66.62	0.00482554886444102\\
66.63	0.00482724283052165\\
66.64	0.00482893704450171\\
66.65	0.00483063150422313\\
66.66	0.00483232620751805\\
66.67	0.00483402115220876\\
66.68	0.00483571633610766\\
66.69	0.00483741175701725\\
66.7	0.00483910741273008\\
66.71	0.00484080330102868\\
66.72	0.00484249941968557\\
66.73	0.00484419576646319\\
66.74	0.00484589233911387\\
66.75	0.0048475891353798\\
66.76	0.00484928615299298\\
66.77	0.00485098338967518\\
66.78	0.0048526808431379\\
66.79	0.00485437851108235\\
66.8	0.00485607639119937\\
66.81	0.00485777448116943\\
66.82	0.00485947277866258\\
66.83	0.00486117128133841\\
66.84	0.00486286998684597\\
66.85	0.00486456889282381\\
66.86	0.00486626799689985\\
66.87	0.00486796729669143\\
66.88	0.00486966678980517\\
66.89	0.00487136647383703\\
66.9	0.00487306634637219\\
66.91	0.00487476640498504\\
66.92	0.00487646664723915\\
66.93	0.00487816707068721\\
66.94	0.00487986767287099\\
66.95	0.0048815684513213\\
66.96	0.00488326940355795\\
66.97	0.00488497052708971\\
66.98	0.00488667181941429\\
66.99	0.00488837327801822\\
67	0.0048900749003769\\
67.01	0.00489177668395449\\
67.02	0.00489347862620392\\
67.03	0.00489518072456681\\
67.04	0.00489688297647341\\
67.05	0.00489858537934263\\
67.06	0.00490028793058192\\
67.07	0.00490199062758725\\
67.08	0.00490369346774311\\
67.09	0.00490539644842237\\
67.1	0.00490709956698634\\
67.11	0.00490880282078466\\
67.12	0.00491050620715527\\
67.13	0.00491220972342438\\
67.14	0.0049139133669064\\
67.15	0.00491561713490393\\
67.16	0.00491732102470765\\
67.17	0.00491902503359638\\
67.18	0.00492072915883692\\
67.19	0.00492243339768408\\
67.2	0.0049241377473806\\
67.21	0.00492584220515712\\
67.22	0.00492754676823213\\
67.23	0.0049292514338119\\
67.24	0.00493095619909047\\
67.25	0.0049326610612496\\
67.26	0.00493436601745869\\
67.27	0.00493607106487474\\
67.28	0.00493777620064235\\
67.29	0.00493948142189363\\
67.3	0.00494118672574811\\
67.31	0.00494289210931281\\
67.32	0.0049445975696821\\
67.33	0.00494630310393764\\
67.34	0.00494800870914842\\
67.35	0.00494971438237062\\
67.36	0.00495142012064761\\
67.37	0.0049531259210099\\
67.38	0.00495483178047505\\
67.39	0.0049565376960477\\
67.4	0.00495824366471943\\
67.41	0.00495994968346875\\
67.42	0.00496165574926109\\
67.43	0.00496336185904869\\
67.44	0.00496506800977055\\
67.45	0.00496677419835242\\
67.46	0.00496848042170675\\
67.47	0.00497018667673258\\
67.48	0.00497189296031555\\
67.49	0.00497359926932783\\
67.5	0.00497530560062805\\
67.51	0.00497701195106126\\
67.52	0.00497871831745891\\
67.53	0.00498042469663874\\
67.54	0.00498213108540478\\
67.55	0.00498383748054725\\
67.56	0.00498554387884256\\
67.57	0.00498725027705321\\
67.58	0.00498895667192774\\
67.59	0.00499066306020074\\
67.6	0.00499236943859271\\
67.61	0.00499407580381007\\
67.62	0.00499578215254506\\
67.63	0.00499748848147573\\
67.64	0.00499919478726585\\
67.65	0.00500090106656487\\
67.66	0.00500260731600788\\
67.67	0.00500431353221552\\
67.68	0.00500601971179396\\
67.69	0.00500772585133482\\
67.7	0.00500943194741513\\
67.71	0.00501113799659728\\
67.72	0.00501284399542894\\
67.73	0.00501454994044302\\
67.74	0.00501625582815761\\
67.75	0.00501796165507593\\
67.76	0.00501966741768625\\
67.77	0.00502137311246189\\
67.78	0.00502307873586109\\
67.79	0.005024784284327\\
67.8	0.00502648975428762\\
67.81	0.0050281951421557\\
67.82	0.00502990044432876\\
67.83	0.00503160565718895\\
67.84	0.00503331077710305\\
67.85	0.00503501580042238\\
67.86	0.00503672072348275\\
67.87	0.0050384255426044\\
67.88	0.00504013025409196\\
67.89	0.00504183485423436\\
67.9	0.00504353933930479\\
67.91	0.00504524370556062\\
67.92	0.00504694794924338\\
67.93	0.00504865206657866\\
67.94	0.00505035605377607\\
67.95	0.00505205990702917\\
67.96	0.00505376362251542\\
67.97	0.00505546719639609\\
67.98	0.00505717062481626\\
67.99	0.00505887390390469\\
68	0.00506057702977379\\
68.01	0.00506227999851957\\
68.02	0.00506398280622156\\
68.03	0.00506568544894274\\
68.04	0.0050673879227295\\
68.05	0.00506909022361157\\
68.06	0.00507079234760195\\
68.07	0.00507249429069684\\
68.08	0.00507419604887561\\
68.09	0.00507589761810069\\
68.1	0.00507759899431755\\
68.11	0.0050793001734546\\
68.12	0.00508100104626924\\
68.13	0.00508270154588124\\
68.14	0.00508440166853683\\
68.15	0.00508610141046893\\
68.16	0.00508780076789721\\
68.17	0.005089499737028\\
68.18	0.00509119831405421\\
68.19	0.00509289649515535\\
68.2	0.00509459427649746\\
68.21	0.00509629165423306\\
68.22	0.00509798862450112\\
68.23	0.005099685183427\\
68.24	0.00510138132712244\\
68.25	0.00510307705168545\\
68.26	0.00510477235320034\\
68.27	0.00510646722773765\\
68.28	0.00510816167135408\\
68.29	0.00510985568009249\\
68.3	0.00511154924998179\\
68.31	0.00511324237703699\\
68.32	0.00511493505725906\\
68.33	0.00511662728663496\\
68.34	0.00511831906113752\\
68.35	0.0051200103767255\\
68.36	0.00512170122934343\\
68.37	0.00512339161492164\\
68.38	0.00512508152937619\\
68.39	0.00512677096860882\\
68.4	0.00512845992850693\\
68.41	0.0051301484049435\\
68.42	0.00513183639377707\\
68.43	0.00513352389085167\\
68.44	0.0051352108919968\\
68.45	0.00513689739302739\\
68.46	0.00513858338974369\\
68.47	0.00514026887793131\\
68.48	0.00514195385336113\\
68.49	0.00514363831178923\\
68.5	0.00514532224895691\\
68.51	0.00514700566059057\\
68.52	0.00514868854240172\\
68.53	0.00515037089008689\\
68.54	0.00515205269932763\\
68.55	0.0051537339657904\\
68.56	0.00515541468512659\\
68.57	0.00515709485297244\\
68.58	0.00515877446494897\\
68.59	0.00516045351666198\\
68.6	0.00516213200370198\\
68.61	0.00516380992164411\\
68.62	0.00516548726604815\\
68.63	0.00516716403245846\\
68.64	0.00516884021640389\\
68.65	0.00517051581339776\\
68.66	0.00517219081893783\\
68.67	0.00517386522850621\\
68.68	0.00517553903756935\\
68.69	0.00517721224157798\\
68.7	0.00517888483596703\\
68.71	0.00518055681615565\\
68.72	0.00518222817754709\\
68.73	0.00518389891552867\\
68.74	0.00518556902547177\\
68.75	0.00518723850273175\\
68.76	0.00518890734264788\\
68.77	0.00519057554054334\\
68.78	0.00519224309172513\\
68.79	0.00519390999148403\\
68.8	0.00519557623509458\\
68.81	0.00519724181781497\\
68.82	0.00519890673488707\\
68.83	0.00520057098153631\\
68.84	0.00520223455297164\\
68.85	0.00520389744438553\\
68.86	0.00520555965095388\\
68.87	0.00520722116783596\\
68.88	0.00520888199017439\\
68.89	0.00521054211309506\\
68.9	0.00521220153170713\\
68.91	0.00521386024110291\\
68.92	0.00521551823635785\\
68.93	0.00521717551253052\\
68.94	0.00521883206411153\\
68.95	0.00522048788547248\\
68.96	0.00522214297096265\\
68.97	0.00522379731490887\\
68.98	0.00522545091161548\\
68.99	0.00522710375536423\\
69	0.0052287558404142\\
69.01	0.00523040716100169\\
69.02	0.00523205771134017\\
69.03	0.00523370748562013\\
69.04	0.00523535647800908\\
69.05	0.00523700468265137\\
69.06	0.00523865209366816\\
69.07	0.00524029870515731\\
69.08	0.0052419445111933\\
69.09	0.00524358950582711\\
69.1	0.00524523368308617\\
69.11	0.00524687703697424\\
69.12	0.00524851956147133\\
69.13	0.00525016125053364\\
69.14	0.00525180209809337\\
69.15	0.00525344209805875\\
69.16	0.00525508124431384\\
69.17	0.00525671953071854\\
69.18	0.00525835695110839\\
69.19	0.00525999349929457\\
69.2	0.00526162916906375\\
69.21	0.00526326395417798\\
69.22	0.00526489784837468\\
69.23	0.00526653084536644\\
69.24	0.00526816293884098\\
69.25	0.00526979412246106\\
69.26	0.00527142438986439\\
69.27	0.00527305373466346\\
69.28	0.00527468215044553\\
69.29	0.00527630963077249\\
69.3	0.00527793616918077\\
69.31	0.00527956175918123\\
69.32	0.00528118639425908\\
69.33	0.00528281006787378\\
69.34	0.00528443277345891\\
69.35	0.00528605450442211\\
69.36	0.00528767525414495\\
69.37	0.00528929501598284\\
69.38	0.00529091378326495\\
69.39	0.00529253154929404\\
69.4	0.00529414830734647\\
69.41	0.00529576405067196\\
69.42	0.00529737877249362\\
69.43	0.00529899246600774\\
69.44	0.00530060512438375\\
69.45	0.00530221674076412\\
69.46	0.00530382730826418\\
69.47	0.00530543681997211\\
69.48	0.00530704526894878\\
69.49	0.00530865264822764\\
69.5	0.00531025895081465\\
69.51	0.00531186416968815\\
69.52	0.00531346829779874\\
69.53	0.00531507132806919\\
69.54	0.00531667325339434\\
69.55	0.00531827406664098\\
69.56	0.00531987376064774\\
69.57	0.00532147232822496\\
69.58	0.00532306976215464\\
69.59	0.00532466605519026\\
69.6	0.00532626120005671\\
69.61	0.0053278551894502\\
69.62	0.00532944801603806\\
69.63	0.00533103967245872\\
69.64	0.00533263015132159\\
69.65	0.00533421944520686\\
69.66	0.00533580754666549\\
69.67	0.00533739444821903\\
69.68	0.00533898014235956\\
69.69	0.00534056462154951\\
69.7	0.00534214787822158\\
69.71	0.00534372990477865\\
69.72	0.00534531069359363\\
69.73	0.00534689023700932\\
69.74	0.00534846852733836\\
69.75	0.00535004555686304\\
69.76	0.00535162131783525\\
69.77	0.00535319580247631\\
69.78	0.00535476900297686\\
69.79	0.00535634091149679\\
69.8	0.00535791152016502\\
69.81	0.0053594808210795\\
69.82	0.00536104880630697\\
69.83	0.00536261546788295\\
69.84	0.00536418079781152\\
69.85	0.00536574478806527\\
69.86	0.00536730743058515\\
69.87	0.00536886871728031\\
69.88	0.00537042864002807\\
69.89	0.00537198719067368\\
69.9	0.0053735443610303\\
69.91	0.00537510014287879\\
69.92	0.00537665452796764\\
69.93	0.00537820750801282\\
69.94	0.00537975968419097\\
69.95	0.00538131217004729\\
69.96	0.00538286496456235\\
69.97	0.00538441806671286\\
69.98	0.00538597147547152\\
69.99	0.00538752518980714\\
70	0.00538907920868453\\
70.01	0.00539063353106456\\
70.02	0.0053921881559041\\
70.03	0.00539374308215602\\
70.04	0.00539529830876918\\
70.05	0.00539685383468842\\
70.06	0.00539840965885453\\
70.07	0.00539996578020428\\
70.08	0.00540152219767035\\
70.09	0.00540307891018136\\
70.1	0.00540463591666182\\
70.11	0.00540619321603217\\
70.12	0.00540775080720871\\
70.13	0.00540930868910364\\
70.14	0.005410866860625\\
70.15	0.00541242532067666\\
70.16	0.00541398406815836\\
70.17	0.00541554310196564\\
70.18	0.00541710242098985\\
70.19	0.00541866202411811\\
70.2	0.00542022191023337\\
70.21	0.00542178207821431\\
70.22	0.00542334252693535\\
70.23	0.00542490325526668\\
70.24	0.00542646426207422\\
70.25	0.00542802554621956\\
70.26	0.00542958710656003\\
70.27	0.00543114894194862\\
70.28	0.005432711051234\\
70.29	0.00543427343326049\\
70.3	0.00543583608686807\\
70.31	0.00543739901089232\\
70.32	0.00543896220416446\\
70.33	0.00544052566551131\\
70.34	0.00544208939375524\\
70.35	0.00544365338771425\\
70.36	0.00544521764620186\\
70.37	0.00544678216802714\\
70.38	0.0054483469519947\\
70.39	0.00544991199690464\\
70.4	0.00545147730155261\\
70.41	0.0054530428647297\\
70.42	0.00545460868522249\\
70.43	0.00545617476181303\\
70.44	0.00545774109327878\\
70.45	0.00545930767839267\\
70.46	0.005460874515923\\
70.47	0.00546244160463352\\
70.48	0.00546400894328333\\
70.49	0.00546557653062689\\
70.5	0.00546714436541405\\
70.51	0.00546871244638998\\
70.52	0.00547028077229517\\
70.53	0.00547184934186544\\
70.54	0.00547341815383188\\
70.55	0.00547498720692088\\
70.56	0.00547655649985408\\
70.57	0.00547812603134839\\
70.58	0.00547969580011593\\
70.59	0.00548126580486406\\
70.6	0.00548283604429532\\
70.61	0.00548440651710745\\
70.62	0.00548597722199339\\
70.63	0.0054875481576412\\
70.64	0.00548911932273408\\
70.65	0.00549069071595039\\
70.66	0.00549226233596357\\
70.67	0.00549383418144216\\
70.68	0.00549540625104979\\
70.69	0.00549697854344514\\
70.7	0.00549855105728195\\
70.71	0.00550012379120898\\
70.72	0.00550169674387002\\
70.73	0.00550326991390385\\
70.74	0.00550484329994421\\
70.75	0.00550641690061984\\
70.76	0.00550799071455443\\
70.77	0.00550956474036658\\
70.78	0.00551113897666983\\
70.79	0.00551271342207259\\
70.8	0.00551428807517822\\
70.81	0.00551586293458486\\
70.82	0.00551743799888559\\
70.83	0.00551901326666825\\
70.84	0.00552058873651556\\
70.85	0.00552216440700499\\
70.86	0.00552374027670883\\
70.87	0.00552531634419413\\
70.88	0.00552689260802268\\
70.89	0.00552846906675102\\
70.9	0.00553004571893041\\
70.91	0.0055316225631068\\
70.92	0.00553319959782081\\
70.93	0.00553477682160775\\
70.94	0.00553635423299757\\
70.95	0.00553793183051485\\
70.96	0.00553950961267878\\
70.97	0.00554108757800315\\
70.98	0.00554266572499635\\
70.99	0.00554424405216136\\
71	0.00554582255799575\\
71.01	0.00554740124099169\\
71.02	0.00554898009963589\\
71.03	0.00555055913240961\\
71.04	0.00555213833778862\\
71.05	0.00555371771424322\\
71.06	0.00555529726023818\\
71.07	0.00555687697423275\\
71.08	0.00555845685468062\\
71.09	0.00556003690002994\\
71.1	0.00556161710872325\\
71.11	0.00556319747919753\\
71.12	0.0055647780098841\\
71.13	0.00556635869920866\\
71.14	0.00556793954559128\\
71.15	0.00556952054744633\\
71.16	0.00557110170318248\\
71.17	0.00557268301120274\\
71.18	0.00557426446990434\\
71.19	0.0055758460776788\\
71.2	0.00557742783291187\\
71.21	0.00557900973398353\\
71.22	0.00558059177926792\\
71.23	0.0055821739671334\\
71.24	0.00558375629594249\\
71.25	0.00558533876405184\\
71.26	0.00558692136981222\\
71.27	0.00558850411156854\\
71.28	0.00559008698765977\\
71.29	0.00559166999641895\\
71.3	0.00559325313617318\\
71.31	0.00559483640524359\\
71.32	0.00559641980194532\\
71.33	0.0055980033245875\\
71.34	0.00559958697147322\\
71.35	0.00560117074089957\\
71.36	0.00560275463115752\\
71.37	0.00560433864053198\\
71.38	0.00560592276730177\\
71.39	0.00560750700973957\\
71.4	0.00560909136611189\\
71.41	0.00561067583467915\\
71.42	0.0056122604136955\\
71.43	0.00561384510140894\\
71.44	0.00561542989606125\\
71.45	0.00561701479588794\\
71.46	0.00561859979911827\\
71.47	0.00562018490397522\\
71.48	0.00562177010867546\\
71.49	0.00562335541142935\\
71.5	0.00562494081044088\\
71.51	0.0056265263039077\\
71.52	0.00562811189002108\\
71.53	0.00562969756696585\\
71.54	0.00563128333292044\\
71.55	0.00563286918605685\\
71.56	0.00563445512454056\\
71.57	0.00563604114653062\\
71.58	0.00563762725017953\\
71.59	0.00563921343363328\\
71.6	0.00564079969503131\\
71.61	0.00564238603250647\\
71.62	0.00564397244418503\\
71.63	0.00564555892818665\\
71.64	0.00564714548262434\\
71.65	0.00564873210560446\\
71.66	0.00565031879522669\\
71.67	0.00565190554958402\\
71.68	0.00565349236676271\\
71.69	0.00565507924484226\\
71.7	0.00565666618189542\\
71.71	0.00565825317598816\\
71.72	0.00565984022517964\\
71.73	0.00566142732752217\\
71.74	0.00566301448106121\\
71.75	0.00566460168383537\\
71.76	0.00566618893387633\\
71.77	0.00566777622920887\\
71.78	0.00566936356785082\\
71.79	0.00567095094781303\\
71.8	0.00567253836709942\\
71.81	0.00567412582370682\\
71.82	0.00567571331562508\\
71.83	0.00567730084083699\\
71.84	0.00567888839731824\\
71.85	0.00568047598303742\\
71.86	0.00568206359595604\\
71.87	0.00568365123402839\\
71.88	0.00568523889520165\\
71.89	0.00568682657741578\\
71.9	0.00568841427860354\\
71.91	0.0056900019966904\\
71.92	0.00569158972959464\\
71.93	0.00569317747522719\\
71.94	0.00569476523149171\\
71.95	0.00569635299628449\\
71.96	0.00569794076749449\\
71.97	0.00569952854300328\\
71.98	0.00570111632068501\\
71.99	0.0057027040984064\\
72	0.00570429187402673\\
72.01	0.00570587964539781\\
72.02	0.0057074674103639\\
72.03	0.00570905516676178\\
72.04	0.00571064291242066\\
72.05	0.00571223064516216\\
72.06	0.00571381836280031\\
72.07	0.00571540606314153\\
72.08	0.00571699374398455\\
72.09	0.00571858140312046\\
72.1	0.00572016903833263\\
72.11	0.00572175664739671\\
72.12	0.00572334422808059\\
72.13	0.00572493177814439\\
72.14	0.00572651929534043\\
72.15	0.00572810677741319\\
72.16	0.00572969422209932\\
72.17	0.00573128162712755\\
72.18	0.00573286899021876\\
72.19	0.00573445630908585\\
72.2	0.00573604358143379\\
72.21	0.00573763080495957\\
72.22	0.00573921797735214\\
72.23	0.00574080509629247\\
72.24	0.00574239215945341\\
72.25	0.00574397916449976\\
72.26	0.0057455661090882\\
72.27	0.00574715299086726\\
72.28	0.0057487398074773\\
72.29	0.00575032655655051\\
72.3	0.00575191323571083\\
72.31	0.00575349984257397\\
72.32	0.00575508637474737\\
72.33	0.00575667282983014\\
72.34	0.0057582592054131\\
72.35	0.00575984549907868\\
72.36	0.00576143170840094\\
72.37	0.00576301783094553\\
72.38	0.00576460386426967\\
72.39	0.00576618980592208\\
72.4	0.00576777565344302\\
72.41	0.00576936140436423\\
72.42	0.00577094705620886\\
72.43	0.00577253260649151\\
72.44	0.00577411805271819\\
72.45	0.00577570339238624\\
72.46	0.00577728862298435\\
72.47	0.00577887374199252\\
72.48	0.00578045874688202\\
72.49	0.0057820436351154\\
72.5	0.0057836284041464\\
72.51	0.00578521305141996\\
72.52	0.00578679757437218\\
72.53	0.00578838197043032\\
72.54	0.00578996623701271\\
72.55	0.00579155037152878\\
72.56	0.00579313437137901\\
72.57	0.00579471823395486\\
72.58	0.00579630195663883\\
72.59	0.00579788553680434\\
72.6	0.00579946897181575\\
72.61	0.00580105225902832\\
72.62	0.00580263539578816\\
72.63	0.00580421837943225\\
72.64	0.00580580120728834\\
72.65	0.00580738387667499\\
72.66	0.00580896638490148\\
72.67	0.00581054872926782\\
72.68	0.0058121309070647\\
72.69	0.00581371291557346\\
72.7	0.00581529475206608\\
72.71	0.00581687641380511\\
72.72	0.00581845789804366\\
72.73	0.00582003920202539\\
72.74	0.00582162032298444\\
72.75	0.00582320125814542\\
72.76	0.00582478200472338\\
72.77	0.00582636255992375\\
72.78	0.00582794292094237\\
72.79	0.00582952308496539\\
72.8	0.00583110304916927\\
72.81	0.00583268281072074\\
72.82	0.00583426236677679\\
72.83	0.0058358417144846\\
72.84	0.00583742085098154\\
72.85	0.00583899977339511\\
72.86	0.00584057847884294\\
72.87	0.00584215696443271\\
72.88	0.00584373522726218\\
72.89	0.00584531326441909\\
72.9	0.00584689107298117\\
72.91	0.0058484686500161\\
72.92	0.00585004599258147\\
72.93	0.00585162309772473\\
72.94	0.0058531999624832\\
72.95	0.005854776583884\\
72.96	0.00585635295894401\\
72.97	0.00585792908466986\\
72.98	0.0058595049580579\\
72.99	0.00586108057609414\\
73	0.00586265593575422\\
73.01	0.0058642310340034\\
73.02	0.00586580586779651\\
73.03	0.00586738043407787\\
73.04	0.00586895472978135\\
73.05	0.00587052875183027\\
73.06	0.00587210249713736\\
73.07	0.00587367596260474\\
73.08	0.00587524914512392\\
73.09	0.00587682204157568\\
73.1	0.00587839464883012\\
73.11	0.00587996696374659\\
73.12	0.00588153898317361\\
73.13	0.00588311070394893\\
73.14	0.0058846821228994\\
73.15	0.005886253236841\\
73.16	0.00588782404257873\\
73.17	0.00588939453690668\\
73.18	0.00589096471660787\\
73.19	0.00589253457845432\\
73.2	0.00589410411920694\\
73.21	0.00589567333561554\\
73.22	0.00589724222441875\\
73.23	0.00589881078234401\\
73.24	0.00590037900610752\\
73.25	0.00590194689241424\\
73.26	0.00590351443795778\\
73.27	0.00590508163942041\\
73.28	0.00590664849347304\\
73.29	0.0059082149967751\\
73.3	0.00590978114597461\\
73.31	0.00591134693770805\\
73.32	0.00591291236860038\\
73.33	0.00591447743526496\\
73.34	0.00591604213430352\\
73.35	0.00591760646230617\\
73.36	0.00591917041585126\\
73.37	0.00592073399150546\\
73.38	0.0059222971858236\\
73.39	0.00592385999534873\\
73.4	0.00592542241661202\\
73.41	0.00592698444613275\\
73.42	0.00592854608041823\\
73.43	0.00593010731596381\\
73.44	0.00593166814925282\\
73.45	0.0059332285767565\\
73.46	0.00593478859493399\\
73.47	0.00593634820023229\\
73.48	0.00593790738908621\\
73.49	0.00593946615791831\\
73.5	0.00594102450313889\\
73.51	0.00594258242114593\\
73.52	0.00594413990832505\\
73.53	0.00594569696104946\\
73.54	0.00594725357567993\\
73.55	0.00594880974856477\\
73.56	0.00595036547603969\\
73.57	0.00595192075442791\\
73.58	0.00595347558003997\\
73.59	0.00595502994917379\\
73.6	0.00595658385811455\\
73.61	0.0059581373031347\\
73.62	0.00595969028049391\\
73.63	0.00596124278643899\\
73.64	0.00596279481720388\\
73.65	0.00596434636900959\\
73.66	0.00596589743806417\\
73.67	0.00596744802056262\\
73.68	0.00596899811268694\\
73.69	0.00597054771060596\\
73.7	0.00597209681047541\\
73.71	0.00597364540843777\\
73.72	0.00597519350062232\\
73.73	0.00597674108314504\\
73.74	0.00597828815210856\\
73.75	0.00597983470360213\\
73.76	0.0059813807337016\\
73.77	0.00598292623846929\\
73.78	0.00598447121395404\\
73.79	0.00598601565619111\\
73.8	0.00598755956120213\\
73.81	0.00598910292499508\\
73.82	0.0059906457435642\\
73.83	0.00599218801289\\
73.84	0.00599372972893917\\
73.85	0.00599527088766452\\
73.86	0.00599681148500497\\
73.87	0.0059983515168855\\
73.88	0.00599989097921704\\
73.89	0.00600142986789651\\
73.9	0.00600296817880671\\
73.91	0.00600450590781627\\
73.92	0.00600604305077962\\
73.93	0.00600757960353697\\
73.94	0.00600911556191417\\
73.95	0.00601065092172277\\
73.96	0.00601218567875987\\
73.97	0.00601371982880812\\
73.98	0.00601525336763569\\
73.99	0.00601678629099616\\
74	0.0060183185946285\\
74.01	0.00601985027425704\\
74.02	0.00602138132559136\\
74.03	0.00602291174432629\\
74.04	0.00602444152614185\\
74.05	0.00602597066670316\\
74.06	0.00602749916166044\\
74.07	0.00602902700664889\\
74.08	0.00603055419728871\\
74.09	0.006032080729185\\
74.1	0.00603360659792773\\
74.11	0.00603513179909164\\
74.12	0.00603665632823627\\
74.13	0.0060381801809058\\
74.14	0.00603970335262908\\
74.15	0.00604122583891954\\
74.16	0.00604274763527513\\
74.17	0.00604426873717828\\
74.18	0.00604578914009583\\
74.19	0.00604730883947899\\
74.2	0.00604882783076325\\
74.21	0.00605034610936839\\
74.22	0.00605186367069833\\
74.23	0.00605338051014115\\
74.24	0.00605489662306899\\
74.25	0.00605641200483802\\
74.26	0.00605792665078837\\
74.27	0.00605944055624405\\
74.28	0.00606095371651292\\
74.29	0.00606246612688664\\
74.3	0.00606397778264056\\
74.31	0.00606548867903373\\
74.32	0.00606699881130876\\
74.33	0.00606850817469185\\
74.34	0.00607001676439264\\
74.35	0.00607152457560424\\
74.36	0.00607303160350306\\
74.37	0.00607453784324887\\
74.38	0.00607604328998464\\
74.39	0.00607754793883654\\
74.4	0.00607905178491384\\
74.41	0.00608055482330886\\
74.42	0.00608205704909692\\
74.43	0.00608355845733626\\
74.44	0.00608505904306799\\
74.45	0.006086558801316\\
74.46	0.00608805772708695\\
74.47	0.00608955581537013\\
74.48	0.00609105306113747\\
74.49	0.00609254945934342\\
74.5	0.00609404500492493\\
74.51	0.00609553969280134\\
74.52	0.00609703351787432\\
74.53	0.00609852647502787\\
74.54	0.00610001855912816\\
74.55	0.0061015097650235\\
74.56	0.00610300008754433\\
74.57	0.00610448952150304\\
74.58	0.00610597806169398\\
74.59	0.00610746570289341\\
74.6	0.00610895243985936\\
74.61	0.0061104382673316\\
74.62	0.00611192318003157\\
74.63	0.00611340717266231\\
74.64	0.0061148902399084\\
74.65	0.00611637237643586\\
74.66	0.00611785357689211\\
74.67	0.00611933383590586\\
74.68	0.00612081314808711\\
74.69	0.00612229150802699\\
74.7	0.00612376891029775\\
74.71	0.00612524534945266\\
74.72	0.00612672082002596\\
74.73	0.00612819531653275\\
74.74	0.00612966883346895\\
74.75	0.00613114136531121\\
74.76	0.00613261290651684\\
74.77	0.00613408345152375\\
74.78	0.00613555299475032\\
74.79	0.00613702153059539\\
74.8	0.00613848905343816\\
74.81	0.00613995555763811\\
74.82	0.00614142103753489\\
74.83	0.00614288548744833\\
74.84	0.00614434890167826\\
74.85	0.0061458112745045\\
74.86	0.00614727260018677\\
74.87	0.00614873287296458\\
74.88	0.00615019208705719\\
74.89	0.00615165023666351\\
74.9	0.00615310731596201\\
74.91	0.00615456331911067\\
74.92	0.00615601824024685\\
74.93	0.00615747207348727\\
74.94	0.00615892481292787\\
74.95	0.00616037645264378\\
74.96	0.00616182698668917\\
74.97	0.00616327640909724\\
74.98	0.00616472471388009\\
74.99	0.00616617189502865\\
75	0.00616761794651257\\
75.01	0.0061690628622802\\
75.02	0.00617050663625841\\
75.03	0.00617194926235259\\
75.04	0.0061733907344465\\
75.05	0.00617483104640223\\
75.06	0.00617627019206008\\
75.07	0.00617770816523848\\
75.08	0.00617914495973389\\
75.09	0.00618058056932076\\
75.1	0.00618201498775135\\
75.11	0.00618344820875625\\
75.12	0.00618488022604447\\
75.13	0.00618631103330343\\
75.14	0.00618774062419882\\
75.15	0.00618916899237458\\
75.16	0.00619059613145279\\
75.17	0.00619202203503361\\
75.18	0.0061934466966952\\
75.19	0.00619487010999361\\
75.2	0.00619629226846278\\
75.21	0.00619771316561437\\
75.22	0.00619913279493777\\
75.23	0.00620055114989995\\
75.24	0.00620196822394542\\
75.25	0.00620338401049615\\
75.26	0.00620479850295147\\
75.27	0.00620621169468799\\
75.28	0.00620762357905958\\
75.29	0.00620903414939721\\
75.3	0.0062104433990089\\
75.31	0.00621185132117967\\
75.32	0.00621325790917141\\
75.33	0.00621466315622284\\
75.34	0.00621606705554938\\
75.35	0.00621746960034314\\
75.36	0.00621887078377275\\
75.37	0.00622027059898336\\
75.38	0.00622166903909652\\
75.39	0.00622306609721007\\
75.4	0.00622446176639811\\
75.41	0.00622585603971089\\
75.42	0.00622724891017471\\
75.43	0.00622864037079186\\
75.44	0.00623003041454055\\
75.45	0.00623141903437478\\
75.46	0.00623280622322428\\
75.47	0.00623419197399442\\
75.48	0.00623557627956615\\
75.49	0.00623695913279588\\
75.5	0.00623834052651538\\
75.51	0.00623972045353174\\
75.52	0.00624109970330925\\
75.53	0.00624247883365666\\
75.54	0.00624385784370188\\
75.55	0.00624523673257493\\
75.56	0.00624661549940799\\
75.57	0.00624799414333543\\
75.58	0.00624937266349384\\
75.59	0.00625075105902205\\
75.6	0.00625212932906117\\
75.61	0.00625350747275461\\
75.62	0.00625488548924812\\
75.63	0.00625626337768981\\
75.64	0.0062576411372302\\
75.65	0.00625901876702221\\
75.66	0.00626039626622126\\
75.67	0.00626177363398521\\
75.68	0.00626315086947447\\
75.69	0.00626452797185197\\
75.7	0.00626590494028327\\
75.71	0.0062672817739365\\
75.72	0.00626865847198246\\
75.73	0.00627003503359461\\
75.74	0.00627141145794913\\
75.75	0.00627278774422494\\
75.76	0.00627416389160375\\
75.77	0.00627553989927004\\
75.78	0.00627691576641118\\
75.79	0.00627829149221736\\
75.8	0.00627966707588173\\
75.81	0.00628104251660036\\
75.82	0.00628241781357228\\
75.83	0.00628379296599956\\
75.84	0.00628516797308729\\
75.85	0.00628654283404366\\
75.86	0.00628791754807994\\
75.87	0.0062892921144106\\
75.88	0.00629066653225325\\
75.89	0.00629204080082874\\
75.9	0.00629341491936117\\
75.91	0.00629478888707793\\
75.92	0.00629616270320976\\
75.93	0.00629753636699072\\
75.94	0.00629890987765831\\
75.95	0.00630028323445346\\
75.96	0.00630165643662055\\
75.97	0.00630302948340751\\
75.98	0.00630440237406579\\
75.99	0.00630577510785045\\
76	0.00630714768402015\\
76.01	0.00630852010183723\\
76.02	0.00630989236056775\\
76.03	0.00631126445948148\\
76.04	0.00631263639785198\\
76.05	0.00631400817495664\\
76.06	0.00631537979007668\\
76.07	0.00631675124249728\\
76.08	0.00631812253150748\\
76.09	0.00631949365640036\\
76.1	0.00632086461647299\\
76.11	0.00632223541102651\\
76.12	0.00632360603936616\\
76.13	0.00632497650080132\\
76.14	0.00632634679464556\\
76.15	0.00632771692021667\\
76.16	0.00632908687683671\\
76.17	0.00633045666383205\\
76.18	0.00633182628053341\\
76.19	0.00633319572627591\\
76.2	0.0063345650003991\\
76.21	0.00633593410224704\\
76.22	0.00633730303116826\\
76.23	0.00633867178651592\\
76.24	0.00634004036764775\\
76.25	0.00634140877392617\\
76.26	0.00634277700471828\\
76.27	0.00634414505939593\\
76.28	0.00634551293733577\\
76.29	0.00634688063791929\\
76.3	0.00634824816053285\\
76.31	0.00634961550456776\\
76.32	0.00635098266942029\\
76.33	0.00635234965449174\\
76.34	0.00635371645918848\\
76.35	0.006355083082922\\
76.36	0.00635644952510894\\
76.37	0.00635781578517117\\
76.38	0.00635918186253581\\
76.39	0.0063605477566353\\
76.4	0.00636191346690743\\
76.41	0.0063632789927954\\
76.42	0.00636464433374785\\
76.43	0.00636600948921896\\
76.44	0.00636737445866841\\
76.45	0.00636873924156155\\
76.46	0.00637010383736934\\
76.47	0.00637146824556847\\
76.48	0.00637283246564137\\
76.49	0.00637419649707629\\
76.5	0.00637556033936734\\
76.51	0.00637692399201454\\
76.52	0.00637828745452387\\
76.53	0.00637965072640735\\
76.54	0.00638101380718305\\
76.55	0.00638237669637516\\
76.56	0.00638373939351407\\
76.57	0.00638510189813637\\
76.58	0.00638646420978498\\
76.59	0.00638782632800911\\
76.6	0.0063891882523644\\
76.61	0.00639054998241292\\
76.62	0.00639191151772328\\
76.63	0.0063932728578706\\
76.64	0.00639463400243667\\
76.65	0.00639599495100991\\
76.66	0.00639735570318551\\
76.67	0.00639871625856542\\
76.68	0.00640007661675845\\
76.69	0.00640143677738032\\
76.7	0.0064027967400537\\
76.71	0.00640415650440829\\
76.72	0.00640551607008087\\
76.73	0.00640687543671535\\
76.74	0.00640823460396285\\
76.75	0.00640959357148174\\
76.76	0.00641095233893774\\
76.77	0.0064123109060039\\
76.78	0.00641366927236075\\
76.79	0.00641502762775041\\
76.8	0.00641638601127061\\
76.81	0.00641774442232639\\
76.82	0.00641910286032433\\
76.83	0.00642046132467256\\
76.84	0.00642181981478078\\
76.85	0.00642317833006031\\
76.86	0.00642453686992405\\
76.87	0.00642589543378656\\
76.88	0.00642725402106405\\
76.89	0.0064286126311744\\
76.9	0.00642997126353721\\
76.91	0.00643132991757375\\
76.92	0.00643268859270709\\
76.93	0.006434047288362\\
76.94	0.00643540600396506\\
76.95	0.00643676473894466\\
76.96	0.00643812349273099\\
76.97	0.00643948226475609\\
76.98	0.00644084105445387\\
76.99	0.00644219986126012\\
77	0.00644355868461256\\
77.01	0.0064449175239508\\
77.02	0.00644627637871644\\
77.03	0.00644763524835304\\
77.04	0.00644899413230616\\
77.05	0.00645035303002336\\
77.06	0.00645171194095429\\
77.07	0.00645307086455062\\
77.08	0.00645442980026612\\
77.09	0.00645578874755669\\
77.1	0.00645714770588034\\
77.11	0.00645850667469725\\
77.12	0.00645986565346978\\
77.13	0.0064612246416625\\
77.14	0.0064625836387422\\
77.15	0.00646394264417794\\
77.16	0.00646530165744104\\
77.17	0.00646666067800512\\
77.18	0.00646801970534614\\
77.19	0.00646937873894242\\
77.2	0.00647073777827462\\
77.21	0.00647209682282584\\
77.22	0.00647345587208158\\
77.23	0.00647481492552979\\
77.24	0.00647617398266093\\
77.25	0.00647753304296793\\
77.26	0.00647889210594625\\
77.27	0.00648025117109392\\
77.28	0.00648161023791153\\
77.29	0.0064829693059023\\
77.3	0.00648432837457207\\
77.31	0.00648568744342934\\
77.32	0.00648704651198529\\
77.33	0.00648840557975384\\
77.34	0.00648976464625162\\
77.35	0.00649112371099804\\
77.36	0.0064924827735153\\
77.37	0.00649384183332844\\
77.38	0.00649520088996534\\
77.39	0.00649655994295675\\
77.4	0.00649791899183634\\
77.41	0.00649927803614072\\
77.42	0.00650063707540944\\
77.43	0.00650199610918507\\
77.44	0.00650335513701319\\
77.45	0.00650471415844244\\
77.46	0.00650607317302453\\
77.47	0.00650743218031427\\
77.48	0.00650879117986964\\
77.49	0.00651015017125176\\
77.5	0.00651150915402498\\
77.51	0.00651286812775684\\
77.52	0.0065142270920182\\
77.53	0.00651558604638313\\
77.54	0.0065169449904291\\
77.55	0.0065183039237369\\
77.56	0.00651966284589068\\
77.57	0.00652102175647806\\
77.58	0.00652238065509006\\
77.59	0.00652373954132121\\
77.6	0.00652509841476954\\
77.61	0.00652645727503662\\
77.62	0.00652781612172761\\
77.63	0.00652917495445127\\
77.64	0.00653053377281999\\
77.65	0.00653189257644987\\
77.66	0.00653325136496069\\
77.67	0.00653461013797598\\
77.68	0.00653596889512305\\
77.69	0.00653732763603301\\
77.7	0.00653868636034084\\
77.71	0.00654004506768536\\
77.72	0.00654140375770934\\
77.73	0.00654276243005949\\
77.74	0.00654412108438648\\
77.75	0.00654547972034505\\
77.76	0.00654683833759392\\
77.77	0.00654819693579599\\
77.78	0.00654955551461821\\
77.79	0.00655091407373174\\
77.8	0.00655227261281192\\
77.81	0.00655363113153833\\
77.82	0.00655498962959484\\
77.83	0.00655634810666961\\
77.84	0.00655770656245514\\
77.85	0.00655906499664836\\
77.86	0.00656042340895058\\
77.87	0.00656178179906759\\
77.88	0.00656314016670969\\
77.89	0.0065644985115917\\
77.9	0.00656585683343304\\
77.91	0.00656721513195774\\
77.92	0.00656857340689448\\
77.93	0.00656993165797665\\
77.94	0.00657128988494237\\
77.95	0.00657264808753455\\
77.96	0.0065740062655009\\
77.97	0.00657536441859401\\
77.98	0.00657672254657136\\
77.99	0.00657808064919537\\
78	0.00657943872623346\\
78.01	0.00658079677745805\\
78.02	0.00658215480264665\\
78.03	0.00658351280158188\\
78.04	0.00658487077405151\\
78.05	0.00658622871984849\\
78.06	0.00658758663877104\\
78.07	0.00658894453062264\\
78.08	0.00659030239521212\\
78.09	0.00659166023235365\\
78.1	0.00659301804186683\\
78.11	0.00659437582357668\\
78.12	0.00659573357731377\\
78.13	0.00659709130291414\\
78.14	0.00659844900021947\\
78.15	0.006599806669077\\
78.16	0.00660116430933969\\
78.17	0.00660252192086618\\
78.18	0.00660387950352086\\
78.19	0.00660523705717393\\
78.2	0.00660659458170143\\
78.21	0.00660795207698528\\
78.22	0.00660930954291332\\
78.23	0.00661066697937938\\
78.24	0.00661202438628332\\
78.25	0.00661338176353104\\
78.26	0.00661473911103458\\
78.27	0.0066160964287121\\
78.28	0.00661745371648801\\
78.29	0.00661881097429292\\
78.3	0.00662016820206379\\
78.31	0.00662152539974389\\
78.32	0.00662288256728289\\
78.33	0.00662423970463691\\
78.34	0.00662559681176854\\
78.35	0.00662695388864692\\
78.36	0.00662831093524777\\
78.37	0.00662966795155344\\
78.38	0.00663102493755296\\
78.39	0.0066323818932421\\
78.4	0.00663373881862341\\
78.41	0.00663509571370625\\
78.42	0.00663645257850688\\
78.43	0.0066378094130485\\
78.44	0.00663916621736127\\
78.45	0.00664052299148238\\
78.46	0.00664187973545613\\
78.47	0.00664323644933392\\
78.48	0.00664459313317436\\
78.49	0.00664594978704331\\
78.5	0.00664730641101387\\
78.51	0.00664866300516655\\
78.52	0.0066500195695892\\
78.53	0.00665137610437715\\
78.54	0.00665273260963322\\
78.55	0.0066540890854678\\
78.56	0.00665544553199888\\
78.57	0.00665680194935211\\
78.58	0.00665815833766088\\
78.59	0.00665951469706633\\
78.6	0.00666087102771746\\
78.61	0.00666222732977111\\
78.62	0.0066635836033921\\
78.63	0.00666493984875325\\
78.64	0.00666629606603539\\
78.65	0.00666765225542752\\
78.66	0.00666900841712676\\
78.67	0.00667036455133848\\
78.68	0.00667172065827633\\
78.69	0.00667307673816228\\
78.7	0.00667443279122674\\
78.71	0.00667578881770855\\
78.72	0.00667714481785505\\
78.73	0.0066785007919222\\
78.74	0.00667985674017457\\
78.75	0.00668121266288542\\
78.76	0.00668256856033679\\
78.77	0.00668392443281952\\
78.78	0.00668528028063332\\
78.79	0.00668663610408686\\
78.8	0.00668799190349779\\
78.81	0.00668934767919285\\
78.82	0.00669070343150787\\
78.83	0.0066920591607879\\
78.84	0.00669341486738721\\
78.85	0.00669477055166941\\
78.86	0.00669612621400748\\
78.87	0.00669748185478383\\
78.88	0.00669883747439039\\
78.89	0.00670019307322865\\
78.9	0.00670154865170975\\
78.91	0.00670290421025451\\
78.92	0.00670425974929354\\
78.93	0.00670561526926725\\
78.94	0.00670697077062599\\
78.95	0.00670832625383004\\
78.96	0.00670968171934975\\
78.97	0.00671103716766553\\
78.98	0.00671239259926799\\
78.99	0.00671374801465797\\
79	0.00671510341434659\\
79.01	0.00671645879885539\\
79.02	0.00671781416871632\\
79.03	0.00671916952447185\\
79.04	0.00672052486667504\\
79.05	0.00672188019588961\\
79.06	0.00672323551268999\\
79.07	0.0067245908176614\\
79.08	0.00672594611139998\\
79.09	0.00672730139451274\\
79.1	0.00672865666761774\\
79.11	0.00673001193134414\\
79.12	0.00673136718633222\\
79.13	0.00673272243323353\\
79.14	0.00673407767271089\\
79.15	0.00673543290543853\\
79.16	0.00673678813210212\\
79.17	0.00673814335339888\\
79.18	0.00673949857003761\\
79.19	0.00674085378273881\\
79.2	0.00674220899223473\\
79.21	0.00674356419926946\\
79.22	0.00674491940459901\\
79.23	0.00674627460899136\\
79.24	0.00674762981322659\\
79.25	0.00674898501809689\\
79.26	0.00675034022440671\\
79.27	0.00675169543297278\\
79.28	0.00675305064462424\\
79.29	0.00675440586020266\\
79.3	0.00675576108056219\\
79.31	0.0067571163065696\\
79.32	0.00675847153910435\\
79.33	0.0067598267790587\\
79.34	0.00676118202733779\\
79.35	0.00676253728485972\\
79.36	0.0067638925525556\\
79.37	0.00676524783136969\\
79.38	0.00676660312225946\\
79.39	0.00676795842619564\\
79.4	0.00676931374416236\\
79.41	0.00677066907715721\\
79.42	0.00677202442619132\\
79.43	0.00677337979228946\\
79.44	0.00677473517649011\\
79.45	0.00677609057984557\\
79.46	0.00677744600342203\\
79.47	0.00677880144829966\\
79.48	0.00678015691557271\\
79.49	0.00678151240634958\\
79.5	0.00678286792175294\\
79.51	0.00678422346291979\\
79.52	0.00678557903100155\\
79.53	0.00678693462716419\\
79.54	0.00678829025258826\\
79.55	0.00678964590846905\\
79.56	0.00679100159601664\\
79.57	0.00679235731645597\\
79.58	0.00679371307102702\\
79.59	0.00679506886098481\\
79.6	0.00679642468759953\\
79.61	0.00679778055215668\\
79.62	0.00679913645595708\\
79.63	0.00680049240031703\\
79.64	0.00680184838656839\\
79.65	0.00680320441605866\\
79.66	0.0068045604901511\\
79.67	0.00680591661022483\\
79.68	0.0068072727776749\\
79.69	0.00680862899391241\\
79.7	0.00680998526036461\\
79.71	0.006811341578475\\
79.72	0.00681269794970343\\
79.73	0.00681405437552617\\
79.74	0.00681541085743608\\
79.75	0.00681676739694264\\
79.76	0.00681812399557211\\
79.77	0.00681948065486759\\
79.78	0.00682083737638916\\
79.79	0.00682219416171392\\
79.8	0.00682355101243622\\
79.81	0.00682490793016762\\
79.82	0.00682626491653709\\
79.83	0.00682762197319108\\
79.84	0.00682897910179365\\
79.85	0.00683033630402655\\
79.86	0.00683169358158936\\
79.87	0.00683305093619955\\
79.88	0.00683440836959265\\
79.89	0.00683576588352233\\
79.9	0.00683712347976049\\
79.91	0.00683848116009741\\
79.92	0.00683983892634185\\
79.93	0.00684119678032113\\
79.94	0.0068425547238813\\
79.95	0.0068439127588872\\
79.96	0.0068452708872226\\
79.97	0.00684662911079032\\
79.98	0.00684798743151235\\
79.99	0.00684934585132992\\
80	0.00685070437220366\\
80.01	0.00685206299611372\\
};
\addplot [color=mycolor1,solid]
  table[row sep=crcr]{%
80.01	0.00685206299611372\\
80.02	0.00685342172505986\\
80.03	0.0068547805610616\\
80.04	0.00685613950615831\\
80.05	0.00685749856240932\\
80.06	0.00685885773189409\\
80.07	0.0068602170167123\\
80.08	0.00686157641898395\\
80.09	0.00686293594084953\\
80.1	0.0068642955844701\\
80.11	0.00686565535202744\\
80.12	0.00686701524572418\\
80.13	0.00686837526778387\\
80.14	0.00686973542045118\\
80.15	0.00687109570599198\\
80.16	0.00687245612669346\\
80.17	0.00687381668486431\\
80.18	0.00687517738283478\\
80.19	0.00687653822295685\\
80.2	0.00687789920760436\\
80.21	0.00687926033917313\\
80.22	0.00688062162008107\\
80.23	0.00688198305276835\\
80.24	0.00688334463969751\\
80.25	0.00688470638335358\\
80.26	0.00688606828624426\\
80.27	0.00688743035090001\\
80.28	0.00688879257987417\\
80.29	0.00689015497574318\\
80.3	0.00689151754110659\\
80.31	0.00689288027858734\\
80.32	0.00689424319083176\\
80.33	0.00689560628050981\\
80.34	0.00689696955031516\\
80.35	0.00689833300296537\\
80.36	0.00689969664120199\\
80.37	0.00690106046779073\\
80.38	0.00690242448552159\\
80.39	0.00690378869720901\\
80.4	0.00690515310569201\\
80.41	0.00690651771383432\\
80.42	0.00690788252270627\\
80.43	0.0069092475321261\\
80.44	0.0069106127419129\\
80.45	0.00691197815188654\\
80.46	0.00691334376186768\\
80.47	0.00691470957167783\\
80.48	0.00691607558113929\\
80.49	0.00691744179007521\\
80.5	0.00691880819830953\\
80.51	0.00692017480566704\\
80.52	0.0069215416119734\\
80.53	0.00692290861705504\\
80.54	0.0069242758207393\\
80.55	0.00692564322285434\\
80.56	0.00692701082322918\\
80.57	0.0069283786216937\\
80.58	0.00692974661807866\\
80.59	0.00693111481221565\\
80.6	0.00693248320393717\\
80.61	0.0069338517930766\\
80.62	0.00693522057946817\\
80.63	0.00693658956294704\\
80.64	0.00693795874334923\\
80.65	0.00693932812051167\\
80.66	0.0069406976942722\\
80.67	0.00694206746446956\\
80.68	0.00694343743094339\\
80.69	0.00694480759353427\\
80.7	0.00694617795208369\\
80.71	0.00694754850643407\\
80.72	0.00694891925642875\\
80.73	0.00695029020191202\\
80.74	0.0069516613427291\\
80.75	0.00695303267872617\\
80.76	0.00695440420975033\\
80.77	0.00695577593564968\\
80.78	0.00695714785627323\\
80.79	0.00695851997147101\\
80.8	0.00695989228109397\\
80.81	0.00696126478499406\\
80.82	0.0069626374830242\\
80.83	0.0069640103750383\\
80.84	0.00696538346089126\\
80.85	0.00696675674043896\\
80.86	0.0069681302135383\\
80.87	0.00696950388004717\\
80.88	0.00697087773982447\\
80.89	0.0069722517927301\\
80.9	0.00697362603862501\\
80.91	0.00697500047737115\\
80.92	0.0069763751088315\\
80.93	0.00697774993287007\\
80.94	0.00697912494935192\\
80.95	0.00698050015814314\\
80.96	0.00698187555911087\\
80.97	0.00698325115212331\\
80.98	0.00698462693704971\\
80.99	0.00698600291376038\\
81	0.00698737908212669\\
81.01	0.0069887554420211\\
81.02	0.00699013199331714\\
81.03	0.00699150873588942\\
81.04	0.00699288566961363\\
81.05	0.00699426279436656\\
81.06	0.00699564011002608\\
81.07	0.00699701761647118\\
81.08	0.00699839531358196\\
81.09	0.00699977320123961\\
81.1	0.00700115127932644\\
81.11	0.00700252954772589\\
81.12	0.00700390800632253\\
81.13	0.00700528665500205\\
81.14	0.00700666549365126\\
81.15	0.00700804452215815\\
81.16	0.00700942374041182\\
81.17	0.00701080314830253\\
81.18	0.00701218274572169\\
81.19	0.0070135625325619\\
81.2	0.00701494250871688\\
81.21	0.00701632267408156\\
81.22	0.007017703028552\\
81.23	0.0070190835720255\\
81.24	0.00702046430440047\\
81.25	0.00702184522557657\\
81.26	0.00702322633545463\\
81.27	0.00702460763393668\\
81.28	0.00702598912092595\\
81.29	0.00702737079632689\\
81.3	0.00702875266004515\\
81.31	0.00703013471198762\\
81.32	0.00703151695206237\\
81.33	0.00703289938017876\\
81.34	0.00703428199624734\\
81.35	0.00703566480017989\\
81.36	0.00703704779188948\\
81.37	0.00703843097129037\\
81.38	0.00703981433829812\\
81.39	0.00704119789282952\\
81.4	0.00704258163480265\\
81.41	0.00704396556413681\\
81.42	0.00704534968075262\\
81.43	0.00704673398457195\\
81.44	0.00704811847551796\\
81.45	0.0070495031535151\\
81.46	0.00705088801848909\\
81.47	0.00705227307036698\\
81.48	0.0070536583090771\\
81.49	0.00705504373454909\\
81.5	0.00705642934671389\\
81.51	0.00705781514550378\\
81.52	0.00705920113085234\\
81.53	0.00706058730269448\\
81.54	0.00706197366096645\\
81.55	0.00706336020560581\\
81.56	0.00706474693655149\\
81.57	0.00706613385374374\\
81.58	0.00706752095712418\\
81.59	0.00706890824663575\\
81.6	0.00707029572222278\\
81.61	0.00707168338383096\\
81.62	0.00707307123140734\\
81.63	0.00707445926490034\\
81.64	0.00707584748425976\\
81.65	0.00707723588943678\\
81.66	0.00707862448038399\\
81.67	0.00708001325705532\\
81.68	0.00708140221940615\\
81.69	0.00708279136739324\\
81.7	0.00708418070097473\\
81.71	0.00708557022011022\\
81.72	0.00708695992476067\\
81.73	0.00708834981488853\\
81.74	0.00708973989045759\\
81.75	0.00709113015143312\\
81.76	0.00709252059778183\\
81.77	0.00709391122947184\\
81.78	0.00709530204647272\\
81.79	0.00709669304875549\\
81.8	0.00709808423629264\\
81.81	0.00709947560905807\\
81.82	0.00710086716702719\\
81.83	0.00710225891017684\\
81.84	0.00710365083848535\\
81.85	0.00710504295193251\\
81.86	0.00710643525049961\\
81.87	0.00710782773416939\\
81.88	0.00710922040292611\\
81.89	0.0071106132567555\\
81.9	0.00711200629564479\\
81.91	0.00711339951958271\\
81.92	0.00711479292855952\\
81.93	0.00711618652256695\\
81.94	0.00711758030159827\\
81.95	0.00711897426564826\\
81.96	0.00712036841471322\\
81.97	0.00712176274879099\\
81.98	0.00712315726788091\\
81.99	0.0071245519719839\\
82	0.00712594686110237\\
82.01	0.00712734193524032\\
82.02	0.00712873719440326\\
82.03	0.00713013263859829\\
82.04	0.00713152826783403\\
82.05	0.00713292408212068\\
82.06	0.00713432008147\\
82.07	0.00713571626589532\\
82.08	0.00713711263541155\\
82.09	0.00713850919003516\\
82.1	0.00713990592978421\\
82.11	0.00714130285467836\\
82.12	0.00714269996473883\\
82.13	0.00714409725998845\\
82.14	0.00714549474045166\\
82.15	0.00714689240615446\\
82.16	0.00714829025712451\\
82.17	0.00714968829339103\\
82.18	0.00715108651498489\\
82.19	0.00715248492193855\\
82.2	0.00715388351428609\\
82.21	0.00715528229206323\\
82.22	0.00715668125530733\\
82.23	0.00715808040405735\\
82.24	0.00715947973835391\\
82.25	0.00716087925823924\\
82.26	0.00716227896375725\\
82.27	0.00716367885495349\\
82.28	0.00716507893187512\\
82.29	0.00716647919457101\\
82.3	0.00716787964309164\\
82.31	0.00716928027748919\\
82.32	0.00717068109781748\\
82.33	0.00717208210413201\\
82.34	0.00717348329648995\\
82.35	0.00717488467495013\\
82.36	0.00717628623957308\\
82.37	0.00717768799042103\\
82.38	0.00717908992755794\\
82.39	0.00718049205104947\\
82.4	0.00718189436096303\\
82.41	0.00718329685736775\\
82.42	0.00718469954033446\\
82.43	0.00718610240993577\\
82.44	0.00718750546624599\\
82.45	0.00718890870934119\\
82.46	0.00719031213929916\\
82.47	0.00719171575619944\\
82.48	0.00719311956012333\\
82.49	0.00719452355115388\\
82.5	0.00719592772937585\\
82.51	0.00719733209487582\\
82.52	0.00719873664774207\\
82.53	0.00720014138806467\\
82.54	0.00720154631593543\\
82.55	0.00720295143144796\\
82.56	0.0072043567346976\\
82.57	0.00720576222578146\\
82.58	0.00720716790479846\\
82.59	0.00720857377184924\\
82.6	0.00720997982703625\\
82.61	0.0072113860704637\\
82.62	0.0072127925022376\\
82.63	0.00721419912246573\\
82.64	0.00721560593125764\\
82.65	0.00721701292872468\\
82.66	0.00721842011497998\\
82.67	0.00721982749013848\\
82.68	0.00722123505431688\\
82.69	0.0072226428076337\\
82.7	0.00722405075020923\\
82.71	0.00722545888216557\\
82.72	0.00722686720362662\\
82.73	0.00722827571471807\\
82.74	0.00722968441556741\\
82.75	0.00723109330630395\\
82.76	0.00723250238705878\\
82.77	0.00723391165796481\\
82.78	0.00723532111915674\\
82.79	0.0072367307707711\\
82.8	0.0072381406129462\\
82.81	0.00723955064582219\\
82.82	0.00724096086954102\\
82.83	0.00724237128424643\\
82.84	0.00724378189008401\\
82.85	0.00724519268720112\\
82.86	0.00724660367574698\\
82.87	0.00724801485587258\\
82.88	0.00724942622773076\\
82.89	0.00725083779147616\\
82.9	0.00725224954726524\\
82.91	0.00725366149525628\\
82.92	0.00725507363560937\\
82.93	0.00725648596848642\\
82.94	0.00725789849405117\\
82.95	0.00725931121246916\\
82.96	0.00726072412390776\\
82.97	0.00726213722853617\\
82.98	0.00726355052652538\\
82.99	0.00726496401804821\\
83	0.00726637770327932\\
83.01	0.00726779158239517\\
83.02	0.00726920565557403\\
83.03	0.00727061992299599\\
83.04	0.007272034384843\\
83.05	0.00727344904129876\\
83.06	0.00727486389254884\\
83.07	0.00727627893878059\\
83.08	0.0072776941801832\\
83.09	0.00727910961694767\\
83.1	0.0072805252492668\\
83.11	0.00728194107733522\\
83.12	0.00728335710134937\\
83.13	0.00728477332150748\\
83.14	0.00728618973800962\\
83.15	0.00728760635105766\\
83.16	0.00728902316085525\\
83.17	0.00729044016760789\\
83.18	0.00729185737152286\\
83.19	0.00729327477280924\\
83.2	0.00729469237167791\\
83.21	0.00729611016834157\\
83.22	0.00729752816301471\\
83.23	0.00729894635591359\\
83.24	0.00730036474725632\\
83.25	0.00730178333726273\\
83.26	0.0073032021261545\\
83.27	0.00730462111415508\\
83.28	0.00730604030148968\\
83.29	0.00730745968838534\\
83.3	0.00730887927507084\\
83.31	0.00731029906177675\\
83.32	0.00731171904873544\\
83.33	0.00731313923618102\\
83.34	0.00731455962434938\\
83.35	0.00731598021347818\\
83.36	0.00731740100380684\\
83.37	0.00731882199557655\\
83.38	0.00732024318903025\\
83.39	0.00732166458441263\\
83.4	0.00732308618197014\\
83.41	0.00732450798195099\\
83.42	0.00732592998460508\\
83.43	0.00732735219018412\\
83.44	0.0073287745989415\\
83.45	0.00733019721113239\\
83.46	0.00733162002701365\\
83.47	0.00733304304684387\\
83.48	0.00733446627088338\\
83.49	0.0073358896993942\\
83.5	0.00733731333264009\\
83.51	0.00733873717088649\\
83.52	0.00734016121440055\\
83.53	0.00734158546345112\\
83.54	0.00734300991830874\\
83.55	0.00734443457924563\\
83.56	0.00734585944653571\\
83.57	0.00734728452045456\\
83.58	0.00734870980127943\\
83.59	0.00735013528928925\\
83.6	0.00735156098476461\\
83.61	0.00735298688798773\\
83.62	0.0073544129992425\\
83.63	0.00735583931881444\\
83.64	0.00735726584699072\\
83.65	0.00735869258406015\\
83.66	0.00736011953031311\\
83.67	0.00736154668604167\\
83.68	0.00736297405153946\\
83.69	0.00736440162710174\\
83.7	0.00736582941302533\\
83.71	0.0073672574096087\\
83.72	0.00736868561715185\\
83.73	0.00737011403595638\\
83.74	0.00737154266632545\\
83.75	0.0073729715085638\\
83.76	0.0073744005629777\\
83.77	0.00737582982987497\\
83.78	0.00737725930956498\\
83.79	0.00737868900235862\\
83.8	0.00738011890856831\\
83.81	0.00738154902850797\\
83.82	0.00738297936249303\\
83.83	0.00738440991084042\\
83.84	0.00738584067386855\\
83.85	0.00738727165189733\\
83.86	0.0073887028452481\\
83.87	0.00739013425424369\\
83.88	0.00739156587920838\\
83.89	0.00739299772046789\\
83.9	0.00739442977834934\\
83.91	0.00739586205318133\\
83.92	0.00739729454529383\\
83.93	0.00739872725501822\\
83.94	0.00740016018268726\\
83.95	0.00740159332863513\\
83.96	0.00740302669319734\\
83.97	0.00740446027671078\\
83.98	0.00740589407951368\\
83.99	0.0074073281019456\\
84	0.00740876234434746\\
84.01	0.00741019680706147\\
84.02	0.00741163149043113\\
84.03	0.00741306639480126\\
84.04	0.00741450152051794\\
84.05	0.00741593686792855\\
84.06	0.00741737243738167\\
84.07	0.00741880822922717\\
84.08	0.00742024424381614\\
84.09	0.00742168048150088\\
84.1	0.00742311694263489\\
84.11	0.00742455362757288\\
84.12	0.00742599053649315\\
84.13	0.00742742766937362\\
84.14	0.00742886502619391\\
84.15	0.00743030260693534\\
84.16	0.00743174041158092\\
84.17	0.00743317844011541\\
84.18	0.00743461669252526\\
84.19	0.00743605516879868\\
84.2	0.0074374938689256\\
84.21	0.00743893279289771\\
84.22	0.00744037194070845\\
84.23	0.007441811312353\\
84.24	0.00744325090782837\\
84.25	0.00744469072713328\\
84.26	0.00744613077026829\\
84.27	0.00744757103723572\\
84.28	0.0074490115280397\\
84.29	0.0074504522426862\\
84.3	0.00745189318118296\\
84.31	0.00745333434353958\\
84.32	0.00745477572976747\\
84.33	0.00745621733987991\\
84.34	0.007457659173892\\
84.35	0.00745910123182073\\
84.36	0.00746054351368492\\
84.37	0.00746198601950529\\
84.38	0.00746342874930444\\
84.39	0.00746487170310683\\
84.4	0.00746631488093887\\
84.41	0.00746775828282883\\
84.42	0.0074692019088069\\
84.43	0.00747064575890522\\
84.44	0.00747208983315782\\
84.45	0.00747353413160071\\
84.46	0.00747497865427182\\
84.47	0.00747642340121103\\
84.48	0.00747786837246019\\
84.49	0.00747931356806312\\
84.5	0.00748075898806563\\
84.51	0.0074822046325155\\
84.52	0.0074836505014625\\
84.53	0.00748509659495843\\
84.54	0.00748654291305707\\
84.55	0.00748798945581423\\
84.56	0.00748943622328776\\
84.57	0.00749088321553753\\
84.58	0.00749233043262546\\
84.59	0.00749377787461551\\
84.6	0.00749522554157372\\
84.61	0.00749667343356819\\
84.62	0.00749812155066908\\
84.63	0.00749956989294866\\
84.64	0.00750101846048127\\
84.65	0.00750246725334337\\
84.66	0.00750391627161352\\
84.67	0.0075053655153724\\
84.68	0.0075068149847028\\
84.69	0.00750826467968966\\
84.7	0.00750971460042008\\
84.71	0.00751116474698326\\
84.72	0.00751261511947061\\
84.73	0.00751406571797568\\
84.74	0.0075155165425942\\
84.75	0.00751696759342408\\
84.76	0.00751841887056543\\
84.77	0.00751987037412057\\
84.78	0.00752132210419399\\
84.79	0.00752277406089243\\
84.8	0.00752422624432484\\
84.81	0.00752567865460243\\
84.82	0.00752713129183862\\
84.83	0.00752858415614908\\
84.84	0.00753003724765177\\
84.85	0.00753149056646688\\
84.86	0.00753294411271689\\
84.87	0.00753439788652658\\
84.88	0.007535851888023\\
84.89	0.00753730611733551\\
84.9	0.00753876057459576\\
84.91	0.00754021525993774\\
84.92	0.00754167017349906\\
84.93	0.00754312531542116\\
84.94	0.00754458068584934\\
84.95	0.00754603628493279\\
84.96	0.00754749211282461\\
84.97	0.00754894816968182\\
84.98	0.00755040445566543\\
84.99	0.00755186097094039\\
85	0.00755331771567571\\
85.01	0.00755477469004441\\
85.02	0.00755623189422357\\
85.03	0.00755768932839438\\
85.04	0.00755914699274214\\
85.05	0.00756060488745627\\
85.06	0.00756206301273038\\
85.07	0.00756352136876227\\
85.08	0.00756497995575398\\
85.09	0.00756643877391176\\
85.1	0.00756789782344616\\
85.11	0.00756935710457203\\
85.12	0.00757081661750856\\
85.13	0.00757227636247929\\
85.14	0.00757373633971213\\
85.15	0.00757519654943943\\
85.16	0.00757665699189796\\
85.17	0.00757811766732898\\
85.18	0.00757957857597823\\
85.19	0.00758103971809597\\
85.2	0.00758250109393703\\
85.21	0.00758396270376082\\
85.22	0.00758542454783135\\
85.23	0.00758688662641728\\
85.24	0.00758834893979194\\
85.25	0.00758981148823336\\
85.26	0.00759127427202428\\
85.27	0.00759273729145221\\
85.28	0.00759420054680945\\
85.29	0.0075956640383931\\
85.3	0.00759712776650513\\
85.31	0.00759859173145237\\
85.32	0.00760005593354655\\
85.33	0.00760152037310435\\
85.34	0.00760298505044741\\
85.35	0.00760444996590237\\
85.36	0.0076059151198009\\
85.37	0.00760738051247972\\
85.38	0.00760884614428064\\
85.39	0.00761031201555061\\
85.4	0.00761177812664172\\
85.41	0.00761324447791124\\
85.42	0.00761471106972164\\
85.43	0.00761617790244069\\
85.44	0.00761764497644139\\
85.45	0.00761911229210206\\
85.46	0.00762057984980639\\
85.47	0.00762204764994342\\
85.48	0.00762351569290761\\
85.49	0.00762498397909886\\
85.5	0.00762645250892255\\
85.51	0.00762792128278955\\
85.52	0.00762939030111629\\
85.53	0.00763085956432478\\
85.54	0.00763232907284261\\
85.55	0.00763379882710303\\
85.56	0.00763526882754498\\
85.57	0.00763673907461308\\
85.58	0.00763820956875771\\
85.59	0.00763968031043503\\
85.6	0.00764115130010701\\
85.61	0.00764262253824146\\
85.62	0.00764409402531209\\
85.63	0.00764556576179852\\
85.64	0.00764703774818633\\
85.65	0.00764850998496706\\
85.66	0.0076499824726383\\
85.67	0.0076514552117037\\
85.68	0.007652928202673\\
85.69	0.00765440144606207\\
85.7	0.00765587494239295\\
85.71	0.00765734869219389\\
85.72	0.00765882269599937\\
85.73	0.00766029695435017\\
85.74	0.00766177146779335\\
85.75	0.00766324623688236\\
85.76	0.00766472126217701\\
85.77	0.00766619654424356\\
85.78	0.00766767208365472\\
85.79	0.00766914788098971\\
85.8	0.0076706239368343\\
85.81	0.0076721002517808\\
85.82	0.0076735768264282\\
85.83	0.0076750536613821\\
85.84	0.0076765307572548\\
85.85	0.00767800811466536\\
85.86	0.00767948573423958\\
85.87	0.0076809636166101\\
85.88	0.0076824417624164\\
85.89	0.00768392017230485\\
85.9	0.00768539884692876\\
85.91	0.0076868777869484\\
85.92	0.00768835699303108\\
85.93	0.00768983646585113\\
85.94	0.00769131620608999\\
85.95	0.00769279621443625\\
85.96	0.00769427649158564\\
85.97	0.00769575703824114\\
85.98	0.00769723785511298\\
85.99	0.00769871894291867\\
86	0.0077002003023831\\
86.01	0.00770168193423851\\
86.02	0.00770316383922459\\
86.03	0.00770464601808849\\
86.04	0.00770612847158487\\
86.05	0.00770761120047596\\
86.06	0.00770909420553156\\
86.07	0.00771057748752915\\
86.08	0.00771206104725386\\
86.09	0.00771354488549857\\
86.1	0.00771502900306393\\
86.11	0.00771651340075839\\
86.12	0.00771799807939829\\
86.13	0.00771948303980785\\
86.14	0.00772096828281926\\
86.15	0.0077224538092727\\
86.16	0.00772393962001639\\
86.17	0.00772542571590663\\
86.18	0.00772691209780787\\
86.19	0.00772839876659271\\
86.2	0.007729885723142\\
86.21	0.00773137296834486\\
86.22	0.00773286050309871\\
86.23	0.00773434832830933\\
86.24	0.00773583644489095\\
86.25	0.00773732485376622\\
86.26	0.00773881355586631\\
86.27	0.00774030255213093\\
86.28	0.00774179184350841\\
86.29	0.00774328143095572\\
86.3	0.00774477131543853\\
86.31	0.00774626149793124\\
86.32	0.00774775197941706\\
86.33	0.00774924276088806\\
86.34	0.00775073384334517\\
86.35	0.00775222522779828\\
86.36	0.00775371691526626\\
86.37	0.00775520890677703\\
86.38	0.00775670120336763\\
86.39	0.00775819380608418\\
86.4	0.00775968671598205\\
86.41	0.00776117993412581\\
86.42	0.00776267346158937\\
86.43	0.00776416729945596\\
86.44	0.00776566144881819\\
86.45	0.00776715591077816\\
86.46	0.00776865068644744\\
86.47	0.00777014577694719\\
86.48	0.00777164118340812\\
86.49	0.00777313690697065\\
86.5	0.0077746329487849\\
86.51	0.00777612931001074\\
86.52	0.00777762599181787\\
86.53	0.00777912299538587\\
86.54	0.00778062032190423\\
86.55	0.00778211797257243\\
86.56	0.00778361594859999\\
86.57	0.00778511425120652\\
86.58	0.00778661288162177\\
86.59	0.00778811184108569\\
86.6	0.00778961113084849\\
86.61	0.00779111075217069\\
86.62	0.00779261070632319\\
86.63	0.00779411099458729\\
86.64	0.00779561161825479\\
86.65	0.00779711257862803\\
86.66	0.00779861387701992\\
86.67	0.00780011551475406\\
86.68	0.00780161749316473\\
86.69	0.00780311981359699\\
86.7	0.00780462247740672\\
86.71	0.00780612548596068\\
86.72	0.0078076288406366\\
86.73	0.00780913254282318\\
86.74	0.00781063659392019\\
86.75	0.00781214099533854\\
86.76	0.00781364574850028\\
86.77	0.00781515085483876\\
86.78	0.00781665631579855\\
86.79	0.00781816213283567\\
86.8	0.00781966830741748\\
86.81	0.00782117484102289\\
86.82	0.00782268173424422\\
86.83	0.00782418898728899\\
86.84	0.00782569660036539\\
86.85	0.00782720457368234\\
86.86	0.00782871290744944\\
86.87	0.00783022160187698\\
86.88	0.00783173065717592\\
86.89	0.00783324007355792\\
86.9	0.00783474985123528\\
86.91	0.00783625999042096\\
86.92	0.00783777049132859\\
86.93	0.00783928135417245\\
86.94	0.00784079257916745\\
86.95	0.00784230416652913\\
86.96	0.00784381611647366\\
86.97	0.00784532842921785\\
86.98	0.00784684110497909\\
86.99	0.00784835414397539\\
87	0.00784986754642535\\
87.01	0.00785138131254818\\
87.02	0.00785289544256364\\
87.03	0.0078544099366921\\
87.04	0.00785592479515445\\
87.05	0.00785744001817219\\
87.06	0.00785895560596733\\
87.07	0.00786047155876244\\
87.08	0.00786198787678063\\
87.09	0.00786350456024551\\
87.1	0.00786502160938123\\
87.11	0.00786653902441243\\
87.12	0.00786805680556427\\
87.13	0.00786957495306239\\
87.14	0.00787109346713289\\
87.15	0.00787261234800238\\
87.16	0.00787413159589791\\
87.17	0.00787565121104699\\
87.18	0.00787717119367756\\
87.19	0.00787869154401803\\
87.2	0.00788021226229718\\
87.21	0.00788173334874426\\
87.22	0.0078832548035889\\
87.23	0.00788477662706111\\
87.24	0.00788629881939131\\
87.25	0.00788782138081029\\
87.26	0.00788934431154919\\
87.27	0.00789086761183952\\
87.28	0.00789239128191311\\
87.29	0.00789391532200216\\
87.3	0.00789543973233916\\
87.31	0.00789696451315692\\
87.32	0.00789848966468853\\
87.33	0.00790001518716741\\
87.34	0.00790154108082722\\
87.35	0.00790306734590189\\
87.36	0.00790459398262561\\
87.37	0.00790612099123282\\
87.38	0.00790764837195816\\
87.39	0.00790917612503652\\
87.4	0.00791070425070296\\
87.41	0.00791223274919277\\
87.42	0.00791376162074138\\
87.43	0.00791529086558443\\
87.44	0.00791682048395768\\
87.45	0.00791835047609703\\
87.46	0.00791988084223854\\
87.47	0.00792141158261835\\
87.48	0.00792294269747272\\
87.49	0.007924474187038\\
87.5	0.00792600605155059\\
87.51	0.00792753829124698\\
87.52	0.00792907090636369\\
87.53	0.00793060389713727\\
87.54	0.00793213726380429\\
87.55	0.00793367100660133\\
87.56	0.00793520512576495\\
87.57	0.0079367396215317\\
87.58	0.00793827449413806\\
87.59	0.00793980974382048\\
87.6	0.0079413453708153\\
87.61	0.00794288137535884\\
87.62	0.00794441775768726\\
87.63	0.0079459545180366\\
87.64	0.00794749165664281\\
87.65	0.00794902917374166\\
87.66	0.00795056706956874\\
87.67	0.00795210534435948\\
87.68	0.00795364399834912\\
87.69	0.00795518303177265\\
87.7	0.00795672244486485\\
87.71	0.00795826223786023\\
87.72	0.00795980241099306\\
87.73	0.00796134296449731\\
87.74	0.00796288389860664\\
87.75	0.0079644252135544\\
87.76	0.00796596690957358\\
87.77	0.00796750898689685\\
87.78	0.00796905144575648\\
87.79	0.00797059428638434\\
87.8	0.00797213750901191\\
87.81	0.00797368111387021\\
87.82	0.00797522510118984\\
87.83	0.0079767694712009\\
87.84	0.00797831422413302\\
87.85	0.00797985936021532\\
87.86	0.00798140487967638\\
87.87	0.00798295078274425\\
87.88	0.00798449706964638\\
87.89	0.00798604374060966\\
87.9	0.00798759079586034\\
87.91	0.00798913823562408\\
87.92	0.00799068606012584\\
87.93	0.00799223426958994\\
87.94	0.00799378286423999\\
87.95	0.00799533184429889\\
87.96	0.00799688120998879\\
87.97	0.00799843096153108\\
87.98	0.00799998109914639\\
87.99	0.00800153162305452\\
88	0.00800308253347444\\
88.01	0.00800463383062428\\
88.02	0.00800618551472129\\
88.03	0.00800773758598183\\
88.04	0.00800929004462131\\
88.05	0.00801084289085424\\
88.06	0.00801239612489412\\
88.07	0.00801394974695348\\
88.08	0.00801550375724382\\
88.09	0.00801705815597559\\
88.1	0.00801861294335818\\
88.11	0.0080201681195999\\
88.12	0.00802172368490792\\
88.13	0.00802327963948827\\
88.14	0.00802483598354582\\
88.15	0.00802639271728423\\
88.16	0.00802794984090594\\
88.17	0.00802950735461215\\
88.18	0.00803106525860277\\
88.19	0.00803262355307642\\
88.2	0.00803418223823037\\
88.21	0.00803574131426055\\
88.22	0.0080373007813615\\
88.23	0.00803886063972633\\
88.24	0.00804042088954672\\
88.25	0.00804198153101288\\
88.26	0.00804354256431351\\
88.27	0.00804510398963579\\
88.28	0.00804666580716534\\
88.29	0.00804822801708618\\
88.3	0.00804979061958072\\
88.31	0.00805135361482972\\
88.32	0.00805291700301226\\
88.33	0.00805448078430571\\
88.34	0.00805604495888571\\
88.35	0.00805760952692609\\
88.36	0.00805917448859893\\
88.37	0.00806073984407444\\
88.38	0.00806230559352096\\
88.39	0.00806387173710495\\
88.4	0.00806543827499091\\
88.41	0.0080670052073414\\
88.42	0.00806857253431697\\
88.43	0.00807014025607614\\
88.44	0.00807170837277536\\
88.45	0.00807327688456898\\
88.46	0.00807484579160921\\
88.47	0.00807641509404611\\
88.48	0.00807798479202751\\
88.49	0.00807955488569901\\
88.5	0.00808112537520394\\
88.51	0.00808269626068332\\
88.52	0.00808426754227581\\
88.53	0.00808583922011769\\
88.54	0.00808741129434283\\
88.55	0.00808898376508262\\
88.56	0.00809055663246598\\
88.57	0.00809212989661928\\
88.58	0.00809370355766632\\
88.59	0.00809527761572829\\
88.6	0.00809685207092374\\
88.61	0.00809842692336852\\
88.62	0.00810000217317576\\
88.63	0.00810157782045583\\
88.64	0.00810315386531628\\
88.65	0.00810473030786181\\
88.66	0.00810630714819425\\
88.67	0.00810788438641248\\
88.68	0.00810946202261243\\
88.69	0.00811104005688701\\
88.7	0.00811261848932605\\
88.71	0.00811419732001634\\
88.72	0.00811577654904147\\
88.73	0.00811735617648189\\
88.74	0.00811893620241481\\
88.75	0.00812051662691415\\
88.76	0.00812209745005055\\
88.77	0.00812367867189127\\
88.78	0.00812526029250017\\
88.79	0.00812684231193767\\
88.8	0.00812842473026067\\
88.81	0.00813000754752255\\
88.82	0.00813159076377311\\
88.83	0.0081331743790585\\
88.84	0.00813475839342119\\
88.85	0.00813634280689994\\
88.86	0.0081379276195297\\
88.87	0.00813951283134162\\
88.88	0.00814109844236296\\
88.89	0.00814268445261708\\
88.9	0.00814427086212336\\
88.91	0.00814585767089714\\
88.92	0.00814744487894969\\
88.93	0.00814903248628819\\
88.94	0.0081506204929156\\
88.95	0.00815220889883069\\
88.96	0.00815379770402794\\
88.97	0.00815538690849749\\
88.98	0.0081569765122251\\
88.99	0.00815856651519211\\
89	0.00816015691737536\\
89.01	0.00816174771874713\\
89.02	0.00816333891927514\\
89.03	0.00816493051892242\\
89.04	0.00816652251764732\\
89.05	0.0081681149154034\\
89.06	0.00816970771213941\\
89.07	0.00817130090779925\\
89.08	0.00817289450232184\\
89.09	0.00817448849564114\\
89.1	0.00817608288768605\\
89.11	0.00817767767838037\\
89.12	0.00817927286764273\\
89.13	0.00818086845538652\\
89.14	0.00818246444151987\\
89.15	0.00818406082594554\\
89.16	0.00818565760856092\\
89.17	0.00818725478925789\\
89.18	0.00818885236792282\\
89.19	0.00819045034443835\\
89.2	0.00819204871868663\\
89.21	0.00819364749054929\\
89.22	0.00819524665990746\\
89.23	0.00819684622664177\\
89.24	0.00819844619063233\\
89.25	0.00820004655175874\\
89.26	0.00820164730990011\\
89.27	0.00820324846493501\\
89.28	0.00820485001674151\\
89.29	0.00820645196519716\\
89.3	0.00820805431017899\\
89.31	0.00820965705156352\\
89.32	0.00821126018922673\\
89.33	0.00821286372304409\\
89.34	0.00821446765289054\\
89.35	0.0082160719786405\\
89.36	0.00821767670016786\\
89.37	0.00821928181734598\\
89.38	0.00822088733004769\\
89.39	0.00822249323814528\\
89.4	0.00822409954151051\\
89.41	0.00822570624001461\\
89.42	0.00822731333352826\\
89.43	0.00822892082192164\\
89.44	0.00823052870506433\\
89.45	0.0082321369828254\\
89.46	0.00823374565507339\\
89.47	0.00823535472167628\\
89.48	0.00823696418250149\\
89.49	0.00823857403741591\\
89.5	0.00824018428628589\\
89.51	0.0082417949289772\\
89.52	0.00824340596535508\\
89.53	0.00824501739528421\\
89.54	0.00824662921862871\\
89.55	0.00824824143525216\\
89.56	0.00824985404501757\\
89.57	0.00825146704778737\\
89.58	0.00825308044342348\\
89.59	0.00825469423178721\\
89.6	0.00825630841273933\\
89.61	0.00825792298614004\\
89.62	0.00825953795184898\\
89.63	0.0082611533097252\\
89.64	0.00826276905962721\\
89.65	0.00826438520141293\\
89.66	0.00826600173493972\\
89.67	0.00826761866006436\\
89.68	0.00826923597664306\\
89.69	0.00827085368453146\\
89.7	0.00827247178358459\\
89.71	0.00827409027365696\\
89.72	0.00827570915460246\\
89.73	0.0082773284262744\\
89.74	0.00827894808852552\\
89.75	0.00828056814120797\\
89.76	0.00828218858417333\\
89.77	0.00828380941727259\\
89.78	0.00828543064035613\\
89.79	0.00828705225327378\\
89.8	0.00828867425587475\\
89.81	0.00829029664800769\\
89.82	0.00829191942952063\\
89.83	0.00829354260026103\\
89.84	0.00829516616007574\\
89.85	0.00829679010881103\\
89.86	0.00829841444631258\\
89.87	0.00830003917242545\\
89.88	0.00830166428699412\\
89.89	0.00830328978986248\\
89.9	0.00830491568087381\\
89.91	0.00830654195987079\\
89.92	0.0083081686266955\\
89.93	0.00830979568118942\\
89.94	0.00831142312319343\\
89.95	0.00831305095254782\\
89.96	0.00831467916909224\\
89.97	0.00831630777266576\\
89.98	0.00831793676310687\\
89.99	0.00831956614025339\\
90	0.0083211959039426\\
90.01	0.00832282605401112\\
90.02	0.008324456590295\\
90.03	0.00832608751262966\\
90.04	0.00832771882084991\\
90.05	0.00832935051478997\\
90.06	0.00833098259428342\\
90.07	0.00833261505916325\\
90.08	0.00833424790926184\\
90.09	0.00833588114441094\\
90.1	0.00833751476444171\\
90.11	0.00833914876918467\\
90.12	0.00834078315846975\\
90.13	0.00834241793212626\\
90.14	0.00834405308998288\\
90.15	0.0083456886318677\\
90.16	0.00834732455760819\\
90.17	0.00834896086703118\\
90.18	0.0083505975599629\\
90.19	0.00835223463622899\\
90.2	0.00835387209565443\\
90.21	0.00835550993806362\\
90.22	0.00835714816328032\\
90.23	0.00835878677112768\\
90.24	0.00836042576142824\\
90.25	0.00836206513400392\\
90.26	0.00836370488867601\\
90.27	0.00836534502526521\\
90.28	0.00836698554359158\\
90.29	0.00836862644347458\\
90.3	0.00837026772473304\\
90.31	0.00837190938718519\\
90.32	0.00837355143064862\\
90.33	0.00837519385494034\\
90.34	0.0083768366598767\\
90.35	0.00837847984527347\\
90.36	0.0083801234109458\\
90.37	0.0083817673567082\\
90.38	0.00838341168237461\\
90.39	0.0083850563877583\\
90.4	0.00838670147267199\\
90.41	0.00838834693692773\\
90.42	0.008389992780337\\
90.43	0.00839163900271064\\
90.44	0.0083932856038589\\
90.45	0.00839493258359141\\
90.46	0.00839657994171718\\
90.47	0.00839822767804464\\
90.48	0.00839987579238158\\
90.49	0.00840152428453521\\
90.5	0.00840317315431211\\
90.51	0.00840482240151828\\
90.52	0.00840647202595909\\
90.53	0.00840812202743933\\
90.54	0.00840977240576316\\
90.55	0.00841142316073418\\
90.56	0.00841307429215535\\
90.57	0.00841472579982905\\
90.58	0.00841637768355707\\
90.59	0.00841802994314058\\
90.6	0.00841968257838018\\
90.61	0.00842133558907586\\
90.62	0.00842298897502703\\
90.63	0.00842464273603251\\
90.64	0.00842629687189052\\
90.65	0.00842795138239869\\
90.66	0.00842960626735409\\
90.67	0.00843126152655317\\
90.68	0.00843291715979184\\
90.69	0.00843457316686539\\
90.7	0.00843622954756857\\
90.71	0.00843788630169551\\
90.72	0.0084395434290398\\
90.73	0.00844120092939445\\
90.74	0.00844285880255189\\
90.75	0.008444517048304\\
90.76	0.00844617566644207\\
90.77	0.00844783465675707\\
90.78	0.00844949401903992\\
90.79	0.00845115375308155\\
90.8	0.00845281385867283\\
90.81	0.00845447433560465\\
90.82	0.00845613518366784\\
90.83	0.00845779640265326\\
90.84	0.00845945799235172\\
90.85	0.00846111995255402\\
90.86	0.00846278228305096\\
90.87	0.00846444498363333\\
90.88	0.00846610805409191\\
90.89	0.00846777149421745\\
90.9	0.00846943530380073\\
90.91	0.00847109948263251\\
90.92	0.00847276403050353\\
90.93	0.00847442894720456\\
90.94	0.00847609423252634\\
90.95	0.00847775988625964\\
90.96	0.00847942590819522\\
90.97	0.00848109229812384\\
90.98	0.00848275905583627\\
90.99	0.00848442618112329\\
91	0.00848609367377571\\
91.01	0.0084877615335843\\
91.02	0.00848942976033989\\
91.03	0.00849109835383331\\
91.04	0.00849276731385541\\
91.05	0.00849443664019705\\
91.06	0.0084961063326491\\
91.07	0.00849777639100247\\
91.08	0.00849944681504809\\
91.09	0.00850111760457691\\
91.1	0.00850278875937989\\
91.11	0.00850446027924806\\
91.12	0.00850613216397242\\
91.13	0.00850780441334406\\
91.14	0.00850947702715405\\
91.15	0.00851115000519354\\
91.16	0.00851282334725368\\
91.17	0.00851449705312567\\
91.18	0.00851617112260076\\
91.19	0.00851784555547022\\
91.2	0.00851952035152537\\
91.21	0.00852119551055757\\
91.22	0.00852287103235824\\
91.23	0.00852454691671884\\
91.24	0.00852622316343087\\
91.25	0.00852789977228588\\
91.26	0.00852957674307548\\
91.27	0.00853125407559134\\
91.28	0.00853293176962516\\
91.29	0.00853460982496872\\
91.3	0.00853628824141385\\
91.31	0.00853796701875244\\
91.32	0.00853964615677645\\
91.33	0.00854132565527789\\
91.34	0.00854300551404883\\
91.35	0.00854468573288144\\
91.36	0.00854636631156791\\
91.37	0.00854804724990055\\
91.38	0.00854972854767171\\
91.39	0.00855141020467381\\
91.4	0.00855309222069938\\
91.41	0.00855477459554098\\
91.42	0.00855645732899128\\
91.43	0.00855814042084302\\
91.44	0.00855982387088903\\
91.45	0.00856150767892221\\
91.46	0.00856319184473557\\
91.47	0.00856487636812216\\
91.48	0.00856656124887518\\
91.49	0.00856824648678786\\
91.5	0.00856993208165357\\
91.51	0.00857161803326574\\
91.52	0.00857330434141793\\
91.53	0.00857499100590377\\
91.54	0.00857667802651698\\
91.55	0.00857836540305141\\
91.56	0.00858005313530101\\
91.57	0.0085817412230598\\
91.58	0.00858342966612195\\
91.59	0.0085851184642817\\
91.6	0.00858680761733342\\
91.61	0.0085884971250716\\
91.62	0.00859018698729081\\
91.63	0.00859187720378577\\
91.64	0.00859356777435129\\
91.65	0.00859525869878231\\
91.66	0.0085969499768739\\
91.67	0.00859864160842122\\
91.68	0.00860033359321958\\
91.69	0.0086020259310644\\
91.7	0.00860371862175125\\
91.71	0.00860541166507581\\
91.72	0.00860710506083388\\
91.73	0.0086087988088214\\
91.74	0.00861049290883446\\
91.75	0.00861218736066927\\
91.76	0.00861388216412218\\
91.77	0.00861557731898968\\
91.78	0.00861727282506839\\
91.79	0.00861896868215509\\
91.8	0.00862066489004669\\
91.81	0.00862236144854026\\
91.82	0.008624058357433\\
91.83	0.00862575561652227\\
91.84	0.00862745322560558\\
91.85	0.00862915118448059\\
91.86	0.00863084949294513\\
91.87	0.00863254815079715\\
91.88	0.0086342471578348\\
91.89	0.00863594651385637\\
91.9	0.0086376462186603\\
91.91	0.00863934627204521\\
91.92	0.0086410466738099\\
91.93	0.0086427474237533\\
91.94	0.00864444852167452\\
91.95	0.00864614996737287\\
91.96	0.00864785176064779\\
91.97	0.00864955390129893\\
91.98	0.00865125638912609\\
91.99	0.00865295922392927\\
92	0.00865466240550861\\
92.01	0.00865636593366449\\
92.02	0.00865806980819741\\
92.03	0.00865977402890811\\
92.04	0.00866147859559748\\
92.05	0.00866318350806661\\
92.06	0.00866488876611679\\
92.07	0.00866659436954947\\
92.08	0.00866830031816633\\
92.09	0.00867000661176924\\
92.1	0.00867171325016024\\
92.11	0.0086734202331416\\
92.12	0.00867512756051576\\
92.13	0.0086768352320854\\
92.14	0.00867854324765336\\
92.15	0.00868025160702273\\
92.16	0.00868196030999677\\
92.17	0.00868366935637898\\
92.18	0.00868537874597305\\
92.19	0.0086870884785829\\
92.2	0.00868879855401265\\
92.21	0.00869050897206663\\
92.22	0.00869221973254942\\
92.23	0.00869393083526579\\
92.24	0.00869564228002074\\
92.25	0.0086973540666195\\
92.26	0.00869906619486752\\
92.27	0.00870077866457049\\
92.28	0.0087024914755343\\
92.29	0.00870420462756509\\
92.3	0.00870591812046923\\
92.31	0.00870763195405334\\
92.32	0.00870934612812425\\
92.33	0.00871106064248903\\
92.34	0.00871277549695501\\
92.35	0.00871449069132974\\
92.36	0.00871620622542103\\
92.37	0.00871792209903691\\
92.38	0.0087196383119857\\
92.39	0.00872135486407593\\
92.4	0.00872307175511638\\
92.41	0.00872478898491611\\
92.42	0.0087265065532844\\
92.43	0.00872822446003081\\
92.44	0.00872994270496516\\
92.45	0.0087316612878975\\
92.46	0.00873338020863817\\
92.47	0.00873509946699776\\
92.48	0.00873681906278711\\
92.49	0.00873853899581736\\
92.5	0.00874025926589988\\
92.51	0.00874197987284634\\
92.52	0.00874370081646866\\
92.53	0.00874542209657904\\
92.54	0.00874714371298997\\
92.55	0.00874886566551418\\
92.56	0.00875058795396471\\
92.57	0.00875231057815486\\
92.58	0.00875403353789823\\
92.59	0.00875575683300869\\
92.6	0.00875748046330039\\
92.61	0.00875920442858777\\
92.62	0.00876092872868557\\
92.63	0.00876265336340882\\
92.64	0.00876437833257281\\
92.65	0.00876610363599317\\
92.66	0.00876782927348579\\
92.67	0.00876955524486686\\
92.68	0.00877128154995289\\
92.69	0.00877300818856067\\
92.7	0.00877473516050729\\
92.71	0.00877646246561016\\
92.72	0.00877819010368698\\
92.73	0.00877991807455578\\
92.74	0.00878164637803485\\
92.75	0.00878337501394285\\
92.76	0.0087851039820987\\
92.77	0.00878683328232167\\
92.78	0.00878856291443133\\
92.79	0.00879029287824756\\
92.8	0.00879202317359057\\
92.81	0.0087937538002809\\
92.82	0.00879548475813937\\
92.83	0.00879721604698718\\
92.84	0.0087989476666458\\
92.85	0.00880067961693707\\
92.86	0.00880241189768313\\
92.87	0.00880414450870647\\
92.88	0.0088058774498299\\
92.89	0.00880761072087655\\
92.9	0.00880934432166993\\
92.91	0.00881107825203382\\
92.92	0.00881281251179239\\
92.93	0.00881454710077015\\
92.94	0.0088162820187919\\
92.95	0.00881801726568284\\
92.96	0.00881975284126848\\
92.97	0.0088214887453747\\
92.98	0.00882322497782769\\
92.99	0.00882496153845404\\
93	0.00882669842708064\\
93.01	0.00882843564353477\\
93.02	0.00883017318764404\\
93.03	0.00883191105923643\\
93.04	0.00883364925814026\\
93.05	0.00883538778418424\\
93.06	0.00883712663719739\\
93.07	0.00883886581700915\\
93.08	0.00884060532344926\\
93.09	0.00884234515634789\\
93.1	0.00884408531553551\\
93.11	0.00884582580084302\\
93.12	0.00884756661210163\\
93.13	0.00884930774914297\\
93.14	0.00885104921179901\\
93.15	0.00885279099990211\\
93.16	0.008854533113285\\
93.17	0.00885627555178077\\
93.18	0.00885801831522291\\
93.19	0.00885976140344528\\
93.2	0.00886150481628213\\
93.21	0.00886324855356808\\
93.22	0.00886499261513813\\
93.23	0.00886673700082768\\
93.24	0.0088684817104725\\
93.25	0.00887022674390876\\
93.26	0.00887197210097301\\
93.27	0.0088737177815022\\
93.28	0.00887546378533366\\
93.29	0.00887721011230513\\
93.3	0.00887895676225473\\
93.31	0.00888070373502097\\
93.32	0.00888245103044278\\
93.33	0.00888419864835946\\
93.34	0.00888594658861074\\
93.35	0.00888769485103674\\
93.36	0.00888944343547797\\
93.37	0.00889119234177535\\
93.38	0.00889294156977021\\
93.39	0.0088946911193043\\
93.4	0.00889644099021974\\
93.41	0.00889819118235909\\
93.42	0.00889994169556532\\
93.43	0.00890169252968179\\
93.44	0.00890344368455229\\
93.45	0.00890519516002103\\
93.46	0.00890694695593261\\
93.47	0.00890869907213207\\
93.48	0.00891045150846486\\
93.49	0.00891220426477684\\
93.5	0.00891395734091431\\
93.51	0.00891571073672397\\
93.52	0.00891746445205295\\
93.53	0.00891921848674882\\
93.54	0.00892097284065955\\
93.55	0.00892272751363355\\
93.56	0.00892448250551964\\
93.57	0.0089262378161671\\
93.58	0.0089279934454256\\
93.59	0.00892974939314528\\
93.6	0.00893150565917668\\
93.61	0.00893326224337079\\
93.62	0.00893501914557903\\
93.63	0.00893677636565325\\
93.64	0.00893853390344573\\
93.65	0.00894029175880921\\
93.66	0.00894204993159684\\
93.67	0.00894380842166223\\
93.68	0.00894556722885942\\
93.69	0.00894732635304289\\
93.7	0.00894908579406756\\
93.71	0.0089508455517888\\
93.72	0.00895260562606242\\
93.73	0.00895436601674468\\
93.74	0.00895612672369227\\
93.75	0.00895788774676234\\
93.76	0.00895964908581249\\
93.77	0.00896141074070075\\
93.78	0.00896317271128562\\
93.79	0.00896493499742604\\
93.8	0.00896669759898139\\
93.81	0.00896846051581152\\
93.82	0.00897022374777673\\
93.83	0.00897198729473777\\
93.84	0.00897375115655583\\
93.85	0.00897551533309257\\
93.86	0.00897727982421012\\
93.87	0.00897904462977102\\
93.88	0.00898080974963832\\
93.89	0.0089825751836755\\
93.9	0.00898434093174649\\
93.91	0.00898610699371571\\
93.92	0.008987873369448\\
93.93	0.0089896400588087\\
93.94	0.00899140706166358\\
93.95	0.0089931743778789\\
93.96	0.00899494200732135\\
93.97	0.00899670994985811\\
93.98	0.00899847820535681\\
93.99	0.00900024677368555\\
94	0.00900201565471289\\
94.01	0.00900378484830787\\
94.02	0.00900555435433996\\
94.03	0.00900732417267914\\
94.04	0.00900909430319583\\
94.05	0.00901086474576093\\
94.06	0.0090126355002458\\
94.07	0.00901440656652225\\
94.08	0.0090161779444626\\
94.09	0.00901794963393962\\
94.1	0.00901972163482654\\
94.11	0.00902149394699706\\
94.12	0.00902326657032537\\
94.13	0.0090250395046861\\
94.14	0.00902681274995439\\
94.15	0.00902858630600581\\
94.16	0.00903036017271643\\
94.17	0.00903213434996278\\
94.18	0.00903390883762187\\
94.19	0.00903568363557117\\
94.2	0.00903745874368862\\
94.21	0.00903923416185266\\
94.22	0.00904100988994217\\
94.23	0.00904278592783652\\
94.24	0.00904456227541554\\
94.25	0.00904633893255957\\
94.26	0.00904811589914936\\
94.27	0.00904989317506619\\
94.28	0.00905167076019179\\
94.29	0.00905344865440837\\
94.3	0.00905522685759859\\
94.31	0.00905700536964562\\
94.32	0.00905878419043308\\
94.33	0.00906056331984506\\
94.34	0.00906234275776614\\
94.35	0.00906412250408137\\
94.36	0.00906590255867625\\
94.37	0.00906768292143679\\
94.38	0.00906946359224944\\
94.39	0.00907124457100114\\
94.4	0.00907302585757929\\
94.41	0.00907480745187178\\
94.42	0.00907658935376695\\
94.43	0.00907837156315364\\
94.44	0.00908015407992112\\
94.45	0.00908193690395918\\
94.46	0.00908372003515804\\
94.47	0.0090855034734084\\
94.48	0.00908728721860145\\
94.49	0.00908907127062883\\
94.5	0.00909085562938265\\
94.51	0.00909264029475549\\
94.52	0.00909442526664041\\
94.53	0.00909621054493092\\
94.54	0.009097996129521\\
94.55	0.0090997820203051\\
94.56	0.00910156821717814\\
94.57	0.00910335472003551\\
94.58	0.00910514152877305\\
94.59	0.00910692864328706\\
94.6	0.00910871606347432\\
94.61	0.00911050378923207\\
94.62	0.00911229182045799\\
94.63	0.00911408015705026\\
94.64	0.00911586879890749\\
94.65	0.00911765774592875\\
94.66	0.00911944699801358\\
94.67	0.00912123655506197\\
94.68	0.00912302641697438\\
94.69	0.0091248165836517\\
94.7	0.0091266070549953\\
94.71	0.00912839783090699\\
94.72	0.00913018891128904\\
94.73	0.00913198029604417\\
94.74	0.00913377198507554\\
94.75	0.00913556397828678\\
94.76	0.00913735627558195\\
94.77	0.00913914887686556\\
94.78	0.00914094178204258\\
94.79	0.0091427349910184\\
94.8	0.00914452850369889\\
94.81	0.00914632231999033\\
94.82	0.00914811643979945\\
94.83	0.00914991086303343\\
94.84	0.00915170558959988\\
94.85	0.00915350061940684\\
94.86	0.00915529595236281\\
94.87	0.00915709158837669\\
94.88	0.00915888752735784\\
94.89	0.00916068376921605\\
94.9	0.00916248031386152\\
94.91	0.0091642771612049\\
94.92	0.00916607431115725\\
94.93	0.00916787176363007\\
94.94	0.00916966951853527\\
94.95	0.0091714675757852\\
94.96	0.0091732659352926\\
94.97	0.00917506459697067\\
94.98	0.009176863560733\\
94.99	0.00917866282649358\\
95	0.00918046239416685\\
95.01	0.00918226226366763\\
95.02	0.00918406243491118\\
95.03	0.00918586290781314\\
95.04	0.00918766368228957\\
95.05	0.00918946475825692\\
95.06	0.00919126613563207\\
95.07	0.00919306781433226\\
95.08	0.00919486979427517\\
95.09	0.00919667207537883\\
95.1	0.00919847465756172\\
95.11	0.00920027754074264\\
95.12	0.00920208072484084\\
95.13	0.00920388420977593\\
95.14	0.00920568799546791\\
95.15	0.00920749208183716\\
95.16	0.00920929646880444\\
95.17	0.00921110115629089\\
95.18	0.00921290614421801\\
95.19	0.0092147114325077\\
95.2	0.00921651702108221\\
95.21	0.00921832290986416\\
95.22	0.00922012909877654\\
95.23	0.00922193558774271\\
95.24	0.00922374237668637\\
95.25	0.00922554946553159\\
95.26	0.00922735685420278\\
95.27	0.00922916454262474\\
95.28	0.00923097253072258\\
95.29	0.00923278081842176\\
95.3	0.0092345894056481\\
95.31	0.00923639829232776\\
95.32	0.00923820747838722\\
95.33	0.00924001696375331\\
95.34	0.00924182674835319\\
95.35	0.00924363683211434\\
95.36	0.00924544721496457\\
95.37	0.00924725789683202\\
95.38	0.00924906887764515\\
95.39	0.00925088015733271\\
95.4	0.0092526917358238\\
95.41	0.0092545036130478\\
95.42	0.00925631578893443\\
95.43	0.00925812826341366\\
95.44	0.00925994103641582\\
95.45	0.0092617541078715\\
95.46	0.00926356747771158\\
95.47	0.00926538114586726\\
95.48	0.00926719511226999\\
95.49	0.00926900937685153\\
95.5	0.0092708239395439\\
95.51	0.00927263880027942\\
95.52	0.00927445395899064\\
95.53	0.00927626941561041\\
95.54	0.00927808517007184\\
95.55	0.00927990122230831\\
95.56	0.00928171757225341\\
95.57	0.00928353421984105\\
95.58	0.00928535116500533\\
95.59	0.00928716840768062\\
95.6	0.00928898594780154\\
95.61	0.00929080378530293\\
95.62	0.00929262192011986\\
95.63	0.00929444035218764\\
95.64	0.0092962590814418\\
95.65	0.00929807810781809\\
95.66	0.00929989743125247\\
95.67	0.00930171705168111\\
95.68	0.00930353696904041\\
95.69	0.00930535718326693\\
95.7	0.00930717769429747\\
95.71	0.009308998502069\\
95.72	0.00931081960651868\\
95.73	0.00931264100758387\\
95.74	0.00931446270520208\\
95.75	0.00931628469931103\\
95.76	0.00931810698984858\\
95.77	0.00931992957675278\\
95.78	0.00932175245996181\\
95.79	0.00932357563941404\\
95.8	0.00932539911504797\\
95.81	0.00932722288680226\\
95.82	0.00932904695461568\\
95.83	0.00933087131842718\\
95.84	0.0093326959781758\\
95.85	0.00933452093380073\\
95.86	0.00933634618524128\\
95.87	0.00933817173243686\\
95.88	0.009339997575327\\
95.89	0.00934182371385134\\
95.9	0.0093436501479496\\
95.91	0.0093454768775616\\
95.92	0.00934730390262727\\
95.93	0.0093491312230866\\
95.94	0.00935095883887966\\
95.95	0.0093527867499466\\
95.96	0.00935461495622761\\
95.97	0.00935644345766298\\
95.98	0.00935827225419303\\
95.99	0.00936010134575811\\
96	0.00936193073229866\\
96.01	0.00936376041375512\\
96.02	0.00936559039006797\\
96.03	0.00936742066117771\\
96.04	0.00936925122702487\\
96.05	0.00937108208754999\\
96.06	0.00937291324269359\\
96.07	0.00937474469239623\\
96.08	0.00937657643659843\\
96.09	0.00937840847524072\\
96.1	0.0093802408082636\\
96.11	0.00938207343560755\\
96.12	0.009383906357213\\
96.13	0.00938573957302036\\
96.14	0.00938757308296999\\
96.15	0.0093894068870022\\
96.16	0.00939124098505722\\
96.17	0.00939307537707525\\
96.18	0.00939491006299639\\
96.19	0.00939674504276067\\
96.2	0.00939858031630804\\
96.21	0.00940041588357833\\
96.22	0.00940225174451131\\
96.23	0.0094040878990466\\
96.24	0.00940592434712375\\
96.25	0.00940776108868215\\
96.26	0.00940959812366107\\
96.27	0.00941143545199966\\
96.28	0.0094132730736369\\
96.29	0.00941511098851165\\
96.3	0.00941694919656258\\
96.31	0.00941878769772821\\
96.32	0.00942062649194689\\
96.33	0.00942246557915678\\
96.34	0.00942430495929584\\
96.35	0.00942614463230186\\
96.36	0.0094279845981124\\
96.37	0.00942982485666481\\
96.38	0.00943166540789624\\
96.39	0.00943350625174358\\
96.4	0.00943534738814351\\
96.41	0.00943718881703245\\
96.42	0.00943903053834657\\
96.43	0.00944087255202178\\
96.44	0.00944271485799372\\
96.45	0.00944455745619774\\
96.46	0.00944640034656892\\
96.47	0.00944824352904203\\
96.48	0.00945008700355156\\
96.49	0.00945193077003166\\
96.5	0.00945377482841617\\
96.51	0.0094556191786386\\
96.52	0.00945746382063212\\
96.53	0.00945930875432956\\
96.54	0.00946115397966338\\
96.55	0.00946299949656569\\
96.56	0.00946484530496821\\
96.57	0.00946669140480228\\
96.58	0.00946853779599887\\
96.59	0.00947038447848851\\
96.6	0.00947223145220134\\
96.61	0.00947407871706709\\
96.62	0.00947592627301503\\
96.63	0.00947777411997402\\
96.64	0.00947962225787245\\
96.65	0.00948147068663828\\
96.66	0.00948331940619895\\
96.67	0.00948516841648148\\
96.68	0.00948701771741236\\
96.69	0.00948886730891761\\
96.7	0.00949071719092272\\
96.71	0.00949256736335269\\
96.72	0.00949441782613196\\
96.73	0.00949626857918445\\
96.74	0.00949811962243354\\
96.75	0.00949997095580204\\
96.76	0.00950182257921218\\
96.77	0.00950367449258565\\
96.78	0.0095055266958435\\
96.79	0.00950737918890623\\
96.8	0.00950923197169369\\
96.81	0.00951108504412514\\
96.82	0.00951293840611919\\
96.83	0.0095147920575938\\
96.84	0.0095166459984663\\
96.85	0.00951850022865335\\
96.86	0.00952035474807092\\
96.87	0.00952220955663431\\
96.88	0.00952406465425812\\
96.89	0.00952592004085623\\
96.9	0.00952777571634182\\
96.91	0.00952963168062733\\
96.92	0.00953148793362446\\
96.93	0.00953334447524414\\
96.94	0.00953520130539657\\
96.95	0.00953705842399114\\
96.96	0.00953891583093647\\
96.97	0.00954077352614038\\
96.98	0.00954263150950987\\
96.99	0.00954448978095112\\
97	0.00954634834036948\\
97.01	0.00954820718766945\\
97.02	0.00955006632275467\\
97.03	0.00955192574552791\\
97.04	0.00955378545589105\\
97.05	0.0095556454537451\\
97.06	0.00955750573899012\\
97.07	0.00955936631152529\\
97.08	0.00956122717124883\\
97.09	0.00956308831805804\\
97.1	0.00956494975184924\\
97.11	0.00956681147251778\\
97.12	0.00956867347995805\\
97.13	0.00957053577406342\\
97.14	0.00957239835472627\\
97.15	0.00957426122183794\\
97.16	0.00957612437528876\\
97.17	0.00957798781496799\\
97.18	0.00957985154076384\\
97.19	0.00958171555256344\\
97.2	0.00958357985025284\\
97.21	0.009585444433717\\
97.22	0.00958730930283974\\
97.23	0.00958917445750377\\
97.24	0.00959103989759066\\
97.25	0.00959290562298083\\
97.26	0.00959477163355351\\
97.27	0.00959663792918678\\
97.28	0.0095985045097575\\
97.29	0.00960037137514133\\
97.3	0.00960223852521271\\
97.31	0.00960410595984484\\
97.32	0.00960597367890967\\
97.33	0.00960784168227788\\
97.34	0.00960970996981887\\
97.35	0.00961157854140076\\
97.36	0.00961344739689035\\
97.37	0.00961531653615313\\
97.38	0.00961718595905322\\
97.39	0.00961905566545344\\
97.4	0.00962092565521519\\
97.41	0.00962279592819854\\
97.42	0.00962466648426212\\
97.43	0.00962653732326318\\
97.44	0.00962840844505754\\
97.45	0.00963027984949956\\
97.46	0.00963215153644217\\
97.47	0.00963402350573681\\
97.48	0.00963589575723345\\
97.49	0.00963776829078056\\
97.5	0.00963964110622508\\
97.51	0.00964151420341242\\
97.52	0.00964338758218643\\
97.53	0.00964526124238939\\
97.54	0.00964713518386195\\
97.55	0.00964900940644317\\
97.56	0.0096508839099705\\
97.57	0.00965275869427971\\
97.58	0.00965463375920494\\
97.59	0.00965650910457864\\
97.6	0.00965838473023159\\
97.61	0.00966026063599285\\
97.62	0.00966213682168977\\
97.63	0.00966401328714796\\
97.64	0.00966589003220603\\
97.65	0.00966776705676154\\
97.66	0.00966964436070889\\
97.67	0.00967152194393926\\
97.68	0.0096733998063406\\
97.69	0.00967527794779758\\
97.7	0.00967715636819153\\
97.71	0.00967903506740043\\
97.72	0.00968091404529885\\
97.73	0.00968279330175791\\
97.74	0.00968467283664523\\
97.75	0.00968655264982492\\
97.76	0.00968843274234247\\
97.77	0.00969031311591114\\
97.78	0.00969219377226286\\
97.79	0.00969407471314843\\
97.8	0.00969595589427488\\
97.81	0.00969783706790343\\
97.82	0.00969971823671616\\
97.83	0.0097015994034335\\
97.84	0.00970348057081461\\
97.85	0.00970536174165778\\
97.86	0.0097072429188009\\
97.87	0.00970912410512182\\
97.88	0.00971100530353872\\
97.89	0.00971288651701043\\
97.9	0.0097147612775502\\
97.91	0.00971662898349836\\
97.92	0.00971848956704936\\
97.93	0.00972034295969377\\
97.94	0.00972218909220971\\
97.95	0.00972402789465418\\
97.96	0.00972585929658671\\
97.97	0.00972768322722907\\
97.98	0.00972949961505148\\
97.99	0.0097313083878552\\
98	0.00973310947272936\\
98.01	0.00973490290657201\\
98.02	0.00973668877760677\\
98.03	0.00973846701010021\\
98.04	0.00974023752749241\\
98.05	0.00974200025238653\\
98.06	0.00974375510653817\\
98.07	0.00974550006656619\\
98.08	0.00974723896368855\\
98.09	0.00974897317730456\\
98.1	0.00975070268116027\\
98.11	0.00975242744883675\\
98.12	0.0097541474537496\\
98.13	0.0097558626691484\\
98.14	0.00975757306811625\\
98.15	0.00975927862356923\\
98.16	0.00976097930825596\\
98.17	0.00976267509475716\\
98.18	0.00976436595548521\\
98.19	0.00976605225165897\\
98.2	0.00976773404885576\\
98.21	0.00976941170560958\\
98.22	0.00977108694750106\\
98.23	0.0097727597627094\\
98.24	0.00977443013941011\\
98.25	0.00977609806577754\\
98.26	0.00977776352998746\\
98.27	0.00977942652021976\\
98.28	0.00978108702466129\\
98.29	0.00978274503150872\\
98.3	0.00978440052863336\\
98.31	0.00978605350385307\\
98.32	0.00978770394500337\\
98.33	0.00978935183921957\\
98.34	0.0097909971659391\\
98.35	0.00979263990441897\\
98.36	0.00979428003373453\\
98.37	0.00979591753233545\\
98.38	0.00979755237846595\\
98.39	0.00979918455017923\\
98.4	0.00980081402533612\\
98.41	0.00980244078160381\\
98.42	0.00980406479645448\\
98.43	0.00980568604504303\\
98.44	0.0098073045016238\\
98.45	0.00980892014019995\\
98.46	0.00981053293452088\\
98.47	0.0098121428580794\\
98.48	0.00981374988410908\\
98.49	0.00981535398558175\\
98.5	0.00981695513520473\\
98.51	0.00981855330541826\\
98.52	0.00982014846839274\\
98.53	0.00982174059602603\\
98.54	0.00982332966000768\\
98.55	0.00982491563234843\\
98.56	0.00982649848478402\\
98.57	0.00982807818877242\\
98.58	0.00982965471549099\\
98.59	0.00983122803583362\\
98.6	0.00983279812040779\\
98.61	0.00983436493953169\\
98.62	0.0098359284632312\\
98.63	0.00983748866123693\\
98.64	0.00983904550298116\\
98.65	0.00984059895759477\\
98.66	0.00984214899348074\\
98.67	0.00984369557858251\\
98.68	0.00984523868052557\\
98.69	0.00984677826661434\\
98.7	0.0098483143038289\\
98.71	0.00984984675882157\\
98.72	0.00985137559791348\\
98.73	0.00985290078709106\\
98.74	0.00985442229200414\\
98.75	0.00985594007797\\
98.76	0.00985745410997008\\
98.77	0.00985896435264673\\
98.78	0.00986047077029981\\
98.79	0.00986197332688554\\
98.8	0.0098634719860138\\
98.81	0.00986496671094474\\
98.82	0.00986645746458541\\
98.83	0.00986794420948751\\
98.84	0.00986942690769653\\
98.85	0.00987090552071792\\
98.86	0.00987238000969395\\
98.87	0.00987385033540023\\
98.88	0.00987531645824225\\
98.89	0.00987677833825183\\
98.9	0.00987823593508357\\
98.91	0.00987968920801123\\
98.92	0.00988113811592412\\
98.93	0.00988258261732343\\
98.94	0.00988402267031852\\
98.95	0.00988545823262319\\
98.96	0.0098868892615519\\
98.97	0.00988831571401599\\
98.98	0.00988973754651977\\
98.99	0.00989115471515673\\
99	0.00989256717560554\\
99.01	0.00989397488312613\\
99.02	0.00989537779255572\\
99.03	0.00989677585830473\\
99.04	0.00989816903435277\\
99.05	0.00989955727424447\\
99.06	0.00990094053108541\\
99.07	0.00990231875753784\\
99.08	0.00990369190581653\\
99.09	0.00990505992768447\\
99.1	0.00990642277444855\\
99.11	0.00990778039695523\\
99.12	0.00990913274558614\\
99.13	0.00991047977025362\\
99.14	0.0099118214203963\\
99.15	0.00991315764497448\\
99.16	0.00991448839246565\\
99.17	0.00991581361085984\\
99.18	0.00991713324765496\\
99.19	0.00991844724985212\\
99.2	0.00991975556395086\\
99.21	0.00992105813594439\\
99.22	0.00992235491131472\\
99.23	0.0099236458350278\\
99.24	0.0099249308515286\\
99.25	0.00992620990473608\\
99.26	0.00992748293803825\\
99.27	0.00992874989428702\\
99.28	0.00993001071579311\\
99.29	0.00993126534432089\\
99.3	0.00993251372108315\\
99.31	0.00993375578673581\\
99.32	0.00993499148137264\\
99.33	0.00993622074451984\\
99.34	0.00993744351513067\\
99.35	0.00993865973157995\\
99.36	0.00993986933165849\\
99.37	0.0099410722525676\\
99.38	0.00994226843091338\\
99.39	0.00994345780270105\\
99.4	0.00994464030332924\\
99.41	0.00994581586758415\\
99.42	0.00994698442963372\\
99.43	0.00994814592302171\\
99.44	0.00994930028066174\\
99.45	0.00995044743483127\\
99.46	0.00995158731716552\\
99.47	0.0099527198586513\\
99.48	0.00995384498962087\\
99.49	0.00995496263974562\\
99.5	0.0099560727380298\\
99.51	0.00995717521280411\\
99.52	0.00995826999171929\\
99.53	0.00995935700173958\\
99.54	0.0099604361691362\\
99.55	0.00996150741948069\\
99.56	0.00996257067763823\\
99.57	0.00996362586776092\\
99.58	0.00996467291328088\\
99.59	0.00996571173690344\\
99.6	0.00996674226060019\\
99.61	0.00996776439644508\\
99.62	0.00996877805189459\\
99.63	0.00996978313349925\\
99.64	0.0099707795468949\\
99.65	0.00997176719679398\\
99.66	0.00997274598697662\\
99.67	0.00997371582028172\\
99.68	0.00997467659859792\\
99.69	0.00997562822285446\\
99.7	0.00997657059301199\\
99.71	0.0099775036080533\\
99.72	0.00997842716597387\\
99.73	0.00997934116377243\\
99.74	0.0099802454974414\\
99.75	0.00998114006195719\\
99.76	0.00998202475127047\\
99.77	0.00998289945829632\\
99.78	0.00998376407490432\\
99.79	0.00998461849190844\\
99.8	0.00998546259905699\\
99.81	0.00998629628502236\\
99.82	0.00998711943739068\\
99.83	0.00998793194265142\\
99.84	0.00998873368618689\\
99.85	0.00998952455226156\\
99.86	0.0099903044240114\\
99.87	0.00999107318343302\\
99.88	0.00999183071137277\\
99.89	0.00999257688751569\\
99.9	0.00999331159037439\\
99.91	0.00999403469727783\\
99.92	0.00999474608435993\\
99.93	0.00999544562654817\\
99.94	0.00999613319755202\\
99.95	0.00999680866985125\\
99.96	0.0099974719146842\\
99.97	0.00999812280203583\\
99.98	0.00999876120062581\\
99.99	0.00999938697789635\\
100	0.01\\
};
\addlegendentry{$q=3$};

\addplot [color=green,solid,forget plot]
  table[row sep=crcr]{%
0.01	0.00157538227128272\\
0.02	0.00157555783429435\\
0.03	0.00157573346993446\\
0.04	0.00157590917823627\\
0.05	0.00157608495923298\\
0.06	0.00157626081295787\\
0.07	0.0015764367394442\\
0.08	0.00157661273872527\\
0.09	0.0015767888108344\\
0.1	0.00157696495580494\\
0.11	0.00157714117367026\\
0.12	0.00157731746446374\\
0.13	0.00157749382821881\\
0.14	0.0015776702649689\\
0.15	0.00157784677474749\\
0.16	0.00157802335758805\\
0.17	0.00157820001352409\\
0.18	0.00157837674258916\\
0.19	0.0015785535448168\\
0.2	0.00157873042024061\\
0.21	0.00157890736889418\\
0.22	0.00157908439081116\\
0.23	0.00157926148602518\\
0.24	0.00157943865456993\\
0.25	0.00157961589647911\\
0.26	0.00157979321178646\\
0.27	0.00157997060052571\\
0.28	0.00158014806273064\\
0.29	0.00158032559843506\\
0.3	0.00158050320767277\\
0.31	0.00158068089047764\\
0.32	0.00158085864688352\\
0.33	0.00158103647692431\\
0.34	0.00158121438063394\\
0.35	0.00158139235804634\\
0.36	0.00158157040919549\\
0.37	0.00158174853411537\\
0.38	0.00158192673284\\
0.39	0.00158210500540343\\
0.4	0.00158228335183971\\
0.41	0.00158246177218294\\
0.42	0.00158264026646722\\
0.43	0.0015828188347267\\
0.44	0.00158299747699555\\
0.45	0.00158317619330794\\
0.46	0.0015833549836981\\
0.47	0.00158353384820025\\
0.48	0.00158371278684866\\
0.49	0.00158389179967762\\
0.5	0.00158407088672143\\
0.51	0.00158425004801442\\
0.52	0.00158442928359097\\
0.53	0.00158460859348545\\
0.54	0.00158478797773227\\
0.55	0.00158496743636588\\
0.56	0.00158514696942073\\
0.57	0.00158532657693129\\
0.58	0.00158550625893209\\
0.59	0.00158568601545765\\
0.6	0.00158586584654254\\
0.61	0.00158604575222133\\
0.62	0.00158622573252864\\
0.63	0.0015864057874991\\
0.64	0.00158658591716737\\
0.65	0.00158676612156813\\
0.66	0.0015869464007361\\
0.67	0.00158712675470601\\
0.68	0.00158730718351261\\
0.69	0.00158748768719069\\
0.7	0.00158766826577506\\
0.71	0.00158784891930056\\
0.72	0.00158802964780205\\
0.73	0.00158821045131442\\
0.74	0.00158839132987256\\
0.75	0.00158857228351143\\
0.76	0.00158875331226598\\
0.77	0.00158893441617119\\
0.78	0.0015891155952621\\
0.79	0.00158929684957372\\
0.8	0.00158947817914113\\
0.81	0.00158965958399942\\
0.82	0.00158984106418369\\
0.83	0.0015900226197291\\
0.84	0.00159020425067079\\
0.85	0.00159038595704398\\
0.86	0.00159056773888387\\
0.87	0.00159074959622571\\
0.88	0.00159093152910476\\
0.89	0.00159111353755633\\
0.9	0.00159129562161573\\
0.91	0.00159147778131831\\
0.92	0.00159166001669944\\
0.93	0.0015918423277945\\
0.94	0.00159202471463894\\
0.95	0.00159220717726821\\
0.96	0.00159238971571776\\
0.97	0.00159257233002311\\
0.98	0.00159275502021979\\
0.99	0.00159293778634333\\
1	0.00159312062842934\\
1.01	0.00159330354651341\\
1.02	0.00159348654063117\\
1.03	0.00159366961081828\\
1.04	0.00159385275711042\\
1.05	0.0015940359795433\\
1.06	0.00159421927815266\\
1.07	0.00159440265297427\\
1.08	0.00159458610404389\\
1.09	0.00159476963139737\\
1.1	0.00159495323507052\\
1.11	0.00159513691509923\\
1.12	0.00159532067151938\\
1.13	0.00159550450436689\\
1.14	0.00159568841367771\\
1.15	0.00159587239948782\\
1.16	0.0015960564618332\\
1.17	0.00159624060074988\\
1.18	0.00159642481627392\\
1.19	0.0015966091084414\\
1.2	0.00159679347728842\\
1.21	0.0015969779228511\\
1.22	0.00159716244516561\\
1.23	0.00159734704426814\\
1.24	0.00159753172019488\\
1.25	0.00159771647298209\\
1.26	0.00159790130266602\\
1.27	0.00159808620928296\\
1.28	0.00159827119286923\\
1.29	0.00159845625346118\\
1.3	0.00159864139109517\\
1.31	0.00159882660580761\\
1.32	0.00159901189763492\\
1.33	0.00159919726661353\\
1.34	0.00159938271277995\\
1.35	0.00159956823617067\\
1.36	0.00159975383682222\\
1.37	0.00159993951477116\\
1.38	0.00160012527005407\\
1.39	0.00160031110270757\\
1.4	0.00160049701276829\\
1.41	0.00160068300027291\\
1.42	0.0016008690652581\\
1.43	0.0016010552077606\\
1.44	0.00160124142781716\\
1.45	0.00160142772546454\\
1.46	0.00160161410073954\\
1.47	0.001601800553679\\
1.48	0.00160198708431977\\
1.49	0.00160217369269872\\
1.5	0.00160236037885278\\
1.51	0.00160254714281887\\
1.52	0.00160273398463397\\
1.53	0.00160292090433505\\
1.54	0.00160310790195915\\
1.55	0.00160329497754329\\
1.56	0.00160348213112457\\
1.57	0.00160366936274007\\
1.58	0.00160385667242692\\
1.59	0.00160404406022228\\
1.6	0.00160423152616333\\
1.61	0.00160441907028728\\
1.62	0.00160460669263136\\
1.63	0.00160479439323284\\
1.64	0.00160498217212901\\
1.65	0.00160517002935719\\
1.66	0.00160535796495472\\
1.67	0.00160554597895897\\
1.68	0.00160573407140735\\
1.69	0.00160592224233728\\
1.7	0.00160611049178623\\
1.71	0.00160629881979167\\
1.72	0.00160648722639112\\
1.73	0.0016066757116221\\
1.74	0.0016068642755222\\
1.75	0.001607052918129\\
1.76	0.00160724163948012\\
1.77	0.00160743043961321\\
1.78	0.00160761931856594\\
1.79	0.00160780827637603\\
1.8	0.00160799731308119\\
1.81	0.0016081864287192\\
1.82	0.00160837562332784\\
1.83	0.00160856489694491\\
1.84	0.00160875424960828\\
1.85	0.00160894368135579\\
1.86	0.00160913319222535\\
1.87	0.00160932278225488\\
1.88	0.00160951245148235\\
1.89	0.00160970219994572\\
1.9	0.001609892027683\\
1.91	0.00161008193473224\\
1.92	0.0016102719211315\\
1.93	0.00161046198691887\\
1.94	0.00161065213213245\\
1.95	0.00161084235681042\\
1.96	0.00161103266099093\\
1.97	0.0016112230447122\\
1.98	0.00161141350801244\\
1.99	0.00161160405092992\\
2	0.00161179467350289\\
2.01	0.00161198537576968\\
2.02	0.00161217615776861\\
2.03	0.00161236701953803\\
2.04	0.00161255796111631\\
2.05	0.00161274898254187\\
2.06	0.00161294008385313\\
2.07	0.00161313126508855\\
2.08	0.00161332252628661\\
2.09	0.00161351386748582\\
2.1	0.0016137052887247\\
2.11	0.00161389679004181\\
2.12	0.00161408837147573\\
2.13	0.00161428003306508\\
2.14	0.00161447177484848\\
2.15	0.0016146635968646\\
2.16	0.00161485549915212\\
2.17	0.00161504748174975\\
2.18	0.00161523954469623\\
2.19	0.00161543168803032\\
2.2	0.00161562391179081\\
2.21	0.0016158162160165\\
2.22	0.00161600860074625\\
2.23	0.0016162010660189\\
2.24	0.00161639361187336\\
2.25	0.00161658623834854\\
2.26	0.00161677894548338\\
2.27	0.00161697173331685\\
2.28	0.00161716460188794\\
2.29	0.00161735755123567\\
2.3	0.0016175505813991\\
2.31	0.00161774369241728\\
2.32	0.00161793688432932\\
2.33	0.00161813015717434\\
2.34	0.00161832351099149\\
2.35	0.00161851694581994\\
2.36	0.0016187104616989\\
2.37	0.0016189040586676\\
2.38	0.00161909773676527\\
2.39	0.00161929149603122\\
2.4	0.00161948533650473\\
2.41	0.00161967925822515\\
2.42	0.00161987326123182\\
2.43	0.00162006734556414\\
2.44	0.0016202615112615\\
2.45	0.00162045575836336\\
2.46	0.00162065008690917\\
2.47	0.00162084449693842\\
2.48	0.00162103898849061\\
2.49	0.0016212335616053\\
2.5	0.00162142821632206\\
2.51	0.00162162295268047\\
2.52	0.00162181777072015\\
2.53	0.00162201267048075\\
2.54	0.00162220765200194\\
2.55	0.00162240271532342\\
2.56	0.00162259786048492\\
2.57	0.00162279308752618\\
2.58	0.00162298839648698\\
2.59	0.00162318378740713\\
2.6	0.00162337926032646\\
2.61	0.00162357481528483\\
2.62	0.00162377045232211\\
2.63	0.00162396617147821\\
2.64	0.00162416197279308\\
2.65	0.00162435785630668\\
2.66	0.00162455382205899\\
2.67	0.00162474987009003\\
2.68	0.00162494600043984\\
2.69	0.00162514221314849\\
2.7	0.00162533850825607\\
2.71	0.0016255348858027\\
2.72	0.00162573134582853\\
2.73	0.00162592788837373\\
2.74	0.00162612451347851\\
2.75	0.00162632122118309\\
2.76	0.00162651801152771\\
2.77	0.00162671488455266\\
2.78	0.00162691184029824\\
2.79	0.00162710887880479\\
2.8	0.00162730600011266\\
2.81	0.00162750320426224\\
2.82	0.00162770049129394\\
2.83	0.00162789786124819\\
2.84	0.00162809531416547\\
2.85	0.00162829285008627\\
2.86	0.00162849046905109\\
2.87	0.00162868817110049\\
2.88	0.00162888595627504\\
2.89	0.00162908382461534\\
2.9	0.00162928177616201\\
2.91	0.0016294798109557\\
2.92	0.0016296779290371\\
2.93	0.00162987613044691\\
2.94	0.00163007441522586\\
2.95	0.00163027278341471\\
2.96	0.00163047123505425\\
2.97	0.00163066977018529\\
2.98	0.00163086838884868\\
2.99	0.00163106709108529\\
3	0.001631265876936\\
3.01	0.00163146474644173\\
3.02	0.00163166369964345\\
3.03	0.00163186273658213\\
3.04	0.00163206185729876\\
3.05	0.00163226106183438\\
3.06	0.00163246035023006\\
3.07	0.00163265972252687\\
3.08	0.00163285917876592\\
3.09	0.00163305871898836\\
3.1	0.00163325834323536\\
3.11	0.00163345805154811\\
3.12	0.00163365784396784\\
3.13	0.00163385772053579\\
3.14	0.00163405768129323\\
3.15	0.00163425772628149\\
3.16	0.00163445785554188\\
3.17	0.00163465806911577\\
3.18	0.00163485836704455\\
3.19	0.00163505874936963\\
3.2	0.00163525921613245\\
3.21	0.00163545976737449\\
3.22	0.00163566040313724\\
3.23	0.00163586112346224\\
3.24	0.00163606192839102\\
3.25	0.00163626281796518\\
3.26	0.00163646379222632\\
3.27	0.00163666485121609\\
3.28	0.00163686599497615\\
3.29	0.00163706722354819\\
3.3	0.00163726853697392\\
3.31	0.00163746993529511\\
3.32	0.00163767141855353\\
3.33	0.00163787298679097\\
3.34	0.00163807464004929\\
3.35	0.00163827637837033\\
3.36	0.00163847820179599\\
3.37	0.00163868011036818\\
3.38	0.00163888210412886\\
3.39	0.00163908418311999\\
3.4	0.00163928634738358\\
3.41	0.00163948859696165\\
3.42	0.00163969093189626\\
3.43	0.00163989335222951\\
3.44	0.0016400958580035\\
3.45	0.00164029844926039\\
3.46	0.00164050112604234\\
3.47	0.00164070388839156\\
3.48	0.00164090673635026\\
3.49	0.00164110966996072\\
3.5	0.00164131268926521\\
3.51	0.00164151579430605\\
3.52	0.00164171898512559\\
3.53	0.0016419222617662\\
3.54	0.00164212562427026\\
3.55	0.00164232907268023\\
3.56	0.00164253260703855\\
3.57	0.0016427362273877\\
3.58	0.00164293993377022\\
3.59	0.00164314372622863\\
3.6	0.00164334760480551\\
3.61	0.00164355156954346\\
3.62	0.00164375562048512\\
3.63	0.00164395975767314\\
3.64	0.00164416398115021\\
3.65	0.00164436829095906\\
3.66	0.00164457268714242\\
3.67	0.00164477716974307\\
3.68	0.00164498173880381\\
3.69	0.00164518639436748\\
3.7	0.00164539113647695\\
3.71	0.0016455959651751\\
3.72	0.00164580088050485\\
3.73	0.00164600588250916\\
3.74	0.001646210971231\\
3.75	0.00164641614671338\\
3.76	0.00164662140899935\\
3.77	0.00164682675813196\\
3.78	0.00164703219415432\\
3.79	0.00164723771710956\\
3.8	0.00164744332704081\\
3.81	0.00164764902399128\\
3.82	0.00164785480800417\\
3.83	0.00164806067912274\\
3.84	0.00164826663739025\\
3.85	0.00164847268285\\
3.86	0.00164867881554534\\
3.87	0.00164888503551962\\
3.88	0.00164909134281623\\
3.89	0.0016492977374786\\
3.9	0.00164950421955018\\
3.91	0.00164971078907444\\
3.92	0.00164991744609491\\
3.93	0.00165012419065512\\
3.94	0.00165033102279864\\
3.95	0.00165053794256906\\
3.96	0.00165074495001004\\
3.97	0.00165095204516522\\
3.98	0.00165115922807828\\
3.99	0.00165136649879297\\
4	0.00165157385735303\\
4.01	0.00165178130380222\\
4.02	0.00165198883818438\\
4.03	0.00165219646054333\\
4.04	0.00165240417092295\\
4.05	0.00165261196936713\\
4.06	0.00165281985591983\\
4.07	0.00165302783062498\\
4.08	0.00165323589352659\\
4.09	0.00165344404466868\\
4.1	0.00165365228409529\\
4.11	0.00165386061185053\\
4.12	0.00165406902797849\\
4.13	0.00165427753252332\\
4.14	0.0016544861255292\\
4.15	0.00165469480704034\\
4.16	0.00165490357710096\\
4.17	0.00165511243575533\\
4.18	0.00165532138304775\\
4.19	0.00165553041902256\\
4.2	0.0016557395437241\\
4.21	0.00165594875719676\\
4.22	0.00165615805948497\\
4.23	0.00165636745063318\\
4.24	0.00165657693068586\\
4.25	0.00165678649968753\\
4.26	0.00165699615768274\\
4.27	0.00165720590471606\\
4.28	0.00165741574083209\\
4.29	0.00165762566607546\\
4.3	0.00165783568049085\\
4.31	0.00165804578412296\\
4.32	0.00165825597701651\\
4.33	0.00165846625921626\\
4.34	0.00165867663076701\\
4.35	0.00165888709171357\\
4.36	0.00165909764210081\\
4.37	0.0016593082819736\\
4.38	0.00165951901137687\\
4.39	0.00165972983035555\\
4.4	0.00165994073895464\\
4.41	0.00166015173721914\\
4.42	0.00166036282519409\\
4.43	0.00166057400292456\\
4.44	0.00166078527045567\\
4.45	0.00166099662783254\\
4.46	0.00166120807510035\\
4.47	0.00166141961230429\\
4.48	0.0016616312394896\\
4.49	0.00166184295670153\\
4.5	0.00166205476398539\\
4.51	0.0016622666613865\\
4.52	0.00166247864895022\\
4.53	0.00166269072672194\\
4.54	0.00166290289474708\\
4.55	0.00166311515307108\\
4.56	0.00166332750173945\\
4.57	0.0016635399407977\\
4.58	0.00166375247029137\\
4.59	0.00166396509026605\\
4.6	0.00166417780076734\\
4.61	0.00166439060184091\\
4.62	0.00166460349353241\\
4.63	0.00166481647588757\\
4.64	0.00166502954895213\\
4.65	0.00166524271277185\\
4.66	0.00166545596739256\\
4.67	0.00166566931286007\\
4.68	0.00166588274922028\\
4.69	0.00166609627651907\\
4.7	0.00166630989480239\\
4.71	0.0016665236041162\\
4.72	0.0016667374045065\\
4.73	0.00166695129601933\\
4.74	0.00166716527870076\\
4.75	0.00166737935259688\\
4.76	0.00166759351775382\\
4.77	0.00166780777421775\\
4.78	0.00166802212203485\\
4.79	0.00166823656125137\\
4.8	0.00166845109191356\\
4.81	0.00166866571406772\\
4.82	0.00166888042776017\\
4.83	0.00166909523303728\\
4.84	0.00166931012994544\\
4.85	0.00166952511853106\\
4.86	0.00166974019884063\\
4.87	0.00166995537092061\\
4.88	0.00167017063481755\\
4.89	0.00167038599057799\\
4.9	0.00167060143824853\\
4.91	0.00167081697787579\\
4.92	0.00167103260950643\\
4.93	0.00167124833318715\\
4.94	0.00167146414896466\\
4.95	0.00167168005688573\\
4.96	0.00167189605699714\\
4.97	0.00167211214934571\\
4.98	0.0016723283339783\\
4.99	0.00167254461094182\\
5	0.00167276098028317\\
5.01	0.00167297744204932\\
5.02	0.00167319399628726\\
5.03	0.00167341064304401\\
5.04	0.00167362738236663\\
5.05	0.00167384421430222\\
5.06	0.00167406113889789\\
5.07	0.00167427815620082\\
5.08	0.00167449526625819\\
5.09	0.00167471246911723\\
5.1	0.00167492976482519\\
5.11	0.00167514715342939\\
5.12	0.00167536463497714\\
5.13	0.00167558220951581\\
5.14	0.00167579987709279\\
5.15	0.00167601763775552\\
5.16	0.00167623549155146\\
5.17	0.00167645343852811\\
5.18	0.001676671478733\\
5.19	0.0016768896122137\\
5.2	0.00167710783901782\\
5.21	0.00167732615919298\\
5.22	0.00167754457278686\\
5.23	0.00167776307984716\\
5.24	0.00167798168042163\\
5.25	0.00167820037455803\\
5.26	0.00167841916230417\\
5.27	0.0016786380437079\\
5.28	0.00167885701881709\\
5.29	0.00167907608767966\\
5.3	0.00167929525034354\\
5.31	0.00167951450685672\\
5.32	0.00167973385726722\\
5.33	0.00167995330162308\\
5.34	0.00168017283997239\\
5.35	0.00168039247236328\\
5.36	0.00168061219884388\\
5.37	0.0016808320194624\\
5.38	0.00168105193426706\\
5.39	0.00168127194330612\\
5.4	0.00168149204662787\\
5.41	0.00168171224428064\\
5.42	0.0016819325363128\\
5.43	0.00168215292277274\\
5.44	0.0016823734037089\\
5.45	0.00168259397916975\\
5.46	0.00168281464920379\\
5.47	0.00168303541385957\\
5.48	0.00168325627318566\\
5.49	0.00168347722723067\\
5.5	0.00168369827604326\\
5.51	0.00168391941967209\\
5.52	0.00168414065816588\\
5.53	0.0016843619915734\\
5.54	0.00168458341994343\\
5.55	0.00168480494332479\\
5.56	0.00168502656176634\\
5.57	0.00168524827531698\\
5.58	0.00168547008402564\\
5.59	0.00168569198794129\\
5.6	0.00168591398711292\\
5.61	0.00168613608158959\\
5.62	0.00168635827142035\\
5.63	0.00168658055665433\\
5.64	0.00168680293734066\\
5.65	0.00168702541352854\\
5.66	0.00168724798526718\\
5.67	0.00168747065260583\\
5.68	0.00168769341559378\\
5.69	0.00168791627428036\\
5.7	0.00168813922871493\\
5.71	0.00168836227894691\\
5.72	0.0016885854250257\\
5.73	0.0016888086670008\\
5.74	0.00168903200492171\\
5.75	0.00168925543883797\\
5.76	0.00168947896879916\\
5.77	0.0016897025948549\\
5.78	0.00168992631705486\\
5.79	0.0016901501354487\\
5.8	0.00169037405008617\\
5.81	0.00169059806101703\\
5.82	0.00169082216829108\\
5.83	0.00169104637195816\\
5.84	0.00169127067206814\\
5.85	0.00169149506867092\\
5.86	0.00169171956181647\\
5.87	0.00169194415155476\\
5.88	0.00169216883793582\\
5.89	0.0016923936210097\\
5.9	0.00169261850082651\\
5.91	0.00169284347743636\\
5.92	0.00169306855088944\\
5.93	0.00169329372123596\\
5.94	0.00169351898852615\\
5.95	0.0016937443528103\\
5.96	0.00169396981413873\\
5.97	0.00169419537256178\\
5.98	0.00169442102812987\\
5.99	0.00169464678089341\\
6	0.00169487263090289\\
6.01	0.00169509857820879\\
6.02	0.00169532462286168\\
6.03	0.00169555076491212\\
6.04	0.00169577700441075\\
6.05	0.0016960033414082\\
6.06	0.00169622977595519\\
6.07	0.00169645630810245\\
6.08	0.00169668293790073\\
6.09	0.00169690966540086\\
6.1	0.00169713649065367\\
6.11	0.00169736341371005\\
6.12	0.00169759043462092\\
6.13	0.00169781755343725\\
6.14	0.00169804477021003\\
6.15	0.0016982720849903\\
6.16	0.00169849949782913\\
6.17	0.00169872700877764\\
6.18	0.00169895461788697\\
6.19	0.00169918232520831\\
6.2	0.0016994101307929\\
6.21	0.00169963803469199\\
6.22	0.0016998660369569\\
6.23	0.00170009413763896\\
6.24	0.00170032233678955\\
6.25	0.0017005506344601\\
6.26	0.00170077903070207\\
6.27	0.00170100752556696\\
6.28	0.00170123611910629\\
6.29	0.00170146481137165\\
6.3	0.00170169360241464\\
6.31	0.00170192249228693\\
6.32	0.00170215148104019\\
6.33	0.00170238056872617\\
6.34	0.00170260975539663\\
6.35	0.00170283904110339\\
6.36	0.00170306842589828\\
6.37	0.00170329790983319\\
6.38	0.00170352749296006\\
6.39	0.00170375717533084\\
6.4	0.00170398695699754\\
6.41	0.0017042168380122\\
6.42	0.00170444681842691\\
6.43	0.00170467689829379\\
6.44	0.001704907077665\\
6.45	0.00170513735659274\\
6.46	0.00170536773512925\\
6.47	0.00170559821332682\\
6.48	0.00170582879123776\\
6.49	0.00170605946891443\\
6.5	0.00170629024640924\\
6.51	0.00170652112377462\\
6.52	0.00170675210106305\\
6.53	0.00170698317832705\\
6.54	0.00170721435561918\\
6.55	0.00170744563299204\\
6.56	0.00170767701049826\\
6.57	0.00170790848819053\\
6.58	0.00170814006612157\\
6.59	0.00170837174434414\\
6.6	0.00170860352291102\\
6.61	0.00170883540187508\\
6.62	0.00170906738128918\\
6.63	0.00170929946120624\\
6.64	0.00170953164167924\\
6.65	0.00170976392276116\\
6.66	0.00170999630450504\\
6.67	0.00171022878696398\\
6.68	0.0017104613701911\\
6.69	0.00171069405423955\\
6.7	0.00171092683916255\\
6.71	0.00171115972501333\\
6.72	0.00171139271184519\\
6.73	0.00171162579971144\\
6.74	0.00171185898866547\\
6.75	0.00171209227876067\\
6.76	0.0017123256700505\\
6.77	0.00171255916258845\\
6.78	0.00171279275642804\\
6.79	0.00171302645162285\\
6.8	0.0017132602482265\\
6.81	0.00171349414629263\\
6.82	0.00171372814587496\\
6.83	0.0017139622470272\\
6.84	0.00171419644980315\\
6.85	0.00171443075425663\\
6.86	0.00171466516044149\\
6.87	0.00171489966841164\\
6.88	0.00171513427822102\\
6.89	0.00171536898992363\\
6.9	0.00171560380357349\\
6.91	0.00171583871922466\\
6.92	0.00171607373693127\\
6.93	0.00171630885674746\\
6.94	0.00171654407872744\\
6.95	0.00171677940292544\\
6.96	0.00171701482939574\\
6.97	0.00171725035819265\\
6.98	0.00171748598937056\\
6.99	0.00171772172298384\\
7	0.00171795755908698\\
7.01	0.00171819349773444\\
7.02	0.00171842953898076\\
7.03	0.00171866568288051\\
7.04	0.00171890192948832\\
7.05	0.00171913827885884\\
7.06	0.00171937473104678\\
7.07	0.00171961128610688\\
7.08	0.00171984794409392\\
7.09	0.00172008470506275\\
7.1	0.00172032156906822\\
7.11	0.00172055853616526\\
7.12	0.00172079560640883\\
7.13	0.00172103277985392\\
7.14	0.00172127005655558\\
7.15	0.0017215074365689\\
7.16	0.00172174491994901\\
7.17	0.00172198250675108\\
7.18	0.00172222019703032\\
7.19	0.001722457990842\\
7.2	0.00172269588824141\\
7.21	0.00172293388928391\\
7.22	0.00172317199402489\\
7.23	0.00172341020251977\\
7.24	0.00172364851482403\\
7.25	0.0017238869309932\\
7.26	0.00172412545108283\\
7.27	0.00172436407514853\\
7.28	0.00172460280324596\\
7.29	0.0017248416354308\\
7.3	0.0017250805717588\\
7.31	0.00172531961228573\\
7.32	0.00172555875706742\\
7.33	0.00172579800615975\\
7.34	0.00172603735961862\\
7.35	0.00172627681749999\\
7.36	0.00172651637985986\\
7.37	0.00172675604675428\\
7.38	0.00172699581823935\\
7.39	0.00172723569437118\\
7.4	0.00172747567520597\\
7.41	0.00172771576079993\\
7.42	0.00172795595120933\\
7.43	0.00172819624649048\\
7.44	0.00172843664669975\\
7.45	0.00172867715189352\\
7.46	0.00172891776212825\\
7.47	0.00172915847746043\\
7.48	0.00172939929794658\\
7.49	0.0017296402236433\\
7.5	0.00172988125460719\\
7.51	0.00173012239089494\\
7.52	0.00173036363256326\\
7.53	0.00173060497966891\\
7.54	0.00173084643226869\\
7.55	0.00173108799041945\\
7.56	0.00173132965417809\\
7.57	0.00173157142360154\\
7.58	0.0017318132987468\\
7.59	0.00173205527967088\\
7.6	0.00173229736643087\\
7.61	0.0017325395590839\\
7.62	0.00173278185768711\\
7.63	0.00173302426229774\\
7.64	0.00173326677297303\\
7.65	0.00173350938977029\\
7.66	0.00173375211274688\\
7.67	0.00173399494196018\\
7.68	0.00173423787746763\\
7.69	0.00173448091932674\\
7.7	0.00173472406759501\\
7.71	0.00173496732233005\\
7.72	0.00173521068358947\\
7.73	0.00173545415143094\\
7.74	0.00173569772591219\\
7.75	0.00173594140709097\\
7.76	0.0017361851950251\\
7.77	0.00173642908977244\\
7.78	0.00173667309139088\\
7.79	0.00173691719993837\\
7.8	0.00173716141547291\\
7.81	0.00173740573805255\\
7.82	0.00173765016773538\\
7.83	0.00173789470457953\\
7.84	0.00173813934864318\\
7.85	0.00173838409998457\\
7.86	0.00173862895866198\\
7.87	0.00173887392473371\\
7.88	0.00173911899825815\\
7.89	0.00173936417929371\\
7.9	0.00173960946789887\\
7.91	0.00173985486413213\\
7.92	0.00174010036805204\\
7.93	0.00174034597971723\\
7.94	0.00174059169918634\\
7.95	0.00174083752651807\\
7.96	0.00174108346177118\\
7.97	0.00174132950500446\\
7.98	0.00174157565627676\\
7.99	0.00174182191564696\\
8	0.001742068283174\\
8.01	0.00174231475891689\\
8.02	0.00174256134293464\\
8.03	0.00174280803528636\\
8.04	0.00174305483603115\\
8.05	0.00174330174522822\\
8.06	0.00174354876293677\\
8.07	0.00174379588921609\\
8.08	0.00174404312412551\\
8.09	0.0017442904677244\\
8.1	0.00174453792007217\\
8.11	0.0017447854812283\\
8.12	0.0017450331512523\\
8.13	0.00174528093020374\\
8.14	0.00174552881814224\\
8.15	0.00174577681512746\\
8.16	0.00174602492121911\\
8.17	0.00174627313647695\\
8.18	0.00174652146096079\\
8.19	0.0017467698947305\\
8.2	0.00174701843784598\\
8.21	0.00174726709036718\\
8.22	0.00174751585235412\\
8.23	0.00174776472386684\\
8.24	0.00174801370496545\\
8.25	0.00174826279571012\\
8.26	0.00174851199616103\\
8.27	0.00174876130637844\\
8.28	0.00174901072642266\\
8.29	0.00174926025635403\\
8.3	0.00174950989623296\\
8.31	0.0017497596461199\\
8.32	0.00175000950607535\\
8.33	0.00175025947615985\\
8.34	0.00175050955643401\\
8.35	0.00175075974695849\\
8.36	0.00175101004779398\\
8.37	0.00175126045900123\\
8.38	0.00175151098064104\\
8.39	0.00175176161277426\\
8.4	0.00175201235546179\\
8.41	0.00175226320876459\\
8.42	0.00175251417274365\\
8.43	0.00175276524746003\\
8.44	0.00175301643297483\\
8.45	0.00175326772934921\\
8.46	0.00175351913664436\\
8.47	0.00175377065492155\\
8.48	0.00175402228424208\\
8.49	0.0017542740246673\\
8.5	0.00175452587625862\\
8.51	0.0017547778390775\\
8.52	0.00175502991318546\\
8.53	0.00175528209864404\\
8.54	0.00175553439551487\\
8.55	0.00175578680385959\\
8.56	0.00175603932373994\\
8.57	0.00175629195521767\\
8.58	0.0017565446983546\\
8.59	0.00175679755321261\\
8.6	0.00175705051985359\\
8.61	0.00175730359833955\\
8.62	0.00175755678873248\\
8.63	0.00175781009109448\\
8.64	0.00175806350548766\\
8.65	0.00175831703197421\\
8.66	0.00175857067061636\\
8.67	0.00175882442147638\\
8.68	0.00175907828461663\\
8.69	0.00175933226009947\\
8.7	0.00175958634798736\\
8.71	0.00175984054834279\\
8.72	0.00176009486122829\\
8.73	0.00176034928670647\\
8.74	0.00176060382483998\\
8.75	0.00176085847569152\\
8.76	0.00176111323932383\\
8.77	0.00176136811579974\\
8.78	0.0017616231051821\\
8.79	0.00176187820753382\\
8.8	0.00176213342291787\\
8.81	0.00176238875139727\\
8.82	0.00176264419303509\\
8.83	0.00176289974789445\\
8.84	0.00176315541603855\\
8.85	0.0017634111975306\\
8.86	0.00176366709243389\\
8.87	0.00176392310081176\\
8.88	0.00176417922272761\\
8.89	0.00176443545824488\\
8.9	0.00176469180742707\\
8.91	0.00176494827033773\\
8.92	0.00176520484704046\\
8.93	0.00176546153759895\\
8.94	0.00176571834207688\\
8.95	0.00176597526053804\\
8.96	0.00176623229304625\\
8.97	0.00176648943966538\\
8.98	0.00176674670045936\\
8.99	0.0017670040754922\\
9	0.00176726156482791\\
9.01	0.0017675191685306\\
9.02	0.00176777688666442\\
9.03	0.00176803471929357\\
9.04	0.0017682926664823\\
9.05	0.00176855072829495\\
9.06	0.00176880890479585\\
9.07	0.00176906719604945\\
9.08	0.00176932560212022\\
9.09	0.0017695841230727\\
9.1	0.00176984275897146\\
9.11	0.00177010150988116\\
9.12	0.00177036037586649\\
9.13	0.00177061935699221\\
9.14	0.00177087845332312\\
9.15	0.00177113766492409\\
9.16	0.00177139699186005\\
9.17	0.00177165643419595\\
9.18	0.00177191599199685\\
9.19	0.00177217566532782\\
9.2	0.00177243545425401\\
9.21	0.00177269535884061\\
9.22	0.00177295537915289\\
9.23	0.00177321551525615\\
9.24	0.00177347576721576\\
9.25	0.00177373613509715\\
9.26	0.00177399661896578\\
9.27	0.00177425721888721\\
9.28	0.00177451793492702\\
9.29	0.00177477876715086\\
9.3	0.00177503971562444\\
9.31	0.00177530078041352\\
9.32	0.00177556196158392\\
9.33	0.00177582325920151\\
9.34	0.00177608467333223\\
9.35	0.00177634620404207\\
9.36	0.00177660785139708\\
9.37	0.00177686961546335\\
9.38	0.00177713149630706\\
9.39	0.00177739349399442\\
9.4	0.0017776556085917\\
9.41	0.00177791784016524\\
9.42	0.00177818018878143\\
9.43	0.00177844265450672\\
9.44	0.00177870523740762\\
9.45	0.00177896793755069\\
9.46	0.00177923075500255\\
9.47	0.00177949368982987\\
9.48	0.00177975674209941\\
9.49	0.00178001991187796\\
9.5	0.00178028319923236\\
9.51	0.00178054660422954\\
9.52	0.00178081012693646\\
9.53	0.00178107376742015\\
9.54	0.0017813375257477\\
9.55	0.00178160140198626\\
9.56	0.00178186539620304\\
9.57	0.00178212950846528\\
9.58	0.00178239373884033\\
9.59	0.00178265808739556\\
9.6	0.00178292255419839\\
9.61	0.00178318713931636\\
9.62	0.001783451842817\\
9.63	0.00178371666476793\\
9.64	0.00178398160523683\\
9.65	0.00178424666429144\\
9.66	0.00178451184199955\\
9.67	0.00178477713842902\\
9.68	0.00178504255364777\\
9.69	0.00178530808772377\\
9.7	0.00178557374072504\\
9.71	0.0017858395127197\\
9.72	0.00178610540377588\\
9.73	0.00178637141396181\\
9.74	0.00178663754334576\\
9.75	0.00178690379199606\\
9.76	0.00178717015998111\\
9.77	0.00178743664736936\\
9.78	0.00178770325422934\\
9.79	0.00178796998062961\\
9.8	0.00178823682663882\\
9.81	0.00178850379232565\\
9.82	0.00178877087775887\\
9.83	0.0017890380830073\\
9.84	0.00178930540813982\\
9.85	0.00178957285322536\\
9.86	0.00178984041833294\\
9.87	0.00179010810353161\\
9.88	0.00179037590889049\\
9.89	0.00179064383447879\\
9.9	0.00179091188036573\\
9.91	0.00179118004662063\\
9.92	0.00179144833331287\\
9.93	0.00179171674051187\\
9.94	0.00179198526828713\\
9.95	0.0017922539167082\\
9.96	0.00179252268584471\\
9.97	0.00179279157576632\\
9.98	0.00179306058654279\\
9.99	0.00179332971824391\\
10	0.00179359897093957\\
10.01	0.00179386834469968\\
10.02	0.00179413783959423\\
10.03	0.00179440745569329\\
10.04	0.00179467719306696\\
10.05	0.00179494705178543\\
10.06	0.00179521703191893\\
10.07	0.00179548713353778\\
10.08	0.00179575735671235\\
10.09	0.00179602770151306\\
10.1	0.00179629816801041\\
10.11	0.00179656875627496\\
10.12	0.00179683946637732\\
10.13	0.00179711029838819\\
10.14	0.00179738125237831\\
10.15	0.00179765232841849\\
10.16	0.00179792352657961\\
10.17	0.00179819484693262\\
10.18	0.0017984662895485\\
10.19	0.00179873785449833\\
10.2	0.00179900954185325\\
10.21	0.00179928135168445\\
10.22	0.00179955328406319\\
10.23	0.00179982533906078\\
10.24	0.00180009751674862\\
10.25	0.00180036981719817\\
10.26	0.00180064224048095\\
10.27	0.00180091478666852\\
10.28	0.00180118745583255\\
10.29	0.00180146024804474\\
10.3	0.00180173316337687\\
10.31	0.00180200620190079\\
10.32	0.0018022793636884\\
10.33	0.00180255264881168\\
10.34	0.00180282605734266\\
10.35	0.00180309958935344\\
10.36	0.00180337324491621\\
10.37	0.00180364702410318\\
10.38	0.00180392092698667\\
10.39	0.00180419495363904\\
10.4	0.00180446910413271\\
10.41	0.0018047433785402\\
10.42	0.00180501777693406\\
10.43	0.00180529229938693\\
10.44	0.0018055669459715\\
10.45	0.00180584171676054\\
10.46	0.00180611661182688\\
10.47	0.0018063916312434\\
10.48	0.00180666677508309\\
10.49	0.00180694204341895\\
10.5	0.00180721743632411\\
10.51	0.00180749295387171\\
10.52	0.00180776859613499\\
10.53	0.00180804436318725\\
10.54	0.00180832025510184\\
10.55	0.00180859627195222\\
10.56	0.00180887241381187\\
10.57	0.00180914868075436\\
10.58	0.00180942507285333\\
10.59	0.00180970159018248\\
10.6	0.00180997823281559\\
10.61	0.00181025500082649\\
10.62	0.00181053189428908\\
10.63	0.00181080891327735\\
10.64	0.00181108605786534\\
10.65	0.00181136332812715\\
10.66	0.00181164072413698\\
10.67	0.00181191824596906\\
10.68	0.00181219589369772\\
10.69	0.00181247366739733\\
10.7	0.00181275156714237\\
10.71	0.00181302959300734\\
10.72	0.00181330774506684\\
10.73	0.00181358602339554\\
10.74	0.00181386442806816\\
10.75	0.00181414295915951\\
10.76	0.00181442161674446\\
10.77	0.00181470040089794\\
10.78	0.00181497931169496\\
10.79	0.0018152583492106\\
10.8	0.00181553751352002\\
10.81	0.00181581680469842\\
10.82	0.0018160962228211\\
10.83	0.00181637576796342\\
10.84	0.00181665544020081\\
10.85	0.00181693523960876\\
10.86	0.00181721516626284\\
10.87	0.0018174952202387\\
10.88	0.00181777540161204\\
10.89	0.00181805571045866\\
10.9	0.00181833614685439\\
10.91	0.00181861671087516\\
10.92	0.00181889740259697\\
10.93	0.00181917822209588\\
10.94	0.00181945916944803\\
10.95	0.00181974024472963\\
10.96	0.00182002144801695\\
10.97	0.00182030277938636\\
10.98	0.00182058423891426\\
10.99	0.00182086582667716\\
11	0.00182114754275162\\
11.01	0.00182142938721429\\
11.02	0.00182171136014186\\
11.03	0.00182199346161113\\
11.04	0.00182227569169895\\
11.05	0.00182255805048225\\
11.06	0.00182284053803803\\
11.07	0.00182312315444336\\
11.08	0.00182340589977538\\
11.09	0.00182368877411131\\
11.1	0.00182397177752846\\
11.11	0.00182425491010417\\
11.12	0.00182453817191589\\
11.13	0.00182482156304113\\
11.14	0.00182510508355747\\
11.15	0.00182538873354257\\
11.16	0.00182567251307416\\
11.17	0.00182595642223005\\
11.18	0.00182624046108811\\
11.19	0.0018265246297263\\
11.2	0.00182680892822265\\
11.21	0.00182709335665525\\
11.22	0.00182737791510228\\
11.23	0.001827662603642\\
11.24	0.00182794742235271\\
11.25	0.00182823237131283\\
11.26	0.00182851745060083\\
11.27	0.00182880266029525\\
11.28	0.00182908800047471\\
11.29	0.00182937347121792\\
11.3	0.00182965907260365\\
11.31	0.00182994480471074\\
11.32	0.00183023066761813\\
11.33	0.0018305166614048\\
11.34	0.00183080278614985\\
11.35	0.0018310890419324\\
11.36	0.0018313754288317\\
11.37	0.00183166194692704\\
11.38	0.0018319485962978\\
11.39	0.00183223537702344\\
11.4	0.00183252228918349\\
11.41	0.00183280933285755\\
11.42	0.00183309650812531\\
11.43	0.00183338381506653\\
11.44	0.00183367125376104\\
11.45	0.00183395882428877\\
11.46	0.00183424652672969\\
11.47	0.00183453436116388\\
11.48	0.00183482232767148\\
11.49	0.00183511042633272\\
11.5	0.00183539865722789\\
11.51	0.00183568702043738\\
11.52	0.00183597551604163\\
11.53	0.00183626414412118\\
11.54	0.00183655290475664\\
11.55	0.0018368417980287\\
11.56	0.00183713082401812\\
11.57	0.00183741998280576\\
11.58	0.00183770927447253\\
11.59	0.00183799869909943\\
11.6	0.00183828825676755\\
11.61	0.00183857794755805\\
11.62	0.00183886777155215\\
11.63	0.00183915772883118\\
11.64	0.00183944781947654\\
11.65	0.0018397380435697\\
11.66	0.00184002840119222\\
11.67	0.00184031889242572\\
11.68	0.00184060951735192\\
11.69	0.00184090027605261\\
11.7	0.00184119116860967\\
11.71	0.00184148219510506\\
11.72	0.00184177335562079\\
11.73	0.00184206465023899\\
11.74	0.00184235607904184\\
11.75	0.00184264764211162\\
11.76	0.0018429393395307\\
11.77	0.00184323117138149\\
11.78	0.00184352313774652\\
11.79	0.00184381523870837\\
11.8	0.00184410747434974\\
11.81	0.00184439984475337\\
11.82	0.00184469235000211\\
11.83	0.00184498499017888\\
11.84	0.00184527776536669\\
11.85	0.0018455706756486\\
11.86	0.00184586372110779\\
11.87	0.00184615690182751\\
11.88	0.00184645021789109\\
11.89	0.00184674366938194\\
11.9	0.00184703725638355\\
11.91	0.0018473309789795\\
11.92	0.00184762483725345\\
11.93	0.00184791883128915\\
11.94	0.0018482129611704\\
11.95	0.00184850722698113\\
11.96	0.00184880162880532\\
11.97	0.00184909616672705\\
11.98	0.00184939084083047\\
11.99	0.00184968565119983\\
12	0.00184998059791944\\
12.01	0.00185027568107371\\
12.02	0.00185057090074714\\
12.03	0.0018508662570243\\
12.04	0.00185116174998984\\
12.05	0.00185145737972851\\
12.06	0.00185175314632514\\
12.07	0.00185204904986465\\
12.08	0.00185234509043201\\
12.09	0.00185264126811233\\
12.1	0.00185293758299077\\
12.11	0.00185323403515256\\
12.12	0.00185353062468305\\
12.13	0.00185382735166766\\
12.14	0.0018541242161919\\
12.15	0.00185442121834136\\
12.16	0.00185471835820172\\
12.17	0.00185501563585874\\
12.18	0.00185531305139826\\
12.19	0.00185561060490622\\
12.2	0.00185590829646866\\
12.21	0.00185620612617166\\
12.22	0.00185650409410144\\
12.23	0.00185680220034425\\
12.24	0.00185710044498649\\
12.25	0.00185739882811459\\
12.26	0.00185769734981509\\
12.27	0.00185799601017463\\
12.28	0.00185829480927992\\
12.29	0.00185859374721777\\
12.3	0.00185889282407507\\
12.31	0.00185919203993878\\
12.32	0.00185949139489597\\
12.33	0.00185979088903381\\
12.34	0.00186009052243953\\
12.35	0.00186039029520046\\
12.36	0.00186069020740402\\
12.37	0.0018609902591377\\
12.38	0.00186129045048912\\
12.39	0.00186159078154595\\
12.4	0.00186189125239596\\
12.41	0.00186219186312702\\
12.42	0.00186249261382707\\
12.43	0.00186279350458416\\
12.44	0.00186309453548641\\
12.45	0.00186339570662206\\
12.46	0.00186369701807939\\
12.47	0.00186399846994681\\
12.48	0.00186430006231281\\
12.49	0.00186460179526596\\
12.5	0.00186490366889495\\
12.51	0.00186520568328851\\
12.52	0.00186550783853551\\
12.53	0.00186581013472488\\
12.54	0.00186611257194566\\
12.55	0.00186641515028697\\
12.56	0.00186671786983801\\
12.57	0.0018670207306881\\
12.58	0.00186732373292663\\
12.59	0.00186762687664309\\
12.6	0.00186793016192705\\
12.61	0.00186823358886819\\
12.62	0.00186853715755628\\
12.63	0.00186884086808116\\
12.64	0.00186914472053278\\
12.65	0.00186944871500119\\
12.66	0.00186975285157651\\
12.67	0.00187005713034898\\
12.68	0.0018703615514089\\
12.69	0.00187066611484669\\
12.7	0.00187097082075285\\
12.71	0.00187127566921799\\
12.72	0.00187158066033279\\
12.73	0.00187188579418804\\
12.74	0.00187219107087461\\
12.75	0.00187249649048347\\
12.76	0.0018728020531057\\
12.77	0.00187310775883245\\
12.78	0.00187341360775498\\
12.79	0.00187371959996464\\
12.8	0.00187402573555287\\
12.81	0.00187433201461121\\
12.82	0.00187463843723129\\
12.83	0.00187494500350485\\
12.84	0.0018752517135237\\
12.85	0.00187555856737977\\
12.86	0.00187586556516507\\
12.87	0.00187617270697171\\
12.88	0.00187647999289191\\
12.89	0.00187678742301795\\
12.9	0.00187709499744225\\
12.91	0.00187740271625729\\
12.92	0.00187771057955567\\
12.93	0.00187801858743008\\
12.94	0.0018783267399733\\
12.95	0.0018786350372782\\
12.96	0.00187894347943779\\
12.97	0.00187925206654512\\
12.98	0.00187956079869338\\
12.99	0.00187986967597582\\
13	0.00188017869848583\\
13.01	0.00188048786631688\\
13.02	0.00188079717956252\\
13.03	0.00188110663831642\\
13.04	0.00188141624267234\\
13.05	0.00188172599272415\\
13.06	0.00188203588856579\\
13.07	0.00188234593029133\\
13.08	0.00188265611799493\\
13.09	0.00188296645177084\\
13.1	0.00188327693171342\\
13.11	0.00188358755791712\\
13.12	0.0018838983304765\\
13.13	0.00188420924948621\\
13.14	0.001884520315041\\
13.15	0.00188483152723574\\
13.16	0.00188514288616537\\
13.17	0.00188545439192495\\
13.18	0.00188576604460964\\
13.19	0.00188607784431469\\
13.2	0.00188638979113546\\
13.21	0.00188670188516741\\
13.22	0.0018870141265061\\
13.23	0.00188732651524719\\
13.24	0.00188763905148644\\
13.25	0.00188795173531973\\
13.26	0.001888264566843\\
13.27	0.00188857754615235\\
13.28	0.00188889067334393\\
13.29	0.00188920394851402\\
13.3	0.00188951737175899\\
13.31	0.00188983094317533\\
13.32	0.00189014466285962\\
13.33	0.00189045853090854\\
13.34	0.00189077254741888\\
13.35	0.00189108671248753\\
13.36	0.0018914010262115\\
13.37	0.00189171548868787\\
13.38	0.00189203010001385\\
13.39	0.00189234486028676\\
13.4	0.00189265976960399\\
13.41	0.00189297482806307\\
13.42	0.00189329003576161\\
13.43	0.00189360539279736\\
13.44	0.00189392089926813\\
13.45	0.00189423655527186\\
13.46	0.00189455236090659\\
13.47	0.00189486831627047\\
13.48	0.00189518442146175\\
13.49	0.00189550067657879\\
13.5	0.00189581708172006\\
13.51	0.00189613363698411\\
13.52	0.00189645034246964\\
13.53	0.00189676719827543\\
13.54	0.00189708420450037\\
13.55	0.00189740136124344\\
13.56	0.00189771866860377\\
13.57	0.00189803612668055\\
13.58	0.00189835373557311\\
13.59	0.00189867149538088\\
13.6	0.00189898940620339\\
13.61	0.00189930746814027\\
13.62	0.0018996256812913\\
13.63	0.00189994404575631\\
13.64	0.00190026256163528\\
13.65	0.00190058122902829\\
13.66	0.00190090004803552\\
13.67	0.00190121901875726\\
13.68	0.00190153814129393\\
13.69	0.00190185741574603\\
13.7	0.00190217684221418\\
13.71	0.00190249642079911\\
13.72	0.00190281615160168\\
13.73	0.00190313603472284\\
13.74	0.00190345607026363\\
13.75	0.00190377625832524\\
13.76	0.00190409659900895\\
13.77	0.00190441709241616\\
13.78	0.00190473773864838\\
13.79	0.00190505853780721\\
13.8	0.0019053794899944\\
13.81	0.00190570059531178\\
13.82	0.00190602185386129\\
13.83	0.00190634326574502\\
13.84	0.00190666483106513\\
13.85	0.00190698654992391\\
13.86	0.00190730842242377\\
13.87	0.00190763044866721\\
13.88	0.00190795262875688\\
13.89	0.0019082749627955\\
13.9	0.00190859745088592\\
13.91	0.00190892009313113\\
13.92	0.0019092428896342\\
13.93	0.00190956584049832\\
13.94	0.0019098889458268\\
13.95	0.00191021220572307\\
13.96	0.00191053562029066\\
13.97	0.00191085918963323\\
13.98	0.00191118291385455\\
13.99	0.0019115067930585\\
14	0.00191183082734907\\
14.01	0.00191215501683038\\
14.02	0.00191247936160666\\
14.03	0.00191280386178225\\
14.04	0.00191312851746162\\
14.05	0.00191345332874934\\
14.06	0.0019137782957501\\
14.07	0.00191410341856872\\
14.08	0.00191442869731012\\
14.09	0.00191475413207935\\
14.1	0.00191507972298157\\
14.11	0.00191540547012205\\
14.12	0.0019157313736062\\
14.13	0.00191605743353952\\
14.14	0.00191638365002766\\
14.15	0.00191671002317636\\
14.16	0.00191703655309149\\
14.17	0.00191736323987903\\
14.18	0.0019176900836451\\
14.19	0.00191801708449593\\
14.2	0.00191834424253785\\
14.21	0.00191867155787732\\
14.22	0.00191899903062095\\
14.23	0.00191932666087542\\
14.24	0.00191965444874757\\
14.25	0.00191998239434433\\
14.26	0.00192031049777278\\
14.27	0.00192063875914009\\
14.28	0.00192096717855357\\
14.29	0.00192129575612066\\
14.3	0.0019216244919489\\
14.31	0.00192195338614595\\
14.32	0.00192228243881962\\
14.33	0.00192261165007782\\
14.34	0.00192294102002858\\
14.35	0.00192327054878006\\
14.36	0.00192360023644055\\
14.37	0.00192393008311844\\
14.38	0.00192426008892226\\
14.39	0.00192459025396067\\
14.4	0.00192492057834243\\
14.41	0.00192525106217645\\
14.42	0.00192558170557175\\
14.43	0.00192591250863747\\
14.44	0.00192624347148287\\
14.45	0.00192657459421736\\
14.46	0.00192690587695046\\
14.47	0.0019272373197918\\
14.48	0.00192756892285115\\
14.49	0.00192790068623841\\
14.5	0.0019282326100636\\
14.51	0.00192856469443686\\
14.52	0.00192889693946847\\
14.53	0.00192922934526881\\
14.54	0.00192956191194842\\
14.55	0.00192989463961795\\
14.56	0.00193022752838817\\
14.57	0.00193056057837\\
14.58	0.00193089378967445\\
14.59	0.00193122716241269\\
14.6	0.00193156069669601\\
14.61	0.00193189439263582\\
14.62	0.00193222825034366\\
14.63	0.00193256226993122\\
14.64	0.00193289645151028\\
14.65	0.00193323079519277\\
14.66	0.00193356530109077\\
14.67	0.00193389996931645\\
14.68	0.00193423479998213\\
14.69	0.00193456979320025\\
14.7	0.0019349049490834\\
14.71	0.00193524026774429\\
14.72	0.00193557574929575\\
14.73	0.00193591139385074\\
14.74	0.00193624720152238\\
14.75	0.00193658317242388\\
14.76	0.00193691930666862\\
14.77	0.00193725560437007\\
14.78	0.00193759206564188\\
14.79	0.0019379286905978\\
14.8	0.00193826547935171\\
14.81	0.00193860243201765\\
14.82	0.00193893954870976\\
14.83	0.00193927682954233\\
14.84	0.00193961427462979\\
14.85	0.00193995188408669\\
14.86	0.00194028965802771\\
14.87	0.00194062759656769\\
14.88	0.00194096569982158\\
14.89	0.00194130396790446\\
14.9	0.00194164240093157\\
14.91	0.00194198099901827\\
14.92	0.00194231976228004\\
14.93	0.00194265869083254\\
14.94	0.0019429977847915\\
14.95	0.00194333704427286\\
14.96	0.00194367646939263\\
14.97	0.00194401606026701\\
14.98	0.00194435581701229\\
14.99	0.00194469573974492\\
15	0.0019450358285815\\
15.01	0.00194537608363874\\
15.02	0.00194571650503351\\
15.03	0.00194605709288279\\
15.04	0.00194639784730374\\
15.05	0.00194673876841362\\
15.06	0.00194707985632985\\
15.07	0.00194742111116998\\
15.08	0.0019477625330517\\
15.09	0.00194810412209283\\
15.1	0.00194844587841136\\
15.11	0.00194878780212538\\
15.12	0.00194912989335315\\
15.13	0.00194947215221306\\
15.14	0.00194981457882363\\
15.15	0.00195015717330354\\
15.16	0.0019504999357716\\
15.17	0.00195084286634677\\
15.18	0.00195118596514813\\
15.19	0.00195152923229493\\
15.2	0.00195187266790654\\
15.21	0.00195221627210249\\
15.22	0.00195256004500244\\
15.23	0.00195290398672619\\
15.24	0.00195324809739369\\
15.25	0.00195359237712503\\
15.26	0.00195393682604045\\
15.27	0.00195428144426034\\
15.28	0.00195462623190521\\
15.29	0.00195497118909573\\
15.3	0.00195531631595272\\
15.31	0.00195566161259714\\
15.32	0.00195600707915008\\
15.33	0.00195635271573279\\
15.34	0.00195669852246667\\
15.35	0.00195704449947327\\
15.36	0.00195739064687425\\
15.37	0.00195773696479146\\
15.38	0.00195808345334688\\
15.39	0.00195843011266263\\
15.4	0.00195877694286098\\
15.41	0.00195912394406435\\
15.42	0.00195947111639532\\
15.43	0.0019598184599766\\
15.44	0.00196016597493105\\
15.45	0.0019605136613817\\
15.46	0.00196086151945169\\
15.47	0.00196120954926435\\
15.48	0.00196155775094312\\
15.49	0.00196190612461163\\
15.5	0.00196225467039363\\
15.51	0.00196260338841304\\
15.52	0.00196295227879391\\
15.53	0.00196330134166045\\
15.54	0.00196365057713703\\
15.55	0.00196399998534817\\
15.56	0.00196434956641851\\
15.57	0.0019646993204729\\
15.58	0.00196504924763629\\
15.59	0.0019653993480338\\
15.6	0.00196574962179072\\
15.61	0.00196610006903246\\
15.62	0.00196645068988461\\
15.63	0.0019668014844729\\
15.64	0.00196715245292323\\
15.65	0.00196750359536162\\
15.66	0.00196785491191428\\
15.67	0.00196820640270755\\
15.68	0.00196855806786796\\
15.69	0.00196890990752215\\
15.7	0.00196926192179694\\
15.71	0.00196961411081931\\
15.72	0.00196996647471637\\
15.73	0.00197031901361543\\
15.74	0.00197067172764392\\
15.75	0.00197102461692943\\
15.76	0.00197137768159973\\
15.77	0.00197173092178272\\
15.78	0.00197208433760648\\
15.79	0.00197243792919923\\
15.8	0.00197279169668937\\
15.81	0.00197314564020543\\
15.82	0.00197349975987612\\
15.83	0.0019738540558303\\
15.84	0.00197420852819701\\
15.85	0.00197456317710542\\
15.86	0.00197491800268488\\
15.87	0.00197527300506489\\
15.88	0.00197562818437511\\
15.89	0.00197598354074537\\
15.9	0.00197633907430565\\
15.91	0.00197669478518612\\
15.92	0.00197705067351707\\
15.93	0.00197740673942898\\
15.94	0.00197776298305248\\
15.95	0.00197811940451837\\
15.96	0.00197847600395762\\
15.97	0.00197883278150134\\
15.98	0.00197918973728083\\
15.99	0.00197954687142754\\
16	0.00197990418407308\\
16.01	0.00198026167534923\\
16.02	0.00198061934538794\\
16.03	0.00198097719432133\\
16.04	0.00198133522228167\\
16.05	0.0019816934294014\\
16.06	0.00198205181581312\\
16.07	0.00198241038164963\\
16.08	0.00198276912704385\\
16.09	0.0019831280521289\\
16.1	0.00198348715703806\\
16.11	0.00198384644190476\\
16.12	0.00198420590686262\\
16.13	0.00198456555204543\\
16.14	0.00198492537758713\\
16.15	0.00198528538362183\\
16.16	0.00198564557028384\\
16.17	0.00198600593770759\\
16.18	0.00198636648602774\\
16.19	0.00198672721537906\\
16.2	0.00198708812589653\\
16.21	0.00198744921771529\\
16.22	0.00198781049097064\\
16.23	0.00198817194579808\\
16.24	0.00198853358233324\\
16.25	0.00198889540071197\\
16.26	0.00198925740107025\\
16.27	0.00198961958354427\\
16.28	0.00198998194827035\\
16.29	0.00199034449538503\\
16.3	0.00199070722502499\\
16.31	0.00199107013732709\\
16.32	0.00199143323242839\\
16.33	0.00199179651046609\\
16.34	0.00199215997157758\\
16.35	0.00199252361590044\\
16.36	0.00199288744357239\\
16.37	0.00199325145473137\\
16.38	0.00199361564951545\\
16.39	0.00199398002806292\\
16.4	0.00199434459051222\\
16.41	0.00199470933700198\\
16.42	0.001995074267671\\
16.43	0.00199543938265825\\
16.44	0.00199580468210291\\
16.45	0.00199617016614431\\
16.46	0.00199653583492197\\
16.47	0.00199690168857559\\
16.48	0.00199726772724503\\
16.49	0.00199763395107037\\
16.5	0.00199800036019183\\
16.51	0.00199836695474984\\
16.52	0.001998733734885\\
16.53	0.00199910070073809\\
16.54	0.00199946785245007\\
16.55	0.00199983519016208\\
16.56	0.00200020271401546\\
16.57	0.00200057042415171\\
16.58	0.00200093832071253\\
16.59	0.0020013064038398\\
16.6	0.00200167467367558\\
16.61	0.00200204313036211\\
16.62	0.00200241177404184\\
16.63	0.00200278060485737\\
16.64	0.00200314962295151\\
16.65	0.00200351882846724\\
16.66	0.00200388822154773\\
16.67	0.00200425780233635\\
16.68	0.00200462757097664\\
16.69	0.00200499752761234\\
16.7	0.00200536767238738\\
16.71	0.00200573800544585\\
16.72	0.00200610852693205\\
16.73	0.00200647923699047\\
16.74	0.00200685013576579\\
16.75	0.00200722122340288\\
16.76	0.00200759250004678\\
16.77	0.00200796396584274\\
16.78	0.0020083356209362\\
16.79	0.00200870746547278\\
16.8	0.00200907949959829\\
16.81	0.00200945172345875\\
16.82	0.00200982413720036\\
16.83	0.00201019674096951\\
16.84	0.00201056953491278\\
16.85	0.00201094251917696\\
16.86	0.00201131569390901\\
16.87	0.0020116890592561\\
16.88	0.0020120626153656\\
16.89	0.00201243636238505\\
16.9	0.00201281030046221\\
16.91	0.00201318442974501\\
16.92	0.0020135587503816\\
16.93	0.00201393326252032\\
16.94	0.0020143079663097\\
16.95	0.00201468286189847\\
16.96	0.00201505794943555\\
16.97	0.00201543322907007\\
16.98	0.00201580870095136\\
16.99	0.00201618436522893\\
17	0.0020165602220525\\
17.01	0.00201693627157198\\
17.02	0.00201731251393752\\
17.03	0.00201768894929941\\
17.04	0.00201806557780817\\
17.05	0.00201844239961452\\
17.06	0.0020188194148694\\
17.07	0.00201919662372391\\
17.08	0.00201957402632939\\
17.09	0.00201995162283734\\
17.1	0.00202032941339951\\
17.11	0.00202070739816783\\
17.12	0.00202108557729443\\
17.13	0.00202146395093164\\
17.14	0.00202184251923202\\
17.15	0.00202222128234831\\
17.16	0.00202260024043346\\
17.17	0.00202297939364063\\
17.18	0.00202335874212319\\
17.19	0.0020237382860347\\
17.2	0.00202411802552895\\
17.21	0.00202449796075991\\
17.22	0.00202487809188178\\
17.23	0.00202525841904896\\
17.24	0.00202563894241605\\
17.25	0.00202601966213788\\
17.26	0.00202640057836946\\
17.27	0.00202678169126604\\
17.28	0.00202716300098305\\
17.29	0.00202754450767616\\
17.3	0.00202792621150123\\
17.31	0.00202830811261433\\
17.32	0.00202869021117175\\
17.33	0.00202907250733001\\
17.34	0.00202945500124581\\
17.35	0.00202983769307608\\
17.36	0.00203022058297794\\
17.37	0.00203060367110877\\
17.38	0.00203098695762613\\
17.39	0.0020313704426878\\
17.4	0.00203175412645178\\
17.41	0.00203213800907627\\
17.42	0.00203252209071971\\
17.43	0.00203290637154075\\
17.44	0.00203329085169824\\
17.45	0.00203367553135127\\
17.46	0.00203406041065912\\
17.47	0.00203444548978132\\
17.48	0.00203483076887759\\
17.49	0.00203521624810789\\
17.5	0.00203560192763239\\
17.51	0.00203598780761147\\
17.52	0.00203637388820575\\
17.53	0.00203676016957606\\
17.54	0.00203714665188345\\
17.55	0.00203753333528921\\
17.56	0.00203792021995481\\
17.57	0.00203830730604198\\
17.58	0.00203869459371266\\
17.59	0.00203908208312902\\
17.6	0.00203946977445345\\
17.61	0.00203985766784855\\
17.62	0.00204024576347716\\
17.63	0.00204063406150234\\
17.64	0.00204102256208739\\
17.65	0.00204141126539582\\
17.66	0.00204180017159136\\
17.67	0.00204218928083797\\
17.68	0.00204257859329986\\
17.69	0.00204296810914145\\
17.7	0.00204335782852739\\
17.71	0.00204374775162255\\
17.72	0.00204413787859205\\
17.73	0.00204452820960123\\
17.74	0.00204491874481564\\
17.75	0.00204530948440109\\
17.76	0.00204570042852361\\
17.77	0.00204609157734946\\
17.78	0.00204648293104513\\
17.79	0.00204687448977734\\
17.8	0.00204726625371306\\
17.81	0.00204765822301948\\
17.82	0.00204805039786402\\
17.83	0.00204844277841434\\
17.84	0.00204883536483833\\
17.85	0.00204922815730414\\
17.86	0.00204962115598011\\
17.87	0.00205001436103486\\
17.88	0.00205040777263722\\
17.89	0.00205080139095627\\
17.9	0.00205119521616132\\
17.91	0.00205158924842191\\
17.92	0.00205198348790786\\
17.93	0.00205237793478918\\
17.94	0.00205277258923613\\
17.95	0.00205316745141924\\
17.96	0.00205356252150924\\
17.97	0.00205395779967713\\
17.98	0.00205435328609415\\
17.99	0.00205474898093176\\
18	0.00205514488436169\\
18.01	0.00205554099655589\\
18.02	0.00205593731768657\\
18.03	0.00205633384792616\\
18.04	0.00205673058744738\\
18.05	0.00205712753642314\\
18.06	0.00205752469502664\\
18.07	0.00205792206343131\\
18.08	0.00205831964181081\\
18.09	0.00205871743033907\\
18.1	0.00205911542919027\\
18.11	0.00205951363853881\\
18.12	0.00205991205855937\\
18.13	0.00206031068942686\\
18.14	0.00206070953131644\\
18.15	0.00206110858440352\\
18.16	0.00206150784886379\\
18.17	0.00206190732487315\\
18.18	0.00206230701260777\\
18.19	0.00206270691224407\\
18.2	0.00206310702395873\\
18.21	0.00206350734792866\\
18.22	0.00206390788433106\\
18.23	0.00206430863334336\\
18.24	0.00206470959514324\\
18.25	0.00206511076990865\\
18.26	0.00206551215781779\\
18.27	0.00206591375904912\\
18.28	0.00206631557378135\\
18.29	0.00206671760219345\\
18.3	0.00206711984446466\\
18.31	0.00206752230077445\\
18.32	0.00206792497130258\\
18.33	0.00206832785622905\\
18.34	0.00206873095573412\\
18.35	0.00206913426999833\\
18.36	0.00206953779920246\\
18.37	0.00206994154352756\\
18.38	0.00207034550315495\\
18.39	0.0020707496782662\\
18.4	0.00207115406904314\\
18.41	0.00207155867566788\\
18.42	0.00207196349832278\\
18.43	0.00207236853719048\\
18.44	0.00207277379245388\\
18.45	0.00207317926429613\\
18.46	0.00207358495290066\\
18.47	0.00207399085845118\\
18.48	0.00207439698113165\\
18.49	0.00207480332112629\\
18.5	0.00207520987861962\\
18.51	0.00207561665379639\\
18.52	0.00207602364684167\\
18.53	0.00207643085794074\\
18.54	0.00207683828727921\\
18.55	0.00207724593504292\\
18.56	0.00207765380141801\\
18.57	0.00207806188659086\\
18.58	0.00207847019074816\\
18.59	0.00207887871407685\\
18.6	0.00207928745676415\\
18.61	0.00207969641899756\\
18.62	0.00208010560096485\\
18.63	0.00208051500285407\\
18.64	0.00208092462485355\\
18.65	0.00208133446715188\\
18.66	0.00208174452993796\\
18.67	0.00208215481340093\\
18.68	0.00208256531773025\\
18.69	0.00208297604311562\\
18.7	0.00208338698974705\\
18.71	0.00208379815781482\\
18.72	0.00208420954750949\\
18.73	0.00208462115902189\\
18.74	0.00208503299254317\\
18.75	0.00208544504826473\\
18.76	0.00208585732637825\\
18.77	0.00208626982707573\\
18.78	0.00208668255054942\\
18.79	0.00208709549699188\\
18.8	0.00208750866659593\\
18.81	0.00208792205955471\\
18.82	0.00208833567606161\\
18.83	0.00208874951631035\\
18.84	0.00208916358049491\\
18.85	0.00208957786880955\\
18.86	0.00208999238144886\\
18.87	0.00209040711860769\\
18.88	0.00209082208048118\\
18.89	0.00209123726726478\\
18.9	0.00209165267915421\\
18.91	0.00209206831634551\\
18.92	0.00209248417903499\\
18.93	0.00209290026741927\\
18.94	0.00209331658169525\\
18.95	0.00209373312206015\\
18.96	0.00209414988871146\\
18.97	0.00209456688184697\\
18.98	0.00209498410166479\\
18.99	0.0020954015483633\\
19	0.00209581922214121\\
19.01	0.00209623712319749\\
19.02	0.00209665525173144\\
19.03	0.00209707360794266\\
19.04	0.00209749219203103\\
19.05	0.00209791100419675\\
19.06	0.00209833004464033\\
19.07	0.00209874931356255\\
19.08	0.00209916881116454\\
19.09	0.00209958853764768\\
19.1	0.00210000849321371\\
19.11	0.00210042867806463\\
19.12	0.00210084909240279\\
19.13	0.0021012697364308\\
19.14	0.00210169061035162\\
19.15	0.00210211171436849\\
19.16	0.00210253304868498\\
19.17	0.00210295461350496\\
19.18	0.0021033764090326\\
19.19	0.00210379843547239\\
19.2	0.00210422069302914\\
19.21	0.00210464318190798\\
19.22	0.00210506590231432\\
19.23	0.0021054888544539\\
19.24	0.00210591203853279\\
19.25	0.00210633545475736\\
19.26	0.00210675910333429\\
19.27	0.00210718298447059\\
19.28	0.00210760709837359\\
19.29	0.00210803144525092\\
19.3	0.00210845602531055\\
19.31	0.00210888083876073\\
19.32	0.00210930588581009\\
19.33	0.00210973116666754\\
19.34	0.00211015668154231\\
19.35	0.00211058243064397\\
19.36	0.00211100841418241\\
19.37	0.00211143463236783\\
19.38	0.00211186108541077\\
19.39	0.00211228777352208\\
19.4	0.00211271469691295\\
19.41	0.00211314185579489\\
19.42	0.00211356925037973\\
19.43	0.00211399688087964\\
19.44	0.00211442474750711\\
19.45	0.00211485285047496\\
19.46	0.00211528118999635\\
19.47	0.00211570976628475\\
19.48	0.00211613857955398\\
19.49	0.0021165676300182\\
19.5	0.00211699691789187\\
19.51	0.00211742644338981\\
19.52	0.00211785620672716\\
19.53	0.0021182862081194\\
19.54	0.00211871644778236\\
19.55	0.00211914692593218\\
19.56	0.00211957764278536\\
19.57	0.00212000859855872\\
19.58	0.00212043979346944\\
19.59	0.00212087122773501\\
19.6	0.00212130290157329\\
19.61	0.00212173481520245\\
19.62	0.00212216696884104\\
19.63	0.00212259936270792\\
19.64	0.00212303199702231\\
19.65	0.00212346487200376\\
19.66	0.00212389798787218\\
19.67	0.00212433134484781\\
19.68	0.00212476494315125\\
19.69	0.00212519878300345\\
19.7	0.00212563286462568\\
19.71	0.00212606718823959\\
19.72	0.00212650175406716\\
19.73	0.00212693656233073\\
19.74	0.00212737161325299\\
19.75	0.00212780690705698\\
19.76	0.00212824244396608\\
19.77	0.00212867822420405\\
19.78	0.00212911424799497\\
19.79	0.0021295505155633\\
19.8	0.00212998702713386\\
19.81	0.0021304237829318\\
19.82	0.00213086078318264\\
19.83	0.00213129802811226\\
19.84	0.0021317355179469\\
19.85	0.00213217325291316\\
19.86	0.002132611233238\\
19.87	0.00213304945914872\\
19.88	0.00213348793087302\\
19.89	0.00213392664863893\\
19.9	0.00213436561267485\\
19.91	0.00213480482320957\\
19.92	0.00213524428047222\\
19.93	0.00213568398469228\\
19.94	0.00213612393609963\\
19.95	0.00213656413492451\\
19.96	0.00213700458139751\\
19.97	0.00213744527574961\\
19.98	0.00213788621821216\\
19.99	0.00213832740901685\\
20	0.00213876884839578\\
20.01	0.00213921053658141\\
20.02	0.00213965247380655\\
20.03	0.00214009466030442\\
20.04	0.00214053709630859\\
20.05	0.00214097978205302\\
20.06	0.00214142271777203\\
20.07	0.00214186590370034\\
20.08	0.00214230934007303\\
20.09	0.00214275302712557\\
20.1	0.00214319696509379\\
20.11	0.00214364115421393\\
20.12	0.00214408559472259\\
20.13	0.00214453028685676\\
20.14	0.00214497523085382\\
20.15	0.00214542042695152\\
20.16	0.002145865875388\\
20.17	0.0021463115764018\\
20.18	0.00214675753023182\\
20.19	0.00214720373711738\\
20.2	0.00214765019729816\\
20.21	0.00214809691101424\\
20.22	0.00214854387850609\\
20.23	0.00214899110001458\\
20.24	0.00214943857578095\\
20.25	0.00214988630604687\\
20.26	0.00215033429105435\\
20.27	0.00215078253104585\\
20.28	0.0021512310262642\\
20.29	0.00215167977695261\\
20.3	0.00215212878335472\\
20.31	0.00215257804571455\\
20.32	0.00215302756427651\\
20.33	0.00215347733928544\\
20.34	0.00215392737098655\\
20.35	0.00215437765962548\\
20.36	0.00215482820544824\\
20.37	0.00215527900870127\\
20.38	0.00215573006963141\\
20.39	0.0021561813884859\\
20.4	0.00215663296551239\\
20.41	0.00215708480095893\\
20.42	0.00215753689507398\\
20.43	0.00215798924810643\\
20.44	0.00215844186030555\\
20.45	0.00215889473192104\\
20.46	0.002159347863203\\
20.47	0.00215980125440195\\
20.48	0.00216025490576884\\
20.49	0.00216070881755499\\
20.5	0.00216116299001219\\
20.51	0.00216161742339261\\
20.52	0.00216207211794885\\
20.53	0.00216252707393392\\
20.54	0.00216298229160126\\
20.55	0.00216343777120474\\
20.56	0.00216389351299862\\
20.57	0.00216434951723762\\
20.58	0.00216480578417685\\
20.59	0.00216526231407186\\
20.6	0.00216571910717863\\
20.61	0.00216617616375356\\
20.62	0.00216663348405348\\
20.63	0.00216709106833564\\
20.64	0.00216754891685773\\
20.65	0.00216800702987787\\
20.66	0.00216846540765461\\
20.67	0.00216892405044691\\
20.68	0.00216938295851421\\
20.69	0.00216984213211634\\
20.7	0.00217030157151359\\
20.71	0.00217076127696667\\
20.72	0.00217122124873674\\
20.73	0.0021716814870854\\
20.74	0.00217214199227468\\
20.75	0.00217260276456705\\
20.76	0.00217306380422543\\
20.77	0.00217352511151318\\
20.78	0.00217398668669409\\
20.79	0.00217444853003241\\
20.8	0.00217491064179284\\
20.81	0.0021753730222405\\
20.82	0.00217583567164099\\
20.83	0.00217629859026033\\
20.84	0.00217676177836502\\
20.85	0.00217722523622197\\
20.86	0.00217768896409858\\
20.87	0.00217815296226269\\
20.88	0.00217861723098257\\
20.89	0.00217908177052699\\
20.9	0.00217954658116513\\
20.91	0.00218001166316667\\
20.92	0.00218047701680171\\
20.93	0.00218094264234083\\
20.94	0.00218140854005507\\
20.95	0.00218187471021591\\
20.96	0.00218234115309533\\
20.97	0.00218280786896574\\
20.98	0.00218327485810003\\
20.99	0.00218374212077154\\
21	0.0021842096572541\\
21.01	0.002184677467822\\
21.02	0.00218514555274998\\
21.03	0.00218561391231327\\
21.04	0.00218608254678757\\
21.05	0.00218655145644903\\
21.06	0.00218702064157431\\
21.07	0.00218749010244051\\
21.08	0.00218795983932523\\
21.09	0.00218842985250653\\
21.1	0.00218890014226295\\
21.11	0.00218937070887352\\
21.12	0.00218984155261773\\
21.13	0.00219031267377557\\
21.14	0.0021907840726275\\
21.15	0.00219125574945447\\
21.16	0.00219172770453792\\
21.17	0.00219219993815975\\
21.18	0.00219267245060237\\
21.19	0.00219314524214868\\
21.2	0.00219361831308204\\
21.21	0.00219409166368634\\
21.22	0.00219456529424593\\
21.23	0.00219503920504566\\
21.24	0.00219551339637089\\
21.25	0.00219598786850746\\
21.26	0.00219646262174169\\
21.27	0.00219693765636043\\
21.28	0.002197412972651\\
21.29	0.00219788857090124\\
21.3	0.00219836445139948\\
21.31	0.00219884061443455\\
21.32	0.00219931706029579\\
21.33	0.00219979378927303\\
21.34	0.00220027080165663\\
21.35	0.00220074809773743\\
21.36	0.00220122567780679\\
21.37	0.00220170354215658\\
21.38	0.00220218169107918\\
21.39	0.00220266012486748\\
21.4	0.00220313884381487\\
21.41	0.00220361784821527\\
21.42	0.00220409713836312\\
21.43	0.00220457671455335\\
21.44	0.00220505657708144\\
21.45	0.00220553672624334\\
21.46	0.00220601716233559\\
21.47	0.00220649788565518\\
21.48	0.00220697889649968\\
21.49	0.00220746019516713\\
21.5	0.00220794178195615\\
21.51	0.00220842365716584\\
21.52	0.00220890582109584\\
21.53	0.00220938827404634\\
21.54	0.00220987101631804\\
21.55	0.00221035404821217\\
21.56	0.00221083737003049\\
21.57	0.00221132098207531\\
21.58	0.00221180488464946\\
21.59	0.00221228907805631\\
21.6	0.00221277356259977\\
21.61	0.00221325833858428\\
21.62	0.00221374340631483\\
21.63	0.00221422876609696\\
21.64	0.00221471441823672\\
21.65	0.00221520036304073\\
21.66	0.00221568660081616\\
21.67	0.00221617313187069\\
21.68	0.00221665995651259\\
21.69	0.00221714707505067\\
21.7	0.00221763448779427\\
21.71	0.00221812219505329\\
21.72	0.00221861019713819\\
21.73	0.00221909849435998\\
21.74	0.00221958708703023\\
21.75	0.00222007597546105\\
21.76	0.00222056515996515\\
21.77	0.00222105464085574\\
21.78	0.00222154441844663\\
21.79	0.00222203449305219\\
21.8	0.00222252486498735\\
21.81	0.00222301553456759\\
21.82	0.00222350650210899\\
21.83	0.00222399776792816\\
21.84	0.0022244893323423\\
21.85	0.00222498119566917\\
21.86	0.00222547335822712\\
21.87	0.00222596582033506\\
21.88	0.00222645858231247\\
21.89	0.00222695164447941\\
21.9	0.00222744500715653\\
21.91	0.00222793867066504\\
21.92	0.00222843263532674\\
21.93	0.00222892690146401\\
21.94	0.00222942146939981\\
21.95	0.00222991633945769\\
21.96	0.00223041151196179\\
21.97	0.00223090698723682\\
21.98	0.00223140276560809\\
21.99	0.00223189884740151\\
22	0.00223239523294356\\
22.01	0.00223289192256133\\
22.02	0.00223338891658249\\
22.03	0.00223388621533532\\
22.04	0.00223438381914869\\
22.05	0.00223488172835207\\
22.06	0.00223537994327552\\
22.07	0.00223587846424973\\
22.08	0.00223637729160597\\
22.09	0.00223687642567612\\
22.1	0.00223737586679265\\
22.11	0.00223787561528867\\
22.12	0.00223837567149788\\
22.13	0.00223887603575459\\
22.14	0.00223937670839372\\
22.15	0.00223987768975081\\
22.16	0.00224037898016202\\
22.17	0.00224088057996412\\
22.18	0.00224138248949449\\
22.19	0.00224188470909113\\
22.2	0.00224238723909269\\
22.21	0.0022428900798384\\
22.22	0.00224339323166815\\
22.23	0.00224389669492243\\
22.24	0.00224440046994238\\
22.25	0.00224490455706974\\
22.26	0.0022454089566469\\
22.27	0.00224591366901689\\
22.28	0.00224641869452336\\
22.29	0.00224692403351058\\
22.3	0.00224742968632349\\
22.31	0.00224793565330764\\
22.32	0.00224844193480923\\
22.33	0.00224894853117511\\
22.34	0.00224945544275276\\
22.35	0.00224996266989031\\
22.36	0.00225047021293653\\
22.37	0.00225097807224085\\
22.38	0.00225148624815334\\
22.39	0.00225199474102471\\
22.4	0.00225250355120634\\
22.41	0.00225301267905026\\
22.42	0.00225352212490915\\
22.43	0.00225403188913636\\
22.44	0.00225454197208588\\
22.45	0.00225505237411236\\
22.46	0.00225556309557114\\
22.47	0.00225607413681819\\
22.48	0.00225658549821017\\
22.49	0.0022570971801044\\
22.5	0.00225760918285885\\
22.51	0.0022581215068322\\
22.52	0.00225863415238375\\
22.53	0.00225914711987352\\
22.54	0.00225966040966219\\
22.55	0.0022601740221111\\
22.56	0.00226068795758229\\
22.57	0.00226120221643847\\
22.58	0.00226171679904303\\
22.59	0.00226223170576006\\
22.6	0.00226274693695432\\
22.61	0.00226326249299125\\
22.62	0.002263778374237\\
22.63	0.00226429458105839\\
22.64	0.00226481111382295\\
22.65	0.00226532797289889\\
22.66	0.00226584515865513\\
22.67	0.00226636267146128\\
22.68	0.00226688051168763\\
22.69	0.00226739867970521\\
22.7	0.00226791717588571\\
22.71	0.00226843600060157\\
22.72	0.00226895515422589\\
22.73	0.00226947463713251\\
22.74	0.00226999444969598\\
22.75	0.00227051459229153\\
22.76	0.00227103506529513\\
22.77	0.00227155586908347\\
22.78	0.00227207700403394\\
22.79	0.00227259847052465\\
22.8	0.00227312026893445\\
22.81	0.00227364239964287\\
22.82	0.00227416486303022\\
22.83	0.00227468765947748\\
22.84	0.00227521078936641\\
22.85	0.00227573425307946\\
22.86	0.00227625805099983\\
22.87	0.00227678218351145\\
22.88	0.00227730665099898\\
22.89	0.00227783145384781\\
22.9	0.0022783565924441\\
22.91	0.00227888206717472\\
22.92	0.00227940787842729\\
22.93	0.00227993402659017\\
22.94	0.00228046051205249\\
22.95	0.0022809873352041\\
22.96	0.00228151449643561\\
22.97	0.00228204199613839\\
22.98	0.00228256983470455\\
22.99	0.00228309801252696\\
23	0.00228362652999924\\
23.01	0.0022841553875158\\
23.02	0.00228468458547178\\
23.03	0.00228521412426309\\
23.04	0.00228574400428642\\
23.05	0.0022862742259392\\
23.06	0.00228680478961966\\
23.07	0.00228733569572677\\
23.08	0.0022878669446603\\
23.09	0.00228839853682079\\
23.1	0.00228893047260953\\
23.11	0.00228946275242863\\
23.12	0.00228999537668094\\
23.13	0.00229052834577013\\
23.14	0.00229106166010064\\
23.15	0.00229159532007768\\
23.16	0.00229212932610728\\
23.17	0.00229266367859624\\
23.18	0.00229319837795215\\
23.19	0.0022937334245834\\
23.2	0.0022942688188992\\
23.21	0.00229480456130952\\
23.22	0.00229534065222516\\
23.23	0.0022958770920577\\
23.24	0.00229641388121955\\
23.25	0.00229695102012391\\
23.26	0.00229748850918478\\
23.27	0.00229802634881701\\
23.28	0.00229856453943622\\
23.29	0.00229910308145887\\
23.3	0.00229964197530223\\
23.31	0.0023001812213844\\
23.32	0.00230072082012427\\
23.33	0.0023012607719416\\
23.34	0.00230180107725694\\
23.35	0.00230234173649167\\
23.36	0.00230288275006803\\
23.37	0.00230342411840907\\
23.38	0.00230396584193867\\
23.39	0.00230450792108155\\
23.4	0.00230505035626327\\
23.41	0.00230559314791024\\
23.42	0.00230613629644971\\
23.43	0.00230667980230975\\
23.44	0.00230722366591932\\
23.45	0.0023077678877082\\
23.46	0.00230831246810703\\
23.47	0.00230885740754729\\
23.48	0.00230940270646135\\
23.49	0.00230994836528241\\
23.5	0.00231049438444453\\
23.51	0.00231104076438265\\
23.52	0.00231158750553256\\
23.53	0.00231213460833092\\
23.54	0.00231268207321526\\
23.55	0.002313229900624\\
23.56	0.0023137780909964\\
23.57	0.00231432664477262\\
23.58	0.0023148755623937\\
23.59	0.00231542484430153\\
23.6	0.00231597449093892\\
23.61	0.00231652450274955\\
23.62	0.002317074880178\\
23.63	0.0023176256236697\\
23.64	0.00231817673367102\\
23.65	0.00231872821062921\\
23.66	0.00231928005499239\\
23.67	0.00231983226720962\\
23.68	0.00232038484773084\\
23.69	0.00232093779700688\\
23.7	0.0023214911154895\\
23.71	0.00232204480363137\\
23.72	0.00232259886188604\\
23.73	0.00232315329070801\\
23.74	0.00232370809055267\\
23.75	0.00232426326187635\\
23.76	0.00232481880513628\\
23.77	0.00232537472079061\\
23.78	0.00232593100929845\\
23.79	0.00232648767111979\\
23.8	0.00232704470671559\\
23.81	0.00232760211654771\\
23.82	0.00232815990107898\\
23.83	0.00232871806077312\\
23.84	0.00232927659609485\\
23.85	0.00232983550750977\\
23.86	0.00233039479548446\\
23.87	0.00233095446048646\\
23.88	0.00233151450298422\\
23.89	0.00233207492344716\\
23.9	0.00233263572234568\\
23.91	0.00233319690015109\\
23.92	0.00233375845733569\\
23.93	0.00233432039437274\\
23.94	0.00233488271173646\\
23.95	0.00233544540990203\\
23.96	0.00233600848934561\\
23.97	0.00233657195054434\\
23.98	0.00233713579397631\\
23.99	0.0023377000201206\\
24	0.00233826462945727\\
24.01	0.00233882962246737\\
24.02	0.00233939499963293\\
24.03	0.00233996076143696\\
24.04	0.00234052690836346\\
24.05	0.00234109344089744\\
24.06	0.00234166035952488\\
24.07	0.00234222766473279\\
24.08	0.00234279535700915\\
24.09	0.00234336343684295\\
24.1	0.00234393190472421\\
24.11	0.00234450076114392\\
24.12	0.00234507000659411\\
24.13	0.00234563964156782\\
24.14	0.00234620966655909\\
24.15	0.00234678008206299\\
24.16	0.00234735088857563\\
24.17	0.00234792208659412\\
24.18	0.0023484936766166\\
24.19	0.00234906565914225\\
24.2	0.00234963803467127\\
24.21	0.00235021080370491\\
24.22	0.00235078396674546\\
24.23	0.00235135752429622\\
24.24	0.00235193147686157\\
24.25	0.00235250582494693\\
24.26	0.00235308056905875\\
24.27	0.00235365570970454\\
24.28	0.00235423124739288\\
24.29	0.00235480718263339\\
24.3	0.00235538351593677\\
24.31	0.00235596024781475\\
24.32	0.00235653737878017\\
24.33	0.00235711490934689\\
24.34	0.00235769284002989\\
24.35	0.00235827117134519\\
24.36	0.00235884990380991\\
24.37	0.00235942903794222\\
24.38	0.00236000857426141\\
24.39	0.00236058851328784\\
24.4	0.00236116885554295\\
24.41	0.00236174960154929\\
24.42	0.00236233075183048\\
24.43	0.00236291230691127\\
24.44	0.00236349426731748\\
24.45	0.00236407663357606\\
24.46	0.00236465940621505\\
24.47	0.00236524258576359\\
24.48	0.00236582617275197\\
24.49	0.00236641016771155\\
24.5	0.00236699457117485\\
24.51	0.00236757938367547\\
24.52	0.00236816460574818\\
24.53	0.00236875023792883\\
24.54	0.00236933628075444\\
24.55	0.00236992273476314\\
24.56	0.00237050960049422\\
24.57	0.00237109687848807\\
24.58	0.00237168456928626\\
24.59	0.00237227267343149\\
24.6	0.00237286119146762\\
24.61	0.00237345012393964\\
24.62	0.0023740394713937\\
24.63	0.00237462923437714\\
24.64	0.00237521941343842\\
24.65	0.00237581000912719\\
24.66	0.00237640102199426\\
24.67	0.00237699245259161\\
24.68	0.00237758430147239\\
24.69	0.00237817656919093\\
24.7	0.00237876925630275\\
24.71	0.00237936236336453\\
24.72	0.00237995589093416\\
24.73	0.0023805498395707\\
24.74	0.00238114420983443\\
24.75	0.0023817390022868\\
24.76	0.00238233421749046\\
24.77	0.00238292985600929\\
24.78	0.00238352591840835\\
24.79	0.00238412240525391\\
24.8	0.00238471931711346\\
24.81	0.00238531665455571\\
24.82	0.00238591441815057\\
24.83	0.00238651260846921\\
24.84	0.00238711122608398\\
24.85	0.00238771027156848\\
24.86	0.00238830974549756\\
24.87	0.00238890964844727\\
24.88	0.00238950998099492\\
24.89	0.00239011074371907\\
24.9	0.0023907119371995\\
24.91	0.00239131356201726\\
24.92	0.00239191561875465\\
24.93	0.00239251810799521\\
24.94	0.00239312103032377\\
24.95	0.00239372438632639\\
24.96	0.0023943281765904\\
24.97	0.00239493240170443\\
24.98	0.00239553706225835\\
24.99	0.00239614215884332\\
25	0.00239674769205179\\
25.01	0.00239735366247747\\
25.02	0.00239796007071537\\
25.03	0.00239856691736179\\
25.04	0.00239917420301432\\
25.05	0.00239978192827187\\
25.06	0.00240039009373461\\
25.07	0.00240099870000404\\
25.08	0.00240160774768298\\
25.09	0.00240221723737553\\
25.1	0.00240282716968712\\
25.11	0.00240343754522451\\
25.12	0.00240404836459578\\
25.13	0.00240465962841031\\
25.14	0.00240527133727885\\
25.15	0.00240588349181345\\
25.16	0.00240649609262751\\
25.17	0.00240710914033578\\
25.18	0.00240772263555433\\
25.19	0.00240833657890059\\
25.2	0.00240895097099336\\
25.21	0.00240956581245278\\
25.22	0.00241018110390034\\
25.23	0.00241079684595891\\
25.24	0.00241141303925271\\
25.25	0.00241202968440735\\
25.26	0.0024126467820498\\
25.27	0.0024132643328084\\
25.28	0.00241388233731291\\
25.29	0.00241450079619442\\
25.3	0.00241511971008545\\
25.31	0.00241573907961991\\
25.32	0.00241635890543309\\
25.33	0.00241697918816169\\
25.34	0.00241759992844383\\
25.35	0.002418221126919\\
25.36	0.00241884278422815\\
25.37	0.00241946490101359\\
25.38	0.00242008747791912\\
25.39	0.0024207105155899\\
25.4	0.00242133401467255\\
25.41	0.00242195797581513\\
25.42	0.00242258239966712\\
25.43	0.00242320728687944\\
25.44	0.00242383263810446\\
25.45	0.00242445845399599\\
25.46	0.00242508473520931\\
25.47	0.00242571148240114\\
25.48	0.00242633869622966\\
25.49	0.00242696637735452\\
25.5	0.00242759452643684\\
25.51	0.00242822314413921\\
25.52	0.0024288522311257\\
25.53	0.00242948178806186\\
25.54	0.0024301118156147\\
25.55	0.00243074231445276\\
25.56	0.00243137328524604\\
25.57	0.00243200472866607\\
25.58	0.00243263664538584\\
25.59	0.00243326903607988\\
25.6	0.00243390190142421\\
25.61	0.00243453524209637\\
25.62	0.00243516905877542\\
25.63	0.00243580335214195\\
25.64	0.00243643812287805\\
25.65	0.00243707337166738\\
25.66	0.00243770909919509\\
25.67	0.0024383453061479\\
25.68	0.00243898199321407\\
25.69	0.00243961916108341\\
25.7	0.00244025681044726\\
25.71	0.00244089494199855\\
25.72	0.00244153355643174\\
25.73	0.00244217265444287\\
25.74	0.00244281223672956\\
25.75	0.00244345230399098\\
25.76	0.00244409285692791\\
25.77	0.00244473389624267\\
25.78	0.0024453754226392\\
25.79	0.00244601743682303\\
25.8	0.00244665993950128\\
25.81	0.00244730293138266\\
25.82	0.00244794641317751\\
25.83	0.00244859038559775\\
25.84	0.00244923484935693\\
25.85	0.00244987980517024\\
25.86	0.00245052525375444\\
25.87	0.00245117119582797\\
25.88	0.00245181763211089\\
25.89	0.00245246456332487\\
25.9	0.00245311199019324\\
25.91	0.00245375991344099\\
25.92	0.00245440833379474\\
25.93	0.00245505725198278\\
25.94	0.00245570666873505\\
25.95	0.00245635658478315\\
25.96	0.00245700700086037\\
25.97	0.00245765791770166\\
25.98	0.00245830933604365\\
25.99	0.00245896125662465\\
26	0.00245961368018468\\
26.01	0.00246026660746542\\
26.02	0.00246092003921027\\
26.03	0.00246157397616433\\
26.04	0.0024622284190744\\
26.05	0.00246288336868901\\
26.06	0.00246353882575839\\
26.07	0.00246419479103449\\
26.08	0.002464851265271\\
26.09	0.00246550824922334\\
26.1	0.00246616574364866\\
26.11	0.00246682374930586\\
26.12	0.00246748226695558\\
26.13	0.0024681412973602\\
26.14	0.00246880084128389\\
26.15	0.00246946089949255\\
26.16	0.00247012147275386\\
26.17	0.00247078256183728\\
26.18	0.00247144416751403\\
26.19	0.00247210629055712\\
26.2	0.00247276893174134\\
26.21	0.00247343209184329\\
26.22	0.00247409577164134\\
26.23	0.00247475997191569\\
26.24	0.00247542469344832\\
26.25	0.00247608993702304\\
26.26	0.00247675570342549\\
26.27	0.00247742199344309\\
26.28	0.00247808880786513\\
26.29	0.00247875614748272\\
26.3	0.00247942401308879\\
26.31	0.00248009240547815\\
26.32	0.00248076132544743\\
26.33	0.00248143077379512\\
26.34	0.00248210075132156\\
26.35	0.00248277125882899\\
26.36	0.00248344229712148\\
26.37	0.00248411386700499\\
26.38	0.00248478596928738\\
26.39	0.00248545860477836\\
26.4	0.00248613177428955\\
26.41	0.00248680547863448\\
26.42	0.00248747971862857\\
26.43	0.00248815449508914\\
26.44	0.00248882980883542\\
26.45	0.00248950566068859\\
26.46	0.00249018205147172\\
26.47	0.00249085898200983\\
26.48	0.00249153645312987\\
26.49	0.00249221446566072\\
26.5	0.00249289302043323\\
26.51	0.00249357211828018\\
26.52	0.00249425176003632\\
26.53	0.00249493194653835\\
26.54	0.00249561267862496\\
26.55	0.0024962939571368\\
26.56	0.0024969757829165\\
26.57	0.00249765815680868\\
26.58	0.00249834107965995\\
26.59	0.00249902455231892\\
26.6	0.00249970857563619\\
26.61	0.0025003931504644\\
26.62	0.00250107827765818\\
26.63	0.00250176395807417\\
26.64	0.00250245019257107\\
26.65	0.00250313698200959\\
26.66	0.00250382432725248\\
26.67	0.00250451222916454\\
26.68	0.00250520068861261\\
26.69	0.00250588970646559\\
26.7	0.00250657928359446\\
26.71	0.00250726942087223\\
26.72	0.00250796011917401\\
26.73	0.002508651379377\\
26.74	0.00250934320236047\\
26.75	0.00251003558900576\\
26.76	0.00251072854019635\\
26.77	0.0025114220568178\\
26.78	0.00251211613975778\\
26.79	0.00251281078990609\\
26.8	0.00251350600815463\\
26.81	0.00251420179539745\\
26.82	0.00251489815253072\\
26.83	0.00251559508045276\\
26.84	0.00251629258006403\\
26.85	0.00251699065226715\\
26.86	0.00251768929796689\\
26.87	0.0025183885180702\\
26.88	0.0025190883134862\\
26.89	0.00251978868512617\\
26.9	0.00252048963390358\\
26.91	0.00252119116073412\\
26.92	0.00252189326653564\\
26.93	0.00252259595222822\\
26.94	0.00252329921873412\\
26.95	0.00252400306697786\\
26.96	0.00252470749788614\\
26.97	0.00252541251238793\\
26.98	0.0025261181114144\\
26.99	0.00252682429589898\\
27	0.00252753106677736\\
27.01	0.00252823842498746\\
27.02	0.00252894637146948\\
27.03	0.00252965490716588\\
27.04	0.00253036403302141\\
27.05	0.00253107374998308\\
27.06	0.00253178405900022\\
27.07	0.00253249496102442\\
27.08	0.00253320645700959\\
27.09	0.00253391854791197\\
27.1	0.00253463123469008\\
27.11	0.00253534451830477\\
27.12	0.00253605839971924\\
27.13	0.00253677287989901\\
27.14	0.00253748795981194\\
27.15	0.00253820364042826\\
27.16	0.00253891992272053\\
27.17	0.0025396368076637\\
27.18	0.00254035429623507\\
27.19	0.00254107238941432\\
27.2	0.00254179108818353\\
27.21	0.00254251039352717\\
27.22	0.00254323030643209\\
27.23	0.00254395082788756\\
27.24	0.00254467195888526\\
27.25	0.00254539370041929\\
27.26	0.00254611605348618\\
27.27	0.0025468390190849\\
27.28	0.00254756259821683\\
27.29	0.00254828679188585\\
27.3	0.00254901160109825\\
27.31	0.00254973702686281\\
27.32	0.00255046307019077\\
27.33	0.00255118973209584\\
27.34	0.00255191701359423\\
27.35	0.00255264491570464\\
27.36	0.00255337343944827\\
27.37	0.00255410258584882\\
27.38	0.00255483235593251\\
27.39	0.00255556275072807\\
27.4	0.00255629377126677\\
27.41	0.00255702541858244\\
27.42	0.0025577576937114\\
27.43	0.00255849059769257\\
27.44	0.00255922413156742\\
27.45	0.00255995829637996\\
27.46	0.0025606930931768\\
27.47	0.00256142852300714\\
27.48	0.00256216458692274\\
27.49	0.00256290128597798\\
27.5	0.00256363862122984\\
27.51	0.00256437659373791\\
27.52	0.00256511520456441\\
27.53	0.00256585445477418\\
27.54	0.00256659434543469\\
27.55	0.00256733487761607\\
27.56	0.0025680760523911\\
27.57	0.00256881787083521\\
27.58	0.0025695603340265\\
27.59	0.00257030344304576\\
27.6	0.00257104719897643\\
27.61	0.0025717916029047\\
27.62	0.00257253665591939\\
27.63	0.00257328235911208\\
27.64	0.00257402871357705\\
27.65	0.00257477572041129\\
27.66	0.00257552338071455\\
27.67	0.0025762716955893\\
27.68	0.00257702066614076\\
27.69	0.00257777029347692\\
27.7	0.00257852057870851\\
27.71	0.00257927152294907\\
27.72	0.00258002312731488\\
27.73	0.00258077539292504\\
27.74	0.00258152832090144\\
27.75	0.00258228191236877\\
27.76	0.00258303616845454\\
27.77	0.00258379109028908\\
27.78	0.00258454667900557\\
27.79	0.00258530293574001\\
27.8	0.00258605986163125\\
27.81	0.00258681745782099\\
27.82	0.00258757572545383\\
27.83	0.00258833466567721\\
27.84	0.00258909427964147\\
27.85	0.00258985456849983\\
27.86	0.00259061553340843\\
27.87	0.0025913771755263\\
27.88	0.0025921394960154\\
27.89	0.00259290249604059\\
27.9	0.00259366617676971\\
27.91	0.00259443053937352\\
27.92	0.00259519558502572\\
27.93	0.00259596131490301\\
27.94	0.00259672773018501\\
27.95	0.00259749483205436\\
27.96	0.00259826262169668\\
27.97	0.00259903110030059\\
27.98	0.0025998002690577\\
27.99	0.00260057012916264\\
28	0.00260134068181309\\
28.01	0.00260211192820974\\
28.02	0.00260288386955633\\
28.03	0.00260365650705965\\
28.04	0.00260442984192955\\
28.05	0.00260520387537896\\
28.06	0.00260597860862388\\
28.07	0.00260675404288342\\
28.08	0.00260753017937976\\
28.09	0.00260830701933822\\
28.1	0.00260908456398719\\
28.11	0.00260986281455824\\
28.12	0.00261064177228604\\
28.13	0.00261142143840844\\
28.14	0.0026122018141664\\
28.15	0.00261298290080409\\
28.16	0.00261376469956882\\
28.17	0.00261454721171112\\
28.18	0.00261533043848467\\
28.19	0.00261611438114639\\
28.2	0.00261689904095641\\
28.21	0.00261768441917806\\
28.22	0.00261847051707792\\
28.23	0.00261925733592581\\
28.24	0.00262004487699481\\
28.25	0.00262083314156124\\
28.26	0.00262162213090471\\
28.27	0.00262241184630813\\
28.28	0.00262320228905766\\
28.29	0.00262399346044279\\
28.3	0.00262478536175633\\
28.31	0.00262557799429439\\
28.32	0.00262637135935642\\
28.33	0.00262716545824521\\
28.34	0.00262796029226692\\
28.35	0.00262875586273106\\
28.36	0.00262955217095051\\
28.37	0.00263034921824154\\
28.38	0.00263114700592381\\
28.39	0.0026319455353204\\
28.4	0.00263274480775777\\
28.41	0.00263354482456585\\
28.42	0.00263434558707797\\
28.43	0.00263514709663092\\
28.44	0.00263594935456495\\
28.45	0.00263675236222376\\
28.46	0.00263755593112226\\
28.47	0.0026383599304226\\
28.48	0.00263916436051276\\
28.49	0.00263996922178145\\
28.5	0.00264077451461806\\
28.51	0.00264158023941267\\
28.52	0.0026423863965561\\
28.53	0.00264319298643986\\
28.54	0.00264400000945617\\
28.55	0.00264480746599796\\
28.56	0.00264561535645889\\
28.57	0.00264642368123331\\
28.58	0.00264723244071632\\
28.59	0.00264804163530372\\
28.6	0.00264885126539204\\
28.61	0.00264966133137853\\
28.62	0.00265047183366116\\
28.63	0.00265128277263865\\
28.64	0.00265209414871043\\
28.65	0.00265290596227668\\
28.66	0.0026537182137383\\
28.67	0.00265453090349694\\
28.68	0.00265534403195498\\
28.69	0.00265615759951555\\
28.7	0.00265697160658251\\
28.71	0.00265778605356049\\
28.72	0.00265860094085484\\
28.73	0.00265941626887169\\
28.74	0.00266023203801789\\
28.75	0.00266104824870106\\
28.76	0.00266186490132959\\
28.77	0.00266268199631261\\
28.78	0.00266349953406002\\
28.79	0.00266431751498248\\
28.8	0.00266513593949142\\
28.81	0.00266595480799903\\
28.82	0.00266677412091829\\
28.83	0.00266759387866291\\
28.84	0.00266841408164742\\
28.85	0.00266923473028711\\
28.86	0.00267005582499804\\
28.87	0.00267087736619707\\
28.88	0.00267169935430182\\
28.89	0.00267252178973072\\
28.9	0.00267334467290298\\
28.91	0.00267416800423859\\
28.92	0.00267499178415834\\
28.93	0.00267581601308383\\
28.94	0.00267664069143744\\
28.95	0.00267746581964234\\
28.96	0.00267829139812253\\
28.97	0.00267911742730279\\
28.98	0.00267994390760874\\
28.99	0.00268077083946677\\
29	0.0026815982233041\\
29.01	0.00268242605954877\\
29.02	0.00268325434862963\\
29.03	0.00268408309097635\\
29.04	0.00268491228701941\\
29.05	0.00268574193719014\\
29.06	0.00268657204192068\\
29.07	0.00268740260164399\\
29.08	0.00268823361679387\\
29.09	0.00268906508780496\\
29.1	0.00268989701511273\\
29.11	0.0026907293991535\\
29.12	0.0026915622403644\\
29.13	0.00269239553918345\\
29.14	0.00269322929604947\\
29.15	0.00269406351140215\\
29.16	0.00269489818568204\\
29.17	0.00269573331933053\\
29.18	0.00269656891278986\\
29.19	0.00269740496650315\\
29.2	0.00269824148091437\\
29.21	0.00269907845646835\\
29.22	0.00269991589361079\\
29.23	0.00270075379278827\\
29.24	0.00270159215444821\\
29.25	0.00270243097903894\\
29.26	0.00270327026700965\\
29.27	0.00270411001881041\\
29.28	0.00270495023489219\\
29.29	0.0027057909157068\\
29.3	0.00270663206170699\\
29.31	0.00270747367334636\\
29.32	0.00270831575107943\\
29.33	0.00270915829536161\\
29.34	0.00271000130664918\\
29.35	0.00271084478539937\\
29.36	0.00271168873207027\\
29.37	0.00271253314712089\\
29.38	0.00271337803101118\\
29.39	0.00271422338420194\\
29.4	0.00271506920715493\\
29.41	0.00271591550033283\\
29.42	0.00271676226419921\\
29.43	0.00271760949921858\\
29.44	0.00271845720585638\\
29.45	0.00271930538457897\\
29.46	0.00272015403585365\\
29.47	0.00272100316014864\\
29.48	0.0027218527579331\\
29.49	0.00272270282967715\\
29.5	0.00272355337585182\\
29.51	0.00272440439692911\\
29.52	0.00272525589338197\\
29.53	0.00272610786568428\\
29.54	0.00272696031431088\\
29.55	0.00272781323973759\\
29.56	0.00272866664244115\\
29.57	0.00272952052289931\\
29.58	0.00273037488159075\\
29.59	0.00273122971899512\\
29.6	0.00273208503559307\\
29.61	0.00273294083186618\\
29.62	0.00273379710829705\\
29.63	0.00273465386536925\\
29.64	0.00273551110356732\\
29.65	0.00273636882337678\\
29.66	0.00273722702528417\\
29.67	0.002738085709777\\
29.68	0.00273894487734377\\
29.69	0.00273980452847401\\
29.7	0.00274066466365822\\
29.71	0.00274152528338792\\
29.72	0.00274238638815563\\
29.73	0.00274324797845488\\
29.74	0.00274411005478022\\
29.75	0.00274497261762723\\
29.76	0.00274583566749249\\
29.77	0.0027466992048736\\
29.78	0.0027475632302692\\
29.79	0.00274842774417897\\
29.8	0.00274929274710359\\
29.81	0.0027501582395448\\
29.82	0.00275102422200537\\
29.83	0.00275189069498913\\
29.84	0.00275275765900093\\
29.85	0.00275362511454669\\
29.86	0.00275449306213336\\
29.87	0.00275536150226898\\
29.88	0.00275623043546261\\
29.89	0.00275709986222439\\
29.9	0.00275796978306553\\
29.91	0.0027588401984983\\
29.92	0.00275971110903604\\
29.93	0.00276058251519318\\
29.94	0.00276145441748522\\
29.95	0.00276232681642873\\
29.96	0.00276319971254137\\
29.97	0.00276407310634191\\
29.98	0.00276494699835019\\
29.99	0.00276582138908715\\
30	0.00276669627907483\\
30.01	0.00276757166883638\\
30.02	0.00276844755889603\\
30.03	0.00276932394977916\\
30.04	0.00277020084201222\\
30.05	0.0027710782361228\\
30.06	0.00277195613263961\\
30.07	0.00277283453209247\\
30.08	0.00277371343501234\\
30.09	0.00277459284193131\\
30.1	0.00277547275338259\\
30.11	0.00277635316990053\\
30.12	0.00277723409202063\\
30.13	0.00277811552027953\\
30.14	0.00277899745521502\\
30.15	0.00277987989736604\\
30.16	0.00278076284727268\\
30.17	0.00278164630547619\\
30.18	0.00278253027251899\\
30.19	0.00278341474894466\\
30.2	0.00278429973529795\\
30.21	0.00278518523212478\\
30.22	0.00278607123997225\\
30.23	0.00278695775938865\\
30.24	0.00278784479092343\\
30.25	0.00278873233512725\\
30.26	0.00278962039255196\\
30.27	0.0027905089637506\\
30.28	0.00279139804927741\\
30.29	0.00279228764968783\\
30.3	0.00279317776553851\\
30.31	0.00279406839738731\\
30.32	0.00279495954579331\\
30.33	0.0027958512113168\\
30.34	0.0027967433945193\\
30.35	0.00279763609596355\\
30.36	0.00279852931621352\\
30.37	0.00279942305583442\\
30.38	0.00280031731539269\\
30.39	0.00280121209545603\\
30.4	0.00280210739659335\\
30.41	0.00280300321937485\\
30.42	0.00280389956437195\\
30.43	0.00280479643215736\\
30.44	0.00280569382330501\\
30.45	0.00280659173839014\\
30.46	0.00280749017798924\\
30.47	0.00280838914268006\\
30.48	0.00280928863304164\\
30.49	0.00281018864965431\\
30.5	0.00281108919309968\\
30.51	0.00281199026396065\\
30.52	0.00281289186282141\\
30.53	0.00281379399026745\\
30.54	0.00281469664688556\\
30.55	0.00281559983326384\\
30.56	0.0028165035499917\\
30.57	0.00281740779765988\\
30.58	0.0028183125768604\\
30.59	0.00281921788818665\\
30.6	0.00282012373223331\\
30.61	0.00282103010959642\\
30.62	0.00282193702087333\\
30.63	0.00282284446666276\\
30.64	0.00282375244756475\\
30.65	0.00282466096418069\\
30.66	0.00282557001711334\\
30.67	0.0028264796069668\\
30.68	0.00282738973434656\\
30.69	0.00282830039985943\\
30.7	0.00282921160411364\\
30.71	0.00283012334771876\\
30.72	0.00283103563128576\\
30.73	0.00283194845542699\\
30.74	0.00283286182075617\\
30.75	0.00283377572788844\\
30.76	0.00283469017744033\\
30.77	0.00283560517002975\\
30.78	0.00283652070627605\\
30.79	0.00283743678679998\\
30.8	0.00283835341222369\\
30.81	0.00283927058317077\\
30.82	0.00284018830026622\\
30.83	0.00284110656413647\\
30.84	0.0028420253754094\\
30.85	0.00284294473471432\\
30.86	0.00284386464268199\\
30.87	0.00284478509994458\\
30.88	0.00284570610713578\\
30.89	0.00284662766489067\\
30.9	0.00284754977384583\\
30.91	0.00284847243463932\\
30.92	0.00284939564791063\\
30.93	0.00285031941430075\\
30.94	0.00285124373445216\\
30.95	0.0028521686090088\\
30.96	0.00285309403861612\\
30.97	0.00285402002392108\\
30.98	0.0028549465655721\\
30.99	0.00285587366421914\\
31	0.00285680132051366\\
31.01	0.00285772953510862\\
31.02	0.00285865830865853\\
31.03	0.00285958764181941\\
31.04	0.0028605175352488\\
31.05	0.00286144798960578\\
31.06	0.002862379005551\\
31.07	0.00286331058374661\\
31.08	0.00286424272485634\\
31.09	0.00286517542954546\\
31.1	0.00286610869848081\\
31.11	0.0028670425323308\\
31.12	0.0028679769317654\\
31.13	0.00286891189745616\\
31.14	0.00286984743007621\\
31.15	0.00287078353030026\\
31.16	0.00287172019880463\\
31.17	0.00287265743626722\\
31.18	0.00287359524336753\\
31.19	0.0028745336207867\\
31.2	0.00287547256920743\\
31.21	0.00287641208931407\\
31.22	0.0028773521817926\\
31.23	0.00287829284733061\\
31.24	0.00287923408661733\\
31.25	0.00288017590034363\\
31.26	0.00288111828920203\\
31.27	0.00288206125388667\\
31.28	0.0028830047950934\\
31.29	0.00288394891351969\\
31.3	0.00288489360986469\\
31.31	0.0028858388848292\\
31.32	0.00288678473911574\\
31.33	0.00288773117342849\\
31.34	0.00288867818847331\\
31.35	0.00288962578495775\\
31.36	0.00289057396359109\\
31.37	0.00289152272508429\\
31.38	0.00289247207015004\\
31.39	0.00289342199950272\\
31.4	0.00289437251385845\\
31.41	0.00289532361393508\\
31.42	0.00289627530045219\\
31.43	0.0028972275741311\\
31.44	0.00289818043569487\\
31.45	0.00289913388586831\\
31.46	0.00290008792537801\\
31.47	0.0029010425549523\\
31.48	0.00290199777532128\\
31.49	0.00290295358721682\\
31.5	0.00290390999137258\\
31.51	0.00290486698852403\\
31.52	0.00290582457940838\\
31.53	0.00290678276476468\\
31.54	0.00290774154533377\\
31.55	0.0029087009218583\\
31.56	0.00290966089508273\\
31.57	0.00291062146575335\\
31.58	0.00291158263461829\\
31.59	0.00291254440242749\\
31.6	0.00291350676993275\\
31.61	0.00291446973788771\\
31.62	0.00291543330704785\\
31.63	0.00291639747817055\\
31.64	0.002917362252015\\
31.65	0.0029183276293423\\
31.66	0.00291929361091542\\
31.67	0.00292026019749921\\
31.68	0.00292122738986041\\
31.69	0.00292219518876766\\
31.7	0.00292316359499151\\
31.71	0.0029241326093044\\
31.72	0.00292510223248071\\
31.73	0.00292607246529672\\
31.74	0.00292704330853066\\
31.75	0.00292801476296267\\
31.76	0.00292898682937486\\
31.77	0.00292995950855127\\
31.78	0.0029309328012779\\
31.79	0.00293190670834272\\
31.8	0.00293288123053564\\
31.81	0.00293385636864859\\
31.82	0.00293483212347544\\
31.83	0.00293580849581206\\
31.84	0.00293678548645634\\
31.85	0.00293776309620813\\
31.86	0.00293874132586933\\
31.87	0.00293972017624381\\
31.88	0.00294069964813751\\
31.89	0.00294167974235836\\
31.9	0.00294266045971634\\
31.91	0.00294364180102348\\
31.92	0.00294462376709385\\
31.93	0.00294560635874359\\
31.94	0.00294658957679086\\
31.95	0.00294757342205595\\
31.96	0.0029485578953612\\
31.97	0.00294954299753102\\
31.98	0.00295052872939192\\
31.99	0.00295151509177254\\
32	0.00295250208550357\\
32.01	0.00295348971141785\\
32.02	0.00295447797035035\\
32.03	0.00295546686313812\\
32.04	0.00295645639062039\\
32.05	0.00295744655363851\\
32.06	0.00295843735303598\\
32.07	0.00295942878965846\\
32.08	0.00296042086435377\\
32.09	0.00296141357797192\\
32.1	0.00296240693136506\\
32.11	0.00296304776470471\\
32.12	0.00296349143659751\\
32.13	0.00296393529352657\\
32.14	0.0029643793355601\\
32.15	0.0029648235627663\\
32.16	0.00296526797521342\\
32.17	0.00296571257296967\\
32.18	0.0029661573561033\\
32.19	0.00296660232468256\\
32.2	0.00296704747877571\\
32.21	0.00296749281845105\\
32.22	0.00296793834377684\\
32.23	0.00296838405482139\\
32.24	0.00296882995165302\\
32.25	0.00296927603434003\\
32.26	0.00296972230295077\\
32.27	0.00297016875755357\\
32.28	0.0029706153982168\\
32.29	0.00297106222500881\\
32.3	0.00297150923799799\\
32.31	0.00297195643725272\\
32.32	0.00297240382284143\\
32.33	0.00297285139483251\\
32.34	0.0029732991532944\\
32.35	0.00297374709829552\\
32.36	0.00297419522990433\\
32.37	0.00297464354818931\\
32.38	0.00297509205321891\\
32.39	0.00297554074506163\\
32.4	0.00297598962378598\\
32.41	0.00297643868946047\\
32.42	0.00297688794215362\\
32.43	0.00297733738193397\\
32.44	0.00297778700887009\\
32.45	0.00297823682303053\\
32.46	0.00297868682448387\\
32.47	0.00297913701329872\\
32.48	0.00297958738954366\\
32.49	0.00298003795328733\\
32.5	0.00298048870459836\\
32.51	0.00298093964354539\\
32.52	0.00298139077019709\\
32.53	0.00298184208462213\\
32.54	0.0029822935868892\\
32.55	0.00298274527706701\\
32.56	0.00298319715522428\\
32.57	0.00298364922142973\\
32.58	0.00298410147575211\\
32.59	0.0029845539182602\\
32.6	0.00298500654902275\\
32.61	0.00298545936810857\\
32.62	0.00298591237558646\\
32.63	0.00298636557152525\\
32.64	0.00298681895599377\\
32.65	0.00298727252906088\\
32.66	0.00298772629079544\\
32.67	0.00298818024126633\\
32.68	0.00298863438054246\\
32.69	0.00298908870869274\\
32.7	0.00298954322578611\\
32.71	0.00298999793189151\\
32.72	0.0029904528270779\\
32.73	0.00299090791141428\\
32.74	0.00299136318496963\\
32.75	0.00299181864781296\\
32.76	0.0029922743000133\\
32.77	0.00299273014163972\\
32.78	0.00299318617276127\\
32.79	0.00299364239344702\\
32.8	0.00299409880376609\\
32.81	0.00299455540378758\\
32.82	0.00299501219358064\\
32.83	0.00299546917321442\\
32.84	0.00299592634275807\\
32.85	0.0029963837022808\\
32.86	0.0029968412518518\\
32.87	0.00299729899154032\\
32.88	0.00299775692141557\\
32.89	0.00299821504154683\\
32.9	0.00299867335200339\\
32.91	0.00299913185285453\\
32.92	0.00299959054416957\\
32.93	0.00300004942601786\\
32.94	0.00300050849846875\\
32.95	0.00300096776159162\\
32.96	0.00300142721545587\\
32.97	0.00300188686013091\\
32.98	0.00300234669568618\\
32.99	0.00300280672219113\\
33	0.00300326693971525\\
33.01	0.00300372734832802\\
33.02	0.00300418794809897\\
33.03	0.00300464873909763\\
33.04	0.00300510972139357\\
33.05	0.00300557089505637\\
33.06	0.00300603226015563\\
33.07	0.00300649381676098\\
33.08	0.00300695556494205\\
33.09	0.00300741750476852\\
33.1	0.00300787963631009\\
33.11	0.00300834195963644\\
33.12	0.00300880447481734\\
33.13	0.00300926718192252\\
33.14	0.00300973008102177\\
33.15	0.00301019317218491\\
33.16	0.00301065645548174\\
33.17	0.00301111993098211\\
33.18	0.0030115835987559\\
33.19	0.00301204745887302\\
33.2	0.00301251151140337\\
33.21	0.0030129757564169\\
33.22	0.00301344019398359\\
33.23	0.00301390482417341\\
33.24	0.00301436964705639\\
33.25	0.00301483466270258\\
33.26	0.00301529987118204\\
33.27	0.00301576527256485\\
33.28	0.00301623086692114\\
33.29	0.00301669665432105\\
33.3	0.00301716263483476\\
33.31	0.00301762880853246\\
33.32	0.00301809517548437\\
33.33	0.00301856173576073\\
33.34	0.00301902848943183\\
33.35	0.00301949543656796\\
33.36	0.00301996257723946\\
33.37	0.00302042991151667\\
33.38	0.00302089743946999\\
33.39	0.00302136516116982\\
33.4	0.0030218330766866\\
33.41	0.00302230118609081\\
33.42	0.00302276948945294\\
33.43	0.00302323798684351\\
33.44	0.00302370667833307\\
33.45	0.00302417556399221\\
33.46	0.00302464464389154\\
33.47	0.00302511391810169\\
33.48	0.00302558338669335\\
33.49	0.00302605304973721\\
33.5	0.003026522907304\\
33.51	0.00302699295946449\\
33.52	0.00302746320628946\\
33.53	0.00302793364784974\\
33.54	0.00302840428421618\\
33.55	0.00302887511545968\\
33.56	0.00302934614165114\\
33.57	0.00302981736286152\\
33.58	0.0030302887791618\\
33.59	0.00303076039062299\\
33.6	0.00303123219731613\\
33.61	0.00303170419931232\\
33.62	0.00303217639668266\\
33.63	0.0030326487894983\\
33.64	0.00303312137783041\\
33.65	0.00303359416175022\\
33.66	0.00303406714132896\\
33.67	0.00303454031663792\\
33.68	0.00303501368774842\\
33.69	0.00303548725473181\\
33.7	0.00303596101765948\\
33.71	0.00303643497660285\\
33.72	0.00303690913163338\\
33.73	0.00303738348282257\\
33.74	0.00303785803024195\\
33.75	0.00303833277396309\\
33.76	0.00303880771405758\\
33.77	0.00303928285059708\\
33.78	0.00303975818365326\\
33.79	0.00304023371329784\\
33.8	0.00304070943960258\\
33.81	0.00304118536263926\\
33.82	0.00304166148247972\\
33.83	0.00304213779919584\\
33.84	0.00304261431285953\\
33.85	0.00304309102354273\\
33.86	0.00304356793131744\\
33.87	0.00304404503625569\\
33.88	0.00304452233842954\\
33.89	0.00304499983791112\\
33.9	0.00304547753477256\\
33.91	0.00304595542908607\\
33.92	0.00304643352092389\\
33.93	0.00304691181035829\\
33.94	0.00304739029746159\\
33.95	0.00304786898230615\\
33.96	0.00304834786496438\\
33.97	0.00304882694550874\\
33.98	0.00304930622401171\\
33.99	0.00304978570054582\\
34	0.00305026537518367\\
34.01	0.00305074524799789\\
34.02	0.00305122531906113\\
34.03	0.00305170558844611\\
34.04	0.00305218605622561\\
34.05	0.00305266672247241\\
34.06	0.00305314758725939\\
34.07	0.00305362865065943\\
34.08	0.0030541099127455\\
34.09	0.00305459137359057\\
34.1	0.0030550730332677\\
34.11	0.00305555489184998\\
34.12	0.00305603694941054\\
34.13	0.00305651920602258\\
34.14	0.00305700166175933\\
34.15	0.00305748431669409\\
34.16	0.00305796717090017\\
34.17	0.00305845022445098\\
34.18	0.00305893347741995\\
34.19	0.00305941692988057\\
34.2	0.00305990058190639\\
34.21	0.00306038443357098\\
34.22	0.00306086848494801\\
34.23	0.00306135273611115\\
34.24	0.00306183718713417\\
34.25	0.00306232183809087\\
34.26	0.00306280668905509\\
34.27	0.00306329174010076\\
34.28	0.00306377699130184\\
34.29	0.00306426244273234\\
34.3	0.00306474809446634\\
34.31	0.00306523394657798\\
34.32	0.00306571999914144\\
34.33	0.00306620625223097\\
34.34	0.00306669270592086\\
34.35	0.00306717936028547\\
34.36	0.00306766621539922\\
34.37	0.00306815327133659\\
34.38	0.00306864052817209\\
34.39	0.00306912798598034\\
34.4	0.00306961564483597\\
34.41	0.0030701035048137\\
34.42	0.00307059156598829\\
34.43	0.00307107982843458\\
34.44	0.00307156829222747\\
34.45	0.0030720569574419\\
34.46	0.00307254582415289\\
34.47	0.00307303489243553\\
34.48	0.00307352416236495\\
34.49	0.00307401363401635\\
34.5	0.00307450330746502\\
34.51	0.00307499318278627\\
34.52	0.00307548326005551\\
34.53	0.0030759735393482\\
34.54	0.00307646402073988\\
34.55	0.00307695470430613\\
34.56	0.00307744559012262\\
34.57	0.00307793667826507\\
34.58	0.00307842796880928\\
34.59	0.00307891946183113\\
34.6	0.00307941115740653\\
34.61	0.0030799030556115\\
34.62	0.0030803951565221\\
34.63	0.00308088746021448\\
34.64	0.00308137996676485\\
34.65	0.00308187267624949\\
34.66	0.00308236558874476\\
34.67	0.00308285870432708\\
34.68	0.00308335202307296\\
34.69	0.00308384554505896\\
34.7	0.00308433927036172\\
34.71	0.00308483319905798\\
34.72	0.00308532733122452\\
34.73	0.00308582166693822\\
34.74	0.00308631620627601\\
34.75	0.00308681094931493\\
34.76	0.00308730589613206\\
34.77	0.00308780104680459\\
34.78	0.00308829640140976\\
34.79	0.00308879196002491\\
34.8	0.00308928772272746\\
34.81	0.00308978368959487\\
34.82	0.00309027986070474\\
34.83	0.0030907762361347\\
34.84	0.00309127281596248\\
34.85	0.00309176960026591\\
34.86	0.00309226658912286\\
34.87	0.00309276378261134\\
34.88	0.00309326118080939\\
34.89	0.00309375878379515\\
34.9	0.00309425659164686\\
34.91	0.00309475460444283\\
34.92	0.00309525282226146\\
34.93	0.00309575124518124\\
34.94	0.00309624987328073\\
34.95	0.00309674870663861\\
34.96	0.00309724774533362\\
34.97	0.00309774698944459\\
34.98	0.00309824643905045\\
34.99	0.00309874609423023\\
35	0.00309924595506301\\
35.01	0.003099746021628\\
35.02	0.00310024629400449\\
35.03	0.00310074677227186\\
35.04	0.00310124745650959\\
35.05	0.00310174834679725\\
35.06	0.00310224944321449\\
35.07	0.00310275074584108\\
35.08	0.00310325225475686\\
35.09	0.00310375397004179\\
35.1	0.0031042558917759\\
35.11	0.00310475802003936\\
35.12	0.00310526035491238\\
35.13	0.00310576289647531\\
35.14	0.0031062656448086\\
35.15	0.00310676859999278\\
35.16	0.00310727176210849\\
35.17	0.00310777513123647\\
35.18	0.00310827870745757\\
35.19	0.00310878249085274\\
35.2	0.00310928648150301\\
35.21	0.00310979067948956\\
35.22	0.00311029508489363\\
35.23	0.00311079969779658\\
35.24	0.0031113045182799\\
35.25	0.00311180954642516\\
35.26	0.00311231478231403\\
35.27	0.00311282022602833\\
35.28	0.00311332587764994\\
35.29	0.00311383173726088\\
35.3	0.00311433780494329\\
35.31	0.00311484408077938\\
35.32	0.00311535056485152\\
35.33	0.00311585725724214\\
35.34	0.00311636415803384\\
35.35	0.00311687126730929\\
35.36	0.00311737858515131\\
35.37	0.00311788611164281\\
35.38	0.00311839384686682\\
35.39	0.00311890179090649\\
35.4	0.0031194099438451\\
35.41	0.00311991830576603\\
35.42	0.0031204268767528\\
35.43	0.00312093565688903\\
35.44	0.00312144464625848\\
35.45	0.00312195384494501\\
35.46	0.00312246325303262\\
35.47	0.00312297287060544\\
35.48	0.00312348269774771\\
35.49	0.00312399273454379\\
35.5	0.00312450298107819\\
35.51	0.0031250134374355\\
35.52	0.0031255241037005\\
35.53	0.00312603497995805\\
35.54	0.00312654606629318\\
35.55	0.00312705736279101\\
35.56	0.00312756886953681\\
35.57	0.00312808058661598\\
35.58	0.00312859251411407\\
35.59	0.00312910465211674\\
35.6	0.00312961700070977\\
35.61	0.00313012955997912\\
35.62	0.00313064233001086\\
35.63	0.0031311553108912\\
35.64	0.00313166850270648\\
35.65	0.00313218190554321\\
35.66	0.00313269551948799\\
35.67	0.0031332093446276\\
35.68	0.00313372338104894\\
35.69	0.00313423762883908\\
35.7	0.0031347520880852\\
35.71	0.00313526675887466\\
35.72	0.00313578164129492\\
35.73	0.00313629673543362\\
35.74	0.00313681204137854\\
35.75	0.00313732755921761\\
35.76	0.00313784328903891\\
35.77	0.00313835923093064\\
35.78	0.0031388753849812\\
35.79	0.00313939175127911\\
35.8	0.00313990832991306\\
35.81	0.00314042512097187\\
35.82	0.00314094212454454\\
35.83	0.00314145934072021\\
35.84	0.00314197676958818\\
35.85	0.00314249441123792\\
35.86	0.00314301226575904\\
35.87	0.00314353033324132\\
35.88	0.00314404861377469\\
35.89	0.00314456710744926\\
35.9	0.00314508581435529\\
35.91	0.00314560473458321\\
35.92	0.00314612386822362\\
35.93	0.00314664321536726\\
35.94	0.00314716277610506\\
35.95	0.00314768255052813\\
35.96	0.0031482025387277\\
35.97	0.00314872274079523\\
35.98	0.00314924315682231\\
35.99	0.00314976378690072\\
36	0.00315028463112239\\
36.01	0.00315080568957947\\
36.02	0.00315132696236425\\
36.03	0.00315184844956919\\
36.04	0.00315237015128696\\
36.05	0.00315289206761037\\
36.06	0.00315341419863246\\
36.07	0.0031539365444464\\
36.08	0.00315445910514557\\
36.09	0.00315498188082352\\
36.1	0.003155504871574\\
36.11	0.00315602807749093\\
36.12	0.00315655149866843\\
36.13	0.0031570751352008\\
36.14	0.00315759898718253\\
36.15	0.00315812305470829\\
36.16	0.00315864733787297\\
36.17	0.00315917183677163\\
36.18	0.00315969655149951\\
36.19	0.00316022148215208\\
36.2	0.00316074662882498\\
36.21	0.00316127199161405\\
36.22	0.00316179757061535\\
36.23	0.00316232336592513\\
36.24	0.0031628493776398\\
36.25	0.00316337560585605\\
36.26	0.0031639020506707\\
36.27	0.00316442871218081\\
36.28	0.00316495559048365\\
36.29	0.00316548268567669\\
36.3	0.00316600999785759\\
36.31	0.00316653752712424\\
36.32	0.00316706527357474\\
36.33	0.00316759323730741\\
36.34	0.00316812141842077\\
36.35	0.00316864981701354\\
36.36	0.00316917843318468\\
36.37	0.00316970726703338\\
36.38	0.003170236318659\\
36.39	0.00317076558816117\\
36.4	0.00317129507563971\\
36.41	0.00317182478119469\\
36.42	0.00317235470492637\\
36.43	0.00317288484693526\\
36.44	0.00317341520732208\\
36.45	0.00317394578618781\\
36.46	0.00317447658363362\\
36.47	0.00317500759976093\\
36.48	0.0031755388346714\\
36.49	0.00317607028846691\\
36.5	0.00317660196124957\\
36.51	0.00317713385312176\\
36.52	0.00317766596418605\\
36.53	0.00317819829454529\\
36.54	0.00317873084430255\\
36.55	0.00317926361356113\\
36.56	0.00317979660242462\\
36.57	0.00318032981099681\\
36.58	0.00318086323938174\\
36.59	0.00318139688768373\\
36.6	0.00318193075600732\\
36.61	0.00318246484445731\\
36.62	0.00318299915313876\\
36.63	0.00318353368215697\\
36.64	0.00318406843161751\\
36.65	0.0031846034016262\\
36.66	0.00318513859228911\\
36.67	0.00318567400371259\\
36.68	0.00318620963600322\\
36.69	0.00318674548926787\\
36.7	0.00318728156361369\\
36.71	0.00318781785914806\\
36.72	0.00318835437597864\\
36.73	0.00318889111421337\\
36.74	0.00318942807396044\\
36.75	0.00318996525532833\\
36.76	0.0031905026584258\\
36.77	0.00319104028336187\\
36.78	0.00319157813024584\\
36.79	0.0031921161991873\\
36.8	0.00319265449029611\\
36.81	0.00319319300368243\\
36.82	0.00319373173945666\\
36.83	0.00319427069772954\\
36.84	0.00319480987861207\\
36.85	0.00319534928221554\\
36.86	0.00319588890865154\\
36.87	0.00319642875803193\\
36.88	0.0031969688304689\\
36.89	0.00319750912607491\\
36.9	0.00319804964496272\\
36.91	0.0031985903872454\\
36.92	0.0031991313530363\\
36.93	0.00319967254244911\\
36.94	0.0032002139555978\\
36.95	0.00320075559259663\\
36.96	0.0032012974535602\\
36.97	0.00320183953860341\\
36.98	0.00320238184784146\\
36.99	0.00320292438138989\\
37	0.00320346713936452\\
37.01	0.00320401012188151\\
37.02	0.00320455332905734\\
37.03	0.00320509676100882\\
37.04	0.00320564041785304\\
37.05	0.00320618429970746\\
37.06	0.00320672840668985\\
37.07	0.00320727273891831\\
37.08	0.00320781729651127\\
37.09	0.00320836207958749\\
37.1	0.00320890708826606\\
37.11	0.00320945232266642\\
37.12	0.00320999778290834\\
37.13	0.00321054346911193\\
37.14	0.00321108938139763\\
37.15	0.00321163551988625\\
37.16	0.00321218188469893\\
37.17	0.00321272847595717\\
37.18	0.00321327529378279\\
37.19	0.00321382233829798\\
37.2	0.00321436960962531\\
37.21	0.00321491710788767\\
37.22	0.00321546483320831\\
37.23	0.00321601278571086\\
37.24	0.0032165609655193\\
37.25	0.00321710937275798\\
37.26	0.0032176580075516\\
37.27	0.00321820687002526\\
37.28	0.0032187559603044\\
37.29	0.00321930527851483\\
37.3	0.00321985482478277\\
37.31	0.00322040459923479\\
37.32	0.00322095460199784\\
37.33	0.00322150483319925\\
37.34	0.00322205529296675\\
37.35	0.00322260598142843\\
37.36	0.0032231568987128\\
37.37	0.00322370804494872\\
37.38	0.00322425942026548\\
37.39	0.00322481102479274\\
37.4	0.00322536285866055\\
37.41	0.0032259149219994\\
37.42	0.00322646721494013\\
37.43	0.00322701973761402\\
37.44	0.00322757249015274\\
37.45	0.00322812547268837\\
37.46	0.0032286786853534\\
37.47	0.00322923212828072\\
37.48	0.00322978580160366\\
37.49	0.00323033970545596\\
37.5	0.00323089383997176\\
37.51	0.00323144820528563\\
37.52	0.00323200280153258\\
37.53	0.00323255762884803\\
37.54	0.00323311268736785\\
37.55	0.0032336679772283\\
37.56	0.00323422349856611\\
37.57	0.00323477925151845\\
37.58	0.00323533523622289\\
37.59	0.00323589145281748\\
37.6	0.0032364479014407\\
37.61	0.00323700458223146\\
37.62	0.00323756149532914\\
37.63	0.00323811864087357\\
37.64	0.00323867601900502\\
37.65	0.00323923362986423\\
37.66	0.00323979147359238\\
37.67	0.00324034955033113\\
37.68	0.0032409078602226\\
37.69	0.00324146640340937\\
37.7	0.00324202518003448\\
37.71	0.00324258419024147\\
37.72	0.00324314343417432\\
37.73	0.00324370291197751\\
37.74	0.00324426262379599\\
37.75	0.0032448225697752\\
37.76	0.00324538275006105\\
37.77	0.00324594316479996\\
37.78	0.00324650381413882\\
37.79	0.00324706469822501\\
37.8	0.00324762581720642\\
37.81	0.00324818717123144\\
37.82	0.00324874876044894\\
37.83	0.00324931058500831\\
37.84	0.00324987264505945\\
37.85	0.00325043494075276\\
37.86	0.00325099747223915\\
37.87	0.00325156023967005\\
37.88	0.0032521232431974\\
37.89	0.00325268648297369\\
37.9	0.00325324995915188\\
37.91	0.00325381367188551\\
37.92	0.00325437762132862\\
37.93	0.00325494180763579\\
37.94	0.00325550623096212\\
37.95	0.00325607089146327\\
37.96	0.00325663578929544\\
37.97	0.00325720092461534\\
37.98	0.00325776629758028\\
37.99	0.00325833190834807\\
38	0.00325889775707711\\
38.01	0.00325946384392634\\
38.02	0.00326003016905525\\
38.03	0.0032605967326239\\
38.04	0.00326116353479292\\
38.05	0.00326173057572349\\
38.06	0.0032622978555774\\
38.07	0.00326286537451698\\
38.08	0.00326343313270513\\
38.09	0.00326400113030536\\
38.1	0.00326456936748175\\
38.11	0.00326513784439896\\
38.12	0.00326570656122225\\
38.13	0.00326627551811746\\
38.14	0.00326684471525106\\
38.15	0.00326741415279008\\
38.16	0.00326798383090216\\
38.17	0.00326855374975557\\
38.18	0.00326912390951916\\
38.19	0.00326969431036241\\
38.2	0.00327026495245542\\
38.21	0.0032708358359689\\
38.22	0.00327140696107418\\
38.23	0.00327197832794322\\
38.24	0.00327254993674862\\
38.25	0.00327312178766358\\
38.26	0.00327369388086198\\
38.27	0.00327426621651831\\
38.28	0.0032748387948077\\
38.29	0.00327541161590596\\
38.3	0.0032759846799895\\
38.31	0.00327655798723544\\
38.32	0.00327713153782151\\
38.33	0.00327770533192611\\
38.34	0.00327827936972833\\
38.35	0.00327885365140791\\
38.36	0.00327942817714526\\
38.37	0.00328000294712145\\
38.38	0.00328057796151826\\
38.39	0.00328115322051813\\
38.4	0.00328172872430419\\
38.41	0.00328230447306026\\
38.42	0.00328288046697087\\
38.43	0.00328345670622121\\
38.44	0.0032840331909972\\
38.45	0.00328460992148546\\
38.46	0.00328518689787331\\
38.47	0.00328576412034878\\
38.48	0.00328634158910062\\
38.49	0.00328691930431831\\
38.5	0.00328749726619203\\
38.51	0.00328807547491271\\
38.52	0.00328865393067201\\
38.53	0.00328923263366229\\
38.54	0.00328981158407669\\
38.55	0.00329039078210908\\
38.56	0.00329097022795406\\
38.57	0.003291549921807\\
38.58	0.00329212986386401\\
38.59	0.00329271005432199\\
38.6	0.00329329049337856\\
38.61	0.00329387118123212\\
38.62	0.00329445211808185\\
38.63	0.00329503330412771\\
38.64	0.00329561473957043\\
38.65	0.00329619642461151\\
38.66	0.00329677835945326\\
38.67	0.00329736054429876\\
38.68	0.0032979429793519\\
38.69	0.00329852566481736\\
38.7	0.00329910860090064\\
38.71	0.00329969178780802\\
38.72	0.00330027522574662\\
38.73	0.00330085891492437\\
38.74	0.00330144285555\\
38.75	0.00330202704783309\\
38.76	0.00330261149198404\\
38.77	0.00330319618821408\\
38.78	0.0033037811367353\\
38.79	0.00330436633776059\\
38.8	0.00330495179150372\\
38.81	0.0033055374981793\\
38.82	0.00330612345800281\\
38.83	0.00330670967119057\\
38.84	0.00330729613795977\\
38.85	0.00330788285852848\\
38.86	0.00330846983311564\\
38.87	0.00330905706194107\\
38.88	0.00330964454522545\\
38.89	0.00331023228319038\\
38.9	0.00331082027605835\\
38.91	0.00331140852405271\\
38.92	0.00331199702739776\\
38.93	0.00331258578631867\\
38.94	0.00331317480104154\\
38.95	0.00331376407179339\\
38.96	0.00331435359880214\\
38.97	0.00331494338229665\\
38.98	0.00331553342250671\\
38.99	0.00331612371966304\\
39	0.00331671427399731\\
39.01	0.00331730508574213\\
39.02	0.00331789615513104\\
39.03	0.00331848748239857\\
39.04	0.00331907906778017\\
39.05	0.0033196709115123\\
39.06	0.00332026301383235\\
39.07	0.00332085537497871\\
39.08	0.00332144799519073\\
39.09	0.00332204087470876\\
39.1	0.00332263401377415\\
39.11	0.00332322741262921\\
39.12	0.00332382107151727\\
39.13	0.00332441499068267\\
39.14	0.00332500917037077\\
39.15	0.00332560361082792\\
39.16	0.0033261983123015\\
39.17	0.00332679327503992\\
39.18	0.00332738849929262\\
39.19	0.00332798398531008\\
39.2	0.00332857973334382\\
39.21	0.00332917574364641\\
39.22	0.00332977201647146\\
39.23	0.00333036855207366\\
39.24	0.00333096535070874\\
39.25	0.00333156241263353\\
39.26	0.00333215973810591\\
39.27	0.00333275732738484\\
39.28	0.00333335518073039\\
39.29	0.00333395329840368\\
39.3	0.00333455168066697\\
39.31	0.00333515032778359\\
39.32	0.00333574924001802\\
39.33	0.00333634841758985\\
39.34	0.00333694786059512\\
39.35	0.00333754756912988\\
39.36	0.00333814754329021\\
39.37	0.00333874778317223\\
39.38	0.00333934828887206\\
39.39	0.00333994906048588\\
39.4	0.00334055009810984\\
39.41	0.0033411514018402\\
39.42	0.00334175297177319\\
39.43	0.00334235480800508\\
39.44	0.00334295691063218\\
39.45	0.00334355927975083\\
39.46	0.00334416191545739\\
39.47	0.00334476481784826\\
39.48	0.00334536798701988\\
39.49	0.00334597142306871\\
39.5	0.00334657512609125\\
39.51	0.00334717909618402\\
39.52	0.00334778333344361\\
39.53	0.00334838783796659\\
39.54	0.00334899260984962\\
39.55	0.00334959764918936\\
39.56	0.00335020295608253\\
39.57	0.00335080853062587\\
39.58	0.00335141437291617\\
39.59	0.00335202048305024\\
39.6	0.00335262686112495\\
39.61	0.00335323350723721\\
39.62	0.00335384042148396\\
39.63	0.00335444760396218\\
39.64	0.00335505505476889\\
39.65	0.00335566277400116\\
39.66	0.0033562707617561\\
39.67	0.00335687901813089\\
39.68	0.00335748754322269\\
39.69	0.00335809633712876\\
39.7	0.00335870539994639\\
39.71	0.00335931473177292\\
39.72	0.00335992433270572\\
39.73	0.00336053420284222\\
39.74	0.00336114434227991\\
39.75	0.00336175475111632\\
39.76	0.003362365429449\\
39.77	0.00336297637737561\\
39.78	0.0033635875949938\\
39.79	0.00336419908240131\\
39.8	0.00336481083969593\\
39.81	0.00336542286697549\\
39.82	0.00336603516433788\\
39.83	0.00336664773188104\\
39.84	0.00336726056970297\\
39.85	0.00336787367790172\\
39.86	0.00336848705657541\\
39.87	0.00336910070582221\\
39.88	0.00336971462574034\\
39.89	0.0033703288164281\\
39.9	0.00337094327798384\\
39.91	0.00337155801050595\\
39.92	0.00337217301409291\\
39.93	0.00337278828884325\\
39.94	0.00337340383485556\\
39.95	0.00337401965222851\\
39.96	0.00337463574106082\\
39.97	0.00337525210145128\\
39.98	0.00337586873349875\\
39.99	0.00337648563730213\\
40	0.00337710281296044\\
40.01	0.00337772026057272\\
};
\addplot [color=green,solid,forget plot]
  table[row sep=crcr]{%
40.01	0.00337772026057272\\
40.02	0.00337833798023811\\
40.03	0.00337895597205581\\
40.04	0.00337957423612509\\
40.05	0.0033801927725453\\
40.06	0.00338081158141584\\
40.07	0.00338143066283621\\
40.08	0.00338205001690597\\
40.09	0.00338266964372478\\
40.1	0.00338328954339234\\
40.11	0.00338390971600846\\
40.12	0.00338453016167301\\
40.13	0.00338515088048594\\
40.14	0.00338577187254729\\
40.15	0.00338639313795718\\
40.16	0.0033870146768158\\
40.17	0.00338763648922343\\
40.18	0.00338825857528045\\
40.19	0.0033888809350873\\
40.2	0.00338950356874453\\
40.21	0.00339012647635274\\
40.22	0.00339074965801267\\
40.23	0.00339137311382511\\
40.24	0.00339199684389096\\
40.25	0.00339262084831119\\
40.26	0.00339324512718688\\
40.27	0.00339386968061921\\
40.28	0.00339449450870943\\
40.29	0.0033951196115589\\
40.3	0.00339574498926908\\
40.31	0.00339637064194152\\
40.32	0.00339699656967787\\
40.33	0.00339762277257988\\
40.34	0.00339824925074939\\
40.35	0.00339887600428837\\
40.36	0.00339950303329886\\
40.37	0.00340013033788303\\
40.38	0.00340075791814314\\
40.39	0.00340138577418155\\
40.4	0.00340201390610075\\
40.41	0.00340264231400331\\
40.42	0.00340327099799195\\
40.43	0.00340389995816945\\
40.44	0.00340452919463874\\
40.45	0.00340515870750285\\
40.46	0.00340578849686492\\
40.47	0.0034064185628282\\
40.48	0.00340704890549608\\
40.49	0.00340767952497206\\
40.5	0.00340831042135973\\
40.51	0.00340894159476284\\
40.52	0.00340957304528524\\
40.53	0.0034102047730309\\
40.54	0.00341083677810394\\
40.55	0.00341146906060855\\
40.56	0.00341210162064911\\
40.57	0.00341273445833009\\
40.58	0.0034133675737561\\
40.59	0.00341400096703187\\
40.6	0.00341463463826229\\
40.61	0.00341526858755234\\
40.62	0.00341590281500717\\
40.63	0.00341653732073205\\
40.64	0.00341717210483238\\
40.65	0.00341780716741371\\
40.66	0.00341844250858173\\
40.67	0.00341907812844226\\
40.68	0.00341971402710127\\
40.69	0.00342035020466488\\
40.7	0.00342098666123933\\
40.71	0.00342162339693105\\
40.72	0.00342226041184657\\
40.73	0.00342289770610784\\
40.74	0.00342353527984559\\
40.75	0.00342417313319057\\
40.76	0.00342481126627354\\
40.77	0.00342544967922531\\
40.78	0.00342608837217668\\
40.79	0.00342672734525849\\
40.8	0.00342736659860159\\
40.81	0.00342800613233685\\
40.82	0.00342864594659516\\
40.83	0.00342928604150743\\
40.84	0.00342992641720457\\
40.85	0.00343056707381753\\
40.86	0.00343120801147729\\
40.87	0.00343184923031482\\
40.88	0.0034324907304611\\
40.89	0.00343313251204717\\
40.9	0.00343377457520405\\
40.91	0.00343441692006279\\
40.92	0.00343505954675447\\
40.93	0.00343570245541015\\
40.94	0.00343634564616093\\
40.95	0.00343698911913794\\
40.96	0.0034376328744723\\
40.97	0.00343827691229516\\
40.98	0.00343892123273768\\
40.99	0.00343956583593104\\
41	0.00344021072200642\\
41.01	0.00344085589109503\\
41.02	0.00344150134332809\\
41.03	0.00344214707883684\\
41.04	0.00344279309775252\\
41.05	0.0034434394002064\\
41.06	0.00344408598632975\\
41.07	0.00344473285625384\\
41.08	0.00344538001011\\
41.09	0.00344602744802953\\
41.1	0.00344667517014376\\
41.11	0.00344732317658403\\
41.12	0.00344797146748168\\
41.13	0.00344862004296808\\
41.14	0.0034492689031746\\
41.15	0.00344991804823262\\
41.16	0.00345056747827355\\
41.17	0.00345121719342878\\
41.18	0.00345186719382973\\
41.19	0.00345251747960784\\
41.2	0.00345316805089453\\
41.21	0.00345381890782126\\
41.22	0.00345447005051948\\
41.23	0.00345512147912065\\
41.24	0.00345577319375625\\
41.25	0.00345642519455776\\
41.26	0.00345707748165667\\
41.27	0.00345773005518448\\
41.28	0.0034583829152727\\
41.29	0.00345903606205283\\
41.3	0.00345968949565641\\
41.31	0.00346034321621497\\
41.32	0.00346099722386002\\
41.33	0.00346165151872313\\
41.34	0.00346230610093583\\
41.35	0.00346296097062969\\
41.36	0.00346361612793625\\
41.37	0.0034642715729871\\
41.38	0.00346492730591379\\
41.39	0.0034655833268479\\
41.4	0.00346623963592101\\
41.41	0.0034668962332647\\
41.42	0.00346755311901058\\
41.43	0.00346821029329022\\
41.44	0.00346886775623522\\
41.45	0.00346952550797718\\
41.46	0.00347018354864771\\
41.47	0.00347084187837839\\
41.48	0.00347150049730084\\
41.49	0.00347215940554668\\
41.5	0.0034728186032475\\
41.51	0.00347347809053493\\
41.52	0.00347413786754057\\
41.53	0.00347479793439603\\
41.54	0.00347545829123294\\
41.55	0.0034761189381829\\
41.56	0.00347677987537753\\
41.57	0.00347744110294844\\
41.58	0.00347810262102725\\
41.59	0.00347876442974556\\
41.6	0.003479426529235\\
41.61	0.00348008891962717\\
41.62	0.00348075160105367\\
41.63	0.00348141457364611\\
41.64	0.0034820778375361\\
41.65	0.00348274139285523\\
41.66	0.00348340523973509\\
41.67	0.0034840693783073\\
41.68	0.00348473380870343\\
41.69	0.00348539853105508\\
41.7	0.00348606354549381\\
41.71	0.00348672885215121\\
41.72	0.00348739445115884\\
41.73	0.00348806034264829\\
41.74	0.00348872652675111\\
41.75	0.00348939300359884\\
41.76	0.00349005977332304\\
41.77	0.00349072683605526\\
41.78	0.00349139419192702\\
41.79	0.00349206184106986\\
41.8	0.0034927297836153\\
41.81	0.00349339801969484\\
41.82	0.00349406654943998\\
41.83	0.00349473537298224\\
41.84	0.00349540449045309\\
41.85	0.003496073901984\\
41.86	0.00349674360770645\\
41.87	0.0034974136077519\\
41.88	0.00349808390225178\\
41.89	0.00349875449133754\\
41.9	0.0034994253751406\\
41.91	0.00350009655379238\\
41.92	0.00350076802742427\\
41.93	0.00350143979616768\\
41.94	0.00350211186015396\\
41.95	0.0035027842195145\\
41.96	0.00350345687438065\\
41.97	0.00350412982488373\\
41.98	0.00350480307115508\\
41.99	0.00350547661332602\\
42	0.00350615045152782\\
42.01	0.00350682458589178\\
42.02	0.00350749901654916\\
42.03	0.0035081737436312\\
42.04	0.00350884876726916\\
42.05	0.00350952408759424\\
42.06	0.00351019970473764\\
42.07	0.00351087561883055\\
42.08	0.00351155183000414\\
42.09	0.00351222833838955\\
42.1	0.00351290514411792\\
42.11	0.00351358224732035\\
42.12	0.00351425964812795\\
42.13	0.00351493734667178\\
42.14	0.0035156153430829\\
42.15	0.00351629363749233\\
42.16	0.00351697223003111\\
42.17	0.00351765112083021\\
42.18	0.00351833031002061\\
42.19	0.00351900979773326\\
42.2	0.00351968958409909\\
42.21	0.00352036966924901\\
42.22	0.00352105005331388\\
42.23	0.00352173073642457\\
42.24	0.00352241171871193\\
42.25	0.00352309300030676\\
42.26	0.00352377458133984\\
42.27	0.00352445646194195\\
42.28	0.00352513864224381\\
42.29	0.00352582112237614\\
42.3	0.00352650390246963\\
42.31	0.00352718698265493\\
42.32	0.00352787036306267\\
42.33	0.00352855404382347\\
42.34	0.00352923802506789\\
42.35	0.00352992230692649\\
42.36	0.00353060688952977\\
42.37	0.00353129177300825\\
42.38	0.00353197695749238\\
42.39	0.00353266244311258\\
42.4	0.00353334822999928\\
42.41	0.00353403431828282\\
42.42	0.00353472070809356\\
42.43	0.0035354073995618\\
42.44	0.00353609439281783\\
42.45	0.00353678168799188\\
42.46	0.00353746928521417\\
42.47	0.00353815718461488\\
42.48	0.00353884538632415\\
42.49	0.00353953389047208\\
42.5	0.00354022269718877\\
42.51	0.00354091180660424\\
42.52	0.00354160121884851\\
42.53	0.00354229093405154\\
42.54	0.00354298095234327\\
42.55	0.0035436712738536\\
42.56	0.00354436189871238\\
42.57	0.00354505282704944\\
42.58	0.00354574405899455\\
42.59	0.00354643559467747\\
42.6	0.00354712743422789\\
42.61	0.0035478195777755\\
42.62	0.0035485120254499\\
42.63	0.0035492047773807\\
42.64	0.00354989783369743\\
42.65	0.0035505911945296\\
42.66	0.00355128486000666\\
42.67	0.00355197883025805\\
42.68	0.00355267310541312\\
42.69	0.00355336768560122\\
42.7	0.00355406257095164\\
42.71	0.00355475776159362\\
42.72	0.00355545325765636\\
42.73	0.00355614905926902\\
42.74	0.00355684516656071\\
42.75	0.00355754157966048\\
42.76	0.00355823829869735\\
42.77	0.0035589353238003\\
42.78	0.00355963265509824\\
42.79	0.00356033029272005\\
42.8	0.00356102823679455\\
42.81	0.00356172648745052\\
42.82	0.00356242504481669\\
42.83	0.00356312390902172\\
42.84	0.00356382308019425\\
42.85	0.00356452255846284\\
42.86	0.00356522234395602\\
42.87	0.00356592243680226\\
42.88	0.00356662283712997\\
42.89	0.00356732354506752\\
42.9	0.0035680245607432\\
42.91	0.0035687258842853\\
42.92	0.003569427515822\\
42.93	0.00357012945548144\\
42.94	0.00357083170339171\\
42.95	0.00357153425968085\\
42.96	0.00357223712447684\\
42.97	0.00357294029790759\\
42.98	0.00357364378010096\\
42.99	0.00357434757118476\\
43	0.00357505167128672\\
43.01	0.00357575608053452\\
43.02	0.00357646079905579\\
43.03	0.00357716582697809\\
43.04	0.00357787116442893\\
43.05	0.00357857681153573\\
43.06	0.00357928276842587\\
43.07	0.00357998903522667\\
43.08	0.00358069561206538\\
43.09	0.00358140249906919\\
43.1	0.00358210969636519\\
43.11	0.00358281720408047\\
43.12	0.00358352502234201\\
43.13	0.00358423315127673\\
43.14	0.00358494159101148\\
43.15	0.00358565034167306\\
43.16	0.00358635940338818\\
43.17	0.00358706877628349\\
43.18	0.00358777846048559\\
43.19	0.00358848845612098\\
43.2	0.0035891987633161\\
43.21	0.00358990938219732\\
43.22	0.00359062031289095\\
43.23	0.00359133155552321\\
43.24	0.00359204311022025\\
43.25	0.00359275497710816\\
43.26	0.00359346715631294\\
43.27	0.00359417964796054\\
43.28	0.00359489245217678\\
43.29	0.00359560556908748\\
43.3	0.00359631899881832\\
43.31	0.00359703274149493\\
43.32	0.00359774679724287\\
43.33	0.00359846116618762\\
43.34	0.00359917584845455\\
43.35	0.00359989084416899\\
43.36	0.00360060615345617\\
43.37	0.00360132177644123\\
43.38	0.00360203771324925\\
43.39	0.00360275396400522\\
43.4	0.00360347052883405\\
43.41	0.00360418740786056\\
43.42	0.00360490460120948\\
43.43	0.00360562210900548\\
43.44	0.00360633993137312\\
43.45	0.00360705806843688\\
43.46	0.00360777652032116\\
43.47	0.00360849528715027\\
43.48	0.00360921436904842\\
43.49	0.00360993376613977\\
43.5	0.00361065347854834\\
43.51	0.00361137350639809\\
43.52	0.00361209384981288\\
43.53	0.0036128145089165\\
43.54	0.00361353548383261\\
43.55	0.0036142567746848\\
43.56	0.00361497838159657\\
43.57	0.00361570030469132\\
43.58	0.00361642254409236\\
43.59	0.00361714509992289\\
43.6	0.00361786797230603\\
43.61	0.0036185911613648\\
43.62	0.00361931466722212\\
43.63	0.00362003849000082\\
43.64	0.0036207626298236\\
43.65	0.00362148708681311\\
43.66	0.00362221186109187\\
43.67	0.0036229369527823\\
43.68	0.00362366236200672\\
43.69	0.00362438808888735\\
43.7	0.0036251141335463\\
43.71	0.0036258404961056\\
43.72	0.00362656717668714\\
43.73	0.00362729417541274\\
43.74	0.00362802149240409\\
43.75	0.00362874912778278\\
43.76	0.0036294770816703\\
43.77	0.00363020535418802\\
43.78	0.0036309339454572\\
43.79	0.00363166285559901\\
43.8	0.00363239208473449\\
43.81	0.00363312163298458\\
43.82	0.0036338515004701\\
43.83	0.00363458168731175\\
43.84	0.00363531219363016\\
43.85	0.00363604301954578\\
43.86	0.003636774165179\\
43.87	0.00363750563065008\\
43.88	0.00363823741607914\\
43.89	0.0036389695215862\\
43.9	0.00363970194729117\\
43.91	0.00364043469331383\\
43.92	0.00364116775977386\\
43.93	0.00364190114679077\\
43.94	0.003642634854484\\
43.95	0.00364336888297286\\
43.96	0.00364410323237652\\
43.97	0.00364483790281404\\
43.98	0.00364557289440435\\
43.99	0.00364630820726625\\
44	0.00364704384151842\\
44.01	0.00364777979727941\\
44.02	0.00364851607466766\\
44.03	0.00364925267380146\\
44.04	0.00364998959479897\\
44.05	0.00365072683777824\\
44.06	0.00365146440285718\\
44.07	0.00365220229015355\\
44.08	0.00365294049978501\\
44.09	0.00365367903186905\\
44.1	0.00365441788652307\\
44.11	0.0036551570638643\\
44.12	0.00365589656400985\\
44.13	0.00365663638707669\\
44.14	0.00365737653318165\\
44.15	0.00365811700244142\\
44.16	0.00365885779497256\\
44.17	0.00365959891089148\\
44.18	0.00366034035031446\\
44.19	0.00366108211335762\\
44.2	0.00366182420013696\\
44.21	0.00366256661076833\\
44.22	0.00366330934536742\\
44.23	0.00366405240404979\\
44.24	0.00366479578693085\\
44.25	0.00366553949412587\\
44.26	0.00366628352574995\\
44.27	0.00366702788191806\\
44.28	0.00366777256274503\\
44.29	0.00366851756834552\\
44.3	0.00366926289883403\\
44.31	0.00367000855432494\\
44.32	0.00367075453493245\\
44.33	0.00367150084077061\\
44.34	0.00367224747195334\\
44.35	0.00367299442859436\\
44.36	0.00367374171080727\\
44.37	0.0036744893187055\\
44.38	0.00367523725240231\\
44.39	0.0036759855120108\\
44.4	0.00367673409764394\\
44.41	0.00367748300941451\\
44.42	0.00367823224743514\\
44.43	0.00367898181181828\\
44.44	0.00367973170267622\\
44.45	0.0036804819201211\\
44.46	0.0036812324642649\\
44.47	0.00368198333521939\\
44.48	0.00368273453309622\\
44.49	0.00368348605800684\\
44.5	0.00368423791006253\\
44.51	0.00368499008937441\\
44.52	0.00368574259605343\\
44.53	0.00368649543021037\\
44.54	0.00368724859195582\\
44.55	0.0036880020814002\\
44.56	0.00368875589865376\\
44.57	0.00368951004382655\\
44.58	0.00369026451702849\\
44.59	0.00369101931836927\\
44.6	0.00369177444795842\\
44.61	0.0036925299059053\\
44.62	0.00369328569231907\\
44.63	0.00369404180730872\\
44.64	0.00369479825098304\\
44.65	0.00369555502345064\\
44.66	0.00369631212481996\\
44.67	0.00369706955519922\\
44.68	0.00369782731469649\\
44.69	0.00369858540341962\\
44.7	0.00369934382147628\\
44.71	0.00370010256897394\\
44.72	0.0037008616460199\\
44.73	0.00370162105272125\\
44.74	0.00370238078918488\\
44.75	0.00370314085551748\\
44.76	0.00370390125182558\\
44.77	0.00370466197821546\\
44.78	0.00370542303479324\\
44.79	0.00370618442166482\\
44.8	0.00370694613893589\\
44.81	0.00370770818671197\\
44.82	0.00370847056509835\\
44.83	0.00370923327420011\\
44.84	0.00370999631412216\\
44.85	0.00371075968496916\\
44.86	0.00371152338684558\\
44.87	0.00371228741985569\\
44.88	0.00371305178410354\\
44.89	0.00371381647969296\\
44.9	0.00371458150672759\\
44.91	0.00371534686531082\\
44.92	0.00371611255554587\\
44.93	0.00371687857753571\\
44.94	0.00371764493138311\\
44.95	0.0037184116171906\\
44.96	0.00371917863506053\\
44.97	0.00371994598509498\\
44.98	0.00372071366739585\\
44.99	0.00372148168206479\\
45	0.00372225002920323\\
45.01	0.0037230187089124\\
45.02	0.00372378772129326\\
45.03	0.00372455706644658\\
45.04	0.00372532674447287\\
45.05	0.00372609675547243\\
45.06	0.00372686709954533\\
45.07	0.00372763777679139\\
45.08	0.00372840878731021\\
45.09	0.00372918013120116\\
45.1	0.00372995180856334\\
45.11	0.00373072381949566\\
45.12	0.00373149616409675\\
45.13	0.00373226884246503\\
45.14	0.00373304185469867\\
45.15	0.00373381520089557\\
45.16	0.00373458888115342\\
45.17	0.00373536289556965\\
45.18	0.00373613724424145\\
45.19	0.00373691192726577\\
45.2	0.00373768694473928\\
45.21	0.00373846229675844\\
45.22	0.00373923798341941\\
45.23	0.00374001400481814\\
45.24	0.00374079036105031\\
45.25	0.00374156705221134\\
45.26	0.00374234407839641\\
45.27	0.00374312143970041\\
45.28	0.003743899136218\\
45.29	0.00374467716804357\\
45.3	0.00374545553527125\\
45.31	0.00374623423799489\\
45.32	0.00374701327630809\\
45.33	0.0037477926503042\\
45.34	0.00374857236007627\\
45.35	0.00374935240571708\\
45.36	0.00375013278731918\\
45.37	0.00375091350497481\\
45.38	0.00375169455877594\\
45.39	0.0037524759488143\\
45.4	0.00375325767518131\\
45.41	0.00375403973796811\\
45.42	0.00375482213726559\\
45.43	0.00375560487316434\\
45.44	0.00375638794575466\\
45.45	0.0037571713551266\\
45.46	0.00375795510136989\\
45.47	0.00375873918457399\\
45.48	0.00375952360482808\\
45.49	0.00376030836222103\\
45.5	0.00376109345684145\\
45.51	0.00376187888877764\\
45.52	0.00376266465811761\\
45.53	0.00376345076494906\\
45.54	0.00376423720935943\\
45.55	0.00376502399143583\\
45.56	0.00376581111126509\\
45.57	0.00376659856893373\\
45.58	0.00376738636452797\\
45.59	0.00376817449813373\\
45.6	0.00376896296983663\\
45.61	0.00376975177972197\\
45.62	0.00377054092787475\\
45.63	0.00377133041437967\\
45.64	0.00377212023932111\\
45.65	0.00377291040278312\\
45.66	0.00377370090484948\\
45.67	0.0037744917456036\\
45.68	0.00377528292512864\\
45.69	0.00377607444350737\\
45.7	0.00377686630082229\\
45.71	0.00377765849715556\\
45.72	0.00377845103258902\\
45.73	0.00377924390720419\\
45.74	0.00378003712108225\\
45.75	0.00378083067430406\\
45.76	0.00378162456695016\\
45.77	0.00378241879910076\\
45.78	0.00378321337083571\\
45.79	0.00378400828223455\\
45.8	0.00378480353337648\\
45.81	0.00378559912434036\\
45.82	0.00378639505520472\\
45.83	0.00378719132604773\\
45.84	0.00378798793694724\\
45.85	0.00378878488798074\\
45.86	0.00378958217922538\\
45.87	0.00379037981075797\\
45.88	0.00379117778265495\\
45.89	0.00379197609499244\\
45.9	0.00379277474784618\\
45.91	0.00379357374129157\\
45.92	0.00379437307540365\\
45.93	0.00379517275025711\\
45.94	0.00379597276592627\\
45.95	0.00379677312248509\\
45.96	0.00379757382000719\\
45.97	0.0037983748585658\\
45.98	0.00379917623823379\\
45.99	0.00379997795908366\\
46	0.00380078002118757\\
46.01	0.00380158242461726\\
46.02	0.00380238516944413\\
46.03	0.0038031882557392\\
46.04	0.00380399168357311\\
46.05	0.00380479545301613\\
46.06	0.00380559956413813\\
46.07	0.00380640401700862\\
46.08	0.00380720881169672\\
46.09	0.00380801394827116\\
46.1	0.00380881942680029\\
46.11	0.00380962524735206\\
46.12	0.00381043140999404\\
46.13	0.00381123791479341\\
46.14	0.00381204476181694\\
46.15	0.00381285195113103\\
46.16	0.00381365948280164\\
46.17	0.00381446735689438\\
46.18	0.00381527557347442\\
46.19	0.00381608413260654\\
46.2	0.00381689303435512\\
46.21	0.00381770227878413\\
46.22	0.00381851186595712\\
46.23	0.00381932179593724\\
46.24	0.00382013206878723\\
46.25	0.0038209426845694\\
46.26	0.00382175364334567\\
46.27	0.0038225649451775\\
46.28	0.00382337659012597\\
46.29	0.0038241885782517\\
46.3	0.00382500090961493\\
46.31	0.00382581358427543\\
46.32	0.00382662660229258\\
46.33	0.0038274399637253\\
46.34	0.00382825366863207\\
46.35	0.00382906771707098\\
46.36	0.00382988210909964\\
46.37	0.00383069684477524\\
46.38	0.00383151192415453\\
46.39	0.00383232734729381\\
46.4	0.00383314311424895\\
46.41	0.00383395922507535\\
46.42	0.00383477567982798\\
46.43	0.00383559247856136\\
46.44	0.00383640962132956\\
46.45	0.00383722710818615\\
46.46	0.00383804493918432\\
46.47	0.00383886311437674\\
46.48	0.00383968163381566\\
46.49	0.00384050049755282\\
46.5	0.00384131970563953\\
46.51	0.00384213925812664\\
46.52	0.0038429591550645\\
46.53	0.00384377939650302\\
46.54	0.00384459998249159\\
46.55	0.00384542091307918\\
46.56	0.00384624218831425\\
46.57	0.00384706380824477\\
46.58	0.00384788577291827\\
46.59	0.00384870808238175\\
46.6	0.00384953073668175\\
46.61	0.00385035373586431\\
46.62	0.00385117707997498\\
46.63	0.00385200076905882\\
46.64	0.0038528248031604\\
46.65	0.00385364918232379\\
46.66	0.00385447390659254\\
46.67	0.00385529897600972\\
46.68	0.0038561243906179\\
46.69	0.00385695015045912\\
46.7	0.00385777625557494\\
46.71	0.00385860270600637\\
46.72	0.00385942950179394\\
46.73	0.00386025664297765\\
46.74	0.00386108412959699\\
46.75	0.00386191196169092\\
46.76	0.00386274013929788\\
46.77	0.0038635686624558\\
46.78	0.00386439753120205\\
46.79	0.00386522674557351\\
46.8	0.00386605630560649\\
46.81	0.0038668862113368\\
46.82	0.00386771646279967\\
46.83	0.00386854706002985\\
46.84	0.00386937800306149\\
46.85	0.00387020929192823\\
46.86	0.00387104092666316\\
46.87	0.00387187290729881\\
46.88	0.00387270523386716\\
46.89	0.00387353790639965\\
46.9	0.00387437092492714\\
46.91	0.00387520428947997\\
46.92	0.00387603800008787\\
46.93	0.00387687205678005\\
46.94	0.00387770645958512\\
46.95	0.00387854120853116\\
46.96	0.00387937630364563\\
46.97	0.00388021174495546\\
46.98	0.00388104753248698\\
46.99	0.00388188366626594\\
47	0.00388272014631753\\
47.01	0.00388355697266633\\
47.02	0.00388439414533636\\
47.03	0.00388523166435103\\
47.04	0.00388606952973317\\
47.05	0.003886907741505\\
47.06	0.00388774629968817\\
47.07	0.00388858520430372\\
47.08	0.00388942445537207\\
47.09	0.00389026405291305\\
47.1	0.0038911039969459\\
47.11	0.00389194428748922\\
47.12	0.00389278492456102\\
47.13	0.00389362590817868\\
47.14	0.00389446723835896\\
47.15	0.00389530891511803\\
47.16	0.0038961509384714\\
47.17	0.00389699330843397\\
47.18	0.00389783602502002\\
47.19	0.00389867908824317\\
47.2	0.00389952249811644\\
47.21	0.00390036625465219\\
47.22	0.00390121035786216\\
47.23	0.00390205480775743\\
47.24	0.00390289960434843\\
47.25	0.00390374474764497\\
47.26	0.00390459023765619\\
47.27	0.00390543607439058\\
47.28	0.00390628225785596\\
47.29	0.00390712878805952\\
47.3	0.00390797566500776\\
47.31	0.00390882288870654\\
47.32	0.00390967045916103\\
47.33	0.00391051837637575\\
47.34	0.00391136664035453\\
47.35	0.00391221525110052\\
47.36	0.0039130642086162\\
47.37	0.00391391351290338\\
47.38	0.00391476316396317\\
47.39	0.00391561316179598\\
47.4	0.00391646350640156\\
47.41	0.00391731419777894\\
47.42	0.00391816523592647\\
47.43	0.00391901662084178\\
47.44	0.00391986835252182\\
47.45	0.00392072043096282\\
47.46	0.00392157285616029\\
47.47	0.00392242562810907\\
47.48	0.00392327874680325\\
47.49	0.00392413221223619\\
47.5	0.00392498602440056\\
47.51	0.0039258401832883\\
47.52	0.00392669468889061\\
47.53	0.00392754954119796\\
47.54	0.00392840474020011\\
47.55	0.00392926028588605\\
47.56	0.00393011617824404\\
47.57	0.00393097241726162\\
47.58	0.00393182900292555\\
47.59	0.00393268593522187\\
47.6	0.00393354321413585\\
47.61	0.00393440083965201\\
47.62	0.00393525881175411\\
47.63	0.00393611713042514\\
47.64	0.00393697579564734\\
47.65	0.00393783480740217\\
47.66	0.00393869416567033\\
47.67	0.00393955387043173\\
47.68	0.0039404139216655\\
47.69	0.00394127431935001\\
47.7	0.00394213506346282\\
47.71	0.00394299615398071\\
47.72	0.00394385759087969\\
47.73	0.00394471937413494\\
47.74	0.00394558150372087\\
47.75	0.00394644397961106\\
47.76	0.00394730680177831\\
47.77	0.0039481699701946\\
47.78	0.0039490334848311\\
47.79	0.00394989734565817\\
47.8	0.00395076155264534\\
47.81	0.00395162610576132\\
47.82	0.003952491004974\\
47.83	0.00395335625025044\\
47.84	0.00395422184155686\\
47.85	0.00395508777885866\\
47.86	0.00395595406212038\\
47.87	0.00395682069130573\\
47.88	0.00395768766637757\\
47.89	0.00395855498729791\\
47.9	0.0039594226540279\\
47.91	0.00396029066652784\\
47.92	0.00396115902475717\\
47.93	0.00396202772867445\\
47.94	0.00396289677823741\\
47.95	0.00396376617340286\\
47.96	0.00396463591412675\\
47.97	0.00396550600036417\\
47.98	0.00396637643206933\\
47.99	0.0039672472091955\\
48	0.00396811833169512\\
48.01	0.00396898979951969\\
48.02	0.00396986161261986\\
48.03	0.00397073377094535\\
48.04	0.00397160627444496\\
48.05	0.00397247912306661\\
48.06	0.00397335231675729\\
48.07	0.00397422585546309\\
48.08	0.00397509973912916\\
48.09	0.00397597396769973\\
48.1	0.00397684854111811\\
48.11	0.00397772345932668\\
48.12	0.00397859872226686\\
48.13	0.00397947432987917\\
48.14	0.00398035028210315\\
48.15	0.00398122657887741\\
48.16	0.00398210322013961\\
48.17	0.00398298020582644\\
48.18	0.00398385753587364\\
48.19	0.00398473521021599\\
48.2	0.00398561322878729\\
48.21	0.00398649159152039\\
48.22	0.00398737029834714\\
48.23	0.00398824934919843\\
48.24	0.00398912874400416\\
48.25	0.00399000848269323\\
48.26	0.00399088856519356\\
48.27	0.00399176899143208\\
48.28	0.00399264976133472\\
48.29	0.00399353087482637\\
48.3	0.00399441233183098\\
48.31	0.00399529413227141\\
48.32	0.00399617627606956\\
48.33	0.00399705876314629\\
48.34	0.00399794159342142\\
48.35	0.00399882476681377\\
48.36	0.00399970828324112\\
48.37	0.00400059214262018\\
48.38	0.00400147634486667\\
48.39	0.00400236088989521\\
48.4	0.00400324577761942\\
48.41	0.00400413100795181\\
48.42	0.00400501658080389\\
48.43	0.00400590249608606\\
48.44	0.00400678875370768\\
48.45	0.00400767535357702\\
48.46	0.00400856229560127\\
48.47	0.00400944957968657\\
48.48	0.00401033720573794\\
48.49	0.00401122517365932\\
48.5	0.00401211348335356\\
48.51	0.0040130021347224\\
48.52	0.0040138911276665\\
48.53	0.00401478046208539\\
48.54	0.00401567013787749\\
48.55	0.00401656015494012\\
48.56	0.00401745051316944\\
48.57	0.00401834121246053\\
48.58	0.0040192322527073\\
48.59	0.00402012363380256\\
48.6	0.00402101535563795\\
48.61	0.00402190741810397\\
48.62	0.00402279982108997\\
48.63	0.00402369256448418\\
48.64	0.00402458564817362\\
48.65	0.00402547907204416\\
48.66	0.00402637283598054\\
48.67	0.00402726693986626\\
48.68	0.0040281613835837\\
48.69	0.00402905616701403\\
48.7	0.00402995129003724\\
48.71	0.00403084675253212\\
48.72	0.00403174255437627\\
48.73	0.00403263869544609\\
48.74	0.00403353517561675\\
48.75	0.00403443199476223\\
48.76	0.00403532915275529\\
48.77	0.00403622664946747\\
48.78	0.00403712448476908\\
48.79	0.00403802265852918\\
48.8	0.00403892117061563\\
48.81	0.00403982002089501\\
48.82	0.00404071920923267\\
48.83	0.00404161873549271\\
48.84	0.00404251859953798\\
48.85	0.00404341880123005\\
48.86	0.00404431934042922\\
48.87	0.00404522021699455\\
48.88	0.00404612143078377\\
48.89	0.00404702298165337\\
48.9	0.00404792486945855\\
48.91	0.00404882709405318\\
48.92	0.00404972965528987\\
48.93	0.00405063255301991\\
48.94	0.00405153578709327\\
48.95	0.00405243935735863\\
48.96	0.00405334326366333\\
48.97	0.00405424750585339\\
48.98	0.00405515208377352\\
48.99	0.00405605699726704\\
49	0.004056962246176\\
49.01	0.00405786783034105\\
49.02	0.00405877374960149\\
49.03	0.00405968000379531\\
49.04	0.00406058659275907\\
49.05	0.00406149351632801\\
49.06	0.00406240077433598\\
49.07	0.00406330836661545\\
49.08	0.0040642162929975\\
49.09	0.00406512455331183\\
49.1	0.00406603314738673\\
49.11	0.00406694207504909\\
49.12	0.0040678513361244\\
49.13	0.00406876093043674\\
49.14	0.00406967085780876\\
49.15	0.00407058111806168\\
49.16	0.00407149171101531\\
49.17	0.00407240263648801\\
49.18	0.00407331389429669\\
49.19	0.00407422548425682\\
49.2	0.00407513740618243\\
49.21	0.00407604965988606\\
49.22	0.00407696224517881\\
49.23	0.0040778751618703\\
49.24	0.00407878840976866\\
49.25	0.00407970198868057\\
49.26	0.00408061589841117\\
49.27	0.00408153013876416\\
49.28	0.00408244470954169\\
49.29	0.00408335961054444\\
49.3	0.00408427484157156\\
49.31	0.00408519040242068\\
49.32	0.00408610629288789\\
49.33	0.00408702251276778\\
49.34	0.00408793906185338\\
49.35	0.00408885593993619\\
49.36	0.00408977314680613\\
49.37	0.00409069068225161\\
49.38	0.00409160854605943\\
49.39	0.00409252673801485\\
49.4	0.00409344525790155\\
49.41	0.00409436410550162\\
49.42	0.00409528328059557\\
49.43	0.00409620278296231\\
49.44	0.00409712261237916\\
49.45	0.00409804276862182\\
49.46	0.00409896325146439\\
49.47	0.00409988406067932\\
49.48	0.00410080519603748\\
49.49	0.00410172665730806\\
49.5	0.00410264844425865\\
49.51	0.00410357055665517\\
49.52	0.00410449299426188\\
49.53	0.00410541575684141\\
49.54	0.00410633884415471\\
49.55	0.00410726225596104\\
49.56	0.00410818599201801\\
49.57	0.00410911005208152\\
49.58	0.0041100344359058\\
49.59	0.00411095914324335\\
49.6	0.00411188417384498\\
49.61	0.00411280952745981\\
49.62	0.00411373520383521\\
49.63	0.00411466120271683\\
49.64	0.00411558752384858\\
49.65	0.00411651416697265\\
49.66	0.00411744113182946\\
49.67	0.00411836841815769\\
49.68	0.00411929602569425\\
49.69	0.00412022395417429\\
49.7	0.00412115220333119\\
49.71	0.00412208077289651\\
49.72	0.00412300966260007\\
49.73	0.00412393887216986\\
49.74	0.00412486840133208\\
49.75	0.00412579824981111\\
49.76	0.00412672841732952\\
49.77	0.00412765890360806\\
49.78	0.00412858970836563\\
49.79	0.00412952083131931\\
49.8	0.0041304522721843\\
49.81	0.00413138403067398\\
49.82	0.00413231610649985\\
49.83	0.00413324849937154\\
49.84	0.00413418120899682\\
49.85	0.00413511423508155\\
49.86	0.00413604757732972\\
49.87	0.00413698123544339\\
49.88	0.00413791520912277\\
49.89	0.00413884949806608\\
49.9	0.00413978410196969\\
49.91	0.00414071902052799\\
49.92	0.00414165425343346\\
49.93	0.00414258980037661\\
49.94	0.00414352566104603\\
49.95	0.00414446183512832\\
49.96	0.00414539832230811\\
49.97	0.0041463351222681\\
49.98	0.00414727223468895\\
49.99	0.00414820965924936\\
50	0.004149147395626\\
50.01	0.00415008544349357\\
50.02	0.00415102380252474\\
50.03	0.00415196247239014\\
50.04	0.0041529014527584\\
50.05	0.00415384074329608\\
50.06	0.00415478034366771\\
50.07	0.00415572025353575\\
50.08	0.00415666047256062\\
50.09	0.00415760100040063\\
50.1	0.00415854183671204\\
50.11	0.00415948298114902\\
50.12	0.00416042443336364\\
50.13	0.00416136619300585\\
50.14	0.0041623082597235\\
50.15	0.00416325063316231\\
50.16	0.00416419331296589\\
50.17	0.00416513629877569\\
50.18	0.00416607959023102\\
50.19	0.00416702318696903\\
50.2	0.00416796708862471\\
50.21	0.00416891129483087\\
50.22	0.00416985580521815\\
50.23	0.00417080061941501\\
50.24	0.00417174573704768\\
50.25	0.00417269115774022\\
50.26	0.00417363688111443\\
50.27	0.00417458290678991\\
50.28	0.00417552923438405\\
50.29	0.00417647586351196\\
50.3	0.0041774227937865\\
50.31	0.0041783700248183\\
50.32	0.0041793175562157\\
50.33	0.00418026538758477\\
50.34	0.00418121351852927\\
50.35	0.0041821619486507\\
50.36	0.00418311067754824\\
50.37	0.00418405970481874\\
50.38	0.00418500903005675\\
50.39	0.00418595865285448\\
50.4	0.00418690857280179\\
50.41	0.0041878587894862\\
50.42	0.00418880930249288\\
50.43	0.00418976011140461\\
50.44	0.00419071121580179\\
50.45	0.00419166261526246\\
50.46	0.00419261430936222\\
50.47	0.0041935662976743\\
50.48	0.0041945185797695\\
50.49	0.00419547115521619\\
50.5	0.00419642402358031\\
50.51	0.00419737718442537\\
50.52	0.00419833063731239\\
50.53	0.00419928438179996\\
50.54	0.00420023841744419\\
50.55	0.00420119274379869\\
50.56	0.00420214736041458\\
50.57	0.00420310226684047\\
50.58	0.0042040574626225\\
50.59	0.00420501294730424\\
50.6	0.00420596872042676\\
50.61	0.00420692478152854\\
50.62	0.00420788113014557\\
50.63	0.00420883776581123\\
50.64	0.00420979468805635\\
50.65	0.00421075189640917\\
50.66	0.00421170939039534\\
50.67	0.0042126671695379\\
50.68	0.00421362523335728\\
50.69	0.00421458358137128\\
50.7	0.00421554221309509\\
50.71	0.00421650112804122\\
50.72	0.00421746032571957\\
50.73	0.00421841980563733\\
50.74	0.00421937956729904\\
50.75	0.00422033961020653\\
50.76	0.00422129993385899\\
50.77	0.00422226053775283\\
50.78	0.00422322142138179\\
50.79	0.00422418258423686\\
50.8	0.0042251440258063\\
50.81	0.00422610574557563\\
50.82	0.00422706774302758\\
50.83	0.00422803001764215\\
50.84	0.0042289925688965\\
50.85	0.00422995539626505\\
50.86	0.0042309184992194\\
50.87	0.00423188187722833\\
50.88	0.00423284552975779\\
50.89	0.0042338094562709\\
50.9	0.00423477365622794\\
50.91	0.0042357381290863\\
50.92	0.00423670287430054\\
50.93	0.00423766789132229\\
50.94	0.00423863317960034\\
50.95	0.00423959873858053\\
50.96	0.00424056456770582\\
50.97	0.00424153066641622\\
50.98	0.00424249703414881\\
50.99	0.0042434636703377\\
51	0.00424443057441408\\
51.01	0.00424539774580612\\
51.02	0.00424636518393903\\
51.03	0.00424733288823503\\
51.04	0.00424830085811331\\
51.05	0.00424926909299006\\
51.06	0.00425023759227842\\
51.07	0.0042512063553885\\
51.08	0.00425217538172735\\
51.09	0.00425314467069896\\
51.1	0.0042541142217042\\
51.11	0.00425508403414091\\
51.12	0.00425605410740378\\
51.13	0.0042570244408844\\
51.14	0.00425799503397124\\
51.15	0.00425896588604961\\
51.16	0.00425993699650168\\
51.17	0.00426090836470645\\
51.18	0.00426187999003976\\
51.19	0.00426285187187423\\
51.2	0.0042638240095793\\
51.21	0.00426479640252118\\
51.22	0.00426576905006288\\
51.23	0.00426674195156413\\
51.24	0.00426771510638144\\
51.25	0.00426868851386804\\
51.26	0.00426966217337389\\
51.27	0.00427063608424565\\
51.28	0.00427161024582668\\
51.29	0.00427258465745702\\
51.3	0.00427355931847339\\
51.31	0.00427453422820916\\
51.32	0.00427550938599434\\
51.33	0.00427648479115559\\
51.34	0.00427746044301617\\
51.35	0.00427843634089594\\
51.36	0.00427941248411137\\
51.37	0.00428038887197551\\
51.38	0.00428136550379795\\
51.39	0.00428234237888486\\
51.4	0.00428331949653893\\
51.41	0.00428429685605938\\
51.42	0.00428527445674194\\
51.43	0.00428625229787885\\
51.44	0.00428723037875881\\
51.45	0.004288208698667\\
51.46	0.00428918725688508\\
51.47	0.00429016605269111\\
51.48	0.00429114508535961\\
51.49	0.00429212435416151\\
51.5	0.00429310385836414\\
51.51	0.0042940835972312\\
51.52	0.00429506357002278\\
51.53	0.00429604377599533\\
51.54	0.00429702421440164\\
51.55	0.00429800488449082\\
51.56	0.00429898578550832\\
51.57	0.00429996691669586\\
51.58	0.00430094827729147\\
51.59	0.00430192986652946\\
51.6	0.00430291168364036\\
51.61	0.00430389372785099\\
51.62	0.00430487599838437\\
51.63	0.00430585849445973\\
51.64	0.00430684121529253\\
51.65	0.00430782416009439\\
51.66	0.00430880732807309\\
51.67	0.0043097907184326\\
51.68	0.004310774330373\\
51.69	0.00431175816309051\\
51.7	0.00431274221577746\\
51.71	0.00431372648762226\\
51.72	0.00431471097780941\\
51.73	0.00431569568551948\\
51.74	0.00431668060992907\\
51.75	0.00431766575021084\\
51.76	0.00431865110553345\\
51.77	0.00431963667506156\\
51.78	0.00432062245795583\\
51.79	0.00432160845337289\\
51.8	0.00432259466046531\\
51.81	0.00432358107838161\\
51.82	0.00432456770626624\\
51.83	0.00432555454325956\\
51.84	0.00432654158849782\\
51.85	0.00432752884111311\\
51.86	0.00432851630023345\\
51.87	0.00432950396498264\\
51.88	0.00433049183448035\\
51.89	0.00433147990784205\\
51.9	0.004332468184179\\
51.91	0.00433345666259823\\
51.92	0.00433444534220256\\
51.93	0.00433543422209054\\
51.94	0.00433642330135644\\
51.95	0.00433741257909026\\
51.96	0.00433840205437769\\
51.97	0.00433939172630012\\
51.98	0.00434038159393455\\
51.99	0.00434137165635368\\
52	0.00434236191262583\\
52.01	0.0043433523618149\\
52.02	0.00434434300298042\\
52.03	0.00434533383517747\\
52.04	0.00434632485745672\\
52.05	0.00434731606886437\\
52.06	0.00434830746844215\\
52.07	0.00434929905522728\\
52.08	0.0043502908282525\\
52.09	0.004351282786546\\
52.1	0.00435227492913144\\
52.11	0.00435326725502792\\
52.12	0.00435425976324996\\
52.13	0.00435525245280747\\
52.14	0.00435624532270575\\
52.15	0.00435723837194549\\
52.16	0.0043582315995227\\
52.17	0.00435922500442874\\
52.18	0.00436021858565026\\
52.19	0.00436121234216925\\
52.2	0.00436220627296294\\
52.21	0.00436320037700381\\
52.22	0.00436419465325961\\
52.23	0.00436518910069329\\
52.24	0.00436618371826303\\
52.25	0.00436717850492215\\
52.26	0.00436817345961918\\
52.27	0.00436916858129777\\
52.28	0.00437016386889671\\
52.29	0.0043711593213499\\
52.3	0.00437215493758632\\
52.31	0.00437315071653002\\
52.32	0.00437414665710011\\
52.33	0.00437514275821073\\
52.34	0.00437613901877104\\
52.35	0.00437713543768518\\
52.36	0.00437813201385229\\
52.37	0.00437912874616642\\
52.38	0.00438012563351661\\
52.39	0.00438112267478676\\
52.4	0.00438211986885572\\
52.41	0.00438311721459718\\
52.42	0.0043841147108797\\
52.43	0.00438511235656666\\
52.44	0.00438611015051629\\
52.45	0.0043871080915816\\
52.46	0.00438810617861034\\
52.47	0.00438910441044509\\
52.48	0.00439010278592309\\
52.49	0.00439110130387634\\
52.5	0.00439209996313153\\
52.51	0.00439309876251\\
52.52	0.00439409770082778\\
52.53	0.00439509677689549\\
52.54	0.00439609598951839\\
52.55	0.00439709533749633\\
52.56	0.00439809481962372\\
52.57	0.00439909443468953\\
52.58	0.00440009418147724\\
52.59	0.00440109405876485\\
52.6	0.00440209406532484\\
52.61	0.00440309419992415\\
52.62	0.00440409446132417\\
52.63	0.00440509484828071\\
52.64	0.00440609535954398\\
52.65	0.00440709599385856\\
52.66	0.00440809674996338\\
52.67	0.0044090976265917\\
52.68	0.00441009862247112\\
52.69	0.00441109973632351\\
52.7	0.00441210096686501\\
52.71	0.00441310231280599\\
52.72	0.00441410377285107\\
52.73	0.00441510534569905\\
52.74	0.00441610703004291\\
52.75	0.00441710882456979\\
52.76	0.00441811072796097\\
52.77	0.00441911273889182\\
52.78	0.00442011485603181\\
52.79	0.00442111707804447\\
52.8	0.00442211940358737\\
52.81	0.0044231218313121\\
52.82	0.00442412435986424\\
52.83	0.00442512698788335\\
52.84	0.00442612971400293\\
52.85	0.00442713253685041\\
52.86	0.00442813545504712\\
52.87	0.00442913846720825\\
52.88	0.00443014157194286\\
52.89	0.00443114476785384\\
52.9	0.00443214805353789\\
52.91	0.00443315142758547\\
52.92	0.00443415488858082\\
52.93	0.0044351584351019\\
52.94	0.00443616206572038\\
52.95	0.00443716577900161\\
52.96	0.00443816957350462\\
52.97	0.00443917344778206\\
52.98	0.00444017740038018\\
52.99	0.00444118142983885\\
53	0.00444218553469146\\
53.01	0.00444318971346495\\
53.02	0.00444419396467979\\
53.03	0.00444519828684992\\
53.04	0.00444620267848274\\
53.05	0.00444720713807908\\
53.06	0.00444821166413319\\
53.07	0.0044492162551327\\
53.08	0.0044502209095586\\
53.09	0.00445122562588521\\
53.1	0.00445223040258015\\
53.11	0.00445323523810434\\
53.12	0.00445424013091193\\
53.13	0.00445524507945032\\
53.14	0.00445625008216011\\
53.15	0.00445725513747507\\
53.16	0.0044582602438221\\
53.17	0.00445926539962125\\
53.18	0.00446027060328566\\
53.19	0.00446127585322154\\
53.2	0.00446228114782813\\
53.21	0.0044632864854977\\
53.22	0.00446429186461551\\
53.23	0.00446529728355976\\
53.24	0.0044663027407016\\
53.25	0.0044673082344051\\
53.26	0.00446831376302718\\
53.27	0.00446931932491763\\
53.28	0.00447032491841905\\
53.29	0.00447133054186685\\
53.3	0.00447233619358921\\
53.31	0.00447334187190702\\
53.32	0.00447434757513393\\
53.33	0.00447535330157622\\
53.34	0.00447635904953286\\
53.35	0.00447736481729544\\
53.36	0.00447837060314815\\
53.37	0.00447937640536774\\
53.38	0.0044803822222235\\
53.39	0.00448138805197724\\
53.4	0.00448239389288324\\
53.41	0.00448339974318825\\
53.42	0.00448440560113142\\
53.43	0.00448541146494432\\
53.44	0.00448641733285089\\
53.45	0.00448742320306736\\
53.46	0.00448842907380231\\
53.47	0.00448943494325657\\
53.48	0.00449044080962323\\
53.49	0.0044914466710876\\
53.5	0.00449245252582715\\
53.51	0.00449345837201153\\
53.52	0.00449446420780251\\
53.53	0.00449547003135394\\
53.54	0.00449647584081175\\
53.55	0.0044974816343139\\
53.56	0.00449848740999034\\
53.57	0.00449949316596299\\
53.58	0.00450049890034573\\
53.59	0.00450150461124434\\
53.6	0.00450251029675646\\
53.61	0.00450351595497159\\
53.62	0.00450452158397105\\
53.63	0.00450552718182792\\
53.64	0.00450653274660705\\
53.65	0.00450753827636501\\
53.66	0.00450854376915002\\
53.67	0.004509549223002\\
53.68	0.00451055463595246\\
53.69	0.00451156000602453\\
53.7	0.00451256533123284\\
53.71	0.00451357060958361\\
53.72	0.0045145758390745\\
53.73	0.00451558101769466\\
53.74	0.00451658614342463\\
53.75	0.00451759121423638\\
53.76	0.0045185962280932\\
53.77	0.00451960118294974\\
53.78	0.0045206060767519\\
53.79	0.00452161090743686\\
53.8	0.00452261567293304\\
53.81	0.00452362037116001\\
53.82	0.00452462500002852\\
53.83	0.00452562955744042\\
53.84	0.00452663404128866\\
53.85	0.00452763844945724\\
53.86	0.00452864277982119\\
53.87	0.00452964703024647\\
53.88	0.00453065119859004\\
53.89	0.00453165528269975\\
53.9	0.00453265928041432\\
53.91	0.00453366318956332\\
53.92	0.00453466700796711\\
53.93	0.00453567073343683\\
53.94	0.00453667436377436\\
53.95	0.00453767789677227\\
53.96	0.00453868133021378\\
53.97	0.00453968466187275\\
53.98	0.00454068788951362\\
53.99	0.00454169101089137\\
54	0.00454269402375153\\
54.01	0.00454369692583007\\
54.02	0.00454469971485342\\
54.03	0.00454570238853841\\
54.04	0.00454670494459224\\
54.05	0.00454770738071244\\
54.06	0.00454870969458681\\
54.07	0.00454971188389344\\
54.08	0.0045507139463006\\
54.09	0.00455171587946676\\
54.1	0.00455271768104053\\
54.11	0.00455371934866063\\
54.12	0.00455472087995579\\
54.13	0.00455572227254485\\
54.14	0.00455672352403654\\
54.15	0.00455772463202962\\
54.16	0.00455872559411272\\
54.17	0.00455972640786433\\
54.18	0.00456072707085278\\
54.19	0.00456172758063621\\
54.2	0.00456272793476246\\
54.21	0.00456372813076913\\
54.22	0.00456472816618347\\
54.23	0.00456572803852235\\
54.24	0.00456672774529224\\
54.25	0.00456772728398917\\
54.26	0.00456872665209866\\
54.27	0.0045697258470957\\
54.28	0.00457072486644471\\
54.29	0.00457172370759949\\
54.3	0.00457272236800321\\
54.31	0.0045737208450883\\
54.32	0.00457471913627649\\
54.33	0.0045757172389787\\
54.34	0.00457671515059507\\
54.35	0.00457771286851481\\
54.36	0.00457871039011629\\
54.37	0.0045797077127669\\
54.38	0.00458070483382302\\
54.39	0.00458170175063002\\
54.4	0.00458269846052219\\
54.41	0.0045836949608227\\
54.42	0.00458469124884354\\
54.43	0.00458568732188553\\
54.44	0.0045866831772382\\
54.45	0.00458767881217979\\
54.46	0.00458867422397724\\
54.47	0.00458966940988606\\
54.48	0.00459066436715035\\
54.49	0.00459165909300276\\
54.5	0.00459265358466438\\
54.51	0.00459364783934479\\
54.52	0.00459464185424191\\
54.53	0.00459563562654206\\
54.54	0.00459662915341983\\
54.55	0.00459762243203808\\
54.56	0.00459861545954789\\
54.57	0.00459960823308849\\
54.58	0.00460060074978725\\
54.59	0.00460159300675959\\
54.6	0.00460258500110898\\
54.61	0.00460357672992686\\
54.62	0.00460456819029261\\
54.63	0.00460555937927349\\
54.64	0.00460655029392461\\
54.65	0.00460754093128886\\
54.66	0.00460853128839689\\
54.67	0.00460952136226704\\
54.68	0.00461051114990531\\
54.69	0.00461150064830528\\
54.7	0.00461248985444811\\
54.71	0.00461347876530244\\
54.72	0.00461446737782438\\
54.73	0.00461545568895745\\
54.74	0.00461644369563252\\
54.75	0.00461743139476778\\
54.76	0.00461841878326866\\
54.77	0.00461940585802784\\
54.78	0.0046203926159251\\
54.79	0.0046213790538274\\
54.8	0.00462236516858869\\
54.81	0.00462335095704999\\
54.82	0.00462433641603924\\
54.83	0.0046253215423713\\
54.84	0.00462630633284791\\
54.85	0.00462729078425756\\
54.86	0.00462827489337556\\
54.87	0.00462925865696388\\
54.88	0.00463024207177115\\
54.89	0.00463122513453262\\
54.9	0.00463220784197005\\
54.91	0.00463319019079173\\
54.92	0.00463417217769236\\
54.93	0.00463515379935306\\
54.94	0.00463613505244127\\
54.95	0.00463711593361071\\
54.96	0.00463809643950133\\
54.97	0.00463907656673926\\
54.98	0.00464005631193676\\
54.99	0.00464103567169213\\
55	0.00464201464258972\\
55.01	0.00464299322119982\\
55.02	0.00464397140407861\\
55.03	0.00464494918776814\\
55.04	0.00464592656879625\\
55.05	0.00464690354367651\\
55.06	0.00464788010890818\\
55.07	0.00464885626097614\\
55.08	0.00464983199635086\\
55.09	0.00465080731148828\\
55.1	0.00465178220282984\\
55.11	0.00465275666680238\\
55.12	0.00465373069981805\\
55.13	0.00465470429827433\\
55.14	0.00465567745855391\\
55.15	0.00465665017702463\\
55.16	0.00465762245003948\\
55.17	0.00465859427393648\\
55.18	0.00465956564503867\\
55.19	0.00466053655965402\\
55.2	0.00466150701407538\\
55.21	0.0046624770045804\\
55.22	0.00466344652743152\\
55.23	0.00466441557887588\\
55.24	0.00466538415514525\\
55.25	0.004666352252456\\
55.26	0.004667319867009\\
55.27	0.00466828699498959\\
55.28	0.00466925363256753\\
55.29	0.00467021977589691\\
55.3	0.00467118542111607\\
55.31	0.00467215056434762\\
55.32	0.00467311520169828\\
55.33	0.0046740793292589\\
55.34	0.00467504294310433\\
55.35	0.00467600603929342\\
55.36	0.00467696861386891\\
55.37	0.00467793066285738\\
55.38	0.00467889218226922\\
55.39	0.00467985316809851\\
55.4	0.00468081361632297\\
55.41	0.00468177352290396\\
55.42	0.00468273288378632\\
55.43	0.00468369169489839\\
55.44	0.00468464995215186\\
55.45	0.0046856076514418\\
55.46	0.00468656478864653\\
55.47	0.00468752135962755\\
55.48	0.00468847736022953\\
55.49	0.00468943278628019\\
55.5	0.00469038763359025\\
55.51	0.00469134189795339\\
55.52	0.00469229557514615\\
55.53	0.00469324866092785\\
55.54	0.00469420115104058\\
55.55	0.00469515304120909\\
55.56	0.00469610432714071\\
55.57	0.00469705500452534\\
55.58	0.00469800506903532\\
55.59	0.0046989545163254\\
55.6	0.00469990334203262\\
55.61	0.00470085154177633\\
55.62	0.00470179911115803\\
55.63	0.00470274604576137\\
55.64	0.00470369234115202\\
55.65	0.00470463867894219\\
55.66	0.0047055853442016\\
55.67	0.00470653233644948\\
55.68	0.00470747965520182\\
55.69	0.00470842729997135\\
55.7	0.00470937527026748\\
55.71	0.00471032356559634\\
55.72	0.00471127218546071\\
55.73	0.00471222112936008\\
55.74	0.00471317039679056\\
55.75	0.0047141199872449\\
55.76	0.00471506990021247\\
55.77	0.00471602013517924\\
55.78	0.0047169706916278\\
55.79	0.00471792156903725\\
55.8	0.00471887276688332\\
55.81	0.00471982428463821\\
55.82	0.00472077612177069\\
55.83	0.00472172827774604\\
55.84	0.00472268075202599\\
55.85	0.00472363354406879\\
55.86	0.00472458665332914\\
55.87	0.00472554007925818\\
55.88	0.00472649382130346\\
55.89	0.00472744787890898\\
55.9	0.00472840225151509\\
55.91	0.00472935693855855\\
55.92	0.00473031193947249\\
55.93	0.00473126725368634\\
55.94	0.00473222288062589\\
55.95	0.00473317881971325\\
55.96	0.00473413507036679\\
55.97	0.00473509163200119\\
55.98	0.00473604850402738\\
55.99	0.00473700568585251\\
56	0.00473796317687999\\
56.01	0.00473892097650941\\
56.02	0.00473987908413658\\
56.03	0.00474083749915345\\
56.04	0.00474179622094814\\
56.05	0.00474275524890493\\
56.06	0.00474371458240418\\
56.07	0.00474467422082239\\
56.08	0.00474563416353212\\
56.09	0.00474659440990199\\
56.1	0.00474755495929671\\
56.11	0.00474851581107696\\
56.12	0.0047494769645995\\
56.13	0.00475043841921703\\
56.14	0.00475140017427825\\
56.15	0.0047523622291278\\
56.16	0.0047533245831063\\
56.17	0.00475428723555023\\
56.18	0.00475525018579203\\
56.19	0.00475621343315998\\
56.2	0.00475717697697825\\
56.21	0.00475814081656684\\
56.22	0.00475910495124157\\
56.23	0.00476006938031409\\
56.24	0.00476103410309183\\
56.25	0.00476199911887797\\
56.26	0.00476296442697147\\
56.27	0.00476393002666698\\
56.28	0.0047648959172549\\
56.29	0.0047658620980213\\
56.3	0.00476682856824793\\
56.31	0.00476779532721217\\
56.32	0.00476876237418706\\
56.33	0.00476972970844123\\
56.34	0.00477069732923892\\
56.35	0.00477166523583993\\
56.36	0.0047726334274996\\
56.37	0.00477360190346882\\
56.38	0.00477457066299397\\
56.39	0.00477553970531695\\
56.4	0.0047765090296751\\
56.41	0.00477747863530122\\
56.42	0.00477844852142355\\
56.43	0.0047794186872657\\
56.44	0.00478038913204672\\
56.45	0.00478135985498097\\
56.46	0.00478233085527819\\
56.47	0.00478330213214343\\
56.48	0.00478427368477705\\
56.49	0.00478524551237466\\
56.5	0.00478621761412718\\
56.51	0.00478718998922072\\
56.52	0.00478816263683662\\
56.53	0.00478913555615143\\
56.54	0.00479010874633685\\
56.55	0.00479108220655973\\
56.56	0.00479205593598207\\
56.57	0.00479302993376094\\
56.58	0.00479400419904854\\
56.59	0.00479497873099208\\
56.6	0.00479595352873384\\
56.61	0.00479692859141111\\
56.62	0.00479790391815615\\
56.63	0.00479887950809624\\
56.64	0.00479985536035355\\
56.65	0.00480083147404522\\
56.66	0.00480180784828326\\
56.67	0.00480278448217458\\
56.68	0.00480376137482092\\
56.69	0.00480473852531889\\
56.7	0.00480571593275986\\
56.71	0.00480669359623003\\
56.72	0.00480767151481033\\
56.73	0.00480864968757642\\
56.74	0.00480962811359872\\
56.75	0.0048106067919423\\
56.76	0.00481158572166689\\
56.77	0.00481256490182688\\
56.78	0.00481354433147127\\
56.79	0.00481452400964365\\
56.8	0.0048155039353822\\
56.81	0.00481648410771959\\
56.82	0.00481746452568307\\
56.83	0.00481844518829433\\
56.84	0.00481942609456956\\
56.85	0.0048204072435194\\
56.86	0.00482138863414887\\
56.87	0.00482237026545741\\
56.88	0.00482335213643884\\
56.89	0.00482433424608128\\
56.9	0.00482531659336721\\
56.91	0.00482629917727336\\
56.92	0.00482728199677077\\
56.93	0.00482826505082468\\
56.94	0.00482924833839455\\
56.95	0.00483023185843404\\
56.96	0.00483121560989097\\
56.97	0.00483219959170728\\
56.98	0.00483318380281901\\
56.99	0.0048341682421563\\
57	0.00483515290864334\\
57.01	0.00483613780119834\\
57.02	0.00483712291873351\\
57.03	0.00483810826015504\\
57.04	0.00483909382436306\\
57.05	0.00484007961025161\\
57.06	0.00484106561670863\\
57.07	0.00484205183942905\\
57.08	0.00484303816927884\\
57.09	0.0048440246053821\\
57.1	0.00484501114685959\\
57.11	0.00484599779282872\\
57.12	0.00484698454240356\\
57.13	0.00484797139469477\\
57.14	0.00484895834880966\\
57.15	0.00484994540385211\\
57.16	0.00485093255892261\\
57.17	0.00485191981311825\\
57.18	0.00485290716553264\\
57.19	0.00485389461525598\\
57.2	0.00485488216137499\\
57.21	0.00485586980297293\\
57.22	0.00485685753912959\\
57.23	0.00485784536892124\\
57.24	0.00485883329142067\\
57.25	0.00485982130569712\\
57.26	0.00486080941081633\\
57.27	0.00486179760584048\\
57.28	0.00486278588982818\\
57.29	0.00486377426183449\\
57.3	0.0048647627209109\\
57.31	0.00486575126610526\\
57.32	0.00486673989646187\\
57.33	0.00486772861102136\\
57.34	0.00486871740882075\\
57.35	0.00486970628889341\\
57.36	0.00487069525026906\\
57.37	0.00487168429197372\\
57.38	0.00487267341302976\\
57.39	0.00487366261245583\\
57.4	0.00487465188926688\\
57.41	0.00487564124247412\\
57.42	0.00487663067108504\\
57.43	0.00487762017410337\\
57.44	0.00487860975052908\\
57.45	0.00487959939935836\\
57.46	0.0048805891195836\\
57.47	0.0048815789101934\\
57.48	0.00488256877017253\\
57.49	0.00488355869850195\\
57.5	0.00488454869415873\\
57.51	0.00488553875611613\\
57.52	0.0048865288833435\\
57.53	0.00488751907480634\\
57.54	0.00488850932946622\\
57.55	0.0048894996462808\\
57.56	0.00489049002420382\\
57.57	0.00489148046218506\\
57.58	0.00489247095917038\\
57.59	0.00489346151410163\\
57.6	0.0048944521259167\\
57.61	0.00489544279354948\\
57.62	0.00489643351592982\\
57.63	0.00489742429198357\\
57.64	0.00489841512063254\\
57.65	0.00489940600079446\\
57.66	0.00490039693138301\\
57.67	0.00490138791130779\\
57.68	0.00490237893947426\\
57.69	0.00490337001478382\\
57.7	0.0049043611361337\\
57.71	0.004905352302417\\
57.72	0.00490634351252267\\
57.73	0.00490733476533547\\
57.74	0.00490832605973598\\
57.75	0.00490931739460057\\
57.76	0.00491030876880141\\
57.77	0.0049113001812064\\
57.78	0.00491229163067924\\
57.79	0.00491328311607932\\
57.8	0.00491427463626177\\
57.81	0.00491526619007743\\
57.82	0.00491625777637283\\
57.83	0.00491724939399015\\
57.84	0.00491824104176727\\
57.85	0.00491923271853768\\
57.86	0.0049202244231305\\
57.87	0.00492121615437048\\
57.88	0.00492220791107796\\
57.89	0.00492319969206885\\
57.9	0.00492419149615463\\
57.91	0.00492518332214234\\
57.92	0.00492617516883453\\
57.93	0.00492716703502929\\
57.94	0.00492815891952019\\
57.95	0.0049291508210963\\
57.96	0.00493014273854215\\
57.97	0.00493113467063772\\
57.98	0.00493212661615844\\
57.99	0.00493311857387514\\
58	0.00493411054255407\\
58.01	0.00493510252095684\\
58.02	0.00493609450784047\\
58.03	0.00493708650195728\\
58.04	0.00493807850205497\\
58.05	0.00493907050687654\\
58.06	0.00494006251516031\\
58.07	0.00494105452563985\\
58.08	0.00494204653704403\\
58.09	0.00494303854809697\\
58.1	0.004944030557518\\
58.11	0.0049450225640217\\
58.12	0.00494601456631782\\
58.13	0.0049470065631113\\
58.14	0.00494799855310227\\
58.15	0.00494899053498598\\
58.16	0.00494998250745281\\
58.17	0.00495097446918826\\
58.18	0.00495196641887294\\
58.19	0.00495295835518251\\
58.2	0.00495395027678771\\
58.21	0.00495494218235431\\
58.22	0.00495593407054311\\
58.23	0.00495692594000991\\
58.24	0.00495791778940551\\
58.25	0.00495890961737565\\
58.26	0.00495990142256107\\
58.27	0.0049608932035974\\
58.28	0.0049618849591152\\
58.29	0.00496287668773992\\
58.3	0.00496386838809191\\
58.31	0.00496486005878635\\
58.32	0.00496585169843328\\
58.33	0.00496684330563756\\
58.34	0.00496783487899883\\
58.35	0.00496882641711156\\
58.36	0.00496981791856494\\
58.37	0.00497080938194294\\
58.38	0.00497180080582424\\
58.39	0.00497279218878222\\
58.4	0.00497378352938498\\
58.41	0.00497477482619527\\
58.42	0.00497576607777047\\
58.43	0.00497675728266263\\
58.44	0.0049777484394184\\
58.45	0.004978739546579\\
58.46	0.00497973060268025\\
58.47	0.00498072160625251\\
58.48	0.00498171255582067\\
58.49	0.00498270344990415\\
58.5	0.00498369428701684\\
58.51	0.00498468506566713\\
58.52	0.00498567578435784\\
58.53	0.00498666644158623\\
58.54	0.00498765703584397\\
58.55	0.00498864756561712\\
58.56	0.00498963802938614\\
58.57	0.0049906284256258\\
58.58	0.00499161875280523\\
58.59	0.00499260900938786\\
58.6	0.0049935991938314\\
58.61	0.00499458930458786\\
58.62	0.00499557934010346\\
58.63	0.00499656929881866\\
58.64	0.00499755917916814\\
58.65	0.00499854897958076\\
58.66	0.00499953869847952\\
58.67	0.00500052833428159\\
58.68	0.00500151788539826\\
58.69	0.0050025073502349\\
58.7	0.00500349672719097\\
58.71	0.00500448601465998\\
58.72	0.00500547521102949\\
58.73	0.00500646431468106\\
58.74	0.00500745332399025\\
58.75	0.00500844223732656\\
58.76	0.00500943105305347\\
58.77	0.00501041976952838\\
58.78	0.00501140838510259\\
58.79	0.00501239689812127\\
58.8	0.00501338530692345\\
58.81	0.00501437360984201\\
58.82	0.00501536180520363\\
58.83	0.00501634989132879\\
58.84	0.00501733786653173\\
58.85	0.00501832572912043\\
58.86	0.00501931347739661\\
58.87	0.00502030110965567\\
58.88	0.00502128862418669\\
58.89	0.00502227601927242\\
58.9	0.00502326329318922\\
58.91	0.00502425044420705\\
58.92	0.00502523747058947\\
58.93	0.0050262243705936\\
58.94	0.00502721114247009\\
58.95	0.00502819778446309\\
58.96	0.00502918429481027\\
58.97	0.00503017067174272\\
58.98	0.005031156913485\\
58.99	0.00503214301825509\\
59	0.00503312898426434\\
59.01	0.00503411480971749\\
59.02	0.0050351004928126\\
59.03	0.00503608603174106\\
59.04	0.00503707142468757\\
59.05	0.00503805666983009\\
59.06	0.0050390417653398\\
59.07	0.00504002670938113\\
59.08	0.0050410115001117\\
59.09	0.00504199613568229\\
59.1	0.00504298061423683\\
59.11	0.00504396493391238\\
59.12	0.00504494909283906\\
59.13	0.00504593308914012\\
59.14	0.00504691692093179\\
59.15	0.00504790058632336\\
59.16	0.00504888408341709\\
59.17	0.00504986741030822\\
59.18	0.00505085056508493\\
59.19	0.0050518335458283\\
59.2	0.00505281635061231\\
59.21	0.00505379897750381\\
59.22	0.00505478142456246\\
59.23	0.00505576368984075\\
59.24	0.00505674577138396\\
59.25	0.00505772766723011\\
59.26	0.00505870937540995\\
59.27	0.00505969089394694\\
59.28	0.00506067222085721\\
59.29	0.00506165335414955\\
59.3	0.00506263429182536\\
59.31	0.00506361503187864\\
59.32	0.00506459557229596\\
59.33	0.00506557591105641\\
59.34	0.00506655604613161\\
59.35	0.00506753597548568\\
59.36	0.00506851569707516\\
59.37	0.00506949520884905\\
59.38	0.00507047450874873\\
59.39	0.00507145359470796\\
59.4	0.00507243246465285\\
59.41	0.00507341111650181\\
59.42	0.00507438954816555\\
59.43	0.00507536775754705\\
59.44	0.0050763457425415\\
59.45	0.00507732350103629\\
59.46	0.005078301030911\\
59.47	0.00507927833003736\\
59.48	0.00508025539627918\\
59.49	0.00508123222749239\\
59.5	0.00508220882152496\\
59.51	0.00508318517621689\\
59.52	0.00508416128940017\\
59.53	0.00508513715889877\\
59.54	0.00508611278252858\\
59.55	0.00508708815809743\\
59.56	0.00508806328340499\\
59.57	0.0050890381562428\\
59.58	0.00509001277439421\\
59.59	0.00509098713563437\\
59.6	0.00509196123773017\\
59.61	0.00509293507844022\\
59.62	0.00509390865551484\\
59.63	0.00509488196669602\\
59.64	0.00509585500971738\\
59.65	0.00509682778230412\\
59.66	0.00509780028217303\\
59.67	0.00509877250703244\\
59.68	0.00509974445458218\\
59.69	0.00510071612251357\\
59.7	0.00510168750850937\\
59.71	0.00510265861024373\\
59.72	0.00510362942538222\\
59.73	0.00510459995158173\\
59.74	0.00510557018649049\\
59.75	0.00510654012774799\\
59.76	0.00510750977298498\\
59.77	0.00510847911982343\\
59.78	0.00510944816587652\\
59.79	0.00511041690874856\\
59.8	0.00511138534603496\\
59.81	0.00511235347532227\\
59.82	0.00511332129418805\\
59.83	0.00511428880020091\\
59.84	0.00511525599092043\\
59.85	0.00511622286389714\\
59.86	0.00511718941667252\\
59.87	0.00511815564677892\\
59.88	0.00511912155173953\\
59.89	0.00512008712906836\\
59.9	0.00512105237627022\\
59.91	0.00512201729084068\\
59.92	0.00512298187026598\\
59.93	0.0051239461120231\\
59.94	0.00512491001357961\\
59.95	0.00512587357239372\\
59.96	0.00512683678591421\\
59.97	0.00512779965158041\\
59.98	0.00512876216682213\\
59.99	0.00512972432905969\\
60	0.00513068613570381\\
60.01	0.00513164758415562\\
60.02	0.00513260867180659\\
60.03	0.00513356939603857\\
60.04	0.00513452975422365\\
60.05	0.00513548974372419\\
60.06	0.00513644936189278\\
60.07	0.00513740860607216\\
60.08	0.00513836747359524\\
60.09	0.00513932596178504\\
60.1	0.00514028406795463\\
60.11	0.00514124178940713\\
60.12	0.00514219912343564\\
60.13	0.00514315606732323\\
60.14	0.00514411261834288\\
60.15	0.00514506877375747\\
60.16	0.00514602453081971\\
60.17	0.0051469798867721\\
60.18	0.00514793483884695\\
60.19	0.00514888938426627\\
60.2	0.00514984352024177\\
60.21	0.0051507972439748\\
60.22	0.00515175055265635\\
60.23	0.00515270344346697\\
60.24	0.00515365591357674\\
60.25	0.00515460796014524\\
60.26	0.00515555958032151\\
60.27	0.00515651077124401\\
60.28	0.00515746153004057\\
60.29	0.00515841185382836\\
60.3	0.00515936173971386\\
60.31	0.00516031118479278\\
60.32	0.00516126018615008\\
60.33	0.00516220874085986\\
60.34	0.0051631568459854\\
60.35	0.00516410449857903\\
60.36	0.00516505169568218\\
60.37	0.00516599843432524\\
60.38	0.00516694471152761\\
60.39	0.00516789052429761\\
60.4	0.00516883586963245\\
60.41	0.00516978074451818\\
60.42	0.00517072514592966\\
60.43	0.00517166907083051\\
60.44	0.00517261251617308\\
60.45	0.00517355547889837\\
60.46	0.00517449795593604\\
60.47	0.00517543994420434\\
60.48	0.00517638144061006\\
60.49	0.00517732244204851\\
60.5	0.00517826294540343\\
60.51	0.00517920294754703\\
60.52	0.00518014244533985\\
60.53	0.00518108143563077\\
60.54	0.00518201991525699\\
60.55	0.00518295788104392\\
60.56	0.005183895951657\\
60.57	0.00518483419195687\\
60.58	0.00518577260158482\\
60.59	0.00518671118018124\\
60.6	0.00518764992738566\\
60.61	0.00518858884283672\\
60.62	0.00518952792617221\\
60.63	0.00519046717702903\\
60.64	0.00519140659504318\\
60.65	0.00519234617984981\\
60.66	0.00519328593108318\\
60.67	0.00519422584837667\\
60.68	0.00519516593136278\\
60.69	0.00519610617967312\\
60.7	0.00519704659293844\\
60.71	0.00519798717078858\\
60.72	0.00519892791285251\\
60.73	0.00519986881875831\\
60.74	0.00520080988813319\\
60.75	0.00520175112060344\\
60.76	0.00520269251579451\\
60.77	0.00520363407333092\\
60.78	0.00520457579283631\\
60.79	0.00520551767393346\\
60.8	0.00520645971624422\\
60.81	0.00520740191938959\\
60.82	0.00520834428298964\\
60.83	0.00520928680666358\\
60.84	0.0052102294900297\\
60.85	0.00521117233270542\\
60.86	0.00521211533430724\\
60.87	0.00521305849445081\\
60.88	0.00521400181275084\\
60.89	0.00521494528882117\\
60.9	0.00521588892227474\\
60.91	0.00521683271272357\\
60.92	0.0052177766597788\\
60.93	0.00521872076305069\\
60.94	0.00521966502214857\\
60.95	0.00522060943668088\\
60.96	0.00522155400625518\\
60.97	0.00522249873047809\\
60.98	0.00522344360895538\\
60.99	0.00522438864129186\\
61	0.00522533382709149\\
61.01	0.00522627916595729\\
61.02	0.0052272246574914\\
61.03	0.00522817030129503\\
61.04	0.00522911609696852\\
61.05	0.00523006204411128\\
61.06	0.0052310081423218\\
61.07	0.00523195439119771\\
61.08	0.00523290079033569\\
61.09	0.00523384733933153\\
61.1	0.00523479403778013\\
61.11	0.00523574088527543\\
61.12	0.00523668788141052\\
61.13	0.00523763502577754\\
61.14	0.00523858231796773\\
61.15	0.00523952975757144\\
61.16	0.00524047734417808\\
61.17	0.00524142507737615\\
61.18	0.00524237295675327\\
61.19	0.00524332098189611\\
61.2	0.00524426915239045\\
61.21	0.00524521746782115\\
61.22	0.00524616592777214\\
61.23	0.00524711453182647\\
61.24	0.00524806327956624\\
61.25	0.00524901217057267\\
61.26	0.00524996120442604\\
61.27	0.00525091038070571\\
61.28	0.00525185969899015\\
61.29	0.00525280915885688\\
61.3	0.00525375875988252\\
61.31	0.00525470850164279\\
61.32	0.00525565838371246\\
61.33	0.00525660840566539\\
61.34	0.00525755856707455\\
61.35	0.00525850886751195\\
61.36	0.0052594593065487\\
61.37	0.005260409883755\\
61.38	0.00526136059870011\\
61.39	0.00526231145095238\\
61.4	0.00526326244007924\\
61.41	0.0052642135656472\\
61.42	0.00526516482722184\\
61.43	0.00526611622436784\\
61.44	0.00526706775664892\\
61.45	0.00526801942362793\\
61.46	0.00526897122486675\\
61.47	0.00526992315992635\\
61.48	0.0052708752283668\\
61.49	0.00527182742974722\\
61.5	0.00527277976362581\\
61.51	0.00527373222955988\\
61.52	0.00527468482710577\\
61.53	0.00527563755581893\\
61.54	0.00527659041525385\\
61.55	0.00527754340496413\\
61.56	0.00527849652450245\\
61.57	0.00527944977342052\\
61.58	0.00528040315126917\\
61.59	0.00528135665759828\\
61.6	0.00528231029195681\\
61.61	0.00528326405389282\\
61.62	0.0052842179429534\\
61.63	0.00528517195868475\\
61.64	0.00528612610063212\\
61.65	0.00528708036833987\\
61.66	0.00528803476135137\\
61.67	0.00528898927920915\\
61.68	0.00528994392145474\\
61.69	0.00529089868762878\\
61.7	0.00529185357727096\\
61.71	0.00529280858992007\\
61.72	0.00529376372511398\\
61.73	0.0052947189823896\\
61.74	0.00529567436128294\\
61.75	0.00529662986132907\\
61.76	0.00529758548206213\\
61.77	0.00529854122301536\\
61.78	0.00529949708372104\\
61.79	0.00530045306371055\\
61.8	0.00530140916251434\\
61.81	0.0053023653796619\\
61.82	0.00530332171468185\\
61.83	0.00530427816710183\\
61.84	0.00530523473644861\\
61.85	0.00530619142224798\\
61.86	0.00530714822402484\\
61.87	0.00530810514130314\\
61.88	0.00530906217360593\\
61.89	0.00531001932045532\\
61.9	0.00531097658137249\\
61.91	0.00531193395587771\\
61.92	0.00531289144349034\\
61.93	0.00531384904372883\\
61.94	0.0053148067561107\\
61.95	0.00531576458015257\\
61.96	0.00531672251537016\\
61.97	0.00531768056127824\\
61.98	0.0053186387173907\\
61.99	0.0053195969832205\\
62	0.0053205553582797\\
62.01	0.00532151384207943\\
62.02	0.00532247243412993\\
62.03	0.00532343113394051\\
62.04	0.00532438994101959\\
62.05	0.00532534885487465\\
62.06	0.00532630787501229\\
62.07	0.00532726700093816\\
62.08	0.00532822623215705\\
62.09	0.0053291855681728\\
62.1	0.00533014500848836\\
62.11	0.00533110455260576\\
62.12	0.00533206420002613\\
62.13	0.00533302395024968\\
62.14	0.00533398380277574\\
62.15	0.00533494375710269\\
62.16	0.00533590381272805\\
62.17	0.00533686396914838\\
62.18	0.00533782422585939\\
62.19	0.00533878458235584\\
62.2	0.00533974503813162\\
62.21	0.00534070559267966\\
62.22	0.00534166624549206\\
62.23	0.00534262699605996\\
62.24	0.00534358784387361\\
62.25	0.00534454878842238\\
62.26	0.0053455098291947\\
62.27	0.00534647096567813\\
62.28	0.00534743219735932\\
62.29	0.00534839352372399\\
62.3	0.00534935494425703\\
62.31	0.00535031645844235\\
62.32	0.00535127806576301\\
62.33	0.00535223976570116\\
62.34	0.00535320155773806\\
62.35	0.00535416344135405\\
62.36	0.0053551254160286\\
62.37	0.00535608748124027\\
62.38	0.00535704963646672\\
62.39	0.00535801188118475\\
62.4	0.00535897421487023\\
62.41	0.00535993663699814\\
62.42	0.00536089914704259\\
62.43	0.00536186174447678\\
62.44	0.00536282442877303\\
62.45	0.00536378719940277\\
62.46	0.00536475005583653\\
62.47	0.00536571299754397\\
62.48	0.00536667602399386\\
62.49	0.00536763913465407\\
62.5	0.00536860232899159\\
62.51	0.00536956560647255\\
62.52	0.00537052896656215\\
62.53	0.00537149240872475\\
62.54	0.00537245593242382\\
62.55	0.00537341953712192\\
62.56	0.00537438322228077\\
62.57	0.0053753469873612\\
62.58	0.00537631083182315\\
62.59	0.0053772747551257\\
62.6	0.00537823875672704\\
62.61	0.00537920283608449\\
62.62	0.0053801669926545\\
62.63	0.00538113122589266\\
62.64	0.00538209553525366\\
62.65	0.00538305992019135\\
62.66	0.0053840243801587\\
62.67	0.00538498891460779\\
62.68	0.00538595352298989\\
62.69	0.00538691820475534\\
62.7	0.00538788295935367\\
62.71	0.00538884778623351\\
62.72	0.00538981268484265\\
62.73	0.005390777654628\\
62.74	0.00539174269503565\\
62.75	0.00539270780551078\\
62.76	0.00539367298549775\\
62.77	0.00539463823444007\\
62.78	0.00539560355178038\\
62.79	0.00539656893696046\\
62.8	0.00539753438942125\\
62.81	0.00539849990860286\\
62.82	0.00539946549394452\\
62.83	0.00540043114488464\\
62.84	0.00540139686086076\\
62.85	0.0054023626413096\\
62.86	0.00540332848566703\\
62.87	0.00540429439336807\\
62.88	0.00540526036384692\\
62.89	0.00540622639653693\\
62.9	0.00540719249087062\\
62.91	0.00540815864627967\\
62.92	0.00540912486219494\\
62.93	0.00541009113804645\\
62.94	0.00541105747326339\\
62.95	0.00541202386727413\\
62.96	0.00541299031950621\\
62.97	0.00541395682938634\\
62.98	0.00541492339634043\\
62.99	0.00541589001979355\\
63	0.00541685669916996\\
63.01	0.00541782343389308\\
63.02	0.00541879022338557\\
63.03	0.00541975706706922\\
63.04	0.00542072396436504\\
63.05	0.00542169091469323\\
63.06	0.00542265791747317\\
63.07	0.00542362497212345\\
63.08	0.00542459207806185\\
63.09	0.00542555923470533\\
63.1	0.0054265264414701\\
63.11	0.00542749369777151\\
63.12	0.00542846100302417\\
63.13	0.00542942835664186\\
63.14	0.0054303957580376\\
63.15	0.0054313632066236\\
63.16	0.00543233070181129\\
63.17	0.00543329824301131\\
63.18	0.00543426582963352\\
63.19	0.00543523346108701\\
63.2	0.00543620113678009\\
63.21	0.00543716885612028\\
63.22	0.00543813661851436\\
63.23	0.00543910442336829\\
63.24	0.0054400722700873\\
63.25	0.00544104015807585\\
63.26	0.00544200808673761\\
63.27	0.00544297605547553\\
63.28	0.00544394406369177\\
63.29	0.00544491211078774\\
63.3	0.00544588019616411\\
63.31	0.00544684831922079\\
63.32	0.00544781647935692\\
63.33	0.00544878467597094\\
63.34	0.00544975290846052\\
63.35	0.00545072117622257\\
63.36	0.00545168947865329\\
63.37	0.00545265781514815\\
63.38	0.00545362618510185\\
63.39	0.00545459458790841\\
63.4	0.00545556302296107\\
63.41	0.0054565314896524\\
63.42	0.00545749998737419\\
63.43	0.00545846851551757\\
63.44	0.00545943707347291\\
63.45	0.00546040566062989\\
63.46	0.00546137427637746\\
63.47	0.00546234292010389\\
63.48	0.00546331159119672\\
63.49	0.00546428028904281\\
63.5	0.00546524901302831\\
63.51	0.00546621776253866\\
63.52	0.00546718653695864\\
63.53	0.00546815533567231\\
63.54	0.00546912415806308\\
63.55	0.00547009300351364\\
63.56	0.00547106187140602\\
63.57	0.00547203076112158\\
63.58	0.00547299967204098\\
63.59	0.00547396860354425\\
63.6	0.0054749375550107\\
63.61	0.00547590652581904\\
63.62	0.00547687551534728\\
63.63	0.00547784452297277\\
63.64	0.00547881354807223\\
63.65	0.00547978259002173\\
63.66	0.00548075164819666\\
63.67	0.00548172072197181\\
63.68	0.0054826898107213\\
63.69	0.00548365891381864\\
63.7	0.0054846280306367\\
63.71	0.0054855971605477\\
63.72	0.00548656630292327\\
63.73	0.00548753545713439\\
63.74	0.00548850462255145\\
63.75	0.00548947379854421\\
63.76	0.00549044298448182\\
63.77	0.00549141217973283\\
63.78	0.00549238138366519\\
63.79	0.00549335059564625\\
63.8	0.00549431981504277\\
63.81	0.0054952890412209\\
63.82	0.00549625827354625\\
63.83	0.0054972275113838\\
63.84	0.00549819675409797\\
63.85	0.00549916600105262\\
63.86	0.00550013525161103\\
63.87	0.0055011045051359\\
63.88	0.0055020737609894\\
63.89	0.00550304301853312\\
63.9	0.0055040122771281\\
63.91	0.00550498153613485\\
63.92	0.00550595079491332\\
63.93	0.00550692005282292\\
63.94	0.00550788930922253\\
63.95	0.0055088585634705\\
63.96	0.00550982781492466\\
63.97	0.0055107970629423\\
63.98	0.00551176630688022\\
63.99	0.00551273554609467\\
64	0.00551370477994145\\
64.01	0.0055146740077758\\
64.02	0.00551564322895249\\
64.03	0.0055166124428258\\
64.04	0.00551758164874951\\
64.05	0.00551855084607691\\
64.06	0.00551952003416085\\
64.07	0.00552048921235365\\
64.08	0.00552145838000719\\
64.09	0.0055224275364729\\
64.1	0.00552339668110174\\
64.11	0.0055243658132442\\
64.12	0.00552533493225034\\
64.13	0.00552630403746977\\
64.14	0.00552727312825167\\
64.15	0.00552824220394478\\
64.16	0.0055292112638974\\
64.17	0.00553018030745744\\
64.18	0.00553114933397236\\
64.19	0.00553211834278921\\
64.2	0.00553308733325467\\
64.21	0.00553405630471498\\
64.22	0.005535025256516\\
64.23	0.00553599418800321\\
64.24	0.0055369630985217\\
64.25	0.00553793198741617\\
64.26	0.00553890085403095\\
64.27	0.00553986969771001\\
64.28	0.00554083851779696\\
64.29	0.00554180731363507\\
64.3	0.00554277608456722\\
64.31	0.00554374482993598\\
64.32	0.00554471354908358\\
64.33	0.00554568224135189\\
64.34	0.00554665090608249\\
64.35	0.00554761954261661\\
64.36	0.00554858815029521\\
64.37	0.0055495567284589\\
64.38	0.005550525276448\\
64.39	0.00555149379360256\\
64.4	0.0055524622792623\\
64.41	0.00555343073276669\\
64.42	0.00555439915345492\\
64.43	0.00555536754066591\\
64.44	0.00555633589373829\\
64.45	0.00555730421201049\\
64.46	0.00555827249482065\\
64.47	0.00555924074150666\\
64.48	0.00556020895140621\\
64.49	0.00556117712385674\\
64.5	0.00556214525819547\\
64.51	0.00556311335375939\\
64.52	0.00556408140988531\\
64.53	0.00556504942590982\\
64.54	0.00556601740116931\\
64.55	0.00556698533500002\\
64.56	0.00556795322673795\\
64.57	0.00556892107571898\\
64.58	0.00556988888127879\\
64.59	0.00557085664275292\\
64.6	0.00557182435947675\\
64.61	0.00557279203078552\\
64.62	0.00557375965601434\\
64.63	0.00557472723449817\\
64.64	0.00557569476557187\\
64.65	0.00557666224857017\\
64.66	0.0055776296828277\\
64.67	0.00557859706767899\\
64.68	0.00557956440245848\\
64.69	0.00558053168650054\\
64.7	0.00558149891913944\\
64.71	0.00558246609970939\\
64.72	0.00558343322754456\\
64.73	0.00558440030197904\\
64.74	0.00558536732234688\\
64.75	0.00558633428798212\\
64.76	0.00558730119821876\\
64.77	0.00558826805239077\\
64.78	0.00558923484983211\\
64.79	0.00559020158987676\\
64.8	0.00559116827185868\\
64.81	0.00559213489511186\\
64.82	0.0055931014589703\\
64.83	0.00559406796276805\\
64.84	0.00559503440583919\\
64.85	0.00559600078751783\\
64.86	0.00559696710713817\\
64.87	0.00559793336403448\\
64.88	0.00559889955754106\\
64.89	0.00559986568699233\\
64.9	0.00560083175172282\\
64.91	0.00560179775106711\\
64.92	0.00560276368435993\\
64.93	0.00560372955093612\\
64.94	0.00560469535013065\\
64.95	0.00560566108127863\\
64.96	0.00560662674371533\\
64.97	0.00560759233677614\\
64.98	0.00560855785979667\\
64.99	0.00560952331211267\\
65	0.00561048869306009\\
65.01	0.00561145400197507\\
65.02	0.00561241923819397\\
65.03	0.00561338440105335\\
65.04	0.00561434948989\\
65.05	0.00561531450404095\\
65.06	0.00561627944284347\\
65.07	0.00561724430563509\\
65.08	0.00561820909175362\\
65.09	0.00561917380053711\\
65.1	0.00562013843132393\\
65.11	0.00562110298345274\\
65.12	0.00562206745626249\\
65.13	0.00562303184909248\\
65.14	0.00562399616128229\\
65.15	0.00562496039217189\\
65.16	0.00562592454110157\\
65.17	0.00562688860741199\\
65.18	0.00562785259044418\\
65.19	0.00562881648953954\\
65.2	0.00562978030403989\\
65.21	0.00563074403328743\\
65.22	0.00563170767662478\\
65.23	0.005632671233395\\
65.24	0.00563363470294157\\
65.25	0.00563459808460844\\
65.26	0.00563556137773998\\
65.27	0.00563652458168109\\
65.28	0.0056374876957771\\
65.29	0.00563845071937387\\
65.3	0.00563941365181775\\
65.31	0.00564037649245561\\
65.32	0.00564133924063488\\
65.33	0.00564230189570348\\
65.34	0.00564326445700992\\
65.35	0.00564422692390326\\
65.36	0.00564518929573316\\
65.37	0.00564615157184984\\
65.38	0.00564711375160416\\
65.39	0.00564807583434756\\
65.4	0.00564903781943214\\
65.41	0.00564999970621061\\
65.42	0.00565096149403635\\
65.43	0.00565192318226342\\
65.44	0.00565288477024652\\
65.45	0.00565384625734108\\
65.46	0.00565480764290322\\
65.47	0.00565576892628978\\
65.48	0.00565673010685832\\
65.49	0.00565769118396715\\
65.5	0.00565865215697536\\
65.51	0.00565961302524278\\
65.52	0.00566057378813004\\
65.53	0.00566153444499854\\
65.54	0.00566249499521054\\
65.55	0.00566345543812908\\
65.56	0.00566441577311808\\
65.57	0.00566537599954227\\
65.58	0.00566633611676728\\
65.59	0.0056672961241596\\
65.6	0.00566825602108664\\
65.61	0.00566921580691668\\
65.62	0.00567017548101895\\
65.63	0.00567113504276363\\
65.64	0.00567209449152183\\
65.65	0.00567305382666564\\
65.66	0.00567401304756811\\
65.67	0.00567497215360331\\
65.68	0.00567593114414631\\
65.69	0.00567689001857321\\
65.7	0.00567784877626116\\
65.71	0.00567880741658834\\
65.72	0.00567976593893403\\
65.73	0.00568072434267859\\
65.74	0.00568168262720348\\
65.75	0.00568264079189126\\
65.76	0.00568359883612566\\
65.77	0.00568455675929153\\
65.78	0.00568551456077489\\
65.79	0.00568647223996295\\
65.8	0.00568742979624411\\
65.81	0.005688387229008\\
65.82	0.00568934453764545\\
65.83	0.00569030172154855\\
65.84	0.00569125878011065\\
65.85	0.0056922157127264\\
65.86	0.00569317251879171\\
65.87	0.00569412919770382\\
65.88	0.00569508574886129\\
65.89	0.00569604217166406\\
65.9	0.00569699846551337\\
65.91	0.0056979546298119\\
65.92	0.00569891066396368\\
65.93	0.00569986656737418\\
65.94	0.0057008223394503\\
65.95	0.00570177797960039\\
65.96	0.00570273348723425\\
65.97	0.00570368886176317\\
65.98	0.00570464410259997\\
65.99	0.00570559920915895\\
66	0.00570655418085598\\
66.01	0.00570750901710846\\
66.02	0.00570846371733538\\
66.03	0.00570941828095732\\
66.04	0.00571037270739647\\
66.05	0.00571132699607666\\
66.06	0.00571228114642335\\
66.07	0.00571323515786367\\
66.08	0.00571418902982646\\
66.09	0.00571514276174224\\
66.1	0.00571609635304327\\
66.11	0.00571704980316353\\
66.12	0.0057180031115388\\
66.13	0.00571895627760661\\
66.14	0.00571990930080631\\
66.15	0.00572086218057907\\
66.16	0.0057218149163679\\
66.17	0.00572276750761767\\
66.18	0.00572371995377514\\
66.19	0.00572467225428896\\
66.2	0.00572562440860971\\
66.21	0.00572657641618991\\
66.22	0.00572752827648404\\
66.23	0.00572847998894857\\
66.24	0.00572943155304198\\
66.25	0.00573038296822476\\
66.26	0.00573133423395947\\
66.27	0.00573228534971071\\
66.28	0.00573323631494518\\
66.29	0.00573418712913171\\
66.3	0.00573513779174124\\
66.31	0.00573608830224687\\
66.32	0.00573703866012389\\
66.33	0.00573798886484975\\
66.34	0.00573893891590416\\
66.35	0.00573988881276905\\
66.36	0.00574083855492862\\
66.37	0.00574178814186935\\
66.38	0.00574273757308005\\
66.39	0.00574368684805185\\
66.4	0.00574463596627822\\
66.41	0.00574558492725501\\
66.42	0.00574653373048051\\
66.43	0.00574748237545538\\
66.44	0.00574843086168276\\
66.45	0.00574937918866826\\
66.46	0.00575032735591997\\
66.47	0.00575127536294851\\
66.48	0.00575222320926704\\
66.49	0.00575317089439126\\
66.5	0.00575411841783952\\
66.51	0.00575506577913273\\
66.52	0.00575601297779446\\
66.53	0.00575696001335094\\
66.54	0.00575790688533109\\
66.55	0.00575885359326656\\
66.56	0.0057598001366917\\
66.57	0.00576074651514365\\
66.58	0.00576169272816236\\
66.59	0.00576263877529053\\
66.6	0.00576358465607377\\
66.61	0.00576453037006051\\
66.62	0.0057654759168021\\
66.63	0.00576642129585277\\
66.64	0.00576736650676974\\
66.65	0.00576831154911317\\
66.66	0.00576925642244622\\
66.67	0.00577020112633509\\
66.68	0.00577114566034901\\
66.69	0.00577209002406029\\
66.7	0.00577303421704436\\
66.71	0.00577397823887976\\
66.72	0.00577492208914822\\
66.73	0.00577586576743462\\
66.74	0.00577680927332707\\
66.75	0.00577775260641693\\
66.76	0.00577869576629881\\
66.77	0.00577963875257064\\
66.78	0.00578058156483367\\
66.79	0.00578152420269248\\
66.8	0.00578246666575509\\
66.81	0.00578340895363288\\
66.82	0.00578435106594069\\
66.83	0.00578529300229683\\
66.84	0.00578623476232312\\
66.85	0.00578717634564489\\
66.86	0.00578811775189104\\
66.87	0.00578905898069407\\
66.88	0.00579000003169008\\
66.89	0.00579094090451883\\
66.9	0.00579188159882375\\
66.91	0.00579282211425199\\
66.92	0.00579376245045445\\
66.93	0.00579470260708576\\
66.94	0.00579564258380441\\
66.95	0.00579658238027269\\
66.96	0.00579752199615675\\
66.97	0.00579846143112666\\
66.98	0.00579940068485639\\
66.99	0.0058003397570239\\
67	0.00580127864731111\\
67.01	0.00580221735540401\\
67.02	0.0058031558809926\\
67.03	0.00580409422377098\\
67.04	0.00580503238343741\\
67.05	0.00580597035969425\\
67.06	0.0058069081522481\\
67.07	0.00580784576080973\\
67.08	0.00580878318509422\\
67.09	0.00580972042482089\\
67.1	0.00581065747971342\\
67.11	0.00581159434949982\\
67.12	0.00581253103391251\\
67.13	0.00581346753268833\\
67.14	0.00581440384556858\\
67.15	0.00581533997229904\\
67.16	0.00581627591263005\\
67.17	0.00581721166631649\\
67.18	0.00581814723311786\\
67.19	0.00581908261279828\\
67.2	0.00582001780512655\\
67.21	0.00582095280987619\\
67.22	0.00582188762682544\\
67.23	0.00582282225575733\\
67.24	0.00582375669645972\\
67.25	0.00582469094872532\\
67.26	0.00582562501235172\\
67.27	0.00582655888714145\\
67.28	0.00582749257290199\\
67.29	0.00582842606944586\\
67.3	0.00582935937659058\\
67.31	0.00583029249415876\\
67.32	0.00583122542197815\\
67.33	0.00583215815988163\\
67.34	0.00583309070770727\\
67.35	0.00583402306529842\\
67.36	0.00583495523250365\\
67.37	0.00583588720917688\\
67.38	0.00583681899517735\\
67.39	0.00583775059036971\\
67.4	0.00583868199462405\\
67.41	0.00583961320781591\\
67.42	0.00584054422982635\\
67.43	0.00584147506054198\\
67.44	0.00584240569985502\\
67.45	0.0058433361476633\\
67.46	0.00584426640387033\\
67.47	0.00584519646838535\\
67.48	0.00584612634112336\\
67.49	0.00584705602200514\\
67.5	0.00584798551095735\\
67.51	0.00584891480791248\\
67.52	0.005849843912809\\
67.53	0.00585077282559132\\
67.54	0.00585170154620988\\
67.55	0.00585263007462116\\
67.56	0.00585355841078776\\
67.57	0.00585448655467841\\
67.58	0.00585541450626804\\
67.59	0.00585634226553781\\
67.6	0.00585726983247514\\
67.61	0.00585819720707379\\
67.62	0.00585912438933389\\
67.63	0.00586005137926197\\
67.64	0.00586097817687103\\
67.65	0.00586190478218056\\
67.66	0.0058628311952166\\
67.67	0.0058637574160118\\
67.68	0.00586468344460545\\
67.69	0.00586560928104349\\
67.7	0.00586653492537866\\
67.71	0.00586746037767044\\
67.72	0.00586838563798515\\
67.73	0.00586931070639598\\
67.74	0.00587023558298306\\
67.75	0.00587116026783348\\
67.76	0.00587208476104138\\
67.77	0.00587300906270793\\
67.78	0.00587393317294145\\
67.79	0.00587485709185743\\
67.8	0.00587578081957856\\
67.81	0.00587670435623482\\
67.82	0.0058776277019635\\
67.83	0.00587855085690925\\
67.84	0.00587947382122415\\
67.85	0.00588039659506774\\
67.86	0.00588131917860711\\
67.87	0.00588224157201689\\
67.88	0.00588316377547934\\
67.89	0.00588408578918443\\
67.9	0.00588500761332982\\
67.91	0.00588592924812097\\
67.92	0.00588685069377116\\
67.93	0.00588777195050159\\
67.94	0.00588869301854136\\
67.95	0.00588961389812758\\
67.96	0.00589053458950543\\
67.97	0.00589145509292816\\
67.98	0.00589237540865719\\
67.99	0.00589329553696216\\
68	0.00589421547812095\\
68.01	0.0058951352324198\\
68.02	0.0058960548001533\\
68.03	0.0058969741816245\\
68.04	0.0058978933771449\\
68.05	0.00589881238703457\\
68.06	0.0058997312116222\\
68.07	0.0059006498512451\\
68.08	0.00590156830624933\\
68.09	0.00590248657698971\\
68.1	0.0059034046638299\\
68.11	0.00590432256714245\\
68.12	0.00590524039254305\\
68.13	0.0059061582033246\\
68.14	0.00590707599964485\\
68.15	0.00590799378166609\\
68.16	0.00590891154955514\\
68.17	0.00590982930348342\\
68.18	0.00591074704362693\\
68.19	0.00591166477016633\\
68.2	0.00591258248328694\\
68.21	0.00591350018317875\\
68.22	0.00591441787003651\\
68.23	0.0059153355440597\\
68.24	0.00591625320545258\\
68.25	0.00591717085442425\\
68.26	0.00591808849118861\\
68.27	0.00591900611596447\\
68.28	0.00591992372897552\\
68.29	0.00592084133045039\\
68.3	0.00592175892062267\\
68.31	0.00592267649973095\\
68.32	0.00592359406801885\\
68.33	0.00592451162573503\\
68.34	0.00592542917313325\\
68.35	0.00592634671047237\\
68.36	0.00592726423801643\\
68.37	0.00592818175603461\\
68.38	0.00592909926480135\\
68.39	0.00593001676459631\\
68.4	0.00593093425570441\\
68.41	0.00593185173841591\\
68.42	0.00593276921302639\\
68.43	0.00593368667983682\\
68.44	0.00593460413915357\\
68.45	0.00593552159128843\\
68.46	0.0059364390365587\\
68.47	0.00593735647528716\\
68.48	0.00593827390780213\\
68.49	0.00593919133443752\\
68.5	0.0059401087555328\\
68.51	0.00594102617143314\\
68.52	0.00594194358248935\\
68.53	0.00594286098905794\\
68.54	0.00594377839150119\\
68.55	0.00594469579018714\\
68.56	0.00594561318548963\\
68.57	0.00594653057778836\\
68.58	0.00594744796746893\\
68.59	0.0059483653549228\\
68.6	0.00594928274054744\\
68.61	0.00595020012474627\\
68.62	0.00595111750792875\\
68.63	0.0059520348905104\\
68.64	0.0059529522729128\\
68.65	0.00595386965556371\\
68.66	0.00595478703889702\\
68.67	0.00595570442335284\\
68.68	0.0059566218093775\\
68.69	0.00595753919742365\\
68.7	0.00595845658795021\\
68.71	0.00595937398142246\\
68.72	0.00596029137831208\\
68.73	0.00596120877909716\\
68.74	0.00596212618426226\\
68.75	0.00596304359429846\\
68.76	0.00596396100970335\\
68.77	0.0059648784309811\\
68.78	0.00596579585864253\\
68.79	0.00596671329320508\\
68.8	0.00596763073519288\\
68.81	0.00596854818513683\\
68.82	0.00596946564357455\\
68.83	0.00597038311105051\\
68.84	0.00597130058811603\\
68.85	0.00597221807532929\\
68.86	0.00597313557325544\\
68.87	0.00597405308246657\\
68.88	0.00597497060354179\\
68.89	0.00597588813706726\\
68.9	0.00597680568363624\\
68.91	0.0059777232438491\\
68.92	0.00597864081831343\\
68.93	0.00597955840764398\\
68.94	0.00598047601246329\\
68.95	0.00598139363340175\\
68.96	0.0059823112710977\\
68.97	0.00598322892619747\\
68.98	0.0059841465993554\\
68.99	0.00598506429123392\\
69	0.00598598200250356\\
69.01	0.00598689973384306\\
69.02	0.00598781748593936\\
69.03	0.00598873525948768\\
69.04	0.00598965305519158\\
69.05	0.00599057087376299\\
69.06	0.00599148871592226\\
69.07	0.0059924065823982\\
69.08	0.0059933244739282\\
69.09	0.00599424239125818\\
69.1	0.00599516033514272\\
69.11	0.00599607830634508\\
69.12	0.00599699630563724\\
69.13	0.00599791433379999\\
69.14	0.00599883239162295\\
69.15	0.00599975047990464\\
69.16	0.00600066859945254\\
69.17	0.00600158675108311\\
69.18	0.00600250493562187\\
69.19	0.00600342315390346\\
69.2	0.00600434140677168\\
69.21	0.00600525969507954\\
69.22	0.00600617801968935\\
69.23	0.00600709638147272\\
69.24	0.00600801478131068\\
69.25	0.00600893322009366\\
69.26	0.00600985169872161\\
69.27	0.00601077021810404\\
69.28	0.00601168877916005\\
69.29	0.00601260738281842\\
69.3	0.00601352603001765\\
69.31	0.00601444472170604\\
69.32	0.00601536345884171\\
69.33	0.00601628224239267\\
69.34	0.00601720107333692\\
69.35	0.00601811995266244\\
69.36	0.00601903888136732\\
69.37	0.00601995786045974\\
69.38	0.00602087689095811\\
69.39	0.00602179597389108\\
69.4	0.00602271511029762\\
69.41	0.00602363430122707\\
69.42	0.0060245535477392\\
69.43	0.00602547285090429\\
69.44	0.00602639221180316\\
69.45	0.00602731163152725\\
69.46	0.00602823111117871\\
69.47	0.00602915065187039\\
69.48	0.006030070254726\\
69.49	0.00603098992088006\\
69.5	0.00603190965147807\\
69.51	0.00603282944767652\\
69.52	0.00603374931064293\\
69.53	0.00603466924155599\\
69.54	0.00603558924160554\\
69.55	0.0060365093119927\\
69.56	0.0060374294539299\\
69.57	0.00603834966864096\\
69.58	0.00603926995736114\\
69.59	0.00604019032133724\\
69.6	0.00604111076182762\\
69.61	0.0060420312801023\\
69.62	0.00604295187744303\\
69.63	0.00604387255514335\\
69.64	0.00604479331450862\\
69.65	0.00604571415685616\\
69.66	0.00604663508351527\\
69.67	0.00604755609582729\\
69.68	0.00604847719514572\\
69.69	0.00604939838283625\\
69.7	0.00605031966027682\\
69.71	0.00605124102885774\\
69.72	0.00605216248998171\\
69.73	0.00605308404506392\\
69.74	0.00605400569553211\\
69.75	0.00605492744282664\\
69.76	0.00605584928840058\\
69.77	0.00605677123371975\\
69.78	0.00605769328026284\\
69.79	0.00605861542952142\\
69.8	0.00605953768300009\\
69.81	0.00606046004221648\\
69.82	0.00606138250870136\\
69.83	0.00606230508399875\\
69.84	0.0060632277696659\\
69.85	0.00606415056727348\\
69.86	0.00606507347840554\\
69.87	0.0060659965046597\\
69.88	0.00606691964764714\\
69.89	0.00606784290899272\\
69.9	0.00606876629033502\\
69.91	0.0060696897933265\\
69.92	0.00607061341963346\\
69.93	0.0060715371709362\\
69.94	0.00607246104825247\\
69.95	0.00607338505136619\\
69.96	0.00607430918006143\\
69.97	0.00607523343412236\\
69.98	0.00607615781333327\\
69.99	0.00607708231747858\\
70	0.00607800694634279\\
70.01	0.00607893169971058\\
70.02	0.00607985657736671\\
70.03	0.0060807815790961\\
70.04	0.0060817067046838\\
70.05	0.00608263195391497\\
70.06	0.00608355732657494\\
70.07	0.00608448282244917\\
70.08	0.00608540844132325\\
70.09	0.00608633418298293\\
70.1	0.00608726004721413\\
70.11	0.00608818603380288\\
70.12	0.0060891121425354\\
70.13	0.00609003837319805\\
70.14	0.00609096472557735\\
70.15	0.00609189119946001\\
70.16	0.00609281779463289\\
70.17	0.00609374451088301\\
70.18	0.00609467134799758\\
70.19	0.00609559830576398\\
70.2	0.00609652538396976\\
70.21	0.00609745258240269\\
70.22	0.00609837990085069\\
70.23	0.00609930733910187\\
70.24	0.00610023489694455\\
70.25	0.00610116257416723\\
70.26	0.00610209037055861\\
70.27	0.0061030182859076\\
70.28	0.00610394632000332\\
70.29	0.00610487447263509\\
70.3	0.00610580274359242\\
70.31	0.00610673113266508\\
70.32	0.00610765963964302\\
70.33	0.00610858826431641\\
70.34	0.00610951700647568\\
70.35	0.00611044586591145\\
70.36	0.00611137484241459\\
70.37	0.00611230393577619\\
70.38	0.00611323314578759\\
70.39	0.00611416247224037\\
70.4	0.00611509191492635\\
70.41	0.00611602147363759\\
70.42	0.00611695114816643\\
70.43	0.00611788093830541\\
70.44	0.00611881084384739\\
70.45	0.00611974086458545\\
70.46	0.00612067100031295\\
70.47	0.00612160125082349\\
70.48	0.006122531615911\\
70.49	0.00612346209536963\\
70.5	0.00612439268899383\\
70.51	0.00612532339657834\\
70.52	0.00612625421791817\\
70.53	0.00612718515280864\\
70.54	0.00612811620104533\\
70.55	0.00612904736242414\\
70.56	0.00612997863674128\\
70.57	0.00613091002379323\\
70.58	0.00613184152337681\\
70.59	0.00613277313528913\\
70.6	0.00613370485932763\\
70.61	0.00613463669529005\\
70.62	0.00613556864297447\\
70.63	0.00613650070217929\\
70.64	0.00613743287270323\\
70.65	0.00613836515434537\\
70.66	0.0061392975469051\\
70.67	0.00614023005018216\\
70.68	0.00614116266397664\\
70.69	0.00614209538808898\\
70.7	0.00614302822231996\\
70.71	0.00614396116647072\\
70.72	0.00614489422034278\\
70.73	0.00614582738373801\\
70.74	0.00614676065645863\\
70.75	0.00614769403830727\\
70.76	0.00614862752908692\\
70.77	0.00614956112860093\\
70.78	0.00615049483665306\\
70.79	0.00615142865304746\\
70.8	0.00615236257758867\\
70.81	0.0061532966100816\\
70.82	0.00615423075033159\\
70.83	0.00615516499814438\\
70.84	0.00615609935332612\\
70.85	0.00615703381568337\\
70.86	0.00615796838502311\\
70.87	0.00615890306115273\\
70.88	0.00615983784388006\\
70.89	0.00616077273301335\\
70.9	0.0061617077283613\\
70.91	0.00616264282973303\\
70.92	0.00616357803693812\\
70.93	0.00616451334978657\\
70.94	0.00616544876808888\\
70.95	0.00616638429165596\\
70.96	0.0061673199202992\\
70.97	0.00616825565383046\\
70.98	0.00616919149206205\\
70.99	0.00617012743480677\\
71	0.0061710634818779\\
71.01	0.0061719996330892\\
71.02	0.0061729358882549\\
71.03	0.00617387224718974\\
71.04	0.00617480870970896\\
71.05	0.00617574527562829\\
71.06	0.00617668194476396\\
71.07	0.00617761871693274\\
71.08	0.00617855559195187\\
71.09	0.00617949256963915\\
71.1	0.00618042964981288\\
71.11	0.00618136683229189\\
71.12	0.00618230411689555\\
71.13	0.00618324150344377\\
71.14	0.006184178991757\\
71.15	0.00618511658165623\\
71.16	0.00618605427296302\\
71.17	0.00618699206549947\\
71.18	0.00618792995908824\\
71.19	0.00618886795355258\\
71.2	0.00618980604871629\\
71.21	0.00619074424440374\\
71.22	0.00619168254043991\\
71.23	0.00619262093665036\\
71.24	0.00619355943286121\\
71.25	0.00619449802889921\\
71.26	0.0061954367245917\\
71.27	0.00619637551976662\\
71.28	0.00619731441425254\\
71.29	0.00619825340787863\\
71.3	0.00619919250047468\\
71.31	0.00620013169187113\\
71.32	0.00620107098189902\\
71.33	0.00620201037039003\\
71.34	0.00620294985717652\\
71.35	0.00620388944209147\\
71.36	0.0062048291249685\\
71.37	0.00620576890564192\\
71.38	0.00620670878394667\\
71.39	0.00620764875971838\\
71.4	0.00620858883279336\\
71.41	0.00620952900300856\\
71.42	0.00621046927020168\\
71.43	0.00621140963421105\\
71.44	0.00621235009487572\\
71.45	0.00621329065203544\\
71.46	0.00621423130553068\\
71.47	0.00621517205520258\\
71.48	0.00621611290089306\\
71.49	0.0062170538424447\\
71.5	0.00621799487970086\\
71.51	0.00621893601250559\\
71.52	0.00621987724070371\\
71.53	0.00622081856414077\\
71.54	0.0062217599826631\\
71.55	0.00622270149611773\\
71.56	0.00622364310435253\\
71.57	0.00622458480721606\\
71.58	0.00622552660455771\\
71.59	0.00622646849622763\\
71.6	0.00622741048207674\\
71.61	0.00622835256195679\\
71.62	0.00622929473572028\\
71.63	0.00623023700322056\\
71.64	0.00623117936431176\\
71.65	0.00623212181884884\\
71.66	0.00623306436668756\\
71.67	0.00623400700768453\\
71.68	0.00623494974169719\\
71.69	0.00623589256858381\\
71.7	0.00623683548820352\\
71.71	0.00623777850041627\\
71.72	0.00623872160508291\\
71.73	0.00623966480206513\\
71.74	0.00624060809122549\\
71.75	0.00624155147242743\\
71.76	0.00624249494553528\\
71.77	0.00624343851041426\\
71.78	0.00624438216693046\\
71.79	0.0062453259149509\\
71.8	0.00624626975434349\\
71.81	0.00624721368497706\\
71.82	0.00624815770672139\\
71.83	0.00624910181944712\\
71.84	0.00625004602302589\\
71.85	0.00625099031733024\\
71.86	0.00625193470223369\\
71.87	0.00625287917761067\\
71.88	0.00625382374333661\\
71.89	0.00625476839928787\\
71.9	0.00625571314534183\\
71.91	0.00625665798137681\\
71.92	0.00625760290727214\\
71.93	0.00625854792290812\\
71.94	0.00625949302816607\\
71.95	0.00626043822292831\\
71.96	0.00626138350707817\\
71.97	0.00626232888050002\\
71.98	0.00626327434307922\\
71.99	0.00626421989470221\\
72	0.00626516553525643\\
72.01	0.0062661112646304\\
72.02	0.00626705708271367\\
72.03	0.00626800298939687\\
72.04	0.00626894898457169\\
72.05	0.00626989506813091\\
72.06	0.00627084123996836\\
72.07	0.00627178749997902\\
72.08	0.0062727338480589\\
72.09	0.00627368028410517\\
72.1	0.00627462680801608\\
72.11	0.006275573419691\\
72.12	0.00627652011903045\\
72.13	0.00627746690593607\\
72.14	0.00627841378031064\\
72.15	0.00627936074205809\\
72.16	0.00628030779108352\\
72.17	0.00628125492729317\\
72.18	0.00628220215059447\\
72.19	0.00628314946089601\\
72.2	0.0062840968581076\\
72.21	0.00628504434214022\\
72.22	0.00628599191290605\\
72.23	0.00628693957031849\\
72.24	0.00628788731429215\\
72.25	0.00628883514474288\\
72.26	0.00628978306158773\\
72.27	0.00629073106474502\\
72.28	0.00629167915413432\\
72.29	0.00629262732967643\\
72.3	0.00629357559129345\\
72.31	0.00629452393890871\\
72.32	0.00629547237244686\\
72.33	0.00629642089183381\\
72.34	0.00629736949699679\\
72.35	0.0062983181878643\\
72.36	0.0062992669643662\\
72.37	0.00630021582643361\\
72.38	0.00630116477399904\\
72.39	0.0063021138069963\\
72.4	0.00630306292536055\\
72.41	0.00630401212902831\\
72.42	0.00630496141793745\\
72.43	0.00630591079202723\\
72.44	0.00630686025123827\\
72.45	0.00630780979551259\\
72.46	0.0063087594247936\\
72.47	0.0063097091390261\\
72.48	0.00631065893815633\\
72.49	0.00631160882213191\\
72.5	0.00631255879090194\\
72.51	0.00631350884441691\\
72.52	0.00631445898262881\\
72.53	0.00631540920549103\\
72.54	0.00631635951295845\\
72.55	0.00631730990498743\\
72.56	0.0063182603815358\\
72.57	0.00631921094256287\\
72.58	0.00632016158802949\\
72.59	0.00632111231789798\\
72.6	0.00632206313213217\\
72.61	0.00632301403069745\\
72.62	0.00632396501356074\\
72.63	0.00632491608069048\\
72.64	0.00632586723205667\\
72.65	0.0063268184676309\\
72.66	0.00632776978738631\\
72.67	0.00632872119129761\\
72.68	0.00632967267934112\\
72.69	0.00633062425149476\\
72.7	0.00633157590773804\\
72.71	0.00633252764805211\\
72.72	0.00633347947241973\\
72.73	0.00633443138082532\\
72.74	0.00633538337325491\\
72.75	0.00633633544969624\\
72.76	0.00633728761013865\\
72.77	0.00633823985457322\\
72.78	0.00633919218299267\\
72.79	0.00634014459539144\\
72.8	0.00634109709176565\\
72.81	0.00634204967211316\\
72.82	0.00634300233643354\\
72.83	0.0063439550847281\\
72.84	0.0063449079169999\\
72.85	0.00634586083325372\\
72.86	0.00634681383349615\\
72.87	0.00634776691773553\\
72.88	0.00634872008598199\\
72.89	0.00634967333824744\\
72.9	0.00635062667454561\\
72.91	0.00635158009489205\\
72.92	0.00635253359930411\\
72.93	0.006353487187801\\
72.94	0.00635444086040377\\
72.95	0.00635539461713532\\
72.96	0.0063563484580204\\
72.97	0.00635730238308566\\
72.98	0.00635825639235964\\
72.99	0.00635921048587276\\
73	0.00636016466365737\\
73.01	0.00636111892574771\\
73.02	0.00636207327217997\\
73.03	0.00636302770299228\\
73.04	0.00636398221822471\\
73.05	0.0063649368179193\\
73.06	0.00636589150212005\\
73.07	0.00636684627087297\\
73.08	0.00636780112422603\\
73.09	0.00636875606222925\\
73.1	0.00636971108493461\\
73.11	0.00637066619239617\\
73.12	0.00637162138466999\\
73.13	0.00637257666181421\\
73.14	0.006373532023889\\
73.15	0.00637448747095663\\
73.16	0.00637544300308145\\
73.17	0.00637639862032988\\
73.18	0.00637735432277046\\
73.19	0.00637831011047388\\
73.2	0.00637926598351291\\
73.21	0.00638022194196249\\
73.22	0.00638117798589968\\
73.23	0.00638213411540376\\
73.24	0.00638309033055614\\
73.25	0.00638404663144043\\
73.26	0.00638500301814243\\
73.27	0.00638595949075017\\
73.28	0.0063869160493539\\
73.29	0.00638787269404609\\
73.3	0.00638882942492147\\
73.31	0.00638978624207703\\
73.32	0.00639074314561201\\
73.33	0.00639170013562795\\
73.34	0.00639265721222868\\
73.35	0.00639361437552035\\
73.36	0.00639457162561141\\
73.37	0.00639552896261266\\
73.38	0.00639648638663723\\
73.39	0.00639744389780061\\
73.4	0.00639840149622066\\
73.41	0.00639935918201763\\
73.42	0.00640031695531415\\
73.43	0.00640127481623527\\
73.44	0.00640223276490844\\
73.45	0.00640319080146358\\
73.46	0.00640414892603301\\
73.47	0.00640510713875153\\
73.48	0.00640606543975641\\
73.49	0.0064070238291874\\
73.5	0.00640798230718676\\
73.51	0.00640894087389923\\
73.52	0.00640989952947212\\
73.53	0.00641085827405521\\
73.54	0.00641181710780089\\
73.55	0.00641277603086408\\
73.56	0.00641373504340229\\
73.57	0.00641469414557561\\
73.58	0.00641565333754673\\
73.59	0.00641661261948097\\
73.6	0.00641757199154626\\
73.61	0.00641853145391318\\
73.62	0.00641949100675498\\
73.63	0.00642045065024757\\
73.64	0.00642141038456952\\
73.65	0.00642237020990215\\
73.66	0.00642333012642946\\
73.67	0.00642429013433816\\
73.68	0.00642525023381773\\
73.69	0.00642621042506038\\
73.7	0.00642717070826112\\
73.71	0.00642813108361769\\
73.72	0.00642909155133068\\
73.73	0.00643005211160345\\
73.74	0.00643101276464222\\
73.75	0.006431973510656\\
73.76	0.00643293434985671\\
73.77	0.00643389528245909\\
73.78	0.00643485630868079\\
73.79	0.00643581742874234\\
73.8	0.0064367786428672\\
73.81	0.00643773995128175\\
73.82	0.00643870135421529\\
73.83	0.0064396628519001\\
73.84	0.00644062444457143\\
73.85	0.00644158613246751\\
73.86	0.00644254791582955\\
73.87	0.00644350979490182\\
73.88	0.00644447176993157\\
73.89	0.00644543384116913\\
73.9	0.00644639600886789\\
73.91	0.00644735827328432\\
73.92	0.00644832063467794\\
73.93	0.00644928309331143\\
73.94	0.00645024564945056\\
73.95	0.00645120830336426\\
73.96	0.00645217105532459\\
73.97	0.00645313390560681\\
73.98	0.00645409685448933\\
73.99	0.0064550599022538\\
74	0.00645602304918505\\
74.01	0.00645698629557117\\
74.02	0.00645794964170349\\
74.03	0.00645891308787659\\
74.04	0.00645987663438836\\
74.05	0.00646084028153997\\
74.06	0.00646180402963589\\
74.07	0.00646276787898396\\
74.08	0.00646373182989534\\
74.09	0.00646469588268455\\
74.1	0.00646566003766949\\
74.11	0.00646662429517147\\
74.12	0.00646758865551521\\
74.13	0.00646855311902883\\
74.14	0.00646951768604394\\
74.15	0.0064704823568956\\
74.16	0.00647144713192233\\
74.17	0.00647241201146615\\
74.18	0.00647337699587262\\
74.19	0.00647434208549081\\
74.2	0.00647530728067334\\
74.21	0.0064762725817764\\
74.22	0.00647723798915978\\
74.23	0.00647820350318683\\
74.24	0.00647916912422455\\
74.25	0.00648013485264356\\
74.26	0.00648110068881815\\
74.27	0.00648206663312627\\
74.28	0.00648303268594955\\
74.29	0.00648399884767334\\
74.3	0.0064849651186867\\
74.31	0.00648593149938245\\
74.32	0.00648689799015716\\
74.33	0.00648786459141119\\
74.34	0.00648883130354867\\
74.35	0.00648979812697757\\
74.36	0.00649076506210969\\
74.37	0.00649173210936067\\
74.38	0.00649269926915002\\
74.39	0.00649366654190115\\
74.4	0.00649463392804139\\
74.41	0.00649560142800195\\
74.42	0.00649656904221803\\
74.43	0.00649753677112878\\
74.44	0.00649850461517731\\
74.45	0.00649947257481077\\
74.46	0.00650044065048032\\
74.47	0.00650140884264112\\
74.48	0.00650237715175245\\
74.49	0.00650334557827763\\
74.5	0.0065043141226841\\
74.51	0.00650528278544337\\
74.52	0.00650625156703116\\
74.53	0.00650722046792728\\
74.54	0.00650818948861576\\
74.55	0.00650915862958481\\
74.56	0.00651012789132685\\
74.57	0.00651109727433854\\
74.58	0.00651206677912081\\
74.59	0.00651303640617884\\
74.6	0.00651400615602214\\
74.61	0.00651497602916449\\
74.62	0.00651594602612404\\
74.63	0.00651691614742331\\
74.64	0.00651788639358916\\
74.65	0.00651885676515288\\
74.66	0.00651982726265015\\
74.67	0.00652079788662112\\
74.68	0.00652176863761037\\
74.69	0.006522739516167\\
74.7	0.00652371052284456\\
74.71	0.00652468165820119\\
74.72	0.00652565292279952\\
74.73	0.00652662431720676\\
74.74	0.00652759584199473\\
74.75	0.00652856749773983\\
74.76	0.0065295392850231\\
74.77	0.00653051120443024\\
74.78	0.00653148325655161\\
74.79	0.00653245544198229\\
74.8	0.00653342776132206\\
74.81	0.00653440021517543\\
74.82	0.00653537280415169\\
74.83	0.0065363455288649\\
74.84	0.00653731838993393\\
74.85	0.0065382913879825\\
74.86	0.00653926452363916\\
74.87	0.00654023779753731\\
74.88	0.0065412112103153\\
74.89	0.00654218476261634\\
74.9	0.00654315845508864\\
74.91	0.00654413228838533\\
74.92	0.00654510626316453\\
74.93	0.00654608038008941\\
74.94	0.00654705463982813\\
74.95	0.00654802904305393\\
74.96	0.00654900359044511\\
74.97	0.0065499782826851\\
74.98	0.00655095312046243\\
74.99	0.00655192810447081\\
75	0.0065529032354091\\
75.01	0.00655387851398136\\
75.02	0.00655485394089689\\
75.03	0.00655582951687023\\
75.04	0.00655680524262115\\
75.05	0.00655778111887479\\
75.06	0.00655875714636155\\
75.07	0.0065597333258172\\
75.08	0.00656070965798286\\
75.09	0.00656168614360506\\
75.1	0.00656266278343573\\
75.11	0.00656363957823226\\
75.12	0.00656461652875749\\
75.13	0.00656559363577976\\
75.14	0.00656657090007293\\
75.15	0.0065675483224164\\
75.16	0.00656852590359511\\
75.17	0.00656950364439961\\
75.18	0.00657048154562608\\
75.19	0.00657145960807632\\
75.2	0.00657243783255779\\
75.21	0.00657341621988368\\
75.22	0.00657439477087285\\
75.23	0.00657537348634992\\
75.24	0.00657635236714529\\
75.25	0.00657733141409514\\
75.26	0.00657831062804146\\
75.27	0.00657929000983213\\
75.28	0.00658026956032085\\
75.29	0.00658124928036725\\
75.3	0.00658222917083686\\
75.31	0.00658320923260118\\
75.32	0.00658418946653769\\
75.33	0.00658516987352985\\
75.34	0.00658615045446716\\
75.35	0.00658713121024521\\
75.36	0.00658811214176562\\
75.37	0.00658909324993615\\
75.38	0.0065900745356707\\
75.39	0.00659105599988933\\
75.4	0.00659203764351829\\
75.41	0.00659301946749005\\
75.42	0.00659400147274332\\
75.43	0.0065949836602231\\
75.44	0.00659596603088069\\
75.45	0.0065969485856737\\
75.46	0.00659793132556612\\
75.47	0.00659891425152833\\
75.48	0.00659989736453711\\
75.49	0.00660088066557567\\
75.5	0.00660186415563372\\
75.51	0.00660284783570745\\
75.52	0.00660383170640257\\
75.53	0.00660481576804852\\
75.54	0.00660580002097601\\
75.55	0.00660678446551707\\
75.56	0.00660776910200499\\
75.57	0.00660875393077437\\
75.58	0.00660973895216111\\
75.59	0.00661072416650236\\
75.6	0.00661170957413657\\
75.61	0.00661269517540349\\
75.62	0.0066136809706441\\
75.63	0.00661466696020068\\
75.64	0.00661565314441678\\
75.65	0.00661663952363722\\
75.66	0.00661762609820808\\
75.67	0.0066186128684767\\
75.68	0.00661959983479166\\
75.69	0.00662058699750284\\
75.7	0.00662157435696133\\
75.71	0.0066225619135195\\
75.72	0.00662354966753092\\
75.73	0.00662453761935045\\
75.74	0.00662552576933417\\
75.75	0.00662651411783937\\
75.76	0.00662750266522461\\
75.77	0.00662849141184965\\
75.78	0.00662948035807549\\
75.79	0.00663046950426434\\
75.8	0.00663145885077961\\
75.81	0.00663244839798596\\
75.82	0.00663343814624921\\
75.83	0.00663442809593642\\
75.84	0.00663541824741583\\
75.85	0.00663640860105688\\
75.86	0.0066373991572302\\
75.87	0.00663838991630759\\
75.88	0.00663938087866205\\
75.89	0.00664037204466775\\
75.9	0.00664136341470004\\
75.91	0.00664235498913541\\
75.92	0.00664334676835152\\
75.93	0.00664433875272721\\
75.94	0.00664533094264246\\
75.95	0.00664632333847837\\
75.96	0.00664731594061721\\
75.97	0.00664830874944237\\
75.98	0.0066493017653384\\
75.99	0.00665029498869092\\
76	0.00665128841988673\\
76.01	0.00665228205931369\\
76.02	0.0066532759073608\\
76.03	0.00665426996441817\\
76.04	0.00665526423087696\\
76.05	0.00665625870712947\\
76.06	0.00665725339356905\\
76.07	0.00665824829059015\\
76.08	0.00665924339858829\\
76.09	0.00666023871796003\\
76.1	0.00666123424910301\\
76.11	0.00666222999241593\\
76.12	0.0066632259482985\\
76.13	0.00666422211715152\\
76.14	0.00666521849937676\\
76.15	0.00666621509537708\\
76.16	0.0066672119055563\\
76.17	0.0066682089303193\\
76.18	0.00666920617007191\\
76.19	0.00667020362522101\\
76.2	0.00667120129617445\\
76.21	0.00667219918334104\\
76.22	0.00667319728713059\\
76.23	0.00667419560795385\\
76.24	0.00667519414622257\\
76.25	0.00667619290234941\\
76.26	0.00667719187674798\\
76.27	0.00667819106983283\\
76.28	0.00667919048201945\\
76.29	0.00668019011372421\\
76.3	0.00668118996536445\\
76.31	0.00668219003735833\\
76.32	0.00668319033012497\\
76.33	0.00668419084408434\\
76.34	0.0066851915796573\\
76.35	0.00668619253726555\\
76.36	0.00668719371733168\\
76.37	0.00668819512027909\\
76.38	0.00668919674653205\\
76.39	0.00669019859651564\\
76.4	0.00669120067065576\\
76.41	0.00669220296937913\\
76.42	0.00669320549311325\\
76.43	0.00669420824228642\\
76.44	0.00669521121732773\\
76.45	0.00669621441866702\\
76.46	0.0066972178467349\\
76.47	0.00669822150196272\\
76.48	0.00669922538478256\\
76.49	0.00670022949562726\\
76.5	0.00670123383493033\\
76.51	0.00670223840312603\\
76.52	0.00670324320064929\\
76.53	0.00670424822793572\\
76.54	0.00670525348542161\\
76.55	0.00670625897354393\\
76.56	0.00670726469274024\\
76.57	0.00670827064344882\\
76.58	0.00670927682610849\\
76.59	0.00671028324115876\\
76.6	0.00671128988903969\\
76.61	0.00671229677019194\\
76.62	0.00671330388505677\\
76.63	0.00671431123407597\\
76.64	0.00671531881769192\\
76.65	0.0067163266363475\\
76.66	0.00671733469048615\\
76.67	0.00671834298055181\\
76.68	0.0067193515069889\\
76.69	0.00672036027024236\\
76.7	0.00672136927075759\\
76.71	0.00672237850898044\\
76.72	0.00672338798535723\\
76.73	0.00672439770033469\\
76.74	0.00672540765435998\\
76.75	0.00672641784788066\\
76.76	0.00672742828134468\\
76.77	0.00672843895520036\\
76.78	0.00672944986989639\\
76.79	0.00673046102565315\\
76.8	0.00673147242264405\\
76.81	0.00673248406104336\\
76.82	0.00673349594102618\\
76.83	0.00673450806276843\\
76.84	0.00673552042644693\\
76.85	0.00673653303223929\\
76.86	0.00673754588032398\\
76.87	0.00673855897088033\\
76.88	0.0067395723040885\\
76.89	0.00674058588012951\\
76.9	0.00674159969918521\\
76.91	0.00674261376143831\\
76.92	0.00674362806707236\\
76.93	0.00674464261627178\\
76.94	0.00674565740922181\\
76.95	0.00674667244610856\\
76.96	0.00674768772711899\\
76.97	0.00674870325244089\\
76.98	0.00674971902226294\\
76.99	0.00675073503677462\\
77	0.00675175129616632\\
77.01	0.00675276780062926\\
77.02	0.00675378455035549\\
77.03	0.00675480154553794\\
77.04	0.00675581878637039\\
77.05	0.00675683627304747\\
77.06	0.00675785400576467\\
77.07	0.00675887198471832\\
77.08	0.00675989021010564\\
77.09	0.00676090868212467\\
77.1	0.00676192740097431\\
77.11	0.00676294636685435\\
77.12	0.0067639655799654\\
77.13	0.00676498504050893\\
77.14	0.00676600474868729\\
77.15	0.00676702470470366\\
77.16	0.0067680449087621\\
77.17	0.00676906536106751\\
77.18	0.00677008606182566\\
77.19	0.00677110701124317\\
77.2	0.00677212820952751\\
77.21	0.00677314965688702\\
77.22	0.00677417135353089\\
77.23	0.00677519329966917\\
77.24	0.00677621549551276\\
77.25	0.00677723794127344\\
77.26	0.0067782606371638\\
77.27	0.00677928358339734\\
77.28	0.00678030678018838\\
77.29	0.00678133022775211\\
77.3	0.00678235392630458\\
77.31	0.00678337787606268\\
77.32	0.00678440207724416\\
77.33	0.00678542653006763\\
77.34	0.00678645123475256\\
77.35	0.00678747619151926\\
77.36	0.0067885014005889\\
77.37	0.00678952686218351\\
77.38	0.00679055257652596\\
77.39	0.00679157854383997\\
77.4	0.00679260476435013\\
77.41	0.00679363123828186\\
77.42	0.00679465796586146\\
77.43	0.00679568494731604\\
77.44	0.00679671218287358\\
77.45	0.00679773967276292\\
77.46	0.00679876741721372\\
77.47	0.00679979541645651\\
77.48	0.00680082367072265\\
77.49	0.00680185218024436\\
77.5	0.00680288094525469\\
77.51	0.00680390996598756\\
77.52	0.00680493924267768\\
77.53	0.00680596877556065\\
77.54	0.0068069985648729\\
77.55	0.00680802861085169\\
77.56	0.0068090589137351\\
77.57	0.00681008947376211\\
77.58	0.00681112029117246\\
77.59	0.00681215136620679\\
77.6	0.00681318269910652\\
77.61	0.00681421429011393\\
77.62	0.00681524613947214\\
77.63	0.00681627824742508\\
77.64	0.00681731061421751\\
77.65	0.00681834324009504\\
77.66	0.00681937612530406\\
77.67	0.00682040927009184\\
77.68	0.00682144267470644\\
77.69	0.00682247633939674\\
77.7	0.00682351026441245\\
77.71	0.00682454445000409\\
77.72	0.006825578896423\\
77.73	0.00682661360392135\\
77.74	0.00682764857275209\\
77.75	0.006828683803169\\
77.76	0.00682971929542667\\
77.77	0.0068307550497805\\
77.78	0.00683179106648669\\
77.79	0.00683282734580224\\
77.8	0.00683386388798496\\
77.81	0.00683490069329344\\
77.82	0.00683593776198709\\
77.83	0.00683697509432611\\
77.84	0.00683801269057148\\
77.85	0.006839050550985\\
77.86	0.00684008867582921\\
77.87	0.00684112706536749\\
77.88	0.00684216571986396\\
77.89	0.00684320463958355\\
77.9	0.00684424382479195\\
77.91	0.00684528327575564\\
77.92	0.00684632299274187\\
77.93	0.00684736297601865\\
77.94	0.00684840322585479\\
77.95	0.00684944374251982\\
77.96	0.00685048452628407\\
77.97	0.0068515255774186\\
77.98	0.00685256689619526\\
77.99	0.00685360848288665\\
78	0.00685465033776609\\
78.01	0.00685569246110767\\
78.02	0.00685673485318623\\
78.03	0.00685777751427735\\
78.04	0.00685882044465735\\
78.05	0.00685986364460327\\
78.06	0.0068609071143929\\
78.07	0.00686195085430476\\
78.08	0.00686299486461808\\
78.09	0.00686403914561283\\
78.1	0.00686508369756969\\
78.11	0.00686612852077005\\
78.12	0.00686717361549604\\
78.13	0.00686821898203047\\
78.14	0.00686926462065685\\
78.15	0.00687031053165941\\
78.16	0.00687135671532308\\
78.17	0.00687240317193346\\
78.18	0.00687344990177687\\
78.19	0.00687449690514028\\
78.2	0.00687554418231136\\
78.21	0.00687659173357848\\
78.22	0.00687763955923064\\
78.23	0.00687868765955753\\
78.24	0.00687973603484951\\
78.25	0.00688078468539758\\
78.26	0.00688183361149342\\
78.27	0.00688288281342936\\
78.28	0.00688393229149834\\
78.29	0.00688498204599399\\
78.3	0.00688603207721056\\
78.31	0.00688708238544291\\
78.32	0.00688813297098657\\
78.33	0.00688918383413765\\
78.34	0.00689023497519291\\
78.35	0.00689128639444972\\
78.36	0.00689233809220604\\
78.37	0.00689339006876044\\
78.38	0.00689444232441211\\
78.39	0.0068954948594608\\
78.4	0.00689654767420687\\
78.41	0.00689760076895126\\
78.42	0.00689865414399547\\
78.43	0.00689970779964159\\
78.44	0.00690076173619227\\
78.45	0.00690181595395072\\
78.46	0.00690287045322071\\
78.47	0.00690392523430653\\
78.48	0.00690498029751306\\
78.49	0.00690603564314568\\
78.5	0.00690709127151032\\
78.51	0.00690814718291342\\
78.52	0.00690920337766195\\
78.53	0.0069102598560634\\
78.54	0.00691131661842573\\
78.55	0.00691237366505744\\
78.56	0.00691343099626751\\
78.57	0.00691448861236541\\
78.58	0.00691554651366108\\
78.59	0.00691660470046494\\
78.6	0.00691766317308787\\
78.61	0.00691872193184123\\
78.62	0.00691978097703681\\
78.63	0.00692084030898685\\
78.64	0.00692189992800404\\
78.65	0.00692295983440151\\
78.66	0.00692402002849279\\
78.67	0.00692508051059184\\
78.68	0.00692614128101303\\
78.69	0.00692720234007113\\
78.7	0.00692826368808131\\
78.71	0.00692932532535913\\
78.72	0.00693038725222051\\
78.73	0.00693144946898177\\
78.74	0.00693251197595958\\
78.75	0.00693357477347095\\
78.76	0.00693463786183328\\
78.77	0.00693570124136426\\
78.78	0.00693676491238195\\
78.79	0.00693782887520471\\
78.8	0.00693889313015123\\
78.81	0.0069399576775405\\
78.82	0.00694102251769181\\
78.83	0.00694208765092473\\
78.84	0.00694315307755913\\
78.85	0.00694421879791512\\
78.86	0.00694528481231311\\
78.87	0.00694635112107373\\
78.88	0.00694741772451788\\
78.89	0.00694848462296668\\
78.9	0.00694955181674147\\
78.91	0.00695061930616383\\
78.92	0.00695168709155552\\
78.93	0.00695275517323852\\
78.94	0.00695382355153497\\
78.95	0.00695489222676722\\
78.96	0.00695596119925777\\
78.97	0.00695703046932927\\
78.98	0.00695810003730453\\
78.99	0.00695916990350648\\
79	0.0069602400682582\\
79.01	0.00696131053188288\\
79.02	0.0069623812947038\\
79.03	0.00696345235704436\\
79.04	0.00696452371922802\\
79.05	0.00696559538157832\\
79.06	0.00696666734441886\\
79.07	0.00696773960807332\\
79.08	0.00696881217286538\\
79.09	0.00696988503911876\\
79.1	0.00697095820715721\\
79.11	0.00697203167730448\\
79.12	0.00697310544988431\\
79.13	0.00697417952522042\\
79.14	0.00697525390363651\\
79.15	0.00697632858545624\\
79.16	0.0069774035710032\\
79.17	0.00697847886060091\\
79.18	0.00697955445457285\\
79.19	0.00698063035324237\\
79.2	0.00698170655693274\\
79.21	0.00698278306596711\\
79.22	0.00698385988066847\\
79.23	0.00698493700135973\\
79.24	0.00698601442836358\\
79.25	0.00698709216200258\\
79.26	0.00698817020259911\\
79.27	0.00698924855047534\\
79.28	0.00699032720595323\\
79.29	0.00699140616935455\\
79.3	0.00699248544100077\\
79.31	0.00699356502121319\\
79.32	0.00699464491031278\\
79.33	0.00699572510862026\\
79.34	0.00699680561645607\\
79.35	0.00699788643414032\\
79.36	0.0069989675619928\\
79.37	0.00700004900033299\\
79.38	0.00700113074947999\\
79.39	0.00700221280975254\\
79.4	0.00700329518146902\\
79.41	0.00700437786494738\\
79.42	0.00700546086050518\\
79.43	0.00700654416845955\\
79.44	0.00700762778912718\\
79.45	0.00700871172282429\\
79.46	0.00700979596986663\\
79.47	0.00701088053056947\\
79.48	0.00701196540524755\\
79.49	0.00701305059421512\\
79.5	0.00701413609778585\\
79.51	0.00701522191627289\\
79.52	0.00701630804998879\\
79.53	0.00701739449924552\\
79.54	0.00701848126435444\\
79.55	0.0070195683456263\\
79.56	0.00702065574337118\\
79.57	0.00702174345789851\\
79.58	0.00702283148951707\\
79.59	0.00702391983853491\\
79.6	0.00702500850525936\\
79.61	0.00702609748999707\\
79.62	0.00702718679305387\\
79.63	0.00702827641473488\\
79.64	0.0070293663553444\\
79.65	0.00703045661518593\\
79.66	0.00703154719456216\\
79.67	0.00703263809377493\\
79.68	0.00703372931312518\\
79.69	0.00703482085291302\\
79.7	0.00703591271343762\\
79.71	0.00703700489499726\\
79.72	0.00703809739788923\\
79.73	0.0070391902224099\\
79.74	0.00704028336885466\\
79.75	0.00704137683751785\\
79.76	0.00704247062869283\\
79.77	0.00704356474267188\\
79.78	0.00704465917974626\\
79.79	0.0070457539402061\\
79.8	0.00704684902434044\\
79.81	0.00704794443243718\\
79.82	0.00704904016478307\\
79.83	0.0070501362216637\\
79.84	0.00705123260336345\\
79.85	0.00705232931016549\\
79.86	0.00705342634235172\\
79.87	0.00705452370020283\\
79.88	0.00705562138399819\\
79.89	0.00705671939401585\\
79.9	0.00705781773053254\\
79.91	0.00705891639382365\\
79.92	0.00706001538416316\\
79.93	0.00706111470182368\\
79.94	0.00706221434707636\\
79.95	0.00706331432019093\\
79.96	0.0070644146214356\\
79.97	0.00706551525107713\\
79.98	0.00706661620938072\\
79.99	0.00706771749661004\\
80	0.00706881911302716\\
80.01	0.00706992105889258\\
};
\addplot [color=green,solid]
  table[row sep=crcr]{%
80.01	0.00706992105889258\\
80.02	0.00707102333446515\\
80.03	0.00707212594000207\\
80.04	0.00707322887575887\\
80.05	0.00707433214198938\\
80.06	0.00707543573894569\\
80.07	0.00707653966687812\\
80.08	0.00707764392603524\\
80.09	0.00707874851666377\\
80.1	0.00707985343900862\\
80.11	0.00708095869331281\\
80.12	0.00708206427981749\\
80.13	0.00708317019876189\\
80.14	0.00708427645038324\\
80.15	0.00708538303491685\\
80.16	0.00708648995259601\\
80.17	0.00708759720365195\\
80.18	0.00708870478831386\\
80.19	0.00708981270680881\\
80.2	0.00709092095936179\\
80.21	0.00709202954619558\\
80.22	0.00709313846753084\\
80.23	0.00709424772358594\\
80.24	0.00709535731457708\\
80.25	0.00709646724071814\\
80.26	0.00709757750222071\\
80.27	0.00709868809929404\\
80.28	0.00709979903214502\\
80.29	0.00710091030097813\\
80.3	0.00710202190599541\\
80.31	0.00710313384739646\\
80.32	0.00710424612537837\\
80.33	0.0071053587401357\\
80.34	0.00710647169186045\\
80.35	0.00710758498074202\\
80.36	0.00710869860696719\\
80.37	0.00710981257072009\\
80.38	0.00711092687218212\\
80.39	0.00711204151153199\\
80.4	0.00711315648894561\\
80.41	0.0071142718045961\\
80.42	0.00711538745865561\\
80.43	0.00711650345129651\\
80.44	0.00711761978269146\\
80.45	0.00711873645301335\\
80.46	0.00711985346243536\\
80.47	0.00712097081113091\\
80.48	0.00712208849927367\\
80.49	0.00712320652703758\\
80.5	0.00712432489459684\\
80.51	0.0071254436021259\\
80.52	0.00712656264979945\\
80.53	0.00712768203779246\\
80.54	0.00712880176628013\\
80.55	0.00712992183543793\\
80.56	0.00713104224544158\\
80.57	0.00713216299646703\\
80.58	0.00713328408869052\\
80.59	0.0071344055222885\\
80.6	0.00713552729743768\\
80.61	0.00713664941431505\\
80.62	0.00713777187309781\\
80.63	0.00713889467396342\\
80.64	0.00714001781708959\\
80.65	0.00714114130265427\\
80.66	0.00714226513083565\\
80.67	0.00714338930181219\\
80.68	0.00714451381576256\\
80.69	0.00714563867286569\\
80.7	0.00714676387330076\\
80.71	0.00714788941724717\\
80.72	0.00714901530488458\\
80.73	0.00715014153639288\\
80.74	0.0071512681119522\\
80.75	0.00715239503174289\\
80.76	0.00715352229594558\\
80.77	0.0071546499047411\\
80.78	0.00715577785831054\\
80.79	0.0071569061568352\\
80.8	0.00715803480049661\\
80.81	0.00715916378947658\\
80.82	0.00716029312395711\\
80.83	0.00716142280412043\\
80.84	0.00716255283014903\\
80.85	0.0071636832022256\\
80.86	0.00716481392053309\\
80.87	0.00716594498525464\\
80.88	0.00716707639657365\\
80.89	0.00716820815467372\\
80.9	0.0071693402597387\\
80.91	0.00717047271195265\\
80.92	0.00717160551149985\\
80.93	0.00717273865856481\\
80.94	0.00717387215333226\\
80.95	0.00717500599598716\\
80.96	0.00717614018671467\\
80.97	0.00717727472570017\\
80.98	0.00717840961312929\\
80.99	0.00717954484918784\\
81	0.00718068043406186\\
81.01	0.0071818163679376\\
81.02	0.00718295265100153\\
81.03	0.00718408928344034\\
81.04	0.00718522626544092\\
81.05	0.00718636359719036\\
81.06	0.00718750127887599\\
81.07	0.00718863931068533\\
81.08	0.00718977769280611\\
81.09	0.00719091642542626\\
81.1	0.00719205550873392\\
81.11	0.00719319494291746\\
81.12	0.00719433472816541\\
81.13	0.00719547486466653\\
81.14	0.00719661535260978\\
81.15	0.00719775619218431\\
81.16	0.00719889738357949\\
81.17	0.00720003892698485\\
81.18	0.00720118082259016\\
81.19	0.00720232307058537\\
81.2	0.00720346567116061\\
81.21	0.00720460862450622\\
81.22	0.00720575193081273\\
81.23	0.00720689559027086\\
81.24	0.00720803960307152\\
81.25	0.00720918396940582\\
81.26	0.00721032868946504\\
81.27	0.00721147376344066\\
81.28	0.00721261919152434\\
81.29	0.00721376497390793\\
81.3	0.00721491111078345\\
81.31	0.00721605760234314\\
81.32	0.00721720444877936\\
81.33	0.0072183516502847\\
81.34	0.00721949920705192\\
81.35	0.00722064711927393\\
81.36	0.00722179538714386\\
81.37	0.00722294401085499\\
81.38	0.00722409299060076\\
81.39	0.0072252423265748\\
81.4	0.00722639201897092\\
81.41	0.00722754206798309\\
81.42	0.00722869247380545\\
81.43	0.0072298432366323\\
81.44	0.00723099435665812\\
81.45	0.00723214583407754\\
81.46	0.00723329766908538\\
81.47	0.00723444986187658\\
81.48	0.00723560241264629\\
81.49	0.00723675532158978\\
81.5	0.0072379085889025\\
81.51	0.00723906221478005\\
81.52	0.0072402161994182\\
81.53	0.00724137054301285\\
81.54	0.00724252524576006\\
81.55	0.00724368030785607\\
81.56	0.00724483572949722\\
81.57	0.00724599151088006\\
81.58	0.00724714765220123\\
81.59	0.00724830415365755\\
81.6	0.00724946101544599\\
81.61	0.00725061823776363\\
81.62	0.00725177582080773\\
81.63	0.00725293376477567\\
81.64	0.00725409206986496\\
81.65	0.00725525073627328\\
81.66	0.00725640976419843\\
81.67	0.00725756915383832\\
81.68	0.00725872890539105\\
81.69	0.00725988901905479\\
81.7	0.0072610494950279\\
81.71	0.00726221033350882\\
81.72	0.00726337153469616\\
81.73	0.00726453309878861\\
81.74	0.00726569502598502\\
81.75	0.00726685731648436\\
81.76	0.00726801997048571\\
81.77	0.00726918298818828\\
81.78	0.00727034636979141\\
81.79	0.00727151011549452\\
81.8	0.00727267422549718\\
81.81	0.00727383869999906\\
81.82	0.00727500353919996\\
81.83	0.00727616874329978\\
81.84	0.00727733431249851\\
81.85	0.00727850024699628\\
81.86	0.00727966654699333\\
81.87	0.00728083321268996\\
81.88	0.00728200024428662\\
81.89	0.00728316764198385\\
81.9	0.00728433540598229\\
81.91	0.00728550353648267\\
81.92	0.00728667203368582\\
81.93	0.00728784089779268\\
81.94	0.00728901012900426\\
81.95	0.00729017972752169\\
81.96	0.00729134969354617\\
81.97	0.00729252002727901\\
81.98	0.00729369072892159\\
81.99	0.00729486179867538\\
82	0.00729603323674194\\
82.01	0.0072972050433229\\
82.02	0.00729837721862\\
82.03	0.00729954976283503\\
82.04	0.00730072267616987\\
82.05	0.00730189595882648\\
82.06	0.00730306961100689\\
82.07	0.0073042436329132\\
82.08	0.00730541802474758\\
82.09	0.00730659278671229\\
82.1	0.00730776791900963\\
82.11	0.007308943421842\\
82.12	0.00731011929541182\\
82.13	0.00731129553992161\\
82.14	0.00731247215557394\\
82.15	0.00731364914257144\\
82.16	0.00731482650111678\\
82.17	0.00731600423141273\\
82.18	0.00731718233366207\\
82.19	0.00731836080806765\\
82.2	0.00731953965483239\\
82.21	0.00732071887415921\\
82.22	0.00732189846625113\\
82.23	0.00732307843131119\\
82.24	0.00732425876954247\\
82.25	0.00732543948114811\\
82.26	0.00732662056633128\\
82.27	0.00732780202529519\\
82.28	0.00732898385824308\\
82.29	0.00733016606537823\\
82.3	0.00733134864690397\\
82.31	0.00733253160302363\\
82.32	0.00733371493394061\\
82.33	0.00733489863985829\\
82.34	0.00733608272098011\\
82.35	0.00733726717750953\\
82.36	0.00733845200965003\\
82.37	0.00733963721760511\\
82.38	0.00734082280157829\\
82.39	0.0073420087617731\\
82.4	0.0073431950983931\\
82.41	0.00734438181164184\\
82.42	0.00734556890172291\\
82.43	0.0073467563688399\\
82.44	0.00734794421319639\\
82.45	0.007349132434996\\
82.46	0.00735032103444232\\
82.47	0.00735151001173897\\
82.48	0.00735269936708955\\
82.49	0.00735388910069767\\
82.5	0.00735507921276693\\
82.51	0.00735626970350094\\
82.52	0.00735746057310329\\
82.53	0.00735865182177756\\
82.54	0.00735984344972733\\
82.55	0.00736103545715615\\
82.56	0.00736222784426757\\
82.57	0.00736342061126513\\
82.58	0.00736461375835235\\
82.59	0.00736580728573269\\
82.6	0.00736700119360966\\
82.61	0.00736819548218668\\
82.62	0.00736939015166719\\
82.63	0.00737058520225457\\
82.64	0.00737178063415219\\
82.65	0.0073729764475634\\
82.66	0.00737417264269147\\
82.67	0.00737536921973969\\
82.68	0.0073765661789113\\
82.69	0.00737776352040945\\
82.7	0.00737896124443733\\
82.71	0.00738015935119803\\
82.72	0.00738135784089461\\
82.73	0.00738255671373009\\
82.74	0.00738375596990744\\
82.75	0.00738495560962959\\
82.76	0.00738615563309938\\
82.77	0.00738735604051965\\
82.78	0.00738855683209315\\
82.79	0.00738975800802258\\
82.8	0.00739095956851056\\
82.81	0.0073921615137597\\
82.82	0.0073933638439725\\
82.83	0.00739456655935141\\
82.84	0.00739576966009882\\
82.85	0.00739697314641704\\
82.86	0.00739817701850832\\
82.87	0.00739938127657482\\
82.88	0.00740058592081865\\
82.89	0.00740179095144182\\
82.9	0.00740299636864628\\
82.91	0.0074042021726339\\
82.92	0.00740540836360644\\
82.93	0.00740661494176562\\
82.94	0.00740782190731303\\
82.95	0.00740902926045021\\
82.96	0.00741023700137859\\
82.97	0.00741144513029951\\
82.98	0.00741265364741422\\
82.99	0.00741386255292388\\
83	0.00741507184702955\\
83.01	0.00741628152993219\\
83.02	0.00741749160183265\\
83.03	0.0074187020629317\\
83.04	0.00741991291342999\\
83.05	0.00742112415352807\\
83.06	0.00742233578342637\\
83.07	0.00742354780332524\\
83.08	0.00742476021342488\\
83.09	0.0074259730139254\\
83.1	0.00742718620502681\\
83.11	0.00742839978692897\\
83.12	0.00742961375983164\\
83.13	0.00743082812393446\\
83.14	0.00743204287943695\\
83.15	0.00743325802653849\\
83.16	0.00743447356543835\\
83.17	0.00743568949633567\\
83.18	0.00743690581942946\\
83.19	0.0074381225349186\\
83.2	0.00743933964300183\\
83.21	0.00744055714387778\\
83.22	0.00744177503774491\\
83.23	0.00744299332480158\\
83.24	0.00744421200524597\\
83.25	0.00744543107927615\\
83.26	0.00744665054709004\\
83.27	0.00744787040888541\\
83.28	0.00744909066485989\\
83.29	0.00745031131521097\\
83.3	0.00745153236013596\\
83.31	0.00745275379983205\\
83.32	0.00745397563449627\\
83.33	0.00745519786432549\\
83.34	0.00745642048951643\\
83.35	0.00745764351026566\\
83.36	0.00745886692676957\\
83.37	0.00746009073922442\\
83.38	0.00746131494782627\\
83.39	0.00746253955277104\\
83.4	0.0074637645542545\\
83.41	0.00746498995247222\\
83.42	0.00746621574761963\\
83.43	0.00746744193989196\\
83.44	0.00746866852948431\\
83.45	0.00746989551659157\\
83.46	0.00747112290140848\\
83.47	0.00747235068412959\\
83.48	0.00747357886494928\\
83.49	0.00747480744406175\\
83.5	0.00747603642166103\\
83.51	0.00747726579794096\\
83.52	0.00747849557309518\\
83.53	0.00747972574731719\\
83.54	0.00748095632080027\\
83.55	0.00748218729373751\\
83.56	0.00748341866632186\\
83.57	0.00748465043874601\\
83.58	0.00748588261120252\\
83.59	0.00748711518388373\\
83.6	0.00748834815698178\\
83.61	0.00748958153068865\\
83.62	0.0074908153051961\\
83.63	0.00749204948069568\\
83.64	0.00749328405737877\\
83.65	0.00749451903543655\\
83.66	0.00749575441505997\\
83.67	0.00749699019643983\\
83.68	0.00749822637976667\\
83.69	0.00749946296523087\\
83.7	0.00750069995302259\\
83.71	0.00750193734333178\\
83.72	0.00750317513634819\\
83.73	0.00750441333226138\\
83.74	0.00750565193126066\\
83.75	0.00750689093353518\\
83.76	0.00750813033927383\\
83.77	0.00750937014866532\\
83.78	0.00751061036189816\\
83.79	0.00751185097916061\\
83.8	0.00751309200064074\\
83.81	0.00751433342652641\\
83.82	0.00751557525700524\\
83.83	0.00751681749226468\\
83.84	0.0075180601324919\\
83.85	0.0075193031778739\\
83.86	0.00752054662859746\\
83.87	0.00752179048484912\\
83.88	0.0075230347468152\\
83.89	0.00752427941468183\\
83.9	0.00752552448863488\\
83.91	0.00752676996886002\\
83.92	0.00752801585554272\\
83.93	0.00752926214886817\\
83.94	0.00753050884902141\\
83.95	0.00753175595618718\\
83.96	0.00753300347055005\\
83.97	0.00753425139229436\\
83.98	0.00753549972160421\\
83.99	0.00753674845866349\\
84	0.00753799760365583\\
84.01	0.0075392471567647\\
84.02	0.00754049711817328\\
84.03	0.00754174748806456\\
84.04	0.00754299826662129\\
84.05	0.007544249454026\\
84.06	0.007545501050461\\
84.07	0.00754675305610836\\
84.08	0.00754800547114994\\
84.09	0.00754925829576735\\
84.1	0.007550511530142\\
84.11	0.00755176517445506\\
84.12	0.00755301922888766\\
84.13	0.0075542736936211\\
84.14	0.00755552856883683\\
84.15	0.00755678385471648\\
84.16	0.00755803955144182\\
84.17	0.00755929565919483\\
84.18	0.00756055217815759\\
84.19	0.00756180910851239\\
84.2	0.00756306645044166\\
84.21	0.00756432420412798\\
84.22	0.0075655823697541\\
84.23	0.00756684094750295\\
84.24	0.00756809993755756\\
84.25	0.00756935934010118\\
84.26	0.00757061915531717\\
84.27	0.00757187938338907\\
84.28	0.00757314002450055\\
84.29	0.00757440107883546\\
84.3	0.00757566254657779\\
84.31	0.00757692442791167\\
84.32	0.00757818672302141\\
84.33	0.00757944943209144\\
84.34	0.00758071255530635\\
84.35	0.00758197609285088\\
84.36	0.00758324004490992\\
84.37	0.00758450441166851\\
84.38	0.00758576919331182\\
84.39	0.00758703439002517\\
84.4	0.00758830000199404\\
84.41	0.00758956602940404\\
84.42	0.0075908324724409\\
84.43	0.00759209933129055\\
84.44	0.007593366606139\\
84.45	0.00759463429717243\\
84.46	0.00759590240457717\\
84.47	0.00759717092853964\\
84.48	0.00759843986924645\\
84.49	0.00759970922688432\\
84.5	0.00760097900164011\\
84.51	0.00760224919370081\\
84.52	0.00760351980325356\\
84.53	0.0076047908304856\\
84.54	0.00760606227558434\\
84.55	0.00760733413873728\\
84.56	0.00760860642013209\\
84.57	0.00760987911995653\\
84.58	0.00761115223839853\\
84.59	0.0076124257756461\\
84.6	0.00761369973188742\\
84.61	0.00761497410731075\\
84.62	0.00761624890210453\\
84.63	0.00761752411645727\\
84.64	0.00761879975055762\\
84.65	0.00762007580459436\\
84.66	0.00762135227875637\\
84.67	0.00762262917323268\\
84.68	0.0076239064882124\\
84.69	0.0076251842238848\\
84.7	0.00762646238043923\\
84.71	0.00762774095806515\\
84.72	0.00762901995695217\\
84.73	0.00763029937729\\
84.74	0.00763157921926843\\
84.75	0.0076328594830774\\
84.76	0.00763414016890694\\
84.77	0.00763542127694719\\
84.78	0.0076367028073884\\
84.79	0.00763798476042093\\
84.8	0.00763926713623524\\
84.81	0.0076405499350219\\
84.82	0.00764183315697156\\
84.83	0.00764311680227502\\
84.84	0.00764440087112312\\
84.85	0.00764568536370685\\
84.86	0.00764697028021728\\
84.87	0.00764825562084556\\
84.88	0.00764954138578296\\
84.89	0.00765082757522085\\
84.9	0.00765211418935067\\
84.91	0.00765340122836397\\
84.92	0.00765468869245239\\
84.93	0.00765597658180765\\
84.94	0.00765726489662155\\
84.95	0.007658553637086\\
84.96	0.00765984280339296\\
84.97	0.0076611323957345\\
84.98	0.00766242241430274\\
84.99	0.00766371285928988\\
85	0.0076650037308882\\
85.01	0.00766629502929004\\
85.02	0.0076675867546878\\
85.03	0.00766887890727396\\
85.04	0.00767017148724105\\
85.05	0.00767146449478166\\
85.06	0.00767275793008844\\
85.07	0.00767405179335408\\
85.08	0.00767534608477133\\
85.09	0.00767664080453299\\
85.1	0.00767793595283189\\
85.11	0.0076792315298609\\
85.12	0.00768052753581296\\
85.13	0.00768182397088101\\
85.14	0.00768312083525804\\
85.15	0.00768441812913708\\
85.16	0.00768571585271114\\
85.17	0.00768701400617332\\
85.18	0.00768831258971669\\
85.19	0.00768961160353436\\
85.2	0.00769091104781945\\
85.21	0.00769221092276509\\
85.22	0.00769351122856444\\
85.23	0.00769481196541063\\
85.24	0.00769611313349679\\
85.25	0.00769741473301611\\
85.26	0.0076987167641617\\
85.27	0.00770001922712671\\
85.28	0.00770132212210427\\
85.29	0.00770262544928749\\
85.3	0.00770392920886946\\
85.31	0.00770523340104325\\
85.32	0.00770653802600192\\
85.33	0.00770784308393849\\
85.34	0.00770914857504595\\
85.35	0.00771045449951726\\
85.36	0.00771176085754535\\
85.37	0.00771306764932307\\
85.38	0.00771437487504328\\
85.39	0.00771568253489876\\
85.4	0.00771699062908223\\
85.41	0.00771829915778638\\
85.42	0.00771960812120382\\
85.43	0.00772091751952711\\
85.44	0.00772222735294874\\
85.45	0.00772353762166112\\
85.46	0.00772484832585659\\
85.47	0.00772615946572742\\
85.48	0.00772747104146579\\
85.49	0.00772878305326379\\
85.5	0.00773009550131342\\
85.51	0.00773140838580661\\
85.52	0.00773272170693515\\
85.53	0.00773403546489076\\
85.54	0.00773534965986505\\
85.55	0.0077366642920495\\
85.56	0.00773797936163551\\
85.57	0.00773929486881431\\
85.58	0.00774061081377707\\
85.59	0.00774192719671477\\
85.6	0.00774324401781831\\
85.61	0.00774456127727842\\
85.62	0.00774587897528571\\
85.63	0.00774719711203063\\
85.64	0.00774851568770351\\
85.65	0.00774983470249448\\
85.66	0.00775115415659355\\
85.67	0.00775247405019056\\
85.68	0.00775379438347517\\
85.69	0.00775511515663688\\
85.7	0.00775643636986502\\
85.71	0.00775775802334872\\
85.72	0.00775908011727694\\
85.73	0.00776040265183844\\
85.74	0.00776172562722179\\
85.75	0.00776304904361535\\
85.76	0.00776437290120731\\
85.77	0.0077656972001856\\
85.78	0.00776702194073798\\
85.79	0.00776834712305195\\
85.8	0.00776967274731482\\
85.81	0.00777099881371365\\
85.82	0.00777232532243528\\
85.83	0.00777365227366629\\
85.84	0.00777497966759304\\
85.85	0.00777630750440162\\
85.86	0.00777763578427786\\
85.87	0.00777896450740736\\
85.88	0.00778029367397541\\
85.89	0.00778162328416708\\
85.9	0.00778295333816711\\
85.91	0.00778428383616\\
85.92	0.00778561477832993\\
85.93	0.00778694616486083\\
85.94	0.00778827799593627\\
85.95	0.00778961027173957\\
85.96	0.0077909429924537\\
85.97	0.00779227615826134\\
85.98	0.00779360976934484\\
85.99	0.00779494382588623\\
86	0.00779627832806718\\
86.01	0.00779761327606906\\
86.02	0.00779894867007285\\
86.03	0.00780028451025922\\
86.04	0.00780162079680847\\
86.05	0.00780295752990051\\
86.06	0.00780429470971492\\
86.07	0.00780563233643088\\
86.08	0.0078069704102272\\
86.09	0.00780830893128229\\
86.1	0.00780964789977418\\
86.11	0.00781098731588049\\
86.12	0.00781232717977842\\
86.13	0.0078136674916448\\
86.14	0.00781500825165599\\
86.15	0.00781634945998794\\
86.16	0.00781769111681618\\
86.17	0.0078190332223158\\
86.18	0.00782037577666142\\
86.19	0.00782171878002723\\
86.2	0.00782306223258695\\
86.21	0.00782440613451383\\
86.22	0.00782575048598065\\
86.23	0.00782709528715973\\
86.24	0.00782844053822285\\
86.25	0.00782978623934136\\
86.26	0.00783113239068605\\
86.27	0.00783247899242724\\
86.28	0.00783382604473472\\
86.29	0.00783517354777776\\
86.3	0.00783652150172509\\
86.31	0.00783786990674491\\
86.32	0.00783921876300489\\
86.33	0.00784056807067211\\
86.34	0.00784191782991312\\
86.35	0.00784326804089389\\
86.36	0.00784461870377982\\
86.37	0.00784596981873573\\
86.38	0.00784732138592582\\
86.39	0.00784867340551374\\
86.4	0.0078500258776625\\
86.41	0.00785137880253449\\
86.42	0.00785273218029152\\
86.43	0.00785408601109472\\
86.44	0.0078554402951046\\
86.45	0.00785679503248103\\
86.46	0.00785815022338323\\
86.47	0.00785950586796974\\
86.48	0.00786086196639843\\
86.49	0.00786221851882651\\
86.5	0.00786357552541048\\
86.51	0.00786493298630616\\
86.52	0.00786629090166866\\
86.53	0.00786764927165236\\
86.54	0.00786900809641095\\
86.55	0.00787036737609735\\
86.56	0.00787172711086379\\
86.57	0.00787308730086169\\
86.58	0.00787444794624178\\
86.59	0.00787580904715395\\
86.6	0.00787717060374738\\
86.61	0.00787853261617043\\
86.62	0.00787989508457068\\
86.63	0.0078812580090949\\
86.64	0.00788262138988905\\
86.65	0.00788398522709827\\
86.66	0.00788534952086686\\
86.67	0.00788671427133829\\
86.68	0.00788807947865518\\
86.69	0.0078894451429593\\
86.7	0.00789081126439152\\
86.71	0.00789217784309185\\
86.72	0.00789354487919942\\
86.73	0.00789491237285246\\
86.74	0.00789628032418828\\
86.75	0.00789764873334326\\
86.76	0.00789901760045288\\
86.77	0.00790038692565167\\
86.78	0.0079017567090732\\
86.79	0.00790312695085009\\
86.8	0.00790449765111401\\
86.81	0.00790586880999561\\
86.82	0.0079072404276251\\
86.83	0.00790861250413245\\
86.84	0.00790998503964736\\
86.85	0.00791135803429931\\
86.86	0.00791273148821752\\
86.87	0.00791410540153096\\
86.88	0.00791547977436833\\
86.89	0.00791685460685812\\
86.9	0.00791822989912851\\
86.91	0.00791960565130747\\
86.92	0.00792098186352269\\
86.93	0.0079223585359016\\
86.94	0.00792373566857136\\
86.95	0.00792511326165888\\
86.96	0.00792649131529079\\
86.97	0.00792786982959348\\
86.98	0.00792924880469303\\
86.99	0.00793062824071528\\
87	0.00793200813778579\\
87.01	0.00793338849602983\\
87.02	0.00793476931557242\\
87.03	0.00793615059653828\\
87.04	0.00793753233905187\\
87.05	0.00793891454323735\\
87.06	0.00794029720921861\\
87.07	0.00794168033711925\\
87.08	0.00794306392706258\\
87.09	0.00794444797917164\\
87.1	0.00794583249356914\\
87.11	0.00794721747037756\\
87.12	0.00794860290971904\\
87.13	0.00794998881171543\\
87.14	0.0079513751764883\\
87.15	0.00795276200415891\\
87.16	0.00795414929484824\\
87.17	0.00795553704867694\\
87.18	0.00795692526576537\\
87.19	0.00795831394623359\\
87.2	0.00795970309020136\\
87.21	0.00796109269778811\\
87.22	0.00796248276911298\\
87.23	0.00796387330429478\\
87.24	0.00796526430345204\\
87.25	0.00796665576670296\\
87.26	0.00796804769416539\\
87.27	0.00796944008595692\\
87.28	0.00797083294219478\\
87.29	0.00797222626299591\\
87.3	0.00797362004847689\\
87.31	0.00797501429875401\\
87.32	0.00797640901394322\\
87.33	0.00797780419416014\\
87.34	0.00797919983952008\\
87.35	0.00798059595013801\\
87.36	0.00798199252612855\\
87.37	0.007983389567606\\
87.38	0.00798478707468436\\
87.39	0.00798618504747723\\
87.4	0.00798758348609793\\
87.41	0.0079889823906594\\
87.42	0.00799038176127426\\
87.43	0.00799178159805478\\
87.44	0.0079931819011129\\
87.45	0.0079945826705602\\
87.46	0.00799598390650793\\
87.47	0.00799738560906697\\
87.48	0.00799878777834786\\
87.49	0.0080001904144608\\
87.5	0.00800159351751563\\
87.51	0.00800299708762183\\
87.52	0.00800440112488853\\
87.53	0.00800580562942451\\
87.54	0.0080072106013382\\
87.55	0.00800861604073764\\
87.56	0.00801002194773054\\
87.57	0.00801142832242422\\
87.58	0.00801283516492568\\
87.59	0.00801424247534151\\
87.6	0.00801565025377796\\
87.61	0.0080170585003409\\
87.62	0.00801846721513585\\
87.63	0.00801987639826794\\
87.64	0.00802128604984194\\
87.65	0.00802269616996226\\
87.66	0.0080241067587329\\
87.67	0.00802551781625753\\
87.68	0.00802692934263941\\
87.69	0.00802834133798144\\
87.7	0.00802975380238613\\
87.71	0.00803116673595564\\
87.72	0.0080325801387917\\
87.73	0.00803399401099571\\
87.74	0.00803540835266866\\
87.75	0.00803682316391115\\
87.76	0.00803823844482342\\
87.77	0.00803965419550531\\
87.78	0.00804107041605626\\
87.79	0.00804248710657535\\
87.8	0.00804390426716124\\
87.81	0.00804532189791223\\
87.82	0.00804673999892621\\
87.83	0.00804815857030068\\
87.84	0.00804957761213276\\
87.85	0.00805099712451916\\
87.86	0.00805241710755621\\
87.87	0.00805383756133982\\
87.88	0.00805525848596553\\
87.89	0.00805667988152847\\
87.9	0.00805810174812337\\
87.91	0.00805952408584458\\
87.92	0.00806094689478602\\
87.93	0.00806237017504122\\
87.94	0.00806379392670332\\
87.95	0.00806521814986505\\
87.96	0.00806664284461873\\
87.97	0.0080680680110563\\
87.98	0.00806949364926926\\
87.99	0.00807091975934874\\
88	0.00807234634138543\\
88.01	0.00807377339546965\\
88.02	0.00807520092169129\\
88.03	0.00807662892013983\\
88.04	0.00807805739090435\\
88.05	0.00807948633407354\\
88.06	0.00808091574973563\\
88.07	0.0080823456379785\\
88.08	0.00808377599888959\\
88.09	0.00808520683255591\\
88.1	0.00808663813906411\\
88.11	0.00808806991850038\\
88.12	0.00808950217095052\\
88.13	0.00809093489649992\\
88.14	0.00809236809523356\\
88.15	0.008093801767236\\
88.16	0.00809523591259138\\
88.17	0.00809667053138345\\
88.18	0.00809810562369552\\
88.19	0.00809954118961051\\
88.2	0.00810097722921091\\
88.21	0.00810241374257881\\
88.22	0.00810385072979588\\
88.23	0.00810528819094337\\
88.24	0.00810672612610212\\
88.25	0.00810816453535257\\
88.26	0.00810960341877471\\
88.27	0.00811104277644818\\
88.28	0.00811248260845214\\
88.29	0.00811392291486537\\
88.3	0.00811536369576623\\
88.31	0.00811680495123268\\
88.32	0.00811824668134225\\
88.33	0.00811968888617207\\
88.34	0.00812113156579884\\
88.35	0.00812257472029887\\
88.36	0.00812401834974804\\
88.37	0.00812546245422184\\
88.38	0.00812690703379533\\
88.39	0.00812835208854317\\
88.4	0.0081297976185396\\
88.41	0.00813124362385848\\
88.42	0.00813269010457322\\
88.43	0.00813413706075686\\
88.44	0.00813558449248201\\
88.45	0.00813703239982087\\
88.46	0.00813848078284526\\
88.47	0.00813992964162658\\
88.48	0.00814137897623581\\
88.49	0.00814282878674356\\
88.5	0.00814427907322002\\
88.51	0.00814572983573498\\
88.52	0.00814718107435782\\
88.53	0.00814863278915753\\
88.54	0.00815008498020272\\
88.55	0.00815153764756157\\
88.56	0.00815299079130189\\
88.57	0.00815444441149108\\
88.58	0.00815589850819616\\
88.59	0.00815735308148375\\
88.6	0.00815880813142007\\
88.61	0.00816026365807097\\
88.62	0.0081617196615019\\
88.63	0.00816317614177793\\
88.64	0.00816463309896374\\
88.65	0.00816609053312363\\
88.66	0.00816754844432152\\
88.67	0.00816900683262094\\
88.68	0.00817046569808507\\
88.69	0.00817192504077668\\
88.7	0.00817338486075818\\
88.71	0.00817484515809161\\
88.72	0.00817630593283863\\
88.73	0.00817776718506055\\
88.74	0.00817922891481829\\
88.75	0.00818069112217244\\
88.76	0.00818215380718318\\
88.77	0.00818361696991037\\
88.78	0.00818508061041348\\
88.79	0.00818654472875167\\
88.8	0.00818800932498372\\
88.81	0.00818947439916803\\
88.82	0.0081909399513627\\
88.83	0.00819240598162546\\
88.84	0.00819387249001372\\
88.85	0.00819533947658451\\
88.86	0.00819680694139456\\
88.87	0.00819827488450025\\
88.88	0.00819974330595763\\
88.89	0.00820121220582241\\
88.9	0.00820268158415\\
88.91	0.00820415144099548\\
88.92	0.00820562177641357\\
88.93	0.00820709259045874\\
88.94	0.0082085638831851\\
88.95	0.00821003565464646\\
88.96	0.00821150790489634\\
88.97	0.00821298063398793\\
88.98	0.00821445384197416\\
88.99	0.00821592752890762\\
89	0.00821740169484063\\
89.01	0.00821887633982523\\
89.02	0.00822035146391317\\
89.03	0.00822182706715589\\
89.04	0.0082233031496046\\
89.05	0.00822477971131021\\
89.06	0.00822625675232337\\
89.07	0.00822773427269446\\
89.08	0.0082292122724736\\
89.09	0.00823069075171066\\
89.1	0.00823216971045525\\
89.11	0.00823364914875676\\
89.12	0.00823512906666431\\
89.13	0.00823660946422678\\
89.14	0.00823809034149283\\
89.15	0.00823957169851089\\
89.16	0.00824105353532917\\
89.17	0.00824253585199566\\
89.18	0.00824401864855813\\
89.19	0.00824550192506416\\
89.2	0.00824698568156108\\
89.21	0.00824846991809606\\
89.22	0.00824995463471602\\
89.23	0.00825143983146769\\
89.24	0.00825292550839758\\
89.25	0.00825441166555201\\
89.26	0.00825589830297705\\
89.27	0.00825738542071859\\
89.28	0.0082588730188223\\
89.29	0.00826036109733364\\
89.3	0.00826184965629785\\
89.31	0.00826333869575996\\
89.32	0.00826482821576479\\
89.33	0.00826631821635695\\
89.34	0.00826780869758085\\
89.35	0.00826929965948065\\
89.36	0.00827079110210033\\
89.37	0.00827228302548365\\
89.38	0.00827377542967414\\
89.39	0.00827526831471515\\
89.4	0.00827676168064977\\
89.41	0.00827825552752092\\
89.42	0.00827974985537128\\
89.43	0.00828124466424333\\
89.44	0.00828273995417932\\
89.45	0.00828423572522129\\
89.46	0.00828573197741107\\
89.47	0.00828722871079028\\
89.48	0.00828872592540033\\
89.49	0.00829022362128238\\
89.5	0.00829172179847741\\
89.51	0.00829322045702617\\
89.52	0.0082947195969692\\
89.53	0.00829621921834682\\
89.54	0.00829771932119913\\
89.55	0.00829921990556603\\
89.56	0.00830072097148718\\
89.57	0.00830222251900204\\
89.58	0.00830372454814985\\
89.59	0.00830522705896964\\
89.6	0.00830673005150019\\
89.61	0.00830823352578012\\
89.62	0.00830973748184778\\
89.63	0.00831124191974133\\
89.64	0.00831274683949871\\
89.65	0.00831425224115763\\
89.66	0.0083157581247556\\
89.67	0.0083172644903299\\
89.68	0.00831877133791759\\
89.69	0.00832027866755552\\
89.7	0.00832178647928032\\
89.71	0.0083232947731284\\
89.72	0.00832480354913594\\
89.73	0.00832631280733894\\
89.74	0.00832782254777313\\
89.75	0.00832933277047405\\
89.76	0.00833084347547703\\
89.77	0.00833235466281715\\
89.78	0.0083338663325293\\
89.79	0.00833537848464813\\
89.8	0.00833689111920809\\
89.81	0.00833840423624339\\
89.82	0.00833991783578804\\
89.83	0.00834143191787581\\
89.84	0.00834294648254027\\
89.85	0.00834446152981475\\
89.86	0.00834597705973237\\
89.87	0.00834749307232604\\
89.88	0.00834900956762844\\
89.89	0.00835052654567203\\
89.9	0.00835204400648903\\
89.91	0.00835356195011148\\
89.92	0.00835508037657117\\
89.93	0.00835659928589968\\
89.94	0.00835811867812836\\
89.95	0.00835963855328834\\
89.96	0.00836115891141055\\
89.97	0.00836267975252568\\
89.98	0.00836420107666418\\
89.99	0.00836572288385633\\
90	0.00836724517413213\\
90.01	0.00836876794752141\\
90.02	0.00837029120405374\\
90.03	0.00837181494375849\\
90.04	0.0083733391666648\\
90.05	0.00837486387280159\\
90.06	0.00837638906219756\\
90.07	0.00837791473488118\\
90.08	0.00837944089088071\\
90.09	0.00838096753022417\\
90.1	0.00838249465293938\\
90.11	0.00838402225905392\\
90.12	0.00838555034859515\\
90.13	0.00838707892159022\\
90.14	0.00838860797806604\\
90.15	0.00839013751804931\\
90.16	0.00839166754156649\\
90.17	0.00839319804864386\\
90.18	0.00839472903930741\\
90.19	0.00839626051358296\\
90.2	0.00839779247149609\\
90.21	0.00839932491307215\\
90.22	0.00840085783833628\\
90.23	0.00840239124731339\\
90.24	0.00840392514002815\\
90.25	0.00840545951650504\\
90.26	0.00840699437676829\\
90.27	0.00840852972084192\\
90.28	0.00841006554874971\\
90.29	0.00841160186051523\\
90.3	0.00841313865616182\\
90.31	0.0084146759357126\\
90.32	0.00841621369919046\\
90.33	0.00841775194661807\\
90.34	0.00841929067801787\\
90.35	0.00842082989341209\\
90.36	0.00842236959282272\\
90.37	0.00842390977627152\\
90.38	0.00842545044378005\\
90.39	0.00842699159536961\\
90.4	0.00842853323106131\\
90.41	0.00843007535087602\\
90.42	0.00843161795483437\\
90.43	0.00843316104295679\\
90.44	0.00843470461526347\\
90.45	0.00843624867177437\\
90.46	0.00843779321250923\\
90.47	0.00843933823748757\\
90.48	0.00844088374672869\\
90.49	0.00844242974025163\\
90.5	0.00844397621807524\\
90.51	0.00844552318021813\\
90.52	0.00844707062669867\\
90.53	0.00844861855753504\\
90.54	0.00845016697274514\\
90.55	0.0084517158723467\\
90.56	0.00845326525635718\\
90.57	0.00845481512479383\\
90.58	0.00845636547767368\\
90.59	0.0084579163150135\\
90.6	0.00845946763682988\\
90.61	0.00846101944313915\\
90.62	0.00846257173395741\\
90.63	0.00846412450930056\\
90.64	0.00846567776918424\\
90.65	0.00846723151362388\\
90.66	0.00846878574263467\\
90.67	0.00847034045623159\\
90.68	0.00847189565442936\\
90.69	0.00847345133724251\\
90.7	0.0084750075046853\\
90.71	0.00847656415677179\\
90.72	0.0084781212935158\\
90.73	0.00847967891493092\\
90.74	0.00848123702103052\\
90.75	0.00848279561182771\\
90.76	0.0084843546873354\\
90.77	0.00848591424756627\\
90.78	0.00848747429253274\\
90.79	0.00848903482224703\\
90.8	0.0084905958367211\\
90.81	0.00849215733596669\\
90.82	0.0084937193199953\\
90.83	0.0084952817888182\\
90.84	0.00849684474244641\\
90.85	0.00849840818089074\\
90.86	0.00849997210416173\\
90.87	0.00850153651226971\\
90.88	0.00850310140522474\\
90.89	0.00850466678303666\\
90.9	0.00850623264571507\\
90.91	0.00850779899326931\\
90.92	0.0085093658257085\\
90.93	0.00851093314304149\\
90.94	0.00851250094527691\\
90.95	0.00851406923242314\\
90.96	0.00851563800448829\\
90.97	0.00851720726148025\\
90.98	0.00851877700340666\\
90.99	0.00852034723027488\\
91	0.00852191794209206\\
91.01	0.00852348913886508\\
91.02	0.00852506082060056\\
91.03	0.0085266329873049\\
91.04	0.0085282056389842\\
91.05	0.00852977877564434\\
91.06	0.00853135239729094\\
91.07	0.00853292650392935\\
91.08	0.00853450109556468\\
91.09	0.00853607617220177\\
91.1	0.00853765173384521\\
91.11	0.00853922778049931\\
91.12	0.00854080431216815\\
91.13	0.00854238132885553\\
91.14	0.00854395883056499\\
91.15	0.0085455368172998\\
91.16	0.00854711528906299\\
91.17	0.0085486942458573\\
91.18	0.00855027368768521\\
91.19	0.00855185361454893\\
91.2	0.00855343402645043\\
91.21	0.00855501492339136\\
91.22	0.00855659630537314\\
91.23	0.00855817817239691\\
91.24	0.00855976052446354\\
91.25	0.00856134336157361\\
91.26	0.00856292668372745\\
91.27	0.0085645104909251\\
91.28	0.00856609478316633\\
91.29	0.00856767956045062\\
91.3	0.00856926482277721\\
91.31	0.00857085057014501\\
91.32	0.0085724368025527\\
91.33	0.00857402351999863\\
91.34	0.00857561072248092\\
91.35	0.00857719840999736\\
91.36	0.0085787865825455\\
91.37	0.00858037524012256\\
91.38	0.00858196438272552\\
91.39	0.00858355401035102\\
91.4	0.00858514412299548\\
91.41	0.00858673472065497\\
91.42	0.00858832580332531\\
91.43	0.008589917371002\\
91.44	0.00859150942368028\\
91.45	0.00859310196135506\\
91.46	0.00859469498402098\\
91.47	0.0085962884916724\\
91.48	0.00859788248430334\\
91.49	0.00859947696190757\\
91.5	0.00860107192447852\\
91.51	0.00860266737200935\\
91.52	0.00860426330449291\\
91.53	0.00860585972192175\\
91.54	0.00860745662428811\\
91.55	0.00860905401158394\\
91.56	0.00861065188380087\\
91.57	0.00861225024093024\\
91.58	0.00861384908296308\\
91.59	0.0086154484098901\\
91.6	0.00861704822170171\\
91.61	0.00861864851838801\\
91.62	0.00862024929993878\\
91.63	0.0086218505663435\\
91.64	0.00862345231759135\\
91.65	0.00862505455367115\\
91.66	0.00862665727457144\\
91.67	0.00862826048028044\\
91.68	0.00862986417078604\\
91.69	0.00863146834607582\\
91.7	0.00863307300613703\\
91.71	0.00863467815095661\\
91.72	0.00863628378052118\\
91.73	0.00863788989481701\\
91.74	0.00863949649383009\\
91.75	0.00864110357754603\\
91.76	0.00864271114595016\\
91.77	0.00864431919902745\\
91.78	0.00864592773676255\\
91.79	0.0086475367591398\\
91.8	0.00864914626614316\\
91.81	0.00865075625775629\\
91.82	0.00865236673396253\\
91.83	0.00865397769474485\\
91.84	0.0086555891400859\\
91.85	0.00865720106996799\\
91.86	0.00865881348437309\\
91.87	0.00866042638328283\\
91.88	0.00866203976667849\\
91.89	0.00866365363454102\\
91.9	0.00866526798685101\\
91.91	0.00866688282358873\\
91.92	0.00866849814473407\\
91.93	0.0086701139502666\\
91.94	0.00867173024016552\\
91.95	0.0086733470144097\\
91.96	0.00867496427297763\\
91.97	0.00867658201584747\\
91.98	0.00867820024299702\\
91.99	0.00867981895440372\\
92	0.00868143815004465\\
92.01	0.00868305782989654\\
92.02	0.00868467799393575\\
92.03	0.0086862986421383\\
92.04	0.00868791977447982\\
92.05	0.00868954139093559\\
92.06	0.00869116349148053\\
92.07	0.00869278607608917\\
92.08	0.0086944091447357\\
92.09	0.00869603269739394\\
92.1	0.00869765673403731\\
92.11	0.00869928125463888\\
92.12	0.00870090625917136\\
92.13	0.00870253174760705\\
92.14	0.00870415771991791\\
92.15	0.0087057841760755\\
92.16	0.008707411116051\\
92.17	0.00870903853981522\\
92.18	0.00871066644733859\\
92.19	0.00871229483859115\\
92.2	0.00871392371354255\\
92.21	0.00871555307216208\\
92.22	0.0087171829144186\\
92.23	0.00871881324028064\\
92.24	0.00872044404971628\\
92.25	0.00872207534269324\\
92.26	0.00872370711917884\\
92.27	0.00872533937914002\\
92.28	0.0087269721225433\\
92.29	0.00872860534935482\\
92.3	0.00873023905954031\\
92.31	0.00873187325306512\\
92.32	0.00873350792989416\\
92.33	0.00873514308999198\\
92.34	0.00873677873332271\\
92.35	0.00873841485985005\\
92.36	0.00874005146953733\\
92.37	0.00874168856234745\\
92.38	0.0087433261382429\\
92.39	0.00874496419718576\\
92.4	0.00874660273913769\\
92.41	0.00874824176405996\\
92.42	0.0087498812719134\\
92.43	0.00875152126265842\\
92.44	0.00875316173625502\\
92.45	0.00875480269266278\\
92.46	0.00875644413184085\\
92.47	0.00875808605374797\\
92.48	0.00875972845834243\\
92.49	0.0087613713455821\\
92.5	0.00876301471542445\\
92.51	0.00876465856782648\\
92.52	0.00876630290274478\\
92.53	0.00876794772013549\\
92.54	0.00876959301995434\\
92.55	0.00877123880215659\\
92.56	0.0087728850666971\\
92.57	0.00877453181353026\\
92.58	0.00877617904261002\\
92.59	0.00877782675388989\\
92.6	0.00877947494732296\\
92.61	0.00878112362286184\\
92.62	0.0087827727804587\\
92.63	0.00878442242006527\\
92.64	0.00878607254163282\\
92.65	0.00878772314511216\\
92.66	0.00878937423045367\\
92.67	0.00879102579760725\\
92.68	0.00879267784652234\\
92.69	0.00879433037714794\\
92.7	0.00879598338943258\\
92.71	0.00879763688332431\\
92.72	0.00879929085877073\\
92.73	0.00880094531571899\\
92.74	0.00880260025411574\\
92.75	0.00880425567390717\\
92.76	0.00880591157503901\\
92.77	0.0088075679574565\\
92.78	0.00880922482110441\\
92.79	0.00881088216592705\\
92.8	0.00881253999186822\\
92.81	0.00881419829887127\\
92.82	0.00881585708687904\\
92.83	0.00881751635583391\\
92.84	0.00881917610567775\\
92.85	0.00882083633635197\\
92.86	0.00882249704779746\\
92.87	0.00882415823995466\\
92.88	0.00882581991276346\\
92.89	0.00882748206616331\\
92.9	0.00882914470009313\\
92.91	0.00883080781449135\\
92.92	0.00883247140929591\\
92.93	0.00883413548444423\\
92.94	0.00883580003987323\\
92.95	0.00883746507551934\\
92.96	0.00883913059131848\\
92.97	0.00884079658720602\\
92.98	0.00884246306311689\\
92.99	0.00884413001898544\\
93	0.00884579745474554\\
93.01	0.00884746537033053\\
93.02	0.00884913376567325\\
93.03	0.00885080264070599\\
93.04	0.00885247199536054\\
93.05	0.00885414182956815\\
93.06	0.00885581214325957\\
93.07	0.00885748293636499\\
93.08	0.00885915420881408\\
93.09	0.00886082596053599\\
93.1	0.00886249819145931\\
93.11	0.00886417090151213\\
93.12	0.00886584409062197\\
93.13	0.00886751775871583\\
93.14	0.00886919190572015\\
93.15	0.00887086653156085\\
93.16	0.00887254163616327\\
93.17	0.00887421721945224\\
93.18	0.00887589328135202\\
93.19	0.00887756982178631\\
93.2	0.00887924684067827\\
93.21	0.00888092433795051\\
93.22	0.00888260231352506\\
93.23	0.00888428076732341\\
93.24	0.00888595969926648\\
93.25	0.00888763910927462\\
93.26	0.00888931899726763\\
93.27	0.00889099936316473\\
93.28	0.00889268020688456\\
93.29	0.00889436152834522\\
93.3	0.0088960433274642\\
93.31	0.00889772560415844\\
93.32	0.00889940835834428\\
93.33	0.0089010915899375\\
93.34	0.00890277529885328\\
93.35	0.00890445948500622\\
93.36	0.00890614414831035\\
93.37	0.0089078292886791\\
93.38	0.00890951490602528\\
93.39	0.00891120100026115\\
93.4	0.00891288757129837\\
93.41	0.00891457461904797\\
93.42	0.00891626214342041\\
93.43	0.00891795014432553\\
93.44	0.00891963862167258\\
93.45	0.00892132757537021\\
93.46	0.00892301700532643\\
93.47	0.00892470691144868\\
93.48	0.00892639729364375\\
93.49	0.00892808815181785\\
93.5	0.00892977948587654\\
93.51	0.00893147129572478\\
93.52	0.00893316358126691\\
93.53	0.00893485634240664\\
93.54	0.00893654957904706\\
93.55	0.00893824329109061\\
93.56	0.00893993747843913\\
93.57	0.00894163214099382\\
93.58	0.00894332727865523\\
93.59	0.00894502289132328\\
93.6	0.00894671897889726\\
93.61	0.00894841554127581\\
93.62	0.00895011257835692\\
93.63	0.00895181009003795\\
93.64	0.0089535080762156\\
93.65	0.00895520653678591\\
93.66	0.0089569054716443\\
93.67	0.0089586048806855\\
93.68	0.00896030476380359\\
93.69	0.00896200512089201\\
93.7	0.0089637059518435\\
93.71	0.00896540725655018\\
93.72	0.00896710903490347\\
93.73	0.00896881128679413\\
93.74	0.00897051401211226\\
93.75	0.00897221721074726\\
93.76	0.00897392088258787\\
93.77	0.00897562502752216\\
93.78	0.00897732964543749\\
93.79	0.00897903473622057\\
93.8	0.00898074029975741\\
93.81	0.00898244633593331\\
93.82	0.00898415284463291\\
93.83	0.00898585982574014\\
93.84	0.00898756727913825\\
93.85	0.00898927520470976\\
93.86	0.00899098360233653\\
93.87	0.00899269247189968\\
93.88	0.00899440181327965\\
93.89	0.00899611162635616\\
93.9	0.00899782191100821\\
93.91	0.00899953266711411\\
93.92	0.00900124389455144\\
93.93	0.00900295559319706\\
93.94	0.00900466776292712\\
93.95	0.00900638040361704\\
93.96	0.0090080935151415\\
93.97	0.00900980709737449\\
93.98	0.00901152115018923\\
93.99	0.00901323567345822\\
94	0.00901495066705323\\
94.01	0.0090166661308453\\
94.02	0.00901838206470469\\
94.03	0.00902009846850097\\
94.04	0.00902181534210293\\
94.05	0.00902353268537861\\
94.06	0.00902525049819532\\
94.07	0.00902696878041959\\
94.08	0.00902868753191722\\
94.09	0.00903040675255323\\
94.1	0.00903212644219189\\
94.11	0.00903384660069669\\
94.12	0.00903556722793038\\
94.13	0.00903728832375491\\
94.14	0.00903900988803149\\
94.15	0.00904073192062054\\
94.16	0.00904245442138168\\
94.17	0.00904417739017378\\
94.18	0.00904590082685492\\
94.19	0.00904762473128239\\
94.2	0.0090493491033127\\
94.21	0.00905107394280154\\
94.22	0.00905279924960386\\
94.23	0.00905452502357376\\
94.24	0.00905625126456458\\
94.25	0.00905797797242882\\
94.26	0.00905970514701821\\
94.27	0.00906143278818367\\
94.28	0.00906316089577528\\
94.29	0.00906488946964233\\
94.3	0.0090666185096333\\
94.31	0.00906834801559584\\
94.32	0.00907007798737679\\
94.33	0.00907180842482214\\
94.34	0.00907353932777709\\
94.35	0.00907527069608597\\
94.36	0.00907700252959232\\
94.37	0.00907873482813882\\
94.38	0.00908046759156731\\
94.39	0.0090822008197188\\
94.4	0.00908393451243346\\
94.41	0.00908566866955059\\
94.42	0.00908740329090868\\
94.43	0.00908913837634532\\
94.44	0.00909087392569729\\
94.45	0.00909260993880048\\
94.46	0.00909434641548995\\
94.47	0.00909608335559985\\
94.48	0.00909782075896352\\
94.49	0.00909955862541339\\
94.5	0.00910129695478102\\
94.51	0.00910303574689713\\
94.52	0.00910477500159152\\
94.53	0.00910651471869314\\
94.54	0.00910825489803002\\
94.55	0.00910999553942936\\
94.56	0.00911173664271741\\
94.57	0.00911347820771957\\
94.58	0.00911522023426031\\
94.59	0.00911696272216324\\
94.6	0.00911870567125104\\
94.61	0.00912044908134549\\
94.62	0.00912219295226747\\
94.63	0.00912393728383695\\
94.64	0.00912568207587298\\
94.65	0.00912742732819369\\
94.66	0.00912917304061632\\
94.67	0.00913091921295714\\
94.68	0.00913266584503153\\
94.69	0.00913441293665392\\
94.7	0.00913616048763784\\
94.71	0.00913790849779584\\
94.72	0.00913965696693958\\
94.73	0.00914140589487975\\
94.74	0.00914315528142609\\
94.75	0.00914490512638742\\
94.76	0.0091466554295716\\
94.77	0.00914840619078552\\
94.78	0.00915015740983513\\
94.79	0.00915190908652542\\
94.8	0.00915366122066043\\
94.81	0.00915541381204321\\
94.82	0.00915716686047584\\
94.83	0.00915892036575946\\
94.84	0.00916067432769421\\
94.85	0.00916242874607926\\
94.86	0.0091641836207128\\
94.87	0.00916593895139203\\
94.88	0.00916769473791317\\
94.89	0.00916945098007144\\
94.9	0.00917120767766108\\
94.91	0.00917296483047533\\
94.92	0.00917472243830642\\
94.93	0.00917648050094558\\
94.94	0.00917823901818305\\
94.95	0.00917999798980803\\
94.96	0.00918175741560874\\
94.97	0.00918351729537238\\
94.98	0.00918527762888509\\
94.99	0.00918703841593204\\
95	0.00918879965629735\\
95.01	0.00919056134976411\\
95.02	0.00919232349611439\\
95.03	0.00919408609512922\\
95.04	0.00919584914658857\\
95.05	0.0091976126502714\\
95.06	0.00919937660595562\\
95.07	0.00920114101341807\\
95.08	0.00920290587243456\\
95.09	0.00920467118277984\\
95.1	0.0092064369442276\\
95.11	0.00920820315655047\\
95.12	0.00920996981952\\
95.13	0.0092117369329067\\
95.14	0.00921350449648\\
95.15	0.00921527251000824\\
95.16	0.0092170409732587\\
95.17	0.00921880988599757\\
95.18	0.00922057924798995\\
95.19	0.00922234905899986\\
95.2	0.00922411931879024\\
95.21	0.00922589002712292\\
95.22	0.00922766118375863\\
95.23	0.00922943278845701\\
95.24	0.00923120484097659\\
95.25	0.00923297734107478\\
95.26	0.0092347502885079\\
95.27	0.00923652368303115\\
95.28	0.00923829752439861\\
95.29	0.00924007181236322\\
95.3	0.00924184654667683\\
95.31	0.00924362172709014\\
95.32	0.0092453973533527\\
95.33	0.00924717342521298\\
95.34	0.00924894994241825\\
95.35	0.00925072690471467\\
95.36	0.00925250431184726\\
95.37	0.00925428216355989\\
95.38	0.00925606045959524\\
95.39	0.00925783919969489\\
95.4	0.00925961838359922\\
95.41	0.00926139801104746\\
95.42	0.00926317808177769\\
95.43	0.00926495859552679\\
95.44	0.00926673955203049\\
95.45	0.00926852095102333\\
95.46	0.00927030279223869\\
95.47	0.00927208507540874\\
95.48	0.00927386780026447\\
95.49	0.00927565096653571\\
95.5	0.00927743457395104\\
95.51	0.0092792186222379\\
95.52	0.00928100311112248\\
95.53	0.0092827880403298\\
95.54	0.00928457340958366\\
95.55	0.00928635921860665\\
95.56	0.00928814546712014\\
95.57	0.00928993215484427\\
95.58	0.009291719281498\\
95.59	0.00929350684679901\\
95.6	0.00929529485046379\\
95.61	0.00929708329220759\\
95.62	0.00929887217174441\\
95.63	0.00930066148878701\\
95.64	0.00930245124304692\\
95.65	0.00930424143423443\\
95.66	0.00930603206205855\\
95.67	0.00930782312622705\\
95.68	0.00930961462644646\\
95.69	0.00931140656242202\\
95.7	0.00931319893385773\\
95.71	0.00931499174045629\\
95.72	0.00931678498191916\\
95.73	0.0093185786579465\\
95.74	0.00932037276823722\\
95.75	0.00932216731248891\\
95.76	0.00932396229039789\\
95.77	0.00932575770165921\\
95.78	0.00932755354596659\\
95.79	0.00932934982301247\\
95.8	0.009331146532488\\
95.81	0.00933294367408301\\
95.82	0.00933474124748602\\
95.83	0.00933653925238424\\
95.84	0.00933833768846357\\
95.85	0.00934013655540859\\
95.86	0.00934193585290255\\
95.87	0.00934373558062738\\
95.88	0.00934553573826368\\
95.89	0.00934733632549071\\
95.9	0.00934913734198638\\
95.91	0.0093509387874273\\
95.92	0.0093527406614887\\
95.93	0.00935454296384446\\
95.94	0.00935634569416713\\
95.95	0.00935814885212788\\
95.96	0.00935995243739653\\
95.97	0.00936175644964156\\
95.98	0.00936356088853003\\
95.99	0.00936536575372768\\
96	0.00936717104489885\\
96.01	0.0093689767617065\\
96.02	0.00937078290381222\\
96.03	0.00937258947087621\\
96.04	0.00937439646255727\\
96.05	0.00937620387851284\\
96.06	0.00937801171839891\\
96.07	0.00937981998187011\\
96.08	0.00938162866857966\\
96.09	0.00938343777817936\\
96.1	0.00938524731031961\\
96.11	0.00938705726464938\\
96.12	0.00938886764081624\\
96.13	0.00939067843846631\\
96.14	0.00939248965724433\\
96.15	0.00939430129679355\\
96.16	0.00939611335675583\\
96.17	0.00939792583677157\\
96.18	0.00939973873647974\\
96.19	0.00940155205551786\\
96.2	0.009403365793522\\
96.21	0.00940517995012678\\
96.22	0.00940699452496536\\
96.23	0.00940880951766945\\
96.24	0.00941062492786928\\
96.25	0.00941244075519361\\
96.26	0.00941425699926975\\
96.27	0.00941607365972353\\
96.28	0.00941789073617927\\
96.29	0.00941970822825985\\
96.3	0.00942152613558663\\
96.31	0.0094233444577795\\
96.32	0.00942516319445686\\
96.33	0.00942698234523557\\
96.34	0.00942880190973106\\
96.35	0.00943062188755718\\
96.36	0.00943244227832631\\
96.37	0.00943426308164933\\
96.38	0.00943608429713557\\
96.39	0.00943790592439285\\
96.4	0.00943972796302749\\
96.41	0.00944155041264424\\
96.42	0.00944337327284635\\
96.43	0.00944519654323552\\
96.44	0.00944702022341192\\
96.45	0.00944884431297417\\
96.46	0.00945066881151935\\
96.47	0.00945249371864297\\
96.48	0.00945431903393901\\
96.49	0.00945614475699988\\
96.5	0.00945797088741644\\
96.51	0.00945979742477795\\
96.52	0.00946162436867214\\
96.53	0.00946345171868514\\
96.54	0.00946527947440154\\
96.55	0.00946710763540429\\
96.56	0.00946893620127481\\
96.57	0.0094707651715929\\
96.58	0.00947259454593678\\
96.59	0.00947442432388307\\
96.6	0.0094762545050068\\
96.61	0.00947808508888138\\
96.62	0.00947991607507863\\
96.63	0.00948174746316875\\
96.64	0.00948357925272032\\
96.65	0.00948541144330031\\
96.66	0.00948724403447406\\
96.67	0.0094890770258053\\
96.68	0.00949091041685612\\
96.69	0.00949274420718696\\
96.7	0.00949457839635666\\
96.71	0.00949641298392239\\
96.72	0.00949824796943968\\
96.73	0.00950008335246241\\
96.74	0.00950191913254283\\
96.75	0.0095037553092315\\
96.76	0.00950559188207735\\
96.77	0.00950742885062762\\
96.78	0.00950926621442791\\
96.79	0.00951110397302213\\
96.8	0.00951294212595252\\
96.81	0.00951478067275965\\
96.82	0.00951661961298239\\
96.83	0.00951845894615794\\
96.84	0.0095202986718218\\
96.85	0.0095221387895078\\
96.86	0.00952397929874804\\
96.87	0.00952582019907293\\
96.88	0.00952766149001119\\
96.89	0.00952950317108982\\
96.9	0.00953134524183412\\
96.91	0.00953318770176765\\
96.92	0.00953503055041227\\
96.93	0.00953687378728812\\
96.94	0.0095387174119136\\
96.95	0.00954056142380539\\
96.96	0.00954240582247843\\
96.97	0.00954425060744593\\
96.98	0.00954609577821935\\
96.99	0.00954794133430841\\
97	0.00954978727522108\\
97.01	0.00955163360046357\\
97.02	0.00955348030954036\\
97.03	0.00955532740195414\\
97.04	0.00955717487720585\\
97.05	0.00955902273479466\\
97.06	0.00956087097421799\\
97.07	0.00956271959497146\\
97.08	0.00956456859654891\\
97.09	0.00956641797844241\\
97.1	0.00956826774014226\\
97.11	0.00957011788113695\\
97.12	0.00957196840091317\\
97.13	0.00957381929895585\\
97.14	0.00957567057474808\\
97.15	0.00957752222777117\\
97.16	0.00957937425750463\\
97.17	0.00958122666342614\\
97.18	0.00958307944501157\\
97.19	0.00958493260173499\\
97.2	0.00958678613306863\\
97.21	0.0095886400384829\\
97.22	0.00959049431744639\\
97.23	0.00959234896942586\\
97.24	0.00959420399388621\\
97.25	0.00959605939029054\\
97.26	0.00959791515810007\\
97.27	0.00959977129677421\\
97.28	0.00960162780577049\\
97.29	0.0096034846845446\\
97.3	0.00960534193255039\\
97.31	0.00960719954923981\\
97.32	0.009609057534063\\
97.33	0.00961091588646818\\
97.34	0.00961277460590173\\
97.35	0.00961463369180817\\
97.36	0.0096164931436301\\
97.37	0.00961835296080829\\
97.38	0.00962021314278158\\
97.39	0.00962207368898695\\
97.4	0.00962393459885948\\
97.41	0.00962579587183234\\
97.42	0.0096276575073368\\
97.43	0.00962951950480218\\
97.44	0.00963138186365589\\
97.45	0.00963324458332342\\
97.46	0.00963510766322832\\
97.47	0.00963697110279221\\
97.48	0.00963883490143478\\
97.49	0.00964069905857377\\
97.5	0.00964256357362499\\
97.51	0.0096444284460103\\
97.52	0.0096462936752224\\
97.53	0.00964815926075106\\
97.54	0.00965002520208313\\
97.55	0.00965189149870251\\
97.56	0.00965375815008998\\
97.57	0.00965562515572279\\
97.58	0.00965749251507467\\
97.59	0.00965936022761576\\
97.6	0.00966122829281262\\
97.61	0.00966309671012819\\
97.62	0.00966496547902173\\
97.63	0.00966683459894884\\
97.64	0.0096687040078037\\
97.65	0.00967057345106537\\
97.66	0.00967244293111522\\
97.67	0.00967431245037442\\
97.68	0.00967618201130439\\
97.69	0.00967805161640719\\
97.7	0.00967992126822601\\
97.71	0.00968179096934558\\
97.72	0.00968366072239264\\
97.73	0.00968553053003638\\
97.74	0.00968740039498893\\
97.75	0.00968927032000576\\
97.76	0.00969113540374984\\
97.77	0.00969299287980512\\
97.78	0.00969484267370487\\
97.79	0.00969668471020148\\
97.8	0.00969851895938087\\
97.81	0.00970034559432675\\
97.82	0.00970216453691259\\
97.83	0.00970397570817427\\
97.84	0.00970577902829956\\
97.85	0.00970757441661751\\
97.86	0.00970936179158764\\
97.87	0.00971114107078894\\
97.88	0.00971291217130359\\
97.89	0.00971467500961018\\
97.9	0.00971643598000225\\
97.91	0.00971819561040886\\
97.92	0.00971995389422941\\
97.93	0.00972171082501592\\
97.94	0.00972346639638159\\
97.95	0.00972522060196552\\
97.96	0.00972697245718935\\
97.97	0.00972872025635572\\
97.98	0.00973046397766682\\
97.99	0.00973220321377041\\
98	0.00973393769619323\\
98.01	0.00973566740022747\\
98.02	0.00973739230079825\\
98.03	0.00973911237272436\\
98.04	0.00974082759071918\\
98.05	0.00974253792939164\\
98.06	0.00974424336324722\\
98.07	0.0097459458132758\\
98.08	0.00974764628114663\\
98.09	0.00974934475099875\\
98.1	0.00975104120683862\\
98.11	0.00975273563253935\\
98.12	0.00975442801183999\\
98.13	0.00975611832834473\\
98.14	0.0097578065655222\\
98.15	0.00975949270670473\\
98.16	0.00976117673508765\\
98.17	0.00976285863372854\\
98.18	0.00976453838554661\\
98.19	0.00976621597288069\\
98.2	0.00976789137782039\\
98.21	0.00976956458173835\\
98.22	0.00977123556324665\\
98.23	0.0097729043007487\\
98.24	0.00977457077243703\\
98.25	0.00977623495629106\\
98.26	0.00977789683007482\\
98.27	0.00977955637133466\\
98.28	0.00978121355739692\\
98.29	0.00978286836536555\\
98.3	0.00978452077212016\\
98.31	0.00978617075431369\\
98.32	0.00978781828836994\\
98.33	0.00978946335048191\\
98.34	0.009791105916618\\
98.35	0.00979274596251981\\
98.36	0.00979438346369993\\
98.37	0.00979601839544009\\
98.38	0.00979765073278893\\
98.39	0.00979928045055972\\
98.4	0.00980090752332797\\
98.41	0.00980253192542883\\
98.42	0.00980415363095472\\
98.43	0.00980577261375538\\
98.44	0.00980738884743632\\
98.45	0.00980900230535652\\
98.46	0.00981061296074857\\
98.47	0.00981222078723336\\
98.48	0.00981382575818507\\
98.49	0.00981542784672884\\
98.5	0.00981702702573845\\
98.51	0.00981862326783386\\
98.52	0.00982021654537887\\
98.53	0.00982180683047862\\
98.54	0.00982339409490997\\
98.55	0.00982497830958456\\
98.56	0.0098265594451396\\
98.57	0.00982813747193521\\
98.58	0.00982971236005172\\
98.59	0.00983128407928698\\
98.6	0.00983285259915361\\
98.61	0.00983441788887629\\
98.62	0.00983597991738892\\
98.63	0.00983753865333183\\
98.64	0.00983909406504894\\
98.65	0.00984064612058488\\
98.66	0.00984219478768381\\
98.67	0.00984374003378718\\
98.68	0.00984528182603079\\
98.69	0.00984682013124195\\
98.7	0.00984835491593646\\
98.71	0.00984988614631569\\
98.72	0.0098514137882636\\
98.73	0.00985293780734368\\
98.74	0.0098544581687959\\
98.75	0.00985597483753364\\
98.76	0.00985748777814056\\
98.77	0.00985899695486745\\
98.78	0.00986050233162905\\
98.79	0.00986200387200082\\
98.8	0.00986350153921575\\
98.81	0.00986499529616101\\
98.82	0.0098664851053747\\
98.83	0.00986797092904249\\
98.84	0.00986945272884565\\
98.85	0.00987093046592802\\
98.86	0.00987240410107248\\
98.87	0.00987387359469748\\
98.88	0.00987533890685353\\
98.89	0.00987679999721977\\
98.9	0.00987825682510035\\
98.91	0.00987970934942086\\
98.92	0.00988115752872478\\
98.93	0.00988260132116976\\
98.94	0.00988404068452397\\
98.95	0.00988547557616236\\
98.96	0.00988690595306295\\
98.97	0.009888331771803\\
98.98	0.00988975298855519\\
98.99	0.00989116955908376\\
99	0.00989258143874062\\
99.01	0.00989398858246139\\
99.02	0.00989539094476144\\
99.03	0.00989678847973188\\
99.04	0.00989818114103547\\
99.05	0.00989956888190258\\
99.06	0.00990095165512703\\
99.07	0.00990232941306191\\
99.08	0.00990370210761538\\
99.09	0.00990506969024643\\
99.1	0.00990643211196056\\
99.11	0.00990778932330546\\
99.12	0.00990914127436665\\
99.13	0.00991048791476302\\
99.14	0.0099118291936424\\
99.15	0.00991316505967706\\
99.16	0.00991449546105915\\
99.17	0.00991582034549609\\
99.18	0.00991713966020598\\
99.19	0.00991845335191288\\
99.2	0.00991976136684211\\
99.21	0.00992106365071546\\
99.22	0.00992236014874638\\
99.23	0.00992365080563514\\
99.24	0.00992493556556388\\
99.25	0.00992621437219167\\
99.26	0.00992748716864954\\
99.27	0.00992875389753535\\
99.28	0.00993001450090877\\
99.29	0.00993126892028608\\
99.3	0.00993251709663497\\
99.31	0.00993375897036932\\
99.32	0.00993499448134388\\
99.33	0.00993622356884892\\
99.34	0.00993744617160479\\
99.35	0.00993866222775656\\
99.36	0.0099398716748684\\
99.37	0.00994107444991809\\
99.38	0.0099422704892914\\
99.39	0.00994345972877637\\
99.4	0.00994464210355766\\
99.41	0.00994581754821071\\
99.42	0.00994698599669593\\
99.43	0.00994814738235283\\
99.44	0.00994930163789402\\
99.45	0.00995044869539926\\
99.46	0.00995158848630938\\
99.47	0.00995272094142012\\
99.48	0.00995384599087603\\
99.49	0.00995496356416414\\
99.5	0.00995607359010774\\
99.51	0.00995717599685997\\
99.52	0.00995827071189741\\
99.53	0.0099593576620136\\
99.54	0.00996043677331251\\
99.55	0.0099615079712019\\
99.56	0.00996257118038667\\
99.57	0.00996362632486211\\
99.58	0.00996467332790712\\
99.59	0.00996571211207732\\
99.6	0.00996674259919812\\
99.61	0.00996776470120001\\
99.62	0.00996877832540117\\
99.63	0.00996978337821723\\
99.64	0.00997077976515262\\
99.65	0.00997176739079183\\
99.66	0.0099727461587905\\
99.67	0.00997371597186654\\
99.68	0.00997467673179107\\
99.69	0.00997562833937934\\
99.7	0.00997657069448153\\
99.71	0.00997750369597346\\
99.72	0.00997842724174722\\
99.73	0.0099793412287017\\
99.74	0.00998024555273308\\
99.75	0.00998114010872511\\
99.76	0.00998202479053945\\
99.77	0.00998289949100579\\
99.78	0.00998376410191195\\
99.79	0.00998461851399383\\
99.8	0.00998546261692534\\
99.81	0.00998629629930815\\
99.82	0.0099871194486614\\
99.83	0.00998793195141125\\
99.84	0.00998873369288044\\
99.85	0.00998952455727761\\
99.86	0.00999030442768662\\
99.87	0.00999107318605574\\
99.88	0.0099918307131867\\
99.89	0.00999257688872371\\
99.9	0.00999331159114228\\
99.91	0.00999403469773802\\
99.92	0.00999474608461527\\
99.93	0.00999544562667568\\
99.94	0.00999613319760659\\
99.95	0.00999680866986942\\
99.96	0.00999747191468782\\
99.97	0.00999812280203583\\
99.98	0.00999876120062581\\
99.99	0.00999938697789635\\
100	0.01\\
};
\addlegendentry{$q=4$};

\end{axis}
\end{tikzpicture}% 
  \caption{Continuous Time w/ nFPC}
\end{subfigure}%
\hfill%
\begin{subfigure}{.45\linewidth}
  \centering
  \setlength\figureheight{\linewidth} 
  \setlength\figurewidth{\linewidth}
  \tikzsetnextfilename{dp_dscr_nFPC_z8}
  % This file was created by matlab2tikz.
%
%The latest updates can be retrieved from
%  http://www.mathworks.com/matlabcentral/fileexchange/22022-matlab2tikz-matlab2tikz
%where you can also make suggestions and rate matlab2tikz.
%
\definecolor{mycolor1}{rgb}{1.00000,0.00000,1.00000}%
%
\begin{tikzpicture}[trim axis left, trim axis right]

\begin{axis}[%
width=\figurewidth,
height=\figureheight,
at={(0\figurewidth,0\figureheight)},
scale only axis,
every outer x axis line/.append style={black},
every x tick label/.append style={font=\color{black}},
xmin=0,
xmax=100,
%xlabel={Time},
every outer y axis line/.append style={black},
every y tick label/.append style={font=\color{black}},
ymin=0,
ymax=0.015,
%ylabel={Depth $\delta^+$},
axis background/.style={fill=white},
axis x line*=bottom,
axis y line*=left,
yticklabel style={
        /pgf/number format/fixed,
        /pgf/number format/precision=3
},
scaled y ticks=false,
legend style={legend cell align=left,align=left,draw=black,font=\footnotesize, at={(0.98,0.02)},anchor=south east},
every axis legend/.code={\renewcommand\addlegendentry[2][]{}}  %ignore legend locally
]
\addplot [color=green,dashed]
  table[row sep=crcr]{%
1	0.00960747766527148\\
2	0.00960459404913772\\
3	0.00960158025650961\\
4	0.0095984295068883\\
5	0.0095951345600636\\
6	0.00959168766534378\\
7	0.00958808050086004\\
8	0.00958430410026429\\
9	0.00958034876332209\\
10	0.00957620394581127\\
11	0.00957185812265307\\
12	0.00956729861615749\\
13	0.00956251137851434\\
14	0.00955748071468751\\
15	0.00955218893264844\\
16	0.00954661593090592\\
17	0.0095407388453902\\
18	0.0095345309976458\\
19	0.00952796000285703\\
20	0.00952098564345573\\
21	0.00951355270940135\\
22	0.00950556094064703\\
23	0.00946440543532272\\
24	0.00940906811304978\\
25	0.00935104996681104\\
26	0.00929018477044053\\
27	0.00922629325960029\\
28	0.00915918211537875\\
29	0.00908864275242587\\
30	0.00901444544869592\\
31	0.00893634415218274\\
32	0.0088540786026562\\
33	0.00876737337954217\\
34	0.00867594157014046\\
35	0.00857949305323717\\
36	0.0084777530018618\\
37	0.00837050592739933\\
38	0.00825987440296975\\
39	0.00814615295278687\\
40	0.00802630605978886\\
41	0.00790008363139868\\
42	0.00776735839791016\\
43	0.00762796148881317\\
44	0.00748124346793057\\
45	0.00732643878621186\\
46	0.00716259451917534\\
47	0.00698855139815732\\
48	0.0068028678544128\\
49	0.00660392300924533\\
50	0.00638983923916324\\
51	0.00615832760230671\\
52	0.00590662677333882\\
53	0.00563137494727097\\
54	0.0054833445843688\\
55	0.00531886987268333\\
56	0.00513317052593558\\
57	0.00494020700596234\\
58	0.00474015268939363\\
59	0.00453339226580794\\
60	0.00432059614673591\\
61	0.00410280954628816\\
62	0.00388155907365784\\
63	0.00365897879302804\\
64	0.00343813287075068\\
65	0.00322358765361566\\
66	0.00302187850555284\\
67	0.00284255784210886\\
68	0.00266514204955049\\
69	0.0024962160076035\\
70	0.00234029063557502\\
71	0.00219095723548395\\
72	0.00204945378603438\\
73	0.00191672431015574\\
74	0.00179269703825858\\
75	0.00167417850208544\\
76	0.00155990138313519\\
77	0.00144776499195021\\
78	0.00133806332426327\\
79	0.00123085736165786\\
80	0.00112590794744774\\
81	0.001022807448086\\
82	0.000921068875316224\\
83	0.000819624925291739\\
84	0.000721158445062816\\
85	0.000627458040754679\\
86	0.000539433042325591\\
87	0.000458553981564615\\
88	0.000386070896422318\\
89	0.000319981903864034\\
90	0.000261223634796461\\
91	0.000208334376922877\\
92	0.000159855757659064\\
93	0.00011559139109104\\
94	7.58928809352733e-05\\
95	4.17334792893222e-05\\
96	1.50512561189767e-05\\
97	0\\
98	0\\
99	0\\
100	0\\
};
\addlegendentry{$q=-4$};

\addplot [color=mycolor1,dashed]
  table[row sep=crcr]{%
1	0.00983929166573282\\
2	0.00983906428205305\\
3	0.00983882586633388\\
4	0.00983857578554717\\
5	0.00983831335893858\\
6	0.00983803785259887\\
7	0.00983774847306906\\
8	0.00983744435972472\\
9	0.00983712457558571\\
10	0.0098367880960489\\
11	0.00983643379482422\\
12	0.00983606042608525\\
13	0.00983566660167668\\
14	0.00983525076264436\\
15	0.00983481114598505\\
16	0.00983434574651927\\
17	0.00983385223240286\\
18	0.00983332773461626\\
19	0.0098327684691146\\
20	0.00983216893471232\\
21	0.00983152027173897\\
22	0.00983080876757299\\
23	0.00982768744363728\\
24	0.0098235670630728\\
25	0.00981928738838518\\
26	0.00981484115706208\\
27	0.00981022092880901\\
28	0.00980541977752457\\
29	0.00980043371283091\\
30	0.00979538939892645\\
31	0.00979023857514771\\
32	0.00978489104778528\\
33	0.0097793379880434\\
34	0.00977356953153074\\
35	0.00976757324661039\\
36	0.00976132801915651\\
37	0.0097547784149741\\
38	0.00974576309461378\\
39	0.00973379495395621\\
40	0.00972139712345876\\
41	0.00970856162990793\\
42	0.00969528093924533\\
43	0.00968152053608604\\
44	0.00966724054970232\\
45	0.0096523911529444\\
46	0.00963690992005741\\
47	0.00962071712702947\\
48	0.00960372260376675\\
49	0.00958582096415723\\
50	0.00956687303578892\\
51	0.00954673634763493\\
52	0.00952522247386327\\
53	0.00950199461908947\\
54	0.00933208601272526\\
55	0.00914906495911794\\
56	0.00895583024907716\\
57	0.00875173338744622\\
58	0.00853551301503994\\
59	0.0083056635024672\\
60	0.00806037526506734\\
61	0.00779745918206924\\
62	0.00751425052909292\\
63	0.00720750982711353\\
64	0.00687319695061563\\
65	0.00650633237721363\\
66	0.00610649935007796\\
67	0.00567132953542143\\
68	0.00521149599924193\\
69	0.0047237781675188\\
70	0.00420485594144063\\
71	0.00392783165642717\\
72	0.00365005661796192\\
73	0.00337011298152657\\
74	0.00309332054940902\\
75	0.0028289491960103\\
76	0.00258588780578947\\
77	0.00237580659121768\\
78	0.00217020600928596\\
79	0.00197107065433464\\
80	0.00178060061526567\\
81	0.00160031142738287\\
82	0.00143198780801362\\
83	0.00128088177894017\\
84	0.00114016131158019\\
85	0.00100480900001626\\
86	0.0008729626779357\\
87	0.000742732348917191\\
88	0.000615967170454213\\
89	0.000496927150175562\\
90	0.000387280994571957\\
91	0.000290236695875495\\
92	0.00020881406197963\\
93	0.000141373809608583\\
94	8.68845319872729e-05\\
95	4.45546390724813e-05\\
96	1.50512561189767e-05\\
97	0\\
98	0\\
99	0\\
100	0\\
};
\addlegendentry{$q=-3$};

\addplot [color=red,dashed]
  table[row sep=crcr]{%
1	0.0098806142486084\\
2	0.00988060225975636\\
3	0.00988058968056293\\
4	0.00988057647644751\\
5	0.00988056261020699\\
6	0.00988054804174608\\
7	0.00988053272776883\\
8	0.00988051662141958\\
9	0.00988049967185485\\
10	0.00988048182371878\\
11	0.00988046301648793\\
12	0.00988044318365599\\
13	0.00988042225174318\\
14	0.00988040013902537\\
15	0.00988037675313954\\
16	0.00988035198410419\\
17	0.00988032568714141\\
18	0.0098802976502558\\
19	0.00988026753995375\\
20	0.00988023484962275\\
21	0.00988019900984474\\
22	0.00988016003052746\\
23	0.00988011864542674\\
24	0.00988007495356235\\
25	0.00988002870760913\\
26	0.0098799795577269\\
27	0.0098799268746262\\
28	0.00987986912736902\\
29	0.00987980162433733\\
30	0.00987960134057879\\
31	0.0098793057209037\\
32	0.00987898875891997\\
33	0.00987864861229315\\
34	0.00987828292615525\\
35	0.00987788794115731\\
36	0.00987745644955523\\
37	0.00987697534178638\\
38	0.0098762849839572\\
39	0.00987536352430184\\
40	0.00987440428355443\\
41	0.00987340481784767\\
42	0.00987236224504076\\
43	0.00987127335846734\\
44	0.00987013439139508\\
45	0.0098689406252867\\
46	0.00986768605948119\\
47	0.00986636329663847\\
48	0.00986496346417182\\
49	0.00986347485171751\\
50	0.00986188376801104\\
51	0.00986017258243903\\
52	0.00985831192027749\\
53	0.00985625516009128\\
54	0.00984384693209673\\
55	0.00983095122915345\\
56	0.00981783065784758\\
57	0.0098044722867266\\
58	0.00979086099535044\\
59	0.00977697937044168\\
60	0.00976280785852619\\
61	0.00974832571042725\\
62	0.00973351469664937\\
63	0.00971834670880941\\
64	0.00970274693435033\\
65	0.00968658058077449\\
66	0.0096649363114085\\
67	0.00963313382503779\\
68	0.00960054339029398\\
69	0.0095668477180941\\
70	0.00953149571949198\\
71	0.0092346191537598\\
72	0.00891070222019736\\
73	0.00856060036876301\\
74	0.00817969621194866\\
75	0.00776210868714419\\
76	0.0073004439043749\\
77	0.00678579240107872\\
78	0.00624625223967729\\
79	0.00568168757145003\\
80	0.00508776463775421\\
81	0.00446132423589887\\
82	0.0038015365938485\\
83	0.00311634866014581\\
84	0.00268381681743506\\
85	0.00240021227450085\\
86	0.00214531516129943\\
87	0.00191176180069247\\
88	0.00167818912809278\\
89	0.00144282299053618\\
90	0.0012074225550966\\
91	0.000974345149752143\\
92	0.000746407645706487\\
93	0.000531597859427775\\
94	0.000338443796331356\\
95	0.000176929185035761\\
96	5.89679114988392e-05\\
97	0\\
98	0\\
99	0\\
100	0\\
};
\addlegendentry{$q=-2$};

\addplot [color=blue,dashed]
  table[row sep=crcr]{%
1	0.00993301187418029\\
2	0.00993301123058148\\
3	0.00993301055507122\\
4	0.00993300984572929\\
5	0.00993300910047486\\
6	0.00993300831704409\\
7	0.00993300749296184\\
8	0.00993300662550481\\
9	0.00993300571165251\\
10	0.00993300474802055\\
11	0.00993300373076705\\
12	0.00993300265545041\\
13	0.00993300151677897\\
14	0.00993300030810113\\
15	0.00993299902034793\\
16	0.00993299764017695\\
17	0.00993299614749676\\
18	0.00993299451360253\\
19	0.00993299270435077\\
20	0.00993299069697789\\
21	0.00993298850991085\\
22	0.00993298618645423\\
23	0.00993298372283397\\
24	0.00993298108128142\\
25	0.0099329781843993\\
26	0.00993297486598587\\
27	0.00993297077181648\\
28	0.00993296519982485\\
29	0.00993295703059585\\
30	0.00993293727940716\\
31	0.00993290960977507\\
32	0.00993287982097473\\
33	0.00993284759829078\\
34	0.00993281245654904\\
35	0.00993277368265537\\
36	0.00993273050356053\\
37	0.00993268292431627\\
38	0.00993263231959259\\
39	0.00993257974798528\\
40	0.0099325251006028\\
41	0.00993246825507844\\
42	0.00993240907926109\\
43	0.00993234741822372\\
44	0.00993228307369699\\
45	0.00993221578663254\\
46	0.0099321452204967\\
47	0.00993207092098236\\
48	0.00993199220833431\\
49	0.00993190816985955\\
50	0.00993181762426275\\
51	0.00993171887092656\\
52	0.00993161029110649\\
53	0.00993149279918156\\
54	0.00993136949478849\\
55	0.00993123998481406\\
56	0.00993110352070128\\
57	0.0099309592013298\\
58	0.00993080592161378\\
59	0.00993064227783493\\
60	0.00993046635252949\\
61	0.00993027511840973\\
62	0.00993006254669058\\
63	0.00992981553424646\\
64	0.00992950928850722\\
65	0.00992910357465447\\
66	0.00992823294696101\\
67	0.00992659488509032\\
68	0.009924809579626\\
69	0.00992282442090743\\
70	0.00992056415130438\\
71	0.00990028399773245\\
72	0.00987949004951769\\
73	0.00985850446534031\\
74	0.00983732510932302\\
75	0.00981594960866513\\
76	0.00979440742552927\\
77	0.00977290389215302\\
78	0.00975271679676563\\
79	0.0097324172717149\\
80	0.00971343084399797\\
81	0.00969563918808831\\
82	0.0096786318169564\\
83	0.00965088127755664\\
84	0.0093466932911918\\
85	0.00887160903576858\\
86	0.00834584872617862\\
87	0.00777892140166473\\
88	0.00719204555733556\\
89	0.00658661882228475\\
90	0.0059608239085164\\
91	0.00531283826600714\\
92	0.00464134333654843\\
93	0.00394433142034477\\
94	0.00321894058908532\\
95	0.00246108721279032\\
96	0.00166403223161984\\
97	0.000826308254754571\\
98	0\\
99	0\\
100	0\\
};
\addlegendentry{$q=-1$};

\addplot [color=black,solid]
  table[row sep=crcr]{%
1	0.000624407485134332\\
2	0.000624407519915784\\
3	0.000624407556323891\\
4	0.000624407594375936\\
5	0.000624407634189318\\
6	0.000624407676445276\\
7	0.000624407722593159\\
8	0.000624407775541716\\
9	0.000624407840924937\\
10	0.00062440792887363\\
11	0.000624408055325559\\
12	0.000624408239521597\\
13	0.000624408491433687\\
14	0.000624408789812747\\
15	0.000624409095760126\\
16	0.000624409409680003\\
17	0.000624409732843658\\
18	0.000624410069180706\\
19	0.000624410429732784\\
20	0.000624410841680389\\
21	0.000624411360596894\\
22	0.000624412070382347\\
23	0.000624413069156528\\
24	0.000624414375010669\\
25	0.000624415729590759\\
26	0.000624417088696578\\
27	0.00062441841243596\\
28	0.000624419623701465\\
29	0.000624420622615931\\
30	0.000624421521762283\\
31	0.000624422411946123\\
32	0.000624423272305066\\
33	0.000624424076334788\\
34	0.000624424781981793\\
35	0.00062442532946859\\
36	0.00062442567735886\\
37	0.000624425919786399\\
38	0.00062442618043771\\
39	0.000624426466438152\\
40	0.000624426790513287\\
41	0.000624427179565305\\
42	0.000624427691434862\\
43	0.00062442844809704\\
44	0.00062442969779383\\
45	0.000624431903854089\\
46	0.000624435788265885\\
47	0.000624442007741635\\
48	0.000624449953346516\\
49	0.000624458161669218\\
50	0.000624466383979973\\
51	0.000624474468212722\\
52	0.000624482454241307\\
53	0.000624490887043724\\
54	0.000624500168306528\\
55	0.000624511205196221\\
56	0.000624526087744902\\
57	0.000624548653900136\\
58	0.000624583366510348\\
59	0.000624623327589464\\
60	0.000624666043064573\\
61	0.000624712664097867\\
62	0.000624764543152344\\
63	0.000624821910457352\\
64	0.000624880962371869\\
65	0.000624940624648512\\
66	0.000625011307902986\\
67	0.000625118049124955\\
68	0.000625283759630578\\
69	0.000625453729813828\\
70	0.000625622432522341\\
71	0.000625805723034884\\
72	0.000626010329493192\\
73	0.000626247898949499\\
74	0.0006265436206454\\
75	0.000626935651106565\\
76	0.000627363418305458\\
77	0.00062782784435962\\
78	0.000628335442994521\\
79	0.000628772360726379\\
80	0.000631557283951244\\
81	0.000634607927527444\\
82	0.000637934956761122\\
83	0.000644443068343529\\
84	0.000641738827914312\\
85	0.000624087409766743\\
86	0.000925253858090135\\
87	0.00138013119776012\\
88	0.00189154071321051\\
89	0.00246991304304022\\
90	0.003086718548447\\
91	0.00374453088362059\\
92	0.00444193080109917\\
93	0.00518263306050222\\
94	0.00597107521850114\\
95	0.00681235368486026\\
96	0.00771339553901169\\
97	0.008683018792965\\
98	0.00969870132858304\\
99	0\\
100	0\\
};
\addlegendentry{$q=0$};

\addplot [color=blue,solid]
  table[row sep=crcr]{%
1	8.38301970140858e-06\\
2	8.38386528491081e-06\\
3	8.38474919953782e-06\\
4	8.38567603025214e-06\\
5	8.38664698922755e-06\\
6	8.38766183294111e-06\\
7	8.38872497039531e-06\\
8	8.38984481688585e-06\\
9	8.39103758840753e-06\\
10	8.39233737740437e-06\\
11	8.39381703207104e-06\\
12	8.39562374315354e-06\\
13	8.39801176035047e-06\\
14	8.40126793308153e-06\\
15	8.40528131334117e-06\\
16	8.40943197882418e-06\\
17	8.41372843142993e-06\\
18	8.41818362380834e-06\\
19	8.4228221392232e-06\\
20	8.42769923837403e-06\\
21	8.43294963500891e-06\\
22	8.43890477407434e-06\\
23	8.44634055047764e-06\\
24	8.45683716744612e-06\\
25	8.47241716854141e-06\\
26	8.48895030383199e-06\\
27	8.50600959947571e-06\\
28	8.52363200865932e-06\\
29	8.54186696005317e-06\\
30	8.56077754448614e-06\\
31	8.58042681464239e-06\\
32	8.60083196266657e-06\\
33	8.62190963800097e-06\\
34	8.64365282864932e-06\\
35	8.66610622399652e-06\\
36	8.68935615797893e-06\\
37	8.71350385225643e-06\\
38	8.73861334423382e-06\\
39	8.7647580021751e-06\\
40	8.79202424669478e-06\\
41	8.82052043243899e-06\\
42	8.85040206576697e-06\\
43	8.88194627764206e-06\\
44	8.91574836941675e-06\\
45	8.95321175932681e-06\\
46	8.99775501488281e-06\\
47	9.05734617107499e-06\\
48	9.1463626579966e-06\\
49	9.26397206883214e-06\\
50	9.38846191829702e-06\\
51	9.51773126786104e-06\\
52	9.65116243962592e-06\\
53	9.78910187031467e-06\\
54	9.93198223680158e-06\\
55	1.00805294362259e-05\\
56	1.02368395141276e-05\\
57	1.04121887200929e-05\\
58	1.06447679419695e-05\\
59	1.19882924727538e-05\\
60	1.38069458053338e-05\\
61	1.57168804265525e-05\\
62	1.77344791835758e-05\\
63	1.98883795848268e-05\\
64	2.22201228382913e-05\\
65	2.47018838786381e-05\\
66	2.73319082886247e-05\\
67	3.01694621578692e-05\\
68	3.3406610600724e-05\\
69	5.54696738036023e-05\\
70	8.02426306743528e-05\\
71	0.00010655232745994\\
72	0.000134652598962992\\
73	0.000164869792889474\\
74	0.000197594714388067\\
75	0.000233467384385008\\
76	0.000536437647948485\\
77	0.000956998611238995\\
78	0.00141178194827107\\
79	0.0019069362149086\\
80	0.00244786603977846\\
81	0.00304657142035237\\
82	0.00370195182948479\\
83	0.00439447281833135\\
84	0.00510567944503527\\
85	0.00586596287169202\\
86	0.00634543106950584\\
87	0.00670970308024902\\
88	0.00705895729635169\\
89	0.00738110433715275\\
90	0.00770609030628117\\
91	0.00803174778032167\\
92	0.00836043655066926\\
93	0.00868876266121125\\
94	0.00901176881358957\\
95	0.00932269995257548\\
96	0.00959732068383334\\
97	0.00981217566362703\\
98	0.0099605845957028\\
99	0\\
100	0\\
};
\addlegendentry{$q=1$};

\addplot [color=red,solid]
  table[row sep=crcr]{%
1	1.14758002926896e-05\\
2	1.1485015126246e-05\\
3	1.14946003885409e-05\\
4	1.15046027019851e-05\\
5	1.15150925076826e-05\\
6	1.15260904882415e-05\\
7	1.15375732305172e-05\\
8	1.15495712508768e-05\\
9	1.15621408913412e-05\\
10	1.15753770190491e-05\\
11	1.15894757738868e-05\\
12	1.16049041034829e-05\\
13	1.16228263433142e-05\\
14	1.16459995832758e-05\\
15	1.16796108326478e-05\\
16	1.17248179954996e-05\\
17	1.17715582509505e-05\\
18	1.18199028602877e-05\\
19	1.18699318496234e-05\\
20	1.19217488708434e-05\\
21	1.19755343670989e-05\\
22	1.20317336984719e-05\\
23	1.20917222287143e-05\\
24	1.21601690212965e-05\\
25	1.22535150957733e-05\\
26	1.28965866048559e-05\\
27	1.36176607243976e-05\\
28	1.43624871419559e-05\\
29	1.51323807492452e-05\\
30	1.59288977496959e-05\\
31	1.67539351428091e-05\\
32	1.7609743507198e-05\\
33	1.84984767666977e-05\\
34	1.94205448101078e-05\\
35	2.0377571158722e-05\\
36	2.13720503234054e-05\\
37	2.24073356341282e-05\\
38	2.34873335793314e-05\\
39	2.46152790267093e-05\\
40	2.57947704444773e-05\\
41	2.70298230433428e-05\\
42	2.83249283013289e-05\\
43	2.96851392068227e-05\\
44	3.11163696085861e-05\\
45	3.26262856714941e-05\\
46	3.42265936580611e-05\\
47	3.59409406857809e-05\\
48	3.78388750768602e-05\\
49	4.42477420479792e-05\\
50	5.85564078128645e-05\\
51	7.35100378273756e-05\\
52	8.9085107278434e-05\\
53	0.000105307590632848\\
54	0.000122235832780331\\
55	0.000139936031545904\\
56	0.00015848122500173\\
57	0.000177928864974253\\
58	0.000198357861165649\\
59	0.000218875527543803\\
60	0.000240239807952066\\
61	0.000263020365504108\\
62	0.000287455255416057\\
63	0.000313870836040574\\
64	0.000342793087038499\\
65	0.000559823947389461\\
66	0.000876468875381666\\
67	0.00121239452801232\\
68	0.0015698556027177\\
69	0.00193310088885079\\
70	0.00232193523002754\\
71	0.00274285791819601\\
72	0.00320228465389986\\
73	0.00370803047155976\\
74	0.00427011241454861\\
75	0.0048797154565817\\
76	0.00526184613590384\\
77	0.005562692150318\\
78	0.00586621833219415\\
79	0.00616877405175283\\
80	0.00646496155186\\
81	0.00674500915762293\\
82	0.00700933776719489\\
83	0.0072682264478649\\
84	0.00751750096070531\\
85	0.00775141758586106\\
86	0.00796379421083901\\
87	0.00816690346257312\\
88	0.00836677804294878\\
89	0.0085649295074448\\
90	0.00876160044559847\\
91	0.00895579290222694\\
92	0.00914691133703123\\
93	0.00933189662901577\\
94	0.00949682983647625\\
95	0.00963734641541312\\
96	0.00976239223053009\\
97	0.00987237869677138\\
98	0.0099605845957028\\
99	0\\
100	0\\
};
\addlegendentry{$q=2$};

\addplot [color=mycolor1,solid]
  table[row sep=crcr]{%
1	3.15734924100943e-05\\
2	3.19865904928091e-05\\
3	3.24164972977663e-05\\
4	3.28644115877258e-05\\
5	3.33318347709687e-05\\
6	3.38210776111889e-05\\
7	3.43337556653169e-05\\
8	3.48706723780252e-05\\
9	3.54335202355788e-05\\
10	3.60243831190674e-05\\
11	3.66456552234368e-05\\
12	3.73002526610017e-05\\
13	3.79923360249596e-05\\
14	3.87300616630171e-05\\
15	3.95366524188967e-05\\
16	4.16909625018281e-05\\
17	4.73360540906365e-05\\
18	5.31918706648171e-05\\
19	5.92689338736221e-05\\
20	6.5578146279307e-05\\
21	7.21305802997466e-05\\
22	7.89370078170671e-05\\
23	8.60067579608657e-05\\
24	9.33446543655605e-05\\
25	0.000100942040821112\\
26	0.000108282876571358\\
27	0.000115838622045322\\
28	0.000123676776113569\\
29	0.000131811281652549\\
30	0.000140256357601115\\
31	0.000149026368452964\\
32	0.000158136102587742\\
33	0.000167601383159383\\
34	0.000177437489465774\\
35	0.000187616302660993\\
36	0.000198142355171025\\
37	0.000209037467129171\\
38	0.000220338401675062\\
39	0.000232098041353089\\
40	0.000244353897257368\\
41	0.000257150171801934\\
42	0.000270539503325142\\
43	0.0002845851175605\\
44	0.000299363033194179\\
45	0.00031496580268087\\
46	0.000331508734095055\\
47	0.000349128165644788\\
48	0.000367942389534666\\
49	0.000383822961085242\\
50	0.000393478487088462\\
51	0.000404331818699767\\
52	0.000553940802233106\\
53	0.000762950510103654\\
54	0.000982157755706503\\
55	0.00121243359747923\\
56	0.00145473340171157\\
57	0.00171010916403584\\
58	0.00197966374140053\\
59	0.00226489522249555\\
60	0.00256833340229969\\
61	0.00289295173037276\\
62	0.00324228006131558\\
63	0.00362074196416667\\
64	0.00403387256464842\\
65	0.00429971016883075\\
66	0.00450850260116754\\
67	0.00474099695827202\\
68	0.00497833720054123\\
69	0.00521981265860384\\
70	0.00546375290856427\\
71	0.00570699553842181\\
72	0.00594501967040718\\
73	0.0061718155284326\\
74	0.00637715129575237\\
75	0.00656949614617233\\
76	0.0067545694498105\\
77	0.00693868354532434\\
78	0.00712335168451623\\
79	0.0073021161567019\\
80	0.00747271297532068\\
81	0.00763496642039432\\
82	0.0077899123949703\\
83	0.00794343262583044\\
84	0.00809580387407136\\
85	0.00824734970254782\\
86	0.0083990715623843\\
87	0.00855093187949473\\
88	0.00870315907223708\\
89	0.00885591566908325\\
90	0.00900993787050331\\
91	0.00916203308305026\\
92	0.00929991832107232\\
93	0.00942260939006479\\
94	0.0095405746029741\\
95	0.00965530092373354\\
96	0.00976697658027199\\
97	0.00987237869677138\\
98	0.0099605845957028\\
99	0\\
100	0\\
};
\addlegendentry{$q=3$};

\addplot [color=green,solid]
  table[row sep=crcr]{%
1	0.000228542668731224\\
2	0.000232779694508534\\
3	0.00023717816446103\\
4	0.000241746132465214\\
5	0.000246491289743564\\
6	0.000251420184496261\\
7	0.000256556459694019\\
8	0.00026190852277929\\
9	0.000267461596073126\\
10	0.000273223515831327\\
11	0.000279208745206394\\
12	0.000285434038114318\\
13	0.000291918294102623\\
14	0.000298680407599511\\
15	0.000305728994600443\\
16	0.000311815361374372\\
17	0.000314777690602715\\
18	0.000317823017361306\\
19	0.000320954144918255\\
20	0.000324174561448557\\
21	0.000327488760880062\\
22	0.000330902689693013\\
23	0.000334424364837541\\
24	0.000338064754865845\\
25	0.000341839258433608\\
26	0.000345771789773533\\
27	0.00034989830721003\\
28	0.000354247312078822\\
29	0.000358853124746333\\
30	0.000363759219945719\\
31	0.000369024275825289\\
32	0.000374729625351693\\
33	0.000380997866611204\\
34	0.000388033296742956\\
35	0.000485321635284121\\
36	0.000616029880551078\\
37	0.000751775060026542\\
38	0.000892878815618323\\
39	0.00103969049316254\\
40	0.00119260897081258\\
41	0.00135207793897538\\
42	0.00151859214896133\\
43	0.00169270456503266\\
44	0.00187503453406004\\
45	0.00206627431025713\\
46	0.00226721718091069\\
47	0.0024788172266664\\
48	0.00270217589477159\\
49	0.00293840034966602\\
50	0.00318908990816527\\
51	0.00345666270969956\\
52	0.00360442739682428\\
53	0.00371039619260692\\
54	0.00382486740183529\\
55	0.00394956492797292\\
56	0.00408672402413882\\
57	0.00423926087323422\\
58	0.0044109704137562\\
59	0.00459978889027843\\
60	0.00479146711718958\\
61	0.00498447850355443\\
62	0.00517655272898267\\
63	0.00536436229735506\\
64	0.00554307376960064\\
65	0.0057067246748794\\
66	0.00585530152151234\\
67	0.00598961802059368\\
68	0.0061265623487264\\
69	0.00626559643540984\\
70	0.00640605001134091\\
71	0.00654723068619584\\
72	0.00668870059954689\\
73	0.0068303531780872\\
74	0.00697264355557257\\
75	0.00711372115145026\\
76	0.0072499961052648\\
77	0.00738079132430047\\
78	0.00750570010018372\\
79	0.00762942661937307\\
80	0.00775353102018371\\
81	0.007878736918946\\
82	0.00800577213521258\\
83	0.00813474430415688\\
84	0.00826523623162049\\
85	0.00839748989621356\\
86	0.00853180785227634\\
87	0.00866903106140667\\
88	0.00880963575522944\\
89	0.00895231557028624\\
90	0.00908219203028829\\
91	0.00919991852903407\\
92	0.00931494353133929\\
93	0.00942873570002179\\
94	0.0095424200554176\\
95	0.0096556491500428\\
96	0.00976697658027199\\
97	0.00987237869677138\\
98	0.0099605845957028\\
99	0\\
100	0\\
};
\addlegendentry{$q=4$};

\end{axis}
\end{tikzpicture}%
 
  \caption{Discrete Time w/ nFPC}
\end{subfigure}\\

\leavevmode\smash{\makebox[0pt]{\hspace{-7em}% HORIZONTAL POSITION           
  \rotatebox[origin=l]{90}{\hspace{20em}% VERTICAL POSITION
    Depth $\delta^+$}%
}}\hspace{0pt plus 1filll}\null

Time (s)

\vspace{1cm}
\begin{subfigure}{\linewidth}
  \centering
  \tikzsetnextfilename{deltalegend}
  \documentclass{article}
\usepackage{pgfplots}
\usetikzlibrary{backgrounds}
\pgfplotsset{compat=newest}  
\newlength\figureheight 
\newlength\figurewidth 

\begin{document}
%
%\begin{figure}
%  \centering
%  \setlength\figureheight{\linewidth} 
%  \setlength\figurewidth{\linewidth}
%  \input{/home/anton/Documents/masc/ml/thesis/tikz/ORCL_comp4stoch.tikz}
%  \caption{Backtest strategy comparison}
%  \label{fig:insample}
%\end{figure}
\definecolor{mycolor1}{rgb}{1.00000,0.00000,1.00000}%
\begin{tikzpicture}[framed]
    \begingroup
    % inits/clears the lists (which might be populated from previous
    % axes):
    \csname pgfplots@init@cleared@structures\endcsname
    \pgfplotsset{legend style={at={(0,1)},anchor=north west},legend columns=-1,legend style={draw=none,column sep=1ex},legend entries={$q=-4$,$q=-3$,$q=-2$,$q=-1$}}%
    
    \csname pgfplots@addlegendimage\endcsname{thick,green,dashed,sharp plot}
    \csname pgfplots@addlegendimage\endcsname{thick,mycolor1,dashed,sharp plot}
    \csname pgfplots@addlegendimage\endcsname{thick,red,dashed,sharp plot}
    \csname pgfplots@addlegendimage\endcsname{thick,blue,dashed,sharp plot}

    % draws the legend:
    \csname pgfplots@createlegend\endcsname
    \endgroup

    \begingroup
    % inits/clears the lists (which might be populated from previous
    % axes):
    \csname pgfplots@init@cleared@structures\endcsname
    \pgfplotsset{legend style={at={(3.45,0.5)},anchor=north west},legend columns=-1,legend style={draw=none,column sep=1ex},legend entries={$q=0$}}%

    \csname pgfplots@addlegendimage\endcsname{thick,black,sharp plot}

    % draws the legend:
    \csname pgfplots@createlegend\endcsname
    \endgroup

    \begingroup
    % inits/clears the lists (which might be populated from previous
    % axes):
    \csname pgfplots@init@cleared@structures\endcsname
    \pgfplotsset{legend style={at={(0,0)},anchor=north west},legend columns=-1,legend style={draw=none,column sep=1ex},legend entries={$q=+4$,$q=+3$,$q=+2$,$q=+1$}}%
    
    \csname pgfplots@addlegendimage\endcsname{thick,green,sharp plot}
    \csname pgfplots@addlegendimage\endcsname{thick,mycolor1,sharp plot}
    \csname pgfplots@addlegendimage\endcsname{thick,red,sharp plot}
    \csname pgfplots@addlegendimage\endcsname{thick,blue,sharp plot}

    % draws the legend:
    \csname pgfplots@createlegend\endcsname
    \endgroup
\end{tikzpicture}

\end{document} 
\end{subfigure}%
  \caption{Optimal buy depths $\delta^+$ for Markov state $Z=(\rho = 0, \Delta S = 0)$, implying neutral imbalance and no previous price change. We expect no change in midprice.}
  \label{fig:comp_dp_z8}
\end{figure}

\begin{figure}
\centering
\begin{subfigure}{.45\linewidth}
  \centering
  \setlength\figureheight{\linewidth} 
  \setlength\figurewidth{\linewidth}
  \tikzsetnextfilename{dp_cts_z15}
  % This file was created by matlab2tikz.
%
%The latest updates can be retrieved from
%  http://www.mathworks.com/matlabcentral/fileexchange/22022-matlab2tikz-matlab2tikz
%where you can also make suggestions and rate matlab2tikz.
%
\definecolor{mycolor1}{rgb}{1.00000,0.00000,1.00000}%
%
\begin{tikzpicture}[trim axis left, trim axis right]

\begin{axis}[%
width=\figurewidth,
height=\figureheight,
at={(0\figurewidth,0\figureheight)},
scale only axis,
every outer x axis line/.append style={black},
every x tick label/.append style={font=\color{black}},
xmin=0,
xmax=100,
%xlabel={Time},
every outer y axis line/.append style={black},
every y tick label/.append style={font=\color{black}},
ymin=0,
ymax=0.015,
%ylabel={Depth $\delta^+$},
axis background/.style={fill=white},
axis x line*=bottom,
axis y line*=left,
yticklabel style={
        /pgf/number format/fixed,
        /pgf/number format/precision=3
},
scaled y ticks=false,
legend style={legend cell align=left,align=left,draw=black,font=\footnotesize, at={(0.98,0.02)},anchor=south east},
every axis legend/.code={\renewcommand\addlegendentry[2][]{}}  %ignore legend locally
]
\addplot [color=green,dashed]
  table[row sep=crcr]{%
0.01	0.01\\
1.01	0.01\\
2.01	0.01\\
3.01	0.01\\
4.01	0.01\\
5.01	0.01\\
6.01	0.01\\
7.01	0.01\\
8.01	0.01\\
9.01	0.01\\
10.01	0.01\\
11.01	0.01\\
12.01	0.01\\
13.01	0.01\\
14.01	0.01\\
15.01	0.01\\
16.01	0.01\\
17.01	0.01\\
18.01	0.01\\
19.01	0.01\\
20.01	0.01\\
21.01	0.01\\
22.01	0.01\\
23.01	0.01\\
24.01	0.01\\
25.01	0.01\\
26.01	0.01\\
27.01	0.01\\
28.01	0.01\\
29.01	0.01\\
30.01	0.01\\
31.01	0.01\\
32.01	0.01\\
33.01	0.01\\
34.01	0.01\\
35.01	0.01\\
36.01	0.01\\
37.01	0.01\\
38.01	0.01\\
39.01	0.01\\
40.01	0.01\\
41.01	0.01\\
42.01	0.01\\
43.01	0.01\\
44.01	0.01\\
45.01	0.01\\
46.01	0.01\\
47.01	0.01\\
48.01	0.01\\
49.01	0.01\\
50.01	0.01\\
51.01	0.01\\
52.01	0.01\\
53.01	0.01\\
54.01	0.01\\
55.01	0.01\\
56.01	0.01\\
57.01	0.01\\
58.01	0.01\\
59.01	0.01\\
60.01	0.01\\
61.01	0.01\\
62.01	0.01\\
63.01	0.01\\
64.01	0.01\\
65.01	0.01\\
66.01	0.01\\
67.01	0.01\\
68.01	0.01\\
69.01	0.01\\
70.01	0.01\\
71.01	0.01\\
72.01	0.01\\
73.01	0.01\\
74.01	0.01\\
75.01	0.01\\
76.01	0.01\\
77.01	0.01\\
78.01	0.01\\
79.01	0.01\\
80.01	0.01\\
81.01	0.01\\
82.01	0.01\\
83.01	0.01\\
84.01	0.01\\
85.01	0.01\\
86.01	0.01\\
87.01	0.01\\
88.01	0.01\\
89.01	0.01\\
90.01	0.01\\
91.01	0.01\\
92.01	0.01\\
93.01	0.01\\
94.01	0.01\\
95.01	0.01\\
96.01	0.01\\
97.01	0.01\\
98.01	0.01\\
99.01	0.00671154109295583\\
99.02	0.00666648979566538\\
99.03	0.00662127010377893\\
99.04	0.00657589218059403\\
99.05	0.00653030932509738\\
99.06	0.00648453023571555\\
99.07	0.00643856490098297\\
99.08	0.00639242392068952\\
99.09	0.00634611853474451\\
99.1	0.00629966065349513\\
99.11	0.00625306288962129\\
99.12	0.00620633859169888\\
99.13	0.00615950187952902\\
99.14	0.0061125419965859\\
99.15	0.00606516434576647\\
99.16	0.00601736505574436\\
99.17	0.00596914019869767\\
99.18	0.0059204857881827\\
99.19	0.00587139777688372\\
99.2	0.00582187205430372\\
99.21	0.00577190444433596\\
99.22	0.00572149075281801\\
99.23	0.00567062689318552\\
99.24	0.0056193087241921\\
99.25	0.00556753204802381\\
99.26	0.00551529260831485\\
99.27	0.00546258608805913\\
99.28	0.00540940810741223\\
99.29	0.0053557542213778\\
99.3	0.00530161991737217\\
99.31	0.00524700061266076\\
99.32	0.00519189165165917\\
99.33	0.00513628830309197\\
99.34	0.00508018575700118\\
99.35	0.0050235791215973\\
99.36	0.00496646341994358\\
99.37	0.00490883358644935\\
99.38	0.0048506844630757\\
99.39	0.00479201079544656\\
99.4	0.00473280722875311\\
99.41	0.00467306830343948\\
99.42	0.00461278915159383\\
99.43	0.00455196486264461\\
99.44	0.00449059048163354\\
99.45	0.00442866100885806\\
99.46	0.0043661713995154\\
99.47	0.00430311656334906\\
99.48	0.0042394913639357\\
99.49	0.00417529061784133\\
99.5	0.00411050909416695\\
99.51	0.00404514151408925\\
99.52	0.00397918255039652\\
99.53	0.00391262682701961\\
99.54	0.00384546891855804\\
99.55	0.00377770334980126\\
99.56	0.00370932459524504\\
99.57	0.00364032707860304\\
99.58	0.00357070517231364\\
99.59	0.0035004531970421\\
99.6	0.00342956542117795\\
99.61	0.00335803606032796\\
99.62	0.0032858592768046\\
99.63	0.00321302917911018\\
99.64	0.00313953982141679\\
99.65	0.00306538520304226\\
99.66	0.0029905592679222\\
99.67	0.00291505590407856\\
99.68	0.00283886894308471\\
99.69	0.00276199217558516\\
99.7	0.00268441933711856\\
99.71	0.00260614410548349\\
99.72	0.0025271601001977\\
99.73	0.0024474608819558\\
99.74	0.00236703995208607\\
99.75	0.00228589075200679\\
99.76	0.00220400666268282\\
99.77	0.00212138100408299\\
99.78	0.00203800703463929\\
99.79	0.00195387795070856\\
99.8	0.00186898688603761\\
99.81	0.00178332691123306\\
99.82	0.00169689103323675\\
99.83	0.00160967219480833\\
99.84	0.00152166327401621\\
99.85	0.00143285708373862\\
99.86	0.00134324637117649\\
99.87	0.00125282381737996\\
99.88	0.00116158203679085\\
99.89	0.00106951357680327\\
99.9	0.00097661091734502\\
99.91	0.000882866470482559\\
99.92	0.000788272580052774\\
99.93	0.000692821521324887\\
99.94	0.000596505500696339\\
99.95	0.000499316655426806\\
99.96	0.000401247053414959\\
99.97	0.000302288693022996\\
99.98	0.000202433502954543\\
99.99	0.00010167334219198\\
100	0\\
};
\addlegendentry{$q=-4$};

\addplot [color=mycolor1,dashed]
  table[row sep=crcr]{%
0.01	0.01\\
1.01	0.01\\
2.01	0.01\\
3.01	0.01\\
4.01	0.01\\
5.01	0.01\\
6.01	0.01\\
7.01	0.01\\
8.01	0.01\\
9.01	0.01\\
10.01	0.01\\
11.01	0.01\\
12.01	0.01\\
13.01	0.01\\
14.01	0.01\\
15.01	0.01\\
16.01	0.01\\
17.01	0.01\\
18.01	0.01\\
19.01	0.01\\
20.01	0.01\\
21.01	0.01\\
22.01	0.01\\
23.01	0.01\\
24.01	0.01\\
25.01	0.01\\
26.01	0.01\\
27.01	0.01\\
28.01	0.01\\
29.01	0.01\\
30.01	0.01\\
31.01	0.01\\
32.01	0.01\\
33.01	0.01\\
34.01	0.01\\
35.01	0.01\\
36.01	0.01\\
37.01	0.01\\
38.01	0.01\\
39.01	0.01\\
40.01	0.01\\
41.01	0.01\\
42.01	0.01\\
43.01	0.01\\
44.01	0.01\\
45.01	0.01\\
46.01	0.01\\
47.01	0.01\\
48.01	0.01\\
49.01	0.01\\
50.01	0.01\\
51.01	0.01\\
52.01	0.01\\
53.01	0.01\\
54.01	0.01\\
55.01	0.01\\
56.01	0.01\\
57.01	0.01\\
58.01	0.01\\
59.01	0.01\\
60.01	0.01\\
61.01	0.01\\
62.01	0.01\\
63.01	0.01\\
64.01	0.01\\
65.01	0.01\\
66.01	0.01\\
67.01	0.01\\
68.01	0.01\\
69.01	0.01\\
70.01	0.01\\
71.01	0.01\\
72.01	0.01\\
73.01	0.01\\
74.01	0.01\\
75.01	0.01\\
76.01	0.01\\
77.01	0.01\\
78.01	0.01\\
79.01	0.01\\
80.01	0.01\\
81.01	0.01\\
82.01	0.01\\
83.01	0.01\\
84.01	0.01\\
85.01	0.01\\
86.01	0.01\\
87.01	0.01\\
88.01	0.01\\
89.01	0.01\\
90.01	0.01\\
91.01	0.01\\
92.01	0.01\\
93.01	0.01\\
94.01	0.01\\
95.01	0.01\\
96.01	0.01\\
97.01	0.01\\
98.01	0.01\\
99.01	0.00895053971824736\\
99.02	0.00874693604588305\\
99.03	0.00854181547068092\\
99.04	0.0083351546369417\\
99.05	0.00812698671253183\\
99.06	0.00791728924939752\\
99.07	0.00770603839157192\\
99.08	0.00749320955439558\\
99.09	0.00727877739613037\\
99.1	0.007062715780018\\
99.11	0.0068449977483514\\
99.12	0.00662559549257636\\
99.13	0.00640448031894984\\
99.14	0.00619506749001899\\
99.15	0.00614343065536976\\
99.16	0.00609144762524035\\
99.17	0.00603912153677965\\
99.18	0.00598645582119131\\
99.19	0.00593345421943499\\
99.2	0.00588012078333322\\
99.21	0.0058264598893813\\
99.22	0.00577246463597955\\
99.23	0.00571809972977372\\
99.24	0.00566336885383304\\
99.25	0.0056082760216254\\
99.26	0.00555282559207382\\
99.27	0.00549702228530884\\
99.28	0.0054408711991545\\
99.29	0.00538437782638891\\
99.3	0.00532754807282254\\
99.31	0.00527038827624041\\
99.32	0.0052129052262574\\
99.33	0.0051551061851392\\
99.34	0.00509699890964516\\
99.35	0.00503859167365097\\
99.36	0.00497989329218159\\
99.37	0.00492091315045911\\
99.38	0.0048616612557091\\
99.39	0.00480214823897229\\
99.4	0.00474238538462658\\
99.41	0.00468238466146454\\
99.42	0.00462184456317968\\
99.43	0.00456075958847703\\
99.44	0.00449912481928159\\
99.45	0.00443693528457883\\
99.46	0.00437418595912791\\
99.47	0.00431087176214988\\
99.48	0.00424698765185965\\
99.49	0.00418252865380583\\
99.5	0.00411748975146728\\
99.51	0.00405186588565974\\
99.52	0.00398565195391432\\
99.53	0.00391884280982608\\
99.54	0.00385143326237089\\
99.55	0.00378341807518831\\
99.56	0.00371479196582851\\
99.57	0.00364554960496075\\
99.58	0.00357568561554109\\
99.59	0.00350519457193654\\
99.6	0.00343407099900291\\
99.61	0.00336230937111332\\
99.62	0.00328990411113425\\
99.63	0.00321684958934553\\
99.64	0.00314314012230076\\
99.65	0.00306876997162424\\
99.66	0.00299373334274013\\
99.67	0.00291802438352953\\
99.68	0.00284163718291057\\
99.69	0.00276456578468188\\
99.7	0.00268680417387409\\
99.71	0.00260834627216765\\
99.72	0.00252918593610012\\
99.73	0.00244931695514438\\
99.74	0.00236873304971111\\
99.75	0.00228742786903057\\
99.76	0.00220539498890514\\
99.77	0.00212262790932414\\
99.78	0.00203912005193141\\
99.79	0.0019548647573356\\
99.8	0.00186985528225254\\
99.81	0.00178408479646787\\
99.82	0.00169754637960787\\
99.83	0.00161023301770489\\
99.84	0.00152213759954336\\
99.85	0.001433252912771\\
99.86	0.00134357163975866\\
99.87	0.00125308635319149\\
99.88	0.0011617895113721\\
99.89	0.00106967345321565\\
99.9	0.000976730392914901\\
99.91	0.000882952414253254\\
99.92	0.000788331464541591\\
99.93	0.000692859348149801\\
99.94	0.000596527719603999\\
99.95	0.000499328076217975\\
99.96	0.000401251750225215\\
99.97	0.000302289900375147\\
99.98	0.000202433502954543\\
99.99	0.000101673342191978\\
100	0\\
};
\addlegendentry{$q=-3$};

\addplot [color=red,dashed]
  table[row sep=crcr]{%
0.01	0.01\\
1.01	0.01\\
2.01	0.01\\
3.01	0.01\\
4.01	0.01\\
5.01	0.01\\
6.01	0.01\\
7.01	0.01\\
8.01	0.01\\
9.01	0.01\\
10.01	0.01\\
11.01	0.01\\
12.01	0.01\\
13.01	0.01\\
14.01	0.01\\
15.01	0.01\\
16.01	0.01\\
17.01	0.01\\
18.01	0.01\\
19.01	0.01\\
20.01	0.01\\
21.01	0.01\\
22.01	0.01\\
23.01	0.01\\
24.01	0.01\\
25.01	0.01\\
26.01	0.01\\
27.01	0.01\\
28.01	0.01\\
29.01	0.01\\
30.01	0.01\\
31.01	0.01\\
32.01	0.01\\
33.01	0.01\\
34.01	0.01\\
35.01	0.01\\
36.01	0.01\\
37.01	0.01\\
38.01	0.01\\
39.01	0.01\\
40.01	0.01\\
41.01	0.01\\
42.01	0.01\\
43.01	0.01\\
44.01	0.01\\
45.01	0.01\\
46.01	0.01\\
47.01	0.01\\
48.01	0.01\\
49.01	0.01\\
50.01	0.01\\
51.01	0.01\\
52.01	0.01\\
53.01	0.01\\
54.01	0.01\\
55.01	0.01\\
56.01	0.01\\
57.01	0.01\\
58.01	0.01\\
59.01	0.01\\
60.01	0.01\\
61.01	0.01\\
62.01	0.01\\
63.01	0.01\\
64.01	0.01\\
65.01	0.01\\
66.01	0.01\\
67.01	0.01\\
68.01	0.01\\
69.01	0.01\\
70.01	0.01\\
71.01	0.01\\
72.01	0.01\\
73.01	0.01\\
74.01	0.01\\
75.01	0.01\\
76.01	0.01\\
77.01	0.01\\
78.01	0.01\\
79.01	0.01\\
80.01	0.01\\
81.01	0.01\\
82.01	0.01\\
83.01	0.01\\
84.01	0.01\\
85.01	0.01\\
86.01	0.01\\
87.01	0.01\\
88.01	0.01\\
89.01	0.01\\
90.01	0.01\\
91.01	0.01\\
92.01	0.01\\
93.01	0.01\\
94.01	0.01\\
95.01	0.01\\
96.01	0.01\\
97.01	0.01\\
98.01	0.01\\
99.01	0.01\\
99.02	0.01\\
99.03	0.01\\
99.04	0.01\\
99.05	0.01\\
99.06	0.01\\
99.07	0.01\\
99.08	0.01\\
99.09	0.01\\
99.1	0.01\\
99.11	0.01\\
99.12	0.01\\
99.13	0.01\\
99.14	0.00998658074540667\\
99.15	0.00981394762269664\\
99.16	0.00964021100322028\\
99.17	0.00946535581611112\\
99.18	0.0092893668729081\\
99.19	0.00911222854862415\\
99.2	0.00893392479044062\\
99.21	0.00875443910369502\\
99.22	0.00857376609587646\\
99.23	0.00839192844111959\\
99.24	0.00820890970110557\\
99.25	0.00802469297024153\\
99.26	0.00783926085981311\\
99.27	0.00765259548150028\\
99.28	0.00746467843022284\\
99.29	0.00727549076627984\\
99.3	0.00708501299674452\\
99.31	0.00689322505607424\\
99.32	0.00670010628589174\\
99.33	0.00650563541389112\\
99.34	0.00630979053181893\\
99.35	0.00611254922952013\\
99.36	0.00591388841707558\\
99.37	0.00571378427464273\\
99.38	0.00551221206393733\\
99.39	0.00530914628219126\\
99.4	0.00510456061610345\\
99.41	0.00489842792758349\\
99.42	0.00483354096659049\\
99.43	0.00476839726764135\\
99.44	0.00470269466849563\\
99.45	0.00463643043392819\\
99.46	0.0045696019104531\\
99.47	0.00450220652384934\\
99.48	0.00443422230458845\\
99.49	0.0043656225605152\\
99.5	0.0042964036119007\\
99.51	0.00422656183753986\\
99.52	0.00415609367880683\\
99.53	0.00408499564389557\\
99.54	0.00401326431225524\\
99.55	0.00394089633923041\\
99.56	0.0038678884609169\\
99.57	0.00379423749924454\\
99.58	0.00371994036729902\\
99.59	0.00364499407489556\\
99.6	0.00356939573441804\\
99.61	0.00349314256693811\\
99.62	0.00341623190862955\\
99.63	0.00333866121749428\\
99.64	0.00326042808041742\\
99.65	0.00318153022056972\\
99.66	0.00310196550517733\\
99.67	0.00302173195367941\\
99.68	0.00294082774629635\\
99.69	0.00285925123288629\\
99.7	0.00277700094237731\\
99.71	0.00269407559271374\\
99.72	0.00261047410832228\\
99.73	0.00252619563924148\\
99.74	0.00244123955819906\\
99.75	0.00235560547282003\\
99.76	0.0022692932384859\\
99.77	0.00218230297188465\\
99.78	0.002094635065294\\
99.79	0.00200629020164378\\
99.8	0.00191726937040586\\
99.81	0.001827573884364\\
99.82	0.0017372053973193\\
99.83	0.00164616592279117\\
99.84	0.00155445785377785\\
99.85	0.00146208398364519\\
99.86	0.00136904752821746\\
99.87	0.00127535214914916\\
99.88	0.00118100197866295\\
99.89	0.00108600164574474\\
99.9	0.000990356303894199\\
99.91	0.000894071660039042\\
99.92	0.000797154004978352\\
99.93	0.000699610246319197\\
99.94	0.000601447943428838\\
99.95	0.000502675344543365\\
99.96	0.000403301426184725\\
99.97	0.000303335935050114\\
99.98	0.000202789432550832\\
99.99	0.00010167334219198\\
100	0\\
};
\addlegendentry{$q=-2$};

\addplot [color=blue,dashed]
  table[row sep=crcr]{%
0.01	0.01\\
1.01	0.01\\
2.01	0.01\\
3.01	0.01\\
4.01	0.01\\
5.01	0.01\\
6.01	0.01\\
7.01	0.01\\
8.01	0.01\\
9.01	0.01\\
10.01	0.01\\
11.01	0.01\\
12.01	0.01\\
13.01	0.01\\
14.01	0.01\\
15.01	0.01\\
16.01	0.01\\
17.01	0.01\\
18.01	0.01\\
19.01	0.01\\
20.01	0.01\\
21.01	0.01\\
22.01	0.01\\
23.01	0.01\\
24.01	0.01\\
25.01	0.01\\
26.01	0.01\\
27.01	0.01\\
28.01	0.01\\
29.01	0.01\\
30.01	0.01\\
31.01	0.01\\
32.01	0.01\\
33.01	0.01\\
34.01	0.01\\
35.01	0.01\\
36.01	0.01\\
37.01	0.01\\
38.01	0.01\\
39.01	0.01\\
40.01	0.01\\
41.01	0.01\\
42.01	0.01\\
43.01	0.01\\
44.01	0.01\\
45.01	0.01\\
46.01	0.01\\
47.01	0.01\\
48.01	0.01\\
49.01	0.01\\
50.01	0.01\\
51.01	0.01\\
52.01	0.01\\
53.01	0.01\\
54.01	0.01\\
55.01	0.01\\
56.01	0.01\\
57.01	0.01\\
58.01	0.01\\
59.01	0.01\\
60.01	0.01\\
61.01	0.01\\
62.01	0.01\\
63.01	0.01\\
64.01	0.01\\
65.01	0.01\\
66.01	0.01\\
67.01	0.01\\
68.01	0.01\\
69.01	0.01\\
70.01	0.01\\
71.01	0.01\\
72.01	0.01\\
73.01	0.01\\
74.01	0.01\\
75.01	0.01\\
76.01	0.01\\
77.01	0.01\\
78.01	0.01\\
79.01	0.01\\
80.01	0.01\\
81.01	0.01\\
82.01	0.01\\
83.01	0.01\\
84.01	0.01\\
85.01	0.01\\
86.01	0.01\\
87.01	0.01\\
88.01	0.01\\
89.01	0.01\\
90.01	0.01\\
91.01	0.01\\
92.01	0.01\\
93.01	0.01\\
94.01	0.01\\
95.01	0.01\\
96.01	0.01\\
97.01	0.01\\
98.01	0.01\\
99.01	0.01\\
99.02	0.01\\
99.03	0.01\\
99.04	0.01\\
99.05	0.01\\
99.06	0.01\\
99.07	0.01\\
99.08	0.01\\
99.09	0.01\\
99.1	0.01\\
99.11	0.01\\
99.12	0.01\\
99.13	0.01\\
99.14	0.01\\
99.15	0.01\\
99.16	0.01\\
99.17	0.01\\
99.18	0.01\\
99.19	0.01\\
99.2	0.01\\
99.21	0.01\\
99.22	0.01\\
99.23	0.01\\
99.24	0.01\\
99.25	0.01\\
99.26	0.01\\
99.27	0.01\\
99.28	0.01\\
99.29	0.01\\
99.3	0.01\\
99.31	0.01\\
99.32	0.01\\
99.33	0.01\\
99.34	0.01\\
99.35	0.01\\
99.36	0.01\\
99.37	0.01\\
99.38	0.01\\
99.39	0.01\\
99.4	0.01\\
99.41	0.01\\
99.42	0.0098574925152882\\
99.43	0.00971396905841077\\
99.44	0.00956972004081591\\
99.45	0.0094247362673596\\
99.46	0.0092790082849791\\
99.47	0.00913252639168261\\
99.48	0.00898529998637358\\
99.49	0.00883734285850463\\
99.5	0.00868864559125499\\
99.51	0.00853919851239646\\
99.52	0.00838899168717304\\
99.53	0.00823801491096786\\
99.54	0.00808625770174855\\
99.55	0.00793370929228211\\
99.56	0.00778035862210953\\
99.57	0.00762619432926998\\
99.58	0.00747120474176386\\
99.59	0.00731537786874324\\
99.6	0.00715870139141777\\
99.61	0.00700116265366315\\
99.62	0.00684274865231885\\
99.63	0.00668344602716063\\
99.64	0.00652324105053277\\
99.65	0.00636211961662405\\
99.66	0.00620006723037029\\
99.67	0.00603706899596548\\
99.68	0.0058731096049623\\
99.69	0.00570817332394313\\
99.7	0.00554224398173856\\
99.71	0.00537530495617044\\
99.72	0.00520733916022993\\
99.73	0.00503832902765711\\
99.74	0.00486825649810313\\
99.75	0.00469710300170934\\
99.76	0.00452484944307186\\
99.77	0.00435147618455827\\
99.78	0.00417696302894062\\
99.79	0.00400128920130679\\
99.8	0.00382443333020954\\
99.81	0.00364637342800988\\
99.82	0.00346708687036842\\
99.83	0.00328655037483507\\
99.84	0.00310473997848419\\
99.85	0.00292163101453843\\
99.86	0.00273719808792035\\
99.87	0.00255141504966703\\
99.88	0.00236425497013749\\
99.89	0.0021756901109384\\
99.9	0.00198569189548735\\
99.91	0.0017942310464843\\
99.92	0.00160127761552209\\
99.93	0.00140680072665657\\
99.94	0.00121076854085607\\
99.95	0.00101314821860146\\
99.96	0.000813905880513196\\
99.97	0.000613006565871897\\
99.98	0.000410414188888463\\
99.99	0.000206091492568098\\
100	0\\
};
\addlegendentry{$q=-1$};

\addplot [color=black,solid]
  table[row sep=crcr]{%
0.01	0.000772572677978338\\
1.01	0.000772586960050298\\
2.01	0.000772602072581268\\
3.01	0.000772617778688537\\
4.01	0.000772634104039415\\
5.01	0.000772651075642132\\
6.01	0.000772668721937257\\
7.01	0.000772687072894957\\
8.01	0.000772706160112206\\
9.01	0.000772726016882976\\
10.01	0.000772746678134052\\
11.01	0.000772768179848622\\
12.01	0.000772790556944401\\
13.01	0.000772813838506633\\
14.01	0.000772838047871736\\
15.01	0.000772863234799767\\
16.01	0.000772889479531821\\
17.01	0.000772916839624101\\
18.01	0.000772945370516073\\
19.01	0.00077297513092211\\
20.01	0.000773006183004609\\
21.01	0.000773038592574578\\
22.01	0.000773072429470599\\
23.01	0.000773107768814724\\
24.01	0.000773144696216669\\
25.01	0.000773183330035973\\
26.01	0.000773223914706035\\
27.01	0.00077326719497765\\
28.01	0.000773315809222854\\
29.01	0.000773378937300093\\
30.01	0.000773485074332282\\
31.01	0.000773665534830962\\
32.01	0.000773863872478265\\
33.01	0.000774070091175414\\
34.01	0.000774284597755553\\
35.01	0.000774507831918101\\
36.01	0.000774740270335949\\
37.01	0.000774982431394036\\
38.01	0.000775234880561552\\
39.01	0.000775498236214334\\
40.01	0.000775773175567719\\
41.01	0.000776060442000893\\
42.01	0.000776360864373517\\
43.01	0.000776675402775383\\
44.01	0.000777005141665751\\
45.01	0.000777351244674431\\
46.01	0.000777714973115972\\
47.01	0.00077809758245926\\
48.01	0.000778499957230751\\
49.01	0.000778922178529547\\
50.01	0.000779365701884367\\
51.01	0.000779836269641\\
52.01	0.00078033854888966\\
53.01	0.000780879541511712\\
54.01	0.000781477216495834\\
55.01	0.000782187233415175\\
56.01	0.000783162452931356\\
57.01	0.00078462801604825\\
58.01	0.000786331132114963\\
59.01	0.000788114932563816\\
60.01	0.000789985425763777\\
61.01	0.000791949322413138\\
62.01	0.000794014170178546\\
63.01	0.000796188578580066\\
64.01	0.000798482690067912\\
65.01	0.000800909308936693\\
66.01	0.000803486464933658\\
67.01	0.000806239812929157\\
68.01	0.000809189496130203\\
69.01	0.00081239867048309\\
70.01	0.00081611278869758\\
71.01	0.000821166101299073\\
72.01	0.000828765825611963\\
73.01	0.000839836010926244\\
74.01	0.000855215841775262\\
75.01	0.000872160581526399\\
76.01	0.000891019008814649\\
77.01	0.000913615766736705\\
78.01	0.000947976808092561\\
79.01	0.000988163762707569\\
80.01	0.00103137174952786\\
81.01	0.00108005140637381\\
82.01	0.00113718005134342\\
83.01	0.00120552757710205\\
84.01	0.00130915617875347\\
85.01	0.00142837050157621\\
86.01	0.00155419534551001\\
87.01	0.00168938017131609\\
88.01	0.00184620959744923\\
89.01	0.0020757431789086\\
90.01	0.00233942407469962\\
91.01	0.00261418593362975\\
92.01	0.00290107740241149\\
93.01	0.00320129505312325\\
94.01	0.00351611636090081\\
95.01	0.00384634088144753\\
96.01	0.00419009984971961\\
97.01	0.00455316857454773\\
98.01	0.00498519971313656\\
99.01	0.00582217103975174\\
99.02	0.00583702608917074\\
99.03	0.00585212152213285\\
99.04	0.00586746285631636\\
99.05	0.00588305574613397\\
99.06	0.00589890598649193\\
99.07	0.00591501951666894\\
99.08	0.00593140242431951\\
99.09	0.00594806094960653\\
99.1	0.0059650014894658\\
99.11	0.00598223060201175\\
99.12	0.00599975501108825\\
99.13	0.00601758161096993\\
99.14	0.0060357174712204\\
99.15	0.00605416984171331\\
99.16	0.00607294615782251\\
99.17	0.00609205404578594\\
99.18	0.00611150132825341\\
99.19	0.00613129603002837\\
99.2	0.00615144638400911\\
99.21	0.00617196083733744\\
99.22	0.006192848057764\\
99.23	0.00621411694023918\\
99.24	0.0062357766137396\\
99.25	0.00625783644834031\\
99.26	0.00628030606254335\\
99.27	0.0063031953308743\\
99.28	0.00632651439175847\\
99.29	0.0063502736556895\\
99.3	0.00637448381370353\\
99.31	0.00639915586344442\\
99.32	0.00642430111417673\\
99.33	0.00644993118405764\\
99.34	0.00647605801001219\\
99.35	0.00650269385799353\\
99.36	0.00652985133364663\\
99.37	0.00655754339339478\\
99.38	0.00658574111450058\\
99.39	0.0066144340019461\\
99.4	0.00664363418934559\\
99.41	0.00667335415000014\\
99.42	0.00670360670988301\\
99.43	0.00673440507738807\\
99.44	0.00676574324691763\\
99.45	0.00679761637399577\\
99.46	0.0068300377389763\\
99.47	0.00686302100297037\\
99.48	0.00689658022086231\\
99.49	0.00693072985487318\\
99.5	0.00696548478869991\\
99.51	0.00700086034226031\\
99.52	0.00703687228707563\\
99.53	0.00707353686232452\\
99.54	0.00711087079160439\\
99.55	0.00714889130043846\\
99.56	0.00718761613456919\\
99.57	0.00722706357908156\\
99.58	0.00726725247840242\\
99.59	0.00730820225722517\\
99.6	0.00734993294241264\\
99.61	0.00739246518593405\\
99.62	0.00743582028889647\\
99.63	0.0074800202267347\\
99.64	0.00752508767562849\\
99.65	0.0075710460402206\\
99.66	0.00761791948271452\\
99.67	0.00766573295343635\\
99.68	0.00771451222295164\\
99.69	0.00776428391583597\\
99.7	0.00781507554620144\\
99.71	0.00786691555509222\\
99.72	0.00791983334987064\\
99.73	0.00797385934572439\\
99.74	0.00802902500943545\\
99.75	0.00808536290556256\\
99.76	0.00814290674520107\\
99.77	0.00820169143749746\\
99.78	0.00826175314411004\\
99.79	0.00832312933682349\\
99.8	0.00838585885854211\\
99.81	0.00844998198790592\\
99.82	0.00851554050779452\\
99.83	0.00858257777800699\\
99.84	0.00865113881243126\\
99.85	0.00872127036104468\\
99.86	0.00879302099711813\\
99.87	0.00886644121003055\\
99.88	0.00894158350413849\\
99.89	0.00901850250418713\\
99.9	0.00909725506779586\\
99.91	0.00917790040560339\\
99.92	0.00926050020971464\\
99.93	0.00934511879115603\\
99.94	0.0094318232271174\\
99.95	0.00952068351883868\\
99.96	0.00961177276108944\\
99.97	0.00970516732428955\\
99.98	0.00980094705043254\\
99.99	0.00989919546410015\\
100	0.01\\
};
\addlegendentry{$q=0$};

\addplot [color=blue,solid]
  table[row sep=crcr]{%
0.01	0.00597637174049999\\
1.01	0.00597648412037382\\
2.01	0.00597660970620105\\
3.01	0.00597674021398701\\
4.01	0.00597687585523506\\
5.01	0.00597701685234169\\
6.01	0.00597716343932752\\
7.01	0.00597731586262773\\
8.01	0.00597747438193883\\
9.01	0.00597763927108442\\
10.01	0.00597781081869609\\
11.01	0.00597798932767577\\
12.01	0.00597817510852149\\
13.01	0.00597836844728766\\
14.01	0.005978569518615\\
15.01	0.00597877849145988\\
16.01	0.00597899609220077\\
17.01	0.00597922288937483\\
18.01	0.00597945933480175\\
19.01	0.00597970590653835\\
20.01	0.00597996311028222\\
21.01	0.00598023148060377\\
22.01	0.005980511582169\\
23.01	0.00598080401162112\\
24.01	0.00598110940359987\\
25.01	0.00598142845767925\\
26.01	0.0059817620671399\\
27.01	0.00598211194342043\\
28.01	0.00598248358592138\\
29.01	0.00598289909207564\\
30.01	0.00598345956946713\\
31.01	0.00598475922475163\\
32.01	0.00598643671418286\\
33.01	0.0059881808016102\\
34.01	0.00598999489836592\\
35.01	0.00599188268651035\\
36.01	0.00599384815213209\\
37.01	0.00599589562406416\\
38.01	0.00599802981886118\\
39.01	0.00600025589246188\\
40.01	0.00600257949701351\\
41.01	0.00600500683691016\\
42.01	0.00600754473531797\\
43.01	0.00601020091323166\\
44.01	0.00601298440402638\\
45.01	0.00601590495653862\\
46.01	0.0060189733226658\\
47.01	0.00602220128626651\\
48.01	0.00602560013904741\\
49.01	0.0060291736273496\\
50.01	0.006032912968422\\
51.01	0.00603686096257251\\
52.01	0.00604106277693161\\
53.01	0.00604555228044429\\
54.01	0.00605038664918195\\
55.01	0.00605572878137712\\
56.01	0.0060622670694374\\
57.01	0.00607265491682967\\
58.01	0.00608727136683662\\
59.01	0.00610259539511717\\
60.01	0.00611868041468113\\
61.01	0.00613558603139828\\
62.01	0.00615337902088007\\
63.01	0.00617213453831509\\
64.01	0.00619193766548622\\
65.01	0.00621288523055736\\
66.01	0.00623508623077034\\
67.01	0.0062586559502836\\
68.01	0.00628373881167773\\
69.01	0.00631057064257134\\
70.01	0.00633970694645457\\
71.01	0.00637393934762701\\
72.01	0.00642982899472481\\
73.01	0.0064960493968363\\
74.01	0.00656504558858773\\
75.01	0.00663907700337228\\
76.01	0.00671701575837442\\
77.01	0.00679948725287152\\
78.01	0.00688078554199095\\
79.01	0.00695981358597765\\
80.01	0.00704217145429333\\
81.01	0.00713838900179958\\
82.01	0.00729531942674517\\
83.01	0.0074780597795894\\
84.01	0.00765388445422861\\
85.01	0.00782883409880153\\
86.01	0.00801153835761498\\
87.01	0.00820388145655756\\
88.01	0.00840914274726625\\
89.01	0.00859674809745502\\
90.01	0.00877379354560438\\
91.01	0.00895875007059519\\
92.01	0.00915207189682873\\
93.01	0.00935422007184203\\
94.01	0.00956552578984417\\
95.01	0.00978446542419082\\
96.01	0.00998035950606173\\
97.01	0.01\\
98.01	0.01\\
99.01	0.01\\
99.02	0.01\\
99.03	0.01\\
99.04	0.01\\
99.05	0.01\\
99.06	0.01\\
99.07	0.01\\
99.08	0.01\\
99.09	0.01\\
99.1	0.01\\
99.11	0.01\\
99.12	0.01\\
99.13	0.01\\
99.14	0.01\\
99.15	0.01\\
99.16	0.01\\
99.17	0.01\\
99.18	0.01\\
99.19	0.01\\
99.2	0.01\\
99.21	0.01\\
99.22	0.01\\
99.23	0.01\\
99.24	0.01\\
99.25	0.01\\
99.26	0.01\\
99.27	0.01\\
99.28	0.01\\
99.29	0.01\\
99.3	0.01\\
99.31	0.01\\
99.32	0.01\\
99.33	0.01\\
99.34	0.01\\
99.35	0.01\\
99.36	0.01\\
99.37	0.01\\
99.38	0.01\\
99.39	0.01\\
99.4	0.01\\
99.41	0.01\\
99.42	0.01\\
99.43	0.01\\
99.44	0.01\\
99.45	0.01\\
99.46	0.01\\
99.47	0.01\\
99.48	0.01\\
99.49	0.01\\
99.5	0.01\\
99.51	0.01\\
99.52	0.01\\
99.53	0.01\\
99.54	0.01\\
99.55	0.01\\
99.56	0.01\\
99.57	0.01\\
99.58	0.01\\
99.59	0.01\\
99.6	0.01\\
99.61	0.01\\
99.62	0.01\\
99.63	0.01\\
99.64	0.01\\
99.65	0.01\\
99.66	0.01\\
99.67	0.01\\
99.68	0.01\\
99.69	0.01\\
99.7	0.01\\
99.71	0.01\\
99.72	0.01\\
99.73	0.01\\
99.74	0.01\\
99.75	0.01\\
99.76	0.01\\
99.77	0.01\\
99.78	0.01\\
99.79	0.01\\
99.8	0.01\\
99.81	0.01\\
99.82	0.01\\
99.83	0.01\\
99.84	0.01\\
99.85	0.01\\
99.86	0.01\\
99.87	0.01\\
99.88	0.01\\
99.89	0.01\\
99.9	0.01\\
99.91	0.01\\
99.92	0.01\\
99.93	0.01\\
99.94	0.01\\
99.95	0.01\\
99.96	0.01\\
99.97	0.01\\
99.98	0.01\\
99.99	0.01\\
100	0.01\\
};
\addlegendentry{$q=1$};

\addplot [color=red,solid]
  table[row sep=crcr]{%
0.01	0.00819891155222631\\
1.01	0.00820015753419366\\
2.01	0.00820176826489647\\
3.01	0.00820344226333224\\
4.01	0.00820518224544149\\
5.01	0.00820699106507912\\
6.01	0.00820887172294877\\
7.01	0.00821082737625455\\
8.01	0.00821286134911644\\
9.01	0.00821497714373716\\
10.01	0.00821717845195611\\
11.01	0.00821946916471961\\
12.01	0.00822185336414796\\
13.01	0.00822433520686423\\
14.01	0.00822691825834294\\
15.01	0.00822960421233094\\
16.01	0.00823239983585605\\
17.01	0.00823531337608766\\
18.01	0.00823835054651003\\
19.01	0.0082415173689447\\
20.01	0.00824482019683733\\
21.01	0.00824826572672234\\
22.01	0.00825186100637006\\
23.01	0.00825561343884282\\
24.01	0.00825953078521241\\
25.01	0.00826362119018675\\
26.01	0.00826789338480721\\
27.01	0.00827235806380775\\
28.01	0.00827703713083022\\
29.01	0.00828201829708017\\
30.01	0.00828747770838528\\
31.01	0.00829288931257152\\
32.01	0.00829826043790506\\
33.01	0.00830384442809367\\
34.01	0.00830965196769699\\
35.01	0.00831569454460901\\
36.01	0.00832198453961317\\
37.01	0.00832853532865284\\
38.01	0.00833536139960729\\
39.01	0.00834247848457186\\
40.01	0.00834990370414062\\
41.01	0.00835765570326112\\
42.01	0.00836575474161264\\
43.01	0.00837422333853095\\
44.01	0.00838309015809641\\
45.01	0.00839238543768394\\
46.01	0.00840214010128022\\
47.01	0.00841238743754561\\
48.01	0.00842315580364073\\
49.01	0.00843440870608351\\
50.01	0.00844573230156728\\
51.01	0.00845709682997258\\
52.01	0.00846915109054207\\
53.01	0.00848197429192993\\
54.01	0.0084956852726762\\
55.01	0.00851067049311928\\
56.01	0.00852989465237098\\
57.01	0.00856236215112235\\
58.01	0.00859490772454737\\
59.01	0.00862892747624734\\
60.01	0.00866451082496186\\
61.01	0.00870175497656521\\
62.01	0.00874076592036578\\
63.01	0.00878165962799344\\
64.01	0.00882456346411751\\
65.01	0.00886961735713146\\
66.01	0.00891697273252878\\
67.01	0.00896679583522316\\
68.01	0.00901931261411121\\
69.01	0.0090748541535986\\
70.01	0.00913410484180677\\
71.01	0.00919869070804489\\
72.01	0.00925922041567255\\
73.01	0.00931728092044048\\
74.01	0.00939362413729078\\
75.01	0.00951723551782385\\
76.01	0.00964573885723674\\
77.01	0.00977820665852457\\
78.01	0.0099098698887369\\
79.01	0.01\\
80.01	0.01\\
81.01	0.01\\
82.01	0.01\\
83.01	0.01\\
84.01	0.01\\
85.01	0.01\\
86.01	0.01\\
87.01	0.01\\
88.01	0.01\\
89.01	0.01\\
90.01	0.01\\
91.01	0.01\\
92.01	0.01\\
93.01	0.01\\
94.01	0.01\\
95.01	0.01\\
96.01	0.01\\
97.01	0.01\\
98.01	0.01\\
99.01	0.01\\
99.02	0.01\\
99.03	0.01\\
99.04	0.01\\
99.05	0.01\\
99.06	0.01\\
99.07	0.01\\
99.08	0.01\\
99.09	0.01\\
99.1	0.01\\
99.11	0.01\\
99.12	0.01\\
99.13	0.01\\
99.14	0.01\\
99.15	0.01\\
99.16	0.01\\
99.17	0.01\\
99.18	0.01\\
99.19	0.01\\
99.2	0.01\\
99.21	0.01\\
99.22	0.01\\
99.23	0.01\\
99.24	0.01\\
99.25	0.01\\
99.26	0.01\\
99.27	0.01\\
99.28	0.01\\
99.29	0.01\\
99.3	0.01\\
99.31	0.01\\
99.32	0.01\\
99.33	0.01\\
99.34	0.01\\
99.35	0.01\\
99.36	0.01\\
99.37	0.01\\
99.38	0.01\\
99.39	0.01\\
99.4	0.01\\
99.41	0.01\\
99.42	0.01\\
99.43	0.01\\
99.44	0.01\\
99.45	0.01\\
99.46	0.01\\
99.47	0.01\\
99.48	0.01\\
99.49	0.01\\
99.5	0.01\\
99.51	0.01\\
99.52	0.01\\
99.53	0.01\\
99.54	0.01\\
99.55	0.01\\
99.56	0.01\\
99.57	0.01\\
99.58	0.01\\
99.59	0.01\\
99.6	0.01\\
99.61	0.01\\
99.62	0.01\\
99.63	0.01\\
99.64	0.01\\
99.65	0.01\\
99.66	0.01\\
99.67	0.01\\
99.68	0.01\\
99.69	0.01\\
99.7	0.01\\
99.71	0.01\\
99.72	0.01\\
99.73	0.01\\
99.74	0.01\\
99.75	0.01\\
99.76	0.01\\
99.77	0.01\\
99.78	0.01\\
99.79	0.01\\
99.8	0.01\\
99.81	0.01\\
99.82	0.01\\
99.83	0.01\\
99.84	0.01\\
99.85	0.01\\
99.86	0.01\\
99.87	0.01\\
99.88	0.01\\
99.89	0.01\\
99.9	0.01\\
99.91	0.01\\
99.92	0.01\\
99.93	0.01\\
99.94	0.01\\
99.95	0.01\\
99.96	0.01\\
99.97	0.01\\
99.98	0.01\\
99.99	0.01\\
100	0.01\\
};
\addlegendentry{$q=2$};

\addplot [color=mycolor1,solid]
  table[row sep=crcr]{%
0.01	0.00956306055999642\\
1.01	0.00956475096158899\\
2.01	0.0095662849236269\\
3.01	0.00956787879633217\\
4.01	0.0095695352344733\\
5.01	0.00957125704962719\\
6.01	0.009573047224405\\
7.01	0.00957490892853349\\
8.01	0.00957684553707455\\
9.01	0.00957886065102747\\
10.01	0.00958095812003643\\
11.01	0.00958314206306399\\
12.01	0.00958541685429289\\
13.01	0.009587786810908\\
14.01	0.00959025333986315\\
15.01	0.00959279539499291\\
16.01	0.00959540372759264\\
17.01	0.00959813146562209\\
18.01	0.00960098733127659\\
19.01	0.00960398014324473\\
20.01	0.00960711973595633\\
21.01	0.00961041713057055\\
22.01	0.00961388474206355\\
23.01	0.00961753663241435\\
24.01	0.0096213888279516\\
25.01	0.00962545975608485\\
26.01	0.00962977109592019\\
27.01	0.00963435128774966\\
28.01	0.00963926408165035\\
29.01	0.00964495644760401\\
30.01	0.00965498589107521\\
31.01	0.00966678071723932\\
32.01	0.00967896706654316\\
33.01	0.0096916266455589\\
34.01	0.00970478055239498\\
35.01	0.00971845105333241\\
36.01	0.00973266166620452\\
37.01	0.00974743725152351\\
38.01	0.00976280411205298\\
39.01	0.00977879010045143\\
40.01	0.00979542472996721\\
41.01	0.00981273927414662\\
42.01	0.00983076693023207\\
43.01	0.00984954407629783\\
44.01	0.00986911222972129\\
45.01	0.00988951305005627\\
46.01	0.00991078943308928\\
47.01	0.00993298481973196\\
48.01	0.00995612096218882\\
49.01	0.00997991879443703\\
50.01	0.00999938358127485\\
51.01	0.01\\
52.01	0.01\\
53.01	0.01\\
54.01	0.01\\
55.01	0.01\\
56.01	0.01\\
57.01	0.01\\
58.01	0.01\\
59.01	0.01\\
60.01	0.01\\
61.01	0.01\\
62.01	0.01\\
63.01	0.01\\
64.01	0.01\\
65.01	0.01\\
66.01	0.01\\
67.01	0.01\\
68.01	0.01\\
69.01	0.01\\
70.01	0.01\\
71.01	0.01\\
72.01	0.01\\
73.01	0.01\\
74.01	0.01\\
75.01	0.01\\
76.01	0.01\\
77.01	0.01\\
78.01	0.01\\
79.01	0.01\\
80.01	0.01\\
81.01	0.01\\
82.01	0.01\\
83.01	0.01\\
84.01	0.01\\
85.01	0.01\\
86.01	0.01\\
87.01	0.01\\
88.01	0.01\\
89.01	0.01\\
90.01	0.01\\
91.01	0.01\\
92.01	0.01\\
93.01	0.01\\
94.01	0.01\\
95.01	0.01\\
96.01	0.01\\
97.01	0.01\\
98.01	0.01\\
99.01	0.01\\
99.02	0.01\\
99.03	0.01\\
99.04	0.01\\
99.05	0.01\\
99.06	0.01\\
99.07	0.01\\
99.08	0.01\\
99.09	0.01\\
99.1	0.01\\
99.11	0.01\\
99.12	0.01\\
99.13	0.01\\
99.14	0.01\\
99.15	0.01\\
99.16	0.01\\
99.17	0.01\\
99.18	0.01\\
99.19	0.01\\
99.2	0.01\\
99.21	0.01\\
99.22	0.01\\
99.23	0.01\\
99.24	0.01\\
99.25	0.01\\
99.26	0.01\\
99.27	0.01\\
99.28	0.01\\
99.29	0.01\\
99.3	0.01\\
99.31	0.01\\
99.32	0.01\\
99.33	0.01\\
99.34	0.01\\
99.35	0.01\\
99.36	0.01\\
99.37	0.01\\
99.38	0.01\\
99.39	0.01\\
99.4	0.01\\
99.41	0.01\\
99.42	0.01\\
99.43	0.01\\
99.44	0.01\\
99.45	0.01\\
99.46	0.01\\
99.47	0.01\\
99.48	0.01\\
99.49	0.01\\
99.5	0.01\\
99.51	0.01\\
99.52	0.01\\
99.53	0.01\\
99.54	0.01\\
99.55	0.01\\
99.56	0.01\\
99.57	0.01\\
99.58	0.01\\
99.59	0.01\\
99.6	0.01\\
99.61	0.01\\
99.62	0.01\\
99.63	0.01\\
99.64	0.01\\
99.65	0.01\\
99.66	0.01\\
99.67	0.01\\
99.68	0.01\\
99.69	0.01\\
99.7	0.01\\
99.71	0.01\\
99.72	0.01\\
99.73	0.01\\
99.74	0.01\\
99.75	0.01\\
99.76	0.01\\
99.77	0.01\\
99.78	0.01\\
99.79	0.01\\
99.8	0.01\\
99.81	0.01\\
99.82	0.01\\
99.83	0.01\\
99.84	0.01\\
99.85	0.01\\
99.86	0.01\\
99.87	0.01\\
99.88	0.01\\
99.89	0.01\\
99.9	0.01\\
99.91	0.01\\
99.92	0.01\\
99.93	0.01\\
99.94	0.01\\
99.95	0.01\\
99.96	0.01\\
99.97	0.01\\
99.98	0.01\\
99.99	0.01\\
100	0.01\\
};
\addlegendentry{$q=3$};

\addplot [color=green,solid]
  table[row sep=crcr]{%
0.01	0.01\\
1.01	0.01\\
2.01	0.01\\
3.01	0.01\\
4.01	0.01\\
5.01	0.01\\
6.01	0.01\\
7.01	0.01\\
8.01	0.01\\
9.01	0.01\\
10.01	0.01\\
11.01	0.01\\
12.01	0.01\\
13.01	0.01\\
14.01	0.01\\
15.01	0.01\\
16.01	0.01\\
17.01	0.01\\
18.01	0.01\\
19.01	0.01\\
20.01	0.01\\
21.01	0.01\\
22.01	0.01\\
23.01	0.01\\
24.01	0.01\\
25.01	0.01\\
26.01	0.01\\
27.01	0.01\\
28.01	0.01\\
29.01	0.01\\
30.01	0.01\\
31.01	0.01\\
32.01	0.01\\
33.01	0.01\\
34.01	0.01\\
35.01	0.01\\
36.01	0.01\\
37.01	0.01\\
38.01	0.01\\
39.01	0.01\\
40.01	0.01\\
41.01	0.01\\
42.01	0.01\\
43.01	0.01\\
44.01	0.01\\
45.01	0.01\\
46.01	0.01\\
47.01	0.01\\
48.01	0.01\\
49.01	0.01\\
50.01	0.01\\
51.01	0.01\\
52.01	0.01\\
53.01	0.01\\
54.01	0.01\\
55.01	0.01\\
56.01	0.01\\
57.01	0.01\\
58.01	0.01\\
59.01	0.01\\
60.01	0.01\\
61.01	0.01\\
62.01	0.01\\
63.01	0.01\\
64.01	0.01\\
65.01	0.01\\
66.01	0.01\\
67.01	0.01\\
68.01	0.01\\
69.01	0.01\\
70.01	0.01\\
71.01	0.01\\
72.01	0.01\\
73.01	0.01\\
74.01	0.01\\
75.01	0.01\\
76.01	0.01\\
77.01	0.01\\
78.01	0.01\\
79.01	0.01\\
80.01	0.01\\
81.01	0.01\\
82.01	0.01\\
83.01	0.01\\
84.01	0.01\\
85.01	0.01\\
86.01	0.01\\
87.01	0.01\\
88.01	0.01\\
89.01	0.01\\
90.01	0.01\\
91.01	0.01\\
92.01	0.01\\
93.01	0.01\\
94.01	0.01\\
95.01	0.01\\
96.01	0.01\\
97.01	0.01\\
98.01	0.01\\
99.01	0.01\\
99.02	0.01\\
99.03	0.01\\
99.04	0.01\\
99.05	0.01\\
99.06	0.01\\
99.07	0.01\\
99.08	0.01\\
99.09	0.01\\
99.1	0.01\\
99.11	0.01\\
99.12	0.01\\
99.13	0.01\\
99.14	0.01\\
99.15	0.01\\
99.16	0.01\\
99.17	0.01\\
99.18	0.01\\
99.19	0.01\\
99.2	0.01\\
99.21	0.01\\
99.22	0.01\\
99.23	0.01\\
99.24	0.01\\
99.25	0.01\\
99.26	0.01\\
99.27	0.01\\
99.28	0.01\\
99.29	0.01\\
99.3	0.01\\
99.31	0.01\\
99.32	0.01\\
99.33	0.01\\
99.34	0.01\\
99.35	0.01\\
99.36	0.01\\
99.37	0.01\\
99.38	0.01\\
99.39	0.01\\
99.4	0.01\\
99.41	0.01\\
99.42	0.01\\
99.43	0.01\\
99.44	0.01\\
99.45	0.01\\
99.46	0.01\\
99.47	0.01\\
99.48	0.01\\
99.49	0.01\\
99.5	0.01\\
99.51	0.01\\
99.52	0.01\\
99.53	0.01\\
99.54	0.01\\
99.55	0.01\\
99.56	0.01\\
99.57	0.01\\
99.58	0.01\\
99.59	0.01\\
99.6	0.01\\
99.61	0.01\\
99.62	0.01\\
99.63	0.01\\
99.64	0.01\\
99.65	0.01\\
99.66	0.01\\
99.67	0.01\\
99.68	0.01\\
99.69	0.01\\
99.7	0.01\\
99.71	0.01\\
99.72	0.01\\
99.73	0.01\\
99.74	0.01\\
99.75	0.01\\
99.76	0.01\\
99.77	0.01\\
99.78	0.01\\
99.79	0.01\\
99.8	0.01\\
99.81	0.01\\
99.82	0.01\\
99.83	0.01\\
99.84	0.01\\
99.85	0.01\\
99.86	0.01\\
99.87	0.01\\
99.88	0.01\\
99.89	0.01\\
99.9	0.01\\
99.91	0.01\\
99.92	0.01\\
99.93	0.01\\
99.94	0.01\\
99.95	0.01\\
99.96	0.01\\
99.97	0.01\\
99.98	0.01\\
99.99	0.01\\
100	0.01\\
};
\addlegendentry{$q=4$};

\end{axis}
\end{tikzpicture}%

  \caption{Continuous Time}
\end{subfigure}%
\hfill%
\begin{subfigure}{.45\linewidth}
  \centering
  \setlength\figureheight{\linewidth} 
  \setlength\figurewidth{\linewidth}
  \tikzsetnextfilename{dp_dscr_z15}
  % This file was created by matlab2tikz.
%
%The latest updates can be retrieved from
%  http://www.mathworks.com/matlabcentral/fileexchange/22022-matlab2tikz-matlab2tikz
%where you can also make suggestions and rate matlab2tikz.
%
\definecolor{mycolor1}{rgb}{0.00000,1.00000,0.14286}%
\definecolor{mycolor2}{rgb}{0.00000,1.00000,0.28571}%
\definecolor{mycolor3}{rgb}{0.00000,1.00000,0.42857}%
\definecolor{mycolor4}{rgb}{0.00000,1.00000,0.57143}%
\definecolor{mycolor5}{rgb}{0.00000,1.00000,0.71429}%
\definecolor{mycolor6}{rgb}{0.00000,1.00000,0.85714}%
\definecolor{mycolor7}{rgb}{0.00000,1.00000,1.00000}%
\definecolor{mycolor8}{rgb}{0.00000,0.87500,1.00000}%
\definecolor{mycolor9}{rgb}{0.00000,0.62500,1.00000}%
\definecolor{mycolor10}{rgb}{0.12500,0.00000,1.00000}%
\definecolor{mycolor11}{rgb}{0.25000,0.00000,1.00000}%
\definecolor{mycolor12}{rgb}{0.37500,0.00000,1.00000}%
\definecolor{mycolor13}{rgb}{0.50000,0.00000,1.00000}%
\definecolor{mycolor14}{rgb}{0.62500,0.00000,1.00000}%
\definecolor{mycolor15}{rgb}{0.75000,0.00000,1.00000}%
\definecolor{mycolor16}{rgb}{0.87500,0.00000,1.00000}%
\definecolor{mycolor17}{rgb}{1.00000,0.00000,1.00000}%
\definecolor{mycolor18}{rgb}{1.00000,0.00000,0.87500}%
\definecolor{mycolor19}{rgb}{1.00000,0.00000,0.62500}%
\definecolor{mycolor20}{rgb}{0.85714,0.00000,0.00000}%
\definecolor{mycolor21}{rgb}{0.71429,0.00000,0.00000}%
%
\begin{tikzpicture}

\begin{axis}[%
width=4.1in,
height=3.803in,
at={(0.809in,0.513in)},
scale only axis,
point meta min=0,
point meta max=1,
every outer x axis line/.append style={black},
every x tick label/.append style={font=\color{black}},
xmin=0,
xmax=600,
every outer y axis line/.append style={black},
every y tick label/.append style={font=\color{black}},
ymin=0,
ymax=0.007,
axis background/.style={fill=white},
axis x line*=bottom,
axis y line*=left,
colormap={mymap}{[1pt] rgb(0pt)=(0,1,0); rgb(7pt)=(0,1,1); rgb(15pt)=(0,0,1); rgb(23pt)=(1,0,1); rgb(31pt)=(1,0,0); rgb(38pt)=(0,0,0)},
colorbar,
colorbar style={separate axis lines,every outer x axis line/.append style={black},every x tick label/.append style={font=\color{black}},every outer y axis line/.append style={black},every y tick label/.append style={font=\color{black}},yticklabels={{-19},{-17},{-15},{-13},{-11},{-9},{-7},{-5},{-3},{-1},{1},{3},{5},{7},{9},{11},{13},{15},{17},{19}}}
]
\addplot [color=green,solid,forget plot]
  table[row sep=crcr]{%
1	0\\
2	0\\
3	0\\
4	0\\
5	0\\
6	0\\
7	0\\
8	0\\
9	0\\
10	0\\
11	0\\
12	0\\
13	0\\
14	0\\
15	0\\
16	0\\
17	0\\
18	0\\
19	0\\
20	0\\
21	0\\
22	0\\
23	0\\
24	0\\
25	0\\
26	0\\
27	0\\
28	0\\
29	0\\
30	0\\
31	0\\
32	0\\
33	0\\
34	0\\
35	0\\
36	0\\
37	0\\
38	0\\
39	0\\
40	0\\
41	0\\
42	0\\
43	0\\
44	0\\
45	0\\
46	0\\
47	0\\
48	0\\
49	0\\
50	0\\
51	0\\
52	0\\
53	0\\
54	0\\
55	0\\
56	0\\
57	0\\
58	0\\
59	0\\
60	0\\
61	0\\
62	0\\
63	0\\
64	0\\
65	0\\
66	0\\
67	0\\
68	0\\
69	0\\
70	0\\
71	0\\
72	0\\
73	0\\
74	0\\
75	0\\
76	0\\
77	0\\
78	0\\
79	0\\
80	0\\
81	0\\
82	0\\
83	0\\
84	0\\
85	0\\
86	0\\
87	0\\
88	0\\
89	0\\
90	0\\
91	0\\
92	0\\
93	0\\
94	0\\
95	0\\
96	0\\
97	0\\
98	0\\
99	0\\
100	0\\
101	0\\
102	0\\
103	0\\
104	0\\
105	0\\
106	0\\
107	0\\
108	0\\
109	0\\
110	0\\
111	0\\
112	0\\
113	0\\
114	0\\
115	0\\
116	0\\
117	0\\
118	0\\
119	0\\
120	0\\
121	0\\
122	0\\
123	0\\
124	0\\
125	0\\
126	0\\
127	0\\
128	0\\
129	0\\
130	0\\
131	0\\
132	0\\
133	0\\
134	0\\
135	0\\
136	0\\
137	0\\
138	0\\
139	0\\
140	0\\
141	0\\
142	0\\
143	0\\
144	0\\
145	0\\
146	0\\
147	0\\
148	0\\
149	0\\
150	0\\
151	0\\
152	0\\
153	0\\
154	0\\
155	0\\
156	0\\
157	0\\
158	0\\
159	0\\
160	0\\
161	0\\
162	0\\
163	0\\
164	0\\
165	0\\
166	0\\
167	0\\
168	0\\
169	0\\
170	0\\
171	0\\
172	0\\
173	0\\
174	0\\
175	0\\
176	0\\
177	0\\
178	0\\
179	0\\
180	0\\
181	0\\
182	0\\
183	0\\
184	0\\
185	0\\
186	0\\
187	0\\
188	0\\
189	0\\
190	0\\
191	0\\
192	0\\
193	0\\
194	0\\
195	0\\
196	0\\
197	0\\
198	0\\
199	0\\
200	0\\
201	0\\
202	0\\
203	0\\
204	0\\
205	0\\
206	0\\
207	0\\
208	0\\
209	0\\
210	0\\
211	0\\
212	0\\
213	0\\
214	0\\
215	0\\
216	0\\
217	0\\
218	0\\
219	0\\
220	0\\
221	0\\
222	0\\
223	0\\
224	0\\
225	0\\
226	0\\
227	0\\
228	0\\
229	0\\
230	0\\
231	0\\
232	0\\
233	0\\
234	0\\
235	0\\
236	0\\
237	0\\
238	0\\
239	0\\
240	0\\
241	0\\
242	0\\
243	0\\
244	0\\
245	0\\
246	0\\
247	0\\
248	0\\
249	0\\
250	0\\
251	0\\
252	0\\
253	0\\
254	0\\
255	0\\
256	0\\
257	0\\
258	0\\
259	0\\
260	0\\
261	0\\
262	0\\
263	0\\
264	0\\
265	0\\
266	0\\
267	0\\
268	0\\
269	0\\
270	0\\
271	0\\
272	0\\
273	0\\
274	0\\
275	0\\
276	0\\
277	0\\
278	0\\
279	0\\
280	0\\
281	0\\
282	0\\
283	0\\
284	0\\
285	0\\
286	0\\
287	0\\
288	0\\
289	0\\
290	0\\
291	0\\
292	0\\
293	0\\
294	0\\
295	0\\
296	0\\
297	0\\
298	0\\
299	0\\
300	0\\
301	0\\
302	0\\
303	0\\
304	0\\
305	0\\
306	0\\
307	0\\
308	0\\
309	0\\
310	0\\
311	0\\
312	0\\
313	0\\
314	0\\
315	0\\
316	0\\
317	0\\
318	0\\
319	0\\
320	0\\
321	0\\
322	0\\
323	0\\
324	0\\
325	0\\
326	0\\
327	0\\
328	0\\
329	0\\
330	0\\
331	0\\
332	0\\
333	0\\
334	0\\
335	0\\
336	0\\
337	0\\
338	0\\
339	0\\
340	0\\
341	0\\
342	0\\
343	0\\
344	0\\
345	0\\
346	0\\
347	0\\
348	0\\
349	0\\
350	0\\
351	0\\
352	0\\
353	0\\
354	0\\
355	0\\
356	0\\
357	0\\
358	0\\
359	0\\
360	0\\
361	0\\
362	0\\
363	0\\
364	0\\
365	0\\
366	0\\
367	0\\
368	0\\
369	0\\
370	0\\
371	0\\
372	0\\
373	0\\
374	0\\
375	0\\
376	0\\
377	0\\
378	0\\
379	0\\
380	0\\
381	0\\
382	0\\
383	0\\
384	0\\
385	0\\
386	0\\
387	0\\
388	0\\
389	0\\
390	0\\
391	0\\
392	0\\
393	0\\
394	0\\
395	0\\
396	0\\
397	0\\
398	0\\
399	0\\
400	0\\
401	0\\
402	0\\
403	0\\
404	0\\
405	0\\
406	0\\
407	0\\
408	0\\
409	0\\
410	0\\
411	0\\
412	0\\
413	0\\
414	0\\
415	0\\
416	0\\
417	0\\
418	0\\
419	0\\
420	0\\
421	0\\
422	0\\
423	0\\
424	0\\
425	0\\
426	0\\
427	0\\
428	0\\
429	0\\
430	0\\
431	0\\
432	0\\
433	0\\
434	0\\
435	0\\
436	0\\
437	0\\
438	0\\
439	0\\
440	0\\
441	0\\
442	0\\
443	0\\
444	0\\
445	0\\
446	0\\
447	0\\
448	0\\
449	0\\
450	0\\
451	0\\
452	0\\
453	0\\
454	0\\
455	0\\
456	0\\
457	0\\
458	0\\
459	0\\
460	0\\
461	0\\
462	0\\
463	0\\
464	0\\
465	0\\
466	0\\
467	0\\
468	0\\
469	0\\
470	0\\
471	0\\
472	0\\
473	0\\
474	0\\
475	0\\
476	0\\
477	0\\
478	0\\
479	0\\
480	0\\
481	0\\
482	0\\
483	0\\
484	0\\
485	0\\
486	0\\
487	0\\
488	0\\
489	0\\
490	0\\
491	0\\
492	0\\
493	0\\
494	0\\
495	0\\
496	0\\
497	0\\
498	0\\
499	0\\
500	0\\
501	0\\
502	0\\
503	0\\
504	0\\
505	0\\
506	0\\
507	0\\
508	0\\
509	0\\
510	0\\
511	0\\
512	0\\
513	0\\
514	0\\
515	0\\
516	0\\
517	0\\
518	0\\
519	0\\
520	0\\
521	0\\
522	0\\
523	0\\
524	0\\
525	0\\
526	0\\
527	0\\
528	0\\
529	0\\
530	0\\
531	0\\
532	0\\
533	0\\
534	0\\
535	0\\
536	0\\
537	0\\
538	0\\
539	0\\
540	0\\
541	0\\
542	0\\
543	0\\
544	0\\
545	0\\
546	0\\
547	0\\
548	0\\
549	0\\
550	0\\
551	0\\
552	0\\
553	0\\
554	0\\
555	0\\
556	0\\
557	0\\
558	0\\
559	0\\
560	0\\
561	0\\
562	0\\
563	0\\
564	0\\
565	0\\
566	0\\
567	0\\
568	0\\
569	0\\
570	0\\
571	0\\
572	0\\
573	0\\
574	0\\
575	0\\
576	0\\
577	0\\
578	0\\
579	0\\
580	0\\
581	0\\
582	0\\
583	0\\
584	0\\
585	0\\
586	0\\
587	0\\
588	0\\
589	0\\
590	0\\
591	0\\
592	0\\
593	0\\
594	0\\
595	0\\
596	0\\
597	0\\
598	0\\
599	0\\
600	0\\
};
\addplot [color=mycolor1,solid,forget plot]
  table[row sep=crcr]{%
1	0\\
2	0\\
3	0\\
4	0\\
5	0\\
6	0\\
7	0\\
8	0\\
9	0\\
10	0\\
11	0\\
12	0\\
13	0\\
14	0\\
15	0\\
16	0\\
17	0\\
18	0\\
19	0\\
20	0\\
21	0\\
22	0\\
23	0\\
24	0\\
25	0\\
26	0\\
27	0\\
28	0\\
29	0\\
30	0\\
31	0\\
32	0\\
33	0\\
34	0\\
35	0\\
36	0\\
37	0\\
38	0\\
39	0\\
40	0\\
41	0\\
42	0\\
43	0\\
44	0\\
45	0\\
46	0\\
47	0\\
48	0\\
49	0\\
50	0\\
51	0\\
52	0\\
53	0\\
54	0\\
55	0\\
56	0\\
57	0\\
58	0\\
59	0\\
60	0\\
61	0\\
62	0\\
63	0\\
64	0\\
65	0\\
66	0\\
67	0\\
68	0\\
69	0\\
70	0\\
71	0\\
72	0\\
73	0\\
74	0\\
75	0\\
76	0\\
77	0\\
78	0\\
79	0\\
80	0\\
81	0\\
82	0\\
83	0\\
84	0\\
85	0\\
86	0\\
87	0\\
88	0\\
89	0\\
90	0\\
91	0\\
92	0\\
93	0\\
94	0\\
95	0\\
96	0\\
97	0\\
98	0\\
99	0\\
100	0\\
101	0\\
102	0\\
103	0\\
104	0\\
105	0\\
106	0\\
107	0\\
108	0\\
109	0\\
110	0\\
111	0\\
112	0\\
113	0\\
114	0\\
115	0\\
116	0\\
117	0\\
118	0\\
119	0\\
120	0\\
121	0\\
122	0\\
123	0\\
124	0\\
125	0\\
126	0\\
127	0\\
128	0\\
129	0\\
130	0\\
131	0\\
132	0\\
133	0\\
134	0\\
135	0\\
136	0\\
137	0\\
138	0\\
139	0\\
140	0\\
141	0\\
142	0\\
143	0\\
144	0\\
145	0\\
146	0\\
147	0\\
148	0\\
149	0\\
150	0\\
151	0\\
152	0\\
153	0\\
154	0\\
155	0\\
156	0\\
157	0\\
158	0\\
159	0\\
160	0\\
161	0\\
162	0\\
163	0\\
164	0\\
165	0\\
166	0\\
167	0\\
168	0\\
169	0\\
170	0\\
171	0\\
172	0\\
173	0\\
174	0\\
175	0\\
176	0\\
177	0\\
178	0\\
179	0\\
180	0\\
181	0\\
182	0\\
183	0\\
184	0\\
185	0\\
186	0\\
187	0\\
188	0\\
189	0\\
190	0\\
191	0\\
192	0\\
193	0\\
194	0\\
195	0\\
196	0\\
197	0\\
198	0\\
199	0\\
200	0\\
201	0\\
202	0\\
203	0\\
204	0\\
205	0\\
206	0\\
207	0\\
208	0\\
209	0\\
210	0\\
211	0\\
212	0\\
213	0\\
214	0\\
215	0\\
216	0\\
217	0\\
218	0\\
219	0\\
220	0\\
221	0\\
222	0\\
223	0\\
224	0\\
225	0\\
226	0\\
227	0\\
228	0\\
229	0\\
230	0\\
231	0\\
232	0\\
233	0\\
234	0\\
235	0\\
236	0\\
237	0\\
238	0\\
239	0\\
240	0\\
241	0\\
242	0\\
243	0\\
244	0\\
245	0\\
246	0\\
247	0\\
248	0\\
249	0\\
250	0\\
251	0\\
252	0\\
253	0\\
254	0\\
255	0\\
256	0\\
257	0\\
258	0\\
259	0\\
260	0\\
261	0\\
262	0\\
263	0\\
264	0\\
265	0\\
266	0\\
267	0\\
268	0\\
269	0\\
270	0\\
271	0\\
272	0\\
273	0\\
274	0\\
275	0\\
276	0\\
277	0\\
278	0\\
279	0\\
280	0\\
281	0\\
282	0\\
283	0\\
284	0\\
285	0\\
286	0\\
287	0\\
288	0\\
289	0\\
290	0\\
291	0\\
292	0\\
293	0\\
294	0\\
295	0\\
296	0\\
297	0\\
298	0\\
299	0\\
300	0\\
301	0\\
302	0\\
303	0\\
304	0\\
305	0\\
306	0\\
307	0\\
308	0\\
309	0\\
310	0\\
311	0\\
312	0\\
313	0\\
314	0\\
315	0\\
316	0\\
317	0\\
318	0\\
319	0\\
320	0\\
321	0\\
322	0\\
323	0\\
324	0\\
325	0\\
326	0\\
327	0\\
328	0\\
329	0\\
330	0\\
331	0\\
332	0\\
333	0\\
334	0\\
335	0\\
336	0\\
337	0\\
338	0\\
339	0\\
340	0\\
341	0\\
342	0\\
343	0\\
344	0\\
345	0\\
346	0\\
347	0\\
348	0\\
349	0\\
350	0\\
351	0\\
352	0\\
353	0\\
354	0\\
355	0\\
356	0\\
357	0\\
358	0\\
359	0\\
360	0\\
361	0\\
362	0\\
363	0\\
364	0\\
365	0\\
366	0\\
367	0\\
368	0\\
369	0\\
370	0\\
371	0\\
372	0\\
373	0\\
374	0\\
375	0\\
376	0\\
377	0\\
378	0\\
379	0\\
380	0\\
381	0\\
382	0\\
383	0\\
384	0\\
385	0\\
386	0\\
387	0\\
388	0\\
389	0\\
390	0\\
391	0\\
392	0\\
393	0\\
394	0\\
395	0\\
396	0\\
397	0\\
398	0\\
399	0\\
400	0\\
401	0\\
402	0\\
403	0\\
404	0\\
405	0\\
406	0\\
407	0\\
408	0\\
409	0\\
410	0\\
411	0\\
412	0\\
413	0\\
414	0\\
415	0\\
416	0\\
417	0\\
418	0\\
419	0\\
420	0\\
421	0\\
422	0\\
423	0\\
424	0\\
425	0\\
426	0\\
427	0\\
428	0\\
429	0\\
430	0\\
431	0\\
432	0\\
433	0\\
434	0\\
435	0\\
436	0\\
437	0\\
438	0\\
439	0\\
440	0\\
441	0\\
442	0\\
443	0\\
444	0\\
445	0\\
446	0\\
447	0\\
448	0\\
449	0\\
450	0\\
451	0\\
452	0\\
453	0\\
454	0\\
455	0\\
456	0\\
457	0\\
458	0\\
459	0\\
460	0\\
461	0\\
462	0\\
463	0\\
464	0\\
465	0\\
466	0\\
467	0\\
468	0\\
469	0\\
470	0\\
471	0\\
472	0\\
473	0\\
474	0\\
475	0\\
476	0\\
477	0\\
478	0\\
479	0\\
480	0\\
481	0\\
482	0\\
483	0\\
484	0\\
485	0\\
486	0\\
487	0\\
488	0\\
489	0\\
490	0\\
491	0\\
492	0\\
493	0\\
494	0\\
495	0\\
496	0\\
497	0\\
498	0\\
499	0\\
500	0\\
501	0\\
502	0\\
503	0\\
504	0\\
505	0\\
506	0\\
507	0\\
508	0\\
509	0\\
510	0\\
511	0\\
512	0\\
513	0\\
514	0\\
515	0\\
516	0\\
517	0\\
518	0\\
519	0\\
520	0\\
521	0\\
522	0\\
523	0\\
524	0\\
525	0\\
526	0\\
527	0\\
528	0\\
529	0\\
530	0\\
531	0\\
532	0\\
533	0\\
534	0\\
535	0\\
536	0\\
537	0\\
538	0\\
539	0\\
540	0\\
541	0\\
542	0\\
543	0\\
544	0\\
545	0\\
546	0\\
547	0\\
548	0\\
549	0\\
550	0\\
551	0\\
552	0\\
553	0\\
554	0\\
555	0\\
556	0\\
557	0\\
558	0\\
559	0\\
560	0\\
561	0\\
562	0\\
563	0\\
564	0\\
565	0\\
566	0\\
567	0\\
568	0\\
569	0\\
570	0\\
571	0\\
572	0\\
573	0\\
574	0\\
575	0\\
576	0\\
577	0\\
578	0\\
579	0\\
580	0\\
581	0\\
582	0\\
583	0\\
584	0\\
585	0\\
586	0\\
587	0\\
588	0\\
589	0\\
590	0\\
591	0\\
592	0\\
593	0\\
594	0\\
595	0\\
596	0\\
597	0\\
598	0\\
599	0\\
600	0\\
};
\addplot [color=mycolor2,solid,forget plot]
  table[row sep=crcr]{%
1	0\\
2	0\\
3	0\\
4	0\\
5	0\\
6	0\\
7	0\\
8	0\\
9	0\\
10	0\\
11	0\\
12	0\\
13	0\\
14	0\\
15	0\\
16	0\\
17	0\\
18	0\\
19	0\\
20	0\\
21	0\\
22	0\\
23	0\\
24	0\\
25	0\\
26	0\\
27	0\\
28	0\\
29	0\\
30	0\\
31	0\\
32	0\\
33	0\\
34	0\\
35	0\\
36	0\\
37	0\\
38	0\\
39	0\\
40	0\\
41	0\\
42	0\\
43	0\\
44	0\\
45	0\\
46	0\\
47	0\\
48	0\\
49	0\\
50	0\\
51	0\\
52	0\\
53	0\\
54	0\\
55	0\\
56	0\\
57	0\\
58	0\\
59	0\\
60	0\\
61	0\\
62	0\\
63	0\\
64	0\\
65	0\\
66	0\\
67	0\\
68	0\\
69	0\\
70	0\\
71	0\\
72	0\\
73	0\\
74	0\\
75	0\\
76	0\\
77	0\\
78	0\\
79	0\\
80	0\\
81	0\\
82	0\\
83	0\\
84	0\\
85	0\\
86	0\\
87	0\\
88	0\\
89	0\\
90	0\\
91	0\\
92	0\\
93	0\\
94	0\\
95	0\\
96	0\\
97	0\\
98	0\\
99	0\\
100	0\\
101	0\\
102	0\\
103	0\\
104	0\\
105	0\\
106	0\\
107	0\\
108	0\\
109	0\\
110	0\\
111	0\\
112	0\\
113	0\\
114	0\\
115	0\\
116	0\\
117	0\\
118	0\\
119	0\\
120	0\\
121	0\\
122	0\\
123	0\\
124	0\\
125	0\\
126	0\\
127	0\\
128	0\\
129	0\\
130	0\\
131	0\\
132	0\\
133	0\\
134	0\\
135	0\\
136	0\\
137	0\\
138	0\\
139	0\\
140	0\\
141	0\\
142	0\\
143	0\\
144	0\\
145	0\\
146	0\\
147	0\\
148	0\\
149	0\\
150	0\\
151	0\\
152	0\\
153	0\\
154	0\\
155	0\\
156	0\\
157	0\\
158	0\\
159	0\\
160	0\\
161	0\\
162	0\\
163	0\\
164	0\\
165	0\\
166	0\\
167	0\\
168	0\\
169	0\\
170	0\\
171	0\\
172	0\\
173	0\\
174	0\\
175	0\\
176	0\\
177	0\\
178	0\\
179	0\\
180	0\\
181	0\\
182	0\\
183	0\\
184	0\\
185	0\\
186	0\\
187	0\\
188	0\\
189	0\\
190	0\\
191	0\\
192	0\\
193	0\\
194	0\\
195	0\\
196	0\\
197	0\\
198	0\\
199	0\\
200	0\\
201	0\\
202	0\\
203	0\\
204	0\\
205	0\\
206	0\\
207	0\\
208	0\\
209	0\\
210	0\\
211	0\\
212	0\\
213	0\\
214	0\\
215	0\\
216	0\\
217	0\\
218	0\\
219	0\\
220	0\\
221	0\\
222	0\\
223	0\\
224	0\\
225	0\\
226	0\\
227	0\\
228	0\\
229	0\\
230	0\\
231	0\\
232	0\\
233	0\\
234	0\\
235	0\\
236	0\\
237	0\\
238	0\\
239	0\\
240	0\\
241	0\\
242	0\\
243	0\\
244	0\\
245	0\\
246	0\\
247	0\\
248	0\\
249	0\\
250	0\\
251	0\\
252	0\\
253	0\\
254	0\\
255	0\\
256	0\\
257	0\\
258	0\\
259	0\\
260	0\\
261	0\\
262	0\\
263	0\\
264	0\\
265	0\\
266	0\\
267	0\\
268	0\\
269	0\\
270	0\\
271	0\\
272	0\\
273	0\\
274	0\\
275	0\\
276	0\\
277	0\\
278	0\\
279	0\\
280	0\\
281	0\\
282	0\\
283	0\\
284	0\\
285	0\\
286	0\\
287	0\\
288	0\\
289	0\\
290	0\\
291	0\\
292	0\\
293	0\\
294	0\\
295	0\\
296	0\\
297	0\\
298	0\\
299	0\\
300	0\\
301	0\\
302	0\\
303	0\\
304	0\\
305	0\\
306	0\\
307	0\\
308	0\\
309	0\\
310	0\\
311	0\\
312	0\\
313	0\\
314	0\\
315	0\\
316	0\\
317	0\\
318	0\\
319	0\\
320	0\\
321	0\\
322	0\\
323	0\\
324	0\\
325	0\\
326	0\\
327	0\\
328	0\\
329	0\\
330	0\\
331	0\\
332	0\\
333	0\\
334	0\\
335	0\\
336	0\\
337	0\\
338	0\\
339	0\\
340	0\\
341	0\\
342	0\\
343	0\\
344	0\\
345	0\\
346	0\\
347	0\\
348	0\\
349	0\\
350	0\\
351	0\\
352	0\\
353	0\\
354	0\\
355	0\\
356	0\\
357	0\\
358	0\\
359	0\\
360	0\\
361	0\\
362	0\\
363	0\\
364	0\\
365	0\\
366	0\\
367	0\\
368	0\\
369	0\\
370	0\\
371	0\\
372	0\\
373	0\\
374	0\\
375	0\\
376	0\\
377	0\\
378	0\\
379	0\\
380	0\\
381	0\\
382	0\\
383	0\\
384	0\\
385	0\\
386	0\\
387	0\\
388	0\\
389	0\\
390	0\\
391	0\\
392	0\\
393	0\\
394	0\\
395	0\\
396	0\\
397	0\\
398	0\\
399	0\\
400	0\\
401	0\\
402	0\\
403	0\\
404	0\\
405	0\\
406	0\\
407	0\\
408	0\\
409	0\\
410	0\\
411	0\\
412	0\\
413	0\\
414	0\\
415	0\\
416	0\\
417	0\\
418	0\\
419	0\\
420	0\\
421	0\\
422	0\\
423	0\\
424	0\\
425	0\\
426	0\\
427	0\\
428	0\\
429	0\\
430	0\\
431	0\\
432	0\\
433	0\\
434	0\\
435	0\\
436	0\\
437	0\\
438	0\\
439	0\\
440	0\\
441	0\\
442	0\\
443	0\\
444	0\\
445	0\\
446	0\\
447	0\\
448	0\\
449	0\\
450	0\\
451	0\\
452	0\\
453	0\\
454	0\\
455	0\\
456	0\\
457	0\\
458	0\\
459	0\\
460	0\\
461	0\\
462	0\\
463	0\\
464	0\\
465	0\\
466	0\\
467	0\\
468	0\\
469	0\\
470	0\\
471	0\\
472	0\\
473	0\\
474	0\\
475	0\\
476	0\\
477	0\\
478	0\\
479	0\\
480	0\\
481	0\\
482	0\\
483	0\\
484	0\\
485	0\\
486	0\\
487	0\\
488	0\\
489	0\\
490	0\\
491	0\\
492	0\\
493	0\\
494	0\\
495	0\\
496	0\\
497	0\\
498	0\\
499	0\\
500	0\\
501	0\\
502	0\\
503	0\\
504	0\\
505	0\\
506	0\\
507	0\\
508	0\\
509	0\\
510	0\\
511	0\\
512	0\\
513	0\\
514	0\\
515	0\\
516	0\\
517	0\\
518	0\\
519	0\\
520	0\\
521	0\\
522	0\\
523	0\\
524	0\\
525	0\\
526	0\\
527	0\\
528	0\\
529	0\\
530	0\\
531	0\\
532	0\\
533	0\\
534	0\\
535	0\\
536	0\\
537	0\\
538	0\\
539	0\\
540	0\\
541	0\\
542	0\\
543	0\\
544	0\\
545	0\\
546	0\\
547	0\\
548	0\\
549	0\\
550	0\\
551	0\\
552	0\\
553	0\\
554	0\\
555	0\\
556	0\\
557	0\\
558	0\\
559	0\\
560	0\\
561	0\\
562	0\\
563	0\\
564	0\\
565	0\\
566	0\\
567	0\\
568	0\\
569	0\\
570	0\\
571	0\\
572	0\\
573	0\\
574	0\\
575	0\\
576	0\\
577	0\\
578	0\\
579	0\\
580	0\\
581	0\\
582	0\\
583	0\\
584	0\\
585	0\\
586	0\\
587	0\\
588	0\\
589	0\\
590	0\\
591	0\\
592	0\\
593	0\\
594	0\\
595	0\\
596	0\\
597	0\\
598	0\\
599	0\\
600	0\\
};
\addplot [color=mycolor3,solid,forget plot]
  table[row sep=crcr]{%
1	0\\
2	0\\
3	0\\
4	0\\
5	0\\
6	0\\
7	0\\
8	0\\
9	0\\
10	0\\
11	0\\
12	0\\
13	0\\
14	0\\
15	0\\
16	0\\
17	0\\
18	0\\
19	0\\
20	0\\
21	0\\
22	0\\
23	0\\
24	0\\
25	0\\
26	0\\
27	0\\
28	0\\
29	0\\
30	0\\
31	0\\
32	0\\
33	0\\
34	0\\
35	0\\
36	0\\
37	0\\
38	0\\
39	0\\
40	0\\
41	0\\
42	0\\
43	0\\
44	0\\
45	0\\
46	0\\
47	0\\
48	0\\
49	0\\
50	0\\
51	0\\
52	0\\
53	0\\
54	0\\
55	0\\
56	0\\
57	0\\
58	0\\
59	0\\
60	0\\
61	0\\
62	0\\
63	0\\
64	0\\
65	0\\
66	0\\
67	0\\
68	0\\
69	0\\
70	0\\
71	0\\
72	0\\
73	0\\
74	0\\
75	0\\
76	0\\
77	0\\
78	0\\
79	0\\
80	0\\
81	0\\
82	0\\
83	0\\
84	0\\
85	0\\
86	0\\
87	0\\
88	0\\
89	0\\
90	0\\
91	0\\
92	0\\
93	0\\
94	0\\
95	0\\
96	0\\
97	0\\
98	0\\
99	0\\
100	0\\
101	0\\
102	0\\
103	0\\
104	0\\
105	0\\
106	0\\
107	0\\
108	0\\
109	0\\
110	0\\
111	0\\
112	0\\
113	0\\
114	0\\
115	0\\
116	0\\
117	0\\
118	0\\
119	0\\
120	0\\
121	0\\
122	0\\
123	0\\
124	0\\
125	0\\
126	0\\
127	0\\
128	0\\
129	0\\
130	0\\
131	0\\
132	0\\
133	0\\
134	0\\
135	0\\
136	0\\
137	0\\
138	0\\
139	0\\
140	0\\
141	0\\
142	0\\
143	0\\
144	0\\
145	0\\
146	0\\
147	0\\
148	0\\
149	0\\
150	0\\
151	0\\
152	0\\
153	0\\
154	0\\
155	0\\
156	0\\
157	0\\
158	0\\
159	0\\
160	0\\
161	0\\
162	0\\
163	0\\
164	0\\
165	0\\
166	0\\
167	0\\
168	0\\
169	0\\
170	0\\
171	0\\
172	0\\
173	0\\
174	0\\
175	0\\
176	0\\
177	0\\
178	0\\
179	0\\
180	0\\
181	0\\
182	0\\
183	0\\
184	0\\
185	0\\
186	0\\
187	0\\
188	0\\
189	0\\
190	0\\
191	0\\
192	0\\
193	0\\
194	0\\
195	0\\
196	0\\
197	0\\
198	0\\
199	0\\
200	0\\
201	0\\
202	0\\
203	0\\
204	0\\
205	0\\
206	0\\
207	0\\
208	0\\
209	0\\
210	0\\
211	0\\
212	0\\
213	0\\
214	0\\
215	0\\
216	0\\
217	0\\
218	0\\
219	0\\
220	0\\
221	0\\
222	0\\
223	0\\
224	0\\
225	0\\
226	0\\
227	0\\
228	0\\
229	0\\
230	0\\
231	0\\
232	0\\
233	0\\
234	0\\
235	0\\
236	0\\
237	0\\
238	0\\
239	0\\
240	0\\
241	0\\
242	0\\
243	0\\
244	0\\
245	0\\
246	0\\
247	0\\
248	0\\
249	0\\
250	0\\
251	0\\
252	0\\
253	0\\
254	0\\
255	0\\
256	0\\
257	0\\
258	0\\
259	0\\
260	0\\
261	0\\
262	0\\
263	0\\
264	0\\
265	0\\
266	0\\
267	0\\
268	0\\
269	0\\
270	0\\
271	0\\
272	0\\
273	0\\
274	0\\
275	0\\
276	0\\
277	0\\
278	0\\
279	0\\
280	0\\
281	0\\
282	0\\
283	0\\
284	0\\
285	0\\
286	0\\
287	0\\
288	0\\
289	0\\
290	0\\
291	0\\
292	0\\
293	0\\
294	0\\
295	0\\
296	0\\
297	0\\
298	0\\
299	0\\
300	0\\
301	0\\
302	0\\
303	0\\
304	0\\
305	0\\
306	0\\
307	0\\
308	0\\
309	0\\
310	0\\
311	0\\
312	0\\
313	0\\
314	0\\
315	0\\
316	0\\
317	0\\
318	0\\
319	0\\
320	0\\
321	0\\
322	0\\
323	0\\
324	0\\
325	0\\
326	0\\
327	0\\
328	0\\
329	0\\
330	0\\
331	0\\
332	0\\
333	0\\
334	0\\
335	0\\
336	0\\
337	0\\
338	0\\
339	0\\
340	0\\
341	0\\
342	0\\
343	0\\
344	0\\
345	0\\
346	0\\
347	0\\
348	0\\
349	0\\
350	0\\
351	0\\
352	0\\
353	0\\
354	0\\
355	0\\
356	0\\
357	0\\
358	0\\
359	0\\
360	0\\
361	0\\
362	0\\
363	0\\
364	0\\
365	0\\
366	0\\
367	0\\
368	0\\
369	0\\
370	0\\
371	0\\
372	0\\
373	0\\
374	0\\
375	0\\
376	0\\
377	0\\
378	0\\
379	0\\
380	0\\
381	0\\
382	0\\
383	0\\
384	0\\
385	0\\
386	0\\
387	0\\
388	0\\
389	0\\
390	0\\
391	0\\
392	0\\
393	0\\
394	0\\
395	0\\
396	0\\
397	0\\
398	0\\
399	0\\
400	0\\
401	0\\
402	0\\
403	0\\
404	0\\
405	0\\
406	0\\
407	0\\
408	0\\
409	0\\
410	0\\
411	0\\
412	0\\
413	0\\
414	0\\
415	0\\
416	0\\
417	0\\
418	0\\
419	0\\
420	0\\
421	0\\
422	0\\
423	0\\
424	0\\
425	0\\
426	0\\
427	0\\
428	0\\
429	0\\
430	0\\
431	0\\
432	0\\
433	0\\
434	0\\
435	0\\
436	0\\
437	0\\
438	0\\
439	0\\
440	0\\
441	0\\
442	0\\
443	0\\
444	0\\
445	0\\
446	0\\
447	0\\
448	0\\
449	0\\
450	0\\
451	0\\
452	0\\
453	0\\
454	0\\
455	0\\
456	0\\
457	0\\
458	0\\
459	0\\
460	0\\
461	0\\
462	0\\
463	0\\
464	0\\
465	0\\
466	0\\
467	0\\
468	0\\
469	0\\
470	0\\
471	0\\
472	0\\
473	0\\
474	0\\
475	0\\
476	0\\
477	0\\
478	0\\
479	0\\
480	0\\
481	0\\
482	0\\
483	0\\
484	0\\
485	0\\
486	0\\
487	0\\
488	0\\
489	0\\
490	0\\
491	0\\
492	0\\
493	0\\
494	0\\
495	0\\
496	0\\
497	0\\
498	0\\
499	0\\
500	0\\
501	0\\
502	0\\
503	0\\
504	0\\
505	0\\
506	0\\
507	0\\
508	0\\
509	0\\
510	0\\
511	0\\
512	0\\
513	0\\
514	0\\
515	0\\
516	0\\
517	0\\
518	0\\
519	0\\
520	0\\
521	0\\
522	0\\
523	0\\
524	0\\
525	0\\
526	0\\
527	0\\
528	0\\
529	0\\
530	0\\
531	0\\
532	0\\
533	0\\
534	0\\
535	0\\
536	0\\
537	0\\
538	0\\
539	0\\
540	0\\
541	0\\
542	0\\
543	0\\
544	0\\
545	0\\
546	0\\
547	0\\
548	0\\
549	0\\
550	0\\
551	0\\
552	0\\
553	0\\
554	0\\
555	0\\
556	0\\
557	0\\
558	0\\
559	0\\
560	0\\
561	0\\
562	0\\
563	0\\
564	0\\
565	0\\
566	0\\
567	0\\
568	0\\
569	0\\
570	0\\
571	0\\
572	0\\
573	0\\
574	0\\
575	0\\
576	0\\
577	0\\
578	0\\
579	0\\
580	0\\
581	0\\
582	0\\
583	0\\
584	0\\
585	0\\
586	0\\
587	0\\
588	0\\
589	0\\
590	0\\
591	0\\
592	0\\
593	0\\
594	0\\
595	0\\
596	0\\
597	0\\
598	0\\
599	0\\
600	0\\
};
\addplot [color=mycolor4,solid,forget plot]
  table[row sep=crcr]{%
1	0\\
2	0\\
3	0\\
4	0\\
5	0\\
6	0\\
7	0\\
8	0\\
9	0\\
10	0\\
11	0\\
12	0\\
13	0\\
14	0\\
15	0\\
16	0\\
17	0\\
18	0\\
19	0\\
20	0\\
21	0\\
22	0\\
23	0\\
24	0\\
25	0\\
26	0\\
27	0\\
28	0\\
29	0\\
30	0\\
31	0\\
32	0\\
33	0\\
34	0\\
35	0\\
36	0\\
37	0\\
38	0\\
39	0\\
40	0\\
41	0\\
42	0\\
43	0\\
44	0\\
45	0\\
46	0\\
47	0\\
48	0\\
49	0\\
50	0\\
51	0\\
52	0\\
53	0\\
54	0\\
55	0\\
56	0\\
57	0\\
58	0\\
59	0\\
60	0\\
61	0\\
62	0\\
63	0\\
64	0\\
65	0\\
66	0\\
67	0\\
68	0\\
69	0\\
70	0\\
71	0\\
72	0\\
73	0\\
74	0\\
75	0\\
76	0\\
77	0\\
78	0\\
79	0\\
80	0\\
81	0\\
82	0\\
83	0\\
84	0\\
85	0\\
86	0\\
87	0\\
88	0\\
89	0\\
90	0\\
91	0\\
92	0\\
93	0\\
94	0\\
95	0\\
96	0\\
97	0\\
98	0\\
99	0\\
100	0\\
101	0\\
102	0\\
103	0\\
104	0\\
105	0\\
106	0\\
107	0\\
108	0\\
109	0\\
110	0\\
111	0\\
112	0\\
113	0\\
114	0\\
115	0\\
116	0\\
117	0\\
118	0\\
119	0\\
120	0\\
121	0\\
122	0\\
123	0\\
124	0\\
125	0\\
126	0\\
127	0\\
128	0\\
129	0\\
130	0\\
131	0\\
132	0\\
133	0\\
134	0\\
135	0\\
136	0\\
137	0\\
138	0\\
139	0\\
140	0\\
141	0\\
142	0\\
143	0\\
144	0\\
145	0\\
146	0\\
147	0\\
148	0\\
149	0\\
150	0\\
151	0\\
152	0\\
153	0\\
154	0\\
155	0\\
156	0\\
157	0\\
158	0\\
159	0\\
160	0\\
161	0\\
162	0\\
163	0\\
164	0\\
165	0\\
166	0\\
167	0\\
168	0\\
169	0\\
170	0\\
171	0\\
172	0\\
173	0\\
174	0\\
175	0\\
176	0\\
177	0\\
178	0\\
179	0\\
180	0\\
181	0\\
182	0\\
183	0\\
184	0\\
185	0\\
186	0\\
187	0\\
188	0\\
189	0\\
190	0\\
191	0\\
192	0\\
193	0\\
194	0\\
195	0\\
196	0\\
197	0\\
198	0\\
199	0\\
200	0\\
201	0\\
202	0\\
203	0\\
204	0\\
205	0\\
206	0\\
207	0\\
208	0\\
209	0\\
210	0\\
211	0\\
212	0\\
213	0\\
214	0\\
215	0\\
216	0\\
217	0\\
218	0\\
219	0\\
220	0\\
221	0\\
222	0\\
223	0\\
224	0\\
225	0\\
226	0\\
227	0\\
228	0\\
229	0\\
230	0\\
231	0\\
232	0\\
233	0\\
234	0\\
235	0\\
236	0\\
237	0\\
238	0\\
239	0\\
240	0\\
241	0\\
242	0\\
243	0\\
244	0\\
245	0\\
246	0\\
247	0\\
248	0\\
249	0\\
250	0\\
251	0\\
252	0\\
253	0\\
254	0\\
255	0\\
256	0\\
257	0\\
258	0\\
259	0\\
260	0\\
261	0\\
262	0\\
263	0\\
264	0\\
265	0\\
266	0\\
267	0\\
268	0\\
269	0\\
270	0\\
271	0\\
272	0\\
273	0\\
274	0\\
275	0\\
276	0\\
277	0\\
278	0\\
279	0\\
280	0\\
281	0\\
282	0\\
283	0\\
284	0\\
285	0\\
286	0\\
287	0\\
288	0\\
289	0\\
290	0\\
291	0\\
292	0\\
293	0\\
294	0\\
295	0\\
296	0\\
297	0\\
298	0\\
299	0\\
300	0\\
301	0\\
302	0\\
303	0\\
304	0\\
305	0\\
306	0\\
307	0\\
308	0\\
309	0\\
310	0\\
311	0\\
312	0\\
313	0\\
314	0\\
315	0\\
316	0\\
317	0\\
318	0\\
319	0\\
320	0\\
321	0\\
322	0\\
323	0\\
324	0\\
325	0\\
326	0\\
327	0\\
328	0\\
329	0\\
330	0\\
331	0\\
332	0\\
333	0\\
334	0\\
335	0\\
336	0\\
337	0\\
338	0\\
339	0\\
340	0\\
341	0\\
342	0\\
343	0\\
344	0\\
345	0\\
346	0\\
347	0\\
348	0\\
349	0\\
350	0\\
351	0\\
352	0\\
353	0\\
354	0\\
355	0\\
356	0\\
357	0\\
358	0\\
359	0\\
360	0\\
361	0\\
362	0\\
363	0\\
364	0\\
365	0\\
366	0\\
367	0\\
368	0\\
369	0\\
370	0\\
371	0\\
372	0\\
373	0\\
374	0\\
375	0\\
376	0\\
377	0\\
378	0\\
379	0\\
380	0\\
381	0\\
382	0\\
383	0\\
384	0\\
385	0\\
386	0\\
387	0\\
388	0\\
389	0\\
390	0\\
391	0\\
392	0\\
393	0\\
394	0\\
395	0\\
396	0\\
397	0\\
398	0\\
399	0\\
400	0\\
401	0\\
402	0\\
403	0\\
404	0\\
405	0\\
406	0\\
407	0\\
408	0\\
409	0\\
410	0\\
411	0\\
412	0\\
413	0\\
414	0\\
415	0\\
416	0\\
417	0\\
418	0\\
419	0\\
420	0\\
421	0\\
422	0\\
423	0\\
424	0\\
425	0\\
426	0\\
427	0\\
428	0\\
429	0\\
430	0\\
431	0\\
432	0\\
433	0\\
434	0\\
435	0\\
436	0\\
437	0\\
438	0\\
439	0\\
440	0\\
441	0\\
442	0\\
443	0\\
444	0\\
445	0\\
446	0\\
447	0\\
448	0\\
449	0\\
450	0\\
451	0\\
452	0\\
453	0\\
454	0\\
455	0\\
456	0\\
457	0\\
458	0\\
459	0\\
460	0\\
461	0\\
462	0\\
463	0\\
464	0\\
465	0\\
466	0\\
467	0\\
468	0\\
469	0\\
470	0\\
471	0\\
472	0\\
473	0\\
474	0\\
475	0\\
476	0\\
477	0\\
478	0\\
479	0\\
480	0\\
481	0\\
482	0\\
483	0\\
484	0\\
485	0\\
486	0\\
487	0\\
488	0\\
489	0\\
490	0\\
491	0\\
492	0\\
493	0\\
494	0\\
495	0\\
496	0\\
497	0\\
498	0\\
499	0\\
500	0\\
501	0\\
502	0\\
503	0\\
504	0\\
505	0\\
506	0\\
507	0\\
508	0\\
509	0\\
510	0\\
511	0\\
512	0\\
513	0\\
514	0\\
515	0\\
516	0\\
517	0\\
518	0\\
519	0\\
520	0\\
521	0\\
522	0\\
523	0\\
524	0\\
525	0\\
526	0\\
527	0\\
528	0\\
529	0\\
530	0\\
531	0\\
532	0\\
533	0\\
534	0\\
535	0\\
536	0\\
537	0\\
538	0\\
539	0\\
540	0\\
541	0\\
542	0\\
543	0\\
544	0\\
545	0\\
546	0\\
547	0\\
548	0\\
549	0\\
550	0\\
551	0\\
552	0\\
553	0\\
554	0\\
555	0\\
556	0\\
557	0\\
558	0\\
559	0\\
560	0\\
561	0\\
562	0\\
563	0\\
564	0\\
565	0\\
566	0\\
567	0\\
568	0\\
569	0\\
570	0\\
571	0\\
572	0\\
573	0\\
574	0\\
575	0\\
576	0\\
577	0\\
578	0\\
579	0\\
580	0\\
581	0\\
582	0\\
583	0\\
584	0\\
585	0\\
586	0\\
587	0\\
588	0\\
589	0\\
590	0\\
591	0\\
592	0\\
593	0\\
594	0\\
595	0\\
596	0\\
597	0\\
598	0\\
599	0\\
600	0\\
};
\addplot [color=mycolor5,solid,forget plot]
  table[row sep=crcr]{%
1	0\\
2	0\\
3	0\\
4	0\\
5	0\\
6	0\\
7	0\\
8	0\\
9	0\\
10	0\\
11	0\\
12	0\\
13	0\\
14	0\\
15	0\\
16	0\\
17	0\\
18	0\\
19	0\\
20	0\\
21	0\\
22	0\\
23	0\\
24	0\\
25	0\\
26	0\\
27	0\\
28	0\\
29	0\\
30	0\\
31	0\\
32	0\\
33	0\\
34	0\\
35	0\\
36	0\\
37	0\\
38	0\\
39	0\\
40	0\\
41	0\\
42	0\\
43	0\\
44	0\\
45	0\\
46	0\\
47	0\\
48	0\\
49	0\\
50	0\\
51	0\\
52	0\\
53	0\\
54	0\\
55	0\\
56	0\\
57	0\\
58	0\\
59	0\\
60	0\\
61	0\\
62	0\\
63	0\\
64	0\\
65	0\\
66	0\\
67	0\\
68	0\\
69	0\\
70	0\\
71	0\\
72	0\\
73	0\\
74	0\\
75	0\\
76	0\\
77	0\\
78	0\\
79	0\\
80	0\\
81	0\\
82	0\\
83	0\\
84	0\\
85	0\\
86	0\\
87	0\\
88	0\\
89	0\\
90	0\\
91	0\\
92	0\\
93	0\\
94	0\\
95	0\\
96	0\\
97	0\\
98	0\\
99	0\\
100	0\\
101	0\\
102	0\\
103	0\\
104	0\\
105	0\\
106	0\\
107	0\\
108	0\\
109	0\\
110	0\\
111	0\\
112	0\\
113	0\\
114	0\\
115	0\\
116	0\\
117	0\\
118	0\\
119	0\\
120	0\\
121	0\\
122	0\\
123	0\\
124	0\\
125	0\\
126	0\\
127	0\\
128	0\\
129	0\\
130	0\\
131	0\\
132	0\\
133	0\\
134	0\\
135	0\\
136	0\\
137	0\\
138	0\\
139	0\\
140	0\\
141	0\\
142	0\\
143	0\\
144	0\\
145	0\\
146	0\\
147	0\\
148	0\\
149	0\\
150	0\\
151	0\\
152	0\\
153	0\\
154	0\\
155	0\\
156	0\\
157	0\\
158	0\\
159	0\\
160	0\\
161	0\\
162	0\\
163	0\\
164	0\\
165	0\\
166	0\\
167	0\\
168	0\\
169	0\\
170	0\\
171	0\\
172	0\\
173	0\\
174	0\\
175	0\\
176	0\\
177	0\\
178	0\\
179	0\\
180	0\\
181	0\\
182	0\\
183	0\\
184	0\\
185	0\\
186	0\\
187	0\\
188	0\\
189	0\\
190	0\\
191	0\\
192	0\\
193	0\\
194	0\\
195	0\\
196	0\\
197	0\\
198	0\\
199	0\\
200	0\\
201	0\\
202	0\\
203	0\\
204	0\\
205	0\\
206	0\\
207	0\\
208	0\\
209	0\\
210	0\\
211	0\\
212	0\\
213	0\\
214	0\\
215	0\\
216	0\\
217	0\\
218	0\\
219	0\\
220	0\\
221	0\\
222	0\\
223	0\\
224	0\\
225	0\\
226	0\\
227	0\\
228	0\\
229	0\\
230	0\\
231	0\\
232	0\\
233	0\\
234	0\\
235	0\\
236	0\\
237	0\\
238	0\\
239	0\\
240	0\\
241	0\\
242	0\\
243	0\\
244	0\\
245	0\\
246	0\\
247	0\\
248	0\\
249	0\\
250	0\\
251	0\\
252	0\\
253	0\\
254	0\\
255	0\\
256	0\\
257	0\\
258	0\\
259	0\\
260	0\\
261	0\\
262	0\\
263	0\\
264	0\\
265	0\\
266	0\\
267	0\\
268	0\\
269	0\\
270	0\\
271	0\\
272	0\\
273	0\\
274	0\\
275	0\\
276	0\\
277	0\\
278	0\\
279	0\\
280	0\\
281	0\\
282	0\\
283	0\\
284	0\\
285	0\\
286	0\\
287	0\\
288	0\\
289	0\\
290	0\\
291	0\\
292	0\\
293	0\\
294	0\\
295	0\\
296	0\\
297	0\\
298	0\\
299	0\\
300	0\\
301	0\\
302	0\\
303	0\\
304	0\\
305	0\\
306	0\\
307	0\\
308	0\\
309	0\\
310	0\\
311	0\\
312	0\\
313	0\\
314	0\\
315	0\\
316	0\\
317	0\\
318	0\\
319	0\\
320	0\\
321	0\\
322	0\\
323	0\\
324	0\\
325	0\\
326	0\\
327	0\\
328	0\\
329	0\\
330	0\\
331	0\\
332	0\\
333	0\\
334	0\\
335	0\\
336	0\\
337	0\\
338	0\\
339	0\\
340	0\\
341	0\\
342	0\\
343	0\\
344	0\\
345	0\\
346	0\\
347	0\\
348	0\\
349	0\\
350	0\\
351	0\\
352	0\\
353	0\\
354	0\\
355	0\\
356	0\\
357	0\\
358	0\\
359	0\\
360	0\\
361	0\\
362	0\\
363	0\\
364	0\\
365	0\\
366	0\\
367	0\\
368	0\\
369	0\\
370	0\\
371	0\\
372	0\\
373	0\\
374	0\\
375	0\\
376	0\\
377	0\\
378	0\\
379	0\\
380	0\\
381	0\\
382	0\\
383	0\\
384	0\\
385	0\\
386	0\\
387	0\\
388	0\\
389	0\\
390	0\\
391	0\\
392	0\\
393	0\\
394	0\\
395	0\\
396	0\\
397	0\\
398	0\\
399	0\\
400	0\\
401	0\\
402	0\\
403	0\\
404	0\\
405	0\\
406	0\\
407	0\\
408	0\\
409	0\\
410	0\\
411	0\\
412	0\\
413	0\\
414	0\\
415	0\\
416	0\\
417	0\\
418	0\\
419	0\\
420	0\\
421	0\\
422	0\\
423	0\\
424	0\\
425	0\\
426	0\\
427	0\\
428	0\\
429	0\\
430	0\\
431	0\\
432	0\\
433	0\\
434	0\\
435	0\\
436	0\\
437	0\\
438	0\\
439	0\\
440	0\\
441	0\\
442	0\\
443	0\\
444	0\\
445	0\\
446	0\\
447	0\\
448	0\\
449	0\\
450	0\\
451	0\\
452	0\\
453	0\\
454	0\\
455	0\\
456	0\\
457	0\\
458	0\\
459	0\\
460	0\\
461	0\\
462	0\\
463	0\\
464	0\\
465	0\\
466	0\\
467	0\\
468	0\\
469	0\\
470	0\\
471	0\\
472	0\\
473	0\\
474	0\\
475	0\\
476	0\\
477	0\\
478	0\\
479	0\\
480	0\\
481	0\\
482	0\\
483	0\\
484	0\\
485	0\\
486	0\\
487	0\\
488	0\\
489	0\\
490	0\\
491	0\\
492	0\\
493	0\\
494	0\\
495	0\\
496	0\\
497	0\\
498	0\\
499	0\\
500	0\\
501	0\\
502	0\\
503	0\\
504	0\\
505	0\\
506	0\\
507	0\\
508	0\\
509	0\\
510	0\\
511	0\\
512	0\\
513	0\\
514	0\\
515	0\\
516	0\\
517	0\\
518	0\\
519	0\\
520	0\\
521	0\\
522	0\\
523	0\\
524	0\\
525	0\\
526	0\\
527	0\\
528	0\\
529	0\\
530	0\\
531	0\\
532	0\\
533	0\\
534	0\\
535	0\\
536	0\\
537	0\\
538	0\\
539	0\\
540	0\\
541	0\\
542	0\\
543	0\\
544	0\\
545	0\\
546	0\\
547	0\\
548	0\\
549	0\\
550	0\\
551	0\\
552	0\\
553	0\\
554	0\\
555	0\\
556	0\\
557	0\\
558	0\\
559	0\\
560	0\\
561	0\\
562	0\\
563	0\\
564	0\\
565	0\\
566	0\\
567	0\\
568	0\\
569	0\\
570	0\\
571	0\\
572	0\\
573	0\\
574	0\\
575	0\\
576	0\\
577	0\\
578	0\\
579	0\\
580	0\\
581	0\\
582	0\\
583	0\\
584	0\\
585	0\\
586	0\\
587	0\\
588	0\\
589	0\\
590	0\\
591	0\\
592	0\\
593	0\\
594	0\\
595	0\\
596	0\\
597	0\\
598	0\\
599	0\\
600	0\\
};
\addplot [color=mycolor6,solid,forget plot]
  table[row sep=crcr]{%
1	0\\
2	0\\
3	0\\
4	0\\
5	0\\
6	0\\
7	0\\
8	0\\
9	0\\
10	0\\
11	0\\
12	0\\
13	0\\
14	0\\
15	0\\
16	0\\
17	0\\
18	0\\
19	0\\
20	0\\
21	0\\
22	0\\
23	0\\
24	0\\
25	0\\
26	0\\
27	0\\
28	0\\
29	0\\
30	0\\
31	0\\
32	0\\
33	0\\
34	0\\
35	0\\
36	0\\
37	0\\
38	0\\
39	0\\
40	0\\
41	0\\
42	0\\
43	0\\
44	0\\
45	0\\
46	0\\
47	0\\
48	0\\
49	0\\
50	0\\
51	0\\
52	0\\
53	0\\
54	0\\
55	0\\
56	0\\
57	0\\
58	0\\
59	0\\
60	0\\
61	0\\
62	0\\
63	0\\
64	0\\
65	0\\
66	0\\
67	0\\
68	0\\
69	0\\
70	0\\
71	0\\
72	0\\
73	0\\
74	0\\
75	0\\
76	0\\
77	0\\
78	0\\
79	0\\
80	0\\
81	0\\
82	0\\
83	0\\
84	0\\
85	0\\
86	0\\
87	0\\
88	0\\
89	0\\
90	0\\
91	0\\
92	0\\
93	0\\
94	0\\
95	0\\
96	0\\
97	0\\
98	0\\
99	0\\
100	0\\
101	0\\
102	0\\
103	0\\
104	0\\
105	0\\
106	0\\
107	0\\
108	0\\
109	0\\
110	0\\
111	0\\
112	0\\
113	0\\
114	0\\
115	0\\
116	0\\
117	0\\
118	0\\
119	0\\
120	0\\
121	0\\
122	0\\
123	0\\
124	0\\
125	0\\
126	0\\
127	0\\
128	0\\
129	0\\
130	0\\
131	0\\
132	0\\
133	0\\
134	0\\
135	0\\
136	0\\
137	0\\
138	0\\
139	0\\
140	0\\
141	0\\
142	0\\
143	0\\
144	0\\
145	0\\
146	0\\
147	0\\
148	0\\
149	0\\
150	0\\
151	0\\
152	0\\
153	0\\
154	0\\
155	0\\
156	0\\
157	0\\
158	0\\
159	0\\
160	0\\
161	0\\
162	0\\
163	0\\
164	0\\
165	0\\
166	0\\
167	0\\
168	0\\
169	0\\
170	0\\
171	0\\
172	0\\
173	0\\
174	0\\
175	0\\
176	0\\
177	0\\
178	0\\
179	0\\
180	0\\
181	0\\
182	0\\
183	0\\
184	0\\
185	0\\
186	0\\
187	0\\
188	0\\
189	0\\
190	0\\
191	0\\
192	0\\
193	0\\
194	0\\
195	0\\
196	0\\
197	0\\
198	0\\
199	0\\
200	0\\
201	0\\
202	0\\
203	0\\
204	0\\
205	0\\
206	0\\
207	0\\
208	0\\
209	0\\
210	0\\
211	0\\
212	0\\
213	0\\
214	0\\
215	0\\
216	0\\
217	0\\
218	0\\
219	0\\
220	0\\
221	0\\
222	0\\
223	0\\
224	0\\
225	0\\
226	0\\
227	0\\
228	0\\
229	0\\
230	0\\
231	0\\
232	0\\
233	0\\
234	0\\
235	0\\
236	0\\
237	0\\
238	0\\
239	0\\
240	0\\
241	0\\
242	0\\
243	0\\
244	0\\
245	0\\
246	0\\
247	0\\
248	0\\
249	0\\
250	0\\
251	0\\
252	0\\
253	0\\
254	0\\
255	0\\
256	0\\
257	0\\
258	0\\
259	0\\
260	0\\
261	0\\
262	0\\
263	0\\
264	0\\
265	0\\
266	0\\
267	0\\
268	0\\
269	0\\
270	0\\
271	0\\
272	0\\
273	0\\
274	0\\
275	0\\
276	0\\
277	0\\
278	0\\
279	0\\
280	0\\
281	0\\
282	0\\
283	0\\
284	0\\
285	0\\
286	0\\
287	0\\
288	0\\
289	0\\
290	0\\
291	0\\
292	0\\
293	0\\
294	0\\
295	0\\
296	0\\
297	0\\
298	0\\
299	0\\
300	0\\
301	0\\
302	0\\
303	0\\
304	0\\
305	0\\
306	0\\
307	0\\
308	0\\
309	0\\
310	0\\
311	0\\
312	0\\
313	0\\
314	0\\
315	0\\
316	0\\
317	0\\
318	0\\
319	0\\
320	0\\
321	0\\
322	0\\
323	0\\
324	0\\
325	0\\
326	0\\
327	0\\
328	0\\
329	0\\
330	0\\
331	0\\
332	0\\
333	0\\
334	0\\
335	0\\
336	0\\
337	0\\
338	0\\
339	0\\
340	0\\
341	0\\
342	0\\
343	0\\
344	0\\
345	0\\
346	0\\
347	0\\
348	0\\
349	0\\
350	0\\
351	0\\
352	0\\
353	0\\
354	0\\
355	0\\
356	0\\
357	0\\
358	0\\
359	0\\
360	0\\
361	0\\
362	0\\
363	0\\
364	0\\
365	0\\
366	0\\
367	0\\
368	0\\
369	0\\
370	0\\
371	0\\
372	0\\
373	0\\
374	0\\
375	0\\
376	0\\
377	0\\
378	0\\
379	0\\
380	0\\
381	0\\
382	0\\
383	0\\
384	0\\
385	0\\
386	0\\
387	0\\
388	0\\
389	0\\
390	0\\
391	0\\
392	0\\
393	0\\
394	0\\
395	0\\
396	0\\
397	0\\
398	0\\
399	0\\
400	0\\
401	0\\
402	0\\
403	0\\
404	0\\
405	0\\
406	0\\
407	0\\
408	0\\
409	0\\
410	0\\
411	0\\
412	0\\
413	0\\
414	0\\
415	0\\
416	0\\
417	0\\
418	0\\
419	0\\
420	0\\
421	0\\
422	0\\
423	0\\
424	0\\
425	0\\
426	0\\
427	0\\
428	0\\
429	0\\
430	0\\
431	0\\
432	0\\
433	0\\
434	0\\
435	0\\
436	0\\
437	0\\
438	0\\
439	0\\
440	0\\
441	0\\
442	0\\
443	0\\
444	0\\
445	0\\
446	0\\
447	0\\
448	0\\
449	0\\
450	0\\
451	0\\
452	0\\
453	0\\
454	0\\
455	0\\
456	0\\
457	0\\
458	0\\
459	0\\
460	0\\
461	0\\
462	0\\
463	0\\
464	0\\
465	0\\
466	0\\
467	0\\
468	0\\
469	0\\
470	0\\
471	0\\
472	0\\
473	0\\
474	0\\
475	0\\
476	0\\
477	0\\
478	0\\
479	0\\
480	0\\
481	0\\
482	0\\
483	0\\
484	0\\
485	0\\
486	0\\
487	0\\
488	0\\
489	0\\
490	0\\
491	0\\
492	0\\
493	0\\
494	0\\
495	0\\
496	0\\
497	0\\
498	0\\
499	0\\
500	0\\
501	0\\
502	0\\
503	0\\
504	0\\
505	0\\
506	0\\
507	0\\
508	0\\
509	0\\
510	0\\
511	0\\
512	0\\
513	0\\
514	0\\
515	0\\
516	0\\
517	0\\
518	0\\
519	0\\
520	0\\
521	0\\
522	0\\
523	0\\
524	0\\
525	0\\
526	0\\
527	0\\
528	0\\
529	0\\
530	0\\
531	0\\
532	0\\
533	0\\
534	0\\
535	0\\
536	0\\
537	0\\
538	0\\
539	0\\
540	0\\
541	0\\
542	0\\
543	0\\
544	0\\
545	0\\
546	0\\
547	0\\
548	0\\
549	0\\
550	0\\
551	0\\
552	0\\
553	0\\
554	0\\
555	0\\
556	0\\
557	0\\
558	0\\
559	0\\
560	0\\
561	0\\
562	0\\
563	0\\
564	0\\
565	0\\
566	0\\
567	0\\
568	0\\
569	0\\
570	0\\
571	0\\
572	0\\
573	0\\
574	0\\
575	0\\
576	0\\
577	0\\
578	0\\
579	0\\
580	0\\
581	0\\
582	0\\
583	0\\
584	0\\
585	0\\
586	0\\
587	0\\
588	0\\
589	0\\
590	0\\
591	0\\
592	0\\
593	0\\
594	0\\
595	0\\
596	0\\
597	0\\
598	0\\
599	0\\
600	0\\
};
\addplot [color=mycolor7,solid,forget plot]
  table[row sep=crcr]{%
1	0\\
2	0\\
3	0\\
4	0\\
5	0\\
6	0\\
7	0\\
8	0\\
9	0\\
10	0\\
11	0\\
12	0\\
13	0\\
14	0\\
15	0\\
16	0\\
17	0\\
18	0\\
19	0\\
20	0\\
21	0\\
22	0\\
23	0\\
24	0\\
25	0\\
26	0\\
27	0\\
28	0\\
29	0\\
30	0\\
31	0\\
32	0\\
33	0\\
34	0\\
35	0\\
36	0\\
37	0\\
38	0\\
39	0\\
40	0\\
41	0\\
42	0\\
43	0\\
44	0\\
45	0\\
46	0\\
47	0\\
48	0\\
49	0\\
50	0\\
51	0\\
52	0\\
53	0\\
54	0\\
55	0\\
56	0\\
57	0\\
58	0\\
59	0\\
60	0\\
61	0\\
62	0\\
63	0\\
64	0\\
65	0\\
66	0\\
67	0\\
68	0\\
69	0\\
70	0\\
71	0\\
72	0\\
73	0\\
74	0\\
75	0\\
76	0\\
77	0\\
78	0\\
79	0\\
80	0\\
81	0\\
82	0\\
83	0\\
84	0\\
85	0\\
86	0\\
87	0\\
88	0\\
89	0\\
90	0\\
91	0\\
92	0\\
93	0\\
94	0\\
95	0\\
96	0\\
97	0\\
98	0\\
99	0\\
100	0\\
101	0\\
102	0\\
103	0\\
104	0\\
105	0\\
106	0\\
107	0\\
108	0\\
109	0\\
110	0\\
111	0\\
112	0\\
113	0\\
114	0\\
115	0\\
116	0\\
117	0\\
118	0\\
119	0\\
120	0\\
121	0\\
122	0\\
123	0\\
124	0\\
125	0\\
126	0\\
127	0\\
128	0\\
129	0\\
130	0\\
131	0\\
132	0\\
133	0\\
134	0\\
135	0\\
136	0\\
137	0\\
138	0\\
139	0\\
140	0\\
141	0\\
142	0\\
143	0\\
144	0\\
145	0\\
146	0\\
147	0\\
148	0\\
149	0\\
150	0\\
151	0\\
152	0\\
153	0\\
154	0\\
155	0\\
156	0\\
157	0\\
158	0\\
159	0\\
160	0\\
161	0\\
162	0\\
163	0\\
164	0\\
165	0\\
166	0\\
167	0\\
168	0\\
169	0\\
170	0\\
171	0\\
172	0\\
173	0\\
174	0\\
175	0\\
176	0\\
177	0\\
178	0\\
179	0\\
180	0\\
181	0\\
182	0\\
183	0\\
184	0\\
185	0\\
186	0\\
187	0\\
188	0\\
189	0\\
190	0\\
191	0\\
192	0\\
193	0\\
194	0\\
195	0\\
196	0\\
197	0\\
198	0\\
199	0\\
200	0\\
201	0\\
202	0\\
203	0\\
204	0\\
205	0\\
206	0\\
207	0\\
208	0\\
209	0\\
210	0\\
211	0\\
212	0\\
213	0\\
214	0\\
215	0\\
216	0\\
217	0\\
218	0\\
219	0\\
220	0\\
221	0\\
222	0\\
223	0\\
224	0\\
225	0\\
226	0\\
227	0\\
228	0\\
229	0\\
230	0\\
231	0\\
232	0\\
233	0\\
234	0\\
235	0\\
236	0\\
237	0\\
238	0\\
239	0\\
240	0\\
241	0\\
242	0\\
243	0\\
244	0\\
245	0\\
246	0\\
247	0\\
248	0\\
249	0\\
250	0\\
251	0\\
252	0\\
253	0\\
254	0\\
255	0\\
256	0\\
257	0\\
258	0\\
259	0\\
260	0\\
261	0\\
262	0\\
263	0\\
264	0\\
265	0\\
266	0\\
267	0\\
268	0\\
269	0\\
270	0\\
271	0\\
272	0\\
273	0\\
274	0\\
275	0\\
276	0\\
277	0\\
278	0\\
279	0\\
280	0\\
281	0\\
282	0\\
283	0\\
284	0\\
285	0\\
286	0\\
287	0\\
288	0\\
289	0\\
290	0\\
291	0\\
292	0\\
293	0\\
294	0\\
295	0\\
296	0\\
297	0\\
298	0\\
299	0\\
300	0\\
301	0\\
302	0\\
303	0\\
304	0\\
305	0\\
306	0\\
307	0\\
308	0\\
309	0\\
310	0\\
311	0\\
312	0\\
313	0\\
314	0\\
315	0\\
316	0\\
317	0\\
318	0\\
319	0\\
320	0\\
321	0\\
322	0\\
323	0\\
324	0\\
325	0\\
326	0\\
327	0\\
328	0\\
329	0\\
330	0\\
331	0\\
332	0\\
333	0\\
334	0\\
335	0\\
336	0\\
337	0\\
338	0\\
339	0\\
340	0\\
341	0\\
342	0\\
343	0\\
344	0\\
345	0\\
346	0\\
347	0\\
348	0\\
349	0\\
350	0\\
351	0\\
352	0\\
353	0\\
354	0\\
355	0\\
356	0\\
357	0\\
358	0\\
359	0\\
360	0\\
361	0\\
362	0\\
363	0\\
364	0\\
365	0\\
366	0\\
367	0\\
368	0\\
369	0\\
370	0\\
371	0\\
372	0\\
373	0\\
374	0\\
375	0\\
376	0\\
377	0\\
378	0\\
379	0\\
380	0\\
381	0\\
382	0\\
383	0\\
384	0\\
385	0\\
386	0\\
387	0\\
388	0\\
389	0\\
390	0\\
391	0\\
392	0\\
393	0\\
394	0\\
395	0\\
396	0\\
397	0\\
398	0\\
399	0\\
400	0\\
401	0\\
402	0\\
403	0\\
404	0\\
405	0\\
406	0\\
407	0\\
408	0\\
409	0\\
410	0\\
411	0\\
412	0\\
413	0\\
414	0\\
415	0\\
416	0\\
417	0\\
418	0\\
419	0\\
420	0\\
421	0\\
422	0\\
423	0\\
424	0\\
425	0\\
426	0\\
427	0\\
428	0\\
429	0\\
430	0\\
431	0\\
432	0\\
433	0\\
434	0\\
435	0\\
436	0\\
437	0\\
438	0\\
439	0\\
440	0\\
441	0\\
442	0\\
443	0\\
444	0\\
445	0\\
446	0\\
447	0\\
448	0\\
449	0\\
450	0\\
451	0\\
452	0\\
453	0\\
454	0\\
455	0\\
456	0\\
457	0\\
458	0\\
459	0\\
460	0\\
461	0\\
462	0\\
463	0\\
464	0\\
465	0\\
466	0\\
467	0\\
468	0\\
469	0\\
470	0\\
471	0\\
472	0\\
473	0\\
474	0\\
475	0\\
476	0\\
477	0\\
478	0\\
479	0\\
480	0\\
481	0\\
482	0\\
483	0\\
484	0\\
485	0\\
486	0\\
487	0\\
488	0\\
489	0\\
490	0\\
491	0\\
492	0\\
493	0\\
494	0\\
495	0\\
496	0\\
497	0\\
498	0\\
499	0\\
500	0\\
501	0\\
502	0\\
503	0\\
504	0\\
505	0\\
506	0\\
507	0\\
508	0\\
509	0\\
510	0\\
511	0\\
512	0\\
513	0\\
514	0\\
515	0\\
516	0\\
517	0\\
518	0\\
519	0\\
520	0\\
521	0\\
522	0\\
523	0\\
524	0\\
525	0\\
526	0\\
527	0\\
528	0\\
529	0\\
530	0\\
531	0\\
532	0\\
533	0\\
534	0\\
535	0\\
536	0\\
537	0\\
538	0\\
539	0\\
540	0\\
541	0\\
542	0\\
543	0\\
544	0\\
545	0\\
546	0\\
547	0\\
548	0\\
549	0\\
550	0\\
551	0\\
552	0\\
553	0\\
554	0\\
555	0\\
556	0\\
557	0\\
558	0\\
559	0\\
560	0\\
561	0\\
562	0\\
563	0\\
564	0\\
565	0\\
566	0\\
567	0\\
568	0\\
569	0\\
570	0\\
571	0\\
572	0\\
573	0\\
574	0\\
575	0\\
576	0\\
577	0\\
578	0\\
579	0\\
580	0\\
581	0\\
582	0\\
583	0\\
584	0\\
585	0\\
586	0\\
587	0\\
588	0\\
589	0\\
590	0\\
591	0\\
592	0\\
593	0\\
594	0\\
595	0\\
596	0\\
597	0\\
598	0\\
599	0\\
600	0\\
};
\addplot [color=mycolor8,solid,forget plot]
  table[row sep=crcr]{%
1	0\\
2	0\\
3	0\\
4	0\\
5	0\\
6	0\\
7	0\\
8	0\\
9	0\\
10	0\\
11	0\\
12	0\\
13	0\\
14	0\\
15	0\\
16	0\\
17	0\\
18	0\\
19	0\\
20	0\\
21	0\\
22	0\\
23	0\\
24	0\\
25	0\\
26	0\\
27	0\\
28	0\\
29	0\\
30	0\\
31	0\\
32	0\\
33	0\\
34	0\\
35	0\\
36	0\\
37	0\\
38	0\\
39	0\\
40	0\\
41	0\\
42	0\\
43	0\\
44	0\\
45	0\\
46	0\\
47	0\\
48	0\\
49	0\\
50	0\\
51	0\\
52	0\\
53	0\\
54	0\\
55	0\\
56	0\\
57	0\\
58	0\\
59	0\\
60	0\\
61	0\\
62	0\\
63	0\\
64	0\\
65	0\\
66	0\\
67	0\\
68	0\\
69	0\\
70	0\\
71	0\\
72	0\\
73	0\\
74	0\\
75	0\\
76	0\\
77	0\\
78	0\\
79	0\\
80	0\\
81	0\\
82	0\\
83	0\\
84	0\\
85	0\\
86	0\\
87	0\\
88	0\\
89	0\\
90	0\\
91	0\\
92	0\\
93	0\\
94	0\\
95	0\\
96	0\\
97	0\\
98	0\\
99	0\\
100	0\\
101	0\\
102	0\\
103	0\\
104	0\\
105	0\\
106	0\\
107	0\\
108	0\\
109	0\\
110	0\\
111	0\\
112	0\\
113	0\\
114	0\\
115	0\\
116	0\\
117	0\\
118	0\\
119	0\\
120	0\\
121	0\\
122	0\\
123	0\\
124	0\\
125	0\\
126	0\\
127	0\\
128	0\\
129	0\\
130	0\\
131	0\\
132	0\\
133	0\\
134	0\\
135	0\\
136	0\\
137	0\\
138	0\\
139	0\\
140	0\\
141	0\\
142	0\\
143	0\\
144	0\\
145	0\\
146	0\\
147	0\\
148	0\\
149	0\\
150	0\\
151	0\\
152	0\\
153	0\\
154	0\\
155	0\\
156	0\\
157	0\\
158	0\\
159	0\\
160	0\\
161	0\\
162	0\\
163	0\\
164	0\\
165	0\\
166	0\\
167	0\\
168	0\\
169	0\\
170	0\\
171	0\\
172	0\\
173	0\\
174	0\\
175	0\\
176	0\\
177	0\\
178	0\\
179	0\\
180	0\\
181	0\\
182	0\\
183	0\\
184	0\\
185	0\\
186	0\\
187	0\\
188	0\\
189	0\\
190	0\\
191	0\\
192	0\\
193	0\\
194	0\\
195	0\\
196	0\\
197	0\\
198	0\\
199	0\\
200	0\\
201	0\\
202	0\\
203	0\\
204	0\\
205	0\\
206	0\\
207	0\\
208	0\\
209	0\\
210	0\\
211	0\\
212	0\\
213	0\\
214	0\\
215	0\\
216	0\\
217	0\\
218	0\\
219	0\\
220	0\\
221	0\\
222	0\\
223	0\\
224	0\\
225	0\\
226	0\\
227	0\\
228	0\\
229	0\\
230	0\\
231	0\\
232	0\\
233	0\\
234	0\\
235	0\\
236	0\\
237	0\\
238	0\\
239	0\\
240	0\\
241	0\\
242	0\\
243	0\\
244	0\\
245	0\\
246	0\\
247	0\\
248	0\\
249	0\\
250	0\\
251	0\\
252	0\\
253	0\\
254	0\\
255	0\\
256	0\\
257	0\\
258	0\\
259	0\\
260	0\\
261	0\\
262	0\\
263	0\\
264	0\\
265	0\\
266	0\\
267	0\\
268	0\\
269	0\\
270	0\\
271	0\\
272	0\\
273	0\\
274	0\\
275	0\\
276	0\\
277	0\\
278	0\\
279	0\\
280	0\\
281	0\\
282	0\\
283	0\\
284	0\\
285	0\\
286	0\\
287	0\\
288	0\\
289	0\\
290	0\\
291	0\\
292	0\\
293	0\\
294	0\\
295	0\\
296	0\\
297	0\\
298	0\\
299	0\\
300	0\\
301	0\\
302	0\\
303	0\\
304	0\\
305	0\\
306	0\\
307	0\\
308	0\\
309	0\\
310	0\\
311	0\\
312	0\\
313	0\\
314	0\\
315	0\\
316	0\\
317	0\\
318	0\\
319	0\\
320	0\\
321	0\\
322	0\\
323	0\\
324	0\\
325	0\\
326	0\\
327	0\\
328	0\\
329	0\\
330	0\\
331	0\\
332	0\\
333	0\\
334	0\\
335	0\\
336	0\\
337	0\\
338	0\\
339	0\\
340	0\\
341	0\\
342	0\\
343	0\\
344	0\\
345	0\\
346	0\\
347	0\\
348	0\\
349	0\\
350	0\\
351	0\\
352	0\\
353	0\\
354	0\\
355	0\\
356	0\\
357	0\\
358	0\\
359	0\\
360	0\\
361	0\\
362	0\\
363	0\\
364	0\\
365	0\\
366	0\\
367	0\\
368	0\\
369	0\\
370	0\\
371	0\\
372	0\\
373	0\\
374	0\\
375	0\\
376	0\\
377	0\\
378	0\\
379	0\\
380	0\\
381	0\\
382	0\\
383	0\\
384	0\\
385	0\\
386	0\\
387	0\\
388	0\\
389	0\\
390	0\\
391	0\\
392	0\\
393	0\\
394	0\\
395	0\\
396	0\\
397	0\\
398	0\\
399	0\\
400	0\\
401	0\\
402	0\\
403	0\\
404	0\\
405	0\\
406	0\\
407	0\\
408	0\\
409	0\\
410	0\\
411	0\\
412	0\\
413	0\\
414	0\\
415	0\\
416	0\\
417	0\\
418	0\\
419	0\\
420	0\\
421	0\\
422	0\\
423	0\\
424	0\\
425	0\\
426	0\\
427	0\\
428	0\\
429	0\\
430	0\\
431	0\\
432	0\\
433	0\\
434	0\\
435	0\\
436	0\\
437	0\\
438	0\\
439	0\\
440	0\\
441	0\\
442	0\\
443	0\\
444	0\\
445	0\\
446	0\\
447	0\\
448	0\\
449	0\\
450	0\\
451	0\\
452	0\\
453	0\\
454	0\\
455	0\\
456	0\\
457	0\\
458	0\\
459	0\\
460	0\\
461	0\\
462	0\\
463	0\\
464	0\\
465	0\\
466	0\\
467	0\\
468	0\\
469	0\\
470	0\\
471	0\\
472	0\\
473	0\\
474	0\\
475	0\\
476	0\\
477	0\\
478	0\\
479	0\\
480	0\\
481	0\\
482	0\\
483	0\\
484	0\\
485	0\\
486	0\\
487	0\\
488	0\\
489	0\\
490	0\\
491	0\\
492	0\\
493	0\\
494	0\\
495	0\\
496	0\\
497	0\\
498	0\\
499	0\\
500	0\\
501	0\\
502	0\\
503	0\\
504	0\\
505	0\\
506	0\\
507	0\\
508	0\\
509	0\\
510	0\\
511	0\\
512	0\\
513	0\\
514	0\\
515	0\\
516	0\\
517	0\\
518	0\\
519	0\\
520	0\\
521	0\\
522	0\\
523	0\\
524	0\\
525	0\\
526	0\\
527	0\\
528	0\\
529	0\\
530	0\\
531	0\\
532	0\\
533	0\\
534	0\\
535	0\\
536	0\\
537	0\\
538	0\\
539	0\\
540	0\\
541	0\\
542	0\\
543	0\\
544	0\\
545	0\\
546	0\\
547	0\\
548	0\\
549	0\\
550	0\\
551	0\\
552	0\\
553	0\\
554	0\\
555	0\\
556	0\\
557	0\\
558	0\\
559	0\\
560	0\\
561	0\\
562	0\\
563	0\\
564	0\\
565	0\\
566	0\\
567	0\\
568	0\\
569	0\\
570	0\\
571	0\\
572	0\\
573	0\\
574	0\\
575	0\\
576	0\\
577	0\\
578	0\\
579	0\\
580	0\\
581	0\\
582	0\\
583	0\\
584	0\\
585	0\\
586	0\\
587	0\\
588	0\\
589	0\\
590	0\\
591	0\\
592	0\\
593	0\\
594	0\\
595	0\\
596	0\\
597	0\\
598	0\\
599	0\\
600	0\\
};
\addplot [color=blue!25!mycolor7,solid,forget plot]
  table[row sep=crcr]{%
1	0\\
2	0\\
3	0\\
4	0\\
5	0\\
6	0\\
7	0\\
8	0\\
9	0\\
10	0\\
11	0\\
12	0\\
13	0\\
14	0\\
15	0\\
16	0\\
17	0\\
18	0\\
19	0\\
20	0\\
21	0\\
22	0\\
23	0\\
24	0\\
25	0\\
26	0\\
27	0\\
28	0\\
29	0\\
30	0\\
31	0\\
32	0\\
33	0\\
34	0\\
35	0\\
36	0\\
37	0\\
38	0\\
39	0\\
40	0\\
41	0\\
42	0\\
43	0\\
44	0\\
45	0\\
46	0\\
47	0\\
48	0\\
49	0\\
50	0\\
51	0\\
52	0\\
53	0\\
54	0\\
55	0\\
56	0\\
57	0\\
58	0\\
59	0\\
60	0\\
61	0\\
62	0\\
63	0\\
64	0\\
65	0\\
66	0\\
67	0\\
68	0\\
69	0\\
70	0\\
71	0\\
72	0\\
73	0\\
74	0\\
75	0\\
76	0\\
77	0\\
78	0\\
79	0\\
80	0\\
81	0\\
82	0\\
83	0\\
84	0\\
85	0\\
86	0\\
87	0\\
88	0\\
89	0\\
90	0\\
91	0\\
92	0\\
93	0\\
94	0\\
95	0\\
96	0\\
97	0\\
98	0\\
99	0\\
100	0\\
101	0\\
102	0\\
103	0\\
104	0\\
105	0\\
106	0\\
107	0\\
108	0\\
109	0\\
110	0\\
111	0\\
112	0\\
113	0\\
114	0\\
115	0\\
116	0\\
117	0\\
118	0\\
119	0\\
120	0\\
121	0\\
122	0\\
123	0\\
124	0\\
125	0\\
126	0\\
127	0\\
128	0\\
129	0\\
130	0\\
131	0\\
132	0\\
133	0\\
134	0\\
135	0\\
136	0\\
137	0\\
138	0\\
139	0\\
140	0\\
141	0\\
142	0\\
143	0\\
144	0\\
145	0\\
146	0\\
147	0\\
148	0\\
149	0\\
150	0\\
151	0\\
152	0\\
153	0\\
154	0\\
155	0\\
156	0\\
157	0\\
158	0\\
159	0\\
160	0\\
161	0\\
162	0\\
163	0\\
164	0\\
165	0\\
166	0\\
167	0\\
168	0\\
169	0\\
170	0\\
171	0\\
172	0\\
173	0\\
174	0\\
175	0\\
176	0\\
177	0\\
178	0\\
179	0\\
180	0\\
181	0\\
182	0\\
183	0\\
184	0\\
185	0\\
186	0\\
187	0\\
188	0\\
189	0\\
190	0\\
191	0\\
192	0\\
193	0\\
194	0\\
195	0\\
196	0\\
197	0\\
198	0\\
199	0\\
200	0\\
201	0\\
202	0\\
203	0\\
204	0\\
205	0\\
206	0\\
207	0\\
208	0\\
209	0\\
210	0\\
211	0\\
212	0\\
213	0\\
214	0\\
215	0\\
216	0\\
217	0\\
218	0\\
219	0\\
220	0\\
221	0\\
222	0\\
223	0\\
224	0\\
225	0\\
226	0\\
227	0\\
228	0\\
229	0\\
230	0\\
231	0\\
232	0\\
233	0\\
234	0\\
235	0\\
236	0\\
237	0\\
238	0\\
239	0\\
240	0\\
241	0\\
242	0\\
243	0\\
244	0\\
245	0\\
246	0\\
247	0\\
248	0\\
249	0\\
250	0\\
251	0\\
252	0\\
253	0\\
254	0\\
255	0\\
256	0\\
257	0\\
258	0\\
259	0\\
260	0\\
261	0\\
262	0\\
263	0\\
264	0\\
265	0\\
266	0\\
267	0\\
268	0\\
269	0\\
270	0\\
271	0\\
272	0\\
273	0\\
274	0\\
275	0\\
276	0\\
277	0\\
278	0\\
279	0\\
280	0\\
281	0\\
282	0\\
283	0\\
284	0\\
285	0\\
286	0\\
287	0\\
288	0\\
289	0\\
290	0\\
291	0\\
292	0\\
293	0\\
294	0\\
295	0\\
296	0\\
297	0\\
298	0\\
299	0\\
300	0\\
301	0\\
302	0\\
303	0\\
304	0\\
305	0\\
306	0\\
307	0\\
308	0\\
309	0\\
310	0\\
311	0\\
312	0\\
313	0\\
314	0\\
315	0\\
316	0\\
317	0\\
318	0\\
319	0\\
320	0\\
321	0\\
322	0\\
323	0\\
324	0\\
325	0\\
326	0\\
327	0\\
328	0\\
329	0\\
330	0\\
331	0\\
332	0\\
333	0\\
334	0\\
335	0\\
336	0\\
337	0\\
338	0\\
339	0\\
340	0\\
341	0\\
342	0\\
343	0\\
344	0\\
345	0\\
346	0\\
347	0\\
348	0\\
349	0\\
350	0\\
351	0\\
352	0\\
353	0\\
354	0\\
355	0\\
356	0\\
357	0\\
358	0\\
359	0\\
360	0\\
361	0\\
362	0\\
363	0\\
364	0\\
365	0\\
366	0\\
367	0\\
368	0\\
369	0\\
370	0\\
371	0\\
372	0\\
373	0\\
374	0\\
375	0\\
376	0\\
377	0\\
378	0\\
379	0\\
380	0\\
381	0\\
382	0\\
383	0\\
384	0\\
385	0\\
386	0\\
387	0\\
388	0\\
389	0\\
390	0\\
391	0\\
392	0\\
393	0\\
394	0\\
395	0\\
396	0\\
397	0\\
398	0\\
399	0\\
400	0\\
401	0\\
402	0\\
403	0\\
404	0\\
405	0\\
406	0\\
407	0\\
408	0\\
409	0\\
410	0\\
411	0\\
412	0\\
413	0\\
414	0\\
415	0\\
416	0\\
417	0\\
418	0\\
419	0\\
420	0\\
421	0\\
422	0\\
423	0\\
424	0\\
425	0\\
426	0\\
427	0\\
428	0\\
429	0\\
430	0\\
431	0\\
432	0\\
433	0\\
434	0\\
435	0\\
436	0\\
437	0\\
438	0\\
439	0\\
440	0\\
441	0\\
442	0\\
443	0\\
444	0\\
445	0\\
446	0\\
447	0\\
448	0\\
449	0\\
450	0\\
451	0\\
452	0\\
453	0\\
454	0\\
455	0\\
456	0\\
457	0\\
458	0\\
459	0\\
460	0\\
461	0\\
462	0\\
463	0\\
464	0\\
465	0\\
466	0\\
467	0\\
468	0\\
469	0\\
470	0\\
471	0\\
472	0\\
473	0\\
474	0\\
475	0\\
476	0\\
477	0\\
478	0\\
479	0\\
480	0\\
481	0\\
482	0\\
483	0\\
484	0\\
485	0\\
486	0\\
487	0\\
488	0\\
489	0\\
490	0\\
491	0\\
492	0\\
493	0\\
494	0\\
495	0\\
496	0\\
497	0\\
498	0\\
499	0\\
500	0\\
501	0\\
502	0\\
503	0\\
504	0\\
505	0\\
506	0\\
507	0\\
508	0\\
509	0\\
510	0\\
511	0\\
512	0\\
513	0\\
514	0\\
515	0\\
516	0\\
517	0\\
518	0\\
519	0\\
520	0\\
521	0\\
522	0\\
523	0\\
524	0\\
525	0\\
526	0\\
527	0\\
528	0\\
529	0\\
530	0\\
531	0\\
532	0\\
533	0\\
534	0\\
535	0\\
536	0\\
537	0\\
538	0\\
539	0\\
540	0\\
541	0\\
542	0\\
543	0\\
544	0\\
545	0\\
546	0\\
547	0\\
548	0\\
549	0\\
550	0\\
551	0\\
552	0\\
553	0\\
554	0\\
555	0\\
556	0\\
557	0\\
558	0\\
559	0\\
560	0\\
561	0\\
562	0\\
563	0\\
564	0\\
565	0\\
566	0\\
567	0\\
568	0\\
569	0\\
570	0\\
571	0\\
572	0\\
573	0\\
574	0\\
575	0\\
576	0\\
577	0\\
578	0\\
579	0\\
580	0\\
581	0\\
582	0\\
583	0\\
584	0\\
585	0\\
586	0\\
587	0\\
588	0\\
589	0\\
590	0\\
591	0\\
592	0\\
593	0\\
594	0\\
595	0\\
596	0\\
597	0\\
598	0\\
599	0\\
600	0\\
};
\addplot [color=mycolor9,solid,forget plot]
  table[row sep=crcr]{%
1	0\\
2	0\\
3	0\\
4	0\\
5	0\\
6	0\\
7	0\\
8	0\\
9	0\\
10	0\\
11	0\\
12	0\\
13	0\\
14	0\\
15	0\\
16	0\\
17	0\\
18	0\\
19	0\\
20	0\\
21	0\\
22	0\\
23	0\\
24	0\\
25	0\\
26	0\\
27	0\\
28	0\\
29	0\\
30	0\\
31	0\\
32	0\\
33	0\\
34	0\\
35	0\\
36	0\\
37	0\\
38	0\\
39	0\\
40	0\\
41	0\\
42	0\\
43	0\\
44	0\\
45	0\\
46	0\\
47	0\\
48	0\\
49	0\\
50	0\\
51	0\\
52	0\\
53	0\\
54	0\\
55	0\\
56	0\\
57	0\\
58	0\\
59	0\\
60	0\\
61	0\\
62	0\\
63	0\\
64	0\\
65	0\\
66	0\\
67	0\\
68	0\\
69	0\\
70	0\\
71	0\\
72	0\\
73	0\\
74	0\\
75	0\\
76	0\\
77	0\\
78	0\\
79	0\\
80	0\\
81	0\\
82	0\\
83	0\\
84	0\\
85	0\\
86	0\\
87	0\\
88	0\\
89	0\\
90	0\\
91	0\\
92	0\\
93	0\\
94	0\\
95	0\\
96	0\\
97	0\\
98	0\\
99	0\\
100	0\\
101	0\\
102	0\\
103	0\\
104	0\\
105	0\\
106	0\\
107	0\\
108	0\\
109	0\\
110	0\\
111	0\\
112	0\\
113	0\\
114	0\\
115	0\\
116	0\\
117	0\\
118	0\\
119	0\\
120	0\\
121	0\\
122	0\\
123	0\\
124	0\\
125	0\\
126	0\\
127	0\\
128	0\\
129	0\\
130	0\\
131	0\\
132	0\\
133	0\\
134	0\\
135	0\\
136	0\\
137	0\\
138	0\\
139	0\\
140	0\\
141	0\\
142	0\\
143	0\\
144	0\\
145	0\\
146	0\\
147	0\\
148	0\\
149	0\\
150	0\\
151	0\\
152	0\\
153	0\\
154	0\\
155	0\\
156	0\\
157	0\\
158	0\\
159	0\\
160	0\\
161	0\\
162	0\\
163	0\\
164	0\\
165	0\\
166	0\\
167	0\\
168	0\\
169	0\\
170	0\\
171	0\\
172	0\\
173	0\\
174	0\\
175	0\\
176	0\\
177	0\\
178	0\\
179	0\\
180	0\\
181	0\\
182	0\\
183	0\\
184	0\\
185	0\\
186	0\\
187	0\\
188	0\\
189	0\\
190	0\\
191	0\\
192	0\\
193	0\\
194	0\\
195	0\\
196	0\\
197	0\\
198	0\\
199	0\\
200	0\\
201	0\\
202	0\\
203	0\\
204	0\\
205	0\\
206	0\\
207	0\\
208	0\\
209	0\\
210	0\\
211	0\\
212	0\\
213	0\\
214	0\\
215	0\\
216	0\\
217	0\\
218	0\\
219	0\\
220	0\\
221	0\\
222	0\\
223	0\\
224	0\\
225	0\\
226	0\\
227	0\\
228	0\\
229	0\\
230	0\\
231	0\\
232	0\\
233	0\\
234	0\\
235	0\\
236	0\\
237	0\\
238	0\\
239	0\\
240	0\\
241	0\\
242	0\\
243	0\\
244	0\\
245	0\\
246	0\\
247	0\\
248	0\\
249	0\\
250	0\\
251	0\\
252	0\\
253	0\\
254	0\\
255	0\\
256	0\\
257	0\\
258	0\\
259	0\\
260	0\\
261	0\\
262	0\\
263	0\\
264	0\\
265	0\\
266	0\\
267	0\\
268	0\\
269	0\\
270	0\\
271	0\\
272	0\\
273	0\\
274	0\\
275	0\\
276	0\\
277	0\\
278	0\\
279	0\\
280	0\\
281	0\\
282	0\\
283	0\\
284	0\\
285	0\\
286	0\\
287	0\\
288	0\\
289	0\\
290	0\\
291	0\\
292	0\\
293	0\\
294	0\\
295	0\\
296	0\\
297	0\\
298	0\\
299	0\\
300	0\\
301	0\\
302	0\\
303	0\\
304	0\\
305	0\\
306	0\\
307	0\\
308	0\\
309	0\\
310	0\\
311	0\\
312	0\\
313	0\\
314	0\\
315	0\\
316	0\\
317	0\\
318	0\\
319	0\\
320	0\\
321	0\\
322	0\\
323	0\\
324	0\\
325	0\\
326	0\\
327	0\\
328	0\\
329	0\\
330	0\\
331	0\\
332	0\\
333	0\\
334	0\\
335	0\\
336	0\\
337	0\\
338	0\\
339	0\\
340	0\\
341	0\\
342	0\\
343	0\\
344	0\\
345	0\\
346	0\\
347	0\\
348	0\\
349	0\\
350	0\\
351	0\\
352	0\\
353	0\\
354	0\\
355	0\\
356	0\\
357	0\\
358	0\\
359	0\\
360	0\\
361	0\\
362	0\\
363	0\\
364	0\\
365	0\\
366	0\\
367	0\\
368	0\\
369	0\\
370	0\\
371	0\\
372	0\\
373	0\\
374	0\\
375	0\\
376	0\\
377	0\\
378	0\\
379	0\\
380	0\\
381	0\\
382	0\\
383	0\\
384	0\\
385	0\\
386	0\\
387	0\\
388	0\\
389	0\\
390	0\\
391	0\\
392	0\\
393	0\\
394	0\\
395	0\\
396	0\\
397	0\\
398	0\\
399	0\\
400	0\\
401	0\\
402	0\\
403	0\\
404	0\\
405	0\\
406	0\\
407	0\\
408	0\\
409	0\\
410	0\\
411	0\\
412	0\\
413	0\\
414	0\\
415	0\\
416	0\\
417	0\\
418	0\\
419	0\\
420	0\\
421	0\\
422	0\\
423	0\\
424	0\\
425	0\\
426	0\\
427	0\\
428	0\\
429	0\\
430	0\\
431	0\\
432	0\\
433	0\\
434	0\\
435	0\\
436	0\\
437	0\\
438	0\\
439	0\\
440	0\\
441	0\\
442	0\\
443	0\\
444	0\\
445	0\\
446	0\\
447	0\\
448	0\\
449	0\\
450	0\\
451	0\\
452	0\\
453	0\\
454	0\\
455	0\\
456	0\\
457	0\\
458	0\\
459	0\\
460	0\\
461	0\\
462	0\\
463	0\\
464	0\\
465	0\\
466	0\\
467	0\\
468	0\\
469	0\\
470	0\\
471	0\\
472	0\\
473	0\\
474	0\\
475	0\\
476	0\\
477	0\\
478	0\\
479	0\\
480	0\\
481	0\\
482	0\\
483	0\\
484	0\\
485	0\\
486	0\\
487	0\\
488	0\\
489	0\\
490	0\\
491	0\\
492	0\\
493	0\\
494	0\\
495	0\\
496	0\\
497	0\\
498	0\\
499	0\\
500	0\\
501	0\\
502	0\\
503	0\\
504	0\\
505	0\\
506	0\\
507	0\\
508	0\\
509	0\\
510	0\\
511	0\\
512	0\\
513	0\\
514	0\\
515	0\\
516	0\\
517	0\\
518	0\\
519	0\\
520	0\\
521	0\\
522	0\\
523	0\\
524	0\\
525	0\\
526	0\\
527	0\\
528	0\\
529	0\\
530	0\\
531	0\\
532	0\\
533	0\\
534	0\\
535	0\\
536	0\\
537	0\\
538	0\\
539	0\\
540	0\\
541	0\\
542	0\\
543	0\\
544	0\\
545	0\\
546	0\\
547	0\\
548	0\\
549	0\\
550	0\\
551	0\\
552	0\\
553	0\\
554	0\\
555	0\\
556	0\\
557	0\\
558	0\\
559	0\\
560	0\\
561	0\\
562	0\\
563	0\\
564	0\\
565	0\\
566	0\\
567	0\\
568	0\\
569	0\\
570	0\\
571	0\\
572	0\\
573	0\\
574	0\\
575	0\\
576	0\\
577	0\\
578	0\\
579	0\\
580	0\\
581	0\\
582	0\\
583	0\\
584	0\\
585	0\\
586	0\\
587	0\\
588	0\\
589	0\\
590	0\\
591	0\\
592	0\\
593	0\\
594	0\\
595	0\\
596	0\\
597	0\\
598	0\\
599	0\\
600	0\\
};
\addplot [color=blue!50!mycolor7,solid,forget plot]
  table[row sep=crcr]{%
1	0\\
2	0\\
3	0\\
4	0\\
5	0\\
6	0\\
7	0\\
8	0\\
9	0\\
10	0\\
11	0\\
12	0\\
13	0\\
14	0\\
15	0\\
16	0\\
17	0\\
18	0\\
19	0\\
20	0\\
21	0\\
22	0\\
23	0\\
24	0\\
25	0\\
26	0\\
27	0\\
28	0\\
29	0\\
30	0\\
31	0\\
32	0\\
33	0\\
34	0\\
35	0\\
36	0\\
37	0\\
38	0\\
39	0\\
40	0\\
41	0\\
42	0\\
43	0\\
44	0\\
45	0\\
46	0\\
47	0\\
48	0\\
49	0\\
50	0\\
51	0\\
52	0\\
53	0\\
54	0\\
55	0\\
56	0\\
57	0\\
58	0\\
59	0\\
60	0\\
61	0\\
62	0\\
63	0\\
64	0\\
65	0\\
66	0\\
67	0\\
68	0\\
69	0\\
70	0\\
71	0\\
72	0\\
73	0\\
74	0\\
75	0\\
76	0\\
77	0\\
78	0\\
79	0\\
80	0\\
81	0\\
82	0\\
83	0\\
84	0\\
85	0\\
86	0\\
87	0\\
88	0\\
89	0\\
90	0\\
91	0\\
92	0\\
93	0\\
94	0\\
95	0\\
96	0\\
97	0\\
98	0\\
99	0\\
100	0\\
101	0\\
102	0\\
103	0\\
104	0\\
105	0\\
106	0\\
107	0\\
108	0\\
109	0\\
110	0\\
111	0\\
112	0\\
113	0\\
114	0\\
115	0\\
116	0\\
117	0\\
118	0\\
119	0\\
120	0\\
121	0\\
122	0\\
123	0\\
124	0\\
125	0\\
126	0\\
127	0\\
128	0\\
129	0\\
130	0\\
131	0\\
132	0\\
133	0\\
134	0\\
135	0\\
136	0\\
137	0\\
138	0\\
139	0\\
140	0\\
141	0\\
142	0\\
143	0\\
144	0\\
145	0\\
146	0\\
147	0\\
148	0\\
149	0\\
150	0\\
151	0\\
152	0\\
153	0\\
154	0\\
155	0\\
156	0\\
157	0\\
158	0\\
159	0\\
160	0\\
161	0\\
162	0\\
163	0\\
164	0\\
165	0\\
166	0\\
167	0\\
168	0\\
169	0\\
170	0\\
171	0\\
172	0\\
173	0\\
174	0\\
175	0\\
176	0\\
177	0\\
178	0\\
179	0\\
180	0\\
181	0\\
182	0\\
183	0\\
184	0\\
185	0\\
186	0\\
187	0\\
188	0\\
189	0\\
190	0\\
191	0\\
192	0\\
193	0\\
194	0\\
195	0\\
196	0\\
197	0\\
198	0\\
199	0\\
200	0\\
201	0\\
202	0\\
203	0\\
204	0\\
205	0\\
206	0\\
207	0\\
208	0\\
209	0\\
210	0\\
211	0\\
212	0\\
213	0\\
214	0\\
215	0\\
216	0\\
217	0\\
218	0\\
219	0\\
220	0\\
221	0\\
222	0\\
223	0\\
224	0\\
225	0\\
226	0\\
227	0\\
228	0\\
229	0\\
230	0\\
231	0\\
232	0\\
233	0\\
234	0\\
235	0\\
236	0\\
237	0\\
238	0\\
239	0\\
240	0\\
241	0\\
242	0\\
243	0\\
244	0\\
245	0\\
246	0\\
247	0\\
248	0\\
249	0\\
250	0\\
251	0\\
252	0\\
253	0\\
254	0\\
255	0\\
256	0\\
257	0\\
258	0\\
259	0\\
260	0\\
261	0\\
262	0\\
263	0\\
264	0\\
265	0\\
266	0\\
267	0\\
268	0\\
269	0\\
270	0\\
271	0\\
272	0\\
273	0\\
274	0\\
275	0\\
276	0\\
277	0\\
278	0\\
279	0\\
280	0\\
281	0\\
282	0\\
283	0\\
284	0\\
285	0\\
286	0\\
287	0\\
288	0\\
289	0\\
290	0\\
291	0\\
292	0\\
293	0\\
294	0\\
295	0\\
296	0\\
297	0\\
298	0\\
299	0\\
300	0\\
301	0\\
302	0\\
303	0\\
304	0\\
305	0\\
306	0\\
307	0\\
308	0\\
309	0\\
310	0\\
311	0\\
312	0\\
313	0\\
314	0\\
315	0\\
316	0\\
317	0\\
318	0\\
319	0\\
320	0\\
321	0\\
322	0\\
323	0\\
324	0\\
325	0\\
326	0\\
327	0\\
328	0\\
329	0\\
330	0\\
331	0\\
332	0\\
333	0\\
334	0\\
335	0\\
336	0\\
337	0\\
338	0\\
339	0\\
340	0\\
341	0\\
342	0\\
343	0\\
344	0\\
345	0\\
346	0\\
347	0\\
348	0\\
349	0\\
350	0\\
351	0\\
352	0\\
353	0\\
354	0\\
355	0\\
356	0\\
357	0\\
358	0\\
359	0\\
360	0\\
361	0\\
362	0\\
363	0\\
364	0\\
365	0\\
366	0\\
367	0\\
368	0\\
369	0\\
370	0\\
371	0\\
372	0\\
373	0\\
374	0\\
375	0\\
376	0\\
377	0\\
378	0\\
379	0\\
380	0\\
381	0\\
382	0\\
383	0\\
384	0\\
385	0\\
386	0\\
387	0\\
388	0\\
389	0\\
390	0\\
391	0\\
392	0\\
393	0\\
394	0\\
395	0\\
396	0\\
397	0\\
398	0\\
399	0\\
400	0\\
401	0\\
402	0\\
403	0\\
404	0\\
405	0\\
406	0\\
407	0\\
408	0\\
409	0\\
410	0\\
411	0\\
412	0\\
413	0\\
414	0\\
415	0\\
416	0\\
417	0\\
418	0\\
419	0\\
420	0\\
421	0\\
422	0\\
423	0\\
424	0\\
425	0\\
426	0\\
427	0\\
428	0\\
429	0\\
430	0\\
431	0\\
432	0\\
433	0\\
434	0\\
435	0\\
436	0\\
437	0\\
438	0\\
439	0\\
440	0\\
441	0\\
442	0\\
443	0\\
444	0\\
445	0\\
446	0\\
447	0\\
448	0\\
449	0\\
450	0\\
451	0\\
452	0\\
453	0\\
454	0\\
455	0\\
456	0\\
457	0\\
458	0\\
459	0\\
460	0\\
461	0\\
462	0\\
463	0\\
464	0\\
465	0\\
466	0\\
467	0\\
468	0\\
469	0\\
470	0\\
471	0\\
472	0\\
473	0\\
474	0\\
475	0\\
476	0\\
477	0\\
478	0\\
479	0\\
480	0\\
481	0\\
482	0\\
483	0\\
484	0\\
485	0\\
486	0\\
487	0\\
488	0\\
489	0\\
490	0\\
491	0\\
492	0\\
493	0\\
494	0\\
495	0\\
496	0\\
497	0\\
498	0\\
499	0\\
500	0\\
501	0\\
502	0\\
503	0\\
504	0\\
505	0\\
506	0\\
507	0\\
508	0\\
509	0\\
510	0\\
511	0\\
512	0\\
513	0\\
514	0\\
515	0\\
516	0\\
517	0\\
518	0\\
519	0\\
520	0\\
521	0\\
522	0\\
523	0\\
524	0\\
525	0\\
526	0\\
527	0\\
528	0\\
529	0\\
530	0\\
531	0\\
532	0\\
533	0\\
534	0\\
535	0\\
536	0\\
537	0\\
538	0\\
539	0\\
540	0\\
541	0\\
542	0\\
543	0\\
544	0\\
545	0\\
546	0\\
547	0\\
548	0\\
549	0\\
550	0\\
551	0\\
552	0\\
553	0\\
554	0\\
555	0\\
556	0\\
557	0\\
558	0\\
559	0\\
560	0\\
561	0\\
562	0\\
563	0\\
564	0\\
565	0\\
566	0\\
567	0\\
568	0\\
569	0\\
570	0\\
571	0\\
572	0\\
573	0\\
574	0\\
575	0\\
576	0\\
577	0\\
578	0\\
579	0\\
580	0\\
581	0\\
582	0\\
583	0\\
584	0\\
585	0\\
586	0\\
587	0\\
588	0\\
589	0\\
590	0\\
591	0\\
592	0\\
593	0\\
594	0\\
595	0\\
596	0\\
597	0\\
598	0\\
599	0\\
600	0\\
};
\addplot [color=blue!40!mycolor9,solid,forget plot]
  table[row sep=crcr]{%
1	0.000572289361079503\\
2	0.000572279828444312\\
3	0.000572270135439682\\
4	0.000572260279371882\\
5	0.000572250257501964\\
6	0.000572240067045005\\
7	0.000572229705169346\\
8	0.00057221916899576\\
9	0.000572208455596701\\
10	0.000572197561995472\\
11	0.000572186485165404\\
12	0.000572175222028997\\
13	0.000572163769457075\\
14	0.000572152124267915\\
15	0.000572140283226384\\
16	0.000572128243043005\\
17	0.000572116000373054\\
18	0.000572103551815636\\
19	0.000572090893912736\\
20	0.000572078023148238\\
21	0.000572064935946996\\
22	0.000572051628673754\\
23	0.000572038097632211\\
24	0.000572024339063951\\
25	0.000572010349147427\\
26	0.000571996123996853\\
27	0.000571981659661137\\
28	0.000571966952122788\\
29	0.00057195199729678\\
30	0.000571936791029434\\
31	0.000571921329097228\\
32	0.000571905607205632\\
33	0.000571889620987887\\
34	0.000571873366003836\\
35	0.000571856837738625\\
36	0.000571840031601462\\
37	0.000571822942924336\\
38	0.000571805566960702\\
39	0.000571787898884128\\
40	0.00057176993378699\\
41	0.000571751666679058\\
42	0.000571733092486115\\
43	0.000571714206048501\\
44	0.000571695002119696\\
45	0.000571675475364842\\
46	0.000571655620359213\\
47	0.000571635431586721\\
48	0.00057161490343834\\
49	0.000571594030210513\\
50	0.000571572806103606\\
51	0.000571551225220185\\
52	0.000571529281563428\\
53	0.000571506969035371\\
54	0.000571484281435221\\
55	0.000571461212457569\\
56	0.00057143775569063\\
57	0.000571413904614426\\
58	0.000571389652598912\\
59	0.000571364992902124\\
60	0.000571339918668244\\
61	0.000571314422925635\\
62	0.000571288498584887\\
63	0.000571262138436795\\
64	0.000571235335150263\\
65	0.000571208081270274\\
66	0.00057118036921569\\
67	0.000571152191277152\\
68	0.000571123539614814\\
69	0.00057109440625612\\
70	0.000571064783093518\\
71	0.000571034661882132\\
72	0.000571004034237361\\
73	0.000570972891632495\\
74	0.000570941225396272\\
75	0.000570909026710325\\
76	0.000570876286606628\\
77	0.000570842995964968\\
78	0.000570809145510212\\
79	0.0005707747258097\\
80	0.00057073972727042\\
81	0.000570704140136268\\
82	0.000570667954485155\\
83	0.000570631160226152\\
84	0.000570593747096516\\
85	0.000570555704658669\\
86	0.000570517022297149\\
87	0.000570477689215458\\
88	0.000570437694432896\\
89	0.000570397026781319\\
90	0.000570355674901806\\
91	0.000570313627241323\\
92	0.000570270872049269\\
93	0.00057022739737396\\
94	0.000570183191059074\\
95	0.000570138240739997\\
96	0.000570092533840138\\
97	0.000570046057567128\\
98	0.000569998798908942\\
99	0.00056995074462999\\
100	0.000569901881267093\\
101	0.000569852195125393\\
102	0.00056980167227418\\
103	0.00056975029854262\\
104	0.000569698059515415\\
105	0.000569644940528375\\
106	0.000569590926663907\\
107	0.000569536002746355\\
108	0.000569480153337341\\
109	0.000569423362730919\\
110	0.000569365614948678\\
111	0.000569306893734748\\
112	0.000569247182550685\\
113	0.000569186464570213\\
114	0.000569124722673949\\
115	0.000569061939443922\\
116	0.000568998097158043\\
117	0.000568933177784415\\
118	0.000568867162975534\\
119	0.000568800034062384\\
120	0.000568731772048393\\
121	0.000568662357603254\\
122	0.000568591771056625\\
123	0.00056851999239169\\
124	0.000568447001238563\\
125	0.000568372776867582\\
126	0.000568297298182457\\
127	0.000568220543713245\\
128	0.000568142491609186\\
129	0.000568063119631404\\
130	0.000567982405145427\\
131	0.000567900325113538\\
132	0.000567816856087025\\
133	0.000567731974198204\\
134	0.000567645655152298\\
135	0.000567557874219155\\
136	0.000567468606224778\\
137	0.000567377825542686\\
138	0.000567285506085131\\
139	0.000567191621294093\\
140	0.000567096144132122\\
141	0.000566999047073011\\
142	0.00056690030209229\\
143	0.000566799880657502\\
144	0.000566697753718357\\
145	0.000566593891696704\\
146	0.000566488264476299\\
147	0.000566380841392455\\
148	0.000566271591221453\\
149	0.000566160482169865\\
150	0.0005660474818637\\
151	0.000565932557337377\\
152	0.000565815675022591\\
153	0.000565696800737025\\
154	0.000565575899672972\\
155	0.000565452936385781\\
156	0.000565327874782326\\
157	0.00056520067810925\\
158	0.000565071308941251\\
159	0.000564939729169241\\
160	0.000564805899988544\\
161	0.000564669781887021\\
162	0.000564531334633243\\
163	0.00056439051726466\\
164	0.000564247288075846\\
165	0.000564101604606802\\
166	0.000563953423631374\\
167	0.000563802701145759\\
168	0.000563649392357172\\
169	0.000563493451672658\\
170	0.000563334832688151\\
171	0.000563173488177561\\
172	0.000563009370082275\\
173	0.000562842429500666\\
174	0.000562672616677942\\
175	0.000562499880996109\\
176	0.000562324170964073\\
177	0.000562145434207974\\
178	0.000561963617461486\\
179	0.00056177866655611\\
180	0.000561590526411449\\
181	0.000561399141025247\\
182	0.000561204453462971\\
183	0.000561006405846999\\
184	0.000560804939345069\\
185	0.000560599994157678\\
186	0.000560391509504379\\
187	0.000560179423608616\\
188	0.000559963673680764\\
189	0.000559744195899196\\
190	0.00055952092538907\\
191	0.000559293796198689\\
192	0.000559062741273323\\
193	0.000558827692426567\\
194	0.000558588580309725\\
195	0.000558345334379785\\
196	0.000558097882867651\\
197	0.000557846152747879\\
198	0.000557590069711211\\
199	0.000557329558141205\\
200	0.000557064541090551\\
201	0.000556794940256949\\
202	0.000556520675958512\\
203	0.000556241667108771\\
204	0.000555957831191096\\
205	0.000555669084232686\\
206	0.000555375340778087\\
207	0.000555076513862098\\
208	0.00055477251498228\\
209	0.000554463254070845\\
210	0.000554148639466027\\
211	0.000553828577882921\\
212	0.00055350297438373\\
213	0.000553171732347426\\
214	0.000552834753438846\\
215	0.000552491937577186\\
216	0.000552143182903838\\
217	0.000551788385749663\\
218	0.000551427440601567\\
219	0.000551060240068446\\
220	0.000550686674846437\\
221	0.000550306633683534\\
222	0.000549920003343464\\
223	0.000549526668568835\\
224	0.000549126512043608\\
225	0.000548719414354775\\
226	0.000548305253953295\\
227	0.000547883907114225\\
228	0.0005474552478961\\
229	0.000547019148099482\\
230	0.000546575477224634\\
231	0.000546124102428374\\
232	0.000545664888480091\\
233	0.000545197697716812\\
234	0.000544722389997394\\
235	0.000544238822655763\\
236	0.000543746850453227\\
237	0.00054324632552977\\
238	0.000542737097354379\\
239	0.000542219012674355\\
240	0.000541691915463587\\
241	0.000541155646869651\\
242	0.000540610045160008\\
243	0.000540054945666896\\
244	0.000539490180731165\\
245	0.000538915579644934\\
246	0.000538330968592991\\
247	0.000537736170593055\\
248	0.000537131005434672\\
249	0.00053651528961688\\
250	0.000535888836284555\\
251	0.000535251455163334\\
252	0.000534602952493253\\
253	0.000533943130960853\\
254	0.00053327178962992\\
255	0.000532588723870659\\
256	0.000531893725287404\\
257	0.000531186581644703\\
258	0.000530467076791844\\
259	0.000529734990585686\\
260	0.000528990098811893\\
261	0.000528232173104341\\
262	0.00052746098086278\\
263	0.000526676285168801\\
264	0.000525877844699746\\
265	0.000525065413640908\\
266	0.000524238741595681\\
267	0.000523397573493714\\
268	0.000522541649497094\\
269	0.000521670704904323\\
270	0.000520784470052275\\
271	0.000519882670215849\\
272	0.000518965025505488\\
273	0.000518031250762205\\
274	0.000517081055450443\\
275	0.000516114143548415\\
276	0.000515130213435959\\
277	0.000514128957779876\\
278	0.00051311006341665\\
279	0.000512073211232512\\
280	0.00051101807604073\\
281	0.000509944326456171\\
282	0.000508851624766938\\
283	0.000507739626803091\\
284	0.000506607981802348\\
285	0.000505456332272721\\
286	0.000504284313852017\\
287	0.000503091555164066\\
288	0.000501877677671679\\
289	0.00050064229552625\\
290	0.000499385015413872\\
291	0.000498105436397899\\
292	0.000496803149758002\\
293	0.00049547773882538\\
294	0.000494128778814315\\
295	0.000492755836649867\\
296	0.000491358470791584\\
297	0.000489936231053213\\
298	0.000488488658418356\\
299	0.000487015284851931\\
300	0.00048551563310734\\
301	0.000483989216529351\\
302	0.000482435538852573\\
303	0.000480854093995371\\
304	0.000479244365849313\\
305	0.000477605828063928\\
306	0.000475937943826749\\
307	0.000474240165638641\\
308	0.000472511935084228\\
309	0.000470752682597514\\
310	0.000468961827222439\\
311	0.000467138776368593\\
312	0.000465282925561809\\
313	0.000463393658189717\\
314	0.000461470345242214\\
315	0.000459512345046848\\
316	0.000457519002999091\\
317	0.000455489651287555\\
318	0.000453423608614047\\
319	0.000451320179908691\\
320	0.000449178656040015\\
321	0.000446998313520135\\
322	0.00044477841420512\\
323	0.000442518204990639\\
324	0.000440216917503031\\
325	0.000437873767785933\\
326	0.000435487955982651\\
327	0.000433058666014573\\
328	0.000430585065255739\\
329	0.000428066304203932\\
330	0.000425501516148543\\
331	0.000422889816835637\\
332	0.000420230304130622\\
333	0.00041752205767883\\
334	0.000414764138564654\\
335	0.000411955588969695\\
336	0.000409095431830549\\
337	0.000406182670496853\\
338	0.000403216288390314\\
339	0.000400195248665453\\
340	0.000397118493873001\\
341	0.000393984945626641\\
342	0.000390793504274319\\
343	0.000387543048574891\\
344	0.00038423243538136\\
345	0.000380860499331881\\
346	0.000377426052549601\\
347	0.000373927884352812\\
348	0.000370364760976633\\
349	0.000366735425307715\\
350	0.000363038596633317\\
351	0.000359272970406324\\
352	0.000355437218027721\\
353	0.000351529986647915\\
354	0.000347549898988649\\
355	0.000343495553186723\\
356	0.000339365522661187\\
357	0.000335158356005238\\
358	0.000330872576904045\\
359	0.000326506684079658\\
360	0.000322059151263819\\
361	0.000317528427199376\\
362	0.000312912935670473\\
363	0.000308211075561671\\
364	0.000303421220945299\\
365	0.000298541721196293\\
366	0.000293570901132558\\
367	0.000288507061178865\\
368	0.000283348477551088\\
369	0.000278093402456643\\
370	0.000272740064306439\\
371	0.000267286667932114\\
372	0.000261731394801327\\
373	0.000256072403222602\\
374	0.000250307828529804\\
375	0.000244435783234991\\
376	0.000238454357136863\\
377	0.000232361617370447\\
378	0.000226155608382433\\
379	0.000219834351815218\\
380	0.000213395846281269\\
381	0.000206838067008912\\
382	0.000200158965339931\\
383	0.000193356468059979\\
384	0.000186428476544544\\
385	0.000179372865706474\\
386	0.00017218748273451\\
387	0.000164870145606742\\
388	0.000157418641337741\\
389	0.000149830723905435\\
390	0.000142104111922943\\
391	0.00013423648565384\\
392	0.000126225483485972\\
393	0.000118068697859579\\
394	0.00010976367063733\\
395	0.000101307887735736\\
396	9.26987715723901e-05\\
397	8.39336683081326e-05\\
398	7.50098166564155e-05\\
399	6.59242533909565e-05\\
400	5.66735099699828e-05\\
401	4.72526162782195e-05\\
402	3.76517861404188e-05\\
403	2.78452662667821e-05\\
404	1.77533983572104e-05\\
405	7.11205651264132e-06\\
406	0\\
407	0\\
408	0\\
409	0\\
410	0\\
411	0\\
412	0\\
413	0\\
414	0\\
415	0\\
416	0\\
417	0\\
418	0\\
419	0\\
420	0\\
421	0\\
422	0\\
423	0\\
424	0\\
425	0\\
426	0\\
427	0\\
428	0\\
429	0\\
430	0\\
431	0\\
432	0\\
433	0\\
434	0\\
435	0\\
436	0\\
437	0\\
438	0\\
439	0\\
440	0\\
441	0\\
442	0\\
443	0\\
444	0\\
445	0\\
446	0\\
447	0\\
448	0\\
449	0\\
450	0\\
451	0\\
452	0\\
453	0\\
454	0\\
455	0\\
456	0\\
457	0\\
458	0\\
459	0\\
460	0\\
461	0\\
462	0\\
463	0\\
464	0\\
465	0\\
466	0\\
467	0\\
468	0\\
469	0\\
470	0\\
471	0\\
472	0\\
473	0\\
474	0\\
475	0\\
476	0\\
477	0\\
478	0\\
479	0\\
480	0\\
481	0\\
482	0\\
483	0\\
484	0\\
485	0\\
486	0\\
487	0\\
488	0\\
489	0\\
490	0\\
491	0\\
492	0\\
493	0\\
494	0\\
495	0\\
496	0\\
497	0\\
498	0\\
499	0\\
500	0\\
501	0\\
502	0\\
503	0\\
504	0\\
505	0\\
506	0\\
507	0\\
508	0\\
509	0\\
510	0\\
511	0\\
512	0\\
513	0\\
514	0\\
515	0\\
516	0\\
517	0\\
518	0\\
519	0\\
520	0\\
521	0\\
522	0\\
523	0\\
524	0\\
525	0\\
526	0\\
527	0\\
528	0\\
529	0\\
530	0\\
531	0\\
532	0\\
533	0\\
534	0\\
535	0\\
536	0\\
537	0\\
538	0\\
539	0\\
540	0\\
541	0\\
542	0\\
543	0\\
544	0\\
545	0\\
546	0\\
547	0\\
548	0\\
549	0\\
550	0\\
551	0\\
552	0\\
553	0\\
554	0\\
555	0\\
556	0\\
557	0\\
558	0\\
559	0\\
560	0\\
561	0\\
562	0\\
563	0\\
564	0\\
565	0\\
566	0\\
567	0\\
568	0\\
569	0\\
570	0\\
571	0\\
572	0\\
573	0\\
574	0\\
575	0\\
576	0\\
577	0\\
578	0\\
579	0\\
580	0\\
581	0\\
582	0\\
583	0\\
584	0\\
585	0\\
586	0\\
587	0\\
588	0\\
589	0\\
590	0\\
591	0\\
592	0\\
593	0\\
594	0\\
595	0\\
596	0\\
597	0\\
598	0\\
599	0\\
600	0\\
};
\addplot [color=blue!75!mycolor7,solid,forget plot]
  table[row sep=crcr]{%
1	0.00162295980517879\\
2	0.00162295046252148\\
3	0.00162294096273216\\
4	0.00162293130316988\\
5	0.00162292148114924\\
6	0.0016229114939396\\
7	0.00162290133876437\\
8	0.00162289101280022\\
9	0.00162288051317626\\
10	0.00162286983697323\\
11	0.00162285898122273\\
12	0.00162284794290632\\
13	0.00162283671895473\\
14	0.00162282530624696\\
15	0.00162281370160937\\
16	0.00162280190181487\\
17	0.00162278990358195\\
18	0.00162277770357375\\
19	0.00162276529839714\\
20	0.00162275268460179\\
21	0.00162273985867911\\
22	0.00162272681706135\\
23	0.00162271355612052\\
24	0.0016227000721674\\
25	0.00162268636145049\\
26	0.00162267242015493\\
27	0.00162265824440144\\
28	0.00162264383024521\\
29	0.00162262917367475\\
30	0.00162261427061081\\
31	0.00162259911690516\\
32	0.00162258370833943\\
33	0.00162256804062393\\
34	0.0016225521093964\\
35	0.00162253591022075\\
36	0.00162251943858585\\
37	0.00162250268990415\\
38	0.00162248565951048\\
39	0.00162246834266063\\
40	0.00162245073452999\\
41	0.00162243283021221\\
42	0.00162241462471777\\
43	0.00162239611297254\\
44	0.00162237728981633\\
45	0.00162235815000139\\
46	0.0016223386881909\\
47	0.00162231889895745\\
48	0.00162229877678144\\
49	0.00162227831604951\\
50	0.00162225751105288\\
51	0.00162223635598575\\
52	0.00162221484494356\\
53	0.00162219297192129\\
54	0.00162217073081174\\
55	0.00162214811540372\\
56	0.00162212511938027\\
57	0.00162210173631675\\
58	0.00162207795967905\\
59	0.0016220537828216\\
60	0.00162202919898548\\
61	0.00162200420129641\\
62	0.00162197878276273\\
63	0.00162195293627333\\
64	0.00162192665459562\\
65	0.00162189993037326\\
66	0.00162187275612416\\
67	0.00162184512423811\\
68	0.00162181702697464\\
69	0.00162178845646065\\
70	0.00162175940468811\\
71	0.00162172986351164\\
72	0.00162169982464612\\
73	0.00162166927966421\\
74	0.00162163821999377\\
75	0.00162160663691539\\
76	0.00162157452155972\\
77	0.00162154186490478\\
78	0.00162150865777329\\
79	0.00162147489082988\\
80	0.00162144055457829\\
81	0.00162140563935846\\
82	0.00162137013534365\\
83	0.00162133403253743\\
84	0.00162129732077062\\
85	0.00162125998969823\\
86	0.00162122202879629\\
87	0.00162118342735862\\
88	0.00162114417449357\\
89	0.00162110425912068\\
90	0.00162106366996724\\
91	0.00162102239556486\\
92	0.00162098042424589\\
93	0.00162093774413989\\
94	0.00162089434316986\\
95	0.00162085020904855\\
96	0.00162080532927466\\
97	0.00162075969112888\\
98	0.00162071328167\\
99	0.00162066608773084\\
100	0.00162061809591408\\
101	0.00162056929258815\\
102	0.0016205196638829\\
103	0.00162046919568523\\
104	0.00162041787363467\\
105	0.00162036568311883\\
106	0.00162031260926882\\
107	0.00162025863695449\\
108	0.00162020375077967\\
109	0.00162014793507727\\
110	0.0016200911739043\\
111	0.00162003345103682\\
112	0.00161997474996473\\
113	0.00161991505388653\\
114	0.00161985434570394\\
115	0.00161979260801643\\
116	0.00161972982311564\\
117	0.00161966597297978\\
118	0.00161960103926773\\
119	0.00161953500331324\\
120	0.00161946784611892\\
121	0.00161939954835011\\
122	0.00161933009032872\\
123	0.00161925945202685\\
124	0.00161918761306037\\
125	0.00161911455268241\\
126	0.00161904024977661\\
127	0.00161896468285042\\
128	0.00161888783002816\\
129	0.00161880966904399\\
130	0.00161873017723482\\
131	0.00161864933153302\\
132	0.00161856710845903\\
133	0.00161848348411394\\
134	0.00161839843417181\\
135	0.00161831193387193\\
136	0.00161822395801102\\
137	0.00161813448093525\\
138	0.00161804347653212\\
139	0.00161795091822225\\
140	0.00161785677895108\\
141	0.00161776103118042\\
142	0.00161766364687985\\
143	0.00161756459751812\\
144	0.00161746385405431\\
145	0.00161736138692892\\
146	0.00161725716605491\\
147	0.00161715116080853\\
148	0.00161704334002016\\
149	0.00161693367196489\\
150	0.00161682212435318\\
151	0.00161670866432125\\
152	0.00161659325842153\\
153	0.00161647587261285\\
154	0.00161635647225068\\
155	0.0016162350220772\\
156	0.00161611148621123\\
157	0.00161598582813821\\
158	0.00161585801069996\\
159	0.0016157279960844\\
160	0.00161559574581517\\
161	0.00161546122074116\\
162	0.00161532438102591\\
163	0.00161518518613694\\
164	0.00161504359483497\\
165	0.00161489956516297\\
166	0.00161475305443516\\
167	0.00161460401922579\\
168	0.00161445241535785\\
169	0.00161429819789152\\
170	0.00161414132111245\\
171	0.00161398173851994\\
172	0.00161381940281468\\
173	0.00161365426588638\\
174	0.00161348627880102\\
175	0.00161331539178782\\
176	0.00161314155422576\\
177	0.00161296471462976\\
178	0.00161278482063631\\
179	0.00161260181898868\\
180	0.00161241565552155\\
181	0.001612226275145\\
182	0.00161203362182788\\
183	0.00161183763858056\\
184	0.00161163826743685\\
185	0.00161143544943531\\
186	0.00161122912459969\\
187	0.00161101923191878\\
188	0.00161080570932537\\
189	0.00161058849367476\\
190	0.00161036752072246\\
191	0.00161014272510164\\
192	0.00160991404030002\\
193	0.00160968139863671\\
194	0.00160944473123896\\
195	0.00160920396801896\\
196	0.0016089590376509\\
197	0.00160870986754805\\
198	0.00160845638383983\\
199	0.00160819851134835\\
200	0.00160793617356469\\
201	0.00160766929262451\\
202	0.00160739778928338\\
203	0.00160712158289149\\
204	0.00160684059136798\\
205	0.00160655473117476\\
206	0.00160626391728973\\
207	0.00160596806317967\\
208	0.00160566708077244\\
209	0.00160536088042871\\
210	0.00160504937091316\\
211	0.00160473245936512\\
212	0.0016044100512686\\
213	0.0016040820504218\\
214	0.001603748358906\\
215	0.00160340887705384\\
216	0.00160306350341704\\
217	0.00160271213473337\\
218	0.00160235466589315\\
219	0.00160199098990493\\
220	0.00160162099786064\\
221	0.0016012445789\\
222	0.00160086162017417\\
223	0.00160047200680884\\
224	0.00160007562186647\\
225	0.00159967234630785\\
226	0.00159926205895291\\
227	0.00159884463644074\\
228	0.00159841995318881\\
229	0.00159798788135147\\
230	0.00159754829077751\\
231	0.00159710104896702\\
232	0.00159664602102728\\
233	0.00159618306962786\\
234	0.00159571205495483\\
235	0.001595232834664\\
236	0.00159474526383333\\
237	0.00159424919491431\\
238	0.00159374447768245\\
239	0.00159323095918677\\
240	0.00159270848369826\\
241	0.00159217689265735\\
242	0.00159163602462031\\
243	0.00159108571520466\\
244	0.00159052579703339\\
245	0.00158995609967814\\
246	0.00158937644960128\\
247	0.00158878667009675\\
248	0.00158818658122983\\
249	0.00158757599977561\\
250	0.00158695473915633\\
251	0.00158632260937742\\
252	0.00158567941696233\\
253	0.00158502496488597\\
254	0.00158435905250693\\
255	0.00158368147549834\\
256	0.00158299202577729\\
257	0.00158229049143296\\
258	0.00158157665665321\\
259	0.00158085030164991\\
260	0.00158011120258253\\
261	0.0015793591314805\\
262	0.00157859385616387\\
263	0.00157781514016245\\
264	0.00157702274263341\\
265	0.00157621641827724\\
266	0.00157539591725204\\
267	0.00157456098508619\\
268	0.00157371136258933\\
269	0.00157284678576154\\
270	0.00157196698570083\\
271	0.00157107168850885\\
272	0.00157016061519472\\
273	0.0015692334815771\\
274	0.00156828999818434\\
275	0.00156732987015272\\
276	0.00156635279712276\\
277	0.00156535847313365\\
278	0.00156434658651553\\
279	0.00156331681977994\\
280	0.00156226884950806\\
281	0.00156120234623699\\
282	0.00156011697434384\\
283	0.00155901239192775\\
284	0.00155788825068978\\
285	0.00155674419581046\\
286	0.00155557986582534\\
287	0.00155439489249817\\
288	0.00155318890069183\\
289	0.00155196150823705\\
290	0.00155071232579872\\
291	0.00154944095673998\\
292	0.00154814699698386\\
293	0.00154683003487266\\
294	0.00154548965102486\\
295	0.00154412541818962\\
296	0.00154273690109895\\
297	0.00154132365631739\\
298	0.00153988523208923\\
299	0.00153842116818331\\
300	0.00153693099573536\\
301	0.00153541423708787\\
302	0.00153387040562747\\
303	0.00153229900561981\\
304	0.00153069953204209\\
305	0.00152907147041296\\
306	0.00152741429662004\\
307	0.00152572747674503\\
308	0.00152401046688623\\
309	0.00152226271297875\\
310	0.00152048365061228\\
311	0.00151867270484644\\
312	0.0015168292900237\\
313	0.00151495280958008\\
314	0.0015130426558534\\
315	0.00151109820988929\\
316	0.0015091188412449\\
317	0.00150710390779045\\
318	0.00150505275550851\\
319	0.00150296471829112\\
320	0.0015008391177349\\
321	0.00149867526293396\\
322	0.00149647245027083\\
323	0.00149422996320542\\
324	0.00149194707206201\\
325	0.0014896230338143\\
326	0.00148725709186868\\
327	0.00148484847584552\\
328	0.00148239640135883\\
329	0.00147990006979397\\
330	0.0014773586680837\\
331	0.00147477136848247\\
332	0.00147213732833884\\
333	0.00146945568986628\\
334	0.00146672557991202\\
335	0.00146394610972412\\
336	0.0014611163747165\\
337	0.00145823545423205\\
338	0.00145530241130345\\
339	0.0014523162924117\\
340	0.00144927612724208\\
341	0.00144618092843732\\
342	0.00144302969134765\\
343	0.00143982139377739\\
344	0.00143655499572767\\
345	0.00143322943913474\\
346	0.00142984364760332\\
347	0.00142639652613438\\
348	0.0014228869608465\\
349	0.00141931381868992\\
350	0.00141567594715247\\
351	0.00141197217395605\\
352	0.00140820130674252\\
353	0.00140436213274755\\
354	0.00140045341846087\\
355	0.00139647390927118\\
356	0.0013924223290937\\
357	0.00138829737997833\\
358	0.001384097741696\\
359	0.00137982207130068\\
360	0.00137546900266415\\
361	0.00137103714598066\\
362	0.00136652508723811\\
363	0.0013619313876523\\
364	0.00135725458306041\\
365	0.00135249318326998\\
366	0.0013476456713591\\
367	0.00134271050292344\\
368	0.00133768610526583\\
369	0.0013325708765238\\
370	0.0013273631847304\\
371	0.0013220613668038\\
372	0.00131666372746137\\
373	0.00131116853805386\\
374	0.0013055740353159\\
375	0.00129987842002959\\
376	0.00129407985559819\\
377	0.00128817646652833\\
378	0.00128216633681967\\
379	0.00127604750826262\\
380	0.00126981797864585\\
381	0.00126347569987737\\
382	0.00125701857602496\\
383	0.00125044446128388\\
384	0.00124375115788245\\
385	0.00123693641393882\\
386	0.00122999792128435\\
387	0.00122293331327162\\
388	0.00121574016258898\\
389	0.00120841597911064\\
390	0.00120095820781001\\
391	0.00119336422679553\\
392	0.00118563134552309\\
393	0.00117775680323375\\
394	0.00116973776762585\\
395	0.00116157133362488\\
396	0.00115325452171927\\
397	0.00114478427407384\\
398	0.00113615744329277\\
399	0.0011273707596541\\
400	0.00111842073867647\\
401	0.0011093034318014\\
402	0.00110001379547988\\
403	0.00109054426602079\\
404	0.00108088232894597\\
405	0.00107101049869601\\
406	0.00106092951432877\\
407	0.00105065977031644\\
408	0.00104019699284573\\
409	0.00102953677969847\\
410	0.00101867464720291\\
411	0.00100760617591885\\
412	0.000996327407526952\\
413	0.000984835854188185\\
414	0.000973132915070548\\
415	0.000961229198796052\\
416	0.000949154471044381\\
417	0.000936968437368094\\
418	0.000924731355869571\\
419	0.000912268345570118\\
420	0.000899549957971183\\
421	0.00088657119851197\\
422	0.000873327160728552\\
423	0.000859813072293951\\
424	0.000846024350965498\\
425	0.000831956672444576\\
426	0.000817606052518265\\
427	0.000802968946306856\\
428	0.000788042368116034\\
429	0.00077282403565568\\
430	0.000757312543762736\\
431	0.000741507575906905\\
432	0.00072541017123818\\
433	0.000709023104064201\\
434	0.000692351598338899\\
435	0.000675405257300434\\
436	0.00065820468577962\\
437	0.000640806376765546\\
438	0.000623398212885161\\
439	0.000606665885673703\\
440	0.000593192831486634\\
441	0.000581579396905864\\
442	0.000569713302659357\\
443	0.000557586012442185\\
444	0.000545188168945224\\
445	0.000532509458907316\\
446	0.000519538454379433\\
447	0.000506262425739107\\
448	0.000492667121011696\\
449	0.000478736504767938\\
450	0.00046445244806613\\
451	0.000449794358060503\\
452	0.000434738730400776\\
453	0.000419258594188202\\
454	0.000403322780750657\\
455	0.000386894828016613\\
456	0.000369930972609213\\
457	0.000352375618567614\\
458	0.000334148997715952\\
459	0.000315109979488416\\
460	0.000294938204456355\\
461	0.000272749625838252\\
462	0.000245909460098321\\
463	0.000218990310996033\\
464	0.000192182069022129\\
465	0.000164840854180496\\
466	0.000136951103771785\\
467	0.000108481453583217\\
468	7.9346960826296e-05\\
469	4.92801445037634e-05\\
470	1.73859184336513e-05\\
471	0\\
472	0\\
473	0\\
474	0\\
475	0\\
476	0\\
477	0\\
478	0\\
479	0\\
480	0\\
481	0\\
482	0\\
483	0\\
484	0\\
485	0\\
486	0\\
487	0\\
488	0\\
489	0\\
490	0\\
491	0\\
492	0\\
493	0\\
494	0\\
495	0\\
496	0\\
497	0\\
498	0\\
499	0\\
500	0\\
501	0\\
502	0\\
503	0\\
504	0\\
505	0\\
506	0\\
507	0\\
508	0\\
509	0\\
510	0\\
511	0\\
512	0\\
513	0\\
514	0\\
515	0\\
516	0\\
517	0\\
518	0\\
519	0\\
520	0\\
521	0\\
522	0\\
523	0\\
524	0\\
525	0\\
526	0\\
527	0\\
528	0\\
529	0\\
530	0\\
531	0\\
532	0\\
533	0\\
534	0\\
535	0\\
536	0\\
537	0\\
538	0\\
539	0\\
540	0\\
541	0\\
542	0\\
543	0\\
544	0\\
545	0\\
546	0\\
547	0\\
548	0\\
549	0\\
550	0\\
551	0\\
552	0\\
553	0\\
554	0\\
555	0\\
556	0\\
557	0\\
558	0\\
559	0\\
560	0\\
561	0\\
562	0\\
563	0\\
564	0\\
565	0\\
566	0\\
567	0\\
568	0\\
569	0\\
570	0\\
571	0\\
572	0\\
573	0\\
574	0\\
575	0\\
576	0\\
577	0\\
578	0\\
579	0\\
580	0\\
581	0\\
582	0\\
583	0\\
584	0\\
585	0\\
586	0\\
587	0\\
588	0\\
589	0\\
590	0\\
591	0\\
592	0\\
593	0\\
594	0\\
595	0\\
596	0\\
597	0\\
598	0\\
599	0\\
600	0\\
};
\addplot [color=blue!80!mycolor9,solid,forget plot]
  table[row sep=crcr]{%
1	0.0031488158044997\\
2	0.00314881010942653\\
3	0.00314880431857838\\
4	0.00314879843034492\\
5	0.00314879244308868\\
6	0.00314878635514459\\
7	0.00314878016481948\\
8	0.00314877387039164\\
9	0.0031487674701103\\
10	0.00314876096219517\\
11	0.0031487543448359\\
12	0.00314874761619161\\
13	0.00314874077439029\\
14	0.00314873381752838\\
15	0.00314872674367014\\
16	0.00314871955084715\\
17	0.00314871223705772\\
18	0.00314870480026634\\
19	0.00314869723840309\\
20	0.00314868954936306\\
21	0.00314868173100576\\
22	0.00314867378115445\\
23	0.00314866569759563\\
24	0.00314865747807829\\
25	0.00314864912031333\\
26	0.00314864062197292\\
27	0.00314863198068978\\
28	0.00314862319405655\\
29	0.00314861425962505\\
30	0.00314860517490563\\
31	0.00314859593736642\\
32	0.00314858654443263\\
33	0.00314857699348576\\
34	0.00314856728186291\\
35	0.00314855740685594\\
36	0.00314854736571077\\
37	0.0031485371556265\\
38	0.00314852677375466\\
39	0.00314851621719838\\
40	0.0031485054830115\\
41	0.00314849456819778\\
42	0.00314848346970997\\
43	0.00314847218444898\\
44	0.00314846070926293\\
45	0.00314844904094627\\
46	0.0031484371762388\\
47	0.00314842511182474\\
48	0.00314841284433181\\
49	0.00314840037033013\\
50	0.0031483876863313\\
51	0.00314837478878737\\
52	0.00314836167408975\\
53	0.0031483483385682\\
54	0.0031483347784897\\
55	0.00314832099005738\\
56	0.00314830696940938\\
57	0.00314829271261774\\
58	0.00314827821568716\\
59	0.0031482634745539\\
60	0.00314824848508452\\
61	0.00314823324307466\\
62	0.00314821774424781\\
63	0.00314820198425396\\
64	0.0031481859586684\\
65	0.00314816966299032\\
66	0.00314815309264149\\
67	0.00314813624296486\\
68	0.00314811910922321\\
69	0.00314810168659764\\
70	0.00314808397018622\\
71	0.0031480659550024\\
72	0.00314804763597357\\
73	0.00314802900793951\\
74	0.0031480100656508\\
75	0.00314799080376724\\
76	0.00314797121685624\\
77	0.00314795129939111\\
78	0.00314793104574942\\
79	0.00314791045021127\\
80	0.00314788950695751\\
81	0.00314786821006797\\
82	0.00314784655351965\\
83	0.00314782453118485\\
84	0.00314780213682928\\
85	0.00314777936411013\\
86	0.00314775620657416\\
87	0.00314773265765558\\
88	0.00314770871067414\\
89	0.00314768435883298\\
90	0.00314765959521654\\
91	0.00314763441278841\\
92	0.00314760880438911\\
93	0.00314758276273387\\
94	0.00314755628041037\\
95	0.00314752934987637\\
96	0.00314750196345738\\
97	0.00314747411334424\\
98	0.00314744579159069\\
99	0.00314741699011081\\
100	0.00314738770067656\\
101	0.00314735791491511\\
102	0.00314732762430623\\
103	0.00314729682017962\\
104	0.00314726549371214\\
105	0.00314723363592501\\
106	0.00314720123768105\\
107	0.00314716828968168\\
108	0.00314713478246407\\
109	0.00314710070639809\\
110	0.00314706605168329\\
111	0.00314703080834579\\
112	0.00314699496623511\\
113	0.00314695851502099\\
114	0.00314692144419008\\
115	0.00314688374304268\\
116	0.0031468454006893\\
117	0.00314680640604725\\
118	0.00314676674783712\\
119	0.0031467264145793\\
120	0.00314668539459026\\
121	0.00314664367597899\\
122	0.00314660124664317\\
123	0.00314655809426546\\
124	0.0031465142063096\\
125	0.00314646957001652\\
126	0.0031464241724004\\
127	0.00314637800024457\\
128	0.00314633104009748\\
129	0.0031462832782685\\
130	0.00314623470082376\\
131	0.00314618529358183\\
132	0.0031461350421094\\
133	0.00314608393171689\\
134	0.00314603194745396\\
135	0.00314597907410506\\
136	0.00314592529618477\\
137	0.00314587059793318\\
138	0.00314581496331124\\
139	0.00314575837599591\\
140	0.00314570081937539\\
141	0.00314564227654423\\
142	0.00314558273029834\\
143	0.00314552216313004\\
144	0.0031454605572229\\
145	0.0031453978944467\\
146	0.00314533415635217\\
147	0.0031452693241657\\
148	0.00314520337878409\\
149	0.00314513630076908\\
150	0.00314506807034191\\
151	0.0031449986673778\\
152	0.00314492807140031\\
153	0.0031448562615757\\
154	0.0031447832167071\\
155	0.00314470891522876\\
156	0.00314463333520007\\
157	0.00314455645429954\\
158	0.00314447824981876\\
159	0.00314439869865611\\
160	0.00314431777731053\\
161	0.00314423546187506\\
162	0.0031441517280303\\
163	0.0031440665510378\\
164	0.00314397990573318\\
165	0.0031438917665193\\
166	0.00314380210735908\\
167	0.00314371090176827\\
168	0.00314361812280806\\
169	0.00314352374307744\\
170	0.00314342773470537\\
171	0.0031433300693428\\
172	0.0031432307181544\\
173	0.00314312965181013\\
174	0.00314302684047649\\
175	0.00314292225380755\\
176	0.00314281586093575\\
177	0.0031427076304624\\
178	0.00314259753044784\\
179	0.00314248552840148\\
180	0.00314237159127135\\
181	0.00314225568543354\\
182	0.00314213777668128\\
183	0.00314201783021375\\
184	0.00314189581062462\\
185	0.00314177168189044\\
186	0.00314164540735863\\
187	0.00314151694973542\\
188	0.00314138627107351\\
189	0.00314125333275963\\
190	0.00314111809550184\\
191	0.00314098051931684\\
192	0.00314084056351704\\
193	0.00314069818669758\\
194	0.00314055334672318\\
195	0.00314040600071485\\
196	0.00314025610503636\\
197	0.00314010361528055\\
198	0.00313994848625536\\
199	0.00313979067196957\\
200	0.00313963012561829\\
201	0.0031394667995682\\
202	0.00313930064534247\\
203	0.00313913161360541\\
204	0.00313895965414691\\
205	0.00313878471586639\\
206	0.00313860674675669\\
207	0.00313842569388749\\
208	0.00313824150338844\\
209	0.00313805412043207\\
210	0.00313786348921623\\
211	0.00313766955294633\\
212	0.00313747225381716\\
213	0.00313727153299439\\
214	0.00313706733059575\\
215	0.0031368595856718\\
216	0.00313664823618643\\
217	0.00313643321899681\\
218	0.0031362144698332\\
219	0.00313599192327818\\
220	0.0031357655127456\\
221	0.00313553517045906\\
222	0.00313530082743007\\
223	0.00313506241343567\\
224	0.00313481985699579\\
225	0.00313457308535002\\
226	0.00313432202443405\\
227	0.0031340665988556\\
228	0.00313380673186999\\
229	0.00313354234535508\\
230	0.00313327335978592\\
231	0.00313299969420882\\
232	0.00313272126621495\\
233	0.00313243799191342\\
234	0.00313214978590395\\
235	0.00313185656124885\\
236	0.00313155822944468\\
237	0.00313125470039322\\
238	0.00313094588237195\\
239	0.00313063168200402\\
240	0.00313031200422758\\
241	0.00312998675226461\\
242	0.0031296558275891\\
243	0.00312931912989473\\
244	0.00312897655706183\\
245	0.00312862800512388\\
246	0.0031282733682332\\
247	0.00312791253862621\\
248	0.00312754540658795\\
249	0.00312717186041587\\
250	0.00312679178638316\\
251	0.00312640506870121\\
252	0.00312601158948155\\
253	0.00312561122869699\\
254	0.00312520386414209\\
255	0.00312478937139297\\
256	0.00312436762376635\\
257	0.00312393849227786\\
258	0.00312350184559967\\
259	0.00312305755001726\\
260	0.0031226054693856\\
261	0.00312214546508438\\
262	0.00312167739597262\\
263	0.00312120111834239\\
264	0.00312071648587183\\
265	0.00312022334957726\\
266	0.00311972155776458\\
267	0.00311921095597983\\
268	0.00311869138695886\\
269	0.00311816269057627\\
270	0.00311762470379345\\
271	0.00311707726060576\\
272	0.00311652019198895\\
273	0.00311595332584462\\
274	0.00311537648694485\\
275	0.00311478949687602\\
276	0.00311419217398167\\
277	0.00311358433330457\\
278	0.00311296578652785\\
279	0.0031123363419153\\
280	0.00311169580425075\\
281	0.00311104397477658\\
282	0.00311038065113134\\
283	0.00310970562728657\\
284	0.00310901869348254\\
285	0.00310831963616339\\
286	0.00310760823791114\\
287	0.00310688427737895\\
288	0.00310614752922352\\
289	0.00310539776403659\\
290	0.00310463474827559\\
291	0.00310385824419343\\
292	0.00310306800976752\\
293	0.00310226379862781\\
294	0.00310144535998416\\
295	0.00310061243855285\\
296	0.0030997647744822\\
297	0.00309890210327755\\
298	0.00309802415572534\\
299	0.00309713065781651\\
300	0.00309622133066907\\
301	0.00309529589045001\\
302	0.00309435404829648\\
303	0.0030933955102362\\
304	0.00309241997710728\\
305	0.00309142714447724\\
306	0.00309041670256155\\
307	0.00308938833614131\\
308	0.00308834172448047\\
309	0.00308727654124234\\
310	0.00308619245440554\\
311	0.00308508912617928\\
312	0.00308396621291814\\
313	0.00308282336503617\\
314	0.00308166022692044\\
315	0.00308047643684404\\
316	0.00307927162687835\\
317	0.00307804542280484\\
318	0.00307679744402615\\
319	0.00307552730347649\\
320	0.00307423460753143\\
321	0.00307291895591687\\
322	0.00307157994161721\\
323	0.00307021715078276\\
324	0.00306883016263607\\
325	0.00306741854937735\\
326	0.00306598187608868\\
327	0.00306451970063701\\
328	0.0030630315735757\\
329	0.00306151703804454\\
330	0.00305997562966797\\
331	0.00305840687645137\\
332	0.0030568102986751\\
333	0.00305518540878608\\
334	0.00305353171128658\\
335	0.0030518487026199\\
336	0.00305013587105251\\
337	0.00304839269655235\\
338	0.00304661865066269\\
339	0.00304481319637114\\
340	0.00304297578797323\\
341	0.00304110587092995\\
342	0.00303920288171855\\
343	0.00303726624767596\\
344	0.00303529538683399\\
345	0.00303328970774543\\
346	0.00303124860930018\\
347	0.00302917148053046\\
348	0.00302705770040394\\
349	0.00302490663760374\\
350	0.00302271765029403\\
351	0.00302049008587002\\
352	0.00301822328069084\\
353	0.00301591655979405\\
354	0.00301356923659012\\
355	0.00301118061253527\\
356	0.00300874997678115\\
357	0.0030062766057994\\
358	0.00300375976297957\\
359	0.00300119869819815\\
360	0.00299859264735711\\
361	0.00299594083188986\\
362	0.00299324245823253\\
363	0.00299049671725858\\
364	0.00298770278367464\\
365	0.00298485981537552\\
366	0.00298196695275613\\
367	0.00297902331797829\\
368	0.00297602801419028\\
369	0.00297298012469691\\
370	0.00296987871207802\\
371	0.00296672281725328\\
372	0.00296351145849092\\
373	0.00296024363035836\\
374	0.00295691830261222\\
375	0.00295353441902534\\
376	0.00295009089614794\\
377	0.00294658662200026\\
378	0.00294302045469302\\
379	0.00293939122097195\\
380	0.00293569771468192\\
381	0.00293193869514513\\
382	0.00292811288544692\\
383	0.0029242189706215\\
384	0.00292025559572781\\
385	0.00291622136380439\\
386	0.00291211483368896\\
387	0.00290793451768638\\
388	0.00290367887906473\\
389	0.00289934632935552\\
390	0.00289493522542896\\
391	0.00289044386630764\\
392	0.00288587048967122\\
393	0.00288121326799\\
394	0.00287647030419765\\
395	0.00287163962676288\\
396	0.00286671918390948\\
397	0.00286170683650244\\
398	0.00285660034861841\\
399	0.00285139737384472\\
400	0.00284609543383834\\
401	0.00284069188478618\\
402	0.00283518387313868\\
403	0.0028295683130977\\
404	0.00282384200316826\\
405	0.00281800200718434\\
406	0.0028120454234071\\
407	0.00280596865857969\\
408	0.0027997679172769\\
409	0.00279343919198299\\
410	0.00278697825900345\\
411	0.00278038068760407\\
412	0.00277364187577775\\
413	0.00276675713162891\\
414	0.00275972180882898\\
415	0.00275253142415608\\
416	0.00274518141276298\\
417	0.0027376656857939\\
418	0.00272997422962037\\
419	0.00272209965955078\\
420	0.00271403454251588\\
421	0.00270577084354978\\
422	0.00269729985161397\\
423	0.00268861209420816\\
424	0.00267969723882085\\
425	0.00267054397888468\\
426	0.00266113990141535\\
427	0.00265147133286866\\
428	0.00264152315879101\\
429	0.00263127861120611\\
430	0.00262071901415549\\
431	0.00260982346880575\\
432	0.00259856843392222\\
433	0.00258692707942594\\
434	0.00257486804188674\\
435	0.00256235239401671\\
436	0.00254932489016489\\
437	0.00253568608287143\\
438	0.00252119866904994\\
439	0.00250516269361007\\
440	0.00248526366011957\\
441	0.00246288514601484\\
442	0.00243997915080848\\
443	0.00241652916105559\\
444	0.0023925180371453\\
445	0.00236792800599649\\
446	0.00234274065856143\\
447	0.00231693695337167\\
448	0.00229049722759709\\
449	0.00226340121732881\\
450	0.00223562808895717\\
451	0.00220715648332688\\
452	0.00217796457304159\\
453	0.00214803012875291\\
454	0.00211733057709868\\
455	0.0020858429982616\\
456	0.00205354392497408\\
457	0.00202040860346689\\
458	0.00198640900300285\\
459	0.00195150951581333\\
460	0.0019156610278376\\
461	0.00187880925939435\\
462	0.00184101668332657\\
463	0.00180306060647431\\
464	0.00176695657113213\\
465	0.00174002127743041\\
466	0.00171255806413367\\
467	0.00168454871256319\\
468	0.00165596653527007\\
469	0.0016267693496688\\
470	0.00159690408596396\\
471	0.00156638900126274\\
472	0.001535271194343\\
473	0.00150354977120314\\
474	0.00147124695810853\\
475	0.0014384333574062\\
476	0.00140525426127058\\
477	0.00137184892090271\\
478	0.00133779609282825\\
479	0.00130297800531827\\
480	0.0012673700781407\\
481	0.00123094637685959\\
482	0.00119367951199304\\
483	0.00115554057384795\\
484	0.00111649904362321\\
485	0.00107652267505202\\
486	0.00103557736814537\\
487	0.00099362708868859\\
488	0.00095063371984951\\
489	0.000906556731923931\\
490	0.000861352959695491\\
491	0.00081497635688427\\
492	0.000767377725191132\\
493	0.000718504413599808\\
494	0.0006682999778587\\
495	0.000616703772697162\\
496	0.000563650397617503\\
497	0.000509068764401486\\
498	0.000452880105573953\\
499	0.000394992927758715\\
500	0.000335289074672282\\
501	0.00027358392453668\\
502	0.000209511731993176\\
503	0.000142196249937518\\
504	6.92959835515677e-05\\
505	0\\
506	0\\
507	0\\
508	0\\
509	0\\
510	0\\
511	0\\
512	0\\
513	0\\
514	0\\
515	0\\
516	0\\
517	0\\
518	0\\
519	0\\
520	0\\
521	0\\
522	0\\
523	0\\
524	0\\
525	0\\
526	0\\
527	0\\
528	0\\
529	0\\
530	0\\
531	0\\
532	0\\
533	0\\
534	0\\
535	0\\
536	0\\
537	0\\
538	0\\
539	0\\
540	0\\
541	0\\
542	0\\
543	0\\
544	0\\
545	0\\
546	0\\
547	0\\
548	0\\
549	0\\
550	0\\
551	0\\
552	0\\
553	0\\
554	0\\
555	0\\
556	0\\
557	0\\
558	0\\
559	0\\
560	0\\
561	0\\
562	0\\
563	0\\
564	0\\
565	0\\
566	0\\
567	0\\
568	0\\
569	0\\
570	0\\
571	0\\
572	0\\
573	0\\
574	0\\
575	0\\
576	0\\
577	0\\
578	0\\
579	0\\
580	0\\
581	0\\
582	0\\
583	0\\
584	0\\
585	0\\
586	0\\
587	0\\
588	0\\
589	0\\
590	0\\
591	0\\
592	0\\
593	0\\
594	0\\
595	0\\
596	0\\
597	0\\
598	0\\
599	0\\
600	0\\
};
\addplot [color=blue,solid,forget plot]
  table[row sep=crcr]{%
1	0.00391145423208649\\
2	0.00391145359632034\\
3	0.00391145294986301\\
4	0.00391145229253474\\
5	0.00391145162415274\\
6	0.00391145094453115\\
7	0.00391145025348098\\
8	0.00391144955081003\\
9	0.00391144883632287\\
10	0.00391144810982078\\
11	0.00391144737110169\\
12	0.00391144661996008\\
13	0.00391144585618702\\
14	0.00391144507957\\
15	0.00391144428989295\\
16	0.00391144348693615\\
17	0.00391144267047614\\
18	0.00391144184028571\\
19	0.0039114409961338\\
20	0.00391144013778546\\
21	0.00391143926500173\\
22	0.00391143837753964\\
23	0.00391143747515211\\
24	0.00391143655758784\\
25	0.00391143562459133\\
26	0.00391143467590271\\
27	0.00391143371125772\\
28	0.00391143273038765\\
29	0.00391143173301919\\
30	0.00391143071887444\\
31	0.00391142968767075\\
32	0.00391142863912071\\
33	0.00391142757293201\\
34	0.00391142648880739\\
35	0.00391142538644453\\
36	0.003911424265536\\
37	0.00391142312576913\\
38	0.00391142196682593\\
39	0.00391142078838303\\
40	0.00391141959011152\\
41	0.00391141837167694\\
42	0.0039114171327391\\
43	0.00391141587295205\\
44	0.00391141459196392\\
45	0.00391141328941686\\
46	0.00391141196494692\\
47	0.00391141061818395\\
48	0.00391140924875146\\
49	0.00391140785626657\\
50	0.00391140644033982\\
51	0.00391140500057516\\
52	0.00391140353656969\\
53	0.00391140204791371\\
54	0.00391140053419044\\
55	0.003911398994976\\
56	0.00391139742983927\\
57	0.00391139583834171\\
58	0.00391139422003729\\
59	0.00391139257447231\\
60	0.00391139090118532\\
61	0.00391138919970692\\
62	0.00391138746955967\\
63	0.00391138571025791\\
64	0.00391138392130764\\
65	0.00391138210220638\\
66	0.00391138025244296\\
67	0.00391137837149746\\
68	0.00391137645884097\\
69	0.00391137451393546\\
70	0.00391137253623364\\
71	0.00391137052517878\\
72	0.00391136848020449\\
73	0.00391136640073466\\
74	0.00391136428618315\\
75	0.00391136213595374\\
76	0.00391135994943986\\
77	0.00391135772602443\\
78	0.00391135546507968\\
79	0.00391135316596697\\
80	0.00391135082803654\\
81	0.00391134845062738\\
82	0.00391134603306697\\
83	0.00391134357467112\\
84	0.00391134107474372\\
85	0.00391133853257653\\
86	0.003911335947449\\
87	0.00391133331862798\\
88	0.00391133064536757\\
89	0.00391132792690882\\
90	0.00391132516247953\\
91	0.003911322351294\\
92	0.00391131949255279\\
93	0.00391131658544247\\
94	0.00391131362913536\\
95	0.00391131062278926\\
96	0.00391130756554721\\
97	0.00391130445653723\\
98	0.00391130129487198\\
99	0.00391129807964859\\
100	0.00391129480994825\\
101	0.00391129148483605\\
102	0.00391128810336056\\
103	0.00391128466455363\\
104	0.00391128116743004\\
105	0.00391127761098719\\
106	0.00391127399420479\\
107	0.00391127031604454\\
108	0.0039112665754498\\
109	0.00391126277134525\\
110	0.00391125890263657\\
111	0.00391125496821008\\
112	0.00391125096693238\\
113	0.00391124689765002\\
114	0.00391124275918911\\
115	0.00391123855035498\\
116	0.00391123426993176\\
117	0.00391122991668204\\
118	0.00391122548934648\\
119	0.00391122098664338\\
120	0.00391121640726832\\
121	0.00391121174989372\\
122	0.00391120701316845\\
123	0.00391120219571741\\
124	0.00391119729614108\\
125	0.00391119231301512\\
126	0.0039111872448899\\
127	0.00391118209029005\\
128	0.00391117684771408\\
129	0.00391117151563381\\
130	0.003911166092494\\
131	0.0039111605767118\\
132	0.00391115496667634\\
133	0.00391114926074819\\
134	0.00391114345725893\\
135	0.00391113755451056\\
136	0.00391113155077509\\
137	0.00391112544429398\\
138	0.0039111192332776\\
139	0.00391111291590479\\
140	0.00391110649032221\\
141	0.0039110999546439\\
142	0.00391109330695071\\
143	0.0039110865452897\\
144	0.00391107966767366\\
145	0.00391107267208049\\
146	0.00391106555645263\\
147	0.00391105831869653\\
148	0.00391105095668199\\
149	0.00391104346824163\\
150	0.00391103585117025\\
151	0.00391102810322426\\
152	0.00391102022212101\\
153	0.00391101220553822\\
154	0.00391100405111328\\
155	0.00391099575644269\\
156	0.00391098731908134\\
157	0.00391097873654187\\
158	0.003910970006294\\
159	0.00391096112576384\\
160	0.00391095209233318\\
161	0.00391094290333881\\
162	0.00391093355607176\\
163	0.00391092404777658\\
164	0.0039109143756506\\
165	0.00391090453684312\\
166	0.00391089452845464\\
167	0.00391088434753605\\
168	0.00391087399108782\\
169	0.0039108634560591\\
170	0.00391085273934689\\
171	0.00391084183779514\\
172	0.00391083074819378\\
173	0.00391081946727783\\
174	0.0039108079917264\\
175	0.00391079631816164\\
176	0.0039107844431478\\
177	0.00391077236319006\\
178	0.00391076007473353\\
179	0.00391074757416205\\
180	0.00391073485779707\\
181	0.00391072192189647\\
182	0.00391070876265335\\
183	0.00391069537619475\\
184	0.00391068175858045\\
185	0.00391066790580162\\
186	0.00391065381377956\\
187	0.00391063947836435\\
188	0.00391062489533349\\
189	0.00391061006039058\\
190	0.00391059496916392\\
191	0.0039105796172051\\
192	0.00391056399998768\\
193	0.00391054811290567\\
194	0.00391053195127217\\
195	0.0039105155103179\\
196	0.00391049878518971\\
197	0.00391048177094907\\
198	0.00391046446257056\\
199	0.00391044685494027\\
200	0.00391042894285424\\
201	0.00391041072101682\\
202	0.00391039218403905\\
203	0.00391037332643693\\
204	0.00391035414262974\\
205	0.00391033462693829\\
206	0.00391031477358314\\
207	0.0039102945766828\\
208	0.00391027403025186\\
209	0.00391025312819916\\
210	0.00391023186432583\\
211	0.00391021023232338\\
212	0.00391018822577169\\
213	0.003910165838137\\
214	0.00391014306276988\\
215	0.00391011989290308\\
216	0.00391009632164944\\
217	0.00391007234199972\\
218	0.00391004794682034\\
219	0.00391002312885119\\
220	0.00390999788070329\\
221	0.00390997219485646\\
222	0.00390994606365694\\
223	0.003909919479315\\
224	0.0039098924339024\\
225	0.00390986491934995\\
226	0.00390983692744492\\
227	0.00390980844982839\\
228	0.00390977947799267\\
229	0.00390975000327855\\
230	0.00390972001687256\\
231	0.00390968950980415\\
232	0.00390965847294288\\
233	0.00390962689699544\\
234	0.00390959477250276\\
235	0.00390956208983697\\
236	0.00390952883919834\\
237	0.00390949501061212\\
238	0.00390946059392543\\
239	0.00390942557880395\\
240	0.00390938995472869\\
241	0.00390935371099259\\
242	0.00390931683669712\\
243	0.00390927932074883\\
244	0.0039092411518558\\
245	0.00390920231852401\\
246	0.00390916280905373\\
247	0.00390912261153578\\
248	0.0039090817138477\\
249	0.00390904010364997\\
250	0.00390899776838202\\
251	0.00390895469525827\\
252	0.00390891087126407\\
253	0.00390886628315158\\
254	0.00390882091743554\\
255	0.00390877476038906\\
256	0.00390872779803919\\
257	0.00390868001616263\\
258	0.00390863140028114\\
259	0.00390858193565703\\
260	0.00390853160728851\\
261	0.00390848039990502\\
262	0.0039084282979624\\
263	0.00390837528563805\\
264	0.003908321346826\\
265	0.0039082664651319\\
266	0.00390821062386793\\
267	0.00390815380604762\\
268	0.0039080959943806\\
269	0.00390803717126733\\
270	0.00390797731879361\\
271	0.00390791641872519\\
272	0.00390785445250214\\
273	0.00390779140123325\\
274	0.0039077272456903\\
275	0.00390766196630227\\
276	0.00390759554314946\\
277	0.00390752795595753\\
278	0.0039074591840915\\
279	0.00390738920654961\\
280	0.00390731800195713\\
281	0.00390724554856015\\
282	0.00390717182421921\\
283	0.00390709680640285\\
284	0.00390702047218123\\
285	0.00390694279821945\\
286	0.00390686376077104\\
287	0.00390678333567118\\
288	0.00390670149832998\\
289	0.00390661822372562\\
290	0.00390653348639748\\
291	0.00390644726043916\\
292	0.00390635951949148\\
293	0.00390627023673536\\
294	0.00390617938488474\\
295	0.00390608693617933\\
296	0.00390599286237742\\
297	0.00390589713474855\\
298	0.00390579972406615\\
299	0.00390570060060022\\
300	0.00390559973410983\\
301	0.00390549709383573\\
302	0.00390539264849274\\
303	0.00390528636626233\\
304	0.00390517821478497\\
305	0.00390506816115258\\
306	0.00390495617190089\\
307	0.00390484221300179\\
308	0.00390472624985571\\
309	0.00390460824728389\\
310	0.00390448816952074\\
311	0.00390436598020609\\
312	0.00390424164237752\\
313	0.0039041151184626\\
314	0.00390398637027116\\
315	0.00390385535898758\\
316	0.00390372204516303\\
317	0.0039035863887077\\
318	0.00390344834888305\\
319	0.00390330788429402\\
320	0.00390316495288124\\
321	0.00390301951191325\\
322	0.0039028715179786\\
323	0.00390272092697804\\
324	0.00390256769411655\\
325	0.00390241177389542\\
326	0.0039022531201042\\
327	0.00390209168581261\\
328	0.00390192742336232\\
329	0.00390176028435864\\
330	0.00390159021966208\\
331	0.00390141717937973\\
332	0.00390124111285647\\
333	0.00390106196866593\\
334	0.00390087969460124\\
335	0.00390069423766547\\
336	0.0039005055440617\\
337	0.00390031355918274\\
338	0.00390011822760038\\
339	0.00389991949305422\\
340	0.00389971729843982\\
341	0.0038995115857963\\
342	0.00389930229629321\\
343	0.00389908937021664\\
344	0.00389887274695436\\
345	0.00389865236498013\\
346	0.00389842816183678\\
347	0.0038982000741182\\
348	0.00389796803745\\
349	0.00389773198646871\\
350	0.00389749185479949\\
351	0.003897247575032\\
352	0.00389699907869457\\
353	0.00389674629622624\\
354	0.00389648915694669\\
355	0.00389622758902382\\
356	0.00389596151943881\\
357	0.00389569087394843\\
358	0.0038954155770446\\
359	0.00389513555191069\\
360	0.00389485072037473\\
361	0.00389456100285906\\
362	0.00389426631832628\\
363	0.00389396658422146\\
364	0.00389366171641021\\
365	0.00389335162911248\\
366	0.00389303623483195\\
367	0.00389271544428074\\
368	0.00389238916629931\\
369	0.00389205730777134\\
370	0.0038917197735333\\
371	0.00389137646627871\\
372	0.00389102728645676\\
373	0.00389067213216507\\
374	0.0038903108990365\\
375	0.00388994348011961\\
376	0.0038895697657526\\
377	0.00388918964343038\\
378	0.00388880299766439\\
379	0.00388840970983479\\
380	0.00388800965803425\\
381	0.003887602716903\\
382	0.00388718875745404\\
383	0.00388676764688758\\
384	0.00388633924839353\\
385	0.00388590342094045\\
386	0.0038854600190492\\
387	0.00388500889254906\\
388	0.00388454988631361\\
389	0.00388408283997319\\
390	0.00388360758759974\\
391	0.00388312395735876\\
392	0.0038826317711215\\
393	0.00388213084402801\\
394	0.00388162098398701\\
395	0.00388110199109091\\
396	0.00388057365691044\\
397	0.00388003576361041\\
398	0.0038794880827994\\
399	0.00387893037401931\\
400	0.00387836238289632\\
401	0.00387778383944593\\
402	0.00387719445811215\\
403	0.00387659394204268\\
404	0.00387598198951732\\
405	0.00387535828191595\\
406	0.00387472246678478\\
407	0.00387407417172406\\
408	0.0038734130033376\\
409	0.00387273854653858\\
410	0.00387205036454529\\
411	0.00387134799997831\\
412	0.00387063097718826\\
413	0.00386989880449444\\
414	0.00386915097106735\\
415	0.00386838692695242\\
416	0.00386760603773274\\
417	0.00386680756080642\\
418	0.00386599080346426\\
419	0.00386515503338101\\
420	0.00386429945743454\\
421	0.00386342321410664\\
422	0.00386252536473786\\
423	0.00386160488343816\\
424	0.00386066064541408\\
425	0.00385969141341615\\
426	0.00385869582192358\\
427	0.00385767235852507\\
428	0.00385661934161984\\
429	0.00385553489274932\\
430	0.00385441689976431\\
431	0.00385326296140546\\
432	0.00385207028869819\\
433	0.00385083549785069\\
434	0.00384955412228839\\
435	0.00384821940056294\\
436	0.00384681926064951\\
437	0.0038453291565604\\
438	0.00384369704008515\\
439	0.00384182192617243\\
440	0.00383957929471537\\
441	0.00383713441622716\\
442	0.00383464069763525\\
443	0.00383209695164383\\
444	0.0038295019659953\\
445	0.00382685450630891\\
446	0.00382415331976878\\
447	0.00382139713981654\\
448	0.00381858469201794\\
449	0.00381571470126403\\
450	0.00381278590039141\\
451	0.00380979704004449\\
452	0.00380674689886397\\
453	0.00380363429115159\\
454	0.00380045806428968\\
455	0.00379721706609648\\
456	0.00379391003229393\\
457	0.00379053526911188\\
458	0.00378708980565845\\
459	0.00378356710029378\\
460	0.00377995040566156\\
461	0.00377619143105491\\
462	0.0037721334462718\\
463	0.0037672220531096\\
464	0.00375953648037518\\
465	0.00374259590734272\\
466	0.00372533544350868\\
467	0.00370774507665263\\
468	0.00368981399443503\\
469	0.00367153110135167\\
470	0.00365288634541003\\
471	0.00363386936568687\\
472	0.00361446838002628\\
473	0.00359467186430756\\
474	0.00357446889131582\\
475	0.00355384847238614\\
476	0.00353279567495486\\
477	0.00351128549018164\\
478	0.00348929938958904\\
479	0.00346681983209936\\
480	0.00344382812335562\\
481	0.00342030431059674\\
482	0.00339622706474022\\
483	0.00337157354405513\\
484	0.00334631923702561\\
485	0.00332043778241085\\
486	0.00329390076330926\\
487	0.0032666774663388\\
488	0.00323873460222879\\
489	0.0032100359918576\\
490	0.00318054220155812\\
491	0.00315021011715184\\
492	0.00311899244306825\\
493	0.00308683710781825\\
494	0.00305368654728657\\
495	0.00301947681520266\\
496	0.0029841364142171\\
497	0.0029475845893136\\
498	0.00290972839773051\\
499	0.00287045663723923\\
500	0.00282962510700878\\
501	0.00278701696808144\\
502	0.00274222972852771\\
503	0.00269434170656592\\
504	0.00264090605458067\\
505	0.00257465229279033\\
506	0.00249789749396208\\
507	0.00241955633568501\\
508	0.00233969171468241\\
509	0.00225848184594896\\
510	0.00217611043500788\\
511	0.00209209122396565\\
512	0.00200591458078461\\
513	0.00191746610396971\\
514	0.00182662038391383\\
515	0.00173323967234684\\
516	0.00163717243645703\\
517	0.00153825200793115\\
518	0.00143629628630456\\
519	0.00133111244935181\\
520	0.00122252232193912\\
521	0.00111046918952441\\
522	0.000995439653286779\\
523	0.000880092985072983\\
524	0.000775499111761147\\
525	0.000689867989463739\\
526	0.000601110061332961\\
527	0.000508813764893824\\
528	0.000412124598928745\\
529	0.000308800153975222\\
530	0.000192070774563915\\
531	5.80908334167806e-05\\
532	0\\
533	0\\
534	0\\
535	0\\
536	0\\
537	0\\
538	0\\
539	0\\
540	0\\
541	0\\
542	0\\
543	0\\
544	0\\
545	0\\
546	0\\
547	0\\
548	0\\
549	0\\
550	0\\
551	0\\
552	0\\
553	0\\
554	0\\
555	0\\
556	0\\
557	0\\
558	0\\
559	0\\
560	0\\
561	0\\
562	0\\
563	0\\
564	0\\
565	0\\
566	0\\
567	0\\
568	0\\
569	0\\
570	0\\
571	0\\
572	0\\
573	0\\
574	0\\
575	0\\
576	0\\
577	0\\
578	0\\
579	0\\
580	0\\
581	0\\
582	0\\
583	0\\
584	0\\
585	0\\
586	0\\
587	0\\
588	0\\
589	0\\
590	0\\
591	0\\
592	0\\
593	0\\
594	0\\
595	0\\
596	0\\
597	0\\
598	0\\
599	0\\
600	0\\
};
\addplot [color=mycolor10,solid,forget plot]
  table[row sep=crcr]{%
1	0.00400023926843176\\
2	0.00400023923938983\\
3	0.00400023920985951\\
4	0.0040002391798326\\
5	0.00400023914930074\\
6	0.00400023911825545\\
7	0.00400023908668808\\
8	0.00400023905458986\\
9	0.00400023902195187\\
10	0.00400023898876501\\
11	0.00400023895502006\\
12	0.00400023892070763\\
13	0.00400023888581818\\
14	0.00400023885034198\\
15	0.00400023881426918\\
16	0.00400023877758973\\
17	0.00400023874029343\\
18	0.00400023870236989\\
19	0.00400023866380856\\
20	0.00400023862459869\\
21	0.00400023858472939\\
22	0.00400023854418953\\
23	0.00400023850296783\\
24	0.00400023846105281\\
25	0.0040002384184328\\
26	0.00400023837509592\\
27	0.00400023833103009\\
28	0.00400023828622305\\
29	0.00400023824066229\\
30	0.00400023819433513\\
31	0.00400023814722864\\
32	0.0040002380993297\\
33	0.00400023805062494\\
34	0.00400023800110079\\
35	0.00400023795074342\\
36	0.0040002378995388\\
37	0.00400023784747262\\
38	0.00400023779453037\\
39	0.00400023774069726\\
40	0.00400023768595826\\
41	0.00400023763029808\\
42	0.0040002375737012\\
43	0.00400023751615178\\
44	0.00400023745763376\\
45	0.00400023739813077\\
46	0.00400023733762619\\
47	0.00400023727610309\\
48	0.00400023721354428\\
49	0.00400023714993225\\
50	0.0040002370852492\\
51	0.00400023701947703\\
52	0.00400023695259733\\
53	0.00400023688459137\\
54	0.0040002368154401\\
55	0.00400023674512415\\
56	0.00400023667362381\\
57	0.00400023660091903\\
58	0.00400023652698944\\
59	0.00400023645181429\\
60	0.00400023637537249\\
61	0.00400023629764258\\
62	0.00400023621860273\\
63	0.00400023613823075\\
64	0.00400023605650405\\
65	0.00400023597339967\\
66	0.00400023588889424\\
67	0.00400023580296397\\
68	0.00400023571558471\\
69	0.00400023562673184\\
70	0.00400023553638036\\
71	0.0040002354445048\\
72	0.00400023535107926\\
73	0.00400023525607742\\
74	0.00400023515947247\\
75	0.00400023506123715\\
76	0.00400023496134373\\
77	0.00400023485976398\\
78	0.00400023475646921\\
79	0.00400023465143022\\
80	0.00400023454461729\\
81	0.00400023443600019\\
82	0.00400023432554817\\
83	0.00400023421322995\\
84	0.00400023409901369\\
85	0.004000233982867\\
86	0.00400023386475693\\
87	0.00400023374464995\\
88	0.00400023362251195\\
89	0.00400023349830821\\
90	0.00400023337200343\\
91	0.00400023324356165\\
92	0.00400023311294632\\
93	0.00400023298012024\\
94	0.00400023284504553\\
95	0.00400023270768369\\
96	0.00400023256799551\\
97	0.0040002324259411\\
98	0.00400023228147987\\
99	0.00400023213457053\\
100	0.00400023198517103\\
101	0.00400023183323862\\
102	0.00400023167872976\\
103	0.00400023152160017\\
104	0.00400023136180476\\
105	0.00400023119929767\\
106	0.00400023103403222\\
107	0.00400023086596091\\
108	0.00400023069503539\\
109	0.00400023052120645\\
110	0.00400023034442404\\
111	0.00400023016463718\\
112	0.00400022998179403\\
113	0.0040002297958418\\
114	0.00400022960672678\\
115	0.0040002294143943\\
116	0.00400022921878872\\
117	0.00400022901985342\\
118	0.00400022881753078\\
119	0.00400022861176213\\
120	0.00400022840248779\\
121	0.004000228189647\\
122	0.00400022797317793\\
123	0.00400022775301766\\
124	0.00400022752910214\\
125	0.00400022730136618\\
126	0.00400022706974344\\
127	0.00400022683416641\\
128	0.00400022659456636\\
129	0.00400022635087337\\
130	0.00400022610301626\\
131	0.00400022585092258\\
132	0.00400022559451863\\
133	0.00400022533372936\\
134	0.00400022506847843\\
135	0.00400022479868814\\
136	0.00400022452427939\\
137	0.00400022424517172\\
138	0.00400022396128322\\
139	0.00400022367253056\\
140	0.00400022337882892\\
141	0.004000223080092\\
142	0.00400022277623198\\
143	0.00400022246715948\\
144	0.00400022215278359\\
145	0.00400022183301175\\
146	0.00400022150774984\\
147	0.00400022117690204\\
148	0.0040002208403709\\
149	0.00400022049805724\\
150	0.00400022014986016\\
151	0.00400021979567702\\
152	0.00400021943540338\\
153	0.00400021906893297\\
154	0.00400021869615773\\
155	0.00400021831696768\\
156	0.00400021793125095\\
157	0.00400021753889377\\
158	0.00400021713978037\\
159	0.004000216733793\\
160	0.00400021632081189\\
161	0.0040002159007152\\
162	0.00400021547337901\\
163	0.00400021503867728\\
164	0.00400021459648178\\
165	0.00400021414666211\\
166	0.00400021368908563\\
167	0.00400021322361742\\
168	0.00400021275012026\\
169	0.00400021226845458\\
170	0.00400021177847842\\
171	0.00400021128004738\\
172	0.00400021077301461\\
173	0.0040002102572307\\
174	0.0040002097325437\\
175	0.00400020919879907\\
176	0.00400020865583955\\
177	0.00400020810350523\\
178	0.00400020754163341\\
179	0.00400020697005859\\
180	0.00400020638861238\\
181	0.00400020579712349\\
182	0.00400020519541765\\
183	0.00400020458331754\\
184	0.00400020396064278\\
185	0.0040002033272098\\
186	0.00400020268283185\\
187	0.00400020202731888\\
188	0.00400020136047753\\
189	0.00400020068211103\\
190	0.00400019999201917\\
191	0.00400019928999819\\
192	0.00400019857584077\\
193	0.00400019784933593\\
194	0.00400019711026898\\
195	0.00400019635842145\\
196	0.004000195593571\\
197	0.00400019481549139\\
198	0.0040001940239524\\
199	0.0040001932187197\\
200	0.00400019239955489\\
201	0.0040001915662153\\
202	0.004000190718454\\
203	0.0040001898560197\\
204	0.00400018897865665\\
205	0.0040001880861046\\
206	0.00400018717809867\\
207	0.00400018625436929\\
208	0.00400018531464214\\
209	0.00400018435863801\\
210	0.00400018338607277\\
211	0.00400018239665722\\
212	0.00400018139009704\\
213	0.00400018036609271\\
214	0.00400017932433935\\
215	0.0040001782645267\\
216	0.00400017718633895\\
217	0.00400017608945471\\
218	0.00400017497354685\\
219	0.00400017383828242\\
220	0.00400017268332256\\
221	0.00400017150832235\\
222	0.00400017031293074\\
223	0.00400016909679041\\
224	0.00400016785953769\\
225	0.0040001666008024\\
226	0.00400016532020775\\
227	0.00400016401737025\\
228	0.00400016269189953\\
229	0.00400016134339824\\
230	0.00400015997146196\\
231	0.00400015857567902\\
232	0.00400015715563037\\
233	0.00400015571088948\\
234	0.00400015424102217\\
235	0.00400015274558649\\
236	0.00400015122413256\\
237	0.00400014967620246\\
238	0.00400014810133003\\
239	0.00400014649904076\\
240	0.00400014486885164\\
241	0.00400014321027096\\
242	0.00400014152279822\\
243	0.00400013980592388\\
244	0.0040001380591293\\
245	0.00400013628188648\\
246	0.00400013447365795\\
247	0.00400013263389657\\
248	0.00400013076204536\\
249	0.00400012885753732\\
250	0.00400012691979526\\
251	0.00400012494823157\\
252	0.00400012294224811\\
253	0.00400012090123594\\
254	0.00400011882457518\\
255	0.00400011671163477\\
256	0.0040001145617723\\
257	0.00400011237433378\\
258	0.00400011014865345\\
259	0.00400010788405356\\
260	0.00400010557984417\\
261	0.0040001032353229\\
262	0.0040001008497747\\
263	0.00400009842247171\\
264	0.00400009595267291\\
265	0.00400009343962396\\
266	0.00400009088255696\\
267	0.00400008828069019\\
268	0.00400008563322785\\
269	0.00400008293935985\\
270	0.00400008019826155\\
271	0.00400007740909347\\
272	0.00400007457100108\\
273	0.00400007168311449\\
274	0.00400006874454821\\
275	0.00400006575440089\\
276	0.004000062711755\\
277	0.00400005961567661\\
278	0.00400005646521507\\
279	0.00400005325940272\\
280	0.00400004999725463\\
281	0.00400004667776828\\
282	0.00400004329992329\\
283	0.00400003986268108\\
284	0.00400003636498461\\
285	0.00400003280575806\\
286	0.00400002918390648\\
287	0.00400002549831555\\
288	0.00400002174785119\\
289	0.00400001793135928\\
290	0.00400001404766534\\
291	0.00400001009557416\\
292	0.00400000607386954\\
293	0.00400000198131388\\
294	0.0039999978166479\\
295	0.00399999357859029\\
296	0.00399998926583733\\
297	0.00399998487706262\\
298	0.00399998041091665\\
299	0.00399997586602652\\
300	0.00399997124099554\\
301	0.00399996653440293\\
302	0.00399996174480339\\
303	0.00399995687072683\\
304	0.00399995191067793\\
305	0.00399994686313584\\
306	0.00399994172655379\\
307	0.00399993649935871\\
308	0.00399993117995091\\
309	0.00399992576670368\\
310	0.00399992025796293\\
311	0.00399991465204682\\
312	0.00399990894724539\\
313	0.00399990314182019\\
314	0.00399989723400391\\
315	0.00399989122199999\\
316	0.00399988510398228\\
317	0.00399987887809461\\
318	0.00399987254245047\\
319	0.00399986609513257\\
320	0.00399985953419253\\
321	0.00399985285765041\\
322	0.00399984606349439\\
323	0.00399983914968035\\
324	0.00399983211413148\\
325	0.00399982495473788\\
326	0.00399981766935617\\
327	0.00399981025580903\\
328	0.00399980271188486\\
329	0.00399979503533729\\
330	0.00399978722388479\\
331	0.00399977927521018\\
332	0.00399977118696023\\
333	0.00399976295674514\\
334	0.00399975458213807\\
335	0.00399974606067465\\
336	0.00399973738985242\\
337	0.00399972856713033\\
338	0.00399971958992814\\
339	0.0039997104556258\\
340	0.00399970116156288\\
341	0.00399969170503784\\
342	0.00399968208330732\\
343	0.00399967229358544\\
344	0.00399966233304298\\
345	0.00399965219880651\\
346	0.00399964188795746\\
347	0.00399963139753122\\
348	0.00399962072451601\\
349	0.00399960986585184\\
350	0.00399959881842922\\
351	0.00399958757908794\\
352	0.00399957614461565\\
353	0.00399956451174636\\
354	0.00399955267715886\\
355	0.00399954063747494\\
356	0.00399952838925759\\
357	0.003999515929009\\
358	0.00399950325316837\\
359	0.00399949035810966\\
360	0.00399947724013911\\
361	0.00399946389549262\\
362	0.0039994503203329\\
363	0.0039994365107465\\
364	0.00399942246274054\\
365	0.00399940817223933\\
366	0.00399939363508073\\
367	0.00399937884701221\\
368	0.00399936380368681\\
369	0.00399934850065873\\
370	0.00399933293337871\\
371	0.00399931709718917\\
372	0.00399930098731901\\
373	0.00399928459887818\\
374	0.00399926792685191\\
375	0.00399925096609462\\
376	0.00399923371132354\\
377	0.00399921615711194\\
378	0.003999198297882\\
379	0.00399918012789727\\
380	0.00399916164125467\\
381	0.00399914283187607\\
382	0.00399912369349926\\
383	0.00399910421966836\\
384	0.00399908440372357\\
385	0.00399906423879013\\
386	0.0039990437177664\\
387	0.00399902283331092\\
388	0.00399900157782828\\
389	0.00399897994345366\\
390	0.0039989579220355\\
391	0.00399893550511629\\
392	0.00399891268391063\\
393	0.0039988894492798\\
394	0.00399886579170178\\
395	0.00399884170123454\\
396	0.00399881716747023\\
397	0.00399879217947742\\
398	0.00399876672573156\\
399	0.00399874079404327\\
400	0.00399871437151541\\
401	0.00399868744458336\\
402	0.00399865999916131\\
403	0.00399863202070385\\
404	0.00399860349370784\\
405	0.00399857440128768\\
406	0.00399854472560509\\
407	0.00399851444782748\\
408	0.00399848354810668\\
409	0.00399845200558938\\
410	0.00399841979846344\\
411	0.00399838690401134\\
412	0.00399835329856065\\
413	0.00399831895709304\\
414	0.00399828385224411\\
415	0.0039982479529837\\
416	0.00399821122496071\\
417	0.00399817363471804\\
418	0.00399813514658468\\
419	0.00399809572187645\\
420	0.00399805531851538\\
421	0.00399801389059089\\
422	0.00399797138785207\\
423	0.00399792775511748\\
424	0.00399788293158426\\
425	0.00399783685000886\\
426	0.00399778943571111\\
427	0.00399774060530526\\
428	0.00399769026494495\\
429	0.00399763830758216\\
430	0.0039975846080481\\
431	0.0039975290131297\\
432	0.00399747132012266\\
433	0.00399741122957038\\
434	0.00399734824351354\\
435	0.00399728146088603\\
436	0.00399720922077238\\
437	0.00399712866805858\\
438	0.00399703587301835\\
439	0.00399692855921201\\
440	0.00399681364852388\\
441	0.00399669628499214\\
442	0.00399657640237339\\
443	0.00399645393267616\\
444	0.00399632880627346\\
445	0.00399620095205716\\
446	0.00399607029764142\\
447	0.00399593676962055\\
448	0.00399580029388026\\
449	0.00399566079594179\\
450	0.00399551820126536\\
451	0.00399537243530081\\
452	0.00399522342271854\\
453	0.00399507108434469\\
454	0.00399491532797951\\
455	0.00399475602314739\\
456	0.00399459293353915\\
457	0.00399442553680086\\
458	0.00399425254035359\\
459	0.0039940705701746\\
460	0.00399387062197271\\
461	0.00399362865816175\\
462	0.00399328224082247\\
463	0.00399268179826925\\
464	0.00399152034506261\\
465	0.00398942525004241\\
466	0.003987295346337\\
467	0.00398512966627529\\
468	0.00398292721000522\\
469	0.00398068697662909\\
470	0.00397840791133193\\
471	0.00397608888287425\\
472	0.00397372874284859\\
473	0.00397132632195255\\
474	0.00396888037428475\\
475	0.00396638943872895\\
476	0.00396385175245868\\
477	0.0039612657638208\\
478	0.00395862990298715\\
479	0.00395594249982952\\
480	0.00395320177500551\\
481	0.00395040582985437\\
482	0.0039475526346765\\
483	0.00394464001520579\\
484	0.00394166563706097\\
485	0.00393862698785699\\
486	0.00393552135645141\\
487	0.00393234580918657\\
488	0.00392909716312208\\
489	0.00392577195493935\\
490	0.00392236640455913\\
491	0.00391887637213836\\
492	0.00391529730638808\\
493	0.0039116241805659\\
494	0.00390785140870817\\
495	0.0039039727252085\\
496	0.00389998098680041\\
497	0.00389586779479689\\
498	0.0038916226814718\\
499	0.00388723122648509\\
500	0.00388267058066376\\
501	0.0038778989473318\\
502	0.00387283208197042\\
503	0.00386729674842656\\
504	0.00386096597286974\\
505	0.0038534149944631\\
506	0.00384507261769542\\
507	0.00383665866359003\\
508	0.00382817739036378\\
509	0.00381962802362775\\
510	0.00381099192896018\\
511	0.00380225392818428\\
512	0.00379341008661713\\
513	0.00378445601683264\\
514	0.00377538679341138\\
515	0.0037661967812744\\
516	0.0037568792082255\\
517	0.00374742495454606\\
518	0.00373781888331623\\
519	0.00372802825804615\\
520	0.00371796508866977\\
521	0.00370736053989885\\
522	0.00369533602505041\\
523	0.00367890636217919\\
524	0.0036476543580472\\
525	0.00359447189176134\\
526	0.00353963152376423\\
527	0.0034829720449033\\
528	0.00342425368577073\\
529	0.00336310709277027\\
530	0.00329908329211729\\
531	0.00323217575810164\\
532	0.00316298165281235\\
533	0.00309166511403029\\
534	0.00301814370668786\\
535	0.00294111313161606\\
536	0.00285499501479115\\
537	0.00274701482624045\\
538	0.00261820957907621\\
539	0.0024858002176776\\
540	0.00234954247109751\\
541	0.00220916502212144\\
542	0.0020643662871162\\
543	0.00191481016471741\\
544	0.00176011232967979\\
545	0.00159980419294568\\
546	0.00143325509308531\\
547	0.00125945611117335\\
548	0.00107633464582706\\
549	0.000878403440878852\\
550	0.000652863504763817\\
551	0.00041853810151925\\
552	0.00017128870103758\\
553	0\\
554	0\\
555	0\\
556	0\\
557	0\\
558	0\\
559	0\\
560	0\\
561	0\\
562	0\\
563	0\\
564	0\\
565	0\\
566	0\\
567	0\\
568	0\\
569	0\\
570	0\\
571	0\\
572	0\\
573	0\\
574	0\\
575	0\\
576	0\\
577	0\\
578	0\\
579	0\\
580	0\\
581	0\\
582	0\\
583	0\\
584	0\\
585	0\\
586	0\\
587	0\\
588	0\\
589	0\\
590	0\\
591	0\\
592	0\\
593	0\\
594	0\\
595	0\\
596	0\\
597	0\\
598	0\\
599	0\\
600	0\\
};
\addplot [color=mycolor11,solid,forget plot]
  table[row sep=crcr]{%
1	0.00406147387421855\\
2	0.00406147387294227\\
3	0.00406147387164454\\
4	0.00406147387032498\\
5	0.00406147386898323\\
6	0.00406147386761891\\
7	0.00406147386623165\\
8	0.00406147386482107\\
9	0.00406147386338676\\
10	0.00406147386192833\\
11	0.00406147386044538\\
12	0.00406147385893749\\
13	0.00406147385740424\\
14	0.0040614738558452\\
15	0.00406147385425995\\
16	0.00406147385264803\\
17	0.004061473851009\\
18	0.00406147384934242\\
19	0.0040614738476478\\
20	0.00406147384592468\\
21	0.00406147384417258\\
22	0.00406147384239101\\
23	0.00406147384057948\\
24	0.00406147383873747\\
25	0.00406147383686449\\
26	0.00406147383495999\\
27	0.00406147383302347\\
28	0.00406147383105436\\
29	0.00406147382905214\\
30	0.00406147382701623\\
31	0.00406147382494607\\
32	0.00406147382284109\\
33	0.00406147382070069\\
34	0.00406147381852427\\
35	0.00406147381631124\\
36	0.00406147381406098\\
37	0.00406147381177285\\
38	0.00406147380944621\\
39	0.00406147380708042\\
40	0.00406147380467482\\
41	0.00406147380222874\\
42	0.00406147379974148\\
43	0.00406147379721236\\
44	0.00406147379464067\\
45	0.0040614737920257\\
46	0.0040614737893667\\
47	0.00406147378666293\\
48	0.00406147378391364\\
49	0.00406147378111807\\
50	0.00406147377827542\\
51	0.0040614737753849\\
52	0.00406147377244571\\
53	0.00406147376945701\\
54	0.00406147376641797\\
55	0.00406147376332775\\
56	0.00406147376018546\\
57	0.00406147375699024\\
58	0.00406147375374118\\
59	0.00406147375043737\\
60	0.00406147374707789\\
61	0.00406147374366179\\
62	0.00406147374018812\\
63	0.00406147373665589\\
64	0.00406147373306411\\
65	0.00406147372941178\\
66	0.00406147372569786\\
67	0.00406147372192131\\
68	0.00406147371808107\\
69	0.00406147371417606\\
70	0.00406147371020517\\
71	0.00406147370616729\\
72	0.00406147370206127\\
73	0.00406147369788597\\
74	0.00406147369364019\\
75	0.00406147368932274\\
76	0.00406147368493241\\
77	0.00406147368046794\\
78	0.00406147367592809\\
79	0.00406147367131156\\
80	0.00406147366661704\\
81	0.00406147366184321\\
82	0.00406147365698872\\
83	0.00406147365205219\\
84	0.00406147364703221\\
85	0.00406147364192737\\
86	0.00406147363673621\\
87	0.00406147363145726\\
88	0.00406147362608902\\
89	0.00406147362062996\\
90	0.00406147361507853\\
91	0.00406147360943315\\
92	0.0040614736036922\\
93	0.00406147359785406\\
94	0.00406147359191705\\
95	0.00406147358587949\\
96	0.00406147357973964\\
97	0.00406147357349575\\
98	0.00406147356714604\\
99	0.00406147356068869\\
100	0.00406147355412185\\
101	0.00406147354744364\\
102	0.00406147354065214\\
103	0.00406147353374541\\
104	0.00406147352672145\\
105	0.00406147351957827\\
106	0.00406147351231378\\
107	0.00406147350492592\\
108	0.00406147349741255\\
109	0.0040614734897715\\
110	0.00406147348200058\\
111	0.00406147347409753\\
112	0.00406147346606009\\
113	0.00406147345788592\\
114	0.00406147344957267\\
115	0.00406147344111792\\
116	0.00406147343251923\\
117	0.0040614734237741\\
118	0.00406147341488001\\
119	0.00406147340583437\\
120	0.00406147339663454\\
121	0.00406147338727787\\
122	0.00406147337776163\\
123	0.00406147336808304\\
124	0.0040614733582393\\
125	0.00406147334822752\\
126	0.0040614733380448\\
127	0.00406147332768816\\
128	0.00406147331715458\\
129	0.00406147330644098\\
130	0.00406147329554423\\
131	0.00406147328446114\\
132	0.00406147327318846\\
133	0.0040614732617229\\
134	0.00406147325006109\\
135	0.00406147323819962\\
136	0.004061473226135\\
137	0.00406147321386371\\
138	0.00406147320138212\\
139	0.00406147318868658\\
140	0.00406147317577335\\
141	0.00406147316263864\\
142	0.00406147314927858\\
143	0.00406147313568924\\
144	0.00406147312186662\\
145	0.00406147310780666\\
146	0.0040614730935052\\
147	0.00406147307895804\\
148	0.00406147306416089\\
149	0.00406147304910937\\
150	0.00406147303379907\\
151	0.00406147301822546\\
152	0.00406147300238396\\
153	0.00406147298626988\\
154	0.00406147296987847\\
155	0.0040614729532049\\
156	0.00406147293624424\\
157	0.0040614729189915\\
158	0.00406147290144159\\
159	0.00406147288358932\\
160	0.00406147286542943\\
161	0.00406147284695656\\
162	0.00406147282816527\\
163	0.00406147280905001\\
164	0.00406147278960514\\
165	0.00406147276982494\\
166	0.00406147274970356\\
167	0.00406147272923508\\
168	0.00406147270841346\\
169	0.00406147268723257\\
170	0.00406147266568616\\
171	0.00406147264376789\\
172	0.0040614726214713\\
173	0.00406147259878982\\
174	0.00406147257571676\\
175	0.00406147255224534\\
176	0.00406147252836863\\
177	0.00406147250407961\\
178	0.00406147247937111\\
179	0.00406147245423586\\
180	0.00406147242866646\\
181	0.00406147240265536\\
182	0.0040614723761949\\
183	0.00406147234927729\\
184	0.00406147232189459\\
185	0.00406147229403871\\
186	0.00406147226570145\\
187	0.00406147223687445\\
188	0.00406147220754919\\
189	0.00406147217771702\\
190	0.00406147214736913\\
191	0.00406147211649655\\
192	0.00406147208509017\\
193	0.0040614720531407\\
194	0.00406147202063869\\
195	0.00406147198757453\\
196	0.00406147195393844\\
197	0.00406147191972046\\
198	0.00406147188491047\\
199	0.00406147184949814\\
200	0.00406147181347299\\
201	0.00406147177682435\\
202	0.00406147173954134\\
203	0.00406147170161291\\
204	0.00406147166302779\\
205	0.00406147162377455\\
206	0.00406147158384153\\
207	0.00406147154321685\\
208	0.00406147150188845\\
209	0.00406147145984404\\
210	0.00406147141707111\\
211	0.00406147137355695\\
212	0.00406147132928859\\
213	0.00406147128425285\\
214	0.00406147123843631\\
215	0.00406147119182533\\
216	0.00406147114440599\\
217	0.00406147109616416\\
218	0.00406147104708543\\
219	0.00406147099715515\\
220	0.00406147094635841\\
221	0.00406147089468003\\
222	0.00406147084210454\\
223	0.00406147078861623\\
224	0.00406147073419909\\
225	0.00406147067883681\\
226	0.00406147062251282\\
227	0.00406147056521024\\
228	0.00406147050691188\\
229	0.00406147044760024\\
230	0.00406147038725754\\
231	0.00406147032586563\\
232	0.00406147026340609\\
233	0.00406147019986012\\
234	0.00406147013520862\\
235	0.00406147006943212\\
236	0.00406147000251083\\
237	0.00406146993442459\\
238	0.00406146986515287\\
239	0.00406146979467477\\
240	0.00406146972296904\\
241	0.00406146965001403\\
242	0.0040614695757877\\
243	0.00406146950026761\\
244	0.00406146942343094\\
245	0.00406146934525444\\
246	0.00406146926571443\\
247	0.00406146918478683\\
248	0.00406146910244713\\
249	0.00406146901867034\\
250	0.00406146893343106\\
251	0.00406146884670341\\
252	0.00406146875846106\\
253	0.0040614686686772\\
254	0.00406146857732452\\
255	0.00406146848437525\\
256	0.0040614683898011\\
257	0.00406146829357326\\
258	0.00406146819566243\\
259	0.00406146809603876\\
260	0.00406146799467186\\
261	0.00406146789153082\\
262	0.00406146778658415\\
263	0.00406146767979979\\
264	0.00406146757114512\\
265	0.00406146746058691\\
266	0.00406146734809136\\
267	0.00406146723362405\\
268	0.00406146711714993\\
269	0.00406146699863334\\
270	0.00406146687803795\\
271	0.00406146675532681\\
272	0.00406146663046229\\
273	0.00406146650340609\\
274	0.00406146637411921\\
275	0.00406146624256198\\
276	0.00406146610869399\\
277	0.00406146597247412\\
278	0.00406146583386051\\
279	0.00406146569281057\\
280	0.00406146554928092\\
281	0.00406146540322742\\
282	0.00406146525460517\\
283	0.00406146510336842\\
284	0.00406146494947065\\
285	0.00406146479286449\\
286	0.00406146463350175\\
287	0.00406146447133337\\
288	0.00406146430630942\\
289	0.00406146413837912\\
290	0.00406146396749075\\
291	0.00406146379359172\\
292	0.00406146361662849\\
293	0.0040614634365466\\
294	0.00406146325329061\\
295	0.00406146306680416\\
296	0.00406146287702986\\
297	0.00406146268390935\\
298	0.00406146248738325\\
299	0.00406146228739114\\
300	0.00406146208387159\\
301	0.00406146187676207\\
302	0.00406146166599901\\
303	0.00406146145151774\\
304	0.00406146123325248\\
305	0.00406146101113634\\
306	0.00406146078510128\\
307	0.00406146055507813\\
308	0.00406146032099653\\
309	0.00406146008278495\\
310	0.00406145984037066\\
311	0.00406145959367972\\
312	0.00406145934263694\\
313	0.0040614590871659\\
314	0.00406145882718891\\
315	0.00406145856262702\\
316	0.00406145829339994\\
317	0.00406145801942611\\
318	0.00406145774062263\\
319	0.00406145745690525\\
320	0.00406145716818835\\
321	0.00406145687438495\\
322	0.00406145657540666\\
323	0.00406145627116369\\
324	0.00406145596156482\\
325	0.00406145564651736\\
326	0.00406145532592719\\
327	0.00406145499969867\\
328	0.00406145466773469\\
329	0.00406145432993661\\
330	0.00406145398620423\\
331	0.00406145363643581\\
332	0.00406145328052804\\
333	0.00406145291837597\\
334	0.00406145254987306\\
335	0.00406145217491112\\
336	0.00406145179338026\\
337	0.00406145140516892\\
338	0.00406145101016383\\
339	0.00406145060824992\\
340	0.00406145019931037\\
341	0.00406144978322657\\
342	0.00406144935987801\\
343	0.00406144892914235\\
344	0.00406144849089529\\
345	0.00406144804501061\\
346	0.00406144759136004\\
347	0.00406144712981331\\
348	0.00406144666023802\\
349	0.00406144618249962\\
350	0.00406144569646135\\
351	0.00406144520198418\\
352	0.00406144469892675\\
353	0.00406144418714527\\
354	0.00406144366649348\\
355	0.00406144313682254\\
356	0.00406144259798095\\
357	0.00406144204981445\\
358	0.00406144149216595\\
359	0.00406144092487536\\
360	0.00406144034777953\\
361	0.00406143976071209\\
362	0.00406143916350335\\
363	0.00406143855598009\\
364	0.00406143793796551\\
365	0.00406143730927897\\
366	0.00406143666973589\\
367	0.00406143601914752\\
368	0.00406143535732077\\
369	0.00406143468405802\\
370	0.00406143399915687\\
371	0.00406143330240993\\
372	0.00406143259360459\\
373	0.00406143187252274\\
374	0.00406143113894053\\
375	0.00406143039262806\\
376	0.00406142963334911\\
377	0.0040614288608608\\
378	0.00406142807491329\\
379	0.00406142727524937\\
380	0.00406142646160414\\
381	0.00406142563370459\\
382	0.00406142479126918\\
383	0.00406142393400737\\
384	0.00406142306161913\\
385	0.00406142217379441\\
386	0.00406142127021255\\
387	0.00406142035054169\\
388	0.00406141941443797\\
389	0.00406141846154484\\
390	0.0040614174914921\\
391	0.00406141650389494\\
392	0.00406141549835268\\
393	0.00406141447444738\\
394	0.00406141343174207\\
395	0.0040614123697786\\
396	0.00406141128807509\\
397	0.00406141018612313\\
398	0.00406140906338551\\
399	0.00406140791929555\\
400	0.00406140675325923\\
401	0.00406140556465838\\
402	0.00406140435284709\\
403	0.00406140311713441\\
404	0.00406140185677566\\
405	0.00406140057098452\\
406	0.00406139925893173\\
407	0.00406139791974453\\
408	0.00406139655250733\\
409	0.00406139515626258\\
410	0.00406139373000972\\
411	0.00406139227269652\\
412	0.00406139078319603\\
413	0.00406138926026895\\
414	0.00406138770253677\\
415	0.00406138610852019\\
416	0.00406138447673179\\
417	0.00406138280557569\\
418	0.00406138109331847\\
419	0.00406137933807158\\
420	0.00406137753777094\\
421	0.00406137569015324\\
422	0.00406137379272801\\
423	0.00406137184274438\\
424	0.00406136983715\\
425	0.00406136777253767\\
426	0.0040613656450696\\
427	0.00406136345035738\\
428	0.00406136118324853\\
429	0.00406135883741312\\
430	0.00406135640450675\\
431	0.00406135387247213\\
432	0.0040613512222096\\
433	0.00406134842152963\\
434	0.00406134541564776\\
435	0.00406134211645094\\
436	0.0040613384023065\\
437	0.00406133415886565\\
438	0.00406132939785058\\
439	0.00406132437530529\\
440	0.00406131924489778\\
441	0.00406131400369967\\
442	0.00406130864870767\\
443	0.00406130317684924\\
444	0.00406129758499008\\
445	0.00406129186994368\\
446	0.00406128602848235\\
447	0.0040612800573483\\
448	0.00406127395325978\\
449	0.00406126771289894\\
450	0.00406126133284752\\
451	0.00406125480938392\\
452	0.00406124813792591\\
453	0.00406124131158086\\
454	0.00406123431746821\\
455	0.00406122712752537\\
456	0.00406121967581232\\
457	0.00406121180344801\\
458	0.0040612031287363\\
459	0.00406119275532929\\
460	0.00406117867059865\\
461	0.00406115669857275\\
462	0.00406111933689257\\
463	0.0040610566265805\\
464	0.00406096569839961\\
465	0.00406087317548331\\
466	0.00406077901137583\\
467	0.00406068315815308\\
468	0.0040605855668307\\
469	0.00406048618571796\\
470	0.00406038496016992\\
471	0.00406028183402784\\
472	0.00406017674864634\\
473	0.00406006964010068\\
474	0.00405996043544122\\
475	0.00405984905393745\\
476	0.00405973542005483\\
477	0.00405961945579985\\
478	0.00405950107808962\\
479	0.00405938019828523\\
480	0.00405925672166008\\
481	0.00405913054678598\\
482	0.00405900156482686\\
483	0.0040588696587283\\
484	0.00405873470228564\\
485	0.0040585965590667\\
486	0.00405845508117609\\
487	0.00405831010784321\\
488	0.00405816146376941\\
489	0.00405800895716853\\
490	0.00405785237739415\\
491	0.00405769149195838\\
492	0.00405752604254285\\
493	0.00405735573912402\\
494	0.00405718025020847\\
495	0.0040569991845618\\
496	0.00405681205392423\\
497	0.00405661819352445\\
498	0.00405641659189319\\
499	0.00405620553753419\\
500	0.00405598193594154\\
501	0.00405574016564575\\
502	0.00405547073960516\\
503	0.00405516067814581\\
504	0.00405480160717274\\
505	0.00405441381211283\\
506	0.00405402099120583\\
507	0.0040536231225162\\
508	0.00405321987985991\\
509	0.00405281039522229\\
510	0.00405239398982295\\
511	0.00405197033797043\\
512	0.00405153907375548\\
513	0.00405109977356717\\
514	0.00405065191342985\\
515	0.00405019475676502\\
516	0.00404972704919735\\
517	0.00404924617980659\\
518	0.00404874588097109\\
519	0.00404820999895577\\
520	0.00404759604850658\\
521	0.00404679381227801\\
522	0.00404553046041244\\
523	0.00404319768801735\\
524	0.00403876878224668\\
525	0.0040321999423493\\
526	0.00402547244576288\\
527	0.00401856983045933\\
528	0.00401147129796997\\
529	0.0040041569952074\\
530	0.00399662529008306\\
531	0.00398888828751701\\
532	0.00398092784527301\\
533	0.00397268828426009\\
534	0.0039640176156458\\
535	0.00395455217310666\\
536	0.00394369284046989\\
537	0.00393062769130563\\
538	0.00391601606445189\\
539	0.00390129681388448\\
540	0.00388646202356258\\
541	0.00387150286326357\\
542	0.00385640915205039\\
543	0.00384116793333786\\
544	0.0038257600908453\\
545	0.00381015356436016\\
546	0.00379428777375626\\
547	0.00377803953381254\\
548	0.00376116179861637\\
549	0.00374325944500222\\
550	0.00372418624906977\\
551	0.00370500254122745\\
552	0.00368548067614745\\
553	0.00366542135653332\\
554	0.00364300836795439\\
555	0.00360768816192235\\
556	0.00352012484187683\\
557	0.0034200905470597\\
558	0.0033114881502614\\
559	0.00318586122956869\\
560	0.00301590666812544\\
561	0.00282709200229717\\
562	0.00263094421539516\\
563	0.00242651276754218\\
564	0.00221179816707534\\
565	0.00198061034947524\\
566	0.00171394633613848\\
567	0.00143955403725653\\
568	0.00115727971134879\\
569	0.000866069879143256\\
570	0.000563740415471681\\
571	0.000244432763086153\\
572	0\\
573	0\\
574	0\\
575	0\\
576	0\\
577	0\\
578	0\\
579	0\\
580	0\\
581	0\\
582	0\\
583	0\\
584	0\\
585	0\\
586	0\\
587	0\\
588	0\\
589	0\\
590	0\\
591	0\\
592	0\\
593	0\\
594	0\\
595	0\\
596	0\\
597	0\\
598	0\\
599	0\\
600	0\\
};
\addplot [color=mycolor12,solid,forget plot]
  table[row sep=crcr]{%
1	0.00510109179991326\\
2	0.00510109179985886\\
3	0.00510109179980354\\
4	0.00510109179974729\\
5	0.0051010917996901\\
6	0.00510109179963195\\
7	0.00510109179957282\\
8	0.00510109179951269\\
9	0.00510109179945155\\
10	0.00510109179938939\\
11	0.00510109179932617\\
12	0.0051010917992619\\
13	0.00510109179919654\\
14	0.00510109179913009\\
15	0.00510109179906252\\
16	0.00510109179899381\\
17	0.00510109179892395\\
18	0.00510109179885291\\
19	0.00510109179878067\\
20	0.00510109179870723\\
21	0.00510109179863254\\
22	0.0051010917985566\\
23	0.00510109179847938\\
24	0.00510109179840087\\
25	0.00510109179832103\\
26	0.00510109179823985\\
27	0.00510109179815731\\
28	0.00510109179807337\\
29	0.00510109179798803\\
30	0.00510109179790124\\
31	0.005101091797813\\
32	0.00510109179772328\\
33	0.00510109179763204\\
34	0.00510109179753927\\
35	0.00510109179744493\\
36	0.00510109179734901\\
37	0.00510109179725148\\
38	0.00510109179715231\\
39	0.00510109179705146\\
40	0.00510109179694892\\
41	0.00510109179684465\\
42	0.00510109179673863\\
43	0.00510109179663082\\
44	0.0051010917965212\\
45	0.00510109179640974\\
46	0.00510109179629639\\
47	0.00510109179618114\\
48	0.00510109179606395\\
49	0.00510109179594478\\
50	0.00510109179582361\\
51	0.00510109179570039\\
52	0.00510109179557511\\
53	0.00510109179544771\\
54	0.00510109179531816\\
55	0.00510109179518643\\
56	0.00510109179505248\\
57	0.00510109179491628\\
58	0.00510109179477778\\
59	0.00510109179463695\\
60	0.00510109179449374\\
61	0.00510109179434812\\
62	0.00510109179420005\\
63	0.00510109179404947\\
64	0.00510109179389636\\
65	0.00510109179374067\\
66	0.00510109179358235\\
67	0.00510109179342137\\
68	0.00510109179325766\\
69	0.0051010917930912\\
70	0.00510109179292192\\
71	0.00510109179274979\\
72	0.00510109179257475\\
73	0.00510109179239676\\
74	0.00510109179221577\\
75	0.00510109179203172\\
76	0.00510109179184456\\
77	0.00510109179165424\\
78	0.0051010917914607\\
79	0.0051010917912639\\
80	0.00510109179106377\\
81	0.00510109179086026\\
82	0.00510109179065331\\
83	0.00510109179044286\\
84	0.00510109179022885\\
85	0.00510109179001123\\
86	0.00510109178978992\\
87	0.00510109178956487\\
88	0.00510109178933601\\
89	0.00510109178910328\\
90	0.00510109178886661\\
91	0.00510109178862593\\
92	0.00510109178838118\\
93	0.00510109178813228\\
94	0.00510109178787917\\
95	0.00510109178762177\\
96	0.00510109178736\\
97	0.0051010917870938\\
98	0.00510109178682309\\
99	0.00510109178654778\\
100	0.00510109178626781\\
101	0.00510109178598308\\
102	0.00510109178569353\\
103	0.00510109178539906\\
104	0.00510109178509958\\
105	0.00510109178479503\\
106	0.0051010917844853\\
107	0.0051010917841703\\
108	0.00510109178384996\\
109	0.00510109178352417\\
110	0.00510109178319283\\
111	0.00510109178285587\\
112	0.00510109178251317\\
113	0.00510109178216463\\
114	0.00510109178181017\\
115	0.00510109178144967\\
116	0.00510109178108302\\
117	0.00510109178071014\\
118	0.00510109178033089\\
119	0.00510109177994519\\
120	0.0051010917795529\\
121	0.00510109177915392\\
122	0.00510109177874814\\
123	0.00510109177833543\\
124	0.00510109177791568\\
125	0.00510109177748875\\
126	0.00510109177705454\\
127	0.0051010917766129\\
128	0.00510109177616372\\
129	0.00510109177570685\\
130	0.00510109177524217\\
131	0.00510109177476954\\
132	0.00510109177428883\\
133	0.00510109177379988\\
134	0.00510109177330256\\
135	0.00510109177279672\\
136	0.00510109177228222\\
137	0.00510109177175889\\
138	0.0051010917712266\\
139	0.00510109177068517\\
140	0.00510109177013446\\
141	0.0051010917695743\\
142	0.00510109176900452\\
143	0.00510109176842496\\
144	0.00510109176783545\\
145	0.00510109176723581\\
146	0.00510109176662587\\
147	0.00510109176600544\\
148	0.00510109176537435\\
149	0.00510109176473241\\
150	0.00510109176407942\\
151	0.0051010917634152\\
152	0.00510109176273955\\
153	0.00510109176205227\\
154	0.00510109176135316\\
155	0.00510109176064201\\
156	0.00510109175991861\\
157	0.00510109175918275\\
158	0.00510109175843421\\
159	0.00510109175767277\\
160	0.00510109175689821\\
161	0.00510109175611029\\
162	0.00510109175530879\\
163	0.00510109175449347\\
164	0.00510109175366408\\
165	0.00510109175282039\\
166	0.00510109175196214\\
167	0.00510109175108909\\
168	0.00510109175020097\\
169	0.00510109174929752\\
170	0.00510109174837848\\
171	0.00510109174744358\\
172	0.00510109174649253\\
173	0.00510109174552507\\
174	0.0051010917445409\\
175	0.00510109174353974\\
176	0.00510109174252128\\
177	0.00510109174148524\\
178	0.00510109174043131\\
179	0.00510109173935916\\
180	0.0051010917382685\\
181	0.00510109173715899\\
182	0.00510109173603032\\
183	0.00510109173488214\\
184	0.00510109173371412\\
185	0.00510109173252591\\
186	0.00510109173131717\\
187	0.00510109173008753\\
188	0.00510109172883664\\
189	0.00510109172756412\\
190	0.0051010917262696\\
191	0.0051010917249527\\
192	0.00510109172361302\\
193	0.00510109172225017\\
194	0.00510109172086375\\
195	0.00510109171945334\\
196	0.00510109171801854\\
197	0.00510109171655891\\
198	0.00510109171507402\\
199	0.00510109171356343\\
200	0.00510109171202669\\
201	0.00510109171046336\\
202	0.00510109170887296\\
203	0.00510109170725502\\
204	0.00510109170560906\\
205	0.0051010917039346\\
206	0.00510109170223113\\
207	0.00510109170049815\\
208	0.00510109169873514\\
209	0.00510109169694158\\
210	0.00510109169511695\\
211	0.00510109169326067\\
212	0.00510109169137222\\
213	0.00510109168945103\\
214	0.00510109168749652\\
215	0.00510109168550811\\
216	0.00510109168348521\\
217	0.00510109168142722\\
218	0.00510109167933351\\
219	0.00510109167720346\\
220	0.00510109167503645\\
221	0.00510109167283181\\
222	0.00510109167058888\\
223	0.00510109166830701\\
224	0.0051010916659855\\
225	0.00510109166362365\\
226	0.00510109166122077\\
227	0.00510109165877613\\
228	0.00510109165628899\\
229	0.00510109165375861\\
230	0.00510109165118423\\
231	0.00510109164856507\\
232	0.00510109164590035\\
233	0.00510109164318926\\
234	0.005101091640431\\
235	0.00510109163762472\\
236	0.00510109163476959\\
237	0.00510109163186473\\
238	0.00510109162890929\\
239	0.00510109162590235\\
240	0.00510109162284302\\
241	0.00510109161973038\\
242	0.00510109161656347\\
243	0.00510109161334134\\
244	0.00510109161006302\\
245	0.00510109160672752\\
246	0.00510109160333382\\
247	0.00510109159988089\\
248	0.00510109159636769\\
249	0.00510109159279315\\
250	0.00510109158915619\\
251	0.0051010915854557\\
252	0.00510109158169055\\
253	0.00510109157785961\\
254	0.0051010915739617\\
255	0.00510109156999565\\
256	0.00510109156596023\\
257	0.00510109156185422\\
258	0.00510109155767638\\
259	0.00510109155342541\\
260	0.00510109154910004\\
261	0.00510109154469893\\
262	0.00510109154022075\\
263	0.00510109153566412\\
264	0.00510109153102764\\
265	0.00510109152630991\\
266	0.00510109152150948\\
267	0.00510109151662488\\
268	0.0051010915116546\\
269	0.00510109150659713\\
270	0.00510109150145092\\
271	0.0051010914962144\\
272	0.00510109149088594\\
273	0.00510109148546392\\
274	0.00510109147994667\\
275	0.00510109147433249\\
276	0.00510109146861967\\
277	0.00510109146280644\\
278	0.00510109145689103\\
279	0.0051010914508716\\
280	0.00510109144474632\\
281	0.00510109143851329\\
282	0.00510109143217059\\
283	0.00510109142571628\\
284	0.00510109141914836\\
285	0.00510109141246483\\
286	0.00510109140566361\\
287	0.00510109139874261\\
288	0.00510109139169971\\
289	0.00510109138453274\\
290	0.00510109137723948\\
291	0.0051010913698177\\
292	0.00510109136226511\\
293	0.00510109135457938\\
294	0.00510109134675816\\
295	0.00510109133879903\\
296	0.00510109133069954\\
297	0.00510109132245721\\
298	0.00510109131406951\\
299	0.00510109130553385\\
300	0.00510109129684761\\
301	0.00510109128800814\\
302	0.00510109127901271\\
303	0.00510109126985857\\
304	0.00510109126054291\\
305	0.00510109125106289\\
306	0.00510109124141559\\
307	0.00510109123159808\\
308	0.00510109122160734\\
309	0.00510109121144035\\
310	0.00510109120109398\\
311	0.0051010911905651\\
312	0.00510109117985049\\
313	0.0051010911689469\\
314	0.00510109115785102\\
315	0.00510109114655949\\
316	0.00510109113506889\\
317	0.00510109112337573\\
318	0.0051010911114765\\
319	0.0051010910993676\\
320	0.00510109108704538\\
321	0.00510109107450615\\
322	0.00510109106174613\\
323	0.00510109104876151\\
324	0.00510109103554841\\
325	0.00510109102210286\\
326	0.00510109100842088\\
327	0.00510109099449839\\
328	0.00510109098033125\\
329	0.00510109096591527\\
330	0.00510109095124619\\
331	0.00510109093631967\\
332	0.00510109092113132\\
333	0.00510109090567667\\
334	0.00510109088995119\\
335	0.00510109087395028\\
336	0.00510109085766926\\
337	0.00510109084110338\\
338	0.00510109082424782\\
339	0.00510109080709769\\
340	0.00510109078964801\\
341	0.00510109077189374\\
342	0.00510109075382975\\
343	0.00510109073545083\\
344	0.00510109071675169\\
345	0.00510109069772694\\
346	0.00510109067837114\\
347	0.00510109065867872\\
348	0.00510109063864405\\
349	0.00510109061826139\\
350	0.00510109059752491\\
351	0.00510109057642869\\
352	0.00510109055496668\\
353	0.00510109053313275\\
354	0.00510109051092066\\
355	0.00510109048832405\\
356	0.00510109046533644\\
357	0.00510109044195123\\
358	0.0051010904181617\\
359	0.005101090393961\\
360	0.00510109036934214\\
361	0.00510109034429798\\
362	0.00510109031882124\\
363	0.00510109029290449\\
364	0.00510109026654014\\
365	0.0051010902397204\\
366	0.00510109021243735\\
367	0.00510109018468284\\
368	0.00510109015644857\\
369	0.00510109012772599\\
370	0.00510109009850639\\
371	0.00510109006878078\\
372	0.00510109003853998\\
373	0.00510109000777454\\
374	0.00510108997647477\\
375	0.00510108994463069\\
376	0.00510108991223205\\
377	0.0051010898792683\\
378	0.00510108984572858\\
379	0.00510108981160168\\
380	0.00510108977687607\\
381	0.00510108974153983\\
382	0.00510108970558068\\
383	0.00510108966898591\\
384	0.0051010896317424\\
385	0.00510108959383656\\
386	0.00510108955525432\\
387	0.0051010895159811\\
388	0.00510108947600176\\
389	0.00510108943530057\\
390	0.00510108939386118\\
391	0.00510108935166651\\
392	0.00510108930869876\\
393	0.00510108926493928\\
394	0.00510108922036845\\
395	0.00510108917496566\\
396	0.00510108912870914\\
397	0.00510108908157593\\
398	0.00510108903354188\\
399	0.00510108898458167\\
400	0.00510108893466886\\
401	0.00510108888377556\\
402	0.00510108883187196\\
403	0.00510108877892617\\
404	0.00510108872490455\\
405	0.00510108866977168\\
406	0.00510108861349032\\
407	0.00510108855602139\\
408	0.00510108849732398\\
409	0.00510108843735507\\
410	0.00510108837606896\\
411	0.00510108831341628\\
412	0.00510108824934293\\
413	0.00510108818378981\\
414	0.00510108811669441\\
415	0.00510108804799222\\
416	0.00510108797761368\\
417	0.00510108790548305\\
418	0.00510108783151769\\
419	0.00510108775562703\\
420	0.00510108767771156\\
421	0.00510108759766138\\
422	0.00510108751535461\\
423	0.00510108743065502\\
424	0.0051010873434088\\
425	0.00510108725343919\\
426	0.00510108716053712\\
427	0.0051010870644438\\
428	0.00510108696481774\\
429	0.00510108686117289\\
430	0.00510108675276765\\
431	0.00510108663842159\\
432	0.00510108651625388\\
433	0.00510108638341271\\
434	0.00510108623605208\\
435	0.00510108607011444\\
436	0.00510108588358932\\
437	0.00510108567976443\\
438	0.00510108546715944\\
439	0.00510108524996865\\
440	0.00510108502806737\\
441	0.0051010848013279\\
442	0.00510108456961976\\
443	0.00510108433281003\\
444	0.00510108409076368\\
445	0.0051010838433438\\
446	0.00510108359041134\\
447	0.00510108333182357\\
448	0.00510108306742894\\
449	0.00510108279705347\\
450	0.00510108252046636\\
451	0.00510108223729676\\
452	0.00510108194683727\\
453	0.00510108164759044\\
454	0.00510108133624716\\
455	0.00510108100545546\\
456	0.00510108063915062\\
457	0.00510108020337091\\
458	0.00510107962988618\\
459	0.00510107879175149\\
460	0.00510107747983606\\
461	0.00510107541719667\\
462	0.00510107239218583\\
463	0.0051010685759856\\
464	0.00510106469226201\\
465	0.00510106073903992\\
466	0.00510105671427971\\
467	0.00510105261587299\\
468	0.00510104844159795\\
469	0.00510104418911902\\
470	0.00510103985600564\\
471	0.00510103543967317\\
472	0.00510103093728759\\
473	0.00510102634570732\\
474	0.00510102166160921\\
475	0.00510101688175904\\
476	0.00510101200277714\\
477	0.00510100702105975\\
478	0.00510100193275785\\
479	0.00510099673375294\\
480	0.00510099141962911\\
481	0.00510098598564094\\
482	0.0051009804266767\\
483	0.00510097473721582\\
484	0.00510096891127982\\
485	0.0051009629423756\\
486	0.00510095682342939\\
487	0.00510095054670774\\
488	0.0051009441037201\\
489	0.00510093748509278\\
490	0.00510093068039336\\
491	0.00510092367786143\\
492	0.00510091646395133\\
493	0.00510090902248744\\
494	0.00510090133302177\\
495	0.00510089336758987\\
496	0.00510088508440327\\
497	0.00510087641614243\\
498	0.00510086725001991\\
499	0.00510085739884259\\
500	0.00510084657245365\\
501	0.00510083438676113\\
502	0.00510082049579685\\
503	0.00510080494095675\\
504	0.0051007885116239\\
505	0.0051007718624104\\
506	0.00510075498557465\\
507	0.00510073786101444\\
508	0.00510072045661552\\
509	0.00510070274822352\\
510	0.00510068472006531\\
511	0.00510066635294463\\
512	0.00510064762101038\\
513	0.00510062848380644\\
514	0.00510060886667179\\
515	0.0051005886129507\\
516	0.00510056736990364\\
517	0.00510054432491002\\
518	0.00510051762369321\\
519	0.0051004831778553\\
520	0.00510043250933198\\
521	0.00510034977287883\\
522	0.00510021062121271\\
523	0.00509999263205191\\
524	0.00509971271547763\\
525	0.00509942494706406\\
526	0.00509912851553759\\
527	0.00509882253688619\\
528	0.00509850628966137\\
529	0.00509817947339278\\
530	0.00509784152438081\\
531	0.00509748999325015\\
532	0.00509711952279453\\
533	0.00509671942295103\\
534	0.00509627145328356\\
535	0.00509575324774495\\
536	0.00509515058158533\\
537	0.00509449543104951\\
538	0.00509383028532832\\
539	0.00509315442498853\\
540	0.00509246701986117\\
541	0.00509176705844467\\
542	0.00509105318802637\\
543	0.00509032339795339\\
544	0.00508957445850129\\
545	0.00508880094436249\\
546	0.00508799382521893\\
547	0.00508713939509688\\
548	0.0050862221275849\\
549	0.00508523810584875\\
550	0.00508420975301684\\
551	0.00508308849968024\\
552	0.00508175972602962\\
553	0.00507990484665965\\
554	0.00507669461183126\\
555	0.00507044038034362\\
556	0.00505922864793289\\
557	0.00504656909230222\\
558	0.00503244725364375\\
559	0.00501578264226783\\
560	0.00499545046704825\\
561	0.00497398626098588\\
562	0.00495230936879711\\
563	0.00493026676404129\\
564	0.00490753314712123\\
565	0.00488359464059053\\
566	0.0048582284635519\\
567	0.00483327164309529\\
568	0.00480872353417642\\
569	0.00478452488381562\\
570	0.00476060339303416\\
571	0.00473706064438208\\
572	0.0047147640350247\\
573	0.00469513866116233\\
574	0.00467754660897937\\
575	0.00465821856935375\\
576	0.00463326745163829\\
577	0.00459144151574509\\
578	0.00450001461999156\\
579	0.00425903716898513\\
580	0.00396782385977364\\
581	0.00364026773511628\\
582	0.00330161831566113\\
583	0.00294712380411371\\
584	0.00256295996406943\\
585	0.0021687264087653\\
586	0.00176496329681716\\
587	0.00134883630364181\\
588	0.000913380900315357\\
589	0.000438890169453977\\
590	0\\
591	0\\
592	0\\
593	0\\
594	0\\
595	0\\
596	0\\
597	0\\
598	0\\
599	0\\
600	0\\
};
\addplot [color=mycolor13,solid,forget plot]
  table[row sep=crcr]{%
1	0\\
2	0\\
3	0\\
4	0\\
5	0\\
6	0\\
7	0\\
8	0\\
9	0\\
10	0\\
11	0\\
12	0\\
13	0\\
14	0\\
15	0\\
16	0\\
17	0\\
18	0\\
19	0\\
20	0\\
21	0\\
22	0\\
23	0\\
24	0\\
25	0\\
26	0\\
27	0\\
28	0\\
29	0\\
30	0\\
31	0\\
32	0\\
33	0\\
34	0\\
35	0\\
36	0\\
37	0\\
38	0\\
39	0\\
40	0\\
41	0\\
42	0\\
43	0\\
44	0\\
45	0\\
46	0\\
47	0\\
48	0\\
49	0\\
50	0\\
51	0\\
52	0\\
53	0\\
54	0\\
55	0\\
56	0\\
57	0\\
58	0\\
59	0\\
60	0\\
61	0\\
62	0\\
63	0\\
64	0\\
65	0\\
66	0\\
67	0\\
68	0\\
69	0\\
70	0\\
71	0\\
72	0\\
73	0\\
74	0\\
75	0\\
76	0\\
77	0\\
78	0\\
79	0\\
80	0\\
81	0\\
82	0\\
83	0\\
84	0\\
85	0\\
86	0\\
87	0\\
88	0\\
89	0\\
90	0\\
91	0\\
92	0\\
93	0\\
94	0\\
95	0\\
96	0\\
97	0\\
98	0\\
99	0\\
100	0\\
101	0\\
102	0\\
103	0\\
104	0\\
105	0\\
106	0\\
107	0\\
108	0\\
109	0\\
110	0\\
111	0\\
112	0\\
113	0\\
114	0\\
115	0\\
116	0\\
117	0\\
118	0\\
119	0\\
120	0\\
121	0\\
122	0\\
123	0\\
124	0\\
125	0\\
126	0\\
127	0\\
128	0\\
129	0\\
130	0\\
131	0\\
132	0\\
133	0\\
134	0\\
135	0\\
136	0\\
137	0\\
138	0\\
139	0\\
140	0\\
141	0\\
142	0\\
143	0\\
144	0\\
145	0\\
146	0\\
147	0\\
148	0\\
149	0\\
150	0\\
151	0\\
152	0\\
153	0\\
154	0\\
155	0\\
156	0\\
157	0\\
158	0\\
159	0\\
160	0\\
161	0\\
162	0\\
163	0\\
164	0\\
165	0\\
166	0\\
167	0\\
168	0\\
169	0\\
170	0\\
171	0\\
172	0\\
173	0\\
174	0\\
175	0\\
176	0\\
177	0\\
178	0\\
179	0\\
180	0\\
181	0\\
182	0\\
183	0\\
184	0\\
185	0\\
186	0\\
187	0\\
188	0\\
189	0\\
190	0\\
191	0\\
192	0\\
193	0\\
194	0\\
195	0\\
196	0\\
197	0\\
198	0\\
199	0\\
200	0\\
201	0\\
202	0\\
203	0\\
204	0\\
205	0\\
206	0\\
207	0\\
208	0\\
209	0\\
210	0\\
211	0\\
212	0\\
213	0\\
214	0\\
215	0\\
216	0\\
217	0\\
218	0\\
219	0\\
220	0\\
221	0\\
222	0\\
223	0\\
224	0\\
225	0\\
226	0\\
227	0\\
228	0\\
229	0\\
230	0\\
231	0\\
232	0\\
233	0\\
234	0\\
235	0\\
236	0\\
237	0\\
238	0\\
239	0\\
240	0\\
241	0\\
242	0\\
243	0\\
244	0\\
245	0\\
246	0\\
247	0\\
248	0\\
249	0\\
250	0\\
251	0\\
252	0\\
253	0\\
254	0\\
255	0\\
256	0\\
257	0\\
258	0\\
259	0\\
260	0\\
261	0\\
262	0\\
263	0\\
264	0\\
265	0\\
266	0\\
267	0\\
268	0\\
269	0\\
270	0\\
271	0\\
272	0\\
273	0\\
274	0\\
275	0\\
276	0\\
277	0\\
278	0\\
279	0\\
280	0\\
281	0\\
282	0\\
283	0\\
284	0\\
285	0\\
286	0\\
287	0\\
288	0\\
289	0\\
290	0\\
291	0\\
292	0\\
293	0\\
294	0\\
295	0\\
296	0\\
297	0\\
298	0\\
299	0\\
300	0\\
301	0\\
302	0\\
303	0\\
304	0\\
305	0\\
306	0\\
307	0\\
308	0\\
309	0\\
310	0\\
311	0\\
312	0\\
313	0\\
314	0\\
315	0\\
316	0\\
317	0\\
318	0\\
319	0\\
320	0\\
321	0\\
322	0\\
323	0\\
324	0\\
325	0\\
326	0\\
327	0\\
328	0\\
329	0\\
330	0\\
331	0\\
332	0\\
333	0\\
334	0\\
335	0\\
336	0\\
337	0\\
338	0\\
339	0\\
340	0\\
341	0\\
342	0\\
343	0\\
344	0\\
345	0\\
346	0\\
347	0\\
348	0\\
349	0\\
350	0\\
351	0\\
352	0\\
353	0\\
354	0\\
355	0\\
356	0\\
357	0\\
358	0\\
359	0\\
360	0\\
361	0\\
362	0\\
363	0\\
364	0\\
365	0\\
366	0\\
367	0\\
368	0\\
369	0\\
370	0\\
371	0\\
372	0\\
373	0\\
374	0\\
375	0\\
376	0\\
377	0\\
378	0\\
379	0\\
380	0\\
381	0\\
382	0\\
383	0\\
384	0\\
385	0\\
386	0\\
387	0\\
388	0\\
389	0\\
390	0\\
391	0\\
392	0\\
393	0\\
394	0\\
395	0\\
396	0\\
397	0\\
398	0\\
399	0\\
400	0\\
401	0\\
402	0\\
403	0\\
404	0\\
405	0\\
406	0\\
407	0\\
408	0\\
409	0\\
410	0\\
411	0\\
412	0\\
413	0\\
414	0\\
415	0\\
416	0\\
417	0\\
418	0\\
419	0\\
420	0\\
421	0\\
422	0\\
423	0\\
424	0\\
425	0\\
426	0\\
427	0\\
428	0\\
429	0\\
430	0\\
431	0\\
432	0\\
433	0\\
434	0\\
435	0\\
436	0\\
437	0\\
438	0\\
439	0\\
440	0\\
441	0\\
442	0\\
443	0\\
444	0\\
445	0\\
446	0\\
447	0\\
448	0\\
449	0\\
450	0\\
451	0\\
452	0\\
453	0\\
454	0\\
455	0\\
456	0\\
457	0\\
458	0\\
459	0\\
460	0\\
461	0\\
462	0\\
463	0\\
464	0\\
465	0\\
466	0\\
467	0\\
468	0\\
469	0\\
470	0\\
471	0\\
472	0\\
473	0\\
474	0\\
475	0\\
476	0\\
477	0\\
478	0\\
479	0\\
480	0\\
481	0\\
482	0\\
483	0\\
484	0\\
485	0\\
486	0\\
487	0\\
488	0\\
489	0\\
490	0\\
491	0\\
492	0\\
493	0\\
494	0\\
495	0\\
496	0\\
497	0\\
498	0\\
499	0\\
500	0\\
501	0\\
502	0\\
503	0\\
504	0\\
505	0\\
506	0\\
507	0\\
508	0\\
509	0\\
510	0\\
511	0\\
512	0\\
513	0\\
514	0\\
515	0\\
516	0\\
517	0\\
518	0\\
519	0\\
520	0\\
521	0\\
522	0\\
523	0\\
524	0\\
525	0\\
526	0\\
527	0\\
528	0\\
529	0\\
530	0\\
531	0\\
532	0\\
533	0\\
534	0\\
535	0\\
536	0\\
537	0\\
538	0\\
539	0\\
540	0\\
541	0\\
542	0\\
543	0\\
544	0\\
545	0\\
546	0\\
547	0\\
548	0\\
549	0\\
550	0\\
551	0\\
552	0\\
553	0\\
554	0\\
555	0\\
556	0\\
557	0\\
558	0\\
559	0\\
560	0\\
561	0\\
562	0\\
563	0\\
564	0\\
565	0\\
566	0\\
567	0\\
568	0\\
569	0\\
570	0\\
571	0\\
572	0\\
573	0\\
574	0\\
575	0\\
576	0\\
577	0\\
578	0\\
579	0\\
580	0\\
581	0\\
582	0\\
583	0\\
584	0\\
585	0\\
586	0\\
587	0\\
588	0\\
589	0\\
590	0\\
591	0\\
592	0\\
593	0\\
594	9.98598075601821e-05\\
595	0.000594592516223747\\
596	0.00167287594548724\\
597	0.00320324344438694\\
598	0.00587935965236376\\
599	0\\
600	0\\
};
\addplot [color=mycolor14,solid,forget plot]
  table[row sep=crcr]{%
1	0\\
2	0\\
3	0\\
4	0\\
5	0\\
6	0\\
7	0\\
8	0\\
9	0\\
10	0\\
11	0\\
12	0\\
13	0\\
14	0\\
15	0\\
16	0\\
17	0\\
18	0\\
19	0\\
20	0\\
21	0\\
22	0\\
23	0\\
24	0\\
25	0\\
26	0\\
27	0\\
28	0\\
29	0\\
30	0\\
31	0\\
32	0\\
33	0\\
34	0\\
35	0\\
36	0\\
37	0\\
38	0\\
39	0\\
40	0\\
41	0\\
42	0\\
43	0\\
44	0\\
45	0\\
46	0\\
47	0\\
48	0\\
49	0\\
50	0\\
51	0\\
52	0\\
53	0\\
54	0\\
55	0\\
56	0\\
57	0\\
58	0\\
59	0\\
60	0\\
61	0\\
62	0\\
63	0\\
64	0\\
65	0\\
66	0\\
67	0\\
68	0\\
69	0\\
70	0\\
71	0\\
72	0\\
73	0\\
74	0\\
75	0\\
76	0\\
77	0\\
78	0\\
79	0\\
80	0\\
81	0\\
82	0\\
83	0\\
84	0\\
85	0\\
86	0\\
87	0\\
88	0\\
89	0\\
90	0\\
91	0\\
92	0\\
93	0\\
94	0\\
95	0\\
96	0\\
97	0\\
98	0\\
99	0\\
100	0\\
101	0\\
102	0\\
103	0\\
104	0\\
105	0\\
106	0\\
107	0\\
108	0\\
109	0\\
110	0\\
111	0\\
112	0\\
113	0\\
114	0\\
115	0\\
116	0\\
117	0\\
118	0\\
119	0\\
120	0\\
121	0\\
122	0\\
123	0\\
124	0\\
125	0\\
126	0\\
127	0\\
128	0\\
129	0\\
130	0\\
131	0\\
132	0\\
133	0\\
134	0\\
135	0\\
136	0\\
137	0\\
138	0\\
139	0\\
140	0\\
141	0\\
142	0\\
143	0\\
144	0\\
145	0\\
146	0\\
147	0\\
148	0\\
149	0\\
150	0\\
151	0\\
152	0\\
153	0\\
154	0\\
155	0\\
156	0\\
157	0\\
158	0\\
159	0\\
160	0\\
161	0\\
162	0\\
163	0\\
164	0\\
165	0\\
166	0\\
167	0\\
168	0\\
169	0\\
170	0\\
171	0\\
172	0\\
173	0\\
174	0\\
175	0\\
176	0\\
177	0\\
178	0\\
179	0\\
180	0\\
181	0\\
182	0\\
183	0\\
184	0\\
185	0\\
186	0\\
187	0\\
188	0\\
189	0\\
190	0\\
191	0\\
192	0\\
193	0\\
194	0\\
195	0\\
196	0\\
197	0\\
198	0\\
199	0\\
200	0\\
201	0\\
202	0\\
203	0\\
204	0\\
205	0\\
206	0\\
207	0\\
208	0\\
209	0\\
210	0\\
211	0\\
212	0\\
213	0\\
214	0\\
215	0\\
216	0\\
217	0\\
218	0\\
219	0\\
220	0\\
221	0\\
222	0\\
223	0\\
224	0\\
225	0\\
226	0\\
227	0\\
228	0\\
229	0\\
230	0\\
231	0\\
232	0\\
233	0\\
234	0\\
235	0\\
236	0\\
237	0\\
238	0\\
239	0\\
240	0\\
241	0\\
242	0\\
243	0\\
244	0\\
245	0\\
246	0\\
247	0\\
248	0\\
249	0\\
250	0\\
251	0\\
252	0\\
253	0\\
254	0\\
255	0\\
256	0\\
257	0\\
258	0\\
259	0\\
260	0\\
261	0\\
262	0\\
263	0\\
264	0\\
265	0\\
266	0\\
267	0\\
268	0\\
269	0\\
270	0\\
271	0\\
272	0\\
273	0\\
274	0\\
275	0\\
276	0\\
277	0\\
278	0\\
279	0\\
280	0\\
281	0\\
282	0\\
283	0\\
284	0\\
285	0\\
286	0\\
287	0\\
288	0\\
289	0\\
290	0\\
291	0\\
292	0\\
293	0\\
294	0\\
295	0\\
296	0\\
297	0\\
298	0\\
299	0\\
300	0\\
301	0\\
302	0\\
303	0\\
304	0\\
305	0\\
306	0\\
307	0\\
308	0\\
309	0\\
310	0\\
311	0\\
312	0\\
313	0\\
314	0\\
315	0\\
316	0\\
317	0\\
318	0\\
319	0\\
320	0\\
321	0\\
322	0\\
323	0\\
324	0\\
325	0\\
326	0\\
327	0\\
328	0\\
329	0\\
330	0\\
331	0\\
332	0\\
333	0\\
334	0\\
335	0\\
336	0\\
337	0\\
338	0\\
339	0\\
340	0\\
341	0\\
342	0\\
343	0\\
344	0\\
345	0\\
346	0\\
347	0\\
348	0\\
349	0\\
350	0\\
351	0\\
352	0\\
353	0\\
354	0\\
355	0\\
356	0\\
357	0\\
358	0\\
359	0\\
360	0\\
361	0\\
362	0\\
363	0\\
364	0\\
365	0\\
366	0\\
367	0\\
368	0\\
369	0\\
370	0\\
371	0\\
372	0\\
373	0\\
374	0\\
375	0\\
376	0\\
377	0\\
378	0\\
379	0\\
380	0\\
381	0\\
382	0\\
383	0\\
384	0\\
385	0\\
386	0\\
387	0\\
388	0\\
389	0\\
390	0\\
391	0\\
392	0\\
393	0\\
394	0\\
395	0\\
396	0\\
397	0\\
398	0\\
399	0\\
400	0\\
401	0\\
402	0\\
403	0\\
404	0\\
405	0\\
406	0\\
407	0\\
408	0\\
409	0\\
410	0\\
411	0\\
412	0\\
413	0\\
414	0\\
415	0\\
416	0\\
417	0\\
418	0\\
419	0\\
420	0\\
421	0\\
422	0\\
423	0\\
424	0\\
425	0\\
426	0\\
427	0\\
428	0\\
429	0\\
430	0\\
431	0\\
432	0\\
433	0\\
434	0\\
435	0\\
436	0\\
437	0\\
438	0\\
439	0\\
440	0\\
441	0\\
442	0\\
443	0\\
444	0\\
445	0\\
446	0\\
447	0\\
448	0\\
449	0\\
450	0\\
451	0\\
452	0\\
453	0\\
454	0\\
455	0\\
456	0\\
457	0\\
458	0\\
459	0\\
460	0\\
461	0\\
462	0\\
463	0\\
464	0\\
465	0\\
466	0\\
467	0\\
468	0\\
469	0\\
470	0\\
471	0\\
472	0\\
473	0\\
474	0\\
475	0\\
476	0\\
477	0\\
478	0\\
479	0\\
480	0\\
481	0\\
482	0\\
483	0\\
484	0\\
485	0\\
486	0\\
487	0\\
488	0\\
489	0\\
490	0\\
491	0\\
492	0\\
493	0\\
494	0\\
495	0\\
496	0\\
497	0\\
498	0\\
499	0\\
500	0\\
501	0\\
502	0\\
503	0\\
504	0\\
505	0\\
506	0\\
507	0\\
508	0\\
509	0\\
510	0\\
511	0\\
512	0\\
513	0\\
514	0\\
515	0\\
516	0\\
517	0\\
518	0\\
519	0\\
520	0\\
521	0\\
522	0\\
523	0\\
524	0\\
525	0\\
526	0\\
527	0\\
528	0\\
529	0\\
530	0\\
531	0\\
532	0\\
533	0\\
534	0\\
535	0\\
536	0\\
537	0\\
538	0\\
539	0\\
540	0\\
541	0\\
542	0\\
543	0\\
544	0\\
545	0\\
546	0\\
547	0\\
548	0\\
549	0\\
550	0\\
551	0\\
552	0\\
553	0\\
554	0\\
555	0\\
556	0\\
557	0\\
558	0\\
559	0\\
560	0\\
561	0\\
562	0\\
563	0\\
564	0\\
565	0\\
566	0\\
567	0\\
568	0\\
569	0\\
570	0\\
571	0\\
572	0\\
573	0\\
574	0\\
575	0\\
576	0\\
577	0\\
578	0\\
579	0\\
580	0\\
581	0\\
582	0\\
583	0\\
584	0\\
585	0\\
586	0.000167079632204449\\
587	0.000456071715075539\\
588	0.000759434260115618\\
589	0.000951398265291226\\
590	0.00114744334461536\\
591	0.0013504417486169\\
592	0.00161095250100881\\
593	0.00219885314480476\\
594	0.00282841440239993\\
595	0.00353909829286836\\
596	0.00399657994176672\\
597	0.0047545345923529\\
598	0.00632942537858856\\
599	0\\
600	0\\
};
\addplot [color=mycolor15,solid,forget plot]
  table[row sep=crcr]{%
1	0\\
2	0\\
3	0\\
4	0\\
5	0\\
6	0\\
7	0\\
8	0\\
9	0\\
10	0\\
11	0\\
12	0\\
13	0\\
14	0\\
15	0\\
16	0\\
17	0\\
18	0\\
19	0\\
20	0\\
21	0\\
22	0\\
23	0\\
24	0\\
25	0\\
26	0\\
27	0\\
28	0\\
29	0\\
30	0\\
31	0\\
32	0\\
33	0\\
34	0\\
35	0\\
36	0\\
37	0\\
38	0\\
39	0\\
40	0\\
41	0\\
42	0\\
43	0\\
44	0\\
45	0\\
46	0\\
47	0\\
48	0\\
49	0\\
50	0\\
51	0\\
52	0\\
53	0\\
54	0\\
55	0\\
56	0\\
57	0\\
58	0\\
59	0\\
60	0\\
61	0\\
62	0\\
63	0\\
64	0\\
65	0\\
66	0\\
67	0\\
68	0\\
69	0\\
70	0\\
71	0\\
72	0\\
73	0\\
74	0\\
75	0\\
76	0\\
77	0\\
78	0\\
79	0\\
80	0\\
81	0\\
82	0\\
83	0\\
84	0\\
85	0\\
86	0\\
87	0\\
88	0\\
89	0\\
90	0\\
91	0\\
92	0\\
93	0\\
94	0\\
95	0\\
96	0\\
97	0\\
98	0\\
99	0\\
100	0\\
101	0\\
102	0\\
103	0\\
104	0\\
105	0\\
106	0\\
107	0\\
108	0\\
109	0\\
110	0\\
111	0\\
112	0\\
113	0\\
114	0\\
115	0\\
116	0\\
117	0\\
118	0\\
119	0\\
120	0\\
121	0\\
122	0\\
123	0\\
124	0\\
125	0\\
126	0\\
127	0\\
128	0\\
129	0\\
130	0\\
131	0\\
132	0\\
133	0\\
134	0\\
135	0\\
136	0\\
137	0\\
138	0\\
139	0\\
140	0\\
141	0\\
142	0\\
143	0\\
144	0\\
145	0\\
146	0\\
147	0\\
148	0\\
149	0\\
150	0\\
151	0\\
152	0\\
153	0\\
154	0\\
155	0\\
156	0\\
157	0\\
158	0\\
159	0\\
160	0\\
161	0\\
162	0\\
163	0\\
164	0\\
165	0\\
166	0\\
167	0\\
168	0\\
169	0\\
170	0\\
171	0\\
172	0\\
173	0\\
174	0\\
175	0\\
176	0\\
177	0\\
178	0\\
179	0\\
180	0\\
181	0\\
182	0\\
183	0\\
184	0\\
185	0\\
186	0\\
187	0\\
188	0\\
189	0\\
190	0\\
191	0\\
192	0\\
193	0\\
194	0\\
195	0\\
196	0\\
197	0\\
198	0\\
199	0\\
200	0\\
201	0\\
202	0\\
203	0\\
204	0\\
205	0\\
206	0\\
207	0\\
208	0\\
209	0\\
210	0\\
211	0\\
212	0\\
213	0\\
214	0\\
215	0\\
216	0\\
217	0\\
218	0\\
219	0\\
220	0\\
221	0\\
222	0\\
223	0\\
224	0\\
225	0\\
226	0\\
227	0\\
228	0\\
229	0\\
230	0\\
231	0\\
232	0\\
233	0\\
234	0\\
235	0\\
236	0\\
237	0\\
238	0\\
239	0\\
240	0\\
241	0\\
242	0\\
243	0\\
244	0\\
245	0\\
246	0\\
247	0\\
248	0\\
249	0\\
250	0\\
251	0\\
252	0\\
253	0\\
254	0\\
255	0\\
256	0\\
257	0\\
258	0\\
259	0\\
260	0\\
261	0\\
262	0\\
263	0\\
264	0\\
265	0\\
266	0\\
267	0\\
268	0\\
269	0\\
270	0\\
271	0\\
272	0\\
273	0\\
274	0\\
275	0\\
276	0\\
277	0\\
278	0\\
279	0\\
280	0\\
281	0\\
282	0\\
283	0\\
284	0\\
285	0\\
286	0\\
287	0\\
288	0\\
289	0\\
290	0\\
291	0\\
292	0\\
293	0\\
294	0\\
295	0\\
296	0\\
297	0\\
298	0\\
299	0\\
300	0\\
301	0\\
302	0\\
303	0\\
304	0\\
305	0\\
306	0\\
307	0\\
308	0\\
309	0\\
310	0\\
311	0\\
312	0\\
313	0\\
314	0\\
315	0\\
316	0\\
317	0\\
318	0\\
319	0\\
320	0\\
321	0\\
322	0\\
323	0\\
324	0\\
325	0\\
326	0\\
327	0\\
328	0\\
329	0\\
330	0\\
331	0\\
332	0\\
333	0\\
334	0\\
335	0\\
336	0\\
337	0\\
338	0\\
339	0\\
340	0\\
341	0\\
342	0\\
343	0\\
344	0\\
345	0\\
346	0\\
347	0\\
348	0\\
349	0\\
350	0\\
351	0\\
352	0\\
353	0\\
354	0\\
355	0\\
356	0\\
357	0\\
358	0\\
359	0\\
360	0\\
361	0\\
362	0\\
363	0\\
364	0\\
365	0\\
366	0\\
367	0\\
368	0\\
369	0\\
370	0\\
371	0\\
372	0\\
373	0\\
374	0\\
375	0\\
376	0\\
377	0\\
378	0\\
379	0\\
380	0\\
381	0\\
382	0\\
383	0\\
384	0\\
385	0\\
386	0\\
387	0\\
388	0\\
389	0\\
390	0\\
391	0\\
392	0\\
393	0\\
394	0\\
395	0\\
396	0\\
397	0\\
398	0\\
399	0\\
400	0\\
401	0\\
402	0\\
403	0\\
404	0\\
405	0\\
406	0\\
407	0\\
408	0\\
409	0\\
410	0\\
411	0\\
412	0\\
413	0\\
414	0\\
415	0\\
416	0\\
417	0\\
418	0\\
419	0\\
420	0\\
421	0\\
422	0\\
423	0\\
424	0\\
425	0\\
426	0\\
427	0\\
428	0\\
429	0\\
430	0\\
431	0\\
432	0\\
433	0\\
434	0\\
435	0\\
436	0\\
437	0\\
438	0\\
439	0\\
440	0\\
441	0\\
442	0\\
443	0\\
444	0\\
445	0\\
446	0\\
447	0\\
448	0\\
449	0\\
450	0\\
451	0\\
452	0\\
453	0\\
454	0\\
455	0\\
456	0\\
457	0\\
458	0\\
459	0\\
460	0\\
461	0\\
462	0\\
463	0\\
464	0\\
465	0\\
466	0\\
467	0\\
468	0\\
469	0\\
470	0\\
471	0\\
472	0\\
473	0\\
474	0\\
475	0\\
476	0\\
477	0\\
478	0\\
479	0\\
480	0\\
481	0\\
482	0\\
483	0\\
484	0\\
485	0\\
486	0\\
487	0\\
488	0\\
489	0\\
490	0\\
491	0\\
492	0\\
493	0\\
494	0\\
495	0\\
496	0\\
497	0\\
498	0\\
499	0\\
500	0\\
501	0\\
502	0\\
503	0\\
504	0\\
505	0\\
506	0\\
507	0\\
508	0\\
509	0\\
510	0\\
511	0\\
512	0\\
513	0\\
514	0\\
515	0\\
516	0\\
517	0\\
518	0\\
519	0\\
520	0\\
521	0\\
522	0\\
523	0\\
524	0\\
525	0\\
526	0\\
527	0\\
528	0\\
529	0\\
530	0\\
531	0\\
532	0\\
533	0\\
534	0\\
535	0\\
536	0\\
537	0\\
538	0\\
539	0\\
540	0\\
541	0\\
542	0\\
543	0\\
544	0\\
545	0\\
546	0\\
547	0\\
548	0\\
549	0\\
550	0\\
551	0\\
552	0\\
553	0\\
554	0\\
555	0\\
556	0\\
557	0\\
558	0\\
559	0\\
560	0\\
561	0\\
562	0\\
563	0\\
564	0\\
565	0\\
566	0\\
567	0\\
568	0\\
569	0\\
570	0\\
571	0\\
572	0\\
573	0\\
574	0\\
575	0\\
576	0\\
577	0\\
578	0\\
579	0.000143936771195565\\
580	0.000398036759884528\\
581	0.00066092679016297\\
582	0.000819713415261157\\
583	0.00096387575301797\\
584	0.00110653927940749\\
585	0.00124960147875673\\
586	0.00139021956195936\\
587	0.00152910028465296\\
588	0.00165893564622842\\
589	0.00185748867593825\\
590	0.00235311191256397\\
591	0.00285266479926953\\
592	0.00331714209625941\\
593	0.00350129544269832\\
594	0.00369876633599398\\
595	0.00393138688805932\\
596	0.00424489192197758\\
597	0.0048700418989439\\
598	0.00632942537858856\\
599	0\\
600	0\\
};
\addplot [color=mycolor16,solid,forget plot]
  table[row sep=crcr]{%
1	0\\
2	0\\
3	0\\
4	0\\
5	0\\
6	0\\
7	0\\
8	0\\
9	0\\
10	0\\
11	0\\
12	0\\
13	0\\
14	0\\
15	0\\
16	0\\
17	0\\
18	0\\
19	0\\
20	0\\
21	0\\
22	0\\
23	0\\
24	0\\
25	0\\
26	0\\
27	0\\
28	0\\
29	0\\
30	0\\
31	0\\
32	0\\
33	0\\
34	0\\
35	0\\
36	0\\
37	0\\
38	0\\
39	0\\
40	0\\
41	0\\
42	0\\
43	0\\
44	0\\
45	0\\
46	0\\
47	0\\
48	0\\
49	0\\
50	0\\
51	0\\
52	0\\
53	0\\
54	0\\
55	0\\
56	0\\
57	0\\
58	0\\
59	0\\
60	0\\
61	0\\
62	0\\
63	0\\
64	0\\
65	0\\
66	0\\
67	0\\
68	0\\
69	0\\
70	0\\
71	0\\
72	0\\
73	0\\
74	0\\
75	0\\
76	0\\
77	0\\
78	0\\
79	0\\
80	0\\
81	0\\
82	0\\
83	0\\
84	0\\
85	0\\
86	0\\
87	0\\
88	0\\
89	0\\
90	0\\
91	0\\
92	0\\
93	0\\
94	0\\
95	0\\
96	0\\
97	0\\
98	0\\
99	0\\
100	0\\
101	0\\
102	0\\
103	0\\
104	0\\
105	0\\
106	0\\
107	0\\
108	0\\
109	0\\
110	0\\
111	0\\
112	0\\
113	0\\
114	0\\
115	0\\
116	0\\
117	0\\
118	0\\
119	0\\
120	0\\
121	0\\
122	0\\
123	0\\
124	0\\
125	0\\
126	0\\
127	0\\
128	0\\
129	0\\
130	0\\
131	0\\
132	0\\
133	0\\
134	0\\
135	0\\
136	0\\
137	0\\
138	0\\
139	0\\
140	0\\
141	0\\
142	0\\
143	0\\
144	0\\
145	0\\
146	0\\
147	0\\
148	0\\
149	0\\
150	0\\
151	0\\
152	0\\
153	0\\
154	0\\
155	0\\
156	0\\
157	0\\
158	0\\
159	0\\
160	0\\
161	0\\
162	0\\
163	0\\
164	0\\
165	0\\
166	0\\
167	0\\
168	0\\
169	0\\
170	0\\
171	0\\
172	0\\
173	0\\
174	0\\
175	0\\
176	0\\
177	0\\
178	0\\
179	0\\
180	0\\
181	0\\
182	0\\
183	0\\
184	0\\
185	0\\
186	0\\
187	0\\
188	0\\
189	0\\
190	0\\
191	0\\
192	0\\
193	0\\
194	0\\
195	0\\
196	0\\
197	0\\
198	0\\
199	0\\
200	0\\
201	0\\
202	0\\
203	0\\
204	0\\
205	0\\
206	0\\
207	0\\
208	0\\
209	0\\
210	0\\
211	0\\
212	0\\
213	0\\
214	0\\
215	0\\
216	0\\
217	0\\
218	0\\
219	0\\
220	0\\
221	0\\
222	0\\
223	0\\
224	0\\
225	0\\
226	0\\
227	0\\
228	0\\
229	0\\
230	0\\
231	0\\
232	0\\
233	0\\
234	0\\
235	0\\
236	0\\
237	0\\
238	0\\
239	0\\
240	0\\
241	0\\
242	0\\
243	0\\
244	0\\
245	0\\
246	0\\
247	0\\
248	0\\
249	0\\
250	0\\
251	0\\
252	0\\
253	0\\
254	0\\
255	0\\
256	0\\
257	0\\
258	0\\
259	0\\
260	0\\
261	0\\
262	0\\
263	0\\
264	0\\
265	0\\
266	0\\
267	0\\
268	0\\
269	0\\
270	0\\
271	0\\
272	0\\
273	0\\
274	0\\
275	0\\
276	0\\
277	0\\
278	0\\
279	0\\
280	0\\
281	0\\
282	0\\
283	0\\
284	0\\
285	0\\
286	0\\
287	0\\
288	0\\
289	0\\
290	0\\
291	0\\
292	0\\
293	0\\
294	0\\
295	0\\
296	0\\
297	0\\
298	0\\
299	0\\
300	0\\
301	0\\
302	0\\
303	0\\
304	0\\
305	0\\
306	0\\
307	0\\
308	0\\
309	0\\
310	0\\
311	0\\
312	0\\
313	0\\
314	0\\
315	0\\
316	0\\
317	0\\
318	0\\
319	0\\
320	0\\
321	0\\
322	0\\
323	0\\
324	0\\
325	0\\
326	0\\
327	0\\
328	0\\
329	0\\
330	0\\
331	0\\
332	0\\
333	0\\
334	0\\
335	0\\
336	0\\
337	0\\
338	0\\
339	0\\
340	0\\
341	0\\
342	0\\
343	0\\
344	0\\
345	0\\
346	0\\
347	0\\
348	0\\
349	0\\
350	0\\
351	0\\
352	0\\
353	0\\
354	0\\
355	0\\
356	0\\
357	0\\
358	0\\
359	0\\
360	0\\
361	0\\
362	0\\
363	0\\
364	0\\
365	0\\
366	0\\
367	0\\
368	0\\
369	0\\
370	0\\
371	0\\
372	0\\
373	0\\
374	0\\
375	0\\
376	0\\
377	0\\
378	0\\
379	0\\
380	0\\
381	0\\
382	0\\
383	0\\
384	0\\
385	0\\
386	0\\
387	0\\
388	0\\
389	0\\
390	0\\
391	0\\
392	0\\
393	0\\
394	0\\
395	0\\
396	0\\
397	0\\
398	0\\
399	0\\
400	0\\
401	0\\
402	0\\
403	0\\
404	0\\
405	0\\
406	0\\
407	0\\
408	0\\
409	0\\
410	0\\
411	0\\
412	0\\
413	0\\
414	0\\
415	0\\
416	0\\
417	0\\
418	0\\
419	0\\
420	0\\
421	0\\
422	0\\
423	0\\
424	0\\
425	0\\
426	0\\
427	0\\
428	0\\
429	0\\
430	0\\
431	0\\
432	0\\
433	0\\
434	0\\
435	0\\
436	0\\
437	0\\
438	0\\
439	0\\
440	0\\
441	0\\
442	0\\
443	0\\
444	0\\
445	0\\
446	0\\
447	0\\
448	0\\
449	0\\
450	0\\
451	0\\
452	0\\
453	0\\
454	0\\
455	0\\
456	0\\
457	0\\
458	0\\
459	0\\
460	0\\
461	0\\
462	0\\
463	0\\
464	0\\
465	0\\
466	0\\
467	0\\
468	0\\
469	0\\
470	0\\
471	0\\
472	0\\
473	0\\
474	0\\
475	0\\
476	0\\
477	0\\
478	0\\
479	0\\
480	0\\
481	0\\
482	0\\
483	0\\
484	0\\
485	0\\
486	0\\
487	0\\
488	0\\
489	0\\
490	0\\
491	0\\
492	0\\
493	0\\
494	0\\
495	0\\
496	0\\
497	0\\
498	0\\
499	0\\
500	0\\
501	0\\
502	0\\
503	0\\
504	0\\
505	0\\
506	0\\
507	0\\
508	0\\
509	0\\
510	0\\
511	0\\
512	0\\
513	0\\
514	0\\
515	0\\
516	0\\
517	0\\
518	0\\
519	0\\
520	0\\
521	0\\
522	0\\
523	0\\
524	0\\
525	0\\
526	0\\
527	0\\
528	0\\
529	0\\
530	0\\
531	0\\
532	0\\
533	0\\
534	0\\
535	0\\
536	0\\
537	0\\
538	0\\
539	0\\
540	0\\
541	0\\
542	0\\
543	0\\
544	0\\
545	0\\
546	0\\
547	0\\
548	0\\
549	0\\
550	0\\
551	0\\
552	0\\
553	0\\
554	0\\
555	0\\
556	0\\
557	0\\
558	0\\
559	0\\
560	0\\
561	0\\
562	0\\
563	0\\
564	0\\
565	0\\
566	0\\
567	0\\
568	0\\
569	0\\
570	0\\
571	0\\
572	0\\
573	0.000161696296701996\\
574	0.000393713864803922\\
575	0.000598899566224693\\
576	0.000720797071750311\\
577	0.000841506955469324\\
578	0.000959997374964595\\
579	0.00107594147230983\\
580	0.00118817755523261\\
581	0.00129194223306109\\
582	0.00138617795658753\\
583	0.00147953228162457\\
584	0.00157379580842985\\
585	0.00166431968199408\\
586	0.00175199502013113\\
587	0.00222288556602185\\
588	0.00270274779694502\\
589	0.00311897222604471\\
590	0.00326220003215725\\
591	0.00340447372974604\\
592	0.00353900409992704\\
593	0.00365273120413177\\
594	0.00378238158822432\\
595	0.0039567480325281\\
596	0.00425297862742962\\
597	0.0048700418989439\\
598	0.00632942537858856\\
599	0\\
600	0\\
};
\addplot [color=mycolor17,solid,forget plot]
  table[row sep=crcr]{%
1	0\\
2	0\\
3	0\\
4	0\\
5	0\\
6	0\\
7	0\\
8	0\\
9	0\\
10	0\\
11	0\\
12	0\\
13	0\\
14	0\\
15	0\\
16	0\\
17	0\\
18	0\\
19	0\\
20	0\\
21	0\\
22	0\\
23	0\\
24	0\\
25	0\\
26	0\\
27	0\\
28	0\\
29	0\\
30	0\\
31	0\\
32	0\\
33	0\\
34	0\\
35	0\\
36	0\\
37	0\\
38	0\\
39	0\\
40	0\\
41	0\\
42	0\\
43	0\\
44	0\\
45	0\\
46	0\\
47	0\\
48	0\\
49	0\\
50	0\\
51	0\\
52	0\\
53	0\\
54	0\\
55	0\\
56	0\\
57	0\\
58	0\\
59	0\\
60	0\\
61	0\\
62	0\\
63	0\\
64	0\\
65	0\\
66	0\\
67	0\\
68	0\\
69	0\\
70	0\\
71	0\\
72	0\\
73	0\\
74	0\\
75	0\\
76	0\\
77	0\\
78	0\\
79	0\\
80	0\\
81	0\\
82	0\\
83	0\\
84	0\\
85	0\\
86	0\\
87	0\\
88	0\\
89	0\\
90	0\\
91	0\\
92	0\\
93	0\\
94	0\\
95	0\\
96	0\\
97	0\\
98	0\\
99	0\\
100	0\\
101	0\\
102	0\\
103	0\\
104	0\\
105	0\\
106	0\\
107	0\\
108	0\\
109	0\\
110	0\\
111	0\\
112	0\\
113	0\\
114	0\\
115	0\\
116	0\\
117	0\\
118	0\\
119	0\\
120	0\\
121	0\\
122	0\\
123	0\\
124	0\\
125	0\\
126	0\\
127	0\\
128	0\\
129	0\\
130	0\\
131	0\\
132	0\\
133	0\\
134	0\\
135	0\\
136	0\\
137	0\\
138	0\\
139	0\\
140	0\\
141	0\\
142	0\\
143	0\\
144	0\\
145	0\\
146	0\\
147	0\\
148	0\\
149	0\\
150	0\\
151	0\\
152	0\\
153	0\\
154	0\\
155	0\\
156	0\\
157	0\\
158	0\\
159	0\\
160	0\\
161	0\\
162	0\\
163	0\\
164	0\\
165	0\\
166	0\\
167	0\\
168	0\\
169	0\\
170	0\\
171	0\\
172	0\\
173	0\\
174	0\\
175	0\\
176	0\\
177	0\\
178	0\\
179	0\\
180	0\\
181	0\\
182	0\\
183	0\\
184	0\\
185	0\\
186	0\\
187	0\\
188	0\\
189	0\\
190	0\\
191	0\\
192	0\\
193	0\\
194	0\\
195	0\\
196	0\\
197	0\\
198	0\\
199	0\\
200	0\\
201	0\\
202	0\\
203	0\\
204	0\\
205	0\\
206	0\\
207	0\\
208	0\\
209	0\\
210	0\\
211	0\\
212	0\\
213	0\\
214	0\\
215	0\\
216	0\\
217	0\\
218	0\\
219	0\\
220	0\\
221	0\\
222	0\\
223	0\\
224	0\\
225	0\\
226	0\\
227	0\\
228	0\\
229	0\\
230	0\\
231	0\\
232	0\\
233	0\\
234	0\\
235	0\\
236	0\\
237	0\\
238	0\\
239	0\\
240	0\\
241	0\\
242	0\\
243	0\\
244	0\\
245	0\\
246	0\\
247	0\\
248	0\\
249	0\\
250	0\\
251	0\\
252	0\\
253	0\\
254	0\\
255	0\\
256	0\\
257	0\\
258	0\\
259	0\\
260	0\\
261	0\\
262	0\\
263	0\\
264	0\\
265	0\\
266	0\\
267	0\\
268	0\\
269	0\\
270	0\\
271	0\\
272	0\\
273	0\\
274	0\\
275	0\\
276	0\\
277	0\\
278	0\\
279	0\\
280	0\\
281	0\\
282	0\\
283	0\\
284	0\\
285	0\\
286	0\\
287	0\\
288	0\\
289	0\\
290	0\\
291	0\\
292	0\\
293	0\\
294	0\\
295	0\\
296	0\\
297	0\\
298	0\\
299	0\\
300	0\\
301	0\\
302	0\\
303	0\\
304	0\\
305	0\\
306	0\\
307	0\\
308	0\\
309	0\\
310	0\\
311	0\\
312	0\\
313	0\\
314	0\\
315	0\\
316	0\\
317	0\\
318	0\\
319	0\\
320	0\\
321	0\\
322	0\\
323	0\\
324	0\\
325	0\\
326	0\\
327	0\\
328	0\\
329	0\\
330	0\\
331	0\\
332	0\\
333	0\\
334	0\\
335	0\\
336	0\\
337	0\\
338	0\\
339	0\\
340	0\\
341	0\\
342	0\\
343	0\\
344	0\\
345	0\\
346	0\\
347	0\\
348	0\\
349	0\\
350	0\\
351	0\\
352	0\\
353	0\\
354	0\\
355	0\\
356	0\\
357	0\\
358	0\\
359	0\\
360	0\\
361	0\\
362	0\\
363	0\\
364	0\\
365	0\\
366	0\\
367	0\\
368	0\\
369	0\\
370	0\\
371	0\\
372	0\\
373	0\\
374	0\\
375	0\\
376	0\\
377	0\\
378	0\\
379	0\\
380	0\\
381	0\\
382	0\\
383	0\\
384	0\\
385	0\\
386	0\\
387	0\\
388	0\\
389	0\\
390	0\\
391	0\\
392	0\\
393	0\\
394	0\\
395	0\\
396	0\\
397	0\\
398	0\\
399	0\\
400	0\\
401	0\\
402	0\\
403	0\\
404	0\\
405	0\\
406	0\\
407	0\\
408	0\\
409	0\\
410	0\\
411	0\\
412	0\\
413	0\\
414	0\\
415	0\\
416	0\\
417	0\\
418	0\\
419	0\\
420	0\\
421	0\\
422	0\\
423	0\\
424	0\\
425	0\\
426	0\\
427	0\\
428	0\\
429	0\\
430	0\\
431	0\\
432	0\\
433	0\\
434	0\\
435	0\\
436	0\\
437	0\\
438	0\\
439	0\\
440	0\\
441	0\\
442	0\\
443	0\\
444	0\\
445	0\\
446	0\\
447	0\\
448	0\\
449	0\\
450	0\\
451	0\\
452	0\\
453	0\\
454	0\\
455	0\\
456	0\\
457	0\\
458	0\\
459	0\\
460	0\\
461	0\\
462	0\\
463	0\\
464	0\\
465	0\\
466	0\\
467	0\\
468	0\\
469	0\\
470	0\\
471	0\\
472	0\\
473	0\\
474	0\\
475	0\\
476	0\\
477	0\\
478	0\\
479	0\\
480	0\\
481	0\\
482	0\\
483	0\\
484	0\\
485	0\\
486	0\\
487	0\\
488	0\\
489	0\\
490	0\\
491	0\\
492	0\\
493	0\\
494	0\\
495	0\\
496	0\\
497	0\\
498	0\\
499	0\\
500	0\\
501	0\\
502	0\\
503	0\\
504	0\\
505	0\\
506	0\\
507	0\\
508	0\\
509	0\\
510	0\\
511	0\\
512	0\\
513	0\\
514	0\\
515	0\\
516	0\\
517	0\\
518	0\\
519	0\\
520	0\\
521	0\\
522	0\\
523	0\\
524	0\\
525	0\\
526	0\\
527	0\\
528	0\\
529	0\\
530	0\\
531	0\\
532	0\\
533	0\\
534	0\\
535	0\\
536	0\\
537	0\\
538	0\\
539	0\\
540	0\\
541	0\\
542	0\\
543	0\\
544	0\\
545	0\\
546	0\\
547	0\\
548	0\\
549	0\\
550	0\\
551	0\\
552	0\\
553	0\\
554	0\\
555	0\\
556	0\\
557	0\\
558	0\\
559	0\\
560	0\\
561	0\\
562	0\\
563	0\\
564	0\\
565	0\\
566	0\\
567	5.88949037576846e-05\\
568	0.000278923522351577\\
569	0.000465215789347868\\
570	0.000573095561531768\\
571	0.000679064755436726\\
572	0.000781407100120683\\
573	0.000879327214528265\\
574	0.000971216243397552\\
575	0.00105576846491719\\
576	0.00113450507723386\\
577	0.00121198502541605\\
578	0.00128801880280719\\
579	0.00136213507180008\\
580	0.00143518170906471\\
581	0.00150992377660234\\
582	0.00158312346660404\\
583	0.00165386180226987\\
584	0.00192128218169381\\
585	0.00239891244873237\\
586	0.00288945262252655\\
587	0.00302337868377814\\
588	0.00315170319741699\\
589	0.00326939440428176\\
590	0.00336308691806496\\
591	0.00345791823244592\\
592	0.00355546598078343\\
593	0.00366084952809565\\
594	0.00378520541414672\\
595	0.00395742119285503\\
596	0.00425297862742962\\
597	0.0048700418989439\\
598	0.00632942537858856\\
599	0\\
600	0\\
};
\addplot [color=mycolor18,solid,forget plot]
  table[row sep=crcr]{%
1	0\\
2	0\\
3	0\\
4	0\\
5	0\\
6	0\\
7	0\\
8	0\\
9	0\\
10	0\\
11	0\\
12	0\\
13	0\\
14	0\\
15	0\\
16	0\\
17	0\\
18	0\\
19	0\\
20	0\\
21	0\\
22	0\\
23	0\\
24	0\\
25	0\\
26	0\\
27	0\\
28	0\\
29	0\\
30	0\\
31	0\\
32	0\\
33	0\\
34	0\\
35	0\\
36	0\\
37	0\\
38	0\\
39	0\\
40	0\\
41	0\\
42	0\\
43	0\\
44	0\\
45	0\\
46	0\\
47	0\\
48	0\\
49	0\\
50	0\\
51	0\\
52	0\\
53	0\\
54	0\\
55	0\\
56	0\\
57	0\\
58	0\\
59	0\\
60	0\\
61	0\\
62	0\\
63	0\\
64	0\\
65	0\\
66	0\\
67	0\\
68	0\\
69	0\\
70	0\\
71	0\\
72	0\\
73	0\\
74	0\\
75	0\\
76	0\\
77	0\\
78	0\\
79	0\\
80	0\\
81	0\\
82	0\\
83	0\\
84	0\\
85	0\\
86	0\\
87	0\\
88	0\\
89	0\\
90	0\\
91	0\\
92	0\\
93	0\\
94	0\\
95	0\\
96	0\\
97	0\\
98	0\\
99	0\\
100	0\\
101	0\\
102	0\\
103	0\\
104	0\\
105	0\\
106	0\\
107	0\\
108	0\\
109	0\\
110	0\\
111	0\\
112	0\\
113	0\\
114	0\\
115	0\\
116	0\\
117	0\\
118	0\\
119	0\\
120	0\\
121	0\\
122	0\\
123	0\\
124	0\\
125	0\\
126	0\\
127	0\\
128	0\\
129	0\\
130	0\\
131	0\\
132	0\\
133	0\\
134	0\\
135	0\\
136	0\\
137	0\\
138	0\\
139	0\\
140	0\\
141	0\\
142	0\\
143	0\\
144	0\\
145	0\\
146	0\\
147	0\\
148	0\\
149	0\\
150	0\\
151	0\\
152	0\\
153	0\\
154	0\\
155	0\\
156	0\\
157	0\\
158	0\\
159	0\\
160	0\\
161	0\\
162	0\\
163	0\\
164	0\\
165	0\\
166	0\\
167	0\\
168	0\\
169	0\\
170	0\\
171	0\\
172	0\\
173	0\\
174	0\\
175	0\\
176	0\\
177	0\\
178	0\\
179	0\\
180	0\\
181	0\\
182	0\\
183	0\\
184	0\\
185	0\\
186	0\\
187	0\\
188	0\\
189	0\\
190	0\\
191	0\\
192	0\\
193	0\\
194	0\\
195	0\\
196	0\\
197	0\\
198	0\\
199	0\\
200	0\\
201	0\\
202	0\\
203	0\\
204	0\\
205	0\\
206	0\\
207	0\\
208	0\\
209	0\\
210	0\\
211	0\\
212	0\\
213	0\\
214	0\\
215	0\\
216	0\\
217	0\\
218	0\\
219	0\\
220	0\\
221	0\\
222	0\\
223	0\\
224	0\\
225	0\\
226	0\\
227	0\\
228	0\\
229	0\\
230	0\\
231	0\\
232	0\\
233	0\\
234	0\\
235	0\\
236	0\\
237	0\\
238	0\\
239	0\\
240	0\\
241	0\\
242	0\\
243	0\\
244	0\\
245	0\\
246	0\\
247	0\\
248	0\\
249	0\\
250	0\\
251	0\\
252	0\\
253	0\\
254	0\\
255	0\\
256	0\\
257	0\\
258	0\\
259	0\\
260	0\\
261	0\\
262	0\\
263	0\\
264	0\\
265	0\\
266	0\\
267	0\\
268	0\\
269	0\\
270	0\\
271	0\\
272	0\\
273	0\\
274	0\\
275	0\\
276	0\\
277	0\\
278	0\\
279	0\\
280	0\\
281	0\\
282	0\\
283	0\\
284	0\\
285	0\\
286	0\\
287	0\\
288	0\\
289	0\\
290	0\\
291	0\\
292	0\\
293	0\\
294	0\\
295	0\\
296	0\\
297	0\\
298	0\\
299	0\\
300	0\\
301	0\\
302	0\\
303	0\\
304	0\\
305	0\\
306	0\\
307	0\\
308	0\\
309	0\\
310	0\\
311	0\\
312	0\\
313	0\\
314	0\\
315	0\\
316	0\\
317	0\\
318	0\\
319	0\\
320	0\\
321	0\\
322	0\\
323	0\\
324	0\\
325	0\\
326	0\\
327	0\\
328	0\\
329	0\\
330	0\\
331	0\\
332	0\\
333	0\\
334	0\\
335	0\\
336	0\\
337	0\\
338	0\\
339	0\\
340	0\\
341	0\\
342	0\\
343	0\\
344	0\\
345	0\\
346	0\\
347	0\\
348	0\\
349	0\\
350	0\\
351	0\\
352	0\\
353	0\\
354	0\\
355	0\\
356	0\\
357	0\\
358	0\\
359	0\\
360	0\\
361	0\\
362	0\\
363	0\\
364	0\\
365	0\\
366	0\\
367	0\\
368	0\\
369	0\\
370	0\\
371	0\\
372	0\\
373	0\\
374	0\\
375	0\\
376	0\\
377	0\\
378	0\\
379	0\\
380	0\\
381	0\\
382	0\\
383	0\\
384	0\\
385	0\\
386	0\\
387	0\\
388	0\\
389	0\\
390	0\\
391	0\\
392	0\\
393	0\\
394	0\\
395	0\\
396	0\\
397	0\\
398	0\\
399	0\\
400	0\\
401	0\\
402	0\\
403	0\\
404	0\\
405	0\\
406	0\\
407	0\\
408	0\\
409	0\\
410	0\\
411	0\\
412	0\\
413	0\\
414	0\\
415	0\\
416	0\\
417	0\\
418	0\\
419	0\\
420	0\\
421	0\\
422	0\\
423	0\\
424	0\\
425	0\\
426	0\\
427	0\\
428	0\\
429	0\\
430	0\\
431	0\\
432	0\\
433	0\\
434	0\\
435	0\\
436	0\\
437	0\\
438	0\\
439	0\\
440	0\\
441	0\\
442	0\\
443	0\\
444	0\\
445	0\\
446	0\\
447	0\\
448	0\\
449	0\\
450	0\\
451	0\\
452	0\\
453	0\\
454	0\\
455	0\\
456	0\\
457	0\\
458	0\\
459	0\\
460	0\\
461	0\\
462	0\\
463	0\\
464	0\\
465	0\\
466	0\\
467	0\\
468	0\\
469	0\\
470	0\\
471	0\\
472	0\\
473	0\\
474	0\\
475	0\\
476	0\\
477	0\\
478	0\\
479	0\\
480	0\\
481	0\\
482	0\\
483	0\\
484	0\\
485	0\\
486	0\\
487	0\\
488	0\\
489	0\\
490	0\\
491	0\\
492	0\\
493	0\\
494	0\\
495	0\\
496	0\\
497	0\\
498	0\\
499	0\\
500	0\\
501	0\\
502	0\\
503	0\\
504	0\\
505	0\\
506	0\\
507	0\\
508	0\\
509	0\\
510	0\\
511	0\\
512	0\\
513	0\\
514	0\\
515	0\\
516	0\\
517	0\\
518	0\\
519	0\\
520	0\\
521	0\\
522	0\\
523	0\\
524	0\\
525	0\\
526	0\\
527	0\\
528	0\\
529	0\\
530	0\\
531	0\\
532	0\\
533	0\\
534	0\\
535	0\\
536	0\\
537	0\\
538	0\\
539	0\\
540	0\\
541	0\\
542	0\\
543	0\\
544	0\\
545	0\\
546	0\\
547	0\\
548	0\\
549	0\\
550	0\\
551	0\\
552	0\\
553	0\\
554	0\\
555	0\\
556	0\\
557	0\\
558	0\\
559	0\\
560	0\\
561	0\\
562	9.90403746872915e-05\\
563	0.000304968769744545\\
564	0.000402367207070976\\
565	0.000497931023480415\\
566	0.000590494868905439\\
567	0.000678077044756074\\
568	0.000758685659673696\\
569	0.000830189121630715\\
570	0.000895650850952982\\
571	0.000961984822474034\\
572	0.0010278398051382\\
573	0.00109225696560847\\
574	0.00115548679208059\\
575	0.00121672171450071\\
576	0.00127705683954306\\
577	0.00133948023067548\\
578	0.00140501315387362\\
579	0.00146934333816017\\
580	0.00153055864356594\\
581	0.00158857538063284\\
582	0.00197419871052425\\
583	0.00246466618649465\\
584	0.00277686403795953\\
585	0.00289976434281973\\
586	0.00301736053854785\\
587	0.00310424045129586\\
588	0.00319120167814864\\
589	0.00327881165853131\\
590	0.00336825554262852\\
591	0.00346025420288071\\
592	0.00355649290755294\\
593	0.00366118457402185\\
594	0.00378527046594782\\
595	0.00395742119285503\\
596	0.00425297862742962\\
597	0.0048700418989439\\
598	0.00632942537858856\\
599	0\\
600	0\\
};
\addplot [color=red!25!mycolor17,solid,forget plot]
  table[row sep=crcr]{%
1	0\\
2	0\\
3	0\\
4	0\\
5	0\\
6	0\\
7	0\\
8	0\\
9	0\\
10	0\\
11	0\\
12	0\\
13	0\\
14	0\\
15	0\\
16	0\\
17	0\\
18	0\\
19	0\\
20	0\\
21	0\\
22	0\\
23	0\\
24	0\\
25	0\\
26	0\\
27	0\\
28	0\\
29	0\\
30	0\\
31	0\\
32	0\\
33	0\\
34	0\\
35	0\\
36	0\\
37	0\\
38	0\\
39	0\\
40	0\\
41	0\\
42	0\\
43	0\\
44	0\\
45	0\\
46	0\\
47	0\\
48	0\\
49	0\\
50	0\\
51	0\\
52	0\\
53	0\\
54	0\\
55	0\\
56	0\\
57	0\\
58	0\\
59	0\\
60	0\\
61	0\\
62	0\\
63	0\\
64	0\\
65	0\\
66	0\\
67	0\\
68	0\\
69	0\\
70	0\\
71	0\\
72	0\\
73	0\\
74	0\\
75	0\\
76	0\\
77	0\\
78	0\\
79	0\\
80	0\\
81	0\\
82	0\\
83	0\\
84	0\\
85	0\\
86	0\\
87	0\\
88	0\\
89	0\\
90	0\\
91	0\\
92	0\\
93	0\\
94	0\\
95	0\\
96	0\\
97	0\\
98	0\\
99	0\\
100	0\\
101	0\\
102	0\\
103	0\\
104	0\\
105	0\\
106	0\\
107	0\\
108	0\\
109	0\\
110	0\\
111	0\\
112	0\\
113	0\\
114	0\\
115	0\\
116	0\\
117	0\\
118	0\\
119	0\\
120	0\\
121	0\\
122	0\\
123	0\\
124	0\\
125	0\\
126	0\\
127	0\\
128	0\\
129	0\\
130	0\\
131	0\\
132	0\\
133	0\\
134	0\\
135	0\\
136	0\\
137	0\\
138	0\\
139	0\\
140	0\\
141	0\\
142	0\\
143	0\\
144	0\\
145	0\\
146	0\\
147	0\\
148	0\\
149	0\\
150	0\\
151	0\\
152	0\\
153	0\\
154	0\\
155	0\\
156	0\\
157	0\\
158	0\\
159	0\\
160	0\\
161	0\\
162	0\\
163	0\\
164	0\\
165	0\\
166	0\\
167	0\\
168	0\\
169	0\\
170	0\\
171	0\\
172	0\\
173	0\\
174	0\\
175	0\\
176	0\\
177	0\\
178	0\\
179	0\\
180	0\\
181	0\\
182	0\\
183	0\\
184	0\\
185	0\\
186	0\\
187	0\\
188	0\\
189	0\\
190	0\\
191	0\\
192	0\\
193	0\\
194	0\\
195	0\\
196	0\\
197	0\\
198	0\\
199	0\\
200	0\\
201	0\\
202	0\\
203	0\\
204	0\\
205	0\\
206	0\\
207	0\\
208	0\\
209	0\\
210	0\\
211	0\\
212	0\\
213	0\\
214	0\\
215	0\\
216	0\\
217	0\\
218	0\\
219	0\\
220	0\\
221	0\\
222	0\\
223	0\\
224	0\\
225	0\\
226	0\\
227	0\\
228	0\\
229	0\\
230	0\\
231	0\\
232	0\\
233	0\\
234	0\\
235	0\\
236	0\\
237	0\\
238	0\\
239	0\\
240	0\\
241	0\\
242	0\\
243	0\\
244	0\\
245	0\\
246	0\\
247	0\\
248	0\\
249	0\\
250	0\\
251	0\\
252	0\\
253	0\\
254	0\\
255	0\\
256	0\\
257	0\\
258	0\\
259	0\\
260	0\\
261	0\\
262	0\\
263	0\\
264	0\\
265	0\\
266	0\\
267	0\\
268	0\\
269	0\\
270	0\\
271	0\\
272	0\\
273	0\\
274	0\\
275	0\\
276	0\\
277	0\\
278	0\\
279	0\\
280	0\\
281	0\\
282	0\\
283	0\\
284	0\\
285	0\\
286	0\\
287	0\\
288	0\\
289	0\\
290	0\\
291	0\\
292	0\\
293	0\\
294	0\\
295	0\\
296	0\\
297	0\\
298	0\\
299	0\\
300	0\\
301	0\\
302	0\\
303	0\\
304	0\\
305	0\\
306	0\\
307	0\\
308	0\\
309	0\\
310	0\\
311	0\\
312	0\\
313	0\\
314	0\\
315	0\\
316	0\\
317	0\\
318	0\\
319	0\\
320	0\\
321	0\\
322	0\\
323	0\\
324	0\\
325	0\\
326	0\\
327	0\\
328	0\\
329	0\\
330	0\\
331	0\\
332	0\\
333	0\\
334	0\\
335	0\\
336	0\\
337	0\\
338	0\\
339	0\\
340	0\\
341	0\\
342	0\\
343	0\\
344	0\\
345	0\\
346	0\\
347	0\\
348	0\\
349	0\\
350	0\\
351	0\\
352	0\\
353	0\\
354	0\\
355	0\\
356	0\\
357	0\\
358	0\\
359	0\\
360	0\\
361	0\\
362	0\\
363	0\\
364	0\\
365	0\\
366	0\\
367	0\\
368	0\\
369	0\\
370	0\\
371	0\\
372	0\\
373	0\\
374	0\\
375	0\\
376	0\\
377	0\\
378	0\\
379	0\\
380	0\\
381	0\\
382	0\\
383	0\\
384	0\\
385	0\\
386	0\\
387	0\\
388	0\\
389	0\\
390	0\\
391	0\\
392	0\\
393	0\\
394	0\\
395	0\\
396	0\\
397	0\\
398	0\\
399	0\\
400	0\\
401	0\\
402	0\\
403	0\\
404	0\\
405	0\\
406	0\\
407	0\\
408	0\\
409	0\\
410	0\\
411	0\\
412	0\\
413	0\\
414	0\\
415	0\\
416	0\\
417	0\\
418	0\\
419	0\\
420	0\\
421	0\\
422	0\\
423	0\\
424	0\\
425	0\\
426	0\\
427	0\\
428	0\\
429	0\\
430	0\\
431	0\\
432	0\\
433	0\\
434	0\\
435	0\\
436	0\\
437	0\\
438	0\\
439	0\\
440	0\\
441	0\\
442	0\\
443	0\\
444	0\\
445	0\\
446	0\\
447	0\\
448	0\\
449	0\\
450	0\\
451	0\\
452	0\\
453	0\\
454	0\\
455	0\\
456	0\\
457	0\\
458	0\\
459	0\\
460	0\\
461	0\\
462	0\\
463	0\\
464	0\\
465	0\\
466	0\\
467	0\\
468	0\\
469	0\\
470	0\\
471	0\\
472	0\\
473	0\\
474	0\\
475	0\\
476	0\\
477	0\\
478	0\\
479	0\\
480	0\\
481	0\\
482	0\\
483	0\\
484	0\\
485	0\\
486	0\\
487	0\\
488	0\\
489	0\\
490	0\\
491	0\\
492	0\\
493	0\\
494	0\\
495	0\\
496	0\\
497	0\\
498	0\\
499	0\\
500	0\\
501	0\\
502	0\\
503	0\\
504	0\\
505	0\\
506	0\\
507	0\\
508	0\\
509	0\\
510	0\\
511	0\\
512	0\\
513	0\\
514	0\\
515	0\\
516	0\\
517	0\\
518	0\\
519	0\\
520	0\\
521	0\\
522	0\\
523	0\\
524	0\\
525	0\\
526	0\\
527	0\\
528	0\\
529	0\\
530	0\\
531	0\\
532	0\\
533	0\\
534	0\\
535	0\\
536	0\\
537	0\\
538	0\\
539	0\\
540	0\\
541	0\\
542	0\\
543	0\\
544	0\\
545	0\\
546	0\\
547	0\\
548	0\\
549	0\\
550	0\\
551	0\\
552	0\\
553	0\\
554	0\\
555	0\\
556	0\\
557	8.01734108211468e-05\\
558	0.000223371265806801\\
559	0.000311803468712962\\
560	0.000397274215465461\\
561	0.000478427144538355\\
562	0.000553370163734247\\
563	0.0006186795095291\\
564	0.000675713681809035\\
565	0.00073191039909959\\
566	0.000787276194635121\\
567	0.000843166680497421\\
568	0.000899372565017849\\
569	0.000954897785455491\\
570	0.00101018526904893\\
571	0.0010637562962866\\
572	0.00111682448780514\\
573	0.00117048000700388\\
574	0.00122509244478267\\
575	0.00128514626774489\\
576	0.00134430788764564\\
577	0.00140097157268809\\
578	0.00145403611431188\\
579	0.00150752687252756\\
580	0.00193194802301185\\
581	0.00243680986819744\\
582	0.00264404234620892\\
583	0.00276021838521539\\
584	0.00285857309192273\\
585	0.00294082185579374\\
586	0.00302342706147123\\
587	0.00310731224695134\\
588	0.00319261476955509\\
589	0.00327956317486525\\
590	0.00336860380042371\\
591	0.00346039193281902\\
592	0.00355653331527142\\
593	0.00366119131364763\\
594	0.00378527046594782\\
595	0.00395742119285503\\
596	0.00425297862742962\\
597	0.0048700418989439\\
598	0.00632942537858856\\
599	0\\
600	0\\
};
\addplot [color=mycolor19,solid,forget plot]
  table[row sep=crcr]{%
1	0\\
2	0\\
3	0\\
4	0\\
5	0\\
6	0\\
7	0\\
8	0\\
9	0\\
10	0\\
11	0\\
12	0\\
13	0\\
14	0\\
15	0\\
16	0\\
17	0\\
18	0\\
19	0\\
20	0\\
21	0\\
22	0\\
23	0\\
24	0\\
25	0\\
26	0\\
27	0\\
28	0\\
29	0\\
30	0\\
31	0\\
32	0\\
33	0\\
34	0\\
35	0\\
36	0\\
37	0\\
38	0\\
39	0\\
40	0\\
41	0\\
42	0\\
43	0\\
44	0\\
45	0\\
46	0\\
47	0\\
48	0\\
49	0\\
50	0\\
51	0\\
52	0\\
53	0\\
54	0\\
55	0\\
56	0\\
57	0\\
58	0\\
59	0\\
60	0\\
61	0\\
62	0\\
63	0\\
64	0\\
65	0\\
66	0\\
67	0\\
68	0\\
69	0\\
70	0\\
71	0\\
72	0\\
73	0\\
74	0\\
75	0\\
76	0\\
77	0\\
78	0\\
79	0\\
80	0\\
81	0\\
82	0\\
83	0\\
84	0\\
85	0\\
86	0\\
87	0\\
88	0\\
89	0\\
90	0\\
91	0\\
92	0\\
93	0\\
94	0\\
95	0\\
96	0\\
97	0\\
98	0\\
99	0\\
100	0\\
101	0\\
102	0\\
103	0\\
104	0\\
105	0\\
106	0\\
107	0\\
108	0\\
109	0\\
110	0\\
111	0\\
112	0\\
113	0\\
114	0\\
115	0\\
116	0\\
117	0\\
118	0\\
119	0\\
120	0\\
121	0\\
122	0\\
123	0\\
124	0\\
125	0\\
126	0\\
127	0\\
128	0\\
129	0\\
130	0\\
131	0\\
132	0\\
133	0\\
134	0\\
135	0\\
136	0\\
137	0\\
138	0\\
139	0\\
140	0\\
141	0\\
142	0\\
143	0\\
144	0\\
145	0\\
146	0\\
147	0\\
148	0\\
149	0\\
150	0\\
151	0\\
152	0\\
153	0\\
154	0\\
155	0\\
156	0\\
157	0\\
158	0\\
159	0\\
160	0\\
161	0\\
162	0\\
163	0\\
164	0\\
165	0\\
166	0\\
167	0\\
168	0\\
169	0\\
170	0\\
171	0\\
172	0\\
173	0\\
174	0\\
175	0\\
176	0\\
177	0\\
178	0\\
179	0\\
180	0\\
181	0\\
182	0\\
183	0\\
184	0\\
185	0\\
186	0\\
187	0\\
188	0\\
189	0\\
190	0\\
191	0\\
192	0\\
193	0\\
194	0\\
195	0\\
196	0\\
197	0\\
198	0\\
199	0\\
200	0\\
201	0\\
202	0\\
203	0\\
204	0\\
205	0\\
206	0\\
207	0\\
208	0\\
209	0\\
210	0\\
211	0\\
212	0\\
213	0\\
214	0\\
215	0\\
216	0\\
217	0\\
218	0\\
219	0\\
220	0\\
221	0\\
222	0\\
223	0\\
224	0\\
225	0\\
226	0\\
227	0\\
228	0\\
229	0\\
230	0\\
231	0\\
232	0\\
233	0\\
234	0\\
235	0\\
236	0\\
237	0\\
238	0\\
239	0\\
240	0\\
241	0\\
242	0\\
243	0\\
244	0\\
245	0\\
246	0\\
247	0\\
248	0\\
249	0\\
250	0\\
251	0\\
252	0\\
253	0\\
254	0\\
255	0\\
256	0\\
257	0\\
258	0\\
259	0\\
260	0\\
261	0\\
262	0\\
263	0\\
264	0\\
265	0\\
266	0\\
267	0\\
268	0\\
269	0\\
270	0\\
271	0\\
272	0\\
273	0\\
274	0\\
275	0\\
276	0\\
277	0\\
278	0\\
279	0\\
280	0\\
281	0\\
282	0\\
283	0\\
284	0\\
285	0\\
286	0\\
287	0\\
288	0\\
289	0\\
290	0\\
291	0\\
292	0\\
293	0\\
294	0\\
295	0\\
296	0\\
297	0\\
298	0\\
299	0\\
300	0\\
301	0\\
302	0\\
303	0\\
304	0\\
305	0\\
306	0\\
307	0\\
308	0\\
309	0\\
310	0\\
311	0\\
312	0\\
313	0\\
314	0\\
315	0\\
316	0\\
317	0\\
318	0\\
319	0\\
320	0\\
321	0\\
322	0\\
323	0\\
324	0\\
325	0\\
326	0\\
327	0\\
328	0\\
329	0\\
330	0\\
331	0\\
332	0\\
333	0\\
334	0\\
335	0\\
336	0\\
337	0\\
338	0\\
339	0\\
340	0\\
341	0\\
342	0\\
343	0\\
344	0\\
345	0\\
346	0\\
347	0\\
348	0\\
349	0\\
350	0\\
351	0\\
352	0\\
353	0\\
354	0\\
355	0\\
356	0\\
357	0\\
358	0\\
359	0\\
360	0\\
361	0\\
362	0\\
363	0\\
364	0\\
365	0\\
366	0\\
367	0\\
368	0\\
369	0\\
370	0\\
371	0\\
372	0\\
373	0\\
374	0\\
375	0\\
376	0\\
377	0\\
378	0\\
379	0\\
380	0\\
381	0\\
382	0\\
383	0\\
384	0\\
385	0\\
386	0\\
387	0\\
388	0\\
389	0\\
390	0\\
391	0\\
392	0\\
393	0\\
394	0\\
395	0\\
396	0\\
397	0\\
398	0\\
399	0\\
400	0\\
401	0\\
402	0\\
403	0\\
404	0\\
405	0\\
406	0\\
407	0\\
408	0\\
409	0\\
410	0\\
411	0\\
412	0\\
413	0\\
414	0\\
415	0\\
416	0\\
417	0\\
418	0\\
419	0\\
420	0\\
421	0\\
422	0\\
423	0\\
424	0\\
425	0\\
426	0\\
427	0\\
428	0\\
429	0\\
430	0\\
431	0\\
432	0\\
433	0\\
434	0\\
435	0\\
436	0\\
437	0\\
438	0\\
439	0\\
440	0\\
441	0\\
442	0\\
443	0\\
444	0\\
445	0\\
446	0\\
447	0\\
448	0\\
449	0\\
450	0\\
451	0\\
452	0\\
453	0\\
454	0\\
455	0\\
456	0\\
457	0\\
458	0\\
459	0\\
460	0\\
461	0\\
462	0\\
463	0\\
464	0\\
465	0\\
466	0\\
467	0\\
468	0\\
469	0\\
470	0\\
471	0\\
472	0\\
473	0\\
474	0\\
475	0\\
476	0\\
477	0\\
478	0\\
479	0\\
480	0\\
481	0\\
482	0\\
483	0\\
484	0\\
485	0\\
486	0\\
487	0\\
488	0\\
489	0\\
490	0\\
491	0\\
492	0\\
493	0\\
494	0\\
495	0\\
496	0\\
497	0\\
498	0\\
499	0\\
500	0\\
501	0\\
502	0\\
503	0\\
504	0\\
505	0\\
506	0\\
507	0\\
508	0\\
509	0\\
510	0\\
511	0\\
512	0\\
513	0\\
514	0\\
515	0\\
516	0\\
517	0\\
518	0\\
519	0\\
520	0\\
521	0\\
522	0\\
523	0\\
524	0\\
525	0\\
526	0\\
527	0\\
528	0\\
529	0\\
530	0\\
531	0\\
532	0\\
533	0\\
534	0\\
535	0\\
536	0\\
537	0\\
538	0\\
539	0\\
540	0\\
541	0\\
542	0\\
543	0\\
544	0\\
545	0\\
546	0\\
547	0\\
548	0\\
549	0\\
550	0\\
551	0\\
552	1.50670897009176e-05\\
553	0.000124382787751889\\
554	0.00020503601022344\\
555	0.000281906524719583\\
556	0.000353475266029824\\
557	0.000416928686445217\\
558	0.000471622760274698\\
559	0.000521843264876268\\
560	0.000570759546781793\\
561	0.000618739159396117\\
562	0.000666121145496063\\
563	0.000713768966029745\\
564	0.000763815747326926\\
565	0.0008137163277131\\
566	0.000863511816317939\\
567	0.000912037857953922\\
568	0.000959785906419871\\
569	0.00100821347014938\\
570	0.00105737215001796\\
571	0.00110730055108194\\
572	0.00116093038778452\\
573	0.00121697978948583\\
574	0.00127198613892506\\
575	0.00132150660081605\\
576	0.00137146061891735\\
577	0.00142187466526321\\
578	0.00180758151954677\\
579	0.00232459949440059\\
580	0.00250414369719064\\
581	0.00261607079037986\\
582	0.00270221271091265\\
583	0.00278105910457229\\
584	0.00286069531223358\\
585	0.00294160817599962\\
586	0.00302387110985324\\
587	0.00310754666812213\\
588	0.00319273161070454\\
589	0.00327961396220161\\
590	0.00336862207624017\\
591	0.00346039677087353\\
592	0.00355653402929194\\
593	0.00366119131364763\\
594	0.00378527046594783\\
595	0.00395742119285503\\
596	0.00425297862742962\\
597	0.0048700418989439\\
598	0.00632942537858856\\
599	0\\
600	0\\
};
\addplot [color=red!50!mycolor17,solid,forget plot]
  table[row sep=crcr]{%
1	0\\
2	0\\
3	0\\
4	0\\
5	0\\
6	0\\
7	0\\
8	0\\
9	0\\
10	0\\
11	0\\
12	0\\
13	0\\
14	0\\
15	0\\
16	0\\
17	0\\
18	0\\
19	0\\
20	0\\
21	0\\
22	0\\
23	0\\
24	0\\
25	0\\
26	0\\
27	0\\
28	0\\
29	0\\
30	0\\
31	0\\
32	0\\
33	0\\
34	0\\
35	0\\
36	0\\
37	0\\
38	0\\
39	0\\
40	0\\
41	0\\
42	0\\
43	0\\
44	0\\
45	0\\
46	0\\
47	0\\
48	0\\
49	0\\
50	0\\
51	0\\
52	0\\
53	0\\
54	0\\
55	0\\
56	0\\
57	0\\
58	0\\
59	0\\
60	0\\
61	0\\
62	0\\
63	0\\
64	0\\
65	0\\
66	0\\
67	0\\
68	0\\
69	0\\
70	0\\
71	0\\
72	0\\
73	0\\
74	0\\
75	0\\
76	0\\
77	0\\
78	0\\
79	0\\
80	0\\
81	0\\
82	0\\
83	0\\
84	0\\
85	0\\
86	0\\
87	0\\
88	0\\
89	0\\
90	0\\
91	0\\
92	0\\
93	0\\
94	0\\
95	0\\
96	0\\
97	0\\
98	0\\
99	0\\
100	0\\
101	0\\
102	0\\
103	0\\
104	0\\
105	0\\
106	0\\
107	0\\
108	0\\
109	0\\
110	0\\
111	0\\
112	0\\
113	0\\
114	0\\
115	0\\
116	0\\
117	0\\
118	0\\
119	0\\
120	0\\
121	0\\
122	0\\
123	0\\
124	0\\
125	0\\
126	0\\
127	0\\
128	0\\
129	0\\
130	0\\
131	0\\
132	0\\
133	0\\
134	0\\
135	0\\
136	0\\
137	0\\
138	0\\
139	0\\
140	0\\
141	0\\
142	0\\
143	0\\
144	0\\
145	0\\
146	0\\
147	0\\
148	0\\
149	0\\
150	0\\
151	0\\
152	0\\
153	0\\
154	0\\
155	0\\
156	0\\
157	0\\
158	0\\
159	0\\
160	0\\
161	0\\
162	0\\
163	0\\
164	0\\
165	0\\
166	0\\
167	0\\
168	0\\
169	0\\
170	0\\
171	0\\
172	0\\
173	0\\
174	0\\
175	0\\
176	0\\
177	0\\
178	0\\
179	0\\
180	0\\
181	0\\
182	0\\
183	0\\
184	0\\
185	0\\
186	0\\
187	0\\
188	0\\
189	0\\
190	0\\
191	0\\
192	0\\
193	0\\
194	0\\
195	0\\
196	0\\
197	0\\
198	0\\
199	0\\
200	0\\
201	0\\
202	0\\
203	0\\
204	0\\
205	0\\
206	0\\
207	0\\
208	0\\
209	0\\
210	0\\
211	0\\
212	0\\
213	0\\
214	0\\
215	0\\
216	0\\
217	0\\
218	0\\
219	0\\
220	0\\
221	0\\
222	0\\
223	0\\
224	0\\
225	0\\
226	0\\
227	0\\
228	0\\
229	0\\
230	0\\
231	0\\
232	0\\
233	0\\
234	0\\
235	0\\
236	0\\
237	0\\
238	0\\
239	0\\
240	0\\
241	0\\
242	0\\
243	0\\
244	0\\
245	0\\
246	0\\
247	0\\
248	0\\
249	0\\
250	0\\
251	0\\
252	0\\
253	0\\
254	0\\
255	0\\
256	0\\
257	0\\
258	0\\
259	0\\
260	0\\
261	0\\
262	0\\
263	0\\
264	0\\
265	0\\
266	0\\
267	0\\
268	0\\
269	0\\
270	0\\
271	0\\
272	0\\
273	0\\
274	0\\
275	0\\
276	0\\
277	0\\
278	0\\
279	0\\
280	0\\
281	0\\
282	0\\
283	0\\
284	0\\
285	0\\
286	0\\
287	0\\
288	0\\
289	0\\
290	0\\
291	0\\
292	0\\
293	0\\
294	0\\
295	0\\
296	0\\
297	0\\
298	0\\
299	0\\
300	0\\
301	0\\
302	0\\
303	0\\
304	0\\
305	0\\
306	0\\
307	0\\
308	0\\
309	0\\
310	0\\
311	0\\
312	0\\
313	0\\
314	0\\
315	0\\
316	0\\
317	0\\
318	0\\
319	0\\
320	0\\
321	0\\
322	0\\
323	0\\
324	0\\
325	0\\
326	0\\
327	0\\
328	0\\
329	0\\
330	0\\
331	0\\
332	0\\
333	0\\
334	0\\
335	0\\
336	0\\
337	0\\
338	0\\
339	0\\
340	0\\
341	0\\
342	0\\
343	0\\
344	0\\
345	0\\
346	0\\
347	0\\
348	0\\
349	0\\
350	0\\
351	0\\
352	0\\
353	0\\
354	0\\
355	0\\
356	0\\
357	0\\
358	0\\
359	0\\
360	0\\
361	0\\
362	0\\
363	0\\
364	0\\
365	0\\
366	0\\
367	0\\
368	0\\
369	0\\
370	0\\
371	0\\
372	0\\
373	0\\
374	0\\
375	0\\
376	0\\
377	0\\
378	0\\
379	0\\
380	0\\
381	0\\
382	0\\
383	0\\
384	0\\
385	0\\
386	0\\
387	0\\
388	0\\
389	0\\
390	0\\
391	0\\
392	0\\
393	0\\
394	0\\
395	0\\
396	0\\
397	0\\
398	0\\
399	0\\
400	0\\
401	0\\
402	0\\
403	0\\
404	0\\
405	0\\
406	0\\
407	0\\
408	0\\
409	0\\
410	0\\
411	0\\
412	0\\
413	0\\
414	0\\
415	0\\
416	0\\
417	0\\
418	0\\
419	0\\
420	0\\
421	0\\
422	0\\
423	0\\
424	0\\
425	0\\
426	0\\
427	0\\
428	0\\
429	0\\
430	0\\
431	0\\
432	0\\
433	0\\
434	0\\
435	0\\
436	0\\
437	0\\
438	0\\
439	0\\
440	0\\
441	0\\
442	0\\
443	0\\
444	0\\
445	0\\
446	0\\
447	0\\
448	0\\
449	0\\
450	0\\
451	0\\
452	0\\
453	0\\
454	0\\
455	0\\
456	0\\
457	0\\
458	0\\
459	0\\
460	0\\
461	0\\
462	0\\
463	0\\
464	0\\
465	0\\
466	0\\
467	0\\
468	0\\
469	0\\
470	0\\
471	0\\
472	0\\
473	0\\
474	0\\
475	0\\
476	0\\
477	0\\
478	0\\
479	0\\
480	0\\
481	0\\
482	0\\
483	0\\
484	0\\
485	0\\
486	0\\
487	0\\
488	0\\
489	0\\
490	0\\
491	0\\
492	0\\
493	0\\
494	0\\
495	0\\
496	0\\
497	0\\
498	0\\
499	0\\
500	0\\
501	0\\
502	0\\
503	0\\
504	0\\
505	0\\
506	0\\
507	0\\
508	0\\
509	0\\
510	0\\
511	0\\
512	0\\
513	0\\
514	0\\
515	0\\
516	0\\
517	0\\
518	0\\
519	0\\
520	0\\
521	0\\
522	0\\
523	0\\
524	0\\
525	0\\
526	0\\
527	0\\
528	0\\
529	0\\
530	0\\
531	0\\
532	0\\
533	0\\
534	0\\
535	0\\
536	0\\
537	0\\
538	0\\
539	0\\
540	0\\
541	0\\
542	0\\
543	0\\
544	0\\
545	0\\
546	0\\
547	0\\
548	1.64327239660599e-05\\
549	9.08311762122886e-05\\
550	0.000161022938856363\\
551	0.000224806955478363\\
552	0.000280485308906282\\
553	0.000327622000208862\\
554	0.000372012735657423\\
555	0.000415398793616001\\
556	0.000457922393037952\\
557	0.000499893593165752\\
558	0.000541753390275113\\
559	0.000584113654120537\\
560	0.000628623132028616\\
561	0.000674257201963313\\
562	0.000719918679057545\\
563	0.000765234608134834\\
564	0.000808741945395788\\
565	0.000852912506720194\\
566	0.00089779277058568\\
567	0.000943386792557535\\
568	0.000989813512646882\\
569	0.00103725286168404\\
570	0.00109022682293433\\
571	0.00114341698775839\\
572	0.00119302706840499\\
573	0.00124014217818203\\
574	0.00128770080075513\\
575	0.00133565715457315\\
576	0.00160918154831211\\
577	0.00213608599489938\\
578	0.00235964274414307\\
579	0.00246903551743822\\
580	0.00255013683265288\\
581	0.00262610814362027\\
582	0.00270308299221297\\
583	0.00278130768962817\\
584	0.00286082862410346\\
585	0.00294168211648289\\
586	0.00302390908465139\\
587	0.003107564358473\\
588	0.00319273875713431\\
589	0.00327961632359304\\
590	0.00336862264580875\\
591	0.00346039684658503\\
592	0.00355653402929193\\
593	0.00366119131364763\\
594	0.00378527046594782\\
595	0.00395742119285503\\
596	0.00425297862742962\\
597	0.0048700418989439\\
598	0.00632942537858856\\
599	0\\
600	0\\
};
\addplot [color=red!40!mycolor19,solid,forget plot]
  table[row sep=crcr]{%
1	0\\
2	0\\
3	0\\
4	0\\
5	0\\
6	0\\
7	0\\
8	0\\
9	0\\
10	0\\
11	0\\
12	0\\
13	0\\
14	0\\
15	0\\
16	0\\
17	0\\
18	0\\
19	0\\
20	0\\
21	0\\
22	0\\
23	0\\
24	0\\
25	0\\
26	0\\
27	0\\
28	0\\
29	0\\
30	0\\
31	0\\
32	0\\
33	0\\
34	0\\
35	0\\
36	0\\
37	0\\
38	0\\
39	0\\
40	0\\
41	0\\
42	0\\
43	0\\
44	0\\
45	0\\
46	0\\
47	0\\
48	0\\
49	0\\
50	0\\
51	0\\
52	0\\
53	0\\
54	0\\
55	0\\
56	0\\
57	0\\
58	0\\
59	0\\
60	0\\
61	0\\
62	0\\
63	0\\
64	0\\
65	0\\
66	0\\
67	0\\
68	0\\
69	0\\
70	0\\
71	0\\
72	0\\
73	0\\
74	0\\
75	0\\
76	0\\
77	0\\
78	0\\
79	0\\
80	0\\
81	0\\
82	0\\
83	0\\
84	0\\
85	0\\
86	0\\
87	0\\
88	0\\
89	0\\
90	0\\
91	0\\
92	0\\
93	0\\
94	0\\
95	0\\
96	0\\
97	0\\
98	0\\
99	0\\
100	0\\
101	0\\
102	0\\
103	0\\
104	0\\
105	0\\
106	0\\
107	0\\
108	0\\
109	0\\
110	0\\
111	0\\
112	0\\
113	0\\
114	0\\
115	0\\
116	0\\
117	0\\
118	0\\
119	0\\
120	0\\
121	0\\
122	0\\
123	0\\
124	0\\
125	0\\
126	0\\
127	0\\
128	0\\
129	0\\
130	0\\
131	0\\
132	0\\
133	0\\
134	0\\
135	0\\
136	0\\
137	0\\
138	0\\
139	0\\
140	0\\
141	0\\
142	0\\
143	0\\
144	0\\
145	0\\
146	0\\
147	0\\
148	0\\
149	0\\
150	0\\
151	0\\
152	0\\
153	0\\
154	0\\
155	0\\
156	0\\
157	0\\
158	0\\
159	0\\
160	0\\
161	0\\
162	0\\
163	0\\
164	0\\
165	0\\
166	0\\
167	0\\
168	0\\
169	0\\
170	0\\
171	0\\
172	0\\
173	0\\
174	0\\
175	0\\
176	0\\
177	0\\
178	0\\
179	0\\
180	0\\
181	0\\
182	0\\
183	0\\
184	0\\
185	0\\
186	0\\
187	0\\
188	0\\
189	0\\
190	0\\
191	0\\
192	0\\
193	0\\
194	0\\
195	0\\
196	0\\
197	0\\
198	0\\
199	0\\
200	0\\
201	0\\
202	0\\
203	0\\
204	0\\
205	0\\
206	0\\
207	0\\
208	0\\
209	0\\
210	0\\
211	0\\
212	0\\
213	0\\
214	0\\
215	0\\
216	0\\
217	0\\
218	0\\
219	0\\
220	0\\
221	0\\
222	0\\
223	0\\
224	0\\
225	0\\
226	0\\
227	0\\
228	0\\
229	0\\
230	0\\
231	0\\
232	0\\
233	0\\
234	0\\
235	0\\
236	0\\
237	0\\
238	0\\
239	0\\
240	0\\
241	0\\
242	0\\
243	0\\
244	0\\
245	0\\
246	0\\
247	0\\
248	0\\
249	0\\
250	0\\
251	0\\
252	0\\
253	0\\
254	0\\
255	0\\
256	0\\
257	0\\
258	0\\
259	0\\
260	0\\
261	0\\
262	0\\
263	0\\
264	0\\
265	0\\
266	0\\
267	0\\
268	0\\
269	0\\
270	0\\
271	0\\
272	0\\
273	0\\
274	0\\
275	0\\
276	0\\
277	0\\
278	0\\
279	0\\
280	0\\
281	0\\
282	0\\
283	0\\
284	0\\
285	0\\
286	0\\
287	0\\
288	0\\
289	0\\
290	0\\
291	0\\
292	0\\
293	0\\
294	0\\
295	0\\
296	0\\
297	0\\
298	0\\
299	0\\
300	0\\
301	0\\
302	0\\
303	0\\
304	0\\
305	0\\
306	0\\
307	0\\
308	0\\
309	0\\
310	0\\
311	0\\
312	0\\
313	0\\
314	0\\
315	0\\
316	0\\
317	0\\
318	0\\
319	0\\
320	0\\
321	0\\
322	0\\
323	0\\
324	0\\
325	0\\
326	0\\
327	0\\
328	0\\
329	0\\
330	0\\
331	0\\
332	0\\
333	0\\
334	0\\
335	0\\
336	0\\
337	0\\
338	0\\
339	0\\
340	0\\
341	0\\
342	0\\
343	0\\
344	0\\
345	0\\
346	0\\
347	0\\
348	0\\
349	0\\
350	0\\
351	0\\
352	0\\
353	0\\
354	0\\
355	0\\
356	0\\
357	0\\
358	0\\
359	0\\
360	0\\
361	0\\
362	0\\
363	0\\
364	0\\
365	0\\
366	0\\
367	0\\
368	0\\
369	0\\
370	0\\
371	0\\
372	0\\
373	0\\
374	0\\
375	0\\
376	0\\
377	0\\
378	0\\
379	0\\
380	0\\
381	0\\
382	0\\
383	0\\
384	0\\
385	0\\
386	0\\
387	0\\
388	0\\
389	0\\
390	0\\
391	0\\
392	0\\
393	0\\
394	0\\
395	0\\
396	0\\
397	0\\
398	0\\
399	0\\
400	0\\
401	0\\
402	0\\
403	0\\
404	0\\
405	0\\
406	0\\
407	0\\
408	0\\
409	0\\
410	0\\
411	0\\
412	0\\
413	0\\
414	0\\
415	0\\
416	0\\
417	0\\
418	0\\
419	0\\
420	0\\
421	0\\
422	0\\
423	0\\
424	0\\
425	0\\
426	0\\
427	0\\
428	0\\
429	0\\
430	0\\
431	0\\
432	0\\
433	0\\
434	0\\
435	0\\
436	0\\
437	0\\
438	0\\
439	0\\
440	0\\
441	0\\
442	0\\
443	0\\
444	0\\
445	0\\
446	0\\
447	0\\
448	0\\
449	0\\
450	0\\
451	0\\
452	0\\
453	0\\
454	0\\
455	0\\
456	0\\
457	0\\
458	0\\
459	0\\
460	0\\
461	0\\
462	0\\
463	0\\
464	0\\
465	0\\
466	0\\
467	0\\
468	0\\
469	0\\
470	0\\
471	0\\
472	0\\
473	0\\
474	0\\
475	0\\
476	0\\
477	0\\
478	0\\
479	0\\
480	0\\
481	0\\
482	0\\
483	0\\
484	0\\
485	0\\
486	0\\
487	0\\
488	0\\
489	0\\
490	0\\
491	0\\
492	0\\
493	0\\
494	0\\
495	0\\
496	0\\
497	0\\
498	0\\
499	0\\
500	0\\
501	0\\
502	0\\
503	0\\
504	0\\
505	0\\
506	0\\
507	0\\
508	0\\
509	0\\
510	0\\
511	0\\
512	0\\
513	0\\
514	0\\
515	0\\
516	0\\
517	0\\
518	0\\
519	0\\
520	0\\
521	0\\
522	0\\
523	0\\
524	0\\
525	0\\
526	0\\
527	0\\
528	0\\
529	0\\
530	0\\
531	0\\
532	0\\
533	0\\
534	0\\
535	0\\
536	0\\
537	0\\
538	0\\
539	0\\
540	0\\
541	0\\
542	0\\
543	0\\
544	0\\
545	3.82391076137234e-05\\
546	9.68023725063791e-05\\
547	0.000147720982546101\\
548	0.000190142250714629\\
549	0.00022979867141449\\
550	0.000268454303467937\\
551	0.000306284037234123\\
552	0.00034374680041995\\
553	0.000381433047254094\\
554	0.000419542603462078\\
555	0.000458172315969797\\
556	0.000497380378338157\\
557	0.00053955610927823\\
558	0.000581988250460856\\
559	0.000624481094669423\\
560	0.000665506456059995\\
561	0.000706089745012563\\
562	0.000747338613311647\\
563	0.000789267713198676\\
564	0.000831888557103819\\
565	0.000875412308035493\\
566	0.000919872431348538\\
567	0.000965414080806018\\
568	0.00101726491970261\\
569	0.00106823974206828\\
570	0.00111385344432899\\
571	0.00115893121983135\\
572	0.00120437493685983\\
573	0.00125042703199772\\
574	0.00134080429263664\\
575	0.00187574822298428\\
576	0.00221187192817254\\
577	0.00232010641060596\\
578	0.00240256038064571\\
579	0.00247601921057826\\
580	0.00255050068707274\\
581	0.00262619072506652\\
582	0.00270312408738541\\
583	0.00278133051979913\\
584	0.00286084071525918\\
585	0.00294168800565563\\
586	0.00302391165742733\\
587	0.00310756532765526\\
588	0.00319273905359195\\
589	0.00327961638930458\\
590	0.00336862265377153\\
591	0.00346039684658503\\
592	0.00355653402929194\\
593	0.00366119131364763\\
594	0.00378527046594783\\
595	0.00395742119285503\\
596	0.00425297862742962\\
597	0.0048700418989439\\
598	0.00632942537858856\\
599	0\\
600	0\\
};
\addplot [color=red!75!mycolor17,solid,forget plot]
  table[row sep=crcr]{%
1	0\\
2	0\\
3	0\\
4	0\\
5	0\\
6	0\\
7	0\\
8	0\\
9	0\\
10	0\\
11	0\\
12	0\\
13	0\\
14	0\\
15	0\\
16	0\\
17	0\\
18	0\\
19	0\\
20	0\\
21	0\\
22	0\\
23	0\\
24	0\\
25	0\\
26	0\\
27	0\\
28	0\\
29	0\\
30	0\\
31	0\\
32	0\\
33	0\\
34	0\\
35	0\\
36	0\\
37	0\\
38	0\\
39	0\\
40	0\\
41	0\\
42	0\\
43	0\\
44	0\\
45	0\\
46	0\\
47	0\\
48	0\\
49	0\\
50	0\\
51	0\\
52	0\\
53	0\\
54	0\\
55	0\\
56	0\\
57	0\\
58	0\\
59	0\\
60	0\\
61	0\\
62	0\\
63	0\\
64	0\\
65	0\\
66	0\\
67	0\\
68	0\\
69	0\\
70	0\\
71	0\\
72	0\\
73	0\\
74	0\\
75	0\\
76	0\\
77	0\\
78	0\\
79	0\\
80	0\\
81	0\\
82	0\\
83	0\\
84	0\\
85	0\\
86	0\\
87	0\\
88	0\\
89	0\\
90	0\\
91	0\\
92	0\\
93	0\\
94	0\\
95	0\\
96	0\\
97	0\\
98	0\\
99	0\\
100	0\\
101	0\\
102	0\\
103	0\\
104	0\\
105	0\\
106	0\\
107	0\\
108	0\\
109	0\\
110	0\\
111	0\\
112	0\\
113	0\\
114	0\\
115	0\\
116	0\\
117	0\\
118	0\\
119	0\\
120	0\\
121	0\\
122	0\\
123	0\\
124	0\\
125	0\\
126	0\\
127	0\\
128	0\\
129	0\\
130	0\\
131	0\\
132	0\\
133	0\\
134	0\\
135	0\\
136	0\\
137	0\\
138	0\\
139	0\\
140	0\\
141	0\\
142	0\\
143	0\\
144	0\\
145	0\\
146	0\\
147	0\\
148	0\\
149	0\\
150	0\\
151	0\\
152	0\\
153	0\\
154	0\\
155	0\\
156	0\\
157	0\\
158	0\\
159	0\\
160	0\\
161	0\\
162	0\\
163	0\\
164	0\\
165	0\\
166	0\\
167	0\\
168	0\\
169	0\\
170	0\\
171	0\\
172	0\\
173	0\\
174	0\\
175	0\\
176	0\\
177	0\\
178	0\\
179	0\\
180	0\\
181	0\\
182	0\\
183	0\\
184	0\\
185	0\\
186	0\\
187	0\\
188	0\\
189	0\\
190	0\\
191	0\\
192	0\\
193	0\\
194	0\\
195	0\\
196	0\\
197	0\\
198	0\\
199	0\\
200	0\\
201	0\\
202	0\\
203	0\\
204	0\\
205	0\\
206	0\\
207	0\\
208	0\\
209	0\\
210	0\\
211	0\\
212	0\\
213	0\\
214	0\\
215	0\\
216	0\\
217	0\\
218	0\\
219	0\\
220	0\\
221	0\\
222	0\\
223	0\\
224	0\\
225	0\\
226	0\\
227	0\\
228	0\\
229	0\\
230	0\\
231	0\\
232	0\\
233	0\\
234	0\\
235	0\\
236	0\\
237	0\\
238	0\\
239	0\\
240	0\\
241	0\\
242	0\\
243	0\\
244	0\\
245	0\\
246	0\\
247	0\\
248	0\\
249	0\\
250	0\\
251	0\\
252	0\\
253	0\\
254	0\\
255	0\\
256	0\\
257	0\\
258	0\\
259	0\\
260	0\\
261	0\\
262	0\\
263	0\\
264	0\\
265	0\\
266	0\\
267	0\\
268	0\\
269	0\\
270	0\\
271	0\\
272	0\\
273	0\\
274	0\\
275	0\\
276	0\\
277	0\\
278	0\\
279	0\\
280	0\\
281	0\\
282	0\\
283	0\\
284	0\\
285	0\\
286	0\\
287	0\\
288	0\\
289	0\\
290	0\\
291	0\\
292	0\\
293	0\\
294	0\\
295	0\\
296	0\\
297	0\\
298	0\\
299	0\\
300	0\\
301	0\\
302	0\\
303	0\\
304	0\\
305	0\\
306	0\\
307	0\\
308	0\\
309	0\\
310	0\\
311	0\\
312	0\\
313	0\\
314	0\\
315	0\\
316	0\\
317	0\\
318	0\\
319	0\\
320	0\\
321	0\\
322	0\\
323	0\\
324	0\\
325	0\\
326	0\\
327	0\\
328	0\\
329	0\\
330	0\\
331	0\\
332	0\\
333	0\\
334	0\\
335	0\\
336	0\\
337	0\\
338	0\\
339	0\\
340	0\\
341	0\\
342	0\\
343	0\\
344	0\\
345	0\\
346	0\\
347	0\\
348	0\\
349	0\\
350	0\\
351	0\\
352	0\\
353	0\\
354	0\\
355	0\\
356	0\\
357	0\\
358	0\\
359	0\\
360	0\\
361	0\\
362	0\\
363	0\\
364	0\\
365	0\\
366	0\\
367	0\\
368	0\\
369	0\\
370	0\\
371	0\\
372	0\\
373	0\\
374	0\\
375	0\\
376	0\\
377	0\\
378	0\\
379	0\\
380	0\\
381	0\\
382	0\\
383	0\\
384	0\\
385	0\\
386	0\\
387	0\\
388	0\\
389	0\\
390	0\\
391	0\\
392	0\\
393	0\\
394	0\\
395	0\\
396	0\\
397	0\\
398	0\\
399	0\\
400	0\\
401	0\\
402	0\\
403	0\\
404	0\\
405	0\\
406	0\\
407	0\\
408	0\\
409	0\\
410	0\\
411	0\\
412	0\\
413	0\\
414	0\\
415	0\\
416	0\\
417	0\\
418	0\\
419	0\\
420	0\\
421	0\\
422	0\\
423	0\\
424	0\\
425	0\\
426	0\\
427	0\\
428	0\\
429	0\\
430	0\\
431	0\\
432	0\\
433	0\\
434	0\\
435	0\\
436	0\\
437	0\\
438	0\\
439	0\\
440	0\\
441	0\\
442	0\\
443	0\\
444	0\\
445	0\\
446	0\\
447	0\\
448	0\\
449	0\\
450	0\\
451	0\\
452	0\\
453	0\\
454	0\\
455	0\\
456	0\\
457	0\\
458	0\\
459	0\\
460	0\\
461	0\\
462	0\\
463	0\\
464	0\\
465	0\\
466	0\\
467	0\\
468	0\\
469	0\\
470	0\\
471	0\\
472	0\\
473	0\\
474	0\\
475	0\\
476	0\\
477	0\\
478	0\\
479	0\\
480	0\\
481	0\\
482	0\\
483	0\\
484	0\\
485	0\\
486	0\\
487	0\\
488	0\\
489	0\\
490	0\\
491	0\\
492	0\\
493	0\\
494	0\\
495	0\\
496	0\\
497	0\\
498	0\\
499	0\\
500	0\\
501	0\\
502	0\\
503	0\\
504	0\\
505	0\\
506	0\\
507	0\\
508	0\\
509	0\\
510	0\\
511	0\\
512	0\\
513	0\\
514	0\\
515	0\\
516	0\\
517	0\\
518	0\\
519	0\\
520	0\\
521	0\\
522	0\\
523	0\\
524	0\\
525	0\\
526	0\\
527	0\\
528	0\\
529	0\\
530	0\\
531	0\\
532	0\\
533	0\\
534	0\\
535	0\\
536	0\\
537	0\\
538	0\\
539	0\\
540	0\\
541	0\\
542	2.04853442762617e-05\\
543	6.06220385510295e-05\\
544	9.64802235070158e-05\\
545	0.000131320163008711\\
546	0.000165361460811259\\
547	0.000199053503989163\\
548	0.000232946229837191\\
549	0.00026724254469209\\
550	0.000302038372857046\\
551	0.000337433066059789\\
552	0.000373524405103253\\
553	0.000410607915064151\\
554	0.000450645605495247\\
555	0.000490655083798201\\
556	0.000530561602653835\\
557	0.000568219493923374\\
558	0.000606305612849557\\
559	0.000645024893172913\\
560	0.000684358053097023\\
561	0.000724443518312681\\
562	0.000765389047441491\\
563	0.000807225433839359\\
564	0.000849982974356452\\
565	0.000893867927586915\\
566	0.000944065882789967\\
567	0.000993239584238119\\
568	0.00103619979679305\\
569	0.00107953810792023\\
570	0.00112318247816187\\
571	0.00116757410108689\\
572	0.00121282583955014\\
573	0.00154610456066244\\
574	0.00206173146214074\\
575	0.00217000160705326\\
576	0.00225943058102288\\
577	0.00233065932806681\\
578	0.00240277285646767\\
579	0.00247604740747313\\
580	0.00255051340445631\\
581	0.00262619770980058\\
582	0.00270312784773939\\
583	0.00278133241566772\\
584	0.00286084158979877\\
585	0.00294168836537039\\
586	0.00302391178438079\\
587	0.00310756536383239\\
588	0.00319273906101823\\
589	0.00327961639013216\\
590	0.00336862265377154\\
591	0.00346039684658504\\
592	0.00355653402929194\\
593	0.00366119131364762\\
594	0.00378527046594782\\
595	0.00395742119285503\\
596	0.00425297862742962\\
597	0.0048700418989439\\
598	0.00632942537858856\\
599	0\\
600	0\\
};
\addplot [color=red!80!mycolor19,solid,forget plot]
  table[row sep=crcr]{%
1	0\\
2	0\\
3	0\\
4	0\\
5	0\\
6	0\\
7	0\\
8	0\\
9	0\\
10	0\\
11	0\\
12	0\\
13	0\\
14	0\\
15	0\\
16	0\\
17	0\\
18	0\\
19	0\\
20	0\\
21	0\\
22	0\\
23	0\\
24	0\\
25	0\\
26	0\\
27	0\\
28	0\\
29	0\\
30	0\\
31	0\\
32	0\\
33	0\\
34	0\\
35	0\\
36	0\\
37	0\\
38	0\\
39	0\\
40	0\\
41	0\\
42	0\\
43	0\\
44	0\\
45	0\\
46	0\\
47	0\\
48	0\\
49	0\\
50	0\\
51	0\\
52	0\\
53	0\\
54	0\\
55	0\\
56	0\\
57	0\\
58	0\\
59	0\\
60	0\\
61	0\\
62	0\\
63	0\\
64	0\\
65	0\\
66	0\\
67	0\\
68	0\\
69	0\\
70	0\\
71	0\\
72	0\\
73	0\\
74	0\\
75	0\\
76	0\\
77	0\\
78	0\\
79	0\\
80	0\\
81	0\\
82	0\\
83	0\\
84	0\\
85	0\\
86	0\\
87	0\\
88	0\\
89	0\\
90	0\\
91	0\\
92	0\\
93	0\\
94	0\\
95	0\\
96	0\\
97	0\\
98	0\\
99	0\\
100	0\\
101	0\\
102	0\\
103	0\\
104	0\\
105	0\\
106	0\\
107	0\\
108	0\\
109	0\\
110	0\\
111	0\\
112	0\\
113	0\\
114	0\\
115	0\\
116	0\\
117	0\\
118	0\\
119	0\\
120	0\\
121	0\\
122	0\\
123	0\\
124	0\\
125	0\\
126	0\\
127	0\\
128	0\\
129	0\\
130	0\\
131	0\\
132	0\\
133	0\\
134	0\\
135	0\\
136	0\\
137	0\\
138	0\\
139	0\\
140	0\\
141	0\\
142	0\\
143	0\\
144	0\\
145	0\\
146	0\\
147	0\\
148	0\\
149	0\\
150	0\\
151	0\\
152	0\\
153	0\\
154	0\\
155	0\\
156	0\\
157	0\\
158	0\\
159	0\\
160	0\\
161	0\\
162	0\\
163	0\\
164	0\\
165	0\\
166	0\\
167	0\\
168	0\\
169	0\\
170	0\\
171	0\\
172	0\\
173	0\\
174	0\\
175	0\\
176	0\\
177	0\\
178	0\\
179	0\\
180	0\\
181	0\\
182	0\\
183	0\\
184	0\\
185	0\\
186	0\\
187	0\\
188	0\\
189	0\\
190	0\\
191	0\\
192	0\\
193	0\\
194	0\\
195	0\\
196	0\\
197	0\\
198	0\\
199	0\\
200	0\\
201	0\\
202	0\\
203	0\\
204	0\\
205	0\\
206	0\\
207	0\\
208	0\\
209	0\\
210	0\\
211	0\\
212	0\\
213	0\\
214	0\\
215	0\\
216	0\\
217	0\\
218	0\\
219	0\\
220	0\\
221	0\\
222	0\\
223	0\\
224	0\\
225	0\\
226	0\\
227	0\\
228	0\\
229	0\\
230	0\\
231	0\\
232	0\\
233	0\\
234	0\\
235	0\\
236	0\\
237	0\\
238	0\\
239	0\\
240	0\\
241	0\\
242	0\\
243	0\\
244	0\\
245	0\\
246	0\\
247	0\\
248	0\\
249	0\\
250	0\\
251	0\\
252	0\\
253	0\\
254	0\\
255	0\\
256	0\\
257	0\\
258	0\\
259	0\\
260	0\\
261	0\\
262	0\\
263	0\\
264	0\\
265	0\\
266	0\\
267	0\\
268	0\\
269	0\\
270	0\\
271	0\\
272	0\\
273	0\\
274	0\\
275	0\\
276	0\\
277	0\\
278	0\\
279	0\\
280	0\\
281	0\\
282	0\\
283	0\\
284	0\\
285	0\\
286	0\\
287	0\\
288	0\\
289	0\\
290	0\\
291	0\\
292	0\\
293	0\\
294	0\\
295	0\\
296	0\\
297	0\\
298	0\\
299	0\\
300	0\\
301	0\\
302	0\\
303	0\\
304	0\\
305	0\\
306	0\\
307	0\\
308	0\\
309	0\\
310	0\\
311	0\\
312	0\\
313	0\\
314	0\\
315	0\\
316	0\\
317	0\\
318	0\\
319	0\\
320	0\\
321	0\\
322	0\\
323	0\\
324	0\\
325	0\\
326	0\\
327	0\\
328	0\\
329	0\\
330	0\\
331	0\\
332	0\\
333	0\\
334	0\\
335	0\\
336	0\\
337	0\\
338	0\\
339	0\\
340	0\\
341	0\\
342	0\\
343	0\\
344	0\\
345	0\\
346	0\\
347	0\\
348	0\\
349	0\\
350	0\\
351	0\\
352	0\\
353	0\\
354	0\\
355	0\\
356	0\\
357	0\\
358	0\\
359	0\\
360	0\\
361	0\\
362	0\\
363	0\\
364	0\\
365	0\\
366	0\\
367	0\\
368	0\\
369	0\\
370	0\\
371	0\\
372	0\\
373	0\\
374	0\\
375	0\\
376	0\\
377	0\\
378	0\\
379	0\\
380	0\\
381	0\\
382	0\\
383	0\\
384	0\\
385	0\\
386	0\\
387	0\\
388	0\\
389	0\\
390	0\\
391	0\\
392	0\\
393	0\\
394	0\\
395	0\\
396	0\\
397	0\\
398	0\\
399	0\\
400	0\\
401	0\\
402	0\\
403	0\\
404	0\\
405	0\\
406	0\\
407	0\\
408	0\\
409	0\\
410	0\\
411	0\\
412	0\\
413	0\\
414	0\\
415	0\\
416	0\\
417	0\\
418	0\\
419	0\\
420	0\\
421	0\\
422	0\\
423	0\\
424	0\\
425	0\\
426	0\\
427	0\\
428	0\\
429	0\\
430	0\\
431	0\\
432	0\\
433	0\\
434	0\\
435	0\\
436	0\\
437	0\\
438	0\\
439	0\\
440	0\\
441	0\\
442	0\\
443	0\\
444	0\\
445	0\\
446	0\\
447	0\\
448	0\\
449	0\\
450	0\\
451	0\\
452	0\\
453	0\\
454	0\\
455	0\\
456	0\\
457	0\\
458	0\\
459	0\\
460	0\\
461	0\\
462	0\\
463	0\\
464	0\\
465	0\\
466	0\\
467	0\\
468	0\\
469	0\\
470	0\\
471	0\\
472	0\\
473	0\\
474	0\\
475	0\\
476	0\\
477	0\\
478	0\\
479	0\\
480	0\\
481	0\\
482	0\\
483	0\\
484	0\\
485	0\\
486	0\\
487	0\\
488	0\\
489	0\\
490	0\\
491	0\\
492	0\\
493	0\\
494	0\\
495	0\\
496	0\\
497	0\\
498	0\\
499	0\\
500	0\\
501	0\\
502	0\\
503	0\\
504	0\\
505	0\\
506	0\\
507	0\\
508	0\\
509	0\\
510	0\\
511	0\\
512	0\\
513	0\\
514	0\\
515	0\\
516	0\\
517	0\\
518	0\\
519	0\\
520	0\\
521	0\\
522	0\\
523	0\\
524	0\\
525	0\\
526	0\\
527	0\\
528	0\\
529	0\\
530	0\\
531	0\\
532	0\\
533	0\\
534	0\\
535	0\\
536	0\\
537	0\\
538	0\\
539	0\\
540	4.09915539253185e-06\\
541	3.50855341422652e-05\\
542	6.56241650374363e-05\\
543	9.62275076458695e-05\\
544	0.000127190880966168\\
545	0.000158606315656507\\
546	0.000190566750982985\\
547	0.000223147011209739\\
548	0.000256388207954706\\
549	0.000290340966566383\\
550	0.000325350059729218\\
551	0.00036325193318367\\
552	0.000401173491610415\\
553	0.000438840853802339\\
554	0.000474293154902854\\
555	0.000510160152892572\\
556	0.000546586920712631\\
557	0.000583562655249454\\
558	0.000621307465471656\\
559	0.00065986095225696\\
560	0.000699251876427312\\
561	0.0007395059311949\\
562	0.000780644872361251\\
563	0.000822720633982572\\
564	0.000870893647640451\\
565	0.000918570800930397\\
566	0.000959929760329168\\
567	0.00100162913418252\\
568	0.00104358144557724\\
569	0.00108631367623227\\
570	0.0011298652614946\\
571	0.00117427238843636\\
572	0.00171542356870662\\
573	0.00201939825841623\\
574	0.00212061739353802\\
575	0.0021898538250397\\
576	0.00225970681651282\\
577	0.00233067087660055\\
578	0.00240277681456536\\
579	0.00247604952405764\\
580	0.00255051455217948\\
581	0.00262619830282023\\
582	0.00270312813271299\\
583	0.00278133254043779\\
584	0.00286084163829612\\
585	0.00294168838147584\\
586	0.00302391178867891\\
587	0.00310756536465452\\
588	0.00319273906110311\\
589	0.00327961639013215\\
590	0.00336862265377152\\
591	0.00346039684658503\\
592	0.00355653402929193\\
593	0.00366119131364762\\
594	0.00378527046594782\\
595	0.00395742119285502\\
596	0.00425297862742962\\
597	0.0048700418989439\\
598	0.00632942537858856\\
599	0\\
600	0\\
};
\addplot [color=red,solid,forget plot]
  table[row sep=crcr]{%
1	0\\
2	0\\
3	0\\
4	0\\
5	0\\
6	0\\
7	0\\
8	0\\
9	0\\
10	0\\
11	0\\
12	0\\
13	0\\
14	0\\
15	0\\
16	0\\
17	0\\
18	0\\
19	0\\
20	0\\
21	0\\
22	0\\
23	0\\
24	0\\
25	0\\
26	0\\
27	0\\
28	0\\
29	0\\
30	0\\
31	0\\
32	0\\
33	0\\
34	0\\
35	0\\
36	0\\
37	0\\
38	0\\
39	0\\
40	0\\
41	0\\
42	0\\
43	0\\
44	0\\
45	0\\
46	0\\
47	0\\
48	0\\
49	0\\
50	0\\
51	0\\
52	0\\
53	0\\
54	0\\
55	0\\
56	0\\
57	0\\
58	0\\
59	0\\
60	0\\
61	0\\
62	0\\
63	0\\
64	0\\
65	0\\
66	0\\
67	0\\
68	0\\
69	0\\
70	0\\
71	0\\
72	0\\
73	0\\
74	0\\
75	0\\
76	0\\
77	0\\
78	0\\
79	0\\
80	0\\
81	0\\
82	0\\
83	0\\
84	0\\
85	0\\
86	0\\
87	0\\
88	0\\
89	0\\
90	0\\
91	0\\
92	0\\
93	0\\
94	0\\
95	0\\
96	0\\
97	0\\
98	0\\
99	0\\
100	0\\
101	0\\
102	0\\
103	0\\
104	0\\
105	0\\
106	0\\
107	0\\
108	0\\
109	0\\
110	0\\
111	0\\
112	0\\
113	0\\
114	0\\
115	0\\
116	0\\
117	0\\
118	0\\
119	0\\
120	0\\
121	0\\
122	0\\
123	0\\
124	0\\
125	0\\
126	0\\
127	0\\
128	0\\
129	0\\
130	0\\
131	0\\
132	0\\
133	0\\
134	0\\
135	0\\
136	0\\
137	0\\
138	0\\
139	0\\
140	0\\
141	0\\
142	0\\
143	0\\
144	0\\
145	0\\
146	0\\
147	0\\
148	0\\
149	0\\
150	0\\
151	0\\
152	0\\
153	0\\
154	0\\
155	0\\
156	0\\
157	0\\
158	0\\
159	0\\
160	0\\
161	0\\
162	0\\
163	0\\
164	0\\
165	0\\
166	0\\
167	0\\
168	0\\
169	0\\
170	0\\
171	0\\
172	0\\
173	0\\
174	0\\
175	0\\
176	0\\
177	0\\
178	0\\
179	0\\
180	0\\
181	0\\
182	0\\
183	0\\
184	0\\
185	0\\
186	0\\
187	0\\
188	0\\
189	0\\
190	0\\
191	0\\
192	0\\
193	0\\
194	0\\
195	0\\
196	0\\
197	0\\
198	0\\
199	0\\
200	0\\
201	0\\
202	0\\
203	0\\
204	0\\
205	0\\
206	0\\
207	0\\
208	0\\
209	0\\
210	0\\
211	0\\
212	0\\
213	0\\
214	0\\
215	0\\
216	0\\
217	0\\
218	0\\
219	0\\
220	0\\
221	0\\
222	0\\
223	0\\
224	0\\
225	0\\
226	0\\
227	0\\
228	0\\
229	0\\
230	0\\
231	0\\
232	0\\
233	0\\
234	0\\
235	0\\
236	0\\
237	0\\
238	0\\
239	0\\
240	0\\
241	0\\
242	0\\
243	0\\
244	0\\
245	0\\
246	0\\
247	0\\
248	0\\
249	0\\
250	0\\
251	0\\
252	0\\
253	0\\
254	0\\
255	0\\
256	0\\
257	0\\
258	0\\
259	0\\
260	0\\
261	0\\
262	0\\
263	0\\
264	0\\
265	0\\
266	0\\
267	0\\
268	0\\
269	0\\
270	0\\
271	0\\
272	0\\
273	0\\
274	0\\
275	0\\
276	0\\
277	0\\
278	0\\
279	0\\
280	0\\
281	0\\
282	0\\
283	0\\
284	0\\
285	0\\
286	0\\
287	0\\
288	0\\
289	0\\
290	0\\
291	0\\
292	0\\
293	0\\
294	0\\
295	0\\
296	0\\
297	0\\
298	0\\
299	0\\
300	0\\
301	0\\
302	0\\
303	0\\
304	0\\
305	0\\
306	0\\
307	0\\
308	0\\
309	0\\
310	0\\
311	0\\
312	0\\
313	0\\
314	0\\
315	0\\
316	0\\
317	0\\
318	0\\
319	0\\
320	0\\
321	0\\
322	0\\
323	0\\
324	0\\
325	0\\
326	0\\
327	0\\
328	0\\
329	0\\
330	0\\
331	0\\
332	0\\
333	0\\
334	0\\
335	0\\
336	0\\
337	0\\
338	0\\
339	0\\
340	0\\
341	0\\
342	0\\
343	0\\
344	0\\
345	0\\
346	0\\
347	0\\
348	0\\
349	0\\
350	0\\
351	0\\
352	0\\
353	0\\
354	0\\
355	0\\
356	0\\
357	0\\
358	0\\
359	0\\
360	0\\
361	0\\
362	0\\
363	0\\
364	0\\
365	0\\
366	0\\
367	0\\
368	0\\
369	0\\
370	0\\
371	0\\
372	0\\
373	0\\
374	0\\
375	0\\
376	0\\
377	0\\
378	0\\
379	0\\
380	0\\
381	0\\
382	0\\
383	0\\
384	0\\
385	0\\
386	0\\
387	0\\
388	0\\
389	0\\
390	0\\
391	0\\
392	0\\
393	0\\
394	0\\
395	0\\
396	0\\
397	0\\
398	0\\
399	0\\
400	0\\
401	0\\
402	0\\
403	0\\
404	0\\
405	0\\
406	0\\
407	0\\
408	0\\
409	0\\
410	0\\
411	0\\
412	0\\
413	0\\
414	0\\
415	0\\
416	0\\
417	0\\
418	0\\
419	0\\
420	0\\
421	0\\
422	0\\
423	0\\
424	0\\
425	0\\
426	0\\
427	0\\
428	0\\
429	0\\
430	0\\
431	0\\
432	0\\
433	0\\
434	0\\
435	0\\
436	0\\
437	0\\
438	0\\
439	0\\
440	0\\
441	0\\
442	0\\
443	0\\
444	0\\
445	0\\
446	0\\
447	0\\
448	0\\
449	0\\
450	0\\
451	0\\
452	0\\
453	0\\
454	0\\
455	0\\
456	0\\
457	0\\
458	0\\
459	0\\
460	0\\
461	0\\
462	0\\
463	0\\
464	0\\
465	0\\
466	0\\
467	0\\
468	0\\
469	0\\
470	0\\
471	0\\
472	0\\
473	0\\
474	0\\
475	0\\
476	0\\
477	0\\
478	0\\
479	0\\
480	0\\
481	0\\
482	0\\
483	0\\
484	0\\
485	0\\
486	0\\
487	0\\
488	0\\
489	0\\
490	0\\
491	0\\
492	0\\
493	0\\
494	0\\
495	0\\
496	0\\
497	0\\
498	0\\
499	0\\
500	0\\
501	0\\
502	0\\
503	0\\
504	0\\
505	0\\
506	0\\
507	0\\
508	0\\
509	0\\
510	0\\
511	0\\
512	0\\
513	0\\
514	0\\
515	0\\
516	0\\
517	0\\
518	0\\
519	0\\
520	0\\
521	0\\
522	0\\
523	0\\
524	0\\
525	0\\
526	0\\
527	0\\
528	0\\
529	0\\
530	0\\
531	0\\
532	0\\
533	0\\
534	0\\
535	0\\
536	0\\
537	0\\
538	0\\
539	0\\
540	2.72097330305856e-05\\
541	5.61210861900328e-05\\
542	8.5585431440554e-05\\
543	0.000115648715291517\\
544	0.000146340270906872\\
545	0.000177684940527103\\
546	0.000209720256944281\\
547	0.000242556230441567\\
548	0.000278557139405352\\
549	0.000314550976192727\\
550	0.00035020981117787\\
551	0.000383724276119911\\
552	0.000417754807534627\\
553	0.000452289907333273\\
554	0.000487256054489089\\
555	0.000522838353760378\\
556	0.000559179818522669\\
557	0.000596309200618129\\
558	0.000634247900431846\\
559	0.000673017586461841\\
560	0.00071263931167027\\
561	0.000753150047157175\\
562	0.000798519250497241\\
563	0.000845250121868294\\
564	0.000885803380539406\\
565	0.000926054546725999\\
566	0.00096653431220477\\
567	0.00100777034248639\\
568	0.00104980318090226\\
569	0.0010926538929629\\
570	0.00126482494133247\\
571	0.00184533543969108\\
572	0.0019701721867511\\
573	0.00205342365444128\\
574	0.00212111322836602\\
575	0.00218986312710426\\
576	0.0022597081208049\\
577	0.0023306715146185\\
578	0.00240277715940921\\
579	0.0024760497051662\\
580	0.00255051464193462\\
581	0.00262619834400932\\
582	0.00270312814987377\\
583	0.00278133254676196\\
584	0.0028608416402799\\
585	0.00294168838197388\\
586	0.00302391178876813\\
587	0.00310756536466312\\
588	0.00319273906110311\\
589	0.00327961639013215\\
590	0.00336862265377153\\
591	0.00346039684658503\\
592	0.00355653402929193\\
593	0.00366119131364763\\
594	0.00378527046594782\\
595	0.00395742119285503\\
596	0.00425297862742962\\
597	0.0048700418989439\\
598	0.00632942537858856\\
599	0\\
600	0\\
};
\addplot [color=mycolor20,solid,forget plot]
  table[row sep=crcr]{%
1	0\\
2	0\\
3	0\\
4	0\\
5	0\\
6	0\\
7	0\\
8	0\\
9	0\\
10	0\\
11	0\\
12	0\\
13	0\\
14	0\\
15	0\\
16	0\\
17	0\\
18	0\\
19	0\\
20	0\\
21	0\\
22	0\\
23	0\\
24	0\\
25	0\\
26	0\\
27	0\\
28	0\\
29	0\\
30	0\\
31	0\\
32	0\\
33	0\\
34	0\\
35	0\\
36	0\\
37	0\\
38	0\\
39	0\\
40	0\\
41	0\\
42	0\\
43	0\\
44	0\\
45	0\\
46	0\\
47	0\\
48	0\\
49	0\\
50	0\\
51	0\\
52	0\\
53	0\\
54	0\\
55	0\\
56	0\\
57	0\\
58	0\\
59	0\\
60	0\\
61	0\\
62	0\\
63	0\\
64	0\\
65	0\\
66	0\\
67	0\\
68	0\\
69	0\\
70	0\\
71	0\\
72	0\\
73	0\\
74	0\\
75	0\\
76	0\\
77	0\\
78	0\\
79	0\\
80	0\\
81	0\\
82	0\\
83	0\\
84	0\\
85	0\\
86	0\\
87	0\\
88	0\\
89	0\\
90	0\\
91	0\\
92	0\\
93	0\\
94	0\\
95	0\\
96	0\\
97	0\\
98	0\\
99	0\\
100	0\\
101	0\\
102	0\\
103	0\\
104	0\\
105	0\\
106	0\\
107	0\\
108	0\\
109	0\\
110	0\\
111	0\\
112	0\\
113	0\\
114	0\\
115	0\\
116	0\\
117	0\\
118	0\\
119	0\\
120	0\\
121	0\\
122	0\\
123	0\\
124	0\\
125	0\\
126	0\\
127	0\\
128	0\\
129	0\\
130	0\\
131	0\\
132	0\\
133	0\\
134	0\\
135	0\\
136	0\\
137	0\\
138	0\\
139	0\\
140	0\\
141	0\\
142	0\\
143	0\\
144	0\\
145	0\\
146	0\\
147	0\\
148	0\\
149	0\\
150	0\\
151	0\\
152	0\\
153	0\\
154	0\\
155	0\\
156	0\\
157	0\\
158	0\\
159	0\\
160	0\\
161	0\\
162	0\\
163	0\\
164	0\\
165	0\\
166	0\\
167	0\\
168	0\\
169	0\\
170	0\\
171	0\\
172	0\\
173	0\\
174	0\\
175	0\\
176	0\\
177	0\\
178	0\\
179	0\\
180	0\\
181	0\\
182	0\\
183	0\\
184	0\\
185	0\\
186	0\\
187	0\\
188	0\\
189	0\\
190	0\\
191	0\\
192	0\\
193	0\\
194	0\\
195	0\\
196	0\\
197	0\\
198	0\\
199	0\\
200	0\\
201	0\\
202	0\\
203	0\\
204	0\\
205	0\\
206	0\\
207	0\\
208	0\\
209	0\\
210	0\\
211	0\\
212	0\\
213	0\\
214	0\\
215	0\\
216	0\\
217	0\\
218	0\\
219	0\\
220	0\\
221	0\\
222	0\\
223	0\\
224	0\\
225	0\\
226	0\\
227	0\\
228	0\\
229	0\\
230	0\\
231	0\\
232	0\\
233	0\\
234	0\\
235	0\\
236	0\\
237	0\\
238	0\\
239	0\\
240	0\\
241	0\\
242	0\\
243	0\\
244	0\\
245	0\\
246	0\\
247	0\\
248	0\\
249	0\\
250	0\\
251	0\\
252	0\\
253	0\\
254	0\\
255	0\\
256	0\\
257	0\\
258	0\\
259	0\\
260	0\\
261	0\\
262	0\\
263	0\\
264	0\\
265	0\\
266	0\\
267	0\\
268	0\\
269	0\\
270	0\\
271	0\\
272	0\\
273	0\\
274	0\\
275	0\\
276	0\\
277	0\\
278	0\\
279	0\\
280	0\\
281	0\\
282	0\\
283	0\\
284	0\\
285	0\\
286	0\\
287	0\\
288	0\\
289	0\\
290	0\\
291	0\\
292	0\\
293	0\\
294	0\\
295	0\\
296	0\\
297	0\\
298	0\\
299	0\\
300	0\\
301	0\\
302	0\\
303	0\\
304	0\\
305	0\\
306	0\\
307	0\\
308	0\\
309	0\\
310	0\\
311	0\\
312	0\\
313	0\\
314	0\\
315	0\\
316	0\\
317	0\\
318	0\\
319	0\\
320	0\\
321	0\\
322	0\\
323	0\\
324	0\\
325	0\\
326	0\\
327	0\\
328	0\\
329	0\\
330	0\\
331	0\\
332	0\\
333	0\\
334	0\\
335	0\\
336	0\\
337	0\\
338	0\\
339	0\\
340	0\\
341	0\\
342	0\\
343	0\\
344	0\\
345	0\\
346	0\\
347	0\\
348	0\\
349	0\\
350	0\\
351	0\\
352	0\\
353	0\\
354	0\\
355	0\\
356	0\\
357	0\\
358	0\\
359	0\\
360	0\\
361	0\\
362	0\\
363	0\\
364	0\\
365	0\\
366	0\\
367	0\\
368	0\\
369	0\\
370	0\\
371	0\\
372	0\\
373	0\\
374	0\\
375	0\\
376	0\\
377	0\\
378	0\\
379	0\\
380	0\\
381	0\\
382	0\\
383	0\\
384	0\\
385	0\\
386	0\\
387	0\\
388	0\\
389	0\\
390	0\\
391	0\\
392	0\\
393	0\\
394	0\\
395	0\\
396	0\\
397	0\\
398	0\\
399	0\\
400	0\\
401	0\\
402	0\\
403	0\\
404	0\\
405	0\\
406	0\\
407	0\\
408	0\\
409	0\\
410	0\\
411	0\\
412	0\\
413	0\\
414	0\\
415	0\\
416	0\\
417	0\\
418	0\\
419	0\\
420	0\\
421	0\\
422	0\\
423	0\\
424	0\\
425	0\\
426	0\\
427	0\\
428	0\\
429	0\\
430	0\\
431	0\\
432	0\\
433	0\\
434	0\\
435	0\\
436	0\\
437	0\\
438	0\\
439	0\\
440	0\\
441	0\\
442	0\\
443	0\\
444	0\\
445	0\\
446	0\\
447	0\\
448	0\\
449	0\\
450	0\\
451	0\\
452	0\\
453	0\\
454	0\\
455	0\\
456	0\\
457	0\\
458	0\\
459	0\\
460	0\\
461	0\\
462	0\\
463	0\\
464	0\\
465	0\\
466	0\\
467	0\\
468	0\\
469	0\\
470	0\\
471	0\\
472	0\\
473	0\\
474	0\\
475	0\\
476	0\\
477	0\\
478	0\\
479	0\\
480	0\\
481	0\\
482	0\\
483	0\\
484	0\\
485	0\\
486	0\\
487	0\\
488	0\\
489	0\\
490	0\\
491	0\\
492	0\\
493	0\\
494	0\\
495	0\\
496	0\\
497	0\\
498	0\\
499	0\\
500	0\\
501	0\\
502	0\\
503	0\\
504	0\\
505	0\\
506	0\\
507	0\\
508	0\\
509	0\\
510	0\\
511	0\\
512	0\\
513	0\\
514	0\\
515	0\\
516	0\\
517	0\\
518	0\\
519	0\\
520	0\\
521	0\\
522	0\\
523	0\\
524	0\\
525	0\\
526	0\\
527	0\\
528	0\\
529	0\\
530	0\\
531	0\\
532	0\\
533	0\\
534	0\\
535	0\\
536	0\\
537	0\\
538	0\\
539	1.49492185966091e-05\\
540	4.3322083531691e-05\\
541	7.23064945701987e-05\\
542	0.000101919514130588\\
543	0.000132186799457533\\
544	0.000163148555289638\\
545	0.000196899412872687\\
546	0.000231163070872471\\
547	0.000265280313724257\\
548	0.000296939134999868\\
549	0.000329084648366537\\
550	0.000361719670533462\\
551	0.000394828145692352\\
552	0.000428606969412651\\
553	0.000463052122313129\\
554	0.000498111370312253\\
555	0.000533863357465193\\
556	0.00057039128795026\\
557	0.000607714878606064\\
558	0.000645854512820227\\
559	0.000684829623640465\\
560	0.000726776106591716\\
561	0.000772673465119796\\
562	0.000813534509467145\\
563	0.000852431652279271\\
564	0.000891516665074081\\
565	0.000931278020432426\\
566	0.000971798618077333\\
567	0.00101309731890282\\
568	0.00105519192697659\\
569	0.00134093035495452\\
570	0.00181655877875805\\
571	0.00191924729878529\\
572	0.00198681216416615\\
573	0.00205343742759807\\
574	0.00212111382330841\\
575	0.00218986332242957\\
576	0.00225970822322969\\
577	0.00233067156883697\\
578	0.00240277718689623\\
579	0.00247604971824475\\
580	0.00255051464767741\\
581	0.00262619834629172\\
582	0.0027031281506736\\
583	0.00278133254699977\\
584	0.00286084164033628\\
585	0.00294168838198338\\
586	0.00302391178876899\\
587	0.00310756536466312\\
588	0.00319273906110312\\
589	0.00327961639013216\\
590	0.00336862265377153\\
591	0.00346039684658503\\
592	0.00355653402929193\\
593	0.00366119131364763\\
594	0.00378527046594782\\
595	0.00395742119285503\\
596	0.00425297862742962\\
597	0.0048700418989439\\
598	0.00632942537858856\\
599	0\\
600	0\\
};
\addplot [color=mycolor21,solid,forget plot]
  table[row sep=crcr]{%
1	0\\
2	0\\
3	0\\
4	0\\
5	0\\
6	0\\
7	0\\
8	0\\
9	0\\
10	0\\
11	0\\
12	0\\
13	0\\
14	0\\
15	0\\
16	0\\
17	0\\
18	0\\
19	0\\
20	0\\
21	0\\
22	0\\
23	0\\
24	0\\
25	0\\
26	0\\
27	0\\
28	0\\
29	0\\
30	0\\
31	0\\
32	0\\
33	0\\
34	0\\
35	0\\
36	0\\
37	0\\
38	0\\
39	0\\
40	0\\
41	0\\
42	0\\
43	0\\
44	0\\
45	0\\
46	0\\
47	0\\
48	0\\
49	0\\
50	0\\
51	0\\
52	0\\
53	0\\
54	0\\
55	0\\
56	0\\
57	0\\
58	0\\
59	0\\
60	0\\
61	0\\
62	0\\
63	0\\
64	0\\
65	0\\
66	0\\
67	0\\
68	0\\
69	0\\
70	0\\
71	0\\
72	0\\
73	0\\
74	0\\
75	0\\
76	0\\
77	0\\
78	0\\
79	0\\
80	0\\
81	0\\
82	0\\
83	0\\
84	0\\
85	0\\
86	0\\
87	0\\
88	0\\
89	0\\
90	0\\
91	0\\
92	0\\
93	0\\
94	0\\
95	0\\
96	0\\
97	0\\
98	0\\
99	0\\
100	0\\
101	0\\
102	0\\
103	0\\
104	0\\
105	0\\
106	0\\
107	0\\
108	0\\
109	0\\
110	0\\
111	0\\
112	0\\
113	0\\
114	0\\
115	0\\
116	0\\
117	0\\
118	0\\
119	0\\
120	0\\
121	0\\
122	0\\
123	0\\
124	0\\
125	0\\
126	0\\
127	0\\
128	0\\
129	0\\
130	0\\
131	0\\
132	0\\
133	0\\
134	0\\
135	0\\
136	0\\
137	0\\
138	0\\
139	0\\
140	0\\
141	0\\
142	0\\
143	0\\
144	0\\
145	0\\
146	0\\
147	0\\
148	0\\
149	0\\
150	0\\
151	0\\
152	0\\
153	0\\
154	0\\
155	0\\
156	0\\
157	0\\
158	0\\
159	0\\
160	0\\
161	0\\
162	0\\
163	0\\
164	0\\
165	0\\
166	0\\
167	0\\
168	0\\
169	0\\
170	0\\
171	0\\
172	0\\
173	0\\
174	0\\
175	0\\
176	0\\
177	0\\
178	0\\
179	0\\
180	0\\
181	0\\
182	0\\
183	0\\
184	0\\
185	0\\
186	0\\
187	0\\
188	0\\
189	0\\
190	0\\
191	0\\
192	0\\
193	0\\
194	0\\
195	0\\
196	0\\
197	0\\
198	0\\
199	0\\
200	0\\
201	0\\
202	0\\
203	0\\
204	0\\
205	0\\
206	0\\
207	0\\
208	0\\
209	0\\
210	0\\
211	0\\
212	0\\
213	0\\
214	0\\
215	0\\
216	0\\
217	0\\
218	0\\
219	0\\
220	0\\
221	0\\
222	0\\
223	0\\
224	0\\
225	0\\
226	0\\
227	0\\
228	0\\
229	0\\
230	0\\
231	0\\
232	0\\
233	0\\
234	0\\
235	0\\
236	0\\
237	0\\
238	0\\
239	0\\
240	0\\
241	0\\
242	0\\
243	0\\
244	0\\
245	0\\
246	0\\
247	0\\
248	0\\
249	0\\
250	0\\
251	0\\
252	0\\
253	0\\
254	0\\
255	0\\
256	0\\
257	0\\
258	0\\
259	0\\
260	0\\
261	0\\
262	0\\
263	0\\
264	0\\
265	0\\
266	0\\
267	0\\
268	0\\
269	0\\
270	0\\
271	0\\
272	0\\
273	0\\
274	0\\
275	0\\
276	0\\
277	0\\
278	0\\
279	0\\
280	0\\
281	0\\
282	0\\
283	0\\
284	0\\
285	0\\
286	0\\
287	0\\
288	0\\
289	0\\
290	0\\
291	0\\
292	0\\
293	0\\
294	0\\
295	0\\
296	0\\
297	0\\
298	0\\
299	0\\
300	0\\
301	0\\
302	0\\
303	0\\
304	0\\
305	0\\
306	0\\
307	0\\
308	0\\
309	0\\
310	0\\
311	0\\
312	0\\
313	0\\
314	0\\
315	0\\
316	0\\
317	0\\
318	0\\
319	0\\
320	0\\
321	0\\
322	0\\
323	0\\
324	0\\
325	0\\
326	0\\
327	0\\
328	0\\
329	0\\
330	0\\
331	0\\
332	0\\
333	0\\
334	0\\
335	0\\
336	0\\
337	0\\
338	0\\
339	0\\
340	0\\
341	0\\
342	0\\
343	0\\
344	0\\
345	0\\
346	0\\
347	0\\
348	0\\
349	0\\
350	0\\
351	0\\
352	0\\
353	0\\
354	0\\
355	0\\
356	0\\
357	0\\
358	0\\
359	0\\
360	0\\
361	0\\
362	0\\
363	0\\
364	0\\
365	0\\
366	0\\
367	0\\
368	0\\
369	0\\
370	0\\
371	0\\
372	0\\
373	0\\
374	0\\
375	0\\
376	0\\
377	0\\
378	0\\
379	0\\
380	0\\
381	0\\
382	0\\
383	0\\
384	0\\
385	0\\
386	0\\
387	0\\
388	0\\
389	0\\
390	0\\
391	0\\
392	0\\
393	0\\
394	0\\
395	0\\
396	0\\
397	0\\
398	0\\
399	0\\
400	0\\
401	0\\
402	0\\
403	0\\
404	0\\
405	0\\
406	0\\
407	0\\
408	0\\
409	0\\
410	0\\
411	0\\
412	0\\
413	0\\
414	0\\
415	0\\
416	0\\
417	0\\
418	0\\
419	0\\
420	0\\
421	0\\
422	0\\
423	0\\
424	0\\
425	0\\
426	0\\
427	0\\
428	0\\
429	0\\
430	0\\
431	0\\
432	0\\
433	0\\
434	0\\
435	0\\
436	0\\
437	0\\
438	0\\
439	0\\
440	0\\
441	0\\
442	0\\
443	0\\
444	0\\
445	0\\
446	0\\
447	0\\
448	0\\
449	0\\
450	0\\
451	0\\
452	0\\
453	0\\
454	0\\
455	0\\
456	0\\
457	0\\
458	0\\
459	0\\
460	0\\
461	0\\
462	0\\
463	0\\
464	0\\
465	0\\
466	0\\
467	0\\
468	0\\
469	0\\
470	0\\
471	0\\
472	0\\
473	0\\
474	0\\
475	0\\
476	0\\
477	0\\
478	0\\
479	0\\
480	0\\
481	0\\
482	0\\
483	0\\
484	0\\
485	0\\
486	0\\
487	0\\
488	0\\
489	0\\
490	0\\
491	0\\
492	0\\
493	0\\
494	0\\
495	0\\
496	0\\
497	0\\
498	0\\
499	0\\
500	0\\
501	0\\
502	0\\
503	0\\
504	0\\
505	0\\
506	0\\
507	0\\
508	0\\
509	0\\
510	0\\
511	0\\
512	0\\
513	0\\
514	0\\
515	0\\
516	0\\
517	0\\
518	0\\
519	0\\
520	0\\
521	0\\
522	0\\
523	0\\
524	0\\
525	0\\
526	0\\
527	0\\
528	0\\
529	0\\
530	0\\
531	0\\
532	0\\
533	0\\
534	0\\
535	0\\
536	0\\
537	0\\
538	1.21694107886508e-06\\
539	2.92045651449222e-05\\
540	5.78049456237067e-05\\
541	8.70536593544758e-05\\
542	0.000118312744446052\\
543	0.000151016328109347\\
544	0.000183614512075941\\
545	0.000214055086743375\\
546	0.000244429244110815\\
547	0.000275261690455392\\
548	0.00030651277983146\\
549	0.000338404008295745\\
550	0.000370954886140724\\
551	0.00040418384758811\\
552	0.000438095852672766\\
553	0.000472675235959531\\
554	0.000507841527916458\\
555	0.000543753467495973\\
556	0.000580443856548278\\
557	0.000617931118089366\\
558	0.000656254294614929\\
559	0.000701069072514467\\
560	0.000743145599645497\\
561	0.000780787630662042\\
562	0.000818610761048887\\
563	0.000856957736854093\\
564	0.000896029558364049\\
565	0.000935845227354032\\
566	0.000976422049464571\\
567	0.00101777751913248\\
568	0.00138028712439642\\
569	0.00176338880029626\\
570	0.00185658807509324\\
571	0.00192121863126382\\
572	0.00198681271195394\\
573	0.00205343749576471\\
574	0.00212111385372883\\
575	0.00218986333840477\\
576	0.00225970823145245\\
577	0.00233067157285982\\
578	0.00240277718873636\\
579	0.00247604971901934\\
580	0.00255051464797174\\
581	0.00262619834639004\\
582	0.00270312815070138\\
583	0.00278133254700601\\
584	0.00286084164033727\\
585	0.00294168838198347\\
586	0.00302391178876899\\
587	0.00310756536466311\\
588	0.00319273906110311\\
589	0.00327961639013215\\
590	0.00336862265377153\\
591	0.00346039684658503\\
592	0.00355653402929193\\
593	0.00366119131364763\\
594	0.00378527046594783\\
595	0.00395742119285503\\
596	0.00425297862742962\\
597	0.0048700418989439\\
598	0.00632942537858856\\
599	0\\
600	0\\
};
\addplot [color=black!20!mycolor21,solid,forget plot]
  table[row sep=crcr]{%
1	0\\
2	0\\
3	0\\
4	0\\
5	0\\
6	0\\
7	0\\
8	0\\
9	0\\
10	0\\
11	0\\
12	0\\
13	0\\
14	0\\
15	0\\
16	0\\
17	0\\
18	0\\
19	0\\
20	0\\
21	0\\
22	0\\
23	0\\
24	0\\
25	0\\
26	0\\
27	0\\
28	0\\
29	0\\
30	0\\
31	0\\
32	0\\
33	0\\
34	0\\
35	0\\
36	0\\
37	0\\
38	0\\
39	0\\
40	0\\
41	0\\
42	0\\
43	0\\
44	0\\
45	0\\
46	0\\
47	0\\
48	0\\
49	0\\
50	0\\
51	0\\
52	0\\
53	0\\
54	0\\
55	0\\
56	0\\
57	0\\
58	0\\
59	0\\
60	0\\
61	0\\
62	0\\
63	0\\
64	0\\
65	0\\
66	0\\
67	0\\
68	0\\
69	0\\
70	0\\
71	0\\
72	0\\
73	0\\
74	0\\
75	0\\
76	0\\
77	0\\
78	0\\
79	0\\
80	0\\
81	0\\
82	0\\
83	0\\
84	0\\
85	0\\
86	0\\
87	0\\
88	0\\
89	0\\
90	0\\
91	0\\
92	0\\
93	0\\
94	0\\
95	0\\
96	0\\
97	0\\
98	0\\
99	0\\
100	0\\
101	0\\
102	0\\
103	0\\
104	0\\
105	0\\
106	0\\
107	0\\
108	0\\
109	0\\
110	0\\
111	0\\
112	0\\
113	0\\
114	0\\
115	0\\
116	0\\
117	0\\
118	0\\
119	0\\
120	0\\
121	0\\
122	0\\
123	0\\
124	0\\
125	0\\
126	0\\
127	0\\
128	0\\
129	0\\
130	0\\
131	0\\
132	0\\
133	0\\
134	0\\
135	0\\
136	0\\
137	0\\
138	0\\
139	0\\
140	0\\
141	0\\
142	0\\
143	0\\
144	0\\
145	0\\
146	0\\
147	0\\
148	0\\
149	0\\
150	0\\
151	0\\
152	0\\
153	0\\
154	0\\
155	0\\
156	0\\
157	0\\
158	0\\
159	0\\
160	0\\
161	0\\
162	0\\
163	0\\
164	0\\
165	0\\
166	0\\
167	0\\
168	0\\
169	0\\
170	0\\
171	0\\
172	0\\
173	0\\
174	0\\
175	0\\
176	0\\
177	0\\
178	0\\
179	0\\
180	0\\
181	0\\
182	0\\
183	0\\
184	0\\
185	0\\
186	0\\
187	0\\
188	0\\
189	0\\
190	0\\
191	0\\
192	0\\
193	0\\
194	0\\
195	0\\
196	0\\
197	0\\
198	0\\
199	0\\
200	0\\
201	0\\
202	0\\
203	0\\
204	0\\
205	0\\
206	0\\
207	0\\
208	0\\
209	0\\
210	0\\
211	0\\
212	0\\
213	0\\
214	0\\
215	0\\
216	0\\
217	0\\
218	0\\
219	0\\
220	0\\
221	0\\
222	0\\
223	0\\
224	0\\
225	0\\
226	0\\
227	0\\
228	0\\
229	0\\
230	0\\
231	0\\
232	0\\
233	0\\
234	0\\
235	0\\
236	0\\
237	0\\
238	0\\
239	0\\
240	0\\
241	0\\
242	0\\
243	0\\
244	0\\
245	0\\
246	0\\
247	0\\
248	0\\
249	0\\
250	0\\
251	0\\
252	0\\
253	0\\
254	0\\
255	0\\
256	0\\
257	0\\
258	0\\
259	0\\
260	0\\
261	0\\
262	0\\
263	0\\
264	0\\
265	0\\
266	0\\
267	0\\
268	0\\
269	0\\
270	0\\
271	0\\
272	0\\
273	0\\
274	0\\
275	0\\
276	0\\
277	0\\
278	0\\
279	0\\
280	0\\
281	0\\
282	0\\
283	0\\
284	0\\
285	0\\
286	0\\
287	0\\
288	0\\
289	0\\
290	0\\
291	0\\
292	0\\
293	0\\
294	0\\
295	0\\
296	0\\
297	0\\
298	0\\
299	0\\
300	0\\
301	0\\
302	0\\
303	0\\
304	0\\
305	0\\
306	0\\
307	0\\
308	0\\
309	0\\
310	0\\
311	0\\
312	0\\
313	0\\
314	0\\
315	0\\
316	0\\
317	0\\
318	0\\
319	0\\
320	0\\
321	0\\
322	0\\
323	0\\
324	0\\
325	0\\
326	0\\
327	0\\
328	0\\
329	0\\
330	0\\
331	0\\
332	0\\
333	0\\
334	0\\
335	0\\
336	0\\
337	0\\
338	0\\
339	0\\
340	0\\
341	0\\
342	0\\
343	0\\
344	0\\
345	0\\
346	0\\
347	0\\
348	0\\
349	0\\
350	0\\
351	0\\
352	0\\
353	0\\
354	0\\
355	0\\
356	0\\
357	0\\
358	0\\
359	0\\
360	0\\
361	0\\
362	0\\
363	0\\
364	0\\
365	0\\
366	0\\
367	0\\
368	0\\
369	0\\
370	0\\
371	0\\
372	0\\
373	0\\
374	0\\
375	0\\
376	0\\
377	0\\
378	0\\
379	0\\
380	0\\
381	0\\
382	0\\
383	0\\
384	0\\
385	0\\
386	0\\
387	0\\
388	0\\
389	0\\
390	0\\
391	0\\
392	0\\
393	0\\
394	0\\
395	0\\
396	0\\
397	0\\
398	0\\
399	0\\
400	0\\
401	0\\
402	0\\
403	0\\
404	0\\
405	0\\
406	0\\
407	0\\
408	0\\
409	0\\
410	0\\
411	0\\
412	0\\
413	0\\
414	0\\
415	0\\
416	0\\
417	0\\
418	0\\
419	0\\
420	0\\
421	0\\
422	0\\
423	0\\
424	0\\
425	0\\
426	0\\
427	0\\
428	0\\
429	0\\
430	0\\
431	0\\
432	0\\
433	0\\
434	0\\
435	0\\
436	0\\
437	0\\
438	0\\
439	0\\
440	0\\
441	0\\
442	0\\
443	0\\
444	0\\
445	0\\
446	0\\
447	0\\
448	0\\
449	0\\
450	0\\
451	0\\
452	0\\
453	0\\
454	0\\
455	0\\
456	0\\
457	0\\
458	0\\
459	0\\
460	0\\
461	0\\
462	0\\
463	0\\
464	0\\
465	0\\
466	0\\
467	0\\
468	0\\
469	0\\
470	0\\
471	0\\
472	0\\
473	0\\
474	0\\
475	0\\
476	0\\
477	0\\
478	0\\
479	0\\
480	0\\
481	0\\
482	0\\
483	0\\
484	0\\
485	0\\
486	0\\
487	0\\
488	0\\
489	0\\
490	0\\
491	0\\
492	0\\
493	0\\
494	0\\
495	0\\
496	0\\
497	0\\
498	0\\
499	0\\
500	0\\
501	0\\
502	0\\
503	0\\
504	0\\
505	0\\
506	0\\
507	0\\
508	0\\
509	0\\
510	0\\
511	0\\
512	0\\
513	0\\
514	0\\
515	0\\
516	0\\
517	0\\
518	0\\
519	0\\
520	0\\
521	0\\
522	0\\
523	0\\
524	0\\
525	0\\
526	0\\
527	0\\
528	0\\
529	0\\
530	0\\
531	0\\
532	0\\
533	0\\
534	0\\
535	0\\
536	0\\
537	0\\
538	1.426719923304e-05\\
539	4.29091327027637e-05\\
540	7.42159152757329e-05\\
541	0.00010540163623223\\
542	0.000135128139185704\\
543	0.000163860507202848\\
544	0.000193019670144231\\
545	0.000222567303902726\\
546	0.00025268187255965\\
547	0.000283418338955152\\
548	0.000314798578510525\\
549	0.000346837628639188\\
550	0.000379549362854509\\
551	0.00041294395250958\\
552	0.0004470207297917\\
553	0.000481745886966618\\
554	0.000517094416528536\\
555	0.000553212180424865\\
556	0.000590118819429489\\
557	0.000630458343480549\\
558	0.000674248409785912\\
559	0.000711217113060013\\
560	0.000747899309859749\\
561	0.0007848972137292\\
562	0.000822585793724534\\
563	0.00086098759293346\\
564	0.000900118540435697\\
565	0.000939995557047767\\
566	0.000980636019795443\\
567	0.00138345405867184\\
568	0.0017080408379249\\
569	0.00179304145117328\\
570	0.00185663516672567\\
571	0.00192121866981793\\
572	0.00198681272153112\\
573	0.00205343750043299\\
574	0.00212111385614151\\
575	0.00218986333960979\\
576	0.00225970823202167\\
577	0.00233067157311048\\
578	0.00240277718883768\\
579	0.00247604971905621\\
580	0.00255051464798352\\
581	0.00262619834639323\\
582	0.00270312815070207\\
583	0.00278133254700611\\
584	0.00286084164033729\\
585	0.00294168838198347\\
586	0.00302391178876899\\
587	0.00310756536466313\\
588	0.00319273906110312\\
589	0.00327961639013215\\
590	0.00336862265377153\\
591	0.00346039684658503\\
592	0.00355653402929193\\
593	0.00366119131364763\\
594	0.00378527046594782\\
595	0.00395742119285503\\
596	0.00425297862742962\\
597	0.0048700418989439\\
598	0.00632942537858856\\
599	0\\
600	0\\
};
\addplot [color=black!50!mycolor20,solid,forget plot]
  table[row sep=crcr]{%
1	0\\
2	0\\
3	0\\
4	0\\
5	0\\
6	0\\
7	0\\
8	0\\
9	0\\
10	0\\
11	0\\
12	0\\
13	0\\
14	0\\
15	0\\
16	0\\
17	0\\
18	0\\
19	0\\
20	0\\
21	0\\
22	0\\
23	0\\
24	0\\
25	0\\
26	0\\
27	0\\
28	0\\
29	0\\
30	0\\
31	0\\
32	0\\
33	0\\
34	0\\
35	0\\
36	0\\
37	0\\
38	0\\
39	0\\
40	0\\
41	0\\
42	0\\
43	0\\
44	0\\
45	0\\
46	0\\
47	0\\
48	0\\
49	0\\
50	0\\
51	0\\
52	0\\
53	0\\
54	0\\
55	0\\
56	0\\
57	0\\
58	0\\
59	0\\
60	0\\
61	0\\
62	0\\
63	0\\
64	0\\
65	0\\
66	0\\
67	0\\
68	0\\
69	0\\
70	0\\
71	0\\
72	0\\
73	0\\
74	0\\
75	0\\
76	0\\
77	0\\
78	0\\
79	0\\
80	0\\
81	0\\
82	0\\
83	0\\
84	0\\
85	0\\
86	0\\
87	0\\
88	0\\
89	0\\
90	0\\
91	0\\
92	0\\
93	0\\
94	0\\
95	0\\
96	0\\
97	0\\
98	0\\
99	0\\
100	0\\
101	0\\
102	0\\
103	0\\
104	0\\
105	0\\
106	0\\
107	0\\
108	0\\
109	0\\
110	0\\
111	0\\
112	0\\
113	0\\
114	0\\
115	0\\
116	0\\
117	0\\
118	0\\
119	0\\
120	0\\
121	0\\
122	0\\
123	0\\
124	0\\
125	0\\
126	0\\
127	0\\
128	0\\
129	0\\
130	0\\
131	0\\
132	0\\
133	0\\
134	0\\
135	0\\
136	0\\
137	0\\
138	0\\
139	0\\
140	0\\
141	0\\
142	0\\
143	0\\
144	0\\
145	0\\
146	0\\
147	0\\
148	0\\
149	0\\
150	0\\
151	0\\
152	0\\
153	0\\
154	0\\
155	0\\
156	0\\
157	0\\
158	0\\
159	0\\
160	0\\
161	0\\
162	0\\
163	0\\
164	0\\
165	0\\
166	0\\
167	0\\
168	0\\
169	0\\
170	0\\
171	0\\
172	0\\
173	0\\
174	0\\
175	0\\
176	0\\
177	0\\
178	0\\
179	0\\
180	0\\
181	0\\
182	0\\
183	0\\
184	0\\
185	0\\
186	0\\
187	0\\
188	0\\
189	0\\
190	0\\
191	0\\
192	0\\
193	0\\
194	0\\
195	0\\
196	0\\
197	0\\
198	0\\
199	0\\
200	0\\
201	0\\
202	0\\
203	0\\
204	0\\
205	0\\
206	0\\
207	0\\
208	0\\
209	0\\
210	0\\
211	0\\
212	0\\
213	0\\
214	0\\
215	0\\
216	0\\
217	0\\
218	0\\
219	0\\
220	0\\
221	0\\
222	0\\
223	0\\
224	0\\
225	0\\
226	0\\
227	0\\
228	0\\
229	0\\
230	0\\
231	0\\
232	0\\
233	0\\
234	0\\
235	0\\
236	0\\
237	0\\
238	0\\
239	0\\
240	0\\
241	0\\
242	0\\
243	0\\
244	0\\
245	0\\
246	0\\
247	0\\
248	0\\
249	0\\
250	0\\
251	0\\
252	0\\
253	0\\
254	0\\
255	0\\
256	0\\
257	0\\
258	0\\
259	0\\
260	0\\
261	0\\
262	0\\
263	0\\
264	0\\
265	0\\
266	0\\
267	0\\
268	0\\
269	0\\
270	0\\
271	0\\
272	0\\
273	0\\
274	0\\
275	0\\
276	0\\
277	0\\
278	0\\
279	0\\
280	0\\
281	0\\
282	0\\
283	0\\
284	0\\
285	0\\
286	0\\
287	0\\
288	0\\
289	0\\
290	0\\
291	0\\
292	0\\
293	0\\
294	0\\
295	0\\
296	0\\
297	0\\
298	0\\
299	0\\
300	0\\
301	0\\
302	0\\
303	0\\
304	0\\
305	0\\
306	0\\
307	0\\
308	0\\
309	0\\
310	0\\
311	0\\
312	0\\
313	0\\
314	0\\
315	0\\
316	0\\
317	0\\
318	0\\
319	0\\
320	0\\
321	0\\
322	0\\
323	0\\
324	0\\
325	0\\
326	0\\
327	0\\
328	0\\
329	0\\
330	0\\
331	0\\
332	0\\
333	0\\
334	0\\
335	0\\
336	0\\
337	0\\
338	0\\
339	0\\
340	0\\
341	0\\
342	0\\
343	0\\
344	0\\
345	0\\
346	0\\
347	0\\
348	0\\
349	0\\
350	0\\
351	0\\
352	0\\
353	0\\
354	0\\
355	0\\
356	0\\
357	0\\
358	0\\
359	0\\
360	0\\
361	0\\
362	0\\
363	0\\
364	0\\
365	0\\
366	0\\
367	0\\
368	0\\
369	0\\
370	0\\
371	0\\
372	0\\
373	0\\
374	0\\
375	0\\
376	0\\
377	0\\
378	0\\
379	0\\
380	0\\
381	0\\
382	0\\
383	0\\
384	0\\
385	0\\
386	0\\
387	0\\
388	0\\
389	0\\
390	0\\
391	0\\
392	0\\
393	0\\
394	0\\
395	0\\
396	0\\
397	0\\
398	0\\
399	0\\
400	0\\
401	0\\
402	0\\
403	0\\
404	0\\
405	0\\
406	0\\
407	0\\
408	0\\
409	0\\
410	0\\
411	0\\
412	0\\
413	0\\
414	0\\
415	0\\
416	0\\
417	0\\
418	0\\
419	0\\
420	0\\
421	0\\
422	0\\
423	0\\
424	0\\
425	0\\
426	0\\
427	0\\
428	0\\
429	0\\
430	0\\
431	0\\
432	0\\
433	0\\
434	0\\
435	0\\
436	0\\
437	0\\
438	0\\
439	0\\
440	0\\
441	0\\
442	0\\
443	0\\
444	0\\
445	0\\
446	0\\
447	0\\
448	0\\
449	0\\
450	0\\
451	0\\
452	0\\
453	0\\
454	0\\
455	0\\
456	0\\
457	0\\
458	0\\
459	0\\
460	0\\
461	0\\
462	0\\
463	0\\
464	0\\
465	0\\
466	0\\
467	0\\
468	0\\
469	0\\
470	0\\
471	0\\
472	0\\
473	0\\
474	0\\
475	0\\
476	0\\
477	0\\
478	0\\
479	0\\
480	0\\
481	0\\
482	0\\
483	0\\
484	0\\
485	0\\
486	0\\
487	0\\
488	0\\
489	0\\
490	0\\
491	0\\
492	0\\
493	0\\
494	0\\
495	0\\
496	0\\
497	0\\
498	0\\
499	0\\
500	0\\
501	0\\
502	0\\
503	0\\
504	0\\
505	0\\
506	0\\
507	0\\
508	0\\
509	0\\
510	0\\
511	0\\
512	0\\
513	0\\
514	0\\
515	0\\
516	0\\
517	0\\
518	0\\
519	0\\
520	0\\
521	0\\
522	0\\
523	0\\
524	0\\
525	0\\
526	0\\
527	0\\
528	0\\
529	0\\
530	0\\
531	0\\
532	0\\
533	0\\
534	0\\
535	0\\
536	0\\
537	6.49828894175203e-07\\
538	3.05652458911548e-05\\
539	5.99404394527822e-05\\
540	8.71312739401948e-05\\
541	0.000114716903641183\\
542	0.000142674508084298\\
543	0.000171100011748483\\
544	0.000200107410759274\\
545	0.000229717106180836\\
546	0.000259944556361032\\
547	0.000290804095303178\\
548	0.000322309893047164\\
549	0.000354475852884199\\
550	0.000387314440674418\\
551	0.000420833252736105\\
552	0.000455025048030861\\
553	0.00048983964048552\\
554	0.000525318728056389\\
555	0.000561574556760999\\
556	0.000603919415462835\\
557	0.00064369719785649\\
558	0.000679340140340759\\
559	0.000715080552741755\\
560	0.000751442297042198\\
561	0.000788488188292752\\
562	0.000826233343400132\\
563	0.000864693097341739\\
564	0.000903883877005878\\
565	0.000943822831240413\\
566	0.00135108213105103\\
567	0.00165100703747856\\
568	0.00173042226223131\\
569	0.0017930426874416\\
570	0.00185663517077488\\
571	0.00192121867121681\\
572	0.00198681272222949\\
573	0.00205343750078598\\
574	0.00212111385631247\\
575	0.00218986333968779\\
576	0.00225970823205476\\
577	0.00233067157312331\\
578	0.00240277718884216\\
579	0.00247604971905759\\
580	0.00255051464798387\\
581	0.0026261983463933\\
582	0.00270312815070208\\
583	0.0027813325470061\\
584	0.00286084164033727\\
585	0.00294168838198346\\
586	0.00302391178876899\\
587	0.00310756536466312\\
588	0.00319273906110312\\
589	0.00327961639013215\\
590	0.00336862265377153\\
591	0.00346039684658503\\
592	0.00355653402929193\\
593	0.00366119131364762\\
594	0.00378527046594782\\
595	0.00395742119285503\\
596	0.00425297862742961\\
597	0.0048700418989439\\
598	0.00632942537858856\\
599	0\\
600	0\\
};
\addplot [color=black!60!mycolor21,solid,forget plot]
  table[row sep=crcr]{%
1	0\\
2	0\\
3	0\\
4	0\\
5	0\\
6	0\\
7	0\\
8	0\\
9	0\\
10	0\\
11	0\\
12	0\\
13	0\\
14	0\\
15	0\\
16	0\\
17	0\\
18	0\\
19	0\\
20	0\\
21	0\\
22	0\\
23	0\\
24	0\\
25	0\\
26	0\\
27	0\\
28	0\\
29	0\\
30	0\\
31	0\\
32	0\\
33	0\\
34	0\\
35	0\\
36	0\\
37	0\\
38	0\\
39	0\\
40	0\\
41	0\\
42	0\\
43	0\\
44	0\\
45	0\\
46	0\\
47	0\\
48	0\\
49	0\\
50	0\\
51	0\\
52	0\\
53	0\\
54	0\\
55	0\\
56	0\\
57	0\\
58	0\\
59	0\\
60	0\\
61	0\\
62	0\\
63	0\\
64	0\\
65	0\\
66	0\\
67	0\\
68	0\\
69	0\\
70	0\\
71	0\\
72	0\\
73	0\\
74	0\\
75	0\\
76	0\\
77	0\\
78	0\\
79	0\\
80	0\\
81	0\\
82	0\\
83	0\\
84	0\\
85	0\\
86	0\\
87	0\\
88	0\\
89	0\\
90	0\\
91	0\\
92	0\\
93	0\\
94	0\\
95	0\\
96	0\\
97	0\\
98	0\\
99	0\\
100	0\\
101	0\\
102	0\\
103	0\\
104	0\\
105	0\\
106	0\\
107	0\\
108	0\\
109	0\\
110	0\\
111	0\\
112	0\\
113	0\\
114	0\\
115	0\\
116	0\\
117	0\\
118	0\\
119	0\\
120	0\\
121	0\\
122	0\\
123	0\\
124	0\\
125	0\\
126	0\\
127	0\\
128	0\\
129	0\\
130	0\\
131	0\\
132	0\\
133	0\\
134	0\\
135	0\\
136	0\\
137	0\\
138	0\\
139	0\\
140	0\\
141	0\\
142	0\\
143	0\\
144	0\\
145	0\\
146	0\\
147	0\\
148	0\\
149	0\\
150	0\\
151	0\\
152	0\\
153	0\\
154	0\\
155	0\\
156	0\\
157	0\\
158	0\\
159	0\\
160	0\\
161	0\\
162	0\\
163	0\\
164	0\\
165	0\\
166	0\\
167	0\\
168	0\\
169	0\\
170	0\\
171	0\\
172	0\\
173	0\\
174	0\\
175	0\\
176	0\\
177	0\\
178	0\\
179	0\\
180	0\\
181	0\\
182	0\\
183	0\\
184	0\\
185	0\\
186	0\\
187	0\\
188	0\\
189	0\\
190	0\\
191	0\\
192	0\\
193	0\\
194	0\\
195	0\\
196	0\\
197	0\\
198	0\\
199	0\\
200	0\\
201	0\\
202	0\\
203	0\\
204	0\\
205	0\\
206	0\\
207	0\\
208	0\\
209	0\\
210	0\\
211	0\\
212	0\\
213	0\\
214	0\\
215	0\\
216	0\\
217	0\\
218	0\\
219	0\\
220	0\\
221	0\\
222	0\\
223	0\\
224	0\\
225	0\\
226	0\\
227	0\\
228	0\\
229	0\\
230	0\\
231	0\\
232	0\\
233	0\\
234	0\\
235	0\\
236	0\\
237	0\\
238	0\\
239	0\\
240	0\\
241	0\\
242	0\\
243	0\\
244	0\\
245	0\\
246	0\\
247	0\\
248	0\\
249	0\\
250	0\\
251	0\\
252	0\\
253	0\\
254	0\\
255	0\\
256	0\\
257	0\\
258	0\\
259	0\\
260	0\\
261	0\\
262	0\\
263	0\\
264	0\\
265	0\\
266	0\\
267	0\\
268	0\\
269	0\\
270	0\\
271	0\\
272	0\\
273	0\\
274	0\\
275	0\\
276	0\\
277	0\\
278	0\\
279	0\\
280	0\\
281	0\\
282	0\\
283	0\\
284	0\\
285	0\\
286	0\\
287	0\\
288	0\\
289	0\\
290	0\\
291	0\\
292	0\\
293	0\\
294	0\\
295	0\\
296	0\\
297	0\\
298	0\\
299	0\\
300	0\\
301	0\\
302	0\\
303	0\\
304	0\\
305	0\\
306	0\\
307	0\\
308	0\\
309	0\\
310	0\\
311	0\\
312	0\\
313	0\\
314	0\\
315	0\\
316	0\\
317	0\\
318	0\\
319	0\\
320	0\\
321	0\\
322	0\\
323	0\\
324	0\\
325	0\\
326	0\\
327	0\\
328	0\\
329	0\\
330	0\\
331	0\\
332	0\\
333	0\\
334	0\\
335	0\\
336	0\\
337	0\\
338	0\\
339	0\\
340	0\\
341	0\\
342	0\\
343	0\\
344	0\\
345	0\\
346	0\\
347	0\\
348	0\\
349	0\\
350	0\\
351	0\\
352	0\\
353	0\\
354	0\\
355	0\\
356	0\\
357	0\\
358	0\\
359	0\\
360	0\\
361	0\\
362	0\\
363	0\\
364	0\\
365	0\\
366	0\\
367	0\\
368	0\\
369	0\\
370	0\\
371	0\\
372	0\\
373	0\\
374	0\\
375	0\\
376	0\\
377	0\\
378	0\\
379	0\\
380	0\\
381	0\\
382	0\\
383	0\\
384	0\\
385	0\\
386	0\\
387	0\\
388	0\\
389	0\\
390	0\\
391	0\\
392	0\\
393	0\\
394	0\\
395	0\\
396	0\\
397	0\\
398	0\\
399	0\\
400	0\\
401	0\\
402	0\\
403	0\\
404	0\\
405	0\\
406	0\\
407	0\\
408	0\\
409	0\\
410	0\\
411	0\\
412	0\\
413	0\\
414	0\\
415	0\\
416	0\\
417	0\\
418	0\\
419	0\\
420	0\\
421	0\\
422	0\\
423	0\\
424	0\\
425	0\\
426	0\\
427	0\\
428	0\\
429	0\\
430	0\\
431	0\\
432	0\\
433	0\\
434	0\\
435	0\\
436	0\\
437	0\\
438	0\\
439	0\\
440	0\\
441	0\\
442	0\\
443	0\\
444	0\\
445	0\\
446	0\\
447	0\\
448	0\\
449	0\\
450	0\\
451	0\\
452	0\\
453	0\\
454	0\\
455	0\\
456	0\\
457	0\\
458	0\\
459	0\\
460	0\\
461	0\\
462	0\\
463	0\\
464	0\\
465	0\\
466	0\\
467	0\\
468	0\\
469	0\\
470	0\\
471	0\\
472	0\\
473	0\\
474	0\\
475	0\\
476	0\\
477	0\\
478	0\\
479	0\\
480	0\\
481	0\\
482	0\\
483	0\\
484	0\\
485	0\\
486	0\\
487	0\\
488	0\\
489	0\\
490	0\\
491	0\\
492	0\\
493	0\\
494	0\\
495	0\\
496	0\\
497	0\\
498	0\\
499	0\\
500	0\\
501	0\\
502	0\\
503	0\\
504	0\\
505	0\\
506	0\\
507	0\\
508	0\\
509	0\\
510	0\\
511	0\\
512	0\\
513	0\\
514	0\\
515	0\\
516	0\\
517	0\\
518	0\\
519	0\\
520	0\\
521	0\\
522	0\\
523	0\\
524	0\\
525	0\\
526	0\\
527	0\\
528	0\\
529	0\\
530	0\\
531	0\\
532	0\\
533	0\\
534	0\\
535	0\\
536	0\\
537	1.41981037307091e-05\\
538	4.03194267452914e-05\\
539	6.68049003892867e-05\\
540	9.36455912896163e-05\\
541	0.000121031570767503\\
542	0.000148981487706946\\
543	0.000177511834873603\\
544	0.000206636133043701\\
545	0.00023636795016899\\
546	0.000266721132093178\\
547	0.000297709825439374\\
548	0.000329348337773177\\
549	0.000361650643511393\\
550	0.000394628923195159\\
551	0.000428289376922921\\
552	0.000462620483140471\\
553	0.000497556235318828\\
554	0.000534404821662524\\
555	0.00057682151016282\\
556	0.000612670727538693\\
557	0.000647448601386753\\
558	0.000682535085499909\\
559	0.000718276611240496\\
560	0.000754689045844615\\
561	0.000791786693129664\\
562	0.000829584407608802\\
563	0.000868098067449268\\
564	0.00090734463744055\\
565	0.00128398061588739\\
566	0.00159228887629074\\
567	0.00166875568695174\\
568	0.00173042230411937\\
569	0.00179304268794214\\
570	0.00185663517097841\\
571	0.00192121867131843\\
572	0.00198681272227957\\
573	0.0020534375008095\\
574	0.00212111385632287\\
575	0.00218986333969204\\
576	0.00225970823205636\\
577	0.00233067157312388\\
578	0.00240277718884235\\
579	0.00247604971905763\\
580	0.00255051464798387\\
581	0.00262619834639329\\
582	0.00270312815070207\\
583	0.00278133254700611\\
584	0.00286084164033729\\
585	0.00294168838198347\\
586	0.00302391178876898\\
587	0.00310756536466311\\
588	0.0031927390611031\\
589	0.00327961639013215\\
590	0.00336862265377153\\
591	0.00346039684658503\\
592	0.00355653402929193\\
593	0.00366119131364762\\
594	0.00378527046594782\\
595	0.00395742119285503\\
596	0.00425297862742963\\
597	0.0048700418989439\\
598	0.00632942537858856\\
599	0\\
600	0\\
};
\addplot [color=black!80!mycolor21,solid,forget plot]
  table[row sep=crcr]{%
1	0\\
2	0\\
3	0\\
4	0\\
5	0\\
6	0\\
7	0\\
8	0\\
9	0\\
10	0\\
11	0\\
12	0\\
13	0\\
14	0\\
15	0\\
16	0\\
17	0\\
18	0\\
19	0\\
20	0\\
21	0\\
22	0\\
23	0\\
24	0\\
25	0\\
26	0\\
27	0\\
28	0\\
29	0\\
30	0\\
31	0\\
32	0\\
33	0\\
34	0\\
35	0\\
36	0\\
37	0\\
38	0\\
39	0\\
40	0\\
41	0\\
42	0\\
43	0\\
44	0\\
45	0\\
46	0\\
47	0\\
48	0\\
49	0\\
50	0\\
51	0\\
52	0\\
53	0\\
54	0\\
55	0\\
56	0\\
57	0\\
58	0\\
59	0\\
60	0\\
61	0\\
62	0\\
63	0\\
64	0\\
65	0\\
66	0\\
67	0\\
68	0\\
69	0\\
70	0\\
71	0\\
72	0\\
73	0\\
74	0\\
75	0\\
76	0\\
77	0\\
78	0\\
79	0\\
80	0\\
81	0\\
82	0\\
83	0\\
84	0\\
85	0\\
86	0\\
87	0\\
88	0\\
89	0\\
90	0\\
91	0\\
92	0\\
93	0\\
94	0\\
95	0\\
96	0\\
97	0\\
98	0\\
99	0\\
100	0\\
101	0\\
102	0\\
103	0\\
104	0\\
105	0\\
106	0\\
107	0\\
108	0\\
109	0\\
110	0\\
111	0\\
112	0\\
113	0\\
114	0\\
115	0\\
116	0\\
117	0\\
118	0\\
119	0\\
120	0\\
121	0\\
122	0\\
123	0\\
124	0\\
125	0\\
126	0\\
127	0\\
128	0\\
129	0\\
130	0\\
131	0\\
132	0\\
133	0\\
134	0\\
135	0\\
136	0\\
137	0\\
138	0\\
139	0\\
140	0\\
141	0\\
142	0\\
143	0\\
144	0\\
145	0\\
146	0\\
147	0\\
148	0\\
149	0\\
150	0\\
151	0\\
152	0\\
153	0\\
154	0\\
155	0\\
156	0\\
157	0\\
158	0\\
159	0\\
160	0\\
161	0\\
162	0\\
163	0\\
164	0\\
165	0\\
166	0\\
167	0\\
168	0\\
169	0\\
170	0\\
171	0\\
172	0\\
173	0\\
174	0\\
175	0\\
176	0\\
177	0\\
178	0\\
179	0\\
180	0\\
181	0\\
182	0\\
183	0\\
184	0\\
185	0\\
186	0\\
187	0\\
188	0\\
189	0\\
190	0\\
191	0\\
192	0\\
193	0\\
194	0\\
195	0\\
196	0\\
197	0\\
198	0\\
199	0\\
200	0\\
201	0\\
202	0\\
203	0\\
204	0\\
205	0\\
206	0\\
207	0\\
208	0\\
209	0\\
210	0\\
211	0\\
212	0\\
213	0\\
214	0\\
215	0\\
216	0\\
217	0\\
218	0\\
219	0\\
220	0\\
221	0\\
222	0\\
223	0\\
224	0\\
225	0\\
226	0\\
227	0\\
228	0\\
229	0\\
230	0\\
231	0\\
232	0\\
233	0\\
234	0\\
235	0\\
236	0\\
237	0\\
238	0\\
239	0\\
240	0\\
241	0\\
242	0\\
243	0\\
244	0\\
245	0\\
246	0\\
247	0\\
248	0\\
249	0\\
250	0\\
251	0\\
252	0\\
253	0\\
254	0\\
255	0\\
256	0\\
257	0\\
258	0\\
259	0\\
260	0\\
261	0\\
262	0\\
263	0\\
264	0\\
265	0\\
266	0\\
267	0\\
268	0\\
269	0\\
270	0\\
271	0\\
272	0\\
273	0\\
274	0\\
275	0\\
276	0\\
277	0\\
278	0\\
279	0\\
280	0\\
281	0\\
282	0\\
283	0\\
284	0\\
285	0\\
286	0\\
287	0\\
288	0\\
289	0\\
290	0\\
291	0\\
292	0\\
293	0\\
294	0\\
295	0\\
296	0\\
297	0\\
298	0\\
299	0\\
300	0\\
301	0\\
302	0\\
303	0\\
304	0\\
305	0\\
306	0\\
307	0\\
308	0\\
309	0\\
310	0\\
311	0\\
312	0\\
313	0\\
314	0\\
315	0\\
316	0\\
317	0\\
318	0\\
319	0\\
320	0\\
321	0\\
322	0\\
323	0\\
324	0\\
325	0\\
326	0\\
327	0\\
328	0\\
329	0\\
330	0\\
331	0\\
332	0\\
333	0\\
334	0\\
335	0\\
336	0\\
337	0\\
338	0\\
339	0\\
340	0\\
341	0\\
342	0\\
343	0\\
344	0\\
345	0\\
346	0\\
347	0\\
348	0\\
349	0\\
350	0\\
351	0\\
352	0\\
353	0\\
354	0\\
355	0\\
356	0\\
357	0\\
358	0\\
359	0\\
360	0\\
361	0\\
362	0\\
363	0\\
364	0\\
365	0\\
366	0\\
367	0\\
368	0\\
369	0\\
370	0\\
371	0\\
372	0\\
373	0\\
374	0\\
375	0\\
376	0\\
377	0\\
378	0\\
379	0\\
380	0\\
381	0\\
382	0\\
383	0\\
384	0\\
385	0\\
386	0\\
387	0\\
388	0\\
389	0\\
390	0\\
391	0\\
392	0\\
393	0\\
394	0\\
395	0\\
396	0\\
397	0\\
398	0\\
399	0\\
400	0\\
401	0\\
402	0\\
403	0\\
404	0\\
405	0\\
406	0\\
407	0\\
408	0\\
409	0\\
410	0\\
411	0\\
412	0\\
413	0\\
414	0\\
415	0\\
416	0\\
417	0\\
418	0\\
419	0\\
420	0\\
421	0\\
422	0\\
423	0\\
424	0\\
425	0\\
426	0\\
427	0\\
428	0\\
429	0\\
430	0\\
431	0\\
432	0\\
433	0\\
434	0\\
435	0\\
436	0\\
437	0\\
438	0\\
439	0\\
440	0\\
441	0\\
442	0\\
443	0\\
444	0\\
445	0\\
446	0\\
447	0\\
448	0\\
449	0\\
450	0\\
451	0\\
452	0\\
453	0\\
454	0\\
455	0\\
456	0\\
457	0\\
458	0\\
459	0\\
460	0\\
461	0\\
462	0\\
463	0\\
464	0\\
465	0\\
466	0\\
467	0\\
468	0\\
469	0\\
470	0\\
471	0\\
472	0\\
473	0\\
474	0\\
475	0\\
476	0\\
477	0\\
478	0\\
479	0\\
480	0\\
481	0\\
482	0\\
483	0\\
484	0\\
485	0\\
486	0\\
487	0\\
488	0\\
489	0\\
490	0\\
491	0\\
492	0\\
493	0\\
494	0\\
495	0\\
496	0\\
497	0\\
498	0\\
499	0\\
500	0\\
501	0\\
502	0\\
503	0\\
504	0\\
505	0\\
506	0\\
507	0\\
508	0\\
509	0\\
510	0\\
511	0\\
512	0\\
513	0\\
514	0\\
515	0\\
516	0\\
517	0\\
518	0\\
519	0\\
520	0\\
521	0\\
522	0\\
523	0\\
524	0\\
525	0\\
526	0\\
527	0\\
528	0\\
529	0\\
530	0\\
531	0\\
532	0\\
533	0\\
534	0\\
535	0\\
536	0\\
537	2.00311818478477e-05\\
538	4.58947831388607e-05\\
539	7.22859642090788e-05\\
540	9.92227167136692e-05\\
541	0.000126717645174376\\
542	0.000154783396446169\\
543	0.000183432792668513\\
544	0.000212678988439243\\
545	0.000242535495908309\\
546	0.000273016183234799\\
547	0.000304135226905855\\
548	0.000335906938006349\\
549	0.000368345226021222\\
550	0.000401462007738327\\
551	0.00043526250918298\\
552	0.00046973121451101\\
553	0.000507806233402199\\
554	0.000547931958570056\\
555	0.000581827365374636\\
556	0.000615782626155859\\
557	0.000650269640106297\\
558	0.000685400368111508\\
559	0.000721188332977737\\
560	0.000757647340897565\\
561	0.00079479173610764\\
562	0.000832636728120108\\
563	0.000871198966466937\\
564	0.00118295200706542\\
565	0.00153193885094584\\
566	0.00160802507255323\\
567	0.00166875568906997\\
568	0.00173042230418516\\
569	0.0017930426879713\\
570	0.00185663517099282\\
571	0.00192121867132536\\
572	0.00198681272228273\\
573	0.00205343750081087\\
574	0.00212111385632341\\
575	0.00218986333969228\\
576	0.00225970823205644\\
577	0.0023306715731239\\
578	0.00240277718884234\\
579	0.00247604971905764\\
580	0.00255051464798387\\
581	0.00262619834639331\\
582	0.00270312815070208\\
583	0.00278133254700611\\
584	0.00286084164033728\\
585	0.00294168838198347\\
586	0.003023911788769\\
587	0.00310756536466313\\
588	0.00319273906110312\\
589	0.00327961639013215\\
590	0.00336862265377153\\
591	0.00346039684658503\\
592	0.00355653402929195\\
593	0.00366119131364764\\
594	0.00378527046594783\\
595	0.00395742119285504\\
596	0.00425297862742962\\
597	0.00487004189894391\\
598	0.00632942537858856\\
599	0\\
600	0\\
};
\addplot [color=black,solid,forget plot]
  table[row sep=crcr]{%
1	0\\
2	0\\
3	0\\
4	0\\
5	0\\
6	0\\
7	0\\
8	0\\
9	0\\
10	0\\
11	0\\
12	0\\
13	0\\
14	0\\
15	0\\
16	0\\
17	0\\
18	0\\
19	0\\
20	0\\
21	0\\
22	0\\
23	0\\
24	0\\
25	0\\
26	0\\
27	0\\
28	0\\
29	0\\
30	0\\
31	0\\
32	0\\
33	0\\
34	0\\
35	0\\
36	0\\
37	0\\
38	0\\
39	0\\
40	0\\
41	0\\
42	0\\
43	0\\
44	0\\
45	0\\
46	0\\
47	0\\
48	0\\
49	0\\
50	0\\
51	0\\
52	0\\
53	0\\
54	0\\
55	0\\
56	0\\
57	0\\
58	0\\
59	0\\
60	0\\
61	0\\
62	0\\
63	0\\
64	0\\
65	0\\
66	0\\
67	0\\
68	0\\
69	0\\
70	0\\
71	0\\
72	0\\
73	0\\
74	0\\
75	0\\
76	0\\
77	0\\
78	0\\
79	0\\
80	0\\
81	0\\
82	0\\
83	0\\
84	0\\
85	0\\
86	0\\
87	0\\
88	0\\
89	0\\
90	0\\
91	0\\
92	0\\
93	0\\
94	0\\
95	0\\
96	0\\
97	0\\
98	0\\
99	0\\
100	0\\
101	0\\
102	0\\
103	0\\
104	0\\
105	0\\
106	0\\
107	0\\
108	0\\
109	0\\
110	0\\
111	0\\
112	0\\
113	0\\
114	0\\
115	0\\
116	0\\
117	0\\
118	0\\
119	0\\
120	0\\
121	0\\
122	0\\
123	0\\
124	0\\
125	0\\
126	0\\
127	0\\
128	0\\
129	0\\
130	0\\
131	0\\
132	0\\
133	0\\
134	0\\
135	0\\
136	0\\
137	0\\
138	0\\
139	0\\
140	0\\
141	0\\
142	0\\
143	0\\
144	0\\
145	0\\
146	0\\
147	0\\
148	0\\
149	0\\
150	0\\
151	0\\
152	0\\
153	0\\
154	0\\
155	0\\
156	0\\
157	0\\
158	0\\
159	0\\
160	0\\
161	0\\
162	0\\
163	0\\
164	0\\
165	0\\
166	0\\
167	0\\
168	0\\
169	0\\
170	0\\
171	0\\
172	0\\
173	0\\
174	0\\
175	0\\
176	0\\
177	0\\
178	0\\
179	0\\
180	0\\
181	0\\
182	0\\
183	0\\
184	0\\
185	0\\
186	0\\
187	0\\
188	0\\
189	0\\
190	0\\
191	0\\
192	0\\
193	0\\
194	0\\
195	0\\
196	0\\
197	0\\
198	0\\
199	0\\
200	0\\
201	0\\
202	0\\
203	0\\
204	0\\
205	0\\
206	0\\
207	0\\
208	0\\
209	0\\
210	0\\
211	0\\
212	0\\
213	0\\
214	0\\
215	0\\
216	0\\
217	0\\
218	0\\
219	0\\
220	0\\
221	0\\
222	0\\
223	0\\
224	0\\
225	0\\
226	0\\
227	0\\
228	0\\
229	0\\
230	0\\
231	0\\
232	0\\
233	0\\
234	0\\
235	0\\
236	0\\
237	0\\
238	0\\
239	0\\
240	0\\
241	0\\
242	0\\
243	0\\
244	0\\
245	0\\
246	0\\
247	0\\
248	0\\
249	0\\
250	0\\
251	0\\
252	0\\
253	0\\
254	0\\
255	0\\
256	0\\
257	0\\
258	0\\
259	0\\
260	0\\
261	0\\
262	0\\
263	0\\
264	0\\
265	0\\
266	0\\
267	0\\
268	0\\
269	0\\
270	0\\
271	0\\
272	0\\
273	0\\
274	0\\
275	0\\
276	0\\
277	0\\
278	0\\
279	0\\
280	0\\
281	0\\
282	0\\
283	0\\
284	0\\
285	0\\
286	0\\
287	0\\
288	0\\
289	0\\
290	0\\
291	0\\
292	0\\
293	0\\
294	0\\
295	0\\
296	0\\
297	0\\
298	0\\
299	0\\
300	0\\
301	0\\
302	0\\
303	0\\
304	0\\
305	0\\
306	0\\
307	0\\
308	0\\
309	0\\
310	0\\
311	0\\
312	0\\
313	0\\
314	0\\
315	0\\
316	0\\
317	0\\
318	0\\
319	0\\
320	0\\
321	0\\
322	0\\
323	0\\
324	0\\
325	0\\
326	0\\
327	0\\
328	0\\
329	0\\
330	0\\
331	0\\
332	0\\
333	0\\
334	0\\
335	0\\
336	0\\
337	0\\
338	0\\
339	0\\
340	0\\
341	0\\
342	0\\
343	0\\
344	0\\
345	0\\
346	0\\
347	0\\
348	0\\
349	0\\
350	0\\
351	0\\
352	0\\
353	0\\
354	0\\
355	0\\
356	0\\
357	0\\
358	0\\
359	0\\
360	0\\
361	0\\
362	0\\
363	0\\
364	0\\
365	0\\
366	0\\
367	0\\
368	0\\
369	0\\
370	0\\
371	0\\
372	0\\
373	0\\
374	0\\
375	0\\
376	0\\
377	0\\
378	0\\
379	0\\
380	0\\
381	0\\
382	0\\
383	0\\
384	0\\
385	0\\
386	0\\
387	0\\
388	0\\
389	0\\
390	0\\
391	0\\
392	0\\
393	0\\
394	0\\
395	0\\
396	0\\
397	0\\
398	0\\
399	0\\
400	0\\
401	0\\
402	0\\
403	0\\
404	0\\
405	0\\
406	0\\
407	0\\
408	0\\
409	0\\
410	0\\
411	0\\
412	0\\
413	0\\
414	0\\
415	0\\
416	0\\
417	0\\
418	0\\
419	0\\
420	0\\
421	0\\
422	0\\
423	0\\
424	0\\
425	0\\
426	0\\
427	0\\
428	0\\
429	0\\
430	0\\
431	0\\
432	0\\
433	0\\
434	0\\
435	0\\
436	0\\
437	0\\
438	0\\
439	0\\
440	0\\
441	0\\
442	0\\
443	0\\
444	0\\
445	0\\
446	0\\
447	0\\
448	0\\
449	0\\
450	0\\
451	0\\
452	0\\
453	0\\
454	0\\
455	0\\
456	0\\
457	0\\
458	0\\
459	0\\
460	0\\
461	0\\
462	0\\
463	0\\
464	0\\
465	0\\
466	0\\
467	0\\
468	0\\
469	0\\
470	0\\
471	0\\
472	0\\
473	0\\
474	0\\
475	0\\
476	0\\
477	0\\
478	0\\
479	0\\
480	0\\
481	0\\
482	0\\
483	0\\
484	0\\
485	0\\
486	0\\
487	0\\
488	0\\
489	0\\
490	0\\
491	0\\
492	0\\
493	0\\
494	0\\
495	0\\
496	0\\
497	0\\
498	0\\
499	0\\
500	0\\
501	0\\
502	0\\
503	0\\
504	0\\
505	0\\
506	0\\
507	0\\
508	0\\
509	0\\
510	0\\
511	0\\
512	0\\
513	0\\
514	0\\
515	0\\
516	0\\
517	0\\
518	0\\
519	0\\
520	0\\
521	0\\
522	0\\
523	0\\
524	0\\
525	0\\
526	0\\
527	0\\
528	0\\
529	0\\
530	0\\
531	0\\
532	0\\
533	0\\
534	0\\
535	0\\
536	0\\
537	2.41449941958518e-05\\
538	5.01062475752778e-05\\
539	7.66056323086619e-05\\
540	0.000103655192469625\\
541	0.000131267283450877\\
542	0.000159454603022728\\
543	0.000188230216867344\\
544	0.0002176075760922\\
545	0.000247600530527473\\
546	0.000278223330210997\\
547	0.000309490592847296\\
548	0.000341417179393022\\
549	0.000374017835784068\\
550	0.000407306284469521\\
551	0.000441293207629602\\
552	0.000480387614922689\\
553	0.000517733822270981\\
554	0.000550900656801395\\
555	0.000584195800718177\\
556	0.000618106994392449\\
557	0.000652650864010514\\
558	0.000687840631956863\\
559	0.000723689894931791\\
560	0.000760212780169657\\
561	0.000797424370009976\\
562	0.000835341903111315\\
563	0.00104794067987499\\
564	0.0014700571270915\\
565	0.00154821321979401\\
566	0.00160802507270126\\
567	0.00166875568907878\\
568	0.00173042230418922\\
569	0.00179304268797327\\
570	0.00185663517099374\\
571	0.00192121867132575\\
572	0.00198681272228291\\
573	0.00205343750081092\\
574	0.00212111385632343\\
575	0.00218986333969225\\
576	0.00225970823205644\\
577	0.0023306715731239\\
578	0.00240277718884235\\
579	0.00247604971905765\\
580	0.00255051464798389\\
581	0.0026261983463933\\
582	0.00270312815070208\\
583	0.00278133254700612\\
584	0.00286084164033729\\
585	0.00294168838198346\\
586	0.00302391178876899\\
587	0.00310756536466312\\
588	0.00319273906110312\\
589	0.00327961639013216\\
590	0.00336862265377154\\
591	0.00346039684658502\\
592	0.00355653402929193\\
593	0.00366119131364762\\
594	0.00378527046594782\\
595	0.00395742119285502\\
596	0.00425297862742962\\
597	0.0048700418989439\\
598	0.00632942537858856\\
599	0\\
600	0\\
};
\end{axis}
\end{tikzpicture}% 
  \caption{Discrete Time}
\end{subfigure}\\
\vspace{1cm}
\begin{subfigure}{.45\linewidth}
  \centering
  \setlength\figureheight{\linewidth} 
  \setlength\figurewidth{\linewidth}
  \tikzsetnextfilename{dp_cts_nFPC_z15}
  % This file was created by matlab2tikz.
%
%The latest updates can be retrieved from
%  http://www.mathworks.com/matlabcentral/fileexchange/22022-matlab2tikz-matlab2tikz
%where you can also make suggestions and rate matlab2tikz.
%
\definecolor{mycolor1}{rgb}{1.00000,0.00000,1.00000}%
%
\begin{tikzpicture}

\begin{axis}[%
width=4.564in,
height=3.803in,
at={(1.067in,0.513in)},
scale only axis,
every outer x axis line/.append style={black},
every x tick label/.append style={font=\color{black}},
xmin=0,
xmax=100,
xlabel={Time},
every outer y axis line/.append style={black},
every y tick label/.append style={font=\color{black}},
ymin=0,
ymax=0.012,
ylabel={Depth $\delta$},
axis background/.style={fill=white},
title={Z=15},
axis x line*=bottom,
axis y line*=left,
legend style={legend cell align=left,align=left,draw=black}
]
\addplot [color=green,dashed,forget plot]
  table[row sep=crcr]{%
0.01	0.01\\
0.02	0.01\\
0.03	0.01\\
0.04	0.01\\
0.05	0.01\\
0.06	0.01\\
0.07	0.01\\
0.08	0.01\\
0.09	0.01\\
0.1	0.01\\
0.11	0.01\\
0.12	0.01\\
0.13	0.01\\
0.14	0.01\\
0.15	0.01\\
0.16	0.01\\
0.17	0.01\\
0.18	0.01\\
0.19	0.01\\
0.2	0.01\\
0.21	0.01\\
0.22	0.01\\
0.23	0.01\\
0.24	0.01\\
0.25	0.01\\
0.26	0.01\\
0.27	0.01\\
0.28	0.01\\
0.29	0.01\\
0.3	0.01\\
0.31	0.01\\
0.32	0.01\\
0.33	0.01\\
0.34	0.01\\
0.35	0.01\\
0.36	0.01\\
0.37	0.01\\
0.38	0.01\\
0.39	0.01\\
0.4	0.01\\
0.41	0.01\\
0.42	0.01\\
0.43	0.01\\
0.44	0.01\\
0.45	0.01\\
0.46	0.01\\
0.47	0.01\\
0.48	0.01\\
0.49	0.01\\
0.5	0.01\\
0.51	0.01\\
0.52	0.01\\
0.53	0.01\\
0.54	0.01\\
0.55	0.01\\
0.56	0.01\\
0.57	0.01\\
0.58	0.01\\
0.59	0.01\\
0.6	0.01\\
0.61	0.01\\
0.62	0.01\\
0.63	0.01\\
0.64	0.01\\
0.65	0.01\\
0.66	0.01\\
0.67	0.01\\
0.68	0.01\\
0.69	0.01\\
0.7	0.01\\
0.71	0.01\\
0.72	0.01\\
0.73	0.01\\
0.74	0.01\\
0.75	0.01\\
0.76	0.01\\
0.77	0.01\\
0.78	0.01\\
0.79	0.01\\
0.8	0.01\\
0.81	0.01\\
0.82	0.01\\
0.83	0.01\\
0.84	0.01\\
0.85	0.01\\
0.86	0.01\\
0.87	0.01\\
0.88	0.01\\
0.89	0.01\\
0.9	0.01\\
0.91	0.01\\
0.92	0.01\\
0.93	0.01\\
0.94	0.01\\
0.95	0.01\\
0.96	0.01\\
0.97	0.01\\
0.98	0.01\\
0.99	0.01\\
1	0.01\\
1.01	0.01\\
1.02	0.01\\
1.03	0.01\\
1.04	0.01\\
1.05	0.01\\
1.06	0.01\\
1.07	0.01\\
1.08	0.01\\
1.09	0.01\\
1.1	0.01\\
1.11	0.01\\
1.12	0.01\\
1.13	0.01\\
1.14	0.01\\
1.15	0.01\\
1.16	0.01\\
1.17	0.01\\
1.18	0.01\\
1.19	0.01\\
1.2	0.01\\
1.21	0.01\\
1.22	0.01\\
1.23	0.01\\
1.24	0.01\\
1.25	0.01\\
1.26	0.01\\
1.27	0.01\\
1.28	0.01\\
1.29	0.01\\
1.3	0.01\\
1.31	0.01\\
1.32	0.01\\
1.33	0.01\\
1.34	0.01\\
1.35	0.01\\
1.36	0.01\\
1.37	0.01\\
1.38	0.01\\
1.39	0.01\\
1.4	0.01\\
1.41	0.01\\
1.42	0.01\\
1.43	0.01\\
1.44	0.01\\
1.45	0.01\\
1.46	0.01\\
1.47	0.01\\
1.48	0.01\\
1.49	0.01\\
1.5	0.01\\
1.51	0.01\\
1.52	0.01\\
1.53	0.01\\
1.54	0.01\\
1.55	0.01\\
1.56	0.01\\
1.57	0.01\\
1.58	0.01\\
1.59	0.01\\
1.6	0.01\\
1.61	0.01\\
1.62	0.01\\
1.63	0.01\\
1.64	0.01\\
1.65	0.01\\
1.66	0.01\\
1.67	0.01\\
1.68	0.01\\
1.69	0.01\\
1.7	0.01\\
1.71	0.01\\
1.72	0.01\\
1.73	0.01\\
1.74	0.01\\
1.75	0.01\\
1.76	0.01\\
1.77	0.01\\
1.78	0.01\\
1.79	0.01\\
1.8	0.01\\
1.81	0.01\\
1.82	0.01\\
1.83	0.01\\
1.84	0.01\\
1.85	0.01\\
1.86	0.01\\
1.87	0.01\\
1.88	0.01\\
1.89	0.01\\
1.9	0.01\\
1.91	0.01\\
1.92	0.01\\
1.93	0.01\\
1.94	0.01\\
1.95	0.01\\
1.96	0.01\\
1.97	0.01\\
1.98	0.01\\
1.99	0.01\\
2	0.01\\
2.01	0.01\\
2.02	0.01\\
2.03	0.01\\
2.04	0.01\\
2.05	0.01\\
2.06	0.01\\
2.07	0.01\\
2.08	0.01\\
2.09	0.01\\
2.1	0.01\\
2.11	0.01\\
2.12	0.01\\
2.13	0.01\\
2.14	0.01\\
2.15	0.01\\
2.16	0.01\\
2.17	0.01\\
2.18	0.01\\
2.19	0.01\\
2.2	0.01\\
2.21	0.01\\
2.22	0.01\\
2.23	0.01\\
2.24	0.01\\
2.25	0.01\\
2.26	0.01\\
2.27	0.01\\
2.28	0.01\\
2.29	0.01\\
2.3	0.01\\
2.31	0.01\\
2.32	0.01\\
2.33	0.01\\
2.34	0.01\\
2.35	0.01\\
2.36	0.01\\
2.37	0.01\\
2.38	0.01\\
2.39	0.01\\
2.4	0.01\\
2.41	0.01\\
2.42	0.01\\
2.43	0.01\\
2.44	0.01\\
2.45	0.01\\
2.46	0.01\\
2.47	0.01\\
2.48	0.01\\
2.49	0.01\\
2.5	0.01\\
2.51	0.01\\
2.52	0.01\\
2.53	0.01\\
2.54	0.01\\
2.55	0.01\\
2.56	0.01\\
2.57	0.01\\
2.58	0.01\\
2.59	0.01\\
2.6	0.01\\
2.61	0.01\\
2.62	0.01\\
2.63	0.01\\
2.64	0.01\\
2.65	0.01\\
2.66	0.01\\
2.67	0.01\\
2.68	0.01\\
2.69	0.01\\
2.7	0.01\\
2.71	0.01\\
2.72	0.01\\
2.73	0.01\\
2.74	0.01\\
2.75	0.01\\
2.76	0.01\\
2.77	0.01\\
2.78	0.01\\
2.79	0.01\\
2.8	0.01\\
2.81	0.01\\
2.82	0.01\\
2.83	0.01\\
2.84	0.01\\
2.85	0.01\\
2.86	0.01\\
2.87	0.01\\
2.88	0.01\\
2.89	0.01\\
2.9	0.01\\
2.91	0.01\\
2.92	0.01\\
2.93	0.01\\
2.94	0.01\\
2.95	0.01\\
2.96	0.01\\
2.97	0.01\\
2.98	0.01\\
2.99	0.01\\
3	0.01\\
3.01	0.01\\
3.02	0.01\\
3.03	0.01\\
3.04	0.01\\
3.05	0.01\\
3.06	0.01\\
3.07	0.01\\
3.08	0.01\\
3.09	0.01\\
3.1	0.01\\
3.11	0.01\\
3.12	0.01\\
3.13	0.01\\
3.14	0.01\\
3.15	0.01\\
3.16	0.01\\
3.17	0.01\\
3.18	0.01\\
3.19	0.01\\
3.2	0.01\\
3.21	0.01\\
3.22	0.01\\
3.23	0.01\\
3.24	0.01\\
3.25	0.01\\
3.26	0.01\\
3.27	0.01\\
3.28	0.01\\
3.29	0.01\\
3.3	0.01\\
3.31	0.01\\
3.32	0.01\\
3.33	0.01\\
3.34	0.01\\
3.35	0.01\\
3.36	0.01\\
3.37	0.01\\
3.38	0.01\\
3.39	0.01\\
3.4	0.01\\
3.41	0.01\\
3.42	0.01\\
3.43	0.01\\
3.44	0.01\\
3.45	0.01\\
3.46	0.01\\
3.47	0.01\\
3.48	0.01\\
3.49	0.01\\
3.5	0.01\\
3.51	0.01\\
3.52	0.01\\
3.53	0.01\\
3.54	0.01\\
3.55	0.01\\
3.56	0.01\\
3.57	0.01\\
3.58	0.01\\
3.59	0.01\\
3.6	0.01\\
3.61	0.01\\
3.62	0.01\\
3.63	0.01\\
3.64	0.01\\
3.65	0.01\\
3.66	0.01\\
3.67	0.01\\
3.68	0.01\\
3.69	0.01\\
3.7	0.01\\
3.71	0.01\\
3.72	0.01\\
3.73	0.01\\
3.74	0.01\\
3.75	0.01\\
3.76	0.01\\
3.77	0.01\\
3.78	0.01\\
3.79	0.01\\
3.8	0.01\\
3.81	0.01\\
3.82	0.01\\
3.83	0.01\\
3.84	0.01\\
3.85	0.01\\
3.86	0.01\\
3.87	0.01\\
3.88	0.01\\
3.89	0.01\\
3.9	0.01\\
3.91	0.01\\
3.92	0.01\\
3.93	0.01\\
3.94	0.01\\
3.95	0.01\\
3.96	0.01\\
3.97	0.01\\
3.98	0.01\\
3.99	0.01\\
4	0.01\\
4.01	0.01\\
4.02	0.01\\
4.03	0.01\\
4.04	0.01\\
4.05	0.01\\
4.06	0.01\\
4.07	0.01\\
4.08	0.01\\
4.09	0.01\\
4.1	0.01\\
4.11	0.01\\
4.12	0.01\\
4.13	0.01\\
4.14	0.01\\
4.15	0.01\\
4.16	0.01\\
4.17	0.01\\
4.18	0.01\\
4.19	0.01\\
4.2	0.01\\
4.21	0.01\\
4.22	0.01\\
4.23	0.01\\
4.24	0.01\\
4.25	0.01\\
4.26	0.01\\
4.27	0.01\\
4.28	0.01\\
4.29	0.01\\
4.3	0.01\\
4.31	0.01\\
4.32	0.01\\
4.33	0.01\\
4.34	0.01\\
4.35	0.01\\
4.36	0.01\\
4.37	0.01\\
4.38	0.01\\
4.39	0.01\\
4.4	0.01\\
4.41	0.01\\
4.42	0.01\\
4.43	0.01\\
4.44	0.01\\
4.45	0.01\\
4.46	0.01\\
4.47	0.01\\
4.48	0.01\\
4.49	0.01\\
4.5	0.01\\
4.51	0.01\\
4.52	0.01\\
4.53	0.01\\
4.54	0.01\\
4.55	0.01\\
4.56	0.01\\
4.57	0.01\\
4.58	0.01\\
4.59	0.01\\
4.6	0.01\\
4.61	0.01\\
4.62	0.01\\
4.63	0.01\\
4.64	0.01\\
4.65	0.01\\
4.66	0.01\\
4.67	0.01\\
4.68	0.01\\
4.69	0.01\\
4.7	0.01\\
4.71	0.01\\
4.72	0.01\\
4.73	0.01\\
4.74	0.01\\
4.75	0.01\\
4.76	0.01\\
4.77	0.01\\
4.78	0.01\\
4.79	0.01\\
4.8	0.01\\
4.81	0.01\\
4.82	0.01\\
4.83	0.01\\
4.84	0.01\\
4.85	0.01\\
4.86	0.01\\
4.87	0.01\\
4.88	0.01\\
4.89	0.01\\
4.9	0.01\\
4.91	0.01\\
4.92	0.01\\
4.93	0.01\\
4.94	0.01\\
4.95	0.01\\
4.96	0.01\\
4.97	0.01\\
4.98	0.01\\
4.99	0.01\\
5	0.01\\
5.01	0.01\\
5.02	0.01\\
5.03	0.01\\
5.04	0.01\\
5.05	0.01\\
5.06	0.01\\
5.07	0.01\\
5.08	0.01\\
5.09	0.01\\
5.1	0.01\\
5.11	0.01\\
5.12	0.01\\
5.13	0.01\\
5.14	0.01\\
5.15	0.01\\
5.16	0.01\\
5.17	0.01\\
5.18	0.01\\
5.19	0.01\\
5.2	0.01\\
5.21	0.01\\
5.22	0.01\\
5.23	0.01\\
5.24	0.01\\
5.25	0.01\\
5.26	0.01\\
5.27	0.01\\
5.28	0.01\\
5.29	0.01\\
5.3	0.01\\
5.31	0.01\\
5.32	0.01\\
5.33	0.01\\
5.34	0.01\\
5.35	0.01\\
5.36	0.01\\
5.37	0.01\\
5.38	0.01\\
5.39	0.01\\
5.4	0.01\\
5.41	0.01\\
5.42	0.01\\
5.43	0.01\\
5.44	0.01\\
5.45	0.01\\
5.46	0.01\\
5.47	0.01\\
5.48	0.01\\
5.49	0.01\\
5.5	0.01\\
5.51	0.01\\
5.52	0.01\\
5.53	0.01\\
5.54	0.01\\
5.55	0.01\\
5.56	0.01\\
5.57	0.01\\
5.58	0.01\\
5.59	0.01\\
5.6	0.01\\
5.61	0.01\\
5.62	0.01\\
5.63	0.01\\
5.64	0.01\\
5.65	0.01\\
5.66	0.01\\
5.67	0.01\\
5.68	0.01\\
5.69	0.01\\
5.7	0.01\\
5.71	0.01\\
5.72	0.01\\
5.73	0.01\\
5.74	0.01\\
5.75	0.01\\
5.76	0.01\\
5.77	0.01\\
5.78	0.01\\
5.79	0.01\\
5.8	0.01\\
5.81	0.01\\
5.82	0.01\\
5.83	0.01\\
5.84	0.01\\
5.85	0.01\\
5.86	0.01\\
5.87	0.01\\
5.88	0.01\\
5.89	0.01\\
5.9	0.01\\
5.91	0.01\\
5.92	0.01\\
5.93	0.01\\
5.94	0.01\\
5.95	0.01\\
5.96	0.01\\
5.97	0.01\\
5.98	0.01\\
5.99	0.01\\
6	0.01\\
6.01	0.01\\
6.02	0.01\\
6.03	0.01\\
6.04	0.01\\
6.05	0.01\\
6.06	0.01\\
6.07	0.01\\
6.08	0.01\\
6.09	0.01\\
6.1	0.01\\
6.11	0.01\\
6.12	0.01\\
6.13	0.01\\
6.14	0.01\\
6.15	0.01\\
6.16	0.01\\
6.17	0.01\\
6.18	0.01\\
6.19	0.01\\
6.2	0.01\\
6.21	0.01\\
6.22	0.01\\
6.23	0.01\\
6.24	0.01\\
6.25	0.01\\
6.26	0.01\\
6.27	0.01\\
6.28	0.01\\
6.29	0.01\\
6.3	0.01\\
6.31	0.01\\
6.32	0.01\\
6.33	0.01\\
6.34	0.01\\
6.35	0.01\\
6.36	0.01\\
6.37	0.01\\
6.38	0.01\\
6.39	0.01\\
6.4	0.01\\
6.41	0.01\\
6.42	0.01\\
6.43	0.01\\
6.44	0.01\\
6.45	0.01\\
6.46	0.01\\
6.47	0.01\\
6.48	0.01\\
6.49	0.01\\
6.5	0.01\\
6.51	0.01\\
6.52	0.01\\
6.53	0.01\\
6.54	0.01\\
6.55	0.01\\
6.56	0.01\\
6.57	0.01\\
6.58	0.01\\
6.59	0.01\\
6.6	0.01\\
6.61	0.01\\
6.62	0.01\\
6.63	0.01\\
6.64	0.01\\
6.65	0.01\\
6.66	0.01\\
6.67	0.01\\
6.68	0.01\\
6.69	0.01\\
6.7	0.01\\
6.71	0.01\\
6.72	0.01\\
6.73	0.01\\
6.74	0.01\\
6.75	0.01\\
6.76	0.01\\
6.77	0.01\\
6.78	0.01\\
6.79	0.01\\
6.8	0.01\\
6.81	0.01\\
6.82	0.01\\
6.83	0.01\\
6.84	0.01\\
6.85	0.01\\
6.86	0.01\\
6.87	0.01\\
6.88	0.01\\
6.89	0.01\\
6.9	0.01\\
6.91	0.01\\
6.92	0.01\\
6.93	0.01\\
6.94	0.01\\
6.95	0.01\\
6.96	0.01\\
6.97	0.01\\
6.98	0.01\\
6.99	0.01\\
7	0.01\\
7.01	0.01\\
7.02	0.01\\
7.03	0.01\\
7.04	0.01\\
7.05	0.01\\
7.06	0.01\\
7.07	0.01\\
7.08	0.01\\
7.09	0.01\\
7.1	0.01\\
7.11	0.01\\
7.12	0.01\\
7.13	0.01\\
7.14	0.01\\
7.15	0.01\\
7.16	0.01\\
7.17	0.01\\
7.18	0.01\\
7.19	0.01\\
7.2	0.01\\
7.21	0.01\\
7.22	0.01\\
7.23	0.01\\
7.24	0.01\\
7.25	0.01\\
7.26	0.01\\
7.27	0.01\\
7.28	0.01\\
7.29	0.01\\
7.3	0.01\\
7.31	0.01\\
7.32	0.01\\
7.33	0.01\\
7.34	0.01\\
7.35	0.01\\
7.36	0.01\\
7.37	0.01\\
7.38	0.01\\
7.39	0.01\\
7.4	0.01\\
7.41	0.01\\
7.42	0.01\\
7.43	0.01\\
7.44	0.01\\
7.45	0.01\\
7.46	0.01\\
7.47	0.01\\
7.48	0.01\\
7.49	0.01\\
7.5	0.01\\
7.51	0.01\\
7.52	0.01\\
7.53	0.01\\
7.54	0.01\\
7.55	0.01\\
7.56	0.01\\
7.57	0.01\\
7.58	0.01\\
7.59	0.01\\
7.6	0.01\\
7.61	0.01\\
7.62	0.01\\
7.63	0.01\\
7.64	0.01\\
7.65	0.01\\
7.66	0.01\\
7.67	0.01\\
7.68	0.01\\
7.69	0.01\\
7.7	0.01\\
7.71	0.01\\
7.72	0.01\\
7.73	0.01\\
7.74	0.01\\
7.75	0.01\\
7.76	0.01\\
7.77	0.01\\
7.78	0.01\\
7.79	0.01\\
7.8	0.01\\
7.81	0.01\\
7.82	0.01\\
7.83	0.01\\
7.84	0.01\\
7.85	0.01\\
7.86	0.01\\
7.87	0.01\\
7.88	0.01\\
7.89	0.01\\
7.9	0.01\\
7.91	0.01\\
7.92	0.01\\
7.93	0.01\\
7.94	0.01\\
7.95	0.01\\
7.96	0.01\\
7.97	0.01\\
7.98	0.01\\
7.99	0.01\\
8	0.01\\
8.01	0.01\\
8.02	0.01\\
8.03	0.01\\
8.04	0.01\\
8.05	0.01\\
8.06	0.01\\
8.07	0.01\\
8.08	0.01\\
8.09	0.01\\
8.1	0.01\\
8.11	0.01\\
8.12	0.01\\
8.13	0.01\\
8.14	0.01\\
8.15	0.01\\
8.16	0.01\\
8.17	0.01\\
8.18	0.01\\
8.19	0.01\\
8.2	0.01\\
8.21	0.01\\
8.22	0.01\\
8.23	0.01\\
8.24	0.01\\
8.25	0.01\\
8.26	0.01\\
8.27	0.01\\
8.28	0.01\\
8.29	0.01\\
8.3	0.01\\
8.31	0.01\\
8.32	0.01\\
8.33	0.01\\
8.34	0.01\\
8.35	0.01\\
8.36	0.01\\
8.37	0.01\\
8.38	0.01\\
8.39	0.01\\
8.4	0.01\\
8.41	0.01\\
8.42	0.01\\
8.43	0.01\\
8.44	0.01\\
8.45	0.01\\
8.46	0.01\\
8.47	0.01\\
8.48	0.01\\
8.49	0.01\\
8.5	0.01\\
8.51	0.01\\
8.52	0.01\\
8.53	0.01\\
8.54	0.01\\
8.55	0.01\\
8.56	0.01\\
8.57	0.01\\
8.58	0.01\\
8.59	0.01\\
8.6	0.01\\
8.61	0.01\\
8.62	0.01\\
8.63	0.01\\
8.64	0.01\\
8.65	0.01\\
8.66	0.01\\
8.67	0.01\\
8.68	0.01\\
8.69	0.01\\
8.7	0.01\\
8.71	0.01\\
8.72	0.01\\
8.73	0.01\\
8.74	0.01\\
8.75	0.01\\
8.76	0.01\\
8.77	0.01\\
8.78	0.01\\
8.79	0.01\\
8.8	0.01\\
8.81	0.01\\
8.82	0.01\\
8.83	0.01\\
8.84	0.01\\
8.85	0.01\\
8.86	0.01\\
8.87	0.01\\
8.88	0.01\\
8.89	0.01\\
8.9	0.01\\
8.91	0.01\\
8.92	0.01\\
8.93	0.01\\
8.94	0.01\\
8.95	0.01\\
8.96	0.01\\
8.97	0.01\\
8.98	0.01\\
8.99	0.01\\
9	0.01\\
9.01	0.01\\
9.02	0.01\\
9.03	0.01\\
9.04	0.01\\
9.05	0.01\\
9.06	0.01\\
9.07	0.01\\
9.08	0.01\\
9.09	0.01\\
9.1	0.01\\
9.11	0.01\\
9.12	0.01\\
9.13	0.01\\
9.14	0.01\\
9.15	0.01\\
9.16	0.01\\
9.17	0.01\\
9.18	0.01\\
9.19	0.01\\
9.2	0.01\\
9.21	0.01\\
9.22	0.01\\
9.23	0.01\\
9.24	0.01\\
9.25	0.01\\
9.26	0.01\\
9.27	0.01\\
9.28	0.01\\
9.29	0.01\\
9.3	0.01\\
9.31	0.01\\
9.32	0.01\\
9.33	0.01\\
9.34	0.01\\
9.35	0.01\\
9.36	0.01\\
9.37	0.01\\
9.38	0.01\\
9.39	0.01\\
9.4	0.01\\
9.41	0.01\\
9.42	0.01\\
9.43	0.01\\
9.44	0.01\\
9.45	0.01\\
9.46	0.01\\
9.47	0.01\\
9.48	0.01\\
9.49	0.01\\
9.5	0.01\\
9.51	0.01\\
9.52	0.01\\
9.53	0.01\\
9.54	0.01\\
9.55	0.01\\
9.56	0.01\\
9.57	0.01\\
9.58	0.01\\
9.59	0.01\\
9.6	0.01\\
9.61	0.01\\
9.62	0.01\\
9.63	0.01\\
9.64	0.01\\
9.65	0.01\\
9.66	0.01\\
9.67	0.01\\
9.68	0.01\\
9.69	0.01\\
9.7	0.01\\
9.71	0.01\\
9.72	0.01\\
9.73	0.01\\
9.74	0.01\\
9.75	0.01\\
9.76	0.01\\
9.77	0.01\\
9.78	0.01\\
9.79	0.01\\
9.8	0.01\\
9.81	0.01\\
9.82	0.01\\
9.83	0.01\\
9.84	0.01\\
9.85	0.01\\
9.86	0.01\\
9.87	0.01\\
9.88	0.01\\
9.89	0.01\\
9.9	0.01\\
9.91	0.01\\
9.92	0.01\\
9.93	0.01\\
9.94	0.01\\
9.95	0.01\\
9.96	0.01\\
9.97	0.01\\
9.98	0.01\\
9.99	0.01\\
10	0.01\\
10.01	0.01\\
10.02	0.01\\
10.03	0.01\\
10.04	0.01\\
10.05	0.01\\
10.06	0.01\\
10.07	0.01\\
10.08	0.01\\
10.09	0.01\\
10.1	0.01\\
10.11	0.01\\
10.12	0.01\\
10.13	0.01\\
10.14	0.01\\
10.15	0.01\\
10.16	0.01\\
10.17	0.01\\
10.18	0.01\\
10.19	0.01\\
10.2	0.01\\
10.21	0.01\\
10.22	0.01\\
10.23	0.01\\
10.24	0.01\\
10.25	0.01\\
10.26	0.01\\
10.27	0.01\\
10.28	0.01\\
10.29	0.01\\
10.3	0.01\\
10.31	0.01\\
10.32	0.01\\
10.33	0.01\\
10.34	0.01\\
10.35	0.01\\
10.36	0.01\\
10.37	0.01\\
10.38	0.01\\
10.39	0.01\\
10.4	0.01\\
10.41	0.01\\
10.42	0.01\\
10.43	0.01\\
10.44	0.01\\
10.45	0.01\\
10.46	0.01\\
10.47	0.01\\
10.48	0.01\\
10.49	0.01\\
10.5	0.01\\
10.51	0.01\\
10.52	0.01\\
10.53	0.01\\
10.54	0.01\\
10.55	0.01\\
10.56	0.01\\
10.57	0.01\\
10.58	0.01\\
10.59	0.01\\
10.6	0.01\\
10.61	0.01\\
10.62	0.01\\
10.63	0.01\\
10.64	0.01\\
10.65	0.01\\
10.66	0.01\\
10.67	0.01\\
10.68	0.01\\
10.69	0.01\\
10.7	0.01\\
10.71	0.01\\
10.72	0.01\\
10.73	0.01\\
10.74	0.01\\
10.75	0.01\\
10.76	0.01\\
10.77	0.01\\
10.78	0.01\\
10.79	0.01\\
10.8	0.01\\
10.81	0.01\\
10.82	0.01\\
10.83	0.01\\
10.84	0.01\\
10.85	0.01\\
10.86	0.01\\
10.87	0.01\\
10.88	0.01\\
10.89	0.01\\
10.9	0.01\\
10.91	0.01\\
10.92	0.01\\
10.93	0.01\\
10.94	0.01\\
10.95	0.01\\
10.96	0.01\\
10.97	0.01\\
10.98	0.01\\
10.99	0.01\\
11	0.01\\
11.01	0.01\\
11.02	0.01\\
11.03	0.01\\
11.04	0.01\\
11.05	0.01\\
11.06	0.01\\
11.07	0.01\\
11.08	0.01\\
11.09	0.01\\
11.1	0.01\\
11.11	0.01\\
11.12	0.01\\
11.13	0.01\\
11.14	0.01\\
11.15	0.01\\
11.16	0.01\\
11.17	0.01\\
11.18	0.01\\
11.19	0.01\\
11.2	0.01\\
11.21	0.01\\
11.22	0.01\\
11.23	0.01\\
11.24	0.01\\
11.25	0.01\\
11.26	0.01\\
11.27	0.01\\
11.28	0.01\\
11.29	0.01\\
11.3	0.01\\
11.31	0.01\\
11.32	0.01\\
11.33	0.01\\
11.34	0.01\\
11.35	0.01\\
11.36	0.01\\
11.37	0.01\\
11.38	0.01\\
11.39	0.01\\
11.4	0.01\\
11.41	0.01\\
11.42	0.01\\
11.43	0.01\\
11.44	0.01\\
11.45	0.01\\
11.46	0.01\\
11.47	0.01\\
11.48	0.01\\
11.49	0.01\\
11.5	0.01\\
11.51	0.01\\
11.52	0.01\\
11.53	0.01\\
11.54	0.01\\
11.55	0.01\\
11.56	0.01\\
11.57	0.01\\
11.58	0.01\\
11.59	0.01\\
11.6	0.01\\
11.61	0.01\\
11.62	0.01\\
11.63	0.01\\
11.64	0.01\\
11.65	0.01\\
11.66	0.01\\
11.67	0.01\\
11.68	0.01\\
11.69	0.01\\
11.7	0.01\\
11.71	0.01\\
11.72	0.01\\
11.73	0.01\\
11.74	0.01\\
11.75	0.01\\
11.76	0.01\\
11.77	0.01\\
11.78	0.01\\
11.79	0.01\\
11.8	0.01\\
11.81	0.01\\
11.82	0.01\\
11.83	0.01\\
11.84	0.01\\
11.85	0.01\\
11.86	0.01\\
11.87	0.01\\
11.88	0.01\\
11.89	0.01\\
11.9	0.01\\
11.91	0.01\\
11.92	0.01\\
11.93	0.01\\
11.94	0.01\\
11.95	0.01\\
11.96	0.01\\
11.97	0.01\\
11.98	0.01\\
11.99	0.01\\
12	0.01\\
12.01	0.01\\
12.02	0.01\\
12.03	0.01\\
12.04	0.01\\
12.05	0.01\\
12.06	0.01\\
12.07	0.01\\
12.08	0.01\\
12.09	0.01\\
12.1	0.01\\
12.11	0.01\\
12.12	0.01\\
12.13	0.01\\
12.14	0.01\\
12.15	0.01\\
12.16	0.01\\
12.17	0.01\\
12.18	0.01\\
12.19	0.01\\
12.2	0.01\\
12.21	0.01\\
12.22	0.01\\
12.23	0.01\\
12.24	0.01\\
12.25	0.01\\
12.26	0.01\\
12.27	0.01\\
12.28	0.01\\
12.29	0.01\\
12.3	0.01\\
12.31	0.01\\
12.32	0.01\\
12.33	0.01\\
12.34	0.01\\
12.35	0.01\\
12.36	0.01\\
12.37	0.01\\
12.38	0.01\\
12.39	0.01\\
12.4	0.01\\
12.41	0.01\\
12.42	0.01\\
12.43	0.01\\
12.44	0.01\\
12.45	0.01\\
12.46	0.01\\
12.47	0.01\\
12.48	0.01\\
12.49	0.01\\
12.5	0.01\\
12.51	0.01\\
12.52	0.01\\
12.53	0.01\\
12.54	0.01\\
12.55	0.01\\
12.56	0.01\\
12.57	0.01\\
12.58	0.01\\
12.59	0.01\\
12.6	0.01\\
12.61	0.01\\
12.62	0.01\\
12.63	0.01\\
12.64	0.01\\
12.65	0.01\\
12.66	0.01\\
12.67	0.01\\
12.68	0.01\\
12.69	0.01\\
12.7	0.01\\
12.71	0.01\\
12.72	0.01\\
12.73	0.01\\
12.74	0.01\\
12.75	0.01\\
12.76	0.01\\
12.77	0.01\\
12.78	0.01\\
12.79	0.01\\
12.8	0.01\\
12.81	0.01\\
12.82	0.01\\
12.83	0.01\\
12.84	0.01\\
12.85	0.01\\
12.86	0.01\\
12.87	0.01\\
12.88	0.01\\
12.89	0.01\\
12.9	0.01\\
12.91	0.01\\
12.92	0.01\\
12.93	0.01\\
12.94	0.01\\
12.95	0.01\\
12.96	0.01\\
12.97	0.01\\
12.98	0.01\\
12.99	0.01\\
13	0.01\\
13.01	0.01\\
13.02	0.01\\
13.03	0.01\\
13.04	0.01\\
13.05	0.01\\
13.06	0.01\\
13.07	0.01\\
13.08	0.01\\
13.09	0.01\\
13.1	0.01\\
13.11	0.01\\
13.12	0.01\\
13.13	0.01\\
13.14	0.01\\
13.15	0.01\\
13.16	0.01\\
13.17	0.01\\
13.18	0.01\\
13.19	0.01\\
13.2	0.01\\
13.21	0.01\\
13.22	0.01\\
13.23	0.01\\
13.24	0.01\\
13.25	0.01\\
13.26	0.01\\
13.27	0.01\\
13.28	0.01\\
13.29	0.01\\
13.3	0.01\\
13.31	0.01\\
13.32	0.01\\
13.33	0.01\\
13.34	0.01\\
13.35	0.01\\
13.36	0.01\\
13.37	0.01\\
13.38	0.01\\
13.39	0.01\\
13.4	0.01\\
13.41	0.01\\
13.42	0.01\\
13.43	0.01\\
13.44	0.01\\
13.45	0.01\\
13.46	0.01\\
13.47	0.01\\
13.48	0.01\\
13.49	0.01\\
13.5	0.01\\
13.51	0.01\\
13.52	0.01\\
13.53	0.01\\
13.54	0.01\\
13.55	0.01\\
13.56	0.01\\
13.57	0.01\\
13.58	0.01\\
13.59	0.01\\
13.6	0.01\\
13.61	0.01\\
13.62	0.01\\
13.63	0.01\\
13.64	0.01\\
13.65	0.01\\
13.66	0.01\\
13.67	0.01\\
13.68	0.01\\
13.69	0.01\\
13.7	0.01\\
13.71	0.01\\
13.72	0.01\\
13.73	0.01\\
13.74	0.01\\
13.75	0.01\\
13.76	0.01\\
13.77	0.01\\
13.78	0.01\\
13.79	0.01\\
13.8	0.01\\
13.81	0.01\\
13.82	0.01\\
13.83	0.01\\
13.84	0.01\\
13.85	0.01\\
13.86	0.01\\
13.87	0.01\\
13.88	0.01\\
13.89	0.01\\
13.9	0.01\\
13.91	0.01\\
13.92	0.01\\
13.93	0.01\\
13.94	0.01\\
13.95	0.01\\
13.96	0.01\\
13.97	0.01\\
13.98	0.01\\
13.99	0.01\\
14	0.01\\
14.01	0.01\\
14.02	0.01\\
14.03	0.01\\
14.04	0.01\\
14.05	0.01\\
14.06	0.01\\
14.07	0.01\\
14.08	0.01\\
14.09	0.01\\
14.1	0.01\\
14.11	0.01\\
14.12	0.01\\
14.13	0.01\\
14.14	0.01\\
14.15	0.01\\
14.16	0.01\\
14.17	0.01\\
14.18	0.01\\
14.19	0.01\\
14.2	0.01\\
14.21	0.01\\
14.22	0.01\\
14.23	0.01\\
14.24	0.01\\
14.25	0.01\\
14.26	0.01\\
14.27	0.01\\
14.28	0.01\\
14.29	0.01\\
14.3	0.01\\
14.31	0.01\\
14.32	0.01\\
14.33	0.01\\
14.34	0.01\\
14.35	0.01\\
14.36	0.01\\
14.37	0.01\\
14.38	0.01\\
14.39	0.01\\
14.4	0.01\\
14.41	0.01\\
14.42	0.01\\
14.43	0.01\\
14.44	0.01\\
14.45	0.01\\
14.46	0.01\\
14.47	0.01\\
14.48	0.01\\
14.49	0.01\\
14.5	0.01\\
14.51	0.01\\
14.52	0.01\\
14.53	0.01\\
14.54	0.01\\
14.55	0.01\\
14.56	0.01\\
14.57	0.01\\
14.58	0.01\\
14.59	0.01\\
14.6	0.01\\
14.61	0.01\\
14.62	0.01\\
14.63	0.01\\
14.64	0.01\\
14.65	0.01\\
14.66	0.01\\
14.67	0.01\\
14.68	0.01\\
14.69	0.01\\
14.7	0.01\\
14.71	0.01\\
14.72	0.01\\
14.73	0.01\\
14.74	0.01\\
14.75	0.01\\
14.76	0.01\\
14.77	0.01\\
14.78	0.01\\
14.79	0.01\\
14.8	0.01\\
14.81	0.01\\
14.82	0.01\\
14.83	0.01\\
14.84	0.01\\
14.85	0.01\\
14.86	0.01\\
14.87	0.01\\
14.88	0.01\\
14.89	0.01\\
14.9	0.01\\
14.91	0.01\\
14.92	0.01\\
14.93	0.01\\
14.94	0.01\\
14.95	0.01\\
14.96	0.01\\
14.97	0.01\\
14.98	0.01\\
14.99	0.01\\
15	0.01\\
15.01	0.01\\
15.02	0.01\\
15.03	0.01\\
15.04	0.01\\
15.05	0.01\\
15.06	0.01\\
15.07	0.01\\
15.08	0.01\\
15.09	0.01\\
15.1	0.01\\
15.11	0.01\\
15.12	0.01\\
15.13	0.01\\
15.14	0.01\\
15.15	0.01\\
15.16	0.01\\
15.17	0.01\\
15.18	0.01\\
15.19	0.01\\
15.2	0.01\\
15.21	0.01\\
15.22	0.01\\
15.23	0.01\\
15.24	0.01\\
15.25	0.01\\
15.26	0.01\\
15.27	0.01\\
15.28	0.01\\
15.29	0.01\\
15.3	0.01\\
15.31	0.01\\
15.32	0.01\\
15.33	0.01\\
15.34	0.01\\
15.35	0.01\\
15.36	0.01\\
15.37	0.01\\
15.38	0.01\\
15.39	0.01\\
15.4	0.01\\
15.41	0.01\\
15.42	0.01\\
15.43	0.01\\
15.44	0.01\\
15.45	0.01\\
15.46	0.01\\
15.47	0.01\\
15.48	0.01\\
15.49	0.01\\
15.5	0.01\\
15.51	0.01\\
15.52	0.01\\
15.53	0.01\\
15.54	0.01\\
15.55	0.01\\
15.56	0.01\\
15.57	0.01\\
15.58	0.01\\
15.59	0.01\\
15.6	0.01\\
15.61	0.01\\
15.62	0.01\\
15.63	0.01\\
15.64	0.01\\
15.65	0.01\\
15.66	0.01\\
15.67	0.01\\
15.68	0.01\\
15.69	0.01\\
15.7	0.01\\
15.71	0.01\\
15.72	0.01\\
15.73	0.01\\
15.74	0.01\\
15.75	0.01\\
15.76	0.01\\
15.77	0.01\\
15.78	0.01\\
15.79	0.01\\
15.8	0.01\\
15.81	0.01\\
15.82	0.01\\
15.83	0.01\\
15.84	0.01\\
15.85	0.01\\
15.86	0.01\\
15.87	0.01\\
15.88	0.01\\
15.89	0.01\\
15.9	0.01\\
15.91	0.01\\
15.92	0.01\\
15.93	0.01\\
15.94	0.01\\
15.95	0.01\\
15.96	0.01\\
15.97	0.01\\
15.98	0.01\\
15.99	0.01\\
16	0.01\\
16.01	0.01\\
16.02	0.01\\
16.03	0.01\\
16.04	0.01\\
16.05	0.01\\
16.06	0.01\\
16.07	0.01\\
16.08	0.01\\
16.09	0.01\\
16.1	0.01\\
16.11	0.01\\
16.12	0.01\\
16.13	0.01\\
16.14	0.01\\
16.15	0.01\\
16.16	0.01\\
16.17	0.01\\
16.18	0.01\\
16.19	0.01\\
16.2	0.01\\
16.21	0.01\\
16.22	0.01\\
16.23	0.01\\
16.24	0.01\\
16.25	0.01\\
16.26	0.01\\
16.27	0.01\\
16.28	0.01\\
16.29	0.01\\
16.3	0.01\\
16.31	0.01\\
16.32	0.01\\
16.33	0.01\\
16.34	0.01\\
16.35	0.01\\
16.36	0.01\\
16.37	0.01\\
16.38	0.01\\
16.39	0.01\\
16.4	0.01\\
16.41	0.01\\
16.42	0.01\\
16.43	0.01\\
16.44	0.01\\
16.45	0.01\\
16.46	0.01\\
16.47	0.01\\
16.48	0.01\\
16.49	0.01\\
16.5	0.01\\
16.51	0.01\\
16.52	0.01\\
16.53	0.01\\
16.54	0.01\\
16.55	0.01\\
16.56	0.01\\
16.57	0.01\\
16.58	0.01\\
16.59	0.01\\
16.6	0.01\\
16.61	0.01\\
16.62	0.01\\
16.63	0.01\\
16.64	0.01\\
16.65	0.01\\
16.66	0.01\\
16.67	0.01\\
16.68	0.01\\
16.69	0.01\\
16.7	0.01\\
16.71	0.01\\
16.72	0.01\\
16.73	0.01\\
16.74	0.01\\
16.75	0.01\\
16.76	0.01\\
16.77	0.01\\
16.78	0.01\\
16.79	0.01\\
16.8	0.01\\
16.81	0.01\\
16.82	0.01\\
16.83	0.01\\
16.84	0.01\\
16.85	0.01\\
16.86	0.01\\
16.87	0.01\\
16.88	0.01\\
16.89	0.01\\
16.9	0.01\\
16.91	0.01\\
16.92	0.01\\
16.93	0.01\\
16.94	0.01\\
16.95	0.01\\
16.96	0.01\\
16.97	0.01\\
16.98	0.01\\
16.99	0.01\\
17	0.01\\
17.01	0.01\\
17.02	0.01\\
17.03	0.01\\
17.04	0.01\\
17.05	0.01\\
17.06	0.01\\
17.07	0.01\\
17.08	0.01\\
17.09	0.01\\
17.1	0.01\\
17.11	0.01\\
17.12	0.01\\
17.13	0.01\\
17.14	0.01\\
17.15	0.01\\
17.16	0.01\\
17.17	0.01\\
17.18	0.01\\
17.19	0.01\\
17.2	0.01\\
17.21	0.01\\
17.22	0.01\\
17.23	0.01\\
17.24	0.01\\
17.25	0.01\\
17.26	0.01\\
17.27	0.01\\
17.28	0.01\\
17.29	0.01\\
17.3	0.01\\
17.31	0.01\\
17.32	0.01\\
17.33	0.01\\
17.34	0.01\\
17.35	0.01\\
17.36	0.01\\
17.37	0.01\\
17.38	0.01\\
17.39	0.01\\
17.4	0.01\\
17.41	0.01\\
17.42	0.01\\
17.43	0.01\\
17.44	0.01\\
17.45	0.01\\
17.46	0.01\\
17.47	0.01\\
17.48	0.01\\
17.49	0.01\\
17.5	0.01\\
17.51	0.01\\
17.52	0.01\\
17.53	0.01\\
17.54	0.01\\
17.55	0.01\\
17.56	0.01\\
17.57	0.01\\
17.58	0.01\\
17.59	0.01\\
17.6	0.01\\
17.61	0.01\\
17.62	0.01\\
17.63	0.01\\
17.64	0.01\\
17.65	0.01\\
17.66	0.01\\
17.67	0.01\\
17.68	0.01\\
17.69	0.01\\
17.7	0.01\\
17.71	0.01\\
17.72	0.01\\
17.73	0.01\\
17.74	0.01\\
17.75	0.01\\
17.76	0.01\\
17.77	0.01\\
17.78	0.01\\
17.79	0.01\\
17.8	0.01\\
17.81	0.01\\
17.82	0.01\\
17.83	0.01\\
17.84	0.01\\
17.85	0.01\\
17.86	0.01\\
17.87	0.01\\
17.88	0.01\\
17.89	0.01\\
17.9	0.01\\
17.91	0.01\\
17.92	0.01\\
17.93	0.01\\
17.94	0.01\\
17.95	0.01\\
17.96	0.01\\
17.97	0.01\\
17.98	0.01\\
17.99	0.01\\
18	0.01\\
18.01	0.01\\
18.02	0.01\\
18.03	0.01\\
18.04	0.01\\
18.05	0.01\\
18.06	0.01\\
18.07	0.01\\
18.08	0.01\\
18.09	0.01\\
18.1	0.01\\
18.11	0.01\\
18.12	0.01\\
18.13	0.01\\
18.14	0.01\\
18.15	0.01\\
18.16	0.01\\
18.17	0.01\\
18.18	0.01\\
18.19	0.01\\
18.2	0.01\\
18.21	0.01\\
18.22	0.01\\
18.23	0.01\\
18.24	0.01\\
18.25	0.01\\
18.26	0.01\\
18.27	0.01\\
18.28	0.01\\
18.29	0.01\\
18.3	0.01\\
18.31	0.01\\
18.32	0.01\\
18.33	0.01\\
18.34	0.01\\
18.35	0.01\\
18.36	0.01\\
18.37	0.01\\
18.38	0.01\\
18.39	0.01\\
18.4	0.01\\
18.41	0.01\\
18.42	0.01\\
18.43	0.01\\
18.44	0.01\\
18.45	0.01\\
18.46	0.01\\
18.47	0.01\\
18.48	0.01\\
18.49	0.01\\
18.5	0.01\\
18.51	0.01\\
18.52	0.01\\
18.53	0.01\\
18.54	0.01\\
18.55	0.01\\
18.56	0.01\\
18.57	0.01\\
18.58	0.01\\
18.59	0.01\\
18.6	0.01\\
18.61	0.01\\
18.62	0.01\\
18.63	0.01\\
18.64	0.01\\
18.65	0.01\\
18.66	0.01\\
18.67	0.01\\
18.68	0.01\\
18.69	0.01\\
18.7	0.01\\
18.71	0.01\\
18.72	0.01\\
18.73	0.01\\
18.74	0.01\\
18.75	0.01\\
18.76	0.01\\
18.77	0.01\\
18.78	0.01\\
18.79	0.01\\
18.8	0.01\\
18.81	0.01\\
18.82	0.01\\
18.83	0.01\\
18.84	0.01\\
18.85	0.01\\
18.86	0.01\\
18.87	0.01\\
18.88	0.01\\
18.89	0.01\\
18.9	0.01\\
18.91	0.01\\
18.92	0.01\\
18.93	0.01\\
18.94	0.01\\
18.95	0.01\\
18.96	0.01\\
18.97	0.01\\
18.98	0.01\\
18.99	0.01\\
19	0.01\\
19.01	0.01\\
19.02	0.01\\
19.03	0.01\\
19.04	0.01\\
19.05	0.01\\
19.06	0.01\\
19.07	0.01\\
19.08	0.01\\
19.09	0.01\\
19.1	0.01\\
19.11	0.01\\
19.12	0.01\\
19.13	0.01\\
19.14	0.01\\
19.15	0.01\\
19.16	0.01\\
19.17	0.01\\
19.18	0.01\\
19.19	0.01\\
19.2	0.01\\
19.21	0.01\\
19.22	0.01\\
19.23	0.01\\
19.24	0.01\\
19.25	0.01\\
19.26	0.01\\
19.27	0.01\\
19.28	0.01\\
19.29	0.01\\
19.3	0.01\\
19.31	0.01\\
19.32	0.01\\
19.33	0.01\\
19.34	0.01\\
19.35	0.01\\
19.36	0.01\\
19.37	0.01\\
19.38	0.01\\
19.39	0.01\\
19.4	0.01\\
19.41	0.01\\
19.42	0.01\\
19.43	0.01\\
19.44	0.01\\
19.45	0.01\\
19.46	0.01\\
19.47	0.01\\
19.48	0.01\\
19.49	0.01\\
19.5	0.01\\
19.51	0.01\\
19.52	0.01\\
19.53	0.01\\
19.54	0.01\\
19.55	0.01\\
19.56	0.01\\
19.57	0.01\\
19.58	0.01\\
19.59	0.01\\
19.6	0.01\\
19.61	0.01\\
19.62	0.01\\
19.63	0.01\\
19.64	0.01\\
19.65	0.01\\
19.66	0.01\\
19.67	0.01\\
19.68	0.01\\
19.69	0.01\\
19.7	0.01\\
19.71	0.01\\
19.72	0.01\\
19.73	0.01\\
19.74	0.01\\
19.75	0.01\\
19.76	0.01\\
19.77	0.01\\
19.78	0.01\\
19.79	0.01\\
19.8	0.01\\
19.81	0.01\\
19.82	0.01\\
19.83	0.01\\
19.84	0.01\\
19.85	0.01\\
19.86	0.01\\
19.87	0.01\\
19.88	0.01\\
19.89	0.01\\
19.9	0.01\\
19.91	0.01\\
19.92	0.01\\
19.93	0.01\\
19.94	0.01\\
19.95	0.01\\
19.96	0.01\\
19.97	0.01\\
19.98	0.01\\
19.99	0.01\\
20	0.01\\
20.01	0.01\\
20.02	0.01\\
20.03	0.01\\
20.04	0.01\\
20.05	0.01\\
20.06	0.01\\
20.07	0.01\\
20.08	0.01\\
20.09	0.01\\
20.1	0.01\\
20.11	0.01\\
20.12	0.01\\
20.13	0.01\\
20.14	0.01\\
20.15	0.01\\
20.16	0.01\\
20.17	0.01\\
20.18	0.01\\
20.19	0.01\\
20.2	0.01\\
20.21	0.01\\
20.22	0.01\\
20.23	0.01\\
20.24	0.01\\
20.25	0.01\\
20.26	0.01\\
20.27	0.01\\
20.28	0.01\\
20.29	0.01\\
20.3	0.01\\
20.31	0.01\\
20.32	0.01\\
20.33	0.01\\
20.34	0.01\\
20.35	0.01\\
20.36	0.01\\
20.37	0.01\\
20.38	0.01\\
20.39	0.01\\
20.4	0.01\\
20.41	0.01\\
20.42	0.01\\
20.43	0.01\\
20.44	0.01\\
20.45	0.01\\
20.46	0.01\\
20.47	0.01\\
20.48	0.01\\
20.49	0.01\\
20.5	0.01\\
20.51	0.01\\
20.52	0.01\\
20.53	0.01\\
20.54	0.01\\
20.55	0.01\\
20.56	0.01\\
20.57	0.01\\
20.58	0.01\\
20.59	0.01\\
20.6	0.01\\
20.61	0.01\\
20.62	0.01\\
20.63	0.01\\
20.64	0.01\\
20.65	0.01\\
20.66	0.01\\
20.67	0.01\\
20.68	0.01\\
20.69	0.01\\
20.7	0.01\\
20.71	0.01\\
20.72	0.01\\
20.73	0.01\\
20.74	0.01\\
20.75	0.01\\
20.76	0.01\\
20.77	0.01\\
20.78	0.01\\
20.79	0.01\\
20.8	0.01\\
20.81	0.01\\
20.82	0.01\\
20.83	0.01\\
20.84	0.01\\
20.85	0.01\\
20.86	0.01\\
20.87	0.01\\
20.88	0.01\\
20.89	0.01\\
20.9	0.01\\
20.91	0.01\\
20.92	0.01\\
20.93	0.01\\
20.94	0.01\\
20.95	0.01\\
20.96	0.01\\
20.97	0.01\\
20.98	0.01\\
20.99	0.01\\
21	0.01\\
21.01	0.01\\
21.02	0.01\\
21.03	0.01\\
21.04	0.01\\
21.05	0.01\\
21.06	0.01\\
21.07	0.01\\
21.08	0.01\\
21.09	0.01\\
21.1	0.01\\
21.11	0.01\\
21.12	0.01\\
21.13	0.01\\
21.14	0.01\\
21.15	0.01\\
21.16	0.01\\
21.17	0.01\\
21.18	0.01\\
21.19	0.01\\
21.2	0.01\\
21.21	0.01\\
21.22	0.01\\
21.23	0.01\\
21.24	0.01\\
21.25	0.01\\
21.26	0.01\\
21.27	0.01\\
21.28	0.01\\
21.29	0.01\\
21.3	0.01\\
21.31	0.01\\
21.32	0.01\\
21.33	0.01\\
21.34	0.01\\
21.35	0.01\\
21.36	0.01\\
21.37	0.01\\
21.38	0.01\\
21.39	0.01\\
21.4	0.01\\
21.41	0.01\\
21.42	0.01\\
21.43	0.01\\
21.44	0.01\\
21.45	0.01\\
21.46	0.01\\
21.47	0.01\\
21.48	0.01\\
21.49	0.01\\
21.5	0.01\\
21.51	0.01\\
21.52	0.01\\
21.53	0.01\\
21.54	0.01\\
21.55	0.01\\
21.56	0.01\\
21.57	0.01\\
21.58	0.01\\
21.59	0.01\\
21.6	0.01\\
21.61	0.01\\
21.62	0.01\\
21.63	0.01\\
21.64	0.01\\
21.65	0.01\\
21.66	0.01\\
21.67	0.01\\
21.68	0.01\\
21.69	0.01\\
21.7	0.01\\
21.71	0.01\\
21.72	0.01\\
21.73	0.01\\
21.74	0.01\\
21.75	0.01\\
21.76	0.01\\
21.77	0.01\\
21.78	0.01\\
21.79	0.01\\
21.8	0.01\\
21.81	0.01\\
21.82	0.01\\
21.83	0.01\\
21.84	0.01\\
21.85	0.01\\
21.86	0.01\\
21.87	0.01\\
21.88	0.01\\
21.89	0.01\\
21.9	0.01\\
21.91	0.01\\
21.92	0.01\\
21.93	0.01\\
21.94	0.01\\
21.95	0.01\\
21.96	0.01\\
21.97	0.01\\
21.98	0.01\\
21.99	0.01\\
22	0.01\\
22.01	0.01\\
22.02	0.01\\
22.03	0.01\\
22.04	0.01\\
22.05	0.01\\
22.06	0.01\\
22.07	0.01\\
22.08	0.01\\
22.09	0.01\\
22.1	0.01\\
22.11	0.01\\
22.12	0.01\\
22.13	0.01\\
22.14	0.01\\
22.15	0.01\\
22.16	0.01\\
22.17	0.01\\
22.18	0.01\\
22.19	0.01\\
22.2	0.01\\
22.21	0.01\\
22.22	0.01\\
22.23	0.01\\
22.24	0.01\\
22.25	0.01\\
22.26	0.01\\
22.27	0.01\\
22.28	0.01\\
22.29	0.01\\
22.3	0.01\\
22.31	0.01\\
22.32	0.01\\
22.33	0.01\\
22.34	0.01\\
22.35	0.01\\
22.36	0.01\\
22.37	0.01\\
22.38	0.01\\
22.39	0.01\\
22.4	0.01\\
22.41	0.01\\
22.42	0.01\\
22.43	0.01\\
22.44	0.01\\
22.45	0.01\\
22.46	0.01\\
22.47	0.01\\
22.48	0.01\\
22.49	0.01\\
22.5	0.01\\
22.51	0.01\\
22.52	0.01\\
22.53	0.01\\
22.54	0.01\\
22.55	0.01\\
22.56	0.01\\
22.57	0.01\\
22.58	0.01\\
22.59	0.01\\
22.6	0.01\\
22.61	0.01\\
22.62	0.01\\
22.63	0.01\\
22.64	0.01\\
22.65	0.01\\
22.66	0.01\\
22.67	0.01\\
22.68	0.01\\
22.69	0.01\\
22.7	0.01\\
22.71	0.01\\
22.72	0.01\\
22.73	0.01\\
22.74	0.01\\
22.75	0.01\\
22.76	0.01\\
22.77	0.01\\
22.78	0.01\\
22.79	0.01\\
22.8	0.01\\
22.81	0.01\\
22.82	0.01\\
22.83	0.01\\
22.84	0.01\\
22.85	0.01\\
22.86	0.01\\
22.87	0.01\\
22.88	0.01\\
22.89	0.01\\
22.9	0.01\\
22.91	0.01\\
22.92	0.01\\
22.93	0.01\\
22.94	0.01\\
22.95	0.01\\
22.96	0.01\\
22.97	0.01\\
22.98	0.01\\
22.99	0.01\\
23	0.01\\
23.01	0.01\\
23.02	0.01\\
23.03	0.01\\
23.04	0.01\\
23.05	0.01\\
23.06	0.01\\
23.07	0.01\\
23.08	0.01\\
23.09	0.01\\
23.1	0.01\\
23.11	0.01\\
23.12	0.01\\
23.13	0.01\\
23.14	0.01\\
23.15	0.01\\
23.16	0.01\\
23.17	0.01\\
23.18	0.01\\
23.19	0.01\\
23.2	0.01\\
23.21	0.01\\
23.22	0.01\\
23.23	0.01\\
23.24	0.01\\
23.25	0.01\\
23.26	0.01\\
23.27	0.01\\
23.28	0.01\\
23.29	0.01\\
23.3	0.01\\
23.31	0.01\\
23.32	0.01\\
23.33	0.01\\
23.34	0.01\\
23.35	0.01\\
23.36	0.01\\
23.37	0.01\\
23.38	0.01\\
23.39	0.01\\
23.4	0.01\\
23.41	0.01\\
23.42	0.01\\
23.43	0.01\\
23.44	0.01\\
23.45	0.01\\
23.46	0.01\\
23.47	0.01\\
23.48	0.01\\
23.49	0.01\\
23.5	0.01\\
23.51	0.01\\
23.52	0.01\\
23.53	0.01\\
23.54	0.01\\
23.55	0.01\\
23.56	0.01\\
23.57	0.01\\
23.58	0.01\\
23.59	0.01\\
23.6	0.01\\
23.61	0.01\\
23.62	0.01\\
23.63	0.01\\
23.64	0.01\\
23.65	0.01\\
23.66	0.01\\
23.67	0.01\\
23.68	0.01\\
23.69	0.01\\
23.7	0.01\\
23.71	0.01\\
23.72	0.01\\
23.73	0.01\\
23.74	0.01\\
23.75	0.01\\
23.76	0.01\\
23.77	0.01\\
23.78	0.01\\
23.79	0.01\\
23.8	0.01\\
23.81	0.01\\
23.82	0.01\\
23.83	0.01\\
23.84	0.01\\
23.85	0.01\\
23.86	0.01\\
23.87	0.01\\
23.88	0.01\\
23.89	0.01\\
23.9	0.01\\
23.91	0.01\\
23.92	0.01\\
23.93	0.01\\
23.94	0.01\\
23.95	0.01\\
23.96	0.01\\
23.97	0.01\\
23.98	0.01\\
23.99	0.01\\
24	0.01\\
24.01	0.01\\
24.02	0.01\\
24.03	0.01\\
24.04	0.01\\
24.05	0.01\\
24.06	0.01\\
24.07	0.01\\
24.08	0.01\\
24.09	0.01\\
24.1	0.01\\
24.11	0.01\\
24.12	0.01\\
24.13	0.01\\
24.14	0.01\\
24.15	0.01\\
24.16	0.01\\
24.17	0.01\\
24.18	0.01\\
24.19	0.01\\
24.2	0.01\\
24.21	0.01\\
24.22	0.01\\
24.23	0.01\\
24.24	0.01\\
24.25	0.01\\
24.26	0.01\\
24.27	0.01\\
24.28	0.01\\
24.29	0.01\\
24.3	0.01\\
24.31	0.01\\
24.32	0.01\\
24.33	0.01\\
24.34	0.01\\
24.35	0.01\\
24.36	0.01\\
24.37	0.01\\
24.38	0.01\\
24.39	0.01\\
24.4	0.01\\
24.41	0.01\\
24.42	0.01\\
24.43	0.01\\
24.44	0.01\\
24.45	0.01\\
24.46	0.01\\
24.47	0.01\\
24.48	0.01\\
24.49	0.01\\
24.5	0.01\\
24.51	0.01\\
24.52	0.01\\
24.53	0.01\\
24.54	0.01\\
24.55	0.01\\
24.56	0.01\\
24.57	0.01\\
24.58	0.01\\
24.59	0.01\\
24.6	0.01\\
24.61	0.01\\
24.62	0.01\\
24.63	0.01\\
24.64	0.01\\
24.65	0.01\\
24.66	0.01\\
24.67	0.01\\
24.68	0.01\\
24.69	0.01\\
24.7	0.01\\
24.71	0.01\\
24.72	0.01\\
24.73	0.01\\
24.74	0.01\\
24.75	0.01\\
24.76	0.01\\
24.77	0.01\\
24.78	0.01\\
24.79	0.01\\
24.8	0.01\\
24.81	0.01\\
24.82	0.01\\
24.83	0.01\\
24.84	0.01\\
24.85	0.01\\
24.86	0.01\\
24.87	0.01\\
24.88	0.01\\
24.89	0.01\\
24.9	0.01\\
24.91	0.01\\
24.92	0.01\\
24.93	0.01\\
24.94	0.01\\
24.95	0.01\\
24.96	0.01\\
24.97	0.01\\
24.98	0.01\\
24.99	0.01\\
25	0.01\\
25.01	0.01\\
25.02	0.01\\
25.03	0.01\\
25.04	0.01\\
25.05	0.01\\
25.06	0.01\\
25.07	0.01\\
25.08	0.01\\
25.09	0.01\\
25.1	0.01\\
25.11	0.01\\
25.12	0.01\\
25.13	0.01\\
25.14	0.01\\
25.15	0.01\\
25.16	0.01\\
25.17	0.01\\
25.18	0.01\\
25.19	0.01\\
25.2	0.01\\
25.21	0.01\\
25.22	0.01\\
25.23	0.01\\
25.24	0.01\\
25.25	0.01\\
25.26	0.01\\
25.27	0.01\\
25.28	0.01\\
25.29	0.01\\
25.3	0.01\\
25.31	0.01\\
25.32	0.01\\
25.33	0.01\\
25.34	0.01\\
25.35	0.01\\
25.36	0.01\\
25.37	0.01\\
25.38	0.01\\
25.39	0.01\\
25.4	0.01\\
25.41	0.01\\
25.42	0.01\\
25.43	0.01\\
25.44	0.01\\
25.45	0.01\\
25.46	0.01\\
25.47	0.01\\
25.48	0.01\\
25.49	0.01\\
25.5	0.01\\
25.51	0.01\\
25.52	0.01\\
25.53	0.01\\
25.54	0.01\\
25.55	0.01\\
25.56	0.01\\
25.57	0.01\\
25.58	0.01\\
25.59	0.01\\
25.6	0.01\\
25.61	0.01\\
25.62	0.01\\
25.63	0.01\\
25.64	0.01\\
25.65	0.01\\
25.66	0.01\\
25.67	0.01\\
25.68	0.01\\
25.69	0.01\\
25.7	0.01\\
25.71	0.01\\
25.72	0.01\\
25.73	0.01\\
25.74	0.01\\
25.75	0.01\\
25.76	0.01\\
25.77	0.01\\
25.78	0.01\\
25.79	0.01\\
25.8	0.01\\
25.81	0.01\\
25.82	0.01\\
25.83	0.01\\
25.84	0.01\\
25.85	0.01\\
25.86	0.01\\
25.87	0.01\\
25.88	0.01\\
25.89	0.01\\
25.9	0.01\\
25.91	0.01\\
25.92	0.01\\
25.93	0.01\\
25.94	0.01\\
25.95	0.01\\
25.96	0.01\\
25.97	0.01\\
25.98	0.01\\
25.99	0.01\\
26	0.01\\
26.01	0.01\\
26.02	0.01\\
26.03	0.01\\
26.04	0.01\\
26.05	0.01\\
26.06	0.01\\
26.07	0.01\\
26.08	0.01\\
26.09	0.01\\
26.1	0.01\\
26.11	0.01\\
26.12	0.01\\
26.13	0.01\\
26.14	0.01\\
26.15	0.01\\
26.16	0.01\\
26.17	0.01\\
26.18	0.01\\
26.19	0.01\\
26.2	0.01\\
26.21	0.01\\
26.22	0.01\\
26.23	0.01\\
26.24	0.01\\
26.25	0.01\\
26.26	0.01\\
26.27	0.01\\
26.28	0.01\\
26.29	0.01\\
26.3	0.01\\
26.31	0.01\\
26.32	0.01\\
26.33	0.01\\
26.34	0.01\\
26.35	0.01\\
26.36	0.01\\
26.37	0.01\\
26.38	0.01\\
26.39	0.01\\
26.4	0.01\\
26.41	0.01\\
26.42	0.01\\
26.43	0.01\\
26.44	0.01\\
26.45	0.01\\
26.46	0.01\\
26.47	0.01\\
26.48	0.01\\
26.49	0.01\\
26.5	0.01\\
26.51	0.01\\
26.52	0.01\\
26.53	0.01\\
26.54	0.01\\
26.55	0.01\\
26.56	0.01\\
26.57	0.01\\
26.58	0.01\\
26.59	0.01\\
26.6	0.01\\
26.61	0.01\\
26.62	0.01\\
26.63	0.01\\
26.64	0.01\\
26.65	0.01\\
26.66	0.01\\
26.67	0.01\\
26.68	0.01\\
26.69	0.01\\
26.7	0.01\\
26.71	0.01\\
26.72	0.01\\
26.73	0.01\\
26.74	0.01\\
26.75	0.01\\
26.76	0.01\\
26.77	0.01\\
26.78	0.01\\
26.79	0.01\\
26.8	0.01\\
26.81	0.01\\
26.82	0.01\\
26.83	0.01\\
26.84	0.01\\
26.85	0.01\\
26.86	0.01\\
26.87	0.01\\
26.88	0.01\\
26.89	0.01\\
26.9	0.01\\
26.91	0.01\\
26.92	0.01\\
26.93	0.01\\
26.94	0.01\\
26.95	0.01\\
26.96	0.01\\
26.97	0.01\\
26.98	0.01\\
26.99	0.01\\
27	0.01\\
27.01	0.01\\
27.02	0.01\\
27.03	0.01\\
27.04	0.01\\
27.05	0.01\\
27.06	0.01\\
27.07	0.01\\
27.08	0.01\\
27.09	0.01\\
27.1	0.01\\
27.11	0.01\\
27.12	0.01\\
27.13	0.01\\
27.14	0.01\\
27.15	0.01\\
27.16	0.01\\
27.17	0.01\\
27.18	0.01\\
27.19	0.01\\
27.2	0.01\\
27.21	0.01\\
27.22	0.01\\
27.23	0.01\\
27.24	0.01\\
27.25	0.01\\
27.26	0.01\\
27.27	0.01\\
27.28	0.01\\
27.29	0.01\\
27.3	0.01\\
27.31	0.01\\
27.32	0.01\\
27.33	0.01\\
27.34	0.01\\
27.35	0.01\\
27.36	0.01\\
27.37	0.01\\
27.38	0.01\\
27.39	0.01\\
27.4	0.01\\
27.41	0.01\\
27.42	0.01\\
27.43	0.01\\
27.44	0.01\\
27.45	0.01\\
27.46	0.01\\
27.47	0.01\\
27.48	0.01\\
27.49	0.01\\
27.5	0.01\\
27.51	0.01\\
27.52	0.01\\
27.53	0.01\\
27.54	0.01\\
27.55	0.01\\
27.56	0.01\\
27.57	0.01\\
27.58	0.01\\
27.59	0.01\\
27.6	0.01\\
27.61	0.01\\
27.62	0.01\\
27.63	0.01\\
27.64	0.01\\
27.65	0.01\\
27.66	0.01\\
27.67	0.01\\
27.68	0.01\\
27.69	0.01\\
27.7	0.01\\
27.71	0.01\\
27.72	0.01\\
27.73	0.01\\
27.74	0.01\\
27.75	0.01\\
27.76	0.01\\
27.77	0.01\\
27.78	0.01\\
27.79	0.01\\
27.8	0.01\\
27.81	0.01\\
27.82	0.01\\
27.83	0.01\\
27.84	0.01\\
27.85	0.01\\
27.86	0.01\\
27.87	0.01\\
27.88	0.01\\
27.89	0.01\\
27.9	0.01\\
27.91	0.01\\
27.92	0.01\\
27.93	0.01\\
27.94	0.01\\
27.95	0.01\\
27.96	0.01\\
27.97	0.01\\
27.98	0.01\\
27.99	0.01\\
28	0.01\\
28.01	0.01\\
28.02	0.01\\
28.03	0.01\\
28.04	0.01\\
28.05	0.01\\
28.06	0.01\\
28.07	0.01\\
28.08	0.01\\
28.09	0.01\\
28.1	0.01\\
28.11	0.01\\
28.12	0.01\\
28.13	0.01\\
28.14	0.01\\
28.15	0.01\\
28.16	0.01\\
28.17	0.01\\
28.18	0.01\\
28.19	0.01\\
28.2	0.01\\
28.21	0.01\\
28.22	0.01\\
28.23	0.01\\
28.24	0.01\\
28.25	0.01\\
28.26	0.01\\
28.27	0.01\\
28.28	0.01\\
28.29	0.01\\
28.3	0.01\\
28.31	0.01\\
28.32	0.01\\
28.33	0.01\\
28.34	0.01\\
28.35	0.01\\
28.36	0.01\\
28.37	0.01\\
28.38	0.01\\
28.39	0.01\\
28.4	0.01\\
28.41	0.01\\
28.42	0.01\\
28.43	0.01\\
28.44	0.01\\
28.45	0.01\\
28.46	0.01\\
28.47	0.01\\
28.48	0.01\\
28.49	0.01\\
28.5	0.01\\
28.51	0.01\\
28.52	0.01\\
28.53	0.01\\
28.54	0.01\\
28.55	0.01\\
28.56	0.01\\
28.57	0.01\\
28.58	0.01\\
28.59	0.01\\
28.6	0.01\\
28.61	0.01\\
28.62	0.01\\
28.63	0.01\\
28.64	0.01\\
28.65	0.01\\
28.66	0.01\\
28.67	0.01\\
28.68	0.01\\
28.69	0.01\\
28.7	0.01\\
28.71	0.01\\
28.72	0.01\\
28.73	0.01\\
28.74	0.01\\
28.75	0.01\\
28.76	0.01\\
28.77	0.01\\
28.78	0.01\\
28.79	0.01\\
28.8	0.01\\
28.81	0.01\\
28.82	0.01\\
28.83	0.01\\
28.84	0.01\\
28.85	0.01\\
28.86	0.01\\
28.87	0.01\\
28.88	0.01\\
28.89	0.01\\
28.9	0.01\\
28.91	0.01\\
28.92	0.01\\
28.93	0.01\\
28.94	0.01\\
28.95	0.01\\
28.96	0.01\\
28.97	0.01\\
28.98	0.01\\
28.99	0.01\\
29	0.01\\
29.01	0.01\\
29.02	0.01\\
29.03	0.01\\
29.04	0.01\\
29.05	0.01\\
29.06	0.01\\
29.07	0.01\\
29.08	0.01\\
29.09	0.01\\
29.1	0.01\\
29.11	0.01\\
29.12	0.01\\
29.13	0.01\\
29.14	0.01\\
29.15	0.01\\
29.16	0.01\\
29.17	0.01\\
29.18	0.01\\
29.19	0.01\\
29.2	0.01\\
29.21	0.01\\
29.22	0.01\\
29.23	0.01\\
29.24	0.01\\
29.25	0.01\\
29.26	0.01\\
29.27	0.01\\
29.28	0.01\\
29.29	0.01\\
29.3	0.01\\
29.31	0.01\\
29.32	0.01\\
29.33	0.01\\
29.34	0.01\\
29.35	0.01\\
29.36	0.01\\
29.37	0.01\\
29.38	0.01\\
29.39	0.01\\
29.4	0.01\\
29.41	0.01\\
29.42	0.01\\
29.43	0.01\\
29.44	0.01\\
29.45	0.01\\
29.46	0.01\\
29.47	0.01\\
29.48	0.01\\
29.49	0.01\\
29.5	0.01\\
29.51	0.01\\
29.52	0.01\\
29.53	0.01\\
29.54	0.01\\
29.55	0.01\\
29.56	0.01\\
29.57	0.01\\
29.58	0.01\\
29.59	0.01\\
29.6	0.01\\
29.61	0.01\\
29.62	0.01\\
29.63	0.01\\
29.64	0.01\\
29.65	0.01\\
29.66	0.01\\
29.67	0.01\\
29.68	0.01\\
29.69	0.01\\
29.7	0.01\\
29.71	0.01\\
29.72	0.01\\
29.73	0.01\\
29.74	0.01\\
29.75	0.01\\
29.76	0.01\\
29.77	0.01\\
29.78	0.01\\
29.79	0.01\\
29.8	0.01\\
29.81	0.01\\
29.82	0.01\\
29.83	0.01\\
29.84	0.01\\
29.85	0.01\\
29.86	0.01\\
29.87	0.01\\
29.88	0.01\\
29.89	0.01\\
29.9	0.01\\
29.91	0.01\\
29.92	0.01\\
29.93	0.01\\
29.94	0.01\\
29.95	0.01\\
29.96	0.01\\
29.97	0.01\\
29.98	0.01\\
29.99	0.01\\
30	0.01\\
30.01	0.01\\
30.02	0.01\\
30.03	0.01\\
30.04	0.01\\
30.05	0.01\\
30.06	0.01\\
30.07	0.01\\
30.08	0.01\\
30.09	0.01\\
30.1	0.01\\
30.11	0.01\\
30.12	0.01\\
30.13	0.01\\
30.14	0.01\\
30.15	0.01\\
30.16	0.01\\
30.17	0.01\\
30.18	0.01\\
30.19	0.01\\
30.2	0.01\\
30.21	0.01\\
30.22	0.01\\
30.23	0.01\\
30.24	0.01\\
30.25	0.01\\
30.26	0.01\\
30.27	0.01\\
30.28	0.01\\
30.29	0.01\\
30.3	0.01\\
30.31	0.01\\
30.32	0.01\\
30.33	0.01\\
30.34	0.01\\
30.35	0.01\\
30.36	0.01\\
30.37	0.01\\
30.38	0.01\\
30.39	0.01\\
30.4	0.01\\
30.41	0.01\\
30.42	0.01\\
30.43	0.01\\
30.44	0.01\\
30.45	0.01\\
30.46	0.01\\
30.47	0.01\\
30.48	0.01\\
30.49	0.01\\
30.5	0.01\\
30.51	0.01\\
30.52	0.01\\
30.53	0.01\\
30.54	0.01\\
30.55	0.01\\
30.56	0.01\\
30.57	0.01\\
30.58	0.01\\
30.59	0.01\\
30.6	0.01\\
30.61	0.01\\
30.62	0.01\\
30.63	0.01\\
30.64	0.01\\
30.65	0.01\\
30.66	0.01\\
30.67	0.01\\
30.68	0.01\\
30.69	0.01\\
30.7	0.01\\
30.71	0.01\\
30.72	0.01\\
30.73	0.01\\
30.74	0.01\\
30.75	0.01\\
30.76	0.01\\
30.77	0.01\\
30.78	0.01\\
30.79	0.01\\
30.8	0.01\\
30.81	0.01\\
30.82	0.01\\
30.83	0.01\\
30.84	0.01\\
30.85	0.01\\
30.86	0.01\\
30.87	0.01\\
30.88	0.01\\
30.89	0.01\\
30.9	0.01\\
30.91	0.01\\
30.92	0.01\\
30.93	0.01\\
30.94	0.01\\
30.95	0.01\\
30.96	0.01\\
30.97	0.01\\
30.98	0.01\\
30.99	0.01\\
31	0.01\\
31.01	0.01\\
31.02	0.01\\
31.03	0.01\\
31.04	0.01\\
31.05	0.01\\
31.06	0.01\\
31.07	0.01\\
31.08	0.01\\
31.09	0.01\\
31.1	0.01\\
31.11	0.01\\
31.12	0.01\\
31.13	0.01\\
31.14	0.01\\
31.15	0.01\\
31.16	0.01\\
31.17	0.01\\
31.18	0.01\\
31.19	0.01\\
31.2	0.01\\
31.21	0.01\\
31.22	0.01\\
31.23	0.01\\
31.24	0.01\\
31.25	0.01\\
31.26	0.01\\
31.27	0.01\\
31.28	0.01\\
31.29	0.01\\
31.3	0.01\\
31.31	0.01\\
31.32	0.01\\
31.33	0.01\\
31.34	0.01\\
31.35	0.01\\
31.36	0.01\\
31.37	0.01\\
31.38	0.01\\
31.39	0.01\\
31.4	0.01\\
31.41	0.01\\
31.42	0.01\\
31.43	0.01\\
31.44	0.01\\
31.45	0.01\\
31.46	0.01\\
31.47	0.01\\
31.48	0.01\\
31.49	0.01\\
31.5	0.01\\
31.51	0.01\\
31.52	0.01\\
31.53	0.01\\
31.54	0.01\\
31.55	0.01\\
31.56	0.01\\
31.57	0.01\\
31.58	0.01\\
31.59	0.01\\
31.6	0.01\\
31.61	0.01\\
31.62	0.01\\
31.63	0.01\\
31.64	0.01\\
31.65	0.01\\
31.66	0.01\\
31.67	0.01\\
31.68	0.01\\
31.69	0.01\\
31.7	0.01\\
31.71	0.01\\
31.72	0.01\\
31.73	0.01\\
31.74	0.01\\
31.75	0.01\\
31.76	0.01\\
31.77	0.01\\
31.78	0.01\\
31.79	0.01\\
31.8	0.01\\
31.81	0.01\\
31.82	0.01\\
31.83	0.01\\
31.84	0.01\\
31.85	0.01\\
31.86	0.01\\
31.87	0.01\\
31.88	0.01\\
31.89	0.01\\
31.9	0.01\\
31.91	0.01\\
31.92	0.01\\
31.93	0.01\\
31.94	0.01\\
31.95	0.01\\
31.96	0.01\\
31.97	0.01\\
31.98	0.01\\
31.99	0.01\\
32	0.01\\
32.01	0.01\\
32.02	0.01\\
32.03	0.01\\
32.04	0.01\\
32.05	0.01\\
32.06	0.01\\
32.07	0.01\\
32.08	0.01\\
32.09	0.01\\
32.1	0.01\\
32.11	0.01\\
32.12	0.01\\
32.13	0.01\\
32.14	0.01\\
32.15	0.01\\
32.16	0.01\\
32.17	0.01\\
32.18	0.01\\
32.19	0.01\\
32.2	0.01\\
32.21	0.01\\
32.22	0.01\\
32.23	0.01\\
32.24	0.01\\
32.25	0.01\\
32.26	0.01\\
32.27	0.01\\
32.28	0.01\\
32.29	0.01\\
32.3	0.01\\
32.31	0.01\\
32.32	0.01\\
32.33	0.01\\
32.34	0.01\\
32.35	0.01\\
32.36	0.01\\
32.37	0.01\\
32.38	0.01\\
32.39	0.01\\
32.4	0.01\\
32.41	0.01\\
32.42	0.01\\
32.43	0.01\\
32.44	0.01\\
32.45	0.01\\
32.46	0.01\\
32.47	0.01\\
32.48	0.01\\
32.49	0.01\\
32.5	0.01\\
32.51	0.01\\
32.52	0.01\\
32.53	0.01\\
32.54	0.01\\
32.55	0.01\\
32.56	0.01\\
32.57	0.01\\
32.58	0.01\\
32.59	0.01\\
32.6	0.01\\
32.61	0.01\\
32.62	0.01\\
32.63	0.01\\
32.64	0.01\\
32.65	0.01\\
32.66	0.01\\
32.67	0.01\\
32.68	0.01\\
32.69	0.01\\
32.7	0.01\\
32.71	0.01\\
32.72	0.01\\
32.73	0.01\\
32.74	0.01\\
32.75	0.01\\
32.76	0.01\\
32.77	0.01\\
32.78	0.01\\
32.79	0.01\\
32.8	0.01\\
32.81	0.01\\
32.82	0.01\\
32.83	0.01\\
32.84	0.01\\
32.85	0.01\\
32.86	0.01\\
32.87	0.01\\
32.88	0.01\\
32.89	0.01\\
32.9	0.01\\
32.91	0.01\\
32.92	0.01\\
32.93	0.01\\
32.94	0.01\\
32.95	0.01\\
32.96	0.01\\
32.97	0.01\\
32.98	0.01\\
32.99	0.01\\
33	0.01\\
33.01	0.01\\
33.02	0.01\\
33.03	0.01\\
33.04	0.01\\
33.05	0.01\\
33.06	0.01\\
33.07	0.01\\
33.08	0.01\\
33.09	0.01\\
33.1	0.01\\
33.11	0.01\\
33.12	0.01\\
33.13	0.01\\
33.14	0.01\\
33.15	0.01\\
33.16	0.01\\
33.17	0.01\\
33.18	0.01\\
33.19	0.01\\
33.2	0.01\\
33.21	0.01\\
33.22	0.01\\
33.23	0.01\\
33.24	0.01\\
33.25	0.01\\
33.26	0.01\\
33.27	0.01\\
33.28	0.01\\
33.29	0.01\\
33.3	0.01\\
33.31	0.01\\
33.32	0.01\\
33.33	0.01\\
33.34	0.01\\
33.35	0.01\\
33.36	0.01\\
33.37	0.01\\
33.38	0.01\\
33.39	0.01\\
33.4	0.01\\
33.41	0.01\\
33.42	0.01\\
33.43	0.01\\
33.44	0.01\\
33.45	0.01\\
33.46	0.01\\
33.47	0.01\\
33.48	0.01\\
33.49	0.01\\
33.5	0.01\\
33.51	0.01\\
33.52	0.01\\
33.53	0.01\\
33.54	0.01\\
33.55	0.01\\
33.56	0.01\\
33.57	0.01\\
33.58	0.01\\
33.59	0.01\\
33.6	0.01\\
33.61	0.01\\
33.62	0.01\\
33.63	0.01\\
33.64	0.01\\
33.65	0.01\\
33.66	0.01\\
33.67	0.01\\
33.68	0.01\\
33.69	0.01\\
33.7	0.01\\
33.71	0.01\\
33.72	0.01\\
33.73	0.01\\
33.74	0.01\\
33.75	0.01\\
33.76	0.01\\
33.77	0.01\\
33.78	0.01\\
33.79	0.01\\
33.8	0.01\\
33.81	0.01\\
33.82	0.01\\
33.83	0.01\\
33.84	0.01\\
33.85	0.01\\
33.86	0.01\\
33.87	0.01\\
33.88	0.01\\
33.89	0.01\\
33.9	0.01\\
33.91	0.01\\
33.92	0.01\\
33.93	0.01\\
33.94	0.01\\
33.95	0.01\\
33.96	0.01\\
33.97	0.01\\
33.98	0.01\\
33.99	0.01\\
34	0.01\\
34.01	0.01\\
34.02	0.01\\
34.03	0.01\\
34.04	0.01\\
34.05	0.01\\
34.06	0.01\\
34.07	0.01\\
34.08	0.01\\
34.09	0.01\\
34.1	0.01\\
34.11	0.01\\
34.12	0.01\\
34.13	0.01\\
34.14	0.01\\
34.15	0.01\\
34.16	0.01\\
34.17	0.01\\
34.18	0.01\\
34.19	0.01\\
34.2	0.01\\
34.21	0.01\\
34.22	0.01\\
34.23	0.01\\
34.24	0.01\\
34.25	0.01\\
34.26	0.01\\
34.27	0.01\\
34.28	0.01\\
34.29	0.01\\
34.3	0.01\\
34.31	0.01\\
34.32	0.01\\
34.33	0.01\\
34.34	0.01\\
34.35	0.01\\
34.36	0.01\\
34.37	0.01\\
34.38	0.01\\
34.39	0.01\\
34.4	0.01\\
34.41	0.01\\
34.42	0.01\\
34.43	0.01\\
34.44	0.01\\
34.45	0.01\\
34.46	0.01\\
34.47	0.01\\
34.48	0.01\\
34.49	0.01\\
34.5	0.01\\
34.51	0.01\\
34.52	0.01\\
34.53	0.01\\
34.54	0.01\\
34.55	0.01\\
34.56	0.01\\
34.57	0.01\\
34.58	0.01\\
34.59	0.01\\
34.6	0.01\\
34.61	0.01\\
34.62	0.01\\
34.63	0.01\\
34.64	0.01\\
34.65	0.01\\
34.66	0.01\\
34.67	0.01\\
34.68	0.01\\
34.69	0.01\\
34.7	0.01\\
34.71	0.01\\
34.72	0.01\\
34.73	0.01\\
34.74	0.01\\
34.75	0.01\\
34.76	0.01\\
34.77	0.01\\
34.78	0.01\\
34.79	0.01\\
34.8	0.01\\
34.81	0.01\\
34.82	0.01\\
34.83	0.01\\
34.84	0.01\\
34.85	0.01\\
34.86	0.01\\
34.87	0.01\\
34.88	0.01\\
34.89	0.01\\
34.9	0.01\\
34.91	0.01\\
34.92	0.01\\
34.93	0.01\\
34.94	0.01\\
34.95	0.01\\
34.96	0.01\\
34.97	0.01\\
34.98	0.01\\
34.99	0.01\\
35	0.01\\
35.01	0.01\\
35.02	0.01\\
35.03	0.01\\
35.04	0.01\\
35.05	0.01\\
35.06	0.01\\
35.07	0.01\\
35.08	0.01\\
35.09	0.01\\
35.1	0.01\\
35.11	0.01\\
35.12	0.01\\
35.13	0.01\\
35.14	0.01\\
35.15	0.01\\
35.16	0.01\\
35.17	0.01\\
35.18	0.01\\
35.19	0.01\\
35.2	0.01\\
35.21	0.01\\
35.22	0.01\\
35.23	0.01\\
35.24	0.01\\
35.25	0.01\\
35.26	0.01\\
35.27	0.01\\
35.28	0.01\\
35.29	0.01\\
35.3	0.01\\
35.31	0.01\\
35.32	0.01\\
35.33	0.01\\
35.34	0.01\\
35.35	0.01\\
35.36	0.01\\
35.37	0.01\\
35.38	0.01\\
35.39	0.01\\
35.4	0.01\\
35.41	0.01\\
35.42	0.01\\
35.43	0.01\\
35.44	0.01\\
35.45	0.01\\
35.46	0.01\\
35.47	0.01\\
35.48	0.01\\
35.49	0.01\\
35.5	0.01\\
35.51	0.01\\
35.52	0.01\\
35.53	0.01\\
35.54	0.01\\
35.55	0.01\\
35.56	0.01\\
35.57	0.01\\
35.58	0.01\\
35.59	0.01\\
35.6	0.01\\
35.61	0.01\\
35.62	0.01\\
35.63	0.01\\
35.64	0.01\\
35.65	0.01\\
35.66	0.01\\
35.67	0.01\\
35.68	0.01\\
35.69	0.01\\
35.7	0.01\\
35.71	0.01\\
35.72	0.01\\
35.73	0.01\\
35.74	0.01\\
35.75	0.01\\
35.76	0.01\\
35.77	0.01\\
35.78	0.01\\
35.79	0.01\\
35.8	0.01\\
35.81	0.01\\
35.82	0.01\\
35.83	0.01\\
35.84	0.01\\
35.85	0.01\\
35.86	0.01\\
35.87	0.01\\
35.88	0.01\\
35.89	0.01\\
35.9	0.01\\
35.91	0.01\\
35.92	0.01\\
35.93	0.01\\
35.94	0.01\\
35.95	0.01\\
35.96	0.01\\
35.97	0.01\\
35.98	0.01\\
35.99	0.01\\
36	0.01\\
36.01	0.01\\
36.02	0.01\\
36.03	0.01\\
36.04	0.01\\
36.05	0.01\\
36.06	0.01\\
36.07	0.01\\
36.08	0.01\\
36.09	0.01\\
36.1	0.01\\
36.11	0.01\\
36.12	0.01\\
36.13	0.01\\
36.14	0.01\\
36.15	0.01\\
36.16	0.01\\
36.17	0.01\\
36.18	0.01\\
36.19	0.01\\
36.2	0.01\\
36.21	0.01\\
36.22	0.01\\
36.23	0.01\\
36.24	0.01\\
36.25	0.01\\
36.26	0.01\\
36.27	0.01\\
36.28	0.01\\
36.29	0.01\\
36.3	0.01\\
36.31	0.01\\
36.32	0.01\\
36.33	0.01\\
36.34	0.01\\
36.35	0.01\\
36.36	0.01\\
36.37	0.01\\
36.38	0.01\\
36.39	0.01\\
36.4	0.01\\
36.41	0.01\\
36.42	0.01\\
36.43	0.01\\
36.44	0.01\\
36.45	0.01\\
36.46	0.01\\
36.47	0.01\\
36.48	0.01\\
36.49	0.01\\
36.5	0.01\\
36.51	0.01\\
36.52	0.01\\
36.53	0.01\\
36.54	0.01\\
36.55	0.01\\
36.56	0.01\\
36.57	0.01\\
36.58	0.01\\
36.59	0.01\\
36.6	0.01\\
36.61	0.01\\
36.62	0.01\\
36.63	0.01\\
36.64	0.01\\
36.65	0.01\\
36.66	0.01\\
36.67	0.01\\
36.68	0.01\\
36.69	0.01\\
36.7	0.01\\
36.71	0.01\\
36.72	0.01\\
36.73	0.01\\
36.74	0.01\\
36.75	0.01\\
36.76	0.01\\
36.77	0.01\\
36.78	0.01\\
36.79	0.01\\
36.8	0.01\\
36.81	0.01\\
36.82	0.01\\
36.83	0.01\\
36.84	0.01\\
36.85	0.01\\
36.86	0.01\\
36.87	0.01\\
36.88	0.01\\
36.89	0.01\\
36.9	0.01\\
36.91	0.01\\
36.92	0.01\\
36.93	0.01\\
36.94	0.01\\
36.95	0.01\\
36.96	0.01\\
36.97	0.01\\
36.98	0.01\\
36.99	0.01\\
37	0.01\\
37.01	0.01\\
37.02	0.01\\
37.03	0.01\\
37.04	0.01\\
37.05	0.01\\
37.06	0.01\\
37.07	0.01\\
37.08	0.01\\
37.09	0.01\\
37.1	0.01\\
37.11	0.01\\
37.12	0.01\\
37.13	0.01\\
37.14	0.01\\
37.15	0.01\\
37.16	0.01\\
37.17	0.01\\
37.18	0.01\\
37.19	0.01\\
37.2	0.01\\
37.21	0.01\\
37.22	0.01\\
37.23	0.01\\
37.24	0.01\\
37.25	0.01\\
37.26	0.01\\
37.27	0.01\\
37.28	0.01\\
37.29	0.01\\
37.3	0.01\\
37.31	0.01\\
37.32	0.01\\
37.33	0.01\\
37.34	0.01\\
37.35	0.01\\
37.36	0.01\\
37.37	0.01\\
37.38	0.01\\
37.39	0.01\\
37.4	0.01\\
37.41	0.01\\
37.42	0.01\\
37.43	0.01\\
37.44	0.01\\
37.45	0.01\\
37.46	0.01\\
37.47	0.01\\
37.48	0.01\\
37.49	0.01\\
37.5	0.01\\
37.51	0.01\\
37.52	0.01\\
37.53	0.01\\
37.54	0.01\\
37.55	0.01\\
37.56	0.01\\
37.57	0.01\\
37.58	0.01\\
37.59	0.01\\
37.6	0.01\\
37.61	0.01\\
37.62	0.01\\
37.63	0.01\\
37.64	0.01\\
37.65	0.01\\
37.66	0.01\\
37.67	0.01\\
37.68	0.01\\
37.69	0.01\\
37.7	0.01\\
37.71	0.01\\
37.72	0.01\\
37.73	0.01\\
37.74	0.01\\
37.75	0.01\\
37.76	0.01\\
37.77	0.01\\
37.78	0.01\\
37.79	0.01\\
37.8	0.01\\
37.81	0.01\\
37.82	0.01\\
37.83	0.01\\
37.84	0.01\\
37.85	0.01\\
37.86	0.01\\
37.87	0.01\\
37.88	0.01\\
37.89	0.01\\
37.9	0.01\\
37.91	0.01\\
37.92	0.01\\
37.93	0.01\\
37.94	0.01\\
37.95	0.01\\
37.96	0.01\\
37.97	0.01\\
37.98	0.01\\
37.99	0.01\\
38	0.01\\
38.01	0.01\\
38.02	0.01\\
38.03	0.01\\
38.04	0.01\\
38.05	0.01\\
38.06	0.01\\
38.07	0.01\\
38.08	0.01\\
38.09	0.01\\
38.1	0.01\\
38.11	0.01\\
38.12	0.01\\
38.13	0.01\\
38.14	0.01\\
38.15	0.01\\
38.16	0.01\\
38.17	0.01\\
38.18	0.01\\
38.19	0.01\\
38.2	0.01\\
38.21	0.01\\
38.22	0.01\\
38.23	0.01\\
38.24	0.01\\
38.25	0.01\\
38.26	0.01\\
38.27	0.01\\
38.28	0.01\\
38.29	0.01\\
38.3	0.01\\
38.31	0.01\\
38.32	0.01\\
38.33	0.01\\
38.34	0.01\\
38.35	0.01\\
38.36	0.01\\
38.37	0.01\\
38.38	0.01\\
38.39	0.01\\
38.4	0.01\\
38.41	0.01\\
38.42	0.01\\
38.43	0.01\\
38.44	0.01\\
38.45	0.01\\
38.46	0.01\\
38.47	0.01\\
38.48	0.01\\
38.49	0.01\\
38.5	0.01\\
38.51	0.01\\
38.52	0.01\\
38.53	0.01\\
38.54	0.01\\
38.55	0.01\\
38.56	0.01\\
38.57	0.01\\
38.58	0.01\\
38.59	0.01\\
38.6	0.01\\
38.61	0.01\\
38.62	0.01\\
38.63	0.01\\
38.64	0.01\\
38.65	0.01\\
38.66	0.01\\
38.67	0.01\\
38.68	0.01\\
38.69	0.01\\
38.7	0.01\\
38.71	0.01\\
38.72	0.01\\
38.73	0.01\\
38.74	0.01\\
38.75	0.01\\
38.76	0.01\\
38.77	0.01\\
38.78	0.01\\
38.79	0.01\\
38.8	0.01\\
38.81	0.01\\
38.82	0.01\\
38.83	0.01\\
38.84	0.01\\
38.85	0.01\\
38.86	0.01\\
38.87	0.01\\
38.88	0.01\\
38.89	0.01\\
38.9	0.01\\
38.91	0.01\\
38.92	0.01\\
38.93	0.01\\
38.94	0.01\\
38.95	0.01\\
38.96	0.01\\
38.97	0.01\\
38.98	0.01\\
38.99	0.01\\
39	0.01\\
39.01	0.01\\
39.02	0.01\\
39.03	0.01\\
39.04	0.01\\
39.05	0.01\\
39.06	0.01\\
39.07	0.01\\
39.08	0.01\\
39.09	0.01\\
39.1	0.01\\
39.11	0.01\\
39.12	0.01\\
39.13	0.01\\
39.14	0.01\\
39.15	0.01\\
39.16	0.01\\
39.17	0.01\\
39.18	0.01\\
39.19	0.01\\
39.2	0.01\\
39.21	0.01\\
39.22	0.01\\
39.23	0.01\\
39.24	0.01\\
39.25	0.01\\
39.26	0.01\\
39.27	0.01\\
39.28	0.01\\
39.29	0.01\\
39.3	0.01\\
39.31	0.01\\
39.32	0.01\\
39.33	0.01\\
39.34	0.01\\
39.35	0.01\\
39.36	0.01\\
39.37	0.01\\
39.38	0.01\\
39.39	0.01\\
39.4	0.01\\
39.41	0.01\\
39.42	0.01\\
39.43	0.01\\
39.44	0.01\\
39.45	0.01\\
39.46	0.01\\
39.47	0.01\\
39.48	0.01\\
39.49	0.01\\
39.5	0.01\\
39.51	0.01\\
39.52	0.01\\
39.53	0.01\\
39.54	0.01\\
39.55	0.01\\
39.56	0.01\\
39.57	0.01\\
39.58	0.01\\
39.59	0.01\\
39.6	0.01\\
39.61	0.01\\
39.62	0.01\\
39.63	0.01\\
39.64	0.01\\
39.65	0.01\\
39.66	0.01\\
39.67	0.01\\
39.68	0.01\\
39.69	0.01\\
39.7	0.01\\
39.71	0.01\\
39.72	0.01\\
39.73	0.01\\
39.74	0.01\\
39.75	0.01\\
39.76	0.01\\
39.77	0.01\\
39.78	0.01\\
39.79	0.01\\
39.8	0.01\\
39.81	0.01\\
39.82	0.01\\
39.83	0.01\\
39.84	0.01\\
39.85	0.01\\
39.86	0.01\\
39.87	0.01\\
39.88	0.01\\
39.89	0.01\\
39.9	0.01\\
39.91	0.01\\
39.92	0.01\\
39.93	0.01\\
39.94	0.01\\
39.95	0.01\\
39.96	0.01\\
39.97	0.01\\
39.98	0.01\\
39.99	0.01\\
40	0.01\\
40.01	0.01\\
};
\addplot [color=green,dashed,forget plot]
  table[row sep=crcr]{%
40.01	0.01\\
40.02	0.01\\
40.03	0.01\\
40.04	0.01\\
40.05	0.01\\
40.06	0.01\\
40.07	0.01\\
40.08	0.01\\
40.09	0.01\\
40.1	0.01\\
40.11	0.01\\
40.12	0.01\\
40.13	0.01\\
40.14	0.01\\
40.15	0.01\\
40.16	0.01\\
40.17	0.01\\
40.18	0.01\\
40.19	0.01\\
40.2	0.01\\
40.21	0.01\\
40.22	0.01\\
40.23	0.01\\
40.24	0.01\\
40.25	0.01\\
40.26	0.01\\
40.27	0.01\\
40.28	0.01\\
40.29	0.01\\
40.3	0.01\\
40.31	0.01\\
40.32	0.01\\
40.33	0.01\\
40.34	0.01\\
40.35	0.01\\
40.36	0.01\\
40.37	0.01\\
40.38	0.01\\
40.39	0.01\\
40.4	0.01\\
40.41	0.01\\
40.42	0.01\\
40.43	0.01\\
40.44	0.01\\
40.45	0.01\\
40.46	0.01\\
40.47	0.01\\
40.48	0.01\\
40.49	0.01\\
40.5	0.01\\
40.51	0.01\\
40.52	0.01\\
40.53	0.01\\
40.54	0.01\\
40.55	0.01\\
40.56	0.01\\
40.57	0.01\\
40.58	0.01\\
40.59	0.01\\
40.6	0.01\\
40.61	0.01\\
40.62	0.01\\
40.63	0.01\\
40.64	0.01\\
40.65	0.01\\
40.66	0.01\\
40.67	0.01\\
40.68	0.01\\
40.69	0.01\\
40.7	0.01\\
40.71	0.01\\
40.72	0.01\\
40.73	0.01\\
40.74	0.01\\
40.75	0.01\\
40.76	0.01\\
40.77	0.01\\
40.78	0.01\\
40.79	0.01\\
40.8	0.01\\
40.81	0.01\\
40.82	0.01\\
40.83	0.01\\
40.84	0.01\\
40.85	0.01\\
40.86	0.01\\
40.87	0.01\\
40.88	0.01\\
40.89	0.01\\
40.9	0.01\\
40.91	0.01\\
40.92	0.01\\
40.93	0.01\\
40.94	0.01\\
40.95	0.01\\
40.96	0.01\\
40.97	0.01\\
40.98	0.01\\
40.99	0.01\\
41	0.01\\
41.01	0.01\\
41.02	0.01\\
41.03	0.01\\
41.04	0.01\\
41.05	0.01\\
41.06	0.01\\
41.07	0.01\\
41.08	0.01\\
41.09	0.01\\
41.1	0.01\\
41.11	0.01\\
41.12	0.01\\
41.13	0.01\\
41.14	0.01\\
41.15	0.01\\
41.16	0.01\\
41.17	0.01\\
41.18	0.01\\
41.19	0.01\\
41.2	0.01\\
41.21	0.01\\
41.22	0.01\\
41.23	0.01\\
41.24	0.01\\
41.25	0.01\\
41.26	0.01\\
41.27	0.01\\
41.28	0.01\\
41.29	0.01\\
41.3	0.01\\
41.31	0.01\\
41.32	0.01\\
41.33	0.01\\
41.34	0.01\\
41.35	0.01\\
41.36	0.01\\
41.37	0.01\\
41.38	0.01\\
41.39	0.01\\
41.4	0.01\\
41.41	0.01\\
41.42	0.01\\
41.43	0.01\\
41.44	0.01\\
41.45	0.01\\
41.46	0.01\\
41.47	0.01\\
41.48	0.01\\
41.49	0.01\\
41.5	0.01\\
41.51	0.01\\
41.52	0.01\\
41.53	0.01\\
41.54	0.01\\
41.55	0.01\\
41.56	0.01\\
41.57	0.01\\
41.58	0.01\\
41.59	0.01\\
41.6	0.01\\
41.61	0.01\\
41.62	0.01\\
41.63	0.01\\
41.64	0.01\\
41.65	0.01\\
41.66	0.01\\
41.67	0.01\\
41.68	0.01\\
41.69	0.01\\
41.7	0.01\\
41.71	0.01\\
41.72	0.01\\
41.73	0.01\\
41.74	0.01\\
41.75	0.01\\
41.76	0.01\\
41.77	0.01\\
41.78	0.01\\
41.79	0.01\\
41.8	0.01\\
41.81	0.01\\
41.82	0.01\\
41.83	0.01\\
41.84	0.01\\
41.85	0.01\\
41.86	0.01\\
41.87	0.01\\
41.88	0.01\\
41.89	0.01\\
41.9	0.01\\
41.91	0.01\\
41.92	0.01\\
41.93	0.01\\
41.94	0.01\\
41.95	0.01\\
41.96	0.01\\
41.97	0.01\\
41.98	0.01\\
41.99	0.01\\
42	0.01\\
42.01	0.01\\
42.02	0.01\\
42.03	0.01\\
42.04	0.01\\
42.05	0.01\\
42.06	0.01\\
42.07	0.01\\
42.08	0.01\\
42.09	0.01\\
42.1	0.01\\
42.11	0.01\\
42.12	0.01\\
42.13	0.01\\
42.14	0.01\\
42.15	0.01\\
42.16	0.01\\
42.17	0.01\\
42.18	0.01\\
42.19	0.01\\
42.2	0.01\\
42.21	0.01\\
42.22	0.01\\
42.23	0.01\\
42.24	0.01\\
42.25	0.01\\
42.26	0.01\\
42.27	0.01\\
42.28	0.01\\
42.29	0.01\\
42.3	0.01\\
42.31	0.01\\
42.32	0.01\\
42.33	0.01\\
42.34	0.01\\
42.35	0.01\\
42.36	0.01\\
42.37	0.01\\
42.38	0.01\\
42.39	0.01\\
42.4	0.01\\
42.41	0.01\\
42.42	0.01\\
42.43	0.01\\
42.44	0.01\\
42.45	0.01\\
42.46	0.01\\
42.47	0.01\\
42.48	0.01\\
42.49	0.01\\
42.5	0.01\\
42.51	0.01\\
42.52	0.01\\
42.53	0.01\\
42.54	0.01\\
42.55	0.01\\
42.56	0.01\\
42.57	0.01\\
42.58	0.01\\
42.59	0.01\\
42.6	0.01\\
42.61	0.01\\
42.62	0.01\\
42.63	0.01\\
42.64	0.01\\
42.65	0.01\\
42.66	0.01\\
42.67	0.01\\
42.68	0.01\\
42.69	0.01\\
42.7	0.01\\
42.71	0.01\\
42.72	0.01\\
42.73	0.01\\
42.74	0.01\\
42.75	0.01\\
42.76	0.01\\
42.77	0.01\\
42.78	0.01\\
42.79	0.01\\
42.8	0.01\\
42.81	0.01\\
42.82	0.01\\
42.83	0.01\\
42.84	0.01\\
42.85	0.01\\
42.86	0.01\\
42.87	0.01\\
42.88	0.01\\
42.89	0.01\\
42.9	0.01\\
42.91	0.01\\
42.92	0.01\\
42.93	0.01\\
42.94	0.01\\
42.95	0.01\\
42.96	0.01\\
42.97	0.01\\
42.98	0.01\\
42.99	0.01\\
43	0.01\\
43.01	0.01\\
43.02	0.01\\
43.03	0.01\\
43.04	0.01\\
43.05	0.01\\
43.06	0.01\\
43.07	0.01\\
43.08	0.01\\
43.09	0.01\\
43.1	0.01\\
43.11	0.01\\
43.12	0.01\\
43.13	0.01\\
43.14	0.01\\
43.15	0.01\\
43.16	0.01\\
43.17	0.01\\
43.18	0.01\\
43.19	0.01\\
43.2	0.01\\
43.21	0.01\\
43.22	0.01\\
43.23	0.01\\
43.24	0.01\\
43.25	0.01\\
43.26	0.01\\
43.27	0.01\\
43.28	0.01\\
43.29	0.01\\
43.3	0.01\\
43.31	0.01\\
43.32	0.01\\
43.33	0.01\\
43.34	0.01\\
43.35	0.01\\
43.36	0.01\\
43.37	0.01\\
43.38	0.01\\
43.39	0.01\\
43.4	0.01\\
43.41	0.01\\
43.42	0.01\\
43.43	0.01\\
43.44	0.01\\
43.45	0.01\\
43.46	0.01\\
43.47	0.01\\
43.48	0.01\\
43.49	0.01\\
43.5	0.01\\
43.51	0.01\\
43.52	0.01\\
43.53	0.01\\
43.54	0.01\\
43.55	0.01\\
43.56	0.01\\
43.57	0.01\\
43.58	0.01\\
43.59	0.01\\
43.6	0.01\\
43.61	0.01\\
43.62	0.01\\
43.63	0.01\\
43.64	0.01\\
43.65	0.01\\
43.66	0.01\\
43.67	0.01\\
43.68	0.01\\
43.69	0.01\\
43.7	0.01\\
43.71	0.01\\
43.72	0.01\\
43.73	0.01\\
43.74	0.01\\
43.75	0.01\\
43.76	0.01\\
43.77	0.01\\
43.78	0.01\\
43.79	0.01\\
43.8	0.01\\
43.81	0.01\\
43.82	0.01\\
43.83	0.01\\
43.84	0.01\\
43.85	0.01\\
43.86	0.01\\
43.87	0.01\\
43.88	0.01\\
43.89	0.01\\
43.9	0.01\\
43.91	0.01\\
43.92	0.01\\
43.93	0.01\\
43.94	0.01\\
43.95	0.01\\
43.96	0.01\\
43.97	0.01\\
43.98	0.01\\
43.99	0.01\\
44	0.01\\
44.01	0.01\\
44.02	0.01\\
44.03	0.01\\
44.04	0.01\\
44.05	0.01\\
44.06	0.01\\
44.07	0.01\\
44.08	0.01\\
44.09	0.01\\
44.1	0.01\\
44.11	0.01\\
44.12	0.01\\
44.13	0.01\\
44.14	0.01\\
44.15	0.01\\
44.16	0.01\\
44.17	0.01\\
44.18	0.01\\
44.19	0.01\\
44.2	0.01\\
44.21	0.01\\
44.22	0.01\\
44.23	0.01\\
44.24	0.01\\
44.25	0.01\\
44.26	0.01\\
44.27	0.01\\
44.28	0.01\\
44.29	0.01\\
44.3	0.01\\
44.31	0.01\\
44.32	0.01\\
44.33	0.01\\
44.34	0.01\\
44.35	0.01\\
44.36	0.01\\
44.37	0.01\\
44.38	0.01\\
44.39	0.01\\
44.4	0.01\\
44.41	0.01\\
44.42	0.01\\
44.43	0.01\\
44.44	0.01\\
44.45	0.01\\
44.46	0.01\\
44.47	0.01\\
44.48	0.01\\
44.49	0.01\\
44.5	0.01\\
44.51	0.01\\
44.52	0.01\\
44.53	0.01\\
44.54	0.01\\
44.55	0.01\\
44.56	0.01\\
44.57	0.01\\
44.58	0.01\\
44.59	0.01\\
44.6	0.01\\
44.61	0.01\\
44.62	0.01\\
44.63	0.01\\
44.64	0.01\\
44.65	0.01\\
44.66	0.01\\
44.67	0.01\\
44.68	0.01\\
44.69	0.01\\
44.7	0.01\\
44.71	0.01\\
44.72	0.01\\
44.73	0.01\\
44.74	0.01\\
44.75	0.01\\
44.76	0.01\\
44.77	0.01\\
44.78	0.01\\
44.79	0.01\\
44.8	0.01\\
44.81	0.01\\
44.82	0.01\\
44.83	0.01\\
44.84	0.01\\
44.85	0.01\\
44.86	0.01\\
44.87	0.01\\
44.88	0.01\\
44.89	0.01\\
44.9	0.01\\
44.91	0.01\\
44.92	0.01\\
44.93	0.01\\
44.94	0.01\\
44.95	0.01\\
44.96	0.01\\
44.97	0.01\\
44.98	0.01\\
44.99	0.01\\
45	0.01\\
45.01	0.01\\
45.02	0.01\\
45.03	0.01\\
45.04	0.01\\
45.05	0.01\\
45.06	0.01\\
45.07	0.01\\
45.08	0.01\\
45.09	0.01\\
45.1	0.01\\
45.11	0.01\\
45.12	0.01\\
45.13	0.01\\
45.14	0.01\\
45.15	0.01\\
45.16	0.01\\
45.17	0.01\\
45.18	0.01\\
45.19	0.01\\
45.2	0.01\\
45.21	0.01\\
45.22	0.01\\
45.23	0.01\\
45.24	0.01\\
45.25	0.01\\
45.26	0.01\\
45.27	0.01\\
45.28	0.01\\
45.29	0.01\\
45.3	0.01\\
45.31	0.01\\
45.32	0.01\\
45.33	0.01\\
45.34	0.01\\
45.35	0.01\\
45.36	0.01\\
45.37	0.01\\
45.38	0.01\\
45.39	0.01\\
45.4	0.01\\
45.41	0.01\\
45.42	0.01\\
45.43	0.01\\
45.44	0.01\\
45.45	0.01\\
45.46	0.01\\
45.47	0.01\\
45.48	0.01\\
45.49	0.01\\
45.5	0.01\\
45.51	0.01\\
45.52	0.01\\
45.53	0.01\\
45.54	0.01\\
45.55	0.01\\
45.56	0.01\\
45.57	0.01\\
45.58	0.01\\
45.59	0.01\\
45.6	0.01\\
45.61	0.01\\
45.62	0.01\\
45.63	0.01\\
45.64	0.01\\
45.65	0.01\\
45.66	0.01\\
45.67	0.01\\
45.68	0.01\\
45.69	0.01\\
45.7	0.01\\
45.71	0.01\\
45.72	0.01\\
45.73	0.01\\
45.74	0.01\\
45.75	0.01\\
45.76	0.01\\
45.77	0.01\\
45.78	0.01\\
45.79	0.01\\
45.8	0.01\\
45.81	0.01\\
45.82	0.01\\
45.83	0.01\\
45.84	0.01\\
45.85	0.01\\
45.86	0.01\\
45.87	0.01\\
45.88	0.01\\
45.89	0.01\\
45.9	0.01\\
45.91	0.01\\
45.92	0.01\\
45.93	0.01\\
45.94	0.01\\
45.95	0.01\\
45.96	0.01\\
45.97	0.01\\
45.98	0.01\\
45.99	0.01\\
46	0.01\\
46.01	0.01\\
46.02	0.01\\
46.03	0.01\\
46.04	0.01\\
46.05	0.01\\
46.06	0.01\\
46.07	0.01\\
46.08	0.01\\
46.09	0.01\\
46.1	0.01\\
46.11	0.01\\
46.12	0.01\\
46.13	0.01\\
46.14	0.01\\
46.15	0.01\\
46.16	0.01\\
46.17	0.01\\
46.18	0.01\\
46.19	0.01\\
46.2	0.01\\
46.21	0.01\\
46.22	0.01\\
46.23	0.01\\
46.24	0.01\\
46.25	0.01\\
46.26	0.01\\
46.27	0.01\\
46.28	0.01\\
46.29	0.01\\
46.3	0.01\\
46.31	0.01\\
46.32	0.01\\
46.33	0.01\\
46.34	0.01\\
46.35	0.01\\
46.36	0.01\\
46.37	0.01\\
46.38	0.01\\
46.39	0.01\\
46.4	0.01\\
46.41	0.01\\
46.42	0.01\\
46.43	0.01\\
46.44	0.01\\
46.45	0.01\\
46.46	0.01\\
46.47	0.01\\
46.48	0.01\\
46.49	0.01\\
46.5	0.01\\
46.51	0.01\\
46.52	0.01\\
46.53	0.01\\
46.54	0.01\\
46.55	0.01\\
46.56	0.01\\
46.57	0.01\\
46.58	0.01\\
46.59	0.01\\
46.6	0.01\\
46.61	0.01\\
46.62	0.01\\
46.63	0.01\\
46.64	0.01\\
46.65	0.01\\
46.66	0.01\\
46.67	0.01\\
46.68	0.01\\
46.69	0.01\\
46.7	0.01\\
46.71	0.01\\
46.72	0.01\\
46.73	0.01\\
46.74	0.01\\
46.75	0.01\\
46.76	0.01\\
46.77	0.01\\
46.78	0.01\\
46.79	0.01\\
46.8	0.01\\
46.81	0.01\\
46.82	0.01\\
46.83	0.01\\
46.84	0.01\\
46.85	0.01\\
46.86	0.01\\
46.87	0.01\\
46.88	0.01\\
46.89	0.01\\
46.9	0.01\\
46.91	0.01\\
46.92	0.01\\
46.93	0.01\\
46.94	0.01\\
46.95	0.01\\
46.96	0.01\\
46.97	0.01\\
46.98	0.01\\
46.99	0.01\\
47	0.01\\
47.01	0.01\\
47.02	0.01\\
47.03	0.01\\
47.04	0.01\\
47.05	0.01\\
47.06	0.01\\
47.07	0.01\\
47.08	0.01\\
47.09	0.01\\
47.1	0.01\\
47.11	0.01\\
47.12	0.01\\
47.13	0.01\\
47.14	0.01\\
47.15	0.01\\
47.16	0.01\\
47.17	0.01\\
47.18	0.01\\
47.19	0.01\\
47.2	0.01\\
47.21	0.01\\
47.22	0.01\\
47.23	0.01\\
47.24	0.01\\
47.25	0.01\\
47.26	0.01\\
47.27	0.01\\
47.28	0.01\\
47.29	0.01\\
47.3	0.01\\
47.31	0.01\\
47.32	0.01\\
47.33	0.01\\
47.34	0.01\\
47.35	0.01\\
47.36	0.01\\
47.37	0.01\\
47.38	0.01\\
47.39	0.01\\
47.4	0.01\\
47.41	0.01\\
47.42	0.01\\
47.43	0.01\\
47.44	0.01\\
47.45	0.01\\
47.46	0.01\\
47.47	0.01\\
47.48	0.01\\
47.49	0.01\\
47.5	0.01\\
47.51	0.01\\
47.52	0.01\\
47.53	0.01\\
47.54	0.01\\
47.55	0.01\\
47.56	0.01\\
47.57	0.01\\
47.58	0.01\\
47.59	0.01\\
47.6	0.01\\
47.61	0.01\\
47.62	0.01\\
47.63	0.01\\
47.64	0.01\\
47.65	0.01\\
47.66	0.01\\
47.67	0.01\\
47.68	0.01\\
47.69	0.01\\
47.7	0.01\\
47.71	0.01\\
47.72	0.01\\
47.73	0.01\\
47.74	0.01\\
47.75	0.01\\
47.76	0.01\\
47.77	0.01\\
47.78	0.01\\
47.79	0.01\\
47.8	0.01\\
47.81	0.01\\
47.82	0.01\\
47.83	0.01\\
47.84	0.01\\
47.85	0.01\\
47.86	0.01\\
47.87	0.01\\
47.88	0.01\\
47.89	0.01\\
47.9	0.01\\
47.91	0.01\\
47.92	0.01\\
47.93	0.01\\
47.94	0.01\\
47.95	0.01\\
47.96	0.01\\
47.97	0.01\\
47.98	0.01\\
47.99	0.01\\
48	0.01\\
48.01	0.01\\
48.02	0.01\\
48.03	0.01\\
48.04	0.01\\
48.05	0.01\\
48.06	0.01\\
48.07	0.01\\
48.08	0.01\\
48.09	0.01\\
48.1	0.01\\
48.11	0.01\\
48.12	0.01\\
48.13	0.01\\
48.14	0.01\\
48.15	0.01\\
48.16	0.01\\
48.17	0.01\\
48.18	0.01\\
48.19	0.01\\
48.2	0.01\\
48.21	0.01\\
48.22	0.01\\
48.23	0.01\\
48.24	0.01\\
48.25	0.01\\
48.26	0.01\\
48.27	0.01\\
48.28	0.01\\
48.29	0.01\\
48.3	0.01\\
48.31	0.01\\
48.32	0.01\\
48.33	0.01\\
48.34	0.01\\
48.35	0.01\\
48.36	0.01\\
48.37	0.01\\
48.38	0.01\\
48.39	0.01\\
48.4	0.01\\
48.41	0.01\\
48.42	0.01\\
48.43	0.01\\
48.44	0.01\\
48.45	0.01\\
48.46	0.01\\
48.47	0.01\\
48.48	0.01\\
48.49	0.01\\
48.5	0.01\\
48.51	0.01\\
48.52	0.01\\
48.53	0.01\\
48.54	0.01\\
48.55	0.01\\
48.56	0.01\\
48.57	0.01\\
48.58	0.01\\
48.59	0.01\\
48.6	0.01\\
48.61	0.01\\
48.62	0.01\\
48.63	0.01\\
48.64	0.01\\
48.65	0.01\\
48.66	0.01\\
48.67	0.01\\
48.68	0.01\\
48.69	0.01\\
48.7	0.01\\
48.71	0.01\\
48.72	0.01\\
48.73	0.01\\
48.74	0.01\\
48.75	0.01\\
48.76	0.01\\
48.77	0.01\\
48.78	0.01\\
48.79	0.01\\
48.8	0.01\\
48.81	0.01\\
48.82	0.01\\
48.83	0.01\\
48.84	0.01\\
48.85	0.01\\
48.86	0.01\\
48.87	0.01\\
48.88	0.01\\
48.89	0.01\\
48.9	0.01\\
48.91	0.01\\
48.92	0.01\\
48.93	0.01\\
48.94	0.01\\
48.95	0.01\\
48.96	0.01\\
48.97	0.01\\
48.98	0.01\\
48.99	0.01\\
49	0.01\\
49.01	0.01\\
49.02	0.01\\
49.03	0.01\\
49.04	0.01\\
49.05	0.01\\
49.06	0.01\\
49.07	0.01\\
49.08	0.01\\
49.09	0.01\\
49.1	0.01\\
49.11	0.01\\
49.12	0.01\\
49.13	0.01\\
49.14	0.01\\
49.15	0.01\\
49.16	0.01\\
49.17	0.01\\
49.18	0.01\\
49.19	0.01\\
49.2	0.01\\
49.21	0.01\\
49.22	0.01\\
49.23	0.01\\
49.24	0.01\\
49.25	0.01\\
49.26	0.01\\
49.27	0.01\\
49.28	0.01\\
49.29	0.01\\
49.3	0.01\\
49.31	0.01\\
49.32	0.01\\
49.33	0.01\\
49.34	0.01\\
49.35	0.01\\
49.36	0.01\\
49.37	0.01\\
49.38	0.01\\
49.39	0.01\\
49.4	0.01\\
49.41	0.01\\
49.42	0.01\\
49.43	0.01\\
49.44	0.01\\
49.45	0.01\\
49.46	0.01\\
49.47	0.01\\
49.48	0.01\\
49.49	0.01\\
49.5	0.01\\
49.51	0.01\\
49.52	0.01\\
49.53	0.01\\
49.54	0.01\\
49.55	0.01\\
49.56	0.01\\
49.57	0.01\\
49.58	0.01\\
49.59	0.01\\
49.6	0.01\\
49.61	0.01\\
49.62	0.01\\
49.63	0.01\\
49.64	0.01\\
49.65	0.01\\
49.66	0.01\\
49.67	0.01\\
49.68	0.01\\
49.69	0.01\\
49.7	0.01\\
49.71	0.01\\
49.72	0.01\\
49.73	0.01\\
49.74	0.01\\
49.75	0.01\\
49.76	0.01\\
49.77	0.01\\
49.78	0.01\\
49.79	0.01\\
49.8	0.01\\
49.81	0.01\\
49.82	0.01\\
49.83	0.01\\
49.84	0.01\\
49.85	0.01\\
49.86	0.01\\
49.87	0.01\\
49.88	0.01\\
49.89	0.01\\
49.9	0.01\\
49.91	0.01\\
49.92	0.01\\
49.93	0.01\\
49.94	0.01\\
49.95	0.01\\
49.96	0.01\\
49.97	0.01\\
49.98	0.01\\
49.99	0.01\\
50	0.01\\
50.01	0.01\\
50.02	0.01\\
50.03	0.01\\
50.04	0.01\\
50.05	0.01\\
50.06	0.01\\
50.07	0.01\\
50.08	0.01\\
50.09	0.01\\
50.1	0.01\\
50.11	0.01\\
50.12	0.01\\
50.13	0.01\\
50.14	0.01\\
50.15	0.01\\
50.16	0.01\\
50.17	0.01\\
50.18	0.01\\
50.19	0.01\\
50.2	0.01\\
50.21	0.01\\
50.22	0.01\\
50.23	0.01\\
50.24	0.01\\
50.25	0.01\\
50.26	0.01\\
50.27	0.01\\
50.28	0.01\\
50.29	0.01\\
50.3	0.01\\
50.31	0.01\\
50.32	0.01\\
50.33	0.01\\
50.34	0.01\\
50.35	0.01\\
50.36	0.01\\
50.37	0.01\\
50.38	0.01\\
50.39	0.01\\
50.4	0.01\\
50.41	0.01\\
50.42	0.01\\
50.43	0.01\\
50.44	0.01\\
50.45	0.01\\
50.46	0.01\\
50.47	0.01\\
50.48	0.01\\
50.49	0.01\\
50.5	0.01\\
50.51	0.01\\
50.52	0.01\\
50.53	0.01\\
50.54	0.01\\
50.55	0.01\\
50.56	0.01\\
50.57	0.01\\
50.58	0.01\\
50.59	0.01\\
50.6	0.01\\
50.61	0.01\\
50.62	0.01\\
50.63	0.01\\
50.64	0.01\\
50.65	0.01\\
50.66	0.01\\
50.67	0.01\\
50.68	0.01\\
50.69	0.01\\
50.7	0.01\\
50.71	0.01\\
50.72	0.01\\
50.73	0.01\\
50.74	0.01\\
50.75	0.01\\
50.76	0.01\\
50.77	0.01\\
50.78	0.01\\
50.79	0.01\\
50.8	0.01\\
50.81	0.01\\
50.82	0.01\\
50.83	0.01\\
50.84	0.01\\
50.85	0.01\\
50.86	0.01\\
50.87	0.01\\
50.88	0.01\\
50.89	0.01\\
50.9	0.01\\
50.91	0.01\\
50.92	0.01\\
50.93	0.01\\
50.94	0.01\\
50.95	0.01\\
50.96	0.01\\
50.97	0.01\\
50.98	0.01\\
50.99	0.01\\
51	0.01\\
51.01	0.01\\
51.02	0.01\\
51.03	0.01\\
51.04	0.01\\
51.05	0.01\\
51.06	0.01\\
51.07	0.01\\
51.08	0.01\\
51.09	0.01\\
51.1	0.01\\
51.11	0.01\\
51.12	0.01\\
51.13	0.01\\
51.14	0.01\\
51.15	0.01\\
51.16	0.01\\
51.17	0.01\\
51.18	0.01\\
51.19	0.01\\
51.2	0.01\\
51.21	0.01\\
51.22	0.01\\
51.23	0.01\\
51.24	0.01\\
51.25	0.01\\
51.26	0.01\\
51.27	0.01\\
51.28	0.01\\
51.29	0.01\\
51.3	0.01\\
51.31	0.01\\
51.32	0.01\\
51.33	0.01\\
51.34	0.01\\
51.35	0.01\\
51.36	0.01\\
51.37	0.01\\
51.38	0.01\\
51.39	0.01\\
51.4	0.01\\
51.41	0.01\\
51.42	0.01\\
51.43	0.01\\
51.44	0.01\\
51.45	0.01\\
51.46	0.01\\
51.47	0.01\\
51.48	0.01\\
51.49	0.01\\
51.5	0.01\\
51.51	0.01\\
51.52	0.01\\
51.53	0.01\\
51.54	0.01\\
51.55	0.01\\
51.56	0.01\\
51.57	0.01\\
51.58	0.01\\
51.59	0.01\\
51.6	0.01\\
51.61	0.01\\
51.62	0.01\\
51.63	0.01\\
51.64	0.01\\
51.65	0.01\\
51.66	0.01\\
51.67	0.01\\
51.68	0.01\\
51.69	0.01\\
51.7	0.01\\
51.71	0.01\\
51.72	0.01\\
51.73	0.01\\
51.74	0.01\\
51.75	0.01\\
51.76	0.01\\
51.77	0.01\\
51.78	0.01\\
51.79	0.01\\
51.8	0.01\\
51.81	0.01\\
51.82	0.01\\
51.83	0.01\\
51.84	0.01\\
51.85	0.01\\
51.86	0.01\\
51.87	0.01\\
51.88	0.01\\
51.89	0.01\\
51.9	0.01\\
51.91	0.01\\
51.92	0.01\\
51.93	0.01\\
51.94	0.01\\
51.95	0.01\\
51.96	0.01\\
51.97	0.01\\
51.98	0.01\\
51.99	0.01\\
52	0.01\\
52.01	0.01\\
52.02	0.01\\
52.03	0.01\\
52.04	0.01\\
52.05	0.01\\
52.06	0.01\\
52.07	0.01\\
52.08	0.01\\
52.09	0.01\\
52.1	0.01\\
52.11	0.01\\
52.12	0.01\\
52.13	0.01\\
52.14	0.01\\
52.15	0.01\\
52.16	0.01\\
52.17	0.01\\
52.18	0.01\\
52.19	0.01\\
52.2	0.01\\
52.21	0.01\\
52.22	0.01\\
52.23	0.01\\
52.24	0.01\\
52.25	0.01\\
52.26	0.01\\
52.27	0.01\\
52.28	0.01\\
52.29	0.01\\
52.3	0.01\\
52.31	0.01\\
52.32	0.01\\
52.33	0.01\\
52.34	0.01\\
52.35	0.01\\
52.36	0.01\\
52.37	0.01\\
52.38	0.01\\
52.39	0.01\\
52.4	0.01\\
52.41	0.01\\
52.42	0.01\\
52.43	0.01\\
52.44	0.01\\
52.45	0.01\\
52.46	0.01\\
52.47	0.01\\
52.48	0.01\\
52.49	0.01\\
52.5	0.01\\
52.51	0.01\\
52.52	0.01\\
52.53	0.01\\
52.54	0.01\\
52.55	0.01\\
52.56	0.01\\
52.57	0.01\\
52.58	0.01\\
52.59	0.01\\
52.6	0.01\\
52.61	0.01\\
52.62	0.01\\
52.63	0.01\\
52.64	0.01\\
52.65	0.01\\
52.66	0.01\\
52.67	0.01\\
52.68	0.01\\
52.69	0.01\\
52.7	0.01\\
52.71	0.01\\
52.72	0.01\\
52.73	0.01\\
52.74	0.01\\
52.75	0.01\\
52.76	0.01\\
52.77	0.01\\
52.78	0.01\\
52.79	0.01\\
52.8	0.01\\
52.81	0.01\\
52.82	0.01\\
52.83	0.01\\
52.84	0.01\\
52.85	0.01\\
52.86	0.01\\
52.87	0.01\\
52.88	0.01\\
52.89	0.01\\
52.9	0.01\\
52.91	0.01\\
52.92	0.01\\
52.93	0.01\\
52.94	0.01\\
52.95	0.01\\
52.96	0.01\\
52.97	0.01\\
52.98	0.01\\
52.99	0.01\\
53	0.01\\
53.01	0.01\\
53.02	0.01\\
53.03	0.01\\
53.04	0.01\\
53.05	0.01\\
53.06	0.01\\
53.07	0.01\\
53.08	0.01\\
53.09	0.01\\
53.1	0.01\\
53.11	0.01\\
53.12	0.01\\
53.13	0.01\\
53.14	0.01\\
53.15	0.01\\
53.16	0.01\\
53.17	0.01\\
53.18	0.01\\
53.19	0.01\\
53.2	0.01\\
53.21	0.01\\
53.22	0.01\\
53.23	0.01\\
53.24	0.01\\
53.25	0.01\\
53.26	0.01\\
53.27	0.01\\
53.28	0.01\\
53.29	0.01\\
53.3	0.01\\
53.31	0.01\\
53.32	0.01\\
53.33	0.01\\
53.34	0.01\\
53.35	0.01\\
53.36	0.01\\
53.37	0.01\\
53.38	0.01\\
53.39	0.01\\
53.4	0.01\\
53.41	0.01\\
53.42	0.01\\
53.43	0.01\\
53.44	0.01\\
53.45	0.01\\
53.46	0.01\\
53.47	0.01\\
53.48	0.01\\
53.49	0.01\\
53.5	0.01\\
53.51	0.01\\
53.52	0.01\\
53.53	0.01\\
53.54	0.01\\
53.55	0.01\\
53.56	0.01\\
53.57	0.01\\
53.58	0.01\\
53.59	0.01\\
53.6	0.01\\
53.61	0.01\\
53.62	0.01\\
53.63	0.01\\
53.64	0.01\\
53.65	0.01\\
53.66	0.01\\
53.67	0.01\\
53.68	0.01\\
53.69	0.01\\
53.7	0.01\\
53.71	0.01\\
53.72	0.01\\
53.73	0.01\\
53.74	0.01\\
53.75	0.01\\
53.76	0.01\\
53.77	0.01\\
53.78	0.01\\
53.79	0.01\\
53.8	0.01\\
53.81	0.01\\
53.82	0.01\\
53.83	0.01\\
53.84	0.01\\
53.85	0.01\\
53.86	0.01\\
53.87	0.01\\
53.88	0.01\\
53.89	0.01\\
53.9	0.01\\
53.91	0.01\\
53.92	0.01\\
53.93	0.01\\
53.94	0.01\\
53.95	0.01\\
53.96	0.01\\
53.97	0.01\\
53.98	0.01\\
53.99	0.01\\
54	0.01\\
54.01	0.01\\
54.02	0.01\\
54.03	0.01\\
54.04	0.01\\
54.05	0.01\\
54.06	0.01\\
54.07	0.01\\
54.08	0.01\\
54.09	0.01\\
54.1	0.01\\
54.11	0.01\\
54.12	0.01\\
54.13	0.01\\
54.14	0.01\\
54.15	0.01\\
54.16	0.01\\
54.17	0.01\\
54.18	0.01\\
54.19	0.01\\
54.2	0.01\\
54.21	0.01\\
54.22	0.01\\
54.23	0.01\\
54.24	0.01\\
54.25	0.01\\
54.26	0.01\\
54.27	0.01\\
54.28	0.01\\
54.29	0.01\\
54.3	0.01\\
54.31	0.01\\
54.32	0.01\\
54.33	0.01\\
54.34	0.01\\
54.35	0.01\\
54.36	0.01\\
54.37	0.01\\
54.38	0.01\\
54.39	0.01\\
54.4	0.01\\
54.41	0.01\\
54.42	0.01\\
54.43	0.01\\
54.44	0.01\\
54.45	0.01\\
54.46	0.01\\
54.47	0.01\\
54.48	0.01\\
54.49	0.01\\
54.5	0.01\\
54.51	0.01\\
54.52	0.01\\
54.53	0.01\\
54.54	0.01\\
54.55	0.01\\
54.56	0.01\\
54.57	0.01\\
54.58	0.01\\
54.59	0.01\\
54.6	0.01\\
54.61	0.01\\
54.62	0.01\\
54.63	0.01\\
54.64	0.01\\
54.65	0.01\\
54.66	0.01\\
54.67	0.01\\
54.68	0.01\\
54.69	0.01\\
54.7	0.01\\
54.71	0.01\\
54.72	0.01\\
54.73	0.01\\
54.74	0.01\\
54.75	0.01\\
54.76	0.01\\
54.77	0.01\\
54.78	0.01\\
54.79	0.01\\
54.8	0.01\\
54.81	0.01\\
54.82	0.01\\
54.83	0.01\\
54.84	0.01\\
54.85	0.01\\
54.86	0.01\\
54.87	0.01\\
54.88	0.01\\
54.89	0.01\\
54.9	0.01\\
54.91	0.01\\
54.92	0.01\\
54.93	0.01\\
54.94	0.01\\
54.95	0.01\\
54.96	0.01\\
54.97	0.01\\
54.98	0.01\\
54.99	0.01\\
55	0.01\\
55.01	0.01\\
55.02	0.01\\
55.03	0.01\\
55.04	0.01\\
55.05	0.01\\
55.06	0.01\\
55.07	0.01\\
55.08	0.01\\
55.09	0.01\\
55.1	0.01\\
55.11	0.01\\
55.12	0.01\\
55.13	0.01\\
55.14	0.01\\
55.15	0.01\\
55.16	0.01\\
55.17	0.01\\
55.18	0.01\\
55.19	0.01\\
55.2	0.01\\
55.21	0.01\\
55.22	0.01\\
55.23	0.01\\
55.24	0.01\\
55.25	0.01\\
55.26	0.01\\
55.27	0.01\\
55.28	0.01\\
55.29	0.01\\
55.3	0.01\\
55.31	0.01\\
55.32	0.01\\
55.33	0.01\\
55.34	0.01\\
55.35	0.01\\
55.36	0.01\\
55.37	0.01\\
55.38	0.01\\
55.39	0.01\\
55.4	0.01\\
55.41	0.01\\
55.42	0.01\\
55.43	0.01\\
55.44	0.01\\
55.45	0.01\\
55.46	0.01\\
55.47	0.01\\
55.48	0.01\\
55.49	0.01\\
55.5	0.01\\
55.51	0.01\\
55.52	0.01\\
55.53	0.01\\
55.54	0.01\\
55.55	0.01\\
55.56	0.01\\
55.57	0.01\\
55.58	0.01\\
55.59	0.01\\
55.6	0.01\\
55.61	0.01\\
55.62	0.01\\
55.63	0.01\\
55.64	0.01\\
55.65	0.01\\
55.66	0.01\\
55.67	0.01\\
55.68	0.01\\
55.69	0.01\\
55.7	0.01\\
55.71	0.01\\
55.72	0.01\\
55.73	0.01\\
55.74	0.01\\
55.75	0.01\\
55.76	0.01\\
55.77	0.01\\
55.78	0.01\\
55.79	0.01\\
55.8	0.01\\
55.81	0.01\\
55.82	0.01\\
55.83	0.01\\
55.84	0.01\\
55.85	0.01\\
55.86	0.01\\
55.87	0.01\\
55.88	0.01\\
55.89	0.01\\
55.9	0.01\\
55.91	0.01\\
55.92	0.01\\
55.93	0.01\\
55.94	0.01\\
55.95	0.01\\
55.96	0.01\\
55.97	0.01\\
55.98	0.01\\
55.99	0.01\\
56	0.01\\
56.01	0.01\\
56.02	0.01\\
56.03	0.01\\
56.04	0.01\\
56.05	0.01\\
56.06	0.01\\
56.07	0.01\\
56.08	0.01\\
56.09	0.01\\
56.1	0.01\\
56.11	0.01\\
56.12	0.01\\
56.13	0.01\\
56.14	0.01\\
56.15	0.01\\
56.16	0.01\\
56.17	0.01\\
56.18	0.01\\
56.19	0.01\\
56.2	0.01\\
56.21	0.01\\
56.22	0.01\\
56.23	0.01\\
56.24	0.01\\
56.25	0.01\\
56.26	0.01\\
56.27	0.01\\
56.28	0.01\\
56.29	0.01\\
56.3	0.01\\
56.31	0.01\\
56.32	0.01\\
56.33	0.01\\
56.34	0.01\\
56.35	0.01\\
56.36	0.01\\
56.37	0.01\\
56.38	0.01\\
56.39	0.01\\
56.4	0.01\\
56.41	0.01\\
56.42	0.01\\
56.43	0.01\\
56.44	0.01\\
56.45	0.01\\
56.46	0.01\\
56.47	0.01\\
56.48	0.01\\
56.49	0.01\\
56.5	0.01\\
56.51	0.01\\
56.52	0.01\\
56.53	0.01\\
56.54	0.01\\
56.55	0.01\\
56.56	0.01\\
56.57	0.01\\
56.58	0.01\\
56.59	0.01\\
56.6	0.01\\
56.61	0.01\\
56.62	0.01\\
56.63	0.01\\
56.64	0.01\\
56.65	0.01\\
56.66	0.01\\
56.67	0.01\\
56.68	0.01\\
56.69	0.01\\
56.7	0.01\\
56.71	0.01\\
56.72	0.01\\
56.73	0.01\\
56.74	0.01\\
56.75	0.01\\
56.76	0.01\\
56.77	0.01\\
56.78	0.01\\
56.79	0.01\\
56.8	0.01\\
56.81	0.01\\
56.82	0.01\\
56.83	0.01\\
56.84	0.01\\
56.85	0.01\\
56.86	0.01\\
56.87	0.01\\
56.88	0.01\\
56.89	0.01\\
56.9	0.01\\
56.91	0.01\\
56.92	0.01\\
56.93	0.01\\
56.94	0.01\\
56.95	0.01\\
56.96	0.01\\
56.97	0.01\\
56.98	0.01\\
56.99	0.01\\
57	0.01\\
57.01	0.01\\
57.02	0.01\\
57.03	0.01\\
57.04	0.01\\
57.05	0.01\\
57.06	0.01\\
57.07	0.01\\
57.08	0.01\\
57.09	0.01\\
57.1	0.01\\
57.11	0.01\\
57.12	0.01\\
57.13	0.01\\
57.14	0.01\\
57.15	0.01\\
57.16	0.01\\
57.17	0.01\\
57.18	0.01\\
57.19	0.01\\
57.2	0.01\\
57.21	0.01\\
57.22	0.01\\
57.23	0.01\\
57.24	0.01\\
57.25	0.01\\
57.26	0.01\\
57.27	0.01\\
57.28	0.01\\
57.29	0.01\\
57.3	0.01\\
57.31	0.01\\
57.32	0.01\\
57.33	0.01\\
57.34	0.01\\
57.35	0.01\\
57.36	0.01\\
57.37	0.01\\
57.38	0.01\\
57.39	0.01\\
57.4	0.01\\
57.41	0.01\\
57.42	0.01\\
57.43	0.01\\
57.44	0.01\\
57.45	0.01\\
57.46	0.01\\
57.47	0.01\\
57.48	0.01\\
57.49	0.01\\
57.5	0.01\\
57.51	0.01\\
57.52	0.01\\
57.53	0.01\\
57.54	0.01\\
57.55	0.01\\
57.56	0.01\\
57.57	0.01\\
57.58	0.01\\
57.59	0.01\\
57.6	0.01\\
57.61	0.01\\
57.62	0.01\\
57.63	0.01\\
57.64	0.01\\
57.65	0.01\\
57.66	0.01\\
57.67	0.01\\
57.68	0.01\\
57.69	0.01\\
57.7	0.01\\
57.71	0.01\\
57.72	0.01\\
57.73	0.01\\
57.74	0.01\\
57.75	0.01\\
57.76	0.01\\
57.77	0.01\\
57.78	0.01\\
57.79	0.01\\
57.8	0.01\\
57.81	0.01\\
57.82	0.01\\
57.83	0.01\\
57.84	0.01\\
57.85	0.01\\
57.86	0.01\\
57.87	0.01\\
57.88	0.01\\
57.89	0.01\\
57.9	0.01\\
57.91	0.01\\
57.92	0.01\\
57.93	0.01\\
57.94	0.01\\
57.95	0.01\\
57.96	0.01\\
57.97	0.01\\
57.98	0.01\\
57.99	0.01\\
58	0.01\\
58.01	0.01\\
58.02	0.01\\
58.03	0.01\\
58.04	0.01\\
58.05	0.01\\
58.06	0.01\\
58.07	0.01\\
58.08	0.01\\
58.09	0.01\\
58.1	0.01\\
58.11	0.01\\
58.12	0.01\\
58.13	0.01\\
58.14	0.01\\
58.15	0.01\\
58.16	0.01\\
58.17	0.01\\
58.18	0.01\\
58.19	0.01\\
58.2	0.01\\
58.21	0.01\\
58.22	0.01\\
58.23	0.01\\
58.24	0.01\\
58.25	0.01\\
58.26	0.01\\
58.27	0.01\\
58.28	0.01\\
58.29	0.01\\
58.3	0.01\\
58.31	0.01\\
58.32	0.01\\
58.33	0.01\\
58.34	0.01\\
58.35	0.01\\
58.36	0.01\\
58.37	0.01\\
58.38	0.01\\
58.39	0.01\\
58.4	0.01\\
58.41	0.01\\
58.42	0.01\\
58.43	0.01\\
58.44	0.01\\
58.45	0.01\\
58.46	0.01\\
58.47	0.01\\
58.48	0.01\\
58.49	0.01\\
58.5	0.01\\
58.51	0.01\\
58.52	0.01\\
58.53	0.01\\
58.54	0.01\\
58.55	0.01\\
58.56	0.01\\
58.57	0.01\\
58.58	0.01\\
58.59	0.01\\
58.6	0.01\\
58.61	0.01\\
58.62	0.01\\
58.63	0.01\\
58.64	0.01\\
58.65	0.01\\
58.66	0.01\\
58.67	0.01\\
58.68	0.01\\
58.69	0.01\\
58.7	0.01\\
58.71	0.01\\
58.72	0.01\\
58.73	0.01\\
58.74	0.01\\
58.75	0.01\\
58.76	0.01\\
58.77	0.01\\
58.78	0.01\\
58.79	0.01\\
58.8	0.01\\
58.81	0.01\\
58.82	0.01\\
58.83	0.01\\
58.84	0.01\\
58.85	0.01\\
58.86	0.01\\
58.87	0.01\\
58.88	0.01\\
58.89	0.01\\
58.9	0.01\\
58.91	0.01\\
58.92	0.01\\
58.93	0.01\\
58.94	0.01\\
58.95	0.01\\
58.96	0.01\\
58.97	0.01\\
58.98	0.01\\
58.99	0.01\\
59	0.01\\
59.01	0.01\\
59.02	0.01\\
59.03	0.01\\
59.04	0.01\\
59.05	0.01\\
59.06	0.01\\
59.07	0.01\\
59.08	0.01\\
59.09	0.01\\
59.1	0.01\\
59.11	0.01\\
59.12	0.01\\
59.13	0.01\\
59.14	0.01\\
59.15	0.01\\
59.16	0.01\\
59.17	0.01\\
59.18	0.01\\
59.19	0.01\\
59.2	0.01\\
59.21	0.01\\
59.22	0.01\\
59.23	0.01\\
59.24	0.01\\
59.25	0.01\\
59.26	0.01\\
59.27	0.01\\
59.28	0.01\\
59.29	0.01\\
59.3	0.01\\
59.31	0.01\\
59.32	0.01\\
59.33	0.01\\
59.34	0.01\\
59.35	0.01\\
59.36	0.01\\
59.37	0.01\\
59.38	0.01\\
59.39	0.01\\
59.4	0.01\\
59.41	0.01\\
59.42	0.01\\
59.43	0.01\\
59.44	0.01\\
59.45	0.01\\
59.46	0.01\\
59.47	0.01\\
59.48	0.01\\
59.49	0.01\\
59.5	0.01\\
59.51	0.01\\
59.52	0.01\\
59.53	0.01\\
59.54	0.01\\
59.55	0.01\\
59.56	0.01\\
59.57	0.01\\
59.58	0.01\\
59.59	0.01\\
59.6	0.01\\
59.61	0.01\\
59.62	0.01\\
59.63	0.01\\
59.64	0.01\\
59.65	0.01\\
59.66	0.01\\
59.67	0.01\\
59.68	0.01\\
59.69	0.01\\
59.7	0.01\\
59.71	0.01\\
59.72	0.01\\
59.73	0.01\\
59.74	0.01\\
59.75	0.01\\
59.76	0.01\\
59.77	0.01\\
59.78	0.01\\
59.79	0.01\\
59.8	0.01\\
59.81	0.01\\
59.82	0.01\\
59.83	0.01\\
59.84	0.01\\
59.85	0.01\\
59.86	0.01\\
59.87	0.01\\
59.88	0.01\\
59.89	0.01\\
59.9	0.01\\
59.91	0.01\\
59.92	0.01\\
59.93	0.01\\
59.94	0.01\\
59.95	0.01\\
59.96	0.01\\
59.97	0.01\\
59.98	0.01\\
59.99	0.01\\
60	0.01\\
60.01	0.01\\
60.02	0.01\\
60.03	0.01\\
60.04	0.01\\
60.05	0.01\\
60.06	0.01\\
60.07	0.01\\
60.08	0.01\\
60.09	0.01\\
60.1	0.01\\
60.11	0.01\\
60.12	0.01\\
60.13	0.01\\
60.14	0.01\\
60.15	0.01\\
60.16	0.01\\
60.17	0.01\\
60.18	0.01\\
60.19	0.01\\
60.2	0.01\\
60.21	0.01\\
60.22	0.01\\
60.23	0.01\\
60.24	0.01\\
60.25	0.01\\
60.26	0.01\\
60.27	0.01\\
60.28	0.01\\
60.29	0.01\\
60.3	0.01\\
60.31	0.01\\
60.32	0.01\\
60.33	0.01\\
60.34	0.01\\
60.35	0.01\\
60.36	0.01\\
60.37	0.01\\
60.38	0.01\\
60.39	0.01\\
60.4	0.01\\
60.41	0.01\\
60.42	0.01\\
60.43	0.01\\
60.44	0.01\\
60.45	0.01\\
60.46	0.01\\
60.47	0.01\\
60.48	0.01\\
60.49	0.01\\
60.5	0.01\\
60.51	0.01\\
60.52	0.01\\
60.53	0.01\\
60.54	0.01\\
60.55	0.01\\
60.56	0.01\\
60.57	0.01\\
60.58	0.01\\
60.59	0.01\\
60.6	0.01\\
60.61	0.01\\
60.62	0.01\\
60.63	0.01\\
60.64	0.01\\
60.65	0.01\\
60.66	0.01\\
60.67	0.01\\
60.68	0.01\\
60.69	0.01\\
60.7	0.01\\
60.71	0.01\\
60.72	0.01\\
60.73	0.01\\
60.74	0.01\\
60.75	0.01\\
60.76	0.01\\
60.77	0.01\\
60.78	0.01\\
60.79	0.01\\
60.8	0.01\\
60.81	0.01\\
60.82	0.01\\
60.83	0.01\\
60.84	0.01\\
60.85	0.01\\
60.86	0.01\\
60.87	0.01\\
60.88	0.01\\
60.89	0.01\\
60.9	0.01\\
60.91	0.01\\
60.92	0.01\\
60.93	0.01\\
60.94	0.01\\
60.95	0.01\\
60.96	0.01\\
60.97	0.01\\
60.98	0.01\\
60.99	0.01\\
61	0.01\\
61.01	0.01\\
61.02	0.01\\
61.03	0.01\\
61.04	0.01\\
61.05	0.01\\
61.06	0.01\\
61.07	0.01\\
61.08	0.01\\
61.09	0.01\\
61.1	0.01\\
61.11	0.01\\
61.12	0.01\\
61.13	0.01\\
61.14	0.01\\
61.15	0.01\\
61.16	0.01\\
61.17	0.01\\
61.18	0.01\\
61.19	0.01\\
61.2	0.01\\
61.21	0.01\\
61.22	0.01\\
61.23	0.01\\
61.24	0.01\\
61.25	0.01\\
61.26	0.01\\
61.27	0.01\\
61.28	0.01\\
61.29	0.01\\
61.3	0.01\\
61.31	0.01\\
61.32	0.01\\
61.33	0.01\\
61.34	0.01\\
61.35	0.01\\
61.36	0.01\\
61.37	0.01\\
61.38	0.01\\
61.39	0.01\\
61.4	0.01\\
61.41	0.01\\
61.42	0.01\\
61.43	0.01\\
61.44	0.01\\
61.45	0.01\\
61.46	0.01\\
61.47	0.01\\
61.48	0.01\\
61.49	0.01\\
61.5	0.01\\
61.51	0.01\\
61.52	0.01\\
61.53	0.01\\
61.54	0.01\\
61.55	0.01\\
61.56	0.01\\
61.57	0.01\\
61.58	0.01\\
61.59	0.01\\
61.6	0.01\\
61.61	0.01\\
61.62	0.01\\
61.63	0.01\\
61.64	0.01\\
61.65	0.01\\
61.66	0.01\\
61.67	0.01\\
61.68	0.01\\
61.69	0.01\\
61.7	0.01\\
61.71	0.01\\
61.72	0.01\\
61.73	0.01\\
61.74	0.01\\
61.75	0.01\\
61.76	0.01\\
61.77	0.01\\
61.78	0.01\\
61.79	0.01\\
61.8	0.01\\
61.81	0.01\\
61.82	0.01\\
61.83	0.01\\
61.84	0.01\\
61.85	0.01\\
61.86	0.01\\
61.87	0.01\\
61.88	0.01\\
61.89	0.01\\
61.9	0.01\\
61.91	0.01\\
61.92	0.01\\
61.93	0.01\\
61.94	0.01\\
61.95	0.01\\
61.96	0.01\\
61.97	0.01\\
61.98	0.01\\
61.99	0.01\\
62	0.01\\
62.01	0.01\\
62.02	0.01\\
62.03	0.01\\
62.04	0.01\\
62.05	0.01\\
62.06	0.01\\
62.07	0.01\\
62.08	0.01\\
62.09	0.01\\
62.1	0.01\\
62.11	0.01\\
62.12	0.01\\
62.13	0.01\\
62.14	0.01\\
62.15	0.01\\
62.16	0.01\\
62.17	0.01\\
62.18	0.01\\
62.19	0.01\\
62.2	0.01\\
62.21	0.01\\
62.22	0.01\\
62.23	0.01\\
62.24	0.01\\
62.25	0.01\\
62.26	0.01\\
62.27	0.01\\
62.28	0.01\\
62.29	0.01\\
62.3	0.01\\
62.31	0.01\\
62.32	0.01\\
62.33	0.01\\
62.34	0.01\\
62.35	0.01\\
62.36	0.01\\
62.37	0.01\\
62.38	0.01\\
62.39	0.01\\
62.4	0.01\\
62.41	0.01\\
62.42	0.01\\
62.43	0.01\\
62.44	0.01\\
62.45	0.01\\
62.46	0.01\\
62.47	0.01\\
62.48	0.01\\
62.49	0.01\\
62.5	0.01\\
62.51	0.01\\
62.52	0.01\\
62.53	0.01\\
62.54	0.01\\
62.55	0.01\\
62.56	0.01\\
62.57	0.01\\
62.58	0.01\\
62.59	0.01\\
62.6	0.01\\
62.61	0.01\\
62.62	0.01\\
62.63	0.01\\
62.64	0.01\\
62.65	0.01\\
62.66	0.01\\
62.67	0.01\\
62.68	0.01\\
62.69	0.01\\
62.7	0.01\\
62.71	0.01\\
62.72	0.01\\
62.73	0.01\\
62.74	0.01\\
62.75	0.01\\
62.76	0.01\\
62.77	0.01\\
62.78	0.01\\
62.79	0.01\\
62.8	0.01\\
62.81	0.01\\
62.82	0.01\\
62.83	0.01\\
62.84	0.01\\
62.85	0.01\\
62.86	0.01\\
62.87	0.01\\
62.88	0.01\\
62.89	0.01\\
62.9	0.01\\
62.91	0.01\\
62.92	0.01\\
62.93	0.01\\
62.94	0.01\\
62.95	0.01\\
62.96	0.01\\
62.97	0.01\\
62.98	0.01\\
62.99	0.01\\
63	0.01\\
63.01	0.01\\
63.02	0.01\\
63.03	0.01\\
63.04	0.01\\
63.05	0.01\\
63.06	0.01\\
63.07	0.01\\
63.08	0.01\\
63.09	0.01\\
63.1	0.01\\
63.11	0.01\\
63.12	0.01\\
63.13	0.01\\
63.14	0.01\\
63.15	0.01\\
63.16	0.01\\
63.17	0.01\\
63.18	0.01\\
63.19	0.01\\
63.2	0.01\\
63.21	0.01\\
63.22	0.01\\
63.23	0.01\\
63.24	0.01\\
63.25	0.01\\
63.26	0.01\\
63.27	0.01\\
63.28	0.01\\
63.29	0.01\\
63.3	0.01\\
63.31	0.01\\
63.32	0.01\\
63.33	0.01\\
63.34	0.01\\
63.35	0.01\\
63.36	0.01\\
63.37	0.01\\
63.38	0.01\\
63.39	0.01\\
63.4	0.01\\
63.41	0.01\\
63.42	0.01\\
63.43	0.01\\
63.44	0.01\\
63.45	0.01\\
63.46	0.01\\
63.47	0.01\\
63.48	0.01\\
63.49	0.01\\
63.5	0.01\\
63.51	0.01\\
63.52	0.01\\
63.53	0.01\\
63.54	0.01\\
63.55	0.01\\
63.56	0.01\\
63.57	0.01\\
63.58	0.01\\
63.59	0.01\\
63.6	0.01\\
63.61	0.01\\
63.62	0.01\\
63.63	0.01\\
63.64	0.01\\
63.65	0.01\\
63.66	0.01\\
63.67	0.01\\
63.68	0.01\\
63.69	0.01\\
63.7	0.01\\
63.71	0.01\\
63.72	0.01\\
63.73	0.01\\
63.74	0.01\\
63.75	0.01\\
63.76	0.01\\
63.77	0.01\\
63.78	0.01\\
63.79	0.01\\
63.8	0.01\\
63.81	0.01\\
63.82	0.01\\
63.83	0.01\\
63.84	0.01\\
63.85	0.01\\
63.86	0.01\\
63.87	0.01\\
63.88	0.01\\
63.89	0.01\\
63.9	0.01\\
63.91	0.01\\
63.92	0.01\\
63.93	0.01\\
63.94	0.01\\
63.95	0.01\\
63.96	0.01\\
63.97	0.01\\
63.98	0.01\\
63.99	0.01\\
64	0.01\\
64.01	0.01\\
64.02	0.01\\
64.03	0.01\\
64.04	0.01\\
64.05	0.01\\
64.06	0.01\\
64.07	0.01\\
64.08	0.01\\
64.09	0.01\\
64.1	0.01\\
64.11	0.01\\
64.12	0.01\\
64.13	0.01\\
64.14	0.01\\
64.15	0.01\\
64.16	0.01\\
64.17	0.01\\
64.18	0.01\\
64.19	0.01\\
64.2	0.01\\
64.21	0.01\\
64.22	0.01\\
64.23	0.01\\
64.24	0.01\\
64.25	0.01\\
64.26	0.01\\
64.27	0.01\\
64.28	0.01\\
64.29	0.01\\
64.3	0.01\\
64.31	0.01\\
64.32	0.01\\
64.33	0.01\\
64.34	0.01\\
64.35	0.01\\
64.36	0.01\\
64.37	0.01\\
64.38	0.01\\
64.39	0.01\\
64.4	0.01\\
64.41	0.01\\
64.42	0.01\\
64.43	0.01\\
64.44	0.01\\
64.45	0.01\\
64.46	0.01\\
64.47	0.01\\
64.48	0.01\\
64.49	0.01\\
64.5	0.01\\
64.51	0.01\\
64.52	0.01\\
64.53	0.01\\
64.54	0.01\\
64.55	0.01\\
64.56	0.01\\
64.57	0.01\\
64.58	0.01\\
64.59	0.01\\
64.6	0.01\\
64.61	0.01\\
64.62	0.01\\
64.63	0.01\\
64.64	0.01\\
64.65	0.01\\
64.66	0.01\\
64.67	0.01\\
64.68	0.01\\
64.69	0.01\\
64.7	0.01\\
64.71	0.01\\
64.72	0.01\\
64.73	0.01\\
64.74	0.01\\
64.75	0.01\\
64.76	0.01\\
64.77	0.01\\
64.78	0.01\\
64.79	0.01\\
64.8	0.01\\
64.81	0.01\\
64.82	0.01\\
64.83	0.01\\
64.84	0.01\\
64.85	0.01\\
64.86	0.01\\
64.87	0.01\\
64.88	0.01\\
64.89	0.01\\
64.9	0.01\\
64.91	0.01\\
64.92	0.01\\
64.93	0.01\\
64.94	0.01\\
64.95	0.01\\
64.96	0.01\\
64.97	0.01\\
64.98	0.01\\
64.99	0.01\\
65	0.01\\
65.01	0.01\\
65.02	0.01\\
65.03	0.01\\
65.04	0.01\\
65.05	0.01\\
65.06	0.01\\
65.07	0.01\\
65.08	0.01\\
65.09	0.01\\
65.1	0.01\\
65.11	0.01\\
65.12	0.01\\
65.13	0.01\\
65.14	0.01\\
65.15	0.01\\
65.16	0.01\\
65.17	0.01\\
65.18	0.01\\
65.19	0.01\\
65.2	0.01\\
65.21	0.01\\
65.22	0.01\\
65.23	0.01\\
65.24	0.01\\
65.25	0.01\\
65.26	0.01\\
65.27	0.01\\
65.28	0.01\\
65.29	0.01\\
65.3	0.01\\
65.31	0.01\\
65.32	0.01\\
65.33	0.01\\
65.34	0.01\\
65.35	0.01\\
65.36	0.01\\
65.37	0.01\\
65.38	0.01\\
65.39	0.01\\
65.4	0.01\\
65.41	0.01\\
65.42	0.01\\
65.43	0.01\\
65.44	0.01\\
65.45	0.01\\
65.46	0.01\\
65.47	0.01\\
65.48	0.01\\
65.49	0.01\\
65.5	0.01\\
65.51	0.01\\
65.52	0.01\\
65.53	0.01\\
65.54	0.01\\
65.55	0.01\\
65.56	0.01\\
65.57	0.01\\
65.58	0.01\\
65.59	0.01\\
65.6	0.01\\
65.61	0.01\\
65.62	0.01\\
65.63	0.01\\
65.64	0.01\\
65.65	0.01\\
65.66	0.01\\
65.67	0.01\\
65.68	0.01\\
65.69	0.01\\
65.7	0.01\\
65.71	0.01\\
65.72	0.01\\
65.73	0.01\\
65.74	0.01\\
65.75	0.01\\
65.76	0.01\\
65.77	0.01\\
65.78	0.01\\
65.79	0.01\\
65.8	0.01\\
65.81	0.01\\
65.82	0.01\\
65.83	0.01\\
65.84	0.01\\
65.85	0.01\\
65.86	0.01\\
65.87	0.01\\
65.88	0.01\\
65.89	0.01\\
65.9	0.01\\
65.91	0.01\\
65.92	0.01\\
65.93	0.01\\
65.94	0.01\\
65.95	0.01\\
65.96	0.01\\
65.97	0.01\\
65.98	0.01\\
65.99	0.01\\
66	0.01\\
66.01	0.01\\
66.02	0.01\\
66.03	0.01\\
66.04	0.01\\
66.05	0.01\\
66.06	0.01\\
66.07	0.01\\
66.08	0.01\\
66.09	0.01\\
66.1	0.01\\
66.11	0.01\\
66.12	0.01\\
66.13	0.01\\
66.14	0.01\\
66.15	0.01\\
66.16	0.01\\
66.17	0.01\\
66.18	0.01\\
66.19	0.01\\
66.2	0.01\\
66.21	0.01\\
66.22	0.01\\
66.23	0.01\\
66.24	0.01\\
66.25	0.01\\
66.26	0.01\\
66.27	0.01\\
66.28	0.01\\
66.29	0.01\\
66.3	0.01\\
66.31	0.01\\
66.32	0.01\\
66.33	0.01\\
66.34	0.01\\
66.35	0.01\\
66.36	0.01\\
66.37	0.01\\
66.38	0.01\\
66.39	0.01\\
66.4	0.01\\
66.41	0.01\\
66.42	0.01\\
66.43	0.01\\
66.44	0.01\\
66.45	0.01\\
66.46	0.01\\
66.47	0.01\\
66.48	0.01\\
66.49	0.01\\
66.5	0.01\\
66.51	0.01\\
66.52	0.01\\
66.53	0.01\\
66.54	0.01\\
66.55	0.01\\
66.56	0.01\\
66.57	0.01\\
66.58	0.01\\
66.59	0.01\\
66.6	0.01\\
66.61	0.01\\
66.62	0.01\\
66.63	0.01\\
66.64	0.01\\
66.65	0.01\\
66.66	0.01\\
66.67	0.01\\
66.68	0.01\\
66.69	0.01\\
66.7	0.01\\
66.71	0.01\\
66.72	0.01\\
66.73	0.01\\
66.74	0.01\\
66.75	0.01\\
66.76	0.01\\
66.77	0.01\\
66.78	0.01\\
66.79	0.01\\
66.8	0.01\\
66.81	0.01\\
66.82	0.01\\
66.83	0.01\\
66.84	0.01\\
66.85	0.01\\
66.86	0.01\\
66.87	0.01\\
66.88	0.01\\
66.89	0.01\\
66.9	0.01\\
66.91	0.01\\
66.92	0.01\\
66.93	0.01\\
66.94	0.01\\
66.95	0.01\\
66.96	0.01\\
66.97	0.01\\
66.98	0.01\\
66.99	0.01\\
67	0.01\\
67.01	0.01\\
67.02	0.01\\
67.03	0.01\\
67.04	0.01\\
67.05	0.01\\
67.06	0.01\\
67.07	0.01\\
67.08	0.01\\
67.09	0.01\\
67.1	0.01\\
67.11	0.01\\
67.12	0.01\\
67.13	0.01\\
67.14	0.01\\
67.15	0.01\\
67.16	0.01\\
67.17	0.01\\
67.18	0.01\\
67.19	0.01\\
67.2	0.01\\
67.21	0.01\\
67.22	0.01\\
67.23	0.01\\
67.24	0.01\\
67.25	0.01\\
67.26	0.01\\
67.27	0.01\\
67.28	0.01\\
67.29	0.01\\
67.3	0.01\\
67.31	0.01\\
67.32	0.01\\
67.33	0.01\\
67.34	0.01\\
67.35	0.01\\
67.36	0.01\\
67.37	0.01\\
67.38	0.01\\
67.39	0.01\\
67.4	0.01\\
67.41	0.01\\
67.42	0.01\\
67.43	0.01\\
67.44	0.01\\
67.45	0.01\\
67.46	0.01\\
67.47	0.01\\
67.48	0.01\\
67.49	0.01\\
67.5	0.01\\
67.51	0.01\\
67.52	0.01\\
67.53	0.01\\
67.54	0.01\\
67.55	0.01\\
67.56	0.01\\
67.57	0.01\\
67.58	0.01\\
67.59	0.01\\
67.6	0.01\\
67.61	0.01\\
67.62	0.01\\
67.63	0.01\\
67.64	0.01\\
67.65	0.01\\
67.66	0.01\\
67.67	0.01\\
67.68	0.01\\
67.69	0.01\\
67.7	0.01\\
67.71	0.01\\
67.72	0.01\\
67.73	0.01\\
67.74	0.01\\
67.75	0.01\\
67.76	0.01\\
67.77	0.01\\
67.78	0.01\\
67.79	0.01\\
67.8	0.01\\
67.81	0.01\\
67.82	0.01\\
67.83	0.01\\
67.84	0.01\\
67.85	0.01\\
67.86	0.01\\
67.87	0.01\\
67.88	0.01\\
67.89	0.01\\
67.9	0.01\\
67.91	0.01\\
67.92	0.01\\
67.93	0.01\\
67.94	0.01\\
67.95	0.01\\
67.96	0.01\\
67.97	0.01\\
67.98	0.01\\
67.99	0.01\\
68	0.01\\
68.01	0.01\\
68.02	0.01\\
68.03	0.01\\
68.04	0.01\\
68.05	0.01\\
68.06	0.01\\
68.07	0.01\\
68.08	0.01\\
68.09	0.01\\
68.1	0.01\\
68.11	0.01\\
68.12	0.01\\
68.13	0.01\\
68.14	0.01\\
68.15	0.01\\
68.16	0.01\\
68.17	0.01\\
68.18	0.01\\
68.19	0.01\\
68.2	0.01\\
68.21	0.01\\
68.22	0.01\\
68.23	0.01\\
68.24	0.01\\
68.25	0.01\\
68.26	0.01\\
68.27	0.01\\
68.28	0.01\\
68.29	0.01\\
68.3	0.01\\
68.31	0.01\\
68.32	0.01\\
68.33	0.01\\
68.34	0.01\\
68.35	0.01\\
68.36	0.01\\
68.37	0.01\\
68.38	0.01\\
68.39	0.01\\
68.4	0.01\\
68.41	0.01\\
68.42	0.01\\
68.43	0.01\\
68.44	0.01\\
68.45	0.01\\
68.46	0.01\\
68.47	0.01\\
68.48	0.01\\
68.49	0.01\\
68.5	0.01\\
68.51	0.01\\
68.52	0.01\\
68.53	0.01\\
68.54	0.01\\
68.55	0.01\\
68.56	0.01\\
68.57	0.01\\
68.58	0.01\\
68.59	0.01\\
68.6	0.01\\
68.61	0.01\\
68.62	0.01\\
68.63	0.01\\
68.64	0.01\\
68.65	0.01\\
68.66	0.01\\
68.67	0.01\\
68.68	0.01\\
68.69	0.01\\
68.7	0.01\\
68.71	0.01\\
68.72	0.01\\
68.73	0.01\\
68.74	0.01\\
68.75	0.01\\
68.76	0.01\\
68.77	0.01\\
68.78	0.01\\
68.79	0.01\\
68.8	0.01\\
68.81	0.01\\
68.82	0.01\\
68.83	0.01\\
68.84	0.01\\
68.85	0.01\\
68.86	0.01\\
68.87	0.01\\
68.88	0.01\\
68.89	0.01\\
68.9	0.01\\
68.91	0.01\\
68.92	0.01\\
68.93	0.01\\
68.94	0.01\\
68.95	0.01\\
68.96	0.01\\
68.97	0.01\\
68.98	0.01\\
68.99	0.01\\
69	0.01\\
69.01	0.01\\
69.02	0.01\\
69.03	0.01\\
69.04	0.01\\
69.05	0.01\\
69.06	0.01\\
69.07	0.01\\
69.08	0.01\\
69.09	0.01\\
69.1	0.01\\
69.11	0.01\\
69.12	0.01\\
69.13	0.01\\
69.14	0.01\\
69.15	0.01\\
69.16	0.01\\
69.17	0.01\\
69.18	0.01\\
69.19	0.01\\
69.2	0.01\\
69.21	0.01\\
69.22	0.01\\
69.23	0.01\\
69.24	0.01\\
69.25	0.01\\
69.26	0.01\\
69.27	0.01\\
69.28	0.01\\
69.29	0.01\\
69.3	0.01\\
69.31	0.01\\
69.32	0.01\\
69.33	0.01\\
69.34	0.01\\
69.35	0.01\\
69.36	0.01\\
69.37	0.01\\
69.38	0.01\\
69.39	0.01\\
69.4	0.01\\
69.41	0.01\\
69.42	0.01\\
69.43	0.01\\
69.44	0.01\\
69.45	0.01\\
69.46	0.01\\
69.47	0.01\\
69.48	0.01\\
69.49	0.01\\
69.5	0.01\\
69.51	0.01\\
69.52	0.01\\
69.53	0.01\\
69.54	0.01\\
69.55	0.01\\
69.56	0.01\\
69.57	0.01\\
69.58	0.01\\
69.59	0.01\\
69.6	0.01\\
69.61	0.01\\
69.62	0.01\\
69.63	0.01\\
69.64	0.01\\
69.65	0.01\\
69.66	0.01\\
69.67	0.01\\
69.68	0.01\\
69.69	0.01\\
69.7	0.01\\
69.71	0.01\\
69.72	0.01\\
69.73	0.01\\
69.74	0.01\\
69.75	0.01\\
69.76	0.01\\
69.77	0.01\\
69.78	0.01\\
69.79	0.01\\
69.8	0.01\\
69.81	0.01\\
69.82	0.01\\
69.83	0.01\\
69.84	0.01\\
69.85	0.01\\
69.86	0.01\\
69.87	0.01\\
69.88	0.01\\
69.89	0.01\\
69.9	0.01\\
69.91	0.01\\
69.92	0.01\\
69.93	0.01\\
69.94	0.01\\
69.95	0.01\\
69.96	0.01\\
69.97	0.01\\
69.98	0.01\\
69.99	0.01\\
70	0.01\\
70.01	0.01\\
70.02	0.01\\
70.03	0.01\\
70.04	0.01\\
70.05	0.01\\
70.06	0.01\\
70.07	0.01\\
70.08	0.01\\
70.09	0.01\\
70.1	0.01\\
70.11	0.01\\
70.12	0.01\\
70.13	0.01\\
70.14	0.01\\
70.15	0.01\\
70.16	0.01\\
70.17	0.01\\
70.18	0.01\\
70.19	0.01\\
70.2	0.01\\
70.21	0.01\\
70.22	0.01\\
70.23	0.01\\
70.24	0.01\\
70.25	0.01\\
70.26	0.01\\
70.27	0.01\\
70.28	0.01\\
70.29	0.01\\
70.3	0.01\\
70.31	0.01\\
70.32	0.01\\
70.33	0.01\\
70.34	0.01\\
70.35	0.01\\
70.36	0.01\\
70.37	0.01\\
70.38	0.01\\
70.39	0.01\\
70.4	0.01\\
70.41	0.01\\
70.42	0.01\\
70.43	0.01\\
70.44	0.01\\
70.45	0.01\\
70.46	0.01\\
70.47	0.01\\
70.48	0.01\\
70.49	0.01\\
70.5	0.01\\
70.51	0.01\\
70.52	0.01\\
70.53	0.01\\
70.54	0.01\\
70.55	0.01\\
70.56	0.01\\
70.57	0.01\\
70.58	0.01\\
70.59	0.01\\
70.6	0.01\\
70.61	0.01\\
70.62	0.01\\
70.63	0.01\\
70.64	0.01\\
70.65	0.01\\
70.66	0.01\\
70.67	0.01\\
70.68	0.01\\
70.69	0.01\\
70.7	0.01\\
70.71	0.01\\
70.72	0.01\\
70.73	0.01\\
70.74	0.01\\
70.75	0.01\\
70.76	0.01\\
70.77	0.01\\
70.78	0.01\\
70.79	0.01\\
70.8	0.01\\
70.81	0.01\\
70.82	0.01\\
70.83	0.01\\
70.84	0.01\\
70.85	0.01\\
70.86	0.01\\
70.87	0.01\\
70.88	0.01\\
70.89	0.01\\
70.9	0.01\\
70.91	0.01\\
70.92	0.01\\
70.93	0.01\\
70.94	0.01\\
70.95	0.01\\
70.96	0.01\\
70.97	0.01\\
70.98	0.01\\
70.99	0.01\\
71	0.01\\
71.01	0.01\\
71.02	0.01\\
71.03	0.01\\
71.04	0.01\\
71.05	0.01\\
71.06	0.01\\
71.07	0.01\\
71.08	0.01\\
71.09	0.01\\
71.1	0.01\\
71.11	0.01\\
71.12	0.01\\
71.13	0.01\\
71.14	0.01\\
71.15	0.01\\
71.16	0.01\\
71.17	0.01\\
71.18	0.01\\
71.19	0.01\\
71.2	0.01\\
71.21	0.01\\
71.22	0.01\\
71.23	0.01\\
71.24	0.01\\
71.25	0.01\\
71.26	0.01\\
71.27	0.01\\
71.28	0.01\\
71.29	0.01\\
71.3	0.01\\
71.31	0.01\\
71.32	0.01\\
71.33	0.01\\
71.34	0.01\\
71.35	0.01\\
71.36	0.01\\
71.37	0.01\\
71.38	0.01\\
71.39	0.01\\
71.4	0.01\\
71.41	0.01\\
71.42	0.01\\
71.43	0.01\\
71.44	0.01\\
71.45	0.01\\
71.46	0.01\\
71.47	0.01\\
71.48	0.01\\
71.49	0.01\\
71.5	0.01\\
71.51	0.01\\
71.52	0.01\\
71.53	0.01\\
71.54	0.01\\
71.55	0.01\\
71.56	0.01\\
71.57	0.01\\
71.58	0.01\\
71.59	0.01\\
71.6	0.01\\
71.61	0.01\\
71.62	0.01\\
71.63	0.01\\
71.64	0.01\\
71.65	0.01\\
71.66	0.01\\
71.67	0.01\\
71.68	0.01\\
71.69	0.01\\
71.7	0.01\\
71.71	0.01\\
71.72	0.01\\
71.73	0.01\\
71.74	0.01\\
71.75	0.01\\
71.76	0.01\\
71.77	0.01\\
71.78	0.01\\
71.79	0.01\\
71.8	0.01\\
71.81	0.01\\
71.82	0.01\\
71.83	0.01\\
71.84	0.01\\
71.85	0.01\\
71.86	0.01\\
71.87	0.01\\
71.88	0.01\\
71.89	0.01\\
71.9	0.01\\
71.91	0.01\\
71.92	0.01\\
71.93	0.01\\
71.94	0.01\\
71.95	0.01\\
71.96	0.01\\
71.97	0.01\\
71.98	0.01\\
71.99	0.01\\
72	0.01\\
72.01	0.01\\
72.02	0.01\\
72.03	0.01\\
72.04	0.01\\
72.05	0.01\\
72.06	0.01\\
72.07	0.01\\
72.08	0.01\\
72.09	0.01\\
72.1	0.01\\
72.11	0.01\\
72.12	0.01\\
72.13	0.01\\
72.14	0.01\\
72.15	0.01\\
72.16	0.01\\
72.17	0.01\\
72.18	0.01\\
72.19	0.01\\
72.2	0.01\\
72.21	0.01\\
72.22	0.01\\
72.23	0.01\\
72.24	0.01\\
72.25	0.01\\
72.26	0.01\\
72.27	0.01\\
72.28	0.01\\
72.29	0.01\\
72.3	0.01\\
72.31	0.01\\
72.32	0.01\\
72.33	0.01\\
72.34	0.01\\
72.35	0.01\\
72.36	0.01\\
72.37	0.01\\
72.38	0.01\\
72.39	0.01\\
72.4	0.01\\
72.41	0.01\\
72.42	0.01\\
72.43	0.01\\
72.44	0.01\\
72.45	0.01\\
72.46	0.01\\
72.47	0.01\\
72.48	0.01\\
72.49	0.01\\
72.5	0.01\\
72.51	0.01\\
72.52	0.01\\
72.53	0.01\\
72.54	0.01\\
72.55	0.01\\
72.56	0.01\\
72.57	0.01\\
72.58	0.01\\
72.59	0.01\\
72.6	0.01\\
72.61	0.01\\
72.62	0.01\\
72.63	0.01\\
72.64	0.01\\
72.65	0.01\\
72.66	0.01\\
72.67	0.01\\
72.68	0.01\\
72.69	0.01\\
72.7	0.01\\
72.71	0.01\\
72.72	0.01\\
72.73	0.01\\
72.74	0.01\\
72.75	0.01\\
72.76	0.01\\
72.77	0.01\\
72.78	0.01\\
72.79	0.01\\
72.8	0.01\\
72.81	0.01\\
72.82	0.01\\
72.83	0.01\\
72.84	0.01\\
72.85	0.01\\
72.86	0.01\\
72.87	0.01\\
72.88	0.01\\
72.89	0.01\\
72.9	0.01\\
72.91	0.01\\
72.92	0.01\\
72.93	0.01\\
72.94	0.01\\
72.95	0.01\\
72.96	0.01\\
72.97	0.01\\
72.98	0.01\\
72.99	0.01\\
73	0.01\\
73.01	0.01\\
73.02	0.01\\
73.03	0.01\\
73.04	0.01\\
73.05	0.01\\
73.06	0.01\\
73.07	0.01\\
73.08	0.01\\
73.09	0.01\\
73.1	0.01\\
73.11	0.01\\
73.12	0.01\\
73.13	0.01\\
73.14	0.01\\
73.15	0.01\\
73.16	0.01\\
73.17	0.01\\
73.18	0.01\\
73.19	0.01\\
73.2	0.01\\
73.21	0.01\\
73.22	0.01\\
73.23	0.01\\
73.24	0.01\\
73.25	0.01\\
73.26	0.01\\
73.27	0.01\\
73.28	0.01\\
73.29	0.01\\
73.3	0.01\\
73.31	0.01\\
73.32	0.01\\
73.33	0.01\\
73.34	0.01\\
73.35	0.01\\
73.36	0.01\\
73.37	0.01\\
73.38	0.01\\
73.39	0.01\\
73.4	0.01\\
73.41	0.01\\
73.42	0.01\\
73.43	0.01\\
73.44	0.01\\
73.45	0.01\\
73.46	0.01\\
73.47	0.01\\
73.48	0.01\\
73.49	0.01\\
73.5	0.01\\
73.51	0.01\\
73.52	0.01\\
73.53	0.01\\
73.54	0.01\\
73.55	0.01\\
73.56	0.01\\
73.57	0.01\\
73.58	0.01\\
73.59	0.01\\
73.6	0.01\\
73.61	0.01\\
73.62	0.01\\
73.63	0.01\\
73.64	0.01\\
73.65	0.01\\
73.66	0.01\\
73.67	0.01\\
73.68	0.01\\
73.69	0.01\\
73.7	0.01\\
73.71	0.01\\
73.72	0.01\\
73.73	0.01\\
73.74	0.01\\
73.75	0.01\\
73.76	0.01\\
73.77	0.01\\
73.78	0.01\\
73.79	0.01\\
73.8	0.01\\
73.81	0.01\\
73.82	0.01\\
73.83	0.01\\
73.84	0.01\\
73.85	0.01\\
73.86	0.01\\
73.87	0.01\\
73.88	0.01\\
73.89	0.01\\
73.9	0.01\\
73.91	0.01\\
73.92	0.01\\
73.93	0.01\\
73.94	0.01\\
73.95	0.01\\
73.96	0.01\\
73.97	0.01\\
73.98	0.01\\
73.99	0.01\\
74	0.01\\
74.01	0.01\\
74.02	0.01\\
74.03	0.01\\
74.04	0.01\\
74.05	0.01\\
74.06	0.01\\
74.07	0.01\\
74.08	0.01\\
74.09	0.01\\
74.1	0.01\\
74.11	0.01\\
74.12	0.01\\
74.13	0.01\\
74.14	0.01\\
74.15	0.01\\
74.16	0.01\\
74.17	0.01\\
74.18	0.01\\
74.19	0.01\\
74.2	0.01\\
74.21	0.01\\
74.22	0.01\\
74.23	0.01\\
74.24	0.01\\
74.25	0.01\\
74.26	0.01\\
74.27	0.01\\
74.28	0.01\\
74.29	0.01\\
74.3	0.01\\
74.31	0.01\\
74.32	0.01\\
74.33	0.01\\
74.34	0.01\\
74.35	0.01\\
74.36	0.01\\
74.37	0.01\\
74.38	0.01\\
74.39	0.01\\
74.4	0.01\\
74.41	0.01\\
74.42	0.01\\
74.43	0.01\\
74.44	0.01\\
74.45	0.01\\
74.46	0.01\\
74.47	0.01\\
74.48	0.01\\
74.49	0.01\\
74.5	0.01\\
74.51	0.01\\
74.52	0.01\\
74.53	0.01\\
74.54	0.01\\
74.55	0.01\\
74.56	0.01\\
74.57	0.01\\
74.58	0.01\\
74.59	0.01\\
74.6	0.01\\
74.61	0.01\\
74.62	0.01\\
74.63	0.01\\
74.64	0.01\\
74.65	0.01\\
74.66	0.01\\
74.67	0.01\\
74.68	0.01\\
74.69	0.01\\
74.7	0.01\\
74.71	0.01\\
74.72	0.01\\
74.73	0.01\\
74.74	0.01\\
74.75	0.01\\
74.76	0.01\\
74.77	0.01\\
74.78	0.01\\
74.79	0.01\\
74.8	0.01\\
74.81	0.01\\
74.82	0.01\\
74.83	0.01\\
74.84	0.01\\
74.85	0.01\\
74.86	0.01\\
74.87	0.01\\
74.88	0.01\\
74.89	0.01\\
74.9	0.01\\
74.91	0.01\\
74.92	0.01\\
74.93	0.01\\
74.94	0.01\\
74.95	0.01\\
74.96	0.01\\
74.97	0.01\\
74.98	0.01\\
74.99	0.01\\
75	0.01\\
75.01	0.01\\
75.02	0.01\\
75.03	0.01\\
75.04	0.01\\
75.05	0.01\\
75.06	0.01\\
75.07	0.01\\
75.08	0.01\\
75.09	0.01\\
75.1	0.01\\
75.11	0.01\\
75.12	0.01\\
75.13	0.01\\
75.14	0.01\\
75.15	0.01\\
75.16	0.01\\
75.17	0.01\\
75.18	0.01\\
75.19	0.01\\
75.2	0.01\\
75.21	0.01\\
75.22	0.01\\
75.23	0.01\\
75.24	0.01\\
75.25	0.01\\
75.26	0.01\\
75.27	0.01\\
75.28	0.01\\
75.29	0.01\\
75.3	0.01\\
75.31	0.01\\
75.32	0.01\\
75.33	0.01\\
75.34	0.01\\
75.35	0.01\\
75.36	0.01\\
75.37	0.01\\
75.38	0.01\\
75.39	0.01\\
75.4	0.01\\
75.41	0.01\\
75.42	0.01\\
75.43	0.01\\
75.44	0.01\\
75.45	0.01\\
75.46	0.01\\
75.47	0.01\\
75.48	0.01\\
75.49	0.01\\
75.5	0.01\\
75.51	0.01\\
75.52	0.01\\
75.53	0.01\\
75.54	0.01\\
75.55	0.01\\
75.56	0.01\\
75.57	0.01\\
75.58	0.01\\
75.59	0.01\\
75.6	0.01\\
75.61	0.01\\
75.62	0.01\\
75.63	0.01\\
75.64	0.01\\
75.65	0.01\\
75.66	0.01\\
75.67	0.01\\
75.68	0.01\\
75.69	0.01\\
75.7	0.01\\
75.71	0.01\\
75.72	0.01\\
75.73	0.01\\
75.74	0.01\\
75.75	0.01\\
75.76	0.01\\
75.77	0.01\\
75.78	0.01\\
75.79	0.01\\
75.8	0.01\\
75.81	0.01\\
75.82	0.01\\
75.83	0.01\\
75.84	0.01\\
75.85	0.01\\
75.86	0.01\\
75.87	0.01\\
75.88	0.01\\
75.89	0.01\\
75.9	0.01\\
75.91	0.01\\
75.92	0.01\\
75.93	0.01\\
75.94	0.01\\
75.95	0.01\\
75.96	0.01\\
75.97	0.01\\
75.98	0.01\\
75.99	0.01\\
76	0.01\\
76.01	0.01\\
76.02	0.01\\
76.03	0.01\\
76.04	0.01\\
76.05	0.01\\
76.06	0.01\\
76.07	0.01\\
76.08	0.01\\
76.09	0.01\\
76.1	0.01\\
76.11	0.01\\
76.12	0.01\\
76.13	0.01\\
76.14	0.01\\
76.15	0.01\\
76.16	0.01\\
76.17	0.01\\
76.18	0.01\\
76.19	0.01\\
76.2	0.01\\
76.21	0.01\\
76.22	0.01\\
76.23	0.01\\
76.24	0.01\\
76.25	0.01\\
76.26	0.01\\
76.27	0.01\\
76.28	0.01\\
76.29	0.01\\
76.3	0.01\\
76.31	0.01\\
76.32	0.01\\
76.33	0.01\\
76.34	0.01\\
76.35	0.01\\
76.36	0.01\\
76.37	0.01\\
76.38	0.01\\
76.39	0.01\\
76.4	0.01\\
76.41	0.01\\
76.42	0.01\\
76.43	0.01\\
76.44	0.01\\
76.45	0.01\\
76.46	0.01\\
76.47	0.01\\
76.48	0.01\\
76.49	0.01\\
76.5	0.01\\
76.51	0.01\\
76.52	0.01\\
76.53	0.01\\
76.54	0.01\\
76.55	0.01\\
76.56	0.01\\
76.57	0.01\\
76.58	0.01\\
76.59	0.01\\
76.6	0.01\\
76.61	0.01\\
76.62	0.01\\
76.63	0.01\\
76.64	0.01\\
76.65	0.01\\
76.66	0.01\\
76.67	0.01\\
76.68	0.01\\
76.69	0.01\\
76.7	0.01\\
76.71	0.01\\
76.72	0.01\\
76.73	0.01\\
76.74	0.01\\
76.75	0.01\\
76.76	0.01\\
76.77	0.01\\
76.78	0.01\\
76.79	0.01\\
76.8	0.01\\
76.81	0.01\\
76.82	0.01\\
76.83	0.01\\
76.84	0.01\\
76.85	0.01\\
76.86	0.01\\
76.87	0.01\\
76.88	0.01\\
76.89	0.01\\
76.9	0.01\\
76.91	0.01\\
76.92	0.01\\
76.93	0.01\\
76.94	0.01\\
76.95	0.01\\
76.96	0.01\\
76.97	0.01\\
76.98	0.01\\
76.99	0.01\\
77	0.01\\
77.01	0.01\\
77.02	0.01\\
77.03	0.01\\
77.04	0.01\\
77.05	0.01\\
77.06	0.01\\
77.07	0.01\\
77.08	0.01\\
77.09	0.01\\
77.1	0.01\\
77.11	0.01\\
77.12	0.01\\
77.13	0.01\\
77.14	0.01\\
77.15	0.01\\
77.16	0.01\\
77.17	0.01\\
77.18	0.01\\
77.19	0.01\\
77.2	0.01\\
77.21	0.01\\
77.22	0.01\\
77.23	0.01\\
77.24	0.01\\
77.25	0.01\\
77.26	0.01\\
77.27	0.01\\
77.28	0.01\\
77.29	0.01\\
77.3	0.01\\
77.31	0.01\\
77.32	0.01\\
77.33	0.01\\
77.34	0.01\\
77.35	0.01\\
77.36	0.01\\
77.37	0.01\\
77.38	0.01\\
77.39	0.01\\
77.4	0.01\\
77.41	0.01\\
77.42	0.01\\
77.43	0.01\\
77.44	0.01\\
77.45	0.01\\
77.46	0.01\\
77.47	0.01\\
77.48	0.01\\
77.49	0.01\\
77.5	0.01\\
77.51	0.01\\
77.52	0.01\\
77.53	0.01\\
77.54	0.01\\
77.55	0.01\\
77.56	0.01\\
77.57	0.01\\
77.58	0.01\\
77.59	0.01\\
77.6	0.01\\
77.61	0.01\\
77.62	0.01\\
77.63	0.01\\
77.64	0.01\\
77.65	0.01\\
77.66	0.01\\
77.67	0.01\\
77.68	0.01\\
77.69	0.01\\
77.7	0.01\\
77.71	0.01\\
77.72	0.01\\
77.73	0.01\\
77.74	0.01\\
77.75	0.01\\
77.76	0.01\\
77.77	0.01\\
77.78	0.01\\
77.79	0.01\\
77.8	0.01\\
77.81	0.01\\
77.82	0.01\\
77.83	0.01\\
77.84	0.01\\
77.85	0.01\\
77.86	0.01\\
77.87	0.01\\
77.88	0.01\\
77.89	0.01\\
77.9	0.01\\
77.91	0.01\\
77.92	0.01\\
77.93	0.01\\
77.94	0.01\\
77.95	0.01\\
77.96	0.01\\
77.97	0.01\\
77.98	0.01\\
77.99	0.01\\
78	0.01\\
78.01	0.01\\
78.02	0.01\\
78.03	0.01\\
78.04	0.01\\
78.05	0.01\\
78.06	0.01\\
78.07	0.01\\
78.08	0.01\\
78.09	0.01\\
78.1	0.01\\
78.11	0.01\\
78.12	0.01\\
78.13	0.01\\
78.14	0.01\\
78.15	0.01\\
78.16	0.01\\
78.17	0.01\\
78.18	0.01\\
78.19	0.01\\
78.2	0.01\\
78.21	0.01\\
78.22	0.01\\
78.23	0.01\\
78.24	0.01\\
78.25	0.01\\
78.26	0.01\\
78.27	0.01\\
78.28	0.01\\
78.29	0.01\\
78.3	0.01\\
78.31	0.01\\
78.32	0.01\\
78.33	0.01\\
78.34	0.01\\
78.35	0.01\\
78.36	0.01\\
78.37	0.01\\
78.38	0.01\\
78.39	0.01\\
78.4	0.01\\
78.41	0.01\\
78.42	0.01\\
78.43	0.01\\
78.44	0.01\\
78.45	0.01\\
78.46	0.01\\
78.47	0.01\\
78.48	0.01\\
78.49	0.01\\
78.5	0.01\\
78.51	0.01\\
78.52	0.01\\
78.53	0.01\\
78.54	0.01\\
78.55	0.01\\
78.56	0.01\\
78.57	0.01\\
78.58	0.01\\
78.59	0.01\\
78.6	0.01\\
78.61	0.01\\
78.62	0.01\\
78.63	0.01\\
78.64	0.01\\
78.65	0.01\\
78.66	0.01\\
78.67	0.01\\
78.68	0.01\\
78.69	0.01\\
78.7	0.01\\
78.71	0.01\\
78.72	0.01\\
78.73	0.01\\
78.74	0.01\\
78.75	0.01\\
78.76	0.01\\
78.77	0.01\\
78.78	0.01\\
78.79	0.01\\
78.8	0.01\\
78.81	0.01\\
78.82	0.01\\
78.83	0.01\\
78.84	0.01\\
78.85	0.01\\
78.86	0.01\\
78.87	0.01\\
78.88	0.01\\
78.89	0.01\\
78.9	0.01\\
78.91	0.01\\
78.92	0.01\\
78.93	0.01\\
78.94	0.01\\
78.95	0.01\\
78.96	0.01\\
78.97	0.01\\
78.98	0.01\\
78.99	0.01\\
79	0.01\\
79.01	0.01\\
79.02	0.01\\
79.03	0.01\\
79.04	0.01\\
79.05	0.01\\
79.06	0.01\\
79.07	0.01\\
79.08	0.01\\
79.09	0.01\\
79.1	0.01\\
79.11	0.01\\
79.12	0.01\\
79.13	0.01\\
79.14	0.01\\
79.15	0.01\\
79.16	0.01\\
79.17	0.01\\
79.18	0.01\\
79.19	0.01\\
79.2	0.01\\
79.21	0.01\\
79.22	0.01\\
79.23	0.01\\
79.24	0.01\\
79.25	0.01\\
79.26	0.01\\
79.27	0.01\\
79.28	0.01\\
79.29	0.01\\
79.3	0.01\\
79.31	0.01\\
79.32	0.01\\
79.33	0.01\\
79.34	0.01\\
79.35	0.01\\
79.36	0.01\\
79.37	0.01\\
79.38	0.01\\
79.39	0.01\\
79.4	0.01\\
79.41	0.01\\
79.42	0.01\\
79.43	0.01\\
79.44	0.01\\
79.45	0.01\\
79.46	0.01\\
79.47	0.01\\
79.48	0.01\\
79.49	0.01\\
79.5	0.01\\
79.51	0.01\\
79.52	0.01\\
79.53	0.01\\
79.54	0.01\\
79.55	0.01\\
79.56	0.01\\
79.57	0.01\\
79.58	0.01\\
79.59	0.01\\
79.6	0.01\\
79.61	0.01\\
79.62	0.01\\
79.63	0.01\\
79.64	0.01\\
79.65	0.01\\
79.66	0.01\\
79.67	0.01\\
79.68	0.01\\
79.69	0.01\\
79.7	0.01\\
79.71	0.01\\
79.72	0.01\\
79.73	0.01\\
79.74	0.01\\
79.75	0.01\\
79.76	0.01\\
79.77	0.01\\
79.78	0.01\\
79.79	0.01\\
79.8	0.01\\
79.81	0.01\\
79.82	0.01\\
79.83	0.01\\
79.84	0.01\\
79.85	0.01\\
79.86	0.01\\
79.87	0.01\\
79.88	0.01\\
79.89	0.01\\
79.9	0.01\\
79.91	0.01\\
79.92	0.01\\
79.93	0.01\\
79.94	0.01\\
79.95	0.01\\
79.96	0.01\\
79.97	0.01\\
79.98	0.01\\
79.99	0.01\\
80	0.01\\
80.01	0.01\\
};
\addplot [color=green,dashed]
  table[row sep=crcr]{%
80.01	0.01\\
80.02	0.01\\
80.03	0.01\\
80.04	0.01\\
80.05	0.01\\
80.06	0.01\\
80.07	0.01\\
80.08	0.01\\
80.09	0.01\\
80.1	0.01\\
80.11	0.01\\
80.12	0.01\\
80.13	0.01\\
80.14	0.01\\
80.15	0.01\\
80.16	0.01\\
80.17	0.01\\
80.18	0.01\\
80.19	0.01\\
80.2	0.01\\
80.21	0.01\\
80.22	0.01\\
80.23	0.01\\
80.24	0.01\\
80.25	0.01\\
80.26	0.01\\
80.27	0.01\\
80.28	0.01\\
80.29	0.01\\
80.3	0.01\\
80.31	0.01\\
80.32	0.01\\
80.33	0.01\\
80.34	0.01\\
80.35	0.01\\
80.36	0.01\\
80.37	0.01\\
80.38	0.01\\
80.39	0.01\\
80.4	0.01\\
80.41	0.01\\
80.42	0.01\\
80.43	0.01\\
80.44	0.01\\
80.45	0.01\\
80.46	0.01\\
80.47	0.01\\
80.48	0.01\\
80.49	0.01\\
80.5	0.01\\
80.51	0.01\\
80.52	0.01\\
80.53	0.01\\
80.54	0.01\\
80.55	0.01\\
80.56	0.01\\
80.57	0.01\\
80.58	0.01\\
80.59	0.01\\
80.6	0.01\\
80.61	0.01\\
80.62	0.01\\
80.63	0.01\\
80.64	0.01\\
80.65	0.01\\
80.66	0.01\\
80.67	0.01\\
80.68	0.01\\
80.69	0.01\\
80.7	0.01\\
80.71	0.01\\
80.72	0.01\\
80.73	0.01\\
80.74	0.01\\
80.75	0.01\\
80.76	0.01\\
80.77	0.01\\
80.78	0.01\\
80.79	0.01\\
80.8	0.01\\
80.81	0.01\\
80.82	0.01\\
80.83	0.01\\
80.84	0.01\\
80.85	0.01\\
80.86	0.01\\
80.87	0.01\\
80.88	0.01\\
80.89	0.01\\
80.9	0.01\\
80.91	0.01\\
80.92	0.01\\
80.93	0.01\\
80.94	0.01\\
80.95	0.01\\
80.96	0.01\\
80.97	0.01\\
80.98	0.01\\
80.99	0.01\\
81	0.01\\
81.01	0.01\\
81.02	0.01\\
81.03	0.01\\
81.04	0.01\\
81.05	0.01\\
81.06	0.01\\
81.07	0.01\\
81.08	0.01\\
81.09	0.01\\
81.1	0.01\\
81.11	0.01\\
81.12	0.01\\
81.13	0.01\\
81.14	0.01\\
81.15	0.01\\
81.16	0.01\\
81.17	0.01\\
81.18	0.01\\
81.19	0.01\\
81.2	0.01\\
81.21	0.01\\
81.22	0.01\\
81.23	0.01\\
81.24	0.01\\
81.25	0.01\\
81.26	0.01\\
81.27	0.01\\
81.28	0.01\\
81.29	0.01\\
81.3	0.01\\
81.31	0.01\\
81.32	0.01\\
81.33	0.01\\
81.34	0.01\\
81.35	0.01\\
81.36	0.01\\
81.37	0.01\\
81.38	0.01\\
81.39	0.01\\
81.4	0.01\\
81.41	0.01\\
81.42	0.01\\
81.43	0.01\\
81.44	0.01\\
81.45	0.01\\
81.46	0.01\\
81.47	0.01\\
81.48	0.01\\
81.49	0.01\\
81.5	0.01\\
81.51	0.01\\
81.52	0.01\\
81.53	0.01\\
81.54	0.01\\
81.55	0.01\\
81.56	0.01\\
81.57	0.01\\
81.58	0.01\\
81.59	0.01\\
81.6	0.01\\
81.61	0.01\\
81.62	0.01\\
81.63	0.01\\
81.64	0.01\\
81.65	0.01\\
81.66	0.01\\
81.67	0.01\\
81.68	0.01\\
81.69	0.01\\
81.7	0.01\\
81.71	0.01\\
81.72	0.01\\
81.73	0.01\\
81.74	0.01\\
81.75	0.01\\
81.76	0.01\\
81.77	0.01\\
81.78	0.01\\
81.79	0.01\\
81.8	0.01\\
81.81	0.01\\
81.82	0.01\\
81.83	0.01\\
81.84	0.01\\
81.85	0.01\\
81.86	0.01\\
81.87	0.01\\
81.88	0.01\\
81.89	0.01\\
81.9	0.01\\
81.91	0.01\\
81.92	0.01\\
81.93	0.01\\
81.94	0.01\\
81.95	0.01\\
81.96	0.01\\
81.97	0.01\\
81.98	0.01\\
81.99	0.01\\
82	0.01\\
82.01	0.01\\
82.02	0.01\\
82.03	0.01\\
82.04	0.01\\
82.05	0.01\\
82.06	0.01\\
82.07	0.01\\
82.08	0.01\\
82.09	0.01\\
82.1	0.01\\
82.11	0.01\\
82.12	0.01\\
82.13	0.01\\
82.14	0.01\\
82.15	0.01\\
82.16	0.01\\
82.17	0.01\\
82.18	0.01\\
82.19	0.01\\
82.2	0.01\\
82.21	0.01\\
82.22	0.01\\
82.23	0.01\\
82.24	0.01\\
82.25	0.01\\
82.26	0.01\\
82.27	0.01\\
82.28	0.01\\
82.29	0.01\\
82.3	0.01\\
82.31	0.01\\
82.32	0.01\\
82.33	0.01\\
82.34	0.01\\
82.35	0.01\\
82.36	0.01\\
82.37	0.01\\
82.38	0.01\\
82.39	0.01\\
82.4	0.01\\
82.41	0.01\\
82.42	0.01\\
82.43	0.01\\
82.44	0.01\\
82.45	0.01\\
82.46	0.01\\
82.47	0.01\\
82.48	0.01\\
82.49	0.01\\
82.5	0.01\\
82.51	0.01\\
82.52	0.01\\
82.53	0.01\\
82.54	0.01\\
82.55	0.01\\
82.56	0.01\\
82.57	0.01\\
82.58	0.01\\
82.59	0.01\\
82.6	0.01\\
82.61	0.01\\
82.62	0.01\\
82.63	0.01\\
82.64	0.01\\
82.65	0.01\\
82.66	0.01\\
82.67	0.01\\
82.68	0.01\\
82.69	0.01\\
82.7	0.01\\
82.71	0.01\\
82.72	0.01\\
82.73	0.01\\
82.74	0.01\\
82.75	0.01\\
82.76	0.01\\
82.77	0.01\\
82.78	0.01\\
82.79	0.01\\
82.8	0.01\\
82.81	0.01\\
82.82	0.01\\
82.83	0.01\\
82.84	0.01\\
82.85	0.01\\
82.86	0.01\\
82.87	0.01\\
82.88	0.01\\
82.89	0.01\\
82.9	0.01\\
82.91	0.01\\
82.92	0.01\\
82.93	0.01\\
82.94	0.01\\
82.95	0.01\\
82.96	0.01\\
82.97	0.01\\
82.98	0.01\\
82.99	0.01\\
83	0.01\\
83.01	0.01\\
83.02	0.01\\
83.03	0.01\\
83.04	0.01\\
83.05	0.01\\
83.06	0.01\\
83.07	0.01\\
83.08	0.01\\
83.09	0.01\\
83.1	0.01\\
83.11	0.01\\
83.12	0.01\\
83.13	0.01\\
83.14	0.01\\
83.15	0.01\\
83.16	0.01\\
83.17	0.01\\
83.18	0.01\\
83.19	0.01\\
83.2	0.01\\
83.21	0.01\\
83.22	0.01\\
83.23	0.01\\
83.24	0.01\\
83.25	0.01\\
83.26	0.01\\
83.27	0.01\\
83.28	0.01\\
83.29	0.01\\
83.3	0.01\\
83.31	0.01\\
83.32	0.01\\
83.33	0.01\\
83.34	0.01\\
83.35	0.01\\
83.36	0.01\\
83.37	0.01\\
83.38	0.01\\
83.39	0.01\\
83.4	0.01\\
83.41	0.01\\
83.42	0.01\\
83.43	0.01\\
83.44	0.01\\
83.45	0.01\\
83.46	0.01\\
83.47	0.01\\
83.48	0.01\\
83.49	0.01\\
83.5	0.01\\
83.51	0.01\\
83.52	0.01\\
83.53	0.01\\
83.54	0.01\\
83.55	0.01\\
83.56	0.01\\
83.57	0.01\\
83.58	0.01\\
83.59	0.01\\
83.6	0.01\\
83.61	0.01\\
83.62	0.01\\
83.63	0.01\\
83.64	0.01\\
83.65	0.01\\
83.66	0.01\\
83.67	0.01\\
83.68	0.01\\
83.69	0.01\\
83.7	0.01\\
83.71	0.01\\
83.72	0.01\\
83.73	0.01\\
83.74	0.01\\
83.75	0.01\\
83.76	0.01\\
83.77	0.01\\
83.78	0.01\\
83.79	0.01\\
83.8	0.01\\
83.81	0.01\\
83.82	0.01\\
83.83	0.01\\
83.84	0.01\\
83.85	0.01\\
83.86	0.01\\
83.87	0.01\\
83.88	0.01\\
83.89	0.01\\
83.9	0.01\\
83.91	0.01\\
83.92	0.01\\
83.93	0.01\\
83.94	0.01\\
83.95	0.01\\
83.96	0.01\\
83.97	0.01\\
83.98	0.01\\
83.99	0.01\\
84	0.01\\
84.01	0.01\\
84.02	0.01\\
84.03	0.01\\
84.04	0.01\\
84.05	0.01\\
84.06	0.01\\
84.07	0.01\\
84.08	0.01\\
84.09	0.01\\
84.1	0.01\\
84.11	0.01\\
84.12	0.01\\
84.13	0.01\\
84.14	0.01\\
84.15	0.01\\
84.16	0.01\\
84.17	0.01\\
84.18	0.01\\
84.19	0.01\\
84.2	0.01\\
84.21	0.01\\
84.22	0.01\\
84.23	0.01\\
84.24	0.01\\
84.25	0.01\\
84.26	0.01\\
84.27	0.01\\
84.28	0.01\\
84.29	0.01\\
84.3	0.01\\
84.31	0.01\\
84.32	0.01\\
84.33	0.01\\
84.34	0.01\\
84.35	0.01\\
84.36	0.01\\
84.37	0.01\\
84.38	0.01\\
84.39	0.01\\
84.4	0.01\\
84.41	0.01\\
84.42	0.01\\
84.43	0.01\\
84.44	0.01\\
84.45	0.01\\
84.46	0.01\\
84.47	0.01\\
84.48	0.01\\
84.49	0.01\\
84.5	0.01\\
84.51	0.01\\
84.52	0.01\\
84.53	0.01\\
84.54	0.01\\
84.55	0.01\\
84.56	0.01\\
84.57	0.01\\
84.58	0.01\\
84.59	0.01\\
84.6	0.01\\
84.61	0.01\\
84.62	0.01\\
84.63	0.01\\
84.64	0.01\\
84.65	0.01\\
84.66	0.01\\
84.67	0.01\\
84.68	0.01\\
84.69	0.01\\
84.7	0.01\\
84.71	0.01\\
84.72	0.01\\
84.73	0.01\\
84.74	0.01\\
84.75	0.01\\
84.76	0.01\\
84.77	0.01\\
84.78	0.01\\
84.79	0.01\\
84.8	0.01\\
84.81	0.01\\
84.82	0.01\\
84.83	0.01\\
84.84	0.01\\
84.85	0.01\\
84.86	0.01\\
84.87	0.01\\
84.88	0.01\\
84.89	0.01\\
84.9	0.01\\
84.91	0.01\\
84.92	0.01\\
84.93	0.01\\
84.94	0.01\\
84.95	0.01\\
84.96	0.01\\
84.97	0.01\\
84.98	0.01\\
84.99	0.01\\
85	0.01\\
85.01	0.01\\
85.02	0.01\\
85.03	0.01\\
85.04	0.01\\
85.05	0.01\\
85.06	0.01\\
85.07	0.01\\
85.08	0.01\\
85.09	0.01\\
85.1	0.01\\
85.11	0.01\\
85.12	0.01\\
85.13	0.01\\
85.14	0.01\\
85.15	0.01\\
85.16	0.01\\
85.17	0.01\\
85.18	0.01\\
85.19	0.01\\
85.2	0.01\\
85.21	0.01\\
85.22	0.01\\
85.23	0.01\\
85.24	0.01\\
85.25	0.01\\
85.26	0.01\\
85.27	0.01\\
85.28	0.01\\
85.29	0.01\\
85.3	0.01\\
85.31	0.01\\
85.32	0.01\\
85.33	0.01\\
85.34	0.01\\
85.35	0.01\\
85.36	0.01\\
85.37	0.01\\
85.38	0.01\\
85.39	0.01\\
85.4	0.01\\
85.41	0.01\\
85.42	0.01\\
85.43	0.01\\
85.44	0.01\\
85.45	0.01\\
85.46	0.01\\
85.47	0.01\\
85.48	0.01\\
85.49	0.01\\
85.5	0.01\\
85.51	0.01\\
85.52	0.01\\
85.53	0.01\\
85.54	0.01\\
85.55	0.01\\
85.56	0.01\\
85.57	0.01\\
85.58	0.01\\
85.59	0.01\\
85.6	0.01\\
85.61	0.01\\
85.62	0.01\\
85.63	0.01\\
85.64	0.01\\
85.65	0.01\\
85.66	0.01\\
85.67	0.01\\
85.68	0.01\\
85.69	0.01\\
85.7	0.01\\
85.71	0.01\\
85.72	0.01\\
85.73	0.01\\
85.74	0.01\\
85.75	0.01\\
85.76	0.01\\
85.77	0.01\\
85.78	0.01\\
85.79	0.01\\
85.8	0.01\\
85.81	0.01\\
85.82	0.01\\
85.83	0.01\\
85.84	0.01\\
85.85	0.01\\
85.86	0.01\\
85.87	0.01\\
85.88	0.01\\
85.89	0.01\\
85.9	0.01\\
85.91	0.01\\
85.92	0.01\\
85.93	0.01\\
85.94	0.01\\
85.95	0.01\\
85.96	0.01\\
85.97	0.01\\
85.98	0.01\\
85.99	0.01\\
86	0.01\\
86.01	0.01\\
86.02	0.01\\
86.03	0.01\\
86.04	0.01\\
86.05	0.01\\
86.06	0.01\\
86.07	0.01\\
86.08	0.01\\
86.09	0.01\\
86.1	0.01\\
86.11	0.01\\
86.12	0.01\\
86.13	0.01\\
86.14	0.01\\
86.15	0.01\\
86.16	0.01\\
86.17	0.01\\
86.18	0.01\\
86.19	0.01\\
86.2	0.01\\
86.21	0.01\\
86.22	0.01\\
86.23	0.01\\
86.24	0.01\\
86.25	0.01\\
86.26	0.01\\
86.27	0.01\\
86.28	0.01\\
86.29	0.01\\
86.3	0.01\\
86.31	0.01\\
86.32	0.01\\
86.33	0.01\\
86.34	0.01\\
86.35	0.01\\
86.36	0.01\\
86.37	0.01\\
86.38	0.01\\
86.39	0.01\\
86.4	0.01\\
86.41	0.01\\
86.42	0.01\\
86.43	0.01\\
86.44	0.01\\
86.45	0.01\\
86.46	0.01\\
86.47	0.01\\
86.48	0.01\\
86.49	0.01\\
86.5	0.01\\
86.51	0.01\\
86.52	0.01\\
86.53	0.01\\
86.54	0.01\\
86.55	0.01\\
86.56	0.01\\
86.57	0.01\\
86.58	0.01\\
86.59	0.01\\
86.6	0.01\\
86.61	0.01\\
86.62	0.01\\
86.63	0.01\\
86.64	0.01\\
86.65	0.01\\
86.66	0.01\\
86.67	0.01\\
86.68	0.01\\
86.69	0.01\\
86.7	0.01\\
86.71	0.01\\
86.72	0.01\\
86.73	0.01\\
86.74	0.01\\
86.75	0.01\\
86.76	0.01\\
86.77	0.01\\
86.78	0.01\\
86.79	0.01\\
86.8	0.01\\
86.81	0.01\\
86.82	0.01\\
86.83	0.01\\
86.84	0.01\\
86.85	0.01\\
86.86	0.01\\
86.87	0.01\\
86.88	0.01\\
86.89	0.01\\
86.9	0.01\\
86.91	0.01\\
86.92	0.01\\
86.93	0.01\\
86.94	0.01\\
86.95	0.01\\
86.96	0.01\\
86.97	0.01\\
86.98	0.01\\
86.99	0.01\\
87	0.01\\
87.01	0.01\\
87.02	0.01\\
87.03	0.01\\
87.04	0.01\\
87.05	0.01\\
87.06	0.01\\
87.07	0.01\\
87.08	0.01\\
87.09	0.01\\
87.1	0.01\\
87.11	0.01\\
87.12	0.01\\
87.13	0.01\\
87.14	0.01\\
87.15	0.01\\
87.16	0.01\\
87.17	0.01\\
87.18	0.01\\
87.19	0.01\\
87.2	0.01\\
87.21	0.01\\
87.22	0.01\\
87.23	0.01\\
87.24	0.01\\
87.25	0.01\\
87.26	0.01\\
87.27	0.01\\
87.28	0.01\\
87.29	0.01\\
87.3	0.01\\
87.31	0.01\\
87.32	0.01\\
87.33	0.01\\
87.34	0.01\\
87.35	0.01\\
87.36	0.01\\
87.37	0.01\\
87.38	0.01\\
87.39	0.01\\
87.4	0.01\\
87.41	0.01\\
87.42	0.01\\
87.43	0.01\\
87.44	0.01\\
87.45	0.01\\
87.46	0.01\\
87.47	0.01\\
87.48	0.01\\
87.49	0.01\\
87.5	0.01\\
87.51	0.01\\
87.52	0.01\\
87.53	0.01\\
87.54	0.01\\
87.55	0.01\\
87.56	0.01\\
87.57	0.01\\
87.58	0.01\\
87.59	0.01\\
87.6	0.01\\
87.61	0.01\\
87.62	0.01\\
87.63	0.01\\
87.64	0.01\\
87.65	0.01\\
87.66	0.01\\
87.67	0.01\\
87.68	0.01\\
87.69	0.01\\
87.7	0.01\\
87.71	0.01\\
87.72	0.01\\
87.73	0.01\\
87.74	0.01\\
87.75	0.01\\
87.76	0.01\\
87.77	0.01\\
87.78	0.01\\
87.79	0.01\\
87.8	0.01\\
87.81	0.01\\
87.82	0.01\\
87.83	0.01\\
87.84	0.01\\
87.85	0.01\\
87.86	0.01\\
87.87	0.01\\
87.88	0.01\\
87.89	0.01\\
87.9	0.01\\
87.91	0.01\\
87.92	0.01\\
87.93	0.01\\
87.94	0.01\\
87.95	0.01\\
87.96	0.01\\
87.97	0.01\\
87.98	0.01\\
87.99	0.01\\
88	0.01\\
88.01	0.01\\
88.02	0.01\\
88.03	0.01\\
88.04	0.01\\
88.05	0.01\\
88.06	0.01\\
88.07	0.01\\
88.08	0.01\\
88.09	0.01\\
88.1	0.01\\
88.11	0.01\\
88.12	0.01\\
88.13	0.01\\
88.14	0.01\\
88.15	0.01\\
88.16	0.01\\
88.17	0.01\\
88.18	0.01\\
88.19	0.01\\
88.2	0.01\\
88.21	0.01\\
88.22	0.01\\
88.23	0.01\\
88.24	0.01\\
88.25	0.01\\
88.26	0.01\\
88.27	0.01\\
88.28	0.01\\
88.29	0.01\\
88.3	0.01\\
88.31	0.01\\
88.32	0.01\\
88.33	0.01\\
88.34	0.01\\
88.35	0.01\\
88.36	0.01\\
88.37	0.01\\
88.38	0.01\\
88.39	0.01\\
88.4	0.01\\
88.41	0.01\\
88.42	0.01\\
88.43	0.01\\
88.44	0.01\\
88.45	0.01\\
88.46	0.01\\
88.47	0.01\\
88.48	0.01\\
88.49	0.01\\
88.5	0.01\\
88.51	0.01\\
88.52	0.01\\
88.53	0.01\\
88.54	0.01\\
88.55	0.01\\
88.56	0.01\\
88.57	0.01\\
88.58	0.01\\
88.59	0.01\\
88.6	0.01\\
88.61	0.01\\
88.62	0.01\\
88.63	0.01\\
88.64	0.01\\
88.65	0.01\\
88.66	0.01\\
88.67	0.01\\
88.68	0.01\\
88.69	0.01\\
88.7	0.01\\
88.71	0.01\\
88.72	0.01\\
88.73	0.01\\
88.74	0.01\\
88.75	0.01\\
88.76	0.01\\
88.77	0.01\\
88.78	0.01\\
88.79	0.01\\
88.8	0.01\\
88.81	0.01\\
88.82	0.01\\
88.83	0.01\\
88.84	0.01\\
88.85	0.01\\
88.86	0.01\\
88.87	0.01\\
88.88	0.01\\
88.89	0.01\\
88.9	0.01\\
88.91	0.01\\
88.92	0.01\\
88.93	0.01\\
88.94	0.01\\
88.95	0.01\\
88.96	0.01\\
88.97	0.01\\
88.98	0.01\\
88.99	0.01\\
89	0.01\\
89.01	0.01\\
89.02	0.01\\
89.03	0.01\\
89.04	0.01\\
89.05	0.01\\
89.06	0.01\\
89.07	0.01\\
89.08	0.01\\
89.09	0.01\\
89.1	0.01\\
89.11	0.01\\
89.12	0.01\\
89.13	0.01\\
89.14	0.01\\
89.15	0.01\\
89.16	0.01\\
89.17	0.01\\
89.18	0.01\\
89.19	0.01\\
89.2	0.01\\
89.21	0.01\\
89.22	0.01\\
89.23	0.01\\
89.24	0.01\\
89.25	0.01\\
89.26	0.01\\
89.27	0.01\\
89.28	0.01\\
89.29	0.01\\
89.3	0.01\\
89.31	0.01\\
89.32	0.01\\
89.33	0.01\\
89.34	0.01\\
89.35	0.01\\
89.36	0.01\\
89.37	0.01\\
89.38	0.01\\
89.39	0.01\\
89.4	0.01\\
89.41	0.01\\
89.42	0.01\\
89.43	0.01\\
89.44	0.01\\
89.45	0.01\\
89.46	0.01\\
89.47	0.01\\
89.48	0.01\\
89.49	0.01\\
89.5	0.01\\
89.51	0.01\\
89.52	0.01\\
89.53	0.01\\
89.54	0.01\\
89.55	0.01\\
89.56	0.01\\
89.57	0.01\\
89.58	0.01\\
89.59	0.01\\
89.6	0.01\\
89.61	0.01\\
89.62	0.01\\
89.63	0.01\\
89.64	0.01\\
89.65	0.01\\
89.66	0.01\\
89.67	0.01\\
89.68	0.01\\
89.69	0.01\\
89.7	0.01\\
89.71	0.01\\
89.72	0.01\\
89.73	0.01\\
89.74	0.01\\
89.75	0.01\\
89.76	0.01\\
89.77	0.01\\
89.78	0.01\\
89.79	0.01\\
89.8	0.01\\
89.81	0.01\\
89.82	0.01\\
89.83	0.01\\
89.84	0.01\\
89.85	0.01\\
89.86	0.01\\
89.87	0.01\\
89.88	0.01\\
89.89	0.01\\
89.9	0.01\\
89.91	0.01\\
89.92	0.01\\
89.93	0.01\\
89.94	0.01\\
89.95	0.01\\
89.96	0.01\\
89.97	0.01\\
89.98	0.01\\
89.99	0.01\\
90	0.01\\
90.01	0.01\\
90.02	0.01\\
90.03	0.01\\
90.04	0.01\\
90.05	0.01\\
90.06	0.01\\
90.07	0.01\\
90.08	0.01\\
90.09	0.01\\
90.1	0.01\\
90.11	0.01\\
90.12	0.01\\
90.13	0.01\\
90.14	0.01\\
90.15	0.01\\
90.16	0.01\\
90.17	0.01\\
90.18	0.01\\
90.19	0.01\\
90.2	0.01\\
90.21	0.01\\
90.22	0.01\\
90.23	0.01\\
90.24	0.01\\
90.25	0.01\\
90.26	0.01\\
90.27	0.01\\
90.28	0.01\\
90.29	0.01\\
90.3	0.01\\
90.31	0.01\\
90.32	0.01\\
90.33	0.01\\
90.34	0.01\\
90.35	0.01\\
90.36	0.01\\
90.37	0.01\\
90.38	0.01\\
90.39	0.01\\
90.4	0.01\\
90.41	0.01\\
90.42	0.01\\
90.43	0.01\\
90.44	0.01\\
90.45	0.01\\
90.46	0.01\\
90.47	0.01\\
90.48	0.01\\
90.49	0.01\\
90.5	0.01\\
90.51	0.01\\
90.52	0.01\\
90.53	0.01\\
90.54	0.01\\
90.55	0.01\\
90.56	0.01\\
90.57	0.01\\
90.58	0.01\\
90.59	0.01\\
90.6	0.01\\
90.61	0.01\\
90.62	0.01\\
90.63	0.01\\
90.64	0.01\\
90.65	0.01\\
90.66	0.01\\
90.67	0.01\\
90.68	0.01\\
90.69	0.01\\
90.7	0.01\\
90.71	0.01\\
90.72	0.01\\
90.73	0.01\\
90.74	0.01\\
90.75	0.01\\
90.76	0.01\\
90.77	0.01\\
90.78	0.01\\
90.79	0.01\\
90.8	0.01\\
90.81	0.01\\
90.82	0.01\\
90.83	0.01\\
90.84	0.01\\
90.85	0.01\\
90.86	0.01\\
90.87	0.01\\
90.88	0.01\\
90.89	0.01\\
90.9	0.01\\
90.91	0.01\\
90.92	0.01\\
90.93	0.01\\
90.94	0.01\\
90.95	0.01\\
90.96	0.01\\
90.97	0.01\\
90.98	0.01\\
90.99	0.01\\
91	0.01\\
91.01	0.01\\
91.02	0.01\\
91.03	0.01\\
91.04	0.01\\
91.05	0.01\\
91.06	0.01\\
91.07	0.01\\
91.08	0.01\\
91.09	0.01\\
91.1	0.01\\
91.11	0.01\\
91.12	0.01\\
91.13	0.01\\
91.14	0.01\\
91.15	0.01\\
91.16	0.01\\
91.17	0.01\\
91.18	0.01\\
91.19	0.01\\
91.2	0.01\\
91.21	0.01\\
91.22	0.01\\
91.23	0.01\\
91.24	0.01\\
91.25	0.01\\
91.26	0.01\\
91.27	0.01\\
91.28	0.01\\
91.29	0.01\\
91.3	0.01\\
91.31	0.01\\
91.32	0.01\\
91.33	0.01\\
91.34	0.01\\
91.35	0.01\\
91.36	0.01\\
91.37	0.01\\
91.38	0.01\\
91.39	0.01\\
91.4	0.01\\
91.41	0.01\\
91.42	0.01\\
91.43	0.01\\
91.44	0.01\\
91.45	0.01\\
91.46	0.01\\
91.47	0.01\\
91.48	0.01\\
91.49	0.01\\
91.5	0.01\\
91.51	0.01\\
91.52	0.01\\
91.53	0.01\\
91.54	0.01\\
91.55	0.01\\
91.56	0.01\\
91.57	0.01\\
91.58	0.01\\
91.59	0.01\\
91.6	0.01\\
91.61	0.01\\
91.62	0.01\\
91.63	0.01\\
91.64	0.01\\
91.65	0.01\\
91.66	0.01\\
91.67	0.01\\
91.68	0.01\\
91.69	0.01\\
91.7	0.01\\
91.71	0.01\\
91.72	0.01\\
91.73	0.01\\
91.74	0.01\\
91.75	0.01\\
91.76	0.01\\
91.77	0.01\\
91.78	0.01\\
91.79	0.01\\
91.8	0.01\\
91.81	0.01\\
91.82	0.01\\
91.83	0.01\\
91.84	0.01\\
91.85	0.01\\
91.86	0.01\\
91.87	0.01\\
91.88	0.01\\
91.89	0.01\\
91.9	0.01\\
91.91	0.01\\
91.92	0.01\\
91.93	0.01\\
91.94	0.01\\
91.95	0.01\\
91.96	0.01\\
91.97	0.01\\
91.98	0.01\\
91.99	0.01\\
92	0.01\\
92.01	0.01\\
92.02	0.01\\
92.03	0.01\\
92.04	0.01\\
92.05	0.01\\
92.06	0.01\\
92.07	0.01\\
92.08	0.01\\
92.09	0.01\\
92.1	0.01\\
92.11	0.01\\
92.12	0.01\\
92.13	0.01\\
92.14	0.01\\
92.15	0.01\\
92.16	0.01\\
92.17	0.01\\
92.18	0.01\\
92.19	0.01\\
92.2	0.01\\
92.21	0.01\\
92.22	0.01\\
92.23	0.01\\
92.24	0.01\\
92.25	0.01\\
92.26	0.01\\
92.27	0.01\\
92.28	0.01\\
92.29	0.01\\
92.3	0.01\\
92.31	0.01\\
92.32	0.01\\
92.33	0.01\\
92.34	0.01\\
92.35	0.01\\
92.36	0.01\\
92.37	0.01\\
92.38	0.01\\
92.39	0.01\\
92.4	0.01\\
92.41	0.01\\
92.42	0.01\\
92.43	0.01\\
92.44	0.01\\
92.45	0.01\\
92.46	0.01\\
92.47	0.01\\
92.48	0.01\\
92.49	0.01\\
92.5	0.01\\
92.51	0.01\\
92.52	0.01\\
92.53	0.01\\
92.54	0.01\\
92.55	0.01\\
92.56	0.01\\
92.57	0.01\\
92.58	0.01\\
92.59	0.01\\
92.6	0.01\\
92.61	0.01\\
92.62	0.01\\
92.63	0.01\\
92.64	0.01\\
92.65	0.01\\
92.66	0.01\\
92.67	0.01\\
92.68	0.01\\
92.69	0.01\\
92.7	0.01\\
92.71	0.01\\
92.72	0.01\\
92.73	0.01\\
92.74	0.01\\
92.75	0.01\\
92.76	0.01\\
92.77	0.01\\
92.78	0.01\\
92.79	0.01\\
92.8	0.01\\
92.81	0.01\\
92.82	0.01\\
92.83	0.01\\
92.84	0.01\\
92.85	0.01\\
92.86	0.01\\
92.87	0.01\\
92.88	0.01\\
92.89	0.01\\
92.9	0.01\\
92.91	0.01\\
92.92	0.01\\
92.93	0.01\\
92.94	0.01\\
92.95	0.01\\
92.96	0.01\\
92.97	0.01\\
92.98	0.01\\
92.99	0.01\\
93	0.01\\
93.01	0.01\\
93.02	0.01\\
93.03	0.01\\
93.04	0.01\\
93.05	0.01\\
93.06	0.01\\
93.07	0.01\\
93.08	0.01\\
93.09	0.01\\
93.1	0.01\\
93.11	0.01\\
93.12	0.01\\
93.13	0.01\\
93.14	0.01\\
93.15	0.01\\
93.16	0.01\\
93.17	0.01\\
93.18	0.01\\
93.19	0.01\\
93.2	0.01\\
93.21	0.01\\
93.22	0.01\\
93.23	0.01\\
93.24	0.01\\
93.25	0.01\\
93.26	0.01\\
93.27	0.01\\
93.28	0.01\\
93.29	0.01\\
93.3	0.01\\
93.31	0.01\\
93.32	0.01\\
93.33	0.01\\
93.34	0.01\\
93.35	0.01\\
93.36	0.01\\
93.37	0.01\\
93.38	0.01\\
93.39	0.01\\
93.4	0.01\\
93.41	0.01\\
93.42	0.01\\
93.43	0.01\\
93.44	0.01\\
93.45	0.01\\
93.46	0.01\\
93.47	0.01\\
93.48	0.01\\
93.49	0.01\\
93.5	0.01\\
93.51	0.01\\
93.52	0.01\\
93.53	0.01\\
93.54	0.01\\
93.55	0.01\\
93.56	0.01\\
93.57	0.01\\
93.58	0.01\\
93.59	0.01\\
93.6	0.01\\
93.61	0.01\\
93.62	0.01\\
93.63	0.01\\
93.64	0.01\\
93.65	0.01\\
93.66	0.01\\
93.67	0.01\\
93.68	0.01\\
93.69	0.01\\
93.7	0.01\\
93.71	0.01\\
93.72	0.01\\
93.73	0.01\\
93.74	0.01\\
93.75	0.01\\
93.76	0.01\\
93.77	0.01\\
93.78	0.01\\
93.79	0.01\\
93.8	0.01\\
93.81	0.01\\
93.82	0.01\\
93.83	0.01\\
93.84	0.01\\
93.85	0.01\\
93.86	0.01\\
93.87	0.01\\
93.88	0.01\\
93.89	0.01\\
93.9	0.01\\
93.91	0.01\\
93.92	0.01\\
93.93	0.01\\
93.94	0.01\\
93.95	0.01\\
93.96	0.01\\
93.97	0.01\\
93.98	0.01\\
93.99	0.01\\
94	0.01\\
94.01	0.01\\
94.02	0.01\\
94.03	0.01\\
94.04	0.01\\
94.05	0.01\\
94.06	0.01\\
94.07	0.01\\
94.08	0.01\\
94.09	0.01\\
94.1	0.01\\
94.11	0.01\\
94.12	0.01\\
94.13	0.01\\
94.14	0.01\\
94.15	0.01\\
94.16	0.01\\
94.17	0.01\\
94.18	0.01\\
94.19	0.01\\
94.2	0.01\\
94.21	0.01\\
94.22	0.01\\
94.23	0.01\\
94.24	0.01\\
94.25	0.01\\
94.26	0.01\\
94.27	0.01\\
94.28	0.01\\
94.29	0.01\\
94.3	0.01\\
94.31	0.01\\
94.32	0.01\\
94.33	0.01\\
94.34	0.01\\
94.35	0.01\\
94.36	0.01\\
94.37	0.01\\
94.38	0.01\\
94.39	0.01\\
94.4	0.01\\
94.41	0.01\\
94.42	0.01\\
94.43	0.01\\
94.44	0.01\\
94.45	0.01\\
94.46	0.01\\
94.47	0.01\\
94.48	0.01\\
94.49	0.01\\
94.5	0.01\\
94.51	0.01\\
94.52	0.01\\
94.53	0.01\\
94.54	0.01\\
94.55	0.01\\
94.56	0.01\\
94.57	0.01\\
94.58	0.01\\
94.59	0.01\\
94.6	0.01\\
94.61	0.01\\
94.62	0.01\\
94.63	0.01\\
94.64	0.01\\
94.65	0.01\\
94.66	0.01\\
94.67	0.01\\
94.68	0.01\\
94.69	0.01\\
94.7	0.01\\
94.71	0.01\\
94.72	0.01\\
94.73	0.01\\
94.74	0.01\\
94.75	0.01\\
94.76	0.01\\
94.77	0.01\\
94.78	0.01\\
94.79	0.01\\
94.8	0.01\\
94.81	0.01\\
94.82	0.01\\
94.83	0.01\\
94.84	0.01\\
94.85	0.01\\
94.86	0.01\\
94.87	0.01\\
94.88	0.01\\
94.89	0.01\\
94.9	0.01\\
94.91	0.01\\
94.92	0.01\\
94.93	0.01\\
94.94	0.01\\
94.95	0.01\\
94.96	0.01\\
94.97	0.01\\
94.98	0.01\\
94.99	0.01\\
95	0.01\\
95.01	0.01\\
95.02	0.01\\
95.03	0.01\\
95.04	0.01\\
95.05	0.01\\
95.06	0.01\\
95.07	0.01\\
95.08	0.01\\
95.09	0.01\\
95.1	0.01\\
95.11	0.01\\
95.12	0.01\\
95.13	0.01\\
95.14	0.01\\
95.15	0.01\\
95.16	0.01\\
95.17	0.01\\
95.18	0.01\\
95.19	0.01\\
95.2	0.01\\
95.21	0.01\\
95.22	0.01\\
95.23	0.01\\
95.24	0.01\\
95.25	0.01\\
95.26	0.01\\
95.27	0.01\\
95.28	0.01\\
95.29	0.01\\
95.3	0.01\\
95.31	0.01\\
95.32	0.01\\
95.33	0.01\\
95.34	0.01\\
95.35	0.01\\
95.36	0.01\\
95.37	0.01\\
95.38	0.01\\
95.39	0.01\\
95.4	0.01\\
95.41	0.01\\
95.42	0.01\\
95.43	0.01\\
95.44	0.01\\
95.45	0.01\\
95.46	0.01\\
95.47	0.01\\
95.48	0.01\\
95.49	0.01\\
95.5	0.01\\
95.51	0.01\\
95.52	0.01\\
95.53	0.01\\
95.54	0.01\\
95.55	0.01\\
95.56	0.01\\
95.57	0.01\\
95.58	0.01\\
95.59	0.01\\
95.6	0.01\\
95.61	0.01\\
95.62	0.01\\
95.63	0.01\\
95.64	0.01\\
95.65	0.01\\
95.66	0.01\\
95.67	0.01\\
95.68	0.01\\
95.69	0.01\\
95.7	0.01\\
95.71	0.01\\
95.72	0.01\\
95.73	0.01\\
95.74	0.01\\
95.75	0.01\\
95.76	0.01\\
95.77	0.01\\
95.78	0.01\\
95.79	0.01\\
95.8	0.01\\
95.81	0.01\\
95.82	0.01\\
95.83	0.01\\
95.84	0.01\\
95.85	0.01\\
95.86	0.01\\
95.87	0.01\\
95.88	0.01\\
95.89	0.01\\
95.9	0.01\\
95.91	0.01\\
95.92	0.01\\
95.93	0.01\\
95.94	0.01\\
95.95	0.01\\
95.96	0.01\\
95.97	0.01\\
95.98	0.01\\
95.99	0.01\\
96	0.01\\
96.01	0.01\\
96.02	0.01\\
96.03	0.01\\
96.04	0.01\\
96.05	0.01\\
96.06	0.01\\
96.07	0.01\\
96.08	0.01\\
96.09	0.01\\
96.1	0.01\\
96.11	0.01\\
96.12	0.01\\
96.13	0.01\\
96.14	0.01\\
96.15	0.01\\
96.16	0.01\\
96.17	0.01\\
96.18	0.01\\
96.19	0.01\\
96.2	0.01\\
96.21	0.01\\
96.22	0.01\\
96.23	0.01\\
96.24	0.01\\
96.25	0.01\\
96.26	0.01\\
96.27	0.01\\
96.28	0.01\\
96.29	0.01\\
96.3	0.01\\
96.31	0.01\\
96.32	0.01\\
96.33	0.01\\
96.34	0.01\\
96.35	0.01\\
96.36	0.01\\
96.37	0.01\\
96.38	0.01\\
96.39	0.01\\
96.4	0.01\\
96.41	0.01\\
96.42	0.01\\
96.43	0.01\\
96.44	0.01\\
96.45	0.01\\
96.46	0.01\\
96.47	0.01\\
96.48	0.01\\
96.49	0.01\\
96.5	0.01\\
96.51	0.01\\
96.52	0.01\\
96.53	0.01\\
96.54	0.01\\
96.55	0.01\\
96.56	0.01\\
96.57	0.01\\
96.58	0.01\\
96.59	0.01\\
96.6	0.01\\
96.61	0.01\\
96.62	0.01\\
96.63	0.01\\
96.64	0.01\\
96.65	0.01\\
96.66	0.01\\
96.67	0.01\\
96.68	0.01\\
96.69	0.01\\
96.7	0.01\\
96.71	0.01\\
96.72	0.01\\
96.73	0.01\\
96.74	0.01\\
96.75	0.01\\
96.76	0.01\\
96.77	0.01\\
96.78	0.01\\
96.79	0.01\\
96.8	0.01\\
96.81	0.01\\
96.82	0.01\\
96.83	0.01\\
96.84	0.01\\
96.85	0.01\\
96.86	0.01\\
96.87	0.01\\
96.88	0.01\\
96.89	0.01\\
96.9	0.01\\
96.91	0.01\\
96.92	0.01\\
96.93	0.01\\
96.94	0.01\\
96.95	0.01\\
96.96	0.01\\
96.97	0.01\\
96.98	0.01\\
96.99	0.01\\
97	0.01\\
97.01	0.01\\
97.02	0.01\\
97.03	0.01\\
97.04	0.01\\
97.05	0.01\\
97.06	0.01\\
97.07	0.01\\
97.08	0.01\\
97.09	0.01\\
97.1	0.01\\
97.11	0.01\\
97.12	0.01\\
97.13	0.01\\
97.14	0.01\\
97.15	0.01\\
97.16	0.01\\
97.17	0.01\\
97.18	0.01\\
97.19	0.01\\
97.2	0.01\\
97.21	0.01\\
97.22	0.01\\
97.23	0.01\\
97.24	0.01\\
97.25	0.01\\
97.26	0.01\\
97.27	0.01\\
97.28	0.01\\
97.29	0.01\\
97.3	0.01\\
97.31	0.01\\
97.32	0.01\\
97.33	0.01\\
97.34	0.01\\
97.35	0.01\\
97.36	0.01\\
97.37	0.01\\
97.38	0.01\\
97.39	0.01\\
97.4	0.01\\
97.41	0.01\\
97.42	0.01\\
97.43	0.01\\
97.44	0.01\\
97.45	0.01\\
97.46	0.01\\
97.47	0.01\\
97.48	0.01\\
97.49	0.01\\
97.5	0.01\\
97.51	0.01\\
97.52	0.01\\
97.53	0.01\\
97.54	0.01\\
97.55	0.01\\
97.56	0.01\\
97.57	0.01\\
97.58	0.01\\
97.59	0.01\\
97.6	0.01\\
97.61	0.01\\
97.62	0.01\\
97.63	0.01\\
97.64	0.01\\
97.65	0.01\\
97.66	0.01\\
97.67	0.01\\
97.68	0.01\\
97.69	0.01\\
97.7	0.01\\
97.71	0.01\\
97.72	0.01\\
97.73	0.01\\
97.74	0.01\\
97.75	0.01\\
97.76	0.01\\
97.77	0.01\\
97.78	0.01\\
97.79	0.01\\
97.8	0.01\\
97.81	0.01\\
97.82	0.01\\
97.83	0.01\\
97.84	0.01\\
97.85	0.01\\
97.86	0.01\\
97.87	0.01\\
97.88	0.01\\
97.89	0.01\\
97.9	0.01\\
97.91	0.01\\
97.92	0.01\\
97.93	0.01\\
97.94	0.01\\
97.95	0.01\\
97.96	0.01\\
97.97	0.01\\
97.98	0.01\\
97.99	0.01\\
98	0.01\\
98.01	0.01\\
98.02	0.01\\
98.03	0.01\\
98.04	0.01\\
98.05	0.00996372752766895\\
98.06	0.00987999614471576\\
98.07	0.00979562027903187\\
98.08	0.00971059366384702\\
98.09	0.00962490996542519\\
98.1	0.00953856277553739\\
98.11	0.00945154560844692\\
98.12	0.00936385190824455\\
98.13	0.00927547504792656\\
98.14	0.00918640832845624\\
98.15	0.00909664497780845\\
98.16	0.00905441357256429\\
98.17	0.00903509579699211\\
98.18	0.00901562220828275\\
98.19	0.00899598765705672\\
98.2	0.00897619085137898\\
98.21	0.00895623048939617\\
98.22	0.00893610526244951\\
98.23	0.00891581385514894\\
98.24	0.00889535494545201\\
98.25	0.00887472720474753\\
98.26	0.0088539292979442\\
98.27	0.00883295988356457\\
98.28	0.00881181761384423\\
98.29	0.00879050113483673\\
98.3	0.0087690090865242\\
98.31	0.00874734010293407\\
98.32	0.00872549281226196\\
98.33	0.00870344911198926\\
98.34	0.00868120602500633\\
98.35	0.00865876163447077\\
98.36	0.00863611405326114\\
98.37	0.00861326137640437\\
98.38	0.00859020168090255\\
98.39	0.00856693302555777\\
98.4	0.00854345345079216\\
98.41	0.00851976097913629\\
98.42	0.00849585361986697\\
98.43	0.00847172936358991\\
98.44	0.00844738618206299\\
98.45	0.00842282202801773\\
98.46	0.00839803483497916\\
98.47	0.00837302251708372\\
98.48	0.0083477829688956\\
98.49	0.00832231406522105\\
98.5	0.0082966136609211\\
98.51	0.0082706795907222\\
98.52	0.00824450966902522\\
98.53	0.00821810168971239\\
98.54	0.0081914534259524\\
98.55	0.00816456263000364\\
98.56	0.00813742703301402\\
98.57	0.00811004434481012\\
98.58	0.00808241225369388\\
98.59	0.00805452842623716\\
98.6	0.00802639050707427\\
98.61	0.00799799611869249\\
98.62	0.00796934286122034\\
98.63	0.00794042831274018\\
98.64	0.00791125003272472\\
98.65	0.00788180555787676\\
98.66	0.00785209240191757\\
98.67	0.00782210805537347\\
98.68	0.00779184998536032\\
98.69	0.00776131563536613\\
98.7	0.00773050242503168\\
98.71	0.00769940774992909\\
98.72	0.0076680289813384\\
98.73	0.00763636346602215\\
98.74	0.00760440852598985\\
98.75	0.00757216145825925\\
98.76	0.00753961953462354\\
98.77	0.00750678000141649\\
98.78	0.00747364007927535\\
98.79	0.00744019696290129\\
98.8	0.00740644782081242\\
98.81	0.00737238979509965\\
98.82	0.00733802000118032\\
98.83	0.00730333552754955\\
98.84	0.00726833343552918\\
98.85	0.00723301075901456\\
98.86	0.00719736450421877\\
98.87	0.00716139164941458\\
98.88	0.007125089144674\\
98.89	0.00708845391160537\\
98.9	0.00705148284308801\\
98.91	0.00701417280300439\\
98.92	0.00697652062596991\\
98.93	0.00693852311706\\
98.94	0.00690017705153485\\
98.95	0.0068614791745615\\
98.96	0.00682242620093329\\
98.97	0.00678301481478685\\
98.98	0.0067432416693163\\
98.99	0.00670310338648488\\
99	0.00666259655673384\\
99.01	0.00662171773868872\\
99.02	0.00658046345886275\\
99.03	0.00653883021135762\\
99.04	0.00649681445756137\\
99.05	0.00645441262584356\\
99.06	0.00641162111124752\\
99.07	0.00636843627517979\\
99.08	0.00632485444509663\\
99.09	0.00628087191418765\\
99.1	0.00623648494105647\\
99.11	0.00619168974939843\\
99.12	0.00614648252767528\\
99.13	0.00610085942878684\\
99.14	0.00605481656973963\\
99.15	0.00600835003131242\\
99.16	0.00596145585771863\\
99.17	0.00591413005626559\\
99.18	0.0058663685970107\\
99.19	0.00581816745002337\\
99.2	0.0057695225507678\\
99.21	0.00572042979718656\\
99.22	0.00567088504935485\\
99.23	0.00562088412913144\\
99.24	0.00557042281980654\\
99.25	0.00551949686574641\\
99.26	0.00546810197203458\\
99.27	0.00541623380410996\\
99.28	0.00536388798740143\\
99.29	0.00531106010695923\\
99.3	0.00525774570708285\\
99.31	0.00520394029094553\\
99.32	0.00514963932021534\\
99.33	0.00509483821467274\\
99.34	0.00503953235182461\\
99.35	0.00498371706651476\\
99.36	0.00492738765053087\\
99.37	0.00487053935220775\\
99.38	0.00481316737602703\\
99.39	0.0047552668822131\\
99.4	0.00469683298632543\\
99.41	0.004637860758847\\
99.42	0.00457834522476916\\
99.43	0.00451828136317241\\
99.44	0.00445766410680361\\
99.45	0.00439648834164902\\
99.46	0.00433474890650368\\
99.47	0.00427244059253659\\
99.48	0.00420955814285212\\
99.49	0.00414609625204719\\
99.5	0.0040820495657645\\
99.51	0.00401741268024164\\
99.52	0.00395218014185599\\
99.53	0.00388634644666552\\
99.54	0.00381990603994536\\
99.55	0.00375285331572\\
99.56	0.00368518261629134\\
99.57	0.00361688823176235\\
99.58	0.00354796439955628\\
99.59	0.0034784053039316\\
99.6	0.00340820507549232\\
99.61	0.00333735779069392\\
99.62	0.00326585747468937\\
99.63	0.00319369810297135\\
99.64	0.00312087359552881\\
99.65	0.00304737781633648\\
99.66	0.00297320457283966\\
99.67	0.00289834761543429\\
99.68	0.00282280063694228\\
99.69	0.00274655727208197\\
99.7	0.00266961109693381\\
99.71	0.00259195562840104\\
99.72	0.00251358432366551\\
99.73	0.00243449057963848\\
99.74	0.00235466773240636\\
99.75	0.00227410905667141\\
99.76	0.0021928077651873\\
99.77	0.00211075700818945\\
99.78	0.00202794987282024\\
99.79	0.00194437938254886\\
99.8	0.0018600384965859\\
99.81	0.00177492010929258\\
99.82	0.0016890170495845\\
99.83	0.00160232208032999\\
99.84	0.00151482789774301\\
99.85	0.0014265271307703\\
99.86	0.00133741234047312\\
99.87	0.00124747601940323\\
99.88	0.00115671059097319\\
99.89	0.00106510840882085\\
99.9	0.000972661756168126\\
99.91	0.000879362845173816\\
99.92	0.00078520381628055\\
99.93	0.000690176737555767\\
99.94	0.000594273604026672\\
99.95	0.000497486337009101\\
99.96	0.000399806783430302\\
99.97	0.00030122671514549\\
99.98	0.000201737828248205\\
99.99	0.00010133174237437\\
100	0\\
};
\addlegendentry{$q=-4$};

\addplot [color=mycolor1,dashed,forget plot]
  table[row sep=crcr]{%
0.01	0.01\\
0.02	0.01\\
0.03	0.01\\
0.04	0.01\\
0.05	0.01\\
0.06	0.01\\
0.07	0.01\\
0.08	0.01\\
0.09	0.01\\
0.1	0.01\\
0.11	0.01\\
0.12	0.01\\
0.13	0.01\\
0.14	0.01\\
0.15	0.01\\
0.16	0.01\\
0.17	0.01\\
0.18	0.01\\
0.19	0.01\\
0.2	0.01\\
0.21	0.01\\
0.22	0.01\\
0.23	0.01\\
0.24	0.01\\
0.25	0.01\\
0.26	0.01\\
0.27	0.01\\
0.28	0.01\\
0.29	0.01\\
0.3	0.01\\
0.31	0.01\\
0.32	0.01\\
0.33	0.01\\
0.34	0.01\\
0.35	0.01\\
0.36	0.01\\
0.37	0.01\\
0.38	0.01\\
0.39	0.01\\
0.4	0.01\\
0.41	0.01\\
0.42	0.01\\
0.43	0.01\\
0.44	0.01\\
0.45	0.01\\
0.46	0.01\\
0.47	0.01\\
0.48	0.01\\
0.49	0.01\\
0.5	0.01\\
0.51	0.01\\
0.52	0.01\\
0.53	0.01\\
0.54	0.01\\
0.55	0.01\\
0.56	0.01\\
0.57	0.01\\
0.58	0.01\\
0.59	0.01\\
0.6	0.01\\
0.61	0.01\\
0.62	0.01\\
0.63	0.01\\
0.64	0.01\\
0.65	0.01\\
0.66	0.01\\
0.67	0.01\\
0.68	0.01\\
0.69	0.01\\
0.7	0.01\\
0.71	0.01\\
0.72	0.01\\
0.73	0.01\\
0.74	0.01\\
0.75	0.01\\
0.76	0.01\\
0.77	0.01\\
0.78	0.01\\
0.79	0.01\\
0.8	0.01\\
0.81	0.01\\
0.82	0.01\\
0.83	0.01\\
0.84	0.01\\
0.85	0.01\\
0.86	0.01\\
0.87	0.01\\
0.88	0.01\\
0.89	0.01\\
0.9	0.01\\
0.91	0.01\\
0.92	0.01\\
0.93	0.01\\
0.94	0.01\\
0.95	0.01\\
0.96	0.01\\
0.97	0.01\\
0.98	0.01\\
0.99	0.01\\
1	0.01\\
1.01	0.01\\
1.02	0.01\\
1.03	0.01\\
1.04	0.01\\
1.05	0.01\\
1.06	0.01\\
1.07	0.01\\
1.08	0.01\\
1.09	0.01\\
1.1	0.01\\
1.11	0.01\\
1.12	0.01\\
1.13	0.01\\
1.14	0.01\\
1.15	0.01\\
1.16	0.01\\
1.17	0.01\\
1.18	0.01\\
1.19	0.01\\
1.2	0.01\\
1.21	0.01\\
1.22	0.01\\
1.23	0.01\\
1.24	0.01\\
1.25	0.01\\
1.26	0.01\\
1.27	0.01\\
1.28	0.01\\
1.29	0.01\\
1.3	0.01\\
1.31	0.01\\
1.32	0.01\\
1.33	0.01\\
1.34	0.01\\
1.35	0.01\\
1.36	0.01\\
1.37	0.01\\
1.38	0.01\\
1.39	0.01\\
1.4	0.01\\
1.41	0.01\\
1.42	0.01\\
1.43	0.01\\
1.44	0.01\\
1.45	0.01\\
1.46	0.01\\
1.47	0.01\\
1.48	0.01\\
1.49	0.01\\
1.5	0.01\\
1.51	0.01\\
1.52	0.01\\
1.53	0.01\\
1.54	0.01\\
1.55	0.01\\
1.56	0.01\\
1.57	0.01\\
1.58	0.01\\
1.59	0.01\\
1.6	0.01\\
1.61	0.01\\
1.62	0.01\\
1.63	0.01\\
1.64	0.01\\
1.65	0.01\\
1.66	0.01\\
1.67	0.01\\
1.68	0.01\\
1.69	0.01\\
1.7	0.01\\
1.71	0.01\\
1.72	0.01\\
1.73	0.01\\
1.74	0.01\\
1.75	0.01\\
1.76	0.01\\
1.77	0.01\\
1.78	0.01\\
1.79	0.01\\
1.8	0.01\\
1.81	0.01\\
1.82	0.01\\
1.83	0.01\\
1.84	0.01\\
1.85	0.01\\
1.86	0.01\\
1.87	0.01\\
1.88	0.01\\
1.89	0.01\\
1.9	0.01\\
1.91	0.01\\
1.92	0.01\\
1.93	0.01\\
1.94	0.01\\
1.95	0.01\\
1.96	0.01\\
1.97	0.01\\
1.98	0.01\\
1.99	0.01\\
2	0.01\\
2.01	0.01\\
2.02	0.01\\
2.03	0.01\\
2.04	0.01\\
2.05	0.01\\
2.06	0.01\\
2.07	0.01\\
2.08	0.01\\
2.09	0.01\\
2.1	0.01\\
2.11	0.01\\
2.12	0.01\\
2.13	0.01\\
2.14	0.01\\
2.15	0.01\\
2.16	0.01\\
2.17	0.01\\
2.18	0.01\\
2.19	0.01\\
2.2	0.01\\
2.21	0.01\\
2.22	0.01\\
2.23	0.01\\
2.24	0.01\\
2.25	0.01\\
2.26	0.01\\
2.27	0.01\\
2.28	0.01\\
2.29	0.01\\
2.3	0.01\\
2.31	0.01\\
2.32	0.01\\
2.33	0.01\\
2.34	0.01\\
2.35	0.01\\
2.36	0.01\\
2.37	0.01\\
2.38	0.01\\
2.39	0.01\\
2.4	0.01\\
2.41	0.01\\
2.42	0.01\\
2.43	0.01\\
2.44	0.01\\
2.45	0.01\\
2.46	0.01\\
2.47	0.01\\
2.48	0.01\\
2.49	0.01\\
2.5	0.01\\
2.51	0.01\\
2.52	0.01\\
2.53	0.01\\
2.54	0.01\\
2.55	0.01\\
2.56	0.01\\
2.57	0.01\\
2.58	0.01\\
2.59	0.01\\
2.6	0.01\\
2.61	0.01\\
2.62	0.01\\
2.63	0.01\\
2.64	0.01\\
2.65	0.01\\
2.66	0.01\\
2.67	0.01\\
2.68	0.01\\
2.69	0.01\\
2.7	0.01\\
2.71	0.01\\
2.72	0.01\\
2.73	0.01\\
2.74	0.01\\
2.75	0.01\\
2.76	0.01\\
2.77	0.01\\
2.78	0.01\\
2.79	0.01\\
2.8	0.01\\
2.81	0.01\\
2.82	0.01\\
2.83	0.01\\
2.84	0.01\\
2.85	0.01\\
2.86	0.01\\
2.87	0.01\\
2.88	0.01\\
2.89	0.01\\
2.9	0.01\\
2.91	0.01\\
2.92	0.01\\
2.93	0.01\\
2.94	0.01\\
2.95	0.01\\
2.96	0.01\\
2.97	0.01\\
2.98	0.01\\
2.99	0.01\\
3	0.01\\
3.01	0.01\\
3.02	0.01\\
3.03	0.01\\
3.04	0.01\\
3.05	0.01\\
3.06	0.01\\
3.07	0.01\\
3.08	0.01\\
3.09	0.01\\
3.1	0.01\\
3.11	0.01\\
3.12	0.01\\
3.13	0.01\\
3.14	0.01\\
3.15	0.01\\
3.16	0.01\\
3.17	0.01\\
3.18	0.01\\
3.19	0.01\\
3.2	0.01\\
3.21	0.01\\
3.22	0.01\\
3.23	0.01\\
3.24	0.01\\
3.25	0.01\\
3.26	0.01\\
3.27	0.01\\
3.28	0.01\\
3.29	0.01\\
3.3	0.01\\
3.31	0.01\\
3.32	0.01\\
3.33	0.01\\
3.34	0.01\\
3.35	0.01\\
3.36	0.01\\
3.37	0.01\\
3.38	0.01\\
3.39	0.01\\
3.4	0.01\\
3.41	0.01\\
3.42	0.01\\
3.43	0.01\\
3.44	0.01\\
3.45	0.01\\
3.46	0.01\\
3.47	0.01\\
3.48	0.01\\
3.49	0.01\\
3.5	0.01\\
3.51	0.01\\
3.52	0.01\\
3.53	0.01\\
3.54	0.01\\
3.55	0.01\\
3.56	0.01\\
3.57	0.01\\
3.58	0.01\\
3.59	0.01\\
3.6	0.01\\
3.61	0.01\\
3.62	0.01\\
3.63	0.01\\
3.64	0.01\\
3.65	0.01\\
3.66	0.01\\
3.67	0.01\\
3.68	0.01\\
3.69	0.01\\
3.7	0.01\\
3.71	0.01\\
3.72	0.01\\
3.73	0.01\\
3.74	0.01\\
3.75	0.01\\
3.76	0.01\\
3.77	0.01\\
3.78	0.01\\
3.79	0.01\\
3.8	0.01\\
3.81	0.01\\
3.82	0.01\\
3.83	0.01\\
3.84	0.01\\
3.85	0.01\\
3.86	0.01\\
3.87	0.01\\
3.88	0.01\\
3.89	0.01\\
3.9	0.01\\
3.91	0.01\\
3.92	0.01\\
3.93	0.01\\
3.94	0.01\\
3.95	0.01\\
3.96	0.01\\
3.97	0.01\\
3.98	0.01\\
3.99	0.01\\
4	0.01\\
4.01	0.01\\
4.02	0.01\\
4.03	0.01\\
4.04	0.01\\
4.05	0.01\\
4.06	0.01\\
4.07	0.01\\
4.08	0.01\\
4.09	0.01\\
4.1	0.01\\
4.11	0.01\\
4.12	0.01\\
4.13	0.01\\
4.14	0.01\\
4.15	0.01\\
4.16	0.01\\
4.17	0.01\\
4.18	0.01\\
4.19	0.01\\
4.2	0.01\\
4.21	0.01\\
4.22	0.01\\
4.23	0.01\\
4.24	0.01\\
4.25	0.01\\
4.26	0.01\\
4.27	0.01\\
4.28	0.01\\
4.29	0.01\\
4.3	0.01\\
4.31	0.01\\
4.32	0.01\\
4.33	0.01\\
4.34	0.01\\
4.35	0.01\\
4.36	0.01\\
4.37	0.01\\
4.38	0.01\\
4.39	0.01\\
4.4	0.01\\
4.41	0.01\\
4.42	0.01\\
4.43	0.01\\
4.44	0.01\\
4.45	0.01\\
4.46	0.01\\
4.47	0.01\\
4.48	0.01\\
4.49	0.01\\
4.5	0.01\\
4.51	0.01\\
4.52	0.01\\
4.53	0.01\\
4.54	0.01\\
4.55	0.01\\
4.56	0.01\\
4.57	0.01\\
4.58	0.01\\
4.59	0.01\\
4.6	0.01\\
4.61	0.01\\
4.62	0.01\\
4.63	0.01\\
4.64	0.01\\
4.65	0.01\\
4.66	0.01\\
4.67	0.01\\
4.68	0.01\\
4.69	0.01\\
4.7	0.01\\
4.71	0.01\\
4.72	0.01\\
4.73	0.01\\
4.74	0.01\\
4.75	0.01\\
4.76	0.01\\
4.77	0.01\\
4.78	0.01\\
4.79	0.01\\
4.8	0.01\\
4.81	0.01\\
4.82	0.01\\
4.83	0.01\\
4.84	0.01\\
4.85	0.01\\
4.86	0.01\\
4.87	0.01\\
4.88	0.01\\
4.89	0.01\\
4.9	0.01\\
4.91	0.01\\
4.92	0.01\\
4.93	0.01\\
4.94	0.01\\
4.95	0.01\\
4.96	0.01\\
4.97	0.01\\
4.98	0.01\\
4.99	0.01\\
5	0.01\\
5.01	0.01\\
5.02	0.01\\
5.03	0.01\\
5.04	0.01\\
5.05	0.01\\
5.06	0.01\\
5.07	0.01\\
5.08	0.01\\
5.09	0.01\\
5.1	0.01\\
5.11	0.01\\
5.12	0.01\\
5.13	0.01\\
5.14	0.01\\
5.15	0.01\\
5.16	0.01\\
5.17	0.01\\
5.18	0.01\\
5.19	0.01\\
5.2	0.01\\
5.21	0.01\\
5.22	0.01\\
5.23	0.01\\
5.24	0.01\\
5.25	0.01\\
5.26	0.01\\
5.27	0.01\\
5.28	0.01\\
5.29	0.01\\
5.3	0.01\\
5.31	0.01\\
5.32	0.01\\
5.33	0.01\\
5.34	0.01\\
5.35	0.01\\
5.36	0.01\\
5.37	0.01\\
5.38	0.01\\
5.39	0.01\\
5.4	0.01\\
5.41	0.01\\
5.42	0.01\\
5.43	0.01\\
5.44	0.01\\
5.45	0.01\\
5.46	0.01\\
5.47	0.01\\
5.48	0.01\\
5.49	0.01\\
5.5	0.01\\
5.51	0.01\\
5.52	0.01\\
5.53	0.01\\
5.54	0.01\\
5.55	0.01\\
5.56	0.01\\
5.57	0.01\\
5.58	0.01\\
5.59	0.01\\
5.6	0.01\\
5.61	0.01\\
5.62	0.01\\
5.63	0.01\\
5.64	0.01\\
5.65	0.01\\
5.66	0.01\\
5.67	0.01\\
5.68	0.01\\
5.69	0.01\\
5.7	0.01\\
5.71	0.01\\
5.72	0.01\\
5.73	0.01\\
5.74	0.01\\
5.75	0.01\\
5.76	0.01\\
5.77	0.01\\
5.78	0.01\\
5.79	0.01\\
5.8	0.01\\
5.81	0.01\\
5.82	0.01\\
5.83	0.01\\
5.84	0.01\\
5.85	0.01\\
5.86	0.01\\
5.87	0.01\\
5.88	0.01\\
5.89	0.01\\
5.9	0.01\\
5.91	0.01\\
5.92	0.01\\
5.93	0.01\\
5.94	0.01\\
5.95	0.01\\
5.96	0.01\\
5.97	0.01\\
5.98	0.01\\
5.99	0.01\\
6	0.01\\
6.01	0.01\\
6.02	0.01\\
6.03	0.01\\
6.04	0.01\\
6.05	0.01\\
6.06	0.01\\
6.07	0.01\\
6.08	0.01\\
6.09	0.01\\
6.1	0.01\\
6.11	0.01\\
6.12	0.01\\
6.13	0.01\\
6.14	0.01\\
6.15	0.01\\
6.16	0.01\\
6.17	0.01\\
6.18	0.01\\
6.19	0.01\\
6.2	0.01\\
6.21	0.01\\
6.22	0.01\\
6.23	0.01\\
6.24	0.01\\
6.25	0.01\\
6.26	0.01\\
6.27	0.01\\
6.28	0.01\\
6.29	0.01\\
6.3	0.01\\
6.31	0.01\\
6.32	0.01\\
6.33	0.01\\
6.34	0.01\\
6.35	0.01\\
6.36	0.01\\
6.37	0.01\\
6.38	0.01\\
6.39	0.01\\
6.4	0.01\\
6.41	0.01\\
6.42	0.01\\
6.43	0.01\\
6.44	0.01\\
6.45	0.01\\
6.46	0.01\\
6.47	0.01\\
6.48	0.01\\
6.49	0.01\\
6.5	0.01\\
6.51	0.01\\
6.52	0.01\\
6.53	0.01\\
6.54	0.01\\
6.55	0.01\\
6.56	0.01\\
6.57	0.01\\
6.58	0.01\\
6.59	0.01\\
6.6	0.01\\
6.61	0.01\\
6.62	0.01\\
6.63	0.01\\
6.64	0.01\\
6.65	0.01\\
6.66	0.01\\
6.67	0.01\\
6.68	0.01\\
6.69	0.01\\
6.7	0.01\\
6.71	0.01\\
6.72	0.01\\
6.73	0.01\\
6.74	0.01\\
6.75	0.01\\
6.76	0.01\\
6.77	0.01\\
6.78	0.01\\
6.79	0.01\\
6.8	0.01\\
6.81	0.01\\
6.82	0.01\\
6.83	0.01\\
6.84	0.01\\
6.85	0.01\\
6.86	0.01\\
6.87	0.01\\
6.88	0.01\\
6.89	0.01\\
6.9	0.01\\
6.91	0.01\\
6.92	0.01\\
6.93	0.01\\
6.94	0.01\\
6.95	0.01\\
6.96	0.01\\
6.97	0.01\\
6.98	0.01\\
6.99	0.01\\
7	0.01\\
7.01	0.01\\
7.02	0.01\\
7.03	0.01\\
7.04	0.01\\
7.05	0.01\\
7.06	0.01\\
7.07	0.01\\
7.08	0.01\\
7.09	0.01\\
7.1	0.01\\
7.11	0.01\\
7.12	0.01\\
7.13	0.01\\
7.14	0.01\\
7.15	0.01\\
7.16	0.01\\
7.17	0.01\\
7.18	0.01\\
7.19	0.01\\
7.2	0.01\\
7.21	0.01\\
7.22	0.01\\
7.23	0.01\\
7.24	0.01\\
7.25	0.01\\
7.26	0.01\\
7.27	0.01\\
7.28	0.01\\
7.29	0.01\\
7.3	0.01\\
7.31	0.01\\
7.32	0.01\\
7.33	0.01\\
7.34	0.01\\
7.35	0.01\\
7.36	0.01\\
7.37	0.01\\
7.38	0.01\\
7.39	0.01\\
7.4	0.01\\
7.41	0.01\\
7.42	0.01\\
7.43	0.01\\
7.44	0.01\\
7.45	0.01\\
7.46	0.01\\
7.47	0.01\\
7.48	0.01\\
7.49	0.01\\
7.5	0.01\\
7.51	0.01\\
7.52	0.01\\
7.53	0.01\\
7.54	0.01\\
7.55	0.01\\
7.56	0.01\\
7.57	0.01\\
7.58	0.01\\
7.59	0.01\\
7.6	0.01\\
7.61	0.01\\
7.62	0.01\\
7.63	0.01\\
7.64	0.01\\
7.65	0.01\\
7.66	0.01\\
7.67	0.01\\
7.68	0.01\\
7.69	0.01\\
7.7	0.01\\
7.71	0.01\\
7.72	0.01\\
7.73	0.01\\
7.74	0.01\\
7.75	0.01\\
7.76	0.01\\
7.77	0.01\\
7.78	0.01\\
7.79	0.01\\
7.8	0.01\\
7.81	0.01\\
7.82	0.01\\
7.83	0.01\\
7.84	0.01\\
7.85	0.01\\
7.86	0.01\\
7.87	0.01\\
7.88	0.01\\
7.89	0.01\\
7.9	0.01\\
7.91	0.01\\
7.92	0.01\\
7.93	0.01\\
7.94	0.01\\
7.95	0.01\\
7.96	0.01\\
7.97	0.01\\
7.98	0.01\\
7.99	0.01\\
8	0.01\\
8.01	0.01\\
8.02	0.01\\
8.03	0.01\\
8.04	0.01\\
8.05	0.01\\
8.06	0.01\\
8.07	0.01\\
8.08	0.01\\
8.09	0.01\\
8.1	0.01\\
8.11	0.01\\
8.12	0.01\\
8.13	0.01\\
8.14	0.01\\
8.15	0.01\\
8.16	0.01\\
8.17	0.01\\
8.18	0.01\\
8.19	0.01\\
8.2	0.01\\
8.21	0.01\\
8.22	0.01\\
8.23	0.01\\
8.24	0.01\\
8.25	0.01\\
8.26	0.01\\
8.27	0.01\\
8.28	0.01\\
8.29	0.01\\
8.3	0.01\\
8.31	0.01\\
8.32	0.01\\
8.33	0.01\\
8.34	0.01\\
8.35	0.01\\
8.36	0.01\\
8.37	0.01\\
8.38	0.01\\
8.39	0.01\\
8.4	0.01\\
8.41	0.01\\
8.42	0.01\\
8.43	0.01\\
8.44	0.01\\
8.45	0.01\\
8.46	0.01\\
8.47	0.01\\
8.48	0.01\\
8.49	0.01\\
8.5	0.01\\
8.51	0.01\\
8.52	0.01\\
8.53	0.01\\
8.54	0.01\\
8.55	0.01\\
8.56	0.01\\
8.57	0.01\\
8.58	0.01\\
8.59	0.01\\
8.6	0.01\\
8.61	0.01\\
8.62	0.01\\
8.63	0.01\\
8.64	0.01\\
8.65	0.01\\
8.66	0.01\\
8.67	0.01\\
8.68	0.01\\
8.69	0.01\\
8.7	0.01\\
8.71	0.01\\
8.72	0.01\\
8.73	0.01\\
8.74	0.01\\
8.75	0.01\\
8.76	0.01\\
8.77	0.01\\
8.78	0.01\\
8.79	0.01\\
8.8	0.01\\
8.81	0.01\\
8.82	0.01\\
8.83	0.01\\
8.84	0.01\\
8.85	0.01\\
8.86	0.01\\
8.87	0.01\\
8.88	0.01\\
8.89	0.01\\
8.9	0.01\\
8.91	0.01\\
8.92	0.01\\
8.93	0.01\\
8.94	0.01\\
8.95	0.01\\
8.96	0.01\\
8.97	0.01\\
8.98	0.01\\
8.99	0.01\\
9	0.01\\
9.01	0.01\\
9.02	0.01\\
9.03	0.01\\
9.04	0.01\\
9.05	0.01\\
9.06	0.01\\
9.07	0.01\\
9.08	0.01\\
9.09	0.01\\
9.1	0.01\\
9.11	0.01\\
9.12	0.01\\
9.13	0.01\\
9.14	0.01\\
9.15	0.01\\
9.16	0.01\\
9.17	0.01\\
9.18	0.01\\
9.19	0.01\\
9.2	0.01\\
9.21	0.01\\
9.22	0.01\\
9.23	0.01\\
9.24	0.01\\
9.25	0.01\\
9.26	0.01\\
9.27	0.01\\
9.28	0.01\\
9.29	0.01\\
9.3	0.01\\
9.31	0.01\\
9.32	0.01\\
9.33	0.01\\
9.34	0.01\\
9.35	0.01\\
9.36	0.01\\
9.37	0.01\\
9.38	0.01\\
9.39	0.01\\
9.4	0.01\\
9.41	0.01\\
9.42	0.01\\
9.43	0.01\\
9.44	0.01\\
9.45	0.01\\
9.46	0.01\\
9.47	0.01\\
9.48	0.01\\
9.49	0.01\\
9.5	0.01\\
9.51	0.01\\
9.52	0.01\\
9.53	0.01\\
9.54	0.01\\
9.55	0.01\\
9.56	0.01\\
9.57	0.01\\
9.58	0.01\\
9.59	0.01\\
9.6	0.01\\
9.61	0.01\\
9.62	0.01\\
9.63	0.01\\
9.64	0.01\\
9.65	0.01\\
9.66	0.01\\
9.67	0.01\\
9.68	0.01\\
9.69	0.01\\
9.7	0.01\\
9.71	0.01\\
9.72	0.01\\
9.73	0.01\\
9.74	0.01\\
9.75	0.01\\
9.76	0.01\\
9.77	0.01\\
9.78	0.01\\
9.79	0.01\\
9.8	0.01\\
9.81	0.01\\
9.82	0.01\\
9.83	0.01\\
9.84	0.01\\
9.85	0.01\\
9.86	0.01\\
9.87	0.01\\
9.88	0.01\\
9.89	0.01\\
9.9	0.01\\
9.91	0.01\\
9.92	0.01\\
9.93	0.01\\
9.94	0.01\\
9.95	0.01\\
9.96	0.01\\
9.97	0.01\\
9.98	0.01\\
9.99	0.01\\
10	0.01\\
10.01	0.01\\
10.02	0.01\\
10.03	0.01\\
10.04	0.01\\
10.05	0.01\\
10.06	0.01\\
10.07	0.01\\
10.08	0.01\\
10.09	0.01\\
10.1	0.01\\
10.11	0.01\\
10.12	0.01\\
10.13	0.01\\
10.14	0.01\\
10.15	0.01\\
10.16	0.01\\
10.17	0.01\\
10.18	0.01\\
10.19	0.01\\
10.2	0.01\\
10.21	0.01\\
10.22	0.01\\
10.23	0.01\\
10.24	0.01\\
10.25	0.01\\
10.26	0.01\\
10.27	0.01\\
10.28	0.01\\
10.29	0.01\\
10.3	0.01\\
10.31	0.01\\
10.32	0.01\\
10.33	0.01\\
10.34	0.01\\
10.35	0.01\\
10.36	0.01\\
10.37	0.01\\
10.38	0.01\\
10.39	0.01\\
10.4	0.01\\
10.41	0.01\\
10.42	0.01\\
10.43	0.01\\
10.44	0.01\\
10.45	0.01\\
10.46	0.01\\
10.47	0.01\\
10.48	0.01\\
10.49	0.01\\
10.5	0.01\\
10.51	0.01\\
10.52	0.01\\
10.53	0.01\\
10.54	0.01\\
10.55	0.01\\
10.56	0.01\\
10.57	0.01\\
10.58	0.01\\
10.59	0.01\\
10.6	0.01\\
10.61	0.01\\
10.62	0.01\\
10.63	0.01\\
10.64	0.01\\
10.65	0.01\\
10.66	0.01\\
10.67	0.01\\
10.68	0.01\\
10.69	0.01\\
10.7	0.01\\
10.71	0.01\\
10.72	0.01\\
10.73	0.01\\
10.74	0.01\\
10.75	0.01\\
10.76	0.01\\
10.77	0.01\\
10.78	0.01\\
10.79	0.01\\
10.8	0.01\\
10.81	0.01\\
10.82	0.01\\
10.83	0.01\\
10.84	0.01\\
10.85	0.01\\
10.86	0.01\\
10.87	0.01\\
10.88	0.01\\
10.89	0.01\\
10.9	0.01\\
10.91	0.01\\
10.92	0.01\\
10.93	0.01\\
10.94	0.01\\
10.95	0.01\\
10.96	0.01\\
10.97	0.01\\
10.98	0.01\\
10.99	0.01\\
11	0.01\\
11.01	0.01\\
11.02	0.01\\
11.03	0.01\\
11.04	0.01\\
11.05	0.01\\
11.06	0.01\\
11.07	0.01\\
11.08	0.01\\
11.09	0.01\\
11.1	0.01\\
11.11	0.01\\
11.12	0.01\\
11.13	0.01\\
11.14	0.01\\
11.15	0.01\\
11.16	0.01\\
11.17	0.01\\
11.18	0.01\\
11.19	0.01\\
11.2	0.01\\
11.21	0.01\\
11.22	0.01\\
11.23	0.01\\
11.24	0.01\\
11.25	0.01\\
11.26	0.01\\
11.27	0.01\\
11.28	0.01\\
11.29	0.01\\
11.3	0.01\\
11.31	0.01\\
11.32	0.01\\
11.33	0.01\\
11.34	0.01\\
11.35	0.01\\
11.36	0.01\\
11.37	0.01\\
11.38	0.01\\
11.39	0.01\\
11.4	0.01\\
11.41	0.01\\
11.42	0.01\\
11.43	0.01\\
11.44	0.01\\
11.45	0.01\\
11.46	0.01\\
11.47	0.01\\
11.48	0.01\\
11.49	0.01\\
11.5	0.01\\
11.51	0.01\\
11.52	0.01\\
11.53	0.01\\
11.54	0.01\\
11.55	0.01\\
11.56	0.01\\
11.57	0.01\\
11.58	0.01\\
11.59	0.01\\
11.6	0.01\\
11.61	0.01\\
11.62	0.01\\
11.63	0.01\\
11.64	0.01\\
11.65	0.01\\
11.66	0.01\\
11.67	0.01\\
11.68	0.01\\
11.69	0.01\\
11.7	0.01\\
11.71	0.01\\
11.72	0.01\\
11.73	0.01\\
11.74	0.01\\
11.75	0.01\\
11.76	0.01\\
11.77	0.01\\
11.78	0.01\\
11.79	0.01\\
11.8	0.01\\
11.81	0.01\\
11.82	0.01\\
11.83	0.01\\
11.84	0.01\\
11.85	0.01\\
11.86	0.01\\
11.87	0.01\\
11.88	0.01\\
11.89	0.01\\
11.9	0.01\\
11.91	0.01\\
11.92	0.01\\
11.93	0.01\\
11.94	0.01\\
11.95	0.01\\
11.96	0.01\\
11.97	0.01\\
11.98	0.01\\
11.99	0.01\\
12	0.01\\
12.01	0.01\\
12.02	0.01\\
12.03	0.01\\
12.04	0.01\\
12.05	0.01\\
12.06	0.01\\
12.07	0.01\\
12.08	0.01\\
12.09	0.01\\
12.1	0.01\\
12.11	0.01\\
12.12	0.01\\
12.13	0.01\\
12.14	0.01\\
12.15	0.01\\
12.16	0.01\\
12.17	0.01\\
12.18	0.01\\
12.19	0.01\\
12.2	0.01\\
12.21	0.01\\
12.22	0.01\\
12.23	0.01\\
12.24	0.01\\
12.25	0.01\\
12.26	0.01\\
12.27	0.01\\
12.28	0.01\\
12.29	0.01\\
12.3	0.01\\
12.31	0.01\\
12.32	0.01\\
12.33	0.01\\
12.34	0.01\\
12.35	0.01\\
12.36	0.01\\
12.37	0.01\\
12.38	0.01\\
12.39	0.01\\
12.4	0.01\\
12.41	0.01\\
12.42	0.01\\
12.43	0.01\\
12.44	0.01\\
12.45	0.01\\
12.46	0.01\\
12.47	0.01\\
12.48	0.01\\
12.49	0.01\\
12.5	0.01\\
12.51	0.01\\
12.52	0.01\\
12.53	0.01\\
12.54	0.01\\
12.55	0.01\\
12.56	0.01\\
12.57	0.01\\
12.58	0.01\\
12.59	0.01\\
12.6	0.01\\
12.61	0.01\\
12.62	0.01\\
12.63	0.01\\
12.64	0.01\\
12.65	0.01\\
12.66	0.01\\
12.67	0.01\\
12.68	0.01\\
12.69	0.01\\
12.7	0.01\\
12.71	0.01\\
12.72	0.01\\
12.73	0.01\\
12.74	0.01\\
12.75	0.01\\
12.76	0.01\\
12.77	0.01\\
12.78	0.01\\
12.79	0.01\\
12.8	0.01\\
12.81	0.01\\
12.82	0.01\\
12.83	0.01\\
12.84	0.01\\
12.85	0.01\\
12.86	0.01\\
12.87	0.01\\
12.88	0.01\\
12.89	0.01\\
12.9	0.01\\
12.91	0.01\\
12.92	0.01\\
12.93	0.01\\
12.94	0.01\\
12.95	0.01\\
12.96	0.01\\
12.97	0.01\\
12.98	0.01\\
12.99	0.01\\
13	0.01\\
13.01	0.01\\
13.02	0.01\\
13.03	0.01\\
13.04	0.01\\
13.05	0.01\\
13.06	0.01\\
13.07	0.01\\
13.08	0.01\\
13.09	0.01\\
13.1	0.01\\
13.11	0.01\\
13.12	0.01\\
13.13	0.01\\
13.14	0.01\\
13.15	0.01\\
13.16	0.01\\
13.17	0.01\\
13.18	0.01\\
13.19	0.01\\
13.2	0.01\\
13.21	0.01\\
13.22	0.01\\
13.23	0.01\\
13.24	0.01\\
13.25	0.01\\
13.26	0.01\\
13.27	0.01\\
13.28	0.01\\
13.29	0.01\\
13.3	0.01\\
13.31	0.01\\
13.32	0.01\\
13.33	0.01\\
13.34	0.01\\
13.35	0.01\\
13.36	0.01\\
13.37	0.01\\
13.38	0.01\\
13.39	0.01\\
13.4	0.01\\
13.41	0.01\\
13.42	0.01\\
13.43	0.01\\
13.44	0.01\\
13.45	0.01\\
13.46	0.01\\
13.47	0.01\\
13.48	0.01\\
13.49	0.01\\
13.5	0.01\\
13.51	0.01\\
13.52	0.01\\
13.53	0.01\\
13.54	0.01\\
13.55	0.01\\
13.56	0.01\\
13.57	0.01\\
13.58	0.01\\
13.59	0.01\\
13.6	0.01\\
13.61	0.01\\
13.62	0.01\\
13.63	0.01\\
13.64	0.01\\
13.65	0.01\\
13.66	0.01\\
13.67	0.01\\
13.68	0.01\\
13.69	0.01\\
13.7	0.01\\
13.71	0.01\\
13.72	0.01\\
13.73	0.01\\
13.74	0.01\\
13.75	0.01\\
13.76	0.01\\
13.77	0.01\\
13.78	0.01\\
13.79	0.01\\
13.8	0.01\\
13.81	0.01\\
13.82	0.01\\
13.83	0.01\\
13.84	0.01\\
13.85	0.01\\
13.86	0.01\\
13.87	0.01\\
13.88	0.01\\
13.89	0.01\\
13.9	0.01\\
13.91	0.01\\
13.92	0.01\\
13.93	0.01\\
13.94	0.01\\
13.95	0.01\\
13.96	0.01\\
13.97	0.01\\
13.98	0.01\\
13.99	0.01\\
14	0.01\\
14.01	0.01\\
14.02	0.01\\
14.03	0.01\\
14.04	0.01\\
14.05	0.01\\
14.06	0.01\\
14.07	0.01\\
14.08	0.01\\
14.09	0.01\\
14.1	0.01\\
14.11	0.01\\
14.12	0.01\\
14.13	0.01\\
14.14	0.01\\
14.15	0.01\\
14.16	0.01\\
14.17	0.01\\
14.18	0.01\\
14.19	0.01\\
14.2	0.01\\
14.21	0.01\\
14.22	0.01\\
14.23	0.01\\
14.24	0.01\\
14.25	0.01\\
14.26	0.01\\
14.27	0.01\\
14.28	0.01\\
14.29	0.01\\
14.3	0.01\\
14.31	0.01\\
14.32	0.01\\
14.33	0.01\\
14.34	0.01\\
14.35	0.01\\
14.36	0.01\\
14.37	0.01\\
14.38	0.01\\
14.39	0.01\\
14.4	0.01\\
14.41	0.01\\
14.42	0.01\\
14.43	0.01\\
14.44	0.01\\
14.45	0.01\\
14.46	0.01\\
14.47	0.01\\
14.48	0.01\\
14.49	0.01\\
14.5	0.01\\
14.51	0.01\\
14.52	0.01\\
14.53	0.01\\
14.54	0.01\\
14.55	0.01\\
14.56	0.01\\
14.57	0.01\\
14.58	0.01\\
14.59	0.01\\
14.6	0.01\\
14.61	0.01\\
14.62	0.01\\
14.63	0.01\\
14.64	0.01\\
14.65	0.01\\
14.66	0.01\\
14.67	0.01\\
14.68	0.01\\
14.69	0.01\\
14.7	0.01\\
14.71	0.01\\
14.72	0.01\\
14.73	0.01\\
14.74	0.01\\
14.75	0.01\\
14.76	0.01\\
14.77	0.01\\
14.78	0.01\\
14.79	0.01\\
14.8	0.01\\
14.81	0.01\\
14.82	0.01\\
14.83	0.01\\
14.84	0.01\\
14.85	0.01\\
14.86	0.01\\
14.87	0.01\\
14.88	0.01\\
14.89	0.01\\
14.9	0.01\\
14.91	0.01\\
14.92	0.01\\
14.93	0.01\\
14.94	0.01\\
14.95	0.01\\
14.96	0.01\\
14.97	0.01\\
14.98	0.01\\
14.99	0.01\\
15	0.01\\
15.01	0.01\\
15.02	0.01\\
15.03	0.01\\
15.04	0.01\\
15.05	0.01\\
15.06	0.01\\
15.07	0.01\\
15.08	0.01\\
15.09	0.01\\
15.1	0.01\\
15.11	0.01\\
15.12	0.01\\
15.13	0.01\\
15.14	0.01\\
15.15	0.01\\
15.16	0.01\\
15.17	0.01\\
15.18	0.01\\
15.19	0.01\\
15.2	0.01\\
15.21	0.01\\
15.22	0.01\\
15.23	0.01\\
15.24	0.01\\
15.25	0.01\\
15.26	0.01\\
15.27	0.01\\
15.28	0.01\\
15.29	0.01\\
15.3	0.01\\
15.31	0.01\\
15.32	0.01\\
15.33	0.01\\
15.34	0.01\\
15.35	0.01\\
15.36	0.01\\
15.37	0.01\\
15.38	0.01\\
15.39	0.01\\
15.4	0.01\\
15.41	0.01\\
15.42	0.01\\
15.43	0.01\\
15.44	0.01\\
15.45	0.01\\
15.46	0.01\\
15.47	0.01\\
15.48	0.01\\
15.49	0.01\\
15.5	0.01\\
15.51	0.01\\
15.52	0.01\\
15.53	0.01\\
15.54	0.01\\
15.55	0.01\\
15.56	0.01\\
15.57	0.01\\
15.58	0.01\\
15.59	0.01\\
15.6	0.01\\
15.61	0.01\\
15.62	0.01\\
15.63	0.01\\
15.64	0.01\\
15.65	0.01\\
15.66	0.01\\
15.67	0.01\\
15.68	0.01\\
15.69	0.01\\
15.7	0.01\\
15.71	0.01\\
15.72	0.01\\
15.73	0.01\\
15.74	0.01\\
15.75	0.01\\
15.76	0.01\\
15.77	0.01\\
15.78	0.01\\
15.79	0.01\\
15.8	0.01\\
15.81	0.01\\
15.82	0.01\\
15.83	0.01\\
15.84	0.01\\
15.85	0.01\\
15.86	0.01\\
15.87	0.01\\
15.88	0.01\\
15.89	0.01\\
15.9	0.01\\
15.91	0.01\\
15.92	0.01\\
15.93	0.01\\
15.94	0.01\\
15.95	0.01\\
15.96	0.01\\
15.97	0.01\\
15.98	0.01\\
15.99	0.01\\
16	0.01\\
16.01	0.01\\
16.02	0.01\\
16.03	0.01\\
16.04	0.01\\
16.05	0.01\\
16.06	0.01\\
16.07	0.01\\
16.08	0.01\\
16.09	0.01\\
16.1	0.01\\
16.11	0.01\\
16.12	0.01\\
16.13	0.01\\
16.14	0.01\\
16.15	0.01\\
16.16	0.01\\
16.17	0.01\\
16.18	0.01\\
16.19	0.01\\
16.2	0.01\\
16.21	0.01\\
16.22	0.01\\
16.23	0.01\\
16.24	0.01\\
16.25	0.01\\
16.26	0.01\\
16.27	0.01\\
16.28	0.01\\
16.29	0.01\\
16.3	0.01\\
16.31	0.01\\
16.32	0.01\\
16.33	0.01\\
16.34	0.01\\
16.35	0.01\\
16.36	0.01\\
16.37	0.01\\
16.38	0.01\\
16.39	0.01\\
16.4	0.01\\
16.41	0.01\\
16.42	0.01\\
16.43	0.01\\
16.44	0.01\\
16.45	0.01\\
16.46	0.01\\
16.47	0.01\\
16.48	0.01\\
16.49	0.01\\
16.5	0.01\\
16.51	0.01\\
16.52	0.01\\
16.53	0.01\\
16.54	0.01\\
16.55	0.01\\
16.56	0.01\\
16.57	0.01\\
16.58	0.01\\
16.59	0.01\\
16.6	0.01\\
16.61	0.01\\
16.62	0.01\\
16.63	0.01\\
16.64	0.01\\
16.65	0.01\\
16.66	0.01\\
16.67	0.01\\
16.68	0.01\\
16.69	0.01\\
16.7	0.01\\
16.71	0.01\\
16.72	0.01\\
16.73	0.01\\
16.74	0.01\\
16.75	0.01\\
16.76	0.01\\
16.77	0.01\\
16.78	0.01\\
16.79	0.01\\
16.8	0.01\\
16.81	0.01\\
16.82	0.01\\
16.83	0.01\\
16.84	0.01\\
16.85	0.01\\
16.86	0.01\\
16.87	0.01\\
16.88	0.01\\
16.89	0.01\\
16.9	0.01\\
16.91	0.01\\
16.92	0.01\\
16.93	0.01\\
16.94	0.01\\
16.95	0.01\\
16.96	0.01\\
16.97	0.01\\
16.98	0.01\\
16.99	0.01\\
17	0.01\\
17.01	0.01\\
17.02	0.01\\
17.03	0.01\\
17.04	0.01\\
17.05	0.01\\
17.06	0.01\\
17.07	0.01\\
17.08	0.01\\
17.09	0.01\\
17.1	0.01\\
17.11	0.01\\
17.12	0.01\\
17.13	0.01\\
17.14	0.01\\
17.15	0.01\\
17.16	0.01\\
17.17	0.01\\
17.18	0.01\\
17.19	0.01\\
17.2	0.01\\
17.21	0.01\\
17.22	0.01\\
17.23	0.01\\
17.24	0.01\\
17.25	0.01\\
17.26	0.01\\
17.27	0.01\\
17.28	0.01\\
17.29	0.01\\
17.3	0.01\\
17.31	0.01\\
17.32	0.01\\
17.33	0.01\\
17.34	0.01\\
17.35	0.01\\
17.36	0.01\\
17.37	0.01\\
17.38	0.01\\
17.39	0.01\\
17.4	0.01\\
17.41	0.01\\
17.42	0.01\\
17.43	0.01\\
17.44	0.01\\
17.45	0.01\\
17.46	0.01\\
17.47	0.01\\
17.48	0.01\\
17.49	0.01\\
17.5	0.01\\
17.51	0.01\\
17.52	0.01\\
17.53	0.01\\
17.54	0.01\\
17.55	0.01\\
17.56	0.01\\
17.57	0.01\\
17.58	0.01\\
17.59	0.01\\
17.6	0.01\\
17.61	0.01\\
17.62	0.01\\
17.63	0.01\\
17.64	0.01\\
17.65	0.01\\
17.66	0.01\\
17.67	0.01\\
17.68	0.01\\
17.69	0.01\\
17.7	0.01\\
17.71	0.01\\
17.72	0.01\\
17.73	0.01\\
17.74	0.01\\
17.75	0.01\\
17.76	0.01\\
17.77	0.01\\
17.78	0.01\\
17.79	0.01\\
17.8	0.01\\
17.81	0.01\\
17.82	0.01\\
17.83	0.01\\
17.84	0.01\\
17.85	0.01\\
17.86	0.01\\
17.87	0.01\\
17.88	0.01\\
17.89	0.01\\
17.9	0.01\\
17.91	0.01\\
17.92	0.01\\
17.93	0.01\\
17.94	0.01\\
17.95	0.01\\
17.96	0.01\\
17.97	0.01\\
17.98	0.01\\
17.99	0.01\\
18	0.01\\
18.01	0.01\\
18.02	0.01\\
18.03	0.01\\
18.04	0.01\\
18.05	0.01\\
18.06	0.01\\
18.07	0.01\\
18.08	0.01\\
18.09	0.01\\
18.1	0.01\\
18.11	0.01\\
18.12	0.01\\
18.13	0.01\\
18.14	0.01\\
18.15	0.01\\
18.16	0.01\\
18.17	0.01\\
18.18	0.01\\
18.19	0.01\\
18.2	0.01\\
18.21	0.01\\
18.22	0.01\\
18.23	0.01\\
18.24	0.01\\
18.25	0.01\\
18.26	0.01\\
18.27	0.01\\
18.28	0.01\\
18.29	0.01\\
18.3	0.01\\
18.31	0.01\\
18.32	0.01\\
18.33	0.01\\
18.34	0.01\\
18.35	0.01\\
18.36	0.01\\
18.37	0.01\\
18.38	0.01\\
18.39	0.01\\
18.4	0.01\\
18.41	0.01\\
18.42	0.01\\
18.43	0.01\\
18.44	0.01\\
18.45	0.01\\
18.46	0.01\\
18.47	0.01\\
18.48	0.01\\
18.49	0.01\\
18.5	0.01\\
18.51	0.01\\
18.52	0.01\\
18.53	0.01\\
18.54	0.01\\
18.55	0.01\\
18.56	0.01\\
18.57	0.01\\
18.58	0.01\\
18.59	0.01\\
18.6	0.01\\
18.61	0.01\\
18.62	0.01\\
18.63	0.01\\
18.64	0.01\\
18.65	0.01\\
18.66	0.01\\
18.67	0.01\\
18.68	0.01\\
18.69	0.01\\
18.7	0.01\\
18.71	0.01\\
18.72	0.01\\
18.73	0.01\\
18.74	0.01\\
18.75	0.01\\
18.76	0.01\\
18.77	0.01\\
18.78	0.01\\
18.79	0.01\\
18.8	0.01\\
18.81	0.01\\
18.82	0.01\\
18.83	0.01\\
18.84	0.01\\
18.85	0.01\\
18.86	0.01\\
18.87	0.01\\
18.88	0.01\\
18.89	0.01\\
18.9	0.01\\
18.91	0.01\\
18.92	0.01\\
18.93	0.01\\
18.94	0.01\\
18.95	0.01\\
18.96	0.01\\
18.97	0.01\\
18.98	0.01\\
18.99	0.01\\
19	0.01\\
19.01	0.01\\
19.02	0.01\\
19.03	0.01\\
19.04	0.01\\
19.05	0.01\\
19.06	0.01\\
19.07	0.01\\
19.08	0.01\\
19.09	0.01\\
19.1	0.01\\
19.11	0.01\\
19.12	0.01\\
19.13	0.01\\
19.14	0.01\\
19.15	0.01\\
19.16	0.01\\
19.17	0.01\\
19.18	0.01\\
19.19	0.01\\
19.2	0.01\\
19.21	0.01\\
19.22	0.01\\
19.23	0.01\\
19.24	0.01\\
19.25	0.01\\
19.26	0.01\\
19.27	0.01\\
19.28	0.01\\
19.29	0.01\\
19.3	0.01\\
19.31	0.01\\
19.32	0.01\\
19.33	0.01\\
19.34	0.01\\
19.35	0.01\\
19.36	0.01\\
19.37	0.01\\
19.38	0.01\\
19.39	0.01\\
19.4	0.01\\
19.41	0.01\\
19.42	0.01\\
19.43	0.01\\
19.44	0.01\\
19.45	0.01\\
19.46	0.01\\
19.47	0.01\\
19.48	0.01\\
19.49	0.01\\
19.5	0.01\\
19.51	0.01\\
19.52	0.01\\
19.53	0.01\\
19.54	0.01\\
19.55	0.01\\
19.56	0.01\\
19.57	0.01\\
19.58	0.01\\
19.59	0.01\\
19.6	0.01\\
19.61	0.01\\
19.62	0.01\\
19.63	0.01\\
19.64	0.01\\
19.65	0.01\\
19.66	0.01\\
19.67	0.01\\
19.68	0.01\\
19.69	0.01\\
19.7	0.01\\
19.71	0.01\\
19.72	0.01\\
19.73	0.01\\
19.74	0.01\\
19.75	0.01\\
19.76	0.01\\
19.77	0.01\\
19.78	0.01\\
19.79	0.01\\
19.8	0.01\\
19.81	0.01\\
19.82	0.01\\
19.83	0.01\\
19.84	0.01\\
19.85	0.01\\
19.86	0.01\\
19.87	0.01\\
19.88	0.01\\
19.89	0.01\\
19.9	0.01\\
19.91	0.01\\
19.92	0.01\\
19.93	0.01\\
19.94	0.01\\
19.95	0.01\\
19.96	0.01\\
19.97	0.01\\
19.98	0.01\\
19.99	0.01\\
20	0.01\\
20.01	0.01\\
20.02	0.01\\
20.03	0.01\\
20.04	0.01\\
20.05	0.01\\
20.06	0.01\\
20.07	0.01\\
20.08	0.01\\
20.09	0.01\\
20.1	0.01\\
20.11	0.01\\
20.12	0.01\\
20.13	0.01\\
20.14	0.01\\
20.15	0.01\\
20.16	0.01\\
20.17	0.01\\
20.18	0.01\\
20.19	0.01\\
20.2	0.01\\
20.21	0.01\\
20.22	0.01\\
20.23	0.01\\
20.24	0.01\\
20.25	0.01\\
20.26	0.01\\
20.27	0.01\\
20.28	0.01\\
20.29	0.01\\
20.3	0.01\\
20.31	0.01\\
20.32	0.01\\
20.33	0.01\\
20.34	0.01\\
20.35	0.01\\
20.36	0.01\\
20.37	0.01\\
20.38	0.01\\
20.39	0.01\\
20.4	0.01\\
20.41	0.01\\
20.42	0.01\\
20.43	0.01\\
20.44	0.01\\
20.45	0.01\\
20.46	0.01\\
20.47	0.01\\
20.48	0.01\\
20.49	0.01\\
20.5	0.01\\
20.51	0.01\\
20.52	0.01\\
20.53	0.01\\
20.54	0.01\\
20.55	0.01\\
20.56	0.01\\
20.57	0.01\\
20.58	0.01\\
20.59	0.01\\
20.6	0.01\\
20.61	0.01\\
20.62	0.01\\
20.63	0.01\\
20.64	0.01\\
20.65	0.01\\
20.66	0.01\\
20.67	0.01\\
20.68	0.01\\
20.69	0.01\\
20.7	0.01\\
20.71	0.01\\
20.72	0.01\\
20.73	0.01\\
20.74	0.01\\
20.75	0.01\\
20.76	0.01\\
20.77	0.01\\
20.78	0.01\\
20.79	0.01\\
20.8	0.01\\
20.81	0.01\\
20.82	0.01\\
20.83	0.01\\
20.84	0.01\\
20.85	0.01\\
20.86	0.01\\
20.87	0.01\\
20.88	0.01\\
20.89	0.01\\
20.9	0.01\\
20.91	0.01\\
20.92	0.01\\
20.93	0.01\\
20.94	0.01\\
20.95	0.01\\
20.96	0.01\\
20.97	0.01\\
20.98	0.01\\
20.99	0.01\\
21	0.01\\
21.01	0.01\\
21.02	0.01\\
21.03	0.01\\
21.04	0.01\\
21.05	0.01\\
21.06	0.01\\
21.07	0.01\\
21.08	0.01\\
21.09	0.01\\
21.1	0.01\\
21.11	0.01\\
21.12	0.01\\
21.13	0.01\\
21.14	0.01\\
21.15	0.01\\
21.16	0.01\\
21.17	0.01\\
21.18	0.01\\
21.19	0.01\\
21.2	0.01\\
21.21	0.01\\
21.22	0.01\\
21.23	0.01\\
21.24	0.01\\
21.25	0.01\\
21.26	0.01\\
21.27	0.01\\
21.28	0.01\\
21.29	0.01\\
21.3	0.01\\
21.31	0.01\\
21.32	0.01\\
21.33	0.01\\
21.34	0.01\\
21.35	0.01\\
21.36	0.01\\
21.37	0.01\\
21.38	0.01\\
21.39	0.01\\
21.4	0.01\\
21.41	0.01\\
21.42	0.01\\
21.43	0.01\\
21.44	0.01\\
21.45	0.01\\
21.46	0.01\\
21.47	0.01\\
21.48	0.01\\
21.49	0.01\\
21.5	0.01\\
21.51	0.01\\
21.52	0.01\\
21.53	0.01\\
21.54	0.01\\
21.55	0.01\\
21.56	0.01\\
21.57	0.01\\
21.58	0.01\\
21.59	0.01\\
21.6	0.01\\
21.61	0.01\\
21.62	0.01\\
21.63	0.01\\
21.64	0.01\\
21.65	0.01\\
21.66	0.01\\
21.67	0.01\\
21.68	0.01\\
21.69	0.01\\
21.7	0.01\\
21.71	0.01\\
21.72	0.01\\
21.73	0.01\\
21.74	0.01\\
21.75	0.01\\
21.76	0.01\\
21.77	0.01\\
21.78	0.01\\
21.79	0.01\\
21.8	0.01\\
21.81	0.01\\
21.82	0.01\\
21.83	0.01\\
21.84	0.01\\
21.85	0.01\\
21.86	0.01\\
21.87	0.01\\
21.88	0.01\\
21.89	0.01\\
21.9	0.01\\
21.91	0.01\\
21.92	0.01\\
21.93	0.01\\
21.94	0.01\\
21.95	0.01\\
21.96	0.01\\
21.97	0.01\\
21.98	0.01\\
21.99	0.01\\
22	0.01\\
22.01	0.01\\
22.02	0.01\\
22.03	0.01\\
22.04	0.01\\
22.05	0.01\\
22.06	0.01\\
22.07	0.01\\
22.08	0.01\\
22.09	0.01\\
22.1	0.01\\
22.11	0.01\\
22.12	0.01\\
22.13	0.01\\
22.14	0.01\\
22.15	0.01\\
22.16	0.01\\
22.17	0.01\\
22.18	0.01\\
22.19	0.01\\
22.2	0.01\\
22.21	0.01\\
22.22	0.01\\
22.23	0.01\\
22.24	0.01\\
22.25	0.01\\
22.26	0.01\\
22.27	0.01\\
22.28	0.01\\
22.29	0.01\\
22.3	0.01\\
22.31	0.01\\
22.32	0.01\\
22.33	0.01\\
22.34	0.01\\
22.35	0.01\\
22.36	0.01\\
22.37	0.01\\
22.38	0.01\\
22.39	0.01\\
22.4	0.01\\
22.41	0.01\\
22.42	0.01\\
22.43	0.01\\
22.44	0.01\\
22.45	0.01\\
22.46	0.01\\
22.47	0.01\\
22.48	0.01\\
22.49	0.01\\
22.5	0.01\\
22.51	0.01\\
22.52	0.01\\
22.53	0.01\\
22.54	0.01\\
22.55	0.01\\
22.56	0.01\\
22.57	0.01\\
22.58	0.01\\
22.59	0.01\\
22.6	0.01\\
22.61	0.01\\
22.62	0.01\\
22.63	0.01\\
22.64	0.01\\
22.65	0.01\\
22.66	0.01\\
22.67	0.01\\
22.68	0.01\\
22.69	0.01\\
22.7	0.01\\
22.71	0.01\\
22.72	0.01\\
22.73	0.01\\
22.74	0.01\\
22.75	0.01\\
22.76	0.01\\
22.77	0.01\\
22.78	0.01\\
22.79	0.01\\
22.8	0.01\\
22.81	0.01\\
22.82	0.01\\
22.83	0.01\\
22.84	0.01\\
22.85	0.01\\
22.86	0.01\\
22.87	0.01\\
22.88	0.01\\
22.89	0.01\\
22.9	0.01\\
22.91	0.01\\
22.92	0.01\\
22.93	0.01\\
22.94	0.01\\
22.95	0.01\\
22.96	0.01\\
22.97	0.01\\
22.98	0.01\\
22.99	0.01\\
23	0.01\\
23.01	0.01\\
23.02	0.01\\
23.03	0.01\\
23.04	0.01\\
23.05	0.01\\
23.06	0.01\\
23.07	0.01\\
23.08	0.01\\
23.09	0.01\\
23.1	0.01\\
23.11	0.01\\
23.12	0.01\\
23.13	0.01\\
23.14	0.01\\
23.15	0.01\\
23.16	0.01\\
23.17	0.01\\
23.18	0.01\\
23.19	0.01\\
23.2	0.01\\
23.21	0.01\\
23.22	0.01\\
23.23	0.01\\
23.24	0.01\\
23.25	0.01\\
23.26	0.01\\
23.27	0.01\\
23.28	0.01\\
23.29	0.01\\
23.3	0.01\\
23.31	0.01\\
23.32	0.01\\
23.33	0.01\\
23.34	0.01\\
23.35	0.01\\
23.36	0.01\\
23.37	0.01\\
23.38	0.01\\
23.39	0.01\\
23.4	0.01\\
23.41	0.01\\
23.42	0.01\\
23.43	0.01\\
23.44	0.01\\
23.45	0.01\\
23.46	0.01\\
23.47	0.01\\
23.48	0.01\\
23.49	0.01\\
23.5	0.01\\
23.51	0.01\\
23.52	0.01\\
23.53	0.01\\
23.54	0.01\\
23.55	0.01\\
23.56	0.01\\
23.57	0.01\\
23.58	0.01\\
23.59	0.01\\
23.6	0.01\\
23.61	0.01\\
23.62	0.01\\
23.63	0.01\\
23.64	0.01\\
23.65	0.01\\
23.66	0.01\\
23.67	0.01\\
23.68	0.01\\
23.69	0.01\\
23.7	0.01\\
23.71	0.01\\
23.72	0.01\\
23.73	0.01\\
23.74	0.01\\
23.75	0.01\\
23.76	0.01\\
23.77	0.01\\
23.78	0.01\\
23.79	0.01\\
23.8	0.01\\
23.81	0.01\\
23.82	0.01\\
23.83	0.01\\
23.84	0.01\\
23.85	0.01\\
23.86	0.01\\
23.87	0.01\\
23.88	0.01\\
23.89	0.01\\
23.9	0.01\\
23.91	0.01\\
23.92	0.01\\
23.93	0.01\\
23.94	0.01\\
23.95	0.01\\
23.96	0.01\\
23.97	0.01\\
23.98	0.01\\
23.99	0.01\\
24	0.01\\
24.01	0.01\\
24.02	0.01\\
24.03	0.01\\
24.04	0.01\\
24.05	0.01\\
24.06	0.01\\
24.07	0.01\\
24.08	0.01\\
24.09	0.01\\
24.1	0.01\\
24.11	0.01\\
24.12	0.01\\
24.13	0.01\\
24.14	0.01\\
24.15	0.01\\
24.16	0.01\\
24.17	0.01\\
24.18	0.01\\
24.19	0.01\\
24.2	0.01\\
24.21	0.01\\
24.22	0.01\\
24.23	0.01\\
24.24	0.01\\
24.25	0.01\\
24.26	0.01\\
24.27	0.01\\
24.28	0.01\\
24.29	0.01\\
24.3	0.01\\
24.31	0.01\\
24.32	0.01\\
24.33	0.01\\
24.34	0.01\\
24.35	0.01\\
24.36	0.01\\
24.37	0.01\\
24.38	0.01\\
24.39	0.01\\
24.4	0.01\\
24.41	0.01\\
24.42	0.01\\
24.43	0.01\\
24.44	0.01\\
24.45	0.01\\
24.46	0.01\\
24.47	0.01\\
24.48	0.01\\
24.49	0.01\\
24.5	0.01\\
24.51	0.01\\
24.52	0.01\\
24.53	0.01\\
24.54	0.01\\
24.55	0.01\\
24.56	0.01\\
24.57	0.01\\
24.58	0.01\\
24.59	0.01\\
24.6	0.01\\
24.61	0.01\\
24.62	0.01\\
24.63	0.01\\
24.64	0.01\\
24.65	0.01\\
24.66	0.01\\
24.67	0.01\\
24.68	0.01\\
24.69	0.01\\
24.7	0.01\\
24.71	0.01\\
24.72	0.01\\
24.73	0.01\\
24.74	0.01\\
24.75	0.01\\
24.76	0.01\\
24.77	0.01\\
24.78	0.01\\
24.79	0.01\\
24.8	0.01\\
24.81	0.01\\
24.82	0.01\\
24.83	0.01\\
24.84	0.01\\
24.85	0.01\\
24.86	0.01\\
24.87	0.01\\
24.88	0.01\\
24.89	0.01\\
24.9	0.01\\
24.91	0.01\\
24.92	0.01\\
24.93	0.01\\
24.94	0.01\\
24.95	0.01\\
24.96	0.01\\
24.97	0.01\\
24.98	0.01\\
24.99	0.01\\
25	0.01\\
25.01	0.01\\
25.02	0.01\\
25.03	0.01\\
25.04	0.01\\
25.05	0.01\\
25.06	0.01\\
25.07	0.01\\
25.08	0.01\\
25.09	0.01\\
25.1	0.01\\
25.11	0.01\\
25.12	0.01\\
25.13	0.01\\
25.14	0.01\\
25.15	0.01\\
25.16	0.01\\
25.17	0.01\\
25.18	0.01\\
25.19	0.01\\
25.2	0.01\\
25.21	0.01\\
25.22	0.01\\
25.23	0.01\\
25.24	0.01\\
25.25	0.01\\
25.26	0.01\\
25.27	0.01\\
25.28	0.01\\
25.29	0.01\\
25.3	0.01\\
25.31	0.01\\
25.32	0.01\\
25.33	0.01\\
25.34	0.01\\
25.35	0.01\\
25.36	0.01\\
25.37	0.01\\
25.38	0.01\\
25.39	0.01\\
25.4	0.01\\
25.41	0.01\\
25.42	0.01\\
25.43	0.01\\
25.44	0.01\\
25.45	0.01\\
25.46	0.01\\
25.47	0.01\\
25.48	0.01\\
25.49	0.01\\
25.5	0.01\\
25.51	0.01\\
25.52	0.01\\
25.53	0.01\\
25.54	0.01\\
25.55	0.01\\
25.56	0.01\\
25.57	0.01\\
25.58	0.01\\
25.59	0.01\\
25.6	0.01\\
25.61	0.01\\
25.62	0.01\\
25.63	0.01\\
25.64	0.01\\
25.65	0.01\\
25.66	0.01\\
25.67	0.01\\
25.68	0.01\\
25.69	0.01\\
25.7	0.01\\
25.71	0.01\\
25.72	0.01\\
25.73	0.01\\
25.74	0.01\\
25.75	0.01\\
25.76	0.01\\
25.77	0.01\\
25.78	0.01\\
25.79	0.01\\
25.8	0.01\\
25.81	0.01\\
25.82	0.01\\
25.83	0.01\\
25.84	0.01\\
25.85	0.01\\
25.86	0.01\\
25.87	0.01\\
25.88	0.01\\
25.89	0.01\\
25.9	0.01\\
25.91	0.01\\
25.92	0.01\\
25.93	0.01\\
25.94	0.01\\
25.95	0.01\\
25.96	0.01\\
25.97	0.01\\
25.98	0.01\\
25.99	0.01\\
26	0.01\\
26.01	0.01\\
26.02	0.01\\
26.03	0.01\\
26.04	0.01\\
26.05	0.01\\
26.06	0.01\\
26.07	0.01\\
26.08	0.01\\
26.09	0.01\\
26.1	0.01\\
26.11	0.01\\
26.12	0.01\\
26.13	0.01\\
26.14	0.01\\
26.15	0.01\\
26.16	0.01\\
26.17	0.01\\
26.18	0.01\\
26.19	0.01\\
26.2	0.01\\
26.21	0.01\\
26.22	0.01\\
26.23	0.01\\
26.24	0.01\\
26.25	0.01\\
26.26	0.01\\
26.27	0.01\\
26.28	0.01\\
26.29	0.01\\
26.3	0.01\\
26.31	0.01\\
26.32	0.01\\
26.33	0.01\\
26.34	0.01\\
26.35	0.01\\
26.36	0.01\\
26.37	0.01\\
26.38	0.01\\
26.39	0.01\\
26.4	0.01\\
26.41	0.01\\
26.42	0.01\\
26.43	0.01\\
26.44	0.01\\
26.45	0.01\\
26.46	0.01\\
26.47	0.01\\
26.48	0.01\\
26.49	0.01\\
26.5	0.01\\
26.51	0.01\\
26.52	0.01\\
26.53	0.01\\
26.54	0.01\\
26.55	0.01\\
26.56	0.01\\
26.57	0.01\\
26.58	0.01\\
26.59	0.01\\
26.6	0.01\\
26.61	0.01\\
26.62	0.01\\
26.63	0.01\\
26.64	0.01\\
26.65	0.01\\
26.66	0.01\\
26.67	0.01\\
26.68	0.01\\
26.69	0.01\\
26.7	0.01\\
26.71	0.01\\
26.72	0.01\\
26.73	0.01\\
26.74	0.01\\
26.75	0.01\\
26.76	0.01\\
26.77	0.01\\
26.78	0.01\\
26.79	0.01\\
26.8	0.01\\
26.81	0.01\\
26.82	0.01\\
26.83	0.01\\
26.84	0.01\\
26.85	0.01\\
26.86	0.01\\
26.87	0.01\\
26.88	0.01\\
26.89	0.01\\
26.9	0.01\\
26.91	0.01\\
26.92	0.01\\
26.93	0.01\\
26.94	0.01\\
26.95	0.01\\
26.96	0.01\\
26.97	0.01\\
26.98	0.01\\
26.99	0.01\\
27	0.01\\
27.01	0.01\\
27.02	0.01\\
27.03	0.01\\
27.04	0.01\\
27.05	0.01\\
27.06	0.01\\
27.07	0.01\\
27.08	0.01\\
27.09	0.01\\
27.1	0.01\\
27.11	0.01\\
27.12	0.01\\
27.13	0.01\\
27.14	0.01\\
27.15	0.01\\
27.16	0.01\\
27.17	0.01\\
27.18	0.01\\
27.19	0.01\\
27.2	0.01\\
27.21	0.01\\
27.22	0.01\\
27.23	0.01\\
27.24	0.01\\
27.25	0.01\\
27.26	0.01\\
27.27	0.01\\
27.28	0.01\\
27.29	0.01\\
27.3	0.01\\
27.31	0.01\\
27.32	0.01\\
27.33	0.01\\
27.34	0.01\\
27.35	0.01\\
27.36	0.01\\
27.37	0.01\\
27.38	0.01\\
27.39	0.01\\
27.4	0.01\\
27.41	0.01\\
27.42	0.01\\
27.43	0.01\\
27.44	0.01\\
27.45	0.01\\
27.46	0.01\\
27.47	0.01\\
27.48	0.01\\
27.49	0.01\\
27.5	0.01\\
27.51	0.01\\
27.52	0.01\\
27.53	0.01\\
27.54	0.01\\
27.55	0.01\\
27.56	0.01\\
27.57	0.01\\
27.58	0.01\\
27.59	0.01\\
27.6	0.01\\
27.61	0.01\\
27.62	0.01\\
27.63	0.01\\
27.64	0.01\\
27.65	0.01\\
27.66	0.01\\
27.67	0.01\\
27.68	0.01\\
27.69	0.01\\
27.7	0.01\\
27.71	0.01\\
27.72	0.01\\
27.73	0.01\\
27.74	0.01\\
27.75	0.01\\
27.76	0.01\\
27.77	0.01\\
27.78	0.01\\
27.79	0.01\\
27.8	0.01\\
27.81	0.01\\
27.82	0.01\\
27.83	0.01\\
27.84	0.01\\
27.85	0.01\\
27.86	0.01\\
27.87	0.01\\
27.88	0.01\\
27.89	0.01\\
27.9	0.01\\
27.91	0.01\\
27.92	0.01\\
27.93	0.01\\
27.94	0.01\\
27.95	0.01\\
27.96	0.01\\
27.97	0.01\\
27.98	0.01\\
27.99	0.01\\
28	0.01\\
28.01	0.01\\
28.02	0.01\\
28.03	0.01\\
28.04	0.01\\
28.05	0.01\\
28.06	0.01\\
28.07	0.01\\
28.08	0.01\\
28.09	0.01\\
28.1	0.01\\
28.11	0.01\\
28.12	0.01\\
28.13	0.01\\
28.14	0.01\\
28.15	0.01\\
28.16	0.01\\
28.17	0.01\\
28.18	0.01\\
28.19	0.01\\
28.2	0.01\\
28.21	0.01\\
28.22	0.01\\
28.23	0.01\\
28.24	0.01\\
28.25	0.01\\
28.26	0.01\\
28.27	0.01\\
28.28	0.01\\
28.29	0.01\\
28.3	0.01\\
28.31	0.01\\
28.32	0.01\\
28.33	0.01\\
28.34	0.01\\
28.35	0.01\\
28.36	0.01\\
28.37	0.01\\
28.38	0.01\\
28.39	0.01\\
28.4	0.01\\
28.41	0.01\\
28.42	0.01\\
28.43	0.01\\
28.44	0.01\\
28.45	0.01\\
28.46	0.01\\
28.47	0.01\\
28.48	0.01\\
28.49	0.01\\
28.5	0.01\\
28.51	0.01\\
28.52	0.01\\
28.53	0.01\\
28.54	0.01\\
28.55	0.01\\
28.56	0.01\\
28.57	0.01\\
28.58	0.01\\
28.59	0.01\\
28.6	0.01\\
28.61	0.01\\
28.62	0.01\\
28.63	0.01\\
28.64	0.01\\
28.65	0.01\\
28.66	0.01\\
28.67	0.01\\
28.68	0.01\\
28.69	0.01\\
28.7	0.01\\
28.71	0.01\\
28.72	0.01\\
28.73	0.01\\
28.74	0.01\\
28.75	0.01\\
28.76	0.01\\
28.77	0.01\\
28.78	0.01\\
28.79	0.01\\
28.8	0.01\\
28.81	0.01\\
28.82	0.01\\
28.83	0.01\\
28.84	0.01\\
28.85	0.01\\
28.86	0.01\\
28.87	0.01\\
28.88	0.01\\
28.89	0.01\\
28.9	0.01\\
28.91	0.01\\
28.92	0.01\\
28.93	0.01\\
28.94	0.01\\
28.95	0.01\\
28.96	0.01\\
28.97	0.01\\
28.98	0.01\\
28.99	0.01\\
29	0.01\\
29.01	0.01\\
29.02	0.01\\
29.03	0.01\\
29.04	0.01\\
29.05	0.01\\
29.06	0.01\\
29.07	0.01\\
29.08	0.01\\
29.09	0.01\\
29.1	0.01\\
29.11	0.01\\
29.12	0.01\\
29.13	0.01\\
29.14	0.01\\
29.15	0.01\\
29.16	0.01\\
29.17	0.01\\
29.18	0.01\\
29.19	0.01\\
29.2	0.01\\
29.21	0.01\\
29.22	0.01\\
29.23	0.01\\
29.24	0.01\\
29.25	0.01\\
29.26	0.01\\
29.27	0.01\\
29.28	0.01\\
29.29	0.01\\
29.3	0.01\\
29.31	0.01\\
29.32	0.01\\
29.33	0.01\\
29.34	0.01\\
29.35	0.01\\
29.36	0.01\\
29.37	0.01\\
29.38	0.01\\
29.39	0.01\\
29.4	0.01\\
29.41	0.01\\
29.42	0.01\\
29.43	0.01\\
29.44	0.01\\
29.45	0.01\\
29.46	0.01\\
29.47	0.01\\
29.48	0.01\\
29.49	0.01\\
29.5	0.01\\
29.51	0.01\\
29.52	0.01\\
29.53	0.01\\
29.54	0.01\\
29.55	0.01\\
29.56	0.01\\
29.57	0.01\\
29.58	0.01\\
29.59	0.01\\
29.6	0.01\\
29.61	0.01\\
29.62	0.01\\
29.63	0.01\\
29.64	0.01\\
29.65	0.01\\
29.66	0.01\\
29.67	0.01\\
29.68	0.01\\
29.69	0.01\\
29.7	0.01\\
29.71	0.01\\
29.72	0.01\\
29.73	0.01\\
29.74	0.01\\
29.75	0.01\\
29.76	0.01\\
29.77	0.01\\
29.78	0.01\\
29.79	0.01\\
29.8	0.01\\
29.81	0.01\\
29.82	0.01\\
29.83	0.01\\
29.84	0.01\\
29.85	0.01\\
29.86	0.01\\
29.87	0.01\\
29.88	0.01\\
29.89	0.01\\
29.9	0.01\\
29.91	0.01\\
29.92	0.01\\
29.93	0.01\\
29.94	0.01\\
29.95	0.01\\
29.96	0.01\\
29.97	0.01\\
29.98	0.01\\
29.99	0.01\\
30	0.01\\
30.01	0.01\\
30.02	0.01\\
30.03	0.01\\
30.04	0.01\\
30.05	0.01\\
30.06	0.01\\
30.07	0.01\\
30.08	0.01\\
30.09	0.01\\
30.1	0.01\\
30.11	0.01\\
30.12	0.01\\
30.13	0.01\\
30.14	0.01\\
30.15	0.01\\
30.16	0.01\\
30.17	0.01\\
30.18	0.01\\
30.19	0.01\\
30.2	0.01\\
30.21	0.01\\
30.22	0.01\\
30.23	0.01\\
30.24	0.01\\
30.25	0.01\\
30.26	0.01\\
30.27	0.01\\
30.28	0.01\\
30.29	0.01\\
30.3	0.01\\
30.31	0.01\\
30.32	0.01\\
30.33	0.01\\
30.34	0.01\\
30.35	0.01\\
30.36	0.01\\
30.37	0.01\\
30.38	0.01\\
30.39	0.01\\
30.4	0.01\\
30.41	0.01\\
30.42	0.01\\
30.43	0.01\\
30.44	0.01\\
30.45	0.01\\
30.46	0.01\\
30.47	0.01\\
30.48	0.01\\
30.49	0.01\\
30.5	0.01\\
30.51	0.01\\
30.52	0.01\\
30.53	0.01\\
30.54	0.01\\
30.55	0.01\\
30.56	0.01\\
30.57	0.01\\
30.58	0.01\\
30.59	0.01\\
30.6	0.01\\
30.61	0.01\\
30.62	0.01\\
30.63	0.01\\
30.64	0.01\\
30.65	0.01\\
30.66	0.01\\
30.67	0.01\\
30.68	0.01\\
30.69	0.01\\
30.7	0.01\\
30.71	0.01\\
30.72	0.01\\
30.73	0.01\\
30.74	0.01\\
30.75	0.01\\
30.76	0.01\\
30.77	0.01\\
30.78	0.01\\
30.79	0.01\\
30.8	0.01\\
30.81	0.01\\
30.82	0.01\\
30.83	0.01\\
30.84	0.01\\
30.85	0.01\\
30.86	0.01\\
30.87	0.01\\
30.88	0.01\\
30.89	0.01\\
30.9	0.01\\
30.91	0.01\\
30.92	0.01\\
30.93	0.01\\
30.94	0.01\\
30.95	0.01\\
30.96	0.01\\
30.97	0.01\\
30.98	0.01\\
30.99	0.01\\
31	0.01\\
31.01	0.01\\
31.02	0.01\\
31.03	0.01\\
31.04	0.01\\
31.05	0.01\\
31.06	0.01\\
31.07	0.01\\
31.08	0.01\\
31.09	0.01\\
31.1	0.01\\
31.11	0.01\\
31.12	0.01\\
31.13	0.01\\
31.14	0.01\\
31.15	0.01\\
31.16	0.01\\
31.17	0.01\\
31.18	0.01\\
31.19	0.01\\
31.2	0.01\\
31.21	0.01\\
31.22	0.01\\
31.23	0.01\\
31.24	0.01\\
31.25	0.01\\
31.26	0.01\\
31.27	0.01\\
31.28	0.01\\
31.29	0.01\\
31.3	0.01\\
31.31	0.01\\
31.32	0.01\\
31.33	0.01\\
31.34	0.01\\
31.35	0.01\\
31.36	0.01\\
31.37	0.01\\
31.38	0.01\\
31.39	0.01\\
31.4	0.01\\
31.41	0.01\\
31.42	0.01\\
31.43	0.01\\
31.44	0.01\\
31.45	0.01\\
31.46	0.01\\
31.47	0.01\\
31.48	0.01\\
31.49	0.01\\
31.5	0.01\\
31.51	0.01\\
31.52	0.01\\
31.53	0.01\\
31.54	0.01\\
31.55	0.01\\
31.56	0.01\\
31.57	0.01\\
31.58	0.01\\
31.59	0.01\\
31.6	0.01\\
31.61	0.01\\
31.62	0.01\\
31.63	0.01\\
31.64	0.01\\
31.65	0.01\\
31.66	0.01\\
31.67	0.01\\
31.68	0.01\\
31.69	0.01\\
31.7	0.01\\
31.71	0.01\\
31.72	0.01\\
31.73	0.01\\
31.74	0.01\\
31.75	0.01\\
31.76	0.01\\
31.77	0.01\\
31.78	0.01\\
31.79	0.01\\
31.8	0.01\\
31.81	0.01\\
31.82	0.01\\
31.83	0.01\\
31.84	0.01\\
31.85	0.01\\
31.86	0.01\\
31.87	0.01\\
31.88	0.01\\
31.89	0.01\\
31.9	0.01\\
31.91	0.01\\
31.92	0.01\\
31.93	0.01\\
31.94	0.01\\
31.95	0.01\\
31.96	0.01\\
31.97	0.01\\
31.98	0.01\\
31.99	0.01\\
32	0.01\\
32.01	0.01\\
32.02	0.01\\
32.03	0.01\\
32.04	0.01\\
32.05	0.01\\
32.06	0.01\\
32.07	0.01\\
32.08	0.01\\
32.09	0.01\\
32.1	0.01\\
32.11	0.01\\
32.12	0.01\\
32.13	0.01\\
32.14	0.01\\
32.15	0.01\\
32.16	0.01\\
32.17	0.01\\
32.18	0.01\\
32.19	0.01\\
32.2	0.01\\
32.21	0.01\\
32.22	0.01\\
32.23	0.01\\
32.24	0.01\\
32.25	0.01\\
32.26	0.01\\
32.27	0.01\\
32.28	0.01\\
32.29	0.01\\
32.3	0.01\\
32.31	0.01\\
32.32	0.01\\
32.33	0.01\\
32.34	0.01\\
32.35	0.01\\
32.36	0.01\\
32.37	0.01\\
32.38	0.01\\
32.39	0.01\\
32.4	0.01\\
32.41	0.01\\
32.42	0.01\\
32.43	0.01\\
32.44	0.01\\
32.45	0.01\\
32.46	0.01\\
32.47	0.01\\
32.48	0.01\\
32.49	0.01\\
32.5	0.01\\
32.51	0.01\\
32.52	0.01\\
32.53	0.01\\
32.54	0.01\\
32.55	0.01\\
32.56	0.01\\
32.57	0.01\\
32.58	0.01\\
32.59	0.01\\
32.6	0.01\\
32.61	0.01\\
32.62	0.01\\
32.63	0.01\\
32.64	0.01\\
32.65	0.01\\
32.66	0.01\\
32.67	0.01\\
32.68	0.01\\
32.69	0.01\\
32.7	0.01\\
32.71	0.01\\
32.72	0.01\\
32.73	0.01\\
32.74	0.01\\
32.75	0.01\\
32.76	0.01\\
32.77	0.01\\
32.78	0.01\\
32.79	0.01\\
32.8	0.01\\
32.81	0.01\\
32.82	0.01\\
32.83	0.01\\
32.84	0.01\\
32.85	0.01\\
32.86	0.01\\
32.87	0.01\\
32.88	0.01\\
32.89	0.01\\
32.9	0.01\\
32.91	0.01\\
32.92	0.01\\
32.93	0.01\\
32.94	0.01\\
32.95	0.01\\
32.96	0.01\\
32.97	0.01\\
32.98	0.01\\
32.99	0.01\\
33	0.01\\
33.01	0.01\\
33.02	0.01\\
33.03	0.01\\
33.04	0.01\\
33.05	0.01\\
33.06	0.01\\
33.07	0.01\\
33.08	0.01\\
33.09	0.01\\
33.1	0.01\\
33.11	0.01\\
33.12	0.01\\
33.13	0.01\\
33.14	0.01\\
33.15	0.01\\
33.16	0.01\\
33.17	0.01\\
33.18	0.01\\
33.19	0.01\\
33.2	0.01\\
33.21	0.01\\
33.22	0.01\\
33.23	0.01\\
33.24	0.01\\
33.25	0.01\\
33.26	0.01\\
33.27	0.01\\
33.28	0.01\\
33.29	0.01\\
33.3	0.01\\
33.31	0.01\\
33.32	0.01\\
33.33	0.01\\
33.34	0.01\\
33.35	0.01\\
33.36	0.01\\
33.37	0.01\\
33.38	0.01\\
33.39	0.01\\
33.4	0.01\\
33.41	0.01\\
33.42	0.01\\
33.43	0.01\\
33.44	0.01\\
33.45	0.01\\
33.46	0.01\\
33.47	0.01\\
33.48	0.01\\
33.49	0.01\\
33.5	0.01\\
33.51	0.01\\
33.52	0.01\\
33.53	0.01\\
33.54	0.01\\
33.55	0.01\\
33.56	0.01\\
33.57	0.01\\
33.58	0.01\\
33.59	0.01\\
33.6	0.01\\
33.61	0.01\\
33.62	0.01\\
33.63	0.01\\
33.64	0.01\\
33.65	0.01\\
33.66	0.01\\
33.67	0.01\\
33.68	0.01\\
33.69	0.01\\
33.7	0.01\\
33.71	0.01\\
33.72	0.01\\
33.73	0.01\\
33.74	0.01\\
33.75	0.01\\
33.76	0.01\\
33.77	0.01\\
33.78	0.01\\
33.79	0.01\\
33.8	0.01\\
33.81	0.01\\
33.82	0.01\\
33.83	0.01\\
33.84	0.01\\
33.85	0.01\\
33.86	0.01\\
33.87	0.01\\
33.88	0.01\\
33.89	0.01\\
33.9	0.01\\
33.91	0.01\\
33.92	0.01\\
33.93	0.01\\
33.94	0.01\\
33.95	0.01\\
33.96	0.01\\
33.97	0.01\\
33.98	0.01\\
33.99	0.01\\
34	0.01\\
34.01	0.01\\
34.02	0.01\\
34.03	0.01\\
34.04	0.01\\
34.05	0.01\\
34.06	0.01\\
34.07	0.01\\
34.08	0.01\\
34.09	0.01\\
34.1	0.01\\
34.11	0.01\\
34.12	0.01\\
34.13	0.01\\
34.14	0.01\\
34.15	0.01\\
34.16	0.01\\
34.17	0.01\\
34.18	0.01\\
34.19	0.01\\
34.2	0.01\\
34.21	0.01\\
34.22	0.01\\
34.23	0.01\\
34.24	0.01\\
34.25	0.01\\
34.26	0.01\\
34.27	0.01\\
34.28	0.01\\
34.29	0.01\\
34.3	0.01\\
34.31	0.01\\
34.32	0.01\\
34.33	0.01\\
34.34	0.01\\
34.35	0.01\\
34.36	0.01\\
34.37	0.01\\
34.38	0.01\\
34.39	0.01\\
34.4	0.01\\
34.41	0.01\\
34.42	0.01\\
34.43	0.01\\
34.44	0.01\\
34.45	0.01\\
34.46	0.01\\
34.47	0.01\\
34.48	0.01\\
34.49	0.01\\
34.5	0.01\\
34.51	0.01\\
34.52	0.01\\
34.53	0.01\\
34.54	0.01\\
34.55	0.01\\
34.56	0.01\\
34.57	0.01\\
34.58	0.01\\
34.59	0.01\\
34.6	0.01\\
34.61	0.01\\
34.62	0.01\\
34.63	0.01\\
34.64	0.01\\
34.65	0.01\\
34.66	0.01\\
34.67	0.01\\
34.68	0.01\\
34.69	0.01\\
34.7	0.01\\
34.71	0.01\\
34.72	0.01\\
34.73	0.01\\
34.74	0.01\\
34.75	0.01\\
34.76	0.01\\
34.77	0.01\\
34.78	0.01\\
34.79	0.01\\
34.8	0.01\\
34.81	0.01\\
34.82	0.01\\
34.83	0.01\\
34.84	0.01\\
34.85	0.01\\
34.86	0.01\\
34.87	0.01\\
34.88	0.01\\
34.89	0.01\\
34.9	0.01\\
34.91	0.01\\
34.92	0.01\\
34.93	0.01\\
34.94	0.01\\
34.95	0.01\\
34.96	0.01\\
34.97	0.01\\
34.98	0.01\\
34.99	0.01\\
35	0.01\\
35.01	0.01\\
35.02	0.01\\
35.03	0.01\\
35.04	0.01\\
35.05	0.01\\
35.06	0.01\\
35.07	0.01\\
35.08	0.01\\
35.09	0.01\\
35.1	0.01\\
35.11	0.01\\
35.12	0.01\\
35.13	0.01\\
35.14	0.01\\
35.15	0.01\\
35.16	0.01\\
35.17	0.01\\
35.18	0.01\\
35.19	0.01\\
35.2	0.01\\
35.21	0.01\\
35.22	0.01\\
35.23	0.01\\
35.24	0.01\\
35.25	0.01\\
35.26	0.01\\
35.27	0.01\\
35.28	0.01\\
35.29	0.01\\
35.3	0.01\\
35.31	0.01\\
35.32	0.01\\
35.33	0.01\\
35.34	0.01\\
35.35	0.01\\
35.36	0.01\\
35.37	0.01\\
35.38	0.01\\
35.39	0.01\\
35.4	0.01\\
35.41	0.01\\
35.42	0.01\\
35.43	0.01\\
35.44	0.01\\
35.45	0.01\\
35.46	0.01\\
35.47	0.01\\
35.48	0.01\\
35.49	0.01\\
35.5	0.01\\
35.51	0.01\\
35.52	0.01\\
35.53	0.01\\
35.54	0.01\\
35.55	0.01\\
35.56	0.01\\
35.57	0.01\\
35.58	0.01\\
35.59	0.01\\
35.6	0.01\\
35.61	0.01\\
35.62	0.01\\
35.63	0.01\\
35.64	0.01\\
35.65	0.01\\
35.66	0.01\\
35.67	0.01\\
35.68	0.01\\
35.69	0.01\\
35.7	0.01\\
35.71	0.01\\
35.72	0.01\\
35.73	0.01\\
35.74	0.01\\
35.75	0.01\\
35.76	0.01\\
35.77	0.01\\
35.78	0.01\\
35.79	0.01\\
35.8	0.01\\
35.81	0.01\\
35.82	0.01\\
35.83	0.01\\
35.84	0.01\\
35.85	0.01\\
35.86	0.01\\
35.87	0.01\\
35.88	0.01\\
35.89	0.01\\
35.9	0.01\\
35.91	0.01\\
35.92	0.01\\
35.93	0.01\\
35.94	0.01\\
35.95	0.01\\
35.96	0.01\\
35.97	0.01\\
35.98	0.01\\
35.99	0.01\\
36	0.01\\
36.01	0.01\\
36.02	0.01\\
36.03	0.01\\
36.04	0.01\\
36.05	0.01\\
36.06	0.01\\
36.07	0.01\\
36.08	0.01\\
36.09	0.01\\
36.1	0.01\\
36.11	0.01\\
36.12	0.01\\
36.13	0.01\\
36.14	0.01\\
36.15	0.01\\
36.16	0.01\\
36.17	0.01\\
36.18	0.01\\
36.19	0.01\\
36.2	0.01\\
36.21	0.01\\
36.22	0.01\\
36.23	0.01\\
36.24	0.01\\
36.25	0.01\\
36.26	0.01\\
36.27	0.01\\
36.28	0.01\\
36.29	0.01\\
36.3	0.01\\
36.31	0.01\\
36.32	0.01\\
36.33	0.01\\
36.34	0.01\\
36.35	0.01\\
36.36	0.01\\
36.37	0.01\\
36.38	0.01\\
36.39	0.01\\
36.4	0.01\\
36.41	0.01\\
36.42	0.01\\
36.43	0.01\\
36.44	0.01\\
36.45	0.01\\
36.46	0.01\\
36.47	0.01\\
36.48	0.01\\
36.49	0.01\\
36.5	0.01\\
36.51	0.01\\
36.52	0.01\\
36.53	0.01\\
36.54	0.01\\
36.55	0.01\\
36.56	0.01\\
36.57	0.01\\
36.58	0.01\\
36.59	0.01\\
36.6	0.01\\
36.61	0.01\\
36.62	0.01\\
36.63	0.01\\
36.64	0.01\\
36.65	0.01\\
36.66	0.01\\
36.67	0.01\\
36.68	0.01\\
36.69	0.01\\
36.7	0.01\\
36.71	0.01\\
36.72	0.01\\
36.73	0.01\\
36.74	0.01\\
36.75	0.01\\
36.76	0.01\\
36.77	0.01\\
36.78	0.01\\
36.79	0.01\\
36.8	0.01\\
36.81	0.01\\
36.82	0.01\\
36.83	0.01\\
36.84	0.01\\
36.85	0.01\\
36.86	0.01\\
36.87	0.01\\
36.88	0.01\\
36.89	0.01\\
36.9	0.01\\
36.91	0.01\\
36.92	0.01\\
36.93	0.01\\
36.94	0.01\\
36.95	0.01\\
36.96	0.01\\
36.97	0.01\\
36.98	0.01\\
36.99	0.01\\
37	0.01\\
37.01	0.01\\
37.02	0.01\\
37.03	0.01\\
37.04	0.01\\
37.05	0.01\\
37.06	0.01\\
37.07	0.01\\
37.08	0.01\\
37.09	0.01\\
37.1	0.01\\
37.11	0.01\\
37.12	0.01\\
37.13	0.01\\
37.14	0.01\\
37.15	0.01\\
37.16	0.01\\
37.17	0.01\\
37.18	0.01\\
37.19	0.01\\
37.2	0.01\\
37.21	0.01\\
37.22	0.01\\
37.23	0.01\\
37.24	0.01\\
37.25	0.01\\
37.26	0.01\\
37.27	0.01\\
37.28	0.01\\
37.29	0.01\\
37.3	0.01\\
37.31	0.01\\
37.32	0.01\\
37.33	0.01\\
37.34	0.01\\
37.35	0.01\\
37.36	0.01\\
37.37	0.01\\
37.38	0.01\\
37.39	0.01\\
37.4	0.01\\
37.41	0.01\\
37.42	0.01\\
37.43	0.01\\
37.44	0.01\\
37.45	0.01\\
37.46	0.01\\
37.47	0.01\\
37.48	0.01\\
37.49	0.01\\
37.5	0.01\\
37.51	0.01\\
37.52	0.01\\
37.53	0.01\\
37.54	0.01\\
37.55	0.01\\
37.56	0.01\\
37.57	0.01\\
37.58	0.01\\
37.59	0.01\\
37.6	0.01\\
37.61	0.01\\
37.62	0.01\\
37.63	0.01\\
37.64	0.01\\
37.65	0.01\\
37.66	0.01\\
37.67	0.01\\
37.68	0.01\\
37.69	0.01\\
37.7	0.01\\
37.71	0.01\\
37.72	0.01\\
37.73	0.01\\
37.74	0.01\\
37.75	0.01\\
37.76	0.01\\
37.77	0.01\\
37.78	0.01\\
37.79	0.01\\
37.8	0.01\\
37.81	0.01\\
37.82	0.01\\
37.83	0.01\\
37.84	0.01\\
37.85	0.01\\
37.86	0.01\\
37.87	0.01\\
37.88	0.01\\
37.89	0.01\\
37.9	0.01\\
37.91	0.01\\
37.92	0.01\\
37.93	0.01\\
37.94	0.01\\
37.95	0.01\\
37.96	0.01\\
37.97	0.01\\
37.98	0.01\\
37.99	0.01\\
38	0.01\\
38.01	0.01\\
38.02	0.01\\
38.03	0.01\\
38.04	0.01\\
38.05	0.01\\
38.06	0.01\\
38.07	0.01\\
38.08	0.01\\
38.09	0.01\\
38.1	0.01\\
38.11	0.01\\
38.12	0.01\\
38.13	0.01\\
38.14	0.01\\
38.15	0.01\\
38.16	0.01\\
38.17	0.01\\
38.18	0.01\\
38.19	0.01\\
38.2	0.01\\
38.21	0.01\\
38.22	0.01\\
38.23	0.01\\
38.24	0.01\\
38.25	0.01\\
38.26	0.01\\
38.27	0.01\\
38.28	0.01\\
38.29	0.01\\
38.3	0.01\\
38.31	0.01\\
38.32	0.01\\
38.33	0.01\\
38.34	0.01\\
38.35	0.01\\
38.36	0.01\\
38.37	0.01\\
38.38	0.01\\
38.39	0.01\\
38.4	0.01\\
38.41	0.01\\
38.42	0.01\\
38.43	0.01\\
38.44	0.01\\
38.45	0.01\\
38.46	0.01\\
38.47	0.01\\
38.48	0.01\\
38.49	0.01\\
38.5	0.01\\
38.51	0.01\\
38.52	0.01\\
38.53	0.01\\
38.54	0.01\\
38.55	0.01\\
38.56	0.01\\
38.57	0.01\\
38.58	0.01\\
38.59	0.01\\
38.6	0.01\\
38.61	0.01\\
38.62	0.01\\
38.63	0.01\\
38.64	0.01\\
38.65	0.01\\
38.66	0.01\\
38.67	0.01\\
38.68	0.01\\
38.69	0.01\\
38.7	0.01\\
38.71	0.01\\
38.72	0.01\\
38.73	0.01\\
38.74	0.01\\
38.75	0.01\\
38.76	0.01\\
38.77	0.01\\
38.78	0.01\\
38.79	0.01\\
38.8	0.01\\
38.81	0.01\\
38.82	0.01\\
38.83	0.01\\
38.84	0.01\\
38.85	0.01\\
38.86	0.01\\
38.87	0.01\\
38.88	0.01\\
38.89	0.01\\
38.9	0.01\\
38.91	0.01\\
38.92	0.01\\
38.93	0.01\\
38.94	0.01\\
38.95	0.01\\
38.96	0.01\\
38.97	0.01\\
38.98	0.01\\
38.99	0.01\\
39	0.01\\
39.01	0.01\\
39.02	0.01\\
39.03	0.01\\
39.04	0.01\\
39.05	0.01\\
39.06	0.01\\
39.07	0.01\\
39.08	0.01\\
39.09	0.01\\
39.1	0.01\\
39.11	0.01\\
39.12	0.01\\
39.13	0.01\\
39.14	0.01\\
39.15	0.01\\
39.16	0.01\\
39.17	0.01\\
39.18	0.01\\
39.19	0.01\\
39.2	0.01\\
39.21	0.01\\
39.22	0.01\\
39.23	0.01\\
39.24	0.01\\
39.25	0.01\\
39.26	0.01\\
39.27	0.01\\
39.28	0.01\\
39.29	0.01\\
39.3	0.01\\
39.31	0.01\\
39.32	0.01\\
39.33	0.01\\
39.34	0.01\\
39.35	0.01\\
39.36	0.01\\
39.37	0.01\\
39.38	0.01\\
39.39	0.01\\
39.4	0.01\\
39.41	0.01\\
39.42	0.01\\
39.43	0.01\\
39.44	0.01\\
39.45	0.01\\
39.46	0.01\\
39.47	0.01\\
39.48	0.01\\
39.49	0.01\\
39.5	0.01\\
39.51	0.01\\
39.52	0.01\\
39.53	0.01\\
39.54	0.01\\
39.55	0.01\\
39.56	0.01\\
39.57	0.01\\
39.58	0.01\\
39.59	0.01\\
39.6	0.01\\
39.61	0.01\\
39.62	0.01\\
39.63	0.01\\
39.64	0.01\\
39.65	0.01\\
39.66	0.01\\
39.67	0.01\\
39.68	0.01\\
39.69	0.01\\
39.7	0.01\\
39.71	0.01\\
39.72	0.01\\
39.73	0.01\\
39.74	0.01\\
39.75	0.01\\
39.76	0.01\\
39.77	0.01\\
39.78	0.01\\
39.79	0.01\\
39.8	0.01\\
39.81	0.01\\
39.82	0.01\\
39.83	0.01\\
39.84	0.01\\
39.85	0.01\\
39.86	0.01\\
39.87	0.01\\
39.88	0.01\\
39.89	0.01\\
39.9	0.01\\
39.91	0.01\\
39.92	0.01\\
39.93	0.01\\
39.94	0.01\\
39.95	0.01\\
39.96	0.01\\
39.97	0.01\\
39.98	0.01\\
39.99	0.01\\
40	0.01\\
40.01	0.01\\
};
\addplot [color=mycolor1,dashed,forget plot]
  table[row sep=crcr]{%
40.01	0.01\\
40.02	0.01\\
40.03	0.01\\
40.04	0.01\\
40.05	0.01\\
40.06	0.01\\
40.07	0.01\\
40.08	0.01\\
40.09	0.01\\
40.1	0.01\\
40.11	0.01\\
40.12	0.01\\
40.13	0.01\\
40.14	0.01\\
40.15	0.01\\
40.16	0.01\\
40.17	0.01\\
40.18	0.01\\
40.19	0.01\\
40.2	0.01\\
40.21	0.01\\
40.22	0.01\\
40.23	0.01\\
40.24	0.01\\
40.25	0.01\\
40.26	0.01\\
40.27	0.01\\
40.28	0.01\\
40.29	0.01\\
40.3	0.01\\
40.31	0.01\\
40.32	0.01\\
40.33	0.01\\
40.34	0.01\\
40.35	0.01\\
40.36	0.01\\
40.37	0.01\\
40.38	0.01\\
40.39	0.01\\
40.4	0.01\\
40.41	0.01\\
40.42	0.01\\
40.43	0.01\\
40.44	0.01\\
40.45	0.01\\
40.46	0.01\\
40.47	0.01\\
40.48	0.01\\
40.49	0.01\\
40.5	0.01\\
40.51	0.01\\
40.52	0.01\\
40.53	0.01\\
40.54	0.01\\
40.55	0.01\\
40.56	0.01\\
40.57	0.01\\
40.58	0.01\\
40.59	0.01\\
40.6	0.01\\
40.61	0.01\\
40.62	0.01\\
40.63	0.01\\
40.64	0.01\\
40.65	0.01\\
40.66	0.01\\
40.67	0.01\\
40.68	0.01\\
40.69	0.01\\
40.7	0.01\\
40.71	0.01\\
40.72	0.01\\
40.73	0.01\\
40.74	0.01\\
40.75	0.01\\
40.76	0.01\\
40.77	0.01\\
40.78	0.01\\
40.79	0.01\\
40.8	0.01\\
40.81	0.01\\
40.82	0.01\\
40.83	0.01\\
40.84	0.01\\
40.85	0.01\\
40.86	0.01\\
40.87	0.01\\
40.88	0.01\\
40.89	0.01\\
40.9	0.01\\
40.91	0.01\\
40.92	0.01\\
40.93	0.01\\
40.94	0.01\\
40.95	0.01\\
40.96	0.01\\
40.97	0.01\\
40.98	0.01\\
40.99	0.01\\
41	0.01\\
41.01	0.01\\
41.02	0.01\\
41.03	0.01\\
41.04	0.01\\
41.05	0.01\\
41.06	0.01\\
41.07	0.01\\
41.08	0.01\\
41.09	0.01\\
41.1	0.01\\
41.11	0.01\\
41.12	0.01\\
41.13	0.01\\
41.14	0.01\\
41.15	0.01\\
41.16	0.01\\
41.17	0.01\\
41.18	0.01\\
41.19	0.01\\
41.2	0.01\\
41.21	0.01\\
41.22	0.01\\
41.23	0.01\\
41.24	0.01\\
41.25	0.01\\
41.26	0.01\\
41.27	0.01\\
41.28	0.01\\
41.29	0.01\\
41.3	0.01\\
41.31	0.01\\
41.32	0.01\\
41.33	0.01\\
41.34	0.01\\
41.35	0.01\\
41.36	0.01\\
41.37	0.01\\
41.38	0.01\\
41.39	0.01\\
41.4	0.01\\
41.41	0.01\\
41.42	0.01\\
41.43	0.01\\
41.44	0.01\\
41.45	0.01\\
41.46	0.01\\
41.47	0.01\\
41.48	0.01\\
41.49	0.01\\
41.5	0.01\\
41.51	0.01\\
41.52	0.01\\
41.53	0.01\\
41.54	0.01\\
41.55	0.01\\
41.56	0.01\\
41.57	0.01\\
41.58	0.01\\
41.59	0.01\\
41.6	0.01\\
41.61	0.01\\
41.62	0.01\\
41.63	0.01\\
41.64	0.01\\
41.65	0.01\\
41.66	0.01\\
41.67	0.01\\
41.68	0.01\\
41.69	0.01\\
41.7	0.01\\
41.71	0.01\\
41.72	0.01\\
41.73	0.01\\
41.74	0.01\\
41.75	0.01\\
41.76	0.01\\
41.77	0.01\\
41.78	0.01\\
41.79	0.01\\
41.8	0.01\\
41.81	0.01\\
41.82	0.01\\
41.83	0.01\\
41.84	0.01\\
41.85	0.01\\
41.86	0.01\\
41.87	0.01\\
41.88	0.01\\
41.89	0.01\\
41.9	0.01\\
41.91	0.01\\
41.92	0.01\\
41.93	0.01\\
41.94	0.01\\
41.95	0.01\\
41.96	0.01\\
41.97	0.01\\
41.98	0.01\\
41.99	0.01\\
42	0.01\\
42.01	0.01\\
42.02	0.01\\
42.03	0.01\\
42.04	0.01\\
42.05	0.01\\
42.06	0.01\\
42.07	0.01\\
42.08	0.01\\
42.09	0.01\\
42.1	0.01\\
42.11	0.01\\
42.12	0.01\\
42.13	0.01\\
42.14	0.01\\
42.15	0.01\\
42.16	0.01\\
42.17	0.01\\
42.18	0.01\\
42.19	0.01\\
42.2	0.01\\
42.21	0.01\\
42.22	0.01\\
42.23	0.01\\
42.24	0.01\\
42.25	0.01\\
42.26	0.01\\
42.27	0.01\\
42.28	0.01\\
42.29	0.01\\
42.3	0.01\\
42.31	0.01\\
42.32	0.01\\
42.33	0.01\\
42.34	0.01\\
42.35	0.01\\
42.36	0.01\\
42.37	0.01\\
42.38	0.01\\
42.39	0.01\\
42.4	0.01\\
42.41	0.01\\
42.42	0.01\\
42.43	0.01\\
42.44	0.01\\
42.45	0.01\\
42.46	0.01\\
42.47	0.01\\
42.48	0.01\\
42.49	0.01\\
42.5	0.01\\
42.51	0.01\\
42.52	0.01\\
42.53	0.01\\
42.54	0.01\\
42.55	0.01\\
42.56	0.01\\
42.57	0.01\\
42.58	0.01\\
42.59	0.01\\
42.6	0.01\\
42.61	0.01\\
42.62	0.01\\
42.63	0.01\\
42.64	0.01\\
42.65	0.01\\
42.66	0.01\\
42.67	0.01\\
42.68	0.01\\
42.69	0.01\\
42.7	0.01\\
42.71	0.01\\
42.72	0.01\\
42.73	0.01\\
42.74	0.01\\
42.75	0.01\\
42.76	0.01\\
42.77	0.01\\
42.78	0.01\\
42.79	0.01\\
42.8	0.01\\
42.81	0.01\\
42.82	0.01\\
42.83	0.01\\
42.84	0.01\\
42.85	0.01\\
42.86	0.01\\
42.87	0.01\\
42.88	0.01\\
42.89	0.01\\
42.9	0.01\\
42.91	0.01\\
42.92	0.01\\
42.93	0.01\\
42.94	0.01\\
42.95	0.01\\
42.96	0.01\\
42.97	0.01\\
42.98	0.01\\
42.99	0.01\\
43	0.01\\
43.01	0.01\\
43.02	0.01\\
43.03	0.01\\
43.04	0.01\\
43.05	0.01\\
43.06	0.01\\
43.07	0.01\\
43.08	0.01\\
43.09	0.01\\
43.1	0.01\\
43.11	0.01\\
43.12	0.01\\
43.13	0.01\\
43.14	0.01\\
43.15	0.01\\
43.16	0.01\\
43.17	0.01\\
43.18	0.01\\
43.19	0.01\\
43.2	0.01\\
43.21	0.01\\
43.22	0.01\\
43.23	0.01\\
43.24	0.01\\
43.25	0.01\\
43.26	0.01\\
43.27	0.01\\
43.28	0.01\\
43.29	0.01\\
43.3	0.01\\
43.31	0.01\\
43.32	0.01\\
43.33	0.01\\
43.34	0.01\\
43.35	0.01\\
43.36	0.01\\
43.37	0.01\\
43.38	0.01\\
43.39	0.01\\
43.4	0.01\\
43.41	0.01\\
43.42	0.01\\
43.43	0.01\\
43.44	0.01\\
43.45	0.01\\
43.46	0.01\\
43.47	0.01\\
43.48	0.01\\
43.49	0.01\\
43.5	0.01\\
43.51	0.01\\
43.52	0.01\\
43.53	0.01\\
43.54	0.01\\
43.55	0.01\\
43.56	0.01\\
43.57	0.01\\
43.58	0.01\\
43.59	0.01\\
43.6	0.01\\
43.61	0.01\\
43.62	0.01\\
43.63	0.01\\
43.64	0.01\\
43.65	0.01\\
43.66	0.01\\
43.67	0.01\\
43.68	0.01\\
43.69	0.01\\
43.7	0.01\\
43.71	0.01\\
43.72	0.01\\
43.73	0.01\\
43.74	0.01\\
43.75	0.01\\
43.76	0.01\\
43.77	0.01\\
43.78	0.01\\
43.79	0.01\\
43.8	0.01\\
43.81	0.01\\
43.82	0.01\\
43.83	0.01\\
43.84	0.01\\
43.85	0.01\\
43.86	0.01\\
43.87	0.01\\
43.88	0.01\\
43.89	0.01\\
43.9	0.01\\
43.91	0.01\\
43.92	0.01\\
43.93	0.01\\
43.94	0.01\\
43.95	0.01\\
43.96	0.01\\
43.97	0.01\\
43.98	0.01\\
43.99	0.01\\
44	0.01\\
44.01	0.01\\
44.02	0.01\\
44.03	0.01\\
44.04	0.01\\
44.05	0.01\\
44.06	0.01\\
44.07	0.01\\
44.08	0.01\\
44.09	0.01\\
44.1	0.01\\
44.11	0.01\\
44.12	0.01\\
44.13	0.01\\
44.14	0.01\\
44.15	0.01\\
44.16	0.01\\
44.17	0.01\\
44.18	0.01\\
44.19	0.01\\
44.2	0.01\\
44.21	0.01\\
44.22	0.01\\
44.23	0.01\\
44.24	0.01\\
44.25	0.01\\
44.26	0.01\\
44.27	0.01\\
44.28	0.01\\
44.29	0.01\\
44.3	0.01\\
44.31	0.01\\
44.32	0.01\\
44.33	0.01\\
44.34	0.01\\
44.35	0.01\\
44.36	0.01\\
44.37	0.01\\
44.38	0.01\\
44.39	0.01\\
44.4	0.01\\
44.41	0.01\\
44.42	0.01\\
44.43	0.01\\
44.44	0.01\\
44.45	0.01\\
44.46	0.01\\
44.47	0.01\\
44.48	0.01\\
44.49	0.01\\
44.5	0.01\\
44.51	0.01\\
44.52	0.01\\
44.53	0.01\\
44.54	0.01\\
44.55	0.01\\
44.56	0.01\\
44.57	0.01\\
44.58	0.01\\
44.59	0.01\\
44.6	0.01\\
44.61	0.01\\
44.62	0.01\\
44.63	0.01\\
44.64	0.01\\
44.65	0.01\\
44.66	0.01\\
44.67	0.01\\
44.68	0.01\\
44.69	0.01\\
44.7	0.01\\
44.71	0.01\\
44.72	0.01\\
44.73	0.01\\
44.74	0.01\\
44.75	0.01\\
44.76	0.01\\
44.77	0.01\\
44.78	0.01\\
44.79	0.01\\
44.8	0.01\\
44.81	0.01\\
44.82	0.01\\
44.83	0.01\\
44.84	0.01\\
44.85	0.01\\
44.86	0.01\\
44.87	0.01\\
44.88	0.01\\
44.89	0.01\\
44.9	0.01\\
44.91	0.01\\
44.92	0.01\\
44.93	0.01\\
44.94	0.01\\
44.95	0.01\\
44.96	0.01\\
44.97	0.01\\
44.98	0.01\\
44.99	0.01\\
45	0.01\\
45.01	0.01\\
45.02	0.01\\
45.03	0.01\\
45.04	0.01\\
45.05	0.01\\
45.06	0.01\\
45.07	0.01\\
45.08	0.01\\
45.09	0.01\\
45.1	0.01\\
45.11	0.01\\
45.12	0.01\\
45.13	0.01\\
45.14	0.01\\
45.15	0.01\\
45.16	0.01\\
45.17	0.01\\
45.18	0.01\\
45.19	0.01\\
45.2	0.01\\
45.21	0.01\\
45.22	0.01\\
45.23	0.01\\
45.24	0.01\\
45.25	0.01\\
45.26	0.01\\
45.27	0.01\\
45.28	0.01\\
45.29	0.01\\
45.3	0.01\\
45.31	0.01\\
45.32	0.01\\
45.33	0.01\\
45.34	0.01\\
45.35	0.01\\
45.36	0.01\\
45.37	0.01\\
45.38	0.01\\
45.39	0.01\\
45.4	0.01\\
45.41	0.01\\
45.42	0.01\\
45.43	0.01\\
45.44	0.01\\
45.45	0.01\\
45.46	0.01\\
45.47	0.01\\
45.48	0.01\\
45.49	0.01\\
45.5	0.01\\
45.51	0.01\\
45.52	0.01\\
45.53	0.01\\
45.54	0.01\\
45.55	0.01\\
45.56	0.01\\
45.57	0.01\\
45.58	0.01\\
45.59	0.01\\
45.6	0.01\\
45.61	0.01\\
45.62	0.01\\
45.63	0.01\\
45.64	0.01\\
45.65	0.01\\
45.66	0.01\\
45.67	0.01\\
45.68	0.01\\
45.69	0.01\\
45.7	0.01\\
45.71	0.01\\
45.72	0.01\\
45.73	0.01\\
45.74	0.01\\
45.75	0.01\\
45.76	0.01\\
45.77	0.01\\
45.78	0.01\\
45.79	0.01\\
45.8	0.01\\
45.81	0.01\\
45.82	0.01\\
45.83	0.01\\
45.84	0.01\\
45.85	0.01\\
45.86	0.01\\
45.87	0.01\\
45.88	0.01\\
45.89	0.01\\
45.9	0.01\\
45.91	0.01\\
45.92	0.01\\
45.93	0.01\\
45.94	0.01\\
45.95	0.01\\
45.96	0.01\\
45.97	0.01\\
45.98	0.01\\
45.99	0.01\\
46	0.01\\
46.01	0.01\\
46.02	0.01\\
46.03	0.01\\
46.04	0.01\\
46.05	0.01\\
46.06	0.01\\
46.07	0.01\\
46.08	0.01\\
46.09	0.01\\
46.1	0.01\\
46.11	0.01\\
46.12	0.01\\
46.13	0.01\\
46.14	0.01\\
46.15	0.01\\
46.16	0.01\\
46.17	0.01\\
46.18	0.01\\
46.19	0.01\\
46.2	0.01\\
46.21	0.01\\
46.22	0.01\\
46.23	0.01\\
46.24	0.01\\
46.25	0.01\\
46.26	0.01\\
46.27	0.01\\
46.28	0.01\\
46.29	0.01\\
46.3	0.01\\
46.31	0.01\\
46.32	0.01\\
46.33	0.01\\
46.34	0.01\\
46.35	0.01\\
46.36	0.01\\
46.37	0.01\\
46.38	0.01\\
46.39	0.01\\
46.4	0.01\\
46.41	0.01\\
46.42	0.01\\
46.43	0.01\\
46.44	0.01\\
46.45	0.01\\
46.46	0.01\\
46.47	0.01\\
46.48	0.01\\
46.49	0.01\\
46.5	0.01\\
46.51	0.01\\
46.52	0.01\\
46.53	0.01\\
46.54	0.01\\
46.55	0.01\\
46.56	0.01\\
46.57	0.01\\
46.58	0.01\\
46.59	0.01\\
46.6	0.01\\
46.61	0.01\\
46.62	0.01\\
46.63	0.01\\
46.64	0.01\\
46.65	0.01\\
46.66	0.01\\
46.67	0.01\\
46.68	0.01\\
46.69	0.01\\
46.7	0.01\\
46.71	0.01\\
46.72	0.01\\
46.73	0.01\\
46.74	0.01\\
46.75	0.01\\
46.76	0.01\\
46.77	0.01\\
46.78	0.01\\
46.79	0.01\\
46.8	0.01\\
46.81	0.01\\
46.82	0.01\\
46.83	0.01\\
46.84	0.01\\
46.85	0.01\\
46.86	0.01\\
46.87	0.01\\
46.88	0.01\\
46.89	0.01\\
46.9	0.01\\
46.91	0.01\\
46.92	0.01\\
46.93	0.01\\
46.94	0.01\\
46.95	0.01\\
46.96	0.01\\
46.97	0.01\\
46.98	0.01\\
46.99	0.01\\
47	0.01\\
47.01	0.01\\
47.02	0.01\\
47.03	0.01\\
47.04	0.01\\
47.05	0.01\\
47.06	0.01\\
47.07	0.01\\
47.08	0.01\\
47.09	0.01\\
47.1	0.01\\
47.11	0.01\\
47.12	0.01\\
47.13	0.01\\
47.14	0.01\\
47.15	0.01\\
47.16	0.01\\
47.17	0.01\\
47.18	0.01\\
47.19	0.01\\
47.2	0.01\\
47.21	0.01\\
47.22	0.01\\
47.23	0.01\\
47.24	0.01\\
47.25	0.01\\
47.26	0.01\\
47.27	0.01\\
47.28	0.01\\
47.29	0.01\\
47.3	0.01\\
47.31	0.01\\
47.32	0.01\\
47.33	0.01\\
47.34	0.01\\
47.35	0.01\\
47.36	0.01\\
47.37	0.01\\
47.38	0.01\\
47.39	0.01\\
47.4	0.01\\
47.41	0.01\\
47.42	0.01\\
47.43	0.01\\
47.44	0.01\\
47.45	0.01\\
47.46	0.01\\
47.47	0.01\\
47.48	0.01\\
47.49	0.01\\
47.5	0.01\\
47.51	0.01\\
47.52	0.01\\
47.53	0.01\\
47.54	0.01\\
47.55	0.01\\
47.56	0.01\\
47.57	0.01\\
47.58	0.01\\
47.59	0.01\\
47.6	0.01\\
47.61	0.01\\
47.62	0.01\\
47.63	0.01\\
47.64	0.01\\
47.65	0.01\\
47.66	0.01\\
47.67	0.01\\
47.68	0.01\\
47.69	0.01\\
47.7	0.01\\
47.71	0.01\\
47.72	0.01\\
47.73	0.01\\
47.74	0.01\\
47.75	0.01\\
47.76	0.01\\
47.77	0.01\\
47.78	0.01\\
47.79	0.01\\
47.8	0.01\\
47.81	0.01\\
47.82	0.01\\
47.83	0.01\\
47.84	0.01\\
47.85	0.01\\
47.86	0.01\\
47.87	0.01\\
47.88	0.01\\
47.89	0.01\\
47.9	0.01\\
47.91	0.01\\
47.92	0.01\\
47.93	0.01\\
47.94	0.01\\
47.95	0.01\\
47.96	0.01\\
47.97	0.01\\
47.98	0.01\\
47.99	0.01\\
48	0.01\\
48.01	0.01\\
48.02	0.01\\
48.03	0.01\\
48.04	0.01\\
48.05	0.01\\
48.06	0.01\\
48.07	0.01\\
48.08	0.01\\
48.09	0.01\\
48.1	0.01\\
48.11	0.01\\
48.12	0.01\\
48.13	0.01\\
48.14	0.01\\
48.15	0.01\\
48.16	0.01\\
48.17	0.01\\
48.18	0.01\\
48.19	0.01\\
48.2	0.01\\
48.21	0.01\\
48.22	0.01\\
48.23	0.01\\
48.24	0.01\\
48.25	0.01\\
48.26	0.01\\
48.27	0.01\\
48.28	0.01\\
48.29	0.01\\
48.3	0.01\\
48.31	0.01\\
48.32	0.01\\
48.33	0.01\\
48.34	0.01\\
48.35	0.01\\
48.36	0.01\\
48.37	0.01\\
48.38	0.01\\
48.39	0.01\\
48.4	0.01\\
48.41	0.01\\
48.42	0.01\\
48.43	0.01\\
48.44	0.01\\
48.45	0.01\\
48.46	0.01\\
48.47	0.01\\
48.48	0.01\\
48.49	0.01\\
48.5	0.01\\
48.51	0.01\\
48.52	0.01\\
48.53	0.01\\
48.54	0.01\\
48.55	0.01\\
48.56	0.01\\
48.57	0.01\\
48.58	0.01\\
48.59	0.01\\
48.6	0.01\\
48.61	0.01\\
48.62	0.01\\
48.63	0.01\\
48.64	0.01\\
48.65	0.01\\
48.66	0.01\\
48.67	0.01\\
48.68	0.01\\
48.69	0.01\\
48.7	0.01\\
48.71	0.01\\
48.72	0.01\\
48.73	0.01\\
48.74	0.01\\
48.75	0.01\\
48.76	0.01\\
48.77	0.01\\
48.78	0.01\\
48.79	0.01\\
48.8	0.01\\
48.81	0.01\\
48.82	0.01\\
48.83	0.01\\
48.84	0.01\\
48.85	0.01\\
48.86	0.01\\
48.87	0.01\\
48.88	0.01\\
48.89	0.01\\
48.9	0.01\\
48.91	0.01\\
48.92	0.01\\
48.93	0.01\\
48.94	0.01\\
48.95	0.01\\
48.96	0.01\\
48.97	0.01\\
48.98	0.01\\
48.99	0.01\\
49	0.01\\
49.01	0.01\\
49.02	0.01\\
49.03	0.01\\
49.04	0.01\\
49.05	0.01\\
49.06	0.01\\
49.07	0.01\\
49.08	0.01\\
49.09	0.01\\
49.1	0.01\\
49.11	0.01\\
49.12	0.01\\
49.13	0.01\\
49.14	0.01\\
49.15	0.01\\
49.16	0.01\\
49.17	0.01\\
49.18	0.01\\
49.19	0.01\\
49.2	0.01\\
49.21	0.01\\
49.22	0.01\\
49.23	0.01\\
49.24	0.01\\
49.25	0.01\\
49.26	0.01\\
49.27	0.01\\
49.28	0.01\\
49.29	0.01\\
49.3	0.01\\
49.31	0.01\\
49.32	0.01\\
49.33	0.01\\
49.34	0.01\\
49.35	0.01\\
49.36	0.01\\
49.37	0.01\\
49.38	0.01\\
49.39	0.01\\
49.4	0.01\\
49.41	0.01\\
49.42	0.01\\
49.43	0.01\\
49.44	0.01\\
49.45	0.01\\
49.46	0.01\\
49.47	0.01\\
49.48	0.01\\
49.49	0.01\\
49.5	0.01\\
49.51	0.01\\
49.52	0.01\\
49.53	0.01\\
49.54	0.01\\
49.55	0.01\\
49.56	0.01\\
49.57	0.01\\
49.58	0.01\\
49.59	0.01\\
49.6	0.01\\
49.61	0.01\\
49.62	0.01\\
49.63	0.01\\
49.64	0.01\\
49.65	0.01\\
49.66	0.01\\
49.67	0.01\\
49.68	0.01\\
49.69	0.01\\
49.7	0.01\\
49.71	0.01\\
49.72	0.01\\
49.73	0.01\\
49.74	0.01\\
49.75	0.01\\
49.76	0.01\\
49.77	0.01\\
49.78	0.01\\
49.79	0.01\\
49.8	0.01\\
49.81	0.01\\
49.82	0.01\\
49.83	0.01\\
49.84	0.01\\
49.85	0.01\\
49.86	0.01\\
49.87	0.01\\
49.88	0.01\\
49.89	0.01\\
49.9	0.01\\
49.91	0.01\\
49.92	0.01\\
49.93	0.01\\
49.94	0.01\\
49.95	0.01\\
49.96	0.01\\
49.97	0.01\\
49.98	0.01\\
49.99	0.01\\
50	0.01\\
50.01	0.01\\
50.02	0.01\\
50.03	0.01\\
50.04	0.01\\
50.05	0.01\\
50.06	0.01\\
50.07	0.01\\
50.08	0.01\\
50.09	0.01\\
50.1	0.01\\
50.11	0.01\\
50.12	0.01\\
50.13	0.01\\
50.14	0.01\\
50.15	0.01\\
50.16	0.01\\
50.17	0.01\\
50.18	0.01\\
50.19	0.01\\
50.2	0.01\\
50.21	0.01\\
50.22	0.01\\
50.23	0.01\\
50.24	0.01\\
50.25	0.01\\
50.26	0.01\\
50.27	0.01\\
50.28	0.01\\
50.29	0.01\\
50.3	0.01\\
50.31	0.01\\
50.32	0.01\\
50.33	0.01\\
50.34	0.01\\
50.35	0.01\\
50.36	0.01\\
50.37	0.01\\
50.38	0.01\\
50.39	0.01\\
50.4	0.01\\
50.41	0.01\\
50.42	0.01\\
50.43	0.01\\
50.44	0.01\\
50.45	0.01\\
50.46	0.01\\
50.47	0.01\\
50.48	0.01\\
50.49	0.01\\
50.5	0.01\\
50.51	0.01\\
50.52	0.01\\
50.53	0.01\\
50.54	0.01\\
50.55	0.01\\
50.56	0.01\\
50.57	0.01\\
50.58	0.01\\
50.59	0.01\\
50.6	0.01\\
50.61	0.01\\
50.62	0.01\\
50.63	0.01\\
50.64	0.01\\
50.65	0.01\\
50.66	0.01\\
50.67	0.01\\
50.68	0.01\\
50.69	0.01\\
50.7	0.01\\
50.71	0.01\\
50.72	0.01\\
50.73	0.01\\
50.74	0.01\\
50.75	0.01\\
50.76	0.01\\
50.77	0.01\\
50.78	0.01\\
50.79	0.01\\
50.8	0.01\\
50.81	0.01\\
50.82	0.01\\
50.83	0.01\\
50.84	0.01\\
50.85	0.01\\
50.86	0.01\\
50.87	0.01\\
50.88	0.01\\
50.89	0.01\\
50.9	0.01\\
50.91	0.01\\
50.92	0.01\\
50.93	0.01\\
50.94	0.01\\
50.95	0.01\\
50.96	0.01\\
50.97	0.01\\
50.98	0.01\\
50.99	0.01\\
51	0.01\\
51.01	0.01\\
51.02	0.01\\
51.03	0.01\\
51.04	0.01\\
51.05	0.01\\
51.06	0.01\\
51.07	0.01\\
51.08	0.01\\
51.09	0.01\\
51.1	0.01\\
51.11	0.01\\
51.12	0.01\\
51.13	0.01\\
51.14	0.01\\
51.15	0.01\\
51.16	0.01\\
51.17	0.01\\
51.18	0.01\\
51.19	0.01\\
51.2	0.01\\
51.21	0.01\\
51.22	0.01\\
51.23	0.01\\
51.24	0.01\\
51.25	0.01\\
51.26	0.01\\
51.27	0.01\\
51.28	0.01\\
51.29	0.01\\
51.3	0.01\\
51.31	0.01\\
51.32	0.01\\
51.33	0.01\\
51.34	0.01\\
51.35	0.01\\
51.36	0.01\\
51.37	0.01\\
51.38	0.01\\
51.39	0.01\\
51.4	0.01\\
51.41	0.01\\
51.42	0.01\\
51.43	0.01\\
51.44	0.01\\
51.45	0.01\\
51.46	0.01\\
51.47	0.01\\
51.48	0.01\\
51.49	0.01\\
51.5	0.01\\
51.51	0.01\\
51.52	0.01\\
51.53	0.01\\
51.54	0.01\\
51.55	0.01\\
51.56	0.01\\
51.57	0.01\\
51.58	0.01\\
51.59	0.01\\
51.6	0.01\\
51.61	0.01\\
51.62	0.01\\
51.63	0.01\\
51.64	0.01\\
51.65	0.01\\
51.66	0.01\\
51.67	0.01\\
51.68	0.01\\
51.69	0.01\\
51.7	0.01\\
51.71	0.01\\
51.72	0.01\\
51.73	0.01\\
51.74	0.01\\
51.75	0.01\\
51.76	0.01\\
51.77	0.01\\
51.78	0.01\\
51.79	0.01\\
51.8	0.01\\
51.81	0.01\\
51.82	0.01\\
51.83	0.01\\
51.84	0.01\\
51.85	0.01\\
51.86	0.01\\
51.87	0.01\\
51.88	0.01\\
51.89	0.01\\
51.9	0.01\\
51.91	0.01\\
51.92	0.01\\
51.93	0.01\\
51.94	0.01\\
51.95	0.01\\
51.96	0.01\\
51.97	0.01\\
51.98	0.01\\
51.99	0.01\\
52	0.01\\
52.01	0.01\\
52.02	0.01\\
52.03	0.01\\
52.04	0.01\\
52.05	0.01\\
52.06	0.01\\
52.07	0.01\\
52.08	0.01\\
52.09	0.01\\
52.1	0.01\\
52.11	0.01\\
52.12	0.01\\
52.13	0.01\\
52.14	0.01\\
52.15	0.01\\
52.16	0.01\\
52.17	0.01\\
52.18	0.01\\
52.19	0.01\\
52.2	0.01\\
52.21	0.01\\
52.22	0.01\\
52.23	0.01\\
52.24	0.01\\
52.25	0.01\\
52.26	0.01\\
52.27	0.01\\
52.28	0.01\\
52.29	0.01\\
52.3	0.01\\
52.31	0.01\\
52.32	0.01\\
52.33	0.01\\
52.34	0.01\\
52.35	0.01\\
52.36	0.01\\
52.37	0.01\\
52.38	0.01\\
52.39	0.01\\
52.4	0.01\\
52.41	0.01\\
52.42	0.01\\
52.43	0.01\\
52.44	0.01\\
52.45	0.01\\
52.46	0.01\\
52.47	0.01\\
52.48	0.01\\
52.49	0.01\\
52.5	0.01\\
52.51	0.01\\
52.52	0.01\\
52.53	0.01\\
52.54	0.01\\
52.55	0.01\\
52.56	0.01\\
52.57	0.01\\
52.58	0.01\\
52.59	0.01\\
52.6	0.01\\
52.61	0.01\\
52.62	0.01\\
52.63	0.01\\
52.64	0.01\\
52.65	0.01\\
52.66	0.01\\
52.67	0.01\\
52.68	0.01\\
52.69	0.01\\
52.7	0.01\\
52.71	0.01\\
52.72	0.01\\
52.73	0.01\\
52.74	0.01\\
52.75	0.01\\
52.76	0.01\\
52.77	0.01\\
52.78	0.01\\
52.79	0.01\\
52.8	0.01\\
52.81	0.01\\
52.82	0.01\\
52.83	0.01\\
52.84	0.01\\
52.85	0.01\\
52.86	0.01\\
52.87	0.01\\
52.88	0.01\\
52.89	0.01\\
52.9	0.01\\
52.91	0.01\\
52.92	0.01\\
52.93	0.01\\
52.94	0.01\\
52.95	0.01\\
52.96	0.01\\
52.97	0.01\\
52.98	0.01\\
52.99	0.01\\
53	0.01\\
53.01	0.01\\
53.02	0.01\\
53.03	0.01\\
53.04	0.01\\
53.05	0.01\\
53.06	0.01\\
53.07	0.01\\
53.08	0.01\\
53.09	0.01\\
53.1	0.01\\
53.11	0.01\\
53.12	0.01\\
53.13	0.01\\
53.14	0.01\\
53.15	0.01\\
53.16	0.01\\
53.17	0.01\\
53.18	0.01\\
53.19	0.01\\
53.2	0.01\\
53.21	0.01\\
53.22	0.01\\
53.23	0.01\\
53.24	0.01\\
53.25	0.01\\
53.26	0.01\\
53.27	0.01\\
53.28	0.01\\
53.29	0.01\\
53.3	0.01\\
53.31	0.01\\
53.32	0.01\\
53.33	0.01\\
53.34	0.01\\
53.35	0.01\\
53.36	0.01\\
53.37	0.01\\
53.38	0.01\\
53.39	0.01\\
53.4	0.01\\
53.41	0.01\\
53.42	0.01\\
53.43	0.01\\
53.44	0.01\\
53.45	0.01\\
53.46	0.01\\
53.47	0.01\\
53.48	0.01\\
53.49	0.01\\
53.5	0.01\\
53.51	0.01\\
53.52	0.01\\
53.53	0.01\\
53.54	0.01\\
53.55	0.01\\
53.56	0.01\\
53.57	0.01\\
53.58	0.01\\
53.59	0.01\\
53.6	0.01\\
53.61	0.01\\
53.62	0.01\\
53.63	0.01\\
53.64	0.01\\
53.65	0.01\\
53.66	0.01\\
53.67	0.01\\
53.68	0.01\\
53.69	0.01\\
53.7	0.01\\
53.71	0.01\\
53.72	0.01\\
53.73	0.01\\
53.74	0.01\\
53.75	0.01\\
53.76	0.01\\
53.77	0.01\\
53.78	0.01\\
53.79	0.01\\
53.8	0.01\\
53.81	0.01\\
53.82	0.01\\
53.83	0.01\\
53.84	0.01\\
53.85	0.01\\
53.86	0.01\\
53.87	0.01\\
53.88	0.01\\
53.89	0.01\\
53.9	0.01\\
53.91	0.01\\
53.92	0.01\\
53.93	0.01\\
53.94	0.01\\
53.95	0.01\\
53.96	0.01\\
53.97	0.01\\
53.98	0.01\\
53.99	0.01\\
54	0.01\\
54.01	0.01\\
54.02	0.01\\
54.03	0.01\\
54.04	0.01\\
54.05	0.01\\
54.06	0.01\\
54.07	0.01\\
54.08	0.01\\
54.09	0.01\\
54.1	0.01\\
54.11	0.01\\
54.12	0.01\\
54.13	0.01\\
54.14	0.01\\
54.15	0.01\\
54.16	0.01\\
54.17	0.01\\
54.18	0.01\\
54.19	0.01\\
54.2	0.01\\
54.21	0.01\\
54.22	0.01\\
54.23	0.01\\
54.24	0.01\\
54.25	0.01\\
54.26	0.01\\
54.27	0.01\\
54.28	0.01\\
54.29	0.01\\
54.3	0.01\\
54.31	0.01\\
54.32	0.01\\
54.33	0.01\\
54.34	0.01\\
54.35	0.01\\
54.36	0.01\\
54.37	0.01\\
54.38	0.01\\
54.39	0.01\\
54.4	0.01\\
54.41	0.01\\
54.42	0.01\\
54.43	0.01\\
54.44	0.01\\
54.45	0.01\\
54.46	0.01\\
54.47	0.01\\
54.48	0.01\\
54.49	0.01\\
54.5	0.01\\
54.51	0.01\\
54.52	0.01\\
54.53	0.01\\
54.54	0.01\\
54.55	0.01\\
54.56	0.01\\
54.57	0.01\\
54.58	0.01\\
54.59	0.01\\
54.6	0.01\\
54.61	0.01\\
54.62	0.01\\
54.63	0.01\\
54.64	0.01\\
54.65	0.01\\
54.66	0.01\\
54.67	0.01\\
54.68	0.01\\
54.69	0.01\\
54.7	0.01\\
54.71	0.01\\
54.72	0.01\\
54.73	0.01\\
54.74	0.01\\
54.75	0.01\\
54.76	0.01\\
54.77	0.01\\
54.78	0.01\\
54.79	0.01\\
54.8	0.01\\
54.81	0.01\\
54.82	0.01\\
54.83	0.01\\
54.84	0.01\\
54.85	0.01\\
54.86	0.01\\
54.87	0.01\\
54.88	0.01\\
54.89	0.01\\
54.9	0.01\\
54.91	0.01\\
54.92	0.01\\
54.93	0.01\\
54.94	0.01\\
54.95	0.01\\
54.96	0.01\\
54.97	0.01\\
54.98	0.01\\
54.99	0.01\\
55	0.01\\
55.01	0.01\\
55.02	0.01\\
55.03	0.01\\
55.04	0.01\\
55.05	0.01\\
55.06	0.01\\
55.07	0.01\\
55.08	0.01\\
55.09	0.01\\
55.1	0.01\\
55.11	0.01\\
55.12	0.01\\
55.13	0.01\\
55.14	0.01\\
55.15	0.01\\
55.16	0.01\\
55.17	0.01\\
55.18	0.01\\
55.19	0.01\\
55.2	0.01\\
55.21	0.01\\
55.22	0.01\\
55.23	0.01\\
55.24	0.01\\
55.25	0.01\\
55.26	0.01\\
55.27	0.01\\
55.28	0.01\\
55.29	0.01\\
55.3	0.01\\
55.31	0.01\\
55.32	0.01\\
55.33	0.01\\
55.34	0.01\\
55.35	0.01\\
55.36	0.01\\
55.37	0.01\\
55.38	0.01\\
55.39	0.01\\
55.4	0.01\\
55.41	0.01\\
55.42	0.01\\
55.43	0.01\\
55.44	0.01\\
55.45	0.01\\
55.46	0.01\\
55.47	0.01\\
55.48	0.01\\
55.49	0.01\\
55.5	0.01\\
55.51	0.01\\
55.52	0.01\\
55.53	0.01\\
55.54	0.01\\
55.55	0.01\\
55.56	0.01\\
55.57	0.01\\
55.58	0.01\\
55.59	0.01\\
55.6	0.01\\
55.61	0.01\\
55.62	0.01\\
55.63	0.01\\
55.64	0.01\\
55.65	0.01\\
55.66	0.01\\
55.67	0.01\\
55.68	0.01\\
55.69	0.01\\
55.7	0.01\\
55.71	0.01\\
55.72	0.01\\
55.73	0.01\\
55.74	0.01\\
55.75	0.01\\
55.76	0.01\\
55.77	0.01\\
55.78	0.01\\
55.79	0.01\\
55.8	0.01\\
55.81	0.01\\
55.82	0.01\\
55.83	0.01\\
55.84	0.01\\
55.85	0.01\\
55.86	0.01\\
55.87	0.01\\
55.88	0.01\\
55.89	0.01\\
55.9	0.01\\
55.91	0.01\\
55.92	0.01\\
55.93	0.01\\
55.94	0.01\\
55.95	0.01\\
55.96	0.01\\
55.97	0.01\\
55.98	0.01\\
55.99	0.01\\
56	0.01\\
56.01	0.01\\
56.02	0.01\\
56.03	0.01\\
56.04	0.01\\
56.05	0.01\\
56.06	0.01\\
56.07	0.01\\
56.08	0.01\\
56.09	0.01\\
56.1	0.01\\
56.11	0.01\\
56.12	0.01\\
56.13	0.01\\
56.14	0.01\\
56.15	0.01\\
56.16	0.01\\
56.17	0.01\\
56.18	0.01\\
56.19	0.01\\
56.2	0.01\\
56.21	0.01\\
56.22	0.01\\
56.23	0.01\\
56.24	0.01\\
56.25	0.01\\
56.26	0.01\\
56.27	0.01\\
56.28	0.01\\
56.29	0.01\\
56.3	0.01\\
56.31	0.01\\
56.32	0.01\\
56.33	0.01\\
56.34	0.01\\
56.35	0.01\\
56.36	0.01\\
56.37	0.01\\
56.38	0.01\\
56.39	0.01\\
56.4	0.01\\
56.41	0.01\\
56.42	0.01\\
56.43	0.01\\
56.44	0.01\\
56.45	0.01\\
56.46	0.01\\
56.47	0.01\\
56.48	0.01\\
56.49	0.01\\
56.5	0.01\\
56.51	0.01\\
56.52	0.01\\
56.53	0.01\\
56.54	0.01\\
56.55	0.01\\
56.56	0.01\\
56.57	0.01\\
56.58	0.01\\
56.59	0.01\\
56.6	0.01\\
56.61	0.01\\
56.62	0.01\\
56.63	0.01\\
56.64	0.01\\
56.65	0.01\\
56.66	0.01\\
56.67	0.01\\
56.68	0.01\\
56.69	0.01\\
56.7	0.01\\
56.71	0.01\\
56.72	0.01\\
56.73	0.01\\
56.74	0.01\\
56.75	0.01\\
56.76	0.01\\
56.77	0.01\\
56.78	0.01\\
56.79	0.01\\
56.8	0.01\\
56.81	0.01\\
56.82	0.01\\
56.83	0.01\\
56.84	0.01\\
56.85	0.01\\
56.86	0.01\\
56.87	0.01\\
56.88	0.01\\
56.89	0.01\\
56.9	0.01\\
56.91	0.01\\
56.92	0.01\\
56.93	0.01\\
56.94	0.01\\
56.95	0.01\\
56.96	0.01\\
56.97	0.01\\
56.98	0.01\\
56.99	0.01\\
57	0.01\\
57.01	0.01\\
57.02	0.01\\
57.03	0.01\\
57.04	0.01\\
57.05	0.01\\
57.06	0.01\\
57.07	0.01\\
57.08	0.01\\
57.09	0.01\\
57.1	0.01\\
57.11	0.01\\
57.12	0.01\\
57.13	0.01\\
57.14	0.01\\
57.15	0.01\\
57.16	0.01\\
57.17	0.01\\
57.18	0.01\\
57.19	0.01\\
57.2	0.01\\
57.21	0.01\\
57.22	0.01\\
57.23	0.01\\
57.24	0.01\\
57.25	0.01\\
57.26	0.01\\
57.27	0.01\\
57.28	0.01\\
57.29	0.01\\
57.3	0.01\\
57.31	0.01\\
57.32	0.01\\
57.33	0.01\\
57.34	0.01\\
57.35	0.01\\
57.36	0.01\\
57.37	0.01\\
57.38	0.01\\
57.39	0.01\\
57.4	0.01\\
57.41	0.01\\
57.42	0.01\\
57.43	0.01\\
57.44	0.01\\
57.45	0.01\\
57.46	0.01\\
57.47	0.01\\
57.48	0.01\\
57.49	0.01\\
57.5	0.01\\
57.51	0.01\\
57.52	0.01\\
57.53	0.01\\
57.54	0.01\\
57.55	0.01\\
57.56	0.01\\
57.57	0.01\\
57.58	0.01\\
57.59	0.01\\
57.6	0.01\\
57.61	0.01\\
57.62	0.01\\
57.63	0.01\\
57.64	0.01\\
57.65	0.01\\
57.66	0.01\\
57.67	0.01\\
57.68	0.01\\
57.69	0.01\\
57.7	0.01\\
57.71	0.01\\
57.72	0.01\\
57.73	0.01\\
57.74	0.01\\
57.75	0.01\\
57.76	0.01\\
57.77	0.01\\
57.78	0.01\\
57.79	0.01\\
57.8	0.01\\
57.81	0.01\\
57.82	0.01\\
57.83	0.01\\
57.84	0.01\\
57.85	0.01\\
57.86	0.01\\
57.87	0.01\\
57.88	0.01\\
57.89	0.01\\
57.9	0.01\\
57.91	0.01\\
57.92	0.01\\
57.93	0.01\\
57.94	0.01\\
57.95	0.01\\
57.96	0.01\\
57.97	0.01\\
57.98	0.01\\
57.99	0.01\\
58	0.01\\
58.01	0.01\\
58.02	0.01\\
58.03	0.01\\
58.04	0.01\\
58.05	0.01\\
58.06	0.01\\
58.07	0.01\\
58.08	0.01\\
58.09	0.01\\
58.1	0.01\\
58.11	0.01\\
58.12	0.01\\
58.13	0.01\\
58.14	0.01\\
58.15	0.01\\
58.16	0.01\\
58.17	0.01\\
58.18	0.01\\
58.19	0.01\\
58.2	0.01\\
58.21	0.01\\
58.22	0.01\\
58.23	0.01\\
58.24	0.01\\
58.25	0.01\\
58.26	0.01\\
58.27	0.01\\
58.28	0.01\\
58.29	0.01\\
58.3	0.01\\
58.31	0.01\\
58.32	0.01\\
58.33	0.01\\
58.34	0.01\\
58.35	0.01\\
58.36	0.01\\
58.37	0.01\\
58.38	0.01\\
58.39	0.01\\
58.4	0.01\\
58.41	0.01\\
58.42	0.01\\
58.43	0.01\\
58.44	0.01\\
58.45	0.01\\
58.46	0.01\\
58.47	0.01\\
58.48	0.01\\
58.49	0.01\\
58.5	0.01\\
58.51	0.01\\
58.52	0.01\\
58.53	0.01\\
58.54	0.01\\
58.55	0.01\\
58.56	0.01\\
58.57	0.01\\
58.58	0.01\\
58.59	0.01\\
58.6	0.01\\
58.61	0.01\\
58.62	0.01\\
58.63	0.01\\
58.64	0.01\\
58.65	0.01\\
58.66	0.01\\
58.67	0.01\\
58.68	0.01\\
58.69	0.01\\
58.7	0.01\\
58.71	0.01\\
58.72	0.01\\
58.73	0.01\\
58.74	0.01\\
58.75	0.01\\
58.76	0.01\\
58.77	0.01\\
58.78	0.01\\
58.79	0.01\\
58.8	0.01\\
58.81	0.01\\
58.82	0.01\\
58.83	0.01\\
58.84	0.01\\
58.85	0.01\\
58.86	0.01\\
58.87	0.01\\
58.88	0.01\\
58.89	0.01\\
58.9	0.01\\
58.91	0.01\\
58.92	0.01\\
58.93	0.01\\
58.94	0.01\\
58.95	0.01\\
58.96	0.01\\
58.97	0.01\\
58.98	0.01\\
58.99	0.01\\
59	0.01\\
59.01	0.01\\
59.02	0.01\\
59.03	0.01\\
59.04	0.01\\
59.05	0.01\\
59.06	0.01\\
59.07	0.01\\
59.08	0.01\\
59.09	0.01\\
59.1	0.01\\
59.11	0.01\\
59.12	0.01\\
59.13	0.01\\
59.14	0.01\\
59.15	0.01\\
59.16	0.01\\
59.17	0.01\\
59.18	0.01\\
59.19	0.01\\
59.2	0.01\\
59.21	0.01\\
59.22	0.01\\
59.23	0.01\\
59.24	0.01\\
59.25	0.01\\
59.26	0.01\\
59.27	0.01\\
59.28	0.01\\
59.29	0.01\\
59.3	0.01\\
59.31	0.01\\
59.32	0.01\\
59.33	0.01\\
59.34	0.01\\
59.35	0.01\\
59.36	0.01\\
59.37	0.01\\
59.38	0.01\\
59.39	0.01\\
59.4	0.01\\
59.41	0.01\\
59.42	0.01\\
59.43	0.01\\
59.44	0.01\\
59.45	0.01\\
59.46	0.01\\
59.47	0.01\\
59.48	0.01\\
59.49	0.01\\
59.5	0.01\\
59.51	0.01\\
59.52	0.01\\
59.53	0.01\\
59.54	0.01\\
59.55	0.01\\
59.56	0.01\\
59.57	0.01\\
59.58	0.01\\
59.59	0.01\\
59.6	0.01\\
59.61	0.01\\
59.62	0.01\\
59.63	0.01\\
59.64	0.01\\
59.65	0.01\\
59.66	0.01\\
59.67	0.01\\
59.68	0.01\\
59.69	0.01\\
59.7	0.01\\
59.71	0.01\\
59.72	0.01\\
59.73	0.01\\
59.74	0.01\\
59.75	0.01\\
59.76	0.01\\
59.77	0.01\\
59.78	0.01\\
59.79	0.01\\
59.8	0.01\\
59.81	0.01\\
59.82	0.01\\
59.83	0.01\\
59.84	0.01\\
59.85	0.01\\
59.86	0.01\\
59.87	0.01\\
59.88	0.01\\
59.89	0.01\\
59.9	0.01\\
59.91	0.01\\
59.92	0.01\\
59.93	0.01\\
59.94	0.01\\
59.95	0.01\\
59.96	0.01\\
59.97	0.01\\
59.98	0.01\\
59.99	0.01\\
60	0.01\\
60.01	0.01\\
60.02	0.01\\
60.03	0.01\\
60.04	0.01\\
60.05	0.01\\
60.06	0.01\\
60.07	0.01\\
60.08	0.01\\
60.09	0.01\\
60.1	0.01\\
60.11	0.01\\
60.12	0.01\\
60.13	0.01\\
60.14	0.01\\
60.15	0.01\\
60.16	0.01\\
60.17	0.01\\
60.18	0.01\\
60.19	0.01\\
60.2	0.01\\
60.21	0.01\\
60.22	0.01\\
60.23	0.01\\
60.24	0.01\\
60.25	0.01\\
60.26	0.01\\
60.27	0.01\\
60.28	0.01\\
60.29	0.01\\
60.3	0.01\\
60.31	0.01\\
60.32	0.01\\
60.33	0.01\\
60.34	0.01\\
60.35	0.01\\
60.36	0.01\\
60.37	0.01\\
60.38	0.01\\
60.39	0.01\\
60.4	0.01\\
60.41	0.01\\
60.42	0.01\\
60.43	0.01\\
60.44	0.01\\
60.45	0.01\\
60.46	0.01\\
60.47	0.01\\
60.48	0.01\\
60.49	0.01\\
60.5	0.01\\
60.51	0.01\\
60.52	0.01\\
60.53	0.01\\
60.54	0.01\\
60.55	0.01\\
60.56	0.01\\
60.57	0.01\\
60.58	0.01\\
60.59	0.01\\
60.6	0.01\\
60.61	0.01\\
60.62	0.01\\
60.63	0.01\\
60.64	0.01\\
60.65	0.01\\
60.66	0.01\\
60.67	0.01\\
60.68	0.01\\
60.69	0.01\\
60.7	0.01\\
60.71	0.01\\
60.72	0.01\\
60.73	0.01\\
60.74	0.01\\
60.75	0.01\\
60.76	0.01\\
60.77	0.01\\
60.78	0.01\\
60.79	0.01\\
60.8	0.01\\
60.81	0.01\\
60.82	0.01\\
60.83	0.01\\
60.84	0.01\\
60.85	0.01\\
60.86	0.01\\
60.87	0.01\\
60.88	0.01\\
60.89	0.01\\
60.9	0.01\\
60.91	0.01\\
60.92	0.01\\
60.93	0.01\\
60.94	0.01\\
60.95	0.01\\
60.96	0.01\\
60.97	0.01\\
60.98	0.01\\
60.99	0.01\\
61	0.01\\
61.01	0.01\\
61.02	0.01\\
61.03	0.01\\
61.04	0.01\\
61.05	0.01\\
61.06	0.01\\
61.07	0.01\\
61.08	0.01\\
61.09	0.01\\
61.1	0.01\\
61.11	0.01\\
61.12	0.01\\
61.13	0.01\\
61.14	0.01\\
61.15	0.01\\
61.16	0.01\\
61.17	0.01\\
61.18	0.01\\
61.19	0.01\\
61.2	0.01\\
61.21	0.01\\
61.22	0.01\\
61.23	0.01\\
61.24	0.01\\
61.25	0.01\\
61.26	0.01\\
61.27	0.01\\
61.28	0.01\\
61.29	0.01\\
61.3	0.01\\
61.31	0.01\\
61.32	0.01\\
61.33	0.01\\
61.34	0.01\\
61.35	0.01\\
61.36	0.01\\
61.37	0.01\\
61.38	0.01\\
61.39	0.01\\
61.4	0.01\\
61.41	0.01\\
61.42	0.01\\
61.43	0.01\\
61.44	0.01\\
61.45	0.01\\
61.46	0.01\\
61.47	0.01\\
61.48	0.01\\
61.49	0.01\\
61.5	0.01\\
61.51	0.01\\
61.52	0.01\\
61.53	0.01\\
61.54	0.01\\
61.55	0.01\\
61.56	0.01\\
61.57	0.01\\
61.58	0.01\\
61.59	0.01\\
61.6	0.01\\
61.61	0.01\\
61.62	0.01\\
61.63	0.01\\
61.64	0.01\\
61.65	0.01\\
61.66	0.01\\
61.67	0.01\\
61.68	0.01\\
61.69	0.01\\
61.7	0.01\\
61.71	0.01\\
61.72	0.01\\
61.73	0.01\\
61.74	0.01\\
61.75	0.01\\
61.76	0.01\\
61.77	0.01\\
61.78	0.01\\
61.79	0.01\\
61.8	0.01\\
61.81	0.01\\
61.82	0.01\\
61.83	0.01\\
61.84	0.01\\
61.85	0.01\\
61.86	0.01\\
61.87	0.01\\
61.88	0.01\\
61.89	0.01\\
61.9	0.01\\
61.91	0.01\\
61.92	0.01\\
61.93	0.01\\
61.94	0.01\\
61.95	0.01\\
61.96	0.01\\
61.97	0.01\\
61.98	0.01\\
61.99	0.01\\
62	0.01\\
62.01	0.01\\
62.02	0.01\\
62.03	0.01\\
62.04	0.01\\
62.05	0.01\\
62.06	0.01\\
62.07	0.01\\
62.08	0.01\\
62.09	0.01\\
62.1	0.01\\
62.11	0.01\\
62.12	0.01\\
62.13	0.01\\
62.14	0.01\\
62.15	0.01\\
62.16	0.01\\
62.17	0.01\\
62.18	0.01\\
62.19	0.01\\
62.2	0.01\\
62.21	0.01\\
62.22	0.01\\
62.23	0.01\\
62.24	0.01\\
62.25	0.01\\
62.26	0.01\\
62.27	0.01\\
62.28	0.01\\
62.29	0.01\\
62.3	0.01\\
62.31	0.01\\
62.32	0.01\\
62.33	0.01\\
62.34	0.01\\
62.35	0.01\\
62.36	0.01\\
62.37	0.01\\
62.38	0.01\\
62.39	0.01\\
62.4	0.01\\
62.41	0.01\\
62.42	0.01\\
62.43	0.01\\
62.44	0.01\\
62.45	0.01\\
62.46	0.01\\
62.47	0.01\\
62.48	0.01\\
62.49	0.01\\
62.5	0.01\\
62.51	0.01\\
62.52	0.01\\
62.53	0.01\\
62.54	0.01\\
62.55	0.01\\
62.56	0.01\\
62.57	0.01\\
62.58	0.01\\
62.59	0.01\\
62.6	0.01\\
62.61	0.01\\
62.62	0.01\\
62.63	0.01\\
62.64	0.01\\
62.65	0.01\\
62.66	0.01\\
62.67	0.01\\
62.68	0.01\\
62.69	0.01\\
62.7	0.01\\
62.71	0.01\\
62.72	0.01\\
62.73	0.01\\
62.74	0.01\\
62.75	0.01\\
62.76	0.01\\
62.77	0.01\\
62.78	0.01\\
62.79	0.01\\
62.8	0.01\\
62.81	0.01\\
62.82	0.01\\
62.83	0.01\\
62.84	0.01\\
62.85	0.01\\
62.86	0.01\\
62.87	0.01\\
62.88	0.01\\
62.89	0.01\\
62.9	0.01\\
62.91	0.01\\
62.92	0.01\\
62.93	0.01\\
62.94	0.01\\
62.95	0.01\\
62.96	0.01\\
62.97	0.01\\
62.98	0.01\\
62.99	0.01\\
63	0.01\\
63.01	0.01\\
63.02	0.01\\
63.03	0.01\\
63.04	0.01\\
63.05	0.01\\
63.06	0.01\\
63.07	0.01\\
63.08	0.01\\
63.09	0.01\\
63.1	0.01\\
63.11	0.01\\
63.12	0.01\\
63.13	0.01\\
63.14	0.01\\
63.15	0.01\\
63.16	0.01\\
63.17	0.01\\
63.18	0.01\\
63.19	0.01\\
63.2	0.01\\
63.21	0.01\\
63.22	0.01\\
63.23	0.01\\
63.24	0.01\\
63.25	0.01\\
63.26	0.01\\
63.27	0.01\\
63.28	0.01\\
63.29	0.01\\
63.3	0.01\\
63.31	0.01\\
63.32	0.01\\
63.33	0.01\\
63.34	0.01\\
63.35	0.01\\
63.36	0.01\\
63.37	0.01\\
63.38	0.01\\
63.39	0.01\\
63.4	0.01\\
63.41	0.01\\
63.42	0.01\\
63.43	0.01\\
63.44	0.01\\
63.45	0.01\\
63.46	0.01\\
63.47	0.01\\
63.48	0.01\\
63.49	0.01\\
63.5	0.01\\
63.51	0.01\\
63.52	0.01\\
63.53	0.01\\
63.54	0.01\\
63.55	0.01\\
63.56	0.01\\
63.57	0.01\\
63.58	0.01\\
63.59	0.01\\
63.6	0.01\\
63.61	0.01\\
63.62	0.01\\
63.63	0.01\\
63.64	0.01\\
63.65	0.01\\
63.66	0.01\\
63.67	0.01\\
63.68	0.01\\
63.69	0.01\\
63.7	0.01\\
63.71	0.01\\
63.72	0.01\\
63.73	0.01\\
63.74	0.01\\
63.75	0.01\\
63.76	0.01\\
63.77	0.01\\
63.78	0.01\\
63.79	0.01\\
63.8	0.01\\
63.81	0.01\\
63.82	0.01\\
63.83	0.01\\
63.84	0.01\\
63.85	0.01\\
63.86	0.01\\
63.87	0.01\\
63.88	0.01\\
63.89	0.01\\
63.9	0.01\\
63.91	0.01\\
63.92	0.01\\
63.93	0.01\\
63.94	0.01\\
63.95	0.01\\
63.96	0.01\\
63.97	0.01\\
63.98	0.01\\
63.99	0.01\\
64	0.01\\
64.01	0.01\\
64.02	0.01\\
64.03	0.01\\
64.04	0.01\\
64.05	0.01\\
64.06	0.01\\
64.07	0.01\\
64.08	0.01\\
64.09	0.01\\
64.1	0.01\\
64.11	0.01\\
64.12	0.01\\
64.13	0.01\\
64.14	0.01\\
64.15	0.01\\
64.16	0.01\\
64.17	0.01\\
64.18	0.01\\
64.19	0.01\\
64.2	0.01\\
64.21	0.01\\
64.22	0.01\\
64.23	0.01\\
64.24	0.01\\
64.25	0.01\\
64.26	0.01\\
64.27	0.01\\
64.28	0.01\\
64.29	0.01\\
64.3	0.01\\
64.31	0.01\\
64.32	0.01\\
64.33	0.01\\
64.34	0.01\\
64.35	0.01\\
64.36	0.01\\
64.37	0.01\\
64.38	0.01\\
64.39	0.01\\
64.4	0.01\\
64.41	0.01\\
64.42	0.01\\
64.43	0.01\\
64.44	0.01\\
64.45	0.01\\
64.46	0.01\\
64.47	0.01\\
64.48	0.01\\
64.49	0.01\\
64.5	0.01\\
64.51	0.01\\
64.52	0.01\\
64.53	0.01\\
64.54	0.01\\
64.55	0.01\\
64.56	0.01\\
64.57	0.01\\
64.58	0.01\\
64.59	0.01\\
64.6	0.01\\
64.61	0.01\\
64.62	0.01\\
64.63	0.01\\
64.64	0.01\\
64.65	0.01\\
64.66	0.01\\
64.67	0.01\\
64.68	0.01\\
64.69	0.01\\
64.7	0.01\\
64.71	0.01\\
64.72	0.01\\
64.73	0.01\\
64.74	0.01\\
64.75	0.01\\
64.76	0.01\\
64.77	0.01\\
64.78	0.01\\
64.79	0.01\\
64.8	0.01\\
64.81	0.01\\
64.82	0.01\\
64.83	0.01\\
64.84	0.01\\
64.85	0.01\\
64.86	0.01\\
64.87	0.01\\
64.88	0.01\\
64.89	0.01\\
64.9	0.01\\
64.91	0.01\\
64.92	0.01\\
64.93	0.01\\
64.94	0.01\\
64.95	0.01\\
64.96	0.01\\
64.97	0.01\\
64.98	0.01\\
64.99	0.01\\
65	0.01\\
65.01	0.01\\
65.02	0.01\\
65.03	0.01\\
65.04	0.01\\
65.05	0.01\\
65.06	0.01\\
65.07	0.01\\
65.08	0.01\\
65.09	0.01\\
65.1	0.01\\
65.11	0.01\\
65.12	0.01\\
65.13	0.01\\
65.14	0.01\\
65.15	0.01\\
65.16	0.01\\
65.17	0.01\\
65.18	0.01\\
65.19	0.01\\
65.2	0.01\\
65.21	0.01\\
65.22	0.01\\
65.23	0.01\\
65.24	0.01\\
65.25	0.01\\
65.26	0.01\\
65.27	0.01\\
65.28	0.01\\
65.29	0.01\\
65.3	0.01\\
65.31	0.01\\
65.32	0.01\\
65.33	0.01\\
65.34	0.01\\
65.35	0.01\\
65.36	0.01\\
65.37	0.01\\
65.38	0.01\\
65.39	0.01\\
65.4	0.01\\
65.41	0.01\\
65.42	0.01\\
65.43	0.01\\
65.44	0.01\\
65.45	0.01\\
65.46	0.01\\
65.47	0.01\\
65.48	0.01\\
65.49	0.01\\
65.5	0.01\\
65.51	0.01\\
65.52	0.01\\
65.53	0.01\\
65.54	0.01\\
65.55	0.01\\
65.56	0.01\\
65.57	0.01\\
65.58	0.01\\
65.59	0.01\\
65.6	0.01\\
65.61	0.01\\
65.62	0.01\\
65.63	0.01\\
65.64	0.01\\
65.65	0.01\\
65.66	0.01\\
65.67	0.01\\
65.68	0.01\\
65.69	0.01\\
65.7	0.01\\
65.71	0.01\\
65.72	0.01\\
65.73	0.01\\
65.74	0.01\\
65.75	0.01\\
65.76	0.01\\
65.77	0.01\\
65.78	0.01\\
65.79	0.01\\
65.8	0.01\\
65.81	0.01\\
65.82	0.01\\
65.83	0.01\\
65.84	0.01\\
65.85	0.01\\
65.86	0.01\\
65.87	0.01\\
65.88	0.01\\
65.89	0.01\\
65.9	0.01\\
65.91	0.01\\
65.92	0.01\\
65.93	0.01\\
65.94	0.01\\
65.95	0.01\\
65.96	0.01\\
65.97	0.01\\
65.98	0.01\\
65.99	0.01\\
66	0.01\\
66.01	0.01\\
66.02	0.01\\
66.03	0.01\\
66.04	0.01\\
66.05	0.01\\
66.06	0.01\\
66.07	0.01\\
66.08	0.01\\
66.09	0.01\\
66.1	0.01\\
66.11	0.01\\
66.12	0.01\\
66.13	0.01\\
66.14	0.01\\
66.15	0.01\\
66.16	0.01\\
66.17	0.01\\
66.18	0.01\\
66.19	0.01\\
66.2	0.01\\
66.21	0.01\\
66.22	0.01\\
66.23	0.01\\
66.24	0.01\\
66.25	0.01\\
66.26	0.01\\
66.27	0.01\\
66.28	0.01\\
66.29	0.01\\
66.3	0.01\\
66.31	0.01\\
66.32	0.01\\
66.33	0.01\\
66.34	0.01\\
66.35	0.01\\
66.36	0.01\\
66.37	0.01\\
66.38	0.01\\
66.39	0.01\\
66.4	0.01\\
66.41	0.01\\
66.42	0.01\\
66.43	0.01\\
66.44	0.01\\
66.45	0.01\\
66.46	0.01\\
66.47	0.01\\
66.48	0.01\\
66.49	0.01\\
66.5	0.01\\
66.51	0.01\\
66.52	0.01\\
66.53	0.01\\
66.54	0.01\\
66.55	0.01\\
66.56	0.01\\
66.57	0.01\\
66.58	0.01\\
66.59	0.01\\
66.6	0.01\\
66.61	0.01\\
66.62	0.01\\
66.63	0.01\\
66.64	0.01\\
66.65	0.01\\
66.66	0.01\\
66.67	0.01\\
66.68	0.01\\
66.69	0.01\\
66.7	0.01\\
66.71	0.01\\
66.72	0.01\\
66.73	0.01\\
66.74	0.01\\
66.75	0.01\\
66.76	0.01\\
66.77	0.01\\
66.78	0.01\\
66.79	0.01\\
66.8	0.01\\
66.81	0.01\\
66.82	0.01\\
66.83	0.01\\
66.84	0.01\\
66.85	0.01\\
66.86	0.01\\
66.87	0.01\\
66.88	0.01\\
66.89	0.01\\
66.9	0.01\\
66.91	0.01\\
66.92	0.01\\
66.93	0.01\\
66.94	0.01\\
66.95	0.01\\
66.96	0.01\\
66.97	0.01\\
66.98	0.01\\
66.99	0.01\\
67	0.01\\
67.01	0.01\\
67.02	0.01\\
67.03	0.01\\
67.04	0.01\\
67.05	0.01\\
67.06	0.01\\
67.07	0.01\\
67.08	0.01\\
67.09	0.01\\
67.1	0.01\\
67.11	0.01\\
67.12	0.01\\
67.13	0.01\\
67.14	0.01\\
67.15	0.01\\
67.16	0.01\\
67.17	0.01\\
67.18	0.01\\
67.19	0.01\\
67.2	0.01\\
67.21	0.01\\
67.22	0.01\\
67.23	0.01\\
67.24	0.01\\
67.25	0.01\\
67.26	0.01\\
67.27	0.01\\
67.28	0.01\\
67.29	0.01\\
67.3	0.01\\
67.31	0.01\\
67.32	0.01\\
67.33	0.01\\
67.34	0.01\\
67.35	0.01\\
67.36	0.01\\
67.37	0.01\\
67.38	0.01\\
67.39	0.01\\
67.4	0.01\\
67.41	0.01\\
67.42	0.01\\
67.43	0.01\\
67.44	0.01\\
67.45	0.01\\
67.46	0.01\\
67.47	0.01\\
67.48	0.01\\
67.49	0.01\\
67.5	0.01\\
67.51	0.01\\
67.52	0.01\\
67.53	0.01\\
67.54	0.01\\
67.55	0.01\\
67.56	0.01\\
67.57	0.01\\
67.58	0.01\\
67.59	0.01\\
67.6	0.01\\
67.61	0.01\\
67.62	0.01\\
67.63	0.01\\
67.64	0.01\\
67.65	0.01\\
67.66	0.01\\
67.67	0.01\\
67.68	0.01\\
67.69	0.01\\
67.7	0.01\\
67.71	0.01\\
67.72	0.01\\
67.73	0.01\\
67.74	0.01\\
67.75	0.01\\
67.76	0.01\\
67.77	0.01\\
67.78	0.01\\
67.79	0.01\\
67.8	0.01\\
67.81	0.01\\
67.82	0.01\\
67.83	0.01\\
67.84	0.01\\
67.85	0.01\\
67.86	0.01\\
67.87	0.01\\
67.88	0.01\\
67.89	0.01\\
67.9	0.01\\
67.91	0.01\\
67.92	0.01\\
67.93	0.01\\
67.94	0.01\\
67.95	0.01\\
67.96	0.01\\
67.97	0.01\\
67.98	0.01\\
67.99	0.01\\
68	0.01\\
68.01	0.01\\
68.02	0.01\\
68.03	0.01\\
68.04	0.01\\
68.05	0.01\\
68.06	0.01\\
68.07	0.01\\
68.08	0.01\\
68.09	0.01\\
68.1	0.01\\
68.11	0.01\\
68.12	0.01\\
68.13	0.01\\
68.14	0.01\\
68.15	0.01\\
68.16	0.01\\
68.17	0.01\\
68.18	0.01\\
68.19	0.01\\
68.2	0.01\\
68.21	0.01\\
68.22	0.01\\
68.23	0.01\\
68.24	0.01\\
68.25	0.01\\
68.26	0.01\\
68.27	0.01\\
68.28	0.01\\
68.29	0.01\\
68.3	0.01\\
68.31	0.01\\
68.32	0.01\\
68.33	0.01\\
68.34	0.01\\
68.35	0.01\\
68.36	0.01\\
68.37	0.01\\
68.38	0.01\\
68.39	0.01\\
68.4	0.01\\
68.41	0.01\\
68.42	0.01\\
68.43	0.01\\
68.44	0.01\\
68.45	0.01\\
68.46	0.01\\
68.47	0.01\\
68.48	0.01\\
68.49	0.01\\
68.5	0.01\\
68.51	0.01\\
68.52	0.01\\
68.53	0.01\\
68.54	0.01\\
68.55	0.01\\
68.56	0.01\\
68.57	0.01\\
68.58	0.01\\
68.59	0.01\\
68.6	0.01\\
68.61	0.01\\
68.62	0.01\\
68.63	0.01\\
68.64	0.01\\
68.65	0.01\\
68.66	0.01\\
68.67	0.01\\
68.68	0.01\\
68.69	0.01\\
68.7	0.01\\
68.71	0.01\\
68.72	0.01\\
68.73	0.01\\
68.74	0.01\\
68.75	0.01\\
68.76	0.01\\
68.77	0.01\\
68.78	0.01\\
68.79	0.01\\
68.8	0.01\\
68.81	0.01\\
68.82	0.01\\
68.83	0.01\\
68.84	0.01\\
68.85	0.01\\
68.86	0.01\\
68.87	0.01\\
68.88	0.01\\
68.89	0.01\\
68.9	0.01\\
68.91	0.01\\
68.92	0.01\\
68.93	0.01\\
68.94	0.01\\
68.95	0.01\\
68.96	0.01\\
68.97	0.01\\
68.98	0.01\\
68.99	0.01\\
69	0.01\\
69.01	0.01\\
69.02	0.01\\
69.03	0.01\\
69.04	0.01\\
69.05	0.01\\
69.06	0.01\\
69.07	0.01\\
69.08	0.01\\
69.09	0.01\\
69.1	0.01\\
69.11	0.01\\
69.12	0.01\\
69.13	0.01\\
69.14	0.01\\
69.15	0.01\\
69.16	0.01\\
69.17	0.01\\
69.18	0.01\\
69.19	0.01\\
69.2	0.01\\
69.21	0.01\\
69.22	0.01\\
69.23	0.01\\
69.24	0.01\\
69.25	0.01\\
69.26	0.01\\
69.27	0.01\\
69.28	0.01\\
69.29	0.01\\
69.3	0.01\\
69.31	0.01\\
69.32	0.01\\
69.33	0.01\\
69.34	0.01\\
69.35	0.01\\
69.36	0.01\\
69.37	0.01\\
69.38	0.01\\
69.39	0.01\\
69.4	0.01\\
69.41	0.01\\
69.42	0.01\\
69.43	0.01\\
69.44	0.01\\
69.45	0.01\\
69.46	0.01\\
69.47	0.01\\
69.48	0.01\\
69.49	0.01\\
69.5	0.01\\
69.51	0.01\\
69.52	0.01\\
69.53	0.01\\
69.54	0.01\\
69.55	0.01\\
69.56	0.01\\
69.57	0.01\\
69.58	0.01\\
69.59	0.01\\
69.6	0.01\\
69.61	0.01\\
69.62	0.01\\
69.63	0.01\\
69.64	0.01\\
69.65	0.01\\
69.66	0.01\\
69.67	0.01\\
69.68	0.01\\
69.69	0.01\\
69.7	0.01\\
69.71	0.01\\
69.72	0.01\\
69.73	0.01\\
69.74	0.01\\
69.75	0.01\\
69.76	0.01\\
69.77	0.01\\
69.78	0.01\\
69.79	0.01\\
69.8	0.01\\
69.81	0.01\\
69.82	0.01\\
69.83	0.01\\
69.84	0.01\\
69.85	0.01\\
69.86	0.01\\
69.87	0.01\\
69.88	0.01\\
69.89	0.01\\
69.9	0.01\\
69.91	0.01\\
69.92	0.01\\
69.93	0.01\\
69.94	0.01\\
69.95	0.01\\
69.96	0.01\\
69.97	0.01\\
69.98	0.01\\
69.99	0.01\\
70	0.01\\
70.01	0.01\\
70.02	0.01\\
70.03	0.01\\
70.04	0.01\\
70.05	0.01\\
70.06	0.01\\
70.07	0.01\\
70.08	0.01\\
70.09	0.01\\
70.1	0.01\\
70.11	0.01\\
70.12	0.01\\
70.13	0.01\\
70.14	0.01\\
70.15	0.01\\
70.16	0.01\\
70.17	0.01\\
70.18	0.01\\
70.19	0.01\\
70.2	0.01\\
70.21	0.01\\
70.22	0.01\\
70.23	0.01\\
70.24	0.01\\
70.25	0.01\\
70.26	0.01\\
70.27	0.01\\
70.28	0.01\\
70.29	0.01\\
70.3	0.01\\
70.31	0.01\\
70.32	0.01\\
70.33	0.01\\
70.34	0.01\\
70.35	0.01\\
70.36	0.01\\
70.37	0.01\\
70.38	0.01\\
70.39	0.01\\
70.4	0.01\\
70.41	0.01\\
70.42	0.01\\
70.43	0.01\\
70.44	0.01\\
70.45	0.01\\
70.46	0.01\\
70.47	0.01\\
70.48	0.01\\
70.49	0.01\\
70.5	0.01\\
70.51	0.01\\
70.52	0.01\\
70.53	0.01\\
70.54	0.01\\
70.55	0.01\\
70.56	0.01\\
70.57	0.01\\
70.58	0.01\\
70.59	0.01\\
70.6	0.01\\
70.61	0.01\\
70.62	0.01\\
70.63	0.01\\
70.64	0.01\\
70.65	0.01\\
70.66	0.01\\
70.67	0.01\\
70.68	0.01\\
70.69	0.01\\
70.7	0.01\\
70.71	0.01\\
70.72	0.01\\
70.73	0.01\\
70.74	0.01\\
70.75	0.01\\
70.76	0.01\\
70.77	0.01\\
70.78	0.01\\
70.79	0.01\\
70.8	0.01\\
70.81	0.01\\
70.82	0.01\\
70.83	0.01\\
70.84	0.01\\
70.85	0.01\\
70.86	0.01\\
70.87	0.01\\
70.88	0.01\\
70.89	0.01\\
70.9	0.01\\
70.91	0.01\\
70.92	0.01\\
70.93	0.01\\
70.94	0.01\\
70.95	0.01\\
70.96	0.01\\
70.97	0.01\\
70.98	0.01\\
70.99	0.01\\
71	0.01\\
71.01	0.01\\
71.02	0.01\\
71.03	0.01\\
71.04	0.01\\
71.05	0.01\\
71.06	0.01\\
71.07	0.01\\
71.08	0.01\\
71.09	0.01\\
71.1	0.01\\
71.11	0.01\\
71.12	0.01\\
71.13	0.01\\
71.14	0.01\\
71.15	0.01\\
71.16	0.01\\
71.17	0.01\\
71.18	0.01\\
71.19	0.01\\
71.2	0.01\\
71.21	0.01\\
71.22	0.01\\
71.23	0.01\\
71.24	0.01\\
71.25	0.01\\
71.26	0.01\\
71.27	0.01\\
71.28	0.01\\
71.29	0.01\\
71.3	0.01\\
71.31	0.01\\
71.32	0.01\\
71.33	0.01\\
71.34	0.01\\
71.35	0.01\\
71.36	0.01\\
71.37	0.01\\
71.38	0.01\\
71.39	0.01\\
71.4	0.01\\
71.41	0.01\\
71.42	0.01\\
71.43	0.01\\
71.44	0.01\\
71.45	0.01\\
71.46	0.01\\
71.47	0.01\\
71.48	0.01\\
71.49	0.01\\
71.5	0.01\\
71.51	0.01\\
71.52	0.01\\
71.53	0.01\\
71.54	0.01\\
71.55	0.01\\
71.56	0.01\\
71.57	0.01\\
71.58	0.01\\
71.59	0.01\\
71.6	0.01\\
71.61	0.01\\
71.62	0.01\\
71.63	0.01\\
71.64	0.01\\
71.65	0.01\\
71.66	0.01\\
71.67	0.01\\
71.68	0.01\\
71.69	0.01\\
71.7	0.01\\
71.71	0.01\\
71.72	0.01\\
71.73	0.01\\
71.74	0.01\\
71.75	0.01\\
71.76	0.01\\
71.77	0.01\\
71.78	0.01\\
71.79	0.01\\
71.8	0.01\\
71.81	0.01\\
71.82	0.01\\
71.83	0.01\\
71.84	0.01\\
71.85	0.01\\
71.86	0.01\\
71.87	0.01\\
71.88	0.01\\
71.89	0.01\\
71.9	0.01\\
71.91	0.01\\
71.92	0.01\\
71.93	0.01\\
71.94	0.01\\
71.95	0.01\\
71.96	0.01\\
71.97	0.01\\
71.98	0.01\\
71.99	0.01\\
72	0.01\\
72.01	0.01\\
72.02	0.01\\
72.03	0.01\\
72.04	0.01\\
72.05	0.01\\
72.06	0.01\\
72.07	0.01\\
72.08	0.01\\
72.09	0.01\\
72.1	0.01\\
72.11	0.01\\
72.12	0.01\\
72.13	0.01\\
72.14	0.01\\
72.15	0.01\\
72.16	0.01\\
72.17	0.01\\
72.18	0.01\\
72.19	0.01\\
72.2	0.01\\
72.21	0.01\\
72.22	0.01\\
72.23	0.01\\
72.24	0.01\\
72.25	0.01\\
72.26	0.01\\
72.27	0.01\\
72.28	0.01\\
72.29	0.01\\
72.3	0.01\\
72.31	0.01\\
72.32	0.01\\
72.33	0.01\\
72.34	0.01\\
72.35	0.01\\
72.36	0.01\\
72.37	0.01\\
72.38	0.01\\
72.39	0.01\\
72.4	0.01\\
72.41	0.01\\
72.42	0.01\\
72.43	0.01\\
72.44	0.01\\
72.45	0.01\\
72.46	0.01\\
72.47	0.01\\
72.48	0.01\\
72.49	0.01\\
72.5	0.01\\
72.51	0.01\\
72.52	0.01\\
72.53	0.01\\
72.54	0.01\\
72.55	0.01\\
72.56	0.01\\
72.57	0.01\\
72.58	0.01\\
72.59	0.01\\
72.6	0.01\\
72.61	0.01\\
72.62	0.01\\
72.63	0.01\\
72.64	0.01\\
72.65	0.01\\
72.66	0.01\\
72.67	0.01\\
72.68	0.01\\
72.69	0.01\\
72.7	0.01\\
72.71	0.01\\
72.72	0.01\\
72.73	0.01\\
72.74	0.01\\
72.75	0.01\\
72.76	0.01\\
72.77	0.01\\
72.78	0.01\\
72.79	0.01\\
72.8	0.01\\
72.81	0.01\\
72.82	0.01\\
72.83	0.01\\
72.84	0.01\\
72.85	0.01\\
72.86	0.01\\
72.87	0.01\\
72.88	0.01\\
72.89	0.01\\
72.9	0.01\\
72.91	0.01\\
72.92	0.01\\
72.93	0.01\\
72.94	0.01\\
72.95	0.01\\
72.96	0.01\\
72.97	0.01\\
72.98	0.01\\
72.99	0.01\\
73	0.01\\
73.01	0.01\\
73.02	0.01\\
73.03	0.01\\
73.04	0.01\\
73.05	0.01\\
73.06	0.01\\
73.07	0.01\\
73.08	0.01\\
73.09	0.01\\
73.1	0.01\\
73.11	0.01\\
73.12	0.01\\
73.13	0.01\\
73.14	0.01\\
73.15	0.01\\
73.16	0.01\\
73.17	0.01\\
73.18	0.01\\
73.19	0.01\\
73.2	0.01\\
73.21	0.01\\
73.22	0.01\\
73.23	0.01\\
73.24	0.01\\
73.25	0.01\\
73.26	0.01\\
73.27	0.01\\
73.28	0.01\\
73.29	0.01\\
73.3	0.01\\
73.31	0.01\\
73.32	0.01\\
73.33	0.01\\
73.34	0.01\\
73.35	0.01\\
73.36	0.01\\
73.37	0.01\\
73.38	0.01\\
73.39	0.01\\
73.4	0.01\\
73.41	0.01\\
73.42	0.01\\
73.43	0.01\\
73.44	0.01\\
73.45	0.01\\
73.46	0.01\\
73.47	0.01\\
73.48	0.01\\
73.49	0.01\\
73.5	0.01\\
73.51	0.01\\
73.52	0.01\\
73.53	0.01\\
73.54	0.01\\
73.55	0.01\\
73.56	0.01\\
73.57	0.01\\
73.58	0.01\\
73.59	0.01\\
73.6	0.01\\
73.61	0.01\\
73.62	0.01\\
73.63	0.01\\
73.64	0.01\\
73.65	0.01\\
73.66	0.01\\
73.67	0.01\\
73.68	0.01\\
73.69	0.01\\
73.7	0.01\\
73.71	0.01\\
73.72	0.01\\
73.73	0.01\\
73.74	0.01\\
73.75	0.01\\
73.76	0.01\\
73.77	0.01\\
73.78	0.01\\
73.79	0.01\\
73.8	0.01\\
73.81	0.01\\
73.82	0.01\\
73.83	0.01\\
73.84	0.01\\
73.85	0.01\\
73.86	0.01\\
73.87	0.01\\
73.88	0.01\\
73.89	0.01\\
73.9	0.01\\
73.91	0.01\\
73.92	0.01\\
73.93	0.01\\
73.94	0.01\\
73.95	0.01\\
73.96	0.01\\
73.97	0.01\\
73.98	0.01\\
73.99	0.01\\
74	0.01\\
74.01	0.01\\
74.02	0.01\\
74.03	0.01\\
74.04	0.01\\
74.05	0.01\\
74.06	0.01\\
74.07	0.01\\
74.08	0.01\\
74.09	0.01\\
74.1	0.01\\
74.11	0.01\\
74.12	0.01\\
74.13	0.01\\
74.14	0.01\\
74.15	0.01\\
74.16	0.01\\
74.17	0.01\\
74.18	0.01\\
74.19	0.01\\
74.2	0.01\\
74.21	0.01\\
74.22	0.01\\
74.23	0.01\\
74.24	0.01\\
74.25	0.01\\
74.26	0.01\\
74.27	0.01\\
74.28	0.01\\
74.29	0.01\\
74.3	0.01\\
74.31	0.01\\
74.32	0.01\\
74.33	0.01\\
74.34	0.01\\
74.35	0.01\\
74.36	0.01\\
74.37	0.01\\
74.38	0.01\\
74.39	0.01\\
74.4	0.01\\
74.41	0.01\\
74.42	0.01\\
74.43	0.01\\
74.44	0.01\\
74.45	0.01\\
74.46	0.01\\
74.47	0.01\\
74.48	0.01\\
74.49	0.01\\
74.5	0.01\\
74.51	0.01\\
74.52	0.01\\
74.53	0.01\\
74.54	0.01\\
74.55	0.01\\
74.56	0.01\\
74.57	0.01\\
74.58	0.01\\
74.59	0.01\\
74.6	0.01\\
74.61	0.01\\
74.62	0.01\\
74.63	0.01\\
74.64	0.01\\
74.65	0.01\\
74.66	0.01\\
74.67	0.01\\
74.68	0.01\\
74.69	0.01\\
74.7	0.01\\
74.71	0.01\\
74.72	0.01\\
74.73	0.01\\
74.74	0.01\\
74.75	0.01\\
74.76	0.01\\
74.77	0.01\\
74.78	0.01\\
74.79	0.01\\
74.8	0.01\\
74.81	0.01\\
74.82	0.01\\
74.83	0.01\\
74.84	0.01\\
74.85	0.01\\
74.86	0.01\\
74.87	0.01\\
74.88	0.01\\
74.89	0.01\\
74.9	0.01\\
74.91	0.01\\
74.92	0.01\\
74.93	0.01\\
74.94	0.01\\
74.95	0.01\\
74.96	0.01\\
74.97	0.01\\
74.98	0.01\\
74.99	0.01\\
75	0.01\\
75.01	0.01\\
75.02	0.01\\
75.03	0.01\\
75.04	0.01\\
75.05	0.01\\
75.06	0.01\\
75.07	0.01\\
75.08	0.01\\
75.09	0.01\\
75.1	0.01\\
75.11	0.01\\
75.12	0.01\\
75.13	0.01\\
75.14	0.01\\
75.15	0.01\\
75.16	0.01\\
75.17	0.01\\
75.18	0.01\\
75.19	0.01\\
75.2	0.01\\
75.21	0.01\\
75.22	0.01\\
75.23	0.01\\
75.24	0.01\\
75.25	0.01\\
75.26	0.01\\
75.27	0.01\\
75.28	0.01\\
75.29	0.01\\
75.3	0.01\\
75.31	0.01\\
75.32	0.01\\
75.33	0.01\\
75.34	0.01\\
75.35	0.01\\
75.36	0.01\\
75.37	0.01\\
75.38	0.01\\
75.39	0.01\\
75.4	0.01\\
75.41	0.01\\
75.42	0.01\\
75.43	0.01\\
75.44	0.01\\
75.45	0.01\\
75.46	0.01\\
75.47	0.01\\
75.48	0.01\\
75.49	0.01\\
75.5	0.01\\
75.51	0.01\\
75.52	0.01\\
75.53	0.01\\
75.54	0.01\\
75.55	0.01\\
75.56	0.01\\
75.57	0.01\\
75.58	0.01\\
75.59	0.01\\
75.6	0.01\\
75.61	0.01\\
75.62	0.01\\
75.63	0.01\\
75.64	0.01\\
75.65	0.01\\
75.66	0.01\\
75.67	0.01\\
75.68	0.01\\
75.69	0.01\\
75.7	0.01\\
75.71	0.01\\
75.72	0.01\\
75.73	0.01\\
75.74	0.01\\
75.75	0.01\\
75.76	0.01\\
75.77	0.01\\
75.78	0.01\\
75.79	0.01\\
75.8	0.01\\
75.81	0.01\\
75.82	0.01\\
75.83	0.01\\
75.84	0.01\\
75.85	0.01\\
75.86	0.01\\
75.87	0.01\\
75.88	0.01\\
75.89	0.01\\
75.9	0.01\\
75.91	0.01\\
75.92	0.01\\
75.93	0.01\\
75.94	0.01\\
75.95	0.01\\
75.96	0.01\\
75.97	0.01\\
75.98	0.01\\
75.99	0.01\\
76	0.01\\
76.01	0.01\\
76.02	0.01\\
76.03	0.01\\
76.04	0.01\\
76.05	0.01\\
76.06	0.01\\
76.07	0.01\\
76.08	0.01\\
76.09	0.01\\
76.1	0.01\\
76.11	0.01\\
76.12	0.01\\
76.13	0.01\\
76.14	0.01\\
76.15	0.01\\
76.16	0.01\\
76.17	0.01\\
76.18	0.01\\
76.19	0.01\\
76.2	0.01\\
76.21	0.01\\
76.22	0.01\\
76.23	0.01\\
76.24	0.01\\
76.25	0.01\\
76.26	0.01\\
76.27	0.01\\
76.28	0.01\\
76.29	0.01\\
76.3	0.01\\
76.31	0.01\\
76.32	0.01\\
76.33	0.01\\
76.34	0.01\\
76.35	0.01\\
76.36	0.01\\
76.37	0.01\\
76.38	0.01\\
76.39	0.01\\
76.4	0.01\\
76.41	0.01\\
76.42	0.01\\
76.43	0.01\\
76.44	0.01\\
76.45	0.01\\
76.46	0.01\\
76.47	0.01\\
76.48	0.01\\
76.49	0.01\\
76.5	0.01\\
76.51	0.01\\
76.52	0.01\\
76.53	0.01\\
76.54	0.01\\
76.55	0.01\\
76.56	0.01\\
76.57	0.01\\
76.58	0.01\\
76.59	0.01\\
76.6	0.01\\
76.61	0.01\\
76.62	0.01\\
76.63	0.01\\
76.64	0.01\\
76.65	0.01\\
76.66	0.01\\
76.67	0.01\\
76.68	0.01\\
76.69	0.01\\
76.7	0.01\\
76.71	0.01\\
76.72	0.01\\
76.73	0.01\\
76.74	0.01\\
76.75	0.01\\
76.76	0.01\\
76.77	0.01\\
76.78	0.01\\
76.79	0.01\\
76.8	0.01\\
76.81	0.01\\
76.82	0.01\\
76.83	0.01\\
76.84	0.01\\
76.85	0.01\\
76.86	0.01\\
76.87	0.01\\
76.88	0.01\\
76.89	0.01\\
76.9	0.01\\
76.91	0.01\\
76.92	0.01\\
76.93	0.01\\
76.94	0.01\\
76.95	0.01\\
76.96	0.01\\
76.97	0.01\\
76.98	0.01\\
76.99	0.01\\
77	0.01\\
77.01	0.01\\
77.02	0.01\\
77.03	0.01\\
77.04	0.01\\
77.05	0.01\\
77.06	0.01\\
77.07	0.01\\
77.08	0.01\\
77.09	0.01\\
77.1	0.01\\
77.11	0.01\\
77.12	0.01\\
77.13	0.01\\
77.14	0.01\\
77.15	0.01\\
77.16	0.01\\
77.17	0.01\\
77.18	0.01\\
77.19	0.01\\
77.2	0.01\\
77.21	0.01\\
77.22	0.01\\
77.23	0.01\\
77.24	0.01\\
77.25	0.01\\
77.26	0.01\\
77.27	0.01\\
77.28	0.01\\
77.29	0.01\\
77.3	0.01\\
77.31	0.01\\
77.32	0.01\\
77.33	0.01\\
77.34	0.01\\
77.35	0.01\\
77.36	0.01\\
77.37	0.01\\
77.38	0.01\\
77.39	0.01\\
77.4	0.01\\
77.41	0.01\\
77.42	0.01\\
77.43	0.01\\
77.44	0.01\\
77.45	0.01\\
77.46	0.01\\
77.47	0.01\\
77.48	0.01\\
77.49	0.01\\
77.5	0.01\\
77.51	0.01\\
77.52	0.01\\
77.53	0.01\\
77.54	0.01\\
77.55	0.01\\
77.56	0.01\\
77.57	0.01\\
77.58	0.01\\
77.59	0.01\\
77.6	0.01\\
77.61	0.01\\
77.62	0.01\\
77.63	0.01\\
77.64	0.01\\
77.65	0.01\\
77.66	0.01\\
77.67	0.01\\
77.68	0.01\\
77.69	0.01\\
77.7	0.01\\
77.71	0.01\\
77.72	0.01\\
77.73	0.01\\
77.74	0.01\\
77.75	0.01\\
77.76	0.01\\
77.77	0.01\\
77.78	0.01\\
77.79	0.01\\
77.8	0.01\\
77.81	0.01\\
77.82	0.01\\
77.83	0.01\\
77.84	0.01\\
77.85	0.01\\
77.86	0.01\\
77.87	0.01\\
77.88	0.01\\
77.89	0.01\\
77.9	0.01\\
77.91	0.01\\
77.92	0.01\\
77.93	0.01\\
77.94	0.01\\
77.95	0.01\\
77.96	0.01\\
77.97	0.01\\
77.98	0.01\\
77.99	0.01\\
78	0.01\\
78.01	0.01\\
78.02	0.01\\
78.03	0.01\\
78.04	0.01\\
78.05	0.01\\
78.06	0.01\\
78.07	0.01\\
78.08	0.01\\
78.09	0.01\\
78.1	0.01\\
78.11	0.01\\
78.12	0.01\\
78.13	0.01\\
78.14	0.01\\
78.15	0.01\\
78.16	0.01\\
78.17	0.01\\
78.18	0.01\\
78.19	0.01\\
78.2	0.01\\
78.21	0.01\\
78.22	0.01\\
78.23	0.01\\
78.24	0.01\\
78.25	0.01\\
78.26	0.01\\
78.27	0.01\\
78.28	0.01\\
78.29	0.01\\
78.3	0.01\\
78.31	0.01\\
78.32	0.01\\
78.33	0.01\\
78.34	0.01\\
78.35	0.01\\
78.36	0.01\\
78.37	0.01\\
78.38	0.01\\
78.39	0.01\\
78.4	0.01\\
78.41	0.01\\
78.42	0.01\\
78.43	0.01\\
78.44	0.01\\
78.45	0.01\\
78.46	0.01\\
78.47	0.01\\
78.48	0.01\\
78.49	0.01\\
78.5	0.01\\
78.51	0.01\\
78.52	0.01\\
78.53	0.01\\
78.54	0.01\\
78.55	0.01\\
78.56	0.01\\
78.57	0.01\\
78.58	0.01\\
78.59	0.01\\
78.6	0.01\\
78.61	0.01\\
78.62	0.01\\
78.63	0.01\\
78.64	0.01\\
78.65	0.01\\
78.66	0.01\\
78.67	0.01\\
78.68	0.01\\
78.69	0.01\\
78.7	0.01\\
78.71	0.01\\
78.72	0.01\\
78.73	0.01\\
78.74	0.01\\
78.75	0.01\\
78.76	0.01\\
78.77	0.01\\
78.78	0.01\\
78.79	0.01\\
78.8	0.01\\
78.81	0.01\\
78.82	0.01\\
78.83	0.01\\
78.84	0.01\\
78.85	0.01\\
78.86	0.01\\
78.87	0.01\\
78.88	0.01\\
78.89	0.01\\
78.9	0.01\\
78.91	0.01\\
78.92	0.01\\
78.93	0.01\\
78.94	0.01\\
78.95	0.01\\
78.96	0.01\\
78.97	0.01\\
78.98	0.01\\
78.99	0.01\\
79	0.01\\
79.01	0.01\\
79.02	0.01\\
79.03	0.01\\
79.04	0.01\\
79.05	0.01\\
79.06	0.01\\
79.07	0.01\\
79.08	0.01\\
79.09	0.01\\
79.1	0.01\\
79.11	0.01\\
79.12	0.01\\
79.13	0.01\\
79.14	0.01\\
79.15	0.01\\
79.16	0.01\\
79.17	0.01\\
79.18	0.01\\
79.19	0.01\\
79.2	0.01\\
79.21	0.01\\
79.22	0.01\\
79.23	0.01\\
79.24	0.01\\
79.25	0.01\\
79.26	0.01\\
79.27	0.01\\
79.28	0.01\\
79.29	0.01\\
79.3	0.01\\
79.31	0.01\\
79.32	0.01\\
79.33	0.01\\
79.34	0.01\\
79.35	0.01\\
79.36	0.01\\
79.37	0.01\\
79.38	0.01\\
79.39	0.01\\
79.4	0.01\\
79.41	0.01\\
79.42	0.01\\
79.43	0.01\\
79.44	0.01\\
79.45	0.01\\
79.46	0.01\\
79.47	0.01\\
79.48	0.01\\
79.49	0.01\\
79.5	0.01\\
79.51	0.01\\
79.52	0.01\\
79.53	0.01\\
79.54	0.01\\
79.55	0.01\\
79.56	0.01\\
79.57	0.01\\
79.58	0.01\\
79.59	0.01\\
79.6	0.01\\
79.61	0.01\\
79.62	0.01\\
79.63	0.01\\
79.64	0.01\\
79.65	0.01\\
79.66	0.01\\
79.67	0.01\\
79.68	0.01\\
79.69	0.01\\
79.7	0.01\\
79.71	0.01\\
79.72	0.01\\
79.73	0.01\\
79.74	0.01\\
79.75	0.01\\
79.76	0.01\\
79.77	0.01\\
79.78	0.01\\
79.79	0.01\\
79.8	0.01\\
79.81	0.01\\
79.82	0.01\\
79.83	0.01\\
79.84	0.01\\
79.85	0.01\\
79.86	0.01\\
79.87	0.01\\
79.88	0.01\\
79.89	0.01\\
79.9	0.01\\
79.91	0.01\\
79.92	0.01\\
79.93	0.01\\
79.94	0.01\\
79.95	0.01\\
79.96	0.01\\
79.97	0.01\\
79.98	0.01\\
79.99	0.01\\
80	0.01\\
80.01	0.01\\
};
\addplot [color=mycolor1,dashed]
  table[row sep=crcr]{%
80.01	0.01\\
80.02	0.01\\
80.03	0.01\\
80.04	0.01\\
80.05	0.01\\
80.06	0.01\\
80.07	0.01\\
80.08	0.01\\
80.09	0.01\\
80.1	0.01\\
80.11	0.01\\
80.12	0.01\\
80.13	0.01\\
80.14	0.01\\
80.15	0.01\\
80.16	0.01\\
80.17	0.01\\
80.18	0.01\\
80.19	0.01\\
80.2	0.01\\
80.21	0.01\\
80.22	0.01\\
80.23	0.01\\
80.24	0.01\\
80.25	0.01\\
80.26	0.01\\
80.27	0.01\\
80.28	0.01\\
80.29	0.01\\
80.3	0.01\\
80.31	0.01\\
80.32	0.01\\
80.33	0.01\\
80.34	0.01\\
80.35	0.01\\
80.36	0.01\\
80.37	0.01\\
80.38	0.01\\
80.39	0.01\\
80.4	0.01\\
80.41	0.01\\
80.42	0.01\\
80.43	0.01\\
80.44	0.01\\
80.45	0.01\\
80.46	0.01\\
80.47	0.01\\
80.48	0.01\\
80.49	0.01\\
80.5	0.01\\
80.51	0.01\\
80.52	0.01\\
80.53	0.01\\
80.54	0.01\\
80.55	0.01\\
80.56	0.01\\
80.57	0.01\\
80.58	0.01\\
80.59	0.01\\
80.6	0.01\\
80.61	0.01\\
80.62	0.01\\
80.63	0.01\\
80.64	0.01\\
80.65	0.01\\
80.66	0.01\\
80.67	0.01\\
80.68	0.01\\
80.69	0.01\\
80.7	0.01\\
80.71	0.01\\
80.72	0.01\\
80.73	0.01\\
80.74	0.01\\
80.75	0.01\\
80.76	0.01\\
80.77	0.01\\
80.78	0.01\\
80.79	0.01\\
80.8	0.01\\
80.81	0.01\\
80.82	0.01\\
80.83	0.01\\
80.84	0.01\\
80.85	0.01\\
80.86	0.01\\
80.87	0.01\\
80.88	0.01\\
80.89	0.01\\
80.9	0.01\\
80.91	0.01\\
80.92	0.01\\
80.93	0.01\\
80.94	0.01\\
80.95	0.01\\
80.96	0.01\\
80.97	0.01\\
80.98	0.01\\
80.99	0.01\\
81	0.01\\
81.01	0.01\\
81.02	0.01\\
81.03	0.01\\
81.04	0.01\\
81.05	0.01\\
81.06	0.01\\
81.07	0.01\\
81.08	0.01\\
81.09	0.01\\
81.1	0.01\\
81.11	0.01\\
81.12	0.01\\
81.13	0.01\\
81.14	0.01\\
81.15	0.01\\
81.16	0.01\\
81.17	0.01\\
81.18	0.01\\
81.19	0.01\\
81.2	0.01\\
81.21	0.01\\
81.22	0.01\\
81.23	0.01\\
81.24	0.01\\
81.25	0.01\\
81.26	0.01\\
81.27	0.01\\
81.28	0.01\\
81.29	0.01\\
81.3	0.01\\
81.31	0.01\\
81.32	0.01\\
81.33	0.01\\
81.34	0.01\\
81.35	0.01\\
81.36	0.01\\
81.37	0.01\\
81.38	0.01\\
81.39	0.01\\
81.4	0.01\\
81.41	0.01\\
81.42	0.01\\
81.43	0.01\\
81.44	0.01\\
81.45	0.01\\
81.46	0.01\\
81.47	0.01\\
81.48	0.01\\
81.49	0.01\\
81.5	0.01\\
81.51	0.01\\
81.52	0.01\\
81.53	0.01\\
81.54	0.01\\
81.55	0.01\\
81.56	0.01\\
81.57	0.01\\
81.58	0.01\\
81.59	0.01\\
81.6	0.01\\
81.61	0.01\\
81.62	0.01\\
81.63	0.01\\
81.64	0.01\\
81.65	0.01\\
81.66	0.01\\
81.67	0.01\\
81.68	0.01\\
81.69	0.01\\
81.7	0.01\\
81.71	0.01\\
81.72	0.01\\
81.73	0.01\\
81.74	0.01\\
81.75	0.01\\
81.76	0.01\\
81.77	0.01\\
81.78	0.01\\
81.79	0.01\\
81.8	0.01\\
81.81	0.01\\
81.82	0.01\\
81.83	0.01\\
81.84	0.01\\
81.85	0.01\\
81.86	0.01\\
81.87	0.01\\
81.88	0.01\\
81.89	0.01\\
81.9	0.01\\
81.91	0.01\\
81.92	0.01\\
81.93	0.01\\
81.94	0.01\\
81.95	0.01\\
81.96	0.01\\
81.97	0.01\\
81.98	0.01\\
81.99	0.01\\
82	0.01\\
82.01	0.01\\
82.02	0.01\\
82.03	0.01\\
82.04	0.01\\
82.05	0.01\\
82.06	0.01\\
82.07	0.01\\
82.08	0.01\\
82.09	0.01\\
82.1	0.01\\
82.11	0.01\\
82.12	0.01\\
82.13	0.01\\
82.14	0.01\\
82.15	0.01\\
82.16	0.01\\
82.17	0.01\\
82.18	0.01\\
82.19	0.01\\
82.2	0.01\\
82.21	0.01\\
82.22	0.01\\
82.23	0.01\\
82.24	0.01\\
82.25	0.01\\
82.26	0.01\\
82.27	0.01\\
82.28	0.01\\
82.29	0.01\\
82.3	0.01\\
82.31	0.01\\
82.32	0.01\\
82.33	0.01\\
82.34	0.01\\
82.35	0.01\\
82.36	0.01\\
82.37	0.01\\
82.38	0.01\\
82.39	0.01\\
82.4	0.01\\
82.41	0.01\\
82.42	0.01\\
82.43	0.01\\
82.44	0.01\\
82.45	0.01\\
82.46	0.01\\
82.47	0.01\\
82.48	0.01\\
82.49	0.01\\
82.5	0.01\\
82.51	0.01\\
82.52	0.01\\
82.53	0.01\\
82.54	0.01\\
82.55	0.01\\
82.56	0.01\\
82.57	0.01\\
82.58	0.01\\
82.59	0.01\\
82.6	0.01\\
82.61	0.01\\
82.62	0.01\\
82.63	0.01\\
82.64	0.01\\
82.65	0.01\\
82.66	0.01\\
82.67	0.01\\
82.68	0.01\\
82.69	0.01\\
82.7	0.01\\
82.71	0.01\\
82.72	0.01\\
82.73	0.01\\
82.74	0.01\\
82.75	0.01\\
82.76	0.01\\
82.77	0.01\\
82.78	0.01\\
82.79	0.01\\
82.8	0.01\\
82.81	0.01\\
82.82	0.01\\
82.83	0.01\\
82.84	0.01\\
82.85	0.01\\
82.86	0.01\\
82.87	0.01\\
82.88	0.01\\
82.89	0.01\\
82.9	0.01\\
82.91	0.01\\
82.92	0.01\\
82.93	0.01\\
82.94	0.01\\
82.95	0.01\\
82.96	0.01\\
82.97	0.01\\
82.98	0.01\\
82.99	0.01\\
83	0.01\\
83.01	0.01\\
83.02	0.01\\
83.03	0.01\\
83.04	0.01\\
83.05	0.01\\
83.06	0.01\\
83.07	0.01\\
83.08	0.01\\
83.09	0.01\\
83.1	0.01\\
83.11	0.01\\
83.12	0.01\\
83.13	0.01\\
83.14	0.01\\
83.15	0.01\\
83.16	0.01\\
83.17	0.01\\
83.18	0.01\\
83.19	0.01\\
83.2	0.01\\
83.21	0.01\\
83.22	0.01\\
83.23	0.01\\
83.24	0.01\\
83.25	0.01\\
83.26	0.01\\
83.27	0.01\\
83.28	0.01\\
83.29	0.01\\
83.3	0.01\\
83.31	0.01\\
83.32	0.01\\
83.33	0.01\\
83.34	0.01\\
83.35	0.01\\
83.36	0.01\\
83.37	0.01\\
83.38	0.01\\
83.39	0.01\\
83.4	0.01\\
83.41	0.01\\
83.42	0.01\\
83.43	0.01\\
83.44	0.01\\
83.45	0.01\\
83.46	0.01\\
83.47	0.01\\
83.48	0.01\\
83.49	0.01\\
83.5	0.01\\
83.51	0.01\\
83.52	0.01\\
83.53	0.01\\
83.54	0.01\\
83.55	0.01\\
83.56	0.01\\
83.57	0.01\\
83.58	0.01\\
83.59	0.01\\
83.6	0.01\\
83.61	0.01\\
83.62	0.01\\
83.63	0.01\\
83.64	0.01\\
83.65	0.01\\
83.66	0.01\\
83.67	0.01\\
83.68	0.01\\
83.69	0.01\\
83.7	0.01\\
83.71	0.01\\
83.72	0.01\\
83.73	0.01\\
83.74	0.01\\
83.75	0.01\\
83.76	0.01\\
83.77	0.01\\
83.78	0.01\\
83.79	0.01\\
83.8	0.01\\
83.81	0.01\\
83.82	0.01\\
83.83	0.01\\
83.84	0.01\\
83.85	0.01\\
83.86	0.01\\
83.87	0.01\\
83.88	0.01\\
83.89	0.01\\
83.9	0.01\\
83.91	0.01\\
83.92	0.01\\
83.93	0.01\\
83.94	0.01\\
83.95	0.01\\
83.96	0.01\\
83.97	0.01\\
83.98	0.01\\
83.99	0.01\\
84	0.01\\
84.01	0.01\\
84.02	0.01\\
84.03	0.01\\
84.04	0.01\\
84.05	0.01\\
84.06	0.01\\
84.07	0.01\\
84.08	0.01\\
84.09	0.01\\
84.1	0.01\\
84.11	0.01\\
84.12	0.01\\
84.13	0.01\\
84.14	0.01\\
84.15	0.01\\
84.16	0.01\\
84.17	0.01\\
84.18	0.01\\
84.19	0.01\\
84.2	0.01\\
84.21	0.01\\
84.22	0.01\\
84.23	0.01\\
84.24	0.01\\
84.25	0.01\\
84.26	0.01\\
84.27	0.01\\
84.28	0.01\\
84.29	0.01\\
84.3	0.01\\
84.31	0.01\\
84.32	0.01\\
84.33	0.01\\
84.34	0.01\\
84.35	0.01\\
84.36	0.01\\
84.37	0.01\\
84.38	0.01\\
84.39	0.01\\
84.4	0.01\\
84.41	0.01\\
84.42	0.01\\
84.43	0.01\\
84.44	0.01\\
84.45	0.01\\
84.46	0.01\\
84.47	0.01\\
84.48	0.01\\
84.49	0.01\\
84.5	0.01\\
84.51	0.01\\
84.52	0.01\\
84.53	0.01\\
84.54	0.01\\
84.55	0.01\\
84.56	0.01\\
84.57	0.01\\
84.58	0.01\\
84.59	0.01\\
84.6	0.01\\
84.61	0.01\\
84.62	0.01\\
84.63	0.01\\
84.64	0.01\\
84.65	0.01\\
84.66	0.01\\
84.67	0.01\\
84.68	0.01\\
84.69	0.01\\
84.7	0.01\\
84.71	0.01\\
84.72	0.01\\
84.73	0.01\\
84.74	0.01\\
84.75	0.01\\
84.76	0.01\\
84.77	0.01\\
84.78	0.01\\
84.79	0.01\\
84.8	0.01\\
84.81	0.01\\
84.82	0.01\\
84.83	0.01\\
84.84	0.01\\
84.85	0.01\\
84.86	0.01\\
84.87	0.01\\
84.88	0.01\\
84.89	0.01\\
84.9	0.01\\
84.91	0.01\\
84.92	0.01\\
84.93	0.01\\
84.94	0.01\\
84.95	0.01\\
84.96	0.01\\
84.97	0.01\\
84.98	0.01\\
84.99	0.01\\
85	0.01\\
85.01	0.01\\
85.02	0.01\\
85.03	0.01\\
85.04	0.01\\
85.05	0.01\\
85.06	0.01\\
85.07	0.01\\
85.08	0.01\\
85.09	0.01\\
85.1	0.01\\
85.11	0.01\\
85.12	0.01\\
85.13	0.01\\
85.14	0.01\\
85.15	0.01\\
85.16	0.01\\
85.17	0.01\\
85.18	0.01\\
85.19	0.01\\
85.2	0.01\\
85.21	0.01\\
85.22	0.01\\
85.23	0.01\\
85.24	0.01\\
85.25	0.01\\
85.26	0.01\\
85.27	0.01\\
85.28	0.01\\
85.29	0.01\\
85.3	0.01\\
85.31	0.01\\
85.32	0.01\\
85.33	0.01\\
85.34	0.01\\
85.35	0.01\\
85.36	0.01\\
85.37	0.01\\
85.38	0.01\\
85.39	0.01\\
85.4	0.01\\
85.41	0.01\\
85.42	0.01\\
85.43	0.01\\
85.44	0.01\\
85.45	0.01\\
85.46	0.01\\
85.47	0.01\\
85.48	0.01\\
85.49	0.01\\
85.5	0.01\\
85.51	0.01\\
85.52	0.01\\
85.53	0.01\\
85.54	0.01\\
85.55	0.01\\
85.56	0.01\\
85.57	0.01\\
85.58	0.01\\
85.59	0.01\\
85.6	0.01\\
85.61	0.01\\
85.62	0.01\\
85.63	0.01\\
85.64	0.01\\
85.65	0.01\\
85.66	0.01\\
85.67	0.01\\
85.68	0.01\\
85.69	0.01\\
85.7	0.01\\
85.71	0.01\\
85.72	0.01\\
85.73	0.01\\
85.74	0.01\\
85.75	0.01\\
85.76	0.01\\
85.77	0.01\\
85.78	0.01\\
85.79	0.01\\
85.8	0.01\\
85.81	0.01\\
85.82	0.01\\
85.83	0.01\\
85.84	0.01\\
85.85	0.01\\
85.86	0.01\\
85.87	0.01\\
85.88	0.01\\
85.89	0.01\\
85.9	0.01\\
85.91	0.01\\
85.92	0.01\\
85.93	0.01\\
85.94	0.01\\
85.95	0.01\\
85.96	0.01\\
85.97	0.01\\
85.98	0.01\\
85.99	0.01\\
86	0.01\\
86.01	0.01\\
86.02	0.01\\
86.03	0.01\\
86.04	0.01\\
86.05	0.01\\
86.06	0.01\\
86.07	0.01\\
86.08	0.01\\
86.09	0.01\\
86.1	0.01\\
86.11	0.01\\
86.12	0.01\\
86.13	0.01\\
86.14	0.01\\
86.15	0.01\\
86.16	0.01\\
86.17	0.01\\
86.18	0.01\\
86.19	0.01\\
86.2	0.01\\
86.21	0.01\\
86.22	0.01\\
86.23	0.01\\
86.24	0.01\\
86.25	0.01\\
86.26	0.01\\
86.27	0.01\\
86.28	0.01\\
86.29	0.01\\
86.3	0.01\\
86.31	0.01\\
86.32	0.01\\
86.33	0.01\\
86.34	0.01\\
86.35	0.01\\
86.36	0.01\\
86.37	0.01\\
86.38	0.01\\
86.39	0.01\\
86.4	0.01\\
86.41	0.01\\
86.42	0.01\\
86.43	0.01\\
86.44	0.01\\
86.45	0.01\\
86.46	0.01\\
86.47	0.01\\
86.48	0.01\\
86.49	0.01\\
86.5	0.01\\
86.51	0.01\\
86.52	0.01\\
86.53	0.01\\
86.54	0.01\\
86.55	0.01\\
86.56	0.01\\
86.57	0.01\\
86.58	0.01\\
86.59	0.01\\
86.6	0.01\\
86.61	0.01\\
86.62	0.01\\
86.63	0.01\\
86.64	0.01\\
86.65	0.01\\
86.66	0.01\\
86.67	0.01\\
86.68	0.01\\
86.69	0.01\\
86.7	0.01\\
86.71	0.01\\
86.72	0.01\\
86.73	0.01\\
86.74	0.01\\
86.75	0.01\\
86.76	0.01\\
86.77	0.01\\
86.78	0.01\\
86.79	0.01\\
86.8	0.01\\
86.81	0.01\\
86.82	0.01\\
86.83	0.01\\
86.84	0.01\\
86.85	0.01\\
86.86	0.01\\
86.87	0.01\\
86.88	0.01\\
86.89	0.01\\
86.9	0.01\\
86.91	0.01\\
86.92	0.01\\
86.93	0.01\\
86.94	0.01\\
86.95	0.01\\
86.96	0.01\\
86.97	0.01\\
86.98	0.01\\
86.99	0.01\\
87	0.01\\
87.01	0.01\\
87.02	0.01\\
87.03	0.01\\
87.04	0.01\\
87.05	0.01\\
87.06	0.01\\
87.07	0.01\\
87.08	0.01\\
87.09	0.01\\
87.1	0.01\\
87.11	0.01\\
87.12	0.01\\
87.13	0.01\\
87.14	0.01\\
87.15	0.01\\
87.16	0.01\\
87.17	0.01\\
87.18	0.01\\
87.19	0.01\\
87.2	0.01\\
87.21	0.01\\
87.22	0.01\\
87.23	0.01\\
87.24	0.01\\
87.25	0.01\\
87.26	0.01\\
87.27	0.01\\
87.28	0.01\\
87.29	0.01\\
87.3	0.01\\
87.31	0.01\\
87.32	0.01\\
87.33	0.01\\
87.34	0.01\\
87.35	0.01\\
87.36	0.01\\
87.37	0.01\\
87.38	0.01\\
87.39	0.01\\
87.4	0.01\\
87.41	0.01\\
87.42	0.01\\
87.43	0.01\\
87.44	0.01\\
87.45	0.01\\
87.46	0.01\\
87.47	0.01\\
87.48	0.01\\
87.49	0.01\\
87.5	0.01\\
87.51	0.01\\
87.52	0.01\\
87.53	0.01\\
87.54	0.01\\
87.55	0.01\\
87.56	0.01\\
87.57	0.01\\
87.58	0.01\\
87.59	0.01\\
87.6	0.01\\
87.61	0.01\\
87.62	0.01\\
87.63	0.01\\
87.64	0.01\\
87.65	0.01\\
87.66	0.01\\
87.67	0.01\\
87.68	0.01\\
87.69	0.01\\
87.7	0.01\\
87.71	0.01\\
87.72	0.01\\
87.73	0.01\\
87.74	0.01\\
87.75	0.01\\
87.76	0.01\\
87.77	0.01\\
87.78	0.01\\
87.79	0.01\\
87.8	0.01\\
87.81	0.01\\
87.82	0.01\\
87.83	0.01\\
87.84	0.01\\
87.85	0.01\\
87.86	0.01\\
87.87	0.01\\
87.88	0.01\\
87.89	0.01\\
87.9	0.01\\
87.91	0.01\\
87.92	0.01\\
87.93	0.01\\
87.94	0.01\\
87.95	0.01\\
87.96	0.01\\
87.97	0.01\\
87.98	0.01\\
87.99	0.01\\
88	0.01\\
88.01	0.01\\
88.02	0.01\\
88.03	0.01\\
88.04	0.01\\
88.05	0.01\\
88.06	0.01\\
88.07	0.01\\
88.08	0.01\\
88.09	0.01\\
88.1	0.01\\
88.11	0.01\\
88.12	0.01\\
88.13	0.01\\
88.14	0.01\\
88.15	0.01\\
88.16	0.01\\
88.17	0.01\\
88.18	0.01\\
88.19	0.01\\
88.2	0.01\\
88.21	0.01\\
88.22	0.01\\
88.23	0.01\\
88.24	0.01\\
88.25	0.01\\
88.26	0.01\\
88.27	0.01\\
88.28	0.01\\
88.29	0.01\\
88.3	0.01\\
88.31	0.01\\
88.32	0.01\\
88.33	0.01\\
88.34	0.01\\
88.35	0.01\\
88.36	0.01\\
88.37	0.01\\
88.38	0.01\\
88.39	0.01\\
88.4	0.01\\
88.41	0.01\\
88.42	0.01\\
88.43	0.01\\
88.44	0.01\\
88.45	0.01\\
88.46	0.01\\
88.47	0.01\\
88.48	0.01\\
88.49	0.01\\
88.5	0.01\\
88.51	0.01\\
88.52	0.01\\
88.53	0.01\\
88.54	0.01\\
88.55	0.01\\
88.56	0.01\\
88.57	0.01\\
88.58	0.01\\
88.59	0.01\\
88.6	0.01\\
88.61	0.01\\
88.62	0.01\\
88.63	0.01\\
88.64	0.01\\
88.65	0.01\\
88.66	0.01\\
88.67	0.01\\
88.68	0.01\\
88.69	0.01\\
88.7	0.01\\
88.71	0.01\\
88.72	0.01\\
88.73	0.01\\
88.74	0.01\\
88.75	0.01\\
88.76	0.01\\
88.77	0.01\\
88.78	0.01\\
88.79	0.01\\
88.8	0.01\\
88.81	0.01\\
88.82	0.01\\
88.83	0.01\\
88.84	0.01\\
88.85	0.01\\
88.86	0.01\\
88.87	0.01\\
88.88	0.01\\
88.89	0.01\\
88.9	0.01\\
88.91	0.01\\
88.92	0.01\\
88.93	0.01\\
88.94	0.01\\
88.95	0.01\\
88.96	0.01\\
88.97	0.01\\
88.98	0.01\\
88.99	0.01\\
89	0.01\\
89.01	0.01\\
89.02	0.01\\
89.03	0.01\\
89.04	0.01\\
89.05	0.01\\
89.06	0.01\\
89.07	0.01\\
89.08	0.01\\
89.09	0.01\\
89.1	0.01\\
89.11	0.01\\
89.12	0.01\\
89.13	0.01\\
89.14	0.01\\
89.15	0.01\\
89.16	0.01\\
89.17	0.01\\
89.18	0.01\\
89.19	0.01\\
89.2	0.01\\
89.21	0.01\\
89.22	0.01\\
89.23	0.01\\
89.24	0.01\\
89.25	0.01\\
89.26	0.01\\
89.27	0.01\\
89.28	0.01\\
89.29	0.01\\
89.3	0.01\\
89.31	0.01\\
89.32	0.01\\
89.33	0.01\\
89.34	0.01\\
89.35	0.01\\
89.36	0.01\\
89.37	0.01\\
89.38	0.01\\
89.39	0.01\\
89.4	0.01\\
89.41	0.01\\
89.42	0.01\\
89.43	0.01\\
89.44	0.01\\
89.45	0.01\\
89.46	0.01\\
89.47	0.01\\
89.48	0.01\\
89.49	0.01\\
89.5	0.01\\
89.51	0.01\\
89.52	0.01\\
89.53	0.01\\
89.54	0.01\\
89.55	0.01\\
89.56	0.01\\
89.57	0.01\\
89.58	0.01\\
89.59	0.01\\
89.6	0.01\\
89.61	0.01\\
89.62	0.01\\
89.63	0.01\\
89.64	0.01\\
89.65	0.01\\
89.66	0.01\\
89.67	0.01\\
89.68	0.01\\
89.69	0.01\\
89.7	0.01\\
89.71	0.01\\
89.72	0.01\\
89.73	0.01\\
89.74	0.01\\
89.75	0.01\\
89.76	0.01\\
89.77	0.01\\
89.78	0.01\\
89.79	0.01\\
89.8	0.01\\
89.81	0.01\\
89.82	0.01\\
89.83	0.01\\
89.84	0.01\\
89.85	0.01\\
89.86	0.01\\
89.87	0.01\\
89.88	0.01\\
89.89	0.01\\
89.9	0.01\\
89.91	0.01\\
89.92	0.01\\
89.93	0.01\\
89.94	0.01\\
89.95	0.01\\
89.96	0.01\\
89.97	0.01\\
89.98	0.01\\
89.99	0.01\\
90	0.01\\
90.01	0.01\\
90.02	0.01\\
90.03	0.01\\
90.04	0.01\\
90.05	0.01\\
90.06	0.01\\
90.07	0.01\\
90.08	0.01\\
90.09	0.01\\
90.1	0.01\\
90.11	0.01\\
90.12	0.01\\
90.13	0.01\\
90.14	0.01\\
90.15	0.01\\
90.16	0.01\\
90.17	0.01\\
90.18	0.01\\
90.19	0.01\\
90.2	0.01\\
90.21	0.01\\
90.22	0.01\\
90.23	0.01\\
90.24	0.01\\
90.25	0.01\\
90.26	0.01\\
90.27	0.01\\
90.28	0.01\\
90.29	0.01\\
90.3	0.01\\
90.31	0.01\\
90.32	0.01\\
90.33	0.01\\
90.34	0.01\\
90.35	0.01\\
90.36	0.01\\
90.37	0.01\\
90.38	0.01\\
90.39	0.01\\
90.4	0.01\\
90.41	0.01\\
90.42	0.01\\
90.43	0.01\\
90.44	0.01\\
90.45	0.01\\
90.46	0.01\\
90.47	0.01\\
90.48	0.01\\
90.49	0.01\\
90.5	0.01\\
90.51	0.01\\
90.52	0.01\\
90.53	0.01\\
90.54	0.01\\
90.55	0.01\\
90.56	0.01\\
90.57	0.01\\
90.58	0.01\\
90.59	0.01\\
90.6	0.01\\
90.61	0.01\\
90.62	0.01\\
90.63	0.01\\
90.64	0.01\\
90.65	0.01\\
90.66	0.01\\
90.67	0.01\\
90.68	0.01\\
90.69	0.01\\
90.7	0.01\\
90.71	0.01\\
90.72	0.01\\
90.73	0.01\\
90.74	0.01\\
90.75	0.01\\
90.76	0.01\\
90.77	0.01\\
90.78	0.01\\
90.79	0.01\\
90.8	0.01\\
90.81	0.01\\
90.82	0.01\\
90.83	0.01\\
90.84	0.01\\
90.85	0.01\\
90.86	0.01\\
90.87	0.01\\
90.88	0.01\\
90.89	0.01\\
90.9	0.01\\
90.91	0.01\\
90.92	0.01\\
90.93	0.01\\
90.94	0.01\\
90.95	0.01\\
90.96	0.01\\
90.97	0.01\\
90.98	0.01\\
90.99	0.01\\
91	0.01\\
91.01	0.01\\
91.02	0.01\\
91.03	0.01\\
91.04	0.01\\
91.05	0.01\\
91.06	0.01\\
91.07	0.01\\
91.08	0.01\\
91.09	0.01\\
91.1	0.01\\
91.11	0.01\\
91.12	0.01\\
91.13	0.01\\
91.14	0.01\\
91.15	0.01\\
91.16	0.01\\
91.17	0.01\\
91.18	0.01\\
91.19	0.01\\
91.2	0.01\\
91.21	0.01\\
91.22	0.01\\
91.23	0.01\\
91.24	0.01\\
91.25	0.01\\
91.26	0.01\\
91.27	0.01\\
91.28	0.01\\
91.29	0.01\\
91.3	0.01\\
91.31	0.01\\
91.32	0.01\\
91.33	0.01\\
91.34	0.01\\
91.35	0.01\\
91.36	0.01\\
91.37	0.01\\
91.38	0.01\\
91.39	0.01\\
91.4	0.01\\
91.41	0.01\\
91.42	0.01\\
91.43	0.01\\
91.44	0.01\\
91.45	0.01\\
91.46	0.01\\
91.47	0.01\\
91.48	0.01\\
91.49	0.01\\
91.5	0.01\\
91.51	0.01\\
91.52	0.01\\
91.53	0.01\\
91.54	0.01\\
91.55	0.01\\
91.56	0.01\\
91.57	0.01\\
91.58	0.01\\
91.59	0.01\\
91.6	0.01\\
91.61	0.01\\
91.62	0.01\\
91.63	0.01\\
91.64	0.01\\
91.65	0.01\\
91.66	0.01\\
91.67	0.01\\
91.68	0.01\\
91.69	0.01\\
91.7	0.01\\
91.71	0.01\\
91.72	0.01\\
91.73	0.01\\
91.74	0.01\\
91.75	0.01\\
91.76	0.01\\
91.77	0.01\\
91.78	0.01\\
91.79	0.01\\
91.8	0.01\\
91.81	0.01\\
91.82	0.01\\
91.83	0.01\\
91.84	0.01\\
91.85	0.01\\
91.86	0.01\\
91.87	0.01\\
91.88	0.01\\
91.89	0.01\\
91.9	0.01\\
91.91	0.01\\
91.92	0.01\\
91.93	0.01\\
91.94	0.01\\
91.95	0.01\\
91.96	0.01\\
91.97	0.01\\
91.98	0.01\\
91.99	0.01\\
92	0.01\\
92.01	0.01\\
92.02	0.01\\
92.03	0.01\\
92.04	0.01\\
92.05	0.01\\
92.06	0.01\\
92.07	0.01\\
92.08	0.01\\
92.09	0.01\\
92.1	0.01\\
92.11	0.01\\
92.12	0.01\\
92.13	0.01\\
92.14	0.01\\
92.15	0.01\\
92.16	0.01\\
92.17	0.01\\
92.18	0.01\\
92.19	0.01\\
92.2	0.01\\
92.21	0.01\\
92.22	0.01\\
92.23	0.01\\
92.24	0.01\\
92.25	0.01\\
92.26	0.01\\
92.27	0.01\\
92.28	0.01\\
92.29	0.01\\
92.3	0.01\\
92.31	0.01\\
92.32	0.01\\
92.33	0.01\\
92.34	0.01\\
92.35	0.01\\
92.36	0.01\\
92.37	0.01\\
92.38	0.01\\
92.39	0.01\\
92.4	0.01\\
92.41	0.01\\
92.42	0.01\\
92.43	0.01\\
92.44	0.01\\
92.45	0.01\\
92.46	0.01\\
92.47	0.01\\
92.48	0.01\\
92.49	0.01\\
92.5	0.01\\
92.51	0.01\\
92.52	0.01\\
92.53	0.01\\
92.54	0.01\\
92.55	0.01\\
92.56	0.01\\
92.57	0.01\\
92.58	0.01\\
92.59	0.01\\
92.6	0.01\\
92.61	0.01\\
92.62	0.01\\
92.63	0.01\\
92.64	0.01\\
92.65	0.01\\
92.66	0.01\\
92.67	0.01\\
92.68	0.01\\
92.69	0.01\\
92.7	0.01\\
92.71	0.01\\
92.72	0.01\\
92.73	0.01\\
92.74	0.01\\
92.75	0.01\\
92.76	0.01\\
92.77	0.01\\
92.78	0.01\\
92.79	0.01\\
92.8	0.01\\
92.81	0.01\\
92.82	0.01\\
92.83	0.01\\
92.84	0.01\\
92.85	0.01\\
92.86	0.01\\
92.87	0.01\\
92.88	0.01\\
92.89	0.01\\
92.9	0.01\\
92.91	0.01\\
92.92	0.01\\
92.93	0.01\\
92.94	0.01\\
92.95	0.01\\
92.96	0.01\\
92.97	0.01\\
92.98	0.01\\
92.99	0.01\\
93	0.01\\
93.01	0.01\\
93.02	0.01\\
93.03	0.01\\
93.04	0.01\\
93.05	0.01\\
93.06	0.01\\
93.07	0.01\\
93.08	0.01\\
93.09	0.01\\
93.1	0.01\\
93.11	0.01\\
93.12	0.01\\
93.13	0.01\\
93.14	0.01\\
93.15	0.01\\
93.16	0.01\\
93.17	0.01\\
93.18	0.01\\
93.19	0.01\\
93.2	0.01\\
93.21	0.01\\
93.22	0.01\\
93.23	0.01\\
93.24	0.01\\
93.25	0.01\\
93.26	0.01\\
93.27	0.01\\
93.28	0.01\\
93.29	0.01\\
93.3	0.01\\
93.31	0.01\\
93.32	0.01\\
93.33	0.01\\
93.34	0.01\\
93.35	0.01\\
93.36	0.01\\
93.37	0.01\\
93.38	0.01\\
93.39	0.01\\
93.4	0.01\\
93.41	0.01\\
93.42	0.01\\
93.43	0.01\\
93.44	0.01\\
93.45	0.01\\
93.46	0.01\\
93.47	0.01\\
93.48	0.01\\
93.49	0.01\\
93.5	0.01\\
93.51	0.01\\
93.52	0.01\\
93.53	0.01\\
93.54	0.01\\
93.55	0.01\\
93.56	0.01\\
93.57	0.01\\
93.58	0.01\\
93.59	0.01\\
93.6	0.01\\
93.61	0.01\\
93.62	0.01\\
93.63	0.01\\
93.64	0.01\\
93.65	0.01\\
93.66	0.01\\
93.67	0.01\\
93.68	0.01\\
93.69	0.01\\
93.7	0.01\\
93.71	0.01\\
93.72	0.01\\
93.73	0.01\\
93.74	0.01\\
93.75	0.01\\
93.76	0.01\\
93.77	0.01\\
93.78	0.01\\
93.79	0.01\\
93.8	0.01\\
93.81	0.01\\
93.82	0.01\\
93.83	0.01\\
93.84	0.01\\
93.85	0.01\\
93.86	0.01\\
93.87	0.01\\
93.88	0.01\\
93.89	0.01\\
93.9	0.01\\
93.91	0.01\\
93.92	0.01\\
93.93	0.01\\
93.94	0.01\\
93.95	0.01\\
93.96	0.01\\
93.97	0.01\\
93.98	0.01\\
93.99	0.01\\
94	0.01\\
94.01	0.01\\
94.02	0.01\\
94.03	0.01\\
94.04	0.01\\
94.05	0.01\\
94.06	0.01\\
94.07	0.01\\
94.08	0.01\\
94.09	0.01\\
94.1	0.01\\
94.11	0.01\\
94.12	0.01\\
94.13	0.01\\
94.14	0.01\\
94.15	0.01\\
94.16	0.01\\
94.17	0.01\\
94.18	0.01\\
94.19	0.01\\
94.2	0.01\\
94.21	0.01\\
94.22	0.01\\
94.23	0.01\\
94.24	0.01\\
94.25	0.01\\
94.26	0.01\\
94.27	0.01\\
94.28	0.01\\
94.29	0.01\\
94.3	0.01\\
94.31	0.01\\
94.32	0.01\\
94.33	0.01\\
94.34	0.01\\
94.35	0.01\\
94.36	0.01\\
94.37	0.01\\
94.38	0.01\\
94.39	0.01\\
94.4	0.01\\
94.41	0.01\\
94.42	0.01\\
94.43	0.01\\
94.44	0.01\\
94.45	0.01\\
94.46	0.01\\
94.47	0.01\\
94.48	0.01\\
94.49	0.01\\
94.5	0.01\\
94.51	0.01\\
94.52	0.01\\
94.53	0.01\\
94.54	0.01\\
94.55	0.01\\
94.56	0.01\\
94.57	0.01\\
94.58	0.01\\
94.59	0.01\\
94.6	0.01\\
94.61	0.01\\
94.62	0.01\\
94.63	0.01\\
94.64	0.01\\
94.65	0.01\\
94.66	0.01\\
94.67	0.01\\
94.68	0.01\\
94.69	0.01\\
94.7	0.01\\
94.71	0.01\\
94.72	0.01\\
94.73	0.01\\
94.74	0.01\\
94.75	0.01\\
94.76	0.01\\
94.77	0.01\\
94.78	0.01\\
94.79	0.01\\
94.8	0.01\\
94.81	0.01\\
94.82	0.01\\
94.83	0.01\\
94.84	0.01\\
94.85	0.01\\
94.86	0.01\\
94.87	0.01\\
94.88	0.01\\
94.89	0.01\\
94.9	0.01\\
94.91	0.01\\
94.92	0.01\\
94.93	0.01\\
94.94	0.01\\
94.95	0.01\\
94.96	0.01\\
94.97	0.01\\
94.98	0.01\\
94.99	0.01\\
95	0.01\\
95.01	0.01\\
95.02	0.01\\
95.03	0.01\\
95.04	0.01\\
95.05	0.01\\
95.06	0.01\\
95.07	0.01\\
95.08	0.01\\
95.09	0.01\\
95.1	0.01\\
95.11	0.01\\
95.12	0.01\\
95.13	0.01\\
95.14	0.01\\
95.15	0.01\\
95.16	0.01\\
95.17	0.01\\
95.18	0.01\\
95.19	0.01\\
95.2	0.01\\
95.21	0.01\\
95.22	0.01\\
95.23	0.01\\
95.24	0.01\\
95.25	0.01\\
95.26	0.01\\
95.27	0.01\\
95.28	0.01\\
95.29	0.01\\
95.3	0.01\\
95.31	0.01\\
95.32	0.01\\
95.33	0.01\\
95.34	0.01\\
95.35	0.01\\
95.36	0.01\\
95.37	0.01\\
95.38	0.01\\
95.39	0.01\\
95.4	0.01\\
95.41	0.01\\
95.42	0.01\\
95.43	0.01\\
95.44	0.01\\
95.45	0.01\\
95.46	0.01\\
95.47	0.01\\
95.48	0.01\\
95.49	0.01\\
95.5	0.01\\
95.51	0.01\\
95.52	0.01\\
95.53	0.01\\
95.54	0.01\\
95.55	0.01\\
95.56	0.01\\
95.57	0.01\\
95.58	0.01\\
95.59	0.01\\
95.6	0.01\\
95.61	0.01\\
95.62	0.01\\
95.63	0.01\\
95.64	0.01\\
95.65	0.01\\
95.66	0.01\\
95.67	0.01\\
95.68	0.01\\
95.69	0.01\\
95.7	0.01\\
95.71	0.01\\
95.72	0.01\\
95.73	0.01\\
95.74	0.01\\
95.75	0.01\\
95.76	0.01\\
95.77	0.01\\
95.78	0.01\\
95.79	0.01\\
95.8	0.01\\
95.81	0.01\\
95.82	0.01\\
95.83	0.01\\
95.84	0.01\\
95.85	0.01\\
95.86	0.01\\
95.87	0.01\\
95.88	0.01\\
95.89	0.01\\
95.9	0.01\\
95.91	0.01\\
95.92	0.01\\
95.93	0.01\\
95.94	0.01\\
95.95	0.01\\
95.96	0.01\\
95.97	0.01\\
95.98	0.01\\
95.99	0.01\\
96	0.01\\
96.01	0.01\\
96.02	0.01\\
96.03	0.01\\
96.04	0.01\\
96.05	0.01\\
96.06	0.01\\
96.07	0.01\\
96.08	0.01\\
96.09	0.01\\
96.1	0.01\\
96.11	0.01\\
96.12	0.01\\
96.13	0.01\\
96.14	0.01\\
96.15	0.01\\
96.16	0.01\\
96.17	0.01\\
96.18	0.01\\
96.19	0.01\\
96.2	0.01\\
96.21	0.01\\
96.22	0.01\\
96.23	0.01\\
96.24	0.01\\
96.25	0.01\\
96.26	0.01\\
96.27	0.01\\
96.28	0.01\\
96.29	0.01\\
96.3	0.01\\
96.31	0.01\\
96.32	0.01\\
96.33	0.01\\
96.34	0.01\\
96.35	0.01\\
96.36	0.01\\
96.37	0.01\\
96.38	0.01\\
96.39	0.01\\
96.4	0.01\\
96.41	0.01\\
96.42	0.01\\
96.43	0.01\\
96.44	0.01\\
96.45	0.01\\
96.46	0.01\\
96.47	0.01\\
96.48	0.01\\
96.49	0.01\\
96.5	0.01\\
96.51	0.01\\
96.52	0.01\\
96.53	0.01\\
96.54	0.01\\
96.55	0.01\\
96.56	0.01\\
96.57	0.01\\
96.58	0.01\\
96.59	0.01\\
96.6	0.01\\
96.61	0.01\\
96.62	0.01\\
96.63	0.01\\
96.64	0.01\\
96.65	0.01\\
96.66	0.01\\
96.67	0.01\\
96.68	0.01\\
96.69	0.01\\
96.7	0.01\\
96.71	0.01\\
96.72	0.01\\
96.73	0.01\\
96.74	0.01\\
96.75	0.01\\
96.76	0.01\\
96.77	0.01\\
96.78	0.01\\
96.79	0.01\\
96.8	0.01\\
96.81	0.01\\
96.82	0.01\\
96.83	0.01\\
96.84	0.01\\
96.85	0.01\\
96.86	0.01\\
96.87	0.01\\
96.88	0.01\\
96.89	0.01\\
96.9	0.01\\
96.91	0.01\\
96.92	0.01\\
96.93	0.01\\
96.94	0.01\\
96.95	0.01\\
96.96	0.01\\
96.97	0.01\\
96.98	0.01\\
96.99	0.01\\
97	0.01\\
97.01	0.01\\
97.02	0.01\\
97.03	0.01\\
97.04	0.01\\
97.05	0.01\\
97.06	0.01\\
97.07	0.01\\
97.08	0.01\\
97.09	0.01\\
97.1	0.01\\
97.11	0.01\\
97.12	0.01\\
97.13	0.01\\
97.14	0.01\\
97.15	0.01\\
97.16	0.01\\
97.17	0.01\\
97.18	0.01\\
97.19	0.01\\
97.2	0.01\\
97.21	0.01\\
97.22	0.01\\
97.23	0.01\\
97.24	0.01\\
97.25	0.01\\
97.26	0.01\\
97.27	0.01\\
97.28	0.01\\
97.29	0.01\\
97.3	0.01\\
97.31	0.01\\
97.32	0.01\\
97.33	0.01\\
97.34	0.01\\
97.35	0.01\\
97.36	0.01\\
97.37	0.01\\
97.38	0.01\\
97.39	0.01\\
97.4	0.01\\
97.41	0.01\\
97.42	0.01\\
97.43	0.01\\
97.44	0.01\\
97.45	0.01\\
97.46	0.01\\
97.47	0.01\\
97.48	0.01\\
97.49	0.01\\
97.5	0.01\\
97.51	0.01\\
97.52	0.01\\
97.53	0.01\\
97.54	0.01\\
97.55	0.01\\
97.56	0.01\\
97.57	0.01\\
97.58	0.01\\
97.59	0.01\\
97.6	0.01\\
97.61	0.01\\
97.62	0.01\\
97.63	0.01\\
97.64	0.01\\
97.65	0.01\\
97.66	0.01\\
97.67	0.01\\
97.68	0.01\\
97.69	0.01\\
97.7	0.01\\
97.71	0.01\\
97.72	0.01\\
97.73	0.01\\
97.74	0.01\\
97.75	0.01\\
97.76	0.01\\
97.77	0.01\\
97.78	0.01\\
97.79	0.01\\
97.8	0.01\\
97.81	0.01\\
97.82	0.01\\
97.83	0.01\\
97.84	0.01\\
97.85	0.01\\
97.86	0.01\\
97.87	0.01\\
97.88	0.01\\
97.89	0.01\\
97.9	0.01\\
97.91	0.01\\
97.92	0.01\\
97.93	0.01\\
97.94	0.01\\
97.95	0.01\\
97.96	0.01\\
97.97	0.01\\
97.98	0.01\\
97.99	0.01\\
98	0.01\\
98.01	0.01\\
98.02	0.01\\
98.03	0.01\\
98.04	0.01\\
98.05	0.01\\
98.06	0.01\\
98.07	0.01\\
98.08	0.01\\
98.09	0.01\\
98.1	0.01\\
98.11	0.01\\
98.12	0.01\\
98.13	0.01\\
98.14	0.01\\
98.15	0.01\\
98.16	0.00995177871030805\\
98.17	0.0098799548118399\\
98.18	0.00980759140952771\\
98.19	0.00973468725080406\\
98.2	0.00966123716359973\\
98.21	0.00958723592161363\\
98.22	0.00951267824364272\\
98.23	0.00943755879290121\\
98.24	0.0093618721763288\\
98.25	0.00928561294388785\\
98.26	0.00920877558784894\\
98.27	0.00913135454206482\\
98.28	0.0090533441812325\\
98.29	0.00897473882014288\\
98.3	0.00889553271291812\\
98.31	0.00881572005223603\\
98.32	0.00873529496854143\\
98.33	0.00870936664103572\\
98.34	0.00868679735443978\\
98.35	0.00866403754486513\\
98.36	0.00864108556610907\\
98.37	0.00861793981098157\\
98.38	0.00859459866094568\\
98.39	0.00857106048609103\\
98.4	0.00854732364644967\\
98.41	0.00852338592357966\\
98.42	0.0084992410349635\\
98.43	0.00847488721120935\\
98.44	0.00845032266989913\\
98.45	0.0084255456155479\\
98.46	0.00840055423956494\\
98.47	0.00837534672021681\\
98.48	0.00834992122259224\\
98.49	0.00832427589856912\\
98.5	0.00829840888678362\\
98.51	0.00827231831260154\\
98.52	0.00824600228809194\\
98.53	0.00821945891200329\\
98.54	0.00819268626974207\\
98.55	0.00816568243335409\\
98.56	0.00813844546192765\\
98.57	0.00811097340435198\\
98.58	0.00808326429622979\\
98.59	0.00805531615987317\\
98.6	0.00802712700430279\\
98.61	0.00799869482525061\\
98.62	0.00797001760516625\\
98.63	0.00794109170234682\\
98.64	0.00791190235367152\\
98.65	0.00788244708039298\\
98.66	0.00785272338012802\\
98.67	0.00782272872662345\\
98.68	0.00779246056951935\\
98.69	0.00776191633410978\\
98.7	0.00773109342107603\\
98.71	0.00769998920623533\\
98.72	0.00766860104029221\\
98.73	0.007636926248587\\
98.74	0.00760496213939973\\
98.75	0.007572706004797\\
98.76	0.00754015511143454\\
98.77	0.00750730670031634\\
98.78	0.00747415798654993\\
98.79	0.00744070615933222\\
98.8	0.00740694838669973\\
98.81	0.0073728818102715\\
98.82	0.00733850354500313\\
98.83	0.00730381067893851\\
98.84	0.00726880027295931\\
98.85	0.00723346936053207\\
98.86	0.007197814947453\\
98.87	0.00716183401159042\\
98.88	0.00712552350262474\\
98.89	0.00708888034178608\\
98.9	0.0070519014215895\\
98.91	0.00701458360556767\\
98.92	0.00697692372799236\\
98.93	0.00693891859360066\\
98.94	0.00690056497731961\\
98.95	0.00686185962398836\\
98.96	0.00682279924807759\\
98.97	0.00678338053340649\\
98.98	0.00674360013285694\\
98.99	0.00670345466808514\\
99	0.00666294072923045\\
99.01	0.00662205487462158\\
99.02	0.00658079363048003\\
99.03	0.00653915349062074\\
99.04	0.00649713091614992\\
99.05	0.00645472233516013\\
99.06	0.00641192414242244\\
99.07	0.00636873269907578\\
99.08	0.00632514433231329\\
99.09	0.00628115533506586\\
99.1	0.00623676196568261\\
99.11	0.00619196044760847\\
99.12	0.00614674696905861\\
99.13	0.00610111768269005\\
99.14	0.00605506870526992\\
99.15	0.00600859611734088\\
99.16	0.00596169596288317\\
99.17	0.00591436424897374\\
99.18	0.00586659694544196\\
99.19	0.00581839001403525\\
99.2	0.00576973939059186\\
99.21	0.00572064097344853\\
99.22	0.00567109062309511\\
99.23	0.00562108416182606\\
99.24	0.00557061737338879\\
99.25	0.00551968600262867\\
99.26	0.00546828575513084\\
99.27	0.00541641229685875\\
99.28	0.00536406125378932\\
99.29	0.00531122821154471\\
99.3	0.00525790871502087\\
99.31	0.00520409826801247\\
99.32	0.0051497923328345\\
99.33	0.0050949863299404\\
99.34	0.00503967563753657\\
99.35	0.00498385559119345\\
99.36	0.00492752148345294\\
99.37	0.00487066856343227\\
99.38	0.00481329203642413\\
99.39	0.00475538706349262\\
99.4	0.00469694876103752\\
99.41	0.00463797220038192\\
99.42	0.00457845240735571\\
99.43	0.00451838436187542\\
99.44	0.00445776299751998\\
99.45	0.00439658320110255\\
99.46	0.00433483981223837\\
99.47	0.00427252762290852\\
99.48	0.0042096413770196\\
99.49	0.00414617576995928\\
99.5	0.00408212544814768\\
99.51	0.00401748500858448\\
99.52	0.00395224899839186\\
99.53	0.00388641191435301\\
99.54	0.00381996820244641\\
99.55	0.00375291225737559\\
99.56	0.00368523842209455\\
99.57	0.00361694098732857\\
99.58	0.00354801419109061\\
99.59	0.00347845221819298\\
99.6	0.00340824919975444\\
99.61	0.00333739921270264\\
99.62	0.00326589628211675\\
99.63	0.00319373438395597\\
99.64	0.003120907438625\\
99.65	0.00304740931045928\\
99.66	0.00297323380720539\\
99.67	0.00289837467949644\\
99.68	0.00282282562032242\\
99.69	0.00274658026449535\\
99.7	0.0026696321881094\\
99.71	0.00259197490799565\\
99.72	0.00251360188117163\\
99.73	0.00243450650428147\\
99.74	0.0023546821130308\\
99.75	0.00227412198161954\\
99.76	0.00219281932216908\\
99.77	0.00211076728414376\\
99.78	0.00202795895376664\\
99.79	0.00194438735342945\\
99.8	0.00186004544109659\\
99.81	0.00177492610970319\\
99.82	0.00168902218654705\\
99.83	0.00160232643267447\\
99.84	0.0015148315422598\\
99.85	0.00142653014197858\\
99.86	0.00133741479037433\\
99.87	0.00124747797721875\\
99.88	0.00115671212286522\\
99.89	0.00106510957759558\\
99.9	0.000972662620960026\\
99.91	0.000879363461110016\\
99.92	0.000785204234124003\\
99.93	0.000690177003325972\\
99.94	0.000594273758596576\\
99.95	0.000497486415676754\\
99.96	0.000399806815463661\\
99.97	0.000301226723298808\\
99.98	0.000201737828248207\\
99.99	0.000101331742374368\\
100	0\\
};
\addlegendentry{$q=-3$};

\addplot [color=red,dashed,forget plot]
  table[row sep=crcr]{%
0.01	0.01\\
0.02	0.01\\
0.03	0.01\\
0.04	0.01\\
0.05	0.01\\
0.06	0.01\\
0.07	0.01\\
0.08	0.01\\
0.09	0.01\\
0.1	0.01\\
0.11	0.01\\
0.12	0.01\\
0.13	0.01\\
0.14	0.01\\
0.15	0.01\\
0.16	0.01\\
0.17	0.01\\
0.18	0.01\\
0.19	0.01\\
0.2	0.01\\
0.21	0.01\\
0.22	0.01\\
0.23	0.01\\
0.24	0.01\\
0.25	0.01\\
0.26	0.01\\
0.27	0.01\\
0.28	0.01\\
0.29	0.01\\
0.3	0.01\\
0.31	0.01\\
0.32	0.01\\
0.33	0.01\\
0.34	0.01\\
0.35	0.01\\
0.36	0.01\\
0.37	0.01\\
0.38	0.01\\
0.39	0.01\\
0.4	0.01\\
0.41	0.01\\
0.42	0.01\\
0.43	0.01\\
0.44	0.01\\
0.45	0.01\\
0.46	0.01\\
0.47	0.01\\
0.48	0.01\\
0.49	0.01\\
0.5	0.01\\
0.51	0.01\\
0.52	0.01\\
0.53	0.01\\
0.54	0.01\\
0.55	0.01\\
0.56	0.01\\
0.57	0.01\\
0.58	0.01\\
0.59	0.01\\
0.6	0.01\\
0.61	0.01\\
0.62	0.01\\
0.63	0.01\\
0.64	0.01\\
0.65	0.01\\
0.66	0.01\\
0.67	0.01\\
0.68	0.01\\
0.69	0.01\\
0.7	0.01\\
0.71	0.01\\
0.72	0.01\\
0.73	0.01\\
0.74	0.01\\
0.75	0.01\\
0.76	0.01\\
0.77	0.01\\
0.78	0.01\\
0.79	0.01\\
0.8	0.01\\
0.81	0.01\\
0.82	0.01\\
0.83	0.01\\
0.84	0.01\\
0.85	0.01\\
0.86	0.01\\
0.87	0.01\\
0.88	0.01\\
0.89	0.01\\
0.9	0.01\\
0.91	0.01\\
0.92	0.01\\
0.93	0.01\\
0.94	0.01\\
0.95	0.01\\
0.96	0.01\\
0.97	0.01\\
0.98	0.01\\
0.99	0.01\\
1	0.01\\
1.01	0.01\\
1.02	0.01\\
1.03	0.01\\
1.04	0.01\\
1.05	0.01\\
1.06	0.01\\
1.07	0.01\\
1.08	0.01\\
1.09	0.01\\
1.1	0.01\\
1.11	0.01\\
1.12	0.01\\
1.13	0.01\\
1.14	0.01\\
1.15	0.01\\
1.16	0.01\\
1.17	0.01\\
1.18	0.01\\
1.19	0.01\\
1.2	0.01\\
1.21	0.01\\
1.22	0.01\\
1.23	0.01\\
1.24	0.01\\
1.25	0.01\\
1.26	0.01\\
1.27	0.01\\
1.28	0.01\\
1.29	0.01\\
1.3	0.01\\
1.31	0.01\\
1.32	0.01\\
1.33	0.01\\
1.34	0.01\\
1.35	0.01\\
1.36	0.01\\
1.37	0.01\\
1.38	0.01\\
1.39	0.01\\
1.4	0.01\\
1.41	0.01\\
1.42	0.01\\
1.43	0.01\\
1.44	0.01\\
1.45	0.01\\
1.46	0.01\\
1.47	0.01\\
1.48	0.01\\
1.49	0.01\\
1.5	0.01\\
1.51	0.01\\
1.52	0.01\\
1.53	0.01\\
1.54	0.01\\
1.55	0.01\\
1.56	0.01\\
1.57	0.01\\
1.58	0.01\\
1.59	0.01\\
1.6	0.01\\
1.61	0.01\\
1.62	0.01\\
1.63	0.01\\
1.64	0.01\\
1.65	0.01\\
1.66	0.01\\
1.67	0.01\\
1.68	0.01\\
1.69	0.01\\
1.7	0.01\\
1.71	0.01\\
1.72	0.01\\
1.73	0.01\\
1.74	0.01\\
1.75	0.01\\
1.76	0.01\\
1.77	0.01\\
1.78	0.01\\
1.79	0.01\\
1.8	0.01\\
1.81	0.01\\
1.82	0.01\\
1.83	0.01\\
1.84	0.01\\
1.85	0.01\\
1.86	0.01\\
1.87	0.01\\
1.88	0.01\\
1.89	0.01\\
1.9	0.01\\
1.91	0.01\\
1.92	0.01\\
1.93	0.01\\
1.94	0.01\\
1.95	0.01\\
1.96	0.01\\
1.97	0.01\\
1.98	0.01\\
1.99	0.01\\
2	0.01\\
2.01	0.01\\
2.02	0.01\\
2.03	0.01\\
2.04	0.01\\
2.05	0.01\\
2.06	0.01\\
2.07	0.01\\
2.08	0.01\\
2.09	0.01\\
2.1	0.01\\
2.11	0.01\\
2.12	0.01\\
2.13	0.01\\
2.14	0.01\\
2.15	0.01\\
2.16	0.01\\
2.17	0.01\\
2.18	0.01\\
2.19	0.01\\
2.2	0.01\\
2.21	0.01\\
2.22	0.01\\
2.23	0.01\\
2.24	0.01\\
2.25	0.01\\
2.26	0.01\\
2.27	0.01\\
2.28	0.01\\
2.29	0.01\\
2.3	0.01\\
2.31	0.01\\
2.32	0.01\\
2.33	0.01\\
2.34	0.01\\
2.35	0.01\\
2.36	0.01\\
2.37	0.01\\
2.38	0.01\\
2.39	0.01\\
2.4	0.01\\
2.41	0.01\\
2.42	0.01\\
2.43	0.01\\
2.44	0.01\\
2.45	0.01\\
2.46	0.01\\
2.47	0.01\\
2.48	0.01\\
2.49	0.01\\
2.5	0.01\\
2.51	0.01\\
2.52	0.01\\
2.53	0.01\\
2.54	0.01\\
2.55	0.01\\
2.56	0.01\\
2.57	0.01\\
2.58	0.01\\
2.59	0.01\\
2.6	0.01\\
2.61	0.01\\
2.62	0.01\\
2.63	0.01\\
2.64	0.01\\
2.65	0.01\\
2.66	0.01\\
2.67	0.01\\
2.68	0.01\\
2.69	0.01\\
2.7	0.01\\
2.71	0.01\\
2.72	0.01\\
2.73	0.01\\
2.74	0.01\\
2.75	0.01\\
2.76	0.01\\
2.77	0.01\\
2.78	0.01\\
2.79	0.01\\
2.8	0.01\\
2.81	0.01\\
2.82	0.01\\
2.83	0.01\\
2.84	0.01\\
2.85	0.01\\
2.86	0.01\\
2.87	0.01\\
2.88	0.01\\
2.89	0.01\\
2.9	0.01\\
2.91	0.01\\
2.92	0.01\\
2.93	0.01\\
2.94	0.01\\
2.95	0.01\\
2.96	0.01\\
2.97	0.01\\
2.98	0.01\\
2.99	0.01\\
3	0.01\\
3.01	0.01\\
3.02	0.01\\
3.03	0.01\\
3.04	0.01\\
3.05	0.01\\
3.06	0.01\\
3.07	0.01\\
3.08	0.01\\
3.09	0.01\\
3.1	0.01\\
3.11	0.01\\
3.12	0.01\\
3.13	0.01\\
3.14	0.01\\
3.15	0.01\\
3.16	0.01\\
3.17	0.01\\
3.18	0.01\\
3.19	0.01\\
3.2	0.01\\
3.21	0.01\\
3.22	0.01\\
3.23	0.01\\
3.24	0.01\\
3.25	0.01\\
3.26	0.01\\
3.27	0.01\\
3.28	0.01\\
3.29	0.01\\
3.3	0.01\\
3.31	0.01\\
3.32	0.01\\
3.33	0.01\\
3.34	0.01\\
3.35	0.01\\
3.36	0.01\\
3.37	0.01\\
3.38	0.01\\
3.39	0.01\\
3.4	0.01\\
3.41	0.01\\
3.42	0.01\\
3.43	0.01\\
3.44	0.01\\
3.45	0.01\\
3.46	0.01\\
3.47	0.01\\
3.48	0.01\\
3.49	0.01\\
3.5	0.01\\
3.51	0.01\\
3.52	0.01\\
3.53	0.01\\
3.54	0.01\\
3.55	0.01\\
3.56	0.01\\
3.57	0.01\\
3.58	0.01\\
3.59	0.01\\
3.6	0.01\\
3.61	0.01\\
3.62	0.01\\
3.63	0.01\\
3.64	0.01\\
3.65	0.01\\
3.66	0.01\\
3.67	0.01\\
3.68	0.01\\
3.69	0.01\\
3.7	0.01\\
3.71	0.01\\
3.72	0.01\\
3.73	0.01\\
3.74	0.01\\
3.75	0.01\\
3.76	0.01\\
3.77	0.01\\
3.78	0.01\\
3.79	0.01\\
3.8	0.01\\
3.81	0.01\\
3.82	0.01\\
3.83	0.01\\
3.84	0.01\\
3.85	0.01\\
3.86	0.01\\
3.87	0.01\\
3.88	0.01\\
3.89	0.01\\
3.9	0.01\\
3.91	0.01\\
3.92	0.01\\
3.93	0.01\\
3.94	0.01\\
3.95	0.01\\
3.96	0.01\\
3.97	0.01\\
3.98	0.01\\
3.99	0.01\\
4	0.01\\
4.01	0.01\\
4.02	0.01\\
4.03	0.01\\
4.04	0.01\\
4.05	0.01\\
4.06	0.01\\
4.07	0.01\\
4.08	0.01\\
4.09	0.01\\
4.1	0.01\\
4.11	0.01\\
4.12	0.01\\
4.13	0.01\\
4.14	0.01\\
4.15	0.01\\
4.16	0.01\\
4.17	0.01\\
4.18	0.01\\
4.19	0.01\\
4.2	0.01\\
4.21	0.01\\
4.22	0.01\\
4.23	0.01\\
4.24	0.01\\
4.25	0.01\\
4.26	0.01\\
4.27	0.01\\
4.28	0.01\\
4.29	0.01\\
4.3	0.01\\
4.31	0.01\\
4.32	0.01\\
4.33	0.01\\
4.34	0.01\\
4.35	0.01\\
4.36	0.01\\
4.37	0.01\\
4.38	0.01\\
4.39	0.01\\
4.4	0.01\\
4.41	0.01\\
4.42	0.01\\
4.43	0.01\\
4.44	0.01\\
4.45	0.01\\
4.46	0.01\\
4.47	0.01\\
4.48	0.01\\
4.49	0.01\\
4.5	0.01\\
4.51	0.01\\
4.52	0.01\\
4.53	0.01\\
4.54	0.01\\
4.55	0.01\\
4.56	0.01\\
4.57	0.01\\
4.58	0.01\\
4.59	0.01\\
4.6	0.01\\
4.61	0.01\\
4.62	0.01\\
4.63	0.01\\
4.64	0.01\\
4.65	0.01\\
4.66	0.01\\
4.67	0.01\\
4.68	0.01\\
4.69	0.01\\
4.7	0.01\\
4.71	0.01\\
4.72	0.01\\
4.73	0.01\\
4.74	0.01\\
4.75	0.01\\
4.76	0.01\\
4.77	0.01\\
4.78	0.01\\
4.79	0.01\\
4.8	0.01\\
4.81	0.01\\
4.82	0.01\\
4.83	0.01\\
4.84	0.01\\
4.85	0.01\\
4.86	0.01\\
4.87	0.01\\
4.88	0.01\\
4.89	0.01\\
4.9	0.01\\
4.91	0.01\\
4.92	0.01\\
4.93	0.01\\
4.94	0.01\\
4.95	0.01\\
4.96	0.01\\
4.97	0.01\\
4.98	0.01\\
4.99	0.01\\
5	0.01\\
5.01	0.01\\
5.02	0.01\\
5.03	0.01\\
5.04	0.01\\
5.05	0.01\\
5.06	0.01\\
5.07	0.01\\
5.08	0.01\\
5.09	0.01\\
5.1	0.01\\
5.11	0.01\\
5.12	0.01\\
5.13	0.01\\
5.14	0.01\\
5.15	0.01\\
5.16	0.01\\
5.17	0.01\\
5.18	0.01\\
5.19	0.01\\
5.2	0.01\\
5.21	0.01\\
5.22	0.01\\
5.23	0.01\\
5.24	0.01\\
5.25	0.01\\
5.26	0.01\\
5.27	0.01\\
5.28	0.01\\
5.29	0.01\\
5.3	0.01\\
5.31	0.01\\
5.32	0.01\\
5.33	0.01\\
5.34	0.01\\
5.35	0.01\\
5.36	0.01\\
5.37	0.01\\
5.38	0.01\\
5.39	0.01\\
5.4	0.01\\
5.41	0.01\\
5.42	0.01\\
5.43	0.01\\
5.44	0.01\\
5.45	0.01\\
5.46	0.01\\
5.47	0.01\\
5.48	0.01\\
5.49	0.01\\
5.5	0.01\\
5.51	0.01\\
5.52	0.01\\
5.53	0.01\\
5.54	0.01\\
5.55	0.01\\
5.56	0.01\\
5.57	0.01\\
5.58	0.01\\
5.59	0.01\\
5.6	0.01\\
5.61	0.01\\
5.62	0.01\\
5.63	0.01\\
5.64	0.01\\
5.65	0.01\\
5.66	0.01\\
5.67	0.01\\
5.68	0.01\\
5.69	0.01\\
5.7	0.01\\
5.71	0.01\\
5.72	0.01\\
5.73	0.01\\
5.74	0.01\\
5.75	0.01\\
5.76	0.01\\
5.77	0.01\\
5.78	0.01\\
5.79	0.01\\
5.8	0.01\\
5.81	0.01\\
5.82	0.01\\
5.83	0.01\\
5.84	0.01\\
5.85	0.01\\
5.86	0.01\\
5.87	0.01\\
5.88	0.01\\
5.89	0.01\\
5.9	0.01\\
5.91	0.01\\
5.92	0.01\\
5.93	0.01\\
5.94	0.01\\
5.95	0.01\\
5.96	0.01\\
5.97	0.01\\
5.98	0.01\\
5.99	0.01\\
6	0.01\\
6.01	0.01\\
6.02	0.01\\
6.03	0.01\\
6.04	0.01\\
6.05	0.01\\
6.06	0.01\\
6.07	0.01\\
6.08	0.01\\
6.09	0.01\\
6.1	0.01\\
6.11	0.01\\
6.12	0.01\\
6.13	0.01\\
6.14	0.01\\
6.15	0.01\\
6.16	0.01\\
6.17	0.01\\
6.18	0.01\\
6.19	0.01\\
6.2	0.01\\
6.21	0.01\\
6.22	0.01\\
6.23	0.01\\
6.24	0.01\\
6.25	0.01\\
6.26	0.01\\
6.27	0.01\\
6.28	0.01\\
6.29	0.01\\
6.3	0.01\\
6.31	0.01\\
6.32	0.01\\
6.33	0.01\\
6.34	0.01\\
6.35	0.01\\
6.36	0.01\\
6.37	0.01\\
6.38	0.01\\
6.39	0.01\\
6.4	0.01\\
6.41	0.01\\
6.42	0.01\\
6.43	0.01\\
6.44	0.01\\
6.45	0.01\\
6.46	0.01\\
6.47	0.01\\
6.48	0.01\\
6.49	0.01\\
6.5	0.01\\
6.51	0.01\\
6.52	0.01\\
6.53	0.01\\
6.54	0.01\\
6.55	0.01\\
6.56	0.01\\
6.57	0.01\\
6.58	0.01\\
6.59	0.01\\
6.6	0.01\\
6.61	0.01\\
6.62	0.01\\
6.63	0.01\\
6.64	0.01\\
6.65	0.01\\
6.66	0.01\\
6.67	0.01\\
6.68	0.01\\
6.69	0.01\\
6.7	0.01\\
6.71	0.01\\
6.72	0.01\\
6.73	0.01\\
6.74	0.01\\
6.75	0.01\\
6.76	0.01\\
6.77	0.01\\
6.78	0.01\\
6.79	0.01\\
6.8	0.01\\
6.81	0.01\\
6.82	0.01\\
6.83	0.01\\
6.84	0.01\\
6.85	0.01\\
6.86	0.01\\
6.87	0.01\\
6.88	0.01\\
6.89	0.01\\
6.9	0.01\\
6.91	0.01\\
6.92	0.01\\
6.93	0.01\\
6.94	0.01\\
6.95	0.01\\
6.96	0.01\\
6.97	0.01\\
6.98	0.01\\
6.99	0.01\\
7	0.01\\
7.01	0.01\\
7.02	0.01\\
7.03	0.01\\
7.04	0.01\\
7.05	0.01\\
7.06	0.01\\
7.07	0.01\\
7.08	0.01\\
7.09	0.01\\
7.1	0.01\\
7.11	0.01\\
7.12	0.01\\
7.13	0.01\\
7.14	0.01\\
7.15	0.01\\
7.16	0.01\\
7.17	0.01\\
7.18	0.01\\
7.19	0.01\\
7.2	0.01\\
7.21	0.01\\
7.22	0.01\\
7.23	0.01\\
7.24	0.01\\
7.25	0.01\\
7.26	0.01\\
7.27	0.01\\
7.28	0.01\\
7.29	0.01\\
7.3	0.01\\
7.31	0.01\\
7.32	0.01\\
7.33	0.01\\
7.34	0.01\\
7.35	0.01\\
7.36	0.01\\
7.37	0.01\\
7.38	0.01\\
7.39	0.01\\
7.4	0.01\\
7.41	0.01\\
7.42	0.01\\
7.43	0.01\\
7.44	0.01\\
7.45	0.01\\
7.46	0.01\\
7.47	0.01\\
7.48	0.01\\
7.49	0.01\\
7.5	0.01\\
7.51	0.01\\
7.52	0.01\\
7.53	0.01\\
7.54	0.01\\
7.55	0.01\\
7.56	0.01\\
7.57	0.01\\
7.58	0.01\\
7.59	0.01\\
7.6	0.01\\
7.61	0.01\\
7.62	0.01\\
7.63	0.01\\
7.64	0.01\\
7.65	0.01\\
7.66	0.01\\
7.67	0.01\\
7.68	0.01\\
7.69	0.01\\
7.7	0.01\\
7.71	0.01\\
7.72	0.01\\
7.73	0.01\\
7.74	0.01\\
7.75	0.01\\
7.76	0.01\\
7.77	0.01\\
7.78	0.01\\
7.79	0.01\\
7.8	0.01\\
7.81	0.01\\
7.82	0.01\\
7.83	0.01\\
7.84	0.01\\
7.85	0.01\\
7.86	0.01\\
7.87	0.01\\
7.88	0.01\\
7.89	0.01\\
7.9	0.01\\
7.91	0.01\\
7.92	0.01\\
7.93	0.01\\
7.94	0.01\\
7.95	0.01\\
7.96	0.01\\
7.97	0.01\\
7.98	0.01\\
7.99	0.01\\
8	0.01\\
8.01	0.01\\
8.02	0.01\\
8.03	0.01\\
8.04	0.01\\
8.05	0.01\\
8.06	0.01\\
8.07	0.01\\
8.08	0.01\\
8.09	0.01\\
8.1	0.01\\
8.11	0.01\\
8.12	0.01\\
8.13	0.01\\
8.14	0.01\\
8.15	0.01\\
8.16	0.01\\
8.17	0.01\\
8.18	0.01\\
8.19	0.01\\
8.2	0.01\\
8.21	0.01\\
8.22	0.01\\
8.23	0.01\\
8.24	0.01\\
8.25	0.01\\
8.26	0.01\\
8.27	0.01\\
8.28	0.01\\
8.29	0.01\\
8.3	0.01\\
8.31	0.01\\
8.32	0.01\\
8.33	0.01\\
8.34	0.01\\
8.35	0.01\\
8.36	0.01\\
8.37	0.01\\
8.38	0.01\\
8.39	0.01\\
8.4	0.01\\
8.41	0.01\\
8.42	0.01\\
8.43	0.01\\
8.44	0.01\\
8.45	0.01\\
8.46	0.01\\
8.47	0.01\\
8.48	0.01\\
8.49	0.01\\
8.5	0.01\\
8.51	0.01\\
8.52	0.01\\
8.53	0.01\\
8.54	0.01\\
8.55	0.01\\
8.56	0.01\\
8.57	0.01\\
8.58	0.01\\
8.59	0.01\\
8.6	0.01\\
8.61	0.01\\
8.62	0.01\\
8.63	0.01\\
8.64	0.01\\
8.65	0.01\\
8.66	0.01\\
8.67	0.01\\
8.68	0.01\\
8.69	0.01\\
8.7	0.01\\
8.71	0.01\\
8.72	0.01\\
8.73	0.01\\
8.74	0.01\\
8.75	0.01\\
8.76	0.01\\
8.77	0.01\\
8.78	0.01\\
8.79	0.01\\
8.8	0.01\\
8.81	0.01\\
8.82	0.01\\
8.83	0.01\\
8.84	0.01\\
8.85	0.01\\
8.86	0.01\\
8.87	0.01\\
8.88	0.01\\
8.89	0.01\\
8.9	0.01\\
8.91	0.01\\
8.92	0.01\\
8.93	0.01\\
8.94	0.01\\
8.95	0.01\\
8.96	0.01\\
8.97	0.01\\
8.98	0.01\\
8.99	0.01\\
9	0.01\\
9.01	0.01\\
9.02	0.01\\
9.03	0.01\\
9.04	0.01\\
9.05	0.01\\
9.06	0.01\\
9.07	0.01\\
9.08	0.01\\
9.09	0.01\\
9.1	0.01\\
9.11	0.01\\
9.12	0.01\\
9.13	0.01\\
9.14	0.01\\
9.15	0.01\\
9.16	0.01\\
9.17	0.01\\
9.18	0.01\\
9.19	0.01\\
9.2	0.01\\
9.21	0.01\\
9.22	0.01\\
9.23	0.01\\
9.24	0.01\\
9.25	0.01\\
9.26	0.01\\
9.27	0.01\\
9.28	0.01\\
9.29	0.01\\
9.3	0.01\\
9.31	0.01\\
9.32	0.01\\
9.33	0.01\\
9.34	0.01\\
9.35	0.01\\
9.36	0.01\\
9.37	0.01\\
9.38	0.01\\
9.39	0.01\\
9.4	0.01\\
9.41	0.01\\
9.42	0.01\\
9.43	0.01\\
9.44	0.01\\
9.45	0.01\\
9.46	0.01\\
9.47	0.01\\
9.48	0.01\\
9.49	0.01\\
9.5	0.01\\
9.51	0.01\\
9.52	0.01\\
9.53	0.01\\
9.54	0.01\\
9.55	0.01\\
9.56	0.01\\
9.57	0.01\\
9.58	0.01\\
9.59	0.01\\
9.6	0.01\\
9.61	0.01\\
9.62	0.01\\
9.63	0.01\\
9.64	0.01\\
9.65	0.01\\
9.66	0.01\\
9.67	0.01\\
9.68	0.01\\
9.69	0.01\\
9.7	0.01\\
9.71	0.01\\
9.72	0.01\\
9.73	0.01\\
9.74	0.01\\
9.75	0.01\\
9.76	0.01\\
9.77	0.01\\
9.78	0.01\\
9.79	0.01\\
9.8	0.01\\
9.81	0.01\\
9.82	0.01\\
9.83	0.01\\
9.84	0.01\\
9.85	0.01\\
9.86	0.01\\
9.87	0.01\\
9.88	0.01\\
9.89	0.01\\
9.9	0.01\\
9.91	0.01\\
9.92	0.01\\
9.93	0.01\\
9.94	0.01\\
9.95	0.01\\
9.96	0.01\\
9.97	0.01\\
9.98	0.01\\
9.99	0.01\\
10	0.01\\
10.01	0.01\\
10.02	0.01\\
10.03	0.01\\
10.04	0.01\\
10.05	0.01\\
10.06	0.01\\
10.07	0.01\\
10.08	0.01\\
10.09	0.01\\
10.1	0.01\\
10.11	0.01\\
10.12	0.01\\
10.13	0.01\\
10.14	0.01\\
10.15	0.01\\
10.16	0.01\\
10.17	0.01\\
10.18	0.01\\
10.19	0.01\\
10.2	0.01\\
10.21	0.01\\
10.22	0.01\\
10.23	0.01\\
10.24	0.01\\
10.25	0.01\\
10.26	0.01\\
10.27	0.01\\
10.28	0.01\\
10.29	0.01\\
10.3	0.01\\
10.31	0.01\\
10.32	0.01\\
10.33	0.01\\
10.34	0.01\\
10.35	0.01\\
10.36	0.01\\
10.37	0.01\\
10.38	0.01\\
10.39	0.01\\
10.4	0.01\\
10.41	0.01\\
10.42	0.01\\
10.43	0.01\\
10.44	0.01\\
10.45	0.01\\
10.46	0.01\\
10.47	0.01\\
10.48	0.01\\
10.49	0.01\\
10.5	0.01\\
10.51	0.01\\
10.52	0.01\\
10.53	0.01\\
10.54	0.01\\
10.55	0.01\\
10.56	0.01\\
10.57	0.01\\
10.58	0.01\\
10.59	0.01\\
10.6	0.01\\
10.61	0.01\\
10.62	0.01\\
10.63	0.01\\
10.64	0.01\\
10.65	0.01\\
10.66	0.01\\
10.67	0.01\\
10.68	0.01\\
10.69	0.01\\
10.7	0.01\\
10.71	0.01\\
10.72	0.01\\
10.73	0.01\\
10.74	0.01\\
10.75	0.01\\
10.76	0.01\\
10.77	0.01\\
10.78	0.01\\
10.79	0.01\\
10.8	0.01\\
10.81	0.01\\
10.82	0.01\\
10.83	0.01\\
10.84	0.01\\
10.85	0.01\\
10.86	0.01\\
10.87	0.01\\
10.88	0.01\\
10.89	0.01\\
10.9	0.01\\
10.91	0.01\\
10.92	0.01\\
10.93	0.01\\
10.94	0.01\\
10.95	0.01\\
10.96	0.01\\
10.97	0.01\\
10.98	0.01\\
10.99	0.01\\
11	0.01\\
11.01	0.01\\
11.02	0.01\\
11.03	0.01\\
11.04	0.01\\
11.05	0.01\\
11.06	0.01\\
11.07	0.01\\
11.08	0.01\\
11.09	0.01\\
11.1	0.01\\
11.11	0.01\\
11.12	0.01\\
11.13	0.01\\
11.14	0.01\\
11.15	0.01\\
11.16	0.01\\
11.17	0.01\\
11.18	0.01\\
11.19	0.01\\
11.2	0.01\\
11.21	0.01\\
11.22	0.01\\
11.23	0.01\\
11.24	0.01\\
11.25	0.01\\
11.26	0.01\\
11.27	0.01\\
11.28	0.01\\
11.29	0.01\\
11.3	0.01\\
11.31	0.01\\
11.32	0.01\\
11.33	0.01\\
11.34	0.01\\
11.35	0.01\\
11.36	0.01\\
11.37	0.01\\
11.38	0.01\\
11.39	0.01\\
11.4	0.01\\
11.41	0.01\\
11.42	0.01\\
11.43	0.01\\
11.44	0.01\\
11.45	0.01\\
11.46	0.01\\
11.47	0.01\\
11.48	0.01\\
11.49	0.01\\
11.5	0.01\\
11.51	0.01\\
11.52	0.01\\
11.53	0.01\\
11.54	0.01\\
11.55	0.01\\
11.56	0.01\\
11.57	0.01\\
11.58	0.01\\
11.59	0.01\\
11.6	0.01\\
11.61	0.01\\
11.62	0.01\\
11.63	0.01\\
11.64	0.01\\
11.65	0.01\\
11.66	0.01\\
11.67	0.01\\
11.68	0.01\\
11.69	0.01\\
11.7	0.01\\
11.71	0.01\\
11.72	0.01\\
11.73	0.01\\
11.74	0.01\\
11.75	0.01\\
11.76	0.01\\
11.77	0.01\\
11.78	0.01\\
11.79	0.01\\
11.8	0.01\\
11.81	0.01\\
11.82	0.01\\
11.83	0.01\\
11.84	0.01\\
11.85	0.01\\
11.86	0.01\\
11.87	0.01\\
11.88	0.01\\
11.89	0.01\\
11.9	0.01\\
11.91	0.01\\
11.92	0.01\\
11.93	0.01\\
11.94	0.01\\
11.95	0.01\\
11.96	0.01\\
11.97	0.01\\
11.98	0.01\\
11.99	0.01\\
12	0.01\\
12.01	0.01\\
12.02	0.01\\
12.03	0.01\\
12.04	0.01\\
12.05	0.01\\
12.06	0.01\\
12.07	0.01\\
12.08	0.01\\
12.09	0.01\\
12.1	0.01\\
12.11	0.01\\
12.12	0.01\\
12.13	0.01\\
12.14	0.01\\
12.15	0.01\\
12.16	0.01\\
12.17	0.01\\
12.18	0.01\\
12.19	0.01\\
12.2	0.01\\
12.21	0.01\\
12.22	0.01\\
12.23	0.01\\
12.24	0.01\\
12.25	0.01\\
12.26	0.01\\
12.27	0.01\\
12.28	0.01\\
12.29	0.01\\
12.3	0.01\\
12.31	0.01\\
12.32	0.01\\
12.33	0.01\\
12.34	0.01\\
12.35	0.01\\
12.36	0.01\\
12.37	0.01\\
12.38	0.01\\
12.39	0.01\\
12.4	0.01\\
12.41	0.01\\
12.42	0.01\\
12.43	0.01\\
12.44	0.01\\
12.45	0.01\\
12.46	0.01\\
12.47	0.01\\
12.48	0.01\\
12.49	0.01\\
12.5	0.01\\
12.51	0.01\\
12.52	0.01\\
12.53	0.01\\
12.54	0.01\\
12.55	0.01\\
12.56	0.01\\
12.57	0.01\\
12.58	0.01\\
12.59	0.01\\
12.6	0.01\\
12.61	0.01\\
12.62	0.01\\
12.63	0.01\\
12.64	0.01\\
12.65	0.01\\
12.66	0.01\\
12.67	0.01\\
12.68	0.01\\
12.69	0.01\\
12.7	0.01\\
12.71	0.01\\
12.72	0.01\\
12.73	0.01\\
12.74	0.01\\
12.75	0.01\\
12.76	0.01\\
12.77	0.01\\
12.78	0.01\\
12.79	0.01\\
12.8	0.01\\
12.81	0.01\\
12.82	0.01\\
12.83	0.01\\
12.84	0.01\\
12.85	0.01\\
12.86	0.01\\
12.87	0.01\\
12.88	0.01\\
12.89	0.01\\
12.9	0.01\\
12.91	0.01\\
12.92	0.01\\
12.93	0.01\\
12.94	0.01\\
12.95	0.01\\
12.96	0.01\\
12.97	0.01\\
12.98	0.01\\
12.99	0.01\\
13	0.01\\
13.01	0.01\\
13.02	0.01\\
13.03	0.01\\
13.04	0.01\\
13.05	0.01\\
13.06	0.01\\
13.07	0.01\\
13.08	0.01\\
13.09	0.01\\
13.1	0.01\\
13.11	0.01\\
13.12	0.01\\
13.13	0.01\\
13.14	0.01\\
13.15	0.01\\
13.16	0.01\\
13.17	0.01\\
13.18	0.01\\
13.19	0.01\\
13.2	0.01\\
13.21	0.01\\
13.22	0.01\\
13.23	0.01\\
13.24	0.01\\
13.25	0.01\\
13.26	0.01\\
13.27	0.01\\
13.28	0.01\\
13.29	0.01\\
13.3	0.01\\
13.31	0.01\\
13.32	0.01\\
13.33	0.01\\
13.34	0.01\\
13.35	0.01\\
13.36	0.01\\
13.37	0.01\\
13.38	0.01\\
13.39	0.01\\
13.4	0.01\\
13.41	0.01\\
13.42	0.01\\
13.43	0.01\\
13.44	0.01\\
13.45	0.01\\
13.46	0.01\\
13.47	0.01\\
13.48	0.01\\
13.49	0.01\\
13.5	0.01\\
13.51	0.01\\
13.52	0.01\\
13.53	0.01\\
13.54	0.01\\
13.55	0.01\\
13.56	0.01\\
13.57	0.01\\
13.58	0.01\\
13.59	0.01\\
13.6	0.01\\
13.61	0.01\\
13.62	0.01\\
13.63	0.01\\
13.64	0.01\\
13.65	0.01\\
13.66	0.01\\
13.67	0.01\\
13.68	0.01\\
13.69	0.01\\
13.7	0.01\\
13.71	0.01\\
13.72	0.01\\
13.73	0.01\\
13.74	0.01\\
13.75	0.01\\
13.76	0.01\\
13.77	0.01\\
13.78	0.01\\
13.79	0.01\\
13.8	0.01\\
13.81	0.01\\
13.82	0.01\\
13.83	0.01\\
13.84	0.01\\
13.85	0.01\\
13.86	0.01\\
13.87	0.01\\
13.88	0.01\\
13.89	0.01\\
13.9	0.01\\
13.91	0.01\\
13.92	0.01\\
13.93	0.01\\
13.94	0.01\\
13.95	0.01\\
13.96	0.01\\
13.97	0.01\\
13.98	0.01\\
13.99	0.01\\
14	0.01\\
14.01	0.01\\
14.02	0.01\\
14.03	0.01\\
14.04	0.01\\
14.05	0.01\\
14.06	0.01\\
14.07	0.01\\
14.08	0.01\\
14.09	0.01\\
14.1	0.01\\
14.11	0.01\\
14.12	0.01\\
14.13	0.01\\
14.14	0.01\\
14.15	0.01\\
14.16	0.01\\
14.17	0.01\\
14.18	0.01\\
14.19	0.01\\
14.2	0.01\\
14.21	0.01\\
14.22	0.01\\
14.23	0.01\\
14.24	0.01\\
14.25	0.01\\
14.26	0.01\\
14.27	0.01\\
14.28	0.01\\
14.29	0.01\\
14.3	0.01\\
14.31	0.01\\
14.32	0.01\\
14.33	0.01\\
14.34	0.01\\
14.35	0.01\\
14.36	0.01\\
14.37	0.01\\
14.38	0.01\\
14.39	0.01\\
14.4	0.01\\
14.41	0.01\\
14.42	0.01\\
14.43	0.01\\
14.44	0.01\\
14.45	0.01\\
14.46	0.01\\
14.47	0.01\\
14.48	0.01\\
14.49	0.01\\
14.5	0.01\\
14.51	0.01\\
14.52	0.01\\
14.53	0.01\\
14.54	0.01\\
14.55	0.01\\
14.56	0.01\\
14.57	0.01\\
14.58	0.01\\
14.59	0.01\\
14.6	0.01\\
14.61	0.01\\
14.62	0.01\\
14.63	0.01\\
14.64	0.01\\
14.65	0.01\\
14.66	0.01\\
14.67	0.01\\
14.68	0.01\\
14.69	0.01\\
14.7	0.01\\
14.71	0.01\\
14.72	0.01\\
14.73	0.01\\
14.74	0.01\\
14.75	0.01\\
14.76	0.01\\
14.77	0.01\\
14.78	0.01\\
14.79	0.01\\
14.8	0.01\\
14.81	0.01\\
14.82	0.01\\
14.83	0.01\\
14.84	0.01\\
14.85	0.01\\
14.86	0.01\\
14.87	0.01\\
14.88	0.01\\
14.89	0.01\\
14.9	0.01\\
14.91	0.01\\
14.92	0.01\\
14.93	0.01\\
14.94	0.01\\
14.95	0.01\\
14.96	0.01\\
14.97	0.01\\
14.98	0.01\\
14.99	0.01\\
15	0.01\\
15.01	0.01\\
15.02	0.01\\
15.03	0.01\\
15.04	0.01\\
15.05	0.01\\
15.06	0.01\\
15.07	0.01\\
15.08	0.01\\
15.09	0.01\\
15.1	0.01\\
15.11	0.01\\
15.12	0.01\\
15.13	0.01\\
15.14	0.01\\
15.15	0.01\\
15.16	0.01\\
15.17	0.01\\
15.18	0.01\\
15.19	0.01\\
15.2	0.01\\
15.21	0.01\\
15.22	0.01\\
15.23	0.01\\
15.24	0.01\\
15.25	0.01\\
15.26	0.01\\
15.27	0.01\\
15.28	0.01\\
15.29	0.01\\
15.3	0.01\\
15.31	0.01\\
15.32	0.01\\
15.33	0.01\\
15.34	0.01\\
15.35	0.01\\
15.36	0.01\\
15.37	0.01\\
15.38	0.01\\
15.39	0.01\\
15.4	0.01\\
15.41	0.01\\
15.42	0.01\\
15.43	0.01\\
15.44	0.01\\
15.45	0.01\\
15.46	0.01\\
15.47	0.01\\
15.48	0.01\\
15.49	0.01\\
15.5	0.01\\
15.51	0.01\\
15.52	0.01\\
15.53	0.01\\
15.54	0.01\\
15.55	0.01\\
15.56	0.01\\
15.57	0.01\\
15.58	0.01\\
15.59	0.01\\
15.6	0.01\\
15.61	0.01\\
15.62	0.01\\
15.63	0.01\\
15.64	0.01\\
15.65	0.01\\
15.66	0.01\\
15.67	0.01\\
15.68	0.01\\
15.69	0.01\\
15.7	0.01\\
15.71	0.01\\
15.72	0.01\\
15.73	0.01\\
15.74	0.01\\
15.75	0.01\\
15.76	0.01\\
15.77	0.01\\
15.78	0.01\\
15.79	0.01\\
15.8	0.01\\
15.81	0.01\\
15.82	0.01\\
15.83	0.01\\
15.84	0.01\\
15.85	0.01\\
15.86	0.01\\
15.87	0.01\\
15.88	0.01\\
15.89	0.01\\
15.9	0.01\\
15.91	0.01\\
15.92	0.01\\
15.93	0.01\\
15.94	0.01\\
15.95	0.01\\
15.96	0.01\\
15.97	0.01\\
15.98	0.01\\
15.99	0.01\\
16	0.01\\
16.01	0.01\\
16.02	0.01\\
16.03	0.01\\
16.04	0.01\\
16.05	0.01\\
16.06	0.01\\
16.07	0.01\\
16.08	0.01\\
16.09	0.01\\
16.1	0.01\\
16.11	0.01\\
16.12	0.01\\
16.13	0.01\\
16.14	0.01\\
16.15	0.01\\
16.16	0.01\\
16.17	0.01\\
16.18	0.01\\
16.19	0.01\\
16.2	0.01\\
16.21	0.01\\
16.22	0.01\\
16.23	0.01\\
16.24	0.01\\
16.25	0.01\\
16.26	0.01\\
16.27	0.01\\
16.28	0.01\\
16.29	0.01\\
16.3	0.01\\
16.31	0.01\\
16.32	0.01\\
16.33	0.01\\
16.34	0.01\\
16.35	0.01\\
16.36	0.01\\
16.37	0.01\\
16.38	0.01\\
16.39	0.01\\
16.4	0.01\\
16.41	0.01\\
16.42	0.01\\
16.43	0.01\\
16.44	0.01\\
16.45	0.01\\
16.46	0.01\\
16.47	0.01\\
16.48	0.01\\
16.49	0.01\\
16.5	0.01\\
16.51	0.01\\
16.52	0.01\\
16.53	0.01\\
16.54	0.01\\
16.55	0.01\\
16.56	0.01\\
16.57	0.01\\
16.58	0.01\\
16.59	0.01\\
16.6	0.01\\
16.61	0.01\\
16.62	0.01\\
16.63	0.01\\
16.64	0.01\\
16.65	0.01\\
16.66	0.01\\
16.67	0.01\\
16.68	0.01\\
16.69	0.01\\
16.7	0.01\\
16.71	0.01\\
16.72	0.01\\
16.73	0.01\\
16.74	0.01\\
16.75	0.01\\
16.76	0.01\\
16.77	0.01\\
16.78	0.01\\
16.79	0.01\\
16.8	0.01\\
16.81	0.01\\
16.82	0.01\\
16.83	0.01\\
16.84	0.01\\
16.85	0.01\\
16.86	0.01\\
16.87	0.01\\
16.88	0.01\\
16.89	0.01\\
16.9	0.01\\
16.91	0.01\\
16.92	0.01\\
16.93	0.01\\
16.94	0.01\\
16.95	0.01\\
16.96	0.01\\
16.97	0.01\\
16.98	0.01\\
16.99	0.01\\
17	0.01\\
17.01	0.01\\
17.02	0.01\\
17.03	0.01\\
17.04	0.01\\
17.05	0.01\\
17.06	0.01\\
17.07	0.01\\
17.08	0.01\\
17.09	0.01\\
17.1	0.01\\
17.11	0.01\\
17.12	0.01\\
17.13	0.01\\
17.14	0.01\\
17.15	0.01\\
17.16	0.01\\
17.17	0.01\\
17.18	0.01\\
17.19	0.01\\
17.2	0.01\\
17.21	0.01\\
17.22	0.01\\
17.23	0.01\\
17.24	0.01\\
17.25	0.01\\
17.26	0.01\\
17.27	0.01\\
17.28	0.01\\
17.29	0.01\\
17.3	0.01\\
17.31	0.01\\
17.32	0.01\\
17.33	0.01\\
17.34	0.01\\
17.35	0.01\\
17.36	0.01\\
17.37	0.01\\
17.38	0.01\\
17.39	0.01\\
17.4	0.01\\
17.41	0.01\\
17.42	0.01\\
17.43	0.01\\
17.44	0.01\\
17.45	0.01\\
17.46	0.01\\
17.47	0.01\\
17.48	0.01\\
17.49	0.01\\
17.5	0.01\\
17.51	0.01\\
17.52	0.01\\
17.53	0.01\\
17.54	0.01\\
17.55	0.01\\
17.56	0.01\\
17.57	0.01\\
17.58	0.01\\
17.59	0.01\\
17.6	0.01\\
17.61	0.01\\
17.62	0.01\\
17.63	0.01\\
17.64	0.01\\
17.65	0.01\\
17.66	0.01\\
17.67	0.01\\
17.68	0.01\\
17.69	0.01\\
17.7	0.01\\
17.71	0.01\\
17.72	0.01\\
17.73	0.01\\
17.74	0.01\\
17.75	0.01\\
17.76	0.01\\
17.77	0.01\\
17.78	0.01\\
17.79	0.01\\
17.8	0.01\\
17.81	0.01\\
17.82	0.01\\
17.83	0.01\\
17.84	0.01\\
17.85	0.01\\
17.86	0.01\\
17.87	0.01\\
17.88	0.01\\
17.89	0.01\\
17.9	0.01\\
17.91	0.01\\
17.92	0.01\\
17.93	0.01\\
17.94	0.01\\
17.95	0.01\\
17.96	0.01\\
17.97	0.01\\
17.98	0.01\\
17.99	0.01\\
18	0.01\\
18.01	0.01\\
18.02	0.01\\
18.03	0.01\\
18.04	0.01\\
18.05	0.01\\
18.06	0.01\\
18.07	0.01\\
18.08	0.01\\
18.09	0.01\\
18.1	0.01\\
18.11	0.01\\
18.12	0.01\\
18.13	0.01\\
18.14	0.01\\
18.15	0.01\\
18.16	0.01\\
18.17	0.01\\
18.18	0.01\\
18.19	0.01\\
18.2	0.01\\
18.21	0.01\\
18.22	0.01\\
18.23	0.01\\
18.24	0.01\\
18.25	0.01\\
18.26	0.01\\
18.27	0.01\\
18.28	0.01\\
18.29	0.01\\
18.3	0.01\\
18.31	0.01\\
18.32	0.01\\
18.33	0.01\\
18.34	0.01\\
18.35	0.01\\
18.36	0.01\\
18.37	0.01\\
18.38	0.01\\
18.39	0.01\\
18.4	0.01\\
18.41	0.01\\
18.42	0.01\\
18.43	0.01\\
18.44	0.01\\
18.45	0.01\\
18.46	0.01\\
18.47	0.01\\
18.48	0.01\\
18.49	0.01\\
18.5	0.01\\
18.51	0.01\\
18.52	0.01\\
18.53	0.01\\
18.54	0.01\\
18.55	0.01\\
18.56	0.01\\
18.57	0.01\\
18.58	0.01\\
18.59	0.01\\
18.6	0.01\\
18.61	0.01\\
18.62	0.01\\
18.63	0.01\\
18.64	0.01\\
18.65	0.01\\
18.66	0.01\\
18.67	0.01\\
18.68	0.01\\
18.69	0.01\\
18.7	0.01\\
18.71	0.01\\
18.72	0.01\\
18.73	0.01\\
18.74	0.01\\
18.75	0.01\\
18.76	0.01\\
18.77	0.01\\
18.78	0.01\\
18.79	0.01\\
18.8	0.01\\
18.81	0.01\\
18.82	0.01\\
18.83	0.01\\
18.84	0.01\\
18.85	0.01\\
18.86	0.01\\
18.87	0.01\\
18.88	0.01\\
18.89	0.01\\
18.9	0.01\\
18.91	0.01\\
18.92	0.01\\
18.93	0.01\\
18.94	0.01\\
18.95	0.01\\
18.96	0.01\\
18.97	0.01\\
18.98	0.01\\
18.99	0.01\\
19	0.01\\
19.01	0.01\\
19.02	0.01\\
19.03	0.01\\
19.04	0.01\\
19.05	0.01\\
19.06	0.01\\
19.07	0.01\\
19.08	0.01\\
19.09	0.01\\
19.1	0.01\\
19.11	0.01\\
19.12	0.01\\
19.13	0.01\\
19.14	0.01\\
19.15	0.01\\
19.16	0.01\\
19.17	0.01\\
19.18	0.01\\
19.19	0.01\\
19.2	0.01\\
19.21	0.01\\
19.22	0.01\\
19.23	0.01\\
19.24	0.01\\
19.25	0.01\\
19.26	0.01\\
19.27	0.01\\
19.28	0.01\\
19.29	0.01\\
19.3	0.01\\
19.31	0.01\\
19.32	0.01\\
19.33	0.01\\
19.34	0.01\\
19.35	0.01\\
19.36	0.01\\
19.37	0.01\\
19.38	0.01\\
19.39	0.01\\
19.4	0.01\\
19.41	0.01\\
19.42	0.01\\
19.43	0.01\\
19.44	0.01\\
19.45	0.01\\
19.46	0.01\\
19.47	0.01\\
19.48	0.01\\
19.49	0.01\\
19.5	0.01\\
19.51	0.01\\
19.52	0.01\\
19.53	0.01\\
19.54	0.01\\
19.55	0.01\\
19.56	0.01\\
19.57	0.01\\
19.58	0.01\\
19.59	0.01\\
19.6	0.01\\
19.61	0.01\\
19.62	0.01\\
19.63	0.01\\
19.64	0.01\\
19.65	0.01\\
19.66	0.01\\
19.67	0.01\\
19.68	0.01\\
19.69	0.01\\
19.7	0.01\\
19.71	0.01\\
19.72	0.01\\
19.73	0.01\\
19.74	0.01\\
19.75	0.01\\
19.76	0.01\\
19.77	0.01\\
19.78	0.01\\
19.79	0.01\\
19.8	0.01\\
19.81	0.01\\
19.82	0.01\\
19.83	0.01\\
19.84	0.01\\
19.85	0.01\\
19.86	0.01\\
19.87	0.01\\
19.88	0.01\\
19.89	0.01\\
19.9	0.01\\
19.91	0.01\\
19.92	0.01\\
19.93	0.01\\
19.94	0.01\\
19.95	0.01\\
19.96	0.01\\
19.97	0.01\\
19.98	0.01\\
19.99	0.01\\
20	0.01\\
20.01	0.01\\
20.02	0.01\\
20.03	0.01\\
20.04	0.01\\
20.05	0.01\\
20.06	0.01\\
20.07	0.01\\
20.08	0.01\\
20.09	0.01\\
20.1	0.01\\
20.11	0.01\\
20.12	0.01\\
20.13	0.01\\
20.14	0.01\\
20.15	0.01\\
20.16	0.01\\
20.17	0.01\\
20.18	0.01\\
20.19	0.01\\
20.2	0.01\\
20.21	0.01\\
20.22	0.01\\
20.23	0.01\\
20.24	0.01\\
20.25	0.01\\
20.26	0.01\\
20.27	0.01\\
20.28	0.01\\
20.29	0.01\\
20.3	0.01\\
20.31	0.01\\
20.32	0.01\\
20.33	0.01\\
20.34	0.01\\
20.35	0.01\\
20.36	0.01\\
20.37	0.01\\
20.38	0.01\\
20.39	0.01\\
20.4	0.01\\
20.41	0.01\\
20.42	0.01\\
20.43	0.01\\
20.44	0.01\\
20.45	0.01\\
20.46	0.01\\
20.47	0.01\\
20.48	0.01\\
20.49	0.01\\
20.5	0.01\\
20.51	0.01\\
20.52	0.01\\
20.53	0.01\\
20.54	0.01\\
20.55	0.01\\
20.56	0.01\\
20.57	0.01\\
20.58	0.01\\
20.59	0.01\\
20.6	0.01\\
20.61	0.01\\
20.62	0.01\\
20.63	0.01\\
20.64	0.01\\
20.65	0.01\\
20.66	0.01\\
20.67	0.01\\
20.68	0.01\\
20.69	0.01\\
20.7	0.01\\
20.71	0.01\\
20.72	0.01\\
20.73	0.01\\
20.74	0.01\\
20.75	0.01\\
20.76	0.01\\
20.77	0.01\\
20.78	0.01\\
20.79	0.01\\
20.8	0.01\\
20.81	0.01\\
20.82	0.01\\
20.83	0.01\\
20.84	0.01\\
20.85	0.01\\
20.86	0.01\\
20.87	0.01\\
20.88	0.01\\
20.89	0.01\\
20.9	0.01\\
20.91	0.01\\
20.92	0.01\\
20.93	0.01\\
20.94	0.01\\
20.95	0.01\\
20.96	0.01\\
20.97	0.01\\
20.98	0.01\\
20.99	0.01\\
21	0.01\\
21.01	0.01\\
21.02	0.01\\
21.03	0.01\\
21.04	0.01\\
21.05	0.01\\
21.06	0.01\\
21.07	0.01\\
21.08	0.01\\
21.09	0.01\\
21.1	0.01\\
21.11	0.01\\
21.12	0.01\\
21.13	0.01\\
21.14	0.01\\
21.15	0.01\\
21.16	0.01\\
21.17	0.01\\
21.18	0.01\\
21.19	0.01\\
21.2	0.01\\
21.21	0.01\\
21.22	0.01\\
21.23	0.01\\
21.24	0.01\\
21.25	0.01\\
21.26	0.01\\
21.27	0.01\\
21.28	0.01\\
21.29	0.01\\
21.3	0.01\\
21.31	0.01\\
21.32	0.01\\
21.33	0.01\\
21.34	0.01\\
21.35	0.01\\
21.36	0.01\\
21.37	0.01\\
21.38	0.01\\
21.39	0.01\\
21.4	0.01\\
21.41	0.01\\
21.42	0.01\\
21.43	0.01\\
21.44	0.01\\
21.45	0.01\\
21.46	0.01\\
21.47	0.01\\
21.48	0.01\\
21.49	0.01\\
21.5	0.01\\
21.51	0.01\\
21.52	0.01\\
21.53	0.01\\
21.54	0.01\\
21.55	0.01\\
21.56	0.01\\
21.57	0.01\\
21.58	0.01\\
21.59	0.01\\
21.6	0.01\\
21.61	0.01\\
21.62	0.01\\
21.63	0.01\\
21.64	0.01\\
21.65	0.01\\
21.66	0.01\\
21.67	0.01\\
21.68	0.01\\
21.69	0.01\\
21.7	0.01\\
21.71	0.01\\
21.72	0.01\\
21.73	0.01\\
21.74	0.01\\
21.75	0.01\\
21.76	0.01\\
21.77	0.01\\
21.78	0.01\\
21.79	0.01\\
21.8	0.01\\
21.81	0.01\\
21.82	0.01\\
21.83	0.01\\
21.84	0.01\\
21.85	0.01\\
21.86	0.01\\
21.87	0.01\\
21.88	0.01\\
21.89	0.01\\
21.9	0.01\\
21.91	0.01\\
21.92	0.01\\
21.93	0.01\\
21.94	0.01\\
21.95	0.01\\
21.96	0.01\\
21.97	0.01\\
21.98	0.01\\
21.99	0.01\\
22	0.01\\
22.01	0.01\\
22.02	0.01\\
22.03	0.01\\
22.04	0.01\\
22.05	0.01\\
22.06	0.01\\
22.07	0.01\\
22.08	0.01\\
22.09	0.01\\
22.1	0.01\\
22.11	0.01\\
22.12	0.01\\
22.13	0.01\\
22.14	0.01\\
22.15	0.01\\
22.16	0.01\\
22.17	0.01\\
22.18	0.01\\
22.19	0.01\\
22.2	0.01\\
22.21	0.01\\
22.22	0.01\\
22.23	0.01\\
22.24	0.01\\
22.25	0.01\\
22.26	0.01\\
22.27	0.01\\
22.28	0.01\\
22.29	0.01\\
22.3	0.01\\
22.31	0.01\\
22.32	0.01\\
22.33	0.01\\
22.34	0.01\\
22.35	0.01\\
22.36	0.01\\
22.37	0.01\\
22.38	0.01\\
22.39	0.01\\
22.4	0.01\\
22.41	0.01\\
22.42	0.01\\
22.43	0.01\\
22.44	0.01\\
22.45	0.01\\
22.46	0.01\\
22.47	0.01\\
22.48	0.01\\
22.49	0.01\\
22.5	0.01\\
22.51	0.01\\
22.52	0.01\\
22.53	0.01\\
22.54	0.01\\
22.55	0.01\\
22.56	0.01\\
22.57	0.01\\
22.58	0.01\\
22.59	0.01\\
22.6	0.01\\
22.61	0.01\\
22.62	0.01\\
22.63	0.01\\
22.64	0.01\\
22.65	0.01\\
22.66	0.01\\
22.67	0.01\\
22.68	0.01\\
22.69	0.01\\
22.7	0.01\\
22.71	0.01\\
22.72	0.01\\
22.73	0.01\\
22.74	0.01\\
22.75	0.01\\
22.76	0.01\\
22.77	0.01\\
22.78	0.01\\
22.79	0.01\\
22.8	0.01\\
22.81	0.01\\
22.82	0.01\\
22.83	0.01\\
22.84	0.01\\
22.85	0.01\\
22.86	0.01\\
22.87	0.01\\
22.88	0.01\\
22.89	0.01\\
22.9	0.01\\
22.91	0.01\\
22.92	0.01\\
22.93	0.01\\
22.94	0.01\\
22.95	0.01\\
22.96	0.01\\
22.97	0.01\\
22.98	0.01\\
22.99	0.01\\
23	0.01\\
23.01	0.01\\
23.02	0.01\\
23.03	0.01\\
23.04	0.01\\
23.05	0.01\\
23.06	0.01\\
23.07	0.01\\
23.08	0.01\\
23.09	0.01\\
23.1	0.01\\
23.11	0.01\\
23.12	0.01\\
23.13	0.01\\
23.14	0.01\\
23.15	0.01\\
23.16	0.01\\
23.17	0.01\\
23.18	0.01\\
23.19	0.01\\
23.2	0.01\\
23.21	0.01\\
23.22	0.01\\
23.23	0.01\\
23.24	0.01\\
23.25	0.01\\
23.26	0.01\\
23.27	0.01\\
23.28	0.01\\
23.29	0.01\\
23.3	0.01\\
23.31	0.01\\
23.32	0.01\\
23.33	0.01\\
23.34	0.01\\
23.35	0.01\\
23.36	0.01\\
23.37	0.01\\
23.38	0.01\\
23.39	0.01\\
23.4	0.01\\
23.41	0.01\\
23.42	0.01\\
23.43	0.01\\
23.44	0.01\\
23.45	0.01\\
23.46	0.01\\
23.47	0.01\\
23.48	0.01\\
23.49	0.01\\
23.5	0.01\\
23.51	0.01\\
23.52	0.01\\
23.53	0.01\\
23.54	0.01\\
23.55	0.01\\
23.56	0.01\\
23.57	0.01\\
23.58	0.01\\
23.59	0.01\\
23.6	0.01\\
23.61	0.01\\
23.62	0.01\\
23.63	0.01\\
23.64	0.01\\
23.65	0.01\\
23.66	0.01\\
23.67	0.01\\
23.68	0.01\\
23.69	0.01\\
23.7	0.01\\
23.71	0.01\\
23.72	0.01\\
23.73	0.01\\
23.74	0.01\\
23.75	0.01\\
23.76	0.01\\
23.77	0.01\\
23.78	0.01\\
23.79	0.01\\
23.8	0.01\\
23.81	0.01\\
23.82	0.01\\
23.83	0.01\\
23.84	0.01\\
23.85	0.01\\
23.86	0.01\\
23.87	0.01\\
23.88	0.01\\
23.89	0.01\\
23.9	0.01\\
23.91	0.01\\
23.92	0.01\\
23.93	0.01\\
23.94	0.01\\
23.95	0.01\\
23.96	0.01\\
23.97	0.01\\
23.98	0.01\\
23.99	0.01\\
24	0.01\\
24.01	0.01\\
24.02	0.01\\
24.03	0.01\\
24.04	0.01\\
24.05	0.01\\
24.06	0.01\\
24.07	0.01\\
24.08	0.01\\
24.09	0.01\\
24.1	0.01\\
24.11	0.01\\
24.12	0.01\\
24.13	0.01\\
24.14	0.01\\
24.15	0.01\\
24.16	0.01\\
24.17	0.01\\
24.18	0.01\\
24.19	0.01\\
24.2	0.01\\
24.21	0.01\\
24.22	0.01\\
24.23	0.01\\
24.24	0.01\\
24.25	0.01\\
24.26	0.01\\
24.27	0.01\\
24.28	0.01\\
24.29	0.01\\
24.3	0.01\\
24.31	0.01\\
24.32	0.01\\
24.33	0.01\\
24.34	0.01\\
24.35	0.01\\
24.36	0.01\\
24.37	0.01\\
24.38	0.01\\
24.39	0.01\\
24.4	0.01\\
24.41	0.01\\
24.42	0.01\\
24.43	0.01\\
24.44	0.01\\
24.45	0.01\\
24.46	0.01\\
24.47	0.01\\
24.48	0.01\\
24.49	0.01\\
24.5	0.01\\
24.51	0.01\\
24.52	0.01\\
24.53	0.01\\
24.54	0.01\\
24.55	0.01\\
24.56	0.01\\
24.57	0.01\\
24.58	0.01\\
24.59	0.01\\
24.6	0.01\\
24.61	0.01\\
24.62	0.01\\
24.63	0.01\\
24.64	0.01\\
24.65	0.01\\
24.66	0.01\\
24.67	0.01\\
24.68	0.01\\
24.69	0.01\\
24.7	0.01\\
24.71	0.01\\
24.72	0.01\\
24.73	0.01\\
24.74	0.01\\
24.75	0.01\\
24.76	0.01\\
24.77	0.01\\
24.78	0.01\\
24.79	0.01\\
24.8	0.01\\
24.81	0.01\\
24.82	0.01\\
24.83	0.01\\
24.84	0.01\\
24.85	0.01\\
24.86	0.01\\
24.87	0.01\\
24.88	0.01\\
24.89	0.01\\
24.9	0.01\\
24.91	0.01\\
24.92	0.01\\
24.93	0.01\\
24.94	0.01\\
24.95	0.01\\
24.96	0.01\\
24.97	0.01\\
24.98	0.01\\
24.99	0.01\\
25	0.01\\
25.01	0.01\\
25.02	0.01\\
25.03	0.01\\
25.04	0.01\\
25.05	0.01\\
25.06	0.01\\
25.07	0.01\\
25.08	0.01\\
25.09	0.01\\
25.1	0.01\\
25.11	0.01\\
25.12	0.01\\
25.13	0.01\\
25.14	0.01\\
25.15	0.01\\
25.16	0.01\\
25.17	0.01\\
25.18	0.01\\
25.19	0.01\\
25.2	0.01\\
25.21	0.01\\
25.22	0.01\\
25.23	0.01\\
25.24	0.01\\
25.25	0.01\\
25.26	0.01\\
25.27	0.01\\
25.28	0.01\\
25.29	0.01\\
25.3	0.01\\
25.31	0.01\\
25.32	0.01\\
25.33	0.01\\
25.34	0.01\\
25.35	0.01\\
25.36	0.01\\
25.37	0.01\\
25.38	0.01\\
25.39	0.01\\
25.4	0.01\\
25.41	0.01\\
25.42	0.01\\
25.43	0.01\\
25.44	0.01\\
25.45	0.01\\
25.46	0.01\\
25.47	0.01\\
25.48	0.01\\
25.49	0.01\\
25.5	0.01\\
25.51	0.01\\
25.52	0.01\\
25.53	0.01\\
25.54	0.01\\
25.55	0.01\\
25.56	0.01\\
25.57	0.01\\
25.58	0.01\\
25.59	0.01\\
25.6	0.01\\
25.61	0.01\\
25.62	0.01\\
25.63	0.01\\
25.64	0.01\\
25.65	0.01\\
25.66	0.01\\
25.67	0.01\\
25.68	0.01\\
25.69	0.01\\
25.7	0.01\\
25.71	0.01\\
25.72	0.01\\
25.73	0.01\\
25.74	0.01\\
25.75	0.01\\
25.76	0.01\\
25.77	0.01\\
25.78	0.01\\
25.79	0.01\\
25.8	0.01\\
25.81	0.01\\
25.82	0.01\\
25.83	0.01\\
25.84	0.01\\
25.85	0.01\\
25.86	0.01\\
25.87	0.01\\
25.88	0.01\\
25.89	0.01\\
25.9	0.01\\
25.91	0.01\\
25.92	0.01\\
25.93	0.01\\
25.94	0.01\\
25.95	0.01\\
25.96	0.01\\
25.97	0.01\\
25.98	0.01\\
25.99	0.01\\
26	0.01\\
26.01	0.01\\
26.02	0.01\\
26.03	0.01\\
26.04	0.01\\
26.05	0.01\\
26.06	0.01\\
26.07	0.01\\
26.08	0.01\\
26.09	0.01\\
26.1	0.01\\
26.11	0.01\\
26.12	0.01\\
26.13	0.01\\
26.14	0.01\\
26.15	0.01\\
26.16	0.01\\
26.17	0.01\\
26.18	0.01\\
26.19	0.01\\
26.2	0.01\\
26.21	0.01\\
26.22	0.01\\
26.23	0.01\\
26.24	0.01\\
26.25	0.01\\
26.26	0.01\\
26.27	0.01\\
26.28	0.01\\
26.29	0.01\\
26.3	0.01\\
26.31	0.01\\
26.32	0.01\\
26.33	0.01\\
26.34	0.01\\
26.35	0.01\\
26.36	0.01\\
26.37	0.01\\
26.38	0.01\\
26.39	0.01\\
26.4	0.01\\
26.41	0.01\\
26.42	0.01\\
26.43	0.01\\
26.44	0.01\\
26.45	0.01\\
26.46	0.01\\
26.47	0.01\\
26.48	0.01\\
26.49	0.01\\
26.5	0.01\\
26.51	0.01\\
26.52	0.01\\
26.53	0.01\\
26.54	0.01\\
26.55	0.01\\
26.56	0.01\\
26.57	0.01\\
26.58	0.01\\
26.59	0.01\\
26.6	0.01\\
26.61	0.01\\
26.62	0.01\\
26.63	0.01\\
26.64	0.01\\
26.65	0.01\\
26.66	0.01\\
26.67	0.01\\
26.68	0.01\\
26.69	0.01\\
26.7	0.01\\
26.71	0.01\\
26.72	0.01\\
26.73	0.01\\
26.74	0.01\\
26.75	0.01\\
26.76	0.01\\
26.77	0.01\\
26.78	0.01\\
26.79	0.01\\
26.8	0.01\\
26.81	0.01\\
26.82	0.01\\
26.83	0.01\\
26.84	0.01\\
26.85	0.01\\
26.86	0.01\\
26.87	0.01\\
26.88	0.01\\
26.89	0.01\\
26.9	0.01\\
26.91	0.01\\
26.92	0.01\\
26.93	0.01\\
26.94	0.01\\
26.95	0.01\\
26.96	0.01\\
26.97	0.01\\
26.98	0.01\\
26.99	0.01\\
27	0.01\\
27.01	0.01\\
27.02	0.01\\
27.03	0.01\\
27.04	0.01\\
27.05	0.01\\
27.06	0.01\\
27.07	0.01\\
27.08	0.01\\
27.09	0.01\\
27.1	0.01\\
27.11	0.01\\
27.12	0.01\\
27.13	0.01\\
27.14	0.01\\
27.15	0.01\\
27.16	0.01\\
27.17	0.01\\
27.18	0.01\\
27.19	0.01\\
27.2	0.01\\
27.21	0.01\\
27.22	0.01\\
27.23	0.01\\
27.24	0.01\\
27.25	0.01\\
27.26	0.01\\
27.27	0.01\\
27.28	0.01\\
27.29	0.01\\
27.3	0.01\\
27.31	0.01\\
27.32	0.01\\
27.33	0.01\\
27.34	0.01\\
27.35	0.01\\
27.36	0.01\\
27.37	0.01\\
27.38	0.01\\
27.39	0.01\\
27.4	0.01\\
27.41	0.01\\
27.42	0.01\\
27.43	0.01\\
27.44	0.01\\
27.45	0.01\\
27.46	0.01\\
27.47	0.01\\
27.48	0.01\\
27.49	0.01\\
27.5	0.01\\
27.51	0.01\\
27.52	0.01\\
27.53	0.01\\
27.54	0.01\\
27.55	0.01\\
27.56	0.01\\
27.57	0.01\\
27.58	0.01\\
27.59	0.01\\
27.6	0.01\\
27.61	0.01\\
27.62	0.01\\
27.63	0.01\\
27.64	0.01\\
27.65	0.01\\
27.66	0.01\\
27.67	0.01\\
27.68	0.01\\
27.69	0.01\\
27.7	0.01\\
27.71	0.01\\
27.72	0.01\\
27.73	0.01\\
27.74	0.01\\
27.75	0.01\\
27.76	0.01\\
27.77	0.01\\
27.78	0.01\\
27.79	0.01\\
27.8	0.01\\
27.81	0.01\\
27.82	0.01\\
27.83	0.01\\
27.84	0.01\\
27.85	0.01\\
27.86	0.01\\
27.87	0.01\\
27.88	0.01\\
27.89	0.01\\
27.9	0.01\\
27.91	0.01\\
27.92	0.01\\
27.93	0.01\\
27.94	0.01\\
27.95	0.01\\
27.96	0.01\\
27.97	0.01\\
27.98	0.01\\
27.99	0.01\\
28	0.01\\
28.01	0.01\\
28.02	0.01\\
28.03	0.01\\
28.04	0.01\\
28.05	0.01\\
28.06	0.01\\
28.07	0.01\\
28.08	0.01\\
28.09	0.01\\
28.1	0.01\\
28.11	0.01\\
28.12	0.01\\
28.13	0.01\\
28.14	0.01\\
28.15	0.01\\
28.16	0.01\\
28.17	0.01\\
28.18	0.01\\
28.19	0.01\\
28.2	0.01\\
28.21	0.01\\
28.22	0.01\\
28.23	0.01\\
28.24	0.01\\
28.25	0.01\\
28.26	0.01\\
28.27	0.01\\
28.28	0.01\\
28.29	0.01\\
28.3	0.01\\
28.31	0.01\\
28.32	0.01\\
28.33	0.01\\
28.34	0.01\\
28.35	0.01\\
28.36	0.01\\
28.37	0.01\\
28.38	0.01\\
28.39	0.01\\
28.4	0.01\\
28.41	0.01\\
28.42	0.01\\
28.43	0.01\\
28.44	0.01\\
28.45	0.01\\
28.46	0.01\\
28.47	0.01\\
28.48	0.01\\
28.49	0.01\\
28.5	0.01\\
28.51	0.01\\
28.52	0.01\\
28.53	0.01\\
28.54	0.01\\
28.55	0.01\\
28.56	0.01\\
28.57	0.01\\
28.58	0.01\\
28.59	0.01\\
28.6	0.01\\
28.61	0.01\\
28.62	0.01\\
28.63	0.01\\
28.64	0.01\\
28.65	0.01\\
28.66	0.01\\
28.67	0.01\\
28.68	0.01\\
28.69	0.01\\
28.7	0.01\\
28.71	0.01\\
28.72	0.01\\
28.73	0.01\\
28.74	0.01\\
28.75	0.01\\
28.76	0.01\\
28.77	0.01\\
28.78	0.01\\
28.79	0.01\\
28.8	0.01\\
28.81	0.01\\
28.82	0.01\\
28.83	0.01\\
28.84	0.01\\
28.85	0.01\\
28.86	0.01\\
28.87	0.01\\
28.88	0.01\\
28.89	0.01\\
28.9	0.01\\
28.91	0.01\\
28.92	0.01\\
28.93	0.01\\
28.94	0.01\\
28.95	0.01\\
28.96	0.01\\
28.97	0.01\\
28.98	0.01\\
28.99	0.01\\
29	0.01\\
29.01	0.01\\
29.02	0.01\\
29.03	0.01\\
29.04	0.01\\
29.05	0.01\\
29.06	0.01\\
29.07	0.01\\
29.08	0.01\\
29.09	0.01\\
29.1	0.01\\
29.11	0.01\\
29.12	0.01\\
29.13	0.01\\
29.14	0.01\\
29.15	0.01\\
29.16	0.01\\
29.17	0.01\\
29.18	0.01\\
29.19	0.01\\
29.2	0.01\\
29.21	0.01\\
29.22	0.01\\
29.23	0.01\\
29.24	0.01\\
29.25	0.01\\
29.26	0.01\\
29.27	0.01\\
29.28	0.01\\
29.29	0.01\\
29.3	0.01\\
29.31	0.01\\
29.32	0.01\\
29.33	0.01\\
29.34	0.01\\
29.35	0.01\\
29.36	0.01\\
29.37	0.01\\
29.38	0.01\\
29.39	0.01\\
29.4	0.01\\
29.41	0.01\\
29.42	0.01\\
29.43	0.01\\
29.44	0.01\\
29.45	0.01\\
29.46	0.01\\
29.47	0.01\\
29.48	0.01\\
29.49	0.01\\
29.5	0.01\\
29.51	0.01\\
29.52	0.01\\
29.53	0.01\\
29.54	0.01\\
29.55	0.01\\
29.56	0.01\\
29.57	0.01\\
29.58	0.01\\
29.59	0.01\\
29.6	0.01\\
29.61	0.01\\
29.62	0.01\\
29.63	0.01\\
29.64	0.01\\
29.65	0.01\\
29.66	0.01\\
29.67	0.01\\
29.68	0.01\\
29.69	0.01\\
29.7	0.01\\
29.71	0.01\\
29.72	0.01\\
29.73	0.01\\
29.74	0.01\\
29.75	0.01\\
29.76	0.01\\
29.77	0.01\\
29.78	0.01\\
29.79	0.01\\
29.8	0.01\\
29.81	0.01\\
29.82	0.01\\
29.83	0.01\\
29.84	0.01\\
29.85	0.01\\
29.86	0.01\\
29.87	0.01\\
29.88	0.01\\
29.89	0.01\\
29.9	0.01\\
29.91	0.01\\
29.92	0.01\\
29.93	0.01\\
29.94	0.01\\
29.95	0.01\\
29.96	0.01\\
29.97	0.01\\
29.98	0.01\\
29.99	0.01\\
30	0.01\\
30.01	0.01\\
30.02	0.01\\
30.03	0.01\\
30.04	0.01\\
30.05	0.01\\
30.06	0.01\\
30.07	0.01\\
30.08	0.01\\
30.09	0.01\\
30.1	0.01\\
30.11	0.01\\
30.12	0.01\\
30.13	0.01\\
30.14	0.01\\
30.15	0.01\\
30.16	0.01\\
30.17	0.01\\
30.18	0.01\\
30.19	0.01\\
30.2	0.01\\
30.21	0.01\\
30.22	0.01\\
30.23	0.01\\
30.24	0.01\\
30.25	0.01\\
30.26	0.01\\
30.27	0.01\\
30.28	0.01\\
30.29	0.01\\
30.3	0.01\\
30.31	0.01\\
30.32	0.01\\
30.33	0.01\\
30.34	0.01\\
30.35	0.01\\
30.36	0.01\\
30.37	0.01\\
30.38	0.01\\
30.39	0.01\\
30.4	0.01\\
30.41	0.01\\
30.42	0.01\\
30.43	0.01\\
30.44	0.01\\
30.45	0.01\\
30.46	0.01\\
30.47	0.01\\
30.48	0.01\\
30.49	0.01\\
30.5	0.01\\
30.51	0.01\\
30.52	0.01\\
30.53	0.01\\
30.54	0.01\\
30.55	0.01\\
30.56	0.01\\
30.57	0.01\\
30.58	0.01\\
30.59	0.01\\
30.6	0.01\\
30.61	0.01\\
30.62	0.01\\
30.63	0.01\\
30.64	0.01\\
30.65	0.01\\
30.66	0.01\\
30.67	0.01\\
30.68	0.01\\
30.69	0.01\\
30.7	0.01\\
30.71	0.01\\
30.72	0.01\\
30.73	0.01\\
30.74	0.01\\
30.75	0.01\\
30.76	0.01\\
30.77	0.01\\
30.78	0.01\\
30.79	0.01\\
30.8	0.01\\
30.81	0.01\\
30.82	0.01\\
30.83	0.01\\
30.84	0.01\\
30.85	0.01\\
30.86	0.01\\
30.87	0.01\\
30.88	0.01\\
30.89	0.01\\
30.9	0.01\\
30.91	0.01\\
30.92	0.01\\
30.93	0.01\\
30.94	0.01\\
30.95	0.01\\
30.96	0.01\\
30.97	0.01\\
30.98	0.01\\
30.99	0.01\\
31	0.01\\
31.01	0.01\\
31.02	0.01\\
31.03	0.01\\
31.04	0.01\\
31.05	0.01\\
31.06	0.01\\
31.07	0.01\\
31.08	0.01\\
31.09	0.01\\
31.1	0.01\\
31.11	0.01\\
31.12	0.01\\
31.13	0.01\\
31.14	0.01\\
31.15	0.01\\
31.16	0.01\\
31.17	0.01\\
31.18	0.01\\
31.19	0.01\\
31.2	0.01\\
31.21	0.01\\
31.22	0.01\\
31.23	0.01\\
31.24	0.01\\
31.25	0.01\\
31.26	0.01\\
31.27	0.01\\
31.28	0.01\\
31.29	0.01\\
31.3	0.01\\
31.31	0.01\\
31.32	0.01\\
31.33	0.01\\
31.34	0.01\\
31.35	0.01\\
31.36	0.01\\
31.37	0.01\\
31.38	0.01\\
31.39	0.01\\
31.4	0.01\\
31.41	0.01\\
31.42	0.01\\
31.43	0.01\\
31.44	0.01\\
31.45	0.01\\
31.46	0.01\\
31.47	0.01\\
31.48	0.01\\
31.49	0.01\\
31.5	0.01\\
31.51	0.01\\
31.52	0.01\\
31.53	0.01\\
31.54	0.01\\
31.55	0.01\\
31.56	0.01\\
31.57	0.01\\
31.58	0.01\\
31.59	0.01\\
31.6	0.01\\
31.61	0.01\\
31.62	0.01\\
31.63	0.01\\
31.64	0.01\\
31.65	0.01\\
31.66	0.01\\
31.67	0.01\\
31.68	0.01\\
31.69	0.01\\
31.7	0.01\\
31.71	0.01\\
31.72	0.01\\
31.73	0.01\\
31.74	0.01\\
31.75	0.01\\
31.76	0.01\\
31.77	0.01\\
31.78	0.01\\
31.79	0.01\\
31.8	0.01\\
31.81	0.01\\
31.82	0.01\\
31.83	0.01\\
31.84	0.01\\
31.85	0.01\\
31.86	0.01\\
31.87	0.01\\
31.88	0.01\\
31.89	0.01\\
31.9	0.01\\
31.91	0.01\\
31.92	0.01\\
31.93	0.01\\
31.94	0.01\\
31.95	0.01\\
31.96	0.01\\
31.97	0.01\\
31.98	0.01\\
31.99	0.01\\
32	0.01\\
32.01	0.01\\
32.02	0.01\\
32.03	0.01\\
32.04	0.01\\
32.05	0.01\\
32.06	0.01\\
32.07	0.01\\
32.08	0.01\\
32.09	0.01\\
32.1	0.01\\
32.11	0.01\\
32.12	0.01\\
32.13	0.01\\
32.14	0.01\\
32.15	0.01\\
32.16	0.01\\
32.17	0.01\\
32.18	0.01\\
32.19	0.01\\
32.2	0.01\\
32.21	0.01\\
32.22	0.01\\
32.23	0.01\\
32.24	0.01\\
32.25	0.01\\
32.26	0.01\\
32.27	0.01\\
32.28	0.01\\
32.29	0.01\\
32.3	0.01\\
32.31	0.01\\
32.32	0.01\\
32.33	0.01\\
32.34	0.01\\
32.35	0.01\\
32.36	0.01\\
32.37	0.01\\
32.38	0.01\\
32.39	0.01\\
32.4	0.01\\
32.41	0.01\\
32.42	0.01\\
32.43	0.01\\
32.44	0.01\\
32.45	0.01\\
32.46	0.01\\
32.47	0.01\\
32.48	0.01\\
32.49	0.01\\
32.5	0.01\\
32.51	0.01\\
32.52	0.01\\
32.53	0.01\\
32.54	0.01\\
32.55	0.01\\
32.56	0.01\\
32.57	0.01\\
32.58	0.01\\
32.59	0.01\\
32.6	0.01\\
32.61	0.01\\
32.62	0.01\\
32.63	0.01\\
32.64	0.01\\
32.65	0.01\\
32.66	0.01\\
32.67	0.01\\
32.68	0.01\\
32.69	0.01\\
32.7	0.01\\
32.71	0.01\\
32.72	0.01\\
32.73	0.01\\
32.74	0.01\\
32.75	0.01\\
32.76	0.01\\
32.77	0.01\\
32.78	0.01\\
32.79	0.01\\
32.8	0.01\\
32.81	0.01\\
32.82	0.01\\
32.83	0.01\\
32.84	0.01\\
32.85	0.01\\
32.86	0.01\\
32.87	0.01\\
32.88	0.01\\
32.89	0.01\\
32.9	0.01\\
32.91	0.01\\
32.92	0.01\\
32.93	0.01\\
32.94	0.01\\
32.95	0.01\\
32.96	0.01\\
32.97	0.01\\
32.98	0.01\\
32.99	0.01\\
33	0.01\\
33.01	0.01\\
33.02	0.01\\
33.03	0.01\\
33.04	0.01\\
33.05	0.01\\
33.06	0.01\\
33.07	0.01\\
33.08	0.01\\
33.09	0.01\\
33.1	0.01\\
33.11	0.01\\
33.12	0.01\\
33.13	0.01\\
33.14	0.01\\
33.15	0.01\\
33.16	0.01\\
33.17	0.01\\
33.18	0.01\\
33.19	0.01\\
33.2	0.01\\
33.21	0.01\\
33.22	0.01\\
33.23	0.01\\
33.24	0.01\\
33.25	0.01\\
33.26	0.01\\
33.27	0.01\\
33.28	0.01\\
33.29	0.01\\
33.3	0.01\\
33.31	0.01\\
33.32	0.01\\
33.33	0.01\\
33.34	0.01\\
33.35	0.01\\
33.36	0.01\\
33.37	0.01\\
33.38	0.01\\
33.39	0.01\\
33.4	0.01\\
33.41	0.01\\
33.42	0.01\\
33.43	0.01\\
33.44	0.01\\
33.45	0.01\\
33.46	0.01\\
33.47	0.01\\
33.48	0.01\\
33.49	0.01\\
33.5	0.01\\
33.51	0.01\\
33.52	0.01\\
33.53	0.01\\
33.54	0.01\\
33.55	0.01\\
33.56	0.01\\
33.57	0.01\\
33.58	0.01\\
33.59	0.01\\
33.6	0.01\\
33.61	0.01\\
33.62	0.01\\
33.63	0.01\\
33.64	0.01\\
33.65	0.01\\
33.66	0.01\\
33.67	0.01\\
33.68	0.01\\
33.69	0.01\\
33.7	0.01\\
33.71	0.01\\
33.72	0.01\\
33.73	0.01\\
33.74	0.01\\
33.75	0.01\\
33.76	0.01\\
33.77	0.01\\
33.78	0.01\\
33.79	0.01\\
33.8	0.01\\
33.81	0.01\\
33.82	0.01\\
33.83	0.01\\
33.84	0.01\\
33.85	0.01\\
33.86	0.01\\
33.87	0.01\\
33.88	0.01\\
33.89	0.01\\
33.9	0.01\\
33.91	0.01\\
33.92	0.01\\
33.93	0.01\\
33.94	0.01\\
33.95	0.01\\
33.96	0.01\\
33.97	0.01\\
33.98	0.01\\
33.99	0.01\\
34	0.01\\
34.01	0.01\\
34.02	0.01\\
34.03	0.01\\
34.04	0.01\\
34.05	0.01\\
34.06	0.01\\
34.07	0.01\\
34.08	0.01\\
34.09	0.01\\
34.1	0.01\\
34.11	0.01\\
34.12	0.01\\
34.13	0.01\\
34.14	0.01\\
34.15	0.01\\
34.16	0.01\\
34.17	0.01\\
34.18	0.01\\
34.19	0.01\\
34.2	0.01\\
34.21	0.01\\
34.22	0.01\\
34.23	0.01\\
34.24	0.01\\
34.25	0.01\\
34.26	0.01\\
34.27	0.01\\
34.28	0.01\\
34.29	0.01\\
34.3	0.01\\
34.31	0.01\\
34.32	0.01\\
34.33	0.01\\
34.34	0.01\\
34.35	0.01\\
34.36	0.01\\
34.37	0.01\\
34.38	0.01\\
34.39	0.01\\
34.4	0.01\\
34.41	0.01\\
34.42	0.01\\
34.43	0.01\\
34.44	0.01\\
34.45	0.01\\
34.46	0.01\\
34.47	0.01\\
34.48	0.01\\
34.49	0.01\\
34.5	0.01\\
34.51	0.01\\
34.52	0.01\\
34.53	0.01\\
34.54	0.01\\
34.55	0.01\\
34.56	0.01\\
34.57	0.01\\
34.58	0.01\\
34.59	0.01\\
34.6	0.01\\
34.61	0.01\\
34.62	0.01\\
34.63	0.01\\
34.64	0.01\\
34.65	0.01\\
34.66	0.01\\
34.67	0.01\\
34.68	0.01\\
34.69	0.01\\
34.7	0.01\\
34.71	0.01\\
34.72	0.01\\
34.73	0.01\\
34.74	0.01\\
34.75	0.01\\
34.76	0.01\\
34.77	0.01\\
34.78	0.01\\
34.79	0.01\\
34.8	0.01\\
34.81	0.01\\
34.82	0.01\\
34.83	0.01\\
34.84	0.01\\
34.85	0.01\\
34.86	0.01\\
34.87	0.01\\
34.88	0.01\\
34.89	0.01\\
34.9	0.01\\
34.91	0.01\\
34.92	0.01\\
34.93	0.01\\
34.94	0.01\\
34.95	0.01\\
34.96	0.01\\
34.97	0.01\\
34.98	0.01\\
34.99	0.01\\
35	0.01\\
35.01	0.01\\
35.02	0.01\\
35.03	0.01\\
35.04	0.01\\
35.05	0.01\\
35.06	0.01\\
35.07	0.01\\
35.08	0.01\\
35.09	0.01\\
35.1	0.01\\
35.11	0.01\\
35.12	0.01\\
35.13	0.01\\
35.14	0.01\\
35.15	0.01\\
35.16	0.01\\
35.17	0.01\\
35.18	0.01\\
35.19	0.01\\
35.2	0.01\\
35.21	0.01\\
35.22	0.01\\
35.23	0.01\\
35.24	0.01\\
35.25	0.01\\
35.26	0.01\\
35.27	0.01\\
35.28	0.01\\
35.29	0.01\\
35.3	0.01\\
35.31	0.01\\
35.32	0.01\\
35.33	0.01\\
35.34	0.01\\
35.35	0.01\\
35.36	0.01\\
35.37	0.01\\
35.38	0.01\\
35.39	0.01\\
35.4	0.01\\
35.41	0.01\\
35.42	0.01\\
35.43	0.01\\
35.44	0.01\\
35.45	0.01\\
35.46	0.01\\
35.47	0.01\\
35.48	0.01\\
35.49	0.01\\
35.5	0.01\\
35.51	0.01\\
35.52	0.01\\
35.53	0.01\\
35.54	0.01\\
35.55	0.01\\
35.56	0.01\\
35.57	0.01\\
35.58	0.01\\
35.59	0.01\\
35.6	0.01\\
35.61	0.01\\
35.62	0.01\\
35.63	0.01\\
35.64	0.01\\
35.65	0.01\\
35.66	0.01\\
35.67	0.01\\
35.68	0.01\\
35.69	0.01\\
35.7	0.01\\
35.71	0.01\\
35.72	0.01\\
35.73	0.01\\
35.74	0.01\\
35.75	0.01\\
35.76	0.01\\
35.77	0.01\\
35.78	0.01\\
35.79	0.01\\
35.8	0.01\\
35.81	0.01\\
35.82	0.01\\
35.83	0.01\\
35.84	0.01\\
35.85	0.01\\
35.86	0.01\\
35.87	0.01\\
35.88	0.01\\
35.89	0.01\\
35.9	0.01\\
35.91	0.01\\
35.92	0.01\\
35.93	0.01\\
35.94	0.01\\
35.95	0.01\\
35.96	0.01\\
35.97	0.01\\
35.98	0.01\\
35.99	0.01\\
36	0.01\\
36.01	0.01\\
36.02	0.01\\
36.03	0.01\\
36.04	0.01\\
36.05	0.01\\
36.06	0.01\\
36.07	0.01\\
36.08	0.01\\
36.09	0.01\\
36.1	0.01\\
36.11	0.01\\
36.12	0.01\\
36.13	0.01\\
36.14	0.01\\
36.15	0.01\\
36.16	0.01\\
36.17	0.01\\
36.18	0.01\\
36.19	0.01\\
36.2	0.01\\
36.21	0.01\\
36.22	0.01\\
36.23	0.01\\
36.24	0.01\\
36.25	0.01\\
36.26	0.01\\
36.27	0.01\\
36.28	0.01\\
36.29	0.01\\
36.3	0.01\\
36.31	0.01\\
36.32	0.01\\
36.33	0.01\\
36.34	0.01\\
36.35	0.01\\
36.36	0.01\\
36.37	0.01\\
36.38	0.01\\
36.39	0.01\\
36.4	0.01\\
36.41	0.01\\
36.42	0.01\\
36.43	0.01\\
36.44	0.01\\
36.45	0.01\\
36.46	0.01\\
36.47	0.01\\
36.48	0.01\\
36.49	0.01\\
36.5	0.01\\
36.51	0.01\\
36.52	0.01\\
36.53	0.01\\
36.54	0.01\\
36.55	0.01\\
36.56	0.01\\
36.57	0.01\\
36.58	0.01\\
36.59	0.01\\
36.6	0.01\\
36.61	0.01\\
36.62	0.01\\
36.63	0.01\\
36.64	0.01\\
36.65	0.01\\
36.66	0.01\\
36.67	0.01\\
36.68	0.01\\
36.69	0.01\\
36.7	0.01\\
36.71	0.01\\
36.72	0.01\\
36.73	0.01\\
36.74	0.01\\
36.75	0.01\\
36.76	0.01\\
36.77	0.01\\
36.78	0.01\\
36.79	0.01\\
36.8	0.01\\
36.81	0.01\\
36.82	0.01\\
36.83	0.01\\
36.84	0.01\\
36.85	0.01\\
36.86	0.01\\
36.87	0.01\\
36.88	0.01\\
36.89	0.01\\
36.9	0.01\\
36.91	0.01\\
36.92	0.01\\
36.93	0.01\\
36.94	0.01\\
36.95	0.01\\
36.96	0.01\\
36.97	0.01\\
36.98	0.01\\
36.99	0.01\\
37	0.01\\
37.01	0.01\\
37.02	0.01\\
37.03	0.01\\
37.04	0.01\\
37.05	0.01\\
37.06	0.01\\
37.07	0.01\\
37.08	0.01\\
37.09	0.01\\
37.1	0.01\\
37.11	0.01\\
37.12	0.01\\
37.13	0.01\\
37.14	0.01\\
37.15	0.01\\
37.16	0.01\\
37.17	0.01\\
37.18	0.01\\
37.19	0.01\\
37.2	0.01\\
37.21	0.01\\
37.22	0.01\\
37.23	0.01\\
37.24	0.01\\
37.25	0.01\\
37.26	0.01\\
37.27	0.01\\
37.28	0.01\\
37.29	0.01\\
37.3	0.01\\
37.31	0.01\\
37.32	0.01\\
37.33	0.01\\
37.34	0.01\\
37.35	0.01\\
37.36	0.01\\
37.37	0.01\\
37.38	0.01\\
37.39	0.01\\
37.4	0.01\\
37.41	0.01\\
37.42	0.01\\
37.43	0.01\\
37.44	0.01\\
37.45	0.01\\
37.46	0.01\\
37.47	0.01\\
37.48	0.01\\
37.49	0.01\\
37.5	0.01\\
37.51	0.01\\
37.52	0.01\\
37.53	0.01\\
37.54	0.01\\
37.55	0.01\\
37.56	0.01\\
37.57	0.01\\
37.58	0.01\\
37.59	0.01\\
37.6	0.01\\
37.61	0.01\\
37.62	0.01\\
37.63	0.01\\
37.64	0.01\\
37.65	0.01\\
37.66	0.01\\
37.67	0.01\\
37.68	0.01\\
37.69	0.01\\
37.7	0.01\\
37.71	0.01\\
37.72	0.01\\
37.73	0.01\\
37.74	0.01\\
37.75	0.01\\
37.76	0.01\\
37.77	0.01\\
37.78	0.01\\
37.79	0.01\\
37.8	0.01\\
37.81	0.01\\
37.82	0.01\\
37.83	0.01\\
37.84	0.01\\
37.85	0.01\\
37.86	0.01\\
37.87	0.01\\
37.88	0.01\\
37.89	0.01\\
37.9	0.01\\
37.91	0.01\\
37.92	0.01\\
37.93	0.01\\
37.94	0.01\\
37.95	0.01\\
37.96	0.01\\
37.97	0.01\\
37.98	0.01\\
37.99	0.01\\
38	0.01\\
38.01	0.01\\
38.02	0.01\\
38.03	0.01\\
38.04	0.01\\
38.05	0.01\\
38.06	0.01\\
38.07	0.01\\
38.08	0.01\\
38.09	0.01\\
38.1	0.01\\
38.11	0.01\\
38.12	0.01\\
38.13	0.01\\
38.14	0.01\\
38.15	0.01\\
38.16	0.01\\
38.17	0.01\\
38.18	0.01\\
38.19	0.01\\
38.2	0.01\\
38.21	0.01\\
38.22	0.01\\
38.23	0.01\\
38.24	0.01\\
38.25	0.01\\
38.26	0.01\\
38.27	0.01\\
38.28	0.01\\
38.29	0.01\\
38.3	0.01\\
38.31	0.01\\
38.32	0.01\\
38.33	0.01\\
38.34	0.01\\
38.35	0.01\\
38.36	0.01\\
38.37	0.01\\
38.38	0.01\\
38.39	0.01\\
38.4	0.01\\
38.41	0.01\\
38.42	0.01\\
38.43	0.01\\
38.44	0.01\\
38.45	0.01\\
38.46	0.01\\
38.47	0.01\\
38.48	0.01\\
38.49	0.01\\
38.5	0.01\\
38.51	0.01\\
38.52	0.01\\
38.53	0.01\\
38.54	0.01\\
38.55	0.01\\
38.56	0.01\\
38.57	0.01\\
38.58	0.01\\
38.59	0.01\\
38.6	0.01\\
38.61	0.01\\
38.62	0.01\\
38.63	0.01\\
38.64	0.01\\
38.65	0.01\\
38.66	0.01\\
38.67	0.01\\
38.68	0.01\\
38.69	0.01\\
38.7	0.01\\
38.71	0.01\\
38.72	0.01\\
38.73	0.01\\
38.74	0.01\\
38.75	0.01\\
38.76	0.01\\
38.77	0.01\\
38.78	0.01\\
38.79	0.01\\
38.8	0.01\\
38.81	0.01\\
38.82	0.01\\
38.83	0.01\\
38.84	0.01\\
38.85	0.01\\
38.86	0.01\\
38.87	0.01\\
38.88	0.01\\
38.89	0.01\\
38.9	0.01\\
38.91	0.01\\
38.92	0.01\\
38.93	0.01\\
38.94	0.01\\
38.95	0.01\\
38.96	0.01\\
38.97	0.01\\
38.98	0.01\\
38.99	0.01\\
39	0.01\\
39.01	0.01\\
39.02	0.01\\
39.03	0.01\\
39.04	0.01\\
39.05	0.01\\
39.06	0.01\\
39.07	0.01\\
39.08	0.01\\
39.09	0.01\\
39.1	0.01\\
39.11	0.01\\
39.12	0.01\\
39.13	0.01\\
39.14	0.01\\
39.15	0.01\\
39.16	0.01\\
39.17	0.01\\
39.18	0.01\\
39.19	0.01\\
39.2	0.01\\
39.21	0.01\\
39.22	0.01\\
39.23	0.01\\
39.24	0.01\\
39.25	0.01\\
39.26	0.01\\
39.27	0.01\\
39.28	0.01\\
39.29	0.01\\
39.3	0.01\\
39.31	0.01\\
39.32	0.01\\
39.33	0.01\\
39.34	0.01\\
39.35	0.01\\
39.36	0.01\\
39.37	0.01\\
39.38	0.01\\
39.39	0.01\\
39.4	0.01\\
39.41	0.01\\
39.42	0.01\\
39.43	0.01\\
39.44	0.01\\
39.45	0.01\\
39.46	0.01\\
39.47	0.01\\
39.48	0.01\\
39.49	0.01\\
39.5	0.01\\
39.51	0.01\\
39.52	0.01\\
39.53	0.01\\
39.54	0.01\\
39.55	0.01\\
39.56	0.01\\
39.57	0.01\\
39.58	0.01\\
39.59	0.01\\
39.6	0.01\\
39.61	0.01\\
39.62	0.01\\
39.63	0.01\\
39.64	0.01\\
39.65	0.01\\
39.66	0.01\\
39.67	0.01\\
39.68	0.01\\
39.69	0.01\\
39.7	0.01\\
39.71	0.01\\
39.72	0.01\\
39.73	0.01\\
39.74	0.01\\
39.75	0.01\\
39.76	0.01\\
39.77	0.01\\
39.78	0.01\\
39.79	0.01\\
39.8	0.01\\
39.81	0.01\\
39.82	0.01\\
39.83	0.01\\
39.84	0.01\\
39.85	0.01\\
39.86	0.01\\
39.87	0.01\\
39.88	0.01\\
39.89	0.01\\
39.9	0.01\\
39.91	0.01\\
39.92	0.01\\
39.93	0.01\\
39.94	0.01\\
39.95	0.01\\
39.96	0.01\\
39.97	0.01\\
39.98	0.01\\
39.99	0.01\\
40	0.01\\
40.01	0.01\\
};
\addplot [color=red,dashed,forget plot]
  table[row sep=crcr]{%
40.01	0.01\\
40.02	0.01\\
40.03	0.01\\
40.04	0.01\\
40.05	0.01\\
40.06	0.01\\
40.07	0.01\\
40.08	0.01\\
40.09	0.01\\
40.1	0.01\\
40.11	0.01\\
40.12	0.01\\
40.13	0.01\\
40.14	0.01\\
40.15	0.01\\
40.16	0.01\\
40.17	0.01\\
40.18	0.01\\
40.19	0.01\\
40.2	0.01\\
40.21	0.01\\
40.22	0.01\\
40.23	0.01\\
40.24	0.01\\
40.25	0.01\\
40.26	0.01\\
40.27	0.01\\
40.28	0.01\\
40.29	0.01\\
40.3	0.01\\
40.31	0.01\\
40.32	0.01\\
40.33	0.01\\
40.34	0.01\\
40.35	0.01\\
40.36	0.01\\
40.37	0.01\\
40.38	0.01\\
40.39	0.01\\
40.4	0.01\\
40.41	0.01\\
40.42	0.01\\
40.43	0.01\\
40.44	0.01\\
40.45	0.01\\
40.46	0.01\\
40.47	0.01\\
40.48	0.01\\
40.49	0.01\\
40.5	0.01\\
40.51	0.01\\
40.52	0.01\\
40.53	0.01\\
40.54	0.01\\
40.55	0.01\\
40.56	0.01\\
40.57	0.01\\
40.58	0.01\\
40.59	0.01\\
40.6	0.01\\
40.61	0.01\\
40.62	0.01\\
40.63	0.01\\
40.64	0.01\\
40.65	0.01\\
40.66	0.01\\
40.67	0.01\\
40.68	0.01\\
40.69	0.01\\
40.7	0.01\\
40.71	0.01\\
40.72	0.01\\
40.73	0.01\\
40.74	0.01\\
40.75	0.01\\
40.76	0.01\\
40.77	0.01\\
40.78	0.01\\
40.79	0.01\\
40.8	0.01\\
40.81	0.01\\
40.82	0.01\\
40.83	0.01\\
40.84	0.01\\
40.85	0.01\\
40.86	0.01\\
40.87	0.01\\
40.88	0.01\\
40.89	0.01\\
40.9	0.01\\
40.91	0.01\\
40.92	0.01\\
40.93	0.01\\
40.94	0.01\\
40.95	0.01\\
40.96	0.01\\
40.97	0.01\\
40.98	0.01\\
40.99	0.01\\
41	0.01\\
41.01	0.01\\
41.02	0.01\\
41.03	0.01\\
41.04	0.01\\
41.05	0.01\\
41.06	0.01\\
41.07	0.01\\
41.08	0.01\\
41.09	0.01\\
41.1	0.01\\
41.11	0.01\\
41.12	0.01\\
41.13	0.01\\
41.14	0.01\\
41.15	0.01\\
41.16	0.01\\
41.17	0.01\\
41.18	0.01\\
41.19	0.01\\
41.2	0.01\\
41.21	0.01\\
41.22	0.01\\
41.23	0.01\\
41.24	0.01\\
41.25	0.01\\
41.26	0.01\\
41.27	0.01\\
41.28	0.01\\
41.29	0.01\\
41.3	0.01\\
41.31	0.01\\
41.32	0.01\\
41.33	0.01\\
41.34	0.01\\
41.35	0.01\\
41.36	0.01\\
41.37	0.01\\
41.38	0.01\\
41.39	0.01\\
41.4	0.01\\
41.41	0.01\\
41.42	0.01\\
41.43	0.01\\
41.44	0.01\\
41.45	0.01\\
41.46	0.01\\
41.47	0.01\\
41.48	0.01\\
41.49	0.01\\
41.5	0.01\\
41.51	0.01\\
41.52	0.01\\
41.53	0.01\\
41.54	0.01\\
41.55	0.01\\
41.56	0.01\\
41.57	0.01\\
41.58	0.01\\
41.59	0.01\\
41.6	0.01\\
41.61	0.01\\
41.62	0.01\\
41.63	0.01\\
41.64	0.01\\
41.65	0.01\\
41.66	0.01\\
41.67	0.01\\
41.68	0.01\\
41.69	0.01\\
41.7	0.01\\
41.71	0.01\\
41.72	0.01\\
41.73	0.01\\
41.74	0.01\\
41.75	0.01\\
41.76	0.01\\
41.77	0.01\\
41.78	0.01\\
41.79	0.01\\
41.8	0.01\\
41.81	0.01\\
41.82	0.01\\
41.83	0.01\\
41.84	0.01\\
41.85	0.01\\
41.86	0.01\\
41.87	0.01\\
41.88	0.01\\
41.89	0.01\\
41.9	0.01\\
41.91	0.01\\
41.92	0.01\\
41.93	0.01\\
41.94	0.01\\
41.95	0.01\\
41.96	0.01\\
41.97	0.01\\
41.98	0.01\\
41.99	0.01\\
42	0.01\\
42.01	0.01\\
42.02	0.01\\
42.03	0.01\\
42.04	0.01\\
42.05	0.01\\
42.06	0.01\\
42.07	0.01\\
42.08	0.01\\
42.09	0.01\\
42.1	0.01\\
42.11	0.01\\
42.12	0.01\\
42.13	0.01\\
42.14	0.01\\
42.15	0.01\\
42.16	0.01\\
42.17	0.01\\
42.18	0.01\\
42.19	0.01\\
42.2	0.01\\
42.21	0.01\\
42.22	0.01\\
42.23	0.01\\
42.24	0.01\\
42.25	0.01\\
42.26	0.01\\
42.27	0.01\\
42.28	0.01\\
42.29	0.01\\
42.3	0.01\\
42.31	0.01\\
42.32	0.01\\
42.33	0.01\\
42.34	0.01\\
42.35	0.01\\
42.36	0.01\\
42.37	0.01\\
42.38	0.01\\
42.39	0.01\\
42.4	0.01\\
42.41	0.01\\
42.42	0.01\\
42.43	0.01\\
42.44	0.01\\
42.45	0.01\\
42.46	0.01\\
42.47	0.01\\
42.48	0.01\\
42.49	0.01\\
42.5	0.01\\
42.51	0.01\\
42.52	0.01\\
42.53	0.01\\
42.54	0.01\\
42.55	0.01\\
42.56	0.01\\
42.57	0.01\\
42.58	0.01\\
42.59	0.01\\
42.6	0.01\\
42.61	0.01\\
42.62	0.01\\
42.63	0.01\\
42.64	0.01\\
42.65	0.01\\
42.66	0.01\\
42.67	0.01\\
42.68	0.01\\
42.69	0.01\\
42.7	0.01\\
42.71	0.01\\
42.72	0.01\\
42.73	0.01\\
42.74	0.01\\
42.75	0.01\\
42.76	0.01\\
42.77	0.01\\
42.78	0.01\\
42.79	0.01\\
42.8	0.01\\
42.81	0.01\\
42.82	0.01\\
42.83	0.01\\
42.84	0.01\\
42.85	0.01\\
42.86	0.01\\
42.87	0.01\\
42.88	0.01\\
42.89	0.01\\
42.9	0.01\\
42.91	0.01\\
42.92	0.01\\
42.93	0.01\\
42.94	0.01\\
42.95	0.01\\
42.96	0.01\\
42.97	0.01\\
42.98	0.01\\
42.99	0.01\\
43	0.01\\
43.01	0.01\\
43.02	0.01\\
43.03	0.01\\
43.04	0.01\\
43.05	0.01\\
43.06	0.01\\
43.07	0.01\\
43.08	0.01\\
43.09	0.01\\
43.1	0.01\\
43.11	0.01\\
43.12	0.01\\
43.13	0.01\\
43.14	0.01\\
43.15	0.01\\
43.16	0.01\\
43.17	0.01\\
43.18	0.01\\
43.19	0.01\\
43.2	0.01\\
43.21	0.01\\
43.22	0.01\\
43.23	0.01\\
43.24	0.01\\
43.25	0.01\\
43.26	0.01\\
43.27	0.01\\
43.28	0.01\\
43.29	0.01\\
43.3	0.01\\
43.31	0.01\\
43.32	0.01\\
43.33	0.01\\
43.34	0.01\\
43.35	0.01\\
43.36	0.01\\
43.37	0.01\\
43.38	0.01\\
43.39	0.01\\
43.4	0.01\\
43.41	0.01\\
43.42	0.01\\
43.43	0.01\\
43.44	0.01\\
43.45	0.01\\
43.46	0.01\\
43.47	0.01\\
43.48	0.01\\
43.49	0.01\\
43.5	0.01\\
43.51	0.01\\
43.52	0.01\\
43.53	0.01\\
43.54	0.01\\
43.55	0.01\\
43.56	0.01\\
43.57	0.01\\
43.58	0.01\\
43.59	0.01\\
43.6	0.01\\
43.61	0.01\\
43.62	0.01\\
43.63	0.01\\
43.64	0.01\\
43.65	0.01\\
43.66	0.01\\
43.67	0.01\\
43.68	0.01\\
43.69	0.01\\
43.7	0.01\\
43.71	0.01\\
43.72	0.01\\
43.73	0.01\\
43.74	0.01\\
43.75	0.01\\
43.76	0.01\\
43.77	0.01\\
43.78	0.01\\
43.79	0.01\\
43.8	0.01\\
43.81	0.01\\
43.82	0.01\\
43.83	0.01\\
43.84	0.01\\
43.85	0.01\\
43.86	0.01\\
43.87	0.01\\
43.88	0.01\\
43.89	0.01\\
43.9	0.01\\
43.91	0.01\\
43.92	0.01\\
43.93	0.01\\
43.94	0.01\\
43.95	0.01\\
43.96	0.01\\
43.97	0.01\\
43.98	0.01\\
43.99	0.01\\
44	0.01\\
44.01	0.01\\
44.02	0.01\\
44.03	0.01\\
44.04	0.01\\
44.05	0.01\\
44.06	0.01\\
44.07	0.01\\
44.08	0.01\\
44.09	0.01\\
44.1	0.01\\
44.11	0.01\\
44.12	0.01\\
44.13	0.01\\
44.14	0.01\\
44.15	0.01\\
44.16	0.01\\
44.17	0.01\\
44.18	0.01\\
44.19	0.01\\
44.2	0.01\\
44.21	0.01\\
44.22	0.01\\
44.23	0.01\\
44.24	0.01\\
44.25	0.01\\
44.26	0.01\\
44.27	0.01\\
44.28	0.01\\
44.29	0.01\\
44.3	0.01\\
44.31	0.01\\
44.32	0.01\\
44.33	0.01\\
44.34	0.01\\
44.35	0.01\\
44.36	0.01\\
44.37	0.01\\
44.38	0.01\\
44.39	0.01\\
44.4	0.01\\
44.41	0.01\\
44.42	0.01\\
44.43	0.01\\
44.44	0.01\\
44.45	0.01\\
44.46	0.01\\
44.47	0.01\\
44.48	0.01\\
44.49	0.01\\
44.5	0.01\\
44.51	0.01\\
44.52	0.01\\
44.53	0.01\\
44.54	0.01\\
44.55	0.01\\
44.56	0.01\\
44.57	0.01\\
44.58	0.01\\
44.59	0.01\\
44.6	0.01\\
44.61	0.01\\
44.62	0.01\\
44.63	0.01\\
44.64	0.01\\
44.65	0.01\\
44.66	0.01\\
44.67	0.01\\
44.68	0.01\\
44.69	0.01\\
44.7	0.01\\
44.71	0.01\\
44.72	0.01\\
44.73	0.01\\
44.74	0.01\\
44.75	0.01\\
44.76	0.01\\
44.77	0.01\\
44.78	0.01\\
44.79	0.01\\
44.8	0.01\\
44.81	0.01\\
44.82	0.01\\
44.83	0.01\\
44.84	0.01\\
44.85	0.01\\
44.86	0.01\\
44.87	0.01\\
44.88	0.01\\
44.89	0.01\\
44.9	0.01\\
44.91	0.01\\
44.92	0.01\\
44.93	0.01\\
44.94	0.01\\
44.95	0.01\\
44.96	0.01\\
44.97	0.01\\
44.98	0.01\\
44.99	0.01\\
45	0.01\\
45.01	0.01\\
45.02	0.01\\
45.03	0.01\\
45.04	0.01\\
45.05	0.01\\
45.06	0.01\\
45.07	0.01\\
45.08	0.01\\
45.09	0.01\\
45.1	0.01\\
45.11	0.01\\
45.12	0.01\\
45.13	0.01\\
45.14	0.01\\
45.15	0.01\\
45.16	0.01\\
45.17	0.01\\
45.18	0.01\\
45.19	0.01\\
45.2	0.01\\
45.21	0.01\\
45.22	0.01\\
45.23	0.01\\
45.24	0.01\\
45.25	0.01\\
45.26	0.01\\
45.27	0.01\\
45.28	0.01\\
45.29	0.01\\
45.3	0.01\\
45.31	0.01\\
45.32	0.01\\
45.33	0.01\\
45.34	0.01\\
45.35	0.01\\
45.36	0.01\\
45.37	0.01\\
45.38	0.01\\
45.39	0.01\\
45.4	0.01\\
45.41	0.01\\
45.42	0.01\\
45.43	0.01\\
45.44	0.01\\
45.45	0.01\\
45.46	0.01\\
45.47	0.01\\
45.48	0.01\\
45.49	0.01\\
45.5	0.01\\
45.51	0.01\\
45.52	0.01\\
45.53	0.01\\
45.54	0.01\\
45.55	0.01\\
45.56	0.01\\
45.57	0.01\\
45.58	0.01\\
45.59	0.01\\
45.6	0.01\\
45.61	0.01\\
45.62	0.01\\
45.63	0.01\\
45.64	0.01\\
45.65	0.01\\
45.66	0.01\\
45.67	0.01\\
45.68	0.01\\
45.69	0.01\\
45.7	0.01\\
45.71	0.01\\
45.72	0.01\\
45.73	0.01\\
45.74	0.01\\
45.75	0.01\\
45.76	0.01\\
45.77	0.01\\
45.78	0.01\\
45.79	0.01\\
45.8	0.01\\
45.81	0.01\\
45.82	0.01\\
45.83	0.01\\
45.84	0.01\\
45.85	0.01\\
45.86	0.01\\
45.87	0.01\\
45.88	0.01\\
45.89	0.01\\
45.9	0.01\\
45.91	0.01\\
45.92	0.01\\
45.93	0.01\\
45.94	0.01\\
45.95	0.01\\
45.96	0.01\\
45.97	0.01\\
45.98	0.01\\
45.99	0.01\\
46	0.01\\
46.01	0.01\\
46.02	0.01\\
46.03	0.01\\
46.04	0.01\\
46.05	0.01\\
46.06	0.01\\
46.07	0.01\\
46.08	0.01\\
46.09	0.01\\
46.1	0.01\\
46.11	0.01\\
46.12	0.01\\
46.13	0.01\\
46.14	0.01\\
46.15	0.01\\
46.16	0.01\\
46.17	0.01\\
46.18	0.01\\
46.19	0.01\\
46.2	0.01\\
46.21	0.01\\
46.22	0.01\\
46.23	0.01\\
46.24	0.01\\
46.25	0.01\\
46.26	0.01\\
46.27	0.01\\
46.28	0.01\\
46.29	0.01\\
46.3	0.01\\
46.31	0.01\\
46.32	0.01\\
46.33	0.01\\
46.34	0.01\\
46.35	0.01\\
46.36	0.01\\
46.37	0.01\\
46.38	0.01\\
46.39	0.01\\
46.4	0.01\\
46.41	0.01\\
46.42	0.01\\
46.43	0.01\\
46.44	0.01\\
46.45	0.01\\
46.46	0.01\\
46.47	0.01\\
46.48	0.01\\
46.49	0.01\\
46.5	0.01\\
46.51	0.01\\
46.52	0.01\\
46.53	0.01\\
46.54	0.01\\
46.55	0.01\\
46.56	0.01\\
46.57	0.01\\
46.58	0.01\\
46.59	0.01\\
46.6	0.01\\
46.61	0.01\\
46.62	0.01\\
46.63	0.01\\
46.64	0.01\\
46.65	0.01\\
46.66	0.01\\
46.67	0.01\\
46.68	0.01\\
46.69	0.01\\
46.7	0.01\\
46.71	0.01\\
46.72	0.01\\
46.73	0.01\\
46.74	0.01\\
46.75	0.01\\
46.76	0.01\\
46.77	0.01\\
46.78	0.01\\
46.79	0.01\\
46.8	0.01\\
46.81	0.01\\
46.82	0.01\\
46.83	0.01\\
46.84	0.01\\
46.85	0.01\\
46.86	0.01\\
46.87	0.01\\
46.88	0.01\\
46.89	0.01\\
46.9	0.01\\
46.91	0.01\\
46.92	0.01\\
46.93	0.01\\
46.94	0.01\\
46.95	0.01\\
46.96	0.01\\
46.97	0.01\\
46.98	0.01\\
46.99	0.01\\
47	0.01\\
47.01	0.01\\
47.02	0.01\\
47.03	0.01\\
47.04	0.01\\
47.05	0.01\\
47.06	0.01\\
47.07	0.01\\
47.08	0.01\\
47.09	0.01\\
47.1	0.01\\
47.11	0.01\\
47.12	0.01\\
47.13	0.01\\
47.14	0.01\\
47.15	0.01\\
47.16	0.01\\
47.17	0.01\\
47.18	0.01\\
47.19	0.01\\
47.2	0.01\\
47.21	0.01\\
47.22	0.01\\
47.23	0.01\\
47.24	0.01\\
47.25	0.01\\
47.26	0.01\\
47.27	0.01\\
47.28	0.01\\
47.29	0.01\\
47.3	0.01\\
47.31	0.01\\
47.32	0.01\\
47.33	0.01\\
47.34	0.01\\
47.35	0.01\\
47.36	0.01\\
47.37	0.01\\
47.38	0.01\\
47.39	0.01\\
47.4	0.01\\
47.41	0.01\\
47.42	0.01\\
47.43	0.01\\
47.44	0.01\\
47.45	0.01\\
47.46	0.01\\
47.47	0.01\\
47.48	0.01\\
47.49	0.01\\
47.5	0.01\\
47.51	0.01\\
47.52	0.01\\
47.53	0.01\\
47.54	0.01\\
47.55	0.01\\
47.56	0.01\\
47.57	0.01\\
47.58	0.01\\
47.59	0.01\\
47.6	0.01\\
47.61	0.01\\
47.62	0.01\\
47.63	0.01\\
47.64	0.01\\
47.65	0.01\\
47.66	0.01\\
47.67	0.01\\
47.68	0.01\\
47.69	0.01\\
47.7	0.01\\
47.71	0.01\\
47.72	0.01\\
47.73	0.01\\
47.74	0.01\\
47.75	0.01\\
47.76	0.01\\
47.77	0.01\\
47.78	0.01\\
47.79	0.01\\
47.8	0.01\\
47.81	0.01\\
47.82	0.01\\
47.83	0.01\\
47.84	0.01\\
47.85	0.01\\
47.86	0.01\\
47.87	0.01\\
47.88	0.01\\
47.89	0.01\\
47.9	0.01\\
47.91	0.01\\
47.92	0.01\\
47.93	0.01\\
47.94	0.01\\
47.95	0.01\\
47.96	0.01\\
47.97	0.01\\
47.98	0.01\\
47.99	0.01\\
48	0.01\\
48.01	0.01\\
48.02	0.01\\
48.03	0.01\\
48.04	0.01\\
48.05	0.01\\
48.06	0.01\\
48.07	0.01\\
48.08	0.01\\
48.09	0.01\\
48.1	0.01\\
48.11	0.01\\
48.12	0.01\\
48.13	0.01\\
48.14	0.01\\
48.15	0.01\\
48.16	0.01\\
48.17	0.01\\
48.18	0.01\\
48.19	0.01\\
48.2	0.01\\
48.21	0.01\\
48.22	0.01\\
48.23	0.01\\
48.24	0.01\\
48.25	0.01\\
48.26	0.01\\
48.27	0.01\\
48.28	0.01\\
48.29	0.01\\
48.3	0.01\\
48.31	0.01\\
48.32	0.01\\
48.33	0.01\\
48.34	0.01\\
48.35	0.01\\
48.36	0.01\\
48.37	0.01\\
48.38	0.01\\
48.39	0.01\\
48.4	0.01\\
48.41	0.01\\
48.42	0.01\\
48.43	0.01\\
48.44	0.01\\
48.45	0.01\\
48.46	0.01\\
48.47	0.01\\
48.48	0.01\\
48.49	0.01\\
48.5	0.01\\
48.51	0.01\\
48.52	0.01\\
48.53	0.01\\
48.54	0.01\\
48.55	0.01\\
48.56	0.01\\
48.57	0.01\\
48.58	0.01\\
48.59	0.01\\
48.6	0.01\\
48.61	0.01\\
48.62	0.01\\
48.63	0.01\\
48.64	0.01\\
48.65	0.01\\
48.66	0.01\\
48.67	0.01\\
48.68	0.01\\
48.69	0.01\\
48.7	0.01\\
48.71	0.01\\
48.72	0.01\\
48.73	0.01\\
48.74	0.01\\
48.75	0.01\\
48.76	0.01\\
48.77	0.01\\
48.78	0.01\\
48.79	0.01\\
48.8	0.01\\
48.81	0.01\\
48.82	0.01\\
48.83	0.01\\
48.84	0.01\\
48.85	0.01\\
48.86	0.01\\
48.87	0.01\\
48.88	0.01\\
48.89	0.01\\
48.9	0.01\\
48.91	0.01\\
48.92	0.01\\
48.93	0.01\\
48.94	0.01\\
48.95	0.01\\
48.96	0.01\\
48.97	0.01\\
48.98	0.01\\
48.99	0.01\\
49	0.01\\
49.01	0.01\\
49.02	0.01\\
49.03	0.01\\
49.04	0.01\\
49.05	0.01\\
49.06	0.01\\
49.07	0.01\\
49.08	0.01\\
49.09	0.01\\
49.1	0.01\\
49.11	0.01\\
49.12	0.01\\
49.13	0.01\\
49.14	0.01\\
49.15	0.01\\
49.16	0.01\\
49.17	0.01\\
49.18	0.01\\
49.19	0.01\\
49.2	0.01\\
49.21	0.01\\
49.22	0.01\\
49.23	0.01\\
49.24	0.01\\
49.25	0.01\\
49.26	0.01\\
49.27	0.01\\
49.28	0.01\\
49.29	0.01\\
49.3	0.01\\
49.31	0.01\\
49.32	0.01\\
49.33	0.01\\
49.34	0.01\\
49.35	0.01\\
49.36	0.01\\
49.37	0.01\\
49.38	0.01\\
49.39	0.01\\
49.4	0.01\\
49.41	0.01\\
49.42	0.01\\
49.43	0.01\\
49.44	0.01\\
49.45	0.01\\
49.46	0.01\\
49.47	0.01\\
49.48	0.01\\
49.49	0.01\\
49.5	0.01\\
49.51	0.01\\
49.52	0.01\\
49.53	0.01\\
49.54	0.01\\
49.55	0.01\\
49.56	0.01\\
49.57	0.01\\
49.58	0.01\\
49.59	0.01\\
49.6	0.01\\
49.61	0.01\\
49.62	0.01\\
49.63	0.01\\
49.64	0.01\\
49.65	0.01\\
49.66	0.01\\
49.67	0.01\\
49.68	0.01\\
49.69	0.01\\
49.7	0.01\\
49.71	0.01\\
49.72	0.01\\
49.73	0.01\\
49.74	0.01\\
49.75	0.01\\
49.76	0.01\\
49.77	0.01\\
49.78	0.01\\
49.79	0.01\\
49.8	0.01\\
49.81	0.01\\
49.82	0.01\\
49.83	0.01\\
49.84	0.01\\
49.85	0.01\\
49.86	0.01\\
49.87	0.01\\
49.88	0.01\\
49.89	0.01\\
49.9	0.01\\
49.91	0.01\\
49.92	0.01\\
49.93	0.01\\
49.94	0.01\\
49.95	0.01\\
49.96	0.01\\
49.97	0.01\\
49.98	0.01\\
49.99	0.01\\
50	0.01\\
50.01	0.01\\
50.02	0.01\\
50.03	0.01\\
50.04	0.01\\
50.05	0.01\\
50.06	0.01\\
50.07	0.01\\
50.08	0.01\\
50.09	0.01\\
50.1	0.01\\
50.11	0.01\\
50.12	0.01\\
50.13	0.01\\
50.14	0.01\\
50.15	0.01\\
50.16	0.01\\
50.17	0.01\\
50.18	0.01\\
50.19	0.01\\
50.2	0.01\\
50.21	0.01\\
50.22	0.01\\
50.23	0.01\\
50.24	0.01\\
50.25	0.01\\
50.26	0.01\\
50.27	0.01\\
50.28	0.01\\
50.29	0.01\\
50.3	0.01\\
50.31	0.01\\
50.32	0.01\\
50.33	0.01\\
50.34	0.01\\
50.35	0.01\\
50.36	0.01\\
50.37	0.01\\
50.38	0.01\\
50.39	0.01\\
50.4	0.01\\
50.41	0.01\\
50.42	0.01\\
50.43	0.01\\
50.44	0.01\\
50.45	0.01\\
50.46	0.01\\
50.47	0.01\\
50.48	0.01\\
50.49	0.01\\
50.5	0.01\\
50.51	0.01\\
50.52	0.01\\
50.53	0.01\\
50.54	0.01\\
50.55	0.01\\
50.56	0.01\\
50.57	0.01\\
50.58	0.01\\
50.59	0.01\\
50.6	0.01\\
50.61	0.01\\
50.62	0.01\\
50.63	0.01\\
50.64	0.01\\
50.65	0.01\\
50.66	0.01\\
50.67	0.01\\
50.68	0.01\\
50.69	0.01\\
50.7	0.01\\
50.71	0.01\\
50.72	0.01\\
50.73	0.01\\
50.74	0.01\\
50.75	0.01\\
50.76	0.01\\
50.77	0.01\\
50.78	0.01\\
50.79	0.01\\
50.8	0.01\\
50.81	0.01\\
50.82	0.01\\
50.83	0.01\\
50.84	0.01\\
50.85	0.01\\
50.86	0.01\\
50.87	0.01\\
50.88	0.01\\
50.89	0.01\\
50.9	0.01\\
50.91	0.01\\
50.92	0.01\\
50.93	0.01\\
50.94	0.01\\
50.95	0.01\\
50.96	0.01\\
50.97	0.01\\
50.98	0.01\\
50.99	0.01\\
51	0.01\\
51.01	0.01\\
51.02	0.01\\
51.03	0.01\\
51.04	0.01\\
51.05	0.01\\
51.06	0.01\\
51.07	0.01\\
51.08	0.01\\
51.09	0.01\\
51.1	0.01\\
51.11	0.01\\
51.12	0.01\\
51.13	0.01\\
51.14	0.01\\
51.15	0.01\\
51.16	0.01\\
51.17	0.01\\
51.18	0.01\\
51.19	0.01\\
51.2	0.01\\
51.21	0.01\\
51.22	0.01\\
51.23	0.01\\
51.24	0.01\\
51.25	0.01\\
51.26	0.01\\
51.27	0.01\\
51.28	0.01\\
51.29	0.01\\
51.3	0.01\\
51.31	0.01\\
51.32	0.01\\
51.33	0.01\\
51.34	0.01\\
51.35	0.01\\
51.36	0.01\\
51.37	0.01\\
51.38	0.01\\
51.39	0.01\\
51.4	0.01\\
51.41	0.01\\
51.42	0.01\\
51.43	0.01\\
51.44	0.01\\
51.45	0.01\\
51.46	0.01\\
51.47	0.01\\
51.48	0.01\\
51.49	0.01\\
51.5	0.01\\
51.51	0.01\\
51.52	0.01\\
51.53	0.01\\
51.54	0.01\\
51.55	0.01\\
51.56	0.01\\
51.57	0.01\\
51.58	0.01\\
51.59	0.01\\
51.6	0.01\\
51.61	0.01\\
51.62	0.01\\
51.63	0.01\\
51.64	0.01\\
51.65	0.01\\
51.66	0.01\\
51.67	0.01\\
51.68	0.01\\
51.69	0.01\\
51.7	0.01\\
51.71	0.01\\
51.72	0.01\\
51.73	0.01\\
51.74	0.01\\
51.75	0.01\\
51.76	0.01\\
51.77	0.01\\
51.78	0.01\\
51.79	0.01\\
51.8	0.01\\
51.81	0.01\\
51.82	0.01\\
51.83	0.01\\
51.84	0.01\\
51.85	0.01\\
51.86	0.01\\
51.87	0.01\\
51.88	0.01\\
51.89	0.01\\
51.9	0.01\\
51.91	0.01\\
51.92	0.01\\
51.93	0.01\\
51.94	0.01\\
51.95	0.01\\
51.96	0.01\\
51.97	0.01\\
51.98	0.01\\
51.99	0.01\\
52	0.01\\
52.01	0.01\\
52.02	0.01\\
52.03	0.01\\
52.04	0.01\\
52.05	0.01\\
52.06	0.01\\
52.07	0.01\\
52.08	0.01\\
52.09	0.01\\
52.1	0.01\\
52.11	0.01\\
52.12	0.01\\
52.13	0.01\\
52.14	0.01\\
52.15	0.01\\
52.16	0.01\\
52.17	0.01\\
52.18	0.01\\
52.19	0.01\\
52.2	0.01\\
52.21	0.01\\
52.22	0.01\\
52.23	0.01\\
52.24	0.01\\
52.25	0.01\\
52.26	0.01\\
52.27	0.01\\
52.28	0.01\\
52.29	0.01\\
52.3	0.01\\
52.31	0.01\\
52.32	0.01\\
52.33	0.01\\
52.34	0.01\\
52.35	0.01\\
52.36	0.01\\
52.37	0.01\\
52.38	0.01\\
52.39	0.01\\
52.4	0.01\\
52.41	0.01\\
52.42	0.01\\
52.43	0.01\\
52.44	0.01\\
52.45	0.01\\
52.46	0.01\\
52.47	0.01\\
52.48	0.01\\
52.49	0.01\\
52.5	0.01\\
52.51	0.01\\
52.52	0.01\\
52.53	0.01\\
52.54	0.01\\
52.55	0.01\\
52.56	0.01\\
52.57	0.01\\
52.58	0.01\\
52.59	0.01\\
52.6	0.01\\
52.61	0.01\\
52.62	0.01\\
52.63	0.01\\
52.64	0.01\\
52.65	0.01\\
52.66	0.01\\
52.67	0.01\\
52.68	0.01\\
52.69	0.01\\
52.7	0.01\\
52.71	0.01\\
52.72	0.01\\
52.73	0.01\\
52.74	0.01\\
52.75	0.01\\
52.76	0.01\\
52.77	0.01\\
52.78	0.01\\
52.79	0.01\\
52.8	0.01\\
52.81	0.01\\
52.82	0.01\\
52.83	0.01\\
52.84	0.01\\
52.85	0.01\\
52.86	0.01\\
52.87	0.01\\
52.88	0.01\\
52.89	0.01\\
52.9	0.01\\
52.91	0.01\\
52.92	0.01\\
52.93	0.01\\
52.94	0.01\\
52.95	0.01\\
52.96	0.01\\
52.97	0.01\\
52.98	0.01\\
52.99	0.01\\
53	0.01\\
53.01	0.01\\
53.02	0.01\\
53.03	0.01\\
53.04	0.01\\
53.05	0.01\\
53.06	0.01\\
53.07	0.01\\
53.08	0.01\\
53.09	0.01\\
53.1	0.01\\
53.11	0.01\\
53.12	0.01\\
53.13	0.01\\
53.14	0.01\\
53.15	0.01\\
53.16	0.01\\
53.17	0.01\\
53.18	0.01\\
53.19	0.01\\
53.2	0.01\\
53.21	0.01\\
53.22	0.01\\
53.23	0.01\\
53.24	0.01\\
53.25	0.01\\
53.26	0.01\\
53.27	0.01\\
53.28	0.01\\
53.29	0.01\\
53.3	0.01\\
53.31	0.01\\
53.32	0.01\\
53.33	0.01\\
53.34	0.01\\
53.35	0.01\\
53.36	0.01\\
53.37	0.01\\
53.38	0.01\\
53.39	0.01\\
53.4	0.01\\
53.41	0.01\\
53.42	0.01\\
53.43	0.01\\
53.44	0.01\\
53.45	0.01\\
53.46	0.01\\
53.47	0.01\\
53.48	0.01\\
53.49	0.01\\
53.5	0.01\\
53.51	0.01\\
53.52	0.01\\
53.53	0.01\\
53.54	0.01\\
53.55	0.01\\
53.56	0.01\\
53.57	0.01\\
53.58	0.01\\
53.59	0.01\\
53.6	0.01\\
53.61	0.01\\
53.62	0.01\\
53.63	0.01\\
53.64	0.01\\
53.65	0.01\\
53.66	0.01\\
53.67	0.01\\
53.68	0.01\\
53.69	0.01\\
53.7	0.01\\
53.71	0.01\\
53.72	0.01\\
53.73	0.01\\
53.74	0.01\\
53.75	0.01\\
53.76	0.01\\
53.77	0.01\\
53.78	0.01\\
53.79	0.01\\
53.8	0.01\\
53.81	0.01\\
53.82	0.01\\
53.83	0.01\\
53.84	0.01\\
53.85	0.01\\
53.86	0.01\\
53.87	0.01\\
53.88	0.01\\
53.89	0.01\\
53.9	0.01\\
53.91	0.01\\
53.92	0.01\\
53.93	0.01\\
53.94	0.01\\
53.95	0.01\\
53.96	0.01\\
53.97	0.01\\
53.98	0.01\\
53.99	0.01\\
54	0.01\\
54.01	0.01\\
54.02	0.01\\
54.03	0.01\\
54.04	0.01\\
54.05	0.01\\
54.06	0.01\\
54.07	0.01\\
54.08	0.01\\
54.09	0.01\\
54.1	0.01\\
54.11	0.01\\
54.12	0.01\\
54.13	0.01\\
54.14	0.01\\
54.15	0.01\\
54.16	0.01\\
54.17	0.01\\
54.18	0.01\\
54.19	0.01\\
54.2	0.01\\
54.21	0.01\\
54.22	0.01\\
54.23	0.01\\
54.24	0.01\\
54.25	0.01\\
54.26	0.01\\
54.27	0.01\\
54.28	0.01\\
54.29	0.01\\
54.3	0.01\\
54.31	0.01\\
54.32	0.01\\
54.33	0.01\\
54.34	0.01\\
54.35	0.01\\
54.36	0.01\\
54.37	0.01\\
54.38	0.01\\
54.39	0.01\\
54.4	0.01\\
54.41	0.01\\
54.42	0.01\\
54.43	0.01\\
54.44	0.01\\
54.45	0.01\\
54.46	0.01\\
54.47	0.01\\
54.48	0.01\\
54.49	0.01\\
54.5	0.01\\
54.51	0.01\\
54.52	0.01\\
54.53	0.01\\
54.54	0.01\\
54.55	0.01\\
54.56	0.01\\
54.57	0.01\\
54.58	0.01\\
54.59	0.01\\
54.6	0.01\\
54.61	0.01\\
54.62	0.01\\
54.63	0.01\\
54.64	0.01\\
54.65	0.01\\
54.66	0.01\\
54.67	0.01\\
54.68	0.01\\
54.69	0.01\\
54.7	0.01\\
54.71	0.01\\
54.72	0.01\\
54.73	0.01\\
54.74	0.01\\
54.75	0.01\\
54.76	0.01\\
54.77	0.01\\
54.78	0.01\\
54.79	0.01\\
54.8	0.01\\
54.81	0.01\\
54.82	0.01\\
54.83	0.01\\
54.84	0.01\\
54.85	0.01\\
54.86	0.01\\
54.87	0.01\\
54.88	0.01\\
54.89	0.01\\
54.9	0.01\\
54.91	0.01\\
54.92	0.01\\
54.93	0.01\\
54.94	0.01\\
54.95	0.01\\
54.96	0.01\\
54.97	0.01\\
54.98	0.01\\
54.99	0.01\\
55	0.01\\
55.01	0.01\\
55.02	0.01\\
55.03	0.01\\
55.04	0.01\\
55.05	0.01\\
55.06	0.01\\
55.07	0.01\\
55.08	0.01\\
55.09	0.01\\
55.1	0.01\\
55.11	0.01\\
55.12	0.01\\
55.13	0.01\\
55.14	0.01\\
55.15	0.01\\
55.16	0.01\\
55.17	0.01\\
55.18	0.01\\
55.19	0.01\\
55.2	0.01\\
55.21	0.01\\
55.22	0.01\\
55.23	0.01\\
55.24	0.01\\
55.25	0.01\\
55.26	0.01\\
55.27	0.01\\
55.28	0.01\\
55.29	0.01\\
55.3	0.01\\
55.31	0.01\\
55.32	0.01\\
55.33	0.01\\
55.34	0.01\\
55.35	0.01\\
55.36	0.01\\
55.37	0.01\\
55.38	0.01\\
55.39	0.01\\
55.4	0.01\\
55.41	0.01\\
55.42	0.01\\
55.43	0.01\\
55.44	0.01\\
55.45	0.01\\
55.46	0.01\\
55.47	0.01\\
55.48	0.01\\
55.49	0.01\\
55.5	0.01\\
55.51	0.01\\
55.52	0.01\\
55.53	0.01\\
55.54	0.01\\
55.55	0.01\\
55.56	0.01\\
55.57	0.01\\
55.58	0.01\\
55.59	0.01\\
55.6	0.01\\
55.61	0.01\\
55.62	0.01\\
55.63	0.01\\
55.64	0.01\\
55.65	0.01\\
55.66	0.01\\
55.67	0.01\\
55.68	0.01\\
55.69	0.01\\
55.7	0.01\\
55.71	0.01\\
55.72	0.01\\
55.73	0.01\\
55.74	0.01\\
55.75	0.01\\
55.76	0.01\\
55.77	0.01\\
55.78	0.01\\
55.79	0.01\\
55.8	0.01\\
55.81	0.01\\
55.82	0.01\\
55.83	0.01\\
55.84	0.01\\
55.85	0.01\\
55.86	0.01\\
55.87	0.01\\
55.88	0.01\\
55.89	0.01\\
55.9	0.01\\
55.91	0.01\\
55.92	0.01\\
55.93	0.01\\
55.94	0.01\\
55.95	0.01\\
55.96	0.01\\
55.97	0.01\\
55.98	0.01\\
55.99	0.01\\
56	0.01\\
56.01	0.01\\
56.02	0.01\\
56.03	0.01\\
56.04	0.01\\
56.05	0.01\\
56.06	0.01\\
56.07	0.01\\
56.08	0.01\\
56.09	0.01\\
56.1	0.01\\
56.11	0.01\\
56.12	0.01\\
56.13	0.01\\
56.14	0.01\\
56.15	0.01\\
56.16	0.01\\
56.17	0.01\\
56.18	0.01\\
56.19	0.01\\
56.2	0.01\\
56.21	0.01\\
56.22	0.01\\
56.23	0.01\\
56.24	0.01\\
56.25	0.01\\
56.26	0.01\\
56.27	0.01\\
56.28	0.01\\
56.29	0.01\\
56.3	0.01\\
56.31	0.01\\
56.32	0.01\\
56.33	0.01\\
56.34	0.01\\
56.35	0.01\\
56.36	0.01\\
56.37	0.01\\
56.38	0.01\\
56.39	0.01\\
56.4	0.01\\
56.41	0.01\\
56.42	0.01\\
56.43	0.01\\
56.44	0.01\\
56.45	0.01\\
56.46	0.01\\
56.47	0.01\\
56.48	0.01\\
56.49	0.01\\
56.5	0.01\\
56.51	0.01\\
56.52	0.01\\
56.53	0.01\\
56.54	0.01\\
56.55	0.01\\
56.56	0.01\\
56.57	0.01\\
56.58	0.01\\
56.59	0.01\\
56.6	0.01\\
56.61	0.01\\
56.62	0.01\\
56.63	0.01\\
56.64	0.01\\
56.65	0.01\\
56.66	0.01\\
56.67	0.01\\
56.68	0.01\\
56.69	0.01\\
56.7	0.01\\
56.71	0.01\\
56.72	0.01\\
56.73	0.01\\
56.74	0.01\\
56.75	0.01\\
56.76	0.01\\
56.77	0.01\\
56.78	0.01\\
56.79	0.01\\
56.8	0.01\\
56.81	0.01\\
56.82	0.01\\
56.83	0.01\\
56.84	0.01\\
56.85	0.01\\
56.86	0.01\\
56.87	0.01\\
56.88	0.01\\
56.89	0.01\\
56.9	0.01\\
56.91	0.01\\
56.92	0.01\\
56.93	0.01\\
56.94	0.01\\
56.95	0.01\\
56.96	0.01\\
56.97	0.01\\
56.98	0.01\\
56.99	0.01\\
57	0.01\\
57.01	0.01\\
57.02	0.01\\
57.03	0.01\\
57.04	0.01\\
57.05	0.01\\
57.06	0.01\\
57.07	0.01\\
57.08	0.01\\
57.09	0.01\\
57.1	0.01\\
57.11	0.01\\
57.12	0.01\\
57.13	0.01\\
57.14	0.01\\
57.15	0.01\\
57.16	0.01\\
57.17	0.01\\
57.18	0.01\\
57.19	0.01\\
57.2	0.01\\
57.21	0.01\\
57.22	0.01\\
57.23	0.01\\
57.24	0.01\\
57.25	0.01\\
57.26	0.01\\
57.27	0.01\\
57.28	0.01\\
57.29	0.01\\
57.3	0.01\\
57.31	0.01\\
57.32	0.01\\
57.33	0.01\\
57.34	0.01\\
57.35	0.01\\
57.36	0.01\\
57.37	0.01\\
57.38	0.01\\
57.39	0.01\\
57.4	0.01\\
57.41	0.01\\
57.42	0.01\\
57.43	0.01\\
57.44	0.01\\
57.45	0.01\\
57.46	0.01\\
57.47	0.01\\
57.48	0.01\\
57.49	0.01\\
57.5	0.01\\
57.51	0.01\\
57.52	0.01\\
57.53	0.01\\
57.54	0.01\\
57.55	0.01\\
57.56	0.01\\
57.57	0.01\\
57.58	0.01\\
57.59	0.01\\
57.6	0.01\\
57.61	0.01\\
57.62	0.01\\
57.63	0.01\\
57.64	0.01\\
57.65	0.01\\
57.66	0.01\\
57.67	0.01\\
57.68	0.01\\
57.69	0.01\\
57.7	0.01\\
57.71	0.01\\
57.72	0.01\\
57.73	0.01\\
57.74	0.01\\
57.75	0.01\\
57.76	0.01\\
57.77	0.01\\
57.78	0.01\\
57.79	0.01\\
57.8	0.01\\
57.81	0.01\\
57.82	0.01\\
57.83	0.01\\
57.84	0.01\\
57.85	0.01\\
57.86	0.01\\
57.87	0.01\\
57.88	0.01\\
57.89	0.01\\
57.9	0.01\\
57.91	0.01\\
57.92	0.01\\
57.93	0.01\\
57.94	0.01\\
57.95	0.01\\
57.96	0.01\\
57.97	0.01\\
57.98	0.01\\
57.99	0.01\\
58	0.01\\
58.01	0.01\\
58.02	0.01\\
58.03	0.01\\
58.04	0.01\\
58.05	0.01\\
58.06	0.01\\
58.07	0.01\\
58.08	0.01\\
58.09	0.01\\
58.1	0.01\\
58.11	0.01\\
58.12	0.01\\
58.13	0.01\\
58.14	0.01\\
58.15	0.01\\
58.16	0.01\\
58.17	0.01\\
58.18	0.01\\
58.19	0.01\\
58.2	0.01\\
58.21	0.01\\
58.22	0.01\\
58.23	0.01\\
58.24	0.01\\
58.25	0.01\\
58.26	0.01\\
58.27	0.01\\
58.28	0.01\\
58.29	0.01\\
58.3	0.01\\
58.31	0.01\\
58.32	0.01\\
58.33	0.01\\
58.34	0.01\\
58.35	0.01\\
58.36	0.01\\
58.37	0.01\\
58.38	0.01\\
58.39	0.01\\
58.4	0.01\\
58.41	0.01\\
58.42	0.01\\
58.43	0.01\\
58.44	0.01\\
58.45	0.01\\
58.46	0.01\\
58.47	0.01\\
58.48	0.01\\
58.49	0.01\\
58.5	0.01\\
58.51	0.01\\
58.52	0.01\\
58.53	0.01\\
58.54	0.01\\
58.55	0.01\\
58.56	0.01\\
58.57	0.01\\
58.58	0.01\\
58.59	0.01\\
58.6	0.01\\
58.61	0.01\\
58.62	0.01\\
58.63	0.01\\
58.64	0.01\\
58.65	0.01\\
58.66	0.01\\
58.67	0.01\\
58.68	0.01\\
58.69	0.01\\
58.7	0.01\\
58.71	0.01\\
58.72	0.01\\
58.73	0.01\\
58.74	0.01\\
58.75	0.01\\
58.76	0.01\\
58.77	0.01\\
58.78	0.01\\
58.79	0.01\\
58.8	0.01\\
58.81	0.01\\
58.82	0.01\\
58.83	0.01\\
58.84	0.01\\
58.85	0.01\\
58.86	0.01\\
58.87	0.01\\
58.88	0.01\\
58.89	0.01\\
58.9	0.01\\
58.91	0.01\\
58.92	0.01\\
58.93	0.01\\
58.94	0.01\\
58.95	0.01\\
58.96	0.01\\
58.97	0.01\\
58.98	0.01\\
58.99	0.01\\
59	0.01\\
59.01	0.01\\
59.02	0.01\\
59.03	0.01\\
59.04	0.01\\
59.05	0.01\\
59.06	0.01\\
59.07	0.01\\
59.08	0.01\\
59.09	0.01\\
59.1	0.01\\
59.11	0.01\\
59.12	0.01\\
59.13	0.01\\
59.14	0.01\\
59.15	0.01\\
59.16	0.01\\
59.17	0.01\\
59.18	0.01\\
59.19	0.01\\
59.2	0.01\\
59.21	0.01\\
59.22	0.01\\
59.23	0.01\\
59.24	0.01\\
59.25	0.01\\
59.26	0.01\\
59.27	0.01\\
59.28	0.01\\
59.29	0.01\\
59.3	0.01\\
59.31	0.01\\
59.32	0.01\\
59.33	0.01\\
59.34	0.01\\
59.35	0.01\\
59.36	0.01\\
59.37	0.01\\
59.38	0.01\\
59.39	0.01\\
59.4	0.01\\
59.41	0.01\\
59.42	0.01\\
59.43	0.01\\
59.44	0.01\\
59.45	0.01\\
59.46	0.01\\
59.47	0.01\\
59.48	0.01\\
59.49	0.01\\
59.5	0.01\\
59.51	0.01\\
59.52	0.01\\
59.53	0.01\\
59.54	0.01\\
59.55	0.01\\
59.56	0.01\\
59.57	0.01\\
59.58	0.01\\
59.59	0.01\\
59.6	0.01\\
59.61	0.01\\
59.62	0.01\\
59.63	0.01\\
59.64	0.01\\
59.65	0.01\\
59.66	0.01\\
59.67	0.01\\
59.68	0.01\\
59.69	0.01\\
59.7	0.01\\
59.71	0.01\\
59.72	0.01\\
59.73	0.01\\
59.74	0.01\\
59.75	0.01\\
59.76	0.01\\
59.77	0.01\\
59.78	0.01\\
59.79	0.01\\
59.8	0.01\\
59.81	0.01\\
59.82	0.01\\
59.83	0.01\\
59.84	0.01\\
59.85	0.01\\
59.86	0.01\\
59.87	0.01\\
59.88	0.01\\
59.89	0.01\\
59.9	0.01\\
59.91	0.01\\
59.92	0.01\\
59.93	0.01\\
59.94	0.01\\
59.95	0.01\\
59.96	0.01\\
59.97	0.01\\
59.98	0.01\\
59.99	0.01\\
60	0.01\\
60.01	0.01\\
60.02	0.01\\
60.03	0.01\\
60.04	0.01\\
60.05	0.01\\
60.06	0.01\\
60.07	0.01\\
60.08	0.01\\
60.09	0.01\\
60.1	0.01\\
60.11	0.01\\
60.12	0.01\\
60.13	0.01\\
60.14	0.01\\
60.15	0.01\\
60.16	0.01\\
60.17	0.01\\
60.18	0.01\\
60.19	0.01\\
60.2	0.01\\
60.21	0.01\\
60.22	0.01\\
60.23	0.01\\
60.24	0.01\\
60.25	0.01\\
60.26	0.01\\
60.27	0.01\\
60.28	0.01\\
60.29	0.01\\
60.3	0.01\\
60.31	0.01\\
60.32	0.01\\
60.33	0.01\\
60.34	0.01\\
60.35	0.01\\
60.36	0.01\\
60.37	0.01\\
60.38	0.01\\
60.39	0.01\\
60.4	0.01\\
60.41	0.01\\
60.42	0.01\\
60.43	0.01\\
60.44	0.01\\
60.45	0.01\\
60.46	0.01\\
60.47	0.01\\
60.48	0.01\\
60.49	0.01\\
60.5	0.01\\
60.51	0.01\\
60.52	0.01\\
60.53	0.01\\
60.54	0.01\\
60.55	0.01\\
60.56	0.01\\
60.57	0.01\\
60.58	0.01\\
60.59	0.01\\
60.6	0.01\\
60.61	0.01\\
60.62	0.01\\
60.63	0.01\\
60.64	0.01\\
60.65	0.01\\
60.66	0.01\\
60.67	0.01\\
60.68	0.01\\
60.69	0.01\\
60.7	0.01\\
60.71	0.01\\
60.72	0.01\\
60.73	0.01\\
60.74	0.01\\
60.75	0.01\\
60.76	0.01\\
60.77	0.01\\
60.78	0.01\\
60.79	0.01\\
60.8	0.01\\
60.81	0.01\\
60.82	0.01\\
60.83	0.01\\
60.84	0.01\\
60.85	0.01\\
60.86	0.01\\
60.87	0.01\\
60.88	0.01\\
60.89	0.01\\
60.9	0.01\\
60.91	0.01\\
60.92	0.01\\
60.93	0.01\\
60.94	0.01\\
60.95	0.01\\
60.96	0.01\\
60.97	0.01\\
60.98	0.01\\
60.99	0.01\\
61	0.01\\
61.01	0.01\\
61.02	0.01\\
61.03	0.01\\
61.04	0.01\\
61.05	0.01\\
61.06	0.01\\
61.07	0.01\\
61.08	0.01\\
61.09	0.01\\
61.1	0.01\\
61.11	0.01\\
61.12	0.01\\
61.13	0.01\\
61.14	0.01\\
61.15	0.01\\
61.16	0.01\\
61.17	0.01\\
61.18	0.01\\
61.19	0.01\\
61.2	0.01\\
61.21	0.01\\
61.22	0.01\\
61.23	0.01\\
61.24	0.01\\
61.25	0.01\\
61.26	0.01\\
61.27	0.01\\
61.28	0.01\\
61.29	0.01\\
61.3	0.01\\
61.31	0.01\\
61.32	0.01\\
61.33	0.01\\
61.34	0.01\\
61.35	0.01\\
61.36	0.01\\
61.37	0.01\\
61.38	0.01\\
61.39	0.01\\
61.4	0.01\\
61.41	0.01\\
61.42	0.01\\
61.43	0.01\\
61.44	0.01\\
61.45	0.01\\
61.46	0.01\\
61.47	0.01\\
61.48	0.01\\
61.49	0.01\\
61.5	0.01\\
61.51	0.01\\
61.52	0.01\\
61.53	0.01\\
61.54	0.01\\
61.55	0.01\\
61.56	0.01\\
61.57	0.01\\
61.58	0.01\\
61.59	0.01\\
61.6	0.01\\
61.61	0.01\\
61.62	0.01\\
61.63	0.01\\
61.64	0.01\\
61.65	0.01\\
61.66	0.01\\
61.67	0.01\\
61.68	0.01\\
61.69	0.01\\
61.7	0.01\\
61.71	0.01\\
61.72	0.01\\
61.73	0.01\\
61.74	0.01\\
61.75	0.01\\
61.76	0.01\\
61.77	0.01\\
61.78	0.01\\
61.79	0.01\\
61.8	0.01\\
61.81	0.01\\
61.82	0.01\\
61.83	0.01\\
61.84	0.01\\
61.85	0.01\\
61.86	0.01\\
61.87	0.01\\
61.88	0.01\\
61.89	0.01\\
61.9	0.01\\
61.91	0.01\\
61.92	0.01\\
61.93	0.01\\
61.94	0.01\\
61.95	0.01\\
61.96	0.01\\
61.97	0.01\\
61.98	0.01\\
61.99	0.01\\
62	0.01\\
62.01	0.01\\
62.02	0.01\\
62.03	0.01\\
62.04	0.01\\
62.05	0.01\\
62.06	0.01\\
62.07	0.01\\
62.08	0.01\\
62.09	0.01\\
62.1	0.01\\
62.11	0.01\\
62.12	0.01\\
62.13	0.01\\
62.14	0.01\\
62.15	0.01\\
62.16	0.01\\
62.17	0.01\\
62.18	0.01\\
62.19	0.01\\
62.2	0.01\\
62.21	0.01\\
62.22	0.01\\
62.23	0.01\\
62.24	0.01\\
62.25	0.01\\
62.26	0.01\\
62.27	0.01\\
62.28	0.01\\
62.29	0.01\\
62.3	0.01\\
62.31	0.01\\
62.32	0.01\\
62.33	0.01\\
62.34	0.01\\
62.35	0.01\\
62.36	0.01\\
62.37	0.01\\
62.38	0.01\\
62.39	0.01\\
62.4	0.01\\
62.41	0.01\\
62.42	0.01\\
62.43	0.01\\
62.44	0.01\\
62.45	0.01\\
62.46	0.01\\
62.47	0.01\\
62.48	0.01\\
62.49	0.01\\
62.5	0.01\\
62.51	0.01\\
62.52	0.01\\
62.53	0.01\\
62.54	0.01\\
62.55	0.01\\
62.56	0.01\\
62.57	0.01\\
62.58	0.01\\
62.59	0.01\\
62.6	0.01\\
62.61	0.01\\
62.62	0.01\\
62.63	0.01\\
62.64	0.01\\
62.65	0.01\\
62.66	0.01\\
62.67	0.01\\
62.68	0.01\\
62.69	0.01\\
62.7	0.01\\
62.71	0.01\\
62.72	0.01\\
62.73	0.01\\
62.74	0.01\\
62.75	0.01\\
62.76	0.01\\
62.77	0.01\\
62.78	0.01\\
62.79	0.01\\
62.8	0.01\\
62.81	0.01\\
62.82	0.01\\
62.83	0.01\\
62.84	0.01\\
62.85	0.01\\
62.86	0.01\\
62.87	0.01\\
62.88	0.01\\
62.89	0.01\\
62.9	0.01\\
62.91	0.01\\
62.92	0.01\\
62.93	0.01\\
62.94	0.01\\
62.95	0.01\\
62.96	0.01\\
62.97	0.01\\
62.98	0.01\\
62.99	0.01\\
63	0.01\\
63.01	0.01\\
63.02	0.01\\
63.03	0.01\\
63.04	0.01\\
63.05	0.01\\
63.06	0.01\\
63.07	0.01\\
63.08	0.01\\
63.09	0.01\\
63.1	0.01\\
63.11	0.01\\
63.12	0.01\\
63.13	0.01\\
63.14	0.01\\
63.15	0.01\\
63.16	0.01\\
63.17	0.01\\
63.18	0.01\\
63.19	0.01\\
63.2	0.01\\
63.21	0.01\\
63.22	0.01\\
63.23	0.01\\
63.24	0.01\\
63.25	0.01\\
63.26	0.01\\
63.27	0.01\\
63.28	0.01\\
63.29	0.01\\
63.3	0.01\\
63.31	0.01\\
63.32	0.01\\
63.33	0.01\\
63.34	0.01\\
63.35	0.01\\
63.36	0.01\\
63.37	0.01\\
63.38	0.01\\
63.39	0.01\\
63.4	0.01\\
63.41	0.01\\
63.42	0.01\\
63.43	0.01\\
63.44	0.01\\
63.45	0.01\\
63.46	0.01\\
63.47	0.01\\
63.48	0.01\\
63.49	0.01\\
63.5	0.01\\
63.51	0.01\\
63.52	0.01\\
63.53	0.01\\
63.54	0.01\\
63.55	0.01\\
63.56	0.01\\
63.57	0.01\\
63.58	0.01\\
63.59	0.01\\
63.6	0.01\\
63.61	0.01\\
63.62	0.01\\
63.63	0.01\\
63.64	0.01\\
63.65	0.01\\
63.66	0.01\\
63.67	0.01\\
63.68	0.01\\
63.69	0.01\\
63.7	0.01\\
63.71	0.01\\
63.72	0.01\\
63.73	0.01\\
63.74	0.01\\
63.75	0.01\\
63.76	0.01\\
63.77	0.01\\
63.78	0.01\\
63.79	0.01\\
63.8	0.01\\
63.81	0.01\\
63.82	0.01\\
63.83	0.01\\
63.84	0.01\\
63.85	0.01\\
63.86	0.01\\
63.87	0.01\\
63.88	0.01\\
63.89	0.01\\
63.9	0.01\\
63.91	0.01\\
63.92	0.01\\
63.93	0.01\\
63.94	0.01\\
63.95	0.01\\
63.96	0.01\\
63.97	0.01\\
63.98	0.01\\
63.99	0.01\\
64	0.01\\
64.01	0.01\\
64.02	0.01\\
64.03	0.01\\
64.04	0.01\\
64.05	0.01\\
64.06	0.01\\
64.07	0.01\\
64.08	0.01\\
64.09	0.01\\
64.1	0.01\\
64.11	0.01\\
64.12	0.01\\
64.13	0.01\\
64.14	0.01\\
64.15	0.01\\
64.16	0.01\\
64.17	0.01\\
64.18	0.01\\
64.19	0.01\\
64.2	0.01\\
64.21	0.01\\
64.22	0.01\\
64.23	0.01\\
64.24	0.01\\
64.25	0.01\\
64.26	0.01\\
64.27	0.01\\
64.28	0.01\\
64.29	0.01\\
64.3	0.01\\
64.31	0.01\\
64.32	0.01\\
64.33	0.01\\
64.34	0.01\\
64.35	0.01\\
64.36	0.01\\
64.37	0.01\\
64.38	0.01\\
64.39	0.01\\
64.4	0.01\\
64.41	0.01\\
64.42	0.01\\
64.43	0.01\\
64.44	0.01\\
64.45	0.01\\
64.46	0.01\\
64.47	0.01\\
64.48	0.01\\
64.49	0.01\\
64.5	0.01\\
64.51	0.01\\
64.52	0.01\\
64.53	0.01\\
64.54	0.01\\
64.55	0.01\\
64.56	0.01\\
64.57	0.01\\
64.58	0.01\\
64.59	0.01\\
64.6	0.01\\
64.61	0.01\\
64.62	0.01\\
64.63	0.01\\
64.64	0.01\\
64.65	0.01\\
64.66	0.01\\
64.67	0.01\\
64.68	0.01\\
64.69	0.01\\
64.7	0.01\\
64.71	0.01\\
64.72	0.01\\
64.73	0.01\\
64.74	0.01\\
64.75	0.01\\
64.76	0.01\\
64.77	0.01\\
64.78	0.01\\
64.79	0.01\\
64.8	0.01\\
64.81	0.01\\
64.82	0.01\\
64.83	0.01\\
64.84	0.01\\
64.85	0.01\\
64.86	0.01\\
64.87	0.01\\
64.88	0.01\\
64.89	0.01\\
64.9	0.01\\
64.91	0.01\\
64.92	0.01\\
64.93	0.01\\
64.94	0.01\\
64.95	0.01\\
64.96	0.01\\
64.97	0.01\\
64.98	0.01\\
64.99	0.01\\
65	0.01\\
65.01	0.01\\
65.02	0.01\\
65.03	0.01\\
65.04	0.01\\
65.05	0.01\\
65.06	0.01\\
65.07	0.01\\
65.08	0.01\\
65.09	0.01\\
65.1	0.01\\
65.11	0.01\\
65.12	0.01\\
65.13	0.01\\
65.14	0.01\\
65.15	0.01\\
65.16	0.01\\
65.17	0.01\\
65.18	0.01\\
65.19	0.01\\
65.2	0.01\\
65.21	0.01\\
65.22	0.01\\
65.23	0.01\\
65.24	0.01\\
65.25	0.01\\
65.26	0.01\\
65.27	0.01\\
65.28	0.01\\
65.29	0.01\\
65.3	0.01\\
65.31	0.01\\
65.32	0.01\\
65.33	0.01\\
65.34	0.01\\
65.35	0.01\\
65.36	0.01\\
65.37	0.01\\
65.38	0.01\\
65.39	0.01\\
65.4	0.01\\
65.41	0.01\\
65.42	0.01\\
65.43	0.01\\
65.44	0.01\\
65.45	0.01\\
65.46	0.01\\
65.47	0.01\\
65.48	0.01\\
65.49	0.01\\
65.5	0.01\\
65.51	0.01\\
65.52	0.01\\
65.53	0.01\\
65.54	0.01\\
65.55	0.01\\
65.56	0.01\\
65.57	0.01\\
65.58	0.01\\
65.59	0.01\\
65.6	0.01\\
65.61	0.01\\
65.62	0.01\\
65.63	0.01\\
65.64	0.01\\
65.65	0.01\\
65.66	0.01\\
65.67	0.01\\
65.68	0.01\\
65.69	0.01\\
65.7	0.01\\
65.71	0.01\\
65.72	0.01\\
65.73	0.01\\
65.74	0.01\\
65.75	0.01\\
65.76	0.01\\
65.77	0.01\\
65.78	0.01\\
65.79	0.01\\
65.8	0.01\\
65.81	0.01\\
65.82	0.01\\
65.83	0.01\\
65.84	0.01\\
65.85	0.01\\
65.86	0.01\\
65.87	0.01\\
65.88	0.01\\
65.89	0.01\\
65.9	0.01\\
65.91	0.01\\
65.92	0.01\\
65.93	0.01\\
65.94	0.01\\
65.95	0.01\\
65.96	0.01\\
65.97	0.01\\
65.98	0.01\\
65.99	0.01\\
66	0.01\\
66.01	0.01\\
66.02	0.01\\
66.03	0.01\\
66.04	0.01\\
66.05	0.01\\
66.06	0.01\\
66.07	0.01\\
66.08	0.01\\
66.09	0.01\\
66.1	0.01\\
66.11	0.01\\
66.12	0.01\\
66.13	0.01\\
66.14	0.01\\
66.15	0.01\\
66.16	0.01\\
66.17	0.01\\
66.18	0.01\\
66.19	0.01\\
66.2	0.01\\
66.21	0.01\\
66.22	0.01\\
66.23	0.01\\
66.24	0.01\\
66.25	0.01\\
66.26	0.01\\
66.27	0.01\\
66.28	0.01\\
66.29	0.01\\
66.3	0.01\\
66.31	0.01\\
66.32	0.01\\
66.33	0.01\\
66.34	0.01\\
66.35	0.01\\
66.36	0.01\\
66.37	0.01\\
66.38	0.01\\
66.39	0.01\\
66.4	0.01\\
66.41	0.01\\
66.42	0.01\\
66.43	0.01\\
66.44	0.01\\
66.45	0.01\\
66.46	0.01\\
66.47	0.01\\
66.48	0.01\\
66.49	0.01\\
66.5	0.01\\
66.51	0.01\\
66.52	0.01\\
66.53	0.01\\
66.54	0.01\\
66.55	0.01\\
66.56	0.01\\
66.57	0.01\\
66.58	0.01\\
66.59	0.01\\
66.6	0.01\\
66.61	0.01\\
66.62	0.01\\
66.63	0.01\\
66.64	0.01\\
66.65	0.01\\
66.66	0.01\\
66.67	0.01\\
66.68	0.01\\
66.69	0.01\\
66.7	0.01\\
66.71	0.01\\
66.72	0.01\\
66.73	0.01\\
66.74	0.01\\
66.75	0.01\\
66.76	0.01\\
66.77	0.01\\
66.78	0.01\\
66.79	0.01\\
66.8	0.01\\
66.81	0.01\\
66.82	0.01\\
66.83	0.01\\
66.84	0.01\\
66.85	0.01\\
66.86	0.01\\
66.87	0.01\\
66.88	0.01\\
66.89	0.01\\
66.9	0.01\\
66.91	0.01\\
66.92	0.01\\
66.93	0.01\\
66.94	0.01\\
66.95	0.01\\
66.96	0.01\\
66.97	0.01\\
66.98	0.01\\
66.99	0.01\\
67	0.01\\
67.01	0.01\\
67.02	0.01\\
67.03	0.01\\
67.04	0.01\\
67.05	0.01\\
67.06	0.01\\
67.07	0.01\\
67.08	0.01\\
67.09	0.01\\
67.1	0.01\\
67.11	0.01\\
67.12	0.01\\
67.13	0.01\\
67.14	0.01\\
67.15	0.01\\
67.16	0.01\\
67.17	0.01\\
67.18	0.01\\
67.19	0.01\\
67.2	0.01\\
67.21	0.01\\
67.22	0.01\\
67.23	0.01\\
67.24	0.01\\
67.25	0.01\\
67.26	0.01\\
67.27	0.01\\
67.28	0.01\\
67.29	0.01\\
67.3	0.01\\
67.31	0.01\\
67.32	0.01\\
67.33	0.01\\
67.34	0.01\\
67.35	0.01\\
67.36	0.01\\
67.37	0.01\\
67.38	0.01\\
67.39	0.01\\
67.4	0.01\\
67.41	0.01\\
67.42	0.01\\
67.43	0.01\\
67.44	0.01\\
67.45	0.01\\
67.46	0.01\\
67.47	0.01\\
67.48	0.01\\
67.49	0.01\\
67.5	0.01\\
67.51	0.01\\
67.52	0.01\\
67.53	0.01\\
67.54	0.01\\
67.55	0.01\\
67.56	0.01\\
67.57	0.01\\
67.58	0.01\\
67.59	0.01\\
67.6	0.01\\
67.61	0.01\\
67.62	0.01\\
67.63	0.01\\
67.64	0.01\\
67.65	0.01\\
67.66	0.01\\
67.67	0.01\\
67.68	0.01\\
67.69	0.01\\
67.7	0.01\\
67.71	0.01\\
67.72	0.01\\
67.73	0.01\\
67.74	0.01\\
67.75	0.01\\
67.76	0.01\\
67.77	0.01\\
67.78	0.01\\
67.79	0.01\\
67.8	0.01\\
67.81	0.01\\
67.82	0.01\\
67.83	0.01\\
67.84	0.01\\
67.85	0.01\\
67.86	0.01\\
67.87	0.01\\
67.88	0.01\\
67.89	0.01\\
67.9	0.01\\
67.91	0.01\\
67.92	0.01\\
67.93	0.01\\
67.94	0.01\\
67.95	0.01\\
67.96	0.01\\
67.97	0.01\\
67.98	0.01\\
67.99	0.01\\
68	0.01\\
68.01	0.01\\
68.02	0.01\\
68.03	0.01\\
68.04	0.01\\
68.05	0.01\\
68.06	0.01\\
68.07	0.01\\
68.08	0.01\\
68.09	0.01\\
68.1	0.01\\
68.11	0.01\\
68.12	0.01\\
68.13	0.01\\
68.14	0.01\\
68.15	0.01\\
68.16	0.01\\
68.17	0.01\\
68.18	0.01\\
68.19	0.01\\
68.2	0.01\\
68.21	0.01\\
68.22	0.01\\
68.23	0.01\\
68.24	0.01\\
68.25	0.01\\
68.26	0.01\\
68.27	0.01\\
68.28	0.01\\
68.29	0.01\\
68.3	0.01\\
68.31	0.01\\
68.32	0.01\\
68.33	0.01\\
68.34	0.01\\
68.35	0.01\\
68.36	0.01\\
68.37	0.01\\
68.38	0.01\\
68.39	0.01\\
68.4	0.01\\
68.41	0.01\\
68.42	0.01\\
68.43	0.01\\
68.44	0.01\\
68.45	0.01\\
68.46	0.01\\
68.47	0.01\\
68.48	0.01\\
68.49	0.01\\
68.5	0.01\\
68.51	0.01\\
68.52	0.01\\
68.53	0.01\\
68.54	0.01\\
68.55	0.01\\
68.56	0.01\\
68.57	0.01\\
68.58	0.01\\
68.59	0.01\\
68.6	0.01\\
68.61	0.01\\
68.62	0.01\\
68.63	0.01\\
68.64	0.01\\
68.65	0.01\\
68.66	0.01\\
68.67	0.01\\
68.68	0.01\\
68.69	0.01\\
68.7	0.01\\
68.71	0.01\\
68.72	0.01\\
68.73	0.01\\
68.74	0.01\\
68.75	0.01\\
68.76	0.01\\
68.77	0.01\\
68.78	0.01\\
68.79	0.01\\
68.8	0.01\\
68.81	0.01\\
68.82	0.01\\
68.83	0.01\\
68.84	0.01\\
68.85	0.01\\
68.86	0.01\\
68.87	0.01\\
68.88	0.01\\
68.89	0.01\\
68.9	0.01\\
68.91	0.01\\
68.92	0.01\\
68.93	0.01\\
68.94	0.01\\
68.95	0.01\\
68.96	0.01\\
68.97	0.01\\
68.98	0.01\\
68.99	0.01\\
69	0.01\\
69.01	0.01\\
69.02	0.01\\
69.03	0.01\\
69.04	0.01\\
69.05	0.01\\
69.06	0.01\\
69.07	0.01\\
69.08	0.01\\
69.09	0.01\\
69.1	0.01\\
69.11	0.01\\
69.12	0.01\\
69.13	0.01\\
69.14	0.01\\
69.15	0.01\\
69.16	0.01\\
69.17	0.01\\
69.18	0.01\\
69.19	0.01\\
69.2	0.01\\
69.21	0.01\\
69.22	0.01\\
69.23	0.01\\
69.24	0.01\\
69.25	0.01\\
69.26	0.01\\
69.27	0.01\\
69.28	0.01\\
69.29	0.01\\
69.3	0.01\\
69.31	0.01\\
69.32	0.01\\
69.33	0.01\\
69.34	0.01\\
69.35	0.01\\
69.36	0.01\\
69.37	0.01\\
69.38	0.01\\
69.39	0.01\\
69.4	0.01\\
69.41	0.01\\
69.42	0.01\\
69.43	0.01\\
69.44	0.01\\
69.45	0.01\\
69.46	0.01\\
69.47	0.01\\
69.48	0.01\\
69.49	0.01\\
69.5	0.01\\
69.51	0.01\\
69.52	0.01\\
69.53	0.01\\
69.54	0.01\\
69.55	0.01\\
69.56	0.01\\
69.57	0.01\\
69.58	0.01\\
69.59	0.01\\
69.6	0.01\\
69.61	0.01\\
69.62	0.01\\
69.63	0.01\\
69.64	0.01\\
69.65	0.01\\
69.66	0.01\\
69.67	0.01\\
69.68	0.01\\
69.69	0.01\\
69.7	0.01\\
69.71	0.01\\
69.72	0.01\\
69.73	0.01\\
69.74	0.01\\
69.75	0.01\\
69.76	0.01\\
69.77	0.01\\
69.78	0.01\\
69.79	0.01\\
69.8	0.01\\
69.81	0.01\\
69.82	0.01\\
69.83	0.01\\
69.84	0.01\\
69.85	0.01\\
69.86	0.01\\
69.87	0.01\\
69.88	0.01\\
69.89	0.01\\
69.9	0.01\\
69.91	0.01\\
69.92	0.01\\
69.93	0.01\\
69.94	0.01\\
69.95	0.01\\
69.96	0.01\\
69.97	0.01\\
69.98	0.01\\
69.99	0.01\\
70	0.01\\
70.01	0.01\\
70.02	0.01\\
70.03	0.01\\
70.04	0.01\\
70.05	0.01\\
70.06	0.01\\
70.07	0.01\\
70.08	0.01\\
70.09	0.01\\
70.1	0.01\\
70.11	0.01\\
70.12	0.01\\
70.13	0.01\\
70.14	0.01\\
70.15	0.01\\
70.16	0.01\\
70.17	0.01\\
70.18	0.01\\
70.19	0.01\\
70.2	0.01\\
70.21	0.01\\
70.22	0.01\\
70.23	0.01\\
70.24	0.01\\
70.25	0.01\\
70.26	0.01\\
70.27	0.01\\
70.28	0.01\\
70.29	0.01\\
70.3	0.01\\
70.31	0.01\\
70.32	0.01\\
70.33	0.01\\
70.34	0.01\\
70.35	0.01\\
70.36	0.01\\
70.37	0.01\\
70.38	0.01\\
70.39	0.01\\
70.4	0.01\\
70.41	0.01\\
70.42	0.01\\
70.43	0.01\\
70.44	0.01\\
70.45	0.01\\
70.46	0.01\\
70.47	0.01\\
70.48	0.01\\
70.49	0.01\\
70.5	0.01\\
70.51	0.01\\
70.52	0.01\\
70.53	0.01\\
70.54	0.01\\
70.55	0.01\\
70.56	0.01\\
70.57	0.01\\
70.58	0.01\\
70.59	0.01\\
70.6	0.01\\
70.61	0.01\\
70.62	0.01\\
70.63	0.01\\
70.64	0.01\\
70.65	0.01\\
70.66	0.01\\
70.67	0.01\\
70.68	0.01\\
70.69	0.01\\
70.7	0.01\\
70.71	0.01\\
70.72	0.01\\
70.73	0.01\\
70.74	0.01\\
70.75	0.01\\
70.76	0.01\\
70.77	0.01\\
70.78	0.01\\
70.79	0.01\\
70.8	0.01\\
70.81	0.01\\
70.82	0.01\\
70.83	0.01\\
70.84	0.01\\
70.85	0.01\\
70.86	0.01\\
70.87	0.01\\
70.88	0.01\\
70.89	0.01\\
70.9	0.01\\
70.91	0.01\\
70.92	0.01\\
70.93	0.01\\
70.94	0.01\\
70.95	0.01\\
70.96	0.01\\
70.97	0.01\\
70.98	0.01\\
70.99	0.01\\
71	0.01\\
71.01	0.01\\
71.02	0.01\\
71.03	0.01\\
71.04	0.01\\
71.05	0.01\\
71.06	0.01\\
71.07	0.01\\
71.08	0.01\\
71.09	0.01\\
71.1	0.01\\
71.11	0.01\\
71.12	0.01\\
71.13	0.01\\
71.14	0.01\\
71.15	0.01\\
71.16	0.01\\
71.17	0.01\\
71.18	0.01\\
71.19	0.01\\
71.2	0.01\\
71.21	0.01\\
71.22	0.01\\
71.23	0.01\\
71.24	0.01\\
71.25	0.01\\
71.26	0.01\\
71.27	0.01\\
71.28	0.01\\
71.29	0.01\\
71.3	0.01\\
71.31	0.01\\
71.32	0.01\\
71.33	0.01\\
71.34	0.01\\
71.35	0.01\\
71.36	0.01\\
71.37	0.01\\
71.38	0.01\\
71.39	0.01\\
71.4	0.01\\
71.41	0.01\\
71.42	0.01\\
71.43	0.01\\
71.44	0.01\\
71.45	0.01\\
71.46	0.01\\
71.47	0.01\\
71.48	0.01\\
71.49	0.01\\
71.5	0.01\\
71.51	0.01\\
71.52	0.01\\
71.53	0.01\\
71.54	0.01\\
71.55	0.01\\
71.56	0.01\\
71.57	0.01\\
71.58	0.01\\
71.59	0.01\\
71.6	0.01\\
71.61	0.01\\
71.62	0.01\\
71.63	0.01\\
71.64	0.01\\
71.65	0.01\\
71.66	0.01\\
71.67	0.01\\
71.68	0.01\\
71.69	0.01\\
71.7	0.01\\
71.71	0.01\\
71.72	0.01\\
71.73	0.01\\
71.74	0.01\\
71.75	0.01\\
71.76	0.01\\
71.77	0.01\\
71.78	0.01\\
71.79	0.01\\
71.8	0.01\\
71.81	0.01\\
71.82	0.01\\
71.83	0.01\\
71.84	0.01\\
71.85	0.01\\
71.86	0.01\\
71.87	0.01\\
71.88	0.01\\
71.89	0.01\\
71.9	0.01\\
71.91	0.01\\
71.92	0.01\\
71.93	0.01\\
71.94	0.01\\
71.95	0.01\\
71.96	0.01\\
71.97	0.01\\
71.98	0.01\\
71.99	0.01\\
72	0.01\\
72.01	0.01\\
72.02	0.01\\
72.03	0.01\\
72.04	0.01\\
72.05	0.01\\
72.06	0.01\\
72.07	0.01\\
72.08	0.01\\
72.09	0.01\\
72.1	0.01\\
72.11	0.01\\
72.12	0.01\\
72.13	0.01\\
72.14	0.01\\
72.15	0.01\\
72.16	0.01\\
72.17	0.01\\
72.18	0.01\\
72.19	0.01\\
72.2	0.01\\
72.21	0.01\\
72.22	0.01\\
72.23	0.01\\
72.24	0.01\\
72.25	0.01\\
72.26	0.01\\
72.27	0.01\\
72.28	0.01\\
72.29	0.01\\
72.3	0.01\\
72.31	0.01\\
72.32	0.01\\
72.33	0.01\\
72.34	0.01\\
72.35	0.01\\
72.36	0.01\\
72.37	0.01\\
72.38	0.01\\
72.39	0.01\\
72.4	0.01\\
72.41	0.01\\
72.42	0.01\\
72.43	0.01\\
72.44	0.01\\
72.45	0.01\\
72.46	0.01\\
72.47	0.01\\
72.48	0.01\\
72.49	0.01\\
72.5	0.01\\
72.51	0.01\\
72.52	0.01\\
72.53	0.01\\
72.54	0.01\\
72.55	0.01\\
72.56	0.01\\
72.57	0.01\\
72.58	0.01\\
72.59	0.01\\
72.6	0.01\\
72.61	0.01\\
72.62	0.01\\
72.63	0.01\\
72.64	0.01\\
72.65	0.01\\
72.66	0.01\\
72.67	0.01\\
72.68	0.01\\
72.69	0.01\\
72.7	0.01\\
72.71	0.01\\
72.72	0.01\\
72.73	0.01\\
72.74	0.01\\
72.75	0.01\\
72.76	0.01\\
72.77	0.01\\
72.78	0.01\\
72.79	0.01\\
72.8	0.01\\
72.81	0.01\\
72.82	0.01\\
72.83	0.01\\
72.84	0.01\\
72.85	0.01\\
72.86	0.01\\
72.87	0.01\\
72.88	0.01\\
72.89	0.01\\
72.9	0.01\\
72.91	0.01\\
72.92	0.01\\
72.93	0.01\\
72.94	0.01\\
72.95	0.01\\
72.96	0.01\\
72.97	0.01\\
72.98	0.01\\
72.99	0.01\\
73	0.01\\
73.01	0.01\\
73.02	0.01\\
73.03	0.01\\
73.04	0.01\\
73.05	0.01\\
73.06	0.01\\
73.07	0.01\\
73.08	0.01\\
73.09	0.01\\
73.1	0.01\\
73.11	0.01\\
73.12	0.01\\
73.13	0.01\\
73.14	0.01\\
73.15	0.01\\
73.16	0.01\\
73.17	0.01\\
73.18	0.01\\
73.19	0.01\\
73.2	0.01\\
73.21	0.01\\
73.22	0.01\\
73.23	0.01\\
73.24	0.01\\
73.25	0.01\\
73.26	0.01\\
73.27	0.01\\
73.28	0.01\\
73.29	0.01\\
73.3	0.01\\
73.31	0.01\\
73.32	0.01\\
73.33	0.01\\
73.34	0.01\\
73.35	0.01\\
73.36	0.01\\
73.37	0.01\\
73.38	0.01\\
73.39	0.01\\
73.4	0.01\\
73.41	0.01\\
73.42	0.01\\
73.43	0.01\\
73.44	0.01\\
73.45	0.01\\
73.46	0.01\\
73.47	0.01\\
73.48	0.01\\
73.49	0.01\\
73.5	0.01\\
73.51	0.01\\
73.52	0.01\\
73.53	0.01\\
73.54	0.01\\
73.55	0.01\\
73.56	0.01\\
73.57	0.01\\
73.58	0.01\\
73.59	0.01\\
73.6	0.01\\
73.61	0.01\\
73.62	0.01\\
73.63	0.01\\
73.64	0.01\\
73.65	0.01\\
73.66	0.01\\
73.67	0.01\\
73.68	0.01\\
73.69	0.01\\
73.7	0.01\\
73.71	0.01\\
73.72	0.01\\
73.73	0.01\\
73.74	0.01\\
73.75	0.01\\
73.76	0.01\\
73.77	0.01\\
73.78	0.01\\
73.79	0.01\\
73.8	0.01\\
73.81	0.01\\
73.82	0.01\\
73.83	0.01\\
73.84	0.01\\
73.85	0.01\\
73.86	0.01\\
73.87	0.01\\
73.88	0.01\\
73.89	0.01\\
73.9	0.01\\
73.91	0.01\\
73.92	0.01\\
73.93	0.01\\
73.94	0.01\\
73.95	0.01\\
73.96	0.01\\
73.97	0.01\\
73.98	0.01\\
73.99	0.01\\
74	0.01\\
74.01	0.01\\
74.02	0.01\\
74.03	0.01\\
74.04	0.01\\
74.05	0.01\\
74.06	0.01\\
74.07	0.01\\
74.08	0.01\\
74.09	0.01\\
74.1	0.01\\
74.11	0.01\\
74.12	0.01\\
74.13	0.01\\
74.14	0.01\\
74.15	0.01\\
74.16	0.01\\
74.17	0.01\\
74.18	0.01\\
74.19	0.01\\
74.2	0.01\\
74.21	0.01\\
74.22	0.01\\
74.23	0.01\\
74.24	0.01\\
74.25	0.01\\
74.26	0.01\\
74.27	0.01\\
74.28	0.01\\
74.29	0.01\\
74.3	0.01\\
74.31	0.01\\
74.32	0.01\\
74.33	0.01\\
74.34	0.01\\
74.35	0.01\\
74.36	0.01\\
74.37	0.01\\
74.38	0.01\\
74.39	0.01\\
74.4	0.01\\
74.41	0.01\\
74.42	0.01\\
74.43	0.01\\
74.44	0.01\\
74.45	0.01\\
74.46	0.01\\
74.47	0.01\\
74.48	0.01\\
74.49	0.01\\
74.5	0.01\\
74.51	0.01\\
74.52	0.01\\
74.53	0.01\\
74.54	0.01\\
74.55	0.01\\
74.56	0.01\\
74.57	0.01\\
74.58	0.01\\
74.59	0.01\\
74.6	0.01\\
74.61	0.01\\
74.62	0.01\\
74.63	0.01\\
74.64	0.01\\
74.65	0.01\\
74.66	0.01\\
74.67	0.01\\
74.68	0.01\\
74.69	0.01\\
74.7	0.01\\
74.71	0.01\\
74.72	0.01\\
74.73	0.01\\
74.74	0.01\\
74.75	0.01\\
74.76	0.01\\
74.77	0.01\\
74.78	0.01\\
74.79	0.01\\
74.8	0.01\\
74.81	0.01\\
74.82	0.01\\
74.83	0.01\\
74.84	0.01\\
74.85	0.01\\
74.86	0.01\\
74.87	0.01\\
74.88	0.01\\
74.89	0.01\\
74.9	0.01\\
74.91	0.01\\
74.92	0.01\\
74.93	0.01\\
74.94	0.01\\
74.95	0.01\\
74.96	0.01\\
74.97	0.01\\
74.98	0.01\\
74.99	0.01\\
75	0.01\\
75.01	0.01\\
75.02	0.01\\
75.03	0.01\\
75.04	0.01\\
75.05	0.01\\
75.06	0.01\\
75.07	0.01\\
75.08	0.01\\
75.09	0.01\\
75.1	0.01\\
75.11	0.01\\
75.12	0.01\\
75.13	0.01\\
75.14	0.01\\
75.15	0.01\\
75.16	0.01\\
75.17	0.01\\
75.18	0.01\\
75.19	0.01\\
75.2	0.01\\
75.21	0.01\\
75.22	0.01\\
75.23	0.01\\
75.24	0.01\\
75.25	0.01\\
75.26	0.01\\
75.27	0.01\\
75.28	0.01\\
75.29	0.01\\
75.3	0.01\\
75.31	0.01\\
75.32	0.01\\
75.33	0.01\\
75.34	0.01\\
75.35	0.01\\
75.36	0.01\\
75.37	0.01\\
75.38	0.01\\
75.39	0.01\\
75.4	0.01\\
75.41	0.01\\
75.42	0.01\\
75.43	0.01\\
75.44	0.01\\
75.45	0.01\\
75.46	0.01\\
75.47	0.01\\
75.48	0.01\\
75.49	0.01\\
75.5	0.01\\
75.51	0.01\\
75.52	0.01\\
75.53	0.01\\
75.54	0.01\\
75.55	0.01\\
75.56	0.01\\
75.57	0.01\\
75.58	0.01\\
75.59	0.01\\
75.6	0.01\\
75.61	0.01\\
75.62	0.01\\
75.63	0.01\\
75.64	0.01\\
75.65	0.01\\
75.66	0.01\\
75.67	0.01\\
75.68	0.01\\
75.69	0.01\\
75.7	0.01\\
75.71	0.01\\
75.72	0.01\\
75.73	0.01\\
75.74	0.01\\
75.75	0.01\\
75.76	0.01\\
75.77	0.01\\
75.78	0.01\\
75.79	0.01\\
75.8	0.01\\
75.81	0.01\\
75.82	0.01\\
75.83	0.01\\
75.84	0.01\\
75.85	0.01\\
75.86	0.01\\
75.87	0.01\\
75.88	0.01\\
75.89	0.01\\
75.9	0.01\\
75.91	0.01\\
75.92	0.01\\
75.93	0.01\\
75.94	0.01\\
75.95	0.01\\
75.96	0.01\\
75.97	0.01\\
75.98	0.01\\
75.99	0.01\\
76	0.01\\
76.01	0.01\\
76.02	0.01\\
76.03	0.01\\
76.04	0.01\\
76.05	0.01\\
76.06	0.01\\
76.07	0.01\\
76.08	0.01\\
76.09	0.01\\
76.1	0.01\\
76.11	0.01\\
76.12	0.01\\
76.13	0.01\\
76.14	0.01\\
76.15	0.01\\
76.16	0.01\\
76.17	0.01\\
76.18	0.01\\
76.19	0.01\\
76.2	0.01\\
76.21	0.01\\
76.22	0.01\\
76.23	0.01\\
76.24	0.01\\
76.25	0.01\\
76.26	0.01\\
76.27	0.01\\
76.28	0.01\\
76.29	0.01\\
76.3	0.01\\
76.31	0.01\\
76.32	0.01\\
76.33	0.01\\
76.34	0.01\\
76.35	0.01\\
76.36	0.01\\
76.37	0.01\\
76.38	0.01\\
76.39	0.01\\
76.4	0.01\\
76.41	0.01\\
76.42	0.01\\
76.43	0.01\\
76.44	0.01\\
76.45	0.01\\
76.46	0.01\\
76.47	0.01\\
76.48	0.01\\
76.49	0.01\\
76.5	0.01\\
76.51	0.01\\
76.52	0.01\\
76.53	0.01\\
76.54	0.01\\
76.55	0.01\\
76.56	0.01\\
76.57	0.01\\
76.58	0.01\\
76.59	0.01\\
76.6	0.01\\
76.61	0.01\\
76.62	0.01\\
76.63	0.01\\
76.64	0.01\\
76.65	0.01\\
76.66	0.01\\
76.67	0.01\\
76.68	0.01\\
76.69	0.01\\
76.7	0.01\\
76.71	0.01\\
76.72	0.01\\
76.73	0.01\\
76.74	0.01\\
76.75	0.01\\
76.76	0.01\\
76.77	0.01\\
76.78	0.01\\
76.79	0.01\\
76.8	0.01\\
76.81	0.01\\
76.82	0.01\\
76.83	0.01\\
76.84	0.01\\
76.85	0.01\\
76.86	0.01\\
76.87	0.01\\
76.88	0.01\\
76.89	0.01\\
76.9	0.01\\
76.91	0.01\\
76.92	0.01\\
76.93	0.01\\
76.94	0.01\\
76.95	0.01\\
76.96	0.01\\
76.97	0.01\\
76.98	0.01\\
76.99	0.01\\
77	0.01\\
77.01	0.01\\
77.02	0.01\\
77.03	0.01\\
77.04	0.01\\
77.05	0.01\\
77.06	0.01\\
77.07	0.01\\
77.08	0.01\\
77.09	0.01\\
77.1	0.01\\
77.11	0.01\\
77.12	0.01\\
77.13	0.01\\
77.14	0.01\\
77.15	0.01\\
77.16	0.01\\
77.17	0.01\\
77.18	0.01\\
77.19	0.01\\
77.2	0.01\\
77.21	0.01\\
77.22	0.01\\
77.23	0.01\\
77.24	0.01\\
77.25	0.01\\
77.26	0.01\\
77.27	0.01\\
77.28	0.01\\
77.29	0.01\\
77.3	0.01\\
77.31	0.01\\
77.32	0.01\\
77.33	0.01\\
77.34	0.01\\
77.35	0.01\\
77.36	0.01\\
77.37	0.01\\
77.38	0.01\\
77.39	0.01\\
77.4	0.01\\
77.41	0.01\\
77.42	0.01\\
77.43	0.01\\
77.44	0.01\\
77.45	0.01\\
77.46	0.01\\
77.47	0.01\\
77.48	0.01\\
77.49	0.01\\
77.5	0.01\\
77.51	0.01\\
77.52	0.01\\
77.53	0.01\\
77.54	0.01\\
77.55	0.01\\
77.56	0.01\\
77.57	0.01\\
77.58	0.01\\
77.59	0.01\\
77.6	0.01\\
77.61	0.01\\
77.62	0.01\\
77.63	0.01\\
77.64	0.01\\
77.65	0.01\\
77.66	0.01\\
77.67	0.01\\
77.68	0.01\\
77.69	0.01\\
77.7	0.01\\
77.71	0.01\\
77.72	0.01\\
77.73	0.01\\
77.74	0.01\\
77.75	0.01\\
77.76	0.01\\
77.77	0.01\\
77.78	0.01\\
77.79	0.01\\
77.8	0.01\\
77.81	0.01\\
77.82	0.01\\
77.83	0.01\\
77.84	0.01\\
77.85	0.01\\
77.86	0.01\\
77.87	0.01\\
77.88	0.01\\
77.89	0.01\\
77.9	0.01\\
77.91	0.01\\
77.92	0.01\\
77.93	0.01\\
77.94	0.01\\
77.95	0.01\\
77.96	0.01\\
77.97	0.01\\
77.98	0.01\\
77.99	0.01\\
78	0.01\\
78.01	0.01\\
78.02	0.01\\
78.03	0.01\\
78.04	0.01\\
78.05	0.01\\
78.06	0.01\\
78.07	0.01\\
78.08	0.01\\
78.09	0.01\\
78.1	0.01\\
78.11	0.01\\
78.12	0.01\\
78.13	0.01\\
78.14	0.01\\
78.15	0.01\\
78.16	0.01\\
78.17	0.01\\
78.18	0.01\\
78.19	0.01\\
78.2	0.01\\
78.21	0.01\\
78.22	0.01\\
78.23	0.01\\
78.24	0.01\\
78.25	0.01\\
78.26	0.01\\
78.27	0.01\\
78.28	0.01\\
78.29	0.01\\
78.3	0.01\\
78.31	0.01\\
78.32	0.01\\
78.33	0.01\\
78.34	0.01\\
78.35	0.01\\
78.36	0.01\\
78.37	0.01\\
78.38	0.01\\
78.39	0.01\\
78.4	0.01\\
78.41	0.01\\
78.42	0.01\\
78.43	0.01\\
78.44	0.01\\
78.45	0.01\\
78.46	0.01\\
78.47	0.01\\
78.48	0.01\\
78.49	0.01\\
78.5	0.01\\
78.51	0.01\\
78.52	0.01\\
78.53	0.01\\
78.54	0.01\\
78.55	0.01\\
78.56	0.01\\
78.57	0.01\\
78.58	0.01\\
78.59	0.01\\
78.6	0.01\\
78.61	0.01\\
78.62	0.01\\
78.63	0.01\\
78.64	0.01\\
78.65	0.01\\
78.66	0.01\\
78.67	0.01\\
78.68	0.01\\
78.69	0.01\\
78.7	0.01\\
78.71	0.01\\
78.72	0.01\\
78.73	0.01\\
78.74	0.01\\
78.75	0.01\\
78.76	0.01\\
78.77	0.01\\
78.78	0.01\\
78.79	0.01\\
78.8	0.01\\
78.81	0.01\\
78.82	0.01\\
78.83	0.01\\
78.84	0.01\\
78.85	0.01\\
78.86	0.01\\
78.87	0.01\\
78.88	0.01\\
78.89	0.01\\
78.9	0.01\\
78.91	0.01\\
78.92	0.01\\
78.93	0.01\\
78.94	0.01\\
78.95	0.01\\
78.96	0.01\\
78.97	0.01\\
78.98	0.01\\
78.99	0.01\\
79	0.01\\
79.01	0.01\\
79.02	0.01\\
79.03	0.01\\
79.04	0.01\\
79.05	0.01\\
79.06	0.01\\
79.07	0.01\\
79.08	0.01\\
79.09	0.01\\
79.1	0.01\\
79.11	0.01\\
79.12	0.01\\
79.13	0.01\\
79.14	0.01\\
79.15	0.01\\
79.16	0.01\\
79.17	0.01\\
79.18	0.01\\
79.19	0.01\\
79.2	0.01\\
79.21	0.01\\
79.22	0.01\\
79.23	0.01\\
79.24	0.01\\
79.25	0.01\\
79.26	0.01\\
79.27	0.01\\
79.28	0.01\\
79.29	0.01\\
79.3	0.01\\
79.31	0.01\\
79.32	0.01\\
79.33	0.01\\
79.34	0.01\\
79.35	0.01\\
79.36	0.01\\
79.37	0.01\\
79.38	0.01\\
79.39	0.01\\
79.4	0.01\\
79.41	0.01\\
79.42	0.01\\
79.43	0.01\\
79.44	0.01\\
79.45	0.01\\
79.46	0.01\\
79.47	0.01\\
79.48	0.01\\
79.49	0.01\\
79.5	0.01\\
79.51	0.01\\
79.52	0.01\\
79.53	0.01\\
79.54	0.01\\
79.55	0.01\\
79.56	0.01\\
79.57	0.01\\
79.58	0.01\\
79.59	0.01\\
79.6	0.01\\
79.61	0.01\\
79.62	0.01\\
79.63	0.01\\
79.64	0.01\\
79.65	0.01\\
79.66	0.01\\
79.67	0.01\\
79.68	0.01\\
79.69	0.01\\
79.7	0.01\\
79.71	0.01\\
79.72	0.01\\
79.73	0.01\\
79.74	0.01\\
79.75	0.01\\
79.76	0.01\\
79.77	0.01\\
79.78	0.01\\
79.79	0.01\\
79.8	0.01\\
79.81	0.01\\
79.82	0.01\\
79.83	0.01\\
79.84	0.01\\
79.85	0.01\\
79.86	0.01\\
79.87	0.01\\
79.88	0.01\\
79.89	0.01\\
79.9	0.01\\
79.91	0.01\\
79.92	0.01\\
79.93	0.01\\
79.94	0.01\\
79.95	0.01\\
79.96	0.01\\
79.97	0.01\\
79.98	0.01\\
79.99	0.01\\
80	0.01\\
80.01	0.01\\
};
\addplot [color=red,dashed]
  table[row sep=crcr]{%
80.01	0.01\\
80.02	0.01\\
80.03	0.01\\
80.04	0.01\\
80.05	0.01\\
80.06	0.01\\
80.07	0.01\\
80.08	0.01\\
80.09	0.01\\
80.1	0.01\\
80.11	0.01\\
80.12	0.01\\
80.13	0.01\\
80.14	0.01\\
80.15	0.01\\
80.16	0.01\\
80.17	0.01\\
80.18	0.01\\
80.19	0.01\\
80.2	0.01\\
80.21	0.01\\
80.22	0.01\\
80.23	0.01\\
80.24	0.01\\
80.25	0.01\\
80.26	0.01\\
80.27	0.01\\
80.28	0.01\\
80.29	0.01\\
80.3	0.01\\
80.31	0.01\\
80.32	0.01\\
80.33	0.01\\
80.34	0.01\\
80.35	0.01\\
80.36	0.01\\
80.37	0.01\\
80.38	0.01\\
80.39	0.01\\
80.4	0.01\\
80.41	0.01\\
80.42	0.01\\
80.43	0.01\\
80.44	0.01\\
80.45	0.01\\
80.46	0.01\\
80.47	0.01\\
80.48	0.01\\
80.49	0.01\\
80.5	0.01\\
80.51	0.01\\
80.52	0.01\\
80.53	0.01\\
80.54	0.01\\
80.55	0.01\\
80.56	0.01\\
80.57	0.01\\
80.58	0.01\\
80.59	0.01\\
80.6	0.01\\
80.61	0.01\\
80.62	0.01\\
80.63	0.01\\
80.64	0.01\\
80.65	0.01\\
80.66	0.01\\
80.67	0.01\\
80.68	0.01\\
80.69	0.01\\
80.7	0.01\\
80.71	0.01\\
80.72	0.01\\
80.73	0.01\\
80.74	0.01\\
80.75	0.01\\
80.76	0.01\\
80.77	0.01\\
80.78	0.01\\
80.79	0.01\\
80.8	0.01\\
80.81	0.01\\
80.82	0.01\\
80.83	0.01\\
80.84	0.01\\
80.85	0.01\\
80.86	0.01\\
80.87	0.01\\
80.88	0.01\\
80.89	0.01\\
80.9	0.01\\
80.91	0.01\\
80.92	0.01\\
80.93	0.01\\
80.94	0.01\\
80.95	0.01\\
80.96	0.01\\
80.97	0.01\\
80.98	0.01\\
80.99	0.01\\
81	0.01\\
81.01	0.01\\
81.02	0.01\\
81.03	0.01\\
81.04	0.01\\
81.05	0.01\\
81.06	0.01\\
81.07	0.01\\
81.08	0.01\\
81.09	0.01\\
81.1	0.01\\
81.11	0.01\\
81.12	0.01\\
81.13	0.01\\
81.14	0.01\\
81.15	0.01\\
81.16	0.01\\
81.17	0.01\\
81.18	0.01\\
81.19	0.01\\
81.2	0.01\\
81.21	0.01\\
81.22	0.01\\
81.23	0.01\\
81.24	0.01\\
81.25	0.01\\
81.26	0.01\\
81.27	0.01\\
81.28	0.01\\
81.29	0.01\\
81.3	0.01\\
81.31	0.01\\
81.32	0.01\\
81.33	0.01\\
81.34	0.01\\
81.35	0.01\\
81.36	0.01\\
81.37	0.01\\
81.38	0.01\\
81.39	0.01\\
81.4	0.01\\
81.41	0.01\\
81.42	0.01\\
81.43	0.01\\
81.44	0.01\\
81.45	0.01\\
81.46	0.01\\
81.47	0.01\\
81.48	0.01\\
81.49	0.01\\
81.5	0.01\\
81.51	0.01\\
81.52	0.01\\
81.53	0.01\\
81.54	0.01\\
81.55	0.01\\
81.56	0.01\\
81.57	0.01\\
81.58	0.01\\
81.59	0.01\\
81.6	0.01\\
81.61	0.01\\
81.62	0.01\\
81.63	0.01\\
81.64	0.01\\
81.65	0.01\\
81.66	0.01\\
81.67	0.01\\
81.68	0.01\\
81.69	0.01\\
81.7	0.01\\
81.71	0.01\\
81.72	0.01\\
81.73	0.01\\
81.74	0.01\\
81.75	0.01\\
81.76	0.01\\
81.77	0.01\\
81.78	0.01\\
81.79	0.01\\
81.8	0.01\\
81.81	0.01\\
81.82	0.01\\
81.83	0.01\\
81.84	0.01\\
81.85	0.01\\
81.86	0.01\\
81.87	0.01\\
81.88	0.01\\
81.89	0.01\\
81.9	0.01\\
81.91	0.01\\
81.92	0.01\\
81.93	0.01\\
81.94	0.01\\
81.95	0.01\\
81.96	0.01\\
81.97	0.01\\
81.98	0.01\\
81.99	0.01\\
82	0.01\\
82.01	0.01\\
82.02	0.01\\
82.03	0.01\\
82.04	0.01\\
82.05	0.01\\
82.06	0.01\\
82.07	0.01\\
82.08	0.01\\
82.09	0.01\\
82.1	0.01\\
82.11	0.01\\
82.12	0.01\\
82.13	0.01\\
82.14	0.01\\
82.15	0.01\\
82.16	0.01\\
82.17	0.01\\
82.18	0.01\\
82.19	0.01\\
82.2	0.01\\
82.21	0.01\\
82.22	0.01\\
82.23	0.01\\
82.24	0.01\\
82.25	0.01\\
82.26	0.01\\
82.27	0.01\\
82.28	0.01\\
82.29	0.01\\
82.3	0.01\\
82.31	0.01\\
82.32	0.01\\
82.33	0.01\\
82.34	0.01\\
82.35	0.01\\
82.36	0.01\\
82.37	0.01\\
82.38	0.01\\
82.39	0.01\\
82.4	0.01\\
82.41	0.01\\
82.42	0.01\\
82.43	0.01\\
82.44	0.01\\
82.45	0.01\\
82.46	0.01\\
82.47	0.01\\
82.48	0.01\\
82.49	0.01\\
82.5	0.01\\
82.51	0.01\\
82.52	0.01\\
82.53	0.01\\
82.54	0.01\\
82.55	0.01\\
82.56	0.01\\
82.57	0.01\\
82.58	0.01\\
82.59	0.01\\
82.6	0.01\\
82.61	0.01\\
82.62	0.01\\
82.63	0.01\\
82.64	0.01\\
82.65	0.01\\
82.66	0.01\\
82.67	0.01\\
82.68	0.01\\
82.69	0.01\\
82.7	0.01\\
82.71	0.01\\
82.72	0.01\\
82.73	0.01\\
82.74	0.01\\
82.75	0.01\\
82.76	0.01\\
82.77	0.01\\
82.78	0.01\\
82.79	0.01\\
82.8	0.01\\
82.81	0.01\\
82.82	0.01\\
82.83	0.01\\
82.84	0.01\\
82.85	0.01\\
82.86	0.01\\
82.87	0.01\\
82.88	0.01\\
82.89	0.01\\
82.9	0.01\\
82.91	0.01\\
82.92	0.01\\
82.93	0.01\\
82.94	0.01\\
82.95	0.01\\
82.96	0.01\\
82.97	0.01\\
82.98	0.01\\
82.99	0.01\\
83	0.01\\
83.01	0.01\\
83.02	0.01\\
83.03	0.01\\
83.04	0.01\\
83.05	0.01\\
83.06	0.01\\
83.07	0.01\\
83.08	0.01\\
83.09	0.01\\
83.1	0.01\\
83.11	0.01\\
83.12	0.01\\
83.13	0.01\\
83.14	0.01\\
83.15	0.01\\
83.16	0.01\\
83.17	0.01\\
83.18	0.01\\
83.19	0.01\\
83.2	0.01\\
83.21	0.01\\
83.22	0.01\\
83.23	0.01\\
83.24	0.01\\
83.25	0.01\\
83.26	0.01\\
83.27	0.01\\
83.28	0.01\\
83.29	0.01\\
83.3	0.01\\
83.31	0.01\\
83.32	0.01\\
83.33	0.01\\
83.34	0.01\\
83.35	0.01\\
83.36	0.01\\
83.37	0.01\\
83.38	0.01\\
83.39	0.01\\
83.4	0.01\\
83.41	0.01\\
83.42	0.01\\
83.43	0.01\\
83.44	0.01\\
83.45	0.01\\
83.46	0.01\\
83.47	0.01\\
83.48	0.01\\
83.49	0.01\\
83.5	0.01\\
83.51	0.01\\
83.52	0.01\\
83.53	0.01\\
83.54	0.01\\
83.55	0.01\\
83.56	0.01\\
83.57	0.01\\
83.58	0.01\\
83.59	0.01\\
83.6	0.01\\
83.61	0.01\\
83.62	0.01\\
83.63	0.01\\
83.64	0.01\\
83.65	0.01\\
83.66	0.01\\
83.67	0.01\\
83.68	0.01\\
83.69	0.01\\
83.7	0.01\\
83.71	0.01\\
83.72	0.01\\
83.73	0.01\\
83.74	0.01\\
83.75	0.01\\
83.76	0.01\\
83.77	0.01\\
83.78	0.01\\
83.79	0.01\\
83.8	0.01\\
83.81	0.01\\
83.82	0.01\\
83.83	0.01\\
83.84	0.01\\
83.85	0.01\\
83.86	0.01\\
83.87	0.01\\
83.88	0.01\\
83.89	0.01\\
83.9	0.01\\
83.91	0.01\\
83.92	0.01\\
83.93	0.01\\
83.94	0.01\\
83.95	0.01\\
83.96	0.01\\
83.97	0.01\\
83.98	0.01\\
83.99	0.01\\
84	0.01\\
84.01	0.01\\
84.02	0.01\\
84.03	0.01\\
84.04	0.01\\
84.05	0.01\\
84.06	0.01\\
84.07	0.01\\
84.08	0.01\\
84.09	0.01\\
84.1	0.01\\
84.11	0.01\\
84.12	0.01\\
84.13	0.01\\
84.14	0.01\\
84.15	0.01\\
84.16	0.01\\
84.17	0.01\\
84.18	0.01\\
84.19	0.01\\
84.2	0.01\\
84.21	0.01\\
84.22	0.01\\
84.23	0.01\\
84.24	0.01\\
84.25	0.01\\
84.26	0.01\\
84.27	0.01\\
84.28	0.01\\
84.29	0.01\\
84.3	0.01\\
84.31	0.01\\
84.32	0.01\\
84.33	0.01\\
84.34	0.01\\
84.35	0.01\\
84.36	0.01\\
84.37	0.01\\
84.38	0.01\\
84.39	0.01\\
84.4	0.01\\
84.41	0.01\\
84.42	0.01\\
84.43	0.01\\
84.44	0.01\\
84.45	0.01\\
84.46	0.01\\
84.47	0.01\\
84.48	0.01\\
84.49	0.01\\
84.5	0.01\\
84.51	0.01\\
84.52	0.01\\
84.53	0.01\\
84.54	0.01\\
84.55	0.01\\
84.56	0.01\\
84.57	0.01\\
84.58	0.01\\
84.59	0.01\\
84.6	0.01\\
84.61	0.01\\
84.62	0.01\\
84.63	0.01\\
84.64	0.01\\
84.65	0.01\\
84.66	0.01\\
84.67	0.01\\
84.68	0.01\\
84.69	0.01\\
84.7	0.01\\
84.71	0.01\\
84.72	0.01\\
84.73	0.01\\
84.74	0.01\\
84.75	0.01\\
84.76	0.01\\
84.77	0.01\\
84.78	0.01\\
84.79	0.01\\
84.8	0.01\\
84.81	0.01\\
84.82	0.01\\
84.83	0.01\\
84.84	0.01\\
84.85	0.01\\
84.86	0.01\\
84.87	0.01\\
84.88	0.01\\
84.89	0.01\\
84.9	0.01\\
84.91	0.01\\
84.92	0.01\\
84.93	0.01\\
84.94	0.01\\
84.95	0.01\\
84.96	0.01\\
84.97	0.01\\
84.98	0.01\\
84.99	0.01\\
85	0.01\\
85.01	0.01\\
85.02	0.01\\
85.03	0.01\\
85.04	0.01\\
85.05	0.01\\
85.06	0.01\\
85.07	0.01\\
85.08	0.01\\
85.09	0.01\\
85.1	0.01\\
85.11	0.01\\
85.12	0.01\\
85.13	0.01\\
85.14	0.01\\
85.15	0.01\\
85.16	0.01\\
85.17	0.01\\
85.18	0.01\\
85.19	0.01\\
85.2	0.01\\
85.21	0.01\\
85.22	0.01\\
85.23	0.01\\
85.24	0.01\\
85.25	0.01\\
85.26	0.01\\
85.27	0.01\\
85.28	0.01\\
85.29	0.01\\
85.3	0.01\\
85.31	0.01\\
85.32	0.01\\
85.33	0.01\\
85.34	0.01\\
85.35	0.01\\
85.36	0.01\\
85.37	0.01\\
85.38	0.01\\
85.39	0.01\\
85.4	0.01\\
85.41	0.01\\
85.42	0.01\\
85.43	0.01\\
85.44	0.01\\
85.45	0.01\\
85.46	0.01\\
85.47	0.01\\
85.48	0.01\\
85.49	0.01\\
85.5	0.01\\
85.51	0.01\\
85.52	0.01\\
85.53	0.01\\
85.54	0.01\\
85.55	0.01\\
85.56	0.01\\
85.57	0.01\\
85.58	0.01\\
85.59	0.01\\
85.6	0.01\\
85.61	0.01\\
85.62	0.01\\
85.63	0.01\\
85.64	0.01\\
85.65	0.01\\
85.66	0.01\\
85.67	0.01\\
85.68	0.01\\
85.69	0.01\\
85.7	0.01\\
85.71	0.01\\
85.72	0.01\\
85.73	0.01\\
85.74	0.01\\
85.75	0.01\\
85.76	0.01\\
85.77	0.01\\
85.78	0.01\\
85.79	0.01\\
85.8	0.01\\
85.81	0.01\\
85.82	0.01\\
85.83	0.01\\
85.84	0.01\\
85.85	0.01\\
85.86	0.01\\
85.87	0.01\\
85.88	0.01\\
85.89	0.01\\
85.9	0.01\\
85.91	0.01\\
85.92	0.01\\
85.93	0.01\\
85.94	0.01\\
85.95	0.01\\
85.96	0.01\\
85.97	0.01\\
85.98	0.01\\
85.99	0.01\\
86	0.01\\
86.01	0.01\\
86.02	0.01\\
86.03	0.01\\
86.04	0.01\\
86.05	0.01\\
86.06	0.01\\
86.07	0.01\\
86.08	0.01\\
86.09	0.01\\
86.1	0.01\\
86.11	0.01\\
86.12	0.01\\
86.13	0.01\\
86.14	0.01\\
86.15	0.01\\
86.16	0.01\\
86.17	0.01\\
86.18	0.01\\
86.19	0.01\\
86.2	0.01\\
86.21	0.01\\
86.22	0.01\\
86.23	0.01\\
86.24	0.01\\
86.25	0.01\\
86.26	0.01\\
86.27	0.01\\
86.28	0.01\\
86.29	0.01\\
86.3	0.01\\
86.31	0.01\\
86.32	0.01\\
86.33	0.01\\
86.34	0.01\\
86.35	0.01\\
86.36	0.01\\
86.37	0.01\\
86.38	0.01\\
86.39	0.01\\
86.4	0.01\\
86.41	0.01\\
86.42	0.01\\
86.43	0.01\\
86.44	0.01\\
86.45	0.01\\
86.46	0.01\\
86.47	0.01\\
86.48	0.01\\
86.49	0.01\\
86.5	0.01\\
86.51	0.01\\
86.52	0.01\\
86.53	0.01\\
86.54	0.01\\
86.55	0.01\\
86.56	0.01\\
86.57	0.01\\
86.58	0.01\\
86.59	0.01\\
86.6	0.01\\
86.61	0.01\\
86.62	0.01\\
86.63	0.01\\
86.64	0.01\\
86.65	0.01\\
86.66	0.01\\
86.67	0.01\\
86.68	0.01\\
86.69	0.01\\
86.7	0.01\\
86.71	0.01\\
86.72	0.01\\
86.73	0.01\\
86.74	0.01\\
86.75	0.01\\
86.76	0.01\\
86.77	0.01\\
86.78	0.01\\
86.79	0.01\\
86.8	0.01\\
86.81	0.01\\
86.82	0.01\\
86.83	0.01\\
86.84	0.01\\
86.85	0.01\\
86.86	0.01\\
86.87	0.01\\
86.88	0.01\\
86.89	0.01\\
86.9	0.01\\
86.91	0.01\\
86.92	0.01\\
86.93	0.01\\
86.94	0.01\\
86.95	0.01\\
86.96	0.01\\
86.97	0.01\\
86.98	0.01\\
86.99	0.01\\
87	0.01\\
87.01	0.01\\
87.02	0.01\\
87.03	0.01\\
87.04	0.01\\
87.05	0.01\\
87.06	0.01\\
87.07	0.01\\
87.08	0.01\\
87.09	0.01\\
87.1	0.01\\
87.11	0.01\\
87.12	0.01\\
87.13	0.01\\
87.14	0.01\\
87.15	0.01\\
87.16	0.01\\
87.17	0.01\\
87.18	0.01\\
87.19	0.01\\
87.2	0.01\\
87.21	0.01\\
87.22	0.01\\
87.23	0.01\\
87.24	0.01\\
87.25	0.01\\
87.26	0.01\\
87.27	0.01\\
87.28	0.01\\
87.29	0.01\\
87.3	0.01\\
87.31	0.01\\
87.32	0.01\\
87.33	0.01\\
87.34	0.01\\
87.35	0.01\\
87.36	0.01\\
87.37	0.01\\
87.38	0.01\\
87.39	0.01\\
87.4	0.01\\
87.41	0.01\\
87.42	0.01\\
87.43	0.01\\
87.44	0.01\\
87.45	0.01\\
87.46	0.01\\
87.47	0.01\\
87.48	0.01\\
87.49	0.01\\
87.5	0.01\\
87.51	0.01\\
87.52	0.01\\
87.53	0.01\\
87.54	0.01\\
87.55	0.01\\
87.56	0.01\\
87.57	0.01\\
87.58	0.01\\
87.59	0.01\\
87.6	0.01\\
87.61	0.01\\
87.62	0.01\\
87.63	0.01\\
87.64	0.01\\
87.65	0.01\\
87.66	0.01\\
87.67	0.01\\
87.68	0.01\\
87.69	0.01\\
87.7	0.01\\
87.71	0.01\\
87.72	0.01\\
87.73	0.01\\
87.74	0.01\\
87.75	0.01\\
87.76	0.01\\
87.77	0.01\\
87.78	0.01\\
87.79	0.01\\
87.8	0.01\\
87.81	0.01\\
87.82	0.01\\
87.83	0.01\\
87.84	0.01\\
87.85	0.01\\
87.86	0.01\\
87.87	0.01\\
87.88	0.01\\
87.89	0.01\\
87.9	0.01\\
87.91	0.01\\
87.92	0.01\\
87.93	0.01\\
87.94	0.01\\
87.95	0.01\\
87.96	0.01\\
87.97	0.01\\
87.98	0.01\\
87.99	0.01\\
88	0.01\\
88.01	0.01\\
88.02	0.01\\
88.03	0.01\\
88.04	0.01\\
88.05	0.01\\
88.06	0.01\\
88.07	0.01\\
88.08	0.01\\
88.09	0.01\\
88.1	0.01\\
88.11	0.01\\
88.12	0.01\\
88.13	0.01\\
88.14	0.01\\
88.15	0.01\\
88.16	0.01\\
88.17	0.01\\
88.18	0.01\\
88.19	0.01\\
88.2	0.01\\
88.21	0.01\\
88.22	0.01\\
88.23	0.01\\
88.24	0.01\\
88.25	0.01\\
88.26	0.01\\
88.27	0.01\\
88.28	0.01\\
88.29	0.01\\
88.3	0.01\\
88.31	0.01\\
88.32	0.01\\
88.33	0.01\\
88.34	0.01\\
88.35	0.01\\
88.36	0.01\\
88.37	0.01\\
88.38	0.01\\
88.39	0.01\\
88.4	0.01\\
88.41	0.01\\
88.42	0.01\\
88.43	0.01\\
88.44	0.01\\
88.45	0.01\\
88.46	0.01\\
88.47	0.01\\
88.48	0.01\\
88.49	0.01\\
88.5	0.01\\
88.51	0.01\\
88.52	0.01\\
88.53	0.01\\
88.54	0.01\\
88.55	0.01\\
88.56	0.01\\
88.57	0.01\\
88.58	0.01\\
88.59	0.01\\
88.6	0.01\\
88.61	0.01\\
88.62	0.01\\
88.63	0.01\\
88.64	0.01\\
88.65	0.01\\
88.66	0.01\\
88.67	0.01\\
88.68	0.01\\
88.69	0.01\\
88.7	0.01\\
88.71	0.01\\
88.72	0.01\\
88.73	0.01\\
88.74	0.01\\
88.75	0.01\\
88.76	0.01\\
88.77	0.01\\
88.78	0.01\\
88.79	0.01\\
88.8	0.01\\
88.81	0.01\\
88.82	0.01\\
88.83	0.01\\
88.84	0.01\\
88.85	0.01\\
88.86	0.01\\
88.87	0.01\\
88.88	0.01\\
88.89	0.01\\
88.9	0.01\\
88.91	0.01\\
88.92	0.01\\
88.93	0.01\\
88.94	0.01\\
88.95	0.01\\
88.96	0.01\\
88.97	0.01\\
88.98	0.01\\
88.99	0.01\\
89	0.01\\
89.01	0.01\\
89.02	0.01\\
89.03	0.01\\
89.04	0.01\\
89.05	0.01\\
89.06	0.01\\
89.07	0.01\\
89.08	0.01\\
89.09	0.01\\
89.1	0.01\\
89.11	0.01\\
89.12	0.01\\
89.13	0.01\\
89.14	0.01\\
89.15	0.01\\
89.16	0.01\\
89.17	0.01\\
89.18	0.01\\
89.19	0.01\\
89.2	0.01\\
89.21	0.01\\
89.22	0.01\\
89.23	0.01\\
89.24	0.01\\
89.25	0.01\\
89.26	0.01\\
89.27	0.01\\
89.28	0.01\\
89.29	0.01\\
89.3	0.01\\
89.31	0.01\\
89.32	0.01\\
89.33	0.01\\
89.34	0.01\\
89.35	0.01\\
89.36	0.01\\
89.37	0.01\\
89.38	0.01\\
89.39	0.01\\
89.4	0.01\\
89.41	0.01\\
89.42	0.01\\
89.43	0.01\\
89.44	0.01\\
89.45	0.01\\
89.46	0.01\\
89.47	0.01\\
89.48	0.01\\
89.49	0.01\\
89.5	0.01\\
89.51	0.01\\
89.52	0.01\\
89.53	0.01\\
89.54	0.01\\
89.55	0.01\\
89.56	0.01\\
89.57	0.01\\
89.58	0.01\\
89.59	0.01\\
89.6	0.01\\
89.61	0.01\\
89.62	0.01\\
89.63	0.01\\
89.64	0.01\\
89.65	0.01\\
89.66	0.01\\
89.67	0.01\\
89.68	0.01\\
89.69	0.01\\
89.7	0.01\\
89.71	0.01\\
89.72	0.01\\
89.73	0.01\\
89.74	0.01\\
89.75	0.01\\
89.76	0.01\\
89.77	0.01\\
89.78	0.01\\
89.79	0.01\\
89.8	0.01\\
89.81	0.01\\
89.82	0.01\\
89.83	0.01\\
89.84	0.01\\
89.85	0.01\\
89.86	0.01\\
89.87	0.01\\
89.88	0.01\\
89.89	0.01\\
89.9	0.01\\
89.91	0.01\\
89.92	0.01\\
89.93	0.01\\
89.94	0.01\\
89.95	0.01\\
89.96	0.01\\
89.97	0.01\\
89.98	0.01\\
89.99	0.01\\
90	0.01\\
90.01	0.01\\
90.02	0.01\\
90.03	0.01\\
90.04	0.01\\
90.05	0.01\\
90.06	0.01\\
90.07	0.01\\
90.08	0.01\\
90.09	0.01\\
90.1	0.01\\
90.11	0.01\\
90.12	0.01\\
90.13	0.01\\
90.14	0.01\\
90.15	0.01\\
90.16	0.01\\
90.17	0.01\\
90.18	0.01\\
90.19	0.01\\
90.2	0.01\\
90.21	0.01\\
90.22	0.01\\
90.23	0.01\\
90.24	0.01\\
90.25	0.01\\
90.26	0.01\\
90.27	0.01\\
90.28	0.01\\
90.29	0.01\\
90.3	0.01\\
90.31	0.01\\
90.32	0.01\\
90.33	0.01\\
90.34	0.01\\
90.35	0.01\\
90.36	0.01\\
90.37	0.01\\
90.38	0.01\\
90.39	0.01\\
90.4	0.01\\
90.41	0.01\\
90.42	0.01\\
90.43	0.01\\
90.44	0.01\\
90.45	0.01\\
90.46	0.01\\
90.47	0.01\\
90.48	0.01\\
90.49	0.01\\
90.5	0.01\\
90.51	0.01\\
90.52	0.01\\
90.53	0.01\\
90.54	0.01\\
90.55	0.01\\
90.56	0.01\\
90.57	0.01\\
90.58	0.01\\
90.59	0.01\\
90.6	0.01\\
90.61	0.01\\
90.62	0.01\\
90.63	0.01\\
90.64	0.01\\
90.65	0.01\\
90.66	0.01\\
90.67	0.01\\
90.68	0.01\\
90.69	0.01\\
90.7	0.01\\
90.71	0.01\\
90.72	0.01\\
90.73	0.01\\
90.74	0.01\\
90.75	0.01\\
90.76	0.01\\
90.77	0.01\\
90.78	0.01\\
90.79	0.01\\
90.8	0.01\\
90.81	0.01\\
90.82	0.01\\
90.83	0.01\\
90.84	0.01\\
90.85	0.01\\
90.86	0.01\\
90.87	0.01\\
90.88	0.01\\
90.89	0.01\\
90.9	0.01\\
90.91	0.01\\
90.92	0.01\\
90.93	0.01\\
90.94	0.01\\
90.95	0.01\\
90.96	0.01\\
90.97	0.01\\
90.98	0.01\\
90.99	0.01\\
91	0.01\\
91.01	0.01\\
91.02	0.01\\
91.03	0.01\\
91.04	0.01\\
91.05	0.01\\
91.06	0.01\\
91.07	0.01\\
91.08	0.01\\
91.09	0.01\\
91.1	0.01\\
91.11	0.01\\
91.12	0.01\\
91.13	0.01\\
91.14	0.01\\
91.15	0.01\\
91.16	0.01\\
91.17	0.01\\
91.18	0.01\\
91.19	0.01\\
91.2	0.01\\
91.21	0.01\\
91.22	0.01\\
91.23	0.01\\
91.24	0.01\\
91.25	0.01\\
91.26	0.01\\
91.27	0.01\\
91.28	0.01\\
91.29	0.01\\
91.3	0.01\\
91.31	0.01\\
91.32	0.01\\
91.33	0.01\\
91.34	0.01\\
91.35	0.01\\
91.36	0.01\\
91.37	0.01\\
91.38	0.01\\
91.39	0.01\\
91.4	0.01\\
91.41	0.01\\
91.42	0.01\\
91.43	0.01\\
91.44	0.01\\
91.45	0.01\\
91.46	0.01\\
91.47	0.01\\
91.48	0.01\\
91.49	0.01\\
91.5	0.01\\
91.51	0.01\\
91.52	0.01\\
91.53	0.01\\
91.54	0.01\\
91.55	0.01\\
91.56	0.01\\
91.57	0.01\\
91.58	0.01\\
91.59	0.01\\
91.6	0.01\\
91.61	0.01\\
91.62	0.01\\
91.63	0.01\\
91.64	0.01\\
91.65	0.01\\
91.66	0.01\\
91.67	0.01\\
91.68	0.01\\
91.69	0.01\\
91.7	0.01\\
91.71	0.01\\
91.72	0.01\\
91.73	0.01\\
91.74	0.01\\
91.75	0.01\\
91.76	0.01\\
91.77	0.01\\
91.78	0.01\\
91.79	0.01\\
91.8	0.01\\
91.81	0.01\\
91.82	0.01\\
91.83	0.01\\
91.84	0.01\\
91.85	0.01\\
91.86	0.01\\
91.87	0.01\\
91.88	0.01\\
91.89	0.01\\
91.9	0.01\\
91.91	0.01\\
91.92	0.01\\
91.93	0.01\\
91.94	0.01\\
91.95	0.01\\
91.96	0.01\\
91.97	0.01\\
91.98	0.01\\
91.99	0.01\\
92	0.01\\
92.01	0.01\\
92.02	0.01\\
92.03	0.01\\
92.04	0.01\\
92.05	0.01\\
92.06	0.01\\
92.07	0.01\\
92.08	0.01\\
92.09	0.01\\
92.1	0.01\\
92.11	0.01\\
92.12	0.01\\
92.13	0.01\\
92.14	0.01\\
92.15	0.01\\
92.16	0.01\\
92.17	0.01\\
92.18	0.01\\
92.19	0.01\\
92.2	0.01\\
92.21	0.01\\
92.22	0.01\\
92.23	0.01\\
92.24	0.01\\
92.25	0.01\\
92.26	0.01\\
92.27	0.01\\
92.28	0.01\\
92.29	0.01\\
92.3	0.01\\
92.31	0.01\\
92.32	0.01\\
92.33	0.01\\
92.34	0.01\\
92.35	0.01\\
92.36	0.01\\
92.37	0.01\\
92.38	0.01\\
92.39	0.01\\
92.4	0.01\\
92.41	0.01\\
92.42	0.01\\
92.43	0.01\\
92.44	0.01\\
92.45	0.01\\
92.46	0.01\\
92.47	0.01\\
92.48	0.01\\
92.49	0.01\\
92.5	0.01\\
92.51	0.01\\
92.52	0.01\\
92.53	0.01\\
92.54	0.01\\
92.55	0.01\\
92.56	0.01\\
92.57	0.01\\
92.58	0.01\\
92.59	0.01\\
92.6	0.01\\
92.61	0.01\\
92.62	0.01\\
92.63	0.01\\
92.64	0.01\\
92.65	0.01\\
92.66	0.01\\
92.67	0.01\\
92.68	0.01\\
92.69	0.01\\
92.7	0.01\\
92.71	0.01\\
92.72	0.01\\
92.73	0.01\\
92.74	0.01\\
92.75	0.01\\
92.76	0.01\\
92.77	0.01\\
92.78	0.01\\
92.79	0.01\\
92.8	0.01\\
92.81	0.01\\
92.82	0.01\\
92.83	0.01\\
92.84	0.01\\
92.85	0.01\\
92.86	0.01\\
92.87	0.01\\
92.88	0.01\\
92.89	0.01\\
92.9	0.01\\
92.91	0.01\\
92.92	0.01\\
92.93	0.01\\
92.94	0.01\\
92.95	0.01\\
92.96	0.01\\
92.97	0.01\\
92.98	0.01\\
92.99	0.01\\
93	0.01\\
93.01	0.01\\
93.02	0.01\\
93.03	0.01\\
93.04	0.01\\
93.05	0.01\\
93.06	0.01\\
93.07	0.01\\
93.08	0.01\\
93.09	0.01\\
93.1	0.01\\
93.11	0.01\\
93.12	0.01\\
93.13	0.01\\
93.14	0.01\\
93.15	0.01\\
93.16	0.01\\
93.17	0.01\\
93.18	0.01\\
93.19	0.01\\
93.2	0.01\\
93.21	0.01\\
93.22	0.01\\
93.23	0.01\\
93.24	0.01\\
93.25	0.01\\
93.26	0.01\\
93.27	0.01\\
93.28	0.01\\
93.29	0.01\\
93.3	0.01\\
93.31	0.01\\
93.32	0.01\\
93.33	0.01\\
93.34	0.01\\
93.35	0.01\\
93.36	0.01\\
93.37	0.01\\
93.38	0.01\\
93.39	0.01\\
93.4	0.01\\
93.41	0.01\\
93.42	0.01\\
93.43	0.01\\
93.44	0.01\\
93.45	0.01\\
93.46	0.01\\
93.47	0.01\\
93.48	0.01\\
93.49	0.01\\
93.5	0.01\\
93.51	0.01\\
93.52	0.01\\
93.53	0.01\\
93.54	0.01\\
93.55	0.01\\
93.56	0.01\\
93.57	0.01\\
93.58	0.01\\
93.59	0.01\\
93.6	0.01\\
93.61	0.01\\
93.62	0.01\\
93.63	0.01\\
93.64	0.01\\
93.65	0.01\\
93.66	0.01\\
93.67	0.01\\
93.68	0.01\\
93.69	0.01\\
93.7	0.01\\
93.71	0.01\\
93.72	0.01\\
93.73	0.01\\
93.74	0.01\\
93.75	0.01\\
93.76	0.01\\
93.77	0.01\\
93.78	0.01\\
93.79	0.01\\
93.8	0.01\\
93.81	0.01\\
93.82	0.01\\
93.83	0.01\\
93.84	0.01\\
93.85	0.01\\
93.86	0.01\\
93.87	0.01\\
93.88	0.01\\
93.89	0.01\\
93.9	0.01\\
93.91	0.01\\
93.92	0.01\\
93.93	0.01\\
93.94	0.01\\
93.95	0.01\\
93.96	0.01\\
93.97	0.01\\
93.98	0.01\\
93.99	0.01\\
94	0.01\\
94.01	0.01\\
94.02	0.01\\
94.03	0.01\\
94.04	0.01\\
94.05	0.01\\
94.06	0.01\\
94.07	0.01\\
94.08	0.01\\
94.09	0.01\\
94.1	0.01\\
94.11	0.01\\
94.12	0.01\\
94.13	0.01\\
94.14	0.01\\
94.15	0.01\\
94.16	0.01\\
94.17	0.01\\
94.18	0.01\\
94.19	0.01\\
94.2	0.01\\
94.21	0.01\\
94.22	0.01\\
94.23	0.01\\
94.24	0.01\\
94.25	0.01\\
94.26	0.01\\
94.27	0.01\\
94.28	0.01\\
94.29	0.01\\
94.3	0.01\\
94.31	0.01\\
94.32	0.01\\
94.33	0.01\\
94.34	0.01\\
94.35	0.01\\
94.36	0.01\\
94.37	0.01\\
94.38	0.01\\
94.39	0.01\\
94.4	0.01\\
94.41	0.01\\
94.42	0.01\\
94.43	0.01\\
94.44	0.01\\
94.45	0.01\\
94.46	0.01\\
94.47	0.01\\
94.48	0.01\\
94.49	0.01\\
94.5	0.01\\
94.51	0.01\\
94.52	0.01\\
94.53	0.01\\
94.54	0.01\\
94.55	0.01\\
94.56	0.01\\
94.57	0.01\\
94.58	0.01\\
94.59	0.01\\
94.6	0.01\\
94.61	0.01\\
94.62	0.01\\
94.63	0.01\\
94.64	0.01\\
94.65	0.01\\
94.66	0.01\\
94.67	0.01\\
94.68	0.01\\
94.69	0.01\\
94.7	0.01\\
94.71	0.01\\
94.72	0.01\\
94.73	0.01\\
94.74	0.01\\
94.75	0.01\\
94.76	0.01\\
94.77	0.01\\
94.78	0.01\\
94.79	0.01\\
94.8	0.01\\
94.81	0.01\\
94.82	0.01\\
94.83	0.01\\
94.84	0.01\\
94.85	0.01\\
94.86	0.01\\
94.87	0.01\\
94.88	0.01\\
94.89	0.01\\
94.9	0.01\\
94.91	0.01\\
94.92	0.01\\
94.93	0.01\\
94.94	0.01\\
94.95	0.01\\
94.96	0.01\\
94.97	0.01\\
94.98	0.01\\
94.99	0.01\\
95	0.01\\
95.01	0.01\\
95.02	0.01\\
95.03	0.01\\
95.04	0.01\\
95.05	0.01\\
95.06	0.01\\
95.07	0.01\\
95.08	0.01\\
95.09	0.01\\
95.1	0.01\\
95.11	0.01\\
95.12	0.01\\
95.13	0.01\\
95.14	0.01\\
95.15	0.01\\
95.16	0.01\\
95.17	0.01\\
95.18	0.01\\
95.19	0.01\\
95.2	0.01\\
95.21	0.01\\
95.22	0.01\\
95.23	0.01\\
95.24	0.01\\
95.25	0.01\\
95.26	0.01\\
95.27	0.01\\
95.28	0.01\\
95.29	0.01\\
95.3	0.01\\
95.31	0.01\\
95.32	0.01\\
95.33	0.01\\
95.34	0.01\\
95.35	0.01\\
95.36	0.01\\
95.37	0.01\\
95.38	0.01\\
95.39	0.01\\
95.4	0.01\\
95.41	0.01\\
95.42	0.01\\
95.43	0.01\\
95.44	0.01\\
95.45	0.01\\
95.46	0.01\\
95.47	0.01\\
95.48	0.01\\
95.49	0.01\\
95.5	0.01\\
95.51	0.01\\
95.52	0.01\\
95.53	0.01\\
95.54	0.01\\
95.55	0.01\\
95.56	0.01\\
95.57	0.01\\
95.58	0.01\\
95.59	0.01\\
95.6	0.01\\
95.61	0.01\\
95.62	0.01\\
95.63	0.01\\
95.64	0.01\\
95.65	0.01\\
95.66	0.01\\
95.67	0.01\\
95.68	0.01\\
95.69	0.01\\
95.7	0.01\\
95.71	0.01\\
95.72	0.01\\
95.73	0.01\\
95.74	0.01\\
95.75	0.01\\
95.76	0.01\\
95.77	0.01\\
95.78	0.01\\
95.79	0.01\\
95.8	0.01\\
95.81	0.01\\
95.82	0.01\\
95.83	0.01\\
95.84	0.01\\
95.85	0.01\\
95.86	0.01\\
95.87	0.01\\
95.88	0.01\\
95.89	0.01\\
95.9	0.01\\
95.91	0.01\\
95.92	0.01\\
95.93	0.01\\
95.94	0.01\\
95.95	0.01\\
95.96	0.01\\
95.97	0.01\\
95.98	0.01\\
95.99	0.01\\
96	0.01\\
96.01	0.01\\
96.02	0.01\\
96.03	0.01\\
96.04	0.01\\
96.05	0.01\\
96.06	0.01\\
96.07	0.01\\
96.08	0.01\\
96.09	0.01\\
96.1	0.01\\
96.11	0.01\\
96.12	0.01\\
96.13	0.01\\
96.14	0.01\\
96.15	0.01\\
96.16	0.01\\
96.17	0.01\\
96.18	0.01\\
96.19	0.01\\
96.2	0.01\\
96.21	0.01\\
96.22	0.01\\
96.23	0.01\\
96.24	0.01\\
96.25	0.01\\
96.26	0.01\\
96.27	0.01\\
96.28	0.01\\
96.29	0.01\\
96.3	0.01\\
96.31	0.01\\
96.32	0.01\\
96.33	0.01\\
96.34	0.01\\
96.35	0.01\\
96.36	0.01\\
96.37	0.01\\
96.38	0.01\\
96.39	0.01\\
96.4	0.01\\
96.41	0.01\\
96.42	0.01\\
96.43	0.01\\
96.44	0.01\\
96.45	0.01\\
96.46	0.01\\
96.47	0.01\\
96.48	0.01\\
96.49	0.01\\
96.5	0.01\\
96.51	0.01\\
96.52	0.01\\
96.53	0.01\\
96.54	0.01\\
96.55	0.01\\
96.56	0.01\\
96.57	0.01\\
96.58	0.01\\
96.59	0.01\\
96.6	0.01\\
96.61	0.01\\
96.62	0.01\\
96.63	0.01\\
96.64	0.01\\
96.65	0.01\\
96.66	0.01\\
96.67	0.01\\
96.68	0.01\\
96.69	0.01\\
96.7	0.01\\
96.71	0.01\\
96.72	0.01\\
96.73	0.01\\
96.74	0.01\\
96.75	0.01\\
96.76	0.01\\
96.77	0.01\\
96.78	0.01\\
96.79	0.01\\
96.8	0.01\\
96.81	0.01\\
96.82	0.01\\
96.83	0.01\\
96.84	0.01\\
96.85	0.01\\
96.86	0.01\\
96.87	0.01\\
96.88	0.01\\
96.89	0.01\\
96.9	0.01\\
96.91	0.01\\
96.92	0.01\\
96.93	0.01\\
96.94	0.01\\
96.95	0.01\\
96.96	0.01\\
96.97	0.01\\
96.98	0.01\\
96.99	0.01\\
97	0.01\\
97.01	0.01\\
97.02	0.01\\
97.03	0.01\\
97.04	0.01\\
97.05	0.01\\
97.06	0.01\\
97.07	0.01\\
97.08	0.01\\
97.09	0.01\\
97.1	0.01\\
97.11	0.01\\
97.12	0.01\\
97.13	0.01\\
97.14	0.01\\
97.15	0.01\\
97.16	0.01\\
97.17	0.01\\
97.18	0.01\\
97.19	0.01\\
97.2	0.01\\
97.21	0.01\\
97.22	0.01\\
97.23	0.01\\
97.24	0.01\\
97.25	0.01\\
97.26	0.01\\
97.27	0.01\\
97.28	0.01\\
97.29	0.01\\
97.3	0.01\\
97.31	0.01\\
97.32	0.01\\
97.33	0.01\\
97.34	0.01\\
97.35	0.01\\
97.36	0.01\\
97.37	0.01\\
97.38	0.01\\
97.39	0.01\\
97.4	0.01\\
97.41	0.01\\
97.42	0.01\\
97.43	0.01\\
97.44	0.01\\
97.45	0.01\\
97.46	0.01\\
97.47	0.01\\
97.48	0.01\\
97.49	0.01\\
97.5	0.01\\
97.51	0.01\\
97.52	0.01\\
97.53	0.01\\
97.54	0.01\\
97.55	0.01\\
97.56	0.01\\
97.57	0.01\\
97.58	0.01\\
97.59	0.01\\
97.6	0.01\\
97.61	0.01\\
97.62	0.01\\
97.63	0.01\\
97.64	0.01\\
97.65	0.01\\
97.66	0.01\\
97.67	0.01\\
97.68	0.01\\
97.69	0.01\\
97.7	0.01\\
97.71	0.01\\
97.72	0.01\\
97.73	0.01\\
97.74	0.01\\
97.75	0.01\\
97.76	0.01\\
97.77	0.01\\
97.78	0.01\\
97.79	0.01\\
97.8	0.01\\
97.81	0.01\\
97.82	0.01\\
97.83	0.01\\
97.84	0.01\\
97.85	0.01\\
97.86	0.01\\
97.87	0.01\\
97.88	0.01\\
97.89	0.01\\
97.9	0.01\\
97.91	0.01\\
97.92	0.01\\
97.93	0.01\\
97.94	0.01\\
97.95	0.01\\
97.96	0.01\\
97.97	0.01\\
97.98	0.01\\
97.99	0.01\\
98	0.01\\
98.01	0.01\\
98.02	0.01\\
98.03	0.01\\
98.04	0.01\\
98.05	0.01\\
98.06	0.01\\
98.07	0.01\\
98.08	0.01\\
98.09	0.01\\
98.1	0.01\\
98.11	0.01\\
98.12	0.01\\
98.13	0.01\\
98.14	0.01\\
98.15	0.01\\
98.16	0.01\\
98.17	0.01\\
98.18	0.01\\
98.19	0.01\\
98.2	0.01\\
98.21	0.01\\
98.22	0.01\\
98.23	0.01\\
98.24	0.01\\
98.25	0.01\\
98.26	0.01\\
98.27	0.01\\
98.28	0.01\\
98.29	0.01\\
98.3	0.01\\
98.31	0.01\\
98.32	0.01\\
98.33	0.00994490160815747\\
98.34	0.00988583813170677\\
98.35	0.00982635360008642\\
98.36	0.00976644407475283\\
98.37	0.00970610557633694\\
98.38	0.00964533408415264\\
98.39	0.00958412553569794\\
98.4	0.00952247582481165\\
98.41	0.00946038136888883\\
98.42	0.00939784258914091\\
98.43	0.00933485533808184\\
98.44	0.00927141542511888\\
98.45	0.00920751861602622\\
98.46	0.00914316063241053\\
98.47	0.00907833715116838\\
98.48	0.00901304380393543\\
98.49	0.00894727617652721\\
98.5	0.00888102980837118\\
98.51	0.00881430019193017\\
98.52	0.00874708277211669\\
98.53	0.00867937294569829\\
98.54	0.00861116606069348\\
98.55	0.00854245741575816\\
98.56	0.00847324225914466\\
98.57	0.00840351578530768\\
98.58	0.00833327313733897\\
98.59	0.00826250940631166\\
98.6	0.00819121963061387\\
98.61	0.00811939879527165\\
98.62	0.00804704183126073\\
98.63	0.0079790913431111\\
98.64	0.00794934915251536\\
98.65	0.00791935816584449\\
98.66	0.00788911610487732\\
98.67	0.00785862067107429\\
98.68	0.00782786954540883\\
98.69	0.00779686038819744\\
98.7	0.00776559091345907\\
98.71	0.0077340588325717\\
98.72	0.00770226183754817\\
98.73	0.00767019760088812\\
98.74	0.00763785640907247\\
98.75	0.00760522778276892\\
98.76	0.00757230906535413\\
98.77	0.00753909757596089\\
98.78	0.00750559060984833\\
98.79	0.00747178525814851\\
98.8	0.0074376747626892\\
98.81	0.00740325627133064\\
98.82	0.00736852690546005\\
98.83	0.00733348375974956\\
98.84	0.00729812390191215\\
98.85	0.00726244437245534\\
98.86	0.00722644218443286\\
98.87	0.00719011432319411\\
98.88	0.00715345774613148\\
98.89	0.0071164693824255\\
98.9	0.00707914613278774\\
98.91	0.0070414848692016\\
98.92	0.00700348243720732\\
98.93	0.00696513565347961\\
98.94	0.00692644130526317\\
98.95	0.0068873961501065\\
98.96	0.00684799691559364\\
98.97	0.00680824029907353\\
98.98	0.00676812296738722\\
98.99	0.0067276415565927\\
99	0.00668679267168765\\
99.01	0.00664557288632967\\
99.02	0.00660397874255443\\
99.03	0.00656200675049135\\
99.04	0.0065196533880771\\
99.05	0.00647691510076665\\
99.06	0.0064337883012421\\
99.07	0.00639026936911908\\
99.08	0.00634635465065084\\
99.09	0.00630204045843002\\
99.1	0.00625732307108794\\
99.11	0.00621219873299165\\
99.12	0.00616666365393848\\
99.13	0.00612071400884833\\
99.14	0.00607434593745342\\
99.15	0.0060275555439858\\
99.16	0.00598033889686234\\
99.17	0.0059326920283674\\
99.18	0.005884610934333\\
99.19	0.00583609157375363\\
99.2	0.0057871298684364\\
99.21	0.00573772170267285\\
99.22	0.00568786292290844\\
99.23	0.00563754933740963\\
99.24	0.00558677671592854\\
99.25	0.00553554078936529\\
99.26	0.00548383724942787\\
99.27	0.00543166174828981\\
99.28	0.00537900989824531\\
99.29	0.00532587727136213\\
99.3	0.00527225939913211\\
99.31	0.00521815177211941\\
99.32	0.00516354983960636\\
99.33	0.00510844900923711\\
99.34	0.00505284464665895\\
99.35	0.00499673207516135\\
99.36	0.00494010657531287\\
99.37	0.00488296338459576\\
99.38	0.00482529769703843\\
99.39	0.00476710466431692\\
99.4	0.00470837945686431\\
99.41	0.00464911720181623\\
99.42	0.00458931298265236\\
99.43	0.0045289618388362\\
99.44	0.00446805876545311\\
99.45	0.00440659871284654\\
99.46	0.00434457658625268\\
99.47	0.00428198724543351\\
99.48	0.00421882550430818\\
99.49	0.00415508613058311\\
99.5	0.00409076384538048\\
99.51	0.00402585332286552\\
99.52	0.00396034918987244\\
99.53	0.00389424602552924\\
99.54	0.00382753836088138\\
99.55	0.00376022067851448\\
99.56	0.00369228741217609\\
99.57	0.00362373294639665\\
99.58	0.00355455161610983\\
99.59	0.00348473770627216\\
99.6	0.00341428545148234\\
99.61	0.00334318903560021\\
99.62	0.00327144259136098\\
99.63	0.00319904019998451\\
99.64	0.00312597589079425\\
99.65	0.00305224364083623\\
99.66	0.00297783737449881\\
99.67	0.00290275096313283\\
99.68	0.00282697822467284\\
99.69	0.0027505129232592\\
99.7	0.00267334876886162\\
99.71	0.00259547941690411\\
99.72	0.00251689846789176\\
99.73	0.00243759947411416\\
99.74	0.00235757593859451\\
99.75	0.00227682130856416\\
99.76	0.00219532897510188\\
99.77	0.00211309227277645\\
99.78	0.00203010447929275\\
99.79	0.00194635881514191\\
99.8	0.00186184844325576\\
99.81	0.00177656646866601\\
99.82	0.00169050593816866\\
99.83	0.00160365983999406\\
99.84	0.00151602110348299\\
99.85	0.00142758259876953\\
99.86	0.00133833713647095\\
99.87	0.00124827746738539\\
99.88	0.00115739628219791\\
99.89	0.00106568621119549\\
99.9	0.000973139823991604\\
99.91	0.000879749629261267\\
99.92	0.000785508074487122\\
99.93	0.000690407545717435\\
99.94	0.000594440367336854\\
99.95	0.000497598801850791\\
99.96	0.000399875049684407\\
99.97	0.000301261248997162\\
99.98	0.000201749475514013\\
99.99	0.00010133174237437\\
100	0\\
};
\addlegendentry{$q=-2$};

\addplot [color=blue,dashed,forget plot]
  table[row sep=crcr]{%
0.01	0.01\\
0.02	0.01\\
0.03	0.01\\
0.04	0.01\\
0.05	0.01\\
0.06	0.01\\
0.07	0.01\\
0.08	0.01\\
0.09	0.01\\
0.1	0.01\\
0.11	0.01\\
0.12	0.01\\
0.13	0.01\\
0.14	0.01\\
0.15	0.01\\
0.16	0.01\\
0.17	0.01\\
0.18	0.01\\
0.19	0.01\\
0.2	0.01\\
0.21	0.01\\
0.22	0.01\\
0.23	0.01\\
0.24	0.01\\
0.25	0.01\\
0.26	0.01\\
0.27	0.01\\
0.28	0.01\\
0.29	0.01\\
0.3	0.01\\
0.31	0.01\\
0.32	0.01\\
0.33	0.01\\
0.34	0.01\\
0.35	0.01\\
0.36	0.01\\
0.37	0.01\\
0.38	0.01\\
0.39	0.01\\
0.4	0.01\\
0.41	0.01\\
0.42	0.01\\
0.43	0.01\\
0.44	0.01\\
0.45	0.01\\
0.46	0.01\\
0.47	0.01\\
0.48	0.01\\
0.49	0.01\\
0.5	0.01\\
0.51	0.01\\
0.52	0.01\\
0.53	0.01\\
0.54	0.01\\
0.55	0.01\\
0.56	0.01\\
0.57	0.01\\
0.58	0.01\\
0.59	0.01\\
0.6	0.01\\
0.61	0.01\\
0.62	0.01\\
0.63	0.01\\
0.64	0.01\\
0.65	0.01\\
0.66	0.01\\
0.67	0.01\\
0.68	0.01\\
0.69	0.01\\
0.7	0.01\\
0.71	0.01\\
0.72	0.01\\
0.73	0.01\\
0.74	0.01\\
0.75	0.01\\
0.76	0.01\\
0.77	0.01\\
0.78	0.01\\
0.79	0.01\\
0.8	0.01\\
0.81	0.01\\
0.82	0.01\\
0.83	0.01\\
0.84	0.01\\
0.85	0.01\\
0.86	0.01\\
0.87	0.01\\
0.88	0.01\\
0.89	0.01\\
0.9	0.01\\
0.91	0.01\\
0.92	0.01\\
0.93	0.01\\
0.94	0.01\\
0.95	0.01\\
0.96	0.01\\
0.97	0.01\\
0.98	0.01\\
0.99	0.01\\
1	0.01\\
1.01	0.01\\
1.02	0.01\\
1.03	0.01\\
1.04	0.01\\
1.05	0.01\\
1.06	0.01\\
1.07	0.01\\
1.08	0.01\\
1.09	0.01\\
1.1	0.01\\
1.11	0.01\\
1.12	0.01\\
1.13	0.01\\
1.14	0.01\\
1.15	0.01\\
1.16	0.01\\
1.17	0.01\\
1.18	0.01\\
1.19	0.01\\
1.2	0.01\\
1.21	0.01\\
1.22	0.01\\
1.23	0.01\\
1.24	0.01\\
1.25	0.01\\
1.26	0.01\\
1.27	0.01\\
1.28	0.01\\
1.29	0.01\\
1.3	0.01\\
1.31	0.01\\
1.32	0.01\\
1.33	0.01\\
1.34	0.01\\
1.35	0.01\\
1.36	0.01\\
1.37	0.01\\
1.38	0.01\\
1.39	0.01\\
1.4	0.01\\
1.41	0.01\\
1.42	0.01\\
1.43	0.01\\
1.44	0.01\\
1.45	0.01\\
1.46	0.01\\
1.47	0.01\\
1.48	0.01\\
1.49	0.01\\
1.5	0.01\\
1.51	0.01\\
1.52	0.01\\
1.53	0.01\\
1.54	0.01\\
1.55	0.01\\
1.56	0.01\\
1.57	0.01\\
1.58	0.01\\
1.59	0.01\\
1.6	0.01\\
1.61	0.01\\
1.62	0.01\\
1.63	0.01\\
1.64	0.01\\
1.65	0.01\\
1.66	0.01\\
1.67	0.01\\
1.68	0.01\\
1.69	0.01\\
1.7	0.01\\
1.71	0.01\\
1.72	0.01\\
1.73	0.01\\
1.74	0.01\\
1.75	0.01\\
1.76	0.01\\
1.77	0.01\\
1.78	0.01\\
1.79	0.01\\
1.8	0.01\\
1.81	0.01\\
1.82	0.01\\
1.83	0.01\\
1.84	0.01\\
1.85	0.01\\
1.86	0.01\\
1.87	0.01\\
1.88	0.01\\
1.89	0.01\\
1.9	0.01\\
1.91	0.01\\
1.92	0.01\\
1.93	0.01\\
1.94	0.01\\
1.95	0.01\\
1.96	0.01\\
1.97	0.01\\
1.98	0.01\\
1.99	0.01\\
2	0.01\\
2.01	0.01\\
2.02	0.01\\
2.03	0.01\\
2.04	0.01\\
2.05	0.01\\
2.06	0.01\\
2.07	0.01\\
2.08	0.01\\
2.09	0.01\\
2.1	0.01\\
2.11	0.01\\
2.12	0.01\\
2.13	0.01\\
2.14	0.01\\
2.15	0.01\\
2.16	0.01\\
2.17	0.01\\
2.18	0.01\\
2.19	0.01\\
2.2	0.01\\
2.21	0.01\\
2.22	0.01\\
2.23	0.01\\
2.24	0.01\\
2.25	0.01\\
2.26	0.01\\
2.27	0.01\\
2.28	0.01\\
2.29	0.01\\
2.3	0.01\\
2.31	0.01\\
2.32	0.01\\
2.33	0.01\\
2.34	0.01\\
2.35	0.01\\
2.36	0.01\\
2.37	0.01\\
2.38	0.01\\
2.39	0.01\\
2.4	0.01\\
2.41	0.01\\
2.42	0.01\\
2.43	0.01\\
2.44	0.01\\
2.45	0.01\\
2.46	0.01\\
2.47	0.01\\
2.48	0.01\\
2.49	0.01\\
2.5	0.01\\
2.51	0.01\\
2.52	0.01\\
2.53	0.01\\
2.54	0.01\\
2.55	0.01\\
2.56	0.01\\
2.57	0.01\\
2.58	0.01\\
2.59	0.01\\
2.6	0.01\\
2.61	0.01\\
2.62	0.01\\
2.63	0.01\\
2.64	0.01\\
2.65	0.01\\
2.66	0.01\\
2.67	0.01\\
2.68	0.01\\
2.69	0.01\\
2.7	0.01\\
2.71	0.01\\
2.72	0.01\\
2.73	0.01\\
2.74	0.01\\
2.75	0.01\\
2.76	0.01\\
2.77	0.01\\
2.78	0.01\\
2.79	0.01\\
2.8	0.01\\
2.81	0.01\\
2.82	0.01\\
2.83	0.01\\
2.84	0.01\\
2.85	0.01\\
2.86	0.01\\
2.87	0.01\\
2.88	0.01\\
2.89	0.01\\
2.9	0.01\\
2.91	0.01\\
2.92	0.01\\
2.93	0.01\\
2.94	0.01\\
2.95	0.01\\
2.96	0.01\\
2.97	0.01\\
2.98	0.01\\
2.99	0.01\\
3	0.01\\
3.01	0.01\\
3.02	0.01\\
3.03	0.01\\
3.04	0.01\\
3.05	0.01\\
3.06	0.01\\
3.07	0.01\\
3.08	0.01\\
3.09	0.01\\
3.1	0.01\\
3.11	0.01\\
3.12	0.01\\
3.13	0.01\\
3.14	0.01\\
3.15	0.01\\
3.16	0.01\\
3.17	0.01\\
3.18	0.01\\
3.19	0.01\\
3.2	0.01\\
3.21	0.01\\
3.22	0.01\\
3.23	0.01\\
3.24	0.01\\
3.25	0.01\\
3.26	0.01\\
3.27	0.01\\
3.28	0.01\\
3.29	0.01\\
3.3	0.01\\
3.31	0.01\\
3.32	0.01\\
3.33	0.01\\
3.34	0.01\\
3.35	0.01\\
3.36	0.01\\
3.37	0.01\\
3.38	0.01\\
3.39	0.01\\
3.4	0.01\\
3.41	0.01\\
3.42	0.01\\
3.43	0.01\\
3.44	0.01\\
3.45	0.01\\
3.46	0.01\\
3.47	0.01\\
3.48	0.01\\
3.49	0.01\\
3.5	0.01\\
3.51	0.01\\
3.52	0.01\\
3.53	0.01\\
3.54	0.01\\
3.55	0.01\\
3.56	0.01\\
3.57	0.01\\
3.58	0.01\\
3.59	0.01\\
3.6	0.01\\
3.61	0.01\\
3.62	0.01\\
3.63	0.01\\
3.64	0.01\\
3.65	0.01\\
3.66	0.01\\
3.67	0.01\\
3.68	0.01\\
3.69	0.01\\
3.7	0.01\\
3.71	0.01\\
3.72	0.01\\
3.73	0.01\\
3.74	0.01\\
3.75	0.01\\
3.76	0.01\\
3.77	0.01\\
3.78	0.01\\
3.79	0.01\\
3.8	0.01\\
3.81	0.01\\
3.82	0.01\\
3.83	0.01\\
3.84	0.01\\
3.85	0.01\\
3.86	0.01\\
3.87	0.01\\
3.88	0.01\\
3.89	0.01\\
3.9	0.01\\
3.91	0.01\\
3.92	0.01\\
3.93	0.01\\
3.94	0.01\\
3.95	0.01\\
3.96	0.01\\
3.97	0.01\\
3.98	0.01\\
3.99	0.01\\
4	0.01\\
4.01	0.01\\
4.02	0.01\\
4.03	0.01\\
4.04	0.01\\
4.05	0.01\\
4.06	0.01\\
4.07	0.01\\
4.08	0.01\\
4.09	0.01\\
4.1	0.01\\
4.11	0.01\\
4.12	0.01\\
4.13	0.01\\
4.14	0.01\\
4.15	0.01\\
4.16	0.01\\
4.17	0.01\\
4.18	0.01\\
4.19	0.01\\
4.2	0.01\\
4.21	0.01\\
4.22	0.01\\
4.23	0.01\\
4.24	0.01\\
4.25	0.01\\
4.26	0.01\\
4.27	0.01\\
4.28	0.01\\
4.29	0.01\\
4.3	0.01\\
4.31	0.01\\
4.32	0.01\\
4.33	0.01\\
4.34	0.01\\
4.35	0.01\\
4.36	0.01\\
4.37	0.01\\
4.38	0.01\\
4.39	0.01\\
4.4	0.01\\
4.41	0.01\\
4.42	0.01\\
4.43	0.01\\
4.44	0.01\\
4.45	0.01\\
4.46	0.01\\
4.47	0.01\\
4.48	0.01\\
4.49	0.01\\
4.5	0.01\\
4.51	0.01\\
4.52	0.01\\
4.53	0.01\\
4.54	0.01\\
4.55	0.01\\
4.56	0.01\\
4.57	0.01\\
4.58	0.01\\
4.59	0.01\\
4.6	0.01\\
4.61	0.01\\
4.62	0.01\\
4.63	0.01\\
4.64	0.01\\
4.65	0.01\\
4.66	0.01\\
4.67	0.01\\
4.68	0.01\\
4.69	0.01\\
4.7	0.01\\
4.71	0.01\\
4.72	0.01\\
4.73	0.01\\
4.74	0.01\\
4.75	0.01\\
4.76	0.01\\
4.77	0.01\\
4.78	0.01\\
4.79	0.01\\
4.8	0.01\\
4.81	0.01\\
4.82	0.01\\
4.83	0.01\\
4.84	0.01\\
4.85	0.01\\
4.86	0.01\\
4.87	0.01\\
4.88	0.01\\
4.89	0.01\\
4.9	0.01\\
4.91	0.01\\
4.92	0.01\\
4.93	0.01\\
4.94	0.01\\
4.95	0.01\\
4.96	0.01\\
4.97	0.01\\
4.98	0.01\\
4.99	0.01\\
5	0.01\\
5.01	0.01\\
5.02	0.01\\
5.03	0.01\\
5.04	0.01\\
5.05	0.01\\
5.06	0.01\\
5.07	0.01\\
5.08	0.01\\
5.09	0.01\\
5.1	0.01\\
5.11	0.01\\
5.12	0.01\\
5.13	0.01\\
5.14	0.01\\
5.15	0.01\\
5.16	0.01\\
5.17	0.01\\
5.18	0.01\\
5.19	0.01\\
5.2	0.01\\
5.21	0.01\\
5.22	0.01\\
5.23	0.01\\
5.24	0.01\\
5.25	0.01\\
5.26	0.01\\
5.27	0.01\\
5.28	0.01\\
5.29	0.01\\
5.3	0.01\\
5.31	0.01\\
5.32	0.01\\
5.33	0.01\\
5.34	0.01\\
5.35	0.01\\
5.36	0.01\\
5.37	0.01\\
5.38	0.01\\
5.39	0.01\\
5.4	0.01\\
5.41	0.01\\
5.42	0.01\\
5.43	0.01\\
5.44	0.01\\
5.45	0.01\\
5.46	0.01\\
5.47	0.01\\
5.48	0.01\\
5.49	0.01\\
5.5	0.01\\
5.51	0.01\\
5.52	0.01\\
5.53	0.01\\
5.54	0.01\\
5.55	0.01\\
5.56	0.01\\
5.57	0.01\\
5.58	0.01\\
5.59	0.01\\
5.6	0.01\\
5.61	0.01\\
5.62	0.01\\
5.63	0.01\\
5.64	0.01\\
5.65	0.01\\
5.66	0.01\\
5.67	0.01\\
5.68	0.01\\
5.69	0.01\\
5.7	0.01\\
5.71	0.01\\
5.72	0.01\\
5.73	0.01\\
5.74	0.01\\
5.75	0.01\\
5.76	0.01\\
5.77	0.01\\
5.78	0.01\\
5.79	0.01\\
5.8	0.01\\
5.81	0.01\\
5.82	0.01\\
5.83	0.01\\
5.84	0.01\\
5.85	0.01\\
5.86	0.01\\
5.87	0.01\\
5.88	0.01\\
5.89	0.01\\
5.9	0.01\\
5.91	0.01\\
5.92	0.01\\
5.93	0.01\\
5.94	0.01\\
5.95	0.01\\
5.96	0.01\\
5.97	0.01\\
5.98	0.01\\
5.99	0.01\\
6	0.01\\
6.01	0.01\\
6.02	0.01\\
6.03	0.01\\
6.04	0.01\\
6.05	0.01\\
6.06	0.01\\
6.07	0.01\\
6.08	0.01\\
6.09	0.01\\
6.1	0.01\\
6.11	0.01\\
6.12	0.01\\
6.13	0.01\\
6.14	0.01\\
6.15	0.01\\
6.16	0.01\\
6.17	0.01\\
6.18	0.01\\
6.19	0.01\\
6.2	0.01\\
6.21	0.01\\
6.22	0.01\\
6.23	0.01\\
6.24	0.01\\
6.25	0.01\\
6.26	0.01\\
6.27	0.01\\
6.28	0.01\\
6.29	0.01\\
6.3	0.01\\
6.31	0.01\\
6.32	0.01\\
6.33	0.01\\
6.34	0.01\\
6.35	0.01\\
6.36	0.01\\
6.37	0.01\\
6.38	0.01\\
6.39	0.01\\
6.4	0.01\\
6.41	0.01\\
6.42	0.01\\
6.43	0.01\\
6.44	0.01\\
6.45	0.01\\
6.46	0.01\\
6.47	0.01\\
6.48	0.01\\
6.49	0.01\\
6.5	0.01\\
6.51	0.01\\
6.52	0.01\\
6.53	0.01\\
6.54	0.01\\
6.55	0.01\\
6.56	0.01\\
6.57	0.01\\
6.58	0.01\\
6.59	0.01\\
6.6	0.01\\
6.61	0.01\\
6.62	0.01\\
6.63	0.01\\
6.64	0.01\\
6.65	0.01\\
6.66	0.01\\
6.67	0.01\\
6.68	0.01\\
6.69	0.01\\
6.7	0.01\\
6.71	0.01\\
6.72	0.01\\
6.73	0.01\\
6.74	0.01\\
6.75	0.01\\
6.76	0.01\\
6.77	0.01\\
6.78	0.01\\
6.79	0.01\\
6.8	0.01\\
6.81	0.01\\
6.82	0.01\\
6.83	0.01\\
6.84	0.01\\
6.85	0.01\\
6.86	0.01\\
6.87	0.01\\
6.88	0.01\\
6.89	0.01\\
6.9	0.01\\
6.91	0.01\\
6.92	0.01\\
6.93	0.01\\
6.94	0.01\\
6.95	0.01\\
6.96	0.01\\
6.97	0.01\\
6.98	0.01\\
6.99	0.01\\
7	0.01\\
7.01	0.01\\
7.02	0.01\\
7.03	0.01\\
7.04	0.01\\
7.05	0.01\\
7.06	0.01\\
7.07	0.01\\
7.08	0.01\\
7.09	0.01\\
7.1	0.01\\
7.11	0.01\\
7.12	0.01\\
7.13	0.01\\
7.14	0.01\\
7.15	0.01\\
7.16	0.01\\
7.17	0.01\\
7.18	0.01\\
7.19	0.01\\
7.2	0.01\\
7.21	0.01\\
7.22	0.01\\
7.23	0.01\\
7.24	0.01\\
7.25	0.01\\
7.26	0.01\\
7.27	0.01\\
7.28	0.01\\
7.29	0.01\\
7.3	0.01\\
7.31	0.01\\
7.32	0.01\\
7.33	0.01\\
7.34	0.01\\
7.35	0.01\\
7.36	0.01\\
7.37	0.01\\
7.38	0.01\\
7.39	0.01\\
7.4	0.01\\
7.41	0.01\\
7.42	0.01\\
7.43	0.01\\
7.44	0.01\\
7.45	0.01\\
7.46	0.01\\
7.47	0.01\\
7.48	0.01\\
7.49	0.01\\
7.5	0.01\\
7.51	0.01\\
7.52	0.01\\
7.53	0.01\\
7.54	0.01\\
7.55	0.01\\
7.56	0.01\\
7.57	0.01\\
7.58	0.01\\
7.59	0.01\\
7.6	0.01\\
7.61	0.01\\
7.62	0.01\\
7.63	0.01\\
7.64	0.01\\
7.65	0.01\\
7.66	0.01\\
7.67	0.01\\
7.68	0.01\\
7.69	0.01\\
7.7	0.01\\
7.71	0.01\\
7.72	0.01\\
7.73	0.01\\
7.74	0.01\\
7.75	0.01\\
7.76	0.01\\
7.77	0.01\\
7.78	0.01\\
7.79	0.01\\
7.8	0.01\\
7.81	0.01\\
7.82	0.01\\
7.83	0.01\\
7.84	0.01\\
7.85	0.01\\
7.86	0.01\\
7.87	0.01\\
7.88	0.01\\
7.89	0.01\\
7.9	0.01\\
7.91	0.01\\
7.92	0.01\\
7.93	0.01\\
7.94	0.01\\
7.95	0.01\\
7.96	0.01\\
7.97	0.01\\
7.98	0.01\\
7.99	0.01\\
8	0.01\\
8.01	0.01\\
8.02	0.01\\
8.03	0.01\\
8.04	0.01\\
8.05	0.01\\
8.06	0.01\\
8.07	0.01\\
8.08	0.01\\
8.09	0.01\\
8.1	0.01\\
8.11	0.01\\
8.12	0.01\\
8.13	0.01\\
8.14	0.01\\
8.15	0.01\\
8.16	0.01\\
8.17	0.01\\
8.18	0.01\\
8.19	0.01\\
8.2	0.01\\
8.21	0.01\\
8.22	0.01\\
8.23	0.01\\
8.24	0.01\\
8.25	0.01\\
8.26	0.01\\
8.27	0.01\\
8.28	0.01\\
8.29	0.01\\
8.3	0.01\\
8.31	0.01\\
8.32	0.01\\
8.33	0.01\\
8.34	0.01\\
8.35	0.01\\
8.36	0.01\\
8.37	0.01\\
8.38	0.01\\
8.39	0.01\\
8.4	0.01\\
8.41	0.01\\
8.42	0.01\\
8.43	0.01\\
8.44	0.01\\
8.45	0.01\\
8.46	0.01\\
8.47	0.01\\
8.48	0.01\\
8.49	0.01\\
8.5	0.01\\
8.51	0.01\\
8.52	0.01\\
8.53	0.01\\
8.54	0.01\\
8.55	0.01\\
8.56	0.01\\
8.57	0.01\\
8.58	0.01\\
8.59	0.01\\
8.6	0.01\\
8.61	0.01\\
8.62	0.01\\
8.63	0.01\\
8.64	0.01\\
8.65	0.01\\
8.66	0.01\\
8.67	0.01\\
8.68	0.01\\
8.69	0.01\\
8.7	0.01\\
8.71	0.01\\
8.72	0.01\\
8.73	0.01\\
8.74	0.01\\
8.75	0.01\\
8.76	0.01\\
8.77	0.01\\
8.78	0.01\\
8.79	0.01\\
8.8	0.01\\
8.81	0.01\\
8.82	0.01\\
8.83	0.01\\
8.84	0.01\\
8.85	0.01\\
8.86	0.01\\
8.87	0.01\\
8.88	0.01\\
8.89	0.01\\
8.9	0.01\\
8.91	0.01\\
8.92	0.01\\
8.93	0.01\\
8.94	0.01\\
8.95	0.01\\
8.96	0.01\\
8.97	0.01\\
8.98	0.01\\
8.99	0.01\\
9	0.01\\
9.01	0.01\\
9.02	0.01\\
9.03	0.01\\
9.04	0.01\\
9.05	0.01\\
9.06	0.01\\
9.07	0.01\\
9.08	0.01\\
9.09	0.01\\
9.1	0.01\\
9.11	0.01\\
9.12	0.01\\
9.13	0.01\\
9.14	0.01\\
9.15	0.01\\
9.16	0.01\\
9.17	0.01\\
9.18	0.01\\
9.19	0.01\\
9.2	0.01\\
9.21	0.01\\
9.22	0.01\\
9.23	0.01\\
9.24	0.01\\
9.25	0.01\\
9.26	0.01\\
9.27	0.01\\
9.28	0.01\\
9.29	0.01\\
9.3	0.01\\
9.31	0.01\\
9.32	0.01\\
9.33	0.01\\
9.34	0.01\\
9.35	0.01\\
9.36	0.01\\
9.37	0.01\\
9.38	0.01\\
9.39	0.01\\
9.4	0.01\\
9.41	0.01\\
9.42	0.01\\
9.43	0.01\\
9.44	0.01\\
9.45	0.01\\
9.46	0.01\\
9.47	0.01\\
9.48	0.01\\
9.49	0.01\\
9.5	0.01\\
9.51	0.01\\
9.52	0.01\\
9.53	0.01\\
9.54	0.01\\
9.55	0.01\\
9.56	0.01\\
9.57	0.01\\
9.58	0.01\\
9.59	0.01\\
9.6	0.01\\
9.61	0.01\\
9.62	0.01\\
9.63	0.01\\
9.64	0.01\\
9.65	0.01\\
9.66	0.01\\
9.67	0.01\\
9.68	0.01\\
9.69	0.01\\
9.7	0.01\\
9.71	0.01\\
9.72	0.01\\
9.73	0.01\\
9.74	0.01\\
9.75	0.01\\
9.76	0.01\\
9.77	0.01\\
9.78	0.01\\
9.79	0.01\\
9.8	0.01\\
9.81	0.01\\
9.82	0.01\\
9.83	0.01\\
9.84	0.01\\
9.85	0.01\\
9.86	0.01\\
9.87	0.01\\
9.88	0.01\\
9.89	0.01\\
9.9	0.01\\
9.91	0.01\\
9.92	0.01\\
9.93	0.01\\
9.94	0.01\\
9.95	0.01\\
9.96	0.01\\
9.97	0.01\\
9.98	0.01\\
9.99	0.01\\
10	0.01\\
10.01	0.01\\
10.02	0.01\\
10.03	0.01\\
10.04	0.01\\
10.05	0.01\\
10.06	0.01\\
10.07	0.01\\
10.08	0.01\\
10.09	0.01\\
10.1	0.01\\
10.11	0.01\\
10.12	0.01\\
10.13	0.01\\
10.14	0.01\\
10.15	0.01\\
10.16	0.01\\
10.17	0.01\\
10.18	0.01\\
10.19	0.01\\
10.2	0.01\\
10.21	0.01\\
10.22	0.01\\
10.23	0.01\\
10.24	0.01\\
10.25	0.01\\
10.26	0.01\\
10.27	0.01\\
10.28	0.01\\
10.29	0.01\\
10.3	0.01\\
10.31	0.01\\
10.32	0.01\\
10.33	0.01\\
10.34	0.01\\
10.35	0.01\\
10.36	0.01\\
10.37	0.01\\
10.38	0.01\\
10.39	0.01\\
10.4	0.01\\
10.41	0.01\\
10.42	0.01\\
10.43	0.01\\
10.44	0.01\\
10.45	0.01\\
10.46	0.01\\
10.47	0.01\\
10.48	0.01\\
10.49	0.01\\
10.5	0.01\\
10.51	0.01\\
10.52	0.01\\
10.53	0.01\\
10.54	0.01\\
10.55	0.01\\
10.56	0.01\\
10.57	0.01\\
10.58	0.01\\
10.59	0.01\\
10.6	0.01\\
10.61	0.01\\
10.62	0.01\\
10.63	0.01\\
10.64	0.01\\
10.65	0.01\\
10.66	0.01\\
10.67	0.01\\
10.68	0.01\\
10.69	0.01\\
10.7	0.01\\
10.71	0.01\\
10.72	0.01\\
10.73	0.01\\
10.74	0.01\\
10.75	0.01\\
10.76	0.01\\
10.77	0.01\\
10.78	0.01\\
10.79	0.01\\
10.8	0.01\\
10.81	0.01\\
10.82	0.01\\
10.83	0.01\\
10.84	0.01\\
10.85	0.01\\
10.86	0.01\\
10.87	0.01\\
10.88	0.01\\
10.89	0.01\\
10.9	0.01\\
10.91	0.01\\
10.92	0.01\\
10.93	0.01\\
10.94	0.01\\
10.95	0.01\\
10.96	0.01\\
10.97	0.01\\
10.98	0.01\\
10.99	0.01\\
11	0.01\\
11.01	0.01\\
11.02	0.01\\
11.03	0.01\\
11.04	0.01\\
11.05	0.01\\
11.06	0.01\\
11.07	0.01\\
11.08	0.01\\
11.09	0.01\\
11.1	0.01\\
11.11	0.01\\
11.12	0.01\\
11.13	0.01\\
11.14	0.01\\
11.15	0.01\\
11.16	0.01\\
11.17	0.01\\
11.18	0.01\\
11.19	0.01\\
11.2	0.01\\
11.21	0.01\\
11.22	0.01\\
11.23	0.01\\
11.24	0.01\\
11.25	0.01\\
11.26	0.01\\
11.27	0.01\\
11.28	0.01\\
11.29	0.01\\
11.3	0.01\\
11.31	0.01\\
11.32	0.01\\
11.33	0.01\\
11.34	0.01\\
11.35	0.01\\
11.36	0.01\\
11.37	0.01\\
11.38	0.01\\
11.39	0.01\\
11.4	0.01\\
11.41	0.01\\
11.42	0.01\\
11.43	0.01\\
11.44	0.01\\
11.45	0.01\\
11.46	0.01\\
11.47	0.01\\
11.48	0.01\\
11.49	0.01\\
11.5	0.01\\
11.51	0.01\\
11.52	0.01\\
11.53	0.01\\
11.54	0.01\\
11.55	0.01\\
11.56	0.01\\
11.57	0.01\\
11.58	0.01\\
11.59	0.01\\
11.6	0.01\\
11.61	0.01\\
11.62	0.01\\
11.63	0.01\\
11.64	0.01\\
11.65	0.01\\
11.66	0.01\\
11.67	0.01\\
11.68	0.01\\
11.69	0.01\\
11.7	0.01\\
11.71	0.01\\
11.72	0.01\\
11.73	0.01\\
11.74	0.01\\
11.75	0.01\\
11.76	0.01\\
11.77	0.01\\
11.78	0.01\\
11.79	0.01\\
11.8	0.01\\
11.81	0.01\\
11.82	0.01\\
11.83	0.01\\
11.84	0.01\\
11.85	0.01\\
11.86	0.01\\
11.87	0.01\\
11.88	0.01\\
11.89	0.01\\
11.9	0.01\\
11.91	0.01\\
11.92	0.01\\
11.93	0.01\\
11.94	0.01\\
11.95	0.01\\
11.96	0.01\\
11.97	0.01\\
11.98	0.01\\
11.99	0.01\\
12	0.01\\
12.01	0.01\\
12.02	0.01\\
12.03	0.01\\
12.04	0.01\\
12.05	0.01\\
12.06	0.01\\
12.07	0.01\\
12.08	0.01\\
12.09	0.01\\
12.1	0.01\\
12.11	0.01\\
12.12	0.01\\
12.13	0.01\\
12.14	0.01\\
12.15	0.01\\
12.16	0.01\\
12.17	0.01\\
12.18	0.01\\
12.19	0.01\\
12.2	0.01\\
12.21	0.01\\
12.22	0.01\\
12.23	0.01\\
12.24	0.01\\
12.25	0.01\\
12.26	0.01\\
12.27	0.01\\
12.28	0.01\\
12.29	0.01\\
12.3	0.01\\
12.31	0.01\\
12.32	0.01\\
12.33	0.01\\
12.34	0.01\\
12.35	0.01\\
12.36	0.01\\
12.37	0.01\\
12.38	0.01\\
12.39	0.01\\
12.4	0.01\\
12.41	0.01\\
12.42	0.01\\
12.43	0.01\\
12.44	0.01\\
12.45	0.01\\
12.46	0.01\\
12.47	0.01\\
12.48	0.01\\
12.49	0.01\\
12.5	0.01\\
12.51	0.01\\
12.52	0.01\\
12.53	0.01\\
12.54	0.01\\
12.55	0.01\\
12.56	0.01\\
12.57	0.01\\
12.58	0.01\\
12.59	0.01\\
12.6	0.01\\
12.61	0.01\\
12.62	0.01\\
12.63	0.01\\
12.64	0.01\\
12.65	0.01\\
12.66	0.01\\
12.67	0.01\\
12.68	0.01\\
12.69	0.01\\
12.7	0.01\\
12.71	0.01\\
12.72	0.01\\
12.73	0.01\\
12.74	0.01\\
12.75	0.01\\
12.76	0.01\\
12.77	0.01\\
12.78	0.01\\
12.79	0.01\\
12.8	0.01\\
12.81	0.01\\
12.82	0.01\\
12.83	0.01\\
12.84	0.01\\
12.85	0.01\\
12.86	0.01\\
12.87	0.01\\
12.88	0.01\\
12.89	0.01\\
12.9	0.01\\
12.91	0.01\\
12.92	0.01\\
12.93	0.01\\
12.94	0.01\\
12.95	0.01\\
12.96	0.01\\
12.97	0.01\\
12.98	0.01\\
12.99	0.01\\
13	0.01\\
13.01	0.01\\
13.02	0.01\\
13.03	0.01\\
13.04	0.01\\
13.05	0.01\\
13.06	0.01\\
13.07	0.01\\
13.08	0.01\\
13.09	0.01\\
13.1	0.01\\
13.11	0.01\\
13.12	0.01\\
13.13	0.01\\
13.14	0.01\\
13.15	0.01\\
13.16	0.01\\
13.17	0.01\\
13.18	0.01\\
13.19	0.01\\
13.2	0.01\\
13.21	0.01\\
13.22	0.01\\
13.23	0.01\\
13.24	0.01\\
13.25	0.01\\
13.26	0.01\\
13.27	0.01\\
13.28	0.01\\
13.29	0.01\\
13.3	0.01\\
13.31	0.01\\
13.32	0.01\\
13.33	0.01\\
13.34	0.01\\
13.35	0.01\\
13.36	0.01\\
13.37	0.01\\
13.38	0.01\\
13.39	0.01\\
13.4	0.01\\
13.41	0.01\\
13.42	0.01\\
13.43	0.01\\
13.44	0.01\\
13.45	0.01\\
13.46	0.01\\
13.47	0.01\\
13.48	0.01\\
13.49	0.01\\
13.5	0.01\\
13.51	0.01\\
13.52	0.01\\
13.53	0.01\\
13.54	0.01\\
13.55	0.01\\
13.56	0.01\\
13.57	0.01\\
13.58	0.01\\
13.59	0.01\\
13.6	0.01\\
13.61	0.01\\
13.62	0.01\\
13.63	0.01\\
13.64	0.01\\
13.65	0.01\\
13.66	0.01\\
13.67	0.01\\
13.68	0.01\\
13.69	0.01\\
13.7	0.01\\
13.71	0.01\\
13.72	0.01\\
13.73	0.01\\
13.74	0.01\\
13.75	0.01\\
13.76	0.01\\
13.77	0.01\\
13.78	0.01\\
13.79	0.01\\
13.8	0.01\\
13.81	0.01\\
13.82	0.01\\
13.83	0.01\\
13.84	0.01\\
13.85	0.01\\
13.86	0.01\\
13.87	0.01\\
13.88	0.01\\
13.89	0.01\\
13.9	0.01\\
13.91	0.01\\
13.92	0.01\\
13.93	0.01\\
13.94	0.01\\
13.95	0.01\\
13.96	0.01\\
13.97	0.01\\
13.98	0.01\\
13.99	0.01\\
14	0.01\\
14.01	0.01\\
14.02	0.01\\
14.03	0.01\\
14.04	0.01\\
14.05	0.01\\
14.06	0.01\\
14.07	0.01\\
14.08	0.01\\
14.09	0.01\\
14.1	0.01\\
14.11	0.01\\
14.12	0.01\\
14.13	0.01\\
14.14	0.01\\
14.15	0.01\\
14.16	0.01\\
14.17	0.01\\
14.18	0.01\\
14.19	0.01\\
14.2	0.01\\
14.21	0.01\\
14.22	0.01\\
14.23	0.01\\
14.24	0.01\\
14.25	0.01\\
14.26	0.01\\
14.27	0.01\\
14.28	0.01\\
14.29	0.01\\
14.3	0.01\\
14.31	0.01\\
14.32	0.01\\
14.33	0.01\\
14.34	0.01\\
14.35	0.01\\
14.36	0.01\\
14.37	0.01\\
14.38	0.01\\
14.39	0.01\\
14.4	0.01\\
14.41	0.01\\
14.42	0.01\\
14.43	0.01\\
14.44	0.01\\
14.45	0.01\\
14.46	0.01\\
14.47	0.01\\
14.48	0.01\\
14.49	0.01\\
14.5	0.01\\
14.51	0.01\\
14.52	0.01\\
14.53	0.01\\
14.54	0.01\\
14.55	0.01\\
14.56	0.01\\
14.57	0.01\\
14.58	0.01\\
14.59	0.01\\
14.6	0.01\\
14.61	0.01\\
14.62	0.01\\
14.63	0.01\\
14.64	0.01\\
14.65	0.01\\
14.66	0.01\\
14.67	0.01\\
14.68	0.01\\
14.69	0.01\\
14.7	0.01\\
14.71	0.01\\
14.72	0.01\\
14.73	0.01\\
14.74	0.01\\
14.75	0.01\\
14.76	0.01\\
14.77	0.01\\
14.78	0.01\\
14.79	0.01\\
14.8	0.01\\
14.81	0.01\\
14.82	0.01\\
14.83	0.01\\
14.84	0.01\\
14.85	0.01\\
14.86	0.01\\
14.87	0.01\\
14.88	0.01\\
14.89	0.01\\
14.9	0.01\\
14.91	0.01\\
14.92	0.01\\
14.93	0.01\\
14.94	0.01\\
14.95	0.01\\
14.96	0.01\\
14.97	0.01\\
14.98	0.01\\
14.99	0.01\\
15	0.01\\
15.01	0.01\\
15.02	0.01\\
15.03	0.01\\
15.04	0.01\\
15.05	0.01\\
15.06	0.01\\
15.07	0.01\\
15.08	0.01\\
15.09	0.01\\
15.1	0.01\\
15.11	0.01\\
15.12	0.01\\
15.13	0.01\\
15.14	0.01\\
15.15	0.01\\
15.16	0.01\\
15.17	0.01\\
15.18	0.01\\
15.19	0.01\\
15.2	0.01\\
15.21	0.01\\
15.22	0.01\\
15.23	0.01\\
15.24	0.01\\
15.25	0.01\\
15.26	0.01\\
15.27	0.01\\
15.28	0.01\\
15.29	0.01\\
15.3	0.01\\
15.31	0.01\\
15.32	0.01\\
15.33	0.01\\
15.34	0.01\\
15.35	0.01\\
15.36	0.01\\
15.37	0.01\\
15.38	0.01\\
15.39	0.01\\
15.4	0.01\\
15.41	0.01\\
15.42	0.01\\
15.43	0.01\\
15.44	0.01\\
15.45	0.01\\
15.46	0.01\\
15.47	0.01\\
15.48	0.01\\
15.49	0.01\\
15.5	0.01\\
15.51	0.01\\
15.52	0.01\\
15.53	0.01\\
15.54	0.01\\
15.55	0.01\\
15.56	0.01\\
15.57	0.01\\
15.58	0.01\\
15.59	0.01\\
15.6	0.01\\
15.61	0.01\\
15.62	0.01\\
15.63	0.01\\
15.64	0.01\\
15.65	0.01\\
15.66	0.01\\
15.67	0.01\\
15.68	0.01\\
15.69	0.01\\
15.7	0.01\\
15.71	0.01\\
15.72	0.01\\
15.73	0.01\\
15.74	0.01\\
15.75	0.01\\
15.76	0.01\\
15.77	0.01\\
15.78	0.01\\
15.79	0.01\\
15.8	0.01\\
15.81	0.01\\
15.82	0.01\\
15.83	0.01\\
15.84	0.01\\
15.85	0.01\\
15.86	0.01\\
15.87	0.01\\
15.88	0.01\\
15.89	0.01\\
15.9	0.01\\
15.91	0.01\\
15.92	0.01\\
15.93	0.01\\
15.94	0.01\\
15.95	0.01\\
15.96	0.01\\
15.97	0.01\\
15.98	0.01\\
15.99	0.01\\
16	0.01\\
16.01	0.01\\
16.02	0.01\\
16.03	0.01\\
16.04	0.01\\
16.05	0.01\\
16.06	0.01\\
16.07	0.01\\
16.08	0.01\\
16.09	0.01\\
16.1	0.01\\
16.11	0.01\\
16.12	0.01\\
16.13	0.01\\
16.14	0.01\\
16.15	0.01\\
16.16	0.01\\
16.17	0.01\\
16.18	0.01\\
16.19	0.01\\
16.2	0.01\\
16.21	0.01\\
16.22	0.01\\
16.23	0.01\\
16.24	0.01\\
16.25	0.01\\
16.26	0.01\\
16.27	0.01\\
16.28	0.01\\
16.29	0.01\\
16.3	0.01\\
16.31	0.01\\
16.32	0.01\\
16.33	0.01\\
16.34	0.01\\
16.35	0.01\\
16.36	0.01\\
16.37	0.01\\
16.38	0.01\\
16.39	0.01\\
16.4	0.01\\
16.41	0.01\\
16.42	0.01\\
16.43	0.01\\
16.44	0.01\\
16.45	0.01\\
16.46	0.01\\
16.47	0.01\\
16.48	0.01\\
16.49	0.01\\
16.5	0.01\\
16.51	0.01\\
16.52	0.01\\
16.53	0.01\\
16.54	0.01\\
16.55	0.01\\
16.56	0.01\\
16.57	0.01\\
16.58	0.01\\
16.59	0.01\\
16.6	0.01\\
16.61	0.01\\
16.62	0.01\\
16.63	0.01\\
16.64	0.01\\
16.65	0.01\\
16.66	0.01\\
16.67	0.01\\
16.68	0.01\\
16.69	0.01\\
16.7	0.01\\
16.71	0.01\\
16.72	0.01\\
16.73	0.01\\
16.74	0.01\\
16.75	0.01\\
16.76	0.01\\
16.77	0.01\\
16.78	0.01\\
16.79	0.01\\
16.8	0.01\\
16.81	0.01\\
16.82	0.01\\
16.83	0.01\\
16.84	0.01\\
16.85	0.01\\
16.86	0.01\\
16.87	0.01\\
16.88	0.01\\
16.89	0.01\\
16.9	0.01\\
16.91	0.01\\
16.92	0.01\\
16.93	0.01\\
16.94	0.01\\
16.95	0.01\\
16.96	0.01\\
16.97	0.01\\
16.98	0.01\\
16.99	0.01\\
17	0.01\\
17.01	0.01\\
17.02	0.01\\
17.03	0.01\\
17.04	0.01\\
17.05	0.01\\
17.06	0.01\\
17.07	0.01\\
17.08	0.01\\
17.09	0.01\\
17.1	0.01\\
17.11	0.01\\
17.12	0.01\\
17.13	0.01\\
17.14	0.01\\
17.15	0.01\\
17.16	0.01\\
17.17	0.01\\
17.18	0.01\\
17.19	0.01\\
17.2	0.01\\
17.21	0.01\\
17.22	0.01\\
17.23	0.01\\
17.24	0.01\\
17.25	0.01\\
17.26	0.01\\
17.27	0.01\\
17.28	0.01\\
17.29	0.01\\
17.3	0.01\\
17.31	0.01\\
17.32	0.01\\
17.33	0.01\\
17.34	0.01\\
17.35	0.01\\
17.36	0.01\\
17.37	0.01\\
17.38	0.01\\
17.39	0.01\\
17.4	0.01\\
17.41	0.01\\
17.42	0.01\\
17.43	0.01\\
17.44	0.01\\
17.45	0.01\\
17.46	0.01\\
17.47	0.01\\
17.48	0.01\\
17.49	0.01\\
17.5	0.01\\
17.51	0.01\\
17.52	0.01\\
17.53	0.01\\
17.54	0.01\\
17.55	0.01\\
17.56	0.01\\
17.57	0.01\\
17.58	0.01\\
17.59	0.01\\
17.6	0.01\\
17.61	0.01\\
17.62	0.01\\
17.63	0.01\\
17.64	0.01\\
17.65	0.01\\
17.66	0.01\\
17.67	0.01\\
17.68	0.01\\
17.69	0.01\\
17.7	0.01\\
17.71	0.01\\
17.72	0.01\\
17.73	0.01\\
17.74	0.01\\
17.75	0.01\\
17.76	0.01\\
17.77	0.01\\
17.78	0.01\\
17.79	0.01\\
17.8	0.01\\
17.81	0.01\\
17.82	0.01\\
17.83	0.01\\
17.84	0.01\\
17.85	0.01\\
17.86	0.01\\
17.87	0.01\\
17.88	0.01\\
17.89	0.01\\
17.9	0.01\\
17.91	0.01\\
17.92	0.01\\
17.93	0.01\\
17.94	0.01\\
17.95	0.01\\
17.96	0.01\\
17.97	0.01\\
17.98	0.01\\
17.99	0.01\\
18	0.01\\
18.01	0.01\\
18.02	0.01\\
18.03	0.01\\
18.04	0.01\\
18.05	0.01\\
18.06	0.01\\
18.07	0.01\\
18.08	0.01\\
18.09	0.01\\
18.1	0.01\\
18.11	0.01\\
18.12	0.01\\
18.13	0.01\\
18.14	0.01\\
18.15	0.01\\
18.16	0.01\\
18.17	0.01\\
18.18	0.01\\
18.19	0.01\\
18.2	0.01\\
18.21	0.01\\
18.22	0.01\\
18.23	0.01\\
18.24	0.01\\
18.25	0.01\\
18.26	0.01\\
18.27	0.01\\
18.28	0.01\\
18.29	0.01\\
18.3	0.01\\
18.31	0.01\\
18.32	0.01\\
18.33	0.01\\
18.34	0.01\\
18.35	0.01\\
18.36	0.01\\
18.37	0.01\\
18.38	0.01\\
18.39	0.01\\
18.4	0.01\\
18.41	0.01\\
18.42	0.01\\
18.43	0.01\\
18.44	0.01\\
18.45	0.01\\
18.46	0.01\\
18.47	0.01\\
18.48	0.01\\
18.49	0.01\\
18.5	0.01\\
18.51	0.01\\
18.52	0.01\\
18.53	0.01\\
18.54	0.01\\
18.55	0.01\\
18.56	0.01\\
18.57	0.01\\
18.58	0.01\\
18.59	0.01\\
18.6	0.01\\
18.61	0.01\\
18.62	0.01\\
18.63	0.01\\
18.64	0.01\\
18.65	0.01\\
18.66	0.01\\
18.67	0.01\\
18.68	0.01\\
18.69	0.01\\
18.7	0.01\\
18.71	0.01\\
18.72	0.01\\
18.73	0.01\\
18.74	0.01\\
18.75	0.01\\
18.76	0.01\\
18.77	0.01\\
18.78	0.01\\
18.79	0.01\\
18.8	0.01\\
18.81	0.01\\
18.82	0.01\\
18.83	0.01\\
18.84	0.01\\
18.85	0.01\\
18.86	0.01\\
18.87	0.01\\
18.88	0.01\\
18.89	0.01\\
18.9	0.01\\
18.91	0.01\\
18.92	0.01\\
18.93	0.01\\
18.94	0.01\\
18.95	0.01\\
18.96	0.01\\
18.97	0.01\\
18.98	0.01\\
18.99	0.01\\
19	0.01\\
19.01	0.01\\
19.02	0.01\\
19.03	0.01\\
19.04	0.01\\
19.05	0.01\\
19.06	0.01\\
19.07	0.01\\
19.08	0.01\\
19.09	0.01\\
19.1	0.01\\
19.11	0.01\\
19.12	0.01\\
19.13	0.01\\
19.14	0.01\\
19.15	0.01\\
19.16	0.01\\
19.17	0.01\\
19.18	0.01\\
19.19	0.01\\
19.2	0.01\\
19.21	0.01\\
19.22	0.01\\
19.23	0.01\\
19.24	0.01\\
19.25	0.01\\
19.26	0.01\\
19.27	0.01\\
19.28	0.01\\
19.29	0.01\\
19.3	0.01\\
19.31	0.01\\
19.32	0.01\\
19.33	0.01\\
19.34	0.01\\
19.35	0.01\\
19.36	0.01\\
19.37	0.01\\
19.38	0.01\\
19.39	0.01\\
19.4	0.01\\
19.41	0.01\\
19.42	0.01\\
19.43	0.01\\
19.44	0.01\\
19.45	0.01\\
19.46	0.01\\
19.47	0.01\\
19.48	0.01\\
19.49	0.01\\
19.5	0.01\\
19.51	0.01\\
19.52	0.01\\
19.53	0.01\\
19.54	0.01\\
19.55	0.01\\
19.56	0.01\\
19.57	0.01\\
19.58	0.01\\
19.59	0.01\\
19.6	0.01\\
19.61	0.01\\
19.62	0.01\\
19.63	0.01\\
19.64	0.01\\
19.65	0.01\\
19.66	0.01\\
19.67	0.01\\
19.68	0.01\\
19.69	0.01\\
19.7	0.01\\
19.71	0.01\\
19.72	0.01\\
19.73	0.01\\
19.74	0.01\\
19.75	0.01\\
19.76	0.01\\
19.77	0.01\\
19.78	0.01\\
19.79	0.01\\
19.8	0.01\\
19.81	0.01\\
19.82	0.01\\
19.83	0.01\\
19.84	0.01\\
19.85	0.01\\
19.86	0.01\\
19.87	0.01\\
19.88	0.01\\
19.89	0.01\\
19.9	0.01\\
19.91	0.01\\
19.92	0.01\\
19.93	0.01\\
19.94	0.01\\
19.95	0.01\\
19.96	0.01\\
19.97	0.01\\
19.98	0.01\\
19.99	0.01\\
20	0.01\\
20.01	0.01\\
20.02	0.01\\
20.03	0.01\\
20.04	0.01\\
20.05	0.01\\
20.06	0.01\\
20.07	0.01\\
20.08	0.01\\
20.09	0.01\\
20.1	0.01\\
20.11	0.01\\
20.12	0.01\\
20.13	0.01\\
20.14	0.01\\
20.15	0.01\\
20.16	0.01\\
20.17	0.01\\
20.18	0.01\\
20.19	0.01\\
20.2	0.01\\
20.21	0.01\\
20.22	0.01\\
20.23	0.01\\
20.24	0.01\\
20.25	0.01\\
20.26	0.01\\
20.27	0.01\\
20.28	0.01\\
20.29	0.01\\
20.3	0.01\\
20.31	0.01\\
20.32	0.01\\
20.33	0.01\\
20.34	0.01\\
20.35	0.01\\
20.36	0.01\\
20.37	0.01\\
20.38	0.01\\
20.39	0.01\\
20.4	0.01\\
20.41	0.01\\
20.42	0.01\\
20.43	0.01\\
20.44	0.01\\
20.45	0.01\\
20.46	0.01\\
20.47	0.01\\
20.48	0.01\\
20.49	0.01\\
20.5	0.01\\
20.51	0.01\\
20.52	0.01\\
20.53	0.01\\
20.54	0.01\\
20.55	0.01\\
20.56	0.01\\
20.57	0.01\\
20.58	0.01\\
20.59	0.01\\
20.6	0.01\\
20.61	0.01\\
20.62	0.01\\
20.63	0.01\\
20.64	0.01\\
20.65	0.01\\
20.66	0.01\\
20.67	0.01\\
20.68	0.01\\
20.69	0.01\\
20.7	0.01\\
20.71	0.01\\
20.72	0.01\\
20.73	0.01\\
20.74	0.01\\
20.75	0.01\\
20.76	0.01\\
20.77	0.01\\
20.78	0.01\\
20.79	0.01\\
20.8	0.01\\
20.81	0.01\\
20.82	0.01\\
20.83	0.01\\
20.84	0.01\\
20.85	0.01\\
20.86	0.01\\
20.87	0.01\\
20.88	0.01\\
20.89	0.01\\
20.9	0.01\\
20.91	0.01\\
20.92	0.01\\
20.93	0.01\\
20.94	0.01\\
20.95	0.01\\
20.96	0.01\\
20.97	0.01\\
20.98	0.01\\
20.99	0.01\\
21	0.01\\
21.01	0.01\\
21.02	0.01\\
21.03	0.01\\
21.04	0.01\\
21.05	0.01\\
21.06	0.01\\
21.07	0.01\\
21.08	0.01\\
21.09	0.01\\
21.1	0.01\\
21.11	0.01\\
21.12	0.01\\
21.13	0.01\\
21.14	0.01\\
21.15	0.01\\
21.16	0.01\\
21.17	0.01\\
21.18	0.01\\
21.19	0.01\\
21.2	0.01\\
21.21	0.01\\
21.22	0.01\\
21.23	0.01\\
21.24	0.01\\
21.25	0.01\\
21.26	0.01\\
21.27	0.01\\
21.28	0.01\\
21.29	0.01\\
21.3	0.01\\
21.31	0.01\\
21.32	0.01\\
21.33	0.01\\
21.34	0.01\\
21.35	0.01\\
21.36	0.01\\
21.37	0.01\\
21.38	0.01\\
21.39	0.01\\
21.4	0.01\\
21.41	0.01\\
21.42	0.01\\
21.43	0.01\\
21.44	0.01\\
21.45	0.01\\
21.46	0.01\\
21.47	0.01\\
21.48	0.01\\
21.49	0.01\\
21.5	0.01\\
21.51	0.01\\
21.52	0.01\\
21.53	0.01\\
21.54	0.01\\
21.55	0.01\\
21.56	0.01\\
21.57	0.01\\
21.58	0.01\\
21.59	0.01\\
21.6	0.01\\
21.61	0.01\\
21.62	0.01\\
21.63	0.01\\
21.64	0.01\\
21.65	0.01\\
21.66	0.01\\
21.67	0.01\\
21.68	0.01\\
21.69	0.01\\
21.7	0.01\\
21.71	0.01\\
21.72	0.01\\
21.73	0.01\\
21.74	0.01\\
21.75	0.01\\
21.76	0.01\\
21.77	0.01\\
21.78	0.01\\
21.79	0.01\\
21.8	0.01\\
21.81	0.01\\
21.82	0.01\\
21.83	0.01\\
21.84	0.01\\
21.85	0.01\\
21.86	0.01\\
21.87	0.01\\
21.88	0.01\\
21.89	0.01\\
21.9	0.01\\
21.91	0.01\\
21.92	0.01\\
21.93	0.01\\
21.94	0.01\\
21.95	0.01\\
21.96	0.01\\
21.97	0.01\\
21.98	0.01\\
21.99	0.01\\
22	0.01\\
22.01	0.01\\
22.02	0.01\\
22.03	0.01\\
22.04	0.01\\
22.05	0.01\\
22.06	0.01\\
22.07	0.01\\
22.08	0.01\\
22.09	0.01\\
22.1	0.01\\
22.11	0.01\\
22.12	0.01\\
22.13	0.01\\
22.14	0.01\\
22.15	0.01\\
22.16	0.01\\
22.17	0.01\\
22.18	0.01\\
22.19	0.01\\
22.2	0.01\\
22.21	0.01\\
22.22	0.01\\
22.23	0.01\\
22.24	0.01\\
22.25	0.01\\
22.26	0.01\\
22.27	0.01\\
22.28	0.01\\
22.29	0.01\\
22.3	0.01\\
22.31	0.01\\
22.32	0.01\\
22.33	0.01\\
22.34	0.01\\
22.35	0.01\\
22.36	0.01\\
22.37	0.01\\
22.38	0.01\\
22.39	0.01\\
22.4	0.01\\
22.41	0.01\\
22.42	0.01\\
22.43	0.01\\
22.44	0.01\\
22.45	0.01\\
22.46	0.01\\
22.47	0.01\\
22.48	0.01\\
22.49	0.01\\
22.5	0.01\\
22.51	0.01\\
22.52	0.01\\
22.53	0.01\\
22.54	0.01\\
22.55	0.01\\
22.56	0.01\\
22.57	0.01\\
22.58	0.01\\
22.59	0.01\\
22.6	0.01\\
22.61	0.01\\
22.62	0.01\\
22.63	0.01\\
22.64	0.01\\
22.65	0.01\\
22.66	0.01\\
22.67	0.01\\
22.68	0.01\\
22.69	0.01\\
22.7	0.01\\
22.71	0.01\\
22.72	0.01\\
22.73	0.01\\
22.74	0.01\\
22.75	0.01\\
22.76	0.01\\
22.77	0.01\\
22.78	0.01\\
22.79	0.01\\
22.8	0.01\\
22.81	0.01\\
22.82	0.01\\
22.83	0.01\\
22.84	0.01\\
22.85	0.01\\
22.86	0.01\\
22.87	0.01\\
22.88	0.01\\
22.89	0.01\\
22.9	0.01\\
22.91	0.01\\
22.92	0.01\\
22.93	0.01\\
22.94	0.01\\
22.95	0.01\\
22.96	0.01\\
22.97	0.01\\
22.98	0.01\\
22.99	0.01\\
23	0.01\\
23.01	0.01\\
23.02	0.01\\
23.03	0.01\\
23.04	0.01\\
23.05	0.01\\
23.06	0.01\\
23.07	0.01\\
23.08	0.01\\
23.09	0.01\\
23.1	0.01\\
23.11	0.01\\
23.12	0.01\\
23.13	0.01\\
23.14	0.01\\
23.15	0.01\\
23.16	0.01\\
23.17	0.01\\
23.18	0.01\\
23.19	0.01\\
23.2	0.01\\
23.21	0.01\\
23.22	0.01\\
23.23	0.01\\
23.24	0.01\\
23.25	0.01\\
23.26	0.01\\
23.27	0.01\\
23.28	0.01\\
23.29	0.01\\
23.3	0.01\\
23.31	0.01\\
23.32	0.01\\
23.33	0.01\\
23.34	0.01\\
23.35	0.01\\
23.36	0.01\\
23.37	0.01\\
23.38	0.01\\
23.39	0.01\\
23.4	0.01\\
23.41	0.01\\
23.42	0.01\\
23.43	0.01\\
23.44	0.01\\
23.45	0.01\\
23.46	0.01\\
23.47	0.01\\
23.48	0.01\\
23.49	0.01\\
23.5	0.01\\
23.51	0.01\\
23.52	0.01\\
23.53	0.01\\
23.54	0.01\\
23.55	0.01\\
23.56	0.01\\
23.57	0.01\\
23.58	0.01\\
23.59	0.01\\
23.6	0.01\\
23.61	0.01\\
23.62	0.01\\
23.63	0.01\\
23.64	0.01\\
23.65	0.01\\
23.66	0.01\\
23.67	0.01\\
23.68	0.01\\
23.69	0.01\\
23.7	0.01\\
23.71	0.01\\
23.72	0.01\\
23.73	0.01\\
23.74	0.01\\
23.75	0.01\\
23.76	0.01\\
23.77	0.01\\
23.78	0.01\\
23.79	0.01\\
23.8	0.01\\
23.81	0.01\\
23.82	0.01\\
23.83	0.01\\
23.84	0.01\\
23.85	0.01\\
23.86	0.01\\
23.87	0.01\\
23.88	0.01\\
23.89	0.01\\
23.9	0.01\\
23.91	0.01\\
23.92	0.01\\
23.93	0.01\\
23.94	0.01\\
23.95	0.01\\
23.96	0.01\\
23.97	0.01\\
23.98	0.01\\
23.99	0.01\\
24	0.01\\
24.01	0.01\\
24.02	0.01\\
24.03	0.01\\
24.04	0.01\\
24.05	0.01\\
24.06	0.01\\
24.07	0.01\\
24.08	0.01\\
24.09	0.01\\
24.1	0.01\\
24.11	0.01\\
24.12	0.01\\
24.13	0.01\\
24.14	0.01\\
24.15	0.01\\
24.16	0.01\\
24.17	0.01\\
24.18	0.01\\
24.19	0.01\\
24.2	0.01\\
24.21	0.01\\
24.22	0.01\\
24.23	0.01\\
24.24	0.01\\
24.25	0.01\\
24.26	0.01\\
24.27	0.01\\
24.28	0.01\\
24.29	0.01\\
24.3	0.01\\
24.31	0.01\\
24.32	0.01\\
24.33	0.01\\
24.34	0.01\\
24.35	0.01\\
24.36	0.01\\
24.37	0.01\\
24.38	0.01\\
24.39	0.01\\
24.4	0.01\\
24.41	0.01\\
24.42	0.01\\
24.43	0.01\\
24.44	0.01\\
24.45	0.01\\
24.46	0.01\\
24.47	0.01\\
24.48	0.01\\
24.49	0.01\\
24.5	0.01\\
24.51	0.01\\
24.52	0.01\\
24.53	0.01\\
24.54	0.01\\
24.55	0.01\\
24.56	0.01\\
24.57	0.01\\
24.58	0.01\\
24.59	0.01\\
24.6	0.01\\
24.61	0.01\\
24.62	0.01\\
24.63	0.01\\
24.64	0.01\\
24.65	0.01\\
24.66	0.01\\
24.67	0.01\\
24.68	0.01\\
24.69	0.01\\
24.7	0.01\\
24.71	0.01\\
24.72	0.01\\
24.73	0.01\\
24.74	0.01\\
24.75	0.01\\
24.76	0.01\\
24.77	0.01\\
24.78	0.01\\
24.79	0.01\\
24.8	0.01\\
24.81	0.01\\
24.82	0.01\\
24.83	0.01\\
24.84	0.01\\
24.85	0.01\\
24.86	0.01\\
24.87	0.01\\
24.88	0.01\\
24.89	0.01\\
24.9	0.01\\
24.91	0.01\\
24.92	0.01\\
24.93	0.01\\
24.94	0.01\\
24.95	0.01\\
24.96	0.01\\
24.97	0.01\\
24.98	0.01\\
24.99	0.01\\
25	0.01\\
25.01	0.01\\
25.02	0.01\\
25.03	0.01\\
25.04	0.01\\
25.05	0.01\\
25.06	0.01\\
25.07	0.01\\
25.08	0.01\\
25.09	0.01\\
25.1	0.01\\
25.11	0.01\\
25.12	0.01\\
25.13	0.01\\
25.14	0.01\\
25.15	0.01\\
25.16	0.01\\
25.17	0.01\\
25.18	0.01\\
25.19	0.01\\
25.2	0.01\\
25.21	0.01\\
25.22	0.01\\
25.23	0.01\\
25.24	0.01\\
25.25	0.01\\
25.26	0.01\\
25.27	0.01\\
25.28	0.01\\
25.29	0.01\\
25.3	0.01\\
25.31	0.01\\
25.32	0.01\\
25.33	0.01\\
25.34	0.01\\
25.35	0.01\\
25.36	0.01\\
25.37	0.01\\
25.38	0.01\\
25.39	0.01\\
25.4	0.01\\
25.41	0.01\\
25.42	0.01\\
25.43	0.01\\
25.44	0.01\\
25.45	0.01\\
25.46	0.01\\
25.47	0.01\\
25.48	0.01\\
25.49	0.01\\
25.5	0.01\\
25.51	0.01\\
25.52	0.01\\
25.53	0.01\\
25.54	0.01\\
25.55	0.01\\
25.56	0.01\\
25.57	0.01\\
25.58	0.01\\
25.59	0.01\\
25.6	0.01\\
25.61	0.01\\
25.62	0.01\\
25.63	0.01\\
25.64	0.01\\
25.65	0.01\\
25.66	0.01\\
25.67	0.01\\
25.68	0.01\\
25.69	0.01\\
25.7	0.01\\
25.71	0.01\\
25.72	0.01\\
25.73	0.01\\
25.74	0.01\\
25.75	0.01\\
25.76	0.01\\
25.77	0.01\\
25.78	0.01\\
25.79	0.01\\
25.8	0.01\\
25.81	0.01\\
25.82	0.01\\
25.83	0.01\\
25.84	0.01\\
25.85	0.01\\
25.86	0.01\\
25.87	0.01\\
25.88	0.01\\
25.89	0.01\\
25.9	0.01\\
25.91	0.01\\
25.92	0.01\\
25.93	0.01\\
25.94	0.01\\
25.95	0.01\\
25.96	0.01\\
25.97	0.01\\
25.98	0.01\\
25.99	0.01\\
26	0.01\\
26.01	0.01\\
26.02	0.01\\
26.03	0.01\\
26.04	0.01\\
26.05	0.01\\
26.06	0.01\\
26.07	0.01\\
26.08	0.01\\
26.09	0.01\\
26.1	0.01\\
26.11	0.01\\
26.12	0.01\\
26.13	0.01\\
26.14	0.01\\
26.15	0.01\\
26.16	0.01\\
26.17	0.01\\
26.18	0.01\\
26.19	0.01\\
26.2	0.01\\
26.21	0.01\\
26.22	0.01\\
26.23	0.01\\
26.24	0.01\\
26.25	0.01\\
26.26	0.01\\
26.27	0.01\\
26.28	0.01\\
26.29	0.01\\
26.3	0.01\\
26.31	0.01\\
26.32	0.01\\
26.33	0.01\\
26.34	0.01\\
26.35	0.01\\
26.36	0.01\\
26.37	0.01\\
26.38	0.01\\
26.39	0.01\\
26.4	0.01\\
26.41	0.01\\
26.42	0.01\\
26.43	0.01\\
26.44	0.01\\
26.45	0.01\\
26.46	0.01\\
26.47	0.01\\
26.48	0.01\\
26.49	0.01\\
26.5	0.01\\
26.51	0.01\\
26.52	0.01\\
26.53	0.01\\
26.54	0.01\\
26.55	0.01\\
26.56	0.01\\
26.57	0.01\\
26.58	0.01\\
26.59	0.01\\
26.6	0.01\\
26.61	0.01\\
26.62	0.01\\
26.63	0.01\\
26.64	0.01\\
26.65	0.01\\
26.66	0.01\\
26.67	0.01\\
26.68	0.01\\
26.69	0.01\\
26.7	0.01\\
26.71	0.01\\
26.72	0.01\\
26.73	0.01\\
26.74	0.01\\
26.75	0.01\\
26.76	0.01\\
26.77	0.01\\
26.78	0.01\\
26.79	0.01\\
26.8	0.01\\
26.81	0.01\\
26.82	0.01\\
26.83	0.01\\
26.84	0.01\\
26.85	0.01\\
26.86	0.01\\
26.87	0.01\\
26.88	0.01\\
26.89	0.01\\
26.9	0.01\\
26.91	0.01\\
26.92	0.01\\
26.93	0.01\\
26.94	0.01\\
26.95	0.01\\
26.96	0.01\\
26.97	0.01\\
26.98	0.01\\
26.99	0.01\\
27	0.01\\
27.01	0.01\\
27.02	0.01\\
27.03	0.01\\
27.04	0.01\\
27.05	0.01\\
27.06	0.01\\
27.07	0.01\\
27.08	0.01\\
27.09	0.01\\
27.1	0.01\\
27.11	0.01\\
27.12	0.01\\
27.13	0.01\\
27.14	0.01\\
27.15	0.01\\
27.16	0.01\\
27.17	0.01\\
27.18	0.01\\
27.19	0.01\\
27.2	0.01\\
27.21	0.01\\
27.22	0.01\\
27.23	0.01\\
27.24	0.01\\
27.25	0.01\\
27.26	0.01\\
27.27	0.01\\
27.28	0.01\\
27.29	0.01\\
27.3	0.01\\
27.31	0.01\\
27.32	0.01\\
27.33	0.01\\
27.34	0.01\\
27.35	0.01\\
27.36	0.01\\
27.37	0.01\\
27.38	0.01\\
27.39	0.01\\
27.4	0.01\\
27.41	0.01\\
27.42	0.01\\
27.43	0.01\\
27.44	0.01\\
27.45	0.01\\
27.46	0.01\\
27.47	0.01\\
27.48	0.01\\
27.49	0.01\\
27.5	0.01\\
27.51	0.01\\
27.52	0.01\\
27.53	0.01\\
27.54	0.01\\
27.55	0.01\\
27.56	0.01\\
27.57	0.01\\
27.58	0.01\\
27.59	0.01\\
27.6	0.01\\
27.61	0.01\\
27.62	0.01\\
27.63	0.01\\
27.64	0.01\\
27.65	0.01\\
27.66	0.01\\
27.67	0.01\\
27.68	0.01\\
27.69	0.01\\
27.7	0.01\\
27.71	0.01\\
27.72	0.01\\
27.73	0.01\\
27.74	0.01\\
27.75	0.01\\
27.76	0.01\\
27.77	0.01\\
27.78	0.01\\
27.79	0.01\\
27.8	0.01\\
27.81	0.01\\
27.82	0.01\\
27.83	0.01\\
27.84	0.01\\
27.85	0.01\\
27.86	0.01\\
27.87	0.01\\
27.88	0.01\\
27.89	0.01\\
27.9	0.01\\
27.91	0.01\\
27.92	0.01\\
27.93	0.01\\
27.94	0.01\\
27.95	0.01\\
27.96	0.01\\
27.97	0.01\\
27.98	0.01\\
27.99	0.01\\
28	0.01\\
28.01	0.01\\
28.02	0.01\\
28.03	0.01\\
28.04	0.01\\
28.05	0.01\\
28.06	0.01\\
28.07	0.01\\
28.08	0.01\\
28.09	0.01\\
28.1	0.01\\
28.11	0.01\\
28.12	0.01\\
28.13	0.01\\
28.14	0.01\\
28.15	0.01\\
28.16	0.01\\
28.17	0.01\\
28.18	0.01\\
28.19	0.01\\
28.2	0.01\\
28.21	0.01\\
28.22	0.01\\
28.23	0.01\\
28.24	0.01\\
28.25	0.01\\
28.26	0.01\\
28.27	0.01\\
28.28	0.01\\
28.29	0.01\\
28.3	0.01\\
28.31	0.01\\
28.32	0.01\\
28.33	0.01\\
28.34	0.01\\
28.35	0.01\\
28.36	0.01\\
28.37	0.01\\
28.38	0.01\\
28.39	0.01\\
28.4	0.01\\
28.41	0.01\\
28.42	0.01\\
28.43	0.01\\
28.44	0.01\\
28.45	0.01\\
28.46	0.01\\
28.47	0.01\\
28.48	0.01\\
28.49	0.01\\
28.5	0.01\\
28.51	0.01\\
28.52	0.01\\
28.53	0.01\\
28.54	0.01\\
28.55	0.01\\
28.56	0.01\\
28.57	0.01\\
28.58	0.01\\
28.59	0.01\\
28.6	0.01\\
28.61	0.01\\
28.62	0.01\\
28.63	0.01\\
28.64	0.01\\
28.65	0.01\\
28.66	0.01\\
28.67	0.01\\
28.68	0.01\\
28.69	0.01\\
28.7	0.01\\
28.71	0.01\\
28.72	0.01\\
28.73	0.01\\
28.74	0.01\\
28.75	0.01\\
28.76	0.01\\
28.77	0.01\\
28.78	0.01\\
28.79	0.01\\
28.8	0.01\\
28.81	0.01\\
28.82	0.01\\
28.83	0.01\\
28.84	0.01\\
28.85	0.01\\
28.86	0.01\\
28.87	0.01\\
28.88	0.01\\
28.89	0.01\\
28.9	0.01\\
28.91	0.01\\
28.92	0.01\\
28.93	0.01\\
28.94	0.01\\
28.95	0.01\\
28.96	0.01\\
28.97	0.01\\
28.98	0.01\\
28.99	0.01\\
29	0.01\\
29.01	0.01\\
29.02	0.01\\
29.03	0.01\\
29.04	0.01\\
29.05	0.01\\
29.06	0.01\\
29.07	0.01\\
29.08	0.01\\
29.09	0.01\\
29.1	0.01\\
29.11	0.01\\
29.12	0.01\\
29.13	0.01\\
29.14	0.01\\
29.15	0.01\\
29.16	0.01\\
29.17	0.01\\
29.18	0.01\\
29.19	0.01\\
29.2	0.01\\
29.21	0.01\\
29.22	0.01\\
29.23	0.01\\
29.24	0.01\\
29.25	0.01\\
29.26	0.01\\
29.27	0.01\\
29.28	0.01\\
29.29	0.01\\
29.3	0.01\\
29.31	0.01\\
29.32	0.01\\
29.33	0.01\\
29.34	0.01\\
29.35	0.01\\
29.36	0.01\\
29.37	0.01\\
29.38	0.01\\
29.39	0.01\\
29.4	0.01\\
29.41	0.01\\
29.42	0.01\\
29.43	0.01\\
29.44	0.01\\
29.45	0.01\\
29.46	0.01\\
29.47	0.01\\
29.48	0.01\\
29.49	0.01\\
29.5	0.01\\
29.51	0.01\\
29.52	0.01\\
29.53	0.01\\
29.54	0.01\\
29.55	0.01\\
29.56	0.01\\
29.57	0.01\\
29.58	0.01\\
29.59	0.01\\
29.6	0.01\\
29.61	0.01\\
29.62	0.01\\
29.63	0.01\\
29.64	0.01\\
29.65	0.01\\
29.66	0.01\\
29.67	0.01\\
29.68	0.01\\
29.69	0.01\\
29.7	0.01\\
29.71	0.01\\
29.72	0.01\\
29.73	0.01\\
29.74	0.01\\
29.75	0.01\\
29.76	0.01\\
29.77	0.01\\
29.78	0.01\\
29.79	0.01\\
29.8	0.01\\
29.81	0.01\\
29.82	0.01\\
29.83	0.01\\
29.84	0.01\\
29.85	0.01\\
29.86	0.01\\
29.87	0.01\\
29.88	0.01\\
29.89	0.01\\
29.9	0.01\\
29.91	0.01\\
29.92	0.01\\
29.93	0.01\\
29.94	0.01\\
29.95	0.01\\
29.96	0.01\\
29.97	0.01\\
29.98	0.01\\
29.99	0.01\\
30	0.01\\
30.01	0.01\\
30.02	0.01\\
30.03	0.01\\
30.04	0.01\\
30.05	0.01\\
30.06	0.01\\
30.07	0.01\\
30.08	0.01\\
30.09	0.01\\
30.1	0.01\\
30.11	0.01\\
30.12	0.01\\
30.13	0.01\\
30.14	0.01\\
30.15	0.01\\
30.16	0.01\\
30.17	0.01\\
30.18	0.01\\
30.19	0.01\\
30.2	0.01\\
30.21	0.01\\
30.22	0.01\\
30.23	0.01\\
30.24	0.01\\
30.25	0.01\\
30.26	0.01\\
30.27	0.01\\
30.28	0.01\\
30.29	0.01\\
30.3	0.01\\
30.31	0.01\\
30.32	0.01\\
30.33	0.01\\
30.34	0.01\\
30.35	0.01\\
30.36	0.01\\
30.37	0.01\\
30.38	0.01\\
30.39	0.01\\
30.4	0.01\\
30.41	0.01\\
30.42	0.01\\
30.43	0.01\\
30.44	0.01\\
30.45	0.01\\
30.46	0.01\\
30.47	0.01\\
30.48	0.01\\
30.49	0.01\\
30.5	0.01\\
30.51	0.01\\
30.52	0.01\\
30.53	0.01\\
30.54	0.01\\
30.55	0.01\\
30.56	0.01\\
30.57	0.01\\
30.58	0.01\\
30.59	0.01\\
30.6	0.01\\
30.61	0.01\\
30.62	0.01\\
30.63	0.01\\
30.64	0.01\\
30.65	0.01\\
30.66	0.01\\
30.67	0.01\\
30.68	0.01\\
30.69	0.01\\
30.7	0.01\\
30.71	0.01\\
30.72	0.01\\
30.73	0.01\\
30.74	0.01\\
30.75	0.01\\
30.76	0.01\\
30.77	0.01\\
30.78	0.01\\
30.79	0.01\\
30.8	0.01\\
30.81	0.01\\
30.82	0.01\\
30.83	0.01\\
30.84	0.01\\
30.85	0.01\\
30.86	0.01\\
30.87	0.01\\
30.88	0.01\\
30.89	0.01\\
30.9	0.01\\
30.91	0.01\\
30.92	0.01\\
30.93	0.01\\
30.94	0.01\\
30.95	0.01\\
30.96	0.01\\
30.97	0.01\\
30.98	0.01\\
30.99	0.01\\
31	0.01\\
31.01	0.01\\
31.02	0.01\\
31.03	0.01\\
31.04	0.01\\
31.05	0.01\\
31.06	0.01\\
31.07	0.01\\
31.08	0.01\\
31.09	0.01\\
31.1	0.01\\
31.11	0.01\\
31.12	0.01\\
31.13	0.01\\
31.14	0.01\\
31.15	0.01\\
31.16	0.01\\
31.17	0.01\\
31.18	0.01\\
31.19	0.01\\
31.2	0.01\\
31.21	0.01\\
31.22	0.01\\
31.23	0.01\\
31.24	0.01\\
31.25	0.01\\
31.26	0.01\\
31.27	0.01\\
31.28	0.01\\
31.29	0.01\\
31.3	0.01\\
31.31	0.01\\
31.32	0.01\\
31.33	0.01\\
31.34	0.01\\
31.35	0.01\\
31.36	0.01\\
31.37	0.01\\
31.38	0.01\\
31.39	0.01\\
31.4	0.01\\
31.41	0.01\\
31.42	0.01\\
31.43	0.01\\
31.44	0.01\\
31.45	0.01\\
31.46	0.01\\
31.47	0.01\\
31.48	0.01\\
31.49	0.01\\
31.5	0.01\\
31.51	0.01\\
31.52	0.01\\
31.53	0.01\\
31.54	0.01\\
31.55	0.01\\
31.56	0.01\\
31.57	0.01\\
31.58	0.01\\
31.59	0.01\\
31.6	0.01\\
31.61	0.01\\
31.62	0.01\\
31.63	0.01\\
31.64	0.01\\
31.65	0.01\\
31.66	0.01\\
31.67	0.01\\
31.68	0.01\\
31.69	0.01\\
31.7	0.01\\
31.71	0.01\\
31.72	0.01\\
31.73	0.01\\
31.74	0.01\\
31.75	0.01\\
31.76	0.01\\
31.77	0.01\\
31.78	0.01\\
31.79	0.01\\
31.8	0.01\\
31.81	0.01\\
31.82	0.01\\
31.83	0.01\\
31.84	0.01\\
31.85	0.01\\
31.86	0.01\\
31.87	0.01\\
31.88	0.01\\
31.89	0.01\\
31.9	0.01\\
31.91	0.01\\
31.92	0.01\\
31.93	0.01\\
31.94	0.01\\
31.95	0.01\\
31.96	0.01\\
31.97	0.01\\
31.98	0.01\\
31.99	0.01\\
32	0.01\\
32.01	0.01\\
32.02	0.01\\
32.03	0.01\\
32.04	0.01\\
32.05	0.01\\
32.06	0.01\\
32.07	0.01\\
32.08	0.01\\
32.09	0.01\\
32.1	0.01\\
32.11	0.01\\
32.12	0.01\\
32.13	0.01\\
32.14	0.01\\
32.15	0.01\\
32.16	0.01\\
32.17	0.01\\
32.18	0.01\\
32.19	0.01\\
32.2	0.01\\
32.21	0.01\\
32.22	0.01\\
32.23	0.01\\
32.24	0.01\\
32.25	0.01\\
32.26	0.01\\
32.27	0.01\\
32.28	0.01\\
32.29	0.01\\
32.3	0.01\\
32.31	0.01\\
32.32	0.01\\
32.33	0.01\\
32.34	0.01\\
32.35	0.01\\
32.36	0.01\\
32.37	0.01\\
32.38	0.01\\
32.39	0.01\\
32.4	0.01\\
32.41	0.01\\
32.42	0.01\\
32.43	0.01\\
32.44	0.01\\
32.45	0.01\\
32.46	0.01\\
32.47	0.01\\
32.48	0.01\\
32.49	0.01\\
32.5	0.01\\
32.51	0.01\\
32.52	0.01\\
32.53	0.01\\
32.54	0.01\\
32.55	0.01\\
32.56	0.01\\
32.57	0.01\\
32.58	0.01\\
32.59	0.01\\
32.6	0.01\\
32.61	0.01\\
32.62	0.01\\
32.63	0.01\\
32.64	0.01\\
32.65	0.01\\
32.66	0.01\\
32.67	0.01\\
32.68	0.01\\
32.69	0.01\\
32.7	0.01\\
32.71	0.01\\
32.72	0.01\\
32.73	0.01\\
32.74	0.01\\
32.75	0.01\\
32.76	0.01\\
32.77	0.01\\
32.78	0.01\\
32.79	0.01\\
32.8	0.01\\
32.81	0.01\\
32.82	0.01\\
32.83	0.01\\
32.84	0.01\\
32.85	0.01\\
32.86	0.01\\
32.87	0.01\\
32.88	0.01\\
32.89	0.01\\
32.9	0.01\\
32.91	0.01\\
32.92	0.01\\
32.93	0.01\\
32.94	0.01\\
32.95	0.01\\
32.96	0.01\\
32.97	0.01\\
32.98	0.01\\
32.99	0.01\\
33	0.01\\
33.01	0.01\\
33.02	0.01\\
33.03	0.01\\
33.04	0.01\\
33.05	0.01\\
33.06	0.01\\
33.07	0.01\\
33.08	0.01\\
33.09	0.01\\
33.1	0.01\\
33.11	0.01\\
33.12	0.01\\
33.13	0.01\\
33.14	0.01\\
33.15	0.01\\
33.16	0.01\\
33.17	0.01\\
33.18	0.01\\
33.19	0.01\\
33.2	0.01\\
33.21	0.01\\
33.22	0.01\\
33.23	0.01\\
33.24	0.01\\
33.25	0.01\\
33.26	0.01\\
33.27	0.01\\
33.28	0.01\\
33.29	0.01\\
33.3	0.01\\
33.31	0.01\\
33.32	0.01\\
33.33	0.01\\
33.34	0.01\\
33.35	0.01\\
33.36	0.01\\
33.37	0.01\\
33.38	0.01\\
33.39	0.01\\
33.4	0.01\\
33.41	0.01\\
33.42	0.01\\
33.43	0.01\\
33.44	0.01\\
33.45	0.01\\
33.46	0.01\\
33.47	0.01\\
33.48	0.01\\
33.49	0.01\\
33.5	0.01\\
33.51	0.01\\
33.52	0.01\\
33.53	0.01\\
33.54	0.01\\
33.55	0.01\\
33.56	0.01\\
33.57	0.01\\
33.58	0.01\\
33.59	0.01\\
33.6	0.01\\
33.61	0.01\\
33.62	0.01\\
33.63	0.01\\
33.64	0.01\\
33.65	0.01\\
33.66	0.01\\
33.67	0.01\\
33.68	0.01\\
33.69	0.01\\
33.7	0.01\\
33.71	0.01\\
33.72	0.01\\
33.73	0.01\\
33.74	0.01\\
33.75	0.01\\
33.76	0.01\\
33.77	0.01\\
33.78	0.01\\
33.79	0.01\\
33.8	0.01\\
33.81	0.01\\
33.82	0.01\\
33.83	0.01\\
33.84	0.01\\
33.85	0.01\\
33.86	0.01\\
33.87	0.01\\
33.88	0.01\\
33.89	0.01\\
33.9	0.01\\
33.91	0.01\\
33.92	0.01\\
33.93	0.01\\
33.94	0.01\\
33.95	0.01\\
33.96	0.01\\
33.97	0.01\\
33.98	0.01\\
33.99	0.01\\
34	0.01\\
34.01	0.01\\
34.02	0.01\\
34.03	0.01\\
34.04	0.01\\
34.05	0.01\\
34.06	0.01\\
34.07	0.01\\
34.08	0.01\\
34.09	0.01\\
34.1	0.01\\
34.11	0.01\\
34.12	0.01\\
34.13	0.01\\
34.14	0.01\\
34.15	0.01\\
34.16	0.01\\
34.17	0.01\\
34.18	0.01\\
34.19	0.01\\
34.2	0.01\\
34.21	0.01\\
34.22	0.01\\
34.23	0.01\\
34.24	0.01\\
34.25	0.01\\
34.26	0.01\\
34.27	0.01\\
34.28	0.01\\
34.29	0.01\\
34.3	0.01\\
34.31	0.01\\
34.32	0.01\\
34.33	0.01\\
34.34	0.01\\
34.35	0.01\\
34.36	0.01\\
34.37	0.01\\
34.38	0.01\\
34.39	0.01\\
34.4	0.01\\
34.41	0.01\\
34.42	0.01\\
34.43	0.01\\
34.44	0.01\\
34.45	0.01\\
34.46	0.01\\
34.47	0.01\\
34.48	0.01\\
34.49	0.01\\
34.5	0.01\\
34.51	0.01\\
34.52	0.01\\
34.53	0.01\\
34.54	0.01\\
34.55	0.01\\
34.56	0.01\\
34.57	0.01\\
34.58	0.01\\
34.59	0.01\\
34.6	0.01\\
34.61	0.01\\
34.62	0.01\\
34.63	0.01\\
34.64	0.01\\
34.65	0.01\\
34.66	0.01\\
34.67	0.01\\
34.68	0.01\\
34.69	0.01\\
34.7	0.01\\
34.71	0.01\\
34.72	0.01\\
34.73	0.01\\
34.74	0.01\\
34.75	0.01\\
34.76	0.01\\
34.77	0.01\\
34.78	0.01\\
34.79	0.01\\
34.8	0.01\\
34.81	0.01\\
34.82	0.01\\
34.83	0.01\\
34.84	0.01\\
34.85	0.01\\
34.86	0.01\\
34.87	0.01\\
34.88	0.01\\
34.89	0.01\\
34.9	0.01\\
34.91	0.01\\
34.92	0.01\\
34.93	0.01\\
34.94	0.01\\
34.95	0.01\\
34.96	0.01\\
34.97	0.01\\
34.98	0.01\\
34.99	0.01\\
35	0.01\\
35.01	0.01\\
35.02	0.01\\
35.03	0.01\\
35.04	0.01\\
35.05	0.01\\
35.06	0.01\\
35.07	0.01\\
35.08	0.01\\
35.09	0.01\\
35.1	0.01\\
35.11	0.01\\
35.12	0.01\\
35.13	0.01\\
35.14	0.01\\
35.15	0.01\\
35.16	0.01\\
35.17	0.01\\
35.18	0.01\\
35.19	0.01\\
35.2	0.01\\
35.21	0.01\\
35.22	0.01\\
35.23	0.01\\
35.24	0.01\\
35.25	0.01\\
35.26	0.01\\
35.27	0.01\\
35.28	0.01\\
35.29	0.01\\
35.3	0.01\\
35.31	0.01\\
35.32	0.01\\
35.33	0.01\\
35.34	0.01\\
35.35	0.01\\
35.36	0.01\\
35.37	0.01\\
35.38	0.01\\
35.39	0.01\\
35.4	0.01\\
35.41	0.01\\
35.42	0.01\\
35.43	0.01\\
35.44	0.01\\
35.45	0.01\\
35.46	0.01\\
35.47	0.01\\
35.48	0.01\\
35.49	0.01\\
35.5	0.01\\
35.51	0.01\\
35.52	0.01\\
35.53	0.01\\
35.54	0.01\\
35.55	0.01\\
35.56	0.01\\
35.57	0.01\\
35.58	0.01\\
35.59	0.01\\
35.6	0.01\\
35.61	0.01\\
35.62	0.01\\
35.63	0.01\\
35.64	0.01\\
35.65	0.01\\
35.66	0.01\\
35.67	0.01\\
35.68	0.01\\
35.69	0.01\\
35.7	0.01\\
35.71	0.01\\
35.72	0.01\\
35.73	0.01\\
35.74	0.01\\
35.75	0.01\\
35.76	0.01\\
35.77	0.01\\
35.78	0.01\\
35.79	0.01\\
35.8	0.01\\
35.81	0.01\\
35.82	0.01\\
35.83	0.01\\
35.84	0.01\\
35.85	0.01\\
35.86	0.01\\
35.87	0.01\\
35.88	0.01\\
35.89	0.01\\
35.9	0.01\\
35.91	0.01\\
35.92	0.01\\
35.93	0.01\\
35.94	0.01\\
35.95	0.01\\
35.96	0.01\\
35.97	0.01\\
35.98	0.01\\
35.99	0.01\\
36	0.01\\
36.01	0.01\\
36.02	0.01\\
36.03	0.01\\
36.04	0.01\\
36.05	0.01\\
36.06	0.01\\
36.07	0.01\\
36.08	0.01\\
36.09	0.01\\
36.1	0.01\\
36.11	0.01\\
36.12	0.01\\
36.13	0.01\\
36.14	0.01\\
36.15	0.01\\
36.16	0.01\\
36.17	0.01\\
36.18	0.01\\
36.19	0.01\\
36.2	0.01\\
36.21	0.01\\
36.22	0.01\\
36.23	0.01\\
36.24	0.01\\
36.25	0.01\\
36.26	0.01\\
36.27	0.01\\
36.28	0.01\\
36.29	0.01\\
36.3	0.01\\
36.31	0.01\\
36.32	0.01\\
36.33	0.01\\
36.34	0.01\\
36.35	0.01\\
36.36	0.01\\
36.37	0.01\\
36.38	0.01\\
36.39	0.01\\
36.4	0.01\\
36.41	0.01\\
36.42	0.01\\
36.43	0.01\\
36.44	0.01\\
36.45	0.01\\
36.46	0.01\\
36.47	0.01\\
36.48	0.01\\
36.49	0.01\\
36.5	0.01\\
36.51	0.01\\
36.52	0.01\\
36.53	0.01\\
36.54	0.01\\
36.55	0.01\\
36.56	0.01\\
36.57	0.01\\
36.58	0.01\\
36.59	0.01\\
36.6	0.01\\
36.61	0.01\\
36.62	0.01\\
36.63	0.01\\
36.64	0.01\\
36.65	0.01\\
36.66	0.01\\
36.67	0.01\\
36.68	0.01\\
36.69	0.01\\
36.7	0.01\\
36.71	0.01\\
36.72	0.01\\
36.73	0.01\\
36.74	0.01\\
36.75	0.01\\
36.76	0.01\\
36.77	0.01\\
36.78	0.01\\
36.79	0.01\\
36.8	0.01\\
36.81	0.01\\
36.82	0.01\\
36.83	0.01\\
36.84	0.01\\
36.85	0.01\\
36.86	0.01\\
36.87	0.01\\
36.88	0.01\\
36.89	0.01\\
36.9	0.01\\
36.91	0.01\\
36.92	0.01\\
36.93	0.01\\
36.94	0.01\\
36.95	0.01\\
36.96	0.01\\
36.97	0.01\\
36.98	0.01\\
36.99	0.01\\
37	0.01\\
37.01	0.01\\
37.02	0.01\\
37.03	0.01\\
37.04	0.01\\
37.05	0.01\\
37.06	0.01\\
37.07	0.01\\
37.08	0.01\\
37.09	0.01\\
37.1	0.01\\
37.11	0.01\\
37.12	0.01\\
37.13	0.01\\
37.14	0.01\\
37.15	0.01\\
37.16	0.01\\
37.17	0.01\\
37.18	0.01\\
37.19	0.01\\
37.2	0.01\\
37.21	0.01\\
37.22	0.01\\
37.23	0.01\\
37.24	0.01\\
37.25	0.01\\
37.26	0.01\\
37.27	0.01\\
37.28	0.01\\
37.29	0.01\\
37.3	0.01\\
37.31	0.01\\
37.32	0.01\\
37.33	0.01\\
37.34	0.01\\
37.35	0.01\\
37.36	0.01\\
37.37	0.01\\
37.38	0.01\\
37.39	0.01\\
37.4	0.01\\
37.41	0.01\\
37.42	0.01\\
37.43	0.01\\
37.44	0.01\\
37.45	0.01\\
37.46	0.01\\
37.47	0.01\\
37.48	0.01\\
37.49	0.01\\
37.5	0.01\\
37.51	0.01\\
37.52	0.01\\
37.53	0.01\\
37.54	0.01\\
37.55	0.01\\
37.56	0.01\\
37.57	0.01\\
37.58	0.01\\
37.59	0.01\\
37.6	0.01\\
37.61	0.01\\
37.62	0.01\\
37.63	0.01\\
37.64	0.01\\
37.65	0.01\\
37.66	0.01\\
37.67	0.01\\
37.68	0.01\\
37.69	0.01\\
37.7	0.01\\
37.71	0.01\\
37.72	0.01\\
37.73	0.01\\
37.74	0.01\\
37.75	0.01\\
37.76	0.01\\
37.77	0.01\\
37.78	0.01\\
37.79	0.01\\
37.8	0.01\\
37.81	0.01\\
37.82	0.01\\
37.83	0.01\\
37.84	0.01\\
37.85	0.01\\
37.86	0.01\\
37.87	0.01\\
37.88	0.01\\
37.89	0.01\\
37.9	0.01\\
37.91	0.01\\
37.92	0.01\\
37.93	0.01\\
37.94	0.01\\
37.95	0.01\\
37.96	0.01\\
37.97	0.01\\
37.98	0.01\\
37.99	0.01\\
38	0.01\\
38.01	0.01\\
38.02	0.01\\
38.03	0.01\\
38.04	0.01\\
38.05	0.01\\
38.06	0.01\\
38.07	0.01\\
38.08	0.01\\
38.09	0.01\\
38.1	0.01\\
38.11	0.01\\
38.12	0.01\\
38.13	0.01\\
38.14	0.01\\
38.15	0.01\\
38.16	0.01\\
38.17	0.01\\
38.18	0.01\\
38.19	0.01\\
38.2	0.01\\
38.21	0.01\\
38.22	0.01\\
38.23	0.01\\
38.24	0.01\\
38.25	0.01\\
38.26	0.01\\
38.27	0.01\\
38.28	0.01\\
38.29	0.01\\
38.3	0.01\\
38.31	0.01\\
38.32	0.01\\
38.33	0.01\\
38.34	0.01\\
38.35	0.01\\
38.36	0.01\\
38.37	0.01\\
38.38	0.01\\
38.39	0.01\\
38.4	0.01\\
38.41	0.01\\
38.42	0.01\\
38.43	0.01\\
38.44	0.01\\
38.45	0.01\\
38.46	0.01\\
38.47	0.01\\
38.48	0.01\\
38.49	0.01\\
38.5	0.01\\
38.51	0.01\\
38.52	0.01\\
38.53	0.01\\
38.54	0.01\\
38.55	0.01\\
38.56	0.01\\
38.57	0.01\\
38.58	0.01\\
38.59	0.01\\
38.6	0.01\\
38.61	0.01\\
38.62	0.01\\
38.63	0.01\\
38.64	0.01\\
38.65	0.01\\
38.66	0.01\\
38.67	0.01\\
38.68	0.01\\
38.69	0.01\\
38.7	0.01\\
38.71	0.01\\
38.72	0.01\\
38.73	0.01\\
38.74	0.01\\
38.75	0.01\\
38.76	0.01\\
38.77	0.01\\
38.78	0.01\\
38.79	0.01\\
38.8	0.01\\
38.81	0.01\\
38.82	0.01\\
38.83	0.01\\
38.84	0.01\\
38.85	0.01\\
38.86	0.01\\
38.87	0.01\\
38.88	0.01\\
38.89	0.01\\
38.9	0.01\\
38.91	0.01\\
38.92	0.01\\
38.93	0.01\\
38.94	0.01\\
38.95	0.01\\
38.96	0.01\\
38.97	0.01\\
38.98	0.01\\
38.99	0.01\\
39	0.01\\
39.01	0.01\\
39.02	0.01\\
39.03	0.01\\
39.04	0.01\\
39.05	0.01\\
39.06	0.01\\
39.07	0.01\\
39.08	0.01\\
39.09	0.01\\
39.1	0.01\\
39.11	0.01\\
39.12	0.01\\
39.13	0.01\\
39.14	0.01\\
39.15	0.01\\
39.16	0.01\\
39.17	0.01\\
39.18	0.01\\
39.19	0.01\\
39.2	0.01\\
39.21	0.01\\
39.22	0.01\\
39.23	0.01\\
39.24	0.01\\
39.25	0.01\\
39.26	0.01\\
39.27	0.01\\
39.28	0.01\\
39.29	0.01\\
39.3	0.01\\
39.31	0.01\\
39.32	0.01\\
39.33	0.01\\
39.34	0.01\\
39.35	0.01\\
39.36	0.01\\
39.37	0.01\\
39.38	0.01\\
39.39	0.01\\
39.4	0.01\\
39.41	0.01\\
39.42	0.01\\
39.43	0.01\\
39.44	0.01\\
39.45	0.01\\
39.46	0.01\\
39.47	0.01\\
39.48	0.01\\
39.49	0.01\\
39.5	0.01\\
39.51	0.01\\
39.52	0.01\\
39.53	0.01\\
39.54	0.01\\
39.55	0.01\\
39.56	0.01\\
39.57	0.01\\
39.58	0.01\\
39.59	0.01\\
39.6	0.01\\
39.61	0.01\\
39.62	0.01\\
39.63	0.01\\
39.64	0.01\\
39.65	0.01\\
39.66	0.01\\
39.67	0.01\\
39.68	0.01\\
39.69	0.01\\
39.7	0.01\\
39.71	0.01\\
39.72	0.01\\
39.73	0.01\\
39.74	0.01\\
39.75	0.01\\
39.76	0.01\\
39.77	0.01\\
39.78	0.01\\
39.79	0.01\\
39.8	0.01\\
39.81	0.01\\
39.82	0.01\\
39.83	0.01\\
39.84	0.01\\
39.85	0.01\\
39.86	0.01\\
39.87	0.01\\
39.88	0.01\\
39.89	0.01\\
39.9	0.01\\
39.91	0.01\\
39.92	0.01\\
39.93	0.01\\
39.94	0.01\\
39.95	0.01\\
39.96	0.01\\
39.97	0.01\\
39.98	0.01\\
39.99	0.01\\
40	0.01\\
40.01	0.01\\
};
\addplot [color=blue,dashed,forget plot]
  table[row sep=crcr]{%
40.01	0.01\\
40.02	0.01\\
40.03	0.01\\
40.04	0.01\\
40.05	0.01\\
40.06	0.01\\
40.07	0.01\\
40.08	0.01\\
40.09	0.01\\
40.1	0.01\\
40.11	0.01\\
40.12	0.01\\
40.13	0.01\\
40.14	0.01\\
40.15	0.01\\
40.16	0.01\\
40.17	0.01\\
40.18	0.01\\
40.19	0.01\\
40.2	0.01\\
40.21	0.01\\
40.22	0.01\\
40.23	0.01\\
40.24	0.01\\
40.25	0.01\\
40.26	0.01\\
40.27	0.01\\
40.28	0.01\\
40.29	0.01\\
40.3	0.01\\
40.31	0.01\\
40.32	0.01\\
40.33	0.01\\
40.34	0.01\\
40.35	0.01\\
40.36	0.01\\
40.37	0.01\\
40.38	0.01\\
40.39	0.01\\
40.4	0.01\\
40.41	0.01\\
40.42	0.01\\
40.43	0.01\\
40.44	0.01\\
40.45	0.01\\
40.46	0.01\\
40.47	0.01\\
40.48	0.01\\
40.49	0.01\\
40.5	0.01\\
40.51	0.01\\
40.52	0.01\\
40.53	0.01\\
40.54	0.01\\
40.55	0.01\\
40.56	0.01\\
40.57	0.01\\
40.58	0.01\\
40.59	0.01\\
40.6	0.01\\
40.61	0.01\\
40.62	0.01\\
40.63	0.01\\
40.64	0.01\\
40.65	0.01\\
40.66	0.01\\
40.67	0.01\\
40.68	0.01\\
40.69	0.01\\
40.7	0.01\\
40.71	0.01\\
40.72	0.01\\
40.73	0.01\\
40.74	0.01\\
40.75	0.01\\
40.76	0.01\\
40.77	0.01\\
40.78	0.01\\
40.79	0.01\\
40.8	0.01\\
40.81	0.01\\
40.82	0.01\\
40.83	0.01\\
40.84	0.01\\
40.85	0.01\\
40.86	0.01\\
40.87	0.01\\
40.88	0.01\\
40.89	0.01\\
40.9	0.01\\
40.91	0.01\\
40.92	0.01\\
40.93	0.01\\
40.94	0.01\\
40.95	0.01\\
40.96	0.01\\
40.97	0.01\\
40.98	0.01\\
40.99	0.01\\
41	0.01\\
41.01	0.01\\
41.02	0.01\\
41.03	0.01\\
41.04	0.01\\
41.05	0.01\\
41.06	0.01\\
41.07	0.01\\
41.08	0.01\\
41.09	0.01\\
41.1	0.01\\
41.11	0.01\\
41.12	0.01\\
41.13	0.01\\
41.14	0.01\\
41.15	0.01\\
41.16	0.01\\
41.17	0.01\\
41.18	0.01\\
41.19	0.01\\
41.2	0.01\\
41.21	0.01\\
41.22	0.01\\
41.23	0.01\\
41.24	0.01\\
41.25	0.01\\
41.26	0.01\\
41.27	0.01\\
41.28	0.01\\
41.29	0.01\\
41.3	0.01\\
41.31	0.01\\
41.32	0.01\\
41.33	0.01\\
41.34	0.01\\
41.35	0.01\\
41.36	0.01\\
41.37	0.01\\
41.38	0.01\\
41.39	0.01\\
41.4	0.01\\
41.41	0.01\\
41.42	0.01\\
41.43	0.01\\
41.44	0.01\\
41.45	0.01\\
41.46	0.01\\
41.47	0.01\\
41.48	0.01\\
41.49	0.01\\
41.5	0.01\\
41.51	0.01\\
41.52	0.01\\
41.53	0.01\\
41.54	0.01\\
41.55	0.01\\
41.56	0.01\\
41.57	0.01\\
41.58	0.01\\
41.59	0.01\\
41.6	0.01\\
41.61	0.01\\
41.62	0.01\\
41.63	0.01\\
41.64	0.01\\
41.65	0.01\\
41.66	0.01\\
41.67	0.01\\
41.68	0.01\\
41.69	0.01\\
41.7	0.01\\
41.71	0.01\\
41.72	0.01\\
41.73	0.01\\
41.74	0.01\\
41.75	0.01\\
41.76	0.01\\
41.77	0.01\\
41.78	0.01\\
41.79	0.01\\
41.8	0.01\\
41.81	0.01\\
41.82	0.01\\
41.83	0.01\\
41.84	0.01\\
41.85	0.01\\
41.86	0.01\\
41.87	0.01\\
41.88	0.01\\
41.89	0.01\\
41.9	0.01\\
41.91	0.01\\
41.92	0.01\\
41.93	0.01\\
41.94	0.01\\
41.95	0.01\\
41.96	0.01\\
41.97	0.01\\
41.98	0.01\\
41.99	0.01\\
42	0.01\\
42.01	0.01\\
42.02	0.01\\
42.03	0.01\\
42.04	0.01\\
42.05	0.01\\
42.06	0.01\\
42.07	0.01\\
42.08	0.01\\
42.09	0.01\\
42.1	0.01\\
42.11	0.01\\
42.12	0.01\\
42.13	0.01\\
42.14	0.01\\
42.15	0.01\\
42.16	0.01\\
42.17	0.01\\
42.18	0.01\\
42.19	0.01\\
42.2	0.01\\
42.21	0.01\\
42.22	0.01\\
42.23	0.01\\
42.24	0.01\\
42.25	0.01\\
42.26	0.01\\
42.27	0.01\\
42.28	0.01\\
42.29	0.01\\
42.3	0.01\\
42.31	0.01\\
42.32	0.01\\
42.33	0.01\\
42.34	0.01\\
42.35	0.01\\
42.36	0.01\\
42.37	0.01\\
42.38	0.01\\
42.39	0.01\\
42.4	0.01\\
42.41	0.01\\
42.42	0.01\\
42.43	0.01\\
42.44	0.01\\
42.45	0.01\\
42.46	0.01\\
42.47	0.01\\
42.48	0.01\\
42.49	0.01\\
42.5	0.01\\
42.51	0.01\\
42.52	0.01\\
42.53	0.01\\
42.54	0.01\\
42.55	0.01\\
42.56	0.01\\
42.57	0.01\\
42.58	0.01\\
42.59	0.01\\
42.6	0.01\\
42.61	0.01\\
42.62	0.01\\
42.63	0.01\\
42.64	0.01\\
42.65	0.01\\
42.66	0.01\\
42.67	0.01\\
42.68	0.01\\
42.69	0.01\\
42.7	0.01\\
42.71	0.01\\
42.72	0.01\\
42.73	0.01\\
42.74	0.01\\
42.75	0.01\\
42.76	0.01\\
42.77	0.01\\
42.78	0.01\\
42.79	0.01\\
42.8	0.01\\
42.81	0.01\\
42.82	0.01\\
42.83	0.01\\
42.84	0.01\\
42.85	0.01\\
42.86	0.01\\
42.87	0.01\\
42.88	0.01\\
42.89	0.01\\
42.9	0.01\\
42.91	0.01\\
42.92	0.01\\
42.93	0.01\\
42.94	0.01\\
42.95	0.01\\
42.96	0.01\\
42.97	0.01\\
42.98	0.01\\
42.99	0.01\\
43	0.01\\
43.01	0.01\\
43.02	0.01\\
43.03	0.01\\
43.04	0.01\\
43.05	0.01\\
43.06	0.01\\
43.07	0.01\\
43.08	0.01\\
43.09	0.01\\
43.1	0.01\\
43.11	0.01\\
43.12	0.01\\
43.13	0.01\\
43.14	0.01\\
43.15	0.01\\
43.16	0.01\\
43.17	0.01\\
43.18	0.01\\
43.19	0.01\\
43.2	0.01\\
43.21	0.01\\
43.22	0.01\\
43.23	0.01\\
43.24	0.01\\
43.25	0.01\\
43.26	0.01\\
43.27	0.01\\
43.28	0.01\\
43.29	0.01\\
43.3	0.01\\
43.31	0.01\\
43.32	0.01\\
43.33	0.01\\
43.34	0.01\\
43.35	0.01\\
43.36	0.01\\
43.37	0.01\\
43.38	0.01\\
43.39	0.01\\
43.4	0.01\\
43.41	0.01\\
43.42	0.01\\
43.43	0.01\\
43.44	0.01\\
43.45	0.01\\
43.46	0.01\\
43.47	0.01\\
43.48	0.01\\
43.49	0.01\\
43.5	0.01\\
43.51	0.01\\
43.52	0.01\\
43.53	0.01\\
43.54	0.01\\
43.55	0.01\\
43.56	0.01\\
43.57	0.01\\
43.58	0.01\\
43.59	0.01\\
43.6	0.01\\
43.61	0.01\\
43.62	0.01\\
43.63	0.01\\
43.64	0.01\\
43.65	0.01\\
43.66	0.01\\
43.67	0.01\\
43.68	0.01\\
43.69	0.01\\
43.7	0.01\\
43.71	0.01\\
43.72	0.01\\
43.73	0.01\\
43.74	0.01\\
43.75	0.01\\
43.76	0.01\\
43.77	0.01\\
43.78	0.01\\
43.79	0.01\\
43.8	0.01\\
43.81	0.01\\
43.82	0.01\\
43.83	0.01\\
43.84	0.01\\
43.85	0.01\\
43.86	0.01\\
43.87	0.01\\
43.88	0.01\\
43.89	0.01\\
43.9	0.01\\
43.91	0.01\\
43.92	0.01\\
43.93	0.01\\
43.94	0.01\\
43.95	0.01\\
43.96	0.01\\
43.97	0.01\\
43.98	0.01\\
43.99	0.01\\
44	0.01\\
44.01	0.01\\
44.02	0.01\\
44.03	0.01\\
44.04	0.01\\
44.05	0.01\\
44.06	0.01\\
44.07	0.01\\
44.08	0.01\\
44.09	0.01\\
44.1	0.01\\
44.11	0.01\\
44.12	0.01\\
44.13	0.01\\
44.14	0.01\\
44.15	0.01\\
44.16	0.01\\
44.17	0.01\\
44.18	0.01\\
44.19	0.01\\
44.2	0.01\\
44.21	0.01\\
44.22	0.01\\
44.23	0.01\\
44.24	0.01\\
44.25	0.01\\
44.26	0.01\\
44.27	0.01\\
44.28	0.01\\
44.29	0.01\\
44.3	0.01\\
44.31	0.01\\
44.32	0.01\\
44.33	0.01\\
44.34	0.01\\
44.35	0.01\\
44.36	0.01\\
44.37	0.01\\
44.38	0.01\\
44.39	0.01\\
44.4	0.01\\
44.41	0.01\\
44.42	0.01\\
44.43	0.01\\
44.44	0.01\\
44.45	0.01\\
44.46	0.01\\
44.47	0.01\\
44.48	0.01\\
44.49	0.01\\
44.5	0.01\\
44.51	0.01\\
44.52	0.01\\
44.53	0.01\\
44.54	0.01\\
44.55	0.01\\
44.56	0.01\\
44.57	0.01\\
44.58	0.01\\
44.59	0.01\\
44.6	0.01\\
44.61	0.01\\
44.62	0.01\\
44.63	0.01\\
44.64	0.01\\
44.65	0.01\\
44.66	0.01\\
44.67	0.01\\
44.68	0.01\\
44.69	0.01\\
44.7	0.01\\
44.71	0.01\\
44.72	0.01\\
44.73	0.01\\
44.74	0.01\\
44.75	0.01\\
44.76	0.01\\
44.77	0.01\\
44.78	0.01\\
44.79	0.01\\
44.8	0.01\\
44.81	0.01\\
44.82	0.01\\
44.83	0.01\\
44.84	0.01\\
44.85	0.01\\
44.86	0.01\\
44.87	0.01\\
44.88	0.01\\
44.89	0.01\\
44.9	0.01\\
44.91	0.01\\
44.92	0.01\\
44.93	0.01\\
44.94	0.01\\
44.95	0.01\\
44.96	0.01\\
44.97	0.01\\
44.98	0.01\\
44.99	0.01\\
45	0.01\\
45.01	0.01\\
45.02	0.01\\
45.03	0.01\\
45.04	0.01\\
45.05	0.01\\
45.06	0.01\\
45.07	0.01\\
45.08	0.01\\
45.09	0.01\\
45.1	0.01\\
45.11	0.01\\
45.12	0.01\\
45.13	0.01\\
45.14	0.01\\
45.15	0.01\\
45.16	0.01\\
45.17	0.01\\
45.18	0.01\\
45.19	0.01\\
45.2	0.01\\
45.21	0.01\\
45.22	0.01\\
45.23	0.01\\
45.24	0.01\\
45.25	0.01\\
45.26	0.01\\
45.27	0.01\\
45.28	0.01\\
45.29	0.01\\
45.3	0.01\\
45.31	0.01\\
45.32	0.01\\
45.33	0.01\\
45.34	0.01\\
45.35	0.01\\
45.36	0.01\\
45.37	0.01\\
45.38	0.01\\
45.39	0.01\\
45.4	0.01\\
45.41	0.01\\
45.42	0.01\\
45.43	0.01\\
45.44	0.01\\
45.45	0.01\\
45.46	0.01\\
45.47	0.01\\
45.48	0.01\\
45.49	0.01\\
45.5	0.01\\
45.51	0.01\\
45.52	0.01\\
45.53	0.01\\
45.54	0.01\\
45.55	0.01\\
45.56	0.01\\
45.57	0.01\\
45.58	0.01\\
45.59	0.01\\
45.6	0.01\\
45.61	0.01\\
45.62	0.01\\
45.63	0.01\\
45.64	0.01\\
45.65	0.01\\
45.66	0.01\\
45.67	0.01\\
45.68	0.01\\
45.69	0.01\\
45.7	0.01\\
45.71	0.01\\
45.72	0.01\\
45.73	0.01\\
45.74	0.01\\
45.75	0.01\\
45.76	0.01\\
45.77	0.01\\
45.78	0.01\\
45.79	0.01\\
45.8	0.01\\
45.81	0.01\\
45.82	0.01\\
45.83	0.01\\
45.84	0.01\\
45.85	0.01\\
45.86	0.01\\
45.87	0.01\\
45.88	0.01\\
45.89	0.01\\
45.9	0.01\\
45.91	0.01\\
45.92	0.01\\
45.93	0.01\\
45.94	0.01\\
45.95	0.01\\
45.96	0.01\\
45.97	0.01\\
45.98	0.01\\
45.99	0.01\\
46	0.01\\
46.01	0.01\\
46.02	0.01\\
46.03	0.01\\
46.04	0.01\\
46.05	0.01\\
46.06	0.01\\
46.07	0.01\\
46.08	0.01\\
46.09	0.01\\
46.1	0.01\\
46.11	0.01\\
46.12	0.01\\
46.13	0.01\\
46.14	0.01\\
46.15	0.01\\
46.16	0.01\\
46.17	0.01\\
46.18	0.01\\
46.19	0.01\\
46.2	0.01\\
46.21	0.01\\
46.22	0.01\\
46.23	0.01\\
46.24	0.01\\
46.25	0.01\\
46.26	0.01\\
46.27	0.01\\
46.28	0.01\\
46.29	0.01\\
46.3	0.01\\
46.31	0.01\\
46.32	0.01\\
46.33	0.01\\
46.34	0.01\\
46.35	0.01\\
46.36	0.01\\
46.37	0.01\\
46.38	0.01\\
46.39	0.01\\
46.4	0.01\\
46.41	0.01\\
46.42	0.01\\
46.43	0.01\\
46.44	0.01\\
46.45	0.01\\
46.46	0.01\\
46.47	0.01\\
46.48	0.01\\
46.49	0.01\\
46.5	0.01\\
46.51	0.01\\
46.52	0.01\\
46.53	0.01\\
46.54	0.01\\
46.55	0.01\\
46.56	0.01\\
46.57	0.01\\
46.58	0.01\\
46.59	0.01\\
46.6	0.01\\
46.61	0.01\\
46.62	0.01\\
46.63	0.01\\
46.64	0.01\\
46.65	0.01\\
46.66	0.01\\
46.67	0.01\\
46.68	0.01\\
46.69	0.01\\
46.7	0.01\\
46.71	0.01\\
46.72	0.01\\
46.73	0.01\\
46.74	0.01\\
46.75	0.01\\
46.76	0.01\\
46.77	0.01\\
46.78	0.01\\
46.79	0.01\\
46.8	0.01\\
46.81	0.01\\
46.82	0.01\\
46.83	0.01\\
46.84	0.01\\
46.85	0.01\\
46.86	0.01\\
46.87	0.01\\
46.88	0.01\\
46.89	0.01\\
46.9	0.01\\
46.91	0.01\\
46.92	0.01\\
46.93	0.01\\
46.94	0.01\\
46.95	0.01\\
46.96	0.01\\
46.97	0.01\\
46.98	0.01\\
46.99	0.01\\
47	0.01\\
47.01	0.01\\
47.02	0.01\\
47.03	0.01\\
47.04	0.01\\
47.05	0.01\\
47.06	0.01\\
47.07	0.01\\
47.08	0.01\\
47.09	0.01\\
47.1	0.01\\
47.11	0.01\\
47.12	0.01\\
47.13	0.01\\
47.14	0.01\\
47.15	0.01\\
47.16	0.01\\
47.17	0.01\\
47.18	0.01\\
47.19	0.01\\
47.2	0.01\\
47.21	0.01\\
47.22	0.01\\
47.23	0.01\\
47.24	0.01\\
47.25	0.01\\
47.26	0.01\\
47.27	0.01\\
47.28	0.01\\
47.29	0.01\\
47.3	0.01\\
47.31	0.01\\
47.32	0.01\\
47.33	0.01\\
47.34	0.01\\
47.35	0.01\\
47.36	0.01\\
47.37	0.01\\
47.38	0.01\\
47.39	0.01\\
47.4	0.01\\
47.41	0.01\\
47.42	0.01\\
47.43	0.01\\
47.44	0.01\\
47.45	0.01\\
47.46	0.01\\
47.47	0.01\\
47.48	0.01\\
47.49	0.01\\
47.5	0.01\\
47.51	0.01\\
47.52	0.01\\
47.53	0.01\\
47.54	0.01\\
47.55	0.01\\
47.56	0.01\\
47.57	0.01\\
47.58	0.01\\
47.59	0.01\\
47.6	0.01\\
47.61	0.01\\
47.62	0.01\\
47.63	0.01\\
47.64	0.01\\
47.65	0.01\\
47.66	0.01\\
47.67	0.01\\
47.68	0.01\\
47.69	0.01\\
47.7	0.01\\
47.71	0.01\\
47.72	0.01\\
47.73	0.01\\
47.74	0.01\\
47.75	0.01\\
47.76	0.01\\
47.77	0.01\\
47.78	0.01\\
47.79	0.01\\
47.8	0.01\\
47.81	0.01\\
47.82	0.01\\
47.83	0.01\\
47.84	0.01\\
47.85	0.01\\
47.86	0.01\\
47.87	0.01\\
47.88	0.01\\
47.89	0.01\\
47.9	0.01\\
47.91	0.01\\
47.92	0.01\\
47.93	0.01\\
47.94	0.01\\
47.95	0.01\\
47.96	0.01\\
47.97	0.01\\
47.98	0.01\\
47.99	0.01\\
48	0.01\\
48.01	0.01\\
48.02	0.01\\
48.03	0.01\\
48.04	0.01\\
48.05	0.01\\
48.06	0.01\\
48.07	0.01\\
48.08	0.01\\
48.09	0.01\\
48.1	0.01\\
48.11	0.01\\
48.12	0.01\\
48.13	0.01\\
48.14	0.01\\
48.15	0.01\\
48.16	0.01\\
48.17	0.01\\
48.18	0.01\\
48.19	0.01\\
48.2	0.01\\
48.21	0.01\\
48.22	0.01\\
48.23	0.01\\
48.24	0.01\\
48.25	0.01\\
48.26	0.01\\
48.27	0.01\\
48.28	0.01\\
48.29	0.01\\
48.3	0.01\\
48.31	0.01\\
48.32	0.01\\
48.33	0.01\\
48.34	0.01\\
48.35	0.01\\
48.36	0.01\\
48.37	0.01\\
48.38	0.01\\
48.39	0.01\\
48.4	0.01\\
48.41	0.01\\
48.42	0.01\\
48.43	0.01\\
48.44	0.01\\
48.45	0.01\\
48.46	0.01\\
48.47	0.01\\
48.48	0.01\\
48.49	0.01\\
48.5	0.01\\
48.51	0.01\\
48.52	0.01\\
48.53	0.01\\
48.54	0.01\\
48.55	0.01\\
48.56	0.01\\
48.57	0.01\\
48.58	0.01\\
48.59	0.01\\
48.6	0.01\\
48.61	0.01\\
48.62	0.01\\
48.63	0.01\\
48.64	0.01\\
48.65	0.01\\
48.66	0.01\\
48.67	0.01\\
48.68	0.01\\
48.69	0.01\\
48.7	0.01\\
48.71	0.01\\
48.72	0.01\\
48.73	0.01\\
48.74	0.01\\
48.75	0.01\\
48.76	0.01\\
48.77	0.01\\
48.78	0.01\\
48.79	0.01\\
48.8	0.01\\
48.81	0.01\\
48.82	0.01\\
48.83	0.01\\
48.84	0.01\\
48.85	0.01\\
48.86	0.01\\
48.87	0.01\\
48.88	0.01\\
48.89	0.01\\
48.9	0.01\\
48.91	0.01\\
48.92	0.01\\
48.93	0.01\\
48.94	0.01\\
48.95	0.01\\
48.96	0.01\\
48.97	0.01\\
48.98	0.01\\
48.99	0.01\\
49	0.01\\
49.01	0.01\\
49.02	0.01\\
49.03	0.01\\
49.04	0.01\\
49.05	0.01\\
49.06	0.01\\
49.07	0.01\\
49.08	0.01\\
49.09	0.01\\
49.1	0.01\\
49.11	0.01\\
49.12	0.01\\
49.13	0.01\\
49.14	0.01\\
49.15	0.01\\
49.16	0.01\\
49.17	0.01\\
49.18	0.01\\
49.19	0.01\\
49.2	0.01\\
49.21	0.01\\
49.22	0.01\\
49.23	0.01\\
49.24	0.01\\
49.25	0.01\\
49.26	0.01\\
49.27	0.01\\
49.28	0.01\\
49.29	0.01\\
49.3	0.01\\
49.31	0.01\\
49.32	0.01\\
49.33	0.01\\
49.34	0.01\\
49.35	0.01\\
49.36	0.01\\
49.37	0.01\\
49.38	0.01\\
49.39	0.01\\
49.4	0.01\\
49.41	0.01\\
49.42	0.01\\
49.43	0.01\\
49.44	0.01\\
49.45	0.01\\
49.46	0.01\\
49.47	0.01\\
49.48	0.01\\
49.49	0.01\\
49.5	0.01\\
49.51	0.01\\
49.52	0.01\\
49.53	0.01\\
49.54	0.01\\
49.55	0.01\\
49.56	0.01\\
49.57	0.01\\
49.58	0.01\\
49.59	0.01\\
49.6	0.01\\
49.61	0.01\\
49.62	0.01\\
49.63	0.01\\
49.64	0.01\\
49.65	0.01\\
49.66	0.01\\
49.67	0.01\\
49.68	0.01\\
49.69	0.01\\
49.7	0.01\\
49.71	0.01\\
49.72	0.01\\
49.73	0.01\\
49.74	0.01\\
49.75	0.01\\
49.76	0.01\\
49.77	0.01\\
49.78	0.01\\
49.79	0.01\\
49.8	0.01\\
49.81	0.01\\
49.82	0.01\\
49.83	0.01\\
49.84	0.01\\
49.85	0.01\\
49.86	0.01\\
49.87	0.01\\
49.88	0.01\\
49.89	0.01\\
49.9	0.01\\
49.91	0.01\\
49.92	0.01\\
49.93	0.01\\
49.94	0.01\\
49.95	0.01\\
49.96	0.01\\
49.97	0.01\\
49.98	0.01\\
49.99	0.01\\
50	0.01\\
50.01	0.01\\
50.02	0.01\\
50.03	0.01\\
50.04	0.01\\
50.05	0.01\\
50.06	0.01\\
50.07	0.01\\
50.08	0.01\\
50.09	0.01\\
50.1	0.01\\
50.11	0.01\\
50.12	0.01\\
50.13	0.01\\
50.14	0.01\\
50.15	0.01\\
50.16	0.01\\
50.17	0.01\\
50.18	0.01\\
50.19	0.01\\
50.2	0.01\\
50.21	0.01\\
50.22	0.01\\
50.23	0.01\\
50.24	0.01\\
50.25	0.01\\
50.26	0.01\\
50.27	0.01\\
50.28	0.01\\
50.29	0.01\\
50.3	0.01\\
50.31	0.01\\
50.32	0.01\\
50.33	0.01\\
50.34	0.01\\
50.35	0.01\\
50.36	0.01\\
50.37	0.01\\
50.38	0.01\\
50.39	0.01\\
50.4	0.01\\
50.41	0.01\\
50.42	0.01\\
50.43	0.01\\
50.44	0.01\\
50.45	0.01\\
50.46	0.01\\
50.47	0.01\\
50.48	0.01\\
50.49	0.01\\
50.5	0.01\\
50.51	0.01\\
50.52	0.01\\
50.53	0.01\\
50.54	0.01\\
50.55	0.01\\
50.56	0.01\\
50.57	0.01\\
50.58	0.01\\
50.59	0.01\\
50.6	0.01\\
50.61	0.01\\
50.62	0.01\\
50.63	0.01\\
50.64	0.01\\
50.65	0.01\\
50.66	0.01\\
50.67	0.01\\
50.68	0.01\\
50.69	0.01\\
50.7	0.01\\
50.71	0.01\\
50.72	0.01\\
50.73	0.01\\
50.74	0.01\\
50.75	0.01\\
50.76	0.01\\
50.77	0.01\\
50.78	0.01\\
50.79	0.01\\
50.8	0.01\\
50.81	0.01\\
50.82	0.01\\
50.83	0.01\\
50.84	0.01\\
50.85	0.01\\
50.86	0.01\\
50.87	0.01\\
50.88	0.01\\
50.89	0.01\\
50.9	0.01\\
50.91	0.01\\
50.92	0.01\\
50.93	0.01\\
50.94	0.01\\
50.95	0.01\\
50.96	0.01\\
50.97	0.01\\
50.98	0.01\\
50.99	0.01\\
51	0.01\\
51.01	0.01\\
51.02	0.01\\
51.03	0.01\\
51.04	0.01\\
51.05	0.01\\
51.06	0.01\\
51.07	0.01\\
51.08	0.01\\
51.09	0.01\\
51.1	0.01\\
51.11	0.01\\
51.12	0.01\\
51.13	0.01\\
51.14	0.01\\
51.15	0.01\\
51.16	0.01\\
51.17	0.01\\
51.18	0.01\\
51.19	0.01\\
51.2	0.01\\
51.21	0.01\\
51.22	0.01\\
51.23	0.01\\
51.24	0.01\\
51.25	0.01\\
51.26	0.01\\
51.27	0.01\\
51.28	0.01\\
51.29	0.01\\
51.3	0.01\\
51.31	0.01\\
51.32	0.01\\
51.33	0.01\\
51.34	0.01\\
51.35	0.01\\
51.36	0.01\\
51.37	0.01\\
51.38	0.01\\
51.39	0.01\\
51.4	0.01\\
51.41	0.01\\
51.42	0.01\\
51.43	0.01\\
51.44	0.01\\
51.45	0.01\\
51.46	0.01\\
51.47	0.01\\
51.48	0.01\\
51.49	0.01\\
51.5	0.01\\
51.51	0.01\\
51.52	0.01\\
51.53	0.01\\
51.54	0.01\\
51.55	0.01\\
51.56	0.01\\
51.57	0.01\\
51.58	0.01\\
51.59	0.01\\
51.6	0.01\\
51.61	0.01\\
51.62	0.01\\
51.63	0.01\\
51.64	0.01\\
51.65	0.01\\
51.66	0.01\\
51.67	0.01\\
51.68	0.01\\
51.69	0.01\\
51.7	0.01\\
51.71	0.01\\
51.72	0.01\\
51.73	0.01\\
51.74	0.01\\
51.75	0.01\\
51.76	0.01\\
51.77	0.01\\
51.78	0.01\\
51.79	0.01\\
51.8	0.01\\
51.81	0.01\\
51.82	0.01\\
51.83	0.01\\
51.84	0.01\\
51.85	0.01\\
51.86	0.01\\
51.87	0.01\\
51.88	0.01\\
51.89	0.01\\
51.9	0.01\\
51.91	0.01\\
51.92	0.01\\
51.93	0.01\\
51.94	0.01\\
51.95	0.01\\
51.96	0.01\\
51.97	0.01\\
51.98	0.01\\
51.99	0.01\\
52	0.01\\
52.01	0.01\\
52.02	0.01\\
52.03	0.01\\
52.04	0.01\\
52.05	0.01\\
52.06	0.01\\
52.07	0.01\\
52.08	0.01\\
52.09	0.01\\
52.1	0.01\\
52.11	0.01\\
52.12	0.01\\
52.13	0.01\\
52.14	0.01\\
52.15	0.01\\
52.16	0.01\\
52.17	0.01\\
52.18	0.01\\
52.19	0.01\\
52.2	0.01\\
52.21	0.01\\
52.22	0.01\\
52.23	0.01\\
52.24	0.01\\
52.25	0.01\\
52.26	0.01\\
52.27	0.01\\
52.28	0.01\\
52.29	0.01\\
52.3	0.01\\
52.31	0.01\\
52.32	0.01\\
52.33	0.01\\
52.34	0.01\\
52.35	0.01\\
52.36	0.01\\
52.37	0.01\\
52.38	0.01\\
52.39	0.01\\
52.4	0.01\\
52.41	0.01\\
52.42	0.01\\
52.43	0.01\\
52.44	0.01\\
52.45	0.01\\
52.46	0.01\\
52.47	0.01\\
52.48	0.01\\
52.49	0.01\\
52.5	0.01\\
52.51	0.01\\
52.52	0.01\\
52.53	0.01\\
52.54	0.01\\
52.55	0.01\\
52.56	0.01\\
52.57	0.01\\
52.58	0.01\\
52.59	0.01\\
52.6	0.01\\
52.61	0.01\\
52.62	0.01\\
52.63	0.01\\
52.64	0.01\\
52.65	0.01\\
52.66	0.01\\
52.67	0.01\\
52.68	0.01\\
52.69	0.01\\
52.7	0.01\\
52.71	0.01\\
52.72	0.01\\
52.73	0.01\\
52.74	0.01\\
52.75	0.01\\
52.76	0.01\\
52.77	0.01\\
52.78	0.01\\
52.79	0.01\\
52.8	0.01\\
52.81	0.01\\
52.82	0.01\\
52.83	0.01\\
52.84	0.01\\
52.85	0.01\\
52.86	0.01\\
52.87	0.01\\
52.88	0.01\\
52.89	0.01\\
52.9	0.01\\
52.91	0.01\\
52.92	0.01\\
52.93	0.01\\
52.94	0.01\\
52.95	0.01\\
52.96	0.01\\
52.97	0.01\\
52.98	0.01\\
52.99	0.01\\
53	0.01\\
53.01	0.01\\
53.02	0.01\\
53.03	0.01\\
53.04	0.01\\
53.05	0.01\\
53.06	0.01\\
53.07	0.01\\
53.08	0.01\\
53.09	0.01\\
53.1	0.01\\
53.11	0.01\\
53.12	0.01\\
53.13	0.01\\
53.14	0.01\\
53.15	0.01\\
53.16	0.01\\
53.17	0.01\\
53.18	0.01\\
53.19	0.01\\
53.2	0.01\\
53.21	0.01\\
53.22	0.01\\
53.23	0.01\\
53.24	0.01\\
53.25	0.01\\
53.26	0.01\\
53.27	0.01\\
53.28	0.01\\
53.29	0.01\\
53.3	0.01\\
53.31	0.01\\
53.32	0.01\\
53.33	0.01\\
53.34	0.01\\
53.35	0.01\\
53.36	0.01\\
53.37	0.01\\
53.38	0.01\\
53.39	0.01\\
53.4	0.01\\
53.41	0.01\\
53.42	0.01\\
53.43	0.01\\
53.44	0.01\\
53.45	0.01\\
53.46	0.01\\
53.47	0.01\\
53.48	0.01\\
53.49	0.01\\
53.5	0.01\\
53.51	0.01\\
53.52	0.01\\
53.53	0.01\\
53.54	0.01\\
53.55	0.01\\
53.56	0.01\\
53.57	0.01\\
53.58	0.01\\
53.59	0.01\\
53.6	0.01\\
53.61	0.01\\
53.62	0.01\\
53.63	0.01\\
53.64	0.01\\
53.65	0.01\\
53.66	0.01\\
53.67	0.01\\
53.68	0.01\\
53.69	0.01\\
53.7	0.01\\
53.71	0.01\\
53.72	0.01\\
53.73	0.01\\
53.74	0.01\\
53.75	0.01\\
53.76	0.01\\
53.77	0.01\\
53.78	0.01\\
53.79	0.01\\
53.8	0.01\\
53.81	0.01\\
53.82	0.01\\
53.83	0.01\\
53.84	0.01\\
53.85	0.01\\
53.86	0.01\\
53.87	0.01\\
53.88	0.01\\
53.89	0.01\\
53.9	0.01\\
53.91	0.01\\
53.92	0.01\\
53.93	0.01\\
53.94	0.01\\
53.95	0.01\\
53.96	0.01\\
53.97	0.01\\
53.98	0.01\\
53.99	0.01\\
54	0.01\\
54.01	0.01\\
54.02	0.01\\
54.03	0.01\\
54.04	0.01\\
54.05	0.01\\
54.06	0.01\\
54.07	0.01\\
54.08	0.01\\
54.09	0.01\\
54.1	0.01\\
54.11	0.01\\
54.12	0.01\\
54.13	0.01\\
54.14	0.01\\
54.15	0.01\\
54.16	0.01\\
54.17	0.01\\
54.18	0.01\\
54.19	0.01\\
54.2	0.01\\
54.21	0.01\\
54.22	0.01\\
54.23	0.01\\
54.24	0.01\\
54.25	0.01\\
54.26	0.01\\
54.27	0.01\\
54.28	0.01\\
54.29	0.01\\
54.3	0.01\\
54.31	0.01\\
54.32	0.01\\
54.33	0.01\\
54.34	0.01\\
54.35	0.01\\
54.36	0.01\\
54.37	0.01\\
54.38	0.01\\
54.39	0.01\\
54.4	0.01\\
54.41	0.01\\
54.42	0.01\\
54.43	0.01\\
54.44	0.01\\
54.45	0.01\\
54.46	0.01\\
54.47	0.01\\
54.48	0.01\\
54.49	0.01\\
54.5	0.01\\
54.51	0.01\\
54.52	0.01\\
54.53	0.01\\
54.54	0.01\\
54.55	0.01\\
54.56	0.01\\
54.57	0.01\\
54.58	0.01\\
54.59	0.01\\
54.6	0.01\\
54.61	0.01\\
54.62	0.01\\
54.63	0.01\\
54.64	0.01\\
54.65	0.01\\
54.66	0.01\\
54.67	0.01\\
54.68	0.01\\
54.69	0.01\\
54.7	0.01\\
54.71	0.01\\
54.72	0.01\\
54.73	0.01\\
54.74	0.01\\
54.75	0.01\\
54.76	0.01\\
54.77	0.01\\
54.78	0.01\\
54.79	0.01\\
54.8	0.01\\
54.81	0.01\\
54.82	0.01\\
54.83	0.01\\
54.84	0.01\\
54.85	0.01\\
54.86	0.01\\
54.87	0.01\\
54.88	0.01\\
54.89	0.01\\
54.9	0.01\\
54.91	0.01\\
54.92	0.01\\
54.93	0.01\\
54.94	0.01\\
54.95	0.01\\
54.96	0.01\\
54.97	0.01\\
54.98	0.01\\
54.99	0.01\\
55	0.01\\
55.01	0.01\\
55.02	0.01\\
55.03	0.01\\
55.04	0.01\\
55.05	0.01\\
55.06	0.01\\
55.07	0.01\\
55.08	0.01\\
55.09	0.01\\
55.1	0.01\\
55.11	0.01\\
55.12	0.01\\
55.13	0.01\\
55.14	0.01\\
55.15	0.01\\
55.16	0.01\\
55.17	0.01\\
55.18	0.01\\
55.19	0.01\\
55.2	0.01\\
55.21	0.01\\
55.22	0.01\\
55.23	0.01\\
55.24	0.01\\
55.25	0.01\\
55.26	0.01\\
55.27	0.01\\
55.28	0.01\\
55.29	0.01\\
55.3	0.01\\
55.31	0.01\\
55.32	0.01\\
55.33	0.01\\
55.34	0.01\\
55.35	0.01\\
55.36	0.01\\
55.37	0.01\\
55.38	0.01\\
55.39	0.01\\
55.4	0.01\\
55.41	0.01\\
55.42	0.01\\
55.43	0.01\\
55.44	0.01\\
55.45	0.01\\
55.46	0.01\\
55.47	0.01\\
55.48	0.01\\
55.49	0.01\\
55.5	0.01\\
55.51	0.01\\
55.52	0.01\\
55.53	0.01\\
55.54	0.01\\
55.55	0.01\\
55.56	0.01\\
55.57	0.01\\
55.58	0.01\\
55.59	0.01\\
55.6	0.01\\
55.61	0.01\\
55.62	0.01\\
55.63	0.01\\
55.64	0.01\\
55.65	0.01\\
55.66	0.01\\
55.67	0.01\\
55.68	0.01\\
55.69	0.01\\
55.7	0.01\\
55.71	0.01\\
55.72	0.01\\
55.73	0.01\\
55.74	0.01\\
55.75	0.01\\
55.76	0.01\\
55.77	0.01\\
55.78	0.01\\
55.79	0.01\\
55.8	0.01\\
55.81	0.01\\
55.82	0.01\\
55.83	0.01\\
55.84	0.01\\
55.85	0.01\\
55.86	0.01\\
55.87	0.01\\
55.88	0.01\\
55.89	0.01\\
55.9	0.01\\
55.91	0.01\\
55.92	0.01\\
55.93	0.01\\
55.94	0.01\\
55.95	0.01\\
55.96	0.01\\
55.97	0.01\\
55.98	0.01\\
55.99	0.01\\
56	0.01\\
56.01	0.01\\
56.02	0.01\\
56.03	0.01\\
56.04	0.01\\
56.05	0.01\\
56.06	0.01\\
56.07	0.01\\
56.08	0.01\\
56.09	0.01\\
56.1	0.01\\
56.11	0.01\\
56.12	0.01\\
56.13	0.01\\
56.14	0.01\\
56.15	0.01\\
56.16	0.01\\
56.17	0.01\\
56.18	0.01\\
56.19	0.01\\
56.2	0.01\\
56.21	0.01\\
56.22	0.01\\
56.23	0.01\\
56.24	0.01\\
56.25	0.01\\
56.26	0.01\\
56.27	0.01\\
56.28	0.01\\
56.29	0.01\\
56.3	0.01\\
56.31	0.01\\
56.32	0.01\\
56.33	0.01\\
56.34	0.01\\
56.35	0.01\\
56.36	0.01\\
56.37	0.01\\
56.38	0.01\\
56.39	0.01\\
56.4	0.01\\
56.41	0.01\\
56.42	0.01\\
56.43	0.01\\
56.44	0.01\\
56.45	0.01\\
56.46	0.01\\
56.47	0.01\\
56.48	0.01\\
56.49	0.01\\
56.5	0.01\\
56.51	0.01\\
56.52	0.01\\
56.53	0.01\\
56.54	0.01\\
56.55	0.01\\
56.56	0.01\\
56.57	0.01\\
56.58	0.01\\
56.59	0.01\\
56.6	0.01\\
56.61	0.01\\
56.62	0.01\\
56.63	0.01\\
56.64	0.01\\
56.65	0.01\\
56.66	0.01\\
56.67	0.01\\
56.68	0.01\\
56.69	0.01\\
56.7	0.01\\
56.71	0.01\\
56.72	0.01\\
56.73	0.01\\
56.74	0.01\\
56.75	0.01\\
56.76	0.01\\
56.77	0.01\\
56.78	0.01\\
56.79	0.01\\
56.8	0.01\\
56.81	0.01\\
56.82	0.01\\
56.83	0.01\\
56.84	0.01\\
56.85	0.01\\
56.86	0.01\\
56.87	0.01\\
56.88	0.01\\
56.89	0.01\\
56.9	0.01\\
56.91	0.01\\
56.92	0.01\\
56.93	0.01\\
56.94	0.01\\
56.95	0.01\\
56.96	0.01\\
56.97	0.01\\
56.98	0.01\\
56.99	0.01\\
57	0.01\\
57.01	0.01\\
57.02	0.01\\
57.03	0.01\\
57.04	0.01\\
57.05	0.01\\
57.06	0.01\\
57.07	0.01\\
57.08	0.01\\
57.09	0.01\\
57.1	0.01\\
57.11	0.01\\
57.12	0.01\\
57.13	0.01\\
57.14	0.01\\
57.15	0.01\\
57.16	0.01\\
57.17	0.01\\
57.18	0.01\\
57.19	0.01\\
57.2	0.01\\
57.21	0.01\\
57.22	0.01\\
57.23	0.01\\
57.24	0.01\\
57.25	0.01\\
57.26	0.01\\
57.27	0.01\\
57.28	0.01\\
57.29	0.01\\
57.3	0.01\\
57.31	0.01\\
57.32	0.01\\
57.33	0.01\\
57.34	0.01\\
57.35	0.01\\
57.36	0.01\\
57.37	0.01\\
57.38	0.01\\
57.39	0.01\\
57.4	0.01\\
57.41	0.01\\
57.42	0.01\\
57.43	0.01\\
57.44	0.01\\
57.45	0.01\\
57.46	0.01\\
57.47	0.01\\
57.48	0.01\\
57.49	0.01\\
57.5	0.01\\
57.51	0.01\\
57.52	0.01\\
57.53	0.01\\
57.54	0.01\\
57.55	0.01\\
57.56	0.01\\
57.57	0.01\\
57.58	0.01\\
57.59	0.01\\
57.6	0.01\\
57.61	0.01\\
57.62	0.01\\
57.63	0.01\\
57.64	0.01\\
57.65	0.01\\
57.66	0.01\\
57.67	0.01\\
57.68	0.01\\
57.69	0.01\\
57.7	0.01\\
57.71	0.01\\
57.72	0.01\\
57.73	0.01\\
57.74	0.01\\
57.75	0.01\\
57.76	0.01\\
57.77	0.01\\
57.78	0.01\\
57.79	0.01\\
57.8	0.01\\
57.81	0.01\\
57.82	0.01\\
57.83	0.01\\
57.84	0.01\\
57.85	0.01\\
57.86	0.01\\
57.87	0.01\\
57.88	0.01\\
57.89	0.01\\
57.9	0.01\\
57.91	0.01\\
57.92	0.01\\
57.93	0.01\\
57.94	0.01\\
57.95	0.01\\
57.96	0.01\\
57.97	0.01\\
57.98	0.01\\
57.99	0.01\\
58	0.01\\
58.01	0.01\\
58.02	0.01\\
58.03	0.01\\
58.04	0.01\\
58.05	0.01\\
58.06	0.01\\
58.07	0.01\\
58.08	0.01\\
58.09	0.01\\
58.1	0.01\\
58.11	0.01\\
58.12	0.01\\
58.13	0.01\\
58.14	0.01\\
58.15	0.01\\
58.16	0.01\\
58.17	0.01\\
58.18	0.01\\
58.19	0.01\\
58.2	0.01\\
58.21	0.01\\
58.22	0.01\\
58.23	0.01\\
58.24	0.01\\
58.25	0.01\\
58.26	0.01\\
58.27	0.01\\
58.28	0.01\\
58.29	0.01\\
58.3	0.01\\
58.31	0.01\\
58.32	0.01\\
58.33	0.01\\
58.34	0.01\\
58.35	0.01\\
58.36	0.01\\
58.37	0.01\\
58.38	0.01\\
58.39	0.01\\
58.4	0.01\\
58.41	0.01\\
58.42	0.01\\
58.43	0.01\\
58.44	0.01\\
58.45	0.01\\
58.46	0.01\\
58.47	0.01\\
58.48	0.01\\
58.49	0.01\\
58.5	0.01\\
58.51	0.01\\
58.52	0.01\\
58.53	0.01\\
58.54	0.01\\
58.55	0.01\\
58.56	0.01\\
58.57	0.01\\
58.58	0.01\\
58.59	0.01\\
58.6	0.01\\
58.61	0.01\\
58.62	0.01\\
58.63	0.01\\
58.64	0.01\\
58.65	0.01\\
58.66	0.01\\
58.67	0.01\\
58.68	0.01\\
58.69	0.01\\
58.7	0.01\\
58.71	0.01\\
58.72	0.01\\
58.73	0.01\\
58.74	0.01\\
58.75	0.01\\
58.76	0.01\\
58.77	0.01\\
58.78	0.01\\
58.79	0.01\\
58.8	0.01\\
58.81	0.01\\
58.82	0.01\\
58.83	0.01\\
58.84	0.01\\
58.85	0.01\\
58.86	0.01\\
58.87	0.01\\
58.88	0.01\\
58.89	0.01\\
58.9	0.01\\
58.91	0.01\\
58.92	0.01\\
58.93	0.01\\
58.94	0.01\\
58.95	0.01\\
58.96	0.01\\
58.97	0.01\\
58.98	0.01\\
58.99	0.01\\
59	0.01\\
59.01	0.01\\
59.02	0.01\\
59.03	0.01\\
59.04	0.01\\
59.05	0.01\\
59.06	0.01\\
59.07	0.01\\
59.08	0.01\\
59.09	0.01\\
59.1	0.01\\
59.11	0.01\\
59.12	0.01\\
59.13	0.01\\
59.14	0.01\\
59.15	0.01\\
59.16	0.01\\
59.17	0.01\\
59.18	0.01\\
59.19	0.01\\
59.2	0.01\\
59.21	0.01\\
59.22	0.01\\
59.23	0.01\\
59.24	0.01\\
59.25	0.01\\
59.26	0.01\\
59.27	0.01\\
59.28	0.01\\
59.29	0.01\\
59.3	0.01\\
59.31	0.01\\
59.32	0.01\\
59.33	0.01\\
59.34	0.01\\
59.35	0.01\\
59.36	0.01\\
59.37	0.01\\
59.38	0.01\\
59.39	0.01\\
59.4	0.01\\
59.41	0.01\\
59.42	0.01\\
59.43	0.01\\
59.44	0.01\\
59.45	0.01\\
59.46	0.01\\
59.47	0.01\\
59.48	0.01\\
59.49	0.01\\
59.5	0.01\\
59.51	0.01\\
59.52	0.01\\
59.53	0.01\\
59.54	0.01\\
59.55	0.01\\
59.56	0.01\\
59.57	0.01\\
59.58	0.01\\
59.59	0.01\\
59.6	0.01\\
59.61	0.01\\
59.62	0.01\\
59.63	0.01\\
59.64	0.01\\
59.65	0.01\\
59.66	0.01\\
59.67	0.01\\
59.68	0.01\\
59.69	0.01\\
59.7	0.01\\
59.71	0.01\\
59.72	0.01\\
59.73	0.01\\
59.74	0.01\\
59.75	0.01\\
59.76	0.01\\
59.77	0.01\\
59.78	0.01\\
59.79	0.01\\
59.8	0.01\\
59.81	0.01\\
59.82	0.01\\
59.83	0.01\\
59.84	0.01\\
59.85	0.01\\
59.86	0.01\\
59.87	0.01\\
59.88	0.01\\
59.89	0.01\\
59.9	0.01\\
59.91	0.01\\
59.92	0.01\\
59.93	0.01\\
59.94	0.01\\
59.95	0.01\\
59.96	0.01\\
59.97	0.01\\
59.98	0.01\\
59.99	0.01\\
60	0.01\\
60.01	0.01\\
60.02	0.01\\
60.03	0.01\\
60.04	0.01\\
60.05	0.01\\
60.06	0.01\\
60.07	0.01\\
60.08	0.01\\
60.09	0.01\\
60.1	0.01\\
60.11	0.01\\
60.12	0.01\\
60.13	0.01\\
60.14	0.01\\
60.15	0.01\\
60.16	0.01\\
60.17	0.01\\
60.18	0.01\\
60.19	0.01\\
60.2	0.01\\
60.21	0.01\\
60.22	0.01\\
60.23	0.01\\
60.24	0.01\\
60.25	0.01\\
60.26	0.01\\
60.27	0.01\\
60.28	0.01\\
60.29	0.01\\
60.3	0.01\\
60.31	0.01\\
60.32	0.01\\
60.33	0.01\\
60.34	0.01\\
60.35	0.01\\
60.36	0.01\\
60.37	0.01\\
60.38	0.01\\
60.39	0.01\\
60.4	0.01\\
60.41	0.01\\
60.42	0.01\\
60.43	0.01\\
60.44	0.01\\
60.45	0.01\\
60.46	0.01\\
60.47	0.01\\
60.48	0.01\\
60.49	0.01\\
60.5	0.01\\
60.51	0.01\\
60.52	0.01\\
60.53	0.01\\
60.54	0.01\\
60.55	0.01\\
60.56	0.01\\
60.57	0.01\\
60.58	0.01\\
60.59	0.01\\
60.6	0.01\\
60.61	0.01\\
60.62	0.01\\
60.63	0.01\\
60.64	0.01\\
60.65	0.01\\
60.66	0.01\\
60.67	0.01\\
60.68	0.01\\
60.69	0.01\\
60.7	0.01\\
60.71	0.01\\
60.72	0.01\\
60.73	0.01\\
60.74	0.01\\
60.75	0.01\\
60.76	0.01\\
60.77	0.01\\
60.78	0.01\\
60.79	0.01\\
60.8	0.01\\
60.81	0.01\\
60.82	0.01\\
60.83	0.01\\
60.84	0.01\\
60.85	0.01\\
60.86	0.01\\
60.87	0.01\\
60.88	0.01\\
60.89	0.01\\
60.9	0.01\\
60.91	0.01\\
60.92	0.01\\
60.93	0.01\\
60.94	0.01\\
60.95	0.01\\
60.96	0.01\\
60.97	0.01\\
60.98	0.01\\
60.99	0.01\\
61	0.01\\
61.01	0.01\\
61.02	0.01\\
61.03	0.01\\
61.04	0.01\\
61.05	0.01\\
61.06	0.01\\
61.07	0.01\\
61.08	0.01\\
61.09	0.01\\
61.1	0.01\\
61.11	0.01\\
61.12	0.01\\
61.13	0.01\\
61.14	0.01\\
61.15	0.01\\
61.16	0.01\\
61.17	0.01\\
61.18	0.01\\
61.19	0.01\\
61.2	0.01\\
61.21	0.01\\
61.22	0.01\\
61.23	0.01\\
61.24	0.01\\
61.25	0.01\\
61.26	0.01\\
61.27	0.01\\
61.28	0.01\\
61.29	0.01\\
61.3	0.01\\
61.31	0.01\\
61.32	0.01\\
61.33	0.01\\
61.34	0.01\\
61.35	0.01\\
61.36	0.01\\
61.37	0.01\\
61.38	0.01\\
61.39	0.01\\
61.4	0.01\\
61.41	0.01\\
61.42	0.01\\
61.43	0.01\\
61.44	0.01\\
61.45	0.01\\
61.46	0.01\\
61.47	0.01\\
61.48	0.01\\
61.49	0.01\\
61.5	0.01\\
61.51	0.01\\
61.52	0.01\\
61.53	0.01\\
61.54	0.01\\
61.55	0.01\\
61.56	0.01\\
61.57	0.01\\
61.58	0.01\\
61.59	0.01\\
61.6	0.01\\
61.61	0.01\\
61.62	0.01\\
61.63	0.01\\
61.64	0.01\\
61.65	0.01\\
61.66	0.01\\
61.67	0.01\\
61.68	0.01\\
61.69	0.01\\
61.7	0.01\\
61.71	0.01\\
61.72	0.01\\
61.73	0.01\\
61.74	0.01\\
61.75	0.01\\
61.76	0.01\\
61.77	0.01\\
61.78	0.01\\
61.79	0.01\\
61.8	0.01\\
61.81	0.01\\
61.82	0.01\\
61.83	0.01\\
61.84	0.01\\
61.85	0.01\\
61.86	0.01\\
61.87	0.01\\
61.88	0.01\\
61.89	0.01\\
61.9	0.01\\
61.91	0.01\\
61.92	0.01\\
61.93	0.01\\
61.94	0.01\\
61.95	0.01\\
61.96	0.01\\
61.97	0.01\\
61.98	0.01\\
61.99	0.01\\
62	0.01\\
62.01	0.01\\
62.02	0.01\\
62.03	0.01\\
62.04	0.01\\
62.05	0.01\\
62.06	0.01\\
62.07	0.01\\
62.08	0.01\\
62.09	0.01\\
62.1	0.01\\
62.11	0.01\\
62.12	0.01\\
62.13	0.01\\
62.14	0.01\\
62.15	0.01\\
62.16	0.01\\
62.17	0.01\\
62.18	0.01\\
62.19	0.01\\
62.2	0.01\\
62.21	0.01\\
62.22	0.01\\
62.23	0.01\\
62.24	0.01\\
62.25	0.01\\
62.26	0.01\\
62.27	0.01\\
62.28	0.01\\
62.29	0.01\\
62.3	0.01\\
62.31	0.01\\
62.32	0.01\\
62.33	0.01\\
62.34	0.01\\
62.35	0.01\\
62.36	0.01\\
62.37	0.01\\
62.38	0.01\\
62.39	0.01\\
62.4	0.01\\
62.41	0.01\\
62.42	0.01\\
62.43	0.01\\
62.44	0.01\\
62.45	0.01\\
62.46	0.01\\
62.47	0.01\\
62.48	0.01\\
62.49	0.01\\
62.5	0.01\\
62.51	0.01\\
62.52	0.01\\
62.53	0.01\\
62.54	0.01\\
62.55	0.01\\
62.56	0.01\\
62.57	0.01\\
62.58	0.01\\
62.59	0.01\\
62.6	0.01\\
62.61	0.01\\
62.62	0.01\\
62.63	0.01\\
62.64	0.01\\
62.65	0.01\\
62.66	0.01\\
62.67	0.01\\
62.68	0.01\\
62.69	0.01\\
62.7	0.01\\
62.71	0.01\\
62.72	0.01\\
62.73	0.01\\
62.74	0.01\\
62.75	0.01\\
62.76	0.01\\
62.77	0.01\\
62.78	0.01\\
62.79	0.01\\
62.8	0.01\\
62.81	0.01\\
62.82	0.01\\
62.83	0.01\\
62.84	0.01\\
62.85	0.01\\
62.86	0.01\\
62.87	0.01\\
62.88	0.01\\
62.89	0.01\\
62.9	0.01\\
62.91	0.01\\
62.92	0.01\\
62.93	0.01\\
62.94	0.01\\
62.95	0.01\\
62.96	0.01\\
62.97	0.01\\
62.98	0.01\\
62.99	0.01\\
63	0.01\\
63.01	0.01\\
63.02	0.01\\
63.03	0.01\\
63.04	0.01\\
63.05	0.01\\
63.06	0.01\\
63.07	0.01\\
63.08	0.01\\
63.09	0.01\\
63.1	0.01\\
63.11	0.01\\
63.12	0.01\\
63.13	0.01\\
63.14	0.01\\
63.15	0.01\\
63.16	0.01\\
63.17	0.01\\
63.18	0.01\\
63.19	0.01\\
63.2	0.01\\
63.21	0.01\\
63.22	0.01\\
63.23	0.01\\
63.24	0.01\\
63.25	0.01\\
63.26	0.01\\
63.27	0.01\\
63.28	0.01\\
63.29	0.01\\
63.3	0.01\\
63.31	0.01\\
63.32	0.01\\
63.33	0.01\\
63.34	0.01\\
63.35	0.01\\
63.36	0.01\\
63.37	0.01\\
63.38	0.01\\
63.39	0.01\\
63.4	0.01\\
63.41	0.01\\
63.42	0.01\\
63.43	0.01\\
63.44	0.01\\
63.45	0.01\\
63.46	0.01\\
63.47	0.01\\
63.48	0.01\\
63.49	0.01\\
63.5	0.01\\
63.51	0.01\\
63.52	0.01\\
63.53	0.01\\
63.54	0.01\\
63.55	0.01\\
63.56	0.01\\
63.57	0.01\\
63.58	0.01\\
63.59	0.01\\
63.6	0.01\\
63.61	0.01\\
63.62	0.01\\
63.63	0.01\\
63.64	0.01\\
63.65	0.01\\
63.66	0.01\\
63.67	0.01\\
63.68	0.01\\
63.69	0.01\\
63.7	0.01\\
63.71	0.01\\
63.72	0.01\\
63.73	0.01\\
63.74	0.01\\
63.75	0.01\\
63.76	0.01\\
63.77	0.01\\
63.78	0.01\\
63.79	0.01\\
63.8	0.01\\
63.81	0.01\\
63.82	0.01\\
63.83	0.01\\
63.84	0.01\\
63.85	0.01\\
63.86	0.01\\
63.87	0.01\\
63.88	0.01\\
63.89	0.01\\
63.9	0.01\\
63.91	0.01\\
63.92	0.01\\
63.93	0.01\\
63.94	0.01\\
63.95	0.01\\
63.96	0.01\\
63.97	0.01\\
63.98	0.01\\
63.99	0.01\\
64	0.01\\
64.01	0.01\\
64.02	0.01\\
64.03	0.01\\
64.04	0.01\\
64.05	0.01\\
64.06	0.01\\
64.07	0.01\\
64.08	0.01\\
64.09	0.01\\
64.1	0.01\\
64.11	0.01\\
64.12	0.01\\
64.13	0.01\\
64.14	0.01\\
64.15	0.01\\
64.16	0.01\\
64.17	0.01\\
64.18	0.01\\
64.19	0.01\\
64.2	0.01\\
64.21	0.01\\
64.22	0.01\\
64.23	0.01\\
64.24	0.01\\
64.25	0.01\\
64.26	0.01\\
64.27	0.01\\
64.28	0.01\\
64.29	0.01\\
64.3	0.01\\
64.31	0.01\\
64.32	0.01\\
64.33	0.01\\
64.34	0.01\\
64.35	0.01\\
64.36	0.01\\
64.37	0.01\\
64.38	0.01\\
64.39	0.01\\
64.4	0.01\\
64.41	0.01\\
64.42	0.01\\
64.43	0.01\\
64.44	0.01\\
64.45	0.01\\
64.46	0.01\\
64.47	0.01\\
64.48	0.01\\
64.49	0.01\\
64.5	0.01\\
64.51	0.01\\
64.52	0.01\\
64.53	0.01\\
64.54	0.01\\
64.55	0.01\\
64.56	0.01\\
64.57	0.01\\
64.58	0.01\\
64.59	0.01\\
64.6	0.01\\
64.61	0.01\\
64.62	0.01\\
64.63	0.01\\
64.64	0.01\\
64.65	0.01\\
64.66	0.01\\
64.67	0.01\\
64.68	0.01\\
64.69	0.01\\
64.7	0.01\\
64.71	0.01\\
64.72	0.01\\
64.73	0.01\\
64.74	0.01\\
64.75	0.01\\
64.76	0.01\\
64.77	0.01\\
64.78	0.01\\
64.79	0.01\\
64.8	0.01\\
64.81	0.01\\
64.82	0.01\\
64.83	0.01\\
64.84	0.01\\
64.85	0.01\\
64.86	0.01\\
64.87	0.01\\
64.88	0.01\\
64.89	0.01\\
64.9	0.01\\
64.91	0.01\\
64.92	0.01\\
64.93	0.01\\
64.94	0.01\\
64.95	0.01\\
64.96	0.01\\
64.97	0.01\\
64.98	0.01\\
64.99	0.01\\
65	0.01\\
65.01	0.01\\
65.02	0.01\\
65.03	0.01\\
65.04	0.01\\
65.05	0.01\\
65.06	0.01\\
65.07	0.01\\
65.08	0.01\\
65.09	0.01\\
65.1	0.01\\
65.11	0.01\\
65.12	0.01\\
65.13	0.01\\
65.14	0.01\\
65.15	0.01\\
65.16	0.01\\
65.17	0.01\\
65.18	0.01\\
65.19	0.01\\
65.2	0.01\\
65.21	0.01\\
65.22	0.01\\
65.23	0.01\\
65.24	0.01\\
65.25	0.01\\
65.26	0.01\\
65.27	0.01\\
65.28	0.01\\
65.29	0.01\\
65.3	0.01\\
65.31	0.01\\
65.32	0.01\\
65.33	0.01\\
65.34	0.01\\
65.35	0.01\\
65.36	0.01\\
65.37	0.01\\
65.38	0.01\\
65.39	0.01\\
65.4	0.01\\
65.41	0.01\\
65.42	0.01\\
65.43	0.01\\
65.44	0.01\\
65.45	0.01\\
65.46	0.01\\
65.47	0.01\\
65.48	0.01\\
65.49	0.01\\
65.5	0.01\\
65.51	0.01\\
65.52	0.01\\
65.53	0.01\\
65.54	0.01\\
65.55	0.01\\
65.56	0.01\\
65.57	0.01\\
65.58	0.01\\
65.59	0.01\\
65.6	0.01\\
65.61	0.01\\
65.62	0.01\\
65.63	0.01\\
65.64	0.01\\
65.65	0.01\\
65.66	0.01\\
65.67	0.01\\
65.68	0.01\\
65.69	0.01\\
65.7	0.01\\
65.71	0.01\\
65.72	0.01\\
65.73	0.01\\
65.74	0.01\\
65.75	0.01\\
65.76	0.01\\
65.77	0.01\\
65.78	0.01\\
65.79	0.01\\
65.8	0.01\\
65.81	0.01\\
65.82	0.01\\
65.83	0.01\\
65.84	0.01\\
65.85	0.01\\
65.86	0.01\\
65.87	0.01\\
65.88	0.01\\
65.89	0.01\\
65.9	0.01\\
65.91	0.01\\
65.92	0.01\\
65.93	0.01\\
65.94	0.01\\
65.95	0.01\\
65.96	0.01\\
65.97	0.01\\
65.98	0.01\\
65.99	0.01\\
66	0.01\\
66.01	0.01\\
66.02	0.01\\
66.03	0.01\\
66.04	0.01\\
66.05	0.01\\
66.06	0.01\\
66.07	0.01\\
66.08	0.01\\
66.09	0.01\\
66.1	0.01\\
66.11	0.01\\
66.12	0.01\\
66.13	0.01\\
66.14	0.01\\
66.15	0.01\\
66.16	0.01\\
66.17	0.01\\
66.18	0.01\\
66.19	0.01\\
66.2	0.01\\
66.21	0.01\\
66.22	0.01\\
66.23	0.01\\
66.24	0.01\\
66.25	0.01\\
66.26	0.01\\
66.27	0.01\\
66.28	0.01\\
66.29	0.01\\
66.3	0.01\\
66.31	0.01\\
66.32	0.01\\
66.33	0.01\\
66.34	0.01\\
66.35	0.01\\
66.36	0.01\\
66.37	0.01\\
66.38	0.01\\
66.39	0.01\\
66.4	0.01\\
66.41	0.01\\
66.42	0.01\\
66.43	0.01\\
66.44	0.01\\
66.45	0.01\\
66.46	0.01\\
66.47	0.01\\
66.48	0.01\\
66.49	0.01\\
66.5	0.01\\
66.51	0.01\\
66.52	0.01\\
66.53	0.01\\
66.54	0.01\\
66.55	0.01\\
66.56	0.01\\
66.57	0.01\\
66.58	0.01\\
66.59	0.01\\
66.6	0.01\\
66.61	0.01\\
66.62	0.01\\
66.63	0.01\\
66.64	0.01\\
66.65	0.01\\
66.66	0.01\\
66.67	0.01\\
66.68	0.01\\
66.69	0.01\\
66.7	0.01\\
66.71	0.01\\
66.72	0.01\\
66.73	0.01\\
66.74	0.01\\
66.75	0.01\\
66.76	0.01\\
66.77	0.01\\
66.78	0.01\\
66.79	0.01\\
66.8	0.01\\
66.81	0.01\\
66.82	0.01\\
66.83	0.01\\
66.84	0.01\\
66.85	0.01\\
66.86	0.01\\
66.87	0.01\\
66.88	0.01\\
66.89	0.01\\
66.9	0.01\\
66.91	0.01\\
66.92	0.01\\
66.93	0.01\\
66.94	0.01\\
66.95	0.01\\
66.96	0.01\\
66.97	0.01\\
66.98	0.01\\
66.99	0.01\\
67	0.01\\
67.01	0.01\\
67.02	0.01\\
67.03	0.01\\
67.04	0.01\\
67.05	0.01\\
67.06	0.01\\
67.07	0.01\\
67.08	0.01\\
67.09	0.01\\
67.1	0.01\\
67.11	0.01\\
67.12	0.01\\
67.13	0.01\\
67.14	0.01\\
67.15	0.01\\
67.16	0.01\\
67.17	0.01\\
67.18	0.01\\
67.19	0.01\\
67.2	0.01\\
67.21	0.01\\
67.22	0.01\\
67.23	0.01\\
67.24	0.01\\
67.25	0.01\\
67.26	0.01\\
67.27	0.01\\
67.28	0.01\\
67.29	0.01\\
67.3	0.01\\
67.31	0.01\\
67.32	0.01\\
67.33	0.01\\
67.34	0.01\\
67.35	0.01\\
67.36	0.01\\
67.37	0.01\\
67.38	0.01\\
67.39	0.01\\
67.4	0.01\\
67.41	0.01\\
67.42	0.01\\
67.43	0.01\\
67.44	0.01\\
67.45	0.01\\
67.46	0.01\\
67.47	0.01\\
67.48	0.01\\
67.49	0.01\\
67.5	0.01\\
67.51	0.01\\
67.52	0.01\\
67.53	0.01\\
67.54	0.01\\
67.55	0.01\\
67.56	0.01\\
67.57	0.01\\
67.58	0.01\\
67.59	0.01\\
67.6	0.01\\
67.61	0.01\\
67.62	0.01\\
67.63	0.01\\
67.64	0.01\\
67.65	0.01\\
67.66	0.01\\
67.67	0.01\\
67.68	0.01\\
67.69	0.01\\
67.7	0.01\\
67.71	0.01\\
67.72	0.01\\
67.73	0.01\\
67.74	0.01\\
67.75	0.01\\
67.76	0.01\\
67.77	0.01\\
67.78	0.01\\
67.79	0.01\\
67.8	0.01\\
67.81	0.01\\
67.82	0.01\\
67.83	0.01\\
67.84	0.01\\
67.85	0.01\\
67.86	0.01\\
67.87	0.01\\
67.88	0.01\\
67.89	0.01\\
67.9	0.01\\
67.91	0.01\\
67.92	0.01\\
67.93	0.01\\
67.94	0.01\\
67.95	0.01\\
67.96	0.01\\
67.97	0.01\\
67.98	0.01\\
67.99	0.01\\
68	0.01\\
68.01	0.01\\
68.02	0.01\\
68.03	0.01\\
68.04	0.01\\
68.05	0.01\\
68.06	0.01\\
68.07	0.01\\
68.08	0.01\\
68.09	0.01\\
68.1	0.01\\
68.11	0.01\\
68.12	0.01\\
68.13	0.01\\
68.14	0.01\\
68.15	0.01\\
68.16	0.01\\
68.17	0.01\\
68.18	0.01\\
68.19	0.01\\
68.2	0.01\\
68.21	0.01\\
68.22	0.01\\
68.23	0.01\\
68.24	0.01\\
68.25	0.01\\
68.26	0.01\\
68.27	0.01\\
68.28	0.01\\
68.29	0.01\\
68.3	0.01\\
68.31	0.01\\
68.32	0.01\\
68.33	0.01\\
68.34	0.01\\
68.35	0.01\\
68.36	0.01\\
68.37	0.01\\
68.38	0.01\\
68.39	0.01\\
68.4	0.01\\
68.41	0.01\\
68.42	0.01\\
68.43	0.01\\
68.44	0.01\\
68.45	0.01\\
68.46	0.01\\
68.47	0.01\\
68.48	0.01\\
68.49	0.01\\
68.5	0.01\\
68.51	0.01\\
68.52	0.01\\
68.53	0.01\\
68.54	0.01\\
68.55	0.01\\
68.56	0.01\\
68.57	0.01\\
68.58	0.01\\
68.59	0.01\\
68.6	0.01\\
68.61	0.01\\
68.62	0.01\\
68.63	0.01\\
68.64	0.01\\
68.65	0.01\\
68.66	0.01\\
68.67	0.01\\
68.68	0.01\\
68.69	0.01\\
68.7	0.01\\
68.71	0.01\\
68.72	0.01\\
68.73	0.01\\
68.74	0.01\\
68.75	0.01\\
68.76	0.01\\
68.77	0.01\\
68.78	0.01\\
68.79	0.01\\
68.8	0.01\\
68.81	0.01\\
68.82	0.01\\
68.83	0.01\\
68.84	0.01\\
68.85	0.01\\
68.86	0.01\\
68.87	0.01\\
68.88	0.01\\
68.89	0.01\\
68.9	0.01\\
68.91	0.01\\
68.92	0.01\\
68.93	0.01\\
68.94	0.01\\
68.95	0.01\\
68.96	0.01\\
68.97	0.01\\
68.98	0.01\\
68.99	0.01\\
69	0.01\\
69.01	0.01\\
69.02	0.01\\
69.03	0.01\\
69.04	0.01\\
69.05	0.01\\
69.06	0.01\\
69.07	0.01\\
69.08	0.01\\
69.09	0.01\\
69.1	0.01\\
69.11	0.01\\
69.12	0.01\\
69.13	0.01\\
69.14	0.01\\
69.15	0.01\\
69.16	0.01\\
69.17	0.01\\
69.18	0.01\\
69.19	0.01\\
69.2	0.01\\
69.21	0.01\\
69.22	0.01\\
69.23	0.01\\
69.24	0.01\\
69.25	0.01\\
69.26	0.01\\
69.27	0.01\\
69.28	0.01\\
69.29	0.01\\
69.3	0.01\\
69.31	0.01\\
69.32	0.01\\
69.33	0.01\\
69.34	0.01\\
69.35	0.01\\
69.36	0.01\\
69.37	0.01\\
69.38	0.01\\
69.39	0.01\\
69.4	0.01\\
69.41	0.01\\
69.42	0.01\\
69.43	0.01\\
69.44	0.01\\
69.45	0.01\\
69.46	0.01\\
69.47	0.01\\
69.48	0.01\\
69.49	0.01\\
69.5	0.01\\
69.51	0.01\\
69.52	0.01\\
69.53	0.01\\
69.54	0.01\\
69.55	0.01\\
69.56	0.01\\
69.57	0.01\\
69.58	0.01\\
69.59	0.01\\
69.6	0.01\\
69.61	0.01\\
69.62	0.01\\
69.63	0.01\\
69.64	0.01\\
69.65	0.01\\
69.66	0.01\\
69.67	0.01\\
69.68	0.01\\
69.69	0.01\\
69.7	0.01\\
69.71	0.01\\
69.72	0.01\\
69.73	0.01\\
69.74	0.01\\
69.75	0.01\\
69.76	0.01\\
69.77	0.01\\
69.78	0.01\\
69.79	0.01\\
69.8	0.01\\
69.81	0.01\\
69.82	0.01\\
69.83	0.01\\
69.84	0.01\\
69.85	0.01\\
69.86	0.01\\
69.87	0.01\\
69.88	0.01\\
69.89	0.01\\
69.9	0.01\\
69.91	0.01\\
69.92	0.01\\
69.93	0.01\\
69.94	0.01\\
69.95	0.01\\
69.96	0.01\\
69.97	0.01\\
69.98	0.01\\
69.99	0.01\\
70	0.01\\
70.01	0.01\\
70.02	0.01\\
70.03	0.01\\
70.04	0.01\\
70.05	0.01\\
70.06	0.01\\
70.07	0.01\\
70.08	0.01\\
70.09	0.01\\
70.1	0.01\\
70.11	0.01\\
70.12	0.01\\
70.13	0.01\\
70.14	0.01\\
70.15	0.01\\
70.16	0.01\\
70.17	0.01\\
70.18	0.01\\
70.19	0.01\\
70.2	0.01\\
70.21	0.01\\
70.22	0.01\\
70.23	0.01\\
70.24	0.01\\
70.25	0.01\\
70.26	0.01\\
70.27	0.01\\
70.28	0.01\\
70.29	0.01\\
70.3	0.01\\
70.31	0.01\\
70.32	0.01\\
70.33	0.01\\
70.34	0.01\\
70.35	0.01\\
70.36	0.01\\
70.37	0.01\\
70.38	0.01\\
70.39	0.01\\
70.4	0.01\\
70.41	0.01\\
70.42	0.01\\
70.43	0.01\\
70.44	0.01\\
70.45	0.01\\
70.46	0.01\\
70.47	0.01\\
70.48	0.01\\
70.49	0.01\\
70.5	0.01\\
70.51	0.01\\
70.52	0.01\\
70.53	0.01\\
70.54	0.01\\
70.55	0.01\\
70.56	0.01\\
70.57	0.01\\
70.58	0.01\\
70.59	0.01\\
70.6	0.01\\
70.61	0.01\\
70.62	0.01\\
70.63	0.01\\
70.64	0.01\\
70.65	0.01\\
70.66	0.01\\
70.67	0.01\\
70.68	0.01\\
70.69	0.01\\
70.7	0.01\\
70.71	0.01\\
70.72	0.01\\
70.73	0.01\\
70.74	0.01\\
70.75	0.01\\
70.76	0.01\\
70.77	0.01\\
70.78	0.01\\
70.79	0.01\\
70.8	0.01\\
70.81	0.01\\
70.82	0.01\\
70.83	0.01\\
70.84	0.01\\
70.85	0.01\\
70.86	0.01\\
70.87	0.01\\
70.88	0.01\\
70.89	0.01\\
70.9	0.01\\
70.91	0.01\\
70.92	0.01\\
70.93	0.01\\
70.94	0.01\\
70.95	0.01\\
70.96	0.01\\
70.97	0.01\\
70.98	0.01\\
70.99	0.01\\
71	0.01\\
71.01	0.01\\
71.02	0.01\\
71.03	0.01\\
71.04	0.01\\
71.05	0.01\\
71.06	0.01\\
71.07	0.01\\
71.08	0.01\\
71.09	0.01\\
71.1	0.01\\
71.11	0.01\\
71.12	0.01\\
71.13	0.01\\
71.14	0.01\\
71.15	0.01\\
71.16	0.01\\
71.17	0.01\\
71.18	0.01\\
71.19	0.01\\
71.2	0.01\\
71.21	0.01\\
71.22	0.01\\
71.23	0.01\\
71.24	0.01\\
71.25	0.01\\
71.26	0.01\\
71.27	0.01\\
71.28	0.01\\
71.29	0.01\\
71.3	0.01\\
71.31	0.01\\
71.32	0.01\\
71.33	0.01\\
71.34	0.01\\
71.35	0.01\\
71.36	0.01\\
71.37	0.01\\
71.38	0.01\\
71.39	0.01\\
71.4	0.01\\
71.41	0.01\\
71.42	0.01\\
71.43	0.01\\
71.44	0.01\\
71.45	0.01\\
71.46	0.01\\
71.47	0.01\\
71.48	0.01\\
71.49	0.01\\
71.5	0.01\\
71.51	0.01\\
71.52	0.01\\
71.53	0.01\\
71.54	0.01\\
71.55	0.01\\
71.56	0.01\\
71.57	0.01\\
71.58	0.01\\
71.59	0.01\\
71.6	0.01\\
71.61	0.01\\
71.62	0.01\\
71.63	0.01\\
71.64	0.01\\
71.65	0.01\\
71.66	0.01\\
71.67	0.01\\
71.68	0.01\\
71.69	0.01\\
71.7	0.01\\
71.71	0.01\\
71.72	0.01\\
71.73	0.01\\
71.74	0.01\\
71.75	0.01\\
71.76	0.01\\
71.77	0.01\\
71.78	0.01\\
71.79	0.01\\
71.8	0.01\\
71.81	0.01\\
71.82	0.01\\
71.83	0.01\\
71.84	0.01\\
71.85	0.01\\
71.86	0.01\\
71.87	0.01\\
71.88	0.01\\
71.89	0.01\\
71.9	0.01\\
71.91	0.01\\
71.92	0.01\\
71.93	0.01\\
71.94	0.01\\
71.95	0.01\\
71.96	0.01\\
71.97	0.01\\
71.98	0.01\\
71.99	0.01\\
72	0.01\\
72.01	0.01\\
72.02	0.01\\
72.03	0.01\\
72.04	0.01\\
72.05	0.01\\
72.06	0.01\\
72.07	0.01\\
72.08	0.01\\
72.09	0.01\\
72.1	0.01\\
72.11	0.01\\
72.12	0.01\\
72.13	0.01\\
72.14	0.01\\
72.15	0.01\\
72.16	0.01\\
72.17	0.01\\
72.18	0.01\\
72.19	0.01\\
72.2	0.01\\
72.21	0.01\\
72.22	0.01\\
72.23	0.01\\
72.24	0.01\\
72.25	0.01\\
72.26	0.01\\
72.27	0.01\\
72.28	0.01\\
72.29	0.01\\
72.3	0.01\\
72.31	0.01\\
72.32	0.01\\
72.33	0.01\\
72.34	0.01\\
72.35	0.01\\
72.36	0.01\\
72.37	0.01\\
72.38	0.01\\
72.39	0.01\\
72.4	0.01\\
72.41	0.01\\
72.42	0.01\\
72.43	0.01\\
72.44	0.01\\
72.45	0.01\\
72.46	0.01\\
72.47	0.01\\
72.48	0.01\\
72.49	0.01\\
72.5	0.01\\
72.51	0.01\\
72.52	0.01\\
72.53	0.01\\
72.54	0.01\\
72.55	0.01\\
72.56	0.01\\
72.57	0.01\\
72.58	0.01\\
72.59	0.01\\
72.6	0.01\\
72.61	0.01\\
72.62	0.01\\
72.63	0.01\\
72.64	0.01\\
72.65	0.01\\
72.66	0.01\\
72.67	0.01\\
72.68	0.01\\
72.69	0.01\\
72.7	0.01\\
72.71	0.01\\
72.72	0.01\\
72.73	0.01\\
72.74	0.01\\
72.75	0.01\\
72.76	0.01\\
72.77	0.01\\
72.78	0.01\\
72.79	0.01\\
72.8	0.01\\
72.81	0.01\\
72.82	0.01\\
72.83	0.01\\
72.84	0.01\\
72.85	0.01\\
72.86	0.01\\
72.87	0.01\\
72.88	0.01\\
72.89	0.01\\
72.9	0.01\\
72.91	0.01\\
72.92	0.01\\
72.93	0.01\\
72.94	0.01\\
72.95	0.01\\
72.96	0.01\\
72.97	0.01\\
72.98	0.01\\
72.99	0.01\\
73	0.01\\
73.01	0.01\\
73.02	0.01\\
73.03	0.01\\
73.04	0.01\\
73.05	0.01\\
73.06	0.01\\
73.07	0.01\\
73.08	0.01\\
73.09	0.01\\
73.1	0.01\\
73.11	0.01\\
73.12	0.01\\
73.13	0.01\\
73.14	0.01\\
73.15	0.01\\
73.16	0.01\\
73.17	0.01\\
73.18	0.01\\
73.19	0.01\\
73.2	0.01\\
73.21	0.01\\
73.22	0.01\\
73.23	0.01\\
73.24	0.01\\
73.25	0.01\\
73.26	0.01\\
73.27	0.01\\
73.28	0.01\\
73.29	0.01\\
73.3	0.01\\
73.31	0.01\\
73.32	0.01\\
73.33	0.01\\
73.34	0.01\\
73.35	0.01\\
73.36	0.01\\
73.37	0.01\\
73.38	0.01\\
73.39	0.01\\
73.4	0.01\\
73.41	0.01\\
73.42	0.01\\
73.43	0.01\\
73.44	0.01\\
73.45	0.01\\
73.46	0.01\\
73.47	0.01\\
73.48	0.01\\
73.49	0.01\\
73.5	0.01\\
73.51	0.01\\
73.52	0.01\\
73.53	0.01\\
73.54	0.01\\
73.55	0.01\\
73.56	0.01\\
73.57	0.01\\
73.58	0.01\\
73.59	0.01\\
73.6	0.01\\
73.61	0.01\\
73.62	0.01\\
73.63	0.01\\
73.64	0.01\\
73.65	0.01\\
73.66	0.01\\
73.67	0.01\\
73.68	0.01\\
73.69	0.01\\
73.7	0.01\\
73.71	0.01\\
73.72	0.01\\
73.73	0.01\\
73.74	0.01\\
73.75	0.01\\
73.76	0.01\\
73.77	0.01\\
73.78	0.01\\
73.79	0.01\\
73.8	0.01\\
73.81	0.01\\
73.82	0.01\\
73.83	0.01\\
73.84	0.01\\
73.85	0.01\\
73.86	0.01\\
73.87	0.01\\
73.88	0.01\\
73.89	0.01\\
73.9	0.01\\
73.91	0.01\\
73.92	0.01\\
73.93	0.01\\
73.94	0.01\\
73.95	0.01\\
73.96	0.01\\
73.97	0.01\\
73.98	0.01\\
73.99	0.01\\
74	0.01\\
74.01	0.01\\
74.02	0.01\\
74.03	0.01\\
74.04	0.01\\
74.05	0.01\\
74.06	0.01\\
74.07	0.01\\
74.08	0.01\\
74.09	0.01\\
74.1	0.01\\
74.11	0.01\\
74.12	0.01\\
74.13	0.01\\
74.14	0.01\\
74.15	0.01\\
74.16	0.01\\
74.17	0.01\\
74.18	0.01\\
74.19	0.01\\
74.2	0.01\\
74.21	0.01\\
74.22	0.01\\
74.23	0.01\\
74.24	0.01\\
74.25	0.01\\
74.26	0.01\\
74.27	0.01\\
74.28	0.01\\
74.29	0.01\\
74.3	0.01\\
74.31	0.01\\
74.32	0.01\\
74.33	0.01\\
74.34	0.01\\
74.35	0.01\\
74.36	0.01\\
74.37	0.01\\
74.38	0.01\\
74.39	0.01\\
74.4	0.01\\
74.41	0.01\\
74.42	0.01\\
74.43	0.01\\
74.44	0.01\\
74.45	0.01\\
74.46	0.01\\
74.47	0.01\\
74.48	0.01\\
74.49	0.01\\
74.5	0.01\\
74.51	0.01\\
74.52	0.01\\
74.53	0.01\\
74.54	0.01\\
74.55	0.01\\
74.56	0.01\\
74.57	0.01\\
74.58	0.01\\
74.59	0.01\\
74.6	0.01\\
74.61	0.01\\
74.62	0.01\\
74.63	0.01\\
74.64	0.01\\
74.65	0.01\\
74.66	0.01\\
74.67	0.01\\
74.68	0.01\\
74.69	0.01\\
74.7	0.01\\
74.71	0.01\\
74.72	0.01\\
74.73	0.01\\
74.74	0.01\\
74.75	0.01\\
74.76	0.01\\
74.77	0.01\\
74.78	0.01\\
74.79	0.01\\
74.8	0.01\\
74.81	0.01\\
74.82	0.01\\
74.83	0.01\\
74.84	0.01\\
74.85	0.01\\
74.86	0.01\\
74.87	0.01\\
74.88	0.01\\
74.89	0.01\\
74.9	0.01\\
74.91	0.01\\
74.92	0.01\\
74.93	0.01\\
74.94	0.01\\
74.95	0.01\\
74.96	0.01\\
74.97	0.01\\
74.98	0.01\\
74.99	0.01\\
75	0.01\\
75.01	0.01\\
75.02	0.01\\
75.03	0.01\\
75.04	0.01\\
75.05	0.01\\
75.06	0.01\\
75.07	0.01\\
75.08	0.01\\
75.09	0.01\\
75.1	0.01\\
75.11	0.01\\
75.12	0.01\\
75.13	0.01\\
75.14	0.01\\
75.15	0.01\\
75.16	0.01\\
75.17	0.01\\
75.18	0.01\\
75.19	0.01\\
75.2	0.01\\
75.21	0.01\\
75.22	0.01\\
75.23	0.01\\
75.24	0.01\\
75.25	0.01\\
75.26	0.01\\
75.27	0.01\\
75.28	0.01\\
75.29	0.01\\
75.3	0.01\\
75.31	0.01\\
75.32	0.01\\
75.33	0.01\\
75.34	0.01\\
75.35	0.01\\
75.36	0.01\\
75.37	0.01\\
75.38	0.01\\
75.39	0.01\\
75.4	0.01\\
75.41	0.01\\
75.42	0.01\\
75.43	0.01\\
75.44	0.01\\
75.45	0.01\\
75.46	0.01\\
75.47	0.01\\
75.48	0.01\\
75.49	0.01\\
75.5	0.01\\
75.51	0.01\\
75.52	0.01\\
75.53	0.01\\
75.54	0.01\\
75.55	0.01\\
75.56	0.01\\
75.57	0.01\\
75.58	0.01\\
75.59	0.01\\
75.6	0.01\\
75.61	0.01\\
75.62	0.01\\
75.63	0.01\\
75.64	0.01\\
75.65	0.01\\
75.66	0.01\\
75.67	0.01\\
75.68	0.01\\
75.69	0.01\\
75.7	0.01\\
75.71	0.01\\
75.72	0.01\\
75.73	0.01\\
75.74	0.01\\
75.75	0.01\\
75.76	0.01\\
75.77	0.01\\
75.78	0.01\\
75.79	0.01\\
75.8	0.01\\
75.81	0.01\\
75.82	0.01\\
75.83	0.01\\
75.84	0.01\\
75.85	0.01\\
75.86	0.01\\
75.87	0.01\\
75.88	0.01\\
75.89	0.01\\
75.9	0.01\\
75.91	0.01\\
75.92	0.01\\
75.93	0.01\\
75.94	0.01\\
75.95	0.01\\
75.96	0.01\\
75.97	0.01\\
75.98	0.01\\
75.99	0.01\\
76	0.01\\
76.01	0.01\\
76.02	0.01\\
76.03	0.01\\
76.04	0.01\\
76.05	0.01\\
76.06	0.01\\
76.07	0.01\\
76.08	0.01\\
76.09	0.01\\
76.1	0.01\\
76.11	0.01\\
76.12	0.01\\
76.13	0.01\\
76.14	0.01\\
76.15	0.01\\
76.16	0.01\\
76.17	0.01\\
76.18	0.01\\
76.19	0.01\\
76.2	0.01\\
76.21	0.01\\
76.22	0.01\\
76.23	0.01\\
76.24	0.01\\
76.25	0.01\\
76.26	0.01\\
76.27	0.01\\
76.28	0.01\\
76.29	0.01\\
76.3	0.01\\
76.31	0.01\\
76.32	0.01\\
76.33	0.01\\
76.34	0.01\\
76.35	0.01\\
76.36	0.01\\
76.37	0.01\\
76.38	0.01\\
76.39	0.01\\
76.4	0.01\\
76.41	0.01\\
76.42	0.01\\
76.43	0.01\\
76.44	0.01\\
76.45	0.01\\
76.46	0.01\\
76.47	0.01\\
76.48	0.01\\
76.49	0.01\\
76.5	0.01\\
76.51	0.01\\
76.52	0.01\\
76.53	0.01\\
76.54	0.01\\
76.55	0.01\\
76.56	0.01\\
76.57	0.01\\
76.58	0.01\\
76.59	0.01\\
76.6	0.01\\
76.61	0.01\\
76.62	0.01\\
76.63	0.01\\
76.64	0.01\\
76.65	0.01\\
76.66	0.01\\
76.67	0.01\\
76.68	0.01\\
76.69	0.01\\
76.7	0.01\\
76.71	0.01\\
76.72	0.01\\
76.73	0.01\\
76.74	0.01\\
76.75	0.01\\
76.76	0.01\\
76.77	0.01\\
76.78	0.01\\
76.79	0.01\\
76.8	0.01\\
76.81	0.01\\
76.82	0.01\\
76.83	0.01\\
76.84	0.01\\
76.85	0.01\\
76.86	0.01\\
76.87	0.01\\
76.88	0.01\\
76.89	0.01\\
76.9	0.01\\
76.91	0.01\\
76.92	0.01\\
76.93	0.01\\
76.94	0.01\\
76.95	0.01\\
76.96	0.01\\
76.97	0.01\\
76.98	0.01\\
76.99	0.01\\
77	0.01\\
77.01	0.01\\
77.02	0.01\\
77.03	0.01\\
77.04	0.01\\
77.05	0.01\\
77.06	0.01\\
77.07	0.01\\
77.08	0.01\\
77.09	0.01\\
77.1	0.01\\
77.11	0.01\\
77.12	0.01\\
77.13	0.01\\
77.14	0.01\\
77.15	0.01\\
77.16	0.01\\
77.17	0.01\\
77.18	0.01\\
77.19	0.01\\
77.2	0.01\\
77.21	0.01\\
77.22	0.01\\
77.23	0.01\\
77.24	0.01\\
77.25	0.01\\
77.26	0.01\\
77.27	0.01\\
77.28	0.01\\
77.29	0.01\\
77.3	0.01\\
77.31	0.01\\
77.32	0.01\\
77.33	0.01\\
77.34	0.01\\
77.35	0.01\\
77.36	0.01\\
77.37	0.01\\
77.38	0.01\\
77.39	0.01\\
77.4	0.01\\
77.41	0.01\\
77.42	0.01\\
77.43	0.01\\
77.44	0.01\\
77.45	0.01\\
77.46	0.01\\
77.47	0.01\\
77.48	0.01\\
77.49	0.01\\
77.5	0.01\\
77.51	0.01\\
77.52	0.01\\
77.53	0.01\\
77.54	0.01\\
77.55	0.01\\
77.56	0.01\\
77.57	0.01\\
77.58	0.01\\
77.59	0.01\\
77.6	0.01\\
77.61	0.01\\
77.62	0.01\\
77.63	0.01\\
77.64	0.01\\
77.65	0.01\\
77.66	0.01\\
77.67	0.01\\
77.68	0.01\\
77.69	0.01\\
77.7	0.01\\
77.71	0.01\\
77.72	0.01\\
77.73	0.01\\
77.74	0.01\\
77.75	0.01\\
77.76	0.01\\
77.77	0.01\\
77.78	0.01\\
77.79	0.01\\
77.8	0.01\\
77.81	0.01\\
77.82	0.01\\
77.83	0.01\\
77.84	0.01\\
77.85	0.01\\
77.86	0.01\\
77.87	0.01\\
77.88	0.01\\
77.89	0.01\\
77.9	0.01\\
77.91	0.01\\
77.92	0.01\\
77.93	0.01\\
77.94	0.01\\
77.95	0.01\\
77.96	0.01\\
77.97	0.01\\
77.98	0.01\\
77.99	0.01\\
78	0.01\\
78.01	0.01\\
78.02	0.01\\
78.03	0.01\\
78.04	0.01\\
78.05	0.01\\
78.06	0.01\\
78.07	0.01\\
78.08	0.01\\
78.09	0.01\\
78.1	0.01\\
78.11	0.01\\
78.12	0.01\\
78.13	0.01\\
78.14	0.01\\
78.15	0.01\\
78.16	0.01\\
78.17	0.01\\
78.18	0.01\\
78.19	0.01\\
78.2	0.01\\
78.21	0.01\\
78.22	0.01\\
78.23	0.01\\
78.24	0.01\\
78.25	0.01\\
78.26	0.01\\
78.27	0.01\\
78.28	0.01\\
78.29	0.01\\
78.3	0.01\\
78.31	0.01\\
78.32	0.01\\
78.33	0.01\\
78.34	0.01\\
78.35	0.01\\
78.36	0.01\\
78.37	0.01\\
78.38	0.01\\
78.39	0.01\\
78.4	0.01\\
78.41	0.01\\
78.42	0.01\\
78.43	0.01\\
78.44	0.01\\
78.45	0.01\\
78.46	0.01\\
78.47	0.01\\
78.48	0.01\\
78.49	0.01\\
78.5	0.01\\
78.51	0.01\\
78.52	0.01\\
78.53	0.01\\
78.54	0.01\\
78.55	0.01\\
78.56	0.01\\
78.57	0.01\\
78.58	0.01\\
78.59	0.01\\
78.6	0.01\\
78.61	0.01\\
78.62	0.01\\
78.63	0.01\\
78.64	0.01\\
78.65	0.01\\
78.66	0.01\\
78.67	0.01\\
78.68	0.01\\
78.69	0.01\\
78.7	0.01\\
78.71	0.01\\
78.72	0.01\\
78.73	0.01\\
78.74	0.01\\
78.75	0.01\\
78.76	0.01\\
78.77	0.01\\
78.78	0.01\\
78.79	0.01\\
78.8	0.01\\
78.81	0.01\\
78.82	0.01\\
78.83	0.01\\
78.84	0.01\\
78.85	0.01\\
78.86	0.01\\
78.87	0.01\\
78.88	0.01\\
78.89	0.01\\
78.9	0.01\\
78.91	0.01\\
78.92	0.01\\
78.93	0.01\\
78.94	0.01\\
78.95	0.01\\
78.96	0.01\\
78.97	0.01\\
78.98	0.01\\
78.99	0.01\\
79	0.01\\
79.01	0.01\\
79.02	0.01\\
79.03	0.01\\
79.04	0.01\\
79.05	0.01\\
79.06	0.01\\
79.07	0.01\\
79.08	0.01\\
79.09	0.01\\
79.1	0.01\\
79.11	0.01\\
79.12	0.01\\
79.13	0.01\\
79.14	0.01\\
79.15	0.01\\
79.16	0.01\\
79.17	0.01\\
79.18	0.01\\
79.19	0.01\\
79.2	0.01\\
79.21	0.01\\
79.22	0.01\\
79.23	0.01\\
79.24	0.01\\
79.25	0.01\\
79.26	0.01\\
79.27	0.01\\
79.28	0.01\\
79.29	0.01\\
79.3	0.01\\
79.31	0.01\\
79.32	0.01\\
79.33	0.01\\
79.34	0.01\\
79.35	0.01\\
79.36	0.01\\
79.37	0.01\\
79.38	0.01\\
79.39	0.01\\
79.4	0.01\\
79.41	0.01\\
79.42	0.01\\
79.43	0.01\\
79.44	0.01\\
79.45	0.01\\
79.46	0.01\\
79.47	0.01\\
79.48	0.01\\
79.49	0.01\\
79.5	0.01\\
79.51	0.01\\
79.52	0.01\\
79.53	0.01\\
79.54	0.01\\
79.55	0.01\\
79.56	0.01\\
79.57	0.01\\
79.58	0.01\\
79.59	0.01\\
79.6	0.01\\
79.61	0.01\\
79.62	0.01\\
79.63	0.01\\
79.64	0.01\\
79.65	0.01\\
79.66	0.01\\
79.67	0.01\\
79.68	0.01\\
79.69	0.01\\
79.7	0.01\\
79.71	0.01\\
79.72	0.01\\
79.73	0.01\\
79.74	0.01\\
79.75	0.01\\
79.76	0.01\\
79.77	0.01\\
79.78	0.01\\
79.79	0.01\\
79.8	0.01\\
79.81	0.01\\
79.82	0.01\\
79.83	0.01\\
79.84	0.01\\
79.85	0.01\\
79.86	0.01\\
79.87	0.01\\
79.88	0.01\\
79.89	0.01\\
79.9	0.01\\
79.91	0.01\\
79.92	0.01\\
79.93	0.01\\
79.94	0.01\\
79.95	0.01\\
79.96	0.01\\
79.97	0.01\\
79.98	0.01\\
79.99	0.01\\
80	0.01\\
80.01	0.01\\
};
\addplot [color=blue,dashed]
  table[row sep=crcr]{%
80.01	0.01\\
80.02	0.01\\
80.03	0.01\\
80.04	0.01\\
80.05	0.01\\
80.06	0.01\\
80.07	0.01\\
80.08	0.01\\
80.09	0.01\\
80.1	0.01\\
80.11	0.01\\
80.12	0.01\\
80.13	0.01\\
80.14	0.01\\
80.15	0.01\\
80.16	0.01\\
80.17	0.01\\
80.18	0.01\\
80.19	0.01\\
80.2	0.01\\
80.21	0.01\\
80.22	0.01\\
80.23	0.01\\
80.24	0.01\\
80.25	0.01\\
80.26	0.01\\
80.27	0.01\\
80.28	0.01\\
80.29	0.01\\
80.3	0.01\\
80.31	0.01\\
80.32	0.01\\
80.33	0.01\\
80.34	0.01\\
80.35	0.01\\
80.36	0.01\\
80.37	0.01\\
80.38	0.01\\
80.39	0.01\\
80.4	0.01\\
80.41	0.01\\
80.42	0.01\\
80.43	0.01\\
80.44	0.01\\
80.45	0.01\\
80.46	0.01\\
80.47	0.01\\
80.48	0.01\\
80.49	0.01\\
80.5	0.01\\
80.51	0.01\\
80.52	0.01\\
80.53	0.01\\
80.54	0.01\\
80.55	0.01\\
80.56	0.01\\
80.57	0.01\\
80.58	0.01\\
80.59	0.01\\
80.6	0.01\\
80.61	0.01\\
80.62	0.01\\
80.63	0.01\\
80.64	0.01\\
80.65	0.01\\
80.66	0.01\\
80.67	0.01\\
80.68	0.01\\
80.69	0.01\\
80.7	0.01\\
80.71	0.01\\
80.72	0.01\\
80.73	0.01\\
80.74	0.01\\
80.75	0.01\\
80.76	0.01\\
80.77	0.01\\
80.78	0.01\\
80.79	0.01\\
80.8	0.01\\
80.81	0.01\\
80.82	0.01\\
80.83	0.01\\
80.84	0.01\\
80.85	0.01\\
80.86	0.01\\
80.87	0.01\\
80.88	0.01\\
80.89	0.01\\
80.9	0.01\\
80.91	0.01\\
80.92	0.01\\
80.93	0.01\\
80.94	0.01\\
80.95	0.01\\
80.96	0.01\\
80.97	0.01\\
80.98	0.01\\
80.99	0.01\\
81	0.01\\
81.01	0.01\\
81.02	0.01\\
81.03	0.01\\
81.04	0.01\\
81.05	0.01\\
81.06	0.01\\
81.07	0.01\\
81.08	0.01\\
81.09	0.01\\
81.1	0.01\\
81.11	0.01\\
81.12	0.01\\
81.13	0.01\\
81.14	0.01\\
81.15	0.01\\
81.16	0.01\\
81.17	0.01\\
81.18	0.01\\
81.19	0.01\\
81.2	0.01\\
81.21	0.01\\
81.22	0.01\\
81.23	0.01\\
81.24	0.01\\
81.25	0.01\\
81.26	0.01\\
81.27	0.01\\
81.28	0.01\\
81.29	0.01\\
81.3	0.01\\
81.31	0.01\\
81.32	0.01\\
81.33	0.01\\
81.34	0.01\\
81.35	0.01\\
81.36	0.01\\
81.37	0.01\\
81.38	0.01\\
81.39	0.01\\
81.4	0.01\\
81.41	0.01\\
81.42	0.01\\
81.43	0.01\\
81.44	0.01\\
81.45	0.01\\
81.46	0.01\\
81.47	0.01\\
81.48	0.01\\
81.49	0.01\\
81.5	0.01\\
81.51	0.01\\
81.52	0.01\\
81.53	0.01\\
81.54	0.01\\
81.55	0.01\\
81.56	0.01\\
81.57	0.01\\
81.58	0.01\\
81.59	0.01\\
81.6	0.01\\
81.61	0.01\\
81.62	0.01\\
81.63	0.01\\
81.64	0.01\\
81.65	0.01\\
81.66	0.01\\
81.67	0.01\\
81.68	0.01\\
81.69	0.01\\
81.7	0.01\\
81.71	0.01\\
81.72	0.01\\
81.73	0.01\\
81.74	0.01\\
81.75	0.01\\
81.76	0.01\\
81.77	0.01\\
81.78	0.01\\
81.79	0.01\\
81.8	0.01\\
81.81	0.01\\
81.82	0.01\\
81.83	0.01\\
81.84	0.01\\
81.85	0.01\\
81.86	0.01\\
81.87	0.01\\
81.88	0.01\\
81.89	0.01\\
81.9	0.01\\
81.91	0.01\\
81.92	0.01\\
81.93	0.01\\
81.94	0.01\\
81.95	0.01\\
81.96	0.01\\
81.97	0.01\\
81.98	0.01\\
81.99	0.01\\
82	0.01\\
82.01	0.01\\
82.02	0.01\\
82.03	0.01\\
82.04	0.01\\
82.05	0.01\\
82.06	0.01\\
82.07	0.01\\
82.08	0.01\\
82.09	0.01\\
82.1	0.01\\
82.11	0.01\\
82.12	0.01\\
82.13	0.01\\
82.14	0.01\\
82.15	0.01\\
82.16	0.01\\
82.17	0.01\\
82.18	0.01\\
82.19	0.01\\
82.2	0.01\\
82.21	0.01\\
82.22	0.01\\
82.23	0.01\\
82.24	0.01\\
82.25	0.01\\
82.26	0.01\\
82.27	0.01\\
82.28	0.01\\
82.29	0.01\\
82.3	0.01\\
82.31	0.01\\
82.32	0.01\\
82.33	0.01\\
82.34	0.01\\
82.35	0.01\\
82.36	0.01\\
82.37	0.01\\
82.38	0.01\\
82.39	0.01\\
82.4	0.01\\
82.41	0.01\\
82.42	0.01\\
82.43	0.01\\
82.44	0.01\\
82.45	0.01\\
82.46	0.01\\
82.47	0.01\\
82.48	0.01\\
82.49	0.01\\
82.5	0.01\\
82.51	0.01\\
82.52	0.01\\
82.53	0.01\\
82.54	0.01\\
82.55	0.01\\
82.56	0.01\\
82.57	0.01\\
82.58	0.01\\
82.59	0.01\\
82.6	0.01\\
82.61	0.01\\
82.62	0.01\\
82.63	0.01\\
82.64	0.01\\
82.65	0.01\\
82.66	0.01\\
82.67	0.01\\
82.68	0.01\\
82.69	0.01\\
82.7	0.01\\
82.71	0.01\\
82.72	0.01\\
82.73	0.01\\
82.74	0.01\\
82.75	0.01\\
82.76	0.01\\
82.77	0.01\\
82.78	0.01\\
82.79	0.01\\
82.8	0.01\\
82.81	0.01\\
82.82	0.01\\
82.83	0.01\\
82.84	0.01\\
82.85	0.01\\
82.86	0.01\\
82.87	0.01\\
82.88	0.01\\
82.89	0.01\\
82.9	0.01\\
82.91	0.01\\
82.92	0.01\\
82.93	0.01\\
82.94	0.01\\
82.95	0.01\\
82.96	0.01\\
82.97	0.01\\
82.98	0.01\\
82.99	0.01\\
83	0.01\\
83.01	0.01\\
83.02	0.01\\
83.03	0.01\\
83.04	0.01\\
83.05	0.01\\
83.06	0.01\\
83.07	0.01\\
83.08	0.01\\
83.09	0.01\\
83.1	0.01\\
83.11	0.01\\
83.12	0.01\\
83.13	0.01\\
83.14	0.01\\
83.15	0.01\\
83.16	0.01\\
83.17	0.01\\
83.18	0.01\\
83.19	0.01\\
83.2	0.01\\
83.21	0.01\\
83.22	0.01\\
83.23	0.01\\
83.24	0.01\\
83.25	0.01\\
83.26	0.01\\
83.27	0.01\\
83.28	0.01\\
83.29	0.01\\
83.3	0.01\\
83.31	0.01\\
83.32	0.01\\
83.33	0.01\\
83.34	0.01\\
83.35	0.01\\
83.36	0.01\\
83.37	0.01\\
83.38	0.01\\
83.39	0.01\\
83.4	0.01\\
83.41	0.01\\
83.42	0.01\\
83.43	0.01\\
83.44	0.01\\
83.45	0.01\\
83.46	0.01\\
83.47	0.01\\
83.48	0.01\\
83.49	0.01\\
83.5	0.01\\
83.51	0.01\\
83.52	0.01\\
83.53	0.01\\
83.54	0.01\\
83.55	0.01\\
83.56	0.01\\
83.57	0.01\\
83.58	0.01\\
83.59	0.01\\
83.6	0.01\\
83.61	0.01\\
83.62	0.01\\
83.63	0.01\\
83.64	0.01\\
83.65	0.01\\
83.66	0.01\\
83.67	0.01\\
83.68	0.01\\
83.69	0.01\\
83.7	0.01\\
83.71	0.01\\
83.72	0.01\\
83.73	0.01\\
83.74	0.01\\
83.75	0.01\\
83.76	0.01\\
83.77	0.01\\
83.78	0.01\\
83.79	0.01\\
83.8	0.01\\
83.81	0.01\\
83.82	0.01\\
83.83	0.01\\
83.84	0.01\\
83.85	0.01\\
83.86	0.01\\
83.87	0.01\\
83.88	0.01\\
83.89	0.01\\
83.9	0.01\\
83.91	0.01\\
83.92	0.01\\
83.93	0.01\\
83.94	0.01\\
83.95	0.01\\
83.96	0.01\\
83.97	0.01\\
83.98	0.01\\
83.99	0.01\\
84	0.01\\
84.01	0.01\\
84.02	0.01\\
84.03	0.01\\
84.04	0.01\\
84.05	0.01\\
84.06	0.01\\
84.07	0.01\\
84.08	0.01\\
84.09	0.01\\
84.1	0.01\\
84.11	0.01\\
84.12	0.01\\
84.13	0.01\\
84.14	0.01\\
84.15	0.01\\
84.16	0.01\\
84.17	0.01\\
84.18	0.01\\
84.19	0.01\\
84.2	0.01\\
84.21	0.01\\
84.22	0.01\\
84.23	0.01\\
84.24	0.01\\
84.25	0.01\\
84.26	0.01\\
84.27	0.01\\
84.28	0.01\\
84.29	0.01\\
84.3	0.01\\
84.31	0.01\\
84.32	0.01\\
84.33	0.01\\
84.34	0.01\\
84.35	0.01\\
84.36	0.01\\
84.37	0.01\\
84.38	0.01\\
84.39	0.01\\
84.4	0.01\\
84.41	0.01\\
84.42	0.01\\
84.43	0.01\\
84.44	0.01\\
84.45	0.01\\
84.46	0.01\\
84.47	0.01\\
84.48	0.01\\
84.49	0.01\\
84.5	0.01\\
84.51	0.01\\
84.52	0.01\\
84.53	0.01\\
84.54	0.01\\
84.55	0.01\\
84.56	0.01\\
84.57	0.01\\
84.58	0.01\\
84.59	0.01\\
84.6	0.01\\
84.61	0.01\\
84.62	0.01\\
84.63	0.01\\
84.64	0.01\\
84.65	0.01\\
84.66	0.01\\
84.67	0.01\\
84.68	0.01\\
84.69	0.01\\
84.7	0.01\\
84.71	0.01\\
84.72	0.01\\
84.73	0.01\\
84.74	0.01\\
84.75	0.01\\
84.76	0.01\\
84.77	0.01\\
84.78	0.01\\
84.79	0.01\\
84.8	0.01\\
84.81	0.01\\
84.82	0.01\\
84.83	0.01\\
84.84	0.01\\
84.85	0.01\\
84.86	0.01\\
84.87	0.01\\
84.88	0.01\\
84.89	0.01\\
84.9	0.01\\
84.91	0.01\\
84.92	0.01\\
84.93	0.01\\
84.94	0.01\\
84.95	0.01\\
84.96	0.01\\
84.97	0.01\\
84.98	0.01\\
84.99	0.01\\
85	0.01\\
85.01	0.01\\
85.02	0.01\\
85.03	0.01\\
85.04	0.01\\
85.05	0.01\\
85.06	0.01\\
85.07	0.01\\
85.08	0.01\\
85.09	0.01\\
85.1	0.01\\
85.11	0.01\\
85.12	0.01\\
85.13	0.01\\
85.14	0.01\\
85.15	0.01\\
85.16	0.01\\
85.17	0.01\\
85.18	0.01\\
85.19	0.01\\
85.2	0.01\\
85.21	0.01\\
85.22	0.01\\
85.23	0.01\\
85.24	0.01\\
85.25	0.01\\
85.26	0.01\\
85.27	0.01\\
85.28	0.01\\
85.29	0.01\\
85.3	0.01\\
85.31	0.01\\
85.32	0.01\\
85.33	0.01\\
85.34	0.01\\
85.35	0.01\\
85.36	0.01\\
85.37	0.01\\
85.38	0.01\\
85.39	0.01\\
85.4	0.01\\
85.41	0.01\\
85.42	0.01\\
85.43	0.01\\
85.44	0.01\\
85.45	0.01\\
85.46	0.01\\
85.47	0.01\\
85.48	0.01\\
85.49	0.01\\
85.5	0.01\\
85.51	0.01\\
85.52	0.01\\
85.53	0.01\\
85.54	0.01\\
85.55	0.01\\
85.56	0.01\\
85.57	0.01\\
85.58	0.01\\
85.59	0.01\\
85.6	0.01\\
85.61	0.01\\
85.62	0.01\\
85.63	0.01\\
85.64	0.01\\
85.65	0.01\\
85.66	0.01\\
85.67	0.01\\
85.68	0.01\\
85.69	0.01\\
85.7	0.01\\
85.71	0.01\\
85.72	0.01\\
85.73	0.01\\
85.74	0.01\\
85.75	0.01\\
85.76	0.01\\
85.77	0.01\\
85.78	0.01\\
85.79	0.01\\
85.8	0.01\\
85.81	0.01\\
85.82	0.01\\
85.83	0.01\\
85.84	0.01\\
85.85	0.01\\
85.86	0.01\\
85.87	0.01\\
85.88	0.01\\
85.89	0.01\\
85.9	0.01\\
85.91	0.01\\
85.92	0.01\\
85.93	0.01\\
85.94	0.01\\
85.95	0.01\\
85.96	0.01\\
85.97	0.01\\
85.98	0.01\\
85.99	0.01\\
86	0.01\\
86.01	0.01\\
86.02	0.01\\
86.03	0.01\\
86.04	0.01\\
86.05	0.01\\
86.06	0.01\\
86.07	0.01\\
86.08	0.01\\
86.09	0.01\\
86.1	0.01\\
86.11	0.01\\
86.12	0.01\\
86.13	0.01\\
86.14	0.01\\
86.15	0.01\\
86.16	0.01\\
86.17	0.01\\
86.18	0.01\\
86.19	0.01\\
86.2	0.01\\
86.21	0.01\\
86.22	0.01\\
86.23	0.01\\
86.24	0.01\\
86.25	0.01\\
86.26	0.01\\
86.27	0.01\\
86.28	0.01\\
86.29	0.01\\
86.3	0.01\\
86.31	0.01\\
86.32	0.01\\
86.33	0.01\\
86.34	0.01\\
86.35	0.01\\
86.36	0.01\\
86.37	0.01\\
86.38	0.01\\
86.39	0.01\\
86.4	0.01\\
86.41	0.01\\
86.42	0.01\\
86.43	0.01\\
86.44	0.01\\
86.45	0.01\\
86.46	0.01\\
86.47	0.01\\
86.48	0.01\\
86.49	0.01\\
86.5	0.01\\
86.51	0.01\\
86.52	0.01\\
86.53	0.01\\
86.54	0.01\\
86.55	0.01\\
86.56	0.01\\
86.57	0.01\\
86.58	0.01\\
86.59	0.01\\
86.6	0.01\\
86.61	0.01\\
86.62	0.01\\
86.63	0.01\\
86.64	0.01\\
86.65	0.01\\
86.66	0.01\\
86.67	0.01\\
86.68	0.01\\
86.69	0.01\\
86.7	0.01\\
86.71	0.01\\
86.72	0.01\\
86.73	0.01\\
86.74	0.01\\
86.75	0.01\\
86.76	0.01\\
86.77	0.01\\
86.78	0.01\\
86.79	0.01\\
86.8	0.01\\
86.81	0.01\\
86.82	0.01\\
86.83	0.01\\
86.84	0.01\\
86.85	0.01\\
86.86	0.01\\
86.87	0.01\\
86.88	0.01\\
86.89	0.01\\
86.9	0.01\\
86.91	0.01\\
86.92	0.01\\
86.93	0.01\\
86.94	0.01\\
86.95	0.01\\
86.96	0.01\\
86.97	0.01\\
86.98	0.01\\
86.99	0.01\\
87	0.01\\
87.01	0.01\\
87.02	0.01\\
87.03	0.01\\
87.04	0.01\\
87.05	0.01\\
87.06	0.01\\
87.07	0.01\\
87.08	0.01\\
87.09	0.01\\
87.1	0.01\\
87.11	0.01\\
87.12	0.01\\
87.13	0.01\\
87.14	0.01\\
87.15	0.01\\
87.16	0.01\\
87.17	0.01\\
87.18	0.01\\
87.19	0.01\\
87.2	0.01\\
87.21	0.01\\
87.22	0.01\\
87.23	0.01\\
87.24	0.01\\
87.25	0.01\\
87.26	0.01\\
87.27	0.01\\
87.28	0.01\\
87.29	0.01\\
87.3	0.01\\
87.31	0.01\\
87.32	0.01\\
87.33	0.01\\
87.34	0.01\\
87.35	0.01\\
87.36	0.01\\
87.37	0.01\\
87.38	0.01\\
87.39	0.01\\
87.4	0.01\\
87.41	0.01\\
87.42	0.01\\
87.43	0.01\\
87.44	0.01\\
87.45	0.01\\
87.46	0.01\\
87.47	0.01\\
87.48	0.01\\
87.49	0.01\\
87.5	0.01\\
87.51	0.01\\
87.52	0.01\\
87.53	0.01\\
87.54	0.01\\
87.55	0.01\\
87.56	0.01\\
87.57	0.01\\
87.58	0.01\\
87.59	0.01\\
87.6	0.01\\
87.61	0.01\\
87.62	0.01\\
87.63	0.01\\
87.64	0.01\\
87.65	0.01\\
87.66	0.01\\
87.67	0.01\\
87.68	0.01\\
87.69	0.01\\
87.7	0.01\\
87.71	0.01\\
87.72	0.01\\
87.73	0.01\\
87.74	0.01\\
87.75	0.01\\
87.76	0.01\\
87.77	0.01\\
87.78	0.01\\
87.79	0.01\\
87.8	0.01\\
87.81	0.01\\
87.82	0.01\\
87.83	0.01\\
87.84	0.01\\
87.85	0.01\\
87.86	0.01\\
87.87	0.01\\
87.88	0.01\\
87.89	0.01\\
87.9	0.01\\
87.91	0.01\\
87.92	0.01\\
87.93	0.01\\
87.94	0.01\\
87.95	0.01\\
87.96	0.01\\
87.97	0.01\\
87.98	0.01\\
87.99	0.01\\
88	0.01\\
88.01	0.01\\
88.02	0.01\\
88.03	0.01\\
88.04	0.01\\
88.05	0.01\\
88.06	0.01\\
88.07	0.01\\
88.08	0.01\\
88.09	0.01\\
88.1	0.01\\
88.11	0.01\\
88.12	0.01\\
88.13	0.01\\
88.14	0.01\\
88.15	0.01\\
88.16	0.01\\
88.17	0.01\\
88.18	0.01\\
88.19	0.01\\
88.2	0.01\\
88.21	0.01\\
88.22	0.01\\
88.23	0.01\\
88.24	0.01\\
88.25	0.01\\
88.26	0.01\\
88.27	0.01\\
88.28	0.01\\
88.29	0.01\\
88.3	0.01\\
88.31	0.01\\
88.32	0.01\\
88.33	0.01\\
88.34	0.01\\
88.35	0.01\\
88.36	0.01\\
88.37	0.01\\
88.38	0.01\\
88.39	0.01\\
88.4	0.01\\
88.41	0.01\\
88.42	0.01\\
88.43	0.01\\
88.44	0.01\\
88.45	0.01\\
88.46	0.01\\
88.47	0.01\\
88.48	0.01\\
88.49	0.01\\
88.5	0.01\\
88.51	0.01\\
88.52	0.01\\
88.53	0.01\\
88.54	0.01\\
88.55	0.01\\
88.56	0.01\\
88.57	0.01\\
88.58	0.01\\
88.59	0.01\\
88.6	0.01\\
88.61	0.01\\
88.62	0.01\\
88.63	0.01\\
88.64	0.01\\
88.65	0.01\\
88.66	0.01\\
88.67	0.01\\
88.68	0.01\\
88.69	0.01\\
88.7	0.01\\
88.71	0.01\\
88.72	0.01\\
88.73	0.01\\
88.74	0.01\\
88.75	0.01\\
88.76	0.01\\
88.77	0.01\\
88.78	0.01\\
88.79	0.01\\
88.8	0.01\\
88.81	0.01\\
88.82	0.01\\
88.83	0.01\\
88.84	0.01\\
88.85	0.01\\
88.86	0.01\\
88.87	0.01\\
88.88	0.01\\
88.89	0.01\\
88.9	0.01\\
88.91	0.01\\
88.92	0.01\\
88.93	0.01\\
88.94	0.01\\
88.95	0.01\\
88.96	0.01\\
88.97	0.01\\
88.98	0.01\\
88.99	0.01\\
89	0.01\\
89.01	0.01\\
89.02	0.01\\
89.03	0.01\\
89.04	0.01\\
89.05	0.01\\
89.06	0.01\\
89.07	0.01\\
89.08	0.01\\
89.09	0.01\\
89.1	0.01\\
89.11	0.01\\
89.12	0.01\\
89.13	0.01\\
89.14	0.01\\
89.15	0.01\\
89.16	0.01\\
89.17	0.01\\
89.18	0.01\\
89.19	0.01\\
89.2	0.01\\
89.21	0.01\\
89.22	0.01\\
89.23	0.01\\
89.24	0.01\\
89.25	0.01\\
89.26	0.01\\
89.27	0.01\\
89.28	0.01\\
89.29	0.01\\
89.3	0.01\\
89.31	0.01\\
89.32	0.01\\
89.33	0.01\\
89.34	0.01\\
89.35	0.01\\
89.36	0.01\\
89.37	0.01\\
89.38	0.01\\
89.39	0.01\\
89.4	0.01\\
89.41	0.01\\
89.42	0.01\\
89.43	0.01\\
89.44	0.01\\
89.45	0.01\\
89.46	0.01\\
89.47	0.01\\
89.48	0.01\\
89.49	0.01\\
89.5	0.01\\
89.51	0.01\\
89.52	0.01\\
89.53	0.01\\
89.54	0.01\\
89.55	0.01\\
89.56	0.01\\
89.57	0.01\\
89.58	0.01\\
89.59	0.01\\
89.6	0.01\\
89.61	0.01\\
89.62	0.01\\
89.63	0.01\\
89.64	0.01\\
89.65	0.01\\
89.66	0.01\\
89.67	0.01\\
89.68	0.01\\
89.69	0.01\\
89.7	0.01\\
89.71	0.01\\
89.72	0.01\\
89.73	0.01\\
89.74	0.01\\
89.75	0.01\\
89.76	0.01\\
89.77	0.01\\
89.78	0.01\\
89.79	0.01\\
89.8	0.01\\
89.81	0.01\\
89.82	0.01\\
89.83	0.01\\
89.84	0.01\\
89.85	0.01\\
89.86	0.01\\
89.87	0.01\\
89.88	0.01\\
89.89	0.01\\
89.9	0.01\\
89.91	0.01\\
89.92	0.01\\
89.93	0.01\\
89.94	0.01\\
89.95	0.01\\
89.96	0.01\\
89.97	0.01\\
89.98	0.01\\
89.99	0.01\\
90	0.01\\
90.01	0.01\\
90.02	0.01\\
90.03	0.01\\
90.04	0.01\\
90.05	0.01\\
90.06	0.01\\
90.07	0.01\\
90.08	0.01\\
90.09	0.01\\
90.1	0.01\\
90.11	0.01\\
90.12	0.01\\
90.13	0.01\\
90.14	0.01\\
90.15	0.01\\
90.16	0.01\\
90.17	0.01\\
90.18	0.01\\
90.19	0.01\\
90.2	0.01\\
90.21	0.01\\
90.22	0.01\\
90.23	0.01\\
90.24	0.01\\
90.25	0.01\\
90.26	0.01\\
90.27	0.01\\
90.28	0.01\\
90.29	0.01\\
90.3	0.01\\
90.31	0.01\\
90.32	0.01\\
90.33	0.01\\
90.34	0.01\\
90.35	0.01\\
90.36	0.01\\
90.37	0.01\\
90.38	0.01\\
90.39	0.01\\
90.4	0.01\\
90.41	0.01\\
90.42	0.01\\
90.43	0.01\\
90.44	0.01\\
90.45	0.01\\
90.46	0.01\\
90.47	0.01\\
90.48	0.01\\
90.49	0.01\\
90.5	0.01\\
90.51	0.01\\
90.52	0.01\\
90.53	0.01\\
90.54	0.01\\
90.55	0.01\\
90.56	0.01\\
90.57	0.01\\
90.58	0.01\\
90.59	0.01\\
90.6	0.01\\
90.61	0.01\\
90.62	0.01\\
90.63	0.01\\
90.64	0.01\\
90.65	0.01\\
90.66	0.01\\
90.67	0.01\\
90.68	0.01\\
90.69	0.01\\
90.7	0.01\\
90.71	0.01\\
90.72	0.01\\
90.73	0.01\\
90.74	0.01\\
90.75	0.01\\
90.76	0.01\\
90.77	0.01\\
90.78	0.01\\
90.79	0.01\\
90.8	0.01\\
90.81	0.01\\
90.82	0.01\\
90.83	0.01\\
90.84	0.01\\
90.85	0.01\\
90.86	0.01\\
90.87	0.01\\
90.88	0.01\\
90.89	0.01\\
90.9	0.01\\
90.91	0.01\\
90.92	0.01\\
90.93	0.01\\
90.94	0.01\\
90.95	0.01\\
90.96	0.01\\
90.97	0.01\\
90.98	0.01\\
90.99	0.01\\
91	0.01\\
91.01	0.01\\
91.02	0.01\\
91.03	0.01\\
91.04	0.01\\
91.05	0.01\\
91.06	0.01\\
91.07	0.01\\
91.08	0.01\\
91.09	0.01\\
91.1	0.01\\
91.11	0.01\\
91.12	0.01\\
91.13	0.01\\
91.14	0.01\\
91.15	0.01\\
91.16	0.01\\
91.17	0.01\\
91.18	0.01\\
91.19	0.01\\
91.2	0.01\\
91.21	0.01\\
91.22	0.01\\
91.23	0.01\\
91.24	0.01\\
91.25	0.01\\
91.26	0.01\\
91.27	0.01\\
91.28	0.01\\
91.29	0.01\\
91.3	0.01\\
91.31	0.01\\
91.32	0.01\\
91.33	0.01\\
91.34	0.01\\
91.35	0.01\\
91.36	0.01\\
91.37	0.01\\
91.38	0.01\\
91.39	0.01\\
91.4	0.01\\
91.41	0.01\\
91.42	0.01\\
91.43	0.01\\
91.44	0.01\\
91.45	0.01\\
91.46	0.01\\
91.47	0.01\\
91.48	0.01\\
91.49	0.01\\
91.5	0.01\\
91.51	0.01\\
91.52	0.01\\
91.53	0.01\\
91.54	0.01\\
91.55	0.01\\
91.56	0.01\\
91.57	0.01\\
91.58	0.01\\
91.59	0.01\\
91.6	0.01\\
91.61	0.01\\
91.62	0.01\\
91.63	0.01\\
91.64	0.01\\
91.65	0.01\\
91.66	0.01\\
91.67	0.01\\
91.68	0.01\\
91.69	0.01\\
91.7	0.01\\
91.71	0.01\\
91.72	0.01\\
91.73	0.01\\
91.74	0.01\\
91.75	0.01\\
91.76	0.01\\
91.77	0.01\\
91.78	0.01\\
91.79	0.01\\
91.8	0.01\\
91.81	0.01\\
91.82	0.01\\
91.83	0.01\\
91.84	0.01\\
91.85	0.01\\
91.86	0.01\\
91.87	0.01\\
91.88	0.01\\
91.89	0.01\\
91.9	0.01\\
91.91	0.01\\
91.92	0.01\\
91.93	0.01\\
91.94	0.01\\
91.95	0.01\\
91.96	0.01\\
91.97	0.01\\
91.98	0.01\\
91.99	0.01\\
92	0.01\\
92.01	0.01\\
92.02	0.01\\
92.03	0.01\\
92.04	0.01\\
92.05	0.01\\
92.06	0.01\\
92.07	0.01\\
92.08	0.01\\
92.09	0.01\\
92.1	0.01\\
92.11	0.01\\
92.12	0.01\\
92.13	0.01\\
92.14	0.01\\
92.15	0.01\\
92.16	0.01\\
92.17	0.01\\
92.18	0.01\\
92.19	0.01\\
92.2	0.01\\
92.21	0.01\\
92.22	0.01\\
92.23	0.01\\
92.24	0.01\\
92.25	0.01\\
92.26	0.01\\
92.27	0.01\\
92.28	0.01\\
92.29	0.01\\
92.3	0.01\\
92.31	0.01\\
92.32	0.01\\
92.33	0.01\\
92.34	0.01\\
92.35	0.01\\
92.36	0.01\\
92.37	0.01\\
92.38	0.01\\
92.39	0.01\\
92.4	0.01\\
92.41	0.01\\
92.42	0.01\\
92.43	0.01\\
92.44	0.01\\
92.45	0.01\\
92.46	0.01\\
92.47	0.01\\
92.48	0.01\\
92.49	0.01\\
92.5	0.01\\
92.51	0.01\\
92.52	0.01\\
92.53	0.01\\
92.54	0.01\\
92.55	0.01\\
92.56	0.01\\
92.57	0.01\\
92.58	0.01\\
92.59	0.01\\
92.6	0.01\\
92.61	0.01\\
92.62	0.01\\
92.63	0.01\\
92.64	0.01\\
92.65	0.01\\
92.66	0.01\\
92.67	0.01\\
92.68	0.01\\
92.69	0.01\\
92.7	0.01\\
92.71	0.01\\
92.72	0.01\\
92.73	0.01\\
92.74	0.01\\
92.75	0.01\\
92.76	0.01\\
92.77	0.01\\
92.78	0.01\\
92.79	0.01\\
92.8	0.01\\
92.81	0.01\\
92.82	0.01\\
92.83	0.01\\
92.84	0.01\\
92.85	0.01\\
92.86	0.01\\
92.87	0.01\\
92.88	0.01\\
92.89	0.01\\
92.9	0.01\\
92.91	0.01\\
92.92	0.01\\
92.93	0.01\\
92.94	0.01\\
92.95	0.01\\
92.96	0.01\\
92.97	0.01\\
92.98	0.01\\
92.99	0.01\\
93	0.01\\
93.01	0.01\\
93.02	0.01\\
93.03	0.01\\
93.04	0.01\\
93.05	0.01\\
93.06	0.01\\
93.07	0.01\\
93.08	0.01\\
93.09	0.01\\
93.1	0.01\\
93.11	0.01\\
93.12	0.01\\
93.13	0.01\\
93.14	0.01\\
93.15	0.01\\
93.16	0.01\\
93.17	0.01\\
93.18	0.01\\
93.19	0.01\\
93.2	0.01\\
93.21	0.01\\
93.22	0.01\\
93.23	0.01\\
93.24	0.01\\
93.25	0.01\\
93.26	0.01\\
93.27	0.01\\
93.28	0.01\\
93.29	0.01\\
93.3	0.01\\
93.31	0.01\\
93.32	0.01\\
93.33	0.01\\
93.34	0.01\\
93.35	0.01\\
93.36	0.01\\
93.37	0.01\\
93.38	0.01\\
93.39	0.01\\
93.4	0.01\\
93.41	0.01\\
93.42	0.01\\
93.43	0.01\\
93.44	0.01\\
93.45	0.01\\
93.46	0.01\\
93.47	0.01\\
93.48	0.01\\
93.49	0.01\\
93.5	0.01\\
93.51	0.01\\
93.52	0.01\\
93.53	0.01\\
93.54	0.01\\
93.55	0.01\\
93.56	0.01\\
93.57	0.01\\
93.58	0.01\\
93.59	0.01\\
93.6	0.01\\
93.61	0.01\\
93.62	0.01\\
93.63	0.01\\
93.64	0.01\\
93.65	0.01\\
93.66	0.01\\
93.67	0.01\\
93.68	0.01\\
93.69	0.01\\
93.7	0.01\\
93.71	0.01\\
93.72	0.01\\
93.73	0.01\\
93.74	0.01\\
93.75	0.01\\
93.76	0.01\\
93.77	0.01\\
93.78	0.01\\
93.79	0.01\\
93.8	0.01\\
93.81	0.01\\
93.82	0.01\\
93.83	0.01\\
93.84	0.01\\
93.85	0.01\\
93.86	0.01\\
93.87	0.01\\
93.88	0.01\\
93.89	0.01\\
93.9	0.01\\
93.91	0.01\\
93.92	0.01\\
93.93	0.01\\
93.94	0.01\\
93.95	0.01\\
93.96	0.01\\
93.97	0.01\\
93.98	0.01\\
93.99	0.01\\
94	0.01\\
94.01	0.01\\
94.02	0.01\\
94.03	0.01\\
94.04	0.01\\
94.05	0.01\\
94.06	0.01\\
94.07	0.01\\
94.08	0.01\\
94.09	0.01\\
94.1	0.01\\
94.11	0.01\\
94.12	0.01\\
94.13	0.01\\
94.14	0.01\\
94.15	0.01\\
94.16	0.01\\
94.17	0.01\\
94.18	0.01\\
94.19	0.01\\
94.2	0.01\\
94.21	0.01\\
94.22	0.01\\
94.23	0.01\\
94.24	0.01\\
94.25	0.01\\
94.26	0.01\\
94.27	0.01\\
94.28	0.01\\
94.29	0.01\\
94.3	0.01\\
94.31	0.01\\
94.32	0.01\\
94.33	0.01\\
94.34	0.01\\
94.35	0.01\\
94.36	0.01\\
94.37	0.01\\
94.38	0.01\\
94.39	0.01\\
94.4	0.01\\
94.41	0.01\\
94.42	0.01\\
94.43	0.01\\
94.44	0.01\\
94.45	0.01\\
94.46	0.01\\
94.47	0.01\\
94.48	0.01\\
94.49	0.01\\
94.5	0.01\\
94.51	0.01\\
94.52	0.01\\
94.53	0.01\\
94.54	0.01\\
94.55	0.01\\
94.56	0.01\\
94.57	0.01\\
94.58	0.01\\
94.59	0.01\\
94.6	0.01\\
94.61	0.01\\
94.62	0.01\\
94.63	0.01\\
94.64	0.01\\
94.65	0.01\\
94.66	0.01\\
94.67	0.01\\
94.68	0.01\\
94.69	0.01\\
94.7	0.01\\
94.71	0.01\\
94.72	0.01\\
94.73	0.01\\
94.74	0.01\\
94.75	0.01\\
94.76	0.01\\
94.77	0.01\\
94.78	0.01\\
94.79	0.01\\
94.8	0.01\\
94.81	0.01\\
94.82	0.01\\
94.83	0.01\\
94.84	0.01\\
94.85	0.01\\
94.86	0.01\\
94.87	0.01\\
94.88	0.01\\
94.89	0.01\\
94.9	0.01\\
94.91	0.01\\
94.92	0.01\\
94.93	0.01\\
94.94	0.01\\
94.95	0.01\\
94.96	0.01\\
94.97	0.01\\
94.98	0.01\\
94.99	0.01\\
95	0.01\\
95.01	0.01\\
95.02	0.01\\
95.03	0.01\\
95.04	0.01\\
95.05	0.01\\
95.06	0.01\\
95.07	0.01\\
95.08	0.01\\
95.09	0.01\\
95.1	0.01\\
95.11	0.01\\
95.12	0.01\\
95.13	0.01\\
95.14	0.01\\
95.15	0.01\\
95.16	0.01\\
95.17	0.01\\
95.18	0.01\\
95.19	0.01\\
95.2	0.01\\
95.21	0.01\\
95.22	0.01\\
95.23	0.01\\
95.24	0.01\\
95.25	0.01\\
95.26	0.01\\
95.27	0.01\\
95.28	0.01\\
95.29	0.01\\
95.3	0.01\\
95.31	0.01\\
95.32	0.01\\
95.33	0.01\\
95.34	0.01\\
95.35	0.01\\
95.36	0.01\\
95.37	0.01\\
95.38	0.01\\
95.39	0.01\\
95.4	0.01\\
95.41	0.01\\
95.42	0.01\\
95.43	0.01\\
95.44	0.01\\
95.45	0.01\\
95.46	0.01\\
95.47	0.01\\
95.48	0.01\\
95.49	0.01\\
95.5	0.01\\
95.51	0.01\\
95.52	0.01\\
95.53	0.01\\
95.54	0.01\\
95.55	0.01\\
95.56	0.01\\
95.57	0.01\\
95.58	0.01\\
95.59	0.01\\
95.6	0.01\\
95.61	0.01\\
95.62	0.01\\
95.63	0.01\\
95.64	0.01\\
95.65	0.01\\
95.66	0.01\\
95.67	0.01\\
95.68	0.01\\
95.69	0.01\\
95.7	0.01\\
95.71	0.01\\
95.72	0.01\\
95.73	0.01\\
95.74	0.01\\
95.75	0.01\\
95.76	0.01\\
95.77	0.01\\
95.78	0.01\\
95.79	0.01\\
95.8	0.01\\
95.81	0.01\\
95.82	0.01\\
95.83	0.01\\
95.84	0.01\\
95.85	0.01\\
95.86	0.01\\
95.87	0.01\\
95.88	0.01\\
95.89	0.01\\
95.9	0.01\\
95.91	0.01\\
95.92	0.01\\
95.93	0.01\\
95.94	0.01\\
95.95	0.01\\
95.96	0.01\\
95.97	0.01\\
95.98	0.01\\
95.99	0.01\\
96	0.01\\
96.01	0.01\\
96.02	0.01\\
96.03	0.01\\
96.04	0.01\\
96.05	0.01\\
96.06	0.01\\
96.07	0.01\\
96.08	0.01\\
96.09	0.01\\
96.1	0.01\\
96.11	0.01\\
96.12	0.01\\
96.13	0.01\\
96.14	0.01\\
96.15	0.01\\
96.16	0.01\\
96.17	0.01\\
96.18	0.01\\
96.19	0.01\\
96.2	0.01\\
96.21	0.01\\
96.22	0.01\\
96.23	0.01\\
96.24	0.01\\
96.25	0.01\\
96.26	0.01\\
96.27	0.01\\
96.28	0.01\\
96.29	0.01\\
96.3	0.01\\
96.31	0.01\\
96.32	0.01\\
96.33	0.01\\
96.34	0.01\\
96.35	0.01\\
96.36	0.01\\
96.37	0.01\\
96.38	0.01\\
96.39	0.01\\
96.4	0.01\\
96.41	0.01\\
96.42	0.01\\
96.43	0.01\\
96.44	0.01\\
96.45	0.01\\
96.46	0.01\\
96.47	0.01\\
96.48	0.01\\
96.49	0.01\\
96.5	0.01\\
96.51	0.01\\
96.52	0.01\\
96.53	0.01\\
96.54	0.01\\
96.55	0.01\\
96.56	0.01\\
96.57	0.01\\
96.58	0.01\\
96.59	0.01\\
96.6	0.01\\
96.61	0.01\\
96.62	0.01\\
96.63	0.01\\
96.64	0.01\\
96.65	0.01\\
96.66	0.01\\
96.67	0.01\\
96.68	0.01\\
96.69	0.01\\
96.7	0.01\\
96.71	0.01\\
96.72	0.01\\
96.73	0.01\\
96.74	0.01\\
96.75	0.01\\
96.76	0.01\\
96.77	0.01\\
96.78	0.01\\
96.79	0.01\\
96.8	0.01\\
96.81	0.01\\
96.82	0.01\\
96.83	0.01\\
96.84	0.01\\
96.85	0.01\\
96.86	0.01\\
96.87	0.01\\
96.88	0.01\\
96.89	0.01\\
96.9	0.01\\
96.91	0.01\\
96.92	0.01\\
96.93	0.01\\
96.94	0.01\\
96.95	0.01\\
96.96	0.01\\
96.97	0.01\\
96.98	0.01\\
96.99	0.01\\
97	0.01\\
97.01	0.01\\
97.02	0.01\\
97.03	0.01\\
97.04	0.01\\
97.05	0.01\\
97.06	0.01\\
97.07	0.01\\
97.08	0.01\\
97.09	0.01\\
97.1	0.01\\
97.11	0.01\\
97.12	0.01\\
97.13	0.01\\
97.14	0.01\\
97.15	0.01\\
97.16	0.01\\
97.17	0.01\\
97.18	0.01\\
97.19	0.01\\
97.2	0.01\\
97.21	0.01\\
97.22	0.01\\
97.23	0.01\\
97.24	0.01\\
97.25	0.01\\
97.26	0.01\\
97.27	0.01\\
97.28	0.01\\
97.29	0.01\\
97.3	0.01\\
97.31	0.01\\
97.32	0.01\\
97.33	0.01\\
97.34	0.01\\
97.35	0.01\\
97.36	0.01\\
97.37	0.01\\
97.38	0.01\\
97.39	0.01\\
97.4	0.01\\
97.41	0.01\\
97.42	0.01\\
97.43	0.01\\
97.44	0.01\\
97.45	0.01\\
97.46	0.01\\
97.47	0.01\\
97.48	0.01\\
97.49	0.01\\
97.5	0.01\\
97.51	0.01\\
97.52	0.01\\
97.53	0.01\\
97.54	0.01\\
97.55	0.01\\
97.56	0.01\\
97.57	0.01\\
97.58	0.01\\
97.59	0.01\\
97.6	0.01\\
97.61	0.01\\
97.62	0.01\\
97.63	0.01\\
97.64	0.01\\
97.65	0.01\\
97.66	0.01\\
97.67	0.01\\
97.68	0.01\\
97.69	0.01\\
97.7	0.01\\
97.71	0.01\\
97.72	0.01\\
97.73	0.01\\
97.74	0.01\\
97.75	0.01\\
97.76	0.01\\
97.77	0.01\\
97.78	0.01\\
97.79	0.01\\
97.8	0.01\\
97.81	0.01\\
97.82	0.01\\
97.83	0.01\\
97.84	0.01\\
97.85	0.01\\
97.86	0.01\\
97.87	0.01\\
97.88	0.01\\
97.89	0.01\\
97.9	0.01\\
97.91	0.01\\
97.92	0.01\\
97.93	0.01\\
97.94	0.01\\
97.95	0.01\\
97.96	0.01\\
97.97	0.01\\
97.98	0.01\\
97.99	0.01\\
98	0.01\\
98.01	0.01\\
98.02	0.01\\
98.03	0.01\\
98.04	0.01\\
98.05	0.01\\
98.06	0.01\\
98.07	0.01\\
98.08	0.01\\
98.09	0.01\\
98.1	0.01\\
98.11	0.01\\
98.12	0.01\\
98.13	0.01\\
98.14	0.01\\
98.15	0.01\\
98.16	0.01\\
98.17	0.01\\
98.18	0.01\\
98.19	0.01\\
98.2	0.01\\
98.21	0.01\\
98.22	0.01\\
98.23	0.01\\
98.24	0.01\\
98.25	0.01\\
98.26	0.01\\
98.27	0.01\\
98.28	0.01\\
98.29	0.01\\
98.3	0.01\\
98.31	0.01\\
98.32	0.01\\
98.33	0.01\\
98.34	0.01\\
98.35	0.01\\
98.36	0.01\\
98.37	0.01\\
98.38	0.01\\
98.39	0.01\\
98.4	0.01\\
98.41	0.01\\
98.42	0.01\\
98.43	0.01\\
98.44	0.01\\
98.45	0.01\\
98.46	0.01\\
98.47	0.01\\
98.48	0.01\\
98.49	0.01\\
98.5	0.01\\
98.51	0.01\\
98.52	0.01\\
98.53	0.01\\
98.54	0.01\\
98.55	0.01\\
98.56	0.01\\
98.57	0.01\\
98.58	0.01\\
98.59	0.01\\
98.6	0.01\\
98.61	0.01\\
98.62	0.01\\
98.63	0.00999505388205829\\
98.64	0.00995136735966437\\
98.65	0.00990739270767156\\
98.66	0.00986312733885803\\
98.67	0.00981856863916014\\
98.68	0.00977371396734272\\
98.69	0.00972856065466484\\
98.7	0.00968310600435172\\
98.71	0.00963734729120488\\
98.72	0.00959128176124934\\
98.73	0.00954490663137584\\
98.74	0.0094982264449798\\
98.75	0.0094512464499936\\
98.76	0.00940396402073698\\
98.77	0.00935637650432044\\
98.78	0.00930848121972065\\
98.79	0.00926027563724096\\
98.8	0.00921176101813882\\
98.81	0.00916293466371986\\
98.82	0.00911379384725402\\
98.83	0.0090643358136209\\
98.84	0.00901455777894991\\
98.85	0.00896445693025523\\
98.86	0.00891403042506543\\
98.87	0.00886327539104762\\
98.88	0.00881218892562615\\
98.89	0.00876076809559573\\
98.9	0.00870900993672882\\
98.91	0.00865691145337731\\
98.92	0.00860446961553091\\
98.93	0.00855168136056717\\
98.94	0.00849854359313442\\
98.95	0.00844505318472779\\
98.96	0.00839120697325908\\
98.97	0.00833700176262016\\
98.98	0.00828243432223994\\
98.99	0.00822750138663475\\
99	0.00817219965495204\\
99.01	0.00811652579050724\\
99.02	0.00806047642031374\\
99.03	0.00800404813460585\\
99.04	0.00794723748635457\\
99.05	0.00789004099077617\\
99.06	0.0078324551248333\\
99.07	0.00777447632672868\\
99.08	0.00771610099539109\\
99.09	0.00765732548995361\\
99.1	0.007598146129224\\
99.11	0.00753855919114704\\
99.12	0.00747856091225866\\
99.13	0.00741814748713178\\
99.14	0.00735731506781378\\
99.15	0.00729605976325524\\
99.16	0.00723437763873002\\
99.17	0.00717226471524643\\
99.18	0.00710971696894931\\
99.19	0.00704673033051301\\
99.2	0.00698330068452509\\
99.21	0.00691942386886035\\
99.22	0.00685509567404525\\
99.23	0.00679031184261236\\
99.24	0.00672506806844475\\
99.25	0.0066593599961101\\
99.26	0.00659318322018437\\
99.27	0.00652653328456483\\
99.28	0.00645940568177213\\
99.29	0.00639179585224147\\
99.3	0.00632369918360239\\
99.31	0.00625511100994711\\
99.32	0.00618602661108714\\
99.33	0.006116441211798\\
99.34	0.00604634998105173\\
99.35	0.00597574803123699\\
99.36	0.00590463041736653\\
99.37	0.00583299213627174\\
99.38	0.00576082812578402\\
99.39	0.00568813326389982\\
99.4	0.00561490236781182\\
99.41	0.00554113019304813\\
99.42	0.00546681143259714\\
99.43	0.00539194071601783\\
99.44	0.00531651260853528\\
99.45	0.00524052161012092\\
99.46	0.00516396215455726\\
99.47	0.00508682860848681\\
99.48	0.00500911527044477\\
99.49	0.00493081636987514\\
99.5	0.00485192606612993\\
99.51	0.00477243844745095\\
99.52	0.00469234752993405\\
99.53	0.00461164725647502\\
99.54	0.00453033149569714\\
99.55	0.00444839404085956\\
99.56	0.00436582860874636\\
99.57	0.00428262883853562\\
99.58	0.00419878829064817\\
99.59	0.00411430044557536\\
99.6	0.00402915870268544\\
99.61	0.00394335637900804\\
99.62	0.00385688670799605\\
99.63	0.00376974283826455\\
99.64	0.00368191783230589\\
99.65	0.00359340466518068\\
99.66	0.00350419622318361\\
99.67	0.00341428530248387\\
99.68	0.00332366460773911\\
99.69	0.00323232675068253\\
99.7	0.00314026424868202\\
99.71	0.00304746952327101\\
99.72	0.0029539348986498\\
99.73	0.0028596526001464\\
99.74	0.00276461475264767\\
99.75	0.00266881337900764\\
99.76	0.00257224039842299\\
99.77	0.0024748876247747\\
99.78	0.00237674676493486\\
99.79	0.00227780941703762\\
99.8	0.00217806706871307\\
99.81	0.00207751109528306\\
99.82	0.00197613275791767\\
99.83	0.0018739232017511\\
99.84	0.00177087345395577\\
99.85	0.00166697442177312\\
99.86	0.00156221689049989\\
99.87	0.00145659152142833\\
99.88	0.00135008884973875\\
99.89	0.00124269928234298\\
99.9	0.00113441309567685\\
99.91	0.00102522043344024\\
99.92	0.000915111304282583\\
99.93	0.000804075579432168\\
99.94	0.000692102990267111\\
99.95	0.000579183125826001\\
99.96	0.00046530543025603\\
99.97	0.000350459200196341\\
99.98	0.000234633582094232\\
99.99	0.000117817569451738\\
100	0\\
};
\addlegendentry{$q=-1$};

\addplot [color=black,solid,forget plot]
  table[row sep=crcr]{%
0.01	0.00837856495180774\\
0.02	0.00837856495180774\\
0.03	0.00837856495180774\\
0.04	0.00837856495180774\\
0.05	0.00837856495180774\\
0.06	0.00837856495180774\\
0.07	0.00837856495180774\\
0.08	0.00837856495180774\\
0.09	0.00837856495180774\\
0.1	0.00837856495180774\\
0.11	0.00837856495180774\\
0.12	0.00837856495180774\\
0.13	0.00837856495180774\\
0.14	0.00837856495180774\\
0.15	0.00837856495180774\\
0.16	0.00837856495180774\\
0.17	0.00837856495180774\\
0.18	0.00837856495180774\\
0.19	0.00837856495180774\\
0.2	0.00837856495180774\\
0.21	0.00837856495180774\\
0.22	0.00837856495180774\\
0.23	0.00837856495180774\\
0.24	0.00837856495180774\\
0.25	0.00837856495180774\\
0.26	0.00837856495180774\\
0.27	0.00837856495180774\\
0.28	0.00837856495180774\\
0.29	0.00837856495180774\\
0.3	0.00837856495180774\\
0.31	0.00837856495180774\\
0.32	0.00837856495180774\\
0.33	0.00837856495180774\\
0.34	0.00837856495180774\\
0.35	0.00837856495180774\\
0.36	0.00837856495180774\\
0.37	0.00837856495180774\\
0.38	0.00837856495180774\\
0.39	0.00837856495180774\\
0.4	0.00837856495180774\\
0.41	0.00837856495180774\\
0.42	0.00837856495180774\\
0.43	0.00837856495180774\\
0.44	0.00837856495180774\\
0.45	0.00837856495180774\\
0.46	0.00837856495180774\\
0.47	0.00837856495180774\\
0.48	0.00837856495180774\\
0.49	0.00837856495180774\\
0.5	0.00837856495180774\\
0.51	0.00837856495180774\\
0.52	0.00837856495180774\\
0.53	0.00837856495180774\\
0.54	0.00837856495180774\\
0.55	0.00837856495180774\\
0.56	0.00837856495180774\\
0.57	0.00837856495180774\\
0.58	0.00837856495180774\\
0.59	0.00837856495180774\\
0.6	0.00837856495180774\\
0.61	0.00837856495180774\\
0.62	0.00837856495180774\\
0.63	0.00837856495180774\\
0.64	0.00837856495180774\\
0.65	0.00837856495180774\\
0.66	0.00837856495180774\\
0.67	0.00837856495180774\\
0.68	0.00837856495180774\\
0.69	0.00837856495180774\\
0.7	0.00837856495180774\\
0.71	0.00837856495180774\\
0.72	0.00837856495180774\\
0.73	0.00837856495180774\\
0.74	0.00837856495180774\\
0.75	0.00837856495180774\\
0.76	0.00837856495180774\\
0.77	0.00837856495180774\\
0.78	0.00837856495180774\\
0.79	0.00837856495180774\\
0.8	0.00837856495180774\\
0.81	0.00837856495180774\\
0.82	0.00837856495180774\\
0.83	0.00837856495180774\\
0.84	0.00837856495180774\\
0.85	0.00837856495180774\\
0.86	0.00837856495180774\\
0.87	0.00837856495180774\\
0.88	0.00837856495180774\\
0.89	0.00837856495180774\\
0.9	0.00837856495180774\\
0.91	0.00837856495180774\\
0.92	0.00837856495180774\\
0.93	0.00837856495180774\\
0.94	0.00837856495180774\\
0.95	0.00837856495180774\\
0.96	0.00837856495180774\\
0.97	0.00837856495180774\\
0.98	0.00837856495180774\\
0.99	0.00837856495180774\\
1	0.00837856495180774\\
1.01	0.00837856495180774\\
1.02	0.00837856495180774\\
1.03	0.00837856495180774\\
1.04	0.00837856495180774\\
1.05	0.00837856495180774\\
1.06	0.00837856495180774\\
1.07	0.00837856495180774\\
1.08	0.00837856495180774\\
1.09	0.00837856495180774\\
1.1	0.00837856495180774\\
1.11	0.00837856495180774\\
1.12	0.00837856495180774\\
1.13	0.00837856495180774\\
1.14	0.00837856495180774\\
1.15	0.00837856495180774\\
1.16	0.00837856495180774\\
1.17	0.00837856495180774\\
1.18	0.00837856495180774\\
1.19	0.00837856495180774\\
1.2	0.00837856495180774\\
1.21	0.00837856495180774\\
1.22	0.00837856495180774\\
1.23	0.00837856495180774\\
1.24	0.00837856495180774\\
1.25	0.00837856495180774\\
1.26	0.00837856495180774\\
1.27	0.00837856495180774\\
1.28	0.00837856495180774\\
1.29	0.00837856495180774\\
1.3	0.00837856495180774\\
1.31	0.00837856495180774\\
1.32	0.00837856495180774\\
1.33	0.00837856495180774\\
1.34	0.00837856495180774\\
1.35	0.00837856495180774\\
1.36	0.00837856495180774\\
1.37	0.00837856495180774\\
1.38	0.00837856495180774\\
1.39	0.00837856495180774\\
1.4	0.00837856495180774\\
1.41	0.00837856495180774\\
1.42	0.00837856495180774\\
1.43	0.00837856495180774\\
1.44	0.00837856495180774\\
1.45	0.00837856495180774\\
1.46	0.00837856495180774\\
1.47	0.00837856495180774\\
1.48	0.00837856495180774\\
1.49	0.00837856495180774\\
1.5	0.00837856495180774\\
1.51	0.00837856495180774\\
1.52	0.00837856495180774\\
1.53	0.00837856495180774\\
1.54	0.00837856495180774\\
1.55	0.00837856495180774\\
1.56	0.00837856495180774\\
1.57	0.00837856495180774\\
1.58	0.00837856495180774\\
1.59	0.00837856495180774\\
1.6	0.00837856495180774\\
1.61	0.00837856495180774\\
1.62	0.00837856495180774\\
1.63	0.00837856495180774\\
1.64	0.00837856495180774\\
1.65	0.00837856495180774\\
1.66	0.00837856495180774\\
1.67	0.00837856495180774\\
1.68	0.00837856495180774\\
1.69	0.00837856495180774\\
1.7	0.00837856495180774\\
1.71	0.00837856495180774\\
1.72	0.00837856495180774\\
1.73	0.00837856495180774\\
1.74	0.00837856495180774\\
1.75	0.00837856495180774\\
1.76	0.00837856495180774\\
1.77	0.00837856495180774\\
1.78	0.00837856495180774\\
1.79	0.00837856495180774\\
1.8	0.00837856495180774\\
1.81	0.00837856495180774\\
1.82	0.00837856495180774\\
1.83	0.00837856495180774\\
1.84	0.00837856495180774\\
1.85	0.00837856495180774\\
1.86	0.00837856495180774\\
1.87	0.00837856495180774\\
1.88	0.00837856495180774\\
1.89	0.00837856495180774\\
1.9	0.00837856495180774\\
1.91	0.00837856495180774\\
1.92	0.00837856495180774\\
1.93	0.00837856495180774\\
1.94	0.00837856495180774\\
1.95	0.00837856495180774\\
1.96	0.00837856495180774\\
1.97	0.00837856495180774\\
1.98	0.00837856495180774\\
1.99	0.00837856495180774\\
2	0.00837856495180774\\
2.01	0.00837856495180774\\
2.02	0.00837856495180774\\
2.03	0.00837856495180774\\
2.04	0.00837856495180774\\
2.05	0.00837856495180774\\
2.06	0.00837856495180774\\
2.07	0.00837856495180774\\
2.08	0.00837856495180774\\
2.09	0.00837856495180774\\
2.1	0.00837856495180774\\
2.11	0.00837856495180774\\
2.12	0.00837856495180774\\
2.13	0.00837856495180774\\
2.14	0.00837856495180774\\
2.15	0.00837856495180774\\
2.16	0.00837856495180774\\
2.17	0.00837856495180774\\
2.18	0.00837856495180774\\
2.19	0.00837856495180774\\
2.2	0.00837856495180774\\
2.21	0.00837856495180774\\
2.22	0.00837856495180774\\
2.23	0.00837856495180774\\
2.24	0.00837856495180774\\
2.25	0.00837856495180774\\
2.26	0.00837856495180774\\
2.27	0.00837856495180774\\
2.28	0.00837856495180774\\
2.29	0.00837856495180774\\
2.3	0.00837856495180774\\
2.31	0.00837856495180774\\
2.32	0.00837856495180774\\
2.33	0.00837856495180774\\
2.34	0.00837856495180774\\
2.35	0.00837856495180774\\
2.36	0.00837856495180774\\
2.37	0.00837856495180774\\
2.38	0.00837856495180774\\
2.39	0.00837856495180774\\
2.4	0.00837856495180774\\
2.41	0.00837856495180774\\
2.42	0.00837856495180774\\
2.43	0.00837856495180774\\
2.44	0.00837856495180774\\
2.45	0.00837856495180774\\
2.46	0.00837856495180774\\
2.47	0.00837856495180774\\
2.48	0.00837856495180774\\
2.49	0.00837856495180774\\
2.5	0.00837856495180774\\
2.51	0.00837856495180774\\
2.52	0.00837856495180774\\
2.53	0.00837856495180774\\
2.54	0.00837856495180774\\
2.55	0.00837856495180774\\
2.56	0.00837856495180774\\
2.57	0.00837856495180774\\
2.58	0.00837856495180774\\
2.59	0.00837856495180774\\
2.6	0.00837856495180774\\
2.61	0.00837856495180774\\
2.62	0.00837856495180774\\
2.63	0.00837856495180774\\
2.64	0.00837856495180774\\
2.65	0.00837856495180774\\
2.66	0.00837856495180774\\
2.67	0.00837856495180774\\
2.68	0.00837856495180774\\
2.69	0.00837856495180774\\
2.7	0.00837856495180774\\
2.71	0.00837856495180774\\
2.72	0.00837856495180774\\
2.73	0.00837856495180774\\
2.74	0.00837856495180774\\
2.75	0.00837856495180774\\
2.76	0.00837856495180774\\
2.77	0.00837856495180774\\
2.78	0.00837856495180774\\
2.79	0.00837856495180774\\
2.8	0.00837856495180774\\
2.81	0.00837856495180774\\
2.82	0.00837856495180774\\
2.83	0.00837856495180774\\
2.84	0.00837856495180774\\
2.85	0.00837856495180774\\
2.86	0.00837856495180774\\
2.87	0.00837856495180774\\
2.88	0.00837856495180774\\
2.89	0.00837856495180774\\
2.9	0.00837856495180774\\
2.91	0.00837856495180774\\
2.92	0.00837856495180774\\
2.93	0.00837856495180774\\
2.94	0.00837856495180774\\
2.95	0.00837856495180774\\
2.96	0.00837856495180774\\
2.97	0.00837856495180774\\
2.98	0.00837856495180774\\
2.99	0.00837856495180774\\
3	0.00837856495180774\\
3.01	0.00837856495180774\\
3.02	0.00837856495180774\\
3.03	0.00837856495180774\\
3.04	0.00837856495180774\\
3.05	0.00837856495180774\\
3.06	0.00837856495180774\\
3.07	0.00837856495180774\\
3.08	0.00837856495180774\\
3.09	0.00837856495180774\\
3.1	0.00837856495180774\\
3.11	0.00837856495180774\\
3.12	0.00837856495180774\\
3.13	0.00837856495180774\\
3.14	0.00837856495180774\\
3.15	0.00837856495180774\\
3.16	0.00837856495180774\\
3.17	0.00837856495180774\\
3.18	0.00837856495180774\\
3.19	0.00837856495180774\\
3.2	0.00837856495180774\\
3.21	0.00837856495180774\\
3.22	0.00837856495180774\\
3.23	0.00837856495180774\\
3.24	0.00837856495180774\\
3.25	0.00837856495180774\\
3.26	0.00837856495180774\\
3.27	0.00837856495180774\\
3.28	0.00837856495180774\\
3.29	0.00837856495180774\\
3.3	0.00837856495180774\\
3.31	0.00837856495180774\\
3.32	0.00837856495180774\\
3.33	0.00837856495180774\\
3.34	0.00837856495180774\\
3.35	0.00837856495180774\\
3.36	0.00837856495180774\\
3.37	0.00837856495180774\\
3.38	0.00837856495180774\\
3.39	0.00837856495180774\\
3.4	0.00837856495180774\\
3.41	0.00837856495180774\\
3.42	0.00837856495180774\\
3.43	0.00837856495180774\\
3.44	0.00837856495180774\\
3.45	0.00837856495180774\\
3.46	0.00837856495180774\\
3.47	0.00837856495180774\\
3.48	0.00837856495180774\\
3.49	0.00837856495180774\\
3.5	0.00837856495180774\\
3.51	0.00837856495180774\\
3.52	0.00837856495180774\\
3.53	0.00837856495180774\\
3.54	0.00837856495180774\\
3.55	0.00837856495180774\\
3.56	0.00837856495180774\\
3.57	0.00837856495180774\\
3.58	0.00837856495180774\\
3.59	0.00837856495180774\\
3.6	0.00837856495180774\\
3.61	0.00837856495180774\\
3.62	0.00837856495180774\\
3.63	0.00837856495180774\\
3.64	0.00837856495180774\\
3.65	0.00837856495180774\\
3.66	0.00837856495180774\\
3.67	0.00837856495180774\\
3.68	0.00837856495180774\\
3.69	0.00837856495180774\\
3.7	0.00837856495180774\\
3.71	0.00837856495180774\\
3.72	0.00837856495180774\\
3.73	0.00837856495180774\\
3.74	0.00837856495180774\\
3.75	0.00837856495180774\\
3.76	0.00837856495180774\\
3.77	0.00837856495180774\\
3.78	0.00837856495180774\\
3.79	0.00837856495180774\\
3.8	0.00837856495180774\\
3.81	0.00837856495180774\\
3.82	0.00837856495180774\\
3.83	0.00837856495180774\\
3.84	0.00837856495180774\\
3.85	0.00837856495180774\\
3.86	0.00837856495180774\\
3.87	0.00837856495180774\\
3.88	0.00837856495180774\\
3.89	0.00837856495180774\\
3.9	0.00837856495180774\\
3.91	0.00837856495180774\\
3.92	0.00837856495180774\\
3.93	0.00837856495180774\\
3.94	0.00837856495180774\\
3.95	0.00837856495180774\\
3.96	0.00837856495180774\\
3.97	0.00837856495180774\\
3.98	0.00837856495180774\\
3.99	0.00837856495180774\\
4	0.00837856495180774\\
4.01	0.00837856495180774\\
4.02	0.00837856495180774\\
4.03	0.00837856495180774\\
4.04	0.00837856495180774\\
4.05	0.00837856495180774\\
4.06	0.00837856495180774\\
4.07	0.00837856495180774\\
4.08	0.00837856495180774\\
4.09	0.00837856495180774\\
4.1	0.00837856495180774\\
4.11	0.00837856495180774\\
4.12	0.00837856495180774\\
4.13	0.00837856495180774\\
4.14	0.00837856495180774\\
4.15	0.00837856495180774\\
4.16	0.00837856495180774\\
4.17	0.00837856495180774\\
4.18	0.00837856495180774\\
4.19	0.00837856495180774\\
4.2	0.00837856495180774\\
4.21	0.00837856495180774\\
4.22	0.00837856495180774\\
4.23	0.00837856495180774\\
4.24	0.00837856495180774\\
4.25	0.00837856495180774\\
4.26	0.00837856495180774\\
4.27	0.00837856495180774\\
4.28	0.00837856495180774\\
4.29	0.00837856495180774\\
4.3	0.00837856495180774\\
4.31	0.00837856495180774\\
4.32	0.00837856495180774\\
4.33	0.00837856495180774\\
4.34	0.00837856495180774\\
4.35	0.00837856495180774\\
4.36	0.00837856495180774\\
4.37	0.00837856495180774\\
4.38	0.00837856495180774\\
4.39	0.00837856495180774\\
4.4	0.00837856495180774\\
4.41	0.00837856495180774\\
4.42	0.00837856495180774\\
4.43	0.00837856495180774\\
4.44	0.00837856495180774\\
4.45	0.00837856495180774\\
4.46	0.00837856495180774\\
4.47	0.00837856495180774\\
4.48	0.00837856495180774\\
4.49	0.00837856495180774\\
4.5	0.00837856495180774\\
4.51	0.00837856495180774\\
4.52	0.00837856495180774\\
4.53	0.00837856495180774\\
4.54	0.00837856495180774\\
4.55	0.00837856495180774\\
4.56	0.00837856495180774\\
4.57	0.00837856495180774\\
4.58	0.00837856495180774\\
4.59	0.00837856495180774\\
4.6	0.00837856495180774\\
4.61	0.00837856495180774\\
4.62	0.00837856495180774\\
4.63	0.00837856495180774\\
4.64	0.00837856495180774\\
4.65	0.00837856495180774\\
4.66	0.00837856495180774\\
4.67	0.00837856495180774\\
4.68	0.00837856495180774\\
4.69	0.00837856495180774\\
4.7	0.00837856495180774\\
4.71	0.00837856495180774\\
4.72	0.00837856495180774\\
4.73	0.00837856495180774\\
4.74	0.00837856495180774\\
4.75	0.00837856495180774\\
4.76	0.00837856495180774\\
4.77	0.00837856495180774\\
4.78	0.00837856495180774\\
4.79	0.00837856495180774\\
4.8	0.00837856495180774\\
4.81	0.00837856495180774\\
4.82	0.00837856495180774\\
4.83	0.00837856495180774\\
4.84	0.00837856495180774\\
4.85	0.00837856495180774\\
4.86	0.00837856495180774\\
4.87	0.00837856495180774\\
4.88	0.00837856495180774\\
4.89	0.00837856495180774\\
4.9	0.00837856495180774\\
4.91	0.00837856495180774\\
4.92	0.00837856495180774\\
4.93	0.00837856495180774\\
4.94	0.00837856495180774\\
4.95	0.00837856495180774\\
4.96	0.00837856495180774\\
4.97	0.00837856495180774\\
4.98	0.00837856495180774\\
4.99	0.00837856495180774\\
5	0.00837856495180774\\
5.01	0.00837856495180774\\
5.02	0.00837856495180774\\
5.03	0.00837856495180774\\
5.04	0.00837856495180774\\
5.05	0.00837856495180774\\
5.06	0.00837856495180774\\
5.07	0.00837856495180774\\
5.08	0.00837856495180774\\
5.09	0.00837856495180774\\
5.1	0.00837856495180774\\
5.11	0.00837856495180774\\
5.12	0.00837856495180774\\
5.13	0.00837856495180774\\
5.14	0.00837856495180774\\
5.15	0.00837856495180774\\
5.16	0.00837856495180774\\
5.17	0.00837856495180774\\
5.18	0.00837856495180774\\
5.19	0.00837856495180774\\
5.2	0.00837856495180774\\
5.21	0.00837856495180774\\
5.22	0.00837856495180774\\
5.23	0.00837856495180774\\
5.24	0.00837856495180774\\
5.25	0.00837856495180774\\
5.26	0.00837856495180774\\
5.27	0.00837856495180774\\
5.28	0.00837856495180774\\
5.29	0.00837856495180774\\
5.3	0.00837856495180774\\
5.31	0.00837856495180774\\
5.32	0.00837856495180774\\
5.33	0.00837856495180774\\
5.34	0.00837856495180774\\
5.35	0.00837856495180774\\
5.36	0.00837856495180774\\
5.37	0.00837856495180774\\
5.38	0.00837856495180774\\
5.39	0.00837856495180774\\
5.4	0.00837856495180774\\
5.41	0.00837856495180774\\
5.42	0.00837856495180774\\
5.43	0.00837856495180774\\
5.44	0.00837856495180774\\
5.45	0.00837856495180774\\
5.46	0.00837856495180774\\
5.47	0.00837856495180774\\
5.48	0.00837856495180774\\
5.49	0.00837856495180774\\
5.5	0.00837856495180774\\
5.51	0.00837856495180774\\
5.52	0.00837856495180774\\
5.53	0.00837856495180774\\
5.54	0.00837856495180774\\
5.55	0.00837856495180774\\
5.56	0.00837856495180774\\
5.57	0.00837856495180774\\
5.58	0.00837856495180774\\
5.59	0.00837856495180774\\
5.6	0.00837856495180774\\
5.61	0.00837856495180774\\
5.62	0.00837856495180774\\
5.63	0.00837856495180774\\
5.64	0.00837856495180774\\
5.65	0.00837856495180774\\
5.66	0.00837856495180774\\
5.67	0.00837856495180774\\
5.68	0.00837856495180774\\
5.69	0.00837856495180774\\
5.7	0.00837856495180774\\
5.71	0.00837856495180774\\
5.72	0.00837856495180774\\
5.73	0.00837856495180774\\
5.74	0.00837856495180774\\
5.75	0.00837856495180774\\
5.76	0.00837856495180774\\
5.77	0.00837856495180774\\
5.78	0.00837856495180774\\
5.79	0.00837856495180774\\
5.8	0.00837856495180774\\
5.81	0.00837856495180774\\
5.82	0.00837856495180774\\
5.83	0.00837856495180774\\
5.84	0.00837856495180774\\
5.85	0.00837856495180774\\
5.86	0.00837856495180774\\
5.87	0.00837856495180774\\
5.88	0.00837856495180774\\
5.89	0.00837856495180774\\
5.9	0.00837856495180774\\
5.91	0.00837856495180774\\
5.92	0.00837856495180774\\
5.93	0.00837856495180774\\
5.94	0.00837856495180774\\
5.95	0.00837856495180774\\
5.96	0.00837856495180774\\
5.97	0.00837856495180774\\
5.98	0.00837856495180774\\
5.99	0.00837856495180774\\
6	0.00837856495180774\\
6.01	0.00837856495180774\\
6.02	0.00837856495180774\\
6.03	0.00837856495180774\\
6.04	0.00837856495180774\\
6.05	0.00837856495180774\\
6.06	0.00837856495180774\\
6.07	0.00837856495180774\\
6.08	0.00837856495180774\\
6.09	0.00837856495180774\\
6.1	0.00837856495180774\\
6.11	0.00837856495180774\\
6.12	0.00837856495180774\\
6.13	0.00837856495180774\\
6.14	0.00837856495180774\\
6.15	0.00837856495180774\\
6.16	0.00837856495180774\\
6.17	0.00837856495180774\\
6.18	0.00837856495180774\\
6.19	0.00837856495180774\\
6.2	0.00837856495180774\\
6.21	0.00837856495180774\\
6.22	0.00837856495180774\\
6.23	0.00837856495180774\\
6.24	0.00837856495180774\\
6.25	0.00837856495180774\\
6.26	0.00837856495180774\\
6.27	0.00837856495180774\\
6.28	0.00837856495180774\\
6.29	0.00837856495180774\\
6.3	0.00837856495180774\\
6.31	0.00837856495180774\\
6.32	0.00837856495180774\\
6.33	0.00837856495180774\\
6.34	0.00837856495180774\\
6.35	0.00837856495180774\\
6.36	0.00837856495180774\\
6.37	0.00837856495180774\\
6.38	0.00837856495180774\\
6.39	0.00837856495180774\\
6.4	0.00837856495180774\\
6.41	0.00837856495180774\\
6.42	0.00837856495180774\\
6.43	0.00837856495180774\\
6.44	0.00837856495180774\\
6.45	0.00837856495180774\\
6.46	0.00837856495180774\\
6.47	0.00837856495180774\\
6.48	0.00837856495180774\\
6.49	0.00837856495180774\\
6.5	0.00837856495180774\\
6.51	0.00837856495180774\\
6.52	0.00837856495180774\\
6.53	0.00837856495180774\\
6.54	0.00837856495180774\\
6.55	0.00837856495180774\\
6.56	0.00837856495180774\\
6.57	0.00837856495180774\\
6.58	0.00837856495180774\\
6.59	0.00837856495180774\\
6.6	0.00837856495180774\\
6.61	0.00837856495180774\\
6.62	0.00837856495180774\\
6.63	0.00837856495180774\\
6.64	0.00837856495180774\\
6.65	0.00837856495180774\\
6.66	0.00837856495180774\\
6.67	0.00837856495180774\\
6.68	0.00837856495180774\\
6.69	0.00837856495180774\\
6.7	0.00837856495180774\\
6.71	0.00837856495180774\\
6.72	0.00837856495180774\\
6.73	0.00837856495180774\\
6.74	0.00837856495180774\\
6.75	0.00837856495180774\\
6.76	0.00837856495180774\\
6.77	0.00837856495180774\\
6.78	0.00837856495180774\\
6.79	0.00837856495180774\\
6.8	0.00837856495180774\\
6.81	0.00837856495180774\\
6.82	0.00837856495180774\\
6.83	0.00837856495180774\\
6.84	0.00837856495180774\\
6.85	0.00837856495180774\\
6.86	0.00837856495180774\\
6.87	0.00837856495180774\\
6.88	0.00837856495180774\\
6.89	0.00837856495180774\\
6.9	0.00837856495180774\\
6.91	0.00837856495180774\\
6.92	0.00837856495180774\\
6.93	0.00837856495180774\\
6.94	0.00837856495180774\\
6.95	0.00837856495180774\\
6.96	0.00837856495180774\\
6.97	0.00837856495180774\\
6.98	0.00837856495180774\\
6.99	0.00837856495180774\\
7	0.00837856495180774\\
7.01	0.00837856495180774\\
7.02	0.00837856495180774\\
7.03	0.00837856495180774\\
7.04	0.00837856495180774\\
7.05	0.00837856495180774\\
7.06	0.00837856495180774\\
7.07	0.00837856495180774\\
7.08	0.00837856495180774\\
7.09	0.00837856495180774\\
7.1	0.00837856495180774\\
7.11	0.00837856495180774\\
7.12	0.00837856495180774\\
7.13	0.00837856495180774\\
7.14	0.00837856495180774\\
7.15	0.00837856495180774\\
7.16	0.00837856495180774\\
7.17	0.00837856495180774\\
7.18	0.00837856495180774\\
7.19	0.00837856495180774\\
7.2	0.00837856495180774\\
7.21	0.00837856495180774\\
7.22	0.00837856495180774\\
7.23	0.00837856495180774\\
7.24	0.00837856495180774\\
7.25	0.00837856495180774\\
7.26	0.00837856495180774\\
7.27	0.00837856495180774\\
7.28	0.00837856495180774\\
7.29	0.00837856495180774\\
7.3	0.00837856495180774\\
7.31	0.00837856495180774\\
7.32	0.00837856495180774\\
7.33	0.00837856495180774\\
7.34	0.00837856495180774\\
7.35	0.00837856495180774\\
7.36	0.00837856495180774\\
7.37	0.00837856495180774\\
7.38	0.00837856495180774\\
7.39	0.00837856495180774\\
7.4	0.00837856495180774\\
7.41	0.00837856495180774\\
7.42	0.00837856495180774\\
7.43	0.00837856495180774\\
7.44	0.00837856495180774\\
7.45	0.00837856495180774\\
7.46	0.00837856495180774\\
7.47	0.00837856495180774\\
7.48	0.00837856495180774\\
7.49	0.00837856495180774\\
7.5	0.00837856495180774\\
7.51	0.00837856495180774\\
7.52	0.00837856495180774\\
7.53	0.00837856495180774\\
7.54	0.00837856495180774\\
7.55	0.00837856495180774\\
7.56	0.00837856495180774\\
7.57	0.00837856495180774\\
7.58	0.00837856495180774\\
7.59	0.00837856495180774\\
7.6	0.00837856495180774\\
7.61	0.00837856495180774\\
7.62	0.00837856495180774\\
7.63	0.00837856495180774\\
7.64	0.00837856495180774\\
7.65	0.00837856495180774\\
7.66	0.00837856495180774\\
7.67	0.00837856495180774\\
7.68	0.00837856495180774\\
7.69	0.00837856495180774\\
7.7	0.00837856495180774\\
7.71	0.00837856495180774\\
7.72	0.00837856495180774\\
7.73	0.00837856495180774\\
7.74	0.00837856495180774\\
7.75	0.00837856495180774\\
7.76	0.00837856495180774\\
7.77	0.00837856495180774\\
7.78	0.00837856495180774\\
7.79	0.00837856495180774\\
7.8	0.00837856495180774\\
7.81	0.00837856495180774\\
7.82	0.00837856495180774\\
7.83	0.00837856495180774\\
7.84	0.00837856495180774\\
7.85	0.00837856495180774\\
7.86	0.00837856495180774\\
7.87	0.00837856495180774\\
7.88	0.00837856495180774\\
7.89	0.00837856495180774\\
7.9	0.00837856495180774\\
7.91	0.00837856495180774\\
7.92	0.00837856495180774\\
7.93	0.00837856495180774\\
7.94	0.00837856495180774\\
7.95	0.00837856495180774\\
7.96	0.00837856495180774\\
7.97	0.00837856495180774\\
7.98	0.00837856495180774\\
7.99	0.00837856495180774\\
8	0.00837856495180774\\
8.01	0.00837856495180774\\
8.02	0.00837856495180774\\
8.03	0.00837856495180774\\
8.04	0.00837856495180774\\
8.05	0.00837856495180774\\
8.06	0.00837856495180774\\
8.07	0.00837856495180774\\
8.08	0.00837856495180774\\
8.09	0.00837856495180774\\
8.1	0.00837856495180774\\
8.11	0.00837856495180774\\
8.12	0.00837856495180774\\
8.13	0.00837856495180774\\
8.14	0.00837856495180774\\
8.15	0.00837856495180774\\
8.16	0.00837856495180774\\
8.17	0.00837856495180774\\
8.18	0.00837856495180774\\
8.19	0.00837856495180774\\
8.2	0.00837856495180774\\
8.21	0.00837856495180774\\
8.22	0.00837856495180774\\
8.23	0.00837856495180774\\
8.24	0.00837856495180774\\
8.25	0.00837856495180774\\
8.26	0.00837856495180774\\
8.27	0.00837856495180774\\
8.28	0.00837856495180774\\
8.29	0.00837856495180774\\
8.3	0.00837856495180774\\
8.31	0.00837856495180774\\
8.32	0.00837856495180774\\
8.33	0.00837856495180774\\
8.34	0.00837856495180774\\
8.35	0.00837856495180774\\
8.36	0.00837856495180774\\
8.37	0.00837856495180774\\
8.38	0.00837856495180774\\
8.39	0.00837856495180774\\
8.4	0.00837856495180774\\
8.41	0.00837856495180774\\
8.42	0.00837856495180774\\
8.43	0.00837856495180774\\
8.44	0.00837856495180774\\
8.45	0.00837856495180774\\
8.46	0.00837856495180774\\
8.47	0.00837856495180774\\
8.48	0.00837856495180774\\
8.49	0.00837856495180774\\
8.5	0.00837856495180774\\
8.51	0.00837856495180774\\
8.52	0.00837856495180774\\
8.53	0.00837856495180774\\
8.54	0.00837856495180774\\
8.55	0.00837856495180774\\
8.56	0.00837856495180774\\
8.57	0.00837856495180774\\
8.58	0.00837856495180774\\
8.59	0.00837856495180774\\
8.6	0.00837856495180774\\
8.61	0.00837856495180774\\
8.62	0.00837856495180774\\
8.63	0.00837856495180774\\
8.64	0.00837856495180774\\
8.65	0.00837856495180774\\
8.66	0.00837856495180774\\
8.67	0.00837856495180774\\
8.68	0.00837856495180774\\
8.69	0.00837856495180774\\
8.7	0.00837856495180774\\
8.71	0.00837856495180774\\
8.72	0.00837856495180774\\
8.73	0.00837856495180774\\
8.74	0.00837856495180774\\
8.75	0.00837856495180774\\
8.76	0.00837856495180774\\
8.77	0.00837856495180774\\
8.78	0.00837856495180774\\
8.79	0.00837856495180774\\
8.8	0.00837856495180774\\
8.81	0.00837856495180774\\
8.82	0.00837856495180774\\
8.83	0.00837856495180774\\
8.84	0.00837856495180774\\
8.85	0.00837856495180774\\
8.86	0.00837856495180774\\
8.87	0.00837856495180774\\
8.88	0.00837856495180774\\
8.89	0.00837856495180774\\
8.9	0.00837856495180774\\
8.91	0.00837856495180774\\
8.92	0.00837856495180774\\
8.93	0.00837856495180774\\
8.94	0.00837856495180774\\
8.95	0.00837856495180774\\
8.96	0.00837856495180774\\
8.97	0.00837856495180774\\
8.98	0.00837856495180774\\
8.99	0.00837856495180774\\
9	0.00837856495180774\\
9.01	0.00837856495180774\\
9.02	0.00837856495180774\\
9.03	0.00837856495180774\\
9.04	0.00837856495180774\\
9.05	0.00837856495180774\\
9.06	0.00837856495180774\\
9.07	0.00837856495180774\\
9.08	0.00837856495180774\\
9.09	0.00837856495180774\\
9.1	0.00837856495180774\\
9.11	0.00837856495180774\\
9.12	0.00837856495180774\\
9.13	0.00837856495180774\\
9.14	0.00837856495180774\\
9.15	0.00837856495180774\\
9.16	0.00837856495180774\\
9.17	0.00837856495180774\\
9.18	0.00837856495180774\\
9.19	0.00837856495180774\\
9.2	0.00837856495180774\\
9.21	0.00837856495180774\\
9.22	0.00837856495180774\\
9.23	0.00837856495180774\\
9.24	0.00837856495180774\\
9.25	0.00837856495180774\\
9.26	0.00837856495180774\\
9.27	0.00837856495180774\\
9.28	0.00837856495180774\\
9.29	0.00837856495180774\\
9.3	0.00837856495180774\\
9.31	0.00837856495180774\\
9.32	0.00837856495180774\\
9.33	0.00837856495180774\\
9.34	0.00837856495180774\\
9.35	0.00837856495180774\\
9.36	0.00837856495180774\\
9.37	0.00837856495180774\\
9.38	0.00837856495180774\\
9.39	0.00837856495180774\\
9.4	0.00837856495180774\\
9.41	0.00837856495180774\\
9.42	0.00837856495180774\\
9.43	0.00837856495180774\\
9.44	0.00837856495180774\\
9.45	0.00837856495180774\\
9.46	0.00837856495180774\\
9.47	0.00837856495180774\\
9.48	0.00837856495180774\\
9.49	0.00837856495180774\\
9.5	0.00837856495180774\\
9.51	0.00837856495180774\\
9.52	0.00837856495180774\\
9.53	0.00837856495180774\\
9.54	0.00837856495180774\\
9.55	0.00837856495180774\\
9.56	0.00837856495180774\\
9.57	0.00837856495180774\\
9.58	0.00837856495180774\\
9.59	0.00837856495180774\\
9.6	0.00837856495180774\\
9.61	0.00837856495180774\\
9.62	0.00837856495180774\\
9.63	0.00837856495180774\\
9.64	0.00837856495180774\\
9.65	0.00837856495180774\\
9.66	0.00837856495180774\\
9.67	0.00837856495180774\\
9.68	0.00837856495180774\\
9.69	0.00837856495180774\\
9.7	0.00837856495180774\\
9.71	0.00837856495180774\\
9.72	0.00837856495180774\\
9.73	0.00837856495180774\\
9.74	0.00837856495180774\\
9.75	0.00837856495180774\\
9.76	0.00837856495180774\\
9.77	0.00837856495180774\\
9.78	0.00837856495180774\\
9.79	0.00837856495180774\\
9.8	0.00837856495180774\\
9.81	0.00837856495180774\\
9.82	0.00837856495180774\\
9.83	0.00837856495180774\\
9.84	0.00837856495180774\\
9.85	0.00837856495180774\\
9.86	0.00837856495180774\\
9.87	0.00837856495180774\\
9.88	0.00837856495180774\\
9.89	0.00837856495180774\\
9.9	0.00837856495180774\\
9.91	0.00837856495180774\\
9.92	0.00837856495180774\\
9.93	0.00837856495180774\\
9.94	0.00837856495180774\\
9.95	0.00837856495180774\\
9.96	0.00837856495180774\\
9.97	0.00837856495180774\\
9.98	0.00837856495180774\\
9.99	0.00837856495180774\\
10	0.00837856495180774\\
10.01	0.00837856495180774\\
10.02	0.00837856495180774\\
10.03	0.00837856495180774\\
10.04	0.00837856495180774\\
10.05	0.00837856495180774\\
10.06	0.00837856495180774\\
10.07	0.00837856495180774\\
10.08	0.00837856495180774\\
10.09	0.00837856495180774\\
10.1	0.00837856495180774\\
10.11	0.00837856495180774\\
10.12	0.00837856495180774\\
10.13	0.00837856495180774\\
10.14	0.00837856495180774\\
10.15	0.00837856495180774\\
10.16	0.00837856495180774\\
10.17	0.00837856495180774\\
10.18	0.00837856495180774\\
10.19	0.00837856495180774\\
10.2	0.00837856495180774\\
10.21	0.00837856495180774\\
10.22	0.00837856495180774\\
10.23	0.00837856495180774\\
10.24	0.00837856495180774\\
10.25	0.00837856495180774\\
10.26	0.00837856495180774\\
10.27	0.00837856495180774\\
10.28	0.00837856495180774\\
10.29	0.00837856495180774\\
10.3	0.00837856495180774\\
10.31	0.00837856495180774\\
10.32	0.00837856495180774\\
10.33	0.00837856495180774\\
10.34	0.00837856495180774\\
10.35	0.00837856495180774\\
10.36	0.00837856495180774\\
10.37	0.00837856495180774\\
10.38	0.00837856495180774\\
10.39	0.00837856495180774\\
10.4	0.00837856495180774\\
10.41	0.00837856495180774\\
10.42	0.00837856495180774\\
10.43	0.00837856495180774\\
10.44	0.00837856495180774\\
10.45	0.00837856495180774\\
10.46	0.00837856495180774\\
10.47	0.00837856495180774\\
10.48	0.00837856495180774\\
10.49	0.00837856495180774\\
10.5	0.00837856495180774\\
10.51	0.00837856495180774\\
10.52	0.00837856495180774\\
10.53	0.00837856495180774\\
10.54	0.00837856495180774\\
10.55	0.00837856495180774\\
10.56	0.00837856495180774\\
10.57	0.00837856495180774\\
10.58	0.00837856495180774\\
10.59	0.00837856495180774\\
10.6	0.00837856495180774\\
10.61	0.00837856495180774\\
10.62	0.00837856495180774\\
10.63	0.00837856495180774\\
10.64	0.00837856495180774\\
10.65	0.00837856495180774\\
10.66	0.00837856495180774\\
10.67	0.00837856495180774\\
10.68	0.00837856495180774\\
10.69	0.00837856495180774\\
10.7	0.00837856495180774\\
10.71	0.00837856495180774\\
10.72	0.00837856495180774\\
10.73	0.00837856495180774\\
10.74	0.00837856495180774\\
10.75	0.00837856495180774\\
10.76	0.00837856495180774\\
10.77	0.00837856495180774\\
10.78	0.00837856495180774\\
10.79	0.00837856495180774\\
10.8	0.00837856495180774\\
10.81	0.00837856495180774\\
10.82	0.00837856495180774\\
10.83	0.00837856495180774\\
10.84	0.00837856495180774\\
10.85	0.00837856495180774\\
10.86	0.00837856495180774\\
10.87	0.00837856495180774\\
10.88	0.00837856495180774\\
10.89	0.00837856495180774\\
10.9	0.00837856495180774\\
10.91	0.00837856495180774\\
10.92	0.00837856495180774\\
10.93	0.00837856495180774\\
10.94	0.00837856495180774\\
10.95	0.00837856495180774\\
10.96	0.00837856495180774\\
10.97	0.00837856495180774\\
10.98	0.00837856495180774\\
10.99	0.00837856495180774\\
11	0.00837856495180774\\
11.01	0.00837856495180774\\
11.02	0.00837856495180774\\
11.03	0.00837856495180774\\
11.04	0.00837856495180774\\
11.05	0.00837856495180774\\
11.06	0.00837856495180774\\
11.07	0.00837856495180774\\
11.08	0.00837856495180774\\
11.09	0.00837856495180774\\
11.1	0.00837856495180774\\
11.11	0.00837856495180774\\
11.12	0.00837856495180774\\
11.13	0.00837856495180774\\
11.14	0.00837856495180774\\
11.15	0.00837856495180774\\
11.16	0.00837856495180774\\
11.17	0.00837856495180774\\
11.18	0.00837856495180774\\
11.19	0.00837856495180774\\
11.2	0.00837856495180774\\
11.21	0.00837856495180774\\
11.22	0.00837856495180774\\
11.23	0.00837856495180774\\
11.24	0.00837856495180774\\
11.25	0.00837856495180774\\
11.26	0.00837856495180774\\
11.27	0.00837856495180774\\
11.28	0.00837856495180774\\
11.29	0.00837856495180774\\
11.3	0.00837856495180774\\
11.31	0.00837856495180774\\
11.32	0.00837856495180774\\
11.33	0.00837856495180774\\
11.34	0.00837856495180774\\
11.35	0.00837856495180774\\
11.36	0.00837856495180774\\
11.37	0.00837856495180774\\
11.38	0.00837856495180774\\
11.39	0.00837856495180774\\
11.4	0.00837856495180774\\
11.41	0.00837856495180774\\
11.42	0.00837856495180774\\
11.43	0.00837856495180774\\
11.44	0.00837856495180774\\
11.45	0.00837856495180774\\
11.46	0.00837856495180774\\
11.47	0.00837856495180774\\
11.48	0.00837856495180774\\
11.49	0.00837856495180774\\
11.5	0.00837856495180774\\
11.51	0.00837856495180774\\
11.52	0.00837856495180774\\
11.53	0.00837856495180774\\
11.54	0.00837856495180774\\
11.55	0.00837856495180774\\
11.56	0.00837856495180774\\
11.57	0.00837856495180774\\
11.58	0.00837856495180774\\
11.59	0.00837856495180774\\
11.6	0.00837856495180774\\
11.61	0.00837856495180774\\
11.62	0.00837856495180774\\
11.63	0.00837856495180774\\
11.64	0.00837856495180774\\
11.65	0.00837856495180774\\
11.66	0.00837856495180774\\
11.67	0.00837856495180774\\
11.68	0.00837856495180774\\
11.69	0.00837856495180774\\
11.7	0.00837856495180774\\
11.71	0.00837856495180774\\
11.72	0.00837856495180774\\
11.73	0.00837856495180774\\
11.74	0.00837856495180774\\
11.75	0.00837856495180774\\
11.76	0.00837856495180774\\
11.77	0.00837856495180774\\
11.78	0.00837856495180774\\
11.79	0.00837856495180774\\
11.8	0.00837856495180774\\
11.81	0.00837856495180774\\
11.82	0.00837856495180774\\
11.83	0.00837856495180774\\
11.84	0.00837856495180774\\
11.85	0.00837856495180774\\
11.86	0.00837856495180774\\
11.87	0.00837856495180774\\
11.88	0.00837856495180774\\
11.89	0.00837856495180774\\
11.9	0.00837856495180774\\
11.91	0.00837856495180774\\
11.92	0.00837856495180774\\
11.93	0.00837856495180774\\
11.94	0.00837856495180774\\
11.95	0.00837856495180774\\
11.96	0.00837856495180774\\
11.97	0.00837856495180774\\
11.98	0.00837856495180774\\
11.99	0.00837856495180774\\
12	0.00837856495180774\\
12.01	0.00837856495180774\\
12.02	0.00837856495180774\\
12.03	0.00837856495180774\\
12.04	0.00837856495180774\\
12.05	0.00837856495180774\\
12.06	0.00837856495180774\\
12.07	0.00837856495180774\\
12.08	0.00837856495180774\\
12.09	0.00837856495180774\\
12.1	0.00837856495180774\\
12.11	0.00837856495180774\\
12.12	0.00837856495180774\\
12.13	0.00837856495180774\\
12.14	0.00837856495180774\\
12.15	0.00837856495180774\\
12.16	0.00837856495180774\\
12.17	0.00837856495180774\\
12.18	0.00837856495180774\\
12.19	0.00837856495180774\\
12.2	0.00837856495180774\\
12.21	0.00837856495180774\\
12.22	0.00837856495180774\\
12.23	0.00837856495180774\\
12.24	0.00837856495180774\\
12.25	0.00837856495180774\\
12.26	0.00837856495180774\\
12.27	0.00837856495180774\\
12.28	0.00837856495180774\\
12.29	0.00837856495180774\\
12.3	0.00837856495180774\\
12.31	0.00837856495180774\\
12.32	0.00837856495180774\\
12.33	0.00837856495180774\\
12.34	0.00837856495180774\\
12.35	0.00837856495180774\\
12.36	0.00837856495180774\\
12.37	0.00837856495180774\\
12.38	0.00837856495180774\\
12.39	0.00837856495180774\\
12.4	0.00837856495180774\\
12.41	0.00837856495180774\\
12.42	0.00837856495180774\\
12.43	0.00837856495180774\\
12.44	0.00837856495180774\\
12.45	0.00837856495180774\\
12.46	0.00837856495180774\\
12.47	0.00837856495180774\\
12.48	0.00837856495180774\\
12.49	0.00837856495180774\\
12.5	0.00837856495180774\\
12.51	0.00837856495180774\\
12.52	0.00837856495180774\\
12.53	0.00837856495180774\\
12.54	0.00837856495180774\\
12.55	0.00837856495180774\\
12.56	0.00837856495180774\\
12.57	0.00837856495180774\\
12.58	0.00837856495180774\\
12.59	0.00837856495180774\\
12.6	0.00837856495180774\\
12.61	0.00837856495180774\\
12.62	0.00837856495180774\\
12.63	0.00837856495180774\\
12.64	0.00837856495180774\\
12.65	0.00837856495180774\\
12.66	0.00837856495180774\\
12.67	0.00837856495180774\\
12.68	0.00837856495180774\\
12.69	0.00837856495180774\\
12.7	0.00837856495180774\\
12.71	0.00837856495180774\\
12.72	0.00837856495180774\\
12.73	0.00837856495180774\\
12.74	0.00837856495180774\\
12.75	0.00837856495180774\\
12.76	0.00837856495180774\\
12.77	0.00837856495180774\\
12.78	0.00837856495180774\\
12.79	0.00837856495180774\\
12.8	0.00837856495180774\\
12.81	0.00837856495180774\\
12.82	0.00837856495180774\\
12.83	0.00837856495180774\\
12.84	0.00837856495180774\\
12.85	0.00837856495180774\\
12.86	0.00837856495180774\\
12.87	0.00837856495180774\\
12.88	0.00837856495180774\\
12.89	0.00837856495180774\\
12.9	0.00837856495180774\\
12.91	0.00837856495180774\\
12.92	0.00837856495180774\\
12.93	0.00837856495180774\\
12.94	0.00837856495180774\\
12.95	0.00837856495180774\\
12.96	0.00837856495180774\\
12.97	0.00837856495180774\\
12.98	0.00837856495180774\\
12.99	0.00837856495180774\\
13	0.00837856495180774\\
13.01	0.00837856495180774\\
13.02	0.00837856495180774\\
13.03	0.00837856495180774\\
13.04	0.00837856495180774\\
13.05	0.00837856495180774\\
13.06	0.00837856495180774\\
13.07	0.00837856495180774\\
13.08	0.00837856495180774\\
13.09	0.00837856495180774\\
13.1	0.00837856495180774\\
13.11	0.00837856495180774\\
13.12	0.00837856495180774\\
13.13	0.00837856495180774\\
13.14	0.00837856495180774\\
13.15	0.00837856495180774\\
13.16	0.00837856495180774\\
13.17	0.00837856495180774\\
13.18	0.00837856495180774\\
13.19	0.00837856495180774\\
13.2	0.00837856495180774\\
13.21	0.00837856495180774\\
13.22	0.00837856495180774\\
13.23	0.00837856495180774\\
13.24	0.00837856495180774\\
13.25	0.00837856495180774\\
13.26	0.00837856495180774\\
13.27	0.00837856495180774\\
13.28	0.00837856495180774\\
13.29	0.00837856495180774\\
13.3	0.00837856495180774\\
13.31	0.00837856495180774\\
13.32	0.00837856495180774\\
13.33	0.00837856495180774\\
13.34	0.00837856495180774\\
13.35	0.00837856495180774\\
13.36	0.00837856495180774\\
13.37	0.00837856495180774\\
13.38	0.00837856495180774\\
13.39	0.00837856495180774\\
13.4	0.00837856495180774\\
13.41	0.00837856495180774\\
13.42	0.00837856495180774\\
13.43	0.00837856495180774\\
13.44	0.00837856495180774\\
13.45	0.00837856495180774\\
13.46	0.00837856495180774\\
13.47	0.00837856495180774\\
13.48	0.00837856495180774\\
13.49	0.00837856495180774\\
13.5	0.00837856495180774\\
13.51	0.00837856495180774\\
13.52	0.00837856495180774\\
13.53	0.00837856495180774\\
13.54	0.00837856495180774\\
13.55	0.00837856495180774\\
13.56	0.00837856495180774\\
13.57	0.00837856495180774\\
13.58	0.00837856495180774\\
13.59	0.00837856495180774\\
13.6	0.00837856495180774\\
13.61	0.00837856495180774\\
13.62	0.00837856495180774\\
13.63	0.00837856495180774\\
13.64	0.00837856495180774\\
13.65	0.00837856495180774\\
13.66	0.00837856495180774\\
13.67	0.00837856495180774\\
13.68	0.00837856495180774\\
13.69	0.00837856495180774\\
13.7	0.00837856495180774\\
13.71	0.00837856495180774\\
13.72	0.00837856495180774\\
13.73	0.00837856495180774\\
13.74	0.00837856495180774\\
13.75	0.00837856495180774\\
13.76	0.00837856495180774\\
13.77	0.00837856495180774\\
13.78	0.00837856495180774\\
13.79	0.00837856495180774\\
13.8	0.00837856495180774\\
13.81	0.00837856495180774\\
13.82	0.00837856495180774\\
13.83	0.00837856495180774\\
13.84	0.00837856495180774\\
13.85	0.00837856495180774\\
13.86	0.00837856495180774\\
13.87	0.00837856495180774\\
13.88	0.00837856495180774\\
13.89	0.00837856495180774\\
13.9	0.00837856495180774\\
13.91	0.00837856495180774\\
13.92	0.00837856495180774\\
13.93	0.00837856495180774\\
13.94	0.00837856495180774\\
13.95	0.00837856495180774\\
13.96	0.00837856495180774\\
13.97	0.00837856495180774\\
13.98	0.00837856495180774\\
13.99	0.00837856495180774\\
14	0.00837856495180774\\
14.01	0.00837856495180774\\
14.02	0.00837856495180774\\
14.03	0.00837856495180774\\
14.04	0.00837856495180774\\
14.05	0.00837856495180774\\
14.06	0.00837856495180774\\
14.07	0.00837856495180774\\
14.08	0.00837856495180774\\
14.09	0.00837856495180774\\
14.1	0.00837856495180774\\
14.11	0.00837856495180774\\
14.12	0.00837856495180774\\
14.13	0.00837856495180774\\
14.14	0.00837856495180774\\
14.15	0.00837856495180774\\
14.16	0.00837856495180774\\
14.17	0.00837856495180774\\
14.18	0.00837856495180774\\
14.19	0.00837856495180774\\
14.2	0.00837856495180774\\
14.21	0.00837856495180774\\
14.22	0.00837856495180774\\
14.23	0.00837856495180774\\
14.24	0.00837856495180774\\
14.25	0.00837856495180774\\
14.26	0.00837856495180774\\
14.27	0.00837856495180774\\
14.28	0.00837856495180774\\
14.29	0.00837856495180774\\
14.3	0.00837856495180774\\
14.31	0.00837856495180774\\
14.32	0.00837856495180774\\
14.33	0.00837856495180774\\
14.34	0.00837856495180774\\
14.35	0.00837856495180774\\
14.36	0.00837856495180774\\
14.37	0.00837856495180774\\
14.38	0.00837856495180774\\
14.39	0.00837856495180774\\
14.4	0.00837856495180774\\
14.41	0.00837856495180774\\
14.42	0.00837856495180774\\
14.43	0.00837856495180774\\
14.44	0.00837856495180774\\
14.45	0.00837856495180774\\
14.46	0.00837856495180774\\
14.47	0.00837856495180774\\
14.48	0.00837856495180774\\
14.49	0.00837856495180774\\
14.5	0.00837856495180774\\
14.51	0.00837856495180774\\
14.52	0.00837856495180774\\
14.53	0.00837856495180774\\
14.54	0.00837856495180774\\
14.55	0.00837856495180774\\
14.56	0.00837856495180774\\
14.57	0.00837856495180774\\
14.58	0.00837856495180774\\
14.59	0.00837856495180774\\
14.6	0.00837856495180774\\
14.61	0.00837856495180774\\
14.62	0.00837856495180774\\
14.63	0.00837856495180774\\
14.64	0.00837856495180774\\
14.65	0.00837856495180774\\
14.66	0.00837856495180774\\
14.67	0.00837856495180774\\
14.68	0.00837856495180774\\
14.69	0.00837856495180774\\
14.7	0.00837856495180774\\
14.71	0.00837856495180774\\
14.72	0.00837856495180774\\
14.73	0.00837856495180774\\
14.74	0.00837856495180774\\
14.75	0.00837856495180774\\
14.76	0.00837856495180774\\
14.77	0.00837856495180774\\
14.78	0.00837856495180774\\
14.79	0.00837856495180774\\
14.8	0.00837856495180774\\
14.81	0.00837856495180774\\
14.82	0.00837856495180774\\
14.83	0.00837856495180774\\
14.84	0.00837856495180774\\
14.85	0.00837856495180774\\
14.86	0.00837856495180774\\
14.87	0.00837856495180774\\
14.88	0.00837856495180774\\
14.89	0.00837856495180774\\
14.9	0.00837856495180774\\
14.91	0.00837856495180774\\
14.92	0.00837856495180774\\
14.93	0.00837856495180774\\
14.94	0.00837856495180774\\
14.95	0.00837856495180774\\
14.96	0.00837856495180774\\
14.97	0.00837856495180774\\
14.98	0.00837856495180774\\
14.99	0.00837856495180774\\
15	0.00837856495180774\\
15.01	0.00837856495180774\\
15.02	0.00837856495180774\\
15.03	0.00837856495180774\\
15.04	0.00837856495180774\\
15.05	0.00837856495180774\\
15.06	0.00837856495180774\\
15.07	0.00837856495180774\\
15.08	0.00837856495180774\\
15.09	0.00837856495180774\\
15.1	0.00837856495180774\\
15.11	0.00837856495180774\\
15.12	0.00837856495180774\\
15.13	0.00837856495180774\\
15.14	0.00837856495180774\\
15.15	0.00837856495180774\\
15.16	0.00837856495180774\\
15.17	0.00837856495180774\\
15.18	0.00837856495180774\\
15.19	0.00837856495180774\\
15.2	0.00837856495180774\\
15.21	0.00837856495180774\\
15.22	0.00837856495180774\\
15.23	0.00837856495180774\\
15.24	0.00837856495180774\\
15.25	0.00837856495180774\\
15.26	0.00837856495180774\\
15.27	0.00837856495180774\\
15.28	0.00837856495180774\\
15.29	0.00837856495180774\\
15.3	0.00837856495180774\\
15.31	0.00837856495180774\\
15.32	0.00837856495180774\\
15.33	0.00837856495180774\\
15.34	0.00837856495180774\\
15.35	0.00837856495180774\\
15.36	0.00837856495180774\\
15.37	0.00837856495180774\\
15.38	0.00837856495180774\\
15.39	0.00837856495180774\\
15.4	0.00837856495180774\\
15.41	0.00837856495180774\\
15.42	0.00837856495180774\\
15.43	0.00837856495180774\\
15.44	0.00837856495180774\\
15.45	0.00837856495180774\\
15.46	0.00837856495180774\\
15.47	0.00837856495180774\\
15.48	0.00837856495180774\\
15.49	0.00837856495180774\\
15.5	0.00837856495180774\\
15.51	0.00837856495180774\\
15.52	0.00837856495180774\\
15.53	0.00837856495180774\\
15.54	0.00837856495180774\\
15.55	0.00837856495180774\\
15.56	0.00837856495180774\\
15.57	0.00837856495180774\\
15.58	0.00837856495180774\\
15.59	0.00837856495180774\\
15.6	0.00837856495180774\\
15.61	0.00837856495180774\\
15.62	0.00837856495180774\\
15.63	0.00837856495180774\\
15.64	0.00837856495180774\\
15.65	0.00837856495180774\\
15.66	0.00837856495180774\\
15.67	0.00837856495180774\\
15.68	0.00837856495180774\\
15.69	0.00837856495180774\\
15.7	0.00837856495180774\\
15.71	0.00837856495180774\\
15.72	0.00837856495180774\\
15.73	0.00837856495180774\\
15.74	0.00837856495180774\\
15.75	0.00837856495180774\\
15.76	0.00837856495180774\\
15.77	0.00837856495180774\\
15.78	0.00837856495180774\\
15.79	0.00837856495180774\\
15.8	0.00837856495180774\\
15.81	0.00837856495180774\\
15.82	0.00837856495180774\\
15.83	0.00837856495180774\\
15.84	0.00837856495180774\\
15.85	0.00837856495180774\\
15.86	0.00837856495180774\\
15.87	0.00837856495180774\\
15.88	0.00837856495180774\\
15.89	0.00837856495180774\\
15.9	0.00837856495180774\\
15.91	0.00837856495180774\\
15.92	0.00837856495180774\\
15.93	0.00837856495180774\\
15.94	0.00837856495180774\\
15.95	0.00837856495180774\\
15.96	0.00837856495180774\\
15.97	0.00837856495180774\\
15.98	0.00837856495180774\\
15.99	0.00837856495180774\\
16	0.00837856495180774\\
16.01	0.00837856495180774\\
16.02	0.00837856495180774\\
16.03	0.00837856495180774\\
16.04	0.00837856495180774\\
16.05	0.00837856495180774\\
16.06	0.00837856495180774\\
16.07	0.00837856495180774\\
16.08	0.00837856495180774\\
16.09	0.00837856495180774\\
16.1	0.00837856495180774\\
16.11	0.00837856495180774\\
16.12	0.00837856495180774\\
16.13	0.00837856495180774\\
16.14	0.00837856495180774\\
16.15	0.00837856495180774\\
16.16	0.00837856495180774\\
16.17	0.00837856495180774\\
16.18	0.00837856495180774\\
16.19	0.00837856495180774\\
16.2	0.00837856495180774\\
16.21	0.00837856495180774\\
16.22	0.00837856495180774\\
16.23	0.00837856495180774\\
16.24	0.00837856495180774\\
16.25	0.00837856495180774\\
16.26	0.00837856495180774\\
16.27	0.00837856495180774\\
16.28	0.00837856495180774\\
16.29	0.00837856495180774\\
16.3	0.00837856495180774\\
16.31	0.00837856495180774\\
16.32	0.00837856495180774\\
16.33	0.00837856495180774\\
16.34	0.00837856495180774\\
16.35	0.00837856495180774\\
16.36	0.00837856495180774\\
16.37	0.00837856495180774\\
16.38	0.00837856495180774\\
16.39	0.00837856495180774\\
16.4	0.00837856495180774\\
16.41	0.00837856495180774\\
16.42	0.00837856495180774\\
16.43	0.00837856495180774\\
16.44	0.00837856495180774\\
16.45	0.00837856495180774\\
16.46	0.00837856495180774\\
16.47	0.00837856495180774\\
16.48	0.00837856495180774\\
16.49	0.00837856495180774\\
16.5	0.00837856495180774\\
16.51	0.00837856495180774\\
16.52	0.00837856495180774\\
16.53	0.00837856495180774\\
16.54	0.00837856495180774\\
16.55	0.00837856495180774\\
16.56	0.00837856495180774\\
16.57	0.00837856495180774\\
16.58	0.00837856495180774\\
16.59	0.00837856495180774\\
16.6	0.00837856495180774\\
16.61	0.00837856495180774\\
16.62	0.00837856495180774\\
16.63	0.00837856495180774\\
16.64	0.00837856495180774\\
16.65	0.00837856495180774\\
16.66	0.00837856495180774\\
16.67	0.00837856495180774\\
16.68	0.00837856495180774\\
16.69	0.00837856495180774\\
16.7	0.00837856495180774\\
16.71	0.00837856495180774\\
16.72	0.00837856495180774\\
16.73	0.00837856495180774\\
16.74	0.00837856495180774\\
16.75	0.00837856495180774\\
16.76	0.00837856495180774\\
16.77	0.00837856495180774\\
16.78	0.00837856495180774\\
16.79	0.00837856495180774\\
16.8	0.00837856495180774\\
16.81	0.00837856495180774\\
16.82	0.00837856495180774\\
16.83	0.00837856495180774\\
16.84	0.00837856495180774\\
16.85	0.00837856495180774\\
16.86	0.00837856495180774\\
16.87	0.00837856495180774\\
16.88	0.00837856495180774\\
16.89	0.00837856495180774\\
16.9	0.00837856495180774\\
16.91	0.00837856495180774\\
16.92	0.00837856495180774\\
16.93	0.00837856495180774\\
16.94	0.00837856495180774\\
16.95	0.00837856495180774\\
16.96	0.00837856495180774\\
16.97	0.00837856495180774\\
16.98	0.00837856495180774\\
16.99	0.00837856495180774\\
17	0.00837856495180774\\
17.01	0.00837856495180774\\
17.02	0.00837856495180774\\
17.03	0.00837856495180774\\
17.04	0.00837856495180774\\
17.05	0.00837856495180774\\
17.06	0.00837856495180774\\
17.07	0.00837856495180774\\
17.08	0.00837856495180774\\
17.09	0.00837856495180774\\
17.1	0.00837856495180774\\
17.11	0.00837856495180774\\
17.12	0.00837856495180774\\
17.13	0.00837856495180774\\
17.14	0.00837856495180774\\
17.15	0.00837856495180774\\
17.16	0.00837856495180774\\
17.17	0.00837856495180774\\
17.18	0.00837856495180774\\
17.19	0.00837856495180774\\
17.2	0.00837856495180774\\
17.21	0.00837856495180774\\
17.22	0.00837856495180774\\
17.23	0.00837856495180774\\
17.24	0.00837856495180774\\
17.25	0.00837856495180774\\
17.26	0.00837856495180774\\
17.27	0.00837856495180774\\
17.28	0.00837856495180774\\
17.29	0.00837856495180774\\
17.3	0.00837856495180774\\
17.31	0.00837856495180774\\
17.32	0.00837856495180774\\
17.33	0.00837856495180774\\
17.34	0.00837856495180774\\
17.35	0.00837856495180774\\
17.36	0.00837856495180774\\
17.37	0.00837856495180774\\
17.38	0.00837856495180774\\
17.39	0.00837856495180774\\
17.4	0.00837856495180774\\
17.41	0.00837856495180774\\
17.42	0.00837856495180774\\
17.43	0.00837856495180774\\
17.44	0.00837856495180774\\
17.45	0.00837856495180774\\
17.46	0.00837856495180774\\
17.47	0.00837856495180774\\
17.48	0.00837856495180774\\
17.49	0.00837856495180774\\
17.5	0.00837856495180774\\
17.51	0.00837856495180774\\
17.52	0.00837856495180774\\
17.53	0.00837856495180774\\
17.54	0.00837856495180774\\
17.55	0.00837856495180774\\
17.56	0.00837856495180774\\
17.57	0.00837856495180774\\
17.58	0.00837856495180774\\
17.59	0.00837856495180774\\
17.6	0.00837856495180774\\
17.61	0.00837856495180774\\
17.62	0.00837856495180774\\
17.63	0.00837856495180774\\
17.64	0.00837856495180774\\
17.65	0.00837856495180774\\
17.66	0.00837856495180774\\
17.67	0.00837856495180774\\
17.68	0.00837856495180774\\
17.69	0.00837856495180774\\
17.7	0.00837856495180774\\
17.71	0.00837856495180774\\
17.72	0.00837856495180774\\
17.73	0.00837856495180774\\
17.74	0.00837856495180774\\
17.75	0.00837856495180774\\
17.76	0.00837856495180774\\
17.77	0.00837856495180774\\
17.78	0.00837856495180774\\
17.79	0.00837856495180774\\
17.8	0.00837856495180774\\
17.81	0.00837856495180774\\
17.82	0.00837856495180774\\
17.83	0.00837856495180774\\
17.84	0.00837856495180774\\
17.85	0.00837856495180774\\
17.86	0.00837856495180774\\
17.87	0.00837856495180774\\
17.88	0.00837856495180774\\
17.89	0.00837856495180774\\
17.9	0.00837856495180774\\
17.91	0.00837856495180774\\
17.92	0.00837856495180774\\
17.93	0.00837856495180774\\
17.94	0.00837856495180774\\
17.95	0.00837856495180774\\
17.96	0.00837856495180774\\
17.97	0.00837856495180774\\
17.98	0.00837856495180774\\
17.99	0.00837856495180774\\
18	0.00837856495180774\\
18.01	0.00837856495180774\\
18.02	0.00837856495180774\\
18.03	0.00837856495180774\\
18.04	0.00837856495180774\\
18.05	0.00837856495180774\\
18.06	0.00837856495180774\\
18.07	0.00837856495180774\\
18.08	0.00837856495180774\\
18.09	0.00837856495180774\\
18.1	0.00837856495180774\\
18.11	0.00837856495180774\\
18.12	0.00837856495180774\\
18.13	0.00837856495180774\\
18.14	0.00837856495180774\\
18.15	0.00837856495180774\\
18.16	0.00837856495180774\\
18.17	0.00837856495180774\\
18.18	0.00837856495180774\\
18.19	0.00837856495180774\\
18.2	0.00837856495180774\\
18.21	0.00837856495180774\\
18.22	0.00837856495180774\\
18.23	0.00837856495180774\\
18.24	0.00837856495180774\\
18.25	0.00837856495180774\\
18.26	0.00837856495180774\\
18.27	0.00837856495180774\\
18.28	0.00837856495180774\\
18.29	0.00837856495180774\\
18.3	0.00837856495180774\\
18.31	0.00837856495180774\\
18.32	0.00837856495180774\\
18.33	0.00837856495180774\\
18.34	0.00837856495180774\\
18.35	0.00837856495180774\\
18.36	0.00837856495180774\\
18.37	0.00837856495180774\\
18.38	0.00837856495180774\\
18.39	0.00837856495180774\\
18.4	0.00837856495180774\\
18.41	0.00837856495180774\\
18.42	0.00837856495180774\\
18.43	0.00837856495180774\\
18.44	0.00837856495180774\\
18.45	0.00837856495180774\\
18.46	0.00837856495180774\\
18.47	0.00837856495180774\\
18.48	0.00837856495180774\\
18.49	0.00837856495180774\\
18.5	0.00837856495180774\\
18.51	0.00837856495180774\\
18.52	0.00837856495180774\\
18.53	0.00837856495180774\\
18.54	0.00837856495180774\\
18.55	0.00837856495180774\\
18.56	0.00837856495180774\\
18.57	0.00837856495180774\\
18.58	0.00837856495180774\\
18.59	0.00837856495180774\\
18.6	0.00837856495180774\\
18.61	0.00837856495180774\\
18.62	0.00837856495180774\\
18.63	0.00837856495180774\\
18.64	0.00837856495180774\\
18.65	0.00837856495180774\\
18.66	0.00837856495180774\\
18.67	0.00837856495180774\\
18.68	0.00837856495180774\\
18.69	0.00837856495180774\\
18.7	0.00837856495180774\\
18.71	0.00837856495180774\\
18.72	0.00837856495180774\\
18.73	0.00837856495180774\\
18.74	0.00837856495180774\\
18.75	0.00837856495180774\\
18.76	0.00837856495180774\\
18.77	0.00837856495180774\\
18.78	0.00837856495180774\\
18.79	0.00837856495180774\\
18.8	0.00837856495180774\\
18.81	0.00837856495180774\\
18.82	0.00837856495180774\\
18.83	0.00837856495180774\\
18.84	0.00837856495180774\\
18.85	0.00837856495180774\\
18.86	0.00837856495180774\\
18.87	0.00837856495180774\\
18.88	0.00837856495180774\\
18.89	0.00837856495180774\\
18.9	0.00837856495180774\\
18.91	0.00837856495180774\\
18.92	0.00837856495180774\\
18.93	0.00837856495180774\\
18.94	0.00837856495180774\\
18.95	0.00837856495180774\\
18.96	0.00837856495180774\\
18.97	0.00837856495180774\\
18.98	0.00837856495180774\\
18.99	0.00837856495180774\\
19	0.00837856495180774\\
19.01	0.00837856495180774\\
19.02	0.00837856495180774\\
19.03	0.00837856495180774\\
19.04	0.00837856495180774\\
19.05	0.00837856495180774\\
19.06	0.00837856495180774\\
19.07	0.00837856495180774\\
19.08	0.00837856495180774\\
19.09	0.00837856495180774\\
19.1	0.00837856495180774\\
19.11	0.00837856495180774\\
19.12	0.00837856495180774\\
19.13	0.00837856495180774\\
19.14	0.00837856495180774\\
19.15	0.00837856495180774\\
19.16	0.00837856495180774\\
19.17	0.00837856495180774\\
19.18	0.00837856495180774\\
19.19	0.00837856495180774\\
19.2	0.00837856495180774\\
19.21	0.00837856495180774\\
19.22	0.00837856495180774\\
19.23	0.00837856495180774\\
19.24	0.00837856495180774\\
19.25	0.00837856495180774\\
19.26	0.00837856495180774\\
19.27	0.00837856495180774\\
19.28	0.00837856495180774\\
19.29	0.00837856495180774\\
19.3	0.00837856495180774\\
19.31	0.00837856495180774\\
19.32	0.00837856495180774\\
19.33	0.00837856495180774\\
19.34	0.00837856495180774\\
19.35	0.00837856495180774\\
19.36	0.00837856495180774\\
19.37	0.00837856495180774\\
19.38	0.00837856495180774\\
19.39	0.00837856495180774\\
19.4	0.00837856495180774\\
19.41	0.00837856495180774\\
19.42	0.00837856495180774\\
19.43	0.00837856495180774\\
19.44	0.00837856495180774\\
19.45	0.00837856495180774\\
19.46	0.00837856495180774\\
19.47	0.00837856495180774\\
19.48	0.00837856495180774\\
19.49	0.00837856495180774\\
19.5	0.00837856495180774\\
19.51	0.00837856495180774\\
19.52	0.00837856495180774\\
19.53	0.00837856495180774\\
19.54	0.00837856495180774\\
19.55	0.00837856495180774\\
19.56	0.00837856495180774\\
19.57	0.00837856495180774\\
19.58	0.00837856495180774\\
19.59	0.00837856495180774\\
19.6	0.00837856495180774\\
19.61	0.00837856495180774\\
19.62	0.00837856495180774\\
19.63	0.00837856495180774\\
19.64	0.00837856495180774\\
19.65	0.00837856495180774\\
19.66	0.00837856495180774\\
19.67	0.00837856495180774\\
19.68	0.00837856495180774\\
19.69	0.00837856495180774\\
19.7	0.00837856495180774\\
19.71	0.00837856495180774\\
19.72	0.00837856495180774\\
19.73	0.00837856495180774\\
19.74	0.00837856495180774\\
19.75	0.00837856495180774\\
19.76	0.00837856495180774\\
19.77	0.00837856495180774\\
19.78	0.00837856495180774\\
19.79	0.00837856495180774\\
19.8	0.00837856495180774\\
19.81	0.00837856495180774\\
19.82	0.00837856495180774\\
19.83	0.00837856495180774\\
19.84	0.00837856495180774\\
19.85	0.00837856495180774\\
19.86	0.00837856495180774\\
19.87	0.00837856495180774\\
19.88	0.00837856495180774\\
19.89	0.00837856495180774\\
19.9	0.00837856495180774\\
19.91	0.00837856495180774\\
19.92	0.00837856495180774\\
19.93	0.00837856495180774\\
19.94	0.00837856495180774\\
19.95	0.00837856495180774\\
19.96	0.00837856495180774\\
19.97	0.00837856495180774\\
19.98	0.00837856495180774\\
19.99	0.00837856495180774\\
20	0.00837856495180774\\
20.01	0.00837856495180774\\
20.02	0.00837856495180774\\
20.03	0.00837856495180774\\
20.04	0.00837856495180774\\
20.05	0.00837856495180774\\
20.06	0.00837856495180774\\
20.07	0.00837856495180774\\
20.08	0.00837856495180774\\
20.09	0.00837856495180774\\
20.1	0.00837856495180774\\
20.11	0.00837856495180774\\
20.12	0.00837856495180774\\
20.13	0.00837856495180774\\
20.14	0.00837856495180774\\
20.15	0.00837856495180774\\
20.16	0.00837856495180774\\
20.17	0.00837856495180774\\
20.18	0.00837856495180774\\
20.19	0.00837856495180774\\
20.2	0.00837856495180774\\
20.21	0.00837856495180774\\
20.22	0.00837856495180774\\
20.23	0.00837856495180774\\
20.24	0.00837856495180774\\
20.25	0.00837856495180774\\
20.26	0.00837856495180774\\
20.27	0.00837856495180774\\
20.28	0.00837856495180774\\
20.29	0.00837856495180774\\
20.3	0.00837856495180774\\
20.31	0.00837856495180774\\
20.32	0.00837856495180774\\
20.33	0.00837856495180774\\
20.34	0.00837856495180774\\
20.35	0.00837856495180774\\
20.36	0.00837856495180774\\
20.37	0.00837856495180774\\
20.38	0.00837856495180774\\
20.39	0.00837856495180774\\
20.4	0.00837856495180774\\
20.41	0.00837856495180774\\
20.42	0.00837856495180774\\
20.43	0.00837856495180774\\
20.44	0.00837856495180774\\
20.45	0.00837856495180774\\
20.46	0.00837856495180774\\
20.47	0.00837856495180774\\
20.48	0.00837856495180774\\
20.49	0.00837856495180774\\
20.5	0.00837856495180774\\
20.51	0.00837856495180774\\
20.52	0.00837856495180774\\
20.53	0.00837856495180774\\
20.54	0.00837856495180774\\
20.55	0.00837856495180774\\
20.56	0.00837856495180774\\
20.57	0.00837856495180774\\
20.58	0.00837856495180774\\
20.59	0.00837856495180774\\
20.6	0.00837856495180774\\
20.61	0.00837856495180774\\
20.62	0.00837856495180774\\
20.63	0.00837856495180774\\
20.64	0.00837856495180774\\
20.65	0.00837856495180774\\
20.66	0.00837856495180774\\
20.67	0.00837856495180774\\
20.68	0.00837856495180774\\
20.69	0.00837856495180774\\
20.7	0.00837856495180774\\
20.71	0.00837856495180774\\
20.72	0.00837856495180774\\
20.73	0.00837856495180774\\
20.74	0.00837856495180774\\
20.75	0.00837856495180774\\
20.76	0.00837856495180774\\
20.77	0.00837856495180774\\
20.78	0.00837856495180774\\
20.79	0.00837856495180774\\
20.8	0.00837856495180774\\
20.81	0.00837856495180774\\
20.82	0.00837856495180774\\
20.83	0.00837856495180774\\
20.84	0.00837856495180774\\
20.85	0.00837856495180774\\
20.86	0.00837856495180774\\
20.87	0.00837856495180774\\
20.88	0.00837856495180774\\
20.89	0.00837856495180774\\
20.9	0.00837856495180774\\
20.91	0.00837856495180774\\
20.92	0.00837856495180774\\
20.93	0.00837856495180774\\
20.94	0.00837856495180774\\
20.95	0.00837856495180774\\
20.96	0.00837856495180774\\
20.97	0.00837856495180774\\
20.98	0.00837856495180774\\
20.99	0.00837856495180774\\
21	0.00837856495180774\\
21.01	0.00837856495180774\\
21.02	0.00837856495180774\\
21.03	0.00837856495180774\\
21.04	0.00837856495180774\\
21.05	0.00837856495180774\\
21.06	0.00837856495180774\\
21.07	0.00837856495180774\\
21.08	0.00837856495180774\\
21.09	0.00837856495180774\\
21.1	0.00837856495180774\\
21.11	0.00837856495180774\\
21.12	0.00837856495180774\\
21.13	0.00837856495180774\\
21.14	0.00837856495180774\\
21.15	0.00837856495180774\\
21.16	0.00837856495180774\\
21.17	0.00837856495180774\\
21.18	0.00837856495180774\\
21.19	0.00837856495180774\\
21.2	0.00837856495180774\\
21.21	0.00837856495180774\\
21.22	0.00837856495180774\\
21.23	0.00837856495180774\\
21.24	0.00837856495180774\\
21.25	0.00837856495180774\\
21.26	0.00837856495180774\\
21.27	0.00837856495180774\\
21.28	0.00837856495180774\\
21.29	0.00837856495180774\\
21.3	0.00837856495180774\\
21.31	0.00837856495180774\\
21.32	0.00837856495180774\\
21.33	0.00837856495180774\\
21.34	0.00837856495180774\\
21.35	0.00837856495180774\\
21.36	0.00837856495180774\\
21.37	0.00837856495180774\\
21.38	0.00837856495180774\\
21.39	0.00837856495180774\\
21.4	0.00837856495180774\\
21.41	0.00837856495180774\\
21.42	0.00837856495180774\\
21.43	0.00837856495180774\\
21.44	0.00837856495180774\\
21.45	0.00837856495180774\\
21.46	0.00837856495180774\\
21.47	0.00837856495180774\\
21.48	0.00837856495180774\\
21.49	0.00837856495180774\\
21.5	0.00837856495180774\\
21.51	0.00837856495180774\\
21.52	0.00837856495180774\\
21.53	0.00837856495180774\\
21.54	0.00837856495180774\\
21.55	0.00837856495180774\\
21.56	0.00837856495180774\\
21.57	0.00837856495180774\\
21.58	0.00837856495180774\\
21.59	0.00837856495180774\\
21.6	0.00837856495180774\\
21.61	0.00837856495180774\\
21.62	0.00837856495180774\\
21.63	0.00837856495180774\\
21.64	0.00837856495180774\\
21.65	0.00837856495180774\\
21.66	0.00837856495180774\\
21.67	0.00837856495180774\\
21.68	0.00837856495180774\\
21.69	0.00837856495180774\\
21.7	0.00837856495180774\\
21.71	0.00837856495180774\\
21.72	0.00837856495180774\\
21.73	0.00837856495180774\\
21.74	0.00837856495180774\\
21.75	0.00837856495180774\\
21.76	0.00837856495180774\\
21.77	0.00837856495180774\\
21.78	0.00837856495180774\\
21.79	0.00837856495180774\\
21.8	0.00837856495180774\\
21.81	0.00837856495180774\\
21.82	0.00837856495180774\\
21.83	0.00837856495180774\\
21.84	0.00837856495180774\\
21.85	0.00837856495180774\\
21.86	0.00837856495180774\\
21.87	0.00837856495180774\\
21.88	0.00837856495180774\\
21.89	0.00837856495180774\\
21.9	0.00837856495180774\\
21.91	0.00837856495180774\\
21.92	0.00837856495180774\\
21.93	0.00837856495180774\\
21.94	0.00837856495180774\\
21.95	0.00837856495180774\\
21.96	0.00837856495180774\\
21.97	0.00837856495180774\\
21.98	0.00837856495180774\\
21.99	0.00837856495180774\\
22	0.00837856495180774\\
22.01	0.00837856495180774\\
22.02	0.00837856495180774\\
22.03	0.00837856495180774\\
22.04	0.00837856495180774\\
22.05	0.00837856495180774\\
22.06	0.00837856495180774\\
22.07	0.00837856495180774\\
22.08	0.00837856495180774\\
22.09	0.00837856495180774\\
22.1	0.00837856495180774\\
22.11	0.00837856495180774\\
22.12	0.00837856495180774\\
22.13	0.00837856495180774\\
22.14	0.00837856495180774\\
22.15	0.00837856495180774\\
22.16	0.00837856495180774\\
22.17	0.00837856495180774\\
22.18	0.00837856495180774\\
22.19	0.00837856495180774\\
22.2	0.00837856495180774\\
22.21	0.00837856495180774\\
22.22	0.00837856495180774\\
22.23	0.00837856495180774\\
22.24	0.00837856495180774\\
22.25	0.00837856495180774\\
22.26	0.00837856495180774\\
22.27	0.00837856495180774\\
22.28	0.00837856495180774\\
22.29	0.00837856495180774\\
22.3	0.00837856495180774\\
22.31	0.00837856495180774\\
22.32	0.00837856495180774\\
22.33	0.00837856495180774\\
22.34	0.00837856495180774\\
22.35	0.00837856495180774\\
22.36	0.00837856495180774\\
22.37	0.00837856495180774\\
22.38	0.00837856495180774\\
22.39	0.00837856495180774\\
22.4	0.00837856495180774\\
22.41	0.00837856495180774\\
22.42	0.00837856495180774\\
22.43	0.00837856495180774\\
22.44	0.00837856495180774\\
22.45	0.00837856495180774\\
22.46	0.00837856495180774\\
22.47	0.00837856495180774\\
22.48	0.00837856495180774\\
22.49	0.00837856495180774\\
22.5	0.00837856495180774\\
22.51	0.00837856495180774\\
22.52	0.00837856495180774\\
22.53	0.00837856495180774\\
22.54	0.00837856495180774\\
22.55	0.00837856495180774\\
22.56	0.00837856495180774\\
22.57	0.00837856495180774\\
22.58	0.00837856495180774\\
22.59	0.00837856495180774\\
22.6	0.00837856495180774\\
22.61	0.00837856495180774\\
22.62	0.00837856495180774\\
22.63	0.00837856495180774\\
22.64	0.00837856495180774\\
22.65	0.00837856495180774\\
22.66	0.00837856495180774\\
22.67	0.00837856495180774\\
22.68	0.00837856495180774\\
22.69	0.00837856495180774\\
22.7	0.00837856495180774\\
22.71	0.00837856495180774\\
22.72	0.00837856495180774\\
22.73	0.00837856495180774\\
22.74	0.00837856495180774\\
22.75	0.00837856495180774\\
22.76	0.00837856495180774\\
22.77	0.00837856495180774\\
22.78	0.00837856495180774\\
22.79	0.00837856495180774\\
22.8	0.00837856495180774\\
22.81	0.00837856495180774\\
22.82	0.00837856495180774\\
22.83	0.00837856495180774\\
22.84	0.00837856495180774\\
22.85	0.00837856495180774\\
22.86	0.00837856495180774\\
22.87	0.00837856495180774\\
22.88	0.00837856495180774\\
22.89	0.00837856495180774\\
22.9	0.00837856495180774\\
22.91	0.00837856495180774\\
22.92	0.00837856495180774\\
22.93	0.00837856495180774\\
22.94	0.00837856495180774\\
22.95	0.00837856495180774\\
22.96	0.00837856495180774\\
22.97	0.00837856495180774\\
22.98	0.00837856495180774\\
22.99	0.00837856495180774\\
23	0.00837856495180774\\
23.01	0.00837856495180774\\
23.02	0.00837856495180774\\
23.03	0.00837856495180774\\
23.04	0.00837856495180774\\
23.05	0.00837856495180774\\
23.06	0.00837856495180774\\
23.07	0.00837856495180774\\
23.08	0.00837856495180774\\
23.09	0.00837856495180774\\
23.1	0.00837856495180774\\
23.11	0.00837856495180774\\
23.12	0.00837856495180774\\
23.13	0.00837856495180774\\
23.14	0.00837856495180774\\
23.15	0.00837856495180774\\
23.16	0.00837856495180774\\
23.17	0.00837856495180774\\
23.18	0.00837856495180774\\
23.19	0.00837856495180774\\
23.2	0.00837856495180774\\
23.21	0.00837856495180774\\
23.22	0.00837856495180774\\
23.23	0.00837856495180774\\
23.24	0.00837856495180774\\
23.25	0.00837856495180774\\
23.26	0.00837856495180774\\
23.27	0.00837856495180774\\
23.28	0.00837856495180774\\
23.29	0.00837856495180774\\
23.3	0.00837856495180774\\
23.31	0.00837856495180774\\
23.32	0.00837856495180774\\
23.33	0.00837856495180774\\
23.34	0.00837856495180774\\
23.35	0.00837856495180774\\
23.36	0.00837856495180774\\
23.37	0.00837856495180774\\
23.38	0.00837856495180774\\
23.39	0.00837856495180774\\
23.4	0.00837856495180774\\
23.41	0.00837856495180774\\
23.42	0.00837856495180774\\
23.43	0.00837856495180774\\
23.44	0.00837856495180774\\
23.45	0.00837856495180774\\
23.46	0.00837856495180774\\
23.47	0.00837856495180774\\
23.48	0.00837856495180774\\
23.49	0.00837856495180774\\
23.5	0.00837856495180774\\
23.51	0.00837856495180774\\
23.52	0.00837856495180774\\
23.53	0.00837856495180774\\
23.54	0.00837856495180774\\
23.55	0.00837856495180774\\
23.56	0.00837856495180774\\
23.57	0.00837856495180774\\
23.58	0.00837856495180774\\
23.59	0.00837856495180774\\
23.6	0.00837856495180774\\
23.61	0.00837856495180774\\
23.62	0.00837856495180774\\
23.63	0.00837856495180774\\
23.64	0.00837856495180774\\
23.65	0.00837856495180774\\
23.66	0.00837856495180774\\
23.67	0.00837856495180774\\
23.68	0.00837856495180774\\
23.69	0.00837856495180774\\
23.7	0.00837856495180774\\
23.71	0.00837856495180774\\
23.72	0.00837856495180774\\
23.73	0.00837856495180774\\
23.74	0.00837856495180774\\
23.75	0.00837856495180774\\
23.76	0.00837856495180774\\
23.77	0.00837856495180774\\
23.78	0.00837856495180774\\
23.79	0.00837856495180774\\
23.8	0.00837856495180774\\
23.81	0.00837856495180774\\
23.82	0.00837856495180774\\
23.83	0.00837856495180774\\
23.84	0.00837856495180774\\
23.85	0.00837856495180774\\
23.86	0.00837856495180774\\
23.87	0.00837856495180774\\
23.88	0.00837856495180774\\
23.89	0.00837856495180774\\
23.9	0.00837856495180774\\
23.91	0.00837856495180774\\
23.92	0.00837856495180774\\
23.93	0.00837856495180774\\
23.94	0.00837856495180774\\
23.95	0.00837856495180774\\
23.96	0.00837856495180774\\
23.97	0.00837856495180774\\
23.98	0.00837856495180774\\
23.99	0.00837856495180774\\
24	0.00837856495180774\\
24.01	0.00837856495180774\\
24.02	0.00837856495180774\\
24.03	0.00837856495180774\\
24.04	0.00837856495180774\\
24.05	0.00837856495180774\\
24.06	0.00837856495180774\\
24.07	0.00837856495180774\\
24.08	0.00837856495180774\\
24.09	0.00837856495180774\\
24.1	0.00837856495180774\\
24.11	0.00837856495180774\\
24.12	0.00837856495180774\\
24.13	0.00837856495180774\\
24.14	0.00837856495180774\\
24.15	0.00837856495180774\\
24.16	0.00837856495180774\\
24.17	0.00837856495180774\\
24.18	0.00837856495180774\\
24.19	0.00837856495180774\\
24.2	0.00837856495180774\\
24.21	0.00837856495180774\\
24.22	0.00837856495180774\\
24.23	0.00837856495180774\\
24.24	0.00837856495180774\\
24.25	0.00837856495180774\\
24.26	0.00837856495180774\\
24.27	0.00837856495180774\\
24.28	0.00837856495180774\\
24.29	0.00837856495180774\\
24.3	0.00837856495180774\\
24.31	0.00837856495180774\\
24.32	0.00837856495180774\\
24.33	0.00837856495180774\\
24.34	0.00837856495180774\\
24.35	0.00837856495180774\\
24.36	0.00837856495180774\\
24.37	0.00837856495180774\\
24.38	0.00837856495180774\\
24.39	0.00837856495180774\\
24.4	0.00837856495180774\\
24.41	0.00837856495180774\\
24.42	0.00837856495180774\\
24.43	0.00837856495180774\\
24.44	0.00837856495180774\\
24.45	0.00837856495180774\\
24.46	0.00837856495180774\\
24.47	0.00837856495180774\\
24.48	0.00837856495180774\\
24.49	0.00837856495180774\\
24.5	0.00837856495180774\\
24.51	0.00837856495180774\\
24.52	0.00837856495180774\\
24.53	0.00837856495180774\\
24.54	0.00837856495180774\\
24.55	0.00837856495180774\\
24.56	0.00837856495180774\\
24.57	0.00837856495180774\\
24.58	0.00837856495180774\\
24.59	0.00837856495180774\\
24.6	0.00837856495180774\\
24.61	0.00837856495180774\\
24.62	0.00837856495180774\\
24.63	0.00837856495180774\\
24.64	0.00837856495180774\\
24.65	0.00837856495180774\\
24.66	0.00837856495180774\\
24.67	0.00837856495180774\\
24.68	0.00837856495180774\\
24.69	0.00837856495180774\\
24.7	0.00837856495180774\\
24.71	0.00837856495180774\\
24.72	0.00837856495180774\\
24.73	0.00837856495180774\\
24.74	0.00837856495180774\\
24.75	0.00837856495180774\\
24.76	0.00837856495180774\\
24.77	0.00837856495180774\\
24.78	0.00837856495180774\\
24.79	0.00837856495180774\\
24.8	0.00837856495180774\\
24.81	0.00837856495180774\\
24.82	0.00837856495180774\\
24.83	0.00837856495180774\\
24.84	0.00837856495180774\\
24.85	0.00837856495180774\\
24.86	0.00837856495180774\\
24.87	0.00837856495180774\\
24.88	0.00837856495180774\\
24.89	0.00837856495180774\\
24.9	0.00837856495180774\\
24.91	0.00837856495180774\\
24.92	0.00837856495180774\\
24.93	0.00837856495180774\\
24.94	0.00837856495180774\\
24.95	0.00837856495180774\\
24.96	0.00837856495180774\\
24.97	0.00837856495180774\\
24.98	0.00837856495180774\\
24.99	0.00837856495180774\\
25	0.00837856495180774\\
25.01	0.00837856495180774\\
25.02	0.00837856495180774\\
25.03	0.00837856495180774\\
25.04	0.00837856495180774\\
25.05	0.00837856495180774\\
25.06	0.00837856495180774\\
25.07	0.00837856495180774\\
25.08	0.00837856495180774\\
25.09	0.00837856495180774\\
25.1	0.00837856495180774\\
25.11	0.00837856495180774\\
25.12	0.00837856495180774\\
25.13	0.00837856495180774\\
25.14	0.00837856495180774\\
25.15	0.00837856495180774\\
25.16	0.00837856495180774\\
25.17	0.00837856495180774\\
25.18	0.00837856495180774\\
25.19	0.00837856495180774\\
25.2	0.00837856495180774\\
25.21	0.00837856495180774\\
25.22	0.00837856495180774\\
25.23	0.00837856495180774\\
25.24	0.00837856495180774\\
25.25	0.00837856495180774\\
25.26	0.00837856495180774\\
25.27	0.00837856495180774\\
25.28	0.00837856495180774\\
25.29	0.00837856495180774\\
25.3	0.00837856495180774\\
25.31	0.00837856495180774\\
25.32	0.00837856495180774\\
25.33	0.00837856495180774\\
25.34	0.00837856495180774\\
25.35	0.00837856495180774\\
25.36	0.00837856495180774\\
25.37	0.00837856495180774\\
25.38	0.00837856495180774\\
25.39	0.00837856495180774\\
25.4	0.00837856495180774\\
25.41	0.00837856495180774\\
25.42	0.00837856495180774\\
25.43	0.00837856495180774\\
25.44	0.00837856495180774\\
25.45	0.00837856495180774\\
25.46	0.00837856495180774\\
25.47	0.00837856495180774\\
25.48	0.00837856495180774\\
25.49	0.00837856495180774\\
25.5	0.00837856495180774\\
25.51	0.00837856495180774\\
25.52	0.00837856495180774\\
25.53	0.00837856495180774\\
25.54	0.00837856495180774\\
25.55	0.00837856495180774\\
25.56	0.00837856495180774\\
25.57	0.00837856495180774\\
25.58	0.00837856495180774\\
25.59	0.00837856495180774\\
25.6	0.00837856495180774\\
25.61	0.00837856495180774\\
25.62	0.00837856495180774\\
25.63	0.00837856495180774\\
25.64	0.00837856495180774\\
25.65	0.00837856495180774\\
25.66	0.00837856495180774\\
25.67	0.00837856495180774\\
25.68	0.00837856495180774\\
25.69	0.00837856495180774\\
25.7	0.00837856495180774\\
25.71	0.00837856495180774\\
25.72	0.00837856495180774\\
25.73	0.00837856495180774\\
25.74	0.00837856495180774\\
25.75	0.00837856495180774\\
25.76	0.00837856495180774\\
25.77	0.00837856495180774\\
25.78	0.00837856495180774\\
25.79	0.00837856495180774\\
25.8	0.00837856495180774\\
25.81	0.00837856495180774\\
25.82	0.00837856495180774\\
25.83	0.00837856495180774\\
25.84	0.00837856495180774\\
25.85	0.00837856495180774\\
25.86	0.00837856495180774\\
25.87	0.00837856495180774\\
25.88	0.00837856495180774\\
25.89	0.00837856495180774\\
25.9	0.00837856495180774\\
25.91	0.00837856495180774\\
25.92	0.00837856495180774\\
25.93	0.00837856495180774\\
25.94	0.00837856495180774\\
25.95	0.00837856495180774\\
25.96	0.00837856495180774\\
25.97	0.00837856495180774\\
25.98	0.00837856495180774\\
25.99	0.00837856495180774\\
26	0.00837856495180774\\
26.01	0.00837856495180774\\
26.02	0.00837856495180774\\
26.03	0.00837856495180774\\
26.04	0.00837856495180774\\
26.05	0.00837856495180774\\
26.06	0.00837856495180774\\
26.07	0.00837856495180774\\
26.08	0.00837856495180774\\
26.09	0.00837856495180774\\
26.1	0.00837856495180774\\
26.11	0.00837856495180774\\
26.12	0.00837856495180774\\
26.13	0.00837856495180774\\
26.14	0.00837856495180774\\
26.15	0.00837856495180774\\
26.16	0.00837856495180774\\
26.17	0.00837856495180774\\
26.18	0.00837856495180774\\
26.19	0.00837856495180774\\
26.2	0.00837856495180774\\
26.21	0.00837856495180774\\
26.22	0.00837856495180774\\
26.23	0.00837856495180774\\
26.24	0.00837856495180774\\
26.25	0.00837856495180774\\
26.26	0.00837856495180774\\
26.27	0.00837856495180774\\
26.28	0.00837856495180774\\
26.29	0.00837856495180774\\
26.3	0.00837856495180774\\
26.31	0.00837856495180774\\
26.32	0.00837856495180774\\
26.33	0.00837856495180774\\
26.34	0.00837856495180774\\
26.35	0.00837856495180774\\
26.36	0.00837856495180774\\
26.37	0.00837856495180774\\
26.38	0.00837856495180774\\
26.39	0.00837856495180774\\
26.4	0.00837856495180774\\
26.41	0.00837856495180774\\
26.42	0.00837856495180774\\
26.43	0.00837856495180774\\
26.44	0.00837856495180774\\
26.45	0.00837856495180774\\
26.46	0.00837856495180774\\
26.47	0.00837856495180774\\
26.48	0.00837856495180774\\
26.49	0.00837856495180774\\
26.5	0.00837856495180774\\
26.51	0.00837856495180774\\
26.52	0.00837856495180774\\
26.53	0.00837856495180774\\
26.54	0.00837856495180774\\
26.55	0.00837856495180774\\
26.56	0.00837856495180774\\
26.57	0.00837856495180774\\
26.58	0.00837856495180774\\
26.59	0.00837856495180774\\
26.6	0.00837856495180774\\
26.61	0.00837856495180774\\
26.62	0.00837856495180774\\
26.63	0.00837856495180774\\
26.64	0.00837856495180774\\
26.65	0.00837856495180774\\
26.66	0.00837856495180774\\
26.67	0.00837856495180774\\
26.68	0.00837856495180774\\
26.69	0.00837856495180774\\
26.7	0.00837856495180774\\
26.71	0.00837856495180774\\
26.72	0.00837856495180774\\
26.73	0.00837856495180774\\
26.74	0.00837856495180774\\
26.75	0.00837856495180774\\
26.76	0.00837856495180774\\
26.77	0.00837856495180774\\
26.78	0.00837856495180774\\
26.79	0.00837856495180774\\
26.8	0.00837856495180774\\
26.81	0.00837856495180774\\
26.82	0.00837856495180774\\
26.83	0.00837856495180774\\
26.84	0.00837856495180774\\
26.85	0.00837856495180774\\
26.86	0.00837856495180774\\
26.87	0.00837856495180774\\
26.88	0.00837856495180774\\
26.89	0.00837856495180774\\
26.9	0.00837856495180774\\
26.91	0.00837856495180774\\
26.92	0.00837856495180774\\
26.93	0.00837856495180774\\
26.94	0.00837856495180774\\
26.95	0.00837856495180774\\
26.96	0.00837856495180774\\
26.97	0.00837856495180774\\
26.98	0.00837856495180774\\
26.99	0.00837856495180774\\
27	0.00837856495180774\\
27.01	0.00837856495180774\\
27.02	0.00837856495180774\\
27.03	0.00837856495180774\\
27.04	0.00837856495180774\\
27.05	0.00837856495180774\\
27.06	0.00837856495180774\\
27.07	0.00837856495180774\\
27.08	0.00837856495180774\\
27.09	0.00837856495180774\\
27.1	0.00837856495180774\\
27.11	0.00837856495180774\\
27.12	0.00837856495180774\\
27.13	0.00837856495180774\\
27.14	0.00837856495180774\\
27.15	0.00837856495180774\\
27.16	0.00837856495180774\\
27.17	0.00837856495180774\\
27.18	0.00837856495180774\\
27.19	0.00837856495180774\\
27.2	0.00837856495180774\\
27.21	0.00837856495180774\\
27.22	0.00837856495180774\\
27.23	0.00837856495180774\\
27.24	0.00837856495180774\\
27.25	0.00837856495180774\\
27.26	0.00837856495180774\\
27.27	0.00837856495180774\\
27.28	0.00837856495180774\\
27.29	0.00837856495180774\\
27.3	0.00837856495180774\\
27.31	0.00837856495180774\\
27.32	0.00837856495180774\\
27.33	0.00837856495180774\\
27.34	0.00837856495180774\\
27.35	0.00837856495180774\\
27.36	0.00837856495180774\\
27.37	0.00837856495180774\\
27.38	0.00837856495180774\\
27.39	0.00837856495180774\\
27.4	0.00837856495180774\\
27.41	0.00837856495180774\\
27.42	0.00837856495180774\\
27.43	0.00837856495180774\\
27.44	0.00837856495180774\\
27.45	0.00837856495180774\\
27.46	0.00837856495180774\\
27.47	0.00837856495180774\\
27.48	0.00837856495180774\\
27.49	0.00837856495180774\\
27.5	0.00837856495180774\\
27.51	0.00837856495180774\\
27.52	0.00837856495180774\\
27.53	0.00837856495180774\\
27.54	0.00837856495180774\\
27.55	0.00837856495180774\\
27.56	0.00837856495180774\\
27.57	0.00837856495180774\\
27.58	0.00837856495180774\\
27.59	0.00837856495180774\\
27.6	0.00837856495180774\\
27.61	0.00837856495180774\\
27.62	0.00837856495180774\\
27.63	0.00837856495180774\\
27.64	0.00837856495180774\\
27.65	0.00837856495180774\\
27.66	0.00837856495180774\\
27.67	0.00837856495180774\\
27.68	0.00837856495180774\\
27.69	0.00837856495180774\\
27.7	0.00837856495180774\\
27.71	0.00837856495180774\\
27.72	0.00837856495180774\\
27.73	0.00837856495180774\\
27.74	0.00837856495180774\\
27.75	0.00837856495180774\\
27.76	0.00837856495180774\\
27.77	0.00837856495180774\\
27.78	0.00837856495180774\\
27.79	0.00837856495180774\\
27.8	0.00837856495180774\\
27.81	0.00837856495180774\\
27.82	0.00837856495180774\\
27.83	0.00837856495180774\\
27.84	0.00837856495180774\\
27.85	0.00837856495180774\\
27.86	0.00837856495180774\\
27.87	0.00837856495180774\\
27.88	0.00837856495180774\\
27.89	0.00837856495180774\\
27.9	0.00837856495180774\\
27.91	0.00837856495180774\\
27.92	0.00837856495180774\\
27.93	0.00837856495180774\\
27.94	0.00837856495180774\\
27.95	0.00837856495180774\\
27.96	0.00837856495180774\\
27.97	0.00837856495180774\\
27.98	0.00837856495180774\\
27.99	0.00837856495180774\\
28	0.00837856495180774\\
28.01	0.00837856495180774\\
28.02	0.00837856495180774\\
28.03	0.00837856495180774\\
28.04	0.00837856495180774\\
28.05	0.00837856495180774\\
28.06	0.00837856495180774\\
28.07	0.00837856495180774\\
28.08	0.00837856495180774\\
28.09	0.00837856495180774\\
28.1	0.00837856495180774\\
28.11	0.00837856495180774\\
28.12	0.00837856495180774\\
28.13	0.00837856495180774\\
28.14	0.00837856495180774\\
28.15	0.00837856495180774\\
28.16	0.00837856495180774\\
28.17	0.00837856495180774\\
28.18	0.00837856495180774\\
28.19	0.00837856495180774\\
28.2	0.00837856495180774\\
28.21	0.00837856495180774\\
28.22	0.00837856495180774\\
28.23	0.00837856495180774\\
28.24	0.00837856495180774\\
28.25	0.00837856495180774\\
28.26	0.00837856495180774\\
28.27	0.00837856495180774\\
28.28	0.00837856495180774\\
28.29	0.00837856495180774\\
28.3	0.00837856495180774\\
28.31	0.00837856495180774\\
28.32	0.00837856495180774\\
28.33	0.00837856495180774\\
28.34	0.00837856495180774\\
28.35	0.00837856495180774\\
28.36	0.00837856495180774\\
28.37	0.00837856495180774\\
28.38	0.00837856495180774\\
28.39	0.00837856495180774\\
28.4	0.00837856495180774\\
28.41	0.00837856495180774\\
28.42	0.00837856495180774\\
28.43	0.00837856495180774\\
28.44	0.00837856495180774\\
28.45	0.00837856495180774\\
28.46	0.00837856495180774\\
28.47	0.00837856495180774\\
28.48	0.00837856495180774\\
28.49	0.00837856495180774\\
28.5	0.00837856495180774\\
28.51	0.00837856495180774\\
28.52	0.00837856495180774\\
28.53	0.00837856495180774\\
28.54	0.00837856495180774\\
28.55	0.00837856495180774\\
28.56	0.00837856495180774\\
28.57	0.00837856495180774\\
28.58	0.00837856495180774\\
28.59	0.00837856495180774\\
28.6	0.00837856495180774\\
28.61	0.00837856495180774\\
28.62	0.00837856495180774\\
28.63	0.00837856495180774\\
28.64	0.00837856495180774\\
28.65	0.00837856495180774\\
28.66	0.00837856495180774\\
28.67	0.00837856495180774\\
28.68	0.00837856495180774\\
28.69	0.00837856495180774\\
28.7	0.00837856495180774\\
28.71	0.00837856495180774\\
28.72	0.00837856495180774\\
28.73	0.00837856495180774\\
28.74	0.00837856495180774\\
28.75	0.00837856495180774\\
28.76	0.00837856495180774\\
28.77	0.00837856495180774\\
28.78	0.00837856495180774\\
28.79	0.00837856495180774\\
28.8	0.00837856495180774\\
28.81	0.00837856495180774\\
28.82	0.00837856495180774\\
28.83	0.00837856495180774\\
28.84	0.00837856495180774\\
28.85	0.00837856495180774\\
28.86	0.00837856495180774\\
28.87	0.00837856495180774\\
28.88	0.00837856495180774\\
28.89	0.00837856495180774\\
28.9	0.00837856495180774\\
28.91	0.00837856495180774\\
28.92	0.00837856495180774\\
28.93	0.00837856495180774\\
28.94	0.00837856495180774\\
28.95	0.00837856495180774\\
28.96	0.00837856495180774\\
28.97	0.00837856495180774\\
28.98	0.00837856495180774\\
28.99	0.00837856495180774\\
29	0.00837856495180774\\
29.01	0.00837856495180774\\
29.02	0.00837856495180774\\
29.03	0.00837856495180774\\
29.04	0.00837856495180774\\
29.05	0.00837856495180774\\
29.06	0.00837856495180774\\
29.07	0.00837856495180774\\
29.08	0.00837856495180774\\
29.09	0.00837856495180774\\
29.1	0.00837856495180774\\
29.11	0.00837856495180774\\
29.12	0.00837856495180774\\
29.13	0.00837856495180774\\
29.14	0.00837856495180774\\
29.15	0.00837856495180774\\
29.16	0.00837856495180774\\
29.17	0.00837856495180774\\
29.18	0.00837856495180774\\
29.19	0.00837856495180774\\
29.2	0.00837856495180774\\
29.21	0.00837856495180774\\
29.22	0.00837856495180774\\
29.23	0.00837856495180774\\
29.24	0.00837856495180774\\
29.25	0.00837856495180774\\
29.26	0.00837856495180774\\
29.27	0.00837856495180774\\
29.28	0.00837856495180774\\
29.29	0.00837856495180774\\
29.3	0.00837856495180774\\
29.31	0.00837856495180774\\
29.32	0.00837856495180774\\
29.33	0.00837856495180774\\
29.34	0.00837856495180774\\
29.35	0.00837856495180774\\
29.36	0.00837856495180774\\
29.37	0.00837856495180774\\
29.38	0.00837856495180774\\
29.39	0.00837856495180774\\
29.4	0.00837856495180774\\
29.41	0.00837856495180774\\
29.42	0.00837856495180774\\
29.43	0.00837856495180774\\
29.44	0.00837856495180774\\
29.45	0.00837856495180774\\
29.46	0.00837856495180774\\
29.47	0.00837856495180774\\
29.48	0.00837856495180774\\
29.49	0.00837856495180774\\
29.5	0.00837856495180774\\
29.51	0.00837856495180774\\
29.52	0.00837856495180774\\
29.53	0.00837856495180774\\
29.54	0.00837856495180774\\
29.55	0.00837856495180774\\
29.56	0.00837856495180774\\
29.57	0.00837856495180774\\
29.58	0.00837856495180774\\
29.59	0.00837856495180774\\
29.6	0.00837856495180774\\
29.61	0.00837856495180774\\
29.62	0.00837856495180774\\
29.63	0.00837856495180774\\
29.64	0.00837856495180774\\
29.65	0.00837856495180774\\
29.66	0.00837856495180774\\
29.67	0.00837856495180774\\
29.68	0.00837856495180774\\
29.69	0.00837856495180774\\
29.7	0.00837856495180774\\
29.71	0.00837856495180774\\
29.72	0.00837856495180774\\
29.73	0.00837856495180774\\
29.74	0.00837856495180774\\
29.75	0.00837856495180774\\
29.76	0.00837856495180774\\
29.77	0.00837856495180774\\
29.78	0.00837856495180774\\
29.79	0.00837856495180774\\
29.8	0.00837856495180774\\
29.81	0.00837856495180774\\
29.82	0.00837856495180774\\
29.83	0.00837856495180774\\
29.84	0.00837856495180774\\
29.85	0.00837856495180774\\
29.86	0.00837856495180774\\
29.87	0.00837856495180774\\
29.88	0.00837856495180774\\
29.89	0.00837856495180774\\
29.9	0.00837856495180774\\
29.91	0.00837856495180774\\
29.92	0.00837856495180774\\
29.93	0.00837856495180774\\
29.94	0.00837856495180774\\
29.95	0.00837856495180774\\
29.96	0.00837856495180774\\
29.97	0.00837856495180774\\
29.98	0.00837856495180774\\
29.99	0.00837856495180774\\
30	0.00837856495180774\\
30.01	0.00837856495180774\\
30.02	0.00837856495180774\\
30.03	0.00837856495180774\\
30.04	0.00837856495180774\\
30.05	0.00837856495180774\\
30.06	0.00837856495180774\\
30.07	0.00837856495180774\\
30.08	0.00837856495180774\\
30.09	0.00837856495180774\\
30.1	0.00837856495180774\\
30.11	0.00837856495180774\\
30.12	0.00837856495180774\\
30.13	0.00837856495180774\\
30.14	0.00837856495180774\\
30.15	0.00837856495180774\\
30.16	0.00837856495180774\\
30.17	0.00837856495180774\\
30.18	0.00837856495180774\\
30.19	0.00837856495180774\\
30.2	0.00837856495180774\\
30.21	0.00837856495180774\\
30.22	0.00837856495180774\\
30.23	0.00837856495180774\\
30.24	0.00837856495180774\\
30.25	0.00837856495180774\\
30.26	0.00837856495180774\\
30.27	0.00837856495180774\\
30.28	0.00837856495180774\\
30.29	0.00837856495180774\\
30.3	0.00837856495180774\\
30.31	0.00837856495180774\\
30.32	0.00837856495180774\\
30.33	0.00837856495180774\\
30.34	0.00837856495180774\\
30.35	0.00837856495180774\\
30.36	0.00837856495180774\\
30.37	0.00837856495180774\\
30.38	0.00837856495180774\\
30.39	0.00837856495180774\\
30.4	0.00837856495180774\\
30.41	0.00837856495180774\\
30.42	0.00837856495180774\\
30.43	0.00837856495180774\\
30.44	0.00837856495180774\\
30.45	0.00837856495180774\\
30.46	0.00837856495180774\\
30.47	0.00837856495180774\\
30.48	0.00837856495180774\\
30.49	0.00837856495180774\\
30.5	0.00837856495180774\\
30.51	0.00837856495180774\\
30.52	0.00837856495180774\\
30.53	0.00837856495180774\\
30.54	0.00837856495180774\\
30.55	0.00837856495180774\\
30.56	0.00837856495180774\\
30.57	0.00837856495180774\\
30.58	0.00837856495180774\\
30.59	0.00837856495180774\\
30.6	0.00837856495180774\\
30.61	0.00837856495180774\\
30.62	0.00837856495180774\\
30.63	0.00837856495180774\\
30.64	0.00837856495180774\\
30.65	0.00837856495180774\\
30.66	0.00837856495180774\\
30.67	0.00837856495180774\\
30.68	0.00837856495180774\\
30.69	0.00837856495180774\\
30.7	0.00837856495180774\\
30.71	0.00837856495180774\\
30.72	0.00837856495180774\\
30.73	0.00837856495180774\\
30.74	0.00837856495180774\\
30.75	0.00837856495180774\\
30.76	0.00837856495180774\\
30.77	0.00837856495180774\\
30.78	0.00837856495180774\\
30.79	0.00837856495180774\\
30.8	0.00837856495180774\\
30.81	0.00837856495180774\\
30.82	0.00837856495180774\\
30.83	0.00837856495180774\\
30.84	0.00837856495180774\\
30.85	0.00837856495180774\\
30.86	0.00837856495180774\\
30.87	0.00837856495180774\\
30.88	0.00837856495180774\\
30.89	0.00837856495180774\\
30.9	0.00837856495180774\\
30.91	0.00837856495180774\\
30.92	0.00837856495180774\\
30.93	0.00837856495180774\\
30.94	0.00837856495180774\\
30.95	0.00837856495180774\\
30.96	0.00837856495180774\\
30.97	0.00837856495180774\\
30.98	0.00837856495180774\\
30.99	0.00837856495180774\\
31	0.00837856495180774\\
31.01	0.00837856495180774\\
31.02	0.00837856495180774\\
31.03	0.00837856495180774\\
31.04	0.00837856495180774\\
31.05	0.00837856495180774\\
31.06	0.00837856495180774\\
31.07	0.00837856495180774\\
31.08	0.00837856495180774\\
31.09	0.00837856495180774\\
31.1	0.00837856495180774\\
31.11	0.00837856495180774\\
31.12	0.00837856495180774\\
31.13	0.00837856495180774\\
31.14	0.00837856495180774\\
31.15	0.00837856495180774\\
31.16	0.00837856495180774\\
31.17	0.00837856495180774\\
31.18	0.00837856495180774\\
31.19	0.00837856495180774\\
31.2	0.00837856495180774\\
31.21	0.00837856495180774\\
31.22	0.00837856495180774\\
31.23	0.00837856495180774\\
31.24	0.00837856495180774\\
31.25	0.00837856495180774\\
31.26	0.00837856495180774\\
31.27	0.00837856495180774\\
31.28	0.00837856495180774\\
31.29	0.00837856495180774\\
31.3	0.00837856495180774\\
31.31	0.00837856495180774\\
31.32	0.00837856495180774\\
31.33	0.00837856495180774\\
31.34	0.00837856495180774\\
31.35	0.00837856495180774\\
31.36	0.00837856495180774\\
31.37	0.00837856495180774\\
31.38	0.00837856495180774\\
31.39	0.00837856495180774\\
31.4	0.00837856495180774\\
31.41	0.00837856495180774\\
31.42	0.00837856495180774\\
31.43	0.00837856495180774\\
31.44	0.00837856495180774\\
31.45	0.00837856495180774\\
31.46	0.00837856495180774\\
31.47	0.00837856495180774\\
31.48	0.00837856495180774\\
31.49	0.00837856495180774\\
31.5	0.00837856495180774\\
31.51	0.00837856495180774\\
31.52	0.00837856495180774\\
31.53	0.00837856495180774\\
31.54	0.00837856495180774\\
31.55	0.00837856495180774\\
31.56	0.00837856495180774\\
31.57	0.00837856495180774\\
31.58	0.00837856495180774\\
31.59	0.00837856495180774\\
31.6	0.00837856495180774\\
31.61	0.00837856495180774\\
31.62	0.00837856495180774\\
31.63	0.00837856495180774\\
31.64	0.00837856495180774\\
31.65	0.00837856495180774\\
31.66	0.00837856495180774\\
31.67	0.00837856495180774\\
31.68	0.00837856495180774\\
31.69	0.00837856495180774\\
31.7	0.00837856495180774\\
31.71	0.00837856495180774\\
31.72	0.00837856495180774\\
31.73	0.00837856495180774\\
31.74	0.00837856495180774\\
31.75	0.00837856495180774\\
31.76	0.00837856495180774\\
31.77	0.00837856495180774\\
31.78	0.00837856495180774\\
31.79	0.00837856495180774\\
31.8	0.00837856495180774\\
31.81	0.00837856495180774\\
31.82	0.00837856495180774\\
31.83	0.00837856495180774\\
31.84	0.00837856495180774\\
31.85	0.00837856495180774\\
31.86	0.00837856495180774\\
31.87	0.00837856495180774\\
31.88	0.00837856495180774\\
31.89	0.00837856495180774\\
31.9	0.00837856495180774\\
31.91	0.00837856495180774\\
31.92	0.00837856495180774\\
31.93	0.00837856495180774\\
31.94	0.00837856495180774\\
31.95	0.00837856495180774\\
31.96	0.00837856495180774\\
31.97	0.00837856495180774\\
31.98	0.00837856495180774\\
31.99	0.00837856495180774\\
32	0.00837856495180774\\
32.01	0.00837856495180774\\
32.02	0.00837856495180774\\
32.03	0.00837856495180774\\
32.04	0.00837856495180774\\
32.05	0.00837856495180774\\
32.06	0.00837856495180774\\
32.07	0.00837856495180774\\
32.08	0.00837856495180774\\
32.09	0.00837856495180774\\
32.1	0.00837856495180774\\
32.11	0.00837856495180774\\
32.12	0.00837856495180774\\
32.13	0.00837856495180774\\
32.14	0.00837856495180774\\
32.15	0.00837856495180774\\
32.16	0.00837856495180774\\
32.17	0.00837856495180774\\
32.18	0.00837856495180774\\
32.19	0.00837856495180774\\
32.2	0.00837856495180774\\
32.21	0.00837856495180774\\
32.22	0.00837856495180774\\
32.23	0.00837856495180774\\
32.24	0.00837856495180774\\
32.25	0.00837856495180774\\
32.26	0.00837856495180774\\
32.27	0.00837856495180774\\
32.28	0.00837856495180774\\
32.29	0.00837856495180774\\
32.3	0.00837856495180774\\
32.31	0.00837856495180774\\
32.32	0.00837856495180774\\
32.33	0.00837856495180774\\
32.34	0.00837856495180774\\
32.35	0.00837856495180774\\
32.36	0.00837856495180774\\
32.37	0.00837856495180774\\
32.38	0.00837856495180774\\
32.39	0.00837856495180774\\
32.4	0.00837856495180774\\
32.41	0.00837856495180774\\
32.42	0.00837856495180774\\
32.43	0.00837856495180774\\
32.44	0.00837856495180774\\
32.45	0.00837856495180774\\
32.46	0.00837856495180774\\
32.47	0.00837856495180774\\
32.48	0.00837856495180774\\
32.49	0.00837856495180774\\
32.5	0.00837856495180774\\
32.51	0.00837856495180774\\
32.52	0.00837856495180774\\
32.53	0.00837856495180774\\
32.54	0.00837856495180774\\
32.55	0.00837856495180774\\
32.56	0.00837856495180774\\
32.57	0.00837856495180774\\
32.58	0.00837856495180774\\
32.59	0.00837856495180774\\
32.6	0.00837856495180774\\
32.61	0.00837856495180774\\
32.62	0.00837856495180774\\
32.63	0.00837856495180774\\
32.64	0.00837856495180774\\
32.65	0.00837856495180774\\
32.66	0.00837856495180774\\
32.67	0.00837856495180774\\
32.68	0.00837856495180774\\
32.69	0.00837856495180774\\
32.7	0.00837856495180774\\
32.71	0.00837856495180774\\
32.72	0.00837856495180774\\
32.73	0.00837856495180774\\
32.74	0.00837856495180774\\
32.75	0.00837856495180774\\
32.76	0.00837856495180774\\
32.77	0.00837856495180774\\
32.78	0.00837856495180774\\
32.79	0.00837856495180774\\
32.8	0.00837856495180774\\
32.81	0.00837856495180774\\
32.82	0.00837856495180774\\
32.83	0.00837856495180774\\
32.84	0.00837856495180774\\
32.85	0.00837856495180774\\
32.86	0.00837856495180774\\
32.87	0.00837856495180774\\
32.88	0.00837856495180774\\
32.89	0.00837856495180774\\
32.9	0.00837856495180774\\
32.91	0.00837856495180774\\
32.92	0.00837856495180774\\
32.93	0.00837856495180774\\
32.94	0.00837856495180774\\
32.95	0.00837856495180774\\
32.96	0.00837856495180774\\
32.97	0.00837856495180774\\
32.98	0.00837856495180774\\
32.99	0.00837856495180774\\
33	0.00837856495180774\\
33.01	0.00837856495180774\\
33.02	0.00837856495180774\\
33.03	0.00837856495180774\\
33.04	0.00837856495180774\\
33.05	0.00837856495180774\\
33.06	0.00837856495180774\\
33.07	0.00837856495180774\\
33.08	0.00837856495180774\\
33.09	0.00837856495180774\\
33.1	0.00837856495180774\\
33.11	0.00837856495180774\\
33.12	0.00837856495180774\\
33.13	0.00837856495180774\\
33.14	0.00837856495180774\\
33.15	0.00837856495180774\\
33.16	0.00837856495180774\\
33.17	0.00837856495180774\\
33.18	0.00837856495180774\\
33.19	0.00837856495180774\\
33.2	0.00837856495180774\\
33.21	0.00837856495180774\\
33.22	0.00837856495180774\\
33.23	0.00837856495180774\\
33.24	0.00837856495180774\\
33.25	0.00837856495180774\\
33.26	0.00837856495180774\\
33.27	0.00837856495180774\\
33.28	0.00837856495180774\\
33.29	0.00837856495180774\\
33.3	0.00837856495180774\\
33.31	0.00837856495180774\\
33.32	0.00837856495180774\\
33.33	0.00837856495180774\\
33.34	0.00837856495180774\\
33.35	0.00837856495180774\\
33.36	0.00837856495180774\\
33.37	0.00837856495180774\\
33.38	0.00837856495180774\\
33.39	0.00837856495180774\\
33.4	0.00837856495180774\\
33.41	0.00837856495180774\\
33.42	0.00837856495180774\\
33.43	0.00837856495180774\\
33.44	0.00837856495180774\\
33.45	0.00837856495180774\\
33.46	0.00837856495180774\\
33.47	0.00837856495180774\\
33.48	0.00837856495180774\\
33.49	0.00837856495180774\\
33.5	0.00837856495180774\\
33.51	0.00837856495180774\\
33.52	0.00837856495180774\\
33.53	0.00837856495180774\\
33.54	0.00837856495180774\\
33.55	0.00837856495180774\\
33.56	0.00837856495180774\\
33.57	0.00837856495180774\\
33.58	0.00837856495180774\\
33.59	0.00837856495180774\\
33.6	0.00837856495180774\\
33.61	0.00837856495180774\\
33.62	0.00837856495180774\\
33.63	0.00837856495180774\\
33.64	0.00837856495180774\\
33.65	0.00837856495180774\\
33.66	0.00837856495180774\\
33.67	0.00837856495180774\\
33.68	0.00837856495180774\\
33.69	0.00837856495180774\\
33.7	0.00837856495180774\\
33.71	0.00837856495180774\\
33.72	0.00837856495180774\\
33.73	0.00837856495180774\\
33.74	0.00837856495180774\\
33.75	0.00837856495180774\\
33.76	0.00837856495180774\\
33.77	0.00837856495180774\\
33.78	0.00837856495180774\\
33.79	0.00837856495180774\\
33.8	0.00837856495180774\\
33.81	0.00837856495180774\\
33.82	0.00837856495180774\\
33.83	0.00837856495180774\\
33.84	0.00837856495180774\\
33.85	0.00837856495180774\\
33.86	0.00837856495180774\\
33.87	0.00837856495180774\\
33.88	0.00837856495180774\\
33.89	0.00837856495180774\\
33.9	0.00837856495180774\\
33.91	0.00837856495180774\\
33.92	0.00837856495180774\\
33.93	0.00837856495180774\\
33.94	0.00837856495180774\\
33.95	0.00837856495180774\\
33.96	0.00837856495180774\\
33.97	0.00837856495180774\\
33.98	0.00837856495180774\\
33.99	0.00837856495180774\\
34	0.00837856495180774\\
34.01	0.00837856495180774\\
34.02	0.00837856495180774\\
34.03	0.00837856495180774\\
34.04	0.00837856495180774\\
34.05	0.00837856495180774\\
34.06	0.00837856495180774\\
34.07	0.00837856495180774\\
34.08	0.00837856495180774\\
34.09	0.00837856495180774\\
34.1	0.00837856495180774\\
34.11	0.00837856495180774\\
34.12	0.00837856495180774\\
34.13	0.00837856495180774\\
34.14	0.00837856495180774\\
34.15	0.00837856495180774\\
34.16	0.00837856495180774\\
34.17	0.00837856495180774\\
34.18	0.00837856495180774\\
34.19	0.00837856495180774\\
34.2	0.00837856495180774\\
34.21	0.00837856495180774\\
34.22	0.00837856495180774\\
34.23	0.00837856495180774\\
34.24	0.00837856495180774\\
34.25	0.00837856495180774\\
34.26	0.00837856495180774\\
34.27	0.00837856495180774\\
34.28	0.00837856495180774\\
34.29	0.00837856495180774\\
34.3	0.00837856495180774\\
34.31	0.00837856495180774\\
34.32	0.00837856495180774\\
34.33	0.00837856495180774\\
34.34	0.00837856495180774\\
34.35	0.00837856495180774\\
34.36	0.00837856495180774\\
34.37	0.00837856495180774\\
34.38	0.00837856495180774\\
34.39	0.00837856495180774\\
34.4	0.00837856495180774\\
34.41	0.00837856495180774\\
34.42	0.00837856495180774\\
34.43	0.00837856495180774\\
34.44	0.00837856495180774\\
34.45	0.00837856495180774\\
34.46	0.00837856495180774\\
34.47	0.00837856495180774\\
34.48	0.00837856495180774\\
34.49	0.00837856495180774\\
34.5	0.00837856495180774\\
34.51	0.00837856495180774\\
34.52	0.00837856495180774\\
34.53	0.00837856495180774\\
34.54	0.00837856495180774\\
34.55	0.00837856495180774\\
34.56	0.00837856495180774\\
34.57	0.00837856495180774\\
34.58	0.00837856495180774\\
34.59	0.00837856495180774\\
34.6	0.00837856495180774\\
34.61	0.00837856495180774\\
34.62	0.00837856495180774\\
34.63	0.00837856495180774\\
34.64	0.00837856495180774\\
34.65	0.00837856495180774\\
34.66	0.00837856495180774\\
34.67	0.00837856495180774\\
34.68	0.00837856495180774\\
34.69	0.00837856495180774\\
34.7	0.00837856495180774\\
34.71	0.00837856495180774\\
34.72	0.00837856495180774\\
34.73	0.00837856495180774\\
34.74	0.00837856495180774\\
34.75	0.00837856495180774\\
34.76	0.00837856495180774\\
34.77	0.00837856495180774\\
34.78	0.00837856495180774\\
34.79	0.00837856495180774\\
34.8	0.00837856495180774\\
34.81	0.00837856495180774\\
34.82	0.00837856495180774\\
34.83	0.00837856495180774\\
34.84	0.00837856495180774\\
34.85	0.00837856495180774\\
34.86	0.00837856495180774\\
34.87	0.00837856495180774\\
34.88	0.00837856495180774\\
34.89	0.00837856495180774\\
34.9	0.00837856495180774\\
34.91	0.00837856495180774\\
34.92	0.00837856495180774\\
34.93	0.00837856495180774\\
34.94	0.00837856495180774\\
34.95	0.00837856495180774\\
34.96	0.00837856495180774\\
34.97	0.00837856495180774\\
34.98	0.00837856495180774\\
34.99	0.00837856495180774\\
35	0.00837856495180774\\
35.01	0.00837856495180774\\
35.02	0.00837856495180774\\
35.03	0.00837856495180774\\
35.04	0.00837856495180774\\
35.05	0.00837856495180774\\
35.06	0.00837856495180774\\
35.07	0.00837856495180774\\
35.08	0.00837856495180774\\
35.09	0.00837856495180774\\
35.1	0.00837856495180774\\
35.11	0.00837856495180774\\
35.12	0.00837856495180774\\
35.13	0.00837856495180774\\
35.14	0.00837856495180774\\
35.15	0.00837856495180774\\
35.16	0.00837856495180774\\
35.17	0.00837856495180774\\
35.18	0.00837856495180774\\
35.19	0.00837856495180774\\
35.2	0.00837856495180774\\
35.21	0.00837856495180774\\
35.22	0.00837856495180774\\
35.23	0.00837856495180774\\
35.24	0.00837856495180774\\
35.25	0.00837856495180774\\
35.26	0.00837856495180774\\
35.27	0.00837856495180774\\
35.28	0.00837856495180774\\
35.29	0.00837856495180774\\
35.3	0.00837856495180774\\
35.31	0.00837856495180774\\
35.32	0.00837856495180774\\
35.33	0.00837856495180774\\
35.34	0.00837856495180774\\
35.35	0.00837856495180774\\
35.36	0.00837856495180774\\
35.37	0.00837856495180774\\
35.38	0.00837856495180774\\
35.39	0.00837856495180774\\
35.4	0.00837856495180774\\
35.41	0.00837856495180774\\
35.42	0.00837856495180774\\
35.43	0.00837856495180774\\
35.44	0.00837856495180774\\
35.45	0.00837856495180774\\
35.46	0.00837856495180774\\
35.47	0.00837856495180774\\
35.48	0.00837856495180774\\
35.49	0.00837856495180774\\
35.5	0.00837856495180774\\
35.51	0.00837856495180774\\
35.52	0.00837856495180774\\
35.53	0.00837856495180774\\
35.54	0.00837856495180774\\
35.55	0.00837856495180774\\
35.56	0.00837856495180774\\
35.57	0.00837856495180774\\
35.58	0.00837856495180774\\
35.59	0.00837856495180774\\
35.6	0.00837856495180774\\
35.61	0.00837856495180774\\
35.62	0.00837856495180774\\
35.63	0.00837856495180774\\
35.64	0.00837856495180774\\
35.65	0.00837856495180774\\
35.66	0.00837856495180774\\
35.67	0.00837856495180774\\
35.68	0.00837856495180774\\
35.69	0.00837856495180774\\
35.7	0.00837856495180774\\
35.71	0.00837856495180774\\
35.72	0.00837856495180774\\
35.73	0.00837856495180774\\
35.74	0.00837856495180774\\
35.75	0.00837856495180774\\
35.76	0.00837856495180774\\
35.77	0.00837856495180774\\
35.78	0.00837856495180774\\
35.79	0.00837856495180774\\
35.8	0.00837856495180774\\
35.81	0.00837856495180774\\
35.82	0.00837856495180774\\
35.83	0.00837856495180774\\
35.84	0.00837856495180774\\
35.85	0.00837856495180774\\
35.86	0.00837856495180774\\
35.87	0.00837856495180774\\
35.88	0.00837856495180774\\
35.89	0.00837856495180774\\
35.9	0.00837856495180774\\
35.91	0.00837856495180774\\
35.92	0.00837856495180774\\
35.93	0.00837856495180774\\
35.94	0.00837856495180774\\
35.95	0.00837856495180774\\
35.96	0.00837856495180774\\
35.97	0.00837856495180774\\
35.98	0.00837856495180774\\
35.99	0.00837856495180774\\
36	0.00837856495180774\\
36.01	0.00837856495180774\\
36.02	0.00837856495180774\\
36.03	0.00837856495180774\\
36.04	0.00837856495180774\\
36.05	0.00837856495180774\\
36.06	0.00837856495180774\\
36.07	0.00837856495180774\\
36.08	0.00837856495180774\\
36.09	0.00837856495180774\\
36.1	0.00837856495180774\\
36.11	0.00837856495180774\\
36.12	0.00837856495180774\\
36.13	0.00837856495180774\\
36.14	0.00837856495180774\\
36.15	0.00837856495180774\\
36.16	0.00837856495180774\\
36.17	0.00837856495180774\\
36.18	0.00837856495180774\\
36.19	0.00837856495180774\\
36.2	0.00837856495180774\\
36.21	0.00837856495180774\\
36.22	0.00837856495180774\\
36.23	0.00837856495180774\\
36.24	0.00837856495180774\\
36.25	0.00837856495180774\\
36.26	0.00837856495180774\\
36.27	0.00837856495180774\\
36.28	0.00837856495180774\\
36.29	0.00837856495180774\\
36.3	0.00837856495180774\\
36.31	0.00837856495180774\\
36.32	0.00837856495180774\\
36.33	0.00837856495180774\\
36.34	0.00837856495180774\\
36.35	0.00837856495180774\\
36.36	0.00837856495180774\\
36.37	0.00837856495180774\\
36.38	0.00837856495180774\\
36.39	0.00837856495180774\\
36.4	0.00837856495180774\\
36.41	0.00837856495180774\\
36.42	0.00837856495180774\\
36.43	0.00837856495180774\\
36.44	0.00837856495180774\\
36.45	0.00837856495180774\\
36.46	0.00837856495180774\\
36.47	0.00837856495180774\\
36.48	0.00837856495180774\\
36.49	0.00837856495180774\\
36.5	0.00837856495180774\\
36.51	0.00837856495180774\\
36.52	0.00837856495180774\\
36.53	0.00837856495180774\\
36.54	0.00837856495180774\\
36.55	0.00837856495180774\\
36.56	0.00837856495180774\\
36.57	0.00837856495180774\\
36.58	0.00837856495180774\\
36.59	0.00837856495180774\\
36.6	0.00837856495180774\\
36.61	0.00837856495180774\\
36.62	0.00837856495180774\\
36.63	0.00837856495180774\\
36.64	0.00837856495180774\\
36.65	0.00837856495180774\\
36.66	0.00837856495180774\\
36.67	0.00837856495180774\\
36.68	0.00837856495180774\\
36.69	0.00837856495180774\\
36.7	0.00837856495180774\\
36.71	0.00837856495180774\\
36.72	0.00837856495180774\\
36.73	0.00837856495180774\\
36.74	0.00837856495180774\\
36.75	0.00837856495180774\\
36.76	0.00837856495180774\\
36.77	0.00837856495180774\\
36.78	0.00837856495180774\\
36.79	0.00837856495180774\\
36.8	0.00837856495180774\\
36.81	0.00837856495180774\\
36.82	0.00837856495180774\\
36.83	0.00837856495180774\\
36.84	0.00837856495180774\\
36.85	0.00837856495180774\\
36.86	0.00837856495180774\\
36.87	0.00837856495180774\\
36.88	0.00837856495180774\\
36.89	0.00837856495180774\\
36.9	0.00837856495180774\\
36.91	0.00837856495180774\\
36.92	0.00837856495180774\\
36.93	0.00837856495180774\\
36.94	0.00837856495180774\\
36.95	0.00837856495180774\\
36.96	0.00837856495180774\\
36.97	0.00837856495180774\\
36.98	0.00837856495180774\\
36.99	0.00837856495180774\\
37	0.00837856495180774\\
37.01	0.00837856495180774\\
37.02	0.00837856495180774\\
37.03	0.00837856495180774\\
37.04	0.00837856495180774\\
37.05	0.00837856495180774\\
37.06	0.00837856495180774\\
37.07	0.00837856495180774\\
37.08	0.00837856495180774\\
37.09	0.00837856495180774\\
37.1	0.00837856495180774\\
37.11	0.00837856495180774\\
37.12	0.00837856495180774\\
37.13	0.00837856495180774\\
37.14	0.00837856495180774\\
37.15	0.00837856495180774\\
37.16	0.00837856495180774\\
37.17	0.00837856495180774\\
37.18	0.00837856495180774\\
37.19	0.00837856495180774\\
37.2	0.00837856495180774\\
37.21	0.00837856495180774\\
37.22	0.00837856495180774\\
37.23	0.00837856495180774\\
37.24	0.00837856495180774\\
37.25	0.00837856495180774\\
37.26	0.00837856495180774\\
37.27	0.00837856495180774\\
37.28	0.00837856495180774\\
37.29	0.00837856495180774\\
37.3	0.00837856495180774\\
37.31	0.00837856495180774\\
37.32	0.00837856495180774\\
37.33	0.00837856495180774\\
37.34	0.00837856495180774\\
37.35	0.00837856495180774\\
37.36	0.00837856495180774\\
37.37	0.00837856495180774\\
37.38	0.00837856495180774\\
37.39	0.00837856495180774\\
37.4	0.00837856495180774\\
37.41	0.00837856495180774\\
37.42	0.00837856495180774\\
37.43	0.00837856495180774\\
37.44	0.00837856495180774\\
37.45	0.00837856495180774\\
37.46	0.00837856495180774\\
37.47	0.00837856495180774\\
37.48	0.00837856495180774\\
37.49	0.00837856495180774\\
37.5	0.00837856495180774\\
37.51	0.00837856495180774\\
37.52	0.00837856495180774\\
37.53	0.00837856495180774\\
37.54	0.00837856495180774\\
37.55	0.00837856495180774\\
37.56	0.00837856495180774\\
37.57	0.00837856495180774\\
37.58	0.00837856495180774\\
37.59	0.00837856495180774\\
37.6	0.00837856495180774\\
37.61	0.00837856495180774\\
37.62	0.00837856495180774\\
37.63	0.00837856495180774\\
37.64	0.00837856495180774\\
37.65	0.00837856495180774\\
37.66	0.00837856495180774\\
37.67	0.00837856495180774\\
37.68	0.00837856495180774\\
37.69	0.00837856495180774\\
37.7	0.00837856495180774\\
37.71	0.00837856495180774\\
37.72	0.00837856495180774\\
37.73	0.00837856495180774\\
37.74	0.00837856495180774\\
37.75	0.00837856495180774\\
37.76	0.00837856495180774\\
37.77	0.00837856495180774\\
37.78	0.00837856495180774\\
37.79	0.00837856495180774\\
37.8	0.00837856495180774\\
37.81	0.00837856495180774\\
37.82	0.00837856495180774\\
37.83	0.00837856495180774\\
37.84	0.00837856495180774\\
37.85	0.00837856495180774\\
37.86	0.00837856495180774\\
37.87	0.00837856495180774\\
37.88	0.00837856495180774\\
37.89	0.00837856495180774\\
37.9	0.00837856495180774\\
37.91	0.00837856495180774\\
37.92	0.00837856495180774\\
37.93	0.00837856495180774\\
37.94	0.00837856495180774\\
37.95	0.00837856495180774\\
37.96	0.00837856495180774\\
37.97	0.00837856495180774\\
37.98	0.00837856495180774\\
37.99	0.00837856495180774\\
38	0.00837856495180774\\
38.01	0.00837856495180774\\
38.02	0.00837856495180774\\
38.03	0.00837856495180774\\
38.04	0.00837856495180774\\
38.05	0.00837856495180774\\
38.06	0.00837856495180774\\
38.07	0.00837856495180774\\
38.08	0.00837856495180774\\
38.09	0.00837856495180774\\
38.1	0.00837856495180774\\
38.11	0.00837856495180774\\
38.12	0.00837856495180774\\
38.13	0.00837856495180774\\
38.14	0.00837856495180774\\
38.15	0.00837856495180774\\
38.16	0.00837856495180774\\
38.17	0.00837856495180774\\
38.18	0.00837856495180774\\
38.19	0.00837856495180774\\
38.2	0.00837856495180774\\
38.21	0.00837856495180774\\
38.22	0.00837856495180774\\
38.23	0.00837856495180774\\
38.24	0.00837856495180774\\
38.25	0.00837856495180774\\
38.26	0.00837856495180774\\
38.27	0.00837856495180774\\
38.28	0.00837856495180774\\
38.29	0.00837856495180774\\
38.3	0.00837856495180774\\
38.31	0.00837856495180774\\
38.32	0.00837856495180774\\
38.33	0.00837856495180774\\
38.34	0.00837856495180774\\
38.35	0.00837856495180774\\
38.36	0.00837856495180774\\
38.37	0.00837856495180774\\
38.38	0.00837856495180774\\
38.39	0.00837856495180774\\
38.4	0.00837856495180774\\
38.41	0.00837856495180774\\
38.42	0.00837856495180774\\
38.43	0.00837856495180774\\
38.44	0.00837856495180774\\
38.45	0.00837856495180774\\
38.46	0.00837856495180774\\
38.47	0.00837856495180774\\
38.48	0.00837856495180774\\
38.49	0.00837856495180774\\
38.5	0.00837856495180774\\
38.51	0.00837856495180774\\
38.52	0.00837856495180774\\
38.53	0.00837856495180774\\
38.54	0.00837856495180774\\
38.55	0.00837856495180774\\
38.56	0.00837856495180774\\
38.57	0.00837856495180774\\
38.58	0.00837856495180774\\
38.59	0.00837856495180774\\
38.6	0.00837856495180774\\
38.61	0.00837856495180774\\
38.62	0.00837856495180774\\
38.63	0.00837856495180774\\
38.64	0.00837856495180774\\
38.65	0.00837856495180774\\
38.66	0.00837856495180774\\
38.67	0.00837856495180774\\
38.68	0.00837856495180774\\
38.69	0.00837856495180775\\
38.7	0.00837856495180775\\
38.71	0.00837856495180775\\
38.72	0.00837856495180775\\
38.73	0.00837856495180775\\
38.74	0.00837856495180775\\
38.75	0.00837856495180775\\
38.76	0.00837856495180775\\
38.77	0.00837856495180775\\
38.78	0.00837856495180775\\
38.79	0.00837856495180775\\
38.8	0.00837856495180775\\
38.81	0.00837856495180775\\
38.82	0.00837856495180775\\
38.83	0.00837856495180775\\
38.84	0.00837856495180775\\
38.85	0.00837856495180775\\
38.86	0.00837856495180775\\
38.87	0.00837856495180775\\
38.88	0.00837856495180775\\
38.89	0.00837856495180775\\
38.9	0.00837856495180775\\
38.91	0.00837856495180775\\
38.92	0.00837856495180775\\
38.93	0.00837856495180775\\
38.94	0.00837856495180775\\
38.95	0.00837856495180775\\
38.96	0.00837856495180775\\
38.97	0.00837856495180775\\
38.98	0.00837856495180775\\
38.99	0.00837856495180775\\
39	0.00837856495180775\\
39.01	0.00837856495180775\\
39.02	0.00837856495180775\\
39.03	0.00837856495180775\\
39.04	0.00837856495180775\\
39.05	0.00837856495180775\\
39.06	0.00837856495180775\\
39.07	0.00837856495180775\\
39.08	0.00837856495180775\\
39.09	0.00837856495180775\\
39.1	0.00837856495180775\\
39.11	0.00837856495180775\\
39.12	0.00837856495180775\\
39.13	0.00837856495180775\\
39.14	0.00837856495180775\\
39.15	0.00837856495180775\\
39.16	0.00837856495180775\\
39.17	0.00837856495180776\\
39.18	0.00837856495180776\\
39.19	0.00837856495180776\\
39.2	0.00837856495180776\\
39.21	0.00837856495180776\\
39.22	0.00837856495180776\\
39.23	0.00837856495180776\\
39.24	0.00837856495180776\\
39.25	0.00837856495180776\\
39.26	0.00837856495180776\\
39.27	0.00837856495180776\\
39.28	0.00837856495180776\\
39.29	0.00837856495180776\\
39.3	0.00837856495180776\\
39.31	0.00837856495180776\\
39.32	0.00837856495180776\\
39.33	0.00837856495180776\\
39.34	0.00837856495180776\\
39.35	0.00837856495180776\\
39.36	0.00837856495180776\\
39.37	0.00837856495180776\\
39.38	0.00837856495180776\\
39.39	0.00837856495180776\\
39.4	0.00837856495180776\\
39.41	0.00837856495180776\\
39.42	0.00837856495180776\\
39.43	0.00837856495180776\\
39.44	0.00837856495180776\\
39.45	0.00837856495180776\\
39.46	0.00837856495180776\\
39.47	0.00837856495180776\\
39.48	0.00837856495180776\\
39.49	0.00837856495180776\\
39.5	0.00837856495180776\\
39.51	0.00837856495180776\\
39.52	0.00837856495180777\\
39.53	0.00837856495180777\\
39.54	0.00837856495180777\\
39.55	0.00837856495180777\\
39.56	0.00837856495180777\\
39.57	0.00837856495180777\\
39.58	0.00837856495180777\\
39.59	0.00837856495180777\\
39.6	0.00837856495180777\\
39.61	0.00837856495180777\\
39.62	0.00837856495180777\\
39.63	0.00837856495180777\\
39.64	0.00837856495180777\\
39.65	0.00837856495180777\\
39.66	0.00837856495180777\\
39.67	0.00837856495180777\\
39.68	0.00837856495180777\\
39.69	0.00837856495180777\\
39.7	0.00837856495180777\\
39.71	0.00837856495180777\\
39.72	0.00837856495180777\\
39.73	0.00837856495180777\\
39.74	0.00837856495180777\\
39.75	0.00837856495180777\\
39.76	0.00837856495180778\\
39.77	0.00837856495180778\\
39.78	0.00837856495180778\\
39.79	0.00837856495180778\\
39.8	0.00837856495180778\\
39.81	0.00837856495180778\\
39.82	0.00837856495180778\\
39.83	0.00837856495180778\\
39.84	0.00837856495180778\\
39.85	0.00837856495180778\\
39.86	0.00837856495180778\\
39.87	0.00837856495180778\\
39.88	0.00837856495180778\\
39.89	0.00837856495180778\\
39.9	0.00837856495180778\\
39.91	0.00837856495180778\\
39.92	0.00837856495180779\\
39.93	0.00837856495180779\\
39.94	0.00837856495180779\\
39.95	0.00837856495180779\\
39.96	0.00837856495180779\\
39.97	0.00837856495180779\\
39.98	0.00837856495180779\\
39.99	0.00837856495180779\\
40	0.00837856495180779\\
40.01	0.00837856495180779\\
};
\addplot [color=black,solid,forget plot]
  table[row sep=crcr]{%
40.01	0.00837856495180779\\
40.02	0.00837856495180779\\
40.03	0.00837856495180779\\
40.04	0.00837856495180779\\
40.05	0.00837856495180779\\
40.06	0.00837856495180779\\
40.07	0.00837856495180779\\
40.08	0.0083785649518078\\
40.09	0.0083785649518078\\
40.1	0.0083785649518078\\
40.11	0.0083785649518078\\
40.12	0.0083785649518078\\
40.13	0.0083785649518078\\
40.14	0.0083785649518078\\
40.15	0.0083785649518078\\
40.16	0.0083785649518078\\
40.17	0.0083785649518078\\
40.18	0.0083785649518078\\
40.19	0.0083785649518078\\
40.2	0.0083785649518078\\
40.21	0.0083785649518078\\
40.22	0.0083785649518078\\
40.23	0.00837856495180781\\
40.24	0.00837856495180781\\
40.25	0.00837856495180781\\
40.26	0.00837856495180781\\
40.27	0.00837856495180781\\
40.28	0.00837856495180781\\
40.29	0.00837856495180781\\
40.3	0.00837856495180781\\
40.31	0.00837856495180781\\
40.32	0.00837856495180781\\
40.33	0.00837856495180781\\
40.34	0.00837856495180781\\
40.35	0.00837856495180782\\
40.36	0.00837856495180782\\
40.37	0.00837856495180782\\
40.38	0.00837856495180782\\
40.39	0.00837856495180782\\
40.4	0.00837856495180782\\
40.41	0.00837856495180782\\
40.42	0.00837856495180782\\
40.43	0.00837856495180782\\
40.44	0.00837856495180783\\
40.45	0.00837856495180783\\
40.46	0.00837856495180783\\
40.47	0.00837856495180783\\
40.48	0.00837856495180783\\
40.49	0.00837856495180783\\
40.5	0.00837856495180783\\
40.51	0.00837856495180783\\
40.52	0.00837856495180783\\
40.53	0.00837856495180784\\
40.54	0.00837856495180784\\
40.55	0.00837856495180784\\
40.56	0.00837856495180784\\
40.57	0.00837856495180784\\
40.58	0.00837856495180784\\
40.59	0.00837856495180784\\
40.6	0.00837856495180784\\
40.61	0.00837856495180784\\
40.62	0.00837856495180784\\
40.63	0.00837856495180785\\
40.64	0.00837856495180785\\
40.65	0.00837856495180785\\
40.66	0.00837856495180785\\
40.67	0.00837856495180785\\
40.68	0.00837856495180785\\
40.69	0.00837856495180785\\
40.7	0.00837856495180785\\
40.71	0.00837856495180786\\
40.72	0.00837856495180786\\
40.73	0.00837856495180786\\
40.74	0.00837856495180786\\
40.75	0.00837856495180786\\
40.76	0.00837856495180786\\
40.77	0.00837856495180786\\
40.78	0.00837856495180787\\
40.79	0.00837856495180787\\
40.8	0.00837856495180787\\
40.81	0.00837856495180787\\
40.82	0.00837856495180787\\
40.83	0.00837856495180787\\
40.84	0.00837856495180788\\
40.85	0.00837856495180788\\
40.86	0.00837856495180788\\
40.87	0.00837856495180788\\
40.88	0.00837856495180788\\
40.89	0.00837856495180788\\
40.9	0.00837856495180788\\
40.91	0.00837856495180789\\
40.92	0.00837856495180789\\
40.93	0.00837856495180789\\
40.94	0.00837856495180789\\
40.95	0.00837856495180789\\
40.96	0.00837856495180789\\
40.97	0.0083785649518079\\
40.98	0.0083785649518079\\
40.99	0.0083785649518079\\
41	0.0083785649518079\\
41.01	0.0083785649518079\\
41.02	0.0083785649518079\\
41.03	0.00837856495180791\\
41.04	0.00837856495180791\\
41.05	0.00837856495180791\\
41.06	0.00837856495180791\\
41.07	0.00837856495180792\\
41.08	0.00837856495180792\\
41.09	0.00837856495180792\\
41.1	0.00837856495180792\\
41.11	0.00837856495180792\\
41.12	0.00837856495180792\\
41.13	0.00837856495180793\\
41.14	0.00837856495180793\\
41.15	0.00837856495180793\\
41.16	0.00837856495180793\\
41.17	0.00837856495180793\\
41.18	0.00837856495180794\\
41.19	0.00837856495180794\\
41.2	0.00837856495180794\\
41.21	0.00837856495180794\\
41.22	0.00837856495180795\\
41.23	0.00837856495180795\\
41.24	0.00837856495180795\\
41.25	0.00837856495180795\\
41.26	0.00837856495180796\\
41.27	0.00837856495180796\\
41.28	0.00837856495180796\\
41.29	0.00837856495180796\\
41.3	0.00837856495180797\\
41.31	0.00837856495180797\\
41.32	0.00837856495180797\\
41.33	0.00837856495180797\\
41.34	0.00837856495180797\\
41.35	0.00837856495180798\\
41.36	0.00837856495180798\\
41.37	0.00837856495180798\\
41.38	0.00837856495180798\\
41.39	0.00837856495180799\\
41.4	0.00837856495180799\\
41.41	0.00837856495180799\\
41.42	0.008378564951808\\
41.43	0.008378564951808\\
41.44	0.008378564951808\\
41.45	0.00837856495180801\\
41.46	0.00837856495180801\\
41.47	0.00837856495180801\\
41.48	0.00837856495180801\\
41.49	0.00837856495180802\\
41.5	0.00837856495180802\\
41.51	0.00837856495180802\\
41.52	0.00837856495180803\\
41.53	0.00837856495180803\\
41.54	0.00837856495180803\\
41.55	0.00837856495180804\\
41.56	0.00837856495180804\\
41.57	0.00837856495180804\\
41.58	0.00837856495180805\\
41.59	0.00837856495180805\\
41.6	0.00837856495180805\\
41.61	0.00837856495180806\\
41.62	0.00837856495180806\\
41.63	0.00837856495180806\\
41.64	0.00837856495180807\\
41.65	0.00837856495180807\\
41.66	0.00837856495180807\\
41.67	0.00837856495180808\\
41.68	0.00837856495180808\\
41.69	0.00837856495180809\\
41.7	0.00837856495180809\\
41.71	0.00837856495180809\\
41.72	0.0083785649518081\\
41.73	0.0083785649518081\\
41.74	0.0083785649518081\\
41.75	0.00837856495180811\\
41.76	0.00837856495180811\\
41.77	0.00837856495180811\\
41.78	0.00837856495180812\\
41.79	0.00837856495180812\\
41.8	0.00837856495180813\\
41.81	0.00837856495180813\\
41.82	0.00837856495180814\\
41.83	0.00837856495180814\\
41.84	0.00837856495180814\\
41.85	0.00837856495180815\\
41.86	0.00837856495180815\\
41.87	0.00837856495180816\\
41.88	0.00837856495180816\\
41.89	0.00837856495180817\\
41.9	0.00837856495180817\\
41.91	0.00837856495180818\\
41.92	0.00837856495180818\\
41.93	0.00837856495180819\\
41.94	0.00837856495180819\\
41.95	0.0083785649518082\\
41.96	0.0083785649518082\\
41.97	0.00837856495180821\\
41.98	0.00837856495180821\\
41.99	0.00837856495180822\\
42	0.00837856495180822\\
42.01	0.00837856495180823\\
42.02	0.00837856495180823\\
42.03	0.00837856495180824\\
42.04	0.00837856495180824\\
42.05	0.00837856495180825\\
42.06	0.00837856495180825\\
42.07	0.00837856495180826\\
42.08	0.00837856495180826\\
42.09	0.00837856495180827\\
42.1	0.00837856495180828\\
42.11	0.00837856495180828\\
42.12	0.00837856495180829\\
42.13	0.00837856495180829\\
42.14	0.0083785649518083\\
42.15	0.00837856495180831\\
42.16	0.00837856495180831\\
42.17	0.00837856495180832\\
42.18	0.00837856495180832\\
42.19	0.00837856495180833\\
42.2	0.00837856495180834\\
42.21	0.00837856495180834\\
42.22	0.00837856495180835\\
42.23	0.00837856495180836\\
42.24	0.00837856495180836\\
42.25	0.00837856495180837\\
42.26	0.00837856495180837\\
42.27	0.00837856495180838\\
42.28	0.00837856495180839\\
42.29	0.0083785649518084\\
42.3	0.0083785649518084\\
42.31	0.00837856495180841\\
42.32	0.00837856495180842\\
42.33	0.00837856495180843\\
42.34	0.00837856495180843\\
42.35	0.00837856495180844\\
42.36	0.00837856495180845\\
42.37	0.00837856495180845\\
42.38	0.00837856495180846\\
42.39	0.00837856495180847\\
42.4	0.00837856495180848\\
42.41	0.00837856495180849\\
42.42	0.00837856495180849\\
42.43	0.0083785649518085\\
42.44	0.00837856495180851\\
42.45	0.00837856495180852\\
42.46	0.00837856495180853\\
42.47	0.00837856495180853\\
42.48	0.00837856495180854\\
42.49	0.00837856495180855\\
42.5	0.00837856495180856\\
42.51	0.00837856495180857\\
42.52	0.00837856495180858\\
42.53	0.00837856495180859\\
42.54	0.0083785649518086\\
42.55	0.00837856495180861\\
42.56	0.00837856495180861\\
42.57	0.00837856495180862\\
42.58	0.00837856495180863\\
42.59	0.00837856495180864\\
42.6	0.00837856495180865\\
42.61	0.00837856495180866\\
42.62	0.00837856495180867\\
42.63	0.00837856495180868\\
42.64	0.00837856495180869\\
42.65	0.0083785649518087\\
42.66	0.00837856495180871\\
42.67	0.00837856495180872\\
42.68	0.00837856495180873\\
42.69	0.00837856495180874\\
42.7	0.00837856495180875\\
42.71	0.00837856495180877\\
42.72	0.00837856495180878\\
42.73	0.00837856495180879\\
42.74	0.0083785649518088\\
42.75	0.00837856495180881\\
42.76	0.00837856495180882\\
42.77	0.00837856495180883\\
42.78	0.00837856495180884\\
42.79	0.00837856495180886\\
42.8	0.00837856495180887\\
42.81	0.00837856495180888\\
42.82	0.00837856495180889\\
42.83	0.0083785649518089\\
42.84	0.00837856495180892\\
42.85	0.00837856495180893\\
42.86	0.00837856495180894\\
42.87	0.00837856495180895\\
42.88	0.00837856495180897\\
42.89	0.00837856495180898\\
42.9	0.00837856495180899\\
42.91	0.00837856495180901\\
42.92	0.00837856495180902\\
42.93	0.00837856495180903\\
42.94	0.00837856495180905\\
42.95	0.00837856495180906\\
42.96	0.00837856495180907\\
42.97	0.00837856495180909\\
42.98	0.0083785649518091\\
42.99	0.00837856495180912\\
43	0.00837856495180913\\
43.01	0.00837856495180915\\
43.02	0.00837856495180916\\
43.03	0.00837856495180918\\
43.04	0.00837856495180919\\
43.05	0.00837856495180921\\
43.06	0.00837856495180922\\
43.07	0.00837856495180924\\
43.08	0.00837856495180925\\
43.09	0.00837856495180927\\
43.1	0.00837856495180929\\
43.11	0.0083785649518093\\
43.12	0.00837856495180932\\
43.13	0.00837856495180934\\
43.14	0.00837856495180935\\
43.15	0.00837856495180937\\
43.16	0.00837856495180939\\
43.17	0.00837856495180941\\
43.18	0.00837856495180942\\
43.19	0.00837856495180944\\
43.2	0.00837856495180946\\
43.21	0.00837856495180948\\
43.22	0.0083785649518095\\
43.23	0.00837856495180951\\
43.24	0.00837856495180953\\
43.25	0.00837856495180955\\
43.26	0.00837856495180957\\
43.27	0.00837856495180959\\
43.28	0.00837856495180961\\
43.29	0.00837856495180963\\
43.3	0.00837856495180965\\
43.31	0.00837856495180967\\
43.32	0.00837856495180969\\
43.33	0.00837856495180971\\
43.34	0.00837856495180973\\
43.35	0.00837856495180975\\
43.36	0.00837856495180977\\
43.37	0.0083785649518098\\
43.38	0.00837856495180982\\
43.39	0.00837856495180984\\
43.4	0.00837856495180986\\
43.41	0.00837856495180988\\
43.42	0.00837856495180991\\
43.43	0.00837856495180993\\
43.44	0.00837856495180995\\
43.45	0.00837856495180997\\
43.46	0.00837856495181\\
43.47	0.00837856495181002\\
43.48	0.00837856495181005\\
43.49	0.00837856495181007\\
43.5	0.0083785649518101\\
43.51	0.00837856495181012\\
43.52	0.00837856495181015\\
43.53	0.00837856495181017\\
43.54	0.0083785649518102\\
43.55	0.00837856495181022\\
43.56	0.00837856495181025\\
43.57	0.00837856495181027\\
43.58	0.0083785649518103\\
43.59	0.00837856495181033\\
43.6	0.00837856495181036\\
43.61	0.00837856495181038\\
43.62	0.00837856495181041\\
43.63	0.00837856495181044\\
43.64	0.00837856495181047\\
43.65	0.0083785649518105\\
43.66	0.00837856495181053\\
43.67	0.00837856495181056\\
43.68	0.00837856495181059\\
43.69	0.00837856495181061\\
43.7	0.00837856495181065\\
43.71	0.00837856495181068\\
43.72	0.00837856495181071\\
43.73	0.00837856495181074\\
43.74	0.00837856495181077\\
43.75	0.0083785649518108\\
43.76	0.00837856495181083\\
43.77	0.00837856495181087\\
43.78	0.0083785649518109\\
43.79	0.00837856495181093\\
43.8	0.00837856495181097\\
43.81	0.008378564951811\\
43.82	0.00837856495181103\\
43.83	0.00837856495181107\\
43.84	0.0083785649518111\\
43.85	0.00837856495181114\\
43.86	0.00837856495181118\\
43.87	0.00837856495181121\\
43.88	0.00837856495181125\\
43.89	0.00837856495181129\\
43.9	0.00837856495181132\\
43.91	0.00837856495181136\\
43.92	0.0083785649518114\\
43.93	0.00837856495181144\\
43.94	0.00837856495181147\\
43.95	0.00837856495181151\\
43.96	0.00837856495181155\\
43.97	0.00837856495181159\\
43.98	0.00837856495181163\\
43.99	0.00837856495181168\\
44	0.00837856495181172\\
44.01	0.00837856495181176\\
44.02	0.0083785649518118\\
44.03	0.00837856495181184\\
44.04	0.00837856495181189\\
44.05	0.00837856495181193\\
44.06	0.00837856495181197\\
44.07	0.00837856495181202\\
44.08	0.00837856495181206\\
44.09	0.00837856495181211\\
44.1	0.00837856495181216\\
44.11	0.0083785649518122\\
44.12	0.00837856495181225\\
44.13	0.0083785649518123\\
44.14	0.00837856495181234\\
44.15	0.00837856495181239\\
44.16	0.00837856495181244\\
44.17	0.00837856495181249\\
44.18	0.00837856495181254\\
44.19	0.00837856495181259\\
44.2	0.00837856495181264\\
44.21	0.00837856495181269\\
44.22	0.00837856495181275\\
44.23	0.0083785649518128\\
44.24	0.00837856495181285\\
44.25	0.0083785649518129\\
44.26	0.00837856495181296\\
44.27	0.00837856495181301\\
44.28	0.00837856495181307\\
44.29	0.00837856495181313\\
44.3	0.00837856495181318\\
44.31	0.00837856495181324\\
44.32	0.0083785649518133\\
44.33	0.00837856495181336\\
44.34	0.00837856495181341\\
44.35	0.00837856495181347\\
44.36	0.00837856495181353\\
44.37	0.00837856495181359\\
44.38	0.00837856495181366\\
44.39	0.00837856495181372\\
44.4	0.00837856495181378\\
44.41	0.00837856495181384\\
44.42	0.00837856495181391\\
44.43	0.00837856495181397\\
44.44	0.00837856495181404\\
44.45	0.0083785649518141\\
44.46	0.00837856495181417\\
44.47	0.00837856495181424\\
44.48	0.00837856495181431\\
44.49	0.00837856495181438\\
44.5	0.00837856495181444\\
44.51	0.00837856495181451\\
44.52	0.00837856495181459\\
44.53	0.00837856495181466\\
44.54	0.00837856495181473\\
44.55	0.0083785649518148\\
44.56	0.00837856495181488\\
44.57	0.00837856495181495\\
44.58	0.00837856495181503\\
44.59	0.0083785649518151\\
44.6	0.00837856495181518\\
44.61	0.00837856495181526\\
44.62	0.00837856495181534\\
44.63	0.00837856495181542\\
44.64	0.0083785649518155\\
44.65	0.00837856495181558\\
44.66	0.00837856495181566\\
44.67	0.00837856495181575\\
44.68	0.00837856495181583\\
44.69	0.00837856495181591\\
44.7	0.008378564951816\\
44.71	0.00837856495181609\\
44.72	0.00837856495181617\\
44.73	0.00837856495181626\\
44.74	0.00837856495181635\\
44.75	0.00837856495181644\\
44.76	0.00837856495181653\\
44.77	0.00837856495181662\\
44.78	0.00837856495181672\\
44.79	0.00837856495181681\\
44.8	0.00837856495181691\\
44.81	0.008378564951817\\
44.82	0.0083785649518171\\
44.83	0.0083785649518172\\
44.84	0.00837856495181729\\
44.85	0.0083785649518174\\
44.86	0.0083785649518175\\
44.87	0.0083785649518176\\
44.88	0.0083785649518177\\
44.89	0.00837856495181781\\
44.9	0.00837856495181791\\
44.91	0.00837856495181802\\
44.92	0.00837856495181813\\
44.93	0.00837856495181823\\
44.94	0.00837856495181834\\
44.95	0.00837856495181846\\
44.96	0.00837856495181857\\
44.97	0.00837856495181868\\
44.98	0.00837856495181879\\
44.99	0.00837856495181891\\
45	0.00837856495181903\\
45.01	0.00837856495181914\\
45.02	0.00837856495181926\\
45.03	0.00837856495181939\\
45.04	0.00837856495181951\\
45.05	0.00837856495181963\\
45.06	0.00837856495181975\\
45.07	0.00837856495181988\\
45.08	0.00837856495182001\\
45.09	0.00837856495182013\\
45.1	0.00837856495182026\\
45.11	0.00837856495182039\\
45.12	0.00837856495182053\\
45.13	0.00837856495182066\\
45.14	0.0083785649518208\\
45.15	0.00837856495182093\\
45.16	0.00837856495182107\\
45.17	0.00837856495182121\\
45.18	0.00837856495182135\\
45.19	0.00837856495182149\\
45.2	0.00837856495182164\\
45.21	0.00837856495182178\\
45.22	0.00837856495182193\\
45.23	0.00837856495182208\\
45.24	0.00837856495182223\\
45.25	0.00837856495182238\\
45.26	0.00837856495182253\\
45.27	0.00837856495182268\\
45.28	0.00837856495182284\\
45.29	0.008378564951823\\
45.3	0.00837856495182316\\
45.31	0.00837856495182332\\
45.32	0.00837856495182348\\
45.33	0.00837856495182364\\
45.34	0.00837856495182381\\
45.35	0.00837856495182398\\
45.36	0.00837856495182415\\
45.37	0.00837856495182432\\
45.38	0.00837856495182449\\
45.39	0.00837856495182467\\
45.4	0.00837856495182484\\
45.41	0.00837856495182502\\
45.42	0.0083785649518252\\
45.43	0.00837856495182538\\
45.44	0.00837856495182557\\
45.45	0.00837856495182576\\
45.46	0.00837856495182594\\
45.47	0.00837856495182613\\
45.48	0.00837856495182632\\
45.49	0.00837856495182652\\
45.5	0.00837856495182671\\
45.51	0.00837856495182691\\
45.52	0.00837856495182711\\
45.53	0.00837856495182731\\
45.54	0.00837856495182752\\
45.55	0.00837856495182772\\
45.56	0.00837856495182793\\
45.57	0.00837856495182814\\
45.58	0.00837856495182836\\
45.59	0.00837856495182857\\
45.6	0.00837856495182879\\
45.61	0.00837856495182901\\
45.62	0.00837856495182923\\
45.63	0.00837856495182945\\
45.64	0.00837856495182968\\
45.65	0.00837856495182991\\
45.66	0.00837856495183014\\
45.67	0.00837856495183037\\
45.68	0.00837856495183061\\
45.69	0.00837856495183085\\
45.7	0.00837856495183109\\
45.71	0.00837856495183133\\
45.72	0.00837856495183158\\
45.73	0.00837856495183182\\
45.74	0.00837856495183208\\
45.75	0.00837856495183233\\
45.76	0.00837856495183258\\
45.77	0.00837856495183284\\
45.78	0.00837856495183311\\
45.79	0.00837856495183337\\
45.8	0.00837856495183364\\
45.81	0.00837856495183391\\
45.82	0.00837856495183418\\
45.83	0.00837856495183445\\
45.84	0.00837856495183473\\
45.85	0.00837856495183501\\
45.86	0.0083785649518353\\
45.87	0.00837856495183558\\
45.88	0.00837856495183587\\
45.89	0.00837856495183617\\
45.9	0.00837856495183646\\
45.91	0.00837856495183676\\
45.92	0.00837856495183706\\
45.93	0.00837856495183737\\
45.94	0.00837856495183768\\
45.95	0.00837856495183799\\
45.96	0.0083785649518383\\
45.97	0.00837856495183862\\
45.98	0.00837856495183894\\
45.99	0.00837856495183927\\
46	0.0083785649518396\\
46.01	0.00837856495183993\\
46.02	0.00837856495184026\\
46.03	0.0083785649518406\\
46.04	0.00837856495184094\\
46.05	0.00837856495184129\\
46.06	0.00837856495184164\\
46.07	0.00837856495184199\\
46.08	0.00837856495184235\\
46.09	0.00837856495184271\\
46.1	0.00837856495184307\\
46.11	0.00837856495184344\\
46.12	0.00837856495184381\\
46.13	0.00837856495184418\\
46.14	0.00837856495184456\\
46.15	0.00837856495184494\\
46.16	0.00837856495184533\\
46.17	0.00837856495184572\\
46.18	0.00837856495184612\\
46.19	0.00837856495184651\\
46.2	0.00837856495184692\\
46.21	0.00837856495184732\\
46.22	0.00837856495184774\\
46.23	0.00837856495184815\\
46.24	0.00837856495184857\\
46.25	0.008378564951849\\
46.26	0.00837856495184943\\
46.27	0.00837856495184986\\
46.28	0.0083785649518503\\
46.29	0.00837856495185074\\
46.3	0.00837856495185118\\
46.31	0.00837856495185164\\
46.32	0.00837856495185209\\
46.33	0.00837856495185255\\
46.34	0.00837856495185302\\
46.35	0.00837856495185349\\
46.36	0.00837856495185396\\
46.37	0.00837856495185444\\
46.38	0.00837856495185493\\
46.39	0.00837856495185542\\
46.4	0.00837856495185591\\
46.41	0.00837856495185641\\
46.42	0.00837856495185692\\
46.43	0.00837856495185743\\
46.44	0.00837856495185794\\
46.45	0.00837856495185847\\
46.46	0.00837856495185899\\
46.47	0.00837856495185952\\
46.48	0.00837856495186006\\
46.49	0.0083785649518606\\
46.5	0.00837856495186115\\
46.51	0.00837856495186171\\
46.52	0.00837856495186227\\
46.53	0.00837856495186283\\
46.54	0.0083785649518634\\
46.55	0.00837856495186398\\
46.56	0.00837856495186456\\
46.57	0.00837856495186515\\
46.58	0.00837856495186575\\
46.59	0.00837856495186635\\
46.6	0.00837856495186696\\
46.61	0.00837856495186757\\
46.62	0.00837856495186819\\
46.63	0.00837856495186882\\
46.64	0.00837856495186945\\
46.65	0.00837856495187009\\
46.66	0.00837856495187074\\
46.67	0.00837856495187139\\
46.68	0.00837856495187205\\
46.69	0.00837856495187272\\
46.7	0.00837856495187339\\
46.71	0.00837856495187407\\
46.72	0.00837856495187476\\
46.73	0.00837856495187545\\
46.74	0.00837856495187616\\
46.75	0.00837856495187687\\
46.76	0.00837856495187758\\
46.77	0.00837856495187831\\
46.78	0.00837856495187904\\
46.79	0.00837856495187977\\
46.8	0.00837856495188052\\
46.81	0.00837856495188127\\
46.82	0.00837856495188204\\
46.83	0.00837856495188281\\
46.84	0.00837856495188358\\
46.85	0.00837856495188437\\
46.86	0.00837856495188516\\
46.87	0.00837856495188596\\
46.88	0.00837856495188677\\
46.89	0.00837856495188759\\
46.9	0.00837856495188842\\
46.91	0.00837856495188925\\
46.92	0.0083785649518901\\
46.93	0.00837856495189095\\
46.94	0.00837856495189181\\
46.95	0.00837856495189268\\
46.96	0.00837856495189356\\
46.97	0.00837856495189445\\
46.98	0.00837856495189535\\
46.99	0.00837856495189625\\
47	0.00837856495189717\\
47.01	0.00837856495189809\\
47.02	0.00837856495189903\\
47.03	0.00837856495189998\\
47.04	0.00837856495190093\\
47.05	0.00837856495190189\\
47.06	0.00837856495190287\\
47.07	0.00837856495190385\\
47.08	0.00837856495190484\\
47.09	0.00837856495190585\\
47.1	0.00837856495190686\\
47.11	0.00837856495190789\\
47.12	0.00837856495190892\\
47.13	0.00837856495190997\\
47.14	0.00837856495191103\\
47.15	0.0083785649519121\\
47.16	0.00837856495191317\\
47.17	0.00837856495191427\\
47.18	0.00837856495191537\\
47.19	0.00837856495191648\\
47.2	0.0083785649519176\\
47.21	0.00837856495191874\\
47.22	0.00837856495191988\\
47.23	0.00837856495192105\\
47.24	0.00837856495192222\\
47.25	0.0083785649519234\\
47.26	0.00837856495192459\\
47.27	0.0083785649519258\\
47.28	0.00837856495192702\\
47.29	0.00837856495192825\\
47.3	0.0083785649519295\\
47.31	0.00837856495193075\\
47.32	0.00837856495193203\\
47.33	0.00837856495193331\\
47.34	0.00837856495193461\\
47.35	0.00837856495193592\\
47.36	0.00837856495193724\\
47.37	0.00837856495193858\\
47.38	0.00837856495193993\\
47.39	0.00837856495194129\\
47.4	0.00837856495194267\\
47.41	0.00837856495194407\\
47.42	0.00837856495194547\\
47.43	0.00837856495194689\\
47.44	0.00837856495194833\\
47.45	0.00837856495194978\\
47.46	0.00837856495195125\\
47.47	0.00837856495195273\\
47.48	0.00837856495195423\\
47.49	0.00837856495195574\\
47.5	0.00837856495195726\\
47.51	0.0083785649519588\\
47.52	0.00837856495196036\\
47.53	0.00837856495196194\\
47.54	0.00837856495196353\\
47.55	0.00837856495196514\\
47.56	0.00837856495196676\\
47.57	0.0083785649519684\\
47.58	0.00837856495197006\\
47.59	0.00837856495197173\\
47.6	0.00837856495197342\\
47.61	0.00837856495197513\\
47.62	0.00837856495197685\\
47.63	0.0083785649519786\\
47.64	0.00837856495198036\\
47.65	0.00837856495198214\\
47.66	0.00837856495198394\\
47.67	0.00837856495198575\\
47.68	0.00837856495198759\\
47.69	0.00837856495198944\\
47.7	0.00837856495199131\\
47.71	0.00837856495199321\\
47.72	0.00837856495199512\\
47.73	0.00837856495199705\\
47.74	0.008378564951999\\
47.75	0.00837856495200097\\
47.76	0.00837856495200296\\
47.77	0.00837856495200497\\
47.78	0.008378564952007\\
47.79	0.00837856495200905\\
47.8	0.00837856495201113\\
47.81	0.00837856495201322\\
47.82	0.00837856495201534\\
47.83	0.00837856495201747\\
47.84	0.00837856495201963\\
47.85	0.00837856495202182\\
47.86	0.00837856495202402\\
47.87	0.00837856495202625\\
47.88	0.0083785649520285\\
47.89	0.00837856495203077\\
47.9	0.00837856495203306\\
47.91	0.00837856495203538\\
47.92	0.00837856495203772\\
47.93	0.00837856495204009\\
47.94	0.00837856495204248\\
47.95	0.0083785649520449\\
47.96	0.00837856495204734\\
47.97	0.0083785649520498\\
47.98	0.00837856495205229\\
47.99	0.00837856495205481\\
48	0.00837856495205735\\
48.01	0.00837856495205992\\
48.02	0.00837856495206251\\
48.03	0.00837856495206513\\
48.04	0.00837856495206778\\
48.05	0.00837856495207045\\
48.06	0.00837856495207315\\
48.07	0.00837856495207588\\
48.08	0.00837856495207864\\
48.09	0.00837856495208142\\
48.1	0.00837856495208423\\
48.11	0.00837856495208708\\
48.12	0.00837856495208995\\
48.13	0.00837856495209285\\
48.14	0.00837856495209578\\
48.15	0.00837856495209874\\
48.16	0.00837856495210173\\
48.17	0.00837856495210475\\
48.18	0.0083785649521078\\
48.19	0.00837856495211088\\
48.2	0.00837856495211399\\
48.21	0.00837856495211714\\
48.22	0.00837856495212032\\
48.23	0.00837856495212353\\
48.24	0.00837856495212677\\
48.25	0.00837856495213005\\
48.26	0.00837856495213336\\
48.27	0.0083785649521367\\
48.28	0.00837856495214008\\
48.29	0.00837856495214349\\
48.3	0.00837856495214694\\
48.31	0.00837856495215042\\
48.32	0.00837856495215393\\
48.33	0.00837856495215749\\
48.34	0.00837856495216108\\
48.35	0.0083785649521647\\
48.36	0.00837856495216836\\
48.37	0.00837856495217206\\
48.38	0.0083785649521758\\
48.39	0.00837856495217958\\
48.4	0.00837856495218339\\
48.41	0.00837856495218725\\
48.42	0.00837856495219114\\
48.43	0.00837856495219507\\
48.44	0.00837856495219904\\
48.45	0.00837856495220306\\
48.46	0.00837856495220711\\
48.47	0.0083785649522112\\
48.48	0.00837856495221534\\
48.49	0.00837856495221952\\
48.5	0.00837856495222374\\
48.51	0.008378564952228\\
48.52	0.00837856495223231\\
48.53	0.00837856495223666\\
48.54	0.00837856495224106\\
48.55	0.0083785649522455\\
48.56	0.00837856495224999\\
48.57	0.00837856495225452\\
48.58	0.00837856495225909\\
48.59	0.00837856495226372\\
48.6	0.00837856495226839\\
48.61	0.00837856495227311\\
48.62	0.00837856495227787\\
48.63	0.00837856495228269\\
48.64	0.00837856495228755\\
48.65	0.00837856495229247\\
48.66	0.00837856495229743\\
48.67	0.00837856495230244\\
48.68	0.00837856495230751\\
48.69	0.00837856495231262\\
48.7	0.00837856495231779\\
48.71	0.00837856495232301\\
48.72	0.00837856495232829\\
48.73	0.00837856495233361\\
48.74	0.008378564952339\\
48.75	0.00837856495234443\\
48.76	0.00837856495234992\\
48.77	0.00837856495235547\\
48.78	0.00837856495236108\\
48.79	0.00837856495236674\\
48.8	0.00837856495237245\\
48.81	0.00837856495237823\\
48.82	0.00837856495238407\\
48.83	0.00837856495238996\\
48.84	0.00837856495239591\\
48.85	0.00837856495240193\\
48.86	0.008378564952408\\
48.87	0.00837856495241414\\
48.88	0.00837856495242034\\
48.89	0.0083785649524266\\
48.9	0.00837856495243293\\
48.91	0.00837856495243932\\
48.92	0.00837856495244577\\
48.93	0.00837856495245229\\
48.94	0.00837856495245888\\
48.95	0.00837856495246553\\
48.96	0.00837856495247225\\
48.97	0.00837856495247904\\
48.98	0.0083785649524859\\
48.99	0.00837856495249283\\
49	0.00837856495249982\\
49.01	0.00837856495250689\\
49.02	0.00837856495251403\\
49.03	0.00837856495252124\\
49.04	0.00837856495252853\\
49.05	0.00837856495253589\\
49.06	0.00837856495254332\\
49.07	0.00837856495255083\\
49.08	0.00837856495255842\\
49.09	0.00837856495256608\\
49.1	0.00837856495257382\\
49.11	0.00837856495258164\\
49.12	0.00837856495258953\\
49.13	0.00837856495259751\\
49.14	0.00837856495260557\\
49.15	0.00837856495261371\\
49.16	0.00837856495262193\\
49.17	0.00837856495263024\\
49.18	0.00837856495263863\\
49.19	0.0083785649526471\\
49.2	0.00837856495265566\\
49.21	0.00837856495266431\\
49.22	0.00837856495267304\\
49.23	0.00837856495268186\\
49.24	0.00837856495269078\\
49.25	0.00837856495269978\\
49.26	0.00837856495270887\\
49.27	0.00837856495271806\\
49.28	0.00837856495272733\\
49.29	0.0083785649527367\\
49.3	0.00837856495274617\\
49.31	0.00837856495275573\\
49.32	0.00837856495276539\\
49.33	0.00837856495277515\\
49.34	0.008378564952785\\
49.35	0.00837856495279496\\
49.36	0.00837856495280501\\
49.37	0.00837856495281517\\
49.38	0.00837856495282543\\
49.39	0.00837856495283579\\
49.4	0.00837856495284626\\
49.41	0.00837856495285683\\
49.42	0.00837856495286751\\
49.43	0.0083785649528783\\
49.44	0.0083785649528892\\
49.45	0.0083785649529002\\
49.46	0.00837856495291132\\
49.47	0.00837856495292255\\
49.48	0.0083785649529339\\
49.49	0.00837856495294536\\
49.5	0.00837856495295693\\
49.51	0.00837856495296862\\
49.52	0.00837856495298043\\
49.53	0.00837856495299236\\
49.54	0.0083785649530044\\
49.55	0.00837856495301657\\
49.56	0.00837856495302887\\
49.57	0.00837856495304128\\
49.58	0.00837856495305382\\
49.59	0.00837856495306649\\
49.6	0.00837856495307929\\
49.61	0.00837856495309221\\
49.62	0.00837856495310527\\
49.63	0.00837856495311845\\
49.64	0.00837856495313177\\
49.65	0.00837856495314523\\
49.66	0.00837856495315882\\
49.67	0.00837856495317254\\
49.68	0.00837856495318641\\
49.69	0.00837856495320041\\
49.7	0.00837856495321456\\
49.71	0.00837856495322884\\
49.72	0.00837856495324328\\
49.73	0.00837856495325785\\
49.74	0.00837856495327258\\
49.75	0.00837856495328745\\
49.76	0.00837856495330247\\
49.77	0.00837856495331764\\
49.78	0.00837856495333297\\
49.79	0.00837856495334845\\
49.8	0.00837856495336408\\
49.81	0.00837856495337987\\
49.82	0.00837856495339583\\
49.83	0.00837856495341194\\
49.84	0.00837856495342821\\
49.85	0.00837856495344465\\
49.86	0.00837856495346125\\
49.87	0.00837856495347802\\
49.88	0.00837856495349496\\
49.89	0.00837856495351207\\
49.9	0.00837856495352935\\
49.91	0.0083785649535468\\
49.92	0.00837856495356443\\
49.93	0.00837856495358224\\
49.94	0.00837856495360023\\
49.95	0.00837856495361839\\
49.96	0.00837856495363674\\
49.97	0.00837856495365527\\
49.98	0.008378564953674\\
49.99	0.0083785649536929\\
50	0.008378564953712\\
50.01	0.00837856495373129\\
50.02	0.00837856495375077\\
50.03	0.00837856495377045\\
50.04	0.00837856495379032\\
50.05	0.0083785649538104\\
50.06	0.00837856495383068\\
50.07	0.00837856495385116\\
50.08	0.00837856495387184\\
50.09	0.00837856495389274\\
50.1	0.00837856495391384\\
50.11	0.00837856495393515\\
50.12	0.00837856495395668\\
50.13	0.00837856495397842\\
50.14	0.00837856495400039\\
50.15	0.00837856495402257\\
50.16	0.00837856495404497\\
50.17	0.0083785649540676\\
50.18	0.00837856495409046\\
50.19	0.00837856495411354\\
50.2	0.00837856495413686\\
50.21	0.00837856495416041\\
50.22	0.00837856495418419\\
50.23	0.00837856495420821\\
50.24	0.00837856495423248\\
50.25	0.00837856495425699\\
50.26	0.00837856495428174\\
50.27	0.00837856495430674\\
50.28	0.00837856495433199\\
50.29	0.00837856495435749\\
50.3	0.00837856495438325\\
50.31	0.00837856495440927\\
50.32	0.00837856495443554\\
50.33	0.00837856495446208\\
50.34	0.00837856495448889\\
50.35	0.00837856495451596\\
50.36	0.00837856495454331\\
50.37	0.00837856495457092\\
50.38	0.00837856495459882\\
50.39	0.00837856495462699\\
50.4	0.00837856495465544\\
50.41	0.00837856495468418\\
50.42	0.00837856495471321\\
50.43	0.00837856495474252\\
50.44	0.00837856495477213\\
50.45	0.00837856495480204\\
50.46	0.00837856495483224\\
50.47	0.00837856495486275\\
50.48	0.00837856495489355\\
50.49	0.00837856495492467\\
50.5	0.0083785649549561\\
50.51	0.00837856495498784\\
50.52	0.0083785649550199\\
50.53	0.00837856495505228\\
50.54	0.00837856495508498\\
50.55	0.00837856495511801\\
50.56	0.00837856495515137\\
50.57	0.00837856495518506\\
50.58	0.00837856495521908\\
50.59	0.00837856495525345\\
50.6	0.00837856495528816\\
50.61	0.00837856495532321\\
50.62	0.00837856495535862\\
50.63	0.00837856495539437\\
50.64	0.00837856495543049\\
50.65	0.00837856495546696\\
50.66	0.0083785649555038\\
50.67	0.008378564955541\\
50.68	0.00837856495557858\\
50.69	0.00837856495561653\\
50.7	0.00837856495565486\\
50.71	0.00837856495569357\\
50.72	0.00837856495573266\\
50.73	0.00837856495577214\\
50.74	0.00837856495581202\\
50.75	0.00837856495585229\\
50.76	0.00837856495589297\\
50.77	0.00837856495593405\\
50.78	0.00837856495597554\\
50.79	0.00837856495601744\\
50.8	0.00837856495605976\\
50.81	0.0083785649561025\\
50.82	0.00837856495614566\\
50.83	0.00837856495618926\\
50.84	0.00837856495623329\\
50.85	0.00837856495627775\\
50.86	0.00837856495632266\\
50.87	0.00837856495636801\\
50.88	0.00837856495641381\\
50.89	0.00837856495646007\\
50.9	0.00837856495650679\\
50.91	0.00837856495655398\\
50.92	0.00837856495660163\\
50.93	0.00837856495664975\\
50.94	0.00837856495669836\\
50.95	0.00837856495674744\\
50.96	0.00837856495679702\\
50.97	0.00837856495684708\\
50.98	0.00837856495689765\\
50.99	0.00837856495694871\\
51	0.00837856495700028\\
51.01	0.00837856495705237\\
51.02	0.00837856495710496\\
51.03	0.00837856495715809\\
51.04	0.00837856495721174\\
51.05	0.00837856495726592\\
51.06	0.00837856495732063\\
51.07	0.00837856495737589\\
51.08	0.0083785649574317\\
51.09	0.00837856495748806\\
51.1	0.00837856495754498\\
51.11	0.00837856495760247\\
51.12	0.00837856495766052\\
51.13	0.00837856495771915\\
51.14	0.00837856495777836\\
51.15	0.00837856495783816\\
51.16	0.00837856495789855\\
51.17	0.00837856495795953\\
51.18	0.00837856495802112\\
51.19	0.00837856495808332\\
51.2	0.00837856495814614\\
51.21	0.00837856495820957\\
51.22	0.00837856495827364\\
51.23	0.00837856495833834\\
51.24	0.00837856495840368\\
51.25	0.00837856495846966\\
51.26	0.0083785649585363\\
51.27	0.0083785649586036\\
51.28	0.00837856495867156\\
51.29	0.00837856495874019\\
51.3	0.0083785649588095\\
51.31	0.0083785649588795\\
51.32	0.00837856495895019\\
51.33	0.00837856495902158\\
51.34	0.00837856495909367\\
51.35	0.00837856495916648\\
51.36	0.00837856495924\\
51.37	0.00837856495931425\\
51.38	0.00837856495938923\\
51.39	0.00837856495946496\\
51.4	0.00837856495954143\\
51.41	0.00837856495961865\\
51.42	0.00837856495969664\\
51.43	0.0083785649597754\\
51.44	0.00837856495985493\\
51.45	0.00837856495993525\\
51.46	0.00837856496001636\\
51.47	0.00837856496009827\\
51.48	0.00837856496018099\\
51.49	0.00837856496026452\\
51.5	0.00837856496034888\\
51.51	0.00837856496043407\\
51.52	0.0083785649605201\\
51.53	0.00837856496060697\\
51.54	0.0083785649606947\\
51.55	0.0083785649607833\\
51.56	0.00837856496087277\\
51.57	0.00837856496096312\\
51.58	0.00837856496105436\\
51.59	0.00837856496114649\\
51.6	0.00837856496123954\\
51.61	0.0083785649613335\\
51.62	0.00837856496142838\\
51.63	0.0083785649615242\\
51.64	0.00837856496162096\\
51.65	0.00837856496171867\\
51.66	0.00837856496181734\\
51.67	0.00837856496191698\\
51.68	0.00837856496201761\\
51.69	0.00837856496211922\\
51.7	0.00837856496222183\\
51.71	0.00837856496232545\\
51.72	0.00837856496243008\\
51.73	0.00837856496253575\\
51.74	0.00837856496264245\\
51.75	0.0083785649627502\\
51.76	0.00837856496285901\\
51.77	0.00837856496296888\\
51.78	0.00837856496307984\\
51.79	0.00837856496319189\\
51.8	0.00837856496330503\\
51.81	0.00837856496341928\\
51.82	0.00837856496353466\\
51.83	0.00837856496365116\\
51.84	0.00837856496376881\\
51.85	0.00837856496388762\\
51.86	0.00837856496400758\\
51.87	0.00837856496412873\\
51.88	0.00837856496425106\\
51.89	0.00837856496437459\\
51.9	0.00837856496449933\\
51.91	0.00837856496462529\\
51.92	0.00837856496475249\\
51.93	0.00837856496488093\\
51.94	0.00837856496501063\\
51.95	0.0083785649651416\\
51.96	0.00837856496527385\\
51.97	0.0083785649654074\\
51.98	0.00837856496554225\\
51.99	0.00837856496567842\\
52	0.00837856496581593\\
52.01	0.00837856496595478\\
52.02	0.00837856496609498\\
52.03	0.00837856496623656\\
52.04	0.00837856496637952\\
52.05	0.00837856496652388\\
52.06	0.00837856496666966\\
52.07	0.00837856496681685\\
52.08	0.00837856496696548\\
52.09	0.00837856496711557\\
52.1	0.00837856496726712\\
52.11	0.00837856496742015\\
52.12	0.00837856496757468\\
52.13	0.00837856496773071\\
52.14	0.00837856496788827\\
52.15	0.00837856496804736\\
52.16	0.00837856496820801\\
52.17	0.00837856496837023\\
52.18	0.00837856496853402\\
52.19	0.00837856496869942\\
52.2	0.00837856496886642\\
52.21	0.00837856496903506\\
52.22	0.00837856496920534\\
52.23	0.00837856496937727\\
52.24	0.00837856496955089\\
52.25	0.00837856496972619\\
52.26	0.00837856496990321\\
52.27	0.00837856497008195\\
52.28	0.00837856497026242\\
52.29	0.00837856497044466\\
52.3	0.00837856497062867\\
52.31	0.00837856497081447\\
52.32	0.00837856497100207\\
52.33	0.00837856497119151\\
52.34	0.00837856497138278\\
52.35	0.00837856497157592\\
52.36	0.00837856497177093\\
52.37	0.00837856497196784\\
52.38	0.00837856497216667\\
52.39	0.00837856497236743\\
52.4	0.00837856497257014\\
52.41	0.00837856497277482\\
52.42	0.00837856497298148\\
52.43	0.00837856497319016\\
52.44	0.00837856497340085\\
52.45	0.0083785649736136\\
52.46	0.00837856497382841\\
52.47	0.00837856497404531\\
52.48	0.00837856497426431\\
52.49	0.00837856497448543\\
52.5	0.0083785649747087\\
52.51	0.00837856497493414\\
52.52	0.00837856497516176\\
52.53	0.00837856497539159\\
52.54	0.00837856497562365\\
52.55	0.00837856497585796\\
52.56	0.00837856497609453\\
52.57	0.0083785649763334\\
52.58	0.00837856497657459\\
52.59	0.0083785649768181\\
52.6	0.00837856497706398\\
52.61	0.00837856497731224\\
52.62	0.0083785649775629\\
52.63	0.00837856497781599\\
52.64	0.00837856497807152\\
52.65	0.00837856497832953\\
52.66	0.00837856497859003\\
52.67	0.00837856497885305\\
52.68	0.00837856497911862\\
52.69	0.00837856497938675\\
52.7	0.00837856497965747\\
52.71	0.00837856497993082\\
52.72	0.0083785649802068\\
52.73	0.00837856498048544\\
52.74	0.00837856498076678\\
52.75	0.00837856498105084\\
52.76	0.00837856498133763\\
52.77	0.0083785649816272\\
52.78	0.00837856498191956\\
52.79	0.00837856498221474\\
52.8	0.00837856498251277\\
52.81	0.00837856498281367\\
52.82	0.00837856498311748\\
52.83	0.00837856498342421\\
52.84	0.0083785649837339\\
52.85	0.00837856498404658\\
52.86	0.00837856498436227\\
52.87	0.008378564984681\\
52.88	0.0083785649850028\\
52.89	0.0083785649853277\\
52.9	0.00837856498565573\\
52.91	0.00837856498598692\\
52.92	0.00837856498632129\\
52.93	0.00837856498665888\\
52.94	0.00837856498699972\\
52.95	0.00837856498734383\\
52.96	0.00837856498769126\\
52.97	0.00837856498804203\\
52.98	0.00837856498839617\\
52.99	0.00837856498875371\\
53	0.00837856498911469\\
53.01	0.00837856498947914\\
53.02	0.00837856498984708\\
53.03	0.00837856499021856\\
53.04	0.00837856499059361\\
53.05	0.00837856499097226\\
53.06	0.00837856499135454\\
53.07	0.00837856499174049\\
53.08	0.00837856499213015\\
53.09	0.00837856499252354\\
53.1	0.0083785649929207\\
53.11	0.00837856499332167\\
53.12	0.00837856499372649\\
53.13	0.00837856499413518\\
53.14	0.00837856499454779\\
53.15	0.00837856499496436\\
53.16	0.00837856499538491\\
53.17	0.00837856499580949\\
53.18	0.00837856499623814\\
53.19	0.00837856499667088\\
53.2	0.00837856499710777\\
53.21	0.00837856499754884\\
53.22	0.00837856499799413\\
53.23	0.00837856499844368\\
53.24	0.00837856499889753\\
53.25	0.00837856499935571\\
53.26	0.00837856499981827\\
53.27	0.00837856500028526\\
53.28	0.0083785650007567\\
53.29	0.00837856500123265\\
53.3	0.00837856500171315\\
53.31	0.00837856500219823\\
53.32	0.00837856500268794\\
53.33	0.00837856500318232\\
53.34	0.00837856500368143\\
53.35	0.00837856500418529\\
53.36	0.00837856500469396\\
53.37	0.00837856500520747\\
53.38	0.00837856500572589\\
53.39	0.00837856500624924\\
53.4	0.00837856500677758\\
53.41	0.00837856500731096\\
53.42	0.00837856500784941\\
53.43	0.008378565008393\\
53.44	0.00837856500894175\\
53.45	0.00837856500949573\\
53.46	0.00837856501005498\\
53.47	0.00837856501061955\\
53.48	0.0083785650111895\\
53.49	0.00837856501176486\\
53.5	0.00837856501234569\\
53.51	0.00837856501293205\\
53.52	0.00837856501352397\\
53.53	0.00837856501412152\\
53.54	0.00837856501472474\\
53.55	0.0083785650153337\\
53.56	0.00837856501594843\\
53.57	0.008378565016569\\
53.58	0.00837856501719546\\
53.59	0.00837856501782786\\
53.6	0.00837856501846626\\
53.61	0.00837856501911071\\
53.62	0.00837856501976128\\
53.63	0.00837856502041801\\
53.64	0.00837856502108097\\
53.65	0.0083785650217502\\
53.66	0.00837856502242578\\
53.67	0.00837856502310775\\
53.68	0.00837856502379618\\
53.69	0.00837856502449112\\
53.7	0.00837856502519264\\
53.71	0.0083785650259008\\
53.72	0.00837856502661566\\
53.73	0.00837856502733727\\
53.74	0.00837856502806571\\
53.75	0.00837856502880104\\
53.76	0.00837856502954331\\
53.77	0.00837856503029259\\
53.78	0.00837856503104895\\
53.79	0.00837856503181245\\
53.8	0.00837856503258316\\
53.81	0.00837856503336114\\
53.82	0.00837856503414647\\
53.83	0.0083785650349392\\
53.84	0.00837856503573941\\
53.85	0.00837856503654716\\
53.86	0.00837856503736253\\
53.87	0.00837856503818558\\
53.88	0.00837856503901638\\
53.89	0.00837856503985501\\
53.9	0.00837856504070154\\
53.91	0.00837856504155604\\
53.92	0.00837856504241858\\
53.93	0.00837856504328923\\
53.94	0.00837856504416808\\
53.95	0.00837856504505519\\
53.96	0.00837856504595064\\
53.97	0.00837856504685451\\
53.98	0.00837856504776687\\
53.99	0.0083785650486878\\
54	0.00837856504961739\\
54.01	0.0083785650505557\\
54.02	0.00837856505150282\\
54.03	0.00837856505245883\\
54.04	0.00837856505342381\\
54.05	0.00837856505439785\\
54.06	0.00837856505538101\\
54.07	0.0083785650563734\\
54.08	0.00837856505737508\\
54.09	0.00837856505838616\\
54.1	0.0083785650594067\\
54.11	0.0083785650604368\\
54.12	0.00837856506147655\\
54.13	0.00837856506252603\\
54.14	0.00837856506358533\\
54.15	0.00837856506465454\\
54.16	0.00837856506573375\\
54.17	0.00837856506682305\\
54.18	0.00837856506792253\\
54.19	0.00837856506903229\\
54.2	0.00837856507015242\\
54.21	0.008378565071283\\
54.22	0.00837856507242415\\
54.23	0.00837856507357595\\
54.24	0.0083785650747385\\
54.25	0.00837856507591189\\
54.26	0.00837856507709623\\
54.27	0.00837856507829161\\
54.28	0.00837856507949814\\
54.29	0.00837856508071591\\
54.3	0.00837856508194502\\
54.31	0.00837856508318559\\
54.32	0.0083785650844377\\
54.33	0.00837856508570147\\
54.34	0.00837856508697701\\
54.35	0.00837856508826441\\
54.36	0.00837856508956378\\
54.37	0.00837856509087523\\
54.38	0.00837856509219888\\
54.39	0.00837856509353483\\
54.4	0.00837856509488319\\
54.41	0.00837856509624407\\
54.42	0.00837856509761759\\
54.43	0.00837856509900385\\
54.44	0.00837856510040298\\
54.45	0.00837856510181509\\
54.46	0.0083785651032403\\
54.47	0.00837856510467872\\
54.48	0.00837856510613047\\
54.49	0.00837856510759567\\
54.5	0.00837856510907445\\
54.51	0.00837856511056692\\
54.52	0.0083785651120732\\
54.53	0.00837856511359343\\
54.54	0.00837856511512772\\
54.55	0.00837856511667621\\
54.56	0.00837856511823901\\
54.57	0.00837856511981626\\
54.58	0.00837856512140808\\
54.59	0.0083785651230146\\
54.6	0.00837856512463597\\
54.61	0.0083785651262723\\
54.62	0.00837856512792374\\
54.63	0.00837856512959041\\
54.64	0.00837856513127246\\
54.65	0.00837856513297001\\
54.66	0.00837856513468322\\
54.67	0.00837856513641221\\
54.68	0.00837856513815713\\
54.69	0.00837856513991812\\
54.7	0.00837856514169533\\
54.71	0.00837856514348889\\
54.72	0.00837856514529895\\
54.73	0.00837856514712566\\
54.74	0.00837856514896917\\
54.75	0.00837856515082963\\
54.76	0.00837856515270717\\
54.77	0.00837856515460197\\
54.78	0.00837856515651417\\
54.79	0.00837856515844392\\
54.8	0.00837856516039138\\
54.81	0.00837856516235671\\
54.82	0.00837856516434006\\
54.83	0.0083785651663416\\
54.84	0.00837856516836148\\
54.85	0.00837856517039987\\
54.86	0.00837856517245693\\
54.87	0.00837856517453282\\
54.88	0.00837856517662772\\
54.89	0.00837856517874178\\
54.9	0.00837856518087518\\
54.91	0.00837856518302809\\
54.92	0.00837856518520068\\
54.93	0.00837856518739312\\
54.94	0.00837856518960558\\
54.95	0.00837856519183826\\
54.96	0.00837856519409131\\
54.97	0.00837856519636492\\
54.98	0.00837856519865928\\
54.99	0.00837856520097456\\
55	0.00837856520331094\\
55.01	0.00837856520566862\\
55.02	0.00837856520804777\\
55.03	0.00837856521044859\\
55.04	0.00837856521287127\\
55.05	0.00837856521531599\\
55.06	0.00837856521778295\\
55.07	0.00837856522027235\\
55.08	0.00837856522278438\\
55.09	0.00837856522531924\\
55.1	0.00837856522787712\\
55.11	0.00837856523045823\\
55.12	0.00837856523306278\\
55.13	0.00837856523569096\\
55.14	0.00837856523834298\\
55.15	0.00837856524101905\\
55.16	0.00837856524371938\\
55.17	0.00837856524644418\\
55.18	0.00837856524919366\\
55.19	0.00837856525196803\\
55.2	0.00837856525476752\\
55.21	0.00837856525759234\\
55.22	0.00837856526044271\\
55.23	0.00837856526331886\\
55.24	0.008378565266221\\
55.25	0.00837856526914936\\
55.26	0.00837856527210416\\
55.27	0.00837856527508565\\
55.28	0.00837856527809404\\
55.29	0.00837856528112958\\
55.3	0.00837856528419249\\
55.31	0.00837856528728301\\
55.32	0.00837856529040138\\
55.33	0.00837856529354784\\
55.34	0.00837856529672263\\
55.35	0.008378565299926\\
55.36	0.00837856530315819\\
55.37	0.00837856530641945\\
55.38	0.00837856530971004\\
55.39	0.00837856531303019\\
55.4	0.00837856531638018\\
55.41	0.00837856531976024\\
55.42	0.00837856532317065\\
55.43	0.00837856532661166\\
55.44	0.00837856533008354\\
55.45	0.00837856533358654\\
55.46	0.00837856533712095\\
55.47	0.00837856534068701\\
55.48	0.00837856534428501\\
55.49	0.00837856534791522\\
55.5	0.00837856535157791\\
55.51	0.00837856535527337\\
55.52	0.00837856535900187\\
55.53	0.00837856536276368\\
55.54	0.00837856536655911\\
55.55	0.00837856537038843\\
55.56	0.00837856537425193\\
55.57	0.00837856537814991\\
55.58	0.00837856538208265\\
55.59	0.00837856538605046\\
55.6	0.00837856539005362\\
55.61	0.00837856539409245\\
55.62	0.00837856539816723\\
55.63	0.00837856540227829\\
55.64	0.00837856540642592\\
55.65	0.00837856541061044\\
55.66	0.00837856541483216\\
55.67	0.00837856541909138\\
55.68	0.00837856542338844\\
55.69	0.00837856542772365\\
55.7	0.00837856543209733\\
55.71	0.0083785654365098\\
55.72	0.0083785654409614\\
55.73	0.00837856544545246\\
55.74	0.00837856544998329\\
55.75	0.00837856545455425\\
55.76	0.00837856545916567\\
55.77	0.00837856546381789\\
55.78	0.00837856546851125\\
55.79	0.00837856547324609\\
55.8	0.00837856547802278\\
55.81	0.00837856548284164\\
55.82	0.00837856548770305\\
55.83	0.00837856549260735\\
55.84	0.00837856549755491\\
55.85	0.00837856550254609\\
55.86	0.00837856550758125\\
55.87	0.00837856551266076\\
55.88	0.00837856551778498\\
55.89	0.0083785655229543\\
55.9	0.00837856552816909\\
55.91	0.00837856553342973\\
55.92	0.00837856553873659\\
55.93	0.00837856554409007\\
55.94	0.00837856554949055\\
55.95	0.00837856555493843\\
55.96	0.00837856556043409\\
55.97	0.00837856556597793\\
55.98	0.00837856557157035\\
55.99	0.00837856557721176\\
56	0.00837856558290256\\
56.01	0.00837856558864316\\
56.02	0.00837856559443397\\
56.03	0.0083785656002754\\
56.04	0.00837856560616788\\
56.05	0.00837856561211182\\
56.06	0.00837856561810766\\
56.07	0.00837856562415581\\
56.08	0.00837856563025671\\
56.09	0.00837856563641078\\
56.1	0.00837856564261848\\
56.11	0.00837856564888024\\
56.12	0.00837856565519651\\
56.13	0.00837856566156772\\
56.14	0.00837856566799433\\
56.15	0.0083785656744768\\
56.16	0.00837856568101558\\
56.17	0.00837856568761114\\
56.18	0.00837856569426393\\
56.19	0.00837856570097443\\
56.2	0.0083785657077431\\
56.21	0.00837856571457042\\
56.22	0.00837856572145687\\
56.23	0.00837856572840293\\
56.24	0.00837856573540908\\
56.25	0.00837856574247582\\
56.26	0.00837856574960363\\
56.27	0.00837856575679301\\
56.28	0.00837856576404447\\
56.29	0.00837856577135849\\
56.3	0.0083785657787356\\
56.31	0.0083785657861763\\
56.32	0.0083785657936811\\
56.33	0.00837856580125052\\
56.34	0.00837856580888509\\
56.35	0.00837856581658533\\
56.36	0.00837856582435176\\
56.37	0.00837856583218493\\
56.38	0.00837856584008537\\
56.39	0.00837856584805362\\
56.4	0.00837856585609022\\
56.41	0.00837856586419572\\
56.42	0.00837856587237068\\
56.43	0.00837856588061565\\
56.44	0.0083785658889312\\
56.45	0.00837856589731788\\
56.46	0.00837856590577626\\
56.47	0.00837856591430693\\
56.48	0.00837856592291045\\
56.49	0.0083785659315874\\
56.5	0.00837856594033837\\
56.51	0.00837856594916396\\
56.52	0.00837856595806474\\
56.53	0.00837856596704133\\
56.54	0.00837856597609432\\
56.55	0.00837856598522431\\
56.56	0.00837856599443193\\
56.57	0.00837856600371777\\
56.58	0.00837856601308247\\
56.59	0.00837856602252663\\
56.6	0.0083785660320509\\
56.61	0.0083785660416559\\
56.62	0.00837856605134227\\
56.63	0.00837856606111065\\
56.64	0.00837856607096168\\
56.65	0.00837856608089602\\
56.66	0.00837856609091431\\
56.67	0.00837856610101722\\
56.68	0.0083785661112054\\
56.69	0.00837856612147954\\
56.7	0.00837856613184029\\
56.71	0.00837856614228833\\
56.72	0.00837856615282435\\
56.73	0.00837856616344904\\
56.74	0.00837856617416308\\
56.75	0.00837856618496716\\
56.76	0.008378566195862\\
56.77	0.00837856620684828\\
56.78	0.00837856621792673\\
56.79	0.00837856622909806\\
56.8	0.00837856624036298\\
56.81	0.00837856625172223\\
56.82	0.00837856626317652\\
56.83	0.00837856627472659\\
56.84	0.00837856628637319\\
56.85	0.00837856629811705\\
56.86	0.00837856630995893\\
56.87	0.00837856632189957\\
56.88	0.00837856633393974\\
56.89	0.0083785663460802\\
56.9	0.00837856635832171\\
56.91	0.00837856637066506\\
56.92	0.00837856638311102\\
56.93	0.00837856639566037\\
56.94	0.00837856640831391\\
56.95	0.00837856642107242\\
56.96	0.00837856643393671\\
56.97	0.00837856644690758\\
56.98	0.00837856645998584\\
56.99	0.00837856647317231\\
57	0.0083785664864678\\
57.01	0.00837856649987315\\
57.02	0.00837856651338917\\
57.03	0.00837856652701672\\
57.04	0.00837856654075662\\
57.05	0.00837856655460974\\
57.06	0.00837856656857691\\
57.07	0.008378566582659\\
57.08	0.00837856659685688\\
57.09	0.0083785666111714\\
57.1	0.00837856662560344\\
57.11	0.00837856664015389\\
57.12	0.00837856665482362\\
57.13	0.00837856666961354\\
57.14	0.00837856668452453\\
57.15	0.0083785666995575\\
57.16	0.00837856671471335\\
57.17	0.008378566729993\\
57.18	0.00837856674539737\\
57.19	0.00837856676092738\\
57.2	0.00837856677658396\\
57.21	0.00837856679236805\\
57.22	0.00837856680828059\\
57.23	0.00837856682432253\\
57.24	0.00837856684049481\\
57.25	0.00837856685679841\\
57.26	0.00837856687323428\\
57.27	0.00837856688980339\\
57.28	0.00837856690650672\\
57.29	0.00837856692334526\\
57.3	0.00837856694031999\\
57.31	0.00837856695743191\\
57.32	0.008378566974682\\
57.33	0.0083785669920713\\
57.34	0.00837856700960079\\
57.35	0.00837856702727151\\
57.36	0.00837856704508447\\
57.37	0.00837856706304071\\
57.38	0.00837856708114126\\
57.39	0.00837856709938716\\
57.4	0.00837856711777947\\
57.41	0.00837856713631923\\
57.42	0.00837856715500751\\
57.43	0.00837856717384537\\
57.44	0.00837856719283388\\
57.45	0.00837856721197413\\
57.46	0.0083785672312672\\
57.47	0.00837856725071418\\
57.48	0.00837856727031616\\
57.49	0.00837856729007426\\
57.5	0.00837856730998957\\
57.51	0.00837856733006322\\
57.52	0.00837856735029632\\
57.53	0.00837856737069002\\
57.54	0.00837856739124543\\
57.55	0.0083785674119637\\
57.56	0.00837856743284598\\
57.57	0.00837856745389343\\
57.58	0.00837856747510719\\
57.59	0.00837856749648844\\
57.6	0.00837856751803835\\
57.61	0.0083785675397581\\
57.62	0.00837856756164887\\
57.63	0.00837856758371186\\
57.64	0.00837856760594825\\
57.65	0.00837856762835927\\
57.66	0.00837856765094611\\
57.67	0.00837856767370999\\
57.68	0.00837856769665215\\
57.69	0.00837856771977379\\
57.7	0.00837856774307617\\
57.71	0.00837856776656053\\
57.72	0.00837856779022811\\
57.73	0.00837856781408017\\
57.74	0.00837856783811797\\
57.75	0.00837856786234277\\
57.76	0.00837856788675587\\
57.77	0.00837856791135852\\
57.78	0.00837856793615203\\
57.79	0.00837856796113769\\
57.8	0.00837856798631679\\
57.81	0.00837856801169064\\
57.82	0.00837856803726056\\
57.83	0.00837856806302787\\
57.84	0.00837856808899388\\
57.85	0.00837856811515994\\
57.86	0.00837856814152739\\
57.87	0.00837856816809756\\
57.88	0.00837856819487182\\
57.89	0.00837856822185151\\
57.9	0.008378568249038\\
57.91	0.00837856827643268\\
57.92	0.0083785683040369\\
57.93	0.00837856833185206\\
57.94	0.00837856835987954\\
57.95	0.00837856838812074\\
57.96	0.00837856841657707\\
57.97	0.00837856844524994\\
57.98	0.00837856847414076\\
57.99	0.00837856850325094\\
58	0.00837856853258193\\
58.01	0.00837856856213516\\
58.02	0.00837856859191206\\
58.03	0.00837856862191408\\
58.04	0.00837856865214268\\
58.05	0.00837856868259932\\
58.06	0.00837856871328546\\
58.07	0.00837856874420258\\
58.08	0.00837856877535215\\
58.09	0.00837856880673566\\
58.1	0.00837856883835459\\
58.11	0.00837856887021046\\
58.12	0.00837856890230475\\
58.13	0.00837856893463898\\
58.14	0.00837856896721466\\
58.15	0.00837856900003332\\
58.16	0.00837856903309649\\
58.17	0.00837856906640569\\
58.18	0.00837856909996246\\
58.19	0.00837856913376835\\
58.2	0.00837856916782492\\
58.21	0.00837856920213371\\
58.22	0.0083785692366963\\
58.23	0.00837856927151425\\
58.24	0.00837856930658913\\
58.25	0.00837856934192253\\
58.26	0.00837856937751603\\
58.27	0.00837856941337123\\
58.28	0.00837856944948972\\
58.29	0.0083785694858731\\
58.3	0.00837856952252298\\
58.31	0.00837856955944098\\
58.32	0.00837856959662872\\
58.33	0.00837856963408782\\
58.34	0.00837856967181991\\
58.35	0.00837856970982663\\
58.36	0.00837856974810962\\
58.37	0.00837856978667052\\
58.38	0.008378569825511\\
58.39	0.0083785698646327\\
58.4	0.00837856990403728\\
58.41	0.00837856994372642\\
58.42	0.00837856998370179\\
58.43	0.00837857002396505\\
58.44	0.00837857006451791\\
58.45	0.00837857010536203\\
58.46	0.00837857014649912\\
58.47	0.00837857018793087\\
58.48	0.00837857022965898\\
58.49	0.00837857027168516\\
58.5	0.00837857031401111\\
58.51	0.00837857035663855\\
58.52	0.00837857039956921\\
58.53	0.00837857044280479\\
58.54	0.00837857048634704\\
58.55	0.00837857053019768\\
58.56	0.00837857057435845\\
58.57	0.00837857061883109\\
58.58	0.00837857066361734\\
58.59	0.00837857070871895\\
58.6	0.00837857075413768\\
58.61	0.00837857079987527\\
58.62	0.00837857084593349\\
58.63	0.00837857089231409\\
58.64	0.00837857093901886\\
58.65	0.00837857098604954\\
58.66	0.00837857103340793\\
58.67	0.00837857108109578\\
58.68	0.00837857112911489\\
58.69	0.00837857117746704\\
58.7	0.008378571226154\\
58.71	0.00837857127517757\\
58.72	0.00837857132453954\\
58.73	0.0083785713742417\\
58.74	0.00837857142428584\\
58.75	0.00837857147467377\\
58.76	0.00837857152540729\\
58.77	0.0083785715764882\\
58.78	0.00837857162791829\\
58.79	0.00837857167969939\\
58.8	0.00837857173183329\\
58.81	0.00837857178432182\\
58.82	0.00837857183716677\\
58.83	0.00837857189036997\\
58.84	0.00837857194393323\\
58.85	0.00837857199785836\\
58.86	0.00837857205214719\\
58.87	0.00837857210680153\\
58.88	0.0083785721618232\\
58.89	0.00837857221721401\\
58.9	0.0083785722729758\\
58.91	0.00837857232911038\\
58.92	0.00837857238561958\\
58.93	0.0083785724425052\\
58.94	0.00837857249976909\\
58.95	0.00837857255741305\\
58.96	0.00837857261543892\\
58.97	0.0083785726738485\\
58.98	0.00837857273264363\\
58.99	0.00837857279182611\\
59	0.00837857285139778\\
59.01	0.00837857291136044\\
59.02	0.00837857297171592\\
59.03	0.00837857303246602\\
59.04	0.00837857309361258\\
59.05	0.00837857315515739\\
59.06	0.00837857321710226\\
59.07	0.00837857327944901\\
59.08	0.00837857334219944\\
59.09	0.00837857340535536\\
59.1	0.00837857346891856\\
59.11	0.00837857353289085\\
59.12	0.00837857359727402\\
59.13	0.00837857366206987\\
59.14	0.00837857372728018\\
59.15	0.00837857379290674\\
59.16	0.00837857385895133\\
59.17	0.00837857392541573\\
59.18	0.00837857399230172\\
59.19	0.00837857405961106\\
59.2	0.00837857412734553\\
59.21	0.00837857419550688\\
59.22	0.00837857426409686\\
59.23	0.00837857433311724\\
59.24	0.00837857440256975\\
59.25	0.00837857447245614\\
59.26	0.00837857454277813\\
59.27	0.00837857461353747\\
59.28	0.00837857468473587\\
59.29	0.00837857475637505\\
59.3	0.00837857482845671\\
59.31	0.00837857490098257\\
59.32	0.00837857497395431\\
59.33	0.00837857504737363\\
59.34	0.0083785751212422\\
59.35	0.00837857519556169\\
59.36	0.00837857527033378\\
59.37	0.00837857534556012\\
59.38	0.00837857542124235\\
59.39	0.00837857549738212\\
59.4	0.00837857557398105\\
59.41	0.00837857565104077\\
59.42	0.00837857572856288\\
59.43	0.008378575806549\\
59.44	0.0083785758850007\\
59.45	0.00837857596391957\\
59.46	0.00837857604330717\\
59.47	0.00837857612316508\\
59.48	0.00837857620349483\\
59.49	0.00837857628429796\\
59.5	0.008378576365576\\
59.51	0.00837857644733046\\
59.52	0.00837857652956283\\
59.53	0.00837857661227461\\
59.54	0.00837857669546727\\
59.55	0.00837857677914227\\
59.56	0.00837857686330105\\
59.57	0.00837857694794506\\
59.58	0.00837857703307571\\
59.59	0.0083785771186944\\
59.6	0.00837857720480253\\
59.61	0.00837857729140147\\
59.62	0.00837857737849258\\
59.63	0.0083785774660772\\
59.64	0.00837857755415667\\
59.65	0.00837857764273228\\
59.66	0.00837857773180535\\
59.67	0.00837857782137713\\
59.68	0.00837857791144891\\
59.69	0.00837857800202191\\
59.7	0.00837857809309736\\
59.71	0.00837857818467647\\
59.72	0.00837857827676042\\
59.73	0.00837857836935039\\
59.74	0.00837857846244752\\
59.75	0.00837857855605294\\
59.76	0.00837857865016775\\
59.77	0.00837857874479306\\
59.78	0.00837857883992992\\
59.79	0.00837857893557938\\
59.8	0.00837857903174246\\
59.81	0.00837857912842017\\
59.82	0.00837857922561349\\
59.83	0.00837857932332338\\
59.84	0.00837857942155076\\
59.85	0.00837857952029656\\
59.86	0.00837857961956166\\
59.87	0.00837857971934691\\
59.88	0.00837857981965317\\
59.89	0.00837857992048124\\
59.9	0.00837858002183191\\
59.91	0.00837858012370595\\
59.92	0.00837858022610409\\
59.93	0.00837858032902704\\
59.94	0.00837858043247548\\
59.95	0.00837858053645007\\
59.96	0.00837858064095144\\
59.97	0.00837858074598019\\
59.98	0.00837858085153689\\
59.99	0.00837858095762208\\
60	0.00837858106423627\\
60.01	0.00837858117137996\\
60.02	0.00837858127905359\\
60.03	0.00837858138725758\\
60.04	0.00837858149599233\\
60.05	0.0083785816052582\\
60.06	0.00837858171505552\\
60.07	0.00837858182538459\\
60.08	0.00837858193624566\\
60.09	0.00837858204763898\\
60.1	0.00837858215956474\\
60.11	0.00837858227202311\\
60.12	0.00837858238501421\\
60.13	0.00837858249853815\\
60.14	0.00837858261259498\\
60.15	0.00837858272718473\\
60.16	0.00837858284230738\\
60.17	0.0083785829579629\\
60.18	0.0083785830741512\\
60.19	0.00837858319087216\\
60.2	0.00837858330812562\\
60.21	0.00837858342591139\\
60.22	0.00837858354422924\\
60.23	0.00837858366307888\\
60.24	0.00837858378246002\\
60.25	0.0083785839023723\\
60.26	0.00837858402281532\\
60.27	0.00837858414378866\\
60.28	0.00837858426529186\\
60.29	0.00837858438732438\\
60.3	0.00837858450988569\\
60.31	0.00837858463297518\\
60.32	0.00837858475659222\\
60.33	0.00837858488073612\\
60.34	0.00837858500540617\\
60.35	0.00837858513060159\\
60.36	0.00837858525632158\\
60.37	0.00837858538256528\\
60.38	0.00837858550933178\\
60.39	0.00837858563662016\\
60.4	0.0083785857644294\\
60.41	0.00837858589275849\\
60.42	0.00837858602160634\\
60.43	0.00837858615097181\\
60.44	0.00837858628085375\\
60.45	0.00837858641125092\\
60.46	0.00837858654216205\\
60.47	0.00837858667358584\\
60.48	0.00837858680552091\\
60.49	0.00837858693796585\\
60.5	0.0083785870709192\\
60.51	0.00837858720437946\\
60.52	0.00837858733834505\\
60.53	0.00837858747281438\\
60.54	0.00837858760778577\\
60.55	0.00837858774325754\\
60.56	0.0083785878792279\\
60.57	0.00837858801569506\\
60.58	0.00837858815265714\\
60.59	0.00837858829011225\\
60.6	0.00837858842805841\\
60.61	0.00837858856649361\\
60.62	0.00837858870541579\\
60.63	0.00837858884482281\\
60.64	0.00837858898471252\\
60.65	0.00837858912508269\\
60.66	0.00837858926593103\\
60.67	0.00837858940725523\\
60.68	0.0083785895490529\\
60.69	0.0083785896913216\\
60.7	0.00837858983405885\\
60.71	0.00837858997726211\\
60.72	0.00837859012092877\\
60.73	0.00837859026505619\\
60.74	0.00837859040964167\\
60.75	0.00837859055468246\\
60.76	0.00837859070017574\\
60.77	0.00837859084611866\\
60.78	0.00837859099250829\\
60.79	0.00837859113934166\\
60.8	0.00837859128661576\\
60.81	0.00837859143432751\\
60.82	0.00837859158247377\\
60.83	0.00837859173105136\\
60.84	0.00837859188005705\\
60.85	0.00837859202948755\\
60.86	0.0083785921793395\\
60.87	0.00837859232960953\\
60.88	0.00837859248029417\\
60.89	0.00837859263138994\\
60.9	0.00837859278289327\\
60.91	0.00837859293480056\\
60.92	0.00837859308710816\\
60.93	0.00837859323981236\\
60.94	0.0083785933929094\\
60.95	0.00837859354639547\\
60.96	0.00837859370026672\\
60.97	0.00837859385451923\\
60.98	0.00837859400914905\\
60.99	0.00837859416415217\\
61	0.00837859431952454\\
61.01	0.00837859447526206\\
61.02	0.00837859463136056\\
61.03	0.00837859478781588\\
61.04	0.00837859494462375\\
61.05	0.00837859510177989\\
61.06	0.00837859525927997\\
61.07	0.00837859541711962\\
61.08	0.00837859557529442\\
61.09	0.00837859573379991\\
61.1	0.00837859589263158\\
61.11	0.00837859605178491\\
61.12	0.00837859621125529\\
61.13	0.00837859637103813\\
61.14	0.00837859653112876\\
61.15	0.00837859669152249\\
61.16	0.00837859685221458\\
61.17	0.00837859701320029\\
61.18	0.00837859717447482\\
61.19	0.00837859733603333\\
61.2	0.00837859749787099\\
61.21	0.00837859765998289\\
61.22	0.00837859782236413\\
61.23	0.00837859798500977\\
61.24	0.00837859814791485\\
61.25	0.00837859831107439\\
61.26	0.00837859847448338\\
61.27	0.00837859863813679\\
61.28	0.00837859880202958\\
61.29	0.0083785989661567\\
61.3	0.00837859913051307\\
61.31	0.0083785992950936\\
61.32	0.00837859945989321\\
61.33	0.00837859962490678\\
61.34	0.00837859979012922\\
61.35	0.00837859995555541\\
61.36	0.00837860012118023\\
61.37	0.00837860028699858\\
61.38	0.00837860045300534\\
61.39	0.00837860061919542\\
61.4	0.00837860078556372\\
61.41	0.00837860095210515\\
61.42	0.00837860111881466\\
61.43	0.00837860128568719\\
61.44	0.0083786014527177\\
61.45	0.00837860161990119\\
61.46	0.00837860178723267\\
61.47	0.00837860195470719\\
61.48	0.00837860212231984\\
61.49	0.00837860229006571\\
61.5	0.00837860245793997\\
61.51	0.00837860262593781\\
61.52	0.00837860279405448\\
61.53	0.00837860296228525\\
61.54	0.00837860313062549\\
61.55	0.0083786032990706\\
61.56	0.00837860346761603\\
61.57	0.00837860363625733\\
61.58	0.00837860380499009\\
61.59	0.00837860397381\\
61.6	0.0083786041427128\\
61.61	0.00837860431169434\\
61.62	0.00837860448075055\\
61.63	0.00837860464987746\\
61.64	0.00837860481907117\\
61.65	0.00837860498832793\\
61.66	0.00837860515764406\\
61.67	0.00837860532701601\\
61.68	0.00837860549644035\\
61.69	0.00837860566591379\\
61.7	0.00837860583543314\\
61.71	0.00837860600499537\\
61.72	0.00837860617459758\\
61.73	0.00837860634423705\\
61.74	0.00837860651391117\\
61.75	0.00837860668361753\\
61.76	0.00837860685335388\\
61.77	0.00837860702311814\\
61.78	0.00837860719290842\\
61.79	0.00837860736272301\\
61.8	0.00837860753256043\\
61.81	0.00837860770241936\\
61.82	0.00837860787229872\\
61.83	0.00837860804219767\\
61.84	0.00837860821211556\\
61.85	0.008378608382052\\
61.86	0.00837860855200685\\
61.87	0.00837860872198022\\
61.88	0.00837860889197249\\
61.89	0.0083786090619843\\
61.9	0.00837860923201658\\
61.91	0.00837860940207029\\
61.92	0.00837860957214639\\
61.93	0.00837860974224586\\
61.94	0.00837860991236968\\
61.95	0.00837861008251885\\
61.96	0.00837861025269437\\
61.97	0.00837861042289728\\
61.98	0.00837861059312859\\
61.99	0.00837861076338935\\
62	0.00837861093368062\\
62.01	0.00837861110400346\\
62.02	0.00837861127435894\\
62.03	0.00837861144474816\\
62.04	0.00837861161517221\\
62.05	0.00837861178563219\\
62.06	0.00837861195612923\\
62.07	0.00837861212666446\\
62.08	0.00837861229723902\\
62.09	0.00837861246785406\\
62.1	0.00837861263851075\\
62.11	0.00837861280921025\\
62.12	0.00837861297995375\\
62.13	0.00837861315074246\\
62.14	0.00837861332157756\\
62.15	0.00837861349246027\\
62.16	0.00837861366339183\\
62.17	0.00837861383437347\\
62.18	0.00837861400540643\\
62.19	0.00837861417649197\\
62.2	0.00837861434763135\\
62.21	0.00837861451882585\\
62.22	0.00837861469007676\\
62.23	0.00837861486138537\\
62.24	0.00837861503275299\\
62.25	0.00837861520418093\\
62.26	0.00837861537567051\\
62.27	0.00837861554722306\\
62.28	0.00837861571883994\\
62.29	0.00837861589052249\\
62.3	0.00837861606227206\\
62.31	0.00837861623409004\\
62.32	0.00837861640597779\\
62.33	0.0083786165779367\\
62.34	0.00837861674996817\\
62.35	0.00837861692207359\\
62.36	0.00837861709425439\\
62.37	0.00837861726651198\\
62.38	0.00837861743884778\\
62.39	0.00837861761126323\\
62.4	0.00837861778375977\\
62.41	0.00837861795633885\\
62.42	0.00837861812900194\\
62.43	0.00837861830175048\\
62.44	0.00837861847458595\\
62.45	0.00837861864750983\\
62.46	0.00837861882052361\\
62.47	0.00837861899362876\\
62.48	0.0083786191668268\\
62.49	0.00837861934011922\\
62.5	0.00837861951350753\\
62.51	0.00837861968699324\\
62.52	0.00837861986057788\\
62.53	0.00837862003426296\\
62.54	0.00837862020805003\\
62.55	0.00837862038194061\\
62.56	0.00837862055593624\\
62.57	0.00837862073003846\\
62.58	0.00837862090424884\\
62.59	0.00837862107856891\\
62.6	0.00837862125300023\\
62.61	0.00837862142754436\\
62.62	0.00837862160220288\\
62.63	0.00837862177697733\\
62.64	0.0083786219518693\\
62.65	0.00837862212688036\\
62.66	0.00837862230201207\\
62.67	0.00837862247726602\\
62.68	0.00837862265264378\\
62.69	0.00837862282814694\\
62.7	0.00837862300377708\\
62.71	0.00837862317953579\\
62.72	0.00837862335542464\\
62.73	0.00837862353144522\\
62.74	0.00837862370759913\\
62.75	0.00837862388388795\\
62.76	0.00837862406031326\\
62.77	0.00837862423687665\\
62.78	0.00837862441357971\\
62.79	0.00837862459042402\\
62.8	0.00837862476741118\\
62.81	0.00837862494454276\\
62.82	0.00837862512182034\\
62.83	0.00837862529924552\\
62.84	0.00837862547681987\\
62.85	0.00837862565454496\\
62.86	0.00837862583242238\\
62.87	0.00837862601045369\\
62.88	0.00837862618864046\\
62.89	0.00837862636698427\\
62.9	0.00837862654548668\\
62.91	0.00837862672414924\\
62.92	0.00837862690297352\\
62.93	0.00837862708196106\\
62.94	0.00837862726111341\\
62.95	0.00837862744043213\\
62.96	0.00837862761991874\\
62.97	0.00837862779957479\\
62.98	0.0083786279794018\\
62.99	0.0083786281594013\\
63	0.00837862833957481\\
63.01	0.00837862851992384\\
63.02	0.00837862870044989\\
63.03	0.00837862888115447\\
63.04	0.00837862906203908\\
63.05	0.00837862924310521\\
63.06	0.00837862942435433\\
63.07	0.00837862960578792\\
63.08	0.00837862978740745\\
63.09	0.00837862996921439\\
63.1	0.00837863015121018\\
63.11	0.00837863033339628\\
63.12	0.00837863051577412\\
63.13	0.00837863069834515\\
63.14	0.00837863088111077\\
63.15	0.00837863106407242\\
63.16	0.00837863124723149\\
63.17	0.00837863143058941\\
63.18	0.00837863161414754\\
63.19	0.0083786317979073\\
63.2	0.00837863198187004\\
63.21	0.00837863216603715\\
63.22	0.00837863235040999\\
63.23	0.00837863253498991\\
63.24	0.00837863271977826\\
63.25	0.00837863290477638\\
63.26	0.00837863308998561\\
63.27	0.00837863327540727\\
63.28	0.00837863346104267\\
63.29	0.00837863364689314\\
63.3	0.00837863383295997\\
63.31	0.00837863401924446\\
63.32	0.0083786342057479\\
63.33	0.00837863439247159\\
63.34	0.00837863457941679\\
63.35	0.00837863476658478\\
63.36	0.00837863495397684\\
63.37	0.00837863514159422\\
63.38	0.00837863532943819\\
63.39	0.00837863551751\\
63.4	0.00837863570581091\\
63.41	0.00837863589434216\\
63.42	0.008378636083105\\
63.43	0.00837863627210068\\
63.44	0.00837863646133044\\
63.45	0.00837863665079552\\
63.46	0.00837863684049718\\
63.47	0.00837863703043665\\
63.48	0.00837863722061518\\
63.49	0.00837863741103403\\
63.5	0.00837863760169444\\
63.51	0.00837863779259768\\
63.52	0.008378637983745\\
63.53	0.00837863817513768\\
63.54	0.008378638366777\\
63.55	0.00837863855866425\\
63.56	0.00837863875080071\\
63.57	0.0083786389431877\\
63.58	0.00837863913582653\\
63.59	0.00837863932871852\\
63.6	0.00837863952186502\\
63.61	0.00837863971526737\\
63.62	0.00837863990892692\\
63.63	0.00837864010284504\\
63.64	0.0083786402970231\\
63.65	0.00837864049146249\\
63.66	0.00837864068616461\\
63.67	0.00837864088113085\\
63.68	0.00837864107636264\\
63.69	0.0083786412718614\\
63.7	0.00837864146762858\\
63.71	0.00837864166366561\\
63.72	0.00837864185997395\\
63.73	0.00837864205655508\\
63.74	0.00837864225341047\\
63.75	0.00837864245054162\\
63.76	0.00837864264795002\\
63.77	0.00837864284563719\\
63.78	0.00837864304360465\\
63.79	0.00837864324185393\\
63.8	0.00837864344038658\\
63.81	0.00837864363920416\\
63.82	0.00837864383830824\\
63.83	0.00837864403770038\\
63.84	0.0083786442373822\\
63.85	0.00837864443735527\\
63.86	0.00837864463762123\\
63.87	0.0083786448381817\\
63.88	0.00837864503903831\\
63.89	0.00837864524019271\\
63.9	0.00837864544164656\\
63.91	0.00837864564340154\\
63.92	0.00837864584545932\\
63.93	0.00837864604782162\\
63.94	0.00837864625049013\\
63.95	0.00837864645346658\\
63.96	0.00837864665675269\\
63.97	0.00837864686035022\\
63.98	0.00837864706426093\\
63.99	0.00837864726848658\\
64	0.00837864747302896\\
64.01	0.00837864767788987\\
64.02	0.0083786478830711\\
64.03	0.00837864808857449\\
64.04	0.00837864829440187\\
64.05	0.00837864850055509\\
64.06	0.008378648707036\\
64.07	0.00837864891384648\\
64.08	0.00837864912098842\\
64.09	0.00837864932846371\\
64.1	0.00837864953627427\\
64.11	0.00837864974442203\\
64.12	0.00837864995290893\\
64.13	0.00837865016173691\\
64.14	0.00837865037090795\\
64.15	0.00837865058042403\\
64.16	0.00837865079028714\\
64.17	0.00837865100049929\\
64.18	0.0083786512110625\\
64.19	0.00837865142197882\\
64.2	0.00837865163325028\\
64.21	0.00837865184487896\\
64.22	0.00837865205686693\\
64.23	0.00837865226921629\\
64.24	0.00837865248192914\\
64.25	0.00837865269500761\\
64.26	0.00837865290845382\\
64.27	0.00837865312226995\\
64.28	0.00837865333645814\\
64.29	0.00837865355102058\\
64.3	0.00837865376595946\\
64.31	0.008378653981277\\
64.32	0.00837865419697542\\
64.33	0.00837865441305696\\
64.34	0.00837865462952387\\
64.35	0.00837865484637843\\
64.36	0.00837865506362292\\
64.37	0.00837865528125964\\
64.38	0.00837865549929091\\
64.39	0.00837865571771906\\
64.4	0.00837865593654644\\
64.41	0.00837865615577541\\
64.42	0.00837865637540834\\
64.43	0.00837865659544764\\
64.44	0.00837865681589572\\
64.45	0.00837865703675499\\
64.46	0.00837865725802791\\
64.47	0.00837865747971693\\
64.48	0.00837865770182452\\
64.49	0.00837865792435318\\
64.5	0.00837865814730541\\
64.51	0.00837865837068374\\
64.52	0.0083786585944907\\
64.53	0.00837865881872886\\
64.54	0.00837865904340077\\
64.55	0.00837865926850904\\
64.56	0.00837865949405628\\
64.57	0.00837865972004509\\
64.58	0.00837865994647813\\
64.59	0.00837866017335804\\
64.6	0.0083786604006875\\
64.61	0.00837866062846921\\
64.62	0.00837866085670587\\
64.63	0.0083786610854002\\
64.64	0.00837866131455494\\
64.65	0.00837866154417287\\
64.66	0.00837866177425674\\
64.67	0.00837866200480936\\
64.68	0.00837866223583354\\
64.69	0.00837866246733211\\
64.7	0.0083786626993079\\
64.71	0.0083786629317638\\
64.72	0.00837866316470267\\
64.73	0.00837866339812743\\
64.74	0.00837866363204098\\
64.75	0.00837866386644626\\
64.76	0.00837866410134623\\
64.77	0.00837866433674386\\
64.78	0.00837866457264214\\
64.79	0.00837866480904407\\
64.8	0.00837866504595268\\
64.81	0.00837866528337102\\
64.82	0.00837866552130214\\
64.83	0.00837866575974914\\
64.84	0.0083786659987151\\
64.85	0.00837866623820316\\
64.86	0.00837866647821643\\
64.87	0.00837866671875808\\
64.88	0.00837866695983129\\
64.89	0.00837866720143923\\
64.9	0.00837866744358513\\
64.91	0.00837866768627222\\
64.92	0.00837866792950374\\
64.93	0.00837866817328295\\
64.94	0.00837866841761316\\
64.95	0.00837866866249765\\
64.96	0.00837866890793976\\
64.97	0.00837866915394283\\
64.98	0.00837866940051022\\
64.99	0.00837866964764531\\
65	0.0083786698953515\\
65.01	0.00837867014363222\\
65.02	0.00837867039249089\\
65.03	0.00837867064193099\\
65.04	0.00837867089195598\\
65.05	0.00837867114256936\\
65.06	0.00837867139377464\\
65.07	0.00837867164557537\\
65.08	0.0083786718979751\\
65.09	0.0083786721509774\\
65.1	0.00837867240458586\\
65.11	0.0083786726588041\\
65.12	0.00837867291363575\\
65.13	0.00837867316908446\\
65.14	0.0083786734251539\\
65.15	0.00837867368184777\\
65.16	0.00837867393916976\\
65.17	0.00837867419712362\\
65.18	0.00837867445571308\\
65.19	0.00837867471494193\\
65.2	0.00837867497481395\\
65.21	0.00837867523533294\\
65.22	0.00837867549650273\\
65.23	0.00837867575832717\\
65.24	0.00837867602081013\\
65.25	0.00837867628395549\\
65.26	0.00837867654776716\\
65.27	0.00837867681224906\\
65.28	0.00837867707740513\\
65.29	0.00837867734323935\\
65.3	0.00837867760975569\\
65.31	0.00837867787695815\\
65.32	0.00837867814485077\\
65.33	0.00837867841343758\\
65.34	0.00837867868272264\\
65.35	0.00837867895271003\\
65.36	0.00837867922340386\\
65.37	0.00837867949480824\\
65.38	0.0083786797669273\\
65.39	0.00837868003976522\\
65.4	0.00837868031332617\\
65.41	0.00837868058761433\\
65.42	0.00837868086263394\\
65.43	0.00837868113838921\\
65.44	0.00837868141488441\\
65.45	0.00837868169212381\\
65.46	0.0083786819701117\\
65.47	0.00837868224885239\\
65.48	0.00837868252835021\\
65.49	0.00837868280860951\\
65.5	0.00837868308963465\\
65.51	0.00837868337143001\\
65.52	0.00837868365400002\\
65.53	0.00837868393734907\\
65.54	0.00837868422148163\\
65.55	0.00837868450640214\\
65.56	0.00837868479211508\\
65.57	0.00837868507862494\\
65.58	0.00837868536593625\\
65.59	0.00837868565405354\\
65.6	0.00837868594298134\\
65.61	0.00837868623272423\\
65.62	0.0083786865232868\\
65.63	0.00837868681467365\\
65.64	0.00837868710688939\\
65.65	0.00837868739993866\\
65.66	0.00837868769382612\\
65.67	0.00837868798855644\\
65.68	0.0083786882841343\\
65.69	0.00837868858056442\\
65.7	0.00837868887785151\\
65.71	0.00837868917600032\\
65.72	0.00837868947501559\\
65.73	0.0083786897749021\\
65.74	0.00837869007566464\\
65.75	0.008378690377308\\
65.76	0.00837869067983702\\
65.77	0.00837869098325651\\
65.78	0.00837869128757134\\
65.79	0.00837869159278637\\
65.8	0.00837869189890647\\
65.81	0.00837869220593655\\
65.82	0.00837869251388151\\
65.83	0.00837869282274628\\
65.84	0.0083786931325358\\
65.85	0.00837869344325502\\
65.86	0.00837869375490891\\
65.87	0.00837869406750244\\
65.88	0.00837869438104062\\
65.89	0.00837869469552846\\
65.9	0.00837869501097096\\
65.91	0.00837869532737317\\
65.92	0.00837869564474013\\
65.93	0.0083786959630769\\
65.94	0.00837869628238856\\
65.95	0.00837869660268018\\
65.96	0.00837869692395686\\
65.97	0.00837869724622371\\
65.98	0.00837869756948583\\
65.99	0.00837869789374837\\
66	0.00837869821901645\\
66.01	0.00837869854529523\\
66.02	0.00837869887258986\\
66.03	0.00837869920090552\\
66.04	0.00837869953024738\\
66.05	0.00837869986062062\\
66.06	0.00837870019203045\\
66.07	0.00837870052448206\\
66.08	0.00837870085798067\\
66.09	0.00837870119253149\\
66.1	0.00837870152813976\\
66.11	0.00837870186481071\\
66.12	0.00837870220254958\\
66.13	0.00837870254136161\\
66.14	0.00837870288125206\\
66.15	0.00837870322222619\\
66.16	0.00837870356428926\\
66.17	0.00837870390744654\\
66.18	0.0083787042517033\\
66.19	0.00837870459706483\\
66.2	0.0083787049435364\\
66.21	0.0083787052911233\\
66.22	0.0083787056398308\\
66.23	0.00837870598966422\\
66.24	0.00837870634062883\\
66.25	0.00837870669272992\\
66.26	0.0083787070459728\\
66.27	0.00837870740036275\\
66.28	0.00837870775590508\\
66.29	0.00837870811260507\\
66.3	0.00837870847046802\\
66.31	0.00837870882949923\\
66.32	0.00837870918970397\\
66.33	0.00837870955108755\\
66.34	0.00837870991365525\\
66.35	0.00837871027741234\\
66.36	0.00837871064236411\\
66.37	0.00837871100851584\\
66.38	0.00837871137587278\\
66.39	0.00837871174444021\\
66.4	0.00837871211422338\\
66.41	0.00837871248522754\\
66.42	0.00837871285745794\\
66.43	0.00837871323091981\\
66.44	0.00837871360561838\\
66.45	0.00837871398155887\\
66.46	0.00837871435874649\\
66.47	0.00837871473718644\\
66.48	0.0083787151168839\\
66.49	0.00837871549784405\\
66.5	0.00837871588007206\\
66.51	0.00837871626357308\\
66.52	0.00837871664835225\\
66.53	0.00837871703441468\\
66.54	0.0083787174217655\\
66.55	0.00837871781040979\\
66.56	0.00837871820035264\\
66.57	0.0083787185915991\\
66.58	0.00837871898415422\\
66.59	0.00837871937802302\\
66.6	0.00837871977321051\\
66.61	0.00837872016972168\\
66.62	0.00837872056756149\\
66.63	0.00837872096673488\\
66.64	0.00837872136724679\\
66.65	0.0083787217691021\\
66.66	0.0083787221723057\\
66.67	0.00837872257686244\\
66.68	0.00837872298277714\\
66.69	0.0083787233900546\\
66.7	0.0083787237986996\\
66.71	0.00837872420871688\\
66.72	0.00837872462011117\\
66.73	0.00837872503288714\\
66.74	0.00837872544704945\\
66.75	0.00837872586260274\\
66.76	0.00837872627955159\\
66.77	0.00837872669790058\\
66.78	0.00837872711765421\\
66.79	0.00837872753881699\\
66.8	0.00837872796139338\\
66.81	0.00837872838538778\\
66.82	0.0083787288108046\\
66.83	0.00837872923764817\\
66.84	0.00837872966592279\\
66.85	0.00837873009563273\\
66.86	0.00837873052678221\\
66.87	0.00837873095937542\\
66.88	0.00837873139341649\\
66.89	0.0083787318289095\\
66.9	0.00837873226585852\\
66.91	0.00837873270426754\\
66.92	0.0083787331441405\\
66.93	0.00837873358548133\\
66.94	0.00837873402829386\\
66.95	0.00837873447258192\\
66.96	0.00837873491834923\\
66.97	0.00837873536559952\\
66.98	0.00837873581433642\\
66.99	0.00837873626456352\\
67	0.00837873671628435\\
67.01	0.0083787371695024\\
67.02	0.00837873762422109\\
67.03	0.00837873808044376\\
67.04	0.00837873853817373\\
67.05	0.00837873899741422\\
67.06	0.00837873945816841\\
67.07	0.00837873992043941\\
67.08	0.00837874038423027\\
67.09	0.00837874084954396\\
67.1	0.00837874131638338\\
67.11	0.00837874178475139\\
67.12	0.00837874225465076\\
67.13	0.00837874272608418\\
67.14	0.00837874319905427\\
67.15	0.00837874367356361\\
67.16	0.00837874414961465\\
67.17	0.00837874462720981\\
67.18	0.00837874510635141\\
67.19	0.00837874558704169\\
67.2	0.00837874606928282\\
67.21	0.00837874655307689\\
67.22	0.0083787470384259\\
67.23	0.00837874752533177\\
67.24	0.00837874801379633\\
67.25	0.00837874850382132\\
67.26	0.00837874899540841\\
67.27	0.00837874948855918\\
67.28	0.00837874998327509\\
67.29	0.00837875047955754\\
67.3	0.00837875097740783\\
67.31	0.00837875147682716\\
67.32	0.00837875197781663\\
67.33	0.00837875248037725\\
67.34	0.00837875298450995\\
67.35	0.00837875349021552\\
67.36	0.00837875399749469\\
67.37	0.00837875450634806\\
67.38	0.00837875501677615\\
67.39	0.00837875552877935\\
67.4	0.00837875604235796\\
67.41	0.00837875655751218\\
67.42	0.00837875707424207\\
67.43	0.00837875759254763\\
67.44	0.0083787581124287\\
67.45	0.00837875863388504\\
67.46	0.00837875915691628\\
67.47	0.00837875968152194\\
67.48	0.00837876020770143\\
67.49	0.00837876073545403\\
67.5	0.00837876126477891\\
67.51	0.00837876179567512\\
67.52	0.00837876232814158\\
67.53	0.00837876286217711\\
67.54	0.00837876339778037\\
67.55	0.00837876393494993\\
67.56	0.00837876447368422\\
67.57	0.00837876501398154\\
67.58	0.00837876555584006\\
67.59	0.00837876609925782\\
67.6	0.00837876664423275\\
67.61	0.00837876719076262\\
67.62	0.00837876773884508\\
67.63	0.00837876828847764\\
67.64	0.00837876883965768\\
67.65	0.00837876939238245\\
67.66	0.00837876994664905\\
67.67	0.00837877050245444\\
67.68	0.00837877105979546\\
67.69	0.00837877161866878\\
67.7	0.00837877217907095\\
67.71	0.00837877274099838\\
67.72	0.00837877330444731\\
67.73	0.00837877386941388\\
67.74	0.00837877443589403\\
67.75	0.0083787750038836\\
67.76	0.00837877557337827\\
67.77	0.00837877614437355\\
67.78	0.00837877671686483\\
67.79	0.00837877729084733\\
67.8	0.00837877786631615\\
67.81	0.0083787784432662\\
67.82	0.00837877902169227\\
67.83	0.00837877960158898\\
67.84	0.0083787801829508\\
67.85	0.00837878076577207\\
67.86	0.00837878135004693\\
67.87	0.00837878193576942\\
67.88	0.00837878252293339\\
67.89	0.00837878311153254\\
67.9	0.00837878370156043\\
67.91	0.00837878429301045\\
67.92	0.00837878488587585\\
67.93	0.0083787854801497\\
67.94	0.00837878607582496\\
67.95	0.00837878667289438\\
67.96	0.00837878727135059\\
67.97	0.00837878787118605\\
67.98	0.00837878847239309\\
67.99	0.00837878907496385\\
68	0.00837878967889033\\
68.01	0.0083787902841644\\
68.02	0.00837879089077773\\
68.03	0.00837879149872189\\
68.04	0.00837879210798825\\
68.05	0.00837879271856806\\
68.06	0.00837879333045242\\
68.07	0.00837879394363225\\
68.08	0.00837879455809835\\
68.09	0.00837879517384137\\
68.1	0.00837879579085181\\
68.11	0.00837879640912001\\
68.12	0.00837879702863619\\
68.13	0.00837879764939042\\
68.14	0.00837879827137261\\
68.15	0.00837879889457257\\
68.16	0.00837879951897994\\
68.17	0.00837880014458423\\
68.18	0.00837880077137483\\
68.19	0.00837880139934099\\
68.2	0.00837880202847183\\
68.21	0.00837880265875635\\
68.22	0.00837880329018342\\
68.23	0.0083788039227418\\
68.24	0.00837880455642011\\
68.25	0.00837880519120688\\
68.26	0.0083788058270905\\
68.27	0.00837880646405929\\
68.28	0.00837880710210142\\
68.29	0.00837880774120499\\
68.3	0.00837880838135798\\
68.31	0.0083788090225483\\
68.32	0.00837880966476376\\
68.33	0.00837881030799207\\
68.34	0.00837881095222087\\
68.35	0.00837881159743772\\
68.36	0.00837881224363012\\
68.37	0.00837881289078548\\
68.38	0.00837881353889117\\
68.39	0.0083788141879345\\
68.4	0.00837881483790271\\
68.41	0.00837881548878302\\
68.42	0.0083788161405626\\
68.43	0.00837881679322858\\
68.44	0.00837881744676808\\
68.45	0.00837881810116818\\
68.46	0.00837881875641597\\
68.47	0.00837881941249853\\
68.48	0.00837882006940292\\
68.49	0.00837882072711624\\
68.5	0.0083788213856256\\
68.51	0.00837882204491812\\
68.52	0.00837882270498097\\
68.53	0.00837882336580137\\
68.54	0.00837882402736658\\
68.55	0.00837882468966394\\
68.56	0.00837882535268083\\
68.57	0.00837882601640475\\
68.58	0.00837882668082328\\
68.59	0.00837882734592409\\
68.6	0.00837882801169498\\
68.61	0.00837882867812387\\
68.62	0.00837882934519884\\
68.63	0.00837883001290808\\
68.64	0.00837883068123999\\
68.65	0.0083788313501831\\
68.66	0.00837883201972616\\
68.67	0.00837883268985813\\
68.68	0.00837883336056816\\
68.69	0.00837883403184564\\
68.7	0.00837883470368023\\
68.71	0.00837883537606183\\
68.72	0.00837883604898062\\
68.73	0.00837883672242708\\
68.74	0.00837883739639201\\
68.75	0.00837883807086653\\
68.76	0.00837883874584211\\
68.77	0.00837883942131057\\
68.78	0.00837884009726414\\
68.79	0.00837884077369544\\
68.8	0.00837884145059752\\
68.81	0.00837884212796386\\
68.82	0.00837884280578841\\
68.83	0.00837884348406563\\
68.84	0.00837884416279046\\
68.85	0.00837884484195837\\
68.86	0.00837884552156541\\
68.87	0.00837884620160819\\
68.88	0.00837884688208393\\
68.89	0.00837884756299048\\
68.9	0.00837884824432634\\
68.91	0.00837884892609069\\
68.92	0.00837884960828343\\
68.93	0.00837885029090518\\
68.94	0.00837885097395669\\
68.95	0.0083788516574387\\
68.96	0.00837885234135197\\
68.97	0.00837885302569724\\
68.98	0.00837885371047526\\
68.99	0.00837885439568679\\
69	0.00837885508133259\\
69.01	0.00837885576741342\\
69.02	0.00837885645393004\\
69.03	0.00837885714088321\\
69.04	0.0083788578282737\\
69.05	0.00837885851610229\\
69.06	0.00837885920436973\\
69.07	0.00837885989307682\\
69.08	0.00837886058222431\\
69.09	0.00837886127181301\\
69.1	0.00837886196184367\\
69.11	0.0083788626523171\\
69.12	0.00837886334323408\\
69.13	0.00837886403459539\\
69.14	0.00837886472640182\\
69.15	0.00837886541865418\\
69.16	0.00837886611135326\\
69.17	0.00837886680449986\\
69.18	0.00837886749809478\\
69.19	0.00837886819213883\\
69.2	0.0083788688866328\\
69.21	0.00837886958157752\\
69.22	0.00837887027697379\\
69.23	0.00837887097282244\\
69.24	0.00837887166912426\\
69.25	0.0083788723658801\\
69.26	0.00837887306309076\\
69.27	0.00837887376075708\\
69.28	0.00837887445887988\\
69.29	0.00837887515746\\
69.3	0.00837887585649826\\
69.31	0.0083788765559955\\
69.32	0.00837887725595256\\
69.33	0.00837887795637028\\
69.34	0.00837887865724951\\
69.35	0.00837887935859109\\
69.36	0.00837888006039587\\
69.37	0.00837888076266471\\
69.38	0.00837888146539845\\
69.39	0.00837888216859796\\
69.4	0.00837888287226409\\
69.41	0.00837888357639772\\
69.42	0.00837888428099969\\
69.43	0.00837888498607089\\
69.44	0.00837888569161219\\
69.45	0.00837888639762445\\
69.46	0.00837888710410855\\
69.47	0.00837888781106538\\
69.48	0.00837888851849582\\
69.49	0.00837888922640075\\
69.5	0.00837888993478107\\
69.51	0.00837889064363765\\
69.52	0.0083788913529714\\
69.53	0.00837889206278321\\
69.54	0.00837889277307398\\
69.55	0.00837889348384462\\
69.56	0.00837889419509603\\
69.57	0.00837889490682912\\
69.58	0.0083788956190448\\
69.59	0.00837889633174398\\
69.6	0.00837889704492759\\
69.61	0.00837889775859654\\
69.62	0.00837889847275176\\
69.63	0.00837889918739416\\
69.64	0.00837889990252469\\
69.65	0.00837890061814427\\
69.66	0.00837890133425385\\
69.67	0.00837890205085435\\
69.68	0.00837890276794672\\
69.69	0.0083789034855319\\
69.7	0.00837890420361085\\
69.71	0.00837890492218451\\
69.72	0.00837890564125384\\
69.73	0.00837890636081979\\
69.74	0.00837890708088332\\
69.75	0.00837890780144541\\
69.76	0.00837890852250701\\
69.77	0.00837890924406909\\
69.78	0.00837890996613263\\
69.79	0.0083789106886986\\
69.8	0.00837891141176799\\
69.81	0.00837891213534178\\
69.82	0.00837891285942095\\
69.83	0.00837891358400649\\
69.84	0.00837891430909939\\
69.85	0.00837891503470065\\
69.86	0.00837891576081128\\
69.87	0.00837891648743227\\
69.88	0.00837891721456462\\
69.89	0.00837891794220936\\
69.9	0.00837891867036749\\
69.91	0.00837891939904003\\
69.92	0.00837892012822799\\
69.93	0.0083789208579324\\
69.94	0.0083789215881543\\
69.95	0.0083789223188947\\
69.96	0.00837892305015464\\
69.97	0.00837892378193517\\
69.98	0.00837892451423731\\
69.99	0.00837892524706212\\
70	0.00837892598041064\\
70.01	0.00837892671428393\\
70.02	0.00837892744868304\\
70.03	0.00837892818360902\\
70.04	0.00837892891906295\\
70.05	0.00837892965504589\\
70.06	0.00837893039155891\\
70.07	0.00837893112860308\\
70.08	0.00837893186617948\\
70.09	0.00837893260428919\\
70.1	0.0083789333429333\\
70.11	0.0083789340821129\\
70.12	0.00837893482182907\\
70.13	0.00837893556208292\\
70.14	0.00837893630287555\\
70.15	0.00837893704420806\\
70.16	0.00837893778608155\\
70.17	0.00837893852849715\\
70.18	0.00837893927145597\\
70.19	0.00837894001495913\\
70.2	0.00837894075900776\\
70.21	0.00837894150360297\\
70.22	0.00837894224874592\\
70.23	0.00837894299443772\\
70.24	0.00837894374067953\\
70.25	0.00837894448747248\\
70.26	0.00837894523481774\\
70.27	0.00837894598271644\\
70.28	0.00837894673116975\\
70.29	0.00837894748017884\\
70.3	0.00837894822974486\\
70.31	0.00837894897986899\\
70.32	0.0083789497305524\\
70.33	0.00837895048179627\\
70.34	0.00837895123360179\\
70.35	0.00837895198597014\\
70.36	0.00837895273890252\\
70.37	0.00837895349240012\\
70.38	0.00837895424646414\\
70.39	0.00837895500109579\\
70.4	0.00837895575629628\\
70.41	0.00837895651206683\\
70.42	0.00837895726840866\\
70.43	0.00837895802532298\\
70.44	0.00837895878281103\\
70.45	0.00837895954087404\\
70.46	0.00837896029951326\\
70.47	0.00837896105872991\\
70.48	0.00837896181852526\\
70.49	0.00837896257890055\\
70.5	0.00837896333985703\\
70.51	0.00837896410139598\\
70.52	0.00837896486351867\\
70.53	0.00837896562622635\\
70.54	0.00837896638952031\\
70.55	0.00837896715340183\\
70.56	0.0083789679178722\\
70.57	0.00837896868293271\\
70.58	0.00837896944858465\\
70.59	0.00837897021482933\\
70.6	0.00837897098166805\\
70.61	0.00837897174910214\\
70.62	0.00837897251713289\\
70.63	0.00837897328576164\\
70.64	0.00837897405498972\\
70.65	0.00837897482481845\\
70.66	0.00837897559524918\\
70.67	0.00837897636628325\\
70.68	0.008378977137922\\
70.69	0.0083789779101668\\
70.7	0.00837897868301899\\
70.71	0.00837897945647995\\
70.72	0.00837898023055105\\
70.73	0.00837898100523366\\
70.74	0.00837898178052917\\
70.75	0.00837898255643895\\
70.76	0.0083789833329644\\
70.77	0.00837898411010693\\
70.78	0.00837898488786792\\
70.79	0.0083789856662488\\
70.8	0.00837898644525098\\
70.81	0.00837898722487587\\
70.82	0.0083789880051249\\
70.83	0.00837898878599951\\
70.84	0.00837898956750113\\
70.85	0.00837899034963121\\
70.86	0.00837899113239118\\
70.87	0.00837899191578252\\
70.88	0.00837899269980667\\
70.89	0.00837899348446511\\
70.9	0.00837899426975931\\
70.91	0.00837899505569073\\
70.92	0.00837899584226088\\
70.93	0.00837899662947124\\
70.94	0.00837899741732329\\
70.95	0.00837899820581855\\
70.96	0.00837899899495853\\
70.97	0.00837899978474472\\
70.98	0.00837900057517866\\
70.99	0.00837900136626188\\
71	0.00837900215799589\\
71.01	0.00837900295038224\\
71.02	0.00837900374342247\\
71.03	0.00837900453711813\\
71.04	0.00837900533147077\\
71.05	0.00837900612648197\\
71.06	0.00837900692215327\\
71.07	0.00837900771848626\\
71.08	0.00837900851548252\\
71.09	0.00837900931314363\\
71.1	0.00837901011147118\\
71.11	0.00837901091046677\\
71.12	0.008379011710132\\
71.13	0.00837901251046849\\
71.14	0.00837901331147785\\
71.15	0.00837901411316169\\
71.16	0.00837901491552166\\
71.17	0.00837901571855937\\
71.18	0.00837901652227648\\
71.19	0.00837901732667463\\
71.2	0.00837901813175546\\
71.21	0.00837901893752065\\
71.22	0.00837901974397185\\
71.23	0.00837902055111073\\
71.24	0.00837902135893897\\
71.25	0.00837902216745826\\
71.26	0.00837902297667027\\
71.27	0.00837902378657671\\
71.28	0.00837902459717928\\
71.29	0.00837902540847968\\
71.3	0.00837902622047964\\
71.31	0.00837902703318085\\
71.32	0.00837902784658506\\
71.33	0.00837902866069399\\
71.34	0.00837902947550938\\
71.35	0.00837903029103298\\
71.36	0.00837903110726653\\
71.37	0.00837903192421178\\
71.38	0.00837903274187051\\
71.39	0.00837903356024448\\
71.4	0.00837903437933545\\
71.41	0.00837903519914522\\
71.42	0.00837903601967556\\
71.43	0.00837903684092827\\
71.44	0.00837903766290514\\
71.45	0.00837903848560798\\
71.46	0.0083790393090386\\
71.47	0.00837904013319881\\
71.48	0.00837904095809044\\
71.49	0.0083790417837153\\
71.5	0.00837904261007524\\
71.51	0.00837904343717208\\
71.52	0.00837904426500768\\
71.53	0.00837904509358389\\
71.54	0.00837904592290256\\
71.55	0.00837904675296555\\
71.56	0.00837904758377472\\
71.57	0.00837904841533196\\
71.58	0.00837904924763914\\
71.59	0.00837905008069815\\
71.6	0.00837905091451086\\
71.61	0.00837905174907919\\
71.62	0.00837905258440502\\
71.63	0.00837905342049027\\
71.64	0.00837905425733685\\
71.65	0.00837905509494666\\
71.66	0.00837905593332165\\
71.67	0.00837905677246372\\
71.68	0.00837905761237482\\
71.69	0.00837905845305689\\
71.7	0.00837905929451186\\
71.71	0.00837906013674169\\
71.72	0.00837906097974832\\
71.73	0.00837906182353372\\
71.74	0.00837906266809985\\
71.75	0.00837906351344868\\
71.76	0.00837906435958218\\
71.77	0.00837906520650233\\
71.78	0.00837906605421111\\
71.79	0.0083790669027105\\
71.8	0.00837906775200251\\
71.81	0.00837906860208913\\
71.82	0.00837906945297236\\
71.83	0.0083790703046542\\
71.84	0.00837907115713667\\
71.85	0.00837907201042178\\
71.86	0.00837907286451154\\
71.87	0.00837907371940799\\
71.88	0.00837907457511314\\
71.89	0.00837907543162904\\
71.9	0.0083790762889577\\
71.91	0.00837907714710119\\
71.92	0.00837907800606152\\
71.93	0.00837907886584076\\
71.94	0.00837907972644095\\
71.95	0.00837908058786415\\
71.96	0.00837908145011241\\
71.97	0.00837908231318779\\
71.98	0.00837908317709236\\
71.99	0.00837908404182818\\
72	0.00837908490739733\\
72.01	0.00837908577380187\\
72.02	0.00837908664104389\\
72.03	0.00837908750912546\\
72.04	0.00837908837804866\\
72.05	0.00837908924781557\\
72.06	0.00837909011842829\\
72.07	0.0083790909898889\\
72.08	0.0083790918621995\\
72.09	0.00837909273536216\\
72.1	0.008379093609379\\
72.11	0.0083790944842521\\
72.12	0.00837909535998357\\
72.13	0.0083790962365755\\
72.14	0.00837909711403\\
72.15	0.00837909799234917\\
72.16	0.00837909887153511\\
72.17	0.00837909975158993\\
72.18	0.00837910063251573\\
72.19	0.00837910151431463\\
72.2	0.00837910239698872\\
72.21	0.00837910328054012\\
72.22	0.00837910416497093\\
72.23	0.00837910505028327\\
72.24	0.00837910593647923\\
72.25	0.00837910682356094\\
72.26	0.00837910771153049\\
72.27	0.00837910860038999\\
72.28	0.00837910949014155\\
72.29	0.00837911038078728\\
72.3	0.00837911127232928\\
72.31	0.00837911216476965\\
72.32	0.00837911305811049\\
72.33	0.00837911395235391\\
72.34	0.008379114847502\\
72.35	0.00837911574355686\\
72.36	0.00837911664052059\\
72.37	0.00837911753839526\\
72.38	0.00837911843718298\\
72.39	0.00837911933688583\\
72.4	0.00837912023750589\\
72.41	0.00837912113904525\\
72.42	0.00837912204150597\\
72.43	0.00837912294489013\\
72.44	0.0083791238491998\\
72.45	0.00837912475443704\\
72.46	0.00837912566060391\\
72.47	0.00837912656770247\\
72.48	0.00837912747573476\\
72.49	0.00837912838470284\\
72.5	0.00837912929460873\\
72.51	0.00837913020545448\\
72.52	0.00837913111724211\\
72.53	0.00837913202997364\\
72.54	0.00837913294365108\\
72.55	0.00837913385827645\\
72.56	0.00837913477385173\\
72.57	0.00837913569037894\\
72.58	0.00837913660786004\\
72.59	0.00837913752629702\\
72.6	0.00837913844569184\\
72.61	0.00837913936604647\\
72.62	0.00837914028736285\\
72.63	0.00837914120964293\\
72.64	0.00837914213288863\\
72.65	0.00837914305710189\\
72.66	0.0083791439822846\\
72.67	0.00837914490843868\\
72.68	0.008379145835566\\
72.69	0.00837914676366845\\
72.7	0.0083791476927479\\
72.71	0.00837914862280619\\
72.72	0.00837914955384517\\
72.73	0.00837915048586666\\
72.74	0.00837915141887249\\
72.75	0.00837915235286445\\
72.76	0.00837915328784433\\
72.77	0.0083791542238139\\
72.78	0.00837915516077492\\
72.79	0.00837915609872913\\
72.8	0.00837915703767826\\
72.81	0.00837915797762401\\
72.82	0.00837915891856809\\
72.83	0.00837915986051216\\
72.84	0.00837916080345789\\
72.85	0.00837916174740691\\
72.86	0.00837916269236084\\
72.87	0.0083791636383213\\
72.88	0.00837916458528986\\
72.89	0.00837916553326809\\
72.9	0.00837916648225753\\
72.91	0.00837916743225969\\
72.92	0.00837916838327609\\
72.93	0.00837916933530819\\
72.94	0.00837917028835746\\
72.95	0.00837917124242532\\
72.96	0.00837917219751319\\
72.97	0.00837917315362244\\
72.98	0.00837917411075444\\
72.99	0.00837917506891053\\
73	0.008379176028092\\
73.01	0.00837917698830014\\
73.02	0.00837917794953621\\
73.03	0.00837917891180143\\
73.04	0.008379179875097\\
73.05	0.00837918083942408\\
73.06	0.00837918180478383\\
73.07	0.00837918277117734\\
73.08	0.00837918373860569\\
73.09	0.00837918470706993\\
73.1	0.00837918567657108\\
73.11	0.00837918664711011\\
73.12	0.00837918761868797\\
73.13	0.00837918859130558\\
73.14	0.00837918956496381\\
73.15	0.0083791905396635\\
73.16	0.00837919151540546\\
73.17	0.00837919249219046\\
73.18	0.00837919347001922\\
73.19	0.00837919444889244\\
73.2	0.00837919542881077\\
73.21	0.00837919640977481\\
73.22	0.00837919739178515\\
73.23	0.0083791983748423\\
73.24	0.00837919935894674\\
73.25	0.00837920034409893\\
73.26	0.00837920133029925\\
73.27	0.00837920231754806\\
73.28	0.00837920330584566\\
73.29	0.00837920429519231\\
73.3	0.00837920528558822\\
73.31	0.00837920627703354\\
73.32	0.00837920726952839\\
73.33	0.00837920826307283\\
73.34	0.00837920925766686\\
73.35	0.00837921025331044\\
73.36	0.00837921125000348\\
73.37	0.0083792122477458\\
73.38	0.00837921324653722\\
73.39	0.00837921424637746\\
73.4	0.00837921524726619\\
73.41	0.00837921624920304\\
73.42	0.00837921725218756\\
73.43	0.00837921825621925\\
73.44	0.00837921926129755\\
73.45	0.00837922026742182\\
73.46	0.00837922127459137\\
73.47	0.00837922228280545\\
73.48	0.00837922329206323\\
73.49	0.00837922430236382\\
73.5	0.00837922531370626\\
73.51	0.00837922632608951\\
73.52	0.00837922733951248\\
73.53	0.00837922835397399\\
73.54	0.00837922936947279\\
73.55	0.00837923038600757\\
73.56	0.00837923140357691\\
73.57	0.00837923242217935\\
73.58	0.00837923344181334\\
73.59	0.00837923446247722\\
73.6	0.0083792354841693\\
73.61	0.00837923650688777\\
73.62	0.00837923753063074\\
73.63	0.00837923855539625\\
73.64	0.00837923958118225\\
73.65	0.00837924060798658\\
73.66	0.00837924163580703\\
73.67	0.00837924266464125\\
73.68	0.00837924369448684\\
73.69	0.00837924472534128\\
73.7	0.00837924575720197\\
73.71	0.0083792467900662\\
73.72	0.00837924782393117\\
73.73	0.00837924885879397\\
73.74	0.00837924989465161\\
73.75	0.00837925093150097\\
73.76	0.00837925196933884\\
73.77	0.0083792530081619\\
73.78	0.00837925404796673\\
73.79	0.00837925508874978\\
73.8	0.00837925613050739\\
73.81	0.00837925717323582\\
73.82	0.00837925821693118\\
73.83	0.00837925926158947\\
73.84	0.00837926030720658\\
73.85	0.00837926135377827\\
73.86	0.00837926240130019\\
73.87	0.00837926344976785\\
73.88	0.00837926449917665\\
73.89	0.00837926554952184\\
73.9	0.00837926660079857\\
73.91	0.00837926765300182\\
73.92	0.00837926870612648\\
73.93	0.00837926976016726\\
73.94	0.00837927081511877\\
73.95	0.00837927187097547\\
73.96	0.00837927292773165\\
73.97	0.00837927398538148\\
73.98	0.008379275043919\\
73.99	0.00837927610333807\\
74	0.00837927716363241\\
74.01	0.0083792782247956\\
74.02	0.00837927928682105\\
74.03	0.00837928034970201\\
74.04	0.00837928141343159\\
74.05	0.00837928247800272\\
74.06	0.00837928354340817\\
74.07	0.00837928460964055\\
74.08	0.0083792856766923\\
74.09	0.00837928674455568\\
74.1	0.00837928781322279\\
74.11	0.00837928888268554\\
74.12	0.00837928995293567\\
74.13	0.00837929102396474\\
74.14	0.00837929209576413\\
74.15	0.00837929316832501\\
74.16	0.00837929424163839\\
74.17	0.00837929531569508\\
74.18	0.00837929639048569\\
74.19	0.00837929746600065\\
74.2	0.00837929854223015\\
74.21	0.00837929961916423\\
74.22	0.0083793006967927\\
74.23	0.00837930177510515\\
74.24	0.00837930285409099\\
74.25	0.00837930393373939\\
74.26	0.00837930501403932\\
74.27	0.00837930609497952\\
74.28	0.00837930717654851\\
74.29	0.00837930825873461\\
74.3	0.00837930934152586\\
74.31	0.00837931042491012\\
74.32	0.00837931150887498\\
74.33	0.00837931259340783\\
74.34	0.00837931367849578\\
74.35	0.00837931476412572\\
74.36	0.00837931585028429\\
74.37	0.00837931693695787\\
74.38	0.0083793180241326\\
74.39	0.00837931911179436\\
74.4	0.00837932019992877\\
74.41	0.00837932128852118\\
74.42	0.00837932237755667\\
74.43	0.00837932346702008\\
74.44	0.00837932455689594\\
74.45	0.00837932564716853\\
74.46	0.00837932673782182\\
74.47	0.00837932782883952\\
74.48	0.00837932892020505\\
74.49	0.00837933001190153\\
74.5	0.00837933110391179\\
74.51	0.00837933219621836\\
74.52	0.00837933328880347\\
74.53	0.00837933438164904\\
74.54	0.00837933547473667\\
74.55	0.00837933656804767\\
74.56	0.00837933766156301\\
74.57	0.00837933875526334\\
74.58	0.00837933984912901\\
74.59	0.00837934094314001\\
74.6	0.008379342037276\\
74.61	0.00837934313151631\\
74.62	0.00837934422583993\\
74.63	0.0083793453202255\\
74.64	0.0083793464146513\\
74.65	0.00837934750909526\\
74.66	0.00837934860353497\\
74.67	0.00837934969794762\\
74.68	0.00837935079231006\\
74.69	0.00837935188659876\\
74.7	0.00837935298078981\\
74.71	0.00837935407485892\\
74.72	0.00837935516878141\\
74.73	0.00837935626253221\\
74.74	0.00837935735608587\\
74.75	0.00837935844941652\\
74.76	0.00837935954249789\\
74.77	0.0083793606353033\\
74.78	0.00837936172780567\\
74.79	0.00837936281997748\\
74.8	0.0083793639117908\\
74.81	0.00837936500321726\\
74.82	0.00837936609422806\\
74.83	0.00837936718479397\\
74.84	0.0083793682748853\\
74.85	0.00837936936447192\\
74.86	0.00837937045352325\\
74.87	0.00837937154200824\\
74.88	0.00837937262989538\\
74.89	0.00837937371715269\\
74.9	0.0083793748037477\\
74.91	0.00837937588964749\\
74.92	0.00837937697481862\\
74.93	0.00837937805922719\\
74.94	0.00837937914283876\\
74.95	0.00837938022561843\\
74.96	0.00837938130753078\\
74.97	0.00837938238853985\\
74.98	0.00837938346860918\\
74.99	0.0083793845477018\\
75	0.00837938562578017\\
75.01	0.00837938670280624\\
75.02	0.00837938777874141\\
75.03	0.00837938885354652\\
75.04	0.00837938992718188\\
75.05	0.00837939099960721\\
75.06	0.00837939207078168\\
75.07	0.00837939314066386\\
75.08	0.00837939420921178\\
75.09	0.00837939527638286\\
75.1	0.00837939634213391\\
75.11	0.00837939740642165\\
75.12	0.00837939846920286\\
75.13	0.00837939953043446\\
75.14	0.00837940059007352\\
75.15	0.00837940164807726\\
75.16	0.0083794027044031\\
75.17	0.00837940375900869\\
75.18	0.00837940481185187\\
75.19	0.00837940586289081\\
75.2	0.0083794069120839\\
75.21	0.00837940795938991\\
75.22	0.00837940900476792\\
75.23	0.00837941004817738\\
75.24	0.00837941108957816\\
75.25	0.00837941212893056\\
75.26	0.00837941316619533\\
75.27	0.00837941420133374\\
75.28	0.00837941523430757\\
75.29	0.00837941626507917\\
75.3	0.00837941729361148\\
75.31	0.00837941831986809\\
75.32	0.00837941934381325\\
75.33	0.00837942036541193\\
75.34	0.00837942138462982\\
75.35	0.00837942240143344\\
75.36	0.00837942341579009\\
75.37	0.00837942442766799\\
75.38	0.00837942543703624\\
75.39	0.00837942644386492\\
75.4	0.00837942744812509\\
75.41	0.0083794284497889\\
75.42	0.00837942944882956\\
75.43	0.00837943044522146\\
75.44	0.00837943143894017\\
75.45	0.00837943242996252\\
75.46	0.00837943341826666\\
75.47	0.00837943440383207\\
75.48	0.00837943538663968\\
75.49	0.00837943636667188\\
75.5	0.00837943734391259\\
75.51	0.00837943831834733\\
75.52	0.00837943928996235\\
75.53	0.008379440258744\\
75.54	0.00837944122467874\\
75.55	0.00837944218775314\\
75.56	0.00837944314795387\\
75.57	0.00837944410526774\\
75.58	0.00837944505968165\\
75.59	0.00837944601118265\\
75.6	0.0083794469597579\\
75.61	0.00837944790539471\\
75.62	0.00837944884808049\\
75.63	0.00837944978780281\\
75.64	0.00837945072454939\\
75.65	0.00837945165830809\\
75.66	0.0083794525890669\\
75.67	0.00837945351681398\\
75.68	0.00837945444153765\\
75.69	0.00837945536322639\\
75.7	0.00837945628186884\\
75.71	0.00837945719745381\\
75.72	0.00837945810997028\\
75.73	0.00837945901940742\\
75.74	0.00837945992575457\\
75.75	0.00837946082900128\\
75.76	0.00837946172913725\\
75.77	0.00837946262615241\\
75.78	0.00837946352003688\\
75.79	0.00837946441078099\\
75.8	0.00837946529837526\\
75.81	0.00837946618281045\\
75.82	0.00837946706407752\\
75.83	0.00837946794216765\\
75.84	0.00837946881707228\\
75.85	0.00837946968878304\\
75.86	0.00837947055729184\\
75.87	0.0083794714225908\\
75.88	0.00837947228467231\\
75.89	0.00837947314352899\\
75.9	0.00837947399915375\\
75.91	0.00837947485153974\\
75.92	0.00837947570068039\\
75.93	0.0083794765465694\\
75.94	0.00837947738920075\\
75.95	0.00837947822856872\\
75.96	0.00837947906466786\\
75.97	0.00837947989749302\\
75.98	0.00837948072703937\\
75.99	0.00837948155330238\\
76	0.00837948237627781\\
76.01	0.00837948319596178\\
76.02	0.00837948401235071\\
76.03	0.00837948482544136\\
76.04	0.00837948563523082\\
76.05	0.00837948644171654\\
76.06	0.0083794872448963\\
76.07	0.00837948804476826\\
76.08	0.00837948884133092\\
76.09	0.00837948963458315\\
76.1	0.00837949042452422\\
76.11	0.00837949121115375\\
76.12	0.00837949199447177\\
76.13	0.00837949277447871\\
76.14	0.00837949355117536\\
76.15	0.00837949432456298\\
76.16	0.00837949509464319\\
76.17	0.00837949586141807\\
76.18	0.0083794966248901\\
76.19	0.00837949738506222\\
76.2	0.00837949814193779\\
76.21	0.00837949889552064\\
76.22	0.00837949964581504\\
76.23	0.00837950039282574\\
76.24	0.00837950113655795\\
76.25	0.00837950187701735\\
76.26	0.00837950261421013\\
76.27	0.00837950334814296\\
76.28	0.00837950407882301\\
76.29	0.00837950480625795\\
76.3	0.00837950553045598\\
76.31	0.00837950625142582\\
76.32	0.00837950696917672\\
76.33	0.00837950768371846\\
76.34	0.00837950839506138\\
76.35	0.00837950910321636\\
76.36	0.00837950980819486\\
76.37	0.0083795105100089\\
76.38	0.00837951120867107\\
76.39	0.00837951190419457\\
76.4	0.00837951259659316\\
76.41	0.00837951328588124\\
76.42	0.0083795139720738\\
76.43	0.00837951465518644\\
76.44	0.0083795153352354\\
76.45	0.00837951601223756\\
76.46	0.00837951668621044\\
76.47	0.0083795173571722\\
76.48	0.00837951802514168\\
76.49	0.00837951869013838\\
76.5	0.00837951935218246\\
76.51	0.0083795200112948\\
76.52	0.00837952066749696\\
76.53	0.00837952132081119\\
76.54	0.00837952197126047\\
76.55	0.00837952261886849\\
76.56	0.00837952326365967\\
76.57	0.00837952390565919\\
76.58	0.00837952454489294\\
76.59	0.00837952518138759\\
76.6	0.00837952581517058\\
76.61	0.0083795264462701\\
76.62	0.00837952707471513\\
76.63	0.00837952770053546\\
76.64	0.00837952832376166\\
76.65	0.00837952894442512\\
76.66	0.00837952956255803\\
76.67	0.00837953017819343\\
76.68	0.00837953079136519\\
76.69	0.008379531402108\\
76.7	0.00837953201045744\\
76.71	0.00837953261644994\\
76.72	0.00837953322012278\\
76.73	0.00837953382151414\\
76.74	0.0083795344206631\\
76.75	0.00837953501760961\\
76.76	0.00837953561239455\\
76.77	0.0083795362050597\\
76.78	0.00837953679564778\\
76.79	0.00837953738420242\\
76.8	0.00837953797076822\\
76.81	0.00837953855539071\\
76.82	0.00837953913811639\\
76.83	0.00837953971899272\\
76.84	0.00837954029806815\\
76.85	0.0083795408753921\\
76.86	0.00837954145101498\\
76.87	0.00837954202498823\\
76.88	0.00837954259736427\\
76.89	0.00837954316819656\\
76.9	0.00837954373753956\\
76.91	0.00837954430544881\\
76.92	0.00837954487198084\\
76.93	0.00837954543719328\\
76.94	0.00837954600114478\\
76.95	0.00837954656389507\\
76.96	0.00837954712550497\\
76.97	0.00837954768603637\\
76.98	0.00837954824555224\\
76.99	0.00837954880411665\\
77	0.0083795493617948\\
77.01	0.00837954991865296\\
77.02	0.00837955047475855\\
77.03	0.00837955103018011\\
77.04	0.00837955158498728\\
77.05	0.00837955213925089\\
77.06	0.00837955269304286\\
77.07	0.00837955324643631\\
77.08	0.00837955379950549\\
77.09	0.0083795543523258\\
77.1	0.00837955490497384\\
77.11	0.00837955545752736\\
77.12	0.00837955601006529\\
77.13	0.00837955656266775\\
77.14	0.00837955711541605\\
77.15	0.00837955766839268\\
77.16	0.00837955822168134\\
77.17	0.00837955877536692\\
77.18	0.00837955932953552\\
77.19	0.00837955988427443\\
77.2	0.00837956043967219\\
77.21	0.00837956099581852\\
77.22	0.00837956155280436\\
77.23	0.00837956211072187\\
77.24	0.00837956266966445\\
77.25	0.00837956322972669\\
77.26	0.00837956379100442\\
77.27	0.00837956435359469\\
77.28	0.00837956491759578\\
77.29	0.00837956548310718\\
77.3	0.00837956605022961\\
77.31	0.00837956661906503\\
77.32	0.00837956718971659\\
77.33	0.00837956776228868\\
77.34	0.00837956833688691\\
77.35	0.0083795689136181\\
77.36	0.00837956949259026\\
77.37	0.00837957007391265\\
77.38	0.00837957065769571\\
77.39	0.00837957124405108\\
77.4	0.0083795718330916\\
77.41	0.0083795724249313\\
77.42	0.00837957301968539\\
77.43	0.00837957361747024\\
77.44	0.00837957421840343\\
77.45	0.00837957482260366\\
77.46	0.00837957543019079\\
77.47	0.00837957604128584\\
77.48	0.00837957665601095\\
77.49	0.00837957727448937\\
77.5	0.00837957789684547\\
77.51	0.00837957852320473\\
77.52	0.00837957915369368\\
77.53	0.00837957978843996\\
77.54	0.00837958042757224\\
77.55	0.00837958107122023\\
77.56	0.00837958171951467\\
77.57	0.0083795823725873\\
77.58	0.00837958303057087\\
77.59	0.00837958369359906\\
77.6	0.00837958436180654\\
77.61	0.00837958503532887\\
77.62	0.00837958571430255\\
77.63	0.00837958639886493\\
77.64	0.00837958708915424\\
77.65	0.00837958778530954\\
77.66	0.00837958848747069\\
77.67	0.00837958919577834\\
77.68	0.00837958991037388\\
77.69	0.00837959063139942\\
77.7	0.00837959135899777\\
77.71	0.00837959209331239\\
77.72	0.00837959283448736\\
77.73	0.00837959358266736\\
77.74	0.00837959433799762\\
77.75	0.00837959510062387\\
77.76	0.00837959587069232\\
77.77	0.00837959664834964\\
77.78	0.00837959743374286\\
77.79	0.00837959822701937\\
77.8	0.00837959902832687\\
77.81	0.0083795998378133\\
77.82	0.00837960065562683\\
77.83	0.00837960148191578\\
77.84	0.00837960231682856\\
77.85	0.00837960316051365\\
77.86	0.00837960401311951\\
77.87	0.00837960487479456\\
77.88	0.00837960574568709\\
77.89	0.00837960662594519\\
77.9	0.00837960751571673\\
77.91	0.00837960841514925\\
77.92	0.00837960932438993\\
77.93	0.00837961024358548\\
77.94	0.00837961117288211\\
77.95	0.00837961211242542\\
77.96	0.00837961306236036\\
77.97	0.00837961402283109\\
77.98	0.00837961499398098\\
77.99	0.00837961597595246\\
78	0.00837961696888698\\
78.01	0.00837961797292488\\
78.02	0.00837961898820533\\
78.03	0.00837962001486621\\
78.04	0.00837962105304405\\
78.05	0.00837962210287388\\
78.06	0.00837962316448918\\
78.07	0.00837962423802205\\
78.08	0.00837962532360605\\
78.09	0.00837962642137627\\
78.1	0.00837962753146928\\
78.11	0.0083796286540232\\
78.12	0.00837962978917767\\
78.13	0.00837963093707393\\
78.14	0.00837963209785477\\
78.15	0.00837963327166459\\
78.16	0.0083796344586494\\
78.17	0.00837963565895684\\
78.18	0.00837963687273623\\
78.19	0.00837963810013851\\
78.2	0.00837963934131637\\
78.21	0.00837964059642415\\
78.22	0.00837964186561795\\
78.23	0.00837964314905562\\
78.24	0.00837964444689676\\
78.25	0.00837964575930276\\
78.26	0.00837964708643682\\
78.27	0.00837964842846396\\
78.28	0.00837964978555104\\
78.29	0.00837965115786681\\
78.3	0.00837965254558189\\
78.31	0.0083796539488688\\
78.32	0.00837965536790202\\
78.33	0.00837965680285795\\
78.34	0.00837965825391497\\
78.35	0.00837965972125348\\
78.36	0.00837966120505587\\
78.37	0.00837966270550658\\
78.38	0.00837966422279212\\
78.39	0.00837966575710106\\
78.4	0.00837966730862412\\
78.41	0.00837966887755411\\
78.42	0.00837967046408603\\
78.43	0.00837967206841705\\
78.44	0.00837967369074654\\
78.45	0.0083796753312761\\
78.46	0.00837967699020959\\
78.47	0.00837967866775314\\
78.48	0.00837968036411521\\
78.49	0.00837968207950655\\
78.5	0.00837968381414032\\
78.51	0.00837968556823201\\
78.52	0.00837968734199955\\
78.53	0.0083796891356633\\
78.54	0.0083796909494461\\
78.55	0.00837969278357324\\
78.56	0.00837969463827258\\
78.57	0.00837969651377449\\
78.58	0.00837969841031193\\
78.59	0.00837970032812047\\
78.6	0.00837970226743831\\
78.61	0.0083797042285063\\
78.62	0.00837970621156799\\
78.63	0.00837970821686968\\
78.64	0.00837971024466038\\
78.65	0.00837971229519192\\
78.66	0.00837971436871891\\
78.67	0.00837971646549884\\
78.68	0.00837971858579207\\
78.69	0.00837972072986185\\
78.7	0.00837972289797439\\
78.71	0.00837972509039889\\
78.72	0.00837972730740752\\
78.73	0.00837972954927554\\
78.74	0.00837973181628124\\
78.75	0.00837973410870606\\
78.76	0.00837973642683456\\
78.77	0.0083797387709545\\
78.78	0.00837974114135683\\
78.79	0.00837974353833578\\
78.8	0.00837974596218885\\
78.81	0.00837974841321687\\
78.82	0.00837975089172405\\
78.83	0.00837975339801797\\
78.84	0.00837975593240966\\
78.85	0.00837975849521364\\
78.86	0.00837976108674792\\
78.87	0.00837976370733408\\
78.88	0.0083797663572973\\
78.89	0.00837976903696637\\
78.9	0.00837977174667378\\
78.91	0.00837977448675571\\
78.92	0.00837977725755213\\
78.93	0.00837978005940678\\
78.94	0.00837978289266724\\
78.95	0.00837978575768501\\
78.96	0.00837978865481547\\
78.97	0.008379791584418\\
78.98	0.008379794546856\\
78.99	0.0083797975424969\\
79	0.00837980057171227\\
79.01	0.00837980363487782\\
79.02	0.00837980673237343\\
79.03	0.00837980986458327\\
79.04	0.00837981303189577\\
79.05	0.00837981623470371\\
79.06	0.00837981947340427\\
79.07	0.00837982274839904\\
79.08	0.00837982606009413\\
79.09	0.00837982940890014\\
79.1	0.00837983279523231\\
79.11	0.00837983621951049\\
79.12	0.00837983968215922\\
79.13	0.00837984318360778\\
79.14	0.00837984672429027\\
79.15	0.0083798503046456\\
79.16	0.00837985392511762\\
79.17	0.00837985758615511\\
79.18	0.00837986128821189\\
79.19	0.00837986503174681\\
79.2	0.00837986881722389\\
79.21	0.0083798726451123\\
79.22	0.00837987651588647\\
79.23	0.00837988043002611\\
79.24	0.0083798843880163\\
79.25	0.00837988839034754\\
79.26	0.00837989243751579\\
79.27	0.00837989653002258\\
79.28	0.00837990066837502\\
79.29	0.00837990485308588\\
79.3	0.00837990908467367\\
79.31	0.00837991336366267\\
79.32	0.00837991769058304\\
79.33	0.00837992206597085\\
79.34	0.00837992649036815\\
79.35	0.00837993096432305\\
79.36	0.00837993548838977\\
79.37	0.00837994006312874\\
79.38	0.00837994468910662\\
79.39	0.00837994936689642\\
79.4	0.00837995409707754\\
79.41	0.00837995888023585\\
79.42	0.00837996371696375\\
79.43	0.00837996860786027\\
79.44	0.00837997355353113\\
79.45	0.0083799785545888\\
79.46	0.0083799836116526\\
79.47	0.00837998872534874\\
79.48	0.00837999389631045\\
79.49	0.00837999912517803\\
79.5	0.00838000441259889\\
79.51	0.00838000975922772\\
79.52	0.00838001516572648\\
79.53	0.00838002063276453\\
79.54	0.0083800261610187\\
79.55	0.0083800317511734\\
79.56	0.00838003740392065\\
79.57	0.00838004311996019\\
79.58	0.00838004889999961\\
79.59	0.00838005474475435\\
79.6	0.00838006065494788\\
79.61	0.00838006663131171\\
79.62	0.00838007267458554\\
79.63	0.0083800787855173\\
79.64	0.0083800849648633\\
79.65	0.00838009121338827\\
79.66	0.00838009753186547\\
79.67	0.00838010392107682\\
79.68	0.00838011038181293\\
79.69	0.00838011691487325\\
79.7	0.00838012352106616\\
79.71	0.00838013020120906\\
79.72	0.00838013695612845\\
79.73	0.00838014378666008\\
79.74	0.00838015069364901\\
79.75	0.00838015767794975\\
79.76	0.00838016474042631\\
79.77	0.00838017188195238\\
79.78	0.00838017910341138\\
79.79	0.00838018640569659\\
79.8	0.00838019378971125\\
79.81	0.0083802012563687\\
79.82	0.00838020880659245\\
79.83	0.00838021644131632\\
79.84	0.00838022416148456\\
79.85	0.00838023196805194\\
79.86	0.0083802398619839\\
79.87	0.00838024784425663\\
79.88	0.00838025591585723\\
79.89	0.0083802640777838\\
79.9	0.00838027233104558\\
79.91	0.00838028067666309\\
79.92	0.00838028911566819\\
79.93	0.00838029764910429\\
79.94	0.00838030627802643\\
79.95	0.00838031500350142\\
79.96	0.00838032382660796\\
79.97	0.00838033274843681\\
79.98	0.00838034177009085\\
79.99	0.00838035089268532\\
80	0.00838036011734786\\
80.01	0.00838036944521871\\
};
\addplot [color=black,solid]
  table[row sep=crcr]{%
80.01	0.00838036944521871\\
80.02	0.00838037887745081\\
80.03	0.00838038841520997\\
80.04	0.00838039805967502\\
80.05	0.00838040781203792\\
80.06	0.00838041767350392\\
80.07	0.00838042764529172\\
80.08	0.00838043772863362\\
80.09	0.00838044792477566\\
80.1	0.00838045823497778\\
80.11	0.00838046866051395\\
80.12	0.00838047920267239\\
80.13	0.00838048986275565\\
80.14	0.00838050064208083\\
80.15	0.00838051154197969\\
80.16	0.00838052256379888\\
80.17	0.00838053370890004\\
80.18	0.00838054497866001\\
80.19	0.00838055637447098\\
80.2	0.00838056789774066\\
80.21	0.00838057954989245\\
80.22	0.00838059133236565\\
80.23	0.00838060324661558\\
80.24	0.00838061529411382\\
80.25	0.00838062747634831\\
80.26	0.00838063979482364\\
80.27	0.00838065225106114\\
80.28	0.00838066484659911\\
80.29	0.008380677582993\\
80.3	0.00838069046181561\\
80.31	0.00838070348465727\\
80.32	0.00838071665312605\\
80.33	0.00838072996884793\\
80.34	0.00838074343346705\\
80.35	0.00838075704864585\\
80.36	0.00838077081606532\\
80.37	0.00838078473742518\\
80.38	0.00838079881444411\\
80.39	0.00838081304885996\\
80.4	0.00838082744242992\\
80.41	0.0083808419969308\\
80.42	0.00838085671415921\\
80.43	0.00838087159593178\\
80.44	0.0083808866440854\\
80.45	0.00838090186047744\\
80.46	0.00838091724698595\\
80.47	0.00838093280550994\\
80.48	0.00838094853796959\\
80.49	0.00838096444630646\\
80.5	0.00838098053248378\\
80.51	0.00838099679848663\\
80.52	0.00838101324632226\\
80.53	0.00838102987802024\\
80.54	0.00838104669563281\\
80.55	0.00838106370123506\\
80.56	0.00838108089692521\\
80.57	0.00838109828482486\\
80.58	0.00838111586707927\\
80.59	0.0083811336458576\\
80.6	0.00838115162335319\\
80.61	0.00838116980178382\\
80.62	0.00838118818339198\\
80.63	0.00838120677044518\\
80.64	0.00838122556523616\\
80.65	0.00838124457008324\\
80.66	0.00838126378733057\\
80.67	0.00838128321934841\\
80.68	0.00838130286853344\\
80.69	0.00838132273730907\\
80.7	0.00838134282812567\\
80.71	0.00838136314346095\\
80.72	0.00838138368582023\\
80.73	0.00838140445773673\\
80.74	0.0083814254617719\\
80.75	0.00838144670051573\\
80.76	0.00838146817658709\\
80.77	0.00838148989263401\\
80.78	0.00838151185133402\\
80.79	0.0083815340553945\\
80.8	0.00838155650755299\\
80.81	0.00838157921057752\\
80.82	0.00838160216726699\\
80.83	0.00838162538045144\\
80.84	0.00838164885299249\\
80.85	0.00838167258778359\\
80.86	0.00838169658775046\\
80.87	0.00838172085585142\\
80.88	0.00838174539507772\\
80.89	0.00838177020845395\\
80.9	0.00838179529903841\\
80.91	0.00838182066992344\\
80.92	0.00838184632423585\\
80.93	0.00838187226513727\\
80.94	0.00838189849582457\\
80.95	0.00838192501953021\\
80.96	0.00838195183952267\\
80.97	0.00838197895910684\\
80.98	0.00838200638162441\\
80.99	0.00838203411045432\\
81	0.00838206214901312\\
81.01	0.00838209050075544\\
81.02	0.00838211916917437\\
81.03	0.00838214815780193\\
81.04	0.00838217747020946\\
81.05	0.00838220711000808\\
81.06	0.00838223708084916\\
81.07	0.0083822673864247\\
81.08	0.00838229803046784\\
81.09	0.00838232901675332\\
81.1	0.00838236034909787\\
81.11	0.00838239203136078\\
81.12	0.0083824240674443\\
81.13	0.00838245646129413\\
81.14	0.00838248921689994\\
81.15	0.00838252233829581\\
81.16	0.00838255582956076\\
81.17	0.00838258969481922\\
81.18	0.00838262393824158\\
81.19	0.00838265856404465\\
81.2	0.0083826935764922\\
81.21	0.00838272897989551\\
81.22	0.00838276477861383\\
81.23	0.00838280097705498\\
81.24	0.00838283757967586\\
81.25	0.00838287459098301\\
81.26	0.00838291201553313\\
81.27	0.00838294985793368\\
81.28	0.00838298812284341\\
81.29	0.00838302681497296\\
81.3	0.00838306593908543\\
81.31	0.00838310549999692\\
81.32	0.00838314550257721\\
81.33	0.00838318595175025\\
81.34	0.00838322685249485\\
81.35	0.00838326820984525\\
81.36	0.00838331002889173\\
81.37	0.00838335231478127\\
81.38	0.00838339507271814\\
81.39	0.00838343830796454\\
81.4	0.0083834820258413\\
81.41	0.00838352623172845\\
81.42	0.00838357093106594\\
81.43	0.00838361612935428\\
81.44	0.00838366183215523\\
81.45	0.00838370804509244\\
81.46	0.00838375477385221\\
81.47	0.00838380202418411\\
81.48	0.00838384980190175\\
81.49	0.00838389811288343\\
81.5	0.00838394696307292\\
81.51	0.00838399635848015\\
81.52	0.00838404630518194\\
81.53	0.00838409680932279\\
81.54	0.00838414787711557\\
81.55	0.00838419951484234\\
81.56	0.00838425172885507\\
81.57	0.00838430452557645\\
81.58	0.00838435791150067\\
81.59	0.00838441189319421\\
81.6	0.00838446647729663\\
81.61	0.00838452167052143\\
81.62	0.00838457747965681\\
81.63	0.00838463391156655\\
81.64	0.00838469097319083\\
81.65	0.00838474867154707\\
81.66	0.00838480701373081\\
81.67	0.00838486600691657\\
81.68	0.00838492565835873\\
81.69	0.00838498597539242\\
81.7	0.0083850469654344\\
81.71	0.008385108635984\\
81.72	0.00838517099462401\\
81.73	0.00838523404902161\\
81.74	0.00838529780692936\\
81.75	0.00838536227618605\\
81.76	0.00838542746471775\\
81.77	0.00838549338053875\\
81.78	0.00838556003175252\\
81.79	0.00838562742655274\\
81.8	0.00838569557322426\\
81.81	0.00838576448014416\\
81.82	0.00838583415578274\\
81.83	0.00838590460870458\\
81.84	0.00838597584756957\\
81.85	0.00838604788113398\\
81.86	0.00838612071825153\\
81.87	0.00838619436787448\\
81.88	0.00838626883905472\\
81.89	0.00838634414094488\\
81.9	0.00838642028279943\\
81.91	0.00838649727397586\\
81.92	0.00838657512393578\\
81.93	0.00838665384224611\\
81.94	0.00838673343858021\\
81.95	0.00838681392271911\\
81.96	0.0083868953045527\\
81.97	0.0083869775940809\\
81.98	0.00838706080141493\\
81.99	0.00838714493677854\\
82	0.00838723001050923\\
82.01	0.00838731603305954\\
82.02	0.00838740301499835\\
82.03	0.00838749096701214\\
82.04	0.00838757989990629\\
82.05	0.00838766982460644\\
82.06	0.0083877607521598\\
82.07	0.00838785269373653\\
82.08	0.00838794566063108\\
82.09	0.00838803966426355\\
82.1	0.00838813471618117\\
82.11	0.00838823082805963\\
82.12	0.00838832801170455\\
82.13	0.00838842627905292\\
82.14	0.00838852564217455\\
82.15	0.00838862611327356\\
82.16	0.00838872770468988\\
82.17	0.00838883042890074\\
82.18	0.0083889342985222\\
82.19	0.00838903932631071\\
82.2	0.00838914552516466\\
82.21	0.00838925290812598\\
82.22	0.0083893614883817\\
82.23	0.00838947127926558\\
82.24	0.00838958229425975\\
82.25	0.00838969454699637\\
82.26	0.00838980805125927\\
82.27	0.00838992282098565\\
82.28	0.00839003887026779\\
82.29	0.00839015621335479\\
82.3	0.00839027486465427\\
82.31	0.00839039483873417\\
82.32	0.00839051615032455\\
82.33	0.00839063881431933\\
82.34	0.00839076284577818\\
82.35	0.00839088825992832\\
82.36	0.00839101507216639\\
82.37	0.00839114209980636\\
82.38	0.00839126932057027\\
82.39	0.00839139673578494\\
82.4	0.00839152434679041\\
82.41	0.00839165215494015\\
82.42	0.0083917801616011\\
82.43	0.00839190836815389\\
82.44	0.00839203677599295\\
82.45	0.00839216538652664\\
82.46	0.00839229420117742\\
82.47	0.00839242322138197\\
82.48	0.00839255244859138\\
82.49	0.00839268188427122\\
82.5	0.00839281152990177\\
82.51	0.00839294138697813\\
82.52	0.00839307145701039\\
82.53	0.00839320174152376\\
82.54	0.00839333224205875\\
82.55	0.00839346296017131\\
82.56	0.00839359389743303\\
82.57	0.00839372505543122\\
82.58	0.00839385643576916\\
82.59	0.0083939880400662\\
82.6	0.00839411986995796\\
82.61	0.0083942519270965\\
82.62	0.00839438421315046\\
82.63	0.00839451672980524\\
82.64	0.0083946494787632\\
82.65	0.0083947824617438\\
82.66	0.0083949156804838\\
82.67	0.0083950491367374\\
82.68	0.00839518283227648\\
82.69	0.00839531676889073\\
82.7	0.00839545094838785\\
82.71	0.00839558537259373\\
82.72	0.00839572004335265\\
82.73	0.00839585496252746\\
82.74	0.00839599013199976\\
82.75	0.00839612555367011\\
82.76	0.00839626122945823\\
82.77	0.00839639716130314\\
82.78	0.00839653335116345\\
82.79	0.00839666980101748\\
82.8	0.00839680651286351\\
82.81	0.00839694348871996\\
82.82	0.00839708073062559\\
82.83	0.00839721824063974\\
82.84	0.00839735602084252\\
82.85	0.00839749407333501\\
82.86	0.00839763240023951\\
82.87	0.00839777100369972\\
82.88	0.00839790988588096\\
82.89	0.00839804904897044\\
82.9	0.00839818849517742\\
82.91	0.00839832822673348\\
82.92	0.00839846824589271\\
82.93	0.00839860855493198\\
82.94	0.00839874915615115\\
82.95	0.0083988900518733\\
82.96	0.00839903124444497\\
82.97	0.00839917273623643\\
82.98	0.00839931452964185\\
82.99	0.00839945662707963\\
83	0.00839959903099258\\
83.01	0.0083997417438482\\
83.02	0.00839988476813893\\
83.03	0.0084000281063824\\
83.04	0.00840017176112167\\
83.05	0.00840031573492551\\
83.06	0.00840046003038865\\
83.07	0.00840060465013207\\
83.08	0.00840074959680322\\
83.09	0.00840089487307632\\
83.1	0.00840104048165262\\
83.11	0.0084011864252607\\
83.12	0.00840133270665671\\
83.13	0.00840147932862467\\
83.14	0.00840162629397674\\
83.15	0.00840177360555354\\
83.16	0.00840192126622437\\
83.17	0.0084020692788876\\
83.18	0.00840221764647085\\
83.19	0.00840236637193138\\
83.2	0.00840251545825635\\
83.21	0.00840266490846311\\
83.22	0.00840281472559954\\
83.23	0.00840296491274432\\
83.24	0.00840311547300728\\
83.25	0.00840326640952969\\
83.26	0.00840341772548459\\
83.27	0.00840356942407708\\
83.28	0.00840372150854469\\
83.29	0.00840387398215769\\
83.3	0.0084040268482194\\
83.31	0.00840418011006653\\
83.32	0.00840433377106957\\
83.33	0.00840448783463303\\
83.34	0.00840464230419587\\
83.35	0.00840479718323182\\
83.36	0.0084049524752497\\
83.37	0.00840510818379381\\
83.38	0.0084052643124443\\
83.39	0.00840542086481745\\
83.4	0.00840557784456615\\
83.41	0.00840573525538017\\
83.42	0.00840589310098657\\
83.43	0.00840605138515008\\
83.44	0.00840621011167347\\
83.45	0.00840636928439793\\
83.46	0.00840652890720344\\
83.47	0.0084066889840092\\
83.48	0.00840684951877398\\
83.49	0.00840701051549655\\
83.5	0.00840717197821606\\
83.51	0.00840733391101244\\
83.52	0.00840749631800682\\
83.53	0.00840765920336196\\
83.54	0.00840782257128262\\
83.55	0.00840798642601602\\
83.56	0.00840815077185225\\
83.57	0.00840831561312468\\
83.58	0.00840848095421044\\
83.59	0.00840864679953081\\
83.6	0.00840881315355167\\
83.61	0.00840898002078396\\
83.62	0.00840914740578412\\
83.63	0.00840931531315456\\
83.64	0.00840948374754408\\
83.65	0.00840965271364835\\
83.66	0.0084098222162104\\
83.67	0.00840999226002107\\
83.68	0.00841016284991947\\
83.69	0.00841033399079349\\
83.7	0.00841050568758026\\
83.71	0.00841067794526666\\
83.72	0.0084108507688898\\
83.73	0.00841102416353752\\
83.74	0.0084111981343489\\
83.75	0.00841137268651476\\
83.76	0.00841154782527816\\
83.77	0.00841172355593496\\
83.78	0.0084118998838343\\
83.79	0.00841207681437914\\
83.8	0.0084122543530268\\
83.81	0.0084124325052895\\
83.82	0.00841261127673487\\
83.83	0.00841279067298656\\
83.84	0.00841297069972471\\
83.85	0.00841315136268659\\
83.86	0.00841333266766711\\
83.87	0.00841351462051939\\
83.88	0.00841369722715539\\
83.89	0.00841388049354639\\
83.9	0.00841406442572368\\
83.91	0.00841424902977908\\
83.92	0.00841443431186554\\
83.93	0.0084146202781978\\
83.94	0.00841480693505292\\
83.95	0.00841499428877094\\
83.96	0.00841518234575548\\
83.97	0.00841537111247437\\
83.98	0.00841556059546028\\
83.99	0.00841575080131136\\
84	0.00841594173669187\\
84.01	0.00841613340833281\\
84.02	0.00841632582303264\\
84.03	0.00841651898765786\\
84.04	0.00841671290914373\\
84.05	0.00841690759449491\\
84.06	0.00841710305078615\\
84.07	0.00841729928516298\\
84.08	0.00841749630484241\\
84.09	0.00841769411711358\\
84.1	0.00841789272933851\\
84.11	0.00841809214895281\\
84.12	0.00841829238346634\\
84.13	0.00841849344046402\\
84.14	0.00841869532760647\\
84.15	0.0084188980526308\\
84.16	0.00841910162335135\\
84.17	0.00841930604766043\\
84.18	0.00841951133352905\\
84.19	0.00841971748900773\\
84.2	0.00841992452222724\\
84.21	0.00842013244139941\\
84.22	0.00842034125481785\\
84.23	0.00842055097085882\\
84.24	0.00842076159798196\\
84.25	0.00842097314473117\\
84.26	0.00842118561973534\\
84.27	0.00842139903170922\\
84.28	0.00842161338945427\\
84.29	0.00842182870185942\\
84.3	0.00842204497790199\\
84.31	0.0084222622266485\\
84.32	0.00842248045725554\\
84.33	0.00842269967897064\\
84.34	0.00842291990113312\\
84.35	0.00842314113317501\\
84.36	0.00842336338462192\\
84.37	0.00842358666509391\\
84.38	0.00842381098430645\\
84.39	0.00842403635207131\\
84.4	0.00842426277829747\\
84.41	0.00842449027299205\\
84.42	0.00842471884626127\\
84.43	0.00842494850831139\\
84.44	0.00842517926944965\\
84.45	0.00842541114008525\\
84.46	0.00842564413073033\\
84.47	0.0084258782520009\\
84.48	0.00842611351461792\\
84.49	0.0084263499294082\\
84.5	0.00842658750730548\\
84.51	0.00842682625935141\\
84.52	0.0084270661966966\\
84.53	0.00842730733060161\\
84.54	0.00842754967243805\\
84.55	0.0084277932336896\\
84.56	0.00842803802595306\\
84.57	0.00842828406093946\\
84.58	0.0084285313504751\\
84.59	0.00842877990650267\\
84.6	0.00842902974108233\\
84.61	0.00842928086639283\\
84.62	0.00842953329473262\\
84.63	0.00842978703852098\\
84.64	0.00843004211029918\\
84.65	0.00843029852273158\\
84.66	0.00843055628860683\\
84.67	0.00843081542083903\\
84.68	0.00843107593246889\\
84.69	0.00843133783666494\\
84.7	0.0084316011467247\\
84.71	0.00843186587607593\\
84.72	0.00843213203827782\\
84.73	0.00843239964702221\\
84.74	0.00843266871613485\\
84.75	0.00843293925957665\\
84.76	0.00843321129144494\\
84.77	0.00843348482597473\\
84.78	0.00843375987754\\
84.79	0.008434036460655\\
84.8	0.00843431458997554\\
84.81	0.00843459428030031\\
84.82	0.00843487554657223\\
84.83	0.00843515840387976\\
84.84	0.00843544286745824\\
84.85	0.00843572895269129\\
84.86	0.00843601667511215\\
84.87	0.00843630605040508\\
84.88	0.00843659709440672\\
84.89	0.00843688982310757\\
84.9	0.00843718425265333\\
84.91	0.00843748039934636\\
84.92	0.00843777741934355\\
84.93	0.00843807519926577\\
84.94	0.00843837374598029\\
84.95	0.00843867306642709\\
84.96	0.00843897316761961\\
84.97	0.00843927405664558\\
84.98	0.00843957574066779\\
84.99	0.00843987822692487\\
85	0.00844018152273213\\
85.01	0.00844048563548238\\
85.02	0.00844079057264672\\
85.03	0.00844109634177542\\
85.04	0.00844140295049873\\
85.05	0.00844171040652775\\
85.06	0.00844201871765527\\
85.07	0.00844232789175667\\
85.08	0.00844263793679081\\
85.09	0.00844294886080084\\
85.1	0.00844326067191521\\
85.11	0.00844357337834848\\
85.12	0.00844388698840231\\
85.13	0.00844420151046633\\
85.14	0.00844451695301913\\
85.15	0.00844483332462915\\
85.16	0.00844515063395568\\
85.17	0.00844546888974982\\
85.18	0.00844578810085543\\
85.19	0.00844610827621015\\
85.2	0.00844642942484638\\
85.21	0.00844675155589229\\
85.22	0.00844707467857282\\
85.23	0.00844739880221076\\
85.24	0.00844772393622774\\
85.25	0.00844805009014529\\
85.26	0.00844837727358591\\
85.27	0.00844870549627415\\
85.28	0.00844903476803768\\
85.29	0.0084493650988084\\
85.3	0.00844969649862353\\
85.31	0.00845002897762671\\
85.32	0.00845036254606919\\
85.33	0.00845069721431093\\
85.34	0.00845103299282173\\
85.35	0.00845136989218247\\
85.36	0.00845170792308621\\
85.37	0.00845204709633941\\
85.38	0.00845238742286316\\
85.39	0.00845272891369436\\
85.4	0.00845307157998694\\
85.41	0.00845341543301315\\
85.42	0.00845376048416474\\
85.43	0.00845410674495432\\
85.44	0.00845445422701654\\
85.45	0.00845480294210945\\
85.46	0.00845515290211579\\
85.47	0.0084555041190443\\
85.48	0.00845585660503107\\
85.49	0.00845621037234086\\
85.5	0.00845656543336849\\
85.51	0.00845692180064021\\
85.52	0.00845727948681507\\
85.53	0.00845763850468636\\
85.54	0.008457998867183\\
85.55	0.00845836058737096\\
85.56	0.00845872367845477\\
85.57	0.00845908815377892\\
85.58	0.00845945402682937\\
85.59	0.00845982131123503\\
85.6	0.0084601900207693\\
85.61	0.00846056016935154\\
85.62	0.00846093177104867\\
85.63	0.00846130484007669\\
85.64	0.00846167939080228\\
85.65	0.00846205543774434\\
85.66	0.00846243299557567\\
85.67	0.00846281207912453\\
85.68	0.00846319270337631\\
85.69	0.00846357488347517\\
85.7	0.00846395863472572\\
85.71	0.00846434397259474\\
85.72	0.00846473091271283\\
85.73	0.00846511947087619\\
85.74	0.00846550966304832\\
85.75	0.00846590150536181\\
85.76	0.00846629501412013\\
85.77	0.00846669020579939\\
85.78	0.00846708709705019\\
85.79	0.00846748570469944\\
85.8	0.00846788604575221\\
85.81	0.00846828813739362\\
85.82	0.00846869199699073\\
85.83	0.00846909764209444\\
85.84	0.00846950509044143\\
85.85	0.00846991435995612\\
85.86	0.00847032546875261\\
85.87	0.00847073843513673\\
85.88	0.00847115327760799\\
85.89	0.00847157001486167\\
85.9	0.00847198866579081\\
85.91	0.00847240924948837\\
85.92	0.00847283178524923\\
85.93	0.0084732562925724\\
85.94	0.0084736827911631\\
85.95	0.00847411130093493\\
85.96	0.0084745418420121\\
85.97	0.00847497443473156\\
85.98	0.0084754090996453\\
85.99	0.00847584585752256\\
86	0.00847628472935214\\
86.01	0.00847672573634465\\
86.02	0.00847716889993487\\
86.03	0.00847761424178409\\
86.04	0.00847806178378249\\
86.05	0.00847851154805148\\
86.06	0.0084789635569462\\
86.07	0.0084794178330579\\
86.08	0.00847987439921644\\
86.09	0.00848033327849276\\
86.1	0.00848079449420145\\
86.11	0.0084812580699032\\
86.12	0.00848172402940749\\
86.13	0.00848219239677508\\
86.14	0.00848266319632071\\
86.15	0.00848313645261569\\
86.16	0.00848361219049065\\
86.17	0.00848409043503815\\
86.18	0.0084845712116155\\
86.19	0.00848505454584748\\
86.2	0.00848554046362912\\
86.21	0.00848602899112854\\
86.22	0.00848652015478975\\
86.23	0.00848701398133562\\
86.24	0.00848751049777065\\
86.25	0.00848800973138402\\
86.26	0.00848851170975249\\
86.27	0.00848901646074341\\
86.28	0.00848952401251772\\
86.29	0.00849003439353307\\
86.3	0.00849054763254681\\
86.31	0.00849106375861918\\
86.32	0.00849158280111642\\
86.33	0.00849210478971398\\
86.34	0.00849262975439969\\
86.35	0.00849315772547703\\
86.36	0.00849368873356841\\
86.37	0.00849422280961848\\
86.38	0.00849475998489744\\
86.39	0.00849530029100447\\
86.4	0.0084958437598711\\
86.41	0.00849639042376467\\
86.42	0.00849694031529182\\
86.43	0.00849749346740199\\
86.44	0.00849804991339098\\
86.45	0.00849860968690453\\
86.46	0.00849917282194195\\
86.47	0.00849973935285979\\
86.48	0.00850030931437549\\
86.49	0.00850088274157118\\
86.5	0.0085014596698974\\
86.51	0.00850204013517695\\
86.52	0.00850262417360868\\
86.53	0.00850321182177145\\
86.54	0.00850380311662799\\
86.55	0.0085043980955289\\
86.56	0.00850499679621665\\
86.57	0.00850559925682961\\
86.58	0.00850620551590615\\
86.59	0.00850681561238878\\
86.6	0.00850742958562826\\
86.61	0.00850804747538789\\
86.62	0.0085086693218477\\
86.63	0.00850929516560876\\
86.64	0.00850992504769754\\
86.65	0.00851055900957026\\
86.66	0.00851119709311731\\
86.67	0.00851183934066778\\
86.68	0.00851248579499388\\
86.69	0.00851313649931557\\
86.7	0.00851379149730513\\
86.71	0.00851445083309184\\
86.72	0.00851511455126663\\
86.73	0.00851578269688686\\
86.74	0.00851645531548108\\
86.75	0.00851713245305393\\
86.76	0.00851781415609093\\
86.77	0.00851850047156349\\
86.78	0.00851919144693388\\
86.79	0.00851988713016023\\
86.8	0.00852058756970166\\
86.81	0.00852129281452336\\
86.82	0.00852200081096595\\
86.83	0.00852271067630885\\
86.84	0.00852342242808432\\
86.85	0.00852413608402093\\
86.86	0.00852485166204575\\
86.87	0.00852556918028666\\
86.88	0.00852628865707464\\
86.89	0.00852701011094607\\
86.9	0.00852773356064515\\
86.91	0.0085284590251262\\
86.92	0.00852918652355613\\
86.93	0.00852991607531685\\
86.94	0.00853064770000772\\
86.95	0.0085313814174481\\
86.96	0.00853211724767978\\
86.97	0.00853285521096961\\
86.98	0.00853359532781204\\
86.99	0.00853433761893175\\
87	0.00853508210528626\\
87.01	0.00853582880806862\\
87.02	0.00853657774871011\\
87.03	0.00853732894888297\\
87.04	0.00853808243050315\\
87.05	0.00853883821573312\\
87.06	0.00853959632698468\\
87.07	0.00854035678692183\\
87.08	0.00854111961846369\\
87.09	0.00854188484478734\\
87.1	0.00854265248933088\\
87.11	0.00854342257579635\\
87.12	0.0085441951281528\\
87.13	0.00854497017063933\\
87.14	0.00854574772776818\\
87.15	0.00854652782432789\\
87.16	0.00854731048538644\\
87.17	0.0085480957362945\\
87.18	0.0085488836026886\\
87.19	0.00854967411049447\\
87.2	0.00855046728593035\\
87.21	0.00855126315551033\\
87.22	0.00855206174604772\\
87.23	0.00855286308465855\\
87.24	0.00855366719876498\\
87.25	0.00855447411609883\\
87.26	0.00855528386470517\\
87.27	0.00855609647294586\\
87.28	0.00855691196950321\\
87.29	0.00855773038338366\\
87.3	0.0085585517439215\\
87.31	0.00855937608078262\\
87.32	0.00856020342396832\\
87.33	0.00856103380381919\\
87.34	0.00856186725101894\\
87.35	0.0085627037965984\\
87.36	0.00856354347193948\\
87.37	0.00856438630877918\\
87.38	0.00856523233921371\\
87.39	0.00856608159570258\\
87.4	0.00856693411107277\\
87.41	0.00856778991852298\\
87.42	0.00856864905162786\\
87.43	0.00856951154434236\\
87.44	0.00857037743100606\\
87.45	0.00857124674634761\\
87.46	0.00857211952548922\\
87.47	0.00857299580395111\\
87.48	0.00857387561765617\\
87.49	0.0085747590029345\\
87.5	0.00857564599652815\\
87.51	0.00857653663559582\\
87.52	0.00857743095771764\\
87.53	0.00857832900090004\\
87.54	0.00857923080358062\\
87.55	0.00858013640463311\\
87.56	0.00858104584337238\\
87.57	0.00858195915955953\\
87.58	0.00858287639340697\\
87.59	0.00858379758558365\\
87.6	0.00858472277722028\\
87.61	0.00858565200991463\\
87.62	0.00858658532573693\\
87.63	0.00858752276723527\\
87.64	0.0085884643774411\\
87.65	0.00858941019987478\\
87.66	0.00859036027855121\\
87.67	0.00859131465798549\\
87.68	0.00859227338319871\\
87.69	0.00859323649972372\\
87.7	0.00859420405361104\\
87.71	0.00859517609143483\\
87.72	0.00859615266029883\\
87.73	0.00859713380784256\\
87.74	0.00859811958224737\\
87.75	0.00859911003224277\\
87.76	0.00860010520711264\\
87.77	0.00860110515670168\\
87.78	0.00860210993142181\\
87.79	0.00860311958225871\\
87.8	0.00860413416077843\\
87.81	0.00860515371913404\\
87.82	0.00860617831007239\\
87.83	0.00860720798694096\\
87.84	0.00860824280369474\\
87.85	0.00860928281490322\\
87.86	0.0086103280757575\\
87.87	0.00861137864207737\\
87.88	0.00861243457031861\\
87.89	0.00861349591758027\\
87.9	0.00861456274161209\\
87.91	0.00861563510082195\\
87.92	0.00861671305428352\\
87.93	0.00861779666174382\\
87.94	0.00861888598363107\\
87.95	0.00861998108106244\\
87.96	0.00862108201585205\\
87.97	0.00862218885051896\\
87.98	0.00862330164829528\\
87.99	0.0086244204731344\\
88	0.00862554538971927\\
88.01	0.0086266764634708\\
88.02	0.00862781376055641\\
88.03	0.00862895734789853\\
88.04	0.00863010729318338\\
88.05	0.00863126366486973\\
88.06	0.00863242653219779\\
88.07	0.00863359596519823\\
88.08	0.00863477203470126\\
88.09	0.00863595481234588\\
88.1	0.00863714437058916\\
88.11	0.00863834078271566\\
88.12	0.00863954412284701\\
88.13	0.0086407544659515\\
88.14	0.00864197188785389\\
88.15	0.00864319646524522\\
88.16	0.00864442827569284\\
88.17	0.00864566739765051\\
88.18	0.00864691391046856\\
88.19	0.00864816789440432\\
88.2	0.00864942943063247\\
88.21	0.00865069860125572\\
88.22	0.00865197548931541\\
88.23	0.00865326017880241\\
88.24	0.00865455275466803\\
88.25	0.00865585330283512\\
88.26	0.00865716191020926\\
88.27	0.00865847866469007\\
88.28	0.00865980365518275\\
88.29	0.00866113697160959\\
88.3	0.00866247870492178\\
88.31	0.00866382894711125\\
88.32	0.00866518779122265\\
88.33	0.00866655533136557\\
88.34	0.00866793166272676\\
88.35	0.0086693168815826\\
88.36	0.00867071108531168\\
88.37	0.00867211437240751\\
88.38	0.00867352684249139\\
88.39	0.00867494859632545\\
88.4	0.0086763797358258\\
88.41	0.0086778203640759\\
88.42	0.00867927058533997\\
88.43	0.00868073050507671\\
88.44	0.00868220022995304\\
88.45	0.0086836798678581\\
88.46	0.00868516952791733\\
88.47	0.00868666932050679\\
88.48	0.00868817935726762\\
88.49	0.00868969975112062\\
88.5	0.00869123061628108\\
88.51	0.00869277206827374\\
88.52	0.0086943242239479\\
88.53	0.00869588720149278\\
88.54	0.008697461120453\\
88.55	0.00869904610174424\\
88.56	0.00870064226766911\\
88.57	0.0087022497419332\\
88.58	0.00870386864966133\\
88.59	0.00870549911741393\\
88.6	0.00870714127320368\\
88.61	0.00870879524651233\\
88.62	0.00871046116830769\\
88.63	0.00871213917106081\\
88.64	0.00871382938876346\\
88.65	0.00871553195694567\\
88.66	0.00871724701269356\\
88.67	0.0087189746946674\\
88.68	0.0087207151431198\\
88.69	0.00872246849991421\\
88.7	0.00872423490854353\\
88.71	0.00872601451414906\\
88.72	0.00872780746353954\\
88.73	0.00872961390521054\\
88.74	0.00873143398936397\\
88.75	0.00873326786792793\\
88.76	0.00873511569457666\\
88.77	0.00873697762475084\\
88.78	0.00873885381567809\\
88.79	0.00874074442639366\\
88.8	0.00874264961776148\\
88.81	0.00874456955249532\\
88.82	0.00874650439518029\\
88.83	0.0087484543122946\\
88.84	0.00875041947223149\\
88.85	0.00875240004532152\\
88.86	0.00875439620385505\\
88.87	0.00875640812210503\\
88.88	0.00875843597635004\\
88.89	0.00876047994489757\\
88.9	0.00876254020810767\\
88.91	0.00876461694841674\\
88.92	0.00876671035036176\\
88.93	0.00876882060060467\\
88.94	0.00877094788795708\\
88.95	0.00877309240340537\\
88.96	0.00877525434013589\\
88.97	0.00877743389356066\\
88.98	0.00877963126134324\\
88.99	0.00878184664342496\\
89	0.00878408024205148\\
89.01	0.00878633226179958\\
89.02	0.00878860290960438\\
89.03	0.00879089239478681\\
89.04	0.00879320092908141\\
89.05	0.00879552872666452\\
89.06	0.00879787600418269\\
89.07	0.00880024298078156\\
89.08	0.008802629878135\\
89.09	0.00880503692047462\\
89.1	0.00880746433461962\\
89.11	0.00880991235000703\\
89.12	0.00881238119872224\\
89.13	0.00881487111553001\\
89.14	0.00881738233790573\\
89.15	0.0088199151060671\\
89.16	0.00882246966300622\\
89.17	0.00882504625452202\\
89.18	0.00882764512925312\\
89.19	0.00883026007577703\\
89.2	0.00883287994319188\\
89.21	0.00883550477209229\\
89.22	0.00883813460353192\\
89.23	0.00884076947902896\\
89.24	0.0088434094405717\\
89.25	0.00884605453062415\\
89.26	0.00884870479213179\\
89.27	0.00885136026852731\\
89.28	0.00885402100373645\\
89.29	0.00885668704218397\\
89.3	0.00885935842879958\\
89.31	0.00886203520902405\\
89.32	0.00886471742881531\\
89.33	0.0088674051346547\\
89.34	0.00887009837355324\\
89.35	0.00887279719305799\\
89.36	0.00887550164125851\\
89.37	0.00887821176679339\\
89.38	0.00888092761885681\\
89.39	0.00888364924720529\\
89.4	0.00888637670216439\\
89.41	0.0088891100346356\\
89.42	0.00889184929610328\\
89.43	0.00889459453864163\\
89.44	0.00889734581492185\\
89.45	0.00890010317821929\\
89.46	0.00890286668242075\\
89.47	0.00890563638203184\\
89.48	0.00890841233218444\\
89.49	0.00891119458864426\\
89.5	0.00891398320781847\\
89.51	0.00891677824676345\\
89.52	0.0089195797631926\\
89.53	0.0089223878154843\\
89.54	0.0089252024626899\\
89.55	0.00892802376454188\\
89.56	0.00893085178146206\\
89.57	0.0089336865745699\\
89.58	0.00893652820569097\\
89.59	0.00893937673736545\\
89.6	0.00894223223285679\\
89.61	0.00894509475616043\\
89.62	0.00894796437201267\\
89.63	0.00895084114589961\\
89.64	0.00895372514406623\\
89.65	0.00895661643352557\\
89.66	0.00895951508206804\\
89.67	0.0089624211582708\\
89.68	0.00896533473150729\\
89.69	0.0089682558719569\\
89.7	0.0089711846506147\\
89.71	0.00897412113930133\\
89.72	0.00897706541067301\\
89.73	0.00898001753823166\\
89.74	0.00898297759633518\\
89.75	0.00898594566020775\\
89.76	0.00898892180595045\\
89.77	0.00899190611055179\\
89.78	0.00899489865189854\\
89.79	0.00899789950878663\\
89.8	0.00900090876093215\\
89.81	0.00900392648898253\\
89.82	0.00900695277452791\\
89.83	0.00900998770011249\\
89.84	0.00901303134924621\\
89.85	0.00901608380641643\\
89.86	0.00901914515709983\\
89.87	0.00902221548777445\\
89.88	0.00902529488593185\\
89.89	0.00902838344008944\\
89.9	0.00903148123980295\\
89.91	0.00903458837567909\\
89.92	0.00903770493938834\\
89.93	0.00904083102367787\\
89.94	0.00904396672238467\\
89.95	0.00904711213044884\\
89.96	0.00905026734392701\\
89.97	0.00905343246000592\\
89.98	0.00905660757701627\\
89.99	0.00905979279444658\\
90	0.00906298821295736\\
90.01	0.00906619393439542\\
90.02	0.00906941006180826\\
90.03	0.00907263669945882\\
90.04	0.00907587395284027\\
90.05	0.009079121928691\\
90.06	0.0090823807350099\\
90.07	0.00908565048107169\\
90.08	0.00908893127744254\\
90.09	0.00909222323599583\\
90.1	0.00909552646992816\\
90.11	0.00909884109377552\\
90.12	0.00910216722342963\\
90.13	0.0091055049761546\\
90.14	0.00910885447060362\\
90.15	0.00911221582683609\\
90.16	0.0091155891663347\\
90.17	0.00911897461202296\\
90.18	0.00912237228828279\\
90.19	0.00912578232097243\\
90.2	0.00912920483744448\\
90.21	0.00913263996656426\\
90.22	0.00913608783872835\\
90.23	0.00913954858588335\\
90.24	0.0091430223415449\\
90.25	0.00914650924081695\\
90.26	0.00915000942041125\\
90.27	0.00915352301866706\\
90.28	0.0091570501755712\\
90.29	0.00916059103277819\\
90.3	0.00916414573363084\\
90.31	0.00916771442318095\\
90.32	0.00917129724821032\\
90.33	0.00917489435725203\\
90.34	0.00917850590061199\\
90.35	0.00918213203039074\\
90.36	0.00918577290050555\\
90.37	0.00918942866671277\\
90.38	0.00919309948663049\\
90.39	0.00919678551976147\\
90.4	0.00920048692751636\\
90.41	0.00920420387323723\\
90.42	0.00920793652222133\\
90.43	0.00921168504174528\\
90.44	0.00921544960108943\\
90.45	0.00921923037156261\\
90.46	0.00922302752652714\\
90.47	0.00922684124142424\\
90.48	0.00923067169379965\\
90.49	0.00923451906332965\\
90.5	0.00923838353184742\\
90.51	0.00924226528336965\\
90.52	0.00924616450412358\\
90.53	0.00925008138257434\\
90.54	0.00925401610945264\\
90.55	0.00925796887778279\\
90.56	0.00926193988291115\\
90.57	0.00926592932253484\\
90.58	0.00926993739673088\\
90.59	0.00927396430798573\\
90.6	0.0092780102612251\\
90.61	0.00928207546384423\\
90.62	0.00928616012573855\\
90.63	0.00929026445933466\\
90.64	0.00929438867962182\\
90.65	0.0092985330041837\\
90.66	0.00930269765323071\\
90.67	0.00930688284963252\\
90.68	0.00931108881895126\\
90.69	0.00931531578947488\\
90.7	0.00931956399225117\\
90.71	0.00932383366112202\\
90.72	0.00932812503275827\\
90.73	0.00933243834669493\\
90.74	0.00933677384536683\\
90.75	0.00934113177414482\\
90.76	0.00934551238137235\\
90.77	0.00934990537612363\\
90.78	0.00935429546733984\\
90.79	0.00935868259535401\\
90.8	0.0093630666997254\\
90.81	0.00936744771922968\\
90.82	0.00937182559184906\\
90.83	0.00937620025476222\\
90.84	0.00938057164433418\\
90.85	0.00938493969610593\\
90.86	0.00938930434478408\\
90.87	0.00939366552423028\\
90.88	0.00939802316745052\\
90.89	0.00940237720658429\\
90.9	0.00940672757289365\\
90.91	0.0094110741967521\\
90.92	0.00941541700763337\\
90.93	0.00941975593409999\\
90.94	0.00942409090379179\\
90.95	0.00942842184341424\\
90.96	0.00943274867872657\\
90.97	0.00943707133452987\\
90.98	0.00944138973465492\\
90.99	0.00944570380194992\\
91	0.00945001345826807\\
91.01	0.00945431862445496\\
91.02	0.00945861922033586\\
91.03	0.00946291516470275\\
91.04	0.00946720637530131\\
91.05	0.00947149276881765\\
91.06	0.00947577426086488\\
91.07	0.0094800507659696\\
91.08	0.00948432219755808\\
91.09	0.0094885884679424\\
91.1	0.00949284948830631\\
91.11	0.00949710516869098\\
91.12	0.00950135541798052\\
91.13	0.00950560014388737\\
91.14	0.00950983925293747\\
91.15	0.00951407265045523\\
91.16	0.00951830024054834\\
91.17	0.00952252192609241\\
91.18	0.00952673760871531\\
91.19	0.00953094718878146\\
91.2	0.00953515056537581\\
91.21	0.00953934763628766\\
91.22	0.00954353829799427\\
91.23	0.00954772244564426\\
91.24	0.00955189997304082\\
91.25	0.00955607077262468\\
91.26	0.00956023473545687\\
91.27	0.00956439175120129\\
91.28	0.00956854170810701\\
91.29	0.0095726844929904\\
91.3	0.00957681999121701\\
91.31	0.00958094808668321\\
91.32	0.00958506866179762\\
91.33	0.00958918159746226\\
91.34	0.00959328677305357\\
91.35	0.00959738406640306\\
91.36	0.00960147335377779\\
91.37	0.00960555450986058\\
91.38	0.00960962740773002\\
91.39	0.00961369191884014\\
91.4	0.00961774791299987\\
91.41	0.0096217952583523\\
91.42	0.00962583382135356\\
91.43	0.0096298634667515\\
91.44	0.00963388405756416\\
91.45	0.0096378954550578\\
91.46	0.00964189751872488\\
91.47	0.00964589010626152\\
91.48	0.0096498730735449\\
91.49	0.00965384627461022\\
91.5	0.00965780956162742\\
91.51	0.00966176278487764\\
91.52	0.00966570579272934\\
91.53	0.00966963843161414\\
91.54	0.00967356054600233\\
91.55	0.00967747197837812\\
91.56	0.00968137256921455\\
91.57	0.00968526215694807\\
91.58	0.00968914057795282\\
91.59	0.00969300766651462\\
91.6	0.00969686325480455\\
91.61	0.00970070717285226\\
91.62	0.00970453924851896\\
91.63	0.00970835930747001\\
91.64	0.00971216717314723\\
91.65	0.00971596266674078\\
91.66	0.00971974560716082\\
91.67	0.00972351581100865\\
91.68	0.00972727309254763\\
91.69	0.00973101726367368\\
91.7	0.00973474813388535\\
91.71	0.00973846551025365\\
91.72	0.00974216919739137\\
91.73	0.00974585899742213\\
91.74	0.00974953470994892\\
91.75	0.00975319613202236\\
91.76	0.00975684305810851\\
91.77	0.00976047528005627\\
91.78	0.0097640925870644\\
91.79	0.00976769476564812\\
91.8	0.00977128159960525\\
91.81	0.00977485286998203\\
91.82	0.00977840835503841\\
91.83	0.00978194783021292\\
91.84	0.00978547106808721\\
91.85	0.00978897783834998\\
91.86	0.00979246790776062\\
91.87	0.00979594104011227\\
91.88	0.0097993969961945\\
91.89	0.00980283553375552\\
91.9	0.00980625640746386\\
91.91	0.00980965936886966\\
91.92	0.00981304416636538\\
91.93	0.00981641054514615\\
91.94	0.00981975824716948\\
91.95	0.00982308701111459\\
91.96	0.00982639657234117\\
91.97	0.00982968666284767\\
91.98	0.00983295701122903\\
91.99	0.00983620734263391\\
92	0.0098394373787214\\
92.01	0.00984264683761719\\
92.02	0.00984583543386915\\
92.03	0.00984900287840246\\
92.04	0.00985214887847408\\
92.05	0.00985527313762675\\
92.06	0.00985837535564231\\
92.07	0.00986145522849459\\
92.08	0.00986451244830159\\
92.09	0.00986754670327715\\
92.1	0.00987055767768196\\
92.11	0.00987354505177405\\
92.12	0.00987650850175861\\
92.13	0.00987944769973716\\
92.14	0.00988236231365623\\
92.15	0.00988525200725525\\
92.16	0.00988811644001391\\
92.17	0.00989095526709884\\
92.18	0.0098937681393096\\
92.19	0.0098965547030241\\
92.2	0.00989931460014422\\
92.21	0.00990204746804165\\
92.22	0.00990475293950134\\
92.23	0.0099074306426642\\
92.24	0.00991008020096919\\
92.25	0.00991270123309459\\
92.26	0.00991529335289864\\
92.27	0.00991785616935945\\
92.28	0.00992038928651416\\
92.29	0.00992289230339732\\
92.3	0.00992536481397866\\
92.31	0.00992780640709991\\
92.32	0.00993021666641104\\
92.33	0.00993259517030559\\
92.34	0.00993494149185534\\
92.35	0.00993725519874404\\
92.36	0.00993953585320049\\
92.37	0.00994178301193074\\
92.38	0.00994399622604946\\
92.39	0.00994617504101053\\
92.4	0.00994831899653674\\
92.41	0.00995042762654871\\
92.42	0.0099525004590929\\
92.43	0.00995453701626878\\
92.44	0.00995653681415515\\
92.45	0.00995849936273552\\
92.46	0.00996042416582265\\
92.47	0.00996231072098219\\
92.48	0.00996415851945534\\
92.49	0.0099659670460807\\
92.5	0.00996773577921507\\
92.51	0.0099694641906534\\
92.52	0.00997115174554775\\
92.53	0.00997279790232527\\
92.54	0.00997440211260528\\
92.55	0.00997596382111525\\
92.56	0.00997748246560591\\
92.57	0.00997895747676527\\
92.58	0.00998038827813168\\
92.59	0.00998177428600582\\
92.6	0.00998311490936171\\
92.61	0.00998440954975659\\
92.62	0.00998565760123985\\
92.63	0.00998685845026079\\
92.64	0.00998801147557538\\
92.65	0.00998911604815188\\
92.66	0.00999017153107534\\
92.67	0.00999117727945111\\
92.68	0.00999213264030703\\
92.69	0.00999303695249467\\
92.7	0.0099938895465893\\
92.71	0.00999468974478876\\
92.72	0.00999543686081112\\
92.73	0.00999613019979123\\
92.74	0.00999676905817598\\
92.75	0.0099973527236184\\
92.76	0.00999788047487057\\
92.77	0.00999835158167526\\
92.78	0.00999876530465632\\
92.79	0.00999912089520785\\
92.8	0.00999941759538211\\
92.81	0.00999965463777609\\
92.82	0.00999983124541684\\
92.83	0.00999994663164551\\
92.84	0.01\\
92.85	0.01\\
92.86	0.01\\
92.87	0.01\\
92.88	0.01\\
92.89	0.01\\
92.9	0.01\\
92.91	0.01\\
92.92	0.01\\
92.93	0.01\\
92.94	0.01\\
92.95	0.01\\
92.96	0.01\\
92.97	0.01\\
92.98	0.01\\
92.99	0.01\\
93	0.01\\
93.01	0.01\\
93.02	0.01\\
93.03	0.01\\
93.04	0.01\\
93.05	0.01\\
93.06	0.01\\
93.07	0.01\\
93.08	0.01\\
93.09	0.01\\
93.1	0.01\\
93.11	0.01\\
93.12	0.01\\
93.13	0.01\\
93.14	0.01\\
93.15	0.01\\
93.16	0.01\\
93.17	0.01\\
93.18	0.01\\
93.19	0.01\\
93.2	0.01\\
93.21	0.01\\
93.22	0.01\\
93.23	0.01\\
93.24	0.01\\
93.25	0.01\\
93.26	0.01\\
93.27	0.01\\
93.28	0.01\\
93.29	0.01\\
93.3	0.01\\
93.31	0.01\\
93.32	0.01\\
93.33	0.01\\
93.34	0.01\\
93.35	0.01\\
93.36	0.01\\
93.37	0.01\\
93.38	0.01\\
93.39	0.01\\
93.4	0.01\\
93.41	0.01\\
93.42	0.01\\
93.43	0.01\\
93.44	0.01\\
93.45	0.01\\
93.46	0.01\\
93.47	0.01\\
93.48	0.01\\
93.49	0.01\\
93.5	0.01\\
93.51	0.01\\
93.52	0.01\\
93.53	0.01\\
93.54	0.01\\
93.55	0.01\\
93.56	0.01\\
93.57	0.01\\
93.58	0.01\\
93.59	0.01\\
93.6	0.01\\
93.61	0.01\\
93.62	0.01\\
93.63	0.01\\
93.64	0.01\\
93.65	0.01\\
93.66	0.01\\
93.67	0.01\\
93.68	0.01\\
93.69	0.01\\
93.7	0.01\\
93.71	0.01\\
93.72	0.01\\
93.73	0.01\\
93.74	0.01\\
93.75	0.01\\
93.76	0.01\\
93.77	0.01\\
93.78	0.01\\
93.79	0.01\\
93.8	0.01\\
93.81	0.01\\
93.82	0.01\\
93.83	0.01\\
93.84	0.01\\
93.85	0.01\\
93.86	0.01\\
93.87	0.01\\
93.88	0.01\\
93.89	0.01\\
93.9	0.01\\
93.91	0.01\\
93.92	0.01\\
93.93	0.01\\
93.94	0.01\\
93.95	0.01\\
93.96	0.01\\
93.97	0.01\\
93.98	0.01\\
93.99	0.01\\
94	0.01\\
94.01	0.01\\
94.02	0.01\\
94.03	0.01\\
94.04	0.01\\
94.05	0.01\\
94.06	0.01\\
94.07	0.01\\
94.08	0.01\\
94.09	0.01\\
94.1	0.01\\
94.11	0.01\\
94.12	0.01\\
94.13	0.01\\
94.14	0.01\\
94.15	0.01\\
94.16	0.01\\
94.17	0.01\\
94.18	0.01\\
94.19	0.01\\
94.2	0.01\\
94.21	0.01\\
94.22	0.01\\
94.23	0.01\\
94.24	0.01\\
94.25	0.01\\
94.26	0.01\\
94.27	0.01\\
94.28	0.01\\
94.29	0.01\\
94.3	0.01\\
94.31	0.01\\
94.32	0.01\\
94.33	0.01\\
94.34	0.01\\
94.35	0.01\\
94.36	0.01\\
94.37	0.01\\
94.38	0.01\\
94.39	0.01\\
94.4	0.01\\
94.41	0.01\\
94.42	0.01\\
94.43	0.01\\
94.44	0.01\\
94.45	0.01\\
94.46	0.01\\
94.47	0.01\\
94.48	0.01\\
94.49	0.01\\
94.5	0.01\\
94.51	0.01\\
94.52	0.01\\
94.53	0.01\\
94.54	0.01\\
94.55	0.01\\
94.56	0.01\\
94.57	0.01\\
94.58	0.01\\
94.59	0.01\\
94.6	0.01\\
94.61	0.01\\
94.62	0.01\\
94.63	0.01\\
94.64	0.01\\
94.65	0.01\\
94.66	0.01\\
94.67	0.01\\
94.68	0.01\\
94.69	0.01\\
94.7	0.01\\
94.71	0.01\\
94.72	0.01\\
94.73	0.01\\
94.74	0.01\\
94.75	0.01\\
94.76	0.01\\
94.77	0.01\\
94.78	0.01\\
94.79	0.01\\
94.8	0.01\\
94.81	0.01\\
94.82	0.01\\
94.83	0.01\\
94.84	0.01\\
94.85	0.01\\
94.86	0.01\\
94.87	0.01\\
94.88	0.01\\
94.89	0.01\\
94.9	0.01\\
94.91	0.01\\
94.92	0.01\\
94.93	0.01\\
94.94	0.01\\
94.95	0.01\\
94.96	0.01\\
94.97	0.01\\
94.98	0.01\\
94.99	0.01\\
95	0.01\\
95.01	0.01\\
95.02	0.01\\
95.03	0.01\\
95.04	0.01\\
95.05	0.01\\
95.06	0.01\\
95.07	0.01\\
95.08	0.01\\
95.09	0.01\\
95.1	0.01\\
95.11	0.01\\
95.12	0.01\\
95.13	0.01\\
95.14	0.01\\
95.15	0.01\\
95.16	0.01\\
95.17	0.01\\
95.18	0.01\\
95.19	0.01\\
95.2	0.01\\
95.21	0.01\\
95.22	0.01\\
95.23	0.01\\
95.24	0.01\\
95.25	0.01\\
95.26	0.01\\
95.27	0.01\\
95.28	0.01\\
95.29	0.01\\
95.3	0.01\\
95.31	0.01\\
95.32	0.01\\
95.33	0.01\\
95.34	0.01\\
95.35	0.01\\
95.36	0.01\\
95.37	0.01\\
95.38	0.01\\
95.39	0.01\\
95.4	0.01\\
95.41	0.01\\
95.42	0.01\\
95.43	0.01\\
95.44	0.01\\
95.45	0.01\\
95.46	0.01\\
95.47	0.01\\
95.48	0.01\\
95.49	0.01\\
95.5	0.01\\
95.51	0.01\\
95.52	0.01\\
95.53	0.01\\
95.54	0.01\\
95.55	0.01\\
95.56	0.01\\
95.57	0.01\\
95.58	0.01\\
95.59	0.01\\
95.6	0.01\\
95.61	0.01\\
95.62	0.01\\
95.63	0.01\\
95.64	0.01\\
95.65	0.01\\
95.66	0.01\\
95.67	0.01\\
95.68	0.01\\
95.69	0.01\\
95.7	0.01\\
95.71	0.01\\
95.72	0.01\\
95.73	0.01\\
95.74	0.01\\
95.75	0.01\\
95.76	0.01\\
95.77	0.01\\
95.78	0.01\\
95.79	0.01\\
95.8	0.01\\
95.81	0.01\\
95.82	0.01\\
95.83	0.01\\
95.84	0.01\\
95.85	0.01\\
95.86	0.01\\
95.87	0.01\\
95.88	0.01\\
95.89	0.01\\
95.9	0.01\\
95.91	0.01\\
95.92	0.01\\
95.93	0.01\\
95.94	0.01\\
95.95	0.01\\
95.96	0.01\\
95.97	0.01\\
95.98	0.01\\
95.99	0.01\\
96	0.01\\
96.01	0.01\\
96.02	0.01\\
96.03	0.01\\
96.04	0.01\\
96.05	0.01\\
96.06	0.01\\
96.07	0.01\\
96.08	0.01\\
96.09	0.01\\
96.1	0.01\\
96.11	0.01\\
96.12	0.01\\
96.13	0.01\\
96.14	0.01\\
96.15	0.01\\
96.16	0.01\\
96.17	0.01\\
96.18	0.01\\
96.19	0.01\\
96.2	0.01\\
96.21	0.01\\
96.22	0.01\\
96.23	0.01\\
96.24	0.01\\
96.25	0.01\\
96.26	0.01\\
96.27	0.01\\
96.28	0.01\\
96.29	0.01\\
96.3	0.01\\
96.31	0.01\\
96.32	0.01\\
96.33	0.01\\
96.34	0.01\\
96.35	0.01\\
96.36	0.01\\
96.37	0.01\\
96.38	0.01\\
96.39	0.01\\
96.4	0.01\\
96.41	0.01\\
96.42	0.01\\
96.43	0.01\\
96.44	0.01\\
96.45	0.01\\
96.46	0.01\\
96.47	0.01\\
96.48	0.01\\
96.49	0.01\\
96.5	0.01\\
96.51	0.01\\
96.52	0.01\\
96.53	0.01\\
96.54	0.01\\
96.55	0.01\\
96.56	0.01\\
96.57	0.01\\
96.58	0.01\\
96.59	0.01\\
96.6	0.01\\
96.61	0.01\\
96.62	0.01\\
96.63	0.01\\
96.64	0.01\\
96.65	0.01\\
96.66	0.01\\
96.67	0.01\\
96.68	0.01\\
96.69	0.01\\
96.7	0.01\\
96.71	0.01\\
96.72	0.01\\
96.73	0.01\\
96.74	0.01\\
96.75	0.01\\
96.76	0.01\\
96.77	0.01\\
96.78	0.01\\
96.79	0.01\\
96.8	0.01\\
96.81	0.01\\
96.82	0.01\\
96.83	0.01\\
96.84	0.01\\
96.85	0.01\\
96.86	0.01\\
96.87	0.01\\
96.88	0.01\\
96.89	0.01\\
96.9	0.01\\
96.91	0.01\\
96.92	0.01\\
96.93	0.01\\
96.94	0.01\\
96.95	0.01\\
96.96	0.01\\
96.97	0.01\\
96.98	0.01\\
96.99	0.01\\
97	0.01\\
97.01	0.01\\
97.02	0.01\\
97.03	0.01\\
97.04	0.01\\
97.05	0.01\\
97.06	0.01\\
97.07	0.01\\
97.08	0.01\\
97.09	0.01\\
97.1	0.01\\
97.11	0.01\\
97.12	0.01\\
97.13	0.01\\
97.14	0.01\\
97.15	0.01\\
97.16	0.01\\
97.17	0.01\\
97.18	0.01\\
97.19	0.01\\
97.2	0.01\\
97.21	0.01\\
97.22	0.01\\
97.23	0.01\\
97.24	0.01\\
97.25	0.01\\
97.26	0.01\\
97.27	0.01\\
97.28	0.01\\
97.29	0.01\\
97.3	0.01\\
97.31	0.01\\
97.32	0.01\\
97.33	0.01\\
97.34	0.01\\
97.35	0.01\\
97.36	0.01\\
97.37	0.01\\
97.38	0.01\\
97.39	0.01\\
97.4	0.01\\
97.41	0.01\\
97.42	0.01\\
97.43	0.01\\
97.44	0.01\\
97.45	0.01\\
97.46	0.01\\
97.47	0.01\\
97.48	0.01\\
97.49	0.01\\
97.5	0.01\\
97.51	0.01\\
97.52	0.01\\
97.53	0.01\\
97.54	0.01\\
97.55	0.01\\
97.56	0.01\\
97.57	0.01\\
97.58	0.01\\
97.59	0.01\\
97.6	0.01\\
97.61	0.01\\
97.62	0.01\\
97.63	0.01\\
97.64	0.01\\
97.65	0.01\\
97.66	0.01\\
97.67	0.01\\
97.68	0.01\\
97.69	0.01\\
97.7	0.01\\
97.71	0.01\\
97.72	0.01\\
97.73	0.01\\
97.74	0.01\\
97.75	0.01\\
97.76	0.01\\
97.77	0.01\\
97.78	0.01\\
97.79	0.01\\
97.8	0.01\\
97.81	0.01\\
97.82	0.01\\
97.83	0.01\\
97.84	0.01\\
97.85	0.01\\
97.86	0.01\\
97.87	0.01\\
97.88	0.01\\
97.89	0.01\\
97.9	0.01\\
97.91	0.01\\
97.92	0.01\\
97.93	0.01\\
97.94	0.01\\
97.95	0.01\\
97.96	0.01\\
97.97	0.01\\
97.98	0.01\\
97.99	0.01\\
98	0.01\\
98.01	0.01\\
98.02	0.01\\
98.03	0.01\\
98.04	0.01\\
98.05	0.01\\
98.06	0.01\\
98.07	0.01\\
98.08	0.01\\
98.09	0.01\\
98.1	0.01\\
98.11	0.01\\
98.12	0.01\\
98.13	0.01\\
98.14	0.01\\
98.15	0.01\\
98.16	0.01\\
98.17	0.01\\
98.18	0.01\\
98.19	0.01\\
98.2	0.01\\
98.21	0.01\\
98.22	0.01\\
98.23	0.01\\
98.24	0.01\\
98.25	0.01\\
98.26	0.01\\
98.27	0.01\\
98.28	0.01\\
98.29	0.01\\
98.3	0.01\\
98.31	0.01\\
98.32	0.01\\
98.33	0.01\\
98.34	0.01\\
98.35	0.01\\
98.36	0.01\\
98.37	0.01\\
98.38	0.01\\
98.39	0.01\\
98.4	0.01\\
98.41	0.01\\
98.42	0.01\\
98.43	0.01\\
98.44	0.01\\
98.45	0.01\\
98.46	0.01\\
98.47	0.01\\
98.48	0.01\\
98.49	0.01\\
98.5	0.01\\
98.51	0.01\\
98.52	0.01\\
98.53	0.01\\
98.54	0.01\\
98.55	0.01\\
98.56	0.01\\
98.57	0.01\\
98.58	0.01\\
98.59	0.01\\
98.6	0.01\\
98.61	0.01\\
98.62	0.01\\
98.63	0.01\\
98.64	0.01\\
98.65	0.01\\
98.66	0.01\\
98.67	0.01\\
98.68	0.01\\
98.69	0.01\\
98.7	0.01\\
98.71	0.01\\
98.72	0.01\\
98.73	0.01\\
98.74	0.01\\
98.75	0.01\\
98.76	0.01\\
98.77	0.01\\
98.78	0.01\\
98.79	0.01\\
98.8	0.01\\
98.81	0.01\\
98.82	0.01\\
98.83	0.01\\
98.84	0.01\\
98.85	0.01\\
98.86	0.01\\
98.87	0.01\\
98.88	0.01\\
98.89	0.01\\
98.9	0.01\\
98.91	0.01\\
98.92	0.01\\
98.93	0.01\\
98.94	0.01\\
98.95	0.01\\
98.96	0.01\\
98.97	0.01\\
98.98	0.01\\
98.99	0.01\\
99	0.01\\
99.01	0.01\\
99.02	0.01\\
99.03	0.01\\
99.04	0.01\\
99.05	0.01\\
99.06	0.01\\
99.07	0.01\\
99.08	0.01\\
99.09	0.01\\
99.1	0.01\\
99.11	0.01\\
99.12	0.01\\
99.13	0.01\\
99.14	0.01\\
99.15	0.01\\
99.16	0.01\\
99.17	0.01\\
99.18	0.01\\
99.19	0.01\\
99.2	0.01\\
99.21	0.01\\
99.22	0.01\\
99.23	0.01\\
99.24	0.01\\
99.25	0.01\\
99.26	0.01\\
99.27	0.01\\
99.28	0.01\\
99.29	0.01\\
99.3	0.01\\
99.31	0.01\\
99.32	0.01\\
99.33	0.01\\
99.34	0.01\\
99.35	0.01\\
99.36	0.01\\
99.37	0.01\\
99.38	0.01\\
99.39	0.01\\
99.4	0.01\\
99.41	0.01\\
99.42	0.01\\
99.43	0.01\\
99.44	0.01\\
99.45	0.01\\
99.46	0.01\\
99.47	0.01\\
99.48	0.01\\
99.49	0.01\\
99.5	0.01\\
99.51	0.01\\
99.52	0.01\\
99.53	0.01\\
99.54	0.01\\
99.55	0.01\\
99.56	0.01\\
99.57	0.01\\
99.58	0.01\\
99.59	0.01\\
99.6	0.01\\
99.61	0.01\\
99.62	0.01\\
99.63	0.01\\
99.64	0.01\\
99.65	0.01\\
99.66	0.01\\
99.67	0.01\\
99.68	0.01\\
99.69	0.01\\
99.7	0.01\\
99.71	0.01\\
99.72	0.01\\
99.73	0.01\\
99.74	0.01\\
99.75	0.01\\
99.76	0.01\\
99.77	0.01\\
99.78	0.01\\
99.79	0.01\\
99.8	0.01\\
99.81	0.01\\
99.82	0.01\\
99.83	0.01\\
99.84	0.01\\
99.85	0.01\\
99.86	0.01\\
99.87	0.01\\
99.88	0.01\\
99.89	0.01\\
99.9	0.01\\
99.91	0.01\\
99.92	0.01\\
99.93	0.01\\
99.94	0.01\\
99.95	0.01\\
99.96	0.01\\
99.97	0.01\\
99.98	0.01\\
99.99	0.01\\
100	0.01\\
};
\addlegendentry{$q=0$};

\addplot [color=blue,solid,forget plot]
  table[row sep=crcr]{%
0.01	0.01\\
0.02	0.01\\
0.03	0.01\\
0.04	0.01\\
0.05	0.01\\
0.06	0.01\\
0.07	0.01\\
0.08	0.01\\
0.09	0.01\\
0.1	0.01\\
0.11	0.01\\
0.12	0.01\\
0.13	0.01\\
0.14	0.01\\
0.15	0.01\\
0.16	0.01\\
0.17	0.01\\
0.18	0.01\\
0.19	0.01\\
0.2	0.01\\
0.21	0.01\\
0.22	0.01\\
0.23	0.01\\
0.24	0.01\\
0.25	0.01\\
0.26	0.01\\
0.27	0.01\\
0.28	0.01\\
0.29	0.01\\
0.3	0.01\\
0.31	0.01\\
0.32	0.01\\
0.33	0.01\\
0.34	0.01\\
0.35	0.01\\
0.36	0.01\\
0.37	0.01\\
0.38	0.01\\
0.39	0.01\\
0.4	0.01\\
0.41	0.01\\
0.42	0.01\\
0.43	0.01\\
0.44	0.01\\
0.45	0.01\\
0.46	0.01\\
0.47	0.01\\
0.48	0.01\\
0.49	0.01\\
0.5	0.01\\
0.51	0.01\\
0.52	0.01\\
0.53	0.01\\
0.54	0.01\\
0.55	0.01\\
0.56	0.01\\
0.57	0.01\\
0.58	0.01\\
0.59	0.01\\
0.6	0.01\\
0.61	0.01\\
0.62	0.01\\
0.63	0.01\\
0.64	0.01\\
0.65	0.01\\
0.66	0.01\\
0.67	0.01\\
0.68	0.01\\
0.69	0.01\\
0.7	0.01\\
0.71	0.01\\
0.72	0.01\\
0.73	0.01\\
0.74	0.01\\
0.75	0.01\\
0.76	0.01\\
0.77	0.01\\
0.78	0.01\\
0.79	0.01\\
0.8	0.01\\
0.81	0.01\\
0.82	0.01\\
0.83	0.01\\
0.84	0.01\\
0.85	0.01\\
0.86	0.01\\
0.87	0.01\\
0.88	0.01\\
0.89	0.01\\
0.9	0.01\\
0.91	0.01\\
0.92	0.01\\
0.93	0.01\\
0.94	0.01\\
0.95	0.01\\
0.96	0.01\\
0.97	0.01\\
0.98	0.01\\
0.99	0.01\\
1	0.01\\
1.01	0.01\\
1.02	0.01\\
1.03	0.01\\
1.04	0.01\\
1.05	0.01\\
1.06	0.01\\
1.07	0.01\\
1.08	0.01\\
1.09	0.01\\
1.1	0.01\\
1.11	0.01\\
1.12	0.01\\
1.13	0.01\\
1.14	0.01\\
1.15	0.01\\
1.16	0.01\\
1.17	0.01\\
1.18	0.01\\
1.19	0.01\\
1.2	0.01\\
1.21	0.01\\
1.22	0.01\\
1.23	0.01\\
1.24	0.01\\
1.25	0.01\\
1.26	0.01\\
1.27	0.01\\
1.28	0.01\\
1.29	0.01\\
1.3	0.01\\
1.31	0.01\\
1.32	0.01\\
1.33	0.01\\
1.34	0.01\\
1.35	0.01\\
1.36	0.01\\
1.37	0.01\\
1.38	0.01\\
1.39	0.01\\
1.4	0.01\\
1.41	0.01\\
1.42	0.01\\
1.43	0.01\\
1.44	0.01\\
1.45	0.01\\
1.46	0.01\\
1.47	0.01\\
1.48	0.01\\
1.49	0.01\\
1.5	0.01\\
1.51	0.01\\
1.52	0.01\\
1.53	0.01\\
1.54	0.01\\
1.55	0.01\\
1.56	0.01\\
1.57	0.01\\
1.58	0.01\\
1.59	0.01\\
1.6	0.01\\
1.61	0.01\\
1.62	0.01\\
1.63	0.01\\
1.64	0.01\\
1.65	0.01\\
1.66	0.01\\
1.67	0.01\\
1.68	0.01\\
1.69	0.01\\
1.7	0.01\\
1.71	0.01\\
1.72	0.01\\
1.73	0.01\\
1.74	0.01\\
1.75	0.01\\
1.76	0.01\\
1.77	0.01\\
1.78	0.01\\
1.79	0.01\\
1.8	0.01\\
1.81	0.01\\
1.82	0.01\\
1.83	0.01\\
1.84	0.01\\
1.85	0.01\\
1.86	0.01\\
1.87	0.01\\
1.88	0.01\\
1.89	0.01\\
1.9	0.01\\
1.91	0.01\\
1.92	0.01\\
1.93	0.01\\
1.94	0.01\\
1.95	0.01\\
1.96	0.01\\
1.97	0.01\\
1.98	0.01\\
1.99	0.01\\
2	0.01\\
2.01	0.01\\
2.02	0.01\\
2.03	0.01\\
2.04	0.01\\
2.05	0.01\\
2.06	0.01\\
2.07	0.01\\
2.08	0.01\\
2.09	0.01\\
2.1	0.01\\
2.11	0.01\\
2.12	0.01\\
2.13	0.01\\
2.14	0.01\\
2.15	0.01\\
2.16	0.01\\
2.17	0.01\\
2.18	0.01\\
2.19	0.01\\
2.2	0.01\\
2.21	0.01\\
2.22	0.01\\
2.23	0.01\\
2.24	0.01\\
2.25	0.01\\
2.26	0.01\\
2.27	0.01\\
2.28	0.01\\
2.29	0.01\\
2.3	0.01\\
2.31	0.01\\
2.32	0.01\\
2.33	0.01\\
2.34	0.01\\
2.35	0.01\\
2.36	0.01\\
2.37	0.01\\
2.38	0.01\\
2.39	0.01\\
2.4	0.01\\
2.41	0.01\\
2.42	0.01\\
2.43	0.01\\
2.44	0.01\\
2.45	0.01\\
2.46	0.01\\
2.47	0.01\\
2.48	0.01\\
2.49	0.01\\
2.5	0.01\\
2.51	0.01\\
2.52	0.01\\
2.53	0.01\\
2.54	0.01\\
2.55	0.01\\
2.56	0.01\\
2.57	0.01\\
2.58	0.01\\
2.59	0.01\\
2.6	0.01\\
2.61	0.01\\
2.62	0.01\\
2.63	0.01\\
2.64	0.01\\
2.65	0.01\\
2.66	0.01\\
2.67	0.01\\
2.68	0.01\\
2.69	0.01\\
2.7	0.01\\
2.71	0.01\\
2.72	0.01\\
2.73	0.01\\
2.74	0.01\\
2.75	0.01\\
2.76	0.01\\
2.77	0.01\\
2.78	0.01\\
2.79	0.01\\
2.8	0.01\\
2.81	0.01\\
2.82	0.01\\
2.83	0.01\\
2.84	0.01\\
2.85	0.01\\
2.86	0.01\\
2.87	0.01\\
2.88	0.01\\
2.89	0.01\\
2.9	0.01\\
2.91	0.01\\
2.92	0.01\\
2.93	0.01\\
2.94	0.01\\
2.95	0.01\\
2.96	0.01\\
2.97	0.01\\
2.98	0.01\\
2.99	0.01\\
3	0.01\\
3.01	0.01\\
3.02	0.01\\
3.03	0.01\\
3.04	0.01\\
3.05	0.01\\
3.06	0.01\\
3.07	0.01\\
3.08	0.01\\
3.09	0.01\\
3.1	0.01\\
3.11	0.01\\
3.12	0.01\\
3.13	0.01\\
3.14	0.01\\
3.15	0.01\\
3.16	0.01\\
3.17	0.01\\
3.18	0.01\\
3.19	0.01\\
3.2	0.01\\
3.21	0.01\\
3.22	0.01\\
3.23	0.01\\
3.24	0.01\\
3.25	0.01\\
3.26	0.01\\
3.27	0.01\\
3.28	0.01\\
3.29	0.01\\
3.3	0.01\\
3.31	0.01\\
3.32	0.01\\
3.33	0.01\\
3.34	0.01\\
3.35	0.01\\
3.36	0.01\\
3.37	0.01\\
3.38	0.01\\
3.39	0.01\\
3.4	0.01\\
3.41	0.01\\
3.42	0.01\\
3.43	0.01\\
3.44	0.01\\
3.45	0.01\\
3.46	0.01\\
3.47	0.01\\
3.48	0.01\\
3.49	0.01\\
3.5	0.01\\
3.51	0.01\\
3.52	0.01\\
3.53	0.01\\
3.54	0.01\\
3.55	0.01\\
3.56	0.01\\
3.57	0.01\\
3.58	0.01\\
3.59	0.01\\
3.6	0.01\\
3.61	0.01\\
3.62	0.01\\
3.63	0.01\\
3.64	0.01\\
3.65	0.01\\
3.66	0.01\\
3.67	0.01\\
3.68	0.01\\
3.69	0.01\\
3.7	0.01\\
3.71	0.01\\
3.72	0.01\\
3.73	0.01\\
3.74	0.01\\
3.75	0.01\\
3.76	0.01\\
3.77	0.01\\
3.78	0.01\\
3.79	0.01\\
3.8	0.01\\
3.81	0.01\\
3.82	0.01\\
3.83	0.01\\
3.84	0.01\\
3.85	0.01\\
3.86	0.01\\
3.87	0.01\\
3.88	0.01\\
3.89	0.01\\
3.9	0.01\\
3.91	0.01\\
3.92	0.01\\
3.93	0.01\\
3.94	0.01\\
3.95	0.01\\
3.96	0.01\\
3.97	0.01\\
3.98	0.01\\
3.99	0.01\\
4	0.01\\
4.01	0.01\\
4.02	0.01\\
4.03	0.01\\
4.04	0.01\\
4.05	0.01\\
4.06	0.01\\
4.07	0.01\\
4.08	0.01\\
4.09	0.01\\
4.1	0.01\\
4.11	0.01\\
4.12	0.01\\
4.13	0.01\\
4.14	0.01\\
4.15	0.01\\
4.16	0.01\\
4.17	0.01\\
4.18	0.01\\
4.19	0.01\\
4.2	0.01\\
4.21	0.01\\
4.22	0.01\\
4.23	0.01\\
4.24	0.01\\
4.25	0.01\\
4.26	0.01\\
4.27	0.01\\
4.28	0.01\\
4.29	0.01\\
4.3	0.01\\
4.31	0.01\\
4.32	0.01\\
4.33	0.01\\
4.34	0.01\\
4.35	0.01\\
4.36	0.01\\
4.37	0.01\\
4.38	0.01\\
4.39	0.01\\
4.4	0.01\\
4.41	0.01\\
4.42	0.01\\
4.43	0.01\\
4.44	0.01\\
4.45	0.01\\
4.46	0.01\\
4.47	0.01\\
4.48	0.01\\
4.49	0.01\\
4.5	0.01\\
4.51	0.01\\
4.52	0.01\\
4.53	0.01\\
4.54	0.01\\
4.55	0.01\\
4.56	0.01\\
4.57	0.01\\
4.58	0.01\\
4.59	0.01\\
4.6	0.01\\
4.61	0.01\\
4.62	0.01\\
4.63	0.01\\
4.64	0.01\\
4.65	0.01\\
4.66	0.01\\
4.67	0.01\\
4.68	0.01\\
4.69	0.01\\
4.7	0.01\\
4.71	0.01\\
4.72	0.01\\
4.73	0.01\\
4.74	0.01\\
4.75	0.01\\
4.76	0.01\\
4.77	0.01\\
4.78	0.01\\
4.79	0.01\\
4.8	0.01\\
4.81	0.01\\
4.82	0.01\\
4.83	0.01\\
4.84	0.01\\
4.85	0.01\\
4.86	0.01\\
4.87	0.01\\
4.88	0.01\\
4.89	0.01\\
4.9	0.01\\
4.91	0.01\\
4.92	0.01\\
4.93	0.01\\
4.94	0.01\\
4.95	0.01\\
4.96	0.01\\
4.97	0.01\\
4.98	0.01\\
4.99	0.01\\
5	0.01\\
5.01	0.01\\
5.02	0.01\\
5.03	0.01\\
5.04	0.01\\
5.05	0.01\\
5.06	0.01\\
5.07	0.01\\
5.08	0.01\\
5.09	0.01\\
5.1	0.01\\
5.11	0.01\\
5.12	0.01\\
5.13	0.01\\
5.14	0.01\\
5.15	0.01\\
5.16	0.01\\
5.17	0.01\\
5.18	0.01\\
5.19	0.01\\
5.2	0.01\\
5.21	0.01\\
5.22	0.01\\
5.23	0.01\\
5.24	0.01\\
5.25	0.01\\
5.26	0.01\\
5.27	0.01\\
5.28	0.01\\
5.29	0.01\\
5.3	0.01\\
5.31	0.01\\
5.32	0.01\\
5.33	0.01\\
5.34	0.01\\
5.35	0.01\\
5.36	0.01\\
5.37	0.01\\
5.38	0.01\\
5.39	0.01\\
5.4	0.01\\
5.41	0.01\\
5.42	0.01\\
5.43	0.01\\
5.44	0.01\\
5.45	0.01\\
5.46	0.01\\
5.47	0.01\\
5.48	0.01\\
5.49	0.01\\
5.5	0.01\\
5.51	0.01\\
5.52	0.01\\
5.53	0.01\\
5.54	0.01\\
5.55	0.01\\
5.56	0.01\\
5.57	0.01\\
5.58	0.01\\
5.59	0.01\\
5.6	0.01\\
5.61	0.01\\
5.62	0.01\\
5.63	0.01\\
5.64	0.01\\
5.65	0.01\\
5.66	0.01\\
5.67	0.01\\
5.68	0.01\\
5.69	0.01\\
5.7	0.01\\
5.71	0.01\\
5.72	0.01\\
5.73	0.01\\
5.74	0.01\\
5.75	0.01\\
5.76	0.01\\
5.77	0.01\\
5.78	0.01\\
5.79	0.01\\
5.8	0.01\\
5.81	0.01\\
5.82	0.01\\
5.83	0.01\\
5.84	0.01\\
5.85	0.01\\
5.86	0.01\\
5.87	0.01\\
5.88	0.01\\
5.89	0.01\\
5.9	0.01\\
5.91	0.01\\
5.92	0.01\\
5.93	0.01\\
5.94	0.01\\
5.95	0.01\\
5.96	0.01\\
5.97	0.01\\
5.98	0.01\\
5.99	0.01\\
6	0.01\\
6.01	0.01\\
6.02	0.01\\
6.03	0.01\\
6.04	0.01\\
6.05	0.01\\
6.06	0.01\\
6.07	0.01\\
6.08	0.01\\
6.09	0.01\\
6.1	0.01\\
6.11	0.01\\
6.12	0.01\\
6.13	0.01\\
6.14	0.01\\
6.15	0.01\\
6.16	0.01\\
6.17	0.01\\
6.18	0.01\\
6.19	0.01\\
6.2	0.01\\
6.21	0.01\\
6.22	0.01\\
6.23	0.01\\
6.24	0.01\\
6.25	0.01\\
6.26	0.01\\
6.27	0.01\\
6.28	0.01\\
6.29	0.01\\
6.3	0.01\\
6.31	0.01\\
6.32	0.01\\
6.33	0.01\\
6.34	0.01\\
6.35	0.01\\
6.36	0.01\\
6.37	0.01\\
6.38	0.01\\
6.39	0.01\\
6.4	0.01\\
6.41	0.01\\
6.42	0.01\\
6.43	0.01\\
6.44	0.01\\
6.45	0.01\\
6.46	0.01\\
6.47	0.01\\
6.48	0.01\\
6.49	0.01\\
6.5	0.01\\
6.51	0.01\\
6.52	0.01\\
6.53	0.01\\
6.54	0.01\\
6.55	0.01\\
6.56	0.01\\
6.57	0.01\\
6.58	0.01\\
6.59	0.01\\
6.6	0.01\\
6.61	0.01\\
6.62	0.01\\
6.63	0.01\\
6.64	0.01\\
6.65	0.01\\
6.66	0.01\\
6.67	0.01\\
6.68	0.01\\
6.69	0.01\\
6.7	0.01\\
6.71	0.01\\
6.72	0.01\\
6.73	0.01\\
6.74	0.01\\
6.75	0.01\\
6.76	0.01\\
6.77	0.01\\
6.78	0.01\\
6.79	0.01\\
6.8	0.01\\
6.81	0.01\\
6.82	0.01\\
6.83	0.01\\
6.84	0.01\\
6.85	0.01\\
6.86	0.01\\
6.87	0.01\\
6.88	0.01\\
6.89	0.01\\
6.9	0.01\\
6.91	0.01\\
6.92	0.01\\
6.93	0.01\\
6.94	0.01\\
6.95	0.01\\
6.96	0.01\\
6.97	0.01\\
6.98	0.01\\
6.99	0.01\\
7	0.01\\
7.01	0.01\\
7.02	0.01\\
7.03	0.01\\
7.04	0.01\\
7.05	0.01\\
7.06	0.01\\
7.07	0.01\\
7.08	0.01\\
7.09	0.01\\
7.1	0.01\\
7.11	0.01\\
7.12	0.01\\
7.13	0.01\\
7.14	0.01\\
7.15	0.01\\
7.16	0.01\\
7.17	0.01\\
7.18	0.01\\
7.19	0.01\\
7.2	0.01\\
7.21	0.01\\
7.22	0.01\\
7.23	0.01\\
7.24	0.01\\
7.25	0.01\\
7.26	0.01\\
7.27	0.01\\
7.28	0.01\\
7.29	0.01\\
7.3	0.01\\
7.31	0.01\\
7.32	0.01\\
7.33	0.01\\
7.34	0.01\\
7.35	0.01\\
7.36	0.01\\
7.37	0.01\\
7.38	0.01\\
7.39	0.01\\
7.4	0.01\\
7.41	0.01\\
7.42	0.01\\
7.43	0.01\\
7.44	0.01\\
7.45	0.01\\
7.46	0.01\\
7.47	0.01\\
7.48	0.01\\
7.49	0.01\\
7.5	0.01\\
7.51	0.01\\
7.52	0.01\\
7.53	0.01\\
7.54	0.01\\
7.55	0.01\\
7.56	0.01\\
7.57	0.01\\
7.58	0.01\\
7.59	0.01\\
7.6	0.01\\
7.61	0.01\\
7.62	0.01\\
7.63	0.01\\
7.64	0.01\\
7.65	0.01\\
7.66	0.01\\
7.67	0.01\\
7.68	0.01\\
7.69	0.01\\
7.7	0.01\\
7.71	0.01\\
7.72	0.01\\
7.73	0.01\\
7.74	0.01\\
7.75	0.01\\
7.76	0.01\\
7.77	0.01\\
7.78	0.01\\
7.79	0.01\\
7.8	0.01\\
7.81	0.01\\
7.82	0.01\\
7.83	0.01\\
7.84	0.01\\
7.85	0.01\\
7.86	0.01\\
7.87	0.01\\
7.88	0.01\\
7.89	0.01\\
7.9	0.01\\
7.91	0.01\\
7.92	0.01\\
7.93	0.01\\
7.94	0.01\\
7.95	0.01\\
7.96	0.01\\
7.97	0.01\\
7.98	0.01\\
7.99	0.01\\
8	0.01\\
8.01	0.01\\
8.02	0.01\\
8.03	0.01\\
8.04	0.01\\
8.05	0.01\\
8.06	0.01\\
8.07	0.01\\
8.08	0.01\\
8.09	0.01\\
8.1	0.01\\
8.11	0.01\\
8.12	0.01\\
8.13	0.01\\
8.14	0.01\\
8.15	0.01\\
8.16	0.01\\
8.17	0.01\\
8.18	0.01\\
8.19	0.01\\
8.2	0.01\\
8.21	0.01\\
8.22	0.01\\
8.23	0.01\\
8.24	0.01\\
8.25	0.01\\
8.26	0.01\\
8.27	0.01\\
8.28	0.01\\
8.29	0.01\\
8.3	0.01\\
8.31	0.01\\
8.32	0.01\\
8.33	0.01\\
8.34	0.01\\
8.35	0.01\\
8.36	0.01\\
8.37	0.01\\
8.38	0.01\\
8.39	0.01\\
8.4	0.01\\
8.41	0.01\\
8.42	0.01\\
8.43	0.01\\
8.44	0.01\\
8.45	0.01\\
8.46	0.01\\
8.47	0.01\\
8.48	0.01\\
8.49	0.01\\
8.5	0.01\\
8.51	0.01\\
8.52	0.01\\
8.53	0.01\\
8.54	0.01\\
8.55	0.01\\
8.56	0.01\\
8.57	0.01\\
8.58	0.01\\
8.59	0.01\\
8.6	0.01\\
8.61	0.01\\
8.62	0.01\\
8.63	0.01\\
8.64	0.01\\
8.65	0.01\\
8.66	0.01\\
8.67	0.01\\
8.68	0.01\\
8.69	0.01\\
8.7	0.01\\
8.71	0.01\\
8.72	0.01\\
8.73	0.01\\
8.74	0.01\\
8.75	0.01\\
8.76	0.01\\
8.77	0.01\\
8.78	0.01\\
8.79	0.01\\
8.8	0.01\\
8.81	0.01\\
8.82	0.01\\
8.83	0.01\\
8.84	0.01\\
8.85	0.01\\
8.86	0.01\\
8.87	0.01\\
8.88	0.01\\
8.89	0.01\\
8.9	0.01\\
8.91	0.01\\
8.92	0.01\\
8.93	0.01\\
8.94	0.01\\
8.95	0.01\\
8.96	0.01\\
8.97	0.01\\
8.98	0.01\\
8.99	0.01\\
9	0.01\\
9.01	0.01\\
9.02	0.01\\
9.03	0.01\\
9.04	0.01\\
9.05	0.01\\
9.06	0.01\\
9.07	0.01\\
9.08	0.01\\
9.09	0.01\\
9.1	0.01\\
9.11	0.01\\
9.12	0.01\\
9.13	0.01\\
9.14	0.01\\
9.15	0.01\\
9.16	0.01\\
9.17	0.01\\
9.18	0.01\\
9.19	0.01\\
9.2	0.01\\
9.21	0.01\\
9.22	0.01\\
9.23	0.01\\
9.24	0.01\\
9.25	0.01\\
9.26	0.01\\
9.27	0.01\\
9.28	0.01\\
9.29	0.01\\
9.3	0.01\\
9.31	0.01\\
9.32	0.01\\
9.33	0.01\\
9.34	0.01\\
9.35	0.01\\
9.36	0.01\\
9.37	0.01\\
9.38	0.01\\
9.39	0.01\\
9.4	0.01\\
9.41	0.01\\
9.42	0.01\\
9.43	0.01\\
9.44	0.01\\
9.45	0.01\\
9.46	0.01\\
9.47	0.01\\
9.48	0.01\\
9.49	0.01\\
9.5	0.01\\
9.51	0.01\\
9.52	0.01\\
9.53	0.01\\
9.54	0.01\\
9.55	0.01\\
9.56	0.01\\
9.57	0.01\\
9.58	0.01\\
9.59	0.01\\
9.6	0.01\\
9.61	0.01\\
9.62	0.01\\
9.63	0.01\\
9.64	0.01\\
9.65	0.01\\
9.66	0.01\\
9.67	0.01\\
9.68	0.01\\
9.69	0.01\\
9.7	0.01\\
9.71	0.01\\
9.72	0.01\\
9.73	0.01\\
9.74	0.01\\
9.75	0.01\\
9.76	0.01\\
9.77	0.01\\
9.78	0.01\\
9.79	0.01\\
9.8	0.01\\
9.81	0.01\\
9.82	0.01\\
9.83	0.01\\
9.84	0.01\\
9.85	0.01\\
9.86	0.01\\
9.87	0.01\\
9.88	0.01\\
9.89	0.01\\
9.9	0.01\\
9.91	0.01\\
9.92	0.01\\
9.93	0.01\\
9.94	0.01\\
9.95	0.01\\
9.96	0.01\\
9.97	0.01\\
9.98	0.01\\
9.99	0.01\\
10	0.01\\
10.01	0.01\\
10.02	0.01\\
10.03	0.01\\
10.04	0.01\\
10.05	0.01\\
10.06	0.01\\
10.07	0.01\\
10.08	0.01\\
10.09	0.01\\
10.1	0.01\\
10.11	0.01\\
10.12	0.01\\
10.13	0.01\\
10.14	0.01\\
10.15	0.01\\
10.16	0.01\\
10.17	0.01\\
10.18	0.01\\
10.19	0.01\\
10.2	0.01\\
10.21	0.01\\
10.22	0.01\\
10.23	0.01\\
10.24	0.01\\
10.25	0.01\\
10.26	0.01\\
10.27	0.01\\
10.28	0.01\\
10.29	0.01\\
10.3	0.01\\
10.31	0.01\\
10.32	0.01\\
10.33	0.01\\
10.34	0.01\\
10.35	0.01\\
10.36	0.01\\
10.37	0.01\\
10.38	0.01\\
10.39	0.01\\
10.4	0.01\\
10.41	0.01\\
10.42	0.01\\
10.43	0.01\\
10.44	0.01\\
10.45	0.01\\
10.46	0.01\\
10.47	0.01\\
10.48	0.01\\
10.49	0.01\\
10.5	0.01\\
10.51	0.01\\
10.52	0.01\\
10.53	0.01\\
10.54	0.01\\
10.55	0.01\\
10.56	0.01\\
10.57	0.01\\
10.58	0.01\\
10.59	0.01\\
10.6	0.01\\
10.61	0.01\\
10.62	0.01\\
10.63	0.01\\
10.64	0.01\\
10.65	0.01\\
10.66	0.01\\
10.67	0.01\\
10.68	0.01\\
10.69	0.01\\
10.7	0.01\\
10.71	0.01\\
10.72	0.01\\
10.73	0.01\\
10.74	0.01\\
10.75	0.01\\
10.76	0.01\\
10.77	0.01\\
10.78	0.01\\
10.79	0.01\\
10.8	0.01\\
10.81	0.01\\
10.82	0.01\\
10.83	0.01\\
10.84	0.01\\
10.85	0.01\\
10.86	0.01\\
10.87	0.01\\
10.88	0.01\\
10.89	0.01\\
10.9	0.01\\
10.91	0.01\\
10.92	0.01\\
10.93	0.01\\
10.94	0.01\\
10.95	0.01\\
10.96	0.01\\
10.97	0.01\\
10.98	0.01\\
10.99	0.01\\
11	0.01\\
11.01	0.01\\
11.02	0.01\\
11.03	0.01\\
11.04	0.01\\
11.05	0.01\\
11.06	0.01\\
11.07	0.01\\
11.08	0.01\\
11.09	0.01\\
11.1	0.01\\
11.11	0.01\\
11.12	0.01\\
11.13	0.01\\
11.14	0.01\\
11.15	0.01\\
11.16	0.01\\
11.17	0.01\\
11.18	0.01\\
11.19	0.01\\
11.2	0.01\\
11.21	0.01\\
11.22	0.01\\
11.23	0.01\\
11.24	0.01\\
11.25	0.01\\
11.26	0.01\\
11.27	0.01\\
11.28	0.01\\
11.29	0.01\\
11.3	0.01\\
11.31	0.01\\
11.32	0.01\\
11.33	0.01\\
11.34	0.01\\
11.35	0.01\\
11.36	0.01\\
11.37	0.01\\
11.38	0.01\\
11.39	0.01\\
11.4	0.01\\
11.41	0.01\\
11.42	0.01\\
11.43	0.01\\
11.44	0.01\\
11.45	0.01\\
11.46	0.01\\
11.47	0.01\\
11.48	0.01\\
11.49	0.01\\
11.5	0.01\\
11.51	0.01\\
11.52	0.01\\
11.53	0.01\\
11.54	0.01\\
11.55	0.01\\
11.56	0.01\\
11.57	0.01\\
11.58	0.01\\
11.59	0.01\\
11.6	0.01\\
11.61	0.01\\
11.62	0.01\\
11.63	0.01\\
11.64	0.01\\
11.65	0.01\\
11.66	0.01\\
11.67	0.01\\
11.68	0.01\\
11.69	0.01\\
11.7	0.01\\
11.71	0.01\\
11.72	0.01\\
11.73	0.01\\
11.74	0.01\\
11.75	0.01\\
11.76	0.01\\
11.77	0.01\\
11.78	0.01\\
11.79	0.01\\
11.8	0.01\\
11.81	0.01\\
11.82	0.01\\
11.83	0.01\\
11.84	0.01\\
11.85	0.01\\
11.86	0.01\\
11.87	0.01\\
11.88	0.01\\
11.89	0.01\\
11.9	0.01\\
11.91	0.01\\
11.92	0.01\\
11.93	0.01\\
11.94	0.01\\
11.95	0.01\\
11.96	0.01\\
11.97	0.01\\
11.98	0.01\\
11.99	0.01\\
12	0.01\\
12.01	0.01\\
12.02	0.01\\
12.03	0.01\\
12.04	0.01\\
12.05	0.01\\
12.06	0.01\\
12.07	0.01\\
12.08	0.01\\
12.09	0.01\\
12.1	0.01\\
12.11	0.01\\
12.12	0.01\\
12.13	0.01\\
12.14	0.01\\
12.15	0.01\\
12.16	0.01\\
12.17	0.01\\
12.18	0.01\\
12.19	0.01\\
12.2	0.01\\
12.21	0.01\\
12.22	0.01\\
12.23	0.01\\
12.24	0.01\\
12.25	0.01\\
12.26	0.01\\
12.27	0.01\\
12.28	0.01\\
12.29	0.01\\
12.3	0.01\\
12.31	0.01\\
12.32	0.01\\
12.33	0.01\\
12.34	0.01\\
12.35	0.01\\
12.36	0.01\\
12.37	0.01\\
12.38	0.01\\
12.39	0.01\\
12.4	0.01\\
12.41	0.01\\
12.42	0.01\\
12.43	0.01\\
12.44	0.01\\
12.45	0.01\\
12.46	0.01\\
12.47	0.01\\
12.48	0.01\\
12.49	0.01\\
12.5	0.01\\
12.51	0.01\\
12.52	0.01\\
12.53	0.01\\
12.54	0.01\\
12.55	0.01\\
12.56	0.01\\
12.57	0.01\\
12.58	0.01\\
12.59	0.01\\
12.6	0.01\\
12.61	0.01\\
12.62	0.01\\
12.63	0.01\\
12.64	0.01\\
12.65	0.01\\
12.66	0.01\\
12.67	0.01\\
12.68	0.01\\
12.69	0.01\\
12.7	0.01\\
12.71	0.01\\
12.72	0.01\\
12.73	0.01\\
12.74	0.01\\
12.75	0.01\\
12.76	0.01\\
12.77	0.01\\
12.78	0.01\\
12.79	0.01\\
12.8	0.01\\
12.81	0.01\\
12.82	0.01\\
12.83	0.01\\
12.84	0.01\\
12.85	0.01\\
12.86	0.01\\
12.87	0.01\\
12.88	0.01\\
12.89	0.01\\
12.9	0.01\\
12.91	0.01\\
12.92	0.01\\
12.93	0.01\\
12.94	0.01\\
12.95	0.01\\
12.96	0.01\\
12.97	0.01\\
12.98	0.01\\
12.99	0.01\\
13	0.01\\
13.01	0.01\\
13.02	0.01\\
13.03	0.01\\
13.04	0.01\\
13.05	0.01\\
13.06	0.01\\
13.07	0.01\\
13.08	0.01\\
13.09	0.01\\
13.1	0.01\\
13.11	0.01\\
13.12	0.01\\
13.13	0.01\\
13.14	0.01\\
13.15	0.01\\
13.16	0.01\\
13.17	0.01\\
13.18	0.01\\
13.19	0.01\\
13.2	0.01\\
13.21	0.01\\
13.22	0.01\\
13.23	0.01\\
13.24	0.01\\
13.25	0.01\\
13.26	0.01\\
13.27	0.01\\
13.28	0.01\\
13.29	0.01\\
13.3	0.01\\
13.31	0.01\\
13.32	0.01\\
13.33	0.01\\
13.34	0.01\\
13.35	0.01\\
13.36	0.01\\
13.37	0.01\\
13.38	0.01\\
13.39	0.01\\
13.4	0.01\\
13.41	0.01\\
13.42	0.01\\
13.43	0.01\\
13.44	0.01\\
13.45	0.01\\
13.46	0.01\\
13.47	0.01\\
13.48	0.01\\
13.49	0.01\\
13.5	0.01\\
13.51	0.01\\
13.52	0.01\\
13.53	0.01\\
13.54	0.01\\
13.55	0.01\\
13.56	0.01\\
13.57	0.01\\
13.58	0.01\\
13.59	0.01\\
13.6	0.01\\
13.61	0.01\\
13.62	0.01\\
13.63	0.01\\
13.64	0.01\\
13.65	0.01\\
13.66	0.01\\
13.67	0.01\\
13.68	0.01\\
13.69	0.01\\
13.7	0.01\\
13.71	0.01\\
13.72	0.01\\
13.73	0.01\\
13.74	0.01\\
13.75	0.01\\
13.76	0.01\\
13.77	0.01\\
13.78	0.01\\
13.79	0.01\\
13.8	0.01\\
13.81	0.01\\
13.82	0.01\\
13.83	0.01\\
13.84	0.01\\
13.85	0.01\\
13.86	0.01\\
13.87	0.01\\
13.88	0.01\\
13.89	0.01\\
13.9	0.01\\
13.91	0.01\\
13.92	0.01\\
13.93	0.01\\
13.94	0.01\\
13.95	0.01\\
13.96	0.01\\
13.97	0.01\\
13.98	0.01\\
13.99	0.01\\
14	0.01\\
14.01	0.01\\
14.02	0.01\\
14.03	0.01\\
14.04	0.01\\
14.05	0.01\\
14.06	0.01\\
14.07	0.01\\
14.08	0.01\\
14.09	0.01\\
14.1	0.01\\
14.11	0.01\\
14.12	0.01\\
14.13	0.01\\
14.14	0.01\\
14.15	0.01\\
14.16	0.01\\
14.17	0.01\\
14.18	0.01\\
14.19	0.01\\
14.2	0.01\\
14.21	0.01\\
14.22	0.01\\
14.23	0.01\\
14.24	0.01\\
14.25	0.01\\
14.26	0.01\\
14.27	0.01\\
14.28	0.01\\
14.29	0.01\\
14.3	0.01\\
14.31	0.01\\
14.32	0.01\\
14.33	0.01\\
14.34	0.01\\
14.35	0.01\\
14.36	0.01\\
14.37	0.01\\
14.38	0.01\\
14.39	0.01\\
14.4	0.01\\
14.41	0.01\\
14.42	0.01\\
14.43	0.01\\
14.44	0.01\\
14.45	0.01\\
14.46	0.01\\
14.47	0.01\\
14.48	0.01\\
14.49	0.01\\
14.5	0.01\\
14.51	0.01\\
14.52	0.01\\
14.53	0.01\\
14.54	0.01\\
14.55	0.01\\
14.56	0.01\\
14.57	0.01\\
14.58	0.01\\
14.59	0.01\\
14.6	0.01\\
14.61	0.01\\
14.62	0.01\\
14.63	0.01\\
14.64	0.01\\
14.65	0.01\\
14.66	0.01\\
14.67	0.01\\
14.68	0.01\\
14.69	0.01\\
14.7	0.01\\
14.71	0.01\\
14.72	0.01\\
14.73	0.01\\
14.74	0.01\\
14.75	0.01\\
14.76	0.01\\
14.77	0.01\\
14.78	0.01\\
14.79	0.01\\
14.8	0.01\\
14.81	0.01\\
14.82	0.01\\
14.83	0.01\\
14.84	0.01\\
14.85	0.01\\
14.86	0.01\\
14.87	0.01\\
14.88	0.01\\
14.89	0.01\\
14.9	0.01\\
14.91	0.01\\
14.92	0.01\\
14.93	0.01\\
14.94	0.01\\
14.95	0.01\\
14.96	0.01\\
14.97	0.01\\
14.98	0.01\\
14.99	0.01\\
15	0.01\\
15.01	0.01\\
15.02	0.01\\
15.03	0.01\\
15.04	0.01\\
15.05	0.01\\
15.06	0.01\\
15.07	0.01\\
15.08	0.01\\
15.09	0.01\\
15.1	0.01\\
15.11	0.01\\
15.12	0.01\\
15.13	0.01\\
15.14	0.01\\
15.15	0.01\\
15.16	0.01\\
15.17	0.01\\
15.18	0.01\\
15.19	0.01\\
15.2	0.01\\
15.21	0.01\\
15.22	0.01\\
15.23	0.01\\
15.24	0.01\\
15.25	0.01\\
15.26	0.01\\
15.27	0.01\\
15.28	0.01\\
15.29	0.01\\
15.3	0.01\\
15.31	0.01\\
15.32	0.01\\
15.33	0.01\\
15.34	0.01\\
15.35	0.01\\
15.36	0.01\\
15.37	0.01\\
15.38	0.01\\
15.39	0.01\\
15.4	0.01\\
15.41	0.01\\
15.42	0.01\\
15.43	0.01\\
15.44	0.01\\
15.45	0.01\\
15.46	0.01\\
15.47	0.01\\
15.48	0.01\\
15.49	0.01\\
15.5	0.01\\
15.51	0.01\\
15.52	0.01\\
15.53	0.01\\
15.54	0.01\\
15.55	0.01\\
15.56	0.01\\
15.57	0.01\\
15.58	0.01\\
15.59	0.01\\
15.6	0.01\\
15.61	0.01\\
15.62	0.01\\
15.63	0.01\\
15.64	0.01\\
15.65	0.01\\
15.66	0.01\\
15.67	0.01\\
15.68	0.01\\
15.69	0.01\\
15.7	0.01\\
15.71	0.01\\
15.72	0.01\\
15.73	0.01\\
15.74	0.01\\
15.75	0.01\\
15.76	0.01\\
15.77	0.01\\
15.78	0.01\\
15.79	0.01\\
15.8	0.01\\
15.81	0.01\\
15.82	0.01\\
15.83	0.01\\
15.84	0.01\\
15.85	0.01\\
15.86	0.01\\
15.87	0.01\\
15.88	0.01\\
15.89	0.01\\
15.9	0.01\\
15.91	0.01\\
15.92	0.01\\
15.93	0.01\\
15.94	0.01\\
15.95	0.01\\
15.96	0.01\\
15.97	0.01\\
15.98	0.01\\
15.99	0.01\\
16	0.01\\
16.01	0.01\\
16.02	0.01\\
16.03	0.01\\
16.04	0.01\\
16.05	0.01\\
16.06	0.01\\
16.07	0.01\\
16.08	0.01\\
16.09	0.01\\
16.1	0.01\\
16.11	0.01\\
16.12	0.01\\
16.13	0.01\\
16.14	0.01\\
16.15	0.01\\
16.16	0.01\\
16.17	0.01\\
16.18	0.01\\
16.19	0.01\\
16.2	0.01\\
16.21	0.01\\
16.22	0.01\\
16.23	0.01\\
16.24	0.01\\
16.25	0.01\\
16.26	0.01\\
16.27	0.01\\
16.28	0.01\\
16.29	0.01\\
16.3	0.01\\
16.31	0.01\\
16.32	0.01\\
16.33	0.01\\
16.34	0.01\\
16.35	0.01\\
16.36	0.01\\
16.37	0.01\\
16.38	0.01\\
16.39	0.01\\
16.4	0.01\\
16.41	0.01\\
16.42	0.01\\
16.43	0.01\\
16.44	0.01\\
16.45	0.01\\
16.46	0.01\\
16.47	0.01\\
16.48	0.01\\
16.49	0.01\\
16.5	0.01\\
16.51	0.01\\
16.52	0.01\\
16.53	0.01\\
16.54	0.01\\
16.55	0.01\\
16.56	0.01\\
16.57	0.01\\
16.58	0.01\\
16.59	0.01\\
16.6	0.01\\
16.61	0.01\\
16.62	0.01\\
16.63	0.01\\
16.64	0.01\\
16.65	0.01\\
16.66	0.01\\
16.67	0.01\\
16.68	0.01\\
16.69	0.01\\
16.7	0.01\\
16.71	0.01\\
16.72	0.01\\
16.73	0.01\\
16.74	0.01\\
16.75	0.01\\
16.76	0.01\\
16.77	0.01\\
16.78	0.01\\
16.79	0.01\\
16.8	0.01\\
16.81	0.01\\
16.82	0.01\\
16.83	0.01\\
16.84	0.01\\
16.85	0.01\\
16.86	0.01\\
16.87	0.01\\
16.88	0.01\\
16.89	0.01\\
16.9	0.01\\
16.91	0.01\\
16.92	0.01\\
16.93	0.01\\
16.94	0.01\\
16.95	0.01\\
16.96	0.01\\
16.97	0.01\\
16.98	0.01\\
16.99	0.01\\
17	0.01\\
17.01	0.01\\
17.02	0.01\\
17.03	0.01\\
17.04	0.01\\
17.05	0.01\\
17.06	0.01\\
17.07	0.01\\
17.08	0.01\\
17.09	0.01\\
17.1	0.01\\
17.11	0.01\\
17.12	0.01\\
17.13	0.01\\
17.14	0.01\\
17.15	0.01\\
17.16	0.01\\
17.17	0.01\\
17.18	0.01\\
17.19	0.01\\
17.2	0.01\\
17.21	0.01\\
17.22	0.01\\
17.23	0.01\\
17.24	0.01\\
17.25	0.01\\
17.26	0.01\\
17.27	0.01\\
17.28	0.01\\
17.29	0.01\\
17.3	0.01\\
17.31	0.01\\
17.32	0.01\\
17.33	0.01\\
17.34	0.01\\
17.35	0.01\\
17.36	0.01\\
17.37	0.01\\
17.38	0.01\\
17.39	0.01\\
17.4	0.01\\
17.41	0.01\\
17.42	0.01\\
17.43	0.01\\
17.44	0.01\\
17.45	0.01\\
17.46	0.01\\
17.47	0.01\\
17.48	0.01\\
17.49	0.01\\
17.5	0.01\\
17.51	0.01\\
17.52	0.01\\
17.53	0.01\\
17.54	0.01\\
17.55	0.01\\
17.56	0.01\\
17.57	0.01\\
17.58	0.01\\
17.59	0.01\\
17.6	0.01\\
17.61	0.01\\
17.62	0.01\\
17.63	0.01\\
17.64	0.01\\
17.65	0.01\\
17.66	0.01\\
17.67	0.01\\
17.68	0.01\\
17.69	0.01\\
17.7	0.01\\
17.71	0.01\\
17.72	0.01\\
17.73	0.01\\
17.74	0.01\\
17.75	0.01\\
17.76	0.01\\
17.77	0.01\\
17.78	0.01\\
17.79	0.01\\
17.8	0.01\\
17.81	0.01\\
17.82	0.01\\
17.83	0.01\\
17.84	0.01\\
17.85	0.01\\
17.86	0.01\\
17.87	0.01\\
17.88	0.01\\
17.89	0.01\\
17.9	0.01\\
17.91	0.01\\
17.92	0.01\\
17.93	0.01\\
17.94	0.01\\
17.95	0.01\\
17.96	0.01\\
17.97	0.01\\
17.98	0.01\\
17.99	0.01\\
18	0.01\\
18.01	0.01\\
18.02	0.01\\
18.03	0.01\\
18.04	0.01\\
18.05	0.01\\
18.06	0.01\\
18.07	0.01\\
18.08	0.01\\
18.09	0.01\\
18.1	0.01\\
18.11	0.01\\
18.12	0.01\\
18.13	0.01\\
18.14	0.01\\
18.15	0.01\\
18.16	0.01\\
18.17	0.01\\
18.18	0.01\\
18.19	0.01\\
18.2	0.01\\
18.21	0.01\\
18.22	0.01\\
18.23	0.01\\
18.24	0.01\\
18.25	0.01\\
18.26	0.01\\
18.27	0.01\\
18.28	0.01\\
18.29	0.01\\
18.3	0.01\\
18.31	0.01\\
18.32	0.01\\
18.33	0.01\\
18.34	0.01\\
18.35	0.01\\
18.36	0.01\\
18.37	0.01\\
18.38	0.01\\
18.39	0.01\\
18.4	0.01\\
18.41	0.01\\
18.42	0.01\\
18.43	0.01\\
18.44	0.01\\
18.45	0.01\\
18.46	0.01\\
18.47	0.01\\
18.48	0.01\\
18.49	0.01\\
18.5	0.01\\
18.51	0.01\\
18.52	0.01\\
18.53	0.01\\
18.54	0.01\\
18.55	0.01\\
18.56	0.01\\
18.57	0.01\\
18.58	0.01\\
18.59	0.01\\
18.6	0.01\\
18.61	0.01\\
18.62	0.01\\
18.63	0.01\\
18.64	0.01\\
18.65	0.01\\
18.66	0.01\\
18.67	0.01\\
18.68	0.01\\
18.69	0.01\\
18.7	0.01\\
18.71	0.01\\
18.72	0.01\\
18.73	0.01\\
18.74	0.01\\
18.75	0.01\\
18.76	0.01\\
18.77	0.01\\
18.78	0.01\\
18.79	0.01\\
18.8	0.01\\
18.81	0.01\\
18.82	0.01\\
18.83	0.01\\
18.84	0.01\\
18.85	0.01\\
18.86	0.01\\
18.87	0.01\\
18.88	0.01\\
18.89	0.01\\
18.9	0.01\\
18.91	0.01\\
18.92	0.01\\
18.93	0.01\\
18.94	0.01\\
18.95	0.01\\
18.96	0.01\\
18.97	0.01\\
18.98	0.01\\
18.99	0.01\\
19	0.01\\
19.01	0.01\\
19.02	0.01\\
19.03	0.01\\
19.04	0.01\\
19.05	0.01\\
19.06	0.01\\
19.07	0.01\\
19.08	0.01\\
19.09	0.01\\
19.1	0.01\\
19.11	0.01\\
19.12	0.01\\
19.13	0.01\\
19.14	0.01\\
19.15	0.01\\
19.16	0.01\\
19.17	0.01\\
19.18	0.01\\
19.19	0.01\\
19.2	0.01\\
19.21	0.01\\
19.22	0.01\\
19.23	0.01\\
19.24	0.01\\
19.25	0.01\\
19.26	0.01\\
19.27	0.01\\
19.28	0.01\\
19.29	0.01\\
19.3	0.01\\
19.31	0.01\\
19.32	0.01\\
19.33	0.01\\
19.34	0.01\\
19.35	0.01\\
19.36	0.01\\
19.37	0.01\\
19.38	0.01\\
19.39	0.01\\
19.4	0.01\\
19.41	0.01\\
19.42	0.01\\
19.43	0.01\\
19.44	0.01\\
19.45	0.01\\
19.46	0.01\\
19.47	0.01\\
19.48	0.01\\
19.49	0.01\\
19.5	0.01\\
19.51	0.01\\
19.52	0.01\\
19.53	0.01\\
19.54	0.01\\
19.55	0.01\\
19.56	0.01\\
19.57	0.01\\
19.58	0.01\\
19.59	0.01\\
19.6	0.01\\
19.61	0.01\\
19.62	0.01\\
19.63	0.01\\
19.64	0.01\\
19.65	0.01\\
19.66	0.01\\
19.67	0.01\\
19.68	0.01\\
19.69	0.01\\
19.7	0.01\\
19.71	0.01\\
19.72	0.01\\
19.73	0.01\\
19.74	0.01\\
19.75	0.01\\
19.76	0.01\\
19.77	0.01\\
19.78	0.01\\
19.79	0.01\\
19.8	0.01\\
19.81	0.01\\
19.82	0.01\\
19.83	0.01\\
19.84	0.01\\
19.85	0.01\\
19.86	0.01\\
19.87	0.01\\
19.88	0.01\\
19.89	0.01\\
19.9	0.01\\
19.91	0.01\\
19.92	0.01\\
19.93	0.01\\
19.94	0.01\\
19.95	0.01\\
19.96	0.01\\
19.97	0.01\\
19.98	0.01\\
19.99	0.01\\
20	0.01\\
20.01	0.01\\
20.02	0.01\\
20.03	0.01\\
20.04	0.01\\
20.05	0.01\\
20.06	0.01\\
20.07	0.01\\
20.08	0.01\\
20.09	0.01\\
20.1	0.01\\
20.11	0.01\\
20.12	0.01\\
20.13	0.01\\
20.14	0.01\\
20.15	0.01\\
20.16	0.01\\
20.17	0.01\\
20.18	0.01\\
20.19	0.01\\
20.2	0.01\\
20.21	0.01\\
20.22	0.01\\
20.23	0.01\\
20.24	0.01\\
20.25	0.01\\
20.26	0.01\\
20.27	0.01\\
20.28	0.01\\
20.29	0.01\\
20.3	0.01\\
20.31	0.01\\
20.32	0.01\\
20.33	0.01\\
20.34	0.01\\
20.35	0.01\\
20.36	0.01\\
20.37	0.01\\
20.38	0.01\\
20.39	0.01\\
20.4	0.01\\
20.41	0.01\\
20.42	0.01\\
20.43	0.01\\
20.44	0.01\\
20.45	0.01\\
20.46	0.01\\
20.47	0.01\\
20.48	0.01\\
20.49	0.01\\
20.5	0.01\\
20.51	0.01\\
20.52	0.01\\
20.53	0.01\\
20.54	0.01\\
20.55	0.01\\
20.56	0.01\\
20.57	0.01\\
20.58	0.01\\
20.59	0.01\\
20.6	0.01\\
20.61	0.01\\
20.62	0.01\\
20.63	0.01\\
20.64	0.01\\
20.65	0.01\\
20.66	0.01\\
20.67	0.01\\
20.68	0.01\\
20.69	0.01\\
20.7	0.01\\
20.71	0.01\\
20.72	0.01\\
20.73	0.01\\
20.74	0.01\\
20.75	0.01\\
20.76	0.01\\
20.77	0.01\\
20.78	0.01\\
20.79	0.01\\
20.8	0.01\\
20.81	0.01\\
20.82	0.01\\
20.83	0.01\\
20.84	0.01\\
20.85	0.01\\
20.86	0.01\\
20.87	0.01\\
20.88	0.01\\
20.89	0.01\\
20.9	0.01\\
20.91	0.01\\
20.92	0.01\\
20.93	0.01\\
20.94	0.01\\
20.95	0.01\\
20.96	0.01\\
20.97	0.01\\
20.98	0.01\\
20.99	0.01\\
21	0.01\\
21.01	0.01\\
21.02	0.01\\
21.03	0.01\\
21.04	0.01\\
21.05	0.01\\
21.06	0.01\\
21.07	0.01\\
21.08	0.01\\
21.09	0.01\\
21.1	0.01\\
21.11	0.01\\
21.12	0.01\\
21.13	0.01\\
21.14	0.01\\
21.15	0.01\\
21.16	0.01\\
21.17	0.01\\
21.18	0.01\\
21.19	0.01\\
21.2	0.01\\
21.21	0.01\\
21.22	0.01\\
21.23	0.01\\
21.24	0.01\\
21.25	0.01\\
21.26	0.01\\
21.27	0.01\\
21.28	0.01\\
21.29	0.01\\
21.3	0.01\\
21.31	0.01\\
21.32	0.01\\
21.33	0.01\\
21.34	0.01\\
21.35	0.01\\
21.36	0.01\\
21.37	0.01\\
21.38	0.01\\
21.39	0.01\\
21.4	0.01\\
21.41	0.01\\
21.42	0.01\\
21.43	0.01\\
21.44	0.01\\
21.45	0.01\\
21.46	0.01\\
21.47	0.01\\
21.48	0.01\\
21.49	0.01\\
21.5	0.01\\
21.51	0.01\\
21.52	0.01\\
21.53	0.01\\
21.54	0.01\\
21.55	0.01\\
21.56	0.01\\
21.57	0.01\\
21.58	0.01\\
21.59	0.01\\
21.6	0.01\\
21.61	0.01\\
21.62	0.01\\
21.63	0.01\\
21.64	0.01\\
21.65	0.01\\
21.66	0.01\\
21.67	0.01\\
21.68	0.01\\
21.69	0.01\\
21.7	0.01\\
21.71	0.01\\
21.72	0.01\\
21.73	0.01\\
21.74	0.01\\
21.75	0.01\\
21.76	0.01\\
21.77	0.01\\
21.78	0.01\\
21.79	0.01\\
21.8	0.01\\
21.81	0.01\\
21.82	0.01\\
21.83	0.01\\
21.84	0.01\\
21.85	0.01\\
21.86	0.01\\
21.87	0.01\\
21.88	0.01\\
21.89	0.01\\
21.9	0.01\\
21.91	0.01\\
21.92	0.01\\
21.93	0.01\\
21.94	0.01\\
21.95	0.01\\
21.96	0.01\\
21.97	0.01\\
21.98	0.01\\
21.99	0.01\\
22	0.01\\
22.01	0.01\\
22.02	0.01\\
22.03	0.01\\
22.04	0.01\\
22.05	0.01\\
22.06	0.01\\
22.07	0.01\\
22.08	0.01\\
22.09	0.01\\
22.1	0.01\\
22.11	0.01\\
22.12	0.01\\
22.13	0.01\\
22.14	0.01\\
22.15	0.01\\
22.16	0.01\\
22.17	0.01\\
22.18	0.01\\
22.19	0.01\\
22.2	0.01\\
22.21	0.01\\
22.22	0.01\\
22.23	0.01\\
22.24	0.01\\
22.25	0.01\\
22.26	0.01\\
22.27	0.01\\
22.28	0.01\\
22.29	0.01\\
22.3	0.01\\
22.31	0.01\\
22.32	0.01\\
22.33	0.01\\
22.34	0.01\\
22.35	0.01\\
22.36	0.01\\
22.37	0.01\\
22.38	0.01\\
22.39	0.01\\
22.4	0.01\\
22.41	0.01\\
22.42	0.01\\
22.43	0.01\\
22.44	0.01\\
22.45	0.01\\
22.46	0.01\\
22.47	0.01\\
22.48	0.01\\
22.49	0.01\\
22.5	0.01\\
22.51	0.01\\
22.52	0.01\\
22.53	0.01\\
22.54	0.01\\
22.55	0.01\\
22.56	0.01\\
22.57	0.01\\
22.58	0.01\\
22.59	0.01\\
22.6	0.01\\
22.61	0.01\\
22.62	0.01\\
22.63	0.01\\
22.64	0.01\\
22.65	0.01\\
22.66	0.01\\
22.67	0.01\\
22.68	0.01\\
22.69	0.01\\
22.7	0.01\\
22.71	0.01\\
22.72	0.01\\
22.73	0.01\\
22.74	0.01\\
22.75	0.01\\
22.76	0.01\\
22.77	0.01\\
22.78	0.01\\
22.79	0.01\\
22.8	0.01\\
22.81	0.01\\
22.82	0.01\\
22.83	0.01\\
22.84	0.01\\
22.85	0.01\\
22.86	0.01\\
22.87	0.01\\
22.88	0.01\\
22.89	0.01\\
22.9	0.01\\
22.91	0.01\\
22.92	0.01\\
22.93	0.01\\
22.94	0.01\\
22.95	0.01\\
22.96	0.01\\
22.97	0.01\\
22.98	0.01\\
22.99	0.01\\
23	0.01\\
23.01	0.01\\
23.02	0.01\\
23.03	0.01\\
23.04	0.01\\
23.05	0.01\\
23.06	0.01\\
23.07	0.01\\
23.08	0.01\\
23.09	0.01\\
23.1	0.01\\
23.11	0.01\\
23.12	0.01\\
23.13	0.01\\
23.14	0.01\\
23.15	0.01\\
23.16	0.01\\
23.17	0.01\\
23.18	0.01\\
23.19	0.01\\
23.2	0.01\\
23.21	0.01\\
23.22	0.01\\
23.23	0.01\\
23.24	0.01\\
23.25	0.01\\
23.26	0.01\\
23.27	0.01\\
23.28	0.01\\
23.29	0.01\\
23.3	0.01\\
23.31	0.01\\
23.32	0.01\\
23.33	0.01\\
23.34	0.01\\
23.35	0.01\\
23.36	0.01\\
23.37	0.01\\
23.38	0.01\\
23.39	0.01\\
23.4	0.01\\
23.41	0.01\\
23.42	0.01\\
23.43	0.01\\
23.44	0.01\\
23.45	0.01\\
23.46	0.01\\
23.47	0.01\\
23.48	0.01\\
23.49	0.01\\
23.5	0.01\\
23.51	0.01\\
23.52	0.01\\
23.53	0.01\\
23.54	0.01\\
23.55	0.01\\
23.56	0.01\\
23.57	0.01\\
23.58	0.01\\
23.59	0.01\\
23.6	0.01\\
23.61	0.01\\
23.62	0.01\\
23.63	0.01\\
23.64	0.01\\
23.65	0.01\\
23.66	0.01\\
23.67	0.01\\
23.68	0.01\\
23.69	0.01\\
23.7	0.01\\
23.71	0.01\\
23.72	0.01\\
23.73	0.01\\
23.74	0.01\\
23.75	0.01\\
23.76	0.01\\
23.77	0.01\\
23.78	0.01\\
23.79	0.01\\
23.8	0.01\\
23.81	0.01\\
23.82	0.01\\
23.83	0.01\\
23.84	0.01\\
23.85	0.01\\
23.86	0.01\\
23.87	0.01\\
23.88	0.01\\
23.89	0.01\\
23.9	0.01\\
23.91	0.01\\
23.92	0.01\\
23.93	0.01\\
23.94	0.01\\
23.95	0.01\\
23.96	0.01\\
23.97	0.01\\
23.98	0.01\\
23.99	0.01\\
24	0.01\\
24.01	0.01\\
24.02	0.01\\
24.03	0.01\\
24.04	0.01\\
24.05	0.01\\
24.06	0.01\\
24.07	0.01\\
24.08	0.01\\
24.09	0.01\\
24.1	0.01\\
24.11	0.01\\
24.12	0.01\\
24.13	0.01\\
24.14	0.01\\
24.15	0.01\\
24.16	0.01\\
24.17	0.01\\
24.18	0.01\\
24.19	0.01\\
24.2	0.01\\
24.21	0.01\\
24.22	0.01\\
24.23	0.01\\
24.24	0.01\\
24.25	0.01\\
24.26	0.01\\
24.27	0.01\\
24.28	0.01\\
24.29	0.01\\
24.3	0.01\\
24.31	0.01\\
24.32	0.01\\
24.33	0.01\\
24.34	0.01\\
24.35	0.01\\
24.36	0.01\\
24.37	0.01\\
24.38	0.01\\
24.39	0.01\\
24.4	0.01\\
24.41	0.01\\
24.42	0.01\\
24.43	0.01\\
24.44	0.01\\
24.45	0.01\\
24.46	0.01\\
24.47	0.01\\
24.48	0.01\\
24.49	0.01\\
24.5	0.01\\
24.51	0.01\\
24.52	0.01\\
24.53	0.01\\
24.54	0.01\\
24.55	0.01\\
24.56	0.01\\
24.57	0.01\\
24.58	0.01\\
24.59	0.01\\
24.6	0.01\\
24.61	0.01\\
24.62	0.01\\
24.63	0.01\\
24.64	0.01\\
24.65	0.01\\
24.66	0.01\\
24.67	0.01\\
24.68	0.01\\
24.69	0.01\\
24.7	0.01\\
24.71	0.01\\
24.72	0.01\\
24.73	0.01\\
24.74	0.01\\
24.75	0.01\\
24.76	0.01\\
24.77	0.01\\
24.78	0.01\\
24.79	0.01\\
24.8	0.01\\
24.81	0.01\\
24.82	0.01\\
24.83	0.01\\
24.84	0.01\\
24.85	0.01\\
24.86	0.01\\
24.87	0.01\\
24.88	0.01\\
24.89	0.01\\
24.9	0.01\\
24.91	0.01\\
24.92	0.01\\
24.93	0.01\\
24.94	0.01\\
24.95	0.01\\
24.96	0.01\\
24.97	0.01\\
24.98	0.01\\
24.99	0.01\\
25	0.01\\
25.01	0.01\\
25.02	0.01\\
25.03	0.01\\
25.04	0.01\\
25.05	0.01\\
25.06	0.01\\
25.07	0.01\\
25.08	0.01\\
25.09	0.01\\
25.1	0.01\\
25.11	0.01\\
25.12	0.01\\
25.13	0.01\\
25.14	0.01\\
25.15	0.01\\
25.16	0.01\\
25.17	0.01\\
25.18	0.01\\
25.19	0.01\\
25.2	0.01\\
25.21	0.01\\
25.22	0.01\\
25.23	0.01\\
25.24	0.01\\
25.25	0.01\\
25.26	0.01\\
25.27	0.01\\
25.28	0.01\\
25.29	0.01\\
25.3	0.01\\
25.31	0.01\\
25.32	0.01\\
25.33	0.01\\
25.34	0.01\\
25.35	0.01\\
25.36	0.01\\
25.37	0.01\\
25.38	0.01\\
25.39	0.01\\
25.4	0.01\\
25.41	0.01\\
25.42	0.01\\
25.43	0.01\\
25.44	0.01\\
25.45	0.01\\
25.46	0.01\\
25.47	0.01\\
25.48	0.01\\
25.49	0.01\\
25.5	0.01\\
25.51	0.01\\
25.52	0.01\\
25.53	0.01\\
25.54	0.01\\
25.55	0.01\\
25.56	0.01\\
25.57	0.01\\
25.58	0.01\\
25.59	0.01\\
25.6	0.01\\
25.61	0.01\\
25.62	0.01\\
25.63	0.01\\
25.64	0.01\\
25.65	0.01\\
25.66	0.01\\
25.67	0.01\\
25.68	0.01\\
25.69	0.01\\
25.7	0.01\\
25.71	0.01\\
25.72	0.01\\
25.73	0.01\\
25.74	0.01\\
25.75	0.01\\
25.76	0.01\\
25.77	0.01\\
25.78	0.01\\
25.79	0.01\\
25.8	0.01\\
25.81	0.01\\
25.82	0.01\\
25.83	0.01\\
25.84	0.01\\
25.85	0.01\\
25.86	0.01\\
25.87	0.01\\
25.88	0.01\\
25.89	0.01\\
25.9	0.01\\
25.91	0.01\\
25.92	0.01\\
25.93	0.01\\
25.94	0.01\\
25.95	0.01\\
25.96	0.01\\
25.97	0.01\\
25.98	0.01\\
25.99	0.01\\
26	0.01\\
26.01	0.01\\
26.02	0.01\\
26.03	0.01\\
26.04	0.01\\
26.05	0.01\\
26.06	0.01\\
26.07	0.01\\
26.08	0.01\\
26.09	0.01\\
26.1	0.01\\
26.11	0.01\\
26.12	0.01\\
26.13	0.01\\
26.14	0.01\\
26.15	0.01\\
26.16	0.01\\
26.17	0.01\\
26.18	0.01\\
26.19	0.01\\
26.2	0.01\\
26.21	0.01\\
26.22	0.01\\
26.23	0.01\\
26.24	0.01\\
26.25	0.01\\
26.26	0.01\\
26.27	0.01\\
26.28	0.01\\
26.29	0.01\\
26.3	0.01\\
26.31	0.01\\
26.32	0.01\\
26.33	0.01\\
26.34	0.01\\
26.35	0.01\\
26.36	0.01\\
26.37	0.01\\
26.38	0.01\\
26.39	0.01\\
26.4	0.01\\
26.41	0.01\\
26.42	0.01\\
26.43	0.01\\
26.44	0.01\\
26.45	0.01\\
26.46	0.01\\
26.47	0.01\\
26.48	0.01\\
26.49	0.01\\
26.5	0.01\\
26.51	0.01\\
26.52	0.01\\
26.53	0.01\\
26.54	0.01\\
26.55	0.01\\
26.56	0.01\\
26.57	0.01\\
26.58	0.01\\
26.59	0.01\\
26.6	0.01\\
26.61	0.01\\
26.62	0.01\\
26.63	0.01\\
26.64	0.01\\
26.65	0.01\\
26.66	0.01\\
26.67	0.01\\
26.68	0.01\\
26.69	0.01\\
26.7	0.01\\
26.71	0.01\\
26.72	0.01\\
26.73	0.01\\
26.74	0.01\\
26.75	0.01\\
26.76	0.01\\
26.77	0.01\\
26.78	0.01\\
26.79	0.01\\
26.8	0.01\\
26.81	0.01\\
26.82	0.01\\
26.83	0.01\\
26.84	0.01\\
26.85	0.01\\
26.86	0.01\\
26.87	0.01\\
26.88	0.01\\
26.89	0.01\\
26.9	0.01\\
26.91	0.01\\
26.92	0.01\\
26.93	0.01\\
26.94	0.01\\
26.95	0.01\\
26.96	0.01\\
26.97	0.01\\
26.98	0.01\\
26.99	0.01\\
27	0.01\\
27.01	0.01\\
27.02	0.01\\
27.03	0.01\\
27.04	0.01\\
27.05	0.01\\
27.06	0.01\\
27.07	0.01\\
27.08	0.01\\
27.09	0.01\\
27.1	0.01\\
27.11	0.01\\
27.12	0.01\\
27.13	0.01\\
27.14	0.01\\
27.15	0.01\\
27.16	0.01\\
27.17	0.01\\
27.18	0.01\\
27.19	0.01\\
27.2	0.01\\
27.21	0.01\\
27.22	0.01\\
27.23	0.01\\
27.24	0.01\\
27.25	0.01\\
27.26	0.01\\
27.27	0.01\\
27.28	0.01\\
27.29	0.01\\
27.3	0.01\\
27.31	0.01\\
27.32	0.01\\
27.33	0.01\\
27.34	0.01\\
27.35	0.01\\
27.36	0.01\\
27.37	0.01\\
27.38	0.01\\
27.39	0.01\\
27.4	0.01\\
27.41	0.01\\
27.42	0.01\\
27.43	0.01\\
27.44	0.01\\
27.45	0.01\\
27.46	0.01\\
27.47	0.01\\
27.48	0.01\\
27.49	0.01\\
27.5	0.01\\
27.51	0.01\\
27.52	0.01\\
27.53	0.01\\
27.54	0.01\\
27.55	0.01\\
27.56	0.01\\
27.57	0.01\\
27.58	0.01\\
27.59	0.01\\
27.6	0.01\\
27.61	0.01\\
27.62	0.01\\
27.63	0.01\\
27.64	0.01\\
27.65	0.01\\
27.66	0.01\\
27.67	0.01\\
27.68	0.01\\
27.69	0.01\\
27.7	0.01\\
27.71	0.01\\
27.72	0.01\\
27.73	0.01\\
27.74	0.01\\
27.75	0.01\\
27.76	0.01\\
27.77	0.01\\
27.78	0.01\\
27.79	0.01\\
27.8	0.01\\
27.81	0.01\\
27.82	0.01\\
27.83	0.01\\
27.84	0.01\\
27.85	0.01\\
27.86	0.01\\
27.87	0.01\\
27.88	0.01\\
27.89	0.01\\
27.9	0.01\\
27.91	0.01\\
27.92	0.01\\
27.93	0.01\\
27.94	0.01\\
27.95	0.01\\
27.96	0.01\\
27.97	0.01\\
27.98	0.01\\
27.99	0.01\\
28	0.01\\
28.01	0.01\\
28.02	0.01\\
28.03	0.01\\
28.04	0.01\\
28.05	0.01\\
28.06	0.01\\
28.07	0.01\\
28.08	0.01\\
28.09	0.01\\
28.1	0.01\\
28.11	0.01\\
28.12	0.01\\
28.13	0.01\\
28.14	0.01\\
28.15	0.01\\
28.16	0.01\\
28.17	0.01\\
28.18	0.01\\
28.19	0.01\\
28.2	0.01\\
28.21	0.01\\
28.22	0.01\\
28.23	0.01\\
28.24	0.01\\
28.25	0.01\\
28.26	0.01\\
28.27	0.01\\
28.28	0.01\\
28.29	0.01\\
28.3	0.01\\
28.31	0.01\\
28.32	0.01\\
28.33	0.01\\
28.34	0.01\\
28.35	0.01\\
28.36	0.01\\
28.37	0.01\\
28.38	0.01\\
28.39	0.01\\
28.4	0.01\\
28.41	0.01\\
28.42	0.01\\
28.43	0.01\\
28.44	0.01\\
28.45	0.01\\
28.46	0.01\\
28.47	0.01\\
28.48	0.01\\
28.49	0.01\\
28.5	0.01\\
28.51	0.01\\
28.52	0.01\\
28.53	0.01\\
28.54	0.01\\
28.55	0.01\\
28.56	0.01\\
28.57	0.01\\
28.58	0.01\\
28.59	0.01\\
28.6	0.01\\
28.61	0.01\\
28.62	0.01\\
28.63	0.01\\
28.64	0.01\\
28.65	0.01\\
28.66	0.01\\
28.67	0.01\\
28.68	0.01\\
28.69	0.01\\
28.7	0.01\\
28.71	0.01\\
28.72	0.01\\
28.73	0.01\\
28.74	0.01\\
28.75	0.01\\
28.76	0.01\\
28.77	0.01\\
28.78	0.01\\
28.79	0.01\\
28.8	0.01\\
28.81	0.01\\
28.82	0.01\\
28.83	0.01\\
28.84	0.01\\
28.85	0.01\\
28.86	0.01\\
28.87	0.01\\
28.88	0.01\\
28.89	0.01\\
28.9	0.01\\
28.91	0.01\\
28.92	0.01\\
28.93	0.01\\
28.94	0.01\\
28.95	0.01\\
28.96	0.01\\
28.97	0.01\\
28.98	0.01\\
28.99	0.01\\
29	0.01\\
29.01	0.01\\
29.02	0.01\\
29.03	0.01\\
29.04	0.01\\
29.05	0.01\\
29.06	0.01\\
29.07	0.01\\
29.08	0.01\\
29.09	0.01\\
29.1	0.01\\
29.11	0.01\\
29.12	0.01\\
29.13	0.01\\
29.14	0.01\\
29.15	0.01\\
29.16	0.01\\
29.17	0.01\\
29.18	0.01\\
29.19	0.01\\
29.2	0.01\\
29.21	0.01\\
29.22	0.01\\
29.23	0.01\\
29.24	0.01\\
29.25	0.01\\
29.26	0.01\\
29.27	0.01\\
29.28	0.01\\
29.29	0.01\\
29.3	0.01\\
29.31	0.01\\
29.32	0.01\\
29.33	0.01\\
29.34	0.01\\
29.35	0.01\\
29.36	0.01\\
29.37	0.01\\
29.38	0.01\\
29.39	0.01\\
29.4	0.01\\
29.41	0.01\\
29.42	0.01\\
29.43	0.01\\
29.44	0.01\\
29.45	0.01\\
29.46	0.01\\
29.47	0.01\\
29.48	0.01\\
29.49	0.01\\
29.5	0.01\\
29.51	0.01\\
29.52	0.01\\
29.53	0.01\\
29.54	0.01\\
29.55	0.01\\
29.56	0.01\\
29.57	0.01\\
29.58	0.01\\
29.59	0.01\\
29.6	0.01\\
29.61	0.01\\
29.62	0.01\\
29.63	0.01\\
29.64	0.01\\
29.65	0.01\\
29.66	0.01\\
29.67	0.01\\
29.68	0.01\\
29.69	0.01\\
29.7	0.01\\
29.71	0.01\\
29.72	0.01\\
29.73	0.01\\
29.74	0.01\\
29.75	0.01\\
29.76	0.01\\
29.77	0.01\\
29.78	0.01\\
29.79	0.01\\
29.8	0.01\\
29.81	0.01\\
29.82	0.01\\
29.83	0.01\\
29.84	0.01\\
29.85	0.01\\
29.86	0.01\\
29.87	0.01\\
29.88	0.01\\
29.89	0.01\\
29.9	0.01\\
29.91	0.01\\
29.92	0.01\\
29.93	0.01\\
29.94	0.01\\
29.95	0.01\\
29.96	0.01\\
29.97	0.01\\
29.98	0.01\\
29.99	0.01\\
30	0.01\\
30.01	0.01\\
30.02	0.01\\
30.03	0.01\\
30.04	0.01\\
30.05	0.01\\
30.06	0.01\\
30.07	0.01\\
30.08	0.01\\
30.09	0.01\\
30.1	0.01\\
30.11	0.01\\
30.12	0.01\\
30.13	0.01\\
30.14	0.01\\
30.15	0.01\\
30.16	0.01\\
30.17	0.01\\
30.18	0.01\\
30.19	0.01\\
30.2	0.01\\
30.21	0.01\\
30.22	0.01\\
30.23	0.01\\
30.24	0.01\\
30.25	0.01\\
30.26	0.01\\
30.27	0.01\\
30.28	0.01\\
30.29	0.01\\
30.3	0.01\\
30.31	0.01\\
30.32	0.01\\
30.33	0.01\\
30.34	0.01\\
30.35	0.01\\
30.36	0.01\\
30.37	0.01\\
30.38	0.01\\
30.39	0.01\\
30.4	0.01\\
30.41	0.01\\
30.42	0.01\\
30.43	0.01\\
30.44	0.01\\
30.45	0.01\\
30.46	0.01\\
30.47	0.01\\
30.48	0.01\\
30.49	0.01\\
30.5	0.01\\
30.51	0.01\\
30.52	0.01\\
30.53	0.01\\
30.54	0.01\\
30.55	0.01\\
30.56	0.01\\
30.57	0.01\\
30.58	0.01\\
30.59	0.01\\
30.6	0.01\\
30.61	0.01\\
30.62	0.01\\
30.63	0.01\\
30.64	0.01\\
30.65	0.01\\
30.66	0.01\\
30.67	0.01\\
30.68	0.01\\
30.69	0.01\\
30.7	0.01\\
30.71	0.01\\
30.72	0.01\\
30.73	0.01\\
30.74	0.01\\
30.75	0.01\\
30.76	0.01\\
30.77	0.01\\
30.78	0.01\\
30.79	0.01\\
30.8	0.01\\
30.81	0.01\\
30.82	0.01\\
30.83	0.01\\
30.84	0.01\\
30.85	0.01\\
30.86	0.01\\
30.87	0.01\\
30.88	0.01\\
30.89	0.01\\
30.9	0.01\\
30.91	0.01\\
30.92	0.01\\
30.93	0.01\\
30.94	0.01\\
30.95	0.01\\
30.96	0.01\\
30.97	0.01\\
30.98	0.01\\
30.99	0.01\\
31	0.01\\
31.01	0.01\\
31.02	0.01\\
31.03	0.01\\
31.04	0.01\\
31.05	0.01\\
31.06	0.01\\
31.07	0.01\\
31.08	0.01\\
31.09	0.01\\
31.1	0.01\\
31.11	0.01\\
31.12	0.01\\
31.13	0.01\\
31.14	0.01\\
31.15	0.01\\
31.16	0.01\\
31.17	0.01\\
31.18	0.01\\
31.19	0.01\\
31.2	0.01\\
31.21	0.01\\
31.22	0.01\\
31.23	0.01\\
31.24	0.01\\
31.25	0.01\\
31.26	0.01\\
31.27	0.01\\
31.28	0.01\\
31.29	0.01\\
31.3	0.01\\
31.31	0.01\\
31.32	0.01\\
31.33	0.01\\
31.34	0.01\\
31.35	0.01\\
31.36	0.01\\
31.37	0.01\\
31.38	0.01\\
31.39	0.01\\
31.4	0.01\\
31.41	0.01\\
31.42	0.01\\
31.43	0.01\\
31.44	0.01\\
31.45	0.01\\
31.46	0.01\\
31.47	0.01\\
31.48	0.01\\
31.49	0.01\\
31.5	0.01\\
31.51	0.01\\
31.52	0.01\\
31.53	0.01\\
31.54	0.01\\
31.55	0.01\\
31.56	0.01\\
31.57	0.01\\
31.58	0.01\\
31.59	0.01\\
31.6	0.01\\
31.61	0.01\\
31.62	0.01\\
31.63	0.01\\
31.64	0.01\\
31.65	0.01\\
31.66	0.01\\
31.67	0.01\\
31.68	0.01\\
31.69	0.01\\
31.7	0.01\\
31.71	0.01\\
31.72	0.01\\
31.73	0.01\\
31.74	0.01\\
31.75	0.01\\
31.76	0.01\\
31.77	0.01\\
31.78	0.01\\
31.79	0.01\\
31.8	0.01\\
31.81	0.01\\
31.82	0.01\\
31.83	0.01\\
31.84	0.01\\
31.85	0.01\\
31.86	0.01\\
31.87	0.01\\
31.88	0.01\\
31.89	0.01\\
31.9	0.01\\
31.91	0.01\\
31.92	0.01\\
31.93	0.01\\
31.94	0.01\\
31.95	0.01\\
31.96	0.01\\
31.97	0.01\\
31.98	0.01\\
31.99	0.01\\
32	0.01\\
32.01	0.01\\
32.02	0.01\\
32.03	0.01\\
32.04	0.01\\
32.05	0.01\\
32.06	0.01\\
32.07	0.01\\
32.08	0.01\\
32.09	0.01\\
32.1	0.01\\
32.11	0.01\\
32.12	0.01\\
32.13	0.01\\
32.14	0.01\\
32.15	0.01\\
32.16	0.01\\
32.17	0.01\\
32.18	0.01\\
32.19	0.01\\
32.2	0.01\\
32.21	0.01\\
32.22	0.01\\
32.23	0.01\\
32.24	0.01\\
32.25	0.01\\
32.26	0.01\\
32.27	0.01\\
32.28	0.01\\
32.29	0.01\\
32.3	0.01\\
32.31	0.01\\
32.32	0.01\\
32.33	0.01\\
32.34	0.01\\
32.35	0.01\\
32.36	0.01\\
32.37	0.01\\
32.38	0.01\\
32.39	0.01\\
32.4	0.01\\
32.41	0.01\\
32.42	0.01\\
32.43	0.01\\
32.44	0.01\\
32.45	0.01\\
32.46	0.01\\
32.47	0.01\\
32.48	0.01\\
32.49	0.01\\
32.5	0.01\\
32.51	0.01\\
32.52	0.01\\
32.53	0.01\\
32.54	0.01\\
32.55	0.01\\
32.56	0.01\\
32.57	0.01\\
32.58	0.01\\
32.59	0.01\\
32.6	0.01\\
32.61	0.01\\
32.62	0.01\\
32.63	0.01\\
32.64	0.01\\
32.65	0.01\\
32.66	0.01\\
32.67	0.01\\
32.68	0.01\\
32.69	0.01\\
32.7	0.01\\
32.71	0.01\\
32.72	0.01\\
32.73	0.01\\
32.74	0.01\\
32.75	0.01\\
32.76	0.01\\
32.77	0.01\\
32.78	0.01\\
32.79	0.01\\
32.8	0.01\\
32.81	0.01\\
32.82	0.01\\
32.83	0.01\\
32.84	0.01\\
32.85	0.01\\
32.86	0.01\\
32.87	0.01\\
32.88	0.01\\
32.89	0.01\\
32.9	0.01\\
32.91	0.01\\
32.92	0.01\\
32.93	0.01\\
32.94	0.01\\
32.95	0.01\\
32.96	0.01\\
32.97	0.01\\
32.98	0.01\\
32.99	0.01\\
33	0.01\\
33.01	0.01\\
33.02	0.01\\
33.03	0.01\\
33.04	0.01\\
33.05	0.01\\
33.06	0.01\\
33.07	0.01\\
33.08	0.01\\
33.09	0.01\\
33.1	0.01\\
33.11	0.01\\
33.12	0.01\\
33.13	0.01\\
33.14	0.01\\
33.15	0.01\\
33.16	0.01\\
33.17	0.01\\
33.18	0.01\\
33.19	0.01\\
33.2	0.01\\
33.21	0.01\\
33.22	0.01\\
33.23	0.01\\
33.24	0.01\\
33.25	0.01\\
33.26	0.01\\
33.27	0.01\\
33.28	0.01\\
33.29	0.01\\
33.3	0.01\\
33.31	0.01\\
33.32	0.01\\
33.33	0.01\\
33.34	0.01\\
33.35	0.01\\
33.36	0.01\\
33.37	0.01\\
33.38	0.01\\
33.39	0.01\\
33.4	0.01\\
33.41	0.01\\
33.42	0.01\\
33.43	0.01\\
33.44	0.01\\
33.45	0.01\\
33.46	0.01\\
33.47	0.01\\
33.48	0.01\\
33.49	0.01\\
33.5	0.01\\
33.51	0.01\\
33.52	0.01\\
33.53	0.01\\
33.54	0.01\\
33.55	0.01\\
33.56	0.01\\
33.57	0.01\\
33.58	0.01\\
33.59	0.01\\
33.6	0.01\\
33.61	0.01\\
33.62	0.01\\
33.63	0.01\\
33.64	0.01\\
33.65	0.01\\
33.66	0.01\\
33.67	0.01\\
33.68	0.01\\
33.69	0.01\\
33.7	0.01\\
33.71	0.01\\
33.72	0.01\\
33.73	0.01\\
33.74	0.01\\
33.75	0.01\\
33.76	0.01\\
33.77	0.01\\
33.78	0.01\\
33.79	0.01\\
33.8	0.01\\
33.81	0.01\\
33.82	0.01\\
33.83	0.01\\
33.84	0.01\\
33.85	0.01\\
33.86	0.01\\
33.87	0.01\\
33.88	0.01\\
33.89	0.01\\
33.9	0.01\\
33.91	0.01\\
33.92	0.01\\
33.93	0.01\\
33.94	0.01\\
33.95	0.01\\
33.96	0.01\\
33.97	0.01\\
33.98	0.01\\
33.99	0.01\\
34	0.01\\
34.01	0.01\\
34.02	0.01\\
34.03	0.01\\
34.04	0.01\\
34.05	0.01\\
34.06	0.01\\
34.07	0.01\\
34.08	0.01\\
34.09	0.01\\
34.1	0.01\\
34.11	0.01\\
34.12	0.01\\
34.13	0.01\\
34.14	0.01\\
34.15	0.01\\
34.16	0.01\\
34.17	0.01\\
34.18	0.01\\
34.19	0.01\\
34.2	0.01\\
34.21	0.01\\
34.22	0.01\\
34.23	0.01\\
34.24	0.01\\
34.25	0.01\\
34.26	0.01\\
34.27	0.01\\
34.28	0.01\\
34.29	0.01\\
34.3	0.01\\
34.31	0.01\\
34.32	0.01\\
34.33	0.01\\
34.34	0.01\\
34.35	0.01\\
34.36	0.01\\
34.37	0.01\\
34.38	0.01\\
34.39	0.01\\
34.4	0.01\\
34.41	0.01\\
34.42	0.01\\
34.43	0.01\\
34.44	0.01\\
34.45	0.01\\
34.46	0.01\\
34.47	0.01\\
34.48	0.01\\
34.49	0.01\\
34.5	0.01\\
34.51	0.01\\
34.52	0.01\\
34.53	0.01\\
34.54	0.01\\
34.55	0.01\\
34.56	0.01\\
34.57	0.01\\
34.58	0.01\\
34.59	0.01\\
34.6	0.01\\
34.61	0.01\\
34.62	0.01\\
34.63	0.01\\
34.64	0.01\\
34.65	0.01\\
34.66	0.01\\
34.67	0.01\\
34.68	0.01\\
34.69	0.01\\
34.7	0.01\\
34.71	0.01\\
34.72	0.01\\
34.73	0.01\\
34.74	0.01\\
34.75	0.01\\
34.76	0.01\\
34.77	0.01\\
34.78	0.01\\
34.79	0.01\\
34.8	0.01\\
34.81	0.01\\
34.82	0.01\\
34.83	0.01\\
34.84	0.01\\
34.85	0.01\\
34.86	0.01\\
34.87	0.01\\
34.88	0.01\\
34.89	0.01\\
34.9	0.01\\
34.91	0.01\\
34.92	0.01\\
34.93	0.01\\
34.94	0.01\\
34.95	0.01\\
34.96	0.01\\
34.97	0.01\\
34.98	0.01\\
34.99	0.01\\
35	0.01\\
35.01	0.01\\
35.02	0.01\\
35.03	0.01\\
35.04	0.01\\
35.05	0.01\\
35.06	0.01\\
35.07	0.01\\
35.08	0.01\\
35.09	0.01\\
35.1	0.01\\
35.11	0.01\\
35.12	0.01\\
35.13	0.01\\
35.14	0.01\\
35.15	0.01\\
35.16	0.01\\
35.17	0.01\\
35.18	0.01\\
35.19	0.01\\
35.2	0.01\\
35.21	0.01\\
35.22	0.01\\
35.23	0.01\\
35.24	0.01\\
35.25	0.01\\
35.26	0.01\\
35.27	0.01\\
35.28	0.01\\
35.29	0.01\\
35.3	0.01\\
35.31	0.01\\
35.32	0.01\\
35.33	0.01\\
35.34	0.01\\
35.35	0.01\\
35.36	0.01\\
35.37	0.01\\
35.38	0.01\\
35.39	0.01\\
35.4	0.01\\
35.41	0.01\\
35.42	0.01\\
35.43	0.01\\
35.44	0.01\\
35.45	0.01\\
35.46	0.01\\
35.47	0.01\\
35.48	0.01\\
35.49	0.01\\
35.5	0.01\\
35.51	0.01\\
35.52	0.01\\
35.53	0.01\\
35.54	0.01\\
35.55	0.01\\
35.56	0.01\\
35.57	0.01\\
35.58	0.01\\
35.59	0.01\\
35.6	0.01\\
35.61	0.01\\
35.62	0.01\\
35.63	0.01\\
35.64	0.01\\
35.65	0.01\\
35.66	0.01\\
35.67	0.01\\
35.68	0.01\\
35.69	0.01\\
35.7	0.01\\
35.71	0.01\\
35.72	0.01\\
35.73	0.01\\
35.74	0.01\\
35.75	0.01\\
35.76	0.01\\
35.77	0.01\\
35.78	0.01\\
35.79	0.01\\
35.8	0.01\\
35.81	0.01\\
35.82	0.01\\
35.83	0.01\\
35.84	0.01\\
35.85	0.01\\
35.86	0.01\\
35.87	0.01\\
35.88	0.01\\
35.89	0.01\\
35.9	0.01\\
35.91	0.01\\
35.92	0.01\\
35.93	0.01\\
35.94	0.01\\
35.95	0.01\\
35.96	0.01\\
35.97	0.01\\
35.98	0.01\\
35.99	0.01\\
36	0.01\\
36.01	0.01\\
36.02	0.01\\
36.03	0.01\\
36.04	0.01\\
36.05	0.01\\
36.06	0.01\\
36.07	0.01\\
36.08	0.01\\
36.09	0.01\\
36.1	0.01\\
36.11	0.01\\
36.12	0.01\\
36.13	0.01\\
36.14	0.01\\
36.15	0.01\\
36.16	0.01\\
36.17	0.01\\
36.18	0.01\\
36.19	0.01\\
36.2	0.01\\
36.21	0.01\\
36.22	0.01\\
36.23	0.01\\
36.24	0.01\\
36.25	0.01\\
36.26	0.01\\
36.27	0.01\\
36.28	0.01\\
36.29	0.01\\
36.3	0.01\\
36.31	0.01\\
36.32	0.01\\
36.33	0.01\\
36.34	0.01\\
36.35	0.01\\
36.36	0.01\\
36.37	0.01\\
36.38	0.01\\
36.39	0.01\\
36.4	0.01\\
36.41	0.01\\
36.42	0.01\\
36.43	0.01\\
36.44	0.01\\
36.45	0.01\\
36.46	0.01\\
36.47	0.01\\
36.48	0.01\\
36.49	0.01\\
36.5	0.01\\
36.51	0.01\\
36.52	0.01\\
36.53	0.01\\
36.54	0.01\\
36.55	0.01\\
36.56	0.01\\
36.57	0.01\\
36.58	0.01\\
36.59	0.01\\
36.6	0.01\\
36.61	0.01\\
36.62	0.01\\
36.63	0.01\\
36.64	0.01\\
36.65	0.01\\
36.66	0.01\\
36.67	0.01\\
36.68	0.01\\
36.69	0.01\\
36.7	0.01\\
36.71	0.01\\
36.72	0.01\\
36.73	0.01\\
36.74	0.01\\
36.75	0.01\\
36.76	0.01\\
36.77	0.01\\
36.78	0.01\\
36.79	0.01\\
36.8	0.01\\
36.81	0.01\\
36.82	0.01\\
36.83	0.01\\
36.84	0.01\\
36.85	0.01\\
36.86	0.01\\
36.87	0.01\\
36.88	0.01\\
36.89	0.01\\
36.9	0.01\\
36.91	0.01\\
36.92	0.01\\
36.93	0.01\\
36.94	0.01\\
36.95	0.01\\
36.96	0.01\\
36.97	0.01\\
36.98	0.01\\
36.99	0.01\\
37	0.01\\
37.01	0.01\\
37.02	0.01\\
37.03	0.01\\
37.04	0.01\\
37.05	0.01\\
37.06	0.01\\
37.07	0.01\\
37.08	0.01\\
37.09	0.01\\
37.1	0.01\\
37.11	0.01\\
37.12	0.01\\
37.13	0.01\\
37.14	0.01\\
37.15	0.01\\
37.16	0.01\\
37.17	0.01\\
37.18	0.01\\
37.19	0.01\\
37.2	0.01\\
37.21	0.01\\
37.22	0.01\\
37.23	0.01\\
37.24	0.01\\
37.25	0.01\\
37.26	0.01\\
37.27	0.01\\
37.28	0.01\\
37.29	0.01\\
37.3	0.01\\
37.31	0.01\\
37.32	0.01\\
37.33	0.01\\
37.34	0.01\\
37.35	0.01\\
37.36	0.01\\
37.37	0.01\\
37.38	0.01\\
37.39	0.01\\
37.4	0.01\\
37.41	0.01\\
37.42	0.01\\
37.43	0.01\\
37.44	0.01\\
37.45	0.01\\
37.46	0.01\\
37.47	0.01\\
37.48	0.01\\
37.49	0.01\\
37.5	0.01\\
37.51	0.01\\
37.52	0.01\\
37.53	0.01\\
37.54	0.01\\
37.55	0.01\\
37.56	0.01\\
37.57	0.01\\
37.58	0.01\\
37.59	0.01\\
37.6	0.01\\
37.61	0.01\\
37.62	0.01\\
37.63	0.01\\
37.64	0.01\\
37.65	0.01\\
37.66	0.01\\
37.67	0.01\\
37.68	0.01\\
37.69	0.01\\
37.7	0.01\\
37.71	0.01\\
37.72	0.01\\
37.73	0.01\\
37.74	0.01\\
37.75	0.01\\
37.76	0.01\\
37.77	0.01\\
37.78	0.01\\
37.79	0.01\\
37.8	0.01\\
37.81	0.01\\
37.82	0.01\\
37.83	0.01\\
37.84	0.01\\
37.85	0.01\\
37.86	0.01\\
37.87	0.01\\
37.88	0.01\\
37.89	0.01\\
37.9	0.01\\
37.91	0.01\\
37.92	0.01\\
37.93	0.01\\
37.94	0.01\\
37.95	0.01\\
37.96	0.01\\
37.97	0.01\\
37.98	0.01\\
37.99	0.01\\
38	0.01\\
38.01	0.01\\
38.02	0.01\\
38.03	0.01\\
38.04	0.01\\
38.05	0.01\\
38.06	0.01\\
38.07	0.01\\
38.08	0.01\\
38.09	0.01\\
38.1	0.01\\
38.11	0.01\\
38.12	0.01\\
38.13	0.01\\
38.14	0.01\\
38.15	0.01\\
38.16	0.01\\
38.17	0.01\\
38.18	0.01\\
38.19	0.01\\
38.2	0.01\\
38.21	0.01\\
38.22	0.01\\
38.23	0.01\\
38.24	0.01\\
38.25	0.01\\
38.26	0.01\\
38.27	0.01\\
38.28	0.01\\
38.29	0.01\\
38.3	0.01\\
38.31	0.01\\
38.32	0.01\\
38.33	0.01\\
38.34	0.01\\
38.35	0.01\\
38.36	0.01\\
38.37	0.01\\
38.38	0.01\\
38.39	0.01\\
38.4	0.01\\
38.41	0.01\\
38.42	0.01\\
38.43	0.01\\
38.44	0.01\\
38.45	0.01\\
38.46	0.01\\
38.47	0.01\\
38.48	0.01\\
38.49	0.01\\
38.5	0.01\\
38.51	0.01\\
38.52	0.01\\
38.53	0.01\\
38.54	0.01\\
38.55	0.01\\
38.56	0.01\\
38.57	0.01\\
38.58	0.01\\
38.59	0.01\\
38.6	0.01\\
38.61	0.01\\
38.62	0.01\\
38.63	0.01\\
38.64	0.01\\
38.65	0.01\\
38.66	0.01\\
38.67	0.01\\
38.68	0.01\\
38.69	0.01\\
38.7	0.01\\
38.71	0.01\\
38.72	0.01\\
38.73	0.01\\
38.74	0.01\\
38.75	0.01\\
38.76	0.01\\
38.77	0.01\\
38.78	0.01\\
38.79	0.01\\
38.8	0.01\\
38.81	0.01\\
38.82	0.01\\
38.83	0.01\\
38.84	0.01\\
38.85	0.01\\
38.86	0.01\\
38.87	0.01\\
38.88	0.01\\
38.89	0.01\\
38.9	0.01\\
38.91	0.01\\
38.92	0.01\\
38.93	0.01\\
38.94	0.01\\
38.95	0.01\\
38.96	0.01\\
38.97	0.01\\
38.98	0.01\\
38.99	0.01\\
39	0.01\\
39.01	0.01\\
39.02	0.01\\
39.03	0.01\\
39.04	0.01\\
39.05	0.01\\
39.06	0.01\\
39.07	0.01\\
39.08	0.01\\
39.09	0.01\\
39.1	0.01\\
39.11	0.01\\
39.12	0.01\\
39.13	0.01\\
39.14	0.01\\
39.15	0.01\\
39.16	0.01\\
39.17	0.01\\
39.18	0.01\\
39.19	0.01\\
39.2	0.01\\
39.21	0.01\\
39.22	0.01\\
39.23	0.01\\
39.24	0.01\\
39.25	0.01\\
39.26	0.01\\
39.27	0.01\\
39.28	0.01\\
39.29	0.01\\
39.3	0.01\\
39.31	0.01\\
39.32	0.01\\
39.33	0.01\\
39.34	0.01\\
39.35	0.01\\
39.36	0.01\\
39.37	0.01\\
39.38	0.01\\
39.39	0.01\\
39.4	0.01\\
39.41	0.01\\
39.42	0.01\\
39.43	0.01\\
39.44	0.01\\
39.45	0.01\\
39.46	0.01\\
39.47	0.01\\
39.48	0.01\\
39.49	0.01\\
39.5	0.01\\
39.51	0.01\\
39.52	0.01\\
39.53	0.01\\
39.54	0.01\\
39.55	0.01\\
39.56	0.01\\
39.57	0.01\\
39.58	0.01\\
39.59	0.01\\
39.6	0.01\\
39.61	0.01\\
39.62	0.01\\
39.63	0.01\\
39.64	0.01\\
39.65	0.01\\
39.66	0.01\\
39.67	0.01\\
39.68	0.01\\
39.69	0.01\\
39.7	0.01\\
39.71	0.01\\
39.72	0.01\\
39.73	0.01\\
39.74	0.01\\
39.75	0.01\\
39.76	0.01\\
39.77	0.01\\
39.78	0.01\\
39.79	0.01\\
39.8	0.01\\
39.81	0.01\\
39.82	0.01\\
39.83	0.01\\
39.84	0.01\\
39.85	0.01\\
39.86	0.01\\
39.87	0.01\\
39.88	0.01\\
39.89	0.01\\
39.9	0.01\\
39.91	0.01\\
39.92	0.01\\
39.93	0.01\\
39.94	0.01\\
39.95	0.01\\
39.96	0.01\\
39.97	0.01\\
39.98	0.01\\
39.99	0.01\\
40	0.01\\
40.01	0.01\\
};
\addplot [color=blue,solid,forget plot]
  table[row sep=crcr]{%
40.01	0.01\\
40.02	0.01\\
40.03	0.01\\
40.04	0.01\\
40.05	0.01\\
40.06	0.01\\
40.07	0.01\\
40.08	0.01\\
40.09	0.01\\
40.1	0.01\\
40.11	0.01\\
40.12	0.01\\
40.13	0.01\\
40.14	0.01\\
40.15	0.01\\
40.16	0.01\\
40.17	0.01\\
40.18	0.01\\
40.19	0.01\\
40.2	0.01\\
40.21	0.01\\
40.22	0.01\\
40.23	0.01\\
40.24	0.01\\
40.25	0.01\\
40.26	0.01\\
40.27	0.01\\
40.28	0.01\\
40.29	0.01\\
40.3	0.01\\
40.31	0.01\\
40.32	0.01\\
40.33	0.01\\
40.34	0.01\\
40.35	0.01\\
40.36	0.01\\
40.37	0.01\\
40.38	0.01\\
40.39	0.01\\
40.4	0.01\\
40.41	0.01\\
40.42	0.01\\
40.43	0.01\\
40.44	0.01\\
40.45	0.01\\
40.46	0.01\\
40.47	0.01\\
40.48	0.01\\
40.49	0.01\\
40.5	0.01\\
40.51	0.01\\
40.52	0.01\\
40.53	0.01\\
40.54	0.01\\
40.55	0.01\\
40.56	0.01\\
40.57	0.01\\
40.58	0.01\\
40.59	0.01\\
40.6	0.01\\
40.61	0.01\\
40.62	0.01\\
40.63	0.01\\
40.64	0.01\\
40.65	0.01\\
40.66	0.01\\
40.67	0.01\\
40.68	0.01\\
40.69	0.01\\
40.7	0.01\\
40.71	0.01\\
40.72	0.01\\
40.73	0.01\\
40.74	0.01\\
40.75	0.01\\
40.76	0.01\\
40.77	0.01\\
40.78	0.01\\
40.79	0.01\\
40.8	0.01\\
40.81	0.01\\
40.82	0.01\\
40.83	0.01\\
40.84	0.01\\
40.85	0.01\\
40.86	0.01\\
40.87	0.01\\
40.88	0.01\\
40.89	0.01\\
40.9	0.01\\
40.91	0.01\\
40.92	0.01\\
40.93	0.01\\
40.94	0.01\\
40.95	0.01\\
40.96	0.01\\
40.97	0.01\\
40.98	0.01\\
40.99	0.01\\
41	0.01\\
41.01	0.01\\
41.02	0.01\\
41.03	0.01\\
41.04	0.01\\
41.05	0.01\\
41.06	0.01\\
41.07	0.01\\
41.08	0.01\\
41.09	0.01\\
41.1	0.01\\
41.11	0.01\\
41.12	0.01\\
41.13	0.01\\
41.14	0.01\\
41.15	0.01\\
41.16	0.01\\
41.17	0.01\\
41.18	0.01\\
41.19	0.01\\
41.2	0.01\\
41.21	0.01\\
41.22	0.01\\
41.23	0.01\\
41.24	0.01\\
41.25	0.01\\
41.26	0.01\\
41.27	0.01\\
41.28	0.01\\
41.29	0.01\\
41.3	0.01\\
41.31	0.01\\
41.32	0.01\\
41.33	0.01\\
41.34	0.01\\
41.35	0.01\\
41.36	0.01\\
41.37	0.01\\
41.38	0.01\\
41.39	0.01\\
41.4	0.01\\
41.41	0.01\\
41.42	0.01\\
41.43	0.01\\
41.44	0.01\\
41.45	0.01\\
41.46	0.01\\
41.47	0.01\\
41.48	0.01\\
41.49	0.01\\
41.5	0.01\\
41.51	0.01\\
41.52	0.01\\
41.53	0.01\\
41.54	0.01\\
41.55	0.01\\
41.56	0.01\\
41.57	0.01\\
41.58	0.01\\
41.59	0.01\\
41.6	0.01\\
41.61	0.01\\
41.62	0.01\\
41.63	0.01\\
41.64	0.01\\
41.65	0.01\\
41.66	0.01\\
41.67	0.01\\
41.68	0.01\\
41.69	0.01\\
41.7	0.01\\
41.71	0.01\\
41.72	0.01\\
41.73	0.01\\
41.74	0.01\\
41.75	0.01\\
41.76	0.01\\
41.77	0.01\\
41.78	0.01\\
41.79	0.01\\
41.8	0.01\\
41.81	0.01\\
41.82	0.01\\
41.83	0.01\\
41.84	0.01\\
41.85	0.01\\
41.86	0.01\\
41.87	0.01\\
41.88	0.01\\
41.89	0.01\\
41.9	0.01\\
41.91	0.01\\
41.92	0.01\\
41.93	0.01\\
41.94	0.01\\
41.95	0.01\\
41.96	0.01\\
41.97	0.01\\
41.98	0.01\\
41.99	0.01\\
42	0.01\\
42.01	0.01\\
42.02	0.01\\
42.03	0.01\\
42.04	0.01\\
42.05	0.01\\
42.06	0.01\\
42.07	0.01\\
42.08	0.01\\
42.09	0.01\\
42.1	0.01\\
42.11	0.01\\
42.12	0.01\\
42.13	0.01\\
42.14	0.01\\
42.15	0.01\\
42.16	0.01\\
42.17	0.01\\
42.18	0.01\\
42.19	0.01\\
42.2	0.01\\
42.21	0.01\\
42.22	0.01\\
42.23	0.01\\
42.24	0.01\\
42.25	0.01\\
42.26	0.01\\
42.27	0.01\\
42.28	0.01\\
42.29	0.01\\
42.3	0.01\\
42.31	0.01\\
42.32	0.01\\
42.33	0.01\\
42.34	0.01\\
42.35	0.01\\
42.36	0.01\\
42.37	0.01\\
42.38	0.01\\
42.39	0.01\\
42.4	0.01\\
42.41	0.01\\
42.42	0.01\\
42.43	0.01\\
42.44	0.01\\
42.45	0.01\\
42.46	0.01\\
42.47	0.01\\
42.48	0.01\\
42.49	0.01\\
42.5	0.01\\
42.51	0.01\\
42.52	0.01\\
42.53	0.01\\
42.54	0.01\\
42.55	0.01\\
42.56	0.01\\
42.57	0.01\\
42.58	0.01\\
42.59	0.01\\
42.6	0.01\\
42.61	0.01\\
42.62	0.01\\
42.63	0.01\\
42.64	0.01\\
42.65	0.01\\
42.66	0.01\\
42.67	0.01\\
42.68	0.01\\
42.69	0.01\\
42.7	0.01\\
42.71	0.01\\
42.72	0.01\\
42.73	0.01\\
42.74	0.01\\
42.75	0.01\\
42.76	0.01\\
42.77	0.01\\
42.78	0.01\\
42.79	0.01\\
42.8	0.01\\
42.81	0.01\\
42.82	0.01\\
42.83	0.01\\
42.84	0.01\\
42.85	0.01\\
42.86	0.01\\
42.87	0.01\\
42.88	0.01\\
42.89	0.01\\
42.9	0.01\\
42.91	0.01\\
42.92	0.01\\
42.93	0.01\\
42.94	0.01\\
42.95	0.01\\
42.96	0.01\\
42.97	0.01\\
42.98	0.01\\
42.99	0.01\\
43	0.01\\
43.01	0.01\\
43.02	0.01\\
43.03	0.01\\
43.04	0.01\\
43.05	0.01\\
43.06	0.01\\
43.07	0.01\\
43.08	0.01\\
43.09	0.01\\
43.1	0.01\\
43.11	0.01\\
43.12	0.01\\
43.13	0.01\\
43.14	0.01\\
43.15	0.01\\
43.16	0.01\\
43.17	0.01\\
43.18	0.01\\
43.19	0.01\\
43.2	0.01\\
43.21	0.01\\
43.22	0.01\\
43.23	0.01\\
43.24	0.01\\
43.25	0.01\\
43.26	0.01\\
43.27	0.01\\
43.28	0.01\\
43.29	0.01\\
43.3	0.01\\
43.31	0.01\\
43.32	0.01\\
43.33	0.01\\
43.34	0.01\\
43.35	0.01\\
43.36	0.01\\
43.37	0.01\\
43.38	0.01\\
43.39	0.01\\
43.4	0.01\\
43.41	0.01\\
43.42	0.01\\
43.43	0.01\\
43.44	0.01\\
43.45	0.01\\
43.46	0.01\\
43.47	0.01\\
43.48	0.01\\
43.49	0.01\\
43.5	0.01\\
43.51	0.01\\
43.52	0.01\\
43.53	0.01\\
43.54	0.01\\
43.55	0.01\\
43.56	0.01\\
43.57	0.01\\
43.58	0.01\\
43.59	0.01\\
43.6	0.01\\
43.61	0.01\\
43.62	0.01\\
43.63	0.01\\
43.64	0.01\\
43.65	0.01\\
43.66	0.01\\
43.67	0.01\\
43.68	0.01\\
43.69	0.01\\
43.7	0.01\\
43.71	0.01\\
43.72	0.01\\
43.73	0.01\\
43.74	0.01\\
43.75	0.01\\
43.76	0.01\\
43.77	0.01\\
43.78	0.01\\
43.79	0.01\\
43.8	0.01\\
43.81	0.01\\
43.82	0.01\\
43.83	0.01\\
43.84	0.01\\
43.85	0.01\\
43.86	0.01\\
43.87	0.01\\
43.88	0.01\\
43.89	0.01\\
43.9	0.01\\
43.91	0.01\\
43.92	0.01\\
43.93	0.01\\
43.94	0.01\\
43.95	0.01\\
43.96	0.01\\
43.97	0.01\\
43.98	0.01\\
43.99	0.01\\
44	0.01\\
44.01	0.01\\
44.02	0.01\\
44.03	0.01\\
44.04	0.01\\
44.05	0.01\\
44.06	0.01\\
44.07	0.01\\
44.08	0.01\\
44.09	0.01\\
44.1	0.01\\
44.11	0.01\\
44.12	0.01\\
44.13	0.01\\
44.14	0.01\\
44.15	0.01\\
44.16	0.01\\
44.17	0.01\\
44.18	0.01\\
44.19	0.01\\
44.2	0.01\\
44.21	0.01\\
44.22	0.01\\
44.23	0.01\\
44.24	0.01\\
44.25	0.01\\
44.26	0.01\\
44.27	0.01\\
44.28	0.01\\
44.29	0.01\\
44.3	0.01\\
44.31	0.01\\
44.32	0.01\\
44.33	0.01\\
44.34	0.01\\
44.35	0.01\\
44.36	0.01\\
44.37	0.01\\
44.38	0.01\\
44.39	0.01\\
44.4	0.01\\
44.41	0.01\\
44.42	0.01\\
44.43	0.01\\
44.44	0.01\\
44.45	0.01\\
44.46	0.01\\
44.47	0.01\\
44.48	0.01\\
44.49	0.01\\
44.5	0.01\\
44.51	0.01\\
44.52	0.01\\
44.53	0.01\\
44.54	0.01\\
44.55	0.01\\
44.56	0.01\\
44.57	0.01\\
44.58	0.01\\
44.59	0.01\\
44.6	0.01\\
44.61	0.01\\
44.62	0.01\\
44.63	0.01\\
44.64	0.01\\
44.65	0.01\\
44.66	0.01\\
44.67	0.01\\
44.68	0.01\\
44.69	0.01\\
44.7	0.01\\
44.71	0.01\\
44.72	0.01\\
44.73	0.01\\
44.74	0.01\\
44.75	0.01\\
44.76	0.01\\
44.77	0.01\\
44.78	0.01\\
44.79	0.01\\
44.8	0.01\\
44.81	0.01\\
44.82	0.01\\
44.83	0.01\\
44.84	0.01\\
44.85	0.01\\
44.86	0.01\\
44.87	0.01\\
44.88	0.01\\
44.89	0.01\\
44.9	0.01\\
44.91	0.01\\
44.92	0.01\\
44.93	0.01\\
44.94	0.01\\
44.95	0.01\\
44.96	0.01\\
44.97	0.01\\
44.98	0.01\\
44.99	0.01\\
45	0.01\\
45.01	0.01\\
45.02	0.01\\
45.03	0.01\\
45.04	0.01\\
45.05	0.01\\
45.06	0.01\\
45.07	0.01\\
45.08	0.01\\
45.09	0.01\\
45.1	0.01\\
45.11	0.01\\
45.12	0.01\\
45.13	0.01\\
45.14	0.01\\
45.15	0.01\\
45.16	0.01\\
45.17	0.01\\
45.18	0.01\\
45.19	0.01\\
45.2	0.01\\
45.21	0.01\\
45.22	0.01\\
45.23	0.01\\
45.24	0.01\\
45.25	0.01\\
45.26	0.01\\
45.27	0.01\\
45.28	0.01\\
45.29	0.01\\
45.3	0.01\\
45.31	0.01\\
45.32	0.01\\
45.33	0.01\\
45.34	0.01\\
45.35	0.01\\
45.36	0.01\\
45.37	0.01\\
45.38	0.01\\
45.39	0.01\\
45.4	0.01\\
45.41	0.01\\
45.42	0.01\\
45.43	0.01\\
45.44	0.01\\
45.45	0.01\\
45.46	0.01\\
45.47	0.01\\
45.48	0.01\\
45.49	0.01\\
45.5	0.01\\
45.51	0.01\\
45.52	0.01\\
45.53	0.01\\
45.54	0.01\\
45.55	0.01\\
45.56	0.01\\
45.57	0.01\\
45.58	0.01\\
45.59	0.01\\
45.6	0.01\\
45.61	0.01\\
45.62	0.01\\
45.63	0.01\\
45.64	0.01\\
45.65	0.01\\
45.66	0.01\\
45.67	0.01\\
45.68	0.01\\
45.69	0.01\\
45.7	0.01\\
45.71	0.01\\
45.72	0.01\\
45.73	0.01\\
45.74	0.01\\
45.75	0.01\\
45.76	0.01\\
45.77	0.01\\
45.78	0.01\\
45.79	0.01\\
45.8	0.01\\
45.81	0.01\\
45.82	0.01\\
45.83	0.01\\
45.84	0.01\\
45.85	0.01\\
45.86	0.01\\
45.87	0.01\\
45.88	0.01\\
45.89	0.01\\
45.9	0.01\\
45.91	0.01\\
45.92	0.01\\
45.93	0.01\\
45.94	0.01\\
45.95	0.01\\
45.96	0.01\\
45.97	0.01\\
45.98	0.01\\
45.99	0.01\\
46	0.01\\
46.01	0.01\\
46.02	0.01\\
46.03	0.01\\
46.04	0.01\\
46.05	0.01\\
46.06	0.01\\
46.07	0.01\\
46.08	0.01\\
46.09	0.01\\
46.1	0.01\\
46.11	0.01\\
46.12	0.01\\
46.13	0.01\\
46.14	0.01\\
46.15	0.01\\
46.16	0.01\\
46.17	0.01\\
46.18	0.01\\
46.19	0.01\\
46.2	0.01\\
46.21	0.01\\
46.22	0.01\\
46.23	0.01\\
46.24	0.01\\
46.25	0.01\\
46.26	0.01\\
46.27	0.01\\
46.28	0.01\\
46.29	0.01\\
46.3	0.01\\
46.31	0.01\\
46.32	0.01\\
46.33	0.01\\
46.34	0.01\\
46.35	0.01\\
46.36	0.01\\
46.37	0.01\\
46.38	0.01\\
46.39	0.01\\
46.4	0.01\\
46.41	0.01\\
46.42	0.01\\
46.43	0.01\\
46.44	0.01\\
46.45	0.01\\
46.46	0.01\\
46.47	0.01\\
46.48	0.01\\
46.49	0.01\\
46.5	0.01\\
46.51	0.01\\
46.52	0.01\\
46.53	0.01\\
46.54	0.01\\
46.55	0.01\\
46.56	0.01\\
46.57	0.01\\
46.58	0.01\\
46.59	0.01\\
46.6	0.01\\
46.61	0.01\\
46.62	0.01\\
46.63	0.01\\
46.64	0.01\\
46.65	0.01\\
46.66	0.01\\
46.67	0.01\\
46.68	0.01\\
46.69	0.01\\
46.7	0.01\\
46.71	0.01\\
46.72	0.01\\
46.73	0.01\\
46.74	0.01\\
46.75	0.01\\
46.76	0.01\\
46.77	0.01\\
46.78	0.01\\
46.79	0.01\\
46.8	0.01\\
46.81	0.01\\
46.82	0.01\\
46.83	0.01\\
46.84	0.01\\
46.85	0.01\\
46.86	0.01\\
46.87	0.01\\
46.88	0.01\\
46.89	0.01\\
46.9	0.01\\
46.91	0.01\\
46.92	0.01\\
46.93	0.01\\
46.94	0.01\\
46.95	0.01\\
46.96	0.01\\
46.97	0.01\\
46.98	0.01\\
46.99	0.01\\
47	0.01\\
47.01	0.01\\
47.02	0.01\\
47.03	0.01\\
47.04	0.01\\
47.05	0.01\\
47.06	0.01\\
47.07	0.01\\
47.08	0.01\\
47.09	0.01\\
47.1	0.01\\
47.11	0.01\\
47.12	0.01\\
47.13	0.01\\
47.14	0.01\\
47.15	0.01\\
47.16	0.01\\
47.17	0.01\\
47.18	0.01\\
47.19	0.01\\
47.2	0.01\\
47.21	0.01\\
47.22	0.01\\
47.23	0.01\\
47.24	0.01\\
47.25	0.01\\
47.26	0.01\\
47.27	0.01\\
47.28	0.01\\
47.29	0.01\\
47.3	0.01\\
47.31	0.01\\
47.32	0.01\\
47.33	0.01\\
47.34	0.01\\
47.35	0.01\\
47.36	0.01\\
47.37	0.01\\
47.38	0.01\\
47.39	0.01\\
47.4	0.01\\
47.41	0.01\\
47.42	0.01\\
47.43	0.01\\
47.44	0.01\\
47.45	0.01\\
47.46	0.01\\
47.47	0.01\\
47.48	0.01\\
47.49	0.01\\
47.5	0.01\\
47.51	0.01\\
47.52	0.01\\
47.53	0.01\\
47.54	0.01\\
47.55	0.01\\
47.56	0.01\\
47.57	0.01\\
47.58	0.01\\
47.59	0.01\\
47.6	0.01\\
47.61	0.01\\
47.62	0.01\\
47.63	0.01\\
47.64	0.01\\
47.65	0.01\\
47.66	0.01\\
47.67	0.01\\
47.68	0.01\\
47.69	0.01\\
47.7	0.01\\
47.71	0.01\\
47.72	0.01\\
47.73	0.01\\
47.74	0.01\\
47.75	0.01\\
47.76	0.01\\
47.77	0.01\\
47.78	0.01\\
47.79	0.01\\
47.8	0.01\\
47.81	0.01\\
47.82	0.01\\
47.83	0.01\\
47.84	0.01\\
47.85	0.01\\
47.86	0.01\\
47.87	0.01\\
47.88	0.01\\
47.89	0.01\\
47.9	0.01\\
47.91	0.01\\
47.92	0.01\\
47.93	0.01\\
47.94	0.01\\
47.95	0.01\\
47.96	0.01\\
47.97	0.01\\
47.98	0.01\\
47.99	0.01\\
48	0.01\\
48.01	0.01\\
48.02	0.01\\
48.03	0.01\\
48.04	0.01\\
48.05	0.01\\
48.06	0.01\\
48.07	0.01\\
48.08	0.01\\
48.09	0.01\\
48.1	0.01\\
48.11	0.01\\
48.12	0.01\\
48.13	0.01\\
48.14	0.01\\
48.15	0.01\\
48.16	0.01\\
48.17	0.01\\
48.18	0.01\\
48.19	0.01\\
48.2	0.01\\
48.21	0.01\\
48.22	0.01\\
48.23	0.01\\
48.24	0.01\\
48.25	0.01\\
48.26	0.01\\
48.27	0.01\\
48.28	0.01\\
48.29	0.01\\
48.3	0.01\\
48.31	0.01\\
48.32	0.01\\
48.33	0.01\\
48.34	0.01\\
48.35	0.01\\
48.36	0.01\\
48.37	0.01\\
48.38	0.01\\
48.39	0.01\\
48.4	0.01\\
48.41	0.01\\
48.42	0.01\\
48.43	0.01\\
48.44	0.01\\
48.45	0.01\\
48.46	0.01\\
48.47	0.01\\
48.48	0.01\\
48.49	0.01\\
48.5	0.01\\
48.51	0.01\\
48.52	0.01\\
48.53	0.01\\
48.54	0.01\\
48.55	0.01\\
48.56	0.01\\
48.57	0.01\\
48.58	0.01\\
48.59	0.01\\
48.6	0.01\\
48.61	0.01\\
48.62	0.01\\
48.63	0.01\\
48.64	0.01\\
48.65	0.01\\
48.66	0.01\\
48.67	0.01\\
48.68	0.01\\
48.69	0.01\\
48.7	0.01\\
48.71	0.01\\
48.72	0.01\\
48.73	0.01\\
48.74	0.01\\
48.75	0.01\\
48.76	0.01\\
48.77	0.01\\
48.78	0.01\\
48.79	0.01\\
48.8	0.01\\
48.81	0.01\\
48.82	0.01\\
48.83	0.01\\
48.84	0.01\\
48.85	0.01\\
48.86	0.01\\
48.87	0.01\\
48.88	0.01\\
48.89	0.01\\
48.9	0.01\\
48.91	0.01\\
48.92	0.01\\
48.93	0.01\\
48.94	0.01\\
48.95	0.01\\
48.96	0.01\\
48.97	0.01\\
48.98	0.01\\
48.99	0.01\\
49	0.01\\
49.01	0.01\\
49.02	0.01\\
49.03	0.01\\
49.04	0.01\\
49.05	0.01\\
49.06	0.01\\
49.07	0.01\\
49.08	0.01\\
49.09	0.01\\
49.1	0.01\\
49.11	0.01\\
49.12	0.01\\
49.13	0.01\\
49.14	0.01\\
49.15	0.01\\
49.16	0.01\\
49.17	0.01\\
49.18	0.01\\
49.19	0.01\\
49.2	0.01\\
49.21	0.01\\
49.22	0.01\\
49.23	0.01\\
49.24	0.01\\
49.25	0.01\\
49.26	0.01\\
49.27	0.01\\
49.28	0.01\\
49.29	0.01\\
49.3	0.01\\
49.31	0.01\\
49.32	0.01\\
49.33	0.01\\
49.34	0.01\\
49.35	0.01\\
49.36	0.01\\
49.37	0.01\\
49.38	0.01\\
49.39	0.01\\
49.4	0.01\\
49.41	0.01\\
49.42	0.01\\
49.43	0.01\\
49.44	0.01\\
49.45	0.01\\
49.46	0.01\\
49.47	0.01\\
49.48	0.01\\
49.49	0.01\\
49.5	0.01\\
49.51	0.01\\
49.52	0.01\\
49.53	0.01\\
49.54	0.01\\
49.55	0.01\\
49.56	0.01\\
49.57	0.01\\
49.58	0.01\\
49.59	0.01\\
49.6	0.01\\
49.61	0.01\\
49.62	0.01\\
49.63	0.01\\
49.64	0.01\\
49.65	0.01\\
49.66	0.01\\
49.67	0.01\\
49.68	0.01\\
49.69	0.01\\
49.7	0.01\\
49.71	0.01\\
49.72	0.01\\
49.73	0.01\\
49.74	0.01\\
49.75	0.01\\
49.76	0.01\\
49.77	0.01\\
49.78	0.01\\
49.79	0.01\\
49.8	0.01\\
49.81	0.01\\
49.82	0.01\\
49.83	0.01\\
49.84	0.01\\
49.85	0.01\\
49.86	0.01\\
49.87	0.01\\
49.88	0.01\\
49.89	0.01\\
49.9	0.01\\
49.91	0.01\\
49.92	0.01\\
49.93	0.01\\
49.94	0.01\\
49.95	0.01\\
49.96	0.01\\
49.97	0.01\\
49.98	0.01\\
49.99	0.01\\
50	0.01\\
50.01	0.01\\
50.02	0.01\\
50.03	0.01\\
50.04	0.01\\
50.05	0.01\\
50.06	0.01\\
50.07	0.01\\
50.08	0.01\\
50.09	0.01\\
50.1	0.01\\
50.11	0.01\\
50.12	0.01\\
50.13	0.01\\
50.14	0.01\\
50.15	0.01\\
50.16	0.01\\
50.17	0.01\\
50.18	0.01\\
50.19	0.01\\
50.2	0.01\\
50.21	0.01\\
50.22	0.01\\
50.23	0.01\\
50.24	0.01\\
50.25	0.01\\
50.26	0.01\\
50.27	0.01\\
50.28	0.01\\
50.29	0.01\\
50.3	0.01\\
50.31	0.01\\
50.32	0.01\\
50.33	0.01\\
50.34	0.01\\
50.35	0.01\\
50.36	0.01\\
50.37	0.01\\
50.38	0.01\\
50.39	0.01\\
50.4	0.01\\
50.41	0.01\\
50.42	0.01\\
50.43	0.01\\
50.44	0.01\\
50.45	0.01\\
50.46	0.01\\
50.47	0.01\\
50.48	0.01\\
50.49	0.01\\
50.5	0.01\\
50.51	0.01\\
50.52	0.01\\
50.53	0.01\\
50.54	0.01\\
50.55	0.01\\
50.56	0.01\\
50.57	0.01\\
50.58	0.01\\
50.59	0.01\\
50.6	0.01\\
50.61	0.01\\
50.62	0.01\\
50.63	0.01\\
50.64	0.01\\
50.65	0.01\\
50.66	0.01\\
50.67	0.01\\
50.68	0.01\\
50.69	0.01\\
50.7	0.01\\
50.71	0.01\\
50.72	0.01\\
50.73	0.01\\
50.74	0.01\\
50.75	0.01\\
50.76	0.01\\
50.77	0.01\\
50.78	0.01\\
50.79	0.01\\
50.8	0.01\\
50.81	0.01\\
50.82	0.01\\
50.83	0.01\\
50.84	0.01\\
50.85	0.01\\
50.86	0.01\\
50.87	0.01\\
50.88	0.01\\
50.89	0.01\\
50.9	0.01\\
50.91	0.01\\
50.92	0.01\\
50.93	0.01\\
50.94	0.01\\
50.95	0.01\\
50.96	0.01\\
50.97	0.01\\
50.98	0.01\\
50.99	0.01\\
51	0.01\\
51.01	0.01\\
51.02	0.01\\
51.03	0.01\\
51.04	0.01\\
51.05	0.01\\
51.06	0.01\\
51.07	0.01\\
51.08	0.01\\
51.09	0.01\\
51.1	0.01\\
51.11	0.01\\
51.12	0.01\\
51.13	0.01\\
51.14	0.01\\
51.15	0.01\\
51.16	0.01\\
51.17	0.01\\
51.18	0.01\\
51.19	0.01\\
51.2	0.01\\
51.21	0.01\\
51.22	0.01\\
51.23	0.01\\
51.24	0.01\\
51.25	0.01\\
51.26	0.01\\
51.27	0.01\\
51.28	0.01\\
51.29	0.01\\
51.3	0.01\\
51.31	0.01\\
51.32	0.01\\
51.33	0.01\\
51.34	0.01\\
51.35	0.01\\
51.36	0.01\\
51.37	0.01\\
51.38	0.01\\
51.39	0.01\\
51.4	0.01\\
51.41	0.01\\
51.42	0.01\\
51.43	0.01\\
51.44	0.01\\
51.45	0.01\\
51.46	0.01\\
51.47	0.01\\
51.48	0.01\\
51.49	0.01\\
51.5	0.01\\
51.51	0.01\\
51.52	0.01\\
51.53	0.01\\
51.54	0.01\\
51.55	0.01\\
51.56	0.01\\
51.57	0.01\\
51.58	0.01\\
51.59	0.01\\
51.6	0.01\\
51.61	0.01\\
51.62	0.01\\
51.63	0.01\\
51.64	0.01\\
51.65	0.01\\
51.66	0.01\\
51.67	0.01\\
51.68	0.01\\
51.69	0.01\\
51.7	0.01\\
51.71	0.01\\
51.72	0.01\\
51.73	0.01\\
51.74	0.01\\
51.75	0.01\\
51.76	0.01\\
51.77	0.01\\
51.78	0.01\\
51.79	0.01\\
51.8	0.01\\
51.81	0.01\\
51.82	0.01\\
51.83	0.01\\
51.84	0.01\\
51.85	0.01\\
51.86	0.01\\
51.87	0.01\\
51.88	0.01\\
51.89	0.01\\
51.9	0.01\\
51.91	0.01\\
51.92	0.01\\
51.93	0.01\\
51.94	0.01\\
51.95	0.01\\
51.96	0.01\\
51.97	0.01\\
51.98	0.01\\
51.99	0.01\\
52	0.01\\
52.01	0.01\\
52.02	0.01\\
52.03	0.01\\
52.04	0.01\\
52.05	0.01\\
52.06	0.01\\
52.07	0.01\\
52.08	0.01\\
52.09	0.01\\
52.1	0.01\\
52.11	0.01\\
52.12	0.01\\
52.13	0.01\\
52.14	0.01\\
52.15	0.01\\
52.16	0.01\\
52.17	0.01\\
52.18	0.01\\
52.19	0.01\\
52.2	0.01\\
52.21	0.01\\
52.22	0.01\\
52.23	0.01\\
52.24	0.01\\
52.25	0.01\\
52.26	0.01\\
52.27	0.01\\
52.28	0.01\\
52.29	0.01\\
52.3	0.01\\
52.31	0.01\\
52.32	0.01\\
52.33	0.01\\
52.34	0.01\\
52.35	0.01\\
52.36	0.01\\
52.37	0.01\\
52.38	0.01\\
52.39	0.01\\
52.4	0.01\\
52.41	0.01\\
52.42	0.01\\
52.43	0.01\\
52.44	0.01\\
52.45	0.01\\
52.46	0.01\\
52.47	0.01\\
52.48	0.01\\
52.49	0.01\\
52.5	0.01\\
52.51	0.01\\
52.52	0.01\\
52.53	0.01\\
52.54	0.01\\
52.55	0.01\\
52.56	0.01\\
52.57	0.01\\
52.58	0.01\\
52.59	0.01\\
52.6	0.01\\
52.61	0.01\\
52.62	0.01\\
52.63	0.01\\
52.64	0.01\\
52.65	0.01\\
52.66	0.01\\
52.67	0.01\\
52.68	0.01\\
52.69	0.01\\
52.7	0.01\\
52.71	0.01\\
52.72	0.01\\
52.73	0.01\\
52.74	0.01\\
52.75	0.01\\
52.76	0.01\\
52.77	0.01\\
52.78	0.01\\
52.79	0.01\\
52.8	0.01\\
52.81	0.01\\
52.82	0.01\\
52.83	0.01\\
52.84	0.01\\
52.85	0.01\\
52.86	0.01\\
52.87	0.01\\
52.88	0.01\\
52.89	0.01\\
52.9	0.01\\
52.91	0.01\\
52.92	0.01\\
52.93	0.01\\
52.94	0.01\\
52.95	0.01\\
52.96	0.01\\
52.97	0.01\\
52.98	0.01\\
52.99	0.01\\
53	0.01\\
53.01	0.01\\
53.02	0.01\\
53.03	0.01\\
53.04	0.01\\
53.05	0.01\\
53.06	0.01\\
53.07	0.01\\
53.08	0.01\\
53.09	0.01\\
53.1	0.01\\
53.11	0.01\\
53.12	0.01\\
53.13	0.01\\
53.14	0.01\\
53.15	0.01\\
53.16	0.01\\
53.17	0.01\\
53.18	0.01\\
53.19	0.01\\
53.2	0.01\\
53.21	0.01\\
53.22	0.01\\
53.23	0.01\\
53.24	0.01\\
53.25	0.01\\
53.26	0.01\\
53.27	0.01\\
53.28	0.01\\
53.29	0.01\\
53.3	0.01\\
53.31	0.01\\
53.32	0.01\\
53.33	0.01\\
53.34	0.01\\
53.35	0.01\\
53.36	0.01\\
53.37	0.01\\
53.38	0.01\\
53.39	0.01\\
53.4	0.01\\
53.41	0.01\\
53.42	0.01\\
53.43	0.01\\
53.44	0.01\\
53.45	0.01\\
53.46	0.01\\
53.47	0.01\\
53.48	0.01\\
53.49	0.01\\
53.5	0.01\\
53.51	0.01\\
53.52	0.01\\
53.53	0.01\\
53.54	0.01\\
53.55	0.01\\
53.56	0.01\\
53.57	0.01\\
53.58	0.01\\
53.59	0.01\\
53.6	0.01\\
53.61	0.01\\
53.62	0.01\\
53.63	0.01\\
53.64	0.01\\
53.65	0.01\\
53.66	0.01\\
53.67	0.01\\
53.68	0.01\\
53.69	0.01\\
53.7	0.01\\
53.71	0.01\\
53.72	0.01\\
53.73	0.01\\
53.74	0.01\\
53.75	0.01\\
53.76	0.01\\
53.77	0.01\\
53.78	0.01\\
53.79	0.01\\
53.8	0.01\\
53.81	0.01\\
53.82	0.01\\
53.83	0.01\\
53.84	0.01\\
53.85	0.01\\
53.86	0.01\\
53.87	0.01\\
53.88	0.01\\
53.89	0.01\\
53.9	0.01\\
53.91	0.01\\
53.92	0.01\\
53.93	0.01\\
53.94	0.01\\
53.95	0.01\\
53.96	0.01\\
53.97	0.01\\
53.98	0.01\\
53.99	0.01\\
54	0.01\\
54.01	0.01\\
54.02	0.01\\
54.03	0.01\\
54.04	0.01\\
54.05	0.01\\
54.06	0.01\\
54.07	0.01\\
54.08	0.01\\
54.09	0.01\\
54.1	0.01\\
54.11	0.01\\
54.12	0.01\\
54.13	0.01\\
54.14	0.01\\
54.15	0.01\\
54.16	0.01\\
54.17	0.01\\
54.18	0.01\\
54.19	0.01\\
54.2	0.01\\
54.21	0.01\\
54.22	0.01\\
54.23	0.01\\
54.24	0.01\\
54.25	0.01\\
54.26	0.01\\
54.27	0.01\\
54.28	0.01\\
54.29	0.01\\
54.3	0.01\\
54.31	0.01\\
54.32	0.01\\
54.33	0.01\\
54.34	0.01\\
54.35	0.01\\
54.36	0.01\\
54.37	0.01\\
54.38	0.01\\
54.39	0.01\\
54.4	0.01\\
54.41	0.01\\
54.42	0.01\\
54.43	0.01\\
54.44	0.01\\
54.45	0.01\\
54.46	0.01\\
54.47	0.01\\
54.48	0.01\\
54.49	0.01\\
54.5	0.01\\
54.51	0.01\\
54.52	0.01\\
54.53	0.01\\
54.54	0.01\\
54.55	0.01\\
54.56	0.01\\
54.57	0.01\\
54.58	0.01\\
54.59	0.01\\
54.6	0.01\\
54.61	0.01\\
54.62	0.01\\
54.63	0.01\\
54.64	0.01\\
54.65	0.01\\
54.66	0.01\\
54.67	0.01\\
54.68	0.01\\
54.69	0.01\\
54.7	0.01\\
54.71	0.01\\
54.72	0.01\\
54.73	0.01\\
54.74	0.01\\
54.75	0.01\\
54.76	0.01\\
54.77	0.01\\
54.78	0.01\\
54.79	0.01\\
54.8	0.01\\
54.81	0.01\\
54.82	0.01\\
54.83	0.01\\
54.84	0.01\\
54.85	0.01\\
54.86	0.01\\
54.87	0.01\\
54.88	0.01\\
54.89	0.01\\
54.9	0.01\\
54.91	0.01\\
54.92	0.01\\
54.93	0.01\\
54.94	0.01\\
54.95	0.01\\
54.96	0.01\\
54.97	0.01\\
54.98	0.01\\
54.99	0.01\\
55	0.01\\
55.01	0.01\\
55.02	0.01\\
55.03	0.01\\
55.04	0.01\\
55.05	0.01\\
55.06	0.01\\
55.07	0.01\\
55.08	0.01\\
55.09	0.01\\
55.1	0.01\\
55.11	0.01\\
55.12	0.01\\
55.13	0.01\\
55.14	0.01\\
55.15	0.01\\
55.16	0.01\\
55.17	0.01\\
55.18	0.01\\
55.19	0.01\\
55.2	0.01\\
55.21	0.01\\
55.22	0.01\\
55.23	0.01\\
55.24	0.01\\
55.25	0.01\\
55.26	0.01\\
55.27	0.01\\
55.28	0.01\\
55.29	0.01\\
55.3	0.01\\
55.31	0.01\\
55.32	0.01\\
55.33	0.01\\
55.34	0.01\\
55.35	0.01\\
55.36	0.01\\
55.37	0.01\\
55.38	0.01\\
55.39	0.01\\
55.4	0.01\\
55.41	0.01\\
55.42	0.01\\
55.43	0.01\\
55.44	0.01\\
55.45	0.01\\
55.46	0.01\\
55.47	0.01\\
55.48	0.01\\
55.49	0.01\\
55.5	0.01\\
55.51	0.01\\
55.52	0.01\\
55.53	0.01\\
55.54	0.01\\
55.55	0.01\\
55.56	0.01\\
55.57	0.01\\
55.58	0.01\\
55.59	0.01\\
55.6	0.01\\
55.61	0.01\\
55.62	0.01\\
55.63	0.01\\
55.64	0.01\\
55.65	0.01\\
55.66	0.01\\
55.67	0.01\\
55.68	0.01\\
55.69	0.01\\
55.7	0.01\\
55.71	0.01\\
55.72	0.01\\
55.73	0.01\\
55.74	0.01\\
55.75	0.01\\
55.76	0.01\\
55.77	0.01\\
55.78	0.01\\
55.79	0.01\\
55.8	0.01\\
55.81	0.01\\
55.82	0.01\\
55.83	0.01\\
55.84	0.01\\
55.85	0.01\\
55.86	0.01\\
55.87	0.01\\
55.88	0.01\\
55.89	0.01\\
55.9	0.01\\
55.91	0.01\\
55.92	0.01\\
55.93	0.01\\
55.94	0.01\\
55.95	0.01\\
55.96	0.01\\
55.97	0.01\\
55.98	0.01\\
55.99	0.01\\
56	0.01\\
56.01	0.01\\
56.02	0.01\\
56.03	0.01\\
56.04	0.01\\
56.05	0.01\\
56.06	0.01\\
56.07	0.01\\
56.08	0.01\\
56.09	0.01\\
56.1	0.01\\
56.11	0.01\\
56.12	0.01\\
56.13	0.01\\
56.14	0.01\\
56.15	0.01\\
56.16	0.01\\
56.17	0.01\\
56.18	0.01\\
56.19	0.01\\
56.2	0.01\\
56.21	0.01\\
56.22	0.01\\
56.23	0.01\\
56.24	0.01\\
56.25	0.01\\
56.26	0.01\\
56.27	0.01\\
56.28	0.01\\
56.29	0.01\\
56.3	0.01\\
56.31	0.01\\
56.32	0.01\\
56.33	0.01\\
56.34	0.01\\
56.35	0.01\\
56.36	0.01\\
56.37	0.01\\
56.38	0.01\\
56.39	0.01\\
56.4	0.01\\
56.41	0.01\\
56.42	0.01\\
56.43	0.01\\
56.44	0.01\\
56.45	0.01\\
56.46	0.01\\
56.47	0.01\\
56.48	0.01\\
56.49	0.01\\
56.5	0.01\\
56.51	0.01\\
56.52	0.01\\
56.53	0.01\\
56.54	0.01\\
56.55	0.01\\
56.56	0.01\\
56.57	0.01\\
56.58	0.01\\
56.59	0.01\\
56.6	0.01\\
56.61	0.01\\
56.62	0.01\\
56.63	0.01\\
56.64	0.01\\
56.65	0.01\\
56.66	0.01\\
56.67	0.01\\
56.68	0.01\\
56.69	0.01\\
56.7	0.01\\
56.71	0.01\\
56.72	0.01\\
56.73	0.01\\
56.74	0.01\\
56.75	0.01\\
56.76	0.01\\
56.77	0.01\\
56.78	0.01\\
56.79	0.01\\
56.8	0.01\\
56.81	0.01\\
56.82	0.01\\
56.83	0.01\\
56.84	0.01\\
56.85	0.01\\
56.86	0.01\\
56.87	0.01\\
56.88	0.01\\
56.89	0.01\\
56.9	0.01\\
56.91	0.01\\
56.92	0.01\\
56.93	0.01\\
56.94	0.01\\
56.95	0.01\\
56.96	0.01\\
56.97	0.01\\
56.98	0.01\\
56.99	0.01\\
57	0.01\\
57.01	0.01\\
57.02	0.01\\
57.03	0.01\\
57.04	0.01\\
57.05	0.01\\
57.06	0.01\\
57.07	0.01\\
57.08	0.01\\
57.09	0.01\\
57.1	0.01\\
57.11	0.01\\
57.12	0.01\\
57.13	0.01\\
57.14	0.01\\
57.15	0.01\\
57.16	0.01\\
57.17	0.01\\
57.18	0.01\\
57.19	0.01\\
57.2	0.01\\
57.21	0.01\\
57.22	0.01\\
57.23	0.01\\
57.24	0.01\\
57.25	0.01\\
57.26	0.01\\
57.27	0.01\\
57.28	0.01\\
57.29	0.01\\
57.3	0.01\\
57.31	0.01\\
57.32	0.01\\
57.33	0.01\\
57.34	0.01\\
57.35	0.01\\
57.36	0.01\\
57.37	0.01\\
57.38	0.01\\
57.39	0.01\\
57.4	0.01\\
57.41	0.01\\
57.42	0.01\\
57.43	0.01\\
57.44	0.01\\
57.45	0.01\\
57.46	0.01\\
57.47	0.01\\
57.48	0.01\\
57.49	0.01\\
57.5	0.01\\
57.51	0.01\\
57.52	0.01\\
57.53	0.01\\
57.54	0.01\\
57.55	0.01\\
57.56	0.01\\
57.57	0.01\\
57.58	0.01\\
57.59	0.01\\
57.6	0.01\\
57.61	0.01\\
57.62	0.01\\
57.63	0.01\\
57.64	0.01\\
57.65	0.01\\
57.66	0.01\\
57.67	0.01\\
57.68	0.01\\
57.69	0.01\\
57.7	0.01\\
57.71	0.01\\
57.72	0.01\\
57.73	0.01\\
57.74	0.01\\
57.75	0.01\\
57.76	0.01\\
57.77	0.01\\
57.78	0.01\\
57.79	0.01\\
57.8	0.01\\
57.81	0.01\\
57.82	0.01\\
57.83	0.01\\
57.84	0.01\\
57.85	0.01\\
57.86	0.01\\
57.87	0.01\\
57.88	0.01\\
57.89	0.01\\
57.9	0.01\\
57.91	0.01\\
57.92	0.01\\
57.93	0.01\\
57.94	0.01\\
57.95	0.01\\
57.96	0.01\\
57.97	0.01\\
57.98	0.01\\
57.99	0.01\\
58	0.01\\
58.01	0.01\\
58.02	0.01\\
58.03	0.01\\
58.04	0.01\\
58.05	0.01\\
58.06	0.01\\
58.07	0.01\\
58.08	0.01\\
58.09	0.01\\
58.1	0.01\\
58.11	0.01\\
58.12	0.01\\
58.13	0.01\\
58.14	0.01\\
58.15	0.01\\
58.16	0.01\\
58.17	0.01\\
58.18	0.01\\
58.19	0.01\\
58.2	0.01\\
58.21	0.01\\
58.22	0.01\\
58.23	0.01\\
58.24	0.01\\
58.25	0.01\\
58.26	0.01\\
58.27	0.01\\
58.28	0.01\\
58.29	0.01\\
58.3	0.01\\
58.31	0.01\\
58.32	0.01\\
58.33	0.01\\
58.34	0.01\\
58.35	0.01\\
58.36	0.01\\
58.37	0.01\\
58.38	0.01\\
58.39	0.01\\
58.4	0.01\\
58.41	0.01\\
58.42	0.01\\
58.43	0.01\\
58.44	0.01\\
58.45	0.01\\
58.46	0.01\\
58.47	0.01\\
58.48	0.01\\
58.49	0.01\\
58.5	0.01\\
58.51	0.01\\
58.52	0.01\\
58.53	0.01\\
58.54	0.01\\
58.55	0.01\\
58.56	0.01\\
58.57	0.01\\
58.58	0.01\\
58.59	0.01\\
58.6	0.01\\
58.61	0.01\\
58.62	0.01\\
58.63	0.01\\
58.64	0.01\\
58.65	0.01\\
58.66	0.01\\
58.67	0.01\\
58.68	0.01\\
58.69	0.01\\
58.7	0.01\\
58.71	0.01\\
58.72	0.01\\
58.73	0.01\\
58.74	0.01\\
58.75	0.01\\
58.76	0.01\\
58.77	0.01\\
58.78	0.01\\
58.79	0.01\\
58.8	0.01\\
58.81	0.01\\
58.82	0.01\\
58.83	0.01\\
58.84	0.01\\
58.85	0.01\\
58.86	0.01\\
58.87	0.01\\
58.88	0.01\\
58.89	0.01\\
58.9	0.01\\
58.91	0.01\\
58.92	0.01\\
58.93	0.01\\
58.94	0.01\\
58.95	0.01\\
58.96	0.01\\
58.97	0.01\\
58.98	0.01\\
58.99	0.01\\
59	0.01\\
59.01	0.01\\
59.02	0.01\\
59.03	0.01\\
59.04	0.01\\
59.05	0.01\\
59.06	0.01\\
59.07	0.01\\
59.08	0.01\\
59.09	0.01\\
59.1	0.01\\
59.11	0.01\\
59.12	0.01\\
59.13	0.01\\
59.14	0.01\\
59.15	0.01\\
59.16	0.01\\
59.17	0.01\\
59.18	0.01\\
59.19	0.01\\
59.2	0.01\\
59.21	0.01\\
59.22	0.01\\
59.23	0.01\\
59.24	0.01\\
59.25	0.01\\
59.26	0.01\\
59.27	0.01\\
59.28	0.01\\
59.29	0.01\\
59.3	0.01\\
59.31	0.01\\
59.32	0.01\\
59.33	0.01\\
59.34	0.01\\
59.35	0.01\\
59.36	0.01\\
59.37	0.01\\
59.38	0.01\\
59.39	0.01\\
59.4	0.01\\
59.41	0.01\\
59.42	0.01\\
59.43	0.01\\
59.44	0.01\\
59.45	0.01\\
59.46	0.01\\
59.47	0.01\\
59.48	0.01\\
59.49	0.01\\
59.5	0.01\\
59.51	0.01\\
59.52	0.01\\
59.53	0.01\\
59.54	0.01\\
59.55	0.01\\
59.56	0.01\\
59.57	0.01\\
59.58	0.01\\
59.59	0.01\\
59.6	0.01\\
59.61	0.01\\
59.62	0.01\\
59.63	0.01\\
59.64	0.01\\
59.65	0.01\\
59.66	0.01\\
59.67	0.01\\
59.68	0.01\\
59.69	0.01\\
59.7	0.01\\
59.71	0.01\\
59.72	0.01\\
59.73	0.01\\
59.74	0.01\\
59.75	0.01\\
59.76	0.01\\
59.77	0.01\\
59.78	0.01\\
59.79	0.01\\
59.8	0.01\\
59.81	0.01\\
59.82	0.01\\
59.83	0.01\\
59.84	0.01\\
59.85	0.01\\
59.86	0.01\\
59.87	0.01\\
59.88	0.01\\
59.89	0.01\\
59.9	0.01\\
59.91	0.01\\
59.92	0.01\\
59.93	0.01\\
59.94	0.01\\
59.95	0.01\\
59.96	0.01\\
59.97	0.01\\
59.98	0.01\\
59.99	0.01\\
60	0.01\\
60.01	0.01\\
60.02	0.01\\
60.03	0.01\\
60.04	0.01\\
60.05	0.01\\
60.06	0.01\\
60.07	0.01\\
60.08	0.01\\
60.09	0.01\\
60.1	0.01\\
60.11	0.01\\
60.12	0.01\\
60.13	0.01\\
60.14	0.01\\
60.15	0.01\\
60.16	0.01\\
60.17	0.01\\
60.18	0.01\\
60.19	0.01\\
60.2	0.01\\
60.21	0.01\\
60.22	0.01\\
60.23	0.01\\
60.24	0.01\\
60.25	0.01\\
60.26	0.01\\
60.27	0.01\\
60.28	0.01\\
60.29	0.01\\
60.3	0.01\\
60.31	0.01\\
60.32	0.01\\
60.33	0.01\\
60.34	0.01\\
60.35	0.01\\
60.36	0.01\\
60.37	0.01\\
60.38	0.01\\
60.39	0.01\\
60.4	0.01\\
60.41	0.01\\
60.42	0.01\\
60.43	0.01\\
60.44	0.01\\
60.45	0.01\\
60.46	0.01\\
60.47	0.01\\
60.48	0.01\\
60.49	0.01\\
60.5	0.01\\
60.51	0.01\\
60.52	0.01\\
60.53	0.01\\
60.54	0.01\\
60.55	0.01\\
60.56	0.01\\
60.57	0.01\\
60.58	0.01\\
60.59	0.01\\
60.6	0.01\\
60.61	0.01\\
60.62	0.01\\
60.63	0.01\\
60.64	0.01\\
60.65	0.01\\
60.66	0.01\\
60.67	0.01\\
60.68	0.01\\
60.69	0.01\\
60.7	0.01\\
60.71	0.01\\
60.72	0.01\\
60.73	0.01\\
60.74	0.01\\
60.75	0.01\\
60.76	0.01\\
60.77	0.01\\
60.78	0.01\\
60.79	0.01\\
60.8	0.01\\
60.81	0.01\\
60.82	0.01\\
60.83	0.01\\
60.84	0.01\\
60.85	0.01\\
60.86	0.01\\
60.87	0.01\\
60.88	0.01\\
60.89	0.01\\
60.9	0.01\\
60.91	0.01\\
60.92	0.01\\
60.93	0.01\\
60.94	0.01\\
60.95	0.01\\
60.96	0.01\\
60.97	0.01\\
60.98	0.01\\
60.99	0.01\\
61	0.01\\
61.01	0.01\\
61.02	0.01\\
61.03	0.01\\
61.04	0.01\\
61.05	0.01\\
61.06	0.01\\
61.07	0.01\\
61.08	0.01\\
61.09	0.01\\
61.1	0.01\\
61.11	0.01\\
61.12	0.01\\
61.13	0.01\\
61.14	0.01\\
61.15	0.01\\
61.16	0.01\\
61.17	0.01\\
61.18	0.01\\
61.19	0.01\\
61.2	0.01\\
61.21	0.01\\
61.22	0.01\\
61.23	0.01\\
61.24	0.01\\
61.25	0.01\\
61.26	0.01\\
61.27	0.01\\
61.28	0.01\\
61.29	0.01\\
61.3	0.01\\
61.31	0.01\\
61.32	0.01\\
61.33	0.01\\
61.34	0.01\\
61.35	0.01\\
61.36	0.01\\
61.37	0.01\\
61.38	0.01\\
61.39	0.01\\
61.4	0.01\\
61.41	0.01\\
61.42	0.01\\
61.43	0.01\\
61.44	0.01\\
61.45	0.01\\
61.46	0.01\\
61.47	0.01\\
61.48	0.01\\
61.49	0.01\\
61.5	0.01\\
61.51	0.01\\
61.52	0.01\\
61.53	0.01\\
61.54	0.01\\
61.55	0.01\\
61.56	0.01\\
61.57	0.01\\
61.58	0.01\\
61.59	0.01\\
61.6	0.01\\
61.61	0.01\\
61.62	0.01\\
61.63	0.01\\
61.64	0.01\\
61.65	0.01\\
61.66	0.01\\
61.67	0.01\\
61.68	0.01\\
61.69	0.01\\
61.7	0.01\\
61.71	0.01\\
61.72	0.01\\
61.73	0.01\\
61.74	0.01\\
61.75	0.01\\
61.76	0.01\\
61.77	0.01\\
61.78	0.01\\
61.79	0.01\\
61.8	0.01\\
61.81	0.01\\
61.82	0.01\\
61.83	0.01\\
61.84	0.01\\
61.85	0.01\\
61.86	0.01\\
61.87	0.01\\
61.88	0.01\\
61.89	0.01\\
61.9	0.01\\
61.91	0.01\\
61.92	0.01\\
61.93	0.01\\
61.94	0.01\\
61.95	0.01\\
61.96	0.01\\
61.97	0.01\\
61.98	0.01\\
61.99	0.01\\
62	0.01\\
62.01	0.01\\
62.02	0.01\\
62.03	0.01\\
62.04	0.01\\
62.05	0.01\\
62.06	0.01\\
62.07	0.01\\
62.08	0.01\\
62.09	0.01\\
62.1	0.01\\
62.11	0.01\\
62.12	0.01\\
62.13	0.01\\
62.14	0.01\\
62.15	0.01\\
62.16	0.01\\
62.17	0.01\\
62.18	0.01\\
62.19	0.01\\
62.2	0.01\\
62.21	0.01\\
62.22	0.01\\
62.23	0.01\\
62.24	0.01\\
62.25	0.01\\
62.26	0.01\\
62.27	0.01\\
62.28	0.01\\
62.29	0.01\\
62.3	0.01\\
62.31	0.01\\
62.32	0.01\\
62.33	0.01\\
62.34	0.01\\
62.35	0.01\\
62.36	0.01\\
62.37	0.01\\
62.38	0.01\\
62.39	0.01\\
62.4	0.01\\
62.41	0.01\\
62.42	0.01\\
62.43	0.01\\
62.44	0.01\\
62.45	0.01\\
62.46	0.01\\
62.47	0.01\\
62.48	0.01\\
62.49	0.01\\
62.5	0.01\\
62.51	0.01\\
62.52	0.01\\
62.53	0.01\\
62.54	0.01\\
62.55	0.01\\
62.56	0.01\\
62.57	0.01\\
62.58	0.01\\
62.59	0.01\\
62.6	0.01\\
62.61	0.01\\
62.62	0.01\\
62.63	0.01\\
62.64	0.01\\
62.65	0.01\\
62.66	0.01\\
62.67	0.01\\
62.68	0.01\\
62.69	0.01\\
62.7	0.01\\
62.71	0.01\\
62.72	0.01\\
62.73	0.01\\
62.74	0.01\\
62.75	0.01\\
62.76	0.01\\
62.77	0.01\\
62.78	0.01\\
62.79	0.01\\
62.8	0.01\\
62.81	0.01\\
62.82	0.01\\
62.83	0.01\\
62.84	0.01\\
62.85	0.01\\
62.86	0.01\\
62.87	0.01\\
62.88	0.01\\
62.89	0.01\\
62.9	0.01\\
62.91	0.01\\
62.92	0.01\\
62.93	0.01\\
62.94	0.01\\
62.95	0.01\\
62.96	0.01\\
62.97	0.01\\
62.98	0.01\\
62.99	0.01\\
63	0.01\\
63.01	0.01\\
63.02	0.01\\
63.03	0.01\\
63.04	0.01\\
63.05	0.01\\
63.06	0.01\\
63.07	0.01\\
63.08	0.01\\
63.09	0.01\\
63.1	0.01\\
63.11	0.01\\
63.12	0.01\\
63.13	0.01\\
63.14	0.01\\
63.15	0.01\\
63.16	0.01\\
63.17	0.01\\
63.18	0.01\\
63.19	0.01\\
63.2	0.01\\
63.21	0.01\\
63.22	0.01\\
63.23	0.01\\
63.24	0.01\\
63.25	0.01\\
63.26	0.01\\
63.27	0.01\\
63.28	0.01\\
63.29	0.01\\
63.3	0.01\\
63.31	0.01\\
63.32	0.01\\
63.33	0.01\\
63.34	0.01\\
63.35	0.01\\
63.36	0.01\\
63.37	0.01\\
63.38	0.01\\
63.39	0.01\\
63.4	0.01\\
63.41	0.01\\
63.42	0.01\\
63.43	0.01\\
63.44	0.01\\
63.45	0.01\\
63.46	0.01\\
63.47	0.01\\
63.48	0.01\\
63.49	0.01\\
63.5	0.01\\
63.51	0.01\\
63.52	0.01\\
63.53	0.01\\
63.54	0.01\\
63.55	0.01\\
63.56	0.01\\
63.57	0.01\\
63.58	0.01\\
63.59	0.01\\
63.6	0.01\\
63.61	0.01\\
63.62	0.01\\
63.63	0.01\\
63.64	0.01\\
63.65	0.01\\
63.66	0.01\\
63.67	0.01\\
63.68	0.01\\
63.69	0.01\\
63.7	0.01\\
63.71	0.01\\
63.72	0.01\\
63.73	0.01\\
63.74	0.01\\
63.75	0.01\\
63.76	0.01\\
63.77	0.01\\
63.78	0.01\\
63.79	0.01\\
63.8	0.01\\
63.81	0.01\\
63.82	0.01\\
63.83	0.01\\
63.84	0.01\\
63.85	0.01\\
63.86	0.01\\
63.87	0.01\\
63.88	0.01\\
63.89	0.01\\
63.9	0.01\\
63.91	0.01\\
63.92	0.01\\
63.93	0.01\\
63.94	0.01\\
63.95	0.01\\
63.96	0.01\\
63.97	0.01\\
63.98	0.01\\
63.99	0.01\\
64	0.01\\
64.01	0.01\\
64.02	0.01\\
64.03	0.01\\
64.04	0.01\\
64.05	0.01\\
64.06	0.01\\
64.07	0.01\\
64.08	0.01\\
64.09	0.01\\
64.1	0.01\\
64.11	0.01\\
64.12	0.01\\
64.13	0.01\\
64.14	0.01\\
64.15	0.01\\
64.16	0.01\\
64.17	0.01\\
64.18	0.01\\
64.19	0.01\\
64.2	0.01\\
64.21	0.01\\
64.22	0.01\\
64.23	0.01\\
64.24	0.01\\
64.25	0.01\\
64.26	0.01\\
64.27	0.01\\
64.28	0.01\\
64.29	0.01\\
64.3	0.01\\
64.31	0.01\\
64.32	0.01\\
64.33	0.01\\
64.34	0.01\\
64.35	0.01\\
64.36	0.01\\
64.37	0.01\\
64.38	0.01\\
64.39	0.01\\
64.4	0.01\\
64.41	0.01\\
64.42	0.01\\
64.43	0.01\\
64.44	0.01\\
64.45	0.01\\
64.46	0.01\\
64.47	0.01\\
64.48	0.01\\
64.49	0.01\\
64.5	0.01\\
64.51	0.01\\
64.52	0.01\\
64.53	0.01\\
64.54	0.01\\
64.55	0.01\\
64.56	0.01\\
64.57	0.01\\
64.58	0.01\\
64.59	0.01\\
64.6	0.01\\
64.61	0.01\\
64.62	0.01\\
64.63	0.01\\
64.64	0.01\\
64.65	0.01\\
64.66	0.01\\
64.67	0.01\\
64.68	0.01\\
64.69	0.01\\
64.7	0.01\\
64.71	0.01\\
64.72	0.01\\
64.73	0.01\\
64.74	0.01\\
64.75	0.01\\
64.76	0.01\\
64.77	0.01\\
64.78	0.01\\
64.79	0.01\\
64.8	0.01\\
64.81	0.01\\
64.82	0.01\\
64.83	0.01\\
64.84	0.01\\
64.85	0.01\\
64.86	0.01\\
64.87	0.01\\
64.88	0.01\\
64.89	0.01\\
64.9	0.01\\
64.91	0.01\\
64.92	0.01\\
64.93	0.01\\
64.94	0.01\\
64.95	0.01\\
64.96	0.01\\
64.97	0.01\\
64.98	0.01\\
64.99	0.01\\
65	0.01\\
65.01	0.01\\
65.02	0.01\\
65.03	0.01\\
65.04	0.01\\
65.05	0.01\\
65.06	0.01\\
65.07	0.01\\
65.08	0.01\\
65.09	0.01\\
65.1	0.01\\
65.11	0.01\\
65.12	0.01\\
65.13	0.01\\
65.14	0.01\\
65.15	0.01\\
65.16	0.01\\
65.17	0.01\\
65.18	0.01\\
65.19	0.01\\
65.2	0.01\\
65.21	0.01\\
65.22	0.01\\
65.23	0.01\\
65.24	0.01\\
65.25	0.01\\
65.26	0.01\\
65.27	0.01\\
65.28	0.01\\
65.29	0.01\\
65.3	0.01\\
65.31	0.01\\
65.32	0.01\\
65.33	0.01\\
65.34	0.01\\
65.35	0.01\\
65.36	0.01\\
65.37	0.01\\
65.38	0.01\\
65.39	0.01\\
65.4	0.01\\
65.41	0.01\\
65.42	0.01\\
65.43	0.01\\
65.44	0.01\\
65.45	0.01\\
65.46	0.01\\
65.47	0.01\\
65.48	0.01\\
65.49	0.01\\
65.5	0.01\\
65.51	0.01\\
65.52	0.01\\
65.53	0.01\\
65.54	0.01\\
65.55	0.01\\
65.56	0.01\\
65.57	0.01\\
65.58	0.01\\
65.59	0.01\\
65.6	0.01\\
65.61	0.01\\
65.62	0.01\\
65.63	0.01\\
65.64	0.01\\
65.65	0.01\\
65.66	0.01\\
65.67	0.01\\
65.68	0.01\\
65.69	0.01\\
65.7	0.01\\
65.71	0.01\\
65.72	0.01\\
65.73	0.01\\
65.74	0.01\\
65.75	0.01\\
65.76	0.01\\
65.77	0.01\\
65.78	0.01\\
65.79	0.01\\
65.8	0.01\\
65.81	0.01\\
65.82	0.01\\
65.83	0.01\\
65.84	0.01\\
65.85	0.01\\
65.86	0.01\\
65.87	0.01\\
65.88	0.01\\
65.89	0.01\\
65.9	0.01\\
65.91	0.01\\
65.92	0.01\\
65.93	0.01\\
65.94	0.01\\
65.95	0.01\\
65.96	0.01\\
65.97	0.01\\
65.98	0.01\\
65.99	0.01\\
66	0.01\\
66.01	0.01\\
66.02	0.01\\
66.03	0.01\\
66.04	0.01\\
66.05	0.01\\
66.06	0.01\\
66.07	0.01\\
66.08	0.01\\
66.09	0.01\\
66.1	0.01\\
66.11	0.01\\
66.12	0.01\\
66.13	0.01\\
66.14	0.01\\
66.15	0.01\\
66.16	0.01\\
66.17	0.01\\
66.18	0.01\\
66.19	0.01\\
66.2	0.01\\
66.21	0.01\\
66.22	0.01\\
66.23	0.01\\
66.24	0.01\\
66.25	0.01\\
66.26	0.01\\
66.27	0.01\\
66.28	0.01\\
66.29	0.01\\
66.3	0.01\\
66.31	0.01\\
66.32	0.01\\
66.33	0.01\\
66.34	0.01\\
66.35	0.01\\
66.36	0.01\\
66.37	0.01\\
66.38	0.01\\
66.39	0.01\\
66.4	0.01\\
66.41	0.01\\
66.42	0.01\\
66.43	0.01\\
66.44	0.01\\
66.45	0.01\\
66.46	0.01\\
66.47	0.01\\
66.48	0.01\\
66.49	0.01\\
66.5	0.01\\
66.51	0.01\\
66.52	0.01\\
66.53	0.01\\
66.54	0.01\\
66.55	0.01\\
66.56	0.01\\
66.57	0.01\\
66.58	0.01\\
66.59	0.01\\
66.6	0.01\\
66.61	0.01\\
66.62	0.01\\
66.63	0.01\\
66.64	0.01\\
66.65	0.01\\
66.66	0.01\\
66.67	0.01\\
66.68	0.01\\
66.69	0.01\\
66.7	0.01\\
66.71	0.01\\
66.72	0.01\\
66.73	0.01\\
66.74	0.01\\
66.75	0.01\\
66.76	0.01\\
66.77	0.01\\
66.78	0.01\\
66.79	0.01\\
66.8	0.01\\
66.81	0.01\\
66.82	0.01\\
66.83	0.01\\
66.84	0.01\\
66.85	0.01\\
66.86	0.01\\
66.87	0.01\\
66.88	0.01\\
66.89	0.01\\
66.9	0.01\\
66.91	0.01\\
66.92	0.01\\
66.93	0.01\\
66.94	0.01\\
66.95	0.01\\
66.96	0.01\\
66.97	0.01\\
66.98	0.01\\
66.99	0.01\\
67	0.01\\
67.01	0.01\\
67.02	0.01\\
67.03	0.01\\
67.04	0.01\\
67.05	0.01\\
67.06	0.01\\
67.07	0.01\\
67.08	0.01\\
67.09	0.01\\
67.1	0.01\\
67.11	0.01\\
67.12	0.01\\
67.13	0.01\\
67.14	0.01\\
67.15	0.01\\
67.16	0.01\\
67.17	0.01\\
67.18	0.01\\
67.19	0.01\\
67.2	0.01\\
67.21	0.01\\
67.22	0.01\\
67.23	0.01\\
67.24	0.01\\
67.25	0.01\\
67.26	0.01\\
67.27	0.01\\
67.28	0.01\\
67.29	0.01\\
67.3	0.01\\
67.31	0.01\\
67.32	0.01\\
67.33	0.01\\
67.34	0.01\\
67.35	0.01\\
67.36	0.01\\
67.37	0.01\\
67.38	0.01\\
67.39	0.01\\
67.4	0.01\\
67.41	0.01\\
67.42	0.01\\
67.43	0.01\\
67.44	0.01\\
67.45	0.01\\
67.46	0.01\\
67.47	0.01\\
67.48	0.01\\
67.49	0.01\\
67.5	0.01\\
67.51	0.01\\
67.52	0.01\\
67.53	0.01\\
67.54	0.01\\
67.55	0.01\\
67.56	0.01\\
67.57	0.01\\
67.58	0.01\\
67.59	0.01\\
67.6	0.01\\
67.61	0.01\\
67.62	0.01\\
67.63	0.01\\
67.64	0.01\\
67.65	0.01\\
67.66	0.01\\
67.67	0.01\\
67.68	0.01\\
67.69	0.01\\
67.7	0.01\\
67.71	0.01\\
67.72	0.01\\
67.73	0.01\\
67.74	0.01\\
67.75	0.01\\
67.76	0.01\\
67.77	0.01\\
67.78	0.01\\
67.79	0.01\\
67.8	0.01\\
67.81	0.01\\
67.82	0.01\\
67.83	0.01\\
67.84	0.01\\
67.85	0.01\\
67.86	0.01\\
67.87	0.01\\
67.88	0.01\\
67.89	0.01\\
67.9	0.01\\
67.91	0.01\\
67.92	0.01\\
67.93	0.01\\
67.94	0.01\\
67.95	0.01\\
67.96	0.01\\
67.97	0.01\\
67.98	0.01\\
67.99	0.01\\
68	0.01\\
68.01	0.01\\
68.02	0.01\\
68.03	0.01\\
68.04	0.01\\
68.05	0.01\\
68.06	0.01\\
68.07	0.01\\
68.08	0.01\\
68.09	0.01\\
68.1	0.01\\
68.11	0.01\\
68.12	0.01\\
68.13	0.01\\
68.14	0.01\\
68.15	0.01\\
68.16	0.01\\
68.17	0.01\\
68.18	0.01\\
68.19	0.01\\
68.2	0.01\\
68.21	0.01\\
68.22	0.01\\
68.23	0.01\\
68.24	0.01\\
68.25	0.01\\
68.26	0.01\\
68.27	0.01\\
68.28	0.01\\
68.29	0.01\\
68.3	0.01\\
68.31	0.01\\
68.32	0.01\\
68.33	0.01\\
68.34	0.01\\
68.35	0.01\\
68.36	0.01\\
68.37	0.01\\
68.38	0.01\\
68.39	0.01\\
68.4	0.01\\
68.41	0.01\\
68.42	0.01\\
68.43	0.01\\
68.44	0.01\\
68.45	0.01\\
68.46	0.01\\
68.47	0.01\\
68.48	0.01\\
68.49	0.01\\
68.5	0.01\\
68.51	0.01\\
68.52	0.01\\
68.53	0.01\\
68.54	0.01\\
68.55	0.01\\
68.56	0.01\\
68.57	0.01\\
68.58	0.01\\
68.59	0.01\\
68.6	0.01\\
68.61	0.01\\
68.62	0.01\\
68.63	0.01\\
68.64	0.01\\
68.65	0.01\\
68.66	0.01\\
68.67	0.01\\
68.68	0.01\\
68.69	0.01\\
68.7	0.01\\
68.71	0.01\\
68.72	0.01\\
68.73	0.01\\
68.74	0.01\\
68.75	0.01\\
68.76	0.01\\
68.77	0.01\\
68.78	0.01\\
68.79	0.01\\
68.8	0.01\\
68.81	0.01\\
68.82	0.01\\
68.83	0.01\\
68.84	0.01\\
68.85	0.01\\
68.86	0.01\\
68.87	0.01\\
68.88	0.01\\
68.89	0.01\\
68.9	0.01\\
68.91	0.01\\
68.92	0.01\\
68.93	0.01\\
68.94	0.01\\
68.95	0.01\\
68.96	0.01\\
68.97	0.01\\
68.98	0.01\\
68.99	0.01\\
69	0.01\\
69.01	0.01\\
69.02	0.01\\
69.03	0.01\\
69.04	0.01\\
69.05	0.01\\
69.06	0.01\\
69.07	0.01\\
69.08	0.01\\
69.09	0.01\\
69.1	0.01\\
69.11	0.01\\
69.12	0.01\\
69.13	0.01\\
69.14	0.01\\
69.15	0.01\\
69.16	0.01\\
69.17	0.01\\
69.18	0.01\\
69.19	0.01\\
69.2	0.01\\
69.21	0.01\\
69.22	0.01\\
69.23	0.01\\
69.24	0.01\\
69.25	0.01\\
69.26	0.01\\
69.27	0.01\\
69.28	0.01\\
69.29	0.01\\
69.3	0.01\\
69.31	0.01\\
69.32	0.01\\
69.33	0.01\\
69.34	0.01\\
69.35	0.01\\
69.36	0.01\\
69.37	0.01\\
69.38	0.01\\
69.39	0.01\\
69.4	0.01\\
69.41	0.01\\
69.42	0.01\\
69.43	0.01\\
69.44	0.01\\
69.45	0.01\\
69.46	0.01\\
69.47	0.01\\
69.48	0.01\\
69.49	0.01\\
69.5	0.01\\
69.51	0.01\\
69.52	0.01\\
69.53	0.01\\
69.54	0.01\\
69.55	0.01\\
69.56	0.01\\
69.57	0.01\\
69.58	0.01\\
69.59	0.01\\
69.6	0.01\\
69.61	0.01\\
69.62	0.01\\
69.63	0.01\\
69.64	0.01\\
69.65	0.01\\
69.66	0.01\\
69.67	0.01\\
69.68	0.01\\
69.69	0.01\\
69.7	0.01\\
69.71	0.01\\
69.72	0.01\\
69.73	0.01\\
69.74	0.01\\
69.75	0.01\\
69.76	0.01\\
69.77	0.01\\
69.78	0.01\\
69.79	0.01\\
69.8	0.01\\
69.81	0.01\\
69.82	0.01\\
69.83	0.01\\
69.84	0.01\\
69.85	0.01\\
69.86	0.01\\
69.87	0.01\\
69.88	0.01\\
69.89	0.01\\
69.9	0.01\\
69.91	0.01\\
69.92	0.01\\
69.93	0.01\\
69.94	0.01\\
69.95	0.01\\
69.96	0.01\\
69.97	0.01\\
69.98	0.01\\
69.99	0.01\\
70	0.01\\
70.01	0.01\\
70.02	0.01\\
70.03	0.01\\
70.04	0.01\\
70.05	0.01\\
70.06	0.01\\
70.07	0.01\\
70.08	0.01\\
70.09	0.01\\
70.1	0.01\\
70.11	0.01\\
70.12	0.01\\
70.13	0.01\\
70.14	0.01\\
70.15	0.01\\
70.16	0.01\\
70.17	0.01\\
70.18	0.01\\
70.19	0.01\\
70.2	0.01\\
70.21	0.01\\
70.22	0.01\\
70.23	0.01\\
70.24	0.01\\
70.25	0.01\\
70.26	0.01\\
70.27	0.01\\
70.28	0.01\\
70.29	0.01\\
70.3	0.01\\
70.31	0.01\\
70.32	0.01\\
70.33	0.01\\
70.34	0.01\\
70.35	0.01\\
70.36	0.01\\
70.37	0.01\\
70.38	0.01\\
70.39	0.01\\
70.4	0.01\\
70.41	0.01\\
70.42	0.01\\
70.43	0.01\\
70.44	0.01\\
70.45	0.01\\
70.46	0.01\\
70.47	0.01\\
70.48	0.01\\
70.49	0.01\\
70.5	0.01\\
70.51	0.01\\
70.52	0.01\\
70.53	0.01\\
70.54	0.01\\
70.55	0.01\\
70.56	0.01\\
70.57	0.01\\
70.58	0.01\\
70.59	0.01\\
70.6	0.01\\
70.61	0.01\\
70.62	0.01\\
70.63	0.01\\
70.64	0.01\\
70.65	0.01\\
70.66	0.01\\
70.67	0.01\\
70.68	0.01\\
70.69	0.01\\
70.7	0.01\\
70.71	0.01\\
70.72	0.01\\
70.73	0.01\\
70.74	0.01\\
70.75	0.01\\
70.76	0.01\\
70.77	0.01\\
70.78	0.01\\
70.79	0.01\\
70.8	0.01\\
70.81	0.01\\
70.82	0.01\\
70.83	0.01\\
70.84	0.01\\
70.85	0.01\\
70.86	0.01\\
70.87	0.01\\
70.88	0.01\\
70.89	0.01\\
70.9	0.01\\
70.91	0.01\\
70.92	0.01\\
70.93	0.01\\
70.94	0.01\\
70.95	0.01\\
70.96	0.01\\
70.97	0.01\\
70.98	0.01\\
70.99	0.01\\
71	0.01\\
71.01	0.01\\
71.02	0.01\\
71.03	0.01\\
71.04	0.01\\
71.05	0.01\\
71.06	0.01\\
71.07	0.01\\
71.08	0.01\\
71.09	0.01\\
71.1	0.01\\
71.11	0.01\\
71.12	0.01\\
71.13	0.01\\
71.14	0.01\\
71.15	0.01\\
71.16	0.01\\
71.17	0.01\\
71.18	0.01\\
71.19	0.01\\
71.2	0.01\\
71.21	0.01\\
71.22	0.01\\
71.23	0.01\\
71.24	0.01\\
71.25	0.01\\
71.26	0.01\\
71.27	0.01\\
71.28	0.01\\
71.29	0.01\\
71.3	0.01\\
71.31	0.01\\
71.32	0.01\\
71.33	0.01\\
71.34	0.01\\
71.35	0.01\\
71.36	0.01\\
71.37	0.01\\
71.38	0.01\\
71.39	0.01\\
71.4	0.01\\
71.41	0.01\\
71.42	0.01\\
71.43	0.01\\
71.44	0.01\\
71.45	0.01\\
71.46	0.01\\
71.47	0.01\\
71.48	0.01\\
71.49	0.01\\
71.5	0.01\\
71.51	0.01\\
71.52	0.01\\
71.53	0.01\\
71.54	0.01\\
71.55	0.01\\
71.56	0.01\\
71.57	0.01\\
71.58	0.01\\
71.59	0.01\\
71.6	0.01\\
71.61	0.01\\
71.62	0.01\\
71.63	0.01\\
71.64	0.01\\
71.65	0.01\\
71.66	0.01\\
71.67	0.01\\
71.68	0.01\\
71.69	0.01\\
71.7	0.01\\
71.71	0.01\\
71.72	0.01\\
71.73	0.01\\
71.74	0.01\\
71.75	0.01\\
71.76	0.01\\
71.77	0.01\\
71.78	0.01\\
71.79	0.01\\
71.8	0.01\\
71.81	0.01\\
71.82	0.01\\
71.83	0.01\\
71.84	0.01\\
71.85	0.01\\
71.86	0.01\\
71.87	0.01\\
71.88	0.01\\
71.89	0.01\\
71.9	0.01\\
71.91	0.01\\
71.92	0.01\\
71.93	0.01\\
71.94	0.01\\
71.95	0.01\\
71.96	0.01\\
71.97	0.01\\
71.98	0.01\\
71.99	0.01\\
72	0.01\\
72.01	0.01\\
72.02	0.01\\
72.03	0.01\\
72.04	0.01\\
72.05	0.01\\
72.06	0.01\\
72.07	0.01\\
72.08	0.01\\
72.09	0.01\\
72.1	0.01\\
72.11	0.01\\
72.12	0.01\\
72.13	0.01\\
72.14	0.01\\
72.15	0.01\\
72.16	0.01\\
72.17	0.01\\
72.18	0.01\\
72.19	0.01\\
72.2	0.01\\
72.21	0.01\\
72.22	0.01\\
72.23	0.01\\
72.24	0.01\\
72.25	0.01\\
72.26	0.01\\
72.27	0.01\\
72.28	0.01\\
72.29	0.01\\
72.3	0.01\\
72.31	0.01\\
72.32	0.01\\
72.33	0.01\\
72.34	0.01\\
72.35	0.01\\
72.36	0.01\\
72.37	0.01\\
72.38	0.01\\
72.39	0.01\\
72.4	0.01\\
72.41	0.01\\
72.42	0.01\\
72.43	0.01\\
72.44	0.01\\
72.45	0.01\\
72.46	0.01\\
72.47	0.01\\
72.48	0.01\\
72.49	0.01\\
72.5	0.01\\
72.51	0.01\\
72.52	0.01\\
72.53	0.01\\
72.54	0.01\\
72.55	0.01\\
72.56	0.01\\
72.57	0.01\\
72.58	0.01\\
72.59	0.01\\
72.6	0.01\\
72.61	0.01\\
72.62	0.01\\
72.63	0.01\\
72.64	0.01\\
72.65	0.01\\
72.66	0.01\\
72.67	0.01\\
72.68	0.01\\
72.69	0.01\\
72.7	0.01\\
72.71	0.01\\
72.72	0.01\\
72.73	0.01\\
72.74	0.01\\
72.75	0.01\\
72.76	0.01\\
72.77	0.01\\
72.78	0.01\\
72.79	0.01\\
72.8	0.01\\
72.81	0.01\\
72.82	0.01\\
72.83	0.01\\
72.84	0.01\\
72.85	0.01\\
72.86	0.01\\
72.87	0.01\\
72.88	0.01\\
72.89	0.01\\
72.9	0.01\\
72.91	0.01\\
72.92	0.01\\
72.93	0.01\\
72.94	0.01\\
72.95	0.01\\
72.96	0.01\\
72.97	0.01\\
72.98	0.01\\
72.99	0.01\\
73	0.01\\
73.01	0.01\\
73.02	0.01\\
73.03	0.01\\
73.04	0.01\\
73.05	0.01\\
73.06	0.01\\
73.07	0.01\\
73.08	0.01\\
73.09	0.01\\
73.1	0.01\\
73.11	0.01\\
73.12	0.01\\
73.13	0.01\\
73.14	0.01\\
73.15	0.01\\
73.16	0.01\\
73.17	0.01\\
73.18	0.01\\
73.19	0.01\\
73.2	0.01\\
73.21	0.01\\
73.22	0.01\\
73.23	0.01\\
73.24	0.01\\
73.25	0.01\\
73.26	0.01\\
73.27	0.01\\
73.28	0.01\\
73.29	0.01\\
73.3	0.01\\
73.31	0.01\\
73.32	0.01\\
73.33	0.01\\
73.34	0.01\\
73.35	0.01\\
73.36	0.01\\
73.37	0.01\\
73.38	0.01\\
73.39	0.01\\
73.4	0.01\\
73.41	0.01\\
73.42	0.01\\
73.43	0.01\\
73.44	0.01\\
73.45	0.01\\
73.46	0.01\\
73.47	0.01\\
73.48	0.01\\
73.49	0.01\\
73.5	0.01\\
73.51	0.01\\
73.52	0.01\\
73.53	0.01\\
73.54	0.01\\
73.55	0.01\\
73.56	0.01\\
73.57	0.01\\
73.58	0.01\\
73.59	0.01\\
73.6	0.01\\
73.61	0.01\\
73.62	0.01\\
73.63	0.01\\
73.64	0.01\\
73.65	0.01\\
73.66	0.01\\
73.67	0.01\\
73.68	0.01\\
73.69	0.01\\
73.7	0.01\\
73.71	0.01\\
73.72	0.01\\
73.73	0.01\\
73.74	0.01\\
73.75	0.01\\
73.76	0.01\\
73.77	0.01\\
73.78	0.01\\
73.79	0.01\\
73.8	0.01\\
73.81	0.01\\
73.82	0.01\\
73.83	0.01\\
73.84	0.01\\
73.85	0.01\\
73.86	0.01\\
73.87	0.01\\
73.88	0.01\\
73.89	0.01\\
73.9	0.01\\
73.91	0.01\\
73.92	0.01\\
73.93	0.01\\
73.94	0.01\\
73.95	0.01\\
73.96	0.01\\
73.97	0.01\\
73.98	0.01\\
73.99	0.01\\
74	0.01\\
74.01	0.01\\
74.02	0.01\\
74.03	0.01\\
74.04	0.01\\
74.05	0.01\\
74.06	0.01\\
74.07	0.01\\
74.08	0.01\\
74.09	0.01\\
74.1	0.01\\
74.11	0.01\\
74.12	0.01\\
74.13	0.01\\
74.14	0.01\\
74.15	0.01\\
74.16	0.01\\
74.17	0.01\\
74.18	0.01\\
74.19	0.01\\
74.2	0.01\\
74.21	0.01\\
74.22	0.01\\
74.23	0.01\\
74.24	0.01\\
74.25	0.01\\
74.26	0.01\\
74.27	0.01\\
74.28	0.01\\
74.29	0.01\\
74.3	0.01\\
74.31	0.01\\
74.32	0.01\\
74.33	0.01\\
74.34	0.01\\
74.35	0.01\\
74.36	0.01\\
74.37	0.01\\
74.38	0.01\\
74.39	0.01\\
74.4	0.01\\
74.41	0.01\\
74.42	0.01\\
74.43	0.01\\
74.44	0.01\\
74.45	0.01\\
74.46	0.01\\
74.47	0.01\\
74.48	0.01\\
74.49	0.01\\
74.5	0.01\\
74.51	0.01\\
74.52	0.01\\
74.53	0.01\\
74.54	0.01\\
74.55	0.01\\
74.56	0.01\\
74.57	0.01\\
74.58	0.01\\
74.59	0.01\\
74.6	0.01\\
74.61	0.01\\
74.62	0.01\\
74.63	0.01\\
74.64	0.01\\
74.65	0.01\\
74.66	0.01\\
74.67	0.01\\
74.68	0.01\\
74.69	0.01\\
74.7	0.01\\
74.71	0.01\\
74.72	0.01\\
74.73	0.01\\
74.74	0.01\\
74.75	0.01\\
74.76	0.01\\
74.77	0.01\\
74.78	0.01\\
74.79	0.01\\
74.8	0.01\\
74.81	0.01\\
74.82	0.01\\
74.83	0.01\\
74.84	0.01\\
74.85	0.01\\
74.86	0.01\\
74.87	0.01\\
74.88	0.01\\
74.89	0.01\\
74.9	0.01\\
74.91	0.01\\
74.92	0.01\\
74.93	0.01\\
74.94	0.01\\
74.95	0.01\\
74.96	0.01\\
74.97	0.01\\
74.98	0.01\\
74.99	0.01\\
75	0.01\\
75.01	0.01\\
75.02	0.01\\
75.03	0.01\\
75.04	0.01\\
75.05	0.01\\
75.06	0.01\\
75.07	0.01\\
75.08	0.01\\
75.09	0.01\\
75.1	0.01\\
75.11	0.01\\
75.12	0.01\\
75.13	0.01\\
75.14	0.01\\
75.15	0.01\\
75.16	0.01\\
75.17	0.01\\
75.18	0.01\\
75.19	0.01\\
75.2	0.01\\
75.21	0.01\\
75.22	0.01\\
75.23	0.01\\
75.24	0.01\\
75.25	0.01\\
75.26	0.01\\
75.27	0.01\\
75.28	0.01\\
75.29	0.01\\
75.3	0.01\\
75.31	0.01\\
75.32	0.01\\
75.33	0.01\\
75.34	0.01\\
75.35	0.01\\
75.36	0.01\\
75.37	0.01\\
75.38	0.01\\
75.39	0.01\\
75.4	0.01\\
75.41	0.01\\
75.42	0.01\\
75.43	0.01\\
75.44	0.01\\
75.45	0.01\\
75.46	0.01\\
75.47	0.01\\
75.48	0.01\\
75.49	0.01\\
75.5	0.01\\
75.51	0.01\\
75.52	0.01\\
75.53	0.01\\
75.54	0.01\\
75.55	0.01\\
75.56	0.01\\
75.57	0.01\\
75.58	0.01\\
75.59	0.01\\
75.6	0.01\\
75.61	0.01\\
75.62	0.01\\
75.63	0.01\\
75.64	0.01\\
75.65	0.01\\
75.66	0.01\\
75.67	0.01\\
75.68	0.01\\
75.69	0.01\\
75.7	0.01\\
75.71	0.01\\
75.72	0.01\\
75.73	0.01\\
75.74	0.01\\
75.75	0.01\\
75.76	0.01\\
75.77	0.01\\
75.78	0.01\\
75.79	0.01\\
75.8	0.01\\
75.81	0.01\\
75.82	0.01\\
75.83	0.01\\
75.84	0.01\\
75.85	0.01\\
75.86	0.01\\
75.87	0.01\\
75.88	0.01\\
75.89	0.01\\
75.9	0.01\\
75.91	0.01\\
75.92	0.01\\
75.93	0.01\\
75.94	0.01\\
75.95	0.01\\
75.96	0.01\\
75.97	0.01\\
75.98	0.01\\
75.99	0.01\\
76	0.01\\
76.01	0.01\\
76.02	0.01\\
76.03	0.01\\
76.04	0.01\\
76.05	0.01\\
76.06	0.01\\
76.07	0.01\\
76.08	0.01\\
76.09	0.01\\
76.1	0.01\\
76.11	0.01\\
76.12	0.01\\
76.13	0.01\\
76.14	0.01\\
76.15	0.01\\
76.16	0.01\\
76.17	0.01\\
76.18	0.01\\
76.19	0.01\\
76.2	0.01\\
76.21	0.01\\
76.22	0.01\\
76.23	0.01\\
76.24	0.01\\
76.25	0.01\\
76.26	0.01\\
76.27	0.01\\
76.28	0.01\\
76.29	0.01\\
76.3	0.01\\
76.31	0.01\\
76.32	0.01\\
76.33	0.01\\
76.34	0.01\\
76.35	0.01\\
76.36	0.01\\
76.37	0.01\\
76.38	0.01\\
76.39	0.01\\
76.4	0.01\\
76.41	0.01\\
76.42	0.01\\
76.43	0.01\\
76.44	0.01\\
76.45	0.01\\
76.46	0.01\\
76.47	0.01\\
76.48	0.01\\
76.49	0.01\\
76.5	0.01\\
76.51	0.01\\
76.52	0.01\\
76.53	0.01\\
76.54	0.01\\
76.55	0.01\\
76.56	0.01\\
76.57	0.01\\
76.58	0.01\\
76.59	0.01\\
76.6	0.01\\
76.61	0.01\\
76.62	0.01\\
76.63	0.01\\
76.64	0.01\\
76.65	0.01\\
76.66	0.01\\
76.67	0.01\\
76.68	0.01\\
76.69	0.01\\
76.7	0.01\\
76.71	0.01\\
76.72	0.01\\
76.73	0.01\\
76.74	0.01\\
76.75	0.01\\
76.76	0.01\\
76.77	0.01\\
76.78	0.01\\
76.79	0.01\\
76.8	0.01\\
76.81	0.01\\
76.82	0.01\\
76.83	0.01\\
76.84	0.01\\
76.85	0.01\\
76.86	0.01\\
76.87	0.01\\
76.88	0.01\\
76.89	0.01\\
76.9	0.01\\
76.91	0.01\\
76.92	0.01\\
76.93	0.01\\
76.94	0.01\\
76.95	0.01\\
76.96	0.01\\
76.97	0.01\\
76.98	0.01\\
76.99	0.01\\
77	0.01\\
77.01	0.01\\
77.02	0.01\\
77.03	0.01\\
77.04	0.01\\
77.05	0.01\\
77.06	0.01\\
77.07	0.01\\
77.08	0.01\\
77.09	0.01\\
77.1	0.01\\
77.11	0.01\\
77.12	0.01\\
77.13	0.01\\
77.14	0.01\\
77.15	0.01\\
77.16	0.01\\
77.17	0.01\\
77.18	0.01\\
77.19	0.01\\
77.2	0.01\\
77.21	0.01\\
77.22	0.01\\
77.23	0.01\\
77.24	0.01\\
77.25	0.01\\
77.26	0.01\\
77.27	0.01\\
77.28	0.01\\
77.29	0.01\\
77.3	0.01\\
77.31	0.01\\
77.32	0.01\\
77.33	0.01\\
77.34	0.01\\
77.35	0.01\\
77.36	0.01\\
77.37	0.01\\
77.38	0.01\\
77.39	0.01\\
77.4	0.01\\
77.41	0.01\\
77.42	0.01\\
77.43	0.01\\
77.44	0.01\\
77.45	0.01\\
77.46	0.01\\
77.47	0.01\\
77.48	0.01\\
77.49	0.01\\
77.5	0.01\\
77.51	0.01\\
77.52	0.01\\
77.53	0.01\\
77.54	0.01\\
77.55	0.01\\
77.56	0.01\\
77.57	0.01\\
77.58	0.01\\
77.59	0.01\\
77.6	0.01\\
77.61	0.01\\
77.62	0.01\\
77.63	0.01\\
77.64	0.01\\
77.65	0.01\\
77.66	0.01\\
77.67	0.01\\
77.68	0.01\\
77.69	0.01\\
77.7	0.01\\
77.71	0.01\\
77.72	0.01\\
77.73	0.01\\
77.74	0.01\\
77.75	0.01\\
77.76	0.01\\
77.77	0.01\\
77.78	0.01\\
77.79	0.01\\
77.8	0.01\\
77.81	0.01\\
77.82	0.01\\
77.83	0.01\\
77.84	0.01\\
77.85	0.01\\
77.86	0.01\\
77.87	0.01\\
77.88	0.01\\
77.89	0.01\\
77.9	0.01\\
77.91	0.01\\
77.92	0.01\\
77.93	0.01\\
77.94	0.01\\
77.95	0.01\\
77.96	0.01\\
77.97	0.01\\
77.98	0.01\\
77.99	0.01\\
78	0.01\\
78.01	0.01\\
78.02	0.01\\
78.03	0.01\\
78.04	0.01\\
78.05	0.01\\
78.06	0.01\\
78.07	0.01\\
78.08	0.01\\
78.09	0.01\\
78.1	0.01\\
78.11	0.01\\
78.12	0.01\\
78.13	0.01\\
78.14	0.01\\
78.15	0.01\\
78.16	0.01\\
78.17	0.01\\
78.18	0.01\\
78.19	0.01\\
78.2	0.01\\
78.21	0.01\\
78.22	0.01\\
78.23	0.01\\
78.24	0.01\\
78.25	0.01\\
78.26	0.01\\
78.27	0.01\\
78.28	0.01\\
78.29	0.01\\
78.3	0.01\\
78.31	0.01\\
78.32	0.01\\
78.33	0.01\\
78.34	0.01\\
78.35	0.01\\
78.36	0.01\\
78.37	0.01\\
78.38	0.01\\
78.39	0.01\\
78.4	0.01\\
78.41	0.01\\
78.42	0.01\\
78.43	0.01\\
78.44	0.01\\
78.45	0.01\\
78.46	0.01\\
78.47	0.01\\
78.48	0.01\\
78.49	0.01\\
78.5	0.01\\
78.51	0.01\\
78.52	0.01\\
78.53	0.01\\
78.54	0.01\\
78.55	0.01\\
78.56	0.01\\
78.57	0.01\\
78.58	0.01\\
78.59	0.01\\
78.6	0.01\\
78.61	0.01\\
78.62	0.01\\
78.63	0.01\\
78.64	0.01\\
78.65	0.01\\
78.66	0.01\\
78.67	0.01\\
78.68	0.01\\
78.69	0.01\\
78.7	0.01\\
78.71	0.01\\
78.72	0.01\\
78.73	0.01\\
78.74	0.01\\
78.75	0.01\\
78.76	0.01\\
78.77	0.01\\
78.78	0.01\\
78.79	0.01\\
78.8	0.01\\
78.81	0.01\\
78.82	0.01\\
78.83	0.01\\
78.84	0.01\\
78.85	0.01\\
78.86	0.01\\
78.87	0.01\\
78.88	0.01\\
78.89	0.01\\
78.9	0.01\\
78.91	0.01\\
78.92	0.01\\
78.93	0.01\\
78.94	0.01\\
78.95	0.01\\
78.96	0.01\\
78.97	0.01\\
78.98	0.01\\
78.99	0.01\\
79	0.01\\
79.01	0.01\\
79.02	0.01\\
79.03	0.01\\
79.04	0.01\\
79.05	0.01\\
79.06	0.01\\
79.07	0.01\\
79.08	0.01\\
79.09	0.01\\
79.1	0.01\\
79.11	0.01\\
79.12	0.01\\
79.13	0.01\\
79.14	0.01\\
79.15	0.01\\
79.16	0.01\\
79.17	0.01\\
79.18	0.01\\
79.19	0.01\\
79.2	0.01\\
79.21	0.01\\
79.22	0.01\\
79.23	0.01\\
79.24	0.01\\
79.25	0.01\\
79.26	0.01\\
79.27	0.01\\
79.28	0.01\\
79.29	0.01\\
79.3	0.01\\
79.31	0.01\\
79.32	0.01\\
79.33	0.01\\
79.34	0.01\\
79.35	0.01\\
79.36	0.01\\
79.37	0.01\\
79.38	0.01\\
79.39	0.01\\
79.4	0.01\\
79.41	0.01\\
79.42	0.01\\
79.43	0.01\\
79.44	0.01\\
79.45	0.01\\
79.46	0.01\\
79.47	0.01\\
79.48	0.01\\
79.49	0.01\\
79.5	0.01\\
79.51	0.01\\
79.52	0.01\\
79.53	0.01\\
79.54	0.01\\
79.55	0.01\\
79.56	0.01\\
79.57	0.01\\
79.58	0.01\\
79.59	0.01\\
79.6	0.01\\
79.61	0.01\\
79.62	0.01\\
79.63	0.01\\
79.64	0.01\\
79.65	0.01\\
79.66	0.01\\
79.67	0.01\\
79.68	0.01\\
79.69	0.01\\
79.7	0.01\\
79.71	0.01\\
79.72	0.01\\
79.73	0.01\\
79.74	0.01\\
79.75	0.01\\
79.76	0.01\\
79.77	0.01\\
79.78	0.01\\
79.79	0.01\\
79.8	0.01\\
79.81	0.01\\
79.82	0.01\\
79.83	0.01\\
79.84	0.01\\
79.85	0.01\\
79.86	0.01\\
79.87	0.01\\
79.88	0.01\\
79.89	0.01\\
79.9	0.01\\
79.91	0.01\\
79.92	0.01\\
79.93	0.01\\
79.94	0.01\\
79.95	0.01\\
79.96	0.01\\
79.97	0.01\\
79.98	0.01\\
79.99	0.01\\
80	0.01\\
80.01	0.01\\
};
\addplot [color=blue,solid]
  table[row sep=crcr]{%
80.01	0.01\\
80.02	0.01\\
80.03	0.01\\
80.04	0.01\\
80.05	0.01\\
80.06	0.01\\
80.07	0.01\\
80.08	0.01\\
80.09	0.01\\
80.1	0.01\\
80.11	0.01\\
80.12	0.01\\
80.13	0.01\\
80.14	0.01\\
80.15	0.01\\
80.16	0.01\\
80.17	0.01\\
80.18	0.01\\
80.19	0.01\\
80.2	0.01\\
80.21	0.01\\
80.22	0.01\\
80.23	0.01\\
80.24	0.01\\
80.25	0.01\\
80.26	0.01\\
80.27	0.01\\
80.28	0.01\\
80.29	0.01\\
80.3	0.01\\
80.31	0.01\\
80.32	0.01\\
80.33	0.01\\
80.34	0.01\\
80.35	0.01\\
80.36	0.01\\
80.37	0.01\\
80.38	0.01\\
80.39	0.01\\
80.4	0.01\\
80.41	0.01\\
80.42	0.01\\
80.43	0.01\\
80.44	0.01\\
80.45	0.01\\
80.46	0.01\\
80.47	0.01\\
80.48	0.01\\
80.49	0.01\\
80.5	0.01\\
80.51	0.01\\
80.52	0.01\\
80.53	0.01\\
80.54	0.01\\
80.55	0.01\\
80.56	0.01\\
80.57	0.01\\
80.58	0.01\\
80.59	0.01\\
80.6	0.01\\
80.61	0.01\\
80.62	0.01\\
80.63	0.01\\
80.64	0.01\\
80.65	0.01\\
80.66	0.01\\
80.67	0.01\\
80.68	0.01\\
80.69	0.01\\
80.7	0.01\\
80.71	0.01\\
80.72	0.01\\
80.73	0.01\\
80.74	0.01\\
80.75	0.01\\
80.76	0.01\\
80.77	0.01\\
80.78	0.01\\
80.79	0.01\\
80.8	0.01\\
80.81	0.01\\
80.82	0.01\\
80.83	0.01\\
80.84	0.01\\
80.85	0.01\\
80.86	0.01\\
80.87	0.01\\
80.88	0.01\\
80.89	0.01\\
80.9	0.01\\
80.91	0.01\\
80.92	0.01\\
80.93	0.01\\
80.94	0.01\\
80.95	0.01\\
80.96	0.01\\
80.97	0.01\\
80.98	0.01\\
80.99	0.01\\
81	0.01\\
81.01	0.01\\
81.02	0.01\\
81.03	0.01\\
81.04	0.01\\
81.05	0.01\\
81.06	0.01\\
81.07	0.01\\
81.08	0.01\\
81.09	0.01\\
81.1	0.01\\
81.11	0.01\\
81.12	0.01\\
81.13	0.01\\
81.14	0.01\\
81.15	0.01\\
81.16	0.01\\
81.17	0.01\\
81.18	0.01\\
81.19	0.01\\
81.2	0.01\\
81.21	0.01\\
81.22	0.01\\
81.23	0.01\\
81.24	0.01\\
81.25	0.01\\
81.26	0.01\\
81.27	0.01\\
81.28	0.01\\
81.29	0.01\\
81.3	0.01\\
81.31	0.01\\
81.32	0.01\\
81.33	0.01\\
81.34	0.01\\
81.35	0.01\\
81.36	0.01\\
81.37	0.01\\
81.38	0.01\\
81.39	0.01\\
81.4	0.01\\
81.41	0.01\\
81.42	0.01\\
81.43	0.01\\
81.44	0.01\\
81.45	0.01\\
81.46	0.01\\
81.47	0.01\\
81.48	0.01\\
81.49	0.01\\
81.5	0.01\\
81.51	0.01\\
81.52	0.01\\
81.53	0.01\\
81.54	0.01\\
81.55	0.01\\
81.56	0.01\\
81.57	0.01\\
81.58	0.01\\
81.59	0.01\\
81.6	0.01\\
81.61	0.01\\
81.62	0.01\\
81.63	0.01\\
81.64	0.01\\
81.65	0.01\\
81.66	0.01\\
81.67	0.01\\
81.68	0.01\\
81.69	0.01\\
81.7	0.01\\
81.71	0.01\\
81.72	0.01\\
81.73	0.01\\
81.74	0.01\\
81.75	0.01\\
81.76	0.01\\
81.77	0.01\\
81.78	0.01\\
81.79	0.01\\
81.8	0.01\\
81.81	0.01\\
81.82	0.01\\
81.83	0.01\\
81.84	0.01\\
81.85	0.01\\
81.86	0.01\\
81.87	0.01\\
81.88	0.01\\
81.89	0.01\\
81.9	0.01\\
81.91	0.01\\
81.92	0.01\\
81.93	0.01\\
81.94	0.01\\
81.95	0.01\\
81.96	0.01\\
81.97	0.01\\
81.98	0.01\\
81.99	0.01\\
82	0.01\\
82.01	0.01\\
82.02	0.01\\
82.03	0.01\\
82.04	0.01\\
82.05	0.01\\
82.06	0.01\\
82.07	0.01\\
82.08	0.01\\
82.09	0.01\\
82.1	0.01\\
82.11	0.01\\
82.12	0.01\\
82.13	0.01\\
82.14	0.01\\
82.15	0.01\\
82.16	0.01\\
82.17	0.01\\
82.18	0.01\\
82.19	0.01\\
82.2	0.01\\
82.21	0.01\\
82.22	0.01\\
82.23	0.01\\
82.24	0.01\\
82.25	0.01\\
82.26	0.01\\
82.27	0.01\\
82.28	0.01\\
82.29	0.01\\
82.3	0.01\\
82.31	0.01\\
82.32	0.01\\
82.33	0.01\\
82.34	0.01\\
82.35	0.01\\
82.36	0.01\\
82.37	0.01\\
82.38	0.01\\
82.39	0.01\\
82.4	0.01\\
82.41	0.01\\
82.42	0.01\\
82.43	0.01\\
82.44	0.01\\
82.45	0.01\\
82.46	0.01\\
82.47	0.01\\
82.48	0.01\\
82.49	0.01\\
82.5	0.01\\
82.51	0.01\\
82.52	0.01\\
82.53	0.01\\
82.54	0.01\\
82.55	0.01\\
82.56	0.01\\
82.57	0.01\\
82.58	0.01\\
82.59	0.01\\
82.6	0.01\\
82.61	0.01\\
82.62	0.01\\
82.63	0.01\\
82.64	0.01\\
82.65	0.01\\
82.66	0.01\\
82.67	0.01\\
82.68	0.01\\
82.69	0.01\\
82.7	0.01\\
82.71	0.01\\
82.72	0.01\\
82.73	0.01\\
82.74	0.01\\
82.75	0.01\\
82.76	0.01\\
82.77	0.01\\
82.78	0.01\\
82.79	0.01\\
82.8	0.01\\
82.81	0.01\\
82.82	0.01\\
82.83	0.01\\
82.84	0.01\\
82.85	0.01\\
82.86	0.01\\
82.87	0.01\\
82.88	0.01\\
82.89	0.01\\
82.9	0.01\\
82.91	0.01\\
82.92	0.01\\
82.93	0.01\\
82.94	0.01\\
82.95	0.01\\
82.96	0.01\\
82.97	0.01\\
82.98	0.01\\
82.99	0.01\\
83	0.01\\
83.01	0.01\\
83.02	0.01\\
83.03	0.01\\
83.04	0.01\\
83.05	0.01\\
83.06	0.01\\
83.07	0.01\\
83.08	0.01\\
83.09	0.01\\
83.1	0.01\\
83.11	0.01\\
83.12	0.01\\
83.13	0.01\\
83.14	0.01\\
83.15	0.01\\
83.16	0.01\\
83.17	0.01\\
83.18	0.01\\
83.19	0.01\\
83.2	0.01\\
83.21	0.01\\
83.22	0.01\\
83.23	0.01\\
83.24	0.01\\
83.25	0.01\\
83.26	0.01\\
83.27	0.01\\
83.28	0.01\\
83.29	0.01\\
83.3	0.01\\
83.31	0.01\\
83.32	0.01\\
83.33	0.01\\
83.34	0.01\\
83.35	0.01\\
83.36	0.01\\
83.37	0.01\\
83.38	0.01\\
83.39	0.01\\
83.4	0.01\\
83.41	0.01\\
83.42	0.01\\
83.43	0.01\\
83.44	0.01\\
83.45	0.01\\
83.46	0.01\\
83.47	0.01\\
83.48	0.01\\
83.49	0.01\\
83.5	0.01\\
83.51	0.01\\
83.52	0.01\\
83.53	0.01\\
83.54	0.01\\
83.55	0.01\\
83.56	0.01\\
83.57	0.01\\
83.58	0.01\\
83.59	0.01\\
83.6	0.01\\
83.61	0.01\\
83.62	0.01\\
83.63	0.01\\
83.64	0.01\\
83.65	0.01\\
83.66	0.01\\
83.67	0.01\\
83.68	0.01\\
83.69	0.01\\
83.7	0.01\\
83.71	0.01\\
83.72	0.01\\
83.73	0.01\\
83.74	0.01\\
83.75	0.01\\
83.76	0.01\\
83.77	0.01\\
83.78	0.01\\
83.79	0.01\\
83.8	0.01\\
83.81	0.01\\
83.82	0.01\\
83.83	0.01\\
83.84	0.01\\
83.85	0.01\\
83.86	0.01\\
83.87	0.01\\
83.88	0.01\\
83.89	0.01\\
83.9	0.01\\
83.91	0.01\\
83.92	0.01\\
83.93	0.01\\
83.94	0.01\\
83.95	0.01\\
83.96	0.01\\
83.97	0.01\\
83.98	0.01\\
83.99	0.01\\
84	0.01\\
84.01	0.01\\
84.02	0.01\\
84.03	0.01\\
84.04	0.01\\
84.05	0.01\\
84.06	0.01\\
84.07	0.01\\
84.08	0.01\\
84.09	0.01\\
84.1	0.01\\
84.11	0.01\\
84.12	0.01\\
84.13	0.01\\
84.14	0.01\\
84.15	0.01\\
84.16	0.01\\
84.17	0.01\\
84.18	0.01\\
84.19	0.01\\
84.2	0.01\\
84.21	0.01\\
84.22	0.01\\
84.23	0.01\\
84.24	0.01\\
84.25	0.01\\
84.26	0.01\\
84.27	0.01\\
84.28	0.01\\
84.29	0.01\\
84.3	0.01\\
84.31	0.01\\
84.32	0.01\\
84.33	0.01\\
84.34	0.01\\
84.35	0.01\\
84.36	0.01\\
84.37	0.01\\
84.38	0.01\\
84.39	0.01\\
84.4	0.01\\
84.41	0.01\\
84.42	0.01\\
84.43	0.01\\
84.44	0.01\\
84.45	0.01\\
84.46	0.01\\
84.47	0.01\\
84.48	0.01\\
84.49	0.01\\
84.5	0.01\\
84.51	0.01\\
84.52	0.01\\
84.53	0.01\\
84.54	0.01\\
84.55	0.01\\
84.56	0.01\\
84.57	0.01\\
84.58	0.01\\
84.59	0.01\\
84.6	0.01\\
84.61	0.01\\
84.62	0.01\\
84.63	0.01\\
84.64	0.01\\
84.65	0.01\\
84.66	0.01\\
84.67	0.01\\
84.68	0.01\\
84.69	0.01\\
84.7	0.01\\
84.71	0.01\\
84.72	0.01\\
84.73	0.01\\
84.74	0.01\\
84.75	0.01\\
84.76	0.01\\
84.77	0.01\\
84.78	0.01\\
84.79	0.01\\
84.8	0.01\\
84.81	0.01\\
84.82	0.01\\
84.83	0.01\\
84.84	0.01\\
84.85	0.01\\
84.86	0.01\\
84.87	0.01\\
84.88	0.01\\
84.89	0.01\\
84.9	0.01\\
84.91	0.01\\
84.92	0.01\\
84.93	0.01\\
84.94	0.01\\
84.95	0.01\\
84.96	0.01\\
84.97	0.01\\
84.98	0.01\\
84.99	0.01\\
85	0.01\\
85.01	0.01\\
85.02	0.01\\
85.03	0.01\\
85.04	0.01\\
85.05	0.01\\
85.06	0.01\\
85.07	0.01\\
85.08	0.01\\
85.09	0.01\\
85.1	0.01\\
85.11	0.01\\
85.12	0.01\\
85.13	0.01\\
85.14	0.01\\
85.15	0.01\\
85.16	0.01\\
85.17	0.01\\
85.18	0.01\\
85.19	0.01\\
85.2	0.01\\
85.21	0.01\\
85.22	0.01\\
85.23	0.01\\
85.24	0.01\\
85.25	0.01\\
85.26	0.01\\
85.27	0.01\\
85.28	0.01\\
85.29	0.01\\
85.3	0.01\\
85.31	0.01\\
85.32	0.01\\
85.33	0.01\\
85.34	0.01\\
85.35	0.01\\
85.36	0.01\\
85.37	0.01\\
85.38	0.01\\
85.39	0.01\\
85.4	0.01\\
85.41	0.01\\
85.42	0.01\\
85.43	0.01\\
85.44	0.01\\
85.45	0.01\\
85.46	0.01\\
85.47	0.01\\
85.48	0.01\\
85.49	0.01\\
85.5	0.01\\
85.51	0.01\\
85.52	0.01\\
85.53	0.01\\
85.54	0.01\\
85.55	0.01\\
85.56	0.01\\
85.57	0.01\\
85.58	0.01\\
85.59	0.01\\
85.6	0.01\\
85.61	0.01\\
85.62	0.01\\
85.63	0.01\\
85.64	0.01\\
85.65	0.01\\
85.66	0.01\\
85.67	0.01\\
85.68	0.01\\
85.69	0.01\\
85.7	0.01\\
85.71	0.01\\
85.72	0.01\\
85.73	0.01\\
85.74	0.01\\
85.75	0.01\\
85.76	0.01\\
85.77	0.01\\
85.78	0.01\\
85.79	0.01\\
85.8	0.01\\
85.81	0.01\\
85.82	0.01\\
85.83	0.01\\
85.84	0.01\\
85.85	0.01\\
85.86	0.01\\
85.87	0.01\\
85.88	0.01\\
85.89	0.01\\
85.9	0.01\\
85.91	0.01\\
85.92	0.01\\
85.93	0.01\\
85.94	0.01\\
85.95	0.01\\
85.96	0.01\\
85.97	0.01\\
85.98	0.01\\
85.99	0.01\\
86	0.01\\
86.01	0.01\\
86.02	0.01\\
86.03	0.01\\
86.04	0.01\\
86.05	0.01\\
86.06	0.01\\
86.07	0.01\\
86.08	0.01\\
86.09	0.01\\
86.1	0.01\\
86.11	0.01\\
86.12	0.01\\
86.13	0.01\\
86.14	0.01\\
86.15	0.01\\
86.16	0.01\\
86.17	0.01\\
86.18	0.01\\
86.19	0.01\\
86.2	0.01\\
86.21	0.01\\
86.22	0.01\\
86.23	0.01\\
86.24	0.01\\
86.25	0.01\\
86.26	0.01\\
86.27	0.01\\
86.28	0.01\\
86.29	0.01\\
86.3	0.01\\
86.31	0.01\\
86.32	0.01\\
86.33	0.01\\
86.34	0.01\\
86.35	0.01\\
86.36	0.01\\
86.37	0.01\\
86.38	0.01\\
86.39	0.01\\
86.4	0.01\\
86.41	0.01\\
86.42	0.01\\
86.43	0.01\\
86.44	0.01\\
86.45	0.01\\
86.46	0.01\\
86.47	0.01\\
86.48	0.01\\
86.49	0.01\\
86.5	0.01\\
86.51	0.01\\
86.52	0.01\\
86.53	0.01\\
86.54	0.01\\
86.55	0.01\\
86.56	0.01\\
86.57	0.01\\
86.58	0.01\\
86.59	0.01\\
86.6	0.01\\
86.61	0.01\\
86.62	0.01\\
86.63	0.01\\
86.64	0.01\\
86.65	0.01\\
86.66	0.01\\
86.67	0.01\\
86.68	0.01\\
86.69	0.01\\
86.7	0.01\\
86.71	0.01\\
86.72	0.01\\
86.73	0.01\\
86.74	0.01\\
86.75	0.01\\
86.76	0.01\\
86.77	0.01\\
86.78	0.01\\
86.79	0.01\\
86.8	0.01\\
86.81	0.01\\
86.82	0.01\\
86.83	0.01\\
86.84	0.01\\
86.85	0.01\\
86.86	0.01\\
86.87	0.01\\
86.88	0.01\\
86.89	0.01\\
86.9	0.01\\
86.91	0.01\\
86.92	0.01\\
86.93	0.01\\
86.94	0.01\\
86.95	0.01\\
86.96	0.01\\
86.97	0.01\\
86.98	0.01\\
86.99	0.01\\
87	0.01\\
87.01	0.01\\
87.02	0.01\\
87.03	0.01\\
87.04	0.01\\
87.05	0.01\\
87.06	0.01\\
87.07	0.01\\
87.08	0.01\\
87.09	0.01\\
87.1	0.01\\
87.11	0.01\\
87.12	0.01\\
87.13	0.01\\
87.14	0.01\\
87.15	0.01\\
87.16	0.01\\
87.17	0.01\\
87.18	0.01\\
87.19	0.01\\
87.2	0.01\\
87.21	0.01\\
87.22	0.01\\
87.23	0.01\\
87.24	0.01\\
87.25	0.01\\
87.26	0.01\\
87.27	0.01\\
87.28	0.01\\
87.29	0.01\\
87.3	0.01\\
87.31	0.01\\
87.32	0.01\\
87.33	0.01\\
87.34	0.01\\
87.35	0.01\\
87.36	0.01\\
87.37	0.01\\
87.38	0.01\\
87.39	0.01\\
87.4	0.01\\
87.41	0.01\\
87.42	0.01\\
87.43	0.01\\
87.44	0.01\\
87.45	0.01\\
87.46	0.01\\
87.47	0.01\\
87.48	0.01\\
87.49	0.01\\
87.5	0.01\\
87.51	0.01\\
87.52	0.01\\
87.53	0.01\\
87.54	0.01\\
87.55	0.01\\
87.56	0.01\\
87.57	0.01\\
87.58	0.01\\
87.59	0.01\\
87.6	0.01\\
87.61	0.01\\
87.62	0.01\\
87.63	0.01\\
87.64	0.01\\
87.65	0.01\\
87.66	0.01\\
87.67	0.01\\
87.68	0.01\\
87.69	0.01\\
87.7	0.01\\
87.71	0.01\\
87.72	0.01\\
87.73	0.01\\
87.74	0.01\\
87.75	0.01\\
87.76	0.01\\
87.77	0.01\\
87.78	0.01\\
87.79	0.01\\
87.8	0.01\\
87.81	0.01\\
87.82	0.01\\
87.83	0.01\\
87.84	0.01\\
87.85	0.01\\
87.86	0.01\\
87.87	0.01\\
87.88	0.01\\
87.89	0.01\\
87.9	0.01\\
87.91	0.01\\
87.92	0.01\\
87.93	0.01\\
87.94	0.01\\
87.95	0.01\\
87.96	0.01\\
87.97	0.01\\
87.98	0.01\\
87.99	0.01\\
88	0.01\\
88.01	0.01\\
88.02	0.01\\
88.03	0.01\\
88.04	0.01\\
88.05	0.01\\
88.06	0.01\\
88.07	0.01\\
88.08	0.01\\
88.09	0.01\\
88.1	0.01\\
88.11	0.01\\
88.12	0.01\\
88.13	0.01\\
88.14	0.01\\
88.15	0.01\\
88.16	0.01\\
88.17	0.01\\
88.18	0.01\\
88.19	0.01\\
88.2	0.01\\
88.21	0.01\\
88.22	0.01\\
88.23	0.01\\
88.24	0.01\\
88.25	0.01\\
88.26	0.01\\
88.27	0.01\\
88.28	0.01\\
88.29	0.01\\
88.3	0.01\\
88.31	0.01\\
88.32	0.01\\
88.33	0.01\\
88.34	0.01\\
88.35	0.01\\
88.36	0.01\\
88.37	0.01\\
88.38	0.01\\
88.39	0.01\\
88.4	0.01\\
88.41	0.01\\
88.42	0.01\\
88.43	0.01\\
88.44	0.01\\
88.45	0.01\\
88.46	0.01\\
88.47	0.01\\
88.48	0.01\\
88.49	0.01\\
88.5	0.01\\
88.51	0.01\\
88.52	0.01\\
88.53	0.01\\
88.54	0.01\\
88.55	0.01\\
88.56	0.01\\
88.57	0.01\\
88.58	0.01\\
88.59	0.01\\
88.6	0.01\\
88.61	0.01\\
88.62	0.01\\
88.63	0.01\\
88.64	0.01\\
88.65	0.01\\
88.66	0.01\\
88.67	0.01\\
88.68	0.01\\
88.69	0.01\\
88.7	0.01\\
88.71	0.01\\
88.72	0.01\\
88.73	0.01\\
88.74	0.01\\
88.75	0.01\\
88.76	0.01\\
88.77	0.01\\
88.78	0.01\\
88.79	0.01\\
88.8	0.01\\
88.81	0.01\\
88.82	0.01\\
88.83	0.01\\
88.84	0.01\\
88.85	0.01\\
88.86	0.01\\
88.87	0.01\\
88.88	0.01\\
88.89	0.01\\
88.9	0.01\\
88.91	0.01\\
88.92	0.01\\
88.93	0.01\\
88.94	0.01\\
88.95	0.01\\
88.96	0.01\\
88.97	0.01\\
88.98	0.01\\
88.99	0.01\\
89	0.01\\
89.01	0.01\\
89.02	0.01\\
89.03	0.01\\
89.04	0.01\\
89.05	0.01\\
89.06	0.01\\
89.07	0.01\\
89.08	0.01\\
89.09	0.01\\
89.1	0.01\\
89.11	0.01\\
89.12	0.01\\
89.13	0.01\\
89.14	0.01\\
89.15	0.01\\
89.16	0.01\\
89.17	0.01\\
89.18	0.01\\
89.19	0.01\\
89.2	0.01\\
89.21	0.01\\
89.22	0.01\\
89.23	0.01\\
89.24	0.01\\
89.25	0.01\\
89.26	0.01\\
89.27	0.01\\
89.28	0.01\\
89.29	0.01\\
89.3	0.01\\
89.31	0.01\\
89.32	0.01\\
89.33	0.01\\
89.34	0.01\\
89.35	0.01\\
89.36	0.01\\
89.37	0.01\\
89.38	0.01\\
89.39	0.01\\
89.4	0.01\\
89.41	0.01\\
89.42	0.01\\
89.43	0.01\\
89.44	0.01\\
89.45	0.01\\
89.46	0.01\\
89.47	0.01\\
89.48	0.01\\
89.49	0.01\\
89.5	0.01\\
89.51	0.01\\
89.52	0.01\\
89.53	0.01\\
89.54	0.01\\
89.55	0.01\\
89.56	0.01\\
89.57	0.01\\
89.58	0.01\\
89.59	0.01\\
89.6	0.01\\
89.61	0.01\\
89.62	0.01\\
89.63	0.01\\
89.64	0.01\\
89.65	0.01\\
89.66	0.01\\
89.67	0.01\\
89.68	0.01\\
89.69	0.01\\
89.7	0.01\\
89.71	0.01\\
89.72	0.01\\
89.73	0.01\\
89.74	0.01\\
89.75	0.01\\
89.76	0.01\\
89.77	0.01\\
89.78	0.01\\
89.79	0.01\\
89.8	0.01\\
89.81	0.01\\
89.82	0.01\\
89.83	0.01\\
89.84	0.01\\
89.85	0.01\\
89.86	0.01\\
89.87	0.01\\
89.88	0.01\\
89.89	0.01\\
89.9	0.01\\
89.91	0.01\\
89.92	0.01\\
89.93	0.01\\
89.94	0.01\\
89.95	0.01\\
89.96	0.01\\
89.97	0.01\\
89.98	0.01\\
89.99	0.01\\
90	0.01\\
90.01	0.01\\
90.02	0.01\\
90.03	0.01\\
90.04	0.01\\
90.05	0.01\\
90.06	0.01\\
90.07	0.01\\
90.08	0.01\\
90.09	0.01\\
90.1	0.01\\
90.11	0.01\\
90.12	0.01\\
90.13	0.01\\
90.14	0.01\\
90.15	0.01\\
90.16	0.01\\
90.17	0.01\\
90.18	0.01\\
90.19	0.01\\
90.2	0.01\\
90.21	0.01\\
90.22	0.01\\
90.23	0.01\\
90.24	0.01\\
90.25	0.01\\
90.26	0.01\\
90.27	0.01\\
90.28	0.01\\
90.29	0.01\\
90.3	0.01\\
90.31	0.01\\
90.32	0.01\\
90.33	0.01\\
90.34	0.01\\
90.35	0.01\\
90.36	0.01\\
90.37	0.01\\
90.38	0.01\\
90.39	0.01\\
90.4	0.01\\
90.41	0.01\\
90.42	0.01\\
90.43	0.01\\
90.44	0.01\\
90.45	0.01\\
90.46	0.01\\
90.47	0.01\\
90.48	0.01\\
90.49	0.01\\
90.5	0.01\\
90.51	0.01\\
90.52	0.01\\
90.53	0.01\\
90.54	0.01\\
90.55	0.01\\
90.56	0.01\\
90.57	0.01\\
90.58	0.01\\
90.59	0.01\\
90.6	0.01\\
90.61	0.01\\
90.62	0.01\\
90.63	0.01\\
90.64	0.01\\
90.65	0.01\\
90.66	0.01\\
90.67	0.01\\
90.68	0.01\\
90.69	0.01\\
90.7	0.01\\
90.71	0.01\\
90.72	0.01\\
90.73	0.01\\
90.74	0.01\\
90.75	0.01\\
90.76	0.01\\
90.77	0.01\\
90.78	0.01\\
90.79	0.01\\
90.8	0.01\\
90.81	0.01\\
90.82	0.01\\
90.83	0.01\\
90.84	0.01\\
90.85	0.01\\
90.86	0.01\\
90.87	0.01\\
90.88	0.01\\
90.89	0.01\\
90.9	0.01\\
90.91	0.01\\
90.92	0.01\\
90.93	0.01\\
90.94	0.01\\
90.95	0.01\\
90.96	0.01\\
90.97	0.01\\
90.98	0.01\\
90.99	0.01\\
91	0.01\\
91.01	0.01\\
91.02	0.01\\
91.03	0.01\\
91.04	0.01\\
91.05	0.01\\
91.06	0.01\\
91.07	0.01\\
91.08	0.01\\
91.09	0.01\\
91.1	0.01\\
91.11	0.01\\
91.12	0.01\\
91.13	0.01\\
91.14	0.01\\
91.15	0.01\\
91.16	0.01\\
91.17	0.01\\
91.18	0.01\\
91.19	0.01\\
91.2	0.01\\
91.21	0.01\\
91.22	0.01\\
91.23	0.01\\
91.24	0.01\\
91.25	0.01\\
91.26	0.01\\
91.27	0.01\\
91.28	0.01\\
91.29	0.01\\
91.3	0.01\\
91.31	0.01\\
91.32	0.01\\
91.33	0.01\\
91.34	0.01\\
91.35	0.01\\
91.36	0.01\\
91.37	0.01\\
91.38	0.01\\
91.39	0.01\\
91.4	0.01\\
91.41	0.01\\
91.42	0.01\\
91.43	0.01\\
91.44	0.01\\
91.45	0.01\\
91.46	0.01\\
91.47	0.01\\
91.48	0.01\\
91.49	0.01\\
91.5	0.01\\
91.51	0.01\\
91.52	0.01\\
91.53	0.01\\
91.54	0.01\\
91.55	0.01\\
91.56	0.01\\
91.57	0.01\\
91.58	0.01\\
91.59	0.01\\
91.6	0.01\\
91.61	0.01\\
91.62	0.01\\
91.63	0.01\\
91.64	0.01\\
91.65	0.01\\
91.66	0.01\\
91.67	0.01\\
91.68	0.01\\
91.69	0.01\\
91.7	0.01\\
91.71	0.01\\
91.72	0.01\\
91.73	0.01\\
91.74	0.01\\
91.75	0.01\\
91.76	0.01\\
91.77	0.01\\
91.78	0.01\\
91.79	0.01\\
91.8	0.01\\
91.81	0.01\\
91.82	0.01\\
91.83	0.01\\
91.84	0.01\\
91.85	0.01\\
91.86	0.01\\
91.87	0.01\\
91.88	0.01\\
91.89	0.01\\
91.9	0.01\\
91.91	0.01\\
91.92	0.01\\
91.93	0.01\\
91.94	0.01\\
91.95	0.01\\
91.96	0.01\\
91.97	0.01\\
91.98	0.01\\
91.99	0.01\\
92	0.01\\
92.01	0.01\\
92.02	0.01\\
92.03	0.01\\
92.04	0.01\\
92.05	0.01\\
92.06	0.01\\
92.07	0.01\\
92.08	0.01\\
92.09	0.01\\
92.1	0.01\\
92.11	0.01\\
92.12	0.01\\
92.13	0.01\\
92.14	0.01\\
92.15	0.01\\
92.16	0.01\\
92.17	0.01\\
92.18	0.01\\
92.19	0.01\\
92.2	0.01\\
92.21	0.01\\
92.22	0.01\\
92.23	0.01\\
92.24	0.01\\
92.25	0.01\\
92.26	0.01\\
92.27	0.01\\
92.28	0.01\\
92.29	0.01\\
92.3	0.01\\
92.31	0.01\\
92.32	0.01\\
92.33	0.01\\
92.34	0.01\\
92.35	0.01\\
92.36	0.01\\
92.37	0.01\\
92.38	0.01\\
92.39	0.01\\
92.4	0.01\\
92.41	0.01\\
92.42	0.01\\
92.43	0.01\\
92.44	0.01\\
92.45	0.01\\
92.46	0.01\\
92.47	0.01\\
92.48	0.01\\
92.49	0.01\\
92.5	0.01\\
92.51	0.01\\
92.52	0.01\\
92.53	0.01\\
92.54	0.01\\
92.55	0.01\\
92.56	0.01\\
92.57	0.01\\
92.58	0.01\\
92.59	0.01\\
92.6	0.01\\
92.61	0.01\\
92.62	0.01\\
92.63	0.01\\
92.64	0.01\\
92.65	0.01\\
92.66	0.01\\
92.67	0.01\\
92.68	0.01\\
92.69	0.01\\
92.7	0.01\\
92.71	0.01\\
92.72	0.01\\
92.73	0.01\\
92.74	0.01\\
92.75	0.01\\
92.76	0.01\\
92.77	0.01\\
92.78	0.01\\
92.79	0.01\\
92.8	0.01\\
92.81	0.01\\
92.82	0.01\\
92.83	0.01\\
92.84	0.01\\
92.85	0.01\\
92.86	0.01\\
92.87	0.01\\
92.88	0.01\\
92.89	0.01\\
92.9	0.01\\
92.91	0.01\\
92.92	0.01\\
92.93	0.01\\
92.94	0.01\\
92.95	0.01\\
92.96	0.01\\
92.97	0.01\\
92.98	0.01\\
92.99	0.01\\
93	0.01\\
93.01	0.01\\
93.02	0.01\\
93.03	0.01\\
93.04	0.01\\
93.05	0.01\\
93.06	0.01\\
93.07	0.01\\
93.08	0.01\\
93.09	0.01\\
93.1	0.01\\
93.11	0.01\\
93.12	0.01\\
93.13	0.01\\
93.14	0.01\\
93.15	0.01\\
93.16	0.01\\
93.17	0.01\\
93.18	0.01\\
93.19	0.01\\
93.2	0.01\\
93.21	0.01\\
93.22	0.01\\
93.23	0.01\\
93.24	0.01\\
93.25	0.01\\
93.26	0.01\\
93.27	0.01\\
93.28	0.01\\
93.29	0.01\\
93.3	0.01\\
93.31	0.01\\
93.32	0.01\\
93.33	0.01\\
93.34	0.01\\
93.35	0.01\\
93.36	0.01\\
93.37	0.01\\
93.38	0.01\\
93.39	0.01\\
93.4	0.01\\
93.41	0.01\\
93.42	0.01\\
93.43	0.01\\
93.44	0.01\\
93.45	0.01\\
93.46	0.01\\
93.47	0.01\\
93.48	0.01\\
93.49	0.01\\
93.5	0.01\\
93.51	0.01\\
93.52	0.01\\
93.53	0.01\\
93.54	0.01\\
93.55	0.01\\
93.56	0.01\\
93.57	0.01\\
93.58	0.01\\
93.59	0.01\\
93.6	0.01\\
93.61	0.01\\
93.62	0.01\\
93.63	0.01\\
93.64	0.01\\
93.65	0.01\\
93.66	0.01\\
93.67	0.01\\
93.68	0.01\\
93.69	0.01\\
93.7	0.01\\
93.71	0.01\\
93.72	0.01\\
93.73	0.01\\
93.74	0.01\\
93.75	0.01\\
93.76	0.01\\
93.77	0.01\\
93.78	0.01\\
93.79	0.01\\
93.8	0.01\\
93.81	0.01\\
93.82	0.01\\
93.83	0.01\\
93.84	0.01\\
93.85	0.01\\
93.86	0.01\\
93.87	0.01\\
93.88	0.01\\
93.89	0.01\\
93.9	0.01\\
93.91	0.01\\
93.92	0.01\\
93.93	0.01\\
93.94	0.01\\
93.95	0.01\\
93.96	0.01\\
93.97	0.01\\
93.98	0.01\\
93.99	0.01\\
94	0.01\\
94.01	0.01\\
94.02	0.01\\
94.03	0.01\\
94.04	0.01\\
94.05	0.01\\
94.06	0.01\\
94.07	0.01\\
94.08	0.01\\
94.09	0.01\\
94.1	0.01\\
94.11	0.01\\
94.12	0.01\\
94.13	0.01\\
94.14	0.01\\
94.15	0.01\\
94.16	0.01\\
94.17	0.01\\
94.18	0.01\\
94.19	0.01\\
94.2	0.01\\
94.21	0.01\\
94.22	0.01\\
94.23	0.01\\
94.24	0.01\\
94.25	0.01\\
94.26	0.01\\
94.27	0.01\\
94.28	0.01\\
94.29	0.01\\
94.3	0.01\\
94.31	0.01\\
94.32	0.01\\
94.33	0.01\\
94.34	0.01\\
94.35	0.01\\
94.36	0.01\\
94.37	0.01\\
94.38	0.01\\
94.39	0.01\\
94.4	0.01\\
94.41	0.01\\
94.42	0.01\\
94.43	0.01\\
94.44	0.01\\
94.45	0.01\\
94.46	0.01\\
94.47	0.01\\
94.48	0.01\\
94.49	0.01\\
94.5	0.01\\
94.51	0.01\\
94.52	0.01\\
94.53	0.01\\
94.54	0.01\\
94.55	0.01\\
94.56	0.01\\
94.57	0.01\\
94.58	0.01\\
94.59	0.01\\
94.6	0.01\\
94.61	0.01\\
94.62	0.01\\
94.63	0.01\\
94.64	0.01\\
94.65	0.01\\
94.66	0.01\\
94.67	0.01\\
94.68	0.01\\
94.69	0.01\\
94.7	0.01\\
94.71	0.01\\
94.72	0.01\\
94.73	0.01\\
94.74	0.01\\
94.75	0.01\\
94.76	0.01\\
94.77	0.01\\
94.78	0.01\\
94.79	0.01\\
94.8	0.01\\
94.81	0.01\\
94.82	0.01\\
94.83	0.01\\
94.84	0.01\\
94.85	0.01\\
94.86	0.01\\
94.87	0.01\\
94.88	0.01\\
94.89	0.01\\
94.9	0.01\\
94.91	0.01\\
94.92	0.01\\
94.93	0.01\\
94.94	0.01\\
94.95	0.01\\
94.96	0.01\\
94.97	0.01\\
94.98	0.01\\
94.99	0.01\\
95	0.01\\
95.01	0.01\\
95.02	0.01\\
95.03	0.01\\
95.04	0.01\\
95.05	0.01\\
95.06	0.01\\
95.07	0.01\\
95.08	0.01\\
95.09	0.01\\
95.1	0.01\\
95.11	0.01\\
95.12	0.01\\
95.13	0.01\\
95.14	0.01\\
95.15	0.01\\
95.16	0.01\\
95.17	0.01\\
95.18	0.01\\
95.19	0.01\\
95.2	0.01\\
95.21	0.01\\
95.22	0.01\\
95.23	0.01\\
95.24	0.01\\
95.25	0.01\\
95.26	0.01\\
95.27	0.01\\
95.28	0.01\\
95.29	0.01\\
95.3	0.01\\
95.31	0.01\\
95.32	0.01\\
95.33	0.01\\
95.34	0.01\\
95.35	0.01\\
95.36	0.01\\
95.37	0.01\\
95.38	0.01\\
95.39	0.01\\
95.4	0.01\\
95.41	0.01\\
95.42	0.01\\
95.43	0.01\\
95.44	0.01\\
95.45	0.01\\
95.46	0.01\\
95.47	0.01\\
95.48	0.01\\
95.49	0.01\\
95.5	0.01\\
95.51	0.01\\
95.52	0.01\\
95.53	0.01\\
95.54	0.01\\
95.55	0.01\\
95.56	0.01\\
95.57	0.01\\
95.58	0.01\\
95.59	0.01\\
95.6	0.01\\
95.61	0.01\\
95.62	0.01\\
95.63	0.01\\
95.64	0.01\\
95.65	0.01\\
95.66	0.01\\
95.67	0.01\\
95.68	0.01\\
95.69	0.01\\
95.7	0.01\\
95.71	0.01\\
95.72	0.01\\
95.73	0.01\\
95.74	0.01\\
95.75	0.01\\
95.76	0.01\\
95.77	0.01\\
95.78	0.01\\
95.79	0.01\\
95.8	0.01\\
95.81	0.01\\
95.82	0.01\\
95.83	0.01\\
95.84	0.01\\
95.85	0.01\\
95.86	0.01\\
95.87	0.01\\
95.88	0.01\\
95.89	0.01\\
95.9	0.01\\
95.91	0.01\\
95.92	0.01\\
95.93	0.01\\
95.94	0.01\\
95.95	0.01\\
95.96	0.01\\
95.97	0.01\\
95.98	0.01\\
95.99	0.01\\
96	0.01\\
96.01	0.01\\
96.02	0.01\\
96.03	0.01\\
96.04	0.01\\
96.05	0.01\\
96.06	0.01\\
96.07	0.01\\
96.08	0.01\\
96.09	0.01\\
96.1	0.01\\
96.11	0.01\\
96.12	0.01\\
96.13	0.01\\
96.14	0.01\\
96.15	0.01\\
96.16	0.01\\
96.17	0.01\\
96.18	0.01\\
96.19	0.01\\
96.2	0.01\\
96.21	0.01\\
96.22	0.01\\
96.23	0.01\\
96.24	0.01\\
96.25	0.01\\
96.26	0.01\\
96.27	0.01\\
96.28	0.01\\
96.29	0.01\\
96.3	0.01\\
96.31	0.01\\
96.32	0.01\\
96.33	0.01\\
96.34	0.01\\
96.35	0.01\\
96.36	0.01\\
96.37	0.01\\
96.38	0.01\\
96.39	0.01\\
96.4	0.01\\
96.41	0.01\\
96.42	0.01\\
96.43	0.01\\
96.44	0.01\\
96.45	0.01\\
96.46	0.01\\
96.47	0.01\\
96.48	0.01\\
96.49	0.01\\
96.5	0.01\\
96.51	0.01\\
96.52	0.01\\
96.53	0.01\\
96.54	0.01\\
96.55	0.01\\
96.56	0.01\\
96.57	0.01\\
96.58	0.01\\
96.59	0.01\\
96.6	0.01\\
96.61	0.01\\
96.62	0.01\\
96.63	0.01\\
96.64	0.01\\
96.65	0.01\\
96.66	0.01\\
96.67	0.01\\
96.68	0.01\\
96.69	0.01\\
96.7	0.01\\
96.71	0.01\\
96.72	0.01\\
96.73	0.01\\
96.74	0.01\\
96.75	0.01\\
96.76	0.01\\
96.77	0.01\\
96.78	0.01\\
96.79	0.01\\
96.8	0.01\\
96.81	0.01\\
96.82	0.01\\
96.83	0.01\\
96.84	0.01\\
96.85	0.01\\
96.86	0.01\\
96.87	0.01\\
96.88	0.01\\
96.89	0.01\\
96.9	0.01\\
96.91	0.01\\
96.92	0.01\\
96.93	0.01\\
96.94	0.01\\
96.95	0.01\\
96.96	0.01\\
96.97	0.01\\
96.98	0.01\\
96.99	0.01\\
97	0.01\\
97.01	0.01\\
97.02	0.01\\
97.03	0.01\\
97.04	0.01\\
97.05	0.01\\
97.06	0.01\\
97.07	0.01\\
97.08	0.01\\
97.09	0.01\\
97.1	0.01\\
97.11	0.01\\
97.12	0.01\\
97.13	0.01\\
97.14	0.01\\
97.15	0.01\\
97.16	0.01\\
97.17	0.01\\
97.18	0.01\\
97.19	0.01\\
97.2	0.01\\
97.21	0.01\\
97.22	0.01\\
97.23	0.01\\
97.24	0.01\\
97.25	0.01\\
97.26	0.01\\
97.27	0.01\\
97.28	0.01\\
97.29	0.01\\
97.3	0.01\\
97.31	0.01\\
97.32	0.01\\
97.33	0.01\\
97.34	0.01\\
97.35	0.01\\
97.36	0.01\\
97.37	0.01\\
97.38	0.01\\
97.39	0.01\\
97.4	0.01\\
97.41	0.01\\
97.42	0.01\\
97.43	0.01\\
97.44	0.01\\
97.45	0.01\\
97.46	0.01\\
97.47	0.01\\
97.48	0.01\\
97.49	0.01\\
97.5	0.01\\
97.51	0.01\\
97.52	0.01\\
97.53	0.01\\
97.54	0.01\\
97.55	0.01\\
97.56	0.01\\
97.57	0.01\\
97.58	0.01\\
97.59	0.01\\
97.6	0.01\\
97.61	0.01\\
97.62	0.01\\
97.63	0.01\\
97.64	0.01\\
97.65	0.01\\
97.66	0.01\\
97.67	0.01\\
97.68	0.01\\
97.69	0.01\\
97.7	0.01\\
97.71	0.01\\
97.72	0.01\\
97.73	0.01\\
97.74	0.01\\
97.75	0.01\\
97.76	0.01\\
97.77	0.01\\
97.78	0.01\\
97.79	0.01\\
97.8	0.01\\
97.81	0.01\\
97.82	0.01\\
97.83	0.01\\
97.84	0.01\\
97.85	0.01\\
97.86	0.01\\
97.87	0.01\\
97.88	0.01\\
97.89	0.01\\
97.9	0.01\\
97.91	0.01\\
97.92	0.01\\
97.93	0.01\\
97.94	0.01\\
97.95	0.01\\
97.96	0.01\\
97.97	0.01\\
97.98	0.01\\
97.99	0.01\\
98	0.01\\
98.01	0.01\\
98.02	0.01\\
98.03	0.01\\
98.04	0.01\\
98.05	0.01\\
98.06	0.01\\
98.07	0.01\\
98.08	0.01\\
98.09	0.01\\
98.1	0.01\\
98.11	0.01\\
98.12	0.01\\
98.13	0.01\\
98.14	0.01\\
98.15	0.01\\
98.16	0.01\\
98.17	0.01\\
98.18	0.01\\
98.19	0.01\\
98.2	0.01\\
98.21	0.01\\
98.22	0.01\\
98.23	0.01\\
98.24	0.01\\
98.25	0.01\\
98.26	0.01\\
98.27	0.01\\
98.28	0.01\\
98.29	0.01\\
98.3	0.01\\
98.31	0.01\\
98.32	0.01\\
98.33	0.01\\
98.34	0.01\\
98.35	0.01\\
98.36	0.01\\
98.37	0.01\\
98.38	0.01\\
98.39	0.01\\
98.4	0.01\\
98.41	0.01\\
98.42	0.01\\
98.43	0.01\\
98.44	0.01\\
98.45	0.01\\
98.46	0.01\\
98.47	0.01\\
98.48	0.01\\
98.49	0.01\\
98.5	0.01\\
98.51	0.01\\
98.52	0.01\\
98.53	0.01\\
98.54	0.01\\
98.55	0.01\\
98.56	0.01\\
98.57	0.01\\
98.58	0.01\\
98.59	0.01\\
98.6	0.01\\
98.61	0.01\\
98.62	0.01\\
98.63	0.01\\
98.64	0.01\\
98.65	0.01\\
98.66	0.01\\
98.67	0.01\\
98.68	0.01\\
98.69	0.01\\
98.7	0.01\\
98.71	0.01\\
98.72	0.01\\
98.73	0.01\\
98.74	0.01\\
98.75	0.01\\
98.76	0.01\\
98.77	0.01\\
98.78	0.01\\
98.79	0.01\\
98.8	0.01\\
98.81	0.01\\
98.82	0.01\\
98.83	0.01\\
98.84	0.01\\
98.85	0.01\\
98.86	0.01\\
98.87	0.01\\
98.88	0.01\\
98.89	0.01\\
98.9	0.01\\
98.91	0.01\\
98.92	0.01\\
98.93	0.01\\
98.94	0.01\\
98.95	0.01\\
98.96	0.01\\
98.97	0.01\\
98.98	0.01\\
98.99	0.01\\
99	0.01\\
99.01	0.01\\
99.02	0.01\\
99.03	0.01\\
99.04	0.01\\
99.05	0.01\\
99.06	0.01\\
99.07	0.01\\
99.08	0.01\\
99.09	0.01\\
99.1	0.01\\
99.11	0.01\\
99.12	0.01\\
99.13	0.01\\
99.14	0.01\\
99.15	0.01\\
99.16	0.01\\
99.17	0.01\\
99.18	0.01\\
99.19	0.01\\
99.2	0.01\\
99.21	0.01\\
99.22	0.01\\
99.23	0.01\\
99.24	0.01\\
99.25	0.01\\
99.26	0.01\\
99.27	0.01\\
99.28	0.01\\
99.29	0.01\\
99.3	0.01\\
99.31	0.01\\
99.32	0.01\\
99.33	0.01\\
99.34	0.01\\
99.35	0.01\\
99.36	0.01\\
99.37	0.01\\
99.38	0.01\\
99.39	0.01\\
99.4	0.01\\
99.41	0.01\\
99.42	0.01\\
99.43	0.01\\
99.44	0.01\\
99.45	0.01\\
99.46	0.01\\
99.47	0.01\\
99.48	0.01\\
99.49	0.01\\
99.5	0.01\\
99.51	0.01\\
99.52	0.01\\
99.53	0.01\\
99.54	0.01\\
99.55	0.01\\
99.56	0.01\\
99.57	0.01\\
99.58	0.01\\
99.59	0.01\\
99.6	0.01\\
99.61	0.01\\
99.62	0.01\\
99.63	0.01\\
99.64	0.01\\
99.65	0.01\\
99.66	0.01\\
99.67	0.01\\
99.68	0.01\\
99.69	0.01\\
99.7	0.01\\
99.71	0.01\\
99.72	0.01\\
99.73	0.01\\
99.74	0.01\\
99.75	0.01\\
99.76	0.01\\
99.77	0.01\\
99.78	0.01\\
99.79	0.01\\
99.8	0.01\\
99.81	0.01\\
99.82	0.01\\
99.83	0.01\\
99.84	0.01\\
99.85	0.01\\
99.86	0.01\\
99.87	0.01\\
99.88	0.01\\
99.89	0.01\\
99.9	0.01\\
99.91	0.01\\
99.92	0.01\\
99.93	0.01\\
99.94	0.01\\
99.95	0.01\\
99.96	0.01\\
99.97	0.01\\
99.98	0.01\\
99.99	0.01\\
100	0.01\\
};
\addlegendentry{$q=1$};

\addplot [color=red,solid,forget plot]
  table[row sep=crcr]{%
0.01	0.01\\
0.02	0.01\\
0.03	0.01\\
0.04	0.01\\
0.05	0.01\\
0.06	0.01\\
0.07	0.01\\
0.08	0.01\\
0.09	0.01\\
0.1	0.01\\
0.11	0.01\\
0.12	0.01\\
0.13	0.01\\
0.14	0.01\\
0.15	0.01\\
0.16	0.01\\
0.17	0.01\\
0.18	0.01\\
0.19	0.01\\
0.2	0.01\\
0.21	0.01\\
0.22	0.01\\
0.23	0.01\\
0.24	0.01\\
0.25	0.01\\
0.26	0.01\\
0.27	0.01\\
0.28	0.01\\
0.29	0.01\\
0.3	0.01\\
0.31	0.01\\
0.32	0.01\\
0.33	0.01\\
0.34	0.01\\
0.35	0.01\\
0.36	0.01\\
0.37	0.01\\
0.38	0.01\\
0.39	0.01\\
0.4	0.01\\
0.41	0.01\\
0.42	0.01\\
0.43	0.01\\
0.44	0.01\\
0.45	0.01\\
0.46	0.01\\
0.47	0.01\\
0.48	0.01\\
0.49	0.01\\
0.5	0.01\\
0.51	0.01\\
0.52	0.01\\
0.53	0.01\\
0.54	0.01\\
0.55	0.01\\
0.56	0.01\\
0.57	0.01\\
0.58	0.01\\
0.59	0.01\\
0.6	0.01\\
0.61	0.01\\
0.62	0.01\\
0.63	0.01\\
0.64	0.01\\
0.65	0.01\\
0.66	0.01\\
0.67	0.01\\
0.68	0.01\\
0.69	0.01\\
0.7	0.01\\
0.71	0.01\\
0.72	0.01\\
0.73	0.01\\
0.74	0.01\\
0.75	0.01\\
0.76	0.01\\
0.77	0.01\\
0.78	0.01\\
0.79	0.01\\
0.8	0.01\\
0.81	0.01\\
0.82	0.01\\
0.83	0.01\\
0.84	0.01\\
0.85	0.01\\
0.86	0.01\\
0.87	0.01\\
0.88	0.01\\
0.89	0.01\\
0.9	0.01\\
0.91	0.01\\
0.92	0.01\\
0.93	0.01\\
0.94	0.01\\
0.95	0.01\\
0.96	0.01\\
0.97	0.01\\
0.98	0.01\\
0.99	0.01\\
1	0.01\\
1.01	0.01\\
1.02	0.01\\
1.03	0.01\\
1.04	0.01\\
1.05	0.01\\
1.06	0.01\\
1.07	0.01\\
1.08	0.01\\
1.09	0.01\\
1.1	0.01\\
1.11	0.01\\
1.12	0.01\\
1.13	0.01\\
1.14	0.01\\
1.15	0.01\\
1.16	0.01\\
1.17	0.01\\
1.18	0.01\\
1.19	0.01\\
1.2	0.01\\
1.21	0.01\\
1.22	0.01\\
1.23	0.01\\
1.24	0.01\\
1.25	0.01\\
1.26	0.01\\
1.27	0.01\\
1.28	0.01\\
1.29	0.01\\
1.3	0.01\\
1.31	0.01\\
1.32	0.01\\
1.33	0.01\\
1.34	0.01\\
1.35	0.01\\
1.36	0.01\\
1.37	0.01\\
1.38	0.01\\
1.39	0.01\\
1.4	0.01\\
1.41	0.01\\
1.42	0.01\\
1.43	0.01\\
1.44	0.01\\
1.45	0.01\\
1.46	0.01\\
1.47	0.01\\
1.48	0.01\\
1.49	0.01\\
1.5	0.01\\
1.51	0.01\\
1.52	0.01\\
1.53	0.01\\
1.54	0.01\\
1.55	0.01\\
1.56	0.01\\
1.57	0.01\\
1.58	0.01\\
1.59	0.01\\
1.6	0.01\\
1.61	0.01\\
1.62	0.01\\
1.63	0.01\\
1.64	0.01\\
1.65	0.01\\
1.66	0.01\\
1.67	0.01\\
1.68	0.01\\
1.69	0.01\\
1.7	0.01\\
1.71	0.01\\
1.72	0.01\\
1.73	0.01\\
1.74	0.01\\
1.75	0.01\\
1.76	0.01\\
1.77	0.01\\
1.78	0.01\\
1.79	0.01\\
1.8	0.01\\
1.81	0.01\\
1.82	0.01\\
1.83	0.01\\
1.84	0.01\\
1.85	0.01\\
1.86	0.01\\
1.87	0.01\\
1.88	0.01\\
1.89	0.01\\
1.9	0.01\\
1.91	0.01\\
1.92	0.01\\
1.93	0.01\\
1.94	0.01\\
1.95	0.01\\
1.96	0.01\\
1.97	0.01\\
1.98	0.01\\
1.99	0.01\\
2	0.01\\
2.01	0.01\\
2.02	0.01\\
2.03	0.01\\
2.04	0.01\\
2.05	0.01\\
2.06	0.01\\
2.07	0.01\\
2.08	0.01\\
2.09	0.01\\
2.1	0.01\\
2.11	0.01\\
2.12	0.01\\
2.13	0.01\\
2.14	0.01\\
2.15	0.01\\
2.16	0.01\\
2.17	0.01\\
2.18	0.01\\
2.19	0.01\\
2.2	0.01\\
2.21	0.01\\
2.22	0.01\\
2.23	0.01\\
2.24	0.01\\
2.25	0.01\\
2.26	0.01\\
2.27	0.01\\
2.28	0.01\\
2.29	0.01\\
2.3	0.01\\
2.31	0.01\\
2.32	0.01\\
2.33	0.01\\
2.34	0.01\\
2.35	0.01\\
2.36	0.01\\
2.37	0.01\\
2.38	0.01\\
2.39	0.01\\
2.4	0.01\\
2.41	0.01\\
2.42	0.01\\
2.43	0.01\\
2.44	0.01\\
2.45	0.01\\
2.46	0.01\\
2.47	0.01\\
2.48	0.01\\
2.49	0.01\\
2.5	0.01\\
2.51	0.01\\
2.52	0.01\\
2.53	0.01\\
2.54	0.01\\
2.55	0.01\\
2.56	0.01\\
2.57	0.01\\
2.58	0.01\\
2.59	0.01\\
2.6	0.01\\
2.61	0.01\\
2.62	0.01\\
2.63	0.01\\
2.64	0.01\\
2.65	0.01\\
2.66	0.01\\
2.67	0.01\\
2.68	0.01\\
2.69	0.01\\
2.7	0.01\\
2.71	0.01\\
2.72	0.01\\
2.73	0.01\\
2.74	0.01\\
2.75	0.01\\
2.76	0.01\\
2.77	0.01\\
2.78	0.01\\
2.79	0.01\\
2.8	0.01\\
2.81	0.01\\
2.82	0.01\\
2.83	0.01\\
2.84	0.01\\
2.85	0.01\\
2.86	0.01\\
2.87	0.01\\
2.88	0.01\\
2.89	0.01\\
2.9	0.01\\
2.91	0.01\\
2.92	0.01\\
2.93	0.01\\
2.94	0.01\\
2.95	0.01\\
2.96	0.01\\
2.97	0.01\\
2.98	0.01\\
2.99	0.01\\
3	0.01\\
3.01	0.01\\
3.02	0.01\\
3.03	0.01\\
3.04	0.01\\
3.05	0.01\\
3.06	0.01\\
3.07	0.01\\
3.08	0.01\\
3.09	0.01\\
3.1	0.01\\
3.11	0.01\\
3.12	0.01\\
3.13	0.01\\
3.14	0.01\\
3.15	0.01\\
3.16	0.01\\
3.17	0.01\\
3.18	0.01\\
3.19	0.01\\
3.2	0.01\\
3.21	0.01\\
3.22	0.01\\
3.23	0.01\\
3.24	0.01\\
3.25	0.01\\
3.26	0.01\\
3.27	0.01\\
3.28	0.01\\
3.29	0.01\\
3.3	0.01\\
3.31	0.01\\
3.32	0.01\\
3.33	0.01\\
3.34	0.01\\
3.35	0.01\\
3.36	0.01\\
3.37	0.01\\
3.38	0.01\\
3.39	0.01\\
3.4	0.01\\
3.41	0.01\\
3.42	0.01\\
3.43	0.01\\
3.44	0.01\\
3.45	0.01\\
3.46	0.01\\
3.47	0.01\\
3.48	0.01\\
3.49	0.01\\
3.5	0.01\\
3.51	0.01\\
3.52	0.01\\
3.53	0.01\\
3.54	0.01\\
3.55	0.01\\
3.56	0.01\\
3.57	0.01\\
3.58	0.01\\
3.59	0.01\\
3.6	0.01\\
3.61	0.01\\
3.62	0.01\\
3.63	0.01\\
3.64	0.01\\
3.65	0.01\\
3.66	0.01\\
3.67	0.01\\
3.68	0.01\\
3.69	0.01\\
3.7	0.01\\
3.71	0.01\\
3.72	0.01\\
3.73	0.01\\
3.74	0.01\\
3.75	0.01\\
3.76	0.01\\
3.77	0.01\\
3.78	0.01\\
3.79	0.01\\
3.8	0.01\\
3.81	0.01\\
3.82	0.01\\
3.83	0.01\\
3.84	0.01\\
3.85	0.01\\
3.86	0.01\\
3.87	0.01\\
3.88	0.01\\
3.89	0.01\\
3.9	0.01\\
3.91	0.01\\
3.92	0.01\\
3.93	0.01\\
3.94	0.01\\
3.95	0.01\\
3.96	0.01\\
3.97	0.01\\
3.98	0.01\\
3.99	0.01\\
4	0.01\\
4.01	0.01\\
4.02	0.01\\
4.03	0.01\\
4.04	0.01\\
4.05	0.01\\
4.06	0.01\\
4.07	0.01\\
4.08	0.01\\
4.09	0.01\\
4.1	0.01\\
4.11	0.01\\
4.12	0.01\\
4.13	0.01\\
4.14	0.01\\
4.15	0.01\\
4.16	0.01\\
4.17	0.01\\
4.18	0.01\\
4.19	0.01\\
4.2	0.01\\
4.21	0.01\\
4.22	0.01\\
4.23	0.01\\
4.24	0.01\\
4.25	0.01\\
4.26	0.01\\
4.27	0.01\\
4.28	0.01\\
4.29	0.01\\
4.3	0.01\\
4.31	0.01\\
4.32	0.01\\
4.33	0.01\\
4.34	0.01\\
4.35	0.01\\
4.36	0.01\\
4.37	0.01\\
4.38	0.01\\
4.39	0.01\\
4.4	0.01\\
4.41	0.01\\
4.42	0.01\\
4.43	0.01\\
4.44	0.01\\
4.45	0.01\\
4.46	0.01\\
4.47	0.01\\
4.48	0.01\\
4.49	0.01\\
4.5	0.01\\
4.51	0.01\\
4.52	0.01\\
4.53	0.01\\
4.54	0.01\\
4.55	0.01\\
4.56	0.01\\
4.57	0.01\\
4.58	0.01\\
4.59	0.01\\
4.6	0.01\\
4.61	0.01\\
4.62	0.01\\
4.63	0.01\\
4.64	0.01\\
4.65	0.01\\
4.66	0.01\\
4.67	0.01\\
4.68	0.01\\
4.69	0.01\\
4.7	0.01\\
4.71	0.01\\
4.72	0.01\\
4.73	0.01\\
4.74	0.01\\
4.75	0.01\\
4.76	0.01\\
4.77	0.01\\
4.78	0.01\\
4.79	0.01\\
4.8	0.01\\
4.81	0.01\\
4.82	0.01\\
4.83	0.01\\
4.84	0.01\\
4.85	0.01\\
4.86	0.01\\
4.87	0.01\\
4.88	0.01\\
4.89	0.01\\
4.9	0.01\\
4.91	0.01\\
4.92	0.01\\
4.93	0.01\\
4.94	0.01\\
4.95	0.01\\
4.96	0.01\\
4.97	0.01\\
4.98	0.01\\
4.99	0.01\\
5	0.01\\
5.01	0.01\\
5.02	0.01\\
5.03	0.01\\
5.04	0.01\\
5.05	0.01\\
5.06	0.01\\
5.07	0.01\\
5.08	0.01\\
5.09	0.01\\
5.1	0.01\\
5.11	0.01\\
5.12	0.01\\
5.13	0.01\\
5.14	0.01\\
5.15	0.01\\
5.16	0.01\\
5.17	0.01\\
5.18	0.01\\
5.19	0.01\\
5.2	0.01\\
5.21	0.01\\
5.22	0.01\\
5.23	0.01\\
5.24	0.01\\
5.25	0.01\\
5.26	0.01\\
5.27	0.01\\
5.28	0.01\\
5.29	0.01\\
5.3	0.01\\
5.31	0.01\\
5.32	0.01\\
5.33	0.01\\
5.34	0.01\\
5.35	0.01\\
5.36	0.01\\
5.37	0.01\\
5.38	0.01\\
5.39	0.01\\
5.4	0.01\\
5.41	0.01\\
5.42	0.01\\
5.43	0.01\\
5.44	0.01\\
5.45	0.01\\
5.46	0.01\\
5.47	0.01\\
5.48	0.01\\
5.49	0.01\\
5.5	0.01\\
5.51	0.01\\
5.52	0.01\\
5.53	0.01\\
5.54	0.01\\
5.55	0.01\\
5.56	0.01\\
5.57	0.01\\
5.58	0.01\\
5.59	0.01\\
5.6	0.01\\
5.61	0.01\\
5.62	0.01\\
5.63	0.01\\
5.64	0.01\\
5.65	0.01\\
5.66	0.01\\
5.67	0.01\\
5.68	0.01\\
5.69	0.01\\
5.7	0.01\\
5.71	0.01\\
5.72	0.01\\
5.73	0.01\\
5.74	0.01\\
5.75	0.01\\
5.76	0.01\\
5.77	0.01\\
5.78	0.01\\
5.79	0.01\\
5.8	0.01\\
5.81	0.01\\
5.82	0.01\\
5.83	0.01\\
5.84	0.01\\
5.85	0.01\\
5.86	0.01\\
5.87	0.01\\
5.88	0.01\\
5.89	0.01\\
5.9	0.01\\
5.91	0.01\\
5.92	0.01\\
5.93	0.01\\
5.94	0.01\\
5.95	0.01\\
5.96	0.01\\
5.97	0.01\\
5.98	0.01\\
5.99	0.01\\
6	0.01\\
6.01	0.01\\
6.02	0.01\\
6.03	0.01\\
6.04	0.01\\
6.05	0.01\\
6.06	0.01\\
6.07	0.01\\
6.08	0.01\\
6.09	0.01\\
6.1	0.01\\
6.11	0.01\\
6.12	0.01\\
6.13	0.01\\
6.14	0.01\\
6.15	0.01\\
6.16	0.01\\
6.17	0.01\\
6.18	0.01\\
6.19	0.01\\
6.2	0.01\\
6.21	0.01\\
6.22	0.01\\
6.23	0.01\\
6.24	0.01\\
6.25	0.01\\
6.26	0.01\\
6.27	0.01\\
6.28	0.01\\
6.29	0.01\\
6.3	0.01\\
6.31	0.01\\
6.32	0.01\\
6.33	0.01\\
6.34	0.01\\
6.35	0.01\\
6.36	0.01\\
6.37	0.01\\
6.38	0.01\\
6.39	0.01\\
6.4	0.01\\
6.41	0.01\\
6.42	0.01\\
6.43	0.01\\
6.44	0.01\\
6.45	0.01\\
6.46	0.01\\
6.47	0.01\\
6.48	0.01\\
6.49	0.01\\
6.5	0.01\\
6.51	0.01\\
6.52	0.01\\
6.53	0.01\\
6.54	0.01\\
6.55	0.01\\
6.56	0.01\\
6.57	0.01\\
6.58	0.01\\
6.59	0.01\\
6.6	0.01\\
6.61	0.01\\
6.62	0.01\\
6.63	0.01\\
6.64	0.01\\
6.65	0.01\\
6.66	0.01\\
6.67	0.01\\
6.68	0.01\\
6.69	0.01\\
6.7	0.01\\
6.71	0.01\\
6.72	0.01\\
6.73	0.01\\
6.74	0.01\\
6.75	0.01\\
6.76	0.01\\
6.77	0.01\\
6.78	0.01\\
6.79	0.01\\
6.8	0.01\\
6.81	0.01\\
6.82	0.01\\
6.83	0.01\\
6.84	0.01\\
6.85	0.01\\
6.86	0.01\\
6.87	0.01\\
6.88	0.01\\
6.89	0.01\\
6.9	0.01\\
6.91	0.01\\
6.92	0.01\\
6.93	0.01\\
6.94	0.01\\
6.95	0.01\\
6.96	0.01\\
6.97	0.01\\
6.98	0.01\\
6.99	0.01\\
7	0.01\\
7.01	0.01\\
7.02	0.01\\
7.03	0.01\\
7.04	0.01\\
7.05	0.01\\
7.06	0.01\\
7.07	0.01\\
7.08	0.01\\
7.09	0.01\\
7.1	0.01\\
7.11	0.01\\
7.12	0.01\\
7.13	0.01\\
7.14	0.01\\
7.15	0.01\\
7.16	0.01\\
7.17	0.01\\
7.18	0.01\\
7.19	0.01\\
7.2	0.01\\
7.21	0.01\\
7.22	0.01\\
7.23	0.01\\
7.24	0.01\\
7.25	0.01\\
7.26	0.01\\
7.27	0.01\\
7.28	0.01\\
7.29	0.01\\
7.3	0.01\\
7.31	0.01\\
7.32	0.01\\
7.33	0.01\\
7.34	0.01\\
7.35	0.01\\
7.36	0.01\\
7.37	0.01\\
7.38	0.01\\
7.39	0.01\\
7.4	0.01\\
7.41	0.01\\
7.42	0.01\\
7.43	0.01\\
7.44	0.01\\
7.45	0.01\\
7.46	0.01\\
7.47	0.01\\
7.48	0.01\\
7.49	0.01\\
7.5	0.01\\
7.51	0.01\\
7.52	0.01\\
7.53	0.01\\
7.54	0.01\\
7.55	0.01\\
7.56	0.01\\
7.57	0.01\\
7.58	0.01\\
7.59	0.01\\
7.6	0.01\\
7.61	0.01\\
7.62	0.01\\
7.63	0.01\\
7.64	0.01\\
7.65	0.01\\
7.66	0.01\\
7.67	0.01\\
7.68	0.01\\
7.69	0.01\\
7.7	0.01\\
7.71	0.01\\
7.72	0.01\\
7.73	0.01\\
7.74	0.01\\
7.75	0.01\\
7.76	0.01\\
7.77	0.01\\
7.78	0.01\\
7.79	0.01\\
7.8	0.01\\
7.81	0.01\\
7.82	0.01\\
7.83	0.01\\
7.84	0.01\\
7.85	0.01\\
7.86	0.01\\
7.87	0.01\\
7.88	0.01\\
7.89	0.01\\
7.9	0.01\\
7.91	0.01\\
7.92	0.01\\
7.93	0.01\\
7.94	0.01\\
7.95	0.01\\
7.96	0.01\\
7.97	0.01\\
7.98	0.01\\
7.99	0.01\\
8	0.01\\
8.01	0.01\\
8.02	0.01\\
8.03	0.01\\
8.04	0.01\\
8.05	0.01\\
8.06	0.01\\
8.07	0.01\\
8.08	0.01\\
8.09	0.01\\
8.1	0.01\\
8.11	0.01\\
8.12	0.01\\
8.13	0.01\\
8.14	0.01\\
8.15	0.01\\
8.16	0.01\\
8.17	0.01\\
8.18	0.01\\
8.19	0.01\\
8.2	0.01\\
8.21	0.01\\
8.22	0.01\\
8.23	0.01\\
8.24	0.01\\
8.25	0.01\\
8.26	0.01\\
8.27	0.01\\
8.28	0.01\\
8.29	0.01\\
8.3	0.01\\
8.31	0.01\\
8.32	0.01\\
8.33	0.01\\
8.34	0.01\\
8.35	0.01\\
8.36	0.01\\
8.37	0.01\\
8.38	0.01\\
8.39	0.01\\
8.4	0.01\\
8.41	0.01\\
8.42	0.01\\
8.43	0.01\\
8.44	0.01\\
8.45	0.01\\
8.46	0.01\\
8.47	0.01\\
8.48	0.01\\
8.49	0.01\\
8.5	0.01\\
8.51	0.01\\
8.52	0.01\\
8.53	0.01\\
8.54	0.01\\
8.55	0.01\\
8.56	0.01\\
8.57	0.01\\
8.58	0.01\\
8.59	0.01\\
8.6	0.01\\
8.61	0.01\\
8.62	0.01\\
8.63	0.01\\
8.64	0.01\\
8.65	0.01\\
8.66	0.01\\
8.67	0.01\\
8.68	0.01\\
8.69	0.01\\
8.7	0.01\\
8.71	0.01\\
8.72	0.01\\
8.73	0.01\\
8.74	0.01\\
8.75	0.01\\
8.76	0.01\\
8.77	0.01\\
8.78	0.01\\
8.79	0.01\\
8.8	0.01\\
8.81	0.01\\
8.82	0.01\\
8.83	0.01\\
8.84	0.01\\
8.85	0.01\\
8.86	0.01\\
8.87	0.01\\
8.88	0.01\\
8.89	0.01\\
8.9	0.01\\
8.91	0.01\\
8.92	0.01\\
8.93	0.01\\
8.94	0.01\\
8.95	0.01\\
8.96	0.01\\
8.97	0.01\\
8.98	0.01\\
8.99	0.01\\
9	0.01\\
9.01	0.01\\
9.02	0.01\\
9.03	0.01\\
9.04	0.01\\
9.05	0.01\\
9.06	0.01\\
9.07	0.01\\
9.08	0.01\\
9.09	0.01\\
9.1	0.01\\
9.11	0.01\\
9.12	0.01\\
9.13	0.01\\
9.14	0.01\\
9.15	0.01\\
9.16	0.01\\
9.17	0.01\\
9.18	0.01\\
9.19	0.01\\
9.2	0.01\\
9.21	0.01\\
9.22	0.01\\
9.23	0.01\\
9.24	0.01\\
9.25	0.01\\
9.26	0.01\\
9.27	0.01\\
9.28	0.01\\
9.29	0.01\\
9.3	0.01\\
9.31	0.01\\
9.32	0.01\\
9.33	0.01\\
9.34	0.01\\
9.35	0.01\\
9.36	0.01\\
9.37	0.01\\
9.38	0.01\\
9.39	0.01\\
9.4	0.01\\
9.41	0.01\\
9.42	0.01\\
9.43	0.01\\
9.44	0.01\\
9.45	0.01\\
9.46	0.01\\
9.47	0.01\\
9.48	0.01\\
9.49	0.01\\
9.5	0.01\\
9.51	0.01\\
9.52	0.01\\
9.53	0.01\\
9.54	0.01\\
9.55	0.01\\
9.56	0.01\\
9.57	0.01\\
9.58	0.01\\
9.59	0.01\\
9.6	0.01\\
9.61	0.01\\
9.62	0.01\\
9.63	0.01\\
9.64	0.01\\
9.65	0.01\\
9.66	0.01\\
9.67	0.01\\
9.68	0.01\\
9.69	0.01\\
9.7	0.01\\
9.71	0.01\\
9.72	0.01\\
9.73	0.01\\
9.74	0.01\\
9.75	0.01\\
9.76	0.01\\
9.77	0.01\\
9.78	0.01\\
9.79	0.01\\
9.8	0.01\\
9.81	0.01\\
9.82	0.01\\
9.83	0.01\\
9.84	0.01\\
9.85	0.01\\
9.86	0.01\\
9.87	0.01\\
9.88	0.01\\
9.89	0.01\\
9.9	0.01\\
9.91	0.01\\
9.92	0.01\\
9.93	0.01\\
9.94	0.01\\
9.95	0.01\\
9.96	0.01\\
9.97	0.01\\
9.98	0.01\\
9.99	0.01\\
10	0.01\\
10.01	0.01\\
10.02	0.01\\
10.03	0.01\\
10.04	0.01\\
10.05	0.01\\
10.06	0.01\\
10.07	0.01\\
10.08	0.01\\
10.09	0.01\\
10.1	0.01\\
10.11	0.01\\
10.12	0.01\\
10.13	0.01\\
10.14	0.01\\
10.15	0.01\\
10.16	0.01\\
10.17	0.01\\
10.18	0.01\\
10.19	0.01\\
10.2	0.01\\
10.21	0.01\\
10.22	0.01\\
10.23	0.01\\
10.24	0.01\\
10.25	0.01\\
10.26	0.01\\
10.27	0.01\\
10.28	0.01\\
10.29	0.01\\
10.3	0.01\\
10.31	0.01\\
10.32	0.01\\
10.33	0.01\\
10.34	0.01\\
10.35	0.01\\
10.36	0.01\\
10.37	0.01\\
10.38	0.01\\
10.39	0.01\\
10.4	0.01\\
10.41	0.01\\
10.42	0.01\\
10.43	0.01\\
10.44	0.01\\
10.45	0.01\\
10.46	0.01\\
10.47	0.01\\
10.48	0.01\\
10.49	0.01\\
10.5	0.01\\
10.51	0.01\\
10.52	0.01\\
10.53	0.01\\
10.54	0.01\\
10.55	0.01\\
10.56	0.01\\
10.57	0.01\\
10.58	0.01\\
10.59	0.01\\
10.6	0.01\\
10.61	0.01\\
10.62	0.01\\
10.63	0.01\\
10.64	0.01\\
10.65	0.01\\
10.66	0.01\\
10.67	0.01\\
10.68	0.01\\
10.69	0.01\\
10.7	0.01\\
10.71	0.01\\
10.72	0.01\\
10.73	0.01\\
10.74	0.01\\
10.75	0.01\\
10.76	0.01\\
10.77	0.01\\
10.78	0.01\\
10.79	0.01\\
10.8	0.01\\
10.81	0.01\\
10.82	0.01\\
10.83	0.01\\
10.84	0.01\\
10.85	0.01\\
10.86	0.01\\
10.87	0.01\\
10.88	0.01\\
10.89	0.01\\
10.9	0.01\\
10.91	0.01\\
10.92	0.01\\
10.93	0.01\\
10.94	0.01\\
10.95	0.01\\
10.96	0.01\\
10.97	0.01\\
10.98	0.01\\
10.99	0.01\\
11	0.01\\
11.01	0.01\\
11.02	0.01\\
11.03	0.01\\
11.04	0.01\\
11.05	0.01\\
11.06	0.01\\
11.07	0.01\\
11.08	0.01\\
11.09	0.01\\
11.1	0.01\\
11.11	0.01\\
11.12	0.01\\
11.13	0.01\\
11.14	0.01\\
11.15	0.01\\
11.16	0.01\\
11.17	0.01\\
11.18	0.01\\
11.19	0.01\\
11.2	0.01\\
11.21	0.01\\
11.22	0.01\\
11.23	0.01\\
11.24	0.01\\
11.25	0.01\\
11.26	0.01\\
11.27	0.01\\
11.28	0.01\\
11.29	0.01\\
11.3	0.01\\
11.31	0.01\\
11.32	0.01\\
11.33	0.01\\
11.34	0.01\\
11.35	0.01\\
11.36	0.01\\
11.37	0.01\\
11.38	0.01\\
11.39	0.01\\
11.4	0.01\\
11.41	0.01\\
11.42	0.01\\
11.43	0.01\\
11.44	0.01\\
11.45	0.01\\
11.46	0.01\\
11.47	0.01\\
11.48	0.01\\
11.49	0.01\\
11.5	0.01\\
11.51	0.01\\
11.52	0.01\\
11.53	0.01\\
11.54	0.01\\
11.55	0.01\\
11.56	0.01\\
11.57	0.01\\
11.58	0.01\\
11.59	0.01\\
11.6	0.01\\
11.61	0.01\\
11.62	0.01\\
11.63	0.01\\
11.64	0.01\\
11.65	0.01\\
11.66	0.01\\
11.67	0.01\\
11.68	0.01\\
11.69	0.01\\
11.7	0.01\\
11.71	0.01\\
11.72	0.01\\
11.73	0.01\\
11.74	0.01\\
11.75	0.01\\
11.76	0.01\\
11.77	0.01\\
11.78	0.01\\
11.79	0.01\\
11.8	0.01\\
11.81	0.01\\
11.82	0.01\\
11.83	0.01\\
11.84	0.01\\
11.85	0.01\\
11.86	0.01\\
11.87	0.01\\
11.88	0.01\\
11.89	0.01\\
11.9	0.01\\
11.91	0.01\\
11.92	0.01\\
11.93	0.01\\
11.94	0.01\\
11.95	0.01\\
11.96	0.01\\
11.97	0.01\\
11.98	0.01\\
11.99	0.01\\
12	0.01\\
12.01	0.01\\
12.02	0.01\\
12.03	0.01\\
12.04	0.01\\
12.05	0.01\\
12.06	0.01\\
12.07	0.01\\
12.08	0.01\\
12.09	0.01\\
12.1	0.01\\
12.11	0.01\\
12.12	0.01\\
12.13	0.01\\
12.14	0.01\\
12.15	0.01\\
12.16	0.01\\
12.17	0.01\\
12.18	0.01\\
12.19	0.01\\
12.2	0.01\\
12.21	0.01\\
12.22	0.01\\
12.23	0.01\\
12.24	0.01\\
12.25	0.01\\
12.26	0.01\\
12.27	0.01\\
12.28	0.01\\
12.29	0.01\\
12.3	0.01\\
12.31	0.01\\
12.32	0.01\\
12.33	0.01\\
12.34	0.01\\
12.35	0.01\\
12.36	0.01\\
12.37	0.01\\
12.38	0.01\\
12.39	0.01\\
12.4	0.01\\
12.41	0.01\\
12.42	0.01\\
12.43	0.01\\
12.44	0.01\\
12.45	0.01\\
12.46	0.01\\
12.47	0.01\\
12.48	0.01\\
12.49	0.01\\
12.5	0.01\\
12.51	0.01\\
12.52	0.01\\
12.53	0.01\\
12.54	0.01\\
12.55	0.01\\
12.56	0.01\\
12.57	0.01\\
12.58	0.01\\
12.59	0.01\\
12.6	0.01\\
12.61	0.01\\
12.62	0.01\\
12.63	0.01\\
12.64	0.01\\
12.65	0.01\\
12.66	0.01\\
12.67	0.01\\
12.68	0.01\\
12.69	0.01\\
12.7	0.01\\
12.71	0.01\\
12.72	0.01\\
12.73	0.01\\
12.74	0.01\\
12.75	0.01\\
12.76	0.01\\
12.77	0.01\\
12.78	0.01\\
12.79	0.01\\
12.8	0.01\\
12.81	0.01\\
12.82	0.01\\
12.83	0.01\\
12.84	0.01\\
12.85	0.01\\
12.86	0.01\\
12.87	0.01\\
12.88	0.01\\
12.89	0.01\\
12.9	0.01\\
12.91	0.01\\
12.92	0.01\\
12.93	0.01\\
12.94	0.01\\
12.95	0.01\\
12.96	0.01\\
12.97	0.01\\
12.98	0.01\\
12.99	0.01\\
13	0.01\\
13.01	0.01\\
13.02	0.01\\
13.03	0.01\\
13.04	0.01\\
13.05	0.01\\
13.06	0.01\\
13.07	0.01\\
13.08	0.01\\
13.09	0.01\\
13.1	0.01\\
13.11	0.01\\
13.12	0.01\\
13.13	0.01\\
13.14	0.01\\
13.15	0.01\\
13.16	0.01\\
13.17	0.01\\
13.18	0.01\\
13.19	0.01\\
13.2	0.01\\
13.21	0.01\\
13.22	0.01\\
13.23	0.01\\
13.24	0.01\\
13.25	0.01\\
13.26	0.01\\
13.27	0.01\\
13.28	0.01\\
13.29	0.01\\
13.3	0.01\\
13.31	0.01\\
13.32	0.01\\
13.33	0.01\\
13.34	0.01\\
13.35	0.01\\
13.36	0.01\\
13.37	0.01\\
13.38	0.01\\
13.39	0.01\\
13.4	0.01\\
13.41	0.01\\
13.42	0.01\\
13.43	0.01\\
13.44	0.01\\
13.45	0.01\\
13.46	0.01\\
13.47	0.01\\
13.48	0.01\\
13.49	0.01\\
13.5	0.01\\
13.51	0.01\\
13.52	0.01\\
13.53	0.01\\
13.54	0.01\\
13.55	0.01\\
13.56	0.01\\
13.57	0.01\\
13.58	0.01\\
13.59	0.01\\
13.6	0.01\\
13.61	0.01\\
13.62	0.01\\
13.63	0.01\\
13.64	0.01\\
13.65	0.01\\
13.66	0.01\\
13.67	0.01\\
13.68	0.01\\
13.69	0.01\\
13.7	0.01\\
13.71	0.01\\
13.72	0.01\\
13.73	0.01\\
13.74	0.01\\
13.75	0.01\\
13.76	0.01\\
13.77	0.01\\
13.78	0.01\\
13.79	0.01\\
13.8	0.01\\
13.81	0.01\\
13.82	0.01\\
13.83	0.01\\
13.84	0.01\\
13.85	0.01\\
13.86	0.01\\
13.87	0.01\\
13.88	0.01\\
13.89	0.01\\
13.9	0.01\\
13.91	0.01\\
13.92	0.01\\
13.93	0.01\\
13.94	0.01\\
13.95	0.01\\
13.96	0.01\\
13.97	0.01\\
13.98	0.01\\
13.99	0.01\\
14	0.01\\
14.01	0.01\\
14.02	0.01\\
14.03	0.01\\
14.04	0.01\\
14.05	0.01\\
14.06	0.01\\
14.07	0.01\\
14.08	0.01\\
14.09	0.01\\
14.1	0.01\\
14.11	0.01\\
14.12	0.01\\
14.13	0.01\\
14.14	0.01\\
14.15	0.01\\
14.16	0.01\\
14.17	0.01\\
14.18	0.01\\
14.19	0.01\\
14.2	0.01\\
14.21	0.01\\
14.22	0.01\\
14.23	0.01\\
14.24	0.01\\
14.25	0.01\\
14.26	0.01\\
14.27	0.01\\
14.28	0.01\\
14.29	0.01\\
14.3	0.01\\
14.31	0.01\\
14.32	0.01\\
14.33	0.01\\
14.34	0.01\\
14.35	0.01\\
14.36	0.01\\
14.37	0.01\\
14.38	0.01\\
14.39	0.01\\
14.4	0.01\\
14.41	0.01\\
14.42	0.01\\
14.43	0.01\\
14.44	0.01\\
14.45	0.01\\
14.46	0.01\\
14.47	0.01\\
14.48	0.01\\
14.49	0.01\\
14.5	0.01\\
14.51	0.01\\
14.52	0.01\\
14.53	0.01\\
14.54	0.01\\
14.55	0.01\\
14.56	0.01\\
14.57	0.01\\
14.58	0.01\\
14.59	0.01\\
14.6	0.01\\
14.61	0.01\\
14.62	0.01\\
14.63	0.01\\
14.64	0.01\\
14.65	0.01\\
14.66	0.01\\
14.67	0.01\\
14.68	0.01\\
14.69	0.01\\
14.7	0.01\\
14.71	0.01\\
14.72	0.01\\
14.73	0.01\\
14.74	0.01\\
14.75	0.01\\
14.76	0.01\\
14.77	0.01\\
14.78	0.01\\
14.79	0.01\\
14.8	0.01\\
14.81	0.01\\
14.82	0.01\\
14.83	0.01\\
14.84	0.01\\
14.85	0.01\\
14.86	0.01\\
14.87	0.01\\
14.88	0.01\\
14.89	0.01\\
14.9	0.01\\
14.91	0.01\\
14.92	0.01\\
14.93	0.01\\
14.94	0.01\\
14.95	0.01\\
14.96	0.01\\
14.97	0.01\\
14.98	0.01\\
14.99	0.01\\
15	0.01\\
15.01	0.01\\
15.02	0.01\\
15.03	0.01\\
15.04	0.01\\
15.05	0.01\\
15.06	0.01\\
15.07	0.01\\
15.08	0.01\\
15.09	0.01\\
15.1	0.01\\
15.11	0.01\\
15.12	0.01\\
15.13	0.01\\
15.14	0.01\\
15.15	0.01\\
15.16	0.01\\
15.17	0.01\\
15.18	0.01\\
15.19	0.01\\
15.2	0.01\\
15.21	0.01\\
15.22	0.01\\
15.23	0.01\\
15.24	0.01\\
15.25	0.01\\
15.26	0.01\\
15.27	0.01\\
15.28	0.01\\
15.29	0.01\\
15.3	0.01\\
15.31	0.01\\
15.32	0.01\\
15.33	0.01\\
15.34	0.01\\
15.35	0.01\\
15.36	0.01\\
15.37	0.01\\
15.38	0.01\\
15.39	0.01\\
15.4	0.01\\
15.41	0.01\\
15.42	0.01\\
15.43	0.01\\
15.44	0.01\\
15.45	0.01\\
15.46	0.01\\
15.47	0.01\\
15.48	0.01\\
15.49	0.01\\
15.5	0.01\\
15.51	0.01\\
15.52	0.01\\
15.53	0.01\\
15.54	0.01\\
15.55	0.01\\
15.56	0.01\\
15.57	0.01\\
15.58	0.01\\
15.59	0.01\\
15.6	0.01\\
15.61	0.01\\
15.62	0.01\\
15.63	0.01\\
15.64	0.01\\
15.65	0.01\\
15.66	0.01\\
15.67	0.01\\
15.68	0.01\\
15.69	0.01\\
15.7	0.01\\
15.71	0.01\\
15.72	0.01\\
15.73	0.01\\
15.74	0.01\\
15.75	0.01\\
15.76	0.01\\
15.77	0.01\\
15.78	0.01\\
15.79	0.01\\
15.8	0.01\\
15.81	0.01\\
15.82	0.01\\
15.83	0.01\\
15.84	0.01\\
15.85	0.01\\
15.86	0.01\\
15.87	0.01\\
15.88	0.01\\
15.89	0.01\\
15.9	0.01\\
15.91	0.01\\
15.92	0.01\\
15.93	0.01\\
15.94	0.01\\
15.95	0.01\\
15.96	0.01\\
15.97	0.01\\
15.98	0.01\\
15.99	0.01\\
16	0.01\\
16.01	0.01\\
16.02	0.01\\
16.03	0.01\\
16.04	0.01\\
16.05	0.01\\
16.06	0.01\\
16.07	0.01\\
16.08	0.01\\
16.09	0.01\\
16.1	0.01\\
16.11	0.01\\
16.12	0.01\\
16.13	0.01\\
16.14	0.01\\
16.15	0.01\\
16.16	0.01\\
16.17	0.01\\
16.18	0.01\\
16.19	0.01\\
16.2	0.01\\
16.21	0.01\\
16.22	0.01\\
16.23	0.01\\
16.24	0.01\\
16.25	0.01\\
16.26	0.01\\
16.27	0.01\\
16.28	0.01\\
16.29	0.01\\
16.3	0.01\\
16.31	0.01\\
16.32	0.01\\
16.33	0.01\\
16.34	0.01\\
16.35	0.01\\
16.36	0.01\\
16.37	0.01\\
16.38	0.01\\
16.39	0.01\\
16.4	0.01\\
16.41	0.01\\
16.42	0.01\\
16.43	0.01\\
16.44	0.01\\
16.45	0.01\\
16.46	0.01\\
16.47	0.01\\
16.48	0.01\\
16.49	0.01\\
16.5	0.01\\
16.51	0.01\\
16.52	0.01\\
16.53	0.01\\
16.54	0.01\\
16.55	0.01\\
16.56	0.01\\
16.57	0.01\\
16.58	0.01\\
16.59	0.01\\
16.6	0.01\\
16.61	0.01\\
16.62	0.01\\
16.63	0.01\\
16.64	0.01\\
16.65	0.01\\
16.66	0.01\\
16.67	0.01\\
16.68	0.01\\
16.69	0.01\\
16.7	0.01\\
16.71	0.01\\
16.72	0.01\\
16.73	0.01\\
16.74	0.01\\
16.75	0.01\\
16.76	0.01\\
16.77	0.01\\
16.78	0.01\\
16.79	0.01\\
16.8	0.01\\
16.81	0.01\\
16.82	0.01\\
16.83	0.01\\
16.84	0.01\\
16.85	0.01\\
16.86	0.01\\
16.87	0.01\\
16.88	0.01\\
16.89	0.01\\
16.9	0.01\\
16.91	0.01\\
16.92	0.01\\
16.93	0.01\\
16.94	0.01\\
16.95	0.01\\
16.96	0.01\\
16.97	0.01\\
16.98	0.01\\
16.99	0.01\\
17	0.01\\
17.01	0.01\\
17.02	0.01\\
17.03	0.01\\
17.04	0.01\\
17.05	0.01\\
17.06	0.01\\
17.07	0.01\\
17.08	0.01\\
17.09	0.01\\
17.1	0.01\\
17.11	0.01\\
17.12	0.01\\
17.13	0.01\\
17.14	0.01\\
17.15	0.01\\
17.16	0.01\\
17.17	0.01\\
17.18	0.01\\
17.19	0.01\\
17.2	0.01\\
17.21	0.01\\
17.22	0.01\\
17.23	0.01\\
17.24	0.01\\
17.25	0.01\\
17.26	0.01\\
17.27	0.01\\
17.28	0.01\\
17.29	0.01\\
17.3	0.01\\
17.31	0.01\\
17.32	0.01\\
17.33	0.01\\
17.34	0.01\\
17.35	0.01\\
17.36	0.01\\
17.37	0.01\\
17.38	0.01\\
17.39	0.01\\
17.4	0.01\\
17.41	0.01\\
17.42	0.01\\
17.43	0.01\\
17.44	0.01\\
17.45	0.01\\
17.46	0.01\\
17.47	0.01\\
17.48	0.01\\
17.49	0.01\\
17.5	0.01\\
17.51	0.01\\
17.52	0.01\\
17.53	0.01\\
17.54	0.01\\
17.55	0.01\\
17.56	0.01\\
17.57	0.01\\
17.58	0.01\\
17.59	0.01\\
17.6	0.01\\
17.61	0.01\\
17.62	0.01\\
17.63	0.01\\
17.64	0.01\\
17.65	0.01\\
17.66	0.01\\
17.67	0.01\\
17.68	0.01\\
17.69	0.01\\
17.7	0.01\\
17.71	0.01\\
17.72	0.01\\
17.73	0.01\\
17.74	0.01\\
17.75	0.01\\
17.76	0.01\\
17.77	0.01\\
17.78	0.01\\
17.79	0.01\\
17.8	0.01\\
17.81	0.01\\
17.82	0.01\\
17.83	0.01\\
17.84	0.01\\
17.85	0.01\\
17.86	0.01\\
17.87	0.01\\
17.88	0.01\\
17.89	0.01\\
17.9	0.01\\
17.91	0.01\\
17.92	0.01\\
17.93	0.01\\
17.94	0.01\\
17.95	0.01\\
17.96	0.01\\
17.97	0.01\\
17.98	0.01\\
17.99	0.01\\
18	0.01\\
18.01	0.01\\
18.02	0.01\\
18.03	0.01\\
18.04	0.01\\
18.05	0.01\\
18.06	0.01\\
18.07	0.01\\
18.08	0.01\\
18.09	0.01\\
18.1	0.01\\
18.11	0.01\\
18.12	0.01\\
18.13	0.01\\
18.14	0.01\\
18.15	0.01\\
18.16	0.01\\
18.17	0.01\\
18.18	0.01\\
18.19	0.01\\
18.2	0.01\\
18.21	0.01\\
18.22	0.01\\
18.23	0.01\\
18.24	0.01\\
18.25	0.01\\
18.26	0.01\\
18.27	0.01\\
18.28	0.01\\
18.29	0.01\\
18.3	0.01\\
18.31	0.01\\
18.32	0.01\\
18.33	0.01\\
18.34	0.01\\
18.35	0.01\\
18.36	0.01\\
18.37	0.01\\
18.38	0.01\\
18.39	0.01\\
18.4	0.01\\
18.41	0.01\\
18.42	0.01\\
18.43	0.01\\
18.44	0.01\\
18.45	0.01\\
18.46	0.01\\
18.47	0.01\\
18.48	0.01\\
18.49	0.01\\
18.5	0.01\\
18.51	0.01\\
18.52	0.01\\
18.53	0.01\\
18.54	0.01\\
18.55	0.01\\
18.56	0.01\\
18.57	0.01\\
18.58	0.01\\
18.59	0.01\\
18.6	0.01\\
18.61	0.01\\
18.62	0.01\\
18.63	0.01\\
18.64	0.01\\
18.65	0.01\\
18.66	0.01\\
18.67	0.01\\
18.68	0.01\\
18.69	0.01\\
18.7	0.01\\
18.71	0.01\\
18.72	0.01\\
18.73	0.01\\
18.74	0.01\\
18.75	0.01\\
18.76	0.01\\
18.77	0.01\\
18.78	0.01\\
18.79	0.01\\
18.8	0.01\\
18.81	0.01\\
18.82	0.01\\
18.83	0.01\\
18.84	0.01\\
18.85	0.01\\
18.86	0.01\\
18.87	0.01\\
18.88	0.01\\
18.89	0.01\\
18.9	0.01\\
18.91	0.01\\
18.92	0.01\\
18.93	0.01\\
18.94	0.01\\
18.95	0.01\\
18.96	0.01\\
18.97	0.01\\
18.98	0.01\\
18.99	0.01\\
19	0.01\\
19.01	0.01\\
19.02	0.01\\
19.03	0.01\\
19.04	0.01\\
19.05	0.01\\
19.06	0.01\\
19.07	0.01\\
19.08	0.01\\
19.09	0.01\\
19.1	0.01\\
19.11	0.01\\
19.12	0.01\\
19.13	0.01\\
19.14	0.01\\
19.15	0.01\\
19.16	0.01\\
19.17	0.01\\
19.18	0.01\\
19.19	0.01\\
19.2	0.01\\
19.21	0.01\\
19.22	0.01\\
19.23	0.01\\
19.24	0.01\\
19.25	0.01\\
19.26	0.01\\
19.27	0.01\\
19.28	0.01\\
19.29	0.01\\
19.3	0.01\\
19.31	0.01\\
19.32	0.01\\
19.33	0.01\\
19.34	0.01\\
19.35	0.01\\
19.36	0.01\\
19.37	0.01\\
19.38	0.01\\
19.39	0.01\\
19.4	0.01\\
19.41	0.01\\
19.42	0.01\\
19.43	0.01\\
19.44	0.01\\
19.45	0.01\\
19.46	0.01\\
19.47	0.01\\
19.48	0.01\\
19.49	0.01\\
19.5	0.01\\
19.51	0.01\\
19.52	0.01\\
19.53	0.01\\
19.54	0.01\\
19.55	0.01\\
19.56	0.01\\
19.57	0.01\\
19.58	0.01\\
19.59	0.01\\
19.6	0.01\\
19.61	0.01\\
19.62	0.01\\
19.63	0.01\\
19.64	0.01\\
19.65	0.01\\
19.66	0.01\\
19.67	0.01\\
19.68	0.01\\
19.69	0.01\\
19.7	0.01\\
19.71	0.01\\
19.72	0.01\\
19.73	0.01\\
19.74	0.01\\
19.75	0.01\\
19.76	0.01\\
19.77	0.01\\
19.78	0.01\\
19.79	0.01\\
19.8	0.01\\
19.81	0.01\\
19.82	0.01\\
19.83	0.01\\
19.84	0.01\\
19.85	0.01\\
19.86	0.01\\
19.87	0.01\\
19.88	0.01\\
19.89	0.01\\
19.9	0.01\\
19.91	0.01\\
19.92	0.01\\
19.93	0.01\\
19.94	0.01\\
19.95	0.01\\
19.96	0.01\\
19.97	0.01\\
19.98	0.01\\
19.99	0.01\\
20	0.01\\
20.01	0.01\\
20.02	0.01\\
20.03	0.01\\
20.04	0.01\\
20.05	0.01\\
20.06	0.01\\
20.07	0.01\\
20.08	0.01\\
20.09	0.01\\
20.1	0.01\\
20.11	0.01\\
20.12	0.01\\
20.13	0.01\\
20.14	0.01\\
20.15	0.01\\
20.16	0.01\\
20.17	0.01\\
20.18	0.01\\
20.19	0.01\\
20.2	0.01\\
20.21	0.01\\
20.22	0.01\\
20.23	0.01\\
20.24	0.01\\
20.25	0.01\\
20.26	0.01\\
20.27	0.01\\
20.28	0.01\\
20.29	0.01\\
20.3	0.01\\
20.31	0.01\\
20.32	0.01\\
20.33	0.01\\
20.34	0.01\\
20.35	0.01\\
20.36	0.01\\
20.37	0.01\\
20.38	0.01\\
20.39	0.01\\
20.4	0.01\\
20.41	0.01\\
20.42	0.01\\
20.43	0.01\\
20.44	0.01\\
20.45	0.01\\
20.46	0.01\\
20.47	0.01\\
20.48	0.01\\
20.49	0.01\\
20.5	0.01\\
20.51	0.01\\
20.52	0.01\\
20.53	0.01\\
20.54	0.01\\
20.55	0.01\\
20.56	0.01\\
20.57	0.01\\
20.58	0.01\\
20.59	0.01\\
20.6	0.01\\
20.61	0.01\\
20.62	0.01\\
20.63	0.01\\
20.64	0.01\\
20.65	0.01\\
20.66	0.01\\
20.67	0.01\\
20.68	0.01\\
20.69	0.01\\
20.7	0.01\\
20.71	0.01\\
20.72	0.01\\
20.73	0.01\\
20.74	0.01\\
20.75	0.01\\
20.76	0.01\\
20.77	0.01\\
20.78	0.01\\
20.79	0.01\\
20.8	0.01\\
20.81	0.01\\
20.82	0.01\\
20.83	0.01\\
20.84	0.01\\
20.85	0.01\\
20.86	0.01\\
20.87	0.01\\
20.88	0.01\\
20.89	0.01\\
20.9	0.01\\
20.91	0.01\\
20.92	0.01\\
20.93	0.01\\
20.94	0.01\\
20.95	0.01\\
20.96	0.01\\
20.97	0.01\\
20.98	0.01\\
20.99	0.01\\
21	0.01\\
21.01	0.01\\
21.02	0.01\\
21.03	0.01\\
21.04	0.01\\
21.05	0.01\\
21.06	0.01\\
21.07	0.01\\
21.08	0.01\\
21.09	0.01\\
21.1	0.01\\
21.11	0.01\\
21.12	0.01\\
21.13	0.01\\
21.14	0.01\\
21.15	0.01\\
21.16	0.01\\
21.17	0.01\\
21.18	0.01\\
21.19	0.01\\
21.2	0.01\\
21.21	0.01\\
21.22	0.01\\
21.23	0.01\\
21.24	0.01\\
21.25	0.01\\
21.26	0.01\\
21.27	0.01\\
21.28	0.01\\
21.29	0.01\\
21.3	0.01\\
21.31	0.01\\
21.32	0.01\\
21.33	0.01\\
21.34	0.01\\
21.35	0.01\\
21.36	0.01\\
21.37	0.01\\
21.38	0.01\\
21.39	0.01\\
21.4	0.01\\
21.41	0.01\\
21.42	0.01\\
21.43	0.01\\
21.44	0.01\\
21.45	0.01\\
21.46	0.01\\
21.47	0.01\\
21.48	0.01\\
21.49	0.01\\
21.5	0.01\\
21.51	0.01\\
21.52	0.01\\
21.53	0.01\\
21.54	0.01\\
21.55	0.01\\
21.56	0.01\\
21.57	0.01\\
21.58	0.01\\
21.59	0.01\\
21.6	0.01\\
21.61	0.01\\
21.62	0.01\\
21.63	0.01\\
21.64	0.01\\
21.65	0.01\\
21.66	0.01\\
21.67	0.01\\
21.68	0.01\\
21.69	0.01\\
21.7	0.01\\
21.71	0.01\\
21.72	0.01\\
21.73	0.01\\
21.74	0.01\\
21.75	0.01\\
21.76	0.01\\
21.77	0.01\\
21.78	0.01\\
21.79	0.01\\
21.8	0.01\\
21.81	0.01\\
21.82	0.01\\
21.83	0.01\\
21.84	0.01\\
21.85	0.01\\
21.86	0.01\\
21.87	0.01\\
21.88	0.01\\
21.89	0.01\\
21.9	0.01\\
21.91	0.01\\
21.92	0.01\\
21.93	0.01\\
21.94	0.01\\
21.95	0.01\\
21.96	0.01\\
21.97	0.01\\
21.98	0.01\\
21.99	0.01\\
22	0.01\\
22.01	0.01\\
22.02	0.01\\
22.03	0.01\\
22.04	0.01\\
22.05	0.01\\
22.06	0.01\\
22.07	0.01\\
22.08	0.01\\
22.09	0.01\\
22.1	0.01\\
22.11	0.01\\
22.12	0.01\\
22.13	0.01\\
22.14	0.01\\
22.15	0.01\\
22.16	0.01\\
22.17	0.01\\
22.18	0.01\\
22.19	0.01\\
22.2	0.01\\
22.21	0.01\\
22.22	0.01\\
22.23	0.01\\
22.24	0.01\\
22.25	0.01\\
22.26	0.01\\
22.27	0.01\\
22.28	0.01\\
22.29	0.01\\
22.3	0.01\\
22.31	0.01\\
22.32	0.01\\
22.33	0.01\\
22.34	0.01\\
22.35	0.01\\
22.36	0.01\\
22.37	0.01\\
22.38	0.01\\
22.39	0.01\\
22.4	0.01\\
22.41	0.01\\
22.42	0.01\\
22.43	0.01\\
22.44	0.01\\
22.45	0.01\\
22.46	0.01\\
22.47	0.01\\
22.48	0.01\\
22.49	0.01\\
22.5	0.01\\
22.51	0.01\\
22.52	0.01\\
22.53	0.01\\
22.54	0.01\\
22.55	0.01\\
22.56	0.01\\
22.57	0.01\\
22.58	0.01\\
22.59	0.01\\
22.6	0.01\\
22.61	0.01\\
22.62	0.01\\
22.63	0.01\\
22.64	0.01\\
22.65	0.01\\
22.66	0.01\\
22.67	0.01\\
22.68	0.01\\
22.69	0.01\\
22.7	0.01\\
22.71	0.01\\
22.72	0.01\\
22.73	0.01\\
22.74	0.01\\
22.75	0.01\\
22.76	0.01\\
22.77	0.01\\
22.78	0.01\\
22.79	0.01\\
22.8	0.01\\
22.81	0.01\\
22.82	0.01\\
22.83	0.01\\
22.84	0.01\\
22.85	0.01\\
22.86	0.01\\
22.87	0.01\\
22.88	0.01\\
22.89	0.01\\
22.9	0.01\\
22.91	0.01\\
22.92	0.01\\
22.93	0.01\\
22.94	0.01\\
22.95	0.01\\
22.96	0.01\\
22.97	0.01\\
22.98	0.01\\
22.99	0.01\\
23	0.01\\
23.01	0.01\\
23.02	0.01\\
23.03	0.01\\
23.04	0.01\\
23.05	0.01\\
23.06	0.01\\
23.07	0.01\\
23.08	0.01\\
23.09	0.01\\
23.1	0.01\\
23.11	0.01\\
23.12	0.01\\
23.13	0.01\\
23.14	0.01\\
23.15	0.01\\
23.16	0.01\\
23.17	0.01\\
23.18	0.01\\
23.19	0.01\\
23.2	0.01\\
23.21	0.01\\
23.22	0.01\\
23.23	0.01\\
23.24	0.01\\
23.25	0.01\\
23.26	0.01\\
23.27	0.01\\
23.28	0.01\\
23.29	0.01\\
23.3	0.01\\
23.31	0.01\\
23.32	0.01\\
23.33	0.01\\
23.34	0.01\\
23.35	0.01\\
23.36	0.01\\
23.37	0.01\\
23.38	0.01\\
23.39	0.01\\
23.4	0.01\\
23.41	0.01\\
23.42	0.01\\
23.43	0.01\\
23.44	0.01\\
23.45	0.01\\
23.46	0.01\\
23.47	0.01\\
23.48	0.01\\
23.49	0.01\\
23.5	0.01\\
23.51	0.01\\
23.52	0.01\\
23.53	0.01\\
23.54	0.01\\
23.55	0.01\\
23.56	0.01\\
23.57	0.01\\
23.58	0.01\\
23.59	0.01\\
23.6	0.01\\
23.61	0.01\\
23.62	0.01\\
23.63	0.01\\
23.64	0.01\\
23.65	0.01\\
23.66	0.01\\
23.67	0.01\\
23.68	0.01\\
23.69	0.01\\
23.7	0.01\\
23.71	0.01\\
23.72	0.01\\
23.73	0.01\\
23.74	0.01\\
23.75	0.01\\
23.76	0.01\\
23.77	0.01\\
23.78	0.01\\
23.79	0.01\\
23.8	0.01\\
23.81	0.01\\
23.82	0.01\\
23.83	0.01\\
23.84	0.01\\
23.85	0.01\\
23.86	0.01\\
23.87	0.01\\
23.88	0.01\\
23.89	0.01\\
23.9	0.01\\
23.91	0.01\\
23.92	0.01\\
23.93	0.01\\
23.94	0.01\\
23.95	0.01\\
23.96	0.01\\
23.97	0.01\\
23.98	0.01\\
23.99	0.01\\
24	0.01\\
24.01	0.01\\
24.02	0.01\\
24.03	0.01\\
24.04	0.01\\
24.05	0.01\\
24.06	0.01\\
24.07	0.01\\
24.08	0.01\\
24.09	0.01\\
24.1	0.01\\
24.11	0.01\\
24.12	0.01\\
24.13	0.01\\
24.14	0.01\\
24.15	0.01\\
24.16	0.01\\
24.17	0.01\\
24.18	0.01\\
24.19	0.01\\
24.2	0.01\\
24.21	0.01\\
24.22	0.01\\
24.23	0.01\\
24.24	0.01\\
24.25	0.01\\
24.26	0.01\\
24.27	0.01\\
24.28	0.01\\
24.29	0.01\\
24.3	0.01\\
24.31	0.01\\
24.32	0.01\\
24.33	0.01\\
24.34	0.01\\
24.35	0.01\\
24.36	0.01\\
24.37	0.01\\
24.38	0.01\\
24.39	0.01\\
24.4	0.01\\
24.41	0.01\\
24.42	0.01\\
24.43	0.01\\
24.44	0.01\\
24.45	0.01\\
24.46	0.01\\
24.47	0.01\\
24.48	0.01\\
24.49	0.01\\
24.5	0.01\\
24.51	0.01\\
24.52	0.01\\
24.53	0.01\\
24.54	0.01\\
24.55	0.01\\
24.56	0.01\\
24.57	0.01\\
24.58	0.01\\
24.59	0.01\\
24.6	0.01\\
24.61	0.01\\
24.62	0.01\\
24.63	0.01\\
24.64	0.01\\
24.65	0.01\\
24.66	0.01\\
24.67	0.01\\
24.68	0.01\\
24.69	0.01\\
24.7	0.01\\
24.71	0.01\\
24.72	0.01\\
24.73	0.01\\
24.74	0.01\\
24.75	0.01\\
24.76	0.01\\
24.77	0.01\\
24.78	0.01\\
24.79	0.01\\
24.8	0.01\\
24.81	0.01\\
24.82	0.01\\
24.83	0.01\\
24.84	0.01\\
24.85	0.01\\
24.86	0.01\\
24.87	0.01\\
24.88	0.01\\
24.89	0.01\\
24.9	0.01\\
24.91	0.01\\
24.92	0.01\\
24.93	0.01\\
24.94	0.01\\
24.95	0.01\\
24.96	0.01\\
24.97	0.01\\
24.98	0.01\\
24.99	0.01\\
25	0.01\\
25.01	0.01\\
25.02	0.01\\
25.03	0.01\\
25.04	0.01\\
25.05	0.01\\
25.06	0.01\\
25.07	0.01\\
25.08	0.01\\
25.09	0.01\\
25.1	0.01\\
25.11	0.01\\
25.12	0.01\\
25.13	0.01\\
25.14	0.01\\
25.15	0.01\\
25.16	0.01\\
25.17	0.01\\
25.18	0.01\\
25.19	0.01\\
25.2	0.01\\
25.21	0.01\\
25.22	0.01\\
25.23	0.01\\
25.24	0.01\\
25.25	0.01\\
25.26	0.01\\
25.27	0.01\\
25.28	0.01\\
25.29	0.01\\
25.3	0.01\\
25.31	0.01\\
25.32	0.01\\
25.33	0.01\\
25.34	0.01\\
25.35	0.01\\
25.36	0.01\\
25.37	0.01\\
25.38	0.01\\
25.39	0.01\\
25.4	0.01\\
25.41	0.01\\
25.42	0.01\\
25.43	0.01\\
25.44	0.01\\
25.45	0.01\\
25.46	0.01\\
25.47	0.01\\
25.48	0.01\\
25.49	0.01\\
25.5	0.01\\
25.51	0.01\\
25.52	0.01\\
25.53	0.01\\
25.54	0.01\\
25.55	0.01\\
25.56	0.01\\
25.57	0.01\\
25.58	0.01\\
25.59	0.01\\
25.6	0.01\\
25.61	0.01\\
25.62	0.01\\
25.63	0.01\\
25.64	0.01\\
25.65	0.01\\
25.66	0.01\\
25.67	0.01\\
25.68	0.01\\
25.69	0.01\\
25.7	0.01\\
25.71	0.01\\
25.72	0.01\\
25.73	0.01\\
25.74	0.01\\
25.75	0.01\\
25.76	0.01\\
25.77	0.01\\
25.78	0.01\\
25.79	0.01\\
25.8	0.01\\
25.81	0.01\\
25.82	0.01\\
25.83	0.01\\
25.84	0.01\\
25.85	0.01\\
25.86	0.01\\
25.87	0.01\\
25.88	0.01\\
25.89	0.01\\
25.9	0.01\\
25.91	0.01\\
25.92	0.01\\
25.93	0.01\\
25.94	0.01\\
25.95	0.01\\
25.96	0.01\\
25.97	0.01\\
25.98	0.01\\
25.99	0.01\\
26	0.01\\
26.01	0.01\\
26.02	0.01\\
26.03	0.01\\
26.04	0.01\\
26.05	0.01\\
26.06	0.01\\
26.07	0.01\\
26.08	0.01\\
26.09	0.01\\
26.1	0.01\\
26.11	0.01\\
26.12	0.01\\
26.13	0.01\\
26.14	0.01\\
26.15	0.01\\
26.16	0.01\\
26.17	0.01\\
26.18	0.01\\
26.19	0.01\\
26.2	0.01\\
26.21	0.01\\
26.22	0.01\\
26.23	0.01\\
26.24	0.01\\
26.25	0.01\\
26.26	0.01\\
26.27	0.01\\
26.28	0.01\\
26.29	0.01\\
26.3	0.01\\
26.31	0.01\\
26.32	0.01\\
26.33	0.01\\
26.34	0.01\\
26.35	0.01\\
26.36	0.01\\
26.37	0.01\\
26.38	0.01\\
26.39	0.01\\
26.4	0.01\\
26.41	0.01\\
26.42	0.01\\
26.43	0.01\\
26.44	0.01\\
26.45	0.01\\
26.46	0.01\\
26.47	0.01\\
26.48	0.01\\
26.49	0.01\\
26.5	0.01\\
26.51	0.01\\
26.52	0.01\\
26.53	0.01\\
26.54	0.01\\
26.55	0.01\\
26.56	0.01\\
26.57	0.01\\
26.58	0.01\\
26.59	0.01\\
26.6	0.01\\
26.61	0.01\\
26.62	0.01\\
26.63	0.01\\
26.64	0.01\\
26.65	0.01\\
26.66	0.01\\
26.67	0.01\\
26.68	0.01\\
26.69	0.01\\
26.7	0.01\\
26.71	0.01\\
26.72	0.01\\
26.73	0.01\\
26.74	0.01\\
26.75	0.01\\
26.76	0.01\\
26.77	0.01\\
26.78	0.01\\
26.79	0.01\\
26.8	0.01\\
26.81	0.01\\
26.82	0.01\\
26.83	0.01\\
26.84	0.01\\
26.85	0.01\\
26.86	0.01\\
26.87	0.01\\
26.88	0.01\\
26.89	0.01\\
26.9	0.01\\
26.91	0.01\\
26.92	0.01\\
26.93	0.01\\
26.94	0.01\\
26.95	0.01\\
26.96	0.01\\
26.97	0.01\\
26.98	0.01\\
26.99	0.01\\
27	0.01\\
27.01	0.01\\
27.02	0.01\\
27.03	0.01\\
27.04	0.01\\
27.05	0.01\\
27.06	0.01\\
27.07	0.01\\
27.08	0.01\\
27.09	0.01\\
27.1	0.01\\
27.11	0.01\\
27.12	0.01\\
27.13	0.01\\
27.14	0.01\\
27.15	0.01\\
27.16	0.01\\
27.17	0.01\\
27.18	0.01\\
27.19	0.01\\
27.2	0.01\\
27.21	0.01\\
27.22	0.01\\
27.23	0.01\\
27.24	0.01\\
27.25	0.01\\
27.26	0.01\\
27.27	0.01\\
27.28	0.01\\
27.29	0.01\\
27.3	0.01\\
27.31	0.01\\
27.32	0.01\\
27.33	0.01\\
27.34	0.01\\
27.35	0.01\\
27.36	0.01\\
27.37	0.01\\
27.38	0.01\\
27.39	0.01\\
27.4	0.01\\
27.41	0.01\\
27.42	0.01\\
27.43	0.01\\
27.44	0.01\\
27.45	0.01\\
27.46	0.01\\
27.47	0.01\\
27.48	0.01\\
27.49	0.01\\
27.5	0.01\\
27.51	0.01\\
27.52	0.01\\
27.53	0.01\\
27.54	0.01\\
27.55	0.01\\
27.56	0.01\\
27.57	0.01\\
27.58	0.01\\
27.59	0.01\\
27.6	0.01\\
27.61	0.01\\
27.62	0.01\\
27.63	0.01\\
27.64	0.01\\
27.65	0.01\\
27.66	0.01\\
27.67	0.01\\
27.68	0.01\\
27.69	0.01\\
27.7	0.01\\
27.71	0.01\\
27.72	0.01\\
27.73	0.01\\
27.74	0.01\\
27.75	0.01\\
27.76	0.01\\
27.77	0.01\\
27.78	0.01\\
27.79	0.01\\
27.8	0.01\\
27.81	0.01\\
27.82	0.01\\
27.83	0.01\\
27.84	0.01\\
27.85	0.01\\
27.86	0.01\\
27.87	0.01\\
27.88	0.01\\
27.89	0.01\\
27.9	0.01\\
27.91	0.01\\
27.92	0.01\\
27.93	0.01\\
27.94	0.01\\
27.95	0.01\\
27.96	0.01\\
27.97	0.01\\
27.98	0.01\\
27.99	0.01\\
28	0.01\\
28.01	0.01\\
28.02	0.01\\
28.03	0.01\\
28.04	0.01\\
28.05	0.01\\
28.06	0.01\\
28.07	0.01\\
28.08	0.01\\
28.09	0.01\\
28.1	0.01\\
28.11	0.01\\
28.12	0.01\\
28.13	0.01\\
28.14	0.01\\
28.15	0.01\\
28.16	0.01\\
28.17	0.01\\
28.18	0.01\\
28.19	0.01\\
28.2	0.01\\
28.21	0.01\\
28.22	0.01\\
28.23	0.01\\
28.24	0.01\\
28.25	0.01\\
28.26	0.01\\
28.27	0.01\\
28.28	0.01\\
28.29	0.01\\
28.3	0.01\\
28.31	0.01\\
28.32	0.01\\
28.33	0.01\\
28.34	0.01\\
28.35	0.01\\
28.36	0.01\\
28.37	0.01\\
28.38	0.01\\
28.39	0.01\\
28.4	0.01\\
28.41	0.01\\
28.42	0.01\\
28.43	0.01\\
28.44	0.01\\
28.45	0.01\\
28.46	0.01\\
28.47	0.01\\
28.48	0.01\\
28.49	0.01\\
28.5	0.01\\
28.51	0.01\\
28.52	0.01\\
28.53	0.01\\
28.54	0.01\\
28.55	0.01\\
28.56	0.01\\
28.57	0.01\\
28.58	0.01\\
28.59	0.01\\
28.6	0.01\\
28.61	0.01\\
28.62	0.01\\
28.63	0.01\\
28.64	0.01\\
28.65	0.01\\
28.66	0.01\\
28.67	0.01\\
28.68	0.01\\
28.69	0.01\\
28.7	0.01\\
28.71	0.01\\
28.72	0.01\\
28.73	0.01\\
28.74	0.01\\
28.75	0.01\\
28.76	0.01\\
28.77	0.01\\
28.78	0.01\\
28.79	0.01\\
28.8	0.01\\
28.81	0.01\\
28.82	0.01\\
28.83	0.01\\
28.84	0.01\\
28.85	0.01\\
28.86	0.01\\
28.87	0.01\\
28.88	0.01\\
28.89	0.01\\
28.9	0.01\\
28.91	0.01\\
28.92	0.01\\
28.93	0.01\\
28.94	0.01\\
28.95	0.01\\
28.96	0.01\\
28.97	0.01\\
28.98	0.01\\
28.99	0.01\\
29	0.01\\
29.01	0.01\\
29.02	0.01\\
29.03	0.01\\
29.04	0.01\\
29.05	0.01\\
29.06	0.01\\
29.07	0.01\\
29.08	0.01\\
29.09	0.01\\
29.1	0.01\\
29.11	0.01\\
29.12	0.01\\
29.13	0.01\\
29.14	0.01\\
29.15	0.01\\
29.16	0.01\\
29.17	0.01\\
29.18	0.01\\
29.19	0.01\\
29.2	0.01\\
29.21	0.01\\
29.22	0.01\\
29.23	0.01\\
29.24	0.01\\
29.25	0.01\\
29.26	0.01\\
29.27	0.01\\
29.28	0.01\\
29.29	0.01\\
29.3	0.01\\
29.31	0.01\\
29.32	0.01\\
29.33	0.01\\
29.34	0.01\\
29.35	0.01\\
29.36	0.01\\
29.37	0.01\\
29.38	0.01\\
29.39	0.01\\
29.4	0.01\\
29.41	0.01\\
29.42	0.01\\
29.43	0.01\\
29.44	0.01\\
29.45	0.01\\
29.46	0.01\\
29.47	0.01\\
29.48	0.01\\
29.49	0.01\\
29.5	0.01\\
29.51	0.01\\
29.52	0.01\\
29.53	0.01\\
29.54	0.01\\
29.55	0.01\\
29.56	0.01\\
29.57	0.01\\
29.58	0.01\\
29.59	0.01\\
29.6	0.01\\
29.61	0.01\\
29.62	0.01\\
29.63	0.01\\
29.64	0.01\\
29.65	0.01\\
29.66	0.01\\
29.67	0.01\\
29.68	0.01\\
29.69	0.01\\
29.7	0.01\\
29.71	0.01\\
29.72	0.01\\
29.73	0.01\\
29.74	0.01\\
29.75	0.01\\
29.76	0.01\\
29.77	0.01\\
29.78	0.01\\
29.79	0.01\\
29.8	0.01\\
29.81	0.01\\
29.82	0.01\\
29.83	0.01\\
29.84	0.01\\
29.85	0.01\\
29.86	0.01\\
29.87	0.01\\
29.88	0.01\\
29.89	0.01\\
29.9	0.01\\
29.91	0.01\\
29.92	0.01\\
29.93	0.01\\
29.94	0.01\\
29.95	0.01\\
29.96	0.01\\
29.97	0.01\\
29.98	0.01\\
29.99	0.01\\
30	0.01\\
30.01	0.01\\
30.02	0.01\\
30.03	0.01\\
30.04	0.01\\
30.05	0.01\\
30.06	0.01\\
30.07	0.01\\
30.08	0.01\\
30.09	0.01\\
30.1	0.01\\
30.11	0.01\\
30.12	0.01\\
30.13	0.01\\
30.14	0.01\\
30.15	0.01\\
30.16	0.01\\
30.17	0.01\\
30.18	0.01\\
30.19	0.01\\
30.2	0.01\\
30.21	0.01\\
30.22	0.01\\
30.23	0.01\\
30.24	0.01\\
30.25	0.01\\
30.26	0.01\\
30.27	0.01\\
30.28	0.01\\
30.29	0.01\\
30.3	0.01\\
30.31	0.01\\
30.32	0.01\\
30.33	0.01\\
30.34	0.01\\
30.35	0.01\\
30.36	0.01\\
30.37	0.01\\
30.38	0.01\\
30.39	0.01\\
30.4	0.01\\
30.41	0.01\\
30.42	0.01\\
30.43	0.01\\
30.44	0.01\\
30.45	0.01\\
30.46	0.01\\
30.47	0.01\\
30.48	0.01\\
30.49	0.01\\
30.5	0.01\\
30.51	0.01\\
30.52	0.01\\
30.53	0.01\\
30.54	0.01\\
30.55	0.01\\
30.56	0.01\\
30.57	0.01\\
30.58	0.01\\
30.59	0.01\\
30.6	0.01\\
30.61	0.01\\
30.62	0.01\\
30.63	0.01\\
30.64	0.01\\
30.65	0.01\\
30.66	0.01\\
30.67	0.01\\
30.68	0.01\\
30.69	0.01\\
30.7	0.01\\
30.71	0.01\\
30.72	0.01\\
30.73	0.01\\
30.74	0.01\\
30.75	0.01\\
30.76	0.01\\
30.77	0.01\\
30.78	0.01\\
30.79	0.01\\
30.8	0.01\\
30.81	0.01\\
30.82	0.01\\
30.83	0.01\\
30.84	0.01\\
30.85	0.01\\
30.86	0.01\\
30.87	0.01\\
30.88	0.01\\
30.89	0.01\\
30.9	0.01\\
30.91	0.01\\
30.92	0.01\\
30.93	0.01\\
30.94	0.01\\
30.95	0.01\\
30.96	0.01\\
30.97	0.01\\
30.98	0.01\\
30.99	0.01\\
31	0.01\\
31.01	0.01\\
31.02	0.01\\
31.03	0.01\\
31.04	0.01\\
31.05	0.01\\
31.06	0.01\\
31.07	0.01\\
31.08	0.01\\
31.09	0.01\\
31.1	0.01\\
31.11	0.01\\
31.12	0.01\\
31.13	0.01\\
31.14	0.01\\
31.15	0.01\\
31.16	0.01\\
31.17	0.01\\
31.18	0.01\\
31.19	0.01\\
31.2	0.01\\
31.21	0.01\\
31.22	0.01\\
31.23	0.01\\
31.24	0.01\\
31.25	0.01\\
31.26	0.01\\
31.27	0.01\\
31.28	0.01\\
31.29	0.01\\
31.3	0.01\\
31.31	0.01\\
31.32	0.01\\
31.33	0.01\\
31.34	0.01\\
31.35	0.01\\
31.36	0.01\\
31.37	0.01\\
31.38	0.01\\
31.39	0.01\\
31.4	0.01\\
31.41	0.01\\
31.42	0.01\\
31.43	0.01\\
31.44	0.01\\
31.45	0.01\\
31.46	0.01\\
31.47	0.01\\
31.48	0.01\\
31.49	0.01\\
31.5	0.01\\
31.51	0.01\\
31.52	0.01\\
31.53	0.01\\
31.54	0.01\\
31.55	0.01\\
31.56	0.01\\
31.57	0.01\\
31.58	0.01\\
31.59	0.01\\
31.6	0.01\\
31.61	0.01\\
31.62	0.01\\
31.63	0.01\\
31.64	0.01\\
31.65	0.01\\
31.66	0.01\\
31.67	0.01\\
31.68	0.01\\
31.69	0.01\\
31.7	0.01\\
31.71	0.01\\
31.72	0.01\\
31.73	0.01\\
31.74	0.01\\
31.75	0.01\\
31.76	0.01\\
31.77	0.01\\
31.78	0.01\\
31.79	0.01\\
31.8	0.01\\
31.81	0.01\\
31.82	0.01\\
31.83	0.01\\
31.84	0.01\\
31.85	0.01\\
31.86	0.01\\
31.87	0.01\\
31.88	0.01\\
31.89	0.01\\
31.9	0.01\\
31.91	0.01\\
31.92	0.01\\
31.93	0.01\\
31.94	0.01\\
31.95	0.01\\
31.96	0.01\\
31.97	0.01\\
31.98	0.01\\
31.99	0.01\\
32	0.01\\
32.01	0.01\\
32.02	0.01\\
32.03	0.01\\
32.04	0.01\\
32.05	0.01\\
32.06	0.01\\
32.07	0.01\\
32.08	0.01\\
32.09	0.01\\
32.1	0.01\\
32.11	0.01\\
32.12	0.01\\
32.13	0.01\\
32.14	0.01\\
32.15	0.01\\
32.16	0.01\\
32.17	0.01\\
32.18	0.01\\
32.19	0.01\\
32.2	0.01\\
32.21	0.01\\
32.22	0.01\\
32.23	0.01\\
32.24	0.01\\
32.25	0.01\\
32.26	0.01\\
32.27	0.01\\
32.28	0.01\\
32.29	0.01\\
32.3	0.01\\
32.31	0.01\\
32.32	0.01\\
32.33	0.01\\
32.34	0.01\\
32.35	0.01\\
32.36	0.01\\
32.37	0.01\\
32.38	0.01\\
32.39	0.01\\
32.4	0.01\\
32.41	0.01\\
32.42	0.01\\
32.43	0.01\\
32.44	0.01\\
32.45	0.01\\
32.46	0.01\\
32.47	0.01\\
32.48	0.01\\
32.49	0.01\\
32.5	0.01\\
32.51	0.01\\
32.52	0.01\\
32.53	0.01\\
32.54	0.01\\
32.55	0.01\\
32.56	0.01\\
32.57	0.01\\
32.58	0.01\\
32.59	0.01\\
32.6	0.01\\
32.61	0.01\\
32.62	0.01\\
32.63	0.01\\
32.64	0.01\\
32.65	0.01\\
32.66	0.01\\
32.67	0.01\\
32.68	0.01\\
32.69	0.01\\
32.7	0.01\\
32.71	0.01\\
32.72	0.01\\
32.73	0.01\\
32.74	0.01\\
32.75	0.01\\
32.76	0.01\\
32.77	0.01\\
32.78	0.01\\
32.79	0.01\\
32.8	0.01\\
32.81	0.01\\
32.82	0.01\\
32.83	0.01\\
32.84	0.01\\
32.85	0.01\\
32.86	0.01\\
32.87	0.01\\
32.88	0.01\\
32.89	0.01\\
32.9	0.01\\
32.91	0.01\\
32.92	0.01\\
32.93	0.01\\
32.94	0.01\\
32.95	0.01\\
32.96	0.01\\
32.97	0.01\\
32.98	0.01\\
32.99	0.01\\
33	0.01\\
33.01	0.01\\
33.02	0.01\\
33.03	0.01\\
33.04	0.01\\
33.05	0.01\\
33.06	0.01\\
33.07	0.01\\
33.08	0.01\\
33.09	0.01\\
33.1	0.01\\
33.11	0.01\\
33.12	0.01\\
33.13	0.01\\
33.14	0.01\\
33.15	0.01\\
33.16	0.01\\
33.17	0.01\\
33.18	0.01\\
33.19	0.01\\
33.2	0.01\\
33.21	0.01\\
33.22	0.01\\
33.23	0.01\\
33.24	0.01\\
33.25	0.01\\
33.26	0.01\\
33.27	0.01\\
33.28	0.01\\
33.29	0.01\\
33.3	0.01\\
33.31	0.01\\
33.32	0.01\\
33.33	0.01\\
33.34	0.01\\
33.35	0.01\\
33.36	0.01\\
33.37	0.01\\
33.38	0.01\\
33.39	0.01\\
33.4	0.01\\
33.41	0.01\\
33.42	0.01\\
33.43	0.01\\
33.44	0.01\\
33.45	0.01\\
33.46	0.01\\
33.47	0.01\\
33.48	0.01\\
33.49	0.01\\
33.5	0.01\\
33.51	0.01\\
33.52	0.01\\
33.53	0.01\\
33.54	0.01\\
33.55	0.01\\
33.56	0.01\\
33.57	0.01\\
33.58	0.01\\
33.59	0.01\\
33.6	0.01\\
33.61	0.01\\
33.62	0.01\\
33.63	0.01\\
33.64	0.01\\
33.65	0.01\\
33.66	0.01\\
33.67	0.01\\
33.68	0.01\\
33.69	0.01\\
33.7	0.01\\
33.71	0.01\\
33.72	0.01\\
33.73	0.01\\
33.74	0.01\\
33.75	0.01\\
33.76	0.01\\
33.77	0.01\\
33.78	0.01\\
33.79	0.01\\
33.8	0.01\\
33.81	0.01\\
33.82	0.01\\
33.83	0.01\\
33.84	0.01\\
33.85	0.01\\
33.86	0.01\\
33.87	0.01\\
33.88	0.01\\
33.89	0.01\\
33.9	0.01\\
33.91	0.01\\
33.92	0.01\\
33.93	0.01\\
33.94	0.01\\
33.95	0.01\\
33.96	0.01\\
33.97	0.01\\
33.98	0.01\\
33.99	0.01\\
34	0.01\\
34.01	0.01\\
34.02	0.01\\
34.03	0.01\\
34.04	0.01\\
34.05	0.01\\
34.06	0.01\\
34.07	0.01\\
34.08	0.01\\
34.09	0.01\\
34.1	0.01\\
34.11	0.01\\
34.12	0.01\\
34.13	0.01\\
34.14	0.01\\
34.15	0.01\\
34.16	0.01\\
34.17	0.01\\
34.18	0.01\\
34.19	0.01\\
34.2	0.01\\
34.21	0.01\\
34.22	0.01\\
34.23	0.01\\
34.24	0.01\\
34.25	0.01\\
34.26	0.01\\
34.27	0.01\\
34.28	0.01\\
34.29	0.01\\
34.3	0.01\\
34.31	0.01\\
34.32	0.01\\
34.33	0.01\\
34.34	0.01\\
34.35	0.01\\
34.36	0.01\\
34.37	0.01\\
34.38	0.01\\
34.39	0.01\\
34.4	0.01\\
34.41	0.01\\
34.42	0.01\\
34.43	0.01\\
34.44	0.01\\
34.45	0.01\\
34.46	0.01\\
34.47	0.01\\
34.48	0.01\\
34.49	0.01\\
34.5	0.01\\
34.51	0.01\\
34.52	0.01\\
34.53	0.01\\
34.54	0.01\\
34.55	0.01\\
34.56	0.01\\
34.57	0.01\\
34.58	0.01\\
34.59	0.01\\
34.6	0.01\\
34.61	0.01\\
34.62	0.01\\
34.63	0.01\\
34.64	0.01\\
34.65	0.01\\
34.66	0.01\\
34.67	0.01\\
34.68	0.01\\
34.69	0.01\\
34.7	0.01\\
34.71	0.01\\
34.72	0.01\\
34.73	0.01\\
34.74	0.01\\
34.75	0.01\\
34.76	0.01\\
34.77	0.01\\
34.78	0.01\\
34.79	0.01\\
34.8	0.01\\
34.81	0.01\\
34.82	0.01\\
34.83	0.01\\
34.84	0.01\\
34.85	0.01\\
34.86	0.01\\
34.87	0.01\\
34.88	0.01\\
34.89	0.01\\
34.9	0.01\\
34.91	0.01\\
34.92	0.01\\
34.93	0.01\\
34.94	0.01\\
34.95	0.01\\
34.96	0.01\\
34.97	0.01\\
34.98	0.01\\
34.99	0.01\\
35	0.01\\
35.01	0.01\\
35.02	0.01\\
35.03	0.01\\
35.04	0.01\\
35.05	0.01\\
35.06	0.01\\
35.07	0.01\\
35.08	0.01\\
35.09	0.01\\
35.1	0.01\\
35.11	0.01\\
35.12	0.01\\
35.13	0.01\\
35.14	0.01\\
35.15	0.01\\
35.16	0.01\\
35.17	0.01\\
35.18	0.01\\
35.19	0.01\\
35.2	0.01\\
35.21	0.01\\
35.22	0.01\\
35.23	0.01\\
35.24	0.01\\
35.25	0.01\\
35.26	0.01\\
35.27	0.01\\
35.28	0.01\\
35.29	0.01\\
35.3	0.01\\
35.31	0.01\\
35.32	0.01\\
35.33	0.01\\
35.34	0.01\\
35.35	0.01\\
35.36	0.01\\
35.37	0.01\\
35.38	0.01\\
35.39	0.01\\
35.4	0.01\\
35.41	0.01\\
35.42	0.01\\
35.43	0.01\\
35.44	0.01\\
35.45	0.01\\
35.46	0.01\\
35.47	0.01\\
35.48	0.01\\
35.49	0.01\\
35.5	0.01\\
35.51	0.01\\
35.52	0.01\\
35.53	0.01\\
35.54	0.01\\
35.55	0.01\\
35.56	0.01\\
35.57	0.01\\
35.58	0.01\\
35.59	0.01\\
35.6	0.01\\
35.61	0.01\\
35.62	0.01\\
35.63	0.01\\
35.64	0.01\\
35.65	0.01\\
35.66	0.01\\
35.67	0.01\\
35.68	0.01\\
35.69	0.01\\
35.7	0.01\\
35.71	0.01\\
35.72	0.01\\
35.73	0.01\\
35.74	0.01\\
35.75	0.01\\
35.76	0.01\\
35.77	0.01\\
35.78	0.01\\
35.79	0.01\\
35.8	0.01\\
35.81	0.01\\
35.82	0.01\\
35.83	0.01\\
35.84	0.01\\
35.85	0.01\\
35.86	0.01\\
35.87	0.01\\
35.88	0.01\\
35.89	0.01\\
35.9	0.01\\
35.91	0.01\\
35.92	0.01\\
35.93	0.01\\
35.94	0.01\\
35.95	0.01\\
35.96	0.01\\
35.97	0.01\\
35.98	0.01\\
35.99	0.01\\
36	0.01\\
36.01	0.01\\
36.02	0.01\\
36.03	0.01\\
36.04	0.01\\
36.05	0.01\\
36.06	0.01\\
36.07	0.01\\
36.08	0.01\\
36.09	0.01\\
36.1	0.01\\
36.11	0.01\\
36.12	0.01\\
36.13	0.01\\
36.14	0.01\\
36.15	0.01\\
36.16	0.01\\
36.17	0.01\\
36.18	0.01\\
36.19	0.01\\
36.2	0.01\\
36.21	0.01\\
36.22	0.01\\
36.23	0.01\\
36.24	0.01\\
36.25	0.01\\
36.26	0.01\\
36.27	0.01\\
36.28	0.01\\
36.29	0.01\\
36.3	0.01\\
36.31	0.01\\
36.32	0.01\\
36.33	0.01\\
36.34	0.01\\
36.35	0.01\\
36.36	0.01\\
36.37	0.01\\
36.38	0.01\\
36.39	0.01\\
36.4	0.01\\
36.41	0.01\\
36.42	0.01\\
36.43	0.01\\
36.44	0.01\\
36.45	0.01\\
36.46	0.01\\
36.47	0.01\\
36.48	0.01\\
36.49	0.01\\
36.5	0.01\\
36.51	0.01\\
36.52	0.01\\
36.53	0.01\\
36.54	0.01\\
36.55	0.01\\
36.56	0.01\\
36.57	0.01\\
36.58	0.01\\
36.59	0.01\\
36.6	0.01\\
36.61	0.01\\
36.62	0.01\\
36.63	0.01\\
36.64	0.01\\
36.65	0.01\\
36.66	0.01\\
36.67	0.01\\
36.68	0.01\\
36.69	0.01\\
36.7	0.01\\
36.71	0.01\\
36.72	0.01\\
36.73	0.01\\
36.74	0.01\\
36.75	0.01\\
36.76	0.01\\
36.77	0.01\\
36.78	0.01\\
36.79	0.01\\
36.8	0.01\\
36.81	0.01\\
36.82	0.01\\
36.83	0.01\\
36.84	0.01\\
36.85	0.01\\
36.86	0.01\\
36.87	0.01\\
36.88	0.01\\
36.89	0.01\\
36.9	0.01\\
36.91	0.01\\
36.92	0.01\\
36.93	0.01\\
36.94	0.01\\
36.95	0.01\\
36.96	0.01\\
36.97	0.01\\
36.98	0.01\\
36.99	0.01\\
37	0.01\\
37.01	0.01\\
37.02	0.01\\
37.03	0.01\\
37.04	0.01\\
37.05	0.01\\
37.06	0.01\\
37.07	0.01\\
37.08	0.01\\
37.09	0.01\\
37.1	0.01\\
37.11	0.01\\
37.12	0.01\\
37.13	0.01\\
37.14	0.01\\
37.15	0.01\\
37.16	0.01\\
37.17	0.01\\
37.18	0.01\\
37.19	0.01\\
37.2	0.01\\
37.21	0.01\\
37.22	0.01\\
37.23	0.01\\
37.24	0.01\\
37.25	0.01\\
37.26	0.01\\
37.27	0.01\\
37.28	0.01\\
37.29	0.01\\
37.3	0.01\\
37.31	0.01\\
37.32	0.01\\
37.33	0.01\\
37.34	0.01\\
37.35	0.01\\
37.36	0.01\\
37.37	0.01\\
37.38	0.01\\
37.39	0.01\\
37.4	0.01\\
37.41	0.01\\
37.42	0.01\\
37.43	0.01\\
37.44	0.01\\
37.45	0.01\\
37.46	0.01\\
37.47	0.01\\
37.48	0.01\\
37.49	0.01\\
37.5	0.01\\
37.51	0.01\\
37.52	0.01\\
37.53	0.01\\
37.54	0.01\\
37.55	0.01\\
37.56	0.01\\
37.57	0.01\\
37.58	0.01\\
37.59	0.01\\
37.6	0.01\\
37.61	0.01\\
37.62	0.01\\
37.63	0.01\\
37.64	0.01\\
37.65	0.01\\
37.66	0.01\\
37.67	0.01\\
37.68	0.01\\
37.69	0.01\\
37.7	0.01\\
37.71	0.01\\
37.72	0.01\\
37.73	0.01\\
37.74	0.01\\
37.75	0.01\\
37.76	0.01\\
37.77	0.01\\
37.78	0.01\\
37.79	0.01\\
37.8	0.01\\
37.81	0.01\\
37.82	0.01\\
37.83	0.01\\
37.84	0.01\\
37.85	0.01\\
37.86	0.01\\
37.87	0.01\\
37.88	0.01\\
37.89	0.01\\
37.9	0.01\\
37.91	0.01\\
37.92	0.01\\
37.93	0.01\\
37.94	0.01\\
37.95	0.01\\
37.96	0.01\\
37.97	0.01\\
37.98	0.01\\
37.99	0.01\\
38	0.01\\
38.01	0.01\\
38.02	0.01\\
38.03	0.01\\
38.04	0.01\\
38.05	0.01\\
38.06	0.01\\
38.07	0.01\\
38.08	0.01\\
38.09	0.01\\
38.1	0.01\\
38.11	0.01\\
38.12	0.01\\
38.13	0.01\\
38.14	0.01\\
38.15	0.01\\
38.16	0.01\\
38.17	0.01\\
38.18	0.01\\
38.19	0.01\\
38.2	0.01\\
38.21	0.01\\
38.22	0.01\\
38.23	0.01\\
38.24	0.01\\
38.25	0.01\\
38.26	0.01\\
38.27	0.01\\
38.28	0.01\\
38.29	0.01\\
38.3	0.01\\
38.31	0.01\\
38.32	0.01\\
38.33	0.01\\
38.34	0.01\\
38.35	0.01\\
38.36	0.01\\
38.37	0.01\\
38.38	0.01\\
38.39	0.01\\
38.4	0.01\\
38.41	0.01\\
38.42	0.01\\
38.43	0.01\\
38.44	0.01\\
38.45	0.01\\
38.46	0.01\\
38.47	0.01\\
38.48	0.01\\
38.49	0.01\\
38.5	0.01\\
38.51	0.01\\
38.52	0.01\\
38.53	0.01\\
38.54	0.01\\
38.55	0.01\\
38.56	0.01\\
38.57	0.01\\
38.58	0.01\\
38.59	0.01\\
38.6	0.01\\
38.61	0.01\\
38.62	0.01\\
38.63	0.01\\
38.64	0.01\\
38.65	0.01\\
38.66	0.01\\
38.67	0.01\\
38.68	0.01\\
38.69	0.01\\
38.7	0.01\\
38.71	0.01\\
38.72	0.01\\
38.73	0.01\\
38.74	0.01\\
38.75	0.01\\
38.76	0.01\\
38.77	0.01\\
38.78	0.01\\
38.79	0.01\\
38.8	0.01\\
38.81	0.01\\
38.82	0.01\\
38.83	0.01\\
38.84	0.01\\
38.85	0.01\\
38.86	0.01\\
38.87	0.01\\
38.88	0.01\\
38.89	0.01\\
38.9	0.01\\
38.91	0.01\\
38.92	0.01\\
38.93	0.01\\
38.94	0.01\\
38.95	0.01\\
38.96	0.01\\
38.97	0.01\\
38.98	0.01\\
38.99	0.01\\
39	0.01\\
39.01	0.01\\
39.02	0.01\\
39.03	0.01\\
39.04	0.01\\
39.05	0.01\\
39.06	0.01\\
39.07	0.01\\
39.08	0.01\\
39.09	0.01\\
39.1	0.01\\
39.11	0.01\\
39.12	0.01\\
39.13	0.01\\
39.14	0.01\\
39.15	0.01\\
39.16	0.01\\
39.17	0.01\\
39.18	0.01\\
39.19	0.01\\
39.2	0.01\\
39.21	0.01\\
39.22	0.01\\
39.23	0.01\\
39.24	0.01\\
39.25	0.01\\
39.26	0.01\\
39.27	0.01\\
39.28	0.01\\
39.29	0.01\\
39.3	0.01\\
39.31	0.01\\
39.32	0.01\\
39.33	0.01\\
39.34	0.01\\
39.35	0.01\\
39.36	0.01\\
39.37	0.01\\
39.38	0.01\\
39.39	0.01\\
39.4	0.01\\
39.41	0.01\\
39.42	0.01\\
39.43	0.01\\
39.44	0.01\\
39.45	0.01\\
39.46	0.01\\
39.47	0.01\\
39.48	0.01\\
39.49	0.01\\
39.5	0.01\\
39.51	0.01\\
39.52	0.01\\
39.53	0.01\\
39.54	0.01\\
39.55	0.01\\
39.56	0.01\\
39.57	0.01\\
39.58	0.01\\
39.59	0.01\\
39.6	0.01\\
39.61	0.01\\
39.62	0.01\\
39.63	0.01\\
39.64	0.01\\
39.65	0.01\\
39.66	0.01\\
39.67	0.01\\
39.68	0.01\\
39.69	0.01\\
39.7	0.01\\
39.71	0.01\\
39.72	0.01\\
39.73	0.01\\
39.74	0.01\\
39.75	0.01\\
39.76	0.01\\
39.77	0.01\\
39.78	0.01\\
39.79	0.01\\
39.8	0.01\\
39.81	0.01\\
39.82	0.01\\
39.83	0.01\\
39.84	0.01\\
39.85	0.01\\
39.86	0.01\\
39.87	0.01\\
39.88	0.01\\
39.89	0.01\\
39.9	0.01\\
39.91	0.01\\
39.92	0.01\\
39.93	0.01\\
39.94	0.01\\
39.95	0.01\\
39.96	0.01\\
39.97	0.01\\
39.98	0.01\\
39.99	0.01\\
40	0.01\\
40.01	0.01\\
};
\addplot [color=red,solid,forget plot]
  table[row sep=crcr]{%
40.01	0.01\\
40.02	0.01\\
40.03	0.01\\
40.04	0.01\\
40.05	0.01\\
40.06	0.01\\
40.07	0.01\\
40.08	0.01\\
40.09	0.01\\
40.1	0.01\\
40.11	0.01\\
40.12	0.01\\
40.13	0.01\\
40.14	0.01\\
40.15	0.01\\
40.16	0.01\\
40.17	0.01\\
40.18	0.01\\
40.19	0.01\\
40.2	0.01\\
40.21	0.01\\
40.22	0.01\\
40.23	0.01\\
40.24	0.01\\
40.25	0.01\\
40.26	0.01\\
40.27	0.01\\
40.28	0.01\\
40.29	0.01\\
40.3	0.01\\
40.31	0.01\\
40.32	0.01\\
40.33	0.01\\
40.34	0.01\\
40.35	0.01\\
40.36	0.01\\
40.37	0.01\\
40.38	0.01\\
40.39	0.01\\
40.4	0.01\\
40.41	0.01\\
40.42	0.01\\
40.43	0.01\\
40.44	0.01\\
40.45	0.01\\
40.46	0.01\\
40.47	0.01\\
40.48	0.01\\
40.49	0.01\\
40.5	0.01\\
40.51	0.01\\
40.52	0.01\\
40.53	0.01\\
40.54	0.01\\
40.55	0.01\\
40.56	0.01\\
40.57	0.01\\
40.58	0.01\\
40.59	0.01\\
40.6	0.01\\
40.61	0.01\\
40.62	0.01\\
40.63	0.01\\
40.64	0.01\\
40.65	0.01\\
40.66	0.01\\
40.67	0.01\\
40.68	0.01\\
40.69	0.01\\
40.7	0.01\\
40.71	0.01\\
40.72	0.01\\
40.73	0.01\\
40.74	0.01\\
40.75	0.01\\
40.76	0.01\\
40.77	0.01\\
40.78	0.01\\
40.79	0.01\\
40.8	0.01\\
40.81	0.01\\
40.82	0.01\\
40.83	0.01\\
40.84	0.01\\
40.85	0.01\\
40.86	0.01\\
40.87	0.01\\
40.88	0.01\\
40.89	0.01\\
40.9	0.01\\
40.91	0.01\\
40.92	0.01\\
40.93	0.01\\
40.94	0.01\\
40.95	0.01\\
40.96	0.01\\
40.97	0.01\\
40.98	0.01\\
40.99	0.01\\
41	0.01\\
41.01	0.01\\
41.02	0.01\\
41.03	0.01\\
41.04	0.01\\
41.05	0.01\\
41.06	0.01\\
41.07	0.01\\
41.08	0.01\\
41.09	0.01\\
41.1	0.01\\
41.11	0.01\\
41.12	0.01\\
41.13	0.01\\
41.14	0.01\\
41.15	0.01\\
41.16	0.01\\
41.17	0.01\\
41.18	0.01\\
41.19	0.01\\
41.2	0.01\\
41.21	0.01\\
41.22	0.01\\
41.23	0.01\\
41.24	0.01\\
41.25	0.01\\
41.26	0.01\\
41.27	0.01\\
41.28	0.01\\
41.29	0.01\\
41.3	0.01\\
41.31	0.01\\
41.32	0.01\\
41.33	0.01\\
41.34	0.01\\
41.35	0.01\\
41.36	0.01\\
41.37	0.01\\
41.38	0.01\\
41.39	0.01\\
41.4	0.01\\
41.41	0.01\\
41.42	0.01\\
41.43	0.01\\
41.44	0.01\\
41.45	0.01\\
41.46	0.01\\
41.47	0.01\\
41.48	0.01\\
41.49	0.01\\
41.5	0.01\\
41.51	0.01\\
41.52	0.01\\
41.53	0.01\\
41.54	0.01\\
41.55	0.01\\
41.56	0.01\\
41.57	0.01\\
41.58	0.01\\
41.59	0.01\\
41.6	0.01\\
41.61	0.01\\
41.62	0.01\\
41.63	0.01\\
41.64	0.01\\
41.65	0.01\\
41.66	0.01\\
41.67	0.01\\
41.68	0.01\\
41.69	0.01\\
41.7	0.01\\
41.71	0.01\\
41.72	0.01\\
41.73	0.01\\
41.74	0.01\\
41.75	0.01\\
41.76	0.01\\
41.77	0.01\\
41.78	0.01\\
41.79	0.01\\
41.8	0.01\\
41.81	0.01\\
41.82	0.01\\
41.83	0.01\\
41.84	0.01\\
41.85	0.01\\
41.86	0.01\\
41.87	0.01\\
41.88	0.01\\
41.89	0.01\\
41.9	0.01\\
41.91	0.01\\
41.92	0.01\\
41.93	0.01\\
41.94	0.01\\
41.95	0.01\\
41.96	0.01\\
41.97	0.01\\
41.98	0.01\\
41.99	0.01\\
42	0.01\\
42.01	0.01\\
42.02	0.01\\
42.03	0.01\\
42.04	0.01\\
42.05	0.01\\
42.06	0.01\\
42.07	0.01\\
42.08	0.01\\
42.09	0.01\\
42.1	0.01\\
42.11	0.01\\
42.12	0.01\\
42.13	0.01\\
42.14	0.01\\
42.15	0.01\\
42.16	0.01\\
42.17	0.01\\
42.18	0.01\\
42.19	0.01\\
42.2	0.01\\
42.21	0.01\\
42.22	0.01\\
42.23	0.01\\
42.24	0.01\\
42.25	0.01\\
42.26	0.01\\
42.27	0.01\\
42.28	0.01\\
42.29	0.01\\
42.3	0.01\\
42.31	0.01\\
42.32	0.01\\
42.33	0.01\\
42.34	0.01\\
42.35	0.01\\
42.36	0.01\\
42.37	0.01\\
42.38	0.01\\
42.39	0.01\\
42.4	0.01\\
42.41	0.01\\
42.42	0.01\\
42.43	0.01\\
42.44	0.01\\
42.45	0.01\\
42.46	0.01\\
42.47	0.01\\
42.48	0.01\\
42.49	0.01\\
42.5	0.01\\
42.51	0.01\\
42.52	0.01\\
42.53	0.01\\
42.54	0.01\\
42.55	0.01\\
42.56	0.01\\
42.57	0.01\\
42.58	0.01\\
42.59	0.01\\
42.6	0.01\\
42.61	0.01\\
42.62	0.01\\
42.63	0.01\\
42.64	0.01\\
42.65	0.01\\
42.66	0.01\\
42.67	0.01\\
42.68	0.01\\
42.69	0.01\\
42.7	0.01\\
42.71	0.01\\
42.72	0.01\\
42.73	0.01\\
42.74	0.01\\
42.75	0.01\\
42.76	0.01\\
42.77	0.01\\
42.78	0.01\\
42.79	0.01\\
42.8	0.01\\
42.81	0.01\\
42.82	0.01\\
42.83	0.01\\
42.84	0.01\\
42.85	0.01\\
42.86	0.01\\
42.87	0.01\\
42.88	0.01\\
42.89	0.01\\
42.9	0.01\\
42.91	0.01\\
42.92	0.01\\
42.93	0.01\\
42.94	0.01\\
42.95	0.01\\
42.96	0.01\\
42.97	0.01\\
42.98	0.01\\
42.99	0.01\\
43	0.01\\
43.01	0.01\\
43.02	0.01\\
43.03	0.01\\
43.04	0.01\\
43.05	0.01\\
43.06	0.01\\
43.07	0.01\\
43.08	0.01\\
43.09	0.01\\
43.1	0.01\\
43.11	0.01\\
43.12	0.01\\
43.13	0.01\\
43.14	0.01\\
43.15	0.01\\
43.16	0.01\\
43.17	0.01\\
43.18	0.01\\
43.19	0.01\\
43.2	0.01\\
43.21	0.01\\
43.22	0.01\\
43.23	0.01\\
43.24	0.01\\
43.25	0.01\\
43.26	0.01\\
43.27	0.01\\
43.28	0.01\\
43.29	0.01\\
43.3	0.01\\
43.31	0.01\\
43.32	0.01\\
43.33	0.01\\
43.34	0.01\\
43.35	0.01\\
43.36	0.01\\
43.37	0.01\\
43.38	0.01\\
43.39	0.01\\
43.4	0.01\\
43.41	0.01\\
43.42	0.01\\
43.43	0.01\\
43.44	0.01\\
43.45	0.01\\
43.46	0.01\\
43.47	0.01\\
43.48	0.01\\
43.49	0.01\\
43.5	0.01\\
43.51	0.01\\
43.52	0.01\\
43.53	0.01\\
43.54	0.01\\
43.55	0.01\\
43.56	0.01\\
43.57	0.01\\
43.58	0.01\\
43.59	0.01\\
43.6	0.01\\
43.61	0.01\\
43.62	0.01\\
43.63	0.01\\
43.64	0.01\\
43.65	0.01\\
43.66	0.01\\
43.67	0.01\\
43.68	0.01\\
43.69	0.01\\
43.7	0.01\\
43.71	0.01\\
43.72	0.01\\
43.73	0.01\\
43.74	0.01\\
43.75	0.01\\
43.76	0.01\\
43.77	0.01\\
43.78	0.01\\
43.79	0.01\\
43.8	0.01\\
43.81	0.01\\
43.82	0.01\\
43.83	0.01\\
43.84	0.01\\
43.85	0.01\\
43.86	0.01\\
43.87	0.01\\
43.88	0.01\\
43.89	0.01\\
43.9	0.01\\
43.91	0.01\\
43.92	0.01\\
43.93	0.01\\
43.94	0.01\\
43.95	0.01\\
43.96	0.01\\
43.97	0.01\\
43.98	0.01\\
43.99	0.01\\
44	0.01\\
44.01	0.01\\
44.02	0.01\\
44.03	0.01\\
44.04	0.01\\
44.05	0.01\\
44.06	0.01\\
44.07	0.01\\
44.08	0.01\\
44.09	0.01\\
44.1	0.01\\
44.11	0.01\\
44.12	0.01\\
44.13	0.01\\
44.14	0.01\\
44.15	0.01\\
44.16	0.01\\
44.17	0.01\\
44.18	0.01\\
44.19	0.01\\
44.2	0.01\\
44.21	0.01\\
44.22	0.01\\
44.23	0.01\\
44.24	0.01\\
44.25	0.01\\
44.26	0.01\\
44.27	0.01\\
44.28	0.01\\
44.29	0.01\\
44.3	0.01\\
44.31	0.01\\
44.32	0.01\\
44.33	0.01\\
44.34	0.01\\
44.35	0.01\\
44.36	0.01\\
44.37	0.01\\
44.38	0.01\\
44.39	0.01\\
44.4	0.01\\
44.41	0.01\\
44.42	0.01\\
44.43	0.01\\
44.44	0.01\\
44.45	0.01\\
44.46	0.01\\
44.47	0.01\\
44.48	0.01\\
44.49	0.01\\
44.5	0.01\\
44.51	0.01\\
44.52	0.01\\
44.53	0.01\\
44.54	0.01\\
44.55	0.01\\
44.56	0.01\\
44.57	0.01\\
44.58	0.01\\
44.59	0.01\\
44.6	0.01\\
44.61	0.01\\
44.62	0.01\\
44.63	0.01\\
44.64	0.01\\
44.65	0.01\\
44.66	0.01\\
44.67	0.01\\
44.68	0.01\\
44.69	0.01\\
44.7	0.01\\
44.71	0.01\\
44.72	0.01\\
44.73	0.01\\
44.74	0.01\\
44.75	0.01\\
44.76	0.01\\
44.77	0.01\\
44.78	0.01\\
44.79	0.01\\
44.8	0.01\\
44.81	0.01\\
44.82	0.01\\
44.83	0.01\\
44.84	0.01\\
44.85	0.01\\
44.86	0.01\\
44.87	0.01\\
44.88	0.01\\
44.89	0.01\\
44.9	0.01\\
44.91	0.01\\
44.92	0.01\\
44.93	0.01\\
44.94	0.01\\
44.95	0.01\\
44.96	0.01\\
44.97	0.01\\
44.98	0.01\\
44.99	0.01\\
45	0.01\\
45.01	0.01\\
45.02	0.01\\
45.03	0.01\\
45.04	0.01\\
45.05	0.01\\
45.06	0.01\\
45.07	0.01\\
45.08	0.01\\
45.09	0.01\\
45.1	0.01\\
45.11	0.01\\
45.12	0.01\\
45.13	0.01\\
45.14	0.01\\
45.15	0.01\\
45.16	0.01\\
45.17	0.01\\
45.18	0.01\\
45.19	0.01\\
45.2	0.01\\
45.21	0.01\\
45.22	0.01\\
45.23	0.01\\
45.24	0.01\\
45.25	0.01\\
45.26	0.01\\
45.27	0.01\\
45.28	0.01\\
45.29	0.01\\
45.3	0.01\\
45.31	0.01\\
45.32	0.01\\
45.33	0.01\\
45.34	0.01\\
45.35	0.01\\
45.36	0.01\\
45.37	0.01\\
45.38	0.01\\
45.39	0.01\\
45.4	0.01\\
45.41	0.01\\
45.42	0.01\\
45.43	0.01\\
45.44	0.01\\
45.45	0.01\\
45.46	0.01\\
45.47	0.01\\
45.48	0.01\\
45.49	0.01\\
45.5	0.01\\
45.51	0.01\\
45.52	0.01\\
45.53	0.01\\
45.54	0.01\\
45.55	0.01\\
45.56	0.01\\
45.57	0.01\\
45.58	0.01\\
45.59	0.01\\
45.6	0.01\\
45.61	0.01\\
45.62	0.01\\
45.63	0.01\\
45.64	0.01\\
45.65	0.01\\
45.66	0.01\\
45.67	0.01\\
45.68	0.01\\
45.69	0.01\\
45.7	0.01\\
45.71	0.01\\
45.72	0.01\\
45.73	0.01\\
45.74	0.01\\
45.75	0.01\\
45.76	0.01\\
45.77	0.01\\
45.78	0.01\\
45.79	0.01\\
45.8	0.01\\
45.81	0.01\\
45.82	0.01\\
45.83	0.01\\
45.84	0.01\\
45.85	0.01\\
45.86	0.01\\
45.87	0.01\\
45.88	0.01\\
45.89	0.01\\
45.9	0.01\\
45.91	0.01\\
45.92	0.01\\
45.93	0.01\\
45.94	0.01\\
45.95	0.01\\
45.96	0.01\\
45.97	0.01\\
45.98	0.01\\
45.99	0.01\\
46	0.01\\
46.01	0.01\\
46.02	0.01\\
46.03	0.01\\
46.04	0.01\\
46.05	0.01\\
46.06	0.01\\
46.07	0.01\\
46.08	0.01\\
46.09	0.01\\
46.1	0.01\\
46.11	0.01\\
46.12	0.01\\
46.13	0.01\\
46.14	0.01\\
46.15	0.01\\
46.16	0.01\\
46.17	0.01\\
46.18	0.01\\
46.19	0.01\\
46.2	0.01\\
46.21	0.01\\
46.22	0.01\\
46.23	0.01\\
46.24	0.01\\
46.25	0.01\\
46.26	0.01\\
46.27	0.01\\
46.28	0.01\\
46.29	0.01\\
46.3	0.01\\
46.31	0.01\\
46.32	0.01\\
46.33	0.01\\
46.34	0.01\\
46.35	0.01\\
46.36	0.01\\
46.37	0.01\\
46.38	0.01\\
46.39	0.01\\
46.4	0.01\\
46.41	0.01\\
46.42	0.01\\
46.43	0.01\\
46.44	0.01\\
46.45	0.01\\
46.46	0.01\\
46.47	0.01\\
46.48	0.01\\
46.49	0.01\\
46.5	0.01\\
46.51	0.01\\
46.52	0.01\\
46.53	0.01\\
46.54	0.01\\
46.55	0.01\\
46.56	0.01\\
46.57	0.01\\
46.58	0.01\\
46.59	0.01\\
46.6	0.01\\
46.61	0.01\\
46.62	0.01\\
46.63	0.01\\
46.64	0.01\\
46.65	0.01\\
46.66	0.01\\
46.67	0.01\\
46.68	0.01\\
46.69	0.01\\
46.7	0.01\\
46.71	0.01\\
46.72	0.01\\
46.73	0.01\\
46.74	0.01\\
46.75	0.01\\
46.76	0.01\\
46.77	0.01\\
46.78	0.01\\
46.79	0.01\\
46.8	0.01\\
46.81	0.01\\
46.82	0.01\\
46.83	0.01\\
46.84	0.01\\
46.85	0.01\\
46.86	0.01\\
46.87	0.01\\
46.88	0.01\\
46.89	0.01\\
46.9	0.01\\
46.91	0.01\\
46.92	0.01\\
46.93	0.01\\
46.94	0.01\\
46.95	0.01\\
46.96	0.01\\
46.97	0.01\\
46.98	0.01\\
46.99	0.01\\
47	0.01\\
47.01	0.01\\
47.02	0.01\\
47.03	0.01\\
47.04	0.01\\
47.05	0.01\\
47.06	0.01\\
47.07	0.01\\
47.08	0.01\\
47.09	0.01\\
47.1	0.01\\
47.11	0.01\\
47.12	0.01\\
47.13	0.01\\
47.14	0.01\\
47.15	0.01\\
47.16	0.01\\
47.17	0.01\\
47.18	0.01\\
47.19	0.01\\
47.2	0.01\\
47.21	0.01\\
47.22	0.01\\
47.23	0.01\\
47.24	0.01\\
47.25	0.01\\
47.26	0.01\\
47.27	0.01\\
47.28	0.01\\
47.29	0.01\\
47.3	0.01\\
47.31	0.01\\
47.32	0.01\\
47.33	0.01\\
47.34	0.01\\
47.35	0.01\\
47.36	0.01\\
47.37	0.01\\
47.38	0.01\\
47.39	0.01\\
47.4	0.01\\
47.41	0.01\\
47.42	0.01\\
47.43	0.01\\
47.44	0.01\\
47.45	0.01\\
47.46	0.01\\
47.47	0.01\\
47.48	0.01\\
47.49	0.01\\
47.5	0.01\\
47.51	0.01\\
47.52	0.01\\
47.53	0.01\\
47.54	0.01\\
47.55	0.01\\
47.56	0.01\\
47.57	0.01\\
47.58	0.01\\
47.59	0.01\\
47.6	0.01\\
47.61	0.01\\
47.62	0.01\\
47.63	0.01\\
47.64	0.01\\
47.65	0.01\\
47.66	0.01\\
47.67	0.01\\
47.68	0.01\\
47.69	0.01\\
47.7	0.01\\
47.71	0.01\\
47.72	0.01\\
47.73	0.01\\
47.74	0.01\\
47.75	0.01\\
47.76	0.01\\
47.77	0.01\\
47.78	0.01\\
47.79	0.01\\
47.8	0.01\\
47.81	0.01\\
47.82	0.01\\
47.83	0.01\\
47.84	0.01\\
47.85	0.01\\
47.86	0.01\\
47.87	0.01\\
47.88	0.01\\
47.89	0.01\\
47.9	0.01\\
47.91	0.01\\
47.92	0.01\\
47.93	0.01\\
47.94	0.01\\
47.95	0.01\\
47.96	0.01\\
47.97	0.01\\
47.98	0.01\\
47.99	0.01\\
48	0.01\\
48.01	0.01\\
48.02	0.01\\
48.03	0.01\\
48.04	0.01\\
48.05	0.01\\
48.06	0.01\\
48.07	0.01\\
48.08	0.01\\
48.09	0.01\\
48.1	0.01\\
48.11	0.01\\
48.12	0.01\\
48.13	0.01\\
48.14	0.01\\
48.15	0.01\\
48.16	0.01\\
48.17	0.01\\
48.18	0.01\\
48.19	0.01\\
48.2	0.01\\
48.21	0.01\\
48.22	0.01\\
48.23	0.01\\
48.24	0.01\\
48.25	0.01\\
48.26	0.01\\
48.27	0.01\\
48.28	0.01\\
48.29	0.01\\
48.3	0.01\\
48.31	0.01\\
48.32	0.01\\
48.33	0.01\\
48.34	0.01\\
48.35	0.01\\
48.36	0.01\\
48.37	0.01\\
48.38	0.01\\
48.39	0.01\\
48.4	0.01\\
48.41	0.01\\
48.42	0.01\\
48.43	0.01\\
48.44	0.01\\
48.45	0.01\\
48.46	0.01\\
48.47	0.01\\
48.48	0.01\\
48.49	0.01\\
48.5	0.01\\
48.51	0.01\\
48.52	0.01\\
48.53	0.01\\
48.54	0.01\\
48.55	0.01\\
48.56	0.01\\
48.57	0.01\\
48.58	0.01\\
48.59	0.01\\
48.6	0.01\\
48.61	0.01\\
48.62	0.01\\
48.63	0.01\\
48.64	0.01\\
48.65	0.01\\
48.66	0.01\\
48.67	0.01\\
48.68	0.01\\
48.69	0.01\\
48.7	0.01\\
48.71	0.01\\
48.72	0.01\\
48.73	0.01\\
48.74	0.01\\
48.75	0.01\\
48.76	0.01\\
48.77	0.01\\
48.78	0.01\\
48.79	0.01\\
48.8	0.01\\
48.81	0.01\\
48.82	0.01\\
48.83	0.01\\
48.84	0.01\\
48.85	0.01\\
48.86	0.01\\
48.87	0.01\\
48.88	0.01\\
48.89	0.01\\
48.9	0.01\\
48.91	0.01\\
48.92	0.01\\
48.93	0.01\\
48.94	0.01\\
48.95	0.01\\
48.96	0.01\\
48.97	0.01\\
48.98	0.01\\
48.99	0.01\\
49	0.01\\
49.01	0.01\\
49.02	0.01\\
49.03	0.01\\
49.04	0.01\\
49.05	0.01\\
49.06	0.01\\
49.07	0.01\\
49.08	0.01\\
49.09	0.01\\
49.1	0.01\\
49.11	0.01\\
49.12	0.01\\
49.13	0.01\\
49.14	0.01\\
49.15	0.01\\
49.16	0.01\\
49.17	0.01\\
49.18	0.01\\
49.19	0.01\\
49.2	0.01\\
49.21	0.01\\
49.22	0.01\\
49.23	0.01\\
49.24	0.01\\
49.25	0.01\\
49.26	0.01\\
49.27	0.01\\
49.28	0.01\\
49.29	0.01\\
49.3	0.01\\
49.31	0.01\\
49.32	0.01\\
49.33	0.01\\
49.34	0.01\\
49.35	0.01\\
49.36	0.01\\
49.37	0.01\\
49.38	0.01\\
49.39	0.01\\
49.4	0.01\\
49.41	0.01\\
49.42	0.01\\
49.43	0.01\\
49.44	0.01\\
49.45	0.01\\
49.46	0.01\\
49.47	0.01\\
49.48	0.01\\
49.49	0.01\\
49.5	0.01\\
49.51	0.01\\
49.52	0.01\\
49.53	0.01\\
49.54	0.01\\
49.55	0.01\\
49.56	0.01\\
49.57	0.01\\
49.58	0.01\\
49.59	0.01\\
49.6	0.01\\
49.61	0.01\\
49.62	0.01\\
49.63	0.01\\
49.64	0.01\\
49.65	0.01\\
49.66	0.01\\
49.67	0.01\\
49.68	0.01\\
49.69	0.01\\
49.7	0.01\\
49.71	0.01\\
49.72	0.01\\
49.73	0.01\\
49.74	0.01\\
49.75	0.01\\
49.76	0.01\\
49.77	0.01\\
49.78	0.01\\
49.79	0.01\\
49.8	0.01\\
49.81	0.01\\
49.82	0.01\\
49.83	0.01\\
49.84	0.01\\
49.85	0.01\\
49.86	0.01\\
49.87	0.01\\
49.88	0.01\\
49.89	0.01\\
49.9	0.01\\
49.91	0.01\\
49.92	0.01\\
49.93	0.01\\
49.94	0.01\\
49.95	0.01\\
49.96	0.01\\
49.97	0.01\\
49.98	0.01\\
49.99	0.01\\
50	0.01\\
50.01	0.01\\
50.02	0.01\\
50.03	0.01\\
50.04	0.01\\
50.05	0.01\\
50.06	0.01\\
50.07	0.01\\
50.08	0.01\\
50.09	0.01\\
50.1	0.01\\
50.11	0.01\\
50.12	0.01\\
50.13	0.01\\
50.14	0.01\\
50.15	0.01\\
50.16	0.01\\
50.17	0.01\\
50.18	0.01\\
50.19	0.01\\
50.2	0.01\\
50.21	0.01\\
50.22	0.01\\
50.23	0.01\\
50.24	0.01\\
50.25	0.01\\
50.26	0.01\\
50.27	0.01\\
50.28	0.01\\
50.29	0.01\\
50.3	0.01\\
50.31	0.01\\
50.32	0.01\\
50.33	0.01\\
50.34	0.01\\
50.35	0.01\\
50.36	0.01\\
50.37	0.01\\
50.38	0.01\\
50.39	0.01\\
50.4	0.01\\
50.41	0.01\\
50.42	0.01\\
50.43	0.01\\
50.44	0.01\\
50.45	0.01\\
50.46	0.01\\
50.47	0.01\\
50.48	0.01\\
50.49	0.01\\
50.5	0.01\\
50.51	0.01\\
50.52	0.01\\
50.53	0.01\\
50.54	0.01\\
50.55	0.01\\
50.56	0.01\\
50.57	0.01\\
50.58	0.01\\
50.59	0.01\\
50.6	0.01\\
50.61	0.01\\
50.62	0.01\\
50.63	0.01\\
50.64	0.01\\
50.65	0.01\\
50.66	0.01\\
50.67	0.01\\
50.68	0.01\\
50.69	0.01\\
50.7	0.01\\
50.71	0.01\\
50.72	0.01\\
50.73	0.01\\
50.74	0.01\\
50.75	0.01\\
50.76	0.01\\
50.77	0.01\\
50.78	0.01\\
50.79	0.01\\
50.8	0.01\\
50.81	0.01\\
50.82	0.01\\
50.83	0.01\\
50.84	0.01\\
50.85	0.01\\
50.86	0.01\\
50.87	0.01\\
50.88	0.01\\
50.89	0.01\\
50.9	0.01\\
50.91	0.01\\
50.92	0.01\\
50.93	0.01\\
50.94	0.01\\
50.95	0.01\\
50.96	0.01\\
50.97	0.01\\
50.98	0.01\\
50.99	0.01\\
51	0.01\\
51.01	0.01\\
51.02	0.01\\
51.03	0.01\\
51.04	0.01\\
51.05	0.01\\
51.06	0.01\\
51.07	0.01\\
51.08	0.01\\
51.09	0.01\\
51.1	0.01\\
51.11	0.01\\
51.12	0.01\\
51.13	0.01\\
51.14	0.01\\
51.15	0.01\\
51.16	0.01\\
51.17	0.01\\
51.18	0.01\\
51.19	0.01\\
51.2	0.01\\
51.21	0.01\\
51.22	0.01\\
51.23	0.01\\
51.24	0.01\\
51.25	0.01\\
51.26	0.01\\
51.27	0.01\\
51.28	0.01\\
51.29	0.01\\
51.3	0.01\\
51.31	0.01\\
51.32	0.01\\
51.33	0.01\\
51.34	0.01\\
51.35	0.01\\
51.36	0.01\\
51.37	0.01\\
51.38	0.01\\
51.39	0.01\\
51.4	0.01\\
51.41	0.01\\
51.42	0.01\\
51.43	0.01\\
51.44	0.01\\
51.45	0.01\\
51.46	0.01\\
51.47	0.01\\
51.48	0.01\\
51.49	0.01\\
51.5	0.01\\
51.51	0.01\\
51.52	0.01\\
51.53	0.01\\
51.54	0.01\\
51.55	0.01\\
51.56	0.01\\
51.57	0.01\\
51.58	0.01\\
51.59	0.01\\
51.6	0.01\\
51.61	0.01\\
51.62	0.01\\
51.63	0.01\\
51.64	0.01\\
51.65	0.01\\
51.66	0.01\\
51.67	0.01\\
51.68	0.01\\
51.69	0.01\\
51.7	0.01\\
51.71	0.01\\
51.72	0.01\\
51.73	0.01\\
51.74	0.01\\
51.75	0.01\\
51.76	0.01\\
51.77	0.01\\
51.78	0.01\\
51.79	0.01\\
51.8	0.01\\
51.81	0.01\\
51.82	0.01\\
51.83	0.01\\
51.84	0.01\\
51.85	0.01\\
51.86	0.01\\
51.87	0.01\\
51.88	0.01\\
51.89	0.01\\
51.9	0.01\\
51.91	0.01\\
51.92	0.01\\
51.93	0.01\\
51.94	0.01\\
51.95	0.01\\
51.96	0.01\\
51.97	0.01\\
51.98	0.01\\
51.99	0.01\\
52	0.01\\
52.01	0.01\\
52.02	0.01\\
52.03	0.01\\
52.04	0.01\\
52.05	0.01\\
52.06	0.01\\
52.07	0.01\\
52.08	0.01\\
52.09	0.01\\
52.1	0.01\\
52.11	0.01\\
52.12	0.01\\
52.13	0.01\\
52.14	0.01\\
52.15	0.01\\
52.16	0.01\\
52.17	0.01\\
52.18	0.01\\
52.19	0.01\\
52.2	0.01\\
52.21	0.01\\
52.22	0.01\\
52.23	0.01\\
52.24	0.01\\
52.25	0.01\\
52.26	0.01\\
52.27	0.01\\
52.28	0.01\\
52.29	0.01\\
52.3	0.01\\
52.31	0.01\\
52.32	0.01\\
52.33	0.01\\
52.34	0.01\\
52.35	0.01\\
52.36	0.01\\
52.37	0.01\\
52.38	0.01\\
52.39	0.01\\
52.4	0.01\\
52.41	0.01\\
52.42	0.01\\
52.43	0.01\\
52.44	0.01\\
52.45	0.01\\
52.46	0.01\\
52.47	0.01\\
52.48	0.01\\
52.49	0.01\\
52.5	0.01\\
52.51	0.01\\
52.52	0.01\\
52.53	0.01\\
52.54	0.01\\
52.55	0.01\\
52.56	0.01\\
52.57	0.01\\
52.58	0.01\\
52.59	0.01\\
52.6	0.01\\
52.61	0.01\\
52.62	0.01\\
52.63	0.01\\
52.64	0.01\\
52.65	0.01\\
52.66	0.01\\
52.67	0.01\\
52.68	0.01\\
52.69	0.01\\
52.7	0.01\\
52.71	0.01\\
52.72	0.01\\
52.73	0.01\\
52.74	0.01\\
52.75	0.01\\
52.76	0.01\\
52.77	0.01\\
52.78	0.01\\
52.79	0.01\\
52.8	0.01\\
52.81	0.01\\
52.82	0.01\\
52.83	0.01\\
52.84	0.01\\
52.85	0.01\\
52.86	0.01\\
52.87	0.01\\
52.88	0.01\\
52.89	0.01\\
52.9	0.01\\
52.91	0.01\\
52.92	0.01\\
52.93	0.01\\
52.94	0.01\\
52.95	0.01\\
52.96	0.01\\
52.97	0.01\\
52.98	0.01\\
52.99	0.01\\
53	0.01\\
53.01	0.01\\
53.02	0.01\\
53.03	0.01\\
53.04	0.01\\
53.05	0.01\\
53.06	0.01\\
53.07	0.01\\
53.08	0.01\\
53.09	0.01\\
53.1	0.01\\
53.11	0.01\\
53.12	0.01\\
53.13	0.01\\
53.14	0.01\\
53.15	0.01\\
53.16	0.01\\
53.17	0.01\\
53.18	0.01\\
53.19	0.01\\
53.2	0.01\\
53.21	0.01\\
53.22	0.01\\
53.23	0.01\\
53.24	0.01\\
53.25	0.01\\
53.26	0.01\\
53.27	0.01\\
53.28	0.01\\
53.29	0.01\\
53.3	0.01\\
53.31	0.01\\
53.32	0.01\\
53.33	0.01\\
53.34	0.01\\
53.35	0.01\\
53.36	0.01\\
53.37	0.01\\
53.38	0.01\\
53.39	0.01\\
53.4	0.01\\
53.41	0.01\\
53.42	0.01\\
53.43	0.01\\
53.44	0.01\\
53.45	0.01\\
53.46	0.01\\
53.47	0.01\\
53.48	0.01\\
53.49	0.01\\
53.5	0.01\\
53.51	0.01\\
53.52	0.01\\
53.53	0.01\\
53.54	0.01\\
53.55	0.01\\
53.56	0.01\\
53.57	0.01\\
53.58	0.01\\
53.59	0.01\\
53.6	0.01\\
53.61	0.01\\
53.62	0.01\\
53.63	0.01\\
53.64	0.01\\
53.65	0.01\\
53.66	0.01\\
53.67	0.01\\
53.68	0.01\\
53.69	0.01\\
53.7	0.01\\
53.71	0.01\\
53.72	0.01\\
53.73	0.01\\
53.74	0.01\\
53.75	0.01\\
53.76	0.01\\
53.77	0.01\\
53.78	0.01\\
53.79	0.01\\
53.8	0.01\\
53.81	0.01\\
53.82	0.01\\
53.83	0.01\\
53.84	0.01\\
53.85	0.01\\
53.86	0.01\\
53.87	0.01\\
53.88	0.01\\
53.89	0.01\\
53.9	0.01\\
53.91	0.01\\
53.92	0.01\\
53.93	0.01\\
53.94	0.01\\
53.95	0.01\\
53.96	0.01\\
53.97	0.01\\
53.98	0.01\\
53.99	0.01\\
54	0.01\\
54.01	0.01\\
54.02	0.01\\
54.03	0.01\\
54.04	0.01\\
54.05	0.01\\
54.06	0.01\\
54.07	0.01\\
54.08	0.01\\
54.09	0.01\\
54.1	0.01\\
54.11	0.01\\
54.12	0.01\\
54.13	0.01\\
54.14	0.01\\
54.15	0.01\\
54.16	0.01\\
54.17	0.01\\
54.18	0.01\\
54.19	0.01\\
54.2	0.01\\
54.21	0.01\\
54.22	0.01\\
54.23	0.01\\
54.24	0.01\\
54.25	0.01\\
54.26	0.01\\
54.27	0.01\\
54.28	0.01\\
54.29	0.01\\
54.3	0.01\\
54.31	0.01\\
54.32	0.01\\
54.33	0.01\\
54.34	0.01\\
54.35	0.01\\
54.36	0.01\\
54.37	0.01\\
54.38	0.01\\
54.39	0.01\\
54.4	0.01\\
54.41	0.01\\
54.42	0.01\\
54.43	0.01\\
54.44	0.01\\
54.45	0.01\\
54.46	0.01\\
54.47	0.01\\
54.48	0.01\\
54.49	0.01\\
54.5	0.01\\
54.51	0.01\\
54.52	0.01\\
54.53	0.01\\
54.54	0.01\\
54.55	0.01\\
54.56	0.01\\
54.57	0.01\\
54.58	0.01\\
54.59	0.01\\
54.6	0.01\\
54.61	0.01\\
54.62	0.01\\
54.63	0.01\\
54.64	0.01\\
54.65	0.01\\
54.66	0.01\\
54.67	0.01\\
54.68	0.01\\
54.69	0.01\\
54.7	0.01\\
54.71	0.01\\
54.72	0.01\\
54.73	0.01\\
54.74	0.01\\
54.75	0.01\\
54.76	0.01\\
54.77	0.01\\
54.78	0.01\\
54.79	0.01\\
54.8	0.01\\
54.81	0.01\\
54.82	0.01\\
54.83	0.01\\
54.84	0.01\\
54.85	0.01\\
54.86	0.01\\
54.87	0.01\\
54.88	0.01\\
54.89	0.01\\
54.9	0.01\\
54.91	0.01\\
54.92	0.01\\
54.93	0.01\\
54.94	0.01\\
54.95	0.01\\
54.96	0.01\\
54.97	0.01\\
54.98	0.01\\
54.99	0.01\\
55	0.01\\
55.01	0.01\\
55.02	0.01\\
55.03	0.01\\
55.04	0.01\\
55.05	0.01\\
55.06	0.01\\
55.07	0.01\\
55.08	0.01\\
55.09	0.01\\
55.1	0.01\\
55.11	0.01\\
55.12	0.01\\
55.13	0.01\\
55.14	0.01\\
55.15	0.01\\
55.16	0.01\\
55.17	0.01\\
55.18	0.01\\
55.19	0.01\\
55.2	0.01\\
55.21	0.01\\
55.22	0.01\\
55.23	0.01\\
55.24	0.01\\
55.25	0.01\\
55.26	0.01\\
55.27	0.01\\
55.28	0.01\\
55.29	0.01\\
55.3	0.01\\
55.31	0.01\\
55.32	0.01\\
55.33	0.01\\
55.34	0.01\\
55.35	0.01\\
55.36	0.01\\
55.37	0.01\\
55.38	0.01\\
55.39	0.01\\
55.4	0.01\\
55.41	0.01\\
55.42	0.01\\
55.43	0.01\\
55.44	0.01\\
55.45	0.01\\
55.46	0.01\\
55.47	0.01\\
55.48	0.01\\
55.49	0.01\\
55.5	0.01\\
55.51	0.01\\
55.52	0.01\\
55.53	0.01\\
55.54	0.01\\
55.55	0.01\\
55.56	0.01\\
55.57	0.01\\
55.58	0.01\\
55.59	0.01\\
55.6	0.01\\
55.61	0.01\\
55.62	0.01\\
55.63	0.01\\
55.64	0.01\\
55.65	0.01\\
55.66	0.01\\
55.67	0.01\\
55.68	0.01\\
55.69	0.01\\
55.7	0.01\\
55.71	0.01\\
55.72	0.01\\
55.73	0.01\\
55.74	0.01\\
55.75	0.01\\
55.76	0.01\\
55.77	0.01\\
55.78	0.01\\
55.79	0.01\\
55.8	0.01\\
55.81	0.01\\
55.82	0.01\\
55.83	0.01\\
55.84	0.01\\
55.85	0.01\\
55.86	0.01\\
55.87	0.01\\
55.88	0.01\\
55.89	0.01\\
55.9	0.01\\
55.91	0.01\\
55.92	0.01\\
55.93	0.01\\
55.94	0.01\\
55.95	0.01\\
55.96	0.01\\
55.97	0.01\\
55.98	0.01\\
55.99	0.01\\
56	0.01\\
56.01	0.01\\
56.02	0.01\\
56.03	0.01\\
56.04	0.01\\
56.05	0.01\\
56.06	0.01\\
56.07	0.01\\
56.08	0.01\\
56.09	0.01\\
56.1	0.01\\
56.11	0.01\\
56.12	0.01\\
56.13	0.01\\
56.14	0.01\\
56.15	0.01\\
56.16	0.01\\
56.17	0.01\\
56.18	0.01\\
56.19	0.01\\
56.2	0.01\\
56.21	0.01\\
56.22	0.01\\
56.23	0.01\\
56.24	0.01\\
56.25	0.01\\
56.26	0.01\\
56.27	0.01\\
56.28	0.01\\
56.29	0.01\\
56.3	0.01\\
56.31	0.01\\
56.32	0.01\\
56.33	0.01\\
56.34	0.01\\
56.35	0.01\\
56.36	0.01\\
56.37	0.01\\
56.38	0.01\\
56.39	0.01\\
56.4	0.01\\
56.41	0.01\\
56.42	0.01\\
56.43	0.01\\
56.44	0.01\\
56.45	0.01\\
56.46	0.01\\
56.47	0.01\\
56.48	0.01\\
56.49	0.01\\
56.5	0.01\\
56.51	0.01\\
56.52	0.01\\
56.53	0.01\\
56.54	0.01\\
56.55	0.01\\
56.56	0.01\\
56.57	0.01\\
56.58	0.01\\
56.59	0.01\\
56.6	0.01\\
56.61	0.01\\
56.62	0.01\\
56.63	0.01\\
56.64	0.01\\
56.65	0.01\\
56.66	0.01\\
56.67	0.01\\
56.68	0.01\\
56.69	0.01\\
56.7	0.01\\
56.71	0.01\\
56.72	0.01\\
56.73	0.01\\
56.74	0.01\\
56.75	0.01\\
56.76	0.01\\
56.77	0.01\\
56.78	0.01\\
56.79	0.01\\
56.8	0.01\\
56.81	0.01\\
56.82	0.01\\
56.83	0.01\\
56.84	0.01\\
56.85	0.01\\
56.86	0.01\\
56.87	0.01\\
56.88	0.01\\
56.89	0.01\\
56.9	0.01\\
56.91	0.01\\
56.92	0.01\\
56.93	0.01\\
56.94	0.01\\
56.95	0.01\\
56.96	0.01\\
56.97	0.01\\
56.98	0.01\\
56.99	0.01\\
57	0.01\\
57.01	0.01\\
57.02	0.01\\
57.03	0.01\\
57.04	0.01\\
57.05	0.01\\
57.06	0.01\\
57.07	0.01\\
57.08	0.01\\
57.09	0.01\\
57.1	0.01\\
57.11	0.01\\
57.12	0.01\\
57.13	0.01\\
57.14	0.01\\
57.15	0.01\\
57.16	0.01\\
57.17	0.01\\
57.18	0.01\\
57.19	0.01\\
57.2	0.01\\
57.21	0.01\\
57.22	0.01\\
57.23	0.01\\
57.24	0.01\\
57.25	0.01\\
57.26	0.01\\
57.27	0.01\\
57.28	0.01\\
57.29	0.01\\
57.3	0.01\\
57.31	0.01\\
57.32	0.01\\
57.33	0.01\\
57.34	0.01\\
57.35	0.01\\
57.36	0.01\\
57.37	0.01\\
57.38	0.01\\
57.39	0.01\\
57.4	0.01\\
57.41	0.01\\
57.42	0.01\\
57.43	0.01\\
57.44	0.01\\
57.45	0.01\\
57.46	0.01\\
57.47	0.01\\
57.48	0.01\\
57.49	0.01\\
57.5	0.01\\
57.51	0.01\\
57.52	0.01\\
57.53	0.01\\
57.54	0.01\\
57.55	0.01\\
57.56	0.01\\
57.57	0.01\\
57.58	0.01\\
57.59	0.01\\
57.6	0.01\\
57.61	0.01\\
57.62	0.01\\
57.63	0.01\\
57.64	0.01\\
57.65	0.01\\
57.66	0.01\\
57.67	0.01\\
57.68	0.01\\
57.69	0.01\\
57.7	0.01\\
57.71	0.01\\
57.72	0.01\\
57.73	0.01\\
57.74	0.01\\
57.75	0.01\\
57.76	0.01\\
57.77	0.01\\
57.78	0.01\\
57.79	0.01\\
57.8	0.01\\
57.81	0.01\\
57.82	0.01\\
57.83	0.01\\
57.84	0.01\\
57.85	0.01\\
57.86	0.01\\
57.87	0.01\\
57.88	0.01\\
57.89	0.01\\
57.9	0.01\\
57.91	0.01\\
57.92	0.01\\
57.93	0.01\\
57.94	0.01\\
57.95	0.01\\
57.96	0.01\\
57.97	0.01\\
57.98	0.01\\
57.99	0.01\\
58	0.01\\
58.01	0.01\\
58.02	0.01\\
58.03	0.01\\
58.04	0.01\\
58.05	0.01\\
58.06	0.01\\
58.07	0.01\\
58.08	0.01\\
58.09	0.01\\
58.1	0.01\\
58.11	0.01\\
58.12	0.01\\
58.13	0.01\\
58.14	0.01\\
58.15	0.01\\
58.16	0.01\\
58.17	0.01\\
58.18	0.01\\
58.19	0.01\\
58.2	0.01\\
58.21	0.01\\
58.22	0.01\\
58.23	0.01\\
58.24	0.01\\
58.25	0.01\\
58.26	0.01\\
58.27	0.01\\
58.28	0.01\\
58.29	0.01\\
58.3	0.01\\
58.31	0.01\\
58.32	0.01\\
58.33	0.01\\
58.34	0.01\\
58.35	0.01\\
58.36	0.01\\
58.37	0.01\\
58.38	0.01\\
58.39	0.01\\
58.4	0.01\\
58.41	0.01\\
58.42	0.01\\
58.43	0.01\\
58.44	0.01\\
58.45	0.01\\
58.46	0.01\\
58.47	0.01\\
58.48	0.01\\
58.49	0.01\\
58.5	0.01\\
58.51	0.01\\
58.52	0.01\\
58.53	0.01\\
58.54	0.01\\
58.55	0.01\\
58.56	0.01\\
58.57	0.01\\
58.58	0.01\\
58.59	0.01\\
58.6	0.01\\
58.61	0.01\\
58.62	0.01\\
58.63	0.01\\
58.64	0.01\\
58.65	0.01\\
58.66	0.01\\
58.67	0.01\\
58.68	0.01\\
58.69	0.01\\
58.7	0.01\\
58.71	0.01\\
58.72	0.01\\
58.73	0.01\\
58.74	0.01\\
58.75	0.01\\
58.76	0.01\\
58.77	0.01\\
58.78	0.01\\
58.79	0.01\\
58.8	0.01\\
58.81	0.01\\
58.82	0.01\\
58.83	0.01\\
58.84	0.01\\
58.85	0.01\\
58.86	0.01\\
58.87	0.01\\
58.88	0.01\\
58.89	0.01\\
58.9	0.01\\
58.91	0.01\\
58.92	0.01\\
58.93	0.01\\
58.94	0.01\\
58.95	0.01\\
58.96	0.01\\
58.97	0.01\\
58.98	0.01\\
58.99	0.01\\
59	0.01\\
59.01	0.01\\
59.02	0.01\\
59.03	0.01\\
59.04	0.01\\
59.05	0.01\\
59.06	0.01\\
59.07	0.01\\
59.08	0.01\\
59.09	0.01\\
59.1	0.01\\
59.11	0.01\\
59.12	0.01\\
59.13	0.01\\
59.14	0.01\\
59.15	0.01\\
59.16	0.01\\
59.17	0.01\\
59.18	0.01\\
59.19	0.01\\
59.2	0.01\\
59.21	0.01\\
59.22	0.01\\
59.23	0.01\\
59.24	0.01\\
59.25	0.01\\
59.26	0.01\\
59.27	0.01\\
59.28	0.01\\
59.29	0.01\\
59.3	0.01\\
59.31	0.01\\
59.32	0.01\\
59.33	0.01\\
59.34	0.01\\
59.35	0.01\\
59.36	0.01\\
59.37	0.01\\
59.38	0.01\\
59.39	0.01\\
59.4	0.01\\
59.41	0.01\\
59.42	0.01\\
59.43	0.01\\
59.44	0.01\\
59.45	0.01\\
59.46	0.01\\
59.47	0.01\\
59.48	0.01\\
59.49	0.01\\
59.5	0.01\\
59.51	0.01\\
59.52	0.01\\
59.53	0.01\\
59.54	0.01\\
59.55	0.01\\
59.56	0.01\\
59.57	0.01\\
59.58	0.01\\
59.59	0.01\\
59.6	0.01\\
59.61	0.01\\
59.62	0.01\\
59.63	0.01\\
59.64	0.01\\
59.65	0.01\\
59.66	0.01\\
59.67	0.01\\
59.68	0.01\\
59.69	0.01\\
59.7	0.01\\
59.71	0.01\\
59.72	0.01\\
59.73	0.01\\
59.74	0.01\\
59.75	0.01\\
59.76	0.01\\
59.77	0.01\\
59.78	0.01\\
59.79	0.01\\
59.8	0.01\\
59.81	0.01\\
59.82	0.01\\
59.83	0.01\\
59.84	0.01\\
59.85	0.01\\
59.86	0.01\\
59.87	0.01\\
59.88	0.01\\
59.89	0.01\\
59.9	0.01\\
59.91	0.01\\
59.92	0.01\\
59.93	0.01\\
59.94	0.01\\
59.95	0.01\\
59.96	0.01\\
59.97	0.01\\
59.98	0.01\\
59.99	0.01\\
60	0.01\\
60.01	0.01\\
60.02	0.01\\
60.03	0.01\\
60.04	0.01\\
60.05	0.01\\
60.06	0.01\\
60.07	0.01\\
60.08	0.01\\
60.09	0.01\\
60.1	0.01\\
60.11	0.01\\
60.12	0.01\\
60.13	0.01\\
60.14	0.01\\
60.15	0.01\\
60.16	0.01\\
60.17	0.01\\
60.18	0.01\\
60.19	0.01\\
60.2	0.01\\
60.21	0.01\\
60.22	0.01\\
60.23	0.01\\
60.24	0.01\\
60.25	0.01\\
60.26	0.01\\
60.27	0.01\\
60.28	0.01\\
60.29	0.01\\
60.3	0.01\\
60.31	0.01\\
60.32	0.01\\
60.33	0.01\\
60.34	0.01\\
60.35	0.01\\
60.36	0.01\\
60.37	0.01\\
60.38	0.01\\
60.39	0.01\\
60.4	0.01\\
60.41	0.01\\
60.42	0.01\\
60.43	0.01\\
60.44	0.01\\
60.45	0.01\\
60.46	0.01\\
60.47	0.01\\
60.48	0.01\\
60.49	0.01\\
60.5	0.01\\
60.51	0.01\\
60.52	0.01\\
60.53	0.01\\
60.54	0.01\\
60.55	0.01\\
60.56	0.01\\
60.57	0.01\\
60.58	0.01\\
60.59	0.01\\
60.6	0.01\\
60.61	0.01\\
60.62	0.01\\
60.63	0.01\\
60.64	0.01\\
60.65	0.01\\
60.66	0.01\\
60.67	0.01\\
60.68	0.01\\
60.69	0.01\\
60.7	0.01\\
60.71	0.01\\
60.72	0.01\\
60.73	0.01\\
60.74	0.01\\
60.75	0.01\\
60.76	0.01\\
60.77	0.01\\
60.78	0.01\\
60.79	0.01\\
60.8	0.01\\
60.81	0.01\\
60.82	0.01\\
60.83	0.01\\
60.84	0.01\\
60.85	0.01\\
60.86	0.01\\
60.87	0.01\\
60.88	0.01\\
60.89	0.01\\
60.9	0.01\\
60.91	0.01\\
60.92	0.01\\
60.93	0.01\\
60.94	0.01\\
60.95	0.01\\
60.96	0.01\\
60.97	0.01\\
60.98	0.01\\
60.99	0.01\\
61	0.01\\
61.01	0.01\\
61.02	0.01\\
61.03	0.01\\
61.04	0.01\\
61.05	0.01\\
61.06	0.01\\
61.07	0.01\\
61.08	0.01\\
61.09	0.01\\
61.1	0.01\\
61.11	0.01\\
61.12	0.01\\
61.13	0.01\\
61.14	0.01\\
61.15	0.01\\
61.16	0.01\\
61.17	0.01\\
61.18	0.01\\
61.19	0.01\\
61.2	0.01\\
61.21	0.01\\
61.22	0.01\\
61.23	0.01\\
61.24	0.01\\
61.25	0.01\\
61.26	0.01\\
61.27	0.01\\
61.28	0.01\\
61.29	0.01\\
61.3	0.01\\
61.31	0.01\\
61.32	0.01\\
61.33	0.01\\
61.34	0.01\\
61.35	0.01\\
61.36	0.01\\
61.37	0.01\\
61.38	0.01\\
61.39	0.01\\
61.4	0.01\\
61.41	0.01\\
61.42	0.01\\
61.43	0.01\\
61.44	0.01\\
61.45	0.01\\
61.46	0.01\\
61.47	0.01\\
61.48	0.01\\
61.49	0.01\\
61.5	0.01\\
61.51	0.01\\
61.52	0.01\\
61.53	0.01\\
61.54	0.01\\
61.55	0.01\\
61.56	0.01\\
61.57	0.01\\
61.58	0.01\\
61.59	0.01\\
61.6	0.01\\
61.61	0.01\\
61.62	0.01\\
61.63	0.01\\
61.64	0.01\\
61.65	0.01\\
61.66	0.01\\
61.67	0.01\\
61.68	0.01\\
61.69	0.01\\
61.7	0.01\\
61.71	0.01\\
61.72	0.01\\
61.73	0.01\\
61.74	0.01\\
61.75	0.01\\
61.76	0.01\\
61.77	0.01\\
61.78	0.01\\
61.79	0.01\\
61.8	0.01\\
61.81	0.01\\
61.82	0.01\\
61.83	0.01\\
61.84	0.01\\
61.85	0.01\\
61.86	0.01\\
61.87	0.01\\
61.88	0.01\\
61.89	0.01\\
61.9	0.01\\
61.91	0.01\\
61.92	0.01\\
61.93	0.01\\
61.94	0.01\\
61.95	0.01\\
61.96	0.01\\
61.97	0.01\\
61.98	0.01\\
61.99	0.01\\
62	0.01\\
62.01	0.01\\
62.02	0.01\\
62.03	0.01\\
62.04	0.01\\
62.05	0.01\\
62.06	0.01\\
62.07	0.01\\
62.08	0.01\\
62.09	0.01\\
62.1	0.01\\
62.11	0.01\\
62.12	0.01\\
62.13	0.01\\
62.14	0.01\\
62.15	0.01\\
62.16	0.01\\
62.17	0.01\\
62.18	0.01\\
62.19	0.01\\
62.2	0.01\\
62.21	0.01\\
62.22	0.01\\
62.23	0.01\\
62.24	0.01\\
62.25	0.01\\
62.26	0.01\\
62.27	0.01\\
62.28	0.01\\
62.29	0.01\\
62.3	0.01\\
62.31	0.01\\
62.32	0.01\\
62.33	0.01\\
62.34	0.01\\
62.35	0.01\\
62.36	0.01\\
62.37	0.01\\
62.38	0.01\\
62.39	0.01\\
62.4	0.01\\
62.41	0.01\\
62.42	0.01\\
62.43	0.01\\
62.44	0.01\\
62.45	0.01\\
62.46	0.01\\
62.47	0.01\\
62.48	0.01\\
62.49	0.01\\
62.5	0.01\\
62.51	0.01\\
62.52	0.01\\
62.53	0.01\\
62.54	0.01\\
62.55	0.01\\
62.56	0.01\\
62.57	0.01\\
62.58	0.01\\
62.59	0.01\\
62.6	0.01\\
62.61	0.01\\
62.62	0.01\\
62.63	0.01\\
62.64	0.01\\
62.65	0.01\\
62.66	0.01\\
62.67	0.01\\
62.68	0.01\\
62.69	0.01\\
62.7	0.01\\
62.71	0.01\\
62.72	0.01\\
62.73	0.01\\
62.74	0.01\\
62.75	0.01\\
62.76	0.01\\
62.77	0.01\\
62.78	0.01\\
62.79	0.01\\
62.8	0.01\\
62.81	0.01\\
62.82	0.01\\
62.83	0.01\\
62.84	0.01\\
62.85	0.01\\
62.86	0.01\\
62.87	0.01\\
62.88	0.01\\
62.89	0.01\\
62.9	0.01\\
62.91	0.01\\
62.92	0.01\\
62.93	0.01\\
62.94	0.01\\
62.95	0.01\\
62.96	0.01\\
62.97	0.01\\
62.98	0.01\\
62.99	0.01\\
63	0.01\\
63.01	0.01\\
63.02	0.01\\
63.03	0.01\\
63.04	0.01\\
63.05	0.01\\
63.06	0.01\\
63.07	0.01\\
63.08	0.01\\
63.09	0.01\\
63.1	0.01\\
63.11	0.01\\
63.12	0.01\\
63.13	0.01\\
63.14	0.01\\
63.15	0.01\\
63.16	0.01\\
63.17	0.01\\
63.18	0.01\\
63.19	0.01\\
63.2	0.01\\
63.21	0.01\\
63.22	0.01\\
63.23	0.01\\
63.24	0.01\\
63.25	0.01\\
63.26	0.01\\
63.27	0.01\\
63.28	0.01\\
63.29	0.01\\
63.3	0.01\\
63.31	0.01\\
63.32	0.01\\
63.33	0.01\\
63.34	0.01\\
63.35	0.01\\
63.36	0.01\\
63.37	0.01\\
63.38	0.01\\
63.39	0.01\\
63.4	0.01\\
63.41	0.01\\
63.42	0.01\\
63.43	0.01\\
63.44	0.01\\
63.45	0.01\\
63.46	0.01\\
63.47	0.01\\
63.48	0.01\\
63.49	0.01\\
63.5	0.01\\
63.51	0.01\\
63.52	0.01\\
63.53	0.01\\
63.54	0.01\\
63.55	0.01\\
63.56	0.01\\
63.57	0.01\\
63.58	0.01\\
63.59	0.01\\
63.6	0.01\\
63.61	0.01\\
63.62	0.01\\
63.63	0.01\\
63.64	0.01\\
63.65	0.01\\
63.66	0.01\\
63.67	0.01\\
63.68	0.01\\
63.69	0.01\\
63.7	0.01\\
63.71	0.01\\
63.72	0.01\\
63.73	0.01\\
63.74	0.01\\
63.75	0.01\\
63.76	0.01\\
63.77	0.01\\
63.78	0.01\\
63.79	0.01\\
63.8	0.01\\
63.81	0.01\\
63.82	0.01\\
63.83	0.01\\
63.84	0.01\\
63.85	0.01\\
63.86	0.01\\
63.87	0.01\\
63.88	0.01\\
63.89	0.01\\
63.9	0.01\\
63.91	0.01\\
63.92	0.01\\
63.93	0.01\\
63.94	0.01\\
63.95	0.01\\
63.96	0.01\\
63.97	0.01\\
63.98	0.01\\
63.99	0.01\\
64	0.01\\
64.01	0.01\\
64.02	0.01\\
64.03	0.01\\
64.04	0.01\\
64.05	0.01\\
64.06	0.01\\
64.07	0.01\\
64.08	0.01\\
64.09	0.01\\
64.1	0.01\\
64.11	0.01\\
64.12	0.01\\
64.13	0.01\\
64.14	0.01\\
64.15	0.01\\
64.16	0.01\\
64.17	0.01\\
64.18	0.01\\
64.19	0.01\\
64.2	0.01\\
64.21	0.01\\
64.22	0.01\\
64.23	0.01\\
64.24	0.01\\
64.25	0.01\\
64.26	0.01\\
64.27	0.01\\
64.28	0.01\\
64.29	0.01\\
64.3	0.01\\
64.31	0.01\\
64.32	0.01\\
64.33	0.01\\
64.34	0.01\\
64.35	0.01\\
64.36	0.01\\
64.37	0.01\\
64.38	0.01\\
64.39	0.01\\
64.4	0.01\\
64.41	0.01\\
64.42	0.01\\
64.43	0.01\\
64.44	0.01\\
64.45	0.01\\
64.46	0.01\\
64.47	0.01\\
64.48	0.01\\
64.49	0.01\\
64.5	0.01\\
64.51	0.01\\
64.52	0.01\\
64.53	0.01\\
64.54	0.01\\
64.55	0.01\\
64.56	0.01\\
64.57	0.01\\
64.58	0.01\\
64.59	0.01\\
64.6	0.01\\
64.61	0.01\\
64.62	0.01\\
64.63	0.01\\
64.64	0.01\\
64.65	0.01\\
64.66	0.01\\
64.67	0.01\\
64.68	0.01\\
64.69	0.01\\
64.7	0.01\\
64.71	0.01\\
64.72	0.01\\
64.73	0.01\\
64.74	0.01\\
64.75	0.01\\
64.76	0.01\\
64.77	0.01\\
64.78	0.01\\
64.79	0.01\\
64.8	0.01\\
64.81	0.01\\
64.82	0.01\\
64.83	0.01\\
64.84	0.01\\
64.85	0.01\\
64.86	0.01\\
64.87	0.01\\
64.88	0.01\\
64.89	0.01\\
64.9	0.01\\
64.91	0.01\\
64.92	0.01\\
64.93	0.01\\
64.94	0.01\\
64.95	0.01\\
64.96	0.01\\
64.97	0.01\\
64.98	0.01\\
64.99	0.01\\
65	0.01\\
65.01	0.01\\
65.02	0.01\\
65.03	0.01\\
65.04	0.01\\
65.05	0.01\\
65.06	0.01\\
65.07	0.01\\
65.08	0.01\\
65.09	0.01\\
65.1	0.01\\
65.11	0.01\\
65.12	0.01\\
65.13	0.01\\
65.14	0.01\\
65.15	0.01\\
65.16	0.01\\
65.17	0.01\\
65.18	0.01\\
65.19	0.01\\
65.2	0.01\\
65.21	0.01\\
65.22	0.01\\
65.23	0.01\\
65.24	0.01\\
65.25	0.01\\
65.26	0.01\\
65.27	0.01\\
65.28	0.01\\
65.29	0.01\\
65.3	0.01\\
65.31	0.01\\
65.32	0.01\\
65.33	0.01\\
65.34	0.01\\
65.35	0.01\\
65.36	0.01\\
65.37	0.01\\
65.38	0.01\\
65.39	0.01\\
65.4	0.01\\
65.41	0.01\\
65.42	0.01\\
65.43	0.01\\
65.44	0.01\\
65.45	0.01\\
65.46	0.01\\
65.47	0.01\\
65.48	0.01\\
65.49	0.01\\
65.5	0.01\\
65.51	0.01\\
65.52	0.01\\
65.53	0.01\\
65.54	0.01\\
65.55	0.01\\
65.56	0.01\\
65.57	0.01\\
65.58	0.01\\
65.59	0.01\\
65.6	0.01\\
65.61	0.01\\
65.62	0.01\\
65.63	0.01\\
65.64	0.01\\
65.65	0.01\\
65.66	0.01\\
65.67	0.01\\
65.68	0.01\\
65.69	0.01\\
65.7	0.01\\
65.71	0.01\\
65.72	0.01\\
65.73	0.01\\
65.74	0.01\\
65.75	0.01\\
65.76	0.01\\
65.77	0.01\\
65.78	0.01\\
65.79	0.01\\
65.8	0.01\\
65.81	0.01\\
65.82	0.01\\
65.83	0.01\\
65.84	0.01\\
65.85	0.01\\
65.86	0.01\\
65.87	0.01\\
65.88	0.01\\
65.89	0.01\\
65.9	0.01\\
65.91	0.01\\
65.92	0.01\\
65.93	0.01\\
65.94	0.01\\
65.95	0.01\\
65.96	0.01\\
65.97	0.01\\
65.98	0.01\\
65.99	0.01\\
66	0.01\\
66.01	0.01\\
66.02	0.01\\
66.03	0.01\\
66.04	0.01\\
66.05	0.01\\
66.06	0.01\\
66.07	0.01\\
66.08	0.01\\
66.09	0.01\\
66.1	0.01\\
66.11	0.01\\
66.12	0.01\\
66.13	0.01\\
66.14	0.01\\
66.15	0.01\\
66.16	0.01\\
66.17	0.01\\
66.18	0.01\\
66.19	0.01\\
66.2	0.01\\
66.21	0.01\\
66.22	0.01\\
66.23	0.01\\
66.24	0.01\\
66.25	0.01\\
66.26	0.01\\
66.27	0.01\\
66.28	0.01\\
66.29	0.01\\
66.3	0.01\\
66.31	0.01\\
66.32	0.01\\
66.33	0.01\\
66.34	0.01\\
66.35	0.01\\
66.36	0.01\\
66.37	0.01\\
66.38	0.01\\
66.39	0.01\\
66.4	0.01\\
66.41	0.01\\
66.42	0.01\\
66.43	0.01\\
66.44	0.01\\
66.45	0.01\\
66.46	0.01\\
66.47	0.01\\
66.48	0.01\\
66.49	0.01\\
66.5	0.01\\
66.51	0.01\\
66.52	0.01\\
66.53	0.01\\
66.54	0.01\\
66.55	0.01\\
66.56	0.01\\
66.57	0.01\\
66.58	0.01\\
66.59	0.01\\
66.6	0.01\\
66.61	0.01\\
66.62	0.01\\
66.63	0.01\\
66.64	0.01\\
66.65	0.01\\
66.66	0.01\\
66.67	0.01\\
66.68	0.01\\
66.69	0.01\\
66.7	0.01\\
66.71	0.01\\
66.72	0.01\\
66.73	0.01\\
66.74	0.01\\
66.75	0.01\\
66.76	0.01\\
66.77	0.01\\
66.78	0.01\\
66.79	0.01\\
66.8	0.01\\
66.81	0.01\\
66.82	0.01\\
66.83	0.01\\
66.84	0.01\\
66.85	0.01\\
66.86	0.01\\
66.87	0.01\\
66.88	0.01\\
66.89	0.01\\
66.9	0.01\\
66.91	0.01\\
66.92	0.01\\
66.93	0.01\\
66.94	0.01\\
66.95	0.01\\
66.96	0.01\\
66.97	0.01\\
66.98	0.01\\
66.99	0.01\\
67	0.01\\
67.01	0.01\\
67.02	0.01\\
67.03	0.01\\
67.04	0.01\\
67.05	0.01\\
67.06	0.01\\
67.07	0.01\\
67.08	0.01\\
67.09	0.01\\
67.1	0.01\\
67.11	0.01\\
67.12	0.01\\
67.13	0.01\\
67.14	0.01\\
67.15	0.01\\
67.16	0.01\\
67.17	0.01\\
67.18	0.01\\
67.19	0.01\\
67.2	0.01\\
67.21	0.01\\
67.22	0.01\\
67.23	0.01\\
67.24	0.01\\
67.25	0.01\\
67.26	0.01\\
67.27	0.01\\
67.28	0.01\\
67.29	0.01\\
67.3	0.01\\
67.31	0.01\\
67.32	0.01\\
67.33	0.01\\
67.34	0.01\\
67.35	0.01\\
67.36	0.01\\
67.37	0.01\\
67.38	0.01\\
67.39	0.01\\
67.4	0.01\\
67.41	0.01\\
67.42	0.01\\
67.43	0.01\\
67.44	0.01\\
67.45	0.01\\
67.46	0.01\\
67.47	0.01\\
67.48	0.01\\
67.49	0.01\\
67.5	0.01\\
67.51	0.01\\
67.52	0.01\\
67.53	0.01\\
67.54	0.01\\
67.55	0.01\\
67.56	0.01\\
67.57	0.01\\
67.58	0.01\\
67.59	0.01\\
67.6	0.01\\
67.61	0.01\\
67.62	0.01\\
67.63	0.01\\
67.64	0.01\\
67.65	0.01\\
67.66	0.01\\
67.67	0.01\\
67.68	0.01\\
67.69	0.01\\
67.7	0.01\\
67.71	0.01\\
67.72	0.01\\
67.73	0.01\\
67.74	0.01\\
67.75	0.01\\
67.76	0.01\\
67.77	0.01\\
67.78	0.01\\
67.79	0.01\\
67.8	0.01\\
67.81	0.01\\
67.82	0.01\\
67.83	0.01\\
67.84	0.01\\
67.85	0.01\\
67.86	0.01\\
67.87	0.01\\
67.88	0.01\\
67.89	0.01\\
67.9	0.01\\
67.91	0.01\\
67.92	0.01\\
67.93	0.01\\
67.94	0.01\\
67.95	0.01\\
67.96	0.01\\
67.97	0.01\\
67.98	0.01\\
67.99	0.01\\
68	0.01\\
68.01	0.01\\
68.02	0.01\\
68.03	0.01\\
68.04	0.01\\
68.05	0.01\\
68.06	0.01\\
68.07	0.01\\
68.08	0.01\\
68.09	0.01\\
68.1	0.01\\
68.11	0.01\\
68.12	0.01\\
68.13	0.01\\
68.14	0.01\\
68.15	0.01\\
68.16	0.01\\
68.17	0.01\\
68.18	0.01\\
68.19	0.01\\
68.2	0.01\\
68.21	0.01\\
68.22	0.01\\
68.23	0.01\\
68.24	0.01\\
68.25	0.01\\
68.26	0.01\\
68.27	0.01\\
68.28	0.01\\
68.29	0.01\\
68.3	0.01\\
68.31	0.01\\
68.32	0.01\\
68.33	0.01\\
68.34	0.01\\
68.35	0.01\\
68.36	0.01\\
68.37	0.01\\
68.38	0.01\\
68.39	0.01\\
68.4	0.01\\
68.41	0.01\\
68.42	0.01\\
68.43	0.01\\
68.44	0.01\\
68.45	0.01\\
68.46	0.01\\
68.47	0.01\\
68.48	0.01\\
68.49	0.01\\
68.5	0.01\\
68.51	0.01\\
68.52	0.01\\
68.53	0.01\\
68.54	0.01\\
68.55	0.01\\
68.56	0.01\\
68.57	0.01\\
68.58	0.01\\
68.59	0.01\\
68.6	0.01\\
68.61	0.01\\
68.62	0.01\\
68.63	0.01\\
68.64	0.01\\
68.65	0.01\\
68.66	0.01\\
68.67	0.01\\
68.68	0.01\\
68.69	0.01\\
68.7	0.01\\
68.71	0.01\\
68.72	0.01\\
68.73	0.01\\
68.74	0.01\\
68.75	0.01\\
68.76	0.01\\
68.77	0.01\\
68.78	0.01\\
68.79	0.01\\
68.8	0.01\\
68.81	0.01\\
68.82	0.01\\
68.83	0.01\\
68.84	0.01\\
68.85	0.01\\
68.86	0.01\\
68.87	0.01\\
68.88	0.01\\
68.89	0.01\\
68.9	0.01\\
68.91	0.01\\
68.92	0.01\\
68.93	0.01\\
68.94	0.01\\
68.95	0.01\\
68.96	0.01\\
68.97	0.01\\
68.98	0.01\\
68.99	0.01\\
69	0.01\\
69.01	0.01\\
69.02	0.01\\
69.03	0.01\\
69.04	0.01\\
69.05	0.01\\
69.06	0.01\\
69.07	0.01\\
69.08	0.01\\
69.09	0.01\\
69.1	0.01\\
69.11	0.01\\
69.12	0.01\\
69.13	0.01\\
69.14	0.01\\
69.15	0.01\\
69.16	0.01\\
69.17	0.01\\
69.18	0.01\\
69.19	0.01\\
69.2	0.01\\
69.21	0.01\\
69.22	0.01\\
69.23	0.01\\
69.24	0.01\\
69.25	0.01\\
69.26	0.01\\
69.27	0.01\\
69.28	0.01\\
69.29	0.01\\
69.3	0.01\\
69.31	0.01\\
69.32	0.01\\
69.33	0.01\\
69.34	0.01\\
69.35	0.01\\
69.36	0.01\\
69.37	0.01\\
69.38	0.01\\
69.39	0.01\\
69.4	0.01\\
69.41	0.01\\
69.42	0.01\\
69.43	0.01\\
69.44	0.01\\
69.45	0.01\\
69.46	0.01\\
69.47	0.01\\
69.48	0.01\\
69.49	0.01\\
69.5	0.01\\
69.51	0.01\\
69.52	0.01\\
69.53	0.01\\
69.54	0.01\\
69.55	0.01\\
69.56	0.01\\
69.57	0.01\\
69.58	0.01\\
69.59	0.01\\
69.6	0.01\\
69.61	0.01\\
69.62	0.01\\
69.63	0.01\\
69.64	0.01\\
69.65	0.01\\
69.66	0.01\\
69.67	0.01\\
69.68	0.01\\
69.69	0.01\\
69.7	0.01\\
69.71	0.01\\
69.72	0.01\\
69.73	0.01\\
69.74	0.01\\
69.75	0.01\\
69.76	0.01\\
69.77	0.01\\
69.78	0.01\\
69.79	0.01\\
69.8	0.01\\
69.81	0.01\\
69.82	0.01\\
69.83	0.01\\
69.84	0.01\\
69.85	0.01\\
69.86	0.01\\
69.87	0.01\\
69.88	0.01\\
69.89	0.01\\
69.9	0.01\\
69.91	0.01\\
69.92	0.01\\
69.93	0.01\\
69.94	0.01\\
69.95	0.01\\
69.96	0.01\\
69.97	0.01\\
69.98	0.01\\
69.99	0.01\\
70	0.01\\
70.01	0.01\\
70.02	0.01\\
70.03	0.01\\
70.04	0.01\\
70.05	0.01\\
70.06	0.01\\
70.07	0.01\\
70.08	0.01\\
70.09	0.01\\
70.1	0.01\\
70.11	0.01\\
70.12	0.01\\
70.13	0.01\\
70.14	0.01\\
70.15	0.01\\
70.16	0.01\\
70.17	0.01\\
70.18	0.01\\
70.19	0.01\\
70.2	0.01\\
70.21	0.01\\
70.22	0.01\\
70.23	0.01\\
70.24	0.01\\
70.25	0.01\\
70.26	0.01\\
70.27	0.01\\
70.28	0.01\\
70.29	0.01\\
70.3	0.01\\
70.31	0.01\\
70.32	0.01\\
70.33	0.01\\
70.34	0.01\\
70.35	0.01\\
70.36	0.01\\
70.37	0.01\\
70.38	0.01\\
70.39	0.01\\
70.4	0.01\\
70.41	0.01\\
70.42	0.01\\
70.43	0.01\\
70.44	0.01\\
70.45	0.01\\
70.46	0.01\\
70.47	0.01\\
70.48	0.01\\
70.49	0.01\\
70.5	0.01\\
70.51	0.01\\
70.52	0.01\\
70.53	0.01\\
70.54	0.01\\
70.55	0.01\\
70.56	0.01\\
70.57	0.01\\
70.58	0.01\\
70.59	0.01\\
70.6	0.01\\
70.61	0.01\\
70.62	0.01\\
70.63	0.01\\
70.64	0.01\\
70.65	0.01\\
70.66	0.01\\
70.67	0.01\\
70.68	0.01\\
70.69	0.01\\
70.7	0.01\\
70.71	0.01\\
70.72	0.01\\
70.73	0.01\\
70.74	0.01\\
70.75	0.01\\
70.76	0.01\\
70.77	0.01\\
70.78	0.01\\
70.79	0.01\\
70.8	0.01\\
70.81	0.01\\
70.82	0.01\\
70.83	0.01\\
70.84	0.01\\
70.85	0.01\\
70.86	0.01\\
70.87	0.01\\
70.88	0.01\\
70.89	0.01\\
70.9	0.01\\
70.91	0.01\\
70.92	0.01\\
70.93	0.01\\
70.94	0.01\\
70.95	0.01\\
70.96	0.01\\
70.97	0.01\\
70.98	0.01\\
70.99	0.01\\
71	0.01\\
71.01	0.01\\
71.02	0.01\\
71.03	0.01\\
71.04	0.01\\
71.05	0.01\\
71.06	0.01\\
71.07	0.01\\
71.08	0.01\\
71.09	0.01\\
71.1	0.01\\
71.11	0.01\\
71.12	0.01\\
71.13	0.01\\
71.14	0.01\\
71.15	0.01\\
71.16	0.01\\
71.17	0.01\\
71.18	0.01\\
71.19	0.01\\
71.2	0.01\\
71.21	0.01\\
71.22	0.01\\
71.23	0.01\\
71.24	0.01\\
71.25	0.01\\
71.26	0.01\\
71.27	0.01\\
71.28	0.01\\
71.29	0.01\\
71.3	0.01\\
71.31	0.01\\
71.32	0.01\\
71.33	0.01\\
71.34	0.01\\
71.35	0.01\\
71.36	0.01\\
71.37	0.01\\
71.38	0.01\\
71.39	0.01\\
71.4	0.01\\
71.41	0.01\\
71.42	0.01\\
71.43	0.01\\
71.44	0.01\\
71.45	0.01\\
71.46	0.01\\
71.47	0.01\\
71.48	0.01\\
71.49	0.01\\
71.5	0.01\\
71.51	0.01\\
71.52	0.01\\
71.53	0.01\\
71.54	0.01\\
71.55	0.01\\
71.56	0.01\\
71.57	0.01\\
71.58	0.01\\
71.59	0.01\\
71.6	0.01\\
71.61	0.01\\
71.62	0.01\\
71.63	0.01\\
71.64	0.01\\
71.65	0.01\\
71.66	0.01\\
71.67	0.01\\
71.68	0.01\\
71.69	0.01\\
71.7	0.01\\
71.71	0.01\\
71.72	0.01\\
71.73	0.01\\
71.74	0.01\\
71.75	0.01\\
71.76	0.01\\
71.77	0.01\\
71.78	0.01\\
71.79	0.01\\
71.8	0.01\\
71.81	0.01\\
71.82	0.01\\
71.83	0.01\\
71.84	0.01\\
71.85	0.01\\
71.86	0.01\\
71.87	0.01\\
71.88	0.01\\
71.89	0.01\\
71.9	0.01\\
71.91	0.01\\
71.92	0.01\\
71.93	0.01\\
71.94	0.01\\
71.95	0.01\\
71.96	0.01\\
71.97	0.01\\
71.98	0.01\\
71.99	0.01\\
72	0.01\\
72.01	0.01\\
72.02	0.01\\
72.03	0.01\\
72.04	0.01\\
72.05	0.01\\
72.06	0.01\\
72.07	0.01\\
72.08	0.01\\
72.09	0.01\\
72.1	0.01\\
72.11	0.01\\
72.12	0.01\\
72.13	0.01\\
72.14	0.01\\
72.15	0.01\\
72.16	0.01\\
72.17	0.01\\
72.18	0.01\\
72.19	0.01\\
72.2	0.01\\
72.21	0.01\\
72.22	0.01\\
72.23	0.01\\
72.24	0.01\\
72.25	0.01\\
72.26	0.01\\
72.27	0.01\\
72.28	0.01\\
72.29	0.01\\
72.3	0.01\\
72.31	0.01\\
72.32	0.01\\
72.33	0.01\\
72.34	0.01\\
72.35	0.01\\
72.36	0.01\\
72.37	0.01\\
72.38	0.01\\
72.39	0.01\\
72.4	0.01\\
72.41	0.01\\
72.42	0.01\\
72.43	0.01\\
72.44	0.01\\
72.45	0.01\\
72.46	0.01\\
72.47	0.01\\
72.48	0.01\\
72.49	0.01\\
72.5	0.01\\
72.51	0.01\\
72.52	0.01\\
72.53	0.01\\
72.54	0.01\\
72.55	0.01\\
72.56	0.01\\
72.57	0.01\\
72.58	0.01\\
72.59	0.01\\
72.6	0.01\\
72.61	0.01\\
72.62	0.01\\
72.63	0.01\\
72.64	0.01\\
72.65	0.01\\
72.66	0.01\\
72.67	0.01\\
72.68	0.01\\
72.69	0.01\\
72.7	0.01\\
72.71	0.01\\
72.72	0.01\\
72.73	0.01\\
72.74	0.01\\
72.75	0.01\\
72.76	0.01\\
72.77	0.01\\
72.78	0.01\\
72.79	0.01\\
72.8	0.01\\
72.81	0.01\\
72.82	0.01\\
72.83	0.01\\
72.84	0.01\\
72.85	0.01\\
72.86	0.01\\
72.87	0.01\\
72.88	0.01\\
72.89	0.01\\
72.9	0.01\\
72.91	0.01\\
72.92	0.01\\
72.93	0.01\\
72.94	0.01\\
72.95	0.01\\
72.96	0.01\\
72.97	0.01\\
72.98	0.01\\
72.99	0.01\\
73	0.01\\
73.01	0.01\\
73.02	0.01\\
73.03	0.01\\
73.04	0.01\\
73.05	0.01\\
73.06	0.01\\
73.07	0.01\\
73.08	0.01\\
73.09	0.01\\
73.1	0.01\\
73.11	0.01\\
73.12	0.01\\
73.13	0.01\\
73.14	0.01\\
73.15	0.01\\
73.16	0.01\\
73.17	0.01\\
73.18	0.01\\
73.19	0.01\\
73.2	0.01\\
73.21	0.01\\
73.22	0.01\\
73.23	0.01\\
73.24	0.01\\
73.25	0.01\\
73.26	0.01\\
73.27	0.01\\
73.28	0.01\\
73.29	0.01\\
73.3	0.01\\
73.31	0.01\\
73.32	0.01\\
73.33	0.01\\
73.34	0.01\\
73.35	0.01\\
73.36	0.01\\
73.37	0.01\\
73.38	0.01\\
73.39	0.01\\
73.4	0.01\\
73.41	0.01\\
73.42	0.01\\
73.43	0.01\\
73.44	0.01\\
73.45	0.01\\
73.46	0.01\\
73.47	0.01\\
73.48	0.01\\
73.49	0.01\\
73.5	0.01\\
73.51	0.01\\
73.52	0.01\\
73.53	0.01\\
73.54	0.01\\
73.55	0.01\\
73.56	0.01\\
73.57	0.01\\
73.58	0.01\\
73.59	0.01\\
73.6	0.01\\
73.61	0.01\\
73.62	0.01\\
73.63	0.01\\
73.64	0.01\\
73.65	0.01\\
73.66	0.01\\
73.67	0.01\\
73.68	0.01\\
73.69	0.01\\
73.7	0.01\\
73.71	0.01\\
73.72	0.01\\
73.73	0.01\\
73.74	0.01\\
73.75	0.01\\
73.76	0.01\\
73.77	0.01\\
73.78	0.01\\
73.79	0.01\\
73.8	0.01\\
73.81	0.01\\
73.82	0.01\\
73.83	0.01\\
73.84	0.01\\
73.85	0.01\\
73.86	0.01\\
73.87	0.01\\
73.88	0.01\\
73.89	0.01\\
73.9	0.01\\
73.91	0.01\\
73.92	0.01\\
73.93	0.01\\
73.94	0.01\\
73.95	0.01\\
73.96	0.01\\
73.97	0.01\\
73.98	0.01\\
73.99	0.01\\
74	0.01\\
74.01	0.01\\
74.02	0.01\\
74.03	0.01\\
74.04	0.01\\
74.05	0.01\\
74.06	0.01\\
74.07	0.01\\
74.08	0.01\\
74.09	0.01\\
74.1	0.01\\
74.11	0.01\\
74.12	0.01\\
74.13	0.01\\
74.14	0.01\\
74.15	0.01\\
74.16	0.01\\
74.17	0.01\\
74.18	0.01\\
74.19	0.01\\
74.2	0.01\\
74.21	0.01\\
74.22	0.01\\
74.23	0.01\\
74.24	0.01\\
74.25	0.01\\
74.26	0.01\\
74.27	0.01\\
74.28	0.01\\
74.29	0.01\\
74.3	0.01\\
74.31	0.01\\
74.32	0.01\\
74.33	0.01\\
74.34	0.01\\
74.35	0.01\\
74.36	0.01\\
74.37	0.01\\
74.38	0.01\\
74.39	0.01\\
74.4	0.01\\
74.41	0.01\\
74.42	0.01\\
74.43	0.01\\
74.44	0.01\\
74.45	0.01\\
74.46	0.01\\
74.47	0.01\\
74.48	0.01\\
74.49	0.01\\
74.5	0.01\\
74.51	0.01\\
74.52	0.01\\
74.53	0.01\\
74.54	0.01\\
74.55	0.01\\
74.56	0.01\\
74.57	0.01\\
74.58	0.01\\
74.59	0.01\\
74.6	0.01\\
74.61	0.01\\
74.62	0.01\\
74.63	0.01\\
74.64	0.01\\
74.65	0.01\\
74.66	0.01\\
74.67	0.01\\
74.68	0.01\\
74.69	0.01\\
74.7	0.01\\
74.71	0.01\\
74.72	0.01\\
74.73	0.01\\
74.74	0.01\\
74.75	0.01\\
74.76	0.01\\
74.77	0.01\\
74.78	0.01\\
74.79	0.01\\
74.8	0.01\\
74.81	0.01\\
74.82	0.01\\
74.83	0.01\\
74.84	0.01\\
74.85	0.01\\
74.86	0.01\\
74.87	0.01\\
74.88	0.01\\
74.89	0.01\\
74.9	0.01\\
74.91	0.01\\
74.92	0.01\\
74.93	0.01\\
74.94	0.01\\
74.95	0.01\\
74.96	0.01\\
74.97	0.01\\
74.98	0.01\\
74.99	0.01\\
75	0.01\\
75.01	0.01\\
75.02	0.01\\
75.03	0.01\\
75.04	0.01\\
75.05	0.01\\
75.06	0.01\\
75.07	0.01\\
75.08	0.01\\
75.09	0.01\\
75.1	0.01\\
75.11	0.01\\
75.12	0.01\\
75.13	0.01\\
75.14	0.01\\
75.15	0.01\\
75.16	0.01\\
75.17	0.01\\
75.18	0.01\\
75.19	0.01\\
75.2	0.01\\
75.21	0.01\\
75.22	0.01\\
75.23	0.01\\
75.24	0.01\\
75.25	0.01\\
75.26	0.01\\
75.27	0.01\\
75.28	0.01\\
75.29	0.01\\
75.3	0.01\\
75.31	0.01\\
75.32	0.01\\
75.33	0.01\\
75.34	0.01\\
75.35	0.01\\
75.36	0.01\\
75.37	0.01\\
75.38	0.01\\
75.39	0.01\\
75.4	0.01\\
75.41	0.01\\
75.42	0.01\\
75.43	0.01\\
75.44	0.01\\
75.45	0.01\\
75.46	0.01\\
75.47	0.01\\
75.48	0.01\\
75.49	0.01\\
75.5	0.01\\
75.51	0.01\\
75.52	0.01\\
75.53	0.01\\
75.54	0.01\\
75.55	0.01\\
75.56	0.01\\
75.57	0.01\\
75.58	0.01\\
75.59	0.01\\
75.6	0.01\\
75.61	0.01\\
75.62	0.01\\
75.63	0.01\\
75.64	0.01\\
75.65	0.01\\
75.66	0.01\\
75.67	0.01\\
75.68	0.01\\
75.69	0.01\\
75.7	0.01\\
75.71	0.01\\
75.72	0.01\\
75.73	0.01\\
75.74	0.01\\
75.75	0.01\\
75.76	0.01\\
75.77	0.01\\
75.78	0.01\\
75.79	0.01\\
75.8	0.01\\
75.81	0.01\\
75.82	0.01\\
75.83	0.01\\
75.84	0.01\\
75.85	0.01\\
75.86	0.01\\
75.87	0.01\\
75.88	0.01\\
75.89	0.01\\
75.9	0.01\\
75.91	0.01\\
75.92	0.01\\
75.93	0.01\\
75.94	0.01\\
75.95	0.01\\
75.96	0.01\\
75.97	0.01\\
75.98	0.01\\
75.99	0.01\\
76	0.01\\
76.01	0.01\\
76.02	0.01\\
76.03	0.01\\
76.04	0.01\\
76.05	0.01\\
76.06	0.01\\
76.07	0.01\\
76.08	0.01\\
76.09	0.01\\
76.1	0.01\\
76.11	0.01\\
76.12	0.01\\
76.13	0.01\\
76.14	0.01\\
76.15	0.01\\
76.16	0.01\\
76.17	0.01\\
76.18	0.01\\
76.19	0.01\\
76.2	0.01\\
76.21	0.01\\
76.22	0.01\\
76.23	0.01\\
76.24	0.01\\
76.25	0.01\\
76.26	0.01\\
76.27	0.01\\
76.28	0.01\\
76.29	0.01\\
76.3	0.01\\
76.31	0.01\\
76.32	0.01\\
76.33	0.01\\
76.34	0.01\\
76.35	0.01\\
76.36	0.01\\
76.37	0.01\\
76.38	0.01\\
76.39	0.01\\
76.4	0.01\\
76.41	0.01\\
76.42	0.01\\
76.43	0.01\\
76.44	0.01\\
76.45	0.01\\
76.46	0.01\\
76.47	0.01\\
76.48	0.01\\
76.49	0.01\\
76.5	0.01\\
76.51	0.01\\
76.52	0.01\\
76.53	0.01\\
76.54	0.01\\
76.55	0.01\\
76.56	0.01\\
76.57	0.01\\
76.58	0.01\\
76.59	0.01\\
76.6	0.01\\
76.61	0.01\\
76.62	0.01\\
76.63	0.01\\
76.64	0.01\\
76.65	0.01\\
76.66	0.01\\
76.67	0.01\\
76.68	0.01\\
76.69	0.01\\
76.7	0.01\\
76.71	0.01\\
76.72	0.01\\
76.73	0.01\\
76.74	0.01\\
76.75	0.01\\
76.76	0.01\\
76.77	0.01\\
76.78	0.01\\
76.79	0.01\\
76.8	0.01\\
76.81	0.01\\
76.82	0.01\\
76.83	0.01\\
76.84	0.01\\
76.85	0.01\\
76.86	0.01\\
76.87	0.01\\
76.88	0.01\\
76.89	0.01\\
76.9	0.01\\
76.91	0.01\\
76.92	0.01\\
76.93	0.01\\
76.94	0.01\\
76.95	0.01\\
76.96	0.01\\
76.97	0.01\\
76.98	0.01\\
76.99	0.01\\
77	0.01\\
77.01	0.01\\
77.02	0.01\\
77.03	0.01\\
77.04	0.01\\
77.05	0.01\\
77.06	0.01\\
77.07	0.01\\
77.08	0.01\\
77.09	0.01\\
77.1	0.01\\
77.11	0.01\\
77.12	0.01\\
77.13	0.01\\
77.14	0.01\\
77.15	0.01\\
77.16	0.01\\
77.17	0.01\\
77.18	0.01\\
77.19	0.01\\
77.2	0.01\\
77.21	0.01\\
77.22	0.01\\
77.23	0.01\\
77.24	0.01\\
77.25	0.01\\
77.26	0.01\\
77.27	0.01\\
77.28	0.01\\
77.29	0.01\\
77.3	0.01\\
77.31	0.01\\
77.32	0.01\\
77.33	0.01\\
77.34	0.01\\
77.35	0.01\\
77.36	0.01\\
77.37	0.01\\
77.38	0.01\\
77.39	0.01\\
77.4	0.01\\
77.41	0.01\\
77.42	0.01\\
77.43	0.01\\
77.44	0.01\\
77.45	0.01\\
77.46	0.01\\
77.47	0.01\\
77.48	0.01\\
77.49	0.01\\
77.5	0.01\\
77.51	0.01\\
77.52	0.01\\
77.53	0.01\\
77.54	0.01\\
77.55	0.01\\
77.56	0.01\\
77.57	0.01\\
77.58	0.01\\
77.59	0.01\\
77.6	0.01\\
77.61	0.01\\
77.62	0.01\\
77.63	0.01\\
77.64	0.01\\
77.65	0.01\\
77.66	0.01\\
77.67	0.01\\
77.68	0.01\\
77.69	0.01\\
77.7	0.01\\
77.71	0.01\\
77.72	0.01\\
77.73	0.01\\
77.74	0.01\\
77.75	0.01\\
77.76	0.01\\
77.77	0.01\\
77.78	0.01\\
77.79	0.01\\
77.8	0.01\\
77.81	0.01\\
77.82	0.01\\
77.83	0.01\\
77.84	0.01\\
77.85	0.01\\
77.86	0.01\\
77.87	0.01\\
77.88	0.01\\
77.89	0.01\\
77.9	0.01\\
77.91	0.01\\
77.92	0.01\\
77.93	0.01\\
77.94	0.01\\
77.95	0.01\\
77.96	0.01\\
77.97	0.01\\
77.98	0.01\\
77.99	0.01\\
78	0.01\\
78.01	0.01\\
78.02	0.01\\
78.03	0.01\\
78.04	0.01\\
78.05	0.01\\
78.06	0.01\\
78.07	0.01\\
78.08	0.01\\
78.09	0.01\\
78.1	0.01\\
78.11	0.01\\
78.12	0.01\\
78.13	0.01\\
78.14	0.01\\
78.15	0.01\\
78.16	0.01\\
78.17	0.01\\
78.18	0.01\\
78.19	0.01\\
78.2	0.01\\
78.21	0.01\\
78.22	0.01\\
78.23	0.01\\
78.24	0.01\\
78.25	0.01\\
78.26	0.01\\
78.27	0.01\\
78.28	0.01\\
78.29	0.01\\
78.3	0.01\\
78.31	0.01\\
78.32	0.01\\
78.33	0.01\\
78.34	0.01\\
78.35	0.01\\
78.36	0.01\\
78.37	0.01\\
78.38	0.01\\
78.39	0.01\\
78.4	0.01\\
78.41	0.01\\
78.42	0.01\\
78.43	0.01\\
78.44	0.01\\
78.45	0.01\\
78.46	0.01\\
78.47	0.01\\
78.48	0.01\\
78.49	0.01\\
78.5	0.01\\
78.51	0.01\\
78.52	0.01\\
78.53	0.01\\
78.54	0.01\\
78.55	0.01\\
78.56	0.01\\
78.57	0.01\\
78.58	0.01\\
78.59	0.01\\
78.6	0.01\\
78.61	0.01\\
78.62	0.01\\
78.63	0.01\\
78.64	0.01\\
78.65	0.01\\
78.66	0.01\\
78.67	0.01\\
78.68	0.01\\
78.69	0.01\\
78.7	0.01\\
78.71	0.01\\
78.72	0.01\\
78.73	0.01\\
78.74	0.01\\
78.75	0.01\\
78.76	0.01\\
78.77	0.01\\
78.78	0.01\\
78.79	0.01\\
78.8	0.01\\
78.81	0.01\\
78.82	0.01\\
78.83	0.01\\
78.84	0.01\\
78.85	0.01\\
78.86	0.01\\
78.87	0.01\\
78.88	0.01\\
78.89	0.01\\
78.9	0.01\\
78.91	0.01\\
78.92	0.01\\
78.93	0.01\\
78.94	0.01\\
78.95	0.01\\
78.96	0.01\\
78.97	0.01\\
78.98	0.01\\
78.99	0.01\\
79	0.01\\
79.01	0.01\\
79.02	0.01\\
79.03	0.01\\
79.04	0.01\\
79.05	0.01\\
79.06	0.01\\
79.07	0.01\\
79.08	0.01\\
79.09	0.01\\
79.1	0.01\\
79.11	0.01\\
79.12	0.01\\
79.13	0.01\\
79.14	0.01\\
79.15	0.01\\
79.16	0.01\\
79.17	0.01\\
79.18	0.01\\
79.19	0.01\\
79.2	0.01\\
79.21	0.01\\
79.22	0.01\\
79.23	0.01\\
79.24	0.01\\
79.25	0.01\\
79.26	0.01\\
79.27	0.01\\
79.28	0.01\\
79.29	0.01\\
79.3	0.01\\
79.31	0.01\\
79.32	0.01\\
79.33	0.01\\
79.34	0.01\\
79.35	0.01\\
79.36	0.01\\
79.37	0.01\\
79.38	0.01\\
79.39	0.01\\
79.4	0.01\\
79.41	0.01\\
79.42	0.01\\
79.43	0.01\\
79.44	0.01\\
79.45	0.01\\
79.46	0.01\\
79.47	0.01\\
79.48	0.01\\
79.49	0.01\\
79.5	0.01\\
79.51	0.01\\
79.52	0.01\\
79.53	0.01\\
79.54	0.01\\
79.55	0.01\\
79.56	0.01\\
79.57	0.01\\
79.58	0.01\\
79.59	0.01\\
79.6	0.01\\
79.61	0.01\\
79.62	0.01\\
79.63	0.01\\
79.64	0.01\\
79.65	0.01\\
79.66	0.01\\
79.67	0.01\\
79.68	0.01\\
79.69	0.01\\
79.7	0.01\\
79.71	0.01\\
79.72	0.01\\
79.73	0.01\\
79.74	0.01\\
79.75	0.01\\
79.76	0.01\\
79.77	0.01\\
79.78	0.01\\
79.79	0.01\\
79.8	0.01\\
79.81	0.01\\
79.82	0.01\\
79.83	0.01\\
79.84	0.01\\
79.85	0.01\\
79.86	0.01\\
79.87	0.01\\
79.88	0.01\\
79.89	0.01\\
79.9	0.01\\
79.91	0.01\\
79.92	0.01\\
79.93	0.01\\
79.94	0.01\\
79.95	0.01\\
79.96	0.01\\
79.97	0.01\\
79.98	0.01\\
79.99	0.01\\
80	0.01\\
80.01	0.01\\
};
\addplot [color=red,solid]
  table[row sep=crcr]{%
80.01	0.01\\
80.02	0.01\\
80.03	0.01\\
80.04	0.01\\
80.05	0.01\\
80.06	0.01\\
80.07	0.01\\
80.08	0.01\\
80.09	0.01\\
80.1	0.01\\
80.11	0.01\\
80.12	0.01\\
80.13	0.01\\
80.14	0.01\\
80.15	0.01\\
80.16	0.01\\
80.17	0.01\\
80.18	0.01\\
80.19	0.01\\
80.2	0.01\\
80.21	0.01\\
80.22	0.01\\
80.23	0.01\\
80.24	0.01\\
80.25	0.01\\
80.26	0.01\\
80.27	0.01\\
80.28	0.01\\
80.29	0.01\\
80.3	0.01\\
80.31	0.01\\
80.32	0.01\\
80.33	0.01\\
80.34	0.01\\
80.35	0.01\\
80.36	0.01\\
80.37	0.01\\
80.38	0.01\\
80.39	0.01\\
80.4	0.01\\
80.41	0.01\\
80.42	0.01\\
80.43	0.01\\
80.44	0.01\\
80.45	0.01\\
80.46	0.01\\
80.47	0.01\\
80.48	0.01\\
80.49	0.01\\
80.5	0.01\\
80.51	0.01\\
80.52	0.01\\
80.53	0.01\\
80.54	0.01\\
80.55	0.01\\
80.56	0.01\\
80.57	0.01\\
80.58	0.01\\
80.59	0.01\\
80.6	0.01\\
80.61	0.01\\
80.62	0.01\\
80.63	0.01\\
80.64	0.01\\
80.65	0.01\\
80.66	0.01\\
80.67	0.01\\
80.68	0.01\\
80.69	0.01\\
80.7	0.01\\
80.71	0.01\\
80.72	0.01\\
80.73	0.01\\
80.74	0.01\\
80.75	0.01\\
80.76	0.01\\
80.77	0.01\\
80.78	0.01\\
80.79	0.01\\
80.8	0.01\\
80.81	0.01\\
80.82	0.01\\
80.83	0.01\\
80.84	0.01\\
80.85	0.01\\
80.86	0.01\\
80.87	0.01\\
80.88	0.01\\
80.89	0.01\\
80.9	0.01\\
80.91	0.01\\
80.92	0.01\\
80.93	0.01\\
80.94	0.01\\
80.95	0.01\\
80.96	0.01\\
80.97	0.01\\
80.98	0.01\\
80.99	0.01\\
81	0.01\\
81.01	0.01\\
81.02	0.01\\
81.03	0.01\\
81.04	0.01\\
81.05	0.01\\
81.06	0.01\\
81.07	0.01\\
81.08	0.01\\
81.09	0.01\\
81.1	0.01\\
81.11	0.01\\
81.12	0.01\\
81.13	0.01\\
81.14	0.01\\
81.15	0.01\\
81.16	0.01\\
81.17	0.01\\
81.18	0.01\\
81.19	0.01\\
81.2	0.01\\
81.21	0.01\\
81.22	0.01\\
81.23	0.01\\
81.24	0.01\\
81.25	0.01\\
81.26	0.01\\
81.27	0.01\\
81.28	0.01\\
81.29	0.01\\
81.3	0.01\\
81.31	0.01\\
81.32	0.01\\
81.33	0.01\\
81.34	0.01\\
81.35	0.01\\
81.36	0.01\\
81.37	0.01\\
81.38	0.01\\
81.39	0.01\\
81.4	0.01\\
81.41	0.01\\
81.42	0.01\\
81.43	0.01\\
81.44	0.01\\
81.45	0.01\\
81.46	0.01\\
81.47	0.01\\
81.48	0.01\\
81.49	0.01\\
81.5	0.01\\
81.51	0.01\\
81.52	0.01\\
81.53	0.01\\
81.54	0.01\\
81.55	0.01\\
81.56	0.01\\
81.57	0.01\\
81.58	0.01\\
81.59	0.01\\
81.6	0.01\\
81.61	0.01\\
81.62	0.01\\
81.63	0.01\\
81.64	0.01\\
81.65	0.01\\
81.66	0.01\\
81.67	0.01\\
81.68	0.01\\
81.69	0.01\\
81.7	0.01\\
81.71	0.01\\
81.72	0.01\\
81.73	0.01\\
81.74	0.01\\
81.75	0.01\\
81.76	0.01\\
81.77	0.01\\
81.78	0.01\\
81.79	0.01\\
81.8	0.01\\
81.81	0.01\\
81.82	0.01\\
81.83	0.01\\
81.84	0.01\\
81.85	0.01\\
81.86	0.01\\
81.87	0.01\\
81.88	0.01\\
81.89	0.01\\
81.9	0.01\\
81.91	0.01\\
81.92	0.01\\
81.93	0.01\\
81.94	0.01\\
81.95	0.01\\
81.96	0.01\\
81.97	0.01\\
81.98	0.01\\
81.99	0.01\\
82	0.01\\
82.01	0.01\\
82.02	0.01\\
82.03	0.01\\
82.04	0.01\\
82.05	0.01\\
82.06	0.01\\
82.07	0.01\\
82.08	0.01\\
82.09	0.01\\
82.1	0.01\\
82.11	0.01\\
82.12	0.01\\
82.13	0.01\\
82.14	0.01\\
82.15	0.01\\
82.16	0.01\\
82.17	0.01\\
82.18	0.01\\
82.19	0.01\\
82.2	0.01\\
82.21	0.01\\
82.22	0.01\\
82.23	0.01\\
82.24	0.01\\
82.25	0.01\\
82.26	0.01\\
82.27	0.01\\
82.28	0.01\\
82.29	0.01\\
82.3	0.01\\
82.31	0.01\\
82.32	0.01\\
82.33	0.01\\
82.34	0.01\\
82.35	0.01\\
82.36	0.01\\
82.37	0.01\\
82.38	0.01\\
82.39	0.01\\
82.4	0.01\\
82.41	0.01\\
82.42	0.01\\
82.43	0.01\\
82.44	0.01\\
82.45	0.01\\
82.46	0.01\\
82.47	0.01\\
82.48	0.01\\
82.49	0.01\\
82.5	0.01\\
82.51	0.01\\
82.52	0.01\\
82.53	0.01\\
82.54	0.01\\
82.55	0.01\\
82.56	0.01\\
82.57	0.01\\
82.58	0.01\\
82.59	0.01\\
82.6	0.01\\
82.61	0.01\\
82.62	0.01\\
82.63	0.01\\
82.64	0.01\\
82.65	0.01\\
82.66	0.01\\
82.67	0.01\\
82.68	0.01\\
82.69	0.01\\
82.7	0.01\\
82.71	0.01\\
82.72	0.01\\
82.73	0.01\\
82.74	0.01\\
82.75	0.01\\
82.76	0.01\\
82.77	0.01\\
82.78	0.01\\
82.79	0.01\\
82.8	0.01\\
82.81	0.01\\
82.82	0.01\\
82.83	0.01\\
82.84	0.01\\
82.85	0.01\\
82.86	0.01\\
82.87	0.01\\
82.88	0.01\\
82.89	0.01\\
82.9	0.01\\
82.91	0.01\\
82.92	0.01\\
82.93	0.01\\
82.94	0.01\\
82.95	0.01\\
82.96	0.01\\
82.97	0.01\\
82.98	0.01\\
82.99	0.01\\
83	0.01\\
83.01	0.01\\
83.02	0.01\\
83.03	0.01\\
83.04	0.01\\
83.05	0.01\\
83.06	0.01\\
83.07	0.01\\
83.08	0.01\\
83.09	0.01\\
83.1	0.01\\
83.11	0.01\\
83.12	0.01\\
83.13	0.01\\
83.14	0.01\\
83.15	0.01\\
83.16	0.01\\
83.17	0.01\\
83.18	0.01\\
83.19	0.01\\
83.2	0.01\\
83.21	0.01\\
83.22	0.01\\
83.23	0.01\\
83.24	0.01\\
83.25	0.01\\
83.26	0.01\\
83.27	0.01\\
83.28	0.01\\
83.29	0.01\\
83.3	0.01\\
83.31	0.01\\
83.32	0.01\\
83.33	0.01\\
83.34	0.01\\
83.35	0.01\\
83.36	0.01\\
83.37	0.01\\
83.38	0.01\\
83.39	0.01\\
83.4	0.01\\
83.41	0.01\\
83.42	0.01\\
83.43	0.01\\
83.44	0.01\\
83.45	0.01\\
83.46	0.01\\
83.47	0.01\\
83.48	0.01\\
83.49	0.01\\
83.5	0.01\\
83.51	0.01\\
83.52	0.01\\
83.53	0.01\\
83.54	0.01\\
83.55	0.01\\
83.56	0.01\\
83.57	0.01\\
83.58	0.01\\
83.59	0.01\\
83.6	0.01\\
83.61	0.01\\
83.62	0.01\\
83.63	0.01\\
83.64	0.01\\
83.65	0.01\\
83.66	0.01\\
83.67	0.01\\
83.68	0.01\\
83.69	0.01\\
83.7	0.01\\
83.71	0.01\\
83.72	0.01\\
83.73	0.01\\
83.74	0.01\\
83.75	0.01\\
83.76	0.01\\
83.77	0.01\\
83.78	0.01\\
83.79	0.01\\
83.8	0.01\\
83.81	0.01\\
83.82	0.01\\
83.83	0.01\\
83.84	0.01\\
83.85	0.01\\
83.86	0.01\\
83.87	0.01\\
83.88	0.01\\
83.89	0.01\\
83.9	0.01\\
83.91	0.01\\
83.92	0.01\\
83.93	0.01\\
83.94	0.01\\
83.95	0.01\\
83.96	0.01\\
83.97	0.01\\
83.98	0.01\\
83.99	0.01\\
84	0.01\\
84.01	0.01\\
84.02	0.01\\
84.03	0.01\\
84.04	0.01\\
84.05	0.01\\
84.06	0.01\\
84.07	0.01\\
84.08	0.01\\
84.09	0.01\\
84.1	0.01\\
84.11	0.01\\
84.12	0.01\\
84.13	0.01\\
84.14	0.01\\
84.15	0.01\\
84.16	0.01\\
84.17	0.01\\
84.18	0.01\\
84.19	0.01\\
84.2	0.01\\
84.21	0.01\\
84.22	0.01\\
84.23	0.01\\
84.24	0.01\\
84.25	0.01\\
84.26	0.01\\
84.27	0.01\\
84.28	0.01\\
84.29	0.01\\
84.3	0.01\\
84.31	0.01\\
84.32	0.01\\
84.33	0.01\\
84.34	0.01\\
84.35	0.01\\
84.36	0.01\\
84.37	0.01\\
84.38	0.01\\
84.39	0.01\\
84.4	0.01\\
84.41	0.01\\
84.42	0.01\\
84.43	0.01\\
84.44	0.01\\
84.45	0.01\\
84.46	0.01\\
84.47	0.01\\
84.48	0.01\\
84.49	0.01\\
84.5	0.01\\
84.51	0.01\\
84.52	0.01\\
84.53	0.01\\
84.54	0.01\\
84.55	0.01\\
84.56	0.01\\
84.57	0.01\\
84.58	0.01\\
84.59	0.01\\
84.6	0.01\\
84.61	0.01\\
84.62	0.01\\
84.63	0.01\\
84.64	0.01\\
84.65	0.01\\
84.66	0.01\\
84.67	0.01\\
84.68	0.01\\
84.69	0.01\\
84.7	0.01\\
84.71	0.01\\
84.72	0.01\\
84.73	0.01\\
84.74	0.01\\
84.75	0.01\\
84.76	0.01\\
84.77	0.01\\
84.78	0.01\\
84.79	0.01\\
84.8	0.01\\
84.81	0.01\\
84.82	0.01\\
84.83	0.01\\
84.84	0.01\\
84.85	0.01\\
84.86	0.01\\
84.87	0.01\\
84.88	0.01\\
84.89	0.01\\
84.9	0.01\\
84.91	0.01\\
84.92	0.01\\
84.93	0.01\\
84.94	0.01\\
84.95	0.01\\
84.96	0.01\\
84.97	0.01\\
84.98	0.01\\
84.99	0.01\\
85	0.01\\
85.01	0.01\\
85.02	0.01\\
85.03	0.01\\
85.04	0.01\\
85.05	0.01\\
85.06	0.01\\
85.07	0.01\\
85.08	0.01\\
85.09	0.01\\
85.1	0.01\\
85.11	0.01\\
85.12	0.01\\
85.13	0.01\\
85.14	0.01\\
85.15	0.01\\
85.16	0.01\\
85.17	0.01\\
85.18	0.01\\
85.19	0.01\\
85.2	0.01\\
85.21	0.01\\
85.22	0.01\\
85.23	0.01\\
85.24	0.01\\
85.25	0.01\\
85.26	0.01\\
85.27	0.01\\
85.28	0.01\\
85.29	0.01\\
85.3	0.01\\
85.31	0.01\\
85.32	0.01\\
85.33	0.01\\
85.34	0.01\\
85.35	0.01\\
85.36	0.01\\
85.37	0.01\\
85.38	0.01\\
85.39	0.01\\
85.4	0.01\\
85.41	0.01\\
85.42	0.01\\
85.43	0.01\\
85.44	0.01\\
85.45	0.01\\
85.46	0.01\\
85.47	0.01\\
85.48	0.01\\
85.49	0.01\\
85.5	0.01\\
85.51	0.01\\
85.52	0.01\\
85.53	0.01\\
85.54	0.01\\
85.55	0.01\\
85.56	0.01\\
85.57	0.01\\
85.58	0.01\\
85.59	0.01\\
85.6	0.01\\
85.61	0.01\\
85.62	0.01\\
85.63	0.01\\
85.64	0.01\\
85.65	0.01\\
85.66	0.01\\
85.67	0.01\\
85.68	0.01\\
85.69	0.01\\
85.7	0.01\\
85.71	0.01\\
85.72	0.01\\
85.73	0.01\\
85.74	0.01\\
85.75	0.01\\
85.76	0.01\\
85.77	0.01\\
85.78	0.01\\
85.79	0.01\\
85.8	0.01\\
85.81	0.01\\
85.82	0.01\\
85.83	0.01\\
85.84	0.01\\
85.85	0.01\\
85.86	0.01\\
85.87	0.01\\
85.88	0.01\\
85.89	0.01\\
85.9	0.01\\
85.91	0.01\\
85.92	0.01\\
85.93	0.01\\
85.94	0.01\\
85.95	0.01\\
85.96	0.01\\
85.97	0.01\\
85.98	0.01\\
85.99	0.01\\
86	0.01\\
86.01	0.01\\
86.02	0.01\\
86.03	0.01\\
86.04	0.01\\
86.05	0.01\\
86.06	0.01\\
86.07	0.01\\
86.08	0.01\\
86.09	0.01\\
86.1	0.01\\
86.11	0.01\\
86.12	0.01\\
86.13	0.01\\
86.14	0.01\\
86.15	0.01\\
86.16	0.01\\
86.17	0.01\\
86.18	0.01\\
86.19	0.01\\
86.2	0.01\\
86.21	0.01\\
86.22	0.01\\
86.23	0.01\\
86.24	0.01\\
86.25	0.01\\
86.26	0.01\\
86.27	0.01\\
86.28	0.01\\
86.29	0.01\\
86.3	0.01\\
86.31	0.01\\
86.32	0.01\\
86.33	0.01\\
86.34	0.01\\
86.35	0.01\\
86.36	0.01\\
86.37	0.01\\
86.38	0.01\\
86.39	0.01\\
86.4	0.01\\
86.41	0.01\\
86.42	0.01\\
86.43	0.01\\
86.44	0.01\\
86.45	0.01\\
86.46	0.01\\
86.47	0.01\\
86.48	0.01\\
86.49	0.01\\
86.5	0.01\\
86.51	0.01\\
86.52	0.01\\
86.53	0.01\\
86.54	0.01\\
86.55	0.01\\
86.56	0.01\\
86.57	0.01\\
86.58	0.01\\
86.59	0.01\\
86.6	0.01\\
86.61	0.01\\
86.62	0.01\\
86.63	0.01\\
86.64	0.01\\
86.65	0.01\\
86.66	0.01\\
86.67	0.01\\
86.68	0.01\\
86.69	0.01\\
86.7	0.01\\
86.71	0.01\\
86.72	0.01\\
86.73	0.01\\
86.74	0.01\\
86.75	0.01\\
86.76	0.01\\
86.77	0.01\\
86.78	0.01\\
86.79	0.01\\
86.8	0.01\\
86.81	0.01\\
86.82	0.01\\
86.83	0.01\\
86.84	0.01\\
86.85	0.01\\
86.86	0.01\\
86.87	0.01\\
86.88	0.01\\
86.89	0.01\\
86.9	0.01\\
86.91	0.01\\
86.92	0.01\\
86.93	0.01\\
86.94	0.01\\
86.95	0.01\\
86.96	0.01\\
86.97	0.01\\
86.98	0.01\\
86.99	0.01\\
87	0.01\\
87.01	0.01\\
87.02	0.01\\
87.03	0.01\\
87.04	0.01\\
87.05	0.01\\
87.06	0.01\\
87.07	0.01\\
87.08	0.01\\
87.09	0.01\\
87.1	0.01\\
87.11	0.01\\
87.12	0.01\\
87.13	0.01\\
87.14	0.01\\
87.15	0.01\\
87.16	0.01\\
87.17	0.01\\
87.18	0.01\\
87.19	0.01\\
87.2	0.01\\
87.21	0.01\\
87.22	0.01\\
87.23	0.01\\
87.24	0.01\\
87.25	0.01\\
87.26	0.01\\
87.27	0.01\\
87.28	0.01\\
87.29	0.01\\
87.3	0.01\\
87.31	0.01\\
87.32	0.01\\
87.33	0.01\\
87.34	0.01\\
87.35	0.01\\
87.36	0.01\\
87.37	0.01\\
87.38	0.01\\
87.39	0.01\\
87.4	0.01\\
87.41	0.01\\
87.42	0.01\\
87.43	0.01\\
87.44	0.01\\
87.45	0.01\\
87.46	0.01\\
87.47	0.01\\
87.48	0.01\\
87.49	0.01\\
87.5	0.01\\
87.51	0.01\\
87.52	0.01\\
87.53	0.01\\
87.54	0.01\\
87.55	0.01\\
87.56	0.01\\
87.57	0.01\\
87.58	0.01\\
87.59	0.01\\
87.6	0.01\\
87.61	0.01\\
87.62	0.01\\
87.63	0.01\\
87.64	0.01\\
87.65	0.01\\
87.66	0.01\\
87.67	0.01\\
87.68	0.01\\
87.69	0.01\\
87.7	0.01\\
87.71	0.01\\
87.72	0.01\\
87.73	0.01\\
87.74	0.01\\
87.75	0.01\\
87.76	0.01\\
87.77	0.01\\
87.78	0.01\\
87.79	0.01\\
87.8	0.01\\
87.81	0.01\\
87.82	0.01\\
87.83	0.01\\
87.84	0.01\\
87.85	0.01\\
87.86	0.01\\
87.87	0.01\\
87.88	0.01\\
87.89	0.01\\
87.9	0.01\\
87.91	0.01\\
87.92	0.01\\
87.93	0.01\\
87.94	0.01\\
87.95	0.01\\
87.96	0.01\\
87.97	0.01\\
87.98	0.01\\
87.99	0.01\\
88	0.01\\
88.01	0.01\\
88.02	0.01\\
88.03	0.01\\
88.04	0.01\\
88.05	0.01\\
88.06	0.01\\
88.07	0.01\\
88.08	0.01\\
88.09	0.01\\
88.1	0.01\\
88.11	0.01\\
88.12	0.01\\
88.13	0.01\\
88.14	0.01\\
88.15	0.01\\
88.16	0.01\\
88.17	0.01\\
88.18	0.01\\
88.19	0.01\\
88.2	0.01\\
88.21	0.01\\
88.22	0.01\\
88.23	0.01\\
88.24	0.01\\
88.25	0.01\\
88.26	0.01\\
88.27	0.01\\
88.28	0.01\\
88.29	0.01\\
88.3	0.01\\
88.31	0.01\\
88.32	0.01\\
88.33	0.01\\
88.34	0.01\\
88.35	0.01\\
88.36	0.01\\
88.37	0.01\\
88.38	0.01\\
88.39	0.01\\
88.4	0.01\\
88.41	0.01\\
88.42	0.01\\
88.43	0.01\\
88.44	0.01\\
88.45	0.01\\
88.46	0.01\\
88.47	0.01\\
88.48	0.01\\
88.49	0.01\\
88.5	0.01\\
88.51	0.01\\
88.52	0.01\\
88.53	0.01\\
88.54	0.01\\
88.55	0.01\\
88.56	0.01\\
88.57	0.01\\
88.58	0.01\\
88.59	0.01\\
88.6	0.01\\
88.61	0.01\\
88.62	0.01\\
88.63	0.01\\
88.64	0.01\\
88.65	0.01\\
88.66	0.01\\
88.67	0.01\\
88.68	0.01\\
88.69	0.01\\
88.7	0.01\\
88.71	0.01\\
88.72	0.01\\
88.73	0.01\\
88.74	0.01\\
88.75	0.01\\
88.76	0.01\\
88.77	0.01\\
88.78	0.01\\
88.79	0.01\\
88.8	0.01\\
88.81	0.01\\
88.82	0.01\\
88.83	0.01\\
88.84	0.01\\
88.85	0.01\\
88.86	0.01\\
88.87	0.01\\
88.88	0.01\\
88.89	0.01\\
88.9	0.01\\
88.91	0.01\\
88.92	0.01\\
88.93	0.01\\
88.94	0.01\\
88.95	0.01\\
88.96	0.01\\
88.97	0.01\\
88.98	0.01\\
88.99	0.01\\
89	0.01\\
89.01	0.01\\
89.02	0.01\\
89.03	0.01\\
89.04	0.01\\
89.05	0.01\\
89.06	0.01\\
89.07	0.01\\
89.08	0.01\\
89.09	0.01\\
89.1	0.01\\
89.11	0.01\\
89.12	0.01\\
89.13	0.01\\
89.14	0.01\\
89.15	0.01\\
89.16	0.01\\
89.17	0.01\\
89.18	0.01\\
89.19	0.01\\
89.2	0.01\\
89.21	0.01\\
89.22	0.01\\
89.23	0.01\\
89.24	0.01\\
89.25	0.01\\
89.26	0.01\\
89.27	0.01\\
89.28	0.01\\
89.29	0.01\\
89.3	0.01\\
89.31	0.01\\
89.32	0.01\\
89.33	0.01\\
89.34	0.01\\
89.35	0.01\\
89.36	0.01\\
89.37	0.01\\
89.38	0.01\\
89.39	0.01\\
89.4	0.01\\
89.41	0.01\\
89.42	0.01\\
89.43	0.01\\
89.44	0.01\\
89.45	0.01\\
89.46	0.01\\
89.47	0.01\\
89.48	0.01\\
89.49	0.01\\
89.5	0.01\\
89.51	0.01\\
89.52	0.01\\
89.53	0.01\\
89.54	0.01\\
89.55	0.01\\
89.56	0.01\\
89.57	0.01\\
89.58	0.01\\
89.59	0.01\\
89.6	0.01\\
89.61	0.01\\
89.62	0.01\\
89.63	0.01\\
89.64	0.01\\
89.65	0.01\\
89.66	0.01\\
89.67	0.01\\
89.68	0.01\\
89.69	0.01\\
89.7	0.01\\
89.71	0.01\\
89.72	0.01\\
89.73	0.01\\
89.74	0.01\\
89.75	0.01\\
89.76	0.01\\
89.77	0.01\\
89.78	0.01\\
89.79	0.01\\
89.8	0.01\\
89.81	0.01\\
89.82	0.01\\
89.83	0.01\\
89.84	0.01\\
89.85	0.01\\
89.86	0.01\\
89.87	0.01\\
89.88	0.01\\
89.89	0.01\\
89.9	0.01\\
89.91	0.01\\
89.92	0.01\\
89.93	0.01\\
89.94	0.01\\
89.95	0.01\\
89.96	0.01\\
89.97	0.01\\
89.98	0.01\\
89.99	0.01\\
90	0.01\\
90.01	0.01\\
90.02	0.01\\
90.03	0.01\\
90.04	0.01\\
90.05	0.01\\
90.06	0.01\\
90.07	0.01\\
90.08	0.01\\
90.09	0.01\\
90.1	0.01\\
90.11	0.01\\
90.12	0.01\\
90.13	0.01\\
90.14	0.01\\
90.15	0.01\\
90.16	0.01\\
90.17	0.01\\
90.18	0.01\\
90.19	0.01\\
90.2	0.01\\
90.21	0.01\\
90.22	0.01\\
90.23	0.01\\
90.24	0.01\\
90.25	0.01\\
90.26	0.01\\
90.27	0.01\\
90.28	0.01\\
90.29	0.01\\
90.3	0.01\\
90.31	0.01\\
90.32	0.01\\
90.33	0.01\\
90.34	0.01\\
90.35	0.01\\
90.36	0.01\\
90.37	0.01\\
90.38	0.01\\
90.39	0.01\\
90.4	0.01\\
90.41	0.01\\
90.42	0.01\\
90.43	0.01\\
90.44	0.01\\
90.45	0.01\\
90.46	0.01\\
90.47	0.01\\
90.48	0.01\\
90.49	0.01\\
90.5	0.01\\
90.51	0.01\\
90.52	0.01\\
90.53	0.01\\
90.54	0.01\\
90.55	0.01\\
90.56	0.01\\
90.57	0.01\\
90.58	0.01\\
90.59	0.01\\
90.6	0.01\\
90.61	0.01\\
90.62	0.01\\
90.63	0.01\\
90.64	0.01\\
90.65	0.01\\
90.66	0.01\\
90.67	0.01\\
90.68	0.01\\
90.69	0.01\\
90.7	0.01\\
90.71	0.01\\
90.72	0.01\\
90.73	0.01\\
90.74	0.01\\
90.75	0.01\\
90.76	0.01\\
90.77	0.01\\
90.78	0.01\\
90.79	0.01\\
90.8	0.01\\
90.81	0.01\\
90.82	0.01\\
90.83	0.01\\
90.84	0.01\\
90.85	0.01\\
90.86	0.01\\
90.87	0.01\\
90.88	0.01\\
90.89	0.01\\
90.9	0.01\\
90.91	0.01\\
90.92	0.01\\
90.93	0.01\\
90.94	0.01\\
90.95	0.01\\
90.96	0.01\\
90.97	0.01\\
90.98	0.01\\
90.99	0.01\\
91	0.01\\
91.01	0.01\\
91.02	0.01\\
91.03	0.01\\
91.04	0.01\\
91.05	0.01\\
91.06	0.01\\
91.07	0.01\\
91.08	0.01\\
91.09	0.01\\
91.1	0.01\\
91.11	0.01\\
91.12	0.01\\
91.13	0.01\\
91.14	0.01\\
91.15	0.01\\
91.16	0.01\\
91.17	0.01\\
91.18	0.01\\
91.19	0.01\\
91.2	0.01\\
91.21	0.01\\
91.22	0.01\\
91.23	0.01\\
91.24	0.01\\
91.25	0.01\\
91.26	0.01\\
91.27	0.01\\
91.28	0.01\\
91.29	0.01\\
91.3	0.01\\
91.31	0.01\\
91.32	0.01\\
91.33	0.01\\
91.34	0.01\\
91.35	0.01\\
91.36	0.01\\
91.37	0.01\\
91.38	0.01\\
91.39	0.01\\
91.4	0.01\\
91.41	0.01\\
91.42	0.01\\
91.43	0.01\\
91.44	0.01\\
91.45	0.01\\
91.46	0.01\\
91.47	0.01\\
91.48	0.01\\
91.49	0.01\\
91.5	0.01\\
91.51	0.01\\
91.52	0.01\\
91.53	0.01\\
91.54	0.01\\
91.55	0.01\\
91.56	0.01\\
91.57	0.01\\
91.58	0.01\\
91.59	0.01\\
91.6	0.01\\
91.61	0.01\\
91.62	0.01\\
91.63	0.01\\
91.64	0.01\\
91.65	0.01\\
91.66	0.01\\
91.67	0.01\\
91.68	0.01\\
91.69	0.01\\
91.7	0.01\\
91.71	0.01\\
91.72	0.01\\
91.73	0.01\\
91.74	0.01\\
91.75	0.01\\
91.76	0.01\\
91.77	0.01\\
91.78	0.01\\
91.79	0.01\\
91.8	0.01\\
91.81	0.01\\
91.82	0.01\\
91.83	0.01\\
91.84	0.01\\
91.85	0.01\\
91.86	0.01\\
91.87	0.01\\
91.88	0.01\\
91.89	0.01\\
91.9	0.01\\
91.91	0.01\\
91.92	0.01\\
91.93	0.01\\
91.94	0.01\\
91.95	0.01\\
91.96	0.01\\
91.97	0.01\\
91.98	0.01\\
91.99	0.01\\
92	0.01\\
92.01	0.01\\
92.02	0.01\\
92.03	0.01\\
92.04	0.01\\
92.05	0.01\\
92.06	0.01\\
92.07	0.01\\
92.08	0.01\\
92.09	0.01\\
92.1	0.01\\
92.11	0.01\\
92.12	0.01\\
92.13	0.01\\
92.14	0.01\\
92.15	0.01\\
92.16	0.01\\
92.17	0.01\\
92.18	0.01\\
92.19	0.01\\
92.2	0.01\\
92.21	0.01\\
92.22	0.01\\
92.23	0.01\\
92.24	0.01\\
92.25	0.01\\
92.26	0.01\\
92.27	0.01\\
92.28	0.01\\
92.29	0.01\\
92.3	0.01\\
92.31	0.01\\
92.32	0.01\\
92.33	0.01\\
92.34	0.01\\
92.35	0.01\\
92.36	0.01\\
92.37	0.01\\
92.38	0.01\\
92.39	0.01\\
92.4	0.01\\
92.41	0.01\\
92.42	0.01\\
92.43	0.01\\
92.44	0.01\\
92.45	0.01\\
92.46	0.01\\
92.47	0.01\\
92.48	0.01\\
92.49	0.01\\
92.5	0.01\\
92.51	0.01\\
92.52	0.01\\
92.53	0.01\\
92.54	0.01\\
92.55	0.01\\
92.56	0.01\\
92.57	0.01\\
92.58	0.01\\
92.59	0.01\\
92.6	0.01\\
92.61	0.01\\
92.62	0.01\\
92.63	0.01\\
92.64	0.01\\
92.65	0.01\\
92.66	0.01\\
92.67	0.01\\
92.68	0.01\\
92.69	0.01\\
92.7	0.01\\
92.71	0.01\\
92.72	0.01\\
92.73	0.01\\
92.74	0.01\\
92.75	0.01\\
92.76	0.01\\
92.77	0.01\\
92.78	0.01\\
92.79	0.01\\
92.8	0.01\\
92.81	0.01\\
92.82	0.01\\
92.83	0.01\\
92.84	0.01\\
92.85	0.01\\
92.86	0.01\\
92.87	0.01\\
92.88	0.01\\
92.89	0.01\\
92.9	0.01\\
92.91	0.01\\
92.92	0.01\\
92.93	0.01\\
92.94	0.01\\
92.95	0.01\\
92.96	0.01\\
92.97	0.01\\
92.98	0.01\\
92.99	0.01\\
93	0.01\\
93.01	0.01\\
93.02	0.01\\
93.03	0.01\\
93.04	0.01\\
93.05	0.01\\
93.06	0.01\\
93.07	0.01\\
93.08	0.01\\
93.09	0.01\\
93.1	0.01\\
93.11	0.01\\
93.12	0.01\\
93.13	0.01\\
93.14	0.01\\
93.15	0.01\\
93.16	0.01\\
93.17	0.01\\
93.18	0.01\\
93.19	0.01\\
93.2	0.01\\
93.21	0.01\\
93.22	0.01\\
93.23	0.01\\
93.24	0.01\\
93.25	0.01\\
93.26	0.01\\
93.27	0.01\\
93.28	0.01\\
93.29	0.01\\
93.3	0.01\\
93.31	0.01\\
93.32	0.01\\
93.33	0.01\\
93.34	0.01\\
93.35	0.01\\
93.36	0.01\\
93.37	0.01\\
93.38	0.01\\
93.39	0.01\\
93.4	0.01\\
93.41	0.01\\
93.42	0.01\\
93.43	0.01\\
93.44	0.01\\
93.45	0.01\\
93.46	0.01\\
93.47	0.01\\
93.48	0.01\\
93.49	0.01\\
93.5	0.01\\
93.51	0.01\\
93.52	0.01\\
93.53	0.01\\
93.54	0.01\\
93.55	0.01\\
93.56	0.01\\
93.57	0.01\\
93.58	0.01\\
93.59	0.01\\
93.6	0.01\\
93.61	0.01\\
93.62	0.01\\
93.63	0.01\\
93.64	0.01\\
93.65	0.01\\
93.66	0.01\\
93.67	0.01\\
93.68	0.01\\
93.69	0.01\\
93.7	0.01\\
93.71	0.01\\
93.72	0.01\\
93.73	0.01\\
93.74	0.01\\
93.75	0.01\\
93.76	0.01\\
93.77	0.01\\
93.78	0.01\\
93.79	0.01\\
93.8	0.01\\
93.81	0.01\\
93.82	0.01\\
93.83	0.01\\
93.84	0.01\\
93.85	0.01\\
93.86	0.01\\
93.87	0.01\\
93.88	0.01\\
93.89	0.01\\
93.9	0.01\\
93.91	0.01\\
93.92	0.01\\
93.93	0.01\\
93.94	0.01\\
93.95	0.01\\
93.96	0.01\\
93.97	0.01\\
93.98	0.01\\
93.99	0.01\\
94	0.01\\
94.01	0.01\\
94.02	0.01\\
94.03	0.01\\
94.04	0.01\\
94.05	0.01\\
94.06	0.01\\
94.07	0.01\\
94.08	0.01\\
94.09	0.01\\
94.1	0.01\\
94.11	0.01\\
94.12	0.01\\
94.13	0.01\\
94.14	0.01\\
94.15	0.01\\
94.16	0.01\\
94.17	0.01\\
94.18	0.01\\
94.19	0.01\\
94.2	0.01\\
94.21	0.01\\
94.22	0.01\\
94.23	0.01\\
94.24	0.01\\
94.25	0.01\\
94.26	0.01\\
94.27	0.01\\
94.28	0.01\\
94.29	0.01\\
94.3	0.01\\
94.31	0.01\\
94.32	0.01\\
94.33	0.01\\
94.34	0.01\\
94.35	0.01\\
94.36	0.01\\
94.37	0.01\\
94.38	0.01\\
94.39	0.01\\
94.4	0.01\\
94.41	0.01\\
94.42	0.01\\
94.43	0.01\\
94.44	0.01\\
94.45	0.01\\
94.46	0.01\\
94.47	0.01\\
94.48	0.01\\
94.49	0.01\\
94.5	0.01\\
94.51	0.01\\
94.52	0.01\\
94.53	0.01\\
94.54	0.01\\
94.55	0.01\\
94.56	0.01\\
94.57	0.01\\
94.58	0.01\\
94.59	0.01\\
94.6	0.01\\
94.61	0.01\\
94.62	0.01\\
94.63	0.01\\
94.64	0.01\\
94.65	0.01\\
94.66	0.01\\
94.67	0.01\\
94.68	0.01\\
94.69	0.01\\
94.7	0.01\\
94.71	0.01\\
94.72	0.01\\
94.73	0.01\\
94.74	0.01\\
94.75	0.01\\
94.76	0.01\\
94.77	0.01\\
94.78	0.01\\
94.79	0.01\\
94.8	0.01\\
94.81	0.01\\
94.82	0.01\\
94.83	0.01\\
94.84	0.01\\
94.85	0.01\\
94.86	0.01\\
94.87	0.01\\
94.88	0.01\\
94.89	0.01\\
94.9	0.01\\
94.91	0.01\\
94.92	0.01\\
94.93	0.01\\
94.94	0.01\\
94.95	0.01\\
94.96	0.01\\
94.97	0.01\\
94.98	0.01\\
94.99	0.01\\
95	0.01\\
95.01	0.01\\
95.02	0.01\\
95.03	0.01\\
95.04	0.01\\
95.05	0.01\\
95.06	0.01\\
95.07	0.01\\
95.08	0.01\\
95.09	0.01\\
95.1	0.01\\
95.11	0.01\\
95.12	0.01\\
95.13	0.01\\
95.14	0.01\\
95.15	0.01\\
95.16	0.01\\
95.17	0.01\\
95.18	0.01\\
95.19	0.01\\
95.2	0.01\\
95.21	0.01\\
95.22	0.01\\
95.23	0.01\\
95.24	0.01\\
95.25	0.01\\
95.26	0.01\\
95.27	0.01\\
95.28	0.01\\
95.29	0.01\\
95.3	0.01\\
95.31	0.01\\
95.32	0.01\\
95.33	0.01\\
95.34	0.01\\
95.35	0.01\\
95.36	0.01\\
95.37	0.01\\
95.38	0.01\\
95.39	0.01\\
95.4	0.01\\
95.41	0.01\\
95.42	0.01\\
95.43	0.01\\
95.44	0.01\\
95.45	0.01\\
95.46	0.01\\
95.47	0.01\\
95.48	0.01\\
95.49	0.01\\
95.5	0.01\\
95.51	0.01\\
95.52	0.01\\
95.53	0.01\\
95.54	0.01\\
95.55	0.01\\
95.56	0.01\\
95.57	0.01\\
95.58	0.01\\
95.59	0.01\\
95.6	0.01\\
95.61	0.01\\
95.62	0.01\\
95.63	0.01\\
95.64	0.01\\
95.65	0.01\\
95.66	0.01\\
95.67	0.01\\
95.68	0.01\\
95.69	0.01\\
95.7	0.01\\
95.71	0.01\\
95.72	0.01\\
95.73	0.01\\
95.74	0.01\\
95.75	0.01\\
95.76	0.01\\
95.77	0.01\\
95.78	0.01\\
95.79	0.01\\
95.8	0.01\\
95.81	0.01\\
95.82	0.01\\
95.83	0.01\\
95.84	0.01\\
95.85	0.01\\
95.86	0.01\\
95.87	0.01\\
95.88	0.01\\
95.89	0.01\\
95.9	0.01\\
95.91	0.01\\
95.92	0.01\\
95.93	0.01\\
95.94	0.01\\
95.95	0.01\\
95.96	0.01\\
95.97	0.01\\
95.98	0.01\\
95.99	0.01\\
96	0.01\\
96.01	0.01\\
96.02	0.01\\
96.03	0.01\\
96.04	0.01\\
96.05	0.01\\
96.06	0.01\\
96.07	0.01\\
96.08	0.01\\
96.09	0.01\\
96.1	0.01\\
96.11	0.01\\
96.12	0.01\\
96.13	0.01\\
96.14	0.01\\
96.15	0.01\\
96.16	0.01\\
96.17	0.01\\
96.18	0.01\\
96.19	0.01\\
96.2	0.01\\
96.21	0.01\\
96.22	0.01\\
96.23	0.01\\
96.24	0.01\\
96.25	0.01\\
96.26	0.01\\
96.27	0.01\\
96.28	0.01\\
96.29	0.01\\
96.3	0.01\\
96.31	0.01\\
96.32	0.01\\
96.33	0.01\\
96.34	0.01\\
96.35	0.01\\
96.36	0.01\\
96.37	0.01\\
96.38	0.01\\
96.39	0.01\\
96.4	0.01\\
96.41	0.01\\
96.42	0.01\\
96.43	0.01\\
96.44	0.01\\
96.45	0.01\\
96.46	0.01\\
96.47	0.01\\
96.48	0.01\\
96.49	0.01\\
96.5	0.01\\
96.51	0.01\\
96.52	0.01\\
96.53	0.01\\
96.54	0.01\\
96.55	0.01\\
96.56	0.01\\
96.57	0.01\\
96.58	0.01\\
96.59	0.01\\
96.6	0.01\\
96.61	0.01\\
96.62	0.01\\
96.63	0.01\\
96.64	0.01\\
96.65	0.01\\
96.66	0.01\\
96.67	0.01\\
96.68	0.01\\
96.69	0.01\\
96.7	0.01\\
96.71	0.01\\
96.72	0.01\\
96.73	0.01\\
96.74	0.01\\
96.75	0.01\\
96.76	0.01\\
96.77	0.01\\
96.78	0.01\\
96.79	0.01\\
96.8	0.01\\
96.81	0.01\\
96.82	0.01\\
96.83	0.01\\
96.84	0.01\\
96.85	0.01\\
96.86	0.01\\
96.87	0.01\\
96.88	0.01\\
96.89	0.01\\
96.9	0.01\\
96.91	0.01\\
96.92	0.01\\
96.93	0.01\\
96.94	0.01\\
96.95	0.01\\
96.96	0.01\\
96.97	0.01\\
96.98	0.01\\
96.99	0.01\\
97	0.01\\
97.01	0.01\\
97.02	0.01\\
97.03	0.01\\
97.04	0.01\\
97.05	0.01\\
97.06	0.01\\
97.07	0.01\\
97.08	0.01\\
97.09	0.01\\
97.1	0.01\\
97.11	0.01\\
97.12	0.01\\
97.13	0.01\\
97.14	0.01\\
97.15	0.01\\
97.16	0.01\\
97.17	0.01\\
97.18	0.01\\
97.19	0.01\\
97.2	0.01\\
97.21	0.01\\
97.22	0.01\\
97.23	0.01\\
97.24	0.01\\
97.25	0.01\\
97.26	0.01\\
97.27	0.01\\
97.28	0.01\\
97.29	0.01\\
97.3	0.01\\
97.31	0.01\\
97.32	0.01\\
97.33	0.01\\
97.34	0.01\\
97.35	0.01\\
97.36	0.01\\
97.37	0.01\\
97.38	0.01\\
97.39	0.01\\
97.4	0.01\\
97.41	0.01\\
97.42	0.01\\
97.43	0.01\\
97.44	0.01\\
97.45	0.01\\
97.46	0.01\\
97.47	0.01\\
97.48	0.01\\
97.49	0.01\\
97.5	0.01\\
97.51	0.01\\
97.52	0.01\\
97.53	0.01\\
97.54	0.01\\
97.55	0.01\\
97.56	0.01\\
97.57	0.01\\
97.58	0.01\\
97.59	0.01\\
97.6	0.01\\
97.61	0.01\\
97.62	0.01\\
97.63	0.01\\
97.64	0.01\\
97.65	0.01\\
97.66	0.01\\
97.67	0.01\\
97.68	0.01\\
97.69	0.01\\
97.7	0.01\\
97.71	0.01\\
97.72	0.01\\
97.73	0.01\\
97.74	0.01\\
97.75	0.01\\
97.76	0.01\\
97.77	0.01\\
97.78	0.01\\
97.79	0.01\\
97.8	0.01\\
97.81	0.01\\
97.82	0.01\\
97.83	0.01\\
97.84	0.01\\
97.85	0.01\\
97.86	0.01\\
97.87	0.01\\
97.88	0.01\\
97.89	0.01\\
97.9	0.01\\
97.91	0.01\\
97.92	0.01\\
97.93	0.01\\
97.94	0.01\\
97.95	0.01\\
97.96	0.01\\
97.97	0.01\\
97.98	0.01\\
97.99	0.01\\
98	0.01\\
98.01	0.01\\
98.02	0.01\\
98.03	0.01\\
98.04	0.01\\
98.05	0.01\\
98.06	0.01\\
98.07	0.01\\
98.08	0.01\\
98.09	0.01\\
98.1	0.01\\
98.11	0.01\\
98.12	0.01\\
98.13	0.01\\
98.14	0.01\\
98.15	0.01\\
98.16	0.01\\
98.17	0.01\\
98.18	0.01\\
98.19	0.01\\
98.2	0.01\\
98.21	0.01\\
98.22	0.01\\
98.23	0.01\\
98.24	0.01\\
98.25	0.01\\
98.26	0.01\\
98.27	0.01\\
98.28	0.01\\
98.29	0.01\\
98.3	0.01\\
98.31	0.01\\
98.32	0.01\\
98.33	0.01\\
98.34	0.01\\
98.35	0.01\\
98.36	0.01\\
98.37	0.01\\
98.38	0.01\\
98.39	0.01\\
98.4	0.01\\
98.41	0.01\\
98.42	0.01\\
98.43	0.01\\
98.44	0.01\\
98.45	0.01\\
98.46	0.01\\
98.47	0.01\\
98.48	0.01\\
98.49	0.01\\
98.5	0.01\\
98.51	0.01\\
98.52	0.01\\
98.53	0.01\\
98.54	0.01\\
98.55	0.01\\
98.56	0.01\\
98.57	0.01\\
98.58	0.01\\
98.59	0.01\\
98.6	0.01\\
98.61	0.01\\
98.62	0.01\\
98.63	0.01\\
98.64	0.01\\
98.65	0.01\\
98.66	0.01\\
98.67	0.01\\
98.68	0.01\\
98.69	0.01\\
98.7	0.01\\
98.71	0.01\\
98.72	0.01\\
98.73	0.01\\
98.74	0.01\\
98.75	0.01\\
98.76	0.01\\
98.77	0.01\\
98.78	0.01\\
98.79	0.01\\
98.8	0.01\\
98.81	0.01\\
98.82	0.01\\
98.83	0.01\\
98.84	0.01\\
98.85	0.01\\
98.86	0.01\\
98.87	0.01\\
98.88	0.01\\
98.89	0.01\\
98.9	0.01\\
98.91	0.01\\
98.92	0.01\\
98.93	0.01\\
98.94	0.01\\
98.95	0.01\\
98.96	0.01\\
98.97	0.01\\
98.98	0.01\\
98.99	0.01\\
99	0.01\\
99.01	0.01\\
99.02	0.01\\
99.03	0.01\\
99.04	0.01\\
99.05	0.01\\
99.06	0.01\\
99.07	0.01\\
99.08	0.01\\
99.09	0.01\\
99.1	0.01\\
99.11	0.01\\
99.12	0.01\\
99.13	0.01\\
99.14	0.01\\
99.15	0.01\\
99.16	0.01\\
99.17	0.01\\
99.18	0.01\\
99.19	0.01\\
99.2	0.01\\
99.21	0.01\\
99.22	0.01\\
99.23	0.01\\
99.24	0.01\\
99.25	0.01\\
99.26	0.01\\
99.27	0.01\\
99.28	0.01\\
99.29	0.01\\
99.3	0.01\\
99.31	0.01\\
99.32	0.01\\
99.33	0.01\\
99.34	0.01\\
99.35	0.01\\
99.36	0.01\\
99.37	0.01\\
99.38	0.01\\
99.39	0.01\\
99.4	0.01\\
99.41	0.01\\
99.42	0.01\\
99.43	0.01\\
99.44	0.01\\
99.45	0.01\\
99.46	0.01\\
99.47	0.01\\
99.48	0.01\\
99.49	0.01\\
99.5	0.01\\
99.51	0.01\\
99.52	0.01\\
99.53	0.01\\
99.54	0.01\\
99.55	0.01\\
99.56	0.01\\
99.57	0.01\\
99.58	0.01\\
99.59	0.01\\
99.6	0.01\\
99.61	0.01\\
99.62	0.01\\
99.63	0.01\\
99.64	0.01\\
99.65	0.01\\
99.66	0.01\\
99.67	0.01\\
99.68	0.01\\
99.69	0.01\\
99.7	0.01\\
99.71	0.01\\
99.72	0.01\\
99.73	0.01\\
99.74	0.01\\
99.75	0.01\\
99.76	0.01\\
99.77	0.01\\
99.78	0.01\\
99.79	0.01\\
99.8	0.01\\
99.81	0.01\\
99.82	0.01\\
99.83	0.01\\
99.84	0.01\\
99.85	0.01\\
99.86	0.01\\
99.87	0.01\\
99.88	0.01\\
99.89	0.01\\
99.9	0.01\\
99.91	0.01\\
99.92	0.01\\
99.93	0.01\\
99.94	0.01\\
99.95	0.01\\
99.96	0.01\\
99.97	0.01\\
99.98	0.01\\
99.99	0.01\\
100	0.01\\
};
\addlegendentry{$q=2$};

\addplot [color=mycolor1,solid,forget plot]
  table[row sep=crcr]{%
0.01	0.01\\
0.02	0.01\\
0.03	0.01\\
0.04	0.01\\
0.05	0.01\\
0.06	0.01\\
0.07	0.01\\
0.08	0.01\\
0.09	0.01\\
0.1	0.01\\
0.11	0.01\\
0.12	0.01\\
0.13	0.01\\
0.14	0.01\\
0.15	0.01\\
0.16	0.01\\
0.17	0.01\\
0.18	0.01\\
0.19	0.01\\
0.2	0.01\\
0.21	0.01\\
0.22	0.01\\
0.23	0.01\\
0.24	0.01\\
0.25	0.01\\
0.26	0.01\\
0.27	0.01\\
0.28	0.01\\
0.29	0.01\\
0.3	0.01\\
0.31	0.01\\
0.32	0.01\\
0.33	0.01\\
0.34	0.01\\
0.35	0.01\\
0.36	0.01\\
0.37	0.01\\
0.38	0.01\\
0.39	0.01\\
0.4	0.01\\
0.41	0.01\\
0.42	0.01\\
0.43	0.01\\
0.44	0.01\\
0.45	0.01\\
0.46	0.01\\
0.47	0.01\\
0.48	0.01\\
0.49	0.01\\
0.5	0.01\\
0.51	0.01\\
0.52	0.01\\
0.53	0.01\\
0.54	0.01\\
0.55	0.01\\
0.56	0.01\\
0.57	0.01\\
0.58	0.01\\
0.59	0.01\\
0.6	0.01\\
0.61	0.01\\
0.62	0.01\\
0.63	0.01\\
0.64	0.01\\
0.65	0.01\\
0.66	0.01\\
0.67	0.01\\
0.68	0.01\\
0.69	0.01\\
0.7	0.01\\
0.71	0.01\\
0.72	0.01\\
0.73	0.01\\
0.74	0.01\\
0.75	0.01\\
0.76	0.01\\
0.77	0.01\\
0.78	0.01\\
0.79	0.01\\
0.8	0.01\\
0.81	0.01\\
0.82	0.01\\
0.83	0.01\\
0.84	0.01\\
0.85	0.01\\
0.86	0.01\\
0.87	0.01\\
0.88	0.01\\
0.89	0.01\\
0.9	0.01\\
0.91	0.01\\
0.92	0.01\\
0.93	0.01\\
0.94	0.01\\
0.95	0.01\\
0.96	0.01\\
0.97	0.01\\
0.98	0.01\\
0.99	0.01\\
1	0.01\\
1.01	0.01\\
1.02	0.01\\
1.03	0.01\\
1.04	0.01\\
1.05	0.01\\
1.06	0.01\\
1.07	0.01\\
1.08	0.01\\
1.09	0.01\\
1.1	0.01\\
1.11	0.01\\
1.12	0.01\\
1.13	0.01\\
1.14	0.01\\
1.15	0.01\\
1.16	0.01\\
1.17	0.01\\
1.18	0.01\\
1.19	0.01\\
1.2	0.01\\
1.21	0.01\\
1.22	0.01\\
1.23	0.01\\
1.24	0.01\\
1.25	0.01\\
1.26	0.01\\
1.27	0.01\\
1.28	0.01\\
1.29	0.01\\
1.3	0.01\\
1.31	0.01\\
1.32	0.01\\
1.33	0.01\\
1.34	0.01\\
1.35	0.01\\
1.36	0.01\\
1.37	0.01\\
1.38	0.01\\
1.39	0.01\\
1.4	0.01\\
1.41	0.01\\
1.42	0.01\\
1.43	0.01\\
1.44	0.01\\
1.45	0.01\\
1.46	0.01\\
1.47	0.01\\
1.48	0.01\\
1.49	0.01\\
1.5	0.01\\
1.51	0.01\\
1.52	0.01\\
1.53	0.01\\
1.54	0.01\\
1.55	0.01\\
1.56	0.01\\
1.57	0.01\\
1.58	0.01\\
1.59	0.01\\
1.6	0.01\\
1.61	0.01\\
1.62	0.01\\
1.63	0.01\\
1.64	0.01\\
1.65	0.01\\
1.66	0.01\\
1.67	0.01\\
1.68	0.01\\
1.69	0.01\\
1.7	0.01\\
1.71	0.01\\
1.72	0.01\\
1.73	0.01\\
1.74	0.01\\
1.75	0.01\\
1.76	0.01\\
1.77	0.01\\
1.78	0.01\\
1.79	0.01\\
1.8	0.01\\
1.81	0.01\\
1.82	0.01\\
1.83	0.01\\
1.84	0.01\\
1.85	0.01\\
1.86	0.01\\
1.87	0.01\\
1.88	0.01\\
1.89	0.01\\
1.9	0.01\\
1.91	0.01\\
1.92	0.01\\
1.93	0.01\\
1.94	0.01\\
1.95	0.01\\
1.96	0.01\\
1.97	0.01\\
1.98	0.01\\
1.99	0.01\\
2	0.01\\
2.01	0.01\\
2.02	0.01\\
2.03	0.01\\
2.04	0.01\\
2.05	0.01\\
2.06	0.01\\
2.07	0.01\\
2.08	0.01\\
2.09	0.01\\
2.1	0.01\\
2.11	0.01\\
2.12	0.01\\
2.13	0.01\\
2.14	0.01\\
2.15	0.01\\
2.16	0.01\\
2.17	0.01\\
2.18	0.01\\
2.19	0.01\\
2.2	0.01\\
2.21	0.01\\
2.22	0.01\\
2.23	0.01\\
2.24	0.01\\
2.25	0.01\\
2.26	0.01\\
2.27	0.01\\
2.28	0.01\\
2.29	0.01\\
2.3	0.01\\
2.31	0.01\\
2.32	0.01\\
2.33	0.01\\
2.34	0.01\\
2.35	0.01\\
2.36	0.01\\
2.37	0.01\\
2.38	0.01\\
2.39	0.01\\
2.4	0.01\\
2.41	0.01\\
2.42	0.01\\
2.43	0.01\\
2.44	0.01\\
2.45	0.01\\
2.46	0.01\\
2.47	0.01\\
2.48	0.01\\
2.49	0.01\\
2.5	0.01\\
2.51	0.01\\
2.52	0.01\\
2.53	0.01\\
2.54	0.01\\
2.55	0.01\\
2.56	0.01\\
2.57	0.01\\
2.58	0.01\\
2.59	0.01\\
2.6	0.01\\
2.61	0.01\\
2.62	0.01\\
2.63	0.01\\
2.64	0.01\\
2.65	0.01\\
2.66	0.01\\
2.67	0.01\\
2.68	0.01\\
2.69	0.01\\
2.7	0.01\\
2.71	0.01\\
2.72	0.01\\
2.73	0.01\\
2.74	0.01\\
2.75	0.01\\
2.76	0.01\\
2.77	0.01\\
2.78	0.01\\
2.79	0.01\\
2.8	0.01\\
2.81	0.01\\
2.82	0.01\\
2.83	0.01\\
2.84	0.01\\
2.85	0.01\\
2.86	0.01\\
2.87	0.01\\
2.88	0.01\\
2.89	0.01\\
2.9	0.01\\
2.91	0.01\\
2.92	0.01\\
2.93	0.01\\
2.94	0.01\\
2.95	0.01\\
2.96	0.01\\
2.97	0.01\\
2.98	0.01\\
2.99	0.01\\
3	0.01\\
3.01	0.01\\
3.02	0.01\\
3.03	0.01\\
3.04	0.01\\
3.05	0.01\\
3.06	0.01\\
3.07	0.01\\
3.08	0.01\\
3.09	0.01\\
3.1	0.01\\
3.11	0.01\\
3.12	0.01\\
3.13	0.01\\
3.14	0.01\\
3.15	0.01\\
3.16	0.01\\
3.17	0.01\\
3.18	0.01\\
3.19	0.01\\
3.2	0.01\\
3.21	0.01\\
3.22	0.01\\
3.23	0.01\\
3.24	0.01\\
3.25	0.01\\
3.26	0.01\\
3.27	0.01\\
3.28	0.01\\
3.29	0.01\\
3.3	0.01\\
3.31	0.01\\
3.32	0.01\\
3.33	0.01\\
3.34	0.01\\
3.35	0.01\\
3.36	0.01\\
3.37	0.01\\
3.38	0.01\\
3.39	0.01\\
3.4	0.01\\
3.41	0.01\\
3.42	0.01\\
3.43	0.01\\
3.44	0.01\\
3.45	0.01\\
3.46	0.01\\
3.47	0.01\\
3.48	0.01\\
3.49	0.01\\
3.5	0.01\\
3.51	0.01\\
3.52	0.01\\
3.53	0.01\\
3.54	0.01\\
3.55	0.01\\
3.56	0.01\\
3.57	0.01\\
3.58	0.01\\
3.59	0.01\\
3.6	0.01\\
3.61	0.01\\
3.62	0.01\\
3.63	0.01\\
3.64	0.01\\
3.65	0.01\\
3.66	0.01\\
3.67	0.01\\
3.68	0.01\\
3.69	0.01\\
3.7	0.01\\
3.71	0.01\\
3.72	0.01\\
3.73	0.01\\
3.74	0.01\\
3.75	0.01\\
3.76	0.01\\
3.77	0.01\\
3.78	0.01\\
3.79	0.01\\
3.8	0.01\\
3.81	0.01\\
3.82	0.01\\
3.83	0.01\\
3.84	0.01\\
3.85	0.01\\
3.86	0.01\\
3.87	0.01\\
3.88	0.01\\
3.89	0.01\\
3.9	0.01\\
3.91	0.01\\
3.92	0.01\\
3.93	0.01\\
3.94	0.01\\
3.95	0.01\\
3.96	0.01\\
3.97	0.01\\
3.98	0.01\\
3.99	0.01\\
4	0.01\\
4.01	0.01\\
4.02	0.01\\
4.03	0.01\\
4.04	0.01\\
4.05	0.01\\
4.06	0.01\\
4.07	0.01\\
4.08	0.01\\
4.09	0.01\\
4.1	0.01\\
4.11	0.01\\
4.12	0.01\\
4.13	0.01\\
4.14	0.01\\
4.15	0.01\\
4.16	0.01\\
4.17	0.01\\
4.18	0.01\\
4.19	0.01\\
4.2	0.01\\
4.21	0.01\\
4.22	0.01\\
4.23	0.01\\
4.24	0.01\\
4.25	0.01\\
4.26	0.01\\
4.27	0.01\\
4.28	0.01\\
4.29	0.01\\
4.3	0.01\\
4.31	0.01\\
4.32	0.01\\
4.33	0.01\\
4.34	0.01\\
4.35	0.01\\
4.36	0.01\\
4.37	0.01\\
4.38	0.01\\
4.39	0.01\\
4.4	0.01\\
4.41	0.01\\
4.42	0.01\\
4.43	0.01\\
4.44	0.01\\
4.45	0.01\\
4.46	0.01\\
4.47	0.01\\
4.48	0.01\\
4.49	0.01\\
4.5	0.01\\
4.51	0.01\\
4.52	0.01\\
4.53	0.01\\
4.54	0.01\\
4.55	0.01\\
4.56	0.01\\
4.57	0.01\\
4.58	0.01\\
4.59	0.01\\
4.6	0.01\\
4.61	0.01\\
4.62	0.01\\
4.63	0.01\\
4.64	0.01\\
4.65	0.01\\
4.66	0.01\\
4.67	0.01\\
4.68	0.01\\
4.69	0.01\\
4.7	0.01\\
4.71	0.01\\
4.72	0.01\\
4.73	0.01\\
4.74	0.01\\
4.75	0.01\\
4.76	0.01\\
4.77	0.01\\
4.78	0.01\\
4.79	0.01\\
4.8	0.01\\
4.81	0.01\\
4.82	0.01\\
4.83	0.01\\
4.84	0.01\\
4.85	0.01\\
4.86	0.01\\
4.87	0.01\\
4.88	0.01\\
4.89	0.01\\
4.9	0.01\\
4.91	0.01\\
4.92	0.01\\
4.93	0.01\\
4.94	0.01\\
4.95	0.01\\
4.96	0.01\\
4.97	0.01\\
4.98	0.01\\
4.99	0.01\\
5	0.01\\
5.01	0.01\\
5.02	0.01\\
5.03	0.01\\
5.04	0.01\\
5.05	0.01\\
5.06	0.01\\
5.07	0.01\\
5.08	0.01\\
5.09	0.01\\
5.1	0.01\\
5.11	0.01\\
5.12	0.01\\
5.13	0.01\\
5.14	0.01\\
5.15	0.01\\
5.16	0.01\\
5.17	0.01\\
5.18	0.01\\
5.19	0.01\\
5.2	0.01\\
5.21	0.01\\
5.22	0.01\\
5.23	0.01\\
5.24	0.01\\
5.25	0.01\\
5.26	0.01\\
5.27	0.01\\
5.28	0.01\\
5.29	0.01\\
5.3	0.01\\
5.31	0.01\\
5.32	0.01\\
5.33	0.01\\
5.34	0.01\\
5.35	0.01\\
5.36	0.01\\
5.37	0.01\\
5.38	0.01\\
5.39	0.01\\
5.4	0.01\\
5.41	0.01\\
5.42	0.01\\
5.43	0.01\\
5.44	0.01\\
5.45	0.01\\
5.46	0.01\\
5.47	0.01\\
5.48	0.01\\
5.49	0.01\\
5.5	0.01\\
5.51	0.01\\
5.52	0.01\\
5.53	0.01\\
5.54	0.01\\
5.55	0.01\\
5.56	0.01\\
5.57	0.01\\
5.58	0.01\\
5.59	0.01\\
5.6	0.01\\
5.61	0.01\\
5.62	0.01\\
5.63	0.01\\
5.64	0.01\\
5.65	0.01\\
5.66	0.01\\
5.67	0.01\\
5.68	0.01\\
5.69	0.01\\
5.7	0.01\\
5.71	0.01\\
5.72	0.01\\
5.73	0.01\\
5.74	0.01\\
5.75	0.01\\
5.76	0.01\\
5.77	0.01\\
5.78	0.01\\
5.79	0.01\\
5.8	0.01\\
5.81	0.01\\
5.82	0.01\\
5.83	0.01\\
5.84	0.01\\
5.85	0.01\\
5.86	0.01\\
5.87	0.01\\
5.88	0.01\\
5.89	0.01\\
5.9	0.01\\
5.91	0.01\\
5.92	0.01\\
5.93	0.01\\
5.94	0.01\\
5.95	0.01\\
5.96	0.01\\
5.97	0.01\\
5.98	0.01\\
5.99	0.01\\
6	0.01\\
6.01	0.01\\
6.02	0.01\\
6.03	0.01\\
6.04	0.01\\
6.05	0.01\\
6.06	0.01\\
6.07	0.01\\
6.08	0.01\\
6.09	0.01\\
6.1	0.01\\
6.11	0.01\\
6.12	0.01\\
6.13	0.01\\
6.14	0.01\\
6.15	0.01\\
6.16	0.01\\
6.17	0.01\\
6.18	0.01\\
6.19	0.01\\
6.2	0.01\\
6.21	0.01\\
6.22	0.01\\
6.23	0.01\\
6.24	0.01\\
6.25	0.01\\
6.26	0.01\\
6.27	0.01\\
6.28	0.01\\
6.29	0.01\\
6.3	0.01\\
6.31	0.01\\
6.32	0.01\\
6.33	0.01\\
6.34	0.01\\
6.35	0.01\\
6.36	0.01\\
6.37	0.01\\
6.38	0.01\\
6.39	0.01\\
6.4	0.01\\
6.41	0.01\\
6.42	0.01\\
6.43	0.01\\
6.44	0.01\\
6.45	0.01\\
6.46	0.01\\
6.47	0.01\\
6.48	0.01\\
6.49	0.01\\
6.5	0.01\\
6.51	0.01\\
6.52	0.01\\
6.53	0.01\\
6.54	0.01\\
6.55	0.01\\
6.56	0.01\\
6.57	0.01\\
6.58	0.01\\
6.59	0.01\\
6.6	0.01\\
6.61	0.01\\
6.62	0.01\\
6.63	0.01\\
6.64	0.01\\
6.65	0.01\\
6.66	0.01\\
6.67	0.01\\
6.68	0.01\\
6.69	0.01\\
6.7	0.01\\
6.71	0.01\\
6.72	0.01\\
6.73	0.01\\
6.74	0.01\\
6.75	0.01\\
6.76	0.01\\
6.77	0.01\\
6.78	0.01\\
6.79	0.01\\
6.8	0.01\\
6.81	0.01\\
6.82	0.01\\
6.83	0.01\\
6.84	0.01\\
6.85	0.01\\
6.86	0.01\\
6.87	0.01\\
6.88	0.01\\
6.89	0.01\\
6.9	0.01\\
6.91	0.01\\
6.92	0.01\\
6.93	0.01\\
6.94	0.01\\
6.95	0.01\\
6.96	0.01\\
6.97	0.01\\
6.98	0.01\\
6.99	0.01\\
7	0.01\\
7.01	0.01\\
7.02	0.01\\
7.03	0.01\\
7.04	0.01\\
7.05	0.01\\
7.06	0.01\\
7.07	0.01\\
7.08	0.01\\
7.09	0.01\\
7.1	0.01\\
7.11	0.01\\
7.12	0.01\\
7.13	0.01\\
7.14	0.01\\
7.15	0.01\\
7.16	0.01\\
7.17	0.01\\
7.18	0.01\\
7.19	0.01\\
7.2	0.01\\
7.21	0.01\\
7.22	0.01\\
7.23	0.01\\
7.24	0.01\\
7.25	0.01\\
7.26	0.01\\
7.27	0.01\\
7.28	0.01\\
7.29	0.01\\
7.3	0.01\\
7.31	0.01\\
7.32	0.01\\
7.33	0.01\\
7.34	0.01\\
7.35	0.01\\
7.36	0.01\\
7.37	0.01\\
7.38	0.01\\
7.39	0.01\\
7.4	0.01\\
7.41	0.01\\
7.42	0.01\\
7.43	0.01\\
7.44	0.01\\
7.45	0.01\\
7.46	0.01\\
7.47	0.01\\
7.48	0.01\\
7.49	0.01\\
7.5	0.01\\
7.51	0.01\\
7.52	0.01\\
7.53	0.01\\
7.54	0.01\\
7.55	0.01\\
7.56	0.01\\
7.57	0.01\\
7.58	0.01\\
7.59	0.01\\
7.6	0.01\\
7.61	0.01\\
7.62	0.01\\
7.63	0.01\\
7.64	0.01\\
7.65	0.01\\
7.66	0.01\\
7.67	0.01\\
7.68	0.01\\
7.69	0.01\\
7.7	0.01\\
7.71	0.01\\
7.72	0.01\\
7.73	0.01\\
7.74	0.01\\
7.75	0.01\\
7.76	0.01\\
7.77	0.01\\
7.78	0.01\\
7.79	0.01\\
7.8	0.01\\
7.81	0.01\\
7.82	0.01\\
7.83	0.01\\
7.84	0.01\\
7.85	0.01\\
7.86	0.01\\
7.87	0.01\\
7.88	0.01\\
7.89	0.01\\
7.9	0.01\\
7.91	0.01\\
7.92	0.01\\
7.93	0.01\\
7.94	0.01\\
7.95	0.01\\
7.96	0.01\\
7.97	0.01\\
7.98	0.01\\
7.99	0.01\\
8	0.01\\
8.01	0.01\\
8.02	0.01\\
8.03	0.01\\
8.04	0.01\\
8.05	0.01\\
8.06	0.01\\
8.07	0.01\\
8.08	0.01\\
8.09	0.01\\
8.1	0.01\\
8.11	0.01\\
8.12	0.01\\
8.13	0.01\\
8.14	0.01\\
8.15	0.01\\
8.16	0.01\\
8.17	0.01\\
8.18	0.01\\
8.19	0.01\\
8.2	0.01\\
8.21	0.01\\
8.22	0.01\\
8.23	0.01\\
8.24	0.01\\
8.25	0.01\\
8.26	0.01\\
8.27	0.01\\
8.28	0.01\\
8.29	0.01\\
8.3	0.01\\
8.31	0.01\\
8.32	0.01\\
8.33	0.01\\
8.34	0.01\\
8.35	0.01\\
8.36	0.01\\
8.37	0.01\\
8.38	0.01\\
8.39	0.01\\
8.4	0.01\\
8.41	0.01\\
8.42	0.01\\
8.43	0.01\\
8.44	0.01\\
8.45	0.01\\
8.46	0.01\\
8.47	0.01\\
8.48	0.01\\
8.49	0.01\\
8.5	0.01\\
8.51	0.01\\
8.52	0.01\\
8.53	0.01\\
8.54	0.01\\
8.55	0.01\\
8.56	0.01\\
8.57	0.01\\
8.58	0.01\\
8.59	0.01\\
8.6	0.01\\
8.61	0.01\\
8.62	0.01\\
8.63	0.01\\
8.64	0.01\\
8.65	0.01\\
8.66	0.01\\
8.67	0.01\\
8.68	0.01\\
8.69	0.01\\
8.7	0.01\\
8.71	0.01\\
8.72	0.01\\
8.73	0.01\\
8.74	0.01\\
8.75	0.01\\
8.76	0.01\\
8.77	0.01\\
8.78	0.01\\
8.79	0.01\\
8.8	0.01\\
8.81	0.01\\
8.82	0.01\\
8.83	0.01\\
8.84	0.01\\
8.85	0.01\\
8.86	0.01\\
8.87	0.01\\
8.88	0.01\\
8.89	0.01\\
8.9	0.01\\
8.91	0.01\\
8.92	0.01\\
8.93	0.01\\
8.94	0.01\\
8.95	0.01\\
8.96	0.01\\
8.97	0.01\\
8.98	0.01\\
8.99	0.01\\
9	0.01\\
9.01	0.01\\
9.02	0.01\\
9.03	0.01\\
9.04	0.01\\
9.05	0.01\\
9.06	0.01\\
9.07	0.01\\
9.08	0.01\\
9.09	0.01\\
9.1	0.01\\
9.11	0.01\\
9.12	0.01\\
9.13	0.01\\
9.14	0.01\\
9.15	0.01\\
9.16	0.01\\
9.17	0.01\\
9.18	0.01\\
9.19	0.01\\
9.2	0.01\\
9.21	0.01\\
9.22	0.01\\
9.23	0.01\\
9.24	0.01\\
9.25	0.01\\
9.26	0.01\\
9.27	0.01\\
9.28	0.01\\
9.29	0.01\\
9.3	0.01\\
9.31	0.01\\
9.32	0.01\\
9.33	0.01\\
9.34	0.01\\
9.35	0.01\\
9.36	0.01\\
9.37	0.01\\
9.38	0.01\\
9.39	0.01\\
9.4	0.01\\
9.41	0.01\\
9.42	0.01\\
9.43	0.01\\
9.44	0.01\\
9.45	0.01\\
9.46	0.01\\
9.47	0.01\\
9.48	0.01\\
9.49	0.01\\
9.5	0.01\\
9.51	0.01\\
9.52	0.01\\
9.53	0.01\\
9.54	0.01\\
9.55	0.01\\
9.56	0.01\\
9.57	0.01\\
9.58	0.01\\
9.59	0.01\\
9.6	0.01\\
9.61	0.01\\
9.62	0.01\\
9.63	0.01\\
9.64	0.01\\
9.65	0.01\\
9.66	0.01\\
9.67	0.01\\
9.68	0.01\\
9.69	0.01\\
9.7	0.01\\
9.71	0.01\\
9.72	0.01\\
9.73	0.01\\
9.74	0.01\\
9.75	0.01\\
9.76	0.01\\
9.77	0.01\\
9.78	0.01\\
9.79	0.01\\
9.8	0.01\\
9.81	0.01\\
9.82	0.01\\
9.83	0.01\\
9.84	0.01\\
9.85	0.01\\
9.86	0.01\\
9.87	0.01\\
9.88	0.01\\
9.89	0.01\\
9.9	0.01\\
9.91	0.01\\
9.92	0.01\\
9.93	0.01\\
9.94	0.01\\
9.95	0.01\\
9.96	0.01\\
9.97	0.01\\
9.98	0.01\\
9.99	0.01\\
10	0.01\\
10.01	0.01\\
10.02	0.01\\
10.03	0.01\\
10.04	0.01\\
10.05	0.01\\
10.06	0.01\\
10.07	0.01\\
10.08	0.01\\
10.09	0.01\\
10.1	0.01\\
10.11	0.01\\
10.12	0.01\\
10.13	0.01\\
10.14	0.01\\
10.15	0.01\\
10.16	0.01\\
10.17	0.01\\
10.18	0.01\\
10.19	0.01\\
10.2	0.01\\
10.21	0.01\\
10.22	0.01\\
10.23	0.01\\
10.24	0.01\\
10.25	0.01\\
10.26	0.01\\
10.27	0.01\\
10.28	0.01\\
10.29	0.01\\
10.3	0.01\\
10.31	0.01\\
10.32	0.01\\
10.33	0.01\\
10.34	0.01\\
10.35	0.01\\
10.36	0.01\\
10.37	0.01\\
10.38	0.01\\
10.39	0.01\\
10.4	0.01\\
10.41	0.01\\
10.42	0.01\\
10.43	0.01\\
10.44	0.01\\
10.45	0.01\\
10.46	0.01\\
10.47	0.01\\
10.48	0.01\\
10.49	0.01\\
10.5	0.01\\
10.51	0.01\\
10.52	0.01\\
10.53	0.01\\
10.54	0.01\\
10.55	0.01\\
10.56	0.01\\
10.57	0.01\\
10.58	0.01\\
10.59	0.01\\
10.6	0.01\\
10.61	0.01\\
10.62	0.01\\
10.63	0.01\\
10.64	0.01\\
10.65	0.01\\
10.66	0.01\\
10.67	0.01\\
10.68	0.01\\
10.69	0.01\\
10.7	0.01\\
10.71	0.01\\
10.72	0.01\\
10.73	0.01\\
10.74	0.01\\
10.75	0.01\\
10.76	0.01\\
10.77	0.01\\
10.78	0.01\\
10.79	0.01\\
10.8	0.01\\
10.81	0.01\\
10.82	0.01\\
10.83	0.01\\
10.84	0.01\\
10.85	0.01\\
10.86	0.01\\
10.87	0.01\\
10.88	0.01\\
10.89	0.01\\
10.9	0.01\\
10.91	0.01\\
10.92	0.01\\
10.93	0.01\\
10.94	0.01\\
10.95	0.01\\
10.96	0.01\\
10.97	0.01\\
10.98	0.01\\
10.99	0.01\\
11	0.01\\
11.01	0.01\\
11.02	0.01\\
11.03	0.01\\
11.04	0.01\\
11.05	0.01\\
11.06	0.01\\
11.07	0.01\\
11.08	0.01\\
11.09	0.01\\
11.1	0.01\\
11.11	0.01\\
11.12	0.01\\
11.13	0.01\\
11.14	0.01\\
11.15	0.01\\
11.16	0.01\\
11.17	0.01\\
11.18	0.01\\
11.19	0.01\\
11.2	0.01\\
11.21	0.01\\
11.22	0.01\\
11.23	0.01\\
11.24	0.01\\
11.25	0.01\\
11.26	0.01\\
11.27	0.01\\
11.28	0.01\\
11.29	0.01\\
11.3	0.01\\
11.31	0.01\\
11.32	0.01\\
11.33	0.01\\
11.34	0.01\\
11.35	0.01\\
11.36	0.01\\
11.37	0.01\\
11.38	0.01\\
11.39	0.01\\
11.4	0.01\\
11.41	0.01\\
11.42	0.01\\
11.43	0.01\\
11.44	0.01\\
11.45	0.01\\
11.46	0.01\\
11.47	0.01\\
11.48	0.01\\
11.49	0.01\\
11.5	0.01\\
11.51	0.01\\
11.52	0.01\\
11.53	0.01\\
11.54	0.01\\
11.55	0.01\\
11.56	0.01\\
11.57	0.01\\
11.58	0.01\\
11.59	0.01\\
11.6	0.01\\
11.61	0.01\\
11.62	0.01\\
11.63	0.01\\
11.64	0.01\\
11.65	0.01\\
11.66	0.01\\
11.67	0.01\\
11.68	0.01\\
11.69	0.01\\
11.7	0.01\\
11.71	0.01\\
11.72	0.01\\
11.73	0.01\\
11.74	0.01\\
11.75	0.01\\
11.76	0.01\\
11.77	0.01\\
11.78	0.01\\
11.79	0.01\\
11.8	0.01\\
11.81	0.01\\
11.82	0.01\\
11.83	0.01\\
11.84	0.01\\
11.85	0.01\\
11.86	0.01\\
11.87	0.01\\
11.88	0.01\\
11.89	0.01\\
11.9	0.01\\
11.91	0.01\\
11.92	0.01\\
11.93	0.01\\
11.94	0.01\\
11.95	0.01\\
11.96	0.01\\
11.97	0.01\\
11.98	0.01\\
11.99	0.01\\
12	0.01\\
12.01	0.01\\
12.02	0.01\\
12.03	0.01\\
12.04	0.01\\
12.05	0.01\\
12.06	0.01\\
12.07	0.01\\
12.08	0.01\\
12.09	0.01\\
12.1	0.01\\
12.11	0.01\\
12.12	0.01\\
12.13	0.01\\
12.14	0.01\\
12.15	0.01\\
12.16	0.01\\
12.17	0.01\\
12.18	0.01\\
12.19	0.01\\
12.2	0.01\\
12.21	0.01\\
12.22	0.01\\
12.23	0.01\\
12.24	0.01\\
12.25	0.01\\
12.26	0.01\\
12.27	0.01\\
12.28	0.01\\
12.29	0.01\\
12.3	0.01\\
12.31	0.01\\
12.32	0.01\\
12.33	0.01\\
12.34	0.01\\
12.35	0.01\\
12.36	0.01\\
12.37	0.01\\
12.38	0.01\\
12.39	0.01\\
12.4	0.01\\
12.41	0.01\\
12.42	0.01\\
12.43	0.01\\
12.44	0.01\\
12.45	0.01\\
12.46	0.01\\
12.47	0.01\\
12.48	0.01\\
12.49	0.01\\
12.5	0.01\\
12.51	0.01\\
12.52	0.01\\
12.53	0.01\\
12.54	0.01\\
12.55	0.01\\
12.56	0.01\\
12.57	0.01\\
12.58	0.01\\
12.59	0.01\\
12.6	0.01\\
12.61	0.01\\
12.62	0.01\\
12.63	0.01\\
12.64	0.01\\
12.65	0.01\\
12.66	0.01\\
12.67	0.01\\
12.68	0.01\\
12.69	0.01\\
12.7	0.01\\
12.71	0.01\\
12.72	0.01\\
12.73	0.01\\
12.74	0.01\\
12.75	0.01\\
12.76	0.01\\
12.77	0.01\\
12.78	0.01\\
12.79	0.01\\
12.8	0.01\\
12.81	0.01\\
12.82	0.01\\
12.83	0.01\\
12.84	0.01\\
12.85	0.01\\
12.86	0.01\\
12.87	0.01\\
12.88	0.01\\
12.89	0.01\\
12.9	0.01\\
12.91	0.01\\
12.92	0.01\\
12.93	0.01\\
12.94	0.01\\
12.95	0.01\\
12.96	0.01\\
12.97	0.01\\
12.98	0.01\\
12.99	0.01\\
13	0.01\\
13.01	0.01\\
13.02	0.01\\
13.03	0.01\\
13.04	0.01\\
13.05	0.01\\
13.06	0.01\\
13.07	0.01\\
13.08	0.01\\
13.09	0.01\\
13.1	0.01\\
13.11	0.01\\
13.12	0.01\\
13.13	0.01\\
13.14	0.01\\
13.15	0.01\\
13.16	0.01\\
13.17	0.01\\
13.18	0.01\\
13.19	0.01\\
13.2	0.01\\
13.21	0.01\\
13.22	0.01\\
13.23	0.01\\
13.24	0.01\\
13.25	0.01\\
13.26	0.01\\
13.27	0.01\\
13.28	0.01\\
13.29	0.01\\
13.3	0.01\\
13.31	0.01\\
13.32	0.01\\
13.33	0.01\\
13.34	0.01\\
13.35	0.01\\
13.36	0.01\\
13.37	0.01\\
13.38	0.01\\
13.39	0.01\\
13.4	0.01\\
13.41	0.01\\
13.42	0.01\\
13.43	0.01\\
13.44	0.01\\
13.45	0.01\\
13.46	0.01\\
13.47	0.01\\
13.48	0.01\\
13.49	0.01\\
13.5	0.01\\
13.51	0.01\\
13.52	0.01\\
13.53	0.01\\
13.54	0.01\\
13.55	0.01\\
13.56	0.01\\
13.57	0.01\\
13.58	0.01\\
13.59	0.01\\
13.6	0.01\\
13.61	0.01\\
13.62	0.01\\
13.63	0.01\\
13.64	0.01\\
13.65	0.01\\
13.66	0.01\\
13.67	0.01\\
13.68	0.01\\
13.69	0.01\\
13.7	0.01\\
13.71	0.01\\
13.72	0.01\\
13.73	0.01\\
13.74	0.01\\
13.75	0.01\\
13.76	0.01\\
13.77	0.01\\
13.78	0.01\\
13.79	0.01\\
13.8	0.01\\
13.81	0.01\\
13.82	0.01\\
13.83	0.01\\
13.84	0.01\\
13.85	0.01\\
13.86	0.01\\
13.87	0.01\\
13.88	0.01\\
13.89	0.01\\
13.9	0.01\\
13.91	0.01\\
13.92	0.01\\
13.93	0.01\\
13.94	0.01\\
13.95	0.01\\
13.96	0.01\\
13.97	0.01\\
13.98	0.01\\
13.99	0.01\\
14	0.01\\
14.01	0.01\\
14.02	0.01\\
14.03	0.01\\
14.04	0.01\\
14.05	0.01\\
14.06	0.01\\
14.07	0.01\\
14.08	0.01\\
14.09	0.01\\
14.1	0.01\\
14.11	0.01\\
14.12	0.01\\
14.13	0.01\\
14.14	0.01\\
14.15	0.01\\
14.16	0.01\\
14.17	0.01\\
14.18	0.01\\
14.19	0.01\\
14.2	0.01\\
14.21	0.01\\
14.22	0.01\\
14.23	0.01\\
14.24	0.01\\
14.25	0.01\\
14.26	0.01\\
14.27	0.01\\
14.28	0.01\\
14.29	0.01\\
14.3	0.01\\
14.31	0.01\\
14.32	0.01\\
14.33	0.01\\
14.34	0.01\\
14.35	0.01\\
14.36	0.01\\
14.37	0.01\\
14.38	0.01\\
14.39	0.01\\
14.4	0.01\\
14.41	0.01\\
14.42	0.01\\
14.43	0.01\\
14.44	0.01\\
14.45	0.01\\
14.46	0.01\\
14.47	0.01\\
14.48	0.01\\
14.49	0.01\\
14.5	0.01\\
14.51	0.01\\
14.52	0.01\\
14.53	0.01\\
14.54	0.01\\
14.55	0.01\\
14.56	0.01\\
14.57	0.01\\
14.58	0.01\\
14.59	0.01\\
14.6	0.01\\
14.61	0.01\\
14.62	0.01\\
14.63	0.01\\
14.64	0.01\\
14.65	0.01\\
14.66	0.01\\
14.67	0.01\\
14.68	0.01\\
14.69	0.01\\
14.7	0.01\\
14.71	0.01\\
14.72	0.01\\
14.73	0.01\\
14.74	0.01\\
14.75	0.01\\
14.76	0.01\\
14.77	0.01\\
14.78	0.01\\
14.79	0.01\\
14.8	0.01\\
14.81	0.01\\
14.82	0.01\\
14.83	0.01\\
14.84	0.01\\
14.85	0.01\\
14.86	0.01\\
14.87	0.01\\
14.88	0.01\\
14.89	0.01\\
14.9	0.01\\
14.91	0.01\\
14.92	0.01\\
14.93	0.01\\
14.94	0.01\\
14.95	0.01\\
14.96	0.01\\
14.97	0.01\\
14.98	0.01\\
14.99	0.01\\
15	0.01\\
15.01	0.01\\
15.02	0.01\\
15.03	0.01\\
15.04	0.01\\
15.05	0.01\\
15.06	0.01\\
15.07	0.01\\
15.08	0.01\\
15.09	0.01\\
15.1	0.01\\
15.11	0.01\\
15.12	0.01\\
15.13	0.01\\
15.14	0.01\\
15.15	0.01\\
15.16	0.01\\
15.17	0.01\\
15.18	0.01\\
15.19	0.01\\
15.2	0.01\\
15.21	0.01\\
15.22	0.01\\
15.23	0.01\\
15.24	0.01\\
15.25	0.01\\
15.26	0.01\\
15.27	0.01\\
15.28	0.01\\
15.29	0.01\\
15.3	0.01\\
15.31	0.01\\
15.32	0.01\\
15.33	0.01\\
15.34	0.01\\
15.35	0.01\\
15.36	0.01\\
15.37	0.01\\
15.38	0.01\\
15.39	0.01\\
15.4	0.01\\
15.41	0.01\\
15.42	0.01\\
15.43	0.01\\
15.44	0.01\\
15.45	0.01\\
15.46	0.01\\
15.47	0.01\\
15.48	0.01\\
15.49	0.01\\
15.5	0.01\\
15.51	0.01\\
15.52	0.01\\
15.53	0.01\\
15.54	0.01\\
15.55	0.01\\
15.56	0.01\\
15.57	0.01\\
15.58	0.01\\
15.59	0.01\\
15.6	0.01\\
15.61	0.01\\
15.62	0.01\\
15.63	0.01\\
15.64	0.01\\
15.65	0.01\\
15.66	0.01\\
15.67	0.01\\
15.68	0.01\\
15.69	0.01\\
15.7	0.01\\
15.71	0.01\\
15.72	0.01\\
15.73	0.01\\
15.74	0.01\\
15.75	0.01\\
15.76	0.01\\
15.77	0.01\\
15.78	0.01\\
15.79	0.01\\
15.8	0.01\\
15.81	0.01\\
15.82	0.01\\
15.83	0.01\\
15.84	0.01\\
15.85	0.01\\
15.86	0.01\\
15.87	0.01\\
15.88	0.01\\
15.89	0.01\\
15.9	0.01\\
15.91	0.01\\
15.92	0.01\\
15.93	0.01\\
15.94	0.01\\
15.95	0.01\\
15.96	0.01\\
15.97	0.01\\
15.98	0.01\\
15.99	0.01\\
16	0.01\\
16.01	0.01\\
16.02	0.01\\
16.03	0.01\\
16.04	0.01\\
16.05	0.01\\
16.06	0.01\\
16.07	0.01\\
16.08	0.01\\
16.09	0.01\\
16.1	0.01\\
16.11	0.01\\
16.12	0.01\\
16.13	0.01\\
16.14	0.01\\
16.15	0.01\\
16.16	0.01\\
16.17	0.01\\
16.18	0.01\\
16.19	0.01\\
16.2	0.01\\
16.21	0.01\\
16.22	0.01\\
16.23	0.01\\
16.24	0.01\\
16.25	0.01\\
16.26	0.01\\
16.27	0.01\\
16.28	0.01\\
16.29	0.01\\
16.3	0.01\\
16.31	0.01\\
16.32	0.01\\
16.33	0.01\\
16.34	0.01\\
16.35	0.01\\
16.36	0.01\\
16.37	0.01\\
16.38	0.01\\
16.39	0.01\\
16.4	0.01\\
16.41	0.01\\
16.42	0.01\\
16.43	0.01\\
16.44	0.01\\
16.45	0.01\\
16.46	0.01\\
16.47	0.01\\
16.48	0.01\\
16.49	0.01\\
16.5	0.01\\
16.51	0.01\\
16.52	0.01\\
16.53	0.01\\
16.54	0.01\\
16.55	0.01\\
16.56	0.01\\
16.57	0.01\\
16.58	0.01\\
16.59	0.01\\
16.6	0.01\\
16.61	0.01\\
16.62	0.01\\
16.63	0.01\\
16.64	0.01\\
16.65	0.01\\
16.66	0.01\\
16.67	0.01\\
16.68	0.01\\
16.69	0.01\\
16.7	0.01\\
16.71	0.01\\
16.72	0.01\\
16.73	0.01\\
16.74	0.01\\
16.75	0.01\\
16.76	0.01\\
16.77	0.01\\
16.78	0.01\\
16.79	0.01\\
16.8	0.01\\
16.81	0.01\\
16.82	0.01\\
16.83	0.01\\
16.84	0.01\\
16.85	0.01\\
16.86	0.01\\
16.87	0.01\\
16.88	0.01\\
16.89	0.01\\
16.9	0.01\\
16.91	0.01\\
16.92	0.01\\
16.93	0.01\\
16.94	0.01\\
16.95	0.01\\
16.96	0.01\\
16.97	0.01\\
16.98	0.01\\
16.99	0.01\\
17	0.01\\
17.01	0.01\\
17.02	0.01\\
17.03	0.01\\
17.04	0.01\\
17.05	0.01\\
17.06	0.01\\
17.07	0.01\\
17.08	0.01\\
17.09	0.01\\
17.1	0.01\\
17.11	0.01\\
17.12	0.01\\
17.13	0.01\\
17.14	0.01\\
17.15	0.01\\
17.16	0.01\\
17.17	0.01\\
17.18	0.01\\
17.19	0.01\\
17.2	0.01\\
17.21	0.01\\
17.22	0.01\\
17.23	0.01\\
17.24	0.01\\
17.25	0.01\\
17.26	0.01\\
17.27	0.01\\
17.28	0.01\\
17.29	0.01\\
17.3	0.01\\
17.31	0.01\\
17.32	0.01\\
17.33	0.01\\
17.34	0.01\\
17.35	0.01\\
17.36	0.01\\
17.37	0.01\\
17.38	0.01\\
17.39	0.01\\
17.4	0.01\\
17.41	0.01\\
17.42	0.01\\
17.43	0.01\\
17.44	0.01\\
17.45	0.01\\
17.46	0.01\\
17.47	0.01\\
17.48	0.01\\
17.49	0.01\\
17.5	0.01\\
17.51	0.01\\
17.52	0.01\\
17.53	0.01\\
17.54	0.01\\
17.55	0.01\\
17.56	0.01\\
17.57	0.01\\
17.58	0.01\\
17.59	0.01\\
17.6	0.01\\
17.61	0.01\\
17.62	0.01\\
17.63	0.01\\
17.64	0.01\\
17.65	0.01\\
17.66	0.01\\
17.67	0.01\\
17.68	0.01\\
17.69	0.01\\
17.7	0.01\\
17.71	0.01\\
17.72	0.01\\
17.73	0.01\\
17.74	0.01\\
17.75	0.01\\
17.76	0.01\\
17.77	0.01\\
17.78	0.01\\
17.79	0.01\\
17.8	0.01\\
17.81	0.01\\
17.82	0.01\\
17.83	0.01\\
17.84	0.01\\
17.85	0.01\\
17.86	0.01\\
17.87	0.01\\
17.88	0.01\\
17.89	0.01\\
17.9	0.01\\
17.91	0.01\\
17.92	0.01\\
17.93	0.01\\
17.94	0.01\\
17.95	0.01\\
17.96	0.01\\
17.97	0.01\\
17.98	0.01\\
17.99	0.01\\
18	0.01\\
18.01	0.01\\
18.02	0.01\\
18.03	0.01\\
18.04	0.01\\
18.05	0.01\\
18.06	0.01\\
18.07	0.01\\
18.08	0.01\\
18.09	0.01\\
18.1	0.01\\
18.11	0.01\\
18.12	0.01\\
18.13	0.01\\
18.14	0.01\\
18.15	0.01\\
18.16	0.01\\
18.17	0.01\\
18.18	0.01\\
18.19	0.01\\
18.2	0.01\\
18.21	0.01\\
18.22	0.01\\
18.23	0.01\\
18.24	0.01\\
18.25	0.01\\
18.26	0.01\\
18.27	0.01\\
18.28	0.01\\
18.29	0.01\\
18.3	0.01\\
18.31	0.01\\
18.32	0.01\\
18.33	0.01\\
18.34	0.01\\
18.35	0.01\\
18.36	0.01\\
18.37	0.01\\
18.38	0.01\\
18.39	0.01\\
18.4	0.01\\
18.41	0.01\\
18.42	0.01\\
18.43	0.01\\
18.44	0.01\\
18.45	0.01\\
18.46	0.01\\
18.47	0.01\\
18.48	0.01\\
18.49	0.01\\
18.5	0.01\\
18.51	0.01\\
18.52	0.01\\
18.53	0.01\\
18.54	0.01\\
18.55	0.01\\
18.56	0.01\\
18.57	0.01\\
18.58	0.01\\
18.59	0.01\\
18.6	0.01\\
18.61	0.01\\
18.62	0.01\\
18.63	0.01\\
18.64	0.01\\
18.65	0.01\\
18.66	0.01\\
18.67	0.01\\
18.68	0.01\\
18.69	0.01\\
18.7	0.01\\
18.71	0.01\\
18.72	0.01\\
18.73	0.01\\
18.74	0.01\\
18.75	0.01\\
18.76	0.01\\
18.77	0.01\\
18.78	0.01\\
18.79	0.01\\
18.8	0.01\\
18.81	0.01\\
18.82	0.01\\
18.83	0.01\\
18.84	0.01\\
18.85	0.01\\
18.86	0.01\\
18.87	0.01\\
18.88	0.01\\
18.89	0.01\\
18.9	0.01\\
18.91	0.01\\
18.92	0.01\\
18.93	0.01\\
18.94	0.01\\
18.95	0.01\\
18.96	0.01\\
18.97	0.01\\
18.98	0.01\\
18.99	0.01\\
19	0.01\\
19.01	0.01\\
19.02	0.01\\
19.03	0.01\\
19.04	0.01\\
19.05	0.01\\
19.06	0.01\\
19.07	0.01\\
19.08	0.01\\
19.09	0.01\\
19.1	0.01\\
19.11	0.01\\
19.12	0.01\\
19.13	0.01\\
19.14	0.01\\
19.15	0.01\\
19.16	0.01\\
19.17	0.01\\
19.18	0.01\\
19.19	0.01\\
19.2	0.01\\
19.21	0.01\\
19.22	0.01\\
19.23	0.01\\
19.24	0.01\\
19.25	0.01\\
19.26	0.01\\
19.27	0.01\\
19.28	0.01\\
19.29	0.01\\
19.3	0.01\\
19.31	0.01\\
19.32	0.01\\
19.33	0.01\\
19.34	0.01\\
19.35	0.01\\
19.36	0.01\\
19.37	0.01\\
19.38	0.01\\
19.39	0.01\\
19.4	0.01\\
19.41	0.01\\
19.42	0.01\\
19.43	0.01\\
19.44	0.01\\
19.45	0.01\\
19.46	0.01\\
19.47	0.01\\
19.48	0.01\\
19.49	0.01\\
19.5	0.01\\
19.51	0.01\\
19.52	0.01\\
19.53	0.01\\
19.54	0.01\\
19.55	0.01\\
19.56	0.01\\
19.57	0.01\\
19.58	0.01\\
19.59	0.01\\
19.6	0.01\\
19.61	0.01\\
19.62	0.01\\
19.63	0.01\\
19.64	0.01\\
19.65	0.01\\
19.66	0.01\\
19.67	0.01\\
19.68	0.01\\
19.69	0.01\\
19.7	0.01\\
19.71	0.01\\
19.72	0.01\\
19.73	0.01\\
19.74	0.01\\
19.75	0.01\\
19.76	0.01\\
19.77	0.01\\
19.78	0.01\\
19.79	0.01\\
19.8	0.01\\
19.81	0.01\\
19.82	0.01\\
19.83	0.01\\
19.84	0.01\\
19.85	0.01\\
19.86	0.01\\
19.87	0.01\\
19.88	0.01\\
19.89	0.01\\
19.9	0.01\\
19.91	0.01\\
19.92	0.01\\
19.93	0.01\\
19.94	0.01\\
19.95	0.01\\
19.96	0.01\\
19.97	0.01\\
19.98	0.01\\
19.99	0.01\\
20	0.01\\
20.01	0.01\\
20.02	0.01\\
20.03	0.01\\
20.04	0.01\\
20.05	0.01\\
20.06	0.01\\
20.07	0.01\\
20.08	0.01\\
20.09	0.01\\
20.1	0.01\\
20.11	0.01\\
20.12	0.01\\
20.13	0.01\\
20.14	0.01\\
20.15	0.01\\
20.16	0.01\\
20.17	0.01\\
20.18	0.01\\
20.19	0.01\\
20.2	0.01\\
20.21	0.01\\
20.22	0.01\\
20.23	0.01\\
20.24	0.01\\
20.25	0.01\\
20.26	0.01\\
20.27	0.01\\
20.28	0.01\\
20.29	0.01\\
20.3	0.01\\
20.31	0.01\\
20.32	0.01\\
20.33	0.01\\
20.34	0.01\\
20.35	0.01\\
20.36	0.01\\
20.37	0.01\\
20.38	0.01\\
20.39	0.01\\
20.4	0.01\\
20.41	0.01\\
20.42	0.01\\
20.43	0.01\\
20.44	0.01\\
20.45	0.01\\
20.46	0.01\\
20.47	0.01\\
20.48	0.01\\
20.49	0.01\\
20.5	0.01\\
20.51	0.01\\
20.52	0.01\\
20.53	0.01\\
20.54	0.01\\
20.55	0.01\\
20.56	0.01\\
20.57	0.01\\
20.58	0.01\\
20.59	0.01\\
20.6	0.01\\
20.61	0.01\\
20.62	0.01\\
20.63	0.01\\
20.64	0.01\\
20.65	0.01\\
20.66	0.01\\
20.67	0.01\\
20.68	0.01\\
20.69	0.01\\
20.7	0.01\\
20.71	0.01\\
20.72	0.01\\
20.73	0.01\\
20.74	0.01\\
20.75	0.01\\
20.76	0.01\\
20.77	0.01\\
20.78	0.01\\
20.79	0.01\\
20.8	0.01\\
20.81	0.01\\
20.82	0.01\\
20.83	0.01\\
20.84	0.01\\
20.85	0.01\\
20.86	0.01\\
20.87	0.01\\
20.88	0.01\\
20.89	0.01\\
20.9	0.01\\
20.91	0.01\\
20.92	0.01\\
20.93	0.01\\
20.94	0.01\\
20.95	0.01\\
20.96	0.01\\
20.97	0.01\\
20.98	0.01\\
20.99	0.01\\
21	0.01\\
21.01	0.01\\
21.02	0.01\\
21.03	0.01\\
21.04	0.01\\
21.05	0.01\\
21.06	0.01\\
21.07	0.01\\
21.08	0.01\\
21.09	0.01\\
21.1	0.01\\
21.11	0.01\\
21.12	0.01\\
21.13	0.01\\
21.14	0.01\\
21.15	0.01\\
21.16	0.01\\
21.17	0.01\\
21.18	0.01\\
21.19	0.01\\
21.2	0.01\\
21.21	0.01\\
21.22	0.01\\
21.23	0.01\\
21.24	0.01\\
21.25	0.01\\
21.26	0.01\\
21.27	0.01\\
21.28	0.01\\
21.29	0.01\\
21.3	0.01\\
21.31	0.01\\
21.32	0.01\\
21.33	0.01\\
21.34	0.01\\
21.35	0.01\\
21.36	0.01\\
21.37	0.01\\
21.38	0.01\\
21.39	0.01\\
21.4	0.01\\
21.41	0.01\\
21.42	0.01\\
21.43	0.01\\
21.44	0.01\\
21.45	0.01\\
21.46	0.01\\
21.47	0.01\\
21.48	0.01\\
21.49	0.01\\
21.5	0.01\\
21.51	0.01\\
21.52	0.01\\
21.53	0.01\\
21.54	0.01\\
21.55	0.01\\
21.56	0.01\\
21.57	0.01\\
21.58	0.01\\
21.59	0.01\\
21.6	0.01\\
21.61	0.01\\
21.62	0.01\\
21.63	0.01\\
21.64	0.01\\
21.65	0.01\\
21.66	0.01\\
21.67	0.01\\
21.68	0.01\\
21.69	0.01\\
21.7	0.01\\
21.71	0.01\\
21.72	0.01\\
21.73	0.01\\
21.74	0.01\\
21.75	0.01\\
21.76	0.01\\
21.77	0.01\\
21.78	0.01\\
21.79	0.01\\
21.8	0.01\\
21.81	0.01\\
21.82	0.01\\
21.83	0.01\\
21.84	0.01\\
21.85	0.01\\
21.86	0.01\\
21.87	0.01\\
21.88	0.01\\
21.89	0.01\\
21.9	0.01\\
21.91	0.01\\
21.92	0.01\\
21.93	0.01\\
21.94	0.01\\
21.95	0.01\\
21.96	0.01\\
21.97	0.01\\
21.98	0.01\\
21.99	0.01\\
22	0.01\\
22.01	0.01\\
22.02	0.01\\
22.03	0.01\\
22.04	0.01\\
22.05	0.01\\
22.06	0.01\\
22.07	0.01\\
22.08	0.01\\
22.09	0.01\\
22.1	0.01\\
22.11	0.01\\
22.12	0.01\\
22.13	0.01\\
22.14	0.01\\
22.15	0.01\\
22.16	0.01\\
22.17	0.01\\
22.18	0.01\\
22.19	0.01\\
22.2	0.01\\
22.21	0.01\\
22.22	0.01\\
22.23	0.01\\
22.24	0.01\\
22.25	0.01\\
22.26	0.01\\
22.27	0.01\\
22.28	0.01\\
22.29	0.01\\
22.3	0.01\\
22.31	0.01\\
22.32	0.01\\
22.33	0.01\\
22.34	0.01\\
22.35	0.01\\
22.36	0.01\\
22.37	0.01\\
22.38	0.01\\
22.39	0.01\\
22.4	0.01\\
22.41	0.01\\
22.42	0.01\\
22.43	0.01\\
22.44	0.01\\
22.45	0.01\\
22.46	0.01\\
22.47	0.01\\
22.48	0.01\\
22.49	0.01\\
22.5	0.01\\
22.51	0.01\\
22.52	0.01\\
22.53	0.01\\
22.54	0.01\\
22.55	0.01\\
22.56	0.01\\
22.57	0.01\\
22.58	0.01\\
22.59	0.01\\
22.6	0.01\\
22.61	0.01\\
22.62	0.01\\
22.63	0.01\\
22.64	0.01\\
22.65	0.01\\
22.66	0.01\\
22.67	0.01\\
22.68	0.01\\
22.69	0.01\\
22.7	0.01\\
22.71	0.01\\
22.72	0.01\\
22.73	0.01\\
22.74	0.01\\
22.75	0.01\\
22.76	0.01\\
22.77	0.01\\
22.78	0.01\\
22.79	0.01\\
22.8	0.01\\
22.81	0.01\\
22.82	0.01\\
22.83	0.01\\
22.84	0.01\\
22.85	0.01\\
22.86	0.01\\
22.87	0.01\\
22.88	0.01\\
22.89	0.01\\
22.9	0.01\\
22.91	0.01\\
22.92	0.01\\
22.93	0.01\\
22.94	0.01\\
22.95	0.01\\
22.96	0.01\\
22.97	0.01\\
22.98	0.01\\
22.99	0.01\\
23	0.01\\
23.01	0.01\\
23.02	0.01\\
23.03	0.01\\
23.04	0.01\\
23.05	0.01\\
23.06	0.01\\
23.07	0.01\\
23.08	0.01\\
23.09	0.01\\
23.1	0.01\\
23.11	0.01\\
23.12	0.01\\
23.13	0.01\\
23.14	0.01\\
23.15	0.01\\
23.16	0.01\\
23.17	0.01\\
23.18	0.01\\
23.19	0.01\\
23.2	0.01\\
23.21	0.01\\
23.22	0.01\\
23.23	0.01\\
23.24	0.01\\
23.25	0.01\\
23.26	0.01\\
23.27	0.01\\
23.28	0.01\\
23.29	0.01\\
23.3	0.01\\
23.31	0.01\\
23.32	0.01\\
23.33	0.01\\
23.34	0.01\\
23.35	0.01\\
23.36	0.01\\
23.37	0.01\\
23.38	0.01\\
23.39	0.01\\
23.4	0.01\\
23.41	0.01\\
23.42	0.01\\
23.43	0.01\\
23.44	0.01\\
23.45	0.01\\
23.46	0.01\\
23.47	0.01\\
23.48	0.01\\
23.49	0.01\\
23.5	0.01\\
23.51	0.01\\
23.52	0.01\\
23.53	0.01\\
23.54	0.01\\
23.55	0.01\\
23.56	0.01\\
23.57	0.01\\
23.58	0.01\\
23.59	0.01\\
23.6	0.01\\
23.61	0.01\\
23.62	0.01\\
23.63	0.01\\
23.64	0.01\\
23.65	0.01\\
23.66	0.01\\
23.67	0.01\\
23.68	0.01\\
23.69	0.01\\
23.7	0.01\\
23.71	0.01\\
23.72	0.01\\
23.73	0.01\\
23.74	0.01\\
23.75	0.01\\
23.76	0.01\\
23.77	0.01\\
23.78	0.01\\
23.79	0.01\\
23.8	0.01\\
23.81	0.01\\
23.82	0.01\\
23.83	0.01\\
23.84	0.01\\
23.85	0.01\\
23.86	0.01\\
23.87	0.01\\
23.88	0.01\\
23.89	0.01\\
23.9	0.01\\
23.91	0.01\\
23.92	0.01\\
23.93	0.01\\
23.94	0.01\\
23.95	0.01\\
23.96	0.01\\
23.97	0.01\\
23.98	0.01\\
23.99	0.01\\
24	0.01\\
24.01	0.01\\
24.02	0.01\\
24.03	0.01\\
24.04	0.01\\
24.05	0.01\\
24.06	0.01\\
24.07	0.01\\
24.08	0.01\\
24.09	0.01\\
24.1	0.01\\
24.11	0.01\\
24.12	0.01\\
24.13	0.01\\
24.14	0.01\\
24.15	0.01\\
24.16	0.01\\
24.17	0.01\\
24.18	0.01\\
24.19	0.01\\
24.2	0.01\\
24.21	0.01\\
24.22	0.01\\
24.23	0.01\\
24.24	0.01\\
24.25	0.01\\
24.26	0.01\\
24.27	0.01\\
24.28	0.01\\
24.29	0.01\\
24.3	0.01\\
24.31	0.01\\
24.32	0.01\\
24.33	0.01\\
24.34	0.01\\
24.35	0.01\\
24.36	0.01\\
24.37	0.01\\
24.38	0.01\\
24.39	0.01\\
24.4	0.01\\
24.41	0.01\\
24.42	0.01\\
24.43	0.01\\
24.44	0.01\\
24.45	0.01\\
24.46	0.01\\
24.47	0.01\\
24.48	0.01\\
24.49	0.01\\
24.5	0.01\\
24.51	0.01\\
24.52	0.01\\
24.53	0.01\\
24.54	0.01\\
24.55	0.01\\
24.56	0.01\\
24.57	0.01\\
24.58	0.01\\
24.59	0.01\\
24.6	0.01\\
24.61	0.01\\
24.62	0.01\\
24.63	0.01\\
24.64	0.01\\
24.65	0.01\\
24.66	0.01\\
24.67	0.01\\
24.68	0.01\\
24.69	0.01\\
24.7	0.01\\
24.71	0.01\\
24.72	0.01\\
24.73	0.01\\
24.74	0.01\\
24.75	0.01\\
24.76	0.01\\
24.77	0.01\\
24.78	0.01\\
24.79	0.01\\
24.8	0.01\\
24.81	0.01\\
24.82	0.01\\
24.83	0.01\\
24.84	0.01\\
24.85	0.01\\
24.86	0.01\\
24.87	0.01\\
24.88	0.01\\
24.89	0.01\\
24.9	0.01\\
24.91	0.01\\
24.92	0.01\\
24.93	0.01\\
24.94	0.01\\
24.95	0.01\\
24.96	0.01\\
24.97	0.01\\
24.98	0.01\\
24.99	0.01\\
25	0.01\\
25.01	0.01\\
25.02	0.01\\
25.03	0.01\\
25.04	0.01\\
25.05	0.01\\
25.06	0.01\\
25.07	0.01\\
25.08	0.01\\
25.09	0.01\\
25.1	0.01\\
25.11	0.01\\
25.12	0.01\\
25.13	0.01\\
25.14	0.01\\
25.15	0.01\\
25.16	0.01\\
25.17	0.01\\
25.18	0.01\\
25.19	0.01\\
25.2	0.01\\
25.21	0.01\\
25.22	0.01\\
25.23	0.01\\
25.24	0.01\\
25.25	0.01\\
25.26	0.01\\
25.27	0.01\\
25.28	0.01\\
25.29	0.01\\
25.3	0.01\\
25.31	0.01\\
25.32	0.01\\
25.33	0.01\\
25.34	0.01\\
25.35	0.01\\
25.36	0.01\\
25.37	0.01\\
25.38	0.01\\
25.39	0.01\\
25.4	0.01\\
25.41	0.01\\
25.42	0.01\\
25.43	0.01\\
25.44	0.01\\
25.45	0.01\\
25.46	0.01\\
25.47	0.01\\
25.48	0.01\\
25.49	0.01\\
25.5	0.01\\
25.51	0.01\\
25.52	0.01\\
25.53	0.01\\
25.54	0.01\\
25.55	0.01\\
25.56	0.01\\
25.57	0.01\\
25.58	0.01\\
25.59	0.01\\
25.6	0.01\\
25.61	0.01\\
25.62	0.01\\
25.63	0.01\\
25.64	0.01\\
25.65	0.01\\
25.66	0.01\\
25.67	0.01\\
25.68	0.01\\
25.69	0.01\\
25.7	0.01\\
25.71	0.01\\
25.72	0.01\\
25.73	0.01\\
25.74	0.01\\
25.75	0.01\\
25.76	0.01\\
25.77	0.01\\
25.78	0.01\\
25.79	0.01\\
25.8	0.01\\
25.81	0.01\\
25.82	0.01\\
25.83	0.01\\
25.84	0.01\\
25.85	0.01\\
25.86	0.01\\
25.87	0.01\\
25.88	0.01\\
25.89	0.01\\
25.9	0.01\\
25.91	0.01\\
25.92	0.01\\
25.93	0.01\\
25.94	0.01\\
25.95	0.01\\
25.96	0.01\\
25.97	0.01\\
25.98	0.01\\
25.99	0.01\\
26	0.01\\
26.01	0.01\\
26.02	0.01\\
26.03	0.01\\
26.04	0.01\\
26.05	0.01\\
26.06	0.01\\
26.07	0.01\\
26.08	0.01\\
26.09	0.01\\
26.1	0.01\\
26.11	0.01\\
26.12	0.01\\
26.13	0.01\\
26.14	0.01\\
26.15	0.01\\
26.16	0.01\\
26.17	0.01\\
26.18	0.01\\
26.19	0.01\\
26.2	0.01\\
26.21	0.01\\
26.22	0.01\\
26.23	0.01\\
26.24	0.01\\
26.25	0.01\\
26.26	0.01\\
26.27	0.01\\
26.28	0.01\\
26.29	0.01\\
26.3	0.01\\
26.31	0.01\\
26.32	0.01\\
26.33	0.01\\
26.34	0.01\\
26.35	0.01\\
26.36	0.01\\
26.37	0.01\\
26.38	0.01\\
26.39	0.01\\
26.4	0.01\\
26.41	0.01\\
26.42	0.01\\
26.43	0.01\\
26.44	0.01\\
26.45	0.01\\
26.46	0.01\\
26.47	0.01\\
26.48	0.01\\
26.49	0.01\\
26.5	0.01\\
26.51	0.01\\
26.52	0.01\\
26.53	0.01\\
26.54	0.01\\
26.55	0.01\\
26.56	0.01\\
26.57	0.01\\
26.58	0.01\\
26.59	0.01\\
26.6	0.01\\
26.61	0.01\\
26.62	0.01\\
26.63	0.01\\
26.64	0.01\\
26.65	0.01\\
26.66	0.01\\
26.67	0.01\\
26.68	0.01\\
26.69	0.01\\
26.7	0.01\\
26.71	0.01\\
26.72	0.01\\
26.73	0.01\\
26.74	0.01\\
26.75	0.01\\
26.76	0.01\\
26.77	0.01\\
26.78	0.01\\
26.79	0.01\\
26.8	0.01\\
26.81	0.01\\
26.82	0.01\\
26.83	0.01\\
26.84	0.01\\
26.85	0.01\\
26.86	0.01\\
26.87	0.01\\
26.88	0.01\\
26.89	0.01\\
26.9	0.01\\
26.91	0.01\\
26.92	0.01\\
26.93	0.01\\
26.94	0.01\\
26.95	0.01\\
26.96	0.01\\
26.97	0.01\\
26.98	0.01\\
26.99	0.01\\
27	0.01\\
27.01	0.01\\
27.02	0.01\\
27.03	0.01\\
27.04	0.01\\
27.05	0.01\\
27.06	0.01\\
27.07	0.01\\
27.08	0.01\\
27.09	0.01\\
27.1	0.01\\
27.11	0.01\\
27.12	0.01\\
27.13	0.01\\
27.14	0.01\\
27.15	0.01\\
27.16	0.01\\
27.17	0.01\\
27.18	0.01\\
27.19	0.01\\
27.2	0.01\\
27.21	0.01\\
27.22	0.01\\
27.23	0.01\\
27.24	0.01\\
27.25	0.01\\
27.26	0.01\\
27.27	0.01\\
27.28	0.01\\
27.29	0.01\\
27.3	0.01\\
27.31	0.01\\
27.32	0.01\\
27.33	0.01\\
27.34	0.01\\
27.35	0.01\\
27.36	0.01\\
27.37	0.01\\
27.38	0.01\\
27.39	0.01\\
27.4	0.01\\
27.41	0.01\\
27.42	0.01\\
27.43	0.01\\
27.44	0.01\\
27.45	0.01\\
27.46	0.01\\
27.47	0.01\\
27.48	0.01\\
27.49	0.01\\
27.5	0.01\\
27.51	0.01\\
27.52	0.01\\
27.53	0.01\\
27.54	0.01\\
27.55	0.01\\
27.56	0.01\\
27.57	0.01\\
27.58	0.01\\
27.59	0.01\\
27.6	0.01\\
27.61	0.01\\
27.62	0.01\\
27.63	0.01\\
27.64	0.01\\
27.65	0.01\\
27.66	0.01\\
27.67	0.01\\
27.68	0.01\\
27.69	0.01\\
27.7	0.01\\
27.71	0.01\\
27.72	0.01\\
27.73	0.01\\
27.74	0.01\\
27.75	0.01\\
27.76	0.01\\
27.77	0.01\\
27.78	0.01\\
27.79	0.01\\
27.8	0.01\\
27.81	0.01\\
27.82	0.01\\
27.83	0.01\\
27.84	0.01\\
27.85	0.01\\
27.86	0.01\\
27.87	0.01\\
27.88	0.01\\
27.89	0.01\\
27.9	0.01\\
27.91	0.01\\
27.92	0.01\\
27.93	0.01\\
27.94	0.01\\
27.95	0.01\\
27.96	0.01\\
27.97	0.01\\
27.98	0.01\\
27.99	0.01\\
28	0.01\\
28.01	0.01\\
28.02	0.01\\
28.03	0.01\\
28.04	0.01\\
28.05	0.01\\
28.06	0.01\\
28.07	0.01\\
28.08	0.01\\
28.09	0.01\\
28.1	0.01\\
28.11	0.01\\
28.12	0.01\\
28.13	0.01\\
28.14	0.01\\
28.15	0.01\\
28.16	0.01\\
28.17	0.01\\
28.18	0.01\\
28.19	0.01\\
28.2	0.01\\
28.21	0.01\\
28.22	0.01\\
28.23	0.01\\
28.24	0.01\\
28.25	0.01\\
28.26	0.01\\
28.27	0.01\\
28.28	0.01\\
28.29	0.01\\
28.3	0.01\\
28.31	0.01\\
28.32	0.01\\
28.33	0.01\\
28.34	0.01\\
28.35	0.01\\
28.36	0.01\\
28.37	0.01\\
28.38	0.01\\
28.39	0.01\\
28.4	0.01\\
28.41	0.01\\
28.42	0.01\\
28.43	0.01\\
28.44	0.01\\
28.45	0.01\\
28.46	0.01\\
28.47	0.01\\
28.48	0.01\\
28.49	0.01\\
28.5	0.01\\
28.51	0.01\\
28.52	0.01\\
28.53	0.01\\
28.54	0.01\\
28.55	0.01\\
28.56	0.01\\
28.57	0.01\\
28.58	0.01\\
28.59	0.01\\
28.6	0.01\\
28.61	0.01\\
28.62	0.01\\
28.63	0.01\\
28.64	0.01\\
28.65	0.01\\
28.66	0.01\\
28.67	0.01\\
28.68	0.01\\
28.69	0.01\\
28.7	0.01\\
28.71	0.01\\
28.72	0.01\\
28.73	0.01\\
28.74	0.01\\
28.75	0.01\\
28.76	0.01\\
28.77	0.01\\
28.78	0.01\\
28.79	0.01\\
28.8	0.01\\
28.81	0.01\\
28.82	0.01\\
28.83	0.01\\
28.84	0.01\\
28.85	0.01\\
28.86	0.01\\
28.87	0.01\\
28.88	0.01\\
28.89	0.01\\
28.9	0.01\\
28.91	0.01\\
28.92	0.01\\
28.93	0.01\\
28.94	0.01\\
28.95	0.01\\
28.96	0.01\\
28.97	0.01\\
28.98	0.01\\
28.99	0.01\\
29	0.01\\
29.01	0.01\\
29.02	0.01\\
29.03	0.01\\
29.04	0.01\\
29.05	0.01\\
29.06	0.01\\
29.07	0.01\\
29.08	0.01\\
29.09	0.01\\
29.1	0.01\\
29.11	0.01\\
29.12	0.01\\
29.13	0.01\\
29.14	0.01\\
29.15	0.01\\
29.16	0.01\\
29.17	0.01\\
29.18	0.01\\
29.19	0.01\\
29.2	0.01\\
29.21	0.01\\
29.22	0.01\\
29.23	0.01\\
29.24	0.01\\
29.25	0.01\\
29.26	0.01\\
29.27	0.01\\
29.28	0.01\\
29.29	0.01\\
29.3	0.01\\
29.31	0.01\\
29.32	0.01\\
29.33	0.01\\
29.34	0.01\\
29.35	0.01\\
29.36	0.01\\
29.37	0.01\\
29.38	0.01\\
29.39	0.01\\
29.4	0.01\\
29.41	0.01\\
29.42	0.01\\
29.43	0.01\\
29.44	0.01\\
29.45	0.01\\
29.46	0.01\\
29.47	0.01\\
29.48	0.01\\
29.49	0.01\\
29.5	0.01\\
29.51	0.01\\
29.52	0.01\\
29.53	0.01\\
29.54	0.01\\
29.55	0.01\\
29.56	0.01\\
29.57	0.01\\
29.58	0.01\\
29.59	0.01\\
29.6	0.01\\
29.61	0.01\\
29.62	0.01\\
29.63	0.01\\
29.64	0.01\\
29.65	0.01\\
29.66	0.01\\
29.67	0.01\\
29.68	0.01\\
29.69	0.01\\
29.7	0.01\\
29.71	0.01\\
29.72	0.01\\
29.73	0.01\\
29.74	0.01\\
29.75	0.01\\
29.76	0.01\\
29.77	0.01\\
29.78	0.01\\
29.79	0.01\\
29.8	0.01\\
29.81	0.01\\
29.82	0.01\\
29.83	0.01\\
29.84	0.01\\
29.85	0.01\\
29.86	0.01\\
29.87	0.01\\
29.88	0.01\\
29.89	0.01\\
29.9	0.01\\
29.91	0.01\\
29.92	0.01\\
29.93	0.01\\
29.94	0.01\\
29.95	0.01\\
29.96	0.01\\
29.97	0.01\\
29.98	0.01\\
29.99	0.01\\
30	0.01\\
30.01	0.01\\
30.02	0.01\\
30.03	0.01\\
30.04	0.01\\
30.05	0.01\\
30.06	0.01\\
30.07	0.01\\
30.08	0.01\\
30.09	0.01\\
30.1	0.01\\
30.11	0.01\\
30.12	0.01\\
30.13	0.01\\
30.14	0.01\\
30.15	0.01\\
30.16	0.01\\
30.17	0.01\\
30.18	0.01\\
30.19	0.01\\
30.2	0.01\\
30.21	0.01\\
30.22	0.01\\
30.23	0.01\\
30.24	0.01\\
30.25	0.01\\
30.26	0.01\\
30.27	0.01\\
30.28	0.01\\
30.29	0.01\\
30.3	0.01\\
30.31	0.01\\
30.32	0.01\\
30.33	0.01\\
30.34	0.01\\
30.35	0.01\\
30.36	0.01\\
30.37	0.01\\
30.38	0.01\\
30.39	0.01\\
30.4	0.01\\
30.41	0.01\\
30.42	0.01\\
30.43	0.01\\
30.44	0.01\\
30.45	0.01\\
30.46	0.01\\
30.47	0.01\\
30.48	0.01\\
30.49	0.01\\
30.5	0.01\\
30.51	0.01\\
30.52	0.01\\
30.53	0.01\\
30.54	0.01\\
30.55	0.01\\
30.56	0.01\\
30.57	0.01\\
30.58	0.01\\
30.59	0.01\\
30.6	0.01\\
30.61	0.01\\
30.62	0.01\\
30.63	0.01\\
30.64	0.01\\
30.65	0.01\\
30.66	0.01\\
30.67	0.01\\
30.68	0.01\\
30.69	0.01\\
30.7	0.01\\
30.71	0.01\\
30.72	0.01\\
30.73	0.01\\
30.74	0.01\\
30.75	0.01\\
30.76	0.01\\
30.77	0.01\\
30.78	0.01\\
30.79	0.01\\
30.8	0.01\\
30.81	0.01\\
30.82	0.01\\
30.83	0.01\\
30.84	0.01\\
30.85	0.01\\
30.86	0.01\\
30.87	0.01\\
30.88	0.01\\
30.89	0.01\\
30.9	0.01\\
30.91	0.01\\
30.92	0.01\\
30.93	0.01\\
30.94	0.01\\
30.95	0.01\\
30.96	0.01\\
30.97	0.01\\
30.98	0.01\\
30.99	0.01\\
31	0.01\\
31.01	0.01\\
31.02	0.01\\
31.03	0.01\\
31.04	0.01\\
31.05	0.01\\
31.06	0.01\\
31.07	0.01\\
31.08	0.01\\
31.09	0.01\\
31.1	0.01\\
31.11	0.01\\
31.12	0.01\\
31.13	0.01\\
31.14	0.01\\
31.15	0.01\\
31.16	0.01\\
31.17	0.01\\
31.18	0.01\\
31.19	0.01\\
31.2	0.01\\
31.21	0.01\\
31.22	0.01\\
31.23	0.01\\
31.24	0.01\\
31.25	0.01\\
31.26	0.01\\
31.27	0.01\\
31.28	0.01\\
31.29	0.01\\
31.3	0.01\\
31.31	0.01\\
31.32	0.01\\
31.33	0.01\\
31.34	0.01\\
31.35	0.01\\
31.36	0.01\\
31.37	0.01\\
31.38	0.01\\
31.39	0.01\\
31.4	0.01\\
31.41	0.01\\
31.42	0.01\\
31.43	0.01\\
31.44	0.01\\
31.45	0.01\\
31.46	0.01\\
31.47	0.01\\
31.48	0.01\\
31.49	0.01\\
31.5	0.01\\
31.51	0.01\\
31.52	0.01\\
31.53	0.01\\
31.54	0.01\\
31.55	0.01\\
31.56	0.01\\
31.57	0.01\\
31.58	0.01\\
31.59	0.01\\
31.6	0.01\\
31.61	0.01\\
31.62	0.01\\
31.63	0.01\\
31.64	0.01\\
31.65	0.01\\
31.66	0.01\\
31.67	0.01\\
31.68	0.01\\
31.69	0.01\\
31.7	0.01\\
31.71	0.01\\
31.72	0.01\\
31.73	0.01\\
31.74	0.01\\
31.75	0.01\\
31.76	0.01\\
31.77	0.01\\
31.78	0.01\\
31.79	0.01\\
31.8	0.01\\
31.81	0.01\\
31.82	0.01\\
31.83	0.01\\
31.84	0.01\\
31.85	0.01\\
31.86	0.01\\
31.87	0.01\\
31.88	0.01\\
31.89	0.01\\
31.9	0.01\\
31.91	0.01\\
31.92	0.01\\
31.93	0.01\\
31.94	0.01\\
31.95	0.01\\
31.96	0.01\\
31.97	0.01\\
31.98	0.01\\
31.99	0.01\\
32	0.01\\
32.01	0.01\\
32.02	0.01\\
32.03	0.01\\
32.04	0.01\\
32.05	0.01\\
32.06	0.01\\
32.07	0.01\\
32.08	0.01\\
32.09	0.01\\
32.1	0.01\\
32.11	0.01\\
32.12	0.01\\
32.13	0.01\\
32.14	0.01\\
32.15	0.01\\
32.16	0.01\\
32.17	0.01\\
32.18	0.01\\
32.19	0.01\\
32.2	0.01\\
32.21	0.01\\
32.22	0.01\\
32.23	0.01\\
32.24	0.01\\
32.25	0.01\\
32.26	0.01\\
32.27	0.01\\
32.28	0.01\\
32.29	0.01\\
32.3	0.01\\
32.31	0.01\\
32.32	0.01\\
32.33	0.01\\
32.34	0.01\\
32.35	0.01\\
32.36	0.01\\
32.37	0.01\\
32.38	0.01\\
32.39	0.01\\
32.4	0.01\\
32.41	0.01\\
32.42	0.01\\
32.43	0.01\\
32.44	0.01\\
32.45	0.01\\
32.46	0.01\\
32.47	0.01\\
32.48	0.01\\
32.49	0.01\\
32.5	0.01\\
32.51	0.01\\
32.52	0.01\\
32.53	0.01\\
32.54	0.01\\
32.55	0.01\\
32.56	0.01\\
32.57	0.01\\
32.58	0.01\\
32.59	0.01\\
32.6	0.01\\
32.61	0.01\\
32.62	0.01\\
32.63	0.01\\
32.64	0.01\\
32.65	0.01\\
32.66	0.01\\
32.67	0.01\\
32.68	0.01\\
32.69	0.01\\
32.7	0.01\\
32.71	0.01\\
32.72	0.01\\
32.73	0.01\\
32.74	0.01\\
32.75	0.01\\
32.76	0.01\\
32.77	0.01\\
32.78	0.01\\
32.79	0.01\\
32.8	0.01\\
32.81	0.01\\
32.82	0.01\\
32.83	0.01\\
32.84	0.01\\
32.85	0.01\\
32.86	0.01\\
32.87	0.01\\
32.88	0.01\\
32.89	0.01\\
32.9	0.01\\
32.91	0.01\\
32.92	0.01\\
32.93	0.01\\
32.94	0.01\\
32.95	0.01\\
32.96	0.01\\
32.97	0.01\\
32.98	0.01\\
32.99	0.01\\
33	0.01\\
33.01	0.01\\
33.02	0.01\\
33.03	0.01\\
33.04	0.01\\
33.05	0.01\\
33.06	0.01\\
33.07	0.01\\
33.08	0.01\\
33.09	0.01\\
33.1	0.01\\
33.11	0.01\\
33.12	0.01\\
33.13	0.01\\
33.14	0.01\\
33.15	0.01\\
33.16	0.01\\
33.17	0.01\\
33.18	0.01\\
33.19	0.01\\
33.2	0.01\\
33.21	0.01\\
33.22	0.01\\
33.23	0.01\\
33.24	0.01\\
33.25	0.01\\
33.26	0.01\\
33.27	0.01\\
33.28	0.01\\
33.29	0.01\\
33.3	0.01\\
33.31	0.01\\
33.32	0.01\\
33.33	0.01\\
33.34	0.01\\
33.35	0.01\\
33.36	0.01\\
33.37	0.01\\
33.38	0.01\\
33.39	0.01\\
33.4	0.01\\
33.41	0.01\\
33.42	0.01\\
33.43	0.01\\
33.44	0.01\\
33.45	0.01\\
33.46	0.01\\
33.47	0.01\\
33.48	0.01\\
33.49	0.01\\
33.5	0.01\\
33.51	0.01\\
33.52	0.01\\
33.53	0.01\\
33.54	0.01\\
33.55	0.01\\
33.56	0.01\\
33.57	0.01\\
33.58	0.01\\
33.59	0.01\\
33.6	0.01\\
33.61	0.01\\
33.62	0.01\\
33.63	0.01\\
33.64	0.01\\
33.65	0.01\\
33.66	0.01\\
33.67	0.01\\
33.68	0.01\\
33.69	0.01\\
33.7	0.01\\
33.71	0.01\\
33.72	0.01\\
33.73	0.01\\
33.74	0.01\\
33.75	0.01\\
33.76	0.01\\
33.77	0.01\\
33.78	0.01\\
33.79	0.01\\
33.8	0.01\\
33.81	0.01\\
33.82	0.01\\
33.83	0.01\\
33.84	0.01\\
33.85	0.01\\
33.86	0.01\\
33.87	0.01\\
33.88	0.01\\
33.89	0.01\\
33.9	0.01\\
33.91	0.01\\
33.92	0.01\\
33.93	0.01\\
33.94	0.01\\
33.95	0.01\\
33.96	0.01\\
33.97	0.01\\
33.98	0.01\\
33.99	0.01\\
34	0.01\\
34.01	0.01\\
34.02	0.01\\
34.03	0.01\\
34.04	0.01\\
34.05	0.01\\
34.06	0.01\\
34.07	0.01\\
34.08	0.01\\
34.09	0.01\\
34.1	0.01\\
34.11	0.01\\
34.12	0.01\\
34.13	0.01\\
34.14	0.01\\
34.15	0.01\\
34.16	0.01\\
34.17	0.01\\
34.18	0.01\\
34.19	0.01\\
34.2	0.01\\
34.21	0.01\\
34.22	0.01\\
34.23	0.01\\
34.24	0.01\\
34.25	0.01\\
34.26	0.01\\
34.27	0.01\\
34.28	0.01\\
34.29	0.01\\
34.3	0.01\\
34.31	0.01\\
34.32	0.01\\
34.33	0.01\\
34.34	0.01\\
34.35	0.01\\
34.36	0.01\\
34.37	0.01\\
34.38	0.01\\
34.39	0.01\\
34.4	0.01\\
34.41	0.01\\
34.42	0.01\\
34.43	0.01\\
34.44	0.01\\
34.45	0.01\\
34.46	0.01\\
34.47	0.01\\
34.48	0.01\\
34.49	0.01\\
34.5	0.01\\
34.51	0.01\\
34.52	0.01\\
34.53	0.01\\
34.54	0.01\\
34.55	0.01\\
34.56	0.01\\
34.57	0.01\\
34.58	0.01\\
34.59	0.01\\
34.6	0.01\\
34.61	0.01\\
34.62	0.01\\
34.63	0.01\\
34.64	0.01\\
34.65	0.01\\
34.66	0.01\\
34.67	0.01\\
34.68	0.01\\
34.69	0.01\\
34.7	0.01\\
34.71	0.01\\
34.72	0.01\\
34.73	0.01\\
34.74	0.01\\
34.75	0.01\\
34.76	0.01\\
34.77	0.01\\
34.78	0.01\\
34.79	0.01\\
34.8	0.01\\
34.81	0.01\\
34.82	0.01\\
34.83	0.01\\
34.84	0.01\\
34.85	0.01\\
34.86	0.01\\
34.87	0.01\\
34.88	0.01\\
34.89	0.01\\
34.9	0.01\\
34.91	0.01\\
34.92	0.01\\
34.93	0.01\\
34.94	0.01\\
34.95	0.01\\
34.96	0.01\\
34.97	0.01\\
34.98	0.01\\
34.99	0.01\\
35	0.01\\
35.01	0.01\\
35.02	0.01\\
35.03	0.01\\
35.04	0.01\\
35.05	0.01\\
35.06	0.01\\
35.07	0.01\\
35.08	0.01\\
35.09	0.01\\
35.1	0.01\\
35.11	0.01\\
35.12	0.01\\
35.13	0.01\\
35.14	0.01\\
35.15	0.01\\
35.16	0.01\\
35.17	0.01\\
35.18	0.01\\
35.19	0.01\\
35.2	0.01\\
35.21	0.01\\
35.22	0.01\\
35.23	0.01\\
35.24	0.01\\
35.25	0.01\\
35.26	0.01\\
35.27	0.01\\
35.28	0.01\\
35.29	0.01\\
35.3	0.01\\
35.31	0.01\\
35.32	0.01\\
35.33	0.01\\
35.34	0.01\\
35.35	0.01\\
35.36	0.01\\
35.37	0.01\\
35.38	0.01\\
35.39	0.01\\
35.4	0.01\\
35.41	0.01\\
35.42	0.01\\
35.43	0.01\\
35.44	0.01\\
35.45	0.01\\
35.46	0.01\\
35.47	0.01\\
35.48	0.01\\
35.49	0.01\\
35.5	0.01\\
35.51	0.01\\
35.52	0.01\\
35.53	0.01\\
35.54	0.01\\
35.55	0.01\\
35.56	0.01\\
35.57	0.01\\
35.58	0.01\\
35.59	0.01\\
35.6	0.01\\
35.61	0.01\\
35.62	0.01\\
35.63	0.01\\
35.64	0.01\\
35.65	0.01\\
35.66	0.01\\
35.67	0.01\\
35.68	0.01\\
35.69	0.01\\
35.7	0.01\\
35.71	0.01\\
35.72	0.01\\
35.73	0.01\\
35.74	0.01\\
35.75	0.01\\
35.76	0.01\\
35.77	0.01\\
35.78	0.01\\
35.79	0.01\\
35.8	0.01\\
35.81	0.01\\
35.82	0.01\\
35.83	0.01\\
35.84	0.01\\
35.85	0.01\\
35.86	0.01\\
35.87	0.01\\
35.88	0.01\\
35.89	0.01\\
35.9	0.01\\
35.91	0.01\\
35.92	0.01\\
35.93	0.01\\
35.94	0.01\\
35.95	0.01\\
35.96	0.01\\
35.97	0.01\\
35.98	0.01\\
35.99	0.01\\
36	0.01\\
36.01	0.01\\
36.02	0.01\\
36.03	0.01\\
36.04	0.01\\
36.05	0.01\\
36.06	0.01\\
36.07	0.01\\
36.08	0.01\\
36.09	0.01\\
36.1	0.01\\
36.11	0.01\\
36.12	0.01\\
36.13	0.01\\
36.14	0.01\\
36.15	0.01\\
36.16	0.01\\
36.17	0.01\\
36.18	0.01\\
36.19	0.01\\
36.2	0.01\\
36.21	0.01\\
36.22	0.01\\
36.23	0.01\\
36.24	0.01\\
36.25	0.01\\
36.26	0.01\\
36.27	0.01\\
36.28	0.01\\
36.29	0.01\\
36.3	0.01\\
36.31	0.01\\
36.32	0.01\\
36.33	0.01\\
36.34	0.01\\
36.35	0.01\\
36.36	0.01\\
36.37	0.01\\
36.38	0.01\\
36.39	0.01\\
36.4	0.01\\
36.41	0.01\\
36.42	0.01\\
36.43	0.01\\
36.44	0.01\\
36.45	0.01\\
36.46	0.01\\
36.47	0.01\\
36.48	0.01\\
36.49	0.01\\
36.5	0.01\\
36.51	0.01\\
36.52	0.01\\
36.53	0.01\\
36.54	0.01\\
36.55	0.01\\
36.56	0.01\\
36.57	0.01\\
36.58	0.01\\
36.59	0.01\\
36.6	0.01\\
36.61	0.01\\
36.62	0.01\\
36.63	0.01\\
36.64	0.01\\
36.65	0.01\\
36.66	0.01\\
36.67	0.01\\
36.68	0.01\\
36.69	0.01\\
36.7	0.01\\
36.71	0.01\\
36.72	0.01\\
36.73	0.01\\
36.74	0.01\\
36.75	0.01\\
36.76	0.01\\
36.77	0.01\\
36.78	0.01\\
36.79	0.01\\
36.8	0.01\\
36.81	0.01\\
36.82	0.01\\
36.83	0.01\\
36.84	0.01\\
36.85	0.01\\
36.86	0.01\\
36.87	0.01\\
36.88	0.01\\
36.89	0.01\\
36.9	0.01\\
36.91	0.01\\
36.92	0.01\\
36.93	0.01\\
36.94	0.01\\
36.95	0.01\\
36.96	0.01\\
36.97	0.01\\
36.98	0.01\\
36.99	0.01\\
37	0.01\\
37.01	0.01\\
37.02	0.01\\
37.03	0.01\\
37.04	0.01\\
37.05	0.01\\
37.06	0.01\\
37.07	0.01\\
37.08	0.01\\
37.09	0.01\\
37.1	0.01\\
37.11	0.01\\
37.12	0.01\\
37.13	0.01\\
37.14	0.01\\
37.15	0.01\\
37.16	0.01\\
37.17	0.01\\
37.18	0.01\\
37.19	0.01\\
37.2	0.01\\
37.21	0.01\\
37.22	0.01\\
37.23	0.01\\
37.24	0.01\\
37.25	0.01\\
37.26	0.01\\
37.27	0.01\\
37.28	0.01\\
37.29	0.01\\
37.3	0.01\\
37.31	0.01\\
37.32	0.01\\
37.33	0.01\\
37.34	0.01\\
37.35	0.01\\
37.36	0.01\\
37.37	0.01\\
37.38	0.01\\
37.39	0.01\\
37.4	0.01\\
37.41	0.01\\
37.42	0.01\\
37.43	0.01\\
37.44	0.01\\
37.45	0.01\\
37.46	0.01\\
37.47	0.01\\
37.48	0.01\\
37.49	0.01\\
37.5	0.01\\
37.51	0.01\\
37.52	0.01\\
37.53	0.01\\
37.54	0.01\\
37.55	0.01\\
37.56	0.01\\
37.57	0.01\\
37.58	0.01\\
37.59	0.01\\
37.6	0.01\\
37.61	0.01\\
37.62	0.01\\
37.63	0.01\\
37.64	0.01\\
37.65	0.01\\
37.66	0.01\\
37.67	0.01\\
37.68	0.01\\
37.69	0.01\\
37.7	0.01\\
37.71	0.01\\
37.72	0.01\\
37.73	0.01\\
37.74	0.01\\
37.75	0.01\\
37.76	0.01\\
37.77	0.01\\
37.78	0.01\\
37.79	0.01\\
37.8	0.01\\
37.81	0.01\\
37.82	0.01\\
37.83	0.01\\
37.84	0.01\\
37.85	0.01\\
37.86	0.01\\
37.87	0.01\\
37.88	0.01\\
37.89	0.01\\
37.9	0.01\\
37.91	0.01\\
37.92	0.01\\
37.93	0.01\\
37.94	0.01\\
37.95	0.01\\
37.96	0.01\\
37.97	0.01\\
37.98	0.01\\
37.99	0.01\\
38	0.01\\
38.01	0.01\\
38.02	0.01\\
38.03	0.01\\
38.04	0.01\\
38.05	0.01\\
38.06	0.01\\
38.07	0.01\\
38.08	0.01\\
38.09	0.01\\
38.1	0.01\\
38.11	0.01\\
38.12	0.01\\
38.13	0.01\\
38.14	0.01\\
38.15	0.01\\
38.16	0.01\\
38.17	0.01\\
38.18	0.01\\
38.19	0.01\\
38.2	0.01\\
38.21	0.01\\
38.22	0.01\\
38.23	0.01\\
38.24	0.01\\
38.25	0.01\\
38.26	0.01\\
38.27	0.01\\
38.28	0.01\\
38.29	0.01\\
38.3	0.01\\
38.31	0.01\\
38.32	0.01\\
38.33	0.01\\
38.34	0.01\\
38.35	0.01\\
38.36	0.01\\
38.37	0.01\\
38.38	0.01\\
38.39	0.01\\
38.4	0.01\\
38.41	0.01\\
38.42	0.01\\
38.43	0.01\\
38.44	0.01\\
38.45	0.01\\
38.46	0.01\\
38.47	0.01\\
38.48	0.01\\
38.49	0.01\\
38.5	0.01\\
38.51	0.01\\
38.52	0.01\\
38.53	0.01\\
38.54	0.01\\
38.55	0.01\\
38.56	0.01\\
38.57	0.01\\
38.58	0.01\\
38.59	0.01\\
38.6	0.01\\
38.61	0.01\\
38.62	0.01\\
38.63	0.01\\
38.64	0.01\\
38.65	0.01\\
38.66	0.01\\
38.67	0.01\\
38.68	0.01\\
38.69	0.01\\
38.7	0.01\\
38.71	0.01\\
38.72	0.01\\
38.73	0.01\\
38.74	0.01\\
38.75	0.01\\
38.76	0.01\\
38.77	0.01\\
38.78	0.01\\
38.79	0.01\\
38.8	0.01\\
38.81	0.01\\
38.82	0.01\\
38.83	0.01\\
38.84	0.01\\
38.85	0.01\\
38.86	0.01\\
38.87	0.01\\
38.88	0.01\\
38.89	0.01\\
38.9	0.01\\
38.91	0.01\\
38.92	0.01\\
38.93	0.01\\
38.94	0.01\\
38.95	0.01\\
38.96	0.01\\
38.97	0.01\\
38.98	0.01\\
38.99	0.01\\
39	0.01\\
39.01	0.01\\
39.02	0.01\\
39.03	0.01\\
39.04	0.01\\
39.05	0.01\\
39.06	0.01\\
39.07	0.01\\
39.08	0.01\\
39.09	0.01\\
39.1	0.01\\
39.11	0.01\\
39.12	0.01\\
39.13	0.01\\
39.14	0.01\\
39.15	0.01\\
39.16	0.01\\
39.17	0.01\\
39.18	0.01\\
39.19	0.01\\
39.2	0.01\\
39.21	0.01\\
39.22	0.01\\
39.23	0.01\\
39.24	0.01\\
39.25	0.01\\
39.26	0.01\\
39.27	0.01\\
39.28	0.01\\
39.29	0.01\\
39.3	0.01\\
39.31	0.01\\
39.32	0.01\\
39.33	0.01\\
39.34	0.01\\
39.35	0.01\\
39.36	0.01\\
39.37	0.01\\
39.38	0.01\\
39.39	0.01\\
39.4	0.01\\
39.41	0.01\\
39.42	0.01\\
39.43	0.01\\
39.44	0.01\\
39.45	0.01\\
39.46	0.01\\
39.47	0.01\\
39.48	0.01\\
39.49	0.01\\
39.5	0.01\\
39.51	0.01\\
39.52	0.01\\
39.53	0.01\\
39.54	0.01\\
39.55	0.01\\
39.56	0.01\\
39.57	0.01\\
39.58	0.01\\
39.59	0.01\\
39.6	0.01\\
39.61	0.01\\
39.62	0.01\\
39.63	0.01\\
39.64	0.01\\
39.65	0.01\\
39.66	0.01\\
39.67	0.01\\
39.68	0.01\\
39.69	0.01\\
39.7	0.01\\
39.71	0.01\\
39.72	0.01\\
39.73	0.01\\
39.74	0.01\\
39.75	0.01\\
39.76	0.01\\
39.77	0.01\\
39.78	0.01\\
39.79	0.01\\
39.8	0.01\\
39.81	0.01\\
39.82	0.01\\
39.83	0.01\\
39.84	0.01\\
39.85	0.01\\
39.86	0.01\\
39.87	0.01\\
39.88	0.01\\
39.89	0.01\\
39.9	0.01\\
39.91	0.01\\
39.92	0.01\\
39.93	0.01\\
39.94	0.01\\
39.95	0.01\\
39.96	0.01\\
39.97	0.01\\
39.98	0.01\\
39.99	0.01\\
40	0.01\\
40.01	0.01\\
};
\addplot [color=mycolor1,solid,forget plot]
  table[row sep=crcr]{%
40.01	0.01\\
40.02	0.01\\
40.03	0.01\\
40.04	0.01\\
40.05	0.01\\
40.06	0.01\\
40.07	0.01\\
40.08	0.01\\
40.09	0.01\\
40.1	0.01\\
40.11	0.01\\
40.12	0.01\\
40.13	0.01\\
40.14	0.01\\
40.15	0.01\\
40.16	0.01\\
40.17	0.01\\
40.18	0.01\\
40.19	0.01\\
40.2	0.01\\
40.21	0.01\\
40.22	0.01\\
40.23	0.01\\
40.24	0.01\\
40.25	0.01\\
40.26	0.01\\
40.27	0.01\\
40.28	0.01\\
40.29	0.01\\
40.3	0.01\\
40.31	0.01\\
40.32	0.01\\
40.33	0.01\\
40.34	0.01\\
40.35	0.01\\
40.36	0.01\\
40.37	0.01\\
40.38	0.01\\
40.39	0.01\\
40.4	0.01\\
40.41	0.01\\
40.42	0.01\\
40.43	0.01\\
40.44	0.01\\
40.45	0.01\\
40.46	0.01\\
40.47	0.01\\
40.48	0.01\\
40.49	0.01\\
40.5	0.01\\
40.51	0.01\\
40.52	0.01\\
40.53	0.01\\
40.54	0.01\\
40.55	0.01\\
40.56	0.01\\
40.57	0.01\\
40.58	0.01\\
40.59	0.01\\
40.6	0.01\\
40.61	0.01\\
40.62	0.01\\
40.63	0.01\\
40.64	0.01\\
40.65	0.01\\
40.66	0.01\\
40.67	0.01\\
40.68	0.01\\
40.69	0.01\\
40.7	0.01\\
40.71	0.01\\
40.72	0.01\\
40.73	0.01\\
40.74	0.01\\
40.75	0.01\\
40.76	0.01\\
40.77	0.01\\
40.78	0.01\\
40.79	0.01\\
40.8	0.01\\
40.81	0.01\\
40.82	0.01\\
40.83	0.01\\
40.84	0.01\\
40.85	0.01\\
40.86	0.01\\
40.87	0.01\\
40.88	0.01\\
40.89	0.01\\
40.9	0.01\\
40.91	0.01\\
40.92	0.01\\
40.93	0.01\\
40.94	0.01\\
40.95	0.01\\
40.96	0.01\\
40.97	0.01\\
40.98	0.01\\
40.99	0.01\\
41	0.01\\
41.01	0.01\\
41.02	0.01\\
41.03	0.01\\
41.04	0.01\\
41.05	0.01\\
41.06	0.01\\
41.07	0.01\\
41.08	0.01\\
41.09	0.01\\
41.1	0.01\\
41.11	0.01\\
41.12	0.01\\
41.13	0.01\\
41.14	0.01\\
41.15	0.01\\
41.16	0.01\\
41.17	0.01\\
41.18	0.01\\
41.19	0.01\\
41.2	0.01\\
41.21	0.01\\
41.22	0.01\\
41.23	0.01\\
41.24	0.01\\
41.25	0.01\\
41.26	0.01\\
41.27	0.01\\
41.28	0.01\\
41.29	0.01\\
41.3	0.01\\
41.31	0.01\\
41.32	0.01\\
41.33	0.01\\
41.34	0.01\\
41.35	0.01\\
41.36	0.01\\
41.37	0.01\\
41.38	0.01\\
41.39	0.01\\
41.4	0.01\\
41.41	0.01\\
41.42	0.01\\
41.43	0.01\\
41.44	0.01\\
41.45	0.01\\
41.46	0.01\\
41.47	0.01\\
41.48	0.01\\
41.49	0.01\\
41.5	0.01\\
41.51	0.01\\
41.52	0.01\\
41.53	0.01\\
41.54	0.01\\
41.55	0.01\\
41.56	0.01\\
41.57	0.01\\
41.58	0.01\\
41.59	0.01\\
41.6	0.01\\
41.61	0.01\\
41.62	0.01\\
41.63	0.01\\
41.64	0.01\\
41.65	0.01\\
41.66	0.01\\
41.67	0.01\\
41.68	0.01\\
41.69	0.01\\
41.7	0.01\\
41.71	0.01\\
41.72	0.01\\
41.73	0.01\\
41.74	0.01\\
41.75	0.01\\
41.76	0.01\\
41.77	0.01\\
41.78	0.01\\
41.79	0.01\\
41.8	0.01\\
41.81	0.01\\
41.82	0.01\\
41.83	0.01\\
41.84	0.01\\
41.85	0.01\\
41.86	0.01\\
41.87	0.01\\
41.88	0.01\\
41.89	0.01\\
41.9	0.01\\
41.91	0.01\\
41.92	0.01\\
41.93	0.01\\
41.94	0.01\\
41.95	0.01\\
41.96	0.01\\
41.97	0.01\\
41.98	0.01\\
41.99	0.01\\
42	0.01\\
42.01	0.01\\
42.02	0.01\\
42.03	0.01\\
42.04	0.01\\
42.05	0.01\\
42.06	0.01\\
42.07	0.01\\
42.08	0.01\\
42.09	0.01\\
42.1	0.01\\
42.11	0.01\\
42.12	0.01\\
42.13	0.01\\
42.14	0.01\\
42.15	0.01\\
42.16	0.01\\
42.17	0.01\\
42.18	0.01\\
42.19	0.01\\
42.2	0.01\\
42.21	0.01\\
42.22	0.01\\
42.23	0.01\\
42.24	0.01\\
42.25	0.01\\
42.26	0.01\\
42.27	0.01\\
42.28	0.01\\
42.29	0.01\\
42.3	0.01\\
42.31	0.01\\
42.32	0.01\\
42.33	0.01\\
42.34	0.01\\
42.35	0.01\\
42.36	0.01\\
42.37	0.01\\
42.38	0.01\\
42.39	0.01\\
42.4	0.01\\
42.41	0.01\\
42.42	0.01\\
42.43	0.01\\
42.44	0.01\\
42.45	0.01\\
42.46	0.01\\
42.47	0.01\\
42.48	0.01\\
42.49	0.01\\
42.5	0.01\\
42.51	0.01\\
42.52	0.01\\
42.53	0.01\\
42.54	0.01\\
42.55	0.01\\
42.56	0.01\\
42.57	0.01\\
42.58	0.01\\
42.59	0.01\\
42.6	0.01\\
42.61	0.01\\
42.62	0.01\\
42.63	0.01\\
42.64	0.01\\
42.65	0.01\\
42.66	0.01\\
42.67	0.01\\
42.68	0.01\\
42.69	0.01\\
42.7	0.01\\
42.71	0.01\\
42.72	0.01\\
42.73	0.01\\
42.74	0.01\\
42.75	0.01\\
42.76	0.01\\
42.77	0.01\\
42.78	0.01\\
42.79	0.01\\
42.8	0.01\\
42.81	0.01\\
42.82	0.01\\
42.83	0.01\\
42.84	0.01\\
42.85	0.01\\
42.86	0.01\\
42.87	0.01\\
42.88	0.01\\
42.89	0.01\\
42.9	0.01\\
42.91	0.01\\
42.92	0.01\\
42.93	0.01\\
42.94	0.01\\
42.95	0.01\\
42.96	0.01\\
42.97	0.01\\
42.98	0.01\\
42.99	0.01\\
43	0.01\\
43.01	0.01\\
43.02	0.01\\
43.03	0.01\\
43.04	0.01\\
43.05	0.01\\
43.06	0.01\\
43.07	0.01\\
43.08	0.01\\
43.09	0.01\\
43.1	0.01\\
43.11	0.01\\
43.12	0.01\\
43.13	0.01\\
43.14	0.01\\
43.15	0.01\\
43.16	0.01\\
43.17	0.01\\
43.18	0.01\\
43.19	0.01\\
43.2	0.01\\
43.21	0.01\\
43.22	0.01\\
43.23	0.01\\
43.24	0.01\\
43.25	0.01\\
43.26	0.01\\
43.27	0.01\\
43.28	0.01\\
43.29	0.01\\
43.3	0.01\\
43.31	0.01\\
43.32	0.01\\
43.33	0.01\\
43.34	0.01\\
43.35	0.01\\
43.36	0.01\\
43.37	0.01\\
43.38	0.01\\
43.39	0.01\\
43.4	0.01\\
43.41	0.01\\
43.42	0.01\\
43.43	0.01\\
43.44	0.01\\
43.45	0.01\\
43.46	0.01\\
43.47	0.01\\
43.48	0.01\\
43.49	0.01\\
43.5	0.01\\
43.51	0.01\\
43.52	0.01\\
43.53	0.01\\
43.54	0.01\\
43.55	0.01\\
43.56	0.01\\
43.57	0.01\\
43.58	0.01\\
43.59	0.01\\
43.6	0.01\\
43.61	0.01\\
43.62	0.01\\
43.63	0.01\\
43.64	0.01\\
43.65	0.01\\
43.66	0.01\\
43.67	0.01\\
43.68	0.01\\
43.69	0.01\\
43.7	0.01\\
43.71	0.01\\
43.72	0.01\\
43.73	0.01\\
43.74	0.01\\
43.75	0.01\\
43.76	0.01\\
43.77	0.01\\
43.78	0.01\\
43.79	0.01\\
43.8	0.01\\
43.81	0.01\\
43.82	0.01\\
43.83	0.01\\
43.84	0.01\\
43.85	0.01\\
43.86	0.01\\
43.87	0.01\\
43.88	0.01\\
43.89	0.01\\
43.9	0.01\\
43.91	0.01\\
43.92	0.01\\
43.93	0.01\\
43.94	0.01\\
43.95	0.01\\
43.96	0.01\\
43.97	0.01\\
43.98	0.01\\
43.99	0.01\\
44	0.01\\
44.01	0.01\\
44.02	0.01\\
44.03	0.01\\
44.04	0.01\\
44.05	0.01\\
44.06	0.01\\
44.07	0.01\\
44.08	0.01\\
44.09	0.01\\
44.1	0.01\\
44.11	0.01\\
44.12	0.01\\
44.13	0.01\\
44.14	0.01\\
44.15	0.01\\
44.16	0.01\\
44.17	0.01\\
44.18	0.01\\
44.19	0.01\\
44.2	0.01\\
44.21	0.01\\
44.22	0.01\\
44.23	0.01\\
44.24	0.01\\
44.25	0.01\\
44.26	0.01\\
44.27	0.01\\
44.28	0.01\\
44.29	0.01\\
44.3	0.01\\
44.31	0.01\\
44.32	0.01\\
44.33	0.01\\
44.34	0.01\\
44.35	0.01\\
44.36	0.01\\
44.37	0.01\\
44.38	0.01\\
44.39	0.01\\
44.4	0.01\\
44.41	0.01\\
44.42	0.01\\
44.43	0.01\\
44.44	0.01\\
44.45	0.01\\
44.46	0.01\\
44.47	0.01\\
44.48	0.01\\
44.49	0.01\\
44.5	0.01\\
44.51	0.01\\
44.52	0.01\\
44.53	0.01\\
44.54	0.01\\
44.55	0.01\\
44.56	0.01\\
44.57	0.01\\
44.58	0.01\\
44.59	0.01\\
44.6	0.01\\
44.61	0.01\\
44.62	0.01\\
44.63	0.01\\
44.64	0.01\\
44.65	0.01\\
44.66	0.01\\
44.67	0.01\\
44.68	0.01\\
44.69	0.01\\
44.7	0.01\\
44.71	0.01\\
44.72	0.01\\
44.73	0.01\\
44.74	0.01\\
44.75	0.01\\
44.76	0.01\\
44.77	0.01\\
44.78	0.01\\
44.79	0.01\\
44.8	0.01\\
44.81	0.01\\
44.82	0.01\\
44.83	0.01\\
44.84	0.01\\
44.85	0.01\\
44.86	0.01\\
44.87	0.01\\
44.88	0.01\\
44.89	0.01\\
44.9	0.01\\
44.91	0.01\\
44.92	0.01\\
44.93	0.01\\
44.94	0.01\\
44.95	0.01\\
44.96	0.01\\
44.97	0.01\\
44.98	0.01\\
44.99	0.01\\
45	0.01\\
45.01	0.01\\
45.02	0.01\\
45.03	0.01\\
45.04	0.01\\
45.05	0.01\\
45.06	0.01\\
45.07	0.01\\
45.08	0.01\\
45.09	0.01\\
45.1	0.01\\
45.11	0.01\\
45.12	0.01\\
45.13	0.01\\
45.14	0.01\\
45.15	0.01\\
45.16	0.01\\
45.17	0.01\\
45.18	0.01\\
45.19	0.01\\
45.2	0.01\\
45.21	0.01\\
45.22	0.01\\
45.23	0.01\\
45.24	0.01\\
45.25	0.01\\
45.26	0.01\\
45.27	0.01\\
45.28	0.01\\
45.29	0.01\\
45.3	0.01\\
45.31	0.01\\
45.32	0.01\\
45.33	0.01\\
45.34	0.01\\
45.35	0.01\\
45.36	0.01\\
45.37	0.01\\
45.38	0.01\\
45.39	0.01\\
45.4	0.01\\
45.41	0.01\\
45.42	0.01\\
45.43	0.01\\
45.44	0.01\\
45.45	0.01\\
45.46	0.01\\
45.47	0.01\\
45.48	0.01\\
45.49	0.01\\
45.5	0.01\\
45.51	0.01\\
45.52	0.01\\
45.53	0.01\\
45.54	0.01\\
45.55	0.01\\
45.56	0.01\\
45.57	0.01\\
45.58	0.01\\
45.59	0.01\\
45.6	0.01\\
45.61	0.01\\
45.62	0.01\\
45.63	0.01\\
45.64	0.01\\
45.65	0.01\\
45.66	0.01\\
45.67	0.01\\
45.68	0.01\\
45.69	0.01\\
45.7	0.01\\
45.71	0.01\\
45.72	0.01\\
45.73	0.01\\
45.74	0.01\\
45.75	0.01\\
45.76	0.01\\
45.77	0.01\\
45.78	0.01\\
45.79	0.01\\
45.8	0.01\\
45.81	0.01\\
45.82	0.01\\
45.83	0.01\\
45.84	0.01\\
45.85	0.01\\
45.86	0.01\\
45.87	0.01\\
45.88	0.01\\
45.89	0.01\\
45.9	0.01\\
45.91	0.01\\
45.92	0.01\\
45.93	0.01\\
45.94	0.01\\
45.95	0.01\\
45.96	0.01\\
45.97	0.01\\
45.98	0.01\\
45.99	0.01\\
46	0.01\\
46.01	0.01\\
46.02	0.01\\
46.03	0.01\\
46.04	0.01\\
46.05	0.01\\
46.06	0.01\\
46.07	0.01\\
46.08	0.01\\
46.09	0.01\\
46.1	0.01\\
46.11	0.01\\
46.12	0.01\\
46.13	0.01\\
46.14	0.01\\
46.15	0.01\\
46.16	0.01\\
46.17	0.01\\
46.18	0.01\\
46.19	0.01\\
46.2	0.01\\
46.21	0.01\\
46.22	0.01\\
46.23	0.01\\
46.24	0.01\\
46.25	0.01\\
46.26	0.01\\
46.27	0.01\\
46.28	0.01\\
46.29	0.01\\
46.3	0.01\\
46.31	0.01\\
46.32	0.01\\
46.33	0.01\\
46.34	0.01\\
46.35	0.01\\
46.36	0.01\\
46.37	0.01\\
46.38	0.01\\
46.39	0.01\\
46.4	0.01\\
46.41	0.01\\
46.42	0.01\\
46.43	0.01\\
46.44	0.01\\
46.45	0.01\\
46.46	0.01\\
46.47	0.01\\
46.48	0.01\\
46.49	0.01\\
46.5	0.01\\
46.51	0.01\\
46.52	0.01\\
46.53	0.01\\
46.54	0.01\\
46.55	0.01\\
46.56	0.01\\
46.57	0.01\\
46.58	0.01\\
46.59	0.01\\
46.6	0.01\\
46.61	0.01\\
46.62	0.01\\
46.63	0.01\\
46.64	0.01\\
46.65	0.01\\
46.66	0.01\\
46.67	0.01\\
46.68	0.01\\
46.69	0.01\\
46.7	0.01\\
46.71	0.01\\
46.72	0.01\\
46.73	0.01\\
46.74	0.01\\
46.75	0.01\\
46.76	0.01\\
46.77	0.01\\
46.78	0.01\\
46.79	0.01\\
46.8	0.01\\
46.81	0.01\\
46.82	0.01\\
46.83	0.01\\
46.84	0.01\\
46.85	0.01\\
46.86	0.01\\
46.87	0.01\\
46.88	0.01\\
46.89	0.01\\
46.9	0.01\\
46.91	0.01\\
46.92	0.01\\
46.93	0.01\\
46.94	0.01\\
46.95	0.01\\
46.96	0.01\\
46.97	0.01\\
46.98	0.01\\
46.99	0.01\\
47	0.01\\
47.01	0.01\\
47.02	0.01\\
47.03	0.01\\
47.04	0.01\\
47.05	0.01\\
47.06	0.01\\
47.07	0.01\\
47.08	0.01\\
47.09	0.01\\
47.1	0.01\\
47.11	0.01\\
47.12	0.01\\
47.13	0.01\\
47.14	0.01\\
47.15	0.01\\
47.16	0.01\\
47.17	0.01\\
47.18	0.01\\
47.19	0.01\\
47.2	0.01\\
47.21	0.01\\
47.22	0.01\\
47.23	0.01\\
47.24	0.01\\
47.25	0.01\\
47.26	0.01\\
47.27	0.01\\
47.28	0.01\\
47.29	0.01\\
47.3	0.01\\
47.31	0.01\\
47.32	0.01\\
47.33	0.01\\
47.34	0.01\\
47.35	0.01\\
47.36	0.01\\
47.37	0.01\\
47.38	0.01\\
47.39	0.01\\
47.4	0.01\\
47.41	0.01\\
47.42	0.01\\
47.43	0.01\\
47.44	0.01\\
47.45	0.01\\
47.46	0.01\\
47.47	0.01\\
47.48	0.01\\
47.49	0.01\\
47.5	0.01\\
47.51	0.01\\
47.52	0.01\\
47.53	0.01\\
47.54	0.01\\
47.55	0.01\\
47.56	0.01\\
47.57	0.01\\
47.58	0.01\\
47.59	0.01\\
47.6	0.01\\
47.61	0.01\\
47.62	0.01\\
47.63	0.01\\
47.64	0.01\\
47.65	0.01\\
47.66	0.01\\
47.67	0.01\\
47.68	0.01\\
47.69	0.01\\
47.7	0.01\\
47.71	0.01\\
47.72	0.01\\
47.73	0.01\\
47.74	0.01\\
47.75	0.01\\
47.76	0.01\\
47.77	0.01\\
47.78	0.01\\
47.79	0.01\\
47.8	0.01\\
47.81	0.01\\
47.82	0.01\\
47.83	0.01\\
47.84	0.01\\
47.85	0.01\\
47.86	0.01\\
47.87	0.01\\
47.88	0.01\\
47.89	0.01\\
47.9	0.01\\
47.91	0.01\\
47.92	0.01\\
47.93	0.01\\
47.94	0.01\\
47.95	0.01\\
47.96	0.01\\
47.97	0.01\\
47.98	0.01\\
47.99	0.01\\
48	0.01\\
48.01	0.01\\
48.02	0.01\\
48.03	0.01\\
48.04	0.01\\
48.05	0.01\\
48.06	0.01\\
48.07	0.01\\
48.08	0.01\\
48.09	0.01\\
48.1	0.01\\
48.11	0.01\\
48.12	0.01\\
48.13	0.01\\
48.14	0.01\\
48.15	0.01\\
48.16	0.01\\
48.17	0.01\\
48.18	0.01\\
48.19	0.01\\
48.2	0.01\\
48.21	0.01\\
48.22	0.01\\
48.23	0.01\\
48.24	0.01\\
48.25	0.01\\
48.26	0.01\\
48.27	0.01\\
48.28	0.01\\
48.29	0.01\\
48.3	0.01\\
48.31	0.01\\
48.32	0.01\\
48.33	0.01\\
48.34	0.01\\
48.35	0.01\\
48.36	0.01\\
48.37	0.01\\
48.38	0.01\\
48.39	0.01\\
48.4	0.01\\
48.41	0.01\\
48.42	0.01\\
48.43	0.01\\
48.44	0.01\\
48.45	0.01\\
48.46	0.01\\
48.47	0.01\\
48.48	0.01\\
48.49	0.01\\
48.5	0.01\\
48.51	0.01\\
48.52	0.01\\
48.53	0.01\\
48.54	0.01\\
48.55	0.01\\
48.56	0.01\\
48.57	0.01\\
48.58	0.01\\
48.59	0.01\\
48.6	0.01\\
48.61	0.01\\
48.62	0.01\\
48.63	0.01\\
48.64	0.01\\
48.65	0.01\\
48.66	0.01\\
48.67	0.01\\
48.68	0.01\\
48.69	0.01\\
48.7	0.01\\
48.71	0.01\\
48.72	0.01\\
48.73	0.01\\
48.74	0.01\\
48.75	0.01\\
48.76	0.01\\
48.77	0.01\\
48.78	0.01\\
48.79	0.01\\
48.8	0.01\\
48.81	0.01\\
48.82	0.01\\
48.83	0.01\\
48.84	0.01\\
48.85	0.01\\
48.86	0.01\\
48.87	0.01\\
48.88	0.01\\
48.89	0.01\\
48.9	0.01\\
48.91	0.01\\
48.92	0.01\\
48.93	0.01\\
48.94	0.01\\
48.95	0.01\\
48.96	0.01\\
48.97	0.01\\
48.98	0.01\\
48.99	0.01\\
49	0.01\\
49.01	0.01\\
49.02	0.01\\
49.03	0.01\\
49.04	0.01\\
49.05	0.01\\
49.06	0.01\\
49.07	0.01\\
49.08	0.01\\
49.09	0.01\\
49.1	0.01\\
49.11	0.01\\
49.12	0.01\\
49.13	0.01\\
49.14	0.01\\
49.15	0.01\\
49.16	0.01\\
49.17	0.01\\
49.18	0.01\\
49.19	0.01\\
49.2	0.01\\
49.21	0.01\\
49.22	0.01\\
49.23	0.01\\
49.24	0.01\\
49.25	0.01\\
49.26	0.01\\
49.27	0.01\\
49.28	0.01\\
49.29	0.01\\
49.3	0.01\\
49.31	0.01\\
49.32	0.01\\
49.33	0.01\\
49.34	0.01\\
49.35	0.01\\
49.36	0.01\\
49.37	0.01\\
49.38	0.01\\
49.39	0.01\\
49.4	0.01\\
49.41	0.01\\
49.42	0.01\\
49.43	0.01\\
49.44	0.01\\
49.45	0.01\\
49.46	0.01\\
49.47	0.01\\
49.48	0.01\\
49.49	0.01\\
49.5	0.01\\
49.51	0.01\\
49.52	0.01\\
49.53	0.01\\
49.54	0.01\\
49.55	0.01\\
49.56	0.01\\
49.57	0.01\\
49.58	0.01\\
49.59	0.01\\
49.6	0.01\\
49.61	0.01\\
49.62	0.01\\
49.63	0.01\\
49.64	0.01\\
49.65	0.01\\
49.66	0.01\\
49.67	0.01\\
49.68	0.01\\
49.69	0.01\\
49.7	0.01\\
49.71	0.01\\
49.72	0.01\\
49.73	0.01\\
49.74	0.01\\
49.75	0.01\\
49.76	0.01\\
49.77	0.01\\
49.78	0.01\\
49.79	0.01\\
49.8	0.01\\
49.81	0.01\\
49.82	0.01\\
49.83	0.01\\
49.84	0.01\\
49.85	0.01\\
49.86	0.01\\
49.87	0.01\\
49.88	0.01\\
49.89	0.01\\
49.9	0.01\\
49.91	0.01\\
49.92	0.01\\
49.93	0.01\\
49.94	0.01\\
49.95	0.01\\
49.96	0.01\\
49.97	0.01\\
49.98	0.01\\
49.99	0.01\\
50	0.01\\
50.01	0.01\\
50.02	0.01\\
50.03	0.01\\
50.04	0.01\\
50.05	0.01\\
50.06	0.01\\
50.07	0.01\\
50.08	0.01\\
50.09	0.01\\
50.1	0.01\\
50.11	0.01\\
50.12	0.01\\
50.13	0.01\\
50.14	0.01\\
50.15	0.01\\
50.16	0.01\\
50.17	0.01\\
50.18	0.01\\
50.19	0.01\\
50.2	0.01\\
50.21	0.01\\
50.22	0.01\\
50.23	0.01\\
50.24	0.01\\
50.25	0.01\\
50.26	0.01\\
50.27	0.01\\
50.28	0.01\\
50.29	0.01\\
50.3	0.01\\
50.31	0.01\\
50.32	0.01\\
50.33	0.01\\
50.34	0.01\\
50.35	0.01\\
50.36	0.01\\
50.37	0.01\\
50.38	0.01\\
50.39	0.01\\
50.4	0.01\\
50.41	0.01\\
50.42	0.01\\
50.43	0.01\\
50.44	0.01\\
50.45	0.01\\
50.46	0.01\\
50.47	0.01\\
50.48	0.01\\
50.49	0.01\\
50.5	0.01\\
50.51	0.01\\
50.52	0.01\\
50.53	0.01\\
50.54	0.01\\
50.55	0.01\\
50.56	0.01\\
50.57	0.01\\
50.58	0.01\\
50.59	0.01\\
50.6	0.01\\
50.61	0.01\\
50.62	0.01\\
50.63	0.01\\
50.64	0.01\\
50.65	0.01\\
50.66	0.01\\
50.67	0.01\\
50.68	0.01\\
50.69	0.01\\
50.7	0.01\\
50.71	0.01\\
50.72	0.01\\
50.73	0.01\\
50.74	0.01\\
50.75	0.01\\
50.76	0.01\\
50.77	0.01\\
50.78	0.01\\
50.79	0.01\\
50.8	0.01\\
50.81	0.01\\
50.82	0.01\\
50.83	0.01\\
50.84	0.01\\
50.85	0.01\\
50.86	0.01\\
50.87	0.01\\
50.88	0.01\\
50.89	0.01\\
50.9	0.01\\
50.91	0.01\\
50.92	0.01\\
50.93	0.01\\
50.94	0.01\\
50.95	0.01\\
50.96	0.01\\
50.97	0.01\\
50.98	0.01\\
50.99	0.01\\
51	0.01\\
51.01	0.01\\
51.02	0.01\\
51.03	0.01\\
51.04	0.01\\
51.05	0.01\\
51.06	0.01\\
51.07	0.01\\
51.08	0.01\\
51.09	0.01\\
51.1	0.01\\
51.11	0.01\\
51.12	0.01\\
51.13	0.01\\
51.14	0.01\\
51.15	0.01\\
51.16	0.01\\
51.17	0.01\\
51.18	0.01\\
51.19	0.01\\
51.2	0.01\\
51.21	0.01\\
51.22	0.01\\
51.23	0.01\\
51.24	0.01\\
51.25	0.01\\
51.26	0.01\\
51.27	0.01\\
51.28	0.01\\
51.29	0.01\\
51.3	0.01\\
51.31	0.01\\
51.32	0.01\\
51.33	0.01\\
51.34	0.01\\
51.35	0.01\\
51.36	0.01\\
51.37	0.01\\
51.38	0.01\\
51.39	0.01\\
51.4	0.01\\
51.41	0.01\\
51.42	0.01\\
51.43	0.01\\
51.44	0.01\\
51.45	0.01\\
51.46	0.01\\
51.47	0.01\\
51.48	0.01\\
51.49	0.01\\
51.5	0.01\\
51.51	0.01\\
51.52	0.01\\
51.53	0.01\\
51.54	0.01\\
51.55	0.01\\
51.56	0.01\\
51.57	0.01\\
51.58	0.01\\
51.59	0.01\\
51.6	0.01\\
51.61	0.01\\
51.62	0.01\\
51.63	0.01\\
51.64	0.01\\
51.65	0.01\\
51.66	0.01\\
51.67	0.01\\
51.68	0.01\\
51.69	0.01\\
51.7	0.01\\
51.71	0.01\\
51.72	0.01\\
51.73	0.01\\
51.74	0.01\\
51.75	0.01\\
51.76	0.01\\
51.77	0.01\\
51.78	0.01\\
51.79	0.01\\
51.8	0.01\\
51.81	0.01\\
51.82	0.01\\
51.83	0.01\\
51.84	0.01\\
51.85	0.01\\
51.86	0.01\\
51.87	0.01\\
51.88	0.01\\
51.89	0.01\\
51.9	0.01\\
51.91	0.01\\
51.92	0.01\\
51.93	0.01\\
51.94	0.01\\
51.95	0.01\\
51.96	0.01\\
51.97	0.01\\
51.98	0.01\\
51.99	0.01\\
52	0.01\\
52.01	0.01\\
52.02	0.01\\
52.03	0.01\\
52.04	0.01\\
52.05	0.01\\
52.06	0.01\\
52.07	0.01\\
52.08	0.01\\
52.09	0.01\\
52.1	0.01\\
52.11	0.01\\
52.12	0.01\\
52.13	0.01\\
52.14	0.01\\
52.15	0.01\\
52.16	0.01\\
52.17	0.01\\
52.18	0.01\\
52.19	0.01\\
52.2	0.01\\
52.21	0.01\\
52.22	0.01\\
52.23	0.01\\
52.24	0.01\\
52.25	0.01\\
52.26	0.01\\
52.27	0.01\\
52.28	0.01\\
52.29	0.01\\
52.3	0.01\\
52.31	0.01\\
52.32	0.01\\
52.33	0.01\\
52.34	0.01\\
52.35	0.01\\
52.36	0.01\\
52.37	0.01\\
52.38	0.01\\
52.39	0.01\\
52.4	0.01\\
52.41	0.01\\
52.42	0.01\\
52.43	0.01\\
52.44	0.01\\
52.45	0.01\\
52.46	0.01\\
52.47	0.01\\
52.48	0.01\\
52.49	0.01\\
52.5	0.01\\
52.51	0.01\\
52.52	0.01\\
52.53	0.01\\
52.54	0.01\\
52.55	0.01\\
52.56	0.01\\
52.57	0.01\\
52.58	0.01\\
52.59	0.01\\
52.6	0.01\\
52.61	0.01\\
52.62	0.01\\
52.63	0.01\\
52.64	0.01\\
52.65	0.01\\
52.66	0.01\\
52.67	0.01\\
52.68	0.01\\
52.69	0.01\\
52.7	0.01\\
52.71	0.01\\
52.72	0.01\\
52.73	0.01\\
52.74	0.01\\
52.75	0.01\\
52.76	0.01\\
52.77	0.01\\
52.78	0.01\\
52.79	0.01\\
52.8	0.01\\
52.81	0.01\\
52.82	0.01\\
52.83	0.01\\
52.84	0.01\\
52.85	0.01\\
52.86	0.01\\
52.87	0.01\\
52.88	0.01\\
52.89	0.01\\
52.9	0.01\\
52.91	0.01\\
52.92	0.01\\
52.93	0.01\\
52.94	0.01\\
52.95	0.01\\
52.96	0.01\\
52.97	0.01\\
52.98	0.01\\
52.99	0.01\\
53	0.01\\
53.01	0.01\\
53.02	0.01\\
53.03	0.01\\
53.04	0.01\\
53.05	0.01\\
53.06	0.01\\
53.07	0.01\\
53.08	0.01\\
53.09	0.01\\
53.1	0.01\\
53.11	0.01\\
53.12	0.01\\
53.13	0.01\\
53.14	0.01\\
53.15	0.01\\
53.16	0.01\\
53.17	0.01\\
53.18	0.01\\
53.19	0.01\\
53.2	0.01\\
53.21	0.01\\
53.22	0.01\\
53.23	0.01\\
53.24	0.01\\
53.25	0.01\\
53.26	0.01\\
53.27	0.01\\
53.28	0.01\\
53.29	0.01\\
53.3	0.01\\
53.31	0.01\\
53.32	0.01\\
53.33	0.01\\
53.34	0.01\\
53.35	0.01\\
53.36	0.01\\
53.37	0.01\\
53.38	0.01\\
53.39	0.01\\
53.4	0.01\\
53.41	0.01\\
53.42	0.01\\
53.43	0.01\\
53.44	0.01\\
53.45	0.01\\
53.46	0.01\\
53.47	0.01\\
53.48	0.01\\
53.49	0.01\\
53.5	0.01\\
53.51	0.01\\
53.52	0.01\\
53.53	0.01\\
53.54	0.01\\
53.55	0.01\\
53.56	0.01\\
53.57	0.01\\
53.58	0.01\\
53.59	0.01\\
53.6	0.01\\
53.61	0.01\\
53.62	0.01\\
53.63	0.01\\
53.64	0.01\\
53.65	0.01\\
53.66	0.01\\
53.67	0.01\\
53.68	0.01\\
53.69	0.01\\
53.7	0.01\\
53.71	0.01\\
53.72	0.01\\
53.73	0.01\\
53.74	0.01\\
53.75	0.01\\
53.76	0.01\\
53.77	0.01\\
53.78	0.01\\
53.79	0.01\\
53.8	0.01\\
53.81	0.01\\
53.82	0.01\\
53.83	0.01\\
53.84	0.01\\
53.85	0.01\\
53.86	0.01\\
53.87	0.01\\
53.88	0.01\\
53.89	0.01\\
53.9	0.01\\
53.91	0.01\\
53.92	0.01\\
53.93	0.01\\
53.94	0.01\\
53.95	0.01\\
53.96	0.01\\
53.97	0.01\\
53.98	0.01\\
53.99	0.01\\
54	0.01\\
54.01	0.01\\
54.02	0.01\\
54.03	0.01\\
54.04	0.01\\
54.05	0.01\\
54.06	0.01\\
54.07	0.01\\
54.08	0.01\\
54.09	0.01\\
54.1	0.01\\
54.11	0.01\\
54.12	0.01\\
54.13	0.01\\
54.14	0.01\\
54.15	0.01\\
54.16	0.01\\
54.17	0.01\\
54.18	0.01\\
54.19	0.01\\
54.2	0.01\\
54.21	0.01\\
54.22	0.01\\
54.23	0.01\\
54.24	0.01\\
54.25	0.01\\
54.26	0.01\\
54.27	0.01\\
54.28	0.01\\
54.29	0.01\\
54.3	0.01\\
54.31	0.01\\
54.32	0.01\\
54.33	0.01\\
54.34	0.01\\
54.35	0.01\\
54.36	0.01\\
54.37	0.01\\
54.38	0.01\\
54.39	0.01\\
54.4	0.01\\
54.41	0.01\\
54.42	0.01\\
54.43	0.01\\
54.44	0.01\\
54.45	0.01\\
54.46	0.01\\
54.47	0.01\\
54.48	0.01\\
54.49	0.01\\
54.5	0.01\\
54.51	0.01\\
54.52	0.01\\
54.53	0.01\\
54.54	0.01\\
54.55	0.01\\
54.56	0.01\\
54.57	0.01\\
54.58	0.01\\
54.59	0.01\\
54.6	0.01\\
54.61	0.01\\
54.62	0.01\\
54.63	0.01\\
54.64	0.01\\
54.65	0.01\\
54.66	0.01\\
54.67	0.01\\
54.68	0.01\\
54.69	0.01\\
54.7	0.01\\
54.71	0.01\\
54.72	0.01\\
54.73	0.01\\
54.74	0.01\\
54.75	0.01\\
54.76	0.01\\
54.77	0.01\\
54.78	0.01\\
54.79	0.01\\
54.8	0.01\\
54.81	0.01\\
54.82	0.01\\
54.83	0.01\\
54.84	0.01\\
54.85	0.01\\
54.86	0.01\\
54.87	0.01\\
54.88	0.01\\
54.89	0.01\\
54.9	0.01\\
54.91	0.01\\
54.92	0.01\\
54.93	0.01\\
54.94	0.01\\
54.95	0.01\\
54.96	0.01\\
54.97	0.01\\
54.98	0.01\\
54.99	0.01\\
55	0.01\\
55.01	0.01\\
55.02	0.01\\
55.03	0.01\\
55.04	0.01\\
55.05	0.01\\
55.06	0.01\\
55.07	0.01\\
55.08	0.01\\
55.09	0.01\\
55.1	0.01\\
55.11	0.01\\
55.12	0.01\\
55.13	0.01\\
55.14	0.01\\
55.15	0.01\\
55.16	0.01\\
55.17	0.01\\
55.18	0.01\\
55.19	0.01\\
55.2	0.01\\
55.21	0.01\\
55.22	0.01\\
55.23	0.01\\
55.24	0.01\\
55.25	0.01\\
55.26	0.01\\
55.27	0.01\\
55.28	0.01\\
55.29	0.01\\
55.3	0.01\\
55.31	0.01\\
55.32	0.01\\
55.33	0.01\\
55.34	0.01\\
55.35	0.01\\
55.36	0.01\\
55.37	0.01\\
55.38	0.01\\
55.39	0.01\\
55.4	0.01\\
55.41	0.01\\
55.42	0.01\\
55.43	0.01\\
55.44	0.01\\
55.45	0.01\\
55.46	0.01\\
55.47	0.01\\
55.48	0.01\\
55.49	0.01\\
55.5	0.01\\
55.51	0.01\\
55.52	0.01\\
55.53	0.01\\
55.54	0.01\\
55.55	0.01\\
55.56	0.01\\
55.57	0.01\\
55.58	0.01\\
55.59	0.01\\
55.6	0.01\\
55.61	0.01\\
55.62	0.01\\
55.63	0.01\\
55.64	0.01\\
55.65	0.01\\
55.66	0.01\\
55.67	0.01\\
55.68	0.01\\
55.69	0.01\\
55.7	0.01\\
55.71	0.01\\
55.72	0.01\\
55.73	0.01\\
55.74	0.01\\
55.75	0.01\\
55.76	0.01\\
55.77	0.01\\
55.78	0.01\\
55.79	0.01\\
55.8	0.01\\
55.81	0.01\\
55.82	0.01\\
55.83	0.01\\
55.84	0.01\\
55.85	0.01\\
55.86	0.01\\
55.87	0.01\\
55.88	0.01\\
55.89	0.01\\
55.9	0.01\\
55.91	0.01\\
55.92	0.01\\
55.93	0.01\\
55.94	0.01\\
55.95	0.01\\
55.96	0.01\\
55.97	0.01\\
55.98	0.01\\
55.99	0.01\\
56	0.01\\
56.01	0.01\\
56.02	0.01\\
56.03	0.01\\
56.04	0.01\\
56.05	0.01\\
56.06	0.01\\
56.07	0.01\\
56.08	0.01\\
56.09	0.01\\
56.1	0.01\\
56.11	0.01\\
56.12	0.01\\
56.13	0.01\\
56.14	0.01\\
56.15	0.01\\
56.16	0.01\\
56.17	0.01\\
56.18	0.01\\
56.19	0.01\\
56.2	0.01\\
56.21	0.01\\
56.22	0.01\\
56.23	0.01\\
56.24	0.01\\
56.25	0.01\\
56.26	0.01\\
56.27	0.01\\
56.28	0.01\\
56.29	0.01\\
56.3	0.01\\
56.31	0.01\\
56.32	0.01\\
56.33	0.01\\
56.34	0.01\\
56.35	0.01\\
56.36	0.01\\
56.37	0.01\\
56.38	0.01\\
56.39	0.01\\
56.4	0.01\\
56.41	0.01\\
56.42	0.01\\
56.43	0.01\\
56.44	0.01\\
56.45	0.01\\
56.46	0.01\\
56.47	0.01\\
56.48	0.01\\
56.49	0.01\\
56.5	0.01\\
56.51	0.01\\
56.52	0.01\\
56.53	0.01\\
56.54	0.01\\
56.55	0.01\\
56.56	0.01\\
56.57	0.01\\
56.58	0.01\\
56.59	0.01\\
56.6	0.01\\
56.61	0.01\\
56.62	0.01\\
56.63	0.01\\
56.64	0.01\\
56.65	0.01\\
56.66	0.01\\
56.67	0.01\\
56.68	0.01\\
56.69	0.01\\
56.7	0.01\\
56.71	0.01\\
56.72	0.01\\
56.73	0.01\\
56.74	0.01\\
56.75	0.01\\
56.76	0.01\\
56.77	0.01\\
56.78	0.01\\
56.79	0.01\\
56.8	0.01\\
56.81	0.01\\
56.82	0.01\\
56.83	0.01\\
56.84	0.01\\
56.85	0.01\\
56.86	0.01\\
56.87	0.01\\
56.88	0.01\\
56.89	0.01\\
56.9	0.01\\
56.91	0.01\\
56.92	0.01\\
56.93	0.01\\
56.94	0.01\\
56.95	0.01\\
56.96	0.01\\
56.97	0.01\\
56.98	0.01\\
56.99	0.01\\
57	0.01\\
57.01	0.01\\
57.02	0.01\\
57.03	0.01\\
57.04	0.01\\
57.05	0.01\\
57.06	0.01\\
57.07	0.01\\
57.08	0.01\\
57.09	0.01\\
57.1	0.01\\
57.11	0.01\\
57.12	0.01\\
57.13	0.01\\
57.14	0.01\\
57.15	0.01\\
57.16	0.01\\
57.17	0.01\\
57.18	0.01\\
57.19	0.01\\
57.2	0.01\\
57.21	0.01\\
57.22	0.01\\
57.23	0.01\\
57.24	0.01\\
57.25	0.01\\
57.26	0.01\\
57.27	0.01\\
57.28	0.01\\
57.29	0.01\\
57.3	0.01\\
57.31	0.01\\
57.32	0.01\\
57.33	0.01\\
57.34	0.01\\
57.35	0.01\\
57.36	0.01\\
57.37	0.01\\
57.38	0.01\\
57.39	0.01\\
57.4	0.01\\
57.41	0.01\\
57.42	0.01\\
57.43	0.01\\
57.44	0.01\\
57.45	0.01\\
57.46	0.01\\
57.47	0.01\\
57.48	0.01\\
57.49	0.01\\
57.5	0.01\\
57.51	0.01\\
57.52	0.01\\
57.53	0.01\\
57.54	0.01\\
57.55	0.01\\
57.56	0.01\\
57.57	0.01\\
57.58	0.01\\
57.59	0.01\\
57.6	0.01\\
57.61	0.01\\
57.62	0.01\\
57.63	0.01\\
57.64	0.01\\
57.65	0.01\\
57.66	0.01\\
57.67	0.01\\
57.68	0.01\\
57.69	0.01\\
57.7	0.01\\
57.71	0.01\\
57.72	0.01\\
57.73	0.01\\
57.74	0.01\\
57.75	0.01\\
57.76	0.01\\
57.77	0.01\\
57.78	0.01\\
57.79	0.01\\
57.8	0.01\\
57.81	0.01\\
57.82	0.01\\
57.83	0.01\\
57.84	0.01\\
57.85	0.01\\
57.86	0.01\\
57.87	0.01\\
57.88	0.01\\
57.89	0.01\\
57.9	0.01\\
57.91	0.01\\
57.92	0.01\\
57.93	0.01\\
57.94	0.01\\
57.95	0.01\\
57.96	0.01\\
57.97	0.01\\
57.98	0.01\\
57.99	0.01\\
58	0.01\\
58.01	0.01\\
58.02	0.01\\
58.03	0.01\\
58.04	0.01\\
58.05	0.01\\
58.06	0.01\\
58.07	0.01\\
58.08	0.01\\
58.09	0.01\\
58.1	0.01\\
58.11	0.01\\
58.12	0.01\\
58.13	0.01\\
58.14	0.01\\
58.15	0.01\\
58.16	0.01\\
58.17	0.01\\
58.18	0.01\\
58.19	0.01\\
58.2	0.01\\
58.21	0.01\\
58.22	0.01\\
58.23	0.01\\
58.24	0.01\\
58.25	0.01\\
58.26	0.01\\
58.27	0.01\\
58.28	0.01\\
58.29	0.01\\
58.3	0.01\\
58.31	0.01\\
58.32	0.01\\
58.33	0.01\\
58.34	0.01\\
58.35	0.01\\
58.36	0.01\\
58.37	0.01\\
58.38	0.01\\
58.39	0.01\\
58.4	0.01\\
58.41	0.01\\
58.42	0.01\\
58.43	0.01\\
58.44	0.01\\
58.45	0.01\\
58.46	0.01\\
58.47	0.01\\
58.48	0.01\\
58.49	0.01\\
58.5	0.01\\
58.51	0.01\\
58.52	0.01\\
58.53	0.01\\
58.54	0.01\\
58.55	0.01\\
58.56	0.01\\
58.57	0.01\\
58.58	0.01\\
58.59	0.01\\
58.6	0.01\\
58.61	0.01\\
58.62	0.01\\
58.63	0.01\\
58.64	0.01\\
58.65	0.01\\
58.66	0.01\\
58.67	0.01\\
58.68	0.01\\
58.69	0.01\\
58.7	0.01\\
58.71	0.01\\
58.72	0.01\\
58.73	0.01\\
58.74	0.01\\
58.75	0.01\\
58.76	0.01\\
58.77	0.01\\
58.78	0.01\\
58.79	0.01\\
58.8	0.01\\
58.81	0.01\\
58.82	0.01\\
58.83	0.01\\
58.84	0.01\\
58.85	0.01\\
58.86	0.01\\
58.87	0.01\\
58.88	0.01\\
58.89	0.01\\
58.9	0.01\\
58.91	0.01\\
58.92	0.01\\
58.93	0.01\\
58.94	0.01\\
58.95	0.01\\
58.96	0.01\\
58.97	0.01\\
58.98	0.01\\
58.99	0.01\\
59	0.01\\
59.01	0.01\\
59.02	0.01\\
59.03	0.01\\
59.04	0.01\\
59.05	0.01\\
59.06	0.01\\
59.07	0.01\\
59.08	0.01\\
59.09	0.01\\
59.1	0.01\\
59.11	0.01\\
59.12	0.01\\
59.13	0.01\\
59.14	0.01\\
59.15	0.01\\
59.16	0.01\\
59.17	0.01\\
59.18	0.01\\
59.19	0.01\\
59.2	0.01\\
59.21	0.01\\
59.22	0.01\\
59.23	0.01\\
59.24	0.01\\
59.25	0.01\\
59.26	0.01\\
59.27	0.01\\
59.28	0.01\\
59.29	0.01\\
59.3	0.01\\
59.31	0.01\\
59.32	0.01\\
59.33	0.01\\
59.34	0.01\\
59.35	0.01\\
59.36	0.01\\
59.37	0.01\\
59.38	0.01\\
59.39	0.01\\
59.4	0.01\\
59.41	0.01\\
59.42	0.01\\
59.43	0.01\\
59.44	0.01\\
59.45	0.01\\
59.46	0.01\\
59.47	0.01\\
59.48	0.01\\
59.49	0.01\\
59.5	0.01\\
59.51	0.01\\
59.52	0.01\\
59.53	0.01\\
59.54	0.01\\
59.55	0.01\\
59.56	0.01\\
59.57	0.01\\
59.58	0.01\\
59.59	0.01\\
59.6	0.01\\
59.61	0.01\\
59.62	0.01\\
59.63	0.01\\
59.64	0.01\\
59.65	0.01\\
59.66	0.01\\
59.67	0.01\\
59.68	0.01\\
59.69	0.01\\
59.7	0.01\\
59.71	0.01\\
59.72	0.01\\
59.73	0.01\\
59.74	0.01\\
59.75	0.01\\
59.76	0.01\\
59.77	0.01\\
59.78	0.01\\
59.79	0.01\\
59.8	0.01\\
59.81	0.01\\
59.82	0.01\\
59.83	0.01\\
59.84	0.01\\
59.85	0.01\\
59.86	0.01\\
59.87	0.01\\
59.88	0.01\\
59.89	0.01\\
59.9	0.01\\
59.91	0.01\\
59.92	0.01\\
59.93	0.01\\
59.94	0.01\\
59.95	0.01\\
59.96	0.01\\
59.97	0.01\\
59.98	0.01\\
59.99	0.01\\
60	0.01\\
60.01	0.01\\
60.02	0.01\\
60.03	0.01\\
60.04	0.01\\
60.05	0.01\\
60.06	0.01\\
60.07	0.01\\
60.08	0.01\\
60.09	0.01\\
60.1	0.01\\
60.11	0.01\\
60.12	0.01\\
60.13	0.01\\
60.14	0.01\\
60.15	0.01\\
60.16	0.01\\
60.17	0.01\\
60.18	0.01\\
60.19	0.01\\
60.2	0.01\\
60.21	0.01\\
60.22	0.01\\
60.23	0.01\\
60.24	0.01\\
60.25	0.01\\
60.26	0.01\\
60.27	0.01\\
60.28	0.01\\
60.29	0.01\\
60.3	0.01\\
60.31	0.01\\
60.32	0.01\\
60.33	0.01\\
60.34	0.01\\
60.35	0.01\\
60.36	0.01\\
60.37	0.01\\
60.38	0.01\\
60.39	0.01\\
60.4	0.01\\
60.41	0.01\\
60.42	0.01\\
60.43	0.01\\
60.44	0.01\\
60.45	0.01\\
60.46	0.01\\
60.47	0.01\\
60.48	0.01\\
60.49	0.01\\
60.5	0.01\\
60.51	0.01\\
60.52	0.01\\
60.53	0.01\\
60.54	0.01\\
60.55	0.01\\
60.56	0.01\\
60.57	0.01\\
60.58	0.01\\
60.59	0.01\\
60.6	0.01\\
60.61	0.01\\
60.62	0.01\\
60.63	0.01\\
60.64	0.01\\
60.65	0.01\\
60.66	0.01\\
60.67	0.01\\
60.68	0.01\\
60.69	0.01\\
60.7	0.01\\
60.71	0.01\\
60.72	0.01\\
60.73	0.01\\
60.74	0.01\\
60.75	0.01\\
60.76	0.01\\
60.77	0.01\\
60.78	0.01\\
60.79	0.01\\
60.8	0.01\\
60.81	0.01\\
60.82	0.01\\
60.83	0.01\\
60.84	0.01\\
60.85	0.01\\
60.86	0.01\\
60.87	0.01\\
60.88	0.01\\
60.89	0.01\\
60.9	0.01\\
60.91	0.01\\
60.92	0.01\\
60.93	0.01\\
60.94	0.01\\
60.95	0.01\\
60.96	0.01\\
60.97	0.01\\
60.98	0.01\\
60.99	0.01\\
61	0.01\\
61.01	0.01\\
61.02	0.01\\
61.03	0.01\\
61.04	0.01\\
61.05	0.01\\
61.06	0.01\\
61.07	0.01\\
61.08	0.01\\
61.09	0.01\\
61.1	0.01\\
61.11	0.01\\
61.12	0.01\\
61.13	0.01\\
61.14	0.01\\
61.15	0.01\\
61.16	0.01\\
61.17	0.01\\
61.18	0.01\\
61.19	0.01\\
61.2	0.01\\
61.21	0.01\\
61.22	0.01\\
61.23	0.01\\
61.24	0.01\\
61.25	0.01\\
61.26	0.01\\
61.27	0.01\\
61.28	0.01\\
61.29	0.01\\
61.3	0.01\\
61.31	0.01\\
61.32	0.01\\
61.33	0.01\\
61.34	0.01\\
61.35	0.01\\
61.36	0.01\\
61.37	0.01\\
61.38	0.01\\
61.39	0.01\\
61.4	0.01\\
61.41	0.01\\
61.42	0.01\\
61.43	0.01\\
61.44	0.01\\
61.45	0.01\\
61.46	0.01\\
61.47	0.01\\
61.48	0.01\\
61.49	0.01\\
61.5	0.01\\
61.51	0.01\\
61.52	0.01\\
61.53	0.01\\
61.54	0.01\\
61.55	0.01\\
61.56	0.01\\
61.57	0.01\\
61.58	0.01\\
61.59	0.01\\
61.6	0.01\\
61.61	0.01\\
61.62	0.01\\
61.63	0.01\\
61.64	0.01\\
61.65	0.01\\
61.66	0.01\\
61.67	0.01\\
61.68	0.01\\
61.69	0.01\\
61.7	0.01\\
61.71	0.01\\
61.72	0.01\\
61.73	0.01\\
61.74	0.01\\
61.75	0.01\\
61.76	0.01\\
61.77	0.01\\
61.78	0.01\\
61.79	0.01\\
61.8	0.01\\
61.81	0.01\\
61.82	0.01\\
61.83	0.01\\
61.84	0.01\\
61.85	0.01\\
61.86	0.01\\
61.87	0.01\\
61.88	0.01\\
61.89	0.01\\
61.9	0.01\\
61.91	0.01\\
61.92	0.01\\
61.93	0.01\\
61.94	0.01\\
61.95	0.01\\
61.96	0.01\\
61.97	0.01\\
61.98	0.01\\
61.99	0.01\\
62	0.01\\
62.01	0.01\\
62.02	0.01\\
62.03	0.01\\
62.04	0.01\\
62.05	0.01\\
62.06	0.01\\
62.07	0.01\\
62.08	0.01\\
62.09	0.01\\
62.1	0.01\\
62.11	0.01\\
62.12	0.01\\
62.13	0.01\\
62.14	0.01\\
62.15	0.01\\
62.16	0.01\\
62.17	0.01\\
62.18	0.01\\
62.19	0.01\\
62.2	0.01\\
62.21	0.01\\
62.22	0.01\\
62.23	0.01\\
62.24	0.01\\
62.25	0.01\\
62.26	0.01\\
62.27	0.01\\
62.28	0.01\\
62.29	0.01\\
62.3	0.01\\
62.31	0.01\\
62.32	0.01\\
62.33	0.01\\
62.34	0.01\\
62.35	0.01\\
62.36	0.01\\
62.37	0.01\\
62.38	0.01\\
62.39	0.01\\
62.4	0.01\\
62.41	0.01\\
62.42	0.01\\
62.43	0.01\\
62.44	0.01\\
62.45	0.01\\
62.46	0.01\\
62.47	0.01\\
62.48	0.01\\
62.49	0.01\\
62.5	0.01\\
62.51	0.01\\
62.52	0.01\\
62.53	0.01\\
62.54	0.01\\
62.55	0.01\\
62.56	0.01\\
62.57	0.01\\
62.58	0.01\\
62.59	0.01\\
62.6	0.01\\
62.61	0.01\\
62.62	0.01\\
62.63	0.01\\
62.64	0.01\\
62.65	0.01\\
62.66	0.01\\
62.67	0.01\\
62.68	0.01\\
62.69	0.01\\
62.7	0.01\\
62.71	0.01\\
62.72	0.01\\
62.73	0.01\\
62.74	0.01\\
62.75	0.01\\
62.76	0.01\\
62.77	0.01\\
62.78	0.01\\
62.79	0.01\\
62.8	0.01\\
62.81	0.01\\
62.82	0.01\\
62.83	0.01\\
62.84	0.01\\
62.85	0.01\\
62.86	0.01\\
62.87	0.01\\
62.88	0.01\\
62.89	0.01\\
62.9	0.01\\
62.91	0.01\\
62.92	0.01\\
62.93	0.01\\
62.94	0.01\\
62.95	0.01\\
62.96	0.01\\
62.97	0.01\\
62.98	0.01\\
62.99	0.01\\
63	0.01\\
63.01	0.01\\
63.02	0.01\\
63.03	0.01\\
63.04	0.01\\
63.05	0.01\\
63.06	0.01\\
63.07	0.01\\
63.08	0.01\\
63.09	0.01\\
63.1	0.01\\
63.11	0.01\\
63.12	0.01\\
63.13	0.01\\
63.14	0.01\\
63.15	0.01\\
63.16	0.01\\
63.17	0.01\\
63.18	0.01\\
63.19	0.01\\
63.2	0.01\\
63.21	0.01\\
63.22	0.01\\
63.23	0.01\\
63.24	0.01\\
63.25	0.01\\
63.26	0.01\\
63.27	0.01\\
63.28	0.01\\
63.29	0.01\\
63.3	0.01\\
63.31	0.01\\
63.32	0.01\\
63.33	0.01\\
63.34	0.01\\
63.35	0.01\\
63.36	0.01\\
63.37	0.01\\
63.38	0.01\\
63.39	0.01\\
63.4	0.01\\
63.41	0.01\\
63.42	0.01\\
63.43	0.01\\
63.44	0.01\\
63.45	0.01\\
63.46	0.01\\
63.47	0.01\\
63.48	0.01\\
63.49	0.01\\
63.5	0.01\\
63.51	0.01\\
63.52	0.01\\
63.53	0.01\\
63.54	0.01\\
63.55	0.01\\
63.56	0.01\\
63.57	0.01\\
63.58	0.01\\
63.59	0.01\\
63.6	0.01\\
63.61	0.01\\
63.62	0.01\\
63.63	0.01\\
63.64	0.01\\
63.65	0.01\\
63.66	0.01\\
63.67	0.01\\
63.68	0.01\\
63.69	0.01\\
63.7	0.01\\
63.71	0.01\\
63.72	0.01\\
63.73	0.01\\
63.74	0.01\\
63.75	0.01\\
63.76	0.01\\
63.77	0.01\\
63.78	0.01\\
63.79	0.01\\
63.8	0.01\\
63.81	0.01\\
63.82	0.01\\
63.83	0.01\\
63.84	0.01\\
63.85	0.01\\
63.86	0.01\\
63.87	0.01\\
63.88	0.01\\
63.89	0.01\\
63.9	0.01\\
63.91	0.01\\
63.92	0.01\\
63.93	0.01\\
63.94	0.01\\
63.95	0.01\\
63.96	0.01\\
63.97	0.01\\
63.98	0.01\\
63.99	0.01\\
64	0.01\\
64.01	0.01\\
64.02	0.01\\
64.03	0.01\\
64.04	0.01\\
64.05	0.01\\
64.06	0.01\\
64.07	0.01\\
64.08	0.01\\
64.09	0.01\\
64.1	0.01\\
64.11	0.01\\
64.12	0.01\\
64.13	0.01\\
64.14	0.01\\
64.15	0.01\\
64.16	0.01\\
64.17	0.01\\
64.18	0.01\\
64.19	0.01\\
64.2	0.01\\
64.21	0.01\\
64.22	0.01\\
64.23	0.01\\
64.24	0.01\\
64.25	0.01\\
64.26	0.01\\
64.27	0.01\\
64.28	0.01\\
64.29	0.01\\
64.3	0.01\\
64.31	0.01\\
64.32	0.01\\
64.33	0.01\\
64.34	0.01\\
64.35	0.01\\
64.36	0.01\\
64.37	0.01\\
64.38	0.01\\
64.39	0.01\\
64.4	0.01\\
64.41	0.01\\
64.42	0.01\\
64.43	0.01\\
64.44	0.01\\
64.45	0.01\\
64.46	0.01\\
64.47	0.01\\
64.48	0.01\\
64.49	0.01\\
64.5	0.01\\
64.51	0.01\\
64.52	0.01\\
64.53	0.01\\
64.54	0.01\\
64.55	0.01\\
64.56	0.01\\
64.57	0.01\\
64.58	0.01\\
64.59	0.01\\
64.6	0.01\\
64.61	0.01\\
64.62	0.01\\
64.63	0.01\\
64.64	0.01\\
64.65	0.01\\
64.66	0.01\\
64.67	0.01\\
64.68	0.01\\
64.69	0.01\\
64.7	0.01\\
64.71	0.01\\
64.72	0.01\\
64.73	0.01\\
64.74	0.01\\
64.75	0.01\\
64.76	0.01\\
64.77	0.01\\
64.78	0.01\\
64.79	0.01\\
64.8	0.01\\
64.81	0.01\\
64.82	0.01\\
64.83	0.01\\
64.84	0.01\\
64.85	0.01\\
64.86	0.01\\
64.87	0.01\\
64.88	0.01\\
64.89	0.01\\
64.9	0.01\\
64.91	0.01\\
64.92	0.01\\
64.93	0.01\\
64.94	0.01\\
64.95	0.01\\
64.96	0.01\\
64.97	0.01\\
64.98	0.01\\
64.99	0.01\\
65	0.01\\
65.01	0.01\\
65.02	0.01\\
65.03	0.01\\
65.04	0.01\\
65.05	0.01\\
65.06	0.01\\
65.07	0.01\\
65.08	0.01\\
65.09	0.01\\
65.1	0.01\\
65.11	0.01\\
65.12	0.01\\
65.13	0.01\\
65.14	0.01\\
65.15	0.01\\
65.16	0.01\\
65.17	0.01\\
65.18	0.01\\
65.19	0.01\\
65.2	0.01\\
65.21	0.01\\
65.22	0.01\\
65.23	0.01\\
65.24	0.01\\
65.25	0.01\\
65.26	0.01\\
65.27	0.01\\
65.28	0.01\\
65.29	0.01\\
65.3	0.01\\
65.31	0.01\\
65.32	0.01\\
65.33	0.01\\
65.34	0.01\\
65.35	0.01\\
65.36	0.01\\
65.37	0.01\\
65.38	0.01\\
65.39	0.01\\
65.4	0.01\\
65.41	0.01\\
65.42	0.01\\
65.43	0.01\\
65.44	0.01\\
65.45	0.01\\
65.46	0.01\\
65.47	0.01\\
65.48	0.01\\
65.49	0.01\\
65.5	0.01\\
65.51	0.01\\
65.52	0.01\\
65.53	0.01\\
65.54	0.01\\
65.55	0.01\\
65.56	0.01\\
65.57	0.01\\
65.58	0.01\\
65.59	0.01\\
65.6	0.01\\
65.61	0.01\\
65.62	0.01\\
65.63	0.01\\
65.64	0.01\\
65.65	0.01\\
65.66	0.01\\
65.67	0.01\\
65.68	0.01\\
65.69	0.01\\
65.7	0.01\\
65.71	0.01\\
65.72	0.01\\
65.73	0.01\\
65.74	0.01\\
65.75	0.01\\
65.76	0.01\\
65.77	0.01\\
65.78	0.01\\
65.79	0.01\\
65.8	0.01\\
65.81	0.01\\
65.82	0.01\\
65.83	0.01\\
65.84	0.01\\
65.85	0.01\\
65.86	0.01\\
65.87	0.01\\
65.88	0.01\\
65.89	0.01\\
65.9	0.01\\
65.91	0.01\\
65.92	0.01\\
65.93	0.01\\
65.94	0.01\\
65.95	0.01\\
65.96	0.01\\
65.97	0.01\\
65.98	0.01\\
65.99	0.01\\
66	0.01\\
66.01	0.01\\
66.02	0.01\\
66.03	0.01\\
66.04	0.01\\
66.05	0.01\\
66.06	0.01\\
66.07	0.01\\
66.08	0.01\\
66.09	0.01\\
66.1	0.01\\
66.11	0.01\\
66.12	0.01\\
66.13	0.01\\
66.14	0.01\\
66.15	0.01\\
66.16	0.01\\
66.17	0.01\\
66.18	0.01\\
66.19	0.01\\
66.2	0.01\\
66.21	0.01\\
66.22	0.01\\
66.23	0.01\\
66.24	0.01\\
66.25	0.01\\
66.26	0.01\\
66.27	0.01\\
66.28	0.01\\
66.29	0.01\\
66.3	0.01\\
66.31	0.01\\
66.32	0.01\\
66.33	0.01\\
66.34	0.01\\
66.35	0.01\\
66.36	0.01\\
66.37	0.01\\
66.38	0.01\\
66.39	0.01\\
66.4	0.01\\
66.41	0.01\\
66.42	0.01\\
66.43	0.01\\
66.44	0.01\\
66.45	0.01\\
66.46	0.01\\
66.47	0.01\\
66.48	0.01\\
66.49	0.01\\
66.5	0.01\\
66.51	0.01\\
66.52	0.01\\
66.53	0.01\\
66.54	0.01\\
66.55	0.01\\
66.56	0.01\\
66.57	0.01\\
66.58	0.01\\
66.59	0.01\\
66.6	0.01\\
66.61	0.01\\
66.62	0.01\\
66.63	0.01\\
66.64	0.01\\
66.65	0.01\\
66.66	0.01\\
66.67	0.01\\
66.68	0.01\\
66.69	0.01\\
66.7	0.01\\
66.71	0.01\\
66.72	0.01\\
66.73	0.01\\
66.74	0.01\\
66.75	0.01\\
66.76	0.01\\
66.77	0.01\\
66.78	0.01\\
66.79	0.01\\
66.8	0.01\\
66.81	0.01\\
66.82	0.01\\
66.83	0.01\\
66.84	0.01\\
66.85	0.01\\
66.86	0.01\\
66.87	0.01\\
66.88	0.01\\
66.89	0.01\\
66.9	0.01\\
66.91	0.01\\
66.92	0.01\\
66.93	0.01\\
66.94	0.01\\
66.95	0.01\\
66.96	0.01\\
66.97	0.01\\
66.98	0.01\\
66.99	0.01\\
67	0.01\\
67.01	0.01\\
67.02	0.01\\
67.03	0.01\\
67.04	0.01\\
67.05	0.01\\
67.06	0.01\\
67.07	0.01\\
67.08	0.01\\
67.09	0.01\\
67.1	0.01\\
67.11	0.01\\
67.12	0.01\\
67.13	0.01\\
67.14	0.01\\
67.15	0.01\\
67.16	0.01\\
67.17	0.01\\
67.18	0.01\\
67.19	0.01\\
67.2	0.01\\
67.21	0.01\\
67.22	0.01\\
67.23	0.01\\
67.24	0.01\\
67.25	0.01\\
67.26	0.01\\
67.27	0.01\\
67.28	0.01\\
67.29	0.01\\
67.3	0.01\\
67.31	0.01\\
67.32	0.01\\
67.33	0.01\\
67.34	0.01\\
67.35	0.01\\
67.36	0.01\\
67.37	0.01\\
67.38	0.01\\
67.39	0.01\\
67.4	0.01\\
67.41	0.01\\
67.42	0.01\\
67.43	0.01\\
67.44	0.01\\
67.45	0.01\\
67.46	0.01\\
67.47	0.01\\
67.48	0.01\\
67.49	0.01\\
67.5	0.01\\
67.51	0.01\\
67.52	0.01\\
67.53	0.01\\
67.54	0.01\\
67.55	0.01\\
67.56	0.01\\
67.57	0.01\\
67.58	0.01\\
67.59	0.01\\
67.6	0.01\\
67.61	0.01\\
67.62	0.01\\
67.63	0.01\\
67.64	0.01\\
67.65	0.01\\
67.66	0.01\\
67.67	0.01\\
67.68	0.01\\
67.69	0.01\\
67.7	0.01\\
67.71	0.01\\
67.72	0.01\\
67.73	0.01\\
67.74	0.01\\
67.75	0.01\\
67.76	0.01\\
67.77	0.01\\
67.78	0.01\\
67.79	0.01\\
67.8	0.01\\
67.81	0.01\\
67.82	0.01\\
67.83	0.01\\
67.84	0.01\\
67.85	0.01\\
67.86	0.01\\
67.87	0.01\\
67.88	0.01\\
67.89	0.01\\
67.9	0.01\\
67.91	0.01\\
67.92	0.01\\
67.93	0.01\\
67.94	0.01\\
67.95	0.01\\
67.96	0.01\\
67.97	0.01\\
67.98	0.01\\
67.99	0.01\\
68	0.01\\
68.01	0.01\\
68.02	0.01\\
68.03	0.01\\
68.04	0.01\\
68.05	0.01\\
68.06	0.01\\
68.07	0.01\\
68.08	0.01\\
68.09	0.01\\
68.1	0.01\\
68.11	0.01\\
68.12	0.01\\
68.13	0.01\\
68.14	0.01\\
68.15	0.01\\
68.16	0.01\\
68.17	0.01\\
68.18	0.01\\
68.19	0.01\\
68.2	0.01\\
68.21	0.01\\
68.22	0.01\\
68.23	0.01\\
68.24	0.01\\
68.25	0.01\\
68.26	0.01\\
68.27	0.01\\
68.28	0.01\\
68.29	0.01\\
68.3	0.01\\
68.31	0.01\\
68.32	0.01\\
68.33	0.01\\
68.34	0.01\\
68.35	0.01\\
68.36	0.01\\
68.37	0.01\\
68.38	0.01\\
68.39	0.01\\
68.4	0.01\\
68.41	0.01\\
68.42	0.01\\
68.43	0.01\\
68.44	0.01\\
68.45	0.01\\
68.46	0.01\\
68.47	0.01\\
68.48	0.01\\
68.49	0.01\\
68.5	0.01\\
68.51	0.01\\
68.52	0.01\\
68.53	0.01\\
68.54	0.01\\
68.55	0.01\\
68.56	0.01\\
68.57	0.01\\
68.58	0.01\\
68.59	0.01\\
68.6	0.01\\
68.61	0.01\\
68.62	0.01\\
68.63	0.01\\
68.64	0.01\\
68.65	0.01\\
68.66	0.01\\
68.67	0.01\\
68.68	0.01\\
68.69	0.01\\
68.7	0.01\\
68.71	0.01\\
68.72	0.01\\
68.73	0.01\\
68.74	0.01\\
68.75	0.01\\
68.76	0.01\\
68.77	0.01\\
68.78	0.01\\
68.79	0.01\\
68.8	0.01\\
68.81	0.01\\
68.82	0.01\\
68.83	0.01\\
68.84	0.01\\
68.85	0.01\\
68.86	0.01\\
68.87	0.01\\
68.88	0.01\\
68.89	0.01\\
68.9	0.01\\
68.91	0.01\\
68.92	0.01\\
68.93	0.01\\
68.94	0.01\\
68.95	0.01\\
68.96	0.01\\
68.97	0.01\\
68.98	0.01\\
68.99	0.01\\
69	0.01\\
69.01	0.01\\
69.02	0.01\\
69.03	0.01\\
69.04	0.01\\
69.05	0.01\\
69.06	0.01\\
69.07	0.01\\
69.08	0.01\\
69.09	0.01\\
69.1	0.01\\
69.11	0.01\\
69.12	0.01\\
69.13	0.01\\
69.14	0.01\\
69.15	0.01\\
69.16	0.01\\
69.17	0.01\\
69.18	0.01\\
69.19	0.01\\
69.2	0.01\\
69.21	0.01\\
69.22	0.01\\
69.23	0.01\\
69.24	0.01\\
69.25	0.01\\
69.26	0.01\\
69.27	0.01\\
69.28	0.01\\
69.29	0.01\\
69.3	0.01\\
69.31	0.01\\
69.32	0.01\\
69.33	0.01\\
69.34	0.01\\
69.35	0.01\\
69.36	0.01\\
69.37	0.01\\
69.38	0.01\\
69.39	0.01\\
69.4	0.01\\
69.41	0.01\\
69.42	0.01\\
69.43	0.01\\
69.44	0.01\\
69.45	0.01\\
69.46	0.01\\
69.47	0.01\\
69.48	0.01\\
69.49	0.01\\
69.5	0.01\\
69.51	0.01\\
69.52	0.01\\
69.53	0.01\\
69.54	0.01\\
69.55	0.01\\
69.56	0.01\\
69.57	0.01\\
69.58	0.01\\
69.59	0.01\\
69.6	0.01\\
69.61	0.01\\
69.62	0.01\\
69.63	0.01\\
69.64	0.01\\
69.65	0.01\\
69.66	0.01\\
69.67	0.01\\
69.68	0.01\\
69.69	0.01\\
69.7	0.01\\
69.71	0.01\\
69.72	0.01\\
69.73	0.01\\
69.74	0.01\\
69.75	0.01\\
69.76	0.01\\
69.77	0.01\\
69.78	0.01\\
69.79	0.01\\
69.8	0.01\\
69.81	0.01\\
69.82	0.01\\
69.83	0.01\\
69.84	0.01\\
69.85	0.01\\
69.86	0.01\\
69.87	0.01\\
69.88	0.01\\
69.89	0.01\\
69.9	0.01\\
69.91	0.01\\
69.92	0.01\\
69.93	0.01\\
69.94	0.01\\
69.95	0.01\\
69.96	0.01\\
69.97	0.01\\
69.98	0.01\\
69.99	0.01\\
70	0.01\\
70.01	0.01\\
70.02	0.01\\
70.03	0.01\\
70.04	0.01\\
70.05	0.01\\
70.06	0.01\\
70.07	0.01\\
70.08	0.01\\
70.09	0.01\\
70.1	0.01\\
70.11	0.01\\
70.12	0.01\\
70.13	0.01\\
70.14	0.01\\
70.15	0.01\\
70.16	0.01\\
70.17	0.01\\
70.18	0.01\\
70.19	0.01\\
70.2	0.01\\
70.21	0.01\\
70.22	0.01\\
70.23	0.01\\
70.24	0.01\\
70.25	0.01\\
70.26	0.01\\
70.27	0.01\\
70.28	0.01\\
70.29	0.01\\
70.3	0.01\\
70.31	0.01\\
70.32	0.01\\
70.33	0.01\\
70.34	0.01\\
70.35	0.01\\
70.36	0.01\\
70.37	0.01\\
70.38	0.01\\
70.39	0.01\\
70.4	0.01\\
70.41	0.01\\
70.42	0.01\\
70.43	0.01\\
70.44	0.01\\
70.45	0.01\\
70.46	0.01\\
70.47	0.01\\
70.48	0.01\\
70.49	0.01\\
70.5	0.01\\
70.51	0.01\\
70.52	0.01\\
70.53	0.01\\
70.54	0.01\\
70.55	0.01\\
70.56	0.01\\
70.57	0.01\\
70.58	0.01\\
70.59	0.01\\
70.6	0.01\\
70.61	0.01\\
70.62	0.01\\
70.63	0.01\\
70.64	0.01\\
70.65	0.01\\
70.66	0.01\\
70.67	0.01\\
70.68	0.01\\
70.69	0.01\\
70.7	0.01\\
70.71	0.01\\
70.72	0.01\\
70.73	0.01\\
70.74	0.01\\
70.75	0.01\\
70.76	0.01\\
70.77	0.01\\
70.78	0.01\\
70.79	0.01\\
70.8	0.01\\
70.81	0.01\\
70.82	0.01\\
70.83	0.01\\
70.84	0.01\\
70.85	0.01\\
70.86	0.01\\
70.87	0.01\\
70.88	0.01\\
70.89	0.01\\
70.9	0.01\\
70.91	0.01\\
70.92	0.01\\
70.93	0.01\\
70.94	0.01\\
70.95	0.01\\
70.96	0.01\\
70.97	0.01\\
70.98	0.01\\
70.99	0.01\\
71	0.01\\
71.01	0.01\\
71.02	0.01\\
71.03	0.01\\
71.04	0.01\\
71.05	0.01\\
71.06	0.01\\
71.07	0.01\\
71.08	0.01\\
71.09	0.01\\
71.1	0.01\\
71.11	0.01\\
71.12	0.01\\
71.13	0.01\\
71.14	0.01\\
71.15	0.01\\
71.16	0.01\\
71.17	0.01\\
71.18	0.01\\
71.19	0.01\\
71.2	0.01\\
71.21	0.01\\
71.22	0.01\\
71.23	0.01\\
71.24	0.01\\
71.25	0.01\\
71.26	0.01\\
71.27	0.01\\
71.28	0.01\\
71.29	0.01\\
71.3	0.01\\
71.31	0.01\\
71.32	0.01\\
71.33	0.01\\
71.34	0.01\\
71.35	0.01\\
71.36	0.01\\
71.37	0.01\\
71.38	0.01\\
71.39	0.01\\
71.4	0.01\\
71.41	0.01\\
71.42	0.01\\
71.43	0.01\\
71.44	0.01\\
71.45	0.01\\
71.46	0.01\\
71.47	0.01\\
71.48	0.01\\
71.49	0.01\\
71.5	0.01\\
71.51	0.01\\
71.52	0.01\\
71.53	0.01\\
71.54	0.01\\
71.55	0.01\\
71.56	0.01\\
71.57	0.01\\
71.58	0.01\\
71.59	0.01\\
71.6	0.01\\
71.61	0.01\\
71.62	0.01\\
71.63	0.01\\
71.64	0.01\\
71.65	0.01\\
71.66	0.01\\
71.67	0.01\\
71.68	0.01\\
71.69	0.01\\
71.7	0.01\\
71.71	0.01\\
71.72	0.01\\
71.73	0.01\\
71.74	0.01\\
71.75	0.01\\
71.76	0.01\\
71.77	0.01\\
71.78	0.01\\
71.79	0.01\\
71.8	0.01\\
71.81	0.01\\
71.82	0.01\\
71.83	0.01\\
71.84	0.01\\
71.85	0.01\\
71.86	0.01\\
71.87	0.01\\
71.88	0.01\\
71.89	0.01\\
71.9	0.01\\
71.91	0.01\\
71.92	0.01\\
71.93	0.01\\
71.94	0.01\\
71.95	0.01\\
71.96	0.01\\
71.97	0.01\\
71.98	0.01\\
71.99	0.01\\
72	0.01\\
72.01	0.01\\
72.02	0.01\\
72.03	0.01\\
72.04	0.01\\
72.05	0.01\\
72.06	0.01\\
72.07	0.01\\
72.08	0.01\\
72.09	0.01\\
72.1	0.01\\
72.11	0.01\\
72.12	0.01\\
72.13	0.01\\
72.14	0.01\\
72.15	0.01\\
72.16	0.01\\
72.17	0.01\\
72.18	0.01\\
72.19	0.01\\
72.2	0.01\\
72.21	0.01\\
72.22	0.01\\
72.23	0.01\\
72.24	0.01\\
72.25	0.01\\
72.26	0.01\\
72.27	0.01\\
72.28	0.01\\
72.29	0.01\\
72.3	0.01\\
72.31	0.01\\
72.32	0.01\\
72.33	0.01\\
72.34	0.01\\
72.35	0.01\\
72.36	0.01\\
72.37	0.01\\
72.38	0.01\\
72.39	0.01\\
72.4	0.01\\
72.41	0.01\\
72.42	0.01\\
72.43	0.01\\
72.44	0.01\\
72.45	0.01\\
72.46	0.01\\
72.47	0.01\\
72.48	0.01\\
72.49	0.01\\
72.5	0.01\\
72.51	0.01\\
72.52	0.01\\
72.53	0.01\\
72.54	0.01\\
72.55	0.01\\
72.56	0.01\\
72.57	0.01\\
72.58	0.01\\
72.59	0.01\\
72.6	0.01\\
72.61	0.01\\
72.62	0.01\\
72.63	0.01\\
72.64	0.01\\
72.65	0.01\\
72.66	0.01\\
72.67	0.01\\
72.68	0.01\\
72.69	0.01\\
72.7	0.01\\
72.71	0.01\\
72.72	0.01\\
72.73	0.01\\
72.74	0.01\\
72.75	0.01\\
72.76	0.01\\
72.77	0.01\\
72.78	0.01\\
72.79	0.01\\
72.8	0.01\\
72.81	0.01\\
72.82	0.01\\
72.83	0.01\\
72.84	0.01\\
72.85	0.01\\
72.86	0.01\\
72.87	0.01\\
72.88	0.01\\
72.89	0.01\\
72.9	0.01\\
72.91	0.01\\
72.92	0.01\\
72.93	0.01\\
72.94	0.01\\
72.95	0.01\\
72.96	0.01\\
72.97	0.01\\
72.98	0.01\\
72.99	0.01\\
73	0.01\\
73.01	0.01\\
73.02	0.01\\
73.03	0.01\\
73.04	0.01\\
73.05	0.01\\
73.06	0.01\\
73.07	0.01\\
73.08	0.01\\
73.09	0.01\\
73.1	0.01\\
73.11	0.01\\
73.12	0.01\\
73.13	0.01\\
73.14	0.01\\
73.15	0.01\\
73.16	0.01\\
73.17	0.01\\
73.18	0.01\\
73.19	0.01\\
73.2	0.01\\
73.21	0.01\\
73.22	0.01\\
73.23	0.01\\
73.24	0.01\\
73.25	0.01\\
73.26	0.01\\
73.27	0.01\\
73.28	0.01\\
73.29	0.01\\
73.3	0.01\\
73.31	0.01\\
73.32	0.01\\
73.33	0.01\\
73.34	0.01\\
73.35	0.01\\
73.36	0.01\\
73.37	0.01\\
73.38	0.01\\
73.39	0.01\\
73.4	0.01\\
73.41	0.01\\
73.42	0.01\\
73.43	0.01\\
73.44	0.01\\
73.45	0.01\\
73.46	0.01\\
73.47	0.01\\
73.48	0.01\\
73.49	0.01\\
73.5	0.01\\
73.51	0.01\\
73.52	0.01\\
73.53	0.01\\
73.54	0.01\\
73.55	0.01\\
73.56	0.01\\
73.57	0.01\\
73.58	0.01\\
73.59	0.01\\
73.6	0.01\\
73.61	0.01\\
73.62	0.01\\
73.63	0.01\\
73.64	0.01\\
73.65	0.01\\
73.66	0.01\\
73.67	0.01\\
73.68	0.01\\
73.69	0.01\\
73.7	0.01\\
73.71	0.01\\
73.72	0.01\\
73.73	0.01\\
73.74	0.01\\
73.75	0.01\\
73.76	0.01\\
73.77	0.01\\
73.78	0.01\\
73.79	0.01\\
73.8	0.01\\
73.81	0.01\\
73.82	0.01\\
73.83	0.01\\
73.84	0.01\\
73.85	0.01\\
73.86	0.01\\
73.87	0.01\\
73.88	0.01\\
73.89	0.01\\
73.9	0.01\\
73.91	0.01\\
73.92	0.01\\
73.93	0.01\\
73.94	0.01\\
73.95	0.01\\
73.96	0.01\\
73.97	0.01\\
73.98	0.01\\
73.99	0.01\\
74	0.01\\
74.01	0.01\\
74.02	0.01\\
74.03	0.01\\
74.04	0.01\\
74.05	0.01\\
74.06	0.01\\
74.07	0.01\\
74.08	0.01\\
74.09	0.01\\
74.1	0.01\\
74.11	0.01\\
74.12	0.01\\
74.13	0.01\\
74.14	0.01\\
74.15	0.01\\
74.16	0.01\\
74.17	0.01\\
74.18	0.01\\
74.19	0.01\\
74.2	0.01\\
74.21	0.01\\
74.22	0.01\\
74.23	0.01\\
74.24	0.01\\
74.25	0.01\\
74.26	0.01\\
74.27	0.01\\
74.28	0.01\\
74.29	0.01\\
74.3	0.01\\
74.31	0.01\\
74.32	0.01\\
74.33	0.01\\
74.34	0.01\\
74.35	0.01\\
74.36	0.01\\
74.37	0.01\\
74.38	0.01\\
74.39	0.01\\
74.4	0.01\\
74.41	0.01\\
74.42	0.01\\
74.43	0.01\\
74.44	0.01\\
74.45	0.01\\
74.46	0.01\\
74.47	0.01\\
74.48	0.01\\
74.49	0.01\\
74.5	0.01\\
74.51	0.01\\
74.52	0.01\\
74.53	0.01\\
74.54	0.01\\
74.55	0.01\\
74.56	0.01\\
74.57	0.01\\
74.58	0.01\\
74.59	0.01\\
74.6	0.01\\
74.61	0.01\\
74.62	0.01\\
74.63	0.01\\
74.64	0.01\\
74.65	0.01\\
74.66	0.01\\
74.67	0.01\\
74.68	0.01\\
74.69	0.01\\
74.7	0.01\\
74.71	0.01\\
74.72	0.01\\
74.73	0.01\\
74.74	0.01\\
74.75	0.01\\
74.76	0.01\\
74.77	0.01\\
74.78	0.01\\
74.79	0.01\\
74.8	0.01\\
74.81	0.01\\
74.82	0.01\\
74.83	0.01\\
74.84	0.01\\
74.85	0.01\\
74.86	0.01\\
74.87	0.01\\
74.88	0.01\\
74.89	0.01\\
74.9	0.01\\
74.91	0.01\\
74.92	0.01\\
74.93	0.01\\
74.94	0.01\\
74.95	0.01\\
74.96	0.01\\
74.97	0.01\\
74.98	0.01\\
74.99	0.01\\
75	0.01\\
75.01	0.01\\
75.02	0.01\\
75.03	0.01\\
75.04	0.01\\
75.05	0.01\\
75.06	0.01\\
75.07	0.01\\
75.08	0.01\\
75.09	0.01\\
75.1	0.01\\
75.11	0.01\\
75.12	0.01\\
75.13	0.01\\
75.14	0.01\\
75.15	0.01\\
75.16	0.01\\
75.17	0.01\\
75.18	0.01\\
75.19	0.01\\
75.2	0.01\\
75.21	0.01\\
75.22	0.01\\
75.23	0.01\\
75.24	0.01\\
75.25	0.01\\
75.26	0.01\\
75.27	0.01\\
75.28	0.01\\
75.29	0.01\\
75.3	0.01\\
75.31	0.01\\
75.32	0.01\\
75.33	0.01\\
75.34	0.01\\
75.35	0.01\\
75.36	0.01\\
75.37	0.01\\
75.38	0.01\\
75.39	0.01\\
75.4	0.01\\
75.41	0.01\\
75.42	0.01\\
75.43	0.01\\
75.44	0.01\\
75.45	0.01\\
75.46	0.01\\
75.47	0.01\\
75.48	0.01\\
75.49	0.01\\
75.5	0.01\\
75.51	0.01\\
75.52	0.01\\
75.53	0.01\\
75.54	0.01\\
75.55	0.01\\
75.56	0.01\\
75.57	0.01\\
75.58	0.01\\
75.59	0.01\\
75.6	0.01\\
75.61	0.01\\
75.62	0.01\\
75.63	0.01\\
75.64	0.01\\
75.65	0.01\\
75.66	0.01\\
75.67	0.01\\
75.68	0.01\\
75.69	0.01\\
75.7	0.01\\
75.71	0.01\\
75.72	0.01\\
75.73	0.01\\
75.74	0.01\\
75.75	0.01\\
75.76	0.01\\
75.77	0.01\\
75.78	0.01\\
75.79	0.01\\
75.8	0.01\\
75.81	0.01\\
75.82	0.01\\
75.83	0.01\\
75.84	0.01\\
75.85	0.01\\
75.86	0.01\\
75.87	0.01\\
75.88	0.01\\
75.89	0.01\\
75.9	0.01\\
75.91	0.01\\
75.92	0.01\\
75.93	0.01\\
75.94	0.01\\
75.95	0.01\\
75.96	0.01\\
75.97	0.01\\
75.98	0.01\\
75.99	0.01\\
76	0.01\\
76.01	0.01\\
76.02	0.01\\
76.03	0.01\\
76.04	0.01\\
76.05	0.01\\
76.06	0.01\\
76.07	0.01\\
76.08	0.01\\
76.09	0.01\\
76.1	0.01\\
76.11	0.01\\
76.12	0.01\\
76.13	0.01\\
76.14	0.01\\
76.15	0.01\\
76.16	0.01\\
76.17	0.01\\
76.18	0.01\\
76.19	0.01\\
76.2	0.01\\
76.21	0.01\\
76.22	0.01\\
76.23	0.01\\
76.24	0.01\\
76.25	0.01\\
76.26	0.01\\
76.27	0.01\\
76.28	0.01\\
76.29	0.01\\
76.3	0.01\\
76.31	0.01\\
76.32	0.01\\
76.33	0.01\\
76.34	0.01\\
76.35	0.01\\
76.36	0.01\\
76.37	0.01\\
76.38	0.01\\
76.39	0.01\\
76.4	0.01\\
76.41	0.01\\
76.42	0.01\\
76.43	0.01\\
76.44	0.01\\
76.45	0.01\\
76.46	0.01\\
76.47	0.01\\
76.48	0.01\\
76.49	0.01\\
76.5	0.01\\
76.51	0.01\\
76.52	0.01\\
76.53	0.01\\
76.54	0.01\\
76.55	0.01\\
76.56	0.01\\
76.57	0.01\\
76.58	0.01\\
76.59	0.01\\
76.6	0.01\\
76.61	0.01\\
76.62	0.01\\
76.63	0.01\\
76.64	0.01\\
76.65	0.01\\
76.66	0.01\\
76.67	0.01\\
76.68	0.01\\
76.69	0.01\\
76.7	0.01\\
76.71	0.01\\
76.72	0.01\\
76.73	0.01\\
76.74	0.01\\
76.75	0.01\\
76.76	0.01\\
76.77	0.01\\
76.78	0.01\\
76.79	0.01\\
76.8	0.01\\
76.81	0.01\\
76.82	0.01\\
76.83	0.01\\
76.84	0.01\\
76.85	0.01\\
76.86	0.01\\
76.87	0.01\\
76.88	0.01\\
76.89	0.01\\
76.9	0.01\\
76.91	0.01\\
76.92	0.01\\
76.93	0.01\\
76.94	0.01\\
76.95	0.01\\
76.96	0.01\\
76.97	0.01\\
76.98	0.01\\
76.99	0.01\\
77	0.01\\
77.01	0.01\\
77.02	0.01\\
77.03	0.01\\
77.04	0.01\\
77.05	0.01\\
77.06	0.01\\
77.07	0.01\\
77.08	0.01\\
77.09	0.01\\
77.1	0.01\\
77.11	0.01\\
77.12	0.01\\
77.13	0.01\\
77.14	0.01\\
77.15	0.01\\
77.16	0.01\\
77.17	0.01\\
77.18	0.01\\
77.19	0.01\\
77.2	0.01\\
77.21	0.01\\
77.22	0.01\\
77.23	0.01\\
77.24	0.01\\
77.25	0.01\\
77.26	0.01\\
77.27	0.01\\
77.28	0.01\\
77.29	0.01\\
77.3	0.01\\
77.31	0.01\\
77.32	0.01\\
77.33	0.01\\
77.34	0.01\\
77.35	0.01\\
77.36	0.01\\
77.37	0.01\\
77.38	0.01\\
77.39	0.01\\
77.4	0.01\\
77.41	0.01\\
77.42	0.01\\
77.43	0.01\\
77.44	0.01\\
77.45	0.01\\
77.46	0.01\\
77.47	0.01\\
77.48	0.01\\
77.49	0.01\\
77.5	0.01\\
77.51	0.01\\
77.52	0.01\\
77.53	0.01\\
77.54	0.01\\
77.55	0.01\\
77.56	0.01\\
77.57	0.01\\
77.58	0.01\\
77.59	0.01\\
77.6	0.01\\
77.61	0.01\\
77.62	0.01\\
77.63	0.01\\
77.64	0.01\\
77.65	0.01\\
77.66	0.01\\
77.67	0.01\\
77.68	0.01\\
77.69	0.01\\
77.7	0.01\\
77.71	0.01\\
77.72	0.01\\
77.73	0.01\\
77.74	0.01\\
77.75	0.01\\
77.76	0.01\\
77.77	0.01\\
77.78	0.01\\
77.79	0.01\\
77.8	0.01\\
77.81	0.01\\
77.82	0.01\\
77.83	0.01\\
77.84	0.01\\
77.85	0.01\\
77.86	0.01\\
77.87	0.01\\
77.88	0.01\\
77.89	0.01\\
77.9	0.01\\
77.91	0.01\\
77.92	0.01\\
77.93	0.01\\
77.94	0.01\\
77.95	0.01\\
77.96	0.01\\
77.97	0.01\\
77.98	0.01\\
77.99	0.01\\
78	0.01\\
78.01	0.01\\
78.02	0.01\\
78.03	0.01\\
78.04	0.01\\
78.05	0.01\\
78.06	0.01\\
78.07	0.01\\
78.08	0.01\\
78.09	0.01\\
78.1	0.01\\
78.11	0.01\\
78.12	0.01\\
78.13	0.01\\
78.14	0.01\\
78.15	0.01\\
78.16	0.01\\
78.17	0.01\\
78.18	0.01\\
78.19	0.01\\
78.2	0.01\\
78.21	0.01\\
78.22	0.01\\
78.23	0.01\\
78.24	0.01\\
78.25	0.01\\
78.26	0.01\\
78.27	0.01\\
78.28	0.01\\
78.29	0.01\\
78.3	0.01\\
78.31	0.01\\
78.32	0.01\\
78.33	0.01\\
78.34	0.01\\
78.35	0.01\\
78.36	0.01\\
78.37	0.01\\
78.38	0.01\\
78.39	0.01\\
78.4	0.01\\
78.41	0.01\\
78.42	0.01\\
78.43	0.01\\
78.44	0.01\\
78.45	0.01\\
78.46	0.01\\
78.47	0.01\\
78.48	0.01\\
78.49	0.01\\
78.5	0.01\\
78.51	0.01\\
78.52	0.01\\
78.53	0.01\\
78.54	0.01\\
78.55	0.01\\
78.56	0.01\\
78.57	0.01\\
78.58	0.01\\
78.59	0.01\\
78.6	0.01\\
78.61	0.01\\
78.62	0.01\\
78.63	0.01\\
78.64	0.01\\
78.65	0.01\\
78.66	0.01\\
78.67	0.01\\
78.68	0.01\\
78.69	0.01\\
78.7	0.01\\
78.71	0.01\\
78.72	0.01\\
78.73	0.01\\
78.74	0.01\\
78.75	0.01\\
78.76	0.01\\
78.77	0.01\\
78.78	0.01\\
78.79	0.01\\
78.8	0.01\\
78.81	0.01\\
78.82	0.01\\
78.83	0.01\\
78.84	0.01\\
78.85	0.01\\
78.86	0.01\\
78.87	0.01\\
78.88	0.01\\
78.89	0.01\\
78.9	0.01\\
78.91	0.01\\
78.92	0.01\\
78.93	0.01\\
78.94	0.01\\
78.95	0.01\\
78.96	0.01\\
78.97	0.01\\
78.98	0.01\\
78.99	0.01\\
79	0.01\\
79.01	0.01\\
79.02	0.01\\
79.03	0.01\\
79.04	0.01\\
79.05	0.01\\
79.06	0.01\\
79.07	0.01\\
79.08	0.01\\
79.09	0.01\\
79.1	0.01\\
79.11	0.01\\
79.12	0.01\\
79.13	0.01\\
79.14	0.01\\
79.15	0.01\\
79.16	0.01\\
79.17	0.01\\
79.18	0.01\\
79.19	0.01\\
79.2	0.01\\
79.21	0.01\\
79.22	0.01\\
79.23	0.01\\
79.24	0.01\\
79.25	0.01\\
79.26	0.01\\
79.27	0.01\\
79.28	0.01\\
79.29	0.01\\
79.3	0.01\\
79.31	0.01\\
79.32	0.01\\
79.33	0.01\\
79.34	0.01\\
79.35	0.01\\
79.36	0.01\\
79.37	0.01\\
79.38	0.01\\
79.39	0.01\\
79.4	0.01\\
79.41	0.01\\
79.42	0.01\\
79.43	0.01\\
79.44	0.01\\
79.45	0.01\\
79.46	0.01\\
79.47	0.01\\
79.48	0.01\\
79.49	0.01\\
79.5	0.01\\
79.51	0.01\\
79.52	0.01\\
79.53	0.01\\
79.54	0.01\\
79.55	0.01\\
79.56	0.01\\
79.57	0.01\\
79.58	0.01\\
79.59	0.01\\
79.6	0.01\\
79.61	0.01\\
79.62	0.01\\
79.63	0.01\\
79.64	0.01\\
79.65	0.01\\
79.66	0.01\\
79.67	0.01\\
79.68	0.01\\
79.69	0.01\\
79.7	0.01\\
79.71	0.01\\
79.72	0.01\\
79.73	0.01\\
79.74	0.01\\
79.75	0.01\\
79.76	0.01\\
79.77	0.01\\
79.78	0.01\\
79.79	0.01\\
79.8	0.01\\
79.81	0.01\\
79.82	0.01\\
79.83	0.01\\
79.84	0.01\\
79.85	0.01\\
79.86	0.01\\
79.87	0.01\\
79.88	0.01\\
79.89	0.01\\
79.9	0.01\\
79.91	0.01\\
79.92	0.01\\
79.93	0.01\\
79.94	0.01\\
79.95	0.01\\
79.96	0.01\\
79.97	0.01\\
79.98	0.01\\
79.99	0.01\\
80	0.01\\
80.01	0.01\\
};
\addplot [color=mycolor1,solid]
  table[row sep=crcr]{%
80.01	0.01\\
80.02	0.01\\
80.03	0.01\\
80.04	0.01\\
80.05	0.01\\
80.06	0.01\\
80.07	0.01\\
80.08	0.01\\
80.09	0.01\\
80.1	0.01\\
80.11	0.01\\
80.12	0.01\\
80.13	0.01\\
80.14	0.01\\
80.15	0.01\\
80.16	0.01\\
80.17	0.01\\
80.18	0.01\\
80.19	0.01\\
80.2	0.01\\
80.21	0.01\\
80.22	0.01\\
80.23	0.01\\
80.24	0.01\\
80.25	0.01\\
80.26	0.01\\
80.27	0.01\\
80.28	0.01\\
80.29	0.01\\
80.3	0.01\\
80.31	0.01\\
80.32	0.01\\
80.33	0.01\\
80.34	0.01\\
80.35	0.01\\
80.36	0.01\\
80.37	0.01\\
80.38	0.01\\
80.39	0.01\\
80.4	0.01\\
80.41	0.01\\
80.42	0.01\\
80.43	0.01\\
80.44	0.01\\
80.45	0.01\\
80.46	0.01\\
80.47	0.01\\
80.48	0.01\\
80.49	0.01\\
80.5	0.01\\
80.51	0.01\\
80.52	0.01\\
80.53	0.01\\
80.54	0.01\\
80.55	0.01\\
80.56	0.01\\
80.57	0.01\\
80.58	0.01\\
80.59	0.01\\
80.6	0.01\\
80.61	0.01\\
80.62	0.01\\
80.63	0.01\\
80.64	0.01\\
80.65	0.01\\
80.66	0.01\\
80.67	0.01\\
80.68	0.01\\
80.69	0.01\\
80.7	0.01\\
80.71	0.01\\
80.72	0.01\\
80.73	0.01\\
80.74	0.01\\
80.75	0.01\\
80.76	0.01\\
80.77	0.01\\
80.78	0.01\\
80.79	0.01\\
80.8	0.01\\
80.81	0.01\\
80.82	0.01\\
80.83	0.01\\
80.84	0.01\\
80.85	0.01\\
80.86	0.01\\
80.87	0.01\\
80.88	0.01\\
80.89	0.01\\
80.9	0.01\\
80.91	0.01\\
80.92	0.01\\
80.93	0.01\\
80.94	0.01\\
80.95	0.01\\
80.96	0.01\\
80.97	0.01\\
80.98	0.01\\
80.99	0.01\\
81	0.01\\
81.01	0.01\\
81.02	0.01\\
81.03	0.01\\
81.04	0.01\\
81.05	0.01\\
81.06	0.01\\
81.07	0.01\\
81.08	0.01\\
81.09	0.01\\
81.1	0.01\\
81.11	0.01\\
81.12	0.01\\
81.13	0.01\\
81.14	0.01\\
81.15	0.01\\
81.16	0.01\\
81.17	0.01\\
81.18	0.01\\
81.19	0.01\\
81.2	0.01\\
81.21	0.01\\
81.22	0.01\\
81.23	0.01\\
81.24	0.01\\
81.25	0.01\\
81.26	0.01\\
81.27	0.01\\
81.28	0.01\\
81.29	0.01\\
81.3	0.01\\
81.31	0.01\\
81.32	0.01\\
81.33	0.01\\
81.34	0.01\\
81.35	0.01\\
81.36	0.01\\
81.37	0.01\\
81.38	0.01\\
81.39	0.01\\
81.4	0.01\\
81.41	0.01\\
81.42	0.01\\
81.43	0.01\\
81.44	0.01\\
81.45	0.01\\
81.46	0.01\\
81.47	0.01\\
81.48	0.01\\
81.49	0.01\\
81.5	0.01\\
81.51	0.01\\
81.52	0.01\\
81.53	0.01\\
81.54	0.01\\
81.55	0.01\\
81.56	0.01\\
81.57	0.01\\
81.58	0.01\\
81.59	0.01\\
81.6	0.01\\
81.61	0.01\\
81.62	0.01\\
81.63	0.01\\
81.64	0.01\\
81.65	0.01\\
81.66	0.01\\
81.67	0.01\\
81.68	0.01\\
81.69	0.01\\
81.7	0.01\\
81.71	0.01\\
81.72	0.01\\
81.73	0.01\\
81.74	0.01\\
81.75	0.01\\
81.76	0.01\\
81.77	0.01\\
81.78	0.01\\
81.79	0.01\\
81.8	0.01\\
81.81	0.01\\
81.82	0.01\\
81.83	0.01\\
81.84	0.01\\
81.85	0.01\\
81.86	0.01\\
81.87	0.01\\
81.88	0.01\\
81.89	0.01\\
81.9	0.01\\
81.91	0.01\\
81.92	0.01\\
81.93	0.01\\
81.94	0.01\\
81.95	0.01\\
81.96	0.01\\
81.97	0.01\\
81.98	0.01\\
81.99	0.01\\
82	0.01\\
82.01	0.01\\
82.02	0.01\\
82.03	0.01\\
82.04	0.01\\
82.05	0.01\\
82.06	0.01\\
82.07	0.01\\
82.08	0.01\\
82.09	0.01\\
82.1	0.01\\
82.11	0.01\\
82.12	0.01\\
82.13	0.01\\
82.14	0.01\\
82.15	0.01\\
82.16	0.01\\
82.17	0.01\\
82.18	0.01\\
82.19	0.01\\
82.2	0.01\\
82.21	0.01\\
82.22	0.01\\
82.23	0.01\\
82.24	0.01\\
82.25	0.01\\
82.26	0.01\\
82.27	0.01\\
82.28	0.01\\
82.29	0.01\\
82.3	0.01\\
82.31	0.01\\
82.32	0.01\\
82.33	0.01\\
82.34	0.01\\
82.35	0.01\\
82.36	0.01\\
82.37	0.01\\
82.38	0.01\\
82.39	0.01\\
82.4	0.01\\
82.41	0.01\\
82.42	0.01\\
82.43	0.01\\
82.44	0.01\\
82.45	0.01\\
82.46	0.01\\
82.47	0.01\\
82.48	0.01\\
82.49	0.01\\
82.5	0.01\\
82.51	0.01\\
82.52	0.01\\
82.53	0.01\\
82.54	0.01\\
82.55	0.01\\
82.56	0.01\\
82.57	0.01\\
82.58	0.01\\
82.59	0.01\\
82.6	0.01\\
82.61	0.01\\
82.62	0.01\\
82.63	0.01\\
82.64	0.01\\
82.65	0.01\\
82.66	0.01\\
82.67	0.01\\
82.68	0.01\\
82.69	0.01\\
82.7	0.01\\
82.71	0.01\\
82.72	0.01\\
82.73	0.01\\
82.74	0.01\\
82.75	0.01\\
82.76	0.01\\
82.77	0.01\\
82.78	0.01\\
82.79	0.01\\
82.8	0.01\\
82.81	0.01\\
82.82	0.01\\
82.83	0.01\\
82.84	0.01\\
82.85	0.01\\
82.86	0.01\\
82.87	0.01\\
82.88	0.01\\
82.89	0.01\\
82.9	0.01\\
82.91	0.01\\
82.92	0.01\\
82.93	0.01\\
82.94	0.01\\
82.95	0.01\\
82.96	0.01\\
82.97	0.01\\
82.98	0.01\\
82.99	0.01\\
83	0.01\\
83.01	0.01\\
83.02	0.01\\
83.03	0.01\\
83.04	0.01\\
83.05	0.01\\
83.06	0.01\\
83.07	0.01\\
83.08	0.01\\
83.09	0.01\\
83.1	0.01\\
83.11	0.01\\
83.12	0.01\\
83.13	0.01\\
83.14	0.01\\
83.15	0.01\\
83.16	0.01\\
83.17	0.01\\
83.18	0.01\\
83.19	0.01\\
83.2	0.01\\
83.21	0.01\\
83.22	0.01\\
83.23	0.01\\
83.24	0.01\\
83.25	0.01\\
83.26	0.01\\
83.27	0.01\\
83.28	0.01\\
83.29	0.01\\
83.3	0.01\\
83.31	0.01\\
83.32	0.01\\
83.33	0.01\\
83.34	0.01\\
83.35	0.01\\
83.36	0.01\\
83.37	0.01\\
83.38	0.01\\
83.39	0.01\\
83.4	0.01\\
83.41	0.01\\
83.42	0.01\\
83.43	0.01\\
83.44	0.01\\
83.45	0.01\\
83.46	0.01\\
83.47	0.01\\
83.48	0.01\\
83.49	0.01\\
83.5	0.01\\
83.51	0.01\\
83.52	0.01\\
83.53	0.01\\
83.54	0.01\\
83.55	0.01\\
83.56	0.01\\
83.57	0.01\\
83.58	0.01\\
83.59	0.01\\
83.6	0.01\\
83.61	0.01\\
83.62	0.01\\
83.63	0.01\\
83.64	0.01\\
83.65	0.01\\
83.66	0.01\\
83.67	0.01\\
83.68	0.01\\
83.69	0.01\\
83.7	0.01\\
83.71	0.01\\
83.72	0.01\\
83.73	0.01\\
83.74	0.01\\
83.75	0.01\\
83.76	0.01\\
83.77	0.01\\
83.78	0.01\\
83.79	0.01\\
83.8	0.01\\
83.81	0.01\\
83.82	0.01\\
83.83	0.01\\
83.84	0.01\\
83.85	0.01\\
83.86	0.01\\
83.87	0.01\\
83.88	0.01\\
83.89	0.01\\
83.9	0.01\\
83.91	0.01\\
83.92	0.01\\
83.93	0.01\\
83.94	0.01\\
83.95	0.01\\
83.96	0.01\\
83.97	0.01\\
83.98	0.01\\
83.99	0.01\\
84	0.01\\
84.01	0.01\\
84.02	0.01\\
84.03	0.01\\
84.04	0.01\\
84.05	0.01\\
84.06	0.01\\
84.07	0.01\\
84.08	0.01\\
84.09	0.01\\
84.1	0.01\\
84.11	0.01\\
84.12	0.01\\
84.13	0.01\\
84.14	0.01\\
84.15	0.01\\
84.16	0.01\\
84.17	0.01\\
84.18	0.01\\
84.19	0.01\\
84.2	0.01\\
84.21	0.01\\
84.22	0.01\\
84.23	0.01\\
84.24	0.01\\
84.25	0.01\\
84.26	0.01\\
84.27	0.01\\
84.28	0.01\\
84.29	0.01\\
84.3	0.01\\
84.31	0.01\\
84.32	0.01\\
84.33	0.01\\
84.34	0.01\\
84.35	0.01\\
84.36	0.01\\
84.37	0.01\\
84.38	0.01\\
84.39	0.01\\
84.4	0.01\\
84.41	0.01\\
84.42	0.01\\
84.43	0.01\\
84.44	0.01\\
84.45	0.01\\
84.46	0.01\\
84.47	0.01\\
84.48	0.01\\
84.49	0.01\\
84.5	0.01\\
84.51	0.01\\
84.52	0.01\\
84.53	0.01\\
84.54	0.01\\
84.55	0.01\\
84.56	0.01\\
84.57	0.01\\
84.58	0.01\\
84.59	0.01\\
84.6	0.01\\
84.61	0.01\\
84.62	0.01\\
84.63	0.01\\
84.64	0.01\\
84.65	0.01\\
84.66	0.01\\
84.67	0.01\\
84.68	0.01\\
84.69	0.01\\
84.7	0.01\\
84.71	0.01\\
84.72	0.01\\
84.73	0.01\\
84.74	0.01\\
84.75	0.01\\
84.76	0.01\\
84.77	0.01\\
84.78	0.01\\
84.79	0.01\\
84.8	0.01\\
84.81	0.01\\
84.82	0.01\\
84.83	0.01\\
84.84	0.01\\
84.85	0.01\\
84.86	0.01\\
84.87	0.01\\
84.88	0.01\\
84.89	0.01\\
84.9	0.01\\
84.91	0.01\\
84.92	0.01\\
84.93	0.01\\
84.94	0.01\\
84.95	0.01\\
84.96	0.01\\
84.97	0.01\\
84.98	0.01\\
84.99	0.01\\
85	0.01\\
85.01	0.01\\
85.02	0.01\\
85.03	0.01\\
85.04	0.01\\
85.05	0.01\\
85.06	0.01\\
85.07	0.01\\
85.08	0.01\\
85.09	0.01\\
85.1	0.01\\
85.11	0.01\\
85.12	0.01\\
85.13	0.01\\
85.14	0.01\\
85.15	0.01\\
85.16	0.01\\
85.17	0.01\\
85.18	0.01\\
85.19	0.01\\
85.2	0.01\\
85.21	0.01\\
85.22	0.01\\
85.23	0.01\\
85.24	0.01\\
85.25	0.01\\
85.26	0.01\\
85.27	0.01\\
85.28	0.01\\
85.29	0.01\\
85.3	0.01\\
85.31	0.01\\
85.32	0.01\\
85.33	0.01\\
85.34	0.01\\
85.35	0.01\\
85.36	0.01\\
85.37	0.01\\
85.38	0.01\\
85.39	0.01\\
85.4	0.01\\
85.41	0.01\\
85.42	0.01\\
85.43	0.01\\
85.44	0.01\\
85.45	0.01\\
85.46	0.01\\
85.47	0.01\\
85.48	0.01\\
85.49	0.01\\
85.5	0.01\\
85.51	0.01\\
85.52	0.01\\
85.53	0.01\\
85.54	0.01\\
85.55	0.01\\
85.56	0.01\\
85.57	0.01\\
85.58	0.01\\
85.59	0.01\\
85.6	0.01\\
85.61	0.01\\
85.62	0.01\\
85.63	0.01\\
85.64	0.01\\
85.65	0.01\\
85.66	0.01\\
85.67	0.01\\
85.68	0.01\\
85.69	0.01\\
85.7	0.01\\
85.71	0.01\\
85.72	0.01\\
85.73	0.01\\
85.74	0.01\\
85.75	0.01\\
85.76	0.01\\
85.77	0.01\\
85.78	0.01\\
85.79	0.01\\
85.8	0.01\\
85.81	0.01\\
85.82	0.01\\
85.83	0.01\\
85.84	0.01\\
85.85	0.01\\
85.86	0.01\\
85.87	0.01\\
85.88	0.01\\
85.89	0.01\\
85.9	0.01\\
85.91	0.01\\
85.92	0.01\\
85.93	0.01\\
85.94	0.01\\
85.95	0.01\\
85.96	0.01\\
85.97	0.01\\
85.98	0.01\\
85.99	0.01\\
86	0.01\\
86.01	0.01\\
86.02	0.01\\
86.03	0.01\\
86.04	0.01\\
86.05	0.01\\
86.06	0.01\\
86.07	0.01\\
86.08	0.01\\
86.09	0.01\\
86.1	0.01\\
86.11	0.01\\
86.12	0.01\\
86.13	0.01\\
86.14	0.01\\
86.15	0.01\\
86.16	0.01\\
86.17	0.01\\
86.18	0.01\\
86.19	0.01\\
86.2	0.01\\
86.21	0.01\\
86.22	0.01\\
86.23	0.01\\
86.24	0.01\\
86.25	0.01\\
86.26	0.01\\
86.27	0.01\\
86.28	0.01\\
86.29	0.01\\
86.3	0.01\\
86.31	0.01\\
86.32	0.01\\
86.33	0.01\\
86.34	0.01\\
86.35	0.01\\
86.36	0.01\\
86.37	0.01\\
86.38	0.01\\
86.39	0.01\\
86.4	0.01\\
86.41	0.01\\
86.42	0.01\\
86.43	0.01\\
86.44	0.01\\
86.45	0.01\\
86.46	0.01\\
86.47	0.01\\
86.48	0.01\\
86.49	0.01\\
86.5	0.01\\
86.51	0.01\\
86.52	0.01\\
86.53	0.01\\
86.54	0.01\\
86.55	0.01\\
86.56	0.01\\
86.57	0.01\\
86.58	0.01\\
86.59	0.01\\
86.6	0.01\\
86.61	0.01\\
86.62	0.01\\
86.63	0.01\\
86.64	0.01\\
86.65	0.01\\
86.66	0.01\\
86.67	0.01\\
86.68	0.01\\
86.69	0.01\\
86.7	0.01\\
86.71	0.01\\
86.72	0.01\\
86.73	0.01\\
86.74	0.01\\
86.75	0.01\\
86.76	0.01\\
86.77	0.01\\
86.78	0.01\\
86.79	0.01\\
86.8	0.01\\
86.81	0.01\\
86.82	0.01\\
86.83	0.01\\
86.84	0.01\\
86.85	0.01\\
86.86	0.01\\
86.87	0.01\\
86.88	0.01\\
86.89	0.01\\
86.9	0.01\\
86.91	0.01\\
86.92	0.01\\
86.93	0.01\\
86.94	0.01\\
86.95	0.01\\
86.96	0.01\\
86.97	0.01\\
86.98	0.01\\
86.99	0.01\\
87	0.01\\
87.01	0.01\\
87.02	0.01\\
87.03	0.01\\
87.04	0.01\\
87.05	0.01\\
87.06	0.01\\
87.07	0.01\\
87.08	0.01\\
87.09	0.01\\
87.1	0.01\\
87.11	0.01\\
87.12	0.01\\
87.13	0.01\\
87.14	0.01\\
87.15	0.01\\
87.16	0.01\\
87.17	0.01\\
87.18	0.01\\
87.19	0.01\\
87.2	0.01\\
87.21	0.01\\
87.22	0.01\\
87.23	0.01\\
87.24	0.01\\
87.25	0.01\\
87.26	0.01\\
87.27	0.01\\
87.28	0.01\\
87.29	0.01\\
87.3	0.01\\
87.31	0.01\\
87.32	0.01\\
87.33	0.01\\
87.34	0.01\\
87.35	0.01\\
87.36	0.01\\
87.37	0.01\\
87.38	0.01\\
87.39	0.01\\
87.4	0.01\\
87.41	0.01\\
87.42	0.01\\
87.43	0.01\\
87.44	0.01\\
87.45	0.01\\
87.46	0.01\\
87.47	0.01\\
87.48	0.01\\
87.49	0.01\\
87.5	0.01\\
87.51	0.01\\
87.52	0.01\\
87.53	0.01\\
87.54	0.01\\
87.55	0.01\\
87.56	0.01\\
87.57	0.01\\
87.58	0.01\\
87.59	0.01\\
87.6	0.01\\
87.61	0.01\\
87.62	0.01\\
87.63	0.01\\
87.64	0.01\\
87.65	0.01\\
87.66	0.01\\
87.67	0.01\\
87.68	0.01\\
87.69	0.01\\
87.7	0.01\\
87.71	0.01\\
87.72	0.01\\
87.73	0.01\\
87.74	0.01\\
87.75	0.01\\
87.76	0.01\\
87.77	0.01\\
87.78	0.01\\
87.79	0.01\\
87.8	0.01\\
87.81	0.01\\
87.82	0.01\\
87.83	0.01\\
87.84	0.01\\
87.85	0.01\\
87.86	0.01\\
87.87	0.01\\
87.88	0.01\\
87.89	0.01\\
87.9	0.01\\
87.91	0.01\\
87.92	0.01\\
87.93	0.01\\
87.94	0.01\\
87.95	0.01\\
87.96	0.01\\
87.97	0.01\\
87.98	0.01\\
87.99	0.01\\
88	0.01\\
88.01	0.01\\
88.02	0.01\\
88.03	0.01\\
88.04	0.01\\
88.05	0.01\\
88.06	0.01\\
88.07	0.01\\
88.08	0.01\\
88.09	0.01\\
88.1	0.01\\
88.11	0.01\\
88.12	0.01\\
88.13	0.01\\
88.14	0.01\\
88.15	0.01\\
88.16	0.01\\
88.17	0.01\\
88.18	0.01\\
88.19	0.01\\
88.2	0.01\\
88.21	0.01\\
88.22	0.01\\
88.23	0.01\\
88.24	0.01\\
88.25	0.01\\
88.26	0.01\\
88.27	0.01\\
88.28	0.01\\
88.29	0.01\\
88.3	0.01\\
88.31	0.01\\
88.32	0.01\\
88.33	0.01\\
88.34	0.01\\
88.35	0.01\\
88.36	0.01\\
88.37	0.01\\
88.38	0.01\\
88.39	0.01\\
88.4	0.01\\
88.41	0.01\\
88.42	0.01\\
88.43	0.01\\
88.44	0.01\\
88.45	0.01\\
88.46	0.01\\
88.47	0.01\\
88.48	0.01\\
88.49	0.01\\
88.5	0.01\\
88.51	0.01\\
88.52	0.01\\
88.53	0.01\\
88.54	0.01\\
88.55	0.01\\
88.56	0.01\\
88.57	0.01\\
88.58	0.01\\
88.59	0.01\\
88.6	0.01\\
88.61	0.01\\
88.62	0.01\\
88.63	0.01\\
88.64	0.01\\
88.65	0.01\\
88.66	0.01\\
88.67	0.01\\
88.68	0.01\\
88.69	0.01\\
88.7	0.01\\
88.71	0.01\\
88.72	0.01\\
88.73	0.01\\
88.74	0.01\\
88.75	0.01\\
88.76	0.01\\
88.77	0.01\\
88.78	0.01\\
88.79	0.01\\
88.8	0.01\\
88.81	0.01\\
88.82	0.01\\
88.83	0.01\\
88.84	0.01\\
88.85	0.01\\
88.86	0.01\\
88.87	0.01\\
88.88	0.01\\
88.89	0.01\\
88.9	0.01\\
88.91	0.01\\
88.92	0.01\\
88.93	0.01\\
88.94	0.01\\
88.95	0.01\\
88.96	0.01\\
88.97	0.01\\
88.98	0.01\\
88.99	0.01\\
89	0.01\\
89.01	0.01\\
89.02	0.01\\
89.03	0.01\\
89.04	0.01\\
89.05	0.01\\
89.06	0.01\\
89.07	0.01\\
89.08	0.01\\
89.09	0.01\\
89.1	0.01\\
89.11	0.01\\
89.12	0.01\\
89.13	0.01\\
89.14	0.01\\
89.15	0.01\\
89.16	0.01\\
89.17	0.01\\
89.18	0.01\\
89.19	0.01\\
89.2	0.01\\
89.21	0.01\\
89.22	0.01\\
89.23	0.01\\
89.24	0.01\\
89.25	0.01\\
89.26	0.01\\
89.27	0.01\\
89.28	0.01\\
89.29	0.01\\
89.3	0.01\\
89.31	0.01\\
89.32	0.01\\
89.33	0.01\\
89.34	0.01\\
89.35	0.01\\
89.36	0.01\\
89.37	0.01\\
89.38	0.01\\
89.39	0.01\\
89.4	0.01\\
89.41	0.01\\
89.42	0.01\\
89.43	0.01\\
89.44	0.01\\
89.45	0.01\\
89.46	0.01\\
89.47	0.01\\
89.48	0.01\\
89.49	0.01\\
89.5	0.01\\
89.51	0.01\\
89.52	0.01\\
89.53	0.01\\
89.54	0.01\\
89.55	0.01\\
89.56	0.01\\
89.57	0.01\\
89.58	0.01\\
89.59	0.01\\
89.6	0.01\\
89.61	0.01\\
89.62	0.01\\
89.63	0.01\\
89.64	0.01\\
89.65	0.01\\
89.66	0.01\\
89.67	0.01\\
89.68	0.01\\
89.69	0.01\\
89.7	0.01\\
89.71	0.01\\
89.72	0.01\\
89.73	0.01\\
89.74	0.01\\
89.75	0.01\\
89.76	0.01\\
89.77	0.01\\
89.78	0.01\\
89.79	0.01\\
89.8	0.01\\
89.81	0.01\\
89.82	0.01\\
89.83	0.01\\
89.84	0.01\\
89.85	0.01\\
89.86	0.01\\
89.87	0.01\\
89.88	0.01\\
89.89	0.01\\
89.9	0.01\\
89.91	0.01\\
89.92	0.01\\
89.93	0.01\\
89.94	0.01\\
89.95	0.01\\
89.96	0.01\\
89.97	0.01\\
89.98	0.01\\
89.99	0.01\\
90	0.01\\
90.01	0.01\\
90.02	0.01\\
90.03	0.01\\
90.04	0.01\\
90.05	0.01\\
90.06	0.01\\
90.07	0.01\\
90.08	0.01\\
90.09	0.01\\
90.1	0.01\\
90.11	0.01\\
90.12	0.01\\
90.13	0.01\\
90.14	0.01\\
90.15	0.01\\
90.16	0.01\\
90.17	0.01\\
90.18	0.01\\
90.19	0.01\\
90.2	0.01\\
90.21	0.01\\
90.22	0.01\\
90.23	0.01\\
90.24	0.01\\
90.25	0.01\\
90.26	0.01\\
90.27	0.01\\
90.28	0.01\\
90.29	0.01\\
90.3	0.01\\
90.31	0.01\\
90.32	0.01\\
90.33	0.01\\
90.34	0.01\\
90.35	0.01\\
90.36	0.01\\
90.37	0.01\\
90.38	0.01\\
90.39	0.01\\
90.4	0.01\\
90.41	0.01\\
90.42	0.01\\
90.43	0.01\\
90.44	0.01\\
90.45	0.01\\
90.46	0.01\\
90.47	0.01\\
90.48	0.01\\
90.49	0.01\\
90.5	0.01\\
90.51	0.01\\
90.52	0.01\\
90.53	0.01\\
90.54	0.01\\
90.55	0.01\\
90.56	0.01\\
90.57	0.01\\
90.58	0.01\\
90.59	0.01\\
90.6	0.01\\
90.61	0.01\\
90.62	0.01\\
90.63	0.01\\
90.64	0.01\\
90.65	0.01\\
90.66	0.01\\
90.67	0.01\\
90.68	0.01\\
90.69	0.01\\
90.7	0.01\\
90.71	0.01\\
90.72	0.01\\
90.73	0.01\\
90.74	0.01\\
90.75	0.01\\
90.76	0.01\\
90.77	0.01\\
90.78	0.01\\
90.79	0.01\\
90.8	0.01\\
90.81	0.01\\
90.82	0.01\\
90.83	0.01\\
90.84	0.01\\
90.85	0.01\\
90.86	0.01\\
90.87	0.01\\
90.88	0.01\\
90.89	0.01\\
90.9	0.01\\
90.91	0.01\\
90.92	0.01\\
90.93	0.01\\
90.94	0.01\\
90.95	0.01\\
90.96	0.01\\
90.97	0.01\\
90.98	0.01\\
90.99	0.01\\
91	0.01\\
91.01	0.01\\
91.02	0.01\\
91.03	0.01\\
91.04	0.01\\
91.05	0.01\\
91.06	0.01\\
91.07	0.01\\
91.08	0.01\\
91.09	0.01\\
91.1	0.01\\
91.11	0.01\\
91.12	0.01\\
91.13	0.01\\
91.14	0.01\\
91.15	0.01\\
91.16	0.01\\
91.17	0.01\\
91.18	0.01\\
91.19	0.01\\
91.2	0.01\\
91.21	0.01\\
91.22	0.01\\
91.23	0.01\\
91.24	0.01\\
91.25	0.01\\
91.26	0.01\\
91.27	0.01\\
91.28	0.01\\
91.29	0.01\\
91.3	0.01\\
91.31	0.01\\
91.32	0.01\\
91.33	0.01\\
91.34	0.01\\
91.35	0.01\\
91.36	0.01\\
91.37	0.01\\
91.38	0.01\\
91.39	0.01\\
91.4	0.01\\
91.41	0.01\\
91.42	0.01\\
91.43	0.01\\
91.44	0.01\\
91.45	0.01\\
91.46	0.01\\
91.47	0.01\\
91.48	0.01\\
91.49	0.01\\
91.5	0.01\\
91.51	0.01\\
91.52	0.01\\
91.53	0.01\\
91.54	0.01\\
91.55	0.01\\
91.56	0.01\\
91.57	0.01\\
91.58	0.01\\
91.59	0.01\\
91.6	0.01\\
91.61	0.01\\
91.62	0.01\\
91.63	0.01\\
91.64	0.01\\
91.65	0.01\\
91.66	0.01\\
91.67	0.01\\
91.68	0.01\\
91.69	0.01\\
91.7	0.01\\
91.71	0.01\\
91.72	0.01\\
91.73	0.01\\
91.74	0.01\\
91.75	0.01\\
91.76	0.01\\
91.77	0.01\\
91.78	0.01\\
91.79	0.01\\
91.8	0.01\\
91.81	0.01\\
91.82	0.01\\
91.83	0.01\\
91.84	0.01\\
91.85	0.01\\
91.86	0.01\\
91.87	0.01\\
91.88	0.01\\
91.89	0.01\\
91.9	0.01\\
91.91	0.01\\
91.92	0.01\\
91.93	0.01\\
91.94	0.01\\
91.95	0.01\\
91.96	0.01\\
91.97	0.01\\
91.98	0.01\\
91.99	0.01\\
92	0.01\\
92.01	0.01\\
92.02	0.01\\
92.03	0.01\\
92.04	0.01\\
92.05	0.01\\
92.06	0.01\\
92.07	0.01\\
92.08	0.01\\
92.09	0.01\\
92.1	0.01\\
92.11	0.01\\
92.12	0.01\\
92.13	0.01\\
92.14	0.01\\
92.15	0.01\\
92.16	0.01\\
92.17	0.01\\
92.18	0.01\\
92.19	0.01\\
92.2	0.01\\
92.21	0.01\\
92.22	0.01\\
92.23	0.01\\
92.24	0.01\\
92.25	0.01\\
92.26	0.01\\
92.27	0.01\\
92.28	0.01\\
92.29	0.01\\
92.3	0.01\\
92.31	0.01\\
92.32	0.01\\
92.33	0.01\\
92.34	0.01\\
92.35	0.01\\
92.36	0.01\\
92.37	0.01\\
92.38	0.01\\
92.39	0.01\\
92.4	0.01\\
92.41	0.01\\
92.42	0.01\\
92.43	0.01\\
92.44	0.01\\
92.45	0.01\\
92.46	0.01\\
92.47	0.01\\
92.48	0.01\\
92.49	0.01\\
92.5	0.01\\
92.51	0.01\\
92.52	0.01\\
92.53	0.01\\
92.54	0.01\\
92.55	0.01\\
92.56	0.01\\
92.57	0.01\\
92.58	0.01\\
92.59	0.01\\
92.6	0.01\\
92.61	0.01\\
92.62	0.01\\
92.63	0.01\\
92.64	0.01\\
92.65	0.01\\
92.66	0.01\\
92.67	0.01\\
92.68	0.01\\
92.69	0.01\\
92.7	0.01\\
92.71	0.01\\
92.72	0.01\\
92.73	0.01\\
92.74	0.01\\
92.75	0.01\\
92.76	0.01\\
92.77	0.01\\
92.78	0.01\\
92.79	0.01\\
92.8	0.01\\
92.81	0.01\\
92.82	0.01\\
92.83	0.01\\
92.84	0.01\\
92.85	0.01\\
92.86	0.01\\
92.87	0.01\\
92.88	0.01\\
92.89	0.01\\
92.9	0.01\\
92.91	0.01\\
92.92	0.01\\
92.93	0.01\\
92.94	0.01\\
92.95	0.01\\
92.96	0.01\\
92.97	0.01\\
92.98	0.01\\
92.99	0.01\\
93	0.01\\
93.01	0.01\\
93.02	0.01\\
93.03	0.01\\
93.04	0.01\\
93.05	0.01\\
93.06	0.01\\
93.07	0.01\\
93.08	0.01\\
93.09	0.01\\
93.1	0.01\\
93.11	0.01\\
93.12	0.01\\
93.13	0.01\\
93.14	0.01\\
93.15	0.01\\
93.16	0.01\\
93.17	0.01\\
93.18	0.01\\
93.19	0.01\\
93.2	0.01\\
93.21	0.01\\
93.22	0.01\\
93.23	0.01\\
93.24	0.01\\
93.25	0.01\\
93.26	0.01\\
93.27	0.01\\
93.28	0.01\\
93.29	0.01\\
93.3	0.01\\
93.31	0.01\\
93.32	0.01\\
93.33	0.01\\
93.34	0.01\\
93.35	0.01\\
93.36	0.01\\
93.37	0.01\\
93.38	0.01\\
93.39	0.01\\
93.4	0.01\\
93.41	0.01\\
93.42	0.01\\
93.43	0.01\\
93.44	0.01\\
93.45	0.01\\
93.46	0.01\\
93.47	0.01\\
93.48	0.01\\
93.49	0.01\\
93.5	0.01\\
93.51	0.01\\
93.52	0.01\\
93.53	0.01\\
93.54	0.01\\
93.55	0.01\\
93.56	0.01\\
93.57	0.01\\
93.58	0.01\\
93.59	0.01\\
93.6	0.01\\
93.61	0.01\\
93.62	0.01\\
93.63	0.01\\
93.64	0.01\\
93.65	0.01\\
93.66	0.01\\
93.67	0.01\\
93.68	0.01\\
93.69	0.01\\
93.7	0.01\\
93.71	0.01\\
93.72	0.01\\
93.73	0.01\\
93.74	0.01\\
93.75	0.01\\
93.76	0.01\\
93.77	0.01\\
93.78	0.01\\
93.79	0.01\\
93.8	0.01\\
93.81	0.01\\
93.82	0.01\\
93.83	0.01\\
93.84	0.01\\
93.85	0.01\\
93.86	0.01\\
93.87	0.01\\
93.88	0.01\\
93.89	0.01\\
93.9	0.01\\
93.91	0.01\\
93.92	0.01\\
93.93	0.01\\
93.94	0.01\\
93.95	0.01\\
93.96	0.01\\
93.97	0.01\\
93.98	0.01\\
93.99	0.01\\
94	0.01\\
94.01	0.01\\
94.02	0.01\\
94.03	0.01\\
94.04	0.01\\
94.05	0.01\\
94.06	0.01\\
94.07	0.01\\
94.08	0.01\\
94.09	0.01\\
94.1	0.01\\
94.11	0.01\\
94.12	0.01\\
94.13	0.01\\
94.14	0.01\\
94.15	0.01\\
94.16	0.01\\
94.17	0.01\\
94.18	0.01\\
94.19	0.01\\
94.2	0.01\\
94.21	0.01\\
94.22	0.01\\
94.23	0.01\\
94.24	0.01\\
94.25	0.01\\
94.26	0.01\\
94.27	0.01\\
94.28	0.01\\
94.29	0.01\\
94.3	0.01\\
94.31	0.01\\
94.32	0.01\\
94.33	0.01\\
94.34	0.01\\
94.35	0.01\\
94.36	0.01\\
94.37	0.01\\
94.38	0.01\\
94.39	0.01\\
94.4	0.01\\
94.41	0.01\\
94.42	0.01\\
94.43	0.01\\
94.44	0.01\\
94.45	0.01\\
94.46	0.01\\
94.47	0.01\\
94.48	0.01\\
94.49	0.01\\
94.5	0.01\\
94.51	0.01\\
94.52	0.01\\
94.53	0.01\\
94.54	0.01\\
94.55	0.01\\
94.56	0.01\\
94.57	0.01\\
94.58	0.01\\
94.59	0.01\\
94.6	0.01\\
94.61	0.01\\
94.62	0.01\\
94.63	0.01\\
94.64	0.01\\
94.65	0.01\\
94.66	0.01\\
94.67	0.01\\
94.68	0.01\\
94.69	0.01\\
94.7	0.01\\
94.71	0.01\\
94.72	0.01\\
94.73	0.01\\
94.74	0.01\\
94.75	0.01\\
94.76	0.01\\
94.77	0.01\\
94.78	0.01\\
94.79	0.01\\
94.8	0.01\\
94.81	0.01\\
94.82	0.01\\
94.83	0.01\\
94.84	0.01\\
94.85	0.01\\
94.86	0.01\\
94.87	0.01\\
94.88	0.01\\
94.89	0.01\\
94.9	0.01\\
94.91	0.01\\
94.92	0.01\\
94.93	0.01\\
94.94	0.01\\
94.95	0.01\\
94.96	0.01\\
94.97	0.01\\
94.98	0.01\\
94.99	0.01\\
95	0.01\\
95.01	0.01\\
95.02	0.01\\
95.03	0.01\\
95.04	0.01\\
95.05	0.01\\
95.06	0.01\\
95.07	0.01\\
95.08	0.01\\
95.09	0.01\\
95.1	0.01\\
95.11	0.01\\
95.12	0.01\\
95.13	0.01\\
95.14	0.01\\
95.15	0.01\\
95.16	0.01\\
95.17	0.01\\
95.18	0.01\\
95.19	0.01\\
95.2	0.01\\
95.21	0.01\\
95.22	0.01\\
95.23	0.01\\
95.24	0.01\\
95.25	0.01\\
95.26	0.01\\
95.27	0.01\\
95.28	0.01\\
95.29	0.01\\
95.3	0.01\\
95.31	0.01\\
95.32	0.01\\
95.33	0.01\\
95.34	0.01\\
95.35	0.01\\
95.36	0.01\\
95.37	0.01\\
95.38	0.01\\
95.39	0.01\\
95.4	0.01\\
95.41	0.01\\
95.42	0.01\\
95.43	0.01\\
95.44	0.01\\
95.45	0.01\\
95.46	0.01\\
95.47	0.01\\
95.48	0.01\\
95.49	0.01\\
95.5	0.01\\
95.51	0.01\\
95.52	0.01\\
95.53	0.01\\
95.54	0.01\\
95.55	0.01\\
95.56	0.01\\
95.57	0.01\\
95.58	0.01\\
95.59	0.01\\
95.6	0.01\\
95.61	0.01\\
95.62	0.01\\
95.63	0.01\\
95.64	0.01\\
95.65	0.01\\
95.66	0.01\\
95.67	0.01\\
95.68	0.01\\
95.69	0.01\\
95.7	0.01\\
95.71	0.01\\
95.72	0.01\\
95.73	0.01\\
95.74	0.01\\
95.75	0.01\\
95.76	0.01\\
95.77	0.01\\
95.78	0.01\\
95.79	0.01\\
95.8	0.01\\
95.81	0.01\\
95.82	0.01\\
95.83	0.01\\
95.84	0.01\\
95.85	0.01\\
95.86	0.01\\
95.87	0.01\\
95.88	0.01\\
95.89	0.01\\
95.9	0.01\\
95.91	0.01\\
95.92	0.01\\
95.93	0.01\\
95.94	0.01\\
95.95	0.01\\
95.96	0.01\\
95.97	0.01\\
95.98	0.01\\
95.99	0.01\\
96	0.01\\
96.01	0.01\\
96.02	0.01\\
96.03	0.01\\
96.04	0.01\\
96.05	0.01\\
96.06	0.01\\
96.07	0.01\\
96.08	0.01\\
96.09	0.01\\
96.1	0.01\\
96.11	0.01\\
96.12	0.01\\
96.13	0.01\\
96.14	0.01\\
96.15	0.01\\
96.16	0.01\\
96.17	0.01\\
96.18	0.01\\
96.19	0.01\\
96.2	0.01\\
96.21	0.01\\
96.22	0.01\\
96.23	0.01\\
96.24	0.01\\
96.25	0.01\\
96.26	0.01\\
96.27	0.01\\
96.28	0.01\\
96.29	0.01\\
96.3	0.01\\
96.31	0.01\\
96.32	0.01\\
96.33	0.01\\
96.34	0.01\\
96.35	0.01\\
96.36	0.01\\
96.37	0.01\\
96.38	0.01\\
96.39	0.01\\
96.4	0.01\\
96.41	0.01\\
96.42	0.01\\
96.43	0.01\\
96.44	0.01\\
96.45	0.01\\
96.46	0.01\\
96.47	0.01\\
96.48	0.01\\
96.49	0.01\\
96.5	0.01\\
96.51	0.01\\
96.52	0.01\\
96.53	0.01\\
96.54	0.01\\
96.55	0.01\\
96.56	0.01\\
96.57	0.01\\
96.58	0.01\\
96.59	0.01\\
96.6	0.01\\
96.61	0.01\\
96.62	0.01\\
96.63	0.01\\
96.64	0.01\\
96.65	0.01\\
96.66	0.01\\
96.67	0.01\\
96.68	0.01\\
96.69	0.01\\
96.7	0.01\\
96.71	0.01\\
96.72	0.01\\
96.73	0.01\\
96.74	0.01\\
96.75	0.01\\
96.76	0.01\\
96.77	0.01\\
96.78	0.01\\
96.79	0.01\\
96.8	0.01\\
96.81	0.01\\
96.82	0.01\\
96.83	0.01\\
96.84	0.01\\
96.85	0.01\\
96.86	0.01\\
96.87	0.01\\
96.88	0.01\\
96.89	0.01\\
96.9	0.01\\
96.91	0.01\\
96.92	0.01\\
96.93	0.01\\
96.94	0.01\\
96.95	0.01\\
96.96	0.01\\
96.97	0.01\\
96.98	0.01\\
96.99	0.01\\
97	0.01\\
97.01	0.01\\
97.02	0.01\\
97.03	0.01\\
97.04	0.01\\
97.05	0.01\\
97.06	0.01\\
97.07	0.01\\
97.08	0.01\\
97.09	0.01\\
97.1	0.01\\
97.11	0.01\\
97.12	0.01\\
97.13	0.01\\
97.14	0.01\\
97.15	0.01\\
97.16	0.01\\
97.17	0.01\\
97.18	0.01\\
97.19	0.01\\
97.2	0.01\\
97.21	0.01\\
97.22	0.01\\
97.23	0.01\\
97.24	0.01\\
97.25	0.01\\
97.26	0.01\\
97.27	0.01\\
97.28	0.01\\
97.29	0.01\\
97.3	0.01\\
97.31	0.01\\
97.32	0.01\\
97.33	0.01\\
97.34	0.01\\
97.35	0.01\\
97.36	0.01\\
97.37	0.01\\
97.38	0.01\\
97.39	0.01\\
97.4	0.01\\
97.41	0.01\\
97.42	0.01\\
97.43	0.01\\
97.44	0.01\\
97.45	0.01\\
97.46	0.01\\
97.47	0.01\\
97.48	0.01\\
97.49	0.01\\
97.5	0.01\\
97.51	0.01\\
97.52	0.01\\
97.53	0.01\\
97.54	0.01\\
97.55	0.01\\
97.56	0.01\\
97.57	0.01\\
97.58	0.01\\
97.59	0.01\\
97.6	0.01\\
97.61	0.01\\
97.62	0.01\\
97.63	0.01\\
97.64	0.01\\
97.65	0.01\\
97.66	0.01\\
97.67	0.01\\
97.68	0.01\\
97.69	0.01\\
97.7	0.01\\
97.71	0.01\\
97.72	0.01\\
97.73	0.01\\
97.74	0.01\\
97.75	0.01\\
97.76	0.01\\
97.77	0.01\\
97.78	0.01\\
97.79	0.01\\
97.8	0.01\\
97.81	0.01\\
97.82	0.01\\
97.83	0.01\\
97.84	0.01\\
97.85	0.01\\
97.86	0.01\\
97.87	0.01\\
97.88	0.01\\
97.89	0.01\\
97.9	0.01\\
97.91	0.01\\
97.92	0.01\\
97.93	0.01\\
97.94	0.01\\
97.95	0.01\\
97.96	0.01\\
97.97	0.01\\
97.98	0.01\\
97.99	0.01\\
98	0.01\\
98.01	0.01\\
98.02	0.01\\
98.03	0.01\\
98.04	0.01\\
98.05	0.01\\
98.06	0.01\\
98.07	0.01\\
98.08	0.01\\
98.09	0.01\\
98.1	0.01\\
98.11	0.01\\
98.12	0.01\\
98.13	0.01\\
98.14	0.01\\
98.15	0.01\\
98.16	0.01\\
98.17	0.01\\
98.18	0.01\\
98.19	0.01\\
98.2	0.01\\
98.21	0.01\\
98.22	0.01\\
98.23	0.01\\
98.24	0.01\\
98.25	0.01\\
98.26	0.01\\
98.27	0.01\\
98.28	0.01\\
98.29	0.01\\
98.3	0.01\\
98.31	0.01\\
98.32	0.01\\
98.33	0.01\\
98.34	0.01\\
98.35	0.01\\
98.36	0.01\\
98.37	0.01\\
98.38	0.01\\
98.39	0.01\\
98.4	0.01\\
98.41	0.01\\
98.42	0.01\\
98.43	0.01\\
98.44	0.01\\
98.45	0.01\\
98.46	0.01\\
98.47	0.01\\
98.48	0.01\\
98.49	0.01\\
98.5	0.01\\
98.51	0.01\\
98.52	0.01\\
98.53	0.01\\
98.54	0.01\\
98.55	0.01\\
98.56	0.01\\
98.57	0.01\\
98.58	0.01\\
98.59	0.01\\
98.6	0.01\\
98.61	0.01\\
98.62	0.01\\
98.63	0.01\\
98.64	0.01\\
98.65	0.01\\
98.66	0.01\\
98.67	0.01\\
98.68	0.01\\
98.69	0.01\\
98.7	0.01\\
98.71	0.01\\
98.72	0.01\\
98.73	0.01\\
98.74	0.01\\
98.75	0.01\\
98.76	0.01\\
98.77	0.01\\
98.78	0.01\\
98.79	0.01\\
98.8	0.01\\
98.81	0.01\\
98.82	0.01\\
98.83	0.01\\
98.84	0.01\\
98.85	0.01\\
98.86	0.01\\
98.87	0.01\\
98.88	0.01\\
98.89	0.01\\
98.9	0.01\\
98.91	0.01\\
98.92	0.01\\
98.93	0.01\\
98.94	0.01\\
98.95	0.01\\
98.96	0.01\\
98.97	0.01\\
98.98	0.01\\
98.99	0.01\\
99	0.01\\
99.01	0.01\\
99.02	0.01\\
99.03	0.01\\
99.04	0.01\\
99.05	0.01\\
99.06	0.01\\
99.07	0.01\\
99.08	0.01\\
99.09	0.01\\
99.1	0.01\\
99.11	0.01\\
99.12	0.01\\
99.13	0.01\\
99.14	0.01\\
99.15	0.01\\
99.16	0.01\\
99.17	0.01\\
99.18	0.01\\
99.19	0.01\\
99.2	0.01\\
99.21	0.01\\
99.22	0.01\\
99.23	0.01\\
99.24	0.01\\
99.25	0.01\\
99.26	0.01\\
99.27	0.01\\
99.28	0.01\\
99.29	0.01\\
99.3	0.01\\
99.31	0.01\\
99.32	0.01\\
99.33	0.01\\
99.34	0.01\\
99.35	0.01\\
99.36	0.01\\
99.37	0.01\\
99.38	0.01\\
99.39	0.01\\
99.4	0.01\\
99.41	0.01\\
99.42	0.01\\
99.43	0.01\\
99.44	0.01\\
99.45	0.01\\
99.46	0.01\\
99.47	0.01\\
99.48	0.01\\
99.49	0.01\\
99.5	0.01\\
99.51	0.01\\
99.52	0.01\\
99.53	0.01\\
99.54	0.01\\
99.55	0.01\\
99.56	0.01\\
99.57	0.01\\
99.58	0.01\\
99.59	0.01\\
99.6	0.01\\
99.61	0.01\\
99.62	0.01\\
99.63	0.01\\
99.64	0.01\\
99.65	0.01\\
99.66	0.01\\
99.67	0.01\\
99.68	0.01\\
99.69	0.01\\
99.7	0.01\\
99.71	0.01\\
99.72	0.01\\
99.73	0.01\\
99.74	0.01\\
99.75	0.01\\
99.76	0.01\\
99.77	0.01\\
99.78	0.01\\
99.79	0.01\\
99.8	0.01\\
99.81	0.01\\
99.82	0.01\\
99.83	0.01\\
99.84	0.01\\
99.85	0.01\\
99.86	0.01\\
99.87	0.01\\
99.88	0.01\\
99.89	0.01\\
99.9	0.01\\
99.91	0.01\\
99.92	0.01\\
99.93	0.01\\
99.94	0.01\\
99.95	0.01\\
99.96	0.01\\
99.97	0.01\\
99.98	0.01\\
99.99	0.01\\
100	0.01\\
};
\addlegendentry{$q=3$};

\addplot [color=green,solid,forget plot]
  table[row sep=crcr]{%
0.01	0.01\\
0.02	0.01\\
0.03	0.01\\
0.04	0.01\\
0.05	0.01\\
0.06	0.01\\
0.07	0.01\\
0.08	0.01\\
0.09	0.01\\
0.1	0.01\\
0.11	0.01\\
0.12	0.01\\
0.13	0.01\\
0.14	0.01\\
0.15	0.01\\
0.16	0.01\\
0.17	0.01\\
0.18	0.01\\
0.19	0.01\\
0.2	0.01\\
0.21	0.01\\
0.22	0.01\\
0.23	0.01\\
0.24	0.01\\
0.25	0.01\\
0.26	0.01\\
0.27	0.01\\
0.28	0.01\\
0.29	0.01\\
0.3	0.01\\
0.31	0.01\\
0.32	0.01\\
0.33	0.01\\
0.34	0.01\\
0.35	0.01\\
0.36	0.01\\
0.37	0.01\\
0.38	0.01\\
0.39	0.01\\
0.4	0.01\\
0.41	0.01\\
0.42	0.01\\
0.43	0.01\\
0.44	0.01\\
0.45	0.01\\
0.46	0.01\\
0.47	0.01\\
0.48	0.01\\
0.49	0.01\\
0.5	0.01\\
0.51	0.01\\
0.52	0.01\\
0.53	0.01\\
0.54	0.01\\
0.55	0.01\\
0.56	0.01\\
0.57	0.01\\
0.58	0.01\\
0.59	0.01\\
0.6	0.01\\
0.61	0.01\\
0.62	0.01\\
0.63	0.01\\
0.64	0.01\\
0.65	0.01\\
0.66	0.01\\
0.67	0.01\\
0.68	0.01\\
0.69	0.01\\
0.7	0.01\\
0.71	0.01\\
0.72	0.01\\
0.73	0.01\\
0.74	0.01\\
0.75	0.01\\
0.76	0.01\\
0.77	0.01\\
0.78	0.01\\
0.79	0.01\\
0.8	0.01\\
0.81	0.01\\
0.82	0.01\\
0.83	0.01\\
0.84	0.01\\
0.85	0.01\\
0.86	0.01\\
0.87	0.01\\
0.88	0.01\\
0.89	0.01\\
0.9	0.01\\
0.91	0.01\\
0.92	0.01\\
0.93	0.01\\
0.94	0.01\\
0.95	0.01\\
0.96	0.01\\
0.97	0.01\\
0.98	0.01\\
0.99	0.01\\
1	0.01\\
1.01	0.01\\
1.02	0.01\\
1.03	0.01\\
1.04	0.01\\
1.05	0.01\\
1.06	0.01\\
1.07	0.01\\
1.08	0.01\\
1.09	0.01\\
1.1	0.01\\
1.11	0.01\\
1.12	0.01\\
1.13	0.01\\
1.14	0.01\\
1.15	0.01\\
1.16	0.01\\
1.17	0.01\\
1.18	0.01\\
1.19	0.01\\
1.2	0.01\\
1.21	0.01\\
1.22	0.01\\
1.23	0.01\\
1.24	0.01\\
1.25	0.01\\
1.26	0.01\\
1.27	0.01\\
1.28	0.01\\
1.29	0.01\\
1.3	0.01\\
1.31	0.01\\
1.32	0.01\\
1.33	0.01\\
1.34	0.01\\
1.35	0.01\\
1.36	0.01\\
1.37	0.01\\
1.38	0.01\\
1.39	0.01\\
1.4	0.01\\
1.41	0.01\\
1.42	0.01\\
1.43	0.01\\
1.44	0.01\\
1.45	0.01\\
1.46	0.01\\
1.47	0.01\\
1.48	0.01\\
1.49	0.01\\
1.5	0.01\\
1.51	0.01\\
1.52	0.01\\
1.53	0.01\\
1.54	0.01\\
1.55	0.01\\
1.56	0.01\\
1.57	0.01\\
1.58	0.01\\
1.59	0.01\\
1.6	0.01\\
1.61	0.01\\
1.62	0.01\\
1.63	0.01\\
1.64	0.01\\
1.65	0.01\\
1.66	0.01\\
1.67	0.01\\
1.68	0.01\\
1.69	0.01\\
1.7	0.01\\
1.71	0.01\\
1.72	0.01\\
1.73	0.01\\
1.74	0.01\\
1.75	0.01\\
1.76	0.01\\
1.77	0.01\\
1.78	0.01\\
1.79	0.01\\
1.8	0.01\\
1.81	0.01\\
1.82	0.01\\
1.83	0.01\\
1.84	0.01\\
1.85	0.01\\
1.86	0.01\\
1.87	0.01\\
1.88	0.01\\
1.89	0.01\\
1.9	0.01\\
1.91	0.01\\
1.92	0.01\\
1.93	0.01\\
1.94	0.01\\
1.95	0.01\\
1.96	0.01\\
1.97	0.01\\
1.98	0.01\\
1.99	0.01\\
2	0.01\\
2.01	0.01\\
2.02	0.01\\
2.03	0.01\\
2.04	0.01\\
2.05	0.01\\
2.06	0.01\\
2.07	0.01\\
2.08	0.01\\
2.09	0.01\\
2.1	0.01\\
2.11	0.01\\
2.12	0.01\\
2.13	0.01\\
2.14	0.01\\
2.15	0.01\\
2.16	0.01\\
2.17	0.01\\
2.18	0.01\\
2.19	0.01\\
2.2	0.01\\
2.21	0.01\\
2.22	0.01\\
2.23	0.01\\
2.24	0.01\\
2.25	0.01\\
2.26	0.01\\
2.27	0.01\\
2.28	0.01\\
2.29	0.01\\
2.3	0.01\\
2.31	0.01\\
2.32	0.01\\
2.33	0.01\\
2.34	0.01\\
2.35	0.01\\
2.36	0.01\\
2.37	0.01\\
2.38	0.01\\
2.39	0.01\\
2.4	0.01\\
2.41	0.01\\
2.42	0.01\\
2.43	0.01\\
2.44	0.01\\
2.45	0.01\\
2.46	0.01\\
2.47	0.01\\
2.48	0.01\\
2.49	0.01\\
2.5	0.01\\
2.51	0.01\\
2.52	0.01\\
2.53	0.01\\
2.54	0.01\\
2.55	0.01\\
2.56	0.01\\
2.57	0.01\\
2.58	0.01\\
2.59	0.01\\
2.6	0.01\\
2.61	0.01\\
2.62	0.01\\
2.63	0.01\\
2.64	0.01\\
2.65	0.01\\
2.66	0.01\\
2.67	0.01\\
2.68	0.01\\
2.69	0.01\\
2.7	0.01\\
2.71	0.01\\
2.72	0.01\\
2.73	0.01\\
2.74	0.01\\
2.75	0.01\\
2.76	0.01\\
2.77	0.01\\
2.78	0.01\\
2.79	0.01\\
2.8	0.01\\
2.81	0.01\\
2.82	0.01\\
2.83	0.01\\
2.84	0.01\\
2.85	0.01\\
2.86	0.01\\
2.87	0.01\\
2.88	0.01\\
2.89	0.01\\
2.9	0.01\\
2.91	0.01\\
2.92	0.01\\
2.93	0.01\\
2.94	0.01\\
2.95	0.01\\
2.96	0.01\\
2.97	0.01\\
2.98	0.01\\
2.99	0.01\\
3	0.01\\
3.01	0.01\\
3.02	0.01\\
3.03	0.01\\
3.04	0.01\\
3.05	0.01\\
3.06	0.01\\
3.07	0.01\\
3.08	0.01\\
3.09	0.01\\
3.1	0.01\\
3.11	0.01\\
3.12	0.01\\
3.13	0.01\\
3.14	0.01\\
3.15	0.01\\
3.16	0.01\\
3.17	0.01\\
3.18	0.01\\
3.19	0.01\\
3.2	0.01\\
3.21	0.01\\
3.22	0.01\\
3.23	0.01\\
3.24	0.01\\
3.25	0.01\\
3.26	0.01\\
3.27	0.01\\
3.28	0.01\\
3.29	0.01\\
3.3	0.01\\
3.31	0.01\\
3.32	0.01\\
3.33	0.01\\
3.34	0.01\\
3.35	0.01\\
3.36	0.01\\
3.37	0.01\\
3.38	0.01\\
3.39	0.01\\
3.4	0.01\\
3.41	0.01\\
3.42	0.01\\
3.43	0.01\\
3.44	0.01\\
3.45	0.01\\
3.46	0.01\\
3.47	0.01\\
3.48	0.01\\
3.49	0.01\\
3.5	0.01\\
3.51	0.01\\
3.52	0.01\\
3.53	0.01\\
3.54	0.01\\
3.55	0.01\\
3.56	0.01\\
3.57	0.01\\
3.58	0.01\\
3.59	0.01\\
3.6	0.01\\
3.61	0.01\\
3.62	0.01\\
3.63	0.01\\
3.64	0.01\\
3.65	0.01\\
3.66	0.01\\
3.67	0.01\\
3.68	0.01\\
3.69	0.01\\
3.7	0.01\\
3.71	0.01\\
3.72	0.01\\
3.73	0.01\\
3.74	0.01\\
3.75	0.01\\
3.76	0.01\\
3.77	0.01\\
3.78	0.01\\
3.79	0.01\\
3.8	0.01\\
3.81	0.01\\
3.82	0.01\\
3.83	0.01\\
3.84	0.01\\
3.85	0.01\\
3.86	0.01\\
3.87	0.01\\
3.88	0.01\\
3.89	0.01\\
3.9	0.01\\
3.91	0.01\\
3.92	0.01\\
3.93	0.01\\
3.94	0.01\\
3.95	0.01\\
3.96	0.01\\
3.97	0.01\\
3.98	0.01\\
3.99	0.01\\
4	0.01\\
4.01	0.01\\
4.02	0.01\\
4.03	0.01\\
4.04	0.01\\
4.05	0.01\\
4.06	0.01\\
4.07	0.01\\
4.08	0.01\\
4.09	0.01\\
4.1	0.01\\
4.11	0.01\\
4.12	0.01\\
4.13	0.01\\
4.14	0.01\\
4.15	0.01\\
4.16	0.01\\
4.17	0.01\\
4.18	0.01\\
4.19	0.01\\
4.2	0.01\\
4.21	0.01\\
4.22	0.01\\
4.23	0.01\\
4.24	0.01\\
4.25	0.01\\
4.26	0.01\\
4.27	0.01\\
4.28	0.01\\
4.29	0.01\\
4.3	0.01\\
4.31	0.01\\
4.32	0.01\\
4.33	0.01\\
4.34	0.01\\
4.35	0.01\\
4.36	0.01\\
4.37	0.01\\
4.38	0.01\\
4.39	0.01\\
4.4	0.01\\
4.41	0.01\\
4.42	0.01\\
4.43	0.01\\
4.44	0.01\\
4.45	0.01\\
4.46	0.01\\
4.47	0.01\\
4.48	0.01\\
4.49	0.01\\
4.5	0.01\\
4.51	0.01\\
4.52	0.01\\
4.53	0.01\\
4.54	0.01\\
4.55	0.01\\
4.56	0.01\\
4.57	0.01\\
4.58	0.01\\
4.59	0.01\\
4.6	0.01\\
4.61	0.01\\
4.62	0.01\\
4.63	0.01\\
4.64	0.01\\
4.65	0.01\\
4.66	0.01\\
4.67	0.01\\
4.68	0.01\\
4.69	0.01\\
4.7	0.01\\
4.71	0.01\\
4.72	0.01\\
4.73	0.01\\
4.74	0.01\\
4.75	0.01\\
4.76	0.01\\
4.77	0.01\\
4.78	0.01\\
4.79	0.01\\
4.8	0.01\\
4.81	0.01\\
4.82	0.01\\
4.83	0.01\\
4.84	0.01\\
4.85	0.01\\
4.86	0.01\\
4.87	0.01\\
4.88	0.01\\
4.89	0.01\\
4.9	0.01\\
4.91	0.01\\
4.92	0.01\\
4.93	0.01\\
4.94	0.01\\
4.95	0.01\\
4.96	0.01\\
4.97	0.01\\
4.98	0.01\\
4.99	0.01\\
5	0.01\\
5.01	0.01\\
5.02	0.01\\
5.03	0.01\\
5.04	0.01\\
5.05	0.01\\
5.06	0.01\\
5.07	0.01\\
5.08	0.01\\
5.09	0.01\\
5.1	0.01\\
5.11	0.01\\
5.12	0.01\\
5.13	0.01\\
5.14	0.01\\
5.15	0.01\\
5.16	0.01\\
5.17	0.01\\
5.18	0.01\\
5.19	0.01\\
5.2	0.01\\
5.21	0.01\\
5.22	0.01\\
5.23	0.01\\
5.24	0.01\\
5.25	0.01\\
5.26	0.01\\
5.27	0.01\\
5.28	0.01\\
5.29	0.01\\
5.3	0.01\\
5.31	0.01\\
5.32	0.01\\
5.33	0.01\\
5.34	0.01\\
5.35	0.01\\
5.36	0.01\\
5.37	0.01\\
5.38	0.01\\
5.39	0.01\\
5.4	0.01\\
5.41	0.01\\
5.42	0.01\\
5.43	0.01\\
5.44	0.01\\
5.45	0.01\\
5.46	0.01\\
5.47	0.01\\
5.48	0.01\\
5.49	0.01\\
5.5	0.01\\
5.51	0.01\\
5.52	0.01\\
5.53	0.01\\
5.54	0.01\\
5.55	0.01\\
5.56	0.01\\
5.57	0.01\\
5.58	0.01\\
5.59	0.01\\
5.6	0.01\\
5.61	0.01\\
5.62	0.01\\
5.63	0.01\\
5.64	0.01\\
5.65	0.01\\
5.66	0.01\\
5.67	0.01\\
5.68	0.01\\
5.69	0.01\\
5.7	0.01\\
5.71	0.01\\
5.72	0.01\\
5.73	0.01\\
5.74	0.01\\
5.75	0.01\\
5.76	0.01\\
5.77	0.01\\
5.78	0.01\\
5.79	0.01\\
5.8	0.01\\
5.81	0.01\\
5.82	0.01\\
5.83	0.01\\
5.84	0.01\\
5.85	0.01\\
5.86	0.01\\
5.87	0.01\\
5.88	0.01\\
5.89	0.01\\
5.9	0.01\\
5.91	0.01\\
5.92	0.01\\
5.93	0.01\\
5.94	0.01\\
5.95	0.01\\
5.96	0.01\\
5.97	0.01\\
5.98	0.01\\
5.99	0.01\\
6	0.01\\
6.01	0.01\\
6.02	0.01\\
6.03	0.01\\
6.04	0.01\\
6.05	0.01\\
6.06	0.01\\
6.07	0.01\\
6.08	0.01\\
6.09	0.01\\
6.1	0.01\\
6.11	0.01\\
6.12	0.01\\
6.13	0.01\\
6.14	0.01\\
6.15	0.01\\
6.16	0.01\\
6.17	0.01\\
6.18	0.01\\
6.19	0.01\\
6.2	0.01\\
6.21	0.01\\
6.22	0.01\\
6.23	0.01\\
6.24	0.01\\
6.25	0.01\\
6.26	0.01\\
6.27	0.01\\
6.28	0.01\\
6.29	0.01\\
6.3	0.01\\
6.31	0.01\\
6.32	0.01\\
6.33	0.01\\
6.34	0.01\\
6.35	0.01\\
6.36	0.01\\
6.37	0.01\\
6.38	0.01\\
6.39	0.01\\
6.4	0.01\\
6.41	0.01\\
6.42	0.01\\
6.43	0.01\\
6.44	0.01\\
6.45	0.01\\
6.46	0.01\\
6.47	0.01\\
6.48	0.01\\
6.49	0.01\\
6.5	0.01\\
6.51	0.01\\
6.52	0.01\\
6.53	0.01\\
6.54	0.01\\
6.55	0.01\\
6.56	0.01\\
6.57	0.01\\
6.58	0.01\\
6.59	0.01\\
6.6	0.01\\
6.61	0.01\\
6.62	0.01\\
6.63	0.01\\
6.64	0.01\\
6.65	0.01\\
6.66	0.01\\
6.67	0.01\\
6.68	0.01\\
6.69	0.01\\
6.7	0.01\\
6.71	0.01\\
6.72	0.01\\
6.73	0.01\\
6.74	0.01\\
6.75	0.01\\
6.76	0.01\\
6.77	0.01\\
6.78	0.01\\
6.79	0.01\\
6.8	0.01\\
6.81	0.01\\
6.82	0.01\\
6.83	0.01\\
6.84	0.01\\
6.85	0.01\\
6.86	0.01\\
6.87	0.01\\
6.88	0.01\\
6.89	0.01\\
6.9	0.01\\
6.91	0.01\\
6.92	0.01\\
6.93	0.01\\
6.94	0.01\\
6.95	0.01\\
6.96	0.01\\
6.97	0.01\\
6.98	0.01\\
6.99	0.01\\
7	0.01\\
7.01	0.01\\
7.02	0.01\\
7.03	0.01\\
7.04	0.01\\
7.05	0.01\\
7.06	0.01\\
7.07	0.01\\
7.08	0.01\\
7.09	0.01\\
7.1	0.01\\
7.11	0.01\\
7.12	0.01\\
7.13	0.01\\
7.14	0.01\\
7.15	0.01\\
7.16	0.01\\
7.17	0.01\\
7.18	0.01\\
7.19	0.01\\
7.2	0.01\\
7.21	0.01\\
7.22	0.01\\
7.23	0.01\\
7.24	0.01\\
7.25	0.01\\
7.26	0.01\\
7.27	0.01\\
7.28	0.01\\
7.29	0.01\\
7.3	0.01\\
7.31	0.01\\
7.32	0.01\\
7.33	0.01\\
7.34	0.01\\
7.35	0.01\\
7.36	0.01\\
7.37	0.01\\
7.38	0.01\\
7.39	0.01\\
7.4	0.01\\
7.41	0.01\\
7.42	0.01\\
7.43	0.01\\
7.44	0.01\\
7.45	0.01\\
7.46	0.01\\
7.47	0.01\\
7.48	0.01\\
7.49	0.01\\
7.5	0.01\\
7.51	0.01\\
7.52	0.01\\
7.53	0.01\\
7.54	0.01\\
7.55	0.01\\
7.56	0.01\\
7.57	0.01\\
7.58	0.01\\
7.59	0.01\\
7.6	0.01\\
7.61	0.01\\
7.62	0.01\\
7.63	0.01\\
7.64	0.01\\
7.65	0.01\\
7.66	0.01\\
7.67	0.01\\
7.68	0.01\\
7.69	0.01\\
7.7	0.01\\
7.71	0.01\\
7.72	0.01\\
7.73	0.01\\
7.74	0.01\\
7.75	0.01\\
7.76	0.01\\
7.77	0.01\\
7.78	0.01\\
7.79	0.01\\
7.8	0.01\\
7.81	0.01\\
7.82	0.01\\
7.83	0.01\\
7.84	0.01\\
7.85	0.01\\
7.86	0.01\\
7.87	0.01\\
7.88	0.01\\
7.89	0.01\\
7.9	0.01\\
7.91	0.01\\
7.92	0.01\\
7.93	0.01\\
7.94	0.01\\
7.95	0.01\\
7.96	0.01\\
7.97	0.01\\
7.98	0.01\\
7.99	0.01\\
8	0.01\\
8.01	0.01\\
8.02	0.01\\
8.03	0.01\\
8.04	0.01\\
8.05	0.01\\
8.06	0.01\\
8.07	0.01\\
8.08	0.01\\
8.09	0.01\\
8.1	0.01\\
8.11	0.01\\
8.12	0.01\\
8.13	0.01\\
8.14	0.01\\
8.15	0.01\\
8.16	0.01\\
8.17	0.01\\
8.18	0.01\\
8.19	0.01\\
8.2	0.01\\
8.21	0.01\\
8.22	0.01\\
8.23	0.01\\
8.24	0.01\\
8.25	0.01\\
8.26	0.01\\
8.27	0.01\\
8.28	0.01\\
8.29	0.01\\
8.3	0.01\\
8.31	0.01\\
8.32	0.01\\
8.33	0.01\\
8.34	0.01\\
8.35	0.01\\
8.36	0.01\\
8.37	0.01\\
8.38	0.01\\
8.39	0.01\\
8.4	0.01\\
8.41	0.01\\
8.42	0.01\\
8.43	0.01\\
8.44	0.01\\
8.45	0.01\\
8.46	0.01\\
8.47	0.01\\
8.48	0.01\\
8.49	0.01\\
8.5	0.01\\
8.51	0.01\\
8.52	0.01\\
8.53	0.01\\
8.54	0.01\\
8.55	0.01\\
8.56	0.01\\
8.57	0.01\\
8.58	0.01\\
8.59	0.01\\
8.6	0.01\\
8.61	0.01\\
8.62	0.01\\
8.63	0.01\\
8.64	0.01\\
8.65	0.01\\
8.66	0.01\\
8.67	0.01\\
8.68	0.01\\
8.69	0.01\\
8.7	0.01\\
8.71	0.01\\
8.72	0.01\\
8.73	0.01\\
8.74	0.01\\
8.75	0.01\\
8.76	0.01\\
8.77	0.01\\
8.78	0.01\\
8.79	0.01\\
8.8	0.01\\
8.81	0.01\\
8.82	0.01\\
8.83	0.01\\
8.84	0.01\\
8.85	0.01\\
8.86	0.01\\
8.87	0.01\\
8.88	0.01\\
8.89	0.01\\
8.9	0.01\\
8.91	0.01\\
8.92	0.01\\
8.93	0.01\\
8.94	0.01\\
8.95	0.01\\
8.96	0.01\\
8.97	0.01\\
8.98	0.01\\
8.99	0.01\\
9	0.01\\
9.01	0.01\\
9.02	0.01\\
9.03	0.01\\
9.04	0.01\\
9.05	0.01\\
9.06	0.01\\
9.07	0.01\\
9.08	0.01\\
9.09	0.01\\
9.1	0.01\\
9.11	0.01\\
9.12	0.01\\
9.13	0.01\\
9.14	0.01\\
9.15	0.01\\
9.16	0.01\\
9.17	0.01\\
9.18	0.01\\
9.19	0.01\\
9.2	0.01\\
9.21	0.01\\
9.22	0.01\\
9.23	0.01\\
9.24	0.01\\
9.25	0.01\\
9.26	0.01\\
9.27	0.01\\
9.28	0.01\\
9.29	0.01\\
9.3	0.01\\
9.31	0.01\\
9.32	0.01\\
9.33	0.01\\
9.34	0.01\\
9.35	0.01\\
9.36	0.01\\
9.37	0.01\\
9.38	0.01\\
9.39	0.01\\
9.4	0.01\\
9.41	0.01\\
9.42	0.01\\
9.43	0.01\\
9.44	0.01\\
9.45	0.01\\
9.46	0.01\\
9.47	0.01\\
9.48	0.01\\
9.49	0.01\\
9.5	0.01\\
9.51	0.01\\
9.52	0.01\\
9.53	0.01\\
9.54	0.01\\
9.55	0.01\\
9.56	0.01\\
9.57	0.01\\
9.58	0.01\\
9.59	0.01\\
9.6	0.01\\
9.61	0.01\\
9.62	0.01\\
9.63	0.01\\
9.64	0.01\\
9.65	0.01\\
9.66	0.01\\
9.67	0.01\\
9.68	0.01\\
9.69	0.01\\
9.7	0.01\\
9.71	0.01\\
9.72	0.01\\
9.73	0.01\\
9.74	0.01\\
9.75	0.01\\
9.76	0.01\\
9.77	0.01\\
9.78	0.01\\
9.79	0.01\\
9.8	0.01\\
9.81	0.01\\
9.82	0.01\\
9.83	0.01\\
9.84	0.01\\
9.85	0.01\\
9.86	0.01\\
9.87	0.01\\
9.88	0.01\\
9.89	0.01\\
9.9	0.01\\
9.91	0.01\\
9.92	0.01\\
9.93	0.01\\
9.94	0.01\\
9.95	0.01\\
9.96	0.01\\
9.97	0.01\\
9.98	0.01\\
9.99	0.01\\
10	0.01\\
10.01	0.01\\
10.02	0.01\\
10.03	0.01\\
10.04	0.01\\
10.05	0.01\\
10.06	0.01\\
10.07	0.01\\
10.08	0.01\\
10.09	0.01\\
10.1	0.01\\
10.11	0.01\\
10.12	0.01\\
10.13	0.01\\
10.14	0.01\\
10.15	0.01\\
10.16	0.01\\
10.17	0.01\\
10.18	0.01\\
10.19	0.01\\
10.2	0.01\\
10.21	0.01\\
10.22	0.01\\
10.23	0.01\\
10.24	0.01\\
10.25	0.01\\
10.26	0.01\\
10.27	0.01\\
10.28	0.01\\
10.29	0.01\\
10.3	0.01\\
10.31	0.01\\
10.32	0.01\\
10.33	0.01\\
10.34	0.01\\
10.35	0.01\\
10.36	0.01\\
10.37	0.01\\
10.38	0.01\\
10.39	0.01\\
10.4	0.01\\
10.41	0.01\\
10.42	0.01\\
10.43	0.01\\
10.44	0.01\\
10.45	0.01\\
10.46	0.01\\
10.47	0.01\\
10.48	0.01\\
10.49	0.01\\
10.5	0.01\\
10.51	0.01\\
10.52	0.01\\
10.53	0.01\\
10.54	0.01\\
10.55	0.01\\
10.56	0.01\\
10.57	0.01\\
10.58	0.01\\
10.59	0.01\\
10.6	0.01\\
10.61	0.01\\
10.62	0.01\\
10.63	0.01\\
10.64	0.01\\
10.65	0.01\\
10.66	0.01\\
10.67	0.01\\
10.68	0.01\\
10.69	0.01\\
10.7	0.01\\
10.71	0.01\\
10.72	0.01\\
10.73	0.01\\
10.74	0.01\\
10.75	0.01\\
10.76	0.01\\
10.77	0.01\\
10.78	0.01\\
10.79	0.01\\
10.8	0.01\\
10.81	0.01\\
10.82	0.01\\
10.83	0.01\\
10.84	0.01\\
10.85	0.01\\
10.86	0.01\\
10.87	0.01\\
10.88	0.01\\
10.89	0.01\\
10.9	0.01\\
10.91	0.01\\
10.92	0.01\\
10.93	0.01\\
10.94	0.01\\
10.95	0.01\\
10.96	0.01\\
10.97	0.01\\
10.98	0.01\\
10.99	0.01\\
11	0.01\\
11.01	0.01\\
11.02	0.01\\
11.03	0.01\\
11.04	0.01\\
11.05	0.01\\
11.06	0.01\\
11.07	0.01\\
11.08	0.01\\
11.09	0.01\\
11.1	0.01\\
11.11	0.01\\
11.12	0.01\\
11.13	0.01\\
11.14	0.01\\
11.15	0.01\\
11.16	0.01\\
11.17	0.01\\
11.18	0.01\\
11.19	0.01\\
11.2	0.01\\
11.21	0.01\\
11.22	0.01\\
11.23	0.01\\
11.24	0.01\\
11.25	0.01\\
11.26	0.01\\
11.27	0.01\\
11.28	0.01\\
11.29	0.01\\
11.3	0.01\\
11.31	0.01\\
11.32	0.01\\
11.33	0.01\\
11.34	0.01\\
11.35	0.01\\
11.36	0.01\\
11.37	0.01\\
11.38	0.01\\
11.39	0.01\\
11.4	0.01\\
11.41	0.01\\
11.42	0.01\\
11.43	0.01\\
11.44	0.01\\
11.45	0.01\\
11.46	0.01\\
11.47	0.01\\
11.48	0.01\\
11.49	0.01\\
11.5	0.01\\
11.51	0.01\\
11.52	0.01\\
11.53	0.01\\
11.54	0.01\\
11.55	0.01\\
11.56	0.01\\
11.57	0.01\\
11.58	0.01\\
11.59	0.01\\
11.6	0.01\\
11.61	0.01\\
11.62	0.01\\
11.63	0.01\\
11.64	0.01\\
11.65	0.01\\
11.66	0.01\\
11.67	0.01\\
11.68	0.01\\
11.69	0.01\\
11.7	0.01\\
11.71	0.01\\
11.72	0.01\\
11.73	0.01\\
11.74	0.01\\
11.75	0.01\\
11.76	0.01\\
11.77	0.01\\
11.78	0.01\\
11.79	0.01\\
11.8	0.01\\
11.81	0.01\\
11.82	0.01\\
11.83	0.01\\
11.84	0.01\\
11.85	0.01\\
11.86	0.01\\
11.87	0.01\\
11.88	0.01\\
11.89	0.01\\
11.9	0.01\\
11.91	0.01\\
11.92	0.01\\
11.93	0.01\\
11.94	0.01\\
11.95	0.01\\
11.96	0.01\\
11.97	0.01\\
11.98	0.01\\
11.99	0.01\\
12	0.01\\
12.01	0.01\\
12.02	0.01\\
12.03	0.01\\
12.04	0.01\\
12.05	0.01\\
12.06	0.01\\
12.07	0.01\\
12.08	0.01\\
12.09	0.01\\
12.1	0.01\\
12.11	0.01\\
12.12	0.01\\
12.13	0.01\\
12.14	0.01\\
12.15	0.01\\
12.16	0.01\\
12.17	0.01\\
12.18	0.01\\
12.19	0.01\\
12.2	0.01\\
12.21	0.01\\
12.22	0.01\\
12.23	0.01\\
12.24	0.01\\
12.25	0.01\\
12.26	0.01\\
12.27	0.01\\
12.28	0.01\\
12.29	0.01\\
12.3	0.01\\
12.31	0.01\\
12.32	0.01\\
12.33	0.01\\
12.34	0.01\\
12.35	0.01\\
12.36	0.01\\
12.37	0.01\\
12.38	0.01\\
12.39	0.01\\
12.4	0.01\\
12.41	0.01\\
12.42	0.01\\
12.43	0.01\\
12.44	0.01\\
12.45	0.01\\
12.46	0.01\\
12.47	0.01\\
12.48	0.01\\
12.49	0.01\\
12.5	0.01\\
12.51	0.01\\
12.52	0.01\\
12.53	0.01\\
12.54	0.01\\
12.55	0.01\\
12.56	0.01\\
12.57	0.01\\
12.58	0.01\\
12.59	0.01\\
12.6	0.01\\
12.61	0.01\\
12.62	0.01\\
12.63	0.01\\
12.64	0.01\\
12.65	0.01\\
12.66	0.01\\
12.67	0.01\\
12.68	0.01\\
12.69	0.01\\
12.7	0.01\\
12.71	0.01\\
12.72	0.01\\
12.73	0.01\\
12.74	0.01\\
12.75	0.01\\
12.76	0.01\\
12.77	0.01\\
12.78	0.01\\
12.79	0.01\\
12.8	0.01\\
12.81	0.01\\
12.82	0.01\\
12.83	0.01\\
12.84	0.01\\
12.85	0.01\\
12.86	0.01\\
12.87	0.01\\
12.88	0.01\\
12.89	0.01\\
12.9	0.01\\
12.91	0.01\\
12.92	0.01\\
12.93	0.01\\
12.94	0.01\\
12.95	0.01\\
12.96	0.01\\
12.97	0.01\\
12.98	0.01\\
12.99	0.01\\
13	0.01\\
13.01	0.01\\
13.02	0.01\\
13.03	0.01\\
13.04	0.01\\
13.05	0.01\\
13.06	0.01\\
13.07	0.01\\
13.08	0.01\\
13.09	0.01\\
13.1	0.01\\
13.11	0.01\\
13.12	0.01\\
13.13	0.01\\
13.14	0.01\\
13.15	0.01\\
13.16	0.01\\
13.17	0.01\\
13.18	0.01\\
13.19	0.01\\
13.2	0.01\\
13.21	0.01\\
13.22	0.01\\
13.23	0.01\\
13.24	0.01\\
13.25	0.01\\
13.26	0.01\\
13.27	0.01\\
13.28	0.01\\
13.29	0.01\\
13.3	0.01\\
13.31	0.01\\
13.32	0.01\\
13.33	0.01\\
13.34	0.01\\
13.35	0.01\\
13.36	0.01\\
13.37	0.01\\
13.38	0.01\\
13.39	0.01\\
13.4	0.01\\
13.41	0.01\\
13.42	0.01\\
13.43	0.01\\
13.44	0.01\\
13.45	0.01\\
13.46	0.01\\
13.47	0.01\\
13.48	0.01\\
13.49	0.01\\
13.5	0.01\\
13.51	0.01\\
13.52	0.01\\
13.53	0.01\\
13.54	0.01\\
13.55	0.01\\
13.56	0.01\\
13.57	0.01\\
13.58	0.01\\
13.59	0.01\\
13.6	0.01\\
13.61	0.01\\
13.62	0.01\\
13.63	0.01\\
13.64	0.01\\
13.65	0.01\\
13.66	0.01\\
13.67	0.01\\
13.68	0.01\\
13.69	0.01\\
13.7	0.01\\
13.71	0.01\\
13.72	0.01\\
13.73	0.01\\
13.74	0.01\\
13.75	0.01\\
13.76	0.01\\
13.77	0.01\\
13.78	0.01\\
13.79	0.01\\
13.8	0.01\\
13.81	0.01\\
13.82	0.01\\
13.83	0.01\\
13.84	0.01\\
13.85	0.01\\
13.86	0.01\\
13.87	0.01\\
13.88	0.01\\
13.89	0.01\\
13.9	0.01\\
13.91	0.01\\
13.92	0.01\\
13.93	0.01\\
13.94	0.01\\
13.95	0.01\\
13.96	0.01\\
13.97	0.01\\
13.98	0.01\\
13.99	0.01\\
14	0.01\\
14.01	0.01\\
14.02	0.01\\
14.03	0.01\\
14.04	0.01\\
14.05	0.01\\
14.06	0.01\\
14.07	0.01\\
14.08	0.01\\
14.09	0.01\\
14.1	0.01\\
14.11	0.01\\
14.12	0.01\\
14.13	0.01\\
14.14	0.01\\
14.15	0.01\\
14.16	0.01\\
14.17	0.01\\
14.18	0.01\\
14.19	0.01\\
14.2	0.01\\
14.21	0.01\\
14.22	0.01\\
14.23	0.01\\
14.24	0.01\\
14.25	0.01\\
14.26	0.01\\
14.27	0.01\\
14.28	0.01\\
14.29	0.01\\
14.3	0.01\\
14.31	0.01\\
14.32	0.01\\
14.33	0.01\\
14.34	0.01\\
14.35	0.01\\
14.36	0.01\\
14.37	0.01\\
14.38	0.01\\
14.39	0.01\\
14.4	0.01\\
14.41	0.01\\
14.42	0.01\\
14.43	0.01\\
14.44	0.01\\
14.45	0.01\\
14.46	0.01\\
14.47	0.01\\
14.48	0.01\\
14.49	0.01\\
14.5	0.01\\
14.51	0.01\\
14.52	0.01\\
14.53	0.01\\
14.54	0.01\\
14.55	0.01\\
14.56	0.01\\
14.57	0.01\\
14.58	0.01\\
14.59	0.01\\
14.6	0.01\\
14.61	0.01\\
14.62	0.01\\
14.63	0.01\\
14.64	0.01\\
14.65	0.01\\
14.66	0.01\\
14.67	0.01\\
14.68	0.01\\
14.69	0.01\\
14.7	0.01\\
14.71	0.01\\
14.72	0.01\\
14.73	0.01\\
14.74	0.01\\
14.75	0.01\\
14.76	0.01\\
14.77	0.01\\
14.78	0.01\\
14.79	0.01\\
14.8	0.01\\
14.81	0.01\\
14.82	0.01\\
14.83	0.01\\
14.84	0.01\\
14.85	0.01\\
14.86	0.01\\
14.87	0.01\\
14.88	0.01\\
14.89	0.01\\
14.9	0.01\\
14.91	0.01\\
14.92	0.01\\
14.93	0.01\\
14.94	0.01\\
14.95	0.01\\
14.96	0.01\\
14.97	0.01\\
14.98	0.01\\
14.99	0.01\\
15	0.01\\
15.01	0.01\\
15.02	0.01\\
15.03	0.01\\
15.04	0.01\\
15.05	0.01\\
15.06	0.01\\
15.07	0.01\\
15.08	0.01\\
15.09	0.01\\
15.1	0.01\\
15.11	0.01\\
15.12	0.01\\
15.13	0.01\\
15.14	0.01\\
15.15	0.01\\
15.16	0.01\\
15.17	0.01\\
15.18	0.01\\
15.19	0.01\\
15.2	0.01\\
15.21	0.01\\
15.22	0.01\\
15.23	0.01\\
15.24	0.01\\
15.25	0.01\\
15.26	0.01\\
15.27	0.01\\
15.28	0.01\\
15.29	0.01\\
15.3	0.01\\
15.31	0.01\\
15.32	0.01\\
15.33	0.01\\
15.34	0.01\\
15.35	0.01\\
15.36	0.01\\
15.37	0.01\\
15.38	0.01\\
15.39	0.01\\
15.4	0.01\\
15.41	0.01\\
15.42	0.01\\
15.43	0.01\\
15.44	0.01\\
15.45	0.01\\
15.46	0.01\\
15.47	0.01\\
15.48	0.01\\
15.49	0.01\\
15.5	0.01\\
15.51	0.01\\
15.52	0.01\\
15.53	0.01\\
15.54	0.01\\
15.55	0.01\\
15.56	0.01\\
15.57	0.01\\
15.58	0.01\\
15.59	0.01\\
15.6	0.01\\
15.61	0.01\\
15.62	0.01\\
15.63	0.01\\
15.64	0.01\\
15.65	0.01\\
15.66	0.01\\
15.67	0.01\\
15.68	0.01\\
15.69	0.01\\
15.7	0.01\\
15.71	0.01\\
15.72	0.01\\
15.73	0.01\\
15.74	0.01\\
15.75	0.01\\
15.76	0.01\\
15.77	0.01\\
15.78	0.01\\
15.79	0.01\\
15.8	0.01\\
15.81	0.01\\
15.82	0.01\\
15.83	0.01\\
15.84	0.01\\
15.85	0.01\\
15.86	0.01\\
15.87	0.01\\
15.88	0.01\\
15.89	0.01\\
15.9	0.01\\
15.91	0.01\\
15.92	0.01\\
15.93	0.01\\
15.94	0.01\\
15.95	0.01\\
15.96	0.01\\
15.97	0.01\\
15.98	0.01\\
15.99	0.01\\
16	0.01\\
16.01	0.01\\
16.02	0.01\\
16.03	0.01\\
16.04	0.01\\
16.05	0.01\\
16.06	0.01\\
16.07	0.01\\
16.08	0.01\\
16.09	0.01\\
16.1	0.01\\
16.11	0.01\\
16.12	0.01\\
16.13	0.01\\
16.14	0.01\\
16.15	0.01\\
16.16	0.01\\
16.17	0.01\\
16.18	0.01\\
16.19	0.01\\
16.2	0.01\\
16.21	0.01\\
16.22	0.01\\
16.23	0.01\\
16.24	0.01\\
16.25	0.01\\
16.26	0.01\\
16.27	0.01\\
16.28	0.01\\
16.29	0.01\\
16.3	0.01\\
16.31	0.01\\
16.32	0.01\\
16.33	0.01\\
16.34	0.01\\
16.35	0.01\\
16.36	0.01\\
16.37	0.01\\
16.38	0.01\\
16.39	0.01\\
16.4	0.01\\
16.41	0.01\\
16.42	0.01\\
16.43	0.01\\
16.44	0.01\\
16.45	0.01\\
16.46	0.01\\
16.47	0.01\\
16.48	0.01\\
16.49	0.01\\
16.5	0.01\\
16.51	0.01\\
16.52	0.01\\
16.53	0.01\\
16.54	0.01\\
16.55	0.01\\
16.56	0.01\\
16.57	0.01\\
16.58	0.01\\
16.59	0.01\\
16.6	0.01\\
16.61	0.01\\
16.62	0.01\\
16.63	0.01\\
16.64	0.01\\
16.65	0.01\\
16.66	0.01\\
16.67	0.01\\
16.68	0.01\\
16.69	0.01\\
16.7	0.01\\
16.71	0.01\\
16.72	0.01\\
16.73	0.01\\
16.74	0.01\\
16.75	0.01\\
16.76	0.01\\
16.77	0.01\\
16.78	0.01\\
16.79	0.01\\
16.8	0.01\\
16.81	0.01\\
16.82	0.01\\
16.83	0.01\\
16.84	0.01\\
16.85	0.01\\
16.86	0.01\\
16.87	0.01\\
16.88	0.01\\
16.89	0.01\\
16.9	0.01\\
16.91	0.01\\
16.92	0.01\\
16.93	0.01\\
16.94	0.01\\
16.95	0.01\\
16.96	0.01\\
16.97	0.01\\
16.98	0.01\\
16.99	0.01\\
17	0.01\\
17.01	0.01\\
17.02	0.01\\
17.03	0.01\\
17.04	0.01\\
17.05	0.01\\
17.06	0.01\\
17.07	0.01\\
17.08	0.01\\
17.09	0.01\\
17.1	0.01\\
17.11	0.01\\
17.12	0.01\\
17.13	0.01\\
17.14	0.01\\
17.15	0.01\\
17.16	0.01\\
17.17	0.01\\
17.18	0.01\\
17.19	0.01\\
17.2	0.01\\
17.21	0.01\\
17.22	0.01\\
17.23	0.01\\
17.24	0.01\\
17.25	0.01\\
17.26	0.01\\
17.27	0.01\\
17.28	0.01\\
17.29	0.01\\
17.3	0.01\\
17.31	0.01\\
17.32	0.01\\
17.33	0.01\\
17.34	0.01\\
17.35	0.01\\
17.36	0.01\\
17.37	0.01\\
17.38	0.01\\
17.39	0.01\\
17.4	0.01\\
17.41	0.01\\
17.42	0.01\\
17.43	0.01\\
17.44	0.01\\
17.45	0.01\\
17.46	0.01\\
17.47	0.01\\
17.48	0.01\\
17.49	0.01\\
17.5	0.01\\
17.51	0.01\\
17.52	0.01\\
17.53	0.01\\
17.54	0.01\\
17.55	0.01\\
17.56	0.01\\
17.57	0.01\\
17.58	0.01\\
17.59	0.01\\
17.6	0.01\\
17.61	0.01\\
17.62	0.01\\
17.63	0.01\\
17.64	0.01\\
17.65	0.01\\
17.66	0.01\\
17.67	0.01\\
17.68	0.01\\
17.69	0.01\\
17.7	0.01\\
17.71	0.01\\
17.72	0.01\\
17.73	0.01\\
17.74	0.01\\
17.75	0.01\\
17.76	0.01\\
17.77	0.01\\
17.78	0.01\\
17.79	0.01\\
17.8	0.01\\
17.81	0.01\\
17.82	0.01\\
17.83	0.01\\
17.84	0.01\\
17.85	0.01\\
17.86	0.01\\
17.87	0.01\\
17.88	0.01\\
17.89	0.01\\
17.9	0.01\\
17.91	0.01\\
17.92	0.01\\
17.93	0.01\\
17.94	0.01\\
17.95	0.01\\
17.96	0.01\\
17.97	0.01\\
17.98	0.01\\
17.99	0.01\\
18	0.01\\
18.01	0.01\\
18.02	0.01\\
18.03	0.01\\
18.04	0.01\\
18.05	0.01\\
18.06	0.01\\
18.07	0.01\\
18.08	0.01\\
18.09	0.01\\
18.1	0.01\\
18.11	0.01\\
18.12	0.01\\
18.13	0.01\\
18.14	0.01\\
18.15	0.01\\
18.16	0.01\\
18.17	0.01\\
18.18	0.01\\
18.19	0.01\\
18.2	0.01\\
18.21	0.01\\
18.22	0.01\\
18.23	0.01\\
18.24	0.01\\
18.25	0.01\\
18.26	0.01\\
18.27	0.01\\
18.28	0.01\\
18.29	0.01\\
18.3	0.01\\
18.31	0.01\\
18.32	0.01\\
18.33	0.01\\
18.34	0.01\\
18.35	0.01\\
18.36	0.01\\
18.37	0.01\\
18.38	0.01\\
18.39	0.01\\
18.4	0.01\\
18.41	0.01\\
18.42	0.01\\
18.43	0.01\\
18.44	0.01\\
18.45	0.01\\
18.46	0.01\\
18.47	0.01\\
18.48	0.01\\
18.49	0.01\\
18.5	0.01\\
18.51	0.01\\
18.52	0.01\\
18.53	0.01\\
18.54	0.01\\
18.55	0.01\\
18.56	0.01\\
18.57	0.01\\
18.58	0.01\\
18.59	0.01\\
18.6	0.01\\
18.61	0.01\\
18.62	0.01\\
18.63	0.01\\
18.64	0.01\\
18.65	0.01\\
18.66	0.01\\
18.67	0.01\\
18.68	0.01\\
18.69	0.01\\
18.7	0.01\\
18.71	0.01\\
18.72	0.01\\
18.73	0.01\\
18.74	0.01\\
18.75	0.01\\
18.76	0.01\\
18.77	0.01\\
18.78	0.01\\
18.79	0.01\\
18.8	0.01\\
18.81	0.01\\
18.82	0.01\\
18.83	0.01\\
18.84	0.01\\
18.85	0.01\\
18.86	0.01\\
18.87	0.01\\
18.88	0.01\\
18.89	0.01\\
18.9	0.01\\
18.91	0.01\\
18.92	0.01\\
18.93	0.01\\
18.94	0.01\\
18.95	0.01\\
18.96	0.01\\
18.97	0.01\\
18.98	0.01\\
18.99	0.01\\
19	0.01\\
19.01	0.01\\
19.02	0.01\\
19.03	0.01\\
19.04	0.01\\
19.05	0.01\\
19.06	0.01\\
19.07	0.01\\
19.08	0.01\\
19.09	0.01\\
19.1	0.01\\
19.11	0.01\\
19.12	0.01\\
19.13	0.01\\
19.14	0.01\\
19.15	0.01\\
19.16	0.01\\
19.17	0.01\\
19.18	0.01\\
19.19	0.01\\
19.2	0.01\\
19.21	0.01\\
19.22	0.01\\
19.23	0.01\\
19.24	0.01\\
19.25	0.01\\
19.26	0.01\\
19.27	0.01\\
19.28	0.01\\
19.29	0.01\\
19.3	0.01\\
19.31	0.01\\
19.32	0.01\\
19.33	0.01\\
19.34	0.01\\
19.35	0.01\\
19.36	0.01\\
19.37	0.01\\
19.38	0.01\\
19.39	0.01\\
19.4	0.01\\
19.41	0.01\\
19.42	0.01\\
19.43	0.01\\
19.44	0.01\\
19.45	0.01\\
19.46	0.01\\
19.47	0.01\\
19.48	0.01\\
19.49	0.01\\
19.5	0.01\\
19.51	0.01\\
19.52	0.01\\
19.53	0.01\\
19.54	0.01\\
19.55	0.01\\
19.56	0.01\\
19.57	0.01\\
19.58	0.01\\
19.59	0.01\\
19.6	0.01\\
19.61	0.01\\
19.62	0.01\\
19.63	0.01\\
19.64	0.01\\
19.65	0.01\\
19.66	0.01\\
19.67	0.01\\
19.68	0.01\\
19.69	0.01\\
19.7	0.01\\
19.71	0.01\\
19.72	0.01\\
19.73	0.01\\
19.74	0.01\\
19.75	0.01\\
19.76	0.01\\
19.77	0.01\\
19.78	0.01\\
19.79	0.01\\
19.8	0.01\\
19.81	0.01\\
19.82	0.01\\
19.83	0.01\\
19.84	0.01\\
19.85	0.01\\
19.86	0.01\\
19.87	0.01\\
19.88	0.01\\
19.89	0.01\\
19.9	0.01\\
19.91	0.01\\
19.92	0.01\\
19.93	0.01\\
19.94	0.01\\
19.95	0.01\\
19.96	0.01\\
19.97	0.01\\
19.98	0.01\\
19.99	0.01\\
20	0.01\\
20.01	0.01\\
20.02	0.01\\
20.03	0.01\\
20.04	0.01\\
20.05	0.01\\
20.06	0.01\\
20.07	0.01\\
20.08	0.01\\
20.09	0.01\\
20.1	0.01\\
20.11	0.01\\
20.12	0.01\\
20.13	0.01\\
20.14	0.01\\
20.15	0.01\\
20.16	0.01\\
20.17	0.01\\
20.18	0.01\\
20.19	0.01\\
20.2	0.01\\
20.21	0.01\\
20.22	0.01\\
20.23	0.01\\
20.24	0.01\\
20.25	0.01\\
20.26	0.01\\
20.27	0.01\\
20.28	0.01\\
20.29	0.01\\
20.3	0.01\\
20.31	0.01\\
20.32	0.01\\
20.33	0.01\\
20.34	0.01\\
20.35	0.01\\
20.36	0.01\\
20.37	0.01\\
20.38	0.01\\
20.39	0.01\\
20.4	0.01\\
20.41	0.01\\
20.42	0.01\\
20.43	0.01\\
20.44	0.01\\
20.45	0.01\\
20.46	0.01\\
20.47	0.01\\
20.48	0.01\\
20.49	0.01\\
20.5	0.01\\
20.51	0.01\\
20.52	0.01\\
20.53	0.01\\
20.54	0.01\\
20.55	0.01\\
20.56	0.01\\
20.57	0.01\\
20.58	0.01\\
20.59	0.01\\
20.6	0.01\\
20.61	0.01\\
20.62	0.01\\
20.63	0.01\\
20.64	0.01\\
20.65	0.01\\
20.66	0.01\\
20.67	0.01\\
20.68	0.01\\
20.69	0.01\\
20.7	0.01\\
20.71	0.01\\
20.72	0.01\\
20.73	0.01\\
20.74	0.01\\
20.75	0.01\\
20.76	0.01\\
20.77	0.01\\
20.78	0.01\\
20.79	0.01\\
20.8	0.01\\
20.81	0.01\\
20.82	0.01\\
20.83	0.01\\
20.84	0.01\\
20.85	0.01\\
20.86	0.01\\
20.87	0.01\\
20.88	0.01\\
20.89	0.01\\
20.9	0.01\\
20.91	0.01\\
20.92	0.01\\
20.93	0.01\\
20.94	0.01\\
20.95	0.01\\
20.96	0.01\\
20.97	0.01\\
20.98	0.01\\
20.99	0.01\\
21	0.01\\
21.01	0.01\\
21.02	0.01\\
21.03	0.01\\
21.04	0.01\\
21.05	0.01\\
21.06	0.01\\
21.07	0.01\\
21.08	0.01\\
21.09	0.01\\
21.1	0.01\\
21.11	0.01\\
21.12	0.01\\
21.13	0.01\\
21.14	0.01\\
21.15	0.01\\
21.16	0.01\\
21.17	0.01\\
21.18	0.01\\
21.19	0.01\\
21.2	0.01\\
21.21	0.01\\
21.22	0.01\\
21.23	0.01\\
21.24	0.01\\
21.25	0.01\\
21.26	0.01\\
21.27	0.01\\
21.28	0.01\\
21.29	0.01\\
21.3	0.01\\
21.31	0.01\\
21.32	0.01\\
21.33	0.01\\
21.34	0.01\\
21.35	0.01\\
21.36	0.01\\
21.37	0.01\\
21.38	0.01\\
21.39	0.01\\
21.4	0.01\\
21.41	0.01\\
21.42	0.01\\
21.43	0.01\\
21.44	0.01\\
21.45	0.01\\
21.46	0.01\\
21.47	0.01\\
21.48	0.01\\
21.49	0.01\\
21.5	0.01\\
21.51	0.01\\
21.52	0.01\\
21.53	0.01\\
21.54	0.01\\
21.55	0.01\\
21.56	0.01\\
21.57	0.01\\
21.58	0.01\\
21.59	0.01\\
21.6	0.01\\
21.61	0.01\\
21.62	0.01\\
21.63	0.01\\
21.64	0.01\\
21.65	0.01\\
21.66	0.01\\
21.67	0.01\\
21.68	0.01\\
21.69	0.01\\
21.7	0.01\\
21.71	0.01\\
21.72	0.01\\
21.73	0.01\\
21.74	0.01\\
21.75	0.01\\
21.76	0.01\\
21.77	0.01\\
21.78	0.01\\
21.79	0.01\\
21.8	0.01\\
21.81	0.01\\
21.82	0.01\\
21.83	0.01\\
21.84	0.01\\
21.85	0.01\\
21.86	0.01\\
21.87	0.01\\
21.88	0.01\\
21.89	0.01\\
21.9	0.01\\
21.91	0.01\\
21.92	0.01\\
21.93	0.01\\
21.94	0.01\\
21.95	0.01\\
21.96	0.01\\
21.97	0.01\\
21.98	0.01\\
21.99	0.01\\
22	0.01\\
22.01	0.01\\
22.02	0.01\\
22.03	0.01\\
22.04	0.01\\
22.05	0.01\\
22.06	0.01\\
22.07	0.01\\
22.08	0.01\\
22.09	0.01\\
22.1	0.01\\
22.11	0.01\\
22.12	0.01\\
22.13	0.01\\
22.14	0.01\\
22.15	0.01\\
22.16	0.01\\
22.17	0.01\\
22.18	0.01\\
22.19	0.01\\
22.2	0.01\\
22.21	0.01\\
22.22	0.01\\
22.23	0.01\\
22.24	0.01\\
22.25	0.01\\
22.26	0.01\\
22.27	0.01\\
22.28	0.01\\
22.29	0.01\\
22.3	0.01\\
22.31	0.01\\
22.32	0.01\\
22.33	0.01\\
22.34	0.01\\
22.35	0.01\\
22.36	0.01\\
22.37	0.01\\
22.38	0.01\\
22.39	0.01\\
22.4	0.01\\
22.41	0.01\\
22.42	0.01\\
22.43	0.01\\
22.44	0.01\\
22.45	0.01\\
22.46	0.01\\
22.47	0.01\\
22.48	0.01\\
22.49	0.01\\
22.5	0.01\\
22.51	0.01\\
22.52	0.01\\
22.53	0.01\\
22.54	0.01\\
22.55	0.01\\
22.56	0.01\\
22.57	0.01\\
22.58	0.01\\
22.59	0.01\\
22.6	0.01\\
22.61	0.01\\
22.62	0.01\\
22.63	0.01\\
22.64	0.01\\
22.65	0.01\\
22.66	0.01\\
22.67	0.01\\
22.68	0.01\\
22.69	0.01\\
22.7	0.01\\
22.71	0.01\\
22.72	0.01\\
22.73	0.01\\
22.74	0.01\\
22.75	0.01\\
22.76	0.01\\
22.77	0.01\\
22.78	0.01\\
22.79	0.01\\
22.8	0.01\\
22.81	0.01\\
22.82	0.01\\
22.83	0.01\\
22.84	0.01\\
22.85	0.01\\
22.86	0.01\\
22.87	0.01\\
22.88	0.01\\
22.89	0.01\\
22.9	0.01\\
22.91	0.01\\
22.92	0.01\\
22.93	0.01\\
22.94	0.01\\
22.95	0.01\\
22.96	0.01\\
22.97	0.01\\
22.98	0.01\\
22.99	0.01\\
23	0.01\\
23.01	0.01\\
23.02	0.01\\
23.03	0.01\\
23.04	0.01\\
23.05	0.01\\
23.06	0.01\\
23.07	0.01\\
23.08	0.01\\
23.09	0.01\\
23.1	0.01\\
23.11	0.01\\
23.12	0.01\\
23.13	0.01\\
23.14	0.01\\
23.15	0.01\\
23.16	0.01\\
23.17	0.01\\
23.18	0.01\\
23.19	0.01\\
23.2	0.01\\
23.21	0.01\\
23.22	0.01\\
23.23	0.01\\
23.24	0.01\\
23.25	0.01\\
23.26	0.01\\
23.27	0.01\\
23.28	0.01\\
23.29	0.01\\
23.3	0.01\\
23.31	0.01\\
23.32	0.01\\
23.33	0.01\\
23.34	0.01\\
23.35	0.01\\
23.36	0.01\\
23.37	0.01\\
23.38	0.01\\
23.39	0.01\\
23.4	0.01\\
23.41	0.01\\
23.42	0.01\\
23.43	0.01\\
23.44	0.01\\
23.45	0.01\\
23.46	0.01\\
23.47	0.01\\
23.48	0.01\\
23.49	0.01\\
23.5	0.01\\
23.51	0.01\\
23.52	0.01\\
23.53	0.01\\
23.54	0.01\\
23.55	0.01\\
23.56	0.01\\
23.57	0.01\\
23.58	0.01\\
23.59	0.01\\
23.6	0.01\\
23.61	0.01\\
23.62	0.01\\
23.63	0.01\\
23.64	0.01\\
23.65	0.01\\
23.66	0.01\\
23.67	0.01\\
23.68	0.01\\
23.69	0.01\\
23.7	0.01\\
23.71	0.01\\
23.72	0.01\\
23.73	0.01\\
23.74	0.01\\
23.75	0.01\\
23.76	0.01\\
23.77	0.01\\
23.78	0.01\\
23.79	0.01\\
23.8	0.01\\
23.81	0.01\\
23.82	0.01\\
23.83	0.01\\
23.84	0.01\\
23.85	0.01\\
23.86	0.01\\
23.87	0.01\\
23.88	0.01\\
23.89	0.01\\
23.9	0.01\\
23.91	0.01\\
23.92	0.01\\
23.93	0.01\\
23.94	0.01\\
23.95	0.01\\
23.96	0.01\\
23.97	0.01\\
23.98	0.01\\
23.99	0.01\\
24	0.01\\
24.01	0.01\\
24.02	0.01\\
24.03	0.01\\
24.04	0.01\\
24.05	0.01\\
24.06	0.01\\
24.07	0.01\\
24.08	0.01\\
24.09	0.01\\
24.1	0.01\\
24.11	0.01\\
24.12	0.01\\
24.13	0.01\\
24.14	0.01\\
24.15	0.01\\
24.16	0.01\\
24.17	0.01\\
24.18	0.01\\
24.19	0.01\\
24.2	0.01\\
24.21	0.01\\
24.22	0.01\\
24.23	0.01\\
24.24	0.01\\
24.25	0.01\\
24.26	0.01\\
24.27	0.01\\
24.28	0.01\\
24.29	0.01\\
24.3	0.01\\
24.31	0.01\\
24.32	0.01\\
24.33	0.01\\
24.34	0.01\\
24.35	0.01\\
24.36	0.01\\
24.37	0.01\\
24.38	0.01\\
24.39	0.01\\
24.4	0.01\\
24.41	0.01\\
24.42	0.01\\
24.43	0.01\\
24.44	0.01\\
24.45	0.01\\
24.46	0.01\\
24.47	0.01\\
24.48	0.01\\
24.49	0.01\\
24.5	0.01\\
24.51	0.01\\
24.52	0.01\\
24.53	0.01\\
24.54	0.01\\
24.55	0.01\\
24.56	0.01\\
24.57	0.01\\
24.58	0.01\\
24.59	0.01\\
24.6	0.01\\
24.61	0.01\\
24.62	0.01\\
24.63	0.01\\
24.64	0.01\\
24.65	0.01\\
24.66	0.01\\
24.67	0.01\\
24.68	0.01\\
24.69	0.01\\
24.7	0.01\\
24.71	0.01\\
24.72	0.01\\
24.73	0.01\\
24.74	0.01\\
24.75	0.01\\
24.76	0.01\\
24.77	0.01\\
24.78	0.01\\
24.79	0.01\\
24.8	0.01\\
24.81	0.01\\
24.82	0.01\\
24.83	0.01\\
24.84	0.01\\
24.85	0.01\\
24.86	0.01\\
24.87	0.01\\
24.88	0.01\\
24.89	0.01\\
24.9	0.01\\
24.91	0.01\\
24.92	0.01\\
24.93	0.01\\
24.94	0.01\\
24.95	0.01\\
24.96	0.01\\
24.97	0.01\\
24.98	0.01\\
24.99	0.01\\
25	0.01\\
25.01	0.01\\
25.02	0.01\\
25.03	0.01\\
25.04	0.01\\
25.05	0.01\\
25.06	0.01\\
25.07	0.01\\
25.08	0.01\\
25.09	0.01\\
25.1	0.01\\
25.11	0.01\\
25.12	0.01\\
25.13	0.01\\
25.14	0.01\\
25.15	0.01\\
25.16	0.01\\
25.17	0.01\\
25.18	0.01\\
25.19	0.01\\
25.2	0.01\\
25.21	0.01\\
25.22	0.01\\
25.23	0.01\\
25.24	0.01\\
25.25	0.01\\
25.26	0.01\\
25.27	0.01\\
25.28	0.01\\
25.29	0.01\\
25.3	0.01\\
25.31	0.01\\
25.32	0.01\\
25.33	0.01\\
25.34	0.01\\
25.35	0.01\\
25.36	0.01\\
25.37	0.01\\
25.38	0.01\\
25.39	0.01\\
25.4	0.01\\
25.41	0.01\\
25.42	0.01\\
25.43	0.01\\
25.44	0.01\\
25.45	0.01\\
25.46	0.01\\
25.47	0.01\\
25.48	0.01\\
25.49	0.01\\
25.5	0.01\\
25.51	0.01\\
25.52	0.01\\
25.53	0.01\\
25.54	0.01\\
25.55	0.01\\
25.56	0.01\\
25.57	0.01\\
25.58	0.01\\
25.59	0.01\\
25.6	0.01\\
25.61	0.01\\
25.62	0.01\\
25.63	0.01\\
25.64	0.01\\
25.65	0.01\\
25.66	0.01\\
25.67	0.01\\
25.68	0.01\\
25.69	0.01\\
25.7	0.01\\
25.71	0.01\\
25.72	0.01\\
25.73	0.01\\
25.74	0.01\\
25.75	0.01\\
25.76	0.01\\
25.77	0.01\\
25.78	0.01\\
25.79	0.01\\
25.8	0.01\\
25.81	0.01\\
25.82	0.01\\
25.83	0.01\\
25.84	0.01\\
25.85	0.01\\
25.86	0.01\\
25.87	0.01\\
25.88	0.01\\
25.89	0.01\\
25.9	0.01\\
25.91	0.01\\
25.92	0.01\\
25.93	0.01\\
25.94	0.01\\
25.95	0.01\\
25.96	0.01\\
25.97	0.01\\
25.98	0.01\\
25.99	0.01\\
26	0.01\\
26.01	0.01\\
26.02	0.01\\
26.03	0.01\\
26.04	0.01\\
26.05	0.01\\
26.06	0.01\\
26.07	0.01\\
26.08	0.01\\
26.09	0.01\\
26.1	0.01\\
26.11	0.01\\
26.12	0.01\\
26.13	0.01\\
26.14	0.01\\
26.15	0.01\\
26.16	0.01\\
26.17	0.01\\
26.18	0.01\\
26.19	0.01\\
26.2	0.01\\
26.21	0.01\\
26.22	0.01\\
26.23	0.01\\
26.24	0.01\\
26.25	0.01\\
26.26	0.01\\
26.27	0.01\\
26.28	0.01\\
26.29	0.01\\
26.3	0.01\\
26.31	0.01\\
26.32	0.01\\
26.33	0.01\\
26.34	0.01\\
26.35	0.01\\
26.36	0.01\\
26.37	0.01\\
26.38	0.01\\
26.39	0.01\\
26.4	0.01\\
26.41	0.01\\
26.42	0.01\\
26.43	0.01\\
26.44	0.01\\
26.45	0.01\\
26.46	0.01\\
26.47	0.01\\
26.48	0.01\\
26.49	0.01\\
26.5	0.01\\
26.51	0.01\\
26.52	0.01\\
26.53	0.01\\
26.54	0.01\\
26.55	0.01\\
26.56	0.01\\
26.57	0.01\\
26.58	0.01\\
26.59	0.01\\
26.6	0.01\\
26.61	0.01\\
26.62	0.01\\
26.63	0.01\\
26.64	0.01\\
26.65	0.01\\
26.66	0.01\\
26.67	0.01\\
26.68	0.01\\
26.69	0.01\\
26.7	0.01\\
26.71	0.01\\
26.72	0.01\\
26.73	0.01\\
26.74	0.01\\
26.75	0.01\\
26.76	0.01\\
26.77	0.01\\
26.78	0.01\\
26.79	0.01\\
26.8	0.01\\
26.81	0.01\\
26.82	0.01\\
26.83	0.01\\
26.84	0.01\\
26.85	0.01\\
26.86	0.01\\
26.87	0.01\\
26.88	0.01\\
26.89	0.01\\
26.9	0.01\\
26.91	0.01\\
26.92	0.01\\
26.93	0.01\\
26.94	0.01\\
26.95	0.01\\
26.96	0.01\\
26.97	0.01\\
26.98	0.01\\
26.99	0.01\\
27	0.01\\
27.01	0.01\\
27.02	0.01\\
27.03	0.01\\
27.04	0.01\\
27.05	0.01\\
27.06	0.01\\
27.07	0.01\\
27.08	0.01\\
27.09	0.01\\
27.1	0.01\\
27.11	0.01\\
27.12	0.01\\
27.13	0.01\\
27.14	0.01\\
27.15	0.01\\
27.16	0.01\\
27.17	0.01\\
27.18	0.01\\
27.19	0.01\\
27.2	0.01\\
27.21	0.01\\
27.22	0.01\\
27.23	0.01\\
27.24	0.01\\
27.25	0.01\\
27.26	0.01\\
27.27	0.01\\
27.28	0.01\\
27.29	0.01\\
27.3	0.01\\
27.31	0.01\\
27.32	0.01\\
27.33	0.01\\
27.34	0.01\\
27.35	0.01\\
27.36	0.01\\
27.37	0.01\\
27.38	0.01\\
27.39	0.01\\
27.4	0.01\\
27.41	0.01\\
27.42	0.01\\
27.43	0.01\\
27.44	0.01\\
27.45	0.01\\
27.46	0.01\\
27.47	0.01\\
27.48	0.01\\
27.49	0.01\\
27.5	0.01\\
27.51	0.01\\
27.52	0.01\\
27.53	0.01\\
27.54	0.01\\
27.55	0.01\\
27.56	0.01\\
27.57	0.01\\
27.58	0.01\\
27.59	0.01\\
27.6	0.01\\
27.61	0.01\\
27.62	0.01\\
27.63	0.01\\
27.64	0.01\\
27.65	0.01\\
27.66	0.01\\
27.67	0.01\\
27.68	0.01\\
27.69	0.01\\
27.7	0.01\\
27.71	0.01\\
27.72	0.01\\
27.73	0.01\\
27.74	0.01\\
27.75	0.01\\
27.76	0.01\\
27.77	0.01\\
27.78	0.01\\
27.79	0.01\\
27.8	0.01\\
27.81	0.01\\
27.82	0.01\\
27.83	0.01\\
27.84	0.01\\
27.85	0.01\\
27.86	0.01\\
27.87	0.01\\
27.88	0.01\\
27.89	0.01\\
27.9	0.01\\
27.91	0.01\\
27.92	0.01\\
27.93	0.01\\
27.94	0.01\\
27.95	0.01\\
27.96	0.01\\
27.97	0.01\\
27.98	0.01\\
27.99	0.01\\
28	0.01\\
28.01	0.01\\
28.02	0.01\\
28.03	0.01\\
28.04	0.01\\
28.05	0.01\\
28.06	0.01\\
28.07	0.01\\
28.08	0.01\\
28.09	0.01\\
28.1	0.01\\
28.11	0.01\\
28.12	0.01\\
28.13	0.01\\
28.14	0.01\\
28.15	0.01\\
28.16	0.01\\
28.17	0.01\\
28.18	0.01\\
28.19	0.01\\
28.2	0.01\\
28.21	0.01\\
28.22	0.01\\
28.23	0.01\\
28.24	0.01\\
28.25	0.01\\
28.26	0.01\\
28.27	0.01\\
28.28	0.01\\
28.29	0.01\\
28.3	0.01\\
28.31	0.01\\
28.32	0.01\\
28.33	0.01\\
28.34	0.01\\
28.35	0.01\\
28.36	0.01\\
28.37	0.01\\
28.38	0.01\\
28.39	0.01\\
28.4	0.01\\
28.41	0.01\\
28.42	0.01\\
28.43	0.01\\
28.44	0.01\\
28.45	0.01\\
28.46	0.01\\
28.47	0.01\\
28.48	0.01\\
28.49	0.01\\
28.5	0.01\\
28.51	0.01\\
28.52	0.01\\
28.53	0.01\\
28.54	0.01\\
28.55	0.01\\
28.56	0.01\\
28.57	0.01\\
28.58	0.01\\
28.59	0.01\\
28.6	0.01\\
28.61	0.01\\
28.62	0.01\\
28.63	0.01\\
28.64	0.01\\
28.65	0.01\\
28.66	0.01\\
28.67	0.01\\
28.68	0.01\\
28.69	0.01\\
28.7	0.01\\
28.71	0.01\\
28.72	0.01\\
28.73	0.01\\
28.74	0.01\\
28.75	0.01\\
28.76	0.01\\
28.77	0.01\\
28.78	0.01\\
28.79	0.01\\
28.8	0.01\\
28.81	0.01\\
28.82	0.01\\
28.83	0.01\\
28.84	0.01\\
28.85	0.01\\
28.86	0.01\\
28.87	0.01\\
28.88	0.01\\
28.89	0.01\\
28.9	0.01\\
28.91	0.01\\
28.92	0.01\\
28.93	0.01\\
28.94	0.01\\
28.95	0.01\\
28.96	0.01\\
28.97	0.01\\
28.98	0.01\\
28.99	0.01\\
29	0.01\\
29.01	0.01\\
29.02	0.01\\
29.03	0.01\\
29.04	0.01\\
29.05	0.01\\
29.06	0.01\\
29.07	0.01\\
29.08	0.01\\
29.09	0.01\\
29.1	0.01\\
29.11	0.01\\
29.12	0.01\\
29.13	0.01\\
29.14	0.01\\
29.15	0.01\\
29.16	0.01\\
29.17	0.01\\
29.18	0.01\\
29.19	0.01\\
29.2	0.01\\
29.21	0.01\\
29.22	0.01\\
29.23	0.01\\
29.24	0.01\\
29.25	0.01\\
29.26	0.01\\
29.27	0.01\\
29.28	0.01\\
29.29	0.01\\
29.3	0.01\\
29.31	0.01\\
29.32	0.01\\
29.33	0.01\\
29.34	0.01\\
29.35	0.01\\
29.36	0.01\\
29.37	0.01\\
29.38	0.01\\
29.39	0.01\\
29.4	0.01\\
29.41	0.01\\
29.42	0.01\\
29.43	0.01\\
29.44	0.01\\
29.45	0.01\\
29.46	0.01\\
29.47	0.01\\
29.48	0.01\\
29.49	0.01\\
29.5	0.01\\
29.51	0.01\\
29.52	0.01\\
29.53	0.01\\
29.54	0.01\\
29.55	0.01\\
29.56	0.01\\
29.57	0.01\\
29.58	0.01\\
29.59	0.01\\
29.6	0.01\\
29.61	0.01\\
29.62	0.01\\
29.63	0.01\\
29.64	0.01\\
29.65	0.01\\
29.66	0.01\\
29.67	0.01\\
29.68	0.01\\
29.69	0.01\\
29.7	0.01\\
29.71	0.01\\
29.72	0.01\\
29.73	0.01\\
29.74	0.01\\
29.75	0.01\\
29.76	0.01\\
29.77	0.01\\
29.78	0.01\\
29.79	0.01\\
29.8	0.01\\
29.81	0.01\\
29.82	0.01\\
29.83	0.01\\
29.84	0.01\\
29.85	0.01\\
29.86	0.01\\
29.87	0.01\\
29.88	0.01\\
29.89	0.01\\
29.9	0.01\\
29.91	0.01\\
29.92	0.01\\
29.93	0.01\\
29.94	0.01\\
29.95	0.01\\
29.96	0.01\\
29.97	0.01\\
29.98	0.01\\
29.99	0.01\\
30	0.01\\
30.01	0.01\\
30.02	0.01\\
30.03	0.01\\
30.04	0.01\\
30.05	0.01\\
30.06	0.01\\
30.07	0.01\\
30.08	0.01\\
30.09	0.01\\
30.1	0.01\\
30.11	0.01\\
30.12	0.01\\
30.13	0.01\\
30.14	0.01\\
30.15	0.01\\
30.16	0.01\\
30.17	0.01\\
30.18	0.01\\
30.19	0.01\\
30.2	0.01\\
30.21	0.01\\
30.22	0.01\\
30.23	0.01\\
30.24	0.01\\
30.25	0.01\\
30.26	0.01\\
30.27	0.01\\
30.28	0.01\\
30.29	0.01\\
30.3	0.01\\
30.31	0.01\\
30.32	0.01\\
30.33	0.01\\
30.34	0.01\\
30.35	0.01\\
30.36	0.01\\
30.37	0.01\\
30.38	0.01\\
30.39	0.01\\
30.4	0.01\\
30.41	0.01\\
30.42	0.01\\
30.43	0.01\\
30.44	0.01\\
30.45	0.01\\
30.46	0.01\\
30.47	0.01\\
30.48	0.01\\
30.49	0.01\\
30.5	0.01\\
30.51	0.01\\
30.52	0.01\\
30.53	0.01\\
30.54	0.01\\
30.55	0.01\\
30.56	0.01\\
30.57	0.01\\
30.58	0.01\\
30.59	0.01\\
30.6	0.01\\
30.61	0.01\\
30.62	0.01\\
30.63	0.01\\
30.64	0.01\\
30.65	0.01\\
30.66	0.01\\
30.67	0.01\\
30.68	0.01\\
30.69	0.01\\
30.7	0.01\\
30.71	0.01\\
30.72	0.01\\
30.73	0.01\\
30.74	0.01\\
30.75	0.01\\
30.76	0.01\\
30.77	0.01\\
30.78	0.01\\
30.79	0.01\\
30.8	0.01\\
30.81	0.01\\
30.82	0.01\\
30.83	0.01\\
30.84	0.01\\
30.85	0.01\\
30.86	0.01\\
30.87	0.01\\
30.88	0.01\\
30.89	0.01\\
30.9	0.01\\
30.91	0.01\\
30.92	0.01\\
30.93	0.01\\
30.94	0.01\\
30.95	0.01\\
30.96	0.01\\
30.97	0.01\\
30.98	0.01\\
30.99	0.01\\
31	0.01\\
31.01	0.01\\
31.02	0.01\\
31.03	0.01\\
31.04	0.01\\
31.05	0.01\\
31.06	0.01\\
31.07	0.01\\
31.08	0.01\\
31.09	0.01\\
31.1	0.01\\
31.11	0.01\\
31.12	0.01\\
31.13	0.01\\
31.14	0.01\\
31.15	0.01\\
31.16	0.01\\
31.17	0.01\\
31.18	0.01\\
31.19	0.01\\
31.2	0.01\\
31.21	0.01\\
31.22	0.01\\
31.23	0.01\\
31.24	0.01\\
31.25	0.01\\
31.26	0.01\\
31.27	0.01\\
31.28	0.01\\
31.29	0.01\\
31.3	0.01\\
31.31	0.01\\
31.32	0.01\\
31.33	0.01\\
31.34	0.01\\
31.35	0.01\\
31.36	0.01\\
31.37	0.01\\
31.38	0.01\\
31.39	0.01\\
31.4	0.01\\
31.41	0.01\\
31.42	0.01\\
31.43	0.01\\
31.44	0.01\\
31.45	0.01\\
31.46	0.01\\
31.47	0.01\\
31.48	0.01\\
31.49	0.01\\
31.5	0.01\\
31.51	0.01\\
31.52	0.01\\
31.53	0.01\\
31.54	0.01\\
31.55	0.01\\
31.56	0.01\\
31.57	0.01\\
31.58	0.01\\
31.59	0.01\\
31.6	0.01\\
31.61	0.01\\
31.62	0.01\\
31.63	0.01\\
31.64	0.01\\
31.65	0.01\\
31.66	0.01\\
31.67	0.01\\
31.68	0.01\\
31.69	0.01\\
31.7	0.01\\
31.71	0.01\\
31.72	0.01\\
31.73	0.01\\
31.74	0.01\\
31.75	0.01\\
31.76	0.01\\
31.77	0.01\\
31.78	0.01\\
31.79	0.01\\
31.8	0.01\\
31.81	0.01\\
31.82	0.01\\
31.83	0.01\\
31.84	0.01\\
31.85	0.01\\
31.86	0.01\\
31.87	0.01\\
31.88	0.01\\
31.89	0.01\\
31.9	0.01\\
31.91	0.01\\
31.92	0.01\\
31.93	0.01\\
31.94	0.01\\
31.95	0.01\\
31.96	0.01\\
31.97	0.01\\
31.98	0.01\\
31.99	0.01\\
32	0.01\\
32.01	0.01\\
32.02	0.01\\
32.03	0.01\\
32.04	0.01\\
32.05	0.01\\
32.06	0.01\\
32.07	0.01\\
32.08	0.01\\
32.09	0.01\\
32.1	0.01\\
32.11	0.01\\
32.12	0.01\\
32.13	0.01\\
32.14	0.01\\
32.15	0.01\\
32.16	0.01\\
32.17	0.01\\
32.18	0.01\\
32.19	0.01\\
32.2	0.01\\
32.21	0.01\\
32.22	0.01\\
32.23	0.01\\
32.24	0.01\\
32.25	0.01\\
32.26	0.01\\
32.27	0.01\\
32.28	0.01\\
32.29	0.01\\
32.3	0.01\\
32.31	0.01\\
32.32	0.01\\
32.33	0.01\\
32.34	0.01\\
32.35	0.01\\
32.36	0.01\\
32.37	0.01\\
32.38	0.01\\
32.39	0.01\\
32.4	0.01\\
32.41	0.01\\
32.42	0.01\\
32.43	0.01\\
32.44	0.01\\
32.45	0.01\\
32.46	0.01\\
32.47	0.01\\
32.48	0.01\\
32.49	0.01\\
32.5	0.01\\
32.51	0.01\\
32.52	0.01\\
32.53	0.01\\
32.54	0.01\\
32.55	0.01\\
32.56	0.01\\
32.57	0.01\\
32.58	0.01\\
32.59	0.01\\
32.6	0.01\\
32.61	0.01\\
32.62	0.01\\
32.63	0.01\\
32.64	0.01\\
32.65	0.01\\
32.66	0.01\\
32.67	0.01\\
32.68	0.01\\
32.69	0.01\\
32.7	0.01\\
32.71	0.01\\
32.72	0.01\\
32.73	0.01\\
32.74	0.01\\
32.75	0.01\\
32.76	0.01\\
32.77	0.01\\
32.78	0.01\\
32.79	0.01\\
32.8	0.01\\
32.81	0.01\\
32.82	0.01\\
32.83	0.01\\
32.84	0.01\\
32.85	0.01\\
32.86	0.01\\
32.87	0.01\\
32.88	0.01\\
32.89	0.01\\
32.9	0.01\\
32.91	0.01\\
32.92	0.01\\
32.93	0.01\\
32.94	0.01\\
32.95	0.01\\
32.96	0.01\\
32.97	0.01\\
32.98	0.01\\
32.99	0.01\\
33	0.01\\
33.01	0.01\\
33.02	0.01\\
33.03	0.01\\
33.04	0.01\\
33.05	0.01\\
33.06	0.01\\
33.07	0.01\\
33.08	0.01\\
33.09	0.01\\
33.1	0.01\\
33.11	0.01\\
33.12	0.01\\
33.13	0.01\\
33.14	0.01\\
33.15	0.01\\
33.16	0.01\\
33.17	0.01\\
33.18	0.01\\
33.19	0.01\\
33.2	0.01\\
33.21	0.01\\
33.22	0.01\\
33.23	0.01\\
33.24	0.01\\
33.25	0.01\\
33.26	0.01\\
33.27	0.01\\
33.28	0.01\\
33.29	0.01\\
33.3	0.01\\
33.31	0.01\\
33.32	0.01\\
33.33	0.01\\
33.34	0.01\\
33.35	0.01\\
33.36	0.01\\
33.37	0.01\\
33.38	0.01\\
33.39	0.01\\
33.4	0.01\\
33.41	0.01\\
33.42	0.01\\
33.43	0.01\\
33.44	0.01\\
33.45	0.01\\
33.46	0.01\\
33.47	0.01\\
33.48	0.01\\
33.49	0.01\\
33.5	0.01\\
33.51	0.01\\
33.52	0.01\\
33.53	0.01\\
33.54	0.01\\
33.55	0.01\\
33.56	0.01\\
33.57	0.01\\
33.58	0.01\\
33.59	0.01\\
33.6	0.01\\
33.61	0.01\\
33.62	0.01\\
33.63	0.01\\
33.64	0.01\\
33.65	0.01\\
33.66	0.01\\
33.67	0.01\\
33.68	0.01\\
33.69	0.01\\
33.7	0.01\\
33.71	0.01\\
33.72	0.01\\
33.73	0.01\\
33.74	0.01\\
33.75	0.01\\
33.76	0.01\\
33.77	0.01\\
33.78	0.01\\
33.79	0.01\\
33.8	0.01\\
33.81	0.01\\
33.82	0.01\\
33.83	0.01\\
33.84	0.01\\
33.85	0.01\\
33.86	0.01\\
33.87	0.01\\
33.88	0.01\\
33.89	0.01\\
33.9	0.01\\
33.91	0.01\\
33.92	0.01\\
33.93	0.01\\
33.94	0.01\\
33.95	0.01\\
33.96	0.01\\
33.97	0.01\\
33.98	0.01\\
33.99	0.01\\
34	0.01\\
34.01	0.01\\
34.02	0.01\\
34.03	0.01\\
34.04	0.01\\
34.05	0.01\\
34.06	0.01\\
34.07	0.01\\
34.08	0.01\\
34.09	0.01\\
34.1	0.01\\
34.11	0.01\\
34.12	0.01\\
34.13	0.01\\
34.14	0.01\\
34.15	0.01\\
34.16	0.01\\
34.17	0.01\\
34.18	0.01\\
34.19	0.01\\
34.2	0.01\\
34.21	0.01\\
34.22	0.01\\
34.23	0.01\\
34.24	0.01\\
34.25	0.01\\
34.26	0.01\\
34.27	0.01\\
34.28	0.01\\
34.29	0.01\\
34.3	0.01\\
34.31	0.01\\
34.32	0.01\\
34.33	0.01\\
34.34	0.01\\
34.35	0.01\\
34.36	0.01\\
34.37	0.01\\
34.38	0.01\\
34.39	0.01\\
34.4	0.01\\
34.41	0.01\\
34.42	0.01\\
34.43	0.01\\
34.44	0.01\\
34.45	0.01\\
34.46	0.01\\
34.47	0.01\\
34.48	0.01\\
34.49	0.01\\
34.5	0.01\\
34.51	0.01\\
34.52	0.01\\
34.53	0.01\\
34.54	0.01\\
34.55	0.01\\
34.56	0.01\\
34.57	0.01\\
34.58	0.01\\
34.59	0.01\\
34.6	0.01\\
34.61	0.01\\
34.62	0.01\\
34.63	0.01\\
34.64	0.01\\
34.65	0.01\\
34.66	0.01\\
34.67	0.01\\
34.68	0.01\\
34.69	0.01\\
34.7	0.01\\
34.71	0.01\\
34.72	0.01\\
34.73	0.01\\
34.74	0.01\\
34.75	0.01\\
34.76	0.01\\
34.77	0.01\\
34.78	0.01\\
34.79	0.01\\
34.8	0.01\\
34.81	0.01\\
34.82	0.01\\
34.83	0.01\\
34.84	0.01\\
34.85	0.01\\
34.86	0.01\\
34.87	0.01\\
34.88	0.01\\
34.89	0.01\\
34.9	0.01\\
34.91	0.01\\
34.92	0.01\\
34.93	0.01\\
34.94	0.01\\
34.95	0.01\\
34.96	0.01\\
34.97	0.01\\
34.98	0.01\\
34.99	0.01\\
35	0.01\\
35.01	0.01\\
35.02	0.01\\
35.03	0.01\\
35.04	0.01\\
35.05	0.01\\
35.06	0.01\\
35.07	0.01\\
35.08	0.01\\
35.09	0.01\\
35.1	0.01\\
35.11	0.01\\
35.12	0.01\\
35.13	0.01\\
35.14	0.01\\
35.15	0.01\\
35.16	0.01\\
35.17	0.01\\
35.18	0.01\\
35.19	0.01\\
35.2	0.01\\
35.21	0.01\\
35.22	0.01\\
35.23	0.01\\
35.24	0.01\\
35.25	0.01\\
35.26	0.01\\
35.27	0.01\\
35.28	0.01\\
35.29	0.01\\
35.3	0.01\\
35.31	0.01\\
35.32	0.01\\
35.33	0.01\\
35.34	0.01\\
35.35	0.01\\
35.36	0.01\\
35.37	0.01\\
35.38	0.01\\
35.39	0.01\\
35.4	0.01\\
35.41	0.01\\
35.42	0.01\\
35.43	0.01\\
35.44	0.01\\
35.45	0.01\\
35.46	0.01\\
35.47	0.01\\
35.48	0.01\\
35.49	0.01\\
35.5	0.01\\
35.51	0.01\\
35.52	0.01\\
35.53	0.01\\
35.54	0.01\\
35.55	0.01\\
35.56	0.01\\
35.57	0.01\\
35.58	0.01\\
35.59	0.01\\
35.6	0.01\\
35.61	0.01\\
35.62	0.01\\
35.63	0.01\\
35.64	0.01\\
35.65	0.01\\
35.66	0.01\\
35.67	0.01\\
35.68	0.01\\
35.69	0.01\\
35.7	0.01\\
35.71	0.01\\
35.72	0.01\\
35.73	0.01\\
35.74	0.01\\
35.75	0.01\\
35.76	0.01\\
35.77	0.01\\
35.78	0.01\\
35.79	0.01\\
35.8	0.01\\
35.81	0.01\\
35.82	0.01\\
35.83	0.01\\
35.84	0.01\\
35.85	0.01\\
35.86	0.01\\
35.87	0.01\\
35.88	0.01\\
35.89	0.01\\
35.9	0.01\\
35.91	0.01\\
35.92	0.01\\
35.93	0.01\\
35.94	0.01\\
35.95	0.01\\
35.96	0.01\\
35.97	0.01\\
35.98	0.01\\
35.99	0.01\\
36	0.01\\
36.01	0.01\\
36.02	0.01\\
36.03	0.01\\
36.04	0.01\\
36.05	0.01\\
36.06	0.01\\
36.07	0.01\\
36.08	0.01\\
36.09	0.01\\
36.1	0.01\\
36.11	0.01\\
36.12	0.01\\
36.13	0.01\\
36.14	0.01\\
36.15	0.01\\
36.16	0.01\\
36.17	0.01\\
36.18	0.01\\
36.19	0.01\\
36.2	0.01\\
36.21	0.01\\
36.22	0.01\\
36.23	0.01\\
36.24	0.01\\
36.25	0.01\\
36.26	0.01\\
36.27	0.01\\
36.28	0.01\\
36.29	0.01\\
36.3	0.01\\
36.31	0.01\\
36.32	0.01\\
36.33	0.01\\
36.34	0.01\\
36.35	0.01\\
36.36	0.01\\
36.37	0.01\\
36.38	0.01\\
36.39	0.01\\
36.4	0.01\\
36.41	0.01\\
36.42	0.01\\
36.43	0.01\\
36.44	0.01\\
36.45	0.01\\
36.46	0.01\\
36.47	0.01\\
36.48	0.01\\
36.49	0.01\\
36.5	0.01\\
36.51	0.01\\
36.52	0.01\\
36.53	0.01\\
36.54	0.01\\
36.55	0.01\\
36.56	0.01\\
36.57	0.01\\
36.58	0.01\\
36.59	0.01\\
36.6	0.01\\
36.61	0.01\\
36.62	0.01\\
36.63	0.01\\
36.64	0.01\\
36.65	0.01\\
36.66	0.01\\
36.67	0.01\\
36.68	0.01\\
36.69	0.01\\
36.7	0.01\\
36.71	0.01\\
36.72	0.01\\
36.73	0.01\\
36.74	0.01\\
36.75	0.01\\
36.76	0.01\\
36.77	0.01\\
36.78	0.01\\
36.79	0.01\\
36.8	0.01\\
36.81	0.01\\
36.82	0.01\\
36.83	0.01\\
36.84	0.01\\
36.85	0.01\\
36.86	0.01\\
36.87	0.01\\
36.88	0.01\\
36.89	0.01\\
36.9	0.01\\
36.91	0.01\\
36.92	0.01\\
36.93	0.01\\
36.94	0.01\\
36.95	0.01\\
36.96	0.01\\
36.97	0.01\\
36.98	0.01\\
36.99	0.01\\
37	0.01\\
37.01	0.01\\
37.02	0.01\\
37.03	0.01\\
37.04	0.01\\
37.05	0.01\\
37.06	0.01\\
37.07	0.01\\
37.08	0.01\\
37.09	0.01\\
37.1	0.01\\
37.11	0.01\\
37.12	0.01\\
37.13	0.01\\
37.14	0.01\\
37.15	0.01\\
37.16	0.01\\
37.17	0.01\\
37.18	0.01\\
37.19	0.01\\
37.2	0.01\\
37.21	0.01\\
37.22	0.01\\
37.23	0.01\\
37.24	0.01\\
37.25	0.01\\
37.26	0.01\\
37.27	0.01\\
37.28	0.01\\
37.29	0.01\\
37.3	0.01\\
37.31	0.01\\
37.32	0.01\\
37.33	0.01\\
37.34	0.01\\
37.35	0.01\\
37.36	0.01\\
37.37	0.01\\
37.38	0.01\\
37.39	0.01\\
37.4	0.01\\
37.41	0.01\\
37.42	0.01\\
37.43	0.01\\
37.44	0.01\\
37.45	0.01\\
37.46	0.01\\
37.47	0.01\\
37.48	0.01\\
37.49	0.01\\
37.5	0.01\\
37.51	0.01\\
37.52	0.01\\
37.53	0.01\\
37.54	0.01\\
37.55	0.01\\
37.56	0.01\\
37.57	0.01\\
37.58	0.01\\
37.59	0.01\\
37.6	0.01\\
37.61	0.01\\
37.62	0.01\\
37.63	0.01\\
37.64	0.01\\
37.65	0.01\\
37.66	0.01\\
37.67	0.01\\
37.68	0.01\\
37.69	0.01\\
37.7	0.01\\
37.71	0.01\\
37.72	0.01\\
37.73	0.01\\
37.74	0.01\\
37.75	0.01\\
37.76	0.01\\
37.77	0.01\\
37.78	0.01\\
37.79	0.01\\
37.8	0.01\\
37.81	0.01\\
37.82	0.01\\
37.83	0.01\\
37.84	0.01\\
37.85	0.01\\
37.86	0.01\\
37.87	0.01\\
37.88	0.01\\
37.89	0.01\\
37.9	0.01\\
37.91	0.01\\
37.92	0.01\\
37.93	0.01\\
37.94	0.01\\
37.95	0.01\\
37.96	0.01\\
37.97	0.01\\
37.98	0.01\\
37.99	0.01\\
38	0.01\\
38.01	0.01\\
38.02	0.01\\
38.03	0.01\\
38.04	0.01\\
38.05	0.01\\
38.06	0.01\\
38.07	0.01\\
38.08	0.01\\
38.09	0.01\\
38.1	0.01\\
38.11	0.01\\
38.12	0.01\\
38.13	0.01\\
38.14	0.01\\
38.15	0.01\\
38.16	0.01\\
38.17	0.01\\
38.18	0.01\\
38.19	0.01\\
38.2	0.01\\
38.21	0.01\\
38.22	0.01\\
38.23	0.01\\
38.24	0.01\\
38.25	0.01\\
38.26	0.01\\
38.27	0.01\\
38.28	0.01\\
38.29	0.01\\
38.3	0.01\\
38.31	0.01\\
38.32	0.01\\
38.33	0.01\\
38.34	0.01\\
38.35	0.01\\
38.36	0.01\\
38.37	0.01\\
38.38	0.01\\
38.39	0.01\\
38.4	0.01\\
38.41	0.01\\
38.42	0.01\\
38.43	0.01\\
38.44	0.01\\
38.45	0.01\\
38.46	0.01\\
38.47	0.01\\
38.48	0.01\\
38.49	0.01\\
38.5	0.01\\
38.51	0.01\\
38.52	0.01\\
38.53	0.01\\
38.54	0.01\\
38.55	0.01\\
38.56	0.01\\
38.57	0.01\\
38.58	0.01\\
38.59	0.01\\
38.6	0.01\\
38.61	0.01\\
38.62	0.01\\
38.63	0.01\\
38.64	0.01\\
38.65	0.01\\
38.66	0.01\\
38.67	0.01\\
38.68	0.01\\
38.69	0.01\\
38.7	0.01\\
38.71	0.01\\
38.72	0.01\\
38.73	0.01\\
38.74	0.01\\
38.75	0.01\\
38.76	0.01\\
38.77	0.01\\
38.78	0.01\\
38.79	0.01\\
38.8	0.01\\
38.81	0.01\\
38.82	0.01\\
38.83	0.01\\
38.84	0.01\\
38.85	0.01\\
38.86	0.01\\
38.87	0.01\\
38.88	0.01\\
38.89	0.01\\
38.9	0.01\\
38.91	0.01\\
38.92	0.01\\
38.93	0.01\\
38.94	0.01\\
38.95	0.01\\
38.96	0.01\\
38.97	0.01\\
38.98	0.01\\
38.99	0.01\\
39	0.01\\
39.01	0.01\\
39.02	0.01\\
39.03	0.01\\
39.04	0.01\\
39.05	0.01\\
39.06	0.01\\
39.07	0.01\\
39.08	0.01\\
39.09	0.01\\
39.1	0.01\\
39.11	0.01\\
39.12	0.01\\
39.13	0.01\\
39.14	0.01\\
39.15	0.01\\
39.16	0.01\\
39.17	0.01\\
39.18	0.01\\
39.19	0.01\\
39.2	0.01\\
39.21	0.01\\
39.22	0.01\\
39.23	0.01\\
39.24	0.01\\
39.25	0.01\\
39.26	0.01\\
39.27	0.01\\
39.28	0.01\\
39.29	0.01\\
39.3	0.01\\
39.31	0.01\\
39.32	0.01\\
39.33	0.01\\
39.34	0.01\\
39.35	0.01\\
39.36	0.01\\
39.37	0.01\\
39.38	0.01\\
39.39	0.01\\
39.4	0.01\\
39.41	0.01\\
39.42	0.01\\
39.43	0.01\\
39.44	0.01\\
39.45	0.01\\
39.46	0.01\\
39.47	0.01\\
39.48	0.01\\
39.49	0.01\\
39.5	0.01\\
39.51	0.01\\
39.52	0.01\\
39.53	0.01\\
39.54	0.01\\
39.55	0.01\\
39.56	0.01\\
39.57	0.01\\
39.58	0.01\\
39.59	0.01\\
39.6	0.01\\
39.61	0.01\\
39.62	0.01\\
39.63	0.01\\
39.64	0.01\\
39.65	0.01\\
39.66	0.01\\
39.67	0.01\\
39.68	0.01\\
39.69	0.01\\
39.7	0.01\\
39.71	0.01\\
39.72	0.01\\
39.73	0.01\\
39.74	0.01\\
39.75	0.01\\
39.76	0.01\\
39.77	0.01\\
39.78	0.01\\
39.79	0.01\\
39.8	0.01\\
39.81	0.01\\
39.82	0.01\\
39.83	0.01\\
39.84	0.01\\
39.85	0.01\\
39.86	0.01\\
39.87	0.01\\
39.88	0.01\\
39.89	0.01\\
39.9	0.01\\
39.91	0.01\\
39.92	0.01\\
39.93	0.01\\
39.94	0.01\\
39.95	0.01\\
39.96	0.01\\
39.97	0.01\\
39.98	0.01\\
39.99	0.01\\
40	0.01\\
40.01	0.01\\
};
\addplot [color=green,solid,forget plot]
  table[row sep=crcr]{%
40.01	0.01\\
40.02	0.01\\
40.03	0.01\\
40.04	0.01\\
40.05	0.01\\
40.06	0.01\\
40.07	0.01\\
40.08	0.01\\
40.09	0.01\\
40.1	0.01\\
40.11	0.01\\
40.12	0.01\\
40.13	0.01\\
40.14	0.01\\
40.15	0.01\\
40.16	0.01\\
40.17	0.01\\
40.18	0.01\\
40.19	0.01\\
40.2	0.01\\
40.21	0.01\\
40.22	0.01\\
40.23	0.01\\
40.24	0.01\\
40.25	0.01\\
40.26	0.01\\
40.27	0.01\\
40.28	0.01\\
40.29	0.01\\
40.3	0.01\\
40.31	0.01\\
40.32	0.01\\
40.33	0.01\\
40.34	0.01\\
40.35	0.01\\
40.36	0.01\\
40.37	0.01\\
40.38	0.01\\
40.39	0.01\\
40.4	0.01\\
40.41	0.01\\
40.42	0.01\\
40.43	0.01\\
40.44	0.01\\
40.45	0.01\\
40.46	0.01\\
40.47	0.01\\
40.48	0.01\\
40.49	0.01\\
40.5	0.01\\
40.51	0.01\\
40.52	0.01\\
40.53	0.01\\
40.54	0.01\\
40.55	0.01\\
40.56	0.01\\
40.57	0.01\\
40.58	0.01\\
40.59	0.01\\
40.6	0.01\\
40.61	0.01\\
40.62	0.01\\
40.63	0.01\\
40.64	0.01\\
40.65	0.01\\
40.66	0.01\\
40.67	0.01\\
40.68	0.01\\
40.69	0.01\\
40.7	0.01\\
40.71	0.01\\
40.72	0.01\\
40.73	0.01\\
40.74	0.01\\
40.75	0.01\\
40.76	0.01\\
40.77	0.01\\
40.78	0.01\\
40.79	0.01\\
40.8	0.01\\
40.81	0.01\\
40.82	0.01\\
40.83	0.01\\
40.84	0.01\\
40.85	0.01\\
40.86	0.01\\
40.87	0.01\\
40.88	0.01\\
40.89	0.01\\
40.9	0.01\\
40.91	0.01\\
40.92	0.01\\
40.93	0.01\\
40.94	0.01\\
40.95	0.01\\
40.96	0.01\\
40.97	0.01\\
40.98	0.01\\
40.99	0.01\\
41	0.01\\
41.01	0.01\\
41.02	0.01\\
41.03	0.01\\
41.04	0.01\\
41.05	0.01\\
41.06	0.01\\
41.07	0.01\\
41.08	0.01\\
41.09	0.01\\
41.1	0.01\\
41.11	0.01\\
41.12	0.01\\
41.13	0.01\\
41.14	0.01\\
41.15	0.01\\
41.16	0.01\\
41.17	0.01\\
41.18	0.01\\
41.19	0.01\\
41.2	0.01\\
41.21	0.01\\
41.22	0.01\\
41.23	0.01\\
41.24	0.01\\
41.25	0.01\\
41.26	0.01\\
41.27	0.01\\
41.28	0.01\\
41.29	0.01\\
41.3	0.01\\
41.31	0.01\\
41.32	0.01\\
41.33	0.01\\
41.34	0.01\\
41.35	0.01\\
41.36	0.01\\
41.37	0.01\\
41.38	0.01\\
41.39	0.01\\
41.4	0.01\\
41.41	0.01\\
41.42	0.01\\
41.43	0.01\\
41.44	0.01\\
41.45	0.01\\
41.46	0.01\\
41.47	0.01\\
41.48	0.01\\
41.49	0.01\\
41.5	0.01\\
41.51	0.01\\
41.52	0.01\\
41.53	0.01\\
41.54	0.01\\
41.55	0.01\\
41.56	0.01\\
41.57	0.01\\
41.58	0.01\\
41.59	0.01\\
41.6	0.01\\
41.61	0.01\\
41.62	0.01\\
41.63	0.01\\
41.64	0.01\\
41.65	0.01\\
41.66	0.01\\
41.67	0.01\\
41.68	0.01\\
41.69	0.01\\
41.7	0.01\\
41.71	0.01\\
41.72	0.01\\
41.73	0.01\\
41.74	0.01\\
41.75	0.01\\
41.76	0.01\\
41.77	0.01\\
41.78	0.01\\
41.79	0.01\\
41.8	0.01\\
41.81	0.01\\
41.82	0.01\\
41.83	0.01\\
41.84	0.01\\
41.85	0.01\\
41.86	0.01\\
41.87	0.01\\
41.88	0.01\\
41.89	0.01\\
41.9	0.01\\
41.91	0.01\\
41.92	0.01\\
41.93	0.01\\
41.94	0.01\\
41.95	0.01\\
41.96	0.01\\
41.97	0.01\\
41.98	0.01\\
41.99	0.01\\
42	0.01\\
42.01	0.01\\
42.02	0.01\\
42.03	0.01\\
42.04	0.01\\
42.05	0.01\\
42.06	0.01\\
42.07	0.01\\
42.08	0.01\\
42.09	0.01\\
42.1	0.01\\
42.11	0.01\\
42.12	0.01\\
42.13	0.01\\
42.14	0.01\\
42.15	0.01\\
42.16	0.01\\
42.17	0.01\\
42.18	0.01\\
42.19	0.01\\
42.2	0.01\\
42.21	0.01\\
42.22	0.01\\
42.23	0.01\\
42.24	0.01\\
42.25	0.01\\
42.26	0.01\\
42.27	0.01\\
42.28	0.01\\
42.29	0.01\\
42.3	0.01\\
42.31	0.01\\
42.32	0.01\\
42.33	0.01\\
42.34	0.01\\
42.35	0.01\\
42.36	0.01\\
42.37	0.01\\
42.38	0.01\\
42.39	0.01\\
42.4	0.01\\
42.41	0.01\\
42.42	0.01\\
42.43	0.01\\
42.44	0.01\\
42.45	0.01\\
42.46	0.01\\
42.47	0.01\\
42.48	0.01\\
42.49	0.01\\
42.5	0.01\\
42.51	0.01\\
42.52	0.01\\
42.53	0.01\\
42.54	0.01\\
42.55	0.01\\
42.56	0.01\\
42.57	0.01\\
42.58	0.01\\
42.59	0.01\\
42.6	0.01\\
42.61	0.01\\
42.62	0.01\\
42.63	0.01\\
42.64	0.01\\
42.65	0.01\\
42.66	0.01\\
42.67	0.01\\
42.68	0.01\\
42.69	0.01\\
42.7	0.01\\
42.71	0.01\\
42.72	0.01\\
42.73	0.01\\
42.74	0.01\\
42.75	0.01\\
42.76	0.01\\
42.77	0.01\\
42.78	0.01\\
42.79	0.01\\
42.8	0.01\\
42.81	0.01\\
42.82	0.01\\
42.83	0.01\\
42.84	0.01\\
42.85	0.01\\
42.86	0.01\\
42.87	0.01\\
42.88	0.01\\
42.89	0.01\\
42.9	0.01\\
42.91	0.01\\
42.92	0.01\\
42.93	0.01\\
42.94	0.01\\
42.95	0.01\\
42.96	0.01\\
42.97	0.01\\
42.98	0.01\\
42.99	0.01\\
43	0.01\\
43.01	0.01\\
43.02	0.01\\
43.03	0.01\\
43.04	0.01\\
43.05	0.01\\
43.06	0.01\\
43.07	0.01\\
43.08	0.01\\
43.09	0.01\\
43.1	0.01\\
43.11	0.01\\
43.12	0.01\\
43.13	0.01\\
43.14	0.01\\
43.15	0.01\\
43.16	0.01\\
43.17	0.01\\
43.18	0.01\\
43.19	0.01\\
43.2	0.01\\
43.21	0.01\\
43.22	0.01\\
43.23	0.01\\
43.24	0.01\\
43.25	0.01\\
43.26	0.01\\
43.27	0.01\\
43.28	0.01\\
43.29	0.01\\
43.3	0.01\\
43.31	0.01\\
43.32	0.01\\
43.33	0.01\\
43.34	0.01\\
43.35	0.01\\
43.36	0.01\\
43.37	0.01\\
43.38	0.01\\
43.39	0.01\\
43.4	0.01\\
43.41	0.01\\
43.42	0.01\\
43.43	0.01\\
43.44	0.01\\
43.45	0.01\\
43.46	0.01\\
43.47	0.01\\
43.48	0.01\\
43.49	0.01\\
43.5	0.01\\
43.51	0.01\\
43.52	0.01\\
43.53	0.01\\
43.54	0.01\\
43.55	0.01\\
43.56	0.01\\
43.57	0.01\\
43.58	0.01\\
43.59	0.01\\
43.6	0.01\\
43.61	0.01\\
43.62	0.01\\
43.63	0.01\\
43.64	0.01\\
43.65	0.01\\
43.66	0.01\\
43.67	0.01\\
43.68	0.01\\
43.69	0.01\\
43.7	0.01\\
43.71	0.01\\
43.72	0.01\\
43.73	0.01\\
43.74	0.01\\
43.75	0.01\\
43.76	0.01\\
43.77	0.01\\
43.78	0.01\\
43.79	0.01\\
43.8	0.01\\
43.81	0.01\\
43.82	0.01\\
43.83	0.01\\
43.84	0.01\\
43.85	0.01\\
43.86	0.01\\
43.87	0.01\\
43.88	0.01\\
43.89	0.01\\
43.9	0.01\\
43.91	0.01\\
43.92	0.01\\
43.93	0.01\\
43.94	0.01\\
43.95	0.01\\
43.96	0.01\\
43.97	0.01\\
43.98	0.01\\
43.99	0.01\\
44	0.01\\
44.01	0.01\\
44.02	0.01\\
44.03	0.01\\
44.04	0.01\\
44.05	0.01\\
44.06	0.01\\
44.07	0.01\\
44.08	0.01\\
44.09	0.01\\
44.1	0.01\\
44.11	0.01\\
44.12	0.01\\
44.13	0.01\\
44.14	0.01\\
44.15	0.01\\
44.16	0.01\\
44.17	0.01\\
44.18	0.01\\
44.19	0.01\\
44.2	0.01\\
44.21	0.01\\
44.22	0.01\\
44.23	0.01\\
44.24	0.01\\
44.25	0.01\\
44.26	0.01\\
44.27	0.01\\
44.28	0.01\\
44.29	0.01\\
44.3	0.01\\
44.31	0.01\\
44.32	0.01\\
44.33	0.01\\
44.34	0.01\\
44.35	0.01\\
44.36	0.01\\
44.37	0.01\\
44.38	0.01\\
44.39	0.01\\
44.4	0.01\\
44.41	0.01\\
44.42	0.01\\
44.43	0.01\\
44.44	0.01\\
44.45	0.01\\
44.46	0.01\\
44.47	0.01\\
44.48	0.01\\
44.49	0.01\\
44.5	0.01\\
44.51	0.01\\
44.52	0.01\\
44.53	0.01\\
44.54	0.01\\
44.55	0.01\\
44.56	0.01\\
44.57	0.01\\
44.58	0.01\\
44.59	0.01\\
44.6	0.01\\
44.61	0.01\\
44.62	0.01\\
44.63	0.01\\
44.64	0.01\\
44.65	0.01\\
44.66	0.01\\
44.67	0.01\\
44.68	0.01\\
44.69	0.01\\
44.7	0.01\\
44.71	0.01\\
44.72	0.01\\
44.73	0.01\\
44.74	0.01\\
44.75	0.01\\
44.76	0.01\\
44.77	0.01\\
44.78	0.01\\
44.79	0.01\\
44.8	0.01\\
44.81	0.01\\
44.82	0.01\\
44.83	0.01\\
44.84	0.01\\
44.85	0.01\\
44.86	0.01\\
44.87	0.01\\
44.88	0.01\\
44.89	0.01\\
44.9	0.01\\
44.91	0.01\\
44.92	0.01\\
44.93	0.01\\
44.94	0.01\\
44.95	0.01\\
44.96	0.01\\
44.97	0.01\\
44.98	0.01\\
44.99	0.01\\
45	0.01\\
45.01	0.01\\
45.02	0.01\\
45.03	0.01\\
45.04	0.01\\
45.05	0.01\\
45.06	0.01\\
45.07	0.01\\
45.08	0.01\\
45.09	0.01\\
45.1	0.01\\
45.11	0.01\\
45.12	0.01\\
45.13	0.01\\
45.14	0.01\\
45.15	0.01\\
45.16	0.01\\
45.17	0.01\\
45.18	0.01\\
45.19	0.01\\
45.2	0.01\\
45.21	0.01\\
45.22	0.01\\
45.23	0.01\\
45.24	0.01\\
45.25	0.01\\
45.26	0.01\\
45.27	0.01\\
45.28	0.01\\
45.29	0.01\\
45.3	0.01\\
45.31	0.01\\
45.32	0.01\\
45.33	0.01\\
45.34	0.01\\
45.35	0.01\\
45.36	0.01\\
45.37	0.01\\
45.38	0.01\\
45.39	0.01\\
45.4	0.01\\
45.41	0.01\\
45.42	0.01\\
45.43	0.01\\
45.44	0.01\\
45.45	0.01\\
45.46	0.01\\
45.47	0.01\\
45.48	0.01\\
45.49	0.01\\
45.5	0.01\\
45.51	0.01\\
45.52	0.01\\
45.53	0.01\\
45.54	0.01\\
45.55	0.01\\
45.56	0.01\\
45.57	0.01\\
45.58	0.01\\
45.59	0.01\\
45.6	0.01\\
45.61	0.01\\
45.62	0.01\\
45.63	0.01\\
45.64	0.01\\
45.65	0.01\\
45.66	0.01\\
45.67	0.01\\
45.68	0.01\\
45.69	0.01\\
45.7	0.01\\
45.71	0.01\\
45.72	0.01\\
45.73	0.01\\
45.74	0.01\\
45.75	0.01\\
45.76	0.01\\
45.77	0.01\\
45.78	0.01\\
45.79	0.01\\
45.8	0.01\\
45.81	0.01\\
45.82	0.01\\
45.83	0.01\\
45.84	0.01\\
45.85	0.01\\
45.86	0.01\\
45.87	0.01\\
45.88	0.01\\
45.89	0.01\\
45.9	0.01\\
45.91	0.01\\
45.92	0.01\\
45.93	0.01\\
45.94	0.01\\
45.95	0.01\\
45.96	0.01\\
45.97	0.01\\
45.98	0.01\\
45.99	0.01\\
46	0.01\\
46.01	0.01\\
46.02	0.01\\
46.03	0.01\\
46.04	0.01\\
46.05	0.01\\
46.06	0.01\\
46.07	0.01\\
46.08	0.01\\
46.09	0.01\\
46.1	0.01\\
46.11	0.01\\
46.12	0.01\\
46.13	0.01\\
46.14	0.01\\
46.15	0.01\\
46.16	0.01\\
46.17	0.01\\
46.18	0.01\\
46.19	0.01\\
46.2	0.01\\
46.21	0.01\\
46.22	0.01\\
46.23	0.01\\
46.24	0.01\\
46.25	0.01\\
46.26	0.01\\
46.27	0.01\\
46.28	0.01\\
46.29	0.01\\
46.3	0.01\\
46.31	0.01\\
46.32	0.01\\
46.33	0.01\\
46.34	0.01\\
46.35	0.01\\
46.36	0.01\\
46.37	0.01\\
46.38	0.01\\
46.39	0.01\\
46.4	0.01\\
46.41	0.01\\
46.42	0.01\\
46.43	0.01\\
46.44	0.01\\
46.45	0.01\\
46.46	0.01\\
46.47	0.01\\
46.48	0.01\\
46.49	0.01\\
46.5	0.01\\
46.51	0.01\\
46.52	0.01\\
46.53	0.01\\
46.54	0.01\\
46.55	0.01\\
46.56	0.01\\
46.57	0.01\\
46.58	0.01\\
46.59	0.01\\
46.6	0.01\\
46.61	0.01\\
46.62	0.01\\
46.63	0.01\\
46.64	0.01\\
46.65	0.01\\
46.66	0.01\\
46.67	0.01\\
46.68	0.01\\
46.69	0.01\\
46.7	0.01\\
46.71	0.01\\
46.72	0.01\\
46.73	0.01\\
46.74	0.01\\
46.75	0.01\\
46.76	0.01\\
46.77	0.01\\
46.78	0.01\\
46.79	0.01\\
46.8	0.01\\
46.81	0.01\\
46.82	0.01\\
46.83	0.01\\
46.84	0.01\\
46.85	0.01\\
46.86	0.01\\
46.87	0.01\\
46.88	0.01\\
46.89	0.01\\
46.9	0.01\\
46.91	0.01\\
46.92	0.01\\
46.93	0.01\\
46.94	0.01\\
46.95	0.01\\
46.96	0.01\\
46.97	0.01\\
46.98	0.01\\
46.99	0.01\\
47	0.01\\
47.01	0.01\\
47.02	0.01\\
47.03	0.01\\
47.04	0.01\\
47.05	0.01\\
47.06	0.01\\
47.07	0.01\\
47.08	0.01\\
47.09	0.01\\
47.1	0.01\\
47.11	0.01\\
47.12	0.01\\
47.13	0.01\\
47.14	0.01\\
47.15	0.01\\
47.16	0.01\\
47.17	0.01\\
47.18	0.01\\
47.19	0.01\\
47.2	0.01\\
47.21	0.01\\
47.22	0.01\\
47.23	0.01\\
47.24	0.01\\
47.25	0.01\\
47.26	0.01\\
47.27	0.01\\
47.28	0.01\\
47.29	0.01\\
47.3	0.01\\
47.31	0.01\\
47.32	0.01\\
47.33	0.01\\
47.34	0.01\\
47.35	0.01\\
47.36	0.01\\
47.37	0.01\\
47.38	0.01\\
47.39	0.01\\
47.4	0.01\\
47.41	0.01\\
47.42	0.01\\
47.43	0.01\\
47.44	0.01\\
47.45	0.01\\
47.46	0.01\\
47.47	0.01\\
47.48	0.01\\
47.49	0.01\\
47.5	0.01\\
47.51	0.01\\
47.52	0.01\\
47.53	0.01\\
47.54	0.01\\
47.55	0.01\\
47.56	0.01\\
47.57	0.01\\
47.58	0.01\\
47.59	0.01\\
47.6	0.01\\
47.61	0.01\\
47.62	0.01\\
47.63	0.01\\
47.64	0.01\\
47.65	0.01\\
47.66	0.01\\
47.67	0.01\\
47.68	0.01\\
47.69	0.01\\
47.7	0.01\\
47.71	0.01\\
47.72	0.01\\
47.73	0.01\\
47.74	0.01\\
47.75	0.01\\
47.76	0.01\\
47.77	0.01\\
47.78	0.01\\
47.79	0.01\\
47.8	0.01\\
47.81	0.01\\
47.82	0.01\\
47.83	0.01\\
47.84	0.01\\
47.85	0.01\\
47.86	0.01\\
47.87	0.01\\
47.88	0.01\\
47.89	0.01\\
47.9	0.01\\
47.91	0.01\\
47.92	0.01\\
47.93	0.01\\
47.94	0.01\\
47.95	0.01\\
47.96	0.01\\
47.97	0.01\\
47.98	0.01\\
47.99	0.01\\
48	0.01\\
48.01	0.01\\
48.02	0.01\\
48.03	0.01\\
48.04	0.01\\
48.05	0.01\\
48.06	0.01\\
48.07	0.01\\
48.08	0.01\\
48.09	0.01\\
48.1	0.01\\
48.11	0.01\\
48.12	0.01\\
48.13	0.01\\
48.14	0.01\\
48.15	0.01\\
48.16	0.01\\
48.17	0.01\\
48.18	0.01\\
48.19	0.01\\
48.2	0.01\\
48.21	0.01\\
48.22	0.01\\
48.23	0.01\\
48.24	0.01\\
48.25	0.01\\
48.26	0.01\\
48.27	0.01\\
48.28	0.01\\
48.29	0.01\\
48.3	0.01\\
48.31	0.01\\
48.32	0.01\\
48.33	0.01\\
48.34	0.01\\
48.35	0.01\\
48.36	0.01\\
48.37	0.01\\
48.38	0.01\\
48.39	0.01\\
48.4	0.01\\
48.41	0.01\\
48.42	0.01\\
48.43	0.01\\
48.44	0.01\\
48.45	0.01\\
48.46	0.01\\
48.47	0.01\\
48.48	0.01\\
48.49	0.01\\
48.5	0.01\\
48.51	0.01\\
48.52	0.01\\
48.53	0.01\\
48.54	0.01\\
48.55	0.01\\
48.56	0.01\\
48.57	0.01\\
48.58	0.01\\
48.59	0.01\\
48.6	0.01\\
48.61	0.01\\
48.62	0.01\\
48.63	0.01\\
48.64	0.01\\
48.65	0.01\\
48.66	0.01\\
48.67	0.01\\
48.68	0.01\\
48.69	0.01\\
48.7	0.01\\
48.71	0.01\\
48.72	0.01\\
48.73	0.01\\
48.74	0.01\\
48.75	0.01\\
48.76	0.01\\
48.77	0.01\\
48.78	0.01\\
48.79	0.01\\
48.8	0.01\\
48.81	0.01\\
48.82	0.01\\
48.83	0.01\\
48.84	0.01\\
48.85	0.01\\
48.86	0.01\\
48.87	0.01\\
48.88	0.01\\
48.89	0.01\\
48.9	0.01\\
48.91	0.01\\
48.92	0.01\\
48.93	0.01\\
48.94	0.01\\
48.95	0.01\\
48.96	0.01\\
48.97	0.01\\
48.98	0.01\\
48.99	0.01\\
49	0.01\\
49.01	0.01\\
49.02	0.01\\
49.03	0.01\\
49.04	0.01\\
49.05	0.01\\
49.06	0.01\\
49.07	0.01\\
49.08	0.01\\
49.09	0.01\\
49.1	0.01\\
49.11	0.01\\
49.12	0.01\\
49.13	0.01\\
49.14	0.01\\
49.15	0.01\\
49.16	0.01\\
49.17	0.01\\
49.18	0.01\\
49.19	0.01\\
49.2	0.01\\
49.21	0.01\\
49.22	0.01\\
49.23	0.01\\
49.24	0.01\\
49.25	0.01\\
49.26	0.01\\
49.27	0.01\\
49.28	0.01\\
49.29	0.01\\
49.3	0.01\\
49.31	0.01\\
49.32	0.01\\
49.33	0.01\\
49.34	0.01\\
49.35	0.01\\
49.36	0.01\\
49.37	0.01\\
49.38	0.01\\
49.39	0.01\\
49.4	0.01\\
49.41	0.01\\
49.42	0.01\\
49.43	0.01\\
49.44	0.01\\
49.45	0.01\\
49.46	0.01\\
49.47	0.01\\
49.48	0.01\\
49.49	0.01\\
49.5	0.01\\
49.51	0.01\\
49.52	0.01\\
49.53	0.01\\
49.54	0.01\\
49.55	0.01\\
49.56	0.01\\
49.57	0.01\\
49.58	0.01\\
49.59	0.01\\
49.6	0.01\\
49.61	0.01\\
49.62	0.01\\
49.63	0.01\\
49.64	0.01\\
49.65	0.01\\
49.66	0.01\\
49.67	0.01\\
49.68	0.01\\
49.69	0.01\\
49.7	0.01\\
49.71	0.01\\
49.72	0.01\\
49.73	0.01\\
49.74	0.01\\
49.75	0.01\\
49.76	0.01\\
49.77	0.01\\
49.78	0.01\\
49.79	0.01\\
49.8	0.01\\
49.81	0.01\\
49.82	0.01\\
49.83	0.01\\
49.84	0.01\\
49.85	0.01\\
49.86	0.01\\
49.87	0.01\\
49.88	0.01\\
49.89	0.01\\
49.9	0.01\\
49.91	0.01\\
49.92	0.01\\
49.93	0.01\\
49.94	0.01\\
49.95	0.01\\
49.96	0.01\\
49.97	0.01\\
49.98	0.01\\
49.99	0.01\\
50	0.01\\
50.01	0.01\\
50.02	0.01\\
50.03	0.01\\
50.04	0.01\\
50.05	0.01\\
50.06	0.01\\
50.07	0.01\\
50.08	0.01\\
50.09	0.01\\
50.1	0.01\\
50.11	0.01\\
50.12	0.01\\
50.13	0.01\\
50.14	0.01\\
50.15	0.01\\
50.16	0.01\\
50.17	0.01\\
50.18	0.01\\
50.19	0.01\\
50.2	0.01\\
50.21	0.01\\
50.22	0.01\\
50.23	0.01\\
50.24	0.01\\
50.25	0.01\\
50.26	0.01\\
50.27	0.01\\
50.28	0.01\\
50.29	0.01\\
50.3	0.01\\
50.31	0.01\\
50.32	0.01\\
50.33	0.01\\
50.34	0.01\\
50.35	0.01\\
50.36	0.01\\
50.37	0.01\\
50.38	0.01\\
50.39	0.01\\
50.4	0.01\\
50.41	0.01\\
50.42	0.01\\
50.43	0.01\\
50.44	0.01\\
50.45	0.01\\
50.46	0.01\\
50.47	0.01\\
50.48	0.01\\
50.49	0.01\\
50.5	0.01\\
50.51	0.01\\
50.52	0.01\\
50.53	0.01\\
50.54	0.01\\
50.55	0.01\\
50.56	0.01\\
50.57	0.01\\
50.58	0.01\\
50.59	0.01\\
50.6	0.01\\
50.61	0.01\\
50.62	0.01\\
50.63	0.01\\
50.64	0.01\\
50.65	0.01\\
50.66	0.01\\
50.67	0.01\\
50.68	0.01\\
50.69	0.01\\
50.7	0.01\\
50.71	0.01\\
50.72	0.01\\
50.73	0.01\\
50.74	0.01\\
50.75	0.01\\
50.76	0.01\\
50.77	0.01\\
50.78	0.01\\
50.79	0.01\\
50.8	0.01\\
50.81	0.01\\
50.82	0.01\\
50.83	0.01\\
50.84	0.01\\
50.85	0.01\\
50.86	0.01\\
50.87	0.01\\
50.88	0.01\\
50.89	0.01\\
50.9	0.01\\
50.91	0.01\\
50.92	0.01\\
50.93	0.01\\
50.94	0.01\\
50.95	0.01\\
50.96	0.01\\
50.97	0.01\\
50.98	0.01\\
50.99	0.01\\
51	0.01\\
51.01	0.01\\
51.02	0.01\\
51.03	0.01\\
51.04	0.01\\
51.05	0.01\\
51.06	0.01\\
51.07	0.01\\
51.08	0.01\\
51.09	0.01\\
51.1	0.01\\
51.11	0.01\\
51.12	0.01\\
51.13	0.01\\
51.14	0.01\\
51.15	0.01\\
51.16	0.01\\
51.17	0.01\\
51.18	0.01\\
51.19	0.01\\
51.2	0.01\\
51.21	0.01\\
51.22	0.01\\
51.23	0.01\\
51.24	0.01\\
51.25	0.01\\
51.26	0.01\\
51.27	0.01\\
51.28	0.01\\
51.29	0.01\\
51.3	0.01\\
51.31	0.01\\
51.32	0.01\\
51.33	0.01\\
51.34	0.01\\
51.35	0.01\\
51.36	0.01\\
51.37	0.01\\
51.38	0.01\\
51.39	0.01\\
51.4	0.01\\
51.41	0.01\\
51.42	0.01\\
51.43	0.01\\
51.44	0.01\\
51.45	0.01\\
51.46	0.01\\
51.47	0.01\\
51.48	0.01\\
51.49	0.01\\
51.5	0.01\\
51.51	0.01\\
51.52	0.01\\
51.53	0.01\\
51.54	0.01\\
51.55	0.01\\
51.56	0.01\\
51.57	0.01\\
51.58	0.01\\
51.59	0.01\\
51.6	0.01\\
51.61	0.01\\
51.62	0.01\\
51.63	0.01\\
51.64	0.01\\
51.65	0.01\\
51.66	0.01\\
51.67	0.01\\
51.68	0.01\\
51.69	0.01\\
51.7	0.01\\
51.71	0.01\\
51.72	0.01\\
51.73	0.01\\
51.74	0.01\\
51.75	0.01\\
51.76	0.01\\
51.77	0.01\\
51.78	0.01\\
51.79	0.01\\
51.8	0.01\\
51.81	0.01\\
51.82	0.01\\
51.83	0.01\\
51.84	0.01\\
51.85	0.01\\
51.86	0.01\\
51.87	0.01\\
51.88	0.01\\
51.89	0.01\\
51.9	0.01\\
51.91	0.01\\
51.92	0.01\\
51.93	0.01\\
51.94	0.01\\
51.95	0.01\\
51.96	0.01\\
51.97	0.01\\
51.98	0.01\\
51.99	0.01\\
52	0.01\\
52.01	0.01\\
52.02	0.01\\
52.03	0.01\\
52.04	0.01\\
52.05	0.01\\
52.06	0.01\\
52.07	0.01\\
52.08	0.01\\
52.09	0.01\\
52.1	0.01\\
52.11	0.01\\
52.12	0.01\\
52.13	0.01\\
52.14	0.01\\
52.15	0.01\\
52.16	0.01\\
52.17	0.01\\
52.18	0.01\\
52.19	0.01\\
52.2	0.01\\
52.21	0.01\\
52.22	0.01\\
52.23	0.01\\
52.24	0.01\\
52.25	0.01\\
52.26	0.01\\
52.27	0.01\\
52.28	0.01\\
52.29	0.01\\
52.3	0.01\\
52.31	0.01\\
52.32	0.01\\
52.33	0.01\\
52.34	0.01\\
52.35	0.01\\
52.36	0.01\\
52.37	0.01\\
52.38	0.01\\
52.39	0.01\\
52.4	0.01\\
52.41	0.01\\
52.42	0.01\\
52.43	0.01\\
52.44	0.01\\
52.45	0.01\\
52.46	0.01\\
52.47	0.01\\
52.48	0.01\\
52.49	0.01\\
52.5	0.01\\
52.51	0.01\\
52.52	0.01\\
52.53	0.01\\
52.54	0.01\\
52.55	0.01\\
52.56	0.01\\
52.57	0.01\\
52.58	0.01\\
52.59	0.01\\
52.6	0.01\\
52.61	0.01\\
52.62	0.01\\
52.63	0.01\\
52.64	0.01\\
52.65	0.01\\
52.66	0.01\\
52.67	0.01\\
52.68	0.01\\
52.69	0.01\\
52.7	0.01\\
52.71	0.01\\
52.72	0.01\\
52.73	0.01\\
52.74	0.01\\
52.75	0.01\\
52.76	0.01\\
52.77	0.01\\
52.78	0.01\\
52.79	0.01\\
52.8	0.01\\
52.81	0.01\\
52.82	0.01\\
52.83	0.01\\
52.84	0.01\\
52.85	0.01\\
52.86	0.01\\
52.87	0.01\\
52.88	0.01\\
52.89	0.01\\
52.9	0.01\\
52.91	0.01\\
52.92	0.01\\
52.93	0.01\\
52.94	0.01\\
52.95	0.01\\
52.96	0.01\\
52.97	0.01\\
52.98	0.01\\
52.99	0.01\\
53	0.01\\
53.01	0.01\\
53.02	0.01\\
53.03	0.01\\
53.04	0.01\\
53.05	0.01\\
53.06	0.01\\
53.07	0.01\\
53.08	0.01\\
53.09	0.01\\
53.1	0.01\\
53.11	0.01\\
53.12	0.01\\
53.13	0.01\\
53.14	0.01\\
53.15	0.01\\
53.16	0.01\\
53.17	0.01\\
53.18	0.01\\
53.19	0.01\\
53.2	0.01\\
53.21	0.01\\
53.22	0.01\\
53.23	0.01\\
53.24	0.01\\
53.25	0.01\\
53.26	0.01\\
53.27	0.01\\
53.28	0.01\\
53.29	0.01\\
53.3	0.01\\
53.31	0.01\\
53.32	0.01\\
53.33	0.01\\
53.34	0.01\\
53.35	0.01\\
53.36	0.01\\
53.37	0.01\\
53.38	0.01\\
53.39	0.01\\
53.4	0.01\\
53.41	0.01\\
53.42	0.01\\
53.43	0.01\\
53.44	0.01\\
53.45	0.01\\
53.46	0.01\\
53.47	0.01\\
53.48	0.01\\
53.49	0.01\\
53.5	0.01\\
53.51	0.01\\
53.52	0.01\\
53.53	0.01\\
53.54	0.01\\
53.55	0.01\\
53.56	0.01\\
53.57	0.01\\
53.58	0.01\\
53.59	0.01\\
53.6	0.01\\
53.61	0.01\\
53.62	0.01\\
53.63	0.01\\
53.64	0.01\\
53.65	0.01\\
53.66	0.01\\
53.67	0.01\\
53.68	0.01\\
53.69	0.01\\
53.7	0.01\\
53.71	0.01\\
53.72	0.01\\
53.73	0.01\\
53.74	0.01\\
53.75	0.01\\
53.76	0.01\\
53.77	0.01\\
53.78	0.01\\
53.79	0.01\\
53.8	0.01\\
53.81	0.01\\
53.82	0.01\\
53.83	0.01\\
53.84	0.01\\
53.85	0.01\\
53.86	0.01\\
53.87	0.01\\
53.88	0.01\\
53.89	0.01\\
53.9	0.01\\
53.91	0.01\\
53.92	0.01\\
53.93	0.01\\
53.94	0.01\\
53.95	0.01\\
53.96	0.01\\
53.97	0.01\\
53.98	0.01\\
53.99	0.01\\
54	0.01\\
54.01	0.01\\
54.02	0.01\\
54.03	0.01\\
54.04	0.01\\
54.05	0.01\\
54.06	0.01\\
54.07	0.01\\
54.08	0.01\\
54.09	0.01\\
54.1	0.01\\
54.11	0.01\\
54.12	0.01\\
54.13	0.01\\
54.14	0.01\\
54.15	0.01\\
54.16	0.01\\
54.17	0.01\\
54.18	0.01\\
54.19	0.01\\
54.2	0.01\\
54.21	0.01\\
54.22	0.01\\
54.23	0.01\\
54.24	0.01\\
54.25	0.01\\
54.26	0.01\\
54.27	0.01\\
54.28	0.01\\
54.29	0.01\\
54.3	0.01\\
54.31	0.01\\
54.32	0.01\\
54.33	0.01\\
54.34	0.01\\
54.35	0.01\\
54.36	0.01\\
54.37	0.01\\
54.38	0.01\\
54.39	0.01\\
54.4	0.01\\
54.41	0.01\\
54.42	0.01\\
54.43	0.01\\
54.44	0.01\\
54.45	0.01\\
54.46	0.01\\
54.47	0.01\\
54.48	0.01\\
54.49	0.01\\
54.5	0.01\\
54.51	0.01\\
54.52	0.01\\
54.53	0.01\\
54.54	0.01\\
54.55	0.01\\
54.56	0.01\\
54.57	0.01\\
54.58	0.01\\
54.59	0.01\\
54.6	0.01\\
54.61	0.01\\
54.62	0.01\\
54.63	0.01\\
54.64	0.01\\
54.65	0.01\\
54.66	0.01\\
54.67	0.01\\
54.68	0.01\\
54.69	0.01\\
54.7	0.01\\
54.71	0.01\\
54.72	0.01\\
54.73	0.01\\
54.74	0.01\\
54.75	0.01\\
54.76	0.01\\
54.77	0.01\\
54.78	0.01\\
54.79	0.01\\
54.8	0.01\\
54.81	0.01\\
54.82	0.01\\
54.83	0.01\\
54.84	0.01\\
54.85	0.01\\
54.86	0.01\\
54.87	0.01\\
54.88	0.01\\
54.89	0.01\\
54.9	0.01\\
54.91	0.01\\
54.92	0.01\\
54.93	0.01\\
54.94	0.01\\
54.95	0.01\\
54.96	0.01\\
54.97	0.01\\
54.98	0.01\\
54.99	0.01\\
55	0.01\\
55.01	0.01\\
55.02	0.01\\
55.03	0.01\\
55.04	0.01\\
55.05	0.01\\
55.06	0.01\\
55.07	0.01\\
55.08	0.01\\
55.09	0.01\\
55.1	0.01\\
55.11	0.01\\
55.12	0.01\\
55.13	0.01\\
55.14	0.01\\
55.15	0.01\\
55.16	0.01\\
55.17	0.01\\
55.18	0.01\\
55.19	0.01\\
55.2	0.01\\
55.21	0.01\\
55.22	0.01\\
55.23	0.01\\
55.24	0.01\\
55.25	0.01\\
55.26	0.01\\
55.27	0.01\\
55.28	0.01\\
55.29	0.01\\
55.3	0.01\\
55.31	0.01\\
55.32	0.01\\
55.33	0.01\\
55.34	0.01\\
55.35	0.01\\
55.36	0.01\\
55.37	0.01\\
55.38	0.01\\
55.39	0.01\\
55.4	0.01\\
55.41	0.01\\
55.42	0.01\\
55.43	0.01\\
55.44	0.01\\
55.45	0.01\\
55.46	0.01\\
55.47	0.01\\
55.48	0.01\\
55.49	0.01\\
55.5	0.01\\
55.51	0.01\\
55.52	0.01\\
55.53	0.01\\
55.54	0.01\\
55.55	0.01\\
55.56	0.01\\
55.57	0.01\\
55.58	0.01\\
55.59	0.01\\
55.6	0.01\\
55.61	0.01\\
55.62	0.01\\
55.63	0.01\\
55.64	0.01\\
55.65	0.01\\
55.66	0.01\\
55.67	0.01\\
55.68	0.01\\
55.69	0.01\\
55.7	0.01\\
55.71	0.01\\
55.72	0.01\\
55.73	0.01\\
55.74	0.01\\
55.75	0.01\\
55.76	0.01\\
55.77	0.01\\
55.78	0.01\\
55.79	0.01\\
55.8	0.01\\
55.81	0.01\\
55.82	0.01\\
55.83	0.01\\
55.84	0.01\\
55.85	0.01\\
55.86	0.01\\
55.87	0.01\\
55.88	0.01\\
55.89	0.01\\
55.9	0.01\\
55.91	0.01\\
55.92	0.01\\
55.93	0.01\\
55.94	0.01\\
55.95	0.01\\
55.96	0.01\\
55.97	0.01\\
55.98	0.01\\
55.99	0.01\\
56	0.01\\
56.01	0.01\\
56.02	0.01\\
56.03	0.01\\
56.04	0.01\\
56.05	0.01\\
56.06	0.01\\
56.07	0.01\\
56.08	0.01\\
56.09	0.01\\
56.1	0.01\\
56.11	0.01\\
56.12	0.01\\
56.13	0.01\\
56.14	0.01\\
56.15	0.01\\
56.16	0.01\\
56.17	0.01\\
56.18	0.01\\
56.19	0.01\\
56.2	0.01\\
56.21	0.01\\
56.22	0.01\\
56.23	0.01\\
56.24	0.01\\
56.25	0.01\\
56.26	0.01\\
56.27	0.01\\
56.28	0.01\\
56.29	0.01\\
56.3	0.01\\
56.31	0.01\\
56.32	0.01\\
56.33	0.01\\
56.34	0.01\\
56.35	0.01\\
56.36	0.01\\
56.37	0.01\\
56.38	0.01\\
56.39	0.01\\
56.4	0.01\\
56.41	0.01\\
56.42	0.01\\
56.43	0.01\\
56.44	0.01\\
56.45	0.01\\
56.46	0.01\\
56.47	0.01\\
56.48	0.01\\
56.49	0.01\\
56.5	0.01\\
56.51	0.01\\
56.52	0.01\\
56.53	0.01\\
56.54	0.01\\
56.55	0.01\\
56.56	0.01\\
56.57	0.01\\
56.58	0.01\\
56.59	0.01\\
56.6	0.01\\
56.61	0.01\\
56.62	0.01\\
56.63	0.01\\
56.64	0.01\\
56.65	0.01\\
56.66	0.01\\
56.67	0.01\\
56.68	0.01\\
56.69	0.01\\
56.7	0.01\\
56.71	0.01\\
56.72	0.01\\
56.73	0.01\\
56.74	0.01\\
56.75	0.01\\
56.76	0.01\\
56.77	0.01\\
56.78	0.01\\
56.79	0.01\\
56.8	0.01\\
56.81	0.01\\
56.82	0.01\\
56.83	0.01\\
56.84	0.01\\
56.85	0.01\\
56.86	0.01\\
56.87	0.01\\
56.88	0.01\\
56.89	0.01\\
56.9	0.01\\
56.91	0.01\\
56.92	0.01\\
56.93	0.01\\
56.94	0.01\\
56.95	0.01\\
56.96	0.01\\
56.97	0.01\\
56.98	0.01\\
56.99	0.01\\
57	0.01\\
57.01	0.01\\
57.02	0.01\\
57.03	0.01\\
57.04	0.01\\
57.05	0.01\\
57.06	0.01\\
57.07	0.01\\
57.08	0.01\\
57.09	0.01\\
57.1	0.01\\
57.11	0.01\\
57.12	0.01\\
57.13	0.01\\
57.14	0.01\\
57.15	0.01\\
57.16	0.01\\
57.17	0.01\\
57.18	0.01\\
57.19	0.01\\
57.2	0.01\\
57.21	0.01\\
57.22	0.01\\
57.23	0.01\\
57.24	0.01\\
57.25	0.01\\
57.26	0.01\\
57.27	0.01\\
57.28	0.01\\
57.29	0.01\\
57.3	0.01\\
57.31	0.01\\
57.32	0.01\\
57.33	0.01\\
57.34	0.01\\
57.35	0.01\\
57.36	0.01\\
57.37	0.01\\
57.38	0.01\\
57.39	0.01\\
57.4	0.01\\
57.41	0.01\\
57.42	0.01\\
57.43	0.01\\
57.44	0.01\\
57.45	0.01\\
57.46	0.01\\
57.47	0.01\\
57.48	0.01\\
57.49	0.01\\
57.5	0.01\\
57.51	0.01\\
57.52	0.01\\
57.53	0.01\\
57.54	0.01\\
57.55	0.01\\
57.56	0.01\\
57.57	0.01\\
57.58	0.01\\
57.59	0.01\\
57.6	0.01\\
57.61	0.01\\
57.62	0.01\\
57.63	0.01\\
57.64	0.01\\
57.65	0.01\\
57.66	0.01\\
57.67	0.01\\
57.68	0.01\\
57.69	0.01\\
57.7	0.01\\
57.71	0.01\\
57.72	0.01\\
57.73	0.01\\
57.74	0.01\\
57.75	0.01\\
57.76	0.01\\
57.77	0.01\\
57.78	0.01\\
57.79	0.01\\
57.8	0.01\\
57.81	0.01\\
57.82	0.01\\
57.83	0.01\\
57.84	0.01\\
57.85	0.01\\
57.86	0.01\\
57.87	0.01\\
57.88	0.01\\
57.89	0.01\\
57.9	0.01\\
57.91	0.01\\
57.92	0.01\\
57.93	0.01\\
57.94	0.01\\
57.95	0.01\\
57.96	0.01\\
57.97	0.01\\
57.98	0.01\\
57.99	0.01\\
58	0.01\\
58.01	0.01\\
58.02	0.01\\
58.03	0.01\\
58.04	0.01\\
58.05	0.01\\
58.06	0.01\\
58.07	0.01\\
58.08	0.01\\
58.09	0.01\\
58.1	0.01\\
58.11	0.01\\
58.12	0.01\\
58.13	0.01\\
58.14	0.01\\
58.15	0.01\\
58.16	0.01\\
58.17	0.01\\
58.18	0.01\\
58.19	0.01\\
58.2	0.01\\
58.21	0.01\\
58.22	0.01\\
58.23	0.01\\
58.24	0.01\\
58.25	0.01\\
58.26	0.01\\
58.27	0.01\\
58.28	0.01\\
58.29	0.01\\
58.3	0.01\\
58.31	0.01\\
58.32	0.01\\
58.33	0.01\\
58.34	0.01\\
58.35	0.01\\
58.36	0.01\\
58.37	0.01\\
58.38	0.01\\
58.39	0.01\\
58.4	0.01\\
58.41	0.01\\
58.42	0.01\\
58.43	0.01\\
58.44	0.01\\
58.45	0.01\\
58.46	0.01\\
58.47	0.01\\
58.48	0.01\\
58.49	0.01\\
58.5	0.01\\
58.51	0.01\\
58.52	0.01\\
58.53	0.01\\
58.54	0.01\\
58.55	0.01\\
58.56	0.01\\
58.57	0.01\\
58.58	0.01\\
58.59	0.01\\
58.6	0.01\\
58.61	0.01\\
58.62	0.01\\
58.63	0.01\\
58.64	0.01\\
58.65	0.01\\
58.66	0.01\\
58.67	0.01\\
58.68	0.01\\
58.69	0.01\\
58.7	0.01\\
58.71	0.01\\
58.72	0.01\\
58.73	0.01\\
58.74	0.01\\
58.75	0.01\\
58.76	0.01\\
58.77	0.01\\
58.78	0.01\\
58.79	0.01\\
58.8	0.01\\
58.81	0.01\\
58.82	0.01\\
58.83	0.01\\
58.84	0.01\\
58.85	0.01\\
58.86	0.01\\
58.87	0.01\\
58.88	0.01\\
58.89	0.01\\
58.9	0.01\\
58.91	0.01\\
58.92	0.01\\
58.93	0.01\\
58.94	0.01\\
58.95	0.01\\
58.96	0.01\\
58.97	0.01\\
58.98	0.01\\
58.99	0.01\\
59	0.01\\
59.01	0.01\\
59.02	0.01\\
59.03	0.01\\
59.04	0.01\\
59.05	0.01\\
59.06	0.01\\
59.07	0.01\\
59.08	0.01\\
59.09	0.01\\
59.1	0.01\\
59.11	0.01\\
59.12	0.01\\
59.13	0.01\\
59.14	0.01\\
59.15	0.01\\
59.16	0.01\\
59.17	0.01\\
59.18	0.01\\
59.19	0.01\\
59.2	0.01\\
59.21	0.01\\
59.22	0.01\\
59.23	0.01\\
59.24	0.01\\
59.25	0.01\\
59.26	0.01\\
59.27	0.01\\
59.28	0.01\\
59.29	0.01\\
59.3	0.01\\
59.31	0.01\\
59.32	0.01\\
59.33	0.01\\
59.34	0.01\\
59.35	0.01\\
59.36	0.01\\
59.37	0.01\\
59.38	0.01\\
59.39	0.01\\
59.4	0.01\\
59.41	0.01\\
59.42	0.01\\
59.43	0.01\\
59.44	0.01\\
59.45	0.01\\
59.46	0.01\\
59.47	0.01\\
59.48	0.01\\
59.49	0.01\\
59.5	0.01\\
59.51	0.01\\
59.52	0.01\\
59.53	0.01\\
59.54	0.01\\
59.55	0.01\\
59.56	0.01\\
59.57	0.01\\
59.58	0.01\\
59.59	0.01\\
59.6	0.01\\
59.61	0.01\\
59.62	0.01\\
59.63	0.01\\
59.64	0.01\\
59.65	0.01\\
59.66	0.01\\
59.67	0.01\\
59.68	0.01\\
59.69	0.01\\
59.7	0.01\\
59.71	0.01\\
59.72	0.01\\
59.73	0.01\\
59.74	0.01\\
59.75	0.01\\
59.76	0.01\\
59.77	0.01\\
59.78	0.01\\
59.79	0.01\\
59.8	0.01\\
59.81	0.01\\
59.82	0.01\\
59.83	0.01\\
59.84	0.01\\
59.85	0.01\\
59.86	0.01\\
59.87	0.01\\
59.88	0.01\\
59.89	0.01\\
59.9	0.01\\
59.91	0.01\\
59.92	0.01\\
59.93	0.01\\
59.94	0.01\\
59.95	0.01\\
59.96	0.01\\
59.97	0.01\\
59.98	0.01\\
59.99	0.01\\
60	0.01\\
60.01	0.01\\
60.02	0.01\\
60.03	0.01\\
60.04	0.01\\
60.05	0.01\\
60.06	0.01\\
60.07	0.01\\
60.08	0.01\\
60.09	0.01\\
60.1	0.01\\
60.11	0.01\\
60.12	0.01\\
60.13	0.01\\
60.14	0.01\\
60.15	0.01\\
60.16	0.01\\
60.17	0.01\\
60.18	0.01\\
60.19	0.01\\
60.2	0.01\\
60.21	0.01\\
60.22	0.01\\
60.23	0.01\\
60.24	0.01\\
60.25	0.01\\
60.26	0.01\\
60.27	0.01\\
60.28	0.01\\
60.29	0.01\\
60.3	0.01\\
60.31	0.01\\
60.32	0.01\\
60.33	0.01\\
60.34	0.01\\
60.35	0.01\\
60.36	0.01\\
60.37	0.01\\
60.38	0.01\\
60.39	0.01\\
60.4	0.01\\
60.41	0.01\\
60.42	0.01\\
60.43	0.01\\
60.44	0.01\\
60.45	0.01\\
60.46	0.01\\
60.47	0.01\\
60.48	0.01\\
60.49	0.01\\
60.5	0.01\\
60.51	0.01\\
60.52	0.01\\
60.53	0.01\\
60.54	0.01\\
60.55	0.01\\
60.56	0.01\\
60.57	0.01\\
60.58	0.01\\
60.59	0.01\\
60.6	0.01\\
60.61	0.01\\
60.62	0.01\\
60.63	0.01\\
60.64	0.01\\
60.65	0.01\\
60.66	0.01\\
60.67	0.01\\
60.68	0.01\\
60.69	0.01\\
60.7	0.01\\
60.71	0.01\\
60.72	0.01\\
60.73	0.01\\
60.74	0.01\\
60.75	0.01\\
60.76	0.01\\
60.77	0.01\\
60.78	0.01\\
60.79	0.01\\
60.8	0.01\\
60.81	0.01\\
60.82	0.01\\
60.83	0.01\\
60.84	0.01\\
60.85	0.01\\
60.86	0.01\\
60.87	0.01\\
60.88	0.01\\
60.89	0.01\\
60.9	0.01\\
60.91	0.01\\
60.92	0.01\\
60.93	0.01\\
60.94	0.01\\
60.95	0.01\\
60.96	0.01\\
60.97	0.01\\
60.98	0.01\\
60.99	0.01\\
61	0.01\\
61.01	0.01\\
61.02	0.01\\
61.03	0.01\\
61.04	0.01\\
61.05	0.01\\
61.06	0.01\\
61.07	0.01\\
61.08	0.01\\
61.09	0.01\\
61.1	0.01\\
61.11	0.01\\
61.12	0.01\\
61.13	0.01\\
61.14	0.01\\
61.15	0.01\\
61.16	0.01\\
61.17	0.01\\
61.18	0.01\\
61.19	0.01\\
61.2	0.01\\
61.21	0.01\\
61.22	0.01\\
61.23	0.01\\
61.24	0.01\\
61.25	0.01\\
61.26	0.01\\
61.27	0.01\\
61.28	0.01\\
61.29	0.01\\
61.3	0.01\\
61.31	0.01\\
61.32	0.01\\
61.33	0.01\\
61.34	0.01\\
61.35	0.01\\
61.36	0.01\\
61.37	0.01\\
61.38	0.01\\
61.39	0.01\\
61.4	0.01\\
61.41	0.01\\
61.42	0.01\\
61.43	0.01\\
61.44	0.01\\
61.45	0.01\\
61.46	0.01\\
61.47	0.01\\
61.48	0.01\\
61.49	0.01\\
61.5	0.01\\
61.51	0.01\\
61.52	0.01\\
61.53	0.01\\
61.54	0.01\\
61.55	0.01\\
61.56	0.01\\
61.57	0.01\\
61.58	0.01\\
61.59	0.01\\
61.6	0.01\\
61.61	0.01\\
61.62	0.01\\
61.63	0.01\\
61.64	0.01\\
61.65	0.01\\
61.66	0.01\\
61.67	0.01\\
61.68	0.01\\
61.69	0.01\\
61.7	0.01\\
61.71	0.01\\
61.72	0.01\\
61.73	0.01\\
61.74	0.01\\
61.75	0.01\\
61.76	0.01\\
61.77	0.01\\
61.78	0.01\\
61.79	0.01\\
61.8	0.01\\
61.81	0.01\\
61.82	0.01\\
61.83	0.01\\
61.84	0.01\\
61.85	0.01\\
61.86	0.01\\
61.87	0.01\\
61.88	0.01\\
61.89	0.01\\
61.9	0.01\\
61.91	0.01\\
61.92	0.01\\
61.93	0.01\\
61.94	0.01\\
61.95	0.01\\
61.96	0.01\\
61.97	0.01\\
61.98	0.01\\
61.99	0.01\\
62	0.01\\
62.01	0.01\\
62.02	0.01\\
62.03	0.01\\
62.04	0.01\\
62.05	0.01\\
62.06	0.01\\
62.07	0.01\\
62.08	0.01\\
62.09	0.01\\
62.1	0.01\\
62.11	0.01\\
62.12	0.01\\
62.13	0.01\\
62.14	0.01\\
62.15	0.01\\
62.16	0.01\\
62.17	0.01\\
62.18	0.01\\
62.19	0.01\\
62.2	0.01\\
62.21	0.01\\
62.22	0.01\\
62.23	0.01\\
62.24	0.01\\
62.25	0.01\\
62.26	0.01\\
62.27	0.01\\
62.28	0.01\\
62.29	0.01\\
62.3	0.01\\
62.31	0.01\\
62.32	0.01\\
62.33	0.01\\
62.34	0.01\\
62.35	0.01\\
62.36	0.01\\
62.37	0.01\\
62.38	0.01\\
62.39	0.01\\
62.4	0.01\\
62.41	0.01\\
62.42	0.01\\
62.43	0.01\\
62.44	0.01\\
62.45	0.01\\
62.46	0.01\\
62.47	0.01\\
62.48	0.01\\
62.49	0.01\\
62.5	0.01\\
62.51	0.01\\
62.52	0.01\\
62.53	0.01\\
62.54	0.01\\
62.55	0.01\\
62.56	0.01\\
62.57	0.01\\
62.58	0.01\\
62.59	0.01\\
62.6	0.01\\
62.61	0.01\\
62.62	0.01\\
62.63	0.01\\
62.64	0.01\\
62.65	0.01\\
62.66	0.01\\
62.67	0.01\\
62.68	0.01\\
62.69	0.01\\
62.7	0.01\\
62.71	0.01\\
62.72	0.01\\
62.73	0.01\\
62.74	0.01\\
62.75	0.01\\
62.76	0.01\\
62.77	0.01\\
62.78	0.01\\
62.79	0.01\\
62.8	0.01\\
62.81	0.01\\
62.82	0.01\\
62.83	0.01\\
62.84	0.01\\
62.85	0.01\\
62.86	0.01\\
62.87	0.01\\
62.88	0.01\\
62.89	0.01\\
62.9	0.01\\
62.91	0.01\\
62.92	0.01\\
62.93	0.01\\
62.94	0.01\\
62.95	0.01\\
62.96	0.01\\
62.97	0.01\\
62.98	0.01\\
62.99	0.01\\
63	0.01\\
63.01	0.01\\
63.02	0.01\\
63.03	0.01\\
63.04	0.01\\
63.05	0.01\\
63.06	0.01\\
63.07	0.01\\
63.08	0.01\\
63.09	0.01\\
63.1	0.01\\
63.11	0.01\\
63.12	0.01\\
63.13	0.01\\
63.14	0.01\\
63.15	0.01\\
63.16	0.01\\
63.17	0.01\\
63.18	0.01\\
63.19	0.01\\
63.2	0.01\\
63.21	0.01\\
63.22	0.01\\
63.23	0.01\\
63.24	0.01\\
63.25	0.01\\
63.26	0.01\\
63.27	0.01\\
63.28	0.01\\
63.29	0.01\\
63.3	0.01\\
63.31	0.01\\
63.32	0.01\\
63.33	0.01\\
63.34	0.01\\
63.35	0.01\\
63.36	0.01\\
63.37	0.01\\
63.38	0.01\\
63.39	0.01\\
63.4	0.01\\
63.41	0.01\\
63.42	0.01\\
63.43	0.01\\
63.44	0.01\\
63.45	0.01\\
63.46	0.01\\
63.47	0.01\\
63.48	0.01\\
63.49	0.01\\
63.5	0.01\\
63.51	0.01\\
63.52	0.01\\
63.53	0.01\\
63.54	0.01\\
63.55	0.01\\
63.56	0.01\\
63.57	0.01\\
63.58	0.01\\
63.59	0.01\\
63.6	0.01\\
63.61	0.01\\
63.62	0.01\\
63.63	0.01\\
63.64	0.01\\
63.65	0.01\\
63.66	0.01\\
63.67	0.01\\
63.68	0.01\\
63.69	0.01\\
63.7	0.01\\
63.71	0.01\\
63.72	0.01\\
63.73	0.01\\
63.74	0.01\\
63.75	0.01\\
63.76	0.01\\
63.77	0.01\\
63.78	0.01\\
63.79	0.01\\
63.8	0.01\\
63.81	0.01\\
63.82	0.01\\
63.83	0.01\\
63.84	0.01\\
63.85	0.01\\
63.86	0.01\\
63.87	0.01\\
63.88	0.01\\
63.89	0.01\\
63.9	0.01\\
63.91	0.01\\
63.92	0.01\\
63.93	0.01\\
63.94	0.01\\
63.95	0.01\\
63.96	0.01\\
63.97	0.01\\
63.98	0.01\\
63.99	0.01\\
64	0.01\\
64.01	0.01\\
64.02	0.01\\
64.03	0.01\\
64.04	0.01\\
64.05	0.01\\
64.06	0.01\\
64.07	0.01\\
64.08	0.01\\
64.09	0.01\\
64.1	0.01\\
64.11	0.01\\
64.12	0.01\\
64.13	0.01\\
64.14	0.01\\
64.15	0.01\\
64.16	0.01\\
64.17	0.01\\
64.18	0.01\\
64.19	0.01\\
64.2	0.01\\
64.21	0.01\\
64.22	0.01\\
64.23	0.01\\
64.24	0.01\\
64.25	0.01\\
64.26	0.01\\
64.27	0.01\\
64.28	0.01\\
64.29	0.01\\
64.3	0.01\\
64.31	0.01\\
64.32	0.01\\
64.33	0.01\\
64.34	0.01\\
64.35	0.01\\
64.36	0.01\\
64.37	0.01\\
64.38	0.01\\
64.39	0.01\\
64.4	0.01\\
64.41	0.01\\
64.42	0.01\\
64.43	0.01\\
64.44	0.01\\
64.45	0.01\\
64.46	0.01\\
64.47	0.01\\
64.48	0.01\\
64.49	0.01\\
64.5	0.01\\
64.51	0.01\\
64.52	0.01\\
64.53	0.01\\
64.54	0.01\\
64.55	0.01\\
64.56	0.01\\
64.57	0.01\\
64.58	0.01\\
64.59	0.01\\
64.6	0.01\\
64.61	0.01\\
64.62	0.01\\
64.63	0.01\\
64.64	0.01\\
64.65	0.01\\
64.66	0.01\\
64.67	0.01\\
64.68	0.01\\
64.69	0.01\\
64.7	0.01\\
64.71	0.01\\
64.72	0.01\\
64.73	0.01\\
64.74	0.01\\
64.75	0.01\\
64.76	0.01\\
64.77	0.01\\
64.78	0.01\\
64.79	0.01\\
64.8	0.01\\
64.81	0.01\\
64.82	0.01\\
64.83	0.01\\
64.84	0.01\\
64.85	0.01\\
64.86	0.01\\
64.87	0.01\\
64.88	0.01\\
64.89	0.01\\
64.9	0.01\\
64.91	0.01\\
64.92	0.01\\
64.93	0.01\\
64.94	0.01\\
64.95	0.01\\
64.96	0.01\\
64.97	0.01\\
64.98	0.01\\
64.99	0.01\\
65	0.01\\
65.01	0.01\\
65.02	0.01\\
65.03	0.01\\
65.04	0.01\\
65.05	0.01\\
65.06	0.01\\
65.07	0.01\\
65.08	0.01\\
65.09	0.01\\
65.1	0.01\\
65.11	0.01\\
65.12	0.01\\
65.13	0.01\\
65.14	0.01\\
65.15	0.01\\
65.16	0.01\\
65.17	0.01\\
65.18	0.01\\
65.19	0.01\\
65.2	0.01\\
65.21	0.01\\
65.22	0.01\\
65.23	0.01\\
65.24	0.01\\
65.25	0.01\\
65.26	0.01\\
65.27	0.01\\
65.28	0.01\\
65.29	0.01\\
65.3	0.01\\
65.31	0.01\\
65.32	0.01\\
65.33	0.01\\
65.34	0.01\\
65.35	0.01\\
65.36	0.01\\
65.37	0.01\\
65.38	0.01\\
65.39	0.01\\
65.4	0.01\\
65.41	0.01\\
65.42	0.01\\
65.43	0.01\\
65.44	0.01\\
65.45	0.01\\
65.46	0.01\\
65.47	0.01\\
65.48	0.01\\
65.49	0.01\\
65.5	0.01\\
65.51	0.01\\
65.52	0.01\\
65.53	0.01\\
65.54	0.01\\
65.55	0.01\\
65.56	0.01\\
65.57	0.01\\
65.58	0.01\\
65.59	0.01\\
65.6	0.01\\
65.61	0.01\\
65.62	0.01\\
65.63	0.01\\
65.64	0.01\\
65.65	0.01\\
65.66	0.01\\
65.67	0.01\\
65.68	0.01\\
65.69	0.01\\
65.7	0.01\\
65.71	0.01\\
65.72	0.01\\
65.73	0.01\\
65.74	0.01\\
65.75	0.01\\
65.76	0.01\\
65.77	0.01\\
65.78	0.01\\
65.79	0.01\\
65.8	0.01\\
65.81	0.01\\
65.82	0.01\\
65.83	0.01\\
65.84	0.01\\
65.85	0.01\\
65.86	0.01\\
65.87	0.01\\
65.88	0.01\\
65.89	0.01\\
65.9	0.01\\
65.91	0.01\\
65.92	0.01\\
65.93	0.01\\
65.94	0.01\\
65.95	0.01\\
65.96	0.01\\
65.97	0.01\\
65.98	0.01\\
65.99	0.01\\
66	0.01\\
66.01	0.01\\
66.02	0.01\\
66.03	0.01\\
66.04	0.01\\
66.05	0.01\\
66.06	0.01\\
66.07	0.01\\
66.08	0.01\\
66.09	0.01\\
66.1	0.01\\
66.11	0.01\\
66.12	0.01\\
66.13	0.01\\
66.14	0.01\\
66.15	0.01\\
66.16	0.01\\
66.17	0.01\\
66.18	0.01\\
66.19	0.01\\
66.2	0.01\\
66.21	0.01\\
66.22	0.01\\
66.23	0.01\\
66.24	0.01\\
66.25	0.01\\
66.26	0.01\\
66.27	0.01\\
66.28	0.01\\
66.29	0.01\\
66.3	0.01\\
66.31	0.01\\
66.32	0.01\\
66.33	0.01\\
66.34	0.01\\
66.35	0.01\\
66.36	0.01\\
66.37	0.01\\
66.38	0.01\\
66.39	0.01\\
66.4	0.01\\
66.41	0.01\\
66.42	0.01\\
66.43	0.01\\
66.44	0.01\\
66.45	0.01\\
66.46	0.01\\
66.47	0.01\\
66.48	0.01\\
66.49	0.01\\
66.5	0.01\\
66.51	0.01\\
66.52	0.01\\
66.53	0.01\\
66.54	0.01\\
66.55	0.01\\
66.56	0.01\\
66.57	0.01\\
66.58	0.01\\
66.59	0.01\\
66.6	0.01\\
66.61	0.01\\
66.62	0.01\\
66.63	0.01\\
66.64	0.01\\
66.65	0.01\\
66.66	0.01\\
66.67	0.01\\
66.68	0.01\\
66.69	0.01\\
66.7	0.01\\
66.71	0.01\\
66.72	0.01\\
66.73	0.01\\
66.74	0.01\\
66.75	0.01\\
66.76	0.01\\
66.77	0.01\\
66.78	0.01\\
66.79	0.01\\
66.8	0.01\\
66.81	0.01\\
66.82	0.01\\
66.83	0.01\\
66.84	0.01\\
66.85	0.01\\
66.86	0.01\\
66.87	0.01\\
66.88	0.01\\
66.89	0.01\\
66.9	0.01\\
66.91	0.01\\
66.92	0.01\\
66.93	0.01\\
66.94	0.01\\
66.95	0.01\\
66.96	0.01\\
66.97	0.01\\
66.98	0.01\\
66.99	0.01\\
67	0.01\\
67.01	0.01\\
67.02	0.01\\
67.03	0.01\\
67.04	0.01\\
67.05	0.01\\
67.06	0.01\\
67.07	0.01\\
67.08	0.01\\
67.09	0.01\\
67.1	0.01\\
67.11	0.01\\
67.12	0.01\\
67.13	0.01\\
67.14	0.01\\
67.15	0.01\\
67.16	0.01\\
67.17	0.01\\
67.18	0.01\\
67.19	0.01\\
67.2	0.01\\
67.21	0.01\\
67.22	0.01\\
67.23	0.01\\
67.24	0.01\\
67.25	0.01\\
67.26	0.01\\
67.27	0.01\\
67.28	0.01\\
67.29	0.01\\
67.3	0.01\\
67.31	0.01\\
67.32	0.01\\
67.33	0.01\\
67.34	0.01\\
67.35	0.01\\
67.36	0.01\\
67.37	0.01\\
67.38	0.01\\
67.39	0.01\\
67.4	0.01\\
67.41	0.01\\
67.42	0.01\\
67.43	0.01\\
67.44	0.01\\
67.45	0.01\\
67.46	0.01\\
67.47	0.01\\
67.48	0.01\\
67.49	0.01\\
67.5	0.01\\
67.51	0.01\\
67.52	0.01\\
67.53	0.01\\
67.54	0.01\\
67.55	0.01\\
67.56	0.01\\
67.57	0.01\\
67.58	0.01\\
67.59	0.01\\
67.6	0.01\\
67.61	0.01\\
67.62	0.01\\
67.63	0.01\\
67.64	0.01\\
67.65	0.01\\
67.66	0.01\\
67.67	0.01\\
67.68	0.01\\
67.69	0.01\\
67.7	0.01\\
67.71	0.01\\
67.72	0.01\\
67.73	0.01\\
67.74	0.01\\
67.75	0.01\\
67.76	0.01\\
67.77	0.01\\
67.78	0.01\\
67.79	0.01\\
67.8	0.01\\
67.81	0.01\\
67.82	0.01\\
67.83	0.01\\
67.84	0.01\\
67.85	0.01\\
67.86	0.01\\
67.87	0.01\\
67.88	0.01\\
67.89	0.01\\
67.9	0.01\\
67.91	0.01\\
67.92	0.01\\
67.93	0.01\\
67.94	0.01\\
67.95	0.01\\
67.96	0.01\\
67.97	0.01\\
67.98	0.01\\
67.99	0.01\\
68	0.01\\
68.01	0.01\\
68.02	0.01\\
68.03	0.01\\
68.04	0.01\\
68.05	0.01\\
68.06	0.01\\
68.07	0.01\\
68.08	0.01\\
68.09	0.01\\
68.1	0.01\\
68.11	0.01\\
68.12	0.01\\
68.13	0.01\\
68.14	0.01\\
68.15	0.01\\
68.16	0.01\\
68.17	0.01\\
68.18	0.01\\
68.19	0.01\\
68.2	0.01\\
68.21	0.01\\
68.22	0.01\\
68.23	0.01\\
68.24	0.01\\
68.25	0.01\\
68.26	0.01\\
68.27	0.01\\
68.28	0.01\\
68.29	0.01\\
68.3	0.01\\
68.31	0.01\\
68.32	0.01\\
68.33	0.01\\
68.34	0.01\\
68.35	0.01\\
68.36	0.01\\
68.37	0.01\\
68.38	0.01\\
68.39	0.01\\
68.4	0.01\\
68.41	0.01\\
68.42	0.01\\
68.43	0.01\\
68.44	0.01\\
68.45	0.01\\
68.46	0.01\\
68.47	0.01\\
68.48	0.01\\
68.49	0.01\\
68.5	0.01\\
68.51	0.01\\
68.52	0.01\\
68.53	0.01\\
68.54	0.01\\
68.55	0.01\\
68.56	0.01\\
68.57	0.01\\
68.58	0.01\\
68.59	0.01\\
68.6	0.01\\
68.61	0.01\\
68.62	0.01\\
68.63	0.01\\
68.64	0.01\\
68.65	0.01\\
68.66	0.01\\
68.67	0.01\\
68.68	0.01\\
68.69	0.01\\
68.7	0.01\\
68.71	0.01\\
68.72	0.01\\
68.73	0.01\\
68.74	0.01\\
68.75	0.01\\
68.76	0.01\\
68.77	0.01\\
68.78	0.01\\
68.79	0.01\\
68.8	0.01\\
68.81	0.01\\
68.82	0.01\\
68.83	0.01\\
68.84	0.01\\
68.85	0.01\\
68.86	0.01\\
68.87	0.01\\
68.88	0.01\\
68.89	0.01\\
68.9	0.01\\
68.91	0.01\\
68.92	0.01\\
68.93	0.01\\
68.94	0.01\\
68.95	0.01\\
68.96	0.01\\
68.97	0.01\\
68.98	0.01\\
68.99	0.01\\
69	0.01\\
69.01	0.01\\
69.02	0.01\\
69.03	0.01\\
69.04	0.01\\
69.05	0.01\\
69.06	0.01\\
69.07	0.01\\
69.08	0.01\\
69.09	0.01\\
69.1	0.01\\
69.11	0.01\\
69.12	0.01\\
69.13	0.01\\
69.14	0.01\\
69.15	0.01\\
69.16	0.01\\
69.17	0.01\\
69.18	0.01\\
69.19	0.01\\
69.2	0.01\\
69.21	0.01\\
69.22	0.01\\
69.23	0.01\\
69.24	0.01\\
69.25	0.01\\
69.26	0.01\\
69.27	0.01\\
69.28	0.01\\
69.29	0.01\\
69.3	0.01\\
69.31	0.01\\
69.32	0.01\\
69.33	0.01\\
69.34	0.01\\
69.35	0.01\\
69.36	0.01\\
69.37	0.01\\
69.38	0.01\\
69.39	0.01\\
69.4	0.01\\
69.41	0.01\\
69.42	0.01\\
69.43	0.01\\
69.44	0.01\\
69.45	0.01\\
69.46	0.01\\
69.47	0.01\\
69.48	0.01\\
69.49	0.01\\
69.5	0.01\\
69.51	0.01\\
69.52	0.01\\
69.53	0.01\\
69.54	0.01\\
69.55	0.01\\
69.56	0.01\\
69.57	0.01\\
69.58	0.01\\
69.59	0.01\\
69.6	0.01\\
69.61	0.01\\
69.62	0.01\\
69.63	0.01\\
69.64	0.01\\
69.65	0.01\\
69.66	0.01\\
69.67	0.01\\
69.68	0.01\\
69.69	0.01\\
69.7	0.01\\
69.71	0.01\\
69.72	0.01\\
69.73	0.01\\
69.74	0.01\\
69.75	0.01\\
69.76	0.01\\
69.77	0.01\\
69.78	0.01\\
69.79	0.01\\
69.8	0.01\\
69.81	0.01\\
69.82	0.01\\
69.83	0.01\\
69.84	0.01\\
69.85	0.01\\
69.86	0.01\\
69.87	0.01\\
69.88	0.01\\
69.89	0.01\\
69.9	0.01\\
69.91	0.01\\
69.92	0.01\\
69.93	0.01\\
69.94	0.01\\
69.95	0.01\\
69.96	0.01\\
69.97	0.01\\
69.98	0.01\\
69.99	0.01\\
70	0.01\\
70.01	0.01\\
70.02	0.01\\
70.03	0.01\\
70.04	0.01\\
70.05	0.01\\
70.06	0.01\\
70.07	0.01\\
70.08	0.01\\
70.09	0.01\\
70.1	0.01\\
70.11	0.01\\
70.12	0.01\\
70.13	0.01\\
70.14	0.01\\
70.15	0.01\\
70.16	0.01\\
70.17	0.01\\
70.18	0.01\\
70.19	0.01\\
70.2	0.01\\
70.21	0.01\\
70.22	0.01\\
70.23	0.01\\
70.24	0.01\\
70.25	0.01\\
70.26	0.01\\
70.27	0.01\\
70.28	0.01\\
70.29	0.01\\
70.3	0.01\\
70.31	0.01\\
70.32	0.01\\
70.33	0.01\\
70.34	0.01\\
70.35	0.01\\
70.36	0.01\\
70.37	0.01\\
70.38	0.01\\
70.39	0.01\\
70.4	0.01\\
70.41	0.01\\
70.42	0.01\\
70.43	0.01\\
70.44	0.01\\
70.45	0.01\\
70.46	0.01\\
70.47	0.01\\
70.48	0.01\\
70.49	0.01\\
70.5	0.01\\
70.51	0.01\\
70.52	0.01\\
70.53	0.01\\
70.54	0.01\\
70.55	0.01\\
70.56	0.01\\
70.57	0.01\\
70.58	0.01\\
70.59	0.01\\
70.6	0.01\\
70.61	0.01\\
70.62	0.01\\
70.63	0.01\\
70.64	0.01\\
70.65	0.01\\
70.66	0.01\\
70.67	0.01\\
70.68	0.01\\
70.69	0.01\\
70.7	0.01\\
70.71	0.01\\
70.72	0.01\\
70.73	0.01\\
70.74	0.01\\
70.75	0.01\\
70.76	0.01\\
70.77	0.01\\
70.78	0.01\\
70.79	0.01\\
70.8	0.01\\
70.81	0.01\\
70.82	0.01\\
70.83	0.01\\
70.84	0.01\\
70.85	0.01\\
70.86	0.01\\
70.87	0.01\\
70.88	0.01\\
70.89	0.01\\
70.9	0.01\\
70.91	0.01\\
70.92	0.01\\
70.93	0.01\\
70.94	0.01\\
70.95	0.01\\
70.96	0.01\\
70.97	0.01\\
70.98	0.01\\
70.99	0.01\\
71	0.01\\
71.01	0.01\\
71.02	0.01\\
71.03	0.01\\
71.04	0.01\\
71.05	0.01\\
71.06	0.01\\
71.07	0.01\\
71.08	0.01\\
71.09	0.01\\
71.1	0.01\\
71.11	0.01\\
71.12	0.01\\
71.13	0.01\\
71.14	0.01\\
71.15	0.01\\
71.16	0.01\\
71.17	0.01\\
71.18	0.01\\
71.19	0.01\\
71.2	0.01\\
71.21	0.01\\
71.22	0.01\\
71.23	0.01\\
71.24	0.01\\
71.25	0.01\\
71.26	0.01\\
71.27	0.01\\
71.28	0.01\\
71.29	0.01\\
71.3	0.01\\
71.31	0.01\\
71.32	0.01\\
71.33	0.01\\
71.34	0.01\\
71.35	0.01\\
71.36	0.01\\
71.37	0.01\\
71.38	0.01\\
71.39	0.01\\
71.4	0.01\\
71.41	0.01\\
71.42	0.01\\
71.43	0.01\\
71.44	0.01\\
71.45	0.01\\
71.46	0.01\\
71.47	0.01\\
71.48	0.01\\
71.49	0.01\\
71.5	0.01\\
71.51	0.01\\
71.52	0.01\\
71.53	0.01\\
71.54	0.01\\
71.55	0.01\\
71.56	0.01\\
71.57	0.01\\
71.58	0.01\\
71.59	0.01\\
71.6	0.01\\
71.61	0.01\\
71.62	0.01\\
71.63	0.01\\
71.64	0.01\\
71.65	0.01\\
71.66	0.01\\
71.67	0.01\\
71.68	0.01\\
71.69	0.01\\
71.7	0.01\\
71.71	0.01\\
71.72	0.01\\
71.73	0.01\\
71.74	0.01\\
71.75	0.01\\
71.76	0.01\\
71.77	0.01\\
71.78	0.01\\
71.79	0.01\\
71.8	0.01\\
71.81	0.01\\
71.82	0.01\\
71.83	0.01\\
71.84	0.01\\
71.85	0.01\\
71.86	0.01\\
71.87	0.01\\
71.88	0.01\\
71.89	0.01\\
71.9	0.01\\
71.91	0.01\\
71.92	0.01\\
71.93	0.01\\
71.94	0.01\\
71.95	0.01\\
71.96	0.01\\
71.97	0.01\\
71.98	0.01\\
71.99	0.01\\
72	0.01\\
72.01	0.01\\
72.02	0.01\\
72.03	0.01\\
72.04	0.01\\
72.05	0.01\\
72.06	0.01\\
72.07	0.01\\
72.08	0.01\\
72.09	0.01\\
72.1	0.01\\
72.11	0.01\\
72.12	0.01\\
72.13	0.01\\
72.14	0.01\\
72.15	0.01\\
72.16	0.01\\
72.17	0.01\\
72.18	0.01\\
72.19	0.01\\
72.2	0.01\\
72.21	0.01\\
72.22	0.01\\
72.23	0.01\\
72.24	0.01\\
72.25	0.01\\
72.26	0.01\\
72.27	0.01\\
72.28	0.01\\
72.29	0.01\\
72.3	0.01\\
72.31	0.01\\
72.32	0.01\\
72.33	0.01\\
72.34	0.01\\
72.35	0.01\\
72.36	0.01\\
72.37	0.01\\
72.38	0.01\\
72.39	0.01\\
72.4	0.01\\
72.41	0.01\\
72.42	0.01\\
72.43	0.01\\
72.44	0.01\\
72.45	0.01\\
72.46	0.01\\
72.47	0.01\\
72.48	0.01\\
72.49	0.01\\
72.5	0.01\\
72.51	0.01\\
72.52	0.01\\
72.53	0.01\\
72.54	0.01\\
72.55	0.01\\
72.56	0.01\\
72.57	0.01\\
72.58	0.01\\
72.59	0.01\\
72.6	0.01\\
72.61	0.01\\
72.62	0.01\\
72.63	0.01\\
72.64	0.01\\
72.65	0.01\\
72.66	0.01\\
72.67	0.01\\
72.68	0.01\\
72.69	0.01\\
72.7	0.01\\
72.71	0.01\\
72.72	0.01\\
72.73	0.01\\
72.74	0.01\\
72.75	0.01\\
72.76	0.01\\
72.77	0.01\\
72.78	0.01\\
72.79	0.01\\
72.8	0.01\\
72.81	0.01\\
72.82	0.01\\
72.83	0.01\\
72.84	0.01\\
72.85	0.01\\
72.86	0.01\\
72.87	0.01\\
72.88	0.01\\
72.89	0.01\\
72.9	0.01\\
72.91	0.01\\
72.92	0.01\\
72.93	0.01\\
72.94	0.01\\
72.95	0.01\\
72.96	0.01\\
72.97	0.01\\
72.98	0.01\\
72.99	0.01\\
73	0.01\\
73.01	0.01\\
73.02	0.01\\
73.03	0.01\\
73.04	0.01\\
73.05	0.01\\
73.06	0.01\\
73.07	0.01\\
73.08	0.01\\
73.09	0.01\\
73.1	0.01\\
73.11	0.01\\
73.12	0.01\\
73.13	0.01\\
73.14	0.01\\
73.15	0.01\\
73.16	0.01\\
73.17	0.01\\
73.18	0.01\\
73.19	0.01\\
73.2	0.01\\
73.21	0.01\\
73.22	0.01\\
73.23	0.01\\
73.24	0.01\\
73.25	0.01\\
73.26	0.01\\
73.27	0.01\\
73.28	0.01\\
73.29	0.01\\
73.3	0.01\\
73.31	0.01\\
73.32	0.01\\
73.33	0.01\\
73.34	0.01\\
73.35	0.01\\
73.36	0.01\\
73.37	0.01\\
73.38	0.01\\
73.39	0.01\\
73.4	0.01\\
73.41	0.01\\
73.42	0.01\\
73.43	0.01\\
73.44	0.01\\
73.45	0.01\\
73.46	0.01\\
73.47	0.01\\
73.48	0.01\\
73.49	0.01\\
73.5	0.01\\
73.51	0.01\\
73.52	0.01\\
73.53	0.01\\
73.54	0.01\\
73.55	0.01\\
73.56	0.01\\
73.57	0.01\\
73.58	0.01\\
73.59	0.01\\
73.6	0.01\\
73.61	0.01\\
73.62	0.01\\
73.63	0.01\\
73.64	0.01\\
73.65	0.01\\
73.66	0.01\\
73.67	0.01\\
73.68	0.01\\
73.69	0.01\\
73.7	0.01\\
73.71	0.01\\
73.72	0.01\\
73.73	0.01\\
73.74	0.01\\
73.75	0.01\\
73.76	0.01\\
73.77	0.01\\
73.78	0.01\\
73.79	0.01\\
73.8	0.01\\
73.81	0.01\\
73.82	0.01\\
73.83	0.01\\
73.84	0.01\\
73.85	0.01\\
73.86	0.01\\
73.87	0.01\\
73.88	0.01\\
73.89	0.01\\
73.9	0.01\\
73.91	0.01\\
73.92	0.01\\
73.93	0.01\\
73.94	0.01\\
73.95	0.01\\
73.96	0.01\\
73.97	0.01\\
73.98	0.01\\
73.99	0.01\\
74	0.01\\
74.01	0.01\\
74.02	0.01\\
74.03	0.01\\
74.04	0.01\\
74.05	0.01\\
74.06	0.01\\
74.07	0.01\\
74.08	0.01\\
74.09	0.01\\
74.1	0.01\\
74.11	0.01\\
74.12	0.01\\
74.13	0.01\\
74.14	0.01\\
74.15	0.01\\
74.16	0.01\\
74.17	0.01\\
74.18	0.01\\
74.19	0.01\\
74.2	0.01\\
74.21	0.01\\
74.22	0.01\\
74.23	0.01\\
74.24	0.01\\
74.25	0.01\\
74.26	0.01\\
74.27	0.01\\
74.28	0.01\\
74.29	0.01\\
74.3	0.01\\
74.31	0.01\\
74.32	0.01\\
74.33	0.01\\
74.34	0.01\\
74.35	0.01\\
74.36	0.01\\
74.37	0.01\\
74.38	0.01\\
74.39	0.01\\
74.4	0.01\\
74.41	0.01\\
74.42	0.01\\
74.43	0.01\\
74.44	0.01\\
74.45	0.01\\
74.46	0.01\\
74.47	0.01\\
74.48	0.01\\
74.49	0.01\\
74.5	0.01\\
74.51	0.01\\
74.52	0.01\\
74.53	0.01\\
74.54	0.01\\
74.55	0.01\\
74.56	0.01\\
74.57	0.01\\
74.58	0.01\\
74.59	0.01\\
74.6	0.01\\
74.61	0.01\\
74.62	0.01\\
74.63	0.01\\
74.64	0.01\\
74.65	0.01\\
74.66	0.01\\
74.67	0.01\\
74.68	0.01\\
74.69	0.01\\
74.7	0.01\\
74.71	0.01\\
74.72	0.01\\
74.73	0.01\\
74.74	0.01\\
74.75	0.01\\
74.76	0.01\\
74.77	0.01\\
74.78	0.01\\
74.79	0.01\\
74.8	0.01\\
74.81	0.01\\
74.82	0.01\\
74.83	0.01\\
74.84	0.01\\
74.85	0.01\\
74.86	0.01\\
74.87	0.01\\
74.88	0.01\\
74.89	0.01\\
74.9	0.01\\
74.91	0.01\\
74.92	0.01\\
74.93	0.01\\
74.94	0.01\\
74.95	0.01\\
74.96	0.01\\
74.97	0.01\\
74.98	0.01\\
74.99	0.01\\
75	0.01\\
75.01	0.01\\
75.02	0.01\\
75.03	0.01\\
75.04	0.01\\
75.05	0.01\\
75.06	0.01\\
75.07	0.01\\
75.08	0.01\\
75.09	0.01\\
75.1	0.01\\
75.11	0.01\\
75.12	0.01\\
75.13	0.01\\
75.14	0.01\\
75.15	0.01\\
75.16	0.01\\
75.17	0.01\\
75.18	0.01\\
75.19	0.01\\
75.2	0.01\\
75.21	0.01\\
75.22	0.01\\
75.23	0.01\\
75.24	0.01\\
75.25	0.01\\
75.26	0.01\\
75.27	0.01\\
75.28	0.01\\
75.29	0.01\\
75.3	0.01\\
75.31	0.01\\
75.32	0.01\\
75.33	0.01\\
75.34	0.01\\
75.35	0.01\\
75.36	0.01\\
75.37	0.01\\
75.38	0.01\\
75.39	0.01\\
75.4	0.01\\
75.41	0.01\\
75.42	0.01\\
75.43	0.01\\
75.44	0.01\\
75.45	0.01\\
75.46	0.01\\
75.47	0.01\\
75.48	0.01\\
75.49	0.01\\
75.5	0.01\\
75.51	0.01\\
75.52	0.01\\
75.53	0.01\\
75.54	0.01\\
75.55	0.01\\
75.56	0.01\\
75.57	0.01\\
75.58	0.01\\
75.59	0.01\\
75.6	0.01\\
75.61	0.01\\
75.62	0.01\\
75.63	0.01\\
75.64	0.01\\
75.65	0.01\\
75.66	0.01\\
75.67	0.01\\
75.68	0.01\\
75.69	0.01\\
75.7	0.01\\
75.71	0.01\\
75.72	0.01\\
75.73	0.01\\
75.74	0.01\\
75.75	0.01\\
75.76	0.01\\
75.77	0.01\\
75.78	0.01\\
75.79	0.01\\
75.8	0.01\\
75.81	0.01\\
75.82	0.01\\
75.83	0.01\\
75.84	0.01\\
75.85	0.01\\
75.86	0.01\\
75.87	0.01\\
75.88	0.01\\
75.89	0.01\\
75.9	0.01\\
75.91	0.01\\
75.92	0.01\\
75.93	0.01\\
75.94	0.01\\
75.95	0.01\\
75.96	0.01\\
75.97	0.01\\
75.98	0.01\\
75.99	0.01\\
76	0.01\\
76.01	0.01\\
76.02	0.01\\
76.03	0.01\\
76.04	0.01\\
76.05	0.01\\
76.06	0.01\\
76.07	0.01\\
76.08	0.01\\
76.09	0.01\\
76.1	0.01\\
76.11	0.01\\
76.12	0.01\\
76.13	0.01\\
76.14	0.01\\
76.15	0.01\\
76.16	0.01\\
76.17	0.01\\
76.18	0.01\\
76.19	0.01\\
76.2	0.01\\
76.21	0.01\\
76.22	0.01\\
76.23	0.01\\
76.24	0.01\\
76.25	0.01\\
76.26	0.01\\
76.27	0.01\\
76.28	0.01\\
76.29	0.01\\
76.3	0.01\\
76.31	0.01\\
76.32	0.01\\
76.33	0.01\\
76.34	0.01\\
76.35	0.01\\
76.36	0.01\\
76.37	0.01\\
76.38	0.01\\
76.39	0.01\\
76.4	0.01\\
76.41	0.01\\
76.42	0.01\\
76.43	0.01\\
76.44	0.01\\
76.45	0.01\\
76.46	0.01\\
76.47	0.01\\
76.48	0.01\\
76.49	0.01\\
76.5	0.01\\
76.51	0.01\\
76.52	0.01\\
76.53	0.01\\
76.54	0.01\\
76.55	0.01\\
76.56	0.01\\
76.57	0.01\\
76.58	0.01\\
76.59	0.01\\
76.6	0.01\\
76.61	0.01\\
76.62	0.01\\
76.63	0.01\\
76.64	0.01\\
76.65	0.01\\
76.66	0.01\\
76.67	0.01\\
76.68	0.01\\
76.69	0.01\\
76.7	0.01\\
76.71	0.01\\
76.72	0.01\\
76.73	0.01\\
76.74	0.01\\
76.75	0.01\\
76.76	0.01\\
76.77	0.01\\
76.78	0.01\\
76.79	0.01\\
76.8	0.01\\
76.81	0.01\\
76.82	0.01\\
76.83	0.01\\
76.84	0.01\\
76.85	0.01\\
76.86	0.01\\
76.87	0.01\\
76.88	0.01\\
76.89	0.01\\
76.9	0.01\\
76.91	0.01\\
76.92	0.01\\
76.93	0.01\\
76.94	0.01\\
76.95	0.01\\
76.96	0.01\\
76.97	0.01\\
76.98	0.01\\
76.99	0.01\\
77	0.01\\
77.01	0.01\\
77.02	0.01\\
77.03	0.01\\
77.04	0.01\\
77.05	0.01\\
77.06	0.01\\
77.07	0.01\\
77.08	0.01\\
77.09	0.01\\
77.1	0.01\\
77.11	0.01\\
77.12	0.01\\
77.13	0.01\\
77.14	0.01\\
77.15	0.01\\
77.16	0.01\\
77.17	0.01\\
77.18	0.01\\
77.19	0.01\\
77.2	0.01\\
77.21	0.01\\
77.22	0.01\\
77.23	0.01\\
77.24	0.01\\
77.25	0.01\\
77.26	0.01\\
77.27	0.01\\
77.28	0.01\\
77.29	0.01\\
77.3	0.01\\
77.31	0.01\\
77.32	0.01\\
77.33	0.01\\
77.34	0.01\\
77.35	0.01\\
77.36	0.01\\
77.37	0.01\\
77.38	0.01\\
77.39	0.01\\
77.4	0.01\\
77.41	0.01\\
77.42	0.01\\
77.43	0.01\\
77.44	0.01\\
77.45	0.01\\
77.46	0.01\\
77.47	0.01\\
77.48	0.01\\
77.49	0.01\\
77.5	0.01\\
77.51	0.01\\
77.52	0.01\\
77.53	0.01\\
77.54	0.01\\
77.55	0.01\\
77.56	0.01\\
77.57	0.01\\
77.58	0.01\\
77.59	0.01\\
77.6	0.01\\
77.61	0.01\\
77.62	0.01\\
77.63	0.01\\
77.64	0.01\\
77.65	0.01\\
77.66	0.01\\
77.67	0.01\\
77.68	0.01\\
77.69	0.01\\
77.7	0.01\\
77.71	0.01\\
77.72	0.01\\
77.73	0.01\\
77.74	0.01\\
77.75	0.01\\
77.76	0.01\\
77.77	0.01\\
77.78	0.01\\
77.79	0.01\\
77.8	0.01\\
77.81	0.01\\
77.82	0.01\\
77.83	0.01\\
77.84	0.01\\
77.85	0.01\\
77.86	0.01\\
77.87	0.01\\
77.88	0.01\\
77.89	0.01\\
77.9	0.01\\
77.91	0.01\\
77.92	0.01\\
77.93	0.01\\
77.94	0.01\\
77.95	0.01\\
77.96	0.01\\
77.97	0.01\\
77.98	0.01\\
77.99	0.01\\
78	0.01\\
78.01	0.01\\
78.02	0.01\\
78.03	0.01\\
78.04	0.01\\
78.05	0.01\\
78.06	0.01\\
78.07	0.01\\
78.08	0.01\\
78.09	0.01\\
78.1	0.01\\
78.11	0.01\\
78.12	0.01\\
78.13	0.01\\
78.14	0.01\\
78.15	0.01\\
78.16	0.01\\
78.17	0.01\\
78.18	0.01\\
78.19	0.01\\
78.2	0.01\\
78.21	0.01\\
78.22	0.01\\
78.23	0.01\\
78.24	0.01\\
78.25	0.01\\
78.26	0.01\\
78.27	0.01\\
78.28	0.01\\
78.29	0.01\\
78.3	0.01\\
78.31	0.01\\
78.32	0.01\\
78.33	0.01\\
78.34	0.01\\
78.35	0.01\\
78.36	0.01\\
78.37	0.01\\
78.38	0.01\\
78.39	0.01\\
78.4	0.01\\
78.41	0.01\\
78.42	0.01\\
78.43	0.01\\
78.44	0.01\\
78.45	0.01\\
78.46	0.01\\
78.47	0.01\\
78.48	0.01\\
78.49	0.01\\
78.5	0.01\\
78.51	0.01\\
78.52	0.01\\
78.53	0.01\\
78.54	0.01\\
78.55	0.01\\
78.56	0.01\\
78.57	0.01\\
78.58	0.01\\
78.59	0.01\\
78.6	0.01\\
78.61	0.01\\
78.62	0.01\\
78.63	0.01\\
78.64	0.01\\
78.65	0.01\\
78.66	0.01\\
78.67	0.01\\
78.68	0.01\\
78.69	0.01\\
78.7	0.01\\
78.71	0.01\\
78.72	0.01\\
78.73	0.01\\
78.74	0.01\\
78.75	0.01\\
78.76	0.01\\
78.77	0.01\\
78.78	0.01\\
78.79	0.01\\
78.8	0.01\\
78.81	0.01\\
78.82	0.01\\
78.83	0.01\\
78.84	0.01\\
78.85	0.01\\
78.86	0.01\\
78.87	0.01\\
78.88	0.01\\
78.89	0.01\\
78.9	0.01\\
78.91	0.01\\
78.92	0.01\\
78.93	0.01\\
78.94	0.01\\
78.95	0.01\\
78.96	0.01\\
78.97	0.01\\
78.98	0.01\\
78.99	0.01\\
79	0.01\\
79.01	0.01\\
79.02	0.01\\
79.03	0.01\\
79.04	0.01\\
79.05	0.01\\
79.06	0.01\\
79.07	0.01\\
79.08	0.01\\
79.09	0.01\\
79.1	0.01\\
79.11	0.01\\
79.12	0.01\\
79.13	0.01\\
79.14	0.01\\
79.15	0.01\\
79.16	0.01\\
79.17	0.01\\
79.18	0.01\\
79.19	0.01\\
79.2	0.01\\
79.21	0.01\\
79.22	0.01\\
79.23	0.01\\
79.24	0.01\\
79.25	0.01\\
79.26	0.01\\
79.27	0.01\\
79.28	0.01\\
79.29	0.01\\
79.3	0.01\\
79.31	0.01\\
79.32	0.01\\
79.33	0.01\\
79.34	0.01\\
79.35	0.01\\
79.36	0.01\\
79.37	0.01\\
79.38	0.01\\
79.39	0.01\\
79.4	0.01\\
79.41	0.01\\
79.42	0.01\\
79.43	0.01\\
79.44	0.01\\
79.45	0.01\\
79.46	0.01\\
79.47	0.01\\
79.48	0.01\\
79.49	0.01\\
79.5	0.01\\
79.51	0.01\\
79.52	0.01\\
79.53	0.01\\
79.54	0.01\\
79.55	0.01\\
79.56	0.01\\
79.57	0.01\\
79.58	0.01\\
79.59	0.01\\
79.6	0.01\\
79.61	0.01\\
79.62	0.01\\
79.63	0.01\\
79.64	0.01\\
79.65	0.01\\
79.66	0.01\\
79.67	0.01\\
79.68	0.01\\
79.69	0.01\\
79.7	0.01\\
79.71	0.01\\
79.72	0.01\\
79.73	0.01\\
79.74	0.01\\
79.75	0.01\\
79.76	0.01\\
79.77	0.01\\
79.78	0.01\\
79.79	0.01\\
79.8	0.01\\
79.81	0.01\\
79.82	0.01\\
79.83	0.01\\
79.84	0.01\\
79.85	0.01\\
79.86	0.01\\
79.87	0.01\\
79.88	0.01\\
79.89	0.01\\
79.9	0.01\\
79.91	0.01\\
79.92	0.01\\
79.93	0.01\\
79.94	0.01\\
79.95	0.01\\
79.96	0.01\\
79.97	0.01\\
79.98	0.01\\
79.99	0.01\\
80	0.01\\
80.01	0.01\\
};
\addplot [color=green,solid]
  table[row sep=crcr]{%
80.01	0.01\\
80.02	0.01\\
80.03	0.01\\
80.04	0.01\\
80.05	0.01\\
80.06	0.01\\
80.07	0.01\\
80.08	0.01\\
80.09	0.01\\
80.1	0.01\\
80.11	0.01\\
80.12	0.01\\
80.13	0.01\\
80.14	0.01\\
80.15	0.01\\
80.16	0.01\\
80.17	0.01\\
80.18	0.01\\
80.19	0.01\\
80.2	0.01\\
80.21	0.01\\
80.22	0.01\\
80.23	0.01\\
80.24	0.01\\
80.25	0.01\\
80.26	0.01\\
80.27	0.01\\
80.28	0.01\\
80.29	0.01\\
80.3	0.01\\
80.31	0.01\\
80.32	0.01\\
80.33	0.01\\
80.34	0.01\\
80.35	0.01\\
80.36	0.01\\
80.37	0.01\\
80.38	0.01\\
80.39	0.01\\
80.4	0.01\\
80.41	0.01\\
80.42	0.01\\
80.43	0.01\\
80.44	0.01\\
80.45	0.01\\
80.46	0.01\\
80.47	0.01\\
80.48	0.01\\
80.49	0.01\\
80.5	0.01\\
80.51	0.01\\
80.52	0.01\\
80.53	0.01\\
80.54	0.01\\
80.55	0.01\\
80.56	0.01\\
80.57	0.01\\
80.58	0.01\\
80.59	0.01\\
80.6	0.01\\
80.61	0.01\\
80.62	0.01\\
80.63	0.01\\
80.64	0.01\\
80.65	0.01\\
80.66	0.01\\
80.67	0.01\\
80.68	0.01\\
80.69	0.01\\
80.7	0.01\\
80.71	0.01\\
80.72	0.01\\
80.73	0.01\\
80.74	0.01\\
80.75	0.01\\
80.76	0.01\\
80.77	0.01\\
80.78	0.01\\
80.79	0.01\\
80.8	0.01\\
80.81	0.01\\
80.82	0.01\\
80.83	0.01\\
80.84	0.01\\
80.85	0.01\\
80.86	0.01\\
80.87	0.01\\
80.88	0.01\\
80.89	0.01\\
80.9	0.01\\
80.91	0.01\\
80.92	0.01\\
80.93	0.01\\
80.94	0.01\\
80.95	0.01\\
80.96	0.01\\
80.97	0.01\\
80.98	0.01\\
80.99	0.01\\
81	0.01\\
81.01	0.01\\
81.02	0.01\\
81.03	0.01\\
81.04	0.01\\
81.05	0.01\\
81.06	0.01\\
81.07	0.01\\
81.08	0.01\\
81.09	0.01\\
81.1	0.01\\
81.11	0.01\\
81.12	0.01\\
81.13	0.01\\
81.14	0.01\\
81.15	0.01\\
81.16	0.01\\
81.17	0.01\\
81.18	0.01\\
81.19	0.01\\
81.2	0.01\\
81.21	0.01\\
81.22	0.01\\
81.23	0.01\\
81.24	0.01\\
81.25	0.01\\
81.26	0.01\\
81.27	0.01\\
81.28	0.01\\
81.29	0.01\\
81.3	0.01\\
81.31	0.01\\
81.32	0.01\\
81.33	0.01\\
81.34	0.01\\
81.35	0.01\\
81.36	0.01\\
81.37	0.01\\
81.38	0.01\\
81.39	0.01\\
81.4	0.01\\
81.41	0.01\\
81.42	0.01\\
81.43	0.01\\
81.44	0.01\\
81.45	0.01\\
81.46	0.01\\
81.47	0.01\\
81.48	0.01\\
81.49	0.01\\
81.5	0.01\\
81.51	0.01\\
81.52	0.01\\
81.53	0.01\\
81.54	0.01\\
81.55	0.01\\
81.56	0.01\\
81.57	0.01\\
81.58	0.01\\
81.59	0.01\\
81.6	0.01\\
81.61	0.01\\
81.62	0.01\\
81.63	0.01\\
81.64	0.01\\
81.65	0.01\\
81.66	0.01\\
81.67	0.01\\
81.68	0.01\\
81.69	0.01\\
81.7	0.01\\
81.71	0.01\\
81.72	0.01\\
81.73	0.01\\
81.74	0.01\\
81.75	0.01\\
81.76	0.01\\
81.77	0.01\\
81.78	0.01\\
81.79	0.01\\
81.8	0.01\\
81.81	0.01\\
81.82	0.01\\
81.83	0.01\\
81.84	0.01\\
81.85	0.01\\
81.86	0.01\\
81.87	0.01\\
81.88	0.01\\
81.89	0.01\\
81.9	0.01\\
81.91	0.01\\
81.92	0.01\\
81.93	0.01\\
81.94	0.01\\
81.95	0.01\\
81.96	0.01\\
81.97	0.01\\
81.98	0.01\\
81.99	0.01\\
82	0.01\\
82.01	0.01\\
82.02	0.01\\
82.03	0.01\\
82.04	0.01\\
82.05	0.01\\
82.06	0.01\\
82.07	0.01\\
82.08	0.01\\
82.09	0.01\\
82.1	0.01\\
82.11	0.01\\
82.12	0.01\\
82.13	0.01\\
82.14	0.01\\
82.15	0.01\\
82.16	0.01\\
82.17	0.01\\
82.18	0.01\\
82.19	0.01\\
82.2	0.01\\
82.21	0.01\\
82.22	0.01\\
82.23	0.01\\
82.24	0.01\\
82.25	0.01\\
82.26	0.01\\
82.27	0.01\\
82.28	0.01\\
82.29	0.01\\
82.3	0.01\\
82.31	0.01\\
82.32	0.01\\
82.33	0.01\\
82.34	0.01\\
82.35	0.01\\
82.36	0.01\\
82.37	0.01\\
82.38	0.01\\
82.39	0.01\\
82.4	0.01\\
82.41	0.01\\
82.42	0.01\\
82.43	0.01\\
82.44	0.01\\
82.45	0.01\\
82.46	0.01\\
82.47	0.01\\
82.48	0.01\\
82.49	0.01\\
82.5	0.01\\
82.51	0.01\\
82.52	0.01\\
82.53	0.01\\
82.54	0.01\\
82.55	0.01\\
82.56	0.01\\
82.57	0.01\\
82.58	0.01\\
82.59	0.01\\
82.6	0.01\\
82.61	0.01\\
82.62	0.01\\
82.63	0.01\\
82.64	0.01\\
82.65	0.01\\
82.66	0.01\\
82.67	0.01\\
82.68	0.01\\
82.69	0.01\\
82.7	0.01\\
82.71	0.01\\
82.72	0.01\\
82.73	0.01\\
82.74	0.01\\
82.75	0.01\\
82.76	0.01\\
82.77	0.01\\
82.78	0.01\\
82.79	0.01\\
82.8	0.01\\
82.81	0.01\\
82.82	0.01\\
82.83	0.01\\
82.84	0.01\\
82.85	0.01\\
82.86	0.01\\
82.87	0.01\\
82.88	0.01\\
82.89	0.01\\
82.9	0.01\\
82.91	0.01\\
82.92	0.01\\
82.93	0.01\\
82.94	0.01\\
82.95	0.01\\
82.96	0.01\\
82.97	0.01\\
82.98	0.01\\
82.99	0.01\\
83	0.01\\
83.01	0.01\\
83.02	0.01\\
83.03	0.01\\
83.04	0.01\\
83.05	0.01\\
83.06	0.01\\
83.07	0.01\\
83.08	0.01\\
83.09	0.01\\
83.1	0.01\\
83.11	0.01\\
83.12	0.01\\
83.13	0.01\\
83.14	0.01\\
83.15	0.01\\
83.16	0.01\\
83.17	0.01\\
83.18	0.01\\
83.19	0.01\\
83.2	0.01\\
83.21	0.01\\
83.22	0.01\\
83.23	0.01\\
83.24	0.01\\
83.25	0.01\\
83.26	0.01\\
83.27	0.01\\
83.28	0.01\\
83.29	0.01\\
83.3	0.01\\
83.31	0.01\\
83.32	0.01\\
83.33	0.01\\
83.34	0.01\\
83.35	0.01\\
83.36	0.01\\
83.37	0.01\\
83.38	0.01\\
83.39	0.01\\
83.4	0.01\\
83.41	0.01\\
83.42	0.01\\
83.43	0.01\\
83.44	0.01\\
83.45	0.01\\
83.46	0.01\\
83.47	0.01\\
83.48	0.01\\
83.49	0.01\\
83.5	0.01\\
83.51	0.01\\
83.52	0.01\\
83.53	0.01\\
83.54	0.01\\
83.55	0.01\\
83.56	0.01\\
83.57	0.01\\
83.58	0.01\\
83.59	0.01\\
83.6	0.01\\
83.61	0.01\\
83.62	0.01\\
83.63	0.01\\
83.64	0.01\\
83.65	0.01\\
83.66	0.01\\
83.67	0.01\\
83.68	0.01\\
83.69	0.01\\
83.7	0.01\\
83.71	0.01\\
83.72	0.01\\
83.73	0.01\\
83.74	0.01\\
83.75	0.01\\
83.76	0.01\\
83.77	0.01\\
83.78	0.01\\
83.79	0.01\\
83.8	0.01\\
83.81	0.01\\
83.82	0.01\\
83.83	0.01\\
83.84	0.01\\
83.85	0.01\\
83.86	0.01\\
83.87	0.01\\
83.88	0.01\\
83.89	0.01\\
83.9	0.01\\
83.91	0.01\\
83.92	0.01\\
83.93	0.01\\
83.94	0.01\\
83.95	0.01\\
83.96	0.01\\
83.97	0.01\\
83.98	0.01\\
83.99	0.01\\
84	0.01\\
84.01	0.01\\
84.02	0.01\\
84.03	0.01\\
84.04	0.01\\
84.05	0.01\\
84.06	0.01\\
84.07	0.01\\
84.08	0.01\\
84.09	0.01\\
84.1	0.01\\
84.11	0.01\\
84.12	0.01\\
84.13	0.01\\
84.14	0.01\\
84.15	0.01\\
84.16	0.01\\
84.17	0.01\\
84.18	0.01\\
84.19	0.01\\
84.2	0.01\\
84.21	0.01\\
84.22	0.01\\
84.23	0.01\\
84.24	0.01\\
84.25	0.01\\
84.26	0.01\\
84.27	0.01\\
84.28	0.01\\
84.29	0.01\\
84.3	0.01\\
84.31	0.01\\
84.32	0.01\\
84.33	0.01\\
84.34	0.01\\
84.35	0.01\\
84.36	0.01\\
84.37	0.01\\
84.38	0.01\\
84.39	0.01\\
84.4	0.01\\
84.41	0.01\\
84.42	0.01\\
84.43	0.01\\
84.44	0.01\\
84.45	0.01\\
84.46	0.01\\
84.47	0.01\\
84.48	0.01\\
84.49	0.01\\
84.5	0.01\\
84.51	0.01\\
84.52	0.01\\
84.53	0.01\\
84.54	0.01\\
84.55	0.01\\
84.56	0.01\\
84.57	0.01\\
84.58	0.01\\
84.59	0.01\\
84.6	0.01\\
84.61	0.01\\
84.62	0.01\\
84.63	0.01\\
84.64	0.01\\
84.65	0.01\\
84.66	0.01\\
84.67	0.01\\
84.68	0.01\\
84.69	0.01\\
84.7	0.01\\
84.71	0.01\\
84.72	0.01\\
84.73	0.01\\
84.74	0.01\\
84.75	0.01\\
84.76	0.01\\
84.77	0.01\\
84.78	0.01\\
84.79	0.01\\
84.8	0.01\\
84.81	0.01\\
84.82	0.01\\
84.83	0.01\\
84.84	0.01\\
84.85	0.01\\
84.86	0.01\\
84.87	0.01\\
84.88	0.01\\
84.89	0.01\\
84.9	0.01\\
84.91	0.01\\
84.92	0.01\\
84.93	0.01\\
84.94	0.01\\
84.95	0.01\\
84.96	0.01\\
84.97	0.01\\
84.98	0.01\\
84.99	0.01\\
85	0.01\\
85.01	0.01\\
85.02	0.01\\
85.03	0.01\\
85.04	0.01\\
85.05	0.01\\
85.06	0.01\\
85.07	0.01\\
85.08	0.01\\
85.09	0.01\\
85.1	0.01\\
85.11	0.01\\
85.12	0.01\\
85.13	0.01\\
85.14	0.01\\
85.15	0.01\\
85.16	0.01\\
85.17	0.01\\
85.18	0.01\\
85.19	0.01\\
85.2	0.01\\
85.21	0.01\\
85.22	0.01\\
85.23	0.01\\
85.24	0.01\\
85.25	0.01\\
85.26	0.01\\
85.27	0.01\\
85.28	0.01\\
85.29	0.01\\
85.3	0.01\\
85.31	0.01\\
85.32	0.01\\
85.33	0.01\\
85.34	0.01\\
85.35	0.01\\
85.36	0.01\\
85.37	0.01\\
85.38	0.01\\
85.39	0.01\\
85.4	0.01\\
85.41	0.01\\
85.42	0.01\\
85.43	0.01\\
85.44	0.01\\
85.45	0.01\\
85.46	0.01\\
85.47	0.01\\
85.48	0.01\\
85.49	0.01\\
85.5	0.01\\
85.51	0.01\\
85.52	0.01\\
85.53	0.01\\
85.54	0.01\\
85.55	0.01\\
85.56	0.01\\
85.57	0.01\\
85.58	0.01\\
85.59	0.01\\
85.6	0.01\\
85.61	0.01\\
85.62	0.01\\
85.63	0.01\\
85.64	0.01\\
85.65	0.01\\
85.66	0.01\\
85.67	0.01\\
85.68	0.01\\
85.69	0.01\\
85.7	0.01\\
85.71	0.01\\
85.72	0.01\\
85.73	0.01\\
85.74	0.01\\
85.75	0.01\\
85.76	0.01\\
85.77	0.01\\
85.78	0.01\\
85.79	0.01\\
85.8	0.01\\
85.81	0.01\\
85.82	0.01\\
85.83	0.01\\
85.84	0.01\\
85.85	0.01\\
85.86	0.01\\
85.87	0.01\\
85.88	0.01\\
85.89	0.01\\
85.9	0.01\\
85.91	0.01\\
85.92	0.01\\
85.93	0.01\\
85.94	0.01\\
85.95	0.01\\
85.96	0.01\\
85.97	0.01\\
85.98	0.01\\
85.99	0.01\\
86	0.01\\
86.01	0.01\\
86.02	0.01\\
86.03	0.01\\
86.04	0.01\\
86.05	0.01\\
86.06	0.01\\
86.07	0.01\\
86.08	0.01\\
86.09	0.01\\
86.1	0.01\\
86.11	0.01\\
86.12	0.01\\
86.13	0.01\\
86.14	0.01\\
86.15	0.01\\
86.16	0.01\\
86.17	0.01\\
86.18	0.01\\
86.19	0.01\\
86.2	0.01\\
86.21	0.01\\
86.22	0.01\\
86.23	0.01\\
86.24	0.01\\
86.25	0.01\\
86.26	0.01\\
86.27	0.01\\
86.28	0.01\\
86.29	0.01\\
86.3	0.01\\
86.31	0.01\\
86.32	0.01\\
86.33	0.01\\
86.34	0.01\\
86.35	0.01\\
86.36	0.01\\
86.37	0.01\\
86.38	0.01\\
86.39	0.01\\
86.4	0.01\\
86.41	0.01\\
86.42	0.01\\
86.43	0.01\\
86.44	0.01\\
86.45	0.01\\
86.46	0.01\\
86.47	0.01\\
86.48	0.01\\
86.49	0.01\\
86.5	0.01\\
86.51	0.01\\
86.52	0.01\\
86.53	0.01\\
86.54	0.01\\
86.55	0.01\\
86.56	0.01\\
86.57	0.01\\
86.58	0.01\\
86.59	0.01\\
86.6	0.01\\
86.61	0.01\\
86.62	0.01\\
86.63	0.01\\
86.64	0.01\\
86.65	0.01\\
86.66	0.01\\
86.67	0.01\\
86.68	0.01\\
86.69	0.01\\
86.7	0.01\\
86.71	0.01\\
86.72	0.01\\
86.73	0.01\\
86.74	0.01\\
86.75	0.01\\
86.76	0.01\\
86.77	0.01\\
86.78	0.01\\
86.79	0.01\\
86.8	0.01\\
86.81	0.01\\
86.82	0.01\\
86.83	0.01\\
86.84	0.01\\
86.85	0.01\\
86.86	0.01\\
86.87	0.01\\
86.88	0.01\\
86.89	0.01\\
86.9	0.01\\
86.91	0.01\\
86.92	0.01\\
86.93	0.01\\
86.94	0.01\\
86.95	0.01\\
86.96	0.01\\
86.97	0.01\\
86.98	0.01\\
86.99	0.01\\
87	0.01\\
87.01	0.01\\
87.02	0.01\\
87.03	0.01\\
87.04	0.01\\
87.05	0.01\\
87.06	0.01\\
87.07	0.01\\
87.08	0.01\\
87.09	0.01\\
87.1	0.01\\
87.11	0.01\\
87.12	0.01\\
87.13	0.01\\
87.14	0.01\\
87.15	0.01\\
87.16	0.01\\
87.17	0.01\\
87.18	0.01\\
87.19	0.01\\
87.2	0.01\\
87.21	0.01\\
87.22	0.01\\
87.23	0.01\\
87.24	0.01\\
87.25	0.01\\
87.26	0.01\\
87.27	0.01\\
87.28	0.01\\
87.29	0.01\\
87.3	0.01\\
87.31	0.01\\
87.32	0.01\\
87.33	0.01\\
87.34	0.01\\
87.35	0.01\\
87.36	0.01\\
87.37	0.01\\
87.38	0.01\\
87.39	0.01\\
87.4	0.01\\
87.41	0.01\\
87.42	0.01\\
87.43	0.01\\
87.44	0.01\\
87.45	0.01\\
87.46	0.01\\
87.47	0.01\\
87.48	0.01\\
87.49	0.01\\
87.5	0.01\\
87.51	0.01\\
87.52	0.01\\
87.53	0.01\\
87.54	0.01\\
87.55	0.01\\
87.56	0.01\\
87.57	0.01\\
87.58	0.01\\
87.59	0.01\\
87.6	0.01\\
87.61	0.01\\
87.62	0.01\\
87.63	0.01\\
87.64	0.01\\
87.65	0.01\\
87.66	0.01\\
87.67	0.01\\
87.68	0.01\\
87.69	0.01\\
87.7	0.01\\
87.71	0.01\\
87.72	0.01\\
87.73	0.01\\
87.74	0.01\\
87.75	0.01\\
87.76	0.01\\
87.77	0.01\\
87.78	0.01\\
87.79	0.01\\
87.8	0.01\\
87.81	0.01\\
87.82	0.01\\
87.83	0.01\\
87.84	0.01\\
87.85	0.01\\
87.86	0.01\\
87.87	0.01\\
87.88	0.01\\
87.89	0.01\\
87.9	0.01\\
87.91	0.01\\
87.92	0.01\\
87.93	0.01\\
87.94	0.01\\
87.95	0.01\\
87.96	0.01\\
87.97	0.01\\
87.98	0.01\\
87.99	0.01\\
88	0.01\\
88.01	0.01\\
88.02	0.01\\
88.03	0.01\\
88.04	0.01\\
88.05	0.01\\
88.06	0.01\\
88.07	0.01\\
88.08	0.01\\
88.09	0.01\\
88.1	0.01\\
88.11	0.01\\
88.12	0.01\\
88.13	0.01\\
88.14	0.01\\
88.15	0.01\\
88.16	0.01\\
88.17	0.01\\
88.18	0.01\\
88.19	0.01\\
88.2	0.01\\
88.21	0.01\\
88.22	0.01\\
88.23	0.01\\
88.24	0.01\\
88.25	0.01\\
88.26	0.01\\
88.27	0.01\\
88.28	0.01\\
88.29	0.01\\
88.3	0.01\\
88.31	0.01\\
88.32	0.01\\
88.33	0.01\\
88.34	0.01\\
88.35	0.01\\
88.36	0.01\\
88.37	0.01\\
88.38	0.01\\
88.39	0.01\\
88.4	0.01\\
88.41	0.01\\
88.42	0.01\\
88.43	0.01\\
88.44	0.01\\
88.45	0.01\\
88.46	0.01\\
88.47	0.01\\
88.48	0.01\\
88.49	0.01\\
88.5	0.01\\
88.51	0.01\\
88.52	0.01\\
88.53	0.01\\
88.54	0.01\\
88.55	0.01\\
88.56	0.01\\
88.57	0.01\\
88.58	0.01\\
88.59	0.01\\
88.6	0.01\\
88.61	0.01\\
88.62	0.01\\
88.63	0.01\\
88.64	0.01\\
88.65	0.01\\
88.66	0.01\\
88.67	0.01\\
88.68	0.01\\
88.69	0.01\\
88.7	0.01\\
88.71	0.01\\
88.72	0.01\\
88.73	0.01\\
88.74	0.01\\
88.75	0.01\\
88.76	0.01\\
88.77	0.01\\
88.78	0.01\\
88.79	0.01\\
88.8	0.01\\
88.81	0.01\\
88.82	0.01\\
88.83	0.01\\
88.84	0.01\\
88.85	0.01\\
88.86	0.01\\
88.87	0.01\\
88.88	0.01\\
88.89	0.01\\
88.9	0.01\\
88.91	0.01\\
88.92	0.01\\
88.93	0.01\\
88.94	0.01\\
88.95	0.01\\
88.96	0.01\\
88.97	0.01\\
88.98	0.01\\
88.99	0.01\\
89	0.01\\
89.01	0.01\\
89.02	0.01\\
89.03	0.01\\
89.04	0.01\\
89.05	0.01\\
89.06	0.01\\
89.07	0.01\\
89.08	0.01\\
89.09	0.01\\
89.1	0.01\\
89.11	0.01\\
89.12	0.01\\
89.13	0.01\\
89.14	0.01\\
89.15	0.01\\
89.16	0.01\\
89.17	0.01\\
89.18	0.01\\
89.19	0.01\\
89.2	0.01\\
89.21	0.01\\
89.22	0.01\\
89.23	0.01\\
89.24	0.01\\
89.25	0.01\\
89.26	0.01\\
89.27	0.01\\
89.28	0.01\\
89.29	0.01\\
89.3	0.01\\
89.31	0.01\\
89.32	0.01\\
89.33	0.01\\
89.34	0.01\\
89.35	0.01\\
89.36	0.01\\
89.37	0.01\\
89.38	0.01\\
89.39	0.01\\
89.4	0.01\\
89.41	0.01\\
89.42	0.01\\
89.43	0.01\\
89.44	0.01\\
89.45	0.01\\
89.46	0.01\\
89.47	0.01\\
89.48	0.01\\
89.49	0.01\\
89.5	0.01\\
89.51	0.01\\
89.52	0.01\\
89.53	0.01\\
89.54	0.01\\
89.55	0.01\\
89.56	0.01\\
89.57	0.01\\
89.58	0.01\\
89.59	0.01\\
89.6	0.01\\
89.61	0.01\\
89.62	0.01\\
89.63	0.01\\
89.64	0.01\\
89.65	0.01\\
89.66	0.01\\
89.67	0.01\\
89.68	0.01\\
89.69	0.01\\
89.7	0.01\\
89.71	0.01\\
89.72	0.01\\
89.73	0.01\\
89.74	0.01\\
89.75	0.01\\
89.76	0.01\\
89.77	0.01\\
89.78	0.01\\
89.79	0.01\\
89.8	0.01\\
89.81	0.01\\
89.82	0.01\\
89.83	0.01\\
89.84	0.01\\
89.85	0.01\\
89.86	0.01\\
89.87	0.01\\
89.88	0.01\\
89.89	0.01\\
89.9	0.01\\
89.91	0.01\\
89.92	0.01\\
89.93	0.01\\
89.94	0.01\\
89.95	0.01\\
89.96	0.01\\
89.97	0.01\\
89.98	0.01\\
89.99	0.01\\
90	0.01\\
90.01	0.01\\
90.02	0.01\\
90.03	0.01\\
90.04	0.01\\
90.05	0.01\\
90.06	0.01\\
90.07	0.01\\
90.08	0.01\\
90.09	0.01\\
90.1	0.01\\
90.11	0.01\\
90.12	0.01\\
90.13	0.01\\
90.14	0.01\\
90.15	0.01\\
90.16	0.01\\
90.17	0.01\\
90.18	0.01\\
90.19	0.01\\
90.2	0.01\\
90.21	0.01\\
90.22	0.01\\
90.23	0.01\\
90.24	0.01\\
90.25	0.01\\
90.26	0.01\\
90.27	0.01\\
90.28	0.01\\
90.29	0.01\\
90.3	0.01\\
90.31	0.01\\
90.32	0.01\\
90.33	0.01\\
90.34	0.01\\
90.35	0.01\\
90.36	0.01\\
90.37	0.01\\
90.38	0.01\\
90.39	0.01\\
90.4	0.01\\
90.41	0.01\\
90.42	0.01\\
90.43	0.01\\
90.44	0.01\\
90.45	0.01\\
90.46	0.01\\
90.47	0.01\\
90.48	0.01\\
90.49	0.01\\
90.5	0.01\\
90.51	0.01\\
90.52	0.01\\
90.53	0.01\\
90.54	0.01\\
90.55	0.01\\
90.56	0.01\\
90.57	0.01\\
90.58	0.01\\
90.59	0.01\\
90.6	0.01\\
90.61	0.01\\
90.62	0.01\\
90.63	0.01\\
90.64	0.01\\
90.65	0.01\\
90.66	0.01\\
90.67	0.01\\
90.68	0.01\\
90.69	0.01\\
90.7	0.01\\
90.71	0.01\\
90.72	0.01\\
90.73	0.01\\
90.74	0.01\\
90.75	0.01\\
90.76	0.01\\
90.77	0.01\\
90.78	0.01\\
90.79	0.01\\
90.8	0.01\\
90.81	0.01\\
90.82	0.01\\
90.83	0.01\\
90.84	0.01\\
90.85	0.01\\
90.86	0.01\\
90.87	0.01\\
90.88	0.01\\
90.89	0.01\\
90.9	0.01\\
90.91	0.01\\
90.92	0.01\\
90.93	0.01\\
90.94	0.01\\
90.95	0.01\\
90.96	0.01\\
90.97	0.01\\
90.98	0.01\\
90.99	0.01\\
91	0.01\\
91.01	0.01\\
91.02	0.01\\
91.03	0.01\\
91.04	0.01\\
91.05	0.01\\
91.06	0.01\\
91.07	0.01\\
91.08	0.01\\
91.09	0.01\\
91.1	0.01\\
91.11	0.01\\
91.12	0.01\\
91.13	0.01\\
91.14	0.01\\
91.15	0.01\\
91.16	0.01\\
91.17	0.01\\
91.18	0.01\\
91.19	0.01\\
91.2	0.01\\
91.21	0.01\\
91.22	0.01\\
91.23	0.01\\
91.24	0.01\\
91.25	0.01\\
91.26	0.01\\
91.27	0.01\\
91.28	0.01\\
91.29	0.01\\
91.3	0.01\\
91.31	0.01\\
91.32	0.01\\
91.33	0.01\\
91.34	0.01\\
91.35	0.01\\
91.36	0.01\\
91.37	0.01\\
91.38	0.01\\
91.39	0.01\\
91.4	0.01\\
91.41	0.01\\
91.42	0.01\\
91.43	0.01\\
91.44	0.01\\
91.45	0.01\\
91.46	0.01\\
91.47	0.01\\
91.48	0.01\\
91.49	0.01\\
91.5	0.01\\
91.51	0.01\\
91.52	0.01\\
91.53	0.01\\
91.54	0.01\\
91.55	0.01\\
91.56	0.01\\
91.57	0.01\\
91.58	0.01\\
91.59	0.01\\
91.6	0.01\\
91.61	0.01\\
91.62	0.01\\
91.63	0.01\\
91.64	0.01\\
91.65	0.01\\
91.66	0.01\\
91.67	0.01\\
91.68	0.01\\
91.69	0.01\\
91.7	0.01\\
91.71	0.01\\
91.72	0.01\\
91.73	0.01\\
91.74	0.01\\
91.75	0.01\\
91.76	0.01\\
91.77	0.01\\
91.78	0.01\\
91.79	0.01\\
91.8	0.01\\
91.81	0.01\\
91.82	0.01\\
91.83	0.01\\
91.84	0.01\\
91.85	0.01\\
91.86	0.01\\
91.87	0.01\\
91.88	0.01\\
91.89	0.01\\
91.9	0.01\\
91.91	0.01\\
91.92	0.01\\
91.93	0.01\\
91.94	0.01\\
91.95	0.01\\
91.96	0.01\\
91.97	0.01\\
91.98	0.01\\
91.99	0.01\\
92	0.01\\
92.01	0.01\\
92.02	0.01\\
92.03	0.01\\
92.04	0.01\\
92.05	0.01\\
92.06	0.01\\
92.07	0.01\\
92.08	0.01\\
92.09	0.01\\
92.1	0.01\\
92.11	0.01\\
92.12	0.01\\
92.13	0.01\\
92.14	0.01\\
92.15	0.01\\
92.16	0.01\\
92.17	0.01\\
92.18	0.01\\
92.19	0.01\\
92.2	0.01\\
92.21	0.01\\
92.22	0.01\\
92.23	0.01\\
92.24	0.01\\
92.25	0.01\\
92.26	0.01\\
92.27	0.01\\
92.28	0.01\\
92.29	0.01\\
92.3	0.01\\
92.31	0.01\\
92.32	0.01\\
92.33	0.01\\
92.34	0.01\\
92.35	0.01\\
92.36	0.01\\
92.37	0.01\\
92.38	0.01\\
92.39	0.01\\
92.4	0.01\\
92.41	0.01\\
92.42	0.01\\
92.43	0.01\\
92.44	0.01\\
92.45	0.01\\
92.46	0.01\\
92.47	0.01\\
92.48	0.01\\
92.49	0.01\\
92.5	0.01\\
92.51	0.01\\
92.52	0.01\\
92.53	0.01\\
92.54	0.01\\
92.55	0.01\\
92.56	0.01\\
92.57	0.01\\
92.58	0.01\\
92.59	0.01\\
92.6	0.01\\
92.61	0.01\\
92.62	0.01\\
92.63	0.01\\
92.64	0.01\\
92.65	0.01\\
92.66	0.01\\
92.67	0.01\\
92.68	0.01\\
92.69	0.01\\
92.7	0.01\\
92.71	0.01\\
92.72	0.01\\
92.73	0.01\\
92.74	0.01\\
92.75	0.01\\
92.76	0.01\\
92.77	0.01\\
92.78	0.01\\
92.79	0.01\\
92.8	0.01\\
92.81	0.01\\
92.82	0.01\\
92.83	0.01\\
92.84	0.01\\
92.85	0.01\\
92.86	0.01\\
92.87	0.01\\
92.88	0.01\\
92.89	0.01\\
92.9	0.01\\
92.91	0.01\\
92.92	0.01\\
92.93	0.01\\
92.94	0.01\\
92.95	0.01\\
92.96	0.01\\
92.97	0.01\\
92.98	0.01\\
92.99	0.01\\
93	0.01\\
93.01	0.01\\
93.02	0.01\\
93.03	0.01\\
93.04	0.01\\
93.05	0.01\\
93.06	0.01\\
93.07	0.01\\
93.08	0.01\\
93.09	0.01\\
93.1	0.01\\
93.11	0.01\\
93.12	0.01\\
93.13	0.01\\
93.14	0.01\\
93.15	0.01\\
93.16	0.01\\
93.17	0.01\\
93.18	0.01\\
93.19	0.01\\
93.2	0.01\\
93.21	0.01\\
93.22	0.01\\
93.23	0.01\\
93.24	0.01\\
93.25	0.01\\
93.26	0.01\\
93.27	0.01\\
93.28	0.01\\
93.29	0.01\\
93.3	0.01\\
93.31	0.01\\
93.32	0.01\\
93.33	0.01\\
93.34	0.01\\
93.35	0.01\\
93.36	0.01\\
93.37	0.01\\
93.38	0.01\\
93.39	0.01\\
93.4	0.01\\
93.41	0.01\\
93.42	0.01\\
93.43	0.01\\
93.44	0.01\\
93.45	0.01\\
93.46	0.01\\
93.47	0.01\\
93.48	0.01\\
93.49	0.01\\
93.5	0.01\\
93.51	0.01\\
93.52	0.01\\
93.53	0.01\\
93.54	0.01\\
93.55	0.01\\
93.56	0.01\\
93.57	0.01\\
93.58	0.01\\
93.59	0.01\\
93.6	0.01\\
93.61	0.01\\
93.62	0.01\\
93.63	0.01\\
93.64	0.01\\
93.65	0.01\\
93.66	0.01\\
93.67	0.01\\
93.68	0.01\\
93.69	0.01\\
93.7	0.01\\
93.71	0.01\\
93.72	0.01\\
93.73	0.01\\
93.74	0.01\\
93.75	0.01\\
93.76	0.01\\
93.77	0.01\\
93.78	0.01\\
93.79	0.01\\
93.8	0.01\\
93.81	0.01\\
93.82	0.01\\
93.83	0.01\\
93.84	0.01\\
93.85	0.01\\
93.86	0.01\\
93.87	0.01\\
93.88	0.01\\
93.89	0.01\\
93.9	0.01\\
93.91	0.01\\
93.92	0.01\\
93.93	0.01\\
93.94	0.01\\
93.95	0.01\\
93.96	0.01\\
93.97	0.01\\
93.98	0.01\\
93.99	0.01\\
94	0.01\\
94.01	0.01\\
94.02	0.01\\
94.03	0.01\\
94.04	0.01\\
94.05	0.01\\
94.06	0.01\\
94.07	0.01\\
94.08	0.01\\
94.09	0.01\\
94.1	0.01\\
94.11	0.01\\
94.12	0.01\\
94.13	0.01\\
94.14	0.01\\
94.15	0.01\\
94.16	0.01\\
94.17	0.01\\
94.18	0.01\\
94.19	0.01\\
94.2	0.01\\
94.21	0.01\\
94.22	0.01\\
94.23	0.01\\
94.24	0.01\\
94.25	0.01\\
94.26	0.01\\
94.27	0.01\\
94.28	0.01\\
94.29	0.01\\
94.3	0.01\\
94.31	0.01\\
94.32	0.01\\
94.33	0.01\\
94.34	0.01\\
94.35	0.01\\
94.36	0.01\\
94.37	0.01\\
94.38	0.01\\
94.39	0.01\\
94.4	0.01\\
94.41	0.01\\
94.42	0.01\\
94.43	0.01\\
94.44	0.01\\
94.45	0.01\\
94.46	0.01\\
94.47	0.01\\
94.48	0.01\\
94.49	0.01\\
94.5	0.01\\
94.51	0.01\\
94.52	0.01\\
94.53	0.01\\
94.54	0.01\\
94.55	0.01\\
94.56	0.01\\
94.57	0.01\\
94.58	0.01\\
94.59	0.01\\
94.6	0.01\\
94.61	0.01\\
94.62	0.01\\
94.63	0.01\\
94.64	0.01\\
94.65	0.01\\
94.66	0.01\\
94.67	0.01\\
94.68	0.01\\
94.69	0.01\\
94.7	0.01\\
94.71	0.01\\
94.72	0.01\\
94.73	0.01\\
94.74	0.01\\
94.75	0.01\\
94.76	0.01\\
94.77	0.01\\
94.78	0.01\\
94.79	0.01\\
94.8	0.01\\
94.81	0.01\\
94.82	0.01\\
94.83	0.01\\
94.84	0.01\\
94.85	0.01\\
94.86	0.01\\
94.87	0.01\\
94.88	0.01\\
94.89	0.01\\
94.9	0.01\\
94.91	0.01\\
94.92	0.01\\
94.93	0.01\\
94.94	0.01\\
94.95	0.01\\
94.96	0.01\\
94.97	0.01\\
94.98	0.01\\
94.99	0.01\\
95	0.01\\
95.01	0.01\\
95.02	0.01\\
95.03	0.01\\
95.04	0.01\\
95.05	0.01\\
95.06	0.01\\
95.07	0.01\\
95.08	0.01\\
95.09	0.01\\
95.1	0.01\\
95.11	0.01\\
95.12	0.01\\
95.13	0.01\\
95.14	0.01\\
95.15	0.01\\
95.16	0.01\\
95.17	0.01\\
95.18	0.01\\
95.19	0.01\\
95.2	0.01\\
95.21	0.01\\
95.22	0.01\\
95.23	0.01\\
95.24	0.01\\
95.25	0.01\\
95.26	0.01\\
95.27	0.01\\
95.28	0.01\\
95.29	0.01\\
95.3	0.01\\
95.31	0.01\\
95.32	0.01\\
95.33	0.01\\
95.34	0.01\\
95.35	0.01\\
95.36	0.01\\
95.37	0.01\\
95.38	0.01\\
95.39	0.01\\
95.4	0.01\\
95.41	0.01\\
95.42	0.01\\
95.43	0.01\\
95.44	0.01\\
95.45	0.01\\
95.46	0.01\\
95.47	0.01\\
95.48	0.01\\
95.49	0.01\\
95.5	0.01\\
95.51	0.01\\
95.52	0.01\\
95.53	0.01\\
95.54	0.01\\
95.55	0.01\\
95.56	0.01\\
95.57	0.01\\
95.58	0.01\\
95.59	0.01\\
95.6	0.01\\
95.61	0.01\\
95.62	0.01\\
95.63	0.01\\
95.64	0.01\\
95.65	0.01\\
95.66	0.01\\
95.67	0.01\\
95.68	0.01\\
95.69	0.01\\
95.7	0.01\\
95.71	0.01\\
95.72	0.01\\
95.73	0.01\\
95.74	0.01\\
95.75	0.01\\
95.76	0.01\\
95.77	0.01\\
95.78	0.01\\
95.79	0.01\\
95.8	0.01\\
95.81	0.01\\
95.82	0.01\\
95.83	0.01\\
95.84	0.01\\
95.85	0.01\\
95.86	0.01\\
95.87	0.01\\
95.88	0.01\\
95.89	0.01\\
95.9	0.01\\
95.91	0.01\\
95.92	0.01\\
95.93	0.01\\
95.94	0.01\\
95.95	0.01\\
95.96	0.01\\
95.97	0.01\\
95.98	0.01\\
95.99	0.01\\
96	0.01\\
96.01	0.01\\
96.02	0.01\\
96.03	0.01\\
96.04	0.01\\
96.05	0.01\\
96.06	0.01\\
96.07	0.01\\
96.08	0.01\\
96.09	0.01\\
96.1	0.01\\
96.11	0.01\\
96.12	0.01\\
96.13	0.01\\
96.14	0.01\\
96.15	0.01\\
96.16	0.01\\
96.17	0.01\\
96.18	0.01\\
96.19	0.01\\
96.2	0.01\\
96.21	0.01\\
96.22	0.01\\
96.23	0.01\\
96.24	0.01\\
96.25	0.01\\
96.26	0.01\\
96.27	0.01\\
96.28	0.01\\
96.29	0.01\\
96.3	0.01\\
96.31	0.01\\
96.32	0.01\\
96.33	0.01\\
96.34	0.01\\
96.35	0.01\\
96.36	0.01\\
96.37	0.01\\
96.38	0.01\\
96.39	0.01\\
96.4	0.01\\
96.41	0.01\\
96.42	0.01\\
96.43	0.01\\
96.44	0.01\\
96.45	0.01\\
96.46	0.01\\
96.47	0.01\\
96.48	0.01\\
96.49	0.01\\
96.5	0.01\\
96.51	0.01\\
96.52	0.01\\
96.53	0.01\\
96.54	0.01\\
96.55	0.01\\
96.56	0.01\\
96.57	0.01\\
96.58	0.01\\
96.59	0.01\\
96.6	0.01\\
96.61	0.01\\
96.62	0.01\\
96.63	0.01\\
96.64	0.01\\
96.65	0.01\\
96.66	0.01\\
96.67	0.01\\
96.68	0.01\\
96.69	0.01\\
96.7	0.01\\
96.71	0.01\\
96.72	0.01\\
96.73	0.01\\
96.74	0.01\\
96.75	0.01\\
96.76	0.01\\
96.77	0.01\\
96.78	0.01\\
96.79	0.01\\
96.8	0.01\\
96.81	0.01\\
96.82	0.01\\
96.83	0.01\\
96.84	0.01\\
96.85	0.01\\
96.86	0.01\\
96.87	0.01\\
96.88	0.01\\
96.89	0.01\\
96.9	0.01\\
96.91	0.01\\
96.92	0.01\\
96.93	0.01\\
96.94	0.01\\
96.95	0.01\\
96.96	0.01\\
96.97	0.01\\
96.98	0.01\\
96.99	0.01\\
97	0.01\\
97.01	0.01\\
97.02	0.01\\
97.03	0.01\\
97.04	0.01\\
97.05	0.01\\
97.06	0.01\\
97.07	0.01\\
97.08	0.01\\
97.09	0.01\\
97.1	0.01\\
97.11	0.01\\
97.12	0.01\\
97.13	0.01\\
97.14	0.01\\
97.15	0.01\\
97.16	0.01\\
97.17	0.01\\
97.18	0.01\\
97.19	0.01\\
97.2	0.01\\
97.21	0.01\\
97.22	0.01\\
97.23	0.01\\
97.24	0.01\\
97.25	0.01\\
97.26	0.01\\
97.27	0.01\\
97.28	0.01\\
97.29	0.01\\
97.3	0.01\\
97.31	0.01\\
97.32	0.01\\
97.33	0.01\\
97.34	0.01\\
97.35	0.01\\
97.36	0.01\\
97.37	0.01\\
97.38	0.01\\
97.39	0.01\\
97.4	0.01\\
97.41	0.01\\
97.42	0.01\\
97.43	0.01\\
97.44	0.01\\
97.45	0.01\\
97.46	0.01\\
97.47	0.01\\
97.48	0.01\\
97.49	0.01\\
97.5	0.01\\
97.51	0.01\\
97.52	0.01\\
97.53	0.01\\
97.54	0.01\\
97.55	0.01\\
97.56	0.01\\
97.57	0.01\\
97.58	0.01\\
97.59	0.01\\
97.6	0.01\\
97.61	0.01\\
97.62	0.01\\
97.63	0.01\\
97.64	0.01\\
97.65	0.01\\
97.66	0.01\\
97.67	0.01\\
97.68	0.01\\
97.69	0.01\\
97.7	0.01\\
97.71	0.01\\
97.72	0.01\\
97.73	0.01\\
97.74	0.01\\
97.75	0.01\\
97.76	0.01\\
97.77	0.01\\
97.78	0.01\\
97.79	0.01\\
97.8	0.01\\
97.81	0.01\\
97.82	0.01\\
97.83	0.01\\
97.84	0.01\\
97.85	0.01\\
97.86	0.01\\
97.87	0.01\\
97.88	0.01\\
97.89	0.01\\
97.9	0.01\\
97.91	0.01\\
97.92	0.01\\
97.93	0.01\\
97.94	0.01\\
97.95	0.01\\
97.96	0.01\\
97.97	0.01\\
97.98	0.01\\
97.99	0.01\\
98	0.01\\
98.01	0.01\\
98.02	0.01\\
98.03	0.01\\
98.04	0.01\\
98.05	0.01\\
98.06	0.01\\
98.07	0.01\\
98.08	0.01\\
98.09	0.01\\
98.1	0.01\\
98.11	0.01\\
98.12	0.01\\
98.13	0.01\\
98.14	0.01\\
98.15	0.01\\
98.16	0.01\\
98.17	0.01\\
98.18	0.01\\
98.19	0.01\\
98.2	0.01\\
98.21	0.01\\
98.22	0.01\\
98.23	0.01\\
98.24	0.01\\
98.25	0.01\\
98.26	0.01\\
98.27	0.01\\
98.28	0.01\\
98.29	0.01\\
98.3	0.01\\
98.31	0.01\\
98.32	0.01\\
98.33	0.01\\
98.34	0.01\\
98.35	0.01\\
98.36	0.01\\
98.37	0.01\\
98.38	0.01\\
98.39	0.01\\
98.4	0.01\\
98.41	0.01\\
98.42	0.01\\
98.43	0.01\\
98.44	0.01\\
98.45	0.01\\
98.46	0.01\\
98.47	0.01\\
98.48	0.01\\
98.49	0.01\\
98.5	0.01\\
98.51	0.01\\
98.52	0.01\\
98.53	0.01\\
98.54	0.01\\
98.55	0.01\\
98.56	0.01\\
98.57	0.01\\
98.58	0.01\\
98.59	0.01\\
98.6	0.01\\
98.61	0.01\\
98.62	0.01\\
98.63	0.01\\
98.64	0.01\\
98.65	0.01\\
98.66	0.01\\
98.67	0.01\\
98.68	0.01\\
98.69	0.01\\
98.7	0.01\\
98.71	0.01\\
98.72	0.01\\
98.73	0.01\\
98.74	0.01\\
98.75	0.01\\
98.76	0.01\\
98.77	0.01\\
98.78	0.01\\
98.79	0.01\\
98.8	0.01\\
98.81	0.01\\
98.82	0.01\\
98.83	0.01\\
98.84	0.01\\
98.85	0.01\\
98.86	0.01\\
98.87	0.01\\
98.88	0.01\\
98.89	0.01\\
98.9	0.01\\
98.91	0.01\\
98.92	0.01\\
98.93	0.01\\
98.94	0.01\\
98.95	0.01\\
98.96	0.01\\
98.97	0.01\\
98.98	0.01\\
98.99	0.01\\
99	0.01\\
99.01	0.01\\
99.02	0.01\\
99.03	0.01\\
99.04	0.01\\
99.05	0.01\\
99.06	0.01\\
99.07	0.01\\
99.08	0.01\\
99.09	0.01\\
99.1	0.01\\
99.11	0.01\\
99.12	0.01\\
99.13	0.01\\
99.14	0.01\\
99.15	0.01\\
99.16	0.01\\
99.17	0.01\\
99.18	0.01\\
99.19	0.01\\
99.2	0.01\\
99.21	0.01\\
99.22	0.01\\
99.23	0.01\\
99.24	0.01\\
99.25	0.01\\
99.26	0.01\\
99.27	0.01\\
99.28	0.01\\
99.29	0.01\\
99.3	0.01\\
99.31	0.01\\
99.32	0.01\\
99.33	0.01\\
99.34	0.01\\
99.35	0.01\\
99.36	0.01\\
99.37	0.01\\
99.38	0.01\\
99.39	0.01\\
99.4	0.01\\
99.41	0.01\\
99.42	0.01\\
99.43	0.01\\
99.44	0.01\\
99.45	0.01\\
99.46	0.01\\
99.47	0.01\\
99.48	0.01\\
99.49	0.01\\
99.5	0.01\\
99.51	0.01\\
99.52	0.01\\
99.53	0.01\\
99.54	0.01\\
99.55	0.01\\
99.56	0.01\\
99.57	0.01\\
99.58	0.01\\
99.59	0.01\\
99.6	0.01\\
99.61	0.01\\
99.62	0.01\\
99.63	0.01\\
99.64	0.01\\
99.65	0.01\\
99.66	0.01\\
99.67	0.01\\
99.68	0.01\\
99.69	0.01\\
99.7	0.01\\
99.71	0.01\\
99.72	0.01\\
99.73	0.01\\
99.74	0.01\\
99.75	0.01\\
99.76	0.01\\
99.77	0.01\\
99.78	0.01\\
99.79	0.01\\
99.8	0.01\\
99.81	0.01\\
99.82	0.01\\
99.83	0.01\\
99.84	0.01\\
99.85	0.01\\
99.86	0.01\\
99.87	0.01\\
99.88	0.01\\
99.89	0.01\\
99.9	0.01\\
99.91	0.01\\
99.92	0.01\\
99.93	0.01\\
99.94	0.01\\
99.95	0.01\\
99.96	0.01\\
99.97	0.01\\
99.98	0.01\\
99.99	0.01\\
100	0.01\\
};
\addlegendentry{$q=4$};

\end{axis}
\end{tikzpicture}% 
  \caption{Continuous Time w/ nFPC}
\end{subfigure}%
\hfill%
\begin{subfigure}{.45\linewidth}
  \centering
  \setlength\figureheight{\linewidth} 
  \setlength\figurewidth{\linewidth}
  \tikzsetnextfilename{dp_dscr_nFPC_z15}
  % This file was created by matlab2tikz.
%
%The latest updates can be retrieved from
%  http://www.mathworks.com/matlabcentral/fileexchange/22022-matlab2tikz-matlab2tikz
%where you can also make suggestions and rate matlab2tikz.
%
\definecolor{mycolor1}{rgb}{0.00000,1.00000,0.14286}%
\definecolor{mycolor2}{rgb}{0.00000,1.00000,0.28571}%
\definecolor{mycolor3}{rgb}{0.00000,1.00000,0.42857}%
\definecolor{mycolor4}{rgb}{0.00000,1.00000,0.57143}%
\definecolor{mycolor5}{rgb}{0.00000,1.00000,0.71429}%
\definecolor{mycolor6}{rgb}{0.00000,1.00000,0.85714}%
\definecolor{mycolor7}{rgb}{0.00000,1.00000,1.00000}%
\definecolor{mycolor8}{rgb}{0.00000,0.87500,1.00000}%
\definecolor{mycolor9}{rgb}{0.00000,0.62500,1.00000}%
\definecolor{mycolor10}{rgb}{0.12500,0.00000,1.00000}%
\definecolor{mycolor11}{rgb}{0.25000,0.00000,1.00000}%
\definecolor{mycolor12}{rgb}{0.37500,0.00000,1.00000}%
\definecolor{mycolor13}{rgb}{0.50000,0.00000,1.00000}%
\definecolor{mycolor14}{rgb}{0.62500,0.00000,1.00000}%
\definecolor{mycolor15}{rgb}{0.75000,0.00000,1.00000}%
\definecolor{mycolor16}{rgb}{0.87500,0.00000,1.00000}%
\definecolor{mycolor17}{rgb}{1.00000,0.00000,1.00000}%
\definecolor{mycolor18}{rgb}{1.00000,0.00000,0.87500}%
\definecolor{mycolor19}{rgb}{1.00000,0.00000,0.62500}%
\definecolor{mycolor20}{rgb}{0.85714,0.00000,0.00000}%
\definecolor{mycolor21}{rgb}{0.71429,0.00000,0.00000}%
%
\begin{tikzpicture}[trim axis left, trim axis right]

\begin{axis}[%
width=\figurewidth,
height=\figureheight,
at={(0\figurewidth,0\figureheight)},
scale only axis,
every outer x axis line/.append style={black},
every x tick label/.append style={font=\color{black}},
xmin=0,
xmax=600,
every outer y axis line/.append style={black},
every y tick label/.append style={font=\color{black}},
ymin=0,
ymax=0.014,
axis background/.style={fill=white},
axis x line*=bottom,
axis y line*=left,
yticklabel style={
        /pgf/number format/fixed,
        /pgf/number format/precision=3
},
scaled y ticks=false
]
\addplot [color=green,solid,forget plot]
  table[row sep=crcr]{%
1	0\\
2	0\\
3	0\\
4	0\\
5	0\\
6	0\\
7	0\\
8	0\\
9	0\\
10	0\\
11	0\\
12	0\\
13	0\\
14	0\\
15	0\\
16	0\\
17	0\\
18	0\\
19	0\\
20	0\\
21	0\\
22	0\\
23	0\\
24	0\\
25	0\\
26	0\\
27	0\\
28	0\\
29	0\\
30	0\\
31	0\\
32	0\\
33	0\\
34	0\\
35	0\\
36	0\\
37	0\\
38	0\\
39	0\\
40	0\\
41	0\\
42	0\\
43	0\\
44	0\\
45	0\\
46	0\\
47	0\\
48	0\\
49	0\\
50	0\\
51	0\\
52	0\\
53	0\\
54	0\\
55	0\\
56	0\\
57	0\\
58	0\\
59	0\\
60	0\\
61	0\\
62	0\\
63	0\\
64	0\\
65	0\\
66	0\\
67	0\\
68	0\\
69	0\\
70	0\\
71	0\\
72	0\\
73	0\\
74	0\\
75	0\\
76	0\\
77	0\\
78	0\\
79	0\\
80	0\\
81	0\\
82	0\\
83	0\\
84	0\\
85	0\\
86	0\\
87	0\\
88	0\\
89	0\\
90	0\\
91	0\\
92	0\\
93	0\\
94	0\\
95	0\\
96	0\\
97	0\\
98	0\\
99	0\\
100	0\\
101	0\\
102	0\\
103	0\\
104	0\\
105	0\\
106	0\\
107	0\\
108	0\\
109	0\\
110	0\\
111	0\\
112	0\\
113	0\\
114	0\\
115	0\\
116	0\\
117	0\\
118	0\\
119	0\\
120	0\\
121	0\\
122	0\\
123	0\\
124	0\\
125	0\\
126	0\\
127	0\\
128	0\\
129	0\\
130	0\\
131	0\\
132	0\\
133	0\\
134	0\\
135	0\\
136	0\\
137	0\\
138	0\\
139	0\\
140	0\\
141	0\\
142	0\\
143	0\\
144	0\\
145	0\\
146	0\\
147	0\\
148	0\\
149	0\\
150	0\\
151	0\\
152	0\\
153	0\\
154	0\\
155	0\\
156	0\\
157	0\\
158	0\\
159	0\\
160	0\\
161	0\\
162	0\\
163	0\\
164	0\\
165	0\\
166	0\\
167	0\\
168	0\\
169	0\\
170	0\\
171	0\\
172	0\\
173	0\\
174	0\\
175	0\\
176	0\\
177	0\\
178	0\\
179	0\\
180	0\\
181	0\\
182	0\\
183	0\\
184	0\\
185	0\\
186	0\\
187	0\\
188	0\\
189	0\\
190	0\\
191	0\\
192	0\\
193	0\\
194	0\\
195	0\\
196	0\\
197	0\\
198	0\\
199	0\\
200	0\\
201	0\\
202	0\\
203	0\\
204	0\\
205	0\\
206	0\\
207	0\\
208	0\\
209	0\\
210	0\\
211	0\\
212	0\\
213	0\\
214	0\\
215	0\\
216	0\\
217	0\\
218	0\\
219	0\\
220	0\\
221	0\\
222	0\\
223	0\\
224	0\\
225	0\\
226	0\\
227	0\\
228	0\\
229	0\\
230	0\\
231	0\\
232	0\\
233	0\\
234	0\\
235	0\\
236	0\\
237	0\\
238	0\\
239	0\\
240	0\\
241	0\\
242	0\\
243	0\\
244	0\\
245	0\\
246	0\\
247	0\\
248	0\\
249	0\\
250	0\\
251	0\\
252	0\\
253	0\\
254	0\\
255	0\\
256	0\\
257	0\\
258	0\\
259	0\\
260	0\\
261	0\\
262	0\\
263	0\\
264	0\\
265	0\\
266	0\\
267	0\\
268	0\\
269	0\\
270	0\\
271	0\\
272	0\\
273	0\\
274	0\\
275	0\\
276	0\\
277	0\\
278	0\\
279	0\\
280	0\\
281	0\\
282	0\\
283	0\\
284	0\\
285	0\\
286	0\\
287	0\\
288	0\\
289	0\\
290	0\\
291	0\\
292	0\\
293	0\\
294	0\\
295	0\\
296	0\\
297	0\\
298	0\\
299	0\\
300	0\\
301	0\\
302	0\\
303	0\\
304	0\\
305	0\\
306	0\\
307	0\\
308	0\\
309	0\\
310	0\\
311	0\\
312	0\\
313	0\\
314	0\\
315	0\\
316	0\\
317	0\\
318	0\\
319	0\\
320	0\\
321	0\\
322	0\\
323	0\\
324	0\\
325	0\\
326	0\\
327	0\\
328	0\\
329	0\\
330	0\\
331	0\\
332	0\\
333	0\\
334	0\\
335	0\\
336	0\\
337	0\\
338	0\\
339	0\\
340	0\\
341	0\\
342	0\\
343	0\\
344	0\\
345	0\\
346	0\\
347	0\\
348	0\\
349	0\\
350	0\\
351	0\\
352	0\\
353	0\\
354	0\\
355	0\\
356	0\\
357	0\\
358	0\\
359	0\\
360	0\\
361	0\\
362	0\\
363	0\\
364	0\\
365	0\\
366	0\\
367	0\\
368	0\\
369	0\\
370	0\\
371	0\\
372	0\\
373	0\\
374	0\\
375	0\\
376	0\\
377	0\\
378	0\\
379	0\\
380	0\\
381	0\\
382	0\\
383	0\\
384	0\\
385	0\\
386	0\\
387	0\\
388	0\\
389	0\\
390	0\\
391	0\\
392	0\\
393	0\\
394	0\\
395	0\\
396	0\\
397	0\\
398	0\\
399	0\\
400	0\\
401	0\\
402	0\\
403	0\\
404	0\\
405	0\\
406	0\\
407	0\\
408	0\\
409	0\\
410	0\\
411	0\\
412	0\\
413	0\\
414	0\\
415	0\\
416	0\\
417	0\\
418	0\\
419	0\\
420	0\\
421	0\\
422	0\\
423	0\\
424	0\\
425	0\\
426	0\\
427	0\\
428	0\\
429	0\\
430	0\\
431	0\\
432	0\\
433	0\\
434	0\\
435	0\\
436	0\\
437	0\\
438	0\\
439	0\\
440	0\\
441	0\\
442	0\\
443	0\\
444	0\\
445	0\\
446	0\\
447	0\\
448	0\\
449	0\\
450	0\\
451	0\\
452	0\\
453	0\\
454	0\\
455	0\\
456	0\\
457	0\\
458	0\\
459	0\\
460	0\\
461	0\\
462	0\\
463	0\\
464	0\\
465	0\\
466	0\\
467	0\\
468	0\\
469	0\\
470	0\\
471	0\\
472	0\\
473	0\\
474	0\\
475	0\\
476	0\\
477	0\\
478	0\\
479	0\\
480	0\\
481	0\\
482	0\\
483	0\\
484	0\\
485	0\\
486	0\\
487	0\\
488	0\\
489	0\\
490	0\\
491	0\\
492	0\\
493	0\\
494	0\\
495	0\\
496	0\\
497	0\\
498	0\\
499	0\\
500	0\\
501	0\\
502	0\\
503	0\\
504	0\\
505	0\\
506	0\\
507	0\\
508	0\\
509	0\\
510	0\\
511	0\\
512	0\\
513	0\\
514	0\\
515	0\\
516	0\\
517	0\\
518	0\\
519	0\\
520	0\\
521	0\\
522	0\\
523	0\\
524	0\\
525	0\\
526	0\\
527	0\\
528	0\\
529	0\\
530	0\\
531	0\\
532	0\\
533	0\\
534	0\\
535	0\\
536	0\\
537	0\\
538	0\\
539	0\\
540	0\\
541	0\\
542	0\\
543	0\\
544	0\\
545	0\\
546	0\\
547	0\\
548	0\\
549	0\\
550	0\\
551	0\\
552	0\\
553	0\\
554	0\\
555	0\\
556	0\\
557	0\\
558	0\\
559	0\\
560	0\\
561	0\\
562	0\\
563	0\\
564	0\\
565	0\\
566	0\\
567	0\\
568	0\\
569	0\\
570	0\\
571	0\\
572	0\\
573	0\\
574	0\\
575	0\\
576	0\\
577	0\\
578	0\\
579	0\\
580	0\\
581	0\\
582	0\\
583	0\\
584	0\\
585	0\\
586	0\\
587	0\\
588	0\\
589	0\\
590	0\\
591	0\\
592	0\\
593	0\\
594	0\\
595	0\\
596	0\\
597	0\\
598	0\\
599	0\\
600	0\\
};
\addplot [color=mycolor1,solid,forget plot]
  table[row sep=crcr]{%
1	0\\
2	0\\
3	0\\
4	0\\
5	0\\
6	0\\
7	0\\
8	0\\
9	0\\
10	0\\
11	0\\
12	0\\
13	0\\
14	0\\
15	0\\
16	0\\
17	0\\
18	0\\
19	0\\
20	0\\
21	0\\
22	0\\
23	0\\
24	0\\
25	0\\
26	0\\
27	0\\
28	0\\
29	0\\
30	0\\
31	0\\
32	0\\
33	0\\
34	0\\
35	0\\
36	0\\
37	0\\
38	0\\
39	0\\
40	0\\
41	0\\
42	0\\
43	0\\
44	0\\
45	0\\
46	0\\
47	0\\
48	0\\
49	0\\
50	0\\
51	0\\
52	0\\
53	0\\
54	0\\
55	0\\
56	0\\
57	0\\
58	0\\
59	0\\
60	0\\
61	0\\
62	0\\
63	0\\
64	0\\
65	0\\
66	0\\
67	0\\
68	0\\
69	0\\
70	0\\
71	0\\
72	0\\
73	0\\
74	0\\
75	0\\
76	0\\
77	0\\
78	0\\
79	0\\
80	0\\
81	0\\
82	0\\
83	0\\
84	0\\
85	0\\
86	0\\
87	0\\
88	0\\
89	0\\
90	0\\
91	0\\
92	0\\
93	0\\
94	0\\
95	0\\
96	0\\
97	0\\
98	0\\
99	0\\
100	0\\
101	0\\
102	0\\
103	0\\
104	0\\
105	0\\
106	0\\
107	0\\
108	0\\
109	0\\
110	0\\
111	0\\
112	0\\
113	0\\
114	0\\
115	0\\
116	0\\
117	0\\
118	0\\
119	0\\
120	0\\
121	0\\
122	0\\
123	0\\
124	0\\
125	0\\
126	0\\
127	0\\
128	0\\
129	0\\
130	0\\
131	0\\
132	0\\
133	0\\
134	0\\
135	0\\
136	0\\
137	0\\
138	0\\
139	0\\
140	0\\
141	0\\
142	0\\
143	0\\
144	0\\
145	0\\
146	0\\
147	0\\
148	0\\
149	0\\
150	0\\
151	0\\
152	0\\
153	0\\
154	0\\
155	0\\
156	0\\
157	0\\
158	0\\
159	0\\
160	0\\
161	0\\
162	0\\
163	0\\
164	0\\
165	0\\
166	0\\
167	0\\
168	0\\
169	0\\
170	0\\
171	0\\
172	0\\
173	0\\
174	0\\
175	0\\
176	0\\
177	0\\
178	0\\
179	0\\
180	0\\
181	0\\
182	0\\
183	0\\
184	0\\
185	0\\
186	0\\
187	0\\
188	0\\
189	0\\
190	0\\
191	0\\
192	0\\
193	0\\
194	0\\
195	0\\
196	0\\
197	0\\
198	0\\
199	0\\
200	0\\
201	0\\
202	0\\
203	0\\
204	0\\
205	0\\
206	0\\
207	0\\
208	0\\
209	0\\
210	0\\
211	0\\
212	0\\
213	0\\
214	0\\
215	0\\
216	0\\
217	0\\
218	0\\
219	0\\
220	0\\
221	0\\
222	0\\
223	0\\
224	0\\
225	0\\
226	0\\
227	0\\
228	0\\
229	0\\
230	0\\
231	0\\
232	0\\
233	0\\
234	0\\
235	0\\
236	0\\
237	0\\
238	0\\
239	0\\
240	0\\
241	0\\
242	0\\
243	0\\
244	0\\
245	0\\
246	0\\
247	0\\
248	0\\
249	0\\
250	0\\
251	0\\
252	0\\
253	0\\
254	0\\
255	0\\
256	0\\
257	0\\
258	0\\
259	0\\
260	0\\
261	0\\
262	0\\
263	0\\
264	0\\
265	0\\
266	0\\
267	0\\
268	0\\
269	0\\
270	0\\
271	0\\
272	0\\
273	0\\
274	0\\
275	0\\
276	0\\
277	0\\
278	0\\
279	0\\
280	0\\
281	0\\
282	0\\
283	0\\
284	0\\
285	0\\
286	0\\
287	0\\
288	0\\
289	0\\
290	0\\
291	0\\
292	0\\
293	0\\
294	0\\
295	0\\
296	0\\
297	0\\
298	0\\
299	0\\
300	0\\
301	0\\
302	0\\
303	0\\
304	0\\
305	0\\
306	0\\
307	0\\
308	0\\
309	0\\
310	0\\
311	0\\
312	0\\
313	0\\
314	0\\
315	0\\
316	0\\
317	0\\
318	0\\
319	0\\
320	0\\
321	0\\
322	0\\
323	0\\
324	0\\
325	0\\
326	0\\
327	0\\
328	0\\
329	0\\
330	0\\
331	0\\
332	0\\
333	0\\
334	0\\
335	0\\
336	0\\
337	0\\
338	0\\
339	0\\
340	0\\
341	0\\
342	0\\
343	0\\
344	0\\
345	0\\
346	0\\
347	0\\
348	0\\
349	0\\
350	0\\
351	0\\
352	0\\
353	0\\
354	0\\
355	0\\
356	0\\
357	0\\
358	0\\
359	0\\
360	0\\
361	0\\
362	0\\
363	0\\
364	0\\
365	0\\
366	0\\
367	0\\
368	0\\
369	0\\
370	0\\
371	0\\
372	0\\
373	0\\
374	0\\
375	0\\
376	0\\
377	0\\
378	0\\
379	0\\
380	0\\
381	0\\
382	0\\
383	0\\
384	0\\
385	0\\
386	0\\
387	0\\
388	0\\
389	0\\
390	0\\
391	0\\
392	0\\
393	0\\
394	0\\
395	0\\
396	0\\
397	0\\
398	0\\
399	0\\
400	0\\
401	0\\
402	0\\
403	0\\
404	0\\
405	0\\
406	0\\
407	0\\
408	0\\
409	0\\
410	0\\
411	0\\
412	0\\
413	0\\
414	0\\
415	0\\
416	0\\
417	0\\
418	0\\
419	0\\
420	0\\
421	0\\
422	0\\
423	0\\
424	0\\
425	0\\
426	0\\
427	0\\
428	0\\
429	0\\
430	0\\
431	0\\
432	0\\
433	0\\
434	0\\
435	0\\
436	0\\
437	0\\
438	0\\
439	0\\
440	0\\
441	0\\
442	0\\
443	0\\
444	0\\
445	0\\
446	0\\
447	0\\
448	0\\
449	0\\
450	0\\
451	0\\
452	0\\
453	0\\
454	0\\
455	0\\
456	0\\
457	0\\
458	0\\
459	0\\
460	0\\
461	0\\
462	0\\
463	0\\
464	0\\
465	0\\
466	0\\
467	0\\
468	0\\
469	0\\
470	0\\
471	0\\
472	0\\
473	0\\
474	0\\
475	0\\
476	0\\
477	0\\
478	0\\
479	0\\
480	0\\
481	0\\
482	0\\
483	0\\
484	0\\
485	0\\
486	0\\
487	0\\
488	0\\
489	0\\
490	0\\
491	0\\
492	0\\
493	0\\
494	0\\
495	0\\
496	0\\
497	0\\
498	0\\
499	0\\
500	0\\
501	0\\
502	0\\
503	0\\
504	0\\
505	0\\
506	0\\
507	0\\
508	0\\
509	0\\
510	0\\
511	0\\
512	0\\
513	0\\
514	0\\
515	0\\
516	0\\
517	0\\
518	0\\
519	0\\
520	0\\
521	0\\
522	0\\
523	0\\
524	0\\
525	0\\
526	0\\
527	0\\
528	0\\
529	0\\
530	0\\
531	0\\
532	0\\
533	0\\
534	0\\
535	0\\
536	0\\
537	0\\
538	0\\
539	0\\
540	0\\
541	0\\
542	0\\
543	0\\
544	0\\
545	0\\
546	0\\
547	0\\
548	0\\
549	0\\
550	0\\
551	0\\
552	0\\
553	0\\
554	0\\
555	0\\
556	0\\
557	0\\
558	0\\
559	0\\
560	0\\
561	0\\
562	0\\
563	0\\
564	0\\
565	0\\
566	0\\
567	0\\
568	0\\
569	0\\
570	0\\
571	0\\
572	0\\
573	0\\
574	0\\
575	0\\
576	0\\
577	0\\
578	0\\
579	0\\
580	0\\
581	0\\
582	0\\
583	0\\
584	0\\
585	0\\
586	0\\
587	0\\
588	0\\
589	0\\
590	0\\
591	0\\
592	0\\
593	0\\
594	0\\
595	0\\
596	0\\
597	0\\
598	0\\
599	0\\
600	0\\
};
\addplot [color=mycolor2,solid,forget plot]
  table[row sep=crcr]{%
1	0\\
2	0\\
3	0\\
4	0\\
5	0\\
6	0\\
7	0\\
8	0\\
9	0\\
10	0\\
11	0\\
12	0\\
13	0\\
14	0\\
15	0\\
16	0\\
17	0\\
18	0\\
19	0\\
20	0\\
21	0\\
22	0\\
23	0\\
24	0\\
25	0\\
26	0\\
27	0\\
28	0\\
29	0\\
30	0\\
31	0\\
32	0\\
33	0\\
34	0\\
35	0\\
36	0\\
37	0\\
38	0\\
39	0\\
40	0\\
41	0\\
42	0\\
43	0\\
44	0\\
45	0\\
46	0\\
47	0\\
48	0\\
49	0\\
50	0\\
51	0\\
52	0\\
53	0\\
54	0\\
55	0\\
56	0\\
57	0\\
58	0\\
59	0\\
60	0\\
61	0\\
62	0\\
63	0\\
64	0\\
65	0\\
66	0\\
67	0\\
68	0\\
69	0\\
70	0\\
71	0\\
72	0\\
73	0\\
74	0\\
75	0\\
76	0\\
77	0\\
78	0\\
79	0\\
80	0\\
81	0\\
82	0\\
83	0\\
84	0\\
85	0\\
86	0\\
87	0\\
88	0\\
89	0\\
90	0\\
91	0\\
92	0\\
93	0\\
94	0\\
95	0\\
96	0\\
97	0\\
98	0\\
99	0\\
100	0\\
101	0\\
102	0\\
103	0\\
104	0\\
105	0\\
106	0\\
107	0\\
108	0\\
109	0\\
110	0\\
111	0\\
112	0\\
113	0\\
114	0\\
115	0\\
116	0\\
117	0\\
118	0\\
119	0\\
120	0\\
121	0\\
122	0\\
123	0\\
124	0\\
125	0\\
126	0\\
127	0\\
128	0\\
129	0\\
130	0\\
131	0\\
132	0\\
133	0\\
134	0\\
135	0\\
136	0\\
137	0\\
138	0\\
139	0\\
140	0\\
141	0\\
142	0\\
143	0\\
144	0\\
145	0\\
146	0\\
147	0\\
148	0\\
149	0\\
150	0\\
151	0\\
152	0\\
153	0\\
154	0\\
155	0\\
156	0\\
157	0\\
158	0\\
159	0\\
160	0\\
161	0\\
162	0\\
163	0\\
164	0\\
165	0\\
166	0\\
167	0\\
168	0\\
169	0\\
170	0\\
171	0\\
172	0\\
173	0\\
174	0\\
175	0\\
176	0\\
177	0\\
178	0\\
179	0\\
180	0\\
181	0\\
182	0\\
183	0\\
184	0\\
185	0\\
186	0\\
187	0\\
188	0\\
189	0\\
190	0\\
191	0\\
192	0\\
193	0\\
194	0\\
195	0\\
196	0\\
197	0\\
198	0\\
199	0\\
200	0\\
201	0\\
202	0\\
203	0\\
204	0\\
205	0\\
206	0\\
207	0\\
208	0\\
209	0\\
210	0\\
211	0\\
212	0\\
213	0\\
214	0\\
215	0\\
216	0\\
217	0\\
218	0\\
219	0\\
220	0\\
221	0\\
222	0\\
223	0\\
224	0\\
225	0\\
226	0\\
227	0\\
228	0\\
229	0\\
230	0\\
231	0\\
232	0\\
233	0\\
234	0\\
235	0\\
236	0\\
237	0\\
238	0\\
239	0\\
240	0\\
241	0\\
242	0\\
243	0\\
244	0\\
245	0\\
246	0\\
247	0\\
248	0\\
249	0\\
250	0\\
251	0\\
252	0\\
253	0\\
254	0\\
255	0\\
256	0\\
257	0\\
258	0\\
259	0\\
260	0\\
261	0\\
262	0\\
263	0\\
264	0\\
265	0\\
266	0\\
267	0\\
268	0\\
269	0\\
270	0\\
271	0\\
272	0\\
273	0\\
274	0\\
275	0\\
276	0\\
277	0\\
278	0\\
279	0\\
280	0\\
281	0\\
282	0\\
283	0\\
284	0\\
285	0\\
286	0\\
287	0\\
288	0\\
289	0\\
290	0\\
291	0\\
292	0\\
293	0\\
294	0\\
295	0\\
296	0\\
297	0\\
298	0\\
299	0\\
300	0\\
301	0\\
302	0\\
303	0\\
304	0\\
305	0\\
306	0\\
307	0\\
308	0\\
309	0\\
310	0\\
311	0\\
312	0\\
313	0\\
314	0\\
315	0\\
316	0\\
317	0\\
318	0\\
319	0\\
320	0\\
321	0\\
322	0\\
323	0\\
324	0\\
325	0\\
326	0\\
327	0\\
328	0\\
329	0\\
330	0\\
331	0\\
332	0\\
333	0\\
334	0\\
335	0\\
336	0\\
337	0\\
338	0\\
339	0\\
340	0\\
341	0\\
342	0\\
343	0\\
344	0\\
345	0\\
346	0\\
347	0\\
348	0\\
349	0\\
350	0\\
351	0\\
352	0\\
353	0\\
354	0\\
355	0\\
356	0\\
357	0\\
358	0\\
359	0\\
360	0\\
361	0\\
362	0\\
363	0\\
364	0\\
365	0\\
366	0\\
367	0\\
368	0\\
369	0\\
370	0\\
371	0\\
372	0\\
373	0\\
374	0\\
375	0\\
376	0\\
377	0\\
378	0\\
379	0\\
380	0\\
381	0\\
382	0\\
383	0\\
384	0\\
385	0\\
386	0\\
387	0\\
388	0\\
389	0\\
390	0\\
391	0\\
392	0\\
393	0\\
394	0\\
395	0\\
396	0\\
397	0\\
398	0\\
399	0\\
400	0\\
401	0\\
402	0\\
403	0\\
404	0\\
405	0\\
406	0\\
407	0\\
408	0\\
409	0\\
410	0\\
411	0\\
412	0\\
413	0\\
414	0\\
415	0\\
416	0\\
417	0\\
418	0\\
419	0\\
420	0\\
421	0\\
422	0\\
423	0\\
424	0\\
425	0\\
426	0\\
427	0\\
428	0\\
429	0\\
430	0\\
431	0\\
432	0\\
433	0\\
434	0\\
435	0\\
436	0\\
437	0\\
438	0\\
439	0\\
440	0\\
441	0\\
442	0\\
443	0\\
444	0\\
445	0\\
446	0\\
447	0\\
448	0\\
449	0\\
450	0\\
451	0\\
452	0\\
453	0\\
454	0\\
455	0\\
456	0\\
457	0\\
458	0\\
459	0\\
460	0\\
461	0\\
462	0\\
463	0\\
464	0\\
465	0\\
466	0\\
467	0\\
468	0\\
469	0\\
470	0\\
471	0\\
472	0\\
473	0\\
474	0\\
475	0\\
476	0\\
477	0\\
478	0\\
479	0\\
480	0\\
481	0\\
482	0\\
483	0\\
484	0\\
485	0\\
486	0\\
487	0\\
488	0\\
489	0\\
490	0\\
491	0\\
492	0\\
493	0\\
494	0\\
495	0\\
496	0\\
497	0\\
498	0\\
499	0\\
500	0\\
501	0\\
502	0\\
503	0\\
504	0\\
505	0\\
506	0\\
507	0\\
508	0\\
509	0\\
510	0\\
511	0\\
512	0\\
513	0\\
514	0\\
515	0\\
516	0\\
517	0\\
518	0\\
519	0\\
520	0\\
521	0\\
522	0\\
523	0\\
524	0\\
525	0\\
526	0\\
527	0\\
528	0\\
529	0\\
530	0\\
531	0\\
532	0\\
533	0\\
534	0\\
535	0\\
536	0\\
537	0\\
538	0\\
539	0\\
540	0\\
541	0\\
542	0\\
543	0\\
544	0\\
545	0\\
546	0\\
547	0\\
548	0\\
549	0\\
550	0\\
551	0\\
552	0\\
553	0\\
554	0\\
555	0\\
556	0\\
557	0\\
558	0\\
559	0\\
560	0\\
561	0\\
562	0\\
563	0\\
564	0\\
565	0\\
566	0\\
567	0\\
568	0\\
569	0\\
570	0\\
571	0\\
572	0\\
573	0\\
574	0\\
575	0\\
576	0\\
577	0\\
578	0\\
579	0\\
580	0\\
581	0\\
582	0\\
583	0\\
584	0\\
585	0\\
586	0\\
587	0\\
588	0\\
589	0\\
590	0\\
591	0\\
592	0\\
593	0\\
594	0\\
595	0\\
596	0\\
597	0\\
598	0\\
599	0\\
600	0\\
};
\addplot [color=mycolor3,solid,forget plot]
  table[row sep=crcr]{%
1	0\\
2	0\\
3	0\\
4	0\\
5	0\\
6	0\\
7	0\\
8	0\\
9	0\\
10	0\\
11	0\\
12	0\\
13	0\\
14	0\\
15	0\\
16	0\\
17	0\\
18	0\\
19	0\\
20	0\\
21	0\\
22	0\\
23	0\\
24	0\\
25	0\\
26	0\\
27	0\\
28	0\\
29	0\\
30	0\\
31	0\\
32	0\\
33	0\\
34	0\\
35	0\\
36	0\\
37	0\\
38	0\\
39	0\\
40	0\\
41	0\\
42	0\\
43	0\\
44	0\\
45	0\\
46	0\\
47	0\\
48	0\\
49	0\\
50	0\\
51	0\\
52	0\\
53	0\\
54	0\\
55	0\\
56	0\\
57	0\\
58	0\\
59	0\\
60	0\\
61	0\\
62	0\\
63	0\\
64	0\\
65	0\\
66	0\\
67	0\\
68	0\\
69	0\\
70	0\\
71	0\\
72	0\\
73	0\\
74	0\\
75	0\\
76	0\\
77	0\\
78	0\\
79	0\\
80	0\\
81	0\\
82	0\\
83	0\\
84	0\\
85	0\\
86	0\\
87	0\\
88	0\\
89	0\\
90	0\\
91	0\\
92	0\\
93	0\\
94	0\\
95	0\\
96	0\\
97	0\\
98	0\\
99	0\\
100	0\\
101	0\\
102	0\\
103	0\\
104	0\\
105	0\\
106	0\\
107	0\\
108	0\\
109	0\\
110	0\\
111	0\\
112	0\\
113	0\\
114	0\\
115	0\\
116	0\\
117	0\\
118	0\\
119	0\\
120	0\\
121	0\\
122	0\\
123	0\\
124	0\\
125	0\\
126	0\\
127	0\\
128	0\\
129	0\\
130	0\\
131	0\\
132	0\\
133	0\\
134	0\\
135	0\\
136	0\\
137	0\\
138	0\\
139	0\\
140	0\\
141	0\\
142	0\\
143	0\\
144	0\\
145	0\\
146	0\\
147	0\\
148	0\\
149	0\\
150	0\\
151	0\\
152	0\\
153	0\\
154	0\\
155	0\\
156	0\\
157	0\\
158	0\\
159	0\\
160	0\\
161	0\\
162	0\\
163	0\\
164	0\\
165	0\\
166	0\\
167	0\\
168	0\\
169	0\\
170	0\\
171	0\\
172	0\\
173	0\\
174	0\\
175	0\\
176	0\\
177	0\\
178	0\\
179	0\\
180	0\\
181	0\\
182	0\\
183	0\\
184	0\\
185	0\\
186	0\\
187	0\\
188	0\\
189	0\\
190	0\\
191	0\\
192	0\\
193	0\\
194	0\\
195	0\\
196	0\\
197	0\\
198	0\\
199	0\\
200	0\\
201	0\\
202	0\\
203	0\\
204	0\\
205	0\\
206	0\\
207	0\\
208	0\\
209	0\\
210	0\\
211	0\\
212	0\\
213	0\\
214	0\\
215	0\\
216	0\\
217	0\\
218	0\\
219	0\\
220	0\\
221	0\\
222	0\\
223	0\\
224	0\\
225	0\\
226	0\\
227	0\\
228	0\\
229	0\\
230	0\\
231	0\\
232	0\\
233	0\\
234	0\\
235	0\\
236	0\\
237	0\\
238	0\\
239	0\\
240	0\\
241	0\\
242	0\\
243	0\\
244	0\\
245	0\\
246	0\\
247	0\\
248	0\\
249	0\\
250	0\\
251	0\\
252	0\\
253	0\\
254	0\\
255	0\\
256	0\\
257	0\\
258	0\\
259	0\\
260	0\\
261	0\\
262	0\\
263	0\\
264	0\\
265	0\\
266	0\\
267	0\\
268	0\\
269	0\\
270	0\\
271	0\\
272	0\\
273	0\\
274	0\\
275	0\\
276	0\\
277	0\\
278	0\\
279	0\\
280	0\\
281	0\\
282	0\\
283	0\\
284	0\\
285	0\\
286	0\\
287	0\\
288	0\\
289	0\\
290	0\\
291	0\\
292	0\\
293	0\\
294	0\\
295	0\\
296	0\\
297	0\\
298	0\\
299	0\\
300	0\\
301	0\\
302	0\\
303	0\\
304	0\\
305	0\\
306	0\\
307	0\\
308	0\\
309	0\\
310	0\\
311	0\\
312	0\\
313	0\\
314	0\\
315	0\\
316	0\\
317	0\\
318	0\\
319	0\\
320	0\\
321	0\\
322	0\\
323	0\\
324	0\\
325	0\\
326	0\\
327	0\\
328	0\\
329	0\\
330	0\\
331	0\\
332	0\\
333	0\\
334	0\\
335	0\\
336	0\\
337	0\\
338	0\\
339	0\\
340	0\\
341	0\\
342	0\\
343	0\\
344	0\\
345	0\\
346	0\\
347	0\\
348	0\\
349	0\\
350	0\\
351	0\\
352	0\\
353	0\\
354	0\\
355	0\\
356	0\\
357	0\\
358	0\\
359	0\\
360	0\\
361	0\\
362	0\\
363	0\\
364	0\\
365	0\\
366	0\\
367	0\\
368	0\\
369	0\\
370	0\\
371	0\\
372	0\\
373	0\\
374	0\\
375	0\\
376	0\\
377	0\\
378	0\\
379	0\\
380	0\\
381	0\\
382	0\\
383	0\\
384	0\\
385	0\\
386	0\\
387	0\\
388	0\\
389	0\\
390	0\\
391	0\\
392	0\\
393	0\\
394	0\\
395	0\\
396	0\\
397	0\\
398	0\\
399	0\\
400	0\\
401	0\\
402	0\\
403	0\\
404	0\\
405	0\\
406	0\\
407	0\\
408	0\\
409	0\\
410	0\\
411	0\\
412	0\\
413	0\\
414	0\\
415	0\\
416	0\\
417	0\\
418	0\\
419	0\\
420	0\\
421	0\\
422	0\\
423	0\\
424	0\\
425	0\\
426	0\\
427	0\\
428	0\\
429	0\\
430	0\\
431	0\\
432	0\\
433	0\\
434	0\\
435	0\\
436	0\\
437	0\\
438	0\\
439	0\\
440	0\\
441	0\\
442	0\\
443	0\\
444	0\\
445	0\\
446	0\\
447	0\\
448	0\\
449	0\\
450	0\\
451	0\\
452	0\\
453	0\\
454	0\\
455	0\\
456	0\\
457	0\\
458	0\\
459	0\\
460	0\\
461	0\\
462	0\\
463	0\\
464	0\\
465	0\\
466	0\\
467	0\\
468	0\\
469	0\\
470	0\\
471	0\\
472	0\\
473	0\\
474	0\\
475	0\\
476	0\\
477	0\\
478	0\\
479	0\\
480	0\\
481	0\\
482	0\\
483	0\\
484	0\\
485	0\\
486	0\\
487	0\\
488	0\\
489	0\\
490	0\\
491	0\\
492	0\\
493	0\\
494	0\\
495	0\\
496	0\\
497	0\\
498	0\\
499	0\\
500	0\\
501	0\\
502	0\\
503	0\\
504	0\\
505	0\\
506	0\\
507	0\\
508	0\\
509	0\\
510	0\\
511	0\\
512	0\\
513	0\\
514	0\\
515	0\\
516	0\\
517	0\\
518	0\\
519	0\\
520	0\\
521	0\\
522	0\\
523	0\\
524	0\\
525	0\\
526	0\\
527	0\\
528	0\\
529	0\\
530	0\\
531	0\\
532	0\\
533	0\\
534	0\\
535	0\\
536	0\\
537	0\\
538	0\\
539	0\\
540	0\\
541	0\\
542	0\\
543	0\\
544	0\\
545	0\\
546	0\\
547	0\\
548	0\\
549	0\\
550	0\\
551	0\\
552	0\\
553	0\\
554	0\\
555	0\\
556	0\\
557	0\\
558	0\\
559	0\\
560	0\\
561	0\\
562	0\\
563	0\\
564	0\\
565	0\\
566	0\\
567	0\\
568	0\\
569	0\\
570	0\\
571	0\\
572	0\\
573	0\\
574	0\\
575	0\\
576	0\\
577	0\\
578	0\\
579	0\\
580	0\\
581	0\\
582	0\\
583	0\\
584	0\\
585	0\\
586	0\\
587	0\\
588	0\\
589	0\\
590	0\\
591	0\\
592	0\\
593	0\\
594	0\\
595	0\\
596	0\\
597	0\\
598	0\\
599	0\\
600	0\\
};
\addplot [color=mycolor4,solid,forget plot]
  table[row sep=crcr]{%
1	0\\
2	0\\
3	0\\
4	0\\
5	0\\
6	0\\
7	0\\
8	0\\
9	0\\
10	0\\
11	0\\
12	0\\
13	0\\
14	0\\
15	0\\
16	0\\
17	0\\
18	0\\
19	0\\
20	0\\
21	0\\
22	0\\
23	0\\
24	0\\
25	0\\
26	0\\
27	0\\
28	0\\
29	0\\
30	0\\
31	0\\
32	0\\
33	0\\
34	0\\
35	0\\
36	0\\
37	0\\
38	0\\
39	0\\
40	0\\
41	0\\
42	0\\
43	0\\
44	0\\
45	0\\
46	0\\
47	0\\
48	0\\
49	0\\
50	0\\
51	0\\
52	0\\
53	0\\
54	0\\
55	0\\
56	0\\
57	0\\
58	0\\
59	0\\
60	0\\
61	0\\
62	0\\
63	0\\
64	0\\
65	0\\
66	0\\
67	0\\
68	0\\
69	0\\
70	0\\
71	0\\
72	0\\
73	0\\
74	0\\
75	0\\
76	0\\
77	0\\
78	0\\
79	0\\
80	0\\
81	0\\
82	0\\
83	0\\
84	0\\
85	0\\
86	0\\
87	0\\
88	0\\
89	0\\
90	0\\
91	0\\
92	0\\
93	0\\
94	0\\
95	0\\
96	0\\
97	0\\
98	0\\
99	0\\
100	0\\
101	0\\
102	0\\
103	0\\
104	0\\
105	0\\
106	0\\
107	0\\
108	0\\
109	0\\
110	0\\
111	0\\
112	0\\
113	0\\
114	0\\
115	0\\
116	0\\
117	0\\
118	0\\
119	0\\
120	0\\
121	0\\
122	0\\
123	0\\
124	0\\
125	0\\
126	0\\
127	0\\
128	0\\
129	0\\
130	0\\
131	0\\
132	0\\
133	0\\
134	0\\
135	0\\
136	0\\
137	0\\
138	0\\
139	0\\
140	0\\
141	0\\
142	0\\
143	0\\
144	0\\
145	0\\
146	0\\
147	0\\
148	0\\
149	0\\
150	0\\
151	0\\
152	0\\
153	0\\
154	0\\
155	0\\
156	0\\
157	0\\
158	0\\
159	0\\
160	0\\
161	0\\
162	0\\
163	0\\
164	0\\
165	0\\
166	0\\
167	0\\
168	0\\
169	0\\
170	0\\
171	0\\
172	0\\
173	0\\
174	0\\
175	0\\
176	0\\
177	0\\
178	0\\
179	0\\
180	0\\
181	0\\
182	0\\
183	0\\
184	0\\
185	0\\
186	0\\
187	0\\
188	0\\
189	0\\
190	0\\
191	0\\
192	0\\
193	0\\
194	0\\
195	0\\
196	0\\
197	0\\
198	0\\
199	0\\
200	0\\
201	0\\
202	0\\
203	0\\
204	0\\
205	0\\
206	0\\
207	0\\
208	0\\
209	0\\
210	0\\
211	0\\
212	0\\
213	0\\
214	0\\
215	0\\
216	0\\
217	0\\
218	0\\
219	0\\
220	0\\
221	0\\
222	0\\
223	0\\
224	0\\
225	0\\
226	0\\
227	0\\
228	0\\
229	0\\
230	0\\
231	0\\
232	0\\
233	0\\
234	0\\
235	0\\
236	0\\
237	0\\
238	0\\
239	0\\
240	0\\
241	0\\
242	0\\
243	0\\
244	0\\
245	0\\
246	0\\
247	0\\
248	0\\
249	0\\
250	0\\
251	0\\
252	0\\
253	0\\
254	0\\
255	0\\
256	0\\
257	0\\
258	0\\
259	0\\
260	0\\
261	0\\
262	0\\
263	0\\
264	0\\
265	0\\
266	0\\
267	0\\
268	0\\
269	0\\
270	0\\
271	0\\
272	0\\
273	0\\
274	0\\
275	0\\
276	0\\
277	0\\
278	0\\
279	0\\
280	0\\
281	0\\
282	0\\
283	0\\
284	0\\
285	0\\
286	0\\
287	0\\
288	0\\
289	0\\
290	0\\
291	0\\
292	0\\
293	0\\
294	0\\
295	0\\
296	0\\
297	0\\
298	0\\
299	0\\
300	0\\
301	0\\
302	0\\
303	0\\
304	0\\
305	0\\
306	0\\
307	0\\
308	0\\
309	0\\
310	0\\
311	0\\
312	0\\
313	0\\
314	0\\
315	0\\
316	0\\
317	0\\
318	0\\
319	0\\
320	0\\
321	0\\
322	0\\
323	0\\
324	0\\
325	0\\
326	0\\
327	0\\
328	0\\
329	0\\
330	0\\
331	0\\
332	0\\
333	0\\
334	0\\
335	0\\
336	0\\
337	0\\
338	0\\
339	0\\
340	0\\
341	0\\
342	0\\
343	0\\
344	0\\
345	0\\
346	0\\
347	0\\
348	0\\
349	0\\
350	0\\
351	0\\
352	0\\
353	0\\
354	0\\
355	0\\
356	0\\
357	0\\
358	0\\
359	0\\
360	0\\
361	0\\
362	0\\
363	0\\
364	0\\
365	0\\
366	0\\
367	0\\
368	0\\
369	0\\
370	0\\
371	0\\
372	0\\
373	0\\
374	0\\
375	0\\
376	0\\
377	0\\
378	0\\
379	0\\
380	0\\
381	0\\
382	0\\
383	0\\
384	0\\
385	0\\
386	0\\
387	0\\
388	0\\
389	0\\
390	0\\
391	0\\
392	0\\
393	0\\
394	0\\
395	0\\
396	0\\
397	0\\
398	0\\
399	0\\
400	0\\
401	0\\
402	0\\
403	0\\
404	0\\
405	0\\
406	0\\
407	0\\
408	0\\
409	0\\
410	0\\
411	0\\
412	0\\
413	0\\
414	0\\
415	0\\
416	0\\
417	0\\
418	0\\
419	0\\
420	0\\
421	0\\
422	0\\
423	0\\
424	0\\
425	0\\
426	0\\
427	0\\
428	0\\
429	0\\
430	0\\
431	0\\
432	0\\
433	0\\
434	0\\
435	0\\
436	0\\
437	0\\
438	0\\
439	0\\
440	0\\
441	0\\
442	0\\
443	0\\
444	0\\
445	0\\
446	0\\
447	0\\
448	0\\
449	0\\
450	0\\
451	0\\
452	0\\
453	0\\
454	0\\
455	0\\
456	0\\
457	0\\
458	0\\
459	0\\
460	0\\
461	0\\
462	0\\
463	0\\
464	0\\
465	0\\
466	0\\
467	0\\
468	0\\
469	0\\
470	0\\
471	0\\
472	0\\
473	0\\
474	0\\
475	0\\
476	0\\
477	0\\
478	0\\
479	0\\
480	0\\
481	0\\
482	0\\
483	0\\
484	0\\
485	0\\
486	0\\
487	0\\
488	0\\
489	0\\
490	0\\
491	0\\
492	0\\
493	0\\
494	0\\
495	0\\
496	0\\
497	0\\
498	0\\
499	0\\
500	0\\
501	0\\
502	0\\
503	0\\
504	0\\
505	0\\
506	0\\
507	0\\
508	0\\
509	0\\
510	0\\
511	0\\
512	0\\
513	0\\
514	0\\
515	0\\
516	0\\
517	0\\
518	0\\
519	0\\
520	0\\
521	0\\
522	0\\
523	0\\
524	0\\
525	0\\
526	0\\
527	0\\
528	0\\
529	0\\
530	0\\
531	0\\
532	0\\
533	0\\
534	0\\
535	0\\
536	0\\
537	0\\
538	0\\
539	0\\
540	0\\
541	0\\
542	0\\
543	0\\
544	0\\
545	0\\
546	0\\
547	0\\
548	0\\
549	0\\
550	0\\
551	0\\
552	0\\
553	0\\
554	0\\
555	0\\
556	0\\
557	0\\
558	0\\
559	0\\
560	0\\
561	0\\
562	0\\
563	0\\
564	0\\
565	0\\
566	0\\
567	0\\
568	0\\
569	0\\
570	0\\
571	0\\
572	0\\
573	0\\
574	0\\
575	0\\
576	0\\
577	0\\
578	0\\
579	0\\
580	0\\
581	0\\
582	0\\
583	0\\
584	0\\
585	0\\
586	0\\
587	0\\
588	0\\
589	0\\
590	0\\
591	0\\
592	0\\
593	0\\
594	0\\
595	0\\
596	0\\
597	0\\
598	0\\
599	0\\
600	0\\
};
\addplot [color=mycolor5,solid,forget plot]
  table[row sep=crcr]{%
1	0\\
2	0\\
3	0\\
4	0\\
5	0\\
6	0\\
7	0\\
8	0\\
9	0\\
10	0\\
11	0\\
12	0\\
13	0\\
14	0\\
15	0\\
16	0\\
17	0\\
18	0\\
19	0\\
20	0\\
21	0\\
22	0\\
23	0\\
24	0\\
25	0\\
26	0\\
27	0\\
28	0\\
29	0\\
30	0\\
31	0\\
32	0\\
33	0\\
34	0\\
35	0\\
36	0\\
37	0\\
38	0\\
39	0\\
40	0\\
41	0\\
42	0\\
43	0\\
44	0\\
45	0\\
46	0\\
47	0\\
48	0\\
49	0\\
50	0\\
51	0\\
52	0\\
53	0\\
54	0\\
55	0\\
56	0\\
57	0\\
58	0\\
59	0\\
60	0\\
61	0\\
62	0\\
63	0\\
64	0\\
65	0\\
66	0\\
67	0\\
68	0\\
69	0\\
70	0\\
71	0\\
72	0\\
73	0\\
74	0\\
75	0\\
76	0\\
77	0\\
78	0\\
79	0\\
80	0\\
81	0\\
82	0\\
83	0\\
84	0\\
85	0\\
86	0\\
87	0\\
88	0\\
89	0\\
90	0\\
91	0\\
92	0\\
93	0\\
94	0\\
95	0\\
96	0\\
97	0\\
98	0\\
99	0\\
100	0\\
101	0\\
102	0\\
103	0\\
104	0\\
105	0\\
106	0\\
107	0\\
108	0\\
109	0\\
110	0\\
111	0\\
112	0\\
113	0\\
114	0\\
115	0\\
116	0\\
117	0\\
118	0\\
119	0\\
120	0\\
121	0\\
122	0\\
123	0\\
124	0\\
125	0\\
126	0\\
127	0\\
128	0\\
129	0\\
130	0\\
131	0\\
132	0\\
133	0\\
134	0\\
135	0\\
136	0\\
137	0\\
138	0\\
139	0\\
140	0\\
141	0\\
142	0\\
143	0\\
144	0\\
145	0\\
146	0\\
147	0\\
148	0\\
149	0\\
150	0\\
151	0\\
152	0\\
153	0\\
154	0\\
155	0\\
156	0\\
157	0\\
158	0\\
159	0\\
160	0\\
161	0\\
162	0\\
163	0\\
164	0\\
165	0\\
166	0\\
167	0\\
168	0\\
169	0\\
170	0\\
171	0\\
172	0\\
173	0\\
174	0\\
175	0\\
176	0\\
177	0\\
178	0\\
179	0\\
180	0\\
181	0\\
182	0\\
183	0\\
184	0\\
185	0\\
186	0\\
187	0\\
188	0\\
189	0\\
190	0\\
191	0\\
192	0\\
193	0\\
194	0\\
195	0\\
196	0\\
197	0\\
198	0\\
199	0\\
200	0\\
201	0\\
202	0\\
203	0\\
204	0\\
205	0\\
206	0\\
207	0\\
208	0\\
209	0\\
210	0\\
211	0\\
212	0\\
213	0\\
214	0\\
215	0\\
216	0\\
217	0\\
218	0\\
219	0\\
220	0\\
221	0\\
222	0\\
223	0\\
224	0\\
225	0\\
226	0\\
227	0\\
228	0\\
229	0\\
230	0\\
231	0\\
232	0\\
233	0\\
234	0\\
235	0\\
236	0\\
237	0\\
238	0\\
239	0\\
240	0\\
241	0\\
242	0\\
243	0\\
244	0\\
245	0\\
246	0\\
247	0\\
248	0\\
249	0\\
250	0\\
251	0\\
252	0\\
253	0\\
254	0\\
255	0\\
256	0\\
257	0\\
258	0\\
259	0\\
260	0\\
261	0\\
262	0\\
263	0\\
264	0\\
265	0\\
266	0\\
267	0\\
268	0\\
269	0\\
270	0\\
271	0\\
272	0\\
273	0\\
274	0\\
275	0\\
276	0\\
277	0\\
278	0\\
279	0\\
280	0\\
281	0\\
282	0\\
283	0\\
284	0\\
285	0\\
286	0\\
287	0\\
288	0\\
289	0\\
290	0\\
291	0\\
292	0\\
293	0\\
294	0\\
295	0\\
296	0\\
297	0\\
298	0\\
299	0\\
300	0\\
301	0\\
302	0\\
303	0\\
304	0\\
305	0\\
306	0\\
307	0\\
308	0\\
309	0\\
310	0\\
311	0\\
312	0\\
313	0\\
314	0\\
315	0\\
316	0\\
317	0\\
318	0\\
319	0\\
320	0\\
321	0\\
322	0\\
323	0\\
324	0\\
325	0\\
326	0\\
327	0\\
328	0\\
329	0\\
330	0\\
331	0\\
332	0\\
333	0\\
334	0\\
335	0\\
336	0\\
337	0\\
338	0\\
339	0\\
340	0\\
341	0\\
342	0\\
343	0\\
344	0\\
345	0\\
346	0\\
347	0\\
348	0\\
349	0\\
350	0\\
351	0\\
352	0\\
353	0\\
354	0\\
355	0\\
356	0\\
357	0\\
358	0\\
359	0\\
360	0\\
361	0\\
362	0\\
363	0\\
364	0\\
365	0\\
366	0\\
367	0\\
368	0\\
369	0\\
370	0\\
371	0\\
372	0\\
373	0\\
374	0\\
375	0\\
376	0\\
377	0\\
378	0\\
379	0\\
380	0\\
381	0\\
382	0\\
383	0\\
384	0\\
385	0\\
386	0\\
387	0\\
388	0\\
389	0\\
390	0\\
391	0\\
392	0\\
393	0\\
394	0\\
395	0\\
396	0\\
397	0\\
398	0\\
399	0\\
400	0\\
401	0\\
402	0\\
403	0\\
404	0\\
405	0\\
406	0\\
407	0\\
408	0\\
409	0\\
410	0\\
411	0\\
412	0\\
413	0\\
414	0\\
415	0\\
416	0\\
417	0\\
418	0\\
419	0\\
420	0\\
421	0\\
422	0\\
423	0\\
424	0\\
425	0\\
426	0\\
427	0\\
428	0\\
429	0\\
430	0\\
431	0\\
432	0\\
433	0\\
434	0\\
435	0\\
436	0\\
437	0\\
438	0\\
439	0\\
440	0\\
441	0\\
442	0\\
443	0\\
444	0\\
445	0\\
446	0\\
447	0\\
448	0\\
449	0\\
450	0\\
451	0\\
452	0\\
453	0\\
454	0\\
455	0\\
456	0\\
457	0\\
458	0\\
459	0\\
460	0\\
461	0\\
462	0\\
463	0\\
464	0\\
465	0\\
466	0\\
467	0\\
468	0\\
469	0\\
470	0\\
471	0\\
472	0\\
473	0\\
474	0\\
475	0\\
476	0\\
477	0\\
478	0\\
479	0\\
480	0\\
481	0\\
482	0\\
483	0\\
484	0\\
485	0\\
486	0\\
487	0\\
488	0\\
489	0\\
490	0\\
491	0\\
492	0\\
493	0\\
494	0\\
495	0\\
496	0\\
497	0\\
498	0\\
499	0\\
500	0\\
501	0\\
502	0\\
503	0\\
504	0\\
505	0\\
506	0\\
507	0\\
508	0\\
509	0\\
510	0\\
511	0\\
512	0\\
513	0\\
514	0\\
515	0\\
516	0\\
517	0\\
518	0\\
519	0\\
520	0\\
521	0\\
522	0\\
523	0\\
524	0\\
525	0\\
526	0\\
527	0\\
528	0\\
529	0\\
530	0\\
531	0\\
532	0\\
533	0\\
534	0\\
535	0\\
536	0\\
537	0\\
538	0\\
539	0\\
540	0\\
541	0\\
542	0\\
543	0\\
544	0\\
545	0\\
546	0\\
547	0\\
548	0\\
549	0\\
550	0\\
551	0\\
552	0\\
553	0\\
554	0\\
555	0\\
556	0\\
557	0\\
558	0\\
559	0\\
560	0\\
561	0\\
562	0\\
563	0\\
564	0\\
565	0\\
566	0\\
567	0\\
568	0\\
569	0\\
570	0\\
571	0\\
572	0\\
573	0\\
574	0\\
575	0\\
576	0\\
577	0\\
578	0\\
579	0\\
580	0\\
581	0\\
582	0\\
583	0\\
584	0\\
585	0\\
586	0\\
587	0\\
588	0\\
589	0\\
590	0\\
591	0\\
592	0\\
593	0\\
594	0\\
595	0\\
596	0\\
597	0\\
598	0\\
599	0\\
600	0\\
};
\addplot [color=mycolor6,solid,forget plot]
  table[row sep=crcr]{%
1	0\\
2	0\\
3	0\\
4	0\\
5	0\\
6	0\\
7	0\\
8	0\\
9	0\\
10	0\\
11	0\\
12	0\\
13	0\\
14	0\\
15	0\\
16	0\\
17	0\\
18	0\\
19	0\\
20	0\\
21	0\\
22	0\\
23	0\\
24	0\\
25	0\\
26	0\\
27	0\\
28	0\\
29	0\\
30	0\\
31	0\\
32	0\\
33	0\\
34	0\\
35	0\\
36	0\\
37	0\\
38	0\\
39	0\\
40	0\\
41	0\\
42	0\\
43	0\\
44	0\\
45	0\\
46	0\\
47	0\\
48	0\\
49	0\\
50	0\\
51	0\\
52	0\\
53	0\\
54	0\\
55	0\\
56	0\\
57	0\\
58	0\\
59	0\\
60	0\\
61	0\\
62	0\\
63	0\\
64	0\\
65	0\\
66	0\\
67	0\\
68	0\\
69	0\\
70	0\\
71	0\\
72	0\\
73	0\\
74	0\\
75	0\\
76	0\\
77	0\\
78	0\\
79	0\\
80	0\\
81	0\\
82	0\\
83	0\\
84	0\\
85	0\\
86	0\\
87	0\\
88	0\\
89	0\\
90	0\\
91	0\\
92	0\\
93	0\\
94	0\\
95	0\\
96	0\\
97	0\\
98	0\\
99	0\\
100	0\\
101	0\\
102	0\\
103	0\\
104	0\\
105	0\\
106	0\\
107	0\\
108	0\\
109	0\\
110	0\\
111	0\\
112	0\\
113	0\\
114	0\\
115	0\\
116	0\\
117	0\\
118	0\\
119	0\\
120	0\\
121	0\\
122	0\\
123	0\\
124	0\\
125	0\\
126	0\\
127	0\\
128	0\\
129	0\\
130	0\\
131	0\\
132	0\\
133	0\\
134	0\\
135	0\\
136	0\\
137	0\\
138	0\\
139	0\\
140	0\\
141	0\\
142	0\\
143	0\\
144	0\\
145	0\\
146	0\\
147	0\\
148	0\\
149	0\\
150	0\\
151	0\\
152	0\\
153	0\\
154	0\\
155	0\\
156	0\\
157	0\\
158	0\\
159	0\\
160	0\\
161	0\\
162	0\\
163	0\\
164	0\\
165	0\\
166	0\\
167	0\\
168	0\\
169	0\\
170	0\\
171	0\\
172	0\\
173	0\\
174	0\\
175	0\\
176	0\\
177	0\\
178	0\\
179	0\\
180	0\\
181	0\\
182	0\\
183	0\\
184	0\\
185	0\\
186	0\\
187	0\\
188	0\\
189	0\\
190	0\\
191	0\\
192	0\\
193	0\\
194	0\\
195	0\\
196	0\\
197	0\\
198	0\\
199	0\\
200	0\\
201	0\\
202	0\\
203	0\\
204	0\\
205	0\\
206	0\\
207	0\\
208	0\\
209	0\\
210	0\\
211	0\\
212	0\\
213	0\\
214	0\\
215	0\\
216	0\\
217	0\\
218	0\\
219	0\\
220	0\\
221	0\\
222	0\\
223	0\\
224	0\\
225	0\\
226	0\\
227	0\\
228	0\\
229	0\\
230	0\\
231	0\\
232	0\\
233	0\\
234	0\\
235	0\\
236	0\\
237	0\\
238	0\\
239	0\\
240	0\\
241	0\\
242	0\\
243	0\\
244	0\\
245	0\\
246	0\\
247	0\\
248	0\\
249	0\\
250	0\\
251	0\\
252	0\\
253	0\\
254	0\\
255	0\\
256	0\\
257	0\\
258	0\\
259	0\\
260	0\\
261	0\\
262	0\\
263	0\\
264	0\\
265	0\\
266	0\\
267	0\\
268	0\\
269	0\\
270	0\\
271	0\\
272	0\\
273	0\\
274	0\\
275	0\\
276	0\\
277	0\\
278	0\\
279	0\\
280	0\\
281	0\\
282	0\\
283	0\\
284	0\\
285	0\\
286	0\\
287	0\\
288	0\\
289	0\\
290	0\\
291	0\\
292	0\\
293	0\\
294	0\\
295	0\\
296	0\\
297	0\\
298	0\\
299	0\\
300	0\\
301	0\\
302	0\\
303	0\\
304	0\\
305	0\\
306	0\\
307	0\\
308	0\\
309	0\\
310	0\\
311	0\\
312	0\\
313	0\\
314	0\\
315	0\\
316	0\\
317	0\\
318	0\\
319	0\\
320	0\\
321	0\\
322	0\\
323	0\\
324	0\\
325	0\\
326	0\\
327	0\\
328	0\\
329	0\\
330	0\\
331	0\\
332	0\\
333	0\\
334	0\\
335	0\\
336	0\\
337	0\\
338	0\\
339	0\\
340	0\\
341	0\\
342	0\\
343	0\\
344	0\\
345	0\\
346	0\\
347	0\\
348	0\\
349	0\\
350	0\\
351	0\\
352	0\\
353	0\\
354	0\\
355	0\\
356	0\\
357	0\\
358	0\\
359	0\\
360	0\\
361	0\\
362	0\\
363	0\\
364	0\\
365	0\\
366	0\\
367	0\\
368	0\\
369	0\\
370	0\\
371	0\\
372	0\\
373	0\\
374	0\\
375	0\\
376	0\\
377	0\\
378	0\\
379	0\\
380	0\\
381	0\\
382	0\\
383	0\\
384	0\\
385	0\\
386	0\\
387	0\\
388	0\\
389	0\\
390	0\\
391	0\\
392	0\\
393	0\\
394	0\\
395	0\\
396	0\\
397	0\\
398	0\\
399	0\\
400	0\\
401	0\\
402	0\\
403	0\\
404	0\\
405	0\\
406	0\\
407	0\\
408	0\\
409	0\\
410	0\\
411	0\\
412	0\\
413	0\\
414	0\\
415	0\\
416	0\\
417	0\\
418	0\\
419	0\\
420	0\\
421	0\\
422	0\\
423	0\\
424	0\\
425	0\\
426	0\\
427	0\\
428	0\\
429	0\\
430	0\\
431	0\\
432	0\\
433	0\\
434	0\\
435	0\\
436	0\\
437	0\\
438	0\\
439	0\\
440	0\\
441	0\\
442	0\\
443	0\\
444	0\\
445	0\\
446	0\\
447	0\\
448	0\\
449	0\\
450	0\\
451	0\\
452	0\\
453	0\\
454	0\\
455	0\\
456	0\\
457	0\\
458	0\\
459	0\\
460	0\\
461	0\\
462	0\\
463	0\\
464	0\\
465	0\\
466	0\\
467	0\\
468	0\\
469	0\\
470	0\\
471	0\\
472	0\\
473	0\\
474	0\\
475	0\\
476	0\\
477	0\\
478	0\\
479	0\\
480	0\\
481	0\\
482	0\\
483	0\\
484	0\\
485	0\\
486	0\\
487	0\\
488	0\\
489	0\\
490	0\\
491	0\\
492	0\\
493	0\\
494	0\\
495	0\\
496	0\\
497	0\\
498	0\\
499	0\\
500	0\\
501	0\\
502	0\\
503	0\\
504	0\\
505	0\\
506	0\\
507	0\\
508	0\\
509	0\\
510	0\\
511	0\\
512	0\\
513	0\\
514	0\\
515	0\\
516	0\\
517	0\\
518	0\\
519	0\\
520	0\\
521	0\\
522	0\\
523	0\\
524	0\\
525	0\\
526	0\\
527	0\\
528	0\\
529	0\\
530	0\\
531	0\\
532	0\\
533	0\\
534	0\\
535	0\\
536	0\\
537	0\\
538	0\\
539	0\\
540	0\\
541	0\\
542	0\\
543	0\\
544	0\\
545	0\\
546	0\\
547	0\\
548	0\\
549	0\\
550	0\\
551	0\\
552	0\\
553	0\\
554	0\\
555	0\\
556	0\\
557	0\\
558	0\\
559	0\\
560	0\\
561	0\\
562	0\\
563	0\\
564	0\\
565	0\\
566	0\\
567	0\\
568	0\\
569	0\\
570	0\\
571	0\\
572	0\\
573	0\\
574	0\\
575	0\\
576	0\\
577	0\\
578	0\\
579	0\\
580	0\\
581	0\\
582	0\\
583	0\\
584	0\\
585	0\\
586	0\\
587	0\\
588	0\\
589	0\\
590	0\\
591	0\\
592	0\\
593	0\\
594	0\\
595	0\\
596	0\\
597	0\\
598	0\\
599	0\\
600	0\\
};
\addplot [color=mycolor7,solid,forget plot]
  table[row sep=crcr]{%
1	0\\
2	0\\
3	0\\
4	0\\
5	0\\
6	0\\
7	0\\
8	0\\
9	0\\
10	0\\
11	0\\
12	0\\
13	0\\
14	0\\
15	0\\
16	0\\
17	0\\
18	0\\
19	0\\
20	0\\
21	0\\
22	0\\
23	0\\
24	0\\
25	0\\
26	0\\
27	0\\
28	0\\
29	0\\
30	0\\
31	0\\
32	0\\
33	0\\
34	0\\
35	0\\
36	0\\
37	0\\
38	0\\
39	0\\
40	0\\
41	0\\
42	0\\
43	0\\
44	0\\
45	0\\
46	0\\
47	0\\
48	0\\
49	0\\
50	0\\
51	0\\
52	0\\
53	0\\
54	0\\
55	0\\
56	0\\
57	0\\
58	0\\
59	0\\
60	0\\
61	0\\
62	0\\
63	0\\
64	0\\
65	0\\
66	0\\
67	0\\
68	0\\
69	0\\
70	0\\
71	0\\
72	0\\
73	0\\
74	0\\
75	0\\
76	0\\
77	0\\
78	0\\
79	0\\
80	0\\
81	0\\
82	0\\
83	0\\
84	0\\
85	0\\
86	0\\
87	0\\
88	0\\
89	0\\
90	0\\
91	0\\
92	0\\
93	0\\
94	0\\
95	0\\
96	0\\
97	0\\
98	0\\
99	0\\
100	0\\
101	0\\
102	0\\
103	0\\
104	0\\
105	0\\
106	0\\
107	0\\
108	0\\
109	0\\
110	0\\
111	0\\
112	0\\
113	0\\
114	0\\
115	0\\
116	0\\
117	0\\
118	0\\
119	0\\
120	0\\
121	0\\
122	0\\
123	0\\
124	0\\
125	0\\
126	0\\
127	0\\
128	0\\
129	0\\
130	0\\
131	0\\
132	0\\
133	0\\
134	0\\
135	0\\
136	0\\
137	0\\
138	0\\
139	0\\
140	0\\
141	0\\
142	0\\
143	0\\
144	0\\
145	0\\
146	0\\
147	0\\
148	0\\
149	0\\
150	0\\
151	0\\
152	0\\
153	0\\
154	0\\
155	0\\
156	0\\
157	0\\
158	0\\
159	0\\
160	0\\
161	0\\
162	0\\
163	0\\
164	0\\
165	0\\
166	0\\
167	0\\
168	0\\
169	0\\
170	0\\
171	0\\
172	0\\
173	0\\
174	0\\
175	0\\
176	0\\
177	0\\
178	0\\
179	0\\
180	0\\
181	0\\
182	0\\
183	0\\
184	0\\
185	0\\
186	0\\
187	0\\
188	0\\
189	0\\
190	0\\
191	0\\
192	0\\
193	0\\
194	0\\
195	0\\
196	0\\
197	0\\
198	0\\
199	0\\
200	0\\
201	0\\
202	0\\
203	0\\
204	0\\
205	0\\
206	0\\
207	0\\
208	0\\
209	0\\
210	0\\
211	0\\
212	0\\
213	0\\
214	0\\
215	0\\
216	0\\
217	0\\
218	0\\
219	0\\
220	0\\
221	0\\
222	0\\
223	0\\
224	0\\
225	0\\
226	0\\
227	0\\
228	0\\
229	0\\
230	0\\
231	0\\
232	0\\
233	0\\
234	0\\
235	0\\
236	0\\
237	0\\
238	0\\
239	0\\
240	0\\
241	0\\
242	0\\
243	0\\
244	0\\
245	0\\
246	0\\
247	0\\
248	0\\
249	0\\
250	0\\
251	0\\
252	0\\
253	0\\
254	0\\
255	0\\
256	0\\
257	0\\
258	0\\
259	0\\
260	0\\
261	0\\
262	0\\
263	0\\
264	0\\
265	0\\
266	0\\
267	0\\
268	0\\
269	0\\
270	0\\
271	0\\
272	0\\
273	0\\
274	0\\
275	0\\
276	0\\
277	0\\
278	0\\
279	0\\
280	0\\
281	0\\
282	0\\
283	0\\
284	0\\
285	0\\
286	0\\
287	0\\
288	0\\
289	0\\
290	0\\
291	0\\
292	0\\
293	0\\
294	0\\
295	0\\
296	0\\
297	0\\
298	0\\
299	0\\
300	0\\
301	0\\
302	0\\
303	0\\
304	0\\
305	0\\
306	0\\
307	0\\
308	0\\
309	0\\
310	0\\
311	0\\
312	0\\
313	0\\
314	0\\
315	0\\
316	0\\
317	0\\
318	0\\
319	0\\
320	0\\
321	0\\
322	0\\
323	0\\
324	0\\
325	0\\
326	0\\
327	0\\
328	0\\
329	0\\
330	0\\
331	0\\
332	0\\
333	0\\
334	0\\
335	0\\
336	0\\
337	0\\
338	0\\
339	0\\
340	0\\
341	0\\
342	0\\
343	0\\
344	0\\
345	0\\
346	0\\
347	0\\
348	0\\
349	0\\
350	0\\
351	0\\
352	0\\
353	0\\
354	0\\
355	0\\
356	0\\
357	0\\
358	0\\
359	0\\
360	0\\
361	0\\
362	0\\
363	0\\
364	0\\
365	0\\
366	0\\
367	0\\
368	0\\
369	0\\
370	0\\
371	0\\
372	0\\
373	0\\
374	0\\
375	0\\
376	0\\
377	0\\
378	0\\
379	0\\
380	0\\
381	0\\
382	0\\
383	0\\
384	0\\
385	0\\
386	0\\
387	0\\
388	0\\
389	0\\
390	0\\
391	0\\
392	0\\
393	0\\
394	0\\
395	0\\
396	0\\
397	0\\
398	0\\
399	0\\
400	0\\
401	0\\
402	0\\
403	0\\
404	0\\
405	0\\
406	0\\
407	0\\
408	0\\
409	0\\
410	0\\
411	0\\
412	0\\
413	0\\
414	0\\
415	0\\
416	0\\
417	0\\
418	0\\
419	0\\
420	0\\
421	0\\
422	0\\
423	0\\
424	0\\
425	0\\
426	0\\
427	0\\
428	0\\
429	0\\
430	0\\
431	0\\
432	0\\
433	0\\
434	0\\
435	0\\
436	0\\
437	0\\
438	0\\
439	0\\
440	0\\
441	0\\
442	0\\
443	0\\
444	0\\
445	0\\
446	0\\
447	0\\
448	0\\
449	0\\
450	0\\
451	0\\
452	0\\
453	0\\
454	0\\
455	0\\
456	0\\
457	0\\
458	0\\
459	0\\
460	0\\
461	0\\
462	0\\
463	0\\
464	0\\
465	0\\
466	0\\
467	0\\
468	0\\
469	0\\
470	0\\
471	0\\
472	0\\
473	0\\
474	0\\
475	0\\
476	0\\
477	0\\
478	0\\
479	0\\
480	0\\
481	0\\
482	0\\
483	0\\
484	0\\
485	0\\
486	0\\
487	0\\
488	0\\
489	0\\
490	0\\
491	0\\
492	0\\
493	0\\
494	0\\
495	0\\
496	0\\
497	0\\
498	0\\
499	0\\
500	0\\
501	0\\
502	0\\
503	0\\
504	0\\
505	0\\
506	0\\
507	0\\
508	0\\
509	0\\
510	0\\
511	0\\
512	0\\
513	0\\
514	0\\
515	0\\
516	0\\
517	0\\
518	0\\
519	0\\
520	0\\
521	0\\
522	0\\
523	0\\
524	0\\
525	0\\
526	0\\
527	0\\
528	0\\
529	0\\
530	0\\
531	0\\
532	0\\
533	0\\
534	0\\
535	0\\
536	0\\
537	0\\
538	0\\
539	0\\
540	0\\
541	0\\
542	0\\
543	0\\
544	0\\
545	0\\
546	0\\
547	0\\
548	0\\
549	0\\
550	0\\
551	0\\
552	0\\
553	0\\
554	0\\
555	0\\
556	0\\
557	0\\
558	0\\
559	0\\
560	0\\
561	0\\
562	0\\
563	0\\
564	0\\
565	0\\
566	0\\
567	0\\
568	0\\
569	0\\
570	0\\
571	0\\
572	0\\
573	0\\
574	0\\
575	0\\
576	0\\
577	0\\
578	0\\
579	0\\
580	0\\
581	0\\
582	0\\
583	0\\
584	0\\
585	0\\
586	0\\
587	0\\
588	0\\
589	0\\
590	0\\
591	0\\
592	0\\
593	0\\
594	0\\
595	0\\
596	0\\
597	0\\
598	0\\
599	0\\
600	0\\
};
\addplot [color=mycolor8,solid,forget plot]
  table[row sep=crcr]{%
1	0\\
2	0\\
3	0\\
4	0\\
5	0\\
6	0\\
7	0\\
8	0\\
9	0\\
10	0\\
11	0\\
12	0\\
13	0\\
14	0\\
15	0\\
16	0\\
17	0\\
18	0\\
19	0\\
20	0\\
21	0\\
22	0\\
23	0\\
24	0\\
25	0\\
26	0\\
27	0\\
28	0\\
29	0\\
30	0\\
31	0\\
32	0\\
33	0\\
34	0\\
35	0\\
36	0\\
37	0\\
38	0\\
39	0\\
40	0\\
41	0\\
42	0\\
43	0\\
44	0\\
45	0\\
46	0\\
47	0\\
48	0\\
49	0\\
50	0\\
51	0\\
52	0\\
53	0\\
54	0\\
55	0\\
56	0\\
57	0\\
58	0\\
59	0\\
60	0\\
61	0\\
62	0\\
63	0\\
64	0\\
65	0\\
66	0\\
67	0\\
68	0\\
69	0\\
70	0\\
71	0\\
72	0\\
73	0\\
74	0\\
75	0\\
76	0\\
77	0\\
78	0\\
79	0\\
80	0\\
81	0\\
82	0\\
83	0\\
84	0\\
85	0\\
86	0\\
87	0\\
88	0\\
89	0\\
90	0\\
91	0\\
92	0\\
93	0\\
94	0\\
95	0\\
96	0\\
97	0\\
98	0\\
99	0\\
100	0\\
101	0\\
102	0\\
103	0\\
104	0\\
105	0\\
106	0\\
107	0\\
108	0\\
109	0\\
110	0\\
111	0\\
112	0\\
113	0\\
114	0\\
115	0\\
116	0\\
117	0\\
118	0\\
119	0\\
120	0\\
121	0\\
122	0\\
123	0\\
124	0\\
125	0\\
126	0\\
127	0\\
128	0\\
129	0\\
130	0\\
131	0\\
132	0\\
133	0\\
134	0\\
135	0\\
136	0\\
137	0\\
138	0\\
139	0\\
140	0\\
141	0\\
142	0\\
143	0\\
144	0\\
145	0\\
146	0\\
147	0\\
148	0\\
149	0\\
150	0\\
151	0\\
152	0\\
153	0\\
154	0\\
155	0\\
156	0\\
157	0\\
158	0\\
159	0\\
160	0\\
161	0\\
162	0\\
163	0\\
164	0\\
165	0\\
166	0\\
167	0\\
168	0\\
169	0\\
170	0\\
171	0\\
172	0\\
173	0\\
174	0\\
175	0\\
176	0\\
177	0\\
178	0\\
179	0\\
180	0\\
181	0\\
182	0\\
183	0\\
184	0\\
185	0\\
186	0\\
187	0\\
188	0\\
189	0\\
190	0\\
191	0\\
192	0\\
193	0\\
194	0\\
195	0\\
196	0\\
197	0\\
198	0\\
199	0\\
200	0\\
201	0\\
202	0\\
203	0\\
204	0\\
205	0\\
206	0\\
207	0\\
208	0\\
209	0\\
210	0\\
211	0\\
212	0\\
213	0\\
214	0\\
215	0\\
216	0\\
217	0\\
218	0\\
219	0\\
220	0\\
221	0\\
222	0\\
223	0\\
224	0\\
225	0\\
226	0\\
227	0\\
228	0\\
229	0\\
230	0\\
231	0\\
232	0\\
233	0\\
234	0\\
235	0\\
236	0\\
237	0\\
238	0\\
239	0\\
240	0\\
241	0\\
242	0\\
243	0\\
244	0\\
245	0\\
246	0\\
247	0\\
248	0\\
249	0\\
250	0\\
251	0\\
252	0\\
253	0\\
254	0\\
255	0\\
256	0\\
257	0\\
258	0\\
259	0\\
260	0\\
261	0\\
262	0\\
263	0\\
264	0\\
265	0\\
266	0\\
267	0\\
268	0\\
269	0\\
270	0\\
271	0\\
272	0\\
273	0\\
274	0\\
275	0\\
276	0\\
277	0\\
278	0\\
279	0\\
280	0\\
281	0\\
282	0\\
283	0\\
284	0\\
285	0\\
286	0\\
287	0\\
288	0\\
289	0\\
290	0\\
291	0\\
292	0\\
293	0\\
294	0\\
295	0\\
296	0\\
297	0\\
298	0\\
299	0\\
300	0\\
301	0\\
302	0\\
303	0\\
304	0\\
305	0\\
306	0\\
307	0\\
308	0\\
309	0\\
310	0\\
311	0\\
312	0\\
313	0\\
314	0\\
315	0\\
316	0\\
317	0\\
318	0\\
319	0\\
320	0\\
321	0\\
322	0\\
323	0\\
324	0\\
325	0\\
326	0\\
327	0\\
328	0\\
329	0\\
330	0\\
331	0\\
332	0\\
333	0\\
334	0\\
335	0\\
336	0\\
337	0\\
338	0\\
339	0\\
340	0\\
341	0\\
342	0\\
343	0\\
344	0\\
345	0\\
346	0\\
347	0\\
348	0\\
349	0\\
350	0\\
351	0\\
352	0\\
353	0\\
354	0\\
355	0\\
356	0\\
357	0\\
358	0\\
359	0\\
360	0\\
361	0\\
362	0\\
363	0\\
364	0\\
365	0\\
366	0\\
367	0\\
368	0\\
369	0\\
370	0\\
371	0\\
372	0\\
373	0\\
374	0\\
375	0\\
376	0\\
377	0\\
378	0\\
379	0\\
380	0\\
381	0\\
382	0\\
383	0\\
384	0\\
385	0\\
386	0\\
387	0\\
388	0\\
389	0\\
390	0\\
391	0\\
392	0\\
393	0\\
394	0\\
395	0\\
396	0\\
397	0\\
398	0\\
399	0\\
400	0\\
401	0\\
402	0\\
403	0\\
404	0\\
405	0\\
406	0\\
407	0\\
408	0\\
409	0\\
410	0\\
411	0\\
412	0\\
413	0\\
414	0\\
415	0\\
416	0\\
417	0\\
418	0\\
419	0\\
420	0\\
421	0\\
422	0\\
423	0\\
424	0\\
425	0\\
426	0\\
427	0\\
428	0\\
429	0\\
430	0\\
431	0\\
432	0\\
433	0\\
434	0\\
435	0\\
436	0\\
437	0\\
438	0\\
439	0\\
440	0\\
441	0\\
442	0\\
443	0\\
444	0\\
445	0\\
446	0\\
447	0\\
448	0\\
449	0\\
450	0\\
451	0\\
452	0\\
453	0\\
454	0\\
455	0\\
456	0\\
457	0\\
458	0\\
459	0\\
460	0\\
461	0\\
462	0\\
463	0\\
464	0\\
465	0\\
466	0\\
467	0\\
468	0\\
469	0\\
470	0\\
471	0\\
472	0\\
473	0\\
474	0\\
475	0\\
476	0\\
477	0\\
478	0\\
479	0\\
480	0\\
481	0\\
482	0\\
483	0\\
484	0\\
485	0\\
486	0\\
487	0\\
488	0\\
489	0\\
490	0\\
491	0\\
492	0\\
493	0\\
494	0\\
495	0\\
496	0\\
497	0\\
498	0\\
499	0\\
500	0\\
501	0\\
502	0\\
503	0\\
504	0\\
505	0\\
506	0\\
507	0\\
508	0\\
509	0\\
510	0\\
511	0\\
512	0\\
513	0\\
514	0\\
515	0\\
516	0\\
517	0\\
518	0\\
519	0\\
520	0\\
521	0\\
522	0\\
523	0\\
524	0\\
525	0\\
526	0\\
527	0\\
528	0\\
529	0\\
530	0\\
531	0\\
532	0\\
533	0\\
534	0\\
535	0\\
536	0\\
537	0\\
538	0\\
539	0\\
540	0\\
541	0\\
542	0\\
543	0\\
544	0\\
545	0\\
546	0\\
547	0\\
548	0\\
549	0\\
550	0\\
551	0\\
552	0\\
553	0\\
554	0\\
555	0\\
556	0\\
557	0\\
558	0\\
559	0\\
560	0\\
561	0\\
562	0\\
563	0\\
564	0\\
565	0\\
566	0\\
567	0\\
568	0\\
569	0\\
570	0\\
571	0\\
572	0\\
573	0\\
574	0\\
575	0\\
576	0\\
577	0\\
578	0\\
579	0\\
580	0\\
581	0\\
582	0\\
583	0\\
584	0\\
585	0\\
586	0\\
587	0\\
588	0\\
589	0\\
590	0\\
591	0\\
592	0\\
593	0\\
594	0\\
595	0\\
596	0\\
597	0\\
598	0\\
599	0\\
600	0\\
};
\addplot [color=blue!25!mycolor7,solid,forget plot]
  table[row sep=crcr]{%
1	0\\
2	0\\
3	0\\
4	0\\
5	0\\
6	0\\
7	0\\
8	0\\
9	0\\
10	0\\
11	0\\
12	0\\
13	0\\
14	0\\
15	0\\
16	0\\
17	0\\
18	0\\
19	0\\
20	0\\
21	0\\
22	0\\
23	0\\
24	0\\
25	0\\
26	0\\
27	0\\
28	0\\
29	0\\
30	0\\
31	0\\
32	0\\
33	0\\
34	0\\
35	0\\
36	0\\
37	0\\
38	0\\
39	0\\
40	0\\
41	0\\
42	0\\
43	0\\
44	0\\
45	0\\
46	0\\
47	0\\
48	0\\
49	0\\
50	0\\
51	0\\
52	0\\
53	0\\
54	0\\
55	0\\
56	0\\
57	0\\
58	0\\
59	0\\
60	0\\
61	0\\
62	0\\
63	0\\
64	0\\
65	0\\
66	0\\
67	0\\
68	0\\
69	0\\
70	0\\
71	0\\
72	0\\
73	0\\
74	0\\
75	0\\
76	0\\
77	0\\
78	0\\
79	0\\
80	0\\
81	0\\
82	0\\
83	0\\
84	0\\
85	0\\
86	0\\
87	0\\
88	0\\
89	0\\
90	0\\
91	0\\
92	0\\
93	0\\
94	0\\
95	0\\
96	0\\
97	0\\
98	0\\
99	0\\
100	0\\
101	0\\
102	0\\
103	0\\
104	0\\
105	0\\
106	0\\
107	0\\
108	0\\
109	0\\
110	0\\
111	0\\
112	0\\
113	0\\
114	0\\
115	0\\
116	0\\
117	0\\
118	0\\
119	0\\
120	0\\
121	0\\
122	0\\
123	0\\
124	0\\
125	0\\
126	0\\
127	0\\
128	0\\
129	0\\
130	0\\
131	0\\
132	0\\
133	0\\
134	0\\
135	0\\
136	0\\
137	0\\
138	0\\
139	0\\
140	0\\
141	0\\
142	0\\
143	0\\
144	0\\
145	0\\
146	0\\
147	0\\
148	0\\
149	0\\
150	0\\
151	0\\
152	0\\
153	0\\
154	0\\
155	0\\
156	0\\
157	0\\
158	0\\
159	0\\
160	0\\
161	0\\
162	0\\
163	0\\
164	0\\
165	0\\
166	0\\
167	0\\
168	0\\
169	0\\
170	0\\
171	0\\
172	0\\
173	0\\
174	0\\
175	0\\
176	0\\
177	0\\
178	0\\
179	0\\
180	0\\
181	0\\
182	0\\
183	0\\
184	0\\
185	0\\
186	0\\
187	0\\
188	0\\
189	0\\
190	0\\
191	0\\
192	0\\
193	0\\
194	0\\
195	0\\
196	0\\
197	0\\
198	0\\
199	0\\
200	0\\
201	0\\
202	0\\
203	0\\
204	0\\
205	0\\
206	0\\
207	0\\
208	0\\
209	0\\
210	0\\
211	0\\
212	0\\
213	0\\
214	0\\
215	0\\
216	0\\
217	0\\
218	0\\
219	0\\
220	0\\
221	0\\
222	0\\
223	0\\
224	0\\
225	0\\
226	0\\
227	0\\
228	0\\
229	0\\
230	0\\
231	0\\
232	0\\
233	0\\
234	0\\
235	0\\
236	0\\
237	0\\
238	0\\
239	0\\
240	0\\
241	0\\
242	0\\
243	0\\
244	0\\
245	0\\
246	0\\
247	0\\
248	0\\
249	0\\
250	0\\
251	0\\
252	0\\
253	0\\
254	0\\
255	0\\
256	0\\
257	0\\
258	0\\
259	0\\
260	0\\
261	0\\
262	0\\
263	0\\
264	0\\
265	0\\
266	0\\
267	0\\
268	0\\
269	0\\
270	0\\
271	0\\
272	0\\
273	0\\
274	0\\
275	0\\
276	0\\
277	0\\
278	0\\
279	0\\
280	0\\
281	0\\
282	0\\
283	0\\
284	0\\
285	0\\
286	0\\
287	0\\
288	0\\
289	0\\
290	0\\
291	0\\
292	0\\
293	0\\
294	0\\
295	0\\
296	0\\
297	0\\
298	0\\
299	0\\
300	0\\
301	0\\
302	0\\
303	0\\
304	0\\
305	0\\
306	0\\
307	0\\
308	0\\
309	0\\
310	0\\
311	0\\
312	0\\
313	0\\
314	0\\
315	0\\
316	0\\
317	0\\
318	0\\
319	0\\
320	0\\
321	0\\
322	0\\
323	0\\
324	0\\
325	0\\
326	0\\
327	0\\
328	0\\
329	0\\
330	0\\
331	0\\
332	0\\
333	0\\
334	0\\
335	0\\
336	0\\
337	0\\
338	0\\
339	0\\
340	0\\
341	0\\
342	0\\
343	0\\
344	0\\
345	0\\
346	0\\
347	0\\
348	0\\
349	0\\
350	0\\
351	0\\
352	0\\
353	0\\
354	0\\
355	0\\
356	0\\
357	0\\
358	0\\
359	0\\
360	0\\
361	0\\
362	0\\
363	0\\
364	0\\
365	0\\
366	0\\
367	0\\
368	0\\
369	0\\
370	0\\
371	0\\
372	0\\
373	0\\
374	0\\
375	0\\
376	0\\
377	0\\
378	0\\
379	0\\
380	0\\
381	0\\
382	0\\
383	0\\
384	0\\
385	0\\
386	0\\
387	0\\
388	0\\
389	0\\
390	0\\
391	0\\
392	0\\
393	0\\
394	0\\
395	0\\
396	0\\
397	0\\
398	0\\
399	0\\
400	0\\
401	0\\
402	0\\
403	0\\
404	0\\
405	0\\
406	0\\
407	0\\
408	0\\
409	0\\
410	0\\
411	0\\
412	0\\
413	0\\
414	0\\
415	0\\
416	0\\
417	0\\
418	0\\
419	0\\
420	0\\
421	0\\
422	0\\
423	0\\
424	0\\
425	0\\
426	0\\
427	0\\
428	0\\
429	0\\
430	0\\
431	0\\
432	0\\
433	0\\
434	0\\
435	0\\
436	0\\
437	0\\
438	0\\
439	0\\
440	0\\
441	0\\
442	0\\
443	0\\
444	0\\
445	0\\
446	0\\
447	0\\
448	0\\
449	0\\
450	0\\
451	0\\
452	0\\
453	0\\
454	0\\
455	0\\
456	0\\
457	0\\
458	0\\
459	0\\
460	0\\
461	0\\
462	0\\
463	0\\
464	0\\
465	0\\
466	0\\
467	0\\
468	0\\
469	0\\
470	0\\
471	0\\
472	0\\
473	0\\
474	0\\
475	0\\
476	0\\
477	0\\
478	0\\
479	0\\
480	0\\
481	0\\
482	0\\
483	0\\
484	0\\
485	0\\
486	0\\
487	0\\
488	0\\
489	0\\
490	0\\
491	0\\
492	0\\
493	0\\
494	0\\
495	0\\
496	0\\
497	0\\
498	0\\
499	0\\
500	0\\
501	0\\
502	0\\
503	0\\
504	0\\
505	0\\
506	0\\
507	0\\
508	0\\
509	0\\
510	0\\
511	0\\
512	0\\
513	0\\
514	0\\
515	0\\
516	0\\
517	0\\
518	0\\
519	0\\
520	0\\
521	0\\
522	0\\
523	0\\
524	0\\
525	0\\
526	0\\
527	0\\
528	0\\
529	0\\
530	0\\
531	0\\
532	0\\
533	0\\
534	0\\
535	0\\
536	0\\
537	0\\
538	0\\
539	0\\
540	0\\
541	0\\
542	0\\
543	0\\
544	0\\
545	0\\
546	0\\
547	0\\
548	0\\
549	0\\
550	0\\
551	0\\
552	0\\
553	0\\
554	0\\
555	0\\
556	0\\
557	0\\
558	0\\
559	0\\
560	0\\
561	0\\
562	0\\
563	0\\
564	0\\
565	0\\
566	0\\
567	0\\
568	0\\
569	0\\
570	0\\
571	0\\
572	0\\
573	0\\
574	0\\
575	0\\
576	0\\
577	0\\
578	0\\
579	0\\
580	0\\
581	0\\
582	0\\
583	0\\
584	0\\
585	0\\
586	0\\
587	0\\
588	0\\
589	0\\
590	0\\
591	0\\
592	0\\
593	0\\
594	0\\
595	0\\
596	0\\
597	0\\
598	0\\
599	0\\
600	0\\
};
\addplot [color=mycolor9,solid,forget plot]
  table[row sep=crcr]{%
1	0\\
2	0\\
3	0\\
4	0\\
5	0\\
6	0\\
7	0\\
8	0\\
9	0\\
10	0\\
11	0\\
12	0\\
13	0\\
14	0\\
15	0\\
16	0\\
17	0\\
18	0\\
19	0\\
20	0\\
21	0\\
22	0\\
23	0\\
24	0\\
25	0\\
26	0\\
27	0\\
28	0\\
29	0\\
30	0\\
31	0\\
32	0\\
33	0\\
34	0\\
35	0\\
36	0\\
37	0\\
38	0\\
39	0\\
40	0\\
41	0\\
42	0\\
43	0\\
44	0\\
45	0\\
46	0\\
47	0\\
48	0\\
49	0\\
50	0\\
51	0\\
52	0\\
53	0\\
54	0\\
55	0\\
56	0\\
57	0\\
58	0\\
59	0\\
60	0\\
61	0\\
62	0\\
63	0\\
64	0\\
65	0\\
66	0\\
67	0\\
68	0\\
69	0\\
70	0\\
71	0\\
72	0\\
73	0\\
74	0\\
75	0\\
76	0\\
77	0\\
78	0\\
79	0\\
80	0\\
81	0\\
82	0\\
83	0\\
84	0\\
85	0\\
86	0\\
87	0\\
88	0\\
89	0\\
90	0\\
91	0\\
92	0\\
93	0\\
94	0\\
95	0\\
96	0\\
97	0\\
98	0\\
99	0\\
100	0\\
101	0\\
102	0\\
103	0\\
104	0\\
105	0\\
106	0\\
107	0\\
108	0\\
109	0\\
110	0\\
111	0\\
112	0\\
113	0\\
114	0\\
115	0\\
116	0\\
117	0\\
118	0\\
119	0\\
120	0\\
121	0\\
122	0\\
123	0\\
124	0\\
125	0\\
126	0\\
127	0\\
128	0\\
129	0\\
130	0\\
131	0\\
132	0\\
133	0\\
134	0\\
135	0\\
136	0\\
137	0\\
138	0\\
139	0\\
140	0\\
141	0\\
142	0\\
143	0\\
144	0\\
145	0\\
146	0\\
147	0\\
148	0\\
149	0\\
150	0\\
151	0\\
152	0\\
153	0\\
154	0\\
155	0\\
156	0\\
157	0\\
158	0\\
159	0\\
160	0\\
161	0\\
162	0\\
163	0\\
164	0\\
165	0\\
166	0\\
167	0\\
168	0\\
169	0\\
170	0\\
171	0\\
172	0\\
173	0\\
174	0\\
175	0\\
176	0\\
177	0\\
178	0\\
179	0\\
180	0\\
181	0\\
182	0\\
183	0\\
184	0\\
185	0\\
186	0\\
187	0\\
188	0\\
189	0\\
190	0\\
191	0\\
192	0\\
193	0\\
194	0\\
195	0\\
196	0\\
197	0\\
198	0\\
199	0\\
200	0\\
201	0\\
202	0\\
203	0\\
204	0\\
205	0\\
206	0\\
207	0\\
208	0\\
209	0\\
210	0\\
211	0\\
212	0\\
213	0\\
214	0\\
215	0\\
216	0\\
217	0\\
218	0\\
219	0\\
220	0\\
221	0\\
222	0\\
223	0\\
224	0\\
225	0\\
226	0\\
227	0\\
228	0\\
229	0\\
230	0\\
231	0\\
232	0\\
233	0\\
234	0\\
235	0\\
236	0\\
237	0\\
238	0\\
239	0\\
240	0\\
241	0\\
242	0\\
243	0\\
244	0\\
245	0\\
246	0\\
247	0\\
248	0\\
249	0\\
250	0\\
251	0\\
252	0\\
253	0\\
254	0\\
255	0\\
256	0\\
257	0\\
258	0\\
259	0\\
260	0\\
261	0\\
262	0\\
263	0\\
264	0\\
265	0\\
266	0\\
267	0\\
268	0\\
269	0\\
270	0\\
271	0\\
272	0\\
273	0\\
274	0\\
275	0\\
276	0\\
277	0\\
278	0\\
279	0\\
280	0\\
281	0\\
282	0\\
283	0\\
284	0\\
285	0\\
286	0\\
287	0\\
288	0\\
289	0\\
290	0\\
291	0\\
292	0\\
293	0\\
294	0\\
295	0\\
296	0\\
297	0\\
298	0\\
299	0\\
300	0\\
301	0\\
302	0\\
303	0\\
304	0\\
305	0\\
306	0\\
307	0\\
308	0\\
309	0\\
310	0\\
311	0\\
312	0\\
313	0\\
314	0\\
315	0\\
316	0\\
317	0\\
318	0\\
319	0\\
320	0\\
321	0\\
322	0\\
323	0\\
324	0\\
325	0\\
326	0\\
327	0\\
328	0\\
329	0\\
330	0\\
331	0\\
332	0\\
333	0\\
334	0\\
335	0\\
336	0\\
337	0\\
338	0\\
339	0\\
340	0\\
341	0\\
342	0\\
343	0\\
344	0\\
345	0\\
346	0\\
347	0\\
348	0\\
349	0\\
350	0\\
351	0\\
352	0\\
353	0\\
354	0\\
355	0\\
356	0\\
357	0\\
358	0\\
359	0\\
360	0\\
361	0\\
362	0\\
363	0\\
364	0\\
365	0\\
366	0\\
367	0\\
368	0\\
369	0\\
370	0\\
371	0\\
372	0\\
373	0\\
374	0\\
375	0\\
376	0\\
377	0\\
378	0\\
379	0\\
380	0\\
381	0\\
382	0\\
383	0\\
384	0\\
385	0\\
386	0\\
387	0\\
388	0\\
389	0\\
390	0\\
391	0\\
392	0\\
393	0\\
394	0\\
395	0\\
396	0\\
397	0\\
398	0\\
399	0\\
400	0\\
401	0\\
402	0\\
403	0\\
404	0\\
405	0\\
406	0\\
407	0\\
408	0\\
409	0\\
410	0\\
411	0\\
412	0\\
413	0\\
414	0\\
415	0\\
416	0\\
417	0\\
418	0\\
419	0\\
420	0\\
421	0\\
422	0\\
423	0\\
424	0\\
425	0\\
426	0\\
427	0\\
428	0\\
429	0\\
430	0\\
431	0\\
432	0\\
433	0\\
434	0\\
435	0\\
436	0\\
437	0\\
438	0\\
439	0\\
440	0\\
441	0\\
442	0\\
443	0\\
444	0\\
445	0\\
446	0\\
447	0\\
448	0\\
449	0\\
450	0\\
451	0\\
452	0\\
453	0\\
454	0\\
455	0\\
456	0\\
457	0\\
458	0\\
459	0\\
460	0\\
461	0\\
462	0\\
463	0\\
464	0\\
465	0\\
466	0\\
467	0\\
468	0\\
469	0\\
470	0\\
471	0\\
472	0\\
473	0\\
474	0\\
475	0\\
476	0\\
477	0\\
478	0\\
479	0\\
480	0\\
481	0\\
482	0\\
483	0\\
484	0\\
485	0\\
486	0\\
487	0\\
488	0\\
489	0\\
490	0\\
491	0\\
492	0\\
493	0\\
494	0\\
495	0\\
496	0\\
497	0\\
498	0\\
499	0\\
500	0\\
501	0\\
502	0\\
503	0\\
504	0\\
505	0\\
506	0\\
507	0\\
508	0\\
509	0\\
510	0\\
511	0\\
512	0\\
513	0\\
514	0\\
515	0\\
516	0\\
517	0\\
518	0\\
519	0\\
520	0\\
521	0\\
522	0\\
523	0\\
524	0\\
525	0\\
526	0\\
527	0\\
528	0\\
529	0\\
530	0\\
531	0\\
532	0\\
533	0\\
534	0\\
535	0\\
536	0\\
537	0\\
538	0\\
539	0\\
540	0\\
541	0\\
542	0\\
543	0\\
544	0\\
545	0\\
546	0\\
547	0\\
548	0\\
549	0\\
550	0\\
551	0\\
552	0\\
553	0\\
554	0\\
555	0\\
556	0\\
557	0\\
558	0\\
559	0\\
560	0\\
561	0\\
562	0\\
563	0\\
564	0\\
565	0\\
566	0\\
567	0\\
568	0\\
569	0\\
570	0\\
571	0\\
572	0\\
573	0\\
574	0\\
575	0\\
576	0\\
577	0\\
578	0\\
579	0\\
580	0\\
581	0\\
582	0\\
583	0\\
584	0\\
585	0\\
586	0\\
587	0\\
588	0\\
589	0\\
590	0\\
591	0\\
592	0\\
593	0\\
594	0\\
595	0\\
596	0\\
597	0\\
598	0\\
599	0\\
600	0\\
};
\addplot [color=blue!50!mycolor7,solid,forget plot]
  table[row sep=crcr]{%
1	0\\
2	0\\
3	0\\
4	0\\
5	0\\
6	0\\
7	0\\
8	0\\
9	0\\
10	0\\
11	0\\
12	0\\
13	0\\
14	0\\
15	0\\
16	0\\
17	0\\
18	0\\
19	0\\
20	0\\
21	0\\
22	0\\
23	0\\
24	0\\
25	0\\
26	0\\
27	0\\
28	0\\
29	0\\
30	0\\
31	0\\
32	0\\
33	0\\
34	0\\
35	0\\
36	0\\
37	0\\
38	0\\
39	0\\
40	0\\
41	0\\
42	0\\
43	0\\
44	0\\
45	0\\
46	0\\
47	0\\
48	0\\
49	0\\
50	0\\
51	0\\
52	0\\
53	0\\
54	0\\
55	0\\
56	0\\
57	0\\
58	0\\
59	0\\
60	0\\
61	0\\
62	0\\
63	0\\
64	0\\
65	0\\
66	0\\
67	0\\
68	0\\
69	0\\
70	0\\
71	0\\
72	0\\
73	0\\
74	0\\
75	0\\
76	0\\
77	0\\
78	0\\
79	0\\
80	0\\
81	0\\
82	0\\
83	0\\
84	0\\
85	0\\
86	0\\
87	0\\
88	0\\
89	0\\
90	0\\
91	0\\
92	0\\
93	0\\
94	0\\
95	0\\
96	0\\
97	0\\
98	0\\
99	0\\
100	0\\
101	0\\
102	0\\
103	0\\
104	0\\
105	0\\
106	0\\
107	0\\
108	0\\
109	0\\
110	0\\
111	0\\
112	0\\
113	0\\
114	0\\
115	0\\
116	0\\
117	0\\
118	0\\
119	0\\
120	0\\
121	0\\
122	0\\
123	0\\
124	0\\
125	0\\
126	0\\
127	0\\
128	0\\
129	0\\
130	0\\
131	0\\
132	0\\
133	0\\
134	0\\
135	0\\
136	0\\
137	0\\
138	0\\
139	0\\
140	0\\
141	0\\
142	0\\
143	0\\
144	0\\
145	0\\
146	0\\
147	0\\
148	0\\
149	0\\
150	0\\
151	0\\
152	0\\
153	0\\
154	0\\
155	0\\
156	0\\
157	0\\
158	0\\
159	0\\
160	0\\
161	0\\
162	0\\
163	0\\
164	0\\
165	0\\
166	0\\
167	0\\
168	0\\
169	0\\
170	0\\
171	0\\
172	0\\
173	0\\
174	0\\
175	0\\
176	0\\
177	0\\
178	0\\
179	0\\
180	0\\
181	0\\
182	0\\
183	0\\
184	0\\
185	0\\
186	0\\
187	0\\
188	0\\
189	0\\
190	0\\
191	0\\
192	0\\
193	0\\
194	0\\
195	0\\
196	0\\
197	0\\
198	0\\
199	0\\
200	0\\
201	0\\
202	0\\
203	0\\
204	0\\
205	0\\
206	0\\
207	0\\
208	0\\
209	0\\
210	0\\
211	0\\
212	0\\
213	0\\
214	0\\
215	0\\
216	0\\
217	0\\
218	0\\
219	0\\
220	0\\
221	0\\
222	0\\
223	0\\
224	0\\
225	0\\
226	0\\
227	0\\
228	0\\
229	0\\
230	0\\
231	0\\
232	0\\
233	0\\
234	0\\
235	0\\
236	0\\
237	0\\
238	0\\
239	0\\
240	0\\
241	0\\
242	0\\
243	0\\
244	0\\
245	0\\
246	0\\
247	0\\
248	0\\
249	0\\
250	0\\
251	0\\
252	0\\
253	0\\
254	0\\
255	0\\
256	0\\
257	0\\
258	0\\
259	0\\
260	0\\
261	0\\
262	0\\
263	0\\
264	0\\
265	0\\
266	0\\
267	0\\
268	0\\
269	0\\
270	0\\
271	0\\
272	0\\
273	0\\
274	0\\
275	0\\
276	0\\
277	0\\
278	0\\
279	0\\
280	0\\
281	0\\
282	0\\
283	0\\
284	0\\
285	0\\
286	0\\
287	0\\
288	0\\
289	0\\
290	0\\
291	0\\
292	0\\
293	0\\
294	0\\
295	0\\
296	0\\
297	0\\
298	0\\
299	0\\
300	0\\
301	0\\
302	0\\
303	0\\
304	0\\
305	0\\
306	0\\
307	0\\
308	0\\
309	0\\
310	0\\
311	0\\
312	0\\
313	0\\
314	0\\
315	0\\
316	0\\
317	0\\
318	0\\
319	0\\
320	0\\
321	0\\
322	0\\
323	0\\
324	0\\
325	0\\
326	0\\
327	0\\
328	0\\
329	0\\
330	0\\
331	0\\
332	0\\
333	0\\
334	0\\
335	0\\
336	0\\
337	0\\
338	0\\
339	0\\
340	0\\
341	0\\
342	0\\
343	0\\
344	0\\
345	0\\
346	0\\
347	0\\
348	0\\
349	0\\
350	0\\
351	0\\
352	0\\
353	0\\
354	0\\
355	0\\
356	0\\
357	0\\
358	0\\
359	0\\
360	0\\
361	0\\
362	0\\
363	0\\
364	0\\
365	0\\
366	0\\
367	0\\
368	0\\
369	0\\
370	0\\
371	0\\
372	0\\
373	0\\
374	0\\
375	0\\
376	0\\
377	0\\
378	0\\
379	0\\
380	0\\
381	0\\
382	0\\
383	0\\
384	0\\
385	0\\
386	0\\
387	0\\
388	0\\
389	0\\
390	0\\
391	0\\
392	0\\
393	0\\
394	0\\
395	0\\
396	0\\
397	0\\
398	0\\
399	0\\
400	0\\
401	0\\
402	0\\
403	0\\
404	0\\
405	0\\
406	0\\
407	0\\
408	0\\
409	0\\
410	0\\
411	0\\
412	0\\
413	0\\
414	0\\
415	0\\
416	0\\
417	0\\
418	0\\
419	0\\
420	0\\
421	0\\
422	0\\
423	0\\
424	0\\
425	0\\
426	0\\
427	0\\
428	0\\
429	0\\
430	0\\
431	0\\
432	0\\
433	0\\
434	0\\
435	0\\
436	0\\
437	0\\
438	0\\
439	0\\
440	0\\
441	0\\
442	0\\
443	0\\
444	0\\
445	0\\
446	0\\
447	0\\
448	0\\
449	0\\
450	0\\
451	0\\
452	0\\
453	0\\
454	0\\
455	0\\
456	0\\
457	0\\
458	0\\
459	0\\
460	0\\
461	0\\
462	0\\
463	0\\
464	0\\
465	0\\
466	0\\
467	0\\
468	0\\
469	0\\
470	0\\
471	0\\
472	0\\
473	0\\
474	0\\
475	0\\
476	0\\
477	0\\
478	0\\
479	0\\
480	0\\
481	0\\
482	0\\
483	0\\
484	0\\
485	0\\
486	0\\
487	0\\
488	0\\
489	0\\
490	0\\
491	0\\
492	0\\
493	0\\
494	0\\
495	0\\
496	0\\
497	0\\
498	0\\
499	0\\
500	0\\
501	0\\
502	0\\
503	0\\
504	0\\
505	0\\
506	0\\
507	0\\
508	0\\
509	0\\
510	0\\
511	0\\
512	0\\
513	0\\
514	0\\
515	0\\
516	0\\
517	0\\
518	0\\
519	0\\
520	0\\
521	0\\
522	0\\
523	0\\
524	0\\
525	0\\
526	0\\
527	0\\
528	0\\
529	0\\
530	0\\
531	0\\
532	0\\
533	0\\
534	0\\
535	0\\
536	0\\
537	0\\
538	0\\
539	0\\
540	0\\
541	0\\
542	0\\
543	0\\
544	0\\
545	0\\
546	0\\
547	0\\
548	0\\
549	0\\
550	0\\
551	0\\
552	0\\
553	0\\
554	0\\
555	0\\
556	0\\
557	0\\
558	0\\
559	0\\
560	0\\
561	0\\
562	0\\
563	0\\
564	0\\
565	0\\
566	0\\
567	0\\
568	0\\
569	0\\
570	0\\
571	0\\
572	0\\
573	0\\
574	0\\
575	0\\
576	0\\
577	0\\
578	0\\
579	0\\
580	0\\
581	0\\
582	0\\
583	0\\
584	0\\
585	0\\
586	0\\
587	0\\
588	0\\
589	0\\
590	0\\
591	0\\
592	0\\
593	0\\
594	0\\
595	0\\
596	0\\
597	0\\
598	0\\
599	0\\
600	0\\
};
\addplot [color=blue!40!mycolor9,solid,forget plot]
  table[row sep=crcr]{%
1	0\\
2	0\\
3	0\\
4	0\\
5	0\\
6	0\\
7	0\\
8	0\\
9	0\\
10	0\\
11	0\\
12	0\\
13	0\\
14	0\\
15	0\\
16	0\\
17	0\\
18	0\\
19	0\\
20	0\\
21	0\\
22	0\\
23	0\\
24	0\\
25	0\\
26	0\\
27	0\\
28	0\\
29	0\\
30	0\\
31	0\\
32	0\\
33	0\\
34	0\\
35	0\\
36	0\\
37	0\\
38	0\\
39	0\\
40	0\\
41	0\\
42	0\\
43	0\\
44	0\\
45	0\\
46	0\\
47	0\\
48	0\\
49	0\\
50	0\\
51	0\\
52	0\\
53	0\\
54	0\\
55	0\\
56	0\\
57	0\\
58	0\\
59	0\\
60	0\\
61	0\\
62	0\\
63	0\\
64	0\\
65	0\\
66	0\\
67	0\\
68	0\\
69	0\\
70	0\\
71	0\\
72	0\\
73	0\\
74	0\\
75	0\\
76	0\\
77	0\\
78	0\\
79	0\\
80	0\\
81	0\\
82	0\\
83	0\\
84	0\\
85	0\\
86	0\\
87	0\\
88	0\\
89	0\\
90	0\\
91	0\\
92	0\\
93	0\\
94	0\\
95	0\\
96	0\\
97	0\\
98	0\\
99	0\\
100	0\\
101	0\\
102	0\\
103	0\\
104	0\\
105	0\\
106	0\\
107	0\\
108	0\\
109	0\\
110	0\\
111	0\\
112	0\\
113	0\\
114	0\\
115	0\\
116	0\\
117	0\\
118	0\\
119	0\\
120	0\\
121	0\\
122	0\\
123	0\\
124	0\\
125	0\\
126	0\\
127	0\\
128	0\\
129	0\\
130	0\\
131	0\\
132	0\\
133	0\\
134	0\\
135	0\\
136	0\\
137	0\\
138	0\\
139	0\\
140	0\\
141	0\\
142	0\\
143	0\\
144	0\\
145	0\\
146	0\\
147	0\\
148	0\\
149	0\\
150	0\\
151	0\\
152	0\\
153	0\\
154	0\\
155	0\\
156	0\\
157	0\\
158	0\\
159	0\\
160	0\\
161	0\\
162	0\\
163	0\\
164	0\\
165	0\\
166	0\\
167	0\\
168	0\\
169	0\\
170	0\\
171	0\\
172	0\\
173	0\\
174	0\\
175	0\\
176	0\\
177	0\\
178	0\\
179	0\\
180	0\\
181	0\\
182	0\\
183	0\\
184	0\\
185	0\\
186	0\\
187	0\\
188	0\\
189	0\\
190	0\\
191	0\\
192	0\\
193	0\\
194	0\\
195	0\\
196	0\\
197	0\\
198	0\\
199	0\\
200	0\\
201	0\\
202	0\\
203	0\\
204	0\\
205	0\\
206	0\\
207	0\\
208	0\\
209	0\\
210	0\\
211	0\\
212	0\\
213	0\\
214	0\\
215	0\\
216	0\\
217	0\\
218	0\\
219	0\\
220	0\\
221	0\\
222	0\\
223	0\\
224	0\\
225	0\\
226	0\\
227	0\\
228	0\\
229	0\\
230	0\\
231	0\\
232	0\\
233	0\\
234	0\\
235	0\\
236	0\\
237	0\\
238	0\\
239	0\\
240	0\\
241	0\\
242	0\\
243	0\\
244	0\\
245	0\\
246	0\\
247	0\\
248	0\\
249	0\\
250	0\\
251	0\\
252	0\\
253	0\\
254	0\\
255	0\\
256	0\\
257	0\\
258	0\\
259	0\\
260	0\\
261	0\\
262	0\\
263	0\\
264	0\\
265	0\\
266	0\\
267	0\\
268	0\\
269	0\\
270	0\\
271	0\\
272	0\\
273	0\\
274	0\\
275	0\\
276	0\\
277	0\\
278	0\\
279	0\\
280	0\\
281	0\\
282	0\\
283	0\\
284	0\\
285	0\\
286	0\\
287	0\\
288	0\\
289	0\\
290	0\\
291	0\\
292	0\\
293	0\\
294	0\\
295	0\\
296	0\\
297	0\\
298	0\\
299	0\\
300	0\\
301	0\\
302	0\\
303	0\\
304	0\\
305	0\\
306	0\\
307	0\\
308	0\\
309	0\\
310	0\\
311	0\\
312	0\\
313	0\\
314	0\\
315	0\\
316	0\\
317	0\\
318	0\\
319	0\\
320	0\\
321	0\\
322	0\\
323	0\\
324	0\\
325	0\\
326	0\\
327	0\\
328	0\\
329	0\\
330	0\\
331	0\\
332	0\\
333	0\\
334	0\\
335	0\\
336	0\\
337	0\\
338	0\\
339	0\\
340	0\\
341	0\\
342	0\\
343	0\\
344	0\\
345	0\\
346	0\\
347	0\\
348	0\\
349	0\\
350	0\\
351	0\\
352	0\\
353	0\\
354	0\\
355	0\\
356	0\\
357	0\\
358	0\\
359	0\\
360	0\\
361	0\\
362	0\\
363	0\\
364	0\\
365	0\\
366	0\\
367	0\\
368	0\\
369	0\\
370	0\\
371	0\\
372	0\\
373	0\\
374	0\\
375	0\\
376	0\\
377	0\\
378	0\\
379	0\\
380	0\\
381	0\\
382	0\\
383	0\\
384	0\\
385	0\\
386	0\\
387	0\\
388	0\\
389	0\\
390	0\\
391	0\\
392	0\\
393	0\\
394	0\\
395	0\\
396	0\\
397	0\\
398	0\\
399	0\\
400	0\\
401	0\\
402	0\\
403	0\\
404	0\\
405	0\\
406	0\\
407	0\\
408	0\\
409	0\\
410	0\\
411	0\\
412	0\\
413	0\\
414	0\\
415	0\\
416	0\\
417	0\\
418	0\\
419	0\\
420	0\\
421	0\\
422	0\\
423	0\\
424	0\\
425	0\\
426	0\\
427	0\\
428	0\\
429	0\\
430	0\\
431	0\\
432	0\\
433	0\\
434	0\\
435	0\\
436	0\\
437	0\\
438	0\\
439	0\\
440	0\\
441	0\\
442	0\\
443	0\\
444	0\\
445	0\\
446	0\\
447	0\\
448	0\\
449	0\\
450	0\\
451	0\\
452	0\\
453	0\\
454	0\\
455	0\\
456	0\\
457	0\\
458	0\\
459	0\\
460	0\\
461	0\\
462	0\\
463	0\\
464	0\\
465	0\\
466	0\\
467	0\\
468	0\\
469	0\\
470	0\\
471	0\\
472	0\\
473	0\\
474	0\\
475	0\\
476	0\\
477	0\\
478	0\\
479	0\\
480	0\\
481	0\\
482	0\\
483	0\\
484	0\\
485	0\\
486	0\\
487	0\\
488	0\\
489	0\\
490	0\\
491	0\\
492	0\\
493	0\\
494	0\\
495	0\\
496	0\\
497	0\\
498	0\\
499	0\\
500	0\\
501	0\\
502	0\\
503	0\\
504	0\\
505	0\\
506	0\\
507	0\\
508	0\\
509	0\\
510	0\\
511	0\\
512	0\\
513	0\\
514	0\\
515	0\\
516	0\\
517	0\\
518	0\\
519	0\\
520	0\\
521	0\\
522	0\\
523	0\\
524	0\\
525	0\\
526	0\\
527	0\\
528	0\\
529	0\\
530	0\\
531	0\\
532	0\\
533	0\\
534	0\\
535	0\\
536	0\\
537	0\\
538	0\\
539	0\\
540	0\\
541	0\\
542	0\\
543	0\\
544	0\\
545	0\\
546	0\\
547	0\\
548	0\\
549	0\\
550	0\\
551	0\\
552	0\\
553	0\\
554	0\\
555	0\\
556	0\\
557	0\\
558	0\\
559	0\\
560	0\\
561	0\\
562	0\\
563	0\\
564	0\\
565	0\\
566	0\\
567	0\\
568	0\\
569	0\\
570	0\\
571	0\\
572	0\\
573	0\\
574	0\\
575	0\\
576	0\\
577	0\\
578	0\\
579	0\\
580	0\\
581	0\\
582	0\\
583	0\\
584	0\\
585	0\\
586	0\\
587	0\\
588	0\\
589	0\\
590	0\\
591	0\\
592	0\\
593	0\\
594	0\\
595	0\\
596	0\\
597	0\\
598	0\\
599	0\\
600	0\\
};
\addplot [color=blue!75!mycolor7,solid,forget plot]
  table[row sep=crcr]{%
1	0\\
2	0\\
3	0\\
4	0\\
5	0\\
6	0\\
7	0\\
8	0\\
9	0\\
10	0\\
11	0\\
12	0\\
13	0\\
14	0\\
15	0\\
16	0\\
17	0\\
18	0\\
19	0\\
20	0\\
21	0\\
22	0\\
23	0\\
24	0\\
25	0\\
26	0\\
27	0\\
28	0\\
29	0\\
30	0\\
31	0\\
32	0\\
33	0\\
34	0\\
35	0\\
36	0\\
37	0\\
38	0\\
39	0\\
40	0\\
41	0\\
42	0\\
43	0\\
44	0\\
45	0\\
46	0\\
47	0\\
48	0\\
49	0\\
50	0\\
51	0\\
52	0\\
53	0\\
54	0\\
55	0\\
56	0\\
57	0\\
58	0\\
59	0\\
60	0\\
61	0\\
62	0\\
63	0\\
64	0\\
65	0\\
66	0\\
67	0\\
68	0\\
69	0\\
70	0\\
71	0\\
72	0\\
73	0\\
74	0\\
75	0\\
76	0\\
77	0\\
78	0\\
79	0\\
80	0\\
81	0\\
82	0\\
83	0\\
84	0\\
85	0\\
86	0\\
87	0\\
88	0\\
89	0\\
90	0\\
91	0\\
92	0\\
93	0\\
94	0\\
95	0\\
96	0\\
97	0\\
98	0\\
99	0\\
100	0\\
101	0\\
102	0\\
103	0\\
104	0\\
105	0\\
106	0\\
107	0\\
108	0\\
109	0\\
110	0\\
111	0\\
112	0\\
113	0\\
114	0\\
115	0\\
116	0\\
117	0\\
118	0\\
119	0\\
120	0\\
121	0\\
122	0\\
123	0\\
124	0\\
125	0\\
126	0\\
127	0\\
128	0\\
129	0\\
130	0\\
131	0\\
132	0\\
133	0\\
134	0\\
135	0\\
136	0\\
137	0\\
138	0\\
139	0\\
140	0\\
141	0\\
142	0\\
143	0\\
144	0\\
145	0\\
146	0\\
147	0\\
148	0\\
149	0\\
150	0\\
151	0\\
152	0\\
153	0\\
154	0\\
155	0\\
156	0\\
157	0\\
158	0\\
159	0\\
160	0\\
161	0\\
162	0\\
163	0\\
164	0\\
165	0\\
166	0\\
167	0\\
168	0\\
169	0\\
170	0\\
171	0\\
172	0\\
173	0\\
174	0\\
175	0\\
176	0\\
177	0\\
178	0\\
179	0\\
180	0\\
181	0\\
182	0\\
183	0\\
184	0\\
185	0\\
186	0\\
187	0\\
188	0\\
189	0\\
190	0\\
191	0\\
192	0\\
193	0\\
194	0\\
195	0\\
196	0\\
197	0\\
198	0\\
199	0\\
200	0\\
201	0\\
202	0\\
203	0\\
204	0\\
205	0\\
206	0\\
207	0\\
208	0\\
209	0\\
210	0\\
211	0\\
212	0\\
213	0\\
214	0\\
215	0\\
216	0\\
217	0\\
218	0\\
219	0\\
220	0\\
221	0\\
222	0\\
223	0\\
224	0\\
225	0\\
226	0\\
227	0\\
228	0\\
229	0\\
230	0\\
231	0\\
232	0\\
233	0\\
234	0\\
235	0\\
236	0\\
237	0\\
238	0\\
239	0\\
240	0\\
241	0\\
242	0\\
243	0\\
244	0\\
245	0\\
246	0\\
247	0\\
248	0\\
249	0\\
250	0\\
251	0\\
252	0\\
253	0\\
254	0\\
255	0\\
256	0\\
257	0\\
258	0\\
259	0\\
260	0\\
261	0\\
262	0\\
263	0\\
264	0\\
265	0\\
266	0\\
267	0\\
268	0\\
269	0\\
270	0\\
271	0\\
272	0\\
273	0\\
274	0\\
275	0\\
276	0\\
277	0\\
278	0\\
279	0\\
280	0\\
281	0\\
282	0\\
283	0\\
284	0\\
285	0\\
286	0\\
287	0\\
288	0\\
289	0\\
290	0\\
291	0\\
292	0\\
293	0\\
294	0\\
295	0\\
296	0\\
297	0\\
298	0\\
299	0\\
300	0\\
301	0\\
302	0\\
303	0\\
304	0\\
305	0\\
306	0\\
307	0\\
308	0\\
309	0\\
310	0\\
311	0\\
312	0\\
313	0\\
314	0\\
315	0\\
316	0\\
317	0\\
318	0\\
319	0\\
320	0\\
321	0\\
322	0\\
323	0\\
324	0\\
325	0\\
326	0\\
327	0\\
328	0\\
329	0\\
330	0\\
331	0\\
332	0\\
333	0\\
334	0\\
335	0\\
336	0\\
337	0\\
338	0\\
339	0\\
340	0\\
341	0\\
342	0\\
343	0\\
344	0\\
345	0\\
346	0\\
347	0\\
348	0\\
349	0\\
350	0\\
351	0\\
352	0\\
353	0\\
354	0\\
355	0\\
356	0\\
357	0\\
358	0\\
359	0\\
360	0\\
361	0\\
362	0\\
363	0\\
364	0\\
365	0\\
366	0\\
367	0\\
368	0\\
369	0\\
370	0\\
371	0\\
372	0\\
373	0\\
374	0\\
375	0\\
376	0\\
377	0\\
378	0\\
379	0\\
380	0\\
381	0\\
382	0\\
383	0\\
384	0\\
385	0\\
386	0\\
387	0\\
388	0\\
389	0\\
390	0\\
391	0\\
392	0\\
393	0\\
394	0\\
395	0\\
396	0\\
397	0\\
398	0\\
399	0\\
400	0\\
401	0\\
402	0\\
403	0\\
404	0\\
405	0\\
406	0\\
407	0\\
408	0\\
409	0\\
410	0\\
411	0\\
412	0\\
413	0\\
414	0\\
415	0\\
416	0\\
417	0\\
418	0\\
419	0\\
420	0\\
421	0\\
422	0\\
423	0\\
424	0\\
425	0\\
426	0\\
427	0\\
428	0\\
429	0\\
430	0\\
431	0\\
432	0\\
433	0\\
434	0\\
435	0\\
436	0\\
437	0\\
438	0\\
439	0\\
440	0\\
441	0\\
442	0\\
443	0\\
444	0\\
445	0\\
446	0\\
447	0\\
448	0\\
449	0\\
450	0\\
451	0\\
452	0\\
453	0\\
454	0\\
455	0\\
456	0\\
457	0\\
458	0\\
459	0\\
460	0\\
461	0\\
462	0\\
463	0\\
464	0\\
465	0\\
466	0\\
467	0\\
468	0\\
469	0\\
470	0\\
471	0\\
472	0\\
473	0\\
474	0\\
475	0\\
476	0\\
477	0\\
478	0\\
479	0\\
480	0\\
481	0\\
482	0\\
483	0\\
484	0\\
485	0\\
486	0\\
487	0\\
488	0\\
489	0\\
490	0\\
491	0\\
492	0\\
493	0\\
494	0\\
495	0\\
496	0\\
497	0\\
498	0\\
499	0\\
500	0\\
501	0\\
502	0\\
503	0\\
504	0\\
505	0\\
506	0\\
507	0\\
508	0\\
509	0\\
510	0\\
511	0\\
512	0\\
513	0\\
514	0\\
515	0\\
516	0\\
517	0\\
518	0\\
519	0\\
520	0\\
521	0\\
522	0\\
523	0\\
524	0\\
525	0\\
526	0\\
527	0\\
528	0\\
529	0\\
530	0\\
531	0\\
532	0\\
533	0\\
534	0\\
535	0\\
536	0\\
537	0\\
538	0\\
539	0\\
540	0\\
541	0\\
542	0\\
543	0\\
544	0\\
545	0\\
546	0\\
547	0\\
548	0\\
549	0\\
550	0\\
551	0\\
552	0\\
553	0\\
554	0\\
555	0\\
556	0\\
557	0\\
558	0\\
559	0\\
560	0\\
561	0\\
562	0\\
563	0\\
564	0\\
565	0\\
566	0\\
567	0\\
568	0\\
569	0\\
570	0\\
571	0\\
572	0\\
573	0\\
574	0\\
575	0\\
576	0\\
577	0\\
578	0\\
579	0\\
580	0\\
581	0\\
582	0\\
583	0\\
584	0\\
585	0\\
586	0\\
587	0\\
588	0\\
589	0\\
590	0\\
591	0\\
592	0\\
593	0\\
594	0\\
595	0\\
596	0\\
597	0\\
598	0\\
599	0\\
600	0\\
};
\addplot [color=blue!80!mycolor9,solid,forget plot]
  table[row sep=crcr]{%
1	0.000420463240150569\\
2	0.00042044912030799\\
3	0.000420434759003375\\
4	0.000420420152089871\\
5	0.00042040529534938\\
6	0.000420390184491366\\
7	0.00042037481515161\\
8	0.00042035918289092\\
9	0.000420343283193896\\
10	0.000420327111467586\\
11	0.000420310663040172\\
12	0.000420293933159629\\
13	0.000420276916992343\\
14	0.000420259609621713\\
15	0.000420242006046732\\
16	0.00042022410118055\\
17	0.000420205889849007\\
18	0.000420187366789113\\
19	0.000420168526647563\\
20	0.000420149363979176\\
21	0.000420129873245317\\
22	0.000420110048812314\\
23	0.000420089884949808\\
24	0.000420069375829147\\
25	0.00042004851552164\\
26	0.00042002729799689\\
27	0.000420005717121061\\
28	0.000419983766655072\\
29	0.000419961440252826\\
30	0.000419938731459381\\
31	0.000419915633709058\\
32	0.0004198921403236\\
33	0.000419868244510194\\
34	0.000419843939359556\\
35	0.000419819217843911\\
36	0.000419794072814985\\
37	0.000419768497001938\\
38	0.000419742483009261\\
39	0.000419716023314677\\
40	0.000419689110266941\\
41	0.000419661736083651\\
42	0.000419633892849017\\
43	0.000419605572511561\\
44	0.00041957676688182\\
45	0.000419547467629971\\
46	0.000419517666283459\\
47	0.00041948735422453\\
48	0.000419456522687782\\
49	0.000419425162757618\\
50	0.000419393265365691\\
51	0.000419360821288323\\
52	0.000419327821143804\\
53	0.000419294255389739\\
54	0.000419260114320285\\
55	0.000419225388063368\\
56	0.000419190066577843\\
57	0.00041915413965062\\
58	0.000419117596893731\\
59	0.00041908042774133\\
60	0.000419042621446685\\
61	0.000419004167079077\\
62	0.000418965053520673\\
63	0.000418925269463327\\
64	0.000418884803405359\\
65	0.000418843643648231\\
66	0.000418801778293208\\
67	0.000418759195237954\\
68	0.000418715882173041\\
69	0.000418671826578452\\
70	0.000418627015719973\\
71	0.000418581436645575\\
72	0.000418535076181664\\
73	0.000418487920929352\\
74	0.000418439957260604\\
75	0.000418391171314359\\
76	0.000418341548992521\\
77	0.000418291075956001\\
78	0.000418239737620543\\
79	0.000418187519152627\\
80	0.000418134405465184\\
81	0.000418080381213302\\
82	0.000418025430789865\\
83	0.000417969538321074\\
84	0.000417912687661938\\
85	0.000417854862391662\\
86	0.000417796045808958\\
87	0.000417736220927311\\
88	0.000417675370470104\\
89	0.000417613476865731\\
90	0.00041755052224259\\
91	0.000417486488423966\\
92	0.000417421356922902\\
93	0.000417355108936909\\
94	0.000417287725342635\\
95	0.000417219186690436\\
96	0.000417149473198828\\
97	0.000417078564748892\\
98	0.000417006440878541\\
99	0.000416933080776717\\
100	0.000416858463277496\\
101	0.00041678256685407\\
102	0.000416705369612641\\
103	0.000416626849286235\\
104	0.000416546983228358\\
105	0.000416465748406587\\
106	0.000416383121396063\\
107	0.000416299078372831\\
108	0.000416213595107108\\
109	0.000416126646956417\\
110	0.000416038208858606\\
111	0.000415948255324737\\
112	0.000415856760431929\\
113	0.00041576369781595\\
114	0.000415669040663792\\
115	0.000415572761706077\\
116	0.000415474833209363\\
117	0.000415375226968243\\
118	0.000415273914297402\\
119	0.000415170866023496\\
120	0.000415066052476872\\
121	0.000414959443483184\\
122	0.000414851008354854\\
123	0.000414740715882372\\
124	0.000414628534325447\\
125	0.000414514431404022\\
126	0.000414398374289128\\
127	0.00041428032959356\\
128	0.000414160263362431\\
129	0.000414038141063515\\
130	0.000413913927577448\\
131	0.000413787587187775\\
132	0.000413659083570779\\
133	0.000413528379785166\\
134	0.000413395438261569\\
135	0.000413260220791854\\
136	0.000413122688518262\\
137	0.000412982801922331\\
138	0.000412840520813672\\
139	0.000412695804318507\\
140	0.000412548610868033\\
141	0.000412398898186595\\
142	0.000412246623279623\\
143	0.000412091742421269\\
144	0.000411934211141827\\
145	0.000411773984214908\\
146	0.00041161101564502\\
147	0.000411445258655372\\
148	0.000411276665673553\\
149	0.000411105188317988\\
150	0.000410930777384189\\
151	0.000410753382830757\\
152	0.000410572953765137\\
153	0.000410389438429166\\
154	0.00041020278418429\\
155	0.000410012937496649\\
156	0.00040981984392177\\
157	0.000409623448089111\\
158	0.000409423693686277\\
159	0.000409220523442977\\
160	0.000409013879114696\\
161	0.000408803701466113\\
162	0.000408589930254217\\
163	0.00040837250421111\\
164	0.000408151361026537\\
165	0.00040792643733012\\
166	0.00040769766867327\\
167	0.000407464989510785\\
168	0.000407228333182126\\
169	0.000406987631892374\\
170	0.000406742816692882\\
171	0.000406493817461535\\
172	0.0004062405628827\\
173	0.000405982980426843\\
174	0.000405720996329757\\
175	0.000405454535571414\\
176	0.000405183521854547\\
177	0.000404907877582701\\
178	0.000404627523838044\\
179	0.000404342380358698\\
180	0.000404052365515714\\
181	0.000403757396289626\\
182	0.000403457388246604\\
183	0.000403152255514178\\
184	0.000402841910756524\\
185	0.000402526265149358\\
186	0.000402205228354325\\
187	0.000401878708492977\\
188	0.000401546612120275\\
189	0.000401208844197626\\
190	0.00040086530806543\\
191	0.000400515905415176\\
192	0.000400160536260948\\
193	0.000399799098910555\\
194	0.000399431489936029\\
195	0.000399057604143654\\
196	0.000398677334543464\\
197	0.000398290572318125\\
198	0.000397897206791365\\
199	0.000397497125395721\\
200	0.000397090213639784\\
201	0.000396676355074823\\
202	0.000396255431260803\\
203	0.00039582732173178\\
204	0.000395391903960692\\
205	0.00039494905332346\\
206	0.000394498643062507\\
207	0.000394040544249511\\
208	0.000393574625747573\\
209	0.000393100754172621\\
210	0.000392618793854148\\
211	0.000392128606795174\\
212	0.000391630052631514\\
213	0.00039112298859028\\
214	0.000390607269447579\\
215	0.000390082747485471\\
216	0.000389549272448078\\
217	0.000389006691496911\\
218	0.000388454849165321\\
219	0.000387893587312119\\
220	0.000387322745074294\\
221	0.000386742158818885\\
222	0.000386151662093866\\
223	0.000385551085578178\\
224	0.000384940257030721\\
225	0.000384319001238465\\
226	0.000383687139963456\\
227	0.000383044491888938\\
228	0.00038239087256427\\
229	0.000381726094348952\\
230	0.000381049966355412\\
231	0.000380362294390771\\
232	0.00037966288089746\\
233	0.000378951524892641\\
234	0.000378228021906455\\
235	0.000377492163919073\\
236	0.000376743739296484\\
237	0.000375982532725003\\
238	0.000375208325144516\\
239	0.000374420893680352\\
240	0.000373620011573821\\
241	0.000372805448111359\\
242	0.000371976968552242\\
243	0.00037113433405487\\
244	0.000370277301601518\\
245	0.000369405623921625\\
246	0.00036851904941349\\
247	0.000367617322064392\\
248	0.000366700181369074\\
249	0.000365767362246593\\
250	0.000364818594955471\\
251	0.000363853605007071\\
252	0.000362872113077254\\
253	0.000361873834916212\\
254	0.000360858481256441\\
255	0.000359825757718859\\
256	0.000358775364716957\\
257	0.000357706997359041\\
258	0.000356620345348432\\
259	0.000355515092881627\\
260	0.000354390918544418\\
261	0.000353247495205839\\
262	0.000352084489909975\\
263	0.000350901563765589\\
264	0.000349698371833468\\
265	0.000348474563011527\\
266	0.000347229779917538\\
267	0.000345963658769561\\
268	0.000344675829263897\\
269	0.000343365914450595\\
270	0.000342033530606426\\
271	0.000340678287105166\\
272	0.000339299786285226\\
273	0.000337897623314623\\
274	0.000336471386054643\\
275	0.000335020654925319\\
276	0.000333545002775104\\
277	0.000332043994732079\\
278	0.000330517187994447\\
279	0.000328964131722927\\
280	0.000327384366888339\\
281	0.000325777426113161\\
282	0.000324142833509996\\
283	0.00032248010451688\\
284	0.000320788745729423\\
285	0.000319068254729721\\
286	0.000317318119912109\\
287	0.000315537820305653\\
288	0.000313726825393466\\
289	0.00031188459492877\\
290	0.000310010578747858\\
291	0.000308104216579848\\
292	0.00030616493785344\\
293	0.000304192161500579\\
294	0.000302185295757328\\
295	0.000300143737961838\\
296	0.000298066874349785\\
297	0.000295954079847361\\
298	0.000293804717861986\\
299	0.000291618140071164\\
300	0.00028939368620967\\
301	0.00028713068385552\\
302	0.000284828448215171\\
303	0.00028248628190848\\
304	0.000280103474754065\\
305	0.00027767930355581\\
306	0.000275213031891497\\
307	0.000272703909904661\\
308	0.000270151174100987\\
309	0.000267554047150566\\
310	0.000264911737695903\\
311	0.000262223440161519\\
312	0.000259488334553529\\
313	0.000256705586246752\\
314	0.00025387434585603\\
315	0.00025099374935309\\
316	0.000248062917857218\\
317	0.000245080957487294\\
318	0.00024204695931954\\
319	0.000238959999380692\\
320	0.00023581913868224\\
321	0.000232623423301923\\
322	0.00022937188451976\\
323	0.000226063539016568\\
324	0.000222697389144192\\
325	0.000219272423277816\\
326	0.000215787616262068\\
327	0.000212241929964235\\
328	0.000208634313949584\\
329	0.000204963706295905\\
330	0.000201229034566454\\
331	0.000197429216963232\\
332	0.000193563163685222\\
333	0.000189629778519623\\
334	0.000185627960697774\\
335	0.000181556607051661\\
336	0.000177414614511688\\
337	0.000173200882991794\\
338	0.000168914318714217\\
339	0.000164553838033065\\
340	0.000160118371823967\\
341	0.000155606870515935\\
342	0.000151018309852072\\
343	0.000146351697477284\\
344	0.000141606080464596\\
345	0.000136780553906929\\
346	0.000131874270718485\\
347	0.000126886452809903\\
348	0.000121816403823762\\
349	0.000116663523643309\\
350	0.000111427324916498\\
351	0.000106107451871358\\
352	0.000100703701736782\\
353	9.52160491260074e-05\\
354	8.96446737914473e-05\\
355	8.39899922321503e-05\\
356	7.82526937707919e-05\\
357	7.24337819585433e-05\\
358	6.65346219905049e-05\\
359	6.05569928351092e-05\\
360	5.45031451081293e-05\\
361	4.83758688722205e-05\\
362	4.21785594234741e-05\\
363	3.59152575781306e-05\\
364	2.95905654047414e-05\\
365	2.3209056409204e-05\\
366	1.67726731719349e-05\\
367	1.026963175326e-05\\
368	3.62063219975855e-06\\
369	0\\
370	0\\
371	0\\
372	0\\
373	0\\
374	0\\
375	0\\
376	0\\
377	0\\
378	0\\
379	0\\
380	0\\
381	0\\
382	0\\
383	0\\
384	0\\
385	0\\
386	0\\
387	0\\
388	0\\
389	0\\
390	0\\
391	0\\
392	0\\
393	0\\
394	0\\
395	0\\
396	0\\
397	0\\
398	0\\
399	0\\
400	0\\
401	0\\
402	0\\
403	0\\
404	0\\
405	0\\
406	0\\
407	0\\
408	0\\
409	0\\
410	0\\
411	0\\
412	0\\
413	0\\
414	0\\
415	0\\
416	0\\
417	0\\
418	0\\
419	0\\
420	0\\
421	0\\
422	0\\
423	0\\
424	0\\
425	0\\
426	0\\
427	0\\
428	0\\
429	0\\
430	0\\
431	0\\
432	0\\
433	0\\
434	0\\
435	0\\
436	0\\
437	0\\
438	0\\
439	0\\
440	0\\
441	0\\
442	0\\
443	0\\
444	0\\
445	0\\
446	0\\
447	0\\
448	0\\
449	0\\
450	0\\
451	0\\
452	0\\
453	0\\
454	0\\
455	0\\
456	0\\
457	0\\
458	0\\
459	0\\
460	0\\
461	0\\
462	0\\
463	0\\
464	0\\
465	0\\
466	0\\
467	0\\
468	0\\
469	0\\
470	0\\
471	0\\
472	0\\
473	0\\
474	0\\
475	0\\
476	0\\
477	0\\
478	0\\
479	0\\
480	0\\
481	0\\
482	0\\
483	0\\
484	0\\
485	0\\
486	0\\
487	0\\
488	0\\
489	0\\
490	0\\
491	0\\
492	0\\
493	0\\
494	0\\
495	0\\
496	0\\
497	0\\
498	0\\
499	0\\
500	0\\
501	0\\
502	0\\
503	0\\
504	0\\
505	0\\
506	0\\
507	0\\
508	0\\
509	0\\
510	0\\
511	0\\
512	0\\
513	0\\
514	0\\
515	0\\
516	0\\
517	0\\
518	0\\
519	0\\
520	0\\
521	0\\
522	0\\
523	0\\
524	0\\
525	0\\
526	0\\
527	0\\
528	0\\
529	0\\
530	0\\
531	0\\
532	0\\
533	0\\
534	0\\
535	0\\
536	0\\
537	0\\
538	0\\
539	0\\
540	0\\
541	0\\
542	0\\
543	0\\
544	0\\
545	0\\
546	0\\
547	0\\
548	0\\
549	0\\
550	0\\
551	0\\
552	0\\
553	0\\
554	0\\
555	0\\
556	0\\
557	0\\
558	0\\
559	0\\
560	0\\
561	0\\
562	0\\
563	0\\
564	0\\
565	0\\
566	0\\
567	0\\
568	0\\
569	0\\
570	0\\
571	0\\
572	0\\
573	0\\
574	0\\
575	0\\
576	0\\
577	0\\
578	0\\
579	0\\
580	0\\
581	0\\
582	0\\
583	0\\
584	0\\
585	0\\
586	0\\
587	0\\
588	0\\
589	0\\
590	0\\
591	0\\
592	0\\
593	0\\
594	0\\
595	0\\
596	0\\
597	0\\
598	0\\
599	0\\
600	0\\
};
\addplot [color=blue,solid,forget plot]
  table[row sep=crcr]{%
1	0.00225547679368458\\
2	0.00225546436267677\\
3	0.0022554517189349\\
4	0.00225543885880797\\
5	0.00225542577858241\\
6	0.00225541247448096\\
7	0.00225539894266163\\
8	0.00225538517921661\\
9	0.00225537118017108\\
10	0.00225535694148212\\
11	0.00225534245903754\\
12	0.00225532772865468\\
13	0.00225531274607924\\
14	0.00225529750698401\\
15	0.00225528200696767\\
16	0.00225526624155352\\
17	0.00225525020618817\\
18	0.00225523389624026\\
19	0.00225521730699912\\
20	0.00225520043367341\\
21	0.00225518327138976\\
22	0.00225516581519137\\
23	0.00225514806003657\\
24	0.0022551300007974\\
25	0.00225511163225812\\
26	0.00225509294911375\\
27	0.00225507394596848\\
28	0.00225505461733422\\
29	0.00225503495762892\\
30	0.00225501496117508\\
31	0.00225499462219804\\
32	0.00225497393482437\\
33	0.00225495289308018\\
34	0.00225493149088941\\
35	0.00225490972207209\\
36	0.00225488758034257\\
37	0.00225486505930775\\
38	0.00225484215246521\\
39	0.00225481885320136\\
40	0.00225479515478961\\
41	0.00225477105038835\\
42	0.00225474653303909\\
43	0.00225472159566441\\
44	0.00225469623106597\\
45	0.00225467043192244\\
46	0.00225464419078744\\
47	0.00225461750008738\\
48	0.00225459035211933\\
49	0.00225456273904881\\
50	0.00225453465290756\\
51	0.00225450608559128\\
52	0.0022544770288573\\
53	0.00225444747432227\\
54	0.00225441741345974\\
55	0.00225438683759773\\
56	0.00225435573791633\\
57	0.00225432410544509\\
58	0.00225429193106057\\
59	0.00225425920548367\\
60	0.00225422591927705\\
61	0.0022541920628424\\
62	0.00225415762641777\\
63	0.00225412260007477\\
64	0.00225408697371574\\
65	0.00225405073707093\\
66	0.00225401387969554\\
67	0.0022539763909668\\
68	0.00225393826008092\\
69	0.0022538994760501\\
70	0.00225386002769933\\
71	0.0022538199036633\\
72	0.00225377909238315\\
73	0.00225373758210322\\
74	0.0022536953608677\\
75	0.00225365241651727\\
76	0.00225360873668569\\
77	0.00225356430879626\\
78	0.0022535191200583\\
79	0.00225347315746352\\
80	0.0022534264077824\\
81	0.00225337885756039\\
82	0.00225333049311418\\
83	0.00225328130052779\\
84	0.0022532312656487\\
85	0.00225318037408382\\
86	0.00225312861119549\\
87	0.00225307596209731\\
88	0.00225302241164999\\
89	0.00225296794445707\\
90	0.0022529125448606\\
91	0.00225285619693674\\
92	0.0022527988844913\\
93	0.00225274059105517\\
94	0.0022526812998797\\
95	0.00225262099393201\\
96	0.00225255965589022\\
97	0.00225249726813854\\
98	0.00225243381276239\\
99	0.00225236927154336\\
100	0.00225230362595409\\
101	0.00225223685715307\\
102	0.00225216894597939\\
103	0.00225209987294737\\
104	0.00225202961824105\\
105	0.00225195816170873\\
106	0.00225188548285725\\
107	0.00225181156084631\\
108	0.00225173637448258\\
109	0.00225165990221383\\
110	0.00225158212212285\\
111	0.00225150301192136\\
112	0.00225142254894375\\
113	0.00225134071014072\\
114	0.00225125747207289\\
115	0.00225117281090421\\
116	0.00225108670239529\\
117	0.00225099912189661\\
118	0.0022509100443417\\
119	0.00225081944424\\
120	0.00225072729566989\\
121	0.00225063357227131\\
122	0.00225053824723845\\
123	0.00225044129331226\\
124	0.0022503426827728\\
125	0.0022502423874315\\
126	0.0022501403786233\\
127	0.00225003662719858\\
128	0.00224993110351509\\
129	0.00224982377742956\\
130	0.00224971461828938\\
131	0.00224960359492395\\
132	0.00224949067563599\\
133	0.00224937582819269\\
134	0.00224925901981669\\
135	0.00224914021717691\\
136	0.00224901938637924\\
137	0.00224889649295707\\
138	0.00224877150186167\\
139	0.00224864437745237\\
140	0.00224851508348662\\
141	0.00224838358310987\\
142	0.00224824983884521\\
143	0.00224811381258292\\
144	0.00224797546556979\\
145	0.00224783475839845\\
146	0.00224769165099652\\
147	0.00224754610261519\\
148	0.0022473980718179\\
149	0.00224724751646874\\
150	0.00224709439372072\\
151	0.00224693866000382\\
152	0.00224678027101282\\
153	0.00224661918169497\\
154	0.0022464553462374\\
155	0.00224628871805434\\
156	0.00224611924977416\\
157	0.00224594689322612\\
158	0.00224577159942695\\
159	0.00224559331856717\\
160	0.00224541199999721\\
161	0.00224522759221328\\
162	0.00224504004284297\\
163	0.00224484929863065\\
164	0.00224465530542263\\
165	0.002244458008152\\
166	0.00224425735082327\\
167	0.00224405327649673\\
168	0.00224384572727259\\
169	0.00224363464427471\\
170	0.00224341996763425\\
171	0.00224320163647287\\
172	0.00224297958888574\\
173	0.0022427537619242\\
174	0.00224252409157822\\
175	0.0022422905127584\\
176	0.00224205295927782\\
177	0.00224181136383346\\
178	0.0022415656579874\\
179	0.00224131577214762\\
180	0.00224106163554846\\
181	0.00224080317623084\\
182	0.00224054032102202\\
183	0.00224027299551509\\
184	0.00224000112404807\\
185	0.00223972462968265\\
186	0.00223944343418256\\
187	0.00223915745799157\\
188	0.00223886662021111\\
189	0.00223857083857746\\
190	0.0022382700294386\\
191	0.00223796410773064\\
192	0.00223765298695376\\
193	0.00223733657914782\\
194	0.00223701479486753\\
195	0.00223668754315712\\
196	0.00223635473152461\\
197	0.00223601626591562\\
198	0.0022356720506867\\
199	0.00223532198857818\\
200	0.00223496598068657\\
201	0.00223460392643642\\
202	0.00223423572355172\\
203	0.00223386126802676\\
204	0.00223348045409645\\
205	0.00223309317420614\\
206	0.00223269931898089\\
207	0.00223229877719413\\
208	0.00223189143573586\\
209	0.00223147717958013\\
210	0.00223105589175204\\
211	0.00223062745329407\\
212	0.00223019174323182\\
213	0.00222974863853913\\
214	0.00222929801410253\\
215	0.002228839742685\\
216	0.00222837369488917\\
217	0.00222789973911968\\
218	0.00222741774154498\\
219	0.00222692756605833\\
220	0.00222642907423807\\
221	0.00222592212530717\\
222	0.00222540657609198\\
223	0.00222488228098028\\
224	0.00222434909187839\\
225	0.00222380685816758\\
226	0.0022232554266596\\
227	0.00222269464155133\\
228	0.0022221243443786\\
229	0.00222154437396911\\
230	0.00222095456639435\\
231	0.00222035475492071\\
232	0.00221974476995955\\
233	0.00221912443901628\\
234	0.00221849358663846\\
235	0.00221785203436292\\
236	0.00221719960066168\\
237	0.00221653610088695\\
238	0.00221586134721493\\
239	0.00221517514858848\\
240	0.00221447731065868\\
241	0.00221376763572513\\
242	0.0022130459226751\\
243	0.00221231196692134\\
244	0.00221156556033872\\
245	0.00221080649119944\\
246	0.00221003454410697\\
247	0.00220924949992857\\
248	0.00220845113572638\\
249	0.00220763922468709\\
250	0.00220681353605007\\
251	0.00220597383503397\\
252	0.00220511988276175\\
253	0.00220425143618409\\
254	0.00220336824800114\\
255	0.00220247006658248\\
256	0.00220155663588542\\
257	0.00220062769537139\\
258	0.00219968297992052\\
259	0.00219872221974431\\
260	0.0021977451402962\\
261	0.00219675146218023\\
262	0.00219574090105754\\
263	0.00219471316755064\\
264	0.00219366796714554\\
265	0.00219260500009143\\
266	0.00219152396129811\\
267	0.00219042454023081\\
268	0.00218930642080246\\
269	0.00218816928126336\\
270	0.00218701279408796\\
271	0.00218583662585887\\
272	0.00218464043714798\\
273	0.00218342388239507\\
274	0.00218218660978429\\
275	0.00218092826111787\\
276	0.00217964847168101\\
277	0.00217834687009569\\
278	0.00217702307819071\\
279	0.00217567671085904\\
280	0.00217430737591045\\
281	0.00217291467392013\\
282	0.00217149819807274\\
283	0.00217005753400202\\
284	0.00216859225962569\\
285	0.00216710194497527\\
286	0.00216558615202082\\
287	0.00216404443449025\\
288	0.00216247633768293\\
289	0.00216088139827743\\
290	0.00215925914413306\\
291	0.00215760909408485\\
292	0.00215593075773177\\
293	0.00215422363521778\\
294	0.00215248721700542\\
295	0.00215072098364137\\
296	0.00214892440551389\\
297	0.00214709694260139\\
298	0.00214523804421187\\
299	0.00214334714871261\\
300	0.00214142368324968\\
301	0.00213946706345662\\
302	0.00213747669315173\\
303	0.00213545196402326\\
304	0.00213339225530193\\
305	0.00213129693341986\\
306	0.00212916535165536\\
307	0.00212699684976246\\
308	0.00212479075358427\\
309	0.00212254637464819\\
310	0.00212026300974022\\
311	0.0021179399404549\\
312	0.00211557643272357\\
313	0.0021131717363424\\
314	0.00211072508450726\\
315	0.00210823569325357\\
316	0.00210570276087912\\
317	0.00210312546735227\\
318	0.0021005029736881\\
319	0.00209783442128978\\
320	0.00209511893125242\\
321	0.00209235560362633\\
322	0.00208954351663622\\
323	0.00208668172585272\\
324	0.00208376926331188\\
325	0.00208080513657825\\
326	0.00207778832774644\\
327	0.00207471779237544\\
328	0.00207159245834967\\
329	0.0020684112246598\\
330	0.00206517296009569\\
331	0.00206187650184302\\
332	0.00205852065397409\\
333	0.00205510418582252\\
334	0.00205162583022995\\
335	0.00204808428165185\\
336	0.00204447819410802\\
337	0.00204080617896145\\
338	0.00203706680250769\\
339	0.00203325858335442\\
340	0.00202937998956875\\
341	0.00202542943556714\\
342	0.00202140527871949\\
343	0.00201730581563618\\
344	0.00201312927810233\\
345	0.00200887382861977\\
346	0.00200453755551226\\
347	0.00200011846754379\\
348	0.00199561448799404\\
349	0.00199102344812764\\
350	0.00198634307998626\\
351	0.00198157100842312\\
352	0.00197670474228982\\
353	0.00197174166467503\\
354	0.0019666790220888\\
355	0.00196151391248782\\
356	0.0019562432720273\\
357	0.00195086386027465\\
358	0.00194537224329756\\
359	0.00193976477445752\\
360	0.00193403757236674\\
361	0.00192818649089298\\
362	0.00192220706749014\\
363	0.00191609439936248\\
364	0.00190984276577908\\
365	0.001903444344481\\
366	0.00189688471607222\\
367	0.00189012753770206\\
368	0.00188307204594991\\
369	0.00187532607576948\\
370	0.00186741687321954\\
371	0.00185937371942756\\
372	0.00185119429012358\\
373	0.00184287622349694\\
374	0.00183441711805204\\
375	0.0018258145280678\\
376	0.00181706595713902\\
377	0.00180816886457825\\
378	0.0017991207509713\\
379	0.00178991932478652\\
380	0.00178056209059273\\
381	0.00177104636415697\\
382	0.0017613694237474\\
383	0.00175152851070232\\
384	0.00174152083010462\\
385	0.00173134355156662\\
386	0.00172099381012973\\
387	0.00171046870728252\\
388	0.00169976531209971\\
389	0.00168888066250364\\
390	0.00167781176664776\\
391	0.00166655560442005\\
392	0.00165510912906219\\
393	0.00164346926889762\\
394	0.00163163292916047\\
395	0.00161959699391801\\
396	0.00160735832808923\\
397	0.00159491377959592\\
398	0.00158226018178169\\
399	0.00156939435649901\\
400	0.00155631311892567\\
401	0.00154301328669679\\
402	0.00152949169902099\\
403	0.00151574525607928\\
404	0.00150177098980964\\
405	0.0014875661528874\\
406	0.00147312828649604\\
407	0.00145845627135828\\
408	0.00144355769227254\\
409	0.00142846262402851\\
410	0.00141326501158894\\
411	0.00139822731853432\\
412	0.00138388740253358\\
413	0.00137011704235746\\
414	0.00135607976325594\\
415	0.00134177068139713\\
416	0.0013271838316562\\
417	0.00131231249661213\\
418	0.0012971497300087\\
419	0.00128168834618063\\
420	0.00126592090865607\\
421	0.00124983971672581\\
422	0.00123343678767319\\
423	0.00121670383904977\\
424	0.00119963228876237\\
425	0.00118221324775915\\
426	0.00116443749003025\\
427	0.00114629543543994\\
428	0.0011277771315331\\
429	0.00110887223425105\\
430	0.00108956998748904\\
431	0.00106985920142357\\
432	0.00104972822953443\\
433	0.00102916494424355\\
434	0.00100815671108913\\
435	0.000986690361352274\\
436	0.000964752163052489\\
437	0.000942327790230635\\
438	0.000919402290441712\\
439	0.000895960050373461\\
440	0.000871984759420209\\
441	0.000847459370560392\\
442	0.000822366055769701\\
443	0.000796686145148563\\
444	0.000770400012820854\\
445	0.0007434868148554\\
446	0.000715924066459259\\
447	0.000687689175228605\\
448	0.000658765806217179\\
449	0.000629129746906497\\
450	0.000598755261626512\\
451	0.00056761530781351\\
452	0.00053568151467907\\
453	0.000502924087514885\\
454	0.000469311590360925\\
455	0.000434810187659242\\
456	0.000399380515904592\\
457	0.000362964325713837\\
458	0.000325427915679923\\
459	0.000286121859631025\\
460	0.00024577460918375\\
461	0.000204529627511481\\
462	0.000162346680408577\\
463	0.000119118848819356\\
464	7.45041232746955e-05\\
465	2.6781960106478e-05\\
466	0\\
467	0\\
468	0\\
469	0\\
470	0\\
471	0\\
472	0\\
473	0\\
474	0\\
475	0\\
476	0\\
477	0\\
478	0\\
479	0\\
480	0\\
481	0\\
482	0\\
483	0\\
484	0\\
485	0\\
486	0\\
487	0\\
488	0\\
489	0\\
490	0\\
491	0\\
492	0\\
493	0\\
494	0\\
495	0\\
496	0\\
497	0\\
498	0\\
499	0\\
500	0\\
501	0\\
502	0\\
503	0\\
504	0\\
505	0\\
506	0\\
507	0\\
508	0\\
509	0\\
510	0\\
511	0\\
512	0\\
513	0\\
514	0\\
515	0\\
516	0\\
517	0\\
518	0\\
519	0\\
520	0\\
521	0\\
522	0\\
523	0\\
524	0\\
525	0\\
526	0\\
527	0\\
528	0\\
529	0\\
530	0\\
531	0\\
532	0\\
533	0\\
534	0\\
535	0\\
536	0\\
537	0\\
538	0\\
539	0\\
540	0\\
541	0\\
542	0\\
543	0\\
544	0\\
545	0\\
546	0\\
547	0\\
548	0\\
549	0\\
550	0\\
551	0\\
552	0\\
553	0\\
554	0\\
555	0\\
556	0\\
557	0\\
558	0\\
559	0\\
560	0\\
561	0\\
562	0\\
563	0\\
564	0\\
565	0\\
566	0\\
567	0\\
568	0\\
569	0\\
570	0\\
571	0\\
572	0\\
573	0\\
574	0\\
575	0\\
576	0\\
577	0\\
578	0\\
579	0\\
580	0\\
581	0\\
582	0\\
583	0\\
584	0\\
585	0\\
586	0\\
587	0\\
588	0\\
589	0\\
590	0\\
591	0\\
592	0\\
593	0\\
594	0\\
595	0\\
596	0\\
597	0\\
598	0\\
599	0\\
600	0\\
};
\addplot [color=mycolor10,solid,forget plot]
  table[row sep=crcr]{%
1	0.00367136005952458\\
2	0.0036713542087014\\
3	0.00367134825773797\\
4	0.00367134220491635\\
5	0.00367133604848915\\
6	0.00367132978667907\\
7	0.00367132341767836\\
8	0.00367131693964833\\
9	0.00367131035071882\\
10	0.00367130364898763\\
11	0.00367129683252001\\
12	0.0036712898993481\\
13	0.00367128284747037\\
14	0.003671275674851\\
15	0.00367126837941938\\
16	0.00367126095906944\\
17	0.00367125341165909\\
18	0.00367124573500959\\
19	0.00367123792690494\\
20	0.00367122998509123\\
21	0.00367122190727602\\
22	0.00367121369112766\\
23	0.00367120533427462\\
24	0.00367119683430483\\
25	0.003671188188765\\
26	0.0036711793951599\\
27	0.00367117045095163\\
28	0.00367116135355894\\
29	0.00367115210035647\\
30	0.00367114268867399\\
31	0.00367113311579566\\
32	0.00367112337895925\\
33	0.00367111347535534\\
34	0.00367110340212652\\
35	0.00367109315636661\\
36	0.00367108273511978\\
37	0.00367107213537974\\
38	0.00367106135408891\\
39	0.00367105038813747\\
40	0.00367103923436256\\
41	0.00367102788954732\\
42	0.00367101635042003\\
43	0.00367100461365313\\
44	0.0036709926758623\\
45	0.00367098053360549\\
46	0.00367096818338196\\
47	0.00367095562163127\\
48	0.00367094284473225\\
49	0.00367092984900204\\
50	0.00367091663069497\\
51	0.00367090318600154\\
52	0.00367088951104733\\
53	0.00367087560189192\\
54	0.00367086145452774\\
55	0.00367084706487896\\
56	0.00367083242880035\\
57	0.00367081754207607\\
58	0.0036708024004185\\
59	0.00367078699946705\\
60	0.00367077133478688\\
61	0.00367075540186769\\
62	0.00367073919612242\\
63	0.00367072271288598\\
64	0.00367070594741391\\
65	0.00367068889488107\\
66	0.00367067155038026\\
67	0.00367065390892083\\
68	0.0036706359654273\\
69	0.00367061771473792\\
70	0.00367059915160324\\
71	0.00367058027068457\\
72	0.00367056106655255\\
73	0.0036705415336856\\
74	0.00367052166646835\\
75	0.0036705014591901\\
76	0.00367048090604318\\
77	0.00367046000112136\\
78	0.00367043873841816\\
79	0.00367041711182519\\
80	0.00367039511513042\\
81	0.00367037274201647\\
82	0.0036703499860588\\
83	0.00367032684072394\\
84	0.00367030329936769\\
85	0.00367027935523319\\
86	0.0036702550014491\\
87	0.00367023023102764\\
88	0.00367020503686266\\
89	0.00367017941172767\\
90	0.00367015334827377\\
91	0.00367012683902765\\
92	0.0036700998763895\\
93	0.00367007245263087\\
94	0.00367004455989254\\
95	0.00367001619018233\\
96	0.00366998733537283\\
97	0.00366995798719921\\
98	0.00366992813725689\\
99	0.00366989777699916\\
100	0.00366986689773488\\
101	0.003669835490626\\
102	0.00366980354668515\\
103	0.00366977105677311\\
104	0.00366973801159628\\
105	0.0036697044017041\\
106	0.00366967021748645\\
107	0.00366963544917095\\
108	0.00366960008682027\\
109	0.00366956412032934\\
110	0.00366952753942259\\
111	0.00366949033365109\\
112	0.00366945249238962\\
113	0.00366941400483377\\
114	0.00366937485999693\\
115	0.00366933504670721\\
116	0.00366929455360442\\
117	0.00366925336913683\\
118	0.00366921148155806\\
119	0.00366916887892377\\
120	0.00366912554908839\\
121	0.0036690814797017\\
122	0.00366903665820549\\
123	0.00366899107183002\\
124	0.00366894470759055\\
125	0.00366889755228367\\
126	0.00366884959248373\\
127	0.00366880081453906\\
128	0.00366875120456826\\
129	0.0036687007484563\\
130	0.00366864943185067\\
131	0.0036685972401574\\
132	0.00366854415853704\\
133	0.00366849017190053\\
134	0.00366843526490509\\
135	0.00366837942194993\\
136	0.00366832262717202\\
137	0.00366826486444162\\
138	0.00366820611735795\\
139	0.00366814636924458\\
140	0.00366808560314488\\
141	0.00366802380181734\\
142	0.00366796094773078\\
143	0.00366789702305959\\
144	0.0036678320096788\\
145	0.00366776588915917\\
146	0.00366769864276207\\
147	0.00366763025143433\\
148	0.00366756069580298\\
149	0.00366748995616998\\
150	0.00366741801250679\\
151	0.00366734484444885\\
152	0.00366727043129002\\
153	0.0036671947519769\\
154	0.00366711778510305\\
155	0.0036670395089031\\
156	0.00366695990124682\\
157	0.00366687893963302\\
158	0.0036667966011834\\
159	0.00366671286263627\\
160	0.0036666277003402\\
161	0.00366654109024748\\
162	0.00366645300790763\\
163	0.0036663634284606\\
164	0.00366627232663002\\
165	0.00366617967671628\\
166	0.00366608545258947\\
167	0.00366598962768222\\
168	0.00366589217498246\\
169	0.00366579306702598\\
170	0.00366569227588894\\
171	0.00366558977318025\\
172	0.0036654855300337\\
173	0.0036653795171002\\
174	0.00366527170453961\\
175	0.00366516206201264\\
176	0.00366505055867255\\
177	0.00366493716315665\\
178	0.00366482184357777\\
179	0.00366470456751549\\
180	0.00366458530200726\\
181	0.00366446401353939\\
182	0.00366434066803786\\
183	0.00366421523085895\\
184	0.0036640876667798\\
185	0.00366395793998869\\
186	0.00366382601407527\\
187	0.00366369185202057\\
188	0.00366355541618677\\
189	0.00366341666830697\\
190	0.00366327556947463\\
191	0.00366313208013285\\
192	0.00366298616006357\\
193	0.00366283776837647\\
194	0.00366268686349774\\
195	0.00366253340315862\\
196	0.00366237734438379\\
197	0.0036622186434795\\
198	0.00366205725602156\\
199	0.00366189313684307\\
200	0.00366172624002197\\
201	0.00366155651886834\\
202	0.00366138392591151\\
203	0.00366120841288693\\
204	0.00366102993072281\\
205	0.00366084842952655\\
206	0.00366066385857087\\
207	0.00366047616627979\\
208	0.00366028530021424\\
209	0.00366009120705756\\
210	0.00365989383260061\\
211	0.00365969312172667\\
212	0.00365948901839613\\
213	0.00365928146563077\\
214	0.0036590704054979\\
215	0.0036588557790941\\
216	0.00365863752652877\\
217	0.00365841558690728\\
218	0.00365818989831388\\
219	0.0036579603977943\\
220	0.00365772702133799\\
221	0.00365748970386007\\
222	0.00365724837918293\\
223	0.00365700298001751\\
224	0.00365675343794421\\
225	0.00365649968339345\\
226	0.0036562416456259\\
227	0.00365597925271227\\
228	0.00365571243151281\\
229	0.00365544110765633\\
230	0.00365516520551892\\
231	0.00365488464820216\\
232	0.00365459935751102\\
233	0.00365430925393122\\
234	0.0036540142566063\\
235	0.00365371428331407\\
236	0.00365340925044277\\
237	0.00365309907296666\\
238	0.00365278366442114\\
239	0.00365246293687742\\
240	0.00365213680091666\\
241	0.00365180516560358\\
242	0.00365146793845959\\
243	0.00365112502543529\\
244	0.00365077633088253\\
245	0.00365042175752576\\
246	0.00365006120643292\\
247	0.00364969457698562\\
248	0.00364932176684879\\
249	0.0036489426719396\\
250	0.00364855718639577\\
251	0.00364816520254322\\
252	0.00364776661086298\\
253	0.0036473612999574\\
254	0.0036469491565156\\
255	0.00364653006527817\\
256	0.0036461039090011\\
257	0.00364567056841883\\
258	0.0036452299222065\\
259	0.00364478184694136\\
260	0.00364432621706316\\
261	0.00364386290483372\\
262	0.00364339178029548\\
263	0.00364291271122899\\
264	0.00364242556310947\\
265	0.00364193019906216\\
266	0.00364142647981658\\
267	0.00364091426365965\\
268	0.00364039340638746\\
269	0.00363986376125587\\
270	0.00363932517892966\\
271	0.00363877750743043\\
272	0.00363822059208316\\
273	0.00363765427546155\\
274	0.00363707839733184\\
275	0.0036364927945936\\
276	0.00363589730121682\\
277	0.00363529174818207\\
278	0.00363467596341668\\
279	0.00363404977172866\\
280	0.00363341299473854\\
281	0.00363276545080885\\
282	0.00363210695497132\\
283	0.00363143731885151\\
284	0.0036307563505908\\
285	0.00363006385476552\\
286	0.00362935963230314\\
287	0.00362864348039527\\
288	0.00362791519240729\\
289	0.0036271745577844\\
290	0.00362642136195385\\
291	0.00362565538622313\\
292	0.00362487640767375\\
293	0.00362408419905044\\
294	0.00362327852864536\\
295	0.00362245916017698\\
296	0.0036216258526632\\
297	0.00362077836028839\\
298	0.00361991643226378\\
299	0.00361903981268069\\
300	0.00361814824035613\\
301	0.00361724144867005\\
302	0.00361631916539358\\
303	0.00361538111250761\\
304	0.00361442700601071\\
305	0.00361345655571573\\
306	0.00361246946503382\\
307	0.00361146543074481\\
308	0.00361044414275246\\
309	0.00360940528382247\\
310	0.0036083485293015\\
311	0.00360727354681671\\
312	0.00360617999595942\\
313	0.00360506752795359\\
314	0.00360393578528118\\
315	0.00360278440128144\\
316	0.00360161299972272\\
317	0.00360042119433957\\
318	0.00359920858833189\\
319	0.00359797477382177\\
320	0.0035967193312638\\
321	0.00359544182880375\\
322	0.00359414182157997\\
323	0.00359281885096147\\
324	0.00359147244371551\\
325	0.00359010211109698\\
326	0.00358870734785085\\
327	0.00358728763111789\\
328	0.00358584241923271\\
329	0.00358437115040184\\
330	0.00358287324124808\\
331	0.00358134808520573\\
332	0.00357979505074929\\
333	0.00357821347943625\\
334	0.00357660268374211\\
335	0.00357496194466306\\
336	0.00357329050905873\\
337	0.00357158758670406\\
338	0.00356985234701524\\
339	0.00356808391541051\\
340	0.00356628136926162\\
341	0.00356444373338597\\
342	0.0035625699750233\\
343	0.00356065899823356\\
344	0.00355870963764444\\
345	0.0035567206514678\\
346	0.00355469071369377\\
347	0.00355261840535946\\
348	0.00355050220477556\\
349	0.00354834047657885\\
350	0.0035461314594607\\
351	0.00354387325240232\\
352	0.00354156379922452\\
353	0.00353920087123593\\
354	0.00353678204773568\\
355	0.00353430469408663\\
356	0.00353176593699083\\
357	0.00352916263644676\\
358	0.00352649135382048\\
359	0.00352374831505654\\
360	0.00352092936611005\\
361	0.00351802991340023\\
362	0.00351504482719273\\
363	0.00351196823786181\\
364	0.00350879299930346\\
365	0.00350550909462353\\
366	0.00350209870266089\\
367	0.00349852084308253\\
368	0.00349465963001152\\
369	0.00349010686104009\\
370	0.00348544033389064\\
371	0.00348070343857037\\
372	0.00347589524788633\\
373	0.0034710148278648\\
374	0.00346606123691451\\
375	0.00346103352543611\\
376	0.00345593074002277\\
377	0.00345075194222006\\
378	0.00344549623478454\\
379	0.0034401626745596\\
380	0.00343475028031016\\
381	0.0034292580655807\\
382	0.00342368503853128\\
383	0.00341803020166984\\
384	0.00341229255146185\\
385	0.00340647107779616\\
386	0.00340056476328271\\
387	0.00339457258235557\\
388	0.00338849350015083\\
389	0.00338232647112589\\
390	0.00337607043738275\\
391	0.00336972432665376\\
392	0.0033632870499045\\
393	0.00335675749850444\\
394	0.00335013454091347\\
395	0.00334341701883367\\
396	0.00333660374278532\\
397	0.00332969348709849\\
398	0.00332268498439105\\
399	0.00331557691978132\\
400	0.00330836792542846\\
401	0.00330105657652901\\
402	0.00329364139035451\\
403	0.00328612082939462\\
404	0.00327849330934434\\
405	0.00327075724168134\\
406	0.00326291136982867\\
407	0.00325495636404665\\
408	0.00324689763874729\\
409	0.00323875231388637\\
410	0.00323056097037767\\
411	0.00322238312421398\\
412	0.00321417988858275\\
413	0.0032058389817077\\
414	0.00319735795331092\\
415	0.00318873408097052\\
416	0.00317996444183571\\
417	0.00317104601964941\\
418	0.0031619757002138\\
419	0.00315275026644325\\
420	0.00314336639261583\\
421	0.00313382063738896\\
422	0.00312410943678926\\
423	0.00311422910202385\\
424	0.00310417581523831\\
425	0.00309394561880077\\
426	0.00308353440793545\\
427	0.00307293792291095\\
428	0.00306215174075572\\
429	0.00305117126647352\\
430	0.00303999172372835\\
431	0.00302860814496947\\
432	0.0030170153609647\\
433	0.00300520798970998\\
434	0.00299318042468135\\
435	0.00298092682239276\\
436	0.00296844108921502\\
437	0.00295571686738979\\
438	0.00294274752011353\\
439	0.00292952611540876\\
440	0.00291604540808881\\
441	0.00290229781811762\\
442	0.00288827540156554\\
443	0.00287396980789372\\
444	0.00285937222539627\\
445	0.00284447339254048\\
446	0.00282926403681843\\
447	0.0028137358241746\\
448	0.00279787831489232\\
449	0.00278168049455409\\
450	0.00276513081731052\\
451	0.00274821718821128\\
452	0.00273092691441245\\
453	0.00271324658327365\\
454	0.00269516167107363\\
455	0.00267665513762852\\
456	0.0026577021184723\\
457	0.00263824930352147\\
458	0.00261813291909802\\
459	0.00259690828776238\\
460	0.00257525010215384\\
461	0.00255329429443533\\
462	0.00253101105647798\\
463	0.00250831014118649\\
464	0.00248489435582048\\
465	0.00246008673831945\\
466	0.00243435385994019\\
467	0.00240879072737593\\
468	0.00237947399244218\\
469	0.0023482221547078\\
470	0.00231642921410384\\
471	0.0022841442400962\\
472	0.00225135420759003\\
473	0.00221804954900144\\
474	0.00218423389328816\\
475	0.00214995437136689\\
476	0.00211539324347675\\
477	0.00208111881827806\\
478	0.00204850191368291\\
479	0.00201860096918738\\
480	0.0019879012972538\\
481	0.00195636036166827\\
482	0.00192393154007199\\
483	0.00189056366105782\\
484	0.00185619990513972\\
485	0.00182077534695337\\
486	0.00178420965799812\\
487	0.00174637801333911\\
488	0.00170696088632293\\
489	0.00166344164530458\\
490	0.00161757795728889\\
491	0.00157061342368385\\
492	0.00152251019137918\\
493	0.00147322862401854\\
494	0.00142272720797365\\
495	0.00137096245850899\\
496	0.00131788886726747\\
497	0.00126345902010643\\
498	0.00120762420237788\\
499	0.00115033542055204\\
500	0.00109153966695643\\
501	0.00103116247127905\\
502	0.000969113373693511\\
503	0.000905317299843979\\
504	0.000839604316548774\\
505	0.000771463659200474\\
506	0.000701477745376201\\
507	0.000629526791820348\\
508	0.000554952970295424\\
509	0.000476998662898304\\
510	0.000397384018534237\\
511	0.000316032786533761\\
512	0.00023278160760913\\
513	0.000147012106123333\\
514	5.53656290951778e-05\\
515	0\\
516	0\\
517	0\\
518	0\\
519	0\\
520	0\\
521	0\\
522	0\\
523	0\\
524	0\\
525	0\\
526	0\\
527	0\\
528	0\\
529	0\\
530	0\\
531	0\\
532	0\\
533	0\\
534	0\\
535	0\\
536	0\\
537	0\\
538	0\\
539	0\\
540	0\\
541	0\\
542	0\\
543	0\\
544	0\\
545	0\\
546	0\\
547	0\\
548	0\\
549	0\\
550	0\\
551	0\\
552	0\\
553	0\\
554	0\\
555	0\\
556	0\\
557	0\\
558	0\\
559	0\\
560	0\\
561	0\\
562	0\\
563	0\\
564	0\\
565	0\\
566	0\\
567	0\\
568	0\\
569	0\\
570	0\\
571	0\\
572	0\\
573	0\\
574	0\\
575	0\\
576	0\\
577	0\\
578	0\\
579	0\\
580	0\\
581	0\\
582	0\\
583	0\\
584	0\\
585	0\\
586	0\\
587	0\\
588	0\\
589	0\\
590	0\\
591	0\\
592	0\\
593	0\\
594	0\\
595	0\\
596	0\\
597	0\\
598	0\\
599	0\\
600	0\\
};
\addplot [color=mycolor11,solid,forget plot]
  table[row sep=crcr]{%
1	0.00516053505583464\\
2	0.00516053396220163\\
3	0.00516053284984937\\
4	0.00516053171845674\\
5	0.0051605305676971\\
6	0.00516052939723823\\
7	0.00516052820674222\\
8	0.00516052699586537\\
9	0.00516052576425811\\
10	0.00516052451156487\\
11	0.005160523237424\\
12	0.00516052194146766\\
13	0.00516052062332171\\
14	0.00516051928260563\\
15	0.00516051791893236\\
16	0.00516051653190824\\
17	0.00516051512113286\\
18	0.00516051368619896\\
19	0.00516051222669234\\
20	0.00516051074219169\\
21	0.00516050923226851\\
22	0.00516050769648695\\
23	0.00516050613440375\\
24	0.00516050454556803\\
25	0.00516050292952122\\
26	0.00516050128579693\\
27	0.00516049961392076\\
28	0.00516049791341025\\
29	0.00516049618377466\\
30	0.00516049442451487\\
31	0.00516049263512326\\
32	0.00516049081508352\\
33	0.0051604889638705\\
34	0.00516048708095013\\
35	0.00516048516577919\\
36	0.00516048321780519\\
37	0.00516048123646621\\
38	0.00516047922119073\\
39	0.0051604771713975\\
40	0.00516047508649533\\
41	0.00516047296588294\\
42	0.0051604708089488\\
43	0.00516046861507095\\
44	0.00516046638361682\\
45	0.00516046411394304\\
46	0.0051604618053953\\
47	0.00516045945730809\\
48	0.00516045706900458\\
49	0.00516045463979641\\
50	0.00516045216898348\\
51	0.00516044965585375\\
52	0.00516044709968305\\
53	0.00516044449973488\\
54	0.0051604418552602\\
55	0.0051604391654972\\
56	0.0051604364296711\\
57	0.00516043364699393\\
58	0.00516043081666431\\
59	0.00516042793786721\\
60	0.00516042500977374\\
61	0.00516042203154088\\
62	0.00516041900231132\\
63	0.0051604159212131\\
64	0.00516041278735947\\
65	0.0051604095998486\\
66	0.00516040635776331\\
67	0.00516040306017083\\
68	0.00516039970612255\\
69	0.00516039629465373\\
70	0.00516039282478322\\
71	0.00516038929551324\\
72	0.00516038570582903\\
73	0.00516038205469859\\
74	0.00516037834107243\\
75	0.00516037456388322\\
76	0.00516037072204551\\
77	0.00516036681445543\\
78	0.00516036283999039\\
79	0.00516035879750876\\
80	0.00516035468584951\\
81	0.00516035050383195\\
82	0.00516034625025537\\
83	0.00516034192389869\\
84	0.00516033752352013\\
85	0.00516033304785689\\
86	0.00516032849562474\\
87	0.00516032386551772\\
88	0.00516031915620774\\
89	0.00516031436634421\\
90	0.00516030949455368\\
91	0.00516030453943946\\
92	0.0051602994995812\\
93	0.00516029437353453\\
94	0.00516028915983063\\
95	0.00516028385697584\\
96	0.00516027846345123\\
97	0.00516027297771219\\
98	0.00516026739818799\\
99	0.00516026172328136\\
100	0.00516025595136801\\
101	0.00516025008079623\\
102	0.00516024410988637\\
103	0.00516023803693045\\
104	0.00516023186019159\\
105	0.00516022557790363\\
106	0.00516021918827055\\
107	0.00516021268946602\\
108	0.00516020607963291\\
109	0.0051601993568827\\
110	0.00516019251929506\\
111	0.00516018556491722\\
112	0.00516017849176349\\
113	0.00516017129781467\\
114	0.00516016398101754\\
115	0.00516015653928424\\
116	0.00516014897049172\\
117	0.00516014127248113\\
118	0.00516013344305727\\
119	0.00516012547998791\\
120	0.00516011738100323\\
121	0.00516010914379517\\
122	0.00516010076601679\\
123	0.00516009224528163\\
124	0.00516008357916303\\
125	0.00516007476519348\\
126	0.00516006580086392\\
127	0.00516005668362308\\
128	0.00516004741087672\\
129	0.00516003797998697\\
130	0.00516002838827154\\
131	0.00516001863300304\\
132	0.0051600087114082\\
133	0.0051599986206671\\
134	0.00515998835791238\\
135	0.00515997792022849\\
136	0.00515996730465084\\
137	0.00515995650816502\\
138	0.00515994552770596\\
139	0.00515993436015705\\
140	0.00515992300234932\\
141	0.00515991145106056\\
142	0.00515989970301442\\
143	0.00515988775487952\\
144	0.00515987560326854\\
145	0.00515986324473728\\
146	0.00515985067578372\\
147	0.00515983789284705\\
148	0.00515982489230669\\
149	0.00515981167048129\\
150	0.00515979822362773\\
151	0.00515978454794009\\
152	0.00515977063954862\\
153	0.00515975649451863\\
154	0.00515974210884946\\
155	0.00515972747847336\\
156	0.00515971259925437\\
157	0.00515969746698721\\
158	0.00515968207739608\\
159	0.00515966642613356\\
160	0.00515965050877932\\
161	0.00515963432083899\\
162	0.0051596178577429\\
163	0.00515960111484483\\
164	0.00515958408742071\\
165	0.00515956677066737\\
166	0.00515954915970118\\
167	0.00515953124955675\\
168	0.00515951303518555\\
169	0.0051594945114545\\
170	0.00515947567314463\\
171	0.00515945651494957\\
172	0.00515943703147415\\
173	0.00515941721723292\\
174	0.00515939706664861\\
175	0.00515937657405062\\
176	0.00515935573367348\\
177	0.00515933453965525\\
178	0.00515931298603592\\
179	0.00515929106675577\\
180	0.00515926877565369\\
181	0.00515924610646555\\
182	0.00515922305282241\\
183	0.00515919960824879\\
184	0.00515917576616094\\
185	0.00515915151986495\\
186	0.00515912686255499\\
187	0.00515910178731138\\
188	0.00515907628709871\\
189	0.00515905035476392\\
190	0.0051590239830343\\
191	0.0051589971645155\\
192	0.00515896989168953\\
193	0.0051589421569126\\
194	0.00515891395241311\\
195	0.00515888527028945\\
196	0.00515885610250783\\
197	0.00515882644090004\\
198	0.00515879627716127\\
199	0.00515876560284769\\
200	0.00515873440937424\\
201	0.00515870268801215\\
202	0.00515867042988658\\
203	0.00515863762597415\\
204	0.00515860426710041\\
205	0.00515857034393732\\
206	0.00515853584700062\\
207	0.00515850076664722\\
208	0.00515846509307253\\
209	0.00515842881630766\\
210	0.00515839192621668\\
211	0.0051583544124938\\
212	0.00515831626466043\\
213	0.00515827747206232\\
214	0.00515823802386648\\
215	0.0051581979090582\\
216	0.00515815711643796\\
217	0.00515811563461819\\
218	0.00515807345202016\\
219	0.00515803055687064\\
220	0.00515798693719857\\
221	0.0051579425808317\\
222	0.00515789747539312\\
223	0.00515785160829771\\
224	0.00515780496674857\\
225	0.00515775753773339\\
226	0.00515770930802069\\
227	0.00515766026415605\\
228	0.00515761039245821\\
229	0.00515755967901517\\
230	0.00515750810968016\\
231	0.00515745567006754\\
232	0.00515740234554864\\
233	0.0051573481212475\\
234	0.00515729298203654\\
235	0.00515723691253213\\
236	0.0051571798970901\\
237	0.00515712191980113\\
238	0.00515706296448607\\
239	0.00515700301469116\\
240	0.00515694205368316\\
241	0.00515688006444434\\
242	0.00515681702966746\\
243	0.00515675293175053\\
244	0.00515668775279156\\
245	0.00515662147458317\\
246	0.00515655407860704\\
247	0.00515648554602832\\
248	0.00515641585768987\\
249	0.00515634499410639\\
250	0.00515627293545845\\
251	0.00515619966158632\\
252	0.00515612515198376\\
253	0.00515604938579157\\
254	0.00515597234179111\\
255	0.00515589399839756\\
256	0.0051558143336531\\
257	0.00515573332521991\\
258	0.00515565095037301\\
259	0.00515556718599293\\
260	0.00515548200855819\\
261	0.00515539539413762\\
262	0.00515530731838246\\
263	0.00515521775651835\\
264	0.00515512668333697\\
265	0.00515503407318759\\
266	0.00515493989996835\\
267	0.0051548441371173\\
268	0.00515474675760322\\
269	0.0051546477339162\\
270	0.0051545470380579\\
271	0.00515444464153171\\
272	0.00515434051533257\\
273	0.00515423462993652\\
274	0.00515412695528961\\
275	0.00515401746079621\\
276	0.00515390611530788\\
277	0.00515379288711148\\
278	0.00515367774391692\\
279	0.00515356065284438\\
280	0.00515344158041124\\
281	0.00515332049251849\\
282	0.00515319735443667\\
283	0.00515307213079131\\
284	0.00515294478554787\\
285	0.00515281528199611\\
286	0.00515268358273379\\
287	0.00515254964964986\\
288	0.00515241344390688\\
289	0.00515227492592275\\
290	0.00515213405535171\\
291	0.00515199079106441\\
292	0.00515184509112727\\
293	0.00515169691278076\\
294	0.00515154621241674\\
295	0.00515139294555468\\
296	0.00515123706681681\\
297	0.0051510785299019\\
298	0.00515091728755775\\
299	0.00515075329155227\\
300	0.0051505864926429\\
301	0.0051504168405444\\
302	0.00515024428389477\\
303	0.00515006877021921\\
304	0.00514989024589186\\
305	0.00514970865609519\\
306	0.00514952394477684\\
307	0.00514933605460346\\
308	0.00514914492691124\\
309	0.00514895050165295\\
310	0.00514875271734128\\
311	0.00514855151098949\\
312	0.00514834681804893\\
313	0.00514813857233824\\
314	0.0051479267059676\\
315	0.00514771114925784\\
316	0.00514749183065296\\
317	0.00514726867662538\\
318	0.00514704161157321\\
319	0.00514681055770862\\
320	0.00514657543493639\\
321	0.00514633616072161\\
322	0.00514609264994523\\
323	0.00514584481474628\\
324	0.0051455925643491\\
325	0.00514533580487412\\
326	0.0051450744391301\\
327	0.00514480836638589\\
328	0.00514453748211933\\
329	0.0051442616777406\\
330	0.00514398084028718\\
331	0.005143694852087\\
332	0.00514340359038611\\
333	0.00514310692693669\\
334	0.00514280472754071\\
335	0.00514249685154394\\
336	0.00514218315127439\\
337	0.00514186347141843\\
338	0.00514153764832721\\
339	0.00514120550924475\\
340	0.00514086687144821\\
341	0.00514052154128956\\
342	0.00514016931312651\\
343	0.00513980996812899\\
344	0.00513944327294557\\
345	0.00513906897821246\\
346	0.00513868681688523\\
347	0.00513829650237064\\
348	0.00513789772643345\\
349	0.00513749015684897\\
350	0.0051370734347688\\
351	0.00513664717176195\\
352	0.00513621094648797\\
353	0.00513576430094946\\
354	0.00513530673625371\\
355	0.00513483770777226\\
356	0.00513435661949786\\
357	0.00513386281723007\\
358	0.00513335557972848\\
359	0.00513283410557918\\
360	0.00513229749024203\\
361	0.00513174467925832\\
362	0.00513117436263682\\
363	0.00513058472537829\\
364	0.00512997285665973\\
365	0.00512933339553511\\
366	0.0051286556461704\\
367	0.00512791809116632\\
368	0.00512707556718617\\
369	0.00512621365677636\\
370	0.00512533867006539\\
371	0.00512445043152696\\
372	0.00512354876402859\\
373	0.00512263348865814\\
374	0.00512170442482359\\
375	0.00512076139150791\\
376	0.00511980421108008\\
377	0.00511883271336735\\
378	0.00511784671898175\\
379	0.00511684604105501\\
380	0.00511583049149508\\
381	0.00511479988093934\\
382	0.00511375401868709\\
383	0.00511269271260762\\
384	0.00511161576901979\\
385	0.00511052299253852\\
386	0.00510941418588275\\
387	0.00510828914963941\\
388	0.00510714768197668\\
389	0.00510598957829947\\
390	0.00510481463083948\\
391	0.00510362262817123\\
392	0.00510241335464608\\
393	0.00510118658973624\\
394	0.00509994210728471\\
395	0.0050986796746647\\
396	0.00509739905187089\\
397	0.00509609999060445\\
398	0.0050947822334919\\
399	0.0050934455137228\\
400	0.00509208955564237\\
401	0.00509071407725901\\
402	0.00508931879649426\\
403	0.00508790344576597\\
404	0.00508646781104311\\
405	0.00508501185636\\
406	0.00508353610539178\\
407	0.00508204219128767\\
408	0.00508053380441878\\
409	0.00507901748525117\\
410	0.00507749997480398\\
411	0.00507597297051831\\
412	0.00507441994867542\\
413	0.00507284042967778\\
414	0.00507123388203909\\
415	0.00506959973481857\\
416	0.00506793739819613\\
417	0.00506624626253051\\
418	0.0050645256973056\\
419	0.00506277504987666\\
420	0.00506099364396145\\
421	0.00505918077819892\\
422	0.00505733572583228\\
423	0.00505545773387667\\
424	0.00505354602078519\\
425	0.00505159977490034\\
426	0.00504961815280787\\
427	0.00504760027758672\\
428	0.00504554523694768\\
429	0.00504345208125428\\
430	0.00504131982141767\\
431	0.00503914742665731\\
432	0.00503693382211835\\
433	0.00503467788633475\\
434	0.00503237844852357\\
435	0.00503003428568796\\
436	0.00502764411948796\\
437	0.00502520661279766\\
438	0.00502272036578203\\
439	0.00502018391115925\\
440	0.00501759570803928\\
441	0.00501495413345738\\
442	0.00501225747122418\\
443	0.00500950390204384\\
444	0.00500669151594826\\
445	0.00500381841065281\\
446	0.00500088292228883\\
447	0.0049978829043709\\
448	0.00499481608057586\\
449	0.00499168004997979\\
450	0.00498847227099568\\
451	0.00498519002024984\\
452	0.00498183029229815\\
453	0.00497838954526938\\
454	0.00497486305463673\\
455	0.00497124330908564\\
456	0.00496751623004757\\
457	0.00496365317341908\\
458	0.00495959803985892\\
459	0.00495545847701009\\
460	0.00495125238269577\\
461	0.0049469672450374\\
462	0.00494257469975514\\
463	0.00493801023687081\\
464	0.00493315076640688\\
465	0.00492798717113069\\
466	0.00492242062516231\\
467	0.00491552816158838\\
468	0.00490322237550712\\
469	0.00488875730748457\\
470	0.0048740630976783\\
471	0.0048591286169599\\
472	0.00484394361654972\\
473	0.00482850186063932\\
474	0.00481280854466351\\
475	0.00479689621963633\\
476	0.00478085290830506\\
477	0.00476484288526039\\
478	0.00474893442827621\\
479	0.0047325699405543\\
480	0.00471571948336329\\
481	0.00469834975067911\\
482	0.00468042346135682\\
483	0.00466189768238898\\
484	0.00464271982189616\\
485	0.00462281735643028\\
486	0.00460207060107256\\
487	0.00458023940772211\\
488	0.00455676418504533\\
489	0.00452948477304095\\
490	0.00450046664968492\\
491	0.00447100333840286\\
492	0.00444108528019643\\
493	0.0044107025642607\\
494	0.00437984492848087\\
495	0.00434850177656219\\
496	0.00431666220770141\\
497	0.00428431500858952\\
498	0.00425144835853428\\
499	0.00421804861420509\\
500	0.00418409693491697\\
501	0.00414957075012113\\
502	0.00411444927293261\\
503	0.00407869978160293\\
504	0.00404226856299079\\
505	0.00400519790882657\\
506	0.00396741722630471\\
507	0.0039287588917257\\
508	0.00388876341634874\\
509	0.00384719310549665\\
510	0.00380535624004568\\
511	0.00376318093942821\\
512	0.0037204691751195\\
513	0.00367670479506395\\
514	0.00363073524723451\\
515	0.00358294946602486\\
516	0.00353505653663763\\
517	0.00348709399632349\\
518	0.00343914842590566\\
519	0.00339142939890124\\
520	0.00334444598783759\\
521	0.00329937957587658\\
522	0.00325854851736096\\
523	0.00322419649826356\\
524	0.00318922244626424\\
525	0.00315326103247786\\
526	0.00311514836296493\\
527	0.00307123341315276\\
528	0.0030098509516261\\
529	0.00294020748052116\\
530	0.00286758526591559\\
531	0.00279104864096957\\
532	0.00270989433435107\\
533	0.00262004142303426\\
534	0.00252218784358832\\
535	0.00242178491691014\\
536	0.00231871896930725\\
537	0.00221285935169346\\
538	0.00210403327127977\\
539	0.00199194507799155\\
540	0.00187591619531658\\
541	0.00175411070440876\\
542	0.00162813783804569\\
543	0.00149958178120082\\
544	0.00136835829867665\\
545	0.00123439189756769\\
546	0.00109760179491459\\
547	0.000957899619460617\\
548	0.000815177798917114\\
549	0.000669268858551629\\
550	0.000519806197145862\\
551	0.000365747564450615\\
552	0.000203770911622493\\
553	2.3893647376284e-05\\
554	0\\
555	0\\
556	0\\
557	0\\
558	0\\
559	0\\
560	0\\
561	0\\
562	0\\
563	0\\
564	0\\
565	0\\
566	0\\
567	0\\
568	0\\
569	0\\
570	0\\
571	0\\
572	0\\
573	0\\
574	0\\
575	0\\
576	0\\
577	0\\
578	0\\
579	0\\
580	0\\
581	0\\
582	0\\
583	0\\
584	0\\
585	0\\
586	0\\
587	0\\
588	0\\
589	0\\
590	0\\
591	0\\
592	0\\
593	0\\
594	0\\
595	0\\
596	0\\
597	0\\
598	0\\
599	0\\
600	0\\
};
\addplot [color=mycolor12,solid,forget plot]
  table[row sep=crcr]{%
1	0.00638811260735049\\
2	0.00638811227827753\\
3	0.00638811194357173\\
4	0.00638811160313645\\
5	0.00638811125687342\\
6	0.00638811090468267\\
7	0.00638811054646253\\
8	0.00638811018210957\\
9	0.00638810981151861\\
10	0.00638810943458267\\
11	0.00638810905119292\\
12	0.00638810866123869\\
13	0.00638810826460739\\
14	0.00638810786118453\\
15	0.00638810745085365\\
16	0.0063881070334963\\
17	0.006388106608992\\
18	0.0063881061772182\\
19	0.00638810573805029\\
20	0.00638810529136148\\
21	0.00638810483702286\\
22	0.00638810437490328\\
23	0.00638810390486936\\
24	0.00638810342678545\\
25	0.00638810294051357\\
26	0.00638810244591339\\
27	0.00638810194284216\\
28	0.00638810143115473\\
29	0.00638810091070344\\
30	0.00638810038133811\\
31	0.00638809984290601\\
32	0.00638809929525179\\
33	0.00638809873821745\\
34	0.00638809817164229\\
35	0.00638809759536288\\
36	0.00638809700921297\\
37	0.00638809641302351\\
38	0.00638809580662254\\
39	0.00638809518983516\\
40	0.00638809456248351\\
41	0.00638809392438669\\
42	0.00638809327536069\\
43	0.00638809261521839\\
44	0.00638809194376946\\
45	0.00638809126082035\\
46	0.00638809056617417\\
47	0.00638808985963071\\
48	0.00638808914098634\\
49	0.00638808841003393\\
50	0.00638808766656286\\
51	0.00638808691035891\\
52	0.0063880861412042\\
53	0.00638808535887714\\
54	0.00638808456315237\\
55	0.00638808375380071\\
56	0.00638808293058904\\
57	0.00638808209328031\\
58	0.00638808124163341\\
59	0.00638808037540313\\
60	0.0063880794943401\\
61	0.00638807859819071\\
62	0.00638807768669701\\
63	0.00638807675959669\\
64	0.00638807581662296\\
65	0.00638807485750451\\
66	0.00638807388196541\\
67	0.00638807288972504\\
68	0.00638807188049802\\
69	0.0063880708539941\\
70	0.00638806980991812\\
71	0.0063880687479699\\
72	0.00638806766784416\\
73	0.00638806656923044\\
74	0.00638806545181301\\
75	0.00638806431527078\\
76	0.0063880631592772\\
77	0.00638806198350021\\
78	0.00638806078760207\\
79	0.00638805957123936\\
80	0.00638805833406281\\
81	0.00638805707571723\\
82	0.00638805579584141\\
83	0.00638805449406802\\
84	0.0063880531700235\\
85	0.00638805182332797\\
86	0.00638805045359511\\
87	0.00638804906043206\\
88	0.00638804764343931\\
89	0.00638804620221056\\
90	0.00638804473633268\\
91	0.0063880432453855\\
92	0.00638804172894177\\
93	0.00638804018656701\\
94	0.00638803861781938\\
95	0.00638803702224957\\
96	0.00638803539940067\\
97	0.00638803374880806\\
98	0.00638803206999925\\
99	0.00638803036249378\\
100	0.00638802862580305\\
101	0.00638802685943024\\
102	0.00638802506287009\\
103	0.00638802323560886\\
104	0.00638802137712409\\
105	0.00638801948688452\\
106	0.00638801756434994\\
107	0.00638801560897098\\
108	0.00638801362018904\\
109	0.00638801159743608\\
110	0.00638800954013446\\
111	0.00638800744769684\\
112	0.00638800531952593\\
113	0.00638800315501439\\
114	0.00638800095354464\\
115	0.00638799871448869\\
116	0.00638799643720797\\
117	0.00638799412105314\\
118	0.00638799176536393\\
119	0.00638798936946893\\
120	0.00638798693268543\\
121	0.00638798445431922\\
122	0.00638798193366442\\
123	0.00638797937000323\\
124	0.00638797676260578\\
125	0.00638797411072992\\
126	0.00638797141362101\\
127	0.0063879686705117\\
128	0.00638796588062174\\
129	0.00638796304315773\\
130	0.00638796015731296\\
131	0.00638795722226712\\
132	0.00638795423718612\\
133	0.00638795120122183\\
134	0.00638794811351189\\
135	0.00638794497317941\\
136	0.00638794177933278\\
137	0.0063879385310654\\
138	0.00638793522745544\\
139	0.00638793186756558\\
140	0.00638792845044275\\
141	0.00638792497511786\\
142	0.00638792144060558\\
143	0.00638791784590399\\
144	0.00638791418999439\\
145	0.00638791047184096\\
146	0.0063879066903905\\
147	0.00638790284457213\\
148	0.006387898933297\\
149	0.006387894955458\\
150	0.00638789090992943\\
151	0.00638788679556675\\
152	0.00638788261120617\\
153	0.00638787835566442\\
154	0.0063878740277384\\
155	0.00638786962620481\\
156	0.00638786514981987\\
157	0.00638786059731894\\
158	0.0063878559674162\\
159	0.00638785125880428\\
160	0.0063878464701539\\
161	0.00638784160011354\\
162	0.00638783664730901\\
163	0.00638783161034311\\
164	0.00638782648779528\\
165	0.00638782127822112\\
166	0.00638781598015209\\
167	0.00638781059209505\\
168	0.00638780511253186\\
169	0.00638779953991899\\
170	0.00638779387268706\\
171	0.00638778810924043\\
172	0.00638778224795679\\
173	0.00638777628718663\\
174	0.00638777022525291\\
175	0.00638776406045047\\
176	0.00638775779104569\\
177	0.00638775141527589\\
178	0.00638774493134897\\
179	0.00638773833744281\\
180	0.00638773163170484\\
181	0.00638772481225151\\
182	0.00638771787716775\\
183	0.00638771082450651\\
184	0.00638770365228814\\
185	0.00638769635849991\\
186	0.00638768894109543\\
187	0.00638768139799409\\
188	0.00638767372708052\\
189	0.00638766592620393\\
190	0.00638765799317761\\
191	0.00638764992577827\\
192	0.00638764172174544\\
193	0.00638763337878085\\
194	0.00638762489454782\\
195	0.00638761626667057\\
196	0.00638760749273358\\
197	0.00638759857028096\\
198	0.00638758949681571\\
199	0.00638758026979909\\
200	0.00638757088664987\\
201	0.00638756134474363\\
202	0.00638755164141204\\
203	0.00638754177394213\\
204	0.0063875317395755\\
205	0.00638752153550759\\
206	0.00638751115888687\\
207	0.00638750060681408\\
208	0.0063874898763414\\
209	0.00638747896447161\\
210	0.0063874678681573\\
211	0.00638745658429997\\
212	0.00638744510974919\\
213	0.00638743344130171\\
214	0.00638742157570056\\
215	0.00638740950963414\\
216	0.00638739723973526\\
217	0.00638738476258024\\
218	0.0063873720746879\\
219	0.00638735917251859\\
220	0.00638734605247319\\
221	0.00638733271089208\\
222	0.00638731914405413\\
223	0.00638730534817558\\
224	0.00638729131940902\\
225	0.00638727705384226\\
226	0.00638726254749722\\
227	0.00638724779632878\\
228	0.00638723279622361\\
229	0.00638721754299901\\
230	0.00638720203240168\\
231	0.00638718626010648\\
232	0.0063871702217152\\
233	0.00638715391275527\\
234	0.00638713732867844\\
235	0.00638712046485946\\
236	0.00638710331659474\\
237	0.00638708587910093\\
238	0.00638706814751352\\
239	0.00638705011688543\\
240	0.00638703178218551\\
241	0.00638701313829702\\
242	0.00638699418001616\\
243	0.00638697490205046\\
244	0.00638695529901721\\
245	0.0063869353654418\\
246	0.00638691509575612\\
247	0.00638689448429679\\
248	0.00638687352530348\\
249	0.00638685221291712\\
250	0.00638683054117809\\
251	0.00638680850402438\\
252	0.00638678609528968\\
253	0.00638676330870148\\
254	0.00638674013787907\\
255	0.00638671657633152\\
256	0.00638669261745562\\
257	0.00638666825453377\\
258	0.00638664348073181\\
259	0.00638661828909679\\
260	0.0063865926725547\\
261	0.00638656662390817\\
262	0.00638654013583404\\
263	0.00638651320088097\\
264	0.00638648581146687\\
265	0.00638645795987635\\
266	0.0063864296382581\\
267	0.00638640083862214\\
268	0.00638637155283704\\
269	0.00638634177262707\\
270	0.00638631148956926\\
271	0.00638628069509043\\
272	0.00638624938046401\\
273	0.00638621753680686\\
274	0.00638618515507574\\
275	0.00638615222606409\\
276	0.00638611874039846\\
277	0.00638608468853491\\
278	0.00638605006075521\\
279	0.00638601484716301\\
280	0.00638597903767974\\
281	0.00638594262204055\\
282	0.00638590558978992\\
283	0.00638586793027723\\
284	0.00638582963265214\\
285	0.00638579068585975\\
286	0.00638575107863561\\
287	0.00638571079950053\\
288	0.00638566983675517\\
289	0.00638562817847436\\
290	0.00638558581250126\\
291	0.00638554272644116\\
292	0.00638549890765511\\
293	0.00638545434325314\\
294	0.00638540902008723\\
295	0.00638536292474393\\
296	0.00638531604353653\\
297	0.00638526836249695\\
298	0.00638521986736709\\
299	0.00638517054358977\\
300	0.00638512037629914\\
301	0.00638506935031051\\
302	0.00638501745010968\\
303	0.00638496465984148\\
304	0.00638491096329772\\
305	0.00638485634390431\\
306	0.00638480078470742\\
307	0.00638474426835884\\
308	0.00638468677710006\\
309	0.00638462829274554\\
310	0.00638456879666494\\
311	0.00638450826976433\\
312	0.00638444669246502\\
313	0.00638438404468094\\
314	0.00638432030579448\\
315	0.00638425545463028\\
316	0.00638418946942683\\
317	0.00638412232780574\\
318	0.00638405400673823\\
319	0.00638398448250878\\
320	0.00638391373067538\\
321	0.00638384172602621\\
322	0.00638376844253228\\
323	0.00638369385329557\\
324	0.00638361793049211\\
325	0.00638354064530964\\
326	0.00638346196787895\\
327	0.00638338186719835\\
328	0.00638330031105052\\
329	0.00638321726591067\\
330	0.00638313269684517\\
331	0.00638304656739947\\
332	0.00638295883947404\\
333	0.00638286947318688\\
334	0.0063827784267211\\
335	0.0063826856561557\\
336	0.00638259111527748\\
337	0.00638249475537197\\
338	0.00638239652499067\\
339	0.00638229636969169\\
340	0.00638219423175074\\
341	0.00638209004983853\\
342	0.00638198375866058\\
343	0.00638187528855478\\
344	0.00638176456504129\\
345	0.00638165150831882\\
346	0.00638153603270058\\
347	0.00638141804598205\\
348	0.00638129744873196\\
349	0.00638117413349642\\
350	0.00638104798390462\\
351	0.00638091887366286\\
352	0.00638078666541956\\
353	0.00638065120947726\\
354	0.00638051234231107\\
355	0.00638036988481699\\
356	0.00638022364013296\\
357	0.0063800733906508\\
358	0.00637991889324608\\
359	0.00637975987032702\\
360	0.0063795959907522\\
361	0.00637942682626216\\
362	0.00637925175035374\\
363	0.00637906970942778\\
364	0.00637887874041027\\
365	0.00637867510162692\\
366	0.00637845230977798\\
367	0.00637820317999826\\
368	0.0063779486827292\\
369	0.00637769033540312\\
370	0.00637742808668752\\
371	0.00637716188478615\\
372	0.00637689167741668\\
373	0.00637661741192001\\
374	0.00637633903573714\\
375	0.00637605649738284\\
376	0.00637576974673577\\
377	0.00637547873112774\\
378	0.00637518339603238\\
379	0.00637488368657652\\
380	0.00637457954752541\\
381	0.00637427092326166\\
382	0.00637395775775675\\
383	0.00637363999453385\\
384	0.00637331757662061\\
385	0.00637299044649023\\
386	0.0063726585459893\\
387	0.00637232181625021\\
388	0.00637198019758615\\
389	0.00637163362936644\\
390	0.00637128204986959\\
391	0.00637092539611196\\
392	0.00637056360364985\\
393	0.00637019660635467\\
394	0.00636982433616361\\
395	0.00636944672281565\\
396	0.00636906369359698\\
397	0.00636867517314809\\
398	0.00636828108343642\\
399	0.00636788134409195\\
400	0.00636747587349228\\
401	0.00636706459147168\\
402	0.00636664742611355\\
403	0.00636622433229439\\
404	0.00636579534338305\\
405	0.00636536069391904\\
406	0.00636492099227602\\
407	0.00636447744632584\\
408	0.00636403184281951\\
409	0.00636358533742221\\
410	0.0063631352141959\\
411	0.00636267744808165\\
412	0.0063622118967298\\
413	0.00636173840340145\\
414	0.00636125680084869\\
415	0.00636076691629268\\
416	0.00636026857114308\\
417	0.00635976158068075\\
418	0.00635924575368311\\
419	0.00635872089199831\\
420	0.00635818679017809\\
421	0.00635764323535161\\
422	0.00635709000691684\\
423	0.00635652687590073\\
424	0.00635595360450518\\
425	0.00635536994562446\\
426	0.00635477564233225\\
427	0.00635417042733614\\
428	0.0063535540223977\\
429	0.00635292613771579\\
430	0.00635228647127053\\
431	0.00635163470812541\\
432	0.0063509705196839\\
433	0.0063502935628958\\
434	0.00634960347940539\\
435	0.00634889989462626\\
436	0.00634818241671315\\
437	0.00634745063537106\\
438	0.00634670412039079\\
439	0.00634594241973513\\
440	0.00634516505701116\\
441	0.00634437152861232\\
442	0.00634356130271134\\
443	0.00634273382722985\\
444	0.00634188855859463\\
445	0.0063410249902627\\
446	0.00634014249251388\\
447	0.00633924039810001\\
448	0.00633831800238986\\
449	0.00633737455658749\\
450	0.0063364092506127\\
451	0.00633542117093243\\
452	0.00633440919541388\\
453	0.00633337173729664\\
454	0.00633230615981661\\
455	0.00633120761774576\\
456	0.00633006763021697\\
457	0.00632887711133775\\
458	0.00632766282066212\\
459	0.00632642830815137\\
460	0.00632516878913098\\
461	0.00632387383674367\\
462	0.00632252232735194\\
463	0.00632108164693185\\
464	0.00631953543083207\\
465	0.00631779877024216\\
466	0.00631550642290275\\
467	0.00631181128253835\\
468	0.00630758251061971\\
469	0.0063032886687178\\
470	0.00629892670499869\\
471	0.00629449405557355\\
472	0.00628998981684965\\
473	0.00628541731536539\\
474	0.00628078894352176\\
475	0.00627613232540686\\
476	0.00627148627539327\\
477	0.00626684319508133\\
478	0.00626204858967383\\
479	0.00625711260959464\\
480	0.00625202665348447\\
481	0.00624677980986374\\
482	0.00624135926103719\\
483	0.00623574865960632\\
484	0.00622992393734656\\
485	0.00622384233034361\\
486	0.00621741401359264\\
487	0.00621043315997757\\
488	0.00620244420629681\\
489	0.00619400965776899\\
490	0.0061854524589268\\
491	0.00617676989141488\\
492	0.00616795912551411\\
493	0.0061590172263189\\
494	0.00614994117299156\\
495	0.00614072789386952\\
496	0.00613137429602339\\
497	0.00612187716581177\\
498	0.00611223259079458\\
499	0.00610243450128191\\
500	0.00609247530938651\\
501	0.00608234825131482\\
502	0.0060720425385402\\
503	0.00606153892911889\\
504	0.00605084367335183\\
505	0.00603994004356179\\
506	0.00602877154022212\\
507	0.00601721376170444\\
508	0.00600523421996853\\
509	0.00599317503690893\\
510	0.00598101128349881\\
511	0.00596867831489254\\
512	0.00595601896616613\\
513	0.00594273840786949\\
514	0.00592896458149047\\
515	0.00591514726359842\\
516	0.00590129512228386\\
517	0.00588743147298232\\
518	0.00587361595475236\\
519	0.00585999549864396\\
520	0.00584690599328163\\
521	0.00583500592475455\\
522	0.00582490792822357\\
523	0.0058146042098332\\
524	0.00580398987612161\\
525	0.00579273376978727\\
526	0.00577980378322902\\
527	0.00576202624003748\\
528	0.00572377841157942\\
529	0.00567699475830189\\
530	0.00562869022943325\\
531	0.0055786912893133\\
532	0.00552598034901514\\
533	0.00546614625759935\\
534	0.00539998269028576\\
535	0.00533380742098718\\
536	0.00526762319228078\\
537	0.00520142728789096\\
538	0.00513519630576685\\
539	0.00506883791671606\\
540	0.00500203947876236\\
541	0.00493320527137919\\
542	0.00486390184218159\\
543	0.00479565705149332\\
544	0.00472849590663815\\
545	0.00466244415217564\\
546	0.00459752828388643\\
547	0.0045337742286058\\
548	0.00447120240486943\\
549	0.00440981133989045\\
550	0.00434952421512032\\
551	0.00429001456373431\\
552	0.00423008609656117\\
553	0.0041662083985624\\
554	0.00410400778309653\\
555	0.00404418208447621\\
556	0.00398683441474163\\
557	0.00393197442332481\\
558	0.00387962678014643\\
559	0.00382986338433296\\
560	0.00378275613597533\\
561	0.00373847095787726\\
562	0.00369740504797728\\
563	0.00366052991955191\\
564	0.00363023942217948\\
565	0.00360500619487998\\
566	0.00352973838064425\\
567	0.00344272230220717\\
568	0.00334614191579287\\
569	0.00324691932347805\\
570	0.00314497980923681\\
571	0.0030401664970947\\
572	0.00293180965279257\\
573	0.00281238234143975\\
574	0.00264080565626875\\
575	0.00246652723666886\\
576	0.00228946683670179\\
577	0.00210943841006402\\
578	0.00192677268954282\\
579	0.00174132574584865\\
580	0.00155260207180459\\
581	0.00136033065535838\\
582	0.00116605697170137\\
583	0.000969668956010576\\
584	0.000770929493206139\\
585	0.000569036622337176\\
586	0.000361344392078551\\
587	0.000140168163487911\\
588	0\\
589	0\\
590	0\\
591	0\\
592	0\\
593	0\\
594	0\\
595	0\\
596	0\\
597	0\\
598	0\\
599	0\\
600	0\\
};
\addplot [color=mycolor13,solid,forget plot]
  table[row sep=crcr]{%
1	0.00142341042040566\\
2	0.00142341042040566\\
3	0.00142341042040566\\
4	0.00142341042040566\\
5	0.00142341042040566\\
6	0.00142341042040566\\
7	0.00142341042040566\\
8	0.00142341042040566\\
9	0.00142341042040566\\
10	0.00142341042040566\\
11	0.00142341042040566\\
12	0.00142341042040566\\
13	0.00142341042040566\\
14	0.00142341042040566\\
15	0.00142341042040566\\
16	0.00142341042040566\\
17	0.00142341042040566\\
18	0.00142341042040566\\
19	0.00142341042040566\\
20	0.00142341042040566\\
21	0.00142341042040566\\
22	0.00142341042040566\\
23	0.00142341042040566\\
24	0.00142341042040566\\
25	0.00142341042040566\\
26	0.00142341042040566\\
27	0.00142341042040566\\
28	0.00142341042040566\\
29	0.00142341042040566\\
30	0.00142341042040566\\
31	0.00142341042040566\\
32	0.00142341042040566\\
33	0.00142341042040566\\
34	0.00142341042040566\\
35	0.00142341042040566\\
36	0.00142341042040566\\
37	0.00142341042040566\\
38	0.00142341042040566\\
39	0.00142341042040566\\
40	0.00142341042040566\\
41	0.00142341042040566\\
42	0.00142341042040566\\
43	0.00142341042040566\\
44	0.00142341042040566\\
45	0.00142341042040566\\
46	0.00142341042040566\\
47	0.00142341042040566\\
48	0.00142341042040566\\
49	0.00142341042040566\\
50	0.00142341042040566\\
51	0.00142341042040566\\
52	0.00142341042040566\\
53	0.00142341042040566\\
54	0.00142341042040566\\
55	0.00142341042040566\\
56	0.00142341042040566\\
57	0.00142341042040566\\
58	0.00142341042040566\\
59	0.00142341042040566\\
60	0.00142341042040566\\
61	0.00142341042040566\\
62	0.00142341042040566\\
63	0.00142341042040566\\
64	0.00142341042040566\\
65	0.00142341042040566\\
66	0.00142341042040566\\
67	0.00142341042040566\\
68	0.00142341042040566\\
69	0.00142341042040566\\
70	0.00142341042040566\\
71	0.00142341042040566\\
72	0.00142341042040566\\
73	0.00142341042040566\\
74	0.00142341042040566\\
75	0.00142341042040566\\
76	0.00142341042040566\\
77	0.00142341042040566\\
78	0.00142341042040566\\
79	0.00142341042040566\\
80	0.00142341042040566\\
81	0.00142341042040566\\
82	0.00142341042040566\\
83	0.00142341042040566\\
84	0.00142341042040566\\
85	0.00142341042040566\\
86	0.00142341042040566\\
87	0.00142341042040566\\
88	0.00142341042040566\\
89	0.00142341042040566\\
90	0.00142341042040566\\
91	0.00142341042040566\\
92	0.00142341042040566\\
93	0.00142341042040566\\
94	0.00142341042040566\\
95	0.00142341042040566\\
96	0.00142341042040566\\
97	0.00142341042040566\\
98	0.00142341042040566\\
99	0.00142341042040566\\
100	0.00142341042040566\\
101	0.00142341042040566\\
102	0.00142341042040566\\
103	0.00142341042040566\\
104	0.00142341042040566\\
105	0.00142341042040566\\
106	0.00142341042040566\\
107	0.00142341042040566\\
108	0.00142341042040566\\
109	0.00142341042040566\\
110	0.00142341042040566\\
111	0.00142341042040566\\
112	0.00142341042040566\\
113	0.00142341042040566\\
114	0.00142341042040566\\
115	0.00142341042040566\\
116	0.00142341042040566\\
117	0.00142341042040566\\
118	0.00142341042040566\\
119	0.00142341042040566\\
120	0.00142341042040566\\
121	0.00142341042040566\\
122	0.00142341042040566\\
123	0.00142341042040566\\
124	0.00142341042040566\\
125	0.00142341042040566\\
126	0.00142341042040566\\
127	0.00142341042040566\\
128	0.00142341042040566\\
129	0.00142341042040566\\
130	0.00142341042040566\\
131	0.00142341042040566\\
132	0.00142341042040566\\
133	0.00142341042040566\\
134	0.00142341042040566\\
135	0.00142341042040566\\
136	0.00142341042040566\\
137	0.00142341042040566\\
138	0.00142341042040566\\
139	0.00142341042040566\\
140	0.00142341042040566\\
141	0.00142341042040566\\
142	0.00142341042040566\\
143	0.00142341042040566\\
144	0.00142341042040566\\
145	0.00142341042040566\\
146	0.00142341042040566\\
147	0.00142341042040566\\
148	0.00142341042040566\\
149	0.00142341042040566\\
150	0.00142341042040566\\
151	0.00142341042040566\\
152	0.00142341042040566\\
153	0.00142341042040566\\
154	0.00142341042040566\\
155	0.00142341042040566\\
156	0.00142341042040566\\
157	0.00142341042040566\\
158	0.00142341042040566\\
159	0.00142341042040566\\
160	0.00142341042040566\\
161	0.00142341042040566\\
162	0.00142341042040566\\
163	0.00142341042040566\\
164	0.00142341042040566\\
165	0.00142341042040566\\
166	0.00142341042040566\\
167	0.00142341042040566\\
168	0.00142341042040566\\
169	0.00142341042040566\\
170	0.00142341042040566\\
171	0.00142341042040566\\
172	0.00142341042040566\\
173	0.00142341042040566\\
174	0.00142341042040566\\
175	0.00142341042040566\\
176	0.00142341042040566\\
177	0.00142341042040566\\
178	0.00142341042040566\\
179	0.00142341042040566\\
180	0.00142341042040566\\
181	0.00142341042040566\\
182	0.00142341042040566\\
183	0.00142341042040566\\
184	0.00142341042040566\\
185	0.00142341042040566\\
186	0.00142341042040566\\
187	0.00142341042040566\\
188	0.00142341042040566\\
189	0.00142341042040566\\
190	0.00142341042040566\\
191	0.00142341042040566\\
192	0.00142341042040566\\
193	0.00142341042040566\\
194	0.00142341042040566\\
195	0.00142341042040566\\
196	0.00142341042040566\\
197	0.00142341042040566\\
198	0.00142341042040566\\
199	0.00142341042040566\\
200	0.00142341042040566\\
201	0.00142341042040566\\
202	0.00142341042040566\\
203	0.00142341042040566\\
204	0.00142341042040566\\
205	0.00142341042040566\\
206	0.00142341042040566\\
207	0.00142341042040566\\
208	0.00142341042040566\\
209	0.00142341042040566\\
210	0.00142341042040566\\
211	0.00142341042040566\\
212	0.00142341042040566\\
213	0.00142341042040566\\
214	0.00142341042040566\\
215	0.00142341042040566\\
216	0.00142341042040566\\
217	0.00142341042040566\\
218	0.00142341042040566\\
219	0.00142341042040566\\
220	0.00142341042040566\\
221	0.00142341042040566\\
222	0.00142341042040566\\
223	0.00142341042040566\\
224	0.00142341042040566\\
225	0.00142341042040566\\
226	0.00142341042040566\\
227	0.00142341042040566\\
228	0.00142341042040566\\
229	0.00142341042040566\\
230	0.00142341042040566\\
231	0.00142341042040566\\
232	0.00142341042040566\\
233	0.00142341042040566\\
234	0.00142341042040566\\
235	0.00142341042040566\\
236	0.00142341042040566\\
237	0.00142341042040566\\
238	0.00142341042040566\\
239	0.00142341042040566\\
240	0.00142341042040566\\
241	0.00142341042040566\\
242	0.00142341042040566\\
243	0.00142341042040566\\
244	0.00142341042040566\\
245	0.00142341042040566\\
246	0.00142341042040566\\
247	0.00142341042040566\\
248	0.00142341042040566\\
249	0.00142341042040566\\
250	0.00142341042040566\\
251	0.00142341042040566\\
252	0.00142341042040566\\
253	0.00142341042040566\\
254	0.00142341042040566\\
255	0.00142341042040566\\
256	0.00142341042040566\\
257	0.00142341042040566\\
258	0.00142341042040566\\
259	0.00142341042040566\\
260	0.00142341042040566\\
261	0.00142341042040566\\
262	0.00142341042040566\\
263	0.00142341042040566\\
264	0.00142341042040566\\
265	0.00142341042040566\\
266	0.00142341042040566\\
267	0.00142341042040566\\
268	0.00142341042040566\\
269	0.00142341042040566\\
270	0.00142341042040566\\
271	0.00142341042040566\\
272	0.00142341042040566\\
273	0.00142341042040566\\
274	0.00142341042040566\\
275	0.00142341042040566\\
276	0.00142341042040566\\
277	0.00142341042040566\\
278	0.00142341042040566\\
279	0.00142341042040566\\
280	0.00142341042040566\\
281	0.00142341042040566\\
282	0.00142341042040566\\
283	0.00142341042040566\\
284	0.00142341042040566\\
285	0.00142341042040566\\
286	0.00142341042040566\\
287	0.00142341042040566\\
288	0.00142341042040566\\
289	0.00142341042040566\\
290	0.00142341042040566\\
291	0.00142341042040566\\
292	0.00142341042040566\\
293	0.00142341042040566\\
294	0.00142341042040566\\
295	0.00142341042040566\\
296	0.00142341042040566\\
297	0.00142341042040566\\
298	0.00142341042040566\\
299	0.00142341042040566\\
300	0.00142341042040566\\
301	0.00142341042040566\\
302	0.00142341042040566\\
303	0.00142341042040566\\
304	0.00142341042040566\\
305	0.00142341042040566\\
306	0.00142341042040566\\
307	0.00142341042040566\\
308	0.00142341042040566\\
309	0.00142341042040566\\
310	0.00142341042040566\\
311	0.00142341042040566\\
312	0.00142341042040566\\
313	0.00142341042040566\\
314	0.00142341042040566\\
315	0.00142341042040566\\
316	0.00142341042040566\\
317	0.00142341042040566\\
318	0.00142341042040566\\
319	0.00142341042040566\\
320	0.00142341042040566\\
321	0.00142341042040566\\
322	0.00142341042040566\\
323	0.00142341042040566\\
324	0.00142341042040566\\
325	0.00142341042040566\\
326	0.00142341042040566\\
327	0.00142341042040566\\
328	0.00142341042040566\\
329	0.00142341042040566\\
330	0.00142341042040566\\
331	0.00142341042040566\\
332	0.00142341042040566\\
333	0.00142341042040566\\
334	0.00142341042040566\\
335	0.00142341042040566\\
336	0.00142341042040566\\
337	0.00142341042040566\\
338	0.00142341042040566\\
339	0.00142341042040566\\
340	0.00142341042040566\\
341	0.00142341042040566\\
342	0.00142341042040566\\
343	0.00142341042040566\\
344	0.00142341042040566\\
345	0.00142341042040566\\
346	0.00142341042040566\\
347	0.00142341042040566\\
348	0.00142341042040566\\
349	0.00142341042040566\\
350	0.00142341042040566\\
351	0.00142341042040566\\
352	0.00142341042040566\\
353	0.00142341042040566\\
354	0.00142341042040566\\
355	0.00142341042040566\\
356	0.00142341042040566\\
357	0.00142341042040566\\
358	0.00142341042040566\\
359	0.00142341042040566\\
360	0.00142341042040566\\
361	0.00142341042040566\\
362	0.00142341042040566\\
363	0.00142341042040566\\
364	0.00142341042040566\\
365	0.00142341042040566\\
366	0.00142341042040566\\
367	0.00142341042040566\\
368	0.00142341042040566\\
369	0.00142341042040566\\
370	0.00142341042040566\\
371	0.00142341042040566\\
372	0.00142341042040566\\
373	0.00142341042040566\\
374	0.00142341042040566\\
375	0.00142341042040566\\
376	0.00142341042040566\\
377	0.00142341042040566\\
378	0.00142341042040566\\
379	0.00142341042040566\\
380	0.00142341042040566\\
381	0.00142341042040566\\
382	0.00142341042040566\\
383	0.00142341042040566\\
384	0.00142341042040566\\
385	0.00142341042040566\\
386	0.00142341042040566\\
387	0.00142341042040566\\
388	0.00142341042040566\\
389	0.00142341042040566\\
390	0.00142341042040566\\
391	0.00142341042040566\\
392	0.00142341042040566\\
393	0.00142341042040566\\
394	0.00142341042040566\\
395	0.00142341042040566\\
396	0.00142341042040566\\
397	0.00142341042040566\\
398	0.00142341042040566\\
399	0.00142341042040566\\
400	0.00142341042040566\\
401	0.00142341042040566\\
402	0.00142341042040566\\
403	0.00142341042040566\\
404	0.00142341042040566\\
405	0.00142341042040566\\
406	0.00142341042040566\\
407	0.00142341042040566\\
408	0.00142341042040566\\
409	0.00142341042040566\\
410	0.00142341042040566\\
411	0.00142341042040566\\
412	0.00142341042040566\\
413	0.00142341042040566\\
414	0.00142341042040566\\
415	0.00142341042040566\\
416	0.00142341042040566\\
417	0.00142341042040566\\
418	0.00142341042040566\\
419	0.00142341042040566\\
420	0.00142341042040566\\
421	0.00142341042040566\\
422	0.00142341042040566\\
423	0.00142341042040566\\
424	0.00142341042040566\\
425	0.00142341042040566\\
426	0.00142341042040566\\
427	0.00142341042040566\\
428	0.00142341042040566\\
429	0.00142341042040566\\
430	0.00142341042040566\\
431	0.00142341042040566\\
432	0.00142341042040566\\
433	0.00142341042040566\\
434	0.00142341042040566\\
435	0.00142341042040566\\
436	0.00142341042040566\\
437	0.00142341042040566\\
438	0.00142341042040566\\
439	0.00142341042040566\\
440	0.00142341042040566\\
441	0.00142341042040566\\
442	0.00142341042040566\\
443	0.00142341042040566\\
444	0.00142341042040566\\
445	0.00142341042040566\\
446	0.00142341042040566\\
447	0.00142341042040566\\
448	0.00142341042040566\\
449	0.00142341042040566\\
450	0.00142341042040566\\
451	0.00142341042040566\\
452	0.00142341042040566\\
453	0.00142341042040566\\
454	0.00142341042040566\\
455	0.00142341042040566\\
456	0.00142341042040566\\
457	0.00142341042040566\\
458	0.00142341042040566\\
459	0.00142341042040566\\
460	0.00142341042040566\\
461	0.00142341042040566\\
462	0.00142341042040566\\
463	0.00142341042040566\\
464	0.00142341042040566\\
465	0.00142341042040566\\
466	0.00142341042040566\\
467	0.00142341042040566\\
468	0.00142341042040566\\
469	0.00142341042040566\\
470	0.00142341042040566\\
471	0.00142341042040566\\
472	0.00142341042040566\\
473	0.00142341042040566\\
474	0.00142341042040566\\
475	0.00142341042040566\\
476	0.00142341042040566\\
477	0.00142341042040566\\
478	0.00142341042040566\\
479	0.00142341042040566\\
480	0.00142341042040566\\
481	0.00142341042040566\\
482	0.00142341042040566\\
483	0.00142341042040566\\
484	0.00142341042040566\\
485	0.00142341042040566\\
486	0.00142341042040566\\
487	0.00142341042040566\\
488	0.00142341042040566\\
489	0.00142341042040566\\
490	0.00142341042040566\\
491	0.00142341042040566\\
492	0.00142341042040566\\
493	0.00142341042040566\\
494	0.00142341042040566\\
495	0.00142341042040566\\
496	0.00142341042040566\\
497	0.00142341042040566\\
498	0.00142341042040566\\
499	0.00142341042040566\\
500	0.00142341042040566\\
501	0.00142341042040566\\
502	0.00142341042040566\\
503	0.00142341042040566\\
504	0.00142341042040566\\
505	0.00142341042040566\\
506	0.00142341042040566\\
507	0.00142341042040566\\
508	0.00142341042040566\\
509	0.00142341042040566\\
510	0.00142341042040566\\
511	0.00142341042040566\\
512	0.00142341042040566\\
513	0.00142341042040566\\
514	0.00142341042040566\\
515	0.00142341042040566\\
516	0.00142341042040566\\
517	0.00142341042040566\\
518	0.00142341042040566\\
519	0.00142341042040566\\
520	0.00142341042040566\\
521	0.00142341042040566\\
522	0.00142341042040566\\
523	0.00142341042040566\\
524	0.00142341042040566\\
525	0.00142341042040566\\
526	0.00142341042040566\\
527	0.00142341042040566\\
528	0.00142341042040566\\
529	0.00142341042040566\\
530	0.00142341042040566\\
531	0.00142341042040566\\
532	0.00142341042040566\\
533	0.00142341042040566\\
534	0.00142341042040566\\
535	0.00142341042040566\\
536	0.00142341042040566\\
537	0.00142341042040566\\
538	0.00142341042040566\\
539	0.00142341042040566\\
540	0.00142341042040566\\
541	0.00142341042040566\\
542	0.00142341042040566\\
543	0.00142341042040566\\
544	0.00142341042040566\\
545	0.00142341042040566\\
546	0.00142341042040566\\
547	0.00142341042040566\\
548	0.00142341042040566\\
549	0.00142341042040566\\
550	0.00142341042040566\\
551	0.00142341042040566\\
552	0.00142341042040566\\
553	0.00142341042040566\\
554	0.00142341042040566\\
555	0.00142341042040566\\
556	0.00142341042040566\\
557	0.00142341042040566\\
558	0.00142341042040566\\
559	0.00142341042040566\\
560	0.00142341042040566\\
561	0.00142341042040566\\
562	0.00142341042040566\\
563	0.00142341042040566\\
564	0.00142341042040566\\
565	0.00141808680651778\\
566	0.00132573838914067\\
567	0.00121575937955735\\
568	0.00110234926539761\\
569	0.000991487125274636\\
570	0.000881523074320698\\
571	0.000772562041893994\\
572	0.000667166343829018\\
573	0.000574706660558508\\
574	0.000427188374771164\\
575	0.000287780647258003\\
576	0.000156402139129492\\
577	3.29294685952495e-05\\
578	0\\
579	0\\
580	0\\
581	0\\
582	0\\
583	0\\
584	0\\
585	0\\
586	0\\
587	0\\
588	0\\
589	0\\
590	0\\
591	0\\
592	5.56832960656924e-05\\
593	0.000164162311403634\\
594	0.00028999157753712\\
595	0.000432641173258269\\
596	0.000588326666482334\\
597	0.000749148955320047\\
598	0.00347642786857269\\
599	0\\
600	0\\
};
\addplot [color=mycolor14,solid,forget plot]
  table[row sep=crcr]{%
1	0\\
2	0\\
3	0\\
4	0\\
5	0\\
6	0\\
7	0\\
8	0\\
9	0\\
10	0\\
11	0\\
12	0\\
13	0\\
14	0\\
15	0\\
16	0\\
17	0\\
18	0\\
19	0\\
20	0\\
21	0\\
22	0\\
23	0\\
24	0\\
25	0\\
26	0\\
27	0\\
28	0\\
29	0\\
30	0\\
31	0\\
32	0\\
33	0\\
34	0\\
35	0\\
36	0\\
37	0\\
38	0\\
39	0\\
40	0\\
41	0\\
42	0\\
43	0\\
44	0\\
45	0\\
46	0\\
47	0\\
48	0\\
49	0\\
50	0\\
51	0\\
52	0\\
53	0\\
54	0\\
55	0\\
56	0\\
57	0\\
58	0\\
59	0\\
60	0\\
61	0\\
62	0\\
63	0\\
64	0\\
65	0\\
66	0\\
67	0\\
68	0\\
69	0\\
70	0\\
71	0\\
72	0\\
73	0\\
74	0\\
75	0\\
76	0\\
77	0\\
78	0\\
79	0\\
80	0\\
81	0\\
82	0\\
83	0\\
84	0\\
85	0\\
86	0\\
87	0\\
88	0\\
89	0\\
90	0\\
91	0\\
92	0\\
93	0\\
94	0\\
95	0\\
96	0\\
97	0\\
98	0\\
99	0\\
100	0\\
101	0\\
102	0\\
103	0\\
104	0\\
105	0\\
106	0\\
107	0\\
108	0\\
109	0\\
110	0\\
111	0\\
112	0\\
113	0\\
114	0\\
115	0\\
116	0\\
117	0\\
118	0\\
119	0\\
120	0\\
121	0\\
122	0\\
123	0\\
124	0\\
125	0\\
126	0\\
127	0\\
128	0\\
129	0\\
130	0\\
131	0\\
132	0\\
133	0\\
134	0\\
135	0\\
136	0\\
137	0\\
138	0\\
139	0\\
140	0\\
141	0\\
142	0\\
143	0\\
144	0\\
145	0\\
146	0\\
147	0\\
148	0\\
149	0\\
150	0\\
151	0\\
152	0\\
153	0\\
154	0\\
155	0\\
156	0\\
157	0\\
158	0\\
159	0\\
160	0\\
161	0\\
162	0\\
163	0\\
164	0\\
165	0\\
166	0\\
167	0\\
168	0\\
169	0\\
170	0\\
171	0\\
172	0\\
173	0\\
174	0\\
175	0\\
176	0\\
177	0\\
178	0\\
179	0\\
180	0\\
181	0\\
182	0\\
183	0\\
184	0\\
185	0\\
186	0\\
187	0\\
188	0\\
189	0\\
190	0\\
191	0\\
192	0\\
193	0\\
194	0\\
195	0\\
196	0\\
197	0\\
198	0\\
199	0\\
200	0\\
201	0\\
202	0\\
203	0\\
204	0\\
205	0\\
206	0\\
207	0\\
208	0\\
209	0\\
210	0\\
211	0\\
212	0\\
213	0\\
214	0\\
215	0\\
216	0\\
217	0\\
218	0\\
219	0\\
220	0\\
221	0\\
222	0\\
223	0\\
224	0\\
225	0\\
226	0\\
227	0\\
228	0\\
229	0\\
230	0\\
231	0\\
232	0\\
233	0\\
234	0\\
235	0\\
236	0\\
237	0\\
238	0\\
239	0\\
240	0\\
241	0\\
242	0\\
243	0\\
244	0\\
245	0\\
246	0\\
247	0\\
248	0\\
249	0\\
250	0\\
251	0\\
252	0\\
253	0\\
254	0\\
255	0\\
256	0\\
257	0\\
258	0\\
259	0\\
260	0\\
261	0\\
262	0\\
263	0\\
264	0\\
265	0\\
266	0\\
267	0\\
268	0\\
269	0\\
270	0\\
271	0\\
272	0\\
273	0\\
274	0\\
275	0\\
276	0\\
277	0\\
278	0\\
279	0\\
280	0\\
281	0\\
282	0\\
283	0\\
284	0\\
285	0\\
286	0\\
287	0\\
288	0\\
289	0\\
290	0\\
291	0\\
292	0\\
293	0\\
294	0\\
295	0\\
296	0\\
297	0\\
298	0\\
299	0\\
300	0\\
301	0\\
302	0\\
303	0\\
304	0\\
305	0\\
306	0\\
307	0\\
308	0\\
309	0\\
310	0\\
311	0\\
312	0\\
313	0\\
314	0\\
315	0\\
316	0\\
317	0\\
318	0\\
319	0\\
320	0\\
321	0\\
322	0\\
323	0\\
324	0\\
325	0\\
326	0\\
327	0\\
328	0\\
329	0\\
330	0\\
331	0\\
332	0\\
333	0\\
334	0\\
335	0\\
336	0\\
337	0\\
338	0\\
339	0\\
340	0\\
341	0\\
342	0\\
343	0\\
344	0\\
345	0\\
346	0\\
347	0\\
348	0\\
349	0\\
350	0\\
351	0\\
352	0\\
353	0\\
354	0\\
355	0\\
356	0\\
357	0\\
358	0\\
359	0\\
360	0\\
361	0\\
362	0\\
363	0\\
364	0\\
365	0\\
366	0\\
367	0\\
368	0\\
369	0\\
370	0\\
371	0\\
372	0\\
373	0\\
374	0\\
375	0\\
376	0\\
377	0\\
378	0\\
379	0\\
380	0\\
381	0\\
382	0\\
383	0\\
384	0\\
385	0\\
386	0\\
387	0\\
388	0\\
389	0\\
390	0\\
391	0\\
392	0\\
393	0\\
394	0\\
395	0\\
396	0\\
397	0\\
398	0\\
399	0\\
400	0\\
401	0\\
402	0\\
403	0\\
404	0\\
405	0\\
406	0\\
407	0\\
408	0\\
409	0\\
410	0\\
411	0\\
412	0\\
413	0\\
414	0\\
415	0\\
416	0\\
417	0\\
418	0\\
419	0\\
420	0\\
421	0\\
422	0\\
423	0\\
424	0\\
425	0\\
426	0\\
427	0\\
428	0\\
429	0\\
430	0\\
431	0\\
432	0\\
433	0\\
434	0\\
435	0\\
436	0\\
437	0\\
438	0\\
439	0\\
440	0\\
441	0\\
442	0\\
443	0\\
444	0\\
445	0\\
446	0\\
447	0\\
448	0\\
449	0\\
450	0\\
451	0\\
452	0\\
453	0\\
454	0\\
455	0\\
456	0\\
457	0\\
458	0\\
459	0\\
460	0\\
461	0\\
462	0\\
463	0\\
464	0\\
465	0\\
466	0\\
467	0\\
468	0\\
469	0\\
470	0\\
471	0\\
472	0\\
473	0\\
474	0\\
475	0\\
476	0\\
477	0\\
478	0\\
479	0\\
480	0\\
481	0\\
482	0\\
483	0\\
484	0\\
485	0\\
486	0\\
487	0\\
488	0\\
489	0\\
490	0\\
491	0\\
492	0\\
493	0\\
494	0\\
495	0\\
496	0\\
497	0\\
498	0\\
499	0\\
500	0\\
501	0\\
502	0\\
503	0\\
504	0\\
505	0\\
506	0\\
507	0\\
508	0\\
509	0\\
510	0\\
511	0\\
512	0\\
513	0\\
514	0\\
515	0\\
516	0\\
517	0\\
518	0\\
519	0\\
520	0\\
521	0\\
522	0\\
523	0\\
524	0\\
525	0\\
526	0\\
527	0\\
528	0\\
529	0\\
530	0\\
531	0\\
532	0\\
533	0\\
534	0\\
535	0\\
536	0\\
537	0\\
538	0\\
539	0\\
540	0\\
541	0\\
542	0\\
543	0\\
544	0\\
545	0\\
546	0\\
547	0\\
548	0\\
549	0\\
550	0\\
551	0\\
552	0\\
553	0\\
554	0\\
555	0\\
556	0\\
557	0\\
558	0\\
559	0\\
560	0\\
561	0\\
562	0\\
563	0\\
564	0\\
565	0\\
566	0\\
567	0\\
568	0\\
569	0\\
570	0\\
571	0\\
572	0\\
573	0\\
574	0\\
575	0\\
576	0\\
577	0\\
578	0\\
579	0\\
580	0\\
581	0\\
582	0\\
583	0\\
584	0\\
585	0.000134515837093855\\
586	0.000281838017575936\\
587	0.000435050286072654\\
588	0.000594542785312444\\
589	0.0007604561023499\\
590	0.000933270892818283\\
591	0.00111362236991578\\
592	0.00130217091110698\\
593	0.00149974464812233\\
594	0.00170715331224199\\
595	0.00192583625726673\\
596	0.00215816651678185\\
597	0.00340423165914315\\
598	0.00644286460810295\\
599	0\\
600	0\\
};
\addplot [color=mycolor15,solid,forget plot]
  table[row sep=crcr]{%
1	0\\
2	0\\
3	0\\
4	0\\
5	0\\
6	0\\
7	0\\
8	0\\
9	0\\
10	0\\
11	0\\
12	0\\
13	0\\
14	0\\
15	0\\
16	0\\
17	0\\
18	0\\
19	0\\
20	0\\
21	0\\
22	0\\
23	0\\
24	0\\
25	0\\
26	0\\
27	0\\
28	0\\
29	0\\
30	0\\
31	0\\
32	0\\
33	0\\
34	0\\
35	0\\
36	0\\
37	0\\
38	0\\
39	0\\
40	0\\
41	0\\
42	0\\
43	0\\
44	0\\
45	0\\
46	0\\
47	0\\
48	0\\
49	0\\
50	0\\
51	0\\
52	0\\
53	0\\
54	0\\
55	0\\
56	0\\
57	0\\
58	0\\
59	0\\
60	0\\
61	0\\
62	0\\
63	0\\
64	0\\
65	0\\
66	0\\
67	0\\
68	0\\
69	0\\
70	0\\
71	0\\
72	0\\
73	0\\
74	0\\
75	0\\
76	0\\
77	0\\
78	0\\
79	0\\
80	0\\
81	0\\
82	0\\
83	0\\
84	0\\
85	0\\
86	0\\
87	0\\
88	0\\
89	0\\
90	0\\
91	0\\
92	0\\
93	0\\
94	0\\
95	0\\
96	0\\
97	0\\
98	0\\
99	0\\
100	0\\
101	0\\
102	0\\
103	0\\
104	0\\
105	0\\
106	0\\
107	0\\
108	0\\
109	0\\
110	0\\
111	0\\
112	0\\
113	0\\
114	0\\
115	0\\
116	0\\
117	0\\
118	0\\
119	0\\
120	0\\
121	0\\
122	0\\
123	0\\
124	0\\
125	0\\
126	0\\
127	0\\
128	0\\
129	0\\
130	0\\
131	0\\
132	0\\
133	0\\
134	0\\
135	0\\
136	0\\
137	0\\
138	0\\
139	0\\
140	0\\
141	0\\
142	0\\
143	0\\
144	0\\
145	0\\
146	0\\
147	0\\
148	0\\
149	0\\
150	0\\
151	0\\
152	0\\
153	0\\
154	0\\
155	0\\
156	0\\
157	0\\
158	0\\
159	0\\
160	0\\
161	0\\
162	0\\
163	0\\
164	0\\
165	0\\
166	0\\
167	0\\
168	0\\
169	0\\
170	0\\
171	0\\
172	0\\
173	0\\
174	0\\
175	0\\
176	0\\
177	0\\
178	0\\
179	0\\
180	0\\
181	0\\
182	0\\
183	0\\
184	0\\
185	0\\
186	0\\
187	0\\
188	0\\
189	0\\
190	0\\
191	0\\
192	0\\
193	0\\
194	0\\
195	0\\
196	0\\
197	0\\
198	0\\
199	0\\
200	0\\
201	0\\
202	0\\
203	0\\
204	0\\
205	0\\
206	0\\
207	0\\
208	0\\
209	0\\
210	0\\
211	0\\
212	0\\
213	0\\
214	0\\
215	0\\
216	0\\
217	0\\
218	0\\
219	0\\
220	0\\
221	0\\
222	0\\
223	0\\
224	0\\
225	0\\
226	0\\
227	0\\
228	0\\
229	0\\
230	0\\
231	0\\
232	0\\
233	0\\
234	0\\
235	0\\
236	0\\
237	0\\
238	0\\
239	0\\
240	0\\
241	0\\
242	0\\
243	0\\
244	0\\
245	0\\
246	0\\
247	0\\
248	0\\
249	0\\
250	0\\
251	0\\
252	0\\
253	0\\
254	0\\
255	0\\
256	0\\
257	0\\
258	0\\
259	0\\
260	0\\
261	0\\
262	0\\
263	0\\
264	0\\
265	0\\
266	0\\
267	0\\
268	0\\
269	0\\
270	0\\
271	0\\
272	0\\
273	0\\
274	0\\
275	0\\
276	0\\
277	0\\
278	0\\
279	0\\
280	0\\
281	0\\
282	0\\
283	0\\
284	0\\
285	0\\
286	0\\
287	0\\
288	0\\
289	0\\
290	0\\
291	0\\
292	0\\
293	0\\
294	0\\
295	0\\
296	0\\
297	0\\
298	0\\
299	0\\
300	0\\
301	0\\
302	0\\
303	0\\
304	0\\
305	0\\
306	0\\
307	0\\
308	0\\
309	0\\
310	0\\
311	0\\
312	0\\
313	0\\
314	0\\
315	0\\
316	0\\
317	0\\
318	0\\
319	0\\
320	0\\
321	0\\
322	0\\
323	0\\
324	0\\
325	0\\
326	0\\
327	0\\
328	0\\
329	0\\
330	0\\
331	0\\
332	0\\
333	0\\
334	0\\
335	0\\
336	0\\
337	0\\
338	0\\
339	0\\
340	0\\
341	0\\
342	0\\
343	0\\
344	0\\
345	0\\
346	0\\
347	0\\
348	0\\
349	0\\
350	0\\
351	0\\
352	0\\
353	0\\
354	0\\
355	0\\
356	0\\
357	0\\
358	0\\
359	0\\
360	0\\
361	0\\
362	0\\
363	0\\
364	0\\
365	0\\
366	0\\
367	0\\
368	0\\
369	0\\
370	0\\
371	0\\
372	0\\
373	0\\
374	0\\
375	0\\
376	0\\
377	0\\
378	0\\
379	0\\
380	0\\
381	0\\
382	0\\
383	0\\
384	0\\
385	0\\
386	0\\
387	0\\
388	0\\
389	0\\
390	0\\
391	0\\
392	0\\
393	0\\
394	0\\
395	0\\
396	0\\
397	0\\
398	0\\
399	0\\
400	0\\
401	0\\
402	0\\
403	0\\
404	0\\
405	0\\
406	0\\
407	0\\
408	0\\
409	0\\
410	0\\
411	0\\
412	0\\
413	0\\
414	0\\
415	0\\
416	0\\
417	0\\
418	0\\
419	0\\
420	0\\
421	0\\
422	0\\
423	0\\
424	0\\
425	0\\
426	0\\
427	0\\
428	0\\
429	0\\
430	0\\
431	0\\
432	0\\
433	0\\
434	0\\
435	0\\
436	0\\
437	0\\
438	0\\
439	0\\
440	0\\
441	0\\
442	0\\
443	0\\
444	0\\
445	0\\
446	0\\
447	0\\
448	0\\
449	0\\
450	0\\
451	0\\
452	0\\
453	0\\
454	0\\
455	0\\
456	0\\
457	0\\
458	0\\
459	0\\
460	0\\
461	0\\
462	0\\
463	0\\
464	0\\
465	0\\
466	0\\
467	0\\
468	0\\
469	0\\
470	0\\
471	0\\
472	0\\
473	0\\
474	0\\
475	0\\
476	0\\
477	0\\
478	0\\
479	0\\
480	0\\
481	0\\
482	0\\
483	0\\
484	0\\
485	0\\
486	0\\
487	0\\
488	0\\
489	0\\
490	0\\
491	0\\
492	0\\
493	0\\
494	0\\
495	0\\
496	0\\
497	0\\
498	0\\
499	0\\
500	0\\
501	0\\
502	0\\
503	0\\
504	0\\
505	0\\
506	0\\
507	0\\
508	0\\
509	0\\
510	0\\
511	0\\
512	0\\
513	0\\
514	0\\
515	0\\
516	0\\
517	0\\
518	0\\
519	0\\
520	0\\
521	0\\
522	0\\
523	0\\
524	0\\
525	0\\
526	0\\
527	0\\
528	0\\
529	0\\
530	0\\
531	0\\
532	0\\
533	0\\
534	0\\
535	0\\
536	0\\
537	0\\
538	0\\
539	0\\
540	0\\
541	0\\
542	0\\
543	0\\
544	0\\
545	0\\
546	0\\
547	0\\
548	0\\
549	0\\
550	0\\
551	0\\
552	0\\
553	0\\
554	0\\
555	0\\
556	0\\
557	0\\
558	0\\
559	0\\
560	0\\
561	0\\
562	0\\
563	0\\
564	0\\
565	0\\
566	0\\
567	0\\
568	0\\
569	0\\
570	0\\
571	0\\
572	0\\
573	2.84550322669103e-05\\
574	0.000136105490885286\\
575	0.000246360904768022\\
576	0.00035920980371322\\
577	0.000474576836408107\\
578	0.000592251873280648\\
579	0.000712460866895543\\
580	0.000835098336304564\\
581	0.000931987463067577\\
582	0.00102435477630977\\
583	0.00111851701928916\\
584	0.00121497734033712\\
585	0.00131373271689993\\
586	0.00141477870147017\\
587	0.00151807891510166\\
588	0.00162355054108618\\
589	0.00173106951191172\\
590	0.00184045717402978\\
591	0.00195146461619112\\
592	0.00206375309347527\\
593	0.00217686668353044\\
594	0.00229019015194498\\
595	0.00264340599857734\\
596	0.00335667832726079\\
597	0.00469756192662931\\
598	0.00644286460810295\\
599	0\\
600	0\\
};
\addplot [color=mycolor16,solid,forget plot]
  table[row sep=crcr]{%
1	0\\
2	0\\
3	0\\
4	0\\
5	0\\
6	0\\
7	0\\
8	0\\
9	0\\
10	0\\
11	0\\
12	0\\
13	0\\
14	0\\
15	0\\
16	0\\
17	0\\
18	0\\
19	0\\
20	0\\
21	0\\
22	0\\
23	0\\
24	0\\
25	0\\
26	0\\
27	0\\
28	0\\
29	0\\
30	0\\
31	0\\
32	0\\
33	0\\
34	0\\
35	0\\
36	0\\
37	0\\
38	0\\
39	0\\
40	0\\
41	0\\
42	0\\
43	0\\
44	0\\
45	0\\
46	0\\
47	0\\
48	0\\
49	0\\
50	0\\
51	0\\
52	0\\
53	0\\
54	0\\
55	0\\
56	0\\
57	0\\
58	0\\
59	0\\
60	0\\
61	0\\
62	0\\
63	0\\
64	0\\
65	0\\
66	0\\
67	0\\
68	0\\
69	0\\
70	0\\
71	0\\
72	0\\
73	0\\
74	0\\
75	0\\
76	0\\
77	0\\
78	0\\
79	0\\
80	0\\
81	0\\
82	0\\
83	0\\
84	0\\
85	0\\
86	0\\
87	0\\
88	0\\
89	0\\
90	0\\
91	0\\
92	0\\
93	0\\
94	0\\
95	0\\
96	0\\
97	0\\
98	0\\
99	0\\
100	0\\
101	0\\
102	0\\
103	0\\
104	0\\
105	0\\
106	0\\
107	0\\
108	0\\
109	0\\
110	0\\
111	0\\
112	0\\
113	0\\
114	0\\
115	0\\
116	0\\
117	0\\
118	0\\
119	0\\
120	0\\
121	0\\
122	0\\
123	0\\
124	0\\
125	0\\
126	0\\
127	0\\
128	0\\
129	0\\
130	0\\
131	0\\
132	0\\
133	0\\
134	0\\
135	0\\
136	0\\
137	0\\
138	0\\
139	0\\
140	0\\
141	0\\
142	0\\
143	0\\
144	0\\
145	0\\
146	0\\
147	0\\
148	0\\
149	0\\
150	0\\
151	0\\
152	0\\
153	0\\
154	0\\
155	0\\
156	0\\
157	0\\
158	0\\
159	0\\
160	0\\
161	0\\
162	0\\
163	0\\
164	0\\
165	0\\
166	0\\
167	0\\
168	0\\
169	0\\
170	0\\
171	0\\
172	0\\
173	0\\
174	0\\
175	0\\
176	0\\
177	0\\
178	0\\
179	0\\
180	0\\
181	0\\
182	0\\
183	0\\
184	0\\
185	0\\
186	0\\
187	0\\
188	0\\
189	0\\
190	0\\
191	0\\
192	0\\
193	0\\
194	0\\
195	0\\
196	0\\
197	0\\
198	0\\
199	0\\
200	0\\
201	0\\
202	0\\
203	0\\
204	0\\
205	0\\
206	0\\
207	0\\
208	0\\
209	0\\
210	0\\
211	0\\
212	0\\
213	0\\
214	0\\
215	0\\
216	0\\
217	0\\
218	0\\
219	0\\
220	0\\
221	0\\
222	0\\
223	0\\
224	0\\
225	0\\
226	0\\
227	0\\
228	0\\
229	0\\
230	0\\
231	0\\
232	0\\
233	0\\
234	0\\
235	0\\
236	0\\
237	0\\
238	0\\
239	0\\
240	0\\
241	0\\
242	0\\
243	0\\
244	0\\
245	0\\
246	0\\
247	0\\
248	0\\
249	0\\
250	0\\
251	0\\
252	0\\
253	0\\
254	0\\
255	0\\
256	0\\
257	0\\
258	0\\
259	0\\
260	0\\
261	0\\
262	0\\
263	0\\
264	0\\
265	0\\
266	0\\
267	0\\
268	0\\
269	0\\
270	0\\
271	0\\
272	0\\
273	0\\
274	0\\
275	0\\
276	0\\
277	0\\
278	0\\
279	0\\
280	0\\
281	0\\
282	0\\
283	0\\
284	0\\
285	0\\
286	0\\
287	0\\
288	0\\
289	0\\
290	0\\
291	0\\
292	0\\
293	0\\
294	0\\
295	0\\
296	0\\
297	0\\
298	0\\
299	0\\
300	0\\
301	0\\
302	0\\
303	0\\
304	0\\
305	0\\
306	0\\
307	0\\
308	0\\
309	0\\
310	0\\
311	0\\
312	0\\
313	0\\
314	0\\
315	0\\
316	0\\
317	0\\
318	0\\
319	0\\
320	0\\
321	0\\
322	0\\
323	0\\
324	0\\
325	0\\
326	0\\
327	0\\
328	0\\
329	0\\
330	0\\
331	0\\
332	0\\
333	0\\
334	0\\
335	0\\
336	0\\
337	0\\
338	0\\
339	0\\
340	0\\
341	0\\
342	0\\
343	0\\
344	0\\
345	0\\
346	0\\
347	0\\
348	0\\
349	0\\
350	0\\
351	0\\
352	0\\
353	0\\
354	0\\
355	0\\
356	0\\
357	0\\
358	0\\
359	0\\
360	0\\
361	0\\
362	0\\
363	0\\
364	0\\
365	0\\
366	0\\
367	0\\
368	0\\
369	0\\
370	0\\
371	0\\
372	0\\
373	0\\
374	0\\
375	0\\
376	0\\
377	0\\
378	0\\
379	0\\
380	0\\
381	0\\
382	0\\
383	0\\
384	0\\
385	0\\
386	0\\
387	0\\
388	0\\
389	0\\
390	0\\
391	0\\
392	0\\
393	0\\
394	0\\
395	0\\
396	0\\
397	0\\
398	0\\
399	0\\
400	0\\
401	0\\
402	0\\
403	0\\
404	0\\
405	0\\
406	0\\
407	0\\
408	0\\
409	0\\
410	0\\
411	0\\
412	0\\
413	0\\
414	0\\
415	0\\
416	0\\
417	0\\
418	0\\
419	0\\
420	0\\
421	0\\
422	0\\
423	0\\
424	0\\
425	0\\
426	0\\
427	0\\
428	0\\
429	0\\
430	0\\
431	0\\
432	0\\
433	0\\
434	0\\
435	0\\
436	0\\
437	0\\
438	0\\
439	0\\
440	0\\
441	0\\
442	0\\
443	0\\
444	0\\
445	0\\
446	0\\
447	0\\
448	0\\
449	0\\
450	0\\
451	0\\
452	0\\
453	0\\
454	0\\
455	0\\
456	0\\
457	0\\
458	0\\
459	0\\
460	0\\
461	0\\
462	0\\
463	0\\
464	0\\
465	0\\
466	0\\
467	0\\
468	0\\
469	0\\
470	0\\
471	0\\
472	0\\
473	0\\
474	0\\
475	0\\
476	0\\
477	0\\
478	0\\
479	0\\
480	0\\
481	0\\
482	0\\
483	0\\
484	0\\
485	0\\
486	0\\
487	0\\
488	0\\
489	0\\
490	0\\
491	0\\
492	0\\
493	0\\
494	0\\
495	0\\
496	0\\
497	0\\
498	0\\
499	0\\
500	0\\
501	0\\
502	0\\
503	0\\
504	0\\
505	0\\
506	0\\
507	0\\
508	0\\
509	0\\
510	0\\
511	0\\
512	0\\
513	0\\
514	0\\
515	0\\
516	0\\
517	0\\
518	0\\
519	0\\
520	0\\
521	0\\
522	0\\
523	0\\
524	0\\
525	0\\
526	0\\
527	0\\
528	0\\
529	0\\
530	0\\
531	0\\
532	0\\
533	0\\
534	0\\
535	0\\
536	0\\
537	0\\
538	0\\
539	0\\
540	0\\
541	0\\
542	0\\
543	0\\
544	0\\
545	0\\
546	0\\
547	0\\
548	0\\
549	0\\
550	0\\
551	0\\
552	0\\
553	0\\
554	0\\
555	0\\
556	0\\
557	0\\
558	0\\
559	0\\
560	0\\
561	0\\
562	0\\
563	0\\
564	6.11241591688388e-05\\
565	0.000153833609333673\\
566	0.000247621371611974\\
567	0.000342275882838338\\
568	0.000426454386023444\\
569	0.000490146502349677\\
570	0.000554714051789001\\
571	0.000620092626673308\\
572	0.000686241977351121\\
573	0.000753338581932168\\
574	0.000821311602791528\\
575	0.000890073230408475\\
576	0.000959515814844145\\
577	0.0010295081722191\\
578	0.0010998907600857\\
579	0.00117046937600367\\
580	0.00124101035188772\\
581	0.00131279084750798\\
582	0.00138613243267675\\
583	0.00146105292021269\\
584	0.00153756685550637\\
585	0.00161568839009441\\
586	0.00169543074241313\\
587	0.00177681010814605\\
588	0.00185987394737956\\
589	0.00194472985818468\\
590	0.00203176867243585\\
591	0.00218539535278842\\
592	0.00246148836585481\\
593	0.00274934460856801\\
594	0.00309920221510837\\
595	0.0035236770965596\\
596	0.00425769487060698\\
597	0.00503983077166121\\
598	0.00644286460810295\\
599	0\\
600	0\\
};
\addplot [color=mycolor17,solid,forget plot]
  table[row sep=crcr]{%
1	0\\
2	0\\
3	0\\
4	0\\
5	0\\
6	0\\
7	0\\
8	0\\
9	0\\
10	0\\
11	0\\
12	0\\
13	0\\
14	0\\
15	0\\
16	0\\
17	0\\
18	0\\
19	0\\
20	0\\
21	0\\
22	0\\
23	0\\
24	0\\
25	0\\
26	0\\
27	0\\
28	0\\
29	0\\
30	0\\
31	0\\
32	0\\
33	0\\
34	0\\
35	0\\
36	0\\
37	0\\
38	0\\
39	0\\
40	0\\
41	0\\
42	0\\
43	0\\
44	0\\
45	0\\
46	0\\
47	0\\
48	0\\
49	0\\
50	0\\
51	0\\
52	0\\
53	0\\
54	0\\
55	0\\
56	0\\
57	0\\
58	0\\
59	0\\
60	0\\
61	0\\
62	0\\
63	0\\
64	0\\
65	0\\
66	0\\
67	0\\
68	0\\
69	0\\
70	0\\
71	0\\
72	0\\
73	0\\
74	0\\
75	0\\
76	0\\
77	0\\
78	0\\
79	0\\
80	0\\
81	0\\
82	0\\
83	0\\
84	0\\
85	0\\
86	0\\
87	0\\
88	0\\
89	0\\
90	0\\
91	0\\
92	0\\
93	0\\
94	0\\
95	0\\
96	0\\
97	0\\
98	0\\
99	0\\
100	0\\
101	0\\
102	0\\
103	0\\
104	0\\
105	0\\
106	0\\
107	0\\
108	0\\
109	0\\
110	0\\
111	0\\
112	0\\
113	0\\
114	0\\
115	0\\
116	0\\
117	0\\
118	0\\
119	0\\
120	0\\
121	0\\
122	0\\
123	0\\
124	0\\
125	0\\
126	0\\
127	0\\
128	0\\
129	0\\
130	0\\
131	0\\
132	0\\
133	0\\
134	0\\
135	0\\
136	0\\
137	0\\
138	0\\
139	0\\
140	0\\
141	0\\
142	0\\
143	0\\
144	0\\
145	0\\
146	0\\
147	0\\
148	0\\
149	0\\
150	0\\
151	0\\
152	0\\
153	0\\
154	0\\
155	0\\
156	0\\
157	0\\
158	0\\
159	0\\
160	0\\
161	0\\
162	0\\
163	0\\
164	0\\
165	0\\
166	0\\
167	0\\
168	0\\
169	0\\
170	0\\
171	0\\
172	0\\
173	0\\
174	0\\
175	0\\
176	0\\
177	0\\
178	0\\
179	0\\
180	0\\
181	0\\
182	0\\
183	0\\
184	0\\
185	0\\
186	0\\
187	0\\
188	0\\
189	0\\
190	0\\
191	0\\
192	0\\
193	0\\
194	0\\
195	0\\
196	0\\
197	0\\
198	0\\
199	0\\
200	0\\
201	0\\
202	0\\
203	0\\
204	0\\
205	0\\
206	0\\
207	0\\
208	0\\
209	0\\
210	0\\
211	0\\
212	0\\
213	0\\
214	0\\
215	0\\
216	0\\
217	0\\
218	0\\
219	0\\
220	0\\
221	0\\
222	0\\
223	0\\
224	0\\
225	0\\
226	0\\
227	0\\
228	0\\
229	0\\
230	0\\
231	0\\
232	0\\
233	0\\
234	0\\
235	0\\
236	0\\
237	0\\
238	0\\
239	0\\
240	0\\
241	0\\
242	0\\
243	0\\
244	0\\
245	0\\
246	0\\
247	0\\
248	0\\
249	0\\
250	0\\
251	0\\
252	0\\
253	0\\
254	0\\
255	0\\
256	0\\
257	0\\
258	0\\
259	0\\
260	0\\
261	0\\
262	0\\
263	0\\
264	0\\
265	0\\
266	0\\
267	0\\
268	0\\
269	0\\
270	0\\
271	0\\
272	0\\
273	0\\
274	0\\
275	0\\
276	0\\
277	0\\
278	0\\
279	0\\
280	0\\
281	0\\
282	0\\
283	0\\
284	0\\
285	0\\
286	0\\
287	0\\
288	0\\
289	0\\
290	0\\
291	0\\
292	0\\
293	0\\
294	0\\
295	0\\
296	0\\
297	0\\
298	0\\
299	0\\
300	0\\
301	0\\
302	0\\
303	0\\
304	0\\
305	0\\
306	0\\
307	0\\
308	0\\
309	0\\
310	0\\
311	0\\
312	0\\
313	0\\
314	0\\
315	0\\
316	0\\
317	0\\
318	0\\
319	0\\
320	0\\
321	0\\
322	0\\
323	0\\
324	0\\
325	0\\
326	0\\
327	0\\
328	0\\
329	0\\
330	0\\
331	0\\
332	0\\
333	0\\
334	0\\
335	0\\
336	0\\
337	0\\
338	0\\
339	0\\
340	0\\
341	0\\
342	0\\
343	0\\
344	0\\
345	0\\
346	0\\
347	0\\
348	0\\
349	0\\
350	0\\
351	0\\
352	0\\
353	0\\
354	0\\
355	0\\
356	0\\
357	0\\
358	0\\
359	0\\
360	0\\
361	0\\
362	0\\
363	0\\
364	0\\
365	0\\
366	0\\
367	0\\
368	0\\
369	0\\
370	0\\
371	0\\
372	0\\
373	0\\
374	0\\
375	0\\
376	0\\
377	0\\
378	0\\
379	0\\
380	0\\
381	0\\
382	0\\
383	0\\
384	0\\
385	0\\
386	0\\
387	0\\
388	0\\
389	0\\
390	0\\
391	0\\
392	0\\
393	0\\
394	0\\
395	0\\
396	0\\
397	0\\
398	0\\
399	0\\
400	0\\
401	0\\
402	0\\
403	0\\
404	0\\
405	0\\
406	0\\
407	0\\
408	0\\
409	0\\
410	0\\
411	0\\
412	0\\
413	0\\
414	0\\
415	0\\
416	0\\
417	0\\
418	0\\
419	0\\
420	0\\
421	0\\
422	0\\
423	0\\
424	0\\
425	0\\
426	0\\
427	0\\
428	0\\
429	0\\
430	0\\
431	0\\
432	0\\
433	0\\
434	0\\
435	0\\
436	0\\
437	0\\
438	0\\
439	0\\
440	0\\
441	0\\
442	0\\
443	0\\
444	0\\
445	0\\
446	0\\
447	0\\
448	0\\
449	0\\
450	0\\
451	0\\
452	0\\
453	0\\
454	0\\
455	0\\
456	0\\
457	0\\
458	0\\
459	0\\
460	0\\
461	0\\
462	0\\
463	0\\
464	0\\
465	0\\
466	0\\
467	0\\
468	0\\
469	0\\
470	0\\
471	0\\
472	0\\
473	0\\
474	0\\
475	0\\
476	0\\
477	0\\
478	0\\
479	0\\
480	0\\
481	0\\
482	0\\
483	0\\
484	0\\
485	0\\
486	0\\
487	0\\
488	0\\
489	0\\
490	0\\
491	0\\
492	0\\
493	0\\
494	0\\
495	0\\
496	0\\
497	0\\
498	0\\
499	0\\
500	0\\
501	0\\
502	0\\
503	0\\
504	0\\
505	0\\
506	0\\
507	0\\
508	0\\
509	0\\
510	0\\
511	0\\
512	0\\
513	0\\
514	0\\
515	0\\
516	0\\
517	0\\
518	0\\
519	0\\
520	0\\
521	0\\
522	0\\
523	0\\
524	0\\
525	0\\
526	0\\
527	0\\
528	0\\
529	0\\
530	0\\
531	0\\
532	0\\
533	0\\
534	0\\
535	0\\
536	0\\
537	0\\
538	0\\
539	0\\
540	0\\
541	0\\
542	0\\
543	0\\
544	0\\
545	0\\
546	0\\
547	0\\
548	0\\
549	0\\
550	0\\
551	0\\
552	0\\
553	0\\
554	0\\
555	2.26142481073285e-05\\
556	9.40356632703977e-05\\
557	0.000142834796507731\\
558	0.000192136179755935\\
559	0.000241890470607289\\
560	0.000292049373855764\\
561	0.000342545162924321\\
562	0.00039329687625306\\
563	0.000444211080034401\\
564	0.000495180373312776\\
565	0.000546081753293092\\
566	0.000596774292680829\\
567	0.000647098131380762\\
568	0.00069742347279142\\
569	0.000748696607560934\\
570	0.000800939380062976\\
571	0.000854177294682413\\
572	0.000908439254781951\\
573	0.000963750984062382\\
574	0.00102014366327751\\
575	0.00107765573610446\\
576	0.00113633523369278\\
577	0.00119624324802379\\
578	0.00125745746076917\\
579	0.00132007724518431\\
580	0.00138423112120714\\
581	0.0014499990846374\\
582	0.00151746183138254\\
583	0.001586633971824\\
584	0.00165775679116301\\
585	0.00173067211608291\\
586	0.00180616539338125\\
587	0.0018831539132818\\
588	0.00209862842761386\\
589	0.00237026081777734\\
590	0.00265207182564143\\
591	0.00293299955993958\\
592	0.0031940225228887\\
593	0.00348473277442626\\
594	0.00383471000680442\\
595	0.00412245907862188\\
596	0.00444436306303141\\
597	0.00503983077166121\\
598	0.00644286460810295\\
599	0\\
600	0\\
};
\addplot [color=mycolor18,solid,forget plot]
  table[row sep=crcr]{%
1	0\\
2	0\\
3	0\\
4	0\\
5	0\\
6	0\\
7	0\\
8	0\\
9	0\\
10	0\\
11	0\\
12	0\\
13	0\\
14	0\\
15	0\\
16	0\\
17	0\\
18	0\\
19	0\\
20	0\\
21	0\\
22	0\\
23	0\\
24	0\\
25	0\\
26	0\\
27	0\\
28	0\\
29	0\\
30	0\\
31	0\\
32	0\\
33	0\\
34	0\\
35	0\\
36	0\\
37	0\\
38	0\\
39	0\\
40	0\\
41	0\\
42	0\\
43	0\\
44	0\\
45	0\\
46	0\\
47	0\\
48	0\\
49	0\\
50	0\\
51	0\\
52	0\\
53	0\\
54	0\\
55	0\\
56	0\\
57	0\\
58	0\\
59	0\\
60	0\\
61	0\\
62	0\\
63	0\\
64	0\\
65	0\\
66	0\\
67	0\\
68	0\\
69	0\\
70	0\\
71	0\\
72	0\\
73	0\\
74	0\\
75	0\\
76	0\\
77	0\\
78	0\\
79	0\\
80	0\\
81	0\\
82	0\\
83	0\\
84	0\\
85	0\\
86	0\\
87	0\\
88	0\\
89	0\\
90	0\\
91	0\\
92	0\\
93	0\\
94	0\\
95	0\\
96	0\\
97	0\\
98	0\\
99	0\\
100	0\\
101	0\\
102	0\\
103	0\\
104	0\\
105	0\\
106	0\\
107	0\\
108	0\\
109	0\\
110	0\\
111	0\\
112	0\\
113	0\\
114	0\\
115	0\\
116	0\\
117	0\\
118	0\\
119	0\\
120	0\\
121	0\\
122	0\\
123	0\\
124	0\\
125	0\\
126	0\\
127	0\\
128	0\\
129	0\\
130	0\\
131	0\\
132	0\\
133	0\\
134	0\\
135	0\\
136	0\\
137	0\\
138	0\\
139	0\\
140	0\\
141	0\\
142	0\\
143	0\\
144	0\\
145	0\\
146	0\\
147	0\\
148	0\\
149	0\\
150	0\\
151	0\\
152	0\\
153	0\\
154	0\\
155	0\\
156	0\\
157	0\\
158	0\\
159	0\\
160	0\\
161	0\\
162	0\\
163	0\\
164	0\\
165	0\\
166	0\\
167	0\\
168	0\\
169	0\\
170	0\\
171	0\\
172	0\\
173	0\\
174	0\\
175	0\\
176	0\\
177	0\\
178	0\\
179	0\\
180	0\\
181	0\\
182	0\\
183	0\\
184	0\\
185	0\\
186	0\\
187	0\\
188	0\\
189	0\\
190	0\\
191	0\\
192	0\\
193	0\\
194	0\\
195	0\\
196	0\\
197	0\\
198	0\\
199	0\\
200	0\\
201	0\\
202	0\\
203	0\\
204	0\\
205	0\\
206	0\\
207	0\\
208	0\\
209	0\\
210	0\\
211	0\\
212	0\\
213	0\\
214	0\\
215	0\\
216	0\\
217	0\\
218	0\\
219	0\\
220	0\\
221	0\\
222	0\\
223	0\\
224	0\\
225	0\\
226	0\\
227	0\\
228	0\\
229	0\\
230	0\\
231	0\\
232	0\\
233	0\\
234	0\\
235	0\\
236	0\\
237	0\\
238	0\\
239	0\\
240	0\\
241	0\\
242	0\\
243	0\\
244	0\\
245	0\\
246	0\\
247	0\\
248	0\\
249	0\\
250	0\\
251	0\\
252	0\\
253	0\\
254	0\\
255	0\\
256	0\\
257	0\\
258	0\\
259	0\\
260	0\\
261	0\\
262	0\\
263	0\\
264	0\\
265	0\\
266	0\\
267	0\\
268	0\\
269	0\\
270	0\\
271	0\\
272	0\\
273	0\\
274	0\\
275	0\\
276	0\\
277	0\\
278	0\\
279	0\\
280	0\\
281	0\\
282	0\\
283	0\\
284	0\\
285	0\\
286	0\\
287	0\\
288	0\\
289	0\\
290	0\\
291	0\\
292	0\\
293	0\\
294	0\\
295	0\\
296	0\\
297	0\\
298	0\\
299	0\\
300	0\\
301	0\\
302	0\\
303	0\\
304	0\\
305	0\\
306	0\\
307	0\\
308	0\\
309	0\\
310	0\\
311	0\\
312	0\\
313	0\\
314	0\\
315	0\\
316	0\\
317	0\\
318	0\\
319	0\\
320	0\\
321	0\\
322	0\\
323	0\\
324	0\\
325	0\\
326	0\\
327	0\\
328	0\\
329	0\\
330	0\\
331	0\\
332	0\\
333	0\\
334	0\\
335	0\\
336	0\\
337	0\\
338	0\\
339	0\\
340	0\\
341	0\\
342	0\\
343	0\\
344	0\\
345	0\\
346	0\\
347	0\\
348	0\\
349	0\\
350	0\\
351	0\\
352	0\\
353	0\\
354	0\\
355	0\\
356	0\\
357	0\\
358	0\\
359	0\\
360	0\\
361	0\\
362	0\\
363	0\\
364	0\\
365	0\\
366	0\\
367	0\\
368	0\\
369	0\\
370	0\\
371	0\\
372	0\\
373	0\\
374	0\\
375	0\\
376	0\\
377	0\\
378	0\\
379	0\\
380	0\\
381	0\\
382	0\\
383	0\\
384	0\\
385	0\\
386	0\\
387	0\\
388	0\\
389	0\\
390	0\\
391	0\\
392	0\\
393	0\\
394	0\\
395	0\\
396	0\\
397	0\\
398	0\\
399	0\\
400	0\\
401	0\\
402	0\\
403	0\\
404	0\\
405	0\\
406	0\\
407	0\\
408	0\\
409	0\\
410	0\\
411	0\\
412	0\\
413	0\\
414	0\\
415	0\\
416	0\\
417	0\\
418	0\\
419	0\\
420	0\\
421	0\\
422	0\\
423	0\\
424	0\\
425	0\\
426	0\\
427	0\\
428	0\\
429	0\\
430	0\\
431	0\\
432	0\\
433	0\\
434	0\\
435	0\\
436	0\\
437	0\\
438	0\\
439	0\\
440	0\\
441	0\\
442	0\\
443	0\\
444	0\\
445	0\\
446	0\\
447	0\\
448	0\\
449	0\\
450	0\\
451	0\\
452	0\\
453	0\\
454	0\\
455	0\\
456	0\\
457	0\\
458	0\\
459	0\\
460	0\\
461	0\\
462	0\\
463	0\\
464	0\\
465	0\\
466	0\\
467	0\\
468	0\\
469	0\\
470	0\\
471	0\\
472	0\\
473	0\\
474	0\\
475	0\\
476	0\\
477	0\\
478	0\\
479	0\\
480	0\\
481	0\\
482	0\\
483	0\\
484	0\\
485	0\\
486	0\\
487	0\\
488	0\\
489	0\\
490	0\\
491	0\\
492	0\\
493	0\\
494	0\\
495	0\\
496	0\\
497	0\\
498	0\\
499	0\\
500	0\\
501	0\\
502	0\\
503	0\\
504	0\\
505	0\\
506	0\\
507	0\\
508	0\\
509	0\\
510	0\\
511	0\\
512	0\\
513	0\\
514	0\\
515	0\\
516	0\\
517	0\\
518	0\\
519	0\\
520	0\\
521	0\\
522	0\\
523	0\\
524	0\\
525	0\\
526	0\\
527	0\\
528	0\\
529	0\\
530	0\\
531	0\\
532	0\\
533	0\\
534	0\\
535	0\\
536	0\\
537	0\\
538	0\\
539	0\\
540	0\\
541	0\\
542	0\\
543	0\\
544	0\\
545	0\\
546	0\\
547	0\\
548	0\\
549	2.32520210821688e-05\\
550	6.29822389686342e-05\\
551	0.000102594284985584\\
552	0.000142000070962094\\
553	0.000181109221367284\\
554	0.000219804609413708\\
555	0.000257961564445722\\
556	0.000295780758493718\\
557	0.000334217678983119\\
558	0.000373283831162645\\
559	0.000412993211787373\\
560	0.000453363060135725\\
561	0.000494414867686697\\
562	0.000536175585096235\\
563	0.000578679087463031\\
564	0.000621967988206539\\
565	0.000666095878568849\\
566	0.000711130094990775\\
567	0.000757155082147057\\
568	0.000804247376185817\\
569	0.000852443417848277\\
570	0.00090178212884265\\
571	0.000952305207691893\\
572	0.00100405760152108\\
573	0.00105708760588472\\
574	0.00111144706147627\\
575	0.0011671914887648\\
576	0.00122442232407532\\
577	0.00128313096863521\\
578	0.0013433810070614\\
579	0.00140538131268943\\
580	0.00146902614413746\\
581	0.00153430958574611\\
582	0.00160209512119075\\
583	0.00167132716120464\\
584	0.0017419265568923\\
585	0.0019582562819987\\
586	0.00223293796086997\\
587	0.00252513250926167\\
588	0.00278502806553789\\
589	0.00303946052792341\\
590	0.0033058313870396\\
591	0.00353307097397887\\
592	0.00367241590802784\\
593	0.00382251134333667\\
594	0.00397190834964223\\
595	0.00415280575412962\\
596	0.00444436306303141\\
597	0.00503983077166121\\
598	0.00644286460810295\\
599	0\\
600	0\\
};
\addplot [color=red!25!mycolor17,solid,forget plot]
  table[row sep=crcr]{%
1	0\\
2	0\\
3	0\\
4	0\\
5	0\\
6	0\\
7	0\\
8	0\\
9	0\\
10	0\\
11	0\\
12	0\\
13	0\\
14	0\\
15	0\\
16	0\\
17	0\\
18	0\\
19	0\\
20	0\\
21	0\\
22	0\\
23	0\\
24	0\\
25	0\\
26	0\\
27	0\\
28	0\\
29	0\\
30	0\\
31	0\\
32	0\\
33	0\\
34	0\\
35	0\\
36	0\\
37	0\\
38	0\\
39	0\\
40	0\\
41	0\\
42	0\\
43	0\\
44	0\\
45	0\\
46	0\\
47	0\\
48	0\\
49	0\\
50	0\\
51	0\\
52	0\\
53	0\\
54	0\\
55	0\\
56	0\\
57	0\\
58	0\\
59	0\\
60	0\\
61	0\\
62	0\\
63	0\\
64	0\\
65	0\\
66	0\\
67	0\\
68	0\\
69	0\\
70	0\\
71	0\\
72	0\\
73	0\\
74	0\\
75	0\\
76	0\\
77	0\\
78	0\\
79	0\\
80	0\\
81	0\\
82	0\\
83	0\\
84	0\\
85	0\\
86	0\\
87	0\\
88	0\\
89	0\\
90	0\\
91	0\\
92	0\\
93	0\\
94	0\\
95	0\\
96	0\\
97	0\\
98	0\\
99	0\\
100	0\\
101	0\\
102	0\\
103	0\\
104	0\\
105	0\\
106	0\\
107	0\\
108	0\\
109	0\\
110	0\\
111	0\\
112	0\\
113	0\\
114	0\\
115	0\\
116	0\\
117	0\\
118	0\\
119	0\\
120	0\\
121	0\\
122	0\\
123	0\\
124	0\\
125	0\\
126	0\\
127	0\\
128	0\\
129	0\\
130	0\\
131	0\\
132	0\\
133	0\\
134	0\\
135	0\\
136	0\\
137	0\\
138	0\\
139	0\\
140	0\\
141	0\\
142	0\\
143	0\\
144	0\\
145	0\\
146	0\\
147	0\\
148	0\\
149	0\\
150	0\\
151	0\\
152	0\\
153	0\\
154	0\\
155	0\\
156	0\\
157	0\\
158	0\\
159	0\\
160	0\\
161	0\\
162	0\\
163	0\\
164	0\\
165	0\\
166	0\\
167	0\\
168	0\\
169	0\\
170	0\\
171	0\\
172	0\\
173	0\\
174	0\\
175	0\\
176	0\\
177	0\\
178	0\\
179	0\\
180	0\\
181	0\\
182	0\\
183	0\\
184	0\\
185	0\\
186	0\\
187	0\\
188	0\\
189	0\\
190	0\\
191	0\\
192	0\\
193	0\\
194	0\\
195	0\\
196	0\\
197	0\\
198	0\\
199	0\\
200	0\\
201	0\\
202	0\\
203	0\\
204	0\\
205	0\\
206	0\\
207	0\\
208	0\\
209	0\\
210	0\\
211	0\\
212	0\\
213	0\\
214	0\\
215	0\\
216	0\\
217	0\\
218	0\\
219	0\\
220	0\\
221	0\\
222	0\\
223	0\\
224	0\\
225	0\\
226	0\\
227	0\\
228	0\\
229	0\\
230	0\\
231	0\\
232	0\\
233	0\\
234	0\\
235	0\\
236	0\\
237	0\\
238	0\\
239	0\\
240	0\\
241	0\\
242	0\\
243	0\\
244	0\\
245	0\\
246	0\\
247	0\\
248	0\\
249	0\\
250	0\\
251	0\\
252	0\\
253	0\\
254	0\\
255	0\\
256	0\\
257	0\\
258	0\\
259	0\\
260	0\\
261	0\\
262	0\\
263	0\\
264	0\\
265	0\\
266	0\\
267	0\\
268	0\\
269	0\\
270	0\\
271	0\\
272	0\\
273	0\\
274	0\\
275	0\\
276	0\\
277	0\\
278	0\\
279	0\\
280	0\\
281	0\\
282	0\\
283	0\\
284	0\\
285	0\\
286	0\\
287	0\\
288	0\\
289	0\\
290	0\\
291	0\\
292	0\\
293	0\\
294	0\\
295	0\\
296	0\\
297	0\\
298	0\\
299	0\\
300	0\\
301	0\\
302	0\\
303	0\\
304	0\\
305	0\\
306	0\\
307	0\\
308	0\\
309	0\\
310	0\\
311	0\\
312	0\\
313	0\\
314	0\\
315	0\\
316	0\\
317	0\\
318	0\\
319	0\\
320	0\\
321	0\\
322	0\\
323	0\\
324	0\\
325	0\\
326	0\\
327	0\\
328	0\\
329	0\\
330	0\\
331	0\\
332	0\\
333	0\\
334	0\\
335	0\\
336	0\\
337	0\\
338	0\\
339	0\\
340	0\\
341	0\\
342	0\\
343	0\\
344	0\\
345	0\\
346	0\\
347	0\\
348	0\\
349	0\\
350	0\\
351	0\\
352	0\\
353	0\\
354	0\\
355	0\\
356	0\\
357	0\\
358	0\\
359	0\\
360	0\\
361	0\\
362	0\\
363	0\\
364	0\\
365	0\\
366	0\\
367	0\\
368	0\\
369	0\\
370	0\\
371	0\\
372	0\\
373	0\\
374	0\\
375	0\\
376	0\\
377	0\\
378	0\\
379	0\\
380	0\\
381	0\\
382	0\\
383	0\\
384	0\\
385	0\\
386	0\\
387	0\\
388	0\\
389	0\\
390	0\\
391	0\\
392	0\\
393	0\\
394	0\\
395	0\\
396	0\\
397	0\\
398	0\\
399	0\\
400	0\\
401	0\\
402	0\\
403	0\\
404	0\\
405	0\\
406	0\\
407	0\\
408	0\\
409	0\\
410	0\\
411	0\\
412	0\\
413	0\\
414	0\\
415	0\\
416	0\\
417	0\\
418	0\\
419	0\\
420	0\\
421	0\\
422	0\\
423	0\\
424	0\\
425	0\\
426	0\\
427	0\\
428	0\\
429	0\\
430	0\\
431	0\\
432	0\\
433	0\\
434	0\\
435	0\\
436	0\\
437	0\\
438	0\\
439	0\\
440	0\\
441	0\\
442	0\\
443	0\\
444	0\\
445	0\\
446	0\\
447	0\\
448	0\\
449	0\\
450	0\\
451	0\\
452	0\\
453	0\\
454	0\\
455	0\\
456	0\\
457	0\\
458	0\\
459	0\\
460	0\\
461	0\\
462	0\\
463	0\\
464	0\\
465	0\\
466	0\\
467	0\\
468	0\\
469	0\\
470	0\\
471	0\\
472	0\\
473	0\\
474	0\\
475	0\\
476	0\\
477	0\\
478	0\\
479	0\\
480	0\\
481	0\\
482	0\\
483	0\\
484	0\\
485	0\\
486	0\\
487	0\\
488	0\\
489	0\\
490	0\\
491	0\\
492	0\\
493	0\\
494	0\\
495	0\\
496	0\\
497	0\\
498	0\\
499	0\\
500	0\\
501	0\\
502	0\\
503	0\\
504	0\\
505	0\\
506	0\\
507	0\\
508	0\\
509	0\\
510	0\\
511	0\\
512	0\\
513	0\\
514	0\\
515	0\\
516	0\\
517	0\\
518	0\\
519	0\\
520	0\\
521	0\\
522	0\\
523	0\\
524	0\\
525	0\\
526	0\\
527	0\\
528	0\\
529	0\\
530	0\\
531	0\\
532	0\\
533	0\\
534	0\\
535	0\\
536	0\\
537	0\\
538	0\\
539	0\\
540	0\\
541	0\\
542	0\\
543	0\\
544	0\\
545	1.60914883874299e-05\\
546	4.61802426780253e-05\\
547	7.66629135509208e-05\\
548	0.000107555208654854\\
549	0.000138876541337362\\
550	0.000170650593952384\\
551	0.000202914517604868\\
552	0.000235708465118579\\
553	0.000269081896836238\\
554	0.000303087169673518\\
555	0.000337789238210994\\
556	0.000373251698431509\\
557	0.000409499257060565\\
558	0.000446558227820018\\
559	0.000484456617435164\\
560	0.000523224188690424\\
561	0.000562892484763894\\
562	0.000603494791845105\\
563	0.000645066023735202\\
564	0.000687642494400291\\
565	0.000731261531316776\\
566	0.000775960935897401\\
567	0.000821778458148387\\
568	0.000868751612779525\\
569	0.000916919738931042\\
570	0.000966334419799042\\
571	0.00101705468567766\\
572	0.00106906417782553\\
573	0.00112241341855564\\
574	0.00117715592173479\\
575	0.00123354169018176\\
576	0.00129128803526078\\
577	0.00135048220327041\\
578	0.00141156374788578\\
579	0.00147451749231703\\
580	0.00153861733116412\\
581	0.00160404882434682\\
582	0.0017589725145217\\
583	0.00203755571568777\\
584	0.00233345386073883\\
585	0.00259266855288805\\
586	0.00284513355551299\\
587	0.00310713312521459\\
588	0.0032721022788783\\
589	0.00339546551785458\\
590	0.00351237211383029\\
591	0.00361868673458143\\
592	0.0037291949721508\\
593	0.00384481075529708\\
594	0.00397672044113206\\
595	0.00415280575412961\\
596	0.00444436306303141\\
597	0.00503983077166121\\
598	0.00644286460810295\\
599	0\\
600	0\\
};
\addplot [color=mycolor19,solid,forget plot]
  table[row sep=crcr]{%
1	0\\
2	0\\
3	0\\
4	0\\
5	0\\
6	0\\
7	0\\
8	0\\
9	0\\
10	0\\
11	0\\
12	0\\
13	0\\
14	0\\
15	0\\
16	0\\
17	0\\
18	0\\
19	0\\
20	0\\
21	0\\
22	0\\
23	0\\
24	0\\
25	0\\
26	0\\
27	0\\
28	0\\
29	0\\
30	0\\
31	0\\
32	0\\
33	0\\
34	0\\
35	0\\
36	0\\
37	0\\
38	0\\
39	0\\
40	0\\
41	0\\
42	0\\
43	0\\
44	0\\
45	0\\
46	0\\
47	0\\
48	0\\
49	0\\
50	0\\
51	0\\
52	0\\
53	0\\
54	0\\
55	0\\
56	0\\
57	0\\
58	0\\
59	0\\
60	0\\
61	0\\
62	0\\
63	0\\
64	0\\
65	0\\
66	0\\
67	0\\
68	0\\
69	0\\
70	0\\
71	0\\
72	0\\
73	0\\
74	0\\
75	0\\
76	0\\
77	0\\
78	0\\
79	0\\
80	0\\
81	0\\
82	0\\
83	0\\
84	0\\
85	0\\
86	0\\
87	0\\
88	0\\
89	0\\
90	0\\
91	0\\
92	0\\
93	0\\
94	0\\
95	0\\
96	0\\
97	0\\
98	0\\
99	0\\
100	0\\
101	0\\
102	0\\
103	0\\
104	0\\
105	0\\
106	0\\
107	0\\
108	0\\
109	0\\
110	0\\
111	0\\
112	0\\
113	0\\
114	0\\
115	0\\
116	0\\
117	0\\
118	0\\
119	0\\
120	0\\
121	0\\
122	0\\
123	0\\
124	0\\
125	0\\
126	0\\
127	0\\
128	0\\
129	0\\
130	0\\
131	0\\
132	0\\
133	0\\
134	0\\
135	0\\
136	0\\
137	0\\
138	0\\
139	0\\
140	0\\
141	0\\
142	0\\
143	0\\
144	0\\
145	0\\
146	0\\
147	0\\
148	0\\
149	0\\
150	0\\
151	0\\
152	0\\
153	0\\
154	0\\
155	0\\
156	0\\
157	0\\
158	0\\
159	0\\
160	0\\
161	0\\
162	0\\
163	0\\
164	0\\
165	0\\
166	0\\
167	0\\
168	0\\
169	0\\
170	0\\
171	0\\
172	0\\
173	0\\
174	0\\
175	0\\
176	0\\
177	0\\
178	0\\
179	0\\
180	0\\
181	0\\
182	0\\
183	0\\
184	0\\
185	0\\
186	0\\
187	0\\
188	0\\
189	0\\
190	0\\
191	0\\
192	0\\
193	0\\
194	0\\
195	0\\
196	0\\
197	0\\
198	0\\
199	0\\
200	0\\
201	0\\
202	0\\
203	0\\
204	0\\
205	0\\
206	0\\
207	0\\
208	0\\
209	0\\
210	0\\
211	0\\
212	0\\
213	0\\
214	0\\
215	0\\
216	0\\
217	0\\
218	0\\
219	0\\
220	0\\
221	0\\
222	0\\
223	0\\
224	0\\
225	0\\
226	0\\
227	0\\
228	0\\
229	0\\
230	0\\
231	0\\
232	0\\
233	0\\
234	0\\
235	0\\
236	0\\
237	0\\
238	0\\
239	0\\
240	0\\
241	0\\
242	0\\
243	0\\
244	0\\
245	0\\
246	0\\
247	0\\
248	0\\
249	0\\
250	0\\
251	0\\
252	0\\
253	0\\
254	0\\
255	0\\
256	0\\
257	0\\
258	0\\
259	0\\
260	0\\
261	0\\
262	0\\
263	0\\
264	0\\
265	0\\
266	0\\
267	0\\
268	0\\
269	0\\
270	0\\
271	0\\
272	0\\
273	0\\
274	0\\
275	0\\
276	0\\
277	0\\
278	0\\
279	0\\
280	0\\
281	0\\
282	0\\
283	0\\
284	0\\
285	0\\
286	0\\
287	0\\
288	0\\
289	0\\
290	0\\
291	0\\
292	0\\
293	0\\
294	0\\
295	0\\
296	0\\
297	0\\
298	0\\
299	0\\
300	0\\
301	0\\
302	0\\
303	0\\
304	0\\
305	0\\
306	0\\
307	0\\
308	0\\
309	0\\
310	0\\
311	0\\
312	0\\
313	0\\
314	0\\
315	0\\
316	0\\
317	0\\
318	0\\
319	0\\
320	0\\
321	0\\
322	0\\
323	0\\
324	0\\
325	0\\
326	0\\
327	0\\
328	0\\
329	0\\
330	0\\
331	0\\
332	0\\
333	0\\
334	0\\
335	0\\
336	0\\
337	0\\
338	0\\
339	0\\
340	0\\
341	0\\
342	0\\
343	0\\
344	0\\
345	0\\
346	0\\
347	0\\
348	0\\
349	0\\
350	0\\
351	0\\
352	0\\
353	0\\
354	0\\
355	0\\
356	0\\
357	0\\
358	0\\
359	0\\
360	0\\
361	0\\
362	0\\
363	0\\
364	0\\
365	0\\
366	0\\
367	0\\
368	0\\
369	0\\
370	0\\
371	0\\
372	0\\
373	0\\
374	0\\
375	0\\
376	0\\
377	0\\
378	0\\
379	0\\
380	0\\
381	0\\
382	0\\
383	0\\
384	0\\
385	0\\
386	0\\
387	0\\
388	0\\
389	0\\
390	0\\
391	0\\
392	0\\
393	0\\
394	0\\
395	0\\
396	0\\
397	0\\
398	0\\
399	0\\
400	0\\
401	0\\
402	0\\
403	0\\
404	0\\
405	0\\
406	0\\
407	0\\
408	0\\
409	0\\
410	0\\
411	0\\
412	0\\
413	0\\
414	0\\
415	0\\
416	0\\
417	0\\
418	0\\
419	0\\
420	0\\
421	0\\
422	0\\
423	0\\
424	0\\
425	0\\
426	0\\
427	0\\
428	0\\
429	0\\
430	0\\
431	0\\
432	0\\
433	0\\
434	0\\
435	0\\
436	0\\
437	0\\
438	0\\
439	0\\
440	0\\
441	0\\
442	0\\
443	0\\
444	0\\
445	0\\
446	0\\
447	0\\
448	0\\
449	0\\
450	0\\
451	0\\
452	0\\
453	0\\
454	0\\
455	0\\
456	0\\
457	0\\
458	0\\
459	0\\
460	0\\
461	0\\
462	0\\
463	0\\
464	0\\
465	0\\
466	0\\
467	0\\
468	0\\
469	0\\
470	0\\
471	0\\
472	0\\
473	0\\
474	0\\
475	0\\
476	0\\
477	0\\
478	0\\
479	0\\
480	0\\
481	0\\
482	0\\
483	0\\
484	0\\
485	0\\
486	0\\
487	0\\
488	0\\
489	0\\
490	0\\
491	0\\
492	0\\
493	0\\
494	0\\
495	0\\
496	0\\
497	0\\
498	0\\
499	0\\
500	0\\
501	0\\
502	0\\
503	0\\
504	0\\
505	0\\
506	0\\
507	0\\
508	0\\
509	0\\
510	0\\
511	0\\
512	0\\
513	0\\
514	0\\
515	0\\
516	0\\
517	0\\
518	0\\
519	0\\
520	0\\
521	0\\
522	0\\
523	0\\
524	0\\
525	0\\
526	0\\
527	0\\
528	0\\
529	0\\
530	0\\
531	0\\
532	0\\
533	0\\
534	0\\
535	0\\
536	0\\
537	0\\
538	0\\
539	0\\
540	0\\
541	0\\
542	0\\
543	1.6281523779886e-05\\
544	4.35246637006791e-05\\
545	7.13087122634737e-05\\
546	9.96552821353406e-05\\
547	0.000128587210985315\\
548	0.000158128564717491\\
549	0.000188304258646555\\
550	0.000219151430388676\\
551	0.000250701151250487\\
552	0.000282981360877781\\
553	0.000316017445589515\\
554	0.000349834654432245\\
555	0.000384457427831735\\
556	0.000419909302190068\\
557	0.000456214671566793\\
558	0.000493398798486647\\
559	0.000531487829578367\\
560	0.000570508810018441\\
561	0.000610489697648098\\
562	0.000651459746879398\\
563	0.000693449329616425\\
564	0.000736489807219065\\
565	0.000780619589166497\\
566	0.000825901473907119\\
567	0.000872296831358665\\
568	0.000919843835008248\\
569	0.000968584807559033\\
570	0.00101855905585974\\
571	0.00106996928087748\\
572	0.0011226278711555\\
573	0.00117653489629999\\
574	0.00123176683108536\\
575	0.00128877167129672\\
576	0.00134736204101222\\
577	0.00140700004607742\\
578	0.00146757979920448\\
579	0.00152899875679746\\
580	0.00178174393908874\\
581	0.00207611800840707\\
582	0.00235517143267944\\
583	0.0026041334094422\\
584	0.00286156611296512\\
585	0.00301924574469048\\
586	0.00313397161576881\\
587	0.00323536006699526\\
588	0.00333221291604738\\
589	0.00342954842160734\\
590	0.0035273364337839\\
591	0.00362803120808679\\
592	0.00373275376052986\\
593	0.00384555904940006\\
594	0.00397672044113206\\
595	0.00415280575412961\\
596	0.00444436306303141\\
597	0.00503983077166121\\
598	0.00644286460810295\\
599	0\\
600	0\\
};
\addplot [color=red!50!mycolor17,solid,forget plot]
  table[row sep=crcr]{%
1	0\\
2	0\\
3	0\\
4	0\\
5	0\\
6	0\\
7	0\\
8	0\\
9	0\\
10	0\\
11	0\\
12	0\\
13	0\\
14	0\\
15	0\\
16	0\\
17	0\\
18	0\\
19	0\\
20	0\\
21	0\\
22	0\\
23	0\\
24	0\\
25	0\\
26	0\\
27	0\\
28	0\\
29	0\\
30	0\\
31	0\\
32	0\\
33	0\\
34	0\\
35	0\\
36	0\\
37	0\\
38	0\\
39	0\\
40	0\\
41	0\\
42	0\\
43	0\\
44	0\\
45	0\\
46	0\\
47	0\\
48	0\\
49	0\\
50	0\\
51	0\\
52	0\\
53	0\\
54	0\\
55	0\\
56	0\\
57	0\\
58	0\\
59	0\\
60	0\\
61	0\\
62	0\\
63	0\\
64	0\\
65	0\\
66	0\\
67	0\\
68	0\\
69	0\\
70	0\\
71	0\\
72	0\\
73	0\\
74	0\\
75	0\\
76	0\\
77	0\\
78	0\\
79	0\\
80	0\\
81	0\\
82	0\\
83	0\\
84	0\\
85	0\\
86	0\\
87	0\\
88	0\\
89	0\\
90	0\\
91	0\\
92	0\\
93	0\\
94	0\\
95	0\\
96	0\\
97	0\\
98	0\\
99	0\\
100	0\\
101	0\\
102	0\\
103	0\\
104	0\\
105	0\\
106	0\\
107	0\\
108	0\\
109	0\\
110	0\\
111	0\\
112	0\\
113	0\\
114	0\\
115	0\\
116	0\\
117	0\\
118	0\\
119	0\\
120	0\\
121	0\\
122	0\\
123	0\\
124	0\\
125	0\\
126	0\\
127	0\\
128	0\\
129	0\\
130	0\\
131	0\\
132	0\\
133	0\\
134	0\\
135	0\\
136	0\\
137	0\\
138	0\\
139	0\\
140	0\\
141	0\\
142	0\\
143	0\\
144	0\\
145	0\\
146	0\\
147	0\\
148	0\\
149	0\\
150	0\\
151	0\\
152	0\\
153	0\\
154	0\\
155	0\\
156	0\\
157	0\\
158	0\\
159	0\\
160	0\\
161	0\\
162	0\\
163	0\\
164	0\\
165	0\\
166	0\\
167	0\\
168	0\\
169	0\\
170	0\\
171	0\\
172	0\\
173	0\\
174	0\\
175	0\\
176	0\\
177	0\\
178	0\\
179	0\\
180	0\\
181	0\\
182	0\\
183	0\\
184	0\\
185	0\\
186	0\\
187	0\\
188	0\\
189	0\\
190	0\\
191	0\\
192	0\\
193	0\\
194	0\\
195	0\\
196	0\\
197	0\\
198	0\\
199	0\\
200	0\\
201	0\\
202	0\\
203	0\\
204	0\\
205	0\\
206	0\\
207	0\\
208	0\\
209	0\\
210	0\\
211	0\\
212	0\\
213	0\\
214	0\\
215	0\\
216	0\\
217	0\\
218	0\\
219	0\\
220	0\\
221	0\\
222	0\\
223	0\\
224	0\\
225	0\\
226	0\\
227	0\\
228	0\\
229	0\\
230	0\\
231	0\\
232	0\\
233	0\\
234	0\\
235	0\\
236	0\\
237	0\\
238	0\\
239	0\\
240	0\\
241	0\\
242	0\\
243	0\\
244	0\\
245	0\\
246	0\\
247	0\\
248	0\\
249	0\\
250	0\\
251	0\\
252	0\\
253	0\\
254	0\\
255	0\\
256	0\\
257	0\\
258	0\\
259	0\\
260	0\\
261	0\\
262	0\\
263	0\\
264	0\\
265	0\\
266	0\\
267	0\\
268	0\\
269	0\\
270	0\\
271	0\\
272	0\\
273	0\\
274	0\\
275	0\\
276	0\\
277	0\\
278	0\\
279	0\\
280	0\\
281	0\\
282	0\\
283	0\\
284	0\\
285	0\\
286	0\\
287	0\\
288	0\\
289	0\\
290	0\\
291	0\\
292	0\\
293	0\\
294	0\\
295	0\\
296	0\\
297	0\\
298	0\\
299	0\\
300	0\\
301	0\\
302	0\\
303	0\\
304	0\\
305	0\\
306	0\\
307	0\\
308	0\\
309	0\\
310	0\\
311	0\\
312	0\\
313	0\\
314	0\\
315	0\\
316	0\\
317	0\\
318	0\\
319	0\\
320	0\\
321	0\\
322	0\\
323	0\\
324	0\\
325	0\\
326	0\\
327	0\\
328	0\\
329	0\\
330	0\\
331	0\\
332	0\\
333	0\\
334	0\\
335	0\\
336	0\\
337	0\\
338	0\\
339	0\\
340	0\\
341	0\\
342	0\\
343	0\\
344	0\\
345	0\\
346	0\\
347	0\\
348	0\\
349	0\\
350	0\\
351	0\\
352	0\\
353	0\\
354	0\\
355	0\\
356	0\\
357	0\\
358	0\\
359	0\\
360	0\\
361	0\\
362	0\\
363	0\\
364	0\\
365	0\\
366	0\\
367	0\\
368	0\\
369	0\\
370	0\\
371	0\\
372	0\\
373	0\\
374	0\\
375	0\\
376	0\\
377	0\\
378	0\\
379	0\\
380	0\\
381	0\\
382	0\\
383	0\\
384	0\\
385	0\\
386	0\\
387	0\\
388	0\\
389	0\\
390	0\\
391	0\\
392	0\\
393	0\\
394	0\\
395	0\\
396	0\\
397	0\\
398	0\\
399	0\\
400	0\\
401	0\\
402	0\\
403	0\\
404	0\\
405	0\\
406	0\\
407	0\\
408	0\\
409	0\\
410	0\\
411	0\\
412	0\\
413	0\\
414	0\\
415	0\\
416	0\\
417	0\\
418	0\\
419	0\\
420	0\\
421	0\\
422	0\\
423	0\\
424	0\\
425	0\\
426	0\\
427	0\\
428	0\\
429	0\\
430	0\\
431	0\\
432	0\\
433	0\\
434	0\\
435	0\\
436	0\\
437	0\\
438	0\\
439	0\\
440	0\\
441	0\\
442	0\\
443	0\\
444	0\\
445	0\\
446	0\\
447	0\\
448	0\\
449	0\\
450	0\\
451	0\\
452	0\\
453	0\\
454	0\\
455	0\\
456	0\\
457	0\\
458	0\\
459	0\\
460	0\\
461	0\\
462	0\\
463	0\\
464	0\\
465	0\\
466	0\\
467	0\\
468	0\\
469	0\\
470	0\\
471	0\\
472	0\\
473	0\\
474	0\\
475	0\\
476	0\\
477	0\\
478	0\\
479	0\\
480	0\\
481	0\\
482	0\\
483	0\\
484	0\\
485	0\\
486	0\\
487	0\\
488	0\\
489	0\\
490	0\\
491	0\\
492	0\\
493	0\\
494	0\\
495	0\\
496	0\\
497	0\\
498	0\\
499	0\\
500	0\\
501	0\\
502	0\\
503	0\\
504	0\\
505	0\\
506	0\\
507	0\\
508	0\\
509	0\\
510	0\\
511	0\\
512	0\\
513	0\\
514	0\\
515	0\\
516	0\\
517	0\\
518	0\\
519	0\\
520	0\\
521	0\\
522	0\\
523	0\\
524	0\\
525	0\\
526	0\\
527	0\\
528	0\\
529	0\\
530	0\\
531	0\\
532	0\\
533	0\\
534	0\\
535	0\\
536	0\\
537	0\\
538	0\\
539	0\\
540	0\\
541	0\\
542	2.31819356943729e-05\\
543	4.97808753607325e-05\\
544	7.69628614176778e-05\\
545	0.000104747319125633\\
546	0.000133154249522804\\
547	0.000162204210497156\\
548	0.000191918039953681\\
549	0.000222329556741014\\
550	0.000253464773718292\\
551	0.000285344204403849\\
552	0.00031798791545124\\
553	0.000351416424199199\\
554	0.000385650737194353\\
555	0.00042071243627783\\
556	0.000456623783107471\\
557	0.000493407751197362\\
558	0.000531088593263046\\
559	0.00056969136713803\\
560	0.000609241967672466\\
561	0.000649799792050662\\
562	0.000691347682418427\\
563	0.000733888225033008\\
564	0.000777453378418064\\
565	0.00082207615087447\\
566	0.000867771153143799\\
567	0.000914688584196929\\
568	0.000962856697167533\\
569	0.00101209789593015\\
570	0.00106246397516016\\
571	0.00111390186467904\\
572	0.00116698002710393\\
573	0.00122180007812752\\
574	0.00127748242167784\\
575	0.00133371885625811\\
576	0.00139081764574271\\
577	0.00146406439474163\\
578	0.00175490587469454\\
579	0.00206663004967047\\
580	0.00232104643036537\\
581	0.00257412158610118\\
582	0.00277111122536755\\
583	0.00287961411679738\\
584	0.00297466536060553\\
585	0.00306490423400819\\
586	0.00315466992901367\\
587	0.00324537223361411\\
588	0.0033377976514749\\
589	0.00343211111647921\\
590	0.00352885108616147\\
591	0.00362859107064115\\
592	0.00373286830485375\\
593	0.00384555904940007\\
594	0.00397672044113206\\
595	0.00415280575412962\\
596	0.00444436306303141\\
597	0.00503983077166121\\
598	0.00644286460810295\\
599	0\\
600	0\\
};
\addplot [color=red!40!mycolor19,solid,forget plot]
  table[row sep=crcr]{%
1	0\\
2	0\\
3	0\\
4	0\\
5	0\\
6	0\\
7	0\\
8	0\\
9	0\\
10	0\\
11	0\\
12	0\\
13	0\\
14	0\\
15	0\\
16	0\\
17	0\\
18	0\\
19	0\\
20	0\\
21	0\\
22	0\\
23	0\\
24	0\\
25	0\\
26	0\\
27	0\\
28	0\\
29	0\\
30	0\\
31	0\\
32	0\\
33	0\\
34	0\\
35	0\\
36	0\\
37	0\\
38	0\\
39	0\\
40	0\\
41	0\\
42	0\\
43	0\\
44	0\\
45	0\\
46	0\\
47	0\\
48	0\\
49	0\\
50	0\\
51	0\\
52	0\\
53	0\\
54	0\\
55	0\\
56	0\\
57	0\\
58	0\\
59	0\\
60	0\\
61	0\\
62	0\\
63	0\\
64	0\\
65	0\\
66	0\\
67	0\\
68	0\\
69	0\\
70	0\\
71	0\\
72	0\\
73	0\\
74	0\\
75	0\\
76	0\\
77	0\\
78	0\\
79	0\\
80	0\\
81	0\\
82	0\\
83	0\\
84	0\\
85	0\\
86	0\\
87	0\\
88	0\\
89	0\\
90	0\\
91	0\\
92	0\\
93	0\\
94	0\\
95	0\\
96	0\\
97	0\\
98	0\\
99	0\\
100	0\\
101	0\\
102	0\\
103	0\\
104	0\\
105	0\\
106	0\\
107	0\\
108	0\\
109	0\\
110	0\\
111	0\\
112	0\\
113	0\\
114	0\\
115	0\\
116	0\\
117	0\\
118	0\\
119	0\\
120	0\\
121	0\\
122	0\\
123	0\\
124	0\\
125	0\\
126	0\\
127	0\\
128	0\\
129	0\\
130	0\\
131	0\\
132	0\\
133	0\\
134	0\\
135	0\\
136	0\\
137	0\\
138	0\\
139	0\\
140	0\\
141	0\\
142	0\\
143	0\\
144	0\\
145	0\\
146	0\\
147	0\\
148	0\\
149	0\\
150	0\\
151	0\\
152	0\\
153	0\\
154	0\\
155	0\\
156	0\\
157	0\\
158	0\\
159	0\\
160	0\\
161	0\\
162	0\\
163	0\\
164	0\\
165	0\\
166	0\\
167	0\\
168	0\\
169	0\\
170	0\\
171	0\\
172	0\\
173	0\\
174	0\\
175	0\\
176	0\\
177	0\\
178	0\\
179	0\\
180	0\\
181	0\\
182	0\\
183	0\\
184	0\\
185	0\\
186	0\\
187	0\\
188	0\\
189	0\\
190	0\\
191	0\\
192	0\\
193	0\\
194	0\\
195	0\\
196	0\\
197	0\\
198	0\\
199	0\\
200	0\\
201	0\\
202	0\\
203	0\\
204	0\\
205	0\\
206	0\\
207	0\\
208	0\\
209	0\\
210	0\\
211	0\\
212	0\\
213	0\\
214	0\\
215	0\\
216	0\\
217	0\\
218	0\\
219	0\\
220	0\\
221	0\\
222	0\\
223	0\\
224	0\\
225	0\\
226	0\\
227	0\\
228	0\\
229	0\\
230	0\\
231	0\\
232	0\\
233	0\\
234	0\\
235	0\\
236	0\\
237	0\\
238	0\\
239	0\\
240	0\\
241	0\\
242	0\\
243	0\\
244	0\\
245	0\\
246	0\\
247	0\\
248	0\\
249	0\\
250	0\\
251	0\\
252	0\\
253	0\\
254	0\\
255	0\\
256	0\\
257	0\\
258	0\\
259	0\\
260	0\\
261	0\\
262	0\\
263	0\\
264	0\\
265	0\\
266	0\\
267	0\\
268	0\\
269	0\\
270	0\\
271	0\\
272	0\\
273	0\\
274	0\\
275	0\\
276	0\\
277	0\\
278	0\\
279	0\\
280	0\\
281	0\\
282	0\\
283	0\\
284	0\\
285	0\\
286	0\\
287	0\\
288	0\\
289	0\\
290	0\\
291	0\\
292	0\\
293	0\\
294	0\\
295	0\\
296	0\\
297	0\\
298	0\\
299	0\\
300	0\\
301	0\\
302	0\\
303	0\\
304	0\\
305	0\\
306	0\\
307	0\\
308	0\\
309	0\\
310	0\\
311	0\\
312	0\\
313	0\\
314	0\\
315	0\\
316	0\\
317	0\\
318	0\\
319	0\\
320	0\\
321	0\\
322	0\\
323	0\\
324	0\\
325	0\\
326	0\\
327	0\\
328	0\\
329	0\\
330	0\\
331	0\\
332	0\\
333	0\\
334	0\\
335	0\\
336	0\\
337	0\\
338	0\\
339	0\\
340	0\\
341	0\\
342	0\\
343	0\\
344	0\\
345	0\\
346	0\\
347	0\\
348	0\\
349	0\\
350	0\\
351	0\\
352	0\\
353	0\\
354	0\\
355	0\\
356	0\\
357	0\\
358	0\\
359	0\\
360	0\\
361	0\\
362	0\\
363	0\\
364	0\\
365	0\\
366	0\\
367	0\\
368	0\\
369	0\\
370	0\\
371	0\\
372	0\\
373	0\\
374	0\\
375	0\\
376	0\\
377	0\\
378	0\\
379	0\\
380	0\\
381	0\\
382	0\\
383	0\\
384	0\\
385	0\\
386	0\\
387	0\\
388	0\\
389	0\\
390	0\\
391	0\\
392	0\\
393	0\\
394	0\\
395	0\\
396	0\\
397	0\\
398	0\\
399	0\\
400	0\\
401	0\\
402	0\\
403	0\\
404	0\\
405	0\\
406	0\\
407	0\\
408	0\\
409	0\\
410	0\\
411	0\\
412	0\\
413	0\\
414	0\\
415	0\\
416	0\\
417	0\\
418	0\\
419	0\\
420	0\\
421	0\\
422	0\\
423	0\\
424	0\\
425	0\\
426	0\\
427	0\\
428	0\\
429	0\\
430	0\\
431	0\\
432	0\\
433	0\\
434	0\\
435	0\\
436	0\\
437	0\\
438	0\\
439	0\\
440	0\\
441	0\\
442	0\\
443	0\\
444	0\\
445	0\\
446	0\\
447	0\\
448	0\\
449	0\\
450	0\\
451	0\\
452	0\\
453	0\\
454	0\\
455	0\\
456	0\\
457	0\\
458	0\\
459	0\\
460	0\\
461	0\\
462	0\\
463	0\\
464	0\\
465	0\\
466	0\\
467	0\\
468	0\\
469	0\\
470	0\\
471	0\\
472	0\\
473	0\\
474	0\\
475	0\\
476	0\\
477	0\\
478	0\\
479	0\\
480	0\\
481	0\\
482	0\\
483	0\\
484	0\\
485	0\\
486	0\\
487	0\\
488	0\\
489	0\\
490	0\\
491	0\\
492	0\\
493	0\\
494	0\\
495	0\\
496	0\\
497	0\\
498	0\\
499	0\\
500	0\\
501	0\\
502	0\\
503	0\\
504	0\\
505	0\\
506	0\\
507	0\\
508	0\\
509	0\\
510	0\\
511	0\\
512	0\\
513	0\\
514	0\\
515	0\\
516	0\\
517	0\\
518	0\\
519	0\\
520	0\\
521	0\\
522	0\\
523	0\\
524	0\\
525	0\\
526	0\\
527	0\\
528	0\\
529	0\\
530	0\\
531	0\\
532	0\\
533	0\\
534	0\\
535	0\\
536	0\\
537	0\\
538	0\\
539	0\\
540	0\\
541	2.25145755298818e-05\\
542	4.88271536578352e-05\\
543	7.57244088362082e-05\\
544	0.000103224216248849\\
545	0.000131344948204077\\
546	0.000160105482906932\\
547	0.000189525285251365\\
548	0.000219636184945385\\
549	0.00025046323928381\\
550	0.000282024413180058\\
551	0.000314338152631727\\
552	0.000347423417862681\\
553	0.000381299724883977\\
554	0.000415987720097952\\
555	0.000451508665813917\\
556	0.000487884480583933\\
557	0.000525184470262948\\
558	0.000563348240308322\\
559	0.000602398540839097\\
560	0.00064236432155272\\
561	0.00068325304476226\\
562	0.000725110760189988\\
563	0.000767980392286332\\
564	0.000812039238838757\\
565	0.000857143872510442\\
566	0.000903200365809114\\
567	0.000950252013351359\\
568	0.00099832220385339\\
569	0.0010475654496909\\
570	0.00109902998654938\\
571	0.00115110526262422\\
572	0.00120382222347614\\
573	0.00125713529321454\\
574	0.0013114218600033\\
575	0.00138416572299911\\
576	0.00168362914241537\\
577	0.00200121253368579\\
578	0.00225018105728763\\
579	0.00250573441948143\\
580	0.00263216434825482\\
581	0.00272775687042408\\
582	0.00281259168714533\\
583	0.00289684044792754\\
584	0.00298182678347091\\
585	0.00306822904379038\\
586	0.00315632033977421\\
587	0.00324628153905063\\
588	0.00333822811831753\\
589	0.00343235311897413\\
590	0.00352893796606579\\
591	0.00362860835901882\\
592	0.00373286830485375\\
593	0.00384555904940006\\
594	0.00397672044113206\\
595	0.00415280575412962\\
596	0.00444436306303141\\
597	0.00503983077166121\\
598	0.00644286460810295\\
599	0\\
600	0\\
};
\addplot [color=red!75!mycolor17,solid,forget plot]
  table[row sep=crcr]{%
1	0\\
2	0\\
3	0\\
4	0\\
5	0\\
6	0\\
7	0\\
8	0\\
9	0\\
10	0\\
11	0\\
12	0\\
13	0\\
14	0\\
15	0\\
16	0\\
17	0\\
18	0\\
19	0\\
20	0\\
21	0\\
22	0\\
23	0\\
24	0\\
25	0\\
26	0\\
27	0\\
28	0\\
29	0\\
30	0\\
31	0\\
32	0\\
33	0\\
34	0\\
35	0\\
36	0\\
37	0\\
38	0\\
39	0\\
40	0\\
41	0\\
42	0\\
43	0\\
44	0\\
45	0\\
46	0\\
47	0\\
48	0\\
49	0\\
50	0\\
51	0\\
52	0\\
53	0\\
54	0\\
55	0\\
56	0\\
57	0\\
58	0\\
59	0\\
60	0\\
61	0\\
62	0\\
63	0\\
64	0\\
65	0\\
66	0\\
67	0\\
68	0\\
69	0\\
70	0\\
71	0\\
72	0\\
73	0\\
74	0\\
75	0\\
76	0\\
77	0\\
78	0\\
79	0\\
80	0\\
81	0\\
82	0\\
83	0\\
84	0\\
85	0\\
86	0\\
87	0\\
88	0\\
89	0\\
90	0\\
91	0\\
92	0\\
93	0\\
94	0\\
95	0\\
96	0\\
97	0\\
98	0\\
99	0\\
100	0\\
101	0\\
102	0\\
103	0\\
104	0\\
105	0\\
106	0\\
107	0\\
108	0\\
109	0\\
110	0\\
111	0\\
112	0\\
113	0\\
114	0\\
115	0\\
116	0\\
117	0\\
118	0\\
119	0\\
120	0\\
121	0\\
122	0\\
123	0\\
124	0\\
125	0\\
126	0\\
127	0\\
128	0\\
129	0\\
130	0\\
131	0\\
132	0\\
133	0\\
134	0\\
135	0\\
136	0\\
137	0\\
138	0\\
139	0\\
140	0\\
141	0\\
142	0\\
143	0\\
144	0\\
145	0\\
146	0\\
147	0\\
148	0\\
149	0\\
150	0\\
151	0\\
152	0\\
153	0\\
154	0\\
155	0\\
156	0\\
157	0\\
158	0\\
159	0\\
160	0\\
161	0\\
162	0\\
163	0\\
164	0\\
165	0\\
166	0\\
167	0\\
168	0\\
169	0\\
170	0\\
171	0\\
172	0\\
173	0\\
174	0\\
175	0\\
176	0\\
177	0\\
178	0\\
179	0\\
180	0\\
181	0\\
182	0\\
183	0\\
184	0\\
185	0\\
186	0\\
187	0\\
188	0\\
189	0\\
190	0\\
191	0\\
192	0\\
193	0\\
194	0\\
195	0\\
196	0\\
197	0\\
198	0\\
199	0\\
200	0\\
201	0\\
202	0\\
203	0\\
204	0\\
205	0\\
206	0\\
207	0\\
208	0\\
209	0\\
210	0\\
211	0\\
212	0\\
213	0\\
214	0\\
215	0\\
216	0\\
217	0\\
218	0\\
219	0\\
220	0\\
221	0\\
222	0\\
223	0\\
224	0\\
225	0\\
226	0\\
227	0\\
228	0\\
229	0\\
230	0\\
231	0\\
232	0\\
233	0\\
234	0\\
235	0\\
236	0\\
237	0\\
238	0\\
239	0\\
240	0\\
241	0\\
242	0\\
243	0\\
244	0\\
245	0\\
246	0\\
247	0\\
248	0\\
249	0\\
250	0\\
251	0\\
252	0\\
253	0\\
254	0\\
255	0\\
256	0\\
257	0\\
258	0\\
259	0\\
260	0\\
261	0\\
262	0\\
263	0\\
264	0\\
265	0\\
266	0\\
267	0\\
268	0\\
269	0\\
270	0\\
271	0\\
272	0\\
273	0\\
274	0\\
275	0\\
276	0\\
277	0\\
278	0\\
279	0\\
280	0\\
281	0\\
282	0\\
283	0\\
284	0\\
285	0\\
286	0\\
287	0\\
288	0\\
289	0\\
290	0\\
291	0\\
292	0\\
293	0\\
294	0\\
295	0\\
296	0\\
297	0\\
298	0\\
299	0\\
300	0\\
301	0\\
302	0\\
303	0\\
304	0\\
305	0\\
306	0\\
307	0\\
308	0\\
309	0\\
310	0\\
311	0\\
312	0\\
313	0\\
314	0\\
315	0\\
316	0\\
317	0\\
318	0\\
319	0\\
320	0\\
321	0\\
322	0\\
323	0\\
324	0\\
325	0\\
326	0\\
327	0\\
328	0\\
329	0\\
330	0\\
331	0\\
332	0\\
333	0\\
334	0\\
335	0\\
336	0\\
337	0\\
338	0\\
339	0\\
340	0\\
341	0\\
342	0\\
343	0\\
344	0\\
345	0\\
346	0\\
347	0\\
348	0\\
349	0\\
350	0\\
351	0\\
352	0\\
353	0\\
354	0\\
355	0\\
356	0\\
357	0\\
358	0\\
359	0\\
360	0\\
361	0\\
362	0\\
363	0\\
364	0\\
365	0\\
366	0\\
367	0\\
368	0\\
369	0\\
370	0\\
371	0\\
372	0\\
373	0\\
374	0\\
375	0\\
376	0\\
377	0\\
378	0\\
379	0\\
380	0\\
381	0\\
382	0\\
383	0\\
384	0\\
385	0\\
386	0\\
387	0\\
388	0\\
389	0\\
390	0\\
391	0\\
392	0\\
393	0\\
394	0\\
395	0\\
396	0\\
397	0\\
398	0\\
399	0\\
400	0\\
401	0\\
402	0\\
403	0\\
404	0\\
405	0\\
406	0\\
407	0\\
408	0\\
409	0\\
410	0\\
411	0\\
412	0\\
413	0\\
414	0\\
415	0\\
416	0\\
417	0\\
418	0\\
419	0\\
420	0\\
421	0\\
422	0\\
423	0\\
424	0\\
425	0\\
426	0\\
427	0\\
428	0\\
429	0\\
430	0\\
431	0\\
432	0\\
433	0\\
434	0\\
435	0\\
436	0\\
437	0\\
438	0\\
439	0\\
440	0\\
441	0\\
442	0\\
443	0\\
444	0\\
445	0\\
446	0\\
447	0\\
448	0\\
449	0\\
450	0\\
451	0\\
452	0\\
453	0\\
454	0\\
455	0\\
456	0\\
457	0\\
458	0\\
459	0\\
460	0\\
461	0\\
462	0\\
463	0\\
464	0\\
465	0\\
466	0\\
467	0\\
468	0\\
469	0\\
470	0\\
471	0\\
472	0\\
473	0\\
474	0\\
475	0\\
476	0\\
477	0\\
478	0\\
479	0\\
480	0\\
481	0\\
482	0\\
483	0\\
484	0\\
485	0\\
486	0\\
487	0\\
488	0\\
489	0\\
490	0\\
491	0\\
492	0\\
493	0\\
494	0\\
495	0\\
496	0\\
497	0\\
498	0\\
499	0\\
500	0\\
501	0\\
502	0\\
503	0\\
504	0\\
505	0\\
506	0\\
507	0\\
508	0\\
509	0\\
510	0\\
511	0\\
512	0\\
513	0\\
514	0\\
515	0\\
516	0\\
517	0\\
518	0\\
519	0\\
520	0\\
521	0\\
522	0\\
523	0\\
524	0\\
525	0\\
526	0\\
527	0\\
528	0\\
529	0\\
530	0\\
531	0\\
532	0\\
533	0\\
534	0\\
535	0\\
536	0\\
537	0\\
538	0\\
539	0\\
540	1.80572611580464e-05\\
541	4.4118006200525e-05\\
542	7.07542032362198e-05\\
543	9.79828118109392e-05\\
544	0.000125821262619649\\
545	0.000154287471883814\\
546	0.000183399616108164\\
547	0.000213189224881146\\
548	0.000243680143284574\\
549	0.00027488924859386\\
550	0.000306834372255217\\
551	0.000339533902369266\\
552	0.00037300762948773\\
553	0.000407321246047958\\
554	0.000442407509995026\\
555	0.000478288527401575\\
556	0.000514990430784664\\
557	0.000552506247906185\\
558	0.000590897395376276\\
559	0.000630191158809039\\
560	0.000670413411890405\\
561	0.000711735510261396\\
562	0.00075401250105808\\
563	0.000797191808578931\\
564	0.000841224772035708\\
565	0.000886213838323151\\
566	0.000932258739848622\\
567	0.000979764676566103\\
568	0.00102876658617279\\
569	0.00107851644739995\\
570	0.00112845627858277\\
571	0.00117925389621404\\
572	0.00123071645413494\\
573	0.00128292972718838\\
574	0.00156765047686712\\
575	0.00189256888265377\\
576	0.00214409180946553\\
577	0.00239107643868339\\
578	0.00249057912042086\\
579	0.00257377822812136\\
580	0.00265393568636939\\
581	0.00273394370368479\\
582	0.00281520413723953\\
583	0.00289795905152283\\
584	0.00298235807749845\\
585	0.00306849821799598\\
586	0.00315646723333688\\
587	0.003246352526171\\
588	0.00333826624313612\\
589	0.00343236642273867\\
590	0.00352894054086423\\
591	0.00362860835901881\\
592	0.00373286830485375\\
593	0.00384555904940006\\
594	0.00397672044113205\\
595	0.00415280575412962\\
596	0.00444436306303141\\
597	0.00503983077166121\\
598	0.00644286460810295\\
599	0\\
600	0\\
};
\addplot [color=red!80!mycolor19,solid,forget plot]
  table[row sep=crcr]{%
1	0\\
2	0\\
3	0\\
4	0\\
5	0\\
6	0\\
7	0\\
8	0\\
9	0\\
10	0\\
11	0\\
12	0\\
13	0\\
14	0\\
15	0\\
16	0\\
17	0\\
18	0\\
19	0\\
20	0\\
21	0\\
22	0\\
23	0\\
24	0\\
25	0\\
26	0\\
27	0\\
28	0\\
29	0\\
30	0\\
31	0\\
32	0\\
33	0\\
34	0\\
35	0\\
36	0\\
37	0\\
38	0\\
39	0\\
40	0\\
41	0\\
42	0\\
43	0\\
44	0\\
45	0\\
46	0\\
47	0\\
48	0\\
49	0\\
50	0\\
51	0\\
52	0\\
53	0\\
54	0\\
55	0\\
56	0\\
57	0\\
58	0\\
59	0\\
60	0\\
61	0\\
62	0\\
63	0\\
64	0\\
65	0\\
66	0\\
67	0\\
68	0\\
69	0\\
70	0\\
71	0\\
72	0\\
73	0\\
74	0\\
75	0\\
76	0\\
77	0\\
78	0\\
79	0\\
80	0\\
81	0\\
82	0\\
83	0\\
84	0\\
85	0\\
86	0\\
87	0\\
88	0\\
89	0\\
90	0\\
91	0\\
92	0\\
93	0\\
94	0\\
95	0\\
96	0\\
97	0\\
98	0\\
99	0\\
100	0\\
101	0\\
102	0\\
103	0\\
104	0\\
105	0\\
106	0\\
107	0\\
108	0\\
109	0\\
110	0\\
111	0\\
112	0\\
113	0\\
114	0\\
115	0\\
116	0\\
117	0\\
118	0\\
119	0\\
120	0\\
121	0\\
122	0\\
123	0\\
124	0\\
125	0\\
126	0\\
127	0\\
128	0\\
129	0\\
130	0\\
131	0\\
132	0\\
133	0\\
134	0\\
135	0\\
136	0\\
137	0\\
138	0\\
139	0\\
140	0\\
141	0\\
142	0\\
143	0\\
144	0\\
145	0\\
146	0\\
147	0\\
148	0\\
149	0\\
150	0\\
151	0\\
152	0\\
153	0\\
154	0\\
155	0\\
156	0\\
157	0\\
158	0\\
159	0\\
160	0\\
161	0\\
162	0\\
163	0\\
164	0\\
165	0\\
166	0\\
167	0\\
168	0\\
169	0\\
170	0\\
171	0\\
172	0\\
173	0\\
174	0\\
175	0\\
176	0\\
177	0\\
178	0\\
179	0\\
180	0\\
181	0\\
182	0\\
183	0\\
184	0\\
185	0\\
186	0\\
187	0\\
188	0\\
189	0\\
190	0\\
191	0\\
192	0\\
193	0\\
194	0\\
195	0\\
196	0\\
197	0\\
198	0\\
199	0\\
200	0\\
201	0\\
202	0\\
203	0\\
204	0\\
205	0\\
206	0\\
207	0\\
208	0\\
209	0\\
210	0\\
211	0\\
212	0\\
213	0\\
214	0\\
215	0\\
216	0\\
217	0\\
218	0\\
219	0\\
220	0\\
221	0\\
222	0\\
223	0\\
224	0\\
225	0\\
226	0\\
227	0\\
228	0\\
229	0\\
230	0\\
231	0\\
232	0\\
233	0\\
234	0\\
235	0\\
236	0\\
237	0\\
238	0\\
239	0\\
240	0\\
241	0\\
242	0\\
243	0\\
244	0\\
245	0\\
246	0\\
247	0\\
248	0\\
249	0\\
250	0\\
251	0\\
252	0\\
253	0\\
254	0\\
255	0\\
256	0\\
257	0\\
258	0\\
259	0\\
260	0\\
261	0\\
262	0\\
263	0\\
264	0\\
265	0\\
266	0\\
267	0\\
268	0\\
269	0\\
270	0\\
271	0\\
272	0\\
273	0\\
274	0\\
275	0\\
276	0\\
277	0\\
278	0\\
279	0\\
280	0\\
281	0\\
282	0\\
283	0\\
284	0\\
285	0\\
286	0\\
287	0\\
288	0\\
289	0\\
290	0\\
291	0\\
292	0\\
293	0\\
294	0\\
295	0\\
296	0\\
297	0\\
298	0\\
299	0\\
300	0\\
301	0\\
302	0\\
303	0\\
304	0\\
305	0\\
306	0\\
307	0\\
308	0\\
309	0\\
310	0\\
311	0\\
312	0\\
313	0\\
314	0\\
315	0\\
316	0\\
317	0\\
318	0\\
319	0\\
320	0\\
321	0\\
322	0\\
323	0\\
324	0\\
325	0\\
326	0\\
327	0\\
328	0\\
329	0\\
330	0\\
331	0\\
332	0\\
333	0\\
334	0\\
335	0\\
336	0\\
337	0\\
338	0\\
339	0\\
340	0\\
341	0\\
342	0\\
343	0\\
344	0\\
345	0\\
346	0\\
347	0\\
348	0\\
349	0\\
350	0\\
351	0\\
352	0\\
353	0\\
354	0\\
355	0\\
356	0\\
357	0\\
358	0\\
359	0\\
360	0\\
361	0\\
362	0\\
363	0\\
364	0\\
365	0\\
366	0\\
367	0\\
368	0\\
369	0\\
370	0\\
371	0\\
372	0\\
373	0\\
374	0\\
375	0\\
376	0\\
377	0\\
378	0\\
379	0\\
380	0\\
381	0\\
382	0\\
383	0\\
384	0\\
385	0\\
386	0\\
387	0\\
388	0\\
389	0\\
390	0\\
391	0\\
392	0\\
393	0\\
394	0\\
395	0\\
396	0\\
397	0\\
398	0\\
399	0\\
400	0\\
401	0\\
402	0\\
403	0\\
404	0\\
405	0\\
406	0\\
407	0\\
408	0\\
409	0\\
410	0\\
411	0\\
412	0\\
413	0\\
414	0\\
415	0\\
416	0\\
417	0\\
418	0\\
419	0\\
420	0\\
421	0\\
422	0\\
423	0\\
424	0\\
425	0\\
426	0\\
427	0\\
428	0\\
429	0\\
430	0\\
431	0\\
432	0\\
433	0\\
434	0\\
435	0\\
436	0\\
437	0\\
438	0\\
439	0\\
440	0\\
441	0\\
442	0\\
443	0\\
444	0\\
445	0\\
446	0\\
447	0\\
448	0\\
449	0\\
450	0\\
451	0\\
452	0\\
453	0\\
454	0\\
455	0\\
456	0\\
457	0\\
458	0\\
459	0\\
460	0\\
461	0\\
462	0\\
463	0\\
464	0\\
465	0\\
466	0\\
467	0\\
468	0\\
469	0\\
470	0\\
471	0\\
472	0\\
473	0\\
474	0\\
475	0\\
476	0\\
477	0\\
478	0\\
479	0\\
480	0\\
481	0\\
482	0\\
483	0\\
484	0\\
485	0\\
486	0\\
487	0\\
488	0\\
489	0\\
490	0\\
491	0\\
492	0\\
493	0\\
494	0\\
495	0\\
496	0\\
497	0\\
498	0\\
499	0\\
500	0\\
501	0\\
502	0\\
503	0\\
504	0\\
505	0\\
506	0\\
507	0\\
508	0\\
509	0\\
510	0\\
511	0\\
512	0\\
513	0\\
514	0\\
515	0\\
516	0\\
517	0\\
518	0\\
519	0\\
520	0\\
521	0\\
522	0\\
523	0\\
524	0\\
525	0\\
526	0\\
527	0\\
528	0\\
529	0\\
530	0\\
531	0\\
532	0\\
533	0\\
534	0\\
535	0\\
536	0\\
537	0\\
538	0\\
539	1.09971123564833e-05\\
540	3.67777687958609e-05\\
541	6.31213625684237e-05\\
542	9.00441133591266e-05\\
543	0.00011756267562623\\
544	0.000145694153413752\\
545	0.000174455739705744\\
546	0.000203878155683555\\
547	0.00023398579118046\\
548	0.000264795179737243\\
549	0.000296366118124326\\
550	0.000328644365716624\\
551	0.000361637574244406\\
552	0.000395368372265394\\
553	0.000429826296342454\\
554	0.000465072355643675\\
555	0.00050112828782099\\
556	0.000538014176942687\\
557	0.00057573329173453\\
558	0.000614484468182088\\
559	0.000654187719273962\\
560	0.000694706083181759\\
561	0.00073598111257611\\
562	0.00077813831988304\\
563	0.000821236596718493\\
564	0.000865236156171182\\
565	0.000910969534659842\\
566	0.000957920024337713\\
567	0.00100529101218676\\
568	0.0010530258921727\\
569	0.00110148835858687\\
570	0.00115017727882363\\
571	0.00120010163547213\\
572	0.00140776483562387\\
573	0.00174519063704481\\
574	0.00200611363934863\\
575	0.00225233627854361\\
576	0.00234714808882544\\
577	0.00242410810580428\\
578	0.00250007651960148\\
579	0.00257684337440823\\
580	0.00265485410432229\\
581	0.00273433854836808\\
582	0.00281537804625545\\
583	0.0028980434054312\\
584	0.00298240150095583\\
585	0.00306852173201496\\
586	0.00315647873661191\\
587	0.00324635845014596\\
588	0.00333826825403972\\
589	0.00343236680134408\\
590	0.00352894054086423\\
591	0.00362860835901881\\
592	0.00373286830485375\\
593	0.00384555904940006\\
594	0.00397672044113206\\
595	0.00415280575412962\\
596	0.00444436306303141\\
597	0.00503983077166121\\
598	0.00644286460810295\\
599	0\\
600	0\\
};
\addplot [color=red,solid,forget plot]
  table[row sep=crcr]{%
1	0\\
2	0\\
3	0\\
4	0\\
5	0\\
6	0\\
7	0\\
8	0\\
9	0\\
10	0\\
11	0\\
12	0\\
13	0\\
14	0\\
15	0\\
16	0\\
17	0\\
18	0\\
19	0\\
20	0\\
21	0\\
22	0\\
23	0\\
24	0\\
25	0\\
26	0\\
27	0\\
28	0\\
29	0\\
30	0\\
31	0\\
32	0\\
33	0\\
34	0\\
35	0\\
36	0\\
37	0\\
38	0\\
39	0\\
40	0\\
41	0\\
42	0\\
43	0\\
44	0\\
45	0\\
46	0\\
47	0\\
48	0\\
49	0\\
50	0\\
51	0\\
52	0\\
53	0\\
54	0\\
55	0\\
56	0\\
57	0\\
58	0\\
59	0\\
60	0\\
61	0\\
62	0\\
63	0\\
64	0\\
65	0\\
66	0\\
67	0\\
68	0\\
69	0\\
70	0\\
71	0\\
72	0\\
73	0\\
74	0\\
75	0\\
76	0\\
77	0\\
78	0\\
79	0\\
80	0\\
81	0\\
82	0\\
83	0\\
84	0\\
85	0\\
86	0\\
87	0\\
88	0\\
89	0\\
90	0\\
91	0\\
92	0\\
93	0\\
94	0\\
95	0\\
96	0\\
97	0\\
98	0\\
99	0\\
100	0\\
101	0\\
102	0\\
103	0\\
104	0\\
105	0\\
106	0\\
107	0\\
108	0\\
109	0\\
110	0\\
111	0\\
112	0\\
113	0\\
114	0\\
115	0\\
116	0\\
117	0\\
118	0\\
119	0\\
120	0\\
121	0\\
122	0\\
123	0\\
124	0\\
125	0\\
126	0\\
127	0\\
128	0\\
129	0\\
130	0\\
131	0\\
132	0\\
133	0\\
134	0\\
135	0\\
136	0\\
137	0\\
138	0\\
139	0\\
140	0\\
141	0\\
142	0\\
143	0\\
144	0\\
145	0\\
146	0\\
147	0\\
148	0\\
149	0\\
150	0\\
151	0\\
152	0\\
153	0\\
154	0\\
155	0\\
156	0\\
157	0\\
158	0\\
159	0\\
160	0\\
161	0\\
162	0\\
163	0\\
164	0\\
165	0\\
166	0\\
167	0\\
168	0\\
169	0\\
170	0\\
171	0\\
172	0\\
173	0\\
174	0\\
175	0\\
176	0\\
177	0\\
178	0\\
179	0\\
180	0\\
181	0\\
182	0\\
183	0\\
184	0\\
185	0\\
186	0\\
187	0\\
188	0\\
189	0\\
190	0\\
191	0\\
192	0\\
193	0\\
194	0\\
195	0\\
196	0\\
197	0\\
198	0\\
199	0\\
200	0\\
201	0\\
202	0\\
203	0\\
204	0\\
205	0\\
206	0\\
207	0\\
208	0\\
209	0\\
210	0\\
211	0\\
212	0\\
213	0\\
214	0\\
215	0\\
216	0\\
217	0\\
218	0\\
219	0\\
220	0\\
221	0\\
222	0\\
223	0\\
224	0\\
225	0\\
226	0\\
227	0\\
228	0\\
229	0\\
230	0\\
231	0\\
232	0\\
233	0\\
234	0\\
235	0\\
236	0\\
237	0\\
238	0\\
239	0\\
240	0\\
241	0\\
242	0\\
243	0\\
244	0\\
245	0\\
246	0\\
247	0\\
248	0\\
249	0\\
250	0\\
251	0\\
252	0\\
253	0\\
254	0\\
255	0\\
256	0\\
257	0\\
258	0\\
259	0\\
260	0\\
261	0\\
262	0\\
263	0\\
264	0\\
265	0\\
266	0\\
267	0\\
268	0\\
269	0\\
270	0\\
271	0\\
272	0\\
273	0\\
274	0\\
275	0\\
276	0\\
277	0\\
278	0\\
279	0\\
280	0\\
281	0\\
282	0\\
283	0\\
284	0\\
285	0\\
286	0\\
287	0\\
288	0\\
289	0\\
290	0\\
291	0\\
292	0\\
293	0\\
294	0\\
295	0\\
296	0\\
297	0\\
298	0\\
299	0\\
300	0\\
301	0\\
302	0\\
303	0\\
304	0\\
305	0\\
306	0\\
307	0\\
308	0\\
309	0\\
310	0\\
311	0\\
312	0\\
313	0\\
314	0\\
315	0\\
316	0\\
317	0\\
318	0\\
319	0\\
320	0\\
321	0\\
322	0\\
323	0\\
324	0\\
325	0\\
326	0\\
327	0\\
328	0\\
329	0\\
330	0\\
331	0\\
332	0\\
333	0\\
334	0\\
335	0\\
336	0\\
337	0\\
338	0\\
339	0\\
340	0\\
341	0\\
342	0\\
343	0\\
344	0\\
345	0\\
346	0\\
347	0\\
348	0\\
349	0\\
350	0\\
351	0\\
352	0\\
353	0\\
354	0\\
355	0\\
356	0\\
357	0\\
358	0\\
359	0\\
360	0\\
361	0\\
362	0\\
363	0\\
364	0\\
365	0\\
366	0\\
367	0\\
368	0\\
369	0\\
370	0\\
371	0\\
372	0\\
373	0\\
374	0\\
375	0\\
376	0\\
377	0\\
378	0\\
379	0\\
380	0\\
381	0\\
382	0\\
383	0\\
384	0\\
385	0\\
386	0\\
387	0\\
388	0\\
389	0\\
390	0\\
391	0\\
392	0\\
393	0\\
394	0\\
395	0\\
396	0\\
397	0\\
398	0\\
399	0\\
400	0\\
401	0\\
402	0\\
403	0\\
404	0\\
405	0\\
406	0\\
407	0\\
408	0\\
409	0\\
410	0\\
411	0\\
412	0\\
413	0\\
414	0\\
415	0\\
416	0\\
417	0\\
418	0\\
419	0\\
420	0\\
421	0\\
422	0\\
423	0\\
424	0\\
425	0\\
426	0\\
427	0\\
428	0\\
429	0\\
430	0\\
431	0\\
432	0\\
433	0\\
434	0\\
435	0\\
436	0\\
437	0\\
438	0\\
439	0\\
440	0\\
441	0\\
442	0\\
443	0\\
444	0\\
445	0\\
446	0\\
447	0\\
448	0\\
449	0\\
450	0\\
451	0\\
452	0\\
453	0\\
454	0\\
455	0\\
456	0\\
457	0\\
458	0\\
459	0\\
460	0\\
461	0\\
462	0\\
463	0\\
464	0\\
465	0\\
466	0\\
467	0\\
468	0\\
469	0\\
470	0\\
471	0\\
472	0\\
473	0\\
474	0\\
475	0\\
476	0\\
477	0\\
478	0\\
479	0\\
480	0\\
481	0\\
482	0\\
483	0\\
484	0\\
485	0\\
486	0\\
487	0\\
488	0\\
489	0\\
490	0\\
491	0\\
492	0\\
493	0\\
494	0\\
495	0\\
496	0\\
497	0\\
498	0\\
499	0\\
500	0\\
501	0\\
502	0\\
503	0\\
504	0\\
505	0\\
506	0\\
507	0\\
508	0\\
509	0\\
510	0\\
511	0\\
512	0\\
513	0\\
514	0\\
515	0\\
516	0\\
517	0\\
518	0\\
519	0\\
520	0\\
521	0\\
522	0\\
523	0\\
524	0\\
525	0\\
526	0\\
527	0\\
528	0\\
529	0\\
530	0\\
531	0\\
532	0\\
533	0\\
534	0\\
535	0\\
536	0\\
537	0\\
538	1.87556343843139e-06\\
539	2.73381310324152e-05\\
540	5.33495288270524e-05\\
541	7.99252542932924e-05\\
542	0.000107081251424983\\
543	0.000134834453286783\\
544	0.000163201833838333\\
545	0.000192240295758037\\
546	0.000221947411190644\\
547	0.000252298954717094\\
548	0.000283315360389555\\
549	0.000314986488951994\\
550	0.000347362970500003\\
551	0.000380469167030157\\
552	0.000414321565112513\\
553	0.000448919563346855\\
554	0.00048433294388368\\
555	0.000520634170094015\\
556	0.000557948080283569\\
557	0.000596000691589419\\
558	0.000634749598647108\\
559	0.000674269636362017\\
560	0.000714637530630376\\
561	0.000755805415251623\\
562	0.000797923053763769\\
563	0.000841878133127252\\
564	0.000886913723009165\\
565	0.00093214151302975\\
566	0.000977757767295516\\
567	0.00102380313004\\
568	0.00107036709861644\\
569	0.00111805503001417\\
570	0.00120939113958143\\
571	0.00154663998556902\\
572	0.00183959615121693\\
573	0.00209756278158636\\
574	0.00220493769069792\\
575	0.00227885628634348\\
576	0.00235163153275207\\
577	0.00242542435592874\\
578	0.00250050669328025\\
579	0.00257697908889624\\
580	0.00265491355281054\\
581	0.00273436544093214\\
582	0.00281539134096267\\
583	0.00289805033361497\\
584	0.00298240522740585\\
585	0.00306852356540592\\
586	0.00315647964482259\\
587	0.0032463587502771\\
588	0.0033382683090296\\
589	0.00343236680134408\\
590	0.00352894054086422\\
591	0.00362860835901881\\
592	0.00373286830485375\\
593	0.00384555904940006\\
594	0.00397672044113206\\
595	0.00415280575412962\\
596	0.00444436306303141\\
597	0.00503983077166121\\
598	0.00644286460810295\\
599	0\\
600	0\\
};
\addplot [color=mycolor20,solid,forget plot]
  table[row sep=crcr]{%
1	0\\
2	0\\
3	0\\
4	0\\
5	0\\
6	0\\
7	0\\
8	0\\
9	0\\
10	0\\
11	0\\
12	0\\
13	0\\
14	0\\
15	0\\
16	0\\
17	0\\
18	0\\
19	0\\
20	0\\
21	0\\
22	0\\
23	0\\
24	0\\
25	0\\
26	0\\
27	0\\
28	0\\
29	0\\
30	0\\
31	0\\
32	0\\
33	0\\
34	0\\
35	0\\
36	0\\
37	0\\
38	0\\
39	0\\
40	0\\
41	0\\
42	0\\
43	0\\
44	0\\
45	0\\
46	0\\
47	0\\
48	0\\
49	0\\
50	0\\
51	0\\
52	0\\
53	0\\
54	0\\
55	0\\
56	0\\
57	0\\
58	0\\
59	0\\
60	0\\
61	0\\
62	0\\
63	0\\
64	0\\
65	0\\
66	0\\
67	0\\
68	0\\
69	0\\
70	0\\
71	0\\
72	0\\
73	0\\
74	0\\
75	0\\
76	0\\
77	0\\
78	0\\
79	0\\
80	0\\
81	0\\
82	0\\
83	0\\
84	0\\
85	0\\
86	0\\
87	0\\
88	0\\
89	0\\
90	0\\
91	0\\
92	0\\
93	0\\
94	0\\
95	0\\
96	0\\
97	0\\
98	0\\
99	0\\
100	0\\
101	0\\
102	0\\
103	0\\
104	0\\
105	0\\
106	0\\
107	0\\
108	0\\
109	0\\
110	0\\
111	0\\
112	0\\
113	0\\
114	0\\
115	0\\
116	0\\
117	0\\
118	0\\
119	0\\
120	0\\
121	0\\
122	0\\
123	0\\
124	0\\
125	0\\
126	0\\
127	0\\
128	0\\
129	0\\
130	0\\
131	0\\
132	0\\
133	0\\
134	0\\
135	0\\
136	0\\
137	0\\
138	0\\
139	0\\
140	0\\
141	0\\
142	0\\
143	0\\
144	0\\
145	0\\
146	0\\
147	0\\
148	0\\
149	0\\
150	0\\
151	0\\
152	0\\
153	0\\
154	0\\
155	0\\
156	0\\
157	0\\
158	0\\
159	0\\
160	0\\
161	0\\
162	0\\
163	0\\
164	0\\
165	0\\
166	0\\
167	0\\
168	0\\
169	0\\
170	0\\
171	0\\
172	0\\
173	0\\
174	0\\
175	0\\
176	0\\
177	0\\
178	0\\
179	0\\
180	0\\
181	0\\
182	0\\
183	0\\
184	0\\
185	0\\
186	0\\
187	0\\
188	0\\
189	0\\
190	0\\
191	0\\
192	0\\
193	0\\
194	0\\
195	0\\
196	0\\
197	0\\
198	0\\
199	0\\
200	0\\
201	0\\
202	0\\
203	0\\
204	0\\
205	0\\
206	0\\
207	0\\
208	0\\
209	0\\
210	0\\
211	0\\
212	0\\
213	0\\
214	0\\
215	0\\
216	0\\
217	0\\
218	0\\
219	0\\
220	0\\
221	0\\
222	0\\
223	0\\
224	0\\
225	0\\
226	0\\
227	0\\
228	0\\
229	0\\
230	0\\
231	0\\
232	0\\
233	0\\
234	0\\
235	0\\
236	0\\
237	0\\
238	0\\
239	0\\
240	0\\
241	0\\
242	0\\
243	0\\
244	0\\
245	0\\
246	0\\
247	0\\
248	0\\
249	0\\
250	0\\
251	0\\
252	0\\
253	0\\
254	0\\
255	0\\
256	0\\
257	0\\
258	0\\
259	0\\
260	0\\
261	0\\
262	0\\
263	0\\
264	0\\
265	0\\
266	0\\
267	0\\
268	0\\
269	0\\
270	0\\
271	0\\
272	0\\
273	0\\
274	0\\
275	0\\
276	0\\
277	0\\
278	0\\
279	0\\
280	0\\
281	0\\
282	0\\
283	0\\
284	0\\
285	0\\
286	0\\
287	0\\
288	0\\
289	0\\
290	0\\
291	0\\
292	0\\
293	0\\
294	0\\
295	0\\
296	0\\
297	0\\
298	0\\
299	0\\
300	0\\
301	0\\
302	0\\
303	0\\
304	0\\
305	0\\
306	0\\
307	0\\
308	0\\
309	0\\
310	0\\
311	0\\
312	0\\
313	0\\
314	0\\
315	0\\
316	0\\
317	0\\
318	0\\
319	0\\
320	0\\
321	0\\
322	0\\
323	0\\
324	0\\
325	0\\
326	0\\
327	0\\
328	0\\
329	0\\
330	0\\
331	0\\
332	0\\
333	0\\
334	0\\
335	0\\
336	0\\
337	0\\
338	0\\
339	0\\
340	0\\
341	0\\
342	0\\
343	0\\
344	0\\
345	0\\
346	0\\
347	0\\
348	0\\
349	0\\
350	0\\
351	0\\
352	0\\
353	0\\
354	0\\
355	0\\
356	0\\
357	0\\
358	0\\
359	0\\
360	0\\
361	0\\
362	0\\
363	0\\
364	0\\
365	0\\
366	0\\
367	0\\
368	0\\
369	0\\
370	0\\
371	0\\
372	0\\
373	0\\
374	0\\
375	0\\
376	0\\
377	0\\
378	0\\
379	0\\
380	0\\
381	0\\
382	0\\
383	0\\
384	0\\
385	0\\
386	0\\
387	0\\
388	0\\
389	0\\
390	0\\
391	0\\
392	0\\
393	0\\
394	0\\
395	0\\
396	0\\
397	0\\
398	0\\
399	0\\
400	0\\
401	0\\
402	0\\
403	0\\
404	0\\
405	0\\
406	0\\
407	0\\
408	0\\
409	0\\
410	0\\
411	0\\
412	0\\
413	0\\
414	0\\
415	0\\
416	0\\
417	0\\
418	0\\
419	0\\
420	0\\
421	0\\
422	0\\
423	0\\
424	0\\
425	0\\
426	0\\
427	0\\
428	0\\
429	0\\
430	0\\
431	0\\
432	0\\
433	0\\
434	0\\
435	0\\
436	0\\
437	0\\
438	0\\
439	0\\
440	0\\
441	0\\
442	0\\
443	0\\
444	0\\
445	0\\
446	0\\
447	0\\
448	0\\
449	0\\
450	0\\
451	0\\
452	0\\
453	0\\
454	0\\
455	0\\
456	0\\
457	0\\
458	0\\
459	0\\
460	0\\
461	0\\
462	0\\
463	0\\
464	0\\
465	0\\
466	0\\
467	0\\
468	0\\
469	0\\
470	0\\
471	0\\
472	0\\
473	0\\
474	0\\
475	0\\
476	0\\
477	0\\
478	0\\
479	0\\
480	0\\
481	0\\
482	0\\
483	0\\
484	0\\
485	0\\
486	0\\
487	0\\
488	0\\
489	0\\
490	0\\
491	0\\
492	0\\
493	0\\
494	0\\
495	0\\
496	0\\
497	0\\
498	0\\
499	0\\
500	0\\
501	0\\
502	0\\
503	0\\
504	0\\
505	0\\
506	0\\
507	0\\
508	0\\
509	0\\
510	0\\
511	0\\
512	0\\
513	0\\
514	0\\
515	0\\
516	0\\
517	0\\
518	0\\
519	0\\
520	0\\
521	0\\
522	0\\
523	0\\
524	0\\
525	0\\
526	0\\
527	0\\
528	0\\
529	0\\
530	0\\
531	0\\
532	0\\
533	0\\
534	0\\
535	0\\
536	0\\
537	0\\
538	1.6243643999888e-05\\
539	4.18918417134826e-05\\
540	6.80899130433921e-05\\
541	9.48564491413589e-05\\
542	0.000122248418487188\\
543	0.000150198053260629\\
544	0.000178739275558913\\
545	0.000207878949442831\\
546	0.000237642445840086\\
547	0.000268066066433872\\
548	0.000299164768053091\\
549	0.000330937123302737\\
550	0.000363440569240543\\
551	0.000396699448537722\\
552	0.000430733142917223\\
553	0.000465742743696337\\
554	0.000501529298393566\\
555	0.000537988092118647\\
556	0.000575079712924593\\
557	0.000612943047238721\\
558	0.000651541350800113\\
559	0.000690963850311138\\
560	0.000731349768291028\\
561	0.000773468762908814\\
562	0.000816756500267696\\
563	0.000860107264034101\\
564	0.000903803885929959\\
565	0.000947716400417519\\
566	0.000992275947024247\\
567	0.00103793439844082\\
568	0.00108481900650592\\
569	0.00130432755340724\\
570	0.00164623612034606\\
571	0.00190410168763289\\
572	0.00206499561851727\\
573	0.00213820104703062\\
574	0.00220835825523849\\
575	0.00227946097953584\\
576	0.00235181287951247\\
577	0.00242548455611651\\
578	0.0025005266540219\\
579	0.00257698800197174\\
580	0.00265491768689218\\
581	0.0027343675196084\\
582	0.00281539243421504\\
583	0.00289805091793385\\
584	0.00298240551503779\\
585	0.00306852370283625\\
586	0.00315647968906845\\
587	0.00324635875816935\\
588	0.00333826830902959\\
589	0.00343236680134408\\
590	0.00352894054086422\\
591	0.00362860835901881\\
592	0.00373286830485375\\
593	0.00384555904940006\\
594	0.00397672044113206\\
595	0.00415280575412961\\
596	0.00444436306303141\\
597	0.00503983077166121\\
598	0.00644286460810295\\
599	0\\
600	0\\
};
\addplot [color=mycolor21,solid,forget plot]
  table[row sep=crcr]{%
1	0\\
2	0\\
3	0\\
4	0\\
5	0\\
6	0\\
7	0\\
8	0\\
9	0\\
10	0\\
11	0\\
12	0\\
13	0\\
14	0\\
15	0\\
16	0\\
17	0\\
18	0\\
19	0\\
20	0\\
21	0\\
22	0\\
23	0\\
24	0\\
25	0\\
26	0\\
27	0\\
28	0\\
29	0\\
30	0\\
31	0\\
32	0\\
33	0\\
34	0\\
35	0\\
36	0\\
37	0\\
38	0\\
39	0\\
40	0\\
41	0\\
42	0\\
43	0\\
44	0\\
45	0\\
46	0\\
47	0\\
48	0\\
49	0\\
50	0\\
51	0\\
52	0\\
53	0\\
54	0\\
55	0\\
56	0\\
57	0\\
58	0\\
59	0\\
60	0\\
61	0\\
62	0\\
63	0\\
64	0\\
65	0\\
66	0\\
67	0\\
68	0\\
69	0\\
70	0\\
71	0\\
72	0\\
73	0\\
74	0\\
75	0\\
76	0\\
77	0\\
78	0\\
79	0\\
80	0\\
81	0\\
82	0\\
83	0\\
84	0\\
85	0\\
86	0\\
87	0\\
88	0\\
89	0\\
90	0\\
91	0\\
92	0\\
93	0\\
94	0\\
95	0\\
96	0\\
97	0\\
98	0\\
99	0\\
100	0\\
101	0\\
102	0\\
103	0\\
104	0\\
105	0\\
106	0\\
107	0\\
108	0\\
109	0\\
110	0\\
111	0\\
112	0\\
113	0\\
114	0\\
115	0\\
116	0\\
117	0\\
118	0\\
119	0\\
120	0\\
121	0\\
122	0\\
123	0\\
124	0\\
125	0\\
126	0\\
127	0\\
128	0\\
129	0\\
130	0\\
131	0\\
132	0\\
133	0\\
134	0\\
135	0\\
136	0\\
137	0\\
138	0\\
139	0\\
140	0\\
141	0\\
142	0\\
143	0\\
144	0\\
145	0\\
146	0\\
147	0\\
148	0\\
149	0\\
150	0\\
151	0\\
152	0\\
153	0\\
154	0\\
155	0\\
156	0\\
157	0\\
158	0\\
159	0\\
160	0\\
161	0\\
162	0\\
163	0\\
164	0\\
165	0\\
166	0\\
167	0\\
168	0\\
169	0\\
170	0\\
171	0\\
172	0\\
173	0\\
174	0\\
175	0\\
176	0\\
177	0\\
178	0\\
179	0\\
180	0\\
181	0\\
182	0\\
183	0\\
184	0\\
185	0\\
186	0\\
187	0\\
188	0\\
189	0\\
190	0\\
191	0\\
192	0\\
193	0\\
194	0\\
195	0\\
196	0\\
197	0\\
198	0\\
199	0\\
200	0\\
201	0\\
202	0\\
203	0\\
204	0\\
205	0\\
206	0\\
207	0\\
208	0\\
209	0\\
210	0\\
211	0\\
212	0\\
213	0\\
214	0\\
215	0\\
216	0\\
217	0\\
218	0\\
219	0\\
220	0\\
221	0\\
222	0\\
223	0\\
224	0\\
225	0\\
226	0\\
227	0\\
228	0\\
229	0\\
230	0\\
231	0\\
232	0\\
233	0\\
234	0\\
235	0\\
236	0\\
237	0\\
238	0\\
239	0\\
240	0\\
241	0\\
242	0\\
243	0\\
244	0\\
245	0\\
246	0\\
247	0\\
248	0\\
249	0\\
250	0\\
251	0\\
252	0\\
253	0\\
254	0\\
255	0\\
256	0\\
257	0\\
258	0\\
259	0\\
260	0\\
261	0\\
262	0\\
263	0\\
264	0\\
265	0\\
266	0\\
267	0\\
268	0\\
269	0\\
270	0\\
271	0\\
272	0\\
273	0\\
274	0\\
275	0\\
276	0\\
277	0\\
278	0\\
279	0\\
280	0\\
281	0\\
282	0\\
283	0\\
284	0\\
285	0\\
286	0\\
287	0\\
288	0\\
289	0\\
290	0\\
291	0\\
292	0\\
293	0\\
294	0\\
295	0\\
296	0\\
297	0\\
298	0\\
299	0\\
300	0\\
301	0\\
302	0\\
303	0\\
304	0\\
305	0\\
306	0\\
307	0\\
308	0\\
309	0\\
310	0\\
311	0\\
312	0\\
313	0\\
314	0\\
315	0\\
316	0\\
317	0\\
318	0\\
319	0\\
320	0\\
321	0\\
322	0\\
323	0\\
324	0\\
325	0\\
326	0\\
327	0\\
328	0\\
329	0\\
330	0\\
331	0\\
332	0\\
333	0\\
334	0\\
335	0\\
336	0\\
337	0\\
338	0\\
339	0\\
340	0\\
341	0\\
342	0\\
343	0\\
344	0\\
345	0\\
346	0\\
347	0\\
348	0\\
349	0\\
350	0\\
351	0\\
352	0\\
353	0\\
354	0\\
355	0\\
356	0\\
357	0\\
358	0\\
359	0\\
360	0\\
361	0\\
362	0\\
363	0\\
364	0\\
365	0\\
366	0\\
367	0\\
368	0\\
369	0\\
370	0\\
371	0\\
372	0\\
373	0\\
374	0\\
375	0\\
376	0\\
377	0\\
378	0\\
379	0\\
380	0\\
381	0\\
382	0\\
383	0\\
384	0\\
385	0\\
386	0\\
387	0\\
388	0\\
389	0\\
390	0\\
391	0\\
392	0\\
393	0\\
394	0\\
395	0\\
396	0\\
397	0\\
398	0\\
399	0\\
400	0\\
401	0\\
402	0\\
403	0\\
404	0\\
405	0\\
406	0\\
407	0\\
408	0\\
409	0\\
410	0\\
411	0\\
412	0\\
413	0\\
414	0\\
415	0\\
416	0\\
417	0\\
418	0\\
419	0\\
420	0\\
421	0\\
422	0\\
423	0\\
424	0\\
425	0\\
426	0\\
427	0\\
428	0\\
429	0\\
430	0\\
431	0\\
432	0\\
433	0\\
434	0\\
435	0\\
436	0\\
437	0\\
438	0\\
439	0\\
440	0\\
441	0\\
442	0\\
443	0\\
444	0\\
445	0\\
446	0\\
447	0\\
448	0\\
449	0\\
450	0\\
451	0\\
452	0\\
453	0\\
454	0\\
455	0\\
456	0\\
457	0\\
458	0\\
459	0\\
460	0\\
461	0\\
462	0\\
463	0\\
464	0\\
465	0\\
466	0\\
467	0\\
468	0\\
469	0\\
470	0\\
471	0\\
472	0\\
473	0\\
474	0\\
475	0\\
476	0\\
477	0\\
478	0\\
479	0\\
480	0\\
481	0\\
482	0\\
483	0\\
484	0\\
485	0\\
486	0\\
487	0\\
488	0\\
489	0\\
490	0\\
491	0\\
492	0\\
493	0\\
494	0\\
495	0\\
496	0\\
497	0\\
498	0\\
499	0\\
500	0\\
501	0\\
502	0\\
503	0\\
504	0\\
505	0\\
506	0\\
507	0\\
508	0\\
509	0\\
510	0\\
511	0\\
512	0\\
513	0\\
514	0\\
515	0\\
516	0\\
517	0\\
518	0\\
519	0\\
520	0\\
521	0\\
522	0\\
523	0\\
524	0\\
525	0\\
526	0\\
527	0\\
528	0\\
529	0\\
530	0\\
531	0\\
532	0\\
533	0\\
534	0\\
535	0\\
536	0\\
537	3.82550301366919e-06\\
538	2.9106131297474e-05\\
539	5.49338526053371e-05\\
540	8.12709214509474e-05\\
541	0.000108136282402065\\
542	0.000135519390228108\\
543	0.000163492059064573\\
544	0.000192077027657505\\
545	0.000221277232734906\\
546	0.000251125248797619\\
547	0.00028165907332523\\
548	0.000312894077851403\\
549	0.000344852457097526\\
550	0.000377575892313951\\
551	0.000411272352801193\\
552	0.000445600411375425\\
553	0.000480493797716272\\
554	0.000516041013830302\\
555	0.000552284502233579\\
556	0.000589200581950451\\
557	0.000626998320133113\\
558	0.000665716783444826\\
559	0.000705887875314932\\
560	0.000747561296246264\\
561	0.000789266191669505\\
562	0.000831174369496882\\
563	0.000873178095902131\\
564	0.000915807659414534\\
565	0.000959488231720514\\
566	0.00100432096429517\\
567	0.0010503705610188\\
568	0.00138555014159372\\
569	0.00168867149361451\\
570	0.00192455116100545\\
571	0.00200205571737336\\
572	0.00207001668793317\\
573	0.00213863941240985\\
574	0.00220843933480784\\
575	0.00227948582459503\\
576	0.00235182128121044\\
577	0.00242548747819655\\
578	0.00250052798438547\\
579	0.00257698863349794\\
580	0.0026549180091809\\
581	0.0027343676902359\\
582	0.0028153925248352\\
583	0.00289805096238546\\
584	0.00298240553557085\\
585	0.00306852370928107\\
586	0.00315647969018814\\
587	0.00324635875816936\\
588	0.0033382683090296\\
589	0.00343236680134408\\
590	0.00352894054086422\\
591	0.00362860835901882\\
592	0.00373286830485375\\
593	0.00384555904940006\\
594	0.00397672044113206\\
595	0.00415280575412961\\
596	0.00444436306303141\\
597	0.00503983077166121\\
598	0.00644286460810295\\
599	0\\
600	0\\
};
\addplot [color=black!20!mycolor21,solid,forget plot]
  table[row sep=crcr]{%
1	0\\
2	0\\
3	0\\
4	0\\
5	0\\
6	0\\
7	0\\
8	0\\
9	0\\
10	0\\
11	0\\
12	0\\
13	0\\
14	0\\
15	0\\
16	0\\
17	0\\
18	0\\
19	0\\
20	0\\
21	0\\
22	0\\
23	0\\
24	0\\
25	0\\
26	0\\
27	0\\
28	0\\
29	0\\
30	0\\
31	0\\
32	0\\
33	0\\
34	0\\
35	0\\
36	0\\
37	0\\
38	0\\
39	0\\
40	0\\
41	0\\
42	0\\
43	0\\
44	0\\
45	0\\
46	0\\
47	0\\
48	0\\
49	0\\
50	0\\
51	0\\
52	0\\
53	0\\
54	0\\
55	0\\
56	0\\
57	0\\
58	0\\
59	0\\
60	0\\
61	0\\
62	0\\
63	0\\
64	0\\
65	0\\
66	0\\
67	0\\
68	0\\
69	0\\
70	0\\
71	0\\
72	0\\
73	0\\
74	0\\
75	0\\
76	0\\
77	0\\
78	0\\
79	0\\
80	0\\
81	0\\
82	0\\
83	0\\
84	0\\
85	0\\
86	0\\
87	0\\
88	0\\
89	0\\
90	0\\
91	0\\
92	0\\
93	0\\
94	0\\
95	0\\
96	0\\
97	0\\
98	0\\
99	0\\
100	0\\
101	0\\
102	0\\
103	0\\
104	0\\
105	0\\
106	0\\
107	0\\
108	0\\
109	0\\
110	0\\
111	0\\
112	0\\
113	0\\
114	0\\
115	0\\
116	0\\
117	0\\
118	0\\
119	0\\
120	0\\
121	0\\
122	0\\
123	0\\
124	0\\
125	0\\
126	0\\
127	0\\
128	0\\
129	0\\
130	0\\
131	0\\
132	0\\
133	0\\
134	0\\
135	0\\
136	0\\
137	0\\
138	0\\
139	0\\
140	0\\
141	0\\
142	0\\
143	0\\
144	0\\
145	0\\
146	0\\
147	0\\
148	0\\
149	0\\
150	0\\
151	0\\
152	0\\
153	0\\
154	0\\
155	0\\
156	0\\
157	0\\
158	0\\
159	0\\
160	0\\
161	0\\
162	0\\
163	0\\
164	0\\
165	0\\
166	0\\
167	0\\
168	0\\
169	0\\
170	0\\
171	0\\
172	0\\
173	0\\
174	0\\
175	0\\
176	0\\
177	0\\
178	0\\
179	0\\
180	0\\
181	0\\
182	0\\
183	0\\
184	0\\
185	0\\
186	0\\
187	0\\
188	0\\
189	0\\
190	0\\
191	0\\
192	0\\
193	0\\
194	0\\
195	0\\
196	0\\
197	0\\
198	0\\
199	0\\
200	0\\
201	0\\
202	0\\
203	0\\
204	0\\
205	0\\
206	0\\
207	0\\
208	0\\
209	0\\
210	0\\
211	0\\
212	0\\
213	0\\
214	0\\
215	0\\
216	0\\
217	0\\
218	0\\
219	0\\
220	0\\
221	0\\
222	0\\
223	0\\
224	0\\
225	0\\
226	0\\
227	0\\
228	0\\
229	0\\
230	0\\
231	0\\
232	0\\
233	0\\
234	0\\
235	0\\
236	0\\
237	0\\
238	0\\
239	0\\
240	0\\
241	0\\
242	0\\
243	0\\
244	0\\
245	0\\
246	0\\
247	0\\
248	0\\
249	0\\
250	0\\
251	0\\
252	0\\
253	0\\
254	0\\
255	0\\
256	0\\
257	0\\
258	0\\
259	0\\
260	0\\
261	0\\
262	0\\
263	0\\
264	0\\
265	0\\
266	0\\
267	0\\
268	0\\
269	0\\
270	0\\
271	0\\
272	0\\
273	0\\
274	0\\
275	0\\
276	0\\
277	0\\
278	0\\
279	0\\
280	0\\
281	0\\
282	0\\
283	0\\
284	0\\
285	0\\
286	0\\
287	0\\
288	0\\
289	0\\
290	0\\
291	0\\
292	0\\
293	0\\
294	0\\
295	0\\
296	0\\
297	0\\
298	0\\
299	0\\
300	0\\
301	0\\
302	0\\
303	0\\
304	0\\
305	0\\
306	0\\
307	0\\
308	0\\
309	0\\
310	0\\
311	0\\
312	0\\
313	0\\
314	0\\
315	0\\
316	0\\
317	0\\
318	0\\
319	0\\
320	0\\
321	0\\
322	0\\
323	0\\
324	0\\
325	0\\
326	0\\
327	0\\
328	0\\
329	0\\
330	0\\
331	0\\
332	0\\
333	0\\
334	0\\
335	0\\
336	0\\
337	0\\
338	0\\
339	0\\
340	0\\
341	0\\
342	0\\
343	0\\
344	0\\
345	0\\
346	0\\
347	0\\
348	0\\
349	0\\
350	0\\
351	0\\
352	0\\
353	0\\
354	0\\
355	0\\
356	0\\
357	0\\
358	0\\
359	0\\
360	0\\
361	0\\
362	0\\
363	0\\
364	0\\
365	0\\
366	0\\
367	0\\
368	0\\
369	0\\
370	0\\
371	0\\
372	0\\
373	0\\
374	0\\
375	0\\
376	0\\
377	0\\
378	0\\
379	0\\
380	0\\
381	0\\
382	0\\
383	0\\
384	0\\
385	0\\
386	0\\
387	0\\
388	0\\
389	0\\
390	0\\
391	0\\
392	0\\
393	0\\
394	0\\
395	0\\
396	0\\
397	0\\
398	0\\
399	0\\
400	0\\
401	0\\
402	0\\
403	0\\
404	0\\
405	0\\
406	0\\
407	0\\
408	0\\
409	0\\
410	0\\
411	0\\
412	0\\
413	0\\
414	0\\
415	0\\
416	0\\
417	0\\
418	0\\
419	0\\
420	0\\
421	0\\
422	0\\
423	0\\
424	0\\
425	0\\
426	0\\
427	0\\
428	0\\
429	0\\
430	0\\
431	0\\
432	0\\
433	0\\
434	0\\
435	0\\
436	0\\
437	0\\
438	0\\
439	0\\
440	0\\
441	0\\
442	0\\
443	0\\
444	0\\
445	0\\
446	0\\
447	0\\
448	0\\
449	0\\
450	0\\
451	0\\
452	0\\
453	0\\
454	0\\
455	0\\
456	0\\
457	0\\
458	0\\
459	0\\
460	0\\
461	0\\
462	0\\
463	0\\
464	0\\
465	0\\
466	0\\
467	0\\
468	0\\
469	0\\
470	0\\
471	0\\
472	0\\
473	0\\
474	0\\
475	0\\
476	0\\
477	0\\
478	0\\
479	0\\
480	0\\
481	0\\
482	0\\
483	0\\
484	0\\
485	0\\
486	0\\
487	0\\
488	0\\
489	0\\
490	0\\
491	0\\
492	0\\
493	0\\
494	0\\
495	0\\
496	0\\
497	0\\
498	0\\
499	0\\
500	0\\
501	0\\
502	0\\
503	0\\
504	0\\
505	0\\
506	0\\
507	0\\
508	0\\
509	0\\
510	0\\
511	0\\
512	0\\
513	0\\
514	0\\
515	0\\
516	0\\
517	0\\
518	0\\
519	0\\
520	0\\
521	0\\
522	0\\
523	0\\
524	0\\
525	0\\
526	0\\
527	0\\
528	0\\
529	0\\
530	0\\
531	0\\
532	0\\
533	0\\
534	0\\
535	0\\
536	0\\
537	1.5257515615078e-05\\
538	4.05712495559513e-05\\
539	6.63714757039172e-05\\
540	9.27016742245395e-05\\
541	0.000119575710927847\\
542	0.000147000013995511\\
543	0.00017504869257464\\
544	0.000203735987885211\\
545	0.000233079425442273\\
546	0.000263096693183489\\
547	0.000293803947281356\\
548	0.000325302371137699\\
549	0.000357642385253218\\
550	0.000390582931632927\\
551	0.00042399701337393\\
552	0.000458068777525668\\
553	0.000492727463226109\\
554	0.000528135608302436\\
555	0.000564380253208309\\
556	0.000601502109540624\\
557	0.000639665961938304\\
558	0.000679881186959353\\
559	0.000720197887011426\\
560	0.000760489350549704\\
561	0.00080084928102869\\
562	0.000841683425308702\\
563	0.000883510817814942\\
564	0.000926424297840563\\
565	0.000970477398949334\\
566	0.00106843923578318\\
567	0.00144259009169947\\
568	0.00171200842408046\\
569	0.0018698844738892\\
570	0.00193635519907031\\
571	0.00200267110891418\\
572	0.00207007254415696\\
573	0.00213865022157796\\
574	0.00220844272080478\\
575	0.00227948699417363\\
576	0.0023518217070068\\
577	0.0024254876758363\\
578	0.00250052808022166\\
579	0.0025769886830341\\
580	0.00265491803552198\\
581	0.00273436770413291\\
582	0.00281539253160681\\
583	0.0028980509654155\\
584	0.00298240553649866\\
585	0.00306852370943816\\
586	0.00315647969018814\\
587	0.00324635875816936\\
588	0.0033382683090296\\
589	0.00343236680134408\\
590	0.00352894054086423\\
591	0.00362860835901881\\
592	0.00373286830485375\\
593	0.00384555904940006\\
594	0.00397672044113205\\
595	0.00415280575412962\\
596	0.00444436306303141\\
597	0.00503983077166121\\
598	0.00644286460810295\\
599	0\\
600	0\\
};
\addplot [color=black!50!mycolor20,solid,forget plot]
  table[row sep=crcr]{%
1	0\\
2	0\\
3	0\\
4	0\\
5	0\\
6	0\\
7	0\\
8	0\\
9	0\\
10	0\\
11	0\\
12	0\\
13	0\\
14	0\\
15	0\\
16	0\\
17	0\\
18	0\\
19	0\\
20	0\\
21	0\\
22	0\\
23	0\\
24	0\\
25	0\\
26	0\\
27	0\\
28	0\\
29	0\\
30	0\\
31	0\\
32	0\\
33	0\\
34	0\\
35	0\\
36	0\\
37	0\\
38	0\\
39	0\\
40	0\\
41	0\\
42	0\\
43	0\\
44	0\\
45	0\\
46	0\\
47	0\\
48	0\\
49	0\\
50	0\\
51	0\\
52	0\\
53	0\\
54	0\\
55	0\\
56	0\\
57	0\\
58	0\\
59	0\\
60	0\\
61	0\\
62	0\\
63	0\\
64	0\\
65	0\\
66	0\\
67	0\\
68	0\\
69	0\\
70	0\\
71	0\\
72	0\\
73	0\\
74	0\\
75	0\\
76	0\\
77	0\\
78	0\\
79	0\\
80	0\\
81	0\\
82	0\\
83	0\\
84	0\\
85	0\\
86	0\\
87	0\\
88	0\\
89	0\\
90	0\\
91	0\\
92	0\\
93	0\\
94	0\\
95	0\\
96	0\\
97	0\\
98	0\\
99	0\\
100	0\\
101	0\\
102	0\\
103	0\\
104	0\\
105	0\\
106	0\\
107	0\\
108	0\\
109	0\\
110	0\\
111	0\\
112	0\\
113	0\\
114	0\\
115	0\\
116	0\\
117	0\\
118	0\\
119	0\\
120	0\\
121	0\\
122	0\\
123	0\\
124	0\\
125	0\\
126	0\\
127	0\\
128	0\\
129	0\\
130	0\\
131	0\\
132	0\\
133	0\\
134	0\\
135	0\\
136	0\\
137	0\\
138	0\\
139	0\\
140	0\\
141	0\\
142	0\\
143	0\\
144	0\\
145	0\\
146	0\\
147	0\\
148	0\\
149	0\\
150	0\\
151	0\\
152	0\\
153	0\\
154	0\\
155	0\\
156	0\\
157	0\\
158	0\\
159	0\\
160	0\\
161	0\\
162	0\\
163	0\\
164	0\\
165	0\\
166	0\\
167	0\\
168	0\\
169	0\\
170	0\\
171	0\\
172	0\\
173	0\\
174	0\\
175	0\\
176	0\\
177	0\\
178	0\\
179	0\\
180	0\\
181	0\\
182	0\\
183	0\\
184	0\\
185	0\\
186	0\\
187	0\\
188	0\\
189	0\\
190	0\\
191	0\\
192	0\\
193	0\\
194	0\\
195	0\\
196	0\\
197	0\\
198	0\\
199	0\\
200	0\\
201	0\\
202	0\\
203	0\\
204	0\\
205	0\\
206	0\\
207	0\\
208	0\\
209	0\\
210	0\\
211	0\\
212	0\\
213	0\\
214	0\\
215	0\\
216	0\\
217	0\\
218	0\\
219	0\\
220	0\\
221	0\\
222	0\\
223	0\\
224	0\\
225	0\\
226	0\\
227	0\\
228	0\\
229	0\\
230	0\\
231	0\\
232	0\\
233	0\\
234	0\\
235	0\\
236	0\\
237	0\\
238	0\\
239	0\\
240	0\\
241	0\\
242	0\\
243	0\\
244	0\\
245	0\\
246	0\\
247	0\\
248	0\\
249	0\\
250	0\\
251	0\\
252	0\\
253	0\\
254	0\\
255	0\\
256	0\\
257	0\\
258	0\\
259	0\\
260	0\\
261	0\\
262	0\\
263	0\\
264	0\\
265	0\\
266	0\\
267	0\\
268	0\\
269	0\\
270	0\\
271	0\\
272	0\\
273	0\\
274	0\\
275	0\\
276	0\\
277	0\\
278	0\\
279	0\\
280	0\\
281	0\\
282	0\\
283	0\\
284	0\\
285	0\\
286	0\\
287	0\\
288	0\\
289	0\\
290	0\\
291	0\\
292	0\\
293	0\\
294	0\\
295	0\\
296	0\\
297	0\\
298	0\\
299	0\\
300	0\\
301	0\\
302	0\\
303	0\\
304	0\\
305	0\\
306	0\\
307	0\\
308	0\\
309	0\\
310	0\\
311	0\\
312	0\\
313	0\\
314	0\\
315	0\\
316	0\\
317	0\\
318	0\\
319	0\\
320	0\\
321	0\\
322	0\\
323	0\\
324	0\\
325	0\\
326	0\\
327	0\\
328	0\\
329	0\\
330	0\\
331	0\\
332	0\\
333	0\\
334	0\\
335	0\\
336	0\\
337	0\\
338	0\\
339	0\\
340	0\\
341	0\\
342	0\\
343	0\\
344	0\\
345	0\\
346	0\\
347	0\\
348	0\\
349	0\\
350	0\\
351	0\\
352	0\\
353	0\\
354	0\\
355	0\\
356	0\\
357	0\\
358	0\\
359	0\\
360	0\\
361	0\\
362	0\\
363	0\\
364	0\\
365	0\\
366	0\\
367	0\\
368	0\\
369	0\\
370	0\\
371	0\\
372	0\\
373	0\\
374	0\\
375	0\\
376	0\\
377	0\\
378	0\\
379	0\\
380	0\\
381	0\\
382	0\\
383	0\\
384	0\\
385	0\\
386	0\\
387	0\\
388	0\\
389	0\\
390	0\\
391	0\\
392	0\\
393	0\\
394	0\\
395	0\\
396	0\\
397	0\\
398	0\\
399	0\\
400	0\\
401	0\\
402	0\\
403	0\\
404	0\\
405	0\\
406	0\\
407	0\\
408	0\\
409	0\\
410	0\\
411	0\\
412	0\\
413	0\\
414	0\\
415	0\\
416	0\\
417	0\\
418	0\\
419	0\\
420	0\\
421	0\\
422	0\\
423	0\\
424	0\\
425	0\\
426	0\\
427	0\\
428	0\\
429	0\\
430	0\\
431	0\\
432	0\\
433	0\\
434	0\\
435	0\\
436	0\\
437	0\\
438	0\\
439	0\\
440	0\\
441	0\\
442	0\\
443	0\\
444	0\\
445	0\\
446	0\\
447	0\\
448	0\\
449	0\\
450	0\\
451	0\\
452	0\\
453	0\\
454	0\\
455	0\\
456	0\\
457	0\\
458	0\\
459	0\\
460	0\\
461	0\\
462	0\\
463	0\\
464	0\\
465	0\\
466	0\\
467	0\\
468	0\\
469	0\\
470	0\\
471	0\\
472	0\\
473	0\\
474	0\\
475	0\\
476	0\\
477	0\\
478	0\\
479	0\\
480	0\\
481	0\\
482	0\\
483	0\\
484	0\\
485	0\\
486	0\\
487	0\\
488	0\\
489	0\\
490	0\\
491	0\\
492	0\\
493	0\\
494	0\\
495	0\\
496	0\\
497	0\\
498	0\\
499	0\\
500	0\\
501	0\\
502	0\\
503	0\\
504	0\\
505	0\\
506	0\\
507	0\\
508	0\\
509	0\\
510	0\\
511	0\\
512	0\\
513	0\\
514	0\\
515	0\\
516	0\\
517	0\\
518	0\\
519	0\\
520	0\\
521	0\\
522	0\\
523	0\\
524	0\\
525	0\\
526	0\\
527	0\\
528	0\\
529	0\\
530	0\\
531	0\\
532	0\\
533	0\\
534	0\\
535	0\\
536	3.78359368673223e-07\\
537	2.51996931176172e-05\\
538	5.05127313110625e-05\\
539	7.6341234946441e-05\\
540	0.000102729583218711\\
541	0.000129693335033606\\
542	0.000157267192673928\\
543	0.000185470106681217\\
544	0.000214316398288709\\
545	0.000243821524482409\\
546	0.000274119909113173\\
547	0.000305191906140328\\
548	0.00033677287195165\\
549	0.000368831234562262\\
550	0.000401484922657279\\
551	0.000434681869006069\\
552	0.00046864575352553\\
553	0.000503409758869739\\
554	0.000539018952998663\\
555	0.000575501755643469\\
556	0.000613762678061962\\
557	0.000652925523529455\\
558	0.000691742840127115\\
559	0.000730684352716771\\
560	0.000769821722848036\\
561	0.000809880072409532\\
562	0.000850959223737115\\
563	0.000893107882836268\\
564	0.00093637240331507\\
565	0.00109010374729446\\
566	0.00145489706217203\\
567	0.0017205761280254\\
568	0.00180709087821166\\
569	0.00187129168921907\\
570	0.0019364300875453\\
571	0.00200267818666582\\
572	0.00207007397726512\\
573	0.00213865068083015\\
574	0.00220844288323674\\
575	0.00227948705593812\\
576	0.00235182173622467\\
577	0.00242548769027979\\
578	0.00250052808776776\\
579	0.00257698868705665\\
580	0.00265491803762909\\
581	0.00273436770515037\\
582	0.00281539253204859\\
583	0.00289805096554754\\
584	0.00298240553652045\\
585	0.00306852370943815\\
586	0.00315647969018813\\
587	0.00324635875816935\\
588	0.0033382683090296\\
589	0.00343236680134408\\
590	0.00352894054086423\\
591	0.00362860835901881\\
592	0.00373286830485375\\
593	0.00384555904940006\\
594	0.00397672044113205\\
595	0.00415280575412962\\
596	0.00444436306303141\\
597	0.00503983077166121\\
598	0.00644286460810295\\
599	0\\
600	0\\
};
\addplot [color=black!60!mycolor21,solid,forget plot]
  table[row sep=crcr]{%
1	0\\
2	0\\
3	0\\
4	0\\
5	0\\
6	0\\
7	0\\
8	0\\
9	0\\
10	0\\
11	0\\
12	0\\
13	0\\
14	0\\
15	0\\
16	0\\
17	0\\
18	0\\
19	0\\
20	0\\
21	0\\
22	0\\
23	0\\
24	0\\
25	0\\
26	0\\
27	0\\
28	0\\
29	0\\
30	0\\
31	0\\
32	0\\
33	0\\
34	0\\
35	0\\
36	0\\
37	0\\
38	0\\
39	0\\
40	0\\
41	0\\
42	0\\
43	0\\
44	0\\
45	0\\
46	0\\
47	0\\
48	0\\
49	0\\
50	0\\
51	0\\
52	0\\
53	0\\
54	0\\
55	0\\
56	0\\
57	0\\
58	0\\
59	0\\
60	0\\
61	0\\
62	0\\
63	0\\
64	0\\
65	0\\
66	0\\
67	0\\
68	0\\
69	0\\
70	0\\
71	0\\
72	0\\
73	0\\
74	0\\
75	0\\
76	0\\
77	0\\
78	0\\
79	0\\
80	0\\
81	0\\
82	0\\
83	0\\
84	0\\
85	0\\
86	0\\
87	0\\
88	0\\
89	0\\
90	0\\
91	0\\
92	0\\
93	0\\
94	0\\
95	0\\
96	0\\
97	0\\
98	0\\
99	0\\
100	0\\
101	0\\
102	0\\
103	0\\
104	0\\
105	0\\
106	0\\
107	0\\
108	0\\
109	0\\
110	0\\
111	0\\
112	0\\
113	0\\
114	0\\
115	0\\
116	0\\
117	0\\
118	0\\
119	0\\
120	0\\
121	0\\
122	0\\
123	0\\
124	0\\
125	0\\
126	0\\
127	0\\
128	0\\
129	0\\
130	0\\
131	0\\
132	0\\
133	0\\
134	0\\
135	0\\
136	0\\
137	0\\
138	0\\
139	0\\
140	0\\
141	0\\
142	0\\
143	0\\
144	0\\
145	0\\
146	0\\
147	0\\
148	0\\
149	0\\
150	0\\
151	0\\
152	0\\
153	0\\
154	0\\
155	0\\
156	0\\
157	0\\
158	0\\
159	0\\
160	0\\
161	0\\
162	0\\
163	0\\
164	0\\
165	0\\
166	0\\
167	0\\
168	0\\
169	0\\
170	0\\
171	0\\
172	0\\
173	0\\
174	0\\
175	0\\
176	0\\
177	0\\
178	0\\
179	0\\
180	0\\
181	0\\
182	0\\
183	0\\
184	0\\
185	0\\
186	0\\
187	0\\
188	0\\
189	0\\
190	0\\
191	0\\
192	0\\
193	0\\
194	0\\
195	0\\
196	0\\
197	0\\
198	0\\
199	0\\
200	0\\
201	0\\
202	0\\
203	0\\
204	0\\
205	0\\
206	0\\
207	0\\
208	0\\
209	0\\
210	0\\
211	0\\
212	0\\
213	0\\
214	0\\
215	0\\
216	0\\
217	0\\
218	0\\
219	0\\
220	0\\
221	0\\
222	0\\
223	0\\
224	0\\
225	0\\
226	0\\
227	0\\
228	0\\
229	0\\
230	0\\
231	0\\
232	0\\
233	0\\
234	0\\
235	0\\
236	0\\
237	0\\
238	0\\
239	0\\
240	0\\
241	0\\
242	0\\
243	0\\
244	0\\
245	0\\
246	0\\
247	0\\
248	0\\
249	0\\
250	0\\
251	0\\
252	0\\
253	0\\
254	0\\
255	0\\
256	0\\
257	0\\
258	0\\
259	0\\
260	0\\
261	0\\
262	0\\
263	0\\
264	0\\
265	0\\
266	0\\
267	0\\
268	0\\
269	0\\
270	0\\
271	0\\
272	0\\
273	0\\
274	0\\
275	0\\
276	0\\
277	0\\
278	0\\
279	0\\
280	0\\
281	0\\
282	0\\
283	0\\
284	0\\
285	0\\
286	0\\
287	0\\
288	0\\
289	0\\
290	0\\
291	0\\
292	0\\
293	0\\
294	0\\
295	0\\
296	0\\
297	0\\
298	0\\
299	0\\
300	0\\
301	0\\
302	0\\
303	0\\
304	0\\
305	0\\
306	0\\
307	0\\
308	0\\
309	0\\
310	0\\
311	0\\
312	0\\
313	0\\
314	0\\
315	0\\
316	0\\
317	0\\
318	0\\
319	0\\
320	0\\
321	0\\
322	0\\
323	0\\
324	0\\
325	0\\
326	0\\
327	0\\
328	0\\
329	0\\
330	0\\
331	0\\
332	0\\
333	0\\
334	0\\
335	0\\
336	0\\
337	0\\
338	0\\
339	0\\
340	0\\
341	0\\
342	0\\
343	0\\
344	0\\
345	0\\
346	0\\
347	0\\
348	0\\
349	0\\
350	0\\
351	0\\
352	0\\
353	0\\
354	0\\
355	0\\
356	0\\
357	0\\
358	0\\
359	0\\
360	0\\
361	0\\
362	0\\
363	0\\
364	0\\
365	0\\
366	0\\
367	0\\
368	0\\
369	0\\
370	0\\
371	0\\
372	0\\
373	0\\
374	0\\
375	0\\
376	0\\
377	0\\
378	0\\
379	0\\
380	0\\
381	0\\
382	0\\
383	0\\
384	0\\
385	0\\
386	0\\
387	0\\
388	0\\
389	0\\
390	0\\
391	0\\
392	0\\
393	0\\
394	0\\
395	0\\
396	0\\
397	0\\
398	0\\
399	0\\
400	0\\
401	0\\
402	0\\
403	0\\
404	0\\
405	0\\
406	0\\
407	0\\
408	0\\
409	0\\
410	0\\
411	0\\
412	0\\
413	0\\
414	0\\
415	0\\
416	0\\
417	0\\
418	0\\
419	0\\
420	0\\
421	0\\
422	0\\
423	0\\
424	0\\
425	0\\
426	0\\
427	0\\
428	0\\
429	0\\
430	0\\
431	0\\
432	0\\
433	0\\
434	0\\
435	0\\
436	0\\
437	0\\
438	0\\
439	0\\
440	0\\
441	0\\
442	0\\
443	0\\
444	0\\
445	0\\
446	0\\
447	0\\
448	0\\
449	0\\
450	0\\
451	0\\
452	0\\
453	0\\
454	0\\
455	0\\
456	0\\
457	0\\
458	0\\
459	0\\
460	0\\
461	0\\
462	0\\
463	0\\
464	0\\
465	0\\
466	0\\
467	0\\
468	0\\
469	0\\
470	0\\
471	0\\
472	0\\
473	0\\
474	0\\
475	0\\
476	0\\
477	0\\
478	0\\
479	0\\
480	0\\
481	0\\
482	0\\
483	0\\
484	0\\
485	0\\
486	0\\
487	0\\
488	0\\
489	0\\
490	0\\
491	0\\
492	0\\
493	0\\
494	0\\
495	0\\
496	0\\
497	0\\
498	0\\
499	0\\
500	0\\
501	0\\
502	0\\
503	0\\
504	0\\
505	0\\
506	0\\
507	0\\
508	0\\
509	0\\
510	0\\
511	0\\
512	0\\
513	0\\
514	0\\
515	0\\
516	0\\
517	0\\
518	0\\
519	0\\
520	0\\
521	0\\
522	0\\
523	0\\
524	0\\
525	0\\
526	0\\
527	0\\
528	0\\
529	0\\
530	0\\
531	0\\
532	0\\
533	0\\
534	0\\
535	0\\
536	9.02145954336361e-06\\
537	3.38870820114369e-05\\
538	5.92843281107733e-05\\
539	8.52316144653158e-05\\
540	0.000111744299307854\\
541	0.000138841481860631\\
542	0.000166552064353744\\
543	0.00019489091642814\\
544	0.0002240061624105\\
545	0.000253856225130063\\
546	0.000284149213149703\\
547	0.000314894675820076\\
548	0.000346177518049505\\
549	0.000378018067102527\\
550	0.000410577800625992\\
551	0.000443915441746857\\
552	0.00047805631403732\\
553	0.000513026094184202\\
554	0.00054907441425197\\
555	0.000587044911282405\\
556	0.000624783568171093\\
557	0.000662504517028127\\
558	0.000700022606374216\\
559	0.000738375853450751\\
560	0.000777685135972991\\
561	0.000817995895139835\\
562	0.000859350044913817\\
563	0.000901791512692584\\
564	0.00109093692626027\\
565	0.00144213576563695\\
566	0.00167960457611589\\
567	0.00174425603612933\\
568	0.00180725733542332\\
569	0.00187130073891484\\
570	0.00193643097966661\\
571	0.00200267837569042\\
572	0.00207007403928857\\
573	0.00213865070333988\\
574	0.00220844289215589\\
575	0.00227948706023545\\
576	0.00235182173838593\\
577	0.00242548769141891\\
578	0.00250052808837546\\
579	0.00257698868737254\\
580	0.00265491803777995\\
581	0.00273436770521399\\
582	0.00281539253206718\\
583	0.00289805096555054\\
584	0.00298240553652045\\
585	0.00306852370943814\\
586	0.00315647969018814\\
587	0.00324635875816936\\
588	0.0033382683090296\\
589	0.00343236680134407\\
590	0.00352894054086422\\
591	0.00362860835901882\\
592	0.00373286830485375\\
593	0.00384555904940006\\
594	0.00397672044113206\\
595	0.00415280575412962\\
596	0.00444436306303141\\
597	0.00503983077166121\\
598	0.00644286460810295\\
599	0\\
600	0\\
};
\addplot [color=black!80!mycolor21,solid,forget plot]
  table[row sep=crcr]{%
1	0\\
2	0\\
3	0\\
4	0\\
5	0\\
6	0\\
7	0\\
8	0\\
9	0\\
10	0\\
11	0\\
12	0\\
13	0\\
14	0\\
15	0\\
16	0\\
17	0\\
18	0\\
19	0\\
20	0\\
21	0\\
22	0\\
23	0\\
24	0\\
25	0\\
26	0\\
27	0\\
28	0\\
29	0\\
30	0\\
31	0\\
32	0\\
33	0\\
34	0\\
35	0\\
36	0\\
37	0\\
38	0\\
39	0\\
40	0\\
41	0\\
42	0\\
43	0\\
44	0\\
45	0\\
46	0\\
47	0\\
48	0\\
49	0\\
50	0\\
51	0\\
52	0\\
53	0\\
54	0\\
55	0\\
56	0\\
57	0\\
58	0\\
59	0\\
60	0\\
61	0\\
62	0\\
63	0\\
64	0\\
65	0\\
66	0\\
67	0\\
68	0\\
69	0\\
70	0\\
71	0\\
72	0\\
73	0\\
74	0\\
75	0\\
76	0\\
77	0\\
78	0\\
79	0\\
80	0\\
81	0\\
82	0\\
83	0\\
84	0\\
85	0\\
86	0\\
87	0\\
88	0\\
89	0\\
90	0\\
91	0\\
92	0\\
93	0\\
94	0\\
95	0\\
96	0\\
97	0\\
98	0\\
99	0\\
100	0\\
101	0\\
102	0\\
103	0\\
104	0\\
105	0\\
106	0\\
107	0\\
108	0\\
109	0\\
110	0\\
111	0\\
112	0\\
113	0\\
114	0\\
115	0\\
116	0\\
117	0\\
118	0\\
119	0\\
120	0\\
121	0\\
122	0\\
123	0\\
124	0\\
125	0\\
126	0\\
127	0\\
128	0\\
129	0\\
130	0\\
131	0\\
132	0\\
133	0\\
134	0\\
135	0\\
136	0\\
137	0\\
138	0\\
139	0\\
140	0\\
141	0\\
142	0\\
143	0\\
144	0\\
145	0\\
146	0\\
147	0\\
148	0\\
149	0\\
150	0\\
151	0\\
152	0\\
153	0\\
154	0\\
155	0\\
156	0\\
157	0\\
158	0\\
159	0\\
160	0\\
161	0\\
162	0\\
163	0\\
164	0\\
165	0\\
166	0\\
167	0\\
168	0\\
169	0\\
170	0\\
171	0\\
172	0\\
173	0\\
174	0\\
175	0\\
176	0\\
177	0\\
178	0\\
179	0\\
180	0\\
181	0\\
182	0\\
183	0\\
184	0\\
185	0\\
186	0\\
187	0\\
188	0\\
189	0\\
190	0\\
191	0\\
192	0\\
193	0\\
194	0\\
195	0\\
196	0\\
197	0\\
198	0\\
199	0\\
200	0\\
201	0\\
202	0\\
203	0\\
204	0\\
205	0\\
206	0\\
207	0\\
208	0\\
209	0\\
210	0\\
211	0\\
212	0\\
213	0\\
214	0\\
215	0\\
216	0\\
217	0\\
218	0\\
219	0\\
220	0\\
221	0\\
222	0\\
223	0\\
224	0\\
225	0\\
226	0\\
227	0\\
228	0\\
229	0\\
230	0\\
231	0\\
232	0\\
233	0\\
234	0\\
235	0\\
236	0\\
237	0\\
238	0\\
239	0\\
240	0\\
241	0\\
242	0\\
243	0\\
244	0\\
245	0\\
246	0\\
247	0\\
248	0\\
249	0\\
250	0\\
251	0\\
252	0\\
253	0\\
254	0\\
255	0\\
256	0\\
257	0\\
258	0\\
259	0\\
260	0\\
261	0\\
262	0\\
263	0\\
264	0\\
265	0\\
266	0\\
267	0\\
268	0\\
269	0\\
270	0\\
271	0\\
272	0\\
273	0\\
274	0\\
275	0\\
276	0\\
277	0\\
278	0\\
279	0\\
280	0\\
281	0\\
282	0\\
283	0\\
284	0\\
285	0\\
286	0\\
287	0\\
288	0\\
289	0\\
290	0\\
291	0\\
292	0\\
293	0\\
294	0\\
295	0\\
296	0\\
297	0\\
298	0\\
299	0\\
300	0\\
301	0\\
302	0\\
303	0\\
304	0\\
305	0\\
306	0\\
307	0\\
308	0\\
309	0\\
310	0\\
311	0\\
312	0\\
313	0\\
314	0\\
315	0\\
316	0\\
317	0\\
318	0\\
319	0\\
320	0\\
321	0\\
322	0\\
323	0\\
324	0\\
325	0\\
326	0\\
327	0\\
328	0\\
329	0\\
330	0\\
331	0\\
332	0\\
333	0\\
334	0\\
335	0\\
336	0\\
337	0\\
338	0\\
339	0\\
340	0\\
341	0\\
342	0\\
343	0\\
344	0\\
345	0\\
346	0\\
347	0\\
348	0\\
349	0\\
350	0\\
351	0\\
352	0\\
353	0\\
354	0\\
355	0\\
356	0\\
357	0\\
358	0\\
359	0\\
360	0\\
361	0\\
362	0\\
363	0\\
364	0\\
365	0\\
366	0\\
367	0\\
368	0\\
369	0\\
370	0\\
371	0\\
372	0\\
373	0\\
374	0\\
375	0\\
376	0\\
377	0\\
378	0\\
379	0\\
380	0\\
381	0\\
382	0\\
383	0\\
384	0\\
385	0\\
386	0\\
387	0\\
388	0\\
389	0\\
390	0\\
391	0\\
392	0\\
393	0\\
394	0\\
395	0\\
396	0\\
397	0\\
398	0\\
399	0\\
400	0\\
401	0\\
402	0\\
403	0\\
404	0\\
405	0\\
406	0\\
407	0\\
408	0\\
409	0\\
410	0\\
411	0\\
412	0\\
413	0\\
414	0\\
415	0\\
416	0\\
417	0\\
418	0\\
419	0\\
420	0\\
421	0\\
422	0\\
423	0\\
424	0\\
425	0\\
426	0\\
427	0\\
428	0\\
429	0\\
430	0\\
431	0\\
432	0\\
433	0\\
434	0\\
435	0\\
436	0\\
437	0\\
438	0\\
439	0\\
440	0\\
441	0\\
442	0\\
443	0\\
444	0\\
445	0\\
446	0\\
447	0\\
448	0\\
449	0\\
450	0\\
451	0\\
452	0\\
453	0\\
454	0\\
455	0\\
456	0\\
457	0\\
458	0\\
459	0\\
460	0\\
461	0\\
462	0\\
463	0\\
464	0\\
465	0\\
466	0\\
467	0\\
468	0\\
469	0\\
470	0\\
471	0\\
472	0\\
473	0\\
474	0\\
475	0\\
476	0\\
477	0\\
478	0\\
479	0\\
480	0\\
481	0\\
482	0\\
483	0\\
484	0\\
485	0\\
486	0\\
487	0\\
488	0\\
489	0\\
490	0\\
491	0\\
492	0\\
493	0\\
494	0\\
495	0\\
496	0\\
497	0\\
498	0\\
499	0\\
500	0\\
501	0\\
502	0\\
503	0\\
504	0\\
505	0\\
506	0\\
507	0\\
508	0\\
509	0\\
510	0\\
511	0\\
512	0\\
513	0\\
514	0\\
515	0\\
516	0\\
517	0\\
518	0\\
519	0\\
520	0\\
521	0\\
522	0\\
523	0\\
524	0\\
525	0\\
526	0\\
527	0\\
528	0\\
529	0\\
530	0\\
531	0\\
532	0\\
533	0\\
534	0\\
535	0\\
536	1.67358631738589e-05\\
537	4.17113787144746e-05\\
538	6.72225789001668e-05\\
539	9.32840951840519e-05\\
540	0.000119910708273509\\
541	0.000147126549934032\\
542	0.000175083363931519\\
543	0.000203765392134202\\
544	0.000232847505447229\\
545	0.000262347629977608\\
546	0.000292337815156544\\
547	0.000322869907415254\\
548	0.000354099010332315\\
549	0.000386067917324203\\
550	0.000418801906923749\\
551	0.000452323301093407\\
552	0.000486654809378515\\
553	0.000522613247243088\\
554	0.000559557726971212\\
555	0.000596060905993073\\
556	0.000632410263998094\\
557	0.000669141593433294\\
558	0.000706767938764675\\
559	0.000745333893433888\\
560	0.000784877380144605\\
561	0.000825438279498713\\
562	0.000867058997353549\\
563	0.00106897995898566\\
564	0.00141439733785219\\
565	0.00162107369184165\\
566	0.00168232368728045\\
567	0.00174427557656458\\
568	0.00180725842153865\\
569	0.00187130085080805\\
570	0.00193643100447991\\
571	0.00200267838403535\\
572	0.00207007404240156\\
573	0.00213865070462208\\
574	0.00220844289278456\\
575	0.00227948706055647\\
576	0.00235182173855629\\
577	0.00242548769150975\\
578	0.00250052808842228\\
579	0.00257698868739462\\
580	0.00265491803778901\\
581	0.00273436770521657\\
582	0.00281539253206758\\
583	0.00289805096555055\\
584	0.00298240553652045\\
585	0.00306852370943816\\
586	0.00315647969018813\\
587	0.00324635875816936\\
588	0.0033382683090296\\
589	0.00343236680134409\\
590	0.00352894054086423\\
591	0.00362860835901882\\
592	0.00373286830485376\\
593	0.00384555904940007\\
594	0.00397672044113206\\
595	0.00415280575412962\\
596	0.00444436306303141\\
597	0.00503983077166122\\
598	0.00644286460810295\\
599	0\\
600	0\\
};
\addplot [color=black,solid,forget plot]
  table[row sep=crcr]{%
1	0\\
2	0\\
3	0\\
4	0\\
5	0\\
6	0\\
7	0\\
8	0\\
9	0\\
10	0\\
11	0\\
12	0\\
13	0\\
14	0\\
15	0\\
16	0\\
17	0\\
18	0\\
19	0\\
20	0\\
21	0\\
22	0\\
23	0\\
24	0\\
25	0\\
26	0\\
27	0\\
28	0\\
29	0\\
30	0\\
31	0\\
32	0\\
33	0\\
34	0\\
35	0\\
36	0\\
37	0\\
38	0\\
39	0\\
40	0\\
41	0\\
42	0\\
43	0\\
44	0\\
45	0\\
46	0\\
47	0\\
48	0\\
49	0\\
50	0\\
51	0\\
52	0\\
53	0\\
54	0\\
55	0\\
56	0\\
57	0\\
58	0\\
59	0\\
60	0\\
61	0\\
62	0\\
63	0\\
64	0\\
65	0\\
66	0\\
67	0\\
68	0\\
69	0\\
70	0\\
71	0\\
72	0\\
73	0\\
74	0\\
75	0\\
76	0\\
77	0\\
78	0\\
79	0\\
80	0\\
81	0\\
82	0\\
83	0\\
84	0\\
85	0\\
86	0\\
87	0\\
88	0\\
89	0\\
90	0\\
91	0\\
92	0\\
93	0\\
94	0\\
95	0\\
96	0\\
97	0\\
98	0\\
99	0\\
100	0\\
101	0\\
102	0\\
103	0\\
104	0\\
105	0\\
106	0\\
107	0\\
108	0\\
109	0\\
110	0\\
111	0\\
112	0\\
113	0\\
114	0\\
115	0\\
116	0\\
117	0\\
118	0\\
119	0\\
120	0\\
121	0\\
122	0\\
123	0\\
124	0\\
125	0\\
126	0\\
127	0\\
128	0\\
129	0\\
130	0\\
131	0\\
132	0\\
133	0\\
134	0\\
135	0\\
136	0\\
137	0\\
138	0\\
139	0\\
140	0\\
141	0\\
142	0\\
143	0\\
144	0\\
145	0\\
146	0\\
147	0\\
148	0\\
149	0\\
150	0\\
151	0\\
152	0\\
153	0\\
154	0\\
155	0\\
156	0\\
157	0\\
158	0\\
159	0\\
160	0\\
161	0\\
162	0\\
163	0\\
164	0\\
165	0\\
166	0\\
167	0\\
168	0\\
169	0\\
170	0\\
171	0\\
172	0\\
173	0\\
174	0\\
175	0\\
176	0\\
177	0\\
178	0\\
179	0\\
180	0\\
181	0\\
182	0\\
183	0\\
184	0\\
185	0\\
186	0\\
187	0\\
188	0\\
189	0\\
190	0\\
191	0\\
192	0\\
193	0\\
194	0\\
195	0\\
196	0\\
197	0\\
198	0\\
199	0\\
200	0\\
201	0\\
202	0\\
203	0\\
204	0\\
205	0\\
206	0\\
207	0\\
208	0\\
209	0\\
210	0\\
211	0\\
212	0\\
213	0\\
214	0\\
215	0\\
216	0\\
217	0\\
218	0\\
219	0\\
220	0\\
221	0\\
222	0\\
223	0\\
224	0\\
225	0\\
226	0\\
227	0\\
228	0\\
229	0\\
230	0\\
231	0\\
232	0\\
233	0\\
234	0\\
235	0\\
236	0\\
237	0\\
238	0\\
239	0\\
240	0\\
241	0\\
242	0\\
243	0\\
244	0\\
245	0\\
246	0\\
247	0\\
248	0\\
249	0\\
250	0\\
251	0\\
252	0\\
253	0\\
254	0\\
255	0\\
256	0\\
257	0\\
258	0\\
259	0\\
260	0\\
261	0\\
262	0\\
263	0\\
264	0\\
265	0\\
266	0\\
267	0\\
268	0\\
269	0\\
270	0\\
271	0\\
272	0\\
273	0\\
274	0\\
275	0\\
276	0\\
277	0\\
278	0\\
279	0\\
280	0\\
281	0\\
282	0\\
283	0\\
284	0\\
285	0\\
286	0\\
287	0\\
288	0\\
289	0\\
290	0\\
291	0\\
292	0\\
293	0\\
294	0\\
295	0\\
296	0\\
297	0\\
298	0\\
299	0\\
300	0\\
301	0\\
302	0\\
303	0\\
304	0\\
305	0\\
306	0\\
307	0\\
308	0\\
309	0\\
310	0\\
311	0\\
312	0\\
313	0\\
314	0\\
315	0\\
316	0\\
317	0\\
318	0\\
319	0\\
320	0\\
321	0\\
322	0\\
323	0\\
324	0\\
325	0\\
326	0\\
327	0\\
328	0\\
329	0\\
330	0\\
331	0\\
332	0\\
333	0\\
334	0\\
335	0\\
336	0\\
337	0\\
338	0\\
339	0\\
340	0\\
341	0\\
342	0\\
343	0\\
344	0\\
345	0\\
346	0\\
347	0\\
348	0\\
349	0\\
350	0\\
351	0\\
352	0\\
353	0\\
354	0\\
355	0\\
356	0\\
357	0\\
358	0\\
359	0\\
360	0\\
361	0\\
362	0\\
363	0\\
364	0\\
365	0\\
366	0\\
367	0\\
368	0\\
369	0\\
370	0\\
371	0\\
372	0\\
373	0\\
374	0\\
375	0\\
376	0\\
377	0\\
378	0\\
379	0\\
380	0\\
381	0\\
382	0\\
383	0\\
384	0\\
385	0\\
386	0\\
387	0\\
388	0\\
389	0\\
390	0\\
391	0\\
392	0\\
393	0\\
394	0\\
395	0\\
396	0\\
397	0\\
398	0\\
399	0\\
400	0\\
401	0\\
402	0\\
403	0\\
404	0\\
405	0\\
406	0\\
407	0\\
408	0\\
409	0\\
410	0\\
411	0\\
412	0\\
413	0\\
414	0\\
415	0\\
416	0\\
417	0\\
418	0\\
419	0\\
420	0\\
421	0\\
422	0\\
423	0\\
424	0\\
425	0\\
426	0\\
427	0\\
428	0\\
429	0\\
430	0\\
431	0\\
432	0\\
433	0\\
434	0\\
435	0\\
436	0\\
437	0\\
438	0\\
439	0\\
440	0\\
441	0\\
442	0\\
443	0\\
444	0\\
445	0\\
446	0\\
447	0\\
448	0\\
449	0\\
450	0\\
451	0\\
452	0\\
453	0\\
454	0\\
455	0\\
456	0\\
457	0\\
458	0\\
459	0\\
460	0\\
461	0\\
462	0\\
463	0\\
464	0\\
465	0\\
466	0\\
467	0\\
468	0\\
469	0\\
470	0\\
471	0\\
472	0\\
473	0\\
474	0\\
475	0\\
476	0\\
477	0\\
478	0\\
479	0\\
480	0\\
481	0\\
482	0\\
483	0\\
484	0\\
485	0\\
486	0\\
487	0\\
488	0\\
489	0\\
490	0\\
491	0\\
492	0\\
493	0\\
494	0\\
495	0\\
496	0\\
497	0\\
498	0\\
499	0\\
500	0\\
501	0\\
502	0\\
503	0\\
504	0\\
505	0\\
506	0\\
507	0\\
508	0\\
509	0\\
510	0\\
511	0\\
512	0\\
513	0\\
514	0\\
515	0\\
516	0\\
517	0\\
518	0\\
519	0\\
520	0\\
521	0\\
522	0\\
523	0\\
524	0\\
525	0\\
526	0\\
527	0\\
528	0\\
529	0\\
530	0\\
531	0\\
532	0\\
533	0\\
534	0\\
535	0\\
536	2.3273899000436e-05\\
537	4.83574677493769e-05\\
538	7.39764419503986e-05\\
539	0.000100145056725748\\
540	0.000126927776089947\\
541	0.000154551208357287\\
542	0.000182537594263675\\
543	0.000210877875154721\\
544	0.000239654581585972\\
545	0.000268942646031741\\
546	0.000298901101391198\\
547	0.000329565590919966\\
548	0.000360958489926797\\
549	0.000393100761277305\\
550	0.0004260137847175\\
551	0.000459720061076944\\
552	0.000495493917640628\\
553	0.000531330574309899\\
554	0.000566818867744161\\
555	0.000602062179114877\\
556	0.000638100502800796\\
557	0.000675022523672534\\
558	0.000712862997966947\\
559	0.00075165794188291\\
560	0.000791445930933432\\
561	0.000832268907521252\\
562	0.00101617204181345\\
563	0.00137303696588922\\
564	0.0015613898810912\\
565	0.00162138355504735\\
566	0.00168232596368277\\
567	0.0017442757060301\\
568	0.00180725843549943\\
569	0.00187130085404748\\
570	0.00193643100559733\\
571	0.00200267838446417\\
572	0.00207007404258462\\
573	0.0021386507047133\\
574	0.00220844289283176\\
575	0.00227948706058165\\
576	0.00235182173856971\\
577	0.0024254876915166\\
578	0.00250052808842547\\
579	0.00257698868739587\\
580	0.00265491803778937\\
581	0.00273436770521664\\
582	0.00281539253206758\\
583	0.00289805096555053\\
584	0.00298240553652045\\
585	0.00306852370943815\\
586	0.00315647969018813\\
587	0.00324635875816936\\
588	0.00333826830902961\\
589	0.00343236680134409\\
590	0.00352894054086423\\
591	0.00362860835901881\\
592	0.00373286830485375\\
593	0.00384555904940006\\
594	0.00397672044113206\\
595	0.00415280575412962\\
596	0.00444436306303141\\
597	0.00503983077166122\\
598	0.00644286460810295\\
599	0\\
600	0\\
};
\end{axis}
\end{tikzpicture}%
 
  \caption{Discrete Time w/ nFPC}
\end{subfigure}\\

\leavevmode\smash{\makebox[0pt]{\hspace{-7em}% HORIZONTAL POSITION           
  \rotatebox[origin=l]{90}{\hspace{20em}% VERTICAL POSITION
    Depth $\delta^+$}%
}}\hspace{0pt plus 1filll}\null

Time (s)

\vspace{1cm}
\begin{subfigure}{\linewidth}
  \centering
  \tikzsetnextfilename{deltalegend}
  \documentclass{article}
\usepackage{pgfplots}
\usetikzlibrary{backgrounds}
\pgfplotsset{compat=newest}  
\newlength\figureheight 
\newlength\figurewidth 

\begin{document}
%
%\begin{figure}
%  \centering
%  \setlength\figureheight{\linewidth} 
%  \setlength\figurewidth{\linewidth}
%  \input{/home/anton/Documents/masc/ml/thesis/tikz/ORCL_comp4stoch.tikz}
%  \caption{Backtest strategy comparison}
%  \label{fig:insample}
%\end{figure}
\definecolor{mycolor1}{rgb}{1.00000,0.00000,1.00000}%
\begin{tikzpicture}[framed]
    \begingroup
    % inits/clears the lists (which might be populated from previous
    % axes):
    \csname pgfplots@init@cleared@structures\endcsname
    \pgfplotsset{legend style={at={(0,1)},anchor=north west},legend columns=-1,legend style={draw=none,column sep=1ex},legend entries={$q=-4$,$q=-3$,$q=-2$,$q=-1$}}%
    
    \csname pgfplots@addlegendimage\endcsname{thick,green,dashed,sharp plot}
    \csname pgfplots@addlegendimage\endcsname{thick,mycolor1,dashed,sharp plot}
    \csname pgfplots@addlegendimage\endcsname{thick,red,dashed,sharp plot}
    \csname pgfplots@addlegendimage\endcsname{thick,blue,dashed,sharp plot}

    % draws the legend:
    \csname pgfplots@createlegend\endcsname
    \endgroup

    \begingroup
    % inits/clears the lists (which might be populated from previous
    % axes):
    \csname pgfplots@init@cleared@structures\endcsname
    \pgfplotsset{legend style={at={(3.45,0.5)},anchor=north west},legend columns=-1,legend style={draw=none,column sep=1ex},legend entries={$q=0$}}%

    \csname pgfplots@addlegendimage\endcsname{thick,black,sharp plot}

    % draws the legend:
    \csname pgfplots@createlegend\endcsname
    \endgroup

    \begingroup
    % inits/clears the lists (which might be populated from previous
    % axes):
    \csname pgfplots@init@cleared@structures\endcsname
    \pgfplotsset{legend style={at={(0,0)},anchor=north west},legend columns=-1,legend style={draw=none,column sep=1ex},legend entries={$q=+4$,$q=+3$,$q=+2$,$q=+1$}}%
    
    \csname pgfplots@addlegendimage\endcsname{thick,green,sharp plot}
    \csname pgfplots@addlegendimage\endcsname{thick,mycolor1,sharp plot}
    \csname pgfplots@addlegendimage\endcsname{thick,red,sharp plot}
    \csname pgfplots@addlegendimage\endcsname{thick,blue,sharp plot}

    % draws the legend:
    \csname pgfplots@createlegend\endcsname
    \endgroup
\end{tikzpicture}

\end{document} 
\end{subfigure}%
  \caption{Optimal buy depths $\delta^{+}$ for Markov state $Z=(\rho = +1, \Delta S = +1)$, implying heavy imbalance in favor of buy pressure, and having previously seen an upward price change. We expect the midprice to rise.}
  \label{fig:comp_dp_z15}
\end{figure}

\fxnote{So which of these plots do we like better? the 9 choices of q, or the colormap version of q?}
\begin{figure}
\centering
\begin{subfigure}{.45\linewidth}
  \centering
  \setlength\figureheight{\linewidth} 
  \setlength\figurewidth{\linewidth}
  \tikzsetnextfilename{testdp_cts_z1}
  % This file was created by matlab2tikz.
%
%The latest updates can be retrieved from
%  http://www.mathworks.com/matlabcentral/fileexchange/22022-matlab2tikz-matlab2tikz
%where you can also make suggestions and rate matlab2tikz.
%
\definecolor{mycolor1}{rgb}{0.00000,1.00000,0.14286}%
\definecolor{mycolor2}{rgb}{0.00000,1.00000,0.28571}%
\definecolor{mycolor3}{rgb}{0.00000,1.00000,0.42857}%
\definecolor{mycolor4}{rgb}{0.00000,1.00000,0.57143}%
\definecolor{mycolor5}{rgb}{0.00000,1.00000,0.71429}%
\definecolor{mycolor6}{rgb}{0.00000,1.00000,0.85714}%
\definecolor{mycolor7}{rgb}{0.00000,1.00000,1.00000}%
\definecolor{mycolor8}{rgb}{0.00000,0.87500,1.00000}%
\definecolor{mycolor9}{rgb}{0.00000,0.62500,1.00000}%
\definecolor{mycolor10}{rgb}{0.12500,0.00000,1.00000}%
\definecolor{mycolor11}{rgb}{0.25000,0.00000,1.00000}%
\definecolor{mycolor12}{rgb}{0.37500,0.00000,1.00000}%
\definecolor{mycolor13}{rgb}{0.50000,0.00000,1.00000}%
\definecolor{mycolor14}{rgb}{0.62500,0.00000,1.00000}%
\definecolor{mycolor15}{rgb}{0.75000,0.00000,1.00000}%
\definecolor{mycolor16}{rgb}{0.87500,0.00000,1.00000}%
\definecolor{mycolor17}{rgb}{1.00000,0.00000,1.00000}%
\definecolor{mycolor18}{rgb}{1.00000,0.00000,0.87500}%
\definecolor{mycolor19}{rgb}{1.00000,0.00000,0.62500}%
\definecolor{mycolor20}{rgb}{0.85714,0.00000,0.00000}%
\definecolor{mycolor21}{rgb}{0.71429,0.00000,0.00000}%
%
\begin{tikzpicture}

\begin{axis}[%
width=4.1in,
height=3.803in,
at={(0.809in,0.513in)},
scale only axis,
point meta min=0,
point meta max=1,
every outer x axis line/.append style={black},
every x tick label/.append style={font=\color{black}},
xmin=0,
xmax=600,
every outer y axis line/.append style={black},
every y tick label/.append style={font=\color{black}},
ymin=0,
ymax=0.01,
axis background/.style={fill=white},
axis x line*=bottom,
axis y line*=left,
colormap={mymap}{[1pt] rgb(0pt)=(0,1,0); rgb(7pt)=(0,1,1); rgb(15pt)=(0,0,1); rgb(23pt)=(1,0,1); rgb(31pt)=(1,0,0); rgb(38pt)=(0,0,0)},
colorbar,
colorbar style={separate axis lines,every outer x axis line/.append style={black},every x tick label/.append style={font=\color{black}},every outer y axis line/.append style={black},every y tick label/.append style={font=\color{black}},yticklabels={{-19},{-17},{-15},{-13},{-11},{-9},{-7},{-5},{-3},{-1},{1},{3},{5},{7},{9},{11},{13},{15},{17},{19}}}
]
\addplot [color=green,solid,forget plot]
  table[row sep=crcr]{%
0.01	0\\
1.01	0\\
2.01	0\\
3.01	0\\
4.01	0\\
5.01	0\\
6.01	0\\
7.01	0\\
8.01	0\\
9.01	0\\
10.01	0\\
11.01	0\\
12.01	0\\
13.01	0\\
14.01	0\\
15.01	0\\
16.01	0\\
17.01	0\\
18.01	0\\
19.01	0\\
20.01	0\\
21.01	0\\
22.01	0\\
23.01	0\\
24.01	0\\
25.01	0\\
26.01	0\\
27.01	0\\
28.01	0\\
29.01	0\\
30.01	0\\
31.01	0\\
32.01	0\\
33.01	0\\
34.01	0\\
35.01	0\\
36.01	0\\
37.01	0\\
38.01	0\\
39.01	0\\
40.01	0\\
41.01	0\\
42.01	0\\
43.01	0\\
44.01	0\\
45.01	0\\
46.01	0\\
47.01	0\\
48.01	0\\
49.01	0\\
50.01	0\\
51.01	0\\
52.01	0\\
53.01	0\\
54.01	0\\
55.01	0\\
56.01	0\\
57.01	0\\
58.01	0\\
59.01	0\\
60.01	0\\
61.01	0\\
62.01	0\\
63.01	0\\
64.01	0\\
65.01	0\\
66.01	0\\
67.01	0\\
68.01	0\\
69.01	0\\
70.01	0\\
71.01	0\\
72.01	0\\
73.01	0\\
74.01	0\\
75.01	0\\
76.01	0\\
77.01	0\\
78.01	0\\
79.01	0\\
80.01	0\\
81.01	0\\
82.01	0\\
83.01	0\\
84.01	0\\
85.01	0\\
86.01	0\\
87.01	0\\
88.01	0\\
89.01	0\\
90.01	0\\
91.01	0\\
92.01	0\\
93.01	0\\
94.01	0\\
95.01	0\\
96.01	0\\
97.01	0\\
98.01	0\\
99.01	0\\
100.01	0\\
101.01	0\\
102.01	0\\
103.01	0\\
104.01	0\\
105.01	0\\
106.01	0\\
107.01	0\\
108.01	0\\
109.01	0\\
110.01	0\\
111.01	0\\
112.01	0\\
113.01	0\\
114.01	0\\
115.01	0\\
116.01	0\\
117.01	0\\
118.01	0\\
119.01	0\\
120.01	0\\
121.01	0\\
122.01	0\\
123.01	0\\
124.01	0\\
125.01	0\\
126.01	0\\
127.01	0\\
128.01	0\\
129.01	0\\
130.01	0\\
131.01	0\\
132.01	0\\
133.01	0\\
134.01	0\\
135.01	0\\
136.01	0\\
137.01	0\\
138.01	0\\
139.01	0\\
140.01	0\\
141.01	0\\
142.01	0\\
143.01	0\\
144.01	0\\
145.01	0\\
146.01	0\\
147.01	0\\
148.01	0\\
149.01	0\\
150.01	0\\
151.01	0\\
152.01	0\\
153.01	0\\
154.01	0\\
155.01	0\\
156.01	0\\
157.01	0\\
158.01	0\\
159.01	0\\
160.01	0\\
161.01	0\\
162.01	0\\
163.01	0\\
164.01	0\\
165.01	0\\
166.01	0\\
167.01	0\\
168.01	0\\
169.01	0\\
170.01	0\\
171.01	0\\
172.01	0\\
173.01	0\\
174.01	0\\
175.01	0\\
176.01	0\\
177.01	0\\
178.01	0\\
179.01	0\\
180.01	0\\
181.01	0\\
182.01	0\\
183.01	0\\
184.01	0\\
185.01	0\\
186.01	0\\
187.01	0\\
188.01	0\\
189.01	0\\
190.01	0\\
191.01	0\\
192.01	0\\
193.01	0\\
194.01	0\\
195.01	0\\
196.01	0\\
197.01	0\\
198.01	0\\
199.01	0\\
200.01	0\\
201.01	0\\
202.01	0\\
203.01	0\\
204.01	0\\
205.01	0\\
206.01	0\\
207.01	0\\
208.01	0\\
209.01	0\\
210.01	0\\
211.01	0\\
212.01	0\\
213.01	0\\
214.01	0\\
215.01	0\\
216.01	0\\
217.01	0\\
218.01	0\\
219.01	0\\
220.01	0\\
221.01	0\\
222.01	0\\
223.01	0\\
224.01	0\\
225.01	0\\
226.01	0\\
227.01	0\\
228.01	0\\
229.01	0\\
230.01	0\\
231.01	0\\
232.01	0\\
233.01	0\\
234.01	0\\
235.01	0\\
236.01	0\\
237.01	0\\
238.01	0\\
239.01	0\\
240.01	0\\
241.01	0\\
242.01	0\\
243.01	0\\
244.01	0\\
245.01	0\\
246.01	0\\
247.01	0\\
248.01	0\\
249.01	0\\
250.01	0\\
251.01	0\\
252.01	0\\
253.01	0\\
254.01	0\\
255.01	0\\
256.01	0\\
257.01	0\\
258.01	0\\
259.01	0\\
260.01	0\\
261.01	0\\
262.01	0\\
263.01	0\\
264.01	0\\
265.01	0\\
266.01	0\\
267.01	0\\
268.01	0\\
269.01	0\\
270.01	0\\
271.01	0\\
272.01	0\\
273.01	0\\
274.01	0\\
275.01	0\\
276.01	0\\
277.01	0\\
278.01	0\\
279.01	0\\
280.01	0\\
281.01	0\\
282.01	0\\
283.01	0\\
284.01	0\\
285.01	0\\
286.01	0\\
287.01	0\\
288.01	0\\
289.01	0\\
290.01	0\\
291.01	0\\
292.01	0\\
293.01	0\\
294.01	0\\
295.01	0\\
296.01	0\\
297.01	0\\
298.01	0\\
299.01	0\\
300.01	0\\
301.01	0\\
302.01	0\\
303.01	0\\
304.01	0\\
305.01	0\\
306.01	0\\
307.01	0\\
308.01	0\\
309.01	0\\
310.01	0\\
311.01	0\\
312.01	0\\
313.01	0\\
314.01	0\\
315.01	0\\
316.01	0\\
317.01	0\\
318.01	0\\
319.01	0\\
320.01	0\\
321.01	0\\
322.01	0\\
323.01	0\\
324.01	0\\
325.01	0\\
326.01	0\\
327.01	0\\
328.01	0\\
329.01	0\\
330.01	0\\
331.01	0\\
332.01	0\\
333.01	0\\
334.01	0\\
335.01	0\\
336.01	0\\
337.01	0\\
338.01	0\\
339.01	0\\
340.01	0\\
341.01	0\\
342.01	0\\
343.01	0\\
344.01	0\\
345.01	0\\
346.01	0\\
347.01	0\\
348.01	0\\
349.01	0\\
350.01	0\\
351.01	0\\
352.01	0\\
353.01	0\\
354.01	0\\
355.01	0\\
356.01	0\\
357.01	0\\
358.01	0\\
359.01	0\\
360.01	0\\
361.01	0\\
362.01	0\\
363.01	0\\
364.01	0\\
365.01	0\\
366.01	0\\
367.01	0\\
368.01	0\\
369.01	0\\
370.01	0\\
371.01	0\\
372.01	0\\
373.01	0\\
374.01	0\\
375.01	0\\
376.01	0\\
377.01	0\\
378.01	0\\
379.01	0\\
380.01	0\\
381.01	0\\
382.01	0\\
383.01	0\\
384.01	0\\
385.01	0\\
386.01	0\\
387.01	0\\
388.01	0\\
389.01	0\\
390.01	0\\
391.01	0\\
392.01	0\\
393.01	0\\
394.01	0\\
395.01	0\\
396.01	0\\
397.01	0\\
398.01	0\\
399.01	0\\
400.01	0\\
401.01	0\\
402.01	0\\
403.01	0\\
404.01	0\\
405.01	0\\
406.01	0\\
407.01	0\\
408.01	0\\
409.01	0\\
410.01	0\\
411.01	0\\
412.01	0\\
413.01	0\\
414.01	0\\
415.01	0\\
416.01	0\\
417.01	0\\
418.01	0\\
419.01	0\\
420.01	0\\
421.01	0\\
422.01	0\\
423.01	0\\
424.01	0\\
425.01	0\\
426.01	0\\
427.01	0\\
428.01	0\\
429.01	0\\
430.01	0\\
431.01	0\\
432.01	0\\
433.01	0\\
434.01	0\\
435.01	0\\
436.01	0\\
437.01	0\\
438.01	0\\
439.01	0\\
440.01	0\\
441.01	0\\
442.01	0\\
443.01	0\\
444.01	0\\
445.01	0\\
446.01	0\\
447.01	0\\
448.01	0\\
449.01	0\\
450.01	0\\
451.01	1.73472347597681e-18\\
452.01	0\\
453.01	0\\
454.01	0\\
455.01	0\\
456.01	0\\
457.01	0\\
458.01	0\\
459.01	0\\
460.01	0\\
461.01	0\\
462.01	0\\
463.01	0\\
464.01	0\\
465.01	0\\
466.01	0\\
467.01	0\\
468.01	0\\
469.01	1.73472347597681e-18\\
470.01	0\\
471.01	0\\
472.01	0\\
473.01	0\\
474.01	0\\
475.01	0\\
476.01	1.73472347597681e-18\\
477.01	0\\
478.01	0\\
479.01	0\\
480.01	0\\
481.01	0\\
482.01	0\\
483.01	0\\
484.01	0\\
485.01	0\\
486.01	0\\
487.01	0\\
488.01	0\\
489.01	0\\
490.01	0\\
491.01	0\\
492.01	0\\
493.01	0\\
494.01	0\\
495.01	0\\
496.01	0\\
497.01	0\\
498.01	0\\
499.01	0\\
500.01	0\\
501.01	0\\
502.01	0\\
503.01	0\\
504.01	0\\
505.01	0\\
506.01	0\\
507.01	0\\
508.01	0\\
509.01	0\\
510.01	0\\
511.01	0\\
512.01	1.73472347597681e-18\\
513.01	0\\
514.01	0\\
515.01	0\\
516.01	0\\
517.01	0\\
518.01	0\\
519.01	0\\
520.01	0\\
521.01	0\\
522.01	0\\
523.01	1.73472347597681e-18\\
524.01	0\\
525.01	0\\
526.01	0\\
527.01	0\\
528.01	0\\
529.01	0\\
530.01	0\\
531.01	0\\
532.01	0\\
533.01	0\\
534.01	0\\
535.01	0\\
536.01	1.73472347597681e-18\\
537.01	1.73472347597681e-18\\
538.01	1.73472347597681e-18\\
539.01	0\\
540.01	0\\
541.01	0\\
542.01	0\\
543.01	0\\
544.01	0\\
545.01	1.73472347597681e-18\\
546.01	1.73472347597681e-18\\
547.01	0\\
548.01	0\\
549.01	0\\
550.01	0\\
551.01	0\\
552.01	0\\
553.01	0\\
554.01	0\\
555.01	0\\
556.01	1.73472347597681e-18\\
557.01	0\\
558.01	0\\
559.01	0\\
560.01	1.73472347597681e-18\\
561.01	1.73472347597681e-18\\
562.01	0\\
563.01	0\\
564.01	0\\
565.01	0\\
566.01	0\\
567.01	0\\
568.01	0\\
569.01	0\\
570.01	0\\
571.01	0\\
572.01	0\\
573.01	0\\
574.01	0\\
575.01	0\\
576.01	0\\
577.01	0\\
578.01	1.73472347597681e-18\\
579.01	0\\
580.01	0\\
581.01	0\\
582.01	0\\
583.01	1.73472347597681e-18\\
584.01	0\\
585.01	0\\
586.01	0\\
587.01	0\\
588.01	0\\
589.01	0\\
590.01	0\\
591.01	0\\
592.01	0\\
593.01	0\\
594.01	0\\
595.01	0\\
596.01	0\\
597.01	0\\
598.01	0\\
599.01	0\\
599.02	0\\
599.03	0\\
599.04	0\\
599.05	0\\
599.06	0\\
599.07	0\\
599.08	0\\
599.09	0\\
599.1	0\\
599.11	0\\
599.12	0\\
599.13	0\\
599.14	0\\
599.15	0\\
599.16	0\\
599.17	0\\
599.18	0\\
599.19	0\\
599.2	0\\
599.21	0\\
599.22	0\\
599.23	0\\
599.24	0\\
599.25	0\\
599.26	0\\
599.27	0\\
599.28	0\\
599.29	0\\
599.3	0\\
599.31	0\\
599.32	0\\
599.33	0\\
599.34	0\\
599.35	0\\
599.36	0\\
599.37	0\\
599.38	0\\
599.39	0\\
599.4	0\\
599.41	0\\
599.42	0\\
599.43	0\\
599.44	0\\
599.45	0\\
599.46	0\\
599.47	0\\
599.48	0\\
599.49	0\\
599.5	0\\
599.51	0\\
599.52	0\\
599.53	0\\
599.54	0\\
599.55	0\\
599.56	0\\
599.57	0\\
599.58	0\\
599.59	0\\
599.6	0\\
599.61	0\\
599.62	0\\
599.63	0\\
599.64	0\\
599.65	0\\
599.66	0\\
599.67	0\\
599.68	0\\
599.69	0\\
599.7	0\\
599.71	0\\
599.72	0\\
599.73	0\\
599.74	0\\
599.75	0\\
599.76	0\\
599.77	0\\
599.78	0\\
599.79	0\\
599.8	0\\
599.81	0\\
599.82	0\\
599.83	0\\
599.84	0\\
599.85	0\\
599.86	0\\
599.87	0\\
599.88	0\\
599.89	0\\
599.9	0\\
599.91	0\\
599.92	0\\
599.93	0\\
599.94	0\\
599.95	0\\
599.96	0\\
599.97	0\\
599.98	0\\
599.99	0\\
600	0\\
};
\addplot [color=mycolor1,solid,forget plot]
  table[row sep=crcr]{%
0.01	0\\
1.01	0\\
2.01	0\\
3.01	0\\
4.01	0\\
5.01	0\\
6.01	0\\
7.01	0\\
8.01	0\\
9.01	0\\
10.01	0\\
11.01	0\\
12.01	0\\
13.01	0\\
14.01	0\\
15.01	0\\
16.01	0\\
17.01	0\\
18.01	0\\
19.01	0\\
20.01	0\\
21.01	0\\
22.01	0\\
23.01	0\\
24.01	0\\
25.01	0\\
26.01	0\\
27.01	0\\
28.01	0\\
29.01	0\\
30.01	0\\
31.01	0\\
32.01	0\\
33.01	0\\
34.01	0\\
35.01	0\\
36.01	0\\
37.01	0\\
38.01	0\\
39.01	0\\
40.01	0\\
41.01	0\\
42.01	0\\
43.01	0\\
44.01	0\\
45.01	0\\
46.01	0\\
47.01	0\\
48.01	0\\
49.01	0\\
50.01	0\\
51.01	0\\
52.01	0\\
53.01	0\\
54.01	0\\
55.01	0\\
56.01	0\\
57.01	0\\
58.01	0\\
59.01	0\\
60.01	0\\
61.01	0\\
62.01	0\\
63.01	0\\
64.01	0\\
65.01	0\\
66.01	0\\
67.01	0\\
68.01	0\\
69.01	0\\
70.01	0\\
71.01	0\\
72.01	0\\
73.01	0\\
74.01	0\\
75.01	0\\
76.01	0\\
77.01	0\\
78.01	0\\
79.01	0\\
80.01	0\\
81.01	0\\
82.01	0\\
83.01	0\\
84.01	0\\
85.01	0\\
86.01	0\\
87.01	0\\
88.01	0\\
89.01	0\\
90.01	0\\
91.01	0\\
92.01	0\\
93.01	0\\
94.01	0\\
95.01	0\\
96.01	0\\
97.01	0\\
98.01	0\\
99.01	0\\
100.01	0\\
101.01	0\\
102.01	0\\
103.01	0\\
104.01	0\\
105.01	0\\
106.01	0\\
107.01	0\\
108.01	0\\
109.01	0\\
110.01	0\\
111.01	0\\
112.01	0\\
113.01	0\\
114.01	0\\
115.01	0\\
116.01	0\\
117.01	0\\
118.01	0\\
119.01	0\\
120.01	0\\
121.01	0\\
122.01	0\\
123.01	0\\
124.01	0\\
125.01	0\\
126.01	0\\
127.01	0\\
128.01	0\\
129.01	0\\
130.01	0\\
131.01	0\\
132.01	0\\
133.01	0\\
134.01	0\\
135.01	0\\
136.01	0\\
137.01	0\\
138.01	0\\
139.01	0\\
140.01	0\\
141.01	0\\
142.01	0\\
143.01	0\\
144.01	0\\
145.01	0\\
146.01	0\\
147.01	0\\
148.01	0\\
149.01	0\\
150.01	0\\
151.01	0\\
152.01	0\\
153.01	0\\
154.01	0\\
155.01	0\\
156.01	0\\
157.01	0\\
158.01	0\\
159.01	0\\
160.01	0\\
161.01	0\\
162.01	0\\
163.01	0\\
164.01	0\\
165.01	0\\
166.01	0\\
167.01	0\\
168.01	0\\
169.01	0\\
170.01	0\\
171.01	0\\
172.01	0\\
173.01	0\\
174.01	0\\
175.01	0\\
176.01	0\\
177.01	0\\
178.01	0\\
179.01	0\\
180.01	0\\
181.01	0\\
182.01	0\\
183.01	0\\
184.01	0\\
185.01	0\\
186.01	0\\
187.01	0\\
188.01	0\\
189.01	0\\
190.01	0\\
191.01	0\\
192.01	0\\
193.01	0\\
194.01	0\\
195.01	0\\
196.01	0\\
197.01	0\\
198.01	0\\
199.01	0\\
200.01	0\\
201.01	0\\
202.01	0\\
203.01	0\\
204.01	0\\
205.01	0\\
206.01	0\\
207.01	0\\
208.01	0\\
209.01	0\\
210.01	0\\
211.01	0\\
212.01	0\\
213.01	0\\
214.01	0\\
215.01	0\\
216.01	0\\
217.01	0\\
218.01	0\\
219.01	0\\
220.01	0\\
221.01	0\\
222.01	0\\
223.01	0\\
224.01	0\\
225.01	0\\
226.01	0\\
227.01	0\\
228.01	0\\
229.01	0\\
230.01	0\\
231.01	0\\
232.01	0\\
233.01	0\\
234.01	0\\
235.01	0\\
236.01	0\\
237.01	0\\
238.01	0\\
239.01	0\\
240.01	0\\
241.01	0\\
242.01	0\\
243.01	0\\
244.01	0\\
245.01	0\\
246.01	0\\
247.01	0\\
248.01	0\\
249.01	0\\
250.01	0\\
251.01	0\\
252.01	0\\
253.01	0\\
254.01	0\\
255.01	0\\
256.01	0\\
257.01	0\\
258.01	0\\
259.01	0\\
260.01	0\\
261.01	0\\
262.01	0\\
263.01	0\\
264.01	0\\
265.01	0\\
266.01	0\\
267.01	0\\
268.01	0\\
269.01	0\\
270.01	0\\
271.01	0\\
272.01	0\\
273.01	0\\
274.01	0\\
275.01	0\\
276.01	0\\
277.01	0\\
278.01	0\\
279.01	0\\
280.01	0\\
281.01	0\\
282.01	0\\
283.01	0\\
284.01	0\\
285.01	0\\
286.01	0\\
287.01	0\\
288.01	0\\
289.01	0\\
290.01	0\\
291.01	0\\
292.01	0\\
293.01	0\\
294.01	0\\
295.01	0\\
296.01	0\\
297.01	0\\
298.01	0\\
299.01	0\\
300.01	0\\
301.01	0\\
302.01	0\\
303.01	0\\
304.01	0\\
305.01	0\\
306.01	0\\
307.01	0\\
308.01	0\\
309.01	0\\
310.01	0\\
311.01	0\\
312.01	0\\
313.01	0\\
314.01	0\\
315.01	0\\
316.01	0\\
317.01	0\\
318.01	0\\
319.01	0\\
320.01	0\\
321.01	0\\
322.01	0\\
323.01	0\\
324.01	0\\
325.01	0\\
326.01	0\\
327.01	0\\
328.01	0\\
329.01	0\\
330.01	0\\
331.01	0\\
332.01	0\\
333.01	0\\
334.01	0\\
335.01	0\\
336.01	0\\
337.01	0\\
338.01	0\\
339.01	0\\
340.01	0\\
341.01	0\\
342.01	0\\
343.01	0\\
344.01	0\\
345.01	0\\
346.01	0\\
347.01	0\\
348.01	0\\
349.01	0\\
350.01	0\\
351.01	0\\
352.01	0\\
353.01	0\\
354.01	0\\
355.01	0\\
356.01	0\\
357.01	0\\
358.01	0\\
359.01	0\\
360.01	0\\
361.01	0\\
362.01	0\\
363.01	0\\
364.01	0\\
365.01	0\\
366.01	0\\
367.01	0\\
368.01	0\\
369.01	0\\
370.01	0\\
371.01	0\\
372.01	0\\
373.01	0\\
374.01	0\\
375.01	0\\
376.01	0\\
377.01	0\\
378.01	0\\
379.01	0\\
380.01	0\\
381.01	0\\
382.01	0\\
383.01	0\\
384.01	0\\
385.01	0\\
386.01	0\\
387.01	0\\
388.01	0\\
389.01	0\\
390.01	0\\
391.01	0\\
392.01	0\\
393.01	0\\
394.01	0\\
395.01	0\\
396.01	0\\
397.01	0\\
398.01	0\\
399.01	0\\
400.01	0\\
401.01	0\\
402.01	0\\
403.01	0\\
404.01	0\\
405.01	0\\
406.01	0\\
407.01	0\\
408.01	0\\
409.01	0\\
410.01	0\\
411.01	0\\
412.01	0\\
413.01	0\\
414.01	0\\
415.01	0\\
416.01	0\\
417.01	0\\
418.01	0\\
419.01	0\\
420.01	0\\
421.01	0\\
422.01	0\\
423.01	0\\
424.01	0\\
425.01	0\\
426.01	0\\
427.01	0\\
428.01	0\\
429.01	0\\
430.01	0\\
431.01	0\\
432.01	0\\
433.01	0\\
434.01	0\\
435.01	0\\
436.01	0\\
437.01	0\\
438.01	0\\
439.01	0\\
440.01	0\\
441.01	0\\
442.01	0\\
443.01	0\\
444.01	0\\
445.01	0\\
446.01	0\\
447.01	0\\
448.01	0\\
449.01	0\\
450.01	0\\
451.01	1.73472347597681e-18\\
452.01	0\\
453.01	0\\
454.01	0\\
455.01	0\\
456.01	0\\
457.01	0\\
458.01	0\\
459.01	0\\
460.01	0\\
461.01	0\\
462.01	0\\
463.01	0\\
464.01	0\\
465.01	0\\
466.01	0\\
467.01	0\\
468.01	0\\
469.01	1.73472347597681e-18\\
470.01	0\\
471.01	0\\
472.01	0\\
473.01	0\\
474.01	0\\
475.01	0\\
476.01	1.73472347597681e-18\\
477.01	0\\
478.01	0\\
479.01	0\\
480.01	0\\
481.01	0\\
482.01	0\\
483.01	0\\
484.01	0\\
485.01	0\\
486.01	0\\
487.01	0\\
488.01	0\\
489.01	0\\
490.01	0\\
491.01	0\\
492.01	0\\
493.01	0\\
494.01	0\\
495.01	0\\
496.01	0\\
497.01	0\\
498.01	0\\
499.01	0\\
500.01	0\\
501.01	0\\
502.01	0\\
503.01	0\\
504.01	0\\
505.01	0\\
506.01	0\\
507.01	0\\
508.01	0\\
509.01	0\\
510.01	0\\
511.01	0\\
512.01	1.73472347597681e-18\\
513.01	0\\
514.01	0\\
515.01	0\\
516.01	0\\
517.01	0\\
518.01	0\\
519.01	0\\
520.01	0\\
521.01	0\\
522.01	0\\
523.01	1.73472347597681e-18\\
524.01	0\\
525.01	0\\
526.01	0\\
527.01	0\\
528.01	0\\
529.01	0\\
530.01	0\\
531.01	0\\
532.01	0\\
533.01	0\\
534.01	0\\
535.01	0\\
536.01	1.73472347597681e-18\\
537.01	1.73472347597681e-18\\
538.01	1.73472347597681e-18\\
539.01	0\\
540.01	0\\
541.01	0\\
542.01	0\\
543.01	0\\
544.01	0\\
545.01	1.73472347597681e-18\\
546.01	1.73472347597681e-18\\
547.01	0\\
548.01	0\\
549.01	0\\
550.01	0\\
551.01	0\\
552.01	0\\
553.01	0\\
554.01	0\\
555.01	0\\
556.01	1.73472347597681e-18\\
557.01	0\\
558.01	0\\
559.01	0\\
560.01	1.73472347597681e-18\\
561.01	1.73472347597681e-18\\
562.01	0\\
563.01	0\\
564.01	0\\
565.01	0\\
566.01	0\\
567.01	0\\
568.01	0\\
569.01	0\\
570.01	0\\
571.01	0\\
572.01	0\\
573.01	0\\
574.01	0\\
575.01	0\\
576.01	0\\
577.01	0\\
578.01	1.73472347597681e-18\\
579.01	0\\
580.01	0\\
581.01	0\\
582.01	0\\
583.01	1.73472347597681e-18\\
584.01	0\\
585.01	0\\
586.01	0\\
587.01	0\\
588.01	0\\
589.01	0\\
590.01	0\\
591.01	0\\
592.01	0\\
593.01	0\\
594.01	0\\
595.01	0\\
596.01	0\\
597.01	0\\
598.01	0\\
599.01	0\\
599.02	0\\
599.03	0\\
599.04	0\\
599.05	0\\
599.06	0\\
599.07	0\\
599.08	0\\
599.09	0\\
599.1	0\\
599.11	0\\
599.12	0\\
599.13	0\\
599.14	0\\
599.15	0\\
599.16	0\\
599.17	0\\
599.18	0\\
599.19	0\\
599.2	0\\
599.21	0\\
599.22	0\\
599.23	0\\
599.24	0\\
599.25	0\\
599.26	0\\
599.27	0\\
599.28	0\\
599.29	0\\
599.3	0\\
599.31	0\\
599.32	0\\
599.33	0\\
599.34	0\\
599.35	0\\
599.36	0\\
599.37	0\\
599.38	0\\
599.39	0\\
599.4	0\\
599.41	0\\
599.42	0\\
599.43	0\\
599.44	0\\
599.45	0\\
599.46	0\\
599.47	0\\
599.48	0\\
599.49	0\\
599.5	0\\
599.51	0\\
599.52	0\\
599.53	0\\
599.54	0\\
599.55	0\\
599.56	0\\
599.57	0\\
599.58	0\\
599.59	0\\
599.6	0\\
599.61	0\\
599.62	0\\
599.63	0\\
599.64	0\\
599.65	0\\
599.66	0\\
599.67	0\\
599.68	0\\
599.69	0\\
599.7	0\\
599.71	0\\
599.72	0\\
599.73	0\\
599.74	0\\
599.75	0\\
599.76	0\\
599.77	0\\
599.78	0\\
599.79	0\\
599.8	0\\
599.81	0\\
599.82	0\\
599.83	0\\
599.84	0\\
599.85	0\\
599.86	0\\
599.87	0\\
599.88	0\\
599.89	0\\
599.9	0\\
599.91	0\\
599.92	0\\
599.93	0\\
599.94	0\\
599.95	0\\
599.96	0\\
599.97	0\\
599.98	0\\
599.99	0\\
600	0\\
};
\addplot [color=mycolor2,solid,forget plot]
  table[row sep=crcr]{%
0.01	0\\
1.01	0\\
2.01	0\\
3.01	0\\
4.01	0\\
5.01	0\\
6.01	0\\
7.01	0\\
8.01	0\\
9.01	0\\
10.01	0\\
11.01	0\\
12.01	0\\
13.01	0\\
14.01	0\\
15.01	0\\
16.01	0\\
17.01	0\\
18.01	0\\
19.01	0\\
20.01	0\\
21.01	0\\
22.01	0\\
23.01	0\\
24.01	0\\
25.01	0\\
26.01	0\\
27.01	0\\
28.01	0\\
29.01	0\\
30.01	0\\
31.01	0\\
32.01	0\\
33.01	0\\
34.01	0\\
35.01	0\\
36.01	0\\
37.01	0\\
38.01	0\\
39.01	0\\
40.01	0\\
41.01	0\\
42.01	0\\
43.01	0\\
44.01	0\\
45.01	0\\
46.01	0\\
47.01	0\\
48.01	0\\
49.01	0\\
50.01	0\\
51.01	0\\
52.01	0\\
53.01	0\\
54.01	0\\
55.01	0\\
56.01	0\\
57.01	0\\
58.01	0\\
59.01	0\\
60.01	0\\
61.01	0\\
62.01	0\\
63.01	0\\
64.01	0\\
65.01	0\\
66.01	0\\
67.01	0\\
68.01	0\\
69.01	0\\
70.01	0\\
71.01	0\\
72.01	0\\
73.01	0\\
74.01	0\\
75.01	0\\
76.01	0\\
77.01	0\\
78.01	0\\
79.01	0\\
80.01	0\\
81.01	0\\
82.01	0\\
83.01	0\\
84.01	0\\
85.01	0\\
86.01	0\\
87.01	0\\
88.01	0\\
89.01	0\\
90.01	0\\
91.01	0\\
92.01	0\\
93.01	0\\
94.01	0\\
95.01	0\\
96.01	0\\
97.01	0\\
98.01	0\\
99.01	0\\
100.01	0\\
101.01	0\\
102.01	0\\
103.01	0\\
104.01	0\\
105.01	0\\
106.01	0\\
107.01	0\\
108.01	0\\
109.01	0\\
110.01	0\\
111.01	0\\
112.01	0\\
113.01	0\\
114.01	0\\
115.01	0\\
116.01	0\\
117.01	0\\
118.01	0\\
119.01	0\\
120.01	0\\
121.01	0\\
122.01	0\\
123.01	0\\
124.01	0\\
125.01	0\\
126.01	0\\
127.01	0\\
128.01	0\\
129.01	0\\
130.01	0\\
131.01	0\\
132.01	0\\
133.01	0\\
134.01	0\\
135.01	0\\
136.01	0\\
137.01	0\\
138.01	0\\
139.01	0\\
140.01	0\\
141.01	0\\
142.01	0\\
143.01	0\\
144.01	0\\
145.01	0\\
146.01	0\\
147.01	0\\
148.01	0\\
149.01	0\\
150.01	0\\
151.01	0\\
152.01	0\\
153.01	0\\
154.01	0\\
155.01	0\\
156.01	0\\
157.01	0\\
158.01	0\\
159.01	0\\
160.01	0\\
161.01	0\\
162.01	0\\
163.01	0\\
164.01	0\\
165.01	0\\
166.01	0\\
167.01	0\\
168.01	0\\
169.01	0\\
170.01	0\\
171.01	0\\
172.01	0\\
173.01	0\\
174.01	0\\
175.01	0\\
176.01	0\\
177.01	0\\
178.01	0\\
179.01	0\\
180.01	0\\
181.01	0\\
182.01	0\\
183.01	0\\
184.01	0\\
185.01	0\\
186.01	0\\
187.01	0\\
188.01	0\\
189.01	0\\
190.01	0\\
191.01	0\\
192.01	0\\
193.01	0\\
194.01	0\\
195.01	0\\
196.01	0\\
197.01	0\\
198.01	0\\
199.01	0\\
200.01	0\\
201.01	0\\
202.01	0\\
203.01	0\\
204.01	0\\
205.01	0\\
206.01	0\\
207.01	0\\
208.01	0\\
209.01	0\\
210.01	0\\
211.01	0\\
212.01	0\\
213.01	0\\
214.01	0\\
215.01	0\\
216.01	0\\
217.01	0\\
218.01	0\\
219.01	0\\
220.01	0\\
221.01	0\\
222.01	0\\
223.01	0\\
224.01	0\\
225.01	0\\
226.01	0\\
227.01	0\\
228.01	0\\
229.01	0\\
230.01	0\\
231.01	0\\
232.01	0\\
233.01	0\\
234.01	0\\
235.01	0\\
236.01	0\\
237.01	0\\
238.01	0\\
239.01	0\\
240.01	0\\
241.01	0\\
242.01	0\\
243.01	0\\
244.01	0\\
245.01	0\\
246.01	0\\
247.01	0\\
248.01	0\\
249.01	0\\
250.01	0\\
251.01	0\\
252.01	0\\
253.01	0\\
254.01	0\\
255.01	0\\
256.01	0\\
257.01	0\\
258.01	0\\
259.01	0\\
260.01	0\\
261.01	0\\
262.01	0\\
263.01	0\\
264.01	0\\
265.01	0\\
266.01	0\\
267.01	0\\
268.01	0\\
269.01	0\\
270.01	0\\
271.01	0\\
272.01	0\\
273.01	0\\
274.01	0\\
275.01	0\\
276.01	0\\
277.01	0\\
278.01	0\\
279.01	0\\
280.01	0\\
281.01	0\\
282.01	0\\
283.01	0\\
284.01	0\\
285.01	0\\
286.01	0\\
287.01	0\\
288.01	0\\
289.01	0\\
290.01	0\\
291.01	0\\
292.01	0\\
293.01	0\\
294.01	0\\
295.01	0\\
296.01	0\\
297.01	0\\
298.01	0\\
299.01	0\\
300.01	0\\
301.01	0\\
302.01	0\\
303.01	0\\
304.01	0\\
305.01	0\\
306.01	0\\
307.01	0\\
308.01	0\\
309.01	0\\
310.01	0\\
311.01	0\\
312.01	0\\
313.01	0\\
314.01	0\\
315.01	0\\
316.01	0\\
317.01	0\\
318.01	0\\
319.01	0\\
320.01	0\\
321.01	0\\
322.01	0\\
323.01	0\\
324.01	0\\
325.01	0\\
326.01	0\\
327.01	0\\
328.01	0\\
329.01	0\\
330.01	0\\
331.01	0\\
332.01	0\\
333.01	0\\
334.01	0\\
335.01	0\\
336.01	0\\
337.01	0\\
338.01	0\\
339.01	0\\
340.01	0\\
341.01	0\\
342.01	0\\
343.01	0\\
344.01	0\\
345.01	0\\
346.01	0\\
347.01	0\\
348.01	0\\
349.01	0\\
350.01	0\\
351.01	0\\
352.01	0\\
353.01	0\\
354.01	0\\
355.01	0\\
356.01	0\\
357.01	0\\
358.01	0\\
359.01	0\\
360.01	0\\
361.01	0\\
362.01	0\\
363.01	0\\
364.01	0\\
365.01	0\\
366.01	0\\
367.01	0\\
368.01	0\\
369.01	0\\
370.01	0\\
371.01	0\\
372.01	0\\
373.01	0\\
374.01	0\\
375.01	0\\
376.01	0\\
377.01	0\\
378.01	0\\
379.01	0\\
380.01	0\\
381.01	0\\
382.01	0\\
383.01	0\\
384.01	0\\
385.01	0\\
386.01	0\\
387.01	0\\
388.01	0\\
389.01	0\\
390.01	0\\
391.01	0\\
392.01	0\\
393.01	0\\
394.01	0\\
395.01	0\\
396.01	0\\
397.01	0\\
398.01	0\\
399.01	0\\
400.01	0\\
401.01	0\\
402.01	0\\
403.01	0\\
404.01	0\\
405.01	0\\
406.01	0\\
407.01	0\\
408.01	0\\
409.01	0\\
410.01	0\\
411.01	0\\
412.01	0\\
413.01	0\\
414.01	0\\
415.01	0\\
416.01	0\\
417.01	0\\
418.01	0\\
419.01	0\\
420.01	0\\
421.01	0\\
422.01	0\\
423.01	0\\
424.01	0\\
425.01	0\\
426.01	0\\
427.01	0\\
428.01	0\\
429.01	0\\
430.01	0\\
431.01	0\\
432.01	0\\
433.01	0\\
434.01	0\\
435.01	0\\
436.01	0\\
437.01	0\\
438.01	0\\
439.01	0\\
440.01	0\\
441.01	0\\
442.01	0\\
443.01	0\\
444.01	0\\
445.01	0\\
446.01	0\\
447.01	0\\
448.01	0\\
449.01	0\\
450.01	0\\
451.01	1.73472347597681e-18\\
452.01	0\\
453.01	0\\
454.01	0\\
455.01	0\\
456.01	0\\
457.01	0\\
458.01	0\\
459.01	0\\
460.01	0\\
461.01	0\\
462.01	0\\
463.01	0\\
464.01	0\\
465.01	0\\
466.01	0\\
467.01	0\\
468.01	0\\
469.01	1.73472347597681e-18\\
470.01	0\\
471.01	0\\
472.01	0\\
473.01	0\\
474.01	0\\
475.01	0\\
476.01	1.73472347597681e-18\\
477.01	0\\
478.01	0\\
479.01	0\\
480.01	0\\
481.01	0\\
482.01	0\\
483.01	0\\
484.01	0\\
485.01	0\\
486.01	0\\
487.01	0\\
488.01	0\\
489.01	0\\
490.01	0\\
491.01	0\\
492.01	0\\
493.01	0\\
494.01	0\\
495.01	0\\
496.01	0\\
497.01	0\\
498.01	0\\
499.01	0\\
500.01	0\\
501.01	0\\
502.01	0\\
503.01	0\\
504.01	0\\
505.01	0\\
506.01	0\\
507.01	0\\
508.01	0\\
509.01	0\\
510.01	0\\
511.01	0\\
512.01	1.73472347597681e-18\\
513.01	0\\
514.01	0\\
515.01	0\\
516.01	0\\
517.01	0\\
518.01	0\\
519.01	0\\
520.01	0\\
521.01	0\\
522.01	0\\
523.01	1.73472347597681e-18\\
524.01	0\\
525.01	0\\
526.01	0\\
527.01	0\\
528.01	0\\
529.01	0\\
530.01	0\\
531.01	0\\
532.01	0\\
533.01	0\\
534.01	0\\
535.01	0\\
536.01	1.73472347597681e-18\\
537.01	1.73472347597681e-18\\
538.01	1.73472347597681e-18\\
539.01	0\\
540.01	0\\
541.01	0\\
542.01	0\\
543.01	0\\
544.01	0\\
545.01	1.73472347597681e-18\\
546.01	1.73472347597681e-18\\
547.01	0\\
548.01	0\\
549.01	0\\
550.01	0\\
551.01	0\\
552.01	0\\
553.01	0\\
554.01	0\\
555.01	0\\
556.01	1.73472347597681e-18\\
557.01	0\\
558.01	0\\
559.01	0\\
560.01	1.73472347597681e-18\\
561.01	1.73472347597681e-18\\
562.01	0\\
563.01	0\\
564.01	0\\
565.01	0\\
566.01	0\\
567.01	0\\
568.01	0\\
569.01	0\\
570.01	0\\
571.01	0\\
572.01	0\\
573.01	0\\
574.01	0\\
575.01	0\\
576.01	0\\
577.01	0\\
578.01	1.73472347597681e-18\\
579.01	0\\
580.01	0\\
581.01	0\\
582.01	0\\
583.01	1.73472347597681e-18\\
584.01	0\\
585.01	0\\
586.01	0\\
587.01	0\\
588.01	0\\
589.01	0\\
590.01	0\\
591.01	0\\
592.01	0\\
593.01	0\\
594.01	0\\
595.01	0\\
596.01	0\\
597.01	0\\
598.01	0\\
599.01	0\\
599.02	0\\
599.03	0\\
599.04	0\\
599.05	0\\
599.06	0\\
599.07	0\\
599.08	0\\
599.09	0\\
599.1	0\\
599.11	0\\
599.12	0\\
599.13	0\\
599.14	0\\
599.15	0\\
599.16	0\\
599.17	0\\
599.18	0\\
599.19	0\\
599.2	0\\
599.21	0\\
599.22	0\\
599.23	0\\
599.24	0\\
599.25	0\\
599.26	0\\
599.27	0\\
599.28	0\\
599.29	0\\
599.3	0\\
599.31	0\\
599.32	0\\
599.33	0\\
599.34	0\\
599.35	0\\
599.36	0\\
599.37	0\\
599.38	0\\
599.39	0\\
599.4	0\\
599.41	0\\
599.42	0\\
599.43	0\\
599.44	0\\
599.45	0\\
599.46	0\\
599.47	0\\
599.48	0\\
599.49	0\\
599.5	0\\
599.51	0\\
599.52	0\\
599.53	0\\
599.54	0\\
599.55	0\\
599.56	0\\
599.57	0\\
599.58	0\\
599.59	0\\
599.6	0\\
599.61	0\\
599.62	0\\
599.63	0\\
599.64	0\\
599.65	0\\
599.66	0\\
599.67	0\\
599.68	0\\
599.69	0\\
599.7	0\\
599.71	0\\
599.72	0\\
599.73	0\\
599.74	0\\
599.75	0\\
599.76	0\\
599.77	0\\
599.78	0\\
599.79	0\\
599.8	0\\
599.81	0\\
599.82	0\\
599.83	0\\
599.84	0\\
599.85	0\\
599.86	0\\
599.87	0\\
599.88	0\\
599.89	0\\
599.9	0\\
599.91	0\\
599.92	0\\
599.93	0\\
599.94	0\\
599.95	0\\
599.96	0\\
599.97	0\\
599.98	0\\
599.99	0\\
600	0\\
};
\addplot [color=mycolor3,solid,forget plot]
  table[row sep=crcr]{%
0.01	0\\
1.01	0\\
2.01	0\\
3.01	0\\
4.01	0\\
5.01	0\\
6.01	0\\
7.01	0\\
8.01	0\\
9.01	0\\
10.01	0\\
11.01	0\\
12.01	0\\
13.01	0\\
14.01	0\\
15.01	0\\
16.01	0\\
17.01	0\\
18.01	0\\
19.01	0\\
20.01	0\\
21.01	0\\
22.01	0\\
23.01	0\\
24.01	0\\
25.01	0\\
26.01	0\\
27.01	0\\
28.01	0\\
29.01	0\\
30.01	0\\
31.01	0\\
32.01	0\\
33.01	0\\
34.01	0\\
35.01	0\\
36.01	0\\
37.01	0\\
38.01	0\\
39.01	0\\
40.01	0\\
41.01	0\\
42.01	0\\
43.01	0\\
44.01	0\\
45.01	0\\
46.01	0\\
47.01	0\\
48.01	0\\
49.01	0\\
50.01	0\\
51.01	0\\
52.01	0\\
53.01	0\\
54.01	0\\
55.01	0\\
56.01	0\\
57.01	0\\
58.01	0\\
59.01	0\\
60.01	0\\
61.01	0\\
62.01	0\\
63.01	0\\
64.01	0\\
65.01	0\\
66.01	0\\
67.01	0\\
68.01	0\\
69.01	0\\
70.01	0\\
71.01	0\\
72.01	0\\
73.01	0\\
74.01	0\\
75.01	0\\
76.01	0\\
77.01	0\\
78.01	0\\
79.01	0\\
80.01	0\\
81.01	0\\
82.01	0\\
83.01	0\\
84.01	0\\
85.01	0\\
86.01	0\\
87.01	0\\
88.01	0\\
89.01	0\\
90.01	0\\
91.01	0\\
92.01	0\\
93.01	0\\
94.01	0\\
95.01	0\\
96.01	0\\
97.01	0\\
98.01	0\\
99.01	0\\
100.01	0\\
101.01	0\\
102.01	0\\
103.01	0\\
104.01	0\\
105.01	0\\
106.01	0\\
107.01	0\\
108.01	0\\
109.01	0\\
110.01	0\\
111.01	0\\
112.01	0\\
113.01	0\\
114.01	0\\
115.01	0\\
116.01	0\\
117.01	0\\
118.01	0\\
119.01	0\\
120.01	0\\
121.01	0\\
122.01	0\\
123.01	0\\
124.01	0\\
125.01	0\\
126.01	0\\
127.01	0\\
128.01	0\\
129.01	0\\
130.01	0\\
131.01	0\\
132.01	0\\
133.01	0\\
134.01	0\\
135.01	0\\
136.01	0\\
137.01	0\\
138.01	0\\
139.01	0\\
140.01	0\\
141.01	0\\
142.01	0\\
143.01	0\\
144.01	0\\
145.01	0\\
146.01	0\\
147.01	0\\
148.01	0\\
149.01	0\\
150.01	0\\
151.01	0\\
152.01	0\\
153.01	0\\
154.01	0\\
155.01	0\\
156.01	0\\
157.01	0\\
158.01	0\\
159.01	0\\
160.01	0\\
161.01	0\\
162.01	0\\
163.01	0\\
164.01	0\\
165.01	0\\
166.01	0\\
167.01	0\\
168.01	0\\
169.01	0\\
170.01	0\\
171.01	0\\
172.01	0\\
173.01	0\\
174.01	0\\
175.01	0\\
176.01	0\\
177.01	0\\
178.01	0\\
179.01	0\\
180.01	0\\
181.01	0\\
182.01	0\\
183.01	0\\
184.01	0\\
185.01	0\\
186.01	0\\
187.01	0\\
188.01	0\\
189.01	0\\
190.01	0\\
191.01	0\\
192.01	0\\
193.01	0\\
194.01	0\\
195.01	0\\
196.01	0\\
197.01	0\\
198.01	0\\
199.01	0\\
200.01	0\\
201.01	0\\
202.01	0\\
203.01	0\\
204.01	0\\
205.01	0\\
206.01	0\\
207.01	0\\
208.01	0\\
209.01	0\\
210.01	0\\
211.01	0\\
212.01	0\\
213.01	0\\
214.01	0\\
215.01	0\\
216.01	0\\
217.01	0\\
218.01	0\\
219.01	0\\
220.01	0\\
221.01	0\\
222.01	0\\
223.01	0\\
224.01	0\\
225.01	0\\
226.01	0\\
227.01	0\\
228.01	0\\
229.01	0\\
230.01	0\\
231.01	0\\
232.01	0\\
233.01	0\\
234.01	0\\
235.01	0\\
236.01	0\\
237.01	0\\
238.01	0\\
239.01	0\\
240.01	0\\
241.01	0\\
242.01	0\\
243.01	0\\
244.01	0\\
245.01	0\\
246.01	0\\
247.01	0\\
248.01	0\\
249.01	0\\
250.01	0\\
251.01	0\\
252.01	0\\
253.01	0\\
254.01	0\\
255.01	0\\
256.01	0\\
257.01	0\\
258.01	0\\
259.01	0\\
260.01	0\\
261.01	0\\
262.01	0\\
263.01	0\\
264.01	0\\
265.01	0\\
266.01	0\\
267.01	0\\
268.01	0\\
269.01	0\\
270.01	0\\
271.01	0\\
272.01	0\\
273.01	0\\
274.01	0\\
275.01	0\\
276.01	0\\
277.01	0\\
278.01	0\\
279.01	0\\
280.01	0\\
281.01	0\\
282.01	0\\
283.01	0\\
284.01	0\\
285.01	0\\
286.01	0\\
287.01	0\\
288.01	0\\
289.01	0\\
290.01	0\\
291.01	0\\
292.01	0\\
293.01	0\\
294.01	0\\
295.01	0\\
296.01	0\\
297.01	0\\
298.01	0\\
299.01	0\\
300.01	0\\
301.01	0\\
302.01	0\\
303.01	0\\
304.01	0\\
305.01	0\\
306.01	0\\
307.01	0\\
308.01	0\\
309.01	0\\
310.01	0\\
311.01	0\\
312.01	0\\
313.01	0\\
314.01	0\\
315.01	0\\
316.01	0\\
317.01	0\\
318.01	0\\
319.01	0\\
320.01	0\\
321.01	0\\
322.01	0\\
323.01	0\\
324.01	0\\
325.01	0\\
326.01	0\\
327.01	0\\
328.01	0\\
329.01	0\\
330.01	0\\
331.01	0\\
332.01	0\\
333.01	0\\
334.01	0\\
335.01	0\\
336.01	0\\
337.01	0\\
338.01	0\\
339.01	0\\
340.01	0\\
341.01	0\\
342.01	0\\
343.01	0\\
344.01	0\\
345.01	0\\
346.01	0\\
347.01	0\\
348.01	0\\
349.01	0\\
350.01	0\\
351.01	0\\
352.01	0\\
353.01	0\\
354.01	0\\
355.01	0\\
356.01	0\\
357.01	0\\
358.01	0\\
359.01	0\\
360.01	0\\
361.01	0\\
362.01	0\\
363.01	0\\
364.01	0\\
365.01	0\\
366.01	0\\
367.01	0\\
368.01	0\\
369.01	0\\
370.01	0\\
371.01	0\\
372.01	0\\
373.01	0\\
374.01	0\\
375.01	0\\
376.01	0\\
377.01	0\\
378.01	0\\
379.01	0\\
380.01	0\\
381.01	0\\
382.01	0\\
383.01	0\\
384.01	0\\
385.01	0\\
386.01	0\\
387.01	0\\
388.01	0\\
389.01	0\\
390.01	0\\
391.01	0\\
392.01	0\\
393.01	0\\
394.01	0\\
395.01	0\\
396.01	0\\
397.01	0\\
398.01	0\\
399.01	0\\
400.01	0\\
401.01	0\\
402.01	0\\
403.01	0\\
404.01	0\\
405.01	0\\
406.01	0\\
407.01	0\\
408.01	0\\
409.01	0\\
410.01	0\\
411.01	0\\
412.01	0\\
413.01	0\\
414.01	0\\
415.01	0\\
416.01	0\\
417.01	0\\
418.01	0\\
419.01	0\\
420.01	0\\
421.01	0\\
422.01	0\\
423.01	0\\
424.01	0\\
425.01	0\\
426.01	0\\
427.01	0\\
428.01	0\\
429.01	0\\
430.01	0\\
431.01	0\\
432.01	0\\
433.01	0\\
434.01	0\\
435.01	0\\
436.01	0\\
437.01	0\\
438.01	0\\
439.01	0\\
440.01	0\\
441.01	0\\
442.01	0\\
443.01	0\\
444.01	0\\
445.01	0\\
446.01	0\\
447.01	0\\
448.01	0\\
449.01	0\\
450.01	0\\
451.01	1.73472347597681e-18\\
452.01	0\\
453.01	0\\
454.01	0\\
455.01	0\\
456.01	0\\
457.01	0\\
458.01	0\\
459.01	0\\
460.01	0\\
461.01	0\\
462.01	0\\
463.01	0\\
464.01	0\\
465.01	0\\
466.01	0\\
467.01	0\\
468.01	0\\
469.01	1.73472347597681e-18\\
470.01	0\\
471.01	0\\
472.01	0\\
473.01	0\\
474.01	0\\
475.01	0\\
476.01	1.73472347597681e-18\\
477.01	0\\
478.01	0\\
479.01	0\\
480.01	0\\
481.01	0\\
482.01	0\\
483.01	0\\
484.01	0\\
485.01	0\\
486.01	0\\
487.01	0\\
488.01	0\\
489.01	0\\
490.01	0\\
491.01	0\\
492.01	0\\
493.01	0\\
494.01	0\\
495.01	0\\
496.01	0\\
497.01	0\\
498.01	0\\
499.01	0\\
500.01	0\\
501.01	0\\
502.01	0\\
503.01	0\\
504.01	0\\
505.01	0\\
506.01	0\\
507.01	0\\
508.01	0\\
509.01	0\\
510.01	0\\
511.01	0\\
512.01	1.73472347597681e-18\\
513.01	0\\
514.01	0\\
515.01	0\\
516.01	0\\
517.01	0\\
518.01	0\\
519.01	0\\
520.01	0\\
521.01	0\\
522.01	0\\
523.01	1.73472347597681e-18\\
524.01	0\\
525.01	0\\
526.01	0\\
527.01	0\\
528.01	0\\
529.01	0\\
530.01	0\\
531.01	0\\
532.01	0\\
533.01	0\\
534.01	0\\
535.01	0\\
536.01	1.73472347597681e-18\\
537.01	1.73472347597681e-18\\
538.01	1.73472347597681e-18\\
539.01	0\\
540.01	0\\
541.01	0\\
542.01	0\\
543.01	0\\
544.01	0\\
545.01	1.73472347597681e-18\\
546.01	1.73472347597681e-18\\
547.01	0\\
548.01	0\\
549.01	0\\
550.01	0\\
551.01	0\\
552.01	0\\
553.01	0\\
554.01	0\\
555.01	0\\
556.01	1.73472347597681e-18\\
557.01	0\\
558.01	0\\
559.01	0\\
560.01	1.73472347597681e-18\\
561.01	1.73472347597681e-18\\
562.01	0\\
563.01	0\\
564.01	0\\
565.01	0\\
566.01	0\\
567.01	0\\
568.01	0\\
569.01	0\\
570.01	0\\
571.01	0\\
572.01	0\\
573.01	0\\
574.01	0\\
575.01	0\\
576.01	0\\
577.01	0\\
578.01	1.73472347597681e-18\\
579.01	0\\
580.01	0\\
581.01	0\\
582.01	0\\
583.01	1.73472347597681e-18\\
584.01	0\\
585.01	0\\
586.01	0\\
587.01	0\\
588.01	0\\
589.01	0\\
590.01	0\\
591.01	0\\
592.01	0\\
593.01	0\\
594.01	0\\
595.01	0\\
596.01	0\\
597.01	0\\
598.01	0\\
599.01	0\\
599.02	0\\
599.03	0\\
599.04	0\\
599.05	0\\
599.06	0\\
599.07	0\\
599.08	0\\
599.09	0\\
599.1	0\\
599.11	0\\
599.12	0\\
599.13	0\\
599.14	0\\
599.15	0\\
599.16	0\\
599.17	0\\
599.18	0\\
599.19	0\\
599.2	0\\
599.21	0\\
599.22	0\\
599.23	0\\
599.24	0\\
599.25	0\\
599.26	0\\
599.27	0\\
599.28	0\\
599.29	0\\
599.3	0\\
599.31	0\\
599.32	0\\
599.33	0\\
599.34	0\\
599.35	0\\
599.36	0\\
599.37	0\\
599.38	0\\
599.39	0\\
599.4	0\\
599.41	0\\
599.42	0\\
599.43	0\\
599.44	0\\
599.45	0\\
599.46	0\\
599.47	0\\
599.48	0\\
599.49	0\\
599.5	0\\
599.51	0\\
599.52	0\\
599.53	0\\
599.54	0\\
599.55	0\\
599.56	0\\
599.57	0\\
599.58	0\\
599.59	0\\
599.6	0\\
599.61	0\\
599.62	0\\
599.63	0\\
599.64	0\\
599.65	0\\
599.66	0\\
599.67	0\\
599.68	0\\
599.69	0\\
599.7	0\\
599.71	0\\
599.72	0\\
599.73	0\\
599.74	0\\
599.75	0\\
599.76	0\\
599.77	0\\
599.78	0\\
599.79	0\\
599.8	0\\
599.81	0\\
599.82	0\\
599.83	0\\
599.84	0\\
599.85	0\\
599.86	0\\
599.87	0\\
599.88	0\\
599.89	0\\
599.9	0\\
599.91	0\\
599.92	0\\
599.93	0\\
599.94	0\\
599.95	0\\
599.96	0\\
599.97	0\\
599.98	0\\
599.99	0\\
600	0\\
};
\addplot [color=mycolor4,solid,forget plot]
  table[row sep=crcr]{%
0.01	0\\
1.01	0\\
2.01	0\\
3.01	0\\
4.01	0\\
5.01	0\\
6.01	0\\
7.01	0\\
8.01	0\\
9.01	0\\
10.01	0\\
11.01	0\\
12.01	0\\
13.01	0\\
14.01	0\\
15.01	0\\
16.01	0\\
17.01	0\\
18.01	0\\
19.01	0\\
20.01	0\\
21.01	0\\
22.01	0\\
23.01	0\\
24.01	0\\
25.01	0\\
26.01	0\\
27.01	0\\
28.01	0\\
29.01	0\\
30.01	0\\
31.01	0\\
32.01	0\\
33.01	0\\
34.01	0\\
35.01	0\\
36.01	0\\
37.01	0\\
38.01	0\\
39.01	0\\
40.01	0\\
41.01	0\\
42.01	0\\
43.01	0\\
44.01	0\\
45.01	0\\
46.01	0\\
47.01	0\\
48.01	0\\
49.01	0\\
50.01	0\\
51.01	0\\
52.01	0\\
53.01	0\\
54.01	0\\
55.01	0\\
56.01	0\\
57.01	0\\
58.01	0\\
59.01	0\\
60.01	0\\
61.01	0\\
62.01	0\\
63.01	0\\
64.01	0\\
65.01	0\\
66.01	0\\
67.01	0\\
68.01	0\\
69.01	0\\
70.01	0\\
71.01	0\\
72.01	0\\
73.01	0\\
74.01	0\\
75.01	0\\
76.01	0\\
77.01	0\\
78.01	0\\
79.01	0\\
80.01	0\\
81.01	0\\
82.01	0\\
83.01	0\\
84.01	0\\
85.01	0\\
86.01	0\\
87.01	0\\
88.01	0\\
89.01	0\\
90.01	0\\
91.01	0\\
92.01	0\\
93.01	0\\
94.01	0\\
95.01	0\\
96.01	0\\
97.01	0\\
98.01	0\\
99.01	0\\
100.01	0\\
101.01	0\\
102.01	0\\
103.01	0\\
104.01	0\\
105.01	0\\
106.01	0\\
107.01	0\\
108.01	0\\
109.01	0\\
110.01	0\\
111.01	0\\
112.01	0\\
113.01	0\\
114.01	0\\
115.01	0\\
116.01	0\\
117.01	0\\
118.01	0\\
119.01	0\\
120.01	0\\
121.01	0\\
122.01	0\\
123.01	0\\
124.01	0\\
125.01	0\\
126.01	0\\
127.01	0\\
128.01	0\\
129.01	0\\
130.01	0\\
131.01	0\\
132.01	0\\
133.01	0\\
134.01	0\\
135.01	0\\
136.01	0\\
137.01	0\\
138.01	0\\
139.01	0\\
140.01	0\\
141.01	0\\
142.01	0\\
143.01	0\\
144.01	0\\
145.01	0\\
146.01	0\\
147.01	0\\
148.01	0\\
149.01	0\\
150.01	0\\
151.01	0\\
152.01	0\\
153.01	0\\
154.01	0\\
155.01	0\\
156.01	0\\
157.01	0\\
158.01	0\\
159.01	0\\
160.01	0\\
161.01	0\\
162.01	0\\
163.01	0\\
164.01	0\\
165.01	0\\
166.01	0\\
167.01	0\\
168.01	0\\
169.01	0\\
170.01	0\\
171.01	0\\
172.01	0\\
173.01	0\\
174.01	0\\
175.01	0\\
176.01	0\\
177.01	0\\
178.01	0\\
179.01	0\\
180.01	0\\
181.01	0\\
182.01	0\\
183.01	0\\
184.01	0\\
185.01	0\\
186.01	0\\
187.01	0\\
188.01	0\\
189.01	0\\
190.01	0\\
191.01	0\\
192.01	0\\
193.01	0\\
194.01	0\\
195.01	0\\
196.01	0\\
197.01	0\\
198.01	0\\
199.01	0\\
200.01	0\\
201.01	0\\
202.01	0\\
203.01	0\\
204.01	0\\
205.01	0\\
206.01	0\\
207.01	0\\
208.01	0\\
209.01	0\\
210.01	0\\
211.01	0\\
212.01	0\\
213.01	0\\
214.01	0\\
215.01	0\\
216.01	0\\
217.01	0\\
218.01	0\\
219.01	0\\
220.01	0\\
221.01	0\\
222.01	0\\
223.01	0\\
224.01	0\\
225.01	0\\
226.01	0\\
227.01	0\\
228.01	0\\
229.01	0\\
230.01	0\\
231.01	0\\
232.01	0\\
233.01	0\\
234.01	0\\
235.01	0\\
236.01	0\\
237.01	0\\
238.01	0\\
239.01	0\\
240.01	0\\
241.01	0\\
242.01	0\\
243.01	0\\
244.01	0\\
245.01	0\\
246.01	0\\
247.01	0\\
248.01	0\\
249.01	0\\
250.01	0\\
251.01	0\\
252.01	0\\
253.01	0\\
254.01	0\\
255.01	0\\
256.01	0\\
257.01	0\\
258.01	0\\
259.01	0\\
260.01	0\\
261.01	0\\
262.01	0\\
263.01	0\\
264.01	0\\
265.01	0\\
266.01	0\\
267.01	0\\
268.01	0\\
269.01	0\\
270.01	0\\
271.01	0\\
272.01	0\\
273.01	0\\
274.01	0\\
275.01	0\\
276.01	0\\
277.01	0\\
278.01	0\\
279.01	0\\
280.01	0\\
281.01	0\\
282.01	0\\
283.01	0\\
284.01	0\\
285.01	0\\
286.01	0\\
287.01	0\\
288.01	0\\
289.01	0\\
290.01	0\\
291.01	0\\
292.01	0\\
293.01	0\\
294.01	0\\
295.01	0\\
296.01	0\\
297.01	0\\
298.01	0\\
299.01	0\\
300.01	0\\
301.01	0\\
302.01	0\\
303.01	0\\
304.01	0\\
305.01	0\\
306.01	0\\
307.01	0\\
308.01	0\\
309.01	0\\
310.01	0\\
311.01	0\\
312.01	0\\
313.01	0\\
314.01	0\\
315.01	0\\
316.01	0\\
317.01	0\\
318.01	0\\
319.01	0\\
320.01	0\\
321.01	0\\
322.01	0\\
323.01	0\\
324.01	0\\
325.01	0\\
326.01	0\\
327.01	0\\
328.01	0\\
329.01	0\\
330.01	0\\
331.01	0\\
332.01	0\\
333.01	0\\
334.01	0\\
335.01	0\\
336.01	0\\
337.01	0\\
338.01	0\\
339.01	0\\
340.01	0\\
341.01	0\\
342.01	0\\
343.01	0\\
344.01	0\\
345.01	0\\
346.01	0\\
347.01	0\\
348.01	0\\
349.01	0\\
350.01	0\\
351.01	0\\
352.01	0\\
353.01	0\\
354.01	0\\
355.01	0\\
356.01	0\\
357.01	0\\
358.01	0\\
359.01	0\\
360.01	0\\
361.01	0\\
362.01	0\\
363.01	0\\
364.01	0\\
365.01	0\\
366.01	0\\
367.01	0\\
368.01	0\\
369.01	0\\
370.01	0\\
371.01	0\\
372.01	0\\
373.01	0\\
374.01	0\\
375.01	0\\
376.01	0\\
377.01	0\\
378.01	0\\
379.01	0\\
380.01	0\\
381.01	0\\
382.01	0\\
383.01	0\\
384.01	0\\
385.01	0\\
386.01	0\\
387.01	0\\
388.01	0\\
389.01	0\\
390.01	0\\
391.01	0\\
392.01	0\\
393.01	0\\
394.01	0\\
395.01	0\\
396.01	0\\
397.01	0\\
398.01	0\\
399.01	0\\
400.01	0\\
401.01	0\\
402.01	0\\
403.01	0\\
404.01	0\\
405.01	0\\
406.01	0\\
407.01	0\\
408.01	0\\
409.01	0\\
410.01	0\\
411.01	0\\
412.01	0\\
413.01	0\\
414.01	0\\
415.01	0\\
416.01	0\\
417.01	0\\
418.01	0\\
419.01	0\\
420.01	0\\
421.01	0\\
422.01	0\\
423.01	0\\
424.01	0\\
425.01	0\\
426.01	0\\
427.01	0\\
428.01	0\\
429.01	0\\
430.01	0\\
431.01	0\\
432.01	0\\
433.01	0\\
434.01	0\\
435.01	0\\
436.01	0\\
437.01	0\\
438.01	0\\
439.01	0\\
440.01	0\\
441.01	0\\
442.01	0\\
443.01	0\\
444.01	0\\
445.01	0\\
446.01	0\\
447.01	0\\
448.01	0\\
449.01	0\\
450.01	0\\
451.01	1.73472347597681e-18\\
452.01	0\\
453.01	0\\
454.01	0\\
455.01	0\\
456.01	0\\
457.01	0\\
458.01	0\\
459.01	0\\
460.01	0\\
461.01	0\\
462.01	0\\
463.01	0\\
464.01	0\\
465.01	0\\
466.01	0\\
467.01	0\\
468.01	0\\
469.01	1.73472347597681e-18\\
470.01	0\\
471.01	0\\
472.01	0\\
473.01	0\\
474.01	0\\
475.01	0\\
476.01	1.73472347597681e-18\\
477.01	0\\
478.01	0\\
479.01	0\\
480.01	0\\
481.01	0\\
482.01	0\\
483.01	0\\
484.01	0\\
485.01	0\\
486.01	0\\
487.01	0\\
488.01	0\\
489.01	0\\
490.01	0\\
491.01	0\\
492.01	0\\
493.01	0\\
494.01	0\\
495.01	0\\
496.01	0\\
497.01	0\\
498.01	0\\
499.01	0\\
500.01	0\\
501.01	0\\
502.01	0\\
503.01	0\\
504.01	0\\
505.01	0\\
506.01	0\\
507.01	0\\
508.01	0\\
509.01	0\\
510.01	0\\
511.01	0\\
512.01	1.73472347597681e-18\\
513.01	0\\
514.01	0\\
515.01	0\\
516.01	0\\
517.01	0\\
518.01	0\\
519.01	0\\
520.01	0\\
521.01	0\\
522.01	0\\
523.01	1.73472347597681e-18\\
524.01	0\\
525.01	0\\
526.01	0\\
527.01	0\\
528.01	0\\
529.01	0\\
530.01	0\\
531.01	0\\
532.01	0\\
533.01	0\\
534.01	0\\
535.01	0\\
536.01	1.73472347597681e-18\\
537.01	1.73472347597681e-18\\
538.01	1.73472347597681e-18\\
539.01	0\\
540.01	0\\
541.01	0\\
542.01	0\\
543.01	0\\
544.01	0\\
545.01	1.73472347597681e-18\\
546.01	1.73472347597681e-18\\
547.01	0\\
548.01	0\\
549.01	0\\
550.01	0\\
551.01	0\\
552.01	0\\
553.01	0\\
554.01	0\\
555.01	0\\
556.01	1.73472347597681e-18\\
557.01	0\\
558.01	0\\
559.01	0\\
560.01	1.73472347597681e-18\\
561.01	1.73472347597681e-18\\
562.01	0\\
563.01	0\\
564.01	0\\
565.01	0\\
566.01	0\\
567.01	0\\
568.01	0\\
569.01	0\\
570.01	0\\
571.01	0\\
572.01	0\\
573.01	0\\
574.01	0\\
575.01	0\\
576.01	0\\
577.01	0\\
578.01	1.73472347597681e-18\\
579.01	0\\
580.01	0\\
581.01	0\\
582.01	0\\
583.01	1.73472347597681e-18\\
584.01	0\\
585.01	0\\
586.01	0\\
587.01	0\\
588.01	0\\
589.01	0\\
590.01	0\\
591.01	0\\
592.01	0\\
593.01	0\\
594.01	0\\
595.01	0\\
596.01	0\\
597.01	0\\
598.01	0\\
599.01	0\\
599.02	0\\
599.03	0\\
599.04	0\\
599.05	0\\
599.06	0\\
599.07	0\\
599.08	0\\
599.09	0\\
599.1	0\\
599.11	0\\
599.12	0\\
599.13	0\\
599.14	0\\
599.15	0\\
599.16	0\\
599.17	0\\
599.18	0\\
599.19	0\\
599.2	0\\
599.21	0\\
599.22	0\\
599.23	0\\
599.24	0\\
599.25	0\\
599.26	0\\
599.27	0\\
599.28	0\\
599.29	0\\
599.3	0\\
599.31	0\\
599.32	0\\
599.33	0\\
599.34	0\\
599.35	0\\
599.36	0\\
599.37	0\\
599.38	0\\
599.39	0\\
599.4	0\\
599.41	0\\
599.42	0\\
599.43	0\\
599.44	0\\
599.45	0\\
599.46	0\\
599.47	0\\
599.48	0\\
599.49	0\\
599.5	0\\
599.51	0\\
599.52	0\\
599.53	0\\
599.54	0\\
599.55	0\\
599.56	0\\
599.57	0\\
599.58	0\\
599.59	0\\
599.6	0\\
599.61	0\\
599.62	0\\
599.63	0\\
599.64	0\\
599.65	0\\
599.66	0\\
599.67	0\\
599.68	0\\
599.69	0\\
599.7	0\\
599.71	0\\
599.72	0\\
599.73	0\\
599.74	0\\
599.75	0\\
599.76	0\\
599.77	0\\
599.78	0\\
599.79	0\\
599.8	0\\
599.81	0\\
599.82	0\\
599.83	0\\
599.84	0\\
599.85	0\\
599.86	0\\
599.87	0\\
599.88	0\\
599.89	0\\
599.9	0\\
599.91	0\\
599.92	0\\
599.93	0\\
599.94	0\\
599.95	0\\
599.96	0\\
599.97	0\\
599.98	0\\
599.99	0\\
600	0\\
};
\addplot [color=mycolor5,solid,forget plot]
  table[row sep=crcr]{%
0.01	0\\
1.01	0\\
2.01	0\\
3.01	0\\
4.01	0\\
5.01	0\\
6.01	0\\
7.01	0\\
8.01	0\\
9.01	0\\
10.01	0\\
11.01	0\\
12.01	0\\
13.01	0\\
14.01	0\\
15.01	0\\
16.01	0\\
17.01	0\\
18.01	0\\
19.01	0\\
20.01	0\\
21.01	0\\
22.01	0\\
23.01	0\\
24.01	0\\
25.01	0\\
26.01	0\\
27.01	0\\
28.01	0\\
29.01	0\\
30.01	0\\
31.01	0\\
32.01	0\\
33.01	0\\
34.01	0\\
35.01	0\\
36.01	0\\
37.01	0\\
38.01	0\\
39.01	0\\
40.01	0\\
41.01	0\\
42.01	0\\
43.01	0\\
44.01	0\\
45.01	0\\
46.01	0\\
47.01	0\\
48.01	0\\
49.01	0\\
50.01	0\\
51.01	0\\
52.01	0\\
53.01	0\\
54.01	0\\
55.01	0\\
56.01	0\\
57.01	0\\
58.01	0\\
59.01	0\\
60.01	0\\
61.01	0\\
62.01	0\\
63.01	0\\
64.01	0\\
65.01	0\\
66.01	0\\
67.01	0\\
68.01	0\\
69.01	0\\
70.01	0\\
71.01	0\\
72.01	0\\
73.01	0\\
74.01	0\\
75.01	0\\
76.01	0\\
77.01	0\\
78.01	0\\
79.01	0\\
80.01	0\\
81.01	0\\
82.01	0\\
83.01	0\\
84.01	0\\
85.01	0\\
86.01	0\\
87.01	0\\
88.01	0\\
89.01	0\\
90.01	0\\
91.01	0\\
92.01	0\\
93.01	0\\
94.01	0\\
95.01	0\\
96.01	0\\
97.01	0\\
98.01	0\\
99.01	0\\
100.01	0\\
101.01	0\\
102.01	0\\
103.01	0\\
104.01	0\\
105.01	0\\
106.01	0\\
107.01	0\\
108.01	0\\
109.01	0\\
110.01	0\\
111.01	0\\
112.01	0\\
113.01	0\\
114.01	0\\
115.01	0\\
116.01	0\\
117.01	0\\
118.01	0\\
119.01	0\\
120.01	0\\
121.01	0\\
122.01	0\\
123.01	0\\
124.01	0\\
125.01	0\\
126.01	0\\
127.01	0\\
128.01	0\\
129.01	0\\
130.01	0\\
131.01	0\\
132.01	0\\
133.01	0\\
134.01	0\\
135.01	0\\
136.01	0\\
137.01	0\\
138.01	0\\
139.01	0\\
140.01	0\\
141.01	0\\
142.01	0\\
143.01	0\\
144.01	0\\
145.01	0\\
146.01	0\\
147.01	0\\
148.01	0\\
149.01	0\\
150.01	0\\
151.01	0\\
152.01	0\\
153.01	0\\
154.01	0\\
155.01	0\\
156.01	0\\
157.01	0\\
158.01	0\\
159.01	0\\
160.01	0\\
161.01	0\\
162.01	0\\
163.01	0\\
164.01	0\\
165.01	0\\
166.01	0\\
167.01	0\\
168.01	0\\
169.01	0\\
170.01	0\\
171.01	0\\
172.01	0\\
173.01	0\\
174.01	0\\
175.01	0\\
176.01	0\\
177.01	0\\
178.01	0\\
179.01	0\\
180.01	0\\
181.01	0\\
182.01	0\\
183.01	0\\
184.01	0\\
185.01	0\\
186.01	0\\
187.01	0\\
188.01	0\\
189.01	0\\
190.01	0\\
191.01	0\\
192.01	0\\
193.01	0\\
194.01	0\\
195.01	0\\
196.01	0\\
197.01	0\\
198.01	0\\
199.01	0\\
200.01	0\\
201.01	0\\
202.01	0\\
203.01	0\\
204.01	0\\
205.01	0\\
206.01	0\\
207.01	0\\
208.01	0\\
209.01	0\\
210.01	0\\
211.01	0\\
212.01	0\\
213.01	0\\
214.01	0\\
215.01	0\\
216.01	0\\
217.01	0\\
218.01	0\\
219.01	0\\
220.01	0\\
221.01	0\\
222.01	0\\
223.01	0\\
224.01	0\\
225.01	0\\
226.01	0\\
227.01	0\\
228.01	0\\
229.01	0\\
230.01	0\\
231.01	0\\
232.01	0\\
233.01	0\\
234.01	0\\
235.01	0\\
236.01	0\\
237.01	0\\
238.01	0\\
239.01	0\\
240.01	0\\
241.01	0\\
242.01	0\\
243.01	0\\
244.01	0\\
245.01	0\\
246.01	0\\
247.01	0\\
248.01	0\\
249.01	0\\
250.01	0\\
251.01	0\\
252.01	0\\
253.01	0\\
254.01	0\\
255.01	0\\
256.01	0\\
257.01	0\\
258.01	0\\
259.01	0\\
260.01	0\\
261.01	0\\
262.01	0\\
263.01	0\\
264.01	0\\
265.01	0\\
266.01	0\\
267.01	0\\
268.01	0\\
269.01	0\\
270.01	0\\
271.01	0\\
272.01	0\\
273.01	0\\
274.01	0\\
275.01	0\\
276.01	0\\
277.01	0\\
278.01	0\\
279.01	0\\
280.01	0\\
281.01	0\\
282.01	0\\
283.01	0\\
284.01	0\\
285.01	0\\
286.01	0\\
287.01	0\\
288.01	0\\
289.01	0\\
290.01	0\\
291.01	0\\
292.01	0\\
293.01	0\\
294.01	0\\
295.01	0\\
296.01	0\\
297.01	0\\
298.01	0\\
299.01	0\\
300.01	0\\
301.01	0\\
302.01	0\\
303.01	0\\
304.01	0\\
305.01	0\\
306.01	0\\
307.01	0\\
308.01	0\\
309.01	0\\
310.01	0\\
311.01	0\\
312.01	0\\
313.01	0\\
314.01	0\\
315.01	0\\
316.01	0\\
317.01	0\\
318.01	0\\
319.01	0\\
320.01	0\\
321.01	0\\
322.01	0\\
323.01	0\\
324.01	0\\
325.01	0\\
326.01	0\\
327.01	0\\
328.01	0\\
329.01	0\\
330.01	0\\
331.01	0\\
332.01	0\\
333.01	0\\
334.01	0\\
335.01	0\\
336.01	0\\
337.01	0\\
338.01	0\\
339.01	0\\
340.01	0\\
341.01	0\\
342.01	0\\
343.01	0\\
344.01	0\\
345.01	0\\
346.01	0\\
347.01	0\\
348.01	0\\
349.01	0\\
350.01	0\\
351.01	0\\
352.01	0\\
353.01	0\\
354.01	0\\
355.01	0\\
356.01	0\\
357.01	0\\
358.01	0\\
359.01	0\\
360.01	0\\
361.01	0\\
362.01	0\\
363.01	0\\
364.01	0\\
365.01	0\\
366.01	0\\
367.01	0\\
368.01	0\\
369.01	0\\
370.01	0\\
371.01	0\\
372.01	0\\
373.01	0\\
374.01	0\\
375.01	0\\
376.01	0\\
377.01	0\\
378.01	0\\
379.01	0\\
380.01	0\\
381.01	0\\
382.01	0\\
383.01	0\\
384.01	0\\
385.01	0\\
386.01	0\\
387.01	0\\
388.01	0\\
389.01	0\\
390.01	0\\
391.01	0\\
392.01	0\\
393.01	0\\
394.01	0\\
395.01	0\\
396.01	0\\
397.01	0\\
398.01	0\\
399.01	0\\
400.01	0\\
401.01	0\\
402.01	0\\
403.01	0\\
404.01	0\\
405.01	0\\
406.01	0\\
407.01	0\\
408.01	0\\
409.01	0\\
410.01	0\\
411.01	0\\
412.01	0\\
413.01	0\\
414.01	0\\
415.01	0\\
416.01	0\\
417.01	0\\
418.01	0\\
419.01	0\\
420.01	0\\
421.01	0\\
422.01	0\\
423.01	0\\
424.01	0\\
425.01	0\\
426.01	0\\
427.01	0\\
428.01	0\\
429.01	0\\
430.01	0\\
431.01	0\\
432.01	0\\
433.01	0\\
434.01	0\\
435.01	0\\
436.01	0\\
437.01	0\\
438.01	0\\
439.01	0\\
440.01	0\\
441.01	0\\
442.01	0\\
443.01	0\\
444.01	0\\
445.01	0\\
446.01	0\\
447.01	0\\
448.01	0\\
449.01	0\\
450.01	0\\
451.01	1.73472347597681e-18\\
452.01	0\\
453.01	0\\
454.01	0\\
455.01	0\\
456.01	0\\
457.01	0\\
458.01	0\\
459.01	0\\
460.01	0\\
461.01	0\\
462.01	0\\
463.01	0\\
464.01	0\\
465.01	0\\
466.01	0\\
467.01	0\\
468.01	0\\
469.01	1.73472347597681e-18\\
470.01	0\\
471.01	0\\
472.01	0\\
473.01	0\\
474.01	0\\
475.01	0\\
476.01	1.73472347597681e-18\\
477.01	0\\
478.01	0\\
479.01	0\\
480.01	0\\
481.01	0\\
482.01	0\\
483.01	0\\
484.01	0\\
485.01	0\\
486.01	0\\
487.01	0\\
488.01	0\\
489.01	0\\
490.01	0\\
491.01	0\\
492.01	0\\
493.01	0\\
494.01	0\\
495.01	0\\
496.01	0\\
497.01	0\\
498.01	0\\
499.01	0\\
500.01	0\\
501.01	0\\
502.01	0\\
503.01	0\\
504.01	0\\
505.01	0\\
506.01	0\\
507.01	0\\
508.01	0\\
509.01	0\\
510.01	0\\
511.01	0\\
512.01	1.73472347597681e-18\\
513.01	0\\
514.01	0\\
515.01	0\\
516.01	0\\
517.01	0\\
518.01	0\\
519.01	0\\
520.01	0\\
521.01	0\\
522.01	0\\
523.01	1.73472347597681e-18\\
524.01	0\\
525.01	0\\
526.01	0\\
527.01	0\\
528.01	0\\
529.01	0\\
530.01	0\\
531.01	0\\
532.01	0\\
533.01	0\\
534.01	0\\
535.01	0\\
536.01	1.73472347597681e-18\\
537.01	1.73472347597681e-18\\
538.01	1.73472347597681e-18\\
539.01	0\\
540.01	0\\
541.01	0\\
542.01	0\\
543.01	0\\
544.01	0\\
545.01	1.73472347597681e-18\\
546.01	1.73472347597681e-18\\
547.01	0\\
548.01	0\\
549.01	0\\
550.01	0\\
551.01	0\\
552.01	0\\
553.01	0\\
554.01	0\\
555.01	0\\
556.01	1.73472347597681e-18\\
557.01	0\\
558.01	0\\
559.01	0\\
560.01	1.73472347597681e-18\\
561.01	1.73472347597681e-18\\
562.01	0\\
563.01	0\\
564.01	0\\
565.01	0\\
566.01	0\\
567.01	0\\
568.01	0\\
569.01	0\\
570.01	0\\
571.01	0\\
572.01	0\\
573.01	0\\
574.01	0\\
575.01	0\\
576.01	0\\
577.01	0\\
578.01	1.73472347597681e-18\\
579.01	0\\
580.01	0\\
581.01	0\\
582.01	0\\
583.01	1.73472347597681e-18\\
584.01	0\\
585.01	0\\
586.01	0\\
587.01	0\\
588.01	0\\
589.01	0\\
590.01	0\\
591.01	0\\
592.01	0\\
593.01	0\\
594.01	0\\
595.01	0\\
596.01	0\\
597.01	0\\
598.01	0\\
599.01	0\\
599.02	0\\
599.03	0\\
599.04	0\\
599.05	0\\
599.06	0\\
599.07	0\\
599.08	0\\
599.09	0\\
599.1	0\\
599.11	0\\
599.12	0\\
599.13	0\\
599.14	0\\
599.15	0\\
599.16	0\\
599.17	0\\
599.18	0\\
599.19	0\\
599.2	0\\
599.21	0\\
599.22	0\\
599.23	0\\
599.24	0\\
599.25	0\\
599.26	0\\
599.27	0\\
599.28	0\\
599.29	0\\
599.3	0\\
599.31	0\\
599.32	0\\
599.33	0\\
599.34	0\\
599.35	0\\
599.36	0\\
599.37	0\\
599.38	0\\
599.39	0\\
599.4	0\\
599.41	0\\
599.42	0\\
599.43	0\\
599.44	0\\
599.45	0\\
599.46	0\\
599.47	0\\
599.48	0\\
599.49	0\\
599.5	0\\
599.51	0\\
599.52	0\\
599.53	0\\
599.54	0\\
599.55	0\\
599.56	0\\
599.57	0\\
599.58	0\\
599.59	0\\
599.6	0\\
599.61	0\\
599.62	0\\
599.63	0\\
599.64	0\\
599.65	0\\
599.66	0\\
599.67	0\\
599.68	0\\
599.69	0\\
599.7	0\\
599.71	0\\
599.72	0\\
599.73	0\\
599.74	0\\
599.75	0\\
599.76	0\\
599.77	0\\
599.78	0\\
599.79	0\\
599.8	0\\
599.81	0\\
599.82	0\\
599.83	0\\
599.84	0\\
599.85	0\\
599.86	0\\
599.87	0\\
599.88	0\\
599.89	0\\
599.9	0\\
599.91	0\\
599.92	0\\
599.93	0\\
599.94	0\\
599.95	0\\
599.96	0\\
599.97	0\\
599.98	0\\
599.99	0\\
600	0\\
};
\addplot [color=mycolor6,solid,forget plot]
  table[row sep=crcr]{%
0.01	0\\
1.01	0\\
2.01	0\\
3.01	0\\
4.01	0\\
5.01	0\\
6.01	0\\
7.01	0\\
8.01	0\\
9.01	0\\
10.01	0\\
11.01	0\\
12.01	0\\
13.01	0\\
14.01	0\\
15.01	0\\
16.01	0\\
17.01	0\\
18.01	0\\
19.01	0\\
20.01	0\\
21.01	0\\
22.01	0\\
23.01	0\\
24.01	0\\
25.01	0\\
26.01	0\\
27.01	0\\
28.01	0\\
29.01	0\\
30.01	0\\
31.01	0\\
32.01	0\\
33.01	0\\
34.01	0\\
35.01	0\\
36.01	0\\
37.01	0\\
38.01	0\\
39.01	0\\
40.01	0\\
41.01	0\\
42.01	0\\
43.01	0\\
44.01	0\\
45.01	0\\
46.01	0\\
47.01	0\\
48.01	0\\
49.01	0\\
50.01	0\\
51.01	0\\
52.01	0\\
53.01	0\\
54.01	0\\
55.01	0\\
56.01	0\\
57.01	0\\
58.01	0\\
59.01	0\\
60.01	0\\
61.01	0\\
62.01	0\\
63.01	0\\
64.01	0\\
65.01	0\\
66.01	0\\
67.01	0\\
68.01	0\\
69.01	0\\
70.01	0\\
71.01	0\\
72.01	0\\
73.01	0\\
74.01	0\\
75.01	0\\
76.01	0\\
77.01	0\\
78.01	0\\
79.01	0\\
80.01	0\\
81.01	0\\
82.01	0\\
83.01	0\\
84.01	0\\
85.01	0\\
86.01	0\\
87.01	0\\
88.01	0\\
89.01	0\\
90.01	0\\
91.01	0\\
92.01	0\\
93.01	0\\
94.01	0\\
95.01	0\\
96.01	0\\
97.01	0\\
98.01	0\\
99.01	0\\
100.01	0\\
101.01	0\\
102.01	0\\
103.01	0\\
104.01	0\\
105.01	0\\
106.01	0\\
107.01	0\\
108.01	0\\
109.01	0\\
110.01	0\\
111.01	0\\
112.01	0\\
113.01	0\\
114.01	0\\
115.01	0\\
116.01	0\\
117.01	0\\
118.01	0\\
119.01	0\\
120.01	0\\
121.01	0\\
122.01	0\\
123.01	0\\
124.01	0\\
125.01	0\\
126.01	0\\
127.01	0\\
128.01	0\\
129.01	0\\
130.01	0\\
131.01	0\\
132.01	0\\
133.01	0\\
134.01	0\\
135.01	0\\
136.01	0\\
137.01	0\\
138.01	0\\
139.01	0\\
140.01	0\\
141.01	0\\
142.01	0\\
143.01	0\\
144.01	0\\
145.01	0\\
146.01	0\\
147.01	0\\
148.01	0\\
149.01	0\\
150.01	0\\
151.01	0\\
152.01	0\\
153.01	0\\
154.01	0\\
155.01	0\\
156.01	0\\
157.01	0\\
158.01	0\\
159.01	0\\
160.01	0\\
161.01	0\\
162.01	0\\
163.01	0\\
164.01	0\\
165.01	0\\
166.01	0\\
167.01	0\\
168.01	0\\
169.01	0\\
170.01	0\\
171.01	0\\
172.01	0\\
173.01	0\\
174.01	0\\
175.01	0\\
176.01	0\\
177.01	0\\
178.01	0\\
179.01	0\\
180.01	0\\
181.01	0\\
182.01	0\\
183.01	0\\
184.01	0\\
185.01	0\\
186.01	0\\
187.01	0\\
188.01	0\\
189.01	0\\
190.01	0\\
191.01	0\\
192.01	0\\
193.01	0\\
194.01	0\\
195.01	0\\
196.01	0\\
197.01	0\\
198.01	0\\
199.01	0\\
200.01	0\\
201.01	0\\
202.01	0\\
203.01	0\\
204.01	0\\
205.01	0\\
206.01	0\\
207.01	0\\
208.01	0\\
209.01	0\\
210.01	0\\
211.01	0\\
212.01	0\\
213.01	0\\
214.01	0\\
215.01	0\\
216.01	0\\
217.01	0\\
218.01	0\\
219.01	0\\
220.01	0\\
221.01	0\\
222.01	0\\
223.01	0\\
224.01	0\\
225.01	0\\
226.01	0\\
227.01	0\\
228.01	0\\
229.01	0\\
230.01	0\\
231.01	0\\
232.01	0\\
233.01	0\\
234.01	0\\
235.01	0\\
236.01	0\\
237.01	0\\
238.01	0\\
239.01	0\\
240.01	0\\
241.01	0\\
242.01	0\\
243.01	0\\
244.01	0\\
245.01	0\\
246.01	0\\
247.01	0\\
248.01	0\\
249.01	0\\
250.01	0\\
251.01	0\\
252.01	0\\
253.01	0\\
254.01	0\\
255.01	0\\
256.01	0\\
257.01	0\\
258.01	0\\
259.01	0\\
260.01	0\\
261.01	0\\
262.01	0\\
263.01	0\\
264.01	0\\
265.01	0\\
266.01	0\\
267.01	0\\
268.01	0\\
269.01	0\\
270.01	0\\
271.01	0\\
272.01	0\\
273.01	0\\
274.01	0\\
275.01	0\\
276.01	0\\
277.01	0\\
278.01	0\\
279.01	0\\
280.01	0\\
281.01	0\\
282.01	0\\
283.01	0\\
284.01	0\\
285.01	0\\
286.01	0\\
287.01	0\\
288.01	0\\
289.01	0\\
290.01	0\\
291.01	0\\
292.01	0\\
293.01	0\\
294.01	0\\
295.01	0\\
296.01	0\\
297.01	0\\
298.01	0\\
299.01	0\\
300.01	0\\
301.01	0\\
302.01	0\\
303.01	0\\
304.01	0\\
305.01	0\\
306.01	0\\
307.01	0\\
308.01	0\\
309.01	0\\
310.01	0\\
311.01	0\\
312.01	0\\
313.01	0\\
314.01	0\\
315.01	0\\
316.01	0\\
317.01	0\\
318.01	0\\
319.01	0\\
320.01	0\\
321.01	0\\
322.01	0\\
323.01	0\\
324.01	0\\
325.01	0\\
326.01	0\\
327.01	0\\
328.01	0\\
329.01	0\\
330.01	0\\
331.01	0\\
332.01	0\\
333.01	0\\
334.01	0\\
335.01	0\\
336.01	0\\
337.01	0\\
338.01	0\\
339.01	0\\
340.01	0\\
341.01	0\\
342.01	0\\
343.01	0\\
344.01	0\\
345.01	0\\
346.01	0\\
347.01	0\\
348.01	0\\
349.01	0\\
350.01	0\\
351.01	0\\
352.01	0\\
353.01	0\\
354.01	0\\
355.01	0\\
356.01	0\\
357.01	0\\
358.01	0\\
359.01	0\\
360.01	0\\
361.01	0\\
362.01	0\\
363.01	0\\
364.01	0\\
365.01	0\\
366.01	0\\
367.01	0\\
368.01	0\\
369.01	0\\
370.01	0\\
371.01	0\\
372.01	0\\
373.01	0\\
374.01	0\\
375.01	0\\
376.01	0\\
377.01	0\\
378.01	0\\
379.01	0\\
380.01	0\\
381.01	0\\
382.01	0\\
383.01	0\\
384.01	0\\
385.01	0\\
386.01	0\\
387.01	0\\
388.01	0\\
389.01	0\\
390.01	0\\
391.01	0\\
392.01	0\\
393.01	0\\
394.01	0\\
395.01	0\\
396.01	0\\
397.01	0\\
398.01	0\\
399.01	0\\
400.01	0\\
401.01	0\\
402.01	0\\
403.01	0\\
404.01	0\\
405.01	0\\
406.01	0\\
407.01	0\\
408.01	0\\
409.01	0\\
410.01	0\\
411.01	0\\
412.01	0\\
413.01	0\\
414.01	0\\
415.01	0\\
416.01	0\\
417.01	0\\
418.01	0\\
419.01	0\\
420.01	0\\
421.01	0\\
422.01	0\\
423.01	0\\
424.01	0\\
425.01	0\\
426.01	0\\
427.01	0\\
428.01	0\\
429.01	0\\
430.01	0\\
431.01	0\\
432.01	0\\
433.01	0\\
434.01	0\\
435.01	0\\
436.01	0\\
437.01	0\\
438.01	0\\
439.01	0\\
440.01	0\\
441.01	0\\
442.01	0\\
443.01	0\\
444.01	0\\
445.01	0\\
446.01	0\\
447.01	0\\
448.01	0\\
449.01	0\\
450.01	0\\
451.01	1.73472347597681e-18\\
452.01	0\\
453.01	0\\
454.01	0\\
455.01	0\\
456.01	0\\
457.01	0\\
458.01	0\\
459.01	0\\
460.01	0\\
461.01	0\\
462.01	0\\
463.01	0\\
464.01	0\\
465.01	0\\
466.01	0\\
467.01	0\\
468.01	0\\
469.01	1.73472347597681e-18\\
470.01	0\\
471.01	0\\
472.01	0\\
473.01	0\\
474.01	0\\
475.01	0\\
476.01	1.73472347597681e-18\\
477.01	0\\
478.01	0\\
479.01	0\\
480.01	0\\
481.01	0\\
482.01	0\\
483.01	0\\
484.01	0\\
485.01	0\\
486.01	0\\
487.01	0\\
488.01	0\\
489.01	0\\
490.01	0\\
491.01	0\\
492.01	0\\
493.01	0\\
494.01	0\\
495.01	0\\
496.01	0\\
497.01	0\\
498.01	0\\
499.01	0\\
500.01	0\\
501.01	0\\
502.01	0\\
503.01	0\\
504.01	0\\
505.01	0\\
506.01	0\\
507.01	0\\
508.01	0\\
509.01	0\\
510.01	0\\
511.01	0\\
512.01	1.73472347597681e-18\\
513.01	0\\
514.01	0\\
515.01	0\\
516.01	0\\
517.01	0\\
518.01	0\\
519.01	0\\
520.01	0\\
521.01	0\\
522.01	0\\
523.01	1.73472347597681e-18\\
524.01	0\\
525.01	0\\
526.01	0\\
527.01	0\\
528.01	0\\
529.01	0\\
530.01	0\\
531.01	0\\
532.01	0\\
533.01	0\\
534.01	0\\
535.01	0\\
536.01	1.73472347597681e-18\\
537.01	1.73472347597681e-18\\
538.01	1.73472347597681e-18\\
539.01	0\\
540.01	0\\
541.01	0\\
542.01	0\\
543.01	0\\
544.01	0\\
545.01	1.73472347597681e-18\\
546.01	1.73472347597681e-18\\
547.01	0\\
548.01	0\\
549.01	0\\
550.01	0\\
551.01	0\\
552.01	0\\
553.01	0\\
554.01	0\\
555.01	0\\
556.01	1.73472347597681e-18\\
557.01	0\\
558.01	0\\
559.01	0\\
560.01	1.73472347597681e-18\\
561.01	1.73472347597681e-18\\
562.01	0\\
563.01	0\\
564.01	0\\
565.01	0\\
566.01	0\\
567.01	0\\
568.01	0\\
569.01	0\\
570.01	0\\
571.01	0\\
572.01	0\\
573.01	0\\
574.01	0\\
575.01	0\\
576.01	0\\
577.01	0\\
578.01	1.73472347597681e-18\\
579.01	0\\
580.01	0\\
581.01	0\\
582.01	0\\
583.01	1.73472347597681e-18\\
584.01	0\\
585.01	0\\
586.01	0\\
587.01	0\\
588.01	0\\
589.01	0\\
590.01	0\\
591.01	0\\
592.01	0\\
593.01	0\\
594.01	0\\
595.01	0\\
596.01	0\\
597.01	0\\
598.01	0\\
599.01	0\\
599.02	0\\
599.03	0\\
599.04	0\\
599.05	0\\
599.06	0\\
599.07	0\\
599.08	0\\
599.09	0\\
599.1	0\\
599.11	0\\
599.12	0\\
599.13	0\\
599.14	0\\
599.15	0\\
599.16	0\\
599.17	0\\
599.18	0\\
599.19	0\\
599.2	0\\
599.21	0\\
599.22	0\\
599.23	0\\
599.24	0\\
599.25	0\\
599.26	0\\
599.27	0\\
599.28	0\\
599.29	0\\
599.3	0\\
599.31	0\\
599.32	0\\
599.33	0\\
599.34	0\\
599.35	0\\
599.36	0\\
599.37	0\\
599.38	0\\
599.39	0\\
599.4	0\\
599.41	0\\
599.42	0\\
599.43	0\\
599.44	0\\
599.45	0\\
599.46	0\\
599.47	0\\
599.48	0\\
599.49	0\\
599.5	0\\
599.51	0\\
599.52	0\\
599.53	0\\
599.54	0\\
599.55	0\\
599.56	0\\
599.57	0\\
599.58	0\\
599.59	0\\
599.6	0\\
599.61	0\\
599.62	0\\
599.63	0\\
599.64	0\\
599.65	0\\
599.66	0\\
599.67	0\\
599.68	0\\
599.69	0\\
599.7	0\\
599.71	0\\
599.72	0\\
599.73	0\\
599.74	0\\
599.75	0\\
599.76	0\\
599.77	0\\
599.78	0\\
599.79	0\\
599.8	0\\
599.81	0\\
599.82	0\\
599.83	0\\
599.84	0\\
599.85	0\\
599.86	0\\
599.87	0\\
599.88	0\\
599.89	0\\
599.9	0\\
599.91	0\\
599.92	0\\
599.93	0\\
599.94	0\\
599.95	0\\
599.96	0\\
599.97	0\\
599.98	0\\
599.99	0\\
600	0\\
};
\addplot [color=mycolor7,solid,forget plot]
  table[row sep=crcr]{%
0.01	0\\
1.01	0\\
2.01	0\\
3.01	0\\
4.01	0\\
5.01	0\\
6.01	0\\
7.01	0\\
8.01	0\\
9.01	0\\
10.01	0\\
11.01	0\\
12.01	0\\
13.01	0\\
14.01	0\\
15.01	0\\
16.01	0\\
17.01	0\\
18.01	0\\
19.01	0\\
20.01	0\\
21.01	0\\
22.01	0\\
23.01	0\\
24.01	0\\
25.01	0\\
26.01	0\\
27.01	0\\
28.01	0\\
29.01	0\\
30.01	0\\
31.01	0\\
32.01	0\\
33.01	0\\
34.01	0\\
35.01	0\\
36.01	0\\
37.01	0\\
38.01	0\\
39.01	0\\
40.01	0\\
41.01	0\\
42.01	0\\
43.01	0\\
44.01	0\\
45.01	0\\
46.01	0\\
47.01	0\\
48.01	0\\
49.01	0\\
50.01	0\\
51.01	0\\
52.01	0\\
53.01	0\\
54.01	0\\
55.01	0\\
56.01	0\\
57.01	0\\
58.01	0\\
59.01	0\\
60.01	0\\
61.01	0\\
62.01	0\\
63.01	0\\
64.01	0\\
65.01	0\\
66.01	0\\
67.01	0\\
68.01	0\\
69.01	0\\
70.01	0\\
71.01	0\\
72.01	0\\
73.01	0\\
74.01	0\\
75.01	0\\
76.01	0\\
77.01	0\\
78.01	0\\
79.01	0\\
80.01	0\\
81.01	0\\
82.01	0\\
83.01	0\\
84.01	0\\
85.01	0\\
86.01	0\\
87.01	0\\
88.01	0\\
89.01	0\\
90.01	0\\
91.01	0\\
92.01	0\\
93.01	0\\
94.01	0\\
95.01	0\\
96.01	0\\
97.01	0\\
98.01	0\\
99.01	0\\
100.01	0\\
101.01	0\\
102.01	0\\
103.01	0\\
104.01	0\\
105.01	0\\
106.01	0\\
107.01	0\\
108.01	0\\
109.01	0\\
110.01	0\\
111.01	0\\
112.01	0\\
113.01	0\\
114.01	0\\
115.01	0\\
116.01	0\\
117.01	0\\
118.01	0\\
119.01	0\\
120.01	0\\
121.01	0\\
122.01	0\\
123.01	0\\
124.01	0\\
125.01	0\\
126.01	0\\
127.01	0\\
128.01	0\\
129.01	0\\
130.01	0\\
131.01	0\\
132.01	0\\
133.01	0\\
134.01	0\\
135.01	0\\
136.01	0\\
137.01	0\\
138.01	0\\
139.01	0\\
140.01	0\\
141.01	0\\
142.01	0\\
143.01	0\\
144.01	0\\
145.01	0\\
146.01	0\\
147.01	0\\
148.01	0\\
149.01	0\\
150.01	0\\
151.01	0\\
152.01	0\\
153.01	0\\
154.01	0\\
155.01	0\\
156.01	0\\
157.01	0\\
158.01	0\\
159.01	0\\
160.01	0\\
161.01	0\\
162.01	0\\
163.01	0\\
164.01	0\\
165.01	0\\
166.01	0\\
167.01	0\\
168.01	0\\
169.01	0\\
170.01	0\\
171.01	0\\
172.01	0\\
173.01	0\\
174.01	0\\
175.01	0\\
176.01	0\\
177.01	0\\
178.01	0\\
179.01	0\\
180.01	0\\
181.01	0\\
182.01	0\\
183.01	0\\
184.01	0\\
185.01	0\\
186.01	0\\
187.01	0\\
188.01	0\\
189.01	0\\
190.01	0\\
191.01	0\\
192.01	0\\
193.01	0\\
194.01	0\\
195.01	0\\
196.01	0\\
197.01	0\\
198.01	0\\
199.01	0\\
200.01	0\\
201.01	0\\
202.01	0\\
203.01	0\\
204.01	0\\
205.01	0\\
206.01	0\\
207.01	0\\
208.01	0\\
209.01	0\\
210.01	0\\
211.01	0\\
212.01	0\\
213.01	0\\
214.01	0\\
215.01	0\\
216.01	0\\
217.01	0\\
218.01	0\\
219.01	0\\
220.01	0\\
221.01	0\\
222.01	0\\
223.01	0\\
224.01	0\\
225.01	0\\
226.01	0\\
227.01	0\\
228.01	0\\
229.01	0\\
230.01	0\\
231.01	0\\
232.01	0\\
233.01	0\\
234.01	0\\
235.01	0\\
236.01	0\\
237.01	0\\
238.01	0\\
239.01	0\\
240.01	0\\
241.01	0\\
242.01	0\\
243.01	0\\
244.01	0\\
245.01	0\\
246.01	0\\
247.01	0\\
248.01	0\\
249.01	0\\
250.01	0\\
251.01	0\\
252.01	0\\
253.01	0\\
254.01	0\\
255.01	0\\
256.01	0\\
257.01	0\\
258.01	0\\
259.01	0\\
260.01	0\\
261.01	0\\
262.01	0\\
263.01	0\\
264.01	0\\
265.01	0\\
266.01	0\\
267.01	0\\
268.01	0\\
269.01	0\\
270.01	0\\
271.01	0\\
272.01	0\\
273.01	0\\
274.01	0\\
275.01	0\\
276.01	0\\
277.01	0\\
278.01	0\\
279.01	0\\
280.01	0\\
281.01	0\\
282.01	0\\
283.01	0\\
284.01	0\\
285.01	0\\
286.01	0\\
287.01	0\\
288.01	0\\
289.01	0\\
290.01	0\\
291.01	0\\
292.01	0\\
293.01	0\\
294.01	0\\
295.01	0\\
296.01	0\\
297.01	0\\
298.01	0\\
299.01	0\\
300.01	0\\
301.01	0\\
302.01	0\\
303.01	0\\
304.01	0\\
305.01	0\\
306.01	0\\
307.01	0\\
308.01	0\\
309.01	0\\
310.01	0\\
311.01	0\\
312.01	0\\
313.01	0\\
314.01	0\\
315.01	0\\
316.01	0\\
317.01	0\\
318.01	0\\
319.01	0\\
320.01	0\\
321.01	0\\
322.01	0\\
323.01	0\\
324.01	0\\
325.01	0\\
326.01	0\\
327.01	0\\
328.01	0\\
329.01	0\\
330.01	0\\
331.01	0\\
332.01	0\\
333.01	0\\
334.01	0\\
335.01	0\\
336.01	0\\
337.01	0\\
338.01	0\\
339.01	0\\
340.01	0\\
341.01	0\\
342.01	0\\
343.01	0\\
344.01	0\\
345.01	0\\
346.01	0\\
347.01	0\\
348.01	0\\
349.01	0\\
350.01	0\\
351.01	0\\
352.01	0\\
353.01	0\\
354.01	0\\
355.01	0\\
356.01	0\\
357.01	0\\
358.01	0\\
359.01	0\\
360.01	0\\
361.01	0\\
362.01	0\\
363.01	0\\
364.01	0\\
365.01	0\\
366.01	0\\
367.01	0\\
368.01	0\\
369.01	0\\
370.01	0\\
371.01	0\\
372.01	0\\
373.01	0\\
374.01	0\\
375.01	0\\
376.01	0\\
377.01	0\\
378.01	0\\
379.01	0\\
380.01	0\\
381.01	0\\
382.01	0\\
383.01	0\\
384.01	0\\
385.01	0\\
386.01	0\\
387.01	0\\
388.01	0\\
389.01	0\\
390.01	0\\
391.01	0\\
392.01	0\\
393.01	0\\
394.01	0\\
395.01	0\\
396.01	0\\
397.01	0\\
398.01	0\\
399.01	0\\
400.01	0\\
401.01	0\\
402.01	0\\
403.01	0\\
404.01	0\\
405.01	0\\
406.01	0\\
407.01	0\\
408.01	0\\
409.01	0\\
410.01	0\\
411.01	0\\
412.01	0\\
413.01	0\\
414.01	0\\
415.01	0\\
416.01	0\\
417.01	0\\
418.01	0\\
419.01	0\\
420.01	0\\
421.01	0\\
422.01	0\\
423.01	0\\
424.01	0\\
425.01	0\\
426.01	0\\
427.01	0\\
428.01	0\\
429.01	0\\
430.01	0\\
431.01	0\\
432.01	0\\
433.01	0\\
434.01	0\\
435.01	0\\
436.01	0\\
437.01	0\\
438.01	0\\
439.01	0\\
440.01	0\\
441.01	0\\
442.01	0\\
443.01	0\\
444.01	0\\
445.01	0\\
446.01	0\\
447.01	0\\
448.01	0\\
449.01	0\\
450.01	0\\
451.01	1.73472347597681e-18\\
452.01	0\\
453.01	0\\
454.01	0\\
455.01	0\\
456.01	0\\
457.01	0\\
458.01	0\\
459.01	0\\
460.01	0\\
461.01	0\\
462.01	0\\
463.01	0\\
464.01	0\\
465.01	0\\
466.01	0\\
467.01	0\\
468.01	0\\
469.01	1.73472347597681e-18\\
470.01	0\\
471.01	0\\
472.01	0\\
473.01	0\\
474.01	0\\
475.01	0\\
476.01	1.73472347597681e-18\\
477.01	0\\
478.01	0\\
479.01	0\\
480.01	0\\
481.01	0\\
482.01	0\\
483.01	0\\
484.01	0\\
485.01	0\\
486.01	0\\
487.01	0\\
488.01	0\\
489.01	0\\
490.01	0\\
491.01	0\\
492.01	0\\
493.01	0\\
494.01	0\\
495.01	0\\
496.01	0\\
497.01	0\\
498.01	0\\
499.01	0\\
500.01	0\\
501.01	0\\
502.01	0\\
503.01	0\\
504.01	0\\
505.01	0\\
506.01	0\\
507.01	0\\
508.01	0\\
509.01	0\\
510.01	0\\
511.01	0\\
512.01	1.73472347597681e-18\\
513.01	0\\
514.01	0\\
515.01	0\\
516.01	0\\
517.01	0\\
518.01	0\\
519.01	0\\
520.01	0\\
521.01	0\\
522.01	0\\
523.01	1.73472347597681e-18\\
524.01	0\\
525.01	0\\
526.01	0\\
527.01	0\\
528.01	0\\
529.01	0\\
530.01	0\\
531.01	0\\
532.01	0\\
533.01	0\\
534.01	0\\
535.01	0\\
536.01	1.73472347597681e-18\\
537.01	1.73472347597681e-18\\
538.01	1.73472347597681e-18\\
539.01	0\\
540.01	0\\
541.01	0\\
542.01	0\\
543.01	0\\
544.01	0\\
545.01	1.73472347597681e-18\\
546.01	1.73472347597681e-18\\
547.01	0\\
548.01	0\\
549.01	0\\
550.01	0\\
551.01	0\\
552.01	0\\
553.01	0\\
554.01	0\\
555.01	0\\
556.01	1.73472347597681e-18\\
557.01	0\\
558.01	0\\
559.01	0\\
560.01	1.73472347597681e-18\\
561.01	1.73472347597681e-18\\
562.01	0\\
563.01	0\\
564.01	0\\
565.01	0\\
566.01	0\\
567.01	0\\
568.01	0\\
569.01	0\\
570.01	0\\
571.01	0\\
572.01	0\\
573.01	0\\
574.01	0\\
575.01	0\\
576.01	0\\
577.01	0\\
578.01	1.73472347597681e-18\\
579.01	0\\
580.01	0\\
581.01	0\\
582.01	0\\
583.01	1.73472347597681e-18\\
584.01	0\\
585.01	0\\
586.01	0\\
587.01	0\\
588.01	0\\
589.01	0\\
590.01	0\\
591.01	0\\
592.01	0\\
593.01	0\\
594.01	0\\
595.01	0\\
596.01	0\\
597.01	0\\
598.01	0\\
599.01	0\\
599.02	0\\
599.03	0\\
599.04	0\\
599.05	0\\
599.06	0\\
599.07	0\\
599.08	0\\
599.09	0\\
599.1	0\\
599.11	0\\
599.12	0\\
599.13	0\\
599.14	0\\
599.15	0\\
599.16	0\\
599.17	0\\
599.18	0\\
599.19	0\\
599.2	0\\
599.21	0\\
599.22	0\\
599.23	0\\
599.24	0\\
599.25	0\\
599.26	0\\
599.27	0\\
599.28	0\\
599.29	0\\
599.3	0\\
599.31	0\\
599.32	0\\
599.33	0\\
599.34	0\\
599.35	0\\
599.36	0\\
599.37	0\\
599.38	0\\
599.39	0\\
599.4	0\\
599.41	0\\
599.42	0\\
599.43	0\\
599.44	0\\
599.45	0\\
599.46	0\\
599.47	0\\
599.48	0\\
599.49	0\\
599.5	0\\
599.51	0\\
599.52	0\\
599.53	0\\
599.54	0\\
599.55	0\\
599.56	0\\
599.57	0\\
599.58	0\\
599.59	0\\
599.6	0\\
599.61	0\\
599.62	0\\
599.63	0\\
599.64	0\\
599.65	0\\
599.66	0\\
599.67	0\\
599.68	0\\
599.69	0\\
599.7	0\\
599.71	0\\
599.72	0\\
599.73	0\\
599.74	0\\
599.75	0\\
599.76	0\\
599.77	0\\
599.78	0\\
599.79	0\\
599.8	0\\
599.81	0\\
599.82	0\\
599.83	0\\
599.84	0\\
599.85	0\\
599.86	0\\
599.87	0\\
599.88	0\\
599.89	0\\
599.9	0\\
599.91	0\\
599.92	0\\
599.93	0\\
599.94	0\\
599.95	0\\
599.96	0\\
599.97	0\\
599.98	0\\
599.99	0\\
600	0\\
};
\addplot [color=mycolor8,solid,forget plot]
  table[row sep=crcr]{%
0.01	0\\
1.01	0\\
2.01	0\\
3.01	0\\
4.01	0\\
5.01	0\\
6.01	0\\
7.01	0\\
8.01	0\\
9.01	0\\
10.01	0\\
11.01	0\\
12.01	0\\
13.01	0\\
14.01	0\\
15.01	0\\
16.01	0\\
17.01	0\\
18.01	0\\
19.01	0\\
20.01	0\\
21.01	0\\
22.01	0\\
23.01	0\\
24.01	0\\
25.01	0\\
26.01	0\\
27.01	0\\
28.01	0\\
29.01	0\\
30.01	0\\
31.01	0\\
32.01	0\\
33.01	0\\
34.01	0\\
35.01	0\\
36.01	0\\
37.01	0\\
38.01	0\\
39.01	0\\
40.01	0\\
41.01	0\\
42.01	0\\
43.01	0\\
44.01	0\\
45.01	0\\
46.01	0\\
47.01	0\\
48.01	0\\
49.01	0\\
50.01	0\\
51.01	0\\
52.01	0\\
53.01	0\\
54.01	0\\
55.01	0\\
56.01	0\\
57.01	0\\
58.01	0\\
59.01	0\\
60.01	0\\
61.01	0\\
62.01	0\\
63.01	0\\
64.01	0\\
65.01	0\\
66.01	0\\
67.01	0\\
68.01	0\\
69.01	0\\
70.01	0\\
71.01	0\\
72.01	0\\
73.01	0\\
74.01	0\\
75.01	0\\
76.01	0\\
77.01	0\\
78.01	0\\
79.01	0\\
80.01	0\\
81.01	0\\
82.01	0\\
83.01	0\\
84.01	0\\
85.01	0\\
86.01	0\\
87.01	0\\
88.01	0\\
89.01	0\\
90.01	0\\
91.01	0\\
92.01	0\\
93.01	0\\
94.01	0\\
95.01	0\\
96.01	0\\
97.01	0\\
98.01	0\\
99.01	0\\
100.01	0\\
101.01	0\\
102.01	0\\
103.01	0\\
104.01	0\\
105.01	0\\
106.01	0\\
107.01	0\\
108.01	0\\
109.01	0\\
110.01	0\\
111.01	0\\
112.01	0\\
113.01	0\\
114.01	0\\
115.01	0\\
116.01	0\\
117.01	0\\
118.01	0\\
119.01	0\\
120.01	0\\
121.01	0\\
122.01	0\\
123.01	0\\
124.01	0\\
125.01	0\\
126.01	0\\
127.01	0\\
128.01	0\\
129.01	0\\
130.01	0\\
131.01	0\\
132.01	0\\
133.01	0\\
134.01	0\\
135.01	0\\
136.01	0\\
137.01	0\\
138.01	0\\
139.01	0\\
140.01	0\\
141.01	0\\
142.01	0\\
143.01	0\\
144.01	0\\
145.01	0\\
146.01	0\\
147.01	0\\
148.01	0\\
149.01	0\\
150.01	0\\
151.01	0\\
152.01	0\\
153.01	0\\
154.01	0\\
155.01	0\\
156.01	0\\
157.01	0\\
158.01	0\\
159.01	0\\
160.01	0\\
161.01	0\\
162.01	0\\
163.01	0\\
164.01	0\\
165.01	0\\
166.01	0\\
167.01	0\\
168.01	0\\
169.01	0\\
170.01	0\\
171.01	0\\
172.01	0\\
173.01	0\\
174.01	0\\
175.01	0\\
176.01	0\\
177.01	0\\
178.01	0\\
179.01	0\\
180.01	0\\
181.01	0\\
182.01	0\\
183.01	0\\
184.01	0\\
185.01	0\\
186.01	0\\
187.01	0\\
188.01	0\\
189.01	0\\
190.01	0\\
191.01	0\\
192.01	0\\
193.01	0\\
194.01	0\\
195.01	0\\
196.01	0\\
197.01	0\\
198.01	0\\
199.01	0\\
200.01	0\\
201.01	0\\
202.01	0\\
203.01	0\\
204.01	0\\
205.01	0\\
206.01	0\\
207.01	0\\
208.01	0\\
209.01	0\\
210.01	0\\
211.01	0\\
212.01	0\\
213.01	0\\
214.01	0\\
215.01	0\\
216.01	0\\
217.01	0\\
218.01	0\\
219.01	0\\
220.01	0\\
221.01	0\\
222.01	0\\
223.01	0\\
224.01	0\\
225.01	0\\
226.01	0\\
227.01	0\\
228.01	0\\
229.01	0\\
230.01	0\\
231.01	0\\
232.01	0\\
233.01	0\\
234.01	0\\
235.01	0\\
236.01	0\\
237.01	0\\
238.01	0\\
239.01	0\\
240.01	0\\
241.01	0\\
242.01	0\\
243.01	0\\
244.01	0\\
245.01	0\\
246.01	0\\
247.01	0\\
248.01	0\\
249.01	0\\
250.01	0\\
251.01	0\\
252.01	0\\
253.01	0\\
254.01	0\\
255.01	0\\
256.01	0\\
257.01	0\\
258.01	0\\
259.01	0\\
260.01	0\\
261.01	0\\
262.01	0\\
263.01	0\\
264.01	0\\
265.01	0\\
266.01	0\\
267.01	0\\
268.01	0\\
269.01	0\\
270.01	0\\
271.01	0\\
272.01	0\\
273.01	0\\
274.01	0\\
275.01	0\\
276.01	0\\
277.01	0\\
278.01	0\\
279.01	0\\
280.01	0\\
281.01	0\\
282.01	0\\
283.01	0\\
284.01	0\\
285.01	0\\
286.01	0\\
287.01	0\\
288.01	0\\
289.01	0\\
290.01	0\\
291.01	0\\
292.01	0\\
293.01	0\\
294.01	0\\
295.01	0\\
296.01	0\\
297.01	0\\
298.01	0\\
299.01	0\\
300.01	0\\
301.01	0\\
302.01	0\\
303.01	0\\
304.01	0\\
305.01	0\\
306.01	0\\
307.01	0\\
308.01	0\\
309.01	0\\
310.01	0\\
311.01	0\\
312.01	0\\
313.01	0\\
314.01	0\\
315.01	0\\
316.01	0\\
317.01	0\\
318.01	0\\
319.01	0\\
320.01	0\\
321.01	0\\
322.01	0\\
323.01	0\\
324.01	0\\
325.01	0\\
326.01	0\\
327.01	0\\
328.01	0\\
329.01	0\\
330.01	0\\
331.01	0\\
332.01	0\\
333.01	0\\
334.01	0\\
335.01	0\\
336.01	0\\
337.01	0\\
338.01	0\\
339.01	0\\
340.01	0\\
341.01	0\\
342.01	0\\
343.01	0\\
344.01	0\\
345.01	0\\
346.01	0\\
347.01	0\\
348.01	0\\
349.01	0\\
350.01	0\\
351.01	0\\
352.01	0\\
353.01	0\\
354.01	0\\
355.01	0\\
356.01	0\\
357.01	0\\
358.01	0\\
359.01	0\\
360.01	0\\
361.01	0\\
362.01	0\\
363.01	0\\
364.01	0\\
365.01	0\\
366.01	0\\
367.01	0\\
368.01	0\\
369.01	0\\
370.01	0\\
371.01	0\\
372.01	0\\
373.01	0\\
374.01	0\\
375.01	0\\
376.01	0\\
377.01	0\\
378.01	0\\
379.01	0\\
380.01	0\\
381.01	0\\
382.01	0\\
383.01	0\\
384.01	0\\
385.01	0\\
386.01	0\\
387.01	0\\
388.01	0\\
389.01	0\\
390.01	0\\
391.01	0\\
392.01	0\\
393.01	0\\
394.01	0\\
395.01	0\\
396.01	0\\
397.01	0\\
398.01	0\\
399.01	0\\
400.01	0\\
401.01	0\\
402.01	0\\
403.01	0\\
404.01	0\\
405.01	0\\
406.01	0\\
407.01	0\\
408.01	0\\
409.01	0\\
410.01	0\\
411.01	0\\
412.01	0\\
413.01	0\\
414.01	0\\
415.01	0\\
416.01	0\\
417.01	0\\
418.01	0\\
419.01	0\\
420.01	0\\
421.01	0\\
422.01	0\\
423.01	0\\
424.01	0\\
425.01	0\\
426.01	0\\
427.01	0\\
428.01	0\\
429.01	0\\
430.01	0\\
431.01	0\\
432.01	0\\
433.01	0\\
434.01	0\\
435.01	0\\
436.01	0\\
437.01	0\\
438.01	0\\
439.01	0\\
440.01	0\\
441.01	0\\
442.01	0\\
443.01	0\\
444.01	0\\
445.01	0\\
446.01	0\\
447.01	0\\
448.01	0\\
449.01	0\\
450.01	0\\
451.01	1.73472347597681e-18\\
452.01	0\\
453.01	0\\
454.01	0\\
455.01	0\\
456.01	0\\
457.01	0\\
458.01	0\\
459.01	0\\
460.01	0\\
461.01	0\\
462.01	0\\
463.01	0\\
464.01	0\\
465.01	0\\
466.01	0\\
467.01	0\\
468.01	0\\
469.01	1.73472347597681e-18\\
470.01	0\\
471.01	0\\
472.01	0\\
473.01	0\\
474.01	0\\
475.01	0\\
476.01	1.73472347597681e-18\\
477.01	0\\
478.01	0\\
479.01	0\\
480.01	0\\
481.01	0\\
482.01	0\\
483.01	0\\
484.01	0\\
485.01	0\\
486.01	0\\
487.01	0\\
488.01	0\\
489.01	0\\
490.01	0\\
491.01	0\\
492.01	0\\
493.01	0\\
494.01	0\\
495.01	0\\
496.01	0\\
497.01	0\\
498.01	0\\
499.01	0\\
500.01	0\\
501.01	0\\
502.01	0\\
503.01	0\\
504.01	0\\
505.01	0\\
506.01	0\\
507.01	0\\
508.01	0\\
509.01	0\\
510.01	0\\
511.01	0\\
512.01	1.73472347597681e-18\\
513.01	0\\
514.01	0\\
515.01	0\\
516.01	0\\
517.01	0\\
518.01	0\\
519.01	0\\
520.01	0\\
521.01	0\\
522.01	0\\
523.01	1.73472347597681e-18\\
524.01	0\\
525.01	0\\
526.01	0\\
527.01	0\\
528.01	0\\
529.01	0\\
530.01	0\\
531.01	0\\
532.01	0\\
533.01	0\\
534.01	0\\
535.01	0\\
536.01	1.73472347597681e-18\\
537.01	1.73472347597681e-18\\
538.01	1.73472347597681e-18\\
539.01	0\\
540.01	0\\
541.01	0\\
542.01	0\\
543.01	0\\
544.01	0\\
545.01	1.73472347597681e-18\\
546.01	1.73472347597681e-18\\
547.01	0\\
548.01	0\\
549.01	0\\
550.01	0\\
551.01	0\\
552.01	0\\
553.01	0\\
554.01	0\\
555.01	0\\
556.01	1.73472347597681e-18\\
557.01	0\\
558.01	0\\
559.01	0\\
560.01	1.73472347597681e-18\\
561.01	1.73472347597681e-18\\
562.01	0\\
563.01	0\\
564.01	0\\
565.01	0\\
566.01	0\\
567.01	0\\
568.01	0\\
569.01	0\\
570.01	0\\
571.01	0\\
572.01	0\\
573.01	0\\
574.01	0\\
575.01	0\\
576.01	0\\
577.01	0\\
578.01	1.73472347597681e-18\\
579.01	0\\
580.01	0\\
581.01	0\\
582.01	0\\
583.01	1.73472347597681e-18\\
584.01	0\\
585.01	0\\
586.01	0\\
587.01	0\\
588.01	0\\
589.01	0\\
590.01	0\\
591.01	0\\
592.01	0\\
593.01	0\\
594.01	0\\
595.01	0\\
596.01	0\\
597.01	0\\
598.01	0\\
599.01	0\\
599.02	0\\
599.03	0\\
599.04	0\\
599.05	0\\
599.06	0\\
599.07	0\\
599.08	0\\
599.09	0\\
599.1	0\\
599.11	0\\
599.12	0\\
599.13	0\\
599.14	0\\
599.15	0\\
599.16	0\\
599.17	0\\
599.18	0\\
599.19	0\\
599.2	0\\
599.21	0\\
599.22	0\\
599.23	0\\
599.24	0\\
599.25	0\\
599.26	0\\
599.27	0\\
599.28	0\\
599.29	0\\
599.3	0\\
599.31	0\\
599.32	0\\
599.33	0\\
599.34	0\\
599.35	0\\
599.36	0\\
599.37	0\\
599.38	0\\
599.39	0\\
599.4	0\\
599.41	0\\
599.42	0\\
599.43	0\\
599.44	0\\
599.45	0\\
599.46	0\\
599.47	0\\
599.48	0\\
599.49	0\\
599.5	0\\
599.51	0\\
599.52	0\\
599.53	0\\
599.54	0\\
599.55	0\\
599.56	0\\
599.57	0\\
599.58	0\\
599.59	0\\
599.6	0\\
599.61	0\\
599.62	0\\
599.63	0\\
599.64	0\\
599.65	0\\
599.66	0\\
599.67	0\\
599.68	0\\
599.69	0\\
599.7	0\\
599.71	0\\
599.72	0\\
599.73	0\\
599.74	0\\
599.75	0\\
599.76	0\\
599.77	0\\
599.78	0\\
599.79	0\\
599.8	0\\
599.81	0\\
599.82	0\\
599.83	0\\
599.84	0\\
599.85	0\\
599.86	0\\
599.87	0\\
599.88	0\\
599.89	0\\
599.9	0\\
599.91	0\\
599.92	0\\
599.93	0\\
599.94	0\\
599.95	0\\
599.96	0\\
599.97	0\\
599.98	0\\
599.99	0\\
600	0\\
};
\addplot [color=blue!25!mycolor7,solid,forget plot]
  table[row sep=crcr]{%
0.01	0\\
1.01	0\\
2.01	0\\
3.01	0\\
4.01	0\\
5.01	0\\
6.01	0\\
7.01	0\\
8.01	0\\
9.01	0\\
10.01	0\\
11.01	0\\
12.01	0\\
13.01	0\\
14.01	0\\
15.01	0\\
16.01	0\\
17.01	0\\
18.01	0\\
19.01	0\\
20.01	0\\
21.01	0\\
22.01	0\\
23.01	0\\
24.01	0\\
25.01	0\\
26.01	0\\
27.01	0\\
28.01	0\\
29.01	0\\
30.01	0\\
31.01	0\\
32.01	0\\
33.01	0\\
34.01	0\\
35.01	0\\
36.01	0\\
37.01	0\\
38.01	0\\
39.01	0\\
40.01	0\\
41.01	0\\
42.01	0\\
43.01	0\\
44.01	0\\
45.01	0\\
46.01	0\\
47.01	0\\
48.01	0\\
49.01	0\\
50.01	0\\
51.01	0\\
52.01	0\\
53.01	0\\
54.01	0\\
55.01	0\\
56.01	0\\
57.01	0\\
58.01	0\\
59.01	0\\
60.01	0\\
61.01	0\\
62.01	0\\
63.01	0\\
64.01	0\\
65.01	0\\
66.01	0\\
67.01	0\\
68.01	0\\
69.01	0\\
70.01	0\\
71.01	0\\
72.01	0\\
73.01	0\\
74.01	0\\
75.01	0\\
76.01	0\\
77.01	0\\
78.01	0\\
79.01	0\\
80.01	0\\
81.01	0\\
82.01	0\\
83.01	0\\
84.01	0\\
85.01	0\\
86.01	0\\
87.01	0\\
88.01	0\\
89.01	0\\
90.01	0\\
91.01	0\\
92.01	0\\
93.01	0\\
94.01	0\\
95.01	0\\
96.01	0\\
97.01	0\\
98.01	0\\
99.01	0\\
100.01	0\\
101.01	0\\
102.01	0\\
103.01	0\\
104.01	0\\
105.01	0\\
106.01	0\\
107.01	0\\
108.01	0\\
109.01	0\\
110.01	0\\
111.01	0\\
112.01	0\\
113.01	0\\
114.01	0\\
115.01	0\\
116.01	0\\
117.01	0\\
118.01	0\\
119.01	0\\
120.01	0\\
121.01	0\\
122.01	0\\
123.01	0\\
124.01	0\\
125.01	0\\
126.01	0\\
127.01	0\\
128.01	0\\
129.01	0\\
130.01	0\\
131.01	0\\
132.01	0\\
133.01	0\\
134.01	0\\
135.01	0\\
136.01	0\\
137.01	0\\
138.01	0\\
139.01	0\\
140.01	0\\
141.01	0\\
142.01	0\\
143.01	0\\
144.01	0\\
145.01	0\\
146.01	0\\
147.01	0\\
148.01	0\\
149.01	0\\
150.01	0\\
151.01	0\\
152.01	0\\
153.01	0\\
154.01	0\\
155.01	0\\
156.01	0\\
157.01	0\\
158.01	0\\
159.01	0\\
160.01	0\\
161.01	0\\
162.01	0\\
163.01	0\\
164.01	0\\
165.01	0\\
166.01	0\\
167.01	0\\
168.01	0\\
169.01	0\\
170.01	0\\
171.01	0\\
172.01	0\\
173.01	0\\
174.01	0\\
175.01	0\\
176.01	0\\
177.01	0\\
178.01	0\\
179.01	0\\
180.01	0\\
181.01	0\\
182.01	0\\
183.01	0\\
184.01	0\\
185.01	0\\
186.01	0\\
187.01	0\\
188.01	0\\
189.01	0\\
190.01	0\\
191.01	0\\
192.01	0\\
193.01	0\\
194.01	0\\
195.01	0\\
196.01	0\\
197.01	0\\
198.01	0\\
199.01	0\\
200.01	0\\
201.01	0\\
202.01	0\\
203.01	0\\
204.01	0\\
205.01	0\\
206.01	0\\
207.01	0\\
208.01	0\\
209.01	0\\
210.01	0\\
211.01	0\\
212.01	0\\
213.01	0\\
214.01	0\\
215.01	0\\
216.01	0\\
217.01	0\\
218.01	0\\
219.01	0\\
220.01	0\\
221.01	0\\
222.01	0\\
223.01	0\\
224.01	0\\
225.01	0\\
226.01	0\\
227.01	0\\
228.01	0\\
229.01	0\\
230.01	0\\
231.01	0\\
232.01	0\\
233.01	0\\
234.01	0\\
235.01	0\\
236.01	0\\
237.01	0\\
238.01	0\\
239.01	0\\
240.01	0\\
241.01	0\\
242.01	0\\
243.01	0\\
244.01	0\\
245.01	0\\
246.01	0\\
247.01	0\\
248.01	0\\
249.01	0\\
250.01	0\\
251.01	0\\
252.01	0\\
253.01	0\\
254.01	0\\
255.01	0\\
256.01	0\\
257.01	0\\
258.01	0\\
259.01	0\\
260.01	0\\
261.01	0\\
262.01	0\\
263.01	0\\
264.01	0\\
265.01	0\\
266.01	0\\
267.01	0\\
268.01	0\\
269.01	0\\
270.01	0\\
271.01	0\\
272.01	0\\
273.01	0\\
274.01	0\\
275.01	0\\
276.01	0\\
277.01	0\\
278.01	0\\
279.01	0\\
280.01	0\\
281.01	0\\
282.01	0\\
283.01	0\\
284.01	0\\
285.01	0\\
286.01	0\\
287.01	0\\
288.01	0\\
289.01	0\\
290.01	0\\
291.01	0\\
292.01	0\\
293.01	0\\
294.01	0\\
295.01	0\\
296.01	0\\
297.01	0\\
298.01	0\\
299.01	0\\
300.01	0\\
301.01	0\\
302.01	0\\
303.01	0\\
304.01	0\\
305.01	0\\
306.01	0\\
307.01	0\\
308.01	0\\
309.01	0\\
310.01	0\\
311.01	0\\
312.01	0\\
313.01	0\\
314.01	0\\
315.01	0\\
316.01	0\\
317.01	0\\
318.01	0\\
319.01	0\\
320.01	0\\
321.01	0\\
322.01	0\\
323.01	0\\
324.01	0\\
325.01	0\\
326.01	0\\
327.01	0\\
328.01	0\\
329.01	0\\
330.01	0\\
331.01	0\\
332.01	0\\
333.01	0\\
334.01	0\\
335.01	0\\
336.01	0\\
337.01	0\\
338.01	0\\
339.01	0\\
340.01	0\\
341.01	0\\
342.01	0\\
343.01	0\\
344.01	0\\
345.01	0\\
346.01	0\\
347.01	0\\
348.01	0\\
349.01	0\\
350.01	0\\
351.01	0\\
352.01	0\\
353.01	0\\
354.01	0\\
355.01	0\\
356.01	0\\
357.01	0\\
358.01	0\\
359.01	0\\
360.01	0\\
361.01	0\\
362.01	0\\
363.01	0\\
364.01	0\\
365.01	0\\
366.01	0\\
367.01	0\\
368.01	0\\
369.01	0\\
370.01	0\\
371.01	0\\
372.01	0\\
373.01	0\\
374.01	0\\
375.01	0\\
376.01	0\\
377.01	0\\
378.01	0\\
379.01	0\\
380.01	0\\
381.01	0\\
382.01	0\\
383.01	0\\
384.01	0\\
385.01	0\\
386.01	0\\
387.01	0\\
388.01	0\\
389.01	0\\
390.01	0\\
391.01	0\\
392.01	0\\
393.01	0\\
394.01	0\\
395.01	0\\
396.01	0\\
397.01	0\\
398.01	0\\
399.01	0\\
400.01	0\\
401.01	0\\
402.01	0\\
403.01	0\\
404.01	0\\
405.01	0\\
406.01	0\\
407.01	0\\
408.01	0\\
409.01	0\\
410.01	0\\
411.01	0\\
412.01	0\\
413.01	0\\
414.01	0\\
415.01	0\\
416.01	0\\
417.01	0\\
418.01	0\\
419.01	0\\
420.01	0\\
421.01	0\\
422.01	0\\
423.01	0\\
424.01	0\\
425.01	0\\
426.01	0\\
427.01	0\\
428.01	0\\
429.01	0\\
430.01	0\\
431.01	0\\
432.01	0\\
433.01	0\\
434.01	0\\
435.01	0\\
436.01	0\\
437.01	0\\
438.01	0\\
439.01	0\\
440.01	0\\
441.01	0\\
442.01	0\\
443.01	0\\
444.01	0\\
445.01	0\\
446.01	0\\
447.01	0\\
448.01	0\\
449.01	0\\
450.01	0\\
451.01	1.73472347597681e-18\\
452.01	0\\
453.01	0\\
454.01	0\\
455.01	0\\
456.01	0\\
457.01	0\\
458.01	0\\
459.01	0\\
460.01	0\\
461.01	0\\
462.01	0\\
463.01	0\\
464.01	0\\
465.01	0\\
466.01	0\\
467.01	0\\
468.01	0\\
469.01	1.73472347597681e-18\\
470.01	0\\
471.01	0\\
472.01	0\\
473.01	0\\
474.01	0\\
475.01	0\\
476.01	1.73472347597681e-18\\
477.01	0\\
478.01	0\\
479.01	0\\
480.01	0\\
481.01	0\\
482.01	0\\
483.01	0\\
484.01	0\\
485.01	0\\
486.01	0\\
487.01	0\\
488.01	0\\
489.01	0\\
490.01	0\\
491.01	0\\
492.01	0\\
493.01	0\\
494.01	0\\
495.01	0\\
496.01	0\\
497.01	0\\
498.01	0\\
499.01	0\\
500.01	0\\
501.01	0\\
502.01	0\\
503.01	0\\
504.01	0\\
505.01	0\\
506.01	0\\
507.01	0\\
508.01	0\\
509.01	0\\
510.01	0\\
511.01	0\\
512.01	1.73472347597681e-18\\
513.01	0\\
514.01	0\\
515.01	0\\
516.01	0\\
517.01	0\\
518.01	0\\
519.01	0\\
520.01	0\\
521.01	0\\
522.01	0\\
523.01	1.73472347597681e-18\\
524.01	0\\
525.01	0\\
526.01	0\\
527.01	0\\
528.01	0\\
529.01	0\\
530.01	0\\
531.01	0\\
532.01	0\\
533.01	0\\
534.01	0\\
535.01	0\\
536.01	1.73472347597681e-18\\
537.01	1.73472347597681e-18\\
538.01	1.73472347597681e-18\\
539.01	0\\
540.01	0\\
541.01	0\\
542.01	0\\
543.01	0\\
544.01	0\\
545.01	1.73472347597681e-18\\
546.01	1.73472347597681e-18\\
547.01	0\\
548.01	0\\
549.01	0\\
550.01	0\\
551.01	0\\
552.01	0\\
553.01	0\\
554.01	0\\
555.01	0\\
556.01	1.73472347597681e-18\\
557.01	0\\
558.01	0\\
559.01	0\\
560.01	1.73472347597681e-18\\
561.01	1.73472347597681e-18\\
562.01	0\\
563.01	0\\
564.01	0\\
565.01	0\\
566.01	0\\
567.01	0\\
568.01	0\\
569.01	0\\
570.01	0\\
571.01	0\\
572.01	0\\
573.01	0\\
574.01	0\\
575.01	0\\
576.01	0\\
577.01	0\\
578.01	1.73472347597681e-18\\
579.01	0\\
580.01	0\\
581.01	0\\
582.01	0\\
583.01	1.73472347597681e-18\\
584.01	0\\
585.01	0\\
586.01	0\\
587.01	0\\
588.01	0\\
589.01	0\\
590.01	0\\
591.01	0\\
592.01	0\\
593.01	0\\
594.01	0\\
595.01	0\\
596.01	0\\
597.01	0\\
598.01	0\\
599.01	0\\
599.02	0\\
599.03	0\\
599.04	0\\
599.05	0\\
599.06	0\\
599.07	0\\
599.08	0\\
599.09	0\\
599.1	0\\
599.11	0\\
599.12	0\\
599.13	0\\
599.14	0\\
599.15	0\\
599.16	0\\
599.17	0\\
599.18	0\\
599.19	0\\
599.2	0\\
599.21	0\\
599.22	0\\
599.23	0\\
599.24	0\\
599.25	0\\
599.26	0\\
599.27	0\\
599.28	0\\
599.29	0\\
599.3	0\\
599.31	0\\
599.32	0\\
599.33	0\\
599.34	0\\
599.35	0\\
599.36	0\\
599.37	0\\
599.38	0\\
599.39	0\\
599.4	0\\
599.41	0\\
599.42	0\\
599.43	0\\
599.44	0\\
599.45	0\\
599.46	0\\
599.47	0\\
599.48	0\\
599.49	0\\
599.5	0\\
599.51	0\\
599.52	0\\
599.53	0\\
599.54	0\\
599.55	0\\
599.56	0\\
599.57	0\\
599.58	0\\
599.59	0\\
599.6	0\\
599.61	0\\
599.62	0\\
599.63	0\\
599.64	0\\
599.65	0\\
599.66	0\\
599.67	0\\
599.68	0\\
599.69	0\\
599.7	0\\
599.71	0\\
599.72	0\\
599.73	0\\
599.74	0\\
599.75	0\\
599.76	0\\
599.77	0\\
599.78	0\\
599.79	0\\
599.8	0\\
599.81	0\\
599.82	0\\
599.83	0\\
599.84	0\\
599.85	0\\
599.86	0\\
599.87	0\\
599.88	0\\
599.89	0\\
599.9	0\\
599.91	0\\
599.92	0\\
599.93	0\\
599.94	0\\
599.95	0\\
599.96	0\\
599.97	0\\
599.98	0\\
599.99	0\\
600	0\\
};
\addplot [color=mycolor9,solid,forget plot]
  table[row sep=crcr]{%
0.01	0\\
1.01	0\\
2.01	0\\
3.01	0\\
4.01	0\\
5.01	0\\
6.01	0\\
7.01	0\\
8.01	0\\
9.01	0\\
10.01	0\\
11.01	0\\
12.01	0\\
13.01	0\\
14.01	0\\
15.01	0\\
16.01	0\\
17.01	0\\
18.01	0\\
19.01	0\\
20.01	0\\
21.01	0\\
22.01	0\\
23.01	0\\
24.01	0\\
25.01	0\\
26.01	0\\
27.01	0\\
28.01	0\\
29.01	0\\
30.01	0\\
31.01	0\\
32.01	0\\
33.01	0\\
34.01	0\\
35.01	0\\
36.01	0\\
37.01	0\\
38.01	0\\
39.01	0\\
40.01	0\\
41.01	0\\
42.01	0\\
43.01	0\\
44.01	0\\
45.01	0\\
46.01	0\\
47.01	0\\
48.01	0\\
49.01	0\\
50.01	0\\
51.01	0\\
52.01	0\\
53.01	0\\
54.01	0\\
55.01	0\\
56.01	0\\
57.01	0\\
58.01	0\\
59.01	0\\
60.01	0\\
61.01	0\\
62.01	0\\
63.01	0\\
64.01	0\\
65.01	0\\
66.01	0\\
67.01	0\\
68.01	0\\
69.01	0\\
70.01	0\\
71.01	0\\
72.01	0\\
73.01	0\\
74.01	0\\
75.01	0\\
76.01	0\\
77.01	0\\
78.01	0\\
79.01	0\\
80.01	0\\
81.01	0\\
82.01	0\\
83.01	0\\
84.01	0\\
85.01	0\\
86.01	0\\
87.01	0\\
88.01	0\\
89.01	0\\
90.01	0\\
91.01	0\\
92.01	0\\
93.01	0\\
94.01	0\\
95.01	0\\
96.01	0\\
97.01	0\\
98.01	0\\
99.01	0\\
100.01	0\\
101.01	0\\
102.01	0\\
103.01	0\\
104.01	0\\
105.01	0\\
106.01	0\\
107.01	0\\
108.01	0\\
109.01	0\\
110.01	0\\
111.01	0\\
112.01	0\\
113.01	0\\
114.01	0\\
115.01	0\\
116.01	0\\
117.01	0\\
118.01	0\\
119.01	0\\
120.01	0\\
121.01	0\\
122.01	0\\
123.01	0\\
124.01	0\\
125.01	0\\
126.01	0\\
127.01	0\\
128.01	0\\
129.01	0\\
130.01	0\\
131.01	0\\
132.01	0\\
133.01	0\\
134.01	0\\
135.01	0\\
136.01	0\\
137.01	0\\
138.01	0\\
139.01	0\\
140.01	0\\
141.01	0\\
142.01	0\\
143.01	0\\
144.01	0\\
145.01	0\\
146.01	0\\
147.01	0\\
148.01	0\\
149.01	0\\
150.01	0\\
151.01	0\\
152.01	0\\
153.01	0\\
154.01	0\\
155.01	0\\
156.01	0\\
157.01	0\\
158.01	0\\
159.01	0\\
160.01	0\\
161.01	0\\
162.01	0\\
163.01	0\\
164.01	0\\
165.01	0\\
166.01	0\\
167.01	0\\
168.01	0\\
169.01	0\\
170.01	0\\
171.01	0\\
172.01	0\\
173.01	0\\
174.01	0\\
175.01	0\\
176.01	0\\
177.01	0\\
178.01	0\\
179.01	0\\
180.01	0\\
181.01	0\\
182.01	0\\
183.01	0\\
184.01	0\\
185.01	0\\
186.01	0\\
187.01	0\\
188.01	0\\
189.01	0\\
190.01	0\\
191.01	0\\
192.01	0\\
193.01	0\\
194.01	0\\
195.01	0\\
196.01	0\\
197.01	0\\
198.01	0\\
199.01	0\\
200.01	0\\
201.01	0\\
202.01	0\\
203.01	0\\
204.01	0\\
205.01	0\\
206.01	0\\
207.01	0\\
208.01	0\\
209.01	0\\
210.01	0\\
211.01	0\\
212.01	0\\
213.01	0\\
214.01	0\\
215.01	0\\
216.01	0\\
217.01	0\\
218.01	0\\
219.01	0\\
220.01	0\\
221.01	0\\
222.01	0\\
223.01	0\\
224.01	0\\
225.01	0\\
226.01	0\\
227.01	0\\
228.01	0\\
229.01	0\\
230.01	0\\
231.01	0\\
232.01	0\\
233.01	0\\
234.01	0\\
235.01	0\\
236.01	0\\
237.01	0\\
238.01	0\\
239.01	0\\
240.01	0\\
241.01	0\\
242.01	0\\
243.01	0\\
244.01	0\\
245.01	0\\
246.01	0\\
247.01	0\\
248.01	0\\
249.01	0\\
250.01	0\\
251.01	0\\
252.01	0\\
253.01	0\\
254.01	0\\
255.01	0\\
256.01	0\\
257.01	0\\
258.01	0\\
259.01	0\\
260.01	0\\
261.01	0\\
262.01	0\\
263.01	0\\
264.01	0\\
265.01	0\\
266.01	0\\
267.01	0\\
268.01	0\\
269.01	0\\
270.01	0\\
271.01	0\\
272.01	0\\
273.01	0\\
274.01	0\\
275.01	0\\
276.01	0\\
277.01	0\\
278.01	0\\
279.01	0\\
280.01	0\\
281.01	0\\
282.01	0\\
283.01	0\\
284.01	0\\
285.01	0\\
286.01	0\\
287.01	0\\
288.01	0\\
289.01	0\\
290.01	0\\
291.01	0\\
292.01	0\\
293.01	0\\
294.01	0\\
295.01	0\\
296.01	0\\
297.01	0\\
298.01	0\\
299.01	0\\
300.01	0\\
301.01	0\\
302.01	0\\
303.01	0\\
304.01	0\\
305.01	0\\
306.01	0\\
307.01	0\\
308.01	0\\
309.01	0\\
310.01	0\\
311.01	0\\
312.01	0\\
313.01	0\\
314.01	0\\
315.01	0\\
316.01	0\\
317.01	0\\
318.01	0\\
319.01	0\\
320.01	0\\
321.01	0\\
322.01	0\\
323.01	0\\
324.01	0\\
325.01	0\\
326.01	0\\
327.01	0\\
328.01	0\\
329.01	0\\
330.01	0\\
331.01	0\\
332.01	0\\
333.01	0\\
334.01	0\\
335.01	0\\
336.01	0\\
337.01	0\\
338.01	0\\
339.01	0\\
340.01	0\\
341.01	0\\
342.01	0\\
343.01	0\\
344.01	0\\
345.01	0\\
346.01	0\\
347.01	0\\
348.01	0\\
349.01	0\\
350.01	0\\
351.01	0\\
352.01	0\\
353.01	0\\
354.01	0\\
355.01	0\\
356.01	0\\
357.01	0\\
358.01	0\\
359.01	0\\
360.01	0\\
361.01	0\\
362.01	0\\
363.01	0\\
364.01	0\\
365.01	0\\
366.01	0\\
367.01	0\\
368.01	0\\
369.01	0\\
370.01	0\\
371.01	0\\
372.01	0\\
373.01	0\\
374.01	0\\
375.01	0\\
376.01	0\\
377.01	0\\
378.01	0\\
379.01	0\\
380.01	0\\
381.01	0\\
382.01	0\\
383.01	0\\
384.01	0\\
385.01	0\\
386.01	0\\
387.01	0\\
388.01	0\\
389.01	0\\
390.01	0\\
391.01	0\\
392.01	0\\
393.01	0\\
394.01	0\\
395.01	0\\
396.01	0\\
397.01	0\\
398.01	0\\
399.01	0\\
400.01	0\\
401.01	0\\
402.01	0\\
403.01	0\\
404.01	0\\
405.01	0\\
406.01	0\\
407.01	0\\
408.01	0\\
409.01	0\\
410.01	0\\
411.01	0\\
412.01	0\\
413.01	0\\
414.01	0\\
415.01	0\\
416.01	0\\
417.01	0\\
418.01	0\\
419.01	0\\
420.01	0\\
421.01	0\\
422.01	0\\
423.01	0\\
424.01	0\\
425.01	0\\
426.01	0\\
427.01	0\\
428.01	0\\
429.01	0\\
430.01	0\\
431.01	0\\
432.01	0\\
433.01	0\\
434.01	0\\
435.01	0\\
436.01	0\\
437.01	0\\
438.01	0\\
439.01	0\\
440.01	0\\
441.01	0\\
442.01	0\\
443.01	0\\
444.01	0\\
445.01	0\\
446.01	0\\
447.01	0\\
448.01	0\\
449.01	0\\
450.01	0\\
451.01	1.73472347597681e-18\\
452.01	0\\
453.01	0\\
454.01	0\\
455.01	0\\
456.01	0\\
457.01	0\\
458.01	0\\
459.01	0\\
460.01	0\\
461.01	0\\
462.01	0\\
463.01	0\\
464.01	0\\
465.01	0\\
466.01	0\\
467.01	0\\
468.01	0\\
469.01	1.73472347597681e-18\\
470.01	0\\
471.01	0\\
472.01	0\\
473.01	0\\
474.01	0\\
475.01	0\\
476.01	1.73472347597681e-18\\
477.01	0\\
478.01	0\\
479.01	0\\
480.01	0\\
481.01	0\\
482.01	0\\
483.01	0\\
484.01	0\\
485.01	0\\
486.01	0\\
487.01	0\\
488.01	0\\
489.01	0\\
490.01	0\\
491.01	0\\
492.01	0\\
493.01	0\\
494.01	0\\
495.01	0\\
496.01	0\\
497.01	0\\
498.01	0\\
499.01	0\\
500.01	0\\
501.01	0\\
502.01	0\\
503.01	0\\
504.01	0\\
505.01	0\\
506.01	0\\
507.01	0\\
508.01	0\\
509.01	0\\
510.01	0\\
511.01	0\\
512.01	1.73472347597681e-18\\
513.01	0\\
514.01	0\\
515.01	0\\
516.01	0\\
517.01	0\\
518.01	0\\
519.01	0\\
520.01	0\\
521.01	0\\
522.01	0\\
523.01	1.73472347597681e-18\\
524.01	0\\
525.01	0\\
526.01	0\\
527.01	0\\
528.01	0\\
529.01	0\\
530.01	0\\
531.01	0\\
532.01	0\\
533.01	0\\
534.01	0\\
535.01	0\\
536.01	1.73472347597681e-18\\
537.01	1.73472347597681e-18\\
538.01	1.73472347597681e-18\\
539.01	0\\
540.01	0\\
541.01	0\\
542.01	0\\
543.01	0\\
544.01	0\\
545.01	1.73472347597681e-18\\
546.01	1.73472347597681e-18\\
547.01	0\\
548.01	0\\
549.01	0\\
550.01	0\\
551.01	0\\
552.01	0\\
553.01	0\\
554.01	0\\
555.01	0\\
556.01	1.73472347597681e-18\\
557.01	0\\
558.01	0\\
559.01	0\\
560.01	1.73472347597681e-18\\
561.01	1.73472347597681e-18\\
562.01	0\\
563.01	0\\
564.01	0\\
565.01	0\\
566.01	0\\
567.01	0\\
568.01	0\\
569.01	0\\
570.01	0\\
571.01	0\\
572.01	0\\
573.01	0\\
574.01	0\\
575.01	0\\
576.01	0\\
577.01	0\\
578.01	1.73472347597681e-18\\
579.01	0\\
580.01	0\\
581.01	0\\
582.01	0\\
583.01	1.73472347597681e-18\\
584.01	0\\
585.01	0\\
586.01	0\\
587.01	0\\
588.01	0\\
589.01	0\\
590.01	0\\
591.01	0\\
592.01	0\\
593.01	0\\
594.01	0\\
595.01	0\\
596.01	0\\
597.01	0\\
598.01	0\\
599.01	0\\
599.02	0\\
599.03	0\\
599.04	0\\
599.05	0\\
599.06	0\\
599.07	0\\
599.08	0\\
599.09	0\\
599.1	0\\
599.11	0\\
599.12	0\\
599.13	0\\
599.14	0\\
599.15	0\\
599.16	0\\
599.17	0\\
599.18	0\\
599.19	0\\
599.2	0\\
599.21	0\\
599.22	0\\
599.23	0\\
599.24	0\\
599.25	0\\
599.26	0\\
599.27	0\\
599.28	0\\
599.29	0\\
599.3	0\\
599.31	0\\
599.32	0\\
599.33	0\\
599.34	0\\
599.35	0\\
599.36	0\\
599.37	0\\
599.38	0\\
599.39	0\\
599.4	0\\
599.41	0\\
599.42	0\\
599.43	0\\
599.44	0\\
599.45	0\\
599.46	0\\
599.47	0\\
599.48	0\\
599.49	0\\
599.5	0\\
599.51	0\\
599.52	0\\
599.53	0\\
599.54	0\\
599.55	0\\
599.56	0\\
599.57	0\\
599.58	0\\
599.59	0\\
599.6	0\\
599.61	0\\
599.62	0\\
599.63	0\\
599.64	0\\
599.65	0\\
599.66	0\\
599.67	0\\
599.68	0\\
599.69	0\\
599.7	0\\
599.71	0\\
599.72	0\\
599.73	0\\
599.74	0\\
599.75	0\\
599.76	0\\
599.77	0\\
599.78	0\\
599.79	0\\
599.8	0\\
599.81	0\\
599.82	0\\
599.83	0\\
599.84	0\\
599.85	0\\
599.86	0\\
599.87	0\\
599.88	0\\
599.89	0\\
599.9	0\\
599.91	0\\
599.92	0\\
599.93	0\\
599.94	0\\
599.95	0\\
599.96	0\\
599.97	0\\
599.98	0\\
599.99	0\\
600	0\\
};
\addplot [color=blue!50!mycolor7,solid,forget plot]
  table[row sep=crcr]{%
0.01	0\\
1.01	0\\
2.01	0\\
3.01	0\\
4.01	0\\
5.01	0\\
6.01	0\\
7.01	0\\
8.01	0\\
9.01	0\\
10.01	0\\
11.01	0\\
12.01	0\\
13.01	0\\
14.01	0\\
15.01	0\\
16.01	0\\
17.01	0\\
18.01	0\\
19.01	0\\
20.01	0\\
21.01	0\\
22.01	0\\
23.01	0\\
24.01	0\\
25.01	0\\
26.01	0\\
27.01	0\\
28.01	0\\
29.01	0\\
30.01	0\\
31.01	0\\
32.01	0\\
33.01	0\\
34.01	0\\
35.01	0\\
36.01	0\\
37.01	0\\
38.01	0\\
39.01	0\\
40.01	0\\
41.01	0\\
42.01	0\\
43.01	0\\
44.01	0\\
45.01	0\\
46.01	0\\
47.01	0\\
48.01	0\\
49.01	0\\
50.01	0\\
51.01	0\\
52.01	0\\
53.01	0\\
54.01	0\\
55.01	0\\
56.01	0\\
57.01	0\\
58.01	0\\
59.01	0\\
60.01	0\\
61.01	0\\
62.01	0\\
63.01	0\\
64.01	0\\
65.01	0\\
66.01	0\\
67.01	0\\
68.01	0\\
69.01	0\\
70.01	0\\
71.01	0\\
72.01	0\\
73.01	0\\
74.01	0\\
75.01	0\\
76.01	0\\
77.01	0\\
78.01	0\\
79.01	0\\
80.01	0\\
81.01	0\\
82.01	0\\
83.01	0\\
84.01	0\\
85.01	0\\
86.01	0\\
87.01	0\\
88.01	0\\
89.01	0\\
90.01	0\\
91.01	0\\
92.01	0\\
93.01	0\\
94.01	0\\
95.01	0\\
96.01	0\\
97.01	0\\
98.01	0\\
99.01	0\\
100.01	0\\
101.01	0\\
102.01	0\\
103.01	0\\
104.01	0\\
105.01	0\\
106.01	0\\
107.01	0\\
108.01	0\\
109.01	0\\
110.01	0\\
111.01	0\\
112.01	0\\
113.01	0\\
114.01	0\\
115.01	0\\
116.01	0\\
117.01	0\\
118.01	0\\
119.01	0\\
120.01	0\\
121.01	0\\
122.01	0\\
123.01	0\\
124.01	0\\
125.01	0\\
126.01	0\\
127.01	0\\
128.01	0\\
129.01	0\\
130.01	0\\
131.01	0\\
132.01	0\\
133.01	0\\
134.01	0\\
135.01	0\\
136.01	0\\
137.01	0\\
138.01	0\\
139.01	0\\
140.01	0\\
141.01	0\\
142.01	0\\
143.01	0\\
144.01	0\\
145.01	0\\
146.01	0\\
147.01	0\\
148.01	0\\
149.01	0\\
150.01	0\\
151.01	0\\
152.01	0\\
153.01	0\\
154.01	0\\
155.01	0\\
156.01	0\\
157.01	0\\
158.01	0\\
159.01	0\\
160.01	0\\
161.01	0\\
162.01	0\\
163.01	0\\
164.01	0\\
165.01	0\\
166.01	0\\
167.01	0\\
168.01	0\\
169.01	0\\
170.01	0\\
171.01	0\\
172.01	0\\
173.01	0\\
174.01	0\\
175.01	0\\
176.01	0\\
177.01	0\\
178.01	0\\
179.01	0\\
180.01	0\\
181.01	0\\
182.01	0\\
183.01	0\\
184.01	0\\
185.01	0\\
186.01	0\\
187.01	0\\
188.01	0\\
189.01	0\\
190.01	0\\
191.01	0\\
192.01	0\\
193.01	0\\
194.01	0\\
195.01	0\\
196.01	0\\
197.01	0\\
198.01	0\\
199.01	0\\
200.01	0\\
201.01	0\\
202.01	0\\
203.01	0\\
204.01	0\\
205.01	0\\
206.01	0\\
207.01	0\\
208.01	0\\
209.01	0\\
210.01	0\\
211.01	0\\
212.01	0\\
213.01	0\\
214.01	0\\
215.01	0\\
216.01	0\\
217.01	0\\
218.01	0\\
219.01	0\\
220.01	0\\
221.01	0\\
222.01	0\\
223.01	0\\
224.01	0\\
225.01	0\\
226.01	0\\
227.01	0\\
228.01	0\\
229.01	0\\
230.01	0\\
231.01	0\\
232.01	0\\
233.01	0\\
234.01	0\\
235.01	0\\
236.01	0\\
237.01	0\\
238.01	0\\
239.01	0\\
240.01	0\\
241.01	0\\
242.01	0\\
243.01	0\\
244.01	0\\
245.01	0\\
246.01	0\\
247.01	0\\
248.01	0\\
249.01	0\\
250.01	0\\
251.01	0\\
252.01	0\\
253.01	0\\
254.01	0\\
255.01	0\\
256.01	0\\
257.01	0\\
258.01	0\\
259.01	0\\
260.01	0\\
261.01	0\\
262.01	0\\
263.01	0\\
264.01	0\\
265.01	0\\
266.01	0\\
267.01	0\\
268.01	0\\
269.01	0\\
270.01	0\\
271.01	0\\
272.01	0\\
273.01	0\\
274.01	0\\
275.01	0\\
276.01	0\\
277.01	0\\
278.01	0\\
279.01	0\\
280.01	0\\
281.01	0\\
282.01	0\\
283.01	0\\
284.01	0\\
285.01	0\\
286.01	0\\
287.01	0\\
288.01	0\\
289.01	0\\
290.01	0\\
291.01	0\\
292.01	0\\
293.01	0\\
294.01	0\\
295.01	0\\
296.01	0\\
297.01	0\\
298.01	0\\
299.01	0\\
300.01	0\\
301.01	0\\
302.01	0\\
303.01	0\\
304.01	0\\
305.01	0\\
306.01	0\\
307.01	0\\
308.01	0\\
309.01	0\\
310.01	0\\
311.01	0\\
312.01	0\\
313.01	0\\
314.01	0\\
315.01	0\\
316.01	0\\
317.01	0\\
318.01	0\\
319.01	0\\
320.01	0\\
321.01	0\\
322.01	0\\
323.01	0\\
324.01	0\\
325.01	0\\
326.01	0\\
327.01	0\\
328.01	0\\
329.01	0\\
330.01	0\\
331.01	0\\
332.01	0\\
333.01	0\\
334.01	0\\
335.01	0\\
336.01	0\\
337.01	0\\
338.01	0\\
339.01	0\\
340.01	0\\
341.01	0\\
342.01	0\\
343.01	0\\
344.01	0\\
345.01	0\\
346.01	0\\
347.01	0\\
348.01	0\\
349.01	0\\
350.01	0\\
351.01	0\\
352.01	0\\
353.01	0\\
354.01	0\\
355.01	0\\
356.01	0\\
357.01	0\\
358.01	0\\
359.01	0\\
360.01	0\\
361.01	0\\
362.01	0\\
363.01	0\\
364.01	0\\
365.01	0\\
366.01	0\\
367.01	0\\
368.01	0\\
369.01	0\\
370.01	0\\
371.01	0\\
372.01	0\\
373.01	0\\
374.01	0\\
375.01	0\\
376.01	0\\
377.01	0\\
378.01	0\\
379.01	0\\
380.01	0\\
381.01	0\\
382.01	0\\
383.01	0\\
384.01	0\\
385.01	0\\
386.01	0\\
387.01	0\\
388.01	0\\
389.01	0\\
390.01	0\\
391.01	0\\
392.01	0\\
393.01	0\\
394.01	0\\
395.01	0\\
396.01	0\\
397.01	0\\
398.01	0\\
399.01	0\\
400.01	0\\
401.01	0\\
402.01	0\\
403.01	0\\
404.01	0\\
405.01	0\\
406.01	0\\
407.01	0\\
408.01	0\\
409.01	0\\
410.01	0\\
411.01	0\\
412.01	0\\
413.01	0\\
414.01	0\\
415.01	0\\
416.01	0\\
417.01	0\\
418.01	0\\
419.01	0\\
420.01	0\\
421.01	0\\
422.01	0\\
423.01	0\\
424.01	0\\
425.01	0\\
426.01	0\\
427.01	0\\
428.01	0\\
429.01	0\\
430.01	0\\
431.01	0\\
432.01	0\\
433.01	0\\
434.01	0\\
435.01	0\\
436.01	0\\
437.01	0\\
438.01	0\\
439.01	0\\
440.01	0\\
441.01	0\\
442.01	0\\
443.01	0\\
444.01	0\\
445.01	0\\
446.01	0\\
447.01	0\\
448.01	0\\
449.01	0\\
450.01	0\\
451.01	1.73472347597681e-18\\
452.01	0\\
453.01	0\\
454.01	0\\
455.01	0\\
456.01	0\\
457.01	0\\
458.01	0\\
459.01	0\\
460.01	0\\
461.01	0\\
462.01	0\\
463.01	0\\
464.01	0\\
465.01	0\\
466.01	0\\
467.01	0\\
468.01	0\\
469.01	1.73472347597681e-18\\
470.01	0\\
471.01	0\\
472.01	0\\
473.01	0\\
474.01	0\\
475.01	0\\
476.01	1.73472347597681e-18\\
477.01	0\\
478.01	0\\
479.01	0\\
480.01	0\\
481.01	0\\
482.01	0\\
483.01	0\\
484.01	0\\
485.01	0\\
486.01	0\\
487.01	0\\
488.01	0\\
489.01	0\\
490.01	0\\
491.01	0\\
492.01	0\\
493.01	0\\
494.01	0\\
495.01	0\\
496.01	0\\
497.01	0\\
498.01	0\\
499.01	0\\
500.01	0\\
501.01	0\\
502.01	0\\
503.01	0\\
504.01	0\\
505.01	0\\
506.01	0\\
507.01	0\\
508.01	0\\
509.01	0\\
510.01	0\\
511.01	0\\
512.01	1.73472347597681e-18\\
513.01	0\\
514.01	0\\
515.01	0\\
516.01	0\\
517.01	0\\
518.01	0\\
519.01	0\\
520.01	0\\
521.01	0\\
522.01	0\\
523.01	1.73472347597681e-18\\
524.01	0\\
525.01	0\\
526.01	0\\
527.01	0\\
528.01	0\\
529.01	0\\
530.01	0\\
531.01	0\\
532.01	0\\
533.01	0\\
534.01	0\\
535.01	0\\
536.01	1.73472347597681e-18\\
537.01	1.73472347597681e-18\\
538.01	1.73472347597681e-18\\
539.01	0\\
540.01	0\\
541.01	0\\
542.01	0\\
543.01	0\\
544.01	0\\
545.01	1.73472347597681e-18\\
546.01	1.73472347597681e-18\\
547.01	0\\
548.01	0\\
549.01	0\\
550.01	0\\
551.01	0\\
552.01	0\\
553.01	0\\
554.01	0\\
555.01	0\\
556.01	1.73472347597681e-18\\
557.01	0\\
558.01	0\\
559.01	0\\
560.01	1.73472347597681e-18\\
561.01	1.73472347597681e-18\\
562.01	0\\
563.01	0\\
564.01	0\\
565.01	0\\
566.01	0\\
567.01	0\\
568.01	0\\
569.01	0\\
570.01	0\\
571.01	0\\
572.01	0\\
573.01	0\\
574.01	0\\
575.01	0\\
576.01	0\\
577.01	0\\
578.01	1.73472347597681e-18\\
579.01	0\\
580.01	0\\
581.01	0\\
582.01	0\\
583.01	1.73472347597681e-18\\
584.01	0\\
585.01	0\\
586.01	0\\
587.01	0\\
588.01	0\\
589.01	0\\
590.01	0\\
591.01	0\\
592.01	0\\
593.01	0\\
594.01	0\\
595.01	0\\
596.01	0\\
597.01	0\\
598.01	0\\
599.01	0\\
599.02	0\\
599.03	0\\
599.04	0\\
599.05	0\\
599.06	0\\
599.07	0\\
599.08	0\\
599.09	0\\
599.1	0\\
599.11	0\\
599.12	0\\
599.13	0\\
599.14	0\\
599.15	0\\
599.16	0\\
599.17	0\\
599.18	0\\
599.19	0\\
599.2	0\\
599.21	0\\
599.22	0\\
599.23	0\\
599.24	0\\
599.25	0\\
599.26	0\\
599.27	0\\
599.28	0\\
599.29	0\\
599.3	0\\
599.31	0\\
599.32	0\\
599.33	0\\
599.34	0\\
599.35	0\\
599.36	0\\
599.37	0\\
599.38	0\\
599.39	0\\
599.4	0\\
599.41	0\\
599.42	0\\
599.43	0\\
599.44	0\\
599.45	0\\
599.46	0\\
599.47	0\\
599.48	0\\
599.49	0\\
599.5	0\\
599.51	0\\
599.52	0\\
599.53	0\\
599.54	0\\
599.55	0\\
599.56	0\\
599.57	0\\
599.58	0\\
599.59	0\\
599.6	0\\
599.61	0\\
599.62	0\\
599.63	0\\
599.64	0\\
599.65	0\\
599.66	0\\
599.67	0\\
599.68	0\\
599.69	0\\
599.7	0\\
599.71	0\\
599.72	0\\
599.73	0\\
599.74	0\\
599.75	0\\
599.76	0\\
599.77	0\\
599.78	0\\
599.79	0\\
599.8	0\\
599.81	0\\
599.82	0\\
599.83	0\\
599.84	0\\
599.85	0\\
599.86	0\\
599.87	0\\
599.88	0\\
599.89	0\\
599.9	0\\
599.91	0\\
599.92	0\\
599.93	0\\
599.94	0\\
599.95	0\\
599.96	0\\
599.97	0\\
599.98	0\\
599.99	0\\
600	0\\
};
\addplot [color=blue!40!mycolor9,solid,forget plot]
  table[row sep=crcr]{%
0.01	0\\
1.01	0\\
2.01	0\\
3.01	0\\
4.01	0\\
5.01	0\\
6.01	0\\
7.01	0\\
8.01	0\\
9.01	0\\
10.01	0\\
11.01	0\\
12.01	0\\
13.01	0\\
14.01	0\\
15.01	0\\
16.01	0\\
17.01	0\\
18.01	0\\
19.01	0\\
20.01	0\\
21.01	0\\
22.01	0\\
23.01	0\\
24.01	0\\
25.01	0\\
26.01	0\\
27.01	0\\
28.01	0\\
29.01	0\\
30.01	0\\
31.01	0\\
32.01	0\\
33.01	0\\
34.01	0\\
35.01	0\\
36.01	0\\
37.01	0\\
38.01	0\\
39.01	0\\
40.01	0\\
41.01	0\\
42.01	0\\
43.01	0\\
44.01	0\\
45.01	0\\
46.01	0\\
47.01	0\\
48.01	0\\
49.01	0\\
50.01	0\\
51.01	0\\
52.01	0\\
53.01	0\\
54.01	0\\
55.01	0\\
56.01	0\\
57.01	0\\
58.01	0\\
59.01	0\\
60.01	0\\
61.01	0\\
62.01	0\\
63.01	0\\
64.01	0\\
65.01	0\\
66.01	0\\
67.01	0\\
68.01	0\\
69.01	0\\
70.01	0\\
71.01	0\\
72.01	0\\
73.01	0\\
74.01	0\\
75.01	0\\
76.01	0\\
77.01	0\\
78.01	0\\
79.01	0\\
80.01	0\\
81.01	0\\
82.01	0\\
83.01	0\\
84.01	0\\
85.01	0\\
86.01	0\\
87.01	0\\
88.01	0\\
89.01	0\\
90.01	0\\
91.01	0\\
92.01	0\\
93.01	0\\
94.01	0\\
95.01	0\\
96.01	0\\
97.01	0\\
98.01	0\\
99.01	0\\
100.01	0\\
101.01	0\\
102.01	0\\
103.01	0\\
104.01	0\\
105.01	0\\
106.01	0\\
107.01	0\\
108.01	0\\
109.01	0\\
110.01	0\\
111.01	0\\
112.01	0\\
113.01	0\\
114.01	0\\
115.01	0\\
116.01	0\\
117.01	0\\
118.01	0\\
119.01	0\\
120.01	0\\
121.01	0\\
122.01	0\\
123.01	0\\
124.01	0\\
125.01	0\\
126.01	0\\
127.01	0\\
128.01	0\\
129.01	0\\
130.01	0\\
131.01	0\\
132.01	0\\
133.01	0\\
134.01	0\\
135.01	0\\
136.01	0\\
137.01	0\\
138.01	0\\
139.01	0\\
140.01	0\\
141.01	0\\
142.01	0\\
143.01	0\\
144.01	0\\
145.01	0\\
146.01	0\\
147.01	0\\
148.01	0\\
149.01	0\\
150.01	0\\
151.01	0\\
152.01	0\\
153.01	0\\
154.01	0\\
155.01	0\\
156.01	0\\
157.01	0\\
158.01	0\\
159.01	0\\
160.01	0\\
161.01	0\\
162.01	0\\
163.01	0\\
164.01	0\\
165.01	0\\
166.01	0\\
167.01	0\\
168.01	0\\
169.01	0\\
170.01	0\\
171.01	0\\
172.01	0\\
173.01	0\\
174.01	0\\
175.01	0\\
176.01	0\\
177.01	0\\
178.01	0\\
179.01	0\\
180.01	0\\
181.01	0\\
182.01	0\\
183.01	0\\
184.01	0\\
185.01	0\\
186.01	0\\
187.01	0\\
188.01	0\\
189.01	0\\
190.01	0\\
191.01	0\\
192.01	0\\
193.01	0\\
194.01	0\\
195.01	0\\
196.01	0\\
197.01	0\\
198.01	0\\
199.01	0\\
200.01	0\\
201.01	0\\
202.01	0\\
203.01	0\\
204.01	0\\
205.01	0\\
206.01	0\\
207.01	0\\
208.01	0\\
209.01	0\\
210.01	0\\
211.01	0\\
212.01	0\\
213.01	0\\
214.01	0\\
215.01	0\\
216.01	0\\
217.01	0\\
218.01	0\\
219.01	0\\
220.01	0\\
221.01	0\\
222.01	0\\
223.01	0\\
224.01	0\\
225.01	0\\
226.01	0\\
227.01	0\\
228.01	0\\
229.01	0\\
230.01	0\\
231.01	0\\
232.01	0\\
233.01	0\\
234.01	0\\
235.01	0\\
236.01	0\\
237.01	0\\
238.01	0\\
239.01	0\\
240.01	0\\
241.01	0\\
242.01	0\\
243.01	0\\
244.01	0\\
245.01	0\\
246.01	0\\
247.01	0\\
248.01	0\\
249.01	0\\
250.01	0\\
251.01	0\\
252.01	0\\
253.01	0\\
254.01	0\\
255.01	0\\
256.01	0\\
257.01	0\\
258.01	0\\
259.01	0\\
260.01	0\\
261.01	0\\
262.01	0\\
263.01	0\\
264.01	0\\
265.01	0\\
266.01	0\\
267.01	0\\
268.01	0\\
269.01	0\\
270.01	0\\
271.01	0\\
272.01	0\\
273.01	0\\
274.01	0\\
275.01	0\\
276.01	0\\
277.01	0\\
278.01	0\\
279.01	0\\
280.01	0\\
281.01	0\\
282.01	0\\
283.01	0\\
284.01	0\\
285.01	0\\
286.01	0\\
287.01	0\\
288.01	0\\
289.01	0\\
290.01	0\\
291.01	0\\
292.01	0\\
293.01	0\\
294.01	0\\
295.01	0\\
296.01	0\\
297.01	0\\
298.01	0\\
299.01	0\\
300.01	0\\
301.01	0\\
302.01	0\\
303.01	0\\
304.01	0\\
305.01	0\\
306.01	0\\
307.01	0\\
308.01	0\\
309.01	0\\
310.01	0\\
311.01	0\\
312.01	0\\
313.01	0\\
314.01	0\\
315.01	0\\
316.01	0\\
317.01	0\\
318.01	0\\
319.01	0\\
320.01	0\\
321.01	0\\
322.01	0\\
323.01	0\\
324.01	0\\
325.01	0\\
326.01	0\\
327.01	0\\
328.01	0\\
329.01	0\\
330.01	0\\
331.01	0\\
332.01	0\\
333.01	0\\
334.01	0\\
335.01	0\\
336.01	0\\
337.01	0\\
338.01	0\\
339.01	0\\
340.01	0\\
341.01	0\\
342.01	0\\
343.01	0\\
344.01	0\\
345.01	0\\
346.01	0\\
347.01	0\\
348.01	0\\
349.01	0\\
350.01	0\\
351.01	0\\
352.01	0\\
353.01	0\\
354.01	0\\
355.01	0\\
356.01	0\\
357.01	0\\
358.01	0\\
359.01	0\\
360.01	0\\
361.01	0\\
362.01	0\\
363.01	0\\
364.01	0\\
365.01	0\\
366.01	0\\
367.01	0\\
368.01	0\\
369.01	0\\
370.01	0\\
371.01	0\\
372.01	0\\
373.01	0\\
374.01	0\\
375.01	0\\
376.01	0\\
377.01	0\\
378.01	0\\
379.01	0\\
380.01	0\\
381.01	0\\
382.01	0\\
383.01	0\\
384.01	0\\
385.01	0\\
386.01	0\\
387.01	0\\
388.01	0\\
389.01	0\\
390.01	0\\
391.01	0\\
392.01	0\\
393.01	0\\
394.01	0\\
395.01	0\\
396.01	0\\
397.01	0\\
398.01	0\\
399.01	0\\
400.01	0\\
401.01	0\\
402.01	0\\
403.01	0\\
404.01	0\\
405.01	0\\
406.01	0\\
407.01	0\\
408.01	0\\
409.01	0\\
410.01	0\\
411.01	0\\
412.01	0\\
413.01	0\\
414.01	0\\
415.01	0\\
416.01	0\\
417.01	0\\
418.01	0\\
419.01	0\\
420.01	0\\
421.01	0\\
422.01	0\\
423.01	0\\
424.01	0\\
425.01	0\\
426.01	0\\
427.01	0\\
428.01	0\\
429.01	0\\
430.01	0\\
431.01	0\\
432.01	0\\
433.01	0\\
434.01	0\\
435.01	0\\
436.01	0\\
437.01	0\\
438.01	0\\
439.01	0\\
440.01	0\\
441.01	0\\
442.01	0\\
443.01	0\\
444.01	0\\
445.01	0\\
446.01	0\\
447.01	0\\
448.01	0\\
449.01	0\\
450.01	0\\
451.01	1.73472347597681e-18\\
452.01	0\\
453.01	0\\
454.01	0\\
455.01	0\\
456.01	0\\
457.01	0\\
458.01	0\\
459.01	0\\
460.01	0\\
461.01	0\\
462.01	0\\
463.01	0\\
464.01	0\\
465.01	0\\
466.01	0\\
467.01	0\\
468.01	0\\
469.01	1.73472347597681e-18\\
470.01	0\\
471.01	0\\
472.01	0\\
473.01	0\\
474.01	0\\
475.01	0\\
476.01	1.73472347597681e-18\\
477.01	0\\
478.01	0\\
479.01	0\\
480.01	0\\
481.01	0\\
482.01	0\\
483.01	0\\
484.01	0\\
485.01	0\\
486.01	0\\
487.01	0\\
488.01	0\\
489.01	0\\
490.01	0\\
491.01	0\\
492.01	0\\
493.01	0\\
494.01	0\\
495.01	0\\
496.01	0\\
497.01	0\\
498.01	0\\
499.01	0\\
500.01	0\\
501.01	0\\
502.01	0\\
503.01	0\\
504.01	0\\
505.01	0\\
506.01	0\\
507.01	0\\
508.01	0\\
509.01	0\\
510.01	0\\
511.01	0\\
512.01	1.73472347597681e-18\\
513.01	0\\
514.01	0\\
515.01	0\\
516.01	0\\
517.01	0\\
518.01	0\\
519.01	0\\
520.01	0\\
521.01	0\\
522.01	0\\
523.01	1.73472347597681e-18\\
524.01	0\\
525.01	0\\
526.01	0\\
527.01	0\\
528.01	0\\
529.01	0\\
530.01	0\\
531.01	0\\
532.01	0\\
533.01	0\\
534.01	0\\
535.01	0\\
536.01	1.73472347597681e-18\\
537.01	1.73472347597681e-18\\
538.01	1.73472347597681e-18\\
539.01	0\\
540.01	0\\
541.01	0\\
542.01	0\\
543.01	0\\
544.01	0\\
545.01	1.73472347597681e-18\\
546.01	1.73472347597681e-18\\
547.01	0\\
548.01	0\\
549.01	0\\
550.01	0\\
551.01	0\\
552.01	0\\
553.01	0\\
554.01	0\\
555.01	0\\
556.01	1.73472347597681e-18\\
557.01	0\\
558.01	0\\
559.01	0\\
560.01	1.73472347597681e-18\\
561.01	1.73472347597681e-18\\
562.01	0\\
563.01	0\\
564.01	0\\
565.01	0\\
566.01	0\\
567.01	0\\
568.01	0\\
569.01	0\\
570.01	0\\
571.01	0\\
572.01	0\\
573.01	0\\
574.01	0\\
575.01	0\\
576.01	0\\
577.01	0\\
578.01	1.73472347597681e-18\\
579.01	0\\
580.01	0\\
581.01	0\\
582.01	0\\
583.01	1.73472347597681e-18\\
584.01	0\\
585.01	0\\
586.01	0\\
587.01	0\\
588.01	0\\
589.01	0\\
590.01	0\\
591.01	0\\
592.01	0\\
593.01	0\\
594.01	0\\
595.01	0\\
596.01	0\\
597.01	0\\
598.01	0\\
599.01	0\\
599.02	0\\
599.03	0\\
599.04	0\\
599.05	0\\
599.06	0\\
599.07	0\\
599.08	0\\
599.09	0\\
599.1	0\\
599.11	0\\
599.12	0\\
599.13	0\\
599.14	0\\
599.15	0\\
599.16	0\\
599.17	0\\
599.18	0\\
599.19	0\\
599.2	0\\
599.21	0\\
599.22	0\\
599.23	0\\
599.24	0\\
599.25	0\\
599.26	0\\
599.27	0\\
599.28	0\\
599.29	0\\
599.3	0\\
599.31	0\\
599.32	0\\
599.33	0\\
599.34	0\\
599.35	0\\
599.36	0\\
599.37	0\\
599.38	0\\
599.39	0\\
599.4	0\\
599.41	0\\
599.42	0\\
599.43	0\\
599.44	0\\
599.45	0\\
599.46	0\\
599.47	0\\
599.48	0\\
599.49	0\\
599.5	0\\
599.51	0\\
599.52	0\\
599.53	0\\
599.54	0\\
599.55	0\\
599.56	0\\
599.57	0\\
599.58	0\\
599.59	0\\
599.6	0\\
599.61	0\\
599.62	0\\
599.63	0\\
599.64	0\\
599.65	0\\
599.66	0\\
599.67	0\\
599.68	0\\
599.69	0\\
599.7	0\\
599.71	0\\
599.72	0\\
599.73	0\\
599.74	0\\
599.75	0\\
599.76	0\\
599.77	0\\
599.78	0\\
599.79	0\\
599.8	0\\
599.81	0\\
599.82	0\\
599.83	0\\
599.84	0\\
599.85	0\\
599.86	0\\
599.87	0\\
599.88	0\\
599.89	0\\
599.9	0\\
599.91	0\\
599.92	0\\
599.93	0\\
599.94	0\\
599.95	0\\
599.96	0\\
599.97	0\\
599.98	0\\
599.99	0\\
600	0\\
};
\addplot [color=blue!75!mycolor7,solid,forget plot]
  table[row sep=crcr]{%
0.01	0\\
1.01	0\\
2.01	0\\
3.01	0\\
4.01	0\\
5.01	0\\
6.01	0\\
7.01	0\\
8.01	0\\
9.01	0\\
10.01	0\\
11.01	0\\
12.01	0\\
13.01	0\\
14.01	0\\
15.01	0\\
16.01	0\\
17.01	0\\
18.01	0\\
19.01	0\\
20.01	0\\
21.01	0\\
22.01	0\\
23.01	0\\
24.01	0\\
25.01	0\\
26.01	0\\
27.01	0\\
28.01	0\\
29.01	0\\
30.01	0\\
31.01	0\\
32.01	0\\
33.01	0\\
34.01	0\\
35.01	0\\
36.01	0\\
37.01	0\\
38.01	0\\
39.01	0\\
40.01	0\\
41.01	0\\
42.01	0\\
43.01	0\\
44.01	0\\
45.01	0\\
46.01	0\\
47.01	0\\
48.01	0\\
49.01	0\\
50.01	0\\
51.01	0\\
52.01	0\\
53.01	0\\
54.01	0\\
55.01	0\\
56.01	0\\
57.01	0\\
58.01	0\\
59.01	0\\
60.01	0\\
61.01	0\\
62.01	0\\
63.01	0\\
64.01	0\\
65.01	0\\
66.01	0\\
67.01	0\\
68.01	0\\
69.01	0\\
70.01	0\\
71.01	0\\
72.01	0\\
73.01	0\\
74.01	0\\
75.01	0\\
76.01	0\\
77.01	0\\
78.01	0\\
79.01	0\\
80.01	0\\
81.01	0\\
82.01	0\\
83.01	0\\
84.01	0\\
85.01	0\\
86.01	0\\
87.01	0\\
88.01	0\\
89.01	0\\
90.01	0\\
91.01	0\\
92.01	0\\
93.01	0\\
94.01	0\\
95.01	0\\
96.01	0\\
97.01	0\\
98.01	0\\
99.01	0\\
100.01	0\\
101.01	0\\
102.01	0\\
103.01	0\\
104.01	0\\
105.01	0\\
106.01	0\\
107.01	0\\
108.01	0\\
109.01	0\\
110.01	0\\
111.01	0\\
112.01	0\\
113.01	0\\
114.01	0\\
115.01	0\\
116.01	0\\
117.01	0\\
118.01	0\\
119.01	0\\
120.01	0\\
121.01	0\\
122.01	0\\
123.01	0\\
124.01	0\\
125.01	0\\
126.01	0\\
127.01	0\\
128.01	0\\
129.01	0\\
130.01	0\\
131.01	0\\
132.01	0\\
133.01	0\\
134.01	0\\
135.01	0\\
136.01	0\\
137.01	0\\
138.01	0\\
139.01	0\\
140.01	0\\
141.01	0\\
142.01	0\\
143.01	0\\
144.01	0\\
145.01	0\\
146.01	0\\
147.01	0\\
148.01	0\\
149.01	0\\
150.01	0\\
151.01	0\\
152.01	0\\
153.01	0\\
154.01	0\\
155.01	0\\
156.01	0\\
157.01	0\\
158.01	0\\
159.01	0\\
160.01	0\\
161.01	0\\
162.01	0\\
163.01	0\\
164.01	0\\
165.01	0\\
166.01	0\\
167.01	0\\
168.01	0\\
169.01	0\\
170.01	0\\
171.01	0\\
172.01	0\\
173.01	0\\
174.01	0\\
175.01	0\\
176.01	0\\
177.01	0\\
178.01	0\\
179.01	0\\
180.01	0\\
181.01	0\\
182.01	0\\
183.01	0\\
184.01	0\\
185.01	0\\
186.01	0\\
187.01	0\\
188.01	0\\
189.01	0\\
190.01	0\\
191.01	0\\
192.01	0\\
193.01	0\\
194.01	0\\
195.01	0\\
196.01	0\\
197.01	0\\
198.01	0\\
199.01	0\\
200.01	0\\
201.01	0\\
202.01	0\\
203.01	0\\
204.01	0\\
205.01	0\\
206.01	0\\
207.01	0\\
208.01	0\\
209.01	0\\
210.01	0\\
211.01	0\\
212.01	0\\
213.01	0\\
214.01	0\\
215.01	0\\
216.01	0\\
217.01	0\\
218.01	0\\
219.01	0\\
220.01	0\\
221.01	0\\
222.01	0\\
223.01	0\\
224.01	0\\
225.01	0\\
226.01	0\\
227.01	0\\
228.01	0\\
229.01	0\\
230.01	0\\
231.01	0\\
232.01	0\\
233.01	0\\
234.01	0\\
235.01	0\\
236.01	0\\
237.01	0\\
238.01	0\\
239.01	0\\
240.01	0\\
241.01	0\\
242.01	0\\
243.01	0\\
244.01	0\\
245.01	0\\
246.01	0\\
247.01	0\\
248.01	0\\
249.01	0\\
250.01	0\\
251.01	0\\
252.01	0\\
253.01	0\\
254.01	0\\
255.01	0\\
256.01	0\\
257.01	0\\
258.01	0\\
259.01	0\\
260.01	0\\
261.01	0\\
262.01	0\\
263.01	0\\
264.01	0\\
265.01	0\\
266.01	0\\
267.01	0\\
268.01	0\\
269.01	0\\
270.01	0\\
271.01	0\\
272.01	0\\
273.01	0\\
274.01	0\\
275.01	0\\
276.01	0\\
277.01	0\\
278.01	0\\
279.01	0\\
280.01	0\\
281.01	0\\
282.01	0\\
283.01	0\\
284.01	0\\
285.01	0\\
286.01	0\\
287.01	0\\
288.01	0\\
289.01	0\\
290.01	0\\
291.01	0\\
292.01	0\\
293.01	0\\
294.01	0\\
295.01	0\\
296.01	0\\
297.01	0\\
298.01	0\\
299.01	0\\
300.01	0\\
301.01	0\\
302.01	0\\
303.01	0\\
304.01	0\\
305.01	0\\
306.01	0\\
307.01	0\\
308.01	0\\
309.01	0\\
310.01	0\\
311.01	0\\
312.01	0\\
313.01	0\\
314.01	0\\
315.01	0\\
316.01	0\\
317.01	0\\
318.01	0\\
319.01	0\\
320.01	0\\
321.01	0\\
322.01	0\\
323.01	0\\
324.01	0\\
325.01	0\\
326.01	0\\
327.01	0\\
328.01	0\\
329.01	0\\
330.01	0\\
331.01	0\\
332.01	0\\
333.01	0\\
334.01	0\\
335.01	0\\
336.01	0\\
337.01	0\\
338.01	0\\
339.01	0\\
340.01	0\\
341.01	0\\
342.01	0\\
343.01	0\\
344.01	0\\
345.01	0\\
346.01	0\\
347.01	0\\
348.01	0\\
349.01	0\\
350.01	0\\
351.01	0\\
352.01	0\\
353.01	0\\
354.01	0\\
355.01	0\\
356.01	0\\
357.01	0\\
358.01	0\\
359.01	0\\
360.01	0\\
361.01	0\\
362.01	0\\
363.01	0\\
364.01	0\\
365.01	0\\
366.01	0\\
367.01	0\\
368.01	0\\
369.01	0\\
370.01	0\\
371.01	0\\
372.01	0\\
373.01	0\\
374.01	0\\
375.01	0\\
376.01	0\\
377.01	0\\
378.01	0\\
379.01	0\\
380.01	0\\
381.01	0\\
382.01	0\\
383.01	0\\
384.01	0\\
385.01	0\\
386.01	0\\
387.01	0\\
388.01	0\\
389.01	0\\
390.01	0\\
391.01	0\\
392.01	0\\
393.01	0\\
394.01	0\\
395.01	0\\
396.01	0\\
397.01	0\\
398.01	0\\
399.01	0\\
400.01	0\\
401.01	0\\
402.01	0\\
403.01	0\\
404.01	0\\
405.01	0\\
406.01	0\\
407.01	0\\
408.01	0\\
409.01	0\\
410.01	0\\
411.01	0\\
412.01	0\\
413.01	0\\
414.01	0\\
415.01	0\\
416.01	0\\
417.01	0\\
418.01	0\\
419.01	0\\
420.01	0\\
421.01	0\\
422.01	0\\
423.01	0\\
424.01	0\\
425.01	0\\
426.01	0\\
427.01	0\\
428.01	0\\
429.01	0\\
430.01	0\\
431.01	0\\
432.01	0\\
433.01	0\\
434.01	0\\
435.01	0\\
436.01	0\\
437.01	0\\
438.01	0\\
439.01	0\\
440.01	0\\
441.01	0\\
442.01	0\\
443.01	0\\
444.01	0\\
445.01	0\\
446.01	0\\
447.01	0\\
448.01	0\\
449.01	0\\
450.01	0\\
451.01	1.73472347597681e-18\\
452.01	0\\
453.01	0\\
454.01	0\\
455.01	0\\
456.01	0\\
457.01	0\\
458.01	0\\
459.01	0\\
460.01	0\\
461.01	0\\
462.01	0\\
463.01	0\\
464.01	0\\
465.01	0\\
466.01	0\\
467.01	0\\
468.01	0\\
469.01	1.73472347597681e-18\\
470.01	0\\
471.01	0\\
472.01	0\\
473.01	0\\
474.01	0\\
475.01	0\\
476.01	1.73472347597681e-18\\
477.01	0\\
478.01	0\\
479.01	0\\
480.01	0\\
481.01	0\\
482.01	0\\
483.01	0\\
484.01	0\\
485.01	0\\
486.01	0\\
487.01	0\\
488.01	0\\
489.01	0\\
490.01	0\\
491.01	0\\
492.01	0\\
493.01	0\\
494.01	0\\
495.01	0\\
496.01	0\\
497.01	0\\
498.01	0\\
499.01	0\\
500.01	0\\
501.01	0\\
502.01	0\\
503.01	0\\
504.01	0\\
505.01	0\\
506.01	0\\
507.01	0\\
508.01	0\\
509.01	0\\
510.01	0\\
511.01	0\\
512.01	1.73472347597681e-18\\
513.01	0\\
514.01	0\\
515.01	0\\
516.01	0\\
517.01	0\\
518.01	0\\
519.01	0\\
520.01	0\\
521.01	0\\
522.01	0\\
523.01	1.73472347597681e-18\\
524.01	0\\
525.01	0\\
526.01	0\\
527.01	0\\
528.01	0\\
529.01	0\\
530.01	0\\
531.01	0\\
532.01	0\\
533.01	0\\
534.01	0\\
535.01	0\\
536.01	1.73472347597681e-18\\
537.01	1.73472347597681e-18\\
538.01	1.73472347597681e-18\\
539.01	0\\
540.01	0\\
541.01	0\\
542.01	0\\
543.01	0\\
544.01	0\\
545.01	1.73472347597681e-18\\
546.01	1.73472347597681e-18\\
547.01	0\\
548.01	0\\
549.01	0\\
550.01	0\\
551.01	0\\
552.01	0\\
553.01	0\\
554.01	0\\
555.01	0\\
556.01	1.73472347597681e-18\\
557.01	0\\
558.01	0\\
559.01	0\\
560.01	1.73472347597681e-18\\
561.01	1.73472347597681e-18\\
562.01	0\\
563.01	0\\
564.01	0\\
565.01	0\\
566.01	0\\
567.01	0\\
568.01	0\\
569.01	0\\
570.01	0\\
571.01	0\\
572.01	0\\
573.01	0\\
574.01	0\\
575.01	0\\
576.01	0\\
577.01	0\\
578.01	1.73472347597681e-18\\
579.01	0\\
580.01	0\\
581.01	0\\
582.01	0\\
583.01	1.73472347597681e-18\\
584.01	0\\
585.01	0\\
586.01	0\\
587.01	0\\
588.01	0\\
589.01	0\\
590.01	0\\
591.01	0\\
592.01	0\\
593.01	0\\
594.01	0\\
595.01	0\\
596.01	0\\
597.01	0\\
598.01	0\\
599.01	0\\
599.02	0\\
599.03	0\\
599.04	0\\
599.05	0\\
599.06	0\\
599.07	0\\
599.08	0\\
599.09	0\\
599.1	0\\
599.11	0\\
599.12	0\\
599.13	0\\
599.14	0\\
599.15	0\\
599.16	0\\
599.17	0\\
599.18	0\\
599.19	0\\
599.2	0\\
599.21	0\\
599.22	0\\
599.23	0\\
599.24	0\\
599.25	0\\
599.26	0\\
599.27	0\\
599.28	0\\
599.29	0\\
599.3	0\\
599.31	0\\
599.32	0\\
599.33	0\\
599.34	0\\
599.35	0\\
599.36	0\\
599.37	0\\
599.38	0\\
599.39	0\\
599.4	0\\
599.41	0\\
599.42	0\\
599.43	0\\
599.44	0\\
599.45	0\\
599.46	0\\
599.47	0\\
599.48	0\\
599.49	0\\
599.5	0\\
599.51	0\\
599.52	0\\
599.53	0\\
599.54	0\\
599.55	0\\
599.56	0\\
599.57	0\\
599.58	0\\
599.59	0\\
599.6	0\\
599.61	0\\
599.62	0\\
599.63	0\\
599.64	0\\
599.65	0\\
599.66	0\\
599.67	0\\
599.68	0\\
599.69	0\\
599.7	0\\
599.71	0\\
599.72	0\\
599.73	0\\
599.74	0\\
599.75	0\\
599.76	0\\
599.77	0\\
599.78	0\\
599.79	0\\
599.8	0\\
599.81	0\\
599.82	0\\
599.83	0\\
599.84	0\\
599.85	0\\
599.86	0\\
599.87	0\\
599.88	0\\
599.89	0\\
599.9	0\\
599.91	0\\
599.92	0\\
599.93	0\\
599.94	0\\
599.95	0\\
599.96	0\\
599.97	0\\
599.98	0\\
599.99	0\\
600	0\\
};
\addplot [color=blue!80!mycolor9,solid,forget plot]
  table[row sep=crcr]{%
0.01	0\\
1.01	0\\
2.01	0\\
3.01	0\\
4.01	0\\
5.01	0\\
6.01	0\\
7.01	0\\
8.01	0\\
9.01	0\\
10.01	0\\
11.01	0\\
12.01	0\\
13.01	0\\
14.01	0\\
15.01	0\\
16.01	0\\
17.01	0\\
18.01	0\\
19.01	0\\
20.01	0\\
21.01	0\\
22.01	0\\
23.01	0\\
24.01	0\\
25.01	0\\
26.01	0\\
27.01	0\\
28.01	0\\
29.01	0\\
30.01	0\\
31.01	0\\
32.01	0\\
33.01	0\\
34.01	0\\
35.01	0\\
36.01	0\\
37.01	0\\
38.01	0\\
39.01	0\\
40.01	0\\
41.01	0\\
42.01	0\\
43.01	0\\
44.01	0\\
45.01	0\\
46.01	0\\
47.01	0\\
48.01	0\\
49.01	0\\
50.01	0\\
51.01	0\\
52.01	0\\
53.01	0\\
54.01	0\\
55.01	0\\
56.01	0\\
57.01	0\\
58.01	0\\
59.01	0\\
60.01	0\\
61.01	0\\
62.01	0\\
63.01	0\\
64.01	0\\
65.01	0\\
66.01	0\\
67.01	0\\
68.01	0\\
69.01	0\\
70.01	0\\
71.01	0\\
72.01	0\\
73.01	0\\
74.01	0\\
75.01	0\\
76.01	0\\
77.01	0\\
78.01	0\\
79.01	0\\
80.01	0\\
81.01	0\\
82.01	0\\
83.01	0\\
84.01	0\\
85.01	0\\
86.01	0\\
87.01	0\\
88.01	0\\
89.01	0\\
90.01	0\\
91.01	0\\
92.01	0\\
93.01	0\\
94.01	0\\
95.01	0\\
96.01	0\\
97.01	0\\
98.01	0\\
99.01	0\\
100.01	0\\
101.01	0\\
102.01	0\\
103.01	0\\
104.01	0\\
105.01	0\\
106.01	0\\
107.01	0\\
108.01	0\\
109.01	0\\
110.01	0\\
111.01	0\\
112.01	0\\
113.01	0\\
114.01	0\\
115.01	0\\
116.01	0\\
117.01	0\\
118.01	0\\
119.01	0\\
120.01	0\\
121.01	0\\
122.01	0\\
123.01	0\\
124.01	0\\
125.01	0\\
126.01	0\\
127.01	0\\
128.01	0\\
129.01	0\\
130.01	0\\
131.01	0\\
132.01	0\\
133.01	0\\
134.01	0\\
135.01	0\\
136.01	0\\
137.01	0\\
138.01	0\\
139.01	0\\
140.01	0\\
141.01	0\\
142.01	0\\
143.01	0\\
144.01	0\\
145.01	0\\
146.01	0\\
147.01	0\\
148.01	0\\
149.01	0\\
150.01	0\\
151.01	0\\
152.01	0\\
153.01	0\\
154.01	0\\
155.01	0\\
156.01	0\\
157.01	0\\
158.01	0\\
159.01	0\\
160.01	0\\
161.01	0\\
162.01	0\\
163.01	0\\
164.01	0\\
165.01	0\\
166.01	0\\
167.01	0\\
168.01	0\\
169.01	0\\
170.01	0\\
171.01	0\\
172.01	0\\
173.01	0\\
174.01	0\\
175.01	0\\
176.01	0\\
177.01	0\\
178.01	0\\
179.01	0\\
180.01	0\\
181.01	0\\
182.01	0\\
183.01	0\\
184.01	0\\
185.01	0\\
186.01	0\\
187.01	0\\
188.01	0\\
189.01	0\\
190.01	0\\
191.01	0\\
192.01	0\\
193.01	0\\
194.01	0\\
195.01	0\\
196.01	0\\
197.01	0\\
198.01	0\\
199.01	0\\
200.01	0\\
201.01	0\\
202.01	0\\
203.01	0\\
204.01	0\\
205.01	0\\
206.01	0\\
207.01	0\\
208.01	0\\
209.01	0\\
210.01	0\\
211.01	0\\
212.01	0\\
213.01	0\\
214.01	0\\
215.01	0\\
216.01	0\\
217.01	0\\
218.01	0\\
219.01	0\\
220.01	0\\
221.01	0\\
222.01	0\\
223.01	0\\
224.01	0\\
225.01	0\\
226.01	0\\
227.01	0\\
228.01	0\\
229.01	0\\
230.01	0\\
231.01	0\\
232.01	0\\
233.01	0\\
234.01	0\\
235.01	0\\
236.01	0\\
237.01	0\\
238.01	0\\
239.01	0\\
240.01	0\\
241.01	0\\
242.01	0\\
243.01	0\\
244.01	0\\
245.01	0\\
246.01	0\\
247.01	0\\
248.01	0\\
249.01	0\\
250.01	0\\
251.01	0\\
252.01	0\\
253.01	0\\
254.01	0\\
255.01	0\\
256.01	0\\
257.01	0\\
258.01	0\\
259.01	0\\
260.01	0\\
261.01	0\\
262.01	0\\
263.01	0\\
264.01	0\\
265.01	0\\
266.01	0\\
267.01	0\\
268.01	0\\
269.01	0\\
270.01	0\\
271.01	0\\
272.01	0\\
273.01	0\\
274.01	0\\
275.01	0\\
276.01	0\\
277.01	0\\
278.01	0\\
279.01	0\\
280.01	0\\
281.01	0\\
282.01	0\\
283.01	0\\
284.01	0\\
285.01	0\\
286.01	0\\
287.01	0\\
288.01	0\\
289.01	0\\
290.01	0\\
291.01	0\\
292.01	0\\
293.01	0\\
294.01	0\\
295.01	0\\
296.01	0\\
297.01	0\\
298.01	0\\
299.01	0\\
300.01	0\\
301.01	0\\
302.01	0\\
303.01	0\\
304.01	0\\
305.01	0\\
306.01	0\\
307.01	0\\
308.01	0\\
309.01	0\\
310.01	0\\
311.01	0\\
312.01	0\\
313.01	0\\
314.01	0\\
315.01	0\\
316.01	0\\
317.01	0\\
318.01	0\\
319.01	0\\
320.01	0\\
321.01	0\\
322.01	0\\
323.01	0\\
324.01	0\\
325.01	0\\
326.01	0\\
327.01	0\\
328.01	0\\
329.01	0\\
330.01	0\\
331.01	0\\
332.01	0\\
333.01	0\\
334.01	0\\
335.01	0\\
336.01	0\\
337.01	0\\
338.01	0\\
339.01	0\\
340.01	0\\
341.01	0\\
342.01	0\\
343.01	0\\
344.01	0\\
345.01	0\\
346.01	0\\
347.01	0\\
348.01	0\\
349.01	0\\
350.01	0\\
351.01	0\\
352.01	0\\
353.01	0\\
354.01	0\\
355.01	0\\
356.01	0\\
357.01	0\\
358.01	0\\
359.01	0\\
360.01	0\\
361.01	0\\
362.01	0\\
363.01	0\\
364.01	0\\
365.01	0\\
366.01	0\\
367.01	0\\
368.01	0\\
369.01	0\\
370.01	0\\
371.01	0\\
372.01	0\\
373.01	0\\
374.01	0\\
375.01	0\\
376.01	0\\
377.01	0\\
378.01	0\\
379.01	0\\
380.01	0\\
381.01	0\\
382.01	0\\
383.01	0\\
384.01	0\\
385.01	0\\
386.01	0\\
387.01	0\\
388.01	0\\
389.01	0\\
390.01	0\\
391.01	0\\
392.01	0\\
393.01	0\\
394.01	0\\
395.01	0\\
396.01	0\\
397.01	0\\
398.01	0\\
399.01	0\\
400.01	0\\
401.01	0\\
402.01	0\\
403.01	0\\
404.01	0\\
405.01	0\\
406.01	0\\
407.01	0\\
408.01	0\\
409.01	0\\
410.01	0\\
411.01	0\\
412.01	0\\
413.01	0\\
414.01	0\\
415.01	0\\
416.01	0\\
417.01	0\\
418.01	0\\
419.01	0\\
420.01	0\\
421.01	0\\
422.01	0\\
423.01	0\\
424.01	0\\
425.01	0\\
426.01	0\\
427.01	0\\
428.01	0\\
429.01	0\\
430.01	0\\
431.01	0\\
432.01	0\\
433.01	0\\
434.01	0\\
435.01	0\\
436.01	0\\
437.01	0\\
438.01	0\\
439.01	0\\
440.01	0\\
441.01	0\\
442.01	0\\
443.01	0\\
444.01	0\\
445.01	0\\
446.01	0\\
447.01	0\\
448.01	0\\
449.01	0\\
450.01	0\\
451.01	1.73472347597681e-18\\
452.01	0\\
453.01	0\\
454.01	0\\
455.01	0\\
456.01	0\\
457.01	0\\
458.01	0\\
459.01	0\\
460.01	0\\
461.01	0\\
462.01	0\\
463.01	0\\
464.01	0\\
465.01	0\\
466.01	0\\
467.01	0\\
468.01	0\\
469.01	1.73472347597681e-18\\
470.01	0\\
471.01	0\\
472.01	0\\
473.01	0\\
474.01	0\\
475.01	0\\
476.01	1.73472347597681e-18\\
477.01	0\\
478.01	0\\
479.01	0\\
480.01	0\\
481.01	0\\
482.01	0\\
483.01	0\\
484.01	0\\
485.01	0\\
486.01	0\\
487.01	0\\
488.01	0\\
489.01	0\\
490.01	0\\
491.01	0\\
492.01	0\\
493.01	0\\
494.01	0\\
495.01	0\\
496.01	0\\
497.01	0\\
498.01	0\\
499.01	0\\
500.01	0\\
501.01	0\\
502.01	0\\
503.01	0\\
504.01	0\\
505.01	0\\
506.01	0\\
507.01	0\\
508.01	0\\
509.01	0\\
510.01	0\\
511.01	0\\
512.01	1.73472347597681e-18\\
513.01	0\\
514.01	0\\
515.01	0\\
516.01	0\\
517.01	0\\
518.01	0\\
519.01	0\\
520.01	0\\
521.01	0\\
522.01	0\\
523.01	1.73472347597681e-18\\
524.01	0\\
525.01	0\\
526.01	0\\
527.01	0\\
528.01	0\\
529.01	0\\
530.01	0\\
531.01	0\\
532.01	0\\
533.01	0\\
534.01	0\\
535.01	0\\
536.01	1.73472347597681e-18\\
537.01	1.73472347597681e-18\\
538.01	1.73472347597681e-18\\
539.01	0\\
540.01	0\\
541.01	0\\
542.01	0\\
543.01	0\\
544.01	0\\
545.01	1.73472347597681e-18\\
546.01	1.73472347597681e-18\\
547.01	0\\
548.01	0\\
549.01	0\\
550.01	0\\
551.01	0\\
552.01	0\\
553.01	0\\
554.01	0\\
555.01	0\\
556.01	1.73472347597681e-18\\
557.01	0\\
558.01	0\\
559.01	0\\
560.01	1.73472347597681e-18\\
561.01	1.73472347597681e-18\\
562.01	0\\
563.01	0\\
564.01	0\\
565.01	0\\
566.01	0\\
567.01	0\\
568.01	0\\
569.01	0\\
570.01	0\\
571.01	0\\
572.01	0\\
573.01	0\\
574.01	0\\
575.01	0\\
576.01	0\\
577.01	0\\
578.01	1.73472347597681e-18\\
579.01	0\\
580.01	0\\
581.01	0\\
582.01	0\\
583.01	1.73472347597681e-18\\
584.01	0\\
585.01	0\\
586.01	0\\
587.01	0\\
588.01	0\\
589.01	0\\
590.01	0\\
591.01	0\\
592.01	0\\
593.01	0\\
594.01	0\\
595.01	0\\
596.01	0\\
597.01	0\\
598.01	0\\
599.01	0\\
599.02	0\\
599.03	0\\
599.04	0\\
599.05	0\\
599.06	0\\
599.07	0\\
599.08	0\\
599.09	0\\
599.1	0\\
599.11	0\\
599.12	0\\
599.13	0\\
599.14	0\\
599.15	0\\
599.16	0\\
599.17	0\\
599.18	0\\
599.19	0\\
599.2	0\\
599.21	0\\
599.22	0\\
599.23	0\\
599.24	0\\
599.25	0\\
599.26	0\\
599.27	0\\
599.28	0\\
599.29	0\\
599.3	0\\
599.31	0\\
599.32	0\\
599.33	0\\
599.34	0\\
599.35	0\\
599.36	0\\
599.37	0\\
599.38	0\\
599.39	0\\
599.4	0\\
599.41	0\\
599.42	0\\
599.43	0\\
599.44	0\\
599.45	0\\
599.46	0\\
599.47	0\\
599.48	0\\
599.49	0\\
599.5	0\\
599.51	0\\
599.52	0\\
599.53	0\\
599.54	0\\
599.55	0\\
599.56	0\\
599.57	0\\
599.58	0\\
599.59	0\\
599.6	0\\
599.61	0\\
599.62	0\\
599.63	0\\
599.64	0\\
599.65	0\\
599.66	0\\
599.67	0\\
599.68	0\\
599.69	0\\
599.7	0\\
599.71	0\\
599.72	0\\
599.73	0\\
599.74	0\\
599.75	0\\
599.76	0\\
599.77	0\\
599.78	0\\
599.79	0\\
599.8	0\\
599.81	0\\
599.82	0\\
599.83	0\\
599.84	0\\
599.85	0\\
599.86	0\\
599.87	0\\
599.88	0\\
599.89	0\\
599.9	0\\
599.91	0\\
599.92	0\\
599.93	0\\
599.94	0\\
599.95	0\\
599.96	0\\
599.97	0\\
599.98	0\\
599.99	0\\
600	0\\
};
\addplot [color=blue,solid,forget plot]
  table[row sep=crcr]{%
0.01	0\\
1.01	0\\
2.01	0\\
3.01	0\\
4.01	0\\
5.01	0\\
6.01	0\\
7.01	0\\
8.01	0\\
9.01	0\\
10.01	0\\
11.01	0\\
12.01	0\\
13.01	0\\
14.01	0\\
15.01	0\\
16.01	0\\
17.01	0\\
18.01	0\\
19.01	0\\
20.01	0\\
21.01	0\\
22.01	0\\
23.01	0\\
24.01	0\\
25.01	0\\
26.01	0\\
27.01	0\\
28.01	0\\
29.01	0\\
30.01	0\\
31.01	0\\
32.01	0\\
33.01	0\\
34.01	0\\
35.01	0\\
36.01	0\\
37.01	0\\
38.01	0\\
39.01	0\\
40.01	0\\
41.01	0\\
42.01	0\\
43.01	0\\
44.01	0\\
45.01	0\\
46.01	0\\
47.01	0\\
48.01	0\\
49.01	0\\
50.01	0\\
51.01	0\\
52.01	0\\
53.01	0\\
54.01	0\\
55.01	0\\
56.01	0\\
57.01	0\\
58.01	0\\
59.01	0\\
60.01	0\\
61.01	0\\
62.01	0\\
63.01	0\\
64.01	0\\
65.01	0\\
66.01	0\\
67.01	0\\
68.01	0\\
69.01	0\\
70.01	0\\
71.01	0\\
72.01	0\\
73.01	0\\
74.01	0\\
75.01	0\\
76.01	0\\
77.01	0\\
78.01	0\\
79.01	0\\
80.01	0\\
81.01	0\\
82.01	0\\
83.01	0\\
84.01	0\\
85.01	0\\
86.01	0\\
87.01	0\\
88.01	0\\
89.01	0\\
90.01	0\\
91.01	0\\
92.01	0\\
93.01	0\\
94.01	0\\
95.01	0\\
96.01	0\\
97.01	0\\
98.01	0\\
99.01	0\\
100.01	0\\
101.01	0\\
102.01	0\\
103.01	0\\
104.01	0\\
105.01	0\\
106.01	0\\
107.01	0\\
108.01	0\\
109.01	0\\
110.01	0\\
111.01	0\\
112.01	0\\
113.01	0\\
114.01	0\\
115.01	0\\
116.01	0\\
117.01	0\\
118.01	0\\
119.01	0\\
120.01	0\\
121.01	0\\
122.01	0\\
123.01	0\\
124.01	0\\
125.01	0\\
126.01	0\\
127.01	0\\
128.01	0\\
129.01	0\\
130.01	0\\
131.01	0\\
132.01	0\\
133.01	0\\
134.01	0\\
135.01	0\\
136.01	0\\
137.01	0\\
138.01	0\\
139.01	0\\
140.01	0\\
141.01	0\\
142.01	0\\
143.01	0\\
144.01	0\\
145.01	0\\
146.01	0\\
147.01	0\\
148.01	0\\
149.01	0\\
150.01	0\\
151.01	0\\
152.01	0\\
153.01	0\\
154.01	0\\
155.01	0\\
156.01	0\\
157.01	0\\
158.01	0\\
159.01	0\\
160.01	0\\
161.01	0\\
162.01	0\\
163.01	0\\
164.01	0\\
165.01	0\\
166.01	0\\
167.01	0\\
168.01	0\\
169.01	0\\
170.01	0\\
171.01	0\\
172.01	0\\
173.01	0\\
174.01	0\\
175.01	0\\
176.01	0\\
177.01	0\\
178.01	0\\
179.01	0\\
180.01	0\\
181.01	0\\
182.01	0\\
183.01	0\\
184.01	0\\
185.01	0\\
186.01	0\\
187.01	0\\
188.01	0\\
189.01	0\\
190.01	0\\
191.01	0\\
192.01	0\\
193.01	0\\
194.01	0\\
195.01	0\\
196.01	0\\
197.01	0\\
198.01	0\\
199.01	0\\
200.01	0\\
201.01	0\\
202.01	0\\
203.01	0\\
204.01	0\\
205.01	0\\
206.01	0\\
207.01	0\\
208.01	0\\
209.01	0\\
210.01	0\\
211.01	0\\
212.01	0\\
213.01	0\\
214.01	0\\
215.01	0\\
216.01	0\\
217.01	0\\
218.01	0\\
219.01	0\\
220.01	0\\
221.01	0\\
222.01	0\\
223.01	0\\
224.01	0\\
225.01	0\\
226.01	0\\
227.01	0\\
228.01	0\\
229.01	0\\
230.01	0\\
231.01	0\\
232.01	0\\
233.01	0\\
234.01	0\\
235.01	0\\
236.01	0\\
237.01	0\\
238.01	0\\
239.01	0\\
240.01	0\\
241.01	0\\
242.01	0\\
243.01	0\\
244.01	0\\
245.01	0\\
246.01	0\\
247.01	0\\
248.01	0\\
249.01	0\\
250.01	0\\
251.01	0\\
252.01	0\\
253.01	0\\
254.01	0\\
255.01	0\\
256.01	0\\
257.01	0\\
258.01	0\\
259.01	0\\
260.01	0\\
261.01	0\\
262.01	0\\
263.01	0\\
264.01	0\\
265.01	0\\
266.01	0\\
267.01	0\\
268.01	0\\
269.01	0\\
270.01	0\\
271.01	0\\
272.01	0\\
273.01	0\\
274.01	0\\
275.01	0\\
276.01	0\\
277.01	0\\
278.01	0\\
279.01	0\\
280.01	0\\
281.01	0\\
282.01	0\\
283.01	0\\
284.01	0\\
285.01	0\\
286.01	0\\
287.01	0\\
288.01	0\\
289.01	0\\
290.01	0\\
291.01	0\\
292.01	0\\
293.01	0\\
294.01	0\\
295.01	0\\
296.01	0\\
297.01	0\\
298.01	0\\
299.01	0\\
300.01	0\\
301.01	0\\
302.01	0\\
303.01	0\\
304.01	0\\
305.01	0\\
306.01	0\\
307.01	0\\
308.01	0\\
309.01	0\\
310.01	0\\
311.01	0\\
312.01	0\\
313.01	0\\
314.01	0\\
315.01	0\\
316.01	0\\
317.01	0\\
318.01	0\\
319.01	0\\
320.01	0\\
321.01	0\\
322.01	0\\
323.01	0\\
324.01	0\\
325.01	0\\
326.01	0\\
327.01	0\\
328.01	0\\
329.01	0\\
330.01	0\\
331.01	0\\
332.01	0\\
333.01	0\\
334.01	0\\
335.01	0\\
336.01	0\\
337.01	0\\
338.01	0\\
339.01	0\\
340.01	0\\
341.01	0\\
342.01	0\\
343.01	0\\
344.01	0\\
345.01	0\\
346.01	0\\
347.01	0\\
348.01	0\\
349.01	0\\
350.01	0\\
351.01	0\\
352.01	0\\
353.01	0\\
354.01	0\\
355.01	0\\
356.01	0\\
357.01	0\\
358.01	0\\
359.01	0\\
360.01	0\\
361.01	0\\
362.01	0\\
363.01	0\\
364.01	0\\
365.01	0\\
366.01	0\\
367.01	0\\
368.01	0\\
369.01	0\\
370.01	0\\
371.01	0\\
372.01	0\\
373.01	0\\
374.01	0\\
375.01	0\\
376.01	0\\
377.01	0\\
378.01	0\\
379.01	0\\
380.01	0\\
381.01	0\\
382.01	0\\
383.01	0\\
384.01	0\\
385.01	0\\
386.01	0\\
387.01	0\\
388.01	0\\
389.01	0\\
390.01	0\\
391.01	0\\
392.01	0\\
393.01	0\\
394.01	0\\
395.01	0\\
396.01	0\\
397.01	0\\
398.01	0\\
399.01	0\\
400.01	0\\
401.01	0\\
402.01	0\\
403.01	0\\
404.01	0\\
405.01	0\\
406.01	0\\
407.01	0\\
408.01	0\\
409.01	0\\
410.01	0\\
411.01	0\\
412.01	0\\
413.01	0\\
414.01	0\\
415.01	0\\
416.01	0\\
417.01	0\\
418.01	0\\
419.01	0\\
420.01	0\\
421.01	0\\
422.01	0\\
423.01	0\\
424.01	0\\
425.01	0\\
426.01	0\\
427.01	0\\
428.01	0\\
429.01	0\\
430.01	0\\
431.01	0\\
432.01	0\\
433.01	0\\
434.01	0\\
435.01	0\\
436.01	0\\
437.01	0\\
438.01	0\\
439.01	0\\
440.01	0\\
441.01	0\\
442.01	0\\
443.01	0\\
444.01	0\\
445.01	0\\
446.01	0\\
447.01	0\\
448.01	0\\
449.01	0\\
450.01	0\\
451.01	1.73472347597681e-18\\
452.01	0\\
453.01	0\\
454.01	0\\
455.01	0\\
456.01	0\\
457.01	0\\
458.01	0\\
459.01	0\\
460.01	0\\
461.01	0\\
462.01	0\\
463.01	0\\
464.01	0\\
465.01	0\\
466.01	0\\
467.01	0\\
468.01	0\\
469.01	1.73472347597681e-18\\
470.01	0\\
471.01	0\\
472.01	0\\
473.01	0\\
474.01	0\\
475.01	0\\
476.01	1.73472347597681e-18\\
477.01	0\\
478.01	0\\
479.01	0\\
480.01	0\\
481.01	0\\
482.01	0\\
483.01	0\\
484.01	0\\
485.01	0\\
486.01	0\\
487.01	0\\
488.01	0\\
489.01	0\\
490.01	0\\
491.01	0\\
492.01	0\\
493.01	0\\
494.01	0\\
495.01	0\\
496.01	0\\
497.01	0\\
498.01	0\\
499.01	0\\
500.01	0\\
501.01	0\\
502.01	0\\
503.01	0\\
504.01	0\\
505.01	0\\
506.01	0\\
507.01	0\\
508.01	0\\
509.01	0\\
510.01	0\\
511.01	0\\
512.01	1.73472347597681e-18\\
513.01	0\\
514.01	0\\
515.01	0\\
516.01	0\\
517.01	0\\
518.01	0\\
519.01	0\\
520.01	0\\
521.01	0\\
522.01	0\\
523.01	1.73472347597681e-18\\
524.01	0\\
525.01	0\\
526.01	0\\
527.01	0\\
528.01	0\\
529.01	0\\
530.01	0\\
531.01	0\\
532.01	0\\
533.01	0\\
534.01	0\\
535.01	0\\
536.01	1.73472347597681e-18\\
537.01	1.73472347597681e-18\\
538.01	1.73472347597681e-18\\
539.01	0\\
540.01	0\\
541.01	0\\
542.01	0\\
543.01	0\\
544.01	0\\
545.01	1.73472347597681e-18\\
546.01	1.73472347597681e-18\\
547.01	0\\
548.01	0\\
549.01	0\\
550.01	0\\
551.01	0\\
552.01	0\\
553.01	0\\
554.01	0\\
555.01	0\\
556.01	1.73472347597681e-18\\
557.01	0\\
558.01	0\\
559.01	0\\
560.01	1.73472347597681e-18\\
561.01	1.73472347597681e-18\\
562.01	0\\
563.01	0\\
564.01	0\\
565.01	0\\
566.01	0\\
567.01	0\\
568.01	0\\
569.01	0\\
570.01	0\\
571.01	0\\
572.01	0\\
573.01	0\\
574.01	0\\
575.01	0\\
576.01	0\\
577.01	0\\
578.01	1.73472347597681e-18\\
579.01	0\\
580.01	0\\
581.01	0\\
582.01	0\\
583.01	1.73472347597681e-18\\
584.01	0\\
585.01	0\\
586.01	0\\
587.01	0\\
588.01	0\\
589.01	0\\
590.01	0\\
591.01	0\\
592.01	0\\
593.01	0\\
594.01	0\\
595.01	0\\
596.01	0\\
597.01	0\\
598.01	0\\
599.01	0\\
599.02	0\\
599.03	0\\
599.04	0\\
599.05	0\\
599.06	0\\
599.07	0\\
599.08	0\\
599.09	0\\
599.1	0\\
599.11	0\\
599.12	0\\
599.13	0\\
599.14	0\\
599.15	0\\
599.16	0\\
599.17	0\\
599.18	0\\
599.19	0\\
599.2	0\\
599.21	0\\
599.22	0\\
599.23	0\\
599.24	0\\
599.25	0\\
599.26	0\\
599.27	0\\
599.28	0\\
599.29	0\\
599.3	0\\
599.31	0\\
599.32	0\\
599.33	0\\
599.34	0\\
599.35	0\\
599.36	0\\
599.37	0\\
599.38	0\\
599.39	0\\
599.4	0\\
599.41	0\\
599.42	0\\
599.43	0\\
599.44	0\\
599.45	0\\
599.46	0\\
599.47	0\\
599.48	0\\
599.49	0\\
599.5	0\\
599.51	0\\
599.52	0\\
599.53	0\\
599.54	0\\
599.55	0\\
599.56	0\\
599.57	0\\
599.58	0\\
599.59	0\\
599.6	0\\
599.61	0\\
599.62	0\\
599.63	0\\
599.64	0\\
599.65	0\\
599.66	0\\
599.67	0\\
599.68	0\\
599.69	0\\
599.7	0\\
599.71	0\\
599.72	0\\
599.73	0\\
599.74	0\\
599.75	0\\
599.76	0\\
599.77	0\\
599.78	0\\
599.79	0\\
599.8	0\\
599.81	0\\
599.82	0\\
599.83	0\\
599.84	0\\
599.85	0\\
599.86	0\\
599.87	0\\
599.88	0\\
599.89	0\\
599.9	0\\
599.91	0\\
599.92	0\\
599.93	0\\
599.94	0\\
599.95	0\\
599.96	0\\
599.97	0\\
599.98	0\\
599.99	0\\
600	0\\
};
\addplot [color=mycolor10,solid,forget plot]
  table[row sep=crcr]{%
0.01	3.6332990175629e-05\\
1.01	3.6332990175629e-05\\
2.01	3.6332990175629e-05\\
3.01	3.6332990175629e-05\\
4.01	3.6332990175629e-05\\
5.01	3.6332990175629e-05\\
6.01	3.6332990175629e-05\\
7.01	3.6332990175629e-05\\
8.01	3.6332990175629e-05\\
9.01	3.6332990175629e-05\\
10.01	3.6332990175629e-05\\
11.01	3.6332990175629e-05\\
12.01	3.6332990175629e-05\\
13.01	3.6332990175629e-05\\
14.01	3.6332990175629e-05\\
15.01	3.6332990175629e-05\\
16.01	3.6332990175629e-05\\
17.01	3.6332990175629e-05\\
18.01	3.6332990175629e-05\\
19.01	3.6332990175629e-05\\
20.01	3.6332990175629e-05\\
21.01	3.6332990175629e-05\\
22.01	3.6332990175629e-05\\
23.01	3.6332990175629e-05\\
24.01	3.6332990175629e-05\\
25.01	3.6332990175629e-05\\
26.01	3.6332990175629e-05\\
27.01	3.6332990175629e-05\\
28.01	3.6332990175629e-05\\
29.01	3.6332990175629e-05\\
30.01	3.6332990175629e-05\\
31.01	3.6332990175629e-05\\
32.01	3.6332990175629e-05\\
33.01	3.6332990175629e-05\\
34.01	3.6332990175629e-05\\
35.01	3.6332990175629e-05\\
36.01	3.6332990175629e-05\\
37.01	3.6332990175629e-05\\
38.01	3.6332990175629e-05\\
39.01	3.6332990175629e-05\\
40.01	3.6332990175629e-05\\
41.01	3.6332990175629e-05\\
42.01	3.6332990175629e-05\\
43.01	3.6332990175629e-05\\
44.01	3.6332990175629e-05\\
45.01	3.6332990175629e-05\\
46.01	3.6332990175629e-05\\
47.01	3.6332990175629e-05\\
48.01	3.6332990175629e-05\\
49.01	3.6332990175629e-05\\
50.01	3.6332990175629e-05\\
51.01	3.6332990175629e-05\\
52.01	3.6332990175629e-05\\
53.01	3.6332990175629e-05\\
54.01	3.6332990175629e-05\\
55.01	3.6332990175629e-05\\
56.01	3.6332990175629e-05\\
57.01	3.6332990175629e-05\\
58.01	3.6332990175629e-05\\
59.01	3.6332990175629e-05\\
60.01	3.6332990175629e-05\\
61.01	3.6332990175629e-05\\
62.01	3.6332990175629e-05\\
63.01	3.6332990175629e-05\\
64.01	3.6332990175629e-05\\
65.01	3.6332990175629e-05\\
66.01	3.6332990175629e-05\\
67.01	3.6332990175629e-05\\
68.01	3.6332990175629e-05\\
69.01	3.6332990175629e-05\\
70.01	3.6332990175629e-05\\
71.01	3.6332990175629e-05\\
72.01	3.6332990175629e-05\\
73.01	3.6332990175629e-05\\
74.01	3.6332990175629e-05\\
75.01	3.6332990175629e-05\\
76.01	3.6332990175629e-05\\
77.01	3.6332990175629e-05\\
78.01	3.6332990175629e-05\\
79.01	3.6332990175629e-05\\
80.01	3.6332990175629e-05\\
81.01	3.6332990175629e-05\\
82.01	3.6332990175629e-05\\
83.01	3.6332990175629e-05\\
84.01	3.6332990175629e-05\\
85.01	3.6332990175629e-05\\
86.01	3.6332990175629e-05\\
87.01	3.6332990175629e-05\\
88.01	3.6332990175629e-05\\
89.01	3.6332990175629e-05\\
90.01	3.6332990175629e-05\\
91.01	3.6332990175629e-05\\
92.01	3.6332990175629e-05\\
93.01	3.6332990175629e-05\\
94.01	3.6332990175629e-05\\
95.01	3.6332990175629e-05\\
96.01	3.6332990175629e-05\\
97.01	3.6332990175629e-05\\
98.01	3.6332990175629e-05\\
99.01	3.6332990175629e-05\\
100.01	3.6332990175629e-05\\
101.01	3.6332990175629e-05\\
102.01	3.6332990175629e-05\\
103.01	3.6332990175629e-05\\
104.01	3.6332990175629e-05\\
105.01	3.6332990175629e-05\\
106.01	3.6332990175629e-05\\
107.01	3.6332990175629e-05\\
108.01	3.6332990175629e-05\\
109.01	3.6332990175629e-05\\
110.01	3.6332990175629e-05\\
111.01	3.6332990175629e-05\\
112.01	3.6332990175629e-05\\
113.01	3.6332990175629e-05\\
114.01	3.6332990175629e-05\\
115.01	3.6332990175629e-05\\
116.01	3.6332990175629e-05\\
117.01	3.6332990175629e-05\\
118.01	3.6332990175629e-05\\
119.01	3.6332990175629e-05\\
120.01	3.6332990175629e-05\\
121.01	3.6332990175629e-05\\
122.01	3.6332990175629e-05\\
123.01	3.6332990175629e-05\\
124.01	3.6332990175629e-05\\
125.01	3.6332990175629e-05\\
126.01	3.6332990175629e-05\\
127.01	3.6332990175629e-05\\
128.01	3.6332990175629e-05\\
129.01	3.6332990175629e-05\\
130.01	3.6332990175629e-05\\
131.01	3.6332990175629e-05\\
132.01	3.6332990175629e-05\\
133.01	3.6332990175629e-05\\
134.01	3.6332990175629e-05\\
135.01	3.6332990175629e-05\\
136.01	3.6332990175629e-05\\
137.01	3.6332990175629e-05\\
138.01	3.6332990175629e-05\\
139.01	3.6332990175629e-05\\
140.01	3.6332990175629e-05\\
141.01	3.6332990175629e-05\\
142.01	3.6332990175629e-05\\
143.01	3.6332990175629e-05\\
144.01	3.6332990175629e-05\\
145.01	3.6332990175629e-05\\
146.01	3.6332990175629e-05\\
147.01	3.6332990175629e-05\\
148.01	3.6332990175629e-05\\
149.01	3.6332990175629e-05\\
150.01	3.6332990175629e-05\\
151.01	3.6332990175629e-05\\
152.01	3.6332990175629e-05\\
153.01	3.6332990175629e-05\\
154.01	3.6332990175629e-05\\
155.01	3.6332990175629e-05\\
156.01	3.6332990175629e-05\\
157.01	3.6332990175629e-05\\
158.01	3.6332990175629e-05\\
159.01	3.6332990175629e-05\\
160.01	3.6332990175629e-05\\
161.01	3.6332990175629e-05\\
162.01	3.6332990175629e-05\\
163.01	3.6332990175629e-05\\
164.01	3.6332990175629e-05\\
165.01	3.6332990175629e-05\\
166.01	3.6332990175629e-05\\
167.01	3.6332990175629e-05\\
168.01	3.6332990175629e-05\\
169.01	3.6332990175629e-05\\
170.01	3.6332990175629e-05\\
171.01	3.6332990175629e-05\\
172.01	3.6332990175629e-05\\
173.01	3.6332990175629e-05\\
174.01	3.6332990175629e-05\\
175.01	3.6332990175629e-05\\
176.01	3.6332990175629e-05\\
177.01	3.6332990175629e-05\\
178.01	3.6332990175629e-05\\
179.01	3.6332990175629e-05\\
180.01	3.6332990175629e-05\\
181.01	3.6332990175629e-05\\
182.01	3.6332990175629e-05\\
183.01	3.6332990175629e-05\\
184.01	3.6332990175629e-05\\
185.01	3.6332990175629e-05\\
186.01	3.6332990175629e-05\\
187.01	3.6332990175629e-05\\
188.01	3.6332990175629e-05\\
189.01	3.6332990175629e-05\\
190.01	3.6332990175629e-05\\
191.01	3.6332990175629e-05\\
192.01	3.6332990175629e-05\\
193.01	3.6332990175629e-05\\
194.01	3.6332990175629e-05\\
195.01	3.6332990175629e-05\\
196.01	3.6332990175629e-05\\
197.01	3.6332990175629e-05\\
198.01	3.6332990175629e-05\\
199.01	3.6332990175629e-05\\
200.01	3.6332990175629e-05\\
201.01	3.6332990175629e-05\\
202.01	3.6332990175629e-05\\
203.01	3.6332990175629e-05\\
204.01	3.6332990175629e-05\\
205.01	3.6332990175629e-05\\
206.01	3.6332990175629e-05\\
207.01	3.6332990175629e-05\\
208.01	3.6332990175629e-05\\
209.01	3.6332990175629e-05\\
210.01	3.6332990175629e-05\\
211.01	3.6332990175629e-05\\
212.01	3.6332990175629e-05\\
213.01	3.6332990175629e-05\\
214.01	3.6332990175629e-05\\
215.01	3.6332990175629e-05\\
216.01	3.6332990175629e-05\\
217.01	3.6332990175629e-05\\
218.01	3.6332990175629e-05\\
219.01	3.6332990175629e-05\\
220.01	3.6332990175629e-05\\
221.01	3.6332990175629e-05\\
222.01	3.6332990175629e-05\\
223.01	3.6332990175629e-05\\
224.01	3.6332990175629e-05\\
225.01	3.6332990175629e-05\\
226.01	3.6332990175629e-05\\
227.01	3.6332990175629e-05\\
228.01	3.6332990175629e-05\\
229.01	3.6332990175629e-05\\
230.01	3.6332990175629e-05\\
231.01	3.6332990175629e-05\\
232.01	3.6332990175629e-05\\
233.01	3.6332990175629e-05\\
234.01	3.6332990175629e-05\\
235.01	3.6332990175629e-05\\
236.01	3.6332990175629e-05\\
237.01	3.6332990175629e-05\\
238.01	3.6332990175629e-05\\
239.01	3.6332990175629e-05\\
240.01	3.6332990175629e-05\\
241.01	3.6332990175629e-05\\
242.01	3.6332990175629e-05\\
243.01	3.6332990175629e-05\\
244.01	3.6332990175629e-05\\
245.01	3.6332990175629e-05\\
246.01	3.6332990175629e-05\\
247.01	3.6332990175629e-05\\
248.01	3.6332990175629e-05\\
249.01	3.6332990175629e-05\\
250.01	3.6332990175629e-05\\
251.01	3.6332990175629e-05\\
252.01	3.6332990175629e-05\\
253.01	3.6332990175629e-05\\
254.01	3.6332990175629e-05\\
255.01	3.6332990175629e-05\\
256.01	3.6332990175629e-05\\
257.01	3.6332990175629e-05\\
258.01	3.6332990175629e-05\\
259.01	3.6332990175629e-05\\
260.01	3.6332990175629e-05\\
261.01	3.6332990175629e-05\\
262.01	3.6332990175629e-05\\
263.01	3.6332990175629e-05\\
264.01	3.6332990175629e-05\\
265.01	3.6332990175629e-05\\
266.01	3.6332990175629e-05\\
267.01	3.6332990175629e-05\\
268.01	3.6332990175629e-05\\
269.01	3.6332990175629e-05\\
270.01	3.6332990175629e-05\\
271.01	3.6332990175629e-05\\
272.01	3.6332990175629e-05\\
273.01	3.6332990175629e-05\\
274.01	3.6332990175629e-05\\
275.01	3.6332990175629e-05\\
276.01	3.6332990175629e-05\\
277.01	3.6332990175629e-05\\
278.01	3.6332990175629e-05\\
279.01	3.6332990175629e-05\\
280.01	3.6332990175629e-05\\
281.01	3.6332990175629e-05\\
282.01	3.6332990175629e-05\\
283.01	3.6332990175629e-05\\
284.01	3.6332990175629e-05\\
285.01	3.6332990175629e-05\\
286.01	3.6332990175629e-05\\
287.01	3.6332990175629e-05\\
288.01	3.6332990175629e-05\\
289.01	3.6332990175629e-05\\
290.01	3.6332990175629e-05\\
291.01	3.6332990175629e-05\\
292.01	3.6332990175629e-05\\
293.01	3.6332990175629e-05\\
294.01	3.6332990175629e-05\\
295.01	3.6332990175629e-05\\
296.01	3.6332990175629e-05\\
297.01	3.6332990175629e-05\\
298.01	3.6332990175629e-05\\
299.01	3.6332990175629e-05\\
300.01	3.6332990175629e-05\\
301.01	3.6332990175629e-05\\
302.01	3.6332990175629e-05\\
303.01	3.6332990175629e-05\\
304.01	3.6332990175629e-05\\
305.01	3.6332990175629e-05\\
306.01	3.6332990175629e-05\\
307.01	3.6332990175629e-05\\
308.01	3.6332990175629e-05\\
309.01	3.6332990175629e-05\\
310.01	3.6332990175629e-05\\
311.01	3.6332990175629e-05\\
312.01	3.6332990175629e-05\\
313.01	3.6332990175629e-05\\
314.01	3.6332990175629e-05\\
315.01	3.6332990175629e-05\\
316.01	3.6332990175629e-05\\
317.01	3.6332990175629e-05\\
318.01	3.6332990175629e-05\\
319.01	3.6332990175629e-05\\
320.01	3.6332990175629e-05\\
321.01	3.6332990175629e-05\\
322.01	3.6332990175629e-05\\
323.01	3.6332990175629e-05\\
324.01	3.6332990175629e-05\\
325.01	3.6332990175629e-05\\
326.01	3.6332990175629e-05\\
327.01	3.6332990175629e-05\\
328.01	3.6332990175629e-05\\
329.01	3.6332990175629e-05\\
330.01	3.6332990175629e-05\\
331.01	3.6332990175629e-05\\
332.01	3.6332990175629e-05\\
333.01	3.6332990175629e-05\\
334.01	3.6332990175629e-05\\
335.01	3.6332990175629e-05\\
336.01	3.6332990175629e-05\\
337.01	3.6332990175629e-05\\
338.01	3.6332990175629e-05\\
339.01	3.6332990175629e-05\\
340.01	3.6332990175629e-05\\
341.01	3.6332990175629e-05\\
342.01	3.6332990175629e-05\\
343.01	3.6332990175629e-05\\
344.01	3.6332990175629e-05\\
345.01	3.6332990175629e-05\\
346.01	3.6332990175629e-05\\
347.01	3.6332990175629e-05\\
348.01	3.6332990175629e-05\\
349.01	3.6332990175629e-05\\
350.01	3.6332990175629e-05\\
351.01	3.6332990175629e-05\\
352.01	3.6332990175629e-05\\
353.01	3.6332990175629e-05\\
354.01	3.6332990175629e-05\\
355.01	3.6332990175629e-05\\
356.01	3.6332990175629e-05\\
357.01	3.6332990175629e-05\\
358.01	3.6332990175629e-05\\
359.01	3.6332990175629e-05\\
360.01	3.6332990175629e-05\\
361.01	3.6332990175629e-05\\
362.01	3.6332990175629e-05\\
363.01	3.6332990175629e-05\\
364.01	3.6332990175629e-05\\
365.01	3.6332990175629e-05\\
366.01	3.6332990175629e-05\\
367.01	3.6332990175629e-05\\
368.01	3.6332990175629e-05\\
369.01	3.6332990175629e-05\\
370.01	3.6332990175629e-05\\
371.01	3.6332990175629e-05\\
372.01	3.6332990175629e-05\\
373.01	3.6332990175629e-05\\
374.01	3.6332990175629e-05\\
375.01	3.6332990175629e-05\\
376.01	3.6332990175629e-05\\
377.01	3.6332990175629e-05\\
378.01	3.6332990175629e-05\\
379.01	3.6332990175629e-05\\
380.01	3.6332990175629e-05\\
381.01	3.6332990175629e-05\\
382.01	3.6332990175629e-05\\
383.01	3.6332990175629e-05\\
384.01	3.6332990175629e-05\\
385.01	3.6332990175629e-05\\
386.01	3.6332990175629e-05\\
387.01	3.6332990175629e-05\\
388.01	3.6332990175629e-05\\
389.01	3.6332990175629e-05\\
390.01	3.6332990175629e-05\\
391.01	3.6332990175629e-05\\
392.01	3.6332990175629e-05\\
393.01	3.6332990175629e-05\\
394.01	3.6332990175629e-05\\
395.01	3.6332990175629e-05\\
396.01	3.6332990175629e-05\\
397.01	3.6332990175629e-05\\
398.01	3.6332990175629e-05\\
399.01	3.6332990175629e-05\\
400.01	3.6332990175629e-05\\
401.01	3.6332990175629e-05\\
402.01	3.6332990175629e-05\\
403.01	3.6332990175629e-05\\
404.01	3.6332990175629e-05\\
405.01	3.6332990175629e-05\\
406.01	3.6332990175629e-05\\
407.01	3.6332990175629e-05\\
408.01	3.6332990175629e-05\\
409.01	3.6332990175629e-05\\
410.01	3.6332990175629e-05\\
411.01	3.6332990175629e-05\\
412.01	3.6332990175629e-05\\
413.01	3.6332990175629e-05\\
414.01	3.6332990175629e-05\\
415.01	3.6332990175629e-05\\
416.01	3.6332990175629e-05\\
417.01	3.6332990175629e-05\\
418.01	3.6332990175629e-05\\
419.01	3.6332990175629e-05\\
420.01	3.6332990175629e-05\\
421.01	3.6332990175629e-05\\
422.01	3.6332990175629e-05\\
423.01	3.6332990175629e-05\\
424.01	3.6332990175629e-05\\
425.01	3.6332990175629e-05\\
426.01	3.6332990175629e-05\\
427.01	3.6332990175629e-05\\
428.01	3.6332990175629e-05\\
429.01	3.6332990175629e-05\\
430.01	3.6332990175629e-05\\
431.01	3.6332990175629e-05\\
432.01	3.6332990175629e-05\\
433.01	3.6332990175629e-05\\
434.01	3.6332990175629e-05\\
435.01	3.6332990175629e-05\\
436.01	3.6332990175629e-05\\
437.01	3.6332990175629e-05\\
438.01	3.6332990175629e-05\\
439.01	3.6332990175629e-05\\
440.01	3.6332990175629e-05\\
441.01	3.6332990175629e-05\\
442.01	3.6332990175629e-05\\
443.01	3.6332990175629e-05\\
444.01	3.6332990175629e-05\\
445.01	3.63329901756273e-05\\
446.01	3.63329901756099e-05\\
447.01	3.633299017557e-05\\
448.01	3.63329901754642e-05\\
449.01	3.63329901751815e-05\\
450.01	3.63329901744112e-05\\
451.01	3.63329901723244e-05\\
452.01	3.63329901666726e-05\\
453.01	3.63329901513759e-05\\
454.01	3.63329901100322e-05\\
455.01	3.63329899984756e-05\\
456.01	3.63329896982331e-05\\
457.01	3.63329888930751e-05\\
458.01	3.63329867448508e-05\\
459.01	3.63329810541529e-05\\
460.01	3.63329661298185e-05\\
461.01	3.63329275367384e-05\\
462.01	3.63328297068363e-05\\
463.01	3.63325887119854e-05\\
464.01	3.63320195061381e-05\\
465.01	3.63307585765377e-05\\
466.01	3.63282378046548e-05\\
467.01	3.63240050212116e-05\\
468.01	3.63186526850575e-05\\
469.01	3.63131005853765e-05\\
470.01	3.63074235777052e-05\\
471.01	3.6301634700522e-05\\
472.01	3.62957697946677e-05\\
473.01	3.62899174158309e-05\\
474.01	3.62842786465652e-05\\
475.01	3.62792618784288e-05\\
476.01	3.6275545979637e-05\\
477.01	3.62738038318083e-05\\
478.01	3.62736023077313e-05\\
479.01	3.62736023077313e-05\\
480.01	3.62736023077313e-05\\
481.01	3.62736023077313e-05\\
482.01	3.62736023077313e-05\\
483.01	3.62736023077313e-05\\
484.01	3.62736023077313e-05\\
485.01	3.62736023077313e-05\\
486.01	3.62736023077313e-05\\
487.01	3.62736023077313e-05\\
488.01	3.62736023077313e-05\\
489.01	3.62736023077313e-05\\
490.01	3.62736023077313e-05\\
491.01	3.62736023077313e-05\\
492.01	3.62736023077313e-05\\
493.01	3.62736023077313e-05\\
494.01	3.62736023077313e-05\\
495.01	3.62736023077313e-05\\
496.01	3.62736023077313e-05\\
497.01	3.62736023077313e-05\\
498.01	3.62736023077313e-05\\
499.01	3.62736023077313e-05\\
500.01	3.62736023077313e-05\\
501.01	3.62736023077313e-05\\
502.01	3.62736023077313e-05\\
503.01	3.62736023077313e-05\\
504.01	3.62736023077313e-05\\
505.01	3.62736023077313e-05\\
506.01	3.62736023077313e-05\\
507.01	3.62736023077331e-05\\
508.01	3.62736023077331e-05\\
509.01	3.6273602307714e-05\\
510.01	3.6273602307688e-05\\
511.01	3.62736023076082e-05\\
512.01	3.62736023073896e-05\\
513.01	3.62736023067703e-05\\
514.01	3.62736023050304e-05\\
515.01	3.62736023001211e-05\\
516.01	3.62736022862329e-05\\
517.01	3.62736022467593e-05\\
518.01	3.62736021340768e-05\\
519.01	3.62736018107348e-05\\
520.01	3.62736008775299e-05\\
521.01	3.62735981668961e-05\\
522.01	3.62735902382626e-05\\
523.01	3.62735668706579e-05\\
524.01	3.62734974387602e-05\\
525.01	3.6273289345785e-05\\
526.01	3.62726599625867e-05\\
527.01	3.62707381364666e-05\\
528.01	3.62648115461908e-05\\
529.01	3.62463488691057e-05\\
530.01	3.61882420091067e-05\\
531.01	3.6003504478764e-05\\
532.01	3.54104452939367e-05\\
533.01	3.34895990750511e-05\\
534.01	2.86780701776249e-05\\
535.01	2.33349897528021e-05\\
536.01	1.7891461666756e-05\\
537.01	1.23300729334211e-05\\
538.01	6.77559521603671e-06\\
539.01	1.90599427558891e-06\\
540.01	0\\
541.01	0\\
542.01	0\\
543.01	0\\
544.01	0\\
545.01	1.73472347597681e-18\\
546.01	1.73472347597681e-18\\
547.01	0\\
548.01	0\\
549.01	0\\
550.01	0\\
551.01	0\\
552.01	0\\
553.01	0\\
554.01	0\\
555.01	0\\
556.01	1.73472347597681e-18\\
557.01	0\\
558.01	0\\
559.01	0\\
560.01	1.73472347597681e-18\\
561.01	1.73472347597681e-18\\
562.01	0\\
563.01	0\\
564.01	0\\
565.01	0\\
566.01	0\\
567.01	0\\
568.01	0\\
569.01	0\\
570.01	0\\
571.01	0\\
572.01	0\\
573.01	0\\
574.01	0\\
575.01	0\\
576.01	0\\
577.01	0\\
578.01	1.73472347597681e-18\\
579.01	0\\
580.01	0\\
581.01	0\\
582.01	0\\
583.01	1.73472347597681e-18\\
584.01	0\\
585.01	0\\
586.01	0\\
587.01	0\\
588.01	0\\
589.01	0\\
590.01	0\\
591.01	0\\
592.01	0\\
593.01	0\\
594.01	0\\
595.01	0\\
596.01	0\\
597.01	0\\
598.01	0\\
599.01	0\\
599.02	0\\
599.03	0\\
599.04	0\\
599.05	0\\
599.06	0\\
599.07	0\\
599.08	0\\
599.09	0\\
599.1	0\\
599.11	0\\
599.12	0\\
599.13	0\\
599.14	0\\
599.15	0\\
599.16	0\\
599.17	0\\
599.18	0\\
599.19	0\\
599.2	0\\
599.21	0\\
599.22	0\\
599.23	0\\
599.24	0\\
599.25	0\\
599.26	0\\
599.27	0\\
599.28	0\\
599.29	0\\
599.3	0\\
599.31	0\\
599.32	0\\
599.33	0\\
599.34	0\\
599.35	0\\
599.36	0\\
599.37	0\\
599.38	0\\
599.39	0\\
599.4	0\\
599.41	0\\
599.42	0\\
599.43	0\\
599.44	0\\
599.45	0\\
599.46	0\\
599.47	0\\
599.48	0\\
599.49	0\\
599.5	0\\
599.51	0\\
599.52	0\\
599.53	0\\
599.54	0\\
599.55	0\\
599.56	0\\
599.57	0\\
599.58	0\\
599.59	0\\
599.6	0\\
599.61	0\\
599.62	0\\
599.63	0\\
599.64	0\\
599.65	0\\
599.66	0\\
599.67	0\\
599.68	0\\
599.69	0\\
599.7	0\\
599.71	0\\
599.72	0\\
599.73	0\\
599.74	0\\
599.75	0\\
599.76	0\\
599.77	0\\
599.78	0\\
599.79	0\\
599.8	0\\
599.81	0\\
599.82	0\\
599.83	0\\
599.84	0\\
599.85	0\\
599.86	0\\
599.87	0\\
599.88	0\\
599.89	0\\
599.9	0\\
599.91	0\\
599.92	0\\
599.93	0\\
599.94	0\\
599.95	0\\
599.96	0\\
599.97	0\\
599.98	0\\
599.99	0\\
600	0\\
};
\addplot [color=mycolor11,solid,forget plot]
  table[row sep=crcr]{%
0.01	0.00032348326751387\\
1.01	0.00032348326751387\\
2.01	0.00032348326751387\\
3.01	0.00032348326751387\\
4.01	0.00032348326751387\\
5.01	0.00032348326751387\\
6.01	0.00032348326751387\\
7.01	0.00032348326751387\\
8.01	0.00032348326751387\\
9.01	0.00032348326751387\\
10.01	0.00032348326751387\\
11.01	0.00032348326751387\\
12.01	0.00032348326751387\\
13.01	0.00032348326751387\\
14.01	0.00032348326751387\\
15.01	0.00032348326751387\\
16.01	0.00032348326751387\\
17.01	0.00032348326751387\\
18.01	0.00032348326751387\\
19.01	0.00032348326751387\\
20.01	0.00032348326751387\\
21.01	0.00032348326751387\\
22.01	0.00032348326751387\\
23.01	0.00032348326751387\\
24.01	0.00032348326751387\\
25.01	0.00032348326751387\\
26.01	0.00032348326751387\\
27.01	0.00032348326751387\\
28.01	0.00032348326751387\\
29.01	0.00032348326751387\\
30.01	0.00032348326751387\\
31.01	0.00032348326751387\\
32.01	0.00032348326751387\\
33.01	0.00032348326751387\\
34.01	0.00032348326751387\\
35.01	0.00032348326751387\\
36.01	0.00032348326751387\\
37.01	0.00032348326751387\\
38.01	0.00032348326751387\\
39.01	0.00032348326751387\\
40.01	0.00032348326751387\\
41.01	0.00032348326751387\\
42.01	0.00032348326751387\\
43.01	0.00032348326751387\\
44.01	0.00032348326751387\\
45.01	0.00032348326751387\\
46.01	0.00032348326751387\\
47.01	0.00032348326751387\\
48.01	0.00032348326751387\\
49.01	0.00032348326751387\\
50.01	0.00032348326751387\\
51.01	0.00032348326751387\\
52.01	0.00032348326751387\\
53.01	0.00032348326751387\\
54.01	0.00032348326751387\\
55.01	0.00032348326751387\\
56.01	0.00032348326751387\\
57.01	0.00032348326751387\\
58.01	0.00032348326751387\\
59.01	0.00032348326751387\\
60.01	0.00032348326751387\\
61.01	0.00032348326751387\\
62.01	0.00032348326751387\\
63.01	0.00032348326751387\\
64.01	0.00032348326751387\\
65.01	0.00032348326751387\\
66.01	0.00032348326751387\\
67.01	0.00032348326751387\\
68.01	0.00032348326751387\\
69.01	0.00032348326751387\\
70.01	0.00032348326751387\\
71.01	0.00032348326751387\\
72.01	0.00032348326751387\\
73.01	0.00032348326751387\\
74.01	0.00032348326751387\\
75.01	0.00032348326751387\\
76.01	0.00032348326751387\\
77.01	0.00032348326751387\\
78.01	0.00032348326751387\\
79.01	0.00032348326751387\\
80.01	0.00032348326751387\\
81.01	0.00032348326751387\\
82.01	0.00032348326751387\\
83.01	0.00032348326751387\\
84.01	0.00032348326751387\\
85.01	0.00032348326751387\\
86.01	0.00032348326751387\\
87.01	0.00032348326751387\\
88.01	0.00032348326751387\\
89.01	0.00032348326751387\\
90.01	0.00032348326751387\\
91.01	0.00032348326751387\\
92.01	0.00032348326751387\\
93.01	0.00032348326751387\\
94.01	0.00032348326751387\\
95.01	0.00032348326751387\\
96.01	0.00032348326751387\\
97.01	0.00032348326751387\\
98.01	0.00032348326751387\\
99.01	0.00032348326751387\\
100.01	0.00032348326751387\\
101.01	0.00032348326751387\\
102.01	0.00032348326751387\\
103.01	0.00032348326751387\\
104.01	0.00032348326751387\\
105.01	0.00032348326751387\\
106.01	0.00032348326751387\\
107.01	0.00032348326751387\\
108.01	0.00032348326751387\\
109.01	0.00032348326751387\\
110.01	0.00032348326751387\\
111.01	0.00032348326751387\\
112.01	0.00032348326751387\\
113.01	0.00032348326751387\\
114.01	0.00032348326751387\\
115.01	0.00032348326751387\\
116.01	0.00032348326751387\\
117.01	0.00032348326751387\\
118.01	0.00032348326751387\\
119.01	0.00032348326751387\\
120.01	0.00032348326751387\\
121.01	0.00032348326751387\\
122.01	0.00032348326751387\\
123.01	0.00032348326751387\\
124.01	0.00032348326751387\\
125.01	0.00032348326751387\\
126.01	0.00032348326751387\\
127.01	0.00032348326751387\\
128.01	0.00032348326751387\\
129.01	0.00032348326751387\\
130.01	0.00032348326751387\\
131.01	0.00032348326751387\\
132.01	0.00032348326751387\\
133.01	0.00032348326751387\\
134.01	0.00032348326751387\\
135.01	0.00032348326751387\\
136.01	0.00032348326751387\\
137.01	0.00032348326751387\\
138.01	0.00032348326751387\\
139.01	0.00032348326751387\\
140.01	0.00032348326751387\\
141.01	0.00032348326751387\\
142.01	0.00032348326751387\\
143.01	0.00032348326751387\\
144.01	0.00032348326751387\\
145.01	0.00032348326751387\\
146.01	0.00032348326751387\\
147.01	0.00032348326751387\\
148.01	0.00032348326751387\\
149.01	0.00032348326751387\\
150.01	0.00032348326751387\\
151.01	0.00032348326751387\\
152.01	0.00032348326751387\\
153.01	0.00032348326751387\\
154.01	0.00032348326751387\\
155.01	0.00032348326751387\\
156.01	0.00032348326751387\\
157.01	0.00032348326751387\\
158.01	0.00032348326751387\\
159.01	0.00032348326751387\\
160.01	0.00032348326751387\\
161.01	0.00032348326751387\\
162.01	0.00032348326751387\\
163.01	0.00032348326751387\\
164.01	0.00032348326751387\\
165.01	0.00032348326751387\\
166.01	0.00032348326751387\\
167.01	0.00032348326751387\\
168.01	0.00032348326751387\\
169.01	0.00032348326751387\\
170.01	0.00032348326751387\\
171.01	0.00032348326751387\\
172.01	0.00032348326751387\\
173.01	0.00032348326751387\\
174.01	0.00032348326751387\\
175.01	0.00032348326751387\\
176.01	0.00032348326751387\\
177.01	0.00032348326751387\\
178.01	0.00032348326751387\\
179.01	0.00032348326751387\\
180.01	0.00032348326751387\\
181.01	0.00032348326751387\\
182.01	0.00032348326751387\\
183.01	0.00032348326751387\\
184.01	0.00032348326751387\\
185.01	0.00032348326751387\\
186.01	0.00032348326751387\\
187.01	0.00032348326751387\\
188.01	0.00032348326751387\\
189.01	0.00032348326751387\\
190.01	0.00032348326751387\\
191.01	0.00032348326751387\\
192.01	0.00032348326751387\\
193.01	0.00032348326751387\\
194.01	0.00032348326751387\\
195.01	0.00032348326751387\\
196.01	0.00032348326751387\\
197.01	0.00032348326751387\\
198.01	0.00032348326751387\\
199.01	0.00032348326751387\\
200.01	0.00032348326751387\\
201.01	0.00032348326751387\\
202.01	0.00032348326751387\\
203.01	0.00032348326751387\\
204.01	0.00032348326751387\\
205.01	0.00032348326751387\\
206.01	0.00032348326751387\\
207.01	0.00032348326751387\\
208.01	0.00032348326751387\\
209.01	0.00032348326751387\\
210.01	0.00032348326751387\\
211.01	0.00032348326751387\\
212.01	0.00032348326751387\\
213.01	0.00032348326751387\\
214.01	0.00032348326751387\\
215.01	0.00032348326751387\\
216.01	0.00032348326751387\\
217.01	0.00032348326751387\\
218.01	0.00032348326751387\\
219.01	0.00032348326751387\\
220.01	0.00032348326751387\\
221.01	0.00032348326751387\\
222.01	0.00032348326751387\\
223.01	0.00032348326751387\\
224.01	0.00032348326751387\\
225.01	0.00032348326751387\\
226.01	0.00032348326751387\\
227.01	0.00032348326751387\\
228.01	0.00032348326751387\\
229.01	0.00032348326751387\\
230.01	0.00032348326751387\\
231.01	0.00032348326751387\\
232.01	0.00032348326751387\\
233.01	0.00032348326751387\\
234.01	0.00032348326751387\\
235.01	0.00032348326751387\\
236.01	0.00032348326751387\\
237.01	0.00032348326751387\\
238.01	0.00032348326751387\\
239.01	0.00032348326751387\\
240.01	0.00032348326751387\\
241.01	0.00032348326751387\\
242.01	0.00032348326751387\\
243.01	0.00032348326751387\\
244.01	0.00032348326751387\\
245.01	0.00032348326751387\\
246.01	0.00032348326751387\\
247.01	0.00032348326751387\\
248.01	0.00032348326751387\\
249.01	0.00032348326751387\\
250.01	0.00032348326751387\\
251.01	0.00032348326751387\\
252.01	0.00032348326751387\\
253.01	0.00032348326751387\\
254.01	0.00032348326751387\\
255.01	0.00032348326751387\\
256.01	0.00032348326751387\\
257.01	0.00032348326751387\\
258.01	0.00032348326751387\\
259.01	0.00032348326751387\\
260.01	0.00032348326751387\\
261.01	0.00032348326751387\\
262.01	0.00032348326751387\\
263.01	0.00032348326751387\\
264.01	0.00032348326751387\\
265.01	0.00032348326751387\\
266.01	0.00032348326751387\\
267.01	0.00032348326751387\\
268.01	0.00032348326751387\\
269.01	0.00032348326751387\\
270.01	0.00032348326751387\\
271.01	0.00032348326751387\\
272.01	0.00032348326751387\\
273.01	0.00032348326751387\\
274.01	0.00032348326751387\\
275.01	0.00032348326751387\\
276.01	0.00032348326751387\\
277.01	0.00032348326751387\\
278.01	0.00032348326751387\\
279.01	0.00032348326751387\\
280.01	0.00032348326751387\\
281.01	0.00032348326751387\\
282.01	0.00032348326751387\\
283.01	0.00032348326751387\\
284.01	0.00032348326751387\\
285.01	0.00032348326751387\\
286.01	0.00032348326751387\\
287.01	0.00032348326751387\\
288.01	0.00032348326751387\\
289.01	0.00032348326751387\\
290.01	0.00032348326751387\\
291.01	0.00032348326751387\\
292.01	0.00032348326751387\\
293.01	0.00032348326751387\\
294.01	0.00032348326751387\\
295.01	0.00032348326751387\\
296.01	0.00032348326751387\\
297.01	0.00032348326751387\\
298.01	0.00032348326751387\\
299.01	0.00032348326751387\\
300.01	0.00032348326751387\\
301.01	0.00032348326751387\\
302.01	0.00032348326751387\\
303.01	0.00032348326751387\\
304.01	0.00032348326751387\\
305.01	0.00032348326751387\\
306.01	0.00032348326751387\\
307.01	0.00032348326751387\\
308.01	0.00032348326751387\\
309.01	0.00032348326751387\\
310.01	0.00032348326751387\\
311.01	0.00032348326751387\\
312.01	0.00032348326751387\\
313.01	0.00032348326751387\\
314.01	0.00032348326751387\\
315.01	0.00032348326751387\\
316.01	0.00032348326751387\\
317.01	0.00032348326751387\\
318.01	0.00032348326751387\\
319.01	0.00032348326751387\\
320.01	0.00032348326751387\\
321.01	0.00032348326751387\\
322.01	0.00032348326751387\\
323.01	0.00032348326751387\\
324.01	0.00032348326751387\\
325.01	0.00032348326751387\\
326.01	0.00032348326751387\\
327.01	0.00032348326751387\\
328.01	0.00032348326751387\\
329.01	0.00032348326751387\\
330.01	0.00032348326751387\\
331.01	0.00032348326751387\\
332.01	0.00032348326751387\\
333.01	0.00032348326751387\\
334.01	0.00032348326751387\\
335.01	0.00032348326751387\\
336.01	0.00032348326751387\\
337.01	0.00032348326751387\\
338.01	0.00032348326751387\\
339.01	0.00032348326751387\\
340.01	0.00032348326751387\\
341.01	0.00032348326751387\\
342.01	0.00032348326751387\\
343.01	0.00032348326751387\\
344.01	0.00032348326751387\\
345.01	0.00032348326751387\\
346.01	0.00032348326751387\\
347.01	0.00032348326751387\\
348.01	0.00032348326751387\\
349.01	0.00032348326751387\\
350.01	0.00032348326751387\\
351.01	0.00032348326751387\\
352.01	0.00032348326751387\\
353.01	0.00032348326751387\\
354.01	0.00032348326751387\\
355.01	0.00032348326751387\\
356.01	0.00032348326751387\\
357.01	0.00032348326751387\\
358.01	0.00032348326751387\\
359.01	0.00032348326751387\\
360.01	0.00032348326751387\\
361.01	0.00032348326751387\\
362.01	0.00032348326751387\\
363.01	0.00032348326751387\\
364.01	0.00032348326751387\\
365.01	0.00032348326751387\\
366.01	0.00032348326751387\\
367.01	0.00032348326751387\\
368.01	0.00032348326751387\\
369.01	0.00032348326751387\\
370.01	0.00032348326751387\\
371.01	0.00032348326751387\\
372.01	0.00032348326751387\\
373.01	0.00032348326751387\\
374.01	0.00032348326751387\\
375.01	0.00032348326751387\\
376.01	0.00032348326751387\\
377.01	0.00032348326751387\\
378.01	0.00032348326751387\\
379.01	0.00032348326751387\\
380.01	0.00032348326751387\\
381.01	0.00032348326751387\\
382.01	0.00032348326751387\\
383.01	0.00032348326751387\\
384.01	0.00032348326751387\\
385.01	0.00032348326751387\\
386.01	0.00032348326751387\\
387.01	0.00032348326751387\\
388.01	0.00032348326751387\\
389.01	0.00032348326751387\\
390.01	0.00032348326751387\\
391.01	0.00032348326751387\\
392.01	0.00032348326751387\\
393.01	0.00032348326751387\\
394.01	0.00032348326751387\\
395.01	0.00032348326751387\\
396.01	0.00032348326751387\\
397.01	0.00032348326751387\\
398.01	0.00032348326751387\\
399.01	0.00032348326751387\\
400.01	0.00032348326751387\\
401.01	0.00032348326751387\\
402.01	0.00032348326751387\\
403.01	0.00032348326751387\\
404.01	0.00032348326751387\\
405.01	0.00032348326751387\\
406.01	0.00032348326751387\\
407.01	0.00032348326751387\\
408.01	0.00032348326751387\\
409.01	0.00032348326751387\\
410.01	0.00032348326751387\\
411.01	0.00032348326751387\\
412.01	0.00032348326751387\\
413.01	0.00032348326751387\\
414.01	0.00032348326751387\\
415.01	0.00032348326751387\\
416.01	0.00032348326751387\\
417.01	0.00032348326751387\\
418.01	0.00032348326751387\\
419.01	0.00032348326751387\\
420.01	0.00032348326751387\\
421.01	0.00032348326751387\\
422.01	0.00032348326751387\\
423.01	0.00032348326751387\\
424.01	0.00032348326751387\\
425.01	0.00032348326751387\\
426.01	0.00032348326751387\\
427.01	0.00032348326751387\\
428.01	0.00032348326751387\\
429.01	0.00032348326751387\\
430.01	0.00032348326751387\\
431.01	0.00032348326751387\\
432.01	0.00032348326751387\\
433.01	0.00032348326751387\\
434.01	0.00032348326751387\\
435.01	0.00032348326751387\\
436.01	0.00032348326751387\\
437.01	0.00032348326751387\\
438.01	0.00032348326751387\\
439.01	0.00032348326751387\\
440.01	0.00032348326751387\\
441.01	0.00032348326751387\\
442.01	0.00032348326751387\\
443.01	0.00032348326751387\\
444.01	0.000323483267513872\\
445.01	0.000323483267513872\\
446.01	0.000323483267513854\\
447.01	0.000323483267513818\\
448.01	0.000323483267513719\\
449.01	0.00032348326751346\\
450.01	0.000323483267512777\\
451.01	0.00032348326751098\\
452.01	0.000323483267506265\\
453.01	0.000323483267493957\\
454.01	0.00032348326746201\\
455.01	0.000323483267379661\\
456.01	0.000323483267169111\\
457.01	0.000323483266636138\\
458.01	0.000323483265303676\\
459.01	0.000323483262024351\\
460.01	0.000323483254114245\\
461.01	0.000323483235527324\\
462.01	0.000323483193342078\\
463.01	0.000323483101985822\\
464.01	0.000323482916574127\\
465.01	0.000323482573337544\\
466.01	0.000323482016935115\\
467.01	0.000323481268148873\\
468.01	0.000323480441666813\\
469.01	0.000323479595467246\\
470.01	0.000323478733894969\\
471.01	0.000323477863201176\\
472.01	0.000323476997294479\\
473.01	0.000323476164510996\\
474.01	0.000323475416246714\\
475.01	0.000323474829585597\\
476.01	0.000323474481653553\\
477.01	0.000323474371361708\\
478.01	0.000323474365043552\\
479.01	0.000323474365043552\\
480.01	0.000323474365043552\\
481.01	0.000323474365043552\\
482.01	0.000323474365043552\\
483.01	0.000323474365043552\\
484.01	0.000323474365043552\\
485.01	0.000323474365043552\\
486.01	0.000323474365043552\\
487.01	0.000323474365043552\\
488.01	0.000323474365043552\\
489.01	0.000323474365043552\\
490.01	0.000323474365043552\\
491.01	0.000323474365043552\\
492.01	0.000323474365043552\\
493.01	0.000323474365043552\\
494.01	0.000323474365043552\\
495.01	0.000323474365043552\\
496.01	0.000323474365043552\\
497.01	0.000323474365043552\\
498.01	0.000323474365043552\\
499.01	0.000323474365043552\\
500.01	0.000323474365043552\\
501.01	0.000323474365043552\\
502.01	0.000323474365043552\\
503.01	0.000323474365043552\\
504.01	0.000323474365043552\\
505.01	0.000323474365043552\\
506.01	0.000323474365043552\\
507.01	0.000323474365043552\\
508.01	0.000323474365043552\\
509.01	0.000323474365043541\\
510.01	0.000323474365043508\\
511.01	0.000323474365043427\\
512.01	0.000323474365043201\\
513.01	0.000323474365042582\\
514.01	0.000323474365040866\\
515.01	0.000323474365036124\\
516.01	0.000323474365023004\\
517.01	0.000323474364986712\\
518.01	0.000323474364886271\\
519.01	0.000323474364608155\\
520.01	0.000323474363837714\\
521.01	0.000323474361702531\\
522.01	0.000323474355783794\\
523.01	0.00032347433937965\\
524.01	0.000323474293952497\\
525.01	0.000323474168395149\\
526.01	0.000323473822617017\\
527.01	0.000323472876248249\\
528.01	0.000323470312193687\\
529.01	0.000323463476687595\\
530.01	0.0003234457191502\\
531.01	0.000323401507088766\\
532.01	0.000323299349623227\\
533.01	0.000323096583750704\\
534.01	0.000322820494104223\\
535.01	0.000322544563133137\\
536.01	0.000322274405627094\\
537.01	0.000322012012008299\\
538.01	0.000321781721082232\\
539.01	0.000321635878047019\\
540.01	0.000321609567441481\\
541.01	0.000321609567440196\\
542.01	0.00032160956743655\\
543.01	0.000321609567426226\\
544.01	0.000321609567396994\\
545.01	0.000321609567314168\\
546.01	0.000321609567079316\\
547.01	0.00032160956641262\\
548.01	0.000321609564516894\\
549.01	0.000321609559114573\\
550.01	0.000321609543674653\\
551.01	0.00032160949938256\\
552.01	0.00032160937172819\\
553.01	0.000321609001700078\\
554.01	0.000321607921710718\\
555.01	0.000321604744132882\\
556.01	0.00032159530906575\\
557.01	0.000321567009926508\\
558.01	0.00032148121484275\\
559.01	0.000321218239453271\\
560.01	0.000320403516967051\\
561.01	0.000317854518528006\\
562.01	0.000309812765753915\\
563.01	0.000290521225180146\\
564.01	0.000258815636837589\\
565.01	0.00022319689057226\\
566.01	0.000180005113740115\\
567.01	0.000117402829674417\\
568.01	5.15310646992881e-05\\
569.01	4.39279151884069e-06\\
570.01	0\\
571.01	0\\
572.01	0\\
573.01	0\\
574.01	0\\
575.01	0\\
576.01	0\\
577.01	0\\
578.01	1.73472347597681e-18\\
579.01	0\\
580.01	0\\
581.01	0\\
582.01	0\\
583.01	1.73472347597681e-18\\
584.01	0\\
585.01	0\\
586.01	0\\
587.01	0\\
588.01	0\\
589.01	0\\
590.01	0\\
591.01	0\\
592.01	0\\
593.01	0\\
594.01	0\\
595.01	0\\
596.01	0\\
597.01	0\\
598.01	0\\
599.01	0\\
599.02	0\\
599.03	0\\
599.04	0\\
599.05	0\\
599.06	0\\
599.07	0\\
599.08	0\\
599.09	0\\
599.1	0\\
599.11	0\\
599.12	0\\
599.13	0\\
599.14	0\\
599.15	0\\
599.16	0\\
599.17	0\\
599.18	0\\
599.19	0\\
599.2	0\\
599.21	0\\
599.22	0\\
599.23	0\\
599.24	0\\
599.25	0\\
599.26	0\\
599.27	0\\
599.28	0\\
599.29	0\\
599.3	0\\
599.31	0\\
599.32	0\\
599.33	0\\
599.34	0\\
599.35	0\\
599.36	0\\
599.37	0\\
599.38	0\\
599.39	0\\
599.4	0\\
599.41	0\\
599.42	0\\
599.43	0\\
599.44	0\\
599.45	0\\
599.46	0\\
599.47	0\\
599.48	0\\
599.49	0\\
599.5	0\\
599.51	0\\
599.52	0\\
599.53	0\\
599.54	0\\
599.55	0\\
599.56	0\\
599.57	0\\
599.58	0\\
599.59	0\\
599.6	0\\
599.61	0\\
599.62	0\\
599.63	0\\
599.64	0\\
599.65	0\\
599.66	0\\
599.67	0\\
599.68	0\\
599.69	0\\
599.7	0\\
599.71	0\\
599.72	0\\
599.73	0\\
599.74	0\\
599.75	0\\
599.76	0\\
599.77	0\\
599.78	0\\
599.79	0\\
599.8	0\\
599.81	0\\
599.82	0\\
599.83	0\\
599.84	0\\
599.85	0\\
599.86	0\\
599.87	0\\
599.88	0\\
599.89	0\\
599.9	0\\
599.91	0\\
599.92	0\\
599.93	0\\
599.94	0\\
599.95	0\\
599.96	0\\
599.97	0\\
599.98	0\\
599.99	0\\
600	0\\
};
\addplot [color=mycolor12,solid,forget plot]
  table[row sep=crcr]{%
0.01	0.00230832677588388\\
1.01	0.00230832677588388\\
2.01	0.00230832677588388\\
3.01	0.00230832677588388\\
4.01	0.00230832677588388\\
5.01	0.00230832677588388\\
6.01	0.00230832677588388\\
7.01	0.00230832677588388\\
8.01	0.00230832677588388\\
9.01	0.00230832677588388\\
10.01	0.00230832677588388\\
11.01	0.00230832677588388\\
12.01	0.00230832677588388\\
13.01	0.00230832677588388\\
14.01	0.00230832677588388\\
15.01	0.00230832677588388\\
16.01	0.00230832677588388\\
17.01	0.00230832677588388\\
18.01	0.00230832677588388\\
19.01	0.00230832677588388\\
20.01	0.00230832677588388\\
21.01	0.00230832677588388\\
22.01	0.00230832677588388\\
23.01	0.00230832677588388\\
24.01	0.00230832677588388\\
25.01	0.00230832677588388\\
26.01	0.00230832677588388\\
27.01	0.00230832677588388\\
28.01	0.00230832677588388\\
29.01	0.00230832677588388\\
30.01	0.00230832677588388\\
31.01	0.00230832677588388\\
32.01	0.00230832677588388\\
33.01	0.00230832677588388\\
34.01	0.00230832677588388\\
35.01	0.00230832677588388\\
36.01	0.00230832677588388\\
37.01	0.00230832677588388\\
38.01	0.00230832677588388\\
39.01	0.00230832677588388\\
40.01	0.00230832677588388\\
41.01	0.00230832677588388\\
42.01	0.00230832677588388\\
43.01	0.00230832677588388\\
44.01	0.00230832677588388\\
45.01	0.00230832677588388\\
46.01	0.00230832677588388\\
47.01	0.00230832677588388\\
48.01	0.00230832677588388\\
49.01	0.00230832677588388\\
50.01	0.00230832677588388\\
51.01	0.00230832677588388\\
52.01	0.00230832677588388\\
53.01	0.00230832677588388\\
54.01	0.00230832677588388\\
55.01	0.00230832677588388\\
56.01	0.00230832677588388\\
57.01	0.00230832677588388\\
58.01	0.00230832677588388\\
59.01	0.00230832677588388\\
60.01	0.00230832677588388\\
61.01	0.00230832677588388\\
62.01	0.00230832677588388\\
63.01	0.00230832677588388\\
64.01	0.00230832677588388\\
65.01	0.00230832677588388\\
66.01	0.00230832677588388\\
67.01	0.00230832677588388\\
68.01	0.00230832677588388\\
69.01	0.00230832677588388\\
70.01	0.00230832677588388\\
71.01	0.00230832677588388\\
72.01	0.00230832677588388\\
73.01	0.00230832677588388\\
74.01	0.00230832677588388\\
75.01	0.00230832677588388\\
76.01	0.00230832677588388\\
77.01	0.00230832677588388\\
78.01	0.00230832677588388\\
79.01	0.00230832677588388\\
80.01	0.00230832677588388\\
81.01	0.00230832677588388\\
82.01	0.00230832677588388\\
83.01	0.00230832677588388\\
84.01	0.00230832677588388\\
85.01	0.00230832677588388\\
86.01	0.00230832677588388\\
87.01	0.00230832677588388\\
88.01	0.00230832677588388\\
89.01	0.00230832677588388\\
90.01	0.00230832677588388\\
91.01	0.00230832677588388\\
92.01	0.00230832677588388\\
93.01	0.00230832677588388\\
94.01	0.00230832677588388\\
95.01	0.00230832677588388\\
96.01	0.00230832677588388\\
97.01	0.00230832677588388\\
98.01	0.00230832677588388\\
99.01	0.00230832677588388\\
100.01	0.00230832677588388\\
101.01	0.00230832677588388\\
102.01	0.00230832677588388\\
103.01	0.00230832677588388\\
104.01	0.00230832677588388\\
105.01	0.00230832677588388\\
106.01	0.00230832677588388\\
107.01	0.00230832677588388\\
108.01	0.00230832677588388\\
109.01	0.00230832677588388\\
110.01	0.00230832677588388\\
111.01	0.00230832677588388\\
112.01	0.00230832677588388\\
113.01	0.00230832677588388\\
114.01	0.00230832677588388\\
115.01	0.00230832677588388\\
116.01	0.00230832677588388\\
117.01	0.00230832677588388\\
118.01	0.00230832677588388\\
119.01	0.00230832677588388\\
120.01	0.00230832677588388\\
121.01	0.00230832677588388\\
122.01	0.00230832677588388\\
123.01	0.00230832677588388\\
124.01	0.00230832677588388\\
125.01	0.00230832677588388\\
126.01	0.00230832677588388\\
127.01	0.00230832677588388\\
128.01	0.00230832677588388\\
129.01	0.00230832677588388\\
130.01	0.00230832677588388\\
131.01	0.00230832677588388\\
132.01	0.00230832677588388\\
133.01	0.00230832677588388\\
134.01	0.00230832677588388\\
135.01	0.00230832677588388\\
136.01	0.00230832677588388\\
137.01	0.00230832677588388\\
138.01	0.00230832677588388\\
139.01	0.00230832677588388\\
140.01	0.00230832677588388\\
141.01	0.00230832677588388\\
142.01	0.00230832677588388\\
143.01	0.00230832677588388\\
144.01	0.00230832677588388\\
145.01	0.00230832677588388\\
146.01	0.00230832677588388\\
147.01	0.00230832677588388\\
148.01	0.00230832677588388\\
149.01	0.00230832677588388\\
150.01	0.00230832677588388\\
151.01	0.00230832677588388\\
152.01	0.00230832677588388\\
153.01	0.00230832677588388\\
154.01	0.00230832677588388\\
155.01	0.00230832677588388\\
156.01	0.00230832677588388\\
157.01	0.00230832677588388\\
158.01	0.00230832677588388\\
159.01	0.00230832677588388\\
160.01	0.00230832677588388\\
161.01	0.00230832677588388\\
162.01	0.00230832677588388\\
163.01	0.00230832677588388\\
164.01	0.00230832677588388\\
165.01	0.00230832677588388\\
166.01	0.00230832677588388\\
167.01	0.00230832677588388\\
168.01	0.00230832677588388\\
169.01	0.00230832677588388\\
170.01	0.00230832677588388\\
171.01	0.00230832677588388\\
172.01	0.00230832677588388\\
173.01	0.00230832677588388\\
174.01	0.00230832677588388\\
175.01	0.00230832677588388\\
176.01	0.00230832677588388\\
177.01	0.00230832677588388\\
178.01	0.00230832677588388\\
179.01	0.00230832677588388\\
180.01	0.00230832677588388\\
181.01	0.00230832677588388\\
182.01	0.00230832677588388\\
183.01	0.00230832677588388\\
184.01	0.00230832677588388\\
185.01	0.00230832677588388\\
186.01	0.00230832677588388\\
187.01	0.00230832677588388\\
188.01	0.00230832677588388\\
189.01	0.00230832677588388\\
190.01	0.00230832677588388\\
191.01	0.00230832677588388\\
192.01	0.00230832677588388\\
193.01	0.00230832677588388\\
194.01	0.00230832677588388\\
195.01	0.00230832677588388\\
196.01	0.00230832677588388\\
197.01	0.00230832677588388\\
198.01	0.00230832677588388\\
199.01	0.00230832677588388\\
200.01	0.00230832677588388\\
201.01	0.00230832677588388\\
202.01	0.00230832677588388\\
203.01	0.00230832677588388\\
204.01	0.00230832677588388\\
205.01	0.00230832677588388\\
206.01	0.00230832677588388\\
207.01	0.00230832677588388\\
208.01	0.00230832677588388\\
209.01	0.00230832677588388\\
210.01	0.00230832677588388\\
211.01	0.00230832677588388\\
212.01	0.00230832677588388\\
213.01	0.00230832677588388\\
214.01	0.00230832677588388\\
215.01	0.00230832677588388\\
216.01	0.00230832677588388\\
217.01	0.00230832677588388\\
218.01	0.00230832677588388\\
219.01	0.00230832677588388\\
220.01	0.00230832677588388\\
221.01	0.00230832677588388\\
222.01	0.00230832677588388\\
223.01	0.00230832677588388\\
224.01	0.00230832677588388\\
225.01	0.00230832677588388\\
226.01	0.00230832677588388\\
227.01	0.00230832677588388\\
228.01	0.00230832677588388\\
229.01	0.00230832677588388\\
230.01	0.00230832677588388\\
231.01	0.00230832677588388\\
232.01	0.00230832677588388\\
233.01	0.00230832677588388\\
234.01	0.00230832677588388\\
235.01	0.00230832677588388\\
236.01	0.00230832677588388\\
237.01	0.00230832677588388\\
238.01	0.00230832677588388\\
239.01	0.00230832677588388\\
240.01	0.00230832677588388\\
241.01	0.00230832677588388\\
242.01	0.00230832677588388\\
243.01	0.00230832677588388\\
244.01	0.00230832677588388\\
245.01	0.00230832677588388\\
246.01	0.00230832677588388\\
247.01	0.00230832677588388\\
248.01	0.00230832677588388\\
249.01	0.00230832677588388\\
250.01	0.00230832677588388\\
251.01	0.00230832677588388\\
252.01	0.00230832677588388\\
253.01	0.00230832677588388\\
254.01	0.00230832677588388\\
255.01	0.00230832677588388\\
256.01	0.00230832677588388\\
257.01	0.00230832677588388\\
258.01	0.00230832677588388\\
259.01	0.00230832677588388\\
260.01	0.00230832677588388\\
261.01	0.00230832677588388\\
262.01	0.00230832677588388\\
263.01	0.00230832677588388\\
264.01	0.00230832677588388\\
265.01	0.00230832677588388\\
266.01	0.00230832677588388\\
267.01	0.00230832677588388\\
268.01	0.00230832677588388\\
269.01	0.00230832677588388\\
270.01	0.00230832677588388\\
271.01	0.00230832677588388\\
272.01	0.00230832677588388\\
273.01	0.00230832677588388\\
274.01	0.00230832677588388\\
275.01	0.00230832677588388\\
276.01	0.00230832677588388\\
277.01	0.00230832677588388\\
278.01	0.00230832677588388\\
279.01	0.00230832677588388\\
280.01	0.00230832677588388\\
281.01	0.00230832677588388\\
282.01	0.00230832677588388\\
283.01	0.00230832677588388\\
284.01	0.00230832677588388\\
285.01	0.00230832677588388\\
286.01	0.00230832677588388\\
287.01	0.00230832677588388\\
288.01	0.00230832677588388\\
289.01	0.00230832677588388\\
290.01	0.00230832677588388\\
291.01	0.00230832677588388\\
292.01	0.00230832677588388\\
293.01	0.00230832677588388\\
294.01	0.00230832677588388\\
295.01	0.00230832677588388\\
296.01	0.00230832677588388\\
297.01	0.00230832677588388\\
298.01	0.00230832677588388\\
299.01	0.00230832677588388\\
300.01	0.00230832677588388\\
301.01	0.00230832677588388\\
302.01	0.00230832677588388\\
303.01	0.00230832677588388\\
304.01	0.00230832677588388\\
305.01	0.00230832677588388\\
306.01	0.00230832677588388\\
307.01	0.00230832677588388\\
308.01	0.00230832677588388\\
309.01	0.00230832677588388\\
310.01	0.00230832677588388\\
311.01	0.00230832677588388\\
312.01	0.00230832677588388\\
313.01	0.00230832677588388\\
314.01	0.00230832677588388\\
315.01	0.00230832677588388\\
316.01	0.00230832677588388\\
317.01	0.00230832677588388\\
318.01	0.00230832677588388\\
319.01	0.00230832677588388\\
320.01	0.00230832677588388\\
321.01	0.00230832677588388\\
322.01	0.00230832677588388\\
323.01	0.00230832677588388\\
324.01	0.00230832677588388\\
325.01	0.00230832677588388\\
326.01	0.00230832677588388\\
327.01	0.00230832677588388\\
328.01	0.00230832677588388\\
329.01	0.00230832677588388\\
330.01	0.00230832677588388\\
331.01	0.00230832677588388\\
332.01	0.00230832677588388\\
333.01	0.00230832677588388\\
334.01	0.00230832677588388\\
335.01	0.00230832677588388\\
336.01	0.00230832677588388\\
337.01	0.00230832677588388\\
338.01	0.00230832677588388\\
339.01	0.00230832677588388\\
340.01	0.00230832677588388\\
341.01	0.00230832677588388\\
342.01	0.00230832677588388\\
343.01	0.00230832677588388\\
344.01	0.00230832677588388\\
345.01	0.00230832677588388\\
346.01	0.00230832677588388\\
347.01	0.00230832677588388\\
348.01	0.00230832677588388\\
349.01	0.00230832677588388\\
350.01	0.00230832677588388\\
351.01	0.00230832677588388\\
352.01	0.00230832677588388\\
353.01	0.00230832677588388\\
354.01	0.00230832677588388\\
355.01	0.00230832677588388\\
356.01	0.00230832677588388\\
357.01	0.00230832677588388\\
358.01	0.00230832677588388\\
359.01	0.00230832677588388\\
360.01	0.00230832677588388\\
361.01	0.00230832677588388\\
362.01	0.00230832677588388\\
363.01	0.00230832677588388\\
364.01	0.00230832677588388\\
365.01	0.00230832677588388\\
366.01	0.00230832677588388\\
367.01	0.00230832677588388\\
368.01	0.00230832677588388\\
369.01	0.00230832677588388\\
370.01	0.00230832677588388\\
371.01	0.00230832677588388\\
372.01	0.00230832677588388\\
373.01	0.00230832677588388\\
374.01	0.00230832677588388\\
375.01	0.00230832677588388\\
376.01	0.00230832677588388\\
377.01	0.00230832677588388\\
378.01	0.00230832677588388\\
379.01	0.00230832677588388\\
380.01	0.00230832677588388\\
381.01	0.00230832677588388\\
382.01	0.00230832677588388\\
383.01	0.00230832677588388\\
384.01	0.00230832677588388\\
385.01	0.00230832677588388\\
386.01	0.00230832677588388\\
387.01	0.00230832677588388\\
388.01	0.00230832677588388\\
389.01	0.00230832677588388\\
390.01	0.00230832677588388\\
391.01	0.00230832677588388\\
392.01	0.00230832677588388\\
393.01	0.00230832677588388\\
394.01	0.00230832677588388\\
395.01	0.00230832677588388\\
396.01	0.00230832677588388\\
397.01	0.00230832677588388\\
398.01	0.00230832677588388\\
399.01	0.00230832677588388\\
400.01	0.00230832677588388\\
401.01	0.00230832677588388\\
402.01	0.00230832677588388\\
403.01	0.00230832677588388\\
404.01	0.00230832677588388\\
405.01	0.00230832677588388\\
406.01	0.00230832677588388\\
407.01	0.00230832677588388\\
408.01	0.00230832677588388\\
409.01	0.00230832677588388\\
410.01	0.00230832677588388\\
411.01	0.00230832677588388\\
412.01	0.00230832677588388\\
413.01	0.00230832677588388\\
414.01	0.00230832677588388\\
415.01	0.00230832677588388\\
416.01	0.00230832677588388\\
417.01	0.00230832677588388\\
418.01	0.00230832677588388\\
419.01	0.00230832677588388\\
420.01	0.00230832677588388\\
421.01	0.00230832677588388\\
422.01	0.00230832677588388\\
423.01	0.00230832677588388\\
424.01	0.00230832677588388\\
425.01	0.00230832677588388\\
426.01	0.00230832677588388\\
427.01	0.00230832677588388\\
428.01	0.00230832677588388\\
429.01	0.00230832677588388\\
430.01	0.00230832677588388\\
431.01	0.00230832677588388\\
432.01	0.00230832677588388\\
433.01	0.00230832677588388\\
434.01	0.00230832677588388\\
435.01	0.00230832677588388\\
436.01	0.00230832677588388\\
437.01	0.00230832677588388\\
438.01	0.00230832677588388\\
439.01	0.00230832677588388\\
440.01	0.00230832677588388\\
441.01	0.00230832677588388\\
442.01	0.00230832677588388\\
443.01	0.00230832677588388\\
444.01	0.00230832677588388\\
445.01	0.00230832677588388\\
446.01	0.00230832677588388\\
447.01	0.00230832677588385\\
448.01	0.00230832677588378\\
449.01	0.00230832677588358\\
450.01	0.00230832677588308\\
451.01	0.00230832677588178\\
452.01	0.00230832677587848\\
453.01	0.00230832677587017\\
454.01	0.00230832677584946\\
455.01	0.00230832677579852\\
456.01	0.00230832677567521\\
457.01	0.00230832677538211\\
458.01	0.00230832677470067\\
459.01	0.00230832677315791\\
460.01	0.00230832676977686\\
461.01	0.00230832676265952\\
462.01	0.00230832674841642\\
463.01	0.00230832672169671\\
464.01	0.00230832667559282\\
465.01	0.00230832660424001\\
466.01	0.00230832650799597\\
467.01	0.00230832639644697\\
468.01	0.00230832628012502\\
469.01	0.00230832616191159\\
470.01	0.00230832604287346\\
471.01	0.00230832592509532\\
472.01	0.00230832581240644\\
473.01	0.00230832571105018\\
474.01	0.00230832562949901\\
475.01	0.00230832557582186\\
476.01	0.00230832555157799\\
477.01	0.00230832554657476\\
478.01	0.00230832554643708\\
479.01	0.00230832554643708\\
480.01	0.00230832554643708\\
481.01	0.00230832554643708\\
482.01	0.00230832554643708\\
483.01	0.00230832554643708\\
484.01	0.00230832554643708\\
485.01	0.00230832554643708\\
486.01	0.00230832554643708\\
487.01	0.00230832554643708\\
488.01	0.00230832554643708\\
489.01	0.00230832554643708\\
490.01	0.00230832554643708\\
491.01	0.00230832554643708\\
492.01	0.00230832554643708\\
493.01	0.00230832554643708\\
494.01	0.00230832554643708\\
495.01	0.00230832554643708\\
496.01	0.00230832554643708\\
497.01	0.00230832554643708\\
498.01	0.00230832554643708\\
499.01	0.00230832554643708\\
500.01	0.00230832554643708\\
501.01	0.00230832554643708\\
502.01	0.00230832554643708\\
503.01	0.00230832554643708\\
504.01	0.00230832554643708\\
505.01	0.00230832554643708\\
506.01	0.00230832554643708\\
507.01	0.00230832554643708\\
508.01	0.00230832554643708\\
509.01	0.00230832554643708\\
510.01	0.00230832554643706\\
511.01	0.00230832554643699\\
512.01	0.00230832554643681\\
513.01	0.00230832554643632\\
514.01	0.00230832554643497\\
515.01	0.0023083255464313\\
516.01	0.00230832554642139\\
517.01	0.00230832554639465\\
518.01	0.00230832554632288\\
519.01	0.00230832554613126\\
520.01	0.00230832554562292\\
521.01	0.00230832554428438\\
522.01	0.00230832554079122\\
523.01	0.0023083255317732\\
524.01	0.00230832550879643\\
525.01	0.00230832545119599\\
526.01	0.002308325309691\\
527.01	0.00230832497087839\\
528.01	0.00230832418619398\\
529.01	0.00230832244749176\\
530.01	0.00230831882192436\\
531.01	0.00230831189250172\\
532.01	0.00230830028466374\\
533.01	0.00230828450470487\\
534.01	0.00230826815361379\\
535.01	0.00230825300739821\\
536.01	0.00230823904377992\\
537.01	0.00230822681893579\\
538.01	0.00230821812485519\\
539.01	0.00230821447379591\\
540.01	0.00230821417331235\\
541.01	0.002308214173311\\
542.01	0.0023082141733072\\
543.01	0.00230821417329649\\
544.01	0.00230821417326634\\
545.01	0.00230821417318166\\
546.01	0.00230821417294417\\
547.01	0.00230821417227934\\
548.01	0.00230821417042241\\
549.01	0.00230821416524946\\
550.01	0.00230821415088388\\
551.01	0.00230821411113991\\
552.01	0.0023082140016871\\
553.01	0.00230821370195917\\
554.01	0.00230821288695113\\
555.01	0.00230821069054941\\
556.01	0.00230820483924602\\
557.01	0.00230818948614138\\
558.01	0.00230815002122632\\
559.01	0.00230805146354952\\
560.01	0.00230781562226196\\
561.01	0.00230728866371109\\
562.01	0.00230625128843328\\
563.01	0.00230463539557428\\
564.01	0.00230261463947325\\
565.01	0.00230027902002805\\
566.01	0.00229739045615409\\
567.01	0.00229414404146691\\
568.01	0.00229160501562282\\
569.01	0.00229055117153649\\
570.01	0.00229052498630496\\
571.01	0.00229052423870807\\
572.01	0.00229052200145302\\
573.01	0.00229051531235272\\
574.01	0.00229049533245125\\
575.01	0.00229043571706054\\
576.01	0.00229025804248841\\
577.01	0.00228972916792507\\
578.01	0.00228815701010785\\
579.01	0.00228349029843025\\
580.01	0.00226965902075653\\
581.01	0.00222878167195284\\
582.01	0.00213964864116229\\
583.01	0.00201726281566172\\
584.01	0.00185939417254293\\
585.01	0.00164117735174267\\
586.01	0.00140842561566351\\
587.01	0.00116447021169677\\
588.01	0.000899880963021769\\
589.01	0.00058797839348404\\
590.01	0.000224847652437047\\
591.01	3.75983539933726e-06\\
592.01	0\\
593.01	0\\
594.01	0\\
595.01	0\\
596.01	0\\
597.01	0\\
598.01	0\\
599.01	0\\
599.02	0\\
599.03	0\\
599.04	0\\
599.05	0\\
599.06	0\\
599.07	0\\
599.08	0\\
599.09	0\\
599.1	0\\
599.11	0\\
599.12	0\\
599.13	0\\
599.14	0\\
599.15	0\\
599.16	0\\
599.17	0\\
599.18	0\\
599.19	0\\
599.2	0\\
599.21	0\\
599.22	0\\
599.23	0\\
599.24	0\\
599.25	0\\
599.26	0\\
599.27	0\\
599.28	0\\
599.29	0\\
599.3	0\\
599.31	0\\
599.32	0\\
599.33	0\\
599.34	0\\
599.35	0\\
599.36	0\\
599.37	0\\
599.38	0\\
599.39	0\\
599.4	0\\
599.41	0\\
599.42	0\\
599.43	0\\
599.44	0\\
599.45	0\\
599.46	0\\
599.47	0\\
599.48	0\\
599.49	0\\
599.5	0\\
599.51	0\\
599.52	0\\
599.53	0\\
599.54	0\\
599.55	0\\
599.56	0\\
599.57	0\\
599.58	0\\
599.59	0\\
599.6	0\\
599.61	0\\
599.62	0\\
599.63	0\\
599.64	0\\
599.65	0\\
599.66	0\\
599.67	0\\
599.68	0\\
599.69	0\\
599.7	0\\
599.71	0\\
599.72	0\\
599.73	0\\
599.74	0\\
599.75	0\\
599.76	0\\
599.77	0\\
599.78	0\\
599.79	0\\
599.8	0\\
599.81	0\\
599.82	0\\
599.83	0\\
599.84	0\\
599.85	0\\
599.86	0\\
599.87	0\\
599.88	0\\
599.89	0\\
599.9	0\\
599.91	0\\
599.92	0\\
599.93	0\\
599.94	0\\
599.95	0\\
599.96	0\\
599.97	0\\
599.98	0\\
599.99	0\\
600	0\\
};
\addplot [color=mycolor13,solid,forget plot]
  table[row sep=crcr]{%
0.01	0\\
1.01	0\\
2.01	0\\
3.01	0\\
4.01	0\\
5.01	0\\
6.01	0\\
7.01	0\\
8.01	0\\
9.01	0\\
10.01	0\\
11.01	0\\
12.01	0\\
13.01	0\\
14.01	0\\
15.01	0\\
16.01	0\\
17.01	0\\
18.01	0\\
19.01	0\\
20.01	0\\
21.01	0\\
22.01	0\\
23.01	0\\
24.01	0\\
25.01	0\\
26.01	0\\
27.01	0\\
28.01	0\\
29.01	0\\
30.01	0\\
31.01	0\\
32.01	0\\
33.01	0\\
34.01	0\\
35.01	0\\
36.01	0\\
37.01	0\\
38.01	0\\
39.01	0\\
40.01	0\\
41.01	0\\
42.01	0\\
43.01	0\\
44.01	0\\
45.01	0\\
46.01	0\\
47.01	0\\
48.01	0\\
49.01	0\\
50.01	0\\
51.01	0\\
52.01	0\\
53.01	0\\
54.01	0\\
55.01	0\\
56.01	0\\
57.01	0\\
58.01	0\\
59.01	0\\
60.01	0\\
61.01	0\\
62.01	0\\
63.01	0\\
64.01	0\\
65.01	0\\
66.01	0\\
67.01	0\\
68.01	0\\
69.01	0\\
70.01	0\\
71.01	0\\
72.01	0\\
73.01	0\\
74.01	0\\
75.01	0\\
76.01	0\\
77.01	0\\
78.01	0\\
79.01	0\\
80.01	0\\
81.01	0\\
82.01	0\\
83.01	0\\
84.01	0\\
85.01	0\\
86.01	0\\
87.01	0\\
88.01	0\\
89.01	0\\
90.01	0\\
91.01	0\\
92.01	0\\
93.01	0\\
94.01	0\\
95.01	0\\
96.01	0\\
97.01	0\\
98.01	0\\
99.01	0\\
100.01	0\\
101.01	0\\
102.01	0\\
103.01	0\\
104.01	0\\
105.01	0\\
106.01	0\\
107.01	0\\
108.01	0\\
109.01	0\\
110.01	0\\
111.01	0\\
112.01	0\\
113.01	0\\
114.01	0\\
115.01	0\\
116.01	0\\
117.01	0\\
118.01	0\\
119.01	0\\
120.01	0\\
121.01	0\\
122.01	0\\
123.01	0\\
124.01	0\\
125.01	0\\
126.01	0\\
127.01	0\\
128.01	0\\
129.01	0\\
130.01	0\\
131.01	0\\
132.01	0\\
133.01	0\\
134.01	0\\
135.01	0\\
136.01	0\\
137.01	0\\
138.01	0\\
139.01	0\\
140.01	0\\
141.01	0\\
142.01	0\\
143.01	0\\
144.01	0\\
145.01	0\\
146.01	0\\
147.01	0\\
148.01	0\\
149.01	0\\
150.01	0\\
151.01	0\\
152.01	0\\
153.01	0\\
154.01	0\\
155.01	0\\
156.01	0\\
157.01	0\\
158.01	0\\
159.01	0\\
160.01	0\\
161.01	0\\
162.01	0\\
163.01	0\\
164.01	0\\
165.01	0\\
166.01	0\\
167.01	0\\
168.01	0\\
169.01	0\\
170.01	0\\
171.01	0\\
172.01	0\\
173.01	0\\
174.01	0\\
175.01	0\\
176.01	0\\
177.01	0\\
178.01	0\\
179.01	0\\
180.01	0\\
181.01	0\\
182.01	0\\
183.01	0\\
184.01	0\\
185.01	0\\
186.01	0\\
187.01	0\\
188.01	0\\
189.01	0\\
190.01	0\\
191.01	0\\
192.01	0\\
193.01	0\\
194.01	0\\
195.01	0\\
196.01	0\\
197.01	0\\
198.01	0\\
199.01	0\\
200.01	0\\
201.01	0\\
202.01	0\\
203.01	0\\
204.01	0\\
205.01	0\\
206.01	0\\
207.01	0\\
208.01	0\\
209.01	0\\
210.01	0\\
211.01	0\\
212.01	0\\
213.01	0\\
214.01	0\\
215.01	0\\
216.01	0\\
217.01	0\\
218.01	0\\
219.01	0\\
220.01	0\\
221.01	0\\
222.01	0\\
223.01	0\\
224.01	0\\
225.01	0\\
226.01	0\\
227.01	0\\
228.01	0\\
229.01	0\\
230.01	0\\
231.01	0\\
232.01	0\\
233.01	0\\
234.01	0\\
235.01	0\\
236.01	0\\
237.01	0\\
238.01	0\\
239.01	0\\
240.01	0\\
241.01	0\\
242.01	0\\
243.01	0\\
244.01	0\\
245.01	0\\
246.01	0\\
247.01	0\\
248.01	0\\
249.01	0\\
250.01	0\\
251.01	0\\
252.01	0\\
253.01	0\\
254.01	0\\
255.01	0\\
256.01	0\\
257.01	0\\
258.01	0\\
259.01	0\\
260.01	0\\
261.01	0\\
262.01	0\\
263.01	0\\
264.01	0\\
265.01	0\\
266.01	0\\
267.01	0\\
268.01	0\\
269.01	0\\
270.01	0\\
271.01	0\\
272.01	0\\
273.01	0\\
274.01	0\\
275.01	0\\
276.01	0\\
277.01	0\\
278.01	0\\
279.01	0\\
280.01	0\\
281.01	0\\
282.01	0\\
283.01	0\\
284.01	0\\
285.01	0\\
286.01	0\\
287.01	0\\
288.01	0\\
289.01	0\\
290.01	0\\
291.01	0\\
292.01	0\\
293.01	0\\
294.01	0\\
295.01	0\\
296.01	0\\
297.01	0\\
298.01	0\\
299.01	0\\
300.01	0\\
301.01	0\\
302.01	0\\
303.01	0\\
304.01	0\\
305.01	0\\
306.01	0\\
307.01	0\\
308.01	0\\
309.01	0\\
310.01	0\\
311.01	0\\
312.01	0\\
313.01	0\\
314.01	0\\
315.01	0\\
316.01	0\\
317.01	0\\
318.01	0\\
319.01	0\\
320.01	0\\
321.01	0\\
322.01	0\\
323.01	0\\
324.01	0\\
325.01	0\\
326.01	0\\
327.01	0\\
328.01	0\\
329.01	0\\
330.01	0\\
331.01	0\\
332.01	0\\
333.01	0\\
334.01	0\\
335.01	0\\
336.01	0\\
337.01	0\\
338.01	0\\
339.01	0\\
340.01	0\\
341.01	0\\
342.01	0\\
343.01	0\\
344.01	0\\
345.01	0\\
346.01	0\\
347.01	0\\
348.01	0\\
349.01	0\\
350.01	0\\
351.01	0\\
352.01	0\\
353.01	0\\
354.01	0\\
355.01	0\\
356.01	0\\
357.01	0\\
358.01	0\\
359.01	0\\
360.01	0\\
361.01	0\\
362.01	0\\
363.01	0\\
364.01	0\\
365.01	0\\
366.01	0\\
367.01	0\\
368.01	0\\
369.01	0\\
370.01	0\\
371.01	0\\
372.01	0\\
373.01	0\\
374.01	0\\
375.01	0\\
376.01	0\\
377.01	0\\
378.01	0\\
379.01	0\\
380.01	0\\
381.01	0\\
382.01	0\\
383.01	0\\
384.01	0\\
385.01	0\\
386.01	0\\
387.01	0\\
388.01	0\\
389.01	0\\
390.01	0\\
391.01	0\\
392.01	0\\
393.01	0\\
394.01	0\\
395.01	0\\
396.01	0\\
397.01	0\\
398.01	0\\
399.01	0\\
400.01	0\\
401.01	0\\
402.01	0\\
403.01	0\\
404.01	0\\
405.01	0\\
406.01	0\\
407.01	0\\
408.01	0\\
409.01	0\\
410.01	0\\
411.01	0\\
412.01	0\\
413.01	0\\
414.01	0\\
415.01	0\\
416.01	0\\
417.01	0\\
418.01	0\\
419.01	0\\
420.01	0\\
421.01	0\\
422.01	0\\
423.01	0\\
424.01	0\\
425.01	0\\
426.01	0\\
427.01	0\\
428.01	0\\
429.01	0\\
430.01	0\\
431.01	0\\
432.01	0\\
433.01	0\\
434.01	0\\
435.01	0\\
436.01	0\\
437.01	0\\
438.01	0\\
439.01	0\\
440.01	0\\
441.01	0\\
442.01	0\\
443.01	0\\
444.01	0\\
445.01	0\\
446.01	0\\
447.01	0\\
448.01	0\\
449.01	0\\
450.01	0\\
451.01	0\\
452.01	0\\
453.01	0\\
454.01	0\\
455.01	0\\
456.01	0\\
457.01	0\\
458.01	0\\
459.01	0\\
460.01	0\\
461.01	0\\
462.01	0\\
463.01	0\\
464.01	0\\
465.01	0\\
466.01	0\\
467.01	0\\
468.01	0\\
469.01	0\\
470.01	0\\
471.01	0\\
472.01	0\\
473.01	0\\
474.01	0\\
475.01	0\\
476.01	0\\
477.01	0\\
478.01	0\\
479.01	0\\
480.01	0\\
481.01	0\\
482.01	0\\
483.01	0\\
484.01	0\\
485.01	0\\
486.01	0\\
487.01	0\\
488.01	0\\
489.01	0\\
490.01	0\\
491.01	0\\
492.01	0\\
493.01	0\\
494.01	0\\
495.01	0\\
496.01	0\\
497.01	0\\
498.01	0\\
499.01	0\\
500.01	0\\
501.01	0\\
502.01	0\\
503.01	0\\
504.01	0\\
505.01	0\\
506.01	0\\
507.01	0\\
508.01	0\\
509.01	0\\
510.01	0\\
511.01	0\\
512.01	0\\
513.01	0\\
514.01	0\\
515.01	0\\
516.01	0\\
517.01	0\\
518.01	0\\
519.01	0\\
520.01	0\\
521.01	0\\
522.01	0\\
523.01	0\\
524.01	0\\
525.01	0\\
526.01	0\\
527.01	0\\
528.01	0\\
529.01	0\\
530.01	0\\
531.01	0\\
532.01	0\\
533.01	0\\
534.01	0\\
535.01	0\\
536.01	0\\
537.01	0\\
538.01	0\\
539.01	0\\
540.01	0\\
541.01	0\\
542.01	0\\
543.01	0\\
544.01	0\\
545.01	0\\
546.01	0\\
547.01	0\\
548.01	0\\
549.01	0\\
550.01	0\\
551.01	0\\
552.01	0\\
553.01	0\\
554.01	0\\
555.01	0\\
556.01	0\\
557.01	0\\
558.01	0\\
559.01	0\\
560.01	0\\
561.01	0\\
562.01	0\\
563.01	0\\
564.01	0\\
565.01	0\\
566.01	0\\
567.01	0\\
568.01	0\\
569.01	0\\
570.01	0\\
571.01	0\\
572.01	0\\
573.01	0\\
574.01	0\\
575.01	0\\
576.01	0\\
577.01	0\\
578.01	0\\
579.01	0\\
580.01	0\\
581.01	0\\
582.01	0\\
583.01	0\\
584.01	0\\
585.01	0\\
586.01	0\\
587.01	0\\
588.01	0\\
589.01	0\\
590.01	0\\
591.01	0\\
592.01	0\\
593.01	0\\
594.01	0\\
595.01	0\\
596.01	0\\
597.01	0\\
598.01	0\\
599.01	0.0022382818960328\\
599.02	0.00228829715400941\\
599.03	0.00233870844988766\\
599.04	0.00238951982934219\\
599.05	0.00244073538661882\\
599.06	0.00249235926524579\\
599.07	0.00254439565875786\\
599.08	0.00259684881143348\\
599.09	0.00264972301904528\\
599.1	0.0027030226296243\\
599.11	0.00275675204423814\\
599.12	0.00281091571778341\\
599.13	0.00286551815979286\\
599.14	0.00292056393525741\\
599.15	0.00297605766546351\\
599.16	0.00303200402884618\\
599.17	0.00308840776185803\\
599.18	0.00314527365985481\\
599.19	0.00320260657799763\\
599.2	0.00326041143217246\\
599.21	0.00331869319992726\\
599.22	0.00337745692142712\\
599.23	0.00343670770042793\\
599.24	0.00349645070526896\\
599.25	0.00355669116988489\\
599.26	0.0036174343948377\\
599.27	0.00367868574836898\\
599.28	0.00374045066747309\\
599.29	0.00380273465899182\\
599.3	0.00386554330073107\\
599.31	0.00392888224260002\\
599.32	0.00399275720777354\\
599.33	0.00405717399387833\\
599.34	0.00412213847420342\\
599.35	0.00418765659893578\\
599.36	0.00425373439642159\\
599.37	0.00432037797445393\\
599.38	0.00438759352158759\\
599.39	0.00445538730848178\\
599.4	0.00452376568927147\\
599.41	0.0045927351029681\\
599.42	0.00466230207489068\\
599.43	0.0047324732181279\\
599.44	0.00480325523503231\\
599.45	0.0048746549187474\\
599.46	0.00494667915476945\\
599.47	0.005019334922583\\
599.48	0.00509262929729296\\
599.49	0.00516656945129368\\
599.5	0.00524116265597606\\
599.51	0.00531641628347389\\
599.52	0.00539233780845056\\
599.53	0.00546893480992739\\
599.54	0.00554621497315491\\
599.55	0.00562418609152839\\
599.56	0.00570285606854896\\
599.57	0.00578223291983187\\
599.58	0.00586232477516331\\
599.59	0.00594313988060741\\
599.6	0.006024686600665\\
599.61	0.00610697342048581\\
599.62	0.00619000894813599\\
599.63	0.00627380191692264\\
599.64	0.00635836118777732\\
599.65	0.00644369575170061\\
599.66	0.00652981473226971\\
599.67	0.00661672738821123\\
599.68	0.00670444311604154\\
599.69	0.00679297145277696\\
599.7	0.00688232207871629\\
599.71	0.00697250482029821\\
599.72	0.0070635296530363\\
599.73	0.00715540670453446\\
599.74	0.00724814625758565\\
599.75	0.00734175875335703\\
599.76	0.00743625479466474\\
599.77	0.0075316451493416\\
599.78	0.00762794075370135\\
599.79	0.00772515271610301\\
599.8	0.00782329232061922\\
599.81	0.00792237103081269\\
599.82	0.00802240049362481\\
599.83	0.00812339254338108\\
599.84	0.00822535920591771\\
599.85	0.00832831270283452\\
599.86	0.00843226545587912\\
599.87	0.00853723009146771\\
599.88	0.00864321944534821\\
599.89	0.00875024656741154\\
599.9	0.0088583247266573\\
599.91	0.00896746741632041\\
599.92	0.00907768835916546\\
599.93	0.0091890015129561\\
599.94	0.00930142107610699\\
599.95	0.00941496149352625\\
599.96	0.009529637462657\\
599.97	0.00964546393972657\\
599.98	0.00976245614621301\\
599.99	0.00988062957553847\\
600	0.01\\
};
\addplot [color=mycolor14,solid,forget plot]
  table[row sep=crcr]{%
0.01	0\\
1.01	0\\
2.01	0\\
3.01	0\\
4.01	0\\
5.01	0\\
6.01	0\\
7.01	0\\
8.01	0\\
9.01	0\\
10.01	0\\
11.01	0\\
12.01	0\\
13.01	0\\
14.01	0\\
15.01	0\\
16.01	0\\
17.01	0\\
18.01	0\\
19.01	0\\
20.01	0\\
21.01	0\\
22.01	0\\
23.01	0\\
24.01	0\\
25.01	0\\
26.01	0\\
27.01	0\\
28.01	0\\
29.01	0\\
30.01	0\\
31.01	0\\
32.01	0\\
33.01	0\\
34.01	0\\
35.01	0\\
36.01	0\\
37.01	0\\
38.01	0\\
39.01	0\\
40.01	0\\
41.01	0\\
42.01	0\\
43.01	0\\
44.01	0\\
45.01	0\\
46.01	0\\
47.01	0\\
48.01	0\\
49.01	0\\
50.01	0\\
51.01	0\\
52.01	0\\
53.01	0\\
54.01	0\\
55.01	0\\
56.01	0\\
57.01	0\\
58.01	0\\
59.01	0\\
60.01	0\\
61.01	0\\
62.01	0\\
63.01	0\\
64.01	0\\
65.01	0\\
66.01	0\\
67.01	0\\
68.01	0\\
69.01	0\\
70.01	0\\
71.01	0\\
72.01	0\\
73.01	0\\
74.01	0\\
75.01	0\\
76.01	0\\
77.01	0\\
78.01	0\\
79.01	0\\
80.01	0\\
81.01	0\\
82.01	0\\
83.01	0\\
84.01	0\\
85.01	0\\
86.01	0\\
87.01	0\\
88.01	0\\
89.01	0\\
90.01	0\\
91.01	0\\
92.01	0\\
93.01	0\\
94.01	0\\
95.01	0\\
96.01	0\\
97.01	0\\
98.01	0\\
99.01	0\\
100.01	0\\
101.01	0\\
102.01	0\\
103.01	0\\
104.01	0\\
105.01	0\\
106.01	0\\
107.01	0\\
108.01	0\\
109.01	0\\
110.01	0\\
111.01	0\\
112.01	0\\
113.01	0\\
114.01	0\\
115.01	0\\
116.01	0\\
117.01	0\\
118.01	0\\
119.01	0\\
120.01	0\\
121.01	0\\
122.01	0\\
123.01	0\\
124.01	0\\
125.01	0\\
126.01	0\\
127.01	0\\
128.01	0\\
129.01	0\\
130.01	0\\
131.01	0\\
132.01	0\\
133.01	0\\
134.01	0\\
135.01	0\\
136.01	0\\
137.01	0\\
138.01	0\\
139.01	0\\
140.01	0\\
141.01	0\\
142.01	0\\
143.01	0\\
144.01	0\\
145.01	0\\
146.01	0\\
147.01	0\\
148.01	0\\
149.01	0\\
150.01	0\\
151.01	0\\
152.01	0\\
153.01	0\\
154.01	0\\
155.01	0\\
156.01	0\\
157.01	0\\
158.01	0\\
159.01	0\\
160.01	0\\
161.01	0\\
162.01	0\\
163.01	0\\
164.01	0\\
165.01	0\\
166.01	0\\
167.01	0\\
168.01	0\\
169.01	0\\
170.01	0\\
171.01	0\\
172.01	0\\
173.01	0\\
174.01	0\\
175.01	0\\
176.01	0\\
177.01	0\\
178.01	0\\
179.01	0\\
180.01	0\\
181.01	0\\
182.01	0\\
183.01	0\\
184.01	0\\
185.01	0\\
186.01	0\\
187.01	0\\
188.01	0\\
189.01	0\\
190.01	0\\
191.01	0\\
192.01	0\\
193.01	0\\
194.01	0\\
195.01	0\\
196.01	0\\
197.01	0\\
198.01	0\\
199.01	0\\
200.01	0\\
201.01	0\\
202.01	0\\
203.01	0\\
204.01	0\\
205.01	0\\
206.01	0\\
207.01	0\\
208.01	0\\
209.01	0\\
210.01	0\\
211.01	0\\
212.01	0\\
213.01	0\\
214.01	0\\
215.01	0\\
216.01	0\\
217.01	0\\
218.01	0\\
219.01	0\\
220.01	0\\
221.01	0\\
222.01	0\\
223.01	0\\
224.01	0\\
225.01	0\\
226.01	0\\
227.01	0\\
228.01	0\\
229.01	0\\
230.01	0\\
231.01	0\\
232.01	0\\
233.01	0\\
234.01	0\\
235.01	0\\
236.01	0\\
237.01	0\\
238.01	0\\
239.01	0\\
240.01	0\\
241.01	0\\
242.01	0\\
243.01	0\\
244.01	0\\
245.01	0\\
246.01	0\\
247.01	0\\
248.01	0\\
249.01	0\\
250.01	0\\
251.01	0\\
252.01	0\\
253.01	0\\
254.01	0\\
255.01	0\\
256.01	0\\
257.01	0\\
258.01	0\\
259.01	0\\
260.01	0\\
261.01	0\\
262.01	0\\
263.01	0\\
264.01	0\\
265.01	0\\
266.01	0\\
267.01	0\\
268.01	0\\
269.01	0\\
270.01	0\\
271.01	0\\
272.01	0\\
273.01	0\\
274.01	0\\
275.01	0\\
276.01	0\\
277.01	0\\
278.01	0\\
279.01	0\\
280.01	0\\
281.01	0\\
282.01	0\\
283.01	0\\
284.01	0\\
285.01	0\\
286.01	0\\
287.01	0\\
288.01	0\\
289.01	0\\
290.01	0\\
291.01	0\\
292.01	0\\
293.01	0\\
294.01	0\\
295.01	0\\
296.01	0\\
297.01	0\\
298.01	0\\
299.01	0\\
300.01	0\\
301.01	0\\
302.01	0\\
303.01	0\\
304.01	0\\
305.01	0\\
306.01	0\\
307.01	0\\
308.01	0\\
309.01	0\\
310.01	0\\
311.01	0\\
312.01	0\\
313.01	0\\
314.01	0\\
315.01	0\\
316.01	0\\
317.01	0\\
318.01	0\\
319.01	0\\
320.01	0\\
321.01	0\\
322.01	0\\
323.01	0\\
324.01	0\\
325.01	0\\
326.01	0\\
327.01	0\\
328.01	0\\
329.01	0\\
330.01	0\\
331.01	0\\
332.01	0\\
333.01	0\\
334.01	0\\
335.01	0\\
336.01	0\\
337.01	0\\
338.01	0\\
339.01	0\\
340.01	0\\
341.01	0\\
342.01	0\\
343.01	0\\
344.01	0\\
345.01	0\\
346.01	0\\
347.01	0\\
348.01	0\\
349.01	0\\
350.01	0\\
351.01	0\\
352.01	0\\
353.01	0\\
354.01	0\\
355.01	0\\
356.01	0\\
357.01	0\\
358.01	0\\
359.01	0\\
360.01	0\\
361.01	0\\
362.01	0\\
363.01	0\\
364.01	0\\
365.01	0\\
366.01	0\\
367.01	0\\
368.01	0\\
369.01	0\\
370.01	0\\
371.01	0\\
372.01	0\\
373.01	0\\
374.01	0\\
375.01	0\\
376.01	0\\
377.01	0\\
378.01	0\\
379.01	0\\
380.01	0\\
381.01	0\\
382.01	0\\
383.01	0\\
384.01	0\\
385.01	0\\
386.01	0\\
387.01	0\\
388.01	0\\
389.01	0\\
390.01	0\\
391.01	0\\
392.01	0\\
393.01	0\\
394.01	0\\
395.01	0\\
396.01	0\\
397.01	0\\
398.01	0\\
399.01	0\\
400.01	0\\
401.01	0\\
402.01	0\\
403.01	0\\
404.01	0\\
405.01	0\\
406.01	0\\
407.01	0\\
408.01	0\\
409.01	0\\
410.01	0\\
411.01	0\\
412.01	0\\
413.01	0\\
414.01	0\\
415.01	0\\
416.01	0\\
417.01	0\\
418.01	0\\
419.01	0\\
420.01	0\\
421.01	0\\
422.01	0\\
423.01	0\\
424.01	0\\
425.01	0\\
426.01	0\\
427.01	0\\
428.01	0\\
429.01	0\\
430.01	0\\
431.01	0\\
432.01	0\\
433.01	0\\
434.01	0\\
435.01	0\\
436.01	0\\
437.01	0\\
438.01	0\\
439.01	0\\
440.01	0\\
441.01	0\\
442.01	0\\
443.01	0\\
444.01	0\\
445.01	0\\
446.01	0\\
447.01	0\\
448.01	0\\
449.01	0\\
450.01	0\\
451.01	0\\
452.01	0\\
453.01	0\\
454.01	0\\
455.01	0\\
456.01	0\\
457.01	0\\
458.01	0\\
459.01	0\\
460.01	0\\
461.01	0\\
462.01	0\\
463.01	0\\
464.01	0\\
465.01	0\\
466.01	0\\
467.01	0\\
468.01	0\\
469.01	0\\
470.01	0\\
471.01	0\\
472.01	0\\
473.01	0\\
474.01	0\\
475.01	0\\
476.01	0\\
477.01	0\\
478.01	0\\
479.01	0\\
480.01	0\\
481.01	0\\
482.01	0\\
483.01	0\\
484.01	0\\
485.01	0\\
486.01	0\\
487.01	0\\
488.01	0\\
489.01	0\\
490.01	0\\
491.01	0\\
492.01	0\\
493.01	0\\
494.01	0\\
495.01	0\\
496.01	0\\
497.01	0\\
498.01	0\\
499.01	0\\
500.01	0\\
501.01	0\\
502.01	0\\
503.01	0\\
504.01	0\\
505.01	0\\
506.01	0\\
507.01	0\\
508.01	0\\
509.01	0\\
510.01	0\\
511.01	0\\
512.01	0\\
513.01	0\\
514.01	0\\
515.01	0\\
516.01	0\\
517.01	0\\
518.01	0\\
519.01	0\\
520.01	0\\
521.01	0\\
522.01	0\\
523.01	0\\
524.01	0\\
525.01	0\\
526.01	0\\
527.01	0\\
528.01	0\\
529.01	0\\
530.01	0\\
531.01	0\\
532.01	0\\
533.01	0\\
534.01	0\\
535.01	0\\
536.01	0\\
537.01	0\\
538.01	0\\
539.01	0\\
540.01	0\\
541.01	0\\
542.01	0\\
543.01	0\\
544.01	0\\
545.01	0\\
546.01	0\\
547.01	0\\
548.01	0\\
549.01	0\\
550.01	0\\
551.01	0\\
552.01	0\\
553.01	0\\
554.01	0\\
555.01	0\\
556.01	0\\
557.01	0\\
558.01	0\\
559.01	0\\
560.01	0\\
561.01	0\\
562.01	0\\
563.01	0\\
564.01	0\\
565.01	0\\
566.01	0\\
567.01	0\\
568.01	0\\
569.01	0\\
570.01	0\\
571.01	0\\
572.01	0\\
573.01	0\\
574.01	0\\
575.01	0\\
576.01	0\\
577.01	0\\
578.01	0\\
579.01	0\\
580.01	0\\
581.01	0\\
582.01	0\\
583.01	0\\
584.01	0\\
585.01	0\\
586.01	0\\
587.01	0\\
588.01	0\\
589.01	0\\
590.01	0\\
591.01	0\\
592.01	0\\
593.01	0\\
594.01	0\\
595.01	0\\
596.01	0\\
597.01	0\\
598.01	3.69463863061463e-05\\
599.01	0.00379444900894968\\
599.02	0.00383270374370267\\
599.03	0.00387131777133739\\
599.04	0.00391029453419965\\
599.05	0.0039496375063856\\
599.06	0.003989350193997\\
599.07	0.00402943613539713\\
599.08	0.00406989890146739\\
599.09	0.00411074209586436\\
599.1	0.00415196935527735\\
599.11	0.00419358434968641\\
599.12	0.00423559078262063\\
599.13	0.00427799239141662\\
599.14	0.00432079294747728\\
599.15	0.00436399625653049\\
599.16	0.00440760615888791\\
599.17	0.00445162652970357\\
599.18	0.00449606127923222\\
599.19	0.00454091435308739\\
599.2	0.00458618973249894\\
599.21	0.00463189143457006\\
599.22	0.00467802351253352\\
599.23	0.00472459005600711\\
599.24	0.00477159519124805\\
599.25	0.00481904308140627\\
599.26	0.00486693792678842\\
599.27	0.00491528396512382\\
599.28	0.00496408547181374\\
599.29	0.00501334676017881\\
599.3	0.00506307218170441\\
599.31	0.00511326612628387\\
599.32	0.00516393302245909\\
599.33	0.00521507733765866\\
599.34	0.00526670357843294\\
599.35	0.00531881629068601\\
599.36	0.00537142005990418\\
599.37	0.00542451951138086\\
599.38	0.00547811931043734\\
599.39	0.00553222416263933\\
599.4	0.00558683881400888\\
599.41	0.00564196805123132\\
599.42	0.00569761670185686\\
599.43	0.00575378963449661\\
599.44	0.00581049175901236\\
599.45	0.00586772802670002\\
599.46	0.00592550342982103\\
599.47	0.00598382297386091\\
599.48	0.00604269170566519\\
599.49	0.00610211471358306\\
599.5	0.00616209712760199\\
599.51	0.00622264411947269\\
599.52	0.00628376090282387\\
599.53	0.00634545273326615\\
599.54	0.00640772490848447\\
599.55	0.00647058276831837\\
599.56	0.00653403169482929\\
599.57	0.00659807711235419\\
599.58	0.00666272448754477\\
599.59	0.00672797932939122\\
599.6	0.00679384718922988\\
599.61	0.0068603336607336\\
599.62	0.00692744437988406\\
599.63	0.00699518502492475\\
599.64	0.00706356131629375\\
599.65	0.007132579016535\\
599.66	0.00720224393018678\\
599.67	0.00727256190364638\\
599.68	0.00734353882500926\\
599.69	0.00741518062388145\\
599.7	0.00748749327116373\\
599.71	0.00756048277880581\\
599.72	0.00763415519952897\\
599.73	0.00770851662651528\\
599.74	0.00778357319306161\\
599.75	0.00785933107219639\\
599.76	0.00793579647625698\\
599.77	0.00801297565642555\\
599.78	0.00809087490222101\\
599.79	0.00816950054094457\\
599.8	0.00824885893707632\\
599.81	0.00832895649161999\\
599.82	0.00840979964139308\\
599.83	0.00849139485825914\\
599.84	0.00857374864829899\\
599.85	0.0086568675509174\\
599.86	0.00874075813788151\\
599.87	0.00882542701228713\\
599.88	0.00891088080744883\\
599.89	0.00899712618570936\\
599.9	0.00908416983716382\\
599.91	0.00917201847829369\\
599.92	0.00926067885050539\\
599.93	0.00935015771856803\\
599.94	0.00944046186894425\\
599.95	0.00953159810800815\\
599.96	0.00962357326014348\\
599.97	0.00971639416571523\\
599.98	0.00981006767890704\\
599.99	0.00990460066541651\\
600	0.01\\
};
\addplot [color=mycolor15,solid,forget plot]
  table[row sep=crcr]{%
0.01	0\\
1.01	0\\
2.01	0\\
3.01	0\\
4.01	0\\
5.01	0\\
6.01	0\\
7.01	0\\
8.01	0\\
9.01	0\\
10.01	0\\
11.01	0\\
12.01	0\\
13.01	0\\
14.01	0\\
15.01	0\\
16.01	0\\
17.01	0\\
18.01	0\\
19.01	0\\
20.01	0\\
21.01	0\\
22.01	0\\
23.01	0\\
24.01	0\\
25.01	0\\
26.01	0\\
27.01	0\\
28.01	0\\
29.01	0\\
30.01	0\\
31.01	0\\
32.01	0\\
33.01	0\\
34.01	0\\
35.01	0\\
36.01	0\\
37.01	0\\
38.01	0\\
39.01	0\\
40.01	0\\
41.01	0\\
42.01	0\\
43.01	0\\
44.01	0\\
45.01	0\\
46.01	0\\
47.01	0\\
48.01	0\\
49.01	0\\
50.01	0\\
51.01	0\\
52.01	0\\
53.01	0\\
54.01	0\\
55.01	0\\
56.01	0\\
57.01	0\\
58.01	0\\
59.01	0\\
60.01	0\\
61.01	0\\
62.01	0\\
63.01	0\\
64.01	0\\
65.01	0\\
66.01	0\\
67.01	0\\
68.01	0\\
69.01	0\\
70.01	0\\
71.01	0\\
72.01	0\\
73.01	0\\
74.01	0\\
75.01	0\\
76.01	0\\
77.01	0\\
78.01	0\\
79.01	0\\
80.01	0\\
81.01	0\\
82.01	0\\
83.01	0\\
84.01	0\\
85.01	0\\
86.01	0\\
87.01	0\\
88.01	0\\
89.01	0\\
90.01	0\\
91.01	0\\
92.01	0\\
93.01	0\\
94.01	0\\
95.01	0\\
96.01	0\\
97.01	0\\
98.01	0\\
99.01	0\\
100.01	0\\
101.01	0\\
102.01	0\\
103.01	0\\
104.01	0\\
105.01	0\\
106.01	0\\
107.01	0\\
108.01	0\\
109.01	0\\
110.01	0\\
111.01	0\\
112.01	0\\
113.01	0\\
114.01	0\\
115.01	0\\
116.01	0\\
117.01	0\\
118.01	0\\
119.01	0\\
120.01	0\\
121.01	0\\
122.01	0\\
123.01	0\\
124.01	0\\
125.01	0\\
126.01	0\\
127.01	0\\
128.01	0\\
129.01	0\\
130.01	0\\
131.01	0\\
132.01	0\\
133.01	0\\
134.01	0\\
135.01	0\\
136.01	0\\
137.01	0\\
138.01	0\\
139.01	0\\
140.01	0\\
141.01	0\\
142.01	0\\
143.01	0\\
144.01	0\\
145.01	0\\
146.01	0\\
147.01	0\\
148.01	0\\
149.01	0\\
150.01	0\\
151.01	0\\
152.01	0\\
153.01	0\\
154.01	0\\
155.01	0\\
156.01	0\\
157.01	0\\
158.01	0\\
159.01	0\\
160.01	0\\
161.01	0\\
162.01	0\\
163.01	0\\
164.01	0\\
165.01	0\\
166.01	0\\
167.01	0\\
168.01	0\\
169.01	0\\
170.01	0\\
171.01	0\\
172.01	0\\
173.01	0\\
174.01	0\\
175.01	0\\
176.01	0\\
177.01	0\\
178.01	0\\
179.01	0\\
180.01	0\\
181.01	0\\
182.01	0\\
183.01	0\\
184.01	0\\
185.01	0\\
186.01	0\\
187.01	0\\
188.01	0\\
189.01	0\\
190.01	0\\
191.01	0\\
192.01	0\\
193.01	0\\
194.01	0\\
195.01	0\\
196.01	0\\
197.01	0\\
198.01	0\\
199.01	0\\
200.01	0\\
201.01	0\\
202.01	0\\
203.01	0\\
204.01	0\\
205.01	0\\
206.01	0\\
207.01	0\\
208.01	0\\
209.01	0\\
210.01	0\\
211.01	0\\
212.01	0\\
213.01	0\\
214.01	0\\
215.01	0\\
216.01	0\\
217.01	0\\
218.01	0\\
219.01	0\\
220.01	0\\
221.01	0\\
222.01	0\\
223.01	0\\
224.01	0\\
225.01	0\\
226.01	0\\
227.01	0\\
228.01	0\\
229.01	0\\
230.01	0\\
231.01	0\\
232.01	0\\
233.01	0\\
234.01	0\\
235.01	0\\
236.01	0\\
237.01	0\\
238.01	0\\
239.01	0\\
240.01	0\\
241.01	0\\
242.01	0\\
243.01	0\\
244.01	0\\
245.01	0\\
246.01	0\\
247.01	0\\
248.01	0\\
249.01	0\\
250.01	0\\
251.01	0\\
252.01	0\\
253.01	0\\
254.01	0\\
255.01	0\\
256.01	0\\
257.01	0\\
258.01	0\\
259.01	0\\
260.01	0\\
261.01	0\\
262.01	0\\
263.01	0\\
264.01	0\\
265.01	0\\
266.01	0\\
267.01	0\\
268.01	0\\
269.01	0\\
270.01	0\\
271.01	0\\
272.01	0\\
273.01	0\\
274.01	0\\
275.01	0\\
276.01	0\\
277.01	0\\
278.01	0\\
279.01	0\\
280.01	0\\
281.01	0\\
282.01	0\\
283.01	0\\
284.01	0\\
285.01	0\\
286.01	0\\
287.01	0\\
288.01	0\\
289.01	0\\
290.01	0\\
291.01	0\\
292.01	0\\
293.01	0\\
294.01	0\\
295.01	0\\
296.01	0\\
297.01	0\\
298.01	0\\
299.01	0\\
300.01	0\\
301.01	0\\
302.01	0\\
303.01	0\\
304.01	0\\
305.01	0\\
306.01	0\\
307.01	0\\
308.01	0\\
309.01	0\\
310.01	0\\
311.01	0\\
312.01	0\\
313.01	0\\
314.01	0\\
315.01	0\\
316.01	0\\
317.01	0\\
318.01	0\\
319.01	0\\
320.01	0\\
321.01	0\\
322.01	0\\
323.01	0\\
324.01	0\\
325.01	0\\
326.01	0\\
327.01	0\\
328.01	0\\
329.01	0\\
330.01	0\\
331.01	0\\
332.01	0\\
333.01	0\\
334.01	0\\
335.01	0\\
336.01	0\\
337.01	0\\
338.01	0\\
339.01	0\\
340.01	0\\
341.01	0\\
342.01	0\\
343.01	0\\
344.01	0\\
345.01	0\\
346.01	0\\
347.01	0\\
348.01	0\\
349.01	0\\
350.01	0\\
351.01	0\\
352.01	0\\
353.01	0\\
354.01	0\\
355.01	0\\
356.01	0\\
357.01	0\\
358.01	0\\
359.01	0\\
360.01	0\\
361.01	0\\
362.01	0\\
363.01	0\\
364.01	0\\
365.01	0\\
366.01	0\\
367.01	0\\
368.01	0\\
369.01	0\\
370.01	0\\
371.01	0\\
372.01	0\\
373.01	0\\
374.01	0\\
375.01	0\\
376.01	0\\
377.01	0\\
378.01	0\\
379.01	0\\
380.01	0\\
381.01	0\\
382.01	0\\
383.01	0\\
384.01	0\\
385.01	0\\
386.01	0\\
387.01	0\\
388.01	0\\
389.01	0\\
390.01	0\\
391.01	0\\
392.01	0\\
393.01	0\\
394.01	0\\
395.01	0\\
396.01	0\\
397.01	0\\
398.01	0\\
399.01	0\\
400.01	0\\
401.01	0\\
402.01	0\\
403.01	0\\
404.01	0\\
405.01	0\\
406.01	0\\
407.01	0\\
408.01	0\\
409.01	0\\
410.01	0\\
411.01	0\\
412.01	0\\
413.01	0\\
414.01	0\\
415.01	0\\
416.01	0\\
417.01	0\\
418.01	0\\
419.01	0\\
420.01	0\\
421.01	0\\
422.01	0\\
423.01	0\\
424.01	0\\
425.01	0\\
426.01	0\\
427.01	0\\
428.01	0\\
429.01	0\\
430.01	0\\
431.01	0\\
432.01	0\\
433.01	0\\
434.01	0\\
435.01	0\\
436.01	0\\
437.01	0\\
438.01	0\\
439.01	0\\
440.01	0\\
441.01	0\\
442.01	0\\
443.01	0\\
444.01	0\\
445.01	0\\
446.01	0\\
447.01	0\\
448.01	0\\
449.01	0\\
450.01	0\\
451.01	0\\
452.01	0\\
453.01	0\\
454.01	0\\
455.01	0\\
456.01	0\\
457.01	0\\
458.01	0\\
459.01	0\\
460.01	0\\
461.01	0\\
462.01	0\\
463.01	0\\
464.01	0\\
465.01	0\\
466.01	0\\
467.01	0\\
468.01	0\\
469.01	0\\
470.01	0\\
471.01	0\\
472.01	0\\
473.01	0\\
474.01	0\\
475.01	0\\
476.01	0\\
477.01	0\\
478.01	0\\
479.01	0\\
480.01	0\\
481.01	0\\
482.01	0\\
483.01	0\\
484.01	0\\
485.01	0\\
486.01	0\\
487.01	0\\
488.01	0\\
489.01	0\\
490.01	0\\
491.01	0\\
492.01	0\\
493.01	0\\
494.01	0\\
495.01	0\\
496.01	0\\
497.01	0\\
498.01	0\\
499.01	0\\
500.01	0\\
501.01	0\\
502.01	0\\
503.01	0\\
504.01	0\\
505.01	0\\
506.01	0\\
507.01	0\\
508.01	0\\
509.01	0\\
510.01	0\\
511.01	0\\
512.01	0\\
513.01	0\\
514.01	0\\
515.01	0\\
516.01	0\\
517.01	0\\
518.01	0\\
519.01	0\\
520.01	0\\
521.01	0\\
522.01	0\\
523.01	0\\
524.01	0\\
525.01	0\\
526.01	0\\
527.01	0\\
528.01	0\\
529.01	0\\
530.01	0\\
531.01	0\\
532.01	0\\
533.01	0\\
534.01	0\\
535.01	0\\
536.01	0\\
537.01	0\\
538.01	0\\
539.01	0\\
540.01	0\\
541.01	0\\
542.01	0\\
543.01	0\\
544.01	0\\
545.01	0\\
546.01	0\\
547.01	0\\
548.01	0\\
549.01	0\\
550.01	0\\
551.01	0\\
552.01	0\\
553.01	0\\
554.01	0\\
555.01	0\\
556.01	0\\
557.01	0\\
558.01	0\\
559.01	0\\
560.01	0\\
561.01	0\\
562.01	0\\
563.01	0\\
564.01	0\\
565.01	0\\
566.01	0\\
567.01	0\\
568.01	0\\
569.01	0\\
570.01	0\\
571.01	0\\
572.01	0\\
573.01	0\\
574.01	0\\
575.01	0\\
576.01	0\\
577.01	0\\
578.01	0\\
579.01	0\\
580.01	0\\
581.01	0\\
582.01	0\\
583.01	0\\
584.01	0\\
585.01	0\\
586.01	0\\
587.01	0\\
588.01	0\\
589.01	0\\
590.01	0\\
591.01	0\\
592.01	0\\
593.01	0\\
594.01	0\\
595.01	0\\
596.01	0\\
597.01	0\\
598.01	0.00141368560759025\\
599.01	0.00385460611783226\\
599.02	0.00389217668082419\\
599.03	0.00393010478508591\\
599.04	0.00396839388118488\\
599.05	0.00400704745291263\\
599.06	0.00404606901760485\\
599.07	0.00408546212646469\\
599.08	0.00412523036488917\\
599.09	0.00416537735279883\\
599.1	0.00420590674497062\\
599.11	0.00424682223137403\\
599.12	0.0042881275375106\\
599.13	0.00432982642475674\\
599.14	0.0043719226907099\\
599.15	0.0044144201695383\\
599.16	0.00445732273233393\\
599.17	0.00450063428746929\\
599.18	0.00454435878095746\\
599.19	0.00458850019681592\\
599.2	0.00463306255743389\\
599.21	0.00467804992394345\\
599.22	0.0047234663965943\\
599.23	0.00476931611513231\\
599.24	0.00481560325918193\\
599.25	0.00486233204863238\\
599.26	0.00490950673609608\\
599.27	0.00495713160496067\\
599.28	0.0050052109796917\\
599.29	0.00505374922622871\\
599.3	0.00510275075238528\\
599.31	0.00515222000825318\\
599.32	0.00520216148661058\\
599.33	0.00525257972333446\\
599.34	0.00530347929781725\\
599.35	0.00535486483338777\\
599.36	0.00540674099773647\\
599.37	0.00545911250334512\\
599.38	0.00551198410792095\\
599.39	0.00556536061483532\\
599.4	0.00561924687356703\\
599.41	0.00567364778015027\\
599.42	0.00572856827762734\\
599.43	0.00578401335650615\\
599.44	0.00583998805522266\\
599.45	0.00589649746060833\\
599.46	0.00595354670836325\\
599.47	0.00601114098356581\\
599.48	0.00606928552115735\\
599.49	0.00612798560643227\\
599.5	0.00618724657553358\\
599.51	0.00624707381595411\\
599.52	0.00630747276704345\\
599.53	0.00636844892052068\\
599.54	0.00643000782099308\\
599.55	0.00649215506648088\\
599.56	0.00655489630894826\\
599.57	0.00661823725484062\\
599.58	0.00668218366562836\\
599.59	0.00674674135835722\\
599.6	0.00681191620620549\\
599.61	0.006877714139048\\
599.62	0.00694414114402729\\
599.63	0.00701120326613206\\
599.64	0.00707890660878295\\
599.65	0.00714725733442608\\
599.66	0.00721626166513437\\
599.67	0.0072859258832169\\
599.68	0.00735625633183653\\
599.69	0.00742725941563611\\
599.7	0.00749894160137328\\
599.71	0.00757130941856446\\
599.72	0.00764436946013798\\
599.73	0.00771812838309687\\
599.74	0.00779259290919147\\
599.75	0.00786776982560227\\
599.76	0.0079436659856333\\
599.77	0.00802028830941633\\
599.78	0.00809764378462653\\
599.79	0.00817573946720956\\
599.8	0.00825458248212104\\
599.81	0.00833418002407833\\
599.82	0.00841453935832557\\
599.83	0.00849566782141209\\
599.84	0.00857757282198503\\
599.85	0.00866026184159656\\
599.86	0.00874374243552642\\
599.87	0.00882802223362036\\
599.88	0.00891310894114515\\
599.89	0.00899901033966106\\
599.9	0.0090857342879123\\
599.91	0.00917328872273657\\
599.92	0.00926168165999438\\
599.93	0.00935092119551921\\
599.94	0.0094410155060894\\
599.95	0.009531972850423\\
599.96	0.00962380157019665\\
599.97	0.00971651009108966\\
599.98	0.0098101069238547\\
599.99	0.00990460066541651\\
600	0.01\\
};
\addplot [color=mycolor16,solid,forget plot]
  table[row sep=crcr]{%
0.01	0\\
1.01	0\\
2.01	0\\
3.01	0\\
4.01	0\\
5.01	0\\
6.01	0\\
7.01	0\\
8.01	0\\
9.01	0\\
10.01	0\\
11.01	0\\
12.01	0\\
13.01	0\\
14.01	0\\
15.01	0\\
16.01	0\\
17.01	0\\
18.01	0\\
19.01	0\\
20.01	0\\
21.01	0\\
22.01	0\\
23.01	0\\
24.01	0\\
25.01	0\\
26.01	0\\
27.01	0\\
28.01	0\\
29.01	0\\
30.01	0\\
31.01	0\\
32.01	0\\
33.01	0\\
34.01	0\\
35.01	0\\
36.01	0\\
37.01	0\\
38.01	0\\
39.01	0\\
40.01	0\\
41.01	0\\
42.01	0\\
43.01	0\\
44.01	0\\
45.01	0\\
46.01	0\\
47.01	0\\
48.01	0\\
49.01	0\\
50.01	0\\
51.01	0\\
52.01	0\\
53.01	0\\
54.01	0\\
55.01	0\\
56.01	0\\
57.01	0\\
58.01	0\\
59.01	0\\
60.01	0\\
61.01	0\\
62.01	0\\
63.01	0\\
64.01	0\\
65.01	0\\
66.01	0\\
67.01	0\\
68.01	0\\
69.01	0\\
70.01	0\\
71.01	0\\
72.01	0\\
73.01	0\\
74.01	0\\
75.01	0\\
76.01	0\\
77.01	0\\
78.01	0\\
79.01	0\\
80.01	0\\
81.01	0\\
82.01	0\\
83.01	0\\
84.01	0\\
85.01	0\\
86.01	0\\
87.01	0\\
88.01	0\\
89.01	0\\
90.01	0\\
91.01	0\\
92.01	0\\
93.01	0\\
94.01	0\\
95.01	0\\
96.01	0\\
97.01	0\\
98.01	0\\
99.01	0\\
100.01	0\\
101.01	0\\
102.01	0\\
103.01	0\\
104.01	0\\
105.01	0\\
106.01	0\\
107.01	0\\
108.01	0\\
109.01	0\\
110.01	0\\
111.01	0\\
112.01	0\\
113.01	0\\
114.01	0\\
115.01	0\\
116.01	0\\
117.01	0\\
118.01	0\\
119.01	0\\
120.01	0\\
121.01	0\\
122.01	0\\
123.01	0\\
124.01	0\\
125.01	0\\
126.01	0\\
127.01	0\\
128.01	0\\
129.01	0\\
130.01	0\\
131.01	0\\
132.01	0\\
133.01	0\\
134.01	0\\
135.01	0\\
136.01	0\\
137.01	0\\
138.01	0\\
139.01	0\\
140.01	0\\
141.01	0\\
142.01	0\\
143.01	0\\
144.01	0\\
145.01	0\\
146.01	0\\
147.01	0\\
148.01	0\\
149.01	0\\
150.01	0\\
151.01	0\\
152.01	0\\
153.01	0\\
154.01	0\\
155.01	0\\
156.01	0\\
157.01	0\\
158.01	0\\
159.01	0\\
160.01	0\\
161.01	0\\
162.01	0\\
163.01	0\\
164.01	0\\
165.01	0\\
166.01	0\\
167.01	0\\
168.01	0\\
169.01	0\\
170.01	0\\
171.01	0\\
172.01	0\\
173.01	0\\
174.01	0\\
175.01	0\\
176.01	0\\
177.01	0\\
178.01	0\\
179.01	0\\
180.01	0\\
181.01	0\\
182.01	0\\
183.01	0\\
184.01	0\\
185.01	0\\
186.01	0\\
187.01	0\\
188.01	0\\
189.01	0\\
190.01	0\\
191.01	0\\
192.01	0\\
193.01	0\\
194.01	0\\
195.01	0\\
196.01	0\\
197.01	0\\
198.01	0\\
199.01	0\\
200.01	0\\
201.01	0\\
202.01	0\\
203.01	0\\
204.01	0\\
205.01	0\\
206.01	0\\
207.01	0\\
208.01	0\\
209.01	0\\
210.01	0\\
211.01	0\\
212.01	0\\
213.01	0\\
214.01	0\\
215.01	0\\
216.01	0\\
217.01	0\\
218.01	0\\
219.01	0\\
220.01	0\\
221.01	0\\
222.01	0\\
223.01	0\\
224.01	0\\
225.01	0\\
226.01	0\\
227.01	0\\
228.01	0\\
229.01	0\\
230.01	0\\
231.01	0\\
232.01	0\\
233.01	0\\
234.01	0\\
235.01	0\\
236.01	0\\
237.01	0\\
238.01	0\\
239.01	0\\
240.01	0\\
241.01	0\\
242.01	0\\
243.01	0\\
244.01	0\\
245.01	0\\
246.01	0\\
247.01	0\\
248.01	0\\
249.01	0\\
250.01	0\\
251.01	0\\
252.01	0\\
253.01	0\\
254.01	0\\
255.01	0\\
256.01	0\\
257.01	0\\
258.01	0\\
259.01	0\\
260.01	0\\
261.01	0\\
262.01	0\\
263.01	0\\
264.01	0\\
265.01	0\\
266.01	0\\
267.01	0\\
268.01	0\\
269.01	0\\
270.01	0\\
271.01	0\\
272.01	0\\
273.01	0\\
274.01	0\\
275.01	0\\
276.01	0\\
277.01	0\\
278.01	0\\
279.01	0\\
280.01	0\\
281.01	0\\
282.01	0\\
283.01	0\\
284.01	0\\
285.01	0\\
286.01	0\\
287.01	0\\
288.01	0\\
289.01	0\\
290.01	0\\
291.01	0\\
292.01	0\\
293.01	0\\
294.01	0\\
295.01	0\\
296.01	0\\
297.01	0\\
298.01	0\\
299.01	0\\
300.01	0\\
301.01	0\\
302.01	0\\
303.01	0\\
304.01	0\\
305.01	0\\
306.01	0\\
307.01	0\\
308.01	0\\
309.01	0\\
310.01	0\\
311.01	0\\
312.01	0\\
313.01	0\\
314.01	0\\
315.01	0\\
316.01	0\\
317.01	0\\
318.01	0\\
319.01	0\\
320.01	0\\
321.01	0\\
322.01	0\\
323.01	0\\
324.01	0\\
325.01	0\\
326.01	0\\
327.01	0\\
328.01	0\\
329.01	0\\
330.01	0\\
331.01	0\\
332.01	0\\
333.01	0\\
334.01	0\\
335.01	0\\
336.01	0\\
337.01	0\\
338.01	0\\
339.01	0\\
340.01	0\\
341.01	0\\
342.01	0\\
343.01	0\\
344.01	0\\
345.01	0\\
346.01	0\\
347.01	0\\
348.01	0\\
349.01	0\\
350.01	0\\
351.01	0\\
352.01	0\\
353.01	0\\
354.01	0\\
355.01	0\\
356.01	0\\
357.01	0\\
358.01	0\\
359.01	0\\
360.01	0\\
361.01	0\\
362.01	0\\
363.01	0\\
364.01	0\\
365.01	0\\
366.01	0\\
367.01	0\\
368.01	0\\
369.01	0\\
370.01	0\\
371.01	0\\
372.01	0\\
373.01	0\\
374.01	0\\
375.01	0\\
376.01	0\\
377.01	0\\
378.01	0\\
379.01	0\\
380.01	0\\
381.01	0\\
382.01	0\\
383.01	0\\
384.01	0\\
385.01	0\\
386.01	0\\
387.01	0\\
388.01	0\\
389.01	0\\
390.01	0\\
391.01	0\\
392.01	0\\
393.01	0\\
394.01	0\\
395.01	0\\
396.01	0\\
397.01	0\\
398.01	0\\
399.01	0\\
400.01	0\\
401.01	0\\
402.01	0\\
403.01	0\\
404.01	0\\
405.01	0\\
406.01	0\\
407.01	0\\
408.01	0\\
409.01	0\\
410.01	0\\
411.01	0\\
412.01	0\\
413.01	0\\
414.01	0\\
415.01	0\\
416.01	0\\
417.01	0\\
418.01	0\\
419.01	0\\
420.01	0\\
421.01	0\\
422.01	0\\
423.01	0\\
424.01	0\\
425.01	0\\
426.01	0\\
427.01	0\\
428.01	0\\
429.01	0\\
430.01	0\\
431.01	0\\
432.01	0\\
433.01	0\\
434.01	0\\
435.01	0\\
436.01	0\\
437.01	0\\
438.01	0\\
439.01	0\\
440.01	0\\
441.01	0\\
442.01	0\\
443.01	0\\
444.01	0\\
445.01	0\\
446.01	0\\
447.01	0\\
448.01	0\\
449.01	0\\
450.01	0\\
451.01	0\\
452.01	0\\
453.01	0\\
454.01	0\\
455.01	0\\
456.01	0\\
457.01	0\\
458.01	0\\
459.01	0\\
460.01	0\\
461.01	0\\
462.01	0\\
463.01	0\\
464.01	0\\
465.01	0\\
466.01	0\\
467.01	0\\
468.01	0\\
469.01	0\\
470.01	0\\
471.01	0\\
472.01	0\\
473.01	0\\
474.01	0\\
475.01	0\\
476.01	0\\
477.01	0\\
478.01	0\\
479.01	0\\
480.01	0\\
481.01	0\\
482.01	0\\
483.01	0\\
484.01	0\\
485.01	0\\
486.01	0\\
487.01	0\\
488.01	0\\
489.01	0\\
490.01	0\\
491.01	0\\
492.01	0\\
493.01	0\\
494.01	0\\
495.01	0\\
496.01	0\\
497.01	0\\
498.01	0\\
499.01	0\\
500.01	0\\
501.01	0\\
502.01	0\\
503.01	0\\
504.01	0\\
505.01	0\\
506.01	0\\
507.01	0\\
508.01	0\\
509.01	0\\
510.01	0\\
511.01	0\\
512.01	0\\
513.01	0\\
514.01	0\\
515.01	0\\
516.01	0\\
517.01	0\\
518.01	0\\
519.01	0\\
520.01	0\\
521.01	0\\
522.01	0\\
523.01	0\\
524.01	0\\
525.01	0\\
526.01	0\\
527.01	0\\
528.01	0\\
529.01	0\\
530.01	0\\
531.01	0\\
532.01	0\\
533.01	0\\
534.01	0\\
535.01	0\\
536.01	0\\
537.01	0\\
538.01	0\\
539.01	0\\
540.01	0\\
541.01	0\\
542.01	0\\
543.01	0\\
544.01	0\\
545.01	0\\
546.01	0\\
547.01	0\\
548.01	0\\
549.01	0\\
550.01	0\\
551.01	0\\
552.01	0\\
553.01	0\\
554.01	0\\
555.01	0\\
556.01	0\\
557.01	0\\
558.01	0\\
559.01	0\\
560.01	0\\
561.01	0\\
562.01	0\\
563.01	0\\
564.01	0\\
565.01	0\\
566.01	0\\
567.01	0\\
568.01	0\\
569.01	0\\
570.01	0\\
571.01	0\\
572.01	0\\
573.01	0\\
574.01	0\\
575.01	0\\
576.01	0\\
577.01	0\\
578.01	0\\
579.01	0\\
580.01	0\\
581.01	0\\
582.01	0\\
583.01	0\\
584.01	0\\
585.01	0\\
586.01	0\\
587.01	0\\
588.01	0\\
589.01	0\\
590.01	0\\
591.01	0\\
592.01	0\\
593.01	0\\
594.01	0\\
595.01	0\\
596.01	0\\
597.01	0\\
598.01	0.0014374037248278\\
599.01	0.00385656036521221\\
599.02	0.00389409724555795\\
599.03	0.00393199176826388\\
599.04	0.00397024738727775\\
599.05	0.00400886758988391\\
599.06	0.00404785589702493\\
599.07	0.0040872158636263\\
599.08	0.00412695107892428\\
599.09	0.00416706516679696\\
599.1	0.00420756178609841\\
599.11	0.00424844463099618\\
599.12	0.00428971743131196\\
599.13	0.00433138395286554\\
599.14	0.00437344799782218\\
599.15	0.00441591340504319\\
599.16	0.00445878405044\\
599.17	0.00450206384733162\\
599.18	0.0045457567468055\\
599.19	0.00458986673808198\\
599.2	0.00463439784888212\\
599.21	0.00467935414579919\\
599.22	0.00472473973467372\\
599.23	0.00477055876097212\\
599.24	0.00481681541016898\\
599.25	0.0048635139081331\\
599.26	0.00491065851083179\\
599.27	0.00495825350811072\\
599.28	0.00500630323104035\\
599.29	0.00505481205231203\\
599.3	0.00510378438663774\\
599.31	0.00515322469115379\\
599.32	0.0052031374658283\\
599.33	0.00525352725387272\\
599.34	0.00530439864215714\\
599.35	0.00535575626162978\\
599.36	0.00540760478774036\\
599.37	0.00545994894086756\\
599.38	0.00551279348675071\\
599.39	0.00556614323692543\\
599.4	0.00562000304916365\\
599.41	0.00567437782791768\\
599.42	0.00572927252476872\\
599.43	0.00578469213887949\\
599.44	0.00584064171745141\\
599.45	0.00589712635618597\\
599.46	0.00595415119975069\\
599.47	0.00601172144224942\\
599.48	0.00606984232769716\\
599.49	0.00612851915049951\\
599.5	0.00618775725593663\\
599.51	0.00624756204065183\\
599.52	0.00630793895314495\\
599.53	0.00636889349427038\\
599.54	0.00643043121773979\\
599.55	0.00649255773062987\\
599.56	0.00655527869389471\\
599.57	0.00661859982288322\\
599.58	0.00668252688786147\\
599.59	0.00674706571453995\\
599.6	0.00681222218460597\\
599.61	0.00687800223626108\\
599.62	0.00694441186476363\\
599.63	0.00701145712297651\\
599.64	0.00707914412192007\\
599.65	0.00714747903133038\\
599.66	0.00721646808022274\\
599.67	0.00728611755746052\\
599.68	0.00735643381232947\\
599.69	0.00742742325511746\\
599.7	0.00749909235769963\\
599.71	0.00757144765412921\\
599.72	0.00764449574123379\\
599.73	0.00771824327921728\\
599.74	0.00779269699226756\\
599.75	0.00786786366916977\\
599.76	0.00794375016392541\\
599.77	0.0080203633963772\\
599.78	0.00809771035283981\\
599.79	0.00817579808673647\\
599.8	0.00825463371924141\\
599.81	0.00833422443992839\\
599.82	0.00841457750742512\\
599.83	0.00849570025007373\\
599.84	0.00857760006659731\\
599.85	0.00866028442677256\\
599.86	0.0087437608721085\\
599.87	0.00882803701653141\\
599.88	0.00891312054707585\\
599.89	0.00899901922458194\\
599.9	0.00908574088439886\\
599.91	0.00917329343709451\\
599.92	0.00926168486917145\\
599.93	0.00935092324378912\\
599.94	0.00944101670149225\\
599.95	0.00953197346094551\\
599.96	0.00962380181967443\\
599.97	0.00971651015481255\\
599.98	0.0098101069238547\\
599.99	0.00990460066541651\\
600	0.01\\
};
\addplot [color=mycolor17,solid,forget plot]
  table[row sep=crcr]{%
0.01	0\\
1.01	0\\
2.01	0\\
3.01	0\\
4.01	0\\
5.01	0\\
6.01	0\\
7.01	0\\
8.01	0\\
9.01	0\\
10.01	0\\
11.01	0\\
12.01	0\\
13.01	0\\
14.01	0\\
15.01	0\\
16.01	0\\
17.01	0\\
18.01	0\\
19.01	0\\
20.01	0\\
21.01	0\\
22.01	0\\
23.01	0\\
24.01	0\\
25.01	0\\
26.01	0\\
27.01	0\\
28.01	0\\
29.01	0\\
30.01	0\\
31.01	0\\
32.01	0\\
33.01	0\\
34.01	0\\
35.01	0\\
36.01	0\\
37.01	0\\
38.01	0\\
39.01	0\\
40.01	0\\
41.01	0\\
42.01	0\\
43.01	0\\
44.01	0\\
45.01	0\\
46.01	0\\
47.01	0\\
48.01	0\\
49.01	0\\
50.01	0\\
51.01	0\\
52.01	0\\
53.01	0\\
54.01	0\\
55.01	0\\
56.01	0\\
57.01	0\\
58.01	0\\
59.01	0\\
60.01	0\\
61.01	0\\
62.01	0\\
63.01	0\\
64.01	0\\
65.01	0\\
66.01	0\\
67.01	0\\
68.01	0\\
69.01	0\\
70.01	0\\
71.01	0\\
72.01	0\\
73.01	0\\
74.01	0\\
75.01	0\\
76.01	0\\
77.01	0\\
78.01	0\\
79.01	0\\
80.01	0\\
81.01	0\\
82.01	0\\
83.01	0\\
84.01	0\\
85.01	0\\
86.01	0\\
87.01	0\\
88.01	0\\
89.01	0\\
90.01	0\\
91.01	0\\
92.01	0\\
93.01	0\\
94.01	0\\
95.01	0\\
96.01	0\\
97.01	0\\
98.01	0\\
99.01	0\\
100.01	0\\
101.01	0\\
102.01	0\\
103.01	0\\
104.01	0\\
105.01	0\\
106.01	0\\
107.01	0\\
108.01	0\\
109.01	0\\
110.01	0\\
111.01	0\\
112.01	0\\
113.01	0\\
114.01	0\\
115.01	0\\
116.01	0\\
117.01	0\\
118.01	0\\
119.01	0\\
120.01	0\\
121.01	0\\
122.01	0\\
123.01	0\\
124.01	0\\
125.01	0\\
126.01	0\\
127.01	0\\
128.01	0\\
129.01	0\\
130.01	0\\
131.01	0\\
132.01	0\\
133.01	0\\
134.01	0\\
135.01	0\\
136.01	0\\
137.01	0\\
138.01	0\\
139.01	0\\
140.01	0\\
141.01	0\\
142.01	0\\
143.01	0\\
144.01	0\\
145.01	0\\
146.01	0\\
147.01	0\\
148.01	0\\
149.01	0\\
150.01	0\\
151.01	0\\
152.01	0\\
153.01	0\\
154.01	0\\
155.01	0\\
156.01	0\\
157.01	0\\
158.01	0\\
159.01	0\\
160.01	0\\
161.01	0\\
162.01	0\\
163.01	0\\
164.01	0\\
165.01	0\\
166.01	0\\
167.01	0\\
168.01	0\\
169.01	0\\
170.01	0\\
171.01	0\\
172.01	0\\
173.01	0\\
174.01	0\\
175.01	0\\
176.01	0\\
177.01	0\\
178.01	0\\
179.01	0\\
180.01	0\\
181.01	0\\
182.01	0\\
183.01	0\\
184.01	0\\
185.01	0\\
186.01	0\\
187.01	0\\
188.01	0\\
189.01	0\\
190.01	0\\
191.01	0\\
192.01	0\\
193.01	0\\
194.01	0\\
195.01	0\\
196.01	0\\
197.01	0\\
198.01	0\\
199.01	0\\
200.01	0\\
201.01	0\\
202.01	0\\
203.01	0\\
204.01	0\\
205.01	0\\
206.01	0\\
207.01	0\\
208.01	0\\
209.01	0\\
210.01	0\\
211.01	0\\
212.01	0\\
213.01	0\\
214.01	0\\
215.01	0\\
216.01	0\\
217.01	0\\
218.01	0\\
219.01	0\\
220.01	0\\
221.01	0\\
222.01	0\\
223.01	0\\
224.01	0\\
225.01	0\\
226.01	0\\
227.01	0\\
228.01	0\\
229.01	0\\
230.01	0\\
231.01	0\\
232.01	0\\
233.01	0\\
234.01	0\\
235.01	0\\
236.01	0\\
237.01	0\\
238.01	0\\
239.01	0\\
240.01	0\\
241.01	0\\
242.01	0\\
243.01	0\\
244.01	0\\
245.01	0\\
246.01	0\\
247.01	0\\
248.01	0\\
249.01	0\\
250.01	0\\
251.01	0\\
252.01	0\\
253.01	0\\
254.01	0\\
255.01	0\\
256.01	0\\
257.01	0\\
258.01	0\\
259.01	0\\
260.01	0\\
261.01	0\\
262.01	0\\
263.01	0\\
264.01	0\\
265.01	0\\
266.01	0\\
267.01	0\\
268.01	0\\
269.01	0\\
270.01	0\\
271.01	0\\
272.01	0\\
273.01	0\\
274.01	0\\
275.01	0\\
276.01	0\\
277.01	0\\
278.01	0\\
279.01	0\\
280.01	0\\
281.01	0\\
282.01	0\\
283.01	0\\
284.01	0\\
285.01	0\\
286.01	0\\
287.01	0\\
288.01	0\\
289.01	0\\
290.01	0\\
291.01	0\\
292.01	0\\
293.01	0\\
294.01	0\\
295.01	0\\
296.01	0\\
297.01	0\\
298.01	0\\
299.01	0\\
300.01	0\\
301.01	0\\
302.01	0\\
303.01	0\\
304.01	0\\
305.01	0\\
306.01	0\\
307.01	0\\
308.01	0\\
309.01	0\\
310.01	0\\
311.01	0\\
312.01	0\\
313.01	0\\
314.01	0\\
315.01	0\\
316.01	0\\
317.01	0\\
318.01	0\\
319.01	0\\
320.01	0\\
321.01	0\\
322.01	0\\
323.01	0\\
324.01	0\\
325.01	0\\
326.01	0\\
327.01	0\\
328.01	0\\
329.01	0\\
330.01	0\\
331.01	0\\
332.01	0\\
333.01	0\\
334.01	0\\
335.01	0\\
336.01	0\\
337.01	0\\
338.01	0\\
339.01	0\\
340.01	0\\
341.01	0\\
342.01	0\\
343.01	0\\
344.01	0\\
345.01	0\\
346.01	0\\
347.01	0\\
348.01	0\\
349.01	0\\
350.01	0\\
351.01	0\\
352.01	0\\
353.01	0\\
354.01	0\\
355.01	0\\
356.01	0\\
357.01	0\\
358.01	0\\
359.01	0\\
360.01	0\\
361.01	0\\
362.01	0\\
363.01	0\\
364.01	0\\
365.01	0\\
366.01	0\\
367.01	0\\
368.01	0\\
369.01	0\\
370.01	0\\
371.01	0\\
372.01	0\\
373.01	0\\
374.01	0\\
375.01	0\\
376.01	0\\
377.01	0\\
378.01	0\\
379.01	0\\
380.01	0\\
381.01	0\\
382.01	0\\
383.01	0\\
384.01	0\\
385.01	0\\
386.01	0\\
387.01	0\\
388.01	0\\
389.01	0\\
390.01	0\\
391.01	0\\
392.01	0\\
393.01	0\\
394.01	0\\
395.01	0\\
396.01	0\\
397.01	0\\
398.01	0\\
399.01	0\\
400.01	0\\
401.01	0\\
402.01	0\\
403.01	0\\
404.01	0\\
405.01	0\\
406.01	0\\
407.01	0\\
408.01	0\\
409.01	0\\
410.01	0\\
411.01	0\\
412.01	0\\
413.01	0\\
414.01	0\\
415.01	0\\
416.01	0\\
417.01	0\\
418.01	0\\
419.01	0\\
420.01	0\\
421.01	0\\
422.01	0\\
423.01	0\\
424.01	0\\
425.01	0\\
426.01	0\\
427.01	0\\
428.01	0\\
429.01	0\\
430.01	0\\
431.01	0\\
432.01	0\\
433.01	0\\
434.01	0\\
435.01	0\\
436.01	0\\
437.01	0\\
438.01	0\\
439.01	0\\
440.01	0\\
441.01	0\\
442.01	0\\
443.01	0\\
444.01	0\\
445.01	0\\
446.01	0\\
447.01	0\\
448.01	0\\
449.01	0\\
450.01	0\\
451.01	0\\
452.01	0\\
453.01	0\\
454.01	0\\
455.01	0\\
456.01	0\\
457.01	0\\
458.01	0\\
459.01	0\\
460.01	0\\
461.01	0\\
462.01	0\\
463.01	0\\
464.01	0\\
465.01	0\\
466.01	0\\
467.01	0\\
468.01	0\\
469.01	0\\
470.01	0\\
471.01	0\\
472.01	0\\
473.01	0\\
474.01	0\\
475.01	0\\
476.01	0\\
477.01	0\\
478.01	0\\
479.01	0\\
480.01	0\\
481.01	0\\
482.01	0\\
483.01	0\\
484.01	0\\
485.01	0\\
486.01	0\\
487.01	0\\
488.01	0\\
489.01	0\\
490.01	0\\
491.01	0\\
492.01	0\\
493.01	0\\
494.01	0\\
495.01	0\\
496.01	0\\
497.01	0\\
498.01	0\\
499.01	0\\
500.01	0\\
501.01	0\\
502.01	0\\
503.01	0\\
504.01	0\\
505.01	0\\
506.01	0\\
507.01	0\\
508.01	0\\
509.01	0\\
510.01	0\\
511.01	0\\
512.01	0\\
513.01	0\\
514.01	0\\
515.01	0\\
516.01	0\\
517.01	0\\
518.01	0\\
519.01	0\\
520.01	0\\
521.01	0\\
522.01	0\\
523.01	0\\
524.01	0\\
525.01	0\\
526.01	0\\
527.01	0\\
528.01	0\\
529.01	0\\
530.01	0\\
531.01	0\\
532.01	0\\
533.01	0\\
534.01	0\\
535.01	0\\
536.01	0\\
537.01	0\\
538.01	0\\
539.01	0\\
540.01	0\\
541.01	0\\
542.01	0\\
543.01	0\\
544.01	0\\
545.01	0\\
546.01	0\\
547.01	0\\
548.01	0\\
549.01	0\\
550.01	0\\
551.01	0\\
552.01	0\\
553.01	0\\
554.01	0\\
555.01	0\\
556.01	0\\
557.01	0\\
558.01	0\\
559.01	0\\
560.01	0\\
561.01	0\\
562.01	0\\
563.01	0\\
564.01	0\\
565.01	0\\
566.01	0\\
567.01	0\\
568.01	0\\
569.01	0\\
570.01	0\\
571.01	0\\
572.01	0\\
573.01	0\\
574.01	0\\
575.01	0\\
576.01	0\\
577.01	0\\
578.01	0\\
579.01	0\\
580.01	0\\
581.01	0\\
582.01	0\\
583.01	0\\
584.01	0\\
585.01	0\\
586.01	0\\
587.01	0\\
588.01	0\\
589.01	0\\
590.01	0\\
591.01	0\\
592.01	0\\
593.01	0\\
594.01	0\\
595.01	0\\
596.01	0\\
597.01	0\\
598.01	0.00143822749983459\\
599.01	0.00385661144683571\\
599.02	0.00389414709863901\\
599.03	0.00393204040529412\\
599.04	0.00397029482087115\\
599.05	0.00400891383277854\\
599.06	0.00404790096208447\\
599.07	0.00408725976384149\\
599.08	0.00412699382741433\\
599.09	0.0041671067768107\\
599.1	0.00420760227101545\\
599.11	0.00424848400432787\\
599.12	0.00428975570670221\\
599.13	0.00433142114409155\\
599.14	0.00437348411879493\\
599.15	0.00441594846980785\\
599.16	0.00445881807317617\\
599.17	0.00450209684235336\\
599.18	0.00454578872856121\\
599.19	0.00458989772115408\\
599.2	0.00463442784798657\\
599.21	0.00467938317578476\\
599.22	0.00472476781052109\\
599.23	0.00477058589779272\\
599.24	0.00481684162320371\\
599.25	0.00486353921275068\\
599.26	0.0049106829225159\\
599.27	0.00495827704247297\\
599.28	0.00500632590381774\\
599.29	0.00505483387936404\\
599.3	0.00510380538394311\\
599.31	0.00515324487480703\\
599.32	0.00520315685203588\\
599.33	0.00525354585894884\\
599.34	0.00530441648251928\\
599.35	0.00535577335379373\\
599.36	0.005407621148315\\
599.37	0.00545996458654921\\
599.38	0.00551280843431708\\
599.39	0.00556615750322915\\
599.4	0.00562001665112538\\
599.41	0.00567439078251885\\
599.42	0.00572928484904367\\
599.43	0.00578470384990735\\
599.44	0.00584065283234732\\
599.45	0.00589713689209195\\
599.46	0.00595416117382597\\
599.47	0.00601173087166026\\
599.48	0.00606985122960629\\
599.49	0.006128527542055\\
599.5	0.0061877651542603\\
599.51	0.00624756946282722\\
599.52	0.00630794591620473\\
599.53	0.00636890001518324\\
599.54	0.00643043731339698\\
599.55	0.00649256341783105\\
599.56	0.00655528398933347\\
599.57	0.00661860474313204\\
599.58	0.00668253144935614\\
599.59	0.00674706993356369\\
599.6	0.00681222607727294\\
599.61	0.0068780058184995\\
599.62	0.0069444151522985\\
599.63	0.00701146013131191\\
599.64	0.0070791468663211\\
599.65	0.00714748152680477\\
599.66	0.00721647034150212\\
599.67	0.00728611959898148\\
599.68	0.00735643564821442\\
599.69	0.00742742489915527\\
599.7	0.00749909382332627\\
599.71	0.00757144895440832\\
599.72	0.00764449688883733\\
599.73	0.00771824428640635\\
599.74	0.00779269787087345\\
599.75	0.00786786443057545\\
599.76	0.00794375081904746\\
599.77	0.0080203639556484\\
599.78	0.00809771082619259\\
599.79	0.00817579848358721\\
599.8	0.00825463404847609\\
599.81	0.0083342247098895\\
599.82	0.0084145777259002\\
599.83	0.00849570042428592\\
599.84	0.00857760020319794\\
599.85	0.00866028453183631\\
599.86	0.00874376095113144\\
599.87	0.00882803707443219\\
599.88	0.00891312058820066\\
599.89	0.00899901925271357\\
599.9	0.00908574090277037\\
599.91	0.00917329344840818\\
599.92	0.00926168487562357\\
599.93	0.00935092324710127\\
599.94	0.00944101670294984\\
599.95	0.00953197346144448\\
599.96	0.00962380181977693\\
599.97	0.00971651015481255\\
599.98	0.0098101069238547\\
599.99	0.00990460066541651\\
600	0.01\\
};
\addplot [color=mycolor18,solid,forget plot]
  table[row sep=crcr]{%
0.01	0\\
1.01	0\\
2.01	0\\
3.01	0\\
4.01	0\\
5.01	0\\
6.01	0\\
7.01	0\\
8.01	0\\
9.01	0\\
10.01	0\\
11.01	0\\
12.01	0\\
13.01	0\\
14.01	0\\
15.01	0\\
16.01	0\\
17.01	0\\
18.01	0\\
19.01	0\\
20.01	0\\
21.01	0\\
22.01	0\\
23.01	0\\
24.01	0\\
25.01	0\\
26.01	0\\
27.01	0\\
28.01	0\\
29.01	0\\
30.01	0\\
31.01	0\\
32.01	0\\
33.01	0\\
34.01	0\\
35.01	0\\
36.01	0\\
37.01	0\\
38.01	0\\
39.01	0\\
40.01	0\\
41.01	0\\
42.01	0\\
43.01	0\\
44.01	0\\
45.01	0\\
46.01	0\\
47.01	0\\
48.01	0\\
49.01	0\\
50.01	0\\
51.01	0\\
52.01	0\\
53.01	0\\
54.01	0\\
55.01	0\\
56.01	0\\
57.01	0\\
58.01	0\\
59.01	0\\
60.01	0\\
61.01	0\\
62.01	0\\
63.01	0\\
64.01	0\\
65.01	0\\
66.01	0\\
67.01	0\\
68.01	0\\
69.01	0\\
70.01	0\\
71.01	0\\
72.01	0\\
73.01	0\\
74.01	0\\
75.01	0\\
76.01	0\\
77.01	0\\
78.01	0\\
79.01	0\\
80.01	0\\
81.01	0\\
82.01	0\\
83.01	0\\
84.01	0\\
85.01	0\\
86.01	0\\
87.01	0\\
88.01	0\\
89.01	0\\
90.01	0\\
91.01	0\\
92.01	0\\
93.01	0\\
94.01	0\\
95.01	0\\
96.01	0\\
97.01	0\\
98.01	0\\
99.01	0\\
100.01	0\\
101.01	0\\
102.01	0\\
103.01	0\\
104.01	0\\
105.01	0\\
106.01	0\\
107.01	0\\
108.01	0\\
109.01	0\\
110.01	0\\
111.01	0\\
112.01	0\\
113.01	0\\
114.01	0\\
115.01	0\\
116.01	0\\
117.01	0\\
118.01	0\\
119.01	0\\
120.01	0\\
121.01	0\\
122.01	0\\
123.01	0\\
124.01	0\\
125.01	0\\
126.01	0\\
127.01	0\\
128.01	0\\
129.01	0\\
130.01	0\\
131.01	0\\
132.01	0\\
133.01	0\\
134.01	0\\
135.01	0\\
136.01	0\\
137.01	0\\
138.01	0\\
139.01	0\\
140.01	0\\
141.01	0\\
142.01	0\\
143.01	0\\
144.01	0\\
145.01	0\\
146.01	0\\
147.01	0\\
148.01	0\\
149.01	0\\
150.01	0\\
151.01	0\\
152.01	0\\
153.01	0\\
154.01	0\\
155.01	0\\
156.01	0\\
157.01	0\\
158.01	0\\
159.01	0\\
160.01	0\\
161.01	0\\
162.01	0\\
163.01	0\\
164.01	0\\
165.01	0\\
166.01	0\\
167.01	0\\
168.01	0\\
169.01	0\\
170.01	0\\
171.01	0\\
172.01	0\\
173.01	0\\
174.01	0\\
175.01	0\\
176.01	0\\
177.01	0\\
178.01	0\\
179.01	0\\
180.01	0\\
181.01	0\\
182.01	0\\
183.01	0\\
184.01	0\\
185.01	0\\
186.01	0\\
187.01	0\\
188.01	0\\
189.01	0\\
190.01	0\\
191.01	0\\
192.01	0\\
193.01	0\\
194.01	0\\
195.01	0\\
196.01	0\\
197.01	0\\
198.01	0\\
199.01	0\\
200.01	0\\
201.01	0\\
202.01	0\\
203.01	0\\
204.01	0\\
205.01	0\\
206.01	0\\
207.01	0\\
208.01	0\\
209.01	0\\
210.01	0\\
211.01	0\\
212.01	0\\
213.01	0\\
214.01	0\\
215.01	0\\
216.01	0\\
217.01	0\\
218.01	0\\
219.01	0\\
220.01	0\\
221.01	0\\
222.01	0\\
223.01	0\\
224.01	0\\
225.01	0\\
226.01	0\\
227.01	0\\
228.01	0\\
229.01	0\\
230.01	0\\
231.01	0\\
232.01	0\\
233.01	0\\
234.01	0\\
235.01	0\\
236.01	0\\
237.01	0\\
238.01	0\\
239.01	0\\
240.01	0\\
241.01	0\\
242.01	0\\
243.01	0\\
244.01	0\\
245.01	0\\
246.01	0\\
247.01	0\\
248.01	0\\
249.01	0\\
250.01	0\\
251.01	0\\
252.01	0\\
253.01	0\\
254.01	0\\
255.01	0\\
256.01	0\\
257.01	0\\
258.01	0\\
259.01	0\\
260.01	0\\
261.01	0\\
262.01	0\\
263.01	0\\
264.01	0\\
265.01	0\\
266.01	0\\
267.01	0\\
268.01	0\\
269.01	0\\
270.01	0\\
271.01	0\\
272.01	0\\
273.01	0\\
274.01	0\\
275.01	0\\
276.01	0\\
277.01	0\\
278.01	0\\
279.01	0\\
280.01	0\\
281.01	0\\
282.01	0\\
283.01	0\\
284.01	0\\
285.01	0\\
286.01	0\\
287.01	0\\
288.01	0\\
289.01	0\\
290.01	0\\
291.01	0\\
292.01	0\\
293.01	0\\
294.01	0\\
295.01	0\\
296.01	0\\
297.01	0\\
298.01	0\\
299.01	0\\
300.01	0\\
301.01	0\\
302.01	0\\
303.01	0\\
304.01	0\\
305.01	0\\
306.01	0\\
307.01	0\\
308.01	0\\
309.01	0\\
310.01	0\\
311.01	0\\
312.01	0\\
313.01	0\\
314.01	0\\
315.01	0\\
316.01	0\\
317.01	0\\
318.01	0\\
319.01	0\\
320.01	0\\
321.01	0\\
322.01	0\\
323.01	0\\
324.01	0\\
325.01	0\\
326.01	0\\
327.01	0\\
328.01	0\\
329.01	0\\
330.01	0\\
331.01	0\\
332.01	0\\
333.01	0\\
334.01	0\\
335.01	0\\
336.01	0\\
337.01	0\\
338.01	0\\
339.01	0\\
340.01	0\\
341.01	0\\
342.01	0\\
343.01	0\\
344.01	0\\
345.01	0\\
346.01	0\\
347.01	0\\
348.01	0\\
349.01	0\\
350.01	0\\
351.01	0\\
352.01	0\\
353.01	0\\
354.01	0\\
355.01	0\\
356.01	0\\
357.01	0\\
358.01	0\\
359.01	0\\
360.01	0\\
361.01	0\\
362.01	0\\
363.01	0\\
364.01	0\\
365.01	0\\
366.01	0\\
367.01	0\\
368.01	0\\
369.01	0\\
370.01	0\\
371.01	0\\
372.01	0\\
373.01	0\\
374.01	0\\
375.01	0\\
376.01	0\\
377.01	0\\
378.01	0\\
379.01	0\\
380.01	0\\
381.01	0\\
382.01	0\\
383.01	0\\
384.01	0\\
385.01	0\\
386.01	0\\
387.01	0\\
388.01	0\\
389.01	0\\
390.01	0\\
391.01	0\\
392.01	0\\
393.01	0\\
394.01	0\\
395.01	0\\
396.01	0\\
397.01	0\\
398.01	0\\
399.01	0\\
400.01	0\\
401.01	0\\
402.01	0\\
403.01	0\\
404.01	0\\
405.01	0\\
406.01	0\\
407.01	0\\
408.01	0\\
409.01	0\\
410.01	0\\
411.01	0\\
412.01	0\\
413.01	0\\
414.01	0\\
415.01	0\\
416.01	0\\
417.01	0\\
418.01	0\\
419.01	0\\
420.01	0\\
421.01	0\\
422.01	0\\
423.01	0\\
424.01	0\\
425.01	0\\
426.01	0\\
427.01	0\\
428.01	0\\
429.01	0\\
430.01	0\\
431.01	0\\
432.01	0\\
433.01	0\\
434.01	0\\
435.01	0\\
436.01	0\\
437.01	0\\
438.01	0\\
439.01	0\\
440.01	0\\
441.01	0\\
442.01	0\\
443.01	0\\
444.01	0\\
445.01	0\\
446.01	0\\
447.01	0\\
448.01	0\\
449.01	0\\
450.01	0\\
451.01	0\\
452.01	0\\
453.01	0\\
454.01	0\\
455.01	0\\
456.01	0\\
457.01	0\\
458.01	0\\
459.01	0\\
460.01	0\\
461.01	0\\
462.01	0\\
463.01	0\\
464.01	0\\
465.01	0\\
466.01	0\\
467.01	0\\
468.01	0\\
469.01	0\\
470.01	0\\
471.01	0\\
472.01	0\\
473.01	0\\
474.01	0\\
475.01	0\\
476.01	0\\
477.01	0\\
478.01	0\\
479.01	0\\
480.01	0\\
481.01	0\\
482.01	0\\
483.01	0\\
484.01	0\\
485.01	0\\
486.01	0\\
487.01	0\\
488.01	0\\
489.01	0\\
490.01	0\\
491.01	0\\
492.01	0\\
493.01	0\\
494.01	0\\
495.01	0\\
496.01	0\\
497.01	0\\
498.01	0\\
499.01	0\\
500.01	0\\
501.01	0\\
502.01	0\\
503.01	0\\
504.01	0\\
505.01	0\\
506.01	0\\
507.01	0\\
508.01	0\\
509.01	0\\
510.01	0\\
511.01	0\\
512.01	0\\
513.01	0\\
514.01	0\\
515.01	0\\
516.01	0\\
517.01	0\\
518.01	0\\
519.01	0\\
520.01	0\\
521.01	0\\
522.01	0\\
523.01	0\\
524.01	0\\
525.01	0\\
526.01	0\\
527.01	0\\
528.01	0\\
529.01	0\\
530.01	0\\
531.01	0\\
532.01	0\\
533.01	0\\
534.01	0\\
535.01	0\\
536.01	0\\
537.01	0\\
538.01	0\\
539.01	0\\
540.01	0\\
541.01	0\\
542.01	0\\
543.01	0\\
544.01	0\\
545.01	0\\
546.01	0\\
547.01	0\\
548.01	0\\
549.01	0\\
550.01	0\\
551.01	0\\
552.01	0\\
553.01	0\\
554.01	0\\
555.01	0\\
556.01	0\\
557.01	0\\
558.01	0\\
559.01	0\\
560.01	0\\
561.01	0\\
562.01	0\\
563.01	0\\
564.01	0\\
565.01	0\\
566.01	0\\
567.01	0\\
568.01	0\\
569.01	0\\
570.01	0\\
571.01	0\\
572.01	0\\
573.01	0\\
574.01	0\\
575.01	0\\
576.01	0\\
577.01	0\\
578.01	0\\
579.01	0\\
580.01	0\\
581.01	0\\
582.01	0\\
583.01	0\\
584.01	0\\
585.01	0\\
586.01	0\\
587.01	0\\
588.01	0\\
589.01	0\\
590.01	0\\
591.01	0\\
592.01	0\\
593.01	0\\
594.01	0\\
595.01	0\\
596.01	0\\
597.01	0\\
598.01	0.00143832017857744\\
599.01	0.00385661255145558\\
599.02	0.00389414816850622\\
599.03	0.00393204144103766\\
599.04	0.00397029582311886\\
599.05	0.00400891480215704\\
599.06	0.00404790189921909\\
599.07	0.00408726066935627\\
599.08	0.00412699470193188\\
599.09	0.00416710762095217\\
599.1	0.00420760308540043\\
599.11	0.00424848478957431\\
599.12	0.00428975646342631\\
599.13	0.00433142187290768\\
599.14	0.00437348482031551\\
599.15	0.00441594914464327\\
599.16	0.00445881872193463\\
599.17	0.00450209746564076\\
599.18	0.00454578932698104\\
599.19	0.00458989829530726\\
599.2	0.0046344283984713\\
599.21	0.0046793837031964\\
599.22	0.00472476831545197\\
599.23	0.00477058638083201\\
599.24	0.00481684208493721\\
599.25	0.00486353965376069\\
599.26	0.00491068334338074\\
599.27	0.00495827744376709\\
599.28	0.00500632628611153\\
599.29	0.0050548342432236\\
599.3	0.00510380572993011\\
599.31	0.00515324520347841\\
599.32	0.00520315716394368\\
599.33	0.00525354615463998\\
599.34	0.00530441676253529\\
599.35	0.00535577361867056\\
599.36	0.00540762139858276\\
599.37	0.00545996482273196\\
599.38	0.00551280865693252\\
599.39	0.00556615771278843\\
599.4	0.00562001684813284\\
599.41	0.00567439096747174\\
599.42	0.00572928502243194\\
599.43	0.00578470401221333\\
599.44	0.00584065298404554\\
599.45	0.00589713703364883\\
599.46	0.00595416130569955\\
599.47	0.00601173099430001\\
599.48	0.00606985134345282\\
599.49	0.00612852764753978\\
599.5	0.00618776525180547\\
599.51	0.0062475695528453\\
599.52	0.0063079459990984\\
599.53	0.00636890009134512\\
599.54	0.00643043738320938\\
599.55	0.00649256348166578\\
599.56	0.0065552840475516\\
599.57	0.0066186047960837\\
599.58	0.00668253149738038\\
599.59	0.00674706997698825\\
599.6	0.00681222611641409\\
599.61	0.00687800585366193\\
599.62	0.00694441518377516\\
599.63	0.00701146015938389\\
599.64	0.00707914689125756\\
599.65	0.00714748154886281\\
599.66	0.00721647036092677\\
599.67	0.00728611961600567\\
599.68	0.00735643566305893\\
599.69	0.00742742491202878\\
599.7	0.00749909383442539\\
599.71	0.00757144896391765\\
599.72	0.00764449689692958\\
599.73	0.00771824429324247\\
599.74	0.0077926978766028\\
599.75	0.00786786443533595\\
599.76	0.00794375082296586\\
599.77	0.00802036395884057\\
599.78	0.00809771082876376\\
599.79	0.00817579848563237\\
599.8	0.00825463405008032\\
599.81	0.0083342247111284\\
599.82	0.00841457772684034\\
599.83	0.00849570042498529\\
599.84	0.00857760020370652\\
599.85	0.00866028453219658\\
599.86	0.00874376095137895\\
599.87	0.00882803707459619\\
599.88	0.00891312058830471\\
599.89	0.00899901925277615\\
599.9	0.00908574090280557\\
599.91	0.00917329344842634\\
599.92	0.0092616848756319\\
599.93	0.00935092324710449\\
599.94	0.00944101670295078\\
599.95	0.00953197346144465\\
599.96	0.00962380181977693\\
599.97	0.00971651015481255\\
599.98	0.0098101069238547\\
599.99	0.00990460066541651\\
600	0.01\\
};
\addplot [color=red!25!mycolor17,solid,forget plot]
  table[row sep=crcr]{%
0.01	0\\
1.01	0\\
2.01	0\\
3.01	0\\
4.01	0\\
5.01	0\\
6.01	0\\
7.01	0\\
8.01	0\\
9.01	0\\
10.01	0\\
11.01	0\\
12.01	0\\
13.01	0\\
14.01	0\\
15.01	0\\
16.01	0\\
17.01	0\\
18.01	0\\
19.01	0\\
20.01	0\\
21.01	0\\
22.01	0\\
23.01	0\\
24.01	0\\
25.01	0\\
26.01	0\\
27.01	0\\
28.01	0\\
29.01	0\\
30.01	0\\
31.01	0\\
32.01	0\\
33.01	0\\
34.01	0\\
35.01	0\\
36.01	0\\
37.01	0\\
38.01	0\\
39.01	0\\
40.01	0\\
41.01	0\\
42.01	0\\
43.01	0\\
44.01	0\\
45.01	0\\
46.01	0\\
47.01	0\\
48.01	0\\
49.01	0\\
50.01	0\\
51.01	0\\
52.01	0\\
53.01	0\\
54.01	0\\
55.01	0\\
56.01	0\\
57.01	0\\
58.01	0\\
59.01	0\\
60.01	0\\
61.01	0\\
62.01	0\\
63.01	0\\
64.01	0\\
65.01	0\\
66.01	0\\
67.01	0\\
68.01	0\\
69.01	0\\
70.01	0\\
71.01	0\\
72.01	0\\
73.01	0\\
74.01	0\\
75.01	0\\
76.01	0\\
77.01	0\\
78.01	0\\
79.01	0\\
80.01	0\\
81.01	0\\
82.01	0\\
83.01	0\\
84.01	0\\
85.01	0\\
86.01	0\\
87.01	0\\
88.01	0\\
89.01	0\\
90.01	0\\
91.01	0\\
92.01	0\\
93.01	0\\
94.01	0\\
95.01	0\\
96.01	0\\
97.01	0\\
98.01	0\\
99.01	0\\
100.01	0\\
101.01	0\\
102.01	0\\
103.01	0\\
104.01	0\\
105.01	0\\
106.01	0\\
107.01	0\\
108.01	0\\
109.01	0\\
110.01	0\\
111.01	0\\
112.01	0\\
113.01	0\\
114.01	0\\
115.01	0\\
116.01	0\\
117.01	0\\
118.01	0\\
119.01	0\\
120.01	0\\
121.01	0\\
122.01	0\\
123.01	0\\
124.01	0\\
125.01	0\\
126.01	0\\
127.01	0\\
128.01	0\\
129.01	0\\
130.01	0\\
131.01	0\\
132.01	0\\
133.01	0\\
134.01	0\\
135.01	0\\
136.01	0\\
137.01	0\\
138.01	0\\
139.01	0\\
140.01	0\\
141.01	0\\
142.01	0\\
143.01	0\\
144.01	0\\
145.01	0\\
146.01	0\\
147.01	0\\
148.01	0\\
149.01	0\\
150.01	0\\
151.01	0\\
152.01	0\\
153.01	0\\
154.01	0\\
155.01	0\\
156.01	0\\
157.01	0\\
158.01	0\\
159.01	0\\
160.01	0\\
161.01	0\\
162.01	0\\
163.01	0\\
164.01	0\\
165.01	0\\
166.01	0\\
167.01	0\\
168.01	0\\
169.01	0\\
170.01	0\\
171.01	0\\
172.01	0\\
173.01	0\\
174.01	0\\
175.01	0\\
176.01	0\\
177.01	0\\
178.01	0\\
179.01	0\\
180.01	0\\
181.01	0\\
182.01	0\\
183.01	0\\
184.01	0\\
185.01	0\\
186.01	0\\
187.01	0\\
188.01	0\\
189.01	0\\
190.01	0\\
191.01	0\\
192.01	0\\
193.01	0\\
194.01	0\\
195.01	0\\
196.01	0\\
197.01	0\\
198.01	0\\
199.01	0\\
200.01	0\\
201.01	0\\
202.01	0\\
203.01	0\\
204.01	0\\
205.01	0\\
206.01	0\\
207.01	0\\
208.01	0\\
209.01	0\\
210.01	0\\
211.01	0\\
212.01	0\\
213.01	0\\
214.01	0\\
215.01	0\\
216.01	0\\
217.01	0\\
218.01	0\\
219.01	0\\
220.01	0\\
221.01	0\\
222.01	0\\
223.01	0\\
224.01	0\\
225.01	0\\
226.01	0\\
227.01	0\\
228.01	0\\
229.01	0\\
230.01	0\\
231.01	0\\
232.01	0\\
233.01	0\\
234.01	0\\
235.01	0\\
236.01	0\\
237.01	0\\
238.01	0\\
239.01	0\\
240.01	0\\
241.01	0\\
242.01	0\\
243.01	0\\
244.01	0\\
245.01	0\\
246.01	0\\
247.01	0\\
248.01	0\\
249.01	0\\
250.01	0\\
251.01	0\\
252.01	0\\
253.01	0\\
254.01	0\\
255.01	0\\
256.01	0\\
257.01	0\\
258.01	0\\
259.01	0\\
260.01	0\\
261.01	0\\
262.01	0\\
263.01	0\\
264.01	0\\
265.01	0\\
266.01	0\\
267.01	0\\
268.01	0\\
269.01	0\\
270.01	0\\
271.01	0\\
272.01	0\\
273.01	0\\
274.01	0\\
275.01	0\\
276.01	0\\
277.01	0\\
278.01	0\\
279.01	0\\
280.01	0\\
281.01	0\\
282.01	0\\
283.01	0\\
284.01	0\\
285.01	0\\
286.01	0\\
287.01	0\\
288.01	0\\
289.01	0\\
290.01	0\\
291.01	0\\
292.01	0\\
293.01	0\\
294.01	0\\
295.01	0\\
296.01	0\\
297.01	0\\
298.01	0\\
299.01	0\\
300.01	0\\
301.01	0\\
302.01	0\\
303.01	0\\
304.01	0\\
305.01	0\\
306.01	0\\
307.01	0\\
308.01	0\\
309.01	0\\
310.01	0\\
311.01	0\\
312.01	0\\
313.01	0\\
314.01	0\\
315.01	0\\
316.01	0\\
317.01	0\\
318.01	0\\
319.01	0\\
320.01	0\\
321.01	0\\
322.01	0\\
323.01	0\\
324.01	0\\
325.01	0\\
326.01	0\\
327.01	0\\
328.01	0\\
329.01	0\\
330.01	0\\
331.01	0\\
332.01	0\\
333.01	0\\
334.01	0\\
335.01	0\\
336.01	0\\
337.01	0\\
338.01	0\\
339.01	0\\
340.01	0\\
341.01	0\\
342.01	0\\
343.01	0\\
344.01	0\\
345.01	0\\
346.01	0\\
347.01	0\\
348.01	0\\
349.01	0\\
350.01	0\\
351.01	0\\
352.01	0\\
353.01	0\\
354.01	0\\
355.01	0\\
356.01	0\\
357.01	0\\
358.01	0\\
359.01	0\\
360.01	0\\
361.01	0\\
362.01	0\\
363.01	0\\
364.01	0\\
365.01	0\\
366.01	0\\
367.01	0\\
368.01	0\\
369.01	0\\
370.01	0\\
371.01	0\\
372.01	0\\
373.01	0\\
374.01	0\\
375.01	0\\
376.01	0\\
377.01	0\\
378.01	0\\
379.01	0\\
380.01	0\\
381.01	0\\
382.01	0\\
383.01	0\\
384.01	0\\
385.01	0\\
386.01	0\\
387.01	0\\
388.01	0\\
389.01	0\\
390.01	0\\
391.01	0\\
392.01	0\\
393.01	0\\
394.01	0\\
395.01	0\\
396.01	0\\
397.01	0\\
398.01	0\\
399.01	0\\
400.01	0\\
401.01	0\\
402.01	0\\
403.01	0\\
404.01	0\\
405.01	0\\
406.01	0\\
407.01	0\\
408.01	0\\
409.01	0\\
410.01	0\\
411.01	0\\
412.01	0\\
413.01	0\\
414.01	0\\
415.01	0\\
416.01	0\\
417.01	0\\
418.01	0\\
419.01	0\\
420.01	0\\
421.01	0\\
422.01	0\\
423.01	0\\
424.01	0\\
425.01	0\\
426.01	0\\
427.01	0\\
428.01	0\\
429.01	0\\
430.01	0\\
431.01	0\\
432.01	0\\
433.01	0\\
434.01	0\\
435.01	0\\
436.01	0\\
437.01	0\\
438.01	0\\
439.01	0\\
440.01	0\\
441.01	0\\
442.01	0\\
443.01	0\\
444.01	0\\
445.01	0\\
446.01	0\\
447.01	0\\
448.01	0\\
449.01	0\\
450.01	0\\
451.01	0\\
452.01	0\\
453.01	0\\
454.01	0\\
455.01	0\\
456.01	0\\
457.01	0\\
458.01	0\\
459.01	0\\
460.01	0\\
461.01	0\\
462.01	0\\
463.01	0\\
464.01	0\\
465.01	0\\
466.01	0\\
467.01	0\\
468.01	0\\
469.01	0\\
470.01	0\\
471.01	0\\
472.01	0\\
473.01	0\\
474.01	0\\
475.01	0\\
476.01	0\\
477.01	0\\
478.01	0\\
479.01	0\\
480.01	0\\
481.01	0\\
482.01	0\\
483.01	0\\
484.01	0\\
485.01	0\\
486.01	0\\
487.01	0\\
488.01	0\\
489.01	0\\
490.01	0\\
491.01	0\\
492.01	0\\
493.01	0\\
494.01	0\\
495.01	0\\
496.01	0\\
497.01	0\\
498.01	0\\
499.01	0\\
500.01	0\\
501.01	0\\
502.01	0\\
503.01	0\\
504.01	0\\
505.01	0\\
506.01	0\\
507.01	0\\
508.01	0\\
509.01	0\\
510.01	0\\
511.01	0\\
512.01	0\\
513.01	0\\
514.01	0\\
515.01	0\\
516.01	0\\
517.01	0\\
518.01	0\\
519.01	0\\
520.01	0\\
521.01	0\\
522.01	0\\
523.01	0\\
524.01	0\\
525.01	0\\
526.01	0\\
527.01	0\\
528.01	0\\
529.01	0\\
530.01	0\\
531.01	0\\
532.01	0\\
533.01	0\\
534.01	0\\
535.01	0\\
536.01	0\\
537.01	0\\
538.01	0\\
539.01	0\\
540.01	0\\
541.01	0\\
542.01	0\\
543.01	0\\
544.01	0\\
545.01	0\\
546.01	0\\
547.01	0\\
548.01	0\\
549.01	0\\
550.01	0\\
551.01	0\\
552.01	0\\
553.01	0\\
554.01	0\\
555.01	0\\
556.01	0\\
557.01	0\\
558.01	0\\
559.01	0\\
560.01	0\\
561.01	0\\
562.01	0\\
563.01	0\\
564.01	0\\
565.01	0\\
566.01	0\\
567.01	0\\
568.01	0\\
569.01	0\\
570.01	0\\
571.01	0\\
572.01	0\\
573.01	0\\
574.01	0\\
575.01	0\\
576.01	0\\
577.01	0\\
578.01	0\\
579.01	0\\
580.01	0\\
581.01	0\\
582.01	0\\
583.01	0\\
584.01	0\\
585.01	0\\
586.01	0\\
587.01	0\\
588.01	0\\
589.01	0\\
590.01	0\\
591.01	0\\
592.01	0\\
593.01	0\\
594.01	0\\
595.01	0\\
596.01	0\\
597.01	0\\
598.01	0.00143832067364899\\
599.01	0.00385661257178434\\
599.02	0.0038941481880362\\
599.03	0.00393204145978999\\
599.04	0.00397029584111443\\
599.05	0.00400891481941645\\
599.06	0.00404790191576273\\
599.07	0.00408726068520424\\
599.08	0.00412699471710402\\
599.09	0.00416710763546808\\
599.1	0.00420760309927944\\
599.11	0.00424848480283546\\
599.12	0.00428975647608841\\
599.13	0.00433142188498926\\
599.14	0.00437348483183484\\
599.15	0.00441594915561834\\
599.16	0.00445881873238316\\
599.17	0.00450209747558018\\
599.18	0.00454578933642853\\
599.19	0.00458989830427969\\
599.2	0.00463442840698527\\
599.21	0.00467938371126824\\
599.22	0.00472476832309771\\
599.23	0.00477058638806738\\
599.24	0.00481684209177767\\
599.25	0.0048635396602214\\
599.26	0.00491068334947654\\
599.27	0.00495827744951256\\
599.28	0.00500632629152094\\
599.29	0.00505483424831089\\
599.3	0.00510380573470894\\
599.31	0.00515324520796213\\
599.32	0.00520315716814533\\
599.33	0.00525354615857229\\
599.34	0.00530441676621069\\
599.35	0.00535577362210116\\
599.36	0.00540762140178036\\
599.37	0.00545996482570803\\
599.38	0.00551280865969823\\
599.39	0.00556615771535462\\
599.4	0.00562001685051006\\
599.41	0.00567439096967018\\
599.42	0.00572928502446148\\
599.43	0.00578470401408357\\
599.44	0.00584065298576571\\
599.45	0.00589713703522788\\
599.46	0.0059541613071461\\
599.47	0.00601173099562236\\
599.48	0.00606985134465893\\
599.49	0.00612852764863735\\
599.5	0.00618776525280185\\
599.51	0.00624756955374753\\
599.52	0.00630794599991322\\
599.53	0.00636890009207898\\
599.54	0.00643043738386841\\
599.55	0.00649256348225581\\
599.56	0.00655528404807817\\
599.57	0.00661860479655207\\
599.58	0.00668253149779552\\
599.59	0.00674706997735484\\
599.6	0.00681222611673654\\
599.61	0.00687800585394439\\
599.62	0.00694441518402151\\
599.63	0.00701146015959775\\
599.64	0.0070791468914423\\
599.65	0.00714748154902157\\
599.66	0.00721647036106244\\
599.67	0.00728611961612093\\
599.68	0.00735643566315622\\
599.69	0.00742742491211035\\
599.7	0.00749909383449329\\
599.71	0.00757144896397373\\
599.72	0.0076444968969755\\
599.73	0.00771824429327973\\
599.74	0.00779269787663272\\
599.75	0.00786786443535973\\
599.76	0.00794375082298454\\
599.77	0.00802036395885504\\
599.78	0.00809771082877482\\
599.79	0.00817579848564069\\
599.8	0.00825463405008646\\
599.81	0.00833422471113284\\
599.82	0.00841457772684348\\
599.83	0.00849570042498746\\
599.84	0.00857760020370797\\
599.85	0.00866028453219752\\
599.86	0.00874376095137954\\
599.87	0.00882803707459654\\
599.88	0.0089131205883049\\
599.89	0.00899901925277625\\
599.9	0.00908574090280562\\
599.91	0.00917329344842636\\
599.92	0.0092616848756319\\
599.93	0.00935092324710449\\
599.94	0.00944101670295079\\
599.95	0.00953197346144465\\
599.96	0.00962380181977694\\
599.97	0.00971651015481255\\
599.98	0.0098101069238547\\
599.99	0.00990460066541651\\
600	0.01\\
};
\addplot [color=mycolor19,solid,forget plot]
  table[row sep=crcr]{%
0.01	0\\
1.01	0\\
2.01	0\\
3.01	0\\
4.01	0\\
5.01	0\\
6.01	0\\
7.01	0\\
8.01	0\\
9.01	0\\
10.01	0\\
11.01	0\\
12.01	0\\
13.01	0\\
14.01	0\\
15.01	0\\
16.01	0\\
17.01	0\\
18.01	0\\
19.01	0\\
20.01	0\\
21.01	0\\
22.01	0\\
23.01	0\\
24.01	0\\
25.01	0\\
26.01	0\\
27.01	0\\
28.01	0\\
29.01	0\\
30.01	0\\
31.01	0\\
32.01	0\\
33.01	0\\
34.01	0\\
35.01	0\\
36.01	0\\
37.01	0\\
38.01	0\\
39.01	0\\
40.01	0\\
41.01	0\\
42.01	0\\
43.01	0\\
44.01	0\\
45.01	0\\
46.01	0\\
47.01	0\\
48.01	0\\
49.01	0\\
50.01	0\\
51.01	0\\
52.01	0\\
53.01	0\\
54.01	0\\
55.01	0\\
56.01	0\\
57.01	0\\
58.01	0\\
59.01	0\\
60.01	0\\
61.01	0\\
62.01	0\\
63.01	0\\
64.01	0\\
65.01	0\\
66.01	0\\
67.01	0\\
68.01	0\\
69.01	0\\
70.01	0\\
71.01	0\\
72.01	0\\
73.01	0\\
74.01	0\\
75.01	0\\
76.01	0\\
77.01	0\\
78.01	0\\
79.01	0\\
80.01	0\\
81.01	0\\
82.01	0\\
83.01	0\\
84.01	0\\
85.01	0\\
86.01	0\\
87.01	0\\
88.01	0\\
89.01	0\\
90.01	0\\
91.01	0\\
92.01	0\\
93.01	0\\
94.01	0\\
95.01	0\\
96.01	0\\
97.01	0\\
98.01	0\\
99.01	0\\
100.01	0\\
101.01	0\\
102.01	0\\
103.01	0\\
104.01	0\\
105.01	0\\
106.01	0\\
107.01	0\\
108.01	0\\
109.01	0\\
110.01	0\\
111.01	0\\
112.01	0\\
113.01	0\\
114.01	0\\
115.01	0\\
116.01	0\\
117.01	0\\
118.01	0\\
119.01	0\\
120.01	0\\
121.01	0\\
122.01	0\\
123.01	0\\
124.01	0\\
125.01	0\\
126.01	0\\
127.01	0\\
128.01	0\\
129.01	0\\
130.01	0\\
131.01	0\\
132.01	0\\
133.01	0\\
134.01	0\\
135.01	0\\
136.01	0\\
137.01	0\\
138.01	0\\
139.01	0\\
140.01	0\\
141.01	0\\
142.01	0\\
143.01	0\\
144.01	0\\
145.01	0\\
146.01	0\\
147.01	0\\
148.01	0\\
149.01	0\\
150.01	0\\
151.01	0\\
152.01	0\\
153.01	0\\
154.01	0\\
155.01	0\\
156.01	0\\
157.01	0\\
158.01	0\\
159.01	0\\
160.01	0\\
161.01	0\\
162.01	0\\
163.01	0\\
164.01	0\\
165.01	0\\
166.01	0\\
167.01	0\\
168.01	0\\
169.01	0\\
170.01	0\\
171.01	0\\
172.01	0\\
173.01	0\\
174.01	0\\
175.01	0\\
176.01	0\\
177.01	0\\
178.01	0\\
179.01	0\\
180.01	0\\
181.01	0\\
182.01	0\\
183.01	0\\
184.01	0\\
185.01	0\\
186.01	0\\
187.01	0\\
188.01	0\\
189.01	0\\
190.01	0\\
191.01	0\\
192.01	0\\
193.01	0\\
194.01	0\\
195.01	0\\
196.01	0\\
197.01	0\\
198.01	0\\
199.01	0\\
200.01	0\\
201.01	0\\
202.01	0\\
203.01	0\\
204.01	0\\
205.01	0\\
206.01	0\\
207.01	0\\
208.01	0\\
209.01	0\\
210.01	0\\
211.01	0\\
212.01	0\\
213.01	0\\
214.01	0\\
215.01	0\\
216.01	0\\
217.01	0\\
218.01	0\\
219.01	0\\
220.01	0\\
221.01	0\\
222.01	0\\
223.01	0\\
224.01	0\\
225.01	0\\
226.01	0\\
227.01	0\\
228.01	0\\
229.01	0\\
230.01	0\\
231.01	0\\
232.01	0\\
233.01	0\\
234.01	0\\
235.01	0\\
236.01	0\\
237.01	0\\
238.01	0\\
239.01	0\\
240.01	0\\
241.01	0\\
242.01	0\\
243.01	0\\
244.01	0\\
245.01	0\\
246.01	0\\
247.01	0\\
248.01	0\\
249.01	0\\
250.01	0\\
251.01	0\\
252.01	0\\
253.01	0\\
254.01	0\\
255.01	0\\
256.01	0\\
257.01	0\\
258.01	0\\
259.01	0\\
260.01	0\\
261.01	0\\
262.01	0\\
263.01	0\\
264.01	0\\
265.01	0\\
266.01	0\\
267.01	0\\
268.01	0\\
269.01	0\\
270.01	0\\
271.01	0\\
272.01	0\\
273.01	0\\
274.01	0\\
275.01	0\\
276.01	0\\
277.01	0\\
278.01	0\\
279.01	0\\
280.01	0\\
281.01	0\\
282.01	0\\
283.01	0\\
284.01	0\\
285.01	0\\
286.01	0\\
287.01	0\\
288.01	0\\
289.01	0\\
290.01	0\\
291.01	0\\
292.01	0\\
293.01	0\\
294.01	0\\
295.01	0\\
296.01	0\\
297.01	0\\
298.01	0\\
299.01	0\\
300.01	0\\
301.01	0\\
302.01	0\\
303.01	0\\
304.01	0\\
305.01	0\\
306.01	0\\
307.01	0\\
308.01	0\\
309.01	0\\
310.01	0\\
311.01	0\\
312.01	0\\
313.01	0\\
314.01	0\\
315.01	0\\
316.01	0\\
317.01	0\\
318.01	0\\
319.01	0\\
320.01	0\\
321.01	0\\
322.01	0\\
323.01	0\\
324.01	0\\
325.01	0\\
326.01	0\\
327.01	0\\
328.01	0\\
329.01	0\\
330.01	0\\
331.01	0\\
332.01	0\\
333.01	0\\
334.01	0\\
335.01	0\\
336.01	0\\
337.01	0\\
338.01	0\\
339.01	0\\
340.01	0\\
341.01	0\\
342.01	0\\
343.01	0\\
344.01	0\\
345.01	0\\
346.01	0\\
347.01	0\\
348.01	0\\
349.01	0\\
350.01	0\\
351.01	0\\
352.01	0\\
353.01	0\\
354.01	0\\
355.01	0\\
356.01	0\\
357.01	0\\
358.01	0\\
359.01	0\\
360.01	0\\
361.01	0\\
362.01	0\\
363.01	0\\
364.01	0\\
365.01	0\\
366.01	0\\
367.01	0\\
368.01	0\\
369.01	0\\
370.01	0\\
371.01	0\\
372.01	0\\
373.01	0\\
374.01	0\\
375.01	0\\
376.01	0\\
377.01	0\\
378.01	0\\
379.01	0\\
380.01	0\\
381.01	0\\
382.01	0\\
383.01	0\\
384.01	0\\
385.01	0\\
386.01	0\\
387.01	0\\
388.01	0\\
389.01	0\\
390.01	0\\
391.01	0\\
392.01	0\\
393.01	0\\
394.01	0\\
395.01	0\\
396.01	0\\
397.01	0\\
398.01	0\\
399.01	0\\
400.01	0\\
401.01	0\\
402.01	0\\
403.01	0\\
404.01	0\\
405.01	0\\
406.01	0\\
407.01	0\\
408.01	0\\
409.01	0\\
410.01	0\\
411.01	0\\
412.01	0\\
413.01	0\\
414.01	0\\
415.01	0\\
416.01	0\\
417.01	0\\
418.01	0\\
419.01	0\\
420.01	0\\
421.01	0\\
422.01	0\\
423.01	0\\
424.01	0\\
425.01	0\\
426.01	0\\
427.01	0\\
428.01	0\\
429.01	0\\
430.01	0\\
431.01	0\\
432.01	0\\
433.01	0\\
434.01	0\\
435.01	0\\
436.01	0\\
437.01	0\\
438.01	0\\
439.01	0\\
440.01	0\\
441.01	0\\
442.01	0\\
443.01	0\\
444.01	0\\
445.01	0\\
446.01	0\\
447.01	0\\
448.01	0\\
449.01	0\\
450.01	0\\
451.01	0\\
452.01	0\\
453.01	0\\
454.01	0\\
455.01	0\\
456.01	0\\
457.01	0\\
458.01	0\\
459.01	0\\
460.01	0\\
461.01	0\\
462.01	0\\
463.01	0\\
464.01	0\\
465.01	0\\
466.01	0\\
467.01	0\\
468.01	0\\
469.01	0\\
470.01	0\\
471.01	0\\
472.01	0\\
473.01	0\\
474.01	0\\
475.01	0\\
476.01	0\\
477.01	0\\
478.01	0\\
479.01	0\\
480.01	0\\
481.01	0\\
482.01	0\\
483.01	0\\
484.01	0\\
485.01	0\\
486.01	0\\
487.01	0\\
488.01	0\\
489.01	0\\
490.01	0\\
491.01	0\\
492.01	0\\
493.01	0\\
494.01	0\\
495.01	0\\
496.01	0\\
497.01	0\\
498.01	0\\
499.01	0\\
500.01	0\\
501.01	0\\
502.01	0\\
503.01	0\\
504.01	0\\
505.01	0\\
506.01	0\\
507.01	0\\
508.01	0\\
509.01	0\\
510.01	0\\
511.01	0\\
512.01	0\\
513.01	0\\
514.01	0\\
515.01	0\\
516.01	0\\
517.01	0\\
518.01	0\\
519.01	0\\
520.01	0\\
521.01	0\\
522.01	0\\
523.01	0\\
524.01	0\\
525.01	0\\
526.01	0\\
527.01	0\\
528.01	0\\
529.01	0\\
530.01	0\\
531.01	0\\
532.01	0\\
533.01	0\\
534.01	0\\
535.01	0\\
536.01	0\\
537.01	0\\
538.01	0\\
539.01	0\\
540.01	0\\
541.01	0\\
542.01	0\\
543.01	0\\
544.01	0\\
545.01	0\\
546.01	0\\
547.01	0\\
548.01	0\\
549.01	0\\
550.01	0\\
551.01	0\\
552.01	0\\
553.01	0\\
554.01	0\\
555.01	0\\
556.01	0\\
557.01	0\\
558.01	0\\
559.01	0\\
560.01	0\\
561.01	0\\
562.01	0\\
563.01	0\\
564.01	0\\
565.01	0\\
566.01	0\\
567.01	0\\
568.01	0\\
569.01	0\\
570.01	0\\
571.01	0\\
572.01	0\\
573.01	0\\
574.01	0\\
575.01	0\\
576.01	0\\
577.01	0\\
578.01	0\\
579.01	0\\
580.01	0\\
581.01	0\\
582.01	0\\
583.01	0\\
584.01	0\\
585.01	0\\
586.01	0\\
587.01	0\\
588.01	0\\
589.01	0\\
590.01	0\\
591.01	0\\
592.01	0\\
593.01	0\\
594.01	0\\
595.01	0\\
596.01	0\\
597.01	0\\
598.01	0.00143832068163977\\
599.01	0.00385661257211143\\
599.02	0.00389414818834777\\
599.03	0.00393204146008657\\
599.04	0.00397029584139656\\
599.05	0.00400891481968466\\
599.06	0.00404790191601751\\
599.07	0.00408726068544609\\
599.08	0.00412699471733344\\
599.09	0.00416710763568554\\
599.1	0.0042076030994854\\
599.11	0.00424848480303038\\
599.12	0.00428975647627273\\
599.13	0.0043314218851634\\
599.14	0.00437348483199923\\
599.15	0.00441594915577338\\
599.16	0.00445881873252926\\
599.17	0.00450209747571773\\
599.18	0.00454578933655789\\
599.19	0.00458989830440124\\
599.2	0.00463442840709936\\
599.21	0.00467938371137522\\
599.22	0.0047247683231979\\
599.23	0.00477058638816113\\
599.24	0.00481684209186529\\
599.25	0.00486353966030319\\
599.26	0.0049106833495528\\
599.27	0.00495827744958357\\
599.28	0.00500632629158696\\
599.29	0.00505483424837222\\
599.3	0.00510380573476581\\
599.31	0.00515324520801481\\
599.32	0.00520315716819406\\
599.33	0.00525354615861728\\
599.34	0.00530441676625217\\
599.35	0.00535577362213935\\
599.36	0.00540762140181545\\
599.37	0.00545996482574022\\
599.38	0.0055128086597277\\
599.39	0.00556615771538157\\
599.4	0.00562001685053463\\
599.41	0.00567439096969254\\
599.42	0.00572928502448181\\
599.43	0.00578470401410199\\
599.44	0.00584065298578238\\
599.45	0.00589713703524292\\
599.46	0.00595416130715964\\
599.47	0.00601173099563451\\
599.48	0.00606985134466982\\
599.49	0.00612852764864707\\
599.5	0.0061877652528105\\
599.51	0.00624756955375522\\
599.52	0.00630794599992002\\
599.53	0.00636890009208498\\
599.54	0.00643043738387368\\
599.55	0.00649256348226043\\
599.56	0.0065552840480822\\
599.57	0.00661860479655558\\
599.58	0.00668253149779855\\
599.59	0.00674706997735745\\
599.6	0.00681222611673879\\
599.61	0.00687800585394631\\
599.62	0.00694441518402313\\
599.63	0.00701146015959912\\
599.64	0.00707914689144344\\
599.65	0.00714748154902252\\
599.66	0.00721647036106323\\
599.67	0.00728611961612157\\
599.68	0.00735643566315675\\
599.69	0.00742742491211077\\
599.7	0.00749909383449363\\
599.71	0.00757144896397399\\
599.72	0.0076444968969757\\
599.73	0.00771824429327989\\
599.74	0.00779269787663284\\
599.75	0.00786786443535982\\
599.76	0.0079437508229846\\
599.77	0.00802036395885509\\
599.78	0.00809771082877485\\
599.79	0.00817579848564071\\
599.8	0.00825463405008648\\
599.81	0.00833422471113285\\
599.82	0.00841457772684349\\
599.83	0.00849570042498746\\
599.84	0.00857760020370797\\
599.85	0.00866028453219752\\
599.86	0.00874376095137953\\
599.87	0.00882803707459654\\
599.88	0.0089131205883049\\
599.89	0.00899901925277625\\
599.9	0.00908574090280562\\
599.91	0.00917329344842636\\
599.92	0.0092616848756319\\
599.93	0.00935092324710449\\
599.94	0.00944101670295078\\
599.95	0.00953197346144465\\
599.96	0.00962380181977693\\
599.97	0.00971651015481255\\
599.98	0.0098101069238547\\
599.99	0.00990460066541651\\
600	0.01\\
};
\addplot [color=red!50!mycolor17,solid,forget plot]
  table[row sep=crcr]{%
0.01	0\\
1.01	0\\
2.01	0\\
3.01	0\\
4.01	0\\
5.01	0\\
6.01	0\\
7.01	0\\
8.01	0\\
9.01	0\\
10.01	0\\
11.01	0\\
12.01	0\\
13.01	0\\
14.01	0\\
15.01	0\\
16.01	0\\
17.01	0\\
18.01	0\\
19.01	0\\
20.01	0\\
21.01	0\\
22.01	0\\
23.01	0\\
24.01	0\\
25.01	0\\
26.01	0\\
27.01	0\\
28.01	0\\
29.01	0\\
30.01	0\\
31.01	0\\
32.01	0\\
33.01	0\\
34.01	0\\
35.01	0\\
36.01	0\\
37.01	0\\
38.01	0\\
39.01	0\\
40.01	0\\
41.01	0\\
42.01	0\\
43.01	0\\
44.01	0\\
45.01	0\\
46.01	0\\
47.01	0\\
48.01	0\\
49.01	0\\
50.01	0\\
51.01	0\\
52.01	0\\
53.01	0\\
54.01	0\\
55.01	0\\
56.01	0\\
57.01	0\\
58.01	0\\
59.01	0\\
60.01	0\\
61.01	0\\
62.01	0\\
63.01	0\\
64.01	0\\
65.01	0\\
66.01	0\\
67.01	0\\
68.01	0\\
69.01	0\\
70.01	0\\
71.01	0\\
72.01	0\\
73.01	0\\
74.01	0\\
75.01	0\\
76.01	0\\
77.01	0\\
78.01	0\\
79.01	0\\
80.01	0\\
81.01	0\\
82.01	0\\
83.01	0\\
84.01	0\\
85.01	0\\
86.01	0\\
87.01	0\\
88.01	0\\
89.01	0\\
90.01	0\\
91.01	0\\
92.01	0\\
93.01	0\\
94.01	0\\
95.01	0\\
96.01	0\\
97.01	0\\
98.01	0\\
99.01	0\\
100.01	0\\
101.01	0\\
102.01	0\\
103.01	0\\
104.01	0\\
105.01	0\\
106.01	0\\
107.01	0\\
108.01	0\\
109.01	0\\
110.01	0\\
111.01	0\\
112.01	0\\
113.01	0\\
114.01	0\\
115.01	0\\
116.01	0\\
117.01	0\\
118.01	0\\
119.01	0\\
120.01	0\\
121.01	0\\
122.01	0\\
123.01	0\\
124.01	0\\
125.01	0\\
126.01	0\\
127.01	0\\
128.01	0\\
129.01	0\\
130.01	0\\
131.01	0\\
132.01	0\\
133.01	0\\
134.01	0\\
135.01	0\\
136.01	0\\
137.01	0\\
138.01	0\\
139.01	0\\
140.01	0\\
141.01	0\\
142.01	0\\
143.01	0\\
144.01	0\\
145.01	0\\
146.01	0\\
147.01	0\\
148.01	0\\
149.01	0\\
150.01	0\\
151.01	0\\
152.01	0\\
153.01	0\\
154.01	0\\
155.01	0\\
156.01	0\\
157.01	0\\
158.01	0\\
159.01	0\\
160.01	0\\
161.01	0\\
162.01	0\\
163.01	0\\
164.01	0\\
165.01	0\\
166.01	0\\
167.01	0\\
168.01	0\\
169.01	0\\
170.01	0\\
171.01	0\\
172.01	0\\
173.01	0\\
174.01	0\\
175.01	0\\
176.01	0\\
177.01	0\\
178.01	0\\
179.01	0\\
180.01	0\\
181.01	0\\
182.01	0\\
183.01	0\\
184.01	0\\
185.01	0\\
186.01	0\\
187.01	0\\
188.01	0\\
189.01	0\\
190.01	0\\
191.01	0\\
192.01	0\\
193.01	0\\
194.01	0\\
195.01	0\\
196.01	0\\
197.01	0\\
198.01	0\\
199.01	0\\
200.01	0\\
201.01	0\\
202.01	0\\
203.01	0\\
204.01	0\\
205.01	0\\
206.01	0\\
207.01	0\\
208.01	0\\
209.01	0\\
210.01	0\\
211.01	0\\
212.01	0\\
213.01	0\\
214.01	0\\
215.01	0\\
216.01	0\\
217.01	0\\
218.01	0\\
219.01	0\\
220.01	0\\
221.01	0\\
222.01	0\\
223.01	0\\
224.01	0\\
225.01	0\\
226.01	0\\
227.01	0\\
228.01	0\\
229.01	0\\
230.01	0\\
231.01	0\\
232.01	0\\
233.01	0\\
234.01	0\\
235.01	0\\
236.01	0\\
237.01	0\\
238.01	0\\
239.01	0\\
240.01	0\\
241.01	0\\
242.01	0\\
243.01	0\\
244.01	0\\
245.01	0\\
246.01	0\\
247.01	0\\
248.01	0\\
249.01	0\\
250.01	0\\
251.01	0\\
252.01	0\\
253.01	0\\
254.01	0\\
255.01	0\\
256.01	0\\
257.01	0\\
258.01	0\\
259.01	0\\
260.01	0\\
261.01	0\\
262.01	0\\
263.01	0\\
264.01	0\\
265.01	0\\
266.01	0\\
267.01	0\\
268.01	0\\
269.01	0\\
270.01	0\\
271.01	0\\
272.01	0\\
273.01	0\\
274.01	0\\
275.01	0\\
276.01	0\\
277.01	0\\
278.01	0\\
279.01	0\\
280.01	0\\
281.01	0\\
282.01	0\\
283.01	0\\
284.01	0\\
285.01	0\\
286.01	0\\
287.01	0\\
288.01	0\\
289.01	0\\
290.01	0\\
291.01	0\\
292.01	0\\
293.01	0\\
294.01	0\\
295.01	0\\
296.01	0\\
297.01	0\\
298.01	0\\
299.01	0\\
300.01	0\\
301.01	0\\
302.01	0\\
303.01	0\\
304.01	0\\
305.01	0\\
306.01	0\\
307.01	0\\
308.01	0\\
309.01	0\\
310.01	0\\
311.01	0\\
312.01	0\\
313.01	0\\
314.01	0\\
315.01	0\\
316.01	0\\
317.01	0\\
318.01	0\\
319.01	0\\
320.01	0\\
321.01	0\\
322.01	0\\
323.01	0\\
324.01	0\\
325.01	0\\
326.01	0\\
327.01	0\\
328.01	0\\
329.01	0\\
330.01	0\\
331.01	0\\
332.01	0\\
333.01	0\\
334.01	0\\
335.01	0\\
336.01	0\\
337.01	0\\
338.01	0\\
339.01	0\\
340.01	0\\
341.01	0\\
342.01	0\\
343.01	0\\
344.01	0\\
345.01	0\\
346.01	0\\
347.01	0\\
348.01	0\\
349.01	0\\
350.01	0\\
351.01	0\\
352.01	0\\
353.01	0\\
354.01	0\\
355.01	0\\
356.01	0\\
357.01	0\\
358.01	0\\
359.01	0\\
360.01	0\\
361.01	0\\
362.01	0\\
363.01	0\\
364.01	0\\
365.01	0\\
366.01	0\\
367.01	0\\
368.01	0\\
369.01	0\\
370.01	0\\
371.01	0\\
372.01	0\\
373.01	0\\
374.01	0\\
375.01	0\\
376.01	0\\
377.01	0\\
378.01	0\\
379.01	0\\
380.01	0\\
381.01	0\\
382.01	0\\
383.01	0\\
384.01	0\\
385.01	0\\
386.01	0\\
387.01	0\\
388.01	0\\
389.01	0\\
390.01	0\\
391.01	0\\
392.01	0\\
393.01	0\\
394.01	0\\
395.01	0\\
396.01	0\\
397.01	0\\
398.01	0\\
399.01	0\\
400.01	0\\
401.01	0\\
402.01	0\\
403.01	0\\
404.01	0\\
405.01	0\\
406.01	0\\
407.01	0\\
408.01	0\\
409.01	0\\
410.01	0\\
411.01	0\\
412.01	0\\
413.01	0\\
414.01	0\\
415.01	0\\
416.01	0\\
417.01	0\\
418.01	0\\
419.01	0\\
420.01	0\\
421.01	0\\
422.01	0\\
423.01	0\\
424.01	0\\
425.01	0\\
426.01	0\\
427.01	0\\
428.01	0\\
429.01	0\\
430.01	0\\
431.01	0\\
432.01	0\\
433.01	0\\
434.01	0\\
435.01	0\\
436.01	0\\
437.01	0\\
438.01	0\\
439.01	0\\
440.01	0\\
441.01	0\\
442.01	0\\
443.01	0\\
444.01	0\\
445.01	0\\
446.01	0\\
447.01	0\\
448.01	0\\
449.01	0\\
450.01	0\\
451.01	0\\
452.01	0\\
453.01	0\\
454.01	0\\
455.01	0\\
456.01	0\\
457.01	0\\
458.01	0\\
459.01	0\\
460.01	0\\
461.01	0\\
462.01	0\\
463.01	0\\
464.01	0\\
465.01	0\\
466.01	0\\
467.01	0\\
468.01	0\\
469.01	0\\
470.01	0\\
471.01	0\\
472.01	0\\
473.01	0\\
474.01	0\\
475.01	0\\
476.01	0\\
477.01	0\\
478.01	0\\
479.01	0\\
480.01	0\\
481.01	0\\
482.01	0\\
483.01	0\\
484.01	0\\
485.01	0\\
486.01	0\\
487.01	0\\
488.01	0\\
489.01	0\\
490.01	0\\
491.01	0\\
492.01	0\\
493.01	0\\
494.01	0\\
495.01	0\\
496.01	0\\
497.01	0\\
498.01	0\\
499.01	0\\
500.01	0\\
501.01	0\\
502.01	0\\
503.01	0\\
504.01	0\\
505.01	0\\
506.01	0\\
507.01	0\\
508.01	0\\
509.01	0\\
510.01	0\\
511.01	0\\
512.01	0\\
513.01	0\\
514.01	0\\
515.01	0\\
516.01	0\\
517.01	0\\
518.01	0\\
519.01	0\\
520.01	0\\
521.01	0\\
522.01	0\\
523.01	0\\
524.01	0\\
525.01	0\\
526.01	0\\
527.01	0\\
528.01	0\\
529.01	0\\
530.01	0\\
531.01	0\\
532.01	0\\
533.01	0\\
534.01	0\\
535.01	0\\
536.01	0\\
537.01	0\\
538.01	0\\
539.01	0\\
540.01	0\\
541.01	0\\
542.01	0\\
543.01	0\\
544.01	0\\
545.01	0\\
546.01	0\\
547.01	0\\
548.01	0\\
549.01	0\\
550.01	0\\
551.01	0\\
552.01	0\\
553.01	0\\
554.01	0\\
555.01	0\\
556.01	0\\
557.01	0\\
558.01	0\\
559.01	0\\
560.01	0\\
561.01	0\\
562.01	0\\
563.01	0\\
564.01	0\\
565.01	0\\
566.01	0\\
567.01	0\\
568.01	0\\
569.01	0\\
570.01	0\\
571.01	0\\
572.01	0\\
573.01	0\\
574.01	0\\
575.01	0\\
576.01	0\\
577.01	0\\
578.01	0\\
579.01	0\\
580.01	0\\
581.01	0\\
582.01	0\\
583.01	0\\
584.01	0\\
585.01	0\\
586.01	0\\
587.01	0\\
588.01	0\\
589.01	0\\
590.01	0\\
591.01	0\\
592.01	0\\
593.01	0\\
594.01	0\\
595.01	0\\
596.01	0\\
597.01	0.000469956016972223\\
598.01	0.00143832068184645\\
599.01	0.00385661257211617\\
599.02	0.00389414818835224\\
599.03	0.00393204146009079\\
599.04	0.00397029584140053\\
599.05	0.0040089148196884\\
599.06	0.00404790191602104\\
599.07	0.00408726068544941\\
599.08	0.00412699471733656\\
599.09	0.00416710763568846\\
599.1	0.00420760309948814\\
599.11	0.00424848480303295\\
599.12	0.00428975647627514\\
599.13	0.00433142188516566\\
599.14	0.00437348483200135\\
599.15	0.00441594915577536\\
599.16	0.00445881873253109\\
599.17	0.00450209747571945\\
599.18	0.00454578933655949\\
599.19	0.00458989830440273\\
599.2	0.00463442840710074\\
599.21	0.00467938371137649\\
599.22	0.00472476832319909\\
599.23	0.00477058638816223\\
599.24	0.0048168420918663\\
599.25	0.00486353966030412\\
599.26	0.00491068334955366\\
599.27	0.00495827744958435\\
599.28	0.00500632629158769\\
599.29	0.00505483424837289\\
599.3	0.00510380573476642\\
599.31	0.00515324520801536\\
599.32	0.00520315716819456\\
599.33	0.00525354615861774\\
599.34	0.00530441676625258\\
599.35	0.00535577362213972\\
599.36	0.00540762140181579\\
599.37	0.00545996482574053\\
599.38	0.00551280865972798\\
599.39	0.00556615771538182\\
599.4	0.00562001685053486\\
599.41	0.00567439096969275\\
599.42	0.00572928502448199\\
599.43	0.00578470401410216\\
599.44	0.00584065298578252\\
599.45	0.00589713703524304\\
599.46	0.00595416130715974\\
599.47	0.0060117309956346\\
599.48	0.0060698513446699\\
599.49	0.00612852764864714\\
599.5	0.00618776525281056\\
599.51	0.00624756955375527\\
599.52	0.00630794599992007\\
599.53	0.00636890009208501\\
599.54	0.00643043738387371\\
599.55	0.00649256348226045\\
599.56	0.00655528404808222\\
599.57	0.00661860479655559\\
599.58	0.00668253149779856\\
599.59	0.00674706997735745\\
599.6	0.00681222611673878\\
599.61	0.0068780058539463\\
599.62	0.00694441518402313\\
599.63	0.00701146015959912\\
599.64	0.00707914689144345\\
599.65	0.00714748154902253\\
599.66	0.00721647036106323\\
599.67	0.00728611961612158\\
599.68	0.00735643566315675\\
599.69	0.00742742491211078\\
599.7	0.00749909383449363\\
599.71	0.007571448963974\\
599.72	0.00764449689697571\\
599.73	0.00771824429327989\\
599.74	0.00779269787663284\\
599.75	0.00786786443535982\\
599.76	0.0079437508229846\\
599.77	0.0080203639588551\\
599.78	0.00809771082877486\\
599.79	0.00817579848564071\\
599.8	0.00825463405008648\\
599.81	0.00833422471113285\\
599.82	0.00841457772684349\\
599.83	0.00849570042498747\\
599.84	0.00857760020370797\\
599.85	0.00866028453219752\\
599.86	0.00874376095137953\\
599.87	0.00882803707459654\\
599.88	0.0089131205883049\\
599.89	0.00899901925277625\\
599.9	0.00908574090280562\\
599.91	0.00917329344842635\\
599.92	0.0092616848756319\\
599.93	0.00935092324710449\\
599.94	0.00944101670295078\\
599.95	0.00953197346144465\\
599.96	0.00962380181977693\\
599.97	0.00971651015481255\\
599.98	0.0098101069238547\\
599.99	0.00990460066541651\\
600	0.01\\
};
\addplot [color=red!40!mycolor19,solid,forget plot]
  table[row sep=crcr]{%
0.01	0\\
1.01	0\\
2.01	0\\
3.01	0\\
4.01	0\\
5.01	0\\
6.01	0\\
7.01	0\\
8.01	0\\
9.01	0\\
10.01	0\\
11.01	0\\
12.01	0\\
13.01	0\\
14.01	0\\
15.01	0\\
16.01	0\\
17.01	0\\
18.01	0\\
19.01	0\\
20.01	0\\
21.01	0\\
22.01	0\\
23.01	0\\
24.01	0\\
25.01	0\\
26.01	0\\
27.01	0\\
28.01	0\\
29.01	0\\
30.01	0\\
31.01	0\\
32.01	0\\
33.01	0\\
34.01	0\\
35.01	0\\
36.01	0\\
37.01	0\\
38.01	0\\
39.01	0\\
40.01	0\\
41.01	0\\
42.01	0\\
43.01	0\\
44.01	0\\
45.01	0\\
46.01	0\\
47.01	0\\
48.01	0\\
49.01	0\\
50.01	0\\
51.01	0\\
52.01	0\\
53.01	0\\
54.01	0\\
55.01	0\\
56.01	0\\
57.01	0\\
58.01	0\\
59.01	0\\
60.01	0\\
61.01	0\\
62.01	0\\
63.01	0\\
64.01	0\\
65.01	0\\
66.01	0\\
67.01	0\\
68.01	0\\
69.01	0\\
70.01	0\\
71.01	0\\
72.01	0\\
73.01	0\\
74.01	0\\
75.01	0\\
76.01	0\\
77.01	0\\
78.01	0\\
79.01	0\\
80.01	0\\
81.01	0\\
82.01	0\\
83.01	0\\
84.01	0\\
85.01	0\\
86.01	0\\
87.01	0\\
88.01	0\\
89.01	0\\
90.01	0\\
91.01	0\\
92.01	0\\
93.01	0\\
94.01	0\\
95.01	0\\
96.01	0\\
97.01	0\\
98.01	0\\
99.01	0\\
100.01	0\\
101.01	0\\
102.01	0\\
103.01	0\\
104.01	0\\
105.01	0\\
106.01	0\\
107.01	0\\
108.01	0\\
109.01	0\\
110.01	0\\
111.01	0\\
112.01	0\\
113.01	0\\
114.01	0\\
115.01	0\\
116.01	0\\
117.01	0\\
118.01	0\\
119.01	0\\
120.01	0\\
121.01	0\\
122.01	0\\
123.01	0\\
124.01	0\\
125.01	0\\
126.01	0\\
127.01	0\\
128.01	0\\
129.01	0\\
130.01	0\\
131.01	0\\
132.01	0\\
133.01	0\\
134.01	0\\
135.01	0\\
136.01	0\\
137.01	0\\
138.01	0\\
139.01	0\\
140.01	0\\
141.01	0\\
142.01	0\\
143.01	0\\
144.01	0\\
145.01	0\\
146.01	0\\
147.01	0\\
148.01	0\\
149.01	0\\
150.01	0\\
151.01	0\\
152.01	0\\
153.01	0\\
154.01	0\\
155.01	0\\
156.01	0\\
157.01	0\\
158.01	0\\
159.01	0\\
160.01	0\\
161.01	0\\
162.01	0\\
163.01	0\\
164.01	0\\
165.01	0\\
166.01	0\\
167.01	0\\
168.01	0\\
169.01	0\\
170.01	0\\
171.01	0\\
172.01	0\\
173.01	0\\
174.01	0\\
175.01	0\\
176.01	0\\
177.01	0\\
178.01	0\\
179.01	0\\
180.01	0\\
181.01	0\\
182.01	0\\
183.01	0\\
184.01	0\\
185.01	0\\
186.01	0\\
187.01	0\\
188.01	0\\
189.01	0\\
190.01	0\\
191.01	0\\
192.01	0\\
193.01	0\\
194.01	0\\
195.01	0\\
196.01	0\\
197.01	0\\
198.01	0\\
199.01	0\\
200.01	0\\
201.01	0\\
202.01	0\\
203.01	0\\
204.01	0\\
205.01	0\\
206.01	0\\
207.01	0\\
208.01	0\\
209.01	0\\
210.01	0\\
211.01	0\\
212.01	0\\
213.01	0\\
214.01	0\\
215.01	0\\
216.01	0\\
217.01	0\\
218.01	0\\
219.01	0\\
220.01	0\\
221.01	0\\
222.01	0\\
223.01	0\\
224.01	0\\
225.01	0\\
226.01	0\\
227.01	0\\
228.01	0\\
229.01	0\\
230.01	0\\
231.01	0\\
232.01	0\\
233.01	0\\
234.01	0\\
235.01	0\\
236.01	0\\
237.01	0\\
238.01	0\\
239.01	0\\
240.01	0\\
241.01	0\\
242.01	0\\
243.01	0\\
244.01	0\\
245.01	0\\
246.01	0\\
247.01	0\\
248.01	0\\
249.01	0\\
250.01	0\\
251.01	0\\
252.01	0\\
253.01	0\\
254.01	0\\
255.01	0\\
256.01	0\\
257.01	0\\
258.01	0\\
259.01	0\\
260.01	0\\
261.01	0\\
262.01	0\\
263.01	0\\
264.01	0\\
265.01	0\\
266.01	0\\
267.01	0\\
268.01	0\\
269.01	0\\
270.01	0\\
271.01	0\\
272.01	0\\
273.01	0\\
274.01	0\\
275.01	0\\
276.01	0\\
277.01	0\\
278.01	0\\
279.01	0\\
280.01	0\\
281.01	0\\
282.01	0\\
283.01	0\\
284.01	0\\
285.01	0\\
286.01	0\\
287.01	0\\
288.01	0\\
289.01	0\\
290.01	0\\
291.01	0\\
292.01	0\\
293.01	0\\
294.01	0\\
295.01	0\\
296.01	0\\
297.01	0\\
298.01	0\\
299.01	0\\
300.01	0\\
301.01	0\\
302.01	0\\
303.01	0\\
304.01	0\\
305.01	0\\
306.01	0\\
307.01	0\\
308.01	0\\
309.01	0\\
310.01	0\\
311.01	0\\
312.01	0\\
313.01	0\\
314.01	0\\
315.01	0\\
316.01	0\\
317.01	0\\
318.01	0\\
319.01	0\\
320.01	0\\
321.01	0\\
322.01	0\\
323.01	0\\
324.01	0\\
325.01	0\\
326.01	0\\
327.01	0\\
328.01	0\\
329.01	0\\
330.01	0\\
331.01	0\\
332.01	0\\
333.01	0\\
334.01	0\\
335.01	0\\
336.01	0\\
337.01	0\\
338.01	0\\
339.01	0\\
340.01	0\\
341.01	0\\
342.01	0\\
343.01	0\\
344.01	0\\
345.01	0\\
346.01	0\\
347.01	0\\
348.01	0\\
349.01	0\\
350.01	0\\
351.01	0\\
352.01	0\\
353.01	0\\
354.01	0\\
355.01	0\\
356.01	0\\
357.01	0\\
358.01	0\\
359.01	0\\
360.01	0\\
361.01	0\\
362.01	0\\
363.01	0\\
364.01	0\\
365.01	0\\
366.01	0\\
367.01	0\\
368.01	0\\
369.01	0\\
370.01	0\\
371.01	0\\
372.01	0\\
373.01	0\\
374.01	0\\
375.01	0\\
376.01	0\\
377.01	0\\
378.01	0\\
379.01	0\\
380.01	0\\
381.01	0\\
382.01	0\\
383.01	0\\
384.01	0\\
385.01	0\\
386.01	0\\
387.01	0\\
388.01	0\\
389.01	0\\
390.01	0\\
391.01	0\\
392.01	0\\
393.01	0\\
394.01	0\\
395.01	0\\
396.01	0\\
397.01	0\\
398.01	0\\
399.01	0\\
400.01	0\\
401.01	0\\
402.01	0\\
403.01	0\\
404.01	0\\
405.01	0\\
406.01	0\\
407.01	0\\
408.01	0\\
409.01	0\\
410.01	0\\
411.01	0\\
412.01	0\\
413.01	0\\
414.01	0\\
415.01	0\\
416.01	0\\
417.01	0\\
418.01	0\\
419.01	0\\
420.01	0\\
421.01	0\\
422.01	0\\
423.01	0\\
424.01	0\\
425.01	0\\
426.01	0\\
427.01	0\\
428.01	0\\
429.01	0\\
430.01	0\\
431.01	0\\
432.01	0\\
433.01	0\\
434.01	0\\
435.01	0\\
436.01	0\\
437.01	0\\
438.01	0\\
439.01	0\\
440.01	0\\
441.01	0\\
442.01	0\\
443.01	0\\
444.01	0\\
445.01	0\\
446.01	0\\
447.01	0\\
448.01	0\\
449.01	0\\
450.01	0\\
451.01	0\\
452.01	0\\
453.01	0\\
454.01	0\\
455.01	0\\
456.01	0\\
457.01	0\\
458.01	0\\
459.01	0\\
460.01	0\\
461.01	0\\
462.01	0\\
463.01	0\\
464.01	0\\
465.01	0\\
466.01	0\\
467.01	0\\
468.01	0\\
469.01	0\\
470.01	0\\
471.01	0\\
472.01	0\\
473.01	0\\
474.01	0\\
475.01	0\\
476.01	0\\
477.01	0\\
478.01	0\\
479.01	0\\
480.01	0\\
481.01	0\\
482.01	0\\
483.01	0\\
484.01	0\\
485.01	0\\
486.01	0\\
487.01	0\\
488.01	0\\
489.01	0\\
490.01	0\\
491.01	0\\
492.01	0\\
493.01	0\\
494.01	0\\
495.01	0\\
496.01	0\\
497.01	0\\
498.01	0\\
499.01	0\\
500.01	0\\
501.01	0\\
502.01	0\\
503.01	0\\
504.01	0\\
505.01	0\\
506.01	0\\
507.01	0\\
508.01	0\\
509.01	0\\
510.01	0\\
511.01	0\\
512.01	0\\
513.01	0\\
514.01	0\\
515.01	0\\
516.01	0\\
517.01	0\\
518.01	0\\
519.01	0\\
520.01	0\\
521.01	0\\
522.01	0\\
523.01	0\\
524.01	0\\
525.01	0\\
526.01	0\\
527.01	0\\
528.01	0\\
529.01	0\\
530.01	0\\
531.01	0\\
532.01	0\\
533.01	0\\
534.01	0\\
535.01	0\\
536.01	0\\
537.01	0\\
538.01	0\\
539.01	0\\
540.01	0\\
541.01	0\\
542.01	0\\
543.01	0\\
544.01	0\\
545.01	0\\
546.01	0\\
547.01	0\\
548.01	0\\
549.01	0\\
550.01	0\\
551.01	0\\
552.01	0\\
553.01	0\\
554.01	0\\
555.01	0\\
556.01	0\\
557.01	0\\
558.01	0\\
559.01	0\\
560.01	0\\
561.01	0\\
562.01	0\\
563.01	0\\
564.01	0\\
565.01	0\\
566.01	0\\
567.01	0\\
568.01	0\\
569.01	0\\
570.01	0\\
571.01	0\\
572.01	0\\
573.01	0\\
574.01	0\\
575.01	0\\
576.01	0\\
577.01	0\\
578.01	0\\
579.01	0\\
580.01	0\\
581.01	0\\
582.01	0\\
583.01	0\\
584.01	0\\
585.01	0\\
586.01	0\\
587.01	0\\
588.01	0\\
589.01	0\\
590.01	0\\
591.01	0\\
592.01	0\\
593.01	0\\
594.01	0\\
595.01	0\\
596.01	0\\
597.01	0.000480262484284122\\
598.01	0.00143832068185175\\
599.01	0.00385661257211624\\
599.02	0.00389414818835231\\
599.03	0.00393204146009084\\
599.04	0.00397029584140058\\
599.05	0.00400891481968844\\
599.06	0.00404790191602107\\
599.07	0.00408726068544944\\
599.08	0.00412699471733659\\
599.09	0.00416710763568848\\
599.1	0.00420760309948815\\
599.11	0.00424848480303296\\
599.12	0.00428975647627514\\
599.13	0.00433142188516566\\
599.14	0.00437348483200135\\
599.15	0.00441594915577536\\
599.16	0.0044588187325311\\
599.17	0.00450209747571944\\
599.18	0.00454578933655948\\
599.19	0.00458989830440271\\
599.2	0.00463442840710074\\
599.21	0.00467938371137649\\
599.22	0.00472476832319909\\
599.23	0.00477058638816223\\
599.24	0.0048168420918663\\
599.25	0.00486353966030412\\
599.26	0.00491068334955367\\
599.27	0.00495827744958437\\
599.28	0.0050063262915877\\
599.29	0.00505483424837289\\
599.3	0.00510380573476642\\
599.31	0.00515324520801536\\
599.32	0.00520315716819456\\
599.33	0.00525354615861774\\
599.34	0.00530441676625259\\
599.35	0.00535577362213972\\
599.36	0.00540762140181578\\
599.37	0.00545996482574052\\
599.38	0.00551280865972797\\
599.39	0.0055661577153818\\
599.4	0.00562001685053484\\
599.41	0.00567439096969274\\
599.42	0.00572928502448198\\
599.43	0.00578470401410214\\
599.44	0.00584065298578251\\
599.45	0.00589713703524303\\
599.46	0.00595416130715973\\
599.47	0.0060117309956346\\
599.48	0.0060698513446699\\
599.49	0.00612852764864714\\
599.5	0.00618776525281056\\
599.51	0.00624756955375527\\
599.52	0.00630794599992007\\
599.53	0.00636890009208502\\
599.54	0.00643043738387371\\
599.55	0.00649256348226045\\
599.56	0.00655528404808223\\
599.57	0.0066186047965556\\
599.58	0.00668253149779857\\
599.59	0.00674706997735746\\
599.6	0.0068122261167388\\
599.61	0.00687800585394632\\
599.62	0.00694441518402314\\
599.63	0.00701146015959913\\
599.64	0.00707914689144346\\
599.65	0.00714748154902253\\
599.66	0.00721647036106323\\
599.67	0.00728611961612158\\
599.68	0.00735643566315675\\
599.69	0.00742742491211078\\
599.7	0.00749909383449363\\
599.71	0.007571448963974\\
599.72	0.00764449689697571\\
599.73	0.00771824429327989\\
599.74	0.00779269787663285\\
599.75	0.00786786443535982\\
599.76	0.00794375082298461\\
599.77	0.0080203639588551\\
599.78	0.00809771082877486\\
599.79	0.00817579848564071\\
599.8	0.00825463405008648\\
599.81	0.00833422471113285\\
599.82	0.00841457772684349\\
599.83	0.00849570042498747\\
599.84	0.00857760020370798\\
599.85	0.00866028453219752\\
599.86	0.00874376095137954\\
599.87	0.00882803707459654\\
599.88	0.0089131205883049\\
599.89	0.00899901925277625\\
599.9	0.00908574090280562\\
599.91	0.00917329344842636\\
599.92	0.00926168487563191\\
599.93	0.00935092324710449\\
599.94	0.00944101670295079\\
599.95	0.00953197346144465\\
599.96	0.00962380181977694\\
599.97	0.00971651015481255\\
599.98	0.0098101069238547\\
599.99	0.00990460066541651\\
600	0.01\\
};
\addplot [color=red!75!mycolor17,solid,forget plot]
  table[row sep=crcr]{%
0.01	0\\
1.01	0\\
2.01	0\\
3.01	0\\
4.01	0\\
5.01	0\\
6.01	0\\
7.01	0\\
8.01	0\\
9.01	0\\
10.01	0\\
11.01	0\\
12.01	0\\
13.01	0\\
14.01	0\\
15.01	0\\
16.01	0\\
17.01	0\\
18.01	0\\
19.01	0\\
20.01	0\\
21.01	0\\
22.01	0\\
23.01	0\\
24.01	0\\
25.01	0\\
26.01	0\\
27.01	0\\
28.01	0\\
29.01	0\\
30.01	0\\
31.01	0\\
32.01	0\\
33.01	0\\
34.01	0\\
35.01	0\\
36.01	0\\
37.01	0\\
38.01	0\\
39.01	0\\
40.01	0\\
41.01	0\\
42.01	0\\
43.01	0\\
44.01	0\\
45.01	0\\
46.01	0\\
47.01	0\\
48.01	0\\
49.01	0\\
50.01	0\\
51.01	0\\
52.01	0\\
53.01	0\\
54.01	0\\
55.01	0\\
56.01	0\\
57.01	0\\
58.01	0\\
59.01	0\\
60.01	0\\
61.01	0\\
62.01	0\\
63.01	0\\
64.01	0\\
65.01	0\\
66.01	0\\
67.01	0\\
68.01	0\\
69.01	0\\
70.01	0\\
71.01	0\\
72.01	0\\
73.01	0\\
74.01	0\\
75.01	0\\
76.01	0\\
77.01	0\\
78.01	0\\
79.01	0\\
80.01	0\\
81.01	0\\
82.01	0\\
83.01	0\\
84.01	0\\
85.01	0\\
86.01	0\\
87.01	0\\
88.01	0\\
89.01	0\\
90.01	0\\
91.01	0\\
92.01	0\\
93.01	0\\
94.01	0\\
95.01	0\\
96.01	0\\
97.01	0\\
98.01	0\\
99.01	0\\
100.01	0\\
101.01	0\\
102.01	0\\
103.01	0\\
104.01	0\\
105.01	0\\
106.01	0\\
107.01	0\\
108.01	0\\
109.01	0\\
110.01	0\\
111.01	0\\
112.01	0\\
113.01	0\\
114.01	0\\
115.01	0\\
116.01	0\\
117.01	0\\
118.01	0\\
119.01	0\\
120.01	0\\
121.01	0\\
122.01	0\\
123.01	0\\
124.01	0\\
125.01	0\\
126.01	0\\
127.01	0\\
128.01	0\\
129.01	0\\
130.01	0\\
131.01	0\\
132.01	0\\
133.01	0\\
134.01	0\\
135.01	0\\
136.01	0\\
137.01	0\\
138.01	0\\
139.01	0\\
140.01	0\\
141.01	0\\
142.01	0\\
143.01	0\\
144.01	0\\
145.01	0\\
146.01	0\\
147.01	0\\
148.01	0\\
149.01	0\\
150.01	0\\
151.01	0\\
152.01	0\\
153.01	0\\
154.01	0\\
155.01	0\\
156.01	0\\
157.01	0\\
158.01	0\\
159.01	0\\
160.01	0\\
161.01	0\\
162.01	0\\
163.01	0\\
164.01	0\\
165.01	0\\
166.01	0\\
167.01	0\\
168.01	0\\
169.01	0\\
170.01	0\\
171.01	0\\
172.01	0\\
173.01	0\\
174.01	0\\
175.01	0\\
176.01	0\\
177.01	0\\
178.01	0\\
179.01	0\\
180.01	0\\
181.01	0\\
182.01	0\\
183.01	0\\
184.01	0\\
185.01	0\\
186.01	0\\
187.01	0\\
188.01	0\\
189.01	0\\
190.01	0\\
191.01	0\\
192.01	0\\
193.01	0\\
194.01	0\\
195.01	0\\
196.01	0\\
197.01	0\\
198.01	0\\
199.01	0\\
200.01	0\\
201.01	0\\
202.01	0\\
203.01	0\\
204.01	0\\
205.01	0\\
206.01	0\\
207.01	0\\
208.01	0\\
209.01	0\\
210.01	0\\
211.01	0\\
212.01	0\\
213.01	0\\
214.01	0\\
215.01	0\\
216.01	0\\
217.01	0\\
218.01	0\\
219.01	0\\
220.01	0\\
221.01	0\\
222.01	0\\
223.01	0\\
224.01	0\\
225.01	0\\
226.01	0\\
227.01	0\\
228.01	0\\
229.01	0\\
230.01	0\\
231.01	0\\
232.01	0\\
233.01	0\\
234.01	0\\
235.01	0\\
236.01	0\\
237.01	0\\
238.01	0\\
239.01	0\\
240.01	0\\
241.01	0\\
242.01	0\\
243.01	0\\
244.01	0\\
245.01	0\\
246.01	0\\
247.01	0\\
248.01	0\\
249.01	0\\
250.01	0\\
251.01	0\\
252.01	0\\
253.01	0\\
254.01	0\\
255.01	0\\
256.01	0\\
257.01	0\\
258.01	0\\
259.01	0\\
260.01	0\\
261.01	0\\
262.01	0\\
263.01	0\\
264.01	0\\
265.01	0\\
266.01	0\\
267.01	0\\
268.01	0\\
269.01	0\\
270.01	0\\
271.01	0\\
272.01	0\\
273.01	0\\
274.01	0\\
275.01	0\\
276.01	0\\
277.01	0\\
278.01	0\\
279.01	0\\
280.01	0\\
281.01	0\\
282.01	0\\
283.01	0\\
284.01	0\\
285.01	0\\
286.01	0\\
287.01	0\\
288.01	0\\
289.01	0\\
290.01	0\\
291.01	0\\
292.01	0\\
293.01	0\\
294.01	0\\
295.01	0\\
296.01	0\\
297.01	0\\
298.01	0\\
299.01	0\\
300.01	0\\
301.01	0\\
302.01	0\\
303.01	0\\
304.01	0\\
305.01	0\\
306.01	0\\
307.01	0\\
308.01	0\\
309.01	0\\
310.01	0\\
311.01	0\\
312.01	0\\
313.01	0\\
314.01	0\\
315.01	0\\
316.01	0\\
317.01	0\\
318.01	0\\
319.01	0\\
320.01	0\\
321.01	0\\
322.01	0\\
323.01	0\\
324.01	0\\
325.01	0\\
326.01	0\\
327.01	0\\
328.01	0\\
329.01	0\\
330.01	0\\
331.01	0\\
332.01	0\\
333.01	0\\
334.01	0\\
335.01	0\\
336.01	0\\
337.01	0\\
338.01	0\\
339.01	0\\
340.01	0\\
341.01	0\\
342.01	0\\
343.01	0\\
344.01	0\\
345.01	0\\
346.01	0\\
347.01	0\\
348.01	0\\
349.01	0\\
350.01	0\\
351.01	0\\
352.01	0\\
353.01	0\\
354.01	0\\
355.01	0\\
356.01	0\\
357.01	0\\
358.01	0\\
359.01	0\\
360.01	0\\
361.01	0\\
362.01	0\\
363.01	0\\
364.01	0\\
365.01	0\\
366.01	0\\
367.01	0\\
368.01	0\\
369.01	0\\
370.01	0\\
371.01	0\\
372.01	0\\
373.01	0\\
374.01	0\\
375.01	0\\
376.01	0\\
377.01	0\\
378.01	0\\
379.01	0\\
380.01	0\\
381.01	0\\
382.01	0\\
383.01	0\\
384.01	0\\
385.01	0\\
386.01	0\\
387.01	0\\
388.01	0\\
389.01	0\\
390.01	0\\
391.01	0\\
392.01	0\\
393.01	0\\
394.01	0\\
395.01	0\\
396.01	0\\
397.01	0\\
398.01	0\\
399.01	0\\
400.01	0\\
401.01	0\\
402.01	0\\
403.01	0\\
404.01	0\\
405.01	0\\
406.01	0\\
407.01	0\\
408.01	0\\
409.01	0\\
410.01	0\\
411.01	0\\
412.01	0\\
413.01	0\\
414.01	0\\
415.01	0\\
416.01	0\\
417.01	0\\
418.01	0\\
419.01	0\\
420.01	0\\
421.01	0\\
422.01	0\\
423.01	0\\
424.01	0\\
425.01	0\\
426.01	0\\
427.01	0\\
428.01	0\\
429.01	0\\
430.01	0\\
431.01	0\\
432.01	0\\
433.01	0\\
434.01	0\\
435.01	0\\
436.01	0\\
437.01	0\\
438.01	0\\
439.01	0\\
440.01	0\\
441.01	0\\
442.01	0\\
443.01	0\\
444.01	0\\
445.01	0\\
446.01	0\\
447.01	0\\
448.01	0\\
449.01	0\\
450.01	0\\
451.01	0\\
452.01	0\\
453.01	0\\
454.01	0\\
455.01	0\\
456.01	0\\
457.01	0\\
458.01	0\\
459.01	0\\
460.01	0\\
461.01	0\\
462.01	0\\
463.01	0\\
464.01	0\\
465.01	0\\
466.01	0\\
467.01	0\\
468.01	0\\
469.01	0\\
470.01	0\\
471.01	0\\
472.01	0\\
473.01	0\\
474.01	0\\
475.01	0\\
476.01	0\\
477.01	0\\
478.01	0\\
479.01	0\\
480.01	0\\
481.01	0\\
482.01	0\\
483.01	0\\
484.01	0\\
485.01	0\\
486.01	0\\
487.01	0\\
488.01	0\\
489.01	0\\
490.01	0\\
491.01	0\\
492.01	0\\
493.01	0\\
494.01	0\\
495.01	0\\
496.01	0\\
497.01	0\\
498.01	0\\
499.01	0\\
500.01	0\\
501.01	0\\
502.01	0\\
503.01	0\\
504.01	0\\
505.01	0\\
506.01	0\\
507.01	0\\
508.01	0\\
509.01	0\\
510.01	0\\
511.01	0\\
512.01	0\\
513.01	0\\
514.01	0\\
515.01	0\\
516.01	0\\
517.01	0\\
518.01	0\\
519.01	0\\
520.01	0\\
521.01	0\\
522.01	0\\
523.01	0\\
524.01	0\\
525.01	0\\
526.01	0\\
527.01	0\\
528.01	0\\
529.01	0\\
530.01	0\\
531.01	0\\
532.01	0\\
533.01	0\\
534.01	0\\
535.01	0\\
536.01	0\\
537.01	0\\
538.01	0\\
539.01	0\\
540.01	0\\
541.01	0\\
542.01	0\\
543.01	0\\
544.01	0\\
545.01	0\\
546.01	0\\
547.01	0\\
548.01	0\\
549.01	0\\
550.01	0\\
551.01	0\\
552.01	0\\
553.01	0\\
554.01	0\\
555.01	0\\
556.01	0\\
557.01	0\\
558.01	0\\
559.01	0\\
560.01	0\\
561.01	0\\
562.01	0\\
563.01	0\\
564.01	0\\
565.01	0\\
566.01	0\\
567.01	0\\
568.01	0\\
569.01	0\\
570.01	0\\
571.01	0\\
572.01	0\\
573.01	0\\
574.01	0\\
575.01	0\\
576.01	0\\
577.01	0\\
578.01	0\\
579.01	0\\
580.01	0\\
581.01	0\\
582.01	0\\
583.01	0\\
584.01	0\\
585.01	0\\
586.01	0\\
587.01	0\\
588.01	0\\
589.01	0\\
590.01	0\\
591.01	0\\
592.01	0\\
593.01	0\\
594.01	0\\
595.01	0\\
596.01	0\\
597.01	0.000480601427308949\\
598.01	0.0014383206818518\\
599.01	0.00385661257211624\\
599.02	0.00389414818835231\\
599.03	0.00393204146009085\\
599.04	0.00397029584140059\\
599.05	0.00400891481968844\\
599.06	0.00404790191602107\\
599.07	0.00408726068544946\\
599.08	0.00412699471733659\\
599.09	0.00416710763568849\\
599.1	0.00420760309948817\\
599.11	0.00424848480303298\\
599.12	0.00428975647627516\\
599.13	0.00433142188516568\\
599.14	0.00437348483200135\\
599.15	0.00441594915577537\\
599.16	0.00445881873253111\\
599.17	0.00450209747571945\\
599.18	0.00454578933655949\\
599.19	0.00458989830440273\\
599.2	0.00463442840710075\\
599.21	0.00467938371137651\\
599.22	0.00472476832319909\\
599.23	0.00477058638816223\\
599.24	0.00481684209186631\\
599.25	0.00486353966030413\\
599.26	0.00491068334955367\\
599.27	0.00495827744958437\\
599.28	0.0050063262915877\\
599.29	0.00505483424837289\\
599.3	0.00510380573476643\\
599.31	0.00515324520801537\\
599.32	0.00520315716819457\\
599.33	0.00525354615861775\\
599.34	0.0053044167662526\\
599.35	0.00535577362213973\\
599.36	0.0054076214018158\\
599.37	0.00545996482574054\\
599.38	0.00551280865972798\\
599.39	0.00556615771538182\\
599.4	0.00562001685053486\\
599.41	0.00567439096969275\\
599.42	0.00572928502448199\\
599.43	0.00578470401410217\\
599.44	0.00584065298578253\\
599.45	0.00589713703524306\\
599.46	0.00595416130715976\\
599.47	0.00601173099563462\\
599.48	0.00606985134466991\\
599.49	0.00612852764864715\\
599.5	0.00618776525281057\\
599.51	0.00624756955375528\\
599.52	0.00630794599992008\\
599.53	0.00636890009208502\\
599.54	0.00643043738387372\\
599.55	0.00649256348226046\\
599.56	0.00655528404808223\\
599.57	0.0066186047965556\\
599.58	0.00668253149779857\\
599.59	0.00674706997735747\\
599.6	0.0068122261167388\\
599.61	0.00687800585394632\\
599.62	0.00694441518402314\\
599.63	0.00701146015959912\\
599.64	0.00707914689144345\\
599.65	0.00714748154902253\\
599.66	0.00721647036106324\\
599.67	0.00728611961612158\\
599.68	0.00735643566315675\\
599.69	0.00742742491211078\\
599.7	0.00749909383449363\\
599.71	0.00757144896397399\\
599.72	0.0076444968969757\\
599.73	0.00771824429327989\\
599.74	0.00779269787663285\\
599.75	0.00786786443535982\\
599.76	0.00794375082298461\\
599.77	0.0080203639588551\\
599.78	0.00809771082877485\\
599.79	0.00817579848564071\\
599.8	0.00825463405008648\\
599.81	0.00833422471113285\\
599.82	0.00841457772684349\\
599.83	0.00849570042498746\\
599.84	0.00857760020370797\\
599.85	0.00866028453219752\\
599.86	0.00874376095137953\\
599.87	0.00882803707459653\\
599.88	0.0089131205883049\\
599.89	0.00899901925277625\\
599.9	0.00908574090280562\\
599.91	0.00917329344842636\\
599.92	0.0092616848756319\\
599.93	0.00935092324710449\\
599.94	0.00944101670295079\\
599.95	0.00953197346144465\\
599.96	0.00962380181977694\\
599.97	0.00971651015481255\\
599.98	0.0098101069238547\\
599.99	0.00990460066541651\\
600	0.01\\
};
\addplot [color=red!80!mycolor19,solid,forget plot]
  table[row sep=crcr]{%
0.01	0\\
1.01	0\\
2.01	0\\
3.01	0\\
4.01	0\\
5.01	0\\
6.01	0\\
7.01	0\\
8.01	0\\
9.01	0\\
10.01	0\\
11.01	0\\
12.01	0\\
13.01	0\\
14.01	0\\
15.01	0\\
16.01	0\\
17.01	0\\
18.01	0\\
19.01	0\\
20.01	0\\
21.01	0\\
22.01	0\\
23.01	0\\
24.01	0\\
25.01	0\\
26.01	0\\
27.01	0\\
28.01	0\\
29.01	0\\
30.01	0\\
31.01	0\\
32.01	0\\
33.01	0\\
34.01	0\\
35.01	0\\
36.01	0\\
37.01	0\\
38.01	0\\
39.01	0\\
40.01	0\\
41.01	0\\
42.01	0\\
43.01	0\\
44.01	0\\
45.01	0\\
46.01	0\\
47.01	0\\
48.01	0\\
49.01	0\\
50.01	0\\
51.01	0\\
52.01	0\\
53.01	0\\
54.01	0\\
55.01	0\\
56.01	0\\
57.01	0\\
58.01	0\\
59.01	0\\
60.01	0\\
61.01	0\\
62.01	0\\
63.01	0\\
64.01	0\\
65.01	0\\
66.01	0\\
67.01	0\\
68.01	0\\
69.01	0\\
70.01	0\\
71.01	0\\
72.01	0\\
73.01	0\\
74.01	0\\
75.01	0\\
76.01	0\\
77.01	0\\
78.01	0\\
79.01	0\\
80.01	0\\
81.01	0\\
82.01	0\\
83.01	0\\
84.01	0\\
85.01	0\\
86.01	0\\
87.01	0\\
88.01	0\\
89.01	0\\
90.01	0\\
91.01	0\\
92.01	0\\
93.01	0\\
94.01	0\\
95.01	0\\
96.01	0\\
97.01	0\\
98.01	0\\
99.01	0\\
100.01	0\\
101.01	0\\
102.01	0\\
103.01	0\\
104.01	0\\
105.01	0\\
106.01	0\\
107.01	0\\
108.01	0\\
109.01	0\\
110.01	0\\
111.01	0\\
112.01	0\\
113.01	0\\
114.01	0\\
115.01	0\\
116.01	0\\
117.01	0\\
118.01	0\\
119.01	0\\
120.01	0\\
121.01	0\\
122.01	0\\
123.01	0\\
124.01	0\\
125.01	0\\
126.01	0\\
127.01	0\\
128.01	0\\
129.01	0\\
130.01	0\\
131.01	0\\
132.01	0\\
133.01	0\\
134.01	0\\
135.01	0\\
136.01	0\\
137.01	0\\
138.01	0\\
139.01	0\\
140.01	0\\
141.01	0\\
142.01	0\\
143.01	0\\
144.01	0\\
145.01	0\\
146.01	0\\
147.01	0\\
148.01	0\\
149.01	0\\
150.01	0\\
151.01	0\\
152.01	0\\
153.01	0\\
154.01	0\\
155.01	0\\
156.01	0\\
157.01	0\\
158.01	0\\
159.01	0\\
160.01	0\\
161.01	0\\
162.01	0\\
163.01	0\\
164.01	0\\
165.01	0\\
166.01	0\\
167.01	0\\
168.01	0\\
169.01	0\\
170.01	0\\
171.01	0\\
172.01	0\\
173.01	0\\
174.01	0\\
175.01	0\\
176.01	0\\
177.01	0\\
178.01	0\\
179.01	0\\
180.01	0\\
181.01	0\\
182.01	0\\
183.01	0\\
184.01	0\\
185.01	0\\
186.01	0\\
187.01	0\\
188.01	0\\
189.01	0\\
190.01	0\\
191.01	0\\
192.01	0\\
193.01	0\\
194.01	0\\
195.01	0\\
196.01	0\\
197.01	0\\
198.01	0\\
199.01	0\\
200.01	0\\
201.01	0\\
202.01	0\\
203.01	0\\
204.01	0\\
205.01	0\\
206.01	0\\
207.01	0\\
208.01	0\\
209.01	0\\
210.01	0\\
211.01	0\\
212.01	0\\
213.01	0\\
214.01	0\\
215.01	0\\
216.01	0\\
217.01	0\\
218.01	0\\
219.01	0\\
220.01	0\\
221.01	0\\
222.01	0\\
223.01	0\\
224.01	0\\
225.01	0\\
226.01	0\\
227.01	0\\
228.01	0\\
229.01	0\\
230.01	0\\
231.01	0\\
232.01	0\\
233.01	0\\
234.01	0\\
235.01	0\\
236.01	0\\
237.01	0\\
238.01	0\\
239.01	0\\
240.01	0\\
241.01	0\\
242.01	0\\
243.01	0\\
244.01	0\\
245.01	0\\
246.01	0\\
247.01	0\\
248.01	0\\
249.01	0\\
250.01	0\\
251.01	0\\
252.01	0\\
253.01	0\\
254.01	0\\
255.01	0\\
256.01	0\\
257.01	0\\
258.01	0\\
259.01	0\\
260.01	0\\
261.01	0\\
262.01	0\\
263.01	0\\
264.01	0\\
265.01	0\\
266.01	0\\
267.01	0\\
268.01	0\\
269.01	0\\
270.01	0\\
271.01	0\\
272.01	0\\
273.01	0\\
274.01	0\\
275.01	0\\
276.01	0\\
277.01	0\\
278.01	0\\
279.01	0\\
280.01	0\\
281.01	0\\
282.01	0\\
283.01	0\\
284.01	0\\
285.01	0\\
286.01	0\\
287.01	0\\
288.01	0\\
289.01	0\\
290.01	0\\
291.01	0\\
292.01	0\\
293.01	0\\
294.01	0\\
295.01	0\\
296.01	0\\
297.01	0\\
298.01	0\\
299.01	0\\
300.01	0\\
301.01	0\\
302.01	0\\
303.01	0\\
304.01	0\\
305.01	0\\
306.01	0\\
307.01	0\\
308.01	0\\
309.01	0\\
310.01	0\\
311.01	0\\
312.01	0\\
313.01	0\\
314.01	0\\
315.01	0\\
316.01	0\\
317.01	0\\
318.01	0\\
319.01	0\\
320.01	0\\
321.01	0\\
322.01	0\\
323.01	0\\
324.01	0\\
325.01	0\\
326.01	0\\
327.01	0\\
328.01	0\\
329.01	0\\
330.01	0\\
331.01	0\\
332.01	0\\
333.01	0\\
334.01	0\\
335.01	0\\
336.01	0\\
337.01	0\\
338.01	0\\
339.01	0\\
340.01	0\\
341.01	0\\
342.01	0\\
343.01	0\\
344.01	0\\
345.01	0\\
346.01	0\\
347.01	0\\
348.01	0\\
349.01	0\\
350.01	0\\
351.01	0\\
352.01	0\\
353.01	0\\
354.01	0\\
355.01	0\\
356.01	0\\
357.01	0\\
358.01	0\\
359.01	0\\
360.01	0\\
361.01	0\\
362.01	0\\
363.01	0\\
364.01	0\\
365.01	0\\
366.01	0\\
367.01	0\\
368.01	0\\
369.01	0\\
370.01	0\\
371.01	0\\
372.01	0\\
373.01	0\\
374.01	0\\
375.01	0\\
376.01	0\\
377.01	0\\
378.01	0\\
379.01	0\\
380.01	0\\
381.01	0\\
382.01	0\\
383.01	0\\
384.01	0\\
385.01	0\\
386.01	0\\
387.01	0\\
388.01	0\\
389.01	0\\
390.01	0\\
391.01	0\\
392.01	0\\
393.01	0\\
394.01	0\\
395.01	0\\
396.01	0\\
397.01	0\\
398.01	0\\
399.01	0\\
400.01	0\\
401.01	0\\
402.01	0\\
403.01	0\\
404.01	0\\
405.01	0\\
406.01	0\\
407.01	0\\
408.01	0\\
409.01	0\\
410.01	0\\
411.01	0\\
412.01	0\\
413.01	0\\
414.01	0\\
415.01	0\\
416.01	0\\
417.01	0\\
418.01	0\\
419.01	0\\
420.01	0\\
421.01	0\\
422.01	0\\
423.01	0\\
424.01	0\\
425.01	0\\
426.01	0\\
427.01	0\\
428.01	0\\
429.01	0\\
430.01	0\\
431.01	0\\
432.01	0\\
433.01	0\\
434.01	0\\
435.01	0\\
436.01	0\\
437.01	0\\
438.01	0\\
439.01	0\\
440.01	0\\
441.01	0\\
442.01	0\\
443.01	0\\
444.01	0\\
445.01	0\\
446.01	0\\
447.01	0\\
448.01	0\\
449.01	0\\
450.01	0\\
451.01	0\\
452.01	0\\
453.01	0\\
454.01	0\\
455.01	0\\
456.01	0\\
457.01	0\\
458.01	0\\
459.01	0\\
460.01	0\\
461.01	0\\
462.01	0\\
463.01	0\\
464.01	0\\
465.01	0\\
466.01	0\\
467.01	0\\
468.01	0\\
469.01	0\\
470.01	0\\
471.01	0\\
472.01	0\\
473.01	0\\
474.01	0\\
475.01	0\\
476.01	0\\
477.01	0\\
478.01	0\\
479.01	0\\
480.01	0\\
481.01	0\\
482.01	0\\
483.01	0\\
484.01	0\\
485.01	0\\
486.01	0\\
487.01	0\\
488.01	0\\
489.01	0\\
490.01	0\\
491.01	0\\
492.01	0\\
493.01	0\\
494.01	0\\
495.01	0\\
496.01	0\\
497.01	0\\
498.01	0\\
499.01	0\\
500.01	0\\
501.01	0\\
502.01	0\\
503.01	0\\
504.01	0\\
505.01	0\\
506.01	0\\
507.01	0\\
508.01	0\\
509.01	0\\
510.01	0\\
511.01	0\\
512.01	0\\
513.01	0\\
514.01	0\\
515.01	0\\
516.01	0\\
517.01	0\\
518.01	0\\
519.01	0\\
520.01	0\\
521.01	0\\
522.01	0\\
523.01	0\\
524.01	0\\
525.01	0\\
526.01	0\\
527.01	0\\
528.01	0\\
529.01	0\\
530.01	0\\
531.01	0\\
532.01	0\\
533.01	0\\
534.01	0\\
535.01	0\\
536.01	0\\
537.01	0\\
538.01	0\\
539.01	0\\
540.01	0\\
541.01	0\\
542.01	0\\
543.01	0\\
544.01	0\\
545.01	0\\
546.01	0\\
547.01	0\\
548.01	0\\
549.01	0\\
550.01	0\\
551.01	0\\
552.01	0\\
553.01	0\\
554.01	0\\
555.01	0\\
556.01	0\\
557.01	0\\
558.01	0\\
559.01	0\\
560.01	0\\
561.01	0\\
562.01	0\\
563.01	0\\
564.01	0\\
565.01	0\\
566.01	0\\
567.01	0\\
568.01	0\\
569.01	0\\
570.01	0\\
571.01	0\\
572.01	0\\
573.01	0\\
574.01	0\\
575.01	0\\
576.01	0\\
577.01	0\\
578.01	0\\
579.01	0\\
580.01	0\\
581.01	0\\
582.01	0\\
583.01	0\\
584.01	0\\
585.01	0\\
586.01	0\\
587.01	0\\
588.01	0\\
589.01	0\\
590.01	0\\
591.01	0\\
592.01	0\\
593.01	0\\
594.01	0\\
595.01	0\\
596.01	0\\
597.01	0.000480804875715751\\
598.01	0.00143832068185185\\
599.01	0.00385661257211618\\
599.02	0.00389414818835225\\
599.03	0.0039320414600908\\
599.04	0.00397029584140053\\
599.05	0.00400891481968839\\
599.06	0.00404790191602102\\
599.07	0.00408726068544939\\
599.08	0.00412699471733653\\
599.09	0.00416710763568844\\
599.1	0.00420760309948812\\
599.11	0.00424848480303291\\
599.12	0.00428975647627511\\
599.13	0.00433142188516562\\
599.14	0.00437348483200131\\
599.15	0.00441594915577533\\
599.16	0.00445881873253107\\
599.17	0.00450209747571942\\
599.18	0.00454578933655946\\
599.19	0.00458989830440269\\
599.2	0.00463442840710071\\
599.21	0.00467938371137647\\
599.22	0.00472476832319907\\
599.23	0.0047705863881622\\
599.24	0.00481684209186628\\
599.25	0.00486353966030409\\
599.26	0.00491068334955363\\
599.27	0.00495827744958433\\
599.28	0.00500632629158766\\
599.29	0.00505483424837287\\
599.3	0.0051038057347664\\
599.31	0.00515324520801534\\
599.32	0.00520315716819453\\
599.33	0.00525354615861772\\
599.34	0.00530441676625257\\
599.35	0.0053557736221397\\
599.36	0.00540762140181576\\
599.37	0.0054599648257405\\
599.38	0.00551280865972795\\
599.39	0.0055661577153818\\
599.4	0.00562001685053483\\
599.41	0.00567439096969272\\
599.42	0.00572928502448197\\
599.43	0.00578470401410213\\
599.44	0.0058406529857825\\
599.45	0.00589713703524302\\
599.46	0.00595416130715973\\
599.47	0.00601173099563459\\
599.48	0.00606985134466988\\
599.49	0.00612852764864712\\
599.5	0.00618776525281054\\
599.51	0.00624756955375525\\
599.52	0.00630794599992006\\
599.53	0.006368900092085\\
599.54	0.0064304373838737\\
599.55	0.00649256348226045\\
599.56	0.00655528404808222\\
599.57	0.00661860479655559\\
599.58	0.00668253149779856\\
599.59	0.00674706997735745\\
599.6	0.00681222611673878\\
599.61	0.0068780058539463\\
599.62	0.00694441518402313\\
599.63	0.00701146015959912\\
599.64	0.00707914689144345\\
599.65	0.00714748154902253\\
599.66	0.00721647036106324\\
599.67	0.00728611961612158\\
599.68	0.00735643566315675\\
599.69	0.00742742491211078\\
599.7	0.00749909383449363\\
599.71	0.00757144896397399\\
599.72	0.00764449689697571\\
599.73	0.00771824429327988\\
599.74	0.00779269787663284\\
599.75	0.00786786443535982\\
599.76	0.0079437508229846\\
599.77	0.00802036395885509\\
599.78	0.00809771082877485\\
599.79	0.00817579848564071\\
599.8	0.00825463405008648\\
599.81	0.00833422471113284\\
599.82	0.00841457772684349\\
599.83	0.00849570042498746\\
599.84	0.00857760020370797\\
599.85	0.00866028453219752\\
599.86	0.00874376095137954\\
599.87	0.00882803707459654\\
599.88	0.0089131205883049\\
599.89	0.00899901925277625\\
599.9	0.00908574090280562\\
599.91	0.00917329344842635\\
599.92	0.0092616848756319\\
599.93	0.00935092324710449\\
599.94	0.00944101670295078\\
599.95	0.00953197346144465\\
599.96	0.00962380181977693\\
599.97	0.00971651015481255\\
599.98	0.0098101069238547\\
599.99	0.00990460066541651\\
600	0.01\\
};
\addplot [color=red,solid,forget plot]
  table[row sep=crcr]{%
0.01	0\\
1.01	0\\
2.01	0\\
3.01	0\\
4.01	0\\
5.01	0\\
6.01	0\\
7.01	0\\
8.01	0\\
9.01	0\\
10.01	0\\
11.01	0\\
12.01	0\\
13.01	0\\
14.01	0\\
15.01	0\\
16.01	0\\
17.01	0\\
18.01	0\\
19.01	0\\
20.01	0\\
21.01	0\\
22.01	0\\
23.01	0\\
24.01	0\\
25.01	0\\
26.01	0\\
27.01	0\\
28.01	0\\
29.01	0\\
30.01	0\\
31.01	0\\
32.01	0\\
33.01	0\\
34.01	0\\
35.01	0\\
36.01	0\\
37.01	0\\
38.01	0\\
39.01	0\\
40.01	0\\
41.01	0\\
42.01	0\\
43.01	0\\
44.01	0\\
45.01	0\\
46.01	0\\
47.01	0\\
48.01	0\\
49.01	0\\
50.01	0\\
51.01	0\\
52.01	0\\
53.01	0\\
54.01	0\\
55.01	0\\
56.01	0\\
57.01	0\\
58.01	0\\
59.01	0\\
60.01	0\\
61.01	0\\
62.01	0\\
63.01	0\\
64.01	0\\
65.01	0\\
66.01	0\\
67.01	0\\
68.01	0\\
69.01	0\\
70.01	0\\
71.01	0\\
72.01	0\\
73.01	0\\
74.01	0\\
75.01	0\\
76.01	0\\
77.01	0\\
78.01	0\\
79.01	0\\
80.01	0\\
81.01	0\\
82.01	0\\
83.01	0\\
84.01	0\\
85.01	0\\
86.01	0\\
87.01	0\\
88.01	0\\
89.01	0\\
90.01	0\\
91.01	0\\
92.01	0\\
93.01	0\\
94.01	0\\
95.01	0\\
96.01	0\\
97.01	0\\
98.01	0\\
99.01	0\\
100.01	0\\
101.01	0\\
102.01	0\\
103.01	0\\
104.01	0\\
105.01	0\\
106.01	0\\
107.01	0\\
108.01	0\\
109.01	0\\
110.01	0\\
111.01	0\\
112.01	0\\
113.01	0\\
114.01	0\\
115.01	0\\
116.01	0\\
117.01	0\\
118.01	0\\
119.01	0\\
120.01	0\\
121.01	0\\
122.01	0\\
123.01	0\\
124.01	0\\
125.01	0\\
126.01	0\\
127.01	0\\
128.01	0\\
129.01	0\\
130.01	0\\
131.01	0\\
132.01	0\\
133.01	0\\
134.01	0\\
135.01	0\\
136.01	0\\
137.01	0\\
138.01	0\\
139.01	0\\
140.01	0\\
141.01	0\\
142.01	0\\
143.01	0\\
144.01	0\\
145.01	0\\
146.01	0\\
147.01	0\\
148.01	0\\
149.01	0\\
150.01	0\\
151.01	0\\
152.01	0\\
153.01	0\\
154.01	0\\
155.01	0\\
156.01	0\\
157.01	0\\
158.01	0\\
159.01	0\\
160.01	0\\
161.01	0\\
162.01	0\\
163.01	0\\
164.01	0\\
165.01	0\\
166.01	0\\
167.01	0\\
168.01	0\\
169.01	0\\
170.01	0\\
171.01	0\\
172.01	0\\
173.01	0\\
174.01	0\\
175.01	0\\
176.01	0\\
177.01	0\\
178.01	0\\
179.01	0\\
180.01	0\\
181.01	0\\
182.01	0\\
183.01	0\\
184.01	0\\
185.01	0\\
186.01	0\\
187.01	0\\
188.01	0\\
189.01	0\\
190.01	0\\
191.01	0\\
192.01	0\\
193.01	0\\
194.01	0\\
195.01	0\\
196.01	0\\
197.01	0\\
198.01	0\\
199.01	0\\
200.01	0\\
201.01	0\\
202.01	0\\
203.01	0\\
204.01	0\\
205.01	0\\
206.01	0\\
207.01	0\\
208.01	0\\
209.01	0\\
210.01	0\\
211.01	0\\
212.01	0\\
213.01	0\\
214.01	0\\
215.01	0\\
216.01	0\\
217.01	0\\
218.01	0\\
219.01	0\\
220.01	0\\
221.01	0\\
222.01	0\\
223.01	0\\
224.01	0\\
225.01	0\\
226.01	0\\
227.01	0\\
228.01	0\\
229.01	0\\
230.01	0\\
231.01	0\\
232.01	0\\
233.01	0\\
234.01	0\\
235.01	0\\
236.01	0\\
237.01	0\\
238.01	0\\
239.01	0\\
240.01	0\\
241.01	0\\
242.01	0\\
243.01	0\\
244.01	0\\
245.01	0\\
246.01	0\\
247.01	0\\
248.01	0\\
249.01	0\\
250.01	0\\
251.01	0\\
252.01	0\\
253.01	0\\
254.01	0\\
255.01	0\\
256.01	0\\
257.01	0\\
258.01	0\\
259.01	0\\
260.01	0\\
261.01	0\\
262.01	0\\
263.01	0\\
264.01	0\\
265.01	0\\
266.01	0\\
267.01	0\\
268.01	0\\
269.01	0\\
270.01	0\\
271.01	0\\
272.01	0\\
273.01	0\\
274.01	0\\
275.01	0\\
276.01	0\\
277.01	0\\
278.01	0\\
279.01	0\\
280.01	0\\
281.01	0\\
282.01	0\\
283.01	0\\
284.01	0\\
285.01	0\\
286.01	0\\
287.01	0\\
288.01	0\\
289.01	0\\
290.01	0\\
291.01	0\\
292.01	0\\
293.01	0\\
294.01	0\\
295.01	0\\
296.01	0\\
297.01	0\\
298.01	0\\
299.01	0\\
300.01	0\\
301.01	0\\
302.01	0\\
303.01	0\\
304.01	0\\
305.01	0\\
306.01	0\\
307.01	0\\
308.01	0\\
309.01	0\\
310.01	0\\
311.01	0\\
312.01	0\\
313.01	0\\
314.01	0\\
315.01	0\\
316.01	0\\
317.01	0\\
318.01	0\\
319.01	0\\
320.01	0\\
321.01	0\\
322.01	0\\
323.01	0\\
324.01	0\\
325.01	0\\
326.01	0\\
327.01	0\\
328.01	0\\
329.01	0\\
330.01	0\\
331.01	0\\
332.01	0\\
333.01	0\\
334.01	0\\
335.01	0\\
336.01	0\\
337.01	0\\
338.01	0\\
339.01	0\\
340.01	0\\
341.01	0\\
342.01	0\\
343.01	0\\
344.01	0\\
345.01	0\\
346.01	0\\
347.01	0\\
348.01	0\\
349.01	0\\
350.01	0\\
351.01	0\\
352.01	0\\
353.01	0\\
354.01	0\\
355.01	0\\
356.01	0\\
357.01	0\\
358.01	0\\
359.01	0\\
360.01	0\\
361.01	0\\
362.01	0\\
363.01	0\\
364.01	0\\
365.01	0\\
366.01	0\\
367.01	0\\
368.01	0\\
369.01	0\\
370.01	0\\
371.01	0\\
372.01	0\\
373.01	0\\
374.01	0\\
375.01	0\\
376.01	0\\
377.01	0\\
378.01	0\\
379.01	0\\
380.01	0\\
381.01	0\\
382.01	0\\
383.01	0\\
384.01	0\\
385.01	0\\
386.01	0\\
387.01	0\\
388.01	0\\
389.01	0\\
390.01	0\\
391.01	0\\
392.01	0\\
393.01	0\\
394.01	0\\
395.01	0\\
396.01	0\\
397.01	0\\
398.01	0\\
399.01	0\\
400.01	0\\
401.01	0\\
402.01	0\\
403.01	0\\
404.01	0\\
405.01	0\\
406.01	0\\
407.01	0\\
408.01	0\\
409.01	0\\
410.01	0\\
411.01	0\\
412.01	0\\
413.01	0\\
414.01	0\\
415.01	0\\
416.01	0\\
417.01	0\\
418.01	0\\
419.01	0\\
420.01	0\\
421.01	0\\
422.01	0\\
423.01	0\\
424.01	0\\
425.01	0\\
426.01	0\\
427.01	0\\
428.01	0\\
429.01	0\\
430.01	0\\
431.01	0\\
432.01	0\\
433.01	0\\
434.01	0\\
435.01	0\\
436.01	0\\
437.01	0\\
438.01	0\\
439.01	0\\
440.01	0\\
441.01	0\\
442.01	0\\
443.01	0\\
444.01	0\\
445.01	0\\
446.01	0\\
447.01	0\\
448.01	0\\
449.01	0\\
450.01	0\\
451.01	0\\
452.01	0\\
453.01	0\\
454.01	0\\
455.01	0\\
456.01	0\\
457.01	0\\
458.01	0\\
459.01	0\\
460.01	0\\
461.01	0\\
462.01	0\\
463.01	0\\
464.01	0\\
465.01	0\\
466.01	0\\
467.01	0\\
468.01	0\\
469.01	0\\
470.01	0\\
471.01	0\\
472.01	0\\
473.01	0\\
474.01	0\\
475.01	0\\
476.01	0\\
477.01	0\\
478.01	0\\
479.01	0\\
480.01	0\\
481.01	0\\
482.01	0\\
483.01	0\\
484.01	0\\
485.01	0\\
486.01	0\\
487.01	0\\
488.01	0\\
489.01	0\\
490.01	0\\
491.01	0\\
492.01	0\\
493.01	0\\
494.01	0\\
495.01	0\\
496.01	0\\
497.01	0\\
498.01	0\\
499.01	0\\
500.01	0\\
501.01	0\\
502.01	0\\
503.01	0\\
504.01	0\\
505.01	0\\
506.01	0\\
507.01	0\\
508.01	0\\
509.01	0\\
510.01	0\\
511.01	0\\
512.01	0\\
513.01	0\\
514.01	0\\
515.01	0\\
516.01	0\\
517.01	0\\
518.01	0\\
519.01	0\\
520.01	0\\
521.01	0\\
522.01	0\\
523.01	0\\
524.01	0\\
525.01	0\\
526.01	0\\
527.01	0\\
528.01	0\\
529.01	0\\
530.01	0\\
531.01	0\\
532.01	0\\
533.01	0\\
534.01	0\\
535.01	0\\
536.01	0\\
537.01	0\\
538.01	0\\
539.01	0\\
540.01	0\\
541.01	0\\
542.01	0\\
543.01	0\\
544.01	0\\
545.01	0\\
546.01	0\\
547.01	0\\
548.01	0\\
549.01	0\\
550.01	0\\
551.01	0\\
552.01	0\\
553.01	0\\
554.01	0\\
555.01	0\\
556.01	0\\
557.01	0\\
558.01	0\\
559.01	0\\
560.01	0\\
561.01	0\\
562.01	0\\
563.01	0\\
564.01	0\\
565.01	0\\
566.01	0\\
567.01	0\\
568.01	0\\
569.01	0\\
570.01	0\\
571.01	0\\
572.01	0\\
573.01	0\\
574.01	0\\
575.01	0\\
576.01	0\\
577.01	0\\
578.01	0\\
579.01	0\\
580.01	0\\
581.01	0\\
582.01	0\\
583.01	0\\
584.01	0\\
585.01	0\\
586.01	0\\
587.01	0\\
588.01	0\\
589.01	0\\
590.01	0\\
591.01	0\\
592.01	0\\
593.01	0\\
594.01	0\\
595.01	0\\
596.01	0\\
597.01	0.000480974774395043\\
598.01	0.00143832068185183\\
599.01	0.00385661257211624\\
599.02	0.00389414818835231\\
599.03	0.00393204146009085\\
599.04	0.0039702958414006\\
599.05	0.00400891481968847\\
599.06	0.00404790191602111\\
599.07	0.00408726068544948\\
599.08	0.00412699471733663\\
599.09	0.00416710763568853\\
599.1	0.0042076030994882\\
599.11	0.00424848480303301\\
599.12	0.00428975647627519\\
599.13	0.00433142188516572\\
599.14	0.0043734848320014\\
599.15	0.00441594915577541\\
599.16	0.00445881873253114\\
599.17	0.00450209747571947\\
599.18	0.00454578933655952\\
599.19	0.00458989830440275\\
599.2	0.00463442840710077\\
599.21	0.00467938371137651\\
599.22	0.0047247683231991\\
599.23	0.00477058638816223\\
599.24	0.00481684209186631\\
599.25	0.00486353966030413\\
599.26	0.00491068334955368\\
599.27	0.00495827744958438\\
599.28	0.0050063262915877\\
599.29	0.0050548342483729\\
599.3	0.00510380573476643\\
599.31	0.00515324520801537\\
599.32	0.00520315716819457\\
599.33	0.00525354615861776\\
599.34	0.00530441676625259\\
599.35	0.00535577362213973\\
599.36	0.00540762140181579\\
599.37	0.00545996482574052\\
599.38	0.00551280865972798\\
599.39	0.00556615771538182\\
599.4	0.00562001685053486\\
599.41	0.00567439096969275\\
599.42	0.00572928502448199\\
599.43	0.00578470401410216\\
599.44	0.00584065298578253\\
599.45	0.00589713703524305\\
599.46	0.00595416130715976\\
599.47	0.00601173099563462\\
599.48	0.00606985134466992\\
599.49	0.00612852764864716\\
599.5	0.00618776525281058\\
599.51	0.00624756955375529\\
599.52	0.00630794599992009\\
599.53	0.00636890009208504\\
599.54	0.00643043738387374\\
599.55	0.00649256348226048\\
599.56	0.00655528404808225\\
599.57	0.00661860479655562\\
599.58	0.00668253149779859\\
599.59	0.00674706997735749\\
599.6	0.00681222611673882\\
599.61	0.00687800585394634\\
599.62	0.00694441518402316\\
599.63	0.00701146015959914\\
599.64	0.00707914689144347\\
599.65	0.00714748154902254\\
599.66	0.00721647036106325\\
599.67	0.00728611961612159\\
599.68	0.00735643566315677\\
599.69	0.0074274249121108\\
599.7	0.00749909383449365\\
599.71	0.00757144896397401\\
599.72	0.00764449689697572\\
599.73	0.0077182442932799\\
599.74	0.00779269787663286\\
599.75	0.00786786443535983\\
599.76	0.00794375082298461\\
599.77	0.0080203639588551\\
599.78	0.00809771082877486\\
599.79	0.00817579848564072\\
599.8	0.00825463405008648\\
599.81	0.00833422471113285\\
599.82	0.0084145777268435\\
599.83	0.00849570042498747\\
599.84	0.00857760020370798\\
599.85	0.00866028453219753\\
599.86	0.00874376095137954\\
599.87	0.00882803707459654\\
599.88	0.0089131205883049\\
599.89	0.00899901925277625\\
599.9	0.00908574090280562\\
599.91	0.00917329344842636\\
599.92	0.00926168487563191\\
599.93	0.00935092324710449\\
599.94	0.00944101670295078\\
599.95	0.00953197346144465\\
599.96	0.00962380181977693\\
599.97	0.00971651015481255\\
599.98	0.0098101069238547\\
599.99	0.00990460066541651\\
600	0.01\\
};
\addplot [color=mycolor20,solid,forget plot]
  table[row sep=crcr]{%
0.01	0\\
1.01	0\\
2.01	0\\
3.01	0\\
4.01	0\\
5.01	0\\
6.01	0\\
7.01	0\\
8.01	0\\
9.01	0\\
10.01	0\\
11.01	0\\
12.01	0\\
13.01	0\\
14.01	0\\
15.01	0\\
16.01	0\\
17.01	0\\
18.01	0\\
19.01	0\\
20.01	0\\
21.01	0\\
22.01	0\\
23.01	0\\
24.01	0\\
25.01	0\\
26.01	0\\
27.01	0\\
28.01	0\\
29.01	0\\
30.01	0\\
31.01	0\\
32.01	0\\
33.01	0\\
34.01	0\\
35.01	0\\
36.01	0\\
37.01	0\\
38.01	0\\
39.01	0\\
40.01	0\\
41.01	0\\
42.01	0\\
43.01	0\\
44.01	0\\
45.01	0\\
46.01	0\\
47.01	0\\
48.01	0\\
49.01	0\\
50.01	0\\
51.01	0\\
52.01	0\\
53.01	0\\
54.01	0\\
55.01	0\\
56.01	0\\
57.01	0\\
58.01	0\\
59.01	0\\
60.01	0\\
61.01	0\\
62.01	0\\
63.01	0\\
64.01	0\\
65.01	0\\
66.01	0\\
67.01	0\\
68.01	0\\
69.01	0\\
70.01	0\\
71.01	0\\
72.01	0\\
73.01	0\\
74.01	0\\
75.01	0\\
76.01	0\\
77.01	0\\
78.01	0\\
79.01	0\\
80.01	0\\
81.01	0\\
82.01	0\\
83.01	0\\
84.01	0\\
85.01	0\\
86.01	0\\
87.01	0\\
88.01	0\\
89.01	0\\
90.01	0\\
91.01	0\\
92.01	0\\
93.01	0\\
94.01	0\\
95.01	0\\
96.01	0\\
97.01	0\\
98.01	0\\
99.01	0\\
100.01	0\\
101.01	0\\
102.01	0\\
103.01	0\\
104.01	0\\
105.01	0\\
106.01	0\\
107.01	0\\
108.01	0\\
109.01	0\\
110.01	0\\
111.01	0\\
112.01	0\\
113.01	0\\
114.01	0\\
115.01	0\\
116.01	0\\
117.01	0\\
118.01	0\\
119.01	0\\
120.01	0\\
121.01	0\\
122.01	0\\
123.01	0\\
124.01	0\\
125.01	0\\
126.01	0\\
127.01	0\\
128.01	0\\
129.01	0\\
130.01	0\\
131.01	0\\
132.01	0\\
133.01	0\\
134.01	0\\
135.01	0\\
136.01	0\\
137.01	0\\
138.01	0\\
139.01	0\\
140.01	0\\
141.01	0\\
142.01	0\\
143.01	0\\
144.01	0\\
145.01	0\\
146.01	0\\
147.01	0\\
148.01	0\\
149.01	0\\
150.01	0\\
151.01	0\\
152.01	0\\
153.01	0\\
154.01	0\\
155.01	0\\
156.01	0\\
157.01	0\\
158.01	0\\
159.01	0\\
160.01	0\\
161.01	0\\
162.01	0\\
163.01	0\\
164.01	0\\
165.01	0\\
166.01	0\\
167.01	0\\
168.01	0\\
169.01	0\\
170.01	0\\
171.01	0\\
172.01	0\\
173.01	0\\
174.01	0\\
175.01	0\\
176.01	0\\
177.01	0\\
178.01	0\\
179.01	0\\
180.01	0\\
181.01	0\\
182.01	0\\
183.01	0\\
184.01	0\\
185.01	0\\
186.01	0\\
187.01	0\\
188.01	0\\
189.01	0\\
190.01	0\\
191.01	0\\
192.01	0\\
193.01	0\\
194.01	0\\
195.01	0\\
196.01	0\\
197.01	0\\
198.01	0\\
199.01	0\\
200.01	0\\
201.01	0\\
202.01	0\\
203.01	0\\
204.01	0\\
205.01	0\\
206.01	0\\
207.01	0\\
208.01	0\\
209.01	0\\
210.01	0\\
211.01	0\\
212.01	0\\
213.01	0\\
214.01	0\\
215.01	0\\
216.01	0\\
217.01	0\\
218.01	0\\
219.01	0\\
220.01	0\\
221.01	0\\
222.01	0\\
223.01	0\\
224.01	0\\
225.01	0\\
226.01	0\\
227.01	0\\
228.01	0\\
229.01	0\\
230.01	0\\
231.01	0\\
232.01	0\\
233.01	0\\
234.01	0\\
235.01	0\\
236.01	0\\
237.01	0\\
238.01	0\\
239.01	0\\
240.01	0\\
241.01	0\\
242.01	0\\
243.01	0\\
244.01	0\\
245.01	0\\
246.01	0\\
247.01	0\\
248.01	0\\
249.01	0\\
250.01	0\\
251.01	0\\
252.01	0\\
253.01	0\\
254.01	0\\
255.01	0\\
256.01	0\\
257.01	0\\
258.01	0\\
259.01	0\\
260.01	0\\
261.01	0\\
262.01	0\\
263.01	0\\
264.01	0\\
265.01	0\\
266.01	0\\
267.01	0\\
268.01	0\\
269.01	0\\
270.01	0\\
271.01	0\\
272.01	0\\
273.01	0\\
274.01	0\\
275.01	0\\
276.01	0\\
277.01	0\\
278.01	0\\
279.01	0\\
280.01	0\\
281.01	0\\
282.01	0\\
283.01	0\\
284.01	0\\
285.01	0\\
286.01	0\\
287.01	0\\
288.01	0\\
289.01	0\\
290.01	0\\
291.01	0\\
292.01	0\\
293.01	0\\
294.01	0\\
295.01	0\\
296.01	0\\
297.01	0\\
298.01	0\\
299.01	0\\
300.01	0\\
301.01	0\\
302.01	0\\
303.01	0\\
304.01	0\\
305.01	0\\
306.01	0\\
307.01	0\\
308.01	0\\
309.01	0\\
310.01	0\\
311.01	0\\
312.01	0\\
313.01	0\\
314.01	0\\
315.01	0\\
316.01	0\\
317.01	0\\
318.01	0\\
319.01	0\\
320.01	0\\
321.01	0\\
322.01	0\\
323.01	0\\
324.01	0\\
325.01	0\\
326.01	0\\
327.01	0\\
328.01	0\\
329.01	0\\
330.01	0\\
331.01	0\\
332.01	0\\
333.01	0\\
334.01	0\\
335.01	0\\
336.01	0\\
337.01	0\\
338.01	0\\
339.01	0\\
340.01	0\\
341.01	0\\
342.01	0\\
343.01	0\\
344.01	0\\
345.01	0\\
346.01	0\\
347.01	0\\
348.01	0\\
349.01	0\\
350.01	0\\
351.01	0\\
352.01	0\\
353.01	0\\
354.01	0\\
355.01	0\\
356.01	0\\
357.01	0\\
358.01	0\\
359.01	0\\
360.01	0\\
361.01	0\\
362.01	0\\
363.01	0\\
364.01	0\\
365.01	0\\
366.01	0\\
367.01	0\\
368.01	0\\
369.01	0\\
370.01	0\\
371.01	0\\
372.01	0\\
373.01	0\\
374.01	0\\
375.01	0\\
376.01	0\\
377.01	0\\
378.01	0\\
379.01	0\\
380.01	0\\
381.01	0\\
382.01	0\\
383.01	0\\
384.01	0\\
385.01	0\\
386.01	0\\
387.01	0\\
388.01	0\\
389.01	0\\
390.01	0\\
391.01	0\\
392.01	0\\
393.01	0\\
394.01	0\\
395.01	0\\
396.01	0\\
397.01	0\\
398.01	0\\
399.01	0\\
400.01	0\\
401.01	0\\
402.01	0\\
403.01	0\\
404.01	0\\
405.01	0\\
406.01	0\\
407.01	0\\
408.01	0\\
409.01	0\\
410.01	0\\
411.01	0\\
412.01	0\\
413.01	0\\
414.01	0\\
415.01	0\\
416.01	0\\
417.01	0\\
418.01	0\\
419.01	0\\
420.01	0\\
421.01	0\\
422.01	0\\
423.01	0\\
424.01	0\\
425.01	0\\
426.01	0\\
427.01	0\\
428.01	0\\
429.01	0\\
430.01	0\\
431.01	0\\
432.01	0\\
433.01	0\\
434.01	0\\
435.01	0\\
436.01	0\\
437.01	0\\
438.01	0\\
439.01	0\\
440.01	0\\
441.01	0\\
442.01	0\\
443.01	0\\
444.01	0\\
445.01	0\\
446.01	0\\
447.01	0\\
448.01	0\\
449.01	0\\
450.01	0\\
451.01	0\\
452.01	0\\
453.01	0\\
454.01	0\\
455.01	0\\
456.01	0\\
457.01	0\\
458.01	0\\
459.01	0\\
460.01	0\\
461.01	0\\
462.01	0\\
463.01	0\\
464.01	0\\
465.01	0\\
466.01	0\\
467.01	0\\
468.01	0\\
469.01	0\\
470.01	0\\
471.01	0\\
472.01	0\\
473.01	0\\
474.01	0\\
475.01	0\\
476.01	0\\
477.01	0\\
478.01	0\\
479.01	0\\
480.01	0\\
481.01	0\\
482.01	0\\
483.01	0\\
484.01	0\\
485.01	0\\
486.01	0\\
487.01	0\\
488.01	0\\
489.01	0\\
490.01	0\\
491.01	0\\
492.01	0\\
493.01	0\\
494.01	0\\
495.01	0\\
496.01	0\\
497.01	0\\
498.01	0\\
499.01	0\\
500.01	0\\
501.01	0\\
502.01	0\\
503.01	0\\
504.01	0\\
505.01	0\\
506.01	0\\
507.01	0\\
508.01	0\\
509.01	0\\
510.01	0\\
511.01	0\\
512.01	0\\
513.01	0\\
514.01	0\\
515.01	0\\
516.01	0\\
517.01	0\\
518.01	0\\
519.01	0\\
520.01	0\\
521.01	0\\
522.01	0\\
523.01	0\\
524.01	0\\
525.01	0\\
526.01	0\\
527.01	0\\
528.01	0\\
529.01	0\\
530.01	0\\
531.01	0\\
532.01	0\\
533.01	0\\
534.01	0\\
535.01	0\\
536.01	0\\
537.01	0\\
538.01	0\\
539.01	0\\
540.01	0\\
541.01	0\\
542.01	0\\
543.01	0\\
544.01	0\\
545.01	0\\
546.01	0\\
547.01	0\\
548.01	0\\
549.01	0\\
550.01	0\\
551.01	0\\
552.01	0\\
553.01	0\\
554.01	0\\
555.01	0\\
556.01	0\\
557.01	0\\
558.01	0\\
559.01	0\\
560.01	0\\
561.01	0\\
562.01	0\\
563.01	0\\
564.01	0\\
565.01	0\\
566.01	0\\
567.01	0\\
568.01	0\\
569.01	0\\
570.01	0\\
571.01	0\\
572.01	0\\
573.01	0\\
574.01	0\\
575.01	0\\
576.01	0\\
577.01	0\\
578.01	0\\
579.01	0\\
580.01	0\\
581.01	0\\
582.01	0\\
583.01	0\\
584.01	0\\
585.01	0\\
586.01	0\\
587.01	0\\
588.01	0\\
589.01	0\\
590.01	0\\
591.01	0\\
592.01	0\\
593.01	0\\
594.01	0\\
595.01	0\\
596.01	0\\
597.01	0.000481110898971432\\
598.01	0.00143832068185183\\
599.01	0.00385661257211628\\
599.02	0.00389414818835235\\
599.03	0.00393204146009088\\
599.04	0.00397029584140061\\
599.05	0.00400891481968847\\
599.06	0.0040479019160211\\
599.07	0.00408726068544947\\
599.08	0.00412699471733662\\
599.09	0.00416710763568851\\
599.1	0.00420760309948819\\
599.11	0.004248484803033\\
599.12	0.00428975647627518\\
599.13	0.00433142188516569\\
599.14	0.00437348483200137\\
599.15	0.00441594915577538\\
599.16	0.00445881873253112\\
599.17	0.00450209747571947\\
599.18	0.00454578933655951\\
599.19	0.00458989830440275\\
599.2	0.00463442840710077\\
599.21	0.00467938371137652\\
599.22	0.00472476832319911\\
599.23	0.00477058638816225\\
599.24	0.00481684209186631\\
599.25	0.00486353966030412\\
599.26	0.00491068334955366\\
599.27	0.00495827744958435\\
599.28	0.00500632629158769\\
599.29	0.00505483424837289\\
599.3	0.00510380573476643\\
599.31	0.00515324520801537\\
599.32	0.00520315716819457\\
599.33	0.00525354615861776\\
599.34	0.00530441676625261\\
599.35	0.00535577362213975\\
599.36	0.0054076214018158\\
599.37	0.00545996482574054\\
599.38	0.00551280865972799\\
599.39	0.00556615771538183\\
599.4	0.00562001685053486\\
599.41	0.00567439096969276\\
599.42	0.00572928502448199\\
599.43	0.00578470401410217\\
599.44	0.00584065298578253\\
599.45	0.00589713703524306\\
599.46	0.00595416130715976\\
599.47	0.00601173099563461\\
599.48	0.00606985134466991\\
599.49	0.00612852764864715\\
599.5	0.00618776525281058\\
599.51	0.00624756955375528\\
599.52	0.00630794599992008\\
599.53	0.00636890009208502\\
599.54	0.00643043738387371\\
599.55	0.00649256348226045\\
599.56	0.00655528404808223\\
599.57	0.00661860479655559\\
599.58	0.00668253149779856\\
599.59	0.00674706997735745\\
599.6	0.00681222611673879\\
599.61	0.00687800585394631\\
599.62	0.00694441518402313\\
599.63	0.00701146015959912\\
599.64	0.00707914689144344\\
599.65	0.00714748154902252\\
599.66	0.00721647036106322\\
599.67	0.00728611961612156\\
599.68	0.00735643566315675\\
599.69	0.00742742491211077\\
599.7	0.00749909383449363\\
599.71	0.00757144896397399\\
599.72	0.00764449689697571\\
599.73	0.00771824429327989\\
599.74	0.00779269787663284\\
599.75	0.00786786443535982\\
599.76	0.0079437508229846\\
599.77	0.00802036395885509\\
599.78	0.00809771082877485\\
599.79	0.0081757984856407\\
599.8	0.00825463405008647\\
599.81	0.00833422471113284\\
599.82	0.00841457772684348\\
599.83	0.00849570042498746\\
599.84	0.00857760020370798\\
599.85	0.00866028453219752\\
599.86	0.00874376095137953\\
599.87	0.00882803707459654\\
599.88	0.00891312058830489\\
599.89	0.00899901925277625\\
599.9	0.00908574090280562\\
599.91	0.00917329344842636\\
599.92	0.00926168487563191\\
599.93	0.00935092324710449\\
599.94	0.00944101670295078\\
599.95	0.00953197346144465\\
599.96	0.00962380181977694\\
599.97	0.00971651015481255\\
599.98	0.0098101069238547\\
599.99	0.00990460066541651\\
600	0.01\\
};
\addplot [color=mycolor21,solid,forget plot]
  table[row sep=crcr]{%
0.01	0\\
1.01	0\\
2.01	0\\
3.01	0\\
4.01	0\\
5.01	0\\
6.01	0\\
7.01	0\\
8.01	0\\
9.01	0\\
10.01	0\\
11.01	0\\
12.01	0\\
13.01	0\\
14.01	0\\
15.01	0\\
16.01	0\\
17.01	0\\
18.01	0\\
19.01	0\\
20.01	0\\
21.01	0\\
22.01	0\\
23.01	0\\
24.01	0\\
25.01	0\\
26.01	0\\
27.01	0\\
28.01	0\\
29.01	0\\
30.01	0\\
31.01	0\\
32.01	0\\
33.01	0\\
34.01	0\\
35.01	0\\
36.01	0\\
37.01	0\\
38.01	0\\
39.01	0\\
40.01	0\\
41.01	0\\
42.01	0\\
43.01	0\\
44.01	0\\
45.01	0\\
46.01	0\\
47.01	0\\
48.01	0\\
49.01	0\\
50.01	0\\
51.01	0\\
52.01	0\\
53.01	0\\
54.01	0\\
55.01	0\\
56.01	0\\
57.01	0\\
58.01	0\\
59.01	0\\
60.01	0\\
61.01	0\\
62.01	0\\
63.01	0\\
64.01	0\\
65.01	0\\
66.01	0\\
67.01	0\\
68.01	0\\
69.01	0\\
70.01	0\\
71.01	0\\
72.01	0\\
73.01	0\\
74.01	0\\
75.01	0\\
76.01	0\\
77.01	0\\
78.01	0\\
79.01	0\\
80.01	0\\
81.01	0\\
82.01	0\\
83.01	0\\
84.01	0\\
85.01	0\\
86.01	0\\
87.01	0\\
88.01	0\\
89.01	0\\
90.01	0\\
91.01	0\\
92.01	0\\
93.01	0\\
94.01	0\\
95.01	0\\
96.01	0\\
97.01	0\\
98.01	0\\
99.01	0\\
100.01	0\\
101.01	0\\
102.01	0\\
103.01	0\\
104.01	0\\
105.01	0\\
106.01	0\\
107.01	0\\
108.01	0\\
109.01	0\\
110.01	0\\
111.01	0\\
112.01	0\\
113.01	0\\
114.01	0\\
115.01	0\\
116.01	0\\
117.01	0\\
118.01	0\\
119.01	0\\
120.01	0\\
121.01	0\\
122.01	0\\
123.01	0\\
124.01	0\\
125.01	0\\
126.01	0\\
127.01	0\\
128.01	0\\
129.01	0\\
130.01	0\\
131.01	0\\
132.01	0\\
133.01	0\\
134.01	0\\
135.01	0\\
136.01	0\\
137.01	0\\
138.01	0\\
139.01	0\\
140.01	0\\
141.01	0\\
142.01	0\\
143.01	0\\
144.01	0\\
145.01	0\\
146.01	0\\
147.01	0\\
148.01	0\\
149.01	0\\
150.01	0\\
151.01	0\\
152.01	0\\
153.01	0\\
154.01	0\\
155.01	0\\
156.01	0\\
157.01	0\\
158.01	0\\
159.01	0\\
160.01	0\\
161.01	0\\
162.01	0\\
163.01	0\\
164.01	0\\
165.01	0\\
166.01	0\\
167.01	0\\
168.01	0\\
169.01	0\\
170.01	0\\
171.01	0\\
172.01	0\\
173.01	0\\
174.01	0\\
175.01	0\\
176.01	0\\
177.01	0\\
178.01	0\\
179.01	0\\
180.01	0\\
181.01	0\\
182.01	0\\
183.01	0\\
184.01	0\\
185.01	0\\
186.01	0\\
187.01	0\\
188.01	0\\
189.01	0\\
190.01	0\\
191.01	0\\
192.01	0\\
193.01	0\\
194.01	0\\
195.01	0\\
196.01	0\\
197.01	0\\
198.01	0\\
199.01	0\\
200.01	0\\
201.01	0\\
202.01	0\\
203.01	0\\
204.01	0\\
205.01	0\\
206.01	0\\
207.01	0\\
208.01	0\\
209.01	0\\
210.01	0\\
211.01	0\\
212.01	0\\
213.01	0\\
214.01	0\\
215.01	0\\
216.01	0\\
217.01	0\\
218.01	0\\
219.01	0\\
220.01	0\\
221.01	0\\
222.01	0\\
223.01	0\\
224.01	0\\
225.01	0\\
226.01	0\\
227.01	0\\
228.01	0\\
229.01	0\\
230.01	0\\
231.01	0\\
232.01	0\\
233.01	0\\
234.01	0\\
235.01	0\\
236.01	0\\
237.01	0\\
238.01	0\\
239.01	0\\
240.01	0\\
241.01	0\\
242.01	0\\
243.01	0\\
244.01	0\\
245.01	0\\
246.01	0\\
247.01	0\\
248.01	0\\
249.01	0\\
250.01	0\\
251.01	0\\
252.01	0\\
253.01	0\\
254.01	0\\
255.01	0\\
256.01	0\\
257.01	0\\
258.01	0\\
259.01	0\\
260.01	0\\
261.01	0\\
262.01	0\\
263.01	0\\
264.01	0\\
265.01	0\\
266.01	0\\
267.01	0\\
268.01	0\\
269.01	0\\
270.01	0\\
271.01	0\\
272.01	0\\
273.01	0\\
274.01	0\\
275.01	0\\
276.01	0\\
277.01	0\\
278.01	0\\
279.01	0\\
280.01	0\\
281.01	0\\
282.01	0\\
283.01	0\\
284.01	0\\
285.01	0\\
286.01	0\\
287.01	0\\
288.01	0\\
289.01	0\\
290.01	0\\
291.01	0\\
292.01	0\\
293.01	0\\
294.01	0\\
295.01	0\\
296.01	0\\
297.01	0\\
298.01	0\\
299.01	0\\
300.01	0\\
301.01	0\\
302.01	0\\
303.01	0\\
304.01	0\\
305.01	0\\
306.01	0\\
307.01	0\\
308.01	0\\
309.01	0\\
310.01	0\\
311.01	0\\
312.01	0\\
313.01	0\\
314.01	0\\
315.01	0\\
316.01	0\\
317.01	0\\
318.01	0\\
319.01	0\\
320.01	0\\
321.01	0\\
322.01	0\\
323.01	0\\
324.01	0\\
325.01	0\\
326.01	0\\
327.01	0\\
328.01	0\\
329.01	0\\
330.01	0\\
331.01	0\\
332.01	0\\
333.01	0\\
334.01	0\\
335.01	0\\
336.01	0\\
337.01	0\\
338.01	0\\
339.01	0\\
340.01	0\\
341.01	0\\
342.01	0\\
343.01	0\\
344.01	0\\
345.01	0\\
346.01	0\\
347.01	0\\
348.01	0\\
349.01	0\\
350.01	0\\
351.01	0\\
352.01	0\\
353.01	0\\
354.01	0\\
355.01	0\\
356.01	0\\
357.01	0\\
358.01	0\\
359.01	0\\
360.01	0\\
361.01	0\\
362.01	0\\
363.01	0\\
364.01	0\\
365.01	0\\
366.01	0\\
367.01	0\\
368.01	0\\
369.01	0\\
370.01	0\\
371.01	0\\
372.01	0\\
373.01	0\\
374.01	0\\
375.01	0\\
376.01	0\\
377.01	0\\
378.01	0\\
379.01	0\\
380.01	0\\
381.01	0\\
382.01	0\\
383.01	0\\
384.01	0\\
385.01	0\\
386.01	0\\
387.01	0\\
388.01	0\\
389.01	0\\
390.01	0\\
391.01	0\\
392.01	0\\
393.01	0\\
394.01	0\\
395.01	0\\
396.01	0\\
397.01	0\\
398.01	0\\
399.01	0\\
400.01	0\\
401.01	0\\
402.01	0\\
403.01	0\\
404.01	0\\
405.01	0\\
406.01	0\\
407.01	0\\
408.01	0\\
409.01	0\\
410.01	0\\
411.01	0\\
412.01	0\\
413.01	0\\
414.01	0\\
415.01	0\\
416.01	0\\
417.01	0\\
418.01	0\\
419.01	0\\
420.01	0\\
421.01	0\\
422.01	0\\
423.01	0\\
424.01	0\\
425.01	0\\
426.01	0\\
427.01	0\\
428.01	0\\
429.01	0\\
430.01	0\\
431.01	0\\
432.01	0\\
433.01	0\\
434.01	0\\
435.01	0\\
436.01	0\\
437.01	0\\
438.01	0\\
439.01	0\\
440.01	0\\
441.01	0\\
442.01	0\\
443.01	0\\
444.01	0\\
445.01	0\\
446.01	0\\
447.01	0\\
448.01	0\\
449.01	0\\
450.01	0\\
451.01	0\\
452.01	0\\
453.01	0\\
454.01	0\\
455.01	0\\
456.01	0\\
457.01	0\\
458.01	0\\
459.01	0\\
460.01	0\\
461.01	0\\
462.01	0\\
463.01	0\\
464.01	0\\
465.01	0\\
466.01	0\\
467.01	0\\
468.01	0\\
469.01	0\\
470.01	0\\
471.01	0\\
472.01	0\\
473.01	0\\
474.01	0\\
475.01	0\\
476.01	0\\
477.01	0\\
478.01	0\\
479.01	0\\
480.01	0\\
481.01	0\\
482.01	0\\
483.01	0\\
484.01	0\\
485.01	0\\
486.01	0\\
487.01	0\\
488.01	0\\
489.01	0\\
490.01	0\\
491.01	0\\
492.01	0\\
493.01	0\\
494.01	0\\
495.01	0\\
496.01	0\\
497.01	0\\
498.01	0\\
499.01	0\\
500.01	0\\
501.01	0\\
502.01	0\\
503.01	0\\
504.01	0\\
505.01	0\\
506.01	0\\
507.01	0\\
508.01	0\\
509.01	0\\
510.01	0\\
511.01	0\\
512.01	0\\
513.01	0\\
514.01	0\\
515.01	0\\
516.01	0\\
517.01	0\\
518.01	0\\
519.01	0\\
520.01	0\\
521.01	0\\
522.01	0\\
523.01	0\\
524.01	0\\
525.01	0\\
526.01	0\\
527.01	0\\
528.01	0\\
529.01	0\\
530.01	0\\
531.01	0\\
532.01	0\\
533.01	0\\
534.01	0\\
535.01	0\\
536.01	0\\
537.01	0\\
538.01	0\\
539.01	0\\
540.01	0\\
541.01	0\\
542.01	0\\
543.01	0\\
544.01	0\\
545.01	0\\
546.01	0\\
547.01	0\\
548.01	0\\
549.01	0\\
550.01	0\\
551.01	0\\
552.01	0\\
553.01	0\\
554.01	0\\
555.01	0\\
556.01	0\\
557.01	0\\
558.01	0\\
559.01	0\\
560.01	0\\
561.01	0\\
562.01	0\\
563.01	0\\
564.01	0\\
565.01	0\\
566.01	0\\
567.01	0\\
568.01	0\\
569.01	0\\
570.01	0\\
571.01	0\\
572.01	0\\
573.01	0\\
574.01	0\\
575.01	0\\
576.01	0\\
577.01	0\\
578.01	0\\
579.01	0\\
580.01	0\\
581.01	0\\
582.01	0\\
583.01	0\\
584.01	0\\
585.01	0\\
586.01	0\\
587.01	0\\
588.01	0\\
589.01	0\\
590.01	0\\
591.01	0\\
592.01	0\\
593.01	0\\
594.01	0\\
595.01	0\\
596.01	0\\
597.01	0.000481187488961005\\
598.01	0.00143832068185176\\
599.01	0.00385661257211614\\
599.02	0.00389414818835221\\
599.03	0.00393204146009075\\
599.04	0.00397029584140049\\
599.05	0.00400891481968836\\
599.06	0.004047901916021\\
599.07	0.00408726068544937\\
599.08	0.00412699471733652\\
599.09	0.00416710763568842\\
599.1	0.00420760309948809\\
599.11	0.0042484848030329\\
599.12	0.00428975647627509\\
599.13	0.00433142188516562\\
599.14	0.00437348483200131\\
599.15	0.00441594915577533\\
599.16	0.00445881873253107\\
599.17	0.0045020974757194\\
599.18	0.00454578933655944\\
599.19	0.00458989830440268\\
599.2	0.0046344284071007\\
599.21	0.00467938371137645\\
599.22	0.00472476832319906\\
599.23	0.0047705863881622\\
599.24	0.00481684209186628\\
599.25	0.0048635396603041\\
599.26	0.00491068334955364\\
599.27	0.00495827744958434\\
599.28	0.00500632629158766\\
599.29	0.00505483424837286\\
599.3	0.00510380573476639\\
599.31	0.00515324520801533\\
599.32	0.00520315716819451\\
599.33	0.0052535461586177\\
599.34	0.00530441676625255\\
599.35	0.00535577362213968\\
599.36	0.00540762140181575\\
599.37	0.00545996482574049\\
599.38	0.00551280865972795\\
599.39	0.00556615771538178\\
599.4	0.00562001685053481\\
599.41	0.0056743909696927\\
599.42	0.00572928502448196\\
599.43	0.00578470401410212\\
599.44	0.0058406529857825\\
599.45	0.00589713703524301\\
599.46	0.00595416130715971\\
599.47	0.00601173099563457\\
599.48	0.00606985134466987\\
599.49	0.00612852764864711\\
599.5	0.00618776525281053\\
599.51	0.00624756955375524\\
599.52	0.00630794599992004\\
599.53	0.006368900092085\\
599.54	0.00643043738387369\\
599.55	0.00649256348226044\\
599.56	0.00655528404808221\\
599.57	0.00661860479655558\\
599.58	0.00668253149779856\\
599.59	0.00674706997735745\\
599.6	0.00681222611673878\\
599.61	0.0068780058539463\\
599.62	0.00694441518402313\\
599.63	0.00701146015959912\\
599.64	0.00707914689144345\\
599.65	0.00714748154902253\\
599.66	0.00721647036106324\\
599.67	0.00728611961612158\\
599.68	0.00735643566315675\\
599.69	0.00742742491211078\\
599.7	0.00749909383449363\\
599.71	0.00757144896397399\\
599.72	0.00764449689697571\\
599.73	0.00771824429327989\\
599.74	0.00779269787663285\\
599.75	0.00786786443535983\\
599.76	0.00794375082298461\\
599.77	0.0080203639588551\\
599.78	0.00809771082877486\\
599.79	0.00817579848564071\\
599.8	0.00825463405008649\\
599.81	0.00833422471113285\\
599.82	0.00841457772684349\\
599.83	0.00849570042498747\\
599.84	0.00857760020370797\\
599.85	0.00866028453219752\\
599.86	0.00874376095137953\\
599.87	0.00882803707459654\\
599.88	0.0089131205883049\\
599.89	0.00899901925277625\\
599.9	0.00908574090280562\\
599.91	0.00917329344842636\\
599.92	0.0092616848756319\\
599.93	0.00935092324710449\\
599.94	0.00944101670295078\\
599.95	0.00953197346144465\\
599.96	0.00962380181977693\\
599.97	0.00971651015481255\\
599.98	0.0098101069238547\\
599.99	0.00990460066541651\\
600	0.01\\
};
\addplot [color=black!20!mycolor21,solid,forget plot]
  table[row sep=crcr]{%
0.01	0\\
1.01	0\\
2.01	0\\
3.01	0\\
4.01	0\\
5.01	0\\
6.01	0\\
7.01	0\\
8.01	0\\
9.01	0\\
10.01	0\\
11.01	0\\
12.01	0\\
13.01	0\\
14.01	0\\
15.01	0\\
16.01	0\\
17.01	0\\
18.01	0\\
19.01	0\\
20.01	0\\
21.01	0\\
22.01	0\\
23.01	0\\
24.01	0\\
25.01	0\\
26.01	0\\
27.01	0\\
28.01	0\\
29.01	0\\
30.01	0\\
31.01	0\\
32.01	0\\
33.01	0\\
34.01	0\\
35.01	0\\
36.01	0\\
37.01	0\\
38.01	0\\
39.01	0\\
40.01	0\\
41.01	0\\
42.01	0\\
43.01	0\\
44.01	0\\
45.01	0\\
46.01	0\\
47.01	0\\
48.01	0\\
49.01	0\\
50.01	0\\
51.01	0\\
52.01	0\\
53.01	0\\
54.01	0\\
55.01	0\\
56.01	0\\
57.01	0\\
58.01	0\\
59.01	0\\
60.01	0\\
61.01	0\\
62.01	0\\
63.01	0\\
64.01	0\\
65.01	0\\
66.01	0\\
67.01	0\\
68.01	0\\
69.01	0\\
70.01	0\\
71.01	0\\
72.01	0\\
73.01	0\\
74.01	0\\
75.01	0\\
76.01	0\\
77.01	0\\
78.01	0\\
79.01	0\\
80.01	0\\
81.01	0\\
82.01	0\\
83.01	0\\
84.01	0\\
85.01	0\\
86.01	0\\
87.01	0\\
88.01	0\\
89.01	0\\
90.01	0\\
91.01	0\\
92.01	0\\
93.01	0\\
94.01	0\\
95.01	0\\
96.01	0\\
97.01	0\\
98.01	0\\
99.01	0\\
100.01	0\\
101.01	0\\
102.01	0\\
103.01	0\\
104.01	0\\
105.01	0\\
106.01	0\\
107.01	0\\
108.01	0\\
109.01	0\\
110.01	0\\
111.01	0\\
112.01	0\\
113.01	0\\
114.01	0\\
115.01	0\\
116.01	0\\
117.01	0\\
118.01	0\\
119.01	0\\
120.01	0\\
121.01	0\\
122.01	0\\
123.01	0\\
124.01	0\\
125.01	0\\
126.01	0\\
127.01	0\\
128.01	0\\
129.01	0\\
130.01	0\\
131.01	0\\
132.01	0\\
133.01	0\\
134.01	0\\
135.01	0\\
136.01	0\\
137.01	0\\
138.01	0\\
139.01	0\\
140.01	0\\
141.01	0\\
142.01	0\\
143.01	0\\
144.01	0\\
145.01	0\\
146.01	0\\
147.01	0\\
148.01	0\\
149.01	0\\
150.01	0\\
151.01	0\\
152.01	0\\
153.01	0\\
154.01	0\\
155.01	0\\
156.01	0\\
157.01	0\\
158.01	0\\
159.01	0\\
160.01	0\\
161.01	0\\
162.01	0\\
163.01	0\\
164.01	0\\
165.01	0\\
166.01	0\\
167.01	0\\
168.01	0\\
169.01	0\\
170.01	0\\
171.01	0\\
172.01	0\\
173.01	0\\
174.01	0\\
175.01	0\\
176.01	0\\
177.01	0\\
178.01	0\\
179.01	0\\
180.01	0\\
181.01	0\\
182.01	0\\
183.01	0\\
184.01	0\\
185.01	0\\
186.01	0\\
187.01	0\\
188.01	0\\
189.01	0\\
190.01	0\\
191.01	0\\
192.01	0\\
193.01	0\\
194.01	0\\
195.01	0\\
196.01	0\\
197.01	0\\
198.01	0\\
199.01	0\\
200.01	0\\
201.01	0\\
202.01	0\\
203.01	0\\
204.01	0\\
205.01	0\\
206.01	0\\
207.01	0\\
208.01	0\\
209.01	0\\
210.01	0\\
211.01	0\\
212.01	0\\
213.01	0\\
214.01	0\\
215.01	0\\
216.01	0\\
217.01	0\\
218.01	0\\
219.01	0\\
220.01	0\\
221.01	0\\
222.01	0\\
223.01	0\\
224.01	0\\
225.01	0\\
226.01	0\\
227.01	0\\
228.01	0\\
229.01	0\\
230.01	0\\
231.01	0\\
232.01	0\\
233.01	0\\
234.01	0\\
235.01	0\\
236.01	0\\
237.01	0\\
238.01	0\\
239.01	0\\
240.01	0\\
241.01	0\\
242.01	0\\
243.01	0\\
244.01	0\\
245.01	0\\
246.01	0\\
247.01	0\\
248.01	0\\
249.01	0\\
250.01	0\\
251.01	0\\
252.01	0\\
253.01	0\\
254.01	0\\
255.01	0\\
256.01	0\\
257.01	0\\
258.01	0\\
259.01	0\\
260.01	0\\
261.01	0\\
262.01	0\\
263.01	0\\
264.01	0\\
265.01	0\\
266.01	0\\
267.01	0\\
268.01	0\\
269.01	0\\
270.01	0\\
271.01	0\\
272.01	0\\
273.01	0\\
274.01	0\\
275.01	0\\
276.01	0\\
277.01	0\\
278.01	0\\
279.01	0\\
280.01	0\\
281.01	0\\
282.01	0\\
283.01	0\\
284.01	0\\
285.01	0\\
286.01	0\\
287.01	0\\
288.01	0\\
289.01	0\\
290.01	0\\
291.01	0\\
292.01	0\\
293.01	0\\
294.01	0\\
295.01	0\\
296.01	0\\
297.01	0\\
298.01	0\\
299.01	0\\
300.01	0\\
301.01	0\\
302.01	0\\
303.01	0\\
304.01	0\\
305.01	0\\
306.01	0\\
307.01	0\\
308.01	0\\
309.01	0\\
310.01	0\\
311.01	0\\
312.01	0\\
313.01	0\\
314.01	0\\
315.01	0\\
316.01	0\\
317.01	0\\
318.01	0\\
319.01	0\\
320.01	0\\
321.01	0\\
322.01	0\\
323.01	0\\
324.01	0\\
325.01	0\\
326.01	0\\
327.01	0\\
328.01	0\\
329.01	0\\
330.01	0\\
331.01	0\\
332.01	0\\
333.01	0\\
334.01	0\\
335.01	0\\
336.01	0\\
337.01	0\\
338.01	0\\
339.01	0\\
340.01	0\\
341.01	0\\
342.01	0\\
343.01	0\\
344.01	0\\
345.01	0\\
346.01	0\\
347.01	0\\
348.01	0\\
349.01	0\\
350.01	0\\
351.01	0\\
352.01	0\\
353.01	0\\
354.01	0\\
355.01	0\\
356.01	0\\
357.01	0\\
358.01	0\\
359.01	0\\
360.01	0\\
361.01	0\\
362.01	0\\
363.01	0\\
364.01	0\\
365.01	0\\
366.01	0\\
367.01	0\\
368.01	0\\
369.01	0\\
370.01	0\\
371.01	0\\
372.01	0\\
373.01	0\\
374.01	0\\
375.01	0\\
376.01	0\\
377.01	0\\
378.01	0\\
379.01	0\\
380.01	0\\
381.01	0\\
382.01	0\\
383.01	0\\
384.01	0\\
385.01	0\\
386.01	0\\
387.01	0\\
388.01	0\\
389.01	0\\
390.01	0\\
391.01	0\\
392.01	0\\
393.01	0\\
394.01	0\\
395.01	0\\
396.01	0\\
397.01	0\\
398.01	0\\
399.01	0\\
400.01	0\\
401.01	0\\
402.01	0\\
403.01	0\\
404.01	0\\
405.01	0\\
406.01	0\\
407.01	0\\
408.01	0\\
409.01	0\\
410.01	0\\
411.01	0\\
412.01	0\\
413.01	0\\
414.01	0\\
415.01	0\\
416.01	0\\
417.01	0\\
418.01	0\\
419.01	0\\
420.01	0\\
421.01	0\\
422.01	0\\
423.01	0\\
424.01	0\\
425.01	0\\
426.01	0\\
427.01	0\\
428.01	0\\
429.01	0\\
430.01	0\\
431.01	0\\
432.01	0\\
433.01	0\\
434.01	0\\
435.01	0\\
436.01	0\\
437.01	0\\
438.01	0\\
439.01	0\\
440.01	0\\
441.01	0\\
442.01	0\\
443.01	0\\
444.01	0\\
445.01	0\\
446.01	0\\
447.01	0\\
448.01	0\\
449.01	0\\
450.01	0\\
451.01	0\\
452.01	0\\
453.01	0\\
454.01	0\\
455.01	0\\
456.01	0\\
457.01	0\\
458.01	0\\
459.01	0\\
460.01	0\\
461.01	0\\
462.01	0\\
463.01	0\\
464.01	0\\
465.01	0\\
466.01	0\\
467.01	0\\
468.01	0\\
469.01	0\\
470.01	0\\
471.01	0\\
472.01	0\\
473.01	0\\
474.01	0\\
475.01	0\\
476.01	0\\
477.01	0\\
478.01	0\\
479.01	0\\
480.01	0\\
481.01	0\\
482.01	0\\
483.01	0\\
484.01	0\\
485.01	0\\
486.01	0\\
487.01	0\\
488.01	0\\
489.01	0\\
490.01	0\\
491.01	0\\
492.01	0\\
493.01	0\\
494.01	0\\
495.01	0\\
496.01	0\\
497.01	0\\
498.01	0\\
499.01	0\\
500.01	0\\
501.01	0\\
502.01	0\\
503.01	0\\
504.01	0\\
505.01	0\\
506.01	0\\
507.01	0\\
508.01	0\\
509.01	0\\
510.01	0\\
511.01	0\\
512.01	0\\
513.01	0\\
514.01	0\\
515.01	0\\
516.01	0\\
517.01	0\\
518.01	0\\
519.01	0\\
520.01	0\\
521.01	0\\
522.01	0\\
523.01	0\\
524.01	0\\
525.01	0\\
526.01	0\\
527.01	0\\
528.01	0\\
529.01	0\\
530.01	0\\
531.01	0\\
532.01	0\\
533.01	0\\
534.01	0\\
535.01	0\\
536.01	0\\
537.01	0\\
538.01	0\\
539.01	0\\
540.01	0\\
541.01	0\\
542.01	0\\
543.01	0\\
544.01	0\\
545.01	0\\
546.01	0\\
547.01	0\\
548.01	0\\
549.01	0\\
550.01	0\\
551.01	0\\
552.01	0\\
553.01	0\\
554.01	0\\
555.01	0\\
556.01	0\\
557.01	0\\
558.01	0\\
559.01	0\\
560.01	0\\
561.01	0\\
562.01	0\\
563.01	0\\
564.01	0\\
565.01	0\\
566.01	0\\
567.01	0\\
568.01	0\\
569.01	0\\
570.01	0\\
571.01	0\\
572.01	0\\
573.01	0\\
574.01	0\\
575.01	0\\
576.01	0\\
577.01	0\\
578.01	0\\
579.01	0\\
580.01	0\\
581.01	0\\
582.01	0\\
583.01	0\\
584.01	0\\
585.01	0\\
586.01	0\\
587.01	0\\
588.01	0\\
589.01	0\\
590.01	0\\
591.01	0\\
592.01	0\\
593.01	0\\
594.01	0\\
595.01	0\\
596.01	0\\
597.01	0.000481249826739166\\
598.01	0.00143832068185196\\
599.01	0.00385661257211625\\
599.02	0.00389414818835232\\
599.03	0.00393204146009087\\
599.04	0.0039702958414006\\
599.05	0.00400891481968846\\
599.06	0.00404790191602108\\
599.07	0.00408726068544946\\
599.08	0.0041269947173366\\
599.09	0.00416710763568851\\
599.1	0.00420760309948819\\
599.11	0.00424848480303298\\
599.12	0.00428975647627516\\
599.13	0.00433142188516568\\
599.14	0.00437348483200135\\
599.15	0.00441594915577537\\
599.16	0.00445881873253111\\
599.17	0.00450209747571946\\
599.18	0.00454578933655951\\
599.19	0.00458989830440275\\
599.2	0.00463442840710078\\
599.21	0.00467938371137652\\
599.22	0.00472476832319911\\
599.23	0.00477058638816225\\
599.24	0.00481684209186631\\
599.25	0.00486353966030413\\
599.26	0.00491068334955368\\
599.27	0.00495827744958437\\
599.28	0.0050063262915877\\
599.29	0.0050548342483729\\
599.3	0.00510380573476643\\
599.31	0.00515324520801537\\
599.32	0.00520315716819457\\
599.33	0.00525354615861774\\
599.34	0.00530441676625258\\
599.35	0.00535577362213972\\
599.36	0.00540762140181579\\
599.37	0.00545996482574053\\
599.38	0.00551280865972797\\
599.39	0.00556615771538181\\
599.4	0.00562001685053486\\
599.41	0.00567439096969276\\
599.42	0.00572928502448199\\
599.43	0.00578470401410215\\
599.44	0.00584065298578251\\
599.45	0.00589713703524303\\
599.46	0.00595416130715974\\
599.47	0.00601173099563461\\
599.48	0.0060698513446699\\
599.49	0.00612852764864714\\
599.5	0.00618776525281057\\
599.51	0.00624756955375527\\
599.52	0.00630794599992007\\
599.53	0.00636890009208502\\
599.54	0.00643043738387371\\
599.55	0.00649256348226046\\
599.56	0.00655528404808223\\
599.57	0.00661860479655561\\
599.58	0.00668253149779858\\
599.59	0.00674706997735747\\
599.6	0.0068122261167388\\
599.61	0.00687800585394632\\
599.62	0.00694441518402315\\
599.63	0.00701146015959913\\
599.64	0.00707914689144346\\
599.65	0.00714748154902253\\
599.66	0.00721647036106324\\
599.67	0.00728611961612158\\
599.68	0.00735643566315675\\
599.69	0.00742742491211078\\
599.7	0.00749909383449363\\
599.71	0.00757144896397399\\
599.72	0.0076444968969757\\
599.73	0.00771824429327989\\
599.74	0.00779269787663284\\
599.75	0.00786786443535982\\
599.76	0.0079437508229846\\
599.77	0.0080203639588551\\
599.78	0.00809771082877485\\
599.79	0.00817579848564071\\
599.8	0.00825463405008648\\
599.81	0.00833422471113284\\
599.82	0.00841457772684348\\
599.83	0.00849570042498746\\
599.84	0.00857760020370798\\
599.85	0.00866028453219752\\
599.86	0.00874376095137953\\
599.87	0.00882803707459653\\
599.88	0.00891312058830489\\
599.89	0.00899901925277624\\
599.9	0.00908574090280562\\
599.91	0.00917329344842635\\
599.92	0.0092616848756319\\
599.93	0.00935092324710449\\
599.94	0.00944101670295078\\
599.95	0.00953197346144465\\
599.96	0.00962380181977694\\
599.97	0.00971651015481255\\
599.98	0.0098101069238547\\
599.99	0.00990460066541651\\
600	0.01\\
};
\addplot [color=black!50!mycolor20,solid,forget plot]
  table[row sep=crcr]{%
0.01	0\\
1.01	0\\
2.01	0\\
3.01	0\\
4.01	0\\
5.01	0\\
6.01	0\\
7.01	0\\
8.01	0\\
9.01	0\\
10.01	0\\
11.01	0\\
12.01	0\\
13.01	0\\
14.01	0\\
15.01	0\\
16.01	0\\
17.01	0\\
18.01	0\\
19.01	0\\
20.01	0\\
21.01	0\\
22.01	0\\
23.01	0\\
24.01	0\\
25.01	0\\
26.01	0\\
27.01	0\\
28.01	0\\
29.01	0\\
30.01	0\\
31.01	0\\
32.01	0\\
33.01	0\\
34.01	0\\
35.01	0\\
36.01	0\\
37.01	0\\
38.01	0\\
39.01	0\\
40.01	0\\
41.01	0\\
42.01	0\\
43.01	0\\
44.01	0\\
45.01	0\\
46.01	0\\
47.01	0\\
48.01	0\\
49.01	0\\
50.01	0\\
51.01	0\\
52.01	0\\
53.01	0\\
54.01	0\\
55.01	0\\
56.01	0\\
57.01	0\\
58.01	0\\
59.01	0\\
60.01	0\\
61.01	0\\
62.01	0\\
63.01	0\\
64.01	0\\
65.01	0\\
66.01	0\\
67.01	0\\
68.01	0\\
69.01	0\\
70.01	0\\
71.01	0\\
72.01	0\\
73.01	0\\
74.01	0\\
75.01	0\\
76.01	0\\
77.01	0\\
78.01	0\\
79.01	0\\
80.01	0\\
81.01	0\\
82.01	0\\
83.01	0\\
84.01	0\\
85.01	0\\
86.01	0\\
87.01	0\\
88.01	0\\
89.01	0\\
90.01	0\\
91.01	0\\
92.01	0\\
93.01	0\\
94.01	0\\
95.01	0\\
96.01	0\\
97.01	0\\
98.01	0\\
99.01	0\\
100.01	0\\
101.01	0\\
102.01	0\\
103.01	0\\
104.01	0\\
105.01	0\\
106.01	0\\
107.01	0\\
108.01	0\\
109.01	0\\
110.01	0\\
111.01	0\\
112.01	0\\
113.01	0\\
114.01	0\\
115.01	0\\
116.01	0\\
117.01	0\\
118.01	0\\
119.01	0\\
120.01	0\\
121.01	0\\
122.01	0\\
123.01	0\\
124.01	0\\
125.01	0\\
126.01	0\\
127.01	0\\
128.01	0\\
129.01	0\\
130.01	0\\
131.01	0\\
132.01	0\\
133.01	0\\
134.01	0\\
135.01	0\\
136.01	0\\
137.01	0\\
138.01	0\\
139.01	0\\
140.01	0\\
141.01	0\\
142.01	0\\
143.01	0\\
144.01	0\\
145.01	0\\
146.01	0\\
147.01	0\\
148.01	0\\
149.01	0\\
150.01	0\\
151.01	0\\
152.01	0\\
153.01	0\\
154.01	0\\
155.01	0\\
156.01	0\\
157.01	0\\
158.01	0\\
159.01	0\\
160.01	0\\
161.01	0\\
162.01	0\\
163.01	0\\
164.01	0\\
165.01	0\\
166.01	0\\
167.01	0\\
168.01	0\\
169.01	0\\
170.01	0\\
171.01	0\\
172.01	0\\
173.01	0\\
174.01	0\\
175.01	0\\
176.01	0\\
177.01	0\\
178.01	0\\
179.01	0\\
180.01	0\\
181.01	0\\
182.01	0\\
183.01	0\\
184.01	0\\
185.01	0\\
186.01	0\\
187.01	0\\
188.01	0\\
189.01	0\\
190.01	0\\
191.01	0\\
192.01	0\\
193.01	0\\
194.01	0\\
195.01	0\\
196.01	0\\
197.01	0\\
198.01	0\\
199.01	0\\
200.01	0\\
201.01	0\\
202.01	0\\
203.01	0\\
204.01	0\\
205.01	0\\
206.01	0\\
207.01	0\\
208.01	0\\
209.01	0\\
210.01	0\\
211.01	0\\
212.01	0\\
213.01	0\\
214.01	0\\
215.01	0\\
216.01	0\\
217.01	0\\
218.01	0\\
219.01	0\\
220.01	0\\
221.01	0\\
222.01	0\\
223.01	0\\
224.01	0\\
225.01	0\\
226.01	0\\
227.01	0\\
228.01	0\\
229.01	0\\
230.01	0\\
231.01	0\\
232.01	0\\
233.01	0\\
234.01	0\\
235.01	0\\
236.01	0\\
237.01	0\\
238.01	0\\
239.01	0\\
240.01	0\\
241.01	0\\
242.01	0\\
243.01	0\\
244.01	0\\
245.01	0\\
246.01	0\\
247.01	0\\
248.01	0\\
249.01	0\\
250.01	0\\
251.01	0\\
252.01	0\\
253.01	0\\
254.01	0\\
255.01	0\\
256.01	0\\
257.01	0\\
258.01	0\\
259.01	0\\
260.01	0\\
261.01	0\\
262.01	0\\
263.01	0\\
264.01	0\\
265.01	0\\
266.01	0\\
267.01	0\\
268.01	0\\
269.01	0\\
270.01	0\\
271.01	0\\
272.01	0\\
273.01	0\\
274.01	0\\
275.01	0\\
276.01	0\\
277.01	0\\
278.01	0\\
279.01	0\\
280.01	0\\
281.01	0\\
282.01	0\\
283.01	0\\
284.01	0\\
285.01	0\\
286.01	0\\
287.01	0\\
288.01	0\\
289.01	0\\
290.01	0\\
291.01	0\\
292.01	0\\
293.01	0\\
294.01	0\\
295.01	0\\
296.01	0\\
297.01	0\\
298.01	0\\
299.01	0\\
300.01	0\\
301.01	0\\
302.01	0\\
303.01	0\\
304.01	0\\
305.01	0\\
306.01	0\\
307.01	0\\
308.01	0\\
309.01	0\\
310.01	0\\
311.01	0\\
312.01	0\\
313.01	0\\
314.01	0\\
315.01	0\\
316.01	0\\
317.01	0\\
318.01	0\\
319.01	0\\
320.01	0\\
321.01	0\\
322.01	0\\
323.01	0\\
324.01	0\\
325.01	0\\
326.01	0\\
327.01	0\\
328.01	0\\
329.01	0\\
330.01	0\\
331.01	0\\
332.01	0\\
333.01	0\\
334.01	0\\
335.01	0\\
336.01	0\\
337.01	0\\
338.01	0\\
339.01	0\\
340.01	0\\
341.01	0\\
342.01	0\\
343.01	0\\
344.01	0\\
345.01	0\\
346.01	0\\
347.01	0\\
348.01	0\\
349.01	0\\
350.01	0\\
351.01	0\\
352.01	0\\
353.01	0\\
354.01	0\\
355.01	0\\
356.01	0\\
357.01	0\\
358.01	0\\
359.01	0\\
360.01	0\\
361.01	0\\
362.01	0\\
363.01	0\\
364.01	0\\
365.01	0\\
366.01	0\\
367.01	0\\
368.01	0\\
369.01	0\\
370.01	0\\
371.01	0\\
372.01	0\\
373.01	0\\
374.01	0\\
375.01	0\\
376.01	0\\
377.01	0\\
378.01	0\\
379.01	0\\
380.01	0\\
381.01	0\\
382.01	0\\
383.01	0\\
384.01	0\\
385.01	0\\
386.01	0\\
387.01	0\\
388.01	0\\
389.01	0\\
390.01	0\\
391.01	0\\
392.01	0\\
393.01	0\\
394.01	0\\
395.01	0\\
396.01	0\\
397.01	0\\
398.01	0\\
399.01	0\\
400.01	0\\
401.01	0\\
402.01	0\\
403.01	0\\
404.01	0\\
405.01	0\\
406.01	0\\
407.01	0\\
408.01	0\\
409.01	0\\
410.01	0\\
411.01	0\\
412.01	0\\
413.01	0\\
414.01	0\\
415.01	0\\
416.01	0\\
417.01	0\\
418.01	0\\
419.01	0\\
420.01	0\\
421.01	0\\
422.01	0\\
423.01	0\\
424.01	0\\
425.01	0\\
426.01	0\\
427.01	0\\
428.01	0\\
429.01	0\\
430.01	0\\
431.01	0\\
432.01	0\\
433.01	0\\
434.01	0\\
435.01	0\\
436.01	0\\
437.01	0\\
438.01	0\\
439.01	0\\
440.01	0\\
441.01	0\\
442.01	0\\
443.01	0\\
444.01	0\\
445.01	0\\
446.01	0\\
447.01	0\\
448.01	0\\
449.01	0\\
450.01	0\\
451.01	0\\
452.01	0\\
453.01	0\\
454.01	0\\
455.01	0\\
456.01	0\\
457.01	0\\
458.01	0\\
459.01	0\\
460.01	0\\
461.01	0\\
462.01	0\\
463.01	0\\
464.01	0\\
465.01	0\\
466.01	0\\
467.01	0\\
468.01	0\\
469.01	0\\
470.01	0\\
471.01	0\\
472.01	0\\
473.01	0\\
474.01	0\\
475.01	0\\
476.01	0\\
477.01	0\\
478.01	0\\
479.01	0\\
480.01	0\\
481.01	0\\
482.01	0\\
483.01	0\\
484.01	0\\
485.01	0\\
486.01	0\\
487.01	0\\
488.01	0\\
489.01	0\\
490.01	0\\
491.01	0\\
492.01	0\\
493.01	0\\
494.01	0\\
495.01	0\\
496.01	0\\
497.01	0\\
498.01	0\\
499.01	0\\
500.01	0\\
501.01	0\\
502.01	0\\
503.01	0\\
504.01	0\\
505.01	0\\
506.01	0\\
507.01	0\\
508.01	0\\
509.01	0\\
510.01	0\\
511.01	0\\
512.01	0\\
513.01	0\\
514.01	0\\
515.01	0\\
516.01	0\\
517.01	0\\
518.01	0\\
519.01	0\\
520.01	0\\
521.01	0\\
522.01	0\\
523.01	0\\
524.01	0\\
525.01	0\\
526.01	0\\
527.01	0\\
528.01	0\\
529.01	0\\
530.01	0\\
531.01	0\\
532.01	0\\
533.01	0\\
534.01	0\\
535.01	0\\
536.01	0\\
537.01	0\\
538.01	0\\
539.01	0\\
540.01	0\\
541.01	0\\
542.01	0\\
543.01	0\\
544.01	0\\
545.01	0\\
546.01	0\\
547.01	0\\
548.01	0\\
549.01	0\\
550.01	0\\
551.01	0\\
552.01	0\\
553.01	0\\
554.01	0\\
555.01	0\\
556.01	0\\
557.01	0\\
558.01	0\\
559.01	0\\
560.01	0\\
561.01	0\\
562.01	0\\
563.01	0\\
564.01	0\\
565.01	0\\
566.01	0\\
567.01	0\\
568.01	0\\
569.01	0\\
570.01	0\\
571.01	0\\
572.01	0\\
573.01	0\\
574.01	0\\
575.01	0\\
576.01	0\\
577.01	0\\
578.01	0\\
579.01	0\\
580.01	0\\
581.01	0\\
582.01	0\\
583.01	0\\
584.01	0\\
585.01	0\\
586.01	0\\
587.01	0\\
588.01	0\\
589.01	0\\
590.01	0\\
591.01	0\\
592.01	0\\
593.01	0\\
594.01	0\\
595.01	0\\
596.01	0\\
597.01	0.000481304278336731\\
598.01	0.0014383206818519\\
599.01	0.00385661257211628\\
599.02	0.00389414818835235\\
599.03	0.00393204146009089\\
599.04	0.00397029584140063\\
599.05	0.00400891481968849\\
599.06	0.00404790191602111\\
599.07	0.00408726068544948\\
599.08	0.00412699471733663\\
599.09	0.00416710763568853\\
599.1	0.00420760309948821\\
599.11	0.00424848480303303\\
599.12	0.00428975647627521\\
599.13	0.00433142188516572\\
599.14	0.0043734848320014\\
599.15	0.00441594915577541\\
599.16	0.00445881873253114\\
599.17	0.00450209747571947\\
599.18	0.00454578933655951\\
599.19	0.00458989830440273\\
599.2	0.00463442840710075\\
599.21	0.00467938371137651\\
599.22	0.0047247683231991\\
599.23	0.00477058638816223\\
599.24	0.00481684209186631\\
599.25	0.00486353966030412\\
599.26	0.00491068334955366\\
599.27	0.00495827744958437\\
599.28	0.00500632629158769\\
599.29	0.00505483424837289\\
599.3	0.00510380573476643\\
599.31	0.00515324520801538\\
599.32	0.00520315716819458\\
599.33	0.00525354615861777\\
599.34	0.00530441676625261\\
599.35	0.00535577362213975\\
599.36	0.00540762140181582\\
599.37	0.00545996482574056\\
599.38	0.005512808659728\\
599.39	0.00556615771538184\\
599.4	0.00562001685053487\\
599.41	0.00567439096969276\\
599.42	0.005729285024482\\
599.43	0.00578470401410217\\
599.44	0.00584065298578254\\
599.45	0.00589713703524307\\
599.46	0.00595416130715977\\
599.47	0.00601173099563464\\
599.48	0.00606985134466993\\
599.49	0.00612852764864717\\
599.5	0.00618776525281059\\
599.51	0.00624756955375529\\
599.52	0.00630794599992009\\
599.53	0.00636890009208503\\
599.54	0.00643043738387374\\
599.55	0.00649256348226047\\
599.56	0.00655528404808225\\
599.57	0.00661860479655561\\
599.58	0.00668253149779858\\
599.59	0.00674706997735747\\
599.6	0.0068122261167388\\
599.61	0.00687800585394632\\
599.62	0.00694441518402315\\
599.63	0.00701146015959914\\
599.64	0.00707914689144346\\
599.65	0.00714748154902254\\
599.66	0.00721647036106324\\
599.67	0.00728611961612158\\
599.68	0.00735643566315676\\
599.69	0.00742742491211078\\
599.7	0.00749909383449363\\
599.71	0.007571448963974\\
599.72	0.00764449689697572\\
599.73	0.00771824429327989\\
599.74	0.00779269787663285\\
599.75	0.00786786443535983\\
599.76	0.00794375082298461\\
599.77	0.0080203639588551\\
599.78	0.00809771082877486\\
599.79	0.00817579848564072\\
599.8	0.00825463405008649\\
599.81	0.00833422471113286\\
599.82	0.00841457772684349\\
599.83	0.00849570042498747\\
599.84	0.00857760020370798\\
599.85	0.00866028453219753\\
599.86	0.00874376095137954\\
599.87	0.00882803707459654\\
599.88	0.0089131205883049\\
599.89	0.00899901925277626\\
599.9	0.00908574090280562\\
599.91	0.00917329344842636\\
599.92	0.00926168487563191\\
599.93	0.00935092324710449\\
599.94	0.00944101670295079\\
599.95	0.00953197346144465\\
599.96	0.00962380181977693\\
599.97	0.00971651015481255\\
599.98	0.0098101069238547\\
599.99	0.00990460066541651\\
600	0.01\\
};
\addplot [color=black!60!mycolor21,solid,forget plot]
  table[row sep=crcr]{%
0.01	0\\
1.01	0\\
2.01	0\\
3.01	0\\
4.01	0\\
5.01	0\\
6.01	0\\
7.01	0\\
8.01	0\\
9.01	0\\
10.01	0\\
11.01	0\\
12.01	0\\
13.01	0\\
14.01	0\\
15.01	0\\
16.01	0\\
17.01	0\\
18.01	0\\
19.01	0\\
20.01	0\\
21.01	0\\
22.01	0\\
23.01	0\\
24.01	0\\
25.01	0\\
26.01	0\\
27.01	0\\
28.01	0\\
29.01	0\\
30.01	0\\
31.01	0\\
32.01	0\\
33.01	0\\
34.01	0\\
35.01	0\\
36.01	0\\
37.01	0\\
38.01	0\\
39.01	0\\
40.01	0\\
41.01	0\\
42.01	0\\
43.01	0\\
44.01	0\\
45.01	0\\
46.01	0\\
47.01	0\\
48.01	0\\
49.01	0\\
50.01	0\\
51.01	0\\
52.01	0\\
53.01	0\\
54.01	0\\
55.01	0\\
56.01	0\\
57.01	0\\
58.01	0\\
59.01	0\\
60.01	0\\
61.01	0\\
62.01	0\\
63.01	0\\
64.01	0\\
65.01	0\\
66.01	0\\
67.01	0\\
68.01	0\\
69.01	0\\
70.01	0\\
71.01	0\\
72.01	0\\
73.01	0\\
74.01	0\\
75.01	0\\
76.01	0\\
77.01	0\\
78.01	0\\
79.01	0\\
80.01	0\\
81.01	0\\
82.01	0\\
83.01	0\\
84.01	0\\
85.01	0\\
86.01	0\\
87.01	0\\
88.01	0\\
89.01	0\\
90.01	0\\
91.01	0\\
92.01	0\\
93.01	0\\
94.01	0\\
95.01	0\\
96.01	0\\
97.01	0\\
98.01	0\\
99.01	0\\
100.01	0\\
101.01	0\\
102.01	0\\
103.01	0\\
104.01	0\\
105.01	0\\
106.01	0\\
107.01	0\\
108.01	0\\
109.01	0\\
110.01	0\\
111.01	0\\
112.01	0\\
113.01	0\\
114.01	0\\
115.01	0\\
116.01	0\\
117.01	0\\
118.01	0\\
119.01	0\\
120.01	0\\
121.01	0\\
122.01	0\\
123.01	0\\
124.01	0\\
125.01	0\\
126.01	0\\
127.01	0\\
128.01	0\\
129.01	0\\
130.01	0\\
131.01	0\\
132.01	0\\
133.01	0\\
134.01	0\\
135.01	0\\
136.01	0\\
137.01	0\\
138.01	0\\
139.01	0\\
140.01	0\\
141.01	0\\
142.01	0\\
143.01	0\\
144.01	0\\
145.01	0\\
146.01	0\\
147.01	0\\
148.01	0\\
149.01	0\\
150.01	0\\
151.01	0\\
152.01	0\\
153.01	0\\
154.01	0\\
155.01	0\\
156.01	0\\
157.01	0\\
158.01	0\\
159.01	0\\
160.01	0\\
161.01	0\\
162.01	0\\
163.01	0\\
164.01	0\\
165.01	0\\
166.01	0\\
167.01	0\\
168.01	0\\
169.01	0\\
170.01	0\\
171.01	0\\
172.01	0\\
173.01	0\\
174.01	0\\
175.01	0\\
176.01	0\\
177.01	0\\
178.01	0\\
179.01	0\\
180.01	0\\
181.01	0\\
182.01	0\\
183.01	0\\
184.01	0\\
185.01	0\\
186.01	0\\
187.01	0\\
188.01	0\\
189.01	0\\
190.01	0\\
191.01	0\\
192.01	0\\
193.01	0\\
194.01	0\\
195.01	0\\
196.01	0\\
197.01	0\\
198.01	0\\
199.01	0\\
200.01	0\\
201.01	0\\
202.01	0\\
203.01	0\\
204.01	0\\
205.01	0\\
206.01	0\\
207.01	0\\
208.01	0\\
209.01	0\\
210.01	0\\
211.01	0\\
212.01	0\\
213.01	0\\
214.01	0\\
215.01	0\\
216.01	0\\
217.01	0\\
218.01	0\\
219.01	0\\
220.01	0\\
221.01	0\\
222.01	0\\
223.01	0\\
224.01	0\\
225.01	0\\
226.01	0\\
227.01	0\\
228.01	0\\
229.01	0\\
230.01	0\\
231.01	0\\
232.01	0\\
233.01	0\\
234.01	0\\
235.01	0\\
236.01	0\\
237.01	0\\
238.01	0\\
239.01	0\\
240.01	0\\
241.01	0\\
242.01	0\\
243.01	0\\
244.01	0\\
245.01	0\\
246.01	0\\
247.01	0\\
248.01	0\\
249.01	0\\
250.01	0\\
251.01	0\\
252.01	0\\
253.01	0\\
254.01	0\\
255.01	0\\
256.01	0\\
257.01	0\\
258.01	0\\
259.01	0\\
260.01	0\\
261.01	0\\
262.01	0\\
263.01	0\\
264.01	0\\
265.01	0\\
266.01	0\\
267.01	0\\
268.01	0\\
269.01	0\\
270.01	0\\
271.01	0\\
272.01	0\\
273.01	0\\
274.01	0\\
275.01	0\\
276.01	0\\
277.01	0\\
278.01	0\\
279.01	0\\
280.01	0\\
281.01	0\\
282.01	0\\
283.01	0\\
284.01	0\\
285.01	0\\
286.01	0\\
287.01	0\\
288.01	0\\
289.01	0\\
290.01	0\\
291.01	0\\
292.01	0\\
293.01	0\\
294.01	0\\
295.01	0\\
296.01	0\\
297.01	0\\
298.01	0\\
299.01	0\\
300.01	0\\
301.01	0\\
302.01	0\\
303.01	0\\
304.01	0\\
305.01	0\\
306.01	0\\
307.01	0\\
308.01	0\\
309.01	0\\
310.01	0\\
311.01	0\\
312.01	0\\
313.01	0\\
314.01	0\\
315.01	0\\
316.01	0\\
317.01	0\\
318.01	0\\
319.01	0\\
320.01	0\\
321.01	0\\
322.01	0\\
323.01	0\\
324.01	0\\
325.01	0\\
326.01	0\\
327.01	0\\
328.01	0\\
329.01	0\\
330.01	0\\
331.01	0\\
332.01	0\\
333.01	0\\
334.01	0\\
335.01	0\\
336.01	0\\
337.01	0\\
338.01	0\\
339.01	0\\
340.01	0\\
341.01	0\\
342.01	0\\
343.01	0\\
344.01	0\\
345.01	0\\
346.01	0\\
347.01	0\\
348.01	0\\
349.01	0\\
350.01	0\\
351.01	0\\
352.01	0\\
353.01	0\\
354.01	0\\
355.01	0\\
356.01	0\\
357.01	0\\
358.01	0\\
359.01	0\\
360.01	0\\
361.01	0\\
362.01	0\\
363.01	0\\
364.01	0\\
365.01	0\\
366.01	0\\
367.01	0\\
368.01	0\\
369.01	0\\
370.01	0\\
371.01	0\\
372.01	0\\
373.01	0\\
374.01	0\\
375.01	0\\
376.01	0\\
377.01	0\\
378.01	0\\
379.01	0\\
380.01	0\\
381.01	0\\
382.01	0\\
383.01	0\\
384.01	0\\
385.01	0\\
386.01	0\\
387.01	0\\
388.01	0\\
389.01	0\\
390.01	0\\
391.01	0\\
392.01	0\\
393.01	0\\
394.01	0\\
395.01	0\\
396.01	0\\
397.01	0\\
398.01	0\\
399.01	0\\
400.01	0\\
401.01	0\\
402.01	0\\
403.01	0\\
404.01	0\\
405.01	0\\
406.01	0\\
407.01	0\\
408.01	0\\
409.01	0\\
410.01	0\\
411.01	0\\
412.01	0\\
413.01	0\\
414.01	0\\
415.01	0\\
416.01	0\\
417.01	0\\
418.01	0\\
419.01	0\\
420.01	0\\
421.01	0\\
422.01	0\\
423.01	0\\
424.01	0\\
425.01	0\\
426.01	0\\
427.01	0\\
428.01	0\\
429.01	0\\
430.01	0\\
431.01	0\\
432.01	0\\
433.01	0\\
434.01	0\\
435.01	0\\
436.01	0\\
437.01	0\\
438.01	0\\
439.01	0\\
440.01	0\\
441.01	0\\
442.01	0\\
443.01	0\\
444.01	0\\
445.01	0\\
446.01	0\\
447.01	0\\
448.01	0\\
449.01	0\\
450.01	0\\
451.01	0\\
452.01	0\\
453.01	0\\
454.01	0\\
455.01	0\\
456.01	0\\
457.01	0\\
458.01	0\\
459.01	0\\
460.01	0\\
461.01	0\\
462.01	0\\
463.01	0\\
464.01	0\\
465.01	0\\
466.01	0\\
467.01	0\\
468.01	0\\
469.01	0\\
470.01	0\\
471.01	0\\
472.01	0\\
473.01	0\\
474.01	0\\
475.01	0\\
476.01	0\\
477.01	0\\
478.01	0\\
479.01	0\\
480.01	0\\
481.01	0\\
482.01	0\\
483.01	0\\
484.01	0\\
485.01	0\\
486.01	0\\
487.01	0\\
488.01	0\\
489.01	0\\
490.01	0\\
491.01	0\\
492.01	0\\
493.01	0\\
494.01	0\\
495.01	0\\
496.01	0\\
497.01	0\\
498.01	0\\
499.01	0\\
500.01	0\\
501.01	0\\
502.01	0\\
503.01	0\\
504.01	0\\
505.01	0\\
506.01	0\\
507.01	0\\
508.01	0\\
509.01	0\\
510.01	0\\
511.01	0\\
512.01	0\\
513.01	0\\
514.01	0\\
515.01	0\\
516.01	0\\
517.01	0\\
518.01	0\\
519.01	0\\
520.01	0\\
521.01	0\\
522.01	0\\
523.01	0\\
524.01	0\\
525.01	0\\
526.01	0\\
527.01	0\\
528.01	0\\
529.01	0\\
530.01	0\\
531.01	0\\
532.01	0\\
533.01	0\\
534.01	0\\
535.01	0\\
536.01	0\\
537.01	0\\
538.01	0\\
539.01	0\\
540.01	0\\
541.01	0\\
542.01	0\\
543.01	0\\
544.01	0\\
545.01	0\\
546.01	0\\
547.01	0\\
548.01	0\\
549.01	0\\
550.01	0\\
551.01	0\\
552.01	0\\
553.01	0\\
554.01	0\\
555.01	0\\
556.01	0\\
557.01	0\\
558.01	0\\
559.01	0\\
560.01	0\\
561.01	0\\
562.01	0\\
563.01	0\\
564.01	0\\
565.01	0\\
566.01	0\\
567.01	0\\
568.01	0\\
569.01	0\\
570.01	0\\
571.01	0\\
572.01	0\\
573.01	0\\
574.01	0\\
575.01	0\\
576.01	0\\
577.01	0\\
578.01	0\\
579.01	0\\
580.01	0\\
581.01	0\\
582.01	0\\
583.01	0\\
584.01	0\\
585.01	0\\
586.01	0\\
587.01	0\\
588.01	0\\
589.01	0\\
590.01	0\\
591.01	0\\
592.01	0\\
593.01	0\\
594.01	0\\
595.01	0\\
596.01	0\\
597.01	0.000481322506726112\\
598.01	0.00143832068185176\\
599.01	0.00385661257211617\\
599.02	0.00389414818835224\\
599.03	0.00393204146009077\\
599.04	0.00397029584140052\\
599.05	0.00400891481968839\\
599.06	0.00404790191602103\\
599.07	0.0040872606854494\\
599.08	0.00412699471733655\\
599.09	0.00416710763568844\\
599.1	0.0042076030994881\\
599.11	0.00424848480303291\\
599.12	0.00428975647627511\\
599.13	0.00433142188516564\\
599.14	0.00437348483200133\\
599.15	0.00441594915577534\\
599.16	0.00445881873253108\\
599.17	0.00450209747571942\\
599.18	0.00454578933655947\\
599.19	0.00458989830440271\\
599.2	0.00463442840710072\\
599.21	0.00467938371137648\\
599.22	0.00472476832319908\\
599.23	0.00477058638816223\\
599.24	0.00481684209186631\\
599.25	0.00486353966030413\\
599.26	0.00491068334955366\\
599.27	0.00495827744958435\\
599.28	0.00500632629158769\\
599.29	0.00505483424837289\\
599.3	0.00510380573476642\\
599.31	0.00515324520801536\\
599.32	0.00520315716819454\\
599.33	0.00525354615861773\\
599.34	0.00530441676625257\\
599.35	0.0053557736221397\\
599.36	0.00540762140181576\\
599.37	0.0054599648257405\\
599.38	0.00551280865972796\\
599.39	0.00556615771538181\\
599.4	0.00562001685053484\\
599.41	0.00567439096969274\\
599.42	0.00572928502448199\\
599.43	0.00578470401410215\\
599.44	0.00584065298578251\\
599.45	0.00589713703524303\\
599.46	0.00595416130715973\\
599.47	0.00601173099563458\\
599.48	0.00606985134466989\\
599.49	0.00612852764864713\\
599.5	0.00618776525281055\\
599.51	0.00624756955375526\\
599.52	0.00630794599992006\\
599.53	0.006368900092085\\
599.54	0.00643043738387369\\
599.55	0.00649256348226043\\
599.56	0.00655528404808221\\
599.57	0.00661860479655559\\
599.58	0.00668253149779856\\
599.59	0.00674706997735745\\
599.6	0.00681222611673879\\
599.61	0.00687800585394631\\
599.62	0.00694441518402313\\
599.63	0.00701146015959912\\
599.64	0.00707914689144344\\
599.65	0.00714748154902252\\
599.66	0.00721647036106323\\
599.67	0.00728611961612157\\
599.68	0.00735643566315675\\
599.69	0.00742742491211077\\
599.7	0.00749909383449363\\
599.71	0.00757144896397399\\
599.72	0.0076444968969757\\
599.73	0.00771824429327989\\
599.74	0.00779269787663284\\
599.75	0.00786786443535981\\
599.76	0.0079437508229846\\
599.77	0.00802036395885509\\
599.78	0.00809771082877485\\
599.79	0.00817579848564071\\
599.8	0.00825463405008648\\
599.81	0.00833422471113284\\
599.82	0.00841457772684349\\
599.83	0.00849570042498746\\
599.84	0.00857760020370797\\
599.85	0.00866028453219752\\
599.86	0.00874376095137954\\
599.87	0.00882803707459654\\
599.88	0.0089131205883049\\
599.89	0.00899901925277625\\
599.9	0.00908574090280562\\
599.91	0.00917329344842635\\
599.92	0.0092616848756319\\
599.93	0.00935092324710449\\
599.94	0.00944101670295078\\
599.95	0.00953197346144465\\
599.96	0.00962380181977693\\
599.97	0.00971651015481255\\
599.98	0.0098101069238547\\
599.99	0.00990460066541651\\
600	0.01\\
};
\addplot [color=black!80!mycolor21,solid,forget plot]
  table[row sep=crcr]{%
0.01	0\\
1.01	0\\
2.01	0\\
3.01	0\\
4.01	0\\
5.01	0\\
6.01	0\\
7.01	0\\
8.01	0\\
9.01	0\\
10.01	0\\
11.01	0\\
12.01	0\\
13.01	0\\
14.01	0\\
15.01	0\\
16.01	0\\
17.01	0\\
18.01	0\\
19.01	0\\
20.01	0\\
21.01	0\\
22.01	0\\
23.01	0\\
24.01	0\\
25.01	0\\
26.01	0\\
27.01	0\\
28.01	0\\
29.01	0\\
30.01	0\\
31.01	0\\
32.01	0\\
33.01	0\\
34.01	0\\
35.01	0\\
36.01	0\\
37.01	0\\
38.01	0\\
39.01	0\\
40.01	0\\
41.01	0\\
42.01	0\\
43.01	0\\
44.01	0\\
45.01	0\\
46.01	0\\
47.01	0\\
48.01	0\\
49.01	0\\
50.01	0\\
51.01	0\\
52.01	0\\
53.01	0\\
54.01	0\\
55.01	0\\
56.01	0\\
57.01	0\\
58.01	0\\
59.01	0\\
60.01	0\\
61.01	0\\
62.01	0\\
63.01	0\\
64.01	0\\
65.01	0\\
66.01	0\\
67.01	0\\
68.01	0\\
69.01	0\\
70.01	0\\
71.01	0\\
72.01	0\\
73.01	0\\
74.01	0\\
75.01	0\\
76.01	0\\
77.01	0\\
78.01	0\\
79.01	0\\
80.01	0\\
81.01	0\\
82.01	0\\
83.01	0\\
84.01	0\\
85.01	0\\
86.01	0\\
87.01	0\\
88.01	0\\
89.01	0\\
90.01	0\\
91.01	0\\
92.01	0\\
93.01	0\\
94.01	0\\
95.01	0\\
96.01	0\\
97.01	0\\
98.01	0\\
99.01	0\\
100.01	0\\
101.01	0\\
102.01	0\\
103.01	0\\
104.01	0\\
105.01	0\\
106.01	0\\
107.01	0\\
108.01	0\\
109.01	0\\
110.01	0\\
111.01	0\\
112.01	0\\
113.01	0\\
114.01	0\\
115.01	0\\
116.01	0\\
117.01	0\\
118.01	0\\
119.01	0\\
120.01	0\\
121.01	0\\
122.01	0\\
123.01	0\\
124.01	0\\
125.01	0\\
126.01	0\\
127.01	0\\
128.01	0\\
129.01	0\\
130.01	0\\
131.01	0\\
132.01	0\\
133.01	0\\
134.01	0\\
135.01	0\\
136.01	0\\
137.01	0\\
138.01	0\\
139.01	0\\
140.01	0\\
141.01	0\\
142.01	0\\
143.01	0\\
144.01	0\\
145.01	0\\
146.01	0\\
147.01	0\\
148.01	0\\
149.01	0\\
150.01	0\\
151.01	0\\
152.01	0\\
153.01	0\\
154.01	0\\
155.01	0\\
156.01	0\\
157.01	0\\
158.01	0\\
159.01	0\\
160.01	0\\
161.01	0\\
162.01	0\\
163.01	0\\
164.01	0\\
165.01	0\\
166.01	0\\
167.01	0\\
168.01	0\\
169.01	0\\
170.01	0\\
171.01	0\\
172.01	0\\
173.01	0\\
174.01	0\\
175.01	0\\
176.01	0\\
177.01	0\\
178.01	0\\
179.01	0\\
180.01	0\\
181.01	0\\
182.01	0\\
183.01	0\\
184.01	0\\
185.01	0\\
186.01	0\\
187.01	0\\
188.01	0\\
189.01	0\\
190.01	0\\
191.01	0\\
192.01	0\\
193.01	0\\
194.01	0\\
195.01	0\\
196.01	0\\
197.01	0\\
198.01	0\\
199.01	0\\
200.01	0\\
201.01	0\\
202.01	0\\
203.01	0\\
204.01	0\\
205.01	0\\
206.01	0\\
207.01	0\\
208.01	0\\
209.01	0\\
210.01	0\\
211.01	0\\
212.01	0\\
213.01	0\\
214.01	0\\
215.01	0\\
216.01	0\\
217.01	0\\
218.01	0\\
219.01	0\\
220.01	0\\
221.01	0\\
222.01	0\\
223.01	0\\
224.01	0\\
225.01	0\\
226.01	0\\
227.01	0\\
228.01	0\\
229.01	0\\
230.01	0\\
231.01	0\\
232.01	0\\
233.01	0\\
234.01	0\\
235.01	0\\
236.01	0\\
237.01	0\\
238.01	0\\
239.01	0\\
240.01	0\\
241.01	0\\
242.01	0\\
243.01	0\\
244.01	0\\
245.01	0\\
246.01	0\\
247.01	0\\
248.01	0\\
249.01	0\\
250.01	0\\
251.01	0\\
252.01	0\\
253.01	0\\
254.01	0\\
255.01	0\\
256.01	0\\
257.01	0\\
258.01	0\\
259.01	0\\
260.01	0\\
261.01	0\\
262.01	0\\
263.01	0\\
264.01	0\\
265.01	0\\
266.01	0\\
267.01	0\\
268.01	0\\
269.01	0\\
270.01	0\\
271.01	0\\
272.01	0\\
273.01	0\\
274.01	0\\
275.01	0\\
276.01	0\\
277.01	0\\
278.01	0\\
279.01	0\\
280.01	0\\
281.01	0\\
282.01	0\\
283.01	0\\
284.01	0\\
285.01	0\\
286.01	0\\
287.01	0\\
288.01	0\\
289.01	0\\
290.01	0\\
291.01	0\\
292.01	0\\
293.01	0\\
294.01	0\\
295.01	0\\
296.01	0\\
297.01	0\\
298.01	0\\
299.01	0\\
300.01	0\\
301.01	0\\
302.01	0\\
303.01	0\\
304.01	0\\
305.01	0\\
306.01	0\\
307.01	0\\
308.01	0\\
309.01	0\\
310.01	0\\
311.01	0\\
312.01	0\\
313.01	0\\
314.01	0\\
315.01	0\\
316.01	0\\
317.01	0\\
318.01	0\\
319.01	0\\
320.01	0\\
321.01	0\\
322.01	0\\
323.01	0\\
324.01	0\\
325.01	0\\
326.01	0\\
327.01	0\\
328.01	0\\
329.01	0\\
330.01	0\\
331.01	0\\
332.01	0\\
333.01	0\\
334.01	0\\
335.01	0\\
336.01	0\\
337.01	0\\
338.01	0\\
339.01	0\\
340.01	0\\
341.01	0\\
342.01	0\\
343.01	0\\
344.01	0\\
345.01	0\\
346.01	0\\
347.01	0\\
348.01	0\\
349.01	0\\
350.01	0\\
351.01	0\\
352.01	0\\
353.01	0\\
354.01	0\\
355.01	0\\
356.01	0\\
357.01	0\\
358.01	0\\
359.01	0\\
360.01	0\\
361.01	0\\
362.01	0\\
363.01	0\\
364.01	0\\
365.01	0\\
366.01	0\\
367.01	0\\
368.01	0\\
369.01	0\\
370.01	0\\
371.01	0\\
372.01	0\\
373.01	0\\
374.01	0\\
375.01	0\\
376.01	0\\
377.01	0\\
378.01	0\\
379.01	0\\
380.01	0\\
381.01	0\\
382.01	0\\
383.01	0\\
384.01	0\\
385.01	0\\
386.01	0\\
387.01	0\\
388.01	0\\
389.01	0\\
390.01	0\\
391.01	0\\
392.01	0\\
393.01	0\\
394.01	0\\
395.01	0\\
396.01	0\\
397.01	0\\
398.01	0\\
399.01	0\\
400.01	0\\
401.01	0\\
402.01	0\\
403.01	0\\
404.01	0\\
405.01	0\\
406.01	0\\
407.01	0\\
408.01	0\\
409.01	0\\
410.01	0\\
411.01	0\\
412.01	0\\
413.01	0\\
414.01	0\\
415.01	0\\
416.01	0\\
417.01	0\\
418.01	0\\
419.01	0\\
420.01	0\\
421.01	0\\
422.01	0\\
423.01	0\\
424.01	0\\
425.01	0\\
426.01	0\\
427.01	0\\
428.01	0\\
429.01	0\\
430.01	0\\
431.01	0\\
432.01	0\\
433.01	0\\
434.01	0\\
435.01	0\\
436.01	0\\
437.01	0\\
438.01	0\\
439.01	0\\
440.01	0\\
441.01	0\\
442.01	0\\
443.01	0\\
444.01	0\\
445.01	0\\
446.01	0\\
447.01	0\\
448.01	0\\
449.01	0\\
450.01	0\\
451.01	0\\
452.01	0\\
453.01	0\\
454.01	0\\
455.01	0\\
456.01	0\\
457.01	0\\
458.01	0\\
459.01	0\\
460.01	0\\
461.01	0\\
462.01	0\\
463.01	0\\
464.01	0\\
465.01	0\\
466.01	0\\
467.01	0\\
468.01	0\\
469.01	0\\
470.01	0\\
471.01	0\\
472.01	0\\
473.01	0\\
474.01	0\\
475.01	0\\
476.01	0\\
477.01	0\\
478.01	0\\
479.01	0\\
480.01	0\\
481.01	0\\
482.01	0\\
483.01	0\\
484.01	0\\
485.01	0\\
486.01	0\\
487.01	0\\
488.01	0\\
489.01	0\\
490.01	0\\
491.01	0\\
492.01	0\\
493.01	0\\
494.01	0\\
495.01	0\\
496.01	0\\
497.01	0\\
498.01	0\\
499.01	0\\
500.01	0\\
501.01	0\\
502.01	0\\
503.01	0\\
504.01	0\\
505.01	0\\
506.01	0\\
507.01	0\\
508.01	0\\
509.01	0\\
510.01	0\\
511.01	0\\
512.01	0\\
513.01	0\\
514.01	0\\
515.01	0\\
516.01	0\\
517.01	0\\
518.01	0\\
519.01	0\\
520.01	0\\
521.01	0\\
522.01	0\\
523.01	0\\
524.01	0\\
525.01	0\\
526.01	0\\
527.01	0\\
528.01	0\\
529.01	0\\
530.01	0\\
531.01	0\\
532.01	0\\
533.01	0\\
534.01	0\\
535.01	0\\
536.01	0\\
537.01	0\\
538.01	0\\
539.01	0\\
540.01	0\\
541.01	0\\
542.01	0\\
543.01	0\\
544.01	0\\
545.01	0\\
546.01	0\\
547.01	0\\
548.01	0\\
549.01	0\\
550.01	0\\
551.01	0\\
552.01	0\\
553.01	0\\
554.01	0\\
555.01	0\\
556.01	0\\
557.01	0\\
558.01	0\\
559.01	0\\
560.01	0\\
561.01	0\\
562.01	0\\
563.01	0\\
564.01	0\\
565.01	0\\
566.01	0\\
567.01	0\\
568.01	0\\
569.01	0\\
570.01	0\\
571.01	0\\
572.01	0\\
573.01	0\\
574.01	0\\
575.01	0\\
576.01	0\\
577.01	0\\
578.01	0\\
579.01	0\\
580.01	0\\
581.01	0\\
582.01	0\\
583.01	0\\
584.01	0\\
585.01	0\\
586.01	0\\
587.01	0\\
588.01	0\\
589.01	0\\
590.01	0\\
591.01	0\\
592.01	0\\
593.01	0\\
594.01	0\\
595.01	0\\
596.01	0\\
597.01	0.000481328854172736\\
598.01	0.00143832068185187\\
599.01	0.00385661257211624\\
599.02	0.00389414818835231\\
599.03	0.00393204146009085\\
599.04	0.00397029584140059\\
599.05	0.00400891481968844\\
599.06	0.00404790191602107\\
599.07	0.00408726068544944\\
599.08	0.00412699471733657\\
599.09	0.00416710763568848\\
599.1	0.00420760309948816\\
599.11	0.00424848480303296\\
599.12	0.00428975647627514\\
599.13	0.00433142188516565\\
599.14	0.00437348483200133\\
599.15	0.00441594915577534\\
599.16	0.00445881873253108\\
599.17	0.00450209747571943\\
599.18	0.00454578933655947\\
599.19	0.00458989830440271\\
599.2	0.00463442840710074\\
599.21	0.00467938371137648\\
599.22	0.00472476832319907\\
599.23	0.0047705863881622\\
599.24	0.00481684209186627\\
599.25	0.00486353966030409\\
599.26	0.00491068334955364\\
599.27	0.00495827744958434\\
599.28	0.00500632629158768\\
599.29	0.00505483424837287\\
599.3	0.0051038057347664\\
599.31	0.00515324520801534\\
599.32	0.00520315716819454\\
599.33	0.00525354615861773\\
599.34	0.00530441676625258\\
599.35	0.00535577362213972\\
599.36	0.00540762140181578\\
599.37	0.0054599648257405\\
599.38	0.00551280865972795\\
599.39	0.00556615771538178\\
599.4	0.00562001685053483\\
599.41	0.00567439096969272\\
599.42	0.00572928502448196\\
599.43	0.00578470401410212\\
599.44	0.00584065298578249\\
599.45	0.005897137035243\\
599.46	0.00595416130715971\\
599.47	0.00601173099563457\\
599.48	0.00606985134466986\\
599.49	0.0061285276486471\\
599.5	0.00618776525281052\\
599.51	0.00624756955375523\\
599.52	0.00630794599992004\\
599.53	0.006368900092085\\
599.54	0.00643043738387369\\
599.55	0.00649256348226045\\
599.56	0.00655528404808221\\
599.57	0.00661860479655557\\
599.58	0.00668253149779854\\
599.59	0.00674706997735744\\
599.6	0.00681222611673878\\
599.61	0.0068780058539463\\
599.62	0.00694441518402312\\
599.63	0.00701146015959912\\
599.64	0.00707914689144344\\
599.65	0.00714748154902253\\
599.66	0.00721647036106324\\
599.67	0.00728611961612158\\
599.68	0.00735643566315676\\
599.69	0.00742742491211079\\
599.7	0.00749909383449364\\
599.71	0.007571448963974\\
599.72	0.00764449689697572\\
599.73	0.0077182442932799\\
599.74	0.00779269787663286\\
599.75	0.00786786443535983\\
599.76	0.00794375082298462\\
599.77	0.00802036395885511\\
599.78	0.00809771082877486\\
599.79	0.00817579848564072\\
599.8	0.00825463405008649\\
599.81	0.00833422471113286\\
599.82	0.0084145777268435\\
599.83	0.00849570042498747\\
599.84	0.00857760020370798\\
599.85	0.00866028453219752\\
599.86	0.00874376095137954\\
599.87	0.00882803707459654\\
599.88	0.0089131205883049\\
599.89	0.00899901925277625\\
599.9	0.00908574090280562\\
599.91	0.00917329344842635\\
599.92	0.0092616848756319\\
599.93	0.00935092324710449\\
599.94	0.00944101670295079\\
599.95	0.00953197346144465\\
599.96	0.00962380181977693\\
599.97	0.00971651015481255\\
599.98	0.0098101069238547\\
599.99	0.00990460066541651\\
600	0.01\\
};
\addplot [color=black,solid,forget plot]
  table[row sep=crcr]{%
0.01	0\\
1.01	0\\
2.01	0\\
3.01	0\\
4.01	0\\
5.01	0\\
6.01	0\\
7.01	0\\
8.01	0\\
9.01	0\\
10.01	0\\
11.01	0\\
12.01	0\\
13.01	0\\
14.01	0\\
15.01	0\\
16.01	0\\
17.01	0\\
18.01	0\\
19.01	0\\
20.01	0\\
21.01	0\\
22.01	0\\
23.01	0\\
24.01	0\\
25.01	0\\
26.01	0\\
27.01	0\\
28.01	0\\
29.01	0\\
30.01	0\\
31.01	0\\
32.01	0\\
33.01	0\\
34.01	0\\
35.01	0\\
36.01	0\\
37.01	0\\
38.01	0\\
39.01	0\\
40.01	0\\
41.01	0\\
42.01	0\\
43.01	0\\
44.01	0\\
45.01	0\\
46.01	0\\
47.01	0\\
48.01	0\\
49.01	0\\
50.01	0\\
51.01	0\\
52.01	0\\
53.01	0\\
54.01	0\\
55.01	0\\
56.01	0\\
57.01	0\\
58.01	0\\
59.01	0\\
60.01	0\\
61.01	0\\
62.01	0\\
63.01	0\\
64.01	0\\
65.01	0\\
66.01	0\\
67.01	0\\
68.01	0\\
69.01	0\\
70.01	0\\
71.01	0\\
72.01	0\\
73.01	0\\
74.01	0\\
75.01	0\\
76.01	0\\
77.01	0\\
78.01	0\\
79.01	0\\
80.01	0\\
81.01	0\\
82.01	0\\
83.01	0\\
84.01	0\\
85.01	0\\
86.01	0\\
87.01	0\\
88.01	0\\
89.01	0\\
90.01	0\\
91.01	0\\
92.01	0\\
93.01	0\\
94.01	0\\
95.01	0\\
96.01	0\\
97.01	0\\
98.01	0\\
99.01	0\\
100.01	0\\
101.01	0\\
102.01	0\\
103.01	0\\
104.01	0\\
105.01	0\\
106.01	0\\
107.01	0\\
108.01	0\\
109.01	0\\
110.01	0\\
111.01	0\\
112.01	0\\
113.01	0\\
114.01	0\\
115.01	0\\
116.01	0\\
117.01	0\\
118.01	0\\
119.01	0\\
120.01	0\\
121.01	0\\
122.01	0\\
123.01	0\\
124.01	0\\
125.01	0\\
126.01	0\\
127.01	0\\
128.01	0\\
129.01	0\\
130.01	0\\
131.01	0\\
132.01	0\\
133.01	0\\
134.01	0\\
135.01	0\\
136.01	0\\
137.01	0\\
138.01	0\\
139.01	0\\
140.01	0\\
141.01	0\\
142.01	0\\
143.01	0\\
144.01	0\\
145.01	0\\
146.01	0\\
147.01	0\\
148.01	0\\
149.01	0\\
150.01	0\\
151.01	0\\
152.01	0\\
153.01	0\\
154.01	0\\
155.01	0\\
156.01	0\\
157.01	0\\
158.01	0\\
159.01	0\\
160.01	0\\
161.01	0\\
162.01	0\\
163.01	0\\
164.01	0\\
165.01	0\\
166.01	0\\
167.01	0\\
168.01	0\\
169.01	0\\
170.01	0\\
171.01	0\\
172.01	0\\
173.01	0\\
174.01	0\\
175.01	0\\
176.01	0\\
177.01	0\\
178.01	0\\
179.01	0\\
180.01	0\\
181.01	0\\
182.01	0\\
183.01	0\\
184.01	0\\
185.01	0\\
186.01	0\\
187.01	0\\
188.01	0\\
189.01	0\\
190.01	0\\
191.01	0\\
192.01	0\\
193.01	0\\
194.01	0\\
195.01	0\\
196.01	0\\
197.01	0\\
198.01	0\\
199.01	0\\
200.01	0\\
201.01	0\\
202.01	0\\
203.01	0\\
204.01	0\\
205.01	0\\
206.01	0\\
207.01	0\\
208.01	0\\
209.01	0\\
210.01	0\\
211.01	0\\
212.01	0\\
213.01	0\\
214.01	0\\
215.01	0\\
216.01	0\\
217.01	0\\
218.01	0\\
219.01	0\\
220.01	0\\
221.01	0\\
222.01	0\\
223.01	0\\
224.01	0\\
225.01	0\\
226.01	0\\
227.01	0\\
228.01	0\\
229.01	0\\
230.01	0\\
231.01	0\\
232.01	0\\
233.01	0\\
234.01	0\\
235.01	0\\
236.01	0\\
237.01	0\\
238.01	0\\
239.01	0\\
240.01	0\\
241.01	0\\
242.01	0\\
243.01	0\\
244.01	0\\
245.01	0\\
246.01	0\\
247.01	0\\
248.01	0\\
249.01	0\\
250.01	0\\
251.01	0\\
252.01	0\\
253.01	0\\
254.01	0\\
255.01	0\\
256.01	0\\
257.01	0\\
258.01	0\\
259.01	0\\
260.01	0\\
261.01	0\\
262.01	0\\
263.01	0\\
264.01	0\\
265.01	0\\
266.01	0\\
267.01	0\\
268.01	0\\
269.01	0\\
270.01	0\\
271.01	0\\
272.01	0\\
273.01	0\\
274.01	0\\
275.01	0\\
276.01	0\\
277.01	0\\
278.01	0\\
279.01	0\\
280.01	0\\
281.01	0\\
282.01	0\\
283.01	0\\
284.01	0\\
285.01	0\\
286.01	0\\
287.01	0\\
288.01	0\\
289.01	0\\
290.01	0\\
291.01	0\\
292.01	0\\
293.01	0\\
294.01	0\\
295.01	0\\
296.01	0\\
297.01	0\\
298.01	0\\
299.01	0\\
300.01	0\\
301.01	0\\
302.01	0\\
303.01	0\\
304.01	0\\
305.01	0\\
306.01	0\\
307.01	0\\
308.01	0\\
309.01	0\\
310.01	0\\
311.01	0\\
312.01	0\\
313.01	0\\
314.01	0\\
315.01	0\\
316.01	0\\
317.01	0\\
318.01	0\\
319.01	0\\
320.01	0\\
321.01	0\\
322.01	0\\
323.01	0\\
324.01	0\\
325.01	0\\
326.01	0\\
327.01	0\\
328.01	0\\
329.01	0\\
330.01	0\\
331.01	0\\
332.01	0\\
333.01	0\\
334.01	0\\
335.01	0\\
336.01	0\\
337.01	0\\
338.01	0\\
339.01	0\\
340.01	0\\
341.01	0\\
342.01	0\\
343.01	0\\
344.01	0\\
345.01	0\\
346.01	0\\
347.01	0\\
348.01	0\\
349.01	0\\
350.01	0\\
351.01	0\\
352.01	0\\
353.01	0\\
354.01	0\\
355.01	0\\
356.01	0\\
357.01	0\\
358.01	0\\
359.01	0\\
360.01	0\\
361.01	0\\
362.01	0\\
363.01	0\\
364.01	0\\
365.01	0\\
366.01	0\\
367.01	0\\
368.01	0\\
369.01	0\\
370.01	0\\
371.01	0\\
372.01	0\\
373.01	0\\
374.01	0\\
375.01	0\\
376.01	0\\
377.01	0\\
378.01	0\\
379.01	0\\
380.01	0\\
381.01	0\\
382.01	0\\
383.01	0\\
384.01	0\\
385.01	0\\
386.01	0\\
387.01	0\\
388.01	0\\
389.01	0\\
390.01	0\\
391.01	0\\
392.01	0\\
393.01	0\\
394.01	0\\
395.01	0\\
396.01	0\\
397.01	0\\
398.01	0\\
399.01	0\\
400.01	0\\
401.01	0\\
402.01	0\\
403.01	0\\
404.01	0\\
405.01	0\\
406.01	0\\
407.01	0\\
408.01	0\\
409.01	0\\
410.01	0\\
411.01	0\\
412.01	0\\
413.01	0\\
414.01	0\\
415.01	0\\
416.01	0\\
417.01	0\\
418.01	0\\
419.01	0\\
420.01	0\\
421.01	0\\
422.01	0\\
423.01	0\\
424.01	0\\
425.01	0\\
426.01	0\\
427.01	0\\
428.01	0\\
429.01	0\\
430.01	0\\
431.01	0\\
432.01	0\\
433.01	0\\
434.01	0\\
435.01	0\\
436.01	0\\
437.01	0\\
438.01	0\\
439.01	0\\
440.01	0\\
441.01	0\\
442.01	0\\
443.01	0\\
444.01	0\\
445.01	0\\
446.01	0\\
447.01	0\\
448.01	0\\
449.01	0\\
450.01	0\\
451.01	0\\
452.01	0\\
453.01	0\\
454.01	0\\
455.01	0\\
456.01	0\\
457.01	0\\
458.01	0\\
459.01	0\\
460.01	0\\
461.01	0\\
462.01	0\\
463.01	0\\
464.01	0\\
465.01	0\\
466.01	0\\
467.01	0\\
468.01	0\\
469.01	0\\
470.01	0\\
471.01	0\\
472.01	0\\
473.01	0\\
474.01	0\\
475.01	0\\
476.01	0\\
477.01	0\\
478.01	0\\
479.01	0\\
480.01	0\\
481.01	0\\
482.01	0\\
483.01	0\\
484.01	0\\
485.01	0\\
486.01	0\\
487.01	0\\
488.01	0\\
489.01	0\\
490.01	0\\
491.01	0\\
492.01	0\\
493.01	0\\
494.01	0\\
495.01	0\\
496.01	0\\
497.01	0\\
498.01	0\\
499.01	0\\
500.01	0\\
501.01	0\\
502.01	0\\
503.01	0\\
504.01	0\\
505.01	0\\
506.01	0\\
507.01	0\\
508.01	0\\
509.01	0\\
510.01	0\\
511.01	0\\
512.01	0\\
513.01	0\\
514.01	0\\
515.01	0\\
516.01	0\\
517.01	0\\
518.01	0\\
519.01	0\\
520.01	0\\
521.01	0\\
522.01	0\\
523.01	0\\
524.01	0\\
525.01	0\\
526.01	0\\
527.01	0\\
528.01	0\\
529.01	0\\
530.01	0\\
531.01	0\\
532.01	0\\
533.01	0\\
534.01	0\\
535.01	0\\
536.01	0\\
537.01	0\\
538.01	0\\
539.01	0\\
540.01	0\\
541.01	0\\
542.01	0\\
543.01	0\\
544.01	0\\
545.01	0\\
546.01	0\\
547.01	0\\
548.01	0\\
549.01	0\\
550.01	0\\
551.01	0\\
552.01	0\\
553.01	0\\
554.01	0\\
555.01	0\\
556.01	0\\
557.01	0\\
558.01	0\\
559.01	0\\
560.01	0\\
561.01	0\\
562.01	0\\
563.01	0\\
564.01	0\\
565.01	0\\
566.01	0\\
567.01	0\\
568.01	0\\
569.01	0\\
570.01	0\\
571.01	0\\
572.01	0\\
573.01	0\\
574.01	0\\
575.01	0\\
576.01	0\\
577.01	0\\
578.01	0\\
579.01	0\\
580.01	0\\
581.01	0\\
582.01	0\\
583.01	0\\
584.01	0\\
585.01	0\\
586.01	0\\
587.01	0\\
588.01	0\\
589.01	0\\
590.01	0\\
591.01	0\\
592.01	0\\
593.01	0\\
594.01	0\\
595.01	0\\
596.01	0\\
597.01	0.00048133368794398\\
598.01	0.00143832068185187\\
599.01	0.00385661257211624\\
599.02	0.00389414818835231\\
599.03	0.00393204146009084\\
599.04	0.00397029584140057\\
599.05	0.00400891481968844\\
599.06	0.00404790191602107\\
599.07	0.00408726068544944\\
599.08	0.00412699471733659\\
599.09	0.00416710763568849\\
599.1	0.00420760309948817\\
599.11	0.00424848480303298\\
599.12	0.00428975647627516\\
599.13	0.00433142188516569\\
599.14	0.00437348483200138\\
599.15	0.00441594915577539\\
599.16	0.00445881873253112\\
599.17	0.00450209747571946\\
599.18	0.0045457893365595\\
599.19	0.00458989830440273\\
599.2	0.00463442840710075\\
599.21	0.00467938371137651\\
599.22	0.0047247683231991\\
599.23	0.00477058638816223\\
599.24	0.00481684209186631\\
599.25	0.00486353966030412\\
599.26	0.00491068334955366\\
599.27	0.00495827744958435\\
599.28	0.00500632629158768\\
599.29	0.00505483424837287\\
599.3	0.00510380573476642\\
599.31	0.00515324520801536\\
599.32	0.00520315716819456\\
599.33	0.00525354615861774\\
599.34	0.00530441676625258\\
599.35	0.00535577362213972\\
599.36	0.00540762140181579\\
599.37	0.00545996482574053\\
599.38	0.00551280865972797\\
599.39	0.00556615771538181\\
599.4	0.00562001685053484\\
599.41	0.00567439096969274\\
599.42	0.00572928502448199\\
599.43	0.00578470401410215\\
599.44	0.00584065298578253\\
599.45	0.00589713703524306\\
599.46	0.00595416130715976\\
599.47	0.00601173099563462\\
599.48	0.00606985134466992\\
599.49	0.00612852764864716\\
599.5	0.00618776525281058\\
599.51	0.00624756955375529\\
599.52	0.00630794599992009\\
599.53	0.00636890009208503\\
599.54	0.00643043738387374\\
599.55	0.00649256348226047\\
599.56	0.00655528404808225\\
599.57	0.00661860479655563\\
599.58	0.0066825314977986\\
599.59	0.00674706997735748\\
599.6	0.0068122261167388\\
599.61	0.00687800585394632\\
599.62	0.00694441518402315\\
599.63	0.00701146015959914\\
599.64	0.00707914689144346\\
599.65	0.00714748154902254\\
599.66	0.00721647036106325\\
599.67	0.00728611961612159\\
599.68	0.00735643566315677\\
599.69	0.00742742491211079\\
599.7	0.00749909383449364\\
599.71	0.00757144896397401\\
599.72	0.00764449689697572\\
599.73	0.00771824429327989\\
599.74	0.00779269787663285\\
599.75	0.00786786443535983\\
599.76	0.00794375082298461\\
599.77	0.0080203639588551\\
599.78	0.00809771082877486\\
599.79	0.00817579848564071\\
599.8	0.00825463405008648\\
599.81	0.00833422471113285\\
599.82	0.00841457772684349\\
599.83	0.00849570042498746\\
599.84	0.00857760020370797\\
599.85	0.00866028453219752\\
599.86	0.00874376095137953\\
599.87	0.00882803707459654\\
599.88	0.0089131205883049\\
599.89	0.00899901925277625\\
599.9	0.00908574090280562\\
599.91	0.00917329344842635\\
599.92	0.0092616848756319\\
599.93	0.00935092324710449\\
599.94	0.00944101670295078\\
599.95	0.00953197346144464\\
599.96	0.00962380181977693\\
599.97	0.00971651015481255\\
599.98	0.0098101069238547\\
599.99	0.00990460066541651\\
600	0.01\\
};
\end{axis}
\end{tikzpicture}%
  \caption{Continuous Time}
\end{subfigure}%
\hfill%
\begin{subfigure}{.45\linewidth}
  \centering
  \setlength\figureheight{\linewidth} 
  \setlength\figurewidth{\linewidth}
  \tikzsetnextfilename{testdp_dscr_z1}
  % This file was created by matlab2tikz.
%
%The latest updates can be retrieved from
%  http://www.mathworks.com/matlabcentral/fileexchange/22022-matlab2tikz-matlab2tikz
%where you can also make suggestions and rate matlab2tikz.
%
\definecolor{mycolor1}{rgb}{1.00000,0.00000,1.00000}%
%
\begin{tikzpicture}[trim axis left, trim axis right]

\begin{axis}[%
width=\figurewidth,
height=\figureheight,
at={(0\figurewidth,0\figureheight)},
scale only axis,
every outer x axis line/.append style={black},
every x tick label/.append style={font=\color{black}},
xmin=0,
xmax=100,
xlabel={Time},
every outer y axis line/.append style={black},
every y tick label/.append style={font=\color{black}},
ymin=0,
ymax=0.015,
ylabel={Depth $\delta$},
axis background/.style={fill=white},
title={Discrete Time\\$Z=(\rho = -1, \Delta S = -1)$},
axis x line*=bottom,
axis y line*=left,
]
\addplot [color=green,dashed]
  table[row sep=crcr]{%
1	0.0126967216220857\\
2	0.0126911766553508\\
3	0.0126854244957378\\
4	0.0126794571471091\\
5	0.0126732662900411\\
6	0.0126668432738205\\
7	0.0126601790839146\\
8	0.0126532642645715\\
9	0.0126460887824662\\
10	0.0126386414716695\\
11	0.0126309074598112\\
12	0.0126228534328581\\
13	0.0126144738803348\\
14	0.0126057825592762\\
15	0.0125967674054145\\
16	0.0125874158466489\\
17	0.0125777149221112\\
18	0.0125676518288635\\
19	0.0125572160964152\\
20	0.0125464065829753\\
21	0.012535255438371\\
22	0.0125238277676431\\
23	0.0125119439922694\\
24	0.0124995838925062\\
25	0.0124867253188147\\
26	0.0124733448453486\\
27	0.012459419125432\\
28	0.012444933178067\\
29	0.0124298672579279\\
30	0.0124141983529513\\
31	0.0123979000396031\\
32	0.0123809394392587\\
33	0.0123632758539379\\
34	0.0123448831206783\\
35	0.0123258029190969\\
36	0.0123063579176832\\
37	0.0122948431826702\\
38	0.0122826746106595\\
39	0.0122697489999676\\
40	0.0122559127763505\\
41	0.0122409423287981\\
42	0.0122247833367846\\
43	0.0122077090117757\\
44	0.0121896254167272\\
45	0.0121704206085886\\
46	0.0121499381712301\\
47	0.0121278682167617\\
48	0.0121033214534428\\
49	0.012057244947584\\
50	0.0119927062994268\\
51	0.0119258128569804\\
52	0.0118563553512363\\
53	0.0117841151637177\\
54	0.011708955006812\\
55	0.0116305529803687\\
56	0.0115208018328421\\
57	0.0113744009336203\\
58	0.0112246429079446\\
59	0.0110714809763842\\
60	0.0109144901049758\\
61	0.0107540938228196\\
62	0.0105907214355421\\
63	0.0104247831009525\\
64	0.0103389180097715\\
65	0.0102689495413488\\
66	0.0101986182069349\\
67	0.0101280884932016\\
68	0.0100581445773063\\
69	0.00998867805764777\\
70	0.00992023771067554\\
71	0.00985490987643991\\
72	0.00979055286639061\\
73	0.00972465253538295\\
74	0.00965715212588177\\
75	0.00958775863428604\\
76	0.00951625001316514\\
77	0.00944304654577188\\
78	0.0093676337122019\\
79	0.00929040728981676\\
80	0.00921125473484573\\
81	0.00912994292596067\\
82	0.00904545617036328\\
83	0.0089586369742139\\
84	0.00886976238995432\\
85	0.00877877634016461\\
86	0.00868545009930632\\
87	0.00858962774069853\\
88	0.00848937276806187\\
89	0.00838785675903304\\
90	0.00821118656248098\\
91	0.00801216764059995\\
92	0.00773569972584507\\
93	0.00725437032485447\\
94	0.00665687490612621\\
95	0.00613676242368112\\
96	0.00574362775386832\\
97	0.00505491021526543\\
98	0.00355084777151062\\
99	0\\
100	0\\
};
\addlegendentry{$q=-4$};

\addplot [color=mycolor1,dashed]
  table[row sep=crcr]{%
1	0.0129856045043362\\
2	0.0129822538654271\\
3	0.0129787746933184\\
4	0.0129751616878654\\
5	0.0129714092934589\\
6	0.0129675116766721\\
7	0.0129634626912123\\
8	0.0129592558149231\\
9	0.0129548839713748\\
10	0.0129503389645035\\
11	0.0129456098389892\\
12	0.012940681480507\\
13	0.0129355501669858\\
14	0.0129302165834978\\
15	0.0129246716083036\\
16	0.0129189056457788\\
17	0.0129129087727571\\
18	0.0129066712687333\\
19	0.0129001851521933\\
20	0.0128934479980573\\
21	0.0128864682679866\\
22	0.0128792467678925\\
23	0.012871707566661\\
24	0.0128638328128444\\
25	0.0128556033206079\\
26	0.0128469989321532\\
27	0.0128379985213246\\
28	0.0128285776587918\\
29	0.0128187088268875\\
30	0.0128083659511024\\
31	0.012797515380411\\
32	0.0127861180334234\\
33	0.0127741332109041\\
34	0.0127615242852429\\
35	0.0127482662811104\\
36	0.0127366827171888\\
37	0.0127190399444645\\
38	0.0127006520061875\\
39	0.0126814541108304\\
40	0.0126613711872214\\
41	0.0126403373095481\\
42	0.0126182723325326\\
43	0.0125951887515214\\
44	0.0125710562280958\\
45	0.0125457916001141\\
46	0.0125192740366137\\
47	0.0124913180328565\\
48	0.0124617207988576\\
49	0.0124306504650677\\
50	0.0123984429883102\\
51	0.0123648479623234\\
52	0.0123299879846532\\
53	0.0122933179431204\\
54	0.0122544667538672\\
55	0.0122125221004401\\
56	0.0121508932082245\\
57	0.0120667890621589\\
58	0.0119795987945139\\
59	0.0118911409837508\\
60	0.0118150951934112\\
61	0.0117359697918491\\
62	0.0116535008464507\\
63	0.0115668494192426\\
64	0.011425161894664\\
65	0.0112691603761261\\
66	0.0111090933800572\\
67	0.0109447517068695\\
68	0.0107760724444617\\
69	0.0106032322186129\\
70	0.0104266573272997\\
71	0.01024746461265\\
72	0.0101063155023793\\
73	0.010020043962939\\
74	0.00993220351405599\\
75	0.00984293956320838\\
76	0.00975326632229672\\
77	0.00966358591246109\\
78	0.00957552791947833\\
79	0.00948699342712069\\
80	0.00939843088438885\\
81	0.00931038240347798\\
82	0.00922020223285864\\
83	0.00912752453324444\\
84	0.00903245284317054\\
85	0.00893492163961726\\
86	0.00883485623418918\\
87	0.0087322177401472\\
88	0.00862691489199603\\
89	0.00851786573398347\\
90	0.00840566327603898\\
91	0.00829076007526614\\
92	0.00814755572284184\\
93	0.00793531516173855\\
94	0.00770050089829056\\
95	0.00715805987528115\\
96	0.00592456746978032\\
97	0.00505491021526543\\
98	0.00355084777151062\\
99	0\\
100	0\\
};
\addlegendentry{$q=-3$};

\addplot [color=red,dashed]
  table[row sep=crcr]{%
1	0.0133660041345742\\
2	0.0133649985425405\\
3	0.0133639521576256\\
4	0.0133628630839124\\
5	0.0133617293110373\\
6	0.0133605486986157\\
7	0.0133593189514349\\
8	0.0133580375734417\\
9	0.0133567017866217\\
10	0.0133553084353259\\
11	0.0133538541195436\\
12	0.0133523364007293\\
13	0.0133507528690506\\
14	0.0133491000120371\\
15	0.01334737409141\\
16	0.0133455711687318\\
17	0.0133436871912005\\
18	0.0133417181713269\\
19	0.013339660401462\\
20	0.013337510198378\\
21	0.0133352615354812\\
22	0.0133329022991876\\
23	0.0133304251252388\\
24	0.0133278219189346\\
25	0.0133250838125675\\
26	0.0133222010552851\\
27	0.0133191625390637\\
28	0.0133159558056849\\
29	0.0133125674314826\\
30	0.0133089832116995\\
31	0.0133051849132065\\
32	0.0133011505048681\\
33	0.0132968524947565\\
34	0.0132922486223346\\
35	0.01328725578988\\
36	0.0132800809359116\\
37	0.0132693409134395\\
38	0.0132581734571374\\
39	0.0132465486247589\\
40	0.0132344324925976\\
41	0.0132217924733293\\
42	0.0132085788742152\\
43	0.0131947883263311\\
44	0.0131803968337275\\
45	0.0131653605311704\\
46	0.0131496252304385\\
47	0.0131331326480878\\
48	0.0131158780985838\\
49	0.0130978752020875\\
50	0.0130791248421649\\
51	0.0130594825380877\\
52	0.0130386277888343\\
53	0.0130166419419974\\
54	0.0129932301314035\\
55	0.0129681276029774\\
56	0.0129413437312889\\
57	0.012912966270936\\
58	0.012882762385406\\
59	0.0128492213430169\\
60	0.0128037713468365\\
61	0.0127558045353237\\
62	0.0127047605146994\\
63	0.0126495225678913\\
64	0.0125607637022288\\
65	0.0124622749469665\\
66	0.0123611613224151\\
67	0.0122565881576478\\
68	0.0121477380418608\\
69	0.0120342752234703\\
70	0.0119195337674421\\
71	0.0118024553426227\\
72	0.011653872989352\\
73	0.011464738764733\\
74	0.0112707232604999\\
75	0.0110712846402284\\
76	0.010865535551453\\
77	0.0106762299032837\\
78	0.0104843448212728\\
79	0.0102896238788441\\
80	0.0100875909154579\\
81	0.00987756510301356\\
82	0.0097308864213279\\
83	0.00960224991070578\\
84	0.00947184311749152\\
85	0.00933914710917778\\
86	0.00920664389467955\\
87	0.00907652416718605\\
88	0.00894263834053335\\
89	0.00880463054234541\\
90	0.00866298575430099\\
91	0.00851835770472905\\
92	0.00837118372262871\\
93	0.00822205565507644\\
94	0.00807170093885098\\
95	0.00783250554120718\\
96	0.0074324720476095\\
97	0.00543592658335712\\
98	0.00355084777151062\\
99	0\\
100	0\\
};
\addlegendentry{$q=-2$};

\addplot [color=blue,dashed]
  table[row sep=crcr]{%
1	0.0135317146551453\\
2	0.0135315608261053\\
3	0.0135314005390358\\
4	0.0135312334767243\\
5	0.0135310593013548\\
6	0.0135308776511215\\
7	0.0135306881342496\\
8	0.0135304903186408\\
9	0.0135302837194055\\
10	0.0135300678132787\\
11	0.0135298421826273\\
12	0.0135296064071407\\
13	0.0135293599287123\\
14	0.013529102153335\\
15	0.013528832453455\\
16	0.0135285501769609\\
17	0.013528254666963\\
18	0.0135279452874482\\
19	0.0135276213979559\\
20	0.0135272820723854\\
21	0.0135269256598165\\
22	0.0135265510360669\\
23	0.0135261569698328\\
24	0.0135257421060612\\
25	0.0135253049431025\\
26	0.0135248438013243\\
27	0.0135243567981914\\
28	0.0135238418766312\\
29	0.0135232970190877\\
30	0.0135227196415019\\
31	0.013522106606347\\
32	0.0135214539712667\\
33	0.0135207560601477\\
34	0.0135200024793008\\
35	0.0135191666513325\\
36	0.0135172829937066\\
37	0.0135153213663433\\
38	0.0135132821840627\\
39	0.0135111594766569\\
40	0.013508946280691\\
41	0.0135066368013251\\
42	0.0135042284544917\\
43	0.0135017161848457\\
44	0.0134990929245518\\
45	0.0134963509555567\\
46	0.0134934822311073\\
47	0.0134904793381017\\
48	0.0134873357348835\\
49	0.0134840450438983\\
50	0.0134805862622556\\
51	0.0134769348974278\\
52	0.0134727995789019\\
53	0.0134682936817264\\
54	0.0134634607952326\\
55	0.0134582913927417\\
56	0.013452784758758\\
57	0.0134468901237203\\
58	0.0134405336742243\\
59	0.01343289578655\\
60	0.0134189018350896\\
61	0.013404150707504\\
62	0.0133884738742701\\
63	0.0133715682000753\\
64	0.0133537616925606\\
65	0.0133350339586812\\
66	0.0133142847501557\\
67	0.0132919259126887\\
68	0.0132683212132686\\
69	0.0132432959639189\\
70	0.0132141421005538\\
71	0.0131802680002577\\
72	0.0131288097559871\\
73	0.0130538307953416\\
74	0.0129760685627625\\
75	0.012895246920584\\
76	0.0128107313559607\\
77	0.0127083463508467\\
78	0.0125996771817469\\
79	0.0124844310082188\\
80	0.0123634245427691\\
81	0.0122360424318266\\
82	0.0120592949160788\\
83	0.0118621580300995\\
84	0.0116584140531364\\
85	0.0114463534785483\\
86	0.0112271122038623\\
87	0.0110012918727073\\
88	0.0107697666903376\\
89	0.0105268926754227\\
90	0.0102718492305835\\
91	0.0100037436489004\\
92	0.00972105195995043\\
93	0.00942146457752256\\
94	0.009101952469783\\
95	0.00875946745630175\\
96	0.0083888525844387\\
97	0.00758920978615654\\
98	0.00355084777151062\\
99	0\\
100	0\\
};
\addlegendentry{$q=-1$};

\addplot [color=black,solid]
  table[row sep=crcr]{%
1	0.00591981360054696\\
2	0.00591981360054696\\
3	0.00591981360054696\\
4	0.00591981360054696\\
5	0.00591981360054696\\
6	0.00591981360054696\\
7	0.00591981360054696\\
8	0.00591981360054696\\
9	0.00591981360054696\\
10	0.00591981360054696\\
11	0.00591981360054696\\
12	0.00591981360054696\\
13	0.00591981360054696\\
14	0.00591981360054696\\
15	0.00591981360054696\\
16	0.00591981360054696\\
17	0.00591981360054696\\
18	0.00591981360054696\\
19	0.00591981360054696\\
20	0.00591981360054696\\
21	0.00591981360054696\\
22	0.00591981360054696\\
23	0.00591981360054696\\
24	0.00591981360054696\\
25	0.00591981360054696\\
26	0.00591981360054696\\
27	0.00591981360054696\\
28	0.00591981360054696\\
29	0.00591981360054696\\
30	0.00591981360054696\\
31	0.00591981360054696\\
32	0.00591981360054696\\
33	0.00591981360054696\\
34	0.00591981360054696\\
35	0.00591981360054696\\
36	0.00591981360054696\\
37	0.00591981360054696\\
38	0.00591981360054696\\
39	0.00591981360054696\\
40	0.00591981360054696\\
41	0.00591981360054696\\
42	0.00591981360054696\\
43	0.00591981360054696\\
44	0.00591981360054696\\
45	0.00591981360054696\\
46	0.00591981360054696\\
47	0.00591981360054696\\
48	0.00591981360054696\\
49	0.00591981360054696\\
50	0.00591981360054696\\
51	0.00591981360054696\\
52	0.0059197314014602\\
53	0.00591957612551109\\
54	0.00591941017485126\\
55	0.00591923196648035\\
56	0.00591903973740182\\
57	0.00591883147966023\\
58	0.00591860481779524\\
59	0.00591835693701099\\
60	0.0059180844892575\\
61	0.00591778347486291\\
62	0.00591744916699088\\
63	0.00591707593995693\\
64	0.00591665713989317\\
65	0.00591618491490205\\
66	0.00596393855843097\\
67	0.00603036864143203\\
68	0.00610041656553486\\
69	0.0061743871334594\\
70	0.00625042468725859\\
71	0.00635396470808304\\
72	0.00646122841449285\\
73	0.00657204709940429\\
74	0.00672553666866284\\
75	0.00693282394483811\\
76	0.00714140644633385\\
77	0.00733859058735712\\
78	0.00753183746900076\\
79	0.00770503670347887\\
80	0.00788383015588474\\
81	0.00806847568100207\\
82	0.00823066202890321\\
83	0.00838615838767804\\
84	0.00854300538107573\\
85	0.00870117949536376\\
86	0.00886185907997313\\
87	0.00902167742403161\\
88	0.00918586905823183\\
89	0.00936316415379578\\
90	0.00955360283913768\\
91	0.00974697832940791\\
92	0.00994205264286296\\
93	0.0101405890941118\\
94	0.0103511869865303\\
95	0.0105897612397742\\
96	0.010890392895037\\
97	0.0113172670570267\\
98	0.0116898933452315\\
99	0\\
100	0\\
};
\addlegendentry{$q=0$};

\addplot [color=blue,solid]
  table[row sep=crcr]{%
1	0.00451614935682934\\
2	0.00452311676043263\\
3	0.00453027508233792\\
4	0.00453763279764905\\
5	0.00454520219945658\\
6	0.00455300549604212\\
7	0.00456108966975832\\
8	0.00456955742544878\\
9	0.00457859628509222\\
10	0.00458822014349924\\
11	0.00459810515859038\\
12	0.00460826524130698\\
13	0.00461872981492281\\
14	0.00462959478680276\\
15	0.00464124774693008\\
16	0.00465536313543588\\
17	0.00467930414929815\\
18	0.0047156224582261\\
19	0.00475305414923783\\
20	0.00479162545081955\\
21	0.00483132816510595\\
22	0.00487207179032189\\
23	0.00491360710969711\\
24	0.00495554809770774\\
25	0.00499842604340389\\
26	0.00504325978533359\\
27	0.00508970088284333\\
28	0.00513784254184666\\
29	0.00518778600467447\\
30	0.00523964153556477\\
31	0.00529352983872942\\
32	0.00534958458445824\\
33	0.00540795802037152\\
34	0.00546883551277045\\
35	0.00553247639970869\\
36	0.00559933305430107\\
37	0.00567040356050585\\
38	0.00574906251218784\\
39	0.00583787816885141\\
40	0.00592908614578806\\
41	0.00602278644817338\\
42	0.00611908733218572\\
43	0.00621809454499509\\
44	0.00631992047549828\\
45	0.0064246372639008\\
46	0.00653238053549258\\
47	0.00664329530107697\\
48	0.00675753619558864\\
49	0.00687526780032528\\
50	0.00699666755464488\\
51	0.00712194255786227\\
52	0.00725140172825184\\
53	0.00738571419701336\\
54	0.00752619676438501\\
55	0.00767734679906207\\
56	0.00783516490553262\\
57	0.008000939758594\\
58	0.00818375632458021\\
59	0.00837348572120451\\
60	0.0085681593084036\\
61	0.00876798883825665\\
62	0.00897314257728339\\
63	0.00918363852233112\\
64	0.00939904059078436\\
65	0.00961766273198416\\
66	0.00979911843910775\\
67	0.00995107288608682\\
68	0.010048972033477\\
69	0.0101472012981851\\
70	0.0102466420766812\\
71	0.0103272687637985\\
72	0.010407754383832\\
73	0.0104881556812313\\
74	0.0105709812055244\\
75	0.0106492339046446\\
76	0.0107264552869942\\
77	0.01080322383852\\
78	0.0108814839359471\\
79	0.0109617983024247\\
80	0.0110434235149282\\
81	0.0111220709407802\\
82	0.0112004869707467\\
83	0.0112800284837741\\
84	0.0113607313049829\\
85	0.0114424985291478\\
86	0.0115241054748618\\
87	0.0116073275602494\\
88	0.0116925866191128\\
89	0.0117844422105768\\
90	0.0118962795727799\\
91	0.0120265210753947\\
92	0.0121587847108071\\
93	0.0122936710395094\\
94	0.0124330125022277\\
95	0.0125820167352517\\
96	0.0127697443304968\\
97	0.0131086352216726\\
98	0.0135508477715106\\
99	0\\
100	0\\
};
\addlegendentry{$q=1$};

\addplot [color=red,solid]
  table[row sep=crcr]{%
1	0.00823461842574045\\
2	0.00826899602495442\\
3	0.00830442877152966\\
4	0.00834095967252183\\
5	0.00837863555996613\\
6	0.00841750975872967\\
7	0.00845765290518301\\
8	0.00849919332502241\\
9	0.00854249419452816\\
10	0.00858906396359684\\
11	0.00863990299031478\\
12	0.00869238162082222\\
13	0.00874655944120017\\
14	0.00880244889580168\\
15	0.00885983719858435\\
16	0.00891749908383751\\
17	0.0089697006586655\\
18	0.00901440876918852\\
19	0.00906043576208444\\
20	0.00910781177920156\\
21	0.00915653388374698\\
22	0.00920649599065296\\
23	0.00925726322100709\\
24	0.00930729853774357\\
25	0.00935113802307131\\
26	0.00938986627572017\\
27	0.00943282038810339\\
28	0.00947681869098303\\
29	0.00952186967365653\\
30	0.00956797914951497\\
31	0.00961514990469749\\
32	0.00966338147105382\\
33	0.00971267035956913\\
34	0.00976301173111001\\
35	0.00981440520254744\\
36	0.0098668713497011\\
37	0.00992048852035668\\
38	0.00997312244123629\\
39	0.0100211590770693\\
40	0.0100705380917287\\
41	0.0101212936933832\\
42	0.010173470373309\\
43	0.0102271425484395\\
44	0.0102823278688608\\
45	0.0103391539930373\\
46	0.0103974002909443\\
47	0.0104570996066606\\
48	0.0105182837235764\\
49	0.0105809817244718\\
50	0.0106452160441145\\
51	0.0107109919498944\\
52	0.0107782643963154\\
53	0.0108468095872655\\
54	0.0109171923145941\\
55	0.0109860994684566\\
56	0.0110570228370002\\
57	0.0111329976652417\\
58	0.0112159842069506\\
59	0.0113127764473596\\
60	0.0114130129655266\\
61	0.0115146276351316\\
62	0.0116175031699437\\
63	0.0117214496776731\\
64	0.0118261307886065\\
65	0.0119307526606845\\
66	0.0120203652489965\\
67	0.0120984408832417\\
68	0.0121568910226901\\
69	0.0122118012863447\\
70	0.0122652910119001\\
71	0.0123106732973323\\
72	0.0123549332729736\\
73	0.0123984456662247\\
74	0.0124414345932572\\
75	0.0124812997201541\\
76	0.0125194500415809\\
77	0.0125569664043024\\
78	0.0125943812824734\\
79	0.0126309404611539\\
80	0.012665963838849\\
81	0.0126994759494684\\
82	0.0127323389053315\\
83	0.0127647722281926\\
84	0.0127968511000606\\
85	0.0128286531811732\\
86	0.0128603301180877\\
87	0.0128963729173589\\
88	0.0129318836587637\\
89	0.0129675724876054\\
90	0.0130033415127081\\
91	0.0130390115143561\\
92	0.0130750839244785\\
93	0.0131135099546856\\
94	0.0131538072419449\\
95	0.0132031167059528\\
96	0.0132828620400256\\
97	0.0134008680318943\\
98	0.0135508477715106\\
99	0\\
100	0\\
};
\addlegendentry{$q=2$};

\addplot [color=mycolor1,solid]
  table[row sep=crcr]{%
1	0.0104070598822911\\
2	0.0104304807816848\\
3	0.0104545314997298\\
4	0.010479226888311\\
5	0.0105045825049643\\
6	0.0105306151605219\\
7	0.0105573425827522\\
8	0.0105847947376939\\
9	0.0106130709466891\\
10	0.0106426290787016\\
11	0.0106744578220093\\
12	0.010708476001213\\
13	0.0107432761490129\\
14	0.0107788359358436\\
15	0.0108150352469957\\
16	0.0108513302663978\\
17	0.0108853842988561\\
18	0.0109164722361298\\
19	0.0109482602322412\\
20	0.0109806318466408\\
21	0.0110135579561168\\
22	0.0110469730697013\\
23	0.011080699651873\\
24	0.011114173882034\\
25	0.011145388157608\\
26	0.0111741490791111\\
27	0.011201393028691\\
28	0.0112289339368627\\
29	0.011257015815042\\
30	0.0112856372065345\\
31	0.0113147950576296\\
32	0.0113444845167581\\
33	0.0113746987074325\\
34	0.0114054284472627\\
35	0.0114366617439412\\
36	0.0114683819974893\\
37	0.0115005581133852\\
38	0.0115322635464395\\
39	0.0115621491867184\\
40	0.0115928099751743\\
41	0.0116242661110847\\
42	0.011656543489315\\
43	0.01168970345219\\
44	0.0117248550277529\\
45	0.0117612297508009\\
46	0.0117978191438218\\
47	0.0118345701178826\\
48	0.0118714229622976\\
49	0.0119083102694157\\
50	0.0119451547256269\\
51	0.0119818630419612\\
52	0.0120183057633091\\
53	0.0120542430501936\\
54	0.0120896436841004\\
55	0.0121228176639475\\
56	0.0121552305302407\\
57	0.0121864443881354\\
58	0.0122167370122164\\
59	0.01224690152124\\
60	0.0122768848962182\\
61	0.0123066099007683\\
62	0.0123360050500266\\
63	0.0123648760476127\\
64	0.0123929406640282\\
65	0.0124202849765883\\
66	0.0124471178681301\\
67	0.0124732830497997\\
68	0.0124986575848265\\
69	0.0125224753421137\\
70	0.0125465845232579\\
71	0.0125712118834182\\
72	0.0125955234053651\\
73	0.0126196107228879\\
74	0.0126428678643362\\
75	0.0126673160897811\\
76	0.0126947312143728\\
77	0.0127224891406568\\
78	0.0127504433547237\\
79	0.012778096208618\\
80	0.0128056775984883\\
81	0.0128338097045545\\
82	0.0128624321267579\\
83	0.0128914557721687\\
84	0.0129209761633547\\
85	0.0129515363179389\\
86	0.0129843922586545\\
87	0.0130145080675542\\
88	0.0130448637247913\\
89	0.0130754225509084\\
90	0.01310614002025\\
91	0.0131370417925568\\
92	0.0131684013089337\\
93	0.0131992881190797\\
94	0.0132339377541053\\
95	0.0132837658672037\\
96	0.0133434148917366\\
97	0.0134129558428687\\
98	0.0135508477715106\\
99	0\\
100	0\\
};
\addlegendentry{$q=3$};

\addplot [color=green,solid]
  table[row sep=crcr]{%
1	0.0112173328374135\\
2	0.0112304770601278\\
3	0.0112439182131143\\
4	0.0112576571891033\\
5	0.01127169491468\\
6	0.0112860341128129\\
7	0.0113006833622851\\
8	0.0113156343059301\\
9	0.0113308631684719\\
10	0.0113463716299626\\
11	0.0113621611579808\\
12	0.0113782330923898\\
13	0.0113945884538718\\
14	0.0114112278659182\\
15	0.0114281513734847\\
16	0.0114453582960558\\
17	0.0114628742299139\\
18	0.0114807550685414\\
19	0.0114990214645766\\
20	0.0115176414336741\\
21	0.01153662032998\\
22	0.0115559632077397\\
23	0.0115756742070751\\
24	0.0115957546069088\\
25	0.0116161960754078\\
26	0.0116368416532018\\
27	0.0116569176878322\\
28	0.0116777533744845\\
29	0.0117002489592426\\
30	0.011722984948825\\
31	0.0117459461714951\\
32	0.0117691156809455\\
33	0.011792474633478\\
34	0.0118160021561575\\
35	0.0118396751971485\\
36	0.0118634683443689\\
37	0.0118873536627004\\
38	0.0119113147065166\\
39	0.0119353555472121\\
40	0.0119594389041357\\
41	0.011983522112995\\
42	0.0120075530995899\\
43	0.0120314523610817\\
44	0.012054465981151\\
45	0.0120770512355319\\
46	0.0120997330324277\\
47	0.0121224858909302\\
48	0.0121452824402504\\
49	0.0121680933715113\\
50	0.0121908873902304\\
51	0.0122136311461536\\
52	0.0122362886243086\\
53	0.0122588178504758\\
54	0.0122811820870626\\
55	0.0123033696464428\\
56	0.0123253090757543\\
57	0.0123465567831379\\
58	0.0123673593529313\\
59	0.0123881606288535\\
60	0.0124087928970967\\
61	0.0124294775489006\\
62	0.0124502172466953\\
63	0.0124709391635829\\
64	0.0124955518434812\\
65	0.012520604250692\\
66	0.0125454406786979\\
67	0.0125717908948519\\
68	0.0125979829530069\\
69	0.0126239699816152\\
70	0.0126488841755872\\
71	0.0126730572520101\\
72	0.012697528992541\\
73	0.0127223355983777\\
74	0.0127475049668732\\
75	0.0127714878551888\\
76	0.0127940838119219\\
77	0.012817096660328\\
78	0.0128407413392876\\
79	0.0128662424001927\\
80	0.0128927948367257\\
81	0.0129196990953362\\
82	0.0129469289363823\\
83	0.0129744655449398\\
84	0.0130023254225124\\
85	0.013030341472698\\
86	0.0130568081275538\\
87	0.0130818372586511\\
88	0.0131071398482185\\
89	0.0131326828575159\\
90	0.0131585197059812\\
91	0.0131841281185588\\
92	0.0132134166715202\\
93	0.013243972340567\\
94	0.0132770651881871\\
95	0.0133105595018936\\
96	0.0133500513908948\\
97	0.0134129558428687\\
98	0.0135508477715106\\
99	0\\
100	0\\
};
\addlegendentry{$q=4$};

\end{axis}
\end{tikzpicture}%
 
  \caption{Discrete Time}
\end{subfigure}\\
\vspace{1cm}
\begin{subfigure}{.45\linewidth}
  \centering
  \setlength\figureheight{\linewidth} 
  \setlength\figurewidth{\linewidth}
  \tikzsetnextfilename{testdp_cts_nFPC_z1}
  % This file was created by matlab2tikz.
%
%The latest updates can be retrieved from
%  http://www.mathworks.com/matlabcentral/fileexchange/22022-matlab2tikz-matlab2tikz
%where you can also make suggestions and rate matlab2tikz.
%
\definecolor{mycolor1}{rgb}{1.00000,0.00000,1.00000}%
%
\begin{tikzpicture}

\begin{axis}[%
width=4.564in,
height=3.803in,
at={(1.067in,0.513in)},
scale only axis,
every outer x axis line/.append style={black},
every x tick label/.append style={font=\color{black}},
xmin=0,
xmax=100,
xlabel={Time},
every outer y axis line/.append style={black},
every y tick label/.append style={font=\color{black}},
ymin=0,
ymax=0.01,
ylabel={Depth $\delta$},
axis background/.style={fill=white},
title={Z=1},
axis x line*=bottom,
axis y line*=left,
legend style={legend cell align=left,align=left,draw=black}
]
\addplot [color=green,dashed,forget plot]
  table[row sep=crcr]{%
0.01	0\\
0.02	0\\
0.03	0\\
0.04	0\\
0.05	0\\
0.06	0\\
0.07	0\\
0.08	0\\
0.09	0\\
0.1	0\\
0.11	0\\
0.12	0\\
0.13	0\\
0.14	0\\
0.15	0\\
0.16	0\\
0.17	1.73472347597681e-18\\
0.18	1.73472347597681e-18\\
0.19	0\\
0.2	0\\
0.21	0\\
0.22	0\\
0.23	0\\
0.24	0\\
0.25	0\\
0.26	0\\
0.27	0\\
0.28	0\\
0.29	0\\
0.3	0\\
0.31	0\\
0.32	0\\
0.33	0\\
0.34	1.73472347597681e-18\\
0.35	1.73472347597681e-18\\
0.36	0\\
0.37	0\\
0.38	0\\
0.39	0\\
0.4	0\\
0.41	0\\
0.42	0\\
0.43	0\\
0.44	0\\
0.45	0\\
0.46	0\\
0.47	0\\
0.48	0\\
0.49	0\\
0.5	0\\
0.51	1.73472347597681e-18\\
0.52	1.73472347597681e-18\\
0.53	0\\
0.54	0\\
0.55	1.73472347597681e-18\\
0.56	0\\
0.57	0\\
0.58	0\\
0.59	0\\
0.6	1.73472347597681e-18\\
0.61	0\\
0.62	0\\
0.63	1.73472347597681e-18\\
0.64	1.73472347597681e-18\\
0.65	0\\
0.66	0\\
0.67	1.73472347597681e-18\\
0.68	0\\
0.69	0\\
0.7	0\\
0.71	0\\
0.72	0\\
0.73	0\\
0.74	0\\
0.75	0\\
0.76	0\\
0.77	1.73472347597681e-18\\
0.78	0\\
0.79	0\\
0.8	0\\
0.81	0\\
0.82	0\\
0.83	0\\
0.84	0\\
0.85	0\\
0.86	0\\
0.87	0\\
0.88	0\\
0.89	0\\
0.9	0\\
0.91	0\\
0.92	0\\
0.93	0\\
0.94	0\\
0.95	0\\
0.96	0\\
0.97	0\\
0.98	0\\
0.99	1.73472347597681e-18\\
1	0\\
1.01	1.73472347597681e-18\\
1.02	0\\
1.03	0\\
1.04	1.73472347597681e-18\\
1.05	0\\
1.06	0\\
1.07	0\\
1.08	0\\
1.09	0\\
1.1	0\\
1.11	0\\
1.12	0\\
1.13	1.73472347597681e-18\\
1.14	0\\
1.15	0\\
1.16	0\\
1.17	0\\
1.18	0\\
1.19	1.73472347597681e-18\\
1.2	0\\
1.21	0\\
1.22	1.73472347597681e-18\\
1.23	1.73472347597681e-18\\
1.24	0\\
1.25	1.73472347597681e-18\\
1.26	0\\
1.27	0\\
1.28	0\\
1.29	0\\
1.3	0\\
1.31	0\\
1.32	0\\
1.33	0\\
1.34	0\\
1.35	0\\
1.36	0\\
1.37	0\\
1.38	0\\
1.39	0\\
1.4	0\\
1.41	0\\
1.42	0\\
1.43	0\\
1.44	0\\
1.45	0\\
1.46	0\\
1.47	0\\
1.48	0\\
1.49	0\\
1.5	0\\
1.51	0\\
1.52	0\\
1.53	0\\
1.54	0\\
1.55	0\\
1.56	0\\
1.57	0\\
1.58	0\\
1.59	0\\
1.6	0\\
1.61	0\\
1.62	0\\
1.63	0\\
1.64	0\\
1.65	0\\
1.66	0\\
1.67	0\\
1.68	0\\
1.69	0\\
1.7	0\\
1.71	0\\
1.72	0\\
1.73	0\\
1.74	0\\
1.75	0\\
1.76	0\\
1.77	0\\
1.78	0\\
1.79	0\\
1.8	1.73472347597681e-18\\
1.81	0\\
1.82	0\\
1.83	0\\
1.84	0\\
1.85	0\\
1.86	0\\
1.87	0\\
1.88	0\\
1.89	0\\
1.9	0\\
1.91	0\\
1.92	0\\
1.93	0\\
1.94	0\\
1.95	0\\
1.96	1.73472347597681e-18\\
1.97	0\\
1.98	0\\
1.99	0\\
2	0\\
2.01	0\\
2.02	0\\
2.03	1.73472347597681e-18\\
2.04	0\\
2.05	1.73472347597681e-18\\
2.06	1.73472347597681e-18\\
2.07	0\\
2.08	0\\
2.09	1.73472347597681e-18\\
2.1	0\\
2.11	0\\
2.12	0\\
2.13	1.73472347597681e-18\\
2.14	0\\
2.15	0\\
2.16	0\\
2.17	0\\
2.18	0\\
2.19	0\\
2.2	0\\
2.21	0\\
2.22	0\\
2.23	0\\
2.24	1.73472347597681e-18\\
2.25	1.73472347597681e-18\\
2.26	0\\
2.27	0\\
2.28	0\\
2.29	0\\
2.3	1.73472347597681e-18\\
2.31	0\\
2.32	0\\
2.33	0\\
2.34	0\\
2.35	1.73472347597681e-18\\
2.36	0\\
2.37	0\\
2.38	0\\
2.39	0\\
2.4	0\\
2.41	0\\
2.42	0\\
2.43	0\\
2.44	0\\
2.45	1.73472347597681e-18\\
2.46	0\\
2.47	0\\
2.48	0\\
2.49	0\\
2.5	0\\
2.51	0\\
2.52	1.73472347597681e-18\\
2.53	0\\
2.54	0\\
2.55	1.73472347597681e-18\\
2.56	0\\
2.57	0\\
2.58	0\\
2.59	0\\
2.6	0\\
2.61	0\\
2.62	0\\
2.63	0\\
2.64	0\\
2.65	0\\
2.66	0\\
2.67	1.73472347597681e-18\\
2.68	0\\
2.69	0\\
2.7	0\\
2.71	0\\
2.72	0\\
2.73	0\\
2.74	0\\
2.75	0\\
2.76	0\\
2.77	0\\
2.78	0\\
2.79	0\\
2.8	0\\
2.81	0\\
2.82	0\\
2.83	0\\
2.84	0\\
2.85	0\\
2.86	0\\
2.87	0\\
2.88	0\\
2.89	0\\
2.9	0\\
2.91	0\\
2.92	0\\
2.93	0\\
2.94	0\\
2.95	0\\
2.96	1.73472347597681e-18\\
2.97	0\\
2.98	0\\
2.99	0\\
3	1.73472347597681e-18\\
3.01	0\\
3.02	0\\
3.03	0\\
3.04	0\\
3.05	0\\
3.06	0\\
3.07	0\\
3.08	0\\
3.09	0\\
3.1	0\\
3.11	0\\
3.12	0\\
3.13	1.73472347597681e-18\\
3.14	0\\
3.15	0\\
3.16	0\\
3.17	0\\
3.18	0\\
3.19	0\\
3.2	0\\
3.21	1.73472347597681e-18\\
3.22	0\\
3.23	1.73472347597681e-18\\
3.24	0\\
3.25	0\\
3.26	0\\
3.27	0\\
3.28	0\\
3.29	0\\
3.3	0\\
3.31	0\\
3.32	0\\
3.33	0\\
3.34	0\\
3.35	0\\
3.36	0\\
3.37	0\\
3.38	0\\
3.39	1.73472347597681e-18\\
3.4	0\\
3.41	1.73472347597681e-18\\
3.42	1.73472347597681e-18\\
3.43	0\\
3.44	1.73472347597681e-18\\
3.45	0\\
3.46	0\\
3.47	0\\
3.48	0\\
3.49	0\\
3.5	1.73472347597681e-18\\
3.51	0\\
3.52	0\\
3.53	0\\
3.54	0\\
3.55	0\\
3.56	0\\
3.57	0\\
3.58	0\\
3.59	1.73472347597681e-18\\
3.6	0\\
3.61	0\\
3.62	0\\
3.63	0\\
3.64	0\\
3.65	0\\
3.66	0\\
3.67	0\\
3.68	0\\
3.69	0\\
3.7	0\\
3.71	0\\
3.72	0\\
3.73	0\\
3.74	0\\
3.75	0\\
3.76	0\\
3.77	0\\
3.78	0\\
3.79	0\\
3.8	0\\
3.81	1.73472347597681e-18\\
3.82	0\\
3.83	0\\
3.84	0\\
3.85	0\\
3.86	0\\
3.87	0\\
3.88	0\\
3.89	0\\
3.9	0\\
3.91	0\\
3.92	0\\
3.93	0\\
3.94	0\\
3.95	0\\
3.96	0\\
3.97	0\\
3.98	1.73472347597681e-18\\
3.99	0\\
4	0\\
4.01	1.73472347597681e-18\\
4.02	0\\
4.03	0\\
4.04	0\\
4.05	1.73472347597681e-18\\
4.06	0\\
4.07	1.73472347597681e-18\\
4.08	0\\
4.09	0\\
4.1	0\\
4.11	0\\
4.12	0\\
4.13	0\\
4.14	0\\
4.15	1.73472347597681e-18\\
4.16	0\\
4.17	0\\
4.18	0\\
4.19	0\\
4.2	0\\
4.21	1.73472347597681e-18\\
4.22	0\\
4.23	0\\
4.24	0\\
4.25	0\\
4.26	1.73472347597681e-18\\
4.27	0\\
4.28	0\\
4.29	0\\
4.3	0\\
4.31	0\\
4.32	0\\
4.33	0\\
4.34	0\\
4.35	0\\
4.36	0\\
4.37	0\\
4.38	0\\
4.39	0\\
4.4	0\\
4.41	0\\
4.42	0\\
4.43	0\\
4.44	0\\
4.45	0\\
4.46	0\\
4.47	0\\
4.48	0\\
4.49	0\\
4.5	0\\
4.51	1.73472347597681e-18\\
4.52	0\\
4.53	0\\
4.54	0\\
4.55	0\\
4.56	0\\
4.57	0\\
4.58	0\\
4.59	0\\
4.6	0\\
4.61	0\\
4.62	1.73472347597681e-18\\
4.63	0\\
4.64	0\\
4.65	0\\
4.66	0\\
4.67	0\\
4.68	0\\
4.69	0\\
4.7	0\\
4.71	0\\
4.72	0\\
4.73	0\\
4.74	0\\
4.75	0\\
4.76	0\\
4.77	0\\
4.78	0\\
4.79	0\\
4.8	0\\
4.81	0\\
4.82	1.73472347597681e-18\\
4.83	0\\
4.84	0\\
4.85	1.73472347597681e-18\\
4.86	0\\
4.87	0\\
4.88	0\\
4.89	0\\
4.9	0\\
4.91	0\\
4.92	0\\
4.93	0\\
4.94	0\\
4.95	0\\
4.96	0\\
4.97	0\\
4.98	1.73472347597681e-18\\
4.99	0\\
5	0\\
5.01	0\\
5.02	0\\
5.03	0\\
5.04	0\\
5.05	0\\
5.06	1.73472347597681e-18\\
5.07	1.73472347597681e-18\\
5.08	0\\
5.09	0\\
5.1	0\\
5.11	1.73472347597681e-18\\
5.12	0\\
5.13	0\\
5.14	0\\
5.15	0\\
5.16	0\\
5.17	0\\
5.18	0\\
5.19	0\\
5.2	0\\
5.21	0\\
5.22	0\\
5.23	1.73472347597681e-18\\
5.24	0\\
5.25	0\\
5.26	0\\
5.27	0\\
5.28	0\\
5.29	0\\
5.3	0\\
5.31	0\\
5.32	1.73472347597681e-18\\
5.33	0\\
5.34	1.73472347597681e-18\\
5.35	1.73472347597681e-18\\
5.36	0\\
5.37	0\\
5.38	0\\
5.39	0\\
5.4	0\\
5.41	0\\
5.42	0\\
5.43	0\\
5.44	0\\
5.45	0\\
5.46	1.73472347597681e-18\\
5.47	0\\
5.48	0\\
5.49	1.73472347597681e-18\\
5.5	0\\
5.51	0\\
5.52	0\\
5.53	0\\
5.54	0\\
5.55	0\\
5.56	0\\
5.57	1.73472347597681e-18\\
5.58	0\\
5.59	0\\
5.6	0\\
5.61	0\\
5.62	0\\
5.63	1.73472347597681e-18\\
5.64	0\\
5.65	0\\
5.66	0\\
5.67	0\\
5.68	0\\
5.69	1.73472347597681e-18\\
5.7	0\\
5.71	1.73472347597681e-18\\
5.72	0\\
5.73	0\\
5.74	0\\
5.75	0\\
5.76	1.73472347597681e-18\\
5.77	0\\
5.78	0\\
5.79	0\\
5.8	0\\
5.81	0\\
5.82	0\\
5.83	1.73472347597681e-18\\
5.84	0\\
5.85	0\\
5.86	0\\
5.87	0\\
5.88	0\\
5.89	0\\
5.9	0\\
5.91	0\\
5.92	0\\
5.93	0\\
5.94	0\\
5.95	0\\
5.96	0\\
5.97	0\\
5.98	0\\
5.99	0\\
6	0\\
6.01	0\\
6.02	0\\
6.03	0\\
6.04	1.73472347597681e-18\\
6.05	0\\
6.06	0\\
6.07	0\\
6.08	0\\
6.09	0\\
6.1	0\\
6.11	1.73472347597681e-18\\
6.12	0\\
6.13	0\\
6.14	0\\
6.15	0\\
6.16	0\\
6.17	1.73472347597681e-18\\
6.18	0\\
6.19	1.73472347597681e-18\\
6.2	0\\
6.21	0\\
6.22	0\\
6.23	0\\
6.24	0\\
6.25	1.73472347597681e-18\\
6.26	0\\
6.27	0\\
6.28	1.73472347597681e-18\\
6.29	0\\
6.3	0\\
6.31	0\\
6.32	0\\
6.33	1.73472347597681e-18\\
6.34	0\\
6.35	1.73472347597681e-18\\
6.36	1.73472347597681e-18\\
6.37	0\\
6.38	1.73472347597681e-18\\
6.39	0\\
6.4	0\\
6.41	0\\
6.42	1.73472347597681e-18\\
6.43	0\\
6.44	0\\
6.45	0\\
6.46	1.73472347597681e-18\\
6.47	0\\
6.48	1.73472347597681e-18\\
6.49	0\\
6.5	0\\
6.51	0\\
6.52	0\\
6.53	0\\
6.54	0\\
6.55	0\\
6.56	1.73472347597681e-18\\
6.57	0\\
6.58	0\\
6.59	0\\
6.6	0\\
6.61	1.73472347597681e-18\\
6.62	0\\
6.63	0\\
6.64	0\\
6.65	0\\
6.66	0\\
6.67	0\\
6.68	0\\
6.69	0\\
6.7	0\\
6.71	0\\
6.72	0\\
6.73	0\\
6.74	0\\
6.75	0\\
6.76	0\\
6.77	0\\
6.78	0\\
6.79	0\\
6.8	0\\
6.81	0\\
6.82	0\\
6.83	0\\
6.84	0\\
6.85	0\\
6.86	1.73472347597681e-18\\
6.87	0\\
6.88	0\\
6.89	0\\
6.9	0\\
6.91	0\\
6.92	0\\
6.93	0\\
6.94	0\\
6.95	0\\
6.96	0\\
6.97	0\\
6.98	0\\
6.99	0\\
7	0\\
7.01	0\\
7.02	0\\
7.03	1.73472347597681e-18\\
7.04	0\\
7.05	0\\
7.06	0\\
7.07	0\\
7.08	0\\
7.09	0\\
7.1	0\\
7.11	0\\
7.12	0\\
7.13	0\\
7.14	0\\
7.15	0\\
7.16	0\\
7.17	0\\
7.18	0\\
7.19	1.73472347597681e-18\\
7.2	0\\
7.21	0\\
7.22	0\\
7.23	0\\
7.24	0\\
7.25	0\\
7.26	0\\
7.27	1.73472347597681e-18\\
7.28	0\\
7.29	0\\
7.3	0\\
7.31	0\\
7.32	0\\
7.33	0\\
7.34	0\\
7.35	0\\
7.36	0\\
7.37	1.73472347597681e-18\\
7.38	0\\
7.39	0\\
7.4	0\\
7.41	0\\
7.42	0\\
7.43	0\\
7.44	0\\
7.45	0\\
7.46	0\\
7.47	0\\
7.48	0\\
7.49	0\\
7.5	0\\
7.51	0\\
7.52	0\\
7.53	0\\
7.54	0\\
7.55	1.73472347597681e-18\\
7.56	1.73472347597681e-18\\
7.57	0\\
7.58	1.73472347597681e-18\\
7.59	0\\
7.6	0\\
7.61	0\\
7.62	0\\
7.63	0\\
7.64	0\\
7.65	0\\
7.66	0\\
7.67	0\\
7.68	0\\
7.69	0\\
7.7	0\\
7.71	0\\
7.72	0\\
7.73	0\\
7.74	0\\
7.75	0\\
7.76	0\\
7.77	0\\
7.78	0\\
7.79	0\\
7.8	0\\
7.81	0\\
7.82	0\\
7.83	0\\
7.84	0\\
7.85	1.73472347597681e-18\\
7.86	0\\
7.87	0\\
7.88	0\\
7.89	0\\
7.9	0\\
7.91	0\\
7.92	0\\
7.93	0\\
7.94	0\\
7.95	1.73472347597681e-18\\
7.96	0\\
7.97	0\\
7.98	0\\
7.99	0\\
8	0\\
8.01	0\\
8.02	0\\
8.03	1.73472347597681e-18\\
8.04	1.73472347597681e-18\\
8.05	1.73472347597681e-18\\
8.06	0\\
8.07	1.73472347597681e-18\\
8.08	0\\
8.09	1.73472347597681e-18\\
8.1	0\\
8.11	0\\
8.12	0\\
8.13	1.73472347597681e-18\\
8.14	0\\
8.15	0\\
8.16	0\\
8.17	0\\
8.18	0\\
8.19	0\\
8.2	0\\
8.21	0\\
8.22	0\\
8.23	0\\
8.24	0\\
8.25	0\\
8.26	0\\
8.27	0\\
8.28	0\\
8.29	0\\
8.3	0\\
8.31	0\\
8.32	0\\
8.33	0\\
8.34	0\\
8.35	0\\
8.36	0\\
8.37	0\\
8.38	0\\
8.39	1.73472347597681e-18\\
8.4	0\\
8.41	0\\
8.42	0\\
8.43	0\\
8.44	0\\
8.45	1.73472347597681e-18\\
8.46	0\\
8.47	0\\
8.48	0\\
8.49	0\\
8.5	0\\
8.51	0\\
8.52	0\\
8.53	0\\
8.54	0\\
8.55	0\\
8.56	0\\
8.57	0\\
8.58	0\\
8.59	0\\
8.6	0\\
8.61	0\\
8.62	0\\
8.63	0\\
8.64	1.73472347597681e-18\\
8.65	0\\
8.66	0\\
8.67	0\\
8.68	1.73472347597681e-18\\
8.69	0\\
8.7	0\\
8.71	0\\
8.72	0\\
8.73	0\\
8.74	0\\
8.75	0\\
8.76	0\\
8.77	0\\
8.78	0\\
8.79	1.73472347597681e-18\\
8.8	0\\
8.81	0\\
8.82	0\\
8.83	0\\
8.84	1.73472347597681e-18\\
8.85	0\\
8.86	1.73472347597681e-18\\
8.87	0\\
8.88	0\\
8.89	1.73472347597681e-18\\
8.9	0\\
8.91	0\\
8.92	0\\
8.93	0\\
8.94	0\\
8.95	0\\
8.96	0\\
8.97	0\\
8.98	0\\
8.99	0\\
9	0\\
9.01	0\\
9.02	0\\
9.03	1.73472347597681e-18\\
9.04	0\\
9.05	0\\
9.06	0\\
9.07	0\\
9.08	0\\
9.09	0\\
9.1	1.73472347597681e-18\\
9.11	0\\
9.12	0\\
9.13	0\\
9.14	0\\
9.15	1.73472347597681e-18\\
9.16	0\\
9.17	1.73472347597681e-18\\
9.18	0\\
9.19	1.73472347597681e-18\\
9.2	0\\
9.21	0\\
9.22	1.73472347597681e-18\\
9.23	0\\
9.24	0\\
9.25	1.73472347597681e-18\\
9.26	1.73472347597681e-18\\
9.27	0\\
9.28	0\\
9.29	0\\
9.3	0\\
9.31	1.73472347597681e-18\\
9.32	0\\
9.33	1.73472347597681e-18\\
9.34	0\\
9.35	0\\
9.36	0\\
9.37	0\\
9.38	0\\
9.39	1.73472347597681e-18\\
9.4	0\\
9.41	1.73472347597681e-18\\
9.42	0\\
9.43	0\\
9.44	0\\
9.45	0\\
9.46	0\\
9.47	0\\
9.48	0\\
9.49	0\\
9.5	0\\
9.51	1.73472347597681e-18\\
9.52	0\\
9.53	0\\
9.54	0\\
9.55	1.73472347597681e-18\\
9.56	0\\
9.57	0\\
9.58	0\\
9.59	0\\
9.6	0\\
9.61	0\\
9.62	0\\
9.63	0\\
9.64	1.73472347597681e-18\\
9.65	0\\
9.66	0\\
9.67	0\\
9.68	1.73472347597681e-18\\
9.69	0\\
9.7	0\\
9.71	0\\
9.72	1.73472347597681e-18\\
9.73	0\\
9.74	0\\
9.75	0\\
9.76	0\\
9.77	0\\
9.78	1.73472347597681e-18\\
9.79	0\\
9.8	0\\
9.81	1.73472347597681e-18\\
9.82	0\\
9.83	0\\
9.84	0\\
9.85	0\\
9.86	0\\
9.87	1.73472347597681e-18\\
9.88	0\\
9.89	0\\
9.9	1.73472347597681e-18\\
9.91	0\\
9.92	0\\
9.93	0\\
9.94	0\\
9.95	0\\
9.96	0\\
9.97	0\\
9.98	1.73472347597681e-18\\
9.99	0\\
10	0\\
10.01	0\\
10.02	0\\
10.03	0\\
10.04	0\\
10.05	0\\
10.06	0\\
10.07	0\\
10.08	0\\
10.09	0\\
10.1	0\\
10.11	0\\
10.12	0\\
10.13	0\\
10.14	0\\
10.15	0\\
10.16	0\\
10.17	0\\
10.18	0\\
10.19	0\\
10.2	0\\
10.21	0\\
10.22	0\\
10.23	0\\
10.24	0\\
10.25	0\\
10.26	0\\
10.27	0\\
10.28	0\\
10.29	0\\
10.3	0\\
10.31	0\\
10.32	0\\
10.33	0\\
10.34	0\\
10.35	0\\
10.36	1.73472347597681e-18\\
10.37	1.73472347597681e-18\\
10.38	0\\
10.39	0\\
10.4	0\\
10.41	0\\
10.42	1.73472347597681e-18\\
10.43	1.73472347597681e-18\\
10.44	0\\
10.45	0\\
10.46	0\\
10.47	0\\
10.48	0\\
10.49	0\\
10.5	0\\
10.51	0\\
10.52	0\\
10.53	0\\
10.54	0\\
10.55	0\\
10.56	0\\
10.57	0\\
10.58	0\\
10.59	1.73472347597681e-18\\
10.6	0\\
10.61	1.73472347597681e-18\\
10.62	0\\
10.63	0\\
10.64	0\\
10.65	0\\
10.66	0\\
10.67	0\\
10.68	0\\
10.69	0\\
10.7	0\\
10.71	0\\
10.72	0\\
10.73	0\\
10.74	0\\
10.75	0\\
10.76	1.73472347597681e-18\\
10.77	1.73472347597681e-18\\
10.78	0\\
10.79	1.73472347597681e-18\\
10.8	0\\
10.81	0\\
10.82	0\\
10.83	0\\
10.84	0\\
10.85	0\\
10.86	0\\
10.87	0\\
10.88	1.73472347597681e-18\\
10.89	0\\
10.9	1.73472347597681e-18\\
10.91	0\\
10.92	0\\
10.93	0\\
10.94	0\\
10.95	0\\
10.96	0\\
10.97	0\\
10.98	0\\
10.99	0\\
11	0\\
11.01	0\\
11.02	0\\
11.03	0\\
11.04	0\\
11.05	0\\
11.06	0\\
11.07	0\\
11.08	0\\
11.09	0\\
11.1	0\\
11.11	1.73472347597681e-18\\
11.12	0\\
11.13	0\\
11.14	0\\
11.15	0\\
11.16	1.73472347597681e-18\\
11.17	0\\
11.18	0\\
11.19	1.73472347597681e-18\\
11.2	0\\
11.21	0\\
11.22	0\\
11.23	0\\
11.24	1.73472347597681e-18\\
11.25	0\\
11.26	0\\
11.27	0\\
11.28	0\\
11.29	0\\
11.3	0\\
11.31	0\\
11.32	0\\
11.33	0\\
11.34	0\\
11.35	1.73472347597681e-18\\
11.36	0\\
11.37	0\\
11.38	0\\
11.39	1.73472347597681e-18\\
11.4	0\\
11.41	1.73472347597681e-18\\
11.42	0\\
11.43	0\\
11.44	0\\
11.45	0\\
11.46	0\\
11.47	0\\
11.48	0\\
11.49	0\\
11.5	0\\
11.51	0\\
11.52	0\\
11.53	0\\
11.54	0\\
11.55	0\\
11.56	0\\
11.57	0\\
11.58	1.73472347597681e-18\\
11.59	1.73472347597681e-18\\
11.6	0\\
11.61	0\\
11.62	0\\
11.63	1.73472347597681e-18\\
11.64	0\\
11.65	0\\
11.66	0\\
11.67	0\\
11.68	0\\
11.69	1.73472347597681e-18\\
11.7	0\\
11.71	0\\
11.72	0\\
11.73	0\\
11.74	0\\
11.75	0\\
11.76	0\\
11.77	1.73472347597681e-18\\
11.78	0\\
11.79	0\\
11.8	0\\
11.81	0\\
11.82	0\\
11.83	0\\
11.84	0\\
11.85	0\\
11.86	0\\
11.87	0\\
11.88	0\\
11.89	0\\
11.9	0\\
11.91	1.73472347597681e-18\\
11.92	0\\
11.93	0\\
11.94	1.73472347597681e-18\\
11.95	0\\
11.96	0\\
11.97	0\\
11.98	0\\
11.99	0\\
12	0\\
12.01	0\\
12.02	0\\
12.03	0\\
12.04	1.73472347597681e-18\\
12.05	0\\
12.06	0\\
12.07	0\\
12.08	0\\
12.09	1.73472347597681e-18\\
12.1	0\\
12.11	1.73472347597681e-18\\
12.12	0\\
12.13	0\\
12.14	0\\
12.15	0\\
12.16	0\\
12.17	1.73472347597681e-18\\
12.18	1.73472347597681e-18\\
12.19	1.73472347597681e-18\\
12.2	0\\
12.21	0\\
12.22	0\\
12.23	0\\
12.24	0\\
12.25	1.73472347597681e-18\\
12.26	0\\
12.27	0\\
12.28	1.73472347597681e-18\\
12.29	0\\
12.3	0\\
12.31	0\\
12.32	0\\
12.33	0\\
12.34	0\\
12.35	1.73472347597681e-18\\
12.36	0\\
12.37	0\\
12.38	1.73472347597681e-18\\
12.39	0\\
12.4	0\\
12.41	0\\
12.42	0\\
12.43	0\\
12.44	0\\
12.45	0\\
12.46	0\\
12.47	1.73472347597681e-18\\
12.48	0\\
12.49	0\\
12.5	0\\
12.51	0\\
12.52	0\\
12.53	0\\
12.54	0\\
12.55	0\\
12.56	0\\
12.57	0\\
12.58	0\\
12.59	0\\
12.6	0\\
12.61	0\\
12.62	0\\
12.63	0\\
12.64	0\\
12.65	0\\
12.66	0\\
12.67	0\\
12.68	0\\
12.69	0\\
12.7	0\\
12.71	0\\
12.72	1.73472347597681e-18\\
12.73	0\\
12.74	0\\
12.75	0\\
12.76	0\\
12.77	0\\
12.78	0\\
12.79	0\\
12.8	0\\
12.81	1.73472347597681e-18\\
12.82	0\\
12.83	0\\
12.84	1.73472347597681e-18\\
12.85	0\\
12.86	1.73472347597681e-18\\
12.87	0\\
12.88	0\\
12.89	0\\
12.9	0\\
12.91	0\\
12.92	0\\
12.93	0\\
12.94	0\\
12.95	1.73472347597681e-18\\
12.96	0\\
12.97	0\\
12.98	0\\
12.99	0\\
13	0\\
13.01	0\\
13.02	0\\
13.03	0\\
13.04	1.73472347597681e-18\\
13.05	0\\
13.06	0\\
13.07	1.73472347597681e-18\\
13.08	0\\
13.09	0\\
13.1	0\\
13.11	0\\
13.12	0\\
13.13	0\\
13.14	0\\
13.15	0\\
13.16	0\\
13.17	0\\
13.18	0\\
13.19	0\\
13.2	0\\
13.21	0\\
13.22	0\\
13.23	0\\
13.24	0\\
13.25	1.73472347597681e-18\\
13.26	0\\
13.27	0\\
13.28	0\\
13.29	0\\
13.3	0\\
13.31	1.73472347597681e-18\\
13.32	0\\
13.33	1.73472347597681e-18\\
13.34	1.73472347597681e-18\\
13.35	0\\
13.36	1.73472347597681e-18\\
13.37	0\\
13.38	1.73472347597681e-18\\
13.39	0\\
13.4	0\\
13.41	0\\
13.42	0\\
13.43	0\\
13.44	0\\
13.45	0\\
13.46	0\\
13.47	1.73472347597681e-18\\
13.48	0\\
13.49	0\\
13.5	0\\
13.51	0\\
13.52	0\\
13.53	0\\
13.54	0\\
13.55	0\\
13.56	0\\
13.57	0\\
13.58	0\\
13.59	0\\
13.6	0\\
13.61	1.73472347597681e-18\\
13.62	0\\
13.63	1.73472347597681e-18\\
13.64	0\\
13.65	0\\
13.66	1.73472347597681e-18\\
13.67	1.73472347597681e-18\\
13.68	0\\
13.69	0\\
13.7	1.73472347597681e-18\\
13.71	0\\
13.72	1.73472347597681e-18\\
13.73	0\\
13.74	1.73472347597681e-18\\
13.75	0\\
13.76	0\\
13.77	0\\
13.78	0\\
13.79	1.73472347597681e-18\\
13.8	0\\
13.81	1.73472347597681e-18\\
13.82	1.73472347597681e-18\\
13.83	0\\
13.84	0\\
13.85	0\\
13.86	0\\
13.87	0\\
13.88	0\\
13.89	1.73472347597681e-18\\
13.9	0\\
13.91	0\\
13.92	1.73472347597681e-18\\
13.93	0\\
13.94	0\\
13.95	0\\
13.96	0\\
13.97	0\\
13.98	0\\
13.99	1.73472347597681e-18\\
14	1.73472347597681e-18\\
14.01	1.73472347597681e-18\\
14.02	0\\
14.03	0\\
14.04	0\\
14.05	0\\
14.06	0\\
14.07	0\\
14.08	0\\
14.09	0\\
14.1	1.73472347597681e-18\\
14.11	0\\
14.12	0\\
14.13	0\\
14.14	0\\
14.15	0\\
14.16	0\\
14.17	0\\
14.18	0\\
14.19	1.73472347597681e-18\\
14.2	0\\
14.21	0\\
14.22	0\\
14.23	0\\
14.24	0\\
14.25	0\\
14.26	0\\
14.27	0\\
14.28	0\\
14.29	0\\
14.3	0\\
14.31	0\\
14.32	0\\
14.33	1.73472347597681e-18\\
14.34	0\\
14.35	0\\
14.36	0\\
14.37	0\\
14.38	0\\
14.39	1.73472347597681e-18\\
14.4	0\\
14.41	1.73472347597681e-18\\
14.42	0\\
14.43	0\\
14.44	0\\
14.45	0\\
14.46	1.73472347597681e-18\\
14.47	1.73472347597681e-18\\
14.48	0\\
14.49	0\\
14.5	0\\
14.51	0\\
14.52	0\\
14.53	0\\
14.54	0\\
14.55	0\\
14.56	0\\
14.57	0\\
14.58	0\\
14.59	0\\
14.6	0\\
14.61	0\\
14.62	0\\
14.63	1.73472347597681e-18\\
14.64	0\\
14.65	0\\
14.66	0\\
14.67	0\\
14.68	0\\
14.69	0\\
14.7	1.73472347597681e-18\\
14.71	0\\
14.72	1.73472347597681e-18\\
14.73	1.73472347597681e-18\\
14.74	0\\
14.75	0\\
14.76	0\\
14.77	0\\
14.78	0\\
14.79	0\\
14.8	0\\
14.81	0\\
14.82	0\\
14.83	0\\
14.84	1.73472347597681e-18\\
14.85	0\\
14.86	0\\
14.87	0\\
14.88	0\\
14.89	0\\
14.9	0\\
14.91	1.73472347597681e-18\\
14.92	0\\
14.93	0\\
14.94	0\\
14.95	1.73472347597681e-18\\
14.96	1.73472347597681e-18\\
14.97	0\\
14.98	0\\
14.99	0\\
15	0\\
15.01	1.73472347597681e-18\\
15.02	0\\
15.03	0\\
15.04	0\\
15.05	1.73472347597681e-18\\
15.06	0\\
15.07	0\\
15.08	0\\
15.09	0\\
15.1	0\\
15.11	0\\
15.12	0\\
15.13	0\\
15.14	0\\
15.15	0\\
15.16	0\\
15.17	0\\
15.18	0\\
15.19	0\\
15.2	1.73472347597681e-18\\
15.21	0\\
15.22	0\\
15.23	0\\
15.24	1.73472347597681e-18\\
15.25	1.73472347597681e-18\\
15.26	0\\
15.27	0\\
15.28	0\\
15.29	0\\
15.3	0\\
15.31	0\\
15.32	0\\
15.33	1.73472347597681e-18\\
15.34	0\\
15.35	0\\
15.36	0\\
15.37	0\\
15.38	0\\
15.39	0\\
15.4	0\\
15.41	0\\
15.42	0\\
15.43	0\\
15.44	0\\
15.45	0\\
15.46	0\\
15.47	0\\
15.48	0\\
15.49	0\\
15.5	0\\
15.51	0\\
15.52	0\\
15.53	0\\
15.54	0\\
15.55	0\\
15.56	0\\
15.57	0\\
15.58	1.73472347597681e-18\\
15.59	0\\
15.6	0\\
15.61	0\\
15.62	0\\
15.63	1.73472347597681e-18\\
15.64	0\\
15.65	0\\
15.66	0\\
15.67	0\\
15.68	1.73472347597681e-18\\
15.69	0\\
15.7	0\\
15.71	0\\
15.72	0\\
15.73	0\\
15.74	0\\
15.75	1.73472347597681e-18\\
15.76	0\\
15.77	0\\
15.78	0\\
15.79	0\\
15.8	0\\
15.81	0\\
15.82	0\\
15.83	0\\
15.84	0\\
15.85	0\\
15.86	0\\
15.87	0\\
15.88	0\\
15.89	0\\
15.9	0\\
15.91	0\\
15.92	0\\
15.93	0\\
15.94	0\\
15.95	0\\
15.96	0\\
15.97	0\\
15.98	0\\
15.99	0\\
16	1.73472347597681e-18\\
16.01	0\\
16.02	0\\
16.03	0\\
16.04	0\\
16.05	0\\
16.06	0\\
16.07	0\\
16.08	1.73472347597681e-18\\
16.09	0\\
16.1	0\\
16.11	0\\
16.12	0\\
16.13	0\\
16.14	0\\
16.15	0\\
16.16	0\\
16.17	0\\
16.18	0\\
16.19	0\\
16.2	0\\
16.21	0\\
16.22	0\\
16.23	0\\
16.24	0\\
16.25	0\\
16.26	0\\
16.27	0\\
16.28	0\\
16.29	0\\
16.3	0\\
16.31	0\\
16.32	0\\
16.33	0\\
16.34	0\\
16.35	1.73472347597681e-18\\
16.36	0\\
16.37	1.73472347597681e-18\\
16.38	0\\
16.39	0\\
16.4	0\\
16.41	0\\
16.42	0\\
16.43	0\\
16.44	0\\
16.45	0\\
16.46	0\\
16.47	0\\
16.48	0\\
16.49	0\\
16.5	0\\
16.51	0\\
16.52	0\\
16.53	0\\
16.54	0\\
16.55	0\\
16.56	0\\
16.57	0\\
16.58	0\\
16.59	0\\
16.6	0\\
16.61	0\\
16.62	0\\
16.63	0\\
16.64	1.73472347597681e-18\\
16.65	1.73472347597681e-18\\
16.66	0\\
16.67	0\\
16.68	1.73472347597681e-18\\
16.69	0\\
16.7	0\\
16.71	0\\
16.72	0\\
16.73	0\\
16.74	0\\
16.75	0\\
16.76	1.73472347597681e-18\\
16.77	0\\
16.78	1.73472347597681e-18\\
16.79	0\\
16.8	0\\
16.81	0\\
16.82	0\\
16.83	1.73472347597681e-18\\
16.84	1.73472347597681e-18\\
16.85	0\\
16.86	1.73472347597681e-18\\
16.87	0\\
16.88	0\\
16.89	1.73472347597681e-18\\
16.9	0\\
16.91	0\\
16.92	0\\
16.93	1.73472347597681e-18\\
16.94	0\\
16.95	1.73472347597681e-18\\
16.96	0\\
16.97	0\\
16.98	0\\
16.99	1.73472347597681e-18\\
17	1.73472347597681e-18\\
17.01	1.73472347597681e-18\\
17.02	0\\
17.03	0\\
17.04	0\\
17.05	0\\
17.06	0\\
17.07	0\\
17.08	0\\
17.09	1.73472347597681e-18\\
17.1	0\\
17.11	0\\
17.12	0\\
17.13	1.73472347597681e-18\\
17.14	0\\
17.15	0\\
17.16	0\\
17.17	1.73472347597681e-18\\
17.18	0\\
17.19	0\\
17.2	1.73472347597681e-18\\
17.21	0\\
17.22	1.73472347597681e-18\\
17.23	0\\
17.24	0\\
17.25	0\\
17.26	0\\
17.27	0\\
17.28	0\\
17.29	0\\
17.3	0\\
17.31	0\\
17.32	0\\
17.33	0\\
17.34	0\\
17.35	0\\
17.36	0\\
17.37	0\\
17.38	0\\
17.39	0\\
17.4	0\\
17.41	1.73472347597681e-18\\
17.42	1.73472347597681e-18\\
17.43	0\\
17.44	0\\
17.45	0\\
17.46	1.73472347597681e-18\\
17.47	0\\
17.48	0\\
17.49	0\\
17.5	0\\
17.51	0\\
17.52	0\\
17.53	0\\
17.54	0\\
17.55	0\\
17.56	0\\
17.57	0\\
17.58	0\\
17.59	0\\
17.6	0\\
17.61	0\\
17.62	0\\
17.63	0\\
17.64	0\\
17.65	0\\
17.66	0\\
17.67	0\\
17.68	0\\
17.69	0\\
17.7	0\\
17.71	0\\
17.72	0\\
17.73	0\\
17.74	0\\
17.75	0\\
17.76	0\\
17.77	0\\
17.78	0\\
17.79	1.73472347597681e-18\\
17.8	0\\
17.81	1.73472347597681e-18\\
17.82	0\\
17.83	1.73472347597681e-18\\
17.84	0\\
17.85	0\\
17.86	0\\
17.87	1.73472347597681e-18\\
17.88	0\\
17.89	0\\
17.9	0\\
17.91	0\\
17.92	1.73472347597681e-18\\
17.93	0\\
17.94	0\\
17.95	0\\
17.96	0\\
17.97	1.73472347597681e-18\\
17.98	0\\
17.99	0\\
18	0\\
18.01	0\\
18.02	0\\
18.03	0\\
18.04	0\\
18.05	0\\
18.06	0\\
18.07	0\\
18.08	0\\
18.09	0\\
18.1	0\\
18.11	1.73472347597681e-18\\
18.12	0\\
18.13	0\\
18.14	0\\
18.15	0\\
18.16	0\\
18.17	0\\
18.18	0\\
18.19	0\\
18.2	1.73472347597681e-18\\
18.21	0\\
18.22	0\\
18.23	0\\
18.24	0\\
18.25	0\\
18.26	0\\
18.27	1.73472347597681e-18\\
18.28	0\\
18.29	0\\
18.3	0\\
18.31	0\\
18.32	0\\
18.33	0\\
18.34	0\\
18.35	1.73472347597681e-18\\
18.36	1.73472347597681e-18\\
18.37	0\\
18.38	0\\
18.39	0\\
18.4	0\\
18.41	0\\
18.42	0\\
18.43	1.73472347597681e-18\\
18.44	0\\
18.45	0\\
18.46	0\\
18.47	0\\
18.48	0\\
18.49	0\\
18.5	0\\
18.51	0\\
18.52	0\\
18.53	0\\
18.54	0\\
18.55	0\\
18.56	0\\
18.57	0\\
18.58	0\\
18.59	0\\
18.6	0\\
18.61	0\\
18.62	0\\
18.63	0\\
18.64	0\\
18.65	0\\
18.66	0\\
18.67	0\\
18.68	0\\
18.69	0\\
18.7	0\\
18.71	0\\
18.72	0\\
18.73	1.73472347597681e-18\\
18.74	0\\
18.75	0\\
18.76	0\\
18.77	0\\
18.78	1.73472347597681e-18\\
18.79	0\\
18.8	0\\
18.81	0\\
18.82	0\\
18.83	0\\
18.84	0\\
18.85	0\\
18.86	0\\
18.87	0\\
18.88	0\\
18.89	0\\
18.9	0\\
18.91	0\\
18.92	0\\
18.93	0\\
18.94	0\\
18.95	0\\
18.96	0\\
18.97	0\\
18.98	0\\
18.99	0\\
19	0\\
19.01	1.73472347597681e-18\\
19.02	0\\
19.03	0\\
19.04	1.73472347597681e-18\\
19.05	0\\
19.06	0\\
19.07	1.73472347597681e-18\\
19.08	0\\
19.09	0\\
19.1	0\\
19.11	0\\
19.12	1.73472347597681e-18\\
19.13	0\\
19.14	0\\
19.15	0\\
19.16	0\\
19.17	0\\
19.18	0\\
19.19	1.73472347597681e-18\\
19.2	0\\
19.21	0\\
19.22	0\\
19.23	0\\
19.24	0\\
19.25	0\\
19.26	0\\
19.27	0\\
19.28	0\\
19.29	1.73472347597681e-18\\
19.3	0\\
19.31	0\\
19.32	0\\
19.33	0\\
19.34	0\\
19.35	0\\
19.36	0\\
19.37	0\\
19.38	0\\
19.39	0\\
19.4	0\\
19.41	0\\
19.42	0\\
19.43	0\\
19.44	0\\
19.45	0\\
19.46	0\\
19.47	0\\
19.48	0\\
19.49	1.73472347597681e-18\\
19.5	0\\
19.51	0\\
19.52	0\\
19.53	0\\
19.54	0\\
19.55	0\\
19.56	0\\
19.57	0\\
19.58	0\\
19.59	0\\
19.6	0\\
19.61	0\\
19.62	1.73472347597681e-18\\
19.63	1.73472347597681e-18\\
19.64	0\\
19.65	0\\
19.66	1.73472347597681e-18\\
19.67	0\\
19.68	0\\
19.69	0\\
19.7	0\\
19.71	0\\
19.72	0\\
19.73	0\\
19.74	0\\
19.75	0\\
19.76	0\\
19.77	0\\
19.78	0\\
19.79	0\\
19.8	0\\
19.81	1.73472347597681e-18\\
19.82	1.73472347597681e-18\\
19.83	0\\
19.84	0\\
19.85	0\\
19.86	1.73472347597681e-18\\
19.87	0\\
19.88	0\\
19.89	0\\
19.9	0\\
19.91	0\\
19.92	0\\
19.93	0\\
19.94	1.73472347597681e-18\\
19.95	0\\
19.96	0\\
19.97	0\\
19.98	0\\
19.99	0\\
20	0\\
20.01	0\\
20.02	1.73472347597681e-18\\
20.03	0\\
20.04	0\\
20.05	0\\
20.06	0\\
20.07	0\\
20.08	0\\
20.09	1.73472347597681e-18\\
20.1	0\\
20.11	0\\
20.12	0\\
20.13	1.73472347597681e-18\\
20.14	0\\
20.15	0\\
20.16	0\\
20.17	0\\
20.18	0\\
20.19	0\\
20.2	0\\
20.21	0\\
20.22	1.73472347597681e-18\\
20.23	0\\
20.24	0\\
20.25	0\\
20.26	0\\
20.27	0\\
20.28	0\\
20.29	0\\
20.3	0\\
20.31	0\\
20.32	0\\
20.33	0\\
20.34	0\\
20.35	0\\
20.36	0\\
20.37	0\\
20.38	1.73472347597681e-18\\
20.39	0\\
20.4	0\\
20.41	0\\
20.42	1.73472347597681e-18\\
20.43	0\\
20.44	1.73472347597681e-18\\
20.45	1.73472347597681e-18\\
20.46	0\\
20.47	0\\
20.48	0\\
20.49	0\\
20.5	0\\
20.51	0\\
20.52	0\\
20.53	0\\
20.54	0\\
20.55	0\\
20.56	0\\
20.57	0\\
20.58	0\\
20.59	0\\
20.6	1.73472347597681e-18\\
20.61	0\\
20.62	0\\
20.63	0\\
20.64	1.73472347597681e-18\\
20.65	1.73472347597681e-18\\
20.66	1.73472347597681e-18\\
20.67	1.73472347597681e-18\\
20.68	0\\
20.69	0\\
20.7	0\\
20.71	0\\
20.72	0\\
20.73	0\\
20.74	1.73472347597681e-18\\
20.75	0\\
20.76	0\\
20.77	1.73472347597681e-18\\
20.78	0\\
20.79	0\\
20.8	0\\
20.81	0\\
20.82	0\\
20.83	0\\
20.84	1.73472347597681e-18\\
20.85	0\\
20.86	0\\
20.87	0\\
20.88	0\\
20.89	0\\
20.9	0\\
20.91	0\\
20.92	0\\
20.93	0\\
20.94	0\\
20.95	0\\
20.96	0\\
20.97	0\\
20.98	0\\
20.99	0\\
21	1.73472347597681e-18\\
21.01	1.73472347597681e-18\\
21.02	0\\
21.03	0\\
21.04	1.73472347597681e-18\\
21.05	0\\
21.06	0\\
21.07	0\\
21.08	0\\
21.09	0\\
21.1	0\\
21.11	0\\
21.12	0\\
21.13	0\\
21.14	0\\
21.15	0\\
21.16	1.73472347597681e-18\\
21.17	0\\
21.18	0\\
21.19	0\\
21.2	0\\
21.21	1.73472347597681e-18\\
21.22	0\\
21.23	0\\
21.24	0\\
21.25	1.73472347597681e-18\\
21.26	1.73472347597681e-18\\
21.27	1.73472347597681e-18\\
21.28	0\\
21.29	0\\
21.3	0\\
21.31	0\\
21.32	1.73472347597681e-18\\
21.33	0\\
21.34	0\\
21.35	0\\
21.36	0\\
21.37	0\\
21.38	0\\
21.39	1.73472347597681e-18\\
21.4	0\\
21.41	0\\
21.42	0\\
21.43	0\\
21.44	0\\
21.45	0\\
21.46	1.73472347597681e-18\\
21.47	0\\
21.48	0\\
21.49	0\\
21.5	0\\
21.51	0\\
21.52	0\\
21.53	0\\
21.54	0\\
21.55	0\\
21.56	0\\
21.57	1.73472347597681e-18\\
21.58	0\\
21.59	0\\
21.6	0\\
21.61	0\\
21.62	0\\
21.63	0\\
21.64	0\\
21.65	0\\
21.66	0\\
21.67	1.73472347597681e-18\\
21.68	0\\
21.69	0\\
21.7	1.73472347597681e-18\\
21.71	0\\
21.72	0\\
21.73	1.73472347597681e-18\\
21.74	0\\
21.75	0\\
21.76	0\\
21.77	0\\
21.78	1.73472347597681e-18\\
21.79	0\\
21.8	0\\
21.81	0\\
21.82	0\\
21.83	0\\
21.84	0\\
21.85	0\\
21.86	0\\
21.87	0\\
21.88	1.73472347597681e-18\\
21.89	0\\
21.9	0\\
21.91	0\\
21.92	1.73472347597681e-18\\
21.93	0\\
21.94	0\\
21.95	0\\
21.96	0\\
21.97	0\\
21.98	0\\
21.99	0\\
22	0\\
22.01	0\\
22.02	0\\
22.03	0\\
22.04	0\\
22.05	0\\
22.06	0\\
22.07	0\\
22.08	0\\
22.09	0\\
22.1	1.73472347597681e-18\\
22.11	0\\
22.12	1.73472347597681e-18\\
22.13	0\\
22.14	1.73472347597681e-18\\
22.15	1.73472347597681e-18\\
22.16	0\\
22.17	0\\
22.18	0\\
22.19	0\\
22.2	0\\
22.21	0\\
22.22	0\\
22.23	0\\
22.24	0\\
22.25	1.73472347597681e-18\\
22.26	1.73472347597681e-18\\
22.27	0\\
22.28	1.73472347597681e-18\\
22.29	0\\
22.3	0\\
22.31	0\\
22.32	0\\
22.33	0\\
22.34	0\\
22.35	0\\
22.36	0\\
22.37	0\\
22.38	0\\
22.39	0\\
22.4	0\\
22.41	0\\
22.42	1.73472347597681e-18\\
22.43	0\\
22.44	0\\
22.45	0\\
22.46	1.73472347597681e-18\\
22.47	1.73472347597681e-18\\
22.48	1.73472347597681e-18\\
22.49	0\\
22.5	0\\
22.51	0\\
22.52	0\\
22.53	0\\
22.54	0\\
22.55	0\\
22.56	0\\
22.57	0\\
22.58	0\\
22.59	0\\
22.6	0\\
22.61	0\\
22.62	0\\
22.63	0\\
22.64	0\\
22.65	1.73472347597681e-18\\
22.66	0\\
22.67	0\\
22.68	1.73472347597681e-18\\
22.69	1.73472347597681e-18\\
22.7	0\\
22.71	0\\
22.72	1.73472347597681e-18\\
22.73	0\\
22.74	0\\
22.75	0\\
22.76	0\\
22.77	0\\
22.78	0\\
22.79	0\\
22.8	0\\
22.81	0\\
22.82	0\\
22.83	0\\
22.84	0\\
22.85	0\\
22.86	0\\
22.87	0\\
22.88	0\\
22.89	0\\
22.9	0\\
22.91	0\\
22.92	0\\
22.93	0\\
22.94	0\\
22.95	0\\
22.96	0\\
22.97	0\\
22.98	0\\
22.99	0\\
23	0\\
23.01	0\\
23.02	0\\
23.03	0\\
23.04	0\\
23.05	0\\
23.06	0\\
23.07	1.73472347597681e-18\\
23.08	0\\
23.09	0\\
23.1	0\\
23.11	0\\
23.12	0\\
23.13	1.73472347597681e-18\\
23.14	0\\
23.15	0\\
23.16	0\\
23.17	0\\
23.18	0\\
23.19	0\\
23.2	0\\
23.21	1.73472347597681e-18\\
23.22	0\\
23.23	0\\
23.24	0\\
23.25	0\\
23.26	0\\
23.27	0\\
23.28	0\\
23.29	0\\
23.3	0\\
23.31	0\\
23.32	0\\
23.33	0\\
23.34	0\\
23.35	0\\
23.36	0\\
23.37	0\\
23.38	1.73472347597681e-18\\
23.39	0\\
23.4	0\\
23.41	0\\
23.42	0\\
23.43	0\\
23.44	0\\
23.45	0\\
23.46	0\\
23.47	0\\
23.48	0\\
23.49	0\\
23.5	0\\
23.51	0\\
23.52	0\\
23.53	0\\
23.54	0\\
23.55	1.73472347597681e-18\\
23.56	0\\
23.57	0\\
23.58	0\\
23.59	0\\
23.6	0\\
23.61	0\\
23.62	0\\
23.63	0\\
23.64	0\\
23.65	0\\
23.66	0\\
23.67	0\\
23.68	0\\
23.69	0\\
23.7	0\\
23.71	0\\
23.72	0\\
23.73	0\\
23.74	0\\
23.75	0\\
23.76	0\\
23.77	0\\
23.78	0\\
23.79	0\\
23.8	1.73472347597681e-18\\
23.81	0\\
23.82	0\\
23.83	0\\
23.84	0\\
23.85	0\\
23.86	1.73472347597681e-18\\
23.87	0\\
23.88	0\\
23.89	0\\
23.9	0\\
23.91	0\\
23.92	0\\
23.93	0\\
23.94	0\\
23.95	0\\
23.96	0\\
23.97	0\\
23.98	0\\
23.99	0\\
24	0\\
24.01	1.73472347597681e-18\\
24.02	0\\
24.03	0\\
24.04	0\\
24.05	0\\
24.06	0\\
24.07	0\\
24.08	0\\
24.09	0\\
24.1	0\\
24.11	0\\
24.12	0\\
24.13	0\\
24.14	0\\
24.15	0\\
24.16	0\\
24.17	1.73472347597681e-18\\
24.18	1.73472347597681e-18\\
24.19	0\\
24.2	0\\
24.21	0\\
24.22	1.73472347597681e-18\\
24.23	0\\
24.24	0\\
24.25	0\\
24.26	0\\
24.27	0\\
24.28	0\\
24.29	0\\
24.3	0\\
24.31	0\\
24.32	0\\
24.33	1.73472347597681e-18\\
24.34	0\\
24.35	0\\
24.36	0\\
24.37	0\\
24.38	0\\
24.39	0\\
24.4	1.73472347597681e-18\\
24.41	1.73472347597681e-18\\
24.42	0\\
24.43	0\\
24.44	0\\
24.45	0\\
24.46	0\\
24.47	1.73472347597681e-18\\
24.48	1.73472347597681e-18\\
24.49	0\\
24.5	0\\
24.51	0\\
24.52	0\\
24.53	0\\
24.54	0\\
24.55	0\\
24.56	0\\
24.57	0\\
24.58	0\\
24.59	0\\
24.6	0\\
24.61	0\\
24.62	0\\
24.63	0\\
24.64	0\\
24.65	1.73472347597681e-18\\
24.66	0\\
24.67	1.73472347597681e-18\\
24.68	0\\
24.69	0\\
24.7	0\\
24.71	0\\
24.72	0\\
24.73	0\\
24.74	1.73472347597681e-18\\
24.75	0\\
24.76	0\\
24.77	0\\
24.78	0\\
24.79	0\\
24.8	0\\
24.81	0\\
24.82	0\\
24.83	0\\
24.84	0\\
24.85	0\\
24.86	0\\
24.87	0\\
24.88	0\\
24.89	0\\
24.9	0\\
24.91	0\\
24.92	1.73472347597681e-18\\
24.93	0\\
24.94	0\\
24.95	0\\
24.96	0\\
24.97	0\\
24.98	0\\
24.99	0\\
25	0\\
25.01	1.73472347597681e-18\\
25.02	1.73472347597681e-18\\
25.03	0\\
25.04	1.73472347597681e-18\\
25.05	0\\
25.06	0\\
25.07	0\\
25.08	0\\
25.09	0\\
25.1	1.73472347597681e-18\\
25.11	1.73472347597681e-18\\
25.12	0\\
25.13	0\\
25.14	0\\
25.15	0\\
25.16	1.73472347597681e-18\\
25.17	0\\
25.18	1.73472347597681e-18\\
25.19	0\\
25.2	0\\
25.21	0\\
25.22	0\\
25.23	0\\
25.24	1.73472347597681e-18\\
25.25	1.73472347597681e-18\\
25.26	0\\
25.27	0\\
25.28	0\\
25.29	0\\
25.3	0\\
25.31	0\\
25.32	0\\
25.33	0\\
25.34	0\\
25.35	0\\
25.36	0\\
25.37	1.73472347597681e-18\\
25.38	0\\
25.39	0\\
25.4	0\\
25.41	1.73472347597681e-18\\
25.42	0\\
25.43	0\\
25.44	0\\
25.45	0\\
25.46	0\\
25.47	0\\
25.48	0\\
25.49	0\\
25.5	0\\
25.51	0\\
25.52	0\\
25.53	0\\
25.54	0\\
25.55	0\\
25.56	0\\
25.57	0\\
25.58	0\\
25.59	0\\
25.6	0\\
25.61	0\\
25.62	0\\
25.63	0\\
25.64	0\\
25.65	0\\
25.66	0\\
25.67	0\\
25.68	0\\
25.69	0\\
25.7	0\\
25.71	0\\
25.72	1.73472347597681e-18\\
25.73	0\\
25.74	1.73472347597681e-18\\
25.75	0\\
25.76	1.73472347597681e-18\\
25.77	0\\
25.78	1.73472347597681e-18\\
25.79	0\\
25.8	0\\
25.81	0\\
25.82	0\\
25.83	0\\
25.84	0\\
25.85	0\\
25.86	0\\
25.87	1.73472347597681e-18\\
25.88	0\\
25.89	0\\
25.9	0\\
25.91	0\\
25.92	0\\
25.93	0\\
25.94	0\\
25.95	0\\
25.96	1.73472347597681e-18\\
25.97	0\\
25.98	0\\
25.99	0\\
26	0\\
26.01	0\\
26.02	0\\
26.03	0\\
26.04	0\\
26.05	0\\
26.06	1.73472347597681e-18\\
26.07	0\\
26.08	0\\
26.09	1.73472347597681e-18\\
26.1	0\\
26.11	0\\
26.12	0\\
26.13	0\\
26.14	0\\
26.15	0\\
26.16	0\\
26.17	0\\
26.18	0\\
26.19	0\\
26.2	0\\
26.21	0\\
26.22	0\\
26.23	0\\
26.24	0\\
26.25	0\\
26.26	0\\
26.27	0\\
26.28	0\\
26.29	0\\
26.3	0\\
26.31	0\\
26.32	0\\
26.33	0\\
26.34	0\\
26.35	0\\
26.36	0\\
26.37	0\\
26.38	0\\
26.39	1.73472347597681e-18\\
26.4	0\\
26.41	0\\
26.42	0\\
26.43	0\\
26.44	0\\
26.45	0\\
26.46	0\\
26.47	0\\
26.48	0\\
26.49	0\\
26.5	0\\
26.51	0\\
26.52	0\\
26.53	0\\
26.54	0\\
26.55	0\\
26.56	0\\
26.57	0\\
26.58	0\\
26.59	0\\
26.6	0\\
26.61	0\\
26.62	0\\
26.63	0\\
26.64	0\\
26.65	0\\
26.66	0\\
26.67	0\\
26.68	0\\
26.69	0\\
26.7	0\\
26.71	1.73472347597681e-18\\
26.72	0\\
26.73	0\\
26.74	0\\
26.75	0\\
26.76	0\\
26.77	0\\
26.78	0\\
26.79	0\\
26.8	1.73472347597681e-18\\
26.81	0\\
26.82	0\\
26.83	0\\
26.84	1.73472347597681e-18\\
26.85	0\\
26.86	1.73472347597681e-18\\
26.87	0\\
26.88	0\\
26.89	0\\
26.9	0\\
26.91	1.73472347597681e-18\\
26.92	0\\
26.93	0\\
26.94	0\\
26.95	0\\
26.96	0\\
26.97	0\\
26.98	1.73472347597681e-18\\
26.99	0\\
27	0\\
27.01	0\\
27.02	0\\
27.03	0\\
27.04	0\\
27.05	0\\
27.06	0\\
27.07	0\\
27.08	0\\
27.09	0\\
27.1	0\\
27.11	1.73472347597681e-18\\
27.12	0\\
27.13	0\\
27.14	0\\
27.15	0\\
27.16	0\\
27.17	0\\
27.18	0\\
27.19	0\\
27.2	0\\
27.21	0\\
27.22	0\\
27.23	0\\
27.24	0\\
27.25	0\\
27.26	0\\
27.27	0\\
27.28	0\\
27.29	0\\
27.3	0\\
27.31	1.73472347597681e-18\\
27.32	0\\
27.33	0\\
27.34	0\\
27.35	0\\
27.36	0\\
27.37	0\\
27.38	0\\
27.39	0\\
27.4	0\\
27.41	0\\
27.42	0\\
27.43	1.73472347597681e-18\\
27.44	0\\
27.45	0\\
27.46	0\\
27.47	0\\
27.48	0\\
27.49	1.73472347597681e-18\\
27.5	1.73472347597681e-18\\
27.51	0\\
27.52	0\\
27.53	0\\
27.54	0\\
27.55	0\\
27.56	0\\
27.57	0\\
27.58	0\\
27.59	0\\
27.6	0\\
27.61	0\\
27.62	0\\
27.63	1.73472347597681e-18\\
27.64	0\\
27.65	0\\
27.66	0\\
27.67	0\\
27.68	0\\
27.69	0\\
27.7	0\\
27.71	0\\
27.72	0\\
27.73	0\\
27.74	0\\
27.75	1.73472347597681e-18\\
27.76	0\\
27.77	0\\
27.78	0\\
27.79	0\\
27.8	0\\
27.81	0\\
27.82	0\\
27.83	0\\
27.84	0\\
27.85	1.73472347597681e-18\\
27.86	0\\
27.87	1.73472347597681e-18\\
27.88	0\\
27.89	0\\
27.9	0\\
27.91	0\\
27.92	0\\
27.93	0\\
27.94	0\\
27.95	0\\
27.96	0\\
27.97	0\\
27.98	0\\
27.99	0\\
28	1.73472347597681e-18\\
28.01	0\\
28.02	0\\
28.03	1.73472347597681e-18\\
28.04	1.73472347597681e-18\\
28.05	0\\
28.06	0\\
28.07	0\\
28.08	0\\
28.09	0\\
28.1	0\\
28.11	1.73472347597681e-18\\
28.12	0\\
28.13	0\\
28.14	0\\
28.15	0\\
28.16	0\\
28.17	0\\
28.18	0\\
28.19	0\\
28.2	0\\
28.21	0\\
28.22	0\\
28.23	0\\
28.24	0\\
28.25	0\\
28.26	0\\
28.27	0\\
28.28	0\\
28.29	0\\
28.3	0\\
28.31	0\\
28.32	0\\
28.33	0\\
28.34	0\\
28.35	1.73472347597681e-18\\
28.36	0\\
28.37	0\\
28.38	0\\
28.39	0\\
28.4	0\\
28.41	0\\
28.42	0\\
28.43	0\\
28.44	0\\
28.45	0\\
28.46	0\\
28.47	0\\
28.48	0\\
28.49	0\\
28.5	0\\
28.51	0\\
28.52	0\\
28.53	0\\
28.54	1.73472347597681e-18\\
28.55	0\\
28.56	0\\
28.57	0\\
28.58	1.73472347597681e-18\\
28.59	1.73472347597681e-18\\
28.6	0\\
28.61	0\\
28.62	0\\
28.63	0\\
28.64	0\\
28.65	0\\
28.66	0\\
28.67	0\\
28.68	0\\
28.69	1.73472347597681e-18\\
28.7	0\\
28.71	0\\
28.72	0\\
28.73	0\\
28.74	0\\
28.75	0\\
28.76	0\\
28.77	0\\
28.78	0\\
28.79	0\\
28.8	0\\
28.81	1.73472347597681e-18\\
28.82	0\\
28.83	0\\
28.84	0\\
28.85	0\\
28.86	0\\
28.87	0\\
28.88	0\\
28.89	0\\
28.9	0\\
28.91	1.73472347597681e-18\\
28.92	1.73472347597681e-18\\
28.93	0\\
28.94	0\\
28.95	0\\
28.96	0\\
28.97	0\\
28.98	0\\
28.99	0\\
29	0\\
29.01	0\\
29.02	0\\
29.03	0\\
29.04	0\\
29.05	0\\
29.06	0\\
29.07	0\\
29.08	0\\
29.09	0\\
29.1	0\\
29.11	0\\
29.12	0\\
29.13	0\\
29.14	0\\
29.15	0\\
29.16	0\\
29.17	0\\
29.18	1.73472347597681e-18\\
29.19	0\\
29.2	0\\
29.21	0\\
29.22	0\\
29.23	0\\
29.24	0\\
29.25	0\\
29.26	0\\
29.27	0\\
29.28	0\\
29.29	0\\
29.3	0\\
29.31	1.73472347597681e-18\\
29.32	0\\
29.33	0\\
29.34	0\\
29.35	0\\
29.36	0\\
29.37	0\\
29.38	0\\
29.39	0\\
29.4	0\\
29.41	0\\
29.42	0\\
29.43	0\\
29.44	0\\
29.45	0\\
29.46	0\\
29.47	0\\
29.48	0\\
29.49	0\\
29.5	0\\
29.51	0\\
29.52	0\\
29.53	1.73472347597681e-18\\
29.54	0\\
29.55	0\\
29.56	0\\
29.57	0\\
29.58	0\\
29.59	0\\
29.6	0\\
29.61	0\\
29.62	0\\
29.63	1.73472347597681e-18\\
29.64	0\\
29.65	1.73472347597681e-18\\
29.66	1.73472347597681e-18\\
29.67	0\\
29.68	1.73472347597681e-18\\
29.69	0\\
29.7	0\\
29.71	0\\
29.72	1.73472347597681e-18\\
29.73	0\\
29.74	0\\
29.75	0\\
29.76	0\\
29.77	0\\
29.78	0\\
29.79	0\\
29.8	1.73472347597681e-18\\
29.81	0\\
29.82	0\\
29.83	0\\
29.84	0\\
29.85	0\\
29.86	0\\
29.87	1.73472347597681e-18\\
29.88	0\\
29.89	0\\
29.9	0\\
29.91	0\\
29.92	0\\
29.93	1.73472347597681e-18\\
29.94	0\\
29.95	0\\
29.96	0\\
29.97	0\\
29.98	0\\
29.99	0\\
30	1.73472347597681e-18\\
30.01	1.73472347597681e-18\\
30.02	0\\
30.03	0\\
30.04	0\\
30.05	0\\
30.06	0\\
30.07	0\\
30.08	0\\
30.09	0\\
30.1	0\\
30.11	1.73472347597681e-18\\
30.12	0\\
30.13	0\\
30.14	0\\
30.15	0\\
30.16	0\\
30.17	0\\
30.18	0\\
30.19	0\\
30.2	0\\
30.21	0\\
30.22	0\\
30.23	0\\
30.24	0\\
30.25	0\\
30.26	0\\
30.27	0\\
30.28	0\\
30.29	0\\
30.3	1.73472347597681e-18\\
30.31	0\\
30.32	0\\
30.33	0\\
30.34	0\\
30.35	0\\
30.36	0\\
30.37	0\\
30.38	1.73472347597681e-18\\
30.39	0\\
30.4	1.73472347597681e-18\\
30.41	0\\
30.42	0\\
30.43	0\\
30.44	0\\
30.45	0\\
30.46	0\\
30.47	0\\
30.48	0\\
30.49	0\\
30.5	0\\
30.51	0\\
30.52	0\\
30.53	0\\
30.54	0\\
30.55	0\\
30.56	0\\
30.57	0\\
30.58	0\\
30.59	0\\
30.6	0\\
30.61	0\\
30.62	0\\
30.63	0\\
30.64	0\\
30.65	0\\
30.66	0\\
30.67	0\\
30.68	1.73472347597681e-18\\
30.69	0\\
30.7	0\\
30.71	0\\
30.72	1.73472347597681e-18\\
30.73	0\\
30.74	0\\
30.75	0\\
30.76	0\\
30.77	0\\
30.78	0\\
30.79	0\\
30.8	1.73472347597681e-18\\
30.81	1.73472347597681e-18\\
30.82	0\\
30.83	1.73472347597681e-18\\
30.84	0\\
30.85	1.73472347597681e-18\\
30.86	0\\
30.87	0\\
30.88	0\\
30.89	0\\
30.9	0\\
30.91	1.73472347597681e-18\\
30.92	0\\
30.93	0\\
30.94	1.73472347597681e-18\\
30.95	0\\
30.96	0\\
30.97	1.73472347597681e-18\\
30.98	0\\
30.99	1.73472347597681e-18\\
31	0\\
31.01	0\\
31.02	0\\
31.03	0\\
31.04	0\\
31.05	0\\
31.06	0\\
31.07	0\\
31.08	1.73472347597681e-18\\
31.09	0\\
31.1	0\\
31.11	0\\
31.12	0\\
31.13	0\\
31.14	0\\
31.15	0\\
31.16	0\\
31.17	0\\
31.18	0\\
31.19	0\\
31.2	0\\
31.21	1.73472347597681e-18\\
31.22	1.73472347597681e-18\\
31.23	0\\
31.24	0\\
31.25	0\\
31.26	1.73472347597681e-18\\
31.27	0\\
31.28	0\\
31.29	0\\
31.3	0\\
31.31	0\\
31.32	0\\
31.33	0\\
31.34	0\\
31.35	0\\
31.36	0\\
31.37	0\\
31.38	0\\
31.39	1.73472347597681e-18\\
31.4	0\\
31.41	0\\
31.42	1.73472347597681e-18\\
31.43	1.73472347597681e-18\\
31.44	1.73472347597681e-18\\
31.45	0\\
31.46	0\\
31.47	0\\
31.48	0\\
31.49	0\\
31.5	0\\
31.51	0\\
31.52	0\\
31.53	0\\
31.54	0\\
31.55	1.73472347597681e-18\\
31.56	0\\
31.57	0\\
31.58	0\\
31.59	0\\
31.6	0\\
31.61	0\\
31.62	0\\
31.63	0\\
31.64	0\\
31.65	0\\
31.66	0\\
31.67	1.73472347597681e-18\\
31.68	0\\
31.69	0\\
31.7	0\\
31.71	0\\
31.72	0\\
31.73	0\\
31.74	0\\
31.75	0\\
31.76	0\\
31.77	0\\
31.78	0\\
31.79	0\\
31.8	0\\
31.81	0\\
31.82	0\\
31.83	1.73472347597681e-18\\
31.84	0\\
31.85	0\\
31.86	1.73472347597681e-18\\
31.87	0\\
31.88	1.73472347597681e-18\\
31.89	1.73472347597681e-18\\
31.9	0\\
31.91	0\\
31.92	0\\
31.93	0\\
31.94	1.73472347597681e-18\\
31.95	1.73472347597681e-18\\
31.96	0\\
31.97	1.73472347597681e-18\\
31.98	0\\
31.99	0\\
32	0\\
32.01	0\\
32.02	0\\
32.03	0\\
32.04	0\\
32.05	0\\
32.06	0\\
32.07	0\\
32.08	0\\
32.09	0\\
32.1	0\\
32.11	0\\
32.12	0\\
32.13	0\\
32.14	0\\
32.15	0\\
32.16	0\\
32.17	0\\
32.18	0\\
32.19	0\\
32.2	0\\
32.21	0\\
32.22	0\\
32.23	0\\
32.24	0\\
32.25	0\\
32.26	0\\
32.27	0\\
32.28	0\\
32.29	0\\
32.3	0\\
32.31	0\\
32.32	0\\
32.33	0\\
32.34	0\\
32.35	0\\
32.36	0\\
32.37	0\\
32.38	0\\
32.39	0\\
32.4	0\\
32.41	0\\
32.42	0\\
32.43	0\\
32.44	1.73472347597681e-18\\
32.45	1.73472347597681e-18\\
32.46	0\\
32.47	0\\
32.48	0\\
32.49	0\\
32.5	0\\
32.51	0\\
32.52	0\\
32.53	0\\
32.54	0\\
32.55	0\\
32.56	0\\
32.57	0\\
32.58	0\\
32.59	0\\
32.6	0\\
32.61	1.73472347597681e-18\\
32.62	0\\
32.63	0\\
32.64	0\\
32.65	0\\
32.66	0\\
32.67	0\\
32.68	0\\
32.69	0\\
32.7	0\\
32.71	0\\
32.72	0\\
32.73	0\\
32.74	0\\
32.75	0\\
32.76	0\\
32.77	0\\
32.78	0\\
32.79	0\\
32.8	0\\
32.81	0\\
32.82	0\\
32.83	1.73472347597681e-18\\
32.84	1.73472347597681e-18\\
32.85	0\\
32.86	0\\
32.87	0\\
32.88	0\\
32.89	1.73472347597681e-18\\
32.9	0\\
32.91	0\\
32.92	0\\
32.93	0\\
32.94	0\\
32.95	0\\
32.96	0\\
32.97	1.73472347597681e-18\\
32.98	0\\
32.99	0\\
33	0\\
33.01	0\\
33.02	0\\
33.03	0\\
33.04	0\\
33.05	0\\
33.06	0\\
33.07	0\\
33.08	0\\
33.09	0\\
33.1	0\\
33.11	1.73472347597681e-18\\
33.12	0\\
33.13	0\\
33.14	0\\
33.15	0\\
33.16	1.73472347597681e-18\\
33.17	0\\
33.18	0\\
33.19	0\\
33.2	0\\
33.21	0\\
33.22	0\\
33.23	0\\
33.24	0\\
33.25	1.73472347597681e-18\\
33.26	0\\
33.27	1.73472347597681e-18\\
33.28	1.73472347597681e-18\\
33.29	0\\
33.3	1.73472347597681e-18\\
33.31	0\\
33.32	0\\
33.33	0\\
33.34	0\\
33.35	0\\
33.36	1.73472347597681e-18\\
33.37	0\\
33.38	1.73472347597681e-18\\
33.39	0\\
33.4	0\\
33.41	0\\
33.42	0\\
33.43	0\\
33.44	0\\
33.45	0\\
33.46	0\\
33.47	0\\
33.48	0\\
33.49	0\\
33.5	0\\
33.51	0\\
33.52	0\\
33.53	0\\
33.54	0\\
33.55	0\\
33.56	0\\
33.57	0\\
33.58	1.73472347597681e-18\\
33.59	0\\
33.6	1.73472347597681e-18\\
33.61	0\\
33.62	0\\
33.63	0\\
33.64	0\\
33.65	0\\
33.66	0\\
33.67	0\\
33.68	0\\
33.69	0\\
33.7	0\\
33.71	0\\
33.72	0\\
33.73	0\\
33.74	0\\
33.75	0\\
33.76	1.73472347597681e-18\\
33.77	0\\
33.78	0\\
33.79	0\\
33.8	0\\
33.81	0\\
33.82	0\\
33.83	0\\
33.84	0\\
33.85	0\\
33.86	0\\
33.87	0\\
33.88	1.73472347597681e-18\\
33.89	0\\
33.9	0\\
33.91	0\\
33.92	0\\
33.93	0\\
33.94	1.73472347597681e-18\\
33.95	0\\
33.96	0\\
33.97	0\\
33.98	0\\
33.99	1.73472347597681e-18\\
34	0\\
34.01	0\\
34.02	0\\
34.03	0\\
34.04	0\\
34.05	0\\
34.06	0\\
34.07	0\\
34.08	0\\
34.09	0\\
34.1	0\\
34.11	0\\
34.12	0\\
34.13	0\\
34.14	0\\
34.15	0\\
34.16	1.73472347597681e-18\\
34.17	0\\
34.18	0\\
34.19	0\\
34.2	0\\
34.21	0\\
34.22	0\\
34.23	0\\
34.24	0\\
34.25	0\\
34.26	1.73472347597681e-18\\
34.27	0\\
34.28	0\\
34.29	0\\
34.3	0\\
34.31	0\\
34.32	0\\
34.33	0\\
34.34	1.73472347597681e-18\\
34.35	0\\
34.36	0\\
34.37	0\\
34.38	0\\
34.39	0\\
34.4	0\\
34.41	0\\
34.42	1.73472347597681e-18\\
34.43	0\\
34.44	0\\
34.45	0\\
34.46	0\\
34.47	0\\
34.48	0\\
34.49	1.73472347597681e-18\\
34.5	0\\
34.51	0\\
34.52	0\\
34.53	0\\
34.54	0\\
34.55	0\\
34.56	0\\
34.57	0\\
34.58	1.73472347597681e-18\\
34.59	0\\
34.6	0\\
34.61	0\\
34.62	0\\
34.63	0\\
34.64	0\\
34.65	0\\
34.66	0\\
34.67	0\\
34.68	0\\
34.69	0\\
34.7	1.73472347597681e-18\\
34.71	0\\
34.72	0\\
34.73	0\\
34.74	1.73472347597681e-18\\
34.75	1.73472347597681e-18\\
34.76	0\\
34.77	1.73472347597681e-18\\
34.78	1.73472347597681e-18\\
34.79	0\\
34.8	0\\
34.81	0\\
34.82	0\\
34.83	0\\
34.84	0\\
34.85	0\\
34.86	0\\
34.87	0\\
34.88	0\\
34.89	0\\
34.9	0\\
34.91	0\\
34.92	0\\
34.93	0\\
34.94	0\\
34.95	1.73472347597681e-18\\
34.96	1.73472347597681e-18\\
34.97	1.73472347597681e-18\\
34.98	0\\
34.99	0\\
35	0\\
35.01	0\\
35.02	0\\
35.03	0\\
35.04	0\\
35.05	0\\
35.06	0\\
35.07	1.73472347597681e-18\\
35.08	0\\
35.09	1.73472347597681e-18\\
35.1	0\\
35.11	0\\
35.12	0\\
35.13	1.73472347597681e-18\\
35.14	0\\
35.15	0\\
35.16	1.73472347597681e-18\\
35.17	0\\
35.18	0\\
35.19	0\\
35.2	0\\
35.21	0\\
35.22	0\\
35.23	0\\
35.24	0\\
35.25	0\\
35.26	0\\
35.27	0\\
35.28	0\\
35.29	0\\
35.3	0\\
35.31	0\\
35.32	0\\
35.33	0\\
35.34	0\\
35.35	0\\
35.36	0\\
35.37	0\\
35.38	0\\
35.39	0\\
35.4	0\\
35.41	0\\
35.42	0\\
35.43	1.73472347597681e-18\\
35.44	0\\
35.45	0\\
35.46	0\\
35.47	0\\
35.48	0\\
35.49	0\\
35.5	0\\
35.51	0\\
35.52	0\\
35.53	0\\
35.54	0\\
35.55	0\\
35.56	0\\
35.57	0\\
35.58	0\\
35.59	0\\
35.6	0\\
35.61	0\\
35.62	0\\
35.63	0\\
35.64	0\\
35.65	0\\
35.66	0\\
35.67	0\\
35.68	1.73472347597681e-18\\
35.69	1.73472347597681e-18\\
35.7	0\\
35.71	0\\
35.72	0\\
35.73	0\\
35.74	0\\
35.75	0\\
35.76	0\\
35.77	0\\
35.78	0\\
35.79	0\\
35.8	0\\
35.81	0\\
35.82	0\\
35.83	0\\
35.84	0\\
35.85	1.73472347597681e-18\\
35.86	0\\
35.87	0\\
35.88	0\\
35.89	0\\
35.9	0\\
35.91	1.73472347597681e-18\\
35.92	0\\
35.93	0\\
35.94	0\\
35.95	0\\
35.96	0\\
35.97	0\\
35.98	1.73472347597681e-18\\
35.99	0\\
36	0\\
36.01	0\\
36.02	0\\
36.03	0\\
36.04	0\\
36.05	0\\
36.06	0\\
36.07	1.73472347597681e-18\\
36.08	1.73472347597681e-18\\
36.09	0\\
36.1	0\\
36.11	0\\
36.12	0\\
36.13	0\\
36.14	0\\
36.15	0\\
36.16	1.73472347597681e-18\\
36.17	0\\
36.18	0\\
36.19	0\\
36.2	0\\
36.21	0\\
36.22	0\\
36.23	0\\
36.24	0\\
36.25	1.73472347597681e-18\\
36.26	0\\
36.27	0\\
36.28	0\\
36.29	0\\
36.3	0\\
36.31	0\\
36.32	1.73472347597681e-18\\
36.33	0\\
36.34	0\\
36.35	0\\
36.36	0\\
36.37	0\\
36.38	1.73472347597681e-18\\
36.39	0\\
36.4	0\\
36.41	0\\
36.42	0\\
36.43	0\\
36.44	1.73472347597681e-18\\
36.45	0\\
36.46	0\\
36.47	0\\
36.48	0\\
36.49	0\\
36.5	1.73472347597681e-18\\
36.51	0\\
36.52	0\\
36.53	0\\
36.54	0\\
36.55	0\\
36.56	0\\
36.57	0\\
36.58	1.73472347597681e-18\\
36.59	0\\
36.6	1.73472347597681e-18\\
36.61	0\\
36.62	0\\
36.63	1.73472347597681e-18\\
36.64	0\\
36.65	0\\
36.66	0\\
36.67	0\\
36.68	0\\
36.69	0\\
36.7	0\\
36.71	0\\
36.72	0\\
36.73	0\\
36.74	0\\
36.75	0\\
36.76	0\\
36.77	0\\
36.78	0\\
36.79	0\\
36.8	0\\
36.81	1.73472347597681e-18\\
36.82	0\\
36.83	0\\
36.84	0\\
36.85	0\\
36.86	0\\
36.87	0\\
36.88	0\\
36.89	0\\
36.9	0\\
36.91	0\\
36.92	0\\
36.93	0\\
36.94	0\\
36.95	0\\
36.96	0\\
36.97	0\\
36.98	1.73472347597681e-18\\
36.99	0\\
37	0\\
37.01	0\\
37.02	1.73472347597681e-18\\
37.03	0\\
37.04	0\\
37.05	0\\
37.06	0\\
37.07	0\\
37.08	0\\
37.09	0\\
37.1	0\\
37.11	0\\
37.12	0\\
37.13	0\\
37.14	0\\
37.15	0\\
37.16	0\\
37.17	0\\
37.18	0\\
37.19	0\\
37.2	0\\
37.21	0\\
37.22	0\\
37.23	0\\
37.24	0\\
37.25	0\\
37.26	0\\
37.27	0\\
37.28	0\\
37.29	0\\
37.3	1.73472347597681e-18\\
37.31	0\\
37.32	0\\
37.33	0\\
37.34	0\\
37.35	0\\
37.36	0\\
37.37	0\\
37.38	0\\
37.39	0\\
37.4	0\\
37.41	0\\
37.42	0\\
37.43	0\\
37.44	0\\
37.45	0\\
37.46	0\\
37.47	0\\
37.48	0\\
37.49	1.73472347597681e-18\\
37.5	0\\
37.51	0\\
37.52	0\\
37.53	0\\
37.54	0\\
37.55	0\\
37.56	0\\
37.57	0\\
37.58	0\\
37.59	0\\
37.6	0\\
37.61	0\\
37.62	0\\
37.63	0\\
37.64	0\\
37.65	0\\
37.66	0\\
37.67	0\\
37.68	0\\
37.69	0\\
37.7	0\\
37.71	0\\
37.72	0\\
37.73	0\\
37.74	0\\
37.75	0\\
37.76	1.73472347597681e-18\\
37.77	0\\
37.78	0\\
37.79	0\\
37.8	0\\
37.81	0\\
37.82	0\\
37.83	0\\
37.84	0\\
37.85	0\\
37.86	0\\
37.87	0\\
37.88	0\\
37.89	0\\
37.9	0\\
37.91	0\\
37.92	0\\
37.93	0\\
37.94	0\\
37.95	0\\
37.96	0\\
37.97	0\\
37.98	0\\
37.99	0\\
38	1.73472347597681e-18\\
38.01	0\\
38.02	0\\
38.03	0\\
38.04	0\\
38.05	0\\
38.06	0\\
38.07	0\\
38.08	0\\
38.09	0\\
38.1	0\\
38.11	1.73472347597681e-18\\
38.12	1.73472347597681e-18\\
38.13	0\\
38.14	0\\
38.15	0\\
38.16	0\\
38.17	0\\
38.18	0\\
38.19	0\\
38.2	0\\
38.21	0\\
38.22	0\\
38.23	0\\
38.24	0\\
38.25	0\\
38.26	0\\
38.27	1.73472347597681e-18\\
38.28	0\\
38.29	0\\
38.3	0\\
38.31	1.73472347597681e-18\\
38.32	0\\
38.33	0\\
38.34	0\\
38.35	1.73472347597681e-18\\
38.36	0\\
38.37	0\\
38.38	0\\
38.39	0\\
38.4	0\\
38.41	0\\
38.42	0\\
38.43	0\\
38.44	0\\
38.45	0\\
38.46	0\\
38.47	0\\
38.48	1.73472347597681e-18\\
38.49	0\\
38.5	0\\
38.51	0\\
38.52	0\\
38.53	0\\
38.54	0\\
38.55	0\\
38.56	0\\
38.57	0\\
38.58	0\\
38.59	0\\
38.6	0\\
38.61	1.73472347597681e-18\\
38.62	0\\
38.63	1.73472347597681e-18\\
38.64	0\\
38.65	1.73472347597681e-18\\
38.66	0\\
38.67	0\\
38.68	0\\
38.69	0\\
38.7	0\\
38.71	0\\
38.72	0\\
38.73	0\\
38.74	1.73472347597681e-18\\
38.75	0\\
38.76	0\\
38.77	0\\
38.78	0\\
38.79	0\\
38.8	0\\
38.81	0\\
38.82	1.73472347597681e-18\\
38.83	0\\
38.84	0\\
38.85	0\\
38.86	0\\
38.87	1.73472347597681e-18\\
38.88	0\\
38.89	1.73472347597681e-18\\
38.9	0\\
38.91	0\\
38.92	0\\
38.93	0\\
38.94	0\\
38.95	0\\
38.96	0\\
38.97	0\\
38.98	0\\
38.99	0\\
39	0\\
39.01	0\\
39.02	0\\
39.03	0\\
39.04	0\\
39.05	0\\
39.06	0\\
39.07	0\\
39.08	0\\
39.09	1.73472347597681e-18\\
39.1	0\\
39.11	0\\
39.12	0\\
39.13	0\\
39.14	0\\
39.15	0\\
39.16	0\\
39.17	0\\
39.18	0\\
39.19	1.73472347597681e-18\\
39.2	0\\
39.21	0\\
39.22	0\\
39.23	0\\
39.24	0\\
39.25	0\\
39.26	0\\
39.27	0\\
39.28	0\\
39.29	0\\
39.3	0\\
39.31	0\\
39.32	0\\
39.33	0\\
39.34	1.73472347597681e-18\\
39.35	0\\
39.36	0\\
39.37	0\\
39.38	0\\
39.39	0\\
39.4	0\\
39.41	0\\
39.42	0\\
39.43	0\\
39.44	0\\
39.45	0\\
39.46	0\\
39.47	0\\
39.48	0\\
39.49	0\\
39.5	0\\
39.51	0\\
39.52	1.73472347597681e-18\\
39.53	0\\
39.54	0\\
39.55	0\\
39.56	0\\
39.57	0\\
39.58	0\\
39.59	0\\
39.6	0\\
39.61	0\\
39.62	0\\
39.63	0\\
39.64	0\\
39.65	0\\
39.66	0\\
39.67	1.73472347597681e-18\\
39.68	0\\
39.69	0\\
39.7	0\\
39.71	0\\
39.72	0\\
39.73	0\\
39.74	0\\
39.75	1.73472347597681e-18\\
39.76	0\\
39.77	0\\
39.78	0\\
39.79	0\\
39.8	0\\
39.81	0\\
39.82	1.73472347597681e-18\\
39.83	0\\
39.84	0\\
39.85	0\\
39.86	0\\
39.87	0\\
39.88	0\\
39.89	0\\
39.9	0\\
39.91	0\\
39.92	1.73472347597681e-18\\
39.93	0\\
39.94	1.73472347597681e-18\\
39.95	0\\
39.96	0\\
39.97	0\\
39.98	0\\
39.99	0\\
40	0\\
40.01	1.73472347597681e-18\\
};
\addplot [color=green,dashed,forget plot]
  table[row sep=crcr]{%
40.01	1.73472347597681e-18\\
40.02	0\\
40.03	1.73472347597681e-18\\
40.04	0\\
40.05	1.73472347597681e-18\\
40.06	0\\
40.07	1.73472347597681e-18\\
40.08	0\\
40.09	1.73472347597681e-18\\
40.1	1.73472347597681e-18\\
40.11	0\\
40.12	0\\
40.13	0\\
40.14	0\\
40.15	1.73472347597681e-18\\
40.16	0\\
40.17	0\\
40.18	0\\
40.19	0\\
40.2	0\\
40.21	0\\
40.22	1.73472347597681e-18\\
40.23	0\\
40.24	0\\
40.25	1.73472347597681e-18\\
40.26	0\\
40.27	0\\
40.28	0\\
40.29	0\\
40.3	0\\
40.31	1.73472347597681e-18\\
40.32	1.73472347597681e-18\\
40.33	0\\
40.34	0\\
40.35	1.73472347597681e-18\\
40.36	0\\
40.37	0\\
40.38	0\\
40.39	0\\
40.4	1.73472347597681e-18\\
40.41	0\\
40.42	0\\
40.43	1.73472347597681e-18\\
40.44	0\\
40.45	0\\
40.46	0\\
40.47	0\\
40.48	0\\
40.49	0\\
40.5	1.73472347597681e-18\\
40.51	0\\
40.52	0\\
40.53	0\\
40.54	0\\
40.55	0\\
40.56	0\\
40.57	0\\
40.58	1.73472347597681e-18\\
40.59	0\\
40.6	0\\
40.61	1.73472347597681e-18\\
40.62	0\\
40.63	0\\
40.64	0\\
40.65	0\\
40.66	0\\
40.67	0\\
40.68	0\\
40.69	0\\
40.7	0\\
40.71	0\\
40.72	0\\
40.73	0\\
40.74	0\\
40.75	0\\
40.76	0\\
40.77	0\\
40.78	0\\
40.79	0\\
40.8	0\\
40.81	0\\
40.82	0\\
40.83	0\\
40.84	0\\
40.85	0\\
40.86	0\\
40.87	1.73472347597681e-18\\
40.88	0\\
40.89	0\\
40.9	1.73472347597681e-18\\
40.91	0\\
40.92	0\\
40.93	0\\
40.94	0\\
40.95	1.73472347597681e-18\\
40.96	0\\
40.97	1.73472347597681e-18\\
40.98	0\\
40.99	0\\
41	0\\
41.01	0\\
41.02	0\\
41.03	0\\
41.04	0\\
41.05	0\\
41.06	0\\
41.07	0\\
41.08	0\\
41.09	0\\
41.1	0\\
41.11	0\\
41.12	0\\
41.13	0\\
41.14	0\\
41.15	1.73472347597681e-18\\
41.16	0\\
41.17	0\\
41.18	1.73472347597681e-18\\
41.19	0\\
41.2	0\\
41.21	0\\
41.22	0\\
41.23	0\\
41.24	0\\
41.25	0\\
41.26	0\\
41.27	1.73472347597681e-18\\
41.28	0\\
41.29	1.73472347597681e-18\\
41.3	0\\
41.31	0\\
41.32	0\\
41.33	0\\
41.34	0\\
41.35	0\\
41.36	0\\
41.37	0\\
41.38	1.73472347597681e-18\\
41.39	0\\
41.4	0\\
41.41	1.73472347597681e-18\\
41.42	0\\
41.43	0\\
41.44	0\\
41.45	0\\
41.46	0\\
41.47	0\\
41.48	0\\
41.49	0\\
41.5	0\\
41.51	0\\
41.52	0\\
41.53	0\\
41.54	0\\
41.55	0\\
41.56	0\\
41.57	0\\
41.58	0\\
41.59	0\\
41.6	0\\
41.61	1.73472347597681e-18\\
41.62	0\\
41.63	0\\
41.64	0\\
41.65	0\\
41.66	0\\
41.67	0\\
41.68	1.73472347597681e-18\\
41.69	0\\
41.7	0\\
41.71	0\\
41.72	0\\
41.73	0\\
41.74	0\\
41.75	0\\
41.76	0\\
41.77	0\\
41.78	0\\
41.79	0\\
41.8	0\\
41.81	1.73472347597681e-18\\
41.82	0\\
41.83	0\\
41.84	0\\
41.85	1.73472347597681e-18\\
41.86	0\\
41.87	0\\
41.88	0\\
41.89	0\\
41.9	0\\
41.91	0\\
41.92	1.73472347597681e-18\\
41.93	0\\
41.94	0\\
41.95	0\\
41.96	0\\
41.97	0\\
41.98	0\\
41.99	0\\
42	0\\
42.01	0\\
42.02	0\\
42.03	0\\
42.04	0\\
42.05	0\\
42.06	0\\
42.07	0\\
42.08	0\\
42.09	0\\
42.1	0\\
42.11	0\\
42.12	0\\
42.13	0\\
42.14	1.73472347597681e-18\\
42.15	0\\
42.16	1.73472347597681e-18\\
42.17	0\\
42.18	1.73472347597681e-18\\
42.19	0\\
42.2	0\\
42.21	0\\
42.22	0\\
42.23	0\\
42.24	1.73472347597681e-18\\
42.25	0\\
42.26	0\\
42.27	0\\
42.28	0\\
42.29	0\\
42.3	1.73472347597681e-18\\
42.31	0\\
42.32	0\\
42.33	0\\
42.34	0\\
42.35	0\\
42.36	1.73472347597681e-18\\
42.37	0\\
42.38	0\\
42.39	0\\
42.4	0\\
42.41	0\\
42.42	0\\
42.43	0\\
42.44	0\\
42.45	0\\
42.46	0\\
42.47	0\\
42.48	0\\
42.49	0\\
42.5	0\\
42.51	0\\
42.52	0\\
42.53	0\\
42.54	0\\
42.55	0\\
42.56	0\\
42.57	0\\
42.58	0\\
42.59	0\\
42.6	0\\
42.61	0\\
42.62	0\\
42.63	0\\
42.64	0\\
42.65	0\\
42.66	0\\
42.67	0\\
42.68	0\\
42.69	0\\
42.7	0\\
42.71	1.73472347597681e-18\\
42.72	0\\
42.73	0\\
42.74	1.73472347597681e-18\\
42.75	0\\
42.76	0\\
42.77	1.73472347597681e-18\\
42.78	0\\
42.79	0\\
42.8	0\\
42.81	0\\
42.82	0\\
42.83	0\\
42.84	0\\
42.85	0\\
42.86	0\\
42.87	0\\
42.88	0\\
42.89	1.73472347597681e-18\\
42.9	0\\
42.91	0\\
42.92	0\\
42.93	0\\
42.94	0\\
42.95	0\\
42.96	0\\
42.97	0\\
42.98	0\\
42.99	0\\
43	1.73472347597681e-18\\
43.01	0\\
43.02	0\\
43.03	0\\
43.04	0\\
43.05	0\\
43.06	0\\
43.07	0\\
43.08	0\\
43.09	1.73472347597681e-18\\
43.1	0\\
43.11	0\\
43.12	0\\
43.13	1.73472347597681e-18\\
43.14	0\\
43.15	0\\
43.16	0\\
43.17	0\\
43.18	0\\
43.19	0\\
43.2	0\\
43.21	0\\
43.22	0\\
43.23	0\\
43.24	0\\
43.25	0\\
43.26	0\\
43.27	0\\
43.28	0\\
43.29	0\\
43.3	0\\
43.31	0\\
43.32	0\\
43.33	1.73472347597681e-18\\
43.34	0\\
43.35	0\\
43.36	0\\
43.37	0\\
43.38	1.73472347597681e-18\\
43.39	0\\
43.4	0\\
43.41	0\\
43.42	0\\
43.43	0\\
43.44	0\\
43.45	0\\
43.46	0\\
43.47	0\\
43.48	0\\
43.49	0\\
43.5	0\\
43.51	0\\
43.52	1.73472347597681e-18\\
43.53	0\\
43.54	0\\
43.55	1.73472347597681e-18\\
43.56	0\\
43.57	0\\
43.58	0\\
43.59	0\\
43.6	0\\
43.61	0\\
43.62	0\\
43.63	0\\
43.64	0\\
43.65	0\\
43.66	0\\
43.67	0\\
43.68	0\\
43.69	0\\
43.7	0\\
43.71	0\\
43.72	1.73472347597681e-18\\
43.73	0\\
43.74	1.73472347597681e-18\\
43.75	0\\
43.76	0\\
43.77	0\\
43.78	0\\
43.79	0\\
43.8	0\\
43.81	1.73472347597681e-18\\
43.82	0\\
43.83	1.73472347597681e-18\\
43.84	1.73472347597681e-18\\
43.85	0\\
43.86	0\\
43.87	0\\
43.88	0\\
43.89	0\\
43.9	0\\
43.91	0\\
43.92	0\\
43.93	1.73472347597681e-18\\
43.94	0\\
43.95	0\\
43.96	1.73472347597681e-18\\
43.97	0\\
43.98	0\\
43.99	0\\
44	0\\
44.01	0\\
44.02	0\\
44.03	1.73472347597681e-18\\
44.04	0\\
44.05	0\\
44.06	0\\
44.07	0\\
44.08	0\\
44.09	0\\
44.1	0\\
44.11	0\\
44.12	0\\
44.13	0\\
44.14	0\\
44.15	0\\
44.16	0\\
44.17	0\\
44.18	0\\
44.19	0\\
44.2	0\\
44.21	0\\
44.22	0\\
44.23	0\\
44.24	0\\
44.25	0\\
44.26	1.73472347597681e-18\\
44.27	0\\
44.28	0\\
44.29	0\\
44.3	0\\
44.31	0\\
44.32	0\\
44.33	1.73472347597681e-18\\
44.34	0\\
44.35	1.73472347597681e-18\\
44.36	1.73472347597681e-18\\
44.37	0\\
44.38	0\\
44.39	0\\
44.4	0\\
44.41	0\\
44.42	0\\
44.43	0\\
44.44	0\\
44.45	0\\
44.46	1.73472347597681e-18\\
44.47	0\\
44.48	0\\
44.49	0\\
44.5	0\\
44.51	0\\
44.52	0\\
44.53	0\\
44.54	0\\
44.55	0\\
44.56	0\\
44.57	0\\
44.58	0\\
44.59	0\\
44.6	0\\
44.61	0\\
44.62	0\\
44.63	0\\
44.64	1.73472347597681e-18\\
44.65	0\\
44.66	0\\
44.67	0\\
44.68	0\\
44.69	1.73472347597681e-18\\
44.7	0\\
44.71	0\\
44.72	0\\
44.73	0\\
44.74	0\\
44.75	0\\
44.76	0\\
44.77	0\\
44.78	0\\
44.79	0\\
44.8	0\\
44.81	0\\
44.82	0\\
44.83	0\\
44.84	1.73472347597681e-18\\
44.85	1.73472347597681e-18\\
44.86	0\\
44.87	0\\
44.88	0\\
44.89	0\\
44.9	0\\
44.91	1.73472347597681e-18\\
44.92	0\\
44.93	0\\
44.94	0\\
44.95	0\\
44.96	0\\
44.97	0\\
44.98	0\\
44.99	0\\
45	0\\
45.01	0\\
45.02	0\\
45.03	0\\
45.04	0\\
45.05	0\\
45.06	0\\
45.07	0\\
45.08	0\\
45.09	0\\
45.1	0\\
45.11	0\\
45.12	0\\
45.13	1.73472347597681e-18\\
45.14	0\\
45.15	0\\
45.16	0\\
45.17	0\\
45.18	0\\
45.19	0\\
45.2	0\\
45.21	0\\
45.22	0\\
45.23	0\\
45.24	0\\
45.25	0\\
45.26	0\\
45.27	0\\
45.28	0\\
45.29	0\\
45.3	0\\
45.31	0\\
45.32	0\\
45.33	0\\
45.34	0\\
45.35	0\\
45.36	0\\
45.37	0\\
45.38	0\\
45.39	0\\
45.4	1.73472347597681e-18\\
45.41	0\\
45.42	0\\
45.43	0\\
45.44	0\\
45.45	0\\
45.46	0\\
45.47	0\\
45.48	0\\
45.49	0\\
45.5	0\\
45.51	0\\
45.52	0\\
45.53	0\\
45.54	0\\
45.55	0\\
45.56	0\\
45.57	0\\
45.58	0\\
45.59	0\\
45.6	1.73472347597681e-18\\
45.61	0\\
45.62	0\\
45.63	1.73472347597681e-18\\
45.64	0\\
45.65	0\\
45.66	0\\
45.67	0\\
45.68	0\\
45.69	0\\
45.7	0\\
45.71	1.73472347597681e-18\\
45.72	0\\
45.73	0\\
45.74	0\\
45.75	0\\
45.76	0\\
45.77	0\\
45.78	0\\
45.79	0\\
45.8	0\\
45.81	0\\
45.82	0\\
45.83	1.73472347597681e-18\\
45.84	0\\
45.85	0\\
45.86	0\\
45.87	0\\
45.88	0\\
45.89	0\\
45.9	0\\
45.91	0\\
45.92	0\\
45.93	1.73472347597681e-18\\
45.94	0\\
45.95	0\\
45.96	0\\
45.97	0\\
45.98	0\\
45.99	0\\
46	0\\
46.01	0\\
46.02	0\\
46.03	0\\
46.04	0\\
46.05	0\\
46.06	0\\
46.07	0\\
46.08	1.73472347597681e-18\\
46.09	0\\
46.1	0\\
46.11	0\\
46.12	0\\
46.13	0\\
46.14	0\\
46.15	0\\
46.16	0\\
46.17	0\\
46.18	0\\
46.19	0\\
46.2	0\\
46.21	0\\
46.22	0\\
46.23	0\\
46.24	0\\
46.25	1.73472347597681e-18\\
46.26	0\\
46.27	0\\
46.28	0\\
46.29	1.73472347597681e-18\\
46.3	1.73472347597681e-18\\
46.31	0\\
46.32	1.73472347597681e-18\\
46.33	0\\
46.34	0\\
46.35	0\\
46.36	0\\
46.37	0\\
46.38	1.73472347597681e-18\\
46.39	0\\
46.4	0\\
46.41	0\\
46.42	0\\
46.43	0\\
46.44	0\\
46.45	0\\
46.46	0\\
46.47	0\\
46.48	0\\
46.49	0\\
46.5	0\\
46.51	0\\
46.52	0\\
46.53	0\\
46.54	0\\
46.55	0\\
46.56	0\\
46.57	0\\
46.58	0\\
46.59	1.73472347597681e-18\\
46.6	1.73472347597681e-18\\
46.61	0\\
46.62	0\\
46.63	0\\
46.64	0\\
46.65	0\\
46.66	0\\
46.67	0\\
46.68	1.73472347597681e-18\\
46.69	0\\
46.7	0\\
46.71	0\\
46.72	0\\
46.73	0\\
46.74	0\\
46.75	0\\
46.76	0\\
46.77	0\\
46.78	0\\
46.79	0\\
46.8	0\\
46.81	0\\
46.82	0\\
46.83	0\\
46.84	0\\
46.85	0\\
46.86	0\\
46.87	1.73472347597681e-18\\
46.88	0\\
46.89	1.73472347597681e-18\\
46.9	0\\
46.91	0\\
46.92	0\\
46.93	0\\
46.94	0\\
46.95	0\\
46.96	0\\
46.97	0\\
46.98	0\\
46.99	0\\
47	0\\
47.01	0\\
47.02	0\\
47.03	0\\
47.04	0\\
47.05	1.73472347597681e-18\\
47.06	0\\
47.07	0\\
47.08	0\\
47.09	0\\
47.1	1.73472347597681e-18\\
47.11	0\\
47.12	0\\
47.13	0\\
47.14	1.73472347597681e-18\\
47.15	0\\
47.16	1.73472347597681e-18\\
47.17	0\\
47.18	0\\
47.19	0\\
47.2	0\\
47.21	0\\
47.22	0\\
47.23	0\\
47.24	0\\
47.25	0\\
47.26	0\\
47.27	0\\
47.28	0\\
47.29	0\\
47.3	0\\
47.31	0\\
47.32	0\\
47.33	0\\
47.34	0\\
47.35	0\\
47.36	1.73472347597681e-18\\
47.37	0\\
47.38	1.73472347597681e-18\\
47.39	0\\
47.4	0\\
47.41	0\\
47.42	0\\
47.43	0\\
47.44	0\\
47.45	0\\
47.46	1.73472347597681e-18\\
47.47	0\\
47.48	0\\
47.49	0\\
47.5	0\\
47.51	1.73472347597681e-18\\
47.52	1.73472347597681e-18\\
47.53	0\\
47.54	0\\
47.55	0\\
47.56	0\\
47.57	1.73472347597681e-18\\
47.58	1.73472347597681e-18\\
47.59	0\\
47.6	0\\
47.61	0\\
47.62	0\\
47.63	0\\
47.64	0\\
47.65	0\\
47.66	0\\
47.67	0\\
47.68	0\\
47.69	0\\
47.7	0\\
47.71	0\\
47.72	0\\
47.73	0\\
47.74	0\\
47.75	0\\
47.76	1.73472347597681e-18\\
47.77	0\\
47.78	0\\
47.79	0\\
47.8	0\\
47.81	0\\
47.82	0\\
47.83	0\\
47.84	1.73472347597681e-18\\
47.85	0\\
47.86	0\\
47.87	0\\
47.88	0\\
47.89	0\\
47.9	0\\
47.91	0\\
47.92	0\\
47.93	0\\
47.94	0\\
47.95	0\\
47.96	0\\
47.97	1.73472347597681e-18\\
47.98	0\\
47.99	0\\
48	1.73472347597681e-18\\
48.01	0\\
48.02	0\\
48.03	0\\
48.04	0\\
48.05	0\\
48.06	0\\
48.07	1.73472347597681e-18\\
48.08	1.73472347597681e-18\\
48.09	0\\
48.1	0\\
48.11	1.73472347597681e-18\\
48.12	0\\
48.13	0\\
48.14	0\\
48.15	0\\
48.16	0\\
48.17	0\\
48.18	0\\
48.19	0\\
48.2	0\\
48.21	0\\
48.22	0\\
48.23	0\\
48.24	0\\
48.25	0\\
48.26	0\\
48.27	0\\
48.28	0\\
48.29	0\\
48.3	0\\
48.31	0\\
48.32	0\\
48.33	0\\
48.34	0\\
48.35	0\\
48.36	0\\
48.37	0\\
48.38	0\\
48.39	0\\
48.4	0\\
48.41	0\\
48.42	0\\
48.43	0\\
48.44	0\\
48.45	0\\
48.46	0\\
48.47	0\\
48.48	0\\
48.49	0\\
48.5	1.73472347597681e-18\\
48.51	0\\
48.52	0\\
48.53	0\\
48.54	0\\
48.55	0\\
48.56	0\\
48.57	0\\
48.58	0\\
48.59	0\\
48.6	0\\
48.61	0\\
48.62	0\\
48.63	0\\
48.64	0\\
48.65	0\\
48.66	0\\
48.67	0\\
48.68	0\\
48.69	0\\
48.7	0\\
48.71	0\\
48.72	0\\
48.73	1.73472347597681e-18\\
48.74	0\\
48.75	0\\
48.76	0\\
48.77	0\\
48.78	0\\
48.79	0\\
48.8	1.73472347597681e-18\\
48.81	0\\
48.82	0\\
48.83	0\\
48.84	0\\
48.85	0\\
48.86	0\\
48.87	1.73472347597681e-18\\
48.88	1.73472347597681e-18\\
48.89	0\\
48.9	0\\
48.91	0\\
48.92	0\\
48.93	0\\
48.94	0\\
48.95	1.73472347597681e-18\\
48.96	1.73472347597681e-18\\
48.97	0\\
48.98	0\\
48.99	0\\
49	0\\
49.01	0\\
49.02	0\\
49.03	0\\
49.04	1.73472347597681e-18\\
49.05	1.73472347597681e-18\\
49.06	0\\
49.07	0\\
49.08	0\\
49.09	0\\
49.1	0\\
49.11	0\\
49.12	0\\
49.13	1.73472347597681e-18\\
49.14	0\\
49.15	0\\
49.16	0\\
49.17	0\\
49.18	0\\
49.19	0\\
49.2	0\\
49.21	0\\
49.22	0\\
49.23	0\\
49.24	0\\
49.25	0\\
49.26	0\\
49.27	0\\
49.28	0\\
49.29	0\\
49.3	0\\
49.31	0\\
49.32	1.73472347597681e-18\\
49.33	0\\
49.34	0\\
49.35	0\\
49.36	0\\
49.37	0\\
49.38	0\\
49.39	0\\
49.4	0\\
49.41	0\\
49.42	0\\
49.43	0\\
49.44	0\\
49.45	0\\
49.46	0\\
49.47	0\\
49.48	0\\
49.49	0\\
49.5	0\\
49.51	0\\
49.52	0\\
49.53	0\\
49.54	0\\
49.55	0\\
49.56	0\\
49.57	0\\
49.58	0\\
49.59	0\\
49.6	0\\
49.61	0\\
49.62	0\\
49.63	0\\
49.64	0\\
49.65	0\\
49.66	0\\
49.67	1.73472347597681e-18\\
49.68	1.73472347597681e-18\\
49.69	0\\
49.7	0\\
49.71	0\\
49.72	0\\
49.73	0\\
49.74	0\\
49.75	0\\
49.76	0\\
49.77	0\\
49.78	0\\
49.79	0\\
49.8	0\\
49.81	0\\
49.82	0\\
49.83	0\\
49.84	0\\
49.85	0\\
49.86	0\\
49.87	0\\
49.88	0\\
49.89	0\\
49.9	0\\
49.91	0\\
49.92	0\\
49.93	0\\
49.94	1.73472347597681e-18\\
49.95	0\\
49.96	0\\
49.97	0\\
49.98	0\\
49.99	0\\
50	0\\
50.01	0\\
50.02	0\\
50.03	0\\
50.04	0\\
50.05	0\\
50.06	0\\
50.07	0\\
50.08	0\\
50.09	0\\
50.1	0\\
50.11	0\\
50.12	0\\
50.13	0\\
50.14	1.73472347597681e-18\\
50.15	0\\
50.16	0\\
50.17	0\\
50.18	0\\
50.19	0\\
50.2	0\\
50.21	1.73472347597681e-18\\
50.22	0\\
50.23	0\\
50.24	0\\
50.25	0\\
50.26	0\\
50.27	0\\
50.28	0\\
50.29	0\\
50.3	0\\
50.31	1.73472347597681e-18\\
50.32	0\\
50.33	0\\
50.34	0\\
50.35	0\\
50.36	0\\
50.37	1.73472347597681e-18\\
50.38	0\\
50.39	0\\
50.4	0\\
50.41	0\\
50.42	0\\
50.43	0\\
50.44	0\\
50.45	0\\
50.46	0\\
50.47	1.73472347597681e-18\\
50.48	0\\
50.49	0\\
50.5	0\\
50.51	1.73472347597681e-18\\
50.52	0\\
50.53	0\\
50.54	0\\
50.55	0\\
50.56	1.73472347597681e-18\\
50.57	0\\
50.58	0\\
50.59	0\\
50.6	0\\
50.61	0\\
50.62	0\\
50.63	0\\
50.64	0\\
50.65	0\\
50.66	0\\
50.67	0\\
50.68	0\\
50.69	0\\
50.7	0\\
50.71	0\\
50.72	0\\
50.73	0\\
50.74	0\\
50.75	0\\
50.76	0\\
50.77	0\\
50.78	1.73472347597681e-18\\
50.79	0\\
50.8	0\\
50.81	0\\
50.82	0\\
50.83	0\\
50.84	0\\
50.85	0\\
50.86	0\\
50.87	1.73472347597681e-18\\
50.88	1.73472347597681e-18\\
50.89	0\\
50.9	0\\
50.91	0\\
50.92	0\\
50.93	1.73472347597681e-18\\
50.94	0\\
50.95	0\\
50.96	1.73472347597681e-18\\
50.97	0\\
50.98	0\\
50.99	0\\
51	0\\
51.01	1.73472347597681e-18\\
51.02	0\\
51.03	0\\
51.04	0\\
51.05	0\\
51.06	0\\
51.07	0\\
51.08	0\\
51.09	0\\
51.1	0\\
51.11	0\\
51.12	0\\
51.13	0\\
51.14	0\\
51.15	0\\
51.16	0\\
51.17	0\\
51.18	0\\
51.19	1.73472347597681e-18\\
51.2	0\\
51.21	0\\
51.22	0\\
51.23	1.73472347597681e-18\\
51.24	0\\
51.25	0\\
51.26	0\\
51.27	0\\
51.28	0\\
51.29	0\\
51.3	0\\
51.31	0\\
51.32	1.73472347597681e-18\\
51.33	0\\
51.34	0\\
51.35	0\\
51.36	0\\
51.37	1.73472347597681e-18\\
51.38	0\\
51.39	0\\
51.4	0\\
51.41	0\\
51.42	1.73472347597681e-18\\
51.43	0\\
51.44	0\\
51.45	0\\
51.46	1.73472347597681e-18\\
51.47	0\\
51.48	0\\
51.49	1.73472347597681e-18\\
51.5	0\\
51.51	0\\
51.52	0\\
51.53	0\\
51.54	0\\
51.55	0\\
51.56	0\\
51.57	0\\
51.58	1.73472347597681e-18\\
51.59	0\\
51.6	0\\
51.61	0\\
51.62	0\\
51.63	0\\
51.64	0\\
51.65	0\\
51.66	0\\
51.67	0\\
51.68	1.73472347597681e-18\\
51.69	0\\
51.7	1.73472347597681e-18\\
51.71	0\\
51.72	0\\
51.73	0\\
51.74	0\\
51.75	0\\
51.76	0\\
51.77	0\\
51.78	0\\
51.79	0\\
51.8	0\\
51.81	0\\
51.82	0\\
51.83	0\\
51.84	0\\
51.85	0\\
51.86	0\\
51.87	0\\
51.88	0\\
51.89	0\\
51.9	0\\
51.91	0\\
51.92	0\\
51.93	1.73472347597681e-18\\
51.94	0\\
51.95	0\\
51.96	0\\
51.97	0\\
51.98	0\\
51.99	0\\
52	0\\
52.01	0\\
52.02	0\\
52.03	0\\
52.04	0\\
52.05	0\\
52.06	0\\
52.07	0\\
52.08	0\\
52.09	0\\
52.1	0\\
52.11	0\\
52.12	0\\
52.13	0\\
52.14	0\\
52.15	0\\
52.16	1.73472347597681e-18\\
52.17	1.73472347597681e-18\\
52.18	0\\
52.19	0\\
52.2	0\\
52.21	0\\
52.22	1.73472347597681e-18\\
52.23	0\\
52.24	0\\
52.25	0\\
52.26	0\\
52.27	0\\
52.28	0\\
52.29	0\\
52.3	0\\
52.31	0\\
52.32	0\\
52.33	0\\
52.34	0\\
52.35	0\\
52.36	0\\
52.37	1.73472347597681e-18\\
52.38	0\\
52.39	1.73472347597681e-18\\
52.4	1.73472347597681e-18\\
52.41	0\\
52.42	0\\
52.43	0\\
52.44	0\\
52.45	0\\
52.46	0\\
52.47	0\\
52.48	0\\
52.49	0\\
52.5	0\\
52.51	0\\
52.52	0\\
52.53	0\\
52.54	0\\
52.55	0\\
52.56	0\\
52.57	0\\
52.58	0\\
52.59	0\\
52.6	0\\
52.61	1.73472347597681e-18\\
52.62	1.73472347597681e-18\\
52.63	0\\
52.64	1.73472347597681e-18\\
52.65	0\\
52.66	0\\
52.67	0\\
52.68	0\\
52.69	0\\
52.7	0\\
52.71	0\\
52.72	1.73472347597681e-18\\
52.73	0\\
52.74	0\\
52.75	0\\
52.76	0\\
52.77	0\\
52.78	0\\
52.79	0\\
52.8	0\\
52.81	0\\
52.82	0\\
52.83	0\\
52.84	0\\
52.85	1.73472347597681e-18\\
52.86	0\\
52.87	1.73472347597681e-18\\
52.88	0\\
52.89	0\\
52.9	1.73472347597681e-18\\
52.91	0\\
52.92	0\\
52.93	0\\
52.94	0\\
52.95	1.73472347597681e-18\\
52.96	0\\
52.97	0\\
52.98	0\\
52.99	0\\
53	0\\
53.01	1.73472347597681e-18\\
53.02	0\\
53.03	0\\
53.04	0\\
53.05	0\\
53.06	0\\
53.07	0\\
53.08	0\\
53.09	0\\
53.1	1.73472347597681e-18\\
53.11	1.73472347597681e-18\\
53.12	0\\
53.13	0\\
53.14	0\\
53.15	0\\
53.16	0\\
53.17	0\\
53.18	0\\
53.19	0\\
53.2	1.73472347597681e-18\\
53.21	0\\
53.22	0\\
53.23	0\\
53.24	0\\
53.25	0\\
53.26	0\\
53.27	1.73472347597681e-18\\
53.28	0\\
53.29	0\\
53.3	0\\
53.31	0\\
53.32	0\\
53.33	0\\
53.34	0\\
53.35	0\\
53.36	0\\
53.37	0\\
53.38	0\\
53.39	0\\
53.4	0\\
53.41	0\\
53.42	0\\
53.43	0\\
53.44	0\\
53.45	1.73472347597681e-18\\
53.46	0\\
53.47	0\\
53.48	0\\
53.49	0\\
53.5	0\\
53.51	0\\
53.52	1.73472347597681e-18\\
53.53	1.73472347597681e-18\\
53.54	0\\
53.55	0\\
53.56	0\\
53.57	0\\
53.58	1.73472347597681e-18\\
53.59	0\\
53.6	0\\
53.61	0\\
53.62	0\\
53.63	0\\
53.64	0\\
53.65	0\\
53.66	0\\
53.67	0\\
53.68	0\\
53.69	0\\
53.7	0\\
53.71	0\\
53.72	0\\
53.73	0\\
53.74	0\\
53.75	0\\
53.76	0\\
53.77	0\\
53.78	0\\
53.79	1.73472347597681e-18\\
53.8	0\\
53.81	0\\
53.82	0\\
53.83	0\\
53.84	0\\
53.85	1.73472347597681e-18\\
53.86	0\\
53.87	0\\
53.88	0\\
53.89	0\\
53.9	0\\
53.91	0\\
53.92	0\\
53.93	0\\
53.94	0\\
53.95	0\\
53.96	0\\
53.97	0\\
53.98	0\\
53.99	0\\
54	0\\
54.01	0\\
54.02	0\\
54.03	0\\
54.04	0\\
54.05	0\\
54.06	0\\
54.07	1.73472347597681e-18\\
54.08	0\\
54.09	1.73472347597681e-18\\
54.1	0\\
54.11	1.73472347597681e-18\\
54.12	0\\
54.13	1.73472347597681e-18\\
54.14	0\\
54.15	1.73472347597681e-18\\
54.16	0\\
54.17	0\\
54.18	0\\
54.19	0\\
54.2	1.73472347597681e-18\\
54.21	1.73472347597681e-18\\
54.22	0\\
54.23	0\\
54.24	0\\
54.25	0\\
54.26	0\\
54.27	0\\
54.28	0\\
54.29	0\\
54.3	0\\
54.31	0\\
54.32	1.73472347597681e-18\\
54.33	0\\
54.34	0\\
54.35	0\\
54.36	0\\
54.37	0\\
54.38	0\\
54.39	0\\
54.4	0\\
54.41	0\\
54.42	0\\
54.43	0\\
54.44	0\\
54.45	1.73472347597681e-18\\
54.46	0\\
54.47	0\\
54.48	0\\
54.49	0\\
54.5	0\\
54.51	0\\
54.52	0\\
54.53	0\\
54.54	0\\
54.55	0\\
54.56	0\\
54.57	0\\
54.58	0\\
54.59	0\\
54.6	0\\
54.61	0\\
54.62	0\\
54.63	0\\
54.64	0\\
54.65	1.73472347597681e-18\\
54.66	0\\
54.67	0\\
54.68	0\\
54.69	1.73472347597681e-18\\
54.7	0\\
54.71	0\\
54.72	0\\
54.73	0\\
54.74	0\\
54.75	0\\
54.76	0\\
54.77	0\\
54.78	0\\
54.79	0\\
54.8	0\\
54.81	0\\
54.82	0\\
54.83	0\\
54.84	0\\
54.85	0\\
54.86	0\\
54.87	0\\
54.88	0\\
54.89	0\\
54.9	1.73472347597681e-18\\
54.91	0\\
54.92	0\\
54.93	0\\
54.94	1.73472347597681e-18\\
54.95	0\\
54.96	0\\
54.97	0\\
54.98	1.73472347597681e-18\\
54.99	0\\
55	0\\
55.01	0\\
55.02	0\\
55.03	0\\
55.04	0\\
55.05	0\\
55.06	0\\
55.07	1.73472347597681e-18\\
55.08	1.73472347597681e-18\\
55.09	0\\
55.1	0\\
55.11	0\\
55.12	0\\
55.13	0\\
55.14	1.73472347597681e-18\\
55.15	0\\
55.16	1.73472347597681e-18\\
55.17	1.73472347597681e-18\\
55.18	0\\
55.19	1.73472347597681e-18\\
55.2	0\\
55.21	0\\
55.22	0\\
55.23	0\\
55.24	0\\
55.25	0\\
55.26	0\\
55.27	0\\
55.28	0\\
55.29	0\\
55.3	0\\
55.31	0\\
55.32	0\\
55.33	0\\
55.34	0\\
55.35	0\\
55.36	1.73472347597681e-18\\
55.37	0\\
55.38	0\\
55.39	0\\
55.4	0\\
55.41	0\\
55.42	0\\
55.43	0\\
55.44	0\\
55.45	0\\
55.46	0\\
55.47	1.73472347597681e-18\\
55.48	0\\
55.49	1.73472347597681e-18\\
55.5	0\\
55.51	0\\
55.52	0\\
55.53	0\\
55.54	0\\
55.55	0\\
55.56	1.73472347597681e-18\\
55.57	0\\
55.58	0\\
55.59	0\\
55.6	0\\
55.61	0\\
55.62	0\\
55.63	0\\
55.64	0\\
55.65	0\\
55.66	0\\
55.67	0\\
55.68	0\\
55.69	0\\
55.7	0\\
55.71	0\\
55.72	0\\
55.73	1.73472347597681e-18\\
55.74	0\\
55.75	0\\
55.76	0\\
55.77	1.73472347597681e-18\\
55.78	0\\
55.79	0\\
55.8	0\\
55.81	0\\
55.82	0\\
55.83	0\\
55.84	0\\
55.85	0\\
55.86	0\\
55.87	0\\
55.88	0\\
55.89	0\\
55.9	0\\
55.91	0\\
55.92	0\\
55.93	0\\
55.94	0\\
55.95	1.73472347597681e-18\\
55.96	0\\
55.97	0\\
55.98	0\\
55.99	0\\
56	0\\
56.01	1.73472347597681e-18\\
56.02	0\\
56.03	0\\
56.04	1.73472347597681e-18\\
56.05	1.73472347597681e-18\\
56.06	0\\
56.07	1.73472347597681e-18\\
56.08	0\\
56.09	0\\
56.1	0\\
56.11	0\\
56.12	0\\
56.13	0\\
56.14	0\\
56.15	0\\
56.16	0\\
56.17	0\\
56.18	0\\
56.19	0\\
56.2	0\\
56.21	0\\
56.22	0\\
56.23	0\\
56.24	1.73472347597681e-18\\
56.25	0\\
56.26	0\\
56.27	0\\
56.28	0\\
56.29	1.73472347597681e-18\\
56.3	0\\
56.31	0\\
56.32	0\\
56.33	0\\
56.34	0\\
56.35	0\\
56.36	0\\
56.37	0\\
56.38	0\\
56.39	0\\
56.4	0\\
56.41	0\\
56.42	0\\
56.43	0\\
56.44	0\\
56.45	0\\
56.46	0\\
56.47	0\\
56.48	0\\
56.49	0\\
56.5	0\\
56.51	0\\
56.52	0\\
56.53	0\\
56.54	0\\
56.55	0\\
56.56	0\\
56.57	0\\
56.58	0\\
56.59	0\\
56.6	0\\
56.61	0\\
56.62	0\\
56.63	0\\
56.64	0\\
56.65	0\\
56.66	0\\
56.67	0\\
56.68	0\\
56.69	0\\
56.7	1.73472347597681e-18\\
56.71	1.73472347597681e-18\\
56.72	0\\
56.73	0\\
56.74	1.73472347597681e-18\\
56.75	1.73472347597681e-18\\
56.76	0\\
56.77	0\\
56.78	0\\
56.79	1.73472347597681e-18\\
56.8	0\\
56.81	0\\
56.82	1.73472347597681e-18\\
56.83	0\\
56.84	0\\
56.85	0\\
56.86	0\\
56.87	0\\
56.88	0\\
56.89	0\\
56.9	0\\
56.91	0\\
56.92	0\\
56.93	0\\
56.94	0\\
56.95	0\\
56.96	0\\
56.97	0\\
56.98	0\\
56.99	0\\
57	1.73472347597681e-18\\
57.01	0\\
57.02	0\\
57.03	0\\
57.04	0\\
57.05	1.73472347597681e-18\\
57.06	1.73472347597681e-18\\
57.07	0\\
57.08	0\\
57.09	0\\
57.1	0\\
57.11	0\\
57.12	0\\
57.13	0\\
57.14	0\\
57.15	0\\
57.16	0\\
57.17	0\\
57.18	0\\
57.19	0\\
57.2	0\\
57.21	0\\
57.22	0\\
57.23	0\\
57.24	0\\
57.25	0\\
57.26	0\\
57.27	1.73472347597681e-18\\
57.28	0\\
57.29	0\\
57.3	0\\
57.31	0\\
57.32	0\\
57.33	0\\
57.34	0\\
57.35	0\\
57.36	0\\
57.37	0\\
57.38	1.73472347597681e-18\\
57.39	0\\
57.4	0\\
57.41	0\\
57.42	0\\
57.43	0\\
57.44	0\\
57.45	0\\
57.46	0\\
57.47	0\\
57.48	0\\
57.49	0\\
57.5	0\\
57.51	1.73472347597681e-18\\
57.52	0\\
57.53	0\\
57.54	0\\
57.55	0\\
57.56	0\\
57.57	0\\
57.58	0\\
57.59	0\\
57.6	0\\
57.61	0\\
57.62	0\\
57.63	0\\
57.64	0\\
57.65	0\\
57.66	0\\
57.67	0\\
57.68	1.73472347597681e-18\\
57.69	1.73472347597681e-18\\
57.7	0\\
57.71	0\\
57.72	0\\
57.73	0\\
57.74	0\\
57.75	0\\
57.76	0\\
57.77	1.73472347597681e-18\\
57.78	0\\
57.79	0\\
57.8	0\\
57.81	0\\
57.82	0\\
57.83	0\\
57.84	1.73472347597681e-18\\
57.85	1.73472347597681e-18\\
57.86	0\\
57.87	1.73472347597681e-18\\
57.88	0\\
57.89	0\\
57.9	0\\
57.91	0\\
57.92	0\\
57.93	0\\
57.94	0\\
57.95	0\\
57.96	0\\
57.97	0\\
57.98	0\\
57.99	0\\
58	1.73472347597681e-18\\
58.01	0\\
58.02	0\\
58.03	0\\
58.04	0\\
58.05	0\\
58.06	0\\
58.07	0\\
58.08	0\\
58.09	0\\
58.1	0\\
58.11	1.73472347597681e-18\\
58.12	0\\
58.13	1.73472347597681e-18\\
58.14	0\\
58.15	0\\
58.16	0\\
58.17	0\\
58.18	0\\
58.19	0\\
58.2	0\\
58.21	0\\
58.22	1.73472347597681e-18\\
58.23	0\\
58.24	0\\
58.25	1.73472347597681e-18\\
58.26	0\\
58.27	0\\
58.28	0\\
58.29	1.73472347597681e-18\\
58.3	0\\
58.31	1.73472347597681e-18\\
58.32	0\\
58.33	0\\
58.34	1.73472347597681e-18\\
58.35	1.73472347597681e-18\\
58.36	0\\
58.37	0\\
58.38	0\\
58.39	0\\
58.4	0\\
58.41	1.73472347597681e-18\\
58.42	1.73472347597681e-18\\
58.43	0\\
58.44	0\\
58.45	0\\
58.46	0\\
58.47	0\\
58.48	0\\
58.49	0\\
58.5	0\\
58.51	0\\
58.52	1.73472347597681e-18\\
58.53	0\\
58.54	0\\
58.55	0\\
58.56	0\\
58.57	0\\
58.58	0\\
58.59	0\\
58.6	0\\
58.61	0\\
58.62	0\\
58.63	0\\
58.64	0\\
58.65	0\\
58.66	0\\
58.67	1.73472347597681e-18\\
58.68	0\\
58.69	0\\
58.7	0\\
58.71	0\\
58.72	0\\
58.73	0\\
58.74	1.73472347597681e-18\\
58.75	0\\
58.76	0\\
58.77	0\\
58.78	0\\
58.79	1.73472347597681e-18\\
58.8	0\\
58.81	0\\
58.82	0\\
58.83	0\\
58.84	0\\
58.85	0\\
58.86	0\\
58.87	0\\
58.88	1.73472347597681e-18\\
58.89	0\\
58.9	0\\
58.91	0\\
58.92	0\\
58.93	0\\
58.94	0\\
58.95	0\\
58.96	0\\
58.97	0\\
58.98	1.73472347597681e-18\\
58.99	0\\
59	0\\
59.01	0\\
59.02	0\\
59.03	0\\
59.04	0\\
59.05	0\\
59.06	0\\
59.07	0\\
59.08	0\\
59.09	0\\
59.1	0\\
59.11	0\\
59.12	0\\
59.13	0\\
59.14	0\\
59.15	0\\
59.16	0\\
59.17	1.73472347597681e-18\\
59.18	0\\
59.19	0\\
59.2	1.73472347597681e-18\\
59.21	0\\
59.22	0\\
59.23	0\\
59.24	0\\
59.25	1.73472347597681e-18\\
59.26	0\\
59.27	0\\
59.28	0\\
59.29	0\\
59.3	1.73472347597681e-18\\
59.31	0\\
59.32	1.73472347597681e-18\\
59.33	0\\
59.34	0\\
59.35	0\\
59.36	0\\
59.37	0\\
59.38	0\\
59.39	0\\
59.4	0\\
59.41	1.73472347597681e-18\\
59.42	1.73472347597681e-18\\
59.43	0\\
59.44	0\\
59.45	0\\
59.46	0\\
59.47	0\\
59.48	0\\
59.49	0\\
59.5	1.73472347597681e-18\\
59.51	1.73472347597681e-18\\
59.52	0\\
59.53	0\\
59.54	0\\
59.55	0\\
59.56	0\\
59.57	0\\
59.58	0\\
59.59	1.73472347597681e-18\\
59.6	0\\
59.61	0\\
59.62	1.73472347597681e-18\\
59.63	0\\
59.64	0\\
59.65	0\\
59.66	1.73472347597681e-18\\
59.67	0\\
59.68	0\\
59.69	0\\
59.7	0\\
59.71	0\\
59.72	0\\
59.73	0\\
59.74	0\\
59.75	0\\
59.76	0\\
59.77	0\\
59.78	0\\
59.79	0\\
59.8	1.73472347597681e-18\\
59.81	0\\
59.82	0\\
59.83	0\\
59.84	0\\
59.85	0\\
59.86	1.73472347597681e-18\\
59.87	0\\
59.88	1.73472347597681e-18\\
59.89	0\\
59.9	0\\
59.91	0\\
59.92	0\\
59.93	1.73472347597681e-18\\
59.94	0\\
59.95	0\\
59.96	0\\
59.97	0\\
59.98	0\\
59.99	0\\
60	0\\
60.01	1.73472347597681e-18\\
60.02	0\\
60.03	0\\
60.04	0\\
60.05	0\\
60.06	0\\
60.07	0\\
60.08	0\\
60.09	0\\
60.1	0\\
60.11	0\\
60.12	0\\
60.13	1.73472347597681e-18\\
60.14	0\\
60.15	0\\
60.16	0\\
60.17	0\\
60.18	0\\
60.19	0\\
60.2	0\\
60.21	0\\
60.22	0\\
60.23	0\\
60.24	0\\
60.25	0\\
60.26	0\\
60.27	0\\
60.28	0\\
60.29	0\\
60.3	0\\
60.31	0\\
60.32	0\\
60.33	0\\
60.34	0\\
60.35	0\\
60.36	0\\
60.37	0\\
60.38	0\\
60.39	1.73472347597681e-18\\
60.4	0\\
60.41	0\\
60.42	0\\
60.43	0\\
60.44	0\\
60.45	0\\
60.46	0\\
60.47	0\\
60.48	0\\
60.49	1.73472347597681e-18\\
60.5	0\\
60.51	0\\
60.52	0\\
60.53	0\\
60.54	0\\
60.55	0\\
60.56	0\\
60.57	0\\
60.58	0\\
60.59	0\\
60.6	0\\
60.61	0\\
60.62	0\\
60.63	0\\
60.64	0\\
60.65	0\\
60.66	1.73472347597681e-18\\
60.67	0\\
60.68	0\\
60.69	0\\
60.7	0\\
60.71	0\\
60.72	0\\
60.73	0\\
60.74	0\\
60.75	0\\
60.76	0\\
60.77	1.73472347597681e-18\\
60.78	0\\
60.79	0\\
60.8	0\\
60.81	0\\
60.82	0\\
60.83	0\\
60.84	0\\
60.85	0\\
60.86	0\\
60.87	0\\
60.88	0\\
60.89	0\\
60.9	0\\
60.91	0\\
60.92	0\\
60.93	0\\
60.94	0\\
60.95	0\\
60.96	0\\
60.97	0\\
60.98	0\\
60.99	0\\
61	0\\
61.01	1.73472347597681e-18\\
61.02	0\\
61.03	0\\
61.04	0\\
61.05	0\\
61.06	1.73472347597681e-18\\
61.07	0\\
61.08	0\\
61.09	0\\
61.1	0\\
61.11	0\\
61.12	0\\
61.13	0\\
61.14	1.73472347597681e-18\\
61.15	0\\
61.16	0\\
61.17	0\\
61.18	0\\
61.19	0\\
61.2	0\\
61.21	0\\
61.22	0\\
61.23	0\\
61.24	0\\
61.25	0\\
61.26	0\\
61.27	0\\
61.28	1.73472347597681e-18\\
61.29	0\\
61.3	0\\
61.31	0\\
61.32	0\\
61.33	0\\
61.34	0\\
61.35	0\\
61.36	0\\
61.37	0\\
61.38	0\\
61.39	0\\
61.4	0\\
61.41	0\\
61.42	0\\
61.43	1.73472347597681e-18\\
61.44	0\\
61.45	0\\
61.46	1.73472347597681e-18\\
61.47	1.73472347597681e-18\\
61.48	0\\
61.49	1.73472347597681e-18\\
61.5	0\\
61.51	0\\
61.52	0\\
61.53	0\\
61.54	0\\
61.55	0\\
61.56	0\\
61.57	0\\
61.58	0\\
61.59	0\\
61.6	0\\
61.61	0\\
61.62	0\\
61.63	0\\
61.64	0\\
61.65	1.73472347597681e-18\\
61.66	0\\
61.67	1.73472347597681e-18\\
61.68	0\\
61.69	0\\
61.7	0\\
61.71	0\\
61.72	0\\
61.73	0\\
61.74	0\\
61.75	0\\
61.76	0\\
61.77	1.73472347597681e-18\\
61.78	0\\
61.79	0\\
61.8	0\\
61.81	0\\
61.82	1.73472347597681e-18\\
61.83	0\\
61.84	1.73472347597681e-18\\
61.85	0\\
61.86	0\\
61.87	0\\
61.88	1.73472347597681e-18\\
61.89	0\\
61.9	0\\
61.91	0\\
61.92	0\\
61.93	0\\
61.94	0\\
61.95	0\\
61.96	0\\
61.97	1.73472347597681e-18\\
61.98	0\\
61.99	0\\
62	0\\
62.01	0\\
62.02	0\\
62.03	0\\
62.04	1.73472347597681e-18\\
62.05	0\\
62.06	0\\
62.07	0\\
62.08	0\\
62.09	0\\
62.1	0\\
62.11	0\\
62.12	0\\
62.13	0\\
62.14	0\\
62.15	0\\
62.16	0\\
62.17	1.73472347597681e-18\\
62.18	0\\
62.19	0\\
62.2	0\\
62.21	0\\
62.22	0\\
62.23	0\\
62.24	0\\
62.25	0\\
62.26	0\\
62.27	0\\
62.28	0\\
62.29	0\\
62.3	0\\
62.31	0\\
62.32	0\\
62.33	0\\
62.34	0\\
62.35	0\\
62.36	0\\
62.37	0\\
62.38	0\\
62.39	0\\
62.4	0\\
62.41	0\\
62.42	0\\
62.43	0\\
62.44	0\\
62.45	0\\
62.46	1.73472347597681e-18\\
62.47	0\\
62.48	0\\
62.49	1.73472347597681e-18\\
62.5	0\\
62.51	0\\
62.52	0\\
62.53	0\\
62.54	0\\
62.55	0\\
62.56	0\\
62.57	0\\
62.58	0\\
62.59	0\\
62.6	0\\
62.61	0\\
62.62	0\\
62.63	0\\
62.64	0\\
62.65	0\\
62.66	0\\
62.67	0\\
62.68	0\\
62.69	0\\
62.7	0\\
62.71	0\\
62.72	0\\
62.73	0\\
62.74	0\\
62.75	0\\
62.76	0\\
62.77	1.73472347597681e-18\\
62.78	0\\
62.79	0\\
62.8	0\\
62.81	0\\
62.82	0\\
62.83	0\\
62.84	1.73472347597681e-18\\
62.85	0\\
62.86	0\\
62.87	0\\
62.88	0\\
62.89	0\\
62.9	0\\
62.91	0\\
62.92	0\\
62.93	0\\
62.94	0\\
62.95	0\\
62.96	0\\
62.97	0\\
62.98	0\\
62.99	1.73472347597681e-18\\
63	0\\
63.01	0\\
63.02	1.73472347597681e-18\\
63.03	0\\
63.04	0\\
63.05	1.73472347597681e-18\\
63.06	0\\
63.07	0\\
63.08	0\\
63.09	0\\
63.1	0\\
63.11	0\\
63.12	0\\
63.13	0\\
63.14	0\\
63.15	0\\
63.16	0\\
63.17	0\\
63.18	1.73472347597681e-18\\
63.19	0\\
63.2	0\\
63.21	0\\
63.22	0\\
63.23	0\\
63.24	0\\
63.25	0\\
63.26	0\\
63.27	0\\
63.28	0\\
63.29	0\\
63.3	0\\
63.31	0\\
63.32	0\\
63.33	0\\
63.34	0\\
63.35	0\\
63.36	0\\
63.37	1.73472347597681e-18\\
63.38	0\\
63.39	0\\
63.4	0\\
63.41	0\\
63.42	0\\
63.43	0\\
63.44	0\\
63.45	0\\
63.46	0\\
63.47	0\\
63.48	0\\
63.49	1.73472347597681e-18\\
63.5	0\\
63.51	0\\
63.52	0\\
63.53	0\\
63.54	0\\
63.55	0\\
63.56	0\\
63.57	0\\
63.58	0\\
63.59	0\\
63.6	0\\
63.61	0\\
63.62	0\\
63.63	0\\
63.64	0\\
63.65	0\\
63.66	0\\
63.67	0\\
63.68	0\\
63.69	0\\
63.7	0\\
63.71	0\\
63.72	0\\
63.73	0\\
63.74	0\\
63.75	0\\
63.76	0\\
63.77	0\\
63.78	0\\
63.79	0\\
63.8	1.73472347597681e-18\\
63.81	0\\
63.82	0\\
63.83	0\\
63.84	0\\
63.85	0\\
63.86	0\\
63.87	0\\
63.88	0\\
63.89	0\\
63.9	0\\
63.91	0\\
63.92	1.73472347597681e-18\\
63.93	0\\
63.94	0\\
63.95	0\\
63.96	1.73472347597681e-18\\
63.97	1.73472347597681e-18\\
63.98	0\\
63.99	0\\
64	0\\
64.01	0\\
64.02	0\\
64.03	0\\
64.04	0\\
64.05	0\\
64.06	1.73472347597681e-18\\
64.07	0\\
64.08	0\\
64.09	0\\
64.1	0\\
64.11	0\\
64.12	0\\
64.13	1.73472347597681e-18\\
64.14	0\\
64.15	0\\
64.16	0\\
64.17	0\\
64.18	0\\
64.19	0\\
64.2	0\\
64.21	0\\
64.22	0\\
64.23	0\\
64.24	0\\
64.25	0\\
64.26	0\\
64.27	0\\
64.28	0\\
64.29	1.73472347597681e-18\\
64.3	0\\
64.31	0\\
64.32	0\\
64.33	0\\
64.34	1.73472347597681e-18\\
64.35	0\\
64.36	0\\
64.37	0\\
64.38	0\\
64.39	0\\
64.4	0\\
64.41	0\\
64.42	1.73472347597681e-18\\
64.43	0\\
64.44	0\\
64.45	0\\
64.46	1.73472347597681e-18\\
64.47	0\\
64.48	0\\
64.49	0\\
64.5	0\\
64.51	0\\
64.52	0\\
64.53	0\\
64.54	0\\
64.55	0\\
64.56	0\\
64.57	0\\
64.58	0\\
64.59	1.73472347597681e-18\\
64.6	0\\
64.61	0\\
64.62	0\\
64.63	0\\
64.64	0\\
64.65	1.73472347597681e-18\\
64.66	0\\
64.67	0\\
64.68	0\\
64.69	0\\
64.7	0\\
64.71	0\\
64.72	0\\
64.73	0\\
64.74	0\\
64.75	1.73472347597681e-18\\
64.76	0\\
64.77	0\\
64.78	0\\
64.79	0\\
64.8	0\\
64.81	0\\
64.82	0\\
64.83	1.73472347597681e-18\\
64.84	1.73472347597681e-18\\
64.85	0\\
64.86	0\\
64.87	1.73472347597681e-18\\
64.88	0\\
64.89	0\\
64.9	0\\
64.91	0\\
64.92	0\\
64.93	0\\
64.94	0\\
64.95	0\\
64.96	0\\
64.97	1.73472347597681e-18\\
64.98	0\\
64.99	1.73472347597681e-18\\
65	0\\
65.01	0\\
65.02	0\\
65.03	0\\
65.04	0\\
65.05	0\\
65.06	0\\
65.07	1.73472347597681e-18\\
65.08	0\\
65.09	0\\
65.1	0\\
65.11	0\\
65.12	0\\
65.13	0\\
65.14	0\\
65.15	0\\
65.16	0\\
65.17	0\\
65.18	1.73472347597681e-18\\
65.19	0\\
65.2	0\\
65.21	0\\
65.22	0\\
65.23	0\\
65.24	0\\
65.25	0\\
65.26	0\\
65.27	0\\
65.28	0\\
65.29	0\\
65.3	0\\
65.31	0\\
65.32	0\\
65.33	0\\
65.34	0\\
65.35	0\\
65.36	0\\
65.37	1.73472347597681e-18\\
65.38	0\\
65.39	0\\
65.4	0\\
65.41	0\\
65.42	0\\
65.43	0\\
65.44	0\\
65.45	0\\
65.46	0\\
65.47	0\\
65.48	0\\
65.49	0\\
65.5	0\\
65.51	0\\
65.52	0\\
65.53	0\\
65.54	0\\
65.55	0\\
65.56	0\\
65.57	0\\
65.58	0\\
65.59	0\\
65.6	0\\
65.61	0\\
65.62	0\\
65.63	0\\
65.64	0\\
65.65	1.73472347597681e-18\\
65.66	0\\
65.67	0\\
65.68	1.73472347597681e-18\\
65.69	0\\
65.7	0\\
65.71	0\\
65.72	0\\
65.73	0\\
65.74	0\\
65.75	0\\
65.76	0\\
65.77	0\\
65.78	0\\
65.79	0\\
65.8	1.73472347597681e-18\\
65.81	1.73472347597681e-18\\
65.82	0\\
65.83	1.73472347597681e-18\\
65.84	0\\
65.85	0\\
65.86	1.73472347597681e-18\\
65.87	0\\
65.88	0\\
65.89	0\\
65.9	0\\
65.91	0\\
65.92	1.73472347597681e-18\\
65.93	1.73472347597681e-18\\
65.94	0\\
65.95	0\\
65.96	0\\
65.97	0\\
65.98	0\\
65.99	0\\
66	0\\
66.01	0\\
66.02	0\\
66.03	0\\
66.04	0\\
66.05	0\\
66.06	1.73472347597681e-18\\
66.07	0\\
66.08	0\\
66.09	0\\
66.1	0\\
66.11	0\\
66.12	0\\
66.13	0\\
66.14	0\\
66.15	1.73472347597681e-18\\
66.16	0\\
66.17	0\\
66.18	0\\
66.19	0\\
66.2	0\\
66.21	0\\
66.22	0\\
66.23	1.73472347597681e-18\\
66.24	0\\
66.25	0\\
66.26	0\\
66.27	0\\
66.28	0\\
66.29	1.73472347597681e-18\\
66.3	0\\
66.31	0\\
66.32	0\\
66.33	0\\
66.34	0\\
66.35	1.73472347597681e-18\\
66.36	0\\
66.37	1.73472347597681e-18\\
66.38	0\\
66.39	0\\
66.4	0\\
66.41	0\\
66.42	0\\
66.43	0\\
66.44	0\\
66.45	0\\
66.46	0\\
66.47	0\\
66.48	0\\
66.49	1.73472347597681e-18\\
66.5	0\\
66.51	0\\
66.52	0\\
66.53	0\\
66.54	0\\
66.55	0\\
66.56	0\\
66.57	1.73472347597681e-18\\
66.58	0\\
66.59	0\\
66.6	0\\
66.61	0\\
66.62	0\\
66.63	0\\
66.64	0\\
66.65	1.73472347597681e-18\\
66.66	0\\
66.67	0\\
66.68	1.73472347597681e-18\\
66.69	0\\
66.7	0\\
66.71	0\\
66.72	0\\
66.73	0\\
66.74	0\\
66.75	0\\
66.76	0\\
66.77	0\\
66.78	0\\
66.79	0\\
66.8	0\\
66.81	0\\
66.82	0\\
66.83	0\\
66.84	0\\
66.85	0\\
66.86	0\\
66.87	0\\
66.88	0\\
66.89	0\\
66.9	0\\
66.91	1.73472347597681e-18\\
66.92	0\\
66.93	0\\
66.94	1.73472347597681e-18\\
66.95	0\\
66.96	0\\
66.97	1.73472347597681e-18\\
66.98	0\\
66.99	0\\
67	0\\
67.01	0\\
67.02	1.73472347597681e-18\\
67.03	0\\
67.04	0\\
67.05	1.73472347597681e-18\\
67.06	0\\
67.07	0\\
67.08	1.73472347597681e-18\\
67.09	0\\
67.1	0\\
67.11	0\\
67.12	0\\
67.13	0\\
67.14	0\\
67.15	0\\
67.16	0\\
67.17	0\\
67.18	0\\
67.19	0\\
67.2	0\\
67.21	0\\
67.22	0\\
67.23	0\\
67.24	0\\
67.25	0\\
67.26	0\\
67.27	0\\
67.28	0\\
67.29	0\\
67.3	0\\
67.31	0\\
67.32	0\\
67.33	0\\
67.34	0\\
67.35	0\\
67.36	1.73472347597681e-18\\
67.37	0\\
67.38	0\\
67.39	0\\
67.4	0\\
67.41	1.73472347597681e-18\\
67.42	0\\
67.43	0\\
67.44	0\\
67.45	0\\
67.46	1.73472347597681e-18\\
67.47	0\\
67.48	0\\
67.49	0\\
67.5	1.73472347597681e-18\\
67.51	0\\
67.52	0\\
67.53	0\\
67.54	0\\
67.55	0\\
67.56	0\\
67.57	0\\
67.58	0\\
67.59	0\\
67.6	0\\
67.61	1.73472347597681e-18\\
67.62	0\\
67.63	0\\
67.64	1.73472347597681e-18\\
67.65	0\\
67.66	0\\
67.67	0\\
67.68	0\\
67.69	0\\
67.7	0\\
67.71	0\\
67.72	0\\
67.73	0\\
67.74	0\\
67.75	0\\
67.76	0\\
67.77	0\\
67.78	0\\
67.79	0\\
67.8	0\\
67.81	0\\
67.82	0\\
67.83	0\\
67.84	0\\
67.85	1.73472347597681e-18\\
67.86	0\\
67.87	0\\
67.88	0\\
67.89	0\\
67.9	0\\
67.91	0\\
67.92	0\\
67.93	0\\
67.94	0\\
67.95	0\\
67.96	0\\
67.97	0\\
67.98	0\\
67.99	0\\
68	0\\
68.01	0\\
68.02	1.73472347597681e-18\\
68.03	0\\
68.04	0\\
68.05	0\\
68.06	0\\
68.07	0\\
68.08	0\\
68.09	1.73472347597681e-18\\
68.1	0\\
68.11	0\\
68.12	0\\
68.13	0\\
68.14	0\\
68.15	0\\
68.16	0\\
68.17	1.73472347597681e-18\\
68.18	0\\
68.19	1.73472347597681e-18\\
68.2	0\\
68.21	0\\
68.22	0\\
68.23	0\\
68.24	0\\
68.25	0\\
68.26	0\\
68.27	0\\
68.28	0\\
68.29	0\\
68.3	1.73472347597681e-18\\
68.31	0\\
68.32	0\\
68.33	0\\
68.34	0\\
68.35	0\\
68.36	1.73472347597681e-18\\
68.37	0\\
68.38	0\\
68.39	0\\
68.4	0\\
68.41	0\\
68.42	0\\
68.43	1.73472347597681e-18\\
68.44	0\\
68.45	0\\
68.46	0\\
68.47	0\\
68.48	0\\
68.49	0\\
68.5	0\\
68.51	0\\
68.52	0\\
68.53	0\\
68.54	0\\
68.55	0\\
68.56	0\\
68.57	0\\
68.58	0\\
68.59	0\\
68.6	0\\
68.61	0\\
68.62	0\\
68.63	0\\
68.64	0\\
68.65	0\\
68.66	0\\
68.67	0\\
68.68	0\\
68.69	0\\
68.7	0\\
68.71	1.73472347597681e-18\\
68.72	0\\
68.73	0\\
68.74	0\\
68.75	0\\
68.76	0\\
68.77	0\\
68.78	0\\
68.79	0\\
68.8	0\\
68.81	0\\
68.82	0\\
68.83	0\\
68.84	0\\
68.85	0\\
68.86	0\\
68.87	0\\
68.88	0\\
68.89	0\\
68.9	0\\
68.91	0\\
68.92	0\\
68.93	0\\
68.94	0\\
68.95	0\\
68.96	0\\
68.97	0\\
68.98	0\\
68.99	0\\
69	1.73472347597681e-18\\
69.01	0\\
69.02	0\\
69.03	0\\
69.04	0\\
69.05	1.73472347597681e-18\\
69.06	0\\
69.07	0\\
69.08	0\\
69.09	0\\
69.1	0\\
69.11	0\\
69.12	0\\
69.13	1.73472347597681e-18\\
69.14	0\\
69.15	1.73472347597681e-18\\
69.16	0\\
69.17	0\\
69.18	0\\
69.19	1.73472347597681e-18\\
69.2	0\\
69.21	1.73472347597681e-18\\
69.22	0\\
69.23	0\\
69.24	0\\
69.25	0\\
69.26	0\\
69.27	0\\
69.28	0\\
69.29	0\\
69.3	0\\
69.31	0\\
69.32	0\\
69.33	0\\
69.34	0\\
69.35	0\\
69.36	0\\
69.37	1.73472347597681e-18\\
69.38	0\\
69.39	0\\
69.4	0\\
69.41	0\\
69.42	0\\
69.43	1.73472347597681e-18\\
69.44	0\\
69.45	1.73472347597681e-18\\
69.46	0\\
69.47	0\\
69.48	0\\
69.49	0\\
69.5	0\\
69.51	0\\
69.52	0\\
69.53	0\\
69.54	0\\
69.55	0\\
69.56	0\\
69.57	1.73472347597681e-18\\
69.58	1.73472347597681e-18\\
69.59	0\\
69.6	0\\
69.61	0\\
69.62	0\\
69.63	0\\
69.64	1.73472347597681e-18\\
69.65	0\\
69.66	0\\
69.67	1.73472347597681e-18\\
69.68	0\\
69.69	0\\
69.7	0\\
69.71	0\\
69.72	0\\
69.73	0\\
69.74	0\\
69.75	0\\
69.76	0\\
69.77	0\\
69.78	0\\
69.79	0\\
69.8	0\\
69.81	0\\
69.82	0\\
69.83	0\\
69.84	0\\
69.85	0\\
69.86	0\\
69.87	0\\
69.88	1.73472347597681e-18\\
69.89	0\\
69.9	0\\
69.91	0\\
69.92	0\\
69.93	0\\
69.94	0\\
69.95	1.73472347597681e-18\\
69.96	0\\
69.97	0\\
69.98	0\\
69.99	0\\
70	0\\
70.01	0\\
70.02	0\\
70.03	0\\
70.04	0\\
70.05	0\\
70.06	0\\
70.07	0\\
70.08	0\\
70.09	0\\
70.1	1.73472347597681e-18\\
70.11	0\\
70.12	0\\
70.13	0\\
70.14	1.73472347597681e-18\\
70.15	0\\
70.16	0\\
70.17	0\\
70.18	0\\
70.19	0\\
70.2	0\\
70.21	0\\
70.22	0\\
70.23	0\\
70.24	0\\
70.25	0\\
70.26	0\\
70.27	0\\
70.28	0\\
70.29	0\\
70.3	0\\
70.31	0\\
70.32	0\\
70.33	1.73472347597681e-18\\
70.34	0\\
70.35	0\\
70.36	0\\
70.37	0\\
70.38	0\\
70.39	0\\
70.4	1.73472347597681e-18\\
70.41	0\\
70.42	0\\
70.43	0\\
70.44	0\\
70.45	0\\
70.46	1.73472347597681e-18\\
70.47	0\\
70.48	0\\
70.49	0\\
70.5	0\\
70.51	0\\
70.52	0\\
70.53	0\\
70.54	0\\
70.55	0\\
70.56	0\\
70.57	0\\
70.58	0\\
70.59	0\\
70.6	0\\
70.61	0\\
70.62	0\\
70.63	0\\
70.64	0\\
70.65	1.73472347597681e-18\\
70.66	0\\
70.67	0\\
70.68	0\\
70.69	0\\
70.7	0\\
70.71	1.73472347597681e-18\\
70.72	0\\
70.73	0\\
70.74	0\\
70.75	0\\
70.76	0\\
70.77	0\\
70.78	0\\
70.79	0\\
70.8	0\\
70.81	0\\
70.82	1.73472347597681e-18\\
70.83	0\\
70.84	0\\
70.85	1.73472347597681e-18\\
70.86	0\\
70.87	0\\
70.88	0\\
70.89	0\\
70.9	0\\
70.91	0\\
70.92	1.73472347597681e-18\\
70.93	0\\
70.94	0\\
70.95	0\\
70.96	0\\
70.97	0\\
70.98	0\\
70.99	0\\
71	0\\
71.01	0\\
71.02	0\\
71.03	0\\
71.04	0\\
71.05	0\\
71.06	0\\
71.07	0\\
71.08	1.73472347597681e-18\\
71.09	1.73472347597681e-18\\
71.1	0\\
71.11	0\\
71.12	0\\
71.13	0\\
71.14	0\\
71.15	0\\
71.16	0\\
71.17	0\\
71.18	1.73472347597681e-18\\
71.19	0\\
71.2	0\\
71.21	0\\
71.22	0\\
71.23	0\\
71.24	0\\
71.25	0\\
71.26	0\\
71.27	1.73472347597681e-18\\
71.28	0\\
71.29	0\\
71.3	0\\
71.31	0\\
71.32	1.73472347597681e-18\\
71.33	0\\
71.34	0\\
71.35	0\\
71.36	1.73472347597681e-18\\
71.37	0\\
71.38	0\\
71.39	0\\
71.4	0\\
71.41	0\\
71.42	0\\
71.43	1.73472347597681e-18\\
71.44	0\\
71.45	0\\
71.46	0\\
71.47	0\\
71.48	0\\
71.49	0\\
71.5	0\\
71.51	0\\
71.52	1.73472347597681e-18\\
71.53	0\\
71.54	0\\
71.55	0\\
71.56	0\\
71.57	0\\
71.58	0\\
71.59	0\\
71.6	0\\
71.61	1.73472347597681e-18\\
71.62	0\\
71.63	1.73472347597681e-18\\
71.64	0\\
71.65	0\\
71.66	0\\
71.67	0\\
71.68	0\\
71.69	0\\
71.7	0\\
71.71	1.73472347597681e-18\\
71.72	0\\
71.73	0\\
71.74	0\\
71.75	0\\
71.76	0\\
71.77	0\\
71.78	0\\
71.79	0\\
71.8	0\\
71.81	0\\
71.82	0\\
71.83	1.73472347597681e-18\\
71.84	0\\
71.85	0\\
71.86	0\\
71.87	0\\
71.88	0\\
71.89	1.73472347597681e-18\\
71.9	0\\
71.91	0\\
71.92	0\\
71.93	0\\
71.94	0\\
71.95	0\\
71.96	1.73472347597681e-18\\
71.97	0\\
71.98	0\\
71.99	0\\
72	0\\
72.01	0\\
72.02	0\\
72.03	0\\
72.04	0\\
72.05	0\\
72.06	0\\
72.07	0\\
72.08	0\\
72.09	0\\
72.1	0\\
72.11	0\\
72.12	1.73472347597681e-18\\
72.13	0\\
72.14	0\\
72.15	0\\
72.16	0\\
72.17	0\\
72.18	0\\
72.19	1.73472347597681e-18\\
72.2	0\\
72.21	0\\
72.22	0\\
72.23	0\\
72.24	0\\
72.25	0\\
72.26	0\\
72.27	0\\
72.28	0\\
72.29	0\\
72.3	0\\
72.31	0\\
72.32	0\\
72.33	0\\
72.34	0\\
72.35	0\\
72.36	0\\
72.37	0\\
72.38	0\\
72.39	0\\
72.4	0\\
72.41	0\\
72.42	0\\
72.43	0\\
72.44	0\\
72.45	1.73472347597681e-18\\
72.46	0\\
72.47	0\\
72.48	0\\
72.49	0\\
72.5	0\\
72.51	0\\
72.52	0\\
72.53	0\\
72.54	0\\
72.55	1.73472347597681e-18\\
72.56	0\\
72.57	0\\
72.58	0\\
72.59	0\\
72.6	0\\
72.61	0\\
72.62	1.73472347597681e-18\\
72.63	0\\
72.64	0\\
72.65	0\\
72.66	0\\
72.67	0\\
72.68	0\\
72.69	0\\
72.7	1.73472347597681e-18\\
72.71	0\\
72.72	0\\
72.73	1.73472347597681e-18\\
72.74	0\\
72.75	0\\
72.76	0\\
72.77	0\\
72.78	1.73472347597681e-18\\
72.79	0\\
72.8	0\\
72.81	0\\
72.82	0\\
72.83	0\\
72.84	0\\
72.85	1.73472347597681e-18\\
72.86	0\\
72.87	0\\
72.88	0\\
72.89	1.73472347597681e-18\\
72.9	0\\
72.91	0\\
72.92	0\\
72.93	0\\
72.94	0\\
72.95	0\\
72.96	0\\
72.97	0\\
72.98	0\\
72.99	0\\
73	0\\
73.01	0\\
73.02	1.73472347597681e-18\\
73.03	0\\
73.04	0\\
73.05	0\\
73.06	0\\
73.07	0\\
73.08	0\\
73.09	0\\
73.1	0\\
73.11	0\\
73.12	0\\
73.13	1.73472347597681e-18\\
73.14	0\\
73.15	0\\
73.16	0\\
73.17	0\\
73.18	0\\
73.19	0\\
73.2	0\\
73.21	0\\
73.22	0\\
73.23	1.73472347597681e-18\\
73.24	0\\
73.25	0\\
73.26	0\\
73.27	1.73472347597681e-18\\
73.28	0\\
73.29	0\\
73.3	0\\
73.31	0\\
73.32	0\\
73.33	0\\
73.34	0\\
73.35	0\\
73.36	0\\
73.37	0\\
73.38	0\\
73.39	1.73472347597681e-18\\
73.4	0\\
73.41	0\\
73.42	0\\
73.43	0\\
73.44	0\\
73.45	0\\
73.46	0\\
73.47	0\\
73.48	0\\
73.49	0\\
73.5	0\\
73.51	1.73472347597681e-18\\
73.52	1.73472347597681e-18\\
73.53	0\\
73.54	0\\
73.55	1.73472347597681e-18\\
73.56	0\\
73.57	0\\
73.58	0\\
73.59	0\\
73.6	0\\
73.61	0\\
73.62	0\\
73.63	0\\
73.64	0\\
73.65	0\\
73.66	1.73472347597681e-18\\
73.67	0\\
73.68	0\\
73.69	0\\
73.7	0\\
73.71	0\\
73.72	0\\
73.73	1.73472347597681e-18\\
73.74	0\\
73.75	0\\
73.76	1.73472347597681e-18\\
73.77	0\\
73.78	0\\
73.79	0\\
73.8	1.73472347597681e-18\\
73.81	0\\
73.82	0\\
73.83	0\\
73.84	0\\
73.85	0\\
73.86	0\\
73.87	0\\
73.88	0\\
73.89	0\\
73.9	0\\
73.91	0\\
73.92	0\\
73.93	0\\
73.94	0\\
73.95	0\\
73.96	0\\
73.97	0\\
73.98	1.73472347597681e-18\\
73.99	0\\
74	0\\
74.01	1.73472347597681e-18\\
74.02	0\\
74.03	0\\
74.04	0\\
74.05	0\\
74.06	0\\
74.07	0\\
74.08	0\\
74.09	0\\
74.1	1.73472347597681e-18\\
74.11	1.73472347597681e-18\\
74.12	1.73472347597681e-18\\
74.13	0\\
74.14	1.73472347597681e-18\\
74.15	0\\
74.16	0\\
74.17	1.73472347597681e-18\\
74.18	0\\
74.19	0\\
74.2	0\\
74.21	0\\
74.22	0\\
74.23	0\\
74.24	0\\
74.25	0\\
74.26	0\\
74.27	0\\
74.28	0\\
74.29	0\\
74.3	1.73472347597681e-18\\
74.31	0\\
74.32	0\\
74.33	0\\
74.34	0\\
74.35	0\\
74.36	0\\
74.37	1.73472347597681e-18\\
74.38	0\\
74.39	0\\
74.4	0\\
74.41	0\\
74.42	0\\
74.43	0\\
74.44	0\\
74.45	0\\
74.46	0\\
74.47	0\\
74.48	0\\
74.49	0\\
74.5	0\\
74.51	0\\
74.52	0\\
74.53	0\\
74.54	0\\
74.55	0\\
74.56	0\\
74.57	0\\
74.58	0\\
74.59	0\\
74.6	1.73472347597681e-18\\
74.61	0\\
74.62	1.73472347597681e-18\\
74.63	0\\
74.64	0\\
74.65	1.73472347597681e-18\\
74.66	0\\
74.67	0\\
74.68	0\\
74.69	0\\
74.7	1.73472347597681e-18\\
74.71	0\\
74.72	0\\
74.73	0\\
74.74	0\\
74.75	0\\
74.76	0\\
74.77	0\\
74.78	0\\
74.79	0\\
74.8	0\\
74.81	0\\
74.82	0\\
74.83	1.73472347597681e-18\\
74.84	0\\
74.85	0\\
74.86	0\\
74.87	1.73472347597681e-18\\
74.88	0\\
74.89	0\\
74.9	0\\
74.91	0\\
74.92	0\\
74.93	0\\
74.94	0\\
74.95	0\\
74.96	0\\
74.97	0\\
74.98	0\\
74.99	0\\
75	1.73472347597681e-18\\
75.01	0\\
75.02	0\\
75.03	1.73472347597681e-18\\
75.04	0\\
75.05	0\\
75.06	1.73472347597681e-18\\
75.07	0\\
75.08	0\\
75.09	0\\
75.1	0\\
75.11	0\\
75.12	0\\
75.13	0\\
75.14	1.73472347597681e-18\\
75.15	0\\
75.16	0\\
75.17	0\\
75.18	0\\
75.19	0\\
75.2	0\\
75.21	0\\
75.22	0\\
75.23	0\\
75.24	0\\
75.25	0\\
75.26	0\\
75.27	0\\
75.28	0\\
75.29	0\\
75.3	0\\
75.31	0\\
75.32	0\\
75.33	0\\
75.34	0\\
75.35	0\\
75.36	1.73472347597681e-18\\
75.37	0\\
75.38	0\\
75.39	0\\
75.4	0\\
75.41	0\\
75.42	0\\
75.43	0\\
75.44	1.73472347597681e-18\\
75.45	0\\
75.46	1.73472347597681e-18\\
75.47	0\\
75.48	0\\
75.49	0\\
75.5	0\\
75.51	0\\
75.52	0\\
75.53	0\\
75.54	0\\
75.55	0\\
75.56	1.73472347597681e-18\\
75.57	0\\
75.58	0\\
75.59	0\\
75.6	0\\
75.61	0\\
75.62	0\\
75.63	0\\
75.64	0\\
75.65	0\\
75.66	0\\
75.67	0\\
75.68	0\\
75.69	0\\
75.7	0\\
75.71	0\\
75.72	0\\
75.73	1.73472347597681e-18\\
75.74	0\\
75.75	0\\
75.76	0\\
75.77	0\\
75.78	0\\
75.79	0\\
75.8	0\\
75.81	0\\
75.82	0\\
75.83	0\\
75.84	0\\
75.85	0\\
75.86	0\\
75.87	0\\
75.88	0\\
75.89	0\\
75.9	0\\
75.91	0\\
75.92	0\\
75.93	1.73472347597681e-18\\
75.94	0\\
75.95	0\\
75.96	0\\
75.97	0\\
75.98	0\\
75.99	0\\
76	0\\
76.01	0\\
76.02	1.73472347597681e-18\\
76.03	0\\
76.04	0\\
76.05	0\\
76.06	0\\
76.07	0\\
76.08	0\\
76.09	0\\
76.1	0\\
76.11	0\\
76.12	0\\
76.13	1.73472347597681e-18\\
76.14	0\\
76.15	0\\
76.16	1.73472347597681e-18\\
76.17	1.73472347597681e-18\\
76.18	0\\
76.19	0\\
76.2	0\\
76.21	0\\
76.22	0\\
76.23	0\\
76.24	0\\
76.25	0\\
76.26	0\\
76.27	0\\
76.28	0\\
76.29	0\\
76.3	0\\
76.31	0\\
76.32	0\\
76.33	1.73472347597681e-18\\
76.34	0\\
76.35	0\\
76.36	0\\
76.37	0\\
76.38	1.73472347597681e-18\\
76.39	0\\
76.4	0\\
76.41	1.73472347597681e-18\\
76.42	1.73472347597681e-18\\
76.43	0\\
76.44	0\\
76.45	0\\
76.46	1.73472347597681e-18\\
76.47	0\\
76.48	0\\
76.49	0\\
76.5	0\\
76.51	0\\
76.52	1.73472347597681e-18\\
76.53	0\\
76.54	1.73472347597681e-18\\
76.55	1.73472347597681e-18\\
76.56	0\\
76.57	0\\
76.58	0\\
76.59	0\\
76.6	0\\
76.61	0\\
76.62	0\\
76.63	0\\
76.64	0\\
76.65	0\\
76.66	0\\
76.67	0\\
76.68	0\\
76.69	0\\
76.7	0\\
76.71	0\\
76.72	0\\
76.73	0\\
76.74	0\\
76.75	0\\
76.76	0\\
76.77	0\\
76.78	0\\
76.79	0\\
76.8	0\\
76.81	0\\
76.82	0\\
76.83	0\\
76.84	0\\
76.85	0\\
76.86	0\\
76.87	0\\
76.88	0\\
76.89	1.73472347597681e-18\\
76.9	0\\
76.91	0\\
76.92	0\\
76.93	0\\
76.94	1.73472347597681e-18\\
76.95	0\\
76.96	0\\
76.97	0\\
76.98	0\\
76.99	0\\
77	0\\
77.01	0\\
77.02	0\\
77.03	0\\
77.04	1.73472347597681e-18\\
77.05	0\\
77.06	0\\
77.07	0\\
77.08	0\\
77.09	0\\
77.1	0\\
77.11	0\\
77.12	0\\
77.13	0\\
77.14	0\\
77.15	0\\
77.16	1.73472347597681e-18\\
77.17	0\\
77.18	0\\
77.19	1.73472347597681e-18\\
77.2	0\\
77.21	0\\
77.22	0\\
77.23	0\\
77.24	0\\
77.25	0\\
77.26	0\\
77.27	0\\
77.28	0\\
77.29	0\\
77.3	0\\
77.31	0\\
77.32	0\\
77.33	0\\
77.34	0\\
77.35	0\\
77.36	0\\
77.37	0\\
77.38	0\\
77.39	0\\
77.4	0\\
77.41	1.73472347597681e-18\\
77.42	0\\
77.43	0\\
77.44	0\\
77.45	0\\
77.46	0\\
77.47	0\\
77.48	0\\
77.49	1.73472347597681e-18\\
77.5	0\\
77.51	0\\
77.52	0\\
77.53	0\\
77.54	1.73472347597681e-18\\
77.55	0\\
77.56	0\\
77.57	0\\
77.58	0\\
77.59	0\\
77.6	0\\
77.61	0\\
77.62	0\\
77.63	0\\
77.64	0\\
77.65	1.73472347597681e-18\\
77.66	0\\
77.67	0\\
77.68	0\\
77.69	0\\
77.7	0\\
77.71	0\\
77.72	0\\
77.73	0\\
77.74	0\\
77.75	1.73472347597681e-18\\
77.76	0\\
77.77	1.73472347597681e-18\\
77.78	0\\
77.79	1.73472347597681e-18\\
77.8	0\\
77.81	0\\
77.82	0\\
77.83	0\\
77.84	0\\
77.85	1.73472347597681e-18\\
77.86	0\\
77.87	0\\
77.88	0\\
77.89	0\\
77.9	0\\
77.91	0\\
77.92	0\\
77.93	0\\
77.94	0\\
77.95	0\\
77.96	0\\
77.97	0\\
77.98	0\\
77.99	0\\
78	0\\
78.01	0\\
78.02	0\\
78.03	0\\
78.04	1.73472347597681e-18\\
78.05	0\\
78.06	0\\
78.07	0\\
78.08	0\\
78.09	0\\
78.1	1.73472347597681e-18\\
78.11	0\\
78.12	0\\
78.13	0\\
78.14	0\\
78.15	0\\
78.16	0\\
78.17	0\\
78.18	0\\
78.19	0\\
78.2	1.73472347597681e-18\\
78.21	1.73472347597681e-18\\
78.22	0\\
78.23	0\\
78.24	0\\
78.25	1.73472347597681e-18\\
78.26	0\\
78.27	0\\
78.28	0\\
78.29	1.73472347597681e-18\\
78.3	0\\
78.31	0\\
78.32	0\\
78.33	0\\
78.34	0\\
78.35	0\\
78.36	0\\
78.37	0\\
78.38	0\\
78.39	0\\
78.4	1.73472347597681e-18\\
78.41	1.73472347597681e-18\\
78.42	0\\
78.43	0\\
78.44	0\\
78.45	0\\
78.46	0\\
78.47	0\\
78.48	0\\
78.49	0\\
78.5	0\\
78.51	0\\
78.52	0\\
78.53	0\\
78.54	1.73472347597681e-18\\
78.55	0\\
78.56	0\\
78.57	0\\
78.58	0\\
78.59	0\\
78.6	0\\
78.61	0\\
78.62	1.73472347597681e-18\\
78.63	0\\
78.64	0\\
78.65	0\\
78.66	0\\
78.67	0\\
78.68	0\\
78.69	0\\
78.7	0\\
78.71	0\\
78.72	0\\
78.73	0\\
78.74	0\\
78.75	0\\
78.76	0\\
78.77	0\\
78.78	0\\
78.79	0\\
78.8	0\\
78.81	0\\
78.82	0\\
78.83	0\\
78.84	0\\
78.85	0\\
78.86	0\\
78.87	1.73472347597681e-18\\
78.88	0\\
78.89	0\\
78.9	1.73472347597681e-18\\
78.91	0\\
78.92	1.73472347597681e-18\\
78.93	0\\
78.94	0\\
78.95	1.73472347597681e-18\\
78.96	0\\
78.97	0\\
78.98	1.73472347597681e-18\\
78.99	0\\
79	0\\
79.01	0\\
79.02	0\\
79.03	0\\
79.04	0\\
79.05	0\\
79.06	0\\
79.07	0\\
79.08	1.73472347597681e-18\\
79.09	0\\
79.1	0\\
79.11	0\\
79.12	0\\
79.13	0\\
79.14	0\\
79.15	0\\
79.16	0\\
79.17	0\\
79.18	0\\
79.19	0\\
79.2	0\\
79.21	0\\
79.22	0\\
79.23	0\\
79.24	0\\
79.25	0\\
79.26	0\\
79.27	0\\
79.28	0\\
79.29	0\\
79.3	1.73472347597681e-18\\
79.31	0\\
79.32	0\\
79.33	0\\
79.34	0\\
79.35	1.73472347597681e-18\\
79.36	0\\
79.37	0\\
79.38	0\\
79.39	0\\
79.4	0\\
79.41	0\\
79.42	0\\
79.43	0\\
79.44	0\\
79.45	0\\
79.46	1.73472347597681e-18\\
79.47	0\\
79.48	0\\
79.49	0\\
79.5	0\\
79.51	0\\
79.52	0\\
79.53	1.73472347597681e-18\\
79.54	0\\
79.55	0\\
79.56	1.73472347597681e-18\\
79.57	1.73472347597681e-18\\
79.58	0\\
79.59	1.73472347597681e-18\\
79.6	0\\
79.61	0\\
79.62	0\\
79.63	0\\
79.64	0\\
79.65	0\\
79.66	0\\
79.67	0\\
79.68	0\\
79.69	0\\
79.7	0\\
79.71	0\\
79.72	0\\
79.73	0\\
79.74	0\\
79.75	0\\
79.76	0\\
79.77	0\\
79.78	0\\
79.79	1.73472347597681e-18\\
79.8	0\\
79.81	0\\
79.82	1.73472347597681e-18\\
79.83	0\\
79.84	0\\
79.85	0\\
79.86	0\\
79.87	1.73472347597681e-18\\
79.88	0\\
79.89	0\\
79.9	0\\
79.91	0\\
79.92	0\\
79.93	0\\
79.94	1.73472347597681e-18\\
79.95	0\\
79.96	0\\
79.97	1.73472347597681e-18\\
79.98	1.73472347597681e-18\\
79.99	1.73472347597681e-18\\
80	0\\
80.01	0\\
};
\addplot [color=green,dashed]
  table[row sep=crcr]{%
80.01	0\\
80.02	0\\
80.03	1.73472347597681e-18\\
80.04	0\\
80.05	0\\
80.06	0\\
80.07	0\\
80.08	0\\
80.09	0\\
80.1	0\\
80.11	1.73472347597681e-18\\
80.12	0\\
80.13	1.73472347597681e-18\\
80.14	0\\
80.15	0\\
80.16	0\\
80.17	1.73472347597681e-18\\
80.18	1.73472347597681e-18\\
80.19	0\\
80.2	0\\
80.21	1.73472347597681e-18\\
80.22	0\\
80.23	1.73472347597681e-18\\
80.24	0\\
80.25	1.73472347597681e-18\\
80.26	0\\
80.27	0\\
80.28	0\\
80.29	0\\
80.3	0\\
80.31	0\\
80.32	0\\
80.33	0\\
80.34	0\\
80.35	0\\
80.36	0\\
80.37	1.73472347597681e-18\\
80.38	0\\
80.39	0\\
80.4	1.73472347597681e-18\\
80.41	0\\
80.42	0\\
80.43	1.73472347597681e-18\\
80.44	0\\
80.45	0\\
80.46	1.73472347597681e-18\\
80.47	0\\
80.48	0\\
80.49	0\\
80.5	0\\
80.51	0\\
80.52	0\\
80.53	0\\
80.54	0\\
80.55	0\\
80.56	1.73472347597681e-18\\
80.57	0\\
80.58	0\\
80.59	0\\
80.6	1.73472347597681e-18\\
80.61	0\\
80.62	0\\
80.63	0\\
80.64	0\\
80.65	0\\
80.66	0\\
80.67	0\\
80.68	0\\
80.69	1.73472347597681e-18\\
80.7	1.73472347597681e-18\\
80.71	0\\
80.72	0\\
80.73	1.73472347597681e-18\\
80.74	0\\
80.75	0\\
80.76	0\\
80.77	0\\
80.78	0\\
80.79	0\\
80.8	0\\
80.81	1.73472347597681e-18\\
80.82	0\\
80.83	0\\
80.84	0\\
80.85	0\\
80.86	0\\
80.87	0\\
80.88	0\\
80.89	0\\
80.9	1.73472347597681e-18\\
80.91	0\\
80.92	0\\
80.93	0\\
80.94	0\\
80.95	1.73472347597681e-18\\
80.96	0\\
80.97	0\\
80.98	0\\
80.99	0\\
81	1.73472347597681e-18\\
81.01	0\\
81.02	0\\
81.03	0\\
81.04	0\\
81.05	0\\
81.06	1.73472347597681e-18\\
81.07	1.73472347597681e-18\\
81.08	0\\
81.09	0\\
81.1	0\\
81.11	0\\
81.12	0\\
81.13	0\\
81.14	0\\
81.15	0\\
81.16	0\\
81.17	0\\
81.18	0\\
81.19	0\\
81.2	0\\
81.21	0\\
81.22	0\\
81.23	0\\
81.24	0\\
81.25	0\\
81.26	0\\
81.27	0\\
81.28	1.73472347597681e-18\\
81.29	1.73472347597681e-18\\
81.3	0\\
81.31	0\\
81.32	0\\
81.33	0\\
81.34	1.73472347597681e-18\\
81.35	0\\
81.36	0\\
81.37	0\\
81.38	0\\
81.39	0\\
81.4	1.73472347597681e-18\\
81.41	0\\
81.42	0\\
81.43	0\\
81.44	0\\
81.45	0\\
81.46	0\\
81.47	0\\
81.48	0\\
81.49	1.73472347597681e-18\\
81.5	0\\
81.51	0\\
81.52	0\\
81.53	0\\
81.54	1.73472347597681e-18\\
81.55	0\\
81.56	0\\
81.57	0\\
81.58	0\\
81.59	0\\
81.6	0\\
81.61	0\\
81.62	0\\
81.63	0\\
81.64	0\\
81.65	0\\
81.66	0\\
81.67	0\\
81.68	1.73472347597681e-18\\
81.69	0\\
81.7	0\\
81.71	1.73472347597681e-18\\
81.72	0\\
81.73	0\\
81.74	0\\
81.75	0\\
81.76	1.73472347597681e-18\\
81.77	0\\
81.78	0\\
81.79	1.73472347597681e-18\\
81.8	0\\
81.81	0\\
81.82	0\\
81.83	0\\
81.84	0\\
81.85	0\\
81.86	0\\
81.87	0\\
81.88	0\\
81.89	0\\
81.9	0\\
81.91	0\\
81.92	0\\
81.93	0\\
81.94	1.73472347597681e-18\\
81.95	0\\
81.96	1.73472347597681e-18\\
81.97	0\\
81.98	0\\
81.99	1.73472347597681e-18\\
82	0\\
82.01	1.73472347597681e-18\\
82.02	1.73472347597681e-18\\
82.03	1.73472347597681e-18\\
82.04	0\\
82.05	1.73472347597681e-18\\
82.06	0\\
82.07	1.73472347597681e-18\\
82.08	0\\
82.09	0\\
82.1	0\\
82.11	0\\
82.12	0\\
82.13	0\\
82.14	0\\
82.15	0\\
82.16	0\\
82.17	0\\
82.18	0\\
82.19	0\\
82.2	0\\
82.21	0\\
82.22	0\\
82.23	0\\
82.24	0\\
82.25	0\\
82.26	0\\
82.27	0\\
82.28	0\\
82.29	0\\
82.3	0\\
82.31	0\\
82.32	0\\
82.33	1.73472347597681e-18\\
82.34	0\\
82.35	1.73472347597681e-18\\
82.36	0\\
82.37	0\\
82.38	0\\
82.39	0\\
82.4	0\\
82.41	0\\
82.42	0\\
82.43	0\\
82.44	0\\
82.45	0\\
82.46	0\\
82.47	0\\
82.48	0\\
82.49	0\\
82.5	1.73472347597681e-18\\
82.51	0\\
82.52	0\\
82.53	0\\
82.54	0\\
82.55	0\\
82.56	0\\
82.57	0\\
82.58	0\\
82.59	0\\
82.6	0\\
82.61	0\\
82.62	0\\
82.63	1.73472347597681e-18\\
82.64	1.73472347597681e-18\\
82.65	1.73472347597681e-18\\
82.66	0\\
82.67	0\\
82.68	0\\
82.69	0\\
82.7	0\\
82.71	0\\
82.72	0\\
82.73	0\\
82.74	0\\
82.75	0\\
82.76	0\\
82.77	1.73472347597681e-18\\
82.78	0\\
82.79	0\\
82.8	0\\
82.81	0\\
82.82	0\\
82.83	0\\
82.84	0\\
82.85	1.73472347597681e-18\\
82.86	0\\
82.87	0\\
82.88	0\\
82.89	1.73472347597681e-18\\
82.9	0\\
82.91	0\\
82.92	0\\
82.93	0\\
82.94	0\\
82.95	0\\
82.96	0\\
82.97	0\\
82.98	0\\
82.99	0\\
83	0\\
83.01	0\\
83.02	0\\
83.03	0\\
83.04	0\\
83.05	1.73472347597681e-18\\
83.06	0\\
83.07	0\\
83.08	0\\
83.09	0\\
83.1	1.73472347597681e-18\\
83.11	1.73472347597681e-18\\
83.12	1.73472347597681e-18\\
83.13	0\\
83.14	0\\
83.15	0\\
83.16	0\\
83.17	0\\
83.18	0\\
83.19	0\\
83.2	1.73472347597681e-18\\
83.21	0\\
83.22	0\\
83.23	1.73472347597681e-18\\
83.24	0\\
83.25	0\\
83.26	1.73472347597681e-18\\
83.27	0\\
83.28	0\\
83.29	1.73472347597681e-18\\
83.3	0\\
83.31	0\\
83.32	0\\
83.33	0\\
83.34	0\\
83.35	0\\
83.36	0\\
83.37	0\\
83.38	0\\
83.39	0\\
83.4	0\\
83.41	0\\
83.42	0\\
83.43	0\\
83.44	0\\
83.45	0\\
83.46	0\\
83.47	1.73472347597681e-18\\
83.48	0\\
83.49	1.73472347597681e-18\\
83.5	0\\
83.51	0\\
83.52	0\\
83.53	0\\
83.54	0\\
83.55	0\\
83.56	0\\
83.57	0\\
83.58	0\\
83.59	0\\
83.6	1.73472347597681e-18\\
83.61	1.73472347597681e-18\\
83.62	0\\
83.63	1.73472347597681e-18\\
83.64	0\\
83.65	0\\
83.66	0\\
83.67	0\\
83.68	0\\
83.69	1.73472347597681e-18\\
83.7	0\\
83.71	0\\
83.72	0\\
83.73	0\\
83.74	0\\
83.75	0\\
83.76	0\\
83.77	0\\
83.78	0\\
83.79	1.73472347597681e-18\\
83.8	0\\
83.81	0\\
83.82	0\\
83.83	0\\
83.84	0\\
83.85	0\\
83.86	0\\
83.87	0\\
83.88	0\\
83.89	0\\
83.9	0\\
83.91	0\\
83.92	0\\
83.93	0\\
83.94	0\\
83.95	0\\
83.96	0\\
83.97	0\\
83.98	0\\
83.99	0\\
84	0\\
84.01	0\\
84.02	0\\
84.03	0\\
84.04	0\\
84.05	0\\
84.06	0\\
84.07	1.73472347597681e-18\\
84.08	0\\
84.09	0\\
84.1	0\\
84.11	1.73472347597681e-18\\
84.12	0\\
84.13	0\\
84.14	0\\
84.15	0\\
84.16	0\\
84.17	0\\
84.18	0\\
84.19	0\\
84.2	0\\
84.21	0\\
84.22	1.73472347597681e-18\\
84.23	0\\
84.24	0\\
84.25	0\\
84.26	1.73472347597681e-18\\
84.27	0\\
84.28	0\\
84.29	0\\
84.3	0\\
84.31	0\\
84.32	1.73472347597681e-18\\
84.33	0\\
84.34	0\\
84.35	1.73472347597681e-18\\
84.36	1.73472347597681e-18\\
84.37	0\\
84.38	0\\
84.39	0\\
84.4	0\\
84.41	1.73472347597681e-18\\
84.42	1.73472347597681e-18\\
84.43	0\\
84.44	0\\
84.45	0\\
84.46	0\\
84.47	0\\
84.48	0\\
84.49	0\\
84.5	1.73472347597681e-18\\
84.51	0\\
84.52	0\\
84.53	0\\
84.54	0\\
84.55	0\\
84.56	0\\
84.57	0\\
84.58	0\\
84.59	1.73472347597681e-18\\
84.6	0\\
84.61	0\\
84.62	0\\
84.63	0\\
84.64	0\\
84.65	0\\
84.66	1.73472347597681e-18\\
84.67	1.73472347597681e-18\\
84.68	0\\
84.69	0\\
84.7	0\\
84.71	0\\
84.72	0\\
84.73	0\\
84.74	0\\
84.75	0\\
84.76	1.73472347597681e-18\\
84.77	0\\
84.78	0\\
84.79	0\\
84.8	0\\
84.81	0\\
84.82	0\\
84.83	0\\
84.84	0\\
84.85	0\\
84.86	0\\
84.87	0\\
84.88	0\\
84.89	0\\
84.9	0\\
84.91	0\\
84.92	0\\
84.93	0\\
84.94	0\\
84.95	0\\
84.96	0\\
84.97	0\\
84.98	0\\
84.99	0\\
85	0\\
85.01	0\\
85.02	0\\
85.03	0\\
85.04	0\\
85.05	0\\
85.06	0\\
85.07	0\\
85.08	0\\
85.09	0\\
85.1	0\\
85.11	0\\
85.12	0\\
85.13	0\\
85.14	0\\
85.15	0\\
85.16	0\\
85.17	0\\
85.18	0\\
85.19	0\\
85.2	0\\
85.21	0\\
85.22	0\\
85.23	1.73472347597681e-18\\
85.24	0\\
85.25	0\\
85.26	1.73472347597681e-18\\
85.27	0\\
85.28	1.73472347597681e-18\\
85.29	0\\
85.3	0\\
85.31	1.73472347597681e-18\\
85.32	0\\
85.33	1.73472347597681e-18\\
85.34	0\\
85.35	0\\
85.36	0\\
85.37	0\\
85.38	0\\
85.39	0\\
85.4	0\\
85.41	0\\
85.42	0\\
85.43	0\\
85.44	0\\
85.45	0\\
85.46	0\\
85.47	0\\
85.48	0\\
85.49	0\\
85.5	0\\
85.51	0\\
85.52	0\\
85.53	1.73472347597681e-18\\
85.54	0\\
85.55	0\\
85.56	0\\
85.57	0\\
85.58	0\\
85.59	0\\
85.6	0\\
85.61	0\\
85.62	0\\
85.63	0\\
85.64	0\\
85.65	0\\
85.66	0\\
85.67	0\\
85.68	0\\
85.69	0\\
85.7	0\\
85.71	0\\
85.72	0\\
85.73	0\\
85.74	0\\
85.75	0\\
85.76	0\\
85.77	0\\
85.78	0\\
85.79	0\\
85.8	0\\
85.81	1.73472347597681e-18\\
85.82	0\\
85.83	0\\
85.84	0\\
85.85	0\\
85.86	0\\
85.87	0\\
85.88	0\\
85.89	0\\
85.9	0\\
85.91	0\\
85.92	0\\
85.93	0\\
85.94	0\\
85.95	0\\
85.96	0\\
85.97	1.73472347597681e-18\\
85.98	0\\
85.99	0\\
86	0\\
86.01	0\\
86.02	1.73472347597681e-18\\
86.03	1.73472347597681e-18\\
86.04	0\\
86.05	0\\
86.06	0\\
86.07	0\\
86.08	0\\
86.09	0\\
86.1	0\\
86.11	0\\
86.12	0\\
86.13	0\\
86.14	0\\
86.15	0\\
86.16	0\\
86.17	0\\
86.18	0\\
86.19	0\\
86.2	0\\
86.21	0\\
86.22	0\\
86.23	0\\
86.24	0\\
86.25	1.73472347597681e-18\\
86.26	1.73472347597681e-18\\
86.27	0\\
86.28	0\\
86.29	0\\
86.3	0\\
86.31	0\\
86.32	0\\
86.33	1.73472347597681e-18\\
86.34	0\\
86.35	0\\
86.36	0\\
86.37	0\\
86.38	0\\
86.39	0\\
86.4	0\\
86.41	0\\
86.42	0\\
86.43	0\\
86.44	0\\
86.45	0\\
86.46	0\\
86.47	0\\
86.48	0\\
86.49	0\\
86.5	0\\
86.51	0\\
86.52	0\\
86.53	0\\
86.54	0\\
86.55	1.73472347597681e-18\\
86.56	0\\
86.57	0\\
86.58	0\\
86.59	0\\
86.6	0\\
86.61	0\\
86.62	0\\
86.63	0\\
86.64	0\\
86.65	0\\
86.66	0\\
86.67	0\\
86.68	0\\
86.69	0\\
86.7	1.73472347597681e-18\\
86.71	1.73472347597681e-18\\
86.72	0\\
86.73	0\\
86.74	0\\
86.75	0\\
86.76	0\\
86.77	0\\
86.78	0\\
86.79	0\\
86.8	0\\
86.81	0\\
86.82	0\\
86.83	0\\
86.84	0\\
86.85	0\\
86.86	0\\
86.87	0\\
86.88	0\\
86.89	0\\
86.9	0\\
86.91	0\\
86.92	0\\
86.93	0\\
86.94	0\\
86.95	0\\
86.96	0\\
86.97	1.73472347597681e-18\\
86.98	0\\
86.99	0\\
87	0\\
87.01	0\\
87.02	0\\
87.03	0\\
87.04	0\\
87.05	0\\
87.06	0\\
87.07	0\\
87.08	0\\
87.09	0\\
87.1	0\\
87.11	0\\
87.12	0\\
87.13	0\\
87.14	0\\
87.15	0\\
87.16	0\\
87.17	0\\
87.18	0\\
87.19	0\\
87.2	0\\
87.21	0\\
87.22	0\\
87.23	0\\
87.24	0\\
87.25	0\\
87.26	0\\
87.27	1.73472347597681e-18\\
87.28	0\\
87.29	0\\
87.3	0\\
87.31	0\\
87.32	0\\
87.33	0\\
87.34	0\\
87.35	0\\
87.36	0\\
87.37	0\\
87.38	0\\
87.39	0\\
87.4	0\\
87.41	0\\
87.42	0\\
87.43	0\\
87.44	0\\
87.45	0\\
87.46	0\\
87.47	0\\
87.48	0\\
87.49	0\\
87.5	0\\
87.51	1.73472347597681e-18\\
87.52	0\\
87.53	0\\
87.54	0\\
87.55	0\\
87.56	0\\
87.57	0\\
87.58	0\\
87.59	0\\
87.6	0\\
87.61	0\\
87.62	1.73472347597681e-18\\
87.63	0\\
87.64	0\\
87.65	0\\
87.66	0\\
87.67	0\\
87.68	0\\
87.69	1.73472347597681e-18\\
87.7	1.73472347597681e-18\\
87.71	0\\
87.72	0\\
87.73	0\\
87.74	0\\
87.75	0\\
87.76	0\\
87.77	0\\
87.78	0\\
87.79	0\\
87.8	0\\
87.81	0\\
87.82	0\\
87.83	1.73472347597681e-18\\
87.84	0\\
87.85	0\\
87.86	0\\
87.87	0\\
87.88	0\\
87.89	0\\
87.9	0\\
87.91	0\\
87.92	0\\
87.93	0\\
87.94	0\\
87.95	0\\
87.96	0\\
87.97	0\\
87.98	0\\
87.99	0\\
88	0\\
88.01	0\\
88.02	0\\
88.03	0\\
88.04	0\\
88.05	0\\
88.06	0\\
88.07	0\\
88.08	0\\
88.09	0\\
88.1	0\\
88.11	0\\
88.12	0\\
88.13	0\\
88.14	0\\
88.15	1.73472347597681e-18\\
88.16	0\\
88.17	0\\
88.18	0\\
88.19	0\\
88.2	0\\
88.21	0\\
88.22	0\\
88.23	0\\
88.24	0\\
88.25	0\\
88.26	0\\
88.27	0\\
88.28	0\\
88.29	0\\
88.3	0\\
88.31	0\\
88.32	0\\
88.33	0\\
88.34	0\\
88.35	0\\
88.36	0\\
88.37	0\\
88.38	0\\
88.39	0\\
88.4	0\\
88.41	0\\
88.42	0\\
88.43	0\\
88.44	0\\
88.45	0\\
88.46	1.73472347597681e-18\\
88.47	0\\
88.48	0\\
88.49	0\\
88.5	0\\
88.51	0\\
88.52	0\\
88.53	0\\
88.54	0\\
88.55	0\\
88.56	0\\
88.57	0\\
88.58	0\\
88.59	0\\
88.6	0\\
88.61	0\\
88.62	0\\
88.63	0\\
88.64	0\\
88.65	0\\
88.66	0\\
88.67	0\\
88.68	0\\
88.69	0\\
88.7	0\\
88.71	0\\
88.72	0\\
88.73	0\\
88.74	0\\
88.75	0\\
88.76	0\\
88.77	0\\
88.78	0\\
88.79	0\\
88.8	0\\
88.81	0\\
88.82	0\\
88.83	0\\
88.84	0\\
88.85	0\\
88.86	0\\
88.87	0\\
88.88	0\\
88.89	0\\
88.9	0\\
88.91	0\\
88.92	0\\
88.93	0\\
88.94	0\\
88.95	0\\
88.96	0\\
88.97	0\\
88.98	0\\
88.99	0\\
89	0\\
89.01	0\\
89.02	0\\
89.03	0\\
89.04	0\\
89.05	0\\
89.06	0\\
89.07	0\\
89.08	0\\
89.09	0\\
89.1	0\\
89.11	0\\
89.12	0\\
89.13	0\\
89.14	0\\
89.15	0\\
89.16	0\\
89.17	0\\
89.18	0\\
89.19	0\\
89.2	0\\
89.21	0\\
89.22	0\\
89.23	0\\
89.24	0\\
89.25	0\\
89.26	0\\
89.27	0\\
89.28	0\\
89.29	0\\
89.3	0\\
89.31	0\\
89.32	0\\
89.33	0\\
89.34	0\\
89.35	0\\
89.36	0\\
89.37	0\\
89.38	1.73472347597681e-18\\
89.39	0\\
89.4	0\\
89.41	0\\
89.42	0\\
89.43	0\\
89.44	0\\
89.45	0\\
89.46	0\\
89.47	0\\
89.48	0\\
89.49	0\\
89.5	0\\
89.51	0\\
89.52	0\\
89.53	0\\
89.54	0\\
89.55	0\\
89.56	0\\
89.57	0\\
89.58	0\\
89.59	0\\
89.6	0\\
89.61	0\\
89.62	0\\
89.63	0\\
89.64	0\\
89.65	1.73472347597681e-18\\
89.66	0\\
89.67	0\\
89.68	1.73472347597681e-18\\
89.69	0\\
89.7	0\\
89.71	0\\
89.72	0\\
89.73	0\\
89.74	0\\
89.75	0\\
89.76	0\\
89.77	0\\
89.78	0\\
89.79	0\\
89.8	0\\
89.81	0\\
89.82	0\\
89.83	0\\
89.84	0\\
89.85	0\\
89.86	0\\
89.87	0\\
89.88	0\\
89.89	0\\
89.9	0\\
89.91	0\\
89.92	0\\
89.93	0\\
89.94	0\\
89.95	0\\
89.96	0\\
89.97	0\\
89.98	0\\
89.99	0\\
90	0\\
90.01	0\\
90.02	0\\
90.03	0\\
90.04	0\\
90.05	0\\
90.06	1.73472347597681e-18\\
90.07	0\\
90.08	0\\
90.09	0\\
90.1	0\\
90.11	0\\
90.12	0\\
90.13	0\\
90.14	0\\
90.15	0\\
90.16	0\\
90.17	0\\
90.18	1.73472347597681e-18\\
90.19	0\\
90.2	0\\
90.21	0\\
90.22	0\\
90.23	0\\
90.24	0\\
90.25	0\\
90.26	0\\
90.27	0\\
90.28	0\\
90.29	0\\
90.3	0\\
90.31	0\\
90.32	0\\
90.33	1.73472347597681e-18\\
90.34	0\\
90.35	0\\
90.36	0\\
90.37	0\\
90.38	0\\
90.39	0\\
90.4	0\\
90.41	0\\
90.42	0\\
90.43	0\\
90.44	0\\
90.45	0\\
90.46	0\\
90.47	0\\
90.48	0\\
90.49	0\\
90.5	0\\
90.51	0\\
90.52	0\\
90.53	0\\
90.54	0\\
90.55	0\\
90.56	0\\
90.57	0\\
90.58	0\\
90.59	0\\
90.6	0\\
90.61	0\\
90.62	0\\
90.63	0\\
90.64	0\\
90.65	0\\
90.66	0\\
90.67	0\\
90.68	0\\
90.69	0\\
90.7	0\\
90.71	0\\
90.72	0\\
90.73	0\\
90.74	0\\
90.75	0\\
90.76	0\\
90.77	0\\
90.78	0\\
90.79	0\\
90.8	0\\
90.81	0\\
90.82	0\\
90.83	0\\
90.84	0\\
90.85	0\\
90.86	0\\
90.87	0\\
90.88	0\\
90.89	0\\
90.9	0\\
90.91	0\\
90.92	0\\
90.93	0\\
90.94	0\\
90.95	0\\
90.96	0\\
90.97	0\\
90.98	0\\
90.99	0\\
91	0\\
91.01	0\\
91.02	0\\
91.03	0\\
91.04	0\\
91.05	0\\
91.06	0\\
91.07	0\\
91.08	0\\
91.09	0\\
91.1	0\\
91.11	0\\
91.12	0\\
91.13	0\\
91.14	0\\
91.15	0\\
91.16	0\\
91.17	0\\
91.18	0\\
91.19	0\\
91.2	0\\
91.21	0\\
91.22	0\\
91.23	0\\
91.24	0\\
91.25	0\\
91.26	0\\
91.27	0\\
91.28	0\\
91.29	0\\
91.3	0\\
91.31	0\\
91.32	0\\
91.33	0\\
91.34	0\\
91.35	0\\
91.36	0\\
91.37	0\\
91.38	0\\
91.39	0\\
91.4	0\\
91.41	0\\
91.42	0\\
91.43	0\\
91.44	0\\
91.45	0\\
91.46	0\\
91.47	0\\
91.48	0\\
91.49	0\\
91.5	0\\
91.51	0\\
91.52	0\\
91.53	0\\
91.54	0\\
91.55	0\\
91.56	0\\
91.57	0\\
91.58	0\\
91.59	0\\
91.6	0\\
91.61	0\\
91.62	0\\
91.63	0\\
91.64	0\\
91.65	0\\
91.66	0\\
91.67	0\\
91.68	0\\
91.69	0\\
91.7	0\\
91.71	0\\
91.72	0\\
91.73	0\\
91.74	0\\
91.75	0\\
91.76	0\\
91.77	0\\
91.78	0\\
91.79	0\\
91.8	0\\
91.81	0\\
91.82	0\\
91.83	0\\
91.84	0\\
91.85	0\\
91.86	0\\
91.87	0\\
91.88	0\\
91.89	0\\
91.9	0\\
91.91	0\\
91.92	0\\
91.93	0\\
91.94	0\\
91.95	0\\
91.96	0\\
91.97	0\\
91.98	0\\
91.99	0\\
92	0\\
92.01	0\\
92.02	0\\
92.03	0\\
92.04	0\\
92.05	0\\
92.06	0\\
92.07	0\\
92.08	0\\
92.09	0\\
92.1	0\\
92.11	0\\
92.12	0\\
92.13	0\\
92.14	0\\
92.15	0\\
92.16	0\\
92.17	0\\
92.18	0\\
92.19	0\\
92.2	0\\
92.21	0\\
92.22	0\\
92.23	0\\
92.24	0\\
92.25	0\\
92.26	0\\
92.27	0\\
92.28	0\\
92.29	0\\
92.3	0\\
92.31	0\\
92.32	0\\
92.33	0\\
92.34	0\\
92.35	0\\
92.36	0\\
92.37	0\\
92.38	0\\
92.39	0\\
92.4	0\\
92.41	0\\
92.42	0\\
92.43	0\\
92.44	0\\
92.45	0\\
92.46	0\\
92.47	0\\
92.48	0\\
92.49	0\\
92.5	0\\
92.51	0\\
92.52	0\\
92.53	0\\
92.54	0\\
92.55	0\\
92.56	0\\
92.57	0\\
92.58	0\\
92.59	0\\
92.6	0\\
92.61	0\\
92.62	0\\
92.63	0\\
92.64	0\\
92.65	0\\
92.66	0\\
92.67	0\\
92.68	0\\
92.69	0\\
92.7	0\\
92.71	0\\
92.72	0\\
92.73	0\\
92.74	0\\
92.75	0\\
92.76	0\\
92.77	0\\
92.78	0\\
92.79	0\\
92.8	0\\
92.81	0\\
92.82	0\\
92.83	0\\
92.84	0\\
92.85	0\\
92.86	0\\
92.87	0\\
92.88	0\\
92.89	0\\
92.9	0\\
92.91	0\\
92.92	0\\
92.93	0\\
92.94	0\\
92.95	0\\
92.96	0\\
92.97	0\\
92.98	0\\
92.99	0\\
93	0\\
93.01	0\\
93.02	0\\
93.03	0\\
93.04	0\\
93.05	0\\
93.06	0\\
93.07	0\\
93.08	0\\
93.09	0\\
93.1	0\\
93.11	0\\
93.12	0\\
93.13	0\\
93.14	0\\
93.15	0\\
93.16	0\\
93.17	0\\
93.18	0\\
93.19	0\\
93.2	0\\
93.21	0\\
93.22	0\\
93.23	0\\
93.24	0\\
93.25	0\\
93.26	0\\
93.27	0\\
93.28	0\\
93.29	0\\
93.3	0\\
93.31	0\\
93.32	0\\
93.33	0\\
93.34	0\\
93.35	0\\
93.36	0\\
93.37	0\\
93.38	0\\
93.39	0\\
93.4	0\\
93.41	0\\
93.42	0\\
93.43	0\\
93.44	0\\
93.45	0\\
93.46	0\\
93.47	0\\
93.48	0\\
93.49	0\\
93.5	0\\
93.51	0\\
93.52	0\\
93.53	0\\
93.54	0\\
93.55	0\\
93.56	0\\
93.57	0\\
93.58	0\\
93.59	0\\
93.6	0\\
93.61	0\\
93.62	0\\
93.63	0\\
93.64	0\\
93.65	0\\
93.66	0\\
93.67	0\\
93.68	0\\
93.69	0\\
93.7	0\\
93.71	0\\
93.72	0\\
93.73	0\\
93.74	0\\
93.75	0\\
93.76	0\\
93.77	0\\
93.78	0\\
93.79	0\\
93.8	0\\
93.81	0\\
93.82	0\\
93.83	0\\
93.84	0\\
93.85	0\\
93.86	0\\
93.87	0\\
93.88	0\\
93.89	0\\
93.9	0\\
93.91	0\\
93.92	0\\
93.93	0\\
93.94	0\\
93.95	0\\
93.96	0\\
93.97	0\\
93.98	0\\
93.99	0\\
94	0\\
94.01	0\\
94.02	0\\
94.03	0\\
94.04	0\\
94.05	0\\
94.06	0\\
94.07	0\\
94.08	0\\
94.09	0\\
94.1	0\\
94.11	0\\
94.12	0\\
94.13	0\\
94.14	0\\
94.15	0\\
94.16	0\\
94.17	0\\
94.18	0\\
94.19	0\\
94.2	0\\
94.21	0\\
94.22	0\\
94.23	0\\
94.24	0\\
94.25	0\\
94.26	0\\
94.27	0\\
94.28	0\\
94.29	0\\
94.3	0\\
94.31	0\\
94.32	0\\
94.33	0\\
94.34	0\\
94.35	0\\
94.36	0\\
94.37	0\\
94.38	0\\
94.39	0\\
94.4	0\\
94.41	0\\
94.42	0\\
94.43	0\\
94.44	0\\
94.45	0\\
94.46	0\\
94.47	0\\
94.48	0\\
94.49	0\\
94.5	0\\
94.51	0\\
94.52	0\\
94.53	0\\
94.54	0\\
94.55	0\\
94.56	0\\
94.57	0\\
94.58	0\\
94.59	0\\
94.6	0\\
94.61	0\\
94.62	0\\
94.63	0\\
94.64	0\\
94.65	0\\
94.66	0\\
94.67	0\\
94.68	0\\
94.69	0\\
94.7	0\\
94.71	0\\
94.72	0\\
94.73	0\\
94.74	0\\
94.75	0\\
94.76	0\\
94.77	0\\
94.78	0\\
94.79	0\\
94.8	0\\
94.81	0\\
94.82	0\\
94.83	0\\
94.84	0\\
94.85	0\\
94.86	0\\
94.87	0\\
94.88	0\\
94.89	0\\
94.9	0\\
94.91	0\\
94.92	0\\
94.93	0\\
94.94	0\\
94.95	0\\
94.96	0\\
94.97	0\\
94.98	0\\
94.99	0\\
95	0\\
95.01	0\\
95.02	0\\
95.03	0\\
95.04	0\\
95.05	0\\
95.06	0\\
95.07	0\\
95.08	0\\
95.09	0\\
95.1	0\\
95.11	0\\
95.12	0\\
95.13	0\\
95.14	0\\
95.15	0\\
95.16	0\\
95.17	0\\
95.18	0\\
95.19	0\\
95.2	0\\
95.21	0\\
95.22	0\\
95.23	0\\
95.24	0\\
95.25	0\\
95.26	0\\
95.27	0\\
95.28	0\\
95.29	0\\
95.3	0\\
95.31	0\\
95.32	0\\
95.33	0\\
95.34	0\\
95.35	0\\
95.36	0\\
95.37	0\\
95.38	0\\
95.39	0\\
95.4	0\\
95.41	0\\
95.42	0\\
95.43	0\\
95.44	0\\
95.45	0\\
95.46	0\\
95.47	0\\
95.48	0\\
95.49	0\\
95.5	0\\
95.51	0\\
95.52	0\\
95.53	0\\
95.54	0\\
95.55	0\\
95.56	0\\
95.57	0\\
95.58	0\\
95.59	0\\
95.6	0\\
95.61	0\\
95.62	0\\
95.63	0\\
95.64	0\\
95.65	0\\
95.66	0\\
95.67	0\\
95.68	0\\
95.69	0\\
95.7	0\\
95.71	0\\
95.72	0\\
95.73	0\\
95.74	0\\
95.75	0\\
95.76	0\\
95.77	0\\
95.78	0\\
95.79	0\\
95.8	0\\
95.81	0\\
95.82	0\\
95.83	0\\
95.84	0\\
95.85	0\\
95.86	0\\
95.87	0\\
95.88	0\\
95.89	0\\
95.9	0\\
95.91	0\\
95.92	0\\
95.93	0\\
95.94	0\\
95.95	0\\
95.96	0\\
95.97	0\\
95.98	0\\
95.99	0\\
96	0\\
96.01	0\\
96.02	0\\
96.03	0\\
96.04	0\\
96.05	0\\
96.06	0\\
96.07	0\\
96.08	0\\
96.09	0\\
96.1	0\\
96.11	0\\
96.12	0\\
96.13	0\\
96.14	0\\
96.15	0\\
96.16	0\\
96.17	0\\
96.18	0\\
96.19	0\\
96.2	0\\
96.21	0\\
96.22	0\\
96.23	0\\
96.24	0\\
96.25	0\\
96.26	0\\
96.27	0\\
96.28	0\\
96.29	0\\
96.3	0\\
96.31	0\\
96.32	0\\
96.33	0\\
96.34	0\\
96.35	0\\
96.36	0\\
96.37	0\\
96.38	0\\
96.39	0\\
96.4	0\\
96.41	0\\
96.42	0\\
96.43	0\\
96.44	0\\
96.45	0\\
96.46	0\\
96.47	0\\
96.48	0\\
96.49	0\\
96.5	0\\
96.51	0\\
96.52	0\\
96.53	0\\
96.54	0\\
96.55	0\\
96.56	0\\
96.57	0\\
96.58	0\\
96.59	0\\
96.6	0\\
96.61	0\\
96.62	0\\
96.63	0\\
96.64	0\\
96.65	0\\
96.66	0\\
96.67	0\\
96.68	0\\
96.69	0\\
96.7	0\\
96.71	0\\
96.72	0\\
96.73	0\\
96.74	0\\
96.75	0\\
96.76	0\\
96.77	0\\
96.78	0\\
96.79	0\\
96.8	0\\
96.81	0\\
96.82	0\\
96.83	0\\
96.84	0\\
96.85	0\\
96.86	0\\
96.87	0\\
96.88	0\\
96.89	0\\
96.9	0\\
96.91	0\\
96.92	0\\
96.93	0\\
96.94	0\\
96.95	0\\
96.96	0\\
96.97	0\\
96.98	0\\
96.99	0\\
97	0\\
97.01	0\\
97.02	0\\
97.03	0\\
97.04	0\\
97.05	0\\
97.06	0\\
97.07	0\\
97.08	0\\
97.09	0\\
97.1	0\\
97.11	0\\
97.12	0\\
97.13	0\\
97.14	0\\
97.15	0\\
97.16	0\\
97.17	0\\
97.18	0\\
97.19	0\\
97.2	0\\
97.21	0\\
97.22	0\\
97.23	0\\
97.24	0\\
97.25	0\\
97.26	0\\
97.27	0\\
97.28	0\\
97.29	0\\
97.3	0\\
97.31	0\\
97.32	0\\
97.33	0\\
97.34	0\\
97.35	0\\
97.36	0\\
97.37	0\\
97.38	0\\
97.39	0\\
97.4	0\\
97.41	0\\
97.42	0\\
97.43	0\\
97.44	0\\
97.45	0\\
97.46	0\\
97.47	0\\
97.48	0\\
97.49	0\\
97.5	0\\
97.51	0\\
97.52	0\\
97.53	0\\
97.54	0\\
97.55	0\\
97.56	0\\
97.57	0\\
97.58	0\\
97.59	0\\
97.6	0\\
97.61	0\\
97.62	0\\
97.63	0\\
97.64	0\\
97.65	0\\
97.66	0\\
97.67	0\\
97.68	0\\
97.69	0\\
97.7	0\\
97.71	0\\
97.72	0\\
97.73	0\\
97.74	0\\
97.75	0\\
97.76	0\\
97.77	0\\
97.78	0\\
97.79	0\\
97.8	0\\
97.81	0\\
97.82	0\\
97.83	0\\
97.84	0\\
97.85	0\\
97.86	0\\
97.87	0\\
97.88	0\\
97.89	0\\
97.9	0\\
97.91	0\\
97.92	0\\
97.93	0\\
97.94	0\\
97.95	0\\
97.96	0\\
97.97	0\\
97.98	0\\
97.99	0\\
98	0\\
98.01	0\\
98.02	0\\
98.03	0\\
98.04	0\\
98.05	0\\
98.06	0\\
98.07	0\\
98.08	0\\
98.09	0\\
98.1	0\\
98.11	0\\
98.12	0\\
98.13	0\\
98.14	0\\
98.15	0\\
98.16	0\\
98.17	0\\
98.18	0\\
98.19	0\\
98.2	0\\
98.21	0\\
98.22	0\\
98.23	0\\
98.24	0\\
98.25	0\\
98.26	0\\
98.27	0\\
98.28	0\\
98.29	0\\
98.3	0\\
98.31	0\\
98.32	0\\
98.33	0\\
98.34	0\\
98.35	0\\
98.36	0\\
98.37	0\\
98.38	0\\
98.39	0\\
98.4	0\\
98.41	0\\
98.42	0\\
98.43	0\\
98.44	0\\
98.45	0\\
98.46	0\\
98.47	0\\
98.48	0\\
98.49	0\\
98.5	0\\
98.51	0\\
98.52	0\\
98.53	0\\
98.54	0\\
98.55	0\\
98.56	0\\
98.57	0\\
98.58	0\\
98.59	0\\
98.6	0\\
98.61	0\\
98.62	0\\
98.63	0\\
98.64	0\\
98.65	0\\
98.66	0\\
98.67	0\\
98.68	0\\
98.69	0\\
98.7	0\\
98.71	0\\
98.72	0\\
98.73	0\\
98.74	0\\
98.75	0\\
98.76	0\\
98.77	0\\
98.78	0\\
98.79	0\\
98.8	0\\
98.81	0\\
98.82	0\\
98.83	0\\
98.84	0\\
98.85	0\\
98.86	0\\
98.87	0\\
98.88	0\\
98.89	0\\
98.9	0\\
98.91	0\\
98.92	0\\
98.93	0\\
98.94	0\\
98.95	0\\
98.96	0\\
98.97	0\\
98.98	0\\
98.99	0\\
99	0\\
99.01	0\\
99.02	0\\
99.03	0\\
99.04	0\\
99.05	0\\
99.06	0\\
99.07	0\\
99.08	0\\
99.09	0\\
99.1	0\\
99.11	0\\
99.12	0\\
99.13	0\\
99.14	0\\
99.15	0\\
99.16	0\\
99.17	0\\
99.18	0\\
99.19	0\\
99.2	0\\
99.21	0\\
99.22	0\\
99.23	0\\
99.24	0\\
99.25	0\\
99.26	0\\
99.27	0\\
99.28	0\\
99.29	0\\
99.3	0\\
99.31	0\\
99.32	0\\
99.33	0\\
99.34	0\\
99.35	0\\
99.36	0\\
99.37	0\\
99.38	0\\
99.39	0\\
99.4	0\\
99.41	0\\
99.42	0\\
99.43	0\\
99.44	0\\
99.45	0\\
99.46	0\\
99.47	0\\
99.48	0\\
99.49	0\\
99.5	0\\
99.51	0\\
99.52	0\\
99.53	0\\
99.54	0\\
99.55	0\\
99.56	0\\
99.57	0\\
99.58	0\\
99.59	0\\
99.6	0\\
99.61	0\\
99.62	0\\
99.63	0\\
99.64	0\\
99.65	0\\
99.66	0\\
99.67	0\\
99.68	0\\
99.69	0\\
99.7	0\\
99.71	0\\
99.72	0\\
99.73	0\\
99.74	0\\
99.75	0\\
99.76	0\\
99.77	0\\
99.78	0\\
99.79	0\\
99.8	0\\
99.81	0\\
99.82	0\\
99.83	0\\
99.84	0\\
99.85	0\\
99.86	0\\
99.87	0\\
99.88	0\\
99.89	0\\
99.9	0\\
99.91	0\\
99.92	0\\
99.93	0\\
99.94	0\\
99.95	0\\
99.96	0\\
99.97	0\\
99.98	0\\
99.99	0\\
100	0\\
};
\addlegendentry{$q=-4$};

\addplot [color=mycolor1,dashed,forget plot]
  table[row sep=crcr]{%
0.01	0\\
0.02	0\\
0.03	0\\
0.04	0\\
0.05	0\\
0.06	0\\
0.07	0\\
0.08	0\\
0.09	0\\
0.1	0\\
0.11	0\\
0.12	0\\
0.13	0\\
0.14	0\\
0.15	0\\
0.16	0\\
0.17	1.73472347597681e-18\\
0.18	1.73472347597681e-18\\
0.19	0\\
0.2	0\\
0.21	0\\
0.22	0\\
0.23	0\\
0.24	0\\
0.25	0\\
0.26	0\\
0.27	0\\
0.28	0\\
0.29	0\\
0.3	0\\
0.31	0\\
0.32	0\\
0.33	0\\
0.34	1.73472347597681e-18\\
0.35	1.73472347597681e-18\\
0.36	0\\
0.37	0\\
0.38	0\\
0.39	0\\
0.4	0\\
0.41	0\\
0.42	0\\
0.43	0\\
0.44	0\\
0.45	0\\
0.46	0\\
0.47	0\\
0.48	0\\
0.49	0\\
0.5	0\\
0.51	1.73472347597681e-18\\
0.52	1.73472347597681e-18\\
0.53	0\\
0.54	0\\
0.55	1.73472347597681e-18\\
0.56	0\\
0.57	0\\
0.58	0\\
0.59	0\\
0.6	1.73472347597681e-18\\
0.61	0\\
0.62	0\\
0.63	1.73472347597681e-18\\
0.64	1.73472347597681e-18\\
0.65	0\\
0.66	0\\
0.67	1.73472347597681e-18\\
0.68	0\\
0.69	0\\
0.7	0\\
0.71	0\\
0.72	0\\
0.73	0\\
0.74	0\\
0.75	0\\
0.76	0\\
0.77	1.73472347597681e-18\\
0.78	0\\
0.79	0\\
0.8	0\\
0.81	0\\
0.82	0\\
0.83	0\\
0.84	0\\
0.85	0\\
0.86	0\\
0.87	0\\
0.88	0\\
0.89	0\\
0.9	0\\
0.91	0\\
0.92	0\\
0.93	0\\
0.94	0\\
0.95	0\\
0.96	0\\
0.97	0\\
0.98	0\\
0.99	1.73472347597681e-18\\
1	0\\
1.01	1.73472347597681e-18\\
1.02	0\\
1.03	0\\
1.04	1.73472347597681e-18\\
1.05	0\\
1.06	0\\
1.07	0\\
1.08	0\\
1.09	0\\
1.1	0\\
1.11	0\\
1.12	0\\
1.13	1.73472347597681e-18\\
1.14	0\\
1.15	0\\
1.16	0\\
1.17	0\\
1.18	0\\
1.19	1.73472347597681e-18\\
1.2	0\\
1.21	0\\
1.22	1.73472347597681e-18\\
1.23	1.73472347597681e-18\\
1.24	0\\
1.25	1.73472347597681e-18\\
1.26	0\\
1.27	0\\
1.28	0\\
1.29	0\\
1.3	0\\
1.31	0\\
1.32	0\\
1.33	0\\
1.34	0\\
1.35	0\\
1.36	0\\
1.37	0\\
1.38	0\\
1.39	0\\
1.4	0\\
1.41	0\\
1.42	0\\
1.43	0\\
1.44	0\\
1.45	0\\
1.46	0\\
1.47	0\\
1.48	0\\
1.49	0\\
1.5	0\\
1.51	0\\
1.52	0\\
1.53	0\\
1.54	0\\
1.55	0\\
1.56	0\\
1.57	0\\
1.58	0\\
1.59	0\\
1.6	0\\
1.61	0\\
1.62	0\\
1.63	0\\
1.64	0\\
1.65	0\\
1.66	0\\
1.67	0\\
1.68	0\\
1.69	0\\
1.7	0\\
1.71	0\\
1.72	0\\
1.73	0\\
1.74	0\\
1.75	0\\
1.76	0\\
1.77	0\\
1.78	0\\
1.79	0\\
1.8	1.73472347597681e-18\\
1.81	0\\
1.82	0\\
1.83	0\\
1.84	0\\
1.85	0\\
1.86	0\\
1.87	0\\
1.88	0\\
1.89	0\\
1.9	0\\
1.91	0\\
1.92	0\\
1.93	0\\
1.94	0\\
1.95	0\\
1.96	1.73472347597681e-18\\
1.97	0\\
1.98	0\\
1.99	0\\
2	0\\
2.01	0\\
2.02	0\\
2.03	1.73472347597681e-18\\
2.04	0\\
2.05	1.73472347597681e-18\\
2.06	1.73472347597681e-18\\
2.07	0\\
2.08	0\\
2.09	1.73472347597681e-18\\
2.1	0\\
2.11	0\\
2.12	0\\
2.13	1.73472347597681e-18\\
2.14	0\\
2.15	0\\
2.16	0\\
2.17	0\\
2.18	0\\
2.19	0\\
2.2	0\\
2.21	0\\
2.22	0\\
2.23	0\\
2.24	1.73472347597681e-18\\
2.25	1.73472347597681e-18\\
2.26	0\\
2.27	0\\
2.28	0\\
2.29	0\\
2.3	1.73472347597681e-18\\
2.31	0\\
2.32	0\\
2.33	0\\
2.34	0\\
2.35	1.73472347597681e-18\\
2.36	0\\
2.37	0\\
2.38	0\\
2.39	0\\
2.4	0\\
2.41	0\\
2.42	0\\
2.43	0\\
2.44	0\\
2.45	1.73472347597681e-18\\
2.46	0\\
2.47	0\\
2.48	0\\
2.49	0\\
2.5	0\\
2.51	0\\
2.52	1.73472347597681e-18\\
2.53	0\\
2.54	0\\
2.55	1.73472347597681e-18\\
2.56	0\\
2.57	0\\
2.58	0\\
2.59	0\\
2.6	0\\
2.61	0\\
2.62	0\\
2.63	0\\
2.64	0\\
2.65	0\\
2.66	0\\
2.67	1.73472347597681e-18\\
2.68	0\\
2.69	0\\
2.7	0\\
2.71	0\\
2.72	0\\
2.73	0\\
2.74	0\\
2.75	0\\
2.76	0\\
2.77	0\\
2.78	0\\
2.79	0\\
2.8	0\\
2.81	0\\
2.82	0\\
2.83	0\\
2.84	0\\
2.85	0\\
2.86	0\\
2.87	0\\
2.88	0\\
2.89	0\\
2.9	0\\
2.91	0\\
2.92	0\\
2.93	0\\
2.94	0\\
2.95	0\\
2.96	1.73472347597681e-18\\
2.97	0\\
2.98	0\\
2.99	0\\
3	1.73472347597681e-18\\
3.01	0\\
3.02	0\\
3.03	0\\
3.04	0\\
3.05	0\\
3.06	0\\
3.07	0\\
3.08	0\\
3.09	0\\
3.1	0\\
3.11	0\\
3.12	0\\
3.13	1.73472347597681e-18\\
3.14	0\\
3.15	0\\
3.16	0\\
3.17	0\\
3.18	0\\
3.19	0\\
3.2	0\\
3.21	1.73472347597681e-18\\
3.22	0\\
3.23	1.73472347597681e-18\\
3.24	0\\
3.25	0\\
3.26	0\\
3.27	0\\
3.28	0\\
3.29	0\\
3.3	0\\
3.31	0\\
3.32	0\\
3.33	0\\
3.34	0\\
3.35	0\\
3.36	0\\
3.37	0\\
3.38	0\\
3.39	1.73472347597681e-18\\
3.4	0\\
3.41	1.73472347597681e-18\\
3.42	1.73472347597681e-18\\
3.43	0\\
3.44	1.73472347597681e-18\\
3.45	0\\
3.46	0\\
3.47	0\\
3.48	0\\
3.49	0\\
3.5	1.73472347597681e-18\\
3.51	0\\
3.52	0\\
3.53	0\\
3.54	0\\
3.55	0\\
3.56	0\\
3.57	0\\
3.58	0\\
3.59	1.73472347597681e-18\\
3.6	0\\
3.61	0\\
3.62	0\\
3.63	0\\
3.64	0\\
3.65	0\\
3.66	0\\
3.67	0\\
3.68	0\\
3.69	0\\
3.7	0\\
3.71	0\\
3.72	0\\
3.73	0\\
3.74	0\\
3.75	0\\
3.76	0\\
3.77	0\\
3.78	0\\
3.79	0\\
3.8	0\\
3.81	1.73472347597681e-18\\
3.82	0\\
3.83	0\\
3.84	0\\
3.85	0\\
3.86	0\\
3.87	0\\
3.88	0\\
3.89	0\\
3.9	0\\
3.91	0\\
3.92	0\\
3.93	0\\
3.94	0\\
3.95	0\\
3.96	0\\
3.97	0\\
3.98	1.73472347597681e-18\\
3.99	0\\
4	0\\
4.01	1.73472347597681e-18\\
4.02	0\\
4.03	0\\
4.04	0\\
4.05	1.73472347597681e-18\\
4.06	0\\
4.07	1.73472347597681e-18\\
4.08	0\\
4.09	0\\
4.1	0\\
4.11	0\\
4.12	0\\
4.13	0\\
4.14	0\\
4.15	1.73472347597681e-18\\
4.16	0\\
4.17	0\\
4.18	0\\
4.19	0\\
4.2	0\\
4.21	1.73472347597681e-18\\
4.22	0\\
4.23	0\\
4.24	0\\
4.25	0\\
4.26	1.73472347597681e-18\\
4.27	0\\
4.28	0\\
4.29	0\\
4.3	0\\
4.31	0\\
4.32	0\\
4.33	0\\
4.34	0\\
4.35	0\\
4.36	0\\
4.37	0\\
4.38	0\\
4.39	0\\
4.4	0\\
4.41	0\\
4.42	0\\
4.43	0\\
4.44	0\\
4.45	0\\
4.46	0\\
4.47	0\\
4.48	0\\
4.49	0\\
4.5	0\\
4.51	1.73472347597681e-18\\
4.52	0\\
4.53	0\\
4.54	0\\
4.55	0\\
4.56	0\\
4.57	0\\
4.58	0\\
4.59	0\\
4.6	0\\
4.61	0\\
4.62	1.73472347597681e-18\\
4.63	0\\
4.64	0\\
4.65	0\\
4.66	0\\
4.67	0\\
4.68	0\\
4.69	0\\
4.7	0\\
4.71	0\\
4.72	0\\
4.73	0\\
4.74	0\\
4.75	0\\
4.76	0\\
4.77	0\\
4.78	0\\
4.79	0\\
4.8	0\\
4.81	0\\
4.82	1.73472347597681e-18\\
4.83	0\\
4.84	0\\
4.85	1.73472347597681e-18\\
4.86	0\\
4.87	0\\
4.88	0\\
4.89	0\\
4.9	0\\
4.91	0\\
4.92	0\\
4.93	0\\
4.94	0\\
4.95	0\\
4.96	0\\
4.97	0\\
4.98	1.73472347597681e-18\\
4.99	0\\
5	0\\
5.01	0\\
5.02	0\\
5.03	0\\
5.04	0\\
5.05	0\\
5.06	1.73472347597681e-18\\
5.07	1.73472347597681e-18\\
5.08	0\\
5.09	0\\
5.1	0\\
5.11	1.73472347597681e-18\\
5.12	0\\
5.13	0\\
5.14	0\\
5.15	0\\
5.16	0\\
5.17	0\\
5.18	0\\
5.19	0\\
5.2	0\\
5.21	0\\
5.22	0\\
5.23	1.73472347597681e-18\\
5.24	0\\
5.25	0\\
5.26	0\\
5.27	0\\
5.28	0\\
5.29	0\\
5.3	0\\
5.31	0\\
5.32	1.73472347597681e-18\\
5.33	0\\
5.34	1.73472347597681e-18\\
5.35	1.73472347597681e-18\\
5.36	0\\
5.37	0\\
5.38	0\\
5.39	0\\
5.4	0\\
5.41	0\\
5.42	0\\
5.43	0\\
5.44	0\\
5.45	0\\
5.46	1.73472347597681e-18\\
5.47	0\\
5.48	0\\
5.49	1.73472347597681e-18\\
5.5	0\\
5.51	0\\
5.52	0\\
5.53	0\\
5.54	0\\
5.55	0\\
5.56	0\\
5.57	1.73472347597681e-18\\
5.58	0\\
5.59	0\\
5.6	0\\
5.61	0\\
5.62	0\\
5.63	1.73472347597681e-18\\
5.64	0\\
5.65	0\\
5.66	0\\
5.67	0\\
5.68	0\\
5.69	1.73472347597681e-18\\
5.7	0\\
5.71	1.73472347597681e-18\\
5.72	0\\
5.73	0\\
5.74	0\\
5.75	0\\
5.76	1.73472347597681e-18\\
5.77	0\\
5.78	0\\
5.79	0\\
5.8	0\\
5.81	0\\
5.82	0\\
5.83	1.73472347597681e-18\\
5.84	0\\
5.85	0\\
5.86	0\\
5.87	0\\
5.88	0\\
5.89	0\\
5.9	0\\
5.91	0\\
5.92	0\\
5.93	0\\
5.94	0\\
5.95	0\\
5.96	0\\
5.97	0\\
5.98	0\\
5.99	0\\
6	0\\
6.01	0\\
6.02	0\\
6.03	0\\
6.04	1.73472347597681e-18\\
6.05	0\\
6.06	0\\
6.07	0\\
6.08	0\\
6.09	0\\
6.1	0\\
6.11	1.73472347597681e-18\\
6.12	0\\
6.13	0\\
6.14	0\\
6.15	0\\
6.16	0\\
6.17	1.73472347597681e-18\\
6.18	0\\
6.19	1.73472347597681e-18\\
6.2	0\\
6.21	0\\
6.22	0\\
6.23	0\\
6.24	0\\
6.25	1.73472347597681e-18\\
6.26	0\\
6.27	0\\
6.28	1.73472347597681e-18\\
6.29	0\\
6.3	0\\
6.31	0\\
6.32	0\\
6.33	1.73472347597681e-18\\
6.34	0\\
6.35	1.73472347597681e-18\\
6.36	1.73472347597681e-18\\
6.37	0\\
6.38	1.73472347597681e-18\\
6.39	0\\
6.4	0\\
6.41	0\\
6.42	1.73472347597681e-18\\
6.43	0\\
6.44	0\\
6.45	0\\
6.46	1.73472347597681e-18\\
6.47	0\\
6.48	1.73472347597681e-18\\
6.49	0\\
6.5	0\\
6.51	0\\
6.52	0\\
6.53	0\\
6.54	0\\
6.55	0\\
6.56	1.73472347597681e-18\\
6.57	0\\
6.58	0\\
6.59	0\\
6.6	0\\
6.61	1.73472347597681e-18\\
6.62	0\\
6.63	0\\
6.64	0\\
6.65	0\\
6.66	0\\
6.67	0\\
6.68	0\\
6.69	0\\
6.7	0\\
6.71	0\\
6.72	0\\
6.73	0\\
6.74	0\\
6.75	0\\
6.76	0\\
6.77	0\\
6.78	0\\
6.79	0\\
6.8	0\\
6.81	0\\
6.82	0\\
6.83	0\\
6.84	0\\
6.85	0\\
6.86	1.73472347597681e-18\\
6.87	0\\
6.88	0\\
6.89	0\\
6.9	0\\
6.91	0\\
6.92	0\\
6.93	0\\
6.94	0\\
6.95	0\\
6.96	0\\
6.97	0\\
6.98	0\\
6.99	0\\
7	0\\
7.01	0\\
7.02	0\\
7.03	1.73472347597681e-18\\
7.04	0\\
7.05	0\\
7.06	0\\
7.07	0\\
7.08	0\\
7.09	0\\
7.1	0\\
7.11	0\\
7.12	0\\
7.13	0\\
7.14	0\\
7.15	0\\
7.16	0\\
7.17	0\\
7.18	0\\
7.19	1.73472347597681e-18\\
7.2	0\\
7.21	0\\
7.22	0\\
7.23	0\\
7.24	0\\
7.25	0\\
7.26	0\\
7.27	1.73472347597681e-18\\
7.28	0\\
7.29	0\\
7.3	0\\
7.31	0\\
7.32	0\\
7.33	0\\
7.34	0\\
7.35	0\\
7.36	0\\
7.37	1.73472347597681e-18\\
7.38	0\\
7.39	0\\
7.4	0\\
7.41	0\\
7.42	0\\
7.43	0\\
7.44	0\\
7.45	0\\
7.46	0\\
7.47	0\\
7.48	0\\
7.49	0\\
7.5	0\\
7.51	0\\
7.52	0\\
7.53	0\\
7.54	0\\
7.55	1.73472347597681e-18\\
7.56	1.73472347597681e-18\\
7.57	0\\
7.58	1.73472347597681e-18\\
7.59	0\\
7.6	0\\
7.61	0\\
7.62	0\\
7.63	0\\
7.64	0\\
7.65	0\\
7.66	0\\
7.67	0\\
7.68	0\\
7.69	0\\
7.7	0\\
7.71	0\\
7.72	0\\
7.73	0\\
7.74	0\\
7.75	0\\
7.76	0\\
7.77	0\\
7.78	0\\
7.79	0\\
7.8	0\\
7.81	0\\
7.82	0\\
7.83	0\\
7.84	0\\
7.85	1.73472347597681e-18\\
7.86	0\\
7.87	0\\
7.88	0\\
7.89	0\\
7.9	0\\
7.91	0\\
7.92	0\\
7.93	0\\
7.94	0\\
7.95	1.73472347597681e-18\\
7.96	0\\
7.97	0\\
7.98	0\\
7.99	0\\
8	0\\
8.01	0\\
8.02	0\\
8.03	1.73472347597681e-18\\
8.04	1.73472347597681e-18\\
8.05	1.73472347597681e-18\\
8.06	0\\
8.07	1.73472347597681e-18\\
8.08	0\\
8.09	1.73472347597681e-18\\
8.1	0\\
8.11	0\\
8.12	0\\
8.13	1.73472347597681e-18\\
8.14	0\\
8.15	0\\
8.16	0\\
8.17	0\\
8.18	0\\
8.19	0\\
8.2	0\\
8.21	0\\
8.22	0\\
8.23	0\\
8.24	0\\
8.25	0\\
8.26	0\\
8.27	0\\
8.28	0\\
8.29	0\\
8.3	0\\
8.31	0\\
8.32	0\\
8.33	0\\
8.34	0\\
8.35	0\\
8.36	0\\
8.37	0\\
8.38	0\\
8.39	1.73472347597681e-18\\
8.4	0\\
8.41	0\\
8.42	0\\
8.43	0\\
8.44	0\\
8.45	1.73472347597681e-18\\
8.46	0\\
8.47	0\\
8.48	0\\
8.49	0\\
8.5	0\\
8.51	0\\
8.52	0\\
8.53	0\\
8.54	0\\
8.55	0\\
8.56	0\\
8.57	0\\
8.58	0\\
8.59	0\\
8.6	0\\
8.61	0\\
8.62	0\\
8.63	0\\
8.64	1.73472347597681e-18\\
8.65	0\\
8.66	0\\
8.67	0\\
8.68	1.73472347597681e-18\\
8.69	0\\
8.7	0\\
8.71	0\\
8.72	0\\
8.73	0\\
8.74	0\\
8.75	0\\
8.76	0\\
8.77	0\\
8.78	0\\
8.79	1.73472347597681e-18\\
8.8	0\\
8.81	0\\
8.82	0\\
8.83	0\\
8.84	1.73472347597681e-18\\
8.85	0\\
8.86	1.73472347597681e-18\\
8.87	0\\
8.88	0\\
8.89	1.73472347597681e-18\\
8.9	0\\
8.91	0\\
8.92	0\\
8.93	0\\
8.94	0\\
8.95	0\\
8.96	0\\
8.97	0\\
8.98	0\\
8.99	0\\
9	0\\
9.01	0\\
9.02	0\\
9.03	1.73472347597681e-18\\
9.04	0\\
9.05	0\\
9.06	0\\
9.07	0\\
9.08	0\\
9.09	0\\
9.1	1.73472347597681e-18\\
9.11	0\\
9.12	0\\
9.13	0\\
9.14	0\\
9.15	1.73472347597681e-18\\
9.16	0\\
9.17	1.73472347597681e-18\\
9.18	0\\
9.19	1.73472347597681e-18\\
9.2	0\\
9.21	0\\
9.22	1.73472347597681e-18\\
9.23	0\\
9.24	0\\
9.25	1.73472347597681e-18\\
9.26	1.73472347597681e-18\\
9.27	0\\
9.28	0\\
9.29	0\\
9.3	0\\
9.31	1.73472347597681e-18\\
9.32	0\\
9.33	1.73472347597681e-18\\
9.34	0\\
9.35	0\\
9.36	0\\
9.37	0\\
9.38	0\\
9.39	1.73472347597681e-18\\
9.4	0\\
9.41	1.73472347597681e-18\\
9.42	0\\
9.43	0\\
9.44	0\\
9.45	0\\
9.46	0\\
9.47	0\\
9.48	0\\
9.49	0\\
9.5	0\\
9.51	1.73472347597681e-18\\
9.52	0\\
9.53	0\\
9.54	0\\
9.55	1.73472347597681e-18\\
9.56	0\\
9.57	0\\
9.58	0\\
9.59	0\\
9.6	0\\
9.61	0\\
9.62	0\\
9.63	0\\
9.64	1.73472347597681e-18\\
9.65	0\\
9.66	0\\
9.67	0\\
9.68	1.73472347597681e-18\\
9.69	0\\
9.7	0\\
9.71	0\\
9.72	1.73472347597681e-18\\
9.73	0\\
9.74	0\\
9.75	0\\
9.76	0\\
9.77	0\\
9.78	1.73472347597681e-18\\
9.79	0\\
9.8	0\\
9.81	1.73472347597681e-18\\
9.82	0\\
9.83	0\\
9.84	0\\
9.85	0\\
9.86	0\\
9.87	1.73472347597681e-18\\
9.88	0\\
9.89	0\\
9.9	1.73472347597681e-18\\
9.91	0\\
9.92	0\\
9.93	0\\
9.94	0\\
9.95	0\\
9.96	0\\
9.97	0\\
9.98	1.73472347597681e-18\\
9.99	0\\
10	0\\
10.01	0\\
10.02	0\\
10.03	0\\
10.04	0\\
10.05	0\\
10.06	0\\
10.07	0\\
10.08	0\\
10.09	0\\
10.1	0\\
10.11	0\\
10.12	0\\
10.13	0\\
10.14	0\\
10.15	0\\
10.16	0\\
10.17	0\\
10.18	0\\
10.19	0\\
10.2	0\\
10.21	0\\
10.22	0\\
10.23	0\\
10.24	0\\
10.25	0\\
10.26	0\\
10.27	0\\
10.28	0\\
10.29	0\\
10.3	0\\
10.31	0\\
10.32	0\\
10.33	0\\
10.34	0\\
10.35	0\\
10.36	1.73472347597681e-18\\
10.37	1.73472347597681e-18\\
10.38	0\\
10.39	0\\
10.4	0\\
10.41	0\\
10.42	1.73472347597681e-18\\
10.43	1.73472347597681e-18\\
10.44	0\\
10.45	0\\
10.46	0\\
10.47	0\\
10.48	0\\
10.49	0\\
10.5	0\\
10.51	0\\
10.52	0\\
10.53	0\\
10.54	0\\
10.55	0\\
10.56	0\\
10.57	0\\
10.58	0\\
10.59	1.73472347597681e-18\\
10.6	0\\
10.61	1.73472347597681e-18\\
10.62	0\\
10.63	0\\
10.64	0\\
10.65	0\\
10.66	0\\
10.67	0\\
10.68	0\\
10.69	0\\
10.7	0\\
10.71	0\\
10.72	0\\
10.73	0\\
10.74	0\\
10.75	0\\
10.76	1.73472347597681e-18\\
10.77	1.73472347597681e-18\\
10.78	0\\
10.79	1.73472347597681e-18\\
10.8	0\\
10.81	0\\
10.82	0\\
10.83	0\\
10.84	0\\
10.85	0\\
10.86	0\\
10.87	0\\
10.88	1.73472347597681e-18\\
10.89	0\\
10.9	1.73472347597681e-18\\
10.91	0\\
10.92	0\\
10.93	0\\
10.94	0\\
10.95	0\\
10.96	0\\
10.97	0\\
10.98	0\\
10.99	0\\
11	0\\
11.01	0\\
11.02	0\\
11.03	0\\
11.04	0\\
11.05	0\\
11.06	0\\
11.07	0\\
11.08	0\\
11.09	0\\
11.1	0\\
11.11	1.73472347597681e-18\\
11.12	0\\
11.13	0\\
11.14	0\\
11.15	0\\
11.16	1.73472347597681e-18\\
11.17	0\\
11.18	0\\
11.19	1.73472347597681e-18\\
11.2	0\\
11.21	0\\
11.22	0\\
11.23	0\\
11.24	1.73472347597681e-18\\
11.25	0\\
11.26	0\\
11.27	0\\
11.28	0\\
11.29	0\\
11.3	0\\
11.31	0\\
11.32	0\\
11.33	0\\
11.34	0\\
11.35	1.73472347597681e-18\\
11.36	0\\
11.37	0\\
11.38	0\\
11.39	1.73472347597681e-18\\
11.4	0\\
11.41	1.73472347597681e-18\\
11.42	0\\
11.43	0\\
11.44	0\\
11.45	0\\
11.46	0\\
11.47	0\\
11.48	0\\
11.49	0\\
11.5	0\\
11.51	0\\
11.52	0\\
11.53	0\\
11.54	0\\
11.55	0\\
11.56	0\\
11.57	0\\
11.58	1.73472347597681e-18\\
11.59	1.73472347597681e-18\\
11.6	0\\
11.61	0\\
11.62	0\\
11.63	1.73472347597681e-18\\
11.64	0\\
11.65	0\\
11.66	0\\
11.67	0\\
11.68	0\\
11.69	1.73472347597681e-18\\
11.7	0\\
11.71	0\\
11.72	0\\
11.73	0\\
11.74	0\\
11.75	0\\
11.76	0\\
11.77	1.73472347597681e-18\\
11.78	0\\
11.79	0\\
11.8	0\\
11.81	0\\
11.82	0\\
11.83	0\\
11.84	0\\
11.85	0\\
11.86	0\\
11.87	0\\
11.88	0\\
11.89	0\\
11.9	0\\
11.91	1.73472347597681e-18\\
11.92	0\\
11.93	0\\
11.94	1.73472347597681e-18\\
11.95	0\\
11.96	0\\
11.97	0\\
11.98	0\\
11.99	0\\
12	0\\
12.01	0\\
12.02	0\\
12.03	0\\
12.04	1.73472347597681e-18\\
12.05	0\\
12.06	0\\
12.07	0\\
12.08	0\\
12.09	1.73472347597681e-18\\
12.1	0\\
12.11	1.73472347597681e-18\\
12.12	0\\
12.13	0\\
12.14	0\\
12.15	0\\
12.16	0\\
12.17	1.73472347597681e-18\\
12.18	1.73472347597681e-18\\
12.19	1.73472347597681e-18\\
12.2	0\\
12.21	0\\
12.22	0\\
12.23	0\\
12.24	0\\
12.25	1.73472347597681e-18\\
12.26	0\\
12.27	0\\
12.28	1.73472347597681e-18\\
12.29	0\\
12.3	0\\
12.31	0\\
12.32	0\\
12.33	0\\
12.34	0\\
12.35	1.73472347597681e-18\\
12.36	0\\
12.37	0\\
12.38	1.73472347597681e-18\\
12.39	0\\
12.4	0\\
12.41	0\\
12.42	0\\
12.43	0\\
12.44	0\\
12.45	0\\
12.46	0\\
12.47	1.73472347597681e-18\\
12.48	0\\
12.49	0\\
12.5	0\\
12.51	0\\
12.52	0\\
12.53	0\\
12.54	0\\
12.55	0\\
12.56	0\\
12.57	0\\
12.58	0\\
12.59	0\\
12.6	0\\
12.61	0\\
12.62	0\\
12.63	0\\
12.64	0\\
12.65	0\\
12.66	0\\
12.67	0\\
12.68	0\\
12.69	0\\
12.7	0\\
12.71	0\\
12.72	1.73472347597681e-18\\
12.73	0\\
12.74	0\\
12.75	0\\
12.76	0\\
12.77	0\\
12.78	0\\
12.79	0\\
12.8	0\\
12.81	1.73472347597681e-18\\
12.82	0\\
12.83	0\\
12.84	1.73472347597681e-18\\
12.85	0\\
12.86	1.73472347597681e-18\\
12.87	0\\
12.88	0\\
12.89	0\\
12.9	0\\
12.91	0\\
12.92	0\\
12.93	0\\
12.94	0\\
12.95	1.73472347597681e-18\\
12.96	0\\
12.97	0\\
12.98	0\\
12.99	0\\
13	0\\
13.01	0\\
13.02	0\\
13.03	0\\
13.04	1.73472347597681e-18\\
13.05	0\\
13.06	0\\
13.07	1.73472347597681e-18\\
13.08	0\\
13.09	0\\
13.1	0\\
13.11	0\\
13.12	0\\
13.13	0\\
13.14	0\\
13.15	0\\
13.16	0\\
13.17	0\\
13.18	0\\
13.19	0\\
13.2	0\\
13.21	0\\
13.22	0\\
13.23	0\\
13.24	0\\
13.25	1.73472347597681e-18\\
13.26	0\\
13.27	0\\
13.28	0\\
13.29	0\\
13.3	0\\
13.31	1.73472347597681e-18\\
13.32	0\\
13.33	1.73472347597681e-18\\
13.34	1.73472347597681e-18\\
13.35	0\\
13.36	1.73472347597681e-18\\
13.37	0\\
13.38	1.73472347597681e-18\\
13.39	0\\
13.4	0\\
13.41	0\\
13.42	0\\
13.43	0\\
13.44	0\\
13.45	0\\
13.46	0\\
13.47	1.73472347597681e-18\\
13.48	0\\
13.49	0\\
13.5	0\\
13.51	0\\
13.52	0\\
13.53	0\\
13.54	0\\
13.55	0\\
13.56	0\\
13.57	0\\
13.58	0\\
13.59	0\\
13.6	0\\
13.61	1.73472347597681e-18\\
13.62	0\\
13.63	1.73472347597681e-18\\
13.64	0\\
13.65	0\\
13.66	1.73472347597681e-18\\
13.67	1.73472347597681e-18\\
13.68	0\\
13.69	0\\
13.7	1.73472347597681e-18\\
13.71	0\\
13.72	1.73472347597681e-18\\
13.73	0\\
13.74	1.73472347597681e-18\\
13.75	0\\
13.76	0\\
13.77	0\\
13.78	0\\
13.79	1.73472347597681e-18\\
13.8	0\\
13.81	1.73472347597681e-18\\
13.82	1.73472347597681e-18\\
13.83	0\\
13.84	0\\
13.85	0\\
13.86	0\\
13.87	0\\
13.88	0\\
13.89	1.73472347597681e-18\\
13.9	0\\
13.91	0\\
13.92	1.73472347597681e-18\\
13.93	0\\
13.94	0\\
13.95	0\\
13.96	0\\
13.97	0\\
13.98	0\\
13.99	1.73472347597681e-18\\
14	1.73472347597681e-18\\
14.01	1.73472347597681e-18\\
14.02	0\\
14.03	0\\
14.04	0\\
14.05	0\\
14.06	0\\
14.07	0\\
14.08	0\\
14.09	0\\
14.1	1.73472347597681e-18\\
14.11	0\\
14.12	0\\
14.13	0\\
14.14	0\\
14.15	0\\
14.16	0\\
14.17	0\\
14.18	0\\
14.19	1.73472347597681e-18\\
14.2	0\\
14.21	0\\
14.22	0\\
14.23	0\\
14.24	0\\
14.25	0\\
14.26	0\\
14.27	0\\
14.28	0\\
14.29	0\\
14.3	0\\
14.31	0\\
14.32	0\\
14.33	1.73472347597681e-18\\
14.34	0\\
14.35	0\\
14.36	0\\
14.37	0\\
14.38	0\\
14.39	1.73472347597681e-18\\
14.4	0\\
14.41	1.73472347597681e-18\\
14.42	0\\
14.43	0\\
14.44	0\\
14.45	0\\
14.46	1.73472347597681e-18\\
14.47	1.73472347597681e-18\\
14.48	0\\
14.49	0\\
14.5	0\\
14.51	0\\
14.52	0\\
14.53	0\\
14.54	0\\
14.55	0\\
14.56	0\\
14.57	0\\
14.58	0\\
14.59	0\\
14.6	0\\
14.61	0\\
14.62	0\\
14.63	1.73472347597681e-18\\
14.64	0\\
14.65	0\\
14.66	0\\
14.67	0\\
14.68	0\\
14.69	0\\
14.7	1.73472347597681e-18\\
14.71	0\\
14.72	1.73472347597681e-18\\
14.73	1.73472347597681e-18\\
14.74	0\\
14.75	0\\
14.76	0\\
14.77	0\\
14.78	0\\
14.79	0\\
14.8	0\\
14.81	0\\
14.82	0\\
14.83	0\\
14.84	1.73472347597681e-18\\
14.85	0\\
14.86	0\\
14.87	0\\
14.88	0\\
14.89	0\\
14.9	0\\
14.91	1.73472347597681e-18\\
14.92	0\\
14.93	0\\
14.94	0\\
14.95	1.73472347597681e-18\\
14.96	1.73472347597681e-18\\
14.97	0\\
14.98	0\\
14.99	0\\
15	0\\
15.01	1.73472347597681e-18\\
15.02	0\\
15.03	0\\
15.04	0\\
15.05	1.73472347597681e-18\\
15.06	0\\
15.07	0\\
15.08	0\\
15.09	0\\
15.1	0\\
15.11	0\\
15.12	0\\
15.13	0\\
15.14	0\\
15.15	0\\
15.16	0\\
15.17	0\\
15.18	0\\
15.19	0\\
15.2	1.73472347597681e-18\\
15.21	0\\
15.22	0\\
15.23	0\\
15.24	1.73472347597681e-18\\
15.25	1.73472347597681e-18\\
15.26	0\\
15.27	0\\
15.28	0\\
15.29	0\\
15.3	0\\
15.31	0\\
15.32	0\\
15.33	1.73472347597681e-18\\
15.34	0\\
15.35	0\\
15.36	0\\
15.37	0\\
15.38	0\\
15.39	0\\
15.4	0\\
15.41	0\\
15.42	0\\
15.43	0\\
15.44	0\\
15.45	0\\
15.46	0\\
15.47	0\\
15.48	0\\
15.49	0\\
15.5	0\\
15.51	0\\
15.52	0\\
15.53	0\\
15.54	0\\
15.55	0\\
15.56	0\\
15.57	0\\
15.58	1.73472347597681e-18\\
15.59	0\\
15.6	0\\
15.61	0\\
15.62	0\\
15.63	1.73472347597681e-18\\
15.64	0\\
15.65	0\\
15.66	0\\
15.67	0\\
15.68	1.73472347597681e-18\\
15.69	0\\
15.7	0\\
15.71	0\\
15.72	0\\
15.73	0\\
15.74	0\\
15.75	1.73472347597681e-18\\
15.76	0\\
15.77	0\\
15.78	0\\
15.79	0\\
15.8	0\\
15.81	0\\
15.82	0\\
15.83	0\\
15.84	0\\
15.85	0\\
15.86	0\\
15.87	0\\
15.88	0\\
15.89	0\\
15.9	0\\
15.91	0\\
15.92	0\\
15.93	0\\
15.94	0\\
15.95	0\\
15.96	0\\
15.97	0\\
15.98	0\\
15.99	0\\
16	1.73472347597681e-18\\
16.01	0\\
16.02	0\\
16.03	0\\
16.04	0\\
16.05	0\\
16.06	0\\
16.07	0\\
16.08	1.73472347597681e-18\\
16.09	0\\
16.1	0\\
16.11	0\\
16.12	0\\
16.13	0\\
16.14	0\\
16.15	0\\
16.16	0\\
16.17	0\\
16.18	0\\
16.19	0\\
16.2	0\\
16.21	0\\
16.22	0\\
16.23	0\\
16.24	0\\
16.25	0\\
16.26	0\\
16.27	0\\
16.28	0\\
16.29	0\\
16.3	0\\
16.31	0\\
16.32	0\\
16.33	0\\
16.34	0\\
16.35	1.73472347597681e-18\\
16.36	0\\
16.37	1.73472347597681e-18\\
16.38	0\\
16.39	0\\
16.4	0\\
16.41	0\\
16.42	0\\
16.43	0\\
16.44	0\\
16.45	0\\
16.46	0\\
16.47	0\\
16.48	0\\
16.49	0\\
16.5	0\\
16.51	0\\
16.52	0\\
16.53	0\\
16.54	0\\
16.55	0\\
16.56	0\\
16.57	0\\
16.58	0\\
16.59	0\\
16.6	0\\
16.61	0\\
16.62	0\\
16.63	0\\
16.64	1.73472347597681e-18\\
16.65	1.73472347597681e-18\\
16.66	0\\
16.67	0\\
16.68	1.73472347597681e-18\\
16.69	0\\
16.7	0\\
16.71	0\\
16.72	0\\
16.73	0\\
16.74	0\\
16.75	0\\
16.76	1.73472347597681e-18\\
16.77	0\\
16.78	1.73472347597681e-18\\
16.79	0\\
16.8	0\\
16.81	0\\
16.82	0\\
16.83	1.73472347597681e-18\\
16.84	1.73472347597681e-18\\
16.85	0\\
16.86	1.73472347597681e-18\\
16.87	0\\
16.88	0\\
16.89	1.73472347597681e-18\\
16.9	0\\
16.91	0\\
16.92	0\\
16.93	1.73472347597681e-18\\
16.94	0\\
16.95	1.73472347597681e-18\\
16.96	0\\
16.97	0\\
16.98	0\\
16.99	1.73472347597681e-18\\
17	1.73472347597681e-18\\
17.01	1.73472347597681e-18\\
17.02	0\\
17.03	0\\
17.04	0\\
17.05	0\\
17.06	0\\
17.07	0\\
17.08	0\\
17.09	1.73472347597681e-18\\
17.1	0\\
17.11	0\\
17.12	0\\
17.13	1.73472347597681e-18\\
17.14	0\\
17.15	0\\
17.16	0\\
17.17	1.73472347597681e-18\\
17.18	0\\
17.19	0\\
17.2	1.73472347597681e-18\\
17.21	0\\
17.22	1.73472347597681e-18\\
17.23	0\\
17.24	0\\
17.25	0\\
17.26	0\\
17.27	0\\
17.28	0\\
17.29	0\\
17.3	0\\
17.31	0\\
17.32	0\\
17.33	0\\
17.34	0\\
17.35	0\\
17.36	0\\
17.37	0\\
17.38	0\\
17.39	0\\
17.4	0\\
17.41	1.73472347597681e-18\\
17.42	1.73472347597681e-18\\
17.43	0\\
17.44	0\\
17.45	0\\
17.46	1.73472347597681e-18\\
17.47	0\\
17.48	0\\
17.49	0\\
17.5	0\\
17.51	0\\
17.52	0\\
17.53	0\\
17.54	0\\
17.55	0\\
17.56	0\\
17.57	0\\
17.58	0\\
17.59	0\\
17.6	0\\
17.61	0\\
17.62	0\\
17.63	0\\
17.64	0\\
17.65	0\\
17.66	0\\
17.67	0\\
17.68	0\\
17.69	0\\
17.7	0\\
17.71	0\\
17.72	0\\
17.73	0\\
17.74	0\\
17.75	0\\
17.76	0\\
17.77	0\\
17.78	0\\
17.79	1.73472347597681e-18\\
17.8	0\\
17.81	1.73472347597681e-18\\
17.82	0\\
17.83	1.73472347597681e-18\\
17.84	0\\
17.85	0\\
17.86	0\\
17.87	1.73472347597681e-18\\
17.88	0\\
17.89	0\\
17.9	0\\
17.91	0\\
17.92	1.73472347597681e-18\\
17.93	0\\
17.94	0\\
17.95	0\\
17.96	0\\
17.97	1.73472347597681e-18\\
17.98	0\\
17.99	0\\
18	0\\
18.01	0\\
18.02	0\\
18.03	0\\
18.04	0\\
18.05	0\\
18.06	0\\
18.07	0\\
18.08	0\\
18.09	0\\
18.1	0\\
18.11	1.73472347597681e-18\\
18.12	0\\
18.13	0\\
18.14	0\\
18.15	0\\
18.16	0\\
18.17	0\\
18.18	0\\
18.19	0\\
18.2	1.73472347597681e-18\\
18.21	0\\
18.22	0\\
18.23	0\\
18.24	0\\
18.25	0\\
18.26	0\\
18.27	1.73472347597681e-18\\
18.28	0\\
18.29	0\\
18.3	0\\
18.31	0\\
18.32	0\\
18.33	0\\
18.34	0\\
18.35	1.73472347597681e-18\\
18.36	1.73472347597681e-18\\
18.37	0\\
18.38	0\\
18.39	0\\
18.4	0\\
18.41	0\\
18.42	0\\
18.43	1.73472347597681e-18\\
18.44	0\\
18.45	0\\
18.46	0\\
18.47	0\\
18.48	0\\
18.49	0\\
18.5	0\\
18.51	0\\
18.52	0\\
18.53	0\\
18.54	0\\
18.55	0\\
18.56	0\\
18.57	0\\
18.58	0\\
18.59	0\\
18.6	0\\
18.61	0\\
18.62	0\\
18.63	0\\
18.64	0\\
18.65	0\\
18.66	0\\
18.67	0\\
18.68	0\\
18.69	0\\
18.7	0\\
18.71	0\\
18.72	0\\
18.73	1.73472347597681e-18\\
18.74	0\\
18.75	0\\
18.76	0\\
18.77	0\\
18.78	1.73472347597681e-18\\
18.79	0\\
18.8	0\\
18.81	0\\
18.82	0\\
18.83	0\\
18.84	0\\
18.85	0\\
18.86	0\\
18.87	0\\
18.88	0\\
18.89	0\\
18.9	0\\
18.91	0\\
18.92	0\\
18.93	0\\
18.94	0\\
18.95	0\\
18.96	0\\
18.97	0\\
18.98	0\\
18.99	0\\
19	0\\
19.01	1.73472347597681e-18\\
19.02	0\\
19.03	0\\
19.04	1.73472347597681e-18\\
19.05	0\\
19.06	0\\
19.07	1.73472347597681e-18\\
19.08	0\\
19.09	0\\
19.1	0\\
19.11	0\\
19.12	1.73472347597681e-18\\
19.13	0\\
19.14	0\\
19.15	0\\
19.16	0\\
19.17	0\\
19.18	0\\
19.19	1.73472347597681e-18\\
19.2	0\\
19.21	0\\
19.22	0\\
19.23	0\\
19.24	0\\
19.25	0\\
19.26	0\\
19.27	0\\
19.28	0\\
19.29	1.73472347597681e-18\\
19.3	0\\
19.31	0\\
19.32	0\\
19.33	0\\
19.34	0\\
19.35	0\\
19.36	0\\
19.37	0\\
19.38	0\\
19.39	0\\
19.4	0\\
19.41	0\\
19.42	0\\
19.43	0\\
19.44	0\\
19.45	0\\
19.46	0\\
19.47	0\\
19.48	0\\
19.49	1.73472347597681e-18\\
19.5	0\\
19.51	0\\
19.52	0\\
19.53	0\\
19.54	0\\
19.55	0\\
19.56	0\\
19.57	0\\
19.58	0\\
19.59	0\\
19.6	0\\
19.61	0\\
19.62	1.73472347597681e-18\\
19.63	1.73472347597681e-18\\
19.64	0\\
19.65	0\\
19.66	1.73472347597681e-18\\
19.67	0\\
19.68	0\\
19.69	0\\
19.7	0\\
19.71	0\\
19.72	0\\
19.73	0\\
19.74	0\\
19.75	0\\
19.76	0\\
19.77	0\\
19.78	0\\
19.79	0\\
19.8	0\\
19.81	1.73472347597681e-18\\
19.82	1.73472347597681e-18\\
19.83	0\\
19.84	0\\
19.85	0\\
19.86	1.73472347597681e-18\\
19.87	0\\
19.88	0\\
19.89	0\\
19.9	0\\
19.91	0\\
19.92	0\\
19.93	0\\
19.94	1.73472347597681e-18\\
19.95	0\\
19.96	0\\
19.97	0\\
19.98	0\\
19.99	0\\
20	0\\
20.01	0\\
20.02	1.73472347597681e-18\\
20.03	0\\
20.04	0\\
20.05	0\\
20.06	0\\
20.07	0\\
20.08	0\\
20.09	1.73472347597681e-18\\
20.1	0\\
20.11	0\\
20.12	0\\
20.13	1.73472347597681e-18\\
20.14	0\\
20.15	0\\
20.16	0\\
20.17	0\\
20.18	0\\
20.19	0\\
20.2	0\\
20.21	0\\
20.22	1.73472347597681e-18\\
20.23	0\\
20.24	0\\
20.25	0\\
20.26	0\\
20.27	0\\
20.28	0\\
20.29	0\\
20.3	0\\
20.31	0\\
20.32	0\\
20.33	0\\
20.34	0\\
20.35	0\\
20.36	0\\
20.37	0\\
20.38	1.73472347597681e-18\\
20.39	0\\
20.4	0\\
20.41	0\\
20.42	1.73472347597681e-18\\
20.43	0\\
20.44	1.73472347597681e-18\\
20.45	1.73472347597681e-18\\
20.46	0\\
20.47	0\\
20.48	0\\
20.49	0\\
20.5	0\\
20.51	0\\
20.52	0\\
20.53	0\\
20.54	0\\
20.55	0\\
20.56	0\\
20.57	0\\
20.58	0\\
20.59	0\\
20.6	1.73472347597681e-18\\
20.61	0\\
20.62	0\\
20.63	0\\
20.64	1.73472347597681e-18\\
20.65	1.73472347597681e-18\\
20.66	1.73472347597681e-18\\
20.67	1.73472347597681e-18\\
20.68	0\\
20.69	0\\
20.7	0\\
20.71	0\\
20.72	0\\
20.73	0\\
20.74	1.73472347597681e-18\\
20.75	0\\
20.76	0\\
20.77	1.73472347597681e-18\\
20.78	0\\
20.79	0\\
20.8	0\\
20.81	0\\
20.82	0\\
20.83	0\\
20.84	1.73472347597681e-18\\
20.85	0\\
20.86	0\\
20.87	0\\
20.88	0\\
20.89	0\\
20.9	0\\
20.91	0\\
20.92	0\\
20.93	0\\
20.94	0\\
20.95	0\\
20.96	0\\
20.97	0\\
20.98	0\\
20.99	0\\
21	1.73472347597681e-18\\
21.01	1.73472347597681e-18\\
21.02	0\\
21.03	0\\
21.04	1.73472347597681e-18\\
21.05	0\\
21.06	0\\
21.07	0\\
21.08	0\\
21.09	0\\
21.1	0\\
21.11	0\\
21.12	0\\
21.13	0\\
21.14	0\\
21.15	0\\
21.16	1.73472347597681e-18\\
21.17	0\\
21.18	0\\
21.19	0\\
21.2	0\\
21.21	1.73472347597681e-18\\
21.22	0\\
21.23	0\\
21.24	0\\
21.25	1.73472347597681e-18\\
21.26	1.73472347597681e-18\\
21.27	1.73472347597681e-18\\
21.28	0\\
21.29	0\\
21.3	0\\
21.31	0\\
21.32	1.73472347597681e-18\\
21.33	0\\
21.34	0\\
21.35	0\\
21.36	0\\
21.37	0\\
21.38	0\\
21.39	1.73472347597681e-18\\
21.4	0\\
21.41	0\\
21.42	0\\
21.43	0\\
21.44	0\\
21.45	0\\
21.46	1.73472347597681e-18\\
21.47	0\\
21.48	0\\
21.49	0\\
21.5	0\\
21.51	0\\
21.52	0\\
21.53	0\\
21.54	0\\
21.55	0\\
21.56	0\\
21.57	1.73472347597681e-18\\
21.58	0\\
21.59	0\\
21.6	0\\
21.61	0\\
21.62	0\\
21.63	0\\
21.64	0\\
21.65	0\\
21.66	0\\
21.67	1.73472347597681e-18\\
21.68	0\\
21.69	0\\
21.7	1.73472347597681e-18\\
21.71	0\\
21.72	0\\
21.73	1.73472347597681e-18\\
21.74	0\\
21.75	0\\
21.76	0\\
21.77	0\\
21.78	1.73472347597681e-18\\
21.79	0\\
21.8	0\\
21.81	0\\
21.82	0\\
21.83	0\\
21.84	0\\
21.85	0\\
21.86	0\\
21.87	0\\
21.88	1.73472347597681e-18\\
21.89	0\\
21.9	0\\
21.91	0\\
21.92	1.73472347597681e-18\\
21.93	0\\
21.94	0\\
21.95	0\\
21.96	0\\
21.97	0\\
21.98	0\\
21.99	0\\
22	0\\
22.01	0\\
22.02	0\\
22.03	0\\
22.04	0\\
22.05	0\\
22.06	0\\
22.07	0\\
22.08	0\\
22.09	0\\
22.1	1.73472347597681e-18\\
22.11	0\\
22.12	1.73472347597681e-18\\
22.13	0\\
22.14	1.73472347597681e-18\\
22.15	1.73472347597681e-18\\
22.16	0\\
22.17	0\\
22.18	0\\
22.19	0\\
22.2	0\\
22.21	0\\
22.22	0\\
22.23	0\\
22.24	0\\
22.25	1.73472347597681e-18\\
22.26	1.73472347597681e-18\\
22.27	0\\
22.28	1.73472347597681e-18\\
22.29	0\\
22.3	0\\
22.31	0\\
22.32	0\\
22.33	0\\
22.34	0\\
22.35	0\\
22.36	0\\
22.37	0\\
22.38	0\\
22.39	0\\
22.4	0\\
22.41	0\\
22.42	1.73472347597681e-18\\
22.43	0\\
22.44	0\\
22.45	0\\
22.46	1.73472347597681e-18\\
22.47	1.73472347597681e-18\\
22.48	1.73472347597681e-18\\
22.49	0\\
22.5	0\\
22.51	0\\
22.52	0\\
22.53	0\\
22.54	0\\
22.55	0\\
22.56	0\\
22.57	0\\
22.58	0\\
22.59	0\\
22.6	0\\
22.61	0\\
22.62	0\\
22.63	0\\
22.64	0\\
22.65	1.73472347597681e-18\\
22.66	0\\
22.67	0\\
22.68	1.73472347597681e-18\\
22.69	1.73472347597681e-18\\
22.7	0\\
22.71	0\\
22.72	1.73472347597681e-18\\
22.73	0\\
22.74	0\\
22.75	0\\
22.76	0\\
22.77	0\\
22.78	0\\
22.79	0\\
22.8	0\\
22.81	0\\
22.82	0\\
22.83	0\\
22.84	0\\
22.85	0\\
22.86	0\\
22.87	0\\
22.88	0\\
22.89	0\\
22.9	0\\
22.91	0\\
22.92	0\\
22.93	0\\
22.94	0\\
22.95	0\\
22.96	0\\
22.97	0\\
22.98	0\\
22.99	0\\
23	0\\
23.01	0\\
23.02	0\\
23.03	0\\
23.04	0\\
23.05	0\\
23.06	0\\
23.07	1.73472347597681e-18\\
23.08	0\\
23.09	0\\
23.1	0\\
23.11	0\\
23.12	0\\
23.13	1.73472347597681e-18\\
23.14	0\\
23.15	0\\
23.16	0\\
23.17	0\\
23.18	0\\
23.19	0\\
23.2	0\\
23.21	1.73472347597681e-18\\
23.22	0\\
23.23	0\\
23.24	0\\
23.25	0\\
23.26	0\\
23.27	0\\
23.28	0\\
23.29	0\\
23.3	0\\
23.31	0\\
23.32	0\\
23.33	0\\
23.34	0\\
23.35	0\\
23.36	0\\
23.37	0\\
23.38	1.73472347597681e-18\\
23.39	0\\
23.4	0\\
23.41	0\\
23.42	0\\
23.43	0\\
23.44	0\\
23.45	0\\
23.46	0\\
23.47	0\\
23.48	0\\
23.49	0\\
23.5	0\\
23.51	0\\
23.52	0\\
23.53	0\\
23.54	0\\
23.55	1.73472347597681e-18\\
23.56	0\\
23.57	0\\
23.58	0\\
23.59	0\\
23.6	0\\
23.61	0\\
23.62	0\\
23.63	0\\
23.64	0\\
23.65	0\\
23.66	0\\
23.67	0\\
23.68	0\\
23.69	0\\
23.7	0\\
23.71	0\\
23.72	0\\
23.73	0\\
23.74	0\\
23.75	0\\
23.76	0\\
23.77	0\\
23.78	0\\
23.79	0\\
23.8	1.73472347597681e-18\\
23.81	0\\
23.82	0\\
23.83	0\\
23.84	0\\
23.85	0\\
23.86	1.73472347597681e-18\\
23.87	0\\
23.88	0\\
23.89	0\\
23.9	0\\
23.91	0\\
23.92	0\\
23.93	0\\
23.94	0\\
23.95	0\\
23.96	0\\
23.97	0\\
23.98	0\\
23.99	0\\
24	0\\
24.01	1.73472347597681e-18\\
24.02	0\\
24.03	0\\
24.04	0\\
24.05	0\\
24.06	0\\
24.07	0\\
24.08	0\\
24.09	0\\
24.1	0\\
24.11	0\\
24.12	0\\
24.13	0\\
24.14	0\\
24.15	0\\
24.16	0\\
24.17	1.73472347597681e-18\\
24.18	1.73472347597681e-18\\
24.19	0\\
24.2	0\\
24.21	0\\
24.22	1.73472347597681e-18\\
24.23	0\\
24.24	0\\
24.25	0\\
24.26	0\\
24.27	0\\
24.28	0\\
24.29	0\\
24.3	0\\
24.31	0\\
24.32	0\\
24.33	1.73472347597681e-18\\
24.34	0\\
24.35	0\\
24.36	0\\
24.37	0\\
24.38	0\\
24.39	0\\
24.4	1.73472347597681e-18\\
24.41	1.73472347597681e-18\\
24.42	0\\
24.43	0\\
24.44	0\\
24.45	0\\
24.46	0\\
24.47	1.73472347597681e-18\\
24.48	1.73472347597681e-18\\
24.49	0\\
24.5	0\\
24.51	0\\
24.52	0\\
24.53	0\\
24.54	0\\
24.55	0\\
24.56	0\\
24.57	0\\
24.58	0\\
24.59	0\\
24.6	0\\
24.61	0\\
24.62	0\\
24.63	0\\
24.64	0\\
24.65	1.73472347597681e-18\\
24.66	0\\
24.67	1.73472347597681e-18\\
24.68	0\\
24.69	0\\
24.7	0\\
24.71	0\\
24.72	0\\
24.73	0\\
24.74	1.73472347597681e-18\\
24.75	0\\
24.76	0\\
24.77	0\\
24.78	0\\
24.79	0\\
24.8	0\\
24.81	0\\
24.82	0\\
24.83	0\\
24.84	0\\
24.85	0\\
24.86	0\\
24.87	0\\
24.88	0\\
24.89	0\\
24.9	0\\
24.91	0\\
24.92	1.73472347597681e-18\\
24.93	0\\
24.94	0\\
24.95	0\\
24.96	0\\
24.97	0\\
24.98	0\\
24.99	0\\
25	0\\
25.01	1.73472347597681e-18\\
25.02	1.73472347597681e-18\\
25.03	0\\
25.04	1.73472347597681e-18\\
25.05	0\\
25.06	0\\
25.07	0\\
25.08	0\\
25.09	0\\
25.1	1.73472347597681e-18\\
25.11	1.73472347597681e-18\\
25.12	0\\
25.13	0\\
25.14	0\\
25.15	0\\
25.16	1.73472347597681e-18\\
25.17	0\\
25.18	1.73472347597681e-18\\
25.19	0\\
25.2	0\\
25.21	0\\
25.22	0\\
25.23	0\\
25.24	1.73472347597681e-18\\
25.25	1.73472347597681e-18\\
25.26	0\\
25.27	0\\
25.28	0\\
25.29	0\\
25.3	0\\
25.31	0\\
25.32	0\\
25.33	0\\
25.34	0\\
25.35	0\\
25.36	0\\
25.37	1.73472347597681e-18\\
25.38	0\\
25.39	0\\
25.4	0\\
25.41	1.73472347597681e-18\\
25.42	0\\
25.43	0\\
25.44	0\\
25.45	0\\
25.46	0\\
25.47	0\\
25.48	0\\
25.49	0\\
25.5	0\\
25.51	0\\
25.52	0\\
25.53	0\\
25.54	0\\
25.55	0\\
25.56	0\\
25.57	0\\
25.58	0\\
25.59	0\\
25.6	0\\
25.61	0\\
25.62	0\\
25.63	0\\
25.64	0\\
25.65	0\\
25.66	0\\
25.67	0\\
25.68	0\\
25.69	0\\
25.7	0\\
25.71	0\\
25.72	1.73472347597681e-18\\
25.73	0\\
25.74	1.73472347597681e-18\\
25.75	0\\
25.76	1.73472347597681e-18\\
25.77	0\\
25.78	1.73472347597681e-18\\
25.79	0\\
25.8	0\\
25.81	0\\
25.82	0\\
25.83	0\\
25.84	0\\
25.85	0\\
25.86	0\\
25.87	1.73472347597681e-18\\
25.88	0\\
25.89	0\\
25.9	0\\
25.91	0\\
25.92	0\\
25.93	0\\
25.94	0\\
25.95	0\\
25.96	1.73472347597681e-18\\
25.97	0\\
25.98	0\\
25.99	0\\
26	0\\
26.01	0\\
26.02	0\\
26.03	0\\
26.04	0\\
26.05	0\\
26.06	1.73472347597681e-18\\
26.07	0\\
26.08	0\\
26.09	1.73472347597681e-18\\
26.1	0\\
26.11	0\\
26.12	0\\
26.13	0\\
26.14	0\\
26.15	0\\
26.16	0\\
26.17	0\\
26.18	0\\
26.19	0\\
26.2	0\\
26.21	0\\
26.22	0\\
26.23	0\\
26.24	0\\
26.25	0\\
26.26	0\\
26.27	0\\
26.28	0\\
26.29	0\\
26.3	0\\
26.31	0\\
26.32	0\\
26.33	0\\
26.34	0\\
26.35	0\\
26.36	0\\
26.37	0\\
26.38	0\\
26.39	1.73472347597681e-18\\
26.4	0\\
26.41	0\\
26.42	0\\
26.43	0\\
26.44	0\\
26.45	0\\
26.46	0\\
26.47	0\\
26.48	0\\
26.49	0\\
26.5	0\\
26.51	0\\
26.52	0\\
26.53	0\\
26.54	0\\
26.55	0\\
26.56	0\\
26.57	0\\
26.58	0\\
26.59	0\\
26.6	0\\
26.61	0\\
26.62	0\\
26.63	0\\
26.64	0\\
26.65	0\\
26.66	0\\
26.67	0\\
26.68	0\\
26.69	0\\
26.7	0\\
26.71	1.73472347597681e-18\\
26.72	0\\
26.73	0\\
26.74	0\\
26.75	0\\
26.76	0\\
26.77	0\\
26.78	0\\
26.79	0\\
26.8	1.73472347597681e-18\\
26.81	0\\
26.82	0\\
26.83	0\\
26.84	1.73472347597681e-18\\
26.85	0\\
26.86	1.73472347597681e-18\\
26.87	0\\
26.88	0\\
26.89	0\\
26.9	0\\
26.91	1.73472347597681e-18\\
26.92	0\\
26.93	0\\
26.94	0\\
26.95	0\\
26.96	0\\
26.97	0\\
26.98	1.73472347597681e-18\\
26.99	0\\
27	0\\
27.01	0\\
27.02	0\\
27.03	0\\
27.04	0\\
27.05	0\\
27.06	0\\
27.07	0\\
27.08	0\\
27.09	0\\
27.1	0\\
27.11	1.73472347597681e-18\\
27.12	0\\
27.13	0\\
27.14	0\\
27.15	0\\
27.16	0\\
27.17	0\\
27.18	0\\
27.19	0\\
27.2	0\\
27.21	0\\
27.22	0\\
27.23	0\\
27.24	0\\
27.25	0\\
27.26	0\\
27.27	0\\
27.28	0\\
27.29	0\\
27.3	0\\
27.31	1.73472347597681e-18\\
27.32	0\\
27.33	0\\
27.34	0\\
27.35	0\\
27.36	0\\
27.37	0\\
27.38	0\\
27.39	0\\
27.4	0\\
27.41	0\\
27.42	0\\
27.43	1.73472347597681e-18\\
27.44	0\\
27.45	0\\
27.46	0\\
27.47	0\\
27.48	0\\
27.49	1.73472347597681e-18\\
27.5	1.73472347597681e-18\\
27.51	0\\
27.52	0\\
27.53	0\\
27.54	0\\
27.55	0\\
27.56	0\\
27.57	0\\
27.58	0\\
27.59	0\\
27.6	0\\
27.61	0\\
27.62	0\\
27.63	1.73472347597681e-18\\
27.64	0\\
27.65	0\\
27.66	0\\
27.67	0\\
27.68	0\\
27.69	0\\
27.7	0\\
27.71	0\\
27.72	0\\
27.73	0\\
27.74	0\\
27.75	1.73472347597681e-18\\
27.76	0\\
27.77	0\\
27.78	0\\
27.79	0\\
27.8	0\\
27.81	0\\
27.82	0\\
27.83	0\\
27.84	0\\
27.85	1.73472347597681e-18\\
27.86	0\\
27.87	1.73472347597681e-18\\
27.88	0\\
27.89	0\\
27.9	0\\
27.91	0\\
27.92	0\\
27.93	0\\
27.94	0\\
27.95	0\\
27.96	0\\
27.97	0\\
27.98	0\\
27.99	0\\
28	1.73472347597681e-18\\
28.01	0\\
28.02	0\\
28.03	1.73472347597681e-18\\
28.04	1.73472347597681e-18\\
28.05	0\\
28.06	0\\
28.07	0\\
28.08	0\\
28.09	0\\
28.1	0\\
28.11	1.73472347597681e-18\\
28.12	0\\
28.13	0\\
28.14	0\\
28.15	0\\
28.16	0\\
28.17	0\\
28.18	0\\
28.19	0\\
28.2	0\\
28.21	0\\
28.22	0\\
28.23	0\\
28.24	0\\
28.25	0\\
28.26	0\\
28.27	0\\
28.28	0\\
28.29	0\\
28.3	0\\
28.31	0\\
28.32	0\\
28.33	0\\
28.34	0\\
28.35	1.73472347597681e-18\\
28.36	0\\
28.37	0\\
28.38	0\\
28.39	0\\
28.4	0\\
28.41	0\\
28.42	0\\
28.43	0\\
28.44	0\\
28.45	0\\
28.46	0\\
28.47	0\\
28.48	0\\
28.49	0\\
28.5	0\\
28.51	0\\
28.52	0\\
28.53	0\\
28.54	1.73472347597681e-18\\
28.55	0\\
28.56	0\\
28.57	0\\
28.58	1.73472347597681e-18\\
28.59	1.73472347597681e-18\\
28.6	0\\
28.61	0\\
28.62	0\\
28.63	0\\
28.64	0\\
28.65	0\\
28.66	0\\
28.67	0\\
28.68	0\\
28.69	1.73472347597681e-18\\
28.7	0\\
28.71	0\\
28.72	0\\
28.73	0\\
28.74	0\\
28.75	0\\
28.76	0\\
28.77	0\\
28.78	0\\
28.79	0\\
28.8	0\\
28.81	1.73472347597681e-18\\
28.82	0\\
28.83	0\\
28.84	0\\
28.85	0\\
28.86	0\\
28.87	0\\
28.88	0\\
28.89	0\\
28.9	0\\
28.91	1.73472347597681e-18\\
28.92	1.73472347597681e-18\\
28.93	0\\
28.94	0\\
28.95	0\\
28.96	0\\
28.97	0\\
28.98	0\\
28.99	0\\
29	0\\
29.01	0\\
29.02	0\\
29.03	0\\
29.04	0\\
29.05	0\\
29.06	0\\
29.07	0\\
29.08	0\\
29.09	0\\
29.1	0\\
29.11	0\\
29.12	0\\
29.13	0\\
29.14	0\\
29.15	0\\
29.16	0\\
29.17	0\\
29.18	1.73472347597681e-18\\
29.19	0\\
29.2	0\\
29.21	0\\
29.22	0\\
29.23	0\\
29.24	0\\
29.25	0\\
29.26	0\\
29.27	0\\
29.28	0\\
29.29	0\\
29.3	0\\
29.31	1.73472347597681e-18\\
29.32	0\\
29.33	0\\
29.34	0\\
29.35	0\\
29.36	0\\
29.37	0\\
29.38	0\\
29.39	0\\
29.4	0\\
29.41	0\\
29.42	0\\
29.43	0\\
29.44	0\\
29.45	0\\
29.46	0\\
29.47	0\\
29.48	0\\
29.49	0\\
29.5	0\\
29.51	0\\
29.52	0\\
29.53	1.73472347597681e-18\\
29.54	0\\
29.55	0\\
29.56	0\\
29.57	0\\
29.58	0\\
29.59	0\\
29.6	0\\
29.61	0\\
29.62	0\\
29.63	1.73472347597681e-18\\
29.64	0\\
29.65	1.73472347597681e-18\\
29.66	1.73472347597681e-18\\
29.67	0\\
29.68	1.73472347597681e-18\\
29.69	0\\
29.7	0\\
29.71	0\\
29.72	1.73472347597681e-18\\
29.73	0\\
29.74	0\\
29.75	0\\
29.76	0\\
29.77	0\\
29.78	0\\
29.79	0\\
29.8	1.73472347597681e-18\\
29.81	0\\
29.82	0\\
29.83	0\\
29.84	0\\
29.85	0\\
29.86	0\\
29.87	1.73472347597681e-18\\
29.88	0\\
29.89	0\\
29.9	0\\
29.91	0\\
29.92	0\\
29.93	1.73472347597681e-18\\
29.94	0\\
29.95	0\\
29.96	0\\
29.97	0\\
29.98	0\\
29.99	0\\
30	1.73472347597681e-18\\
30.01	1.73472347597681e-18\\
30.02	0\\
30.03	0\\
30.04	0\\
30.05	0\\
30.06	0\\
30.07	0\\
30.08	0\\
30.09	0\\
30.1	0\\
30.11	1.73472347597681e-18\\
30.12	0\\
30.13	0\\
30.14	0\\
30.15	0\\
30.16	0\\
30.17	0\\
30.18	0\\
30.19	0\\
30.2	0\\
30.21	0\\
30.22	0\\
30.23	0\\
30.24	0\\
30.25	0\\
30.26	0\\
30.27	0\\
30.28	0\\
30.29	0\\
30.3	1.73472347597681e-18\\
30.31	0\\
30.32	0\\
30.33	0\\
30.34	0\\
30.35	0\\
30.36	0\\
30.37	0\\
30.38	1.73472347597681e-18\\
30.39	0\\
30.4	1.73472347597681e-18\\
30.41	0\\
30.42	0\\
30.43	0\\
30.44	0\\
30.45	0\\
30.46	0\\
30.47	0\\
30.48	0\\
30.49	0\\
30.5	0\\
30.51	0\\
30.52	0\\
30.53	0\\
30.54	0\\
30.55	0\\
30.56	0\\
30.57	0\\
30.58	0\\
30.59	0\\
30.6	0\\
30.61	0\\
30.62	0\\
30.63	0\\
30.64	0\\
30.65	0\\
30.66	0\\
30.67	0\\
30.68	1.73472347597681e-18\\
30.69	0\\
30.7	0\\
30.71	0\\
30.72	1.73472347597681e-18\\
30.73	0\\
30.74	0\\
30.75	0\\
30.76	0\\
30.77	0\\
30.78	0\\
30.79	0\\
30.8	1.73472347597681e-18\\
30.81	1.73472347597681e-18\\
30.82	0\\
30.83	1.73472347597681e-18\\
30.84	0\\
30.85	1.73472347597681e-18\\
30.86	0\\
30.87	0\\
30.88	0\\
30.89	0\\
30.9	0\\
30.91	1.73472347597681e-18\\
30.92	0\\
30.93	0\\
30.94	1.73472347597681e-18\\
30.95	0\\
30.96	0\\
30.97	1.73472347597681e-18\\
30.98	0\\
30.99	1.73472347597681e-18\\
31	0\\
31.01	0\\
31.02	0\\
31.03	0\\
31.04	0\\
31.05	0\\
31.06	0\\
31.07	0\\
31.08	1.73472347597681e-18\\
31.09	0\\
31.1	0\\
31.11	0\\
31.12	0\\
31.13	0\\
31.14	0\\
31.15	0\\
31.16	0\\
31.17	0\\
31.18	0\\
31.19	0\\
31.2	0\\
31.21	1.73472347597681e-18\\
31.22	1.73472347597681e-18\\
31.23	0\\
31.24	0\\
31.25	0\\
31.26	1.73472347597681e-18\\
31.27	0\\
31.28	0\\
31.29	0\\
31.3	0\\
31.31	0\\
31.32	0\\
31.33	0\\
31.34	0\\
31.35	0\\
31.36	0\\
31.37	0\\
31.38	0\\
31.39	1.73472347597681e-18\\
31.4	0\\
31.41	0\\
31.42	1.73472347597681e-18\\
31.43	1.73472347597681e-18\\
31.44	1.73472347597681e-18\\
31.45	0\\
31.46	0\\
31.47	0\\
31.48	0\\
31.49	0\\
31.5	0\\
31.51	0\\
31.52	0\\
31.53	0\\
31.54	0\\
31.55	1.73472347597681e-18\\
31.56	0\\
31.57	0\\
31.58	0\\
31.59	0\\
31.6	0\\
31.61	0\\
31.62	0\\
31.63	0\\
31.64	0\\
31.65	0\\
31.66	0\\
31.67	1.73472347597681e-18\\
31.68	0\\
31.69	0\\
31.7	0\\
31.71	0\\
31.72	0\\
31.73	0\\
31.74	0\\
31.75	0\\
31.76	0\\
31.77	0\\
31.78	0\\
31.79	0\\
31.8	0\\
31.81	0\\
31.82	0\\
31.83	1.73472347597681e-18\\
31.84	0\\
31.85	0\\
31.86	1.73472347597681e-18\\
31.87	0\\
31.88	1.73472347597681e-18\\
31.89	1.73472347597681e-18\\
31.9	0\\
31.91	0\\
31.92	0\\
31.93	0\\
31.94	1.73472347597681e-18\\
31.95	1.73472347597681e-18\\
31.96	0\\
31.97	1.73472347597681e-18\\
31.98	0\\
31.99	0\\
32	0\\
32.01	0\\
32.02	0\\
32.03	0\\
32.04	0\\
32.05	0\\
32.06	0\\
32.07	0\\
32.08	0\\
32.09	0\\
32.1	0\\
32.11	0\\
32.12	0\\
32.13	0\\
32.14	0\\
32.15	0\\
32.16	0\\
32.17	0\\
32.18	0\\
32.19	0\\
32.2	0\\
32.21	0\\
32.22	0\\
32.23	0\\
32.24	0\\
32.25	0\\
32.26	0\\
32.27	0\\
32.28	0\\
32.29	0\\
32.3	0\\
32.31	0\\
32.32	0\\
32.33	0\\
32.34	0\\
32.35	0\\
32.36	0\\
32.37	0\\
32.38	0\\
32.39	0\\
32.4	0\\
32.41	0\\
32.42	0\\
32.43	0\\
32.44	1.73472347597681e-18\\
32.45	1.73472347597681e-18\\
32.46	0\\
32.47	0\\
32.48	0\\
32.49	0\\
32.5	0\\
32.51	0\\
32.52	0\\
32.53	0\\
32.54	0\\
32.55	0\\
32.56	0\\
32.57	0\\
32.58	0\\
32.59	0\\
32.6	0\\
32.61	1.73472347597681e-18\\
32.62	0\\
32.63	0\\
32.64	0\\
32.65	0\\
32.66	0\\
32.67	0\\
32.68	0\\
32.69	0\\
32.7	0\\
32.71	0\\
32.72	0\\
32.73	0\\
32.74	0\\
32.75	0\\
32.76	0\\
32.77	0\\
32.78	0\\
32.79	0\\
32.8	0\\
32.81	0\\
32.82	0\\
32.83	1.73472347597681e-18\\
32.84	1.73472347597681e-18\\
32.85	0\\
32.86	0\\
32.87	0\\
32.88	0\\
32.89	1.73472347597681e-18\\
32.9	0\\
32.91	0\\
32.92	0\\
32.93	0\\
32.94	0\\
32.95	0\\
32.96	0\\
32.97	1.73472347597681e-18\\
32.98	0\\
32.99	0\\
33	0\\
33.01	0\\
33.02	0\\
33.03	0\\
33.04	0\\
33.05	0\\
33.06	0\\
33.07	0\\
33.08	0\\
33.09	0\\
33.1	0\\
33.11	1.73472347597681e-18\\
33.12	0\\
33.13	0\\
33.14	0\\
33.15	0\\
33.16	1.73472347597681e-18\\
33.17	0\\
33.18	0\\
33.19	0\\
33.2	0\\
33.21	0\\
33.22	0\\
33.23	0\\
33.24	0\\
33.25	1.73472347597681e-18\\
33.26	0\\
33.27	1.73472347597681e-18\\
33.28	1.73472347597681e-18\\
33.29	0\\
33.3	1.73472347597681e-18\\
33.31	0\\
33.32	0\\
33.33	0\\
33.34	0\\
33.35	0\\
33.36	1.73472347597681e-18\\
33.37	0\\
33.38	1.73472347597681e-18\\
33.39	0\\
33.4	0\\
33.41	0\\
33.42	0\\
33.43	0\\
33.44	0\\
33.45	0\\
33.46	0\\
33.47	0\\
33.48	0\\
33.49	0\\
33.5	0\\
33.51	0\\
33.52	0\\
33.53	0\\
33.54	0\\
33.55	0\\
33.56	0\\
33.57	0\\
33.58	1.73472347597681e-18\\
33.59	0\\
33.6	1.73472347597681e-18\\
33.61	0\\
33.62	0\\
33.63	0\\
33.64	0\\
33.65	0\\
33.66	0\\
33.67	0\\
33.68	0\\
33.69	0\\
33.7	0\\
33.71	0\\
33.72	0\\
33.73	0\\
33.74	0\\
33.75	0\\
33.76	1.73472347597681e-18\\
33.77	0\\
33.78	0\\
33.79	0\\
33.8	0\\
33.81	0\\
33.82	0\\
33.83	0\\
33.84	0\\
33.85	0\\
33.86	0\\
33.87	0\\
33.88	1.73472347597681e-18\\
33.89	0\\
33.9	0\\
33.91	0\\
33.92	0\\
33.93	0\\
33.94	1.73472347597681e-18\\
33.95	0\\
33.96	0\\
33.97	0\\
33.98	0\\
33.99	1.73472347597681e-18\\
34	0\\
34.01	0\\
34.02	0\\
34.03	0\\
34.04	0\\
34.05	0\\
34.06	0\\
34.07	0\\
34.08	0\\
34.09	0\\
34.1	0\\
34.11	0\\
34.12	0\\
34.13	0\\
34.14	0\\
34.15	0\\
34.16	1.73472347597681e-18\\
34.17	0\\
34.18	0\\
34.19	0\\
34.2	0\\
34.21	0\\
34.22	0\\
34.23	0\\
34.24	0\\
34.25	0\\
34.26	1.73472347597681e-18\\
34.27	0\\
34.28	0\\
34.29	0\\
34.3	0\\
34.31	0\\
34.32	0\\
34.33	0\\
34.34	1.73472347597681e-18\\
34.35	0\\
34.36	0\\
34.37	0\\
34.38	0\\
34.39	0\\
34.4	0\\
34.41	0\\
34.42	1.73472347597681e-18\\
34.43	0\\
34.44	0\\
34.45	0\\
34.46	0\\
34.47	0\\
34.48	0\\
34.49	1.73472347597681e-18\\
34.5	0\\
34.51	0\\
34.52	0\\
34.53	0\\
34.54	0\\
34.55	0\\
34.56	0\\
34.57	0\\
34.58	1.73472347597681e-18\\
34.59	0\\
34.6	0\\
34.61	0\\
34.62	0\\
34.63	0\\
34.64	0\\
34.65	0\\
34.66	0\\
34.67	0\\
34.68	0\\
34.69	0\\
34.7	1.73472347597681e-18\\
34.71	0\\
34.72	0\\
34.73	0\\
34.74	1.73472347597681e-18\\
34.75	1.73472347597681e-18\\
34.76	0\\
34.77	1.73472347597681e-18\\
34.78	1.73472347597681e-18\\
34.79	0\\
34.8	0\\
34.81	0\\
34.82	0\\
34.83	0\\
34.84	0\\
34.85	0\\
34.86	0\\
34.87	0\\
34.88	0\\
34.89	0\\
34.9	0\\
34.91	0\\
34.92	0\\
34.93	0\\
34.94	0\\
34.95	1.73472347597681e-18\\
34.96	1.73472347597681e-18\\
34.97	1.73472347597681e-18\\
34.98	0\\
34.99	0\\
35	0\\
35.01	0\\
35.02	0\\
35.03	0\\
35.04	0\\
35.05	0\\
35.06	0\\
35.07	1.73472347597681e-18\\
35.08	0\\
35.09	1.73472347597681e-18\\
35.1	0\\
35.11	0\\
35.12	0\\
35.13	1.73472347597681e-18\\
35.14	0\\
35.15	0\\
35.16	1.73472347597681e-18\\
35.17	0\\
35.18	0\\
35.19	0\\
35.2	0\\
35.21	0\\
35.22	0\\
35.23	0\\
35.24	0\\
35.25	0\\
35.26	0\\
35.27	0\\
35.28	0\\
35.29	0\\
35.3	0\\
35.31	0\\
35.32	0\\
35.33	0\\
35.34	0\\
35.35	0\\
35.36	0\\
35.37	0\\
35.38	0\\
35.39	0\\
35.4	0\\
35.41	0\\
35.42	0\\
35.43	1.73472347597681e-18\\
35.44	0\\
35.45	0\\
35.46	0\\
35.47	0\\
35.48	0\\
35.49	0\\
35.5	0\\
35.51	0\\
35.52	0\\
35.53	0\\
35.54	0\\
35.55	0\\
35.56	0\\
35.57	0\\
35.58	0\\
35.59	0\\
35.6	0\\
35.61	0\\
35.62	0\\
35.63	0\\
35.64	0\\
35.65	0\\
35.66	0\\
35.67	0\\
35.68	1.73472347597681e-18\\
35.69	1.73472347597681e-18\\
35.7	0\\
35.71	0\\
35.72	0\\
35.73	0\\
35.74	0\\
35.75	0\\
35.76	0\\
35.77	0\\
35.78	0\\
35.79	0\\
35.8	0\\
35.81	0\\
35.82	0\\
35.83	0\\
35.84	0\\
35.85	1.73472347597681e-18\\
35.86	0\\
35.87	0\\
35.88	0\\
35.89	0\\
35.9	0\\
35.91	1.73472347597681e-18\\
35.92	0\\
35.93	0\\
35.94	0\\
35.95	0\\
35.96	0\\
35.97	0\\
35.98	1.73472347597681e-18\\
35.99	0\\
36	0\\
36.01	0\\
36.02	0\\
36.03	0\\
36.04	0\\
36.05	0\\
36.06	0\\
36.07	1.73472347597681e-18\\
36.08	1.73472347597681e-18\\
36.09	0\\
36.1	0\\
36.11	0\\
36.12	0\\
36.13	0\\
36.14	0\\
36.15	0\\
36.16	1.73472347597681e-18\\
36.17	0\\
36.18	0\\
36.19	0\\
36.2	0\\
36.21	0\\
36.22	0\\
36.23	0\\
36.24	0\\
36.25	1.73472347597681e-18\\
36.26	0\\
36.27	0\\
36.28	0\\
36.29	0\\
36.3	0\\
36.31	0\\
36.32	1.73472347597681e-18\\
36.33	0\\
36.34	0\\
36.35	0\\
36.36	0\\
36.37	0\\
36.38	1.73472347597681e-18\\
36.39	0\\
36.4	0\\
36.41	0\\
36.42	0\\
36.43	0\\
36.44	1.73472347597681e-18\\
36.45	0\\
36.46	0\\
36.47	0\\
36.48	0\\
36.49	0\\
36.5	1.73472347597681e-18\\
36.51	0\\
36.52	0\\
36.53	0\\
36.54	0\\
36.55	0\\
36.56	0\\
36.57	0\\
36.58	1.73472347597681e-18\\
36.59	0\\
36.6	1.73472347597681e-18\\
36.61	0\\
36.62	0\\
36.63	1.73472347597681e-18\\
36.64	0\\
36.65	0\\
36.66	0\\
36.67	0\\
36.68	0\\
36.69	0\\
36.7	0\\
36.71	0\\
36.72	0\\
36.73	0\\
36.74	0\\
36.75	0\\
36.76	0\\
36.77	0\\
36.78	0\\
36.79	0\\
36.8	0\\
36.81	1.73472347597681e-18\\
36.82	0\\
36.83	0\\
36.84	0\\
36.85	0\\
36.86	0\\
36.87	0\\
36.88	0\\
36.89	0\\
36.9	0\\
36.91	0\\
36.92	0\\
36.93	0\\
36.94	0\\
36.95	0\\
36.96	0\\
36.97	0\\
36.98	1.73472347597681e-18\\
36.99	0\\
37	0\\
37.01	0\\
37.02	1.73472347597681e-18\\
37.03	0\\
37.04	0\\
37.05	0\\
37.06	0\\
37.07	0\\
37.08	0\\
37.09	0\\
37.1	0\\
37.11	0\\
37.12	0\\
37.13	0\\
37.14	0\\
37.15	0\\
37.16	0\\
37.17	0\\
37.18	0\\
37.19	0\\
37.2	0\\
37.21	0\\
37.22	0\\
37.23	0\\
37.24	0\\
37.25	0\\
37.26	0\\
37.27	0\\
37.28	0\\
37.29	0\\
37.3	1.73472347597681e-18\\
37.31	0\\
37.32	0\\
37.33	0\\
37.34	0\\
37.35	0\\
37.36	0\\
37.37	0\\
37.38	0\\
37.39	0\\
37.4	0\\
37.41	0\\
37.42	0\\
37.43	0\\
37.44	0\\
37.45	0\\
37.46	0\\
37.47	0\\
37.48	0\\
37.49	1.73472347597681e-18\\
37.5	0\\
37.51	0\\
37.52	0\\
37.53	0\\
37.54	0\\
37.55	0\\
37.56	0\\
37.57	0\\
37.58	0\\
37.59	0\\
37.6	0\\
37.61	0\\
37.62	0\\
37.63	0\\
37.64	0\\
37.65	0\\
37.66	0\\
37.67	0\\
37.68	0\\
37.69	0\\
37.7	0\\
37.71	0\\
37.72	0\\
37.73	0\\
37.74	0\\
37.75	0\\
37.76	1.73472347597681e-18\\
37.77	0\\
37.78	0\\
37.79	0\\
37.8	0\\
37.81	0\\
37.82	0\\
37.83	0\\
37.84	0\\
37.85	0\\
37.86	0\\
37.87	0\\
37.88	0\\
37.89	0\\
37.9	0\\
37.91	0\\
37.92	0\\
37.93	0\\
37.94	0\\
37.95	0\\
37.96	0\\
37.97	0\\
37.98	0\\
37.99	0\\
38	1.73472347597681e-18\\
38.01	0\\
38.02	0\\
38.03	0\\
38.04	0\\
38.05	0\\
38.06	0\\
38.07	0\\
38.08	0\\
38.09	0\\
38.1	0\\
38.11	1.73472347597681e-18\\
38.12	1.73472347597681e-18\\
38.13	0\\
38.14	0\\
38.15	0\\
38.16	0\\
38.17	0\\
38.18	0\\
38.19	0\\
38.2	0\\
38.21	0\\
38.22	0\\
38.23	0\\
38.24	0\\
38.25	0\\
38.26	0\\
38.27	1.73472347597681e-18\\
38.28	0\\
38.29	0\\
38.3	0\\
38.31	1.73472347597681e-18\\
38.32	0\\
38.33	0\\
38.34	0\\
38.35	1.73472347597681e-18\\
38.36	0\\
38.37	0\\
38.38	0\\
38.39	0\\
38.4	0\\
38.41	0\\
38.42	0\\
38.43	0\\
38.44	0\\
38.45	0\\
38.46	0\\
38.47	0\\
38.48	1.73472347597681e-18\\
38.49	0\\
38.5	0\\
38.51	0\\
38.52	0\\
38.53	0\\
38.54	0\\
38.55	0\\
38.56	0\\
38.57	0\\
38.58	0\\
38.59	0\\
38.6	0\\
38.61	1.73472347597681e-18\\
38.62	0\\
38.63	1.73472347597681e-18\\
38.64	0\\
38.65	1.73472347597681e-18\\
38.66	0\\
38.67	0\\
38.68	0\\
38.69	0\\
38.7	0\\
38.71	0\\
38.72	0\\
38.73	0\\
38.74	1.73472347597681e-18\\
38.75	0\\
38.76	0\\
38.77	0\\
38.78	0\\
38.79	0\\
38.8	0\\
38.81	0\\
38.82	1.73472347597681e-18\\
38.83	0\\
38.84	0\\
38.85	0\\
38.86	0\\
38.87	1.73472347597681e-18\\
38.88	0\\
38.89	1.73472347597681e-18\\
38.9	0\\
38.91	0\\
38.92	0\\
38.93	0\\
38.94	0\\
38.95	0\\
38.96	0\\
38.97	0\\
38.98	0\\
38.99	0\\
39	0\\
39.01	0\\
39.02	0\\
39.03	0\\
39.04	0\\
39.05	0\\
39.06	0\\
39.07	0\\
39.08	0\\
39.09	1.73472347597681e-18\\
39.1	0\\
39.11	0\\
39.12	0\\
39.13	0\\
39.14	0\\
39.15	0\\
39.16	0\\
39.17	0\\
39.18	0\\
39.19	1.73472347597681e-18\\
39.2	0\\
39.21	0\\
39.22	0\\
39.23	0\\
39.24	0\\
39.25	0\\
39.26	0\\
39.27	0\\
39.28	0\\
39.29	0\\
39.3	0\\
39.31	0\\
39.32	0\\
39.33	0\\
39.34	1.73472347597681e-18\\
39.35	0\\
39.36	0\\
39.37	0\\
39.38	0\\
39.39	0\\
39.4	0\\
39.41	0\\
39.42	0\\
39.43	0\\
39.44	0\\
39.45	0\\
39.46	0\\
39.47	0\\
39.48	0\\
39.49	0\\
39.5	0\\
39.51	0\\
39.52	1.73472347597681e-18\\
39.53	0\\
39.54	0\\
39.55	0\\
39.56	0\\
39.57	0\\
39.58	0\\
39.59	0\\
39.6	0\\
39.61	0\\
39.62	0\\
39.63	0\\
39.64	0\\
39.65	0\\
39.66	0\\
39.67	1.73472347597681e-18\\
39.68	0\\
39.69	0\\
39.7	0\\
39.71	0\\
39.72	0\\
39.73	0\\
39.74	0\\
39.75	1.73472347597681e-18\\
39.76	0\\
39.77	0\\
39.78	0\\
39.79	0\\
39.8	0\\
39.81	0\\
39.82	1.73472347597681e-18\\
39.83	0\\
39.84	0\\
39.85	0\\
39.86	0\\
39.87	0\\
39.88	0\\
39.89	0\\
39.9	0\\
39.91	0\\
39.92	1.73472347597681e-18\\
39.93	0\\
39.94	1.73472347597681e-18\\
39.95	0\\
39.96	0\\
39.97	0\\
39.98	0\\
39.99	0\\
40	0\\
40.01	1.73472347597681e-18\\
};
\addplot [color=mycolor1,dashed,forget plot]
  table[row sep=crcr]{%
40.01	1.73472347597681e-18\\
40.02	0\\
40.03	1.73472347597681e-18\\
40.04	0\\
40.05	1.73472347597681e-18\\
40.06	0\\
40.07	1.73472347597681e-18\\
40.08	0\\
40.09	1.73472347597681e-18\\
40.1	1.73472347597681e-18\\
40.11	0\\
40.12	0\\
40.13	0\\
40.14	0\\
40.15	1.73472347597681e-18\\
40.16	0\\
40.17	0\\
40.18	0\\
40.19	0\\
40.2	0\\
40.21	0\\
40.22	1.73472347597681e-18\\
40.23	0\\
40.24	0\\
40.25	1.73472347597681e-18\\
40.26	0\\
40.27	0\\
40.28	0\\
40.29	0\\
40.3	0\\
40.31	1.73472347597681e-18\\
40.32	1.73472347597681e-18\\
40.33	0\\
40.34	0\\
40.35	1.73472347597681e-18\\
40.36	0\\
40.37	0\\
40.38	0\\
40.39	0\\
40.4	1.73472347597681e-18\\
40.41	0\\
40.42	0\\
40.43	1.73472347597681e-18\\
40.44	0\\
40.45	0\\
40.46	0\\
40.47	0\\
40.48	0\\
40.49	0\\
40.5	1.73472347597681e-18\\
40.51	0\\
40.52	0\\
40.53	0\\
40.54	0\\
40.55	0\\
40.56	0\\
40.57	0\\
40.58	1.73472347597681e-18\\
40.59	0\\
40.6	0\\
40.61	1.73472347597681e-18\\
40.62	0\\
40.63	0\\
40.64	0\\
40.65	0\\
40.66	0\\
40.67	0\\
40.68	0\\
40.69	0\\
40.7	0\\
40.71	0\\
40.72	0\\
40.73	0\\
40.74	0\\
40.75	0\\
40.76	0\\
40.77	0\\
40.78	0\\
40.79	0\\
40.8	0\\
40.81	0\\
40.82	0\\
40.83	0\\
40.84	0\\
40.85	0\\
40.86	0\\
40.87	1.73472347597681e-18\\
40.88	0\\
40.89	0\\
40.9	1.73472347597681e-18\\
40.91	0\\
40.92	0\\
40.93	0\\
40.94	0\\
40.95	1.73472347597681e-18\\
40.96	0\\
40.97	1.73472347597681e-18\\
40.98	0\\
40.99	0\\
41	0\\
41.01	0\\
41.02	0\\
41.03	0\\
41.04	0\\
41.05	0\\
41.06	0\\
41.07	0\\
41.08	0\\
41.09	0\\
41.1	0\\
41.11	0\\
41.12	0\\
41.13	0\\
41.14	0\\
41.15	1.73472347597681e-18\\
41.16	0\\
41.17	0\\
41.18	1.73472347597681e-18\\
41.19	0\\
41.2	0\\
41.21	0\\
41.22	0\\
41.23	0\\
41.24	0\\
41.25	0\\
41.26	0\\
41.27	1.73472347597681e-18\\
41.28	0\\
41.29	1.73472347597681e-18\\
41.3	0\\
41.31	0\\
41.32	0\\
41.33	0\\
41.34	0\\
41.35	0\\
41.36	0\\
41.37	0\\
41.38	1.73472347597681e-18\\
41.39	0\\
41.4	0\\
41.41	1.73472347597681e-18\\
41.42	0\\
41.43	0\\
41.44	0\\
41.45	0\\
41.46	0\\
41.47	0\\
41.48	0\\
41.49	0\\
41.5	0\\
41.51	0\\
41.52	0\\
41.53	0\\
41.54	0\\
41.55	0\\
41.56	0\\
41.57	0\\
41.58	0\\
41.59	0\\
41.6	0\\
41.61	1.73472347597681e-18\\
41.62	0\\
41.63	0\\
41.64	0\\
41.65	0\\
41.66	0\\
41.67	0\\
41.68	1.73472347597681e-18\\
41.69	0\\
41.7	0\\
41.71	0\\
41.72	0\\
41.73	0\\
41.74	0\\
41.75	0\\
41.76	0\\
41.77	0\\
41.78	0\\
41.79	0\\
41.8	0\\
41.81	1.73472347597681e-18\\
41.82	0\\
41.83	0\\
41.84	0\\
41.85	1.73472347597681e-18\\
41.86	0\\
41.87	0\\
41.88	0\\
41.89	0\\
41.9	0\\
41.91	0\\
41.92	1.73472347597681e-18\\
41.93	0\\
41.94	0\\
41.95	0\\
41.96	0\\
41.97	0\\
41.98	0\\
41.99	0\\
42	0\\
42.01	0\\
42.02	0\\
42.03	0\\
42.04	0\\
42.05	0\\
42.06	0\\
42.07	0\\
42.08	0\\
42.09	0\\
42.1	0\\
42.11	0\\
42.12	0\\
42.13	0\\
42.14	1.73472347597681e-18\\
42.15	0\\
42.16	1.73472347597681e-18\\
42.17	0\\
42.18	1.73472347597681e-18\\
42.19	0\\
42.2	0\\
42.21	0\\
42.22	0\\
42.23	0\\
42.24	1.73472347597681e-18\\
42.25	0\\
42.26	0\\
42.27	0\\
42.28	0\\
42.29	0\\
42.3	1.73472347597681e-18\\
42.31	0\\
42.32	0\\
42.33	0\\
42.34	0\\
42.35	0\\
42.36	1.73472347597681e-18\\
42.37	0\\
42.38	0\\
42.39	0\\
42.4	0\\
42.41	0\\
42.42	0\\
42.43	0\\
42.44	0\\
42.45	0\\
42.46	0\\
42.47	0\\
42.48	0\\
42.49	0\\
42.5	0\\
42.51	0\\
42.52	0\\
42.53	0\\
42.54	0\\
42.55	0\\
42.56	0\\
42.57	0\\
42.58	0\\
42.59	0\\
42.6	0\\
42.61	0\\
42.62	0\\
42.63	0\\
42.64	0\\
42.65	0\\
42.66	0\\
42.67	0\\
42.68	0\\
42.69	0\\
42.7	0\\
42.71	1.73472347597681e-18\\
42.72	0\\
42.73	0\\
42.74	1.73472347597681e-18\\
42.75	0\\
42.76	0\\
42.77	1.73472347597681e-18\\
42.78	0\\
42.79	0\\
42.8	0\\
42.81	0\\
42.82	0\\
42.83	0\\
42.84	0\\
42.85	0\\
42.86	0\\
42.87	0\\
42.88	0\\
42.89	1.73472347597681e-18\\
42.9	0\\
42.91	0\\
42.92	0\\
42.93	0\\
42.94	0\\
42.95	0\\
42.96	0\\
42.97	0\\
42.98	0\\
42.99	0\\
43	1.73472347597681e-18\\
43.01	0\\
43.02	0\\
43.03	0\\
43.04	0\\
43.05	0\\
43.06	0\\
43.07	0\\
43.08	0\\
43.09	1.73472347597681e-18\\
43.1	0\\
43.11	0\\
43.12	0\\
43.13	1.73472347597681e-18\\
43.14	0\\
43.15	0\\
43.16	0\\
43.17	0\\
43.18	0\\
43.19	0\\
43.2	0\\
43.21	0\\
43.22	0\\
43.23	0\\
43.24	0\\
43.25	0\\
43.26	0\\
43.27	0\\
43.28	0\\
43.29	0\\
43.3	0\\
43.31	0\\
43.32	0\\
43.33	1.73472347597681e-18\\
43.34	0\\
43.35	0\\
43.36	0\\
43.37	0\\
43.38	1.73472347597681e-18\\
43.39	0\\
43.4	0\\
43.41	0\\
43.42	0\\
43.43	0\\
43.44	0\\
43.45	0\\
43.46	0\\
43.47	0\\
43.48	0\\
43.49	0\\
43.5	0\\
43.51	0\\
43.52	1.73472347597681e-18\\
43.53	0\\
43.54	0\\
43.55	1.73472347597681e-18\\
43.56	0\\
43.57	0\\
43.58	0\\
43.59	0\\
43.6	0\\
43.61	0\\
43.62	0\\
43.63	0\\
43.64	0\\
43.65	0\\
43.66	0\\
43.67	0\\
43.68	0\\
43.69	0\\
43.7	0\\
43.71	0\\
43.72	1.73472347597681e-18\\
43.73	0\\
43.74	1.73472347597681e-18\\
43.75	0\\
43.76	0\\
43.77	0\\
43.78	0\\
43.79	0\\
43.8	0\\
43.81	1.73472347597681e-18\\
43.82	0\\
43.83	1.73472347597681e-18\\
43.84	1.73472347597681e-18\\
43.85	0\\
43.86	0\\
43.87	0\\
43.88	0\\
43.89	0\\
43.9	0\\
43.91	0\\
43.92	0\\
43.93	1.73472347597681e-18\\
43.94	0\\
43.95	0\\
43.96	1.73472347597681e-18\\
43.97	0\\
43.98	0\\
43.99	0\\
44	0\\
44.01	0\\
44.02	0\\
44.03	1.73472347597681e-18\\
44.04	0\\
44.05	0\\
44.06	0\\
44.07	0\\
44.08	0\\
44.09	0\\
44.1	0\\
44.11	0\\
44.12	0\\
44.13	0\\
44.14	0\\
44.15	0\\
44.16	0\\
44.17	0\\
44.18	0\\
44.19	0\\
44.2	0\\
44.21	0\\
44.22	0\\
44.23	0\\
44.24	0\\
44.25	0\\
44.26	1.73472347597681e-18\\
44.27	0\\
44.28	0\\
44.29	0\\
44.3	0\\
44.31	0\\
44.32	0\\
44.33	1.73472347597681e-18\\
44.34	0\\
44.35	1.73472347597681e-18\\
44.36	1.73472347597681e-18\\
44.37	0\\
44.38	0\\
44.39	0\\
44.4	0\\
44.41	0\\
44.42	0\\
44.43	0\\
44.44	0\\
44.45	0\\
44.46	1.73472347597681e-18\\
44.47	0\\
44.48	0\\
44.49	0\\
44.5	0\\
44.51	0\\
44.52	0\\
44.53	0\\
44.54	0\\
44.55	0\\
44.56	0\\
44.57	0\\
44.58	0\\
44.59	0\\
44.6	0\\
44.61	0\\
44.62	0\\
44.63	0\\
44.64	1.73472347597681e-18\\
44.65	0\\
44.66	0\\
44.67	0\\
44.68	0\\
44.69	1.73472347597681e-18\\
44.7	0\\
44.71	0\\
44.72	0\\
44.73	0\\
44.74	0\\
44.75	0\\
44.76	0\\
44.77	0\\
44.78	0\\
44.79	0\\
44.8	0\\
44.81	0\\
44.82	0\\
44.83	0\\
44.84	1.73472347597681e-18\\
44.85	1.73472347597681e-18\\
44.86	0\\
44.87	0\\
44.88	0\\
44.89	0\\
44.9	0\\
44.91	1.73472347597681e-18\\
44.92	0\\
44.93	0\\
44.94	0\\
44.95	0\\
44.96	0\\
44.97	0\\
44.98	0\\
44.99	0\\
45	0\\
45.01	0\\
45.02	0\\
45.03	0\\
45.04	0\\
45.05	0\\
45.06	0\\
45.07	0\\
45.08	0\\
45.09	0\\
45.1	0\\
45.11	0\\
45.12	0\\
45.13	1.73472347597681e-18\\
45.14	0\\
45.15	0\\
45.16	0\\
45.17	0\\
45.18	0\\
45.19	0\\
45.2	0\\
45.21	0\\
45.22	0\\
45.23	0\\
45.24	0\\
45.25	0\\
45.26	0\\
45.27	0\\
45.28	0\\
45.29	0\\
45.3	0\\
45.31	0\\
45.32	0\\
45.33	0\\
45.34	0\\
45.35	0\\
45.36	0\\
45.37	0\\
45.38	0\\
45.39	0\\
45.4	1.73472347597681e-18\\
45.41	0\\
45.42	0\\
45.43	0\\
45.44	0\\
45.45	0\\
45.46	0\\
45.47	0\\
45.48	0\\
45.49	0\\
45.5	0\\
45.51	0\\
45.52	0\\
45.53	0\\
45.54	0\\
45.55	0\\
45.56	0\\
45.57	0\\
45.58	0\\
45.59	0\\
45.6	1.73472347597681e-18\\
45.61	0\\
45.62	0\\
45.63	1.73472347597681e-18\\
45.64	0\\
45.65	0\\
45.66	0\\
45.67	0\\
45.68	0\\
45.69	0\\
45.7	0\\
45.71	1.73472347597681e-18\\
45.72	0\\
45.73	0\\
45.74	0\\
45.75	0\\
45.76	0\\
45.77	0\\
45.78	0\\
45.79	0\\
45.8	0\\
45.81	0\\
45.82	0\\
45.83	1.73472347597681e-18\\
45.84	0\\
45.85	0\\
45.86	0\\
45.87	0\\
45.88	0\\
45.89	0\\
45.9	0\\
45.91	0\\
45.92	0\\
45.93	1.73472347597681e-18\\
45.94	0\\
45.95	0\\
45.96	0\\
45.97	0\\
45.98	0\\
45.99	0\\
46	0\\
46.01	0\\
46.02	0\\
46.03	0\\
46.04	0\\
46.05	0\\
46.06	0\\
46.07	0\\
46.08	1.73472347597681e-18\\
46.09	0\\
46.1	0\\
46.11	0\\
46.12	0\\
46.13	0\\
46.14	0\\
46.15	0\\
46.16	0\\
46.17	0\\
46.18	0\\
46.19	0\\
46.2	0\\
46.21	0\\
46.22	0\\
46.23	0\\
46.24	0\\
46.25	1.73472347597681e-18\\
46.26	0\\
46.27	0\\
46.28	0\\
46.29	1.73472347597681e-18\\
46.3	1.73472347597681e-18\\
46.31	0\\
46.32	1.73472347597681e-18\\
46.33	0\\
46.34	0\\
46.35	0\\
46.36	0\\
46.37	0\\
46.38	1.73472347597681e-18\\
46.39	0\\
46.4	0\\
46.41	0\\
46.42	0\\
46.43	0\\
46.44	0\\
46.45	0\\
46.46	0\\
46.47	0\\
46.48	0\\
46.49	0\\
46.5	0\\
46.51	0\\
46.52	0\\
46.53	0\\
46.54	0\\
46.55	0\\
46.56	0\\
46.57	0\\
46.58	0\\
46.59	1.73472347597681e-18\\
46.6	1.73472347597681e-18\\
46.61	0\\
46.62	0\\
46.63	0\\
46.64	0\\
46.65	0\\
46.66	0\\
46.67	0\\
46.68	1.73472347597681e-18\\
46.69	0\\
46.7	0\\
46.71	0\\
46.72	0\\
46.73	0\\
46.74	0\\
46.75	0\\
46.76	0\\
46.77	0\\
46.78	0\\
46.79	0\\
46.8	0\\
46.81	0\\
46.82	0\\
46.83	0\\
46.84	0\\
46.85	0\\
46.86	0\\
46.87	1.73472347597681e-18\\
46.88	0\\
46.89	1.73472347597681e-18\\
46.9	0\\
46.91	0\\
46.92	0\\
46.93	0\\
46.94	0\\
46.95	0\\
46.96	0\\
46.97	0\\
46.98	0\\
46.99	0\\
47	0\\
47.01	0\\
47.02	0\\
47.03	0\\
47.04	0\\
47.05	1.73472347597681e-18\\
47.06	0\\
47.07	0\\
47.08	0\\
47.09	0\\
47.1	1.73472347597681e-18\\
47.11	0\\
47.12	0\\
47.13	0\\
47.14	1.73472347597681e-18\\
47.15	0\\
47.16	1.73472347597681e-18\\
47.17	0\\
47.18	0\\
47.19	0\\
47.2	0\\
47.21	0\\
47.22	0\\
47.23	0\\
47.24	0\\
47.25	0\\
47.26	0\\
47.27	0\\
47.28	0\\
47.29	0\\
47.3	0\\
47.31	0\\
47.32	0\\
47.33	0\\
47.34	0\\
47.35	0\\
47.36	1.73472347597681e-18\\
47.37	0\\
47.38	1.73472347597681e-18\\
47.39	0\\
47.4	0\\
47.41	0\\
47.42	0\\
47.43	0\\
47.44	0\\
47.45	0\\
47.46	1.73472347597681e-18\\
47.47	0\\
47.48	0\\
47.49	0\\
47.5	0\\
47.51	1.73472347597681e-18\\
47.52	1.73472347597681e-18\\
47.53	0\\
47.54	0\\
47.55	0\\
47.56	0\\
47.57	1.73472347597681e-18\\
47.58	1.73472347597681e-18\\
47.59	0\\
47.6	0\\
47.61	0\\
47.62	0\\
47.63	0\\
47.64	0\\
47.65	0\\
47.66	0\\
47.67	0\\
47.68	0\\
47.69	0\\
47.7	0\\
47.71	0\\
47.72	0\\
47.73	0\\
47.74	0\\
47.75	0\\
47.76	1.73472347597681e-18\\
47.77	0\\
47.78	0\\
47.79	0\\
47.8	0\\
47.81	0\\
47.82	0\\
47.83	0\\
47.84	1.73472347597681e-18\\
47.85	0\\
47.86	0\\
47.87	0\\
47.88	0\\
47.89	0\\
47.9	0\\
47.91	0\\
47.92	0\\
47.93	0\\
47.94	0\\
47.95	0\\
47.96	0\\
47.97	1.73472347597681e-18\\
47.98	0\\
47.99	0\\
48	1.73472347597681e-18\\
48.01	0\\
48.02	0\\
48.03	0\\
48.04	0\\
48.05	0\\
48.06	0\\
48.07	1.73472347597681e-18\\
48.08	1.73472347597681e-18\\
48.09	0\\
48.1	0\\
48.11	1.73472347597681e-18\\
48.12	0\\
48.13	0\\
48.14	0\\
48.15	0\\
48.16	0\\
48.17	0\\
48.18	0\\
48.19	0\\
48.2	0\\
48.21	0\\
48.22	0\\
48.23	0\\
48.24	0\\
48.25	0\\
48.26	0\\
48.27	0\\
48.28	0\\
48.29	0\\
48.3	0\\
48.31	0\\
48.32	0\\
48.33	0\\
48.34	0\\
48.35	0\\
48.36	0\\
48.37	0\\
48.38	0\\
48.39	0\\
48.4	0\\
48.41	0\\
48.42	0\\
48.43	0\\
48.44	0\\
48.45	0\\
48.46	0\\
48.47	0\\
48.48	0\\
48.49	0\\
48.5	1.73472347597681e-18\\
48.51	0\\
48.52	0\\
48.53	0\\
48.54	0\\
48.55	0\\
48.56	0\\
48.57	0\\
48.58	0\\
48.59	0\\
48.6	0\\
48.61	0\\
48.62	0\\
48.63	0\\
48.64	0\\
48.65	0\\
48.66	0\\
48.67	0\\
48.68	0\\
48.69	0\\
48.7	0\\
48.71	0\\
48.72	0\\
48.73	1.73472347597681e-18\\
48.74	0\\
48.75	0\\
48.76	0\\
48.77	0\\
48.78	0\\
48.79	0\\
48.8	1.73472347597681e-18\\
48.81	0\\
48.82	0\\
48.83	0\\
48.84	0\\
48.85	0\\
48.86	0\\
48.87	1.73472347597681e-18\\
48.88	1.73472347597681e-18\\
48.89	0\\
48.9	0\\
48.91	0\\
48.92	0\\
48.93	0\\
48.94	0\\
48.95	1.73472347597681e-18\\
48.96	1.73472347597681e-18\\
48.97	0\\
48.98	0\\
48.99	0\\
49	0\\
49.01	0\\
49.02	0\\
49.03	0\\
49.04	1.73472347597681e-18\\
49.05	1.73472347597681e-18\\
49.06	0\\
49.07	0\\
49.08	0\\
49.09	0\\
49.1	0\\
49.11	0\\
49.12	0\\
49.13	1.73472347597681e-18\\
49.14	0\\
49.15	0\\
49.16	0\\
49.17	0\\
49.18	0\\
49.19	0\\
49.2	0\\
49.21	0\\
49.22	0\\
49.23	0\\
49.24	0\\
49.25	0\\
49.26	0\\
49.27	0\\
49.28	0\\
49.29	0\\
49.3	0\\
49.31	0\\
49.32	1.73472347597681e-18\\
49.33	0\\
49.34	0\\
49.35	0\\
49.36	0\\
49.37	0\\
49.38	0\\
49.39	0\\
49.4	0\\
49.41	0\\
49.42	0\\
49.43	0\\
49.44	0\\
49.45	0\\
49.46	0\\
49.47	0\\
49.48	0\\
49.49	0\\
49.5	0\\
49.51	0\\
49.52	0\\
49.53	0\\
49.54	0\\
49.55	0\\
49.56	0\\
49.57	0\\
49.58	0\\
49.59	0\\
49.6	0\\
49.61	0\\
49.62	0\\
49.63	0\\
49.64	0\\
49.65	0\\
49.66	0\\
49.67	1.73472347597681e-18\\
49.68	1.73472347597681e-18\\
49.69	0\\
49.7	0\\
49.71	0\\
49.72	0\\
49.73	0\\
49.74	0\\
49.75	0\\
49.76	0\\
49.77	0\\
49.78	0\\
49.79	0\\
49.8	0\\
49.81	0\\
49.82	0\\
49.83	0\\
49.84	0\\
49.85	0\\
49.86	0\\
49.87	0\\
49.88	0\\
49.89	0\\
49.9	0\\
49.91	0\\
49.92	0\\
49.93	0\\
49.94	1.73472347597681e-18\\
49.95	0\\
49.96	0\\
49.97	0\\
49.98	0\\
49.99	0\\
50	0\\
50.01	0\\
50.02	0\\
50.03	0\\
50.04	0\\
50.05	0\\
50.06	0\\
50.07	0\\
50.08	0\\
50.09	0\\
50.1	0\\
50.11	0\\
50.12	0\\
50.13	0\\
50.14	1.73472347597681e-18\\
50.15	0\\
50.16	0\\
50.17	0\\
50.18	0\\
50.19	0\\
50.2	0\\
50.21	1.73472347597681e-18\\
50.22	0\\
50.23	0\\
50.24	0\\
50.25	0\\
50.26	0\\
50.27	0\\
50.28	0\\
50.29	0\\
50.3	0\\
50.31	1.73472347597681e-18\\
50.32	0\\
50.33	0\\
50.34	0\\
50.35	0\\
50.36	0\\
50.37	1.73472347597681e-18\\
50.38	0\\
50.39	0\\
50.4	0\\
50.41	0\\
50.42	0\\
50.43	0\\
50.44	0\\
50.45	0\\
50.46	0\\
50.47	1.73472347597681e-18\\
50.48	0\\
50.49	0\\
50.5	0\\
50.51	1.73472347597681e-18\\
50.52	0\\
50.53	0\\
50.54	0\\
50.55	0\\
50.56	1.73472347597681e-18\\
50.57	0\\
50.58	0\\
50.59	0\\
50.6	0\\
50.61	0\\
50.62	0\\
50.63	0\\
50.64	0\\
50.65	0\\
50.66	0\\
50.67	0\\
50.68	0\\
50.69	0\\
50.7	0\\
50.71	0\\
50.72	0\\
50.73	0\\
50.74	0\\
50.75	0\\
50.76	0\\
50.77	0\\
50.78	1.73472347597681e-18\\
50.79	0\\
50.8	0\\
50.81	0\\
50.82	0\\
50.83	0\\
50.84	0\\
50.85	0\\
50.86	0\\
50.87	1.73472347597681e-18\\
50.88	1.73472347597681e-18\\
50.89	0\\
50.9	0\\
50.91	0\\
50.92	0\\
50.93	1.73472347597681e-18\\
50.94	0\\
50.95	0\\
50.96	1.73472347597681e-18\\
50.97	0\\
50.98	0\\
50.99	0\\
51	0\\
51.01	1.73472347597681e-18\\
51.02	0\\
51.03	0\\
51.04	0\\
51.05	0\\
51.06	0\\
51.07	0\\
51.08	0\\
51.09	0\\
51.1	0\\
51.11	0\\
51.12	0\\
51.13	0\\
51.14	0\\
51.15	0\\
51.16	0\\
51.17	0\\
51.18	0\\
51.19	1.73472347597681e-18\\
51.2	0\\
51.21	0\\
51.22	0\\
51.23	1.73472347597681e-18\\
51.24	0\\
51.25	0\\
51.26	0\\
51.27	0\\
51.28	0\\
51.29	0\\
51.3	0\\
51.31	0\\
51.32	1.73472347597681e-18\\
51.33	0\\
51.34	0\\
51.35	0\\
51.36	0\\
51.37	1.73472347597681e-18\\
51.38	0\\
51.39	0\\
51.4	0\\
51.41	0\\
51.42	1.73472347597681e-18\\
51.43	0\\
51.44	0\\
51.45	0\\
51.46	1.73472347597681e-18\\
51.47	0\\
51.48	0\\
51.49	1.73472347597681e-18\\
51.5	0\\
51.51	0\\
51.52	0\\
51.53	0\\
51.54	0\\
51.55	0\\
51.56	0\\
51.57	0\\
51.58	1.73472347597681e-18\\
51.59	0\\
51.6	0\\
51.61	0\\
51.62	0\\
51.63	0\\
51.64	0\\
51.65	0\\
51.66	0\\
51.67	0\\
51.68	1.73472347597681e-18\\
51.69	0\\
51.7	1.73472347597681e-18\\
51.71	0\\
51.72	0\\
51.73	0\\
51.74	0\\
51.75	0\\
51.76	0\\
51.77	0\\
51.78	0\\
51.79	0\\
51.8	0\\
51.81	0\\
51.82	0\\
51.83	0\\
51.84	0\\
51.85	0\\
51.86	0\\
51.87	0\\
51.88	0\\
51.89	0\\
51.9	0\\
51.91	0\\
51.92	0\\
51.93	1.73472347597681e-18\\
51.94	0\\
51.95	0\\
51.96	0\\
51.97	0\\
51.98	0\\
51.99	0\\
52	0\\
52.01	0\\
52.02	0\\
52.03	0\\
52.04	0\\
52.05	0\\
52.06	0\\
52.07	0\\
52.08	0\\
52.09	0\\
52.1	0\\
52.11	0\\
52.12	0\\
52.13	0\\
52.14	0\\
52.15	0\\
52.16	1.73472347597681e-18\\
52.17	1.73472347597681e-18\\
52.18	0\\
52.19	0\\
52.2	0\\
52.21	0\\
52.22	1.73472347597681e-18\\
52.23	0\\
52.24	0\\
52.25	0\\
52.26	0\\
52.27	0\\
52.28	0\\
52.29	0\\
52.3	0\\
52.31	0\\
52.32	0\\
52.33	0\\
52.34	0\\
52.35	0\\
52.36	0\\
52.37	1.73472347597681e-18\\
52.38	0\\
52.39	1.73472347597681e-18\\
52.4	1.73472347597681e-18\\
52.41	0\\
52.42	0\\
52.43	0\\
52.44	0\\
52.45	0\\
52.46	0\\
52.47	0\\
52.48	0\\
52.49	0\\
52.5	0\\
52.51	0\\
52.52	0\\
52.53	0\\
52.54	0\\
52.55	0\\
52.56	0\\
52.57	0\\
52.58	0\\
52.59	0\\
52.6	0\\
52.61	1.73472347597681e-18\\
52.62	1.73472347597681e-18\\
52.63	0\\
52.64	1.73472347597681e-18\\
52.65	0\\
52.66	0\\
52.67	0\\
52.68	0\\
52.69	0\\
52.7	0\\
52.71	0\\
52.72	1.73472347597681e-18\\
52.73	0\\
52.74	0\\
52.75	0\\
52.76	0\\
52.77	0\\
52.78	0\\
52.79	0\\
52.8	0\\
52.81	0\\
52.82	0\\
52.83	0\\
52.84	0\\
52.85	1.73472347597681e-18\\
52.86	0\\
52.87	1.73472347597681e-18\\
52.88	0\\
52.89	0\\
52.9	1.73472347597681e-18\\
52.91	0\\
52.92	0\\
52.93	0\\
52.94	0\\
52.95	1.73472347597681e-18\\
52.96	0\\
52.97	0\\
52.98	0\\
52.99	0\\
53	0\\
53.01	1.73472347597681e-18\\
53.02	0\\
53.03	0\\
53.04	0\\
53.05	0\\
53.06	0\\
53.07	0\\
53.08	0\\
53.09	0\\
53.1	1.73472347597681e-18\\
53.11	1.73472347597681e-18\\
53.12	0\\
53.13	0\\
53.14	0\\
53.15	0\\
53.16	0\\
53.17	0\\
53.18	0\\
53.19	0\\
53.2	1.73472347597681e-18\\
53.21	0\\
53.22	0\\
53.23	0\\
53.24	0\\
53.25	0\\
53.26	0\\
53.27	1.73472347597681e-18\\
53.28	0\\
53.29	0\\
53.3	0\\
53.31	0\\
53.32	0\\
53.33	0\\
53.34	0\\
53.35	0\\
53.36	0\\
53.37	0\\
53.38	0\\
53.39	0\\
53.4	0\\
53.41	0\\
53.42	0\\
53.43	0\\
53.44	0\\
53.45	1.73472347597681e-18\\
53.46	0\\
53.47	0\\
53.48	0\\
53.49	0\\
53.5	0\\
53.51	0\\
53.52	1.73472347597681e-18\\
53.53	1.73472347597681e-18\\
53.54	0\\
53.55	0\\
53.56	0\\
53.57	0\\
53.58	1.73472347597681e-18\\
53.59	0\\
53.6	0\\
53.61	0\\
53.62	0\\
53.63	0\\
53.64	0\\
53.65	0\\
53.66	0\\
53.67	0\\
53.68	0\\
53.69	0\\
53.7	0\\
53.71	0\\
53.72	0\\
53.73	0\\
53.74	0\\
53.75	0\\
53.76	0\\
53.77	0\\
53.78	0\\
53.79	1.73472347597681e-18\\
53.8	0\\
53.81	0\\
53.82	0\\
53.83	0\\
53.84	0\\
53.85	1.73472347597681e-18\\
53.86	0\\
53.87	0\\
53.88	0\\
53.89	0\\
53.9	0\\
53.91	0\\
53.92	0\\
53.93	0\\
53.94	0\\
53.95	0\\
53.96	0\\
53.97	0\\
53.98	0\\
53.99	0\\
54	0\\
54.01	0\\
54.02	0\\
54.03	0\\
54.04	0\\
54.05	0\\
54.06	0\\
54.07	1.73472347597681e-18\\
54.08	0\\
54.09	1.73472347597681e-18\\
54.1	0\\
54.11	1.73472347597681e-18\\
54.12	0\\
54.13	1.73472347597681e-18\\
54.14	0\\
54.15	1.73472347597681e-18\\
54.16	0\\
54.17	0\\
54.18	0\\
54.19	0\\
54.2	1.73472347597681e-18\\
54.21	1.73472347597681e-18\\
54.22	0\\
54.23	0\\
54.24	0\\
54.25	0\\
54.26	0\\
54.27	0\\
54.28	0\\
54.29	0\\
54.3	0\\
54.31	0\\
54.32	1.73472347597681e-18\\
54.33	0\\
54.34	0\\
54.35	0\\
54.36	0\\
54.37	0\\
54.38	0\\
54.39	0\\
54.4	0\\
54.41	0\\
54.42	0\\
54.43	0\\
54.44	0\\
54.45	1.73472347597681e-18\\
54.46	0\\
54.47	0\\
54.48	0\\
54.49	0\\
54.5	0\\
54.51	0\\
54.52	0\\
54.53	0\\
54.54	0\\
54.55	0\\
54.56	0\\
54.57	0\\
54.58	0\\
54.59	0\\
54.6	0\\
54.61	0\\
54.62	0\\
54.63	0\\
54.64	0\\
54.65	1.73472347597681e-18\\
54.66	0\\
54.67	0\\
54.68	0\\
54.69	1.73472347597681e-18\\
54.7	0\\
54.71	0\\
54.72	0\\
54.73	0\\
54.74	0\\
54.75	0\\
54.76	0\\
54.77	0\\
54.78	0\\
54.79	0\\
54.8	0\\
54.81	0\\
54.82	0\\
54.83	0\\
54.84	0\\
54.85	0\\
54.86	0\\
54.87	0\\
54.88	0\\
54.89	0\\
54.9	1.73472347597681e-18\\
54.91	0\\
54.92	0\\
54.93	0\\
54.94	1.73472347597681e-18\\
54.95	0\\
54.96	0\\
54.97	0\\
54.98	1.73472347597681e-18\\
54.99	0\\
55	0\\
55.01	0\\
55.02	0\\
55.03	0\\
55.04	0\\
55.05	0\\
55.06	0\\
55.07	1.73472347597681e-18\\
55.08	1.73472347597681e-18\\
55.09	0\\
55.1	0\\
55.11	0\\
55.12	0\\
55.13	0\\
55.14	1.73472347597681e-18\\
55.15	0\\
55.16	1.73472347597681e-18\\
55.17	1.73472347597681e-18\\
55.18	0\\
55.19	1.73472347597681e-18\\
55.2	0\\
55.21	0\\
55.22	0\\
55.23	0\\
55.24	0\\
55.25	0\\
55.26	0\\
55.27	0\\
55.28	0\\
55.29	0\\
55.3	0\\
55.31	0\\
55.32	0\\
55.33	0\\
55.34	0\\
55.35	0\\
55.36	1.73472347597681e-18\\
55.37	0\\
55.38	0\\
55.39	0\\
55.4	0\\
55.41	0\\
55.42	0\\
55.43	0\\
55.44	0\\
55.45	0\\
55.46	0\\
55.47	1.73472347597681e-18\\
55.48	0\\
55.49	1.73472347597681e-18\\
55.5	0\\
55.51	0\\
55.52	0\\
55.53	0\\
55.54	0\\
55.55	0\\
55.56	1.73472347597681e-18\\
55.57	0\\
55.58	0\\
55.59	0\\
55.6	0\\
55.61	0\\
55.62	0\\
55.63	0\\
55.64	0\\
55.65	0\\
55.66	0\\
55.67	0\\
55.68	0\\
55.69	0\\
55.7	0\\
55.71	0\\
55.72	0\\
55.73	1.73472347597681e-18\\
55.74	0\\
55.75	0\\
55.76	0\\
55.77	1.73472347597681e-18\\
55.78	0\\
55.79	0\\
55.8	0\\
55.81	0\\
55.82	0\\
55.83	0\\
55.84	0\\
55.85	0\\
55.86	0\\
55.87	0\\
55.88	0\\
55.89	0\\
55.9	0\\
55.91	0\\
55.92	0\\
55.93	0\\
55.94	0\\
55.95	1.73472347597681e-18\\
55.96	0\\
55.97	0\\
55.98	0\\
55.99	0\\
56	0\\
56.01	1.73472347597681e-18\\
56.02	0\\
56.03	0\\
56.04	1.73472347597681e-18\\
56.05	1.73472347597681e-18\\
56.06	0\\
56.07	1.73472347597681e-18\\
56.08	0\\
56.09	0\\
56.1	0\\
56.11	0\\
56.12	0\\
56.13	0\\
56.14	0\\
56.15	0\\
56.16	0\\
56.17	0\\
56.18	0\\
56.19	0\\
56.2	0\\
56.21	0\\
56.22	0\\
56.23	0\\
56.24	1.73472347597681e-18\\
56.25	0\\
56.26	0\\
56.27	0\\
56.28	0\\
56.29	1.73472347597681e-18\\
56.3	0\\
56.31	0\\
56.32	0\\
56.33	0\\
56.34	0\\
56.35	0\\
56.36	0\\
56.37	0\\
56.38	0\\
56.39	0\\
56.4	0\\
56.41	0\\
56.42	0\\
56.43	0\\
56.44	0\\
56.45	0\\
56.46	0\\
56.47	0\\
56.48	0\\
56.49	0\\
56.5	0\\
56.51	0\\
56.52	0\\
56.53	0\\
56.54	0\\
56.55	0\\
56.56	0\\
56.57	0\\
56.58	0\\
56.59	0\\
56.6	0\\
56.61	0\\
56.62	0\\
56.63	0\\
56.64	0\\
56.65	0\\
56.66	0\\
56.67	0\\
56.68	0\\
56.69	0\\
56.7	1.73472347597681e-18\\
56.71	1.73472347597681e-18\\
56.72	0\\
56.73	0\\
56.74	1.73472347597681e-18\\
56.75	1.73472347597681e-18\\
56.76	0\\
56.77	0\\
56.78	0\\
56.79	1.73472347597681e-18\\
56.8	0\\
56.81	0\\
56.82	1.73472347597681e-18\\
56.83	0\\
56.84	0\\
56.85	0\\
56.86	0\\
56.87	0\\
56.88	0\\
56.89	0\\
56.9	0\\
56.91	0\\
56.92	0\\
56.93	0\\
56.94	0\\
56.95	0\\
56.96	0\\
56.97	0\\
56.98	0\\
56.99	0\\
57	1.73472347597681e-18\\
57.01	0\\
57.02	0\\
57.03	0\\
57.04	0\\
57.05	1.73472347597681e-18\\
57.06	1.73472347597681e-18\\
57.07	0\\
57.08	0\\
57.09	0\\
57.1	0\\
57.11	0\\
57.12	0\\
57.13	0\\
57.14	0\\
57.15	0\\
57.16	0\\
57.17	0\\
57.18	0\\
57.19	0\\
57.2	0\\
57.21	0\\
57.22	0\\
57.23	0\\
57.24	0\\
57.25	0\\
57.26	0\\
57.27	1.73472347597681e-18\\
57.28	0\\
57.29	0\\
57.3	0\\
57.31	0\\
57.32	0\\
57.33	0\\
57.34	0\\
57.35	0\\
57.36	0\\
57.37	0\\
57.38	1.73472347597681e-18\\
57.39	0\\
57.4	0\\
57.41	0\\
57.42	0\\
57.43	0\\
57.44	0\\
57.45	0\\
57.46	0\\
57.47	0\\
57.48	0\\
57.49	0\\
57.5	0\\
57.51	1.73472347597681e-18\\
57.52	0\\
57.53	0\\
57.54	0\\
57.55	0\\
57.56	0\\
57.57	0\\
57.58	0\\
57.59	0\\
57.6	0\\
57.61	0\\
57.62	0\\
57.63	0\\
57.64	0\\
57.65	0\\
57.66	0\\
57.67	0\\
57.68	1.73472347597681e-18\\
57.69	1.73472347597681e-18\\
57.7	0\\
57.71	0\\
57.72	0\\
57.73	0\\
57.74	0\\
57.75	0\\
57.76	0\\
57.77	1.73472347597681e-18\\
57.78	0\\
57.79	0\\
57.8	0\\
57.81	0\\
57.82	0\\
57.83	0\\
57.84	1.73472347597681e-18\\
57.85	1.73472347597681e-18\\
57.86	0\\
57.87	1.73472347597681e-18\\
57.88	0\\
57.89	0\\
57.9	0\\
57.91	0\\
57.92	0\\
57.93	0\\
57.94	0\\
57.95	0\\
57.96	0\\
57.97	0\\
57.98	0\\
57.99	0\\
58	1.73472347597681e-18\\
58.01	0\\
58.02	0\\
58.03	0\\
58.04	0\\
58.05	0\\
58.06	0\\
58.07	0\\
58.08	0\\
58.09	0\\
58.1	0\\
58.11	1.73472347597681e-18\\
58.12	0\\
58.13	1.73472347597681e-18\\
58.14	0\\
58.15	0\\
58.16	0\\
58.17	0\\
58.18	0\\
58.19	0\\
58.2	0\\
58.21	0\\
58.22	1.73472347597681e-18\\
58.23	0\\
58.24	0\\
58.25	1.73472347597681e-18\\
58.26	0\\
58.27	0\\
58.28	0\\
58.29	1.73472347597681e-18\\
58.3	0\\
58.31	1.73472347597681e-18\\
58.32	0\\
58.33	0\\
58.34	1.73472347597681e-18\\
58.35	1.73472347597681e-18\\
58.36	0\\
58.37	0\\
58.38	0\\
58.39	0\\
58.4	0\\
58.41	1.73472347597681e-18\\
58.42	1.73472347597681e-18\\
58.43	0\\
58.44	0\\
58.45	0\\
58.46	0\\
58.47	0\\
58.48	0\\
58.49	0\\
58.5	0\\
58.51	0\\
58.52	1.73472347597681e-18\\
58.53	0\\
58.54	0\\
58.55	0\\
58.56	0\\
58.57	0\\
58.58	0\\
58.59	0\\
58.6	0\\
58.61	0\\
58.62	0\\
58.63	0\\
58.64	0\\
58.65	0\\
58.66	0\\
58.67	1.73472347597681e-18\\
58.68	0\\
58.69	0\\
58.7	0\\
58.71	0\\
58.72	0\\
58.73	0\\
58.74	1.73472347597681e-18\\
58.75	0\\
58.76	0\\
58.77	0\\
58.78	0\\
58.79	1.73472347597681e-18\\
58.8	0\\
58.81	0\\
58.82	0\\
58.83	0\\
58.84	0\\
58.85	0\\
58.86	0\\
58.87	0\\
58.88	1.73472347597681e-18\\
58.89	0\\
58.9	0\\
58.91	0\\
58.92	0\\
58.93	0\\
58.94	0\\
58.95	0\\
58.96	0\\
58.97	0\\
58.98	1.73472347597681e-18\\
58.99	0\\
59	0\\
59.01	0\\
59.02	0\\
59.03	0\\
59.04	0\\
59.05	0\\
59.06	0\\
59.07	0\\
59.08	0\\
59.09	0\\
59.1	0\\
59.11	0\\
59.12	0\\
59.13	0\\
59.14	0\\
59.15	0\\
59.16	0\\
59.17	1.73472347597681e-18\\
59.18	0\\
59.19	0\\
59.2	1.73472347597681e-18\\
59.21	0\\
59.22	0\\
59.23	0\\
59.24	0\\
59.25	1.73472347597681e-18\\
59.26	0\\
59.27	0\\
59.28	0\\
59.29	0\\
59.3	1.73472347597681e-18\\
59.31	0\\
59.32	1.73472347597681e-18\\
59.33	0\\
59.34	0\\
59.35	0\\
59.36	0\\
59.37	0\\
59.38	0\\
59.39	0\\
59.4	0\\
59.41	1.73472347597681e-18\\
59.42	1.73472347597681e-18\\
59.43	0\\
59.44	0\\
59.45	0\\
59.46	0\\
59.47	0\\
59.48	0\\
59.49	0\\
59.5	1.73472347597681e-18\\
59.51	1.73472347597681e-18\\
59.52	0\\
59.53	0\\
59.54	0\\
59.55	0\\
59.56	0\\
59.57	0\\
59.58	0\\
59.59	1.73472347597681e-18\\
59.6	0\\
59.61	0\\
59.62	1.73472347597681e-18\\
59.63	0\\
59.64	0\\
59.65	0\\
59.66	1.73472347597681e-18\\
59.67	0\\
59.68	0\\
59.69	0\\
59.7	0\\
59.71	0\\
59.72	0\\
59.73	0\\
59.74	0\\
59.75	0\\
59.76	0\\
59.77	0\\
59.78	0\\
59.79	0\\
59.8	1.73472347597681e-18\\
59.81	0\\
59.82	0\\
59.83	0\\
59.84	0\\
59.85	0\\
59.86	1.73472347597681e-18\\
59.87	0\\
59.88	1.73472347597681e-18\\
59.89	0\\
59.9	0\\
59.91	0\\
59.92	0\\
59.93	1.73472347597681e-18\\
59.94	0\\
59.95	0\\
59.96	0\\
59.97	0\\
59.98	0\\
59.99	0\\
60	0\\
60.01	1.73472347597681e-18\\
60.02	0\\
60.03	0\\
60.04	0\\
60.05	0\\
60.06	0\\
60.07	0\\
60.08	0\\
60.09	0\\
60.1	0\\
60.11	0\\
60.12	0\\
60.13	1.73472347597681e-18\\
60.14	0\\
60.15	0\\
60.16	0\\
60.17	0\\
60.18	0\\
60.19	0\\
60.2	0\\
60.21	0\\
60.22	0\\
60.23	0\\
60.24	0\\
60.25	0\\
60.26	0\\
60.27	0\\
60.28	0\\
60.29	0\\
60.3	0\\
60.31	0\\
60.32	0\\
60.33	0\\
60.34	0\\
60.35	0\\
60.36	0\\
60.37	0\\
60.38	0\\
60.39	1.73472347597681e-18\\
60.4	0\\
60.41	0\\
60.42	0\\
60.43	0\\
60.44	0\\
60.45	0\\
60.46	0\\
60.47	0\\
60.48	0\\
60.49	1.73472347597681e-18\\
60.5	0\\
60.51	0\\
60.52	0\\
60.53	0\\
60.54	0\\
60.55	0\\
60.56	0\\
60.57	0\\
60.58	0\\
60.59	0\\
60.6	0\\
60.61	0\\
60.62	0\\
60.63	0\\
60.64	0\\
60.65	0\\
60.66	1.73472347597681e-18\\
60.67	0\\
60.68	0\\
60.69	0\\
60.7	0\\
60.71	0\\
60.72	0\\
60.73	0\\
60.74	0\\
60.75	0\\
60.76	0\\
60.77	1.73472347597681e-18\\
60.78	0\\
60.79	0\\
60.8	0\\
60.81	0\\
60.82	0\\
60.83	0\\
60.84	0\\
60.85	0\\
60.86	0\\
60.87	0\\
60.88	0\\
60.89	0\\
60.9	0\\
60.91	0\\
60.92	0\\
60.93	0\\
60.94	0\\
60.95	0\\
60.96	0\\
60.97	0\\
60.98	0\\
60.99	0\\
61	0\\
61.01	1.73472347597681e-18\\
61.02	0\\
61.03	0\\
61.04	0\\
61.05	0\\
61.06	1.73472347597681e-18\\
61.07	0\\
61.08	0\\
61.09	0\\
61.1	0\\
61.11	0\\
61.12	0\\
61.13	0\\
61.14	1.73472347597681e-18\\
61.15	0\\
61.16	0\\
61.17	0\\
61.18	0\\
61.19	0\\
61.2	0\\
61.21	0\\
61.22	0\\
61.23	0\\
61.24	0\\
61.25	0\\
61.26	0\\
61.27	0\\
61.28	1.73472347597681e-18\\
61.29	0\\
61.3	0\\
61.31	0\\
61.32	0\\
61.33	0\\
61.34	0\\
61.35	0\\
61.36	0\\
61.37	0\\
61.38	0\\
61.39	0\\
61.4	0\\
61.41	0\\
61.42	0\\
61.43	1.73472347597681e-18\\
61.44	0\\
61.45	0\\
61.46	1.73472347597681e-18\\
61.47	1.73472347597681e-18\\
61.48	0\\
61.49	1.73472347597681e-18\\
61.5	0\\
61.51	0\\
61.52	0\\
61.53	0\\
61.54	0\\
61.55	0\\
61.56	0\\
61.57	0\\
61.58	0\\
61.59	0\\
61.6	0\\
61.61	0\\
61.62	0\\
61.63	0\\
61.64	0\\
61.65	1.73472347597681e-18\\
61.66	0\\
61.67	1.73472347597681e-18\\
61.68	0\\
61.69	0\\
61.7	0\\
61.71	0\\
61.72	0\\
61.73	0\\
61.74	0\\
61.75	0\\
61.76	0\\
61.77	1.73472347597681e-18\\
61.78	0\\
61.79	0\\
61.8	0\\
61.81	0\\
61.82	1.73472347597681e-18\\
61.83	0\\
61.84	1.73472347597681e-18\\
61.85	0\\
61.86	0\\
61.87	0\\
61.88	1.73472347597681e-18\\
61.89	0\\
61.9	0\\
61.91	0\\
61.92	0\\
61.93	0\\
61.94	0\\
61.95	0\\
61.96	0\\
61.97	1.73472347597681e-18\\
61.98	0\\
61.99	0\\
62	0\\
62.01	0\\
62.02	0\\
62.03	0\\
62.04	1.73472347597681e-18\\
62.05	0\\
62.06	0\\
62.07	0\\
62.08	0\\
62.09	0\\
62.1	0\\
62.11	0\\
62.12	0\\
62.13	0\\
62.14	0\\
62.15	0\\
62.16	0\\
62.17	1.73472347597681e-18\\
62.18	0\\
62.19	0\\
62.2	0\\
62.21	0\\
62.22	0\\
62.23	0\\
62.24	0\\
62.25	0\\
62.26	0\\
62.27	0\\
62.28	0\\
62.29	0\\
62.3	0\\
62.31	0\\
62.32	0\\
62.33	0\\
62.34	0\\
62.35	0\\
62.36	0\\
62.37	0\\
62.38	0\\
62.39	0\\
62.4	0\\
62.41	0\\
62.42	0\\
62.43	0\\
62.44	0\\
62.45	0\\
62.46	1.73472347597681e-18\\
62.47	0\\
62.48	0\\
62.49	1.73472347597681e-18\\
62.5	0\\
62.51	0\\
62.52	0\\
62.53	0\\
62.54	0\\
62.55	0\\
62.56	0\\
62.57	0\\
62.58	0\\
62.59	0\\
62.6	0\\
62.61	0\\
62.62	0\\
62.63	0\\
62.64	0\\
62.65	0\\
62.66	0\\
62.67	0\\
62.68	0\\
62.69	0\\
62.7	0\\
62.71	0\\
62.72	0\\
62.73	0\\
62.74	0\\
62.75	0\\
62.76	0\\
62.77	1.73472347597681e-18\\
62.78	0\\
62.79	0\\
62.8	0\\
62.81	0\\
62.82	0\\
62.83	0\\
62.84	1.73472347597681e-18\\
62.85	0\\
62.86	0\\
62.87	0\\
62.88	0\\
62.89	0\\
62.9	0\\
62.91	0\\
62.92	0\\
62.93	0\\
62.94	0\\
62.95	0\\
62.96	0\\
62.97	0\\
62.98	0\\
62.99	1.73472347597681e-18\\
63	0\\
63.01	0\\
63.02	1.73472347597681e-18\\
63.03	0\\
63.04	0\\
63.05	1.73472347597681e-18\\
63.06	0\\
63.07	0\\
63.08	0\\
63.09	0\\
63.1	0\\
63.11	0\\
63.12	0\\
63.13	0\\
63.14	0\\
63.15	0\\
63.16	0\\
63.17	0\\
63.18	1.73472347597681e-18\\
63.19	0\\
63.2	0\\
63.21	0\\
63.22	0\\
63.23	0\\
63.24	0\\
63.25	0\\
63.26	0\\
63.27	0\\
63.28	0\\
63.29	0\\
63.3	0\\
63.31	0\\
63.32	0\\
63.33	0\\
63.34	0\\
63.35	0\\
63.36	0\\
63.37	1.73472347597681e-18\\
63.38	0\\
63.39	0\\
63.4	0\\
63.41	0\\
63.42	0\\
63.43	0\\
63.44	0\\
63.45	0\\
63.46	0\\
63.47	0\\
63.48	0\\
63.49	1.73472347597681e-18\\
63.5	0\\
63.51	0\\
63.52	0\\
63.53	0\\
63.54	0\\
63.55	0\\
63.56	0\\
63.57	0\\
63.58	0\\
63.59	0\\
63.6	0\\
63.61	0\\
63.62	0\\
63.63	0\\
63.64	0\\
63.65	0\\
63.66	0\\
63.67	0\\
63.68	0\\
63.69	0\\
63.7	0\\
63.71	0\\
63.72	0\\
63.73	0\\
63.74	0\\
63.75	0\\
63.76	0\\
63.77	0\\
63.78	0\\
63.79	0\\
63.8	1.73472347597681e-18\\
63.81	0\\
63.82	0\\
63.83	0\\
63.84	0\\
63.85	0\\
63.86	0\\
63.87	0\\
63.88	0\\
63.89	0\\
63.9	0\\
63.91	0\\
63.92	1.73472347597681e-18\\
63.93	0\\
63.94	0\\
63.95	0\\
63.96	1.73472347597681e-18\\
63.97	1.73472347597681e-18\\
63.98	0\\
63.99	0\\
64	0\\
64.01	0\\
64.02	0\\
64.03	0\\
64.04	0\\
64.05	0\\
64.06	1.73472347597681e-18\\
64.07	0\\
64.08	0\\
64.09	0\\
64.1	0\\
64.11	0\\
64.12	0\\
64.13	1.73472347597681e-18\\
64.14	0\\
64.15	0\\
64.16	0\\
64.17	0\\
64.18	0\\
64.19	0\\
64.2	0\\
64.21	0\\
64.22	0\\
64.23	0\\
64.24	0\\
64.25	0\\
64.26	0\\
64.27	0\\
64.28	0\\
64.29	1.73472347597681e-18\\
64.3	0\\
64.31	0\\
64.32	0\\
64.33	0\\
64.34	1.73472347597681e-18\\
64.35	0\\
64.36	0\\
64.37	0\\
64.38	0\\
64.39	0\\
64.4	0\\
64.41	0\\
64.42	1.73472347597681e-18\\
64.43	0\\
64.44	0\\
64.45	0\\
64.46	1.73472347597681e-18\\
64.47	0\\
64.48	0\\
64.49	0\\
64.5	0\\
64.51	0\\
64.52	0\\
64.53	0\\
64.54	0\\
64.55	0\\
64.56	0\\
64.57	0\\
64.58	0\\
64.59	1.73472347597681e-18\\
64.6	0\\
64.61	0\\
64.62	0\\
64.63	0\\
64.64	0\\
64.65	1.73472347597681e-18\\
64.66	0\\
64.67	0\\
64.68	0\\
64.69	0\\
64.7	0\\
64.71	0\\
64.72	0\\
64.73	0\\
64.74	0\\
64.75	1.73472347597681e-18\\
64.76	0\\
64.77	0\\
64.78	0\\
64.79	0\\
64.8	0\\
64.81	0\\
64.82	0\\
64.83	1.73472347597681e-18\\
64.84	1.73472347597681e-18\\
64.85	0\\
64.86	0\\
64.87	1.73472347597681e-18\\
64.88	0\\
64.89	0\\
64.9	0\\
64.91	0\\
64.92	0\\
64.93	0\\
64.94	0\\
64.95	0\\
64.96	0\\
64.97	1.73472347597681e-18\\
64.98	0\\
64.99	1.73472347597681e-18\\
65	0\\
65.01	0\\
65.02	0\\
65.03	0\\
65.04	0\\
65.05	0\\
65.06	0\\
65.07	1.73472347597681e-18\\
65.08	0\\
65.09	0\\
65.1	0\\
65.11	0\\
65.12	0\\
65.13	0\\
65.14	0\\
65.15	0\\
65.16	0\\
65.17	0\\
65.18	1.73472347597681e-18\\
65.19	0\\
65.2	0\\
65.21	0\\
65.22	0\\
65.23	0\\
65.24	0\\
65.25	0\\
65.26	0\\
65.27	0\\
65.28	0\\
65.29	0\\
65.3	0\\
65.31	0\\
65.32	0\\
65.33	0\\
65.34	0\\
65.35	0\\
65.36	0\\
65.37	1.73472347597681e-18\\
65.38	0\\
65.39	0\\
65.4	0\\
65.41	0\\
65.42	0\\
65.43	0\\
65.44	0\\
65.45	0\\
65.46	0\\
65.47	0\\
65.48	0\\
65.49	0\\
65.5	0\\
65.51	0\\
65.52	0\\
65.53	0\\
65.54	0\\
65.55	0\\
65.56	0\\
65.57	0\\
65.58	0\\
65.59	0\\
65.6	0\\
65.61	0\\
65.62	0\\
65.63	0\\
65.64	0\\
65.65	1.73472347597681e-18\\
65.66	0\\
65.67	0\\
65.68	1.73472347597681e-18\\
65.69	0\\
65.7	0\\
65.71	0\\
65.72	0\\
65.73	0\\
65.74	0\\
65.75	0\\
65.76	0\\
65.77	0\\
65.78	0\\
65.79	0\\
65.8	1.73472347597681e-18\\
65.81	1.73472347597681e-18\\
65.82	0\\
65.83	1.73472347597681e-18\\
65.84	0\\
65.85	0\\
65.86	1.73472347597681e-18\\
65.87	0\\
65.88	0\\
65.89	0\\
65.9	0\\
65.91	0\\
65.92	1.73472347597681e-18\\
65.93	1.73472347597681e-18\\
65.94	0\\
65.95	0\\
65.96	0\\
65.97	0\\
65.98	0\\
65.99	0\\
66	0\\
66.01	0\\
66.02	0\\
66.03	0\\
66.04	0\\
66.05	0\\
66.06	1.73472347597681e-18\\
66.07	0\\
66.08	0\\
66.09	0\\
66.1	0\\
66.11	0\\
66.12	0\\
66.13	0\\
66.14	0\\
66.15	1.73472347597681e-18\\
66.16	0\\
66.17	0\\
66.18	0\\
66.19	0\\
66.2	0\\
66.21	0\\
66.22	0\\
66.23	1.73472347597681e-18\\
66.24	0\\
66.25	0\\
66.26	0\\
66.27	0\\
66.28	0\\
66.29	1.73472347597681e-18\\
66.3	0\\
66.31	0\\
66.32	0\\
66.33	0\\
66.34	0\\
66.35	1.73472347597681e-18\\
66.36	0\\
66.37	1.73472347597681e-18\\
66.38	0\\
66.39	0\\
66.4	0\\
66.41	0\\
66.42	0\\
66.43	0\\
66.44	0\\
66.45	0\\
66.46	0\\
66.47	0\\
66.48	0\\
66.49	1.73472347597681e-18\\
66.5	0\\
66.51	0\\
66.52	0\\
66.53	0\\
66.54	0\\
66.55	0\\
66.56	0\\
66.57	1.73472347597681e-18\\
66.58	0\\
66.59	0\\
66.6	0\\
66.61	0\\
66.62	0\\
66.63	0\\
66.64	0\\
66.65	1.73472347597681e-18\\
66.66	0\\
66.67	0\\
66.68	1.73472347597681e-18\\
66.69	0\\
66.7	0\\
66.71	0\\
66.72	0\\
66.73	0\\
66.74	0\\
66.75	0\\
66.76	0\\
66.77	0\\
66.78	0\\
66.79	0\\
66.8	0\\
66.81	0\\
66.82	0\\
66.83	0\\
66.84	0\\
66.85	0\\
66.86	0\\
66.87	0\\
66.88	0\\
66.89	0\\
66.9	0\\
66.91	1.73472347597681e-18\\
66.92	0\\
66.93	0\\
66.94	1.73472347597681e-18\\
66.95	0\\
66.96	0\\
66.97	1.73472347597681e-18\\
66.98	0\\
66.99	0\\
67	0\\
67.01	0\\
67.02	1.73472347597681e-18\\
67.03	0\\
67.04	0\\
67.05	1.73472347597681e-18\\
67.06	0\\
67.07	0\\
67.08	1.73472347597681e-18\\
67.09	0\\
67.1	0\\
67.11	0\\
67.12	0\\
67.13	0\\
67.14	0\\
67.15	0\\
67.16	0\\
67.17	0\\
67.18	0\\
67.19	0\\
67.2	0\\
67.21	0\\
67.22	0\\
67.23	0\\
67.24	0\\
67.25	0\\
67.26	0\\
67.27	0\\
67.28	0\\
67.29	0\\
67.3	0\\
67.31	0\\
67.32	0\\
67.33	0\\
67.34	0\\
67.35	0\\
67.36	1.73472347597681e-18\\
67.37	0\\
67.38	0\\
67.39	0\\
67.4	0\\
67.41	1.73472347597681e-18\\
67.42	0\\
67.43	0\\
67.44	0\\
67.45	0\\
67.46	1.73472347597681e-18\\
67.47	0\\
67.48	0\\
67.49	0\\
67.5	1.73472347597681e-18\\
67.51	0\\
67.52	0\\
67.53	0\\
67.54	0\\
67.55	0\\
67.56	0\\
67.57	0\\
67.58	0\\
67.59	0\\
67.6	0\\
67.61	1.73472347597681e-18\\
67.62	0\\
67.63	0\\
67.64	1.73472347597681e-18\\
67.65	0\\
67.66	0\\
67.67	0\\
67.68	0\\
67.69	0\\
67.7	0\\
67.71	0\\
67.72	0\\
67.73	0\\
67.74	0\\
67.75	0\\
67.76	0\\
67.77	0\\
67.78	0\\
67.79	0\\
67.8	0\\
67.81	0\\
67.82	0\\
67.83	0\\
67.84	0\\
67.85	1.73472347597681e-18\\
67.86	0\\
67.87	0\\
67.88	0\\
67.89	0\\
67.9	0\\
67.91	0\\
67.92	0\\
67.93	0\\
67.94	0\\
67.95	0\\
67.96	0\\
67.97	0\\
67.98	0\\
67.99	0\\
68	0\\
68.01	0\\
68.02	1.73472347597681e-18\\
68.03	0\\
68.04	0\\
68.05	0\\
68.06	0\\
68.07	0\\
68.08	0\\
68.09	1.73472347597681e-18\\
68.1	0\\
68.11	0\\
68.12	0\\
68.13	0\\
68.14	0\\
68.15	0\\
68.16	0\\
68.17	1.73472347597681e-18\\
68.18	0\\
68.19	1.73472347597681e-18\\
68.2	0\\
68.21	0\\
68.22	0\\
68.23	0\\
68.24	0\\
68.25	0\\
68.26	0\\
68.27	0\\
68.28	0\\
68.29	0\\
68.3	1.73472347597681e-18\\
68.31	0\\
68.32	0\\
68.33	0\\
68.34	0\\
68.35	0\\
68.36	1.73472347597681e-18\\
68.37	0\\
68.38	0\\
68.39	0\\
68.4	0\\
68.41	0\\
68.42	0\\
68.43	1.73472347597681e-18\\
68.44	0\\
68.45	0\\
68.46	0\\
68.47	0\\
68.48	0\\
68.49	0\\
68.5	0\\
68.51	0\\
68.52	0\\
68.53	0\\
68.54	0\\
68.55	0\\
68.56	0\\
68.57	0\\
68.58	0\\
68.59	0\\
68.6	0\\
68.61	0\\
68.62	0\\
68.63	0\\
68.64	0\\
68.65	0\\
68.66	0\\
68.67	0\\
68.68	0\\
68.69	0\\
68.7	0\\
68.71	1.73472347597681e-18\\
68.72	0\\
68.73	0\\
68.74	0\\
68.75	0\\
68.76	0\\
68.77	0\\
68.78	0\\
68.79	0\\
68.8	0\\
68.81	0\\
68.82	0\\
68.83	0\\
68.84	0\\
68.85	0\\
68.86	0\\
68.87	0\\
68.88	0\\
68.89	0\\
68.9	0\\
68.91	0\\
68.92	0\\
68.93	0\\
68.94	0\\
68.95	0\\
68.96	0\\
68.97	0\\
68.98	0\\
68.99	0\\
69	1.73472347597681e-18\\
69.01	0\\
69.02	0\\
69.03	0\\
69.04	0\\
69.05	1.73472347597681e-18\\
69.06	0\\
69.07	0\\
69.08	0\\
69.09	0\\
69.1	0\\
69.11	0\\
69.12	0\\
69.13	1.73472347597681e-18\\
69.14	0\\
69.15	1.73472347597681e-18\\
69.16	0\\
69.17	0\\
69.18	0\\
69.19	1.73472347597681e-18\\
69.2	0\\
69.21	1.73472347597681e-18\\
69.22	0\\
69.23	0\\
69.24	0\\
69.25	0\\
69.26	0\\
69.27	0\\
69.28	0\\
69.29	0\\
69.3	0\\
69.31	0\\
69.32	0\\
69.33	0\\
69.34	0\\
69.35	0\\
69.36	0\\
69.37	1.73472347597681e-18\\
69.38	0\\
69.39	0\\
69.4	0\\
69.41	0\\
69.42	0\\
69.43	1.73472347597681e-18\\
69.44	0\\
69.45	1.73472347597681e-18\\
69.46	0\\
69.47	0\\
69.48	0\\
69.49	0\\
69.5	0\\
69.51	0\\
69.52	0\\
69.53	0\\
69.54	0\\
69.55	0\\
69.56	0\\
69.57	1.73472347597681e-18\\
69.58	1.73472347597681e-18\\
69.59	0\\
69.6	0\\
69.61	0\\
69.62	0\\
69.63	0\\
69.64	1.73472347597681e-18\\
69.65	0\\
69.66	0\\
69.67	1.73472347597681e-18\\
69.68	0\\
69.69	0\\
69.7	0\\
69.71	0\\
69.72	0\\
69.73	0\\
69.74	0\\
69.75	0\\
69.76	0\\
69.77	0\\
69.78	0\\
69.79	0\\
69.8	0\\
69.81	0\\
69.82	0\\
69.83	0\\
69.84	0\\
69.85	0\\
69.86	0\\
69.87	0\\
69.88	1.73472347597681e-18\\
69.89	0\\
69.9	0\\
69.91	0\\
69.92	0\\
69.93	0\\
69.94	0\\
69.95	1.73472347597681e-18\\
69.96	0\\
69.97	0\\
69.98	0\\
69.99	0\\
70	0\\
70.01	0\\
70.02	0\\
70.03	0\\
70.04	0\\
70.05	0\\
70.06	0\\
70.07	0\\
70.08	0\\
70.09	0\\
70.1	1.73472347597681e-18\\
70.11	0\\
70.12	0\\
70.13	0\\
70.14	1.73472347597681e-18\\
70.15	0\\
70.16	0\\
70.17	0\\
70.18	0\\
70.19	0\\
70.2	0\\
70.21	0\\
70.22	0\\
70.23	0\\
70.24	0\\
70.25	0\\
70.26	0\\
70.27	0\\
70.28	0\\
70.29	0\\
70.3	0\\
70.31	0\\
70.32	0\\
70.33	1.73472347597681e-18\\
70.34	0\\
70.35	0\\
70.36	0\\
70.37	0\\
70.38	0\\
70.39	0\\
70.4	1.73472347597681e-18\\
70.41	0\\
70.42	0\\
70.43	0\\
70.44	0\\
70.45	0\\
70.46	1.73472347597681e-18\\
70.47	0\\
70.48	0\\
70.49	0\\
70.5	0\\
70.51	0\\
70.52	0\\
70.53	0\\
70.54	0\\
70.55	0\\
70.56	0\\
70.57	0\\
70.58	0\\
70.59	0\\
70.6	0\\
70.61	0\\
70.62	0\\
70.63	0\\
70.64	0\\
70.65	1.73472347597681e-18\\
70.66	0\\
70.67	0\\
70.68	0\\
70.69	0\\
70.7	0\\
70.71	1.73472347597681e-18\\
70.72	0\\
70.73	0\\
70.74	0\\
70.75	0\\
70.76	0\\
70.77	0\\
70.78	0\\
70.79	0\\
70.8	0\\
70.81	0\\
70.82	1.73472347597681e-18\\
70.83	0\\
70.84	0\\
70.85	1.73472347597681e-18\\
70.86	0\\
70.87	0\\
70.88	0\\
70.89	0\\
70.9	0\\
70.91	0\\
70.92	1.73472347597681e-18\\
70.93	0\\
70.94	0\\
70.95	0\\
70.96	0\\
70.97	0\\
70.98	0\\
70.99	0\\
71	0\\
71.01	0\\
71.02	0\\
71.03	0\\
71.04	0\\
71.05	0\\
71.06	0\\
71.07	0\\
71.08	1.73472347597681e-18\\
71.09	1.73472347597681e-18\\
71.1	0\\
71.11	0\\
71.12	0\\
71.13	0\\
71.14	0\\
71.15	0\\
71.16	0\\
71.17	0\\
71.18	1.73472347597681e-18\\
71.19	0\\
71.2	0\\
71.21	0\\
71.22	0\\
71.23	0\\
71.24	0\\
71.25	0\\
71.26	0\\
71.27	1.73472347597681e-18\\
71.28	0\\
71.29	0\\
71.3	0\\
71.31	0\\
71.32	1.73472347597681e-18\\
71.33	0\\
71.34	0\\
71.35	0\\
71.36	1.73472347597681e-18\\
71.37	0\\
71.38	0\\
71.39	0\\
71.4	0\\
71.41	0\\
71.42	0\\
71.43	1.73472347597681e-18\\
71.44	0\\
71.45	0\\
71.46	0\\
71.47	0\\
71.48	0\\
71.49	0\\
71.5	0\\
71.51	0\\
71.52	1.73472347597681e-18\\
71.53	0\\
71.54	0\\
71.55	0\\
71.56	0\\
71.57	0\\
71.58	0\\
71.59	0\\
71.6	0\\
71.61	1.73472347597681e-18\\
71.62	0\\
71.63	1.73472347597681e-18\\
71.64	0\\
71.65	0\\
71.66	0\\
71.67	0\\
71.68	0\\
71.69	0\\
71.7	0\\
71.71	1.73472347597681e-18\\
71.72	0\\
71.73	0\\
71.74	0\\
71.75	0\\
71.76	0\\
71.77	0\\
71.78	0\\
71.79	0\\
71.8	0\\
71.81	0\\
71.82	0\\
71.83	1.73472347597681e-18\\
71.84	0\\
71.85	0\\
71.86	0\\
71.87	0\\
71.88	0\\
71.89	1.73472347597681e-18\\
71.9	0\\
71.91	0\\
71.92	0\\
71.93	0\\
71.94	0\\
71.95	0\\
71.96	1.73472347597681e-18\\
71.97	0\\
71.98	0\\
71.99	0\\
72	0\\
72.01	0\\
72.02	0\\
72.03	0\\
72.04	0\\
72.05	0\\
72.06	0\\
72.07	0\\
72.08	0\\
72.09	0\\
72.1	0\\
72.11	0\\
72.12	1.73472347597681e-18\\
72.13	0\\
72.14	0\\
72.15	0\\
72.16	0\\
72.17	0\\
72.18	0\\
72.19	1.73472347597681e-18\\
72.2	0\\
72.21	0\\
72.22	0\\
72.23	0\\
72.24	0\\
72.25	0\\
72.26	0\\
72.27	0\\
72.28	0\\
72.29	0\\
72.3	0\\
72.31	0\\
72.32	0\\
72.33	0\\
72.34	0\\
72.35	0\\
72.36	0\\
72.37	0\\
72.38	0\\
72.39	0\\
72.4	0\\
72.41	0\\
72.42	0\\
72.43	0\\
72.44	0\\
72.45	1.73472347597681e-18\\
72.46	0\\
72.47	0\\
72.48	0\\
72.49	0\\
72.5	0\\
72.51	0\\
72.52	0\\
72.53	0\\
72.54	0\\
72.55	1.73472347597681e-18\\
72.56	0\\
72.57	0\\
72.58	0\\
72.59	0\\
72.6	0\\
72.61	0\\
72.62	1.73472347597681e-18\\
72.63	0\\
72.64	0\\
72.65	0\\
72.66	0\\
72.67	0\\
72.68	0\\
72.69	0\\
72.7	1.73472347597681e-18\\
72.71	0\\
72.72	0\\
72.73	1.73472347597681e-18\\
72.74	0\\
72.75	0\\
72.76	0\\
72.77	0\\
72.78	1.73472347597681e-18\\
72.79	0\\
72.8	0\\
72.81	0\\
72.82	0\\
72.83	0\\
72.84	0\\
72.85	1.73472347597681e-18\\
72.86	0\\
72.87	0\\
72.88	0\\
72.89	1.73472347597681e-18\\
72.9	0\\
72.91	0\\
72.92	0\\
72.93	0\\
72.94	0\\
72.95	0\\
72.96	0\\
72.97	0\\
72.98	0\\
72.99	0\\
73	0\\
73.01	0\\
73.02	1.73472347597681e-18\\
73.03	0\\
73.04	0\\
73.05	0\\
73.06	0\\
73.07	0\\
73.08	0\\
73.09	0\\
73.1	0\\
73.11	0\\
73.12	0\\
73.13	1.73472347597681e-18\\
73.14	0\\
73.15	0\\
73.16	0\\
73.17	0\\
73.18	0\\
73.19	0\\
73.2	0\\
73.21	0\\
73.22	0\\
73.23	1.73472347597681e-18\\
73.24	0\\
73.25	0\\
73.26	0\\
73.27	1.73472347597681e-18\\
73.28	0\\
73.29	0\\
73.3	0\\
73.31	0\\
73.32	0\\
73.33	0\\
73.34	0\\
73.35	0\\
73.36	0\\
73.37	0\\
73.38	0\\
73.39	1.73472347597681e-18\\
73.4	0\\
73.41	0\\
73.42	0\\
73.43	0\\
73.44	0\\
73.45	0\\
73.46	0\\
73.47	0\\
73.48	0\\
73.49	0\\
73.5	0\\
73.51	1.73472347597681e-18\\
73.52	1.73472347597681e-18\\
73.53	0\\
73.54	0\\
73.55	1.73472347597681e-18\\
73.56	0\\
73.57	0\\
73.58	0\\
73.59	0\\
73.6	0\\
73.61	0\\
73.62	0\\
73.63	0\\
73.64	0\\
73.65	0\\
73.66	1.73472347597681e-18\\
73.67	0\\
73.68	0\\
73.69	0\\
73.7	0\\
73.71	0\\
73.72	0\\
73.73	1.73472347597681e-18\\
73.74	0\\
73.75	0\\
73.76	1.73472347597681e-18\\
73.77	0\\
73.78	0\\
73.79	0\\
73.8	1.73472347597681e-18\\
73.81	0\\
73.82	0\\
73.83	0\\
73.84	0\\
73.85	0\\
73.86	0\\
73.87	0\\
73.88	0\\
73.89	0\\
73.9	0\\
73.91	0\\
73.92	0\\
73.93	0\\
73.94	0\\
73.95	0\\
73.96	0\\
73.97	0\\
73.98	1.73472347597681e-18\\
73.99	0\\
74	0\\
74.01	1.73472347597681e-18\\
74.02	0\\
74.03	0\\
74.04	0\\
74.05	0\\
74.06	0\\
74.07	0\\
74.08	0\\
74.09	0\\
74.1	1.73472347597681e-18\\
74.11	1.73472347597681e-18\\
74.12	1.73472347597681e-18\\
74.13	0\\
74.14	1.73472347597681e-18\\
74.15	0\\
74.16	0\\
74.17	1.73472347597681e-18\\
74.18	0\\
74.19	0\\
74.2	0\\
74.21	0\\
74.22	0\\
74.23	0\\
74.24	0\\
74.25	0\\
74.26	0\\
74.27	0\\
74.28	0\\
74.29	0\\
74.3	1.73472347597681e-18\\
74.31	0\\
74.32	0\\
74.33	0\\
74.34	0\\
74.35	0\\
74.36	0\\
74.37	1.73472347597681e-18\\
74.38	0\\
74.39	0\\
74.4	0\\
74.41	0\\
74.42	0\\
74.43	0\\
74.44	0\\
74.45	0\\
74.46	0\\
74.47	0\\
74.48	0\\
74.49	0\\
74.5	0\\
74.51	0\\
74.52	0\\
74.53	0\\
74.54	0\\
74.55	0\\
74.56	0\\
74.57	0\\
74.58	0\\
74.59	0\\
74.6	1.73472347597681e-18\\
74.61	0\\
74.62	1.73472347597681e-18\\
74.63	0\\
74.64	0\\
74.65	1.73472347597681e-18\\
74.66	0\\
74.67	0\\
74.68	0\\
74.69	0\\
74.7	1.73472347597681e-18\\
74.71	0\\
74.72	0\\
74.73	0\\
74.74	0\\
74.75	0\\
74.76	0\\
74.77	0\\
74.78	0\\
74.79	0\\
74.8	0\\
74.81	0\\
74.82	0\\
74.83	1.73472347597681e-18\\
74.84	0\\
74.85	0\\
74.86	0\\
74.87	1.73472347597681e-18\\
74.88	0\\
74.89	0\\
74.9	0\\
74.91	0\\
74.92	0\\
74.93	0\\
74.94	0\\
74.95	0\\
74.96	0\\
74.97	0\\
74.98	0\\
74.99	0\\
75	1.73472347597681e-18\\
75.01	0\\
75.02	0\\
75.03	1.73472347597681e-18\\
75.04	0\\
75.05	0\\
75.06	1.73472347597681e-18\\
75.07	0\\
75.08	0\\
75.09	0\\
75.1	0\\
75.11	0\\
75.12	0\\
75.13	0\\
75.14	1.73472347597681e-18\\
75.15	0\\
75.16	0\\
75.17	0\\
75.18	0\\
75.19	0\\
75.2	0\\
75.21	0\\
75.22	0\\
75.23	0\\
75.24	0\\
75.25	0\\
75.26	0\\
75.27	0\\
75.28	0\\
75.29	0\\
75.3	0\\
75.31	0\\
75.32	0\\
75.33	0\\
75.34	0\\
75.35	0\\
75.36	1.73472347597681e-18\\
75.37	0\\
75.38	0\\
75.39	0\\
75.4	0\\
75.41	0\\
75.42	0\\
75.43	0\\
75.44	1.73472347597681e-18\\
75.45	0\\
75.46	1.73472347597681e-18\\
75.47	0\\
75.48	0\\
75.49	0\\
75.5	0\\
75.51	0\\
75.52	0\\
75.53	0\\
75.54	0\\
75.55	0\\
75.56	1.73472347597681e-18\\
75.57	0\\
75.58	0\\
75.59	0\\
75.6	0\\
75.61	0\\
75.62	0\\
75.63	0\\
75.64	0\\
75.65	0\\
75.66	0\\
75.67	0\\
75.68	0\\
75.69	0\\
75.7	0\\
75.71	0\\
75.72	0\\
75.73	1.73472347597681e-18\\
75.74	0\\
75.75	0\\
75.76	0\\
75.77	0\\
75.78	0\\
75.79	0\\
75.8	0\\
75.81	0\\
75.82	0\\
75.83	0\\
75.84	0\\
75.85	0\\
75.86	0\\
75.87	0\\
75.88	0\\
75.89	0\\
75.9	0\\
75.91	0\\
75.92	0\\
75.93	1.73472347597681e-18\\
75.94	0\\
75.95	0\\
75.96	0\\
75.97	0\\
75.98	0\\
75.99	0\\
76	0\\
76.01	0\\
76.02	1.73472347597681e-18\\
76.03	0\\
76.04	0\\
76.05	0\\
76.06	0\\
76.07	0\\
76.08	0\\
76.09	0\\
76.1	0\\
76.11	0\\
76.12	0\\
76.13	1.73472347597681e-18\\
76.14	0\\
76.15	0\\
76.16	1.73472347597681e-18\\
76.17	1.73472347597681e-18\\
76.18	0\\
76.19	0\\
76.2	0\\
76.21	0\\
76.22	0\\
76.23	0\\
76.24	0\\
76.25	0\\
76.26	0\\
76.27	0\\
76.28	0\\
76.29	0\\
76.3	0\\
76.31	0\\
76.32	0\\
76.33	1.73472347597681e-18\\
76.34	0\\
76.35	0\\
76.36	0\\
76.37	0\\
76.38	1.73472347597681e-18\\
76.39	0\\
76.4	0\\
76.41	1.73472347597681e-18\\
76.42	1.73472347597681e-18\\
76.43	0\\
76.44	0\\
76.45	0\\
76.46	1.73472347597681e-18\\
76.47	0\\
76.48	0\\
76.49	0\\
76.5	0\\
76.51	0\\
76.52	1.73472347597681e-18\\
76.53	0\\
76.54	1.73472347597681e-18\\
76.55	1.73472347597681e-18\\
76.56	0\\
76.57	0\\
76.58	0\\
76.59	0\\
76.6	0\\
76.61	0\\
76.62	0\\
76.63	0\\
76.64	0\\
76.65	0\\
76.66	0\\
76.67	0\\
76.68	0\\
76.69	0\\
76.7	0\\
76.71	0\\
76.72	0\\
76.73	0\\
76.74	0\\
76.75	0\\
76.76	0\\
76.77	0\\
76.78	0\\
76.79	0\\
76.8	0\\
76.81	0\\
76.82	0\\
76.83	0\\
76.84	0\\
76.85	0\\
76.86	0\\
76.87	0\\
76.88	0\\
76.89	1.73472347597681e-18\\
76.9	0\\
76.91	0\\
76.92	0\\
76.93	0\\
76.94	1.73472347597681e-18\\
76.95	0\\
76.96	0\\
76.97	0\\
76.98	0\\
76.99	0\\
77	0\\
77.01	0\\
77.02	0\\
77.03	0\\
77.04	1.73472347597681e-18\\
77.05	0\\
77.06	0\\
77.07	0\\
77.08	0\\
77.09	0\\
77.1	0\\
77.11	0\\
77.12	0\\
77.13	0\\
77.14	0\\
77.15	0\\
77.16	1.73472347597681e-18\\
77.17	0\\
77.18	0\\
77.19	1.73472347597681e-18\\
77.2	0\\
77.21	0\\
77.22	0\\
77.23	0\\
77.24	0\\
77.25	0\\
77.26	0\\
77.27	0\\
77.28	0\\
77.29	0\\
77.3	0\\
77.31	0\\
77.32	0\\
77.33	0\\
77.34	0\\
77.35	0\\
77.36	0\\
77.37	0\\
77.38	0\\
77.39	0\\
77.4	0\\
77.41	1.73472347597681e-18\\
77.42	0\\
77.43	0\\
77.44	0\\
77.45	0\\
77.46	0\\
77.47	0\\
77.48	0\\
77.49	1.73472347597681e-18\\
77.5	0\\
77.51	0\\
77.52	0\\
77.53	0\\
77.54	1.73472347597681e-18\\
77.55	0\\
77.56	0\\
77.57	0\\
77.58	0\\
77.59	0\\
77.6	0\\
77.61	0\\
77.62	0\\
77.63	0\\
77.64	0\\
77.65	1.73472347597681e-18\\
77.66	0\\
77.67	0\\
77.68	0\\
77.69	0\\
77.7	0\\
77.71	0\\
77.72	0\\
77.73	0\\
77.74	0\\
77.75	1.73472347597681e-18\\
77.76	0\\
77.77	1.73472347597681e-18\\
77.78	0\\
77.79	1.73472347597681e-18\\
77.8	0\\
77.81	0\\
77.82	0\\
77.83	0\\
77.84	0\\
77.85	1.73472347597681e-18\\
77.86	0\\
77.87	0\\
77.88	0\\
77.89	0\\
77.9	0\\
77.91	0\\
77.92	0\\
77.93	0\\
77.94	0\\
77.95	0\\
77.96	0\\
77.97	0\\
77.98	0\\
77.99	0\\
78	0\\
78.01	0\\
78.02	0\\
78.03	0\\
78.04	1.73472347597681e-18\\
78.05	0\\
78.06	0\\
78.07	0\\
78.08	0\\
78.09	0\\
78.1	1.73472347597681e-18\\
78.11	0\\
78.12	0\\
78.13	0\\
78.14	0\\
78.15	0\\
78.16	0\\
78.17	0\\
78.18	0\\
78.19	0\\
78.2	1.73472347597681e-18\\
78.21	1.73472347597681e-18\\
78.22	0\\
78.23	0\\
78.24	0\\
78.25	1.73472347597681e-18\\
78.26	0\\
78.27	0\\
78.28	0\\
78.29	1.73472347597681e-18\\
78.3	0\\
78.31	0\\
78.32	0\\
78.33	0\\
78.34	0\\
78.35	0\\
78.36	0\\
78.37	0\\
78.38	0\\
78.39	0\\
78.4	1.73472347597681e-18\\
78.41	1.73472347597681e-18\\
78.42	0\\
78.43	0\\
78.44	0\\
78.45	0\\
78.46	0\\
78.47	0\\
78.48	0\\
78.49	0\\
78.5	0\\
78.51	0\\
78.52	0\\
78.53	0\\
78.54	1.73472347597681e-18\\
78.55	0\\
78.56	0\\
78.57	0\\
78.58	0\\
78.59	0\\
78.6	0\\
78.61	0\\
78.62	1.73472347597681e-18\\
78.63	0\\
78.64	0\\
78.65	0\\
78.66	0\\
78.67	0\\
78.68	0\\
78.69	0\\
78.7	0\\
78.71	0\\
78.72	0\\
78.73	0\\
78.74	0\\
78.75	0\\
78.76	0\\
78.77	0\\
78.78	0\\
78.79	0\\
78.8	0\\
78.81	0\\
78.82	0\\
78.83	0\\
78.84	0\\
78.85	0\\
78.86	0\\
78.87	1.73472347597681e-18\\
78.88	0\\
78.89	0\\
78.9	1.73472347597681e-18\\
78.91	0\\
78.92	1.73472347597681e-18\\
78.93	0\\
78.94	0\\
78.95	1.73472347597681e-18\\
78.96	0\\
78.97	0\\
78.98	1.73472347597681e-18\\
78.99	0\\
79	0\\
79.01	0\\
79.02	0\\
79.03	0\\
79.04	0\\
79.05	0\\
79.06	0\\
79.07	0\\
79.08	1.73472347597681e-18\\
79.09	0\\
79.1	0\\
79.11	0\\
79.12	0\\
79.13	0\\
79.14	0\\
79.15	0\\
79.16	0\\
79.17	0\\
79.18	0\\
79.19	0\\
79.2	0\\
79.21	0\\
79.22	0\\
79.23	0\\
79.24	0\\
79.25	0\\
79.26	0\\
79.27	0\\
79.28	0\\
79.29	0\\
79.3	1.73472347597681e-18\\
79.31	0\\
79.32	0\\
79.33	0\\
79.34	0\\
79.35	1.73472347597681e-18\\
79.36	0\\
79.37	0\\
79.38	0\\
79.39	0\\
79.4	0\\
79.41	0\\
79.42	0\\
79.43	0\\
79.44	0\\
79.45	0\\
79.46	1.73472347597681e-18\\
79.47	0\\
79.48	0\\
79.49	0\\
79.5	0\\
79.51	0\\
79.52	0\\
79.53	1.73472347597681e-18\\
79.54	0\\
79.55	0\\
79.56	1.73472347597681e-18\\
79.57	1.73472347597681e-18\\
79.58	0\\
79.59	1.73472347597681e-18\\
79.6	0\\
79.61	0\\
79.62	0\\
79.63	0\\
79.64	0\\
79.65	0\\
79.66	0\\
79.67	0\\
79.68	0\\
79.69	0\\
79.7	0\\
79.71	0\\
79.72	0\\
79.73	0\\
79.74	0\\
79.75	0\\
79.76	0\\
79.77	0\\
79.78	0\\
79.79	1.73472347597681e-18\\
79.8	0\\
79.81	0\\
79.82	1.73472347597681e-18\\
79.83	0\\
79.84	0\\
79.85	0\\
79.86	0\\
79.87	1.73472347597681e-18\\
79.88	0\\
79.89	0\\
79.9	0\\
79.91	0\\
79.92	0\\
79.93	0\\
79.94	1.73472347597681e-18\\
79.95	0\\
79.96	0\\
79.97	1.73472347597681e-18\\
79.98	1.73472347597681e-18\\
79.99	1.73472347597681e-18\\
80	0\\
80.01	0\\
};
\addplot [color=mycolor1,dashed]
  table[row sep=crcr]{%
80.01	0\\
80.02	0\\
80.03	1.73472347597681e-18\\
80.04	0\\
80.05	0\\
80.06	0\\
80.07	0\\
80.08	0\\
80.09	0\\
80.1	0\\
80.11	1.73472347597681e-18\\
80.12	0\\
80.13	1.73472347597681e-18\\
80.14	0\\
80.15	0\\
80.16	0\\
80.17	1.73472347597681e-18\\
80.18	1.73472347597681e-18\\
80.19	0\\
80.2	0\\
80.21	1.73472347597681e-18\\
80.22	0\\
80.23	1.73472347597681e-18\\
80.24	0\\
80.25	1.73472347597681e-18\\
80.26	0\\
80.27	0\\
80.28	0\\
80.29	0\\
80.3	0\\
80.31	0\\
80.32	0\\
80.33	0\\
80.34	0\\
80.35	0\\
80.36	0\\
80.37	1.73472347597681e-18\\
80.38	0\\
80.39	0\\
80.4	1.73472347597681e-18\\
80.41	0\\
80.42	0\\
80.43	1.73472347597681e-18\\
80.44	0\\
80.45	0\\
80.46	1.73472347597681e-18\\
80.47	0\\
80.48	0\\
80.49	0\\
80.5	0\\
80.51	0\\
80.52	0\\
80.53	0\\
80.54	0\\
80.55	0\\
80.56	1.73472347597681e-18\\
80.57	0\\
80.58	0\\
80.59	0\\
80.6	1.73472347597681e-18\\
80.61	0\\
80.62	0\\
80.63	0\\
80.64	0\\
80.65	0\\
80.66	0\\
80.67	0\\
80.68	0\\
80.69	1.73472347597681e-18\\
80.7	1.73472347597681e-18\\
80.71	0\\
80.72	0\\
80.73	1.73472347597681e-18\\
80.74	0\\
80.75	0\\
80.76	0\\
80.77	0\\
80.78	0\\
80.79	0\\
80.8	0\\
80.81	1.73472347597681e-18\\
80.82	0\\
80.83	0\\
80.84	0\\
80.85	0\\
80.86	0\\
80.87	0\\
80.88	0\\
80.89	0\\
80.9	1.73472347597681e-18\\
80.91	0\\
80.92	0\\
80.93	0\\
80.94	0\\
80.95	1.73472347597681e-18\\
80.96	0\\
80.97	0\\
80.98	0\\
80.99	0\\
81	1.73472347597681e-18\\
81.01	0\\
81.02	0\\
81.03	0\\
81.04	0\\
81.05	0\\
81.06	1.73472347597681e-18\\
81.07	1.73472347597681e-18\\
81.08	0\\
81.09	0\\
81.1	0\\
81.11	0\\
81.12	0\\
81.13	0\\
81.14	0\\
81.15	0\\
81.16	0\\
81.17	0\\
81.18	0\\
81.19	0\\
81.2	0\\
81.21	0\\
81.22	0\\
81.23	0\\
81.24	0\\
81.25	0\\
81.26	0\\
81.27	0\\
81.28	1.73472347597681e-18\\
81.29	1.73472347597681e-18\\
81.3	0\\
81.31	0\\
81.32	0\\
81.33	0\\
81.34	1.73472347597681e-18\\
81.35	0\\
81.36	0\\
81.37	0\\
81.38	0\\
81.39	0\\
81.4	1.73472347597681e-18\\
81.41	0\\
81.42	0\\
81.43	0\\
81.44	0\\
81.45	0\\
81.46	0\\
81.47	0\\
81.48	0\\
81.49	1.73472347597681e-18\\
81.5	0\\
81.51	0\\
81.52	0\\
81.53	0\\
81.54	1.73472347597681e-18\\
81.55	0\\
81.56	0\\
81.57	0\\
81.58	0\\
81.59	0\\
81.6	0\\
81.61	0\\
81.62	0\\
81.63	0\\
81.64	0\\
81.65	0\\
81.66	0\\
81.67	0\\
81.68	1.73472347597681e-18\\
81.69	0\\
81.7	0\\
81.71	1.73472347597681e-18\\
81.72	0\\
81.73	0\\
81.74	0\\
81.75	0\\
81.76	1.73472347597681e-18\\
81.77	0\\
81.78	0\\
81.79	1.73472347597681e-18\\
81.8	0\\
81.81	0\\
81.82	0\\
81.83	0\\
81.84	0\\
81.85	0\\
81.86	0\\
81.87	0\\
81.88	0\\
81.89	0\\
81.9	0\\
81.91	0\\
81.92	0\\
81.93	0\\
81.94	1.73472347597681e-18\\
81.95	0\\
81.96	1.73472347597681e-18\\
81.97	0\\
81.98	0\\
81.99	1.73472347597681e-18\\
82	0\\
82.01	1.73472347597681e-18\\
82.02	1.73472347597681e-18\\
82.03	1.73472347597681e-18\\
82.04	0\\
82.05	1.73472347597681e-18\\
82.06	0\\
82.07	1.73472347597681e-18\\
82.08	0\\
82.09	0\\
82.1	0\\
82.11	0\\
82.12	0\\
82.13	0\\
82.14	0\\
82.15	0\\
82.16	0\\
82.17	0\\
82.18	0\\
82.19	0\\
82.2	0\\
82.21	0\\
82.22	0\\
82.23	0\\
82.24	0\\
82.25	0\\
82.26	0\\
82.27	0\\
82.28	0\\
82.29	0\\
82.3	0\\
82.31	0\\
82.32	0\\
82.33	1.73472347597681e-18\\
82.34	0\\
82.35	1.73472347597681e-18\\
82.36	0\\
82.37	0\\
82.38	0\\
82.39	0\\
82.4	0\\
82.41	0\\
82.42	0\\
82.43	0\\
82.44	0\\
82.45	0\\
82.46	0\\
82.47	0\\
82.48	0\\
82.49	0\\
82.5	1.73472347597681e-18\\
82.51	0\\
82.52	0\\
82.53	0\\
82.54	0\\
82.55	0\\
82.56	0\\
82.57	0\\
82.58	0\\
82.59	0\\
82.6	0\\
82.61	0\\
82.62	0\\
82.63	1.73472347597681e-18\\
82.64	1.73472347597681e-18\\
82.65	1.73472347597681e-18\\
82.66	0\\
82.67	0\\
82.68	0\\
82.69	0\\
82.7	0\\
82.71	0\\
82.72	0\\
82.73	0\\
82.74	0\\
82.75	0\\
82.76	0\\
82.77	1.73472347597681e-18\\
82.78	0\\
82.79	0\\
82.8	0\\
82.81	0\\
82.82	0\\
82.83	0\\
82.84	0\\
82.85	1.73472347597681e-18\\
82.86	0\\
82.87	0\\
82.88	0\\
82.89	1.73472347597681e-18\\
82.9	0\\
82.91	0\\
82.92	0\\
82.93	0\\
82.94	0\\
82.95	0\\
82.96	0\\
82.97	0\\
82.98	0\\
82.99	0\\
83	0\\
83.01	0\\
83.02	0\\
83.03	0\\
83.04	0\\
83.05	1.73472347597681e-18\\
83.06	0\\
83.07	0\\
83.08	0\\
83.09	0\\
83.1	1.73472347597681e-18\\
83.11	1.73472347597681e-18\\
83.12	1.73472347597681e-18\\
83.13	0\\
83.14	0\\
83.15	0\\
83.16	0\\
83.17	0\\
83.18	0\\
83.19	0\\
83.2	1.73472347597681e-18\\
83.21	0\\
83.22	0\\
83.23	1.73472347597681e-18\\
83.24	0\\
83.25	0\\
83.26	1.73472347597681e-18\\
83.27	0\\
83.28	0\\
83.29	1.73472347597681e-18\\
83.3	0\\
83.31	0\\
83.32	0\\
83.33	0\\
83.34	0\\
83.35	0\\
83.36	0\\
83.37	0\\
83.38	0\\
83.39	0\\
83.4	0\\
83.41	0\\
83.42	0\\
83.43	0\\
83.44	0\\
83.45	0\\
83.46	0\\
83.47	1.73472347597681e-18\\
83.48	0\\
83.49	1.73472347597681e-18\\
83.5	0\\
83.51	0\\
83.52	0\\
83.53	0\\
83.54	0\\
83.55	0\\
83.56	0\\
83.57	0\\
83.58	0\\
83.59	0\\
83.6	1.73472347597681e-18\\
83.61	1.73472347597681e-18\\
83.62	0\\
83.63	1.73472347597681e-18\\
83.64	0\\
83.65	0\\
83.66	0\\
83.67	0\\
83.68	0\\
83.69	1.73472347597681e-18\\
83.7	0\\
83.71	0\\
83.72	0\\
83.73	0\\
83.74	0\\
83.75	0\\
83.76	0\\
83.77	0\\
83.78	0\\
83.79	1.73472347597681e-18\\
83.8	0\\
83.81	0\\
83.82	0\\
83.83	0\\
83.84	0\\
83.85	0\\
83.86	0\\
83.87	0\\
83.88	0\\
83.89	0\\
83.9	0\\
83.91	0\\
83.92	0\\
83.93	0\\
83.94	0\\
83.95	0\\
83.96	0\\
83.97	0\\
83.98	0\\
83.99	0\\
84	0\\
84.01	0\\
84.02	0\\
84.03	0\\
84.04	0\\
84.05	0\\
84.06	0\\
84.07	1.73472347597681e-18\\
84.08	0\\
84.09	0\\
84.1	0\\
84.11	1.73472347597681e-18\\
84.12	0\\
84.13	0\\
84.14	0\\
84.15	0\\
84.16	0\\
84.17	0\\
84.18	0\\
84.19	0\\
84.2	0\\
84.21	0\\
84.22	1.73472347597681e-18\\
84.23	0\\
84.24	0\\
84.25	0\\
84.26	1.73472347597681e-18\\
84.27	0\\
84.28	0\\
84.29	0\\
84.3	0\\
84.31	0\\
84.32	1.73472347597681e-18\\
84.33	0\\
84.34	0\\
84.35	1.73472347597681e-18\\
84.36	1.73472347597681e-18\\
84.37	0\\
84.38	0\\
84.39	0\\
84.4	0\\
84.41	1.73472347597681e-18\\
84.42	1.73472347597681e-18\\
84.43	0\\
84.44	0\\
84.45	0\\
84.46	0\\
84.47	0\\
84.48	0\\
84.49	0\\
84.5	1.73472347597681e-18\\
84.51	0\\
84.52	0\\
84.53	0\\
84.54	0\\
84.55	0\\
84.56	0\\
84.57	0\\
84.58	0\\
84.59	1.73472347597681e-18\\
84.6	0\\
84.61	0\\
84.62	0\\
84.63	0\\
84.64	0\\
84.65	0\\
84.66	1.73472347597681e-18\\
84.67	1.73472347597681e-18\\
84.68	0\\
84.69	0\\
84.7	0\\
84.71	0\\
84.72	0\\
84.73	0\\
84.74	0\\
84.75	0\\
84.76	1.73472347597681e-18\\
84.77	0\\
84.78	0\\
84.79	0\\
84.8	0\\
84.81	0\\
84.82	0\\
84.83	0\\
84.84	0\\
84.85	0\\
84.86	0\\
84.87	0\\
84.88	0\\
84.89	0\\
84.9	0\\
84.91	0\\
84.92	0\\
84.93	0\\
84.94	0\\
84.95	0\\
84.96	0\\
84.97	0\\
84.98	0\\
84.99	0\\
85	0\\
85.01	0\\
85.02	0\\
85.03	0\\
85.04	0\\
85.05	0\\
85.06	0\\
85.07	0\\
85.08	0\\
85.09	0\\
85.1	0\\
85.11	0\\
85.12	0\\
85.13	0\\
85.14	0\\
85.15	0\\
85.16	0\\
85.17	0\\
85.18	0\\
85.19	0\\
85.2	0\\
85.21	0\\
85.22	0\\
85.23	1.73472347597681e-18\\
85.24	0\\
85.25	0\\
85.26	1.73472347597681e-18\\
85.27	0\\
85.28	1.73472347597681e-18\\
85.29	0\\
85.3	0\\
85.31	1.73472347597681e-18\\
85.32	0\\
85.33	1.73472347597681e-18\\
85.34	0\\
85.35	0\\
85.36	0\\
85.37	0\\
85.38	0\\
85.39	0\\
85.4	0\\
85.41	0\\
85.42	0\\
85.43	0\\
85.44	0\\
85.45	0\\
85.46	0\\
85.47	0\\
85.48	0\\
85.49	0\\
85.5	0\\
85.51	0\\
85.52	0\\
85.53	1.73472347597681e-18\\
85.54	0\\
85.55	0\\
85.56	0\\
85.57	0\\
85.58	0\\
85.59	0\\
85.6	0\\
85.61	0\\
85.62	0\\
85.63	0\\
85.64	0\\
85.65	0\\
85.66	0\\
85.67	0\\
85.68	0\\
85.69	0\\
85.7	0\\
85.71	0\\
85.72	0\\
85.73	0\\
85.74	0\\
85.75	0\\
85.76	0\\
85.77	0\\
85.78	0\\
85.79	0\\
85.8	0\\
85.81	1.73472347597681e-18\\
85.82	0\\
85.83	0\\
85.84	0\\
85.85	0\\
85.86	0\\
85.87	0\\
85.88	0\\
85.89	0\\
85.9	0\\
85.91	0\\
85.92	0\\
85.93	0\\
85.94	0\\
85.95	0\\
85.96	0\\
85.97	1.73472347597681e-18\\
85.98	0\\
85.99	0\\
86	0\\
86.01	0\\
86.02	1.73472347597681e-18\\
86.03	1.73472347597681e-18\\
86.04	0\\
86.05	0\\
86.06	0\\
86.07	0\\
86.08	0\\
86.09	0\\
86.1	0\\
86.11	0\\
86.12	0\\
86.13	0\\
86.14	0\\
86.15	0\\
86.16	0\\
86.17	0\\
86.18	0\\
86.19	0\\
86.2	0\\
86.21	0\\
86.22	0\\
86.23	0\\
86.24	0\\
86.25	1.73472347597681e-18\\
86.26	1.73472347597681e-18\\
86.27	0\\
86.28	0\\
86.29	0\\
86.3	0\\
86.31	0\\
86.32	0\\
86.33	1.73472347597681e-18\\
86.34	0\\
86.35	0\\
86.36	0\\
86.37	0\\
86.38	0\\
86.39	0\\
86.4	0\\
86.41	0\\
86.42	0\\
86.43	0\\
86.44	0\\
86.45	0\\
86.46	0\\
86.47	0\\
86.48	0\\
86.49	0\\
86.5	0\\
86.51	0\\
86.52	0\\
86.53	0\\
86.54	0\\
86.55	1.73472347597681e-18\\
86.56	0\\
86.57	0\\
86.58	0\\
86.59	0\\
86.6	0\\
86.61	0\\
86.62	0\\
86.63	0\\
86.64	0\\
86.65	0\\
86.66	0\\
86.67	0\\
86.68	0\\
86.69	0\\
86.7	1.73472347597681e-18\\
86.71	1.73472347597681e-18\\
86.72	0\\
86.73	0\\
86.74	0\\
86.75	0\\
86.76	0\\
86.77	0\\
86.78	0\\
86.79	0\\
86.8	0\\
86.81	0\\
86.82	0\\
86.83	0\\
86.84	0\\
86.85	0\\
86.86	0\\
86.87	0\\
86.88	0\\
86.89	0\\
86.9	0\\
86.91	0\\
86.92	0\\
86.93	0\\
86.94	0\\
86.95	0\\
86.96	0\\
86.97	1.73472347597681e-18\\
86.98	0\\
86.99	0\\
87	0\\
87.01	0\\
87.02	0\\
87.03	0\\
87.04	0\\
87.05	0\\
87.06	0\\
87.07	0\\
87.08	0\\
87.09	0\\
87.1	0\\
87.11	0\\
87.12	0\\
87.13	0\\
87.14	0\\
87.15	0\\
87.16	0\\
87.17	0\\
87.18	0\\
87.19	0\\
87.2	0\\
87.21	0\\
87.22	0\\
87.23	0\\
87.24	0\\
87.25	0\\
87.26	0\\
87.27	1.73472347597681e-18\\
87.28	0\\
87.29	0\\
87.3	0\\
87.31	0\\
87.32	0\\
87.33	0\\
87.34	0\\
87.35	0\\
87.36	0\\
87.37	0\\
87.38	0\\
87.39	0\\
87.4	0\\
87.41	0\\
87.42	0\\
87.43	0\\
87.44	0\\
87.45	0\\
87.46	0\\
87.47	0\\
87.48	0\\
87.49	0\\
87.5	0\\
87.51	1.73472347597681e-18\\
87.52	0\\
87.53	0\\
87.54	0\\
87.55	0\\
87.56	0\\
87.57	0\\
87.58	0\\
87.59	0\\
87.6	0\\
87.61	0\\
87.62	1.73472347597681e-18\\
87.63	0\\
87.64	0\\
87.65	0\\
87.66	0\\
87.67	0\\
87.68	0\\
87.69	1.73472347597681e-18\\
87.7	1.73472347597681e-18\\
87.71	0\\
87.72	0\\
87.73	0\\
87.74	0\\
87.75	0\\
87.76	0\\
87.77	0\\
87.78	0\\
87.79	0\\
87.8	0\\
87.81	0\\
87.82	0\\
87.83	1.73472347597681e-18\\
87.84	0\\
87.85	0\\
87.86	0\\
87.87	0\\
87.88	0\\
87.89	0\\
87.9	0\\
87.91	0\\
87.92	0\\
87.93	0\\
87.94	0\\
87.95	0\\
87.96	0\\
87.97	0\\
87.98	0\\
87.99	0\\
88	0\\
88.01	0\\
88.02	0\\
88.03	0\\
88.04	0\\
88.05	0\\
88.06	0\\
88.07	0\\
88.08	0\\
88.09	0\\
88.1	0\\
88.11	0\\
88.12	0\\
88.13	0\\
88.14	0\\
88.15	1.73472347597681e-18\\
88.16	0\\
88.17	0\\
88.18	0\\
88.19	0\\
88.2	0\\
88.21	0\\
88.22	0\\
88.23	0\\
88.24	0\\
88.25	0\\
88.26	0\\
88.27	0\\
88.28	0\\
88.29	0\\
88.3	0\\
88.31	0\\
88.32	0\\
88.33	0\\
88.34	0\\
88.35	0\\
88.36	0\\
88.37	0\\
88.38	0\\
88.39	0\\
88.4	0\\
88.41	0\\
88.42	0\\
88.43	0\\
88.44	0\\
88.45	0\\
88.46	1.73472347597681e-18\\
88.47	0\\
88.48	0\\
88.49	0\\
88.5	0\\
88.51	0\\
88.52	0\\
88.53	0\\
88.54	0\\
88.55	0\\
88.56	0\\
88.57	0\\
88.58	0\\
88.59	0\\
88.6	0\\
88.61	0\\
88.62	0\\
88.63	0\\
88.64	0\\
88.65	0\\
88.66	0\\
88.67	0\\
88.68	0\\
88.69	0\\
88.7	0\\
88.71	0\\
88.72	0\\
88.73	0\\
88.74	0\\
88.75	0\\
88.76	0\\
88.77	0\\
88.78	0\\
88.79	0\\
88.8	0\\
88.81	0\\
88.82	0\\
88.83	0\\
88.84	0\\
88.85	0\\
88.86	0\\
88.87	0\\
88.88	0\\
88.89	0\\
88.9	0\\
88.91	0\\
88.92	0\\
88.93	0\\
88.94	0\\
88.95	0\\
88.96	0\\
88.97	0\\
88.98	0\\
88.99	0\\
89	0\\
89.01	0\\
89.02	0\\
89.03	0\\
89.04	0\\
89.05	0\\
89.06	0\\
89.07	0\\
89.08	0\\
89.09	0\\
89.1	0\\
89.11	0\\
89.12	0\\
89.13	0\\
89.14	0\\
89.15	0\\
89.16	0\\
89.17	0\\
89.18	0\\
89.19	0\\
89.2	0\\
89.21	0\\
89.22	0\\
89.23	0\\
89.24	0\\
89.25	0\\
89.26	0\\
89.27	0\\
89.28	0\\
89.29	0\\
89.3	0\\
89.31	0\\
89.32	0\\
89.33	0\\
89.34	0\\
89.35	0\\
89.36	0\\
89.37	0\\
89.38	1.73472347597681e-18\\
89.39	0\\
89.4	0\\
89.41	0\\
89.42	0\\
89.43	0\\
89.44	0\\
89.45	0\\
89.46	0\\
89.47	0\\
89.48	0\\
89.49	0\\
89.5	0\\
89.51	0\\
89.52	0\\
89.53	0\\
89.54	0\\
89.55	0\\
89.56	0\\
89.57	0\\
89.58	0\\
89.59	0\\
89.6	0\\
89.61	0\\
89.62	0\\
89.63	0\\
89.64	0\\
89.65	1.73472347597681e-18\\
89.66	0\\
89.67	0\\
89.68	1.73472347597681e-18\\
89.69	0\\
89.7	0\\
89.71	0\\
89.72	0\\
89.73	0\\
89.74	0\\
89.75	0\\
89.76	0\\
89.77	0\\
89.78	0\\
89.79	0\\
89.8	0\\
89.81	0\\
89.82	0\\
89.83	0\\
89.84	0\\
89.85	0\\
89.86	0\\
89.87	0\\
89.88	0\\
89.89	0\\
89.9	0\\
89.91	0\\
89.92	0\\
89.93	0\\
89.94	0\\
89.95	0\\
89.96	0\\
89.97	0\\
89.98	0\\
89.99	0\\
90	0\\
90.01	0\\
90.02	0\\
90.03	0\\
90.04	0\\
90.05	0\\
90.06	1.73472347597681e-18\\
90.07	0\\
90.08	0\\
90.09	0\\
90.1	0\\
90.11	0\\
90.12	0\\
90.13	0\\
90.14	0\\
90.15	0\\
90.16	0\\
90.17	0\\
90.18	1.73472347597681e-18\\
90.19	0\\
90.2	0\\
90.21	0\\
90.22	0\\
90.23	0\\
90.24	0\\
90.25	0\\
90.26	0\\
90.27	0\\
90.28	0\\
90.29	0\\
90.3	0\\
90.31	0\\
90.32	0\\
90.33	1.73472347597681e-18\\
90.34	0\\
90.35	0\\
90.36	0\\
90.37	0\\
90.38	0\\
90.39	0\\
90.4	0\\
90.41	0\\
90.42	0\\
90.43	0\\
90.44	0\\
90.45	0\\
90.46	0\\
90.47	0\\
90.48	0\\
90.49	0\\
90.5	0\\
90.51	0\\
90.52	0\\
90.53	0\\
90.54	0\\
90.55	0\\
90.56	0\\
90.57	0\\
90.58	0\\
90.59	0\\
90.6	0\\
90.61	0\\
90.62	0\\
90.63	0\\
90.64	0\\
90.65	0\\
90.66	0\\
90.67	0\\
90.68	0\\
90.69	0\\
90.7	0\\
90.71	0\\
90.72	0\\
90.73	0\\
90.74	0\\
90.75	0\\
90.76	0\\
90.77	0\\
90.78	0\\
90.79	0\\
90.8	0\\
90.81	0\\
90.82	0\\
90.83	0\\
90.84	0\\
90.85	0\\
90.86	0\\
90.87	0\\
90.88	0\\
90.89	0\\
90.9	0\\
90.91	0\\
90.92	0\\
90.93	0\\
90.94	0\\
90.95	0\\
90.96	0\\
90.97	0\\
90.98	0\\
90.99	0\\
91	0\\
91.01	0\\
91.02	0\\
91.03	0\\
91.04	0\\
91.05	0\\
91.06	0\\
91.07	0\\
91.08	0\\
91.09	0\\
91.1	0\\
91.11	0\\
91.12	0\\
91.13	0\\
91.14	0\\
91.15	0\\
91.16	0\\
91.17	0\\
91.18	0\\
91.19	0\\
91.2	0\\
91.21	0\\
91.22	0\\
91.23	0\\
91.24	0\\
91.25	0\\
91.26	0\\
91.27	0\\
91.28	0\\
91.29	0\\
91.3	0\\
91.31	0\\
91.32	0\\
91.33	0\\
91.34	0\\
91.35	0\\
91.36	0\\
91.37	0\\
91.38	0\\
91.39	0\\
91.4	0\\
91.41	0\\
91.42	0\\
91.43	0\\
91.44	0\\
91.45	0\\
91.46	0\\
91.47	0\\
91.48	0\\
91.49	0\\
91.5	0\\
91.51	0\\
91.52	0\\
91.53	0\\
91.54	0\\
91.55	0\\
91.56	0\\
91.57	0\\
91.58	0\\
91.59	0\\
91.6	0\\
91.61	0\\
91.62	0\\
91.63	0\\
91.64	0\\
91.65	0\\
91.66	0\\
91.67	0\\
91.68	0\\
91.69	0\\
91.7	0\\
91.71	0\\
91.72	0\\
91.73	0\\
91.74	0\\
91.75	0\\
91.76	0\\
91.77	0\\
91.78	0\\
91.79	0\\
91.8	0\\
91.81	0\\
91.82	0\\
91.83	0\\
91.84	0\\
91.85	0\\
91.86	0\\
91.87	0\\
91.88	0\\
91.89	0\\
91.9	0\\
91.91	0\\
91.92	0\\
91.93	0\\
91.94	0\\
91.95	0\\
91.96	0\\
91.97	0\\
91.98	0\\
91.99	0\\
92	0\\
92.01	0\\
92.02	0\\
92.03	0\\
92.04	0\\
92.05	0\\
92.06	0\\
92.07	0\\
92.08	0\\
92.09	0\\
92.1	0\\
92.11	0\\
92.12	0\\
92.13	0\\
92.14	0\\
92.15	0\\
92.16	0\\
92.17	0\\
92.18	0\\
92.19	0\\
92.2	0\\
92.21	0\\
92.22	0\\
92.23	0\\
92.24	0\\
92.25	0\\
92.26	0\\
92.27	0\\
92.28	0\\
92.29	0\\
92.3	0\\
92.31	0\\
92.32	0\\
92.33	0\\
92.34	0\\
92.35	0\\
92.36	0\\
92.37	0\\
92.38	0\\
92.39	0\\
92.4	0\\
92.41	0\\
92.42	0\\
92.43	0\\
92.44	0\\
92.45	0\\
92.46	0\\
92.47	0\\
92.48	0\\
92.49	0\\
92.5	0\\
92.51	0\\
92.52	0\\
92.53	0\\
92.54	0\\
92.55	0\\
92.56	0\\
92.57	0\\
92.58	0\\
92.59	0\\
92.6	0\\
92.61	0\\
92.62	0\\
92.63	0\\
92.64	0\\
92.65	0\\
92.66	0\\
92.67	0\\
92.68	0\\
92.69	0\\
92.7	0\\
92.71	0\\
92.72	0\\
92.73	0\\
92.74	0\\
92.75	0\\
92.76	0\\
92.77	0\\
92.78	0\\
92.79	0\\
92.8	0\\
92.81	0\\
92.82	0\\
92.83	0\\
92.84	0\\
92.85	0\\
92.86	0\\
92.87	0\\
92.88	0\\
92.89	0\\
92.9	0\\
92.91	0\\
92.92	0\\
92.93	0\\
92.94	0\\
92.95	0\\
92.96	0\\
92.97	0\\
92.98	0\\
92.99	0\\
93	0\\
93.01	0\\
93.02	0\\
93.03	0\\
93.04	0\\
93.05	0\\
93.06	0\\
93.07	0\\
93.08	0\\
93.09	0\\
93.1	0\\
93.11	0\\
93.12	0\\
93.13	0\\
93.14	0\\
93.15	0\\
93.16	0\\
93.17	0\\
93.18	0\\
93.19	0\\
93.2	0\\
93.21	0\\
93.22	0\\
93.23	0\\
93.24	0\\
93.25	0\\
93.26	0\\
93.27	0\\
93.28	0\\
93.29	0\\
93.3	0\\
93.31	0\\
93.32	0\\
93.33	0\\
93.34	0\\
93.35	0\\
93.36	0\\
93.37	0\\
93.38	0\\
93.39	0\\
93.4	0\\
93.41	0\\
93.42	0\\
93.43	0\\
93.44	0\\
93.45	0\\
93.46	0\\
93.47	0\\
93.48	0\\
93.49	0\\
93.5	0\\
93.51	0\\
93.52	0\\
93.53	0\\
93.54	0\\
93.55	0\\
93.56	0\\
93.57	0\\
93.58	0\\
93.59	0\\
93.6	0\\
93.61	0\\
93.62	0\\
93.63	0\\
93.64	0\\
93.65	0\\
93.66	0\\
93.67	0\\
93.68	0\\
93.69	0\\
93.7	0\\
93.71	0\\
93.72	0\\
93.73	0\\
93.74	0\\
93.75	0\\
93.76	0\\
93.77	0\\
93.78	0\\
93.79	0\\
93.8	0\\
93.81	0\\
93.82	0\\
93.83	0\\
93.84	0\\
93.85	0\\
93.86	0\\
93.87	0\\
93.88	0\\
93.89	0\\
93.9	0\\
93.91	0\\
93.92	0\\
93.93	0\\
93.94	0\\
93.95	0\\
93.96	0\\
93.97	0\\
93.98	0\\
93.99	0\\
94	0\\
94.01	0\\
94.02	0\\
94.03	0\\
94.04	0\\
94.05	0\\
94.06	0\\
94.07	0\\
94.08	0\\
94.09	0\\
94.1	0\\
94.11	0\\
94.12	0\\
94.13	0\\
94.14	0\\
94.15	0\\
94.16	0\\
94.17	0\\
94.18	0\\
94.19	0\\
94.2	0\\
94.21	0\\
94.22	0\\
94.23	0\\
94.24	0\\
94.25	0\\
94.26	0\\
94.27	0\\
94.28	0\\
94.29	0\\
94.3	0\\
94.31	0\\
94.32	0\\
94.33	0\\
94.34	0\\
94.35	0\\
94.36	0\\
94.37	0\\
94.38	0\\
94.39	0\\
94.4	0\\
94.41	0\\
94.42	0\\
94.43	0\\
94.44	0\\
94.45	0\\
94.46	0\\
94.47	0\\
94.48	0\\
94.49	0\\
94.5	0\\
94.51	0\\
94.52	0\\
94.53	0\\
94.54	0\\
94.55	0\\
94.56	0\\
94.57	0\\
94.58	0\\
94.59	0\\
94.6	0\\
94.61	0\\
94.62	0\\
94.63	0\\
94.64	0\\
94.65	0\\
94.66	0\\
94.67	0\\
94.68	0\\
94.69	0\\
94.7	0\\
94.71	0\\
94.72	0\\
94.73	0\\
94.74	0\\
94.75	0\\
94.76	0\\
94.77	0\\
94.78	0\\
94.79	0\\
94.8	0\\
94.81	0\\
94.82	0\\
94.83	0\\
94.84	0\\
94.85	0\\
94.86	0\\
94.87	0\\
94.88	0\\
94.89	0\\
94.9	0\\
94.91	0\\
94.92	0\\
94.93	0\\
94.94	0\\
94.95	0\\
94.96	0\\
94.97	0\\
94.98	0\\
94.99	0\\
95	0\\
95.01	0\\
95.02	0\\
95.03	0\\
95.04	0\\
95.05	0\\
95.06	0\\
95.07	0\\
95.08	0\\
95.09	0\\
95.1	0\\
95.11	0\\
95.12	0\\
95.13	0\\
95.14	0\\
95.15	0\\
95.16	0\\
95.17	0\\
95.18	0\\
95.19	0\\
95.2	0\\
95.21	0\\
95.22	0\\
95.23	0\\
95.24	0\\
95.25	0\\
95.26	0\\
95.27	0\\
95.28	0\\
95.29	0\\
95.3	0\\
95.31	0\\
95.32	0\\
95.33	0\\
95.34	0\\
95.35	0\\
95.36	0\\
95.37	0\\
95.38	0\\
95.39	0\\
95.4	0\\
95.41	0\\
95.42	0\\
95.43	0\\
95.44	0\\
95.45	0\\
95.46	0\\
95.47	0\\
95.48	0\\
95.49	0\\
95.5	0\\
95.51	0\\
95.52	0\\
95.53	0\\
95.54	0\\
95.55	0\\
95.56	0\\
95.57	0\\
95.58	0\\
95.59	0\\
95.6	0\\
95.61	0\\
95.62	0\\
95.63	0\\
95.64	0\\
95.65	0\\
95.66	0\\
95.67	0\\
95.68	0\\
95.69	0\\
95.7	0\\
95.71	0\\
95.72	0\\
95.73	0\\
95.74	0\\
95.75	0\\
95.76	0\\
95.77	0\\
95.78	0\\
95.79	0\\
95.8	0\\
95.81	0\\
95.82	0\\
95.83	0\\
95.84	0\\
95.85	0\\
95.86	0\\
95.87	0\\
95.88	0\\
95.89	0\\
95.9	0\\
95.91	0\\
95.92	0\\
95.93	0\\
95.94	0\\
95.95	0\\
95.96	0\\
95.97	0\\
95.98	0\\
95.99	0\\
96	0\\
96.01	0\\
96.02	0\\
96.03	0\\
96.04	0\\
96.05	0\\
96.06	0\\
96.07	0\\
96.08	0\\
96.09	0\\
96.1	0\\
96.11	0\\
96.12	0\\
96.13	0\\
96.14	0\\
96.15	0\\
96.16	0\\
96.17	0\\
96.18	0\\
96.19	0\\
96.2	0\\
96.21	0\\
96.22	0\\
96.23	0\\
96.24	0\\
96.25	0\\
96.26	0\\
96.27	0\\
96.28	0\\
96.29	0\\
96.3	0\\
96.31	0\\
96.32	0\\
96.33	0\\
96.34	0\\
96.35	0\\
96.36	0\\
96.37	0\\
96.38	0\\
96.39	0\\
96.4	0\\
96.41	0\\
96.42	0\\
96.43	0\\
96.44	0\\
96.45	0\\
96.46	0\\
96.47	0\\
96.48	0\\
96.49	0\\
96.5	0\\
96.51	0\\
96.52	0\\
96.53	0\\
96.54	0\\
96.55	0\\
96.56	0\\
96.57	0\\
96.58	0\\
96.59	0\\
96.6	0\\
96.61	0\\
96.62	0\\
96.63	0\\
96.64	0\\
96.65	0\\
96.66	0\\
96.67	0\\
96.68	0\\
96.69	0\\
96.7	0\\
96.71	0\\
96.72	0\\
96.73	0\\
96.74	0\\
96.75	0\\
96.76	0\\
96.77	0\\
96.78	0\\
96.79	0\\
96.8	0\\
96.81	0\\
96.82	0\\
96.83	0\\
96.84	0\\
96.85	0\\
96.86	0\\
96.87	0\\
96.88	0\\
96.89	0\\
96.9	0\\
96.91	0\\
96.92	0\\
96.93	0\\
96.94	0\\
96.95	0\\
96.96	0\\
96.97	0\\
96.98	0\\
96.99	0\\
97	0\\
97.01	0\\
97.02	0\\
97.03	0\\
97.04	0\\
97.05	0\\
97.06	0\\
97.07	0\\
97.08	0\\
97.09	0\\
97.1	0\\
97.11	0\\
97.12	0\\
97.13	0\\
97.14	0\\
97.15	0\\
97.16	0\\
97.17	0\\
97.18	0\\
97.19	0\\
97.2	0\\
97.21	0\\
97.22	0\\
97.23	0\\
97.24	0\\
97.25	0\\
97.26	0\\
97.27	0\\
97.28	0\\
97.29	0\\
97.3	0\\
97.31	0\\
97.32	0\\
97.33	0\\
97.34	0\\
97.35	0\\
97.36	0\\
97.37	0\\
97.38	0\\
97.39	0\\
97.4	0\\
97.41	0\\
97.42	0\\
97.43	0\\
97.44	0\\
97.45	0\\
97.46	0\\
97.47	0\\
97.48	0\\
97.49	0\\
97.5	0\\
97.51	0\\
97.52	0\\
97.53	0\\
97.54	0\\
97.55	0\\
97.56	0\\
97.57	0\\
97.58	0\\
97.59	0\\
97.6	0\\
97.61	0\\
97.62	0\\
97.63	0\\
97.64	0\\
97.65	0\\
97.66	0\\
97.67	0\\
97.68	0\\
97.69	0\\
97.7	0\\
97.71	0\\
97.72	0\\
97.73	0\\
97.74	0\\
97.75	0\\
97.76	0\\
97.77	0\\
97.78	0\\
97.79	0\\
97.8	0\\
97.81	0\\
97.82	0\\
97.83	0\\
97.84	0\\
97.85	0\\
97.86	0\\
97.87	0\\
97.88	0\\
97.89	0\\
97.9	0\\
97.91	0\\
97.92	0\\
97.93	0\\
97.94	0\\
97.95	0\\
97.96	0\\
97.97	0\\
97.98	0\\
97.99	0\\
98	0\\
98.01	0\\
98.02	0\\
98.03	0\\
98.04	0\\
98.05	0\\
98.06	0\\
98.07	0\\
98.08	0\\
98.09	0\\
98.1	0\\
98.11	0\\
98.12	0\\
98.13	0\\
98.14	0\\
98.15	0\\
98.16	0\\
98.17	0\\
98.18	0\\
98.19	0\\
98.2	0\\
98.21	0\\
98.22	0\\
98.23	0\\
98.24	0\\
98.25	0\\
98.26	0\\
98.27	0\\
98.28	0\\
98.29	0\\
98.3	0\\
98.31	0\\
98.32	0\\
98.33	0\\
98.34	0\\
98.35	0\\
98.36	0\\
98.37	0\\
98.38	0\\
98.39	0\\
98.4	0\\
98.41	0\\
98.42	0\\
98.43	0\\
98.44	0\\
98.45	0\\
98.46	0\\
98.47	0\\
98.48	0\\
98.49	0\\
98.5	0\\
98.51	0\\
98.52	0\\
98.53	0\\
98.54	0\\
98.55	0\\
98.56	0\\
98.57	0\\
98.58	0\\
98.59	0\\
98.6	0\\
98.61	0\\
98.62	0\\
98.63	0\\
98.64	0\\
98.65	0\\
98.66	0\\
98.67	0\\
98.68	0\\
98.69	0\\
98.7	0\\
98.71	0\\
98.72	0\\
98.73	0\\
98.74	0\\
98.75	0\\
98.76	0\\
98.77	0\\
98.78	0\\
98.79	0\\
98.8	0\\
98.81	0\\
98.82	0\\
98.83	0\\
98.84	0\\
98.85	0\\
98.86	0\\
98.87	0\\
98.88	0\\
98.89	0\\
98.9	0\\
98.91	0\\
98.92	0\\
98.93	0\\
98.94	0\\
98.95	0\\
98.96	0\\
98.97	0\\
98.98	0\\
98.99	0\\
99	0\\
99.01	0\\
99.02	0\\
99.03	0\\
99.04	0\\
99.05	0\\
99.06	0\\
99.07	0\\
99.08	0\\
99.09	0\\
99.1	0\\
99.11	0\\
99.12	0\\
99.13	0\\
99.14	0\\
99.15	0\\
99.16	0\\
99.17	0\\
99.18	0\\
99.19	0\\
99.2	0\\
99.21	0\\
99.22	0\\
99.23	0\\
99.24	0\\
99.25	0\\
99.26	0\\
99.27	0\\
99.28	0\\
99.29	0\\
99.3	0\\
99.31	0\\
99.32	0\\
99.33	0\\
99.34	0\\
99.35	0\\
99.36	0\\
99.37	0\\
99.38	0\\
99.39	0\\
99.4	0\\
99.41	0\\
99.42	0\\
99.43	0\\
99.44	0\\
99.45	0\\
99.46	0\\
99.47	0\\
99.48	0\\
99.49	0\\
99.5	0\\
99.51	0\\
99.52	0\\
99.53	0\\
99.54	0\\
99.55	0\\
99.56	0\\
99.57	0\\
99.58	0\\
99.59	0\\
99.6	0\\
99.61	0\\
99.62	0\\
99.63	0\\
99.64	0\\
99.65	0\\
99.66	0\\
99.67	0\\
99.68	0\\
99.69	0\\
99.7	0\\
99.71	0\\
99.72	0\\
99.73	0\\
99.74	0\\
99.75	0\\
99.76	0\\
99.77	0\\
99.78	0\\
99.79	0\\
99.8	0\\
99.81	0\\
99.82	0\\
99.83	0\\
99.84	0\\
99.85	0\\
99.86	0\\
99.87	0\\
99.88	0\\
99.89	0\\
99.9	0\\
99.91	0\\
99.92	0\\
99.93	0\\
99.94	0\\
99.95	0\\
99.96	0\\
99.97	0\\
99.98	0\\
99.99	0\\
100	0\\
};
\addlegendentry{$q=-3$};

\addplot [color=red,dashed,forget plot]
  table[row sep=crcr]{%
0.01	0\\
0.02	0\\
0.03	0\\
0.04	0\\
0.05	0\\
0.06	0\\
0.07	0\\
0.08	0\\
0.09	0\\
0.1	0\\
0.11	0\\
0.12	0\\
0.13	0\\
0.14	0\\
0.15	0\\
0.16	0\\
0.17	1.73472347597681e-18\\
0.18	1.73472347597681e-18\\
0.19	0\\
0.2	0\\
0.21	0\\
0.22	0\\
0.23	0\\
0.24	0\\
0.25	0\\
0.26	0\\
0.27	0\\
0.28	0\\
0.29	0\\
0.3	0\\
0.31	0\\
0.32	0\\
0.33	0\\
0.34	1.73472347597681e-18\\
0.35	1.73472347597681e-18\\
0.36	0\\
0.37	0\\
0.38	0\\
0.39	0\\
0.4	0\\
0.41	0\\
0.42	0\\
0.43	0\\
0.44	0\\
0.45	0\\
0.46	0\\
0.47	0\\
0.48	0\\
0.49	0\\
0.5	0\\
0.51	1.73472347597681e-18\\
0.52	1.73472347597681e-18\\
0.53	0\\
0.54	0\\
0.55	1.73472347597681e-18\\
0.56	0\\
0.57	0\\
0.58	0\\
0.59	0\\
0.6	1.73472347597681e-18\\
0.61	0\\
0.62	0\\
0.63	1.73472347597681e-18\\
0.64	1.73472347597681e-18\\
0.65	0\\
0.66	0\\
0.67	1.73472347597681e-18\\
0.68	0\\
0.69	0\\
0.7	0\\
0.71	0\\
0.72	0\\
0.73	0\\
0.74	0\\
0.75	0\\
0.76	0\\
0.77	1.73472347597681e-18\\
0.78	0\\
0.79	0\\
0.8	0\\
0.81	0\\
0.82	0\\
0.83	0\\
0.84	0\\
0.85	0\\
0.86	0\\
0.87	0\\
0.88	0\\
0.89	0\\
0.9	0\\
0.91	0\\
0.92	0\\
0.93	0\\
0.94	0\\
0.95	0\\
0.96	0\\
0.97	0\\
0.98	0\\
0.99	1.73472347597681e-18\\
1	0\\
1.01	1.73472347597681e-18\\
1.02	0\\
1.03	0\\
1.04	1.73472347597681e-18\\
1.05	0\\
1.06	0\\
1.07	0\\
1.08	0\\
1.09	0\\
1.1	0\\
1.11	0\\
1.12	0\\
1.13	1.73472347597681e-18\\
1.14	0\\
1.15	0\\
1.16	0\\
1.17	0\\
1.18	0\\
1.19	1.73472347597681e-18\\
1.2	0\\
1.21	0\\
1.22	1.73472347597681e-18\\
1.23	1.73472347597681e-18\\
1.24	0\\
1.25	1.73472347597681e-18\\
1.26	0\\
1.27	0\\
1.28	0\\
1.29	0\\
1.3	0\\
1.31	0\\
1.32	0\\
1.33	0\\
1.34	0\\
1.35	0\\
1.36	0\\
1.37	0\\
1.38	0\\
1.39	0\\
1.4	0\\
1.41	0\\
1.42	0\\
1.43	0\\
1.44	0\\
1.45	0\\
1.46	0\\
1.47	0\\
1.48	0\\
1.49	0\\
1.5	0\\
1.51	0\\
1.52	0\\
1.53	0\\
1.54	0\\
1.55	0\\
1.56	0\\
1.57	0\\
1.58	0\\
1.59	0\\
1.6	0\\
1.61	0\\
1.62	0\\
1.63	0\\
1.64	0\\
1.65	0\\
1.66	0\\
1.67	0\\
1.68	0\\
1.69	0\\
1.7	0\\
1.71	0\\
1.72	0\\
1.73	0\\
1.74	0\\
1.75	0\\
1.76	0\\
1.77	0\\
1.78	0\\
1.79	0\\
1.8	1.73472347597681e-18\\
1.81	0\\
1.82	0\\
1.83	0\\
1.84	0\\
1.85	0\\
1.86	0\\
1.87	0\\
1.88	0\\
1.89	0\\
1.9	0\\
1.91	0\\
1.92	0\\
1.93	0\\
1.94	0\\
1.95	0\\
1.96	1.73472347597681e-18\\
1.97	0\\
1.98	0\\
1.99	0\\
2	0\\
2.01	0\\
2.02	0\\
2.03	1.73472347597681e-18\\
2.04	0\\
2.05	1.73472347597681e-18\\
2.06	1.73472347597681e-18\\
2.07	0\\
2.08	0\\
2.09	1.73472347597681e-18\\
2.1	0\\
2.11	0\\
2.12	0\\
2.13	1.73472347597681e-18\\
2.14	0\\
2.15	0\\
2.16	0\\
2.17	0\\
2.18	0\\
2.19	0\\
2.2	0\\
2.21	0\\
2.22	0\\
2.23	0\\
2.24	1.73472347597681e-18\\
2.25	1.73472347597681e-18\\
2.26	0\\
2.27	0\\
2.28	0\\
2.29	0\\
2.3	1.73472347597681e-18\\
2.31	0\\
2.32	0\\
2.33	0\\
2.34	0\\
2.35	1.73472347597681e-18\\
2.36	0\\
2.37	0\\
2.38	0\\
2.39	0\\
2.4	0\\
2.41	0\\
2.42	0\\
2.43	0\\
2.44	0\\
2.45	1.73472347597681e-18\\
2.46	0\\
2.47	0\\
2.48	0\\
2.49	0\\
2.5	0\\
2.51	0\\
2.52	1.73472347597681e-18\\
2.53	0\\
2.54	0\\
2.55	1.73472347597681e-18\\
2.56	0\\
2.57	0\\
2.58	0\\
2.59	0\\
2.6	0\\
2.61	0\\
2.62	0\\
2.63	0\\
2.64	0\\
2.65	0\\
2.66	0\\
2.67	1.73472347597681e-18\\
2.68	0\\
2.69	0\\
2.7	0\\
2.71	0\\
2.72	0\\
2.73	0\\
2.74	0\\
2.75	0\\
2.76	0\\
2.77	0\\
2.78	0\\
2.79	0\\
2.8	0\\
2.81	0\\
2.82	0\\
2.83	0\\
2.84	0\\
2.85	0\\
2.86	0\\
2.87	0\\
2.88	0\\
2.89	0\\
2.9	0\\
2.91	0\\
2.92	0\\
2.93	0\\
2.94	0\\
2.95	0\\
2.96	1.73472347597681e-18\\
2.97	0\\
2.98	0\\
2.99	0\\
3	1.73472347597681e-18\\
3.01	0\\
3.02	0\\
3.03	0\\
3.04	0\\
3.05	0\\
3.06	0\\
3.07	0\\
3.08	0\\
3.09	0\\
3.1	0\\
3.11	0\\
3.12	0\\
3.13	1.73472347597681e-18\\
3.14	0\\
3.15	0\\
3.16	0\\
3.17	0\\
3.18	0\\
3.19	0\\
3.2	0\\
3.21	1.73472347597681e-18\\
3.22	0\\
3.23	1.73472347597681e-18\\
3.24	0\\
3.25	0\\
3.26	0\\
3.27	0\\
3.28	0\\
3.29	0\\
3.3	0\\
3.31	0\\
3.32	0\\
3.33	0\\
3.34	0\\
3.35	0\\
3.36	0\\
3.37	0\\
3.38	0\\
3.39	1.73472347597681e-18\\
3.4	0\\
3.41	1.73472347597681e-18\\
3.42	1.73472347597681e-18\\
3.43	0\\
3.44	1.73472347597681e-18\\
3.45	0\\
3.46	0\\
3.47	0\\
3.48	0\\
3.49	0\\
3.5	1.73472347597681e-18\\
3.51	0\\
3.52	0\\
3.53	0\\
3.54	0\\
3.55	0\\
3.56	0\\
3.57	0\\
3.58	0\\
3.59	1.73472347597681e-18\\
3.6	0\\
3.61	0\\
3.62	0\\
3.63	0\\
3.64	0\\
3.65	0\\
3.66	0\\
3.67	0\\
3.68	0\\
3.69	0\\
3.7	0\\
3.71	0\\
3.72	0\\
3.73	0\\
3.74	0\\
3.75	0\\
3.76	0\\
3.77	0\\
3.78	0\\
3.79	0\\
3.8	0\\
3.81	1.73472347597681e-18\\
3.82	0\\
3.83	0\\
3.84	0\\
3.85	0\\
3.86	0\\
3.87	0\\
3.88	0\\
3.89	0\\
3.9	0\\
3.91	0\\
3.92	0\\
3.93	0\\
3.94	0\\
3.95	0\\
3.96	0\\
3.97	0\\
3.98	1.73472347597681e-18\\
3.99	0\\
4	0\\
4.01	1.73472347597681e-18\\
4.02	0\\
4.03	0\\
4.04	0\\
4.05	1.73472347597681e-18\\
4.06	0\\
4.07	1.73472347597681e-18\\
4.08	0\\
4.09	0\\
4.1	0\\
4.11	0\\
4.12	0\\
4.13	0\\
4.14	0\\
4.15	1.73472347597681e-18\\
4.16	0\\
4.17	0\\
4.18	0\\
4.19	0\\
4.2	0\\
4.21	1.73472347597681e-18\\
4.22	0\\
4.23	0\\
4.24	0\\
4.25	0\\
4.26	1.73472347597681e-18\\
4.27	0\\
4.28	0\\
4.29	0\\
4.3	0\\
4.31	0\\
4.32	0\\
4.33	0\\
4.34	0\\
4.35	0\\
4.36	0\\
4.37	0\\
4.38	0\\
4.39	0\\
4.4	0\\
4.41	0\\
4.42	0\\
4.43	0\\
4.44	0\\
4.45	0\\
4.46	0\\
4.47	0\\
4.48	0\\
4.49	0\\
4.5	0\\
4.51	1.73472347597681e-18\\
4.52	0\\
4.53	0\\
4.54	0\\
4.55	0\\
4.56	0\\
4.57	0\\
4.58	0\\
4.59	0\\
4.6	0\\
4.61	0\\
4.62	1.73472347597681e-18\\
4.63	0\\
4.64	0\\
4.65	0\\
4.66	0\\
4.67	0\\
4.68	0\\
4.69	0\\
4.7	0\\
4.71	0\\
4.72	0\\
4.73	0\\
4.74	0\\
4.75	0\\
4.76	0\\
4.77	0\\
4.78	0\\
4.79	0\\
4.8	0\\
4.81	0\\
4.82	1.73472347597681e-18\\
4.83	0\\
4.84	0\\
4.85	1.73472347597681e-18\\
4.86	0\\
4.87	0\\
4.88	0\\
4.89	0\\
4.9	0\\
4.91	0\\
4.92	0\\
4.93	0\\
4.94	0\\
4.95	0\\
4.96	0\\
4.97	0\\
4.98	1.73472347597681e-18\\
4.99	0\\
5	0\\
5.01	0\\
5.02	0\\
5.03	0\\
5.04	0\\
5.05	0\\
5.06	1.73472347597681e-18\\
5.07	1.73472347597681e-18\\
5.08	0\\
5.09	0\\
5.1	0\\
5.11	1.73472347597681e-18\\
5.12	0\\
5.13	0\\
5.14	0\\
5.15	0\\
5.16	0\\
5.17	0\\
5.18	0\\
5.19	0\\
5.2	0\\
5.21	0\\
5.22	0\\
5.23	1.73472347597681e-18\\
5.24	0\\
5.25	0\\
5.26	0\\
5.27	0\\
5.28	0\\
5.29	0\\
5.3	0\\
5.31	0\\
5.32	1.73472347597681e-18\\
5.33	0\\
5.34	1.73472347597681e-18\\
5.35	1.73472347597681e-18\\
5.36	0\\
5.37	0\\
5.38	0\\
5.39	0\\
5.4	0\\
5.41	0\\
5.42	0\\
5.43	0\\
5.44	0\\
5.45	0\\
5.46	1.73472347597681e-18\\
5.47	0\\
5.48	0\\
5.49	1.73472347597681e-18\\
5.5	0\\
5.51	0\\
5.52	0\\
5.53	0\\
5.54	0\\
5.55	0\\
5.56	0\\
5.57	1.73472347597681e-18\\
5.58	0\\
5.59	0\\
5.6	0\\
5.61	0\\
5.62	0\\
5.63	1.73472347597681e-18\\
5.64	0\\
5.65	0\\
5.66	0\\
5.67	0\\
5.68	0\\
5.69	1.73472347597681e-18\\
5.7	0\\
5.71	1.73472347597681e-18\\
5.72	0\\
5.73	0\\
5.74	0\\
5.75	0\\
5.76	1.73472347597681e-18\\
5.77	0\\
5.78	0\\
5.79	0\\
5.8	0\\
5.81	0\\
5.82	0\\
5.83	1.73472347597681e-18\\
5.84	0\\
5.85	0\\
5.86	0\\
5.87	0\\
5.88	0\\
5.89	0\\
5.9	0\\
5.91	0\\
5.92	0\\
5.93	0\\
5.94	0\\
5.95	0\\
5.96	0\\
5.97	0\\
5.98	0\\
5.99	0\\
6	0\\
6.01	0\\
6.02	0\\
6.03	0\\
6.04	1.73472347597681e-18\\
6.05	0\\
6.06	0\\
6.07	0\\
6.08	0\\
6.09	0\\
6.1	0\\
6.11	1.73472347597681e-18\\
6.12	0\\
6.13	0\\
6.14	0\\
6.15	0\\
6.16	0\\
6.17	1.73472347597681e-18\\
6.18	0\\
6.19	1.73472347597681e-18\\
6.2	0\\
6.21	0\\
6.22	0\\
6.23	0\\
6.24	0\\
6.25	1.73472347597681e-18\\
6.26	0\\
6.27	0\\
6.28	1.73472347597681e-18\\
6.29	0\\
6.3	0\\
6.31	0\\
6.32	0\\
6.33	1.73472347597681e-18\\
6.34	0\\
6.35	1.73472347597681e-18\\
6.36	1.73472347597681e-18\\
6.37	0\\
6.38	1.73472347597681e-18\\
6.39	0\\
6.4	0\\
6.41	0\\
6.42	1.73472347597681e-18\\
6.43	0\\
6.44	0\\
6.45	0\\
6.46	1.73472347597681e-18\\
6.47	0\\
6.48	1.73472347597681e-18\\
6.49	0\\
6.5	0\\
6.51	0\\
6.52	0\\
6.53	0\\
6.54	0\\
6.55	0\\
6.56	1.73472347597681e-18\\
6.57	0\\
6.58	0\\
6.59	0\\
6.6	0\\
6.61	1.73472347597681e-18\\
6.62	0\\
6.63	0\\
6.64	0\\
6.65	0\\
6.66	0\\
6.67	0\\
6.68	0\\
6.69	0\\
6.7	0\\
6.71	0\\
6.72	0\\
6.73	0\\
6.74	0\\
6.75	0\\
6.76	0\\
6.77	0\\
6.78	0\\
6.79	0\\
6.8	0\\
6.81	0\\
6.82	0\\
6.83	0\\
6.84	0\\
6.85	0\\
6.86	1.73472347597681e-18\\
6.87	0\\
6.88	0\\
6.89	0\\
6.9	0\\
6.91	0\\
6.92	0\\
6.93	0\\
6.94	0\\
6.95	0\\
6.96	0\\
6.97	0\\
6.98	0\\
6.99	0\\
7	0\\
7.01	0\\
7.02	0\\
7.03	1.73472347597681e-18\\
7.04	0\\
7.05	0\\
7.06	0\\
7.07	0\\
7.08	0\\
7.09	0\\
7.1	0\\
7.11	0\\
7.12	0\\
7.13	0\\
7.14	0\\
7.15	0\\
7.16	0\\
7.17	0\\
7.18	0\\
7.19	1.73472347597681e-18\\
7.2	0\\
7.21	0\\
7.22	0\\
7.23	0\\
7.24	0\\
7.25	0\\
7.26	0\\
7.27	1.73472347597681e-18\\
7.28	0\\
7.29	0\\
7.3	0\\
7.31	0\\
7.32	0\\
7.33	0\\
7.34	0\\
7.35	0\\
7.36	0\\
7.37	1.73472347597681e-18\\
7.38	0\\
7.39	0\\
7.4	0\\
7.41	0\\
7.42	0\\
7.43	0\\
7.44	0\\
7.45	0\\
7.46	0\\
7.47	0\\
7.48	0\\
7.49	0\\
7.5	0\\
7.51	0\\
7.52	0\\
7.53	0\\
7.54	0\\
7.55	1.73472347597681e-18\\
7.56	1.73472347597681e-18\\
7.57	0\\
7.58	1.73472347597681e-18\\
7.59	0\\
7.6	0\\
7.61	0\\
7.62	0\\
7.63	0\\
7.64	0\\
7.65	0\\
7.66	0\\
7.67	0\\
7.68	0\\
7.69	0\\
7.7	0\\
7.71	0\\
7.72	0\\
7.73	0\\
7.74	0\\
7.75	0\\
7.76	0\\
7.77	0\\
7.78	0\\
7.79	0\\
7.8	0\\
7.81	0\\
7.82	0\\
7.83	0\\
7.84	0\\
7.85	1.73472347597681e-18\\
7.86	0\\
7.87	0\\
7.88	0\\
7.89	0\\
7.9	0\\
7.91	0\\
7.92	0\\
7.93	0\\
7.94	0\\
7.95	1.73472347597681e-18\\
7.96	0\\
7.97	0\\
7.98	0\\
7.99	0\\
8	0\\
8.01	0\\
8.02	0\\
8.03	1.73472347597681e-18\\
8.04	1.73472347597681e-18\\
8.05	1.73472347597681e-18\\
8.06	0\\
8.07	1.73472347597681e-18\\
8.08	0\\
8.09	1.73472347597681e-18\\
8.1	0\\
8.11	0\\
8.12	0\\
8.13	1.73472347597681e-18\\
8.14	0\\
8.15	0\\
8.16	0\\
8.17	0\\
8.18	0\\
8.19	0\\
8.2	0\\
8.21	0\\
8.22	0\\
8.23	0\\
8.24	0\\
8.25	0\\
8.26	0\\
8.27	0\\
8.28	0\\
8.29	0\\
8.3	0\\
8.31	0\\
8.32	0\\
8.33	0\\
8.34	0\\
8.35	0\\
8.36	0\\
8.37	0\\
8.38	0\\
8.39	1.73472347597681e-18\\
8.4	0\\
8.41	0\\
8.42	0\\
8.43	0\\
8.44	0\\
8.45	1.73472347597681e-18\\
8.46	0\\
8.47	0\\
8.48	0\\
8.49	0\\
8.5	0\\
8.51	0\\
8.52	0\\
8.53	0\\
8.54	0\\
8.55	0\\
8.56	0\\
8.57	0\\
8.58	0\\
8.59	0\\
8.6	0\\
8.61	0\\
8.62	0\\
8.63	0\\
8.64	1.73472347597681e-18\\
8.65	0\\
8.66	0\\
8.67	0\\
8.68	1.73472347597681e-18\\
8.69	0\\
8.7	0\\
8.71	0\\
8.72	0\\
8.73	0\\
8.74	0\\
8.75	0\\
8.76	0\\
8.77	0\\
8.78	0\\
8.79	1.73472347597681e-18\\
8.8	0\\
8.81	0\\
8.82	0\\
8.83	0\\
8.84	1.73472347597681e-18\\
8.85	0\\
8.86	1.73472347597681e-18\\
8.87	0\\
8.88	0\\
8.89	1.73472347597681e-18\\
8.9	0\\
8.91	0\\
8.92	0\\
8.93	0\\
8.94	0\\
8.95	0\\
8.96	0\\
8.97	0\\
8.98	0\\
8.99	0\\
9	0\\
9.01	0\\
9.02	0\\
9.03	1.73472347597681e-18\\
9.04	0\\
9.05	0\\
9.06	0\\
9.07	0\\
9.08	0\\
9.09	0\\
9.1	1.73472347597681e-18\\
9.11	0\\
9.12	0\\
9.13	0\\
9.14	0\\
9.15	1.73472347597681e-18\\
9.16	0\\
9.17	1.73472347597681e-18\\
9.18	0\\
9.19	1.73472347597681e-18\\
9.2	0\\
9.21	0\\
9.22	1.73472347597681e-18\\
9.23	0\\
9.24	0\\
9.25	1.73472347597681e-18\\
9.26	1.73472347597681e-18\\
9.27	0\\
9.28	0\\
9.29	0\\
9.3	0\\
9.31	1.73472347597681e-18\\
9.32	0\\
9.33	1.73472347597681e-18\\
9.34	0\\
9.35	0\\
9.36	0\\
9.37	0\\
9.38	0\\
9.39	1.73472347597681e-18\\
9.4	0\\
9.41	1.73472347597681e-18\\
9.42	0\\
9.43	0\\
9.44	0\\
9.45	0\\
9.46	0\\
9.47	0\\
9.48	0\\
9.49	0\\
9.5	0\\
9.51	1.73472347597681e-18\\
9.52	0\\
9.53	0\\
9.54	0\\
9.55	1.73472347597681e-18\\
9.56	0\\
9.57	0\\
9.58	0\\
9.59	0\\
9.6	0\\
9.61	0\\
9.62	0\\
9.63	0\\
9.64	1.73472347597681e-18\\
9.65	0\\
9.66	0\\
9.67	0\\
9.68	1.73472347597681e-18\\
9.69	0\\
9.7	0\\
9.71	0\\
9.72	1.73472347597681e-18\\
9.73	0\\
9.74	0\\
9.75	0\\
9.76	0\\
9.77	0\\
9.78	1.73472347597681e-18\\
9.79	0\\
9.8	0\\
9.81	1.73472347597681e-18\\
9.82	0\\
9.83	0\\
9.84	0\\
9.85	0\\
9.86	0\\
9.87	1.73472347597681e-18\\
9.88	0\\
9.89	0\\
9.9	1.73472347597681e-18\\
9.91	0\\
9.92	0\\
9.93	0\\
9.94	0\\
9.95	0\\
9.96	0\\
9.97	0\\
9.98	1.73472347597681e-18\\
9.99	0\\
10	0\\
10.01	0\\
10.02	0\\
10.03	0\\
10.04	0\\
10.05	0\\
10.06	0\\
10.07	0\\
10.08	0\\
10.09	0\\
10.1	0\\
10.11	0\\
10.12	0\\
10.13	0\\
10.14	0\\
10.15	0\\
10.16	0\\
10.17	0\\
10.18	0\\
10.19	0\\
10.2	0\\
10.21	0\\
10.22	0\\
10.23	0\\
10.24	0\\
10.25	0\\
10.26	0\\
10.27	0\\
10.28	0\\
10.29	0\\
10.3	0\\
10.31	0\\
10.32	0\\
10.33	0\\
10.34	0\\
10.35	0\\
10.36	1.73472347597681e-18\\
10.37	1.73472347597681e-18\\
10.38	0\\
10.39	0\\
10.4	0\\
10.41	0\\
10.42	1.73472347597681e-18\\
10.43	1.73472347597681e-18\\
10.44	0\\
10.45	0\\
10.46	0\\
10.47	0\\
10.48	0\\
10.49	0\\
10.5	0\\
10.51	0\\
10.52	0\\
10.53	0\\
10.54	0\\
10.55	0\\
10.56	0\\
10.57	0\\
10.58	0\\
10.59	1.73472347597681e-18\\
10.6	0\\
10.61	1.73472347597681e-18\\
10.62	0\\
10.63	0\\
10.64	0\\
10.65	0\\
10.66	0\\
10.67	0\\
10.68	0\\
10.69	0\\
10.7	0\\
10.71	0\\
10.72	0\\
10.73	0\\
10.74	0\\
10.75	0\\
10.76	1.73472347597681e-18\\
10.77	1.73472347597681e-18\\
10.78	0\\
10.79	1.73472347597681e-18\\
10.8	0\\
10.81	0\\
10.82	0\\
10.83	0\\
10.84	0\\
10.85	0\\
10.86	0\\
10.87	0\\
10.88	1.73472347597681e-18\\
10.89	0\\
10.9	1.73472347597681e-18\\
10.91	0\\
10.92	0\\
10.93	0\\
10.94	0\\
10.95	0\\
10.96	0\\
10.97	0\\
10.98	0\\
10.99	0\\
11	0\\
11.01	0\\
11.02	0\\
11.03	0\\
11.04	0\\
11.05	0\\
11.06	0\\
11.07	0\\
11.08	0\\
11.09	0\\
11.1	0\\
11.11	1.73472347597681e-18\\
11.12	0\\
11.13	0\\
11.14	0\\
11.15	0\\
11.16	1.73472347597681e-18\\
11.17	0\\
11.18	0\\
11.19	1.73472347597681e-18\\
11.2	0\\
11.21	0\\
11.22	0\\
11.23	0\\
11.24	1.73472347597681e-18\\
11.25	0\\
11.26	0\\
11.27	0\\
11.28	0\\
11.29	0\\
11.3	0\\
11.31	0\\
11.32	0\\
11.33	0\\
11.34	0\\
11.35	1.73472347597681e-18\\
11.36	0\\
11.37	0\\
11.38	0\\
11.39	1.73472347597681e-18\\
11.4	0\\
11.41	1.73472347597681e-18\\
11.42	0\\
11.43	0\\
11.44	0\\
11.45	0\\
11.46	0\\
11.47	0\\
11.48	0\\
11.49	0\\
11.5	0\\
11.51	0\\
11.52	0\\
11.53	0\\
11.54	0\\
11.55	0\\
11.56	0\\
11.57	0\\
11.58	1.73472347597681e-18\\
11.59	1.73472347597681e-18\\
11.6	0\\
11.61	0\\
11.62	0\\
11.63	1.73472347597681e-18\\
11.64	0\\
11.65	0\\
11.66	0\\
11.67	0\\
11.68	0\\
11.69	1.73472347597681e-18\\
11.7	0\\
11.71	0\\
11.72	0\\
11.73	0\\
11.74	0\\
11.75	0\\
11.76	0\\
11.77	1.73472347597681e-18\\
11.78	0\\
11.79	0\\
11.8	0\\
11.81	0\\
11.82	0\\
11.83	0\\
11.84	0\\
11.85	0\\
11.86	0\\
11.87	0\\
11.88	0\\
11.89	0\\
11.9	0\\
11.91	1.73472347597681e-18\\
11.92	0\\
11.93	0\\
11.94	1.73472347597681e-18\\
11.95	0\\
11.96	0\\
11.97	0\\
11.98	0\\
11.99	0\\
12	0\\
12.01	0\\
12.02	0\\
12.03	0\\
12.04	1.73472347597681e-18\\
12.05	0\\
12.06	0\\
12.07	0\\
12.08	0\\
12.09	1.73472347597681e-18\\
12.1	0\\
12.11	1.73472347597681e-18\\
12.12	0\\
12.13	0\\
12.14	0\\
12.15	0\\
12.16	0\\
12.17	1.73472347597681e-18\\
12.18	1.73472347597681e-18\\
12.19	1.73472347597681e-18\\
12.2	0\\
12.21	0\\
12.22	0\\
12.23	0\\
12.24	0\\
12.25	1.73472347597681e-18\\
12.26	0\\
12.27	0\\
12.28	1.73472347597681e-18\\
12.29	0\\
12.3	0\\
12.31	0\\
12.32	0\\
12.33	0\\
12.34	0\\
12.35	1.73472347597681e-18\\
12.36	0\\
12.37	0\\
12.38	1.73472347597681e-18\\
12.39	0\\
12.4	0\\
12.41	0\\
12.42	0\\
12.43	0\\
12.44	0\\
12.45	0\\
12.46	0\\
12.47	1.73472347597681e-18\\
12.48	0\\
12.49	0\\
12.5	0\\
12.51	0\\
12.52	0\\
12.53	0\\
12.54	0\\
12.55	0\\
12.56	0\\
12.57	0\\
12.58	0\\
12.59	0\\
12.6	0\\
12.61	0\\
12.62	0\\
12.63	0\\
12.64	0\\
12.65	0\\
12.66	0\\
12.67	0\\
12.68	0\\
12.69	0\\
12.7	0\\
12.71	0\\
12.72	1.73472347597681e-18\\
12.73	0\\
12.74	0\\
12.75	0\\
12.76	0\\
12.77	0\\
12.78	0\\
12.79	0\\
12.8	0\\
12.81	1.73472347597681e-18\\
12.82	0\\
12.83	0\\
12.84	1.73472347597681e-18\\
12.85	0\\
12.86	1.73472347597681e-18\\
12.87	0\\
12.88	0\\
12.89	0\\
12.9	0\\
12.91	0\\
12.92	0\\
12.93	0\\
12.94	0\\
12.95	1.73472347597681e-18\\
12.96	0\\
12.97	0\\
12.98	0\\
12.99	0\\
13	0\\
13.01	0\\
13.02	0\\
13.03	0\\
13.04	1.73472347597681e-18\\
13.05	0\\
13.06	0\\
13.07	1.73472347597681e-18\\
13.08	0\\
13.09	0\\
13.1	0\\
13.11	0\\
13.12	0\\
13.13	0\\
13.14	0\\
13.15	0\\
13.16	0\\
13.17	0\\
13.18	0\\
13.19	0\\
13.2	0\\
13.21	0\\
13.22	0\\
13.23	0\\
13.24	0\\
13.25	1.73472347597681e-18\\
13.26	0\\
13.27	0\\
13.28	0\\
13.29	0\\
13.3	0\\
13.31	1.73472347597681e-18\\
13.32	0\\
13.33	1.73472347597681e-18\\
13.34	1.73472347597681e-18\\
13.35	0\\
13.36	1.73472347597681e-18\\
13.37	0\\
13.38	1.73472347597681e-18\\
13.39	0\\
13.4	0\\
13.41	0\\
13.42	0\\
13.43	0\\
13.44	0\\
13.45	0\\
13.46	0\\
13.47	1.73472347597681e-18\\
13.48	0\\
13.49	0\\
13.5	0\\
13.51	0\\
13.52	0\\
13.53	0\\
13.54	0\\
13.55	0\\
13.56	0\\
13.57	0\\
13.58	0\\
13.59	0\\
13.6	0\\
13.61	1.73472347597681e-18\\
13.62	0\\
13.63	1.73472347597681e-18\\
13.64	0\\
13.65	0\\
13.66	1.73472347597681e-18\\
13.67	1.73472347597681e-18\\
13.68	0\\
13.69	0\\
13.7	1.73472347597681e-18\\
13.71	0\\
13.72	1.73472347597681e-18\\
13.73	0\\
13.74	1.73472347597681e-18\\
13.75	0\\
13.76	0\\
13.77	0\\
13.78	0\\
13.79	1.73472347597681e-18\\
13.8	0\\
13.81	1.73472347597681e-18\\
13.82	1.73472347597681e-18\\
13.83	0\\
13.84	0\\
13.85	0\\
13.86	0\\
13.87	0\\
13.88	0\\
13.89	1.73472347597681e-18\\
13.9	0\\
13.91	0\\
13.92	1.73472347597681e-18\\
13.93	0\\
13.94	0\\
13.95	0\\
13.96	0\\
13.97	0\\
13.98	0\\
13.99	1.73472347597681e-18\\
14	1.73472347597681e-18\\
14.01	1.73472347597681e-18\\
14.02	0\\
14.03	0\\
14.04	0\\
14.05	0\\
14.06	0\\
14.07	0\\
14.08	0\\
14.09	0\\
14.1	1.73472347597681e-18\\
14.11	0\\
14.12	0\\
14.13	0\\
14.14	0\\
14.15	0\\
14.16	0\\
14.17	0\\
14.18	0\\
14.19	1.73472347597681e-18\\
14.2	0\\
14.21	0\\
14.22	0\\
14.23	0\\
14.24	0\\
14.25	0\\
14.26	0\\
14.27	0\\
14.28	0\\
14.29	0\\
14.3	0\\
14.31	0\\
14.32	0\\
14.33	1.73472347597681e-18\\
14.34	0\\
14.35	0\\
14.36	0\\
14.37	0\\
14.38	0\\
14.39	1.73472347597681e-18\\
14.4	0\\
14.41	1.73472347597681e-18\\
14.42	0\\
14.43	0\\
14.44	0\\
14.45	0\\
14.46	1.73472347597681e-18\\
14.47	1.73472347597681e-18\\
14.48	0\\
14.49	0\\
14.5	0\\
14.51	0\\
14.52	0\\
14.53	0\\
14.54	0\\
14.55	0\\
14.56	0\\
14.57	0\\
14.58	0\\
14.59	0\\
14.6	0\\
14.61	0\\
14.62	0\\
14.63	1.73472347597681e-18\\
14.64	0\\
14.65	0\\
14.66	0\\
14.67	0\\
14.68	0\\
14.69	0\\
14.7	1.73472347597681e-18\\
14.71	0\\
14.72	1.73472347597681e-18\\
14.73	1.73472347597681e-18\\
14.74	0\\
14.75	0\\
14.76	0\\
14.77	0\\
14.78	0\\
14.79	0\\
14.8	0\\
14.81	0\\
14.82	0\\
14.83	0\\
14.84	1.73472347597681e-18\\
14.85	0\\
14.86	0\\
14.87	0\\
14.88	0\\
14.89	0\\
14.9	0\\
14.91	1.73472347597681e-18\\
14.92	0\\
14.93	0\\
14.94	0\\
14.95	1.73472347597681e-18\\
14.96	1.73472347597681e-18\\
14.97	0\\
14.98	0\\
14.99	0\\
15	0\\
15.01	1.73472347597681e-18\\
15.02	0\\
15.03	0\\
15.04	0\\
15.05	1.73472347597681e-18\\
15.06	0\\
15.07	0\\
15.08	0\\
15.09	0\\
15.1	0\\
15.11	0\\
15.12	0\\
15.13	0\\
15.14	0\\
15.15	0\\
15.16	0\\
15.17	0\\
15.18	0\\
15.19	0\\
15.2	1.73472347597681e-18\\
15.21	0\\
15.22	0\\
15.23	0\\
15.24	1.73472347597681e-18\\
15.25	1.73472347597681e-18\\
15.26	0\\
15.27	0\\
15.28	0\\
15.29	0\\
15.3	0\\
15.31	0\\
15.32	0\\
15.33	1.73472347597681e-18\\
15.34	0\\
15.35	0\\
15.36	0\\
15.37	0\\
15.38	0\\
15.39	0\\
15.4	0\\
15.41	0\\
15.42	0\\
15.43	0\\
15.44	0\\
15.45	0\\
15.46	0\\
15.47	0\\
15.48	0\\
15.49	0\\
15.5	0\\
15.51	0\\
15.52	0\\
15.53	0\\
15.54	0\\
15.55	0\\
15.56	0\\
15.57	0\\
15.58	1.73472347597681e-18\\
15.59	0\\
15.6	0\\
15.61	0\\
15.62	0\\
15.63	1.73472347597681e-18\\
15.64	0\\
15.65	0\\
15.66	0\\
15.67	0\\
15.68	1.73472347597681e-18\\
15.69	0\\
15.7	0\\
15.71	0\\
15.72	0\\
15.73	0\\
15.74	0\\
15.75	1.73472347597681e-18\\
15.76	0\\
15.77	0\\
15.78	0\\
15.79	0\\
15.8	0\\
15.81	0\\
15.82	0\\
15.83	0\\
15.84	0\\
15.85	0\\
15.86	0\\
15.87	0\\
15.88	0\\
15.89	0\\
15.9	0\\
15.91	0\\
15.92	0\\
15.93	0\\
15.94	0\\
15.95	0\\
15.96	0\\
15.97	0\\
15.98	0\\
15.99	0\\
16	1.73472347597681e-18\\
16.01	0\\
16.02	0\\
16.03	0\\
16.04	0\\
16.05	0\\
16.06	0\\
16.07	0\\
16.08	1.73472347597681e-18\\
16.09	0\\
16.1	0\\
16.11	0\\
16.12	0\\
16.13	0\\
16.14	0\\
16.15	0\\
16.16	0\\
16.17	0\\
16.18	0\\
16.19	0\\
16.2	0\\
16.21	0\\
16.22	0\\
16.23	0\\
16.24	0\\
16.25	0\\
16.26	0\\
16.27	0\\
16.28	0\\
16.29	0\\
16.3	0\\
16.31	0\\
16.32	0\\
16.33	0\\
16.34	0\\
16.35	1.73472347597681e-18\\
16.36	0\\
16.37	1.73472347597681e-18\\
16.38	0\\
16.39	0\\
16.4	0\\
16.41	0\\
16.42	0\\
16.43	0\\
16.44	0\\
16.45	0\\
16.46	0\\
16.47	0\\
16.48	0\\
16.49	0\\
16.5	0\\
16.51	0\\
16.52	0\\
16.53	0\\
16.54	0\\
16.55	0\\
16.56	0\\
16.57	0\\
16.58	0\\
16.59	0\\
16.6	0\\
16.61	0\\
16.62	0\\
16.63	0\\
16.64	1.73472347597681e-18\\
16.65	1.73472347597681e-18\\
16.66	0\\
16.67	0\\
16.68	1.73472347597681e-18\\
16.69	0\\
16.7	0\\
16.71	0\\
16.72	0\\
16.73	0\\
16.74	0\\
16.75	0\\
16.76	1.73472347597681e-18\\
16.77	0\\
16.78	1.73472347597681e-18\\
16.79	0\\
16.8	0\\
16.81	0\\
16.82	0\\
16.83	1.73472347597681e-18\\
16.84	1.73472347597681e-18\\
16.85	0\\
16.86	1.73472347597681e-18\\
16.87	0\\
16.88	0\\
16.89	1.73472347597681e-18\\
16.9	0\\
16.91	0\\
16.92	0\\
16.93	1.73472347597681e-18\\
16.94	0\\
16.95	1.73472347597681e-18\\
16.96	0\\
16.97	0\\
16.98	0\\
16.99	1.73472347597681e-18\\
17	1.73472347597681e-18\\
17.01	1.73472347597681e-18\\
17.02	0\\
17.03	0\\
17.04	0\\
17.05	0\\
17.06	0\\
17.07	0\\
17.08	0\\
17.09	1.73472347597681e-18\\
17.1	0\\
17.11	0\\
17.12	0\\
17.13	1.73472347597681e-18\\
17.14	0\\
17.15	0\\
17.16	0\\
17.17	1.73472347597681e-18\\
17.18	0\\
17.19	0\\
17.2	1.73472347597681e-18\\
17.21	0\\
17.22	1.73472347597681e-18\\
17.23	0\\
17.24	0\\
17.25	0\\
17.26	0\\
17.27	0\\
17.28	0\\
17.29	0\\
17.3	0\\
17.31	0\\
17.32	0\\
17.33	0\\
17.34	0\\
17.35	0\\
17.36	0\\
17.37	0\\
17.38	0\\
17.39	0\\
17.4	0\\
17.41	1.73472347597681e-18\\
17.42	1.73472347597681e-18\\
17.43	0\\
17.44	0\\
17.45	0\\
17.46	1.73472347597681e-18\\
17.47	0\\
17.48	0\\
17.49	0\\
17.5	0\\
17.51	0\\
17.52	0\\
17.53	0\\
17.54	0\\
17.55	0\\
17.56	0\\
17.57	0\\
17.58	0\\
17.59	0\\
17.6	0\\
17.61	0\\
17.62	0\\
17.63	0\\
17.64	0\\
17.65	0\\
17.66	0\\
17.67	0\\
17.68	0\\
17.69	0\\
17.7	0\\
17.71	0\\
17.72	0\\
17.73	0\\
17.74	0\\
17.75	0\\
17.76	0\\
17.77	0\\
17.78	0\\
17.79	1.73472347597681e-18\\
17.8	0\\
17.81	1.73472347597681e-18\\
17.82	0\\
17.83	1.73472347597681e-18\\
17.84	0\\
17.85	0\\
17.86	0\\
17.87	1.73472347597681e-18\\
17.88	0\\
17.89	0\\
17.9	0\\
17.91	0\\
17.92	1.73472347597681e-18\\
17.93	0\\
17.94	0\\
17.95	0\\
17.96	0\\
17.97	1.73472347597681e-18\\
17.98	0\\
17.99	0\\
18	0\\
18.01	0\\
18.02	0\\
18.03	0\\
18.04	0\\
18.05	0\\
18.06	0\\
18.07	0\\
18.08	0\\
18.09	0\\
18.1	0\\
18.11	1.73472347597681e-18\\
18.12	0\\
18.13	0\\
18.14	0\\
18.15	0\\
18.16	0\\
18.17	0\\
18.18	0\\
18.19	0\\
18.2	1.73472347597681e-18\\
18.21	0\\
18.22	0\\
18.23	0\\
18.24	0\\
18.25	0\\
18.26	0\\
18.27	1.73472347597681e-18\\
18.28	0\\
18.29	0\\
18.3	0\\
18.31	0\\
18.32	0\\
18.33	0\\
18.34	0\\
18.35	1.73472347597681e-18\\
18.36	1.73472347597681e-18\\
18.37	0\\
18.38	0\\
18.39	0\\
18.4	0\\
18.41	0\\
18.42	0\\
18.43	1.73472347597681e-18\\
18.44	0\\
18.45	0\\
18.46	0\\
18.47	0\\
18.48	0\\
18.49	0\\
18.5	0\\
18.51	0\\
18.52	0\\
18.53	0\\
18.54	0\\
18.55	0\\
18.56	0\\
18.57	0\\
18.58	0\\
18.59	0\\
18.6	0\\
18.61	0\\
18.62	0\\
18.63	0\\
18.64	0\\
18.65	0\\
18.66	0\\
18.67	0\\
18.68	0\\
18.69	0\\
18.7	0\\
18.71	0\\
18.72	0\\
18.73	1.73472347597681e-18\\
18.74	0\\
18.75	0\\
18.76	0\\
18.77	0\\
18.78	1.73472347597681e-18\\
18.79	0\\
18.8	0\\
18.81	0\\
18.82	0\\
18.83	0\\
18.84	0\\
18.85	0\\
18.86	0\\
18.87	0\\
18.88	0\\
18.89	0\\
18.9	0\\
18.91	0\\
18.92	0\\
18.93	0\\
18.94	0\\
18.95	0\\
18.96	0\\
18.97	0\\
18.98	0\\
18.99	0\\
19	0\\
19.01	1.73472347597681e-18\\
19.02	0\\
19.03	0\\
19.04	1.73472347597681e-18\\
19.05	0\\
19.06	0\\
19.07	1.73472347597681e-18\\
19.08	0\\
19.09	0\\
19.1	0\\
19.11	0\\
19.12	1.73472347597681e-18\\
19.13	0\\
19.14	0\\
19.15	0\\
19.16	0\\
19.17	0\\
19.18	0\\
19.19	1.73472347597681e-18\\
19.2	0\\
19.21	0\\
19.22	0\\
19.23	0\\
19.24	0\\
19.25	0\\
19.26	0\\
19.27	0\\
19.28	0\\
19.29	1.73472347597681e-18\\
19.3	0\\
19.31	0\\
19.32	0\\
19.33	0\\
19.34	0\\
19.35	0\\
19.36	0\\
19.37	0\\
19.38	0\\
19.39	0\\
19.4	0\\
19.41	0\\
19.42	0\\
19.43	0\\
19.44	0\\
19.45	0\\
19.46	0\\
19.47	0\\
19.48	0\\
19.49	1.73472347597681e-18\\
19.5	0\\
19.51	0\\
19.52	0\\
19.53	0\\
19.54	0\\
19.55	0\\
19.56	0\\
19.57	0\\
19.58	0\\
19.59	0\\
19.6	0\\
19.61	0\\
19.62	1.73472347597681e-18\\
19.63	1.73472347597681e-18\\
19.64	0\\
19.65	0\\
19.66	1.73472347597681e-18\\
19.67	0\\
19.68	0\\
19.69	0\\
19.7	0\\
19.71	0\\
19.72	0\\
19.73	0\\
19.74	0\\
19.75	0\\
19.76	0\\
19.77	0\\
19.78	0\\
19.79	0\\
19.8	0\\
19.81	1.73472347597681e-18\\
19.82	1.73472347597681e-18\\
19.83	0\\
19.84	0\\
19.85	0\\
19.86	1.73472347597681e-18\\
19.87	0\\
19.88	0\\
19.89	0\\
19.9	0\\
19.91	0\\
19.92	0\\
19.93	0\\
19.94	1.73472347597681e-18\\
19.95	0\\
19.96	0\\
19.97	0\\
19.98	0\\
19.99	0\\
20	0\\
20.01	0\\
20.02	1.73472347597681e-18\\
20.03	0\\
20.04	0\\
20.05	0\\
20.06	0\\
20.07	0\\
20.08	0\\
20.09	1.73472347597681e-18\\
20.1	0\\
20.11	0\\
20.12	0\\
20.13	1.73472347597681e-18\\
20.14	0\\
20.15	0\\
20.16	0\\
20.17	0\\
20.18	0\\
20.19	0\\
20.2	0\\
20.21	0\\
20.22	1.73472347597681e-18\\
20.23	0\\
20.24	0\\
20.25	0\\
20.26	0\\
20.27	0\\
20.28	0\\
20.29	0\\
20.3	0\\
20.31	0\\
20.32	0\\
20.33	0\\
20.34	0\\
20.35	0\\
20.36	0\\
20.37	0\\
20.38	1.73472347597681e-18\\
20.39	0\\
20.4	0\\
20.41	0\\
20.42	1.73472347597681e-18\\
20.43	0\\
20.44	1.73472347597681e-18\\
20.45	1.73472347597681e-18\\
20.46	0\\
20.47	0\\
20.48	0\\
20.49	0\\
20.5	0\\
20.51	0\\
20.52	0\\
20.53	0\\
20.54	0\\
20.55	0\\
20.56	0\\
20.57	0\\
20.58	0\\
20.59	0\\
20.6	1.73472347597681e-18\\
20.61	0\\
20.62	0\\
20.63	0\\
20.64	1.73472347597681e-18\\
20.65	1.73472347597681e-18\\
20.66	1.73472347597681e-18\\
20.67	1.73472347597681e-18\\
20.68	0\\
20.69	0\\
20.7	0\\
20.71	0\\
20.72	0\\
20.73	0\\
20.74	1.73472347597681e-18\\
20.75	0\\
20.76	0\\
20.77	1.73472347597681e-18\\
20.78	0\\
20.79	0\\
20.8	0\\
20.81	0\\
20.82	0\\
20.83	0\\
20.84	1.73472347597681e-18\\
20.85	0\\
20.86	0\\
20.87	0\\
20.88	0\\
20.89	0\\
20.9	0\\
20.91	0\\
20.92	0\\
20.93	0\\
20.94	0\\
20.95	0\\
20.96	0\\
20.97	0\\
20.98	0\\
20.99	0\\
21	1.73472347597681e-18\\
21.01	1.73472347597681e-18\\
21.02	0\\
21.03	0\\
21.04	1.73472347597681e-18\\
21.05	0\\
21.06	0\\
21.07	0\\
21.08	0\\
21.09	0\\
21.1	0\\
21.11	0\\
21.12	0\\
21.13	0\\
21.14	0\\
21.15	0\\
21.16	1.73472347597681e-18\\
21.17	0\\
21.18	0\\
21.19	0\\
21.2	0\\
21.21	1.73472347597681e-18\\
21.22	0\\
21.23	0\\
21.24	0\\
21.25	1.73472347597681e-18\\
21.26	1.73472347597681e-18\\
21.27	1.73472347597681e-18\\
21.28	0\\
21.29	0\\
21.3	0\\
21.31	0\\
21.32	1.73472347597681e-18\\
21.33	0\\
21.34	0\\
21.35	0\\
21.36	0\\
21.37	0\\
21.38	0\\
21.39	1.73472347597681e-18\\
21.4	0\\
21.41	0\\
21.42	0\\
21.43	0\\
21.44	0\\
21.45	0\\
21.46	1.73472347597681e-18\\
21.47	0\\
21.48	0\\
21.49	0\\
21.5	0\\
21.51	0\\
21.52	0\\
21.53	0\\
21.54	0\\
21.55	0\\
21.56	0\\
21.57	1.73472347597681e-18\\
21.58	0\\
21.59	0\\
21.6	0\\
21.61	0\\
21.62	0\\
21.63	0\\
21.64	0\\
21.65	0\\
21.66	0\\
21.67	1.73472347597681e-18\\
21.68	0\\
21.69	0\\
21.7	1.73472347597681e-18\\
21.71	0\\
21.72	0\\
21.73	1.73472347597681e-18\\
21.74	0\\
21.75	0\\
21.76	0\\
21.77	0\\
21.78	1.73472347597681e-18\\
21.79	0\\
21.8	0\\
21.81	0\\
21.82	0\\
21.83	0\\
21.84	0\\
21.85	0\\
21.86	0\\
21.87	0\\
21.88	1.73472347597681e-18\\
21.89	0\\
21.9	0\\
21.91	0\\
21.92	1.73472347597681e-18\\
21.93	0\\
21.94	0\\
21.95	0\\
21.96	0\\
21.97	0\\
21.98	0\\
21.99	0\\
22	0\\
22.01	0\\
22.02	0\\
22.03	0\\
22.04	0\\
22.05	0\\
22.06	0\\
22.07	0\\
22.08	0\\
22.09	0\\
22.1	1.73472347597681e-18\\
22.11	0\\
22.12	1.73472347597681e-18\\
22.13	0\\
22.14	1.73472347597681e-18\\
22.15	1.73472347597681e-18\\
22.16	0\\
22.17	0\\
22.18	0\\
22.19	0\\
22.2	0\\
22.21	0\\
22.22	0\\
22.23	0\\
22.24	0\\
22.25	1.73472347597681e-18\\
22.26	1.73472347597681e-18\\
22.27	0\\
22.28	1.73472347597681e-18\\
22.29	0\\
22.3	0\\
22.31	0\\
22.32	0\\
22.33	0\\
22.34	0\\
22.35	0\\
22.36	0\\
22.37	0\\
22.38	0\\
22.39	0\\
22.4	0\\
22.41	0\\
22.42	1.73472347597681e-18\\
22.43	0\\
22.44	0\\
22.45	0\\
22.46	1.73472347597681e-18\\
22.47	1.73472347597681e-18\\
22.48	1.73472347597681e-18\\
22.49	0\\
22.5	0\\
22.51	0\\
22.52	0\\
22.53	0\\
22.54	0\\
22.55	0\\
22.56	0\\
22.57	0\\
22.58	0\\
22.59	0\\
22.6	0\\
22.61	0\\
22.62	0\\
22.63	0\\
22.64	0\\
22.65	1.73472347597681e-18\\
22.66	0\\
22.67	0\\
22.68	1.73472347597681e-18\\
22.69	1.73472347597681e-18\\
22.7	0\\
22.71	0\\
22.72	1.73472347597681e-18\\
22.73	0\\
22.74	0\\
22.75	0\\
22.76	0\\
22.77	0\\
22.78	0\\
22.79	0\\
22.8	0\\
22.81	0\\
22.82	0\\
22.83	0\\
22.84	0\\
22.85	0\\
22.86	0\\
22.87	0\\
22.88	0\\
22.89	0\\
22.9	0\\
22.91	0\\
22.92	0\\
22.93	0\\
22.94	0\\
22.95	0\\
22.96	0\\
22.97	0\\
22.98	0\\
22.99	0\\
23	0\\
23.01	0\\
23.02	0\\
23.03	0\\
23.04	0\\
23.05	0\\
23.06	0\\
23.07	1.73472347597681e-18\\
23.08	0\\
23.09	0\\
23.1	0\\
23.11	0\\
23.12	0\\
23.13	1.73472347597681e-18\\
23.14	0\\
23.15	0\\
23.16	0\\
23.17	0\\
23.18	0\\
23.19	0\\
23.2	0\\
23.21	1.73472347597681e-18\\
23.22	0\\
23.23	0\\
23.24	0\\
23.25	0\\
23.26	0\\
23.27	0\\
23.28	0\\
23.29	0\\
23.3	0\\
23.31	0\\
23.32	0\\
23.33	0\\
23.34	0\\
23.35	0\\
23.36	0\\
23.37	0\\
23.38	1.73472347597681e-18\\
23.39	0\\
23.4	0\\
23.41	0\\
23.42	0\\
23.43	0\\
23.44	0\\
23.45	0\\
23.46	0\\
23.47	0\\
23.48	0\\
23.49	0\\
23.5	0\\
23.51	0\\
23.52	0\\
23.53	0\\
23.54	0\\
23.55	1.73472347597681e-18\\
23.56	0\\
23.57	0\\
23.58	0\\
23.59	0\\
23.6	0\\
23.61	0\\
23.62	0\\
23.63	0\\
23.64	0\\
23.65	0\\
23.66	0\\
23.67	0\\
23.68	0\\
23.69	0\\
23.7	0\\
23.71	0\\
23.72	0\\
23.73	0\\
23.74	0\\
23.75	0\\
23.76	0\\
23.77	0\\
23.78	0\\
23.79	0\\
23.8	1.73472347597681e-18\\
23.81	0\\
23.82	0\\
23.83	0\\
23.84	0\\
23.85	0\\
23.86	1.73472347597681e-18\\
23.87	0\\
23.88	0\\
23.89	0\\
23.9	0\\
23.91	0\\
23.92	0\\
23.93	0\\
23.94	0\\
23.95	0\\
23.96	0\\
23.97	0\\
23.98	0\\
23.99	0\\
24	0\\
24.01	1.73472347597681e-18\\
24.02	0\\
24.03	0\\
24.04	0\\
24.05	0\\
24.06	0\\
24.07	0\\
24.08	0\\
24.09	0\\
24.1	0\\
24.11	0\\
24.12	0\\
24.13	0\\
24.14	0\\
24.15	0\\
24.16	0\\
24.17	1.73472347597681e-18\\
24.18	1.73472347597681e-18\\
24.19	0\\
24.2	0\\
24.21	0\\
24.22	1.73472347597681e-18\\
24.23	0\\
24.24	0\\
24.25	0\\
24.26	0\\
24.27	0\\
24.28	0\\
24.29	0\\
24.3	0\\
24.31	0\\
24.32	0\\
24.33	1.73472347597681e-18\\
24.34	0\\
24.35	0\\
24.36	0\\
24.37	0\\
24.38	0\\
24.39	0\\
24.4	1.73472347597681e-18\\
24.41	1.73472347597681e-18\\
24.42	0\\
24.43	0\\
24.44	0\\
24.45	0\\
24.46	0\\
24.47	1.73472347597681e-18\\
24.48	1.73472347597681e-18\\
24.49	0\\
24.5	0\\
24.51	0\\
24.52	0\\
24.53	0\\
24.54	0\\
24.55	0\\
24.56	0\\
24.57	0\\
24.58	0\\
24.59	0\\
24.6	0\\
24.61	0\\
24.62	0\\
24.63	0\\
24.64	0\\
24.65	1.73472347597681e-18\\
24.66	0\\
24.67	1.73472347597681e-18\\
24.68	0\\
24.69	0\\
24.7	0\\
24.71	0\\
24.72	0\\
24.73	0\\
24.74	1.73472347597681e-18\\
24.75	0\\
24.76	0\\
24.77	0\\
24.78	0\\
24.79	0\\
24.8	0\\
24.81	0\\
24.82	0\\
24.83	0\\
24.84	0\\
24.85	0\\
24.86	0\\
24.87	0\\
24.88	0\\
24.89	0\\
24.9	0\\
24.91	0\\
24.92	1.73472347597681e-18\\
24.93	0\\
24.94	0\\
24.95	0\\
24.96	0\\
24.97	0\\
24.98	0\\
24.99	0\\
25	0\\
25.01	1.73472347597681e-18\\
25.02	1.73472347597681e-18\\
25.03	0\\
25.04	1.73472347597681e-18\\
25.05	0\\
25.06	0\\
25.07	0\\
25.08	0\\
25.09	0\\
25.1	1.73472347597681e-18\\
25.11	1.73472347597681e-18\\
25.12	0\\
25.13	0\\
25.14	0\\
25.15	0\\
25.16	1.73472347597681e-18\\
25.17	0\\
25.18	1.73472347597681e-18\\
25.19	0\\
25.2	0\\
25.21	0\\
25.22	0\\
25.23	0\\
25.24	1.73472347597681e-18\\
25.25	1.73472347597681e-18\\
25.26	0\\
25.27	0\\
25.28	0\\
25.29	0\\
25.3	0\\
25.31	0\\
25.32	0\\
25.33	0\\
25.34	0\\
25.35	0\\
25.36	0\\
25.37	1.73472347597681e-18\\
25.38	0\\
25.39	0\\
25.4	0\\
25.41	1.73472347597681e-18\\
25.42	0\\
25.43	0\\
25.44	0\\
25.45	0\\
25.46	0\\
25.47	0\\
25.48	0\\
25.49	0\\
25.5	0\\
25.51	0\\
25.52	0\\
25.53	0\\
25.54	0\\
25.55	0\\
25.56	0\\
25.57	0\\
25.58	0\\
25.59	0\\
25.6	0\\
25.61	0\\
25.62	0\\
25.63	0\\
25.64	0\\
25.65	0\\
25.66	0\\
25.67	0\\
25.68	0\\
25.69	0\\
25.7	0\\
25.71	0\\
25.72	1.73472347597681e-18\\
25.73	0\\
25.74	1.73472347597681e-18\\
25.75	0\\
25.76	1.73472347597681e-18\\
25.77	0\\
25.78	1.73472347597681e-18\\
25.79	0\\
25.8	0\\
25.81	0\\
25.82	0\\
25.83	0\\
25.84	0\\
25.85	0\\
25.86	0\\
25.87	1.73472347597681e-18\\
25.88	0\\
25.89	0\\
25.9	0\\
25.91	0\\
25.92	0\\
25.93	0\\
25.94	0\\
25.95	0\\
25.96	1.73472347597681e-18\\
25.97	0\\
25.98	0\\
25.99	0\\
26	0\\
26.01	0\\
26.02	0\\
26.03	0\\
26.04	0\\
26.05	0\\
26.06	1.73472347597681e-18\\
26.07	0\\
26.08	0\\
26.09	1.73472347597681e-18\\
26.1	0\\
26.11	0\\
26.12	0\\
26.13	0\\
26.14	0\\
26.15	0\\
26.16	0\\
26.17	0\\
26.18	0\\
26.19	0\\
26.2	0\\
26.21	0\\
26.22	0\\
26.23	0\\
26.24	0\\
26.25	0\\
26.26	0\\
26.27	0\\
26.28	0\\
26.29	0\\
26.3	0\\
26.31	0\\
26.32	0\\
26.33	0\\
26.34	0\\
26.35	0\\
26.36	0\\
26.37	0\\
26.38	0\\
26.39	1.73472347597681e-18\\
26.4	0\\
26.41	0\\
26.42	0\\
26.43	0\\
26.44	0\\
26.45	0\\
26.46	0\\
26.47	0\\
26.48	0\\
26.49	0\\
26.5	0\\
26.51	0\\
26.52	0\\
26.53	0\\
26.54	0\\
26.55	0\\
26.56	0\\
26.57	0\\
26.58	0\\
26.59	0\\
26.6	0\\
26.61	0\\
26.62	0\\
26.63	0\\
26.64	0\\
26.65	0\\
26.66	0\\
26.67	0\\
26.68	0\\
26.69	0\\
26.7	0\\
26.71	1.73472347597681e-18\\
26.72	0\\
26.73	0\\
26.74	0\\
26.75	0\\
26.76	0\\
26.77	0\\
26.78	0\\
26.79	0\\
26.8	1.73472347597681e-18\\
26.81	0\\
26.82	0\\
26.83	0\\
26.84	1.73472347597681e-18\\
26.85	0\\
26.86	1.73472347597681e-18\\
26.87	0\\
26.88	0\\
26.89	0\\
26.9	0\\
26.91	1.73472347597681e-18\\
26.92	0\\
26.93	0\\
26.94	0\\
26.95	0\\
26.96	0\\
26.97	0\\
26.98	1.73472347597681e-18\\
26.99	0\\
27	0\\
27.01	0\\
27.02	0\\
27.03	0\\
27.04	0\\
27.05	0\\
27.06	0\\
27.07	0\\
27.08	0\\
27.09	0\\
27.1	0\\
27.11	1.73472347597681e-18\\
27.12	0\\
27.13	0\\
27.14	0\\
27.15	0\\
27.16	0\\
27.17	0\\
27.18	0\\
27.19	0\\
27.2	0\\
27.21	0\\
27.22	0\\
27.23	0\\
27.24	0\\
27.25	0\\
27.26	0\\
27.27	0\\
27.28	0\\
27.29	0\\
27.3	0\\
27.31	1.73472347597681e-18\\
27.32	0\\
27.33	0\\
27.34	0\\
27.35	0\\
27.36	0\\
27.37	0\\
27.38	0\\
27.39	0\\
27.4	0\\
27.41	0\\
27.42	0\\
27.43	1.73472347597681e-18\\
27.44	0\\
27.45	0\\
27.46	0\\
27.47	0\\
27.48	0\\
27.49	1.73472347597681e-18\\
27.5	1.73472347597681e-18\\
27.51	0\\
27.52	0\\
27.53	0\\
27.54	0\\
27.55	0\\
27.56	0\\
27.57	0\\
27.58	0\\
27.59	0\\
27.6	0\\
27.61	0\\
27.62	0\\
27.63	1.73472347597681e-18\\
27.64	0\\
27.65	0\\
27.66	0\\
27.67	0\\
27.68	0\\
27.69	0\\
27.7	0\\
27.71	0\\
27.72	0\\
27.73	0\\
27.74	0\\
27.75	1.73472347597681e-18\\
27.76	0\\
27.77	0\\
27.78	0\\
27.79	0\\
27.8	0\\
27.81	0\\
27.82	0\\
27.83	0\\
27.84	0\\
27.85	1.73472347597681e-18\\
27.86	0\\
27.87	1.73472347597681e-18\\
27.88	0\\
27.89	0\\
27.9	0\\
27.91	0\\
27.92	0\\
27.93	0\\
27.94	0\\
27.95	0\\
27.96	0\\
27.97	0\\
27.98	0\\
27.99	0\\
28	1.73472347597681e-18\\
28.01	0\\
28.02	0\\
28.03	1.73472347597681e-18\\
28.04	1.73472347597681e-18\\
28.05	0\\
28.06	0\\
28.07	0\\
28.08	0\\
28.09	0\\
28.1	0\\
28.11	1.73472347597681e-18\\
28.12	0\\
28.13	0\\
28.14	0\\
28.15	0\\
28.16	0\\
28.17	0\\
28.18	0\\
28.19	0\\
28.2	0\\
28.21	0\\
28.22	0\\
28.23	0\\
28.24	0\\
28.25	0\\
28.26	0\\
28.27	0\\
28.28	0\\
28.29	0\\
28.3	0\\
28.31	0\\
28.32	0\\
28.33	0\\
28.34	0\\
28.35	1.73472347597681e-18\\
28.36	0\\
28.37	0\\
28.38	0\\
28.39	0\\
28.4	0\\
28.41	0\\
28.42	0\\
28.43	0\\
28.44	0\\
28.45	0\\
28.46	0\\
28.47	0\\
28.48	0\\
28.49	0\\
28.5	0\\
28.51	0\\
28.52	0\\
28.53	0\\
28.54	1.73472347597681e-18\\
28.55	0\\
28.56	0\\
28.57	0\\
28.58	1.73472347597681e-18\\
28.59	1.73472347597681e-18\\
28.6	0\\
28.61	0\\
28.62	0\\
28.63	0\\
28.64	0\\
28.65	0\\
28.66	0\\
28.67	0\\
28.68	0\\
28.69	1.73472347597681e-18\\
28.7	0\\
28.71	0\\
28.72	0\\
28.73	0\\
28.74	0\\
28.75	0\\
28.76	0\\
28.77	0\\
28.78	0\\
28.79	0\\
28.8	0\\
28.81	1.73472347597681e-18\\
28.82	0\\
28.83	0\\
28.84	0\\
28.85	0\\
28.86	0\\
28.87	0\\
28.88	0\\
28.89	0\\
28.9	0\\
28.91	1.73472347597681e-18\\
28.92	1.73472347597681e-18\\
28.93	0\\
28.94	0\\
28.95	0\\
28.96	0\\
28.97	0\\
28.98	0\\
28.99	0\\
29	0\\
29.01	0\\
29.02	0\\
29.03	0\\
29.04	0\\
29.05	0\\
29.06	0\\
29.07	0\\
29.08	0\\
29.09	0\\
29.1	0\\
29.11	0\\
29.12	0\\
29.13	0\\
29.14	0\\
29.15	0\\
29.16	0\\
29.17	0\\
29.18	1.73472347597681e-18\\
29.19	0\\
29.2	0\\
29.21	0\\
29.22	0\\
29.23	0\\
29.24	0\\
29.25	0\\
29.26	0\\
29.27	0\\
29.28	0\\
29.29	0\\
29.3	0\\
29.31	1.73472347597681e-18\\
29.32	0\\
29.33	0\\
29.34	0\\
29.35	0\\
29.36	0\\
29.37	0\\
29.38	0\\
29.39	0\\
29.4	0\\
29.41	0\\
29.42	0\\
29.43	0\\
29.44	0\\
29.45	0\\
29.46	0\\
29.47	0\\
29.48	0\\
29.49	0\\
29.5	0\\
29.51	0\\
29.52	0\\
29.53	1.73472347597681e-18\\
29.54	0\\
29.55	0\\
29.56	0\\
29.57	0\\
29.58	0\\
29.59	0\\
29.6	0\\
29.61	0\\
29.62	0\\
29.63	1.73472347597681e-18\\
29.64	0\\
29.65	1.73472347597681e-18\\
29.66	1.73472347597681e-18\\
29.67	0\\
29.68	1.73472347597681e-18\\
29.69	0\\
29.7	0\\
29.71	0\\
29.72	1.73472347597681e-18\\
29.73	0\\
29.74	0\\
29.75	0\\
29.76	0\\
29.77	0\\
29.78	0\\
29.79	0\\
29.8	1.73472347597681e-18\\
29.81	0\\
29.82	0\\
29.83	0\\
29.84	0\\
29.85	0\\
29.86	0\\
29.87	1.73472347597681e-18\\
29.88	0\\
29.89	0\\
29.9	0\\
29.91	0\\
29.92	0\\
29.93	1.73472347597681e-18\\
29.94	0\\
29.95	0\\
29.96	0\\
29.97	0\\
29.98	0\\
29.99	0\\
30	1.73472347597681e-18\\
30.01	1.73472347597681e-18\\
30.02	0\\
30.03	0\\
30.04	0\\
30.05	0\\
30.06	0\\
30.07	0\\
30.08	0\\
30.09	0\\
30.1	0\\
30.11	1.73472347597681e-18\\
30.12	0\\
30.13	0\\
30.14	0\\
30.15	0\\
30.16	0\\
30.17	0\\
30.18	0\\
30.19	0\\
30.2	0\\
30.21	0\\
30.22	0\\
30.23	0\\
30.24	0\\
30.25	0\\
30.26	0\\
30.27	0\\
30.28	0\\
30.29	0\\
30.3	1.73472347597681e-18\\
30.31	0\\
30.32	0\\
30.33	0\\
30.34	0\\
30.35	0\\
30.36	0\\
30.37	0\\
30.38	1.73472347597681e-18\\
30.39	0\\
30.4	1.73472347597681e-18\\
30.41	0\\
30.42	0\\
30.43	0\\
30.44	0\\
30.45	0\\
30.46	0\\
30.47	0\\
30.48	0\\
30.49	0\\
30.5	0\\
30.51	0\\
30.52	0\\
30.53	0\\
30.54	0\\
30.55	0\\
30.56	0\\
30.57	0\\
30.58	0\\
30.59	0\\
30.6	0\\
30.61	0\\
30.62	0\\
30.63	0\\
30.64	0\\
30.65	0\\
30.66	0\\
30.67	0\\
30.68	1.73472347597681e-18\\
30.69	0\\
30.7	0\\
30.71	0\\
30.72	1.73472347597681e-18\\
30.73	0\\
30.74	0\\
30.75	0\\
30.76	0\\
30.77	0\\
30.78	0\\
30.79	0\\
30.8	1.73472347597681e-18\\
30.81	1.73472347597681e-18\\
30.82	0\\
30.83	1.73472347597681e-18\\
30.84	0\\
30.85	1.73472347597681e-18\\
30.86	0\\
30.87	0\\
30.88	0\\
30.89	0\\
30.9	0\\
30.91	1.73472347597681e-18\\
30.92	0\\
30.93	0\\
30.94	1.73472347597681e-18\\
30.95	0\\
30.96	0\\
30.97	1.73472347597681e-18\\
30.98	0\\
30.99	1.73472347597681e-18\\
31	0\\
31.01	0\\
31.02	0\\
31.03	0\\
31.04	0\\
31.05	0\\
31.06	0\\
31.07	0\\
31.08	1.73472347597681e-18\\
31.09	0\\
31.1	0\\
31.11	0\\
31.12	0\\
31.13	0\\
31.14	0\\
31.15	0\\
31.16	0\\
31.17	0\\
31.18	0\\
31.19	0\\
31.2	0\\
31.21	1.73472347597681e-18\\
31.22	1.73472347597681e-18\\
31.23	0\\
31.24	0\\
31.25	0\\
31.26	1.73472347597681e-18\\
31.27	0\\
31.28	0\\
31.29	0\\
31.3	0\\
31.31	0\\
31.32	0\\
31.33	0\\
31.34	0\\
31.35	0\\
31.36	0\\
31.37	0\\
31.38	0\\
31.39	1.73472347597681e-18\\
31.4	0\\
31.41	0\\
31.42	1.73472347597681e-18\\
31.43	1.73472347597681e-18\\
31.44	1.73472347597681e-18\\
31.45	0\\
31.46	0\\
31.47	0\\
31.48	0\\
31.49	0\\
31.5	0\\
31.51	0\\
31.52	0\\
31.53	0\\
31.54	0\\
31.55	1.73472347597681e-18\\
31.56	0\\
31.57	0\\
31.58	0\\
31.59	0\\
31.6	0\\
31.61	0\\
31.62	0\\
31.63	0\\
31.64	0\\
31.65	0\\
31.66	0\\
31.67	1.73472347597681e-18\\
31.68	0\\
31.69	0\\
31.7	0\\
31.71	0\\
31.72	0\\
31.73	0\\
31.74	0\\
31.75	0\\
31.76	0\\
31.77	0\\
31.78	0\\
31.79	0\\
31.8	0\\
31.81	0\\
31.82	0\\
31.83	1.73472347597681e-18\\
31.84	0\\
31.85	0\\
31.86	1.73472347597681e-18\\
31.87	0\\
31.88	1.73472347597681e-18\\
31.89	1.73472347597681e-18\\
31.9	0\\
31.91	0\\
31.92	0\\
31.93	0\\
31.94	1.73472347597681e-18\\
31.95	1.73472347597681e-18\\
31.96	0\\
31.97	1.73472347597681e-18\\
31.98	0\\
31.99	0\\
32	0\\
32.01	0\\
32.02	0\\
32.03	0\\
32.04	0\\
32.05	0\\
32.06	0\\
32.07	0\\
32.08	0\\
32.09	0\\
32.1	0\\
32.11	0\\
32.12	0\\
32.13	0\\
32.14	0\\
32.15	0\\
32.16	0\\
32.17	0\\
32.18	0\\
32.19	0\\
32.2	0\\
32.21	0\\
32.22	0\\
32.23	0\\
32.24	0\\
32.25	0\\
32.26	0\\
32.27	0\\
32.28	0\\
32.29	0\\
32.3	0\\
32.31	0\\
32.32	0\\
32.33	0\\
32.34	0\\
32.35	0\\
32.36	0\\
32.37	0\\
32.38	0\\
32.39	0\\
32.4	0\\
32.41	0\\
32.42	0\\
32.43	0\\
32.44	1.73472347597681e-18\\
32.45	1.73472347597681e-18\\
32.46	0\\
32.47	0\\
32.48	0\\
32.49	0\\
32.5	0\\
32.51	0\\
32.52	0\\
32.53	0\\
32.54	0\\
32.55	0\\
32.56	0\\
32.57	0\\
32.58	0\\
32.59	0\\
32.6	0\\
32.61	1.73472347597681e-18\\
32.62	0\\
32.63	0\\
32.64	0\\
32.65	0\\
32.66	0\\
32.67	0\\
32.68	0\\
32.69	0\\
32.7	0\\
32.71	0\\
32.72	0\\
32.73	0\\
32.74	0\\
32.75	0\\
32.76	0\\
32.77	0\\
32.78	0\\
32.79	0\\
32.8	0\\
32.81	0\\
32.82	0\\
32.83	1.73472347597681e-18\\
32.84	1.73472347597681e-18\\
32.85	0\\
32.86	0\\
32.87	0\\
32.88	0\\
32.89	1.73472347597681e-18\\
32.9	0\\
32.91	0\\
32.92	0\\
32.93	0\\
32.94	0\\
32.95	0\\
32.96	0\\
32.97	1.73472347597681e-18\\
32.98	0\\
32.99	0\\
33	0\\
33.01	0\\
33.02	0\\
33.03	0\\
33.04	0\\
33.05	0\\
33.06	0\\
33.07	0\\
33.08	0\\
33.09	0\\
33.1	0\\
33.11	1.73472347597681e-18\\
33.12	0\\
33.13	0\\
33.14	0\\
33.15	0\\
33.16	1.73472347597681e-18\\
33.17	0\\
33.18	0\\
33.19	0\\
33.2	0\\
33.21	0\\
33.22	0\\
33.23	0\\
33.24	0\\
33.25	1.73472347597681e-18\\
33.26	0\\
33.27	1.73472347597681e-18\\
33.28	1.73472347597681e-18\\
33.29	0\\
33.3	1.73472347597681e-18\\
33.31	0\\
33.32	0\\
33.33	0\\
33.34	0\\
33.35	0\\
33.36	1.73472347597681e-18\\
33.37	0\\
33.38	1.73472347597681e-18\\
33.39	0\\
33.4	0\\
33.41	0\\
33.42	0\\
33.43	0\\
33.44	0\\
33.45	0\\
33.46	0\\
33.47	0\\
33.48	0\\
33.49	0\\
33.5	0\\
33.51	0\\
33.52	0\\
33.53	0\\
33.54	0\\
33.55	0\\
33.56	0\\
33.57	0\\
33.58	1.73472347597681e-18\\
33.59	0\\
33.6	1.73472347597681e-18\\
33.61	0\\
33.62	0\\
33.63	0\\
33.64	0\\
33.65	0\\
33.66	0\\
33.67	0\\
33.68	0\\
33.69	0\\
33.7	0\\
33.71	0\\
33.72	0\\
33.73	0\\
33.74	0\\
33.75	0\\
33.76	1.73472347597681e-18\\
33.77	0\\
33.78	0\\
33.79	0\\
33.8	0\\
33.81	0\\
33.82	0\\
33.83	0\\
33.84	0\\
33.85	0\\
33.86	0\\
33.87	0\\
33.88	1.73472347597681e-18\\
33.89	0\\
33.9	0\\
33.91	0\\
33.92	0\\
33.93	0\\
33.94	1.73472347597681e-18\\
33.95	0\\
33.96	0\\
33.97	0\\
33.98	0\\
33.99	1.73472347597681e-18\\
34	0\\
34.01	0\\
34.02	0\\
34.03	0\\
34.04	0\\
34.05	0\\
34.06	0\\
34.07	0\\
34.08	0\\
34.09	0\\
34.1	0\\
34.11	0\\
34.12	0\\
34.13	0\\
34.14	0\\
34.15	0\\
34.16	1.73472347597681e-18\\
34.17	0\\
34.18	0\\
34.19	0\\
34.2	0\\
34.21	0\\
34.22	0\\
34.23	0\\
34.24	0\\
34.25	0\\
34.26	1.73472347597681e-18\\
34.27	0\\
34.28	0\\
34.29	0\\
34.3	0\\
34.31	0\\
34.32	0\\
34.33	0\\
34.34	1.73472347597681e-18\\
34.35	0\\
34.36	0\\
34.37	0\\
34.38	0\\
34.39	0\\
34.4	0\\
34.41	0\\
34.42	1.73472347597681e-18\\
34.43	0\\
34.44	0\\
34.45	0\\
34.46	0\\
34.47	0\\
34.48	0\\
34.49	1.73472347597681e-18\\
34.5	0\\
34.51	0\\
34.52	0\\
34.53	0\\
34.54	0\\
34.55	0\\
34.56	0\\
34.57	0\\
34.58	1.73472347597681e-18\\
34.59	0\\
34.6	0\\
34.61	0\\
34.62	0\\
34.63	0\\
34.64	0\\
34.65	0\\
34.66	0\\
34.67	0\\
34.68	0\\
34.69	0\\
34.7	1.73472347597681e-18\\
34.71	0\\
34.72	0\\
34.73	0\\
34.74	1.73472347597681e-18\\
34.75	1.73472347597681e-18\\
34.76	0\\
34.77	1.73472347597681e-18\\
34.78	1.73472347597681e-18\\
34.79	0\\
34.8	0\\
34.81	0\\
34.82	0\\
34.83	0\\
34.84	0\\
34.85	0\\
34.86	0\\
34.87	0\\
34.88	0\\
34.89	0\\
34.9	0\\
34.91	0\\
34.92	0\\
34.93	0\\
34.94	0\\
34.95	1.73472347597681e-18\\
34.96	1.73472347597681e-18\\
34.97	1.73472347597681e-18\\
34.98	0\\
34.99	0\\
35	0\\
35.01	0\\
35.02	0\\
35.03	0\\
35.04	0\\
35.05	0\\
35.06	0\\
35.07	1.73472347597681e-18\\
35.08	0\\
35.09	1.73472347597681e-18\\
35.1	0\\
35.11	0\\
35.12	0\\
35.13	1.73472347597681e-18\\
35.14	0\\
35.15	0\\
35.16	1.73472347597681e-18\\
35.17	0\\
35.18	0\\
35.19	0\\
35.2	0\\
35.21	0\\
35.22	0\\
35.23	0\\
35.24	0\\
35.25	0\\
35.26	0\\
35.27	0\\
35.28	0\\
35.29	0\\
35.3	0\\
35.31	0\\
35.32	0\\
35.33	0\\
35.34	0\\
35.35	0\\
35.36	0\\
35.37	0\\
35.38	0\\
35.39	0\\
35.4	0\\
35.41	0\\
35.42	0\\
35.43	1.73472347597681e-18\\
35.44	0\\
35.45	0\\
35.46	0\\
35.47	0\\
35.48	0\\
35.49	0\\
35.5	0\\
35.51	0\\
35.52	0\\
35.53	0\\
35.54	0\\
35.55	0\\
35.56	0\\
35.57	0\\
35.58	0\\
35.59	0\\
35.6	0\\
35.61	0\\
35.62	0\\
35.63	0\\
35.64	0\\
35.65	0\\
35.66	0\\
35.67	0\\
35.68	1.73472347597681e-18\\
35.69	1.73472347597681e-18\\
35.7	0\\
35.71	0\\
35.72	0\\
35.73	0\\
35.74	0\\
35.75	0\\
35.76	0\\
35.77	0\\
35.78	0\\
35.79	0\\
35.8	0\\
35.81	0\\
35.82	0\\
35.83	0\\
35.84	0\\
35.85	1.73472347597681e-18\\
35.86	0\\
35.87	0\\
35.88	0\\
35.89	0\\
35.9	0\\
35.91	1.73472347597681e-18\\
35.92	0\\
35.93	0\\
35.94	0\\
35.95	0\\
35.96	0\\
35.97	0\\
35.98	1.73472347597681e-18\\
35.99	0\\
36	0\\
36.01	0\\
36.02	0\\
36.03	0\\
36.04	0\\
36.05	0\\
36.06	0\\
36.07	1.73472347597681e-18\\
36.08	1.73472347597681e-18\\
36.09	0\\
36.1	0\\
36.11	0\\
36.12	0\\
36.13	0\\
36.14	0\\
36.15	0\\
36.16	1.73472347597681e-18\\
36.17	0\\
36.18	0\\
36.19	0\\
36.2	0\\
36.21	0\\
36.22	0\\
36.23	0\\
36.24	0\\
36.25	1.73472347597681e-18\\
36.26	0\\
36.27	0\\
36.28	0\\
36.29	0\\
36.3	0\\
36.31	0\\
36.32	1.73472347597681e-18\\
36.33	0\\
36.34	0\\
36.35	0\\
36.36	0\\
36.37	0\\
36.38	1.73472347597681e-18\\
36.39	0\\
36.4	0\\
36.41	0\\
36.42	0\\
36.43	0\\
36.44	1.73472347597681e-18\\
36.45	0\\
36.46	0\\
36.47	0\\
36.48	0\\
36.49	0\\
36.5	1.73472347597681e-18\\
36.51	0\\
36.52	0\\
36.53	0\\
36.54	0\\
36.55	0\\
36.56	0\\
36.57	0\\
36.58	1.73472347597681e-18\\
36.59	0\\
36.6	1.73472347597681e-18\\
36.61	0\\
36.62	0\\
36.63	1.73472347597681e-18\\
36.64	0\\
36.65	0\\
36.66	0\\
36.67	0\\
36.68	0\\
36.69	0\\
36.7	0\\
36.71	0\\
36.72	0\\
36.73	0\\
36.74	0\\
36.75	0\\
36.76	0\\
36.77	0\\
36.78	0\\
36.79	0\\
36.8	0\\
36.81	1.73472347597681e-18\\
36.82	0\\
36.83	0\\
36.84	0\\
36.85	0\\
36.86	0\\
36.87	0\\
36.88	0\\
36.89	0\\
36.9	0\\
36.91	0\\
36.92	0\\
36.93	0\\
36.94	0\\
36.95	0\\
36.96	0\\
36.97	0\\
36.98	1.73472347597681e-18\\
36.99	0\\
37	0\\
37.01	0\\
37.02	1.73472347597681e-18\\
37.03	0\\
37.04	0\\
37.05	0\\
37.06	0\\
37.07	0\\
37.08	0\\
37.09	0\\
37.1	0\\
37.11	0\\
37.12	0\\
37.13	0\\
37.14	0\\
37.15	0\\
37.16	0\\
37.17	0\\
37.18	0\\
37.19	0\\
37.2	0\\
37.21	0\\
37.22	0\\
37.23	0\\
37.24	0\\
37.25	0\\
37.26	0\\
37.27	0\\
37.28	0\\
37.29	0\\
37.3	1.73472347597681e-18\\
37.31	0\\
37.32	0\\
37.33	0\\
37.34	0\\
37.35	0\\
37.36	0\\
37.37	0\\
37.38	0\\
37.39	0\\
37.4	0\\
37.41	0\\
37.42	0\\
37.43	0\\
37.44	0\\
37.45	0\\
37.46	0\\
37.47	0\\
37.48	0\\
37.49	1.73472347597681e-18\\
37.5	0\\
37.51	0\\
37.52	0\\
37.53	0\\
37.54	0\\
37.55	0\\
37.56	0\\
37.57	0\\
37.58	0\\
37.59	0\\
37.6	0\\
37.61	0\\
37.62	0\\
37.63	0\\
37.64	0\\
37.65	0\\
37.66	0\\
37.67	0\\
37.68	0\\
37.69	0\\
37.7	0\\
37.71	0\\
37.72	0\\
37.73	0\\
37.74	0\\
37.75	0\\
37.76	1.73472347597681e-18\\
37.77	0\\
37.78	0\\
37.79	0\\
37.8	0\\
37.81	0\\
37.82	0\\
37.83	0\\
37.84	0\\
37.85	0\\
37.86	0\\
37.87	0\\
37.88	0\\
37.89	0\\
37.9	0\\
37.91	0\\
37.92	0\\
37.93	0\\
37.94	0\\
37.95	0\\
37.96	0\\
37.97	0\\
37.98	0\\
37.99	0\\
38	1.73472347597681e-18\\
38.01	0\\
38.02	0\\
38.03	0\\
38.04	0\\
38.05	0\\
38.06	0\\
38.07	0\\
38.08	0\\
38.09	0\\
38.1	0\\
38.11	1.73472347597681e-18\\
38.12	1.73472347597681e-18\\
38.13	0\\
38.14	0\\
38.15	0\\
38.16	0\\
38.17	0\\
38.18	0\\
38.19	0\\
38.2	0\\
38.21	0\\
38.22	0\\
38.23	0\\
38.24	0\\
38.25	0\\
38.26	0\\
38.27	1.73472347597681e-18\\
38.28	0\\
38.29	0\\
38.3	0\\
38.31	1.73472347597681e-18\\
38.32	0\\
38.33	0\\
38.34	0\\
38.35	1.73472347597681e-18\\
38.36	0\\
38.37	0\\
38.38	0\\
38.39	0\\
38.4	0\\
38.41	0\\
38.42	0\\
38.43	0\\
38.44	0\\
38.45	0\\
38.46	0\\
38.47	0\\
38.48	1.73472347597681e-18\\
38.49	0\\
38.5	0\\
38.51	0\\
38.52	0\\
38.53	0\\
38.54	0\\
38.55	0\\
38.56	0\\
38.57	0\\
38.58	0\\
38.59	0\\
38.6	0\\
38.61	1.73472347597681e-18\\
38.62	0\\
38.63	1.73472347597681e-18\\
38.64	0\\
38.65	1.73472347597681e-18\\
38.66	0\\
38.67	0\\
38.68	0\\
38.69	0\\
38.7	0\\
38.71	0\\
38.72	0\\
38.73	0\\
38.74	1.73472347597681e-18\\
38.75	0\\
38.76	0\\
38.77	0\\
38.78	0\\
38.79	0\\
38.8	0\\
38.81	0\\
38.82	1.73472347597681e-18\\
38.83	0\\
38.84	0\\
38.85	0\\
38.86	0\\
38.87	1.73472347597681e-18\\
38.88	0\\
38.89	1.73472347597681e-18\\
38.9	0\\
38.91	0\\
38.92	0\\
38.93	0\\
38.94	0\\
38.95	0\\
38.96	0\\
38.97	0\\
38.98	0\\
38.99	0\\
39	0\\
39.01	0\\
39.02	0\\
39.03	0\\
39.04	0\\
39.05	0\\
39.06	0\\
39.07	0\\
39.08	0\\
39.09	1.73472347597681e-18\\
39.1	0\\
39.11	0\\
39.12	0\\
39.13	0\\
39.14	0\\
39.15	0\\
39.16	0\\
39.17	0\\
39.18	0\\
39.19	1.73472347597681e-18\\
39.2	0\\
39.21	0\\
39.22	0\\
39.23	0\\
39.24	0\\
39.25	0\\
39.26	0\\
39.27	0\\
39.28	0\\
39.29	0\\
39.3	0\\
39.31	0\\
39.32	0\\
39.33	0\\
39.34	1.73472347597681e-18\\
39.35	0\\
39.36	0\\
39.37	0\\
39.38	0\\
39.39	0\\
39.4	0\\
39.41	0\\
39.42	0\\
39.43	0\\
39.44	0\\
39.45	0\\
39.46	0\\
39.47	0\\
39.48	0\\
39.49	0\\
39.5	0\\
39.51	0\\
39.52	1.73472347597681e-18\\
39.53	0\\
39.54	0\\
39.55	0\\
39.56	0\\
39.57	0\\
39.58	0\\
39.59	0\\
39.6	0\\
39.61	0\\
39.62	0\\
39.63	0\\
39.64	0\\
39.65	0\\
39.66	0\\
39.67	1.73472347597681e-18\\
39.68	0\\
39.69	0\\
39.7	0\\
39.71	0\\
39.72	0\\
39.73	0\\
39.74	0\\
39.75	1.73472347597681e-18\\
39.76	0\\
39.77	0\\
39.78	0\\
39.79	0\\
39.8	0\\
39.81	0\\
39.82	1.73472347597681e-18\\
39.83	0\\
39.84	0\\
39.85	0\\
39.86	0\\
39.87	0\\
39.88	0\\
39.89	0\\
39.9	0\\
39.91	0\\
39.92	1.73472347597681e-18\\
39.93	0\\
39.94	1.73472347597681e-18\\
39.95	0\\
39.96	0\\
39.97	0\\
39.98	0\\
39.99	0\\
40	0\\
40.01	1.73472347597681e-18\\
};
\addplot [color=red,dashed,forget plot]
  table[row sep=crcr]{%
40.01	1.73472347597681e-18\\
40.02	0\\
40.03	1.73472347597681e-18\\
40.04	0\\
40.05	1.73472347597681e-18\\
40.06	0\\
40.07	1.73472347597681e-18\\
40.08	0\\
40.09	1.73472347597681e-18\\
40.1	1.73472347597681e-18\\
40.11	0\\
40.12	0\\
40.13	0\\
40.14	0\\
40.15	1.73472347597681e-18\\
40.16	0\\
40.17	0\\
40.18	0\\
40.19	0\\
40.2	0\\
40.21	0\\
40.22	1.73472347597681e-18\\
40.23	0\\
40.24	0\\
40.25	1.73472347597681e-18\\
40.26	0\\
40.27	0\\
40.28	0\\
40.29	0\\
40.3	0\\
40.31	1.73472347597681e-18\\
40.32	1.73472347597681e-18\\
40.33	0\\
40.34	0\\
40.35	1.73472347597681e-18\\
40.36	0\\
40.37	0\\
40.38	0\\
40.39	0\\
40.4	1.73472347597681e-18\\
40.41	0\\
40.42	0\\
40.43	1.73472347597681e-18\\
40.44	0\\
40.45	0\\
40.46	0\\
40.47	0\\
40.48	0\\
40.49	0\\
40.5	1.73472347597681e-18\\
40.51	0\\
40.52	0\\
40.53	0\\
40.54	0\\
40.55	0\\
40.56	0\\
40.57	0\\
40.58	1.73472347597681e-18\\
40.59	0\\
40.6	0\\
40.61	1.73472347597681e-18\\
40.62	0\\
40.63	0\\
40.64	0\\
40.65	0\\
40.66	0\\
40.67	0\\
40.68	0\\
40.69	0\\
40.7	0\\
40.71	0\\
40.72	0\\
40.73	0\\
40.74	0\\
40.75	0\\
40.76	0\\
40.77	0\\
40.78	0\\
40.79	0\\
40.8	0\\
40.81	0\\
40.82	0\\
40.83	0\\
40.84	0\\
40.85	0\\
40.86	0\\
40.87	1.73472347597681e-18\\
40.88	0\\
40.89	0\\
40.9	1.73472347597681e-18\\
40.91	0\\
40.92	0\\
40.93	0\\
40.94	0\\
40.95	1.73472347597681e-18\\
40.96	0\\
40.97	1.73472347597681e-18\\
40.98	0\\
40.99	0\\
41	0\\
41.01	0\\
41.02	0\\
41.03	0\\
41.04	0\\
41.05	0\\
41.06	0\\
41.07	0\\
41.08	0\\
41.09	0\\
41.1	0\\
41.11	0\\
41.12	0\\
41.13	0\\
41.14	0\\
41.15	1.73472347597681e-18\\
41.16	0\\
41.17	0\\
41.18	1.73472347597681e-18\\
41.19	0\\
41.2	0\\
41.21	0\\
41.22	0\\
41.23	0\\
41.24	0\\
41.25	0\\
41.26	0\\
41.27	1.73472347597681e-18\\
41.28	0\\
41.29	1.73472347597681e-18\\
41.3	0\\
41.31	0\\
41.32	0\\
41.33	0\\
41.34	0\\
41.35	0\\
41.36	0\\
41.37	0\\
41.38	1.73472347597681e-18\\
41.39	0\\
41.4	0\\
41.41	1.73472347597681e-18\\
41.42	0\\
41.43	0\\
41.44	0\\
41.45	0\\
41.46	0\\
41.47	0\\
41.48	0\\
41.49	0\\
41.5	0\\
41.51	0\\
41.52	0\\
41.53	0\\
41.54	0\\
41.55	0\\
41.56	0\\
41.57	0\\
41.58	0\\
41.59	0\\
41.6	0\\
41.61	1.73472347597681e-18\\
41.62	0\\
41.63	0\\
41.64	0\\
41.65	0\\
41.66	0\\
41.67	0\\
41.68	1.73472347597681e-18\\
41.69	0\\
41.7	0\\
41.71	0\\
41.72	0\\
41.73	0\\
41.74	0\\
41.75	0\\
41.76	0\\
41.77	0\\
41.78	0\\
41.79	0\\
41.8	0\\
41.81	1.73472347597681e-18\\
41.82	0\\
41.83	0\\
41.84	0\\
41.85	1.73472347597681e-18\\
41.86	0\\
41.87	0\\
41.88	0\\
41.89	0\\
41.9	0\\
41.91	0\\
41.92	1.73472347597681e-18\\
41.93	0\\
41.94	0\\
41.95	0\\
41.96	0\\
41.97	0\\
41.98	0\\
41.99	0\\
42	0\\
42.01	0\\
42.02	0\\
42.03	0\\
42.04	0\\
42.05	0\\
42.06	0\\
42.07	0\\
42.08	0\\
42.09	0\\
42.1	0\\
42.11	0\\
42.12	0\\
42.13	0\\
42.14	1.73472347597681e-18\\
42.15	0\\
42.16	1.73472347597681e-18\\
42.17	0\\
42.18	1.73472347597681e-18\\
42.19	0\\
42.2	0\\
42.21	0\\
42.22	0\\
42.23	0\\
42.24	1.73472347597681e-18\\
42.25	0\\
42.26	0\\
42.27	0\\
42.28	0\\
42.29	0\\
42.3	1.73472347597681e-18\\
42.31	0\\
42.32	0\\
42.33	0\\
42.34	0\\
42.35	0\\
42.36	1.73472347597681e-18\\
42.37	0\\
42.38	0\\
42.39	0\\
42.4	0\\
42.41	0\\
42.42	0\\
42.43	0\\
42.44	0\\
42.45	0\\
42.46	0\\
42.47	0\\
42.48	0\\
42.49	0\\
42.5	0\\
42.51	0\\
42.52	0\\
42.53	0\\
42.54	0\\
42.55	0\\
42.56	0\\
42.57	0\\
42.58	0\\
42.59	0\\
42.6	0\\
42.61	0\\
42.62	0\\
42.63	0\\
42.64	0\\
42.65	0\\
42.66	0\\
42.67	0\\
42.68	0\\
42.69	0\\
42.7	0\\
42.71	1.73472347597681e-18\\
42.72	0\\
42.73	0\\
42.74	1.73472347597681e-18\\
42.75	0\\
42.76	0\\
42.77	1.73472347597681e-18\\
42.78	0\\
42.79	0\\
42.8	0\\
42.81	0\\
42.82	0\\
42.83	0\\
42.84	0\\
42.85	0\\
42.86	0\\
42.87	0\\
42.88	0\\
42.89	1.73472347597681e-18\\
42.9	0\\
42.91	0\\
42.92	0\\
42.93	0\\
42.94	0\\
42.95	0\\
42.96	0\\
42.97	0\\
42.98	0\\
42.99	0\\
43	1.73472347597681e-18\\
43.01	0\\
43.02	0\\
43.03	0\\
43.04	0\\
43.05	0\\
43.06	0\\
43.07	0\\
43.08	0\\
43.09	1.73472347597681e-18\\
43.1	0\\
43.11	0\\
43.12	0\\
43.13	1.73472347597681e-18\\
43.14	0\\
43.15	0\\
43.16	0\\
43.17	0\\
43.18	0\\
43.19	0\\
43.2	0\\
43.21	0\\
43.22	0\\
43.23	0\\
43.24	0\\
43.25	0\\
43.26	0\\
43.27	0\\
43.28	0\\
43.29	0\\
43.3	0\\
43.31	0\\
43.32	0\\
43.33	1.73472347597681e-18\\
43.34	0\\
43.35	0\\
43.36	0\\
43.37	0\\
43.38	1.73472347597681e-18\\
43.39	0\\
43.4	0\\
43.41	0\\
43.42	0\\
43.43	0\\
43.44	0\\
43.45	0\\
43.46	0\\
43.47	0\\
43.48	0\\
43.49	0\\
43.5	0\\
43.51	0\\
43.52	1.73472347597681e-18\\
43.53	0\\
43.54	0\\
43.55	1.73472347597681e-18\\
43.56	0\\
43.57	0\\
43.58	0\\
43.59	0\\
43.6	0\\
43.61	0\\
43.62	0\\
43.63	0\\
43.64	0\\
43.65	0\\
43.66	0\\
43.67	0\\
43.68	0\\
43.69	0\\
43.7	0\\
43.71	0\\
43.72	1.73472347597681e-18\\
43.73	0\\
43.74	1.73472347597681e-18\\
43.75	0\\
43.76	0\\
43.77	0\\
43.78	0\\
43.79	0\\
43.8	0\\
43.81	1.73472347597681e-18\\
43.82	0\\
43.83	1.73472347597681e-18\\
43.84	1.73472347597681e-18\\
43.85	0\\
43.86	0\\
43.87	0\\
43.88	0\\
43.89	0\\
43.9	0\\
43.91	0\\
43.92	0\\
43.93	1.73472347597681e-18\\
43.94	0\\
43.95	0\\
43.96	1.73472347597681e-18\\
43.97	0\\
43.98	0\\
43.99	0\\
44	0\\
44.01	0\\
44.02	0\\
44.03	1.73472347597681e-18\\
44.04	0\\
44.05	0\\
44.06	0\\
44.07	0\\
44.08	0\\
44.09	0\\
44.1	0\\
44.11	0\\
44.12	0\\
44.13	0\\
44.14	0\\
44.15	0\\
44.16	0\\
44.17	0\\
44.18	0\\
44.19	0\\
44.2	0\\
44.21	0\\
44.22	0\\
44.23	0\\
44.24	0\\
44.25	0\\
44.26	1.73472347597681e-18\\
44.27	0\\
44.28	0\\
44.29	0\\
44.3	0\\
44.31	0\\
44.32	0\\
44.33	1.73472347597681e-18\\
44.34	0\\
44.35	1.73472347597681e-18\\
44.36	1.73472347597681e-18\\
44.37	0\\
44.38	0\\
44.39	0\\
44.4	0\\
44.41	0\\
44.42	0\\
44.43	0\\
44.44	0\\
44.45	0\\
44.46	1.73472347597681e-18\\
44.47	0\\
44.48	0\\
44.49	0\\
44.5	0\\
44.51	0\\
44.52	0\\
44.53	0\\
44.54	0\\
44.55	0\\
44.56	0\\
44.57	0\\
44.58	0\\
44.59	0\\
44.6	0\\
44.61	0\\
44.62	0\\
44.63	0\\
44.64	1.73472347597681e-18\\
44.65	0\\
44.66	0\\
44.67	0\\
44.68	0\\
44.69	1.73472347597681e-18\\
44.7	0\\
44.71	0\\
44.72	0\\
44.73	0\\
44.74	0\\
44.75	0\\
44.76	0\\
44.77	0\\
44.78	0\\
44.79	0\\
44.8	0\\
44.81	0\\
44.82	0\\
44.83	0\\
44.84	1.73472347597681e-18\\
44.85	1.73472347597681e-18\\
44.86	0\\
44.87	0\\
44.88	0\\
44.89	0\\
44.9	0\\
44.91	1.73472347597681e-18\\
44.92	0\\
44.93	0\\
44.94	0\\
44.95	0\\
44.96	0\\
44.97	0\\
44.98	0\\
44.99	0\\
45	0\\
45.01	0\\
45.02	0\\
45.03	0\\
45.04	0\\
45.05	0\\
45.06	0\\
45.07	0\\
45.08	0\\
45.09	0\\
45.1	0\\
45.11	0\\
45.12	0\\
45.13	1.73472347597681e-18\\
45.14	0\\
45.15	0\\
45.16	0\\
45.17	0\\
45.18	0\\
45.19	0\\
45.2	0\\
45.21	0\\
45.22	0\\
45.23	0\\
45.24	0\\
45.25	0\\
45.26	0\\
45.27	0\\
45.28	0\\
45.29	0\\
45.3	0\\
45.31	0\\
45.32	0\\
45.33	0\\
45.34	0\\
45.35	0\\
45.36	0\\
45.37	0\\
45.38	0\\
45.39	0\\
45.4	1.73472347597681e-18\\
45.41	0\\
45.42	0\\
45.43	0\\
45.44	0\\
45.45	0\\
45.46	0\\
45.47	0\\
45.48	0\\
45.49	0\\
45.5	0\\
45.51	0\\
45.52	0\\
45.53	0\\
45.54	0\\
45.55	0\\
45.56	0\\
45.57	0\\
45.58	0\\
45.59	0\\
45.6	1.73472347597681e-18\\
45.61	0\\
45.62	0\\
45.63	1.73472347597681e-18\\
45.64	0\\
45.65	0\\
45.66	0\\
45.67	0\\
45.68	0\\
45.69	0\\
45.7	0\\
45.71	1.73472347597681e-18\\
45.72	0\\
45.73	0\\
45.74	0\\
45.75	0\\
45.76	0\\
45.77	0\\
45.78	0\\
45.79	0\\
45.8	0\\
45.81	0\\
45.82	0\\
45.83	1.73472347597681e-18\\
45.84	0\\
45.85	0\\
45.86	0\\
45.87	0\\
45.88	0\\
45.89	0\\
45.9	0\\
45.91	0\\
45.92	0\\
45.93	1.73472347597681e-18\\
45.94	0\\
45.95	0\\
45.96	0\\
45.97	0\\
45.98	0\\
45.99	0\\
46	0\\
46.01	0\\
46.02	0\\
46.03	0\\
46.04	0\\
46.05	0\\
46.06	0\\
46.07	0\\
46.08	1.73472347597681e-18\\
46.09	0\\
46.1	0\\
46.11	0\\
46.12	0\\
46.13	0\\
46.14	0\\
46.15	0\\
46.16	0\\
46.17	0\\
46.18	0\\
46.19	0\\
46.2	0\\
46.21	0\\
46.22	0\\
46.23	0\\
46.24	0\\
46.25	1.73472347597681e-18\\
46.26	0\\
46.27	0\\
46.28	0\\
46.29	1.73472347597681e-18\\
46.3	1.73472347597681e-18\\
46.31	0\\
46.32	1.73472347597681e-18\\
46.33	0\\
46.34	0\\
46.35	0\\
46.36	0\\
46.37	0\\
46.38	1.73472347597681e-18\\
46.39	0\\
46.4	0\\
46.41	0\\
46.42	0\\
46.43	0\\
46.44	0\\
46.45	0\\
46.46	0\\
46.47	0\\
46.48	0\\
46.49	0\\
46.5	0\\
46.51	0\\
46.52	0\\
46.53	0\\
46.54	0\\
46.55	0\\
46.56	0\\
46.57	0\\
46.58	0\\
46.59	1.73472347597681e-18\\
46.6	1.73472347597681e-18\\
46.61	0\\
46.62	0\\
46.63	0\\
46.64	0\\
46.65	0\\
46.66	0\\
46.67	0\\
46.68	1.73472347597681e-18\\
46.69	0\\
46.7	0\\
46.71	0\\
46.72	0\\
46.73	0\\
46.74	0\\
46.75	0\\
46.76	0\\
46.77	0\\
46.78	0\\
46.79	0\\
46.8	0\\
46.81	0\\
46.82	0\\
46.83	0\\
46.84	0\\
46.85	0\\
46.86	0\\
46.87	1.73472347597681e-18\\
46.88	0\\
46.89	1.73472347597681e-18\\
46.9	0\\
46.91	0\\
46.92	0\\
46.93	0\\
46.94	0\\
46.95	0\\
46.96	0\\
46.97	0\\
46.98	0\\
46.99	0\\
47	0\\
47.01	0\\
47.02	0\\
47.03	0\\
47.04	0\\
47.05	1.73472347597681e-18\\
47.06	0\\
47.07	0\\
47.08	0\\
47.09	0\\
47.1	1.73472347597681e-18\\
47.11	0\\
47.12	0\\
47.13	0\\
47.14	1.73472347597681e-18\\
47.15	0\\
47.16	1.73472347597681e-18\\
47.17	0\\
47.18	0\\
47.19	0\\
47.2	0\\
47.21	0\\
47.22	0\\
47.23	0\\
47.24	0\\
47.25	0\\
47.26	0\\
47.27	0\\
47.28	0\\
47.29	0\\
47.3	0\\
47.31	0\\
47.32	0\\
47.33	0\\
47.34	0\\
47.35	0\\
47.36	1.73472347597681e-18\\
47.37	0\\
47.38	1.73472347597681e-18\\
47.39	0\\
47.4	0\\
47.41	0\\
47.42	0\\
47.43	0\\
47.44	0\\
47.45	0\\
47.46	1.73472347597681e-18\\
47.47	0\\
47.48	0\\
47.49	0\\
47.5	0\\
47.51	1.73472347597681e-18\\
47.52	1.73472347597681e-18\\
47.53	0\\
47.54	0\\
47.55	0\\
47.56	0\\
47.57	1.73472347597681e-18\\
47.58	1.73472347597681e-18\\
47.59	0\\
47.6	0\\
47.61	0\\
47.62	0\\
47.63	0\\
47.64	0\\
47.65	0\\
47.66	0\\
47.67	0\\
47.68	0\\
47.69	0\\
47.7	0\\
47.71	0\\
47.72	0\\
47.73	0\\
47.74	0\\
47.75	0\\
47.76	1.73472347597681e-18\\
47.77	0\\
47.78	0\\
47.79	0\\
47.8	0\\
47.81	0\\
47.82	0\\
47.83	0\\
47.84	1.73472347597681e-18\\
47.85	0\\
47.86	0\\
47.87	0\\
47.88	0\\
47.89	0\\
47.9	0\\
47.91	0\\
47.92	0\\
47.93	0\\
47.94	0\\
47.95	0\\
47.96	0\\
47.97	1.73472347597681e-18\\
47.98	0\\
47.99	0\\
48	1.73472347597681e-18\\
48.01	0\\
48.02	0\\
48.03	0\\
48.04	0\\
48.05	0\\
48.06	0\\
48.07	1.73472347597681e-18\\
48.08	1.73472347597681e-18\\
48.09	0\\
48.1	0\\
48.11	1.73472347597681e-18\\
48.12	0\\
48.13	0\\
48.14	0\\
48.15	0\\
48.16	0\\
48.17	0\\
48.18	0\\
48.19	0\\
48.2	0\\
48.21	0\\
48.22	0\\
48.23	0\\
48.24	0\\
48.25	0\\
48.26	0\\
48.27	0\\
48.28	0\\
48.29	0\\
48.3	0\\
48.31	0\\
48.32	0\\
48.33	0\\
48.34	0\\
48.35	0\\
48.36	0\\
48.37	0\\
48.38	0\\
48.39	0\\
48.4	0\\
48.41	0\\
48.42	0\\
48.43	0\\
48.44	0\\
48.45	0\\
48.46	0\\
48.47	0\\
48.48	0\\
48.49	0\\
48.5	1.73472347597681e-18\\
48.51	0\\
48.52	0\\
48.53	0\\
48.54	0\\
48.55	0\\
48.56	0\\
48.57	0\\
48.58	0\\
48.59	0\\
48.6	0\\
48.61	0\\
48.62	0\\
48.63	0\\
48.64	0\\
48.65	0\\
48.66	0\\
48.67	0\\
48.68	0\\
48.69	0\\
48.7	0\\
48.71	0\\
48.72	0\\
48.73	1.73472347597681e-18\\
48.74	0\\
48.75	0\\
48.76	0\\
48.77	0\\
48.78	0\\
48.79	0\\
48.8	1.73472347597681e-18\\
48.81	0\\
48.82	0\\
48.83	0\\
48.84	0\\
48.85	0\\
48.86	0\\
48.87	1.73472347597681e-18\\
48.88	1.73472347597681e-18\\
48.89	0\\
48.9	0\\
48.91	0\\
48.92	0\\
48.93	0\\
48.94	0\\
48.95	1.73472347597681e-18\\
48.96	1.73472347597681e-18\\
48.97	0\\
48.98	0\\
48.99	0\\
49	0\\
49.01	0\\
49.02	0\\
49.03	0\\
49.04	1.73472347597681e-18\\
49.05	1.73472347597681e-18\\
49.06	0\\
49.07	0\\
49.08	0\\
49.09	0\\
49.1	0\\
49.11	0\\
49.12	0\\
49.13	1.73472347597681e-18\\
49.14	0\\
49.15	0\\
49.16	0\\
49.17	0\\
49.18	0\\
49.19	0\\
49.2	0\\
49.21	0\\
49.22	0\\
49.23	0\\
49.24	0\\
49.25	0\\
49.26	0\\
49.27	0\\
49.28	0\\
49.29	0\\
49.3	0\\
49.31	0\\
49.32	1.73472347597681e-18\\
49.33	0\\
49.34	0\\
49.35	0\\
49.36	0\\
49.37	0\\
49.38	0\\
49.39	0\\
49.4	0\\
49.41	0\\
49.42	0\\
49.43	0\\
49.44	0\\
49.45	0\\
49.46	0\\
49.47	0\\
49.48	0\\
49.49	0\\
49.5	0\\
49.51	0\\
49.52	0\\
49.53	0\\
49.54	0\\
49.55	0\\
49.56	0\\
49.57	0\\
49.58	0\\
49.59	0\\
49.6	0\\
49.61	0\\
49.62	0\\
49.63	0\\
49.64	0\\
49.65	0\\
49.66	0\\
49.67	1.73472347597681e-18\\
49.68	1.73472347597681e-18\\
49.69	0\\
49.7	0\\
49.71	0\\
49.72	0\\
49.73	0\\
49.74	0\\
49.75	0\\
49.76	0\\
49.77	0\\
49.78	0\\
49.79	0\\
49.8	0\\
49.81	0\\
49.82	0\\
49.83	0\\
49.84	0\\
49.85	0\\
49.86	0\\
49.87	0\\
49.88	0\\
49.89	0\\
49.9	0\\
49.91	0\\
49.92	0\\
49.93	0\\
49.94	1.73472347597681e-18\\
49.95	0\\
49.96	0\\
49.97	0\\
49.98	0\\
49.99	0\\
50	0\\
50.01	0\\
50.02	0\\
50.03	0\\
50.04	0\\
50.05	0\\
50.06	0\\
50.07	0\\
50.08	0\\
50.09	0\\
50.1	0\\
50.11	0\\
50.12	0\\
50.13	0\\
50.14	1.73472347597681e-18\\
50.15	0\\
50.16	0\\
50.17	0\\
50.18	0\\
50.19	0\\
50.2	0\\
50.21	1.73472347597681e-18\\
50.22	0\\
50.23	0\\
50.24	0\\
50.25	0\\
50.26	0\\
50.27	0\\
50.28	0\\
50.29	0\\
50.3	0\\
50.31	1.73472347597681e-18\\
50.32	0\\
50.33	0\\
50.34	0\\
50.35	0\\
50.36	0\\
50.37	1.73472347597681e-18\\
50.38	0\\
50.39	0\\
50.4	0\\
50.41	0\\
50.42	0\\
50.43	0\\
50.44	0\\
50.45	0\\
50.46	0\\
50.47	1.73472347597681e-18\\
50.48	0\\
50.49	0\\
50.5	0\\
50.51	1.73472347597681e-18\\
50.52	0\\
50.53	0\\
50.54	0\\
50.55	0\\
50.56	1.73472347597681e-18\\
50.57	0\\
50.58	0\\
50.59	0\\
50.6	0\\
50.61	0\\
50.62	0\\
50.63	0\\
50.64	0\\
50.65	0\\
50.66	0\\
50.67	0\\
50.68	0\\
50.69	0\\
50.7	0\\
50.71	0\\
50.72	0\\
50.73	0\\
50.74	0\\
50.75	0\\
50.76	0\\
50.77	0\\
50.78	1.73472347597681e-18\\
50.79	0\\
50.8	0\\
50.81	0\\
50.82	0\\
50.83	0\\
50.84	0\\
50.85	0\\
50.86	0\\
50.87	1.73472347597681e-18\\
50.88	1.73472347597681e-18\\
50.89	0\\
50.9	0\\
50.91	0\\
50.92	0\\
50.93	1.73472347597681e-18\\
50.94	0\\
50.95	0\\
50.96	1.73472347597681e-18\\
50.97	0\\
50.98	0\\
50.99	0\\
51	0\\
51.01	1.73472347597681e-18\\
51.02	0\\
51.03	0\\
51.04	0\\
51.05	0\\
51.06	0\\
51.07	0\\
51.08	0\\
51.09	0\\
51.1	0\\
51.11	0\\
51.12	0\\
51.13	0\\
51.14	0\\
51.15	0\\
51.16	0\\
51.17	0\\
51.18	0\\
51.19	1.73472347597681e-18\\
51.2	0\\
51.21	0\\
51.22	0\\
51.23	1.73472347597681e-18\\
51.24	0\\
51.25	0\\
51.26	0\\
51.27	0\\
51.28	0\\
51.29	0\\
51.3	0\\
51.31	0\\
51.32	1.73472347597681e-18\\
51.33	0\\
51.34	0\\
51.35	0\\
51.36	0\\
51.37	1.73472347597681e-18\\
51.38	0\\
51.39	0\\
51.4	0\\
51.41	0\\
51.42	1.73472347597681e-18\\
51.43	0\\
51.44	0\\
51.45	0\\
51.46	1.73472347597681e-18\\
51.47	0\\
51.48	0\\
51.49	1.73472347597681e-18\\
51.5	0\\
51.51	0\\
51.52	0\\
51.53	0\\
51.54	0\\
51.55	0\\
51.56	0\\
51.57	0\\
51.58	1.73472347597681e-18\\
51.59	0\\
51.6	0\\
51.61	0\\
51.62	0\\
51.63	0\\
51.64	0\\
51.65	0\\
51.66	0\\
51.67	0\\
51.68	1.73472347597681e-18\\
51.69	0\\
51.7	1.73472347597681e-18\\
51.71	0\\
51.72	0\\
51.73	0\\
51.74	0\\
51.75	0\\
51.76	0\\
51.77	0\\
51.78	0\\
51.79	0\\
51.8	0\\
51.81	0\\
51.82	0\\
51.83	0\\
51.84	0\\
51.85	0\\
51.86	0\\
51.87	0\\
51.88	0\\
51.89	0\\
51.9	0\\
51.91	0\\
51.92	0\\
51.93	1.73472347597681e-18\\
51.94	0\\
51.95	0\\
51.96	0\\
51.97	0\\
51.98	0\\
51.99	0\\
52	0\\
52.01	0\\
52.02	0\\
52.03	0\\
52.04	0\\
52.05	0\\
52.06	0\\
52.07	0\\
52.08	0\\
52.09	0\\
52.1	0\\
52.11	0\\
52.12	0\\
52.13	0\\
52.14	0\\
52.15	0\\
52.16	1.73472347597681e-18\\
52.17	1.73472347597681e-18\\
52.18	0\\
52.19	0\\
52.2	0\\
52.21	0\\
52.22	1.73472347597681e-18\\
52.23	0\\
52.24	0\\
52.25	0\\
52.26	0\\
52.27	0\\
52.28	0\\
52.29	0\\
52.3	0\\
52.31	0\\
52.32	0\\
52.33	0\\
52.34	0\\
52.35	0\\
52.36	0\\
52.37	1.73472347597681e-18\\
52.38	0\\
52.39	1.73472347597681e-18\\
52.4	1.73472347597681e-18\\
52.41	0\\
52.42	0\\
52.43	0\\
52.44	0\\
52.45	0\\
52.46	0\\
52.47	0\\
52.48	0\\
52.49	0\\
52.5	0\\
52.51	0\\
52.52	0\\
52.53	0\\
52.54	0\\
52.55	0\\
52.56	0\\
52.57	0\\
52.58	0\\
52.59	0\\
52.6	0\\
52.61	1.73472347597681e-18\\
52.62	1.73472347597681e-18\\
52.63	0\\
52.64	1.73472347597681e-18\\
52.65	0\\
52.66	0\\
52.67	0\\
52.68	0\\
52.69	0\\
52.7	0\\
52.71	0\\
52.72	1.73472347597681e-18\\
52.73	0\\
52.74	0\\
52.75	0\\
52.76	0\\
52.77	0\\
52.78	0\\
52.79	0\\
52.8	0\\
52.81	0\\
52.82	0\\
52.83	0\\
52.84	0\\
52.85	1.73472347597681e-18\\
52.86	0\\
52.87	1.73472347597681e-18\\
52.88	0\\
52.89	0\\
52.9	1.73472347597681e-18\\
52.91	0\\
52.92	0\\
52.93	0\\
52.94	0\\
52.95	1.73472347597681e-18\\
52.96	0\\
52.97	0\\
52.98	0\\
52.99	0\\
53	0\\
53.01	1.73472347597681e-18\\
53.02	0\\
53.03	0\\
53.04	0\\
53.05	0\\
53.06	0\\
53.07	0\\
53.08	0\\
53.09	0\\
53.1	1.73472347597681e-18\\
53.11	1.73472347597681e-18\\
53.12	0\\
53.13	0\\
53.14	0\\
53.15	0\\
53.16	0\\
53.17	0\\
53.18	0\\
53.19	0\\
53.2	1.73472347597681e-18\\
53.21	0\\
53.22	0\\
53.23	0\\
53.24	0\\
53.25	0\\
53.26	0\\
53.27	1.73472347597681e-18\\
53.28	0\\
53.29	0\\
53.3	0\\
53.31	0\\
53.32	0\\
53.33	0\\
53.34	0\\
53.35	0\\
53.36	0\\
53.37	0\\
53.38	0\\
53.39	0\\
53.4	0\\
53.41	0\\
53.42	0\\
53.43	0\\
53.44	0\\
53.45	1.73472347597681e-18\\
53.46	0\\
53.47	0\\
53.48	0\\
53.49	0\\
53.5	0\\
53.51	0\\
53.52	1.73472347597681e-18\\
53.53	1.73472347597681e-18\\
53.54	0\\
53.55	0\\
53.56	0\\
53.57	0\\
53.58	1.73472347597681e-18\\
53.59	0\\
53.6	0\\
53.61	0\\
53.62	0\\
53.63	0\\
53.64	0\\
53.65	0\\
53.66	0\\
53.67	0\\
53.68	0\\
53.69	0\\
53.7	0\\
53.71	0\\
53.72	0\\
53.73	0\\
53.74	0\\
53.75	0\\
53.76	0\\
53.77	0\\
53.78	0\\
53.79	1.73472347597681e-18\\
53.8	0\\
53.81	0\\
53.82	0\\
53.83	0\\
53.84	0\\
53.85	1.73472347597681e-18\\
53.86	0\\
53.87	0\\
53.88	0\\
53.89	0\\
53.9	0\\
53.91	0\\
53.92	0\\
53.93	0\\
53.94	0\\
53.95	0\\
53.96	0\\
53.97	0\\
53.98	0\\
53.99	0\\
54	0\\
54.01	0\\
54.02	0\\
54.03	0\\
54.04	0\\
54.05	0\\
54.06	0\\
54.07	1.73472347597681e-18\\
54.08	0\\
54.09	1.73472347597681e-18\\
54.1	0\\
54.11	1.73472347597681e-18\\
54.12	0\\
54.13	1.73472347597681e-18\\
54.14	0\\
54.15	1.73472347597681e-18\\
54.16	0\\
54.17	0\\
54.18	0\\
54.19	0\\
54.2	1.73472347597681e-18\\
54.21	1.73472347597681e-18\\
54.22	0\\
54.23	0\\
54.24	0\\
54.25	0\\
54.26	0\\
54.27	0\\
54.28	0\\
54.29	0\\
54.3	0\\
54.31	0\\
54.32	1.73472347597681e-18\\
54.33	0\\
54.34	0\\
54.35	0\\
54.36	0\\
54.37	0\\
54.38	0\\
54.39	0\\
54.4	0\\
54.41	0\\
54.42	0\\
54.43	0\\
54.44	0\\
54.45	1.73472347597681e-18\\
54.46	0\\
54.47	0\\
54.48	0\\
54.49	0\\
54.5	0\\
54.51	0\\
54.52	0\\
54.53	0\\
54.54	0\\
54.55	0\\
54.56	0\\
54.57	0\\
54.58	0\\
54.59	0\\
54.6	0\\
54.61	0\\
54.62	0\\
54.63	0\\
54.64	0\\
54.65	1.73472347597681e-18\\
54.66	0\\
54.67	0\\
54.68	0\\
54.69	1.73472347597681e-18\\
54.7	0\\
54.71	0\\
54.72	0\\
54.73	0\\
54.74	0\\
54.75	0\\
54.76	0\\
54.77	0\\
54.78	0\\
54.79	0\\
54.8	0\\
54.81	0\\
54.82	0\\
54.83	0\\
54.84	0\\
54.85	0\\
54.86	0\\
54.87	0\\
54.88	0\\
54.89	0\\
54.9	1.73472347597681e-18\\
54.91	0\\
54.92	0\\
54.93	0\\
54.94	1.73472347597681e-18\\
54.95	0\\
54.96	0\\
54.97	0\\
54.98	1.73472347597681e-18\\
54.99	0\\
55	0\\
55.01	0\\
55.02	0\\
55.03	0\\
55.04	0\\
55.05	0\\
55.06	0\\
55.07	1.73472347597681e-18\\
55.08	1.73472347597681e-18\\
55.09	0\\
55.1	0\\
55.11	0\\
55.12	0\\
55.13	0\\
55.14	1.73472347597681e-18\\
55.15	0\\
55.16	1.73472347597681e-18\\
55.17	1.73472347597681e-18\\
55.18	0\\
55.19	1.73472347597681e-18\\
55.2	0\\
55.21	0\\
55.22	0\\
55.23	0\\
55.24	0\\
55.25	0\\
55.26	0\\
55.27	0\\
55.28	0\\
55.29	0\\
55.3	0\\
55.31	0\\
55.32	0\\
55.33	0\\
55.34	0\\
55.35	0\\
55.36	1.73472347597681e-18\\
55.37	0\\
55.38	0\\
55.39	0\\
55.4	0\\
55.41	0\\
55.42	0\\
55.43	0\\
55.44	0\\
55.45	0\\
55.46	0\\
55.47	1.73472347597681e-18\\
55.48	0\\
55.49	1.73472347597681e-18\\
55.5	0\\
55.51	0\\
55.52	0\\
55.53	0\\
55.54	0\\
55.55	0\\
55.56	1.73472347597681e-18\\
55.57	0\\
55.58	0\\
55.59	0\\
55.6	0\\
55.61	0\\
55.62	0\\
55.63	0\\
55.64	0\\
55.65	0\\
55.66	0\\
55.67	0\\
55.68	0\\
55.69	0\\
55.7	0\\
55.71	0\\
55.72	0\\
55.73	1.73472347597681e-18\\
55.74	0\\
55.75	0\\
55.76	0\\
55.77	1.73472347597681e-18\\
55.78	0\\
55.79	0\\
55.8	0\\
55.81	0\\
55.82	0\\
55.83	0\\
55.84	0\\
55.85	0\\
55.86	0\\
55.87	0\\
55.88	0\\
55.89	0\\
55.9	0\\
55.91	0\\
55.92	0\\
55.93	0\\
55.94	0\\
55.95	1.73472347597681e-18\\
55.96	0\\
55.97	0\\
55.98	0\\
55.99	0\\
56	0\\
56.01	1.73472347597681e-18\\
56.02	0\\
56.03	0\\
56.04	1.73472347597681e-18\\
56.05	1.73472347597681e-18\\
56.06	0\\
56.07	1.73472347597681e-18\\
56.08	0\\
56.09	0\\
56.1	0\\
56.11	0\\
56.12	0\\
56.13	0\\
56.14	0\\
56.15	0\\
56.16	0\\
56.17	0\\
56.18	0\\
56.19	0\\
56.2	0\\
56.21	0\\
56.22	0\\
56.23	0\\
56.24	1.73472347597681e-18\\
56.25	0\\
56.26	0\\
56.27	0\\
56.28	0\\
56.29	1.73472347597681e-18\\
56.3	0\\
56.31	0\\
56.32	0\\
56.33	0\\
56.34	0\\
56.35	0\\
56.36	0\\
56.37	0\\
56.38	0\\
56.39	0\\
56.4	0\\
56.41	0\\
56.42	0\\
56.43	0\\
56.44	0\\
56.45	0\\
56.46	0\\
56.47	0\\
56.48	0\\
56.49	0\\
56.5	0\\
56.51	0\\
56.52	0\\
56.53	0\\
56.54	0\\
56.55	0\\
56.56	0\\
56.57	0\\
56.58	0\\
56.59	0\\
56.6	0\\
56.61	0\\
56.62	0\\
56.63	0\\
56.64	0\\
56.65	0\\
56.66	0\\
56.67	0\\
56.68	0\\
56.69	0\\
56.7	1.73472347597681e-18\\
56.71	1.73472347597681e-18\\
56.72	0\\
56.73	0\\
56.74	1.73472347597681e-18\\
56.75	1.73472347597681e-18\\
56.76	0\\
56.77	0\\
56.78	0\\
56.79	1.73472347597681e-18\\
56.8	0\\
56.81	0\\
56.82	1.73472347597681e-18\\
56.83	0\\
56.84	0\\
56.85	0\\
56.86	0\\
56.87	0\\
56.88	0\\
56.89	0\\
56.9	0\\
56.91	0\\
56.92	0\\
56.93	0\\
56.94	0\\
56.95	0\\
56.96	0\\
56.97	0\\
56.98	0\\
56.99	0\\
57	1.73472347597681e-18\\
57.01	0\\
57.02	0\\
57.03	0\\
57.04	0\\
57.05	1.73472347597681e-18\\
57.06	1.73472347597681e-18\\
57.07	0\\
57.08	0\\
57.09	0\\
57.1	0\\
57.11	0\\
57.12	0\\
57.13	0\\
57.14	0\\
57.15	0\\
57.16	0\\
57.17	0\\
57.18	0\\
57.19	0\\
57.2	0\\
57.21	0\\
57.22	0\\
57.23	0\\
57.24	0\\
57.25	0\\
57.26	0\\
57.27	1.73472347597681e-18\\
57.28	0\\
57.29	0\\
57.3	0\\
57.31	0\\
57.32	0\\
57.33	0\\
57.34	0\\
57.35	0\\
57.36	0\\
57.37	0\\
57.38	1.73472347597681e-18\\
57.39	0\\
57.4	0\\
57.41	0\\
57.42	0\\
57.43	0\\
57.44	0\\
57.45	0\\
57.46	0\\
57.47	0\\
57.48	0\\
57.49	0\\
57.5	0\\
57.51	1.73472347597681e-18\\
57.52	0\\
57.53	0\\
57.54	0\\
57.55	0\\
57.56	0\\
57.57	0\\
57.58	0\\
57.59	0\\
57.6	0\\
57.61	0\\
57.62	0\\
57.63	0\\
57.64	0\\
57.65	0\\
57.66	0\\
57.67	0\\
57.68	1.73472347597681e-18\\
57.69	1.73472347597681e-18\\
57.7	0\\
57.71	0\\
57.72	0\\
57.73	0\\
57.74	0\\
57.75	0\\
57.76	0\\
57.77	1.73472347597681e-18\\
57.78	0\\
57.79	0\\
57.8	0\\
57.81	0\\
57.82	0\\
57.83	0\\
57.84	1.73472347597681e-18\\
57.85	1.73472347597681e-18\\
57.86	0\\
57.87	1.73472347597681e-18\\
57.88	0\\
57.89	0\\
57.9	0\\
57.91	0\\
57.92	0\\
57.93	0\\
57.94	0\\
57.95	0\\
57.96	0\\
57.97	0\\
57.98	0\\
57.99	0\\
58	1.73472347597681e-18\\
58.01	0\\
58.02	0\\
58.03	0\\
58.04	0\\
58.05	0\\
58.06	0\\
58.07	0\\
58.08	0\\
58.09	0\\
58.1	0\\
58.11	1.73472347597681e-18\\
58.12	0\\
58.13	1.73472347597681e-18\\
58.14	0\\
58.15	0\\
58.16	0\\
58.17	0\\
58.18	0\\
58.19	0\\
58.2	0\\
58.21	0\\
58.22	1.73472347597681e-18\\
58.23	0\\
58.24	0\\
58.25	1.73472347597681e-18\\
58.26	0\\
58.27	0\\
58.28	0\\
58.29	1.73472347597681e-18\\
58.3	0\\
58.31	1.73472347597681e-18\\
58.32	0\\
58.33	0\\
58.34	1.73472347597681e-18\\
58.35	1.73472347597681e-18\\
58.36	0\\
58.37	0\\
58.38	0\\
58.39	0\\
58.4	0\\
58.41	1.73472347597681e-18\\
58.42	1.73472347597681e-18\\
58.43	0\\
58.44	0\\
58.45	0\\
58.46	0\\
58.47	0\\
58.48	0\\
58.49	0\\
58.5	0\\
58.51	0\\
58.52	1.73472347597681e-18\\
58.53	0\\
58.54	0\\
58.55	0\\
58.56	0\\
58.57	0\\
58.58	0\\
58.59	0\\
58.6	0\\
58.61	0\\
58.62	0\\
58.63	0\\
58.64	0\\
58.65	0\\
58.66	0\\
58.67	1.73472347597681e-18\\
58.68	0\\
58.69	0\\
58.7	0\\
58.71	0\\
58.72	0\\
58.73	0\\
58.74	1.73472347597681e-18\\
58.75	0\\
58.76	0\\
58.77	0\\
58.78	0\\
58.79	1.73472347597681e-18\\
58.8	0\\
58.81	0\\
58.82	0\\
58.83	0\\
58.84	0\\
58.85	0\\
58.86	0\\
58.87	0\\
58.88	1.73472347597681e-18\\
58.89	0\\
58.9	0\\
58.91	0\\
58.92	0\\
58.93	0\\
58.94	0\\
58.95	0\\
58.96	0\\
58.97	0\\
58.98	1.73472347597681e-18\\
58.99	0\\
59	0\\
59.01	0\\
59.02	0\\
59.03	0\\
59.04	0\\
59.05	0\\
59.06	0\\
59.07	0\\
59.08	0\\
59.09	0\\
59.1	0\\
59.11	0\\
59.12	0\\
59.13	0\\
59.14	0\\
59.15	0\\
59.16	0\\
59.17	1.73472347597681e-18\\
59.18	0\\
59.19	0\\
59.2	1.73472347597681e-18\\
59.21	0\\
59.22	0\\
59.23	0\\
59.24	0\\
59.25	1.73472347597681e-18\\
59.26	0\\
59.27	0\\
59.28	0\\
59.29	0\\
59.3	1.73472347597681e-18\\
59.31	0\\
59.32	1.73472347597681e-18\\
59.33	0\\
59.34	0\\
59.35	0\\
59.36	0\\
59.37	0\\
59.38	0\\
59.39	0\\
59.4	0\\
59.41	1.73472347597681e-18\\
59.42	1.73472347597681e-18\\
59.43	0\\
59.44	0\\
59.45	0\\
59.46	0\\
59.47	0\\
59.48	0\\
59.49	0\\
59.5	1.73472347597681e-18\\
59.51	1.73472347597681e-18\\
59.52	0\\
59.53	0\\
59.54	0\\
59.55	0\\
59.56	0\\
59.57	0\\
59.58	0\\
59.59	1.73472347597681e-18\\
59.6	0\\
59.61	0\\
59.62	1.73472347597681e-18\\
59.63	0\\
59.64	0\\
59.65	0\\
59.66	1.73472347597681e-18\\
59.67	0\\
59.68	0\\
59.69	0\\
59.7	0\\
59.71	0\\
59.72	0\\
59.73	0\\
59.74	0\\
59.75	0\\
59.76	0\\
59.77	0\\
59.78	0\\
59.79	0\\
59.8	1.73472347597681e-18\\
59.81	0\\
59.82	0\\
59.83	0\\
59.84	0\\
59.85	0\\
59.86	1.73472347597681e-18\\
59.87	0\\
59.88	1.73472347597681e-18\\
59.89	0\\
59.9	0\\
59.91	0\\
59.92	0\\
59.93	1.73472347597681e-18\\
59.94	0\\
59.95	0\\
59.96	0\\
59.97	0\\
59.98	0\\
59.99	0\\
60	0\\
60.01	1.73472347597681e-18\\
60.02	0\\
60.03	0\\
60.04	0\\
60.05	0\\
60.06	0\\
60.07	0\\
60.08	0\\
60.09	0\\
60.1	0\\
60.11	0\\
60.12	0\\
60.13	1.73472347597681e-18\\
60.14	0\\
60.15	0\\
60.16	0\\
60.17	0\\
60.18	0\\
60.19	0\\
60.2	0\\
60.21	0\\
60.22	0\\
60.23	0\\
60.24	0\\
60.25	0\\
60.26	0\\
60.27	0\\
60.28	0\\
60.29	0\\
60.3	0\\
60.31	0\\
60.32	0\\
60.33	0\\
60.34	0\\
60.35	0\\
60.36	0\\
60.37	0\\
60.38	0\\
60.39	1.73472347597681e-18\\
60.4	0\\
60.41	0\\
60.42	0\\
60.43	0\\
60.44	0\\
60.45	0\\
60.46	0\\
60.47	0\\
60.48	0\\
60.49	1.73472347597681e-18\\
60.5	0\\
60.51	0\\
60.52	0\\
60.53	0\\
60.54	0\\
60.55	0\\
60.56	0\\
60.57	0\\
60.58	0\\
60.59	0\\
60.6	0\\
60.61	0\\
60.62	0\\
60.63	0\\
60.64	0\\
60.65	0\\
60.66	1.73472347597681e-18\\
60.67	0\\
60.68	0\\
60.69	0\\
60.7	0\\
60.71	0\\
60.72	0\\
60.73	0\\
60.74	0\\
60.75	0\\
60.76	0\\
60.77	1.73472347597681e-18\\
60.78	0\\
60.79	0\\
60.8	0\\
60.81	0\\
60.82	0\\
60.83	0\\
60.84	0\\
60.85	0\\
60.86	0\\
60.87	0\\
60.88	0\\
60.89	0\\
60.9	0\\
60.91	0\\
60.92	0\\
60.93	0\\
60.94	0\\
60.95	0\\
60.96	0\\
60.97	0\\
60.98	0\\
60.99	0\\
61	0\\
61.01	1.73472347597681e-18\\
61.02	0\\
61.03	0\\
61.04	0\\
61.05	0\\
61.06	1.73472347597681e-18\\
61.07	0\\
61.08	0\\
61.09	0\\
61.1	0\\
61.11	0\\
61.12	0\\
61.13	0\\
61.14	1.73472347597681e-18\\
61.15	0\\
61.16	0\\
61.17	0\\
61.18	0\\
61.19	0\\
61.2	0\\
61.21	0\\
61.22	0\\
61.23	0\\
61.24	0\\
61.25	0\\
61.26	0\\
61.27	0\\
61.28	1.73472347597681e-18\\
61.29	0\\
61.3	0\\
61.31	0\\
61.32	0\\
61.33	0\\
61.34	0\\
61.35	0\\
61.36	0\\
61.37	0\\
61.38	0\\
61.39	0\\
61.4	0\\
61.41	0\\
61.42	0\\
61.43	1.73472347597681e-18\\
61.44	0\\
61.45	0\\
61.46	1.73472347597681e-18\\
61.47	1.73472347597681e-18\\
61.48	0\\
61.49	1.73472347597681e-18\\
61.5	0\\
61.51	0\\
61.52	0\\
61.53	0\\
61.54	0\\
61.55	0\\
61.56	0\\
61.57	0\\
61.58	0\\
61.59	0\\
61.6	0\\
61.61	0\\
61.62	0\\
61.63	0\\
61.64	0\\
61.65	1.73472347597681e-18\\
61.66	0\\
61.67	1.73472347597681e-18\\
61.68	0\\
61.69	0\\
61.7	0\\
61.71	0\\
61.72	0\\
61.73	0\\
61.74	0\\
61.75	0\\
61.76	0\\
61.77	1.73472347597681e-18\\
61.78	0\\
61.79	0\\
61.8	0\\
61.81	0\\
61.82	1.73472347597681e-18\\
61.83	0\\
61.84	1.73472347597681e-18\\
61.85	0\\
61.86	0\\
61.87	0\\
61.88	1.73472347597681e-18\\
61.89	0\\
61.9	0\\
61.91	0\\
61.92	0\\
61.93	0\\
61.94	0\\
61.95	0\\
61.96	0\\
61.97	1.73472347597681e-18\\
61.98	0\\
61.99	0\\
62	0\\
62.01	0\\
62.02	0\\
62.03	0\\
62.04	1.73472347597681e-18\\
62.05	0\\
62.06	0\\
62.07	0\\
62.08	0\\
62.09	0\\
62.1	0\\
62.11	0\\
62.12	0\\
62.13	0\\
62.14	0\\
62.15	0\\
62.16	0\\
62.17	1.73472347597681e-18\\
62.18	0\\
62.19	0\\
62.2	0\\
62.21	0\\
62.22	0\\
62.23	0\\
62.24	0\\
62.25	0\\
62.26	0\\
62.27	0\\
62.28	0\\
62.29	0\\
62.3	0\\
62.31	0\\
62.32	0\\
62.33	0\\
62.34	0\\
62.35	0\\
62.36	0\\
62.37	0\\
62.38	0\\
62.39	0\\
62.4	0\\
62.41	0\\
62.42	0\\
62.43	0\\
62.44	0\\
62.45	0\\
62.46	1.73472347597681e-18\\
62.47	0\\
62.48	0\\
62.49	1.73472347597681e-18\\
62.5	0\\
62.51	0\\
62.52	0\\
62.53	0\\
62.54	0\\
62.55	0\\
62.56	0\\
62.57	0\\
62.58	0\\
62.59	0\\
62.6	0\\
62.61	0\\
62.62	0\\
62.63	0\\
62.64	0\\
62.65	0\\
62.66	0\\
62.67	0\\
62.68	0\\
62.69	0\\
62.7	0\\
62.71	0\\
62.72	0\\
62.73	0\\
62.74	0\\
62.75	0\\
62.76	0\\
62.77	1.73472347597681e-18\\
62.78	0\\
62.79	0\\
62.8	0\\
62.81	0\\
62.82	0\\
62.83	0\\
62.84	1.73472347597681e-18\\
62.85	0\\
62.86	0\\
62.87	0\\
62.88	0\\
62.89	0\\
62.9	0\\
62.91	0\\
62.92	0\\
62.93	0\\
62.94	0\\
62.95	0\\
62.96	0\\
62.97	0\\
62.98	0\\
62.99	1.73472347597681e-18\\
63	0\\
63.01	0\\
63.02	1.73472347597681e-18\\
63.03	0\\
63.04	0\\
63.05	1.73472347597681e-18\\
63.06	0\\
63.07	0\\
63.08	0\\
63.09	0\\
63.1	0\\
63.11	0\\
63.12	0\\
63.13	0\\
63.14	0\\
63.15	0\\
63.16	0\\
63.17	0\\
63.18	1.73472347597681e-18\\
63.19	0\\
63.2	0\\
63.21	0\\
63.22	0\\
63.23	0\\
63.24	0\\
63.25	0\\
63.26	0\\
63.27	0\\
63.28	0\\
63.29	0\\
63.3	0\\
63.31	0\\
63.32	0\\
63.33	0\\
63.34	0\\
63.35	0\\
63.36	0\\
63.37	1.73472347597681e-18\\
63.38	0\\
63.39	0\\
63.4	0\\
63.41	0\\
63.42	0\\
63.43	0\\
63.44	0\\
63.45	0\\
63.46	0\\
63.47	0\\
63.48	0\\
63.49	1.73472347597681e-18\\
63.5	0\\
63.51	0\\
63.52	0\\
63.53	0\\
63.54	0\\
63.55	0\\
63.56	0\\
63.57	0\\
63.58	0\\
63.59	0\\
63.6	0\\
63.61	0\\
63.62	0\\
63.63	0\\
63.64	0\\
63.65	0\\
63.66	0\\
63.67	0\\
63.68	0\\
63.69	0\\
63.7	0\\
63.71	0\\
63.72	0\\
63.73	0\\
63.74	0\\
63.75	0\\
63.76	0\\
63.77	0\\
63.78	0\\
63.79	0\\
63.8	1.73472347597681e-18\\
63.81	0\\
63.82	0\\
63.83	0\\
63.84	0\\
63.85	0\\
63.86	0\\
63.87	0\\
63.88	0\\
63.89	0\\
63.9	0\\
63.91	0\\
63.92	1.73472347597681e-18\\
63.93	0\\
63.94	0\\
63.95	0\\
63.96	1.73472347597681e-18\\
63.97	1.73472347597681e-18\\
63.98	0\\
63.99	0\\
64	0\\
64.01	0\\
64.02	0\\
64.03	0\\
64.04	0\\
64.05	0\\
64.06	1.73472347597681e-18\\
64.07	0\\
64.08	0\\
64.09	0\\
64.1	0\\
64.11	0\\
64.12	0\\
64.13	1.73472347597681e-18\\
64.14	0\\
64.15	0\\
64.16	0\\
64.17	0\\
64.18	0\\
64.19	0\\
64.2	0\\
64.21	0\\
64.22	0\\
64.23	0\\
64.24	0\\
64.25	0\\
64.26	0\\
64.27	0\\
64.28	0\\
64.29	1.73472347597681e-18\\
64.3	0\\
64.31	0\\
64.32	0\\
64.33	0\\
64.34	1.73472347597681e-18\\
64.35	0\\
64.36	0\\
64.37	0\\
64.38	0\\
64.39	0\\
64.4	0\\
64.41	0\\
64.42	1.73472347597681e-18\\
64.43	0\\
64.44	0\\
64.45	0\\
64.46	1.73472347597681e-18\\
64.47	0\\
64.48	0\\
64.49	0\\
64.5	0\\
64.51	0\\
64.52	0\\
64.53	0\\
64.54	0\\
64.55	0\\
64.56	0\\
64.57	0\\
64.58	0\\
64.59	1.73472347597681e-18\\
64.6	0\\
64.61	0\\
64.62	0\\
64.63	0\\
64.64	0\\
64.65	1.73472347597681e-18\\
64.66	0\\
64.67	0\\
64.68	0\\
64.69	0\\
64.7	0\\
64.71	0\\
64.72	0\\
64.73	0\\
64.74	0\\
64.75	1.73472347597681e-18\\
64.76	0\\
64.77	0\\
64.78	0\\
64.79	0\\
64.8	0\\
64.81	0\\
64.82	0\\
64.83	1.73472347597681e-18\\
64.84	1.73472347597681e-18\\
64.85	0\\
64.86	0\\
64.87	1.73472347597681e-18\\
64.88	0\\
64.89	0\\
64.9	0\\
64.91	0\\
64.92	0\\
64.93	0\\
64.94	0\\
64.95	0\\
64.96	0\\
64.97	1.73472347597681e-18\\
64.98	0\\
64.99	1.73472347597681e-18\\
65	0\\
65.01	0\\
65.02	0\\
65.03	0\\
65.04	0\\
65.05	0\\
65.06	0\\
65.07	1.73472347597681e-18\\
65.08	0\\
65.09	0\\
65.1	0\\
65.11	0\\
65.12	0\\
65.13	0\\
65.14	0\\
65.15	0\\
65.16	0\\
65.17	0\\
65.18	1.73472347597681e-18\\
65.19	0\\
65.2	0\\
65.21	0\\
65.22	0\\
65.23	0\\
65.24	0\\
65.25	0\\
65.26	0\\
65.27	0\\
65.28	0\\
65.29	0\\
65.3	0\\
65.31	0\\
65.32	0\\
65.33	0\\
65.34	0\\
65.35	0\\
65.36	0\\
65.37	1.73472347597681e-18\\
65.38	0\\
65.39	0\\
65.4	0\\
65.41	0\\
65.42	0\\
65.43	0\\
65.44	0\\
65.45	0\\
65.46	0\\
65.47	0\\
65.48	0\\
65.49	0\\
65.5	0\\
65.51	0\\
65.52	0\\
65.53	0\\
65.54	0\\
65.55	0\\
65.56	0\\
65.57	0\\
65.58	0\\
65.59	0\\
65.6	0\\
65.61	0\\
65.62	0\\
65.63	0\\
65.64	0\\
65.65	1.73472347597681e-18\\
65.66	0\\
65.67	0\\
65.68	1.73472347597681e-18\\
65.69	0\\
65.7	0\\
65.71	0\\
65.72	0\\
65.73	0\\
65.74	0\\
65.75	0\\
65.76	0\\
65.77	0\\
65.78	0\\
65.79	0\\
65.8	1.73472347597681e-18\\
65.81	1.73472347597681e-18\\
65.82	0\\
65.83	1.73472347597681e-18\\
65.84	0\\
65.85	0\\
65.86	1.73472347597681e-18\\
65.87	0\\
65.88	0\\
65.89	0\\
65.9	0\\
65.91	0\\
65.92	1.73472347597681e-18\\
65.93	1.73472347597681e-18\\
65.94	0\\
65.95	0\\
65.96	0\\
65.97	0\\
65.98	0\\
65.99	0\\
66	0\\
66.01	0\\
66.02	0\\
66.03	0\\
66.04	0\\
66.05	0\\
66.06	1.73472347597681e-18\\
66.07	0\\
66.08	0\\
66.09	0\\
66.1	0\\
66.11	0\\
66.12	0\\
66.13	0\\
66.14	0\\
66.15	1.73472347597681e-18\\
66.16	0\\
66.17	0\\
66.18	0\\
66.19	0\\
66.2	0\\
66.21	0\\
66.22	0\\
66.23	1.73472347597681e-18\\
66.24	0\\
66.25	0\\
66.26	0\\
66.27	0\\
66.28	0\\
66.29	1.73472347597681e-18\\
66.3	0\\
66.31	0\\
66.32	0\\
66.33	0\\
66.34	0\\
66.35	1.73472347597681e-18\\
66.36	0\\
66.37	1.73472347597681e-18\\
66.38	0\\
66.39	0\\
66.4	0\\
66.41	0\\
66.42	0\\
66.43	0\\
66.44	0\\
66.45	0\\
66.46	0\\
66.47	0\\
66.48	0\\
66.49	1.73472347597681e-18\\
66.5	0\\
66.51	0\\
66.52	0\\
66.53	0\\
66.54	0\\
66.55	0\\
66.56	0\\
66.57	1.73472347597681e-18\\
66.58	0\\
66.59	0\\
66.6	0\\
66.61	0\\
66.62	0\\
66.63	0\\
66.64	0\\
66.65	1.73472347597681e-18\\
66.66	0\\
66.67	0\\
66.68	1.73472347597681e-18\\
66.69	0\\
66.7	0\\
66.71	0\\
66.72	0\\
66.73	0\\
66.74	0\\
66.75	0\\
66.76	0\\
66.77	0\\
66.78	0\\
66.79	0\\
66.8	0\\
66.81	0\\
66.82	0\\
66.83	0\\
66.84	0\\
66.85	0\\
66.86	0\\
66.87	0\\
66.88	0\\
66.89	0\\
66.9	0\\
66.91	1.73472347597681e-18\\
66.92	0\\
66.93	0\\
66.94	1.73472347597681e-18\\
66.95	0\\
66.96	0\\
66.97	1.73472347597681e-18\\
66.98	0\\
66.99	0\\
67	0\\
67.01	0\\
67.02	1.73472347597681e-18\\
67.03	0\\
67.04	0\\
67.05	1.73472347597681e-18\\
67.06	0\\
67.07	0\\
67.08	1.73472347597681e-18\\
67.09	0\\
67.1	0\\
67.11	0\\
67.12	0\\
67.13	0\\
67.14	0\\
67.15	0\\
67.16	0\\
67.17	0\\
67.18	0\\
67.19	0\\
67.2	0\\
67.21	0\\
67.22	0\\
67.23	0\\
67.24	0\\
67.25	0\\
67.26	0\\
67.27	0\\
67.28	0\\
67.29	0\\
67.3	0\\
67.31	0\\
67.32	0\\
67.33	0\\
67.34	0\\
67.35	0\\
67.36	1.73472347597681e-18\\
67.37	0\\
67.38	0\\
67.39	0\\
67.4	0\\
67.41	1.73472347597681e-18\\
67.42	0\\
67.43	0\\
67.44	0\\
67.45	0\\
67.46	1.73472347597681e-18\\
67.47	0\\
67.48	0\\
67.49	0\\
67.5	1.73472347597681e-18\\
67.51	0\\
67.52	0\\
67.53	0\\
67.54	0\\
67.55	0\\
67.56	0\\
67.57	0\\
67.58	0\\
67.59	0\\
67.6	0\\
67.61	1.73472347597681e-18\\
67.62	0\\
67.63	0\\
67.64	1.73472347597681e-18\\
67.65	0\\
67.66	0\\
67.67	0\\
67.68	0\\
67.69	0\\
67.7	0\\
67.71	0\\
67.72	0\\
67.73	0\\
67.74	0\\
67.75	0\\
67.76	0\\
67.77	0\\
67.78	0\\
67.79	0\\
67.8	0\\
67.81	0\\
67.82	0\\
67.83	0\\
67.84	0\\
67.85	1.73472347597681e-18\\
67.86	0\\
67.87	0\\
67.88	0\\
67.89	0\\
67.9	0\\
67.91	0\\
67.92	0\\
67.93	0\\
67.94	0\\
67.95	0\\
67.96	0\\
67.97	0\\
67.98	0\\
67.99	0\\
68	0\\
68.01	0\\
68.02	1.73472347597681e-18\\
68.03	0\\
68.04	0\\
68.05	0\\
68.06	0\\
68.07	0\\
68.08	0\\
68.09	1.73472347597681e-18\\
68.1	0\\
68.11	0\\
68.12	0\\
68.13	0\\
68.14	0\\
68.15	0\\
68.16	0\\
68.17	1.73472347597681e-18\\
68.18	0\\
68.19	1.73472347597681e-18\\
68.2	0\\
68.21	0\\
68.22	0\\
68.23	0\\
68.24	0\\
68.25	0\\
68.26	0\\
68.27	0\\
68.28	0\\
68.29	0\\
68.3	1.73472347597681e-18\\
68.31	0\\
68.32	0\\
68.33	0\\
68.34	0\\
68.35	0\\
68.36	1.73472347597681e-18\\
68.37	0\\
68.38	0\\
68.39	0\\
68.4	0\\
68.41	0\\
68.42	0\\
68.43	1.73472347597681e-18\\
68.44	0\\
68.45	0\\
68.46	0\\
68.47	0\\
68.48	0\\
68.49	0\\
68.5	0\\
68.51	0\\
68.52	0\\
68.53	0\\
68.54	0\\
68.55	0\\
68.56	0\\
68.57	0\\
68.58	0\\
68.59	0\\
68.6	0\\
68.61	0\\
68.62	0\\
68.63	0\\
68.64	0\\
68.65	0\\
68.66	0\\
68.67	0\\
68.68	0\\
68.69	0\\
68.7	0\\
68.71	1.73472347597681e-18\\
68.72	0\\
68.73	0\\
68.74	0\\
68.75	0\\
68.76	0\\
68.77	0\\
68.78	0\\
68.79	0\\
68.8	0\\
68.81	0\\
68.82	0\\
68.83	0\\
68.84	0\\
68.85	0\\
68.86	0\\
68.87	0\\
68.88	0\\
68.89	0\\
68.9	0\\
68.91	0\\
68.92	0\\
68.93	0\\
68.94	0\\
68.95	0\\
68.96	0\\
68.97	0\\
68.98	0\\
68.99	0\\
69	1.73472347597681e-18\\
69.01	0\\
69.02	0\\
69.03	0\\
69.04	0\\
69.05	1.73472347597681e-18\\
69.06	0\\
69.07	0\\
69.08	0\\
69.09	0\\
69.1	0\\
69.11	0\\
69.12	0\\
69.13	1.73472347597681e-18\\
69.14	0\\
69.15	1.73472347597681e-18\\
69.16	0\\
69.17	0\\
69.18	0\\
69.19	1.73472347597681e-18\\
69.2	0\\
69.21	1.73472347597681e-18\\
69.22	0\\
69.23	0\\
69.24	0\\
69.25	0\\
69.26	0\\
69.27	0\\
69.28	0\\
69.29	0\\
69.3	0\\
69.31	0\\
69.32	0\\
69.33	0\\
69.34	0\\
69.35	0\\
69.36	0\\
69.37	1.73472347597681e-18\\
69.38	0\\
69.39	0\\
69.4	0\\
69.41	0\\
69.42	0\\
69.43	1.73472347597681e-18\\
69.44	0\\
69.45	1.73472347597681e-18\\
69.46	0\\
69.47	0\\
69.48	0\\
69.49	0\\
69.5	0\\
69.51	0\\
69.52	0\\
69.53	0\\
69.54	0\\
69.55	0\\
69.56	0\\
69.57	1.73472347597681e-18\\
69.58	1.73472347597681e-18\\
69.59	0\\
69.6	0\\
69.61	0\\
69.62	0\\
69.63	0\\
69.64	1.73472347597681e-18\\
69.65	0\\
69.66	0\\
69.67	1.73472347597681e-18\\
69.68	0\\
69.69	0\\
69.7	0\\
69.71	0\\
69.72	0\\
69.73	0\\
69.74	0\\
69.75	0\\
69.76	0\\
69.77	0\\
69.78	0\\
69.79	0\\
69.8	0\\
69.81	0\\
69.82	0\\
69.83	0\\
69.84	0\\
69.85	0\\
69.86	0\\
69.87	0\\
69.88	1.73472347597681e-18\\
69.89	0\\
69.9	0\\
69.91	0\\
69.92	0\\
69.93	0\\
69.94	0\\
69.95	1.73472347597681e-18\\
69.96	0\\
69.97	0\\
69.98	0\\
69.99	0\\
70	0\\
70.01	0\\
70.02	0\\
70.03	0\\
70.04	0\\
70.05	0\\
70.06	0\\
70.07	0\\
70.08	0\\
70.09	0\\
70.1	1.73472347597681e-18\\
70.11	0\\
70.12	0\\
70.13	0\\
70.14	1.73472347597681e-18\\
70.15	0\\
70.16	0\\
70.17	0\\
70.18	0\\
70.19	0\\
70.2	0\\
70.21	0\\
70.22	0\\
70.23	0\\
70.24	0\\
70.25	0\\
70.26	0\\
70.27	0\\
70.28	0\\
70.29	0\\
70.3	0\\
70.31	0\\
70.32	0\\
70.33	1.73472347597681e-18\\
70.34	0\\
70.35	0\\
70.36	0\\
70.37	0\\
70.38	0\\
70.39	0\\
70.4	1.73472347597681e-18\\
70.41	0\\
70.42	0\\
70.43	0\\
70.44	0\\
70.45	0\\
70.46	1.73472347597681e-18\\
70.47	0\\
70.48	0\\
70.49	0\\
70.5	0\\
70.51	0\\
70.52	0\\
70.53	0\\
70.54	0\\
70.55	0\\
70.56	0\\
70.57	0\\
70.58	0\\
70.59	0\\
70.6	0\\
70.61	0\\
70.62	0\\
70.63	0\\
70.64	0\\
70.65	1.73472347597681e-18\\
70.66	0\\
70.67	0\\
70.68	0\\
70.69	0\\
70.7	0\\
70.71	1.73472347597681e-18\\
70.72	0\\
70.73	0\\
70.74	0\\
70.75	0\\
70.76	0\\
70.77	0\\
70.78	0\\
70.79	0\\
70.8	0\\
70.81	0\\
70.82	1.73472347597681e-18\\
70.83	0\\
70.84	0\\
70.85	1.73472347597681e-18\\
70.86	0\\
70.87	0\\
70.88	0\\
70.89	0\\
70.9	0\\
70.91	0\\
70.92	1.73472347597681e-18\\
70.93	0\\
70.94	0\\
70.95	0\\
70.96	0\\
70.97	0\\
70.98	0\\
70.99	0\\
71	0\\
71.01	0\\
71.02	0\\
71.03	0\\
71.04	0\\
71.05	0\\
71.06	0\\
71.07	0\\
71.08	1.73472347597681e-18\\
71.09	1.73472347597681e-18\\
71.1	0\\
71.11	0\\
71.12	0\\
71.13	0\\
71.14	0\\
71.15	0\\
71.16	0\\
71.17	0\\
71.18	1.73472347597681e-18\\
71.19	0\\
71.2	0\\
71.21	0\\
71.22	0\\
71.23	0\\
71.24	0\\
71.25	0\\
71.26	0\\
71.27	1.73472347597681e-18\\
71.28	0\\
71.29	0\\
71.3	0\\
71.31	0\\
71.32	1.73472347597681e-18\\
71.33	0\\
71.34	0\\
71.35	0\\
71.36	1.73472347597681e-18\\
71.37	0\\
71.38	0\\
71.39	0\\
71.4	0\\
71.41	0\\
71.42	0\\
71.43	1.73472347597681e-18\\
71.44	0\\
71.45	0\\
71.46	0\\
71.47	0\\
71.48	0\\
71.49	0\\
71.5	0\\
71.51	0\\
71.52	1.73472347597681e-18\\
71.53	0\\
71.54	0\\
71.55	0\\
71.56	0\\
71.57	0\\
71.58	0\\
71.59	0\\
71.6	0\\
71.61	1.73472347597681e-18\\
71.62	0\\
71.63	1.73472347597681e-18\\
71.64	0\\
71.65	0\\
71.66	0\\
71.67	0\\
71.68	0\\
71.69	0\\
71.7	0\\
71.71	1.73472347597681e-18\\
71.72	0\\
71.73	0\\
71.74	0\\
71.75	0\\
71.76	0\\
71.77	0\\
71.78	0\\
71.79	0\\
71.8	0\\
71.81	0\\
71.82	0\\
71.83	1.73472347597681e-18\\
71.84	0\\
71.85	0\\
71.86	0\\
71.87	0\\
71.88	0\\
71.89	1.73472347597681e-18\\
71.9	0\\
71.91	0\\
71.92	0\\
71.93	0\\
71.94	0\\
71.95	0\\
71.96	1.73472347597681e-18\\
71.97	0\\
71.98	0\\
71.99	0\\
72	0\\
72.01	0\\
72.02	0\\
72.03	0\\
72.04	0\\
72.05	0\\
72.06	0\\
72.07	0\\
72.08	0\\
72.09	0\\
72.1	0\\
72.11	0\\
72.12	1.73472347597681e-18\\
72.13	0\\
72.14	0\\
72.15	0\\
72.16	0\\
72.17	0\\
72.18	0\\
72.19	1.73472347597681e-18\\
72.2	0\\
72.21	0\\
72.22	0\\
72.23	0\\
72.24	0\\
72.25	0\\
72.26	0\\
72.27	0\\
72.28	0\\
72.29	0\\
72.3	0\\
72.31	0\\
72.32	0\\
72.33	0\\
72.34	0\\
72.35	0\\
72.36	0\\
72.37	0\\
72.38	0\\
72.39	0\\
72.4	0\\
72.41	0\\
72.42	0\\
72.43	0\\
72.44	0\\
72.45	1.73472347597681e-18\\
72.46	0\\
72.47	0\\
72.48	0\\
72.49	0\\
72.5	0\\
72.51	0\\
72.52	0\\
72.53	0\\
72.54	0\\
72.55	1.73472347597681e-18\\
72.56	0\\
72.57	0\\
72.58	0\\
72.59	0\\
72.6	0\\
72.61	0\\
72.62	1.73472347597681e-18\\
72.63	0\\
72.64	0\\
72.65	0\\
72.66	0\\
72.67	0\\
72.68	0\\
72.69	0\\
72.7	1.73472347597681e-18\\
72.71	0\\
72.72	0\\
72.73	1.73472347597681e-18\\
72.74	0\\
72.75	0\\
72.76	0\\
72.77	0\\
72.78	1.73472347597681e-18\\
72.79	0\\
72.8	0\\
72.81	0\\
72.82	0\\
72.83	0\\
72.84	0\\
72.85	1.73472347597681e-18\\
72.86	0\\
72.87	0\\
72.88	0\\
72.89	1.73472347597681e-18\\
72.9	0\\
72.91	0\\
72.92	0\\
72.93	0\\
72.94	0\\
72.95	0\\
72.96	0\\
72.97	0\\
72.98	0\\
72.99	0\\
73	0\\
73.01	0\\
73.02	1.73472347597681e-18\\
73.03	0\\
73.04	0\\
73.05	0\\
73.06	0\\
73.07	0\\
73.08	0\\
73.09	0\\
73.1	0\\
73.11	0\\
73.12	0\\
73.13	1.73472347597681e-18\\
73.14	0\\
73.15	0\\
73.16	0\\
73.17	0\\
73.18	0\\
73.19	0\\
73.2	0\\
73.21	0\\
73.22	0\\
73.23	1.73472347597681e-18\\
73.24	0\\
73.25	0\\
73.26	0\\
73.27	1.73472347597681e-18\\
73.28	0\\
73.29	0\\
73.3	0\\
73.31	0\\
73.32	0\\
73.33	0\\
73.34	0\\
73.35	0\\
73.36	0\\
73.37	0\\
73.38	0\\
73.39	1.73472347597681e-18\\
73.4	0\\
73.41	0\\
73.42	0\\
73.43	0\\
73.44	0\\
73.45	0\\
73.46	0\\
73.47	0\\
73.48	0\\
73.49	0\\
73.5	0\\
73.51	1.73472347597681e-18\\
73.52	1.73472347597681e-18\\
73.53	0\\
73.54	0\\
73.55	1.73472347597681e-18\\
73.56	0\\
73.57	0\\
73.58	0\\
73.59	0\\
73.6	0\\
73.61	0\\
73.62	0\\
73.63	0\\
73.64	0\\
73.65	0\\
73.66	1.73472347597681e-18\\
73.67	0\\
73.68	0\\
73.69	0\\
73.7	0\\
73.71	0\\
73.72	0\\
73.73	1.73472347597681e-18\\
73.74	0\\
73.75	0\\
73.76	1.73472347597681e-18\\
73.77	0\\
73.78	0\\
73.79	0\\
73.8	1.73472347597681e-18\\
73.81	0\\
73.82	0\\
73.83	0\\
73.84	0\\
73.85	0\\
73.86	0\\
73.87	0\\
73.88	0\\
73.89	0\\
73.9	0\\
73.91	0\\
73.92	0\\
73.93	0\\
73.94	0\\
73.95	0\\
73.96	0\\
73.97	0\\
73.98	1.73472347597681e-18\\
73.99	0\\
74	0\\
74.01	1.73472347597681e-18\\
74.02	0\\
74.03	0\\
74.04	0\\
74.05	0\\
74.06	0\\
74.07	0\\
74.08	0\\
74.09	0\\
74.1	1.73472347597681e-18\\
74.11	1.73472347597681e-18\\
74.12	1.73472347597681e-18\\
74.13	0\\
74.14	1.73472347597681e-18\\
74.15	0\\
74.16	0\\
74.17	1.73472347597681e-18\\
74.18	0\\
74.19	0\\
74.2	0\\
74.21	0\\
74.22	0\\
74.23	0\\
74.24	0\\
74.25	0\\
74.26	0\\
74.27	0\\
74.28	0\\
74.29	0\\
74.3	1.73472347597681e-18\\
74.31	0\\
74.32	0\\
74.33	0\\
74.34	0\\
74.35	0\\
74.36	0\\
74.37	1.73472347597681e-18\\
74.38	0\\
74.39	0\\
74.4	0\\
74.41	0\\
74.42	0\\
74.43	0\\
74.44	0\\
74.45	0\\
74.46	0\\
74.47	0\\
74.48	0\\
74.49	0\\
74.5	0\\
74.51	0\\
74.52	0\\
74.53	0\\
74.54	0\\
74.55	0\\
74.56	0\\
74.57	0\\
74.58	0\\
74.59	0\\
74.6	1.73472347597681e-18\\
74.61	0\\
74.62	1.73472347597681e-18\\
74.63	0\\
74.64	0\\
74.65	1.73472347597681e-18\\
74.66	0\\
74.67	0\\
74.68	0\\
74.69	0\\
74.7	1.73472347597681e-18\\
74.71	0\\
74.72	0\\
74.73	0\\
74.74	0\\
74.75	0\\
74.76	0\\
74.77	0\\
74.78	0\\
74.79	0\\
74.8	0\\
74.81	0\\
74.82	0\\
74.83	1.73472347597681e-18\\
74.84	0\\
74.85	0\\
74.86	0\\
74.87	1.73472347597681e-18\\
74.88	0\\
74.89	0\\
74.9	0\\
74.91	0\\
74.92	0\\
74.93	0\\
74.94	0\\
74.95	0\\
74.96	0\\
74.97	0\\
74.98	0\\
74.99	0\\
75	1.73472347597681e-18\\
75.01	0\\
75.02	0\\
75.03	1.73472347597681e-18\\
75.04	0\\
75.05	0\\
75.06	1.73472347597681e-18\\
75.07	0\\
75.08	0\\
75.09	0\\
75.1	0\\
75.11	0\\
75.12	0\\
75.13	0\\
75.14	1.73472347597681e-18\\
75.15	0\\
75.16	0\\
75.17	0\\
75.18	0\\
75.19	0\\
75.2	0\\
75.21	0\\
75.22	0\\
75.23	0\\
75.24	0\\
75.25	0\\
75.26	0\\
75.27	0\\
75.28	0\\
75.29	0\\
75.3	0\\
75.31	0\\
75.32	0\\
75.33	0\\
75.34	0\\
75.35	0\\
75.36	1.73472347597681e-18\\
75.37	0\\
75.38	0\\
75.39	0\\
75.4	0\\
75.41	0\\
75.42	0\\
75.43	0\\
75.44	1.73472347597681e-18\\
75.45	0\\
75.46	1.73472347597681e-18\\
75.47	0\\
75.48	0\\
75.49	0\\
75.5	0\\
75.51	0\\
75.52	0\\
75.53	0\\
75.54	0\\
75.55	0\\
75.56	1.73472347597681e-18\\
75.57	0\\
75.58	0\\
75.59	0\\
75.6	0\\
75.61	0\\
75.62	0\\
75.63	0\\
75.64	0\\
75.65	0\\
75.66	0\\
75.67	0\\
75.68	0\\
75.69	0\\
75.7	0\\
75.71	0\\
75.72	0\\
75.73	1.73472347597681e-18\\
75.74	0\\
75.75	0\\
75.76	0\\
75.77	0\\
75.78	0\\
75.79	0\\
75.8	0\\
75.81	0\\
75.82	0\\
75.83	0\\
75.84	0\\
75.85	0\\
75.86	0\\
75.87	0\\
75.88	0\\
75.89	0\\
75.9	0\\
75.91	0\\
75.92	0\\
75.93	1.73472347597681e-18\\
75.94	0\\
75.95	0\\
75.96	0\\
75.97	0\\
75.98	0\\
75.99	0\\
76	0\\
76.01	0\\
76.02	1.73472347597681e-18\\
76.03	0\\
76.04	0\\
76.05	0\\
76.06	0\\
76.07	0\\
76.08	0\\
76.09	0\\
76.1	0\\
76.11	0\\
76.12	0\\
76.13	1.73472347597681e-18\\
76.14	0\\
76.15	0\\
76.16	1.73472347597681e-18\\
76.17	1.73472347597681e-18\\
76.18	0\\
76.19	0\\
76.2	0\\
76.21	0\\
76.22	0\\
76.23	0\\
76.24	0\\
76.25	0\\
76.26	0\\
76.27	0\\
76.28	0\\
76.29	0\\
76.3	0\\
76.31	0\\
76.32	0\\
76.33	1.73472347597681e-18\\
76.34	0\\
76.35	0\\
76.36	0\\
76.37	0\\
76.38	1.73472347597681e-18\\
76.39	0\\
76.4	0\\
76.41	1.73472347597681e-18\\
76.42	1.73472347597681e-18\\
76.43	0\\
76.44	0\\
76.45	0\\
76.46	1.73472347597681e-18\\
76.47	0\\
76.48	0\\
76.49	0\\
76.5	0\\
76.51	0\\
76.52	1.73472347597681e-18\\
76.53	0\\
76.54	1.73472347597681e-18\\
76.55	1.73472347597681e-18\\
76.56	0\\
76.57	0\\
76.58	0\\
76.59	0\\
76.6	0\\
76.61	0\\
76.62	0\\
76.63	0\\
76.64	0\\
76.65	0\\
76.66	0\\
76.67	0\\
76.68	0\\
76.69	0\\
76.7	0\\
76.71	0\\
76.72	0\\
76.73	0\\
76.74	0\\
76.75	0\\
76.76	0\\
76.77	0\\
76.78	0\\
76.79	0\\
76.8	0\\
76.81	0\\
76.82	0\\
76.83	0\\
76.84	0\\
76.85	0\\
76.86	0\\
76.87	0\\
76.88	0\\
76.89	1.73472347597681e-18\\
76.9	0\\
76.91	0\\
76.92	0\\
76.93	0\\
76.94	1.73472347597681e-18\\
76.95	0\\
76.96	0\\
76.97	0\\
76.98	0\\
76.99	0\\
77	0\\
77.01	0\\
77.02	0\\
77.03	0\\
77.04	1.73472347597681e-18\\
77.05	0\\
77.06	0\\
77.07	0\\
77.08	0\\
77.09	0\\
77.1	0\\
77.11	0\\
77.12	0\\
77.13	0\\
77.14	0\\
77.15	0\\
77.16	1.73472347597681e-18\\
77.17	0\\
77.18	0\\
77.19	1.73472347597681e-18\\
77.2	0\\
77.21	0\\
77.22	0\\
77.23	0\\
77.24	0\\
77.25	0\\
77.26	0\\
77.27	0\\
77.28	0\\
77.29	0\\
77.3	0\\
77.31	0\\
77.32	0\\
77.33	0\\
77.34	0\\
77.35	0\\
77.36	0\\
77.37	0\\
77.38	0\\
77.39	0\\
77.4	0\\
77.41	1.73472347597681e-18\\
77.42	0\\
77.43	0\\
77.44	0\\
77.45	0\\
77.46	0\\
77.47	0\\
77.48	0\\
77.49	1.73472347597681e-18\\
77.5	0\\
77.51	0\\
77.52	0\\
77.53	0\\
77.54	1.73472347597681e-18\\
77.55	0\\
77.56	0\\
77.57	0\\
77.58	0\\
77.59	0\\
77.6	0\\
77.61	0\\
77.62	0\\
77.63	0\\
77.64	0\\
77.65	1.73472347597681e-18\\
77.66	0\\
77.67	0\\
77.68	0\\
77.69	0\\
77.7	0\\
77.71	0\\
77.72	0\\
77.73	0\\
77.74	0\\
77.75	1.73472347597681e-18\\
77.76	0\\
77.77	1.73472347597681e-18\\
77.78	0\\
77.79	1.73472347597681e-18\\
77.8	0\\
77.81	0\\
77.82	0\\
77.83	0\\
77.84	0\\
77.85	1.73472347597681e-18\\
77.86	0\\
77.87	0\\
77.88	0\\
77.89	0\\
77.9	0\\
77.91	0\\
77.92	0\\
77.93	0\\
77.94	0\\
77.95	0\\
77.96	0\\
77.97	0\\
77.98	0\\
77.99	0\\
78	0\\
78.01	0\\
78.02	0\\
78.03	0\\
78.04	1.73472347597681e-18\\
78.05	0\\
78.06	0\\
78.07	0\\
78.08	0\\
78.09	0\\
78.1	1.73472347597681e-18\\
78.11	0\\
78.12	0\\
78.13	0\\
78.14	0\\
78.15	0\\
78.16	0\\
78.17	0\\
78.18	0\\
78.19	0\\
78.2	1.73472347597681e-18\\
78.21	1.73472347597681e-18\\
78.22	0\\
78.23	0\\
78.24	0\\
78.25	1.73472347597681e-18\\
78.26	0\\
78.27	0\\
78.28	0\\
78.29	1.73472347597681e-18\\
78.3	0\\
78.31	0\\
78.32	0\\
78.33	0\\
78.34	0\\
78.35	0\\
78.36	0\\
78.37	0\\
78.38	0\\
78.39	0\\
78.4	1.73472347597681e-18\\
78.41	1.73472347597681e-18\\
78.42	0\\
78.43	0\\
78.44	0\\
78.45	0\\
78.46	0\\
78.47	0\\
78.48	0\\
78.49	0\\
78.5	0\\
78.51	0\\
78.52	0\\
78.53	0\\
78.54	1.73472347597681e-18\\
78.55	0\\
78.56	0\\
78.57	0\\
78.58	0\\
78.59	0\\
78.6	0\\
78.61	0\\
78.62	1.73472347597681e-18\\
78.63	0\\
78.64	0\\
78.65	0\\
78.66	0\\
78.67	0\\
78.68	0\\
78.69	0\\
78.7	0\\
78.71	0\\
78.72	0\\
78.73	0\\
78.74	0\\
78.75	0\\
78.76	0\\
78.77	0\\
78.78	0\\
78.79	0\\
78.8	0\\
78.81	0\\
78.82	0\\
78.83	0\\
78.84	0\\
78.85	0\\
78.86	0\\
78.87	1.73472347597681e-18\\
78.88	0\\
78.89	0\\
78.9	1.73472347597681e-18\\
78.91	0\\
78.92	1.73472347597681e-18\\
78.93	0\\
78.94	0\\
78.95	1.73472347597681e-18\\
78.96	0\\
78.97	0\\
78.98	1.73472347597681e-18\\
78.99	0\\
79	0\\
79.01	0\\
79.02	0\\
79.03	0\\
79.04	0\\
79.05	0\\
79.06	0\\
79.07	0\\
79.08	1.73472347597681e-18\\
79.09	0\\
79.1	0\\
79.11	0\\
79.12	0\\
79.13	0\\
79.14	0\\
79.15	0\\
79.16	0\\
79.17	0\\
79.18	0\\
79.19	0\\
79.2	0\\
79.21	0\\
79.22	0\\
79.23	0\\
79.24	0\\
79.25	0\\
79.26	0\\
79.27	0\\
79.28	0\\
79.29	0\\
79.3	1.73472347597681e-18\\
79.31	0\\
79.32	0\\
79.33	0\\
79.34	0\\
79.35	1.73472347597681e-18\\
79.36	0\\
79.37	0\\
79.38	0\\
79.39	0\\
79.4	0\\
79.41	0\\
79.42	0\\
79.43	0\\
79.44	0\\
79.45	0\\
79.46	1.73472347597681e-18\\
79.47	0\\
79.48	0\\
79.49	0\\
79.5	0\\
79.51	0\\
79.52	0\\
79.53	1.73472347597681e-18\\
79.54	0\\
79.55	0\\
79.56	1.73472347597681e-18\\
79.57	1.73472347597681e-18\\
79.58	0\\
79.59	1.73472347597681e-18\\
79.6	0\\
79.61	0\\
79.62	0\\
79.63	0\\
79.64	0\\
79.65	0\\
79.66	0\\
79.67	0\\
79.68	0\\
79.69	0\\
79.7	0\\
79.71	0\\
79.72	0\\
79.73	0\\
79.74	0\\
79.75	0\\
79.76	0\\
79.77	0\\
79.78	0\\
79.79	1.73472347597681e-18\\
79.8	0\\
79.81	0\\
79.82	1.73472347597681e-18\\
79.83	0\\
79.84	0\\
79.85	0\\
79.86	0\\
79.87	1.73472347597681e-18\\
79.88	0\\
79.89	0\\
79.9	0\\
79.91	0\\
79.92	0\\
79.93	0\\
79.94	1.73472347597681e-18\\
79.95	0\\
79.96	0\\
79.97	1.73472347597681e-18\\
79.98	1.73472347597681e-18\\
79.99	1.73472347597681e-18\\
80	0\\
80.01	0\\
};
\addplot [color=red,dashed]
  table[row sep=crcr]{%
80.01	0\\
80.02	0\\
80.03	1.73472347597681e-18\\
80.04	0\\
80.05	0\\
80.06	0\\
80.07	0\\
80.08	0\\
80.09	0\\
80.1	0\\
80.11	1.73472347597681e-18\\
80.12	0\\
80.13	1.73472347597681e-18\\
80.14	0\\
80.15	0\\
80.16	0\\
80.17	1.73472347597681e-18\\
80.18	1.73472347597681e-18\\
80.19	0\\
80.2	0\\
80.21	1.73472347597681e-18\\
80.22	0\\
80.23	1.73472347597681e-18\\
80.24	0\\
80.25	1.73472347597681e-18\\
80.26	0\\
80.27	0\\
80.28	0\\
80.29	0\\
80.3	0\\
80.31	0\\
80.32	0\\
80.33	0\\
80.34	0\\
80.35	0\\
80.36	0\\
80.37	1.73472347597681e-18\\
80.38	0\\
80.39	0\\
80.4	1.73472347597681e-18\\
80.41	0\\
80.42	0\\
80.43	1.73472347597681e-18\\
80.44	0\\
80.45	0\\
80.46	1.73472347597681e-18\\
80.47	0\\
80.48	0\\
80.49	0\\
80.5	0\\
80.51	0\\
80.52	0\\
80.53	0\\
80.54	0\\
80.55	0\\
80.56	1.73472347597681e-18\\
80.57	0\\
80.58	0\\
80.59	0\\
80.6	1.73472347597681e-18\\
80.61	0\\
80.62	0\\
80.63	0\\
80.64	0\\
80.65	0\\
80.66	0\\
80.67	0\\
80.68	0\\
80.69	1.73472347597681e-18\\
80.7	1.73472347597681e-18\\
80.71	0\\
80.72	0\\
80.73	1.73472347597681e-18\\
80.74	0\\
80.75	0\\
80.76	0\\
80.77	0\\
80.78	0\\
80.79	0\\
80.8	0\\
80.81	1.73472347597681e-18\\
80.82	0\\
80.83	0\\
80.84	0\\
80.85	0\\
80.86	0\\
80.87	0\\
80.88	0\\
80.89	0\\
80.9	1.73472347597681e-18\\
80.91	0\\
80.92	0\\
80.93	0\\
80.94	0\\
80.95	1.73472347597681e-18\\
80.96	0\\
80.97	0\\
80.98	0\\
80.99	0\\
81	1.73472347597681e-18\\
81.01	0\\
81.02	0\\
81.03	0\\
81.04	0\\
81.05	0\\
81.06	1.73472347597681e-18\\
81.07	1.73472347597681e-18\\
81.08	0\\
81.09	0\\
81.1	0\\
81.11	0\\
81.12	0\\
81.13	0\\
81.14	0\\
81.15	0\\
81.16	0\\
81.17	0\\
81.18	0\\
81.19	0\\
81.2	0\\
81.21	0\\
81.22	0\\
81.23	0\\
81.24	0\\
81.25	0\\
81.26	0\\
81.27	0\\
81.28	1.73472347597681e-18\\
81.29	1.73472347597681e-18\\
81.3	0\\
81.31	0\\
81.32	0\\
81.33	0\\
81.34	1.73472347597681e-18\\
81.35	0\\
81.36	0\\
81.37	0\\
81.38	0\\
81.39	0\\
81.4	1.73472347597681e-18\\
81.41	0\\
81.42	0\\
81.43	0\\
81.44	0\\
81.45	0\\
81.46	0\\
81.47	0\\
81.48	0\\
81.49	1.73472347597681e-18\\
81.5	0\\
81.51	0\\
81.52	0\\
81.53	0\\
81.54	1.73472347597681e-18\\
81.55	0\\
81.56	0\\
81.57	0\\
81.58	0\\
81.59	0\\
81.6	0\\
81.61	0\\
81.62	0\\
81.63	0\\
81.64	0\\
81.65	0\\
81.66	0\\
81.67	0\\
81.68	1.73472347597681e-18\\
81.69	0\\
81.7	0\\
81.71	1.73472347597681e-18\\
81.72	0\\
81.73	0\\
81.74	0\\
81.75	0\\
81.76	1.73472347597681e-18\\
81.77	0\\
81.78	0\\
81.79	1.73472347597681e-18\\
81.8	0\\
81.81	0\\
81.82	0\\
81.83	0\\
81.84	0\\
81.85	0\\
81.86	0\\
81.87	0\\
81.88	0\\
81.89	0\\
81.9	0\\
81.91	0\\
81.92	0\\
81.93	0\\
81.94	1.73472347597681e-18\\
81.95	0\\
81.96	1.73472347597681e-18\\
81.97	0\\
81.98	0\\
81.99	1.73472347597681e-18\\
82	0\\
82.01	1.73472347597681e-18\\
82.02	1.73472347597681e-18\\
82.03	1.73472347597681e-18\\
82.04	0\\
82.05	1.73472347597681e-18\\
82.06	0\\
82.07	1.73472347597681e-18\\
82.08	0\\
82.09	0\\
82.1	0\\
82.11	0\\
82.12	0\\
82.13	0\\
82.14	0\\
82.15	0\\
82.16	0\\
82.17	0\\
82.18	0\\
82.19	0\\
82.2	0\\
82.21	0\\
82.22	0\\
82.23	0\\
82.24	0\\
82.25	0\\
82.26	0\\
82.27	0\\
82.28	0\\
82.29	0\\
82.3	0\\
82.31	0\\
82.32	0\\
82.33	1.73472347597681e-18\\
82.34	0\\
82.35	1.73472347597681e-18\\
82.36	0\\
82.37	0\\
82.38	0\\
82.39	0\\
82.4	0\\
82.41	0\\
82.42	0\\
82.43	0\\
82.44	0\\
82.45	0\\
82.46	0\\
82.47	0\\
82.48	0\\
82.49	0\\
82.5	1.73472347597681e-18\\
82.51	0\\
82.52	0\\
82.53	0\\
82.54	0\\
82.55	0\\
82.56	0\\
82.57	0\\
82.58	0\\
82.59	0\\
82.6	0\\
82.61	0\\
82.62	0\\
82.63	1.73472347597681e-18\\
82.64	1.73472347597681e-18\\
82.65	1.73472347597681e-18\\
82.66	0\\
82.67	0\\
82.68	0\\
82.69	0\\
82.7	0\\
82.71	0\\
82.72	0\\
82.73	0\\
82.74	0\\
82.75	0\\
82.76	0\\
82.77	1.73472347597681e-18\\
82.78	0\\
82.79	0\\
82.8	0\\
82.81	0\\
82.82	0\\
82.83	0\\
82.84	0\\
82.85	1.73472347597681e-18\\
82.86	0\\
82.87	0\\
82.88	0\\
82.89	1.73472347597681e-18\\
82.9	0\\
82.91	0\\
82.92	0\\
82.93	0\\
82.94	0\\
82.95	0\\
82.96	0\\
82.97	0\\
82.98	0\\
82.99	0\\
83	0\\
83.01	0\\
83.02	0\\
83.03	0\\
83.04	0\\
83.05	1.73472347597681e-18\\
83.06	0\\
83.07	0\\
83.08	0\\
83.09	0\\
83.1	1.73472347597681e-18\\
83.11	1.73472347597681e-18\\
83.12	1.73472347597681e-18\\
83.13	0\\
83.14	0\\
83.15	0\\
83.16	0\\
83.17	0\\
83.18	0\\
83.19	0\\
83.2	1.73472347597681e-18\\
83.21	0\\
83.22	0\\
83.23	1.73472347597681e-18\\
83.24	0\\
83.25	0\\
83.26	1.73472347597681e-18\\
83.27	0\\
83.28	0\\
83.29	1.73472347597681e-18\\
83.3	0\\
83.31	0\\
83.32	0\\
83.33	0\\
83.34	0\\
83.35	0\\
83.36	0\\
83.37	0\\
83.38	0\\
83.39	0\\
83.4	0\\
83.41	0\\
83.42	0\\
83.43	0\\
83.44	0\\
83.45	0\\
83.46	0\\
83.47	1.73472347597681e-18\\
83.48	0\\
83.49	1.73472347597681e-18\\
83.5	0\\
83.51	0\\
83.52	0\\
83.53	0\\
83.54	0\\
83.55	0\\
83.56	0\\
83.57	0\\
83.58	0\\
83.59	0\\
83.6	1.73472347597681e-18\\
83.61	1.73472347597681e-18\\
83.62	0\\
83.63	1.73472347597681e-18\\
83.64	0\\
83.65	0\\
83.66	0\\
83.67	0\\
83.68	0\\
83.69	1.73472347597681e-18\\
83.7	0\\
83.71	0\\
83.72	0\\
83.73	0\\
83.74	0\\
83.75	0\\
83.76	0\\
83.77	0\\
83.78	0\\
83.79	1.73472347597681e-18\\
83.8	0\\
83.81	0\\
83.82	0\\
83.83	0\\
83.84	0\\
83.85	0\\
83.86	0\\
83.87	0\\
83.88	0\\
83.89	0\\
83.9	0\\
83.91	0\\
83.92	0\\
83.93	0\\
83.94	0\\
83.95	0\\
83.96	0\\
83.97	0\\
83.98	0\\
83.99	0\\
84	0\\
84.01	0\\
84.02	0\\
84.03	0\\
84.04	0\\
84.05	0\\
84.06	0\\
84.07	1.73472347597681e-18\\
84.08	0\\
84.09	0\\
84.1	0\\
84.11	1.73472347597681e-18\\
84.12	0\\
84.13	0\\
84.14	0\\
84.15	0\\
84.16	0\\
84.17	0\\
84.18	0\\
84.19	0\\
84.2	0\\
84.21	0\\
84.22	1.73472347597681e-18\\
84.23	0\\
84.24	0\\
84.25	0\\
84.26	1.73472347597681e-18\\
84.27	0\\
84.28	0\\
84.29	0\\
84.3	0\\
84.31	0\\
84.32	1.73472347597681e-18\\
84.33	0\\
84.34	0\\
84.35	1.73472347597681e-18\\
84.36	1.73472347597681e-18\\
84.37	0\\
84.38	0\\
84.39	0\\
84.4	0\\
84.41	1.73472347597681e-18\\
84.42	1.73472347597681e-18\\
84.43	0\\
84.44	0\\
84.45	0\\
84.46	0\\
84.47	0\\
84.48	0\\
84.49	0\\
84.5	1.73472347597681e-18\\
84.51	0\\
84.52	0\\
84.53	0\\
84.54	0\\
84.55	0\\
84.56	0\\
84.57	0\\
84.58	0\\
84.59	1.73472347597681e-18\\
84.6	0\\
84.61	0\\
84.62	0\\
84.63	0\\
84.64	0\\
84.65	0\\
84.66	1.73472347597681e-18\\
84.67	1.73472347597681e-18\\
84.68	0\\
84.69	0\\
84.7	0\\
84.71	0\\
84.72	0\\
84.73	0\\
84.74	0\\
84.75	0\\
84.76	1.73472347597681e-18\\
84.77	0\\
84.78	0\\
84.79	0\\
84.8	0\\
84.81	0\\
84.82	0\\
84.83	0\\
84.84	0\\
84.85	0\\
84.86	0\\
84.87	0\\
84.88	0\\
84.89	0\\
84.9	0\\
84.91	0\\
84.92	0\\
84.93	0\\
84.94	0\\
84.95	0\\
84.96	0\\
84.97	0\\
84.98	0\\
84.99	0\\
85	0\\
85.01	0\\
85.02	0\\
85.03	0\\
85.04	0\\
85.05	0\\
85.06	0\\
85.07	0\\
85.08	0\\
85.09	0\\
85.1	0\\
85.11	0\\
85.12	0\\
85.13	0\\
85.14	0\\
85.15	0\\
85.16	0\\
85.17	0\\
85.18	0\\
85.19	0\\
85.2	0\\
85.21	0\\
85.22	0\\
85.23	1.73472347597681e-18\\
85.24	0\\
85.25	0\\
85.26	1.73472347597681e-18\\
85.27	0\\
85.28	1.73472347597681e-18\\
85.29	0\\
85.3	0\\
85.31	1.73472347597681e-18\\
85.32	0\\
85.33	1.73472347597681e-18\\
85.34	0\\
85.35	0\\
85.36	0\\
85.37	0\\
85.38	0\\
85.39	0\\
85.4	0\\
85.41	0\\
85.42	0\\
85.43	0\\
85.44	0\\
85.45	0\\
85.46	0\\
85.47	0\\
85.48	0\\
85.49	0\\
85.5	0\\
85.51	0\\
85.52	0\\
85.53	1.73472347597681e-18\\
85.54	0\\
85.55	0\\
85.56	0\\
85.57	0\\
85.58	0\\
85.59	0\\
85.6	0\\
85.61	0\\
85.62	0\\
85.63	0\\
85.64	0\\
85.65	0\\
85.66	0\\
85.67	0\\
85.68	0\\
85.69	0\\
85.7	0\\
85.71	0\\
85.72	0\\
85.73	0\\
85.74	0\\
85.75	0\\
85.76	0\\
85.77	0\\
85.78	0\\
85.79	0\\
85.8	0\\
85.81	1.73472347597681e-18\\
85.82	0\\
85.83	0\\
85.84	0\\
85.85	0\\
85.86	0\\
85.87	0\\
85.88	0\\
85.89	0\\
85.9	0\\
85.91	0\\
85.92	0\\
85.93	0\\
85.94	0\\
85.95	0\\
85.96	0\\
85.97	1.73472347597681e-18\\
85.98	0\\
85.99	0\\
86	0\\
86.01	0\\
86.02	1.73472347597681e-18\\
86.03	1.73472347597681e-18\\
86.04	0\\
86.05	0\\
86.06	0\\
86.07	0\\
86.08	0\\
86.09	0\\
86.1	0\\
86.11	0\\
86.12	0\\
86.13	0\\
86.14	0\\
86.15	0\\
86.16	0\\
86.17	0\\
86.18	0\\
86.19	0\\
86.2	0\\
86.21	0\\
86.22	0\\
86.23	0\\
86.24	0\\
86.25	1.73472347597681e-18\\
86.26	1.73472347597681e-18\\
86.27	0\\
86.28	0\\
86.29	0\\
86.3	0\\
86.31	0\\
86.32	0\\
86.33	1.73472347597681e-18\\
86.34	0\\
86.35	0\\
86.36	0\\
86.37	0\\
86.38	0\\
86.39	0\\
86.4	0\\
86.41	0\\
86.42	0\\
86.43	0\\
86.44	0\\
86.45	0\\
86.46	0\\
86.47	0\\
86.48	0\\
86.49	0\\
86.5	0\\
86.51	0\\
86.52	0\\
86.53	0\\
86.54	0\\
86.55	1.73472347597681e-18\\
86.56	0\\
86.57	0\\
86.58	0\\
86.59	0\\
86.6	0\\
86.61	0\\
86.62	0\\
86.63	0\\
86.64	0\\
86.65	0\\
86.66	0\\
86.67	0\\
86.68	0\\
86.69	0\\
86.7	1.73472347597681e-18\\
86.71	1.73472347597681e-18\\
86.72	0\\
86.73	0\\
86.74	0\\
86.75	0\\
86.76	0\\
86.77	0\\
86.78	0\\
86.79	0\\
86.8	0\\
86.81	0\\
86.82	0\\
86.83	0\\
86.84	0\\
86.85	0\\
86.86	0\\
86.87	0\\
86.88	0\\
86.89	0\\
86.9	0\\
86.91	0\\
86.92	0\\
86.93	0\\
86.94	0\\
86.95	0\\
86.96	0\\
86.97	1.73472347597681e-18\\
86.98	0\\
86.99	0\\
87	0\\
87.01	0\\
87.02	0\\
87.03	0\\
87.04	0\\
87.05	0\\
87.06	0\\
87.07	0\\
87.08	0\\
87.09	0\\
87.1	0\\
87.11	0\\
87.12	0\\
87.13	0\\
87.14	0\\
87.15	0\\
87.16	0\\
87.17	0\\
87.18	0\\
87.19	0\\
87.2	0\\
87.21	0\\
87.22	0\\
87.23	0\\
87.24	0\\
87.25	0\\
87.26	0\\
87.27	1.73472347597681e-18\\
87.28	0\\
87.29	0\\
87.3	0\\
87.31	0\\
87.32	0\\
87.33	0\\
87.34	0\\
87.35	0\\
87.36	0\\
87.37	0\\
87.38	0\\
87.39	0\\
87.4	0\\
87.41	0\\
87.42	0\\
87.43	0\\
87.44	0\\
87.45	0\\
87.46	0\\
87.47	0\\
87.48	0\\
87.49	0\\
87.5	0\\
87.51	1.73472347597681e-18\\
87.52	0\\
87.53	0\\
87.54	0\\
87.55	0\\
87.56	0\\
87.57	0\\
87.58	0\\
87.59	0\\
87.6	0\\
87.61	0\\
87.62	1.73472347597681e-18\\
87.63	0\\
87.64	0\\
87.65	0\\
87.66	0\\
87.67	0\\
87.68	0\\
87.69	1.73472347597681e-18\\
87.7	1.73472347597681e-18\\
87.71	0\\
87.72	0\\
87.73	0\\
87.74	0\\
87.75	0\\
87.76	0\\
87.77	0\\
87.78	0\\
87.79	0\\
87.8	0\\
87.81	0\\
87.82	0\\
87.83	1.73472347597681e-18\\
87.84	0\\
87.85	0\\
87.86	0\\
87.87	0\\
87.88	0\\
87.89	0\\
87.9	0\\
87.91	0\\
87.92	0\\
87.93	0\\
87.94	0\\
87.95	0\\
87.96	0\\
87.97	0\\
87.98	0\\
87.99	0\\
88	0\\
88.01	0\\
88.02	0\\
88.03	0\\
88.04	0\\
88.05	0\\
88.06	0\\
88.07	0\\
88.08	0\\
88.09	0\\
88.1	0\\
88.11	0\\
88.12	0\\
88.13	0\\
88.14	0\\
88.15	1.73472347597681e-18\\
88.16	0\\
88.17	0\\
88.18	0\\
88.19	0\\
88.2	0\\
88.21	0\\
88.22	0\\
88.23	0\\
88.24	0\\
88.25	0\\
88.26	0\\
88.27	0\\
88.28	0\\
88.29	0\\
88.3	0\\
88.31	0\\
88.32	0\\
88.33	0\\
88.34	0\\
88.35	0\\
88.36	0\\
88.37	0\\
88.38	0\\
88.39	0\\
88.4	0\\
88.41	0\\
88.42	0\\
88.43	0\\
88.44	0\\
88.45	0\\
88.46	1.73472347597681e-18\\
88.47	0\\
88.48	0\\
88.49	0\\
88.5	0\\
88.51	0\\
88.52	0\\
88.53	0\\
88.54	0\\
88.55	0\\
88.56	0\\
88.57	0\\
88.58	0\\
88.59	0\\
88.6	0\\
88.61	0\\
88.62	0\\
88.63	0\\
88.64	0\\
88.65	0\\
88.66	0\\
88.67	0\\
88.68	0\\
88.69	0\\
88.7	0\\
88.71	0\\
88.72	0\\
88.73	0\\
88.74	0\\
88.75	0\\
88.76	0\\
88.77	0\\
88.78	0\\
88.79	0\\
88.8	0\\
88.81	0\\
88.82	0\\
88.83	0\\
88.84	0\\
88.85	0\\
88.86	0\\
88.87	0\\
88.88	0\\
88.89	0\\
88.9	0\\
88.91	0\\
88.92	0\\
88.93	0\\
88.94	0\\
88.95	0\\
88.96	0\\
88.97	0\\
88.98	0\\
88.99	0\\
89	0\\
89.01	0\\
89.02	0\\
89.03	0\\
89.04	0\\
89.05	0\\
89.06	0\\
89.07	0\\
89.08	0\\
89.09	0\\
89.1	0\\
89.11	0\\
89.12	0\\
89.13	0\\
89.14	0\\
89.15	0\\
89.16	0\\
89.17	0\\
89.18	0\\
89.19	0\\
89.2	0\\
89.21	0\\
89.22	0\\
89.23	0\\
89.24	0\\
89.25	0\\
89.26	0\\
89.27	0\\
89.28	0\\
89.29	0\\
89.3	0\\
89.31	0\\
89.32	0\\
89.33	0\\
89.34	0\\
89.35	0\\
89.36	0\\
89.37	0\\
89.38	1.73472347597681e-18\\
89.39	0\\
89.4	0\\
89.41	0\\
89.42	0\\
89.43	0\\
89.44	0\\
89.45	0\\
89.46	0\\
89.47	0\\
89.48	0\\
89.49	0\\
89.5	0\\
89.51	0\\
89.52	0\\
89.53	0\\
89.54	0\\
89.55	0\\
89.56	0\\
89.57	0\\
89.58	0\\
89.59	0\\
89.6	0\\
89.61	0\\
89.62	0\\
89.63	0\\
89.64	0\\
89.65	1.73472347597681e-18\\
89.66	0\\
89.67	0\\
89.68	1.73472347597681e-18\\
89.69	0\\
89.7	0\\
89.71	0\\
89.72	0\\
89.73	0\\
89.74	0\\
89.75	0\\
89.76	0\\
89.77	0\\
89.78	0\\
89.79	0\\
89.8	0\\
89.81	0\\
89.82	0\\
89.83	0\\
89.84	0\\
89.85	0\\
89.86	0\\
89.87	0\\
89.88	0\\
89.89	0\\
89.9	0\\
89.91	0\\
89.92	0\\
89.93	0\\
89.94	0\\
89.95	0\\
89.96	0\\
89.97	0\\
89.98	0\\
89.99	0\\
90	0\\
90.01	0\\
90.02	0\\
90.03	0\\
90.04	0\\
90.05	0\\
90.06	1.73472347597681e-18\\
90.07	0\\
90.08	0\\
90.09	0\\
90.1	0\\
90.11	0\\
90.12	0\\
90.13	0\\
90.14	0\\
90.15	0\\
90.16	0\\
90.17	0\\
90.18	1.73472347597681e-18\\
90.19	0\\
90.2	0\\
90.21	0\\
90.22	0\\
90.23	0\\
90.24	0\\
90.25	0\\
90.26	0\\
90.27	0\\
90.28	0\\
90.29	0\\
90.3	0\\
90.31	0\\
90.32	0\\
90.33	1.73472347597681e-18\\
90.34	0\\
90.35	0\\
90.36	0\\
90.37	0\\
90.38	0\\
90.39	0\\
90.4	0\\
90.41	0\\
90.42	0\\
90.43	0\\
90.44	0\\
90.45	0\\
90.46	0\\
90.47	0\\
90.48	0\\
90.49	0\\
90.5	0\\
90.51	0\\
90.52	0\\
90.53	0\\
90.54	0\\
90.55	0\\
90.56	0\\
90.57	0\\
90.58	0\\
90.59	0\\
90.6	0\\
90.61	0\\
90.62	0\\
90.63	0\\
90.64	0\\
90.65	0\\
90.66	0\\
90.67	0\\
90.68	0\\
90.69	0\\
90.7	0\\
90.71	0\\
90.72	0\\
90.73	0\\
90.74	0\\
90.75	0\\
90.76	0\\
90.77	0\\
90.78	0\\
90.79	0\\
90.8	0\\
90.81	0\\
90.82	0\\
90.83	0\\
90.84	0\\
90.85	0\\
90.86	0\\
90.87	0\\
90.88	0\\
90.89	0\\
90.9	0\\
90.91	0\\
90.92	0\\
90.93	0\\
90.94	0\\
90.95	0\\
90.96	0\\
90.97	0\\
90.98	0\\
90.99	0\\
91	0\\
91.01	0\\
91.02	0\\
91.03	0\\
91.04	0\\
91.05	0\\
91.06	0\\
91.07	0\\
91.08	0\\
91.09	0\\
91.1	0\\
91.11	0\\
91.12	0\\
91.13	0\\
91.14	0\\
91.15	0\\
91.16	0\\
91.17	0\\
91.18	0\\
91.19	0\\
91.2	0\\
91.21	0\\
91.22	0\\
91.23	0\\
91.24	0\\
91.25	0\\
91.26	0\\
91.27	0\\
91.28	0\\
91.29	0\\
91.3	0\\
91.31	0\\
91.32	0\\
91.33	0\\
91.34	0\\
91.35	0\\
91.36	0\\
91.37	0\\
91.38	0\\
91.39	0\\
91.4	0\\
91.41	0\\
91.42	0\\
91.43	0\\
91.44	0\\
91.45	0\\
91.46	0\\
91.47	0\\
91.48	0\\
91.49	0\\
91.5	0\\
91.51	0\\
91.52	0\\
91.53	0\\
91.54	0\\
91.55	0\\
91.56	0\\
91.57	0\\
91.58	0\\
91.59	0\\
91.6	0\\
91.61	0\\
91.62	0\\
91.63	0\\
91.64	0\\
91.65	0\\
91.66	0\\
91.67	0\\
91.68	0\\
91.69	0\\
91.7	0\\
91.71	0\\
91.72	0\\
91.73	0\\
91.74	0\\
91.75	0\\
91.76	0\\
91.77	0\\
91.78	0\\
91.79	0\\
91.8	0\\
91.81	0\\
91.82	0\\
91.83	0\\
91.84	0\\
91.85	0\\
91.86	0\\
91.87	0\\
91.88	0\\
91.89	0\\
91.9	0\\
91.91	0\\
91.92	0\\
91.93	0\\
91.94	0\\
91.95	0\\
91.96	0\\
91.97	0\\
91.98	0\\
91.99	0\\
92	0\\
92.01	0\\
92.02	0\\
92.03	0\\
92.04	0\\
92.05	0\\
92.06	0\\
92.07	0\\
92.08	0\\
92.09	0\\
92.1	0\\
92.11	0\\
92.12	0\\
92.13	0\\
92.14	0\\
92.15	0\\
92.16	0\\
92.17	0\\
92.18	0\\
92.19	0\\
92.2	0\\
92.21	0\\
92.22	0\\
92.23	0\\
92.24	0\\
92.25	0\\
92.26	0\\
92.27	0\\
92.28	0\\
92.29	0\\
92.3	0\\
92.31	0\\
92.32	0\\
92.33	0\\
92.34	0\\
92.35	0\\
92.36	0\\
92.37	0\\
92.38	0\\
92.39	0\\
92.4	0\\
92.41	0\\
92.42	0\\
92.43	0\\
92.44	0\\
92.45	0\\
92.46	0\\
92.47	0\\
92.48	0\\
92.49	0\\
92.5	0\\
92.51	0\\
92.52	0\\
92.53	0\\
92.54	0\\
92.55	0\\
92.56	0\\
92.57	0\\
92.58	0\\
92.59	0\\
92.6	0\\
92.61	0\\
92.62	0\\
92.63	0\\
92.64	0\\
92.65	0\\
92.66	0\\
92.67	0\\
92.68	0\\
92.69	0\\
92.7	0\\
92.71	0\\
92.72	0\\
92.73	0\\
92.74	0\\
92.75	0\\
92.76	0\\
92.77	0\\
92.78	0\\
92.79	0\\
92.8	0\\
92.81	0\\
92.82	0\\
92.83	0\\
92.84	0\\
92.85	0\\
92.86	0\\
92.87	0\\
92.88	0\\
92.89	0\\
92.9	0\\
92.91	0\\
92.92	0\\
92.93	0\\
92.94	0\\
92.95	0\\
92.96	0\\
92.97	0\\
92.98	0\\
92.99	0\\
93	0\\
93.01	0\\
93.02	0\\
93.03	0\\
93.04	0\\
93.05	0\\
93.06	0\\
93.07	0\\
93.08	0\\
93.09	0\\
93.1	0\\
93.11	0\\
93.12	0\\
93.13	0\\
93.14	0\\
93.15	0\\
93.16	0\\
93.17	0\\
93.18	0\\
93.19	0\\
93.2	0\\
93.21	0\\
93.22	0\\
93.23	0\\
93.24	0\\
93.25	0\\
93.26	0\\
93.27	0\\
93.28	0\\
93.29	0\\
93.3	0\\
93.31	0\\
93.32	0\\
93.33	0\\
93.34	0\\
93.35	0\\
93.36	0\\
93.37	0\\
93.38	0\\
93.39	0\\
93.4	0\\
93.41	0\\
93.42	0\\
93.43	0\\
93.44	0\\
93.45	0\\
93.46	0\\
93.47	0\\
93.48	0\\
93.49	0\\
93.5	0\\
93.51	0\\
93.52	0\\
93.53	0\\
93.54	0\\
93.55	0\\
93.56	0\\
93.57	0\\
93.58	0\\
93.59	0\\
93.6	0\\
93.61	0\\
93.62	0\\
93.63	0\\
93.64	0\\
93.65	0\\
93.66	0\\
93.67	0\\
93.68	0\\
93.69	0\\
93.7	0\\
93.71	0\\
93.72	0\\
93.73	0\\
93.74	0\\
93.75	0\\
93.76	0\\
93.77	0\\
93.78	0\\
93.79	0\\
93.8	0\\
93.81	0\\
93.82	0\\
93.83	0\\
93.84	0\\
93.85	0\\
93.86	0\\
93.87	0\\
93.88	0\\
93.89	0\\
93.9	0\\
93.91	0\\
93.92	0\\
93.93	0\\
93.94	0\\
93.95	0\\
93.96	0\\
93.97	0\\
93.98	0\\
93.99	0\\
94	0\\
94.01	0\\
94.02	0\\
94.03	0\\
94.04	0\\
94.05	0\\
94.06	0\\
94.07	0\\
94.08	0\\
94.09	0\\
94.1	0\\
94.11	0\\
94.12	0\\
94.13	0\\
94.14	0\\
94.15	0\\
94.16	0\\
94.17	0\\
94.18	0\\
94.19	0\\
94.2	0\\
94.21	0\\
94.22	0\\
94.23	0\\
94.24	0\\
94.25	0\\
94.26	0\\
94.27	0\\
94.28	0\\
94.29	0\\
94.3	0\\
94.31	0\\
94.32	0\\
94.33	0\\
94.34	0\\
94.35	0\\
94.36	0\\
94.37	0\\
94.38	0\\
94.39	0\\
94.4	0\\
94.41	0\\
94.42	0\\
94.43	0\\
94.44	0\\
94.45	0\\
94.46	0\\
94.47	0\\
94.48	0\\
94.49	0\\
94.5	0\\
94.51	0\\
94.52	0\\
94.53	0\\
94.54	0\\
94.55	0\\
94.56	0\\
94.57	0\\
94.58	0\\
94.59	0\\
94.6	0\\
94.61	0\\
94.62	0\\
94.63	0\\
94.64	0\\
94.65	0\\
94.66	0\\
94.67	0\\
94.68	0\\
94.69	0\\
94.7	0\\
94.71	0\\
94.72	0\\
94.73	0\\
94.74	0\\
94.75	0\\
94.76	0\\
94.77	0\\
94.78	0\\
94.79	0\\
94.8	0\\
94.81	0\\
94.82	0\\
94.83	0\\
94.84	0\\
94.85	0\\
94.86	0\\
94.87	0\\
94.88	0\\
94.89	0\\
94.9	0\\
94.91	0\\
94.92	0\\
94.93	0\\
94.94	0\\
94.95	0\\
94.96	0\\
94.97	0\\
94.98	0\\
94.99	0\\
95	0\\
95.01	0\\
95.02	0\\
95.03	0\\
95.04	0\\
95.05	0\\
95.06	0\\
95.07	0\\
95.08	0\\
95.09	0\\
95.1	0\\
95.11	0\\
95.12	0\\
95.13	0\\
95.14	0\\
95.15	0\\
95.16	0\\
95.17	0\\
95.18	0\\
95.19	0\\
95.2	0\\
95.21	0\\
95.22	0\\
95.23	0\\
95.24	0\\
95.25	0\\
95.26	0\\
95.27	0\\
95.28	0\\
95.29	0\\
95.3	0\\
95.31	0\\
95.32	0\\
95.33	0\\
95.34	0\\
95.35	0\\
95.36	0\\
95.37	0\\
95.38	0\\
95.39	0\\
95.4	0\\
95.41	0\\
95.42	0\\
95.43	0\\
95.44	0\\
95.45	0\\
95.46	0\\
95.47	0\\
95.48	0\\
95.49	0\\
95.5	0\\
95.51	0\\
95.52	0\\
95.53	0\\
95.54	0\\
95.55	0\\
95.56	0\\
95.57	0\\
95.58	0\\
95.59	0\\
95.6	0\\
95.61	0\\
95.62	0\\
95.63	0\\
95.64	0\\
95.65	0\\
95.66	0\\
95.67	0\\
95.68	0\\
95.69	0\\
95.7	0\\
95.71	0\\
95.72	0\\
95.73	0\\
95.74	0\\
95.75	0\\
95.76	0\\
95.77	0\\
95.78	0\\
95.79	0\\
95.8	0\\
95.81	0\\
95.82	0\\
95.83	0\\
95.84	0\\
95.85	0\\
95.86	0\\
95.87	0\\
95.88	0\\
95.89	0\\
95.9	0\\
95.91	0\\
95.92	0\\
95.93	0\\
95.94	0\\
95.95	0\\
95.96	0\\
95.97	0\\
95.98	0\\
95.99	0\\
96	0\\
96.01	0\\
96.02	0\\
96.03	0\\
96.04	0\\
96.05	0\\
96.06	0\\
96.07	0\\
96.08	0\\
96.09	0\\
96.1	0\\
96.11	0\\
96.12	0\\
96.13	0\\
96.14	0\\
96.15	0\\
96.16	0\\
96.17	0\\
96.18	0\\
96.19	0\\
96.2	0\\
96.21	0\\
96.22	0\\
96.23	0\\
96.24	0\\
96.25	0\\
96.26	0\\
96.27	0\\
96.28	0\\
96.29	0\\
96.3	0\\
96.31	0\\
96.32	0\\
96.33	0\\
96.34	0\\
96.35	0\\
96.36	0\\
96.37	0\\
96.38	0\\
96.39	0\\
96.4	0\\
96.41	0\\
96.42	0\\
96.43	0\\
96.44	0\\
96.45	0\\
96.46	0\\
96.47	0\\
96.48	0\\
96.49	0\\
96.5	0\\
96.51	0\\
96.52	0\\
96.53	0\\
96.54	0\\
96.55	0\\
96.56	0\\
96.57	0\\
96.58	0\\
96.59	0\\
96.6	0\\
96.61	0\\
96.62	0\\
96.63	0\\
96.64	0\\
96.65	0\\
96.66	0\\
96.67	0\\
96.68	0\\
96.69	0\\
96.7	0\\
96.71	0\\
96.72	0\\
96.73	0\\
96.74	0\\
96.75	0\\
96.76	0\\
96.77	0\\
96.78	0\\
96.79	0\\
96.8	0\\
96.81	0\\
96.82	0\\
96.83	0\\
96.84	0\\
96.85	0\\
96.86	0\\
96.87	0\\
96.88	0\\
96.89	0\\
96.9	0\\
96.91	0\\
96.92	0\\
96.93	0\\
96.94	0\\
96.95	0\\
96.96	0\\
96.97	0\\
96.98	0\\
96.99	0\\
97	0\\
97.01	0\\
97.02	0\\
97.03	0\\
97.04	0\\
97.05	0\\
97.06	0\\
97.07	0\\
97.08	0\\
97.09	0\\
97.1	0\\
97.11	0\\
97.12	0\\
97.13	0\\
97.14	0\\
97.15	0\\
97.16	0\\
97.17	0\\
97.18	0\\
97.19	0\\
97.2	0\\
97.21	0\\
97.22	0\\
97.23	0\\
97.24	0\\
97.25	0\\
97.26	0\\
97.27	0\\
97.28	0\\
97.29	0\\
97.3	0\\
97.31	0\\
97.32	0\\
97.33	0\\
97.34	0\\
97.35	0\\
97.36	0\\
97.37	0\\
97.38	0\\
97.39	0\\
97.4	0\\
97.41	0\\
97.42	0\\
97.43	0\\
97.44	0\\
97.45	0\\
97.46	0\\
97.47	0\\
97.48	0\\
97.49	0\\
97.5	0\\
97.51	0\\
97.52	0\\
97.53	0\\
97.54	0\\
97.55	0\\
97.56	0\\
97.57	0\\
97.58	0\\
97.59	0\\
97.6	0\\
97.61	0\\
97.62	0\\
97.63	0\\
97.64	0\\
97.65	0\\
97.66	0\\
97.67	0\\
97.68	0\\
97.69	0\\
97.7	0\\
97.71	0\\
97.72	0\\
97.73	0\\
97.74	0\\
97.75	0\\
97.76	0\\
97.77	0\\
97.78	0\\
97.79	0\\
97.8	0\\
97.81	0\\
97.82	0\\
97.83	0\\
97.84	0\\
97.85	0\\
97.86	0\\
97.87	0\\
97.88	0\\
97.89	0\\
97.9	0\\
97.91	0\\
97.92	0\\
97.93	0\\
97.94	0\\
97.95	0\\
97.96	0\\
97.97	0\\
97.98	0\\
97.99	0\\
98	0\\
98.01	0\\
98.02	0\\
98.03	0\\
98.04	0\\
98.05	0\\
98.06	0\\
98.07	0\\
98.08	0\\
98.09	0\\
98.1	0\\
98.11	0\\
98.12	0\\
98.13	0\\
98.14	0\\
98.15	0\\
98.16	0\\
98.17	0\\
98.18	0\\
98.19	0\\
98.2	0\\
98.21	0\\
98.22	0\\
98.23	0\\
98.24	0\\
98.25	0\\
98.26	0\\
98.27	0\\
98.28	0\\
98.29	0\\
98.3	0\\
98.31	0\\
98.32	0\\
98.33	0\\
98.34	0\\
98.35	0\\
98.36	0\\
98.37	0\\
98.38	0\\
98.39	0\\
98.4	0\\
98.41	0\\
98.42	0\\
98.43	0\\
98.44	0\\
98.45	0\\
98.46	0\\
98.47	0\\
98.48	0\\
98.49	0\\
98.5	0\\
98.51	0\\
98.52	0\\
98.53	0\\
98.54	0\\
98.55	0\\
98.56	0\\
98.57	0\\
98.58	0\\
98.59	0\\
98.6	0\\
98.61	0\\
98.62	0\\
98.63	0\\
98.64	0\\
98.65	0\\
98.66	0\\
98.67	0\\
98.68	0\\
98.69	0\\
98.7	0\\
98.71	0\\
98.72	0\\
98.73	0\\
98.74	0\\
98.75	0\\
98.76	0\\
98.77	0\\
98.78	0\\
98.79	0\\
98.8	0\\
98.81	0\\
98.82	0\\
98.83	0\\
98.84	0\\
98.85	0\\
98.86	0\\
98.87	0\\
98.88	0\\
98.89	0\\
98.9	0\\
98.91	0\\
98.92	0\\
98.93	0\\
98.94	0\\
98.95	0\\
98.96	0\\
98.97	0\\
98.98	0\\
98.99	0\\
99	0\\
99.01	0\\
99.02	0\\
99.03	0\\
99.04	0\\
99.05	0\\
99.06	0\\
99.07	0\\
99.08	0\\
99.09	0\\
99.1	0\\
99.11	0\\
99.12	0\\
99.13	0\\
99.14	0\\
99.15	0\\
99.16	0\\
99.17	0\\
99.18	0\\
99.19	0\\
99.2	0\\
99.21	0\\
99.22	0\\
99.23	0\\
99.24	0\\
99.25	0\\
99.26	0\\
99.27	0\\
99.28	0\\
99.29	0\\
99.3	0\\
99.31	0\\
99.32	0\\
99.33	0\\
99.34	0\\
99.35	0\\
99.36	0\\
99.37	0\\
99.38	0\\
99.39	0\\
99.4	0\\
99.41	0\\
99.42	0\\
99.43	0\\
99.44	0\\
99.45	0\\
99.46	0\\
99.47	0\\
99.48	0\\
99.49	0\\
99.5	0\\
99.51	0\\
99.52	0\\
99.53	0\\
99.54	0\\
99.55	0\\
99.56	0\\
99.57	0\\
99.58	0\\
99.59	0\\
99.6	0\\
99.61	0\\
99.62	0\\
99.63	0\\
99.64	0\\
99.65	0\\
99.66	0\\
99.67	0\\
99.68	0\\
99.69	0\\
99.7	0\\
99.71	0\\
99.72	0\\
99.73	0\\
99.74	0\\
99.75	0\\
99.76	0\\
99.77	0\\
99.78	0\\
99.79	0\\
99.8	0\\
99.81	0\\
99.82	0\\
99.83	0\\
99.84	0\\
99.85	0\\
99.86	0\\
99.87	0\\
99.88	0\\
99.89	0\\
99.9	0\\
99.91	0\\
99.92	0\\
99.93	0\\
99.94	0\\
99.95	0\\
99.96	0\\
99.97	0\\
99.98	0\\
99.99	0\\
100	0\\
};
\addlegendentry{$q=-2$};

\addplot [color=blue,dashed,forget plot]
  table[row sep=crcr]{%
0.01	0.00201272129290892\\
0.02	0.00201272129049512\\
0.03	0.00201272128807998\\
0.04	0.00201272128566351\\
0.05	0.0020127212832457\\
0.06	0.00201272128082656\\
0.07	0.00201272127840608\\
0.08	0.00201272127598427\\
0.09	0.00201272127356111\\
0.1	0.00201272127113662\\
0.11	0.00201272126871078\\
0.12	0.00201272126628361\\
0.13	0.00201272126385509\\
0.14	0.00201272126142523\\
0.15	0.00201272125899402\\
0.16	0.00201272125656147\\
0.17	0.00201272125412757\\
0.18	0.00201272125169232\\
0.19	0.00201272124925573\\
0.2	0.00201272124681778\\
0.21	0.00201272124437849\\
0.22	0.00201272124193784\\
0.23	0.00201272123949584\\
0.24	0.00201272123705249\\
0.25	0.00201272123460778\\
0.26	0.00201272123216172\\
0.27	0.00201272122971429\\
0.28	0.00201272122726552\\
0.29	0.00201272122481538\\
0.3	0.00201272122236388\\
0.31	0.00201272121991103\\
0.32	0.00201272121745681\\
0.33	0.00201272121500122\\
0.34	0.00201272121254428\\
0.35	0.00201272121008597\\
0.36	0.00201272120762629\\
0.37	0.00201272120516524\\
0.38	0.00201272120270283\\
0.39	0.00201272120023905\\
0.4	0.0020127211977739\\
0.41	0.00201272119530737\\
0.42	0.00201272119283948\\
0.43	0.00201272119037021\\
0.44	0.00201272118789957\\
0.45	0.00201272118542755\\
0.46	0.00201272118295416\\
0.47	0.00201272118047939\\
0.48	0.00201272117800324\\
0.49	0.00201272117552571\\
0.5	0.0020127211730468\\
0.51	0.00201272117056651\\
0.52	0.00201272116808483\\
0.53	0.00201272116560177\\
0.54	0.00201272116311733\\
0.55	0.0020127211606315\\
0.56	0.00201272115814428\\
0.57	0.00201272115565568\\
0.58	0.00201272115316569\\
0.59	0.00201272115067431\\
0.6	0.00201272114818153\\
0.61	0.00201272114568737\\
0.62	0.00201272114319181\\
0.63	0.00201272114069486\\
0.64	0.00201272113819651\\
0.65	0.00201272113569676\\
0.66	0.00201272113319562\\
0.67	0.00201272113069308\\
0.68	0.00201272112818914\\
0.69	0.0020127211256838\\
0.7	0.00201272112317705\\
0.71	0.00201272112066891\\
0.72	0.00201272111815936\\
0.73	0.0020127211156484\\
0.74	0.00201272111313604\\
0.75	0.00201272111062227\\
0.76	0.0020127211081071\\
0.77	0.00201272110559051\\
0.78	0.00201272110307252\\
0.79	0.00201272110055311\\
0.8	0.00201272109803229\\
0.81	0.00201272109551006\\
0.82	0.00201272109298641\\
0.83	0.00201272109046134\\
0.84	0.00201272108793486\\
0.85	0.00201272108540696\\
0.86	0.00201272108287765\\
0.87	0.00201272108034691\\
0.88	0.00201272107781475\\
0.89	0.00201272107528117\\
0.9	0.00201272107274616\\
0.91	0.00201272107020973\\
0.92	0.00201272106767188\\
0.93	0.0020127210651326\\
0.94	0.00201272106259189\\
0.95	0.00201272106004975\\
0.96	0.00201272105750618\\
0.97	0.00201272105496118\\
0.98	0.00201272105241475\\
0.99	0.00201272104986688\\
1	0.00201272104731758\\
1.01	0.00201272104476685\\
1.02	0.00201272104221467\\
1.03	0.00201272103966106\\
1.04	0.00201272103710602\\
1.05	0.00201272103454953\\
1.06	0.0020127210319916\\
1.07	0.00201272102943222\\
1.08	0.00201272102687141\\
1.09	0.00201272102430915\\
1.1	0.00201272102174544\\
1.11	0.00201272101918029\\
1.12	0.00201272101661369\\
1.13	0.00201272101404564\\
1.14	0.00201272101147614\\
1.15	0.00201272100890519\\
1.16	0.00201272100633279\\
1.17	0.00201272100375893\\
1.18	0.00201272100118362\\
1.19	0.00201272099860686\\
1.2	0.00201272099602863\\
1.21	0.00201272099344895\\
1.22	0.00201272099086781\\
1.23	0.00201272098828521\\
1.24	0.00201272098570115\\
1.25	0.00201272098311562\\
1.26	0.00201272098052863\\
1.27	0.00201272097794018\\
1.28	0.00201272097535026\\
1.29	0.00201272097275888\\
1.3	0.00201272097016602\\
1.31	0.0020127209675717\\
1.32	0.0020127209649759\\
1.33	0.00201272096237864\\
1.34	0.0020127209597799\\
1.35	0.00201272095717968\\
1.36	0.002012720954578\\
1.37	0.00201272095197483\\
1.38	0.00201272094937019\\
1.39	0.00201272094676407\\
1.4	0.00201272094415647\\
1.41	0.00201272094154738\\
1.42	0.00201272093893682\\
1.43	0.00201272093632477\\
1.44	0.00201272093371124\\
1.45	0.00201272093109623\\
1.46	0.00201272092847972\\
1.47	0.00201272092586173\\
1.48	0.00201272092324225\\
1.49	0.00201272092062128\\
1.5	0.00201272091799881\\
1.51	0.00201272091537486\\
1.52	0.00201272091274941\\
1.53	0.00201272091012247\\
1.54	0.00201272090749403\\
1.55	0.00201272090486409\\
1.56	0.00201272090223265\\
1.57	0.00201272089959971\\
1.58	0.00201272089696528\\
1.59	0.00201272089432934\\
1.6	0.0020127208916919\\
1.61	0.00201272088905295\\
1.62	0.0020127208864125\\
1.63	0.00201272088377054\\
1.64	0.00201272088112707\\
1.65	0.0020127208784821\\
1.66	0.00201272087583561\\
1.67	0.00201272087318761\\
1.68	0.0020127208705381\\
1.69	0.00201272086788708\\
1.7	0.00201272086523454\\
1.71	0.00201272086258048\\
1.72	0.00201272085992491\\
1.73	0.00201272085726781\\
1.74	0.0020127208546092\\
1.75	0.00201272085194907\\
1.76	0.00201272084928741\\
1.77	0.00201272084662423\\
1.78	0.00201272084395953\\
1.79	0.00201272084129329\\
1.8	0.00201272083862554\\
1.81	0.00201272083595625\\
1.82	0.00201272083328543\\
1.83	0.00201272083061309\\
1.84	0.00201272082793921\\
1.85	0.0020127208252638\\
1.86	0.00201272082258685\\
1.87	0.00201272081990836\\
1.88	0.00201272081722834\\
1.89	0.00201272081454679\\
1.9	0.00201272081186369\\
1.91	0.00201272080917905\\
1.92	0.00201272080649287\\
1.93	0.00201272080380515\\
1.94	0.00201272080111588\\
1.95	0.00201272079842507\\
1.96	0.00201272079573271\\
1.97	0.0020127207930388\\
1.98	0.00201272079034334\\
1.99	0.00201272078764634\\
2	0.00201272078494778\\
2.01	0.00201272078224766\\
2.02	0.002012720779546\\
2.03	0.00201272077684277\\
2.04	0.00201272077413799\\
2.05	0.00201272077143166\\
2.06	0.00201272076872376\\
2.07	0.0020127207660143\\
2.08	0.00201272076330329\\
2.09	0.00201272076059071\\
2.1	0.00201272075787656\\
2.11	0.00201272075516085\\
2.12	0.00201272075244357\\
2.13	0.00201272074972472\\
2.14	0.00201272074700431\\
2.15	0.00201272074428232\\
2.16	0.00201272074155876\\
2.17	0.00201272073883363\\
2.18	0.00201272073610693\\
2.19	0.00201272073337865\\
2.2	0.00201272073064879\\
2.21	0.00201272072791735\\
2.22	0.00201272072518434\\
2.23	0.00201272072244974\\
2.24	0.00201272071971356\\
2.25	0.0020127207169758\\
2.26	0.00201272071423646\\
2.27	0.00201272071149552\\
2.28	0.00201272070875301\\
2.29	0.0020127207060089\\
2.3	0.0020127207032632\\
2.31	0.00201272070051591\\
2.32	0.00201272069776703\\
2.33	0.00201272069501656\\
2.34	0.00201272069226449\\
2.35	0.00201272068951083\\
2.36	0.00201272068675556\\
2.37	0.0020127206839987\\
2.38	0.00201272068124024\\
2.39	0.00201272067848018\\
2.4	0.00201272067571852\\
2.41	0.00201272067295525\\
2.42	0.00201272067019038\\
2.43	0.0020127206674239\\
2.44	0.00201272066465581\\
2.45	0.00201272066188611\\
2.46	0.00201272065911481\\
2.47	0.00201272065634189\\
2.48	0.00201272065356735\\
2.49	0.00201272065079121\\
2.5	0.00201272064801345\\
2.51	0.00201272064523407\\
2.52	0.00201272064245307\\
2.53	0.00201272063967045\\
2.54	0.00201272063688621\\
2.55	0.00201272063410036\\
2.56	0.00201272063131287\\
2.57	0.00201272062852376\\
2.58	0.00201272062573303\\
2.59	0.00201272062294067\\
2.6	0.00201272062014668\\
2.61	0.00201272061735105\\
2.62	0.0020127206145538\\
2.63	0.00201272061175491\\
2.64	0.0020127206089544\\
2.65	0.00201272060615224\\
2.66	0.00201272060334845\\
2.67	0.00201272060054302\\
2.68	0.00201272059773595\\
2.69	0.00201272059492724\\
2.7	0.00201272059211688\\
2.71	0.00201272058930489\\
2.72	0.00201272058649125\\
2.73	0.00201272058367596\\
2.74	0.00201272058085902\\
2.75	0.00201272057804044\\
2.76	0.0020127205752202\\
2.77	0.00201272057239832\\
2.78	0.00201272056957478\\
2.79	0.00201272056674958\\
2.8	0.00201272056392273\\
2.81	0.00201272056109423\\
2.82	0.00201272055826406\\
2.83	0.00201272055543224\\
2.84	0.00201272055259875\\
2.85	0.0020127205497636\\
2.86	0.00201272054692679\\
2.87	0.00201272054408831\\
2.88	0.00201272054124816\\
2.89	0.00201272053840635\\
2.9	0.00201272053556287\\
2.91	0.00201272053271771\\
2.92	0.00201272052987089\\
2.93	0.00201272052702239\\
2.94	0.00201272052417221\\
2.95	0.00201272052132036\\
2.96	0.00201272051846684\\
2.97	0.00201272051561163\\
2.98	0.00201272051275474\\
2.99	0.00201272050989617\\
3	0.00201272050703592\\
3.01	0.00201272050417398\\
3.02	0.00201272050131036\\
3.03	0.00201272049844504\\
3.04	0.00201272049557804\\
3.05	0.00201272049270935\\
3.06	0.00201272048983897\\
3.07	0.00201272048696689\\
3.08	0.00201272048409312\\
3.09	0.00201272048121766\\
3.1	0.00201272047834049\\
3.11	0.00201272047546163\\
3.12	0.00201272047258107\\
3.13	0.00201272046969881\\
3.14	0.00201272046681484\\
3.15	0.00201272046392917\\
3.16	0.00201272046104179\\
3.17	0.00201272045815271\\
3.18	0.00201272045526192\\
3.19	0.00201272045236941\\
3.2	0.0020127204494752\\
3.21	0.00201272044657927\\
3.22	0.00201272044368163\\
3.23	0.00201272044078228\\
3.24	0.0020127204378812\\
3.25	0.00201272043497841\\
3.26	0.00201272043207389\\
3.27	0.00201272042916766\\
3.28	0.0020127204262597\\
3.29	0.00201272042335002\\
3.3	0.00201272042043861\\
3.31	0.00201272041752548\\
3.32	0.00201272041461061\\
3.33	0.00201272041169402\\
3.34	0.00201272040877569\\
3.35	0.00201272040585564\\
3.36	0.00201272040293384\\
3.37	0.00201272040001031\\
3.38	0.00201272039708505\\
3.39	0.00201272039415804\\
3.4	0.0020127203912293\\
3.41	0.00201272038829881\\
3.42	0.00201272038536658\\
3.43	0.0020127203824326\\
3.44	0.00201272037949688\\
3.45	0.00201272037655941\\
3.46	0.00201272037362019\\
3.47	0.00201272037067922\\
3.48	0.0020127203677365\\
3.49	0.00201272036479203\\
3.5	0.0020127203618458\\
3.51	0.00201272035889781\\
3.52	0.00201272035594807\\
3.53	0.00201272035299657\\
3.54	0.0020127203500433\\
3.55	0.00201272034708828\\
3.56	0.00201272034413148\\
3.57	0.00201272034117293\\
3.58	0.00201272033821261\\
3.59	0.00201272033525052\\
3.6	0.00201272033228665\\
3.61	0.00201272032932102\\
3.62	0.00201272032635362\\
3.63	0.00201272032338444\\
3.64	0.00201272032041348\\
3.65	0.00201272031744075\\
3.66	0.00201272031446624\\
3.67	0.00201272031148994\\
3.68	0.00201272030851187\\
3.69	0.00201272030553201\\
3.7	0.00201272030255037\\
3.71	0.00201272029956694\\
3.72	0.00201272029658173\\
3.73	0.00201272029359472\\
3.74	0.00201272029060592\\
3.75	0.00201272028761534\\
3.76	0.00201272028462295\\
3.77	0.00201272028162877\\
3.78	0.0020127202786328\\
3.79	0.00201272027563502\\
3.8	0.00201272027263545\\
3.81	0.00201272026963407\\
3.82	0.00201272026663089\\
3.83	0.00201272026362591\\
3.84	0.00201272026061912\\
3.85	0.00201272025761052\\
3.86	0.00201272025460011\\
3.87	0.0020127202515879\\
3.88	0.00201272024857387\\
3.89	0.00201272024555802\\
3.9	0.00201272024254036\\
3.91	0.00201272023952088\\
3.92	0.00201272023649959\\
3.93	0.00201272023347648\\
3.94	0.00201272023045154\\
3.95	0.00201272022742478\\
3.96	0.00201272022439619\\
3.97	0.00201272022136578\\
3.98	0.00201272021833354\\
3.99	0.00201272021529947\\
4	0.00201272021226357\\
4.01	0.00201272020922584\\
4.02	0.00201272020618628\\
4.03	0.00201272020314487\\
4.04	0.00201272020010163\\
4.05	0.00201272019705656\\
4.06	0.00201272019400964\\
4.07	0.00201272019096088\\
4.08	0.00201272018791027\\
4.09	0.00201272018485782\\
4.1	0.00201272018180352\\
4.11	0.00201272017874738\\
4.12	0.00201272017568939\\
4.13	0.00201272017262954\\
4.14	0.00201272016956784\\
4.15	0.00201272016650429\\
4.16	0.00201272016343888\\
4.17	0.00201272016037161\\
4.18	0.00201272015730248\\
4.19	0.00201272015423149\\
4.2	0.00201272015115864\\
4.21	0.00201272014808392\\
4.22	0.00201272014500734\\
4.23	0.00201272014192889\\
4.24	0.00201272013884857\\
4.25	0.00201272013576639\\
4.26	0.00201272013268232\\
4.27	0.00201272012959639\\
4.28	0.00201272012650858\\
4.29	0.00201272012341889\\
4.3	0.00201272012032732\\
4.31	0.00201272011723387\\
4.32	0.00201272011413854\\
4.33	0.00201272011104133\\
4.34	0.00201272010794223\\
4.35	0.00201272010484125\\
4.36	0.00201272010173837\\
4.37	0.00201272009863361\\
4.38	0.00201272009552695\\
4.39	0.0020127200924184\\
4.4	0.00201272008930796\\
4.41	0.00201272008619562\\
4.42	0.00201272008308138\\
4.43	0.00201272007996524\\
4.44	0.0020127200768472\\
4.45	0.00201272007372725\\
4.46	0.0020127200706054\\
4.47	0.00201272006748164\\
4.48	0.00201272006435598\\
4.49	0.0020127200612284\\
4.5	0.00201272005809892\\
4.51	0.00201272005496752\\
4.52	0.0020127200518342\\
4.53	0.00201272004869897\\
4.54	0.00201272004556182\\
4.55	0.00201272004242275\\
4.56	0.00201272003928176\\
4.57	0.00201272003613884\\
4.58	0.002012720032994\\
4.59	0.00201272002984724\\
4.6	0.00201272002669854\\
4.61	0.00201272002354792\\
4.62	0.00201272002039536\\
4.63	0.00201272001724087\\
4.64	0.00201272001408445\\
4.65	0.00201272001092609\\
4.66	0.00201272000776579\\
4.67	0.00201272000460354\\
4.68	0.00201272000143936\\
4.69	0.00201271999827324\\
4.7	0.00201271999510517\\
4.71	0.00201271999193515\\
4.72	0.00201271998876318\\
4.73	0.00201271998558926\\
4.74	0.0020127199824134\\
4.75	0.00201271997923557\\
4.76	0.00201271997605579\\
4.77	0.00201271997287406\\
4.78	0.00201271996969037\\
4.79	0.00201271996650471\\
4.8	0.00201271996331709\\
4.81	0.00201271996012751\\
4.82	0.00201271995693597\\
4.83	0.00201271995374245\\
4.84	0.00201271995054697\\
4.85	0.00201271994734951\\
4.86	0.00201271994415008\\
4.87	0.00201271994094868\\
4.88	0.0020127199377453\\
4.89	0.00201271993453995\\
4.9	0.00201271993133261\\
4.91	0.00201271992812329\\
4.92	0.002012719924912\\
4.93	0.00201271992169871\\
4.94	0.00201271991848344\\
4.95	0.00201271991526618\\
4.96	0.00201271991204693\\
4.97	0.00201271990882569\\
4.98	0.00201271990560245\\
4.99	0.00201271990237722\\
5	0.00201271989914999\\
5.01	0.00201271989592076\\
5.02	0.00201271989268954\\
5.03	0.00201271988945631\\
5.04	0.00201271988622107\\
5.05	0.00201271988298383\\
5.06	0.00201271987974458\\
5.07	0.00201271987650333\\
5.08	0.00201271987326006\\
5.09	0.00201271987001477\\
5.1	0.00201271986676748\\
5.11	0.00201271986351816\\
5.12	0.00201271986026683\\
5.13	0.00201271985701348\\
5.14	0.00201271985375811\\
5.15	0.00201271985050071\\
5.16	0.00201271984724128\\
5.17	0.00201271984397983\\
5.18	0.00201271984071635\\
5.19	0.00201271983745084\\
5.2	0.0020127198341833\\
5.21	0.00201271983091372\\
5.22	0.0020127198276421\\
5.23	0.00201271982436845\\
5.24	0.00201271982109276\\
5.25	0.00201271981781502\\
5.26	0.00201271981453524\\
5.27	0.00201271981125342\\
5.28	0.00201271980796955\\
5.29	0.00201271980468362\\
5.3	0.00201271980139565\\
5.31	0.00201271979810563\\
5.32	0.00201271979481355\\
5.33	0.00201271979151941\\
5.34	0.00201271978822322\\
5.35	0.00201271978492497\\
5.36	0.00201271978162465\\
5.37	0.00201271977832227\\
5.38	0.00201271977501783\\
5.39	0.00201271977171131\\
5.4	0.00201271976840273\\
5.41	0.00201271976509208\\
5.42	0.00201271976177935\\
5.43	0.00201271975846455\\
5.44	0.00201271975514767\\
5.45	0.00201271975182872\\
5.46	0.00201271974850768\\
5.47	0.00201271974518456\\
5.48	0.00201271974185936\\
5.49	0.00201271973853207\\
5.5	0.00201271973520269\\
5.51	0.00201271973187122\\
5.52	0.00201271972853767\\
5.53	0.00201271972520202\\
5.54	0.00201271972186427\\
5.55	0.00201271971852442\\
5.56	0.00201271971518248\\
5.57	0.00201271971183844\\
5.58	0.00201271970849229\\
5.59	0.00201271970514404\\
5.6	0.00201271970179368\\
5.61	0.00201271969844121\\
5.62	0.00201271969508663\\
5.63	0.00201271969172994\\
5.64	0.00201271968837114\\
5.65	0.00201271968501022\\
5.66	0.00201271968164718\\
5.67	0.00201271967828202\\
5.68	0.00201271967491474\\
5.69	0.00201271967154533\\
5.7	0.0020127196681738\\
5.71	0.00201271966480015\\
5.72	0.00201271966142436\\
5.73	0.00201271965804644\\
5.74	0.00201271965466639\\
5.75	0.0020127196512842\\
5.76	0.00201271964789988\\
5.77	0.00201271964451341\\
5.78	0.00201271964112481\\
5.79	0.00201271963773406\\
5.8	0.00201271963434117\\
5.81	0.00201271963094613\\
5.82	0.00201271962754894\\
5.83	0.0020127196241496\\
5.84	0.00201271962074811\\
5.85	0.00201271961734446\\
5.86	0.00201271961393866\\
5.87	0.0020127196105307\\
5.88	0.00201271960712058\\
5.89	0.00201271960370829\\
5.9	0.00201271960029384\\
5.91	0.00201271959687723\\
5.92	0.00201271959345844\\
5.93	0.00201271959003749\\
5.94	0.00201271958661436\\
5.95	0.00201271958318906\\
5.96	0.00201271957976158\\
5.97	0.00201271957633193\\
5.98	0.00201271957290009\\
5.99	0.00201271956946608\\
6	0.00201271956602987\\
6.01	0.00201271956259149\\
6.02	0.00201271955915091\\
6.03	0.00201271955570814\\
6.04	0.00201271955226319\\
6.05	0.00201271954881604\\
6.06	0.00201271954536669\\
6.07	0.00201271954191514\\
6.08	0.00201271953846139\\
6.09	0.00201271953500545\\
6.1	0.00201271953154729\\
6.11	0.00201271952808694\\
6.12	0.00201271952462437\\
6.13	0.00201271952115959\\
6.14	0.0020127195176926\\
6.15	0.0020127195142234\\
6.16	0.00201271951075198\\
6.17	0.00201271950727835\\
6.18	0.00201271950380249\\
6.19	0.00201271950032441\\
6.2	0.0020127194968441\\
6.21	0.00201271949336157\\
6.22	0.00201271948987681\\
6.23	0.00201271948638983\\
6.24	0.0020127194829006\\
6.25	0.00201271947940915\\
6.26	0.00201271947591546\\
6.27	0.00201271947241952\\
6.28	0.00201271946892135\\
6.29	0.00201271946542094\\
6.3	0.00201271946191828\\
6.31	0.00201271945841337\\
6.32	0.00201271945490622\\
6.33	0.00201271945139682\\
6.34	0.00201271944788516\\
6.35	0.00201271944437125\\
6.36	0.00201271944085508\\
6.37	0.00201271943733665\\
6.38	0.00201271943381596\\
6.39	0.002012719430293\\
6.4	0.00201271942676778\\
6.41	0.0020127194232403\\
6.42	0.00201271941971054\\
6.43	0.00201271941617851\\
6.44	0.00201271941264421\\
6.45	0.00201271940910763\\
6.46	0.00201271940556878\\
6.47	0.00201271940202765\\
6.48	0.00201271939848423\\
6.49	0.00201271939493853\\
6.5	0.00201271939139054\\
6.51	0.00201271938784026\\
6.52	0.0020127193842877\\
6.53	0.00201271938073284\\
6.54	0.00201271937717569\\
6.55	0.00201271937361624\\
6.56	0.00201271937005449\\
6.57	0.00201271936649044\\
6.58	0.00201271936292408\\
6.59	0.00201271935935543\\
6.6	0.00201271935578446\\
6.61	0.00201271935221119\\
6.62	0.0020127193486356\\
6.63	0.0020127193450577\\
6.64	0.00201271934147748\\
6.65	0.00201271933789494\\
6.66	0.00201271933431009\\
6.67	0.00201271933072291\\
6.68	0.00201271932713341\\
6.69	0.00201271932354158\\
6.7	0.00201271931994742\\
6.71	0.00201271931635093\\
6.72	0.00201271931275211\\
6.73	0.00201271930915096\\
6.74	0.00201271930554746\\
6.75	0.00201271930194163\\
6.76	0.00201271929833345\\
6.77	0.00201271929472293\\
6.78	0.00201271929111006\\
6.79	0.00201271928749485\\
6.8	0.00201271928387729\\
6.81	0.00201271928025737\\
6.82	0.0020127192766351\\
6.83	0.00201271927301047\\
6.84	0.00201271926938348\\
6.85	0.00201271926575413\\
6.86	0.00201271926212242\\
6.87	0.00201271925848833\\
6.88	0.00201271925485189\\
6.89	0.00201271925121307\\
6.9	0.00201271924757188\\
6.91	0.00201271924392831\\
6.92	0.00201271924028237\\
6.93	0.00201271923663405\\
6.94	0.00201271923298335\\
6.95	0.00201271922933027\\
6.96	0.00201271922567479\\
6.97	0.00201271922201694\\
6.98	0.00201271921835669\\
6.99	0.00201271921469405\\
7	0.00201271921102901\\
7.01	0.00201271920736158\\
7.02	0.00201271920369175\\
7.03	0.00201271920001951\\
7.04	0.00201271919634488\\
7.05	0.00201271919266783\\
7.06	0.00201271918898838\\
7.07	0.00201271918530652\\
7.08	0.00201271918162225\\
7.09	0.00201271917793556\\
7.1	0.00201271917424646\\
7.11	0.00201271917055493\\
7.12	0.00201271916686098\\
7.13	0.00201271916316462\\
7.14	0.00201271915946582\\
7.15	0.00201271915576459\\
7.16	0.00201271915206094\\
7.17	0.00201271914835485\\
7.18	0.00201271914464633\\
7.19	0.00201271914093537\\
7.2	0.00201271913722197\\
7.21	0.00201271913350613\\
7.22	0.00201271912978784\\
7.23	0.00201271912606711\\
7.24	0.00201271912234392\\
7.25	0.00201271911861829\\
7.26	0.0020127191148902\\
7.27	0.00201271911115966\\
7.28	0.00201271910742666\\
7.29	0.0020127191036912\\
7.3	0.00201271909995328\\
7.31	0.00201271909621289\\
7.32	0.00201271909247003\\
7.33	0.00201271908872471\\
7.34	0.00201271908497691\\
7.35	0.00201271908122664\\
7.36	0.00201271907747389\\
7.37	0.00201271907371866\\
7.38	0.00201271906996095\\
7.39	0.00201271906620076\\
7.4	0.00201271906243808\\
7.41	0.00201271905867291\\
7.42	0.00201271905490526\\
7.43	0.0020127190511351\\
7.44	0.00201271904736246\\
7.45	0.00201271904358732\\
7.46	0.00201271903980967\\
7.47	0.00201271903602953\\
7.48	0.00201271903224687\\
7.49	0.00201271902846172\\
7.5	0.00201271902467405\\
7.51	0.00201271902088387\\
7.52	0.00201271901709118\\
7.53	0.00201271901329597\\
7.54	0.00201271900949824\\
7.55	0.00201271900569799\\
7.56	0.00201271900189521\\
7.57	0.00201271899808991\\
7.58	0.00201271899428209\\
7.59	0.00201271899047173\\
7.6	0.00201271898665883\\
7.61	0.0020127189828434\\
7.62	0.00201271897902544\\
7.63	0.00201271897520493\\
7.64	0.00201271897138188\\
7.65	0.00201271896755629\\
7.66	0.00201271896372814\\
7.67	0.00201271895989745\\
7.68	0.0020127189560642\\
7.69	0.0020127189522284\\
7.7	0.00201271894839004\\
7.71	0.00201271894454913\\
7.72	0.00201271894070565\\
7.73	0.0020127189368596\\
7.74	0.00201271893301099\\
7.75	0.00201271892915981\\
7.76	0.00201271892530606\\
7.77	0.00201271892144973\\
7.78	0.00201271891759082\\
7.79	0.00201271891372934\\
7.8	0.00201271890986528\\
7.81	0.00201271890599863\\
7.82	0.00201271890212939\\
7.83	0.00201271889825757\\
7.84	0.00201271889438315\\
7.85	0.00201271889050614\\
7.86	0.00201271888662653\\
7.87	0.00201271888274433\\
7.88	0.00201271887885952\\
7.89	0.00201271887497211\\
7.9	0.00201271887108209\\
7.91	0.00201271886718946\\
7.92	0.00201271886329422\\
7.93	0.00201271885939637\\
7.94	0.00201271885549591\\
7.95	0.00201271885159282\\
7.96	0.00201271884768711\\
7.97	0.00201271884377878\\
7.98	0.00201271883986782\\
7.99	0.00201271883595423\\
8	0.00201271883203802\\
8.01	0.00201271882811916\\
8.02	0.00201271882419768\\
8.03	0.00201271882027355\\
8.04	0.00201271881634678\\
8.05	0.00201271881241737\\
8.06	0.00201271880848531\\
8.07	0.0020127188045506\\
8.08	0.00201271880061324\\
8.09	0.00201271879667323\\
8.1	0.00201271879273056\\
8.11	0.00201271878878523\\
8.12	0.00201271878483724\\
8.13	0.00201271878088659\\
8.14	0.00201271877693326\\
8.15	0.00201271877297727\\
8.16	0.00201271876901861\\
8.17	0.00201271876505728\\
8.18	0.00201271876109326\\
8.19	0.00201271875712657\\
8.2	0.00201271875315719\\
8.21	0.00201271874918514\\
8.22	0.00201271874521039\\
8.23	0.00201271874123295\\
8.24	0.00201271873725282\\
8.25	0.00201271873327\\
8.26	0.00201271872928448\\
8.27	0.00201271872529626\\
8.28	0.00201271872130534\\
8.29	0.00201271871731171\\
8.3	0.00201271871331537\\
8.31	0.00201271870931632\\
8.32	0.00201271870531456\\
8.33	0.00201271870131009\\
8.34	0.00201271869730289\\
8.35	0.00201271869329298\\
8.36	0.00201271868928034\\
8.37	0.00201271868526498\\
8.38	0.00201271868124688\\
8.39	0.00201271867722606\\
8.4	0.0020127186732025\\
8.41	0.0020127186691762\\
8.42	0.00201271866514717\\
8.43	0.00201271866111539\\
8.44	0.00201271865708087\\
8.45	0.00201271865304361\\
8.46	0.00201271864900359\\
8.47	0.00201271864496082\\
8.48	0.00201271864091529\\
8.49	0.00201271863686701\\
8.5	0.00201271863281597\\
8.51	0.00201271862876216\\
8.52	0.00201271862470559\\
8.53	0.00201271862064625\\
8.54	0.00201271861658414\\
8.55	0.00201271861251926\\
8.56	0.0020127186084516\\
8.57	0.00201271860438116\\
8.58	0.00201271860030794\\
8.59	0.00201271859623193\\
8.6	0.00201271859215314\\
8.61	0.00201271858807155\\
8.62	0.00201271858398718\\
8.63	0.00201271857990001\\
8.64	0.00201271857581004\\
8.65	0.00201271857171727\\
8.66	0.0020127185676217\\
8.67	0.00201271856352332\\
8.68	0.00201271855942214\\
8.69	0.00201271855531814\\
8.7	0.00201271855121132\\
8.71	0.0020127185471017\\
8.72	0.00201271854298925\\
8.73	0.00201271853887398\\
8.74	0.00201271853475588\\
8.75	0.00201271853063496\\
8.76	0.0020127185265112\\
8.77	0.00201271852238462\\
8.78	0.00201271851825519\\
8.79	0.00201271851412293\\
8.8	0.00201271850998783\\
8.81	0.00201271850584988\\
8.82	0.00201271850170909\\
8.83	0.00201271849756545\\
8.84	0.00201271849341895\\
8.85	0.0020127184892696\\
8.86	0.00201271848511739\\
8.87	0.00201271848096232\\
8.88	0.00201271847680438\\
8.89	0.00201271847264359\\
8.9	0.00201271846847992\\
8.91	0.00201271846431337\\
8.92	0.00201271846014396\\
8.93	0.00201271845597167\\
8.94	0.00201271845179649\\
8.95	0.00201271844761844\\
8.96	0.0020127184434375\\
8.97	0.00201271843925367\\
8.98	0.00201271843506694\\
8.99	0.00201271843087733\\
9	0.00201271842668482\\
9.01	0.0020127184224894\\
9.02	0.00201271841829109\\
9.03	0.00201271841408987\\
9.04	0.00201271840988574\\
9.05	0.0020127184056787\\
9.06	0.00201271840146874\\
9.07	0.00201271839725587\\
9.08	0.00201271839304008\\
9.09	0.00201271838882137\\
9.1	0.00201271838459973\\
9.11	0.00201271838037517\\
9.12	0.00201271837614767\\
9.13	0.00201271837191724\\
9.14	0.00201271836768387\\
9.15	0.00201271836344756\\
9.16	0.00201271835920831\\
9.17	0.00201271835496612\\
9.18	0.00201271835072097\\
9.19	0.00201271834647288\\
9.2	0.00201271834222183\\
9.21	0.00201271833796783\\
9.22	0.00201271833371086\\
9.23	0.00201271832945094\\
9.24	0.00201271832518804\\
9.25	0.00201271832092218\\
9.26	0.00201271831665335\\
9.27	0.00201271831238154\\
9.28	0.00201271830810676\\
9.29	0.002012718303829\\
9.3	0.00201271829954825\\
9.31	0.00201271829526452\\
9.32	0.0020127182909778\\
9.33	0.00201271828668809\\
9.34	0.00201271828239538\\
9.35	0.00201271827809968\\
9.36	0.00201271827380098\\
9.37	0.00201271826949927\\
9.38	0.00201271826519456\\
9.39	0.00201271826088684\\
9.4	0.0020127182565761\\
9.41	0.00201271825226236\\
9.42	0.00201271824794559\\
9.43	0.0020127182436258\\
9.44	0.00201271823930299\\
9.45	0.00201271823497715\\
9.46	0.00201271823064828\\
9.47	0.00201271822631638\\
9.48	0.00201271822198145\\
9.49	0.00201271821764347\\
9.5	0.00201271821330245\\
9.51	0.00201271820895839\\
9.52	0.00201271820461128\\
9.53	0.00201271820026113\\
9.54	0.00201271819590791\\
9.55	0.00201271819155164\\
9.56	0.00201271818719232\\
9.57	0.00201271818282992\\
9.58	0.00201271817846447\\
9.59	0.00201271817409594\\
9.6	0.00201271816972435\\
9.61	0.00201271816534968\\
9.62	0.00201271816097193\\
9.63	0.0020127181565911\\
9.64	0.00201271815220719\\
9.65	0.00201271814782019\\
9.66	0.0020127181434301\\
9.67	0.00201271813903692\\
9.68	0.00201271813464064\\
9.69	0.00201271813024126\\
9.7	0.00201271812583879\\
9.71	0.00201271812143321\\
9.72	0.00201271811702452\\
9.73	0.00201271811261272\\
9.74	0.0020127181081978\\
9.75	0.00201271810377977\\
9.76	0.00201271809935862\\
9.77	0.00201271809493434\\
9.78	0.00201271809050694\\
9.79	0.00201271808607641\\
9.8	0.00201271808164275\\
9.81	0.00201271807720596\\
9.82	0.00201271807276602\\
9.83	0.00201271806832294\\
9.84	0.00201271806387672\\
9.85	0.00201271805942735\\
9.86	0.00201271805497483\\
9.87	0.00201271805051916\\
9.88	0.00201271804606032\\
9.89	0.00201271804159833\\
9.9	0.00201271803713318\\
9.91	0.00201271803266486\\
9.92	0.00201271802819337\\
9.93	0.00201271802371871\\
9.94	0.00201271801924087\\
9.95	0.00201271801475985\\
9.96	0.00201271801027565\\
9.97	0.00201271800578827\\
9.98	0.0020127180012977\\
9.99	0.00201271799680394\\
10	0.00201271799230698\\
10.01	0.00201271798780682\\
10.02	0.00201271798330347\\
10.03	0.00201271797879691\\
10.04	0.00201271797428714\\
10.05	0.00201271796977417\\
10.06	0.00201271796525798\\
10.07	0.00201271796073857\\
10.08	0.00201271795621595\\
10.09	0.0020127179516901\\
10.1	0.00201271794716102\\
10.11	0.00201271794262872\\
10.12	0.00201271793809319\\
10.13	0.00201271793355441\\
10.14	0.0020127179290124\\
10.15	0.00201271792446715\\
10.16	0.00201271791991866\\
10.17	0.00201271791536691\\
10.18	0.00201271791081191\\
10.19	0.00201271790625366\\
10.2	0.00201271790169215\\
10.21	0.00201271789712739\\
10.22	0.00201271789255935\\
10.23	0.00201271788798805\\
10.24	0.00201271788341348\\
10.25	0.00201271787883563\\
10.26	0.00201271787425451\\
10.27	0.0020127178696701\\
10.28	0.00201271786508241\\
10.29	0.00201271786049144\\
10.3	0.00201271785589717\\
10.31	0.00201271785129962\\
10.32	0.00201271784669876\\
10.33	0.00201271784209461\\
10.34	0.00201271783748715\\
10.35	0.00201271783287639\\
10.36	0.00201271782826232\\
10.37	0.00201271782364493\\
10.38	0.00201271781902423\\
10.39	0.0020127178144002\\
10.4	0.00201271780977286\\
10.41	0.00201271780514219\\
10.42	0.00201271780050819\\
10.43	0.00201271779587086\\
10.44	0.00201271779123019\\
10.45	0.00201271778658618\\
10.46	0.00201271778193883\\
10.47	0.00201271777728814\\
10.48	0.00201271777263409\\
10.49	0.0020127177679767\\
10.5	0.00201271776331594\\
10.51	0.00201271775865183\\
10.52	0.00201271775398436\\
10.53	0.00201271774931352\\
10.54	0.00201271774463931\\
10.55	0.00201271773996173\\
10.56	0.00201271773528077\\
10.57	0.00201271773059644\\
10.58	0.00201271772590872\\
10.59	0.00201271772121762\\
10.6	0.00201271771652313\\
10.61	0.00201271771182524\\
10.62	0.00201271770712396\\
10.63	0.00201271770241928\\
10.64	0.0020127176977112\\
10.65	0.00201271769299972\\
10.66	0.00201271768828482\\
10.67	0.00201271768356651\\
10.68	0.00201271767884479\\
10.69	0.00201271767411965\\
10.7	0.00201271766939108\\
10.71	0.00201271766465909\\
10.72	0.00201271765992367\\
10.73	0.00201271765518482\\
10.74	0.00201271765044253\\
10.75	0.0020127176456968\\
10.76	0.00201271764094762\\
10.77	0.00201271763619501\\
10.78	0.00201271763143894\\
10.79	0.00201271762667942\\
10.8	0.00201271762191644\\
10.81	0.00201271761715\\
10.82	0.0020127176123801\\
10.83	0.00201271760760673\\
10.84	0.00201271760282989\\
10.85	0.00201271759804958\\
10.86	0.00201271759326579\\
10.87	0.00201271758847852\\
10.88	0.00201271758368777\\
10.89	0.00201271757889353\\
10.9	0.00201271757409579\\
10.91	0.00201271756929457\\
10.92	0.00201271756448984\\
10.93	0.00201271755968162\\
10.94	0.00201271755486989\\
10.95	0.00201271755005465\\
10.96	0.0020127175452359\\
10.97	0.00201271754041363\\
10.98	0.00201271753558785\\
10.99	0.00201271753075854\\
11	0.00201271752592571\\
11.01	0.00201271752108935\\
11.02	0.00201271751624946\\
11.03	0.00201271751140603\\
11.04	0.00201271750655906\\
11.05	0.00201271750170855\\
11.06	0.00201271749685449\\
11.07	0.00201271749199688\\
11.08	0.00201271748713572\\
11.09	0.002012717482271\\
11.1	0.00201271747740272\\
11.11	0.00201271747253088\\
11.12	0.00201271746765547\\
11.13	0.00201271746277648\\
11.14	0.00201271745789392\\
11.15	0.00201271745300779\\
11.16	0.00201271744811807\\
11.17	0.00201271744322477\\
11.18	0.00201271743832788\\
11.19	0.00201271743342739\\
11.2	0.00201271742852331\\
11.21	0.00201271742361563\\
11.22	0.00201271741870435\\
11.23	0.00201271741378946\\
11.24	0.00201271740887096\\
11.25	0.00201271740394885\\
11.26	0.00201271739902311\\
11.27	0.00201271739409376\\
11.28	0.00201271738916078\\
11.29	0.00201271738422418\\
11.3	0.00201271737928394\\
11.31	0.00201271737434007\\
11.32	0.00201271736939256\\
11.33	0.0020127173644414\\
11.34	0.0020127173594866\\
11.35	0.00201271735452816\\
11.36	0.00201271734956605\\
11.37	0.00201271734460029\\
11.38	0.00201271733963087\\
11.39	0.00201271733465779\\
11.4	0.00201271732968103\\
11.41	0.00201271732470061\\
11.42	0.00201271731971651\\
11.43	0.00201271731472873\\
11.44	0.00201271730973727\\
11.45	0.00201271730474212\\
11.46	0.00201271729974329\\
11.47	0.00201271729474076\\
11.48	0.00201271728973453\\
11.49	0.0020127172847246\\
11.5	0.00201271727971097\\
11.51	0.00201271727469363\\
11.52	0.00201271726967258\\
11.53	0.00201271726464781\\
11.54	0.00201271725961932\\
11.55	0.00201271725458712\\
11.56	0.00201271724955118\\
11.57	0.00201271724451152\\
11.58	0.00201271723946812\\
11.59	0.00201271723442098\\
11.6	0.0020127172293701\\
11.61	0.00201271722431548\\
11.62	0.00201271721925711\\
11.63	0.00201271721419499\\
11.64	0.00201271720912911\\
11.65	0.00201271720405947\\
11.66	0.00201271719898607\\
11.67	0.0020127171939089\\
11.68	0.00201271718882796\\
11.69	0.00201271718374324\\
11.7	0.00201271717865475\\
11.71	0.00201271717356247\\
11.72	0.00201271716846641\\
11.73	0.00201271716336656\\
11.74	0.00201271715826292\\
11.75	0.00201271715315548\\
11.76	0.00201271714804423\\
11.77	0.00201271714292919\\
11.78	0.00201271713781033\\
11.79	0.00201271713268767\\
11.8	0.00201271712756118\\
11.81	0.00201271712243088\\
11.82	0.00201271711729675\\
11.83	0.0020127171121588\\
11.84	0.00201271710701702\\
11.85	0.0020127171018714\\
11.86	0.00201271709672194\\
11.87	0.00201271709156864\\
11.88	0.0020127170864115\\
11.89	0.0020127170812505\\
11.9	0.00201271707608565\\
11.91	0.00201271707091695\\
11.92	0.00201271706574438\\
11.93	0.00201271706056795\\
11.94	0.00201271705538764\\
11.95	0.00201271705020347\\
11.96	0.00201271704501542\\
11.97	0.00201271703982349\\
11.98	0.00201271703462767\\
11.99	0.00201271702942796\\
12	0.00201271702422437\\
12.01	0.00201271701901688\\
12.02	0.00201271701380548\\
12.03	0.00201271700859019\\
12.04	0.00201271700337099\\
12.05	0.00201271699814787\\
12.06	0.00201271699292084\\
12.07	0.00201271698768989\\
12.08	0.00201271698245502\\
12.09	0.00201271697721623\\
12.1	0.0020127169719735\\
12.11	0.00201271696672684\\
12.12	0.00201271696147623\\
12.13	0.00201271695622169\\
12.14	0.0020127169509632\\
12.15	0.00201271694570076\\
12.16	0.00201271694043437\\
12.17	0.00201271693516402\\
12.18	0.00201271692988971\\
12.19	0.00201271692461143\\
12.2	0.00201271691932918\\
12.21	0.00201271691404296\\
12.22	0.00201271690875277\\
12.23	0.00201271690345859\\
12.24	0.00201271689816042\\
12.25	0.00201271689285827\\
12.26	0.00201271688755212\\
12.27	0.00201271688224198\\
12.28	0.00201271687692784\\
12.29	0.00201271687160968\\
12.3	0.00201271686628753\\
12.31	0.00201271686096136\\
12.32	0.00201271685563117\\
12.33	0.00201271685029696\\
12.34	0.00201271684495873\\
12.35	0.00201271683961647\\
12.36	0.00201271683427017\\
12.37	0.00201271682891985\\
12.38	0.00201271682356548\\
12.39	0.00201271681820706\\
12.4	0.0020127168128446\\
12.41	0.00201271680747808\\
12.42	0.00201271680210751\\
12.43	0.00201271679673288\\
12.44	0.00201271679135418\\
12.45	0.00201271678597142\\
12.46	0.00201271678058458\\
12.47	0.00201271677519367\\
12.48	0.00201271676979867\\
12.49	0.00201271676439959\\
12.5	0.00201271675899643\\
12.51	0.00201271675358917\\
12.52	0.00201271674817781\\
12.53	0.00201271674276235\\
12.54	0.00201271673734279\\
12.55	0.00201271673191913\\
12.56	0.00201271672649134\\
12.57	0.00201271672105945\\
12.58	0.00201271671562343\\
12.59	0.00201271671018328\\
12.6	0.00201271670473901\\
12.61	0.00201271669929061\\
12.62	0.00201271669383806\\
12.63	0.00201271668838138\\
12.64	0.00201271668292055\\
12.65	0.00201271667745558\\
12.66	0.00201271667198645\\
12.67	0.00201271666651316\\
12.68	0.00201271666103571\\
12.69	0.0020127166555541\\
12.7	0.00201271665006832\\
12.71	0.00201271664457836\\
12.72	0.00201271663908423\\
12.73	0.00201271663358591\\
12.74	0.00201271662808341\\
12.75	0.00201271662257672\\
12.76	0.00201271661706584\\
12.77	0.00201271661155075\\
12.78	0.00201271660603147\\
12.79	0.00201271660050798\\
12.8	0.00201271659498028\\
12.81	0.00201271658944836\\
12.82	0.00201271658391223\\
12.83	0.00201271657837187\\
12.84	0.00201271657282729\\
12.85	0.00201271656727847\\
12.86	0.00201271656172542\\
12.87	0.00201271655616813\\
12.88	0.00201271655060659\\
12.89	0.00201271654504081\\
12.9	0.00201271653947078\\
12.91	0.00201271653389649\\
12.92	0.00201271652831794\\
12.93	0.00201271652273513\\
12.94	0.00201271651714805\\
12.95	0.0020127165115567\\
12.96	0.00201271650596107\\
12.97	0.00201271650036116\\
12.98	0.00201271649475696\\
12.99	0.00201271648914847\\
13	0.0020127164835357\\
13.01	0.00201271647791862\\
13.02	0.00201271647229724\\
13.03	0.00201271646667156\\
13.04	0.00201271646104157\\
13.05	0.00201271645540726\\
13.06	0.00201271644976863\\
13.07	0.00201271644412568\\
13.08	0.0020127164384784\\
13.09	0.00201271643282679\\
13.1	0.00201271642717085\\
13.11	0.00201271642151057\\
13.12	0.00201271641584594\\
13.13	0.00201271641017697\\
13.14	0.00201271640450364\\
13.15	0.00201271639882595\\
13.16	0.00201271639314391\\
13.17	0.0020127163874575\\
13.18	0.00201271638176672\\
13.19	0.00201271637607157\\
13.2	0.00201271637037204\\
13.21	0.00201271636466813\\
13.22	0.00201271635895983\\
13.23	0.00201271635324714\\
13.24	0.00201271634753006\\
13.25	0.00201271634180857\\
13.26	0.00201271633608269\\
13.27	0.00201271633035239\\
13.28	0.00201271632461769\\
13.29	0.00201271631887857\\
13.3	0.00201271631313502\\
13.31	0.00201271630738706\\
13.32	0.00201271630163466\\
13.33	0.00201271629587784\\
13.34	0.00201271629011657\\
13.35	0.00201271628435086\\
13.36	0.00201271627858071\\
13.37	0.0020127162728061\\
13.38	0.00201271626702704\\
13.39	0.00201271626124352\\
13.4	0.00201271625545554\\
13.41	0.00201271624966309\\
13.42	0.00201271624386617\\
13.43	0.00201271623806477\\
13.44	0.00201271623225889\\
13.45	0.00201271622644853\\
13.46	0.00201271622063368\\
13.47	0.00201271621481433\\
13.48	0.00201271620899049\\
13.49	0.00201271620316214\\
13.5	0.00201271619732929\\
13.51	0.00201271619149192\\
13.52	0.00201271618565004\\
13.53	0.00201271617980364\\
13.54	0.00201271617395272\\
13.55	0.00201271616809727\\
13.56	0.00201271616223728\\
13.57	0.00201271615637276\\
13.58	0.0020127161505037\\
13.59	0.00201271614463009\\
13.6	0.00201271613875193\\
13.61	0.00201271613286922\\
13.62	0.00201271612698195\\
13.63	0.00201271612109011\\
13.64	0.0020127161151937\\
13.65	0.00201271610929273\\
13.66	0.00201271610338718\\
13.67	0.00201271609747704\\
13.68	0.00201271609156232\\
13.69	0.00201271608564301\\
13.7	0.00201271607971911\\
13.71	0.00201271607379061\\
13.72	0.00201271606785751\\
13.73	0.0020127160619198\\
13.74	0.00201271605597747\\
13.75	0.00201271605003053\\
13.76	0.00201271604407897\\
13.77	0.00201271603812279\\
13.78	0.00201271603216197\\
13.79	0.00201271602619652\\
13.8	0.00201271602022644\\
13.81	0.00201271601425171\\
13.82	0.00201271600827233\\
13.83	0.0020127160022883\\
13.84	0.00201271599629962\\
13.85	0.00201271599030627\\
13.86	0.00201271598430826\\
13.87	0.00201271597830558\\
13.88	0.00201271597229823\\
13.89	0.0020127159662862\\
13.9	0.00201271596026948\\
13.91	0.00201271595424808\\
13.92	0.00201271594822199\\
13.93	0.0020127159421912\\
13.94	0.00201271593615571\\
13.95	0.00201271593011551\\
13.96	0.0020127159240706\\
13.97	0.00201271591802099\\
13.98	0.00201271591196665\\
13.99	0.00201271590590759\\
14	0.0020127158998438\\
14.01	0.00201271589377528\\
14.02	0.00201271588770202\\
14.03	0.00201271588162402\\
14.04	0.00201271587554128\\
14.05	0.00201271586945379\\
14.06	0.00201271586336154\\
14.07	0.00201271585726453\\
14.08	0.00201271585116276\\
14.09	0.00201271584505622\\
14.1	0.00201271583894491\\
14.11	0.00201271583282882\\
14.12	0.00201271582670795\\
14.13	0.00201271582058229\\
14.14	0.00201271581445185\\
14.15	0.0020127158083166\\
14.16	0.00201271580217656\\
14.17	0.00201271579603172\\
14.18	0.00201271578988206\\
14.19	0.00201271578372759\\
14.2	0.0020127157775683\\
14.21	0.0020127157714042\\
14.22	0.00201271576523526\\
14.23	0.00201271575906149\\
14.24	0.00201271575288289\\
14.25	0.00201271574669945\\
14.26	0.00201271574051116\\
14.27	0.00201271573431802\\
14.28	0.00201271572812002\\
14.29	0.00201271572191717\\
14.3	0.00201271571570946\\
14.31	0.00201271570949687\\
14.32	0.00201271570327942\\
14.33	0.00201271569705708\\
14.34	0.00201271569082987\\
14.35	0.00201271568459776\\
14.36	0.00201271567836077\\
14.37	0.00201271567211888\\
14.38	0.0020127156658721\\
14.39	0.0020127156596204\\
14.4	0.0020127156533638\\
14.41	0.00201271564710228\\
14.42	0.00201271564083585\\
14.43	0.00201271563456449\\
14.44	0.00201271562828821\\
14.45	0.00201271562200699\\
14.46	0.00201271561572084\\
14.47	0.00201271560942974\\
14.48	0.0020127156031337\\
14.49	0.0020127155968327\\
14.5	0.00201271559052676\\
14.51	0.00201271558421585\\
14.52	0.00201271557789997\\
14.53	0.00201271557157913\\
14.54	0.00201271556525332\\
14.55	0.00201271555892252\\
14.56	0.00201271555258675\\
14.57	0.00201271554624598\\
14.58	0.00201271553990022\\
14.59	0.00201271553354947\\
14.6	0.00201271552719371\\
14.61	0.00201271552083295\\
14.62	0.00201271551446718\\
14.63	0.00201271550809639\\
14.64	0.00201271550172058\\
14.65	0.00201271549533975\\
14.66	0.00201271548895388\\
14.67	0.00201271548256299\\
14.68	0.00201271547616705\\
14.69	0.00201271546976607\\
14.7	0.00201271546336004\\
14.71	0.00201271545694896\\
14.72	0.00201271545053283\\
14.73	0.00201271544411163\\
14.74	0.00201271543768536\\
14.75	0.00201271543125402\\
14.76	0.0020127154248176\\
14.77	0.00201271541837611\\
14.78	0.00201271541192953\\
14.79	0.00201271540547786\\
14.8	0.00201271539902109\\
14.81	0.00201271539255922\\
14.82	0.00201271538609226\\
14.83	0.00201271537962018\\
14.84	0.00201271537314299\\
14.85	0.00201271536666067\\
14.86	0.00201271536017324\\
14.87	0.00201271535368068\\
14.88	0.00201271534718298\\
14.89	0.00201271534068015\\
14.9	0.00201271533417218\\
14.91	0.00201271532765906\\
14.92	0.00201271532114079\\
14.93	0.00201271531461737\\
14.94	0.00201271530808878\\
14.95	0.00201271530155503\\
14.96	0.00201271529501611\\
14.97	0.00201271528847201\\
14.98	0.00201271528192273\\
14.99	0.00201271527536827\\
15	0.00201271526880862\\
15.01	0.00201271526224378\\
15.02	0.00201271525567374\\
15.03	0.00201271524909849\\
15.04	0.00201271524251804\\
15.05	0.00201271523593237\\
15.06	0.00201271522934149\\
15.07	0.00201271522274539\\
15.08	0.00201271521614406\\
15.09	0.00201271520953749\\
15.1	0.00201271520292569\\
15.11	0.00201271519630866\\
15.12	0.00201271518968637\\
15.13	0.00201271518305883\\
15.14	0.00201271517642604\\
15.15	0.00201271516978799\\
15.16	0.00201271516314467\\
15.17	0.00201271515649609\\
15.18	0.00201271514984223\\
15.19	0.00201271514318309\\
15.2	0.00201271513651867\\
15.21	0.00201271512984896\\
15.22	0.00201271512317396\\
15.23	0.00201271511649365\\
15.24	0.00201271510980805\\
15.25	0.00201271510311714\\
15.26	0.00201271509642091\\
15.27	0.00201271508971937\\
15.28	0.00201271508301251\\
15.29	0.00201271507630032\\
15.3	0.0020127150695828\\
15.31	0.00201271506285994\\
15.32	0.00201271505613175\\
15.33	0.0020127150493982\\
15.34	0.00201271504265931\\
15.35	0.00201271503591506\\
15.36	0.00201271502916545\\
15.37	0.00201271502241048\\
15.38	0.00201271501565014\\
15.39	0.00201271500888442\\
15.4	0.00201271500211332\\
15.41	0.00201271499533684\\
15.42	0.00201271498855497\\
15.43	0.00201271498176771\\
15.44	0.00201271497497505\\
15.45	0.00201271496817699\\
15.46	0.00201271496137351\\
15.47	0.00201271495456463\\
15.48	0.00201271494775033\\
15.49	0.0020127149409306\\
15.5	0.00201271493410545\\
15.51	0.00201271492727487\\
15.52	0.00201271492043885\\
15.53	0.00201271491359739\\
15.54	0.00201271490675049\\
15.55	0.00201271489989813\\
15.56	0.00201271489304032\\
15.57	0.00201271488617704\\
15.58	0.0020127148793083\\
15.59	0.00201271487243409\\
15.6	0.00201271486555441\\
15.61	0.00201271485866925\\
15.62	0.0020127148517786\\
15.63	0.00201271484488246\\
15.64	0.00201271483798083\\
15.65	0.00201271483107369\\
15.66	0.00201271482416106\\
15.67	0.00201271481724292\\
15.68	0.00201271481031926\\
15.69	0.00201271480339008\\
15.7	0.00201271479645538\\
15.71	0.00201271478951516\\
15.72	0.0020127147825694\\
15.73	0.0020127147756181\\
15.74	0.00201271476866126\\
15.75	0.00201271476169888\\
15.76	0.00201271475473094\\
15.77	0.00201271474775744\\
15.78	0.00201271474077839\\
15.79	0.00201271473379376\\
15.8	0.00201271472680357\\
15.81	0.0020127147198078\\
15.82	0.00201271471280645\\
15.83	0.00201271470579951\\
15.84	0.00201271469878698\\
15.85	0.00201271469176886\\
15.86	0.00201271468474514\\
15.87	0.00201271467771581\\
15.88	0.00201271467068087\\
15.89	0.00201271466364032\\
15.9	0.00201271465659414\\
15.91	0.00201271464954235\\
15.92	0.00201271464248492\\
15.93	0.00201271463542186\\
15.94	0.00201271462835316\\
15.95	0.00201271462127882\\
15.96	0.00201271461419883\\
15.97	0.00201271460711318\\
15.98	0.00201271460002188\\
15.99	0.00201271459292491\\
16	0.00201271458582228\\
16.01	0.00201271457871397\\
16.02	0.00201271457159999\\
16.03	0.00201271456448032\\
16.04	0.00201271455735497\\
16.05	0.00201271455022392\\
16.06	0.00201271454308718\\
16.07	0.00201271453594474\\
16.08	0.00201271452879659\\
16.09	0.00201271452164272\\
16.1	0.00201271451448315\\
16.11	0.00201271450731785\\
16.12	0.00201271450014682\\
16.13	0.00201271449297007\\
16.14	0.00201271448578757\\
16.15	0.00201271447859934\\
16.16	0.00201271447140537\\
16.17	0.00201271446420564\\
16.18	0.00201271445700016\\
16.19	0.00201271444978891\\
16.2	0.00201271444257191\\
16.21	0.00201271443534913\\
16.22	0.00201271442812058\\
16.23	0.00201271442088626\\
16.24	0.00201271441364614\\
16.25	0.00201271440640024\\
16.26	0.00201271439914854\\
16.27	0.00201271439189105\\
16.28	0.00201271438462775\\
16.29	0.00201271437735865\\
16.3	0.00201271437008373\\
16.31	0.002012714362803\\
16.32	0.00201271435551644\\
16.33	0.00201271434822405\\
16.34	0.00201271434092583\\
16.35	0.00201271433362178\\
16.36	0.00201271432631188\\
16.37	0.00201271431899614\\
16.38	0.00201271431167454\\
16.39	0.00201271430434709\\
16.4	0.00201271429701378\\
16.41	0.0020127142896746\\
16.42	0.00201271428232955\\
16.43	0.00201271427497862\\
16.44	0.00201271426762182\\
16.45	0.00201271426025912\\
16.46	0.00201271425289054\\
16.47	0.00201271424551606\\
16.48	0.00201271423813569\\
16.49	0.0020127142307494\\
16.5	0.00201271422335721\\
16.51	0.00201271421595911\\
16.52	0.00201271420855508\\
16.53	0.00201271420114513\\
16.54	0.00201271419372925\\
16.55	0.00201271418630744\\
16.56	0.00201271417887969\\
16.57	0.002012714171446\\
16.58	0.00201271416400635\\
16.59	0.00201271415656076\\
16.6	0.0020127141491092\\
16.61	0.00201271414165169\\
16.62	0.0020127141341882\\
16.63	0.00201271412671875\\
16.64	0.00201271411924332\\
16.65	0.0020127141117619\\
16.66	0.0020127141042745\\
16.67	0.00201271409678111\\
16.68	0.00201271408928172\\
16.69	0.00201271408177633\\
16.7	0.00201271407426493\\
16.71	0.00201271406674752\\
16.72	0.0020127140592241\\
16.73	0.00201271405169465\\
16.74	0.00201271404415919\\
16.75	0.00201271403661769\\
16.76	0.00201271402907015\\
16.77	0.00201271402151658\\
16.78	0.00201271401395696\\
16.79	0.00201271400639129\\
16.8	0.00201271399881957\\
16.81	0.00201271399124179\\
16.82	0.00201271398365794\\
16.83	0.00201271397606803\\
16.84	0.00201271396847205\\
16.85	0.00201271396086998\\
16.86	0.00201271395326184\\
16.87	0.0020127139456476\\
16.88	0.00201271393802727\\
16.89	0.00201271393040085\\
16.9	0.00201271392276832\\
16.91	0.00201271391512969\\
16.92	0.00201271390748494\\
16.93	0.00201271389983408\\
16.94	0.0020127138921771\\
16.95	0.00201271388451399\\
16.96	0.00201271387684475\\
16.97	0.00201271386916937\\
16.98	0.00201271386148785\\
16.99	0.00201271385380019\\
17	0.00201271384610638\\
17.01	0.00201271383840642\\
17.02	0.00201271383070029\\
17.03	0.002012713822988\\
17.04	0.00201271381526954\\
17.05	0.00201271380754491\\
17.06	0.0020127137998141\\
17.07	0.00201271379207711\\
17.08	0.00201271378433393\\
17.09	0.00201271377658455\\
17.1	0.00201271376882898\\
17.11	0.0020127137610672\\
17.12	0.00201271375329922\\
17.13	0.00201271374552503\\
17.14	0.00201271373774462\\
17.15	0.00201271372995799\\
17.16	0.00201271372216513\\
17.17	0.00201271371436604\\
17.18	0.00201271370656071\\
17.19	0.00201271369874915\\
17.2	0.00201271369093134\\
17.21	0.00201271368310728\\
17.22	0.00201271367527697\\
17.23	0.00201271366744039\\
17.24	0.00201271365959755\\
17.25	0.00201271365174845\\
17.26	0.00201271364389307\\
17.27	0.00201271363603141\\
17.28	0.00201271362816347\\
17.29	0.00201271362028924\\
17.3	0.00201271361240872\\
17.31	0.00201271360452191\\
17.32	0.00201271359662879\\
17.33	0.00201271358872936\\
17.34	0.00201271358082362\\
17.35	0.00201271357291157\\
17.36	0.0020127135649932\\
17.37	0.0020127135570685\\
17.38	0.00201271354913747\\
17.39	0.0020127135412001\\
17.4	0.0020127135332564\\
17.41	0.00201271352530636\\
17.42	0.00201271351734996\\
17.43	0.00201271350938721\\
17.44	0.0020127135014181\\
17.45	0.00201271349344263\\
17.46	0.00201271348546079\\
17.47	0.00201271347747258\\
17.48	0.002012713469478\\
17.49	0.00201271346147703\\
17.5	0.00201271345346967\\
17.51	0.00201271344545593\\
17.52	0.00201271343743579\\
17.53	0.00201271342940925\\
17.54	0.0020127134213763\\
17.55	0.00201271341333695\\
17.56	0.00201271340529118\\
17.57	0.00201271339723899\\
17.58	0.00201271338918038\\
17.59	0.00201271338111534\\
17.6	0.00201271337304386\\
17.61	0.00201271336496595\\
17.62	0.0020127133568816\\
17.63	0.00201271334879081\\
17.64	0.00201271334069356\\
17.65	0.00201271333258985\\
17.66	0.00201271332447969\\
17.67	0.00201271331636305\\
17.68	0.00201271330823996\\
17.69	0.00201271330011038\\
17.7	0.00201271329197433\\
17.71	0.00201271328383179\\
17.72	0.00201271327568277\\
17.73	0.00201271326752725\\
17.74	0.00201271325936523\\
17.75	0.00201271325119672\\
17.76	0.00201271324302169\\
17.77	0.00201271323484016\\
17.78	0.00201271322665211\\
17.79	0.00201271321845754\\
17.8	0.00201271321025645\\
17.81	0.00201271320204882\\
17.82	0.00201271319383466\\
17.83	0.00201271318561397\\
17.84	0.00201271317738673\\
17.85	0.00201271316915294\\
17.86	0.00201271316091261\\
17.87	0.00201271315266571\\
17.88	0.00201271314441225\\
17.89	0.00201271313615223\\
17.9	0.00201271312788564\\
17.91	0.00201271311961247\\
17.92	0.00201271311133273\\
17.93	0.0020127131030464\\
17.94	0.00201271309475348\\
17.95	0.00201271308645397\\
17.96	0.00201271307814786\\
17.97	0.00201271306983515\\
17.98	0.00201271306151584\\
17.99	0.00201271305318991\\
18	0.00201271304485737\\
18.01	0.0020127130365182\\
18.02	0.00201271302817242\\
18.03	0.00201271301982\\
18.04	0.00201271301146096\\
18.05	0.00201271300309527\\
18.06	0.00201271299472294\\
18.07	0.00201271298634397\\
18.08	0.00201271297795834\\
18.09	0.00201271296956606\\
18.1	0.00201271296116712\\
18.11	0.00201271295276152\\
18.12	0.00201271294434924\\
18.13	0.00201271293593029\\
18.14	0.00201271292750467\\
18.15	0.00201271291907236\\
18.16	0.00201271291063337\\
18.17	0.00201271290218769\\
18.18	0.00201271289373531\\
18.19	0.00201271288527623\\
18.2	0.00201271287681045\\
18.21	0.00201271286833795\\
18.22	0.00201271285985875\\
18.23	0.00201271285137283\\
18.24	0.00201271284288018\\
18.25	0.00201271283438081\\
18.26	0.00201271282587471\\
18.27	0.00201271281736188\\
18.28	0.00201271280884231\\
18.29	0.00201271280031599\\
18.3	0.00201271279178293\\
18.31	0.00201271278324311\\
18.32	0.00201271277469654\\
18.33	0.0020127127661432\\
18.34	0.0020127127575831\\
18.35	0.00201271274901624\\
18.36	0.00201271274044259\\
18.37	0.00201271273186217\\
18.38	0.00201271272327497\\
18.39	0.00201271271468098\\
18.4	0.0020127127060802\\
18.41	0.00201271269747263\\
18.42	0.00201271268885825\\
18.43	0.00201271268023708\\
18.44	0.00201271267160909\\
18.45	0.00201271266297429\\
18.46	0.00201271265433268\\
18.47	0.00201271264568424\\
18.48	0.00201271263702898\\
18.49	0.00201271262836689\\
18.5	0.00201271261969797\\
18.51	0.00201271261102221\\
18.52	0.00201271260233961\\
18.53	0.00201271259365016\\
18.54	0.00201271258495386\\
18.55	0.00201271257625071\\
18.56	0.0020127125675407\\
18.57	0.00201271255882382\\
18.58	0.00201271255010008\\
18.59	0.00201271254136947\\
18.6	0.00201271253263198\\
18.61	0.00201271252388761\\
18.62	0.00201271251513636\\
18.63	0.00201271250637823\\
18.64	0.0020127124976132\\
18.65	0.00201271248884127\\
18.66	0.00201271248006244\\
18.67	0.00201271247127671\\
18.68	0.00201271246248407\\
18.69	0.00201271245368452\\
18.7	0.00201271244487805\\
18.71	0.00201271243606467\\
18.72	0.00201271242724435\\
18.73	0.00201271241841711\\
18.74	0.00201271240958294\\
18.75	0.00201271240074182\\
18.76	0.00201271239189377\\
18.77	0.00201271238303878\\
18.78	0.00201271237417683\\
18.79	0.00201271236530793\\
18.8	0.00201271235643207\\
18.81	0.00201271234754926\\
18.82	0.00201271233865947\\
18.83	0.00201271232976272\\
18.84	0.002012712320859\\
18.85	0.0020127123119483\\
18.86	0.00201271230303062\\
18.87	0.00201271229410595\\
18.88	0.0020127122851743\\
18.89	0.00201271227623565\\
18.9	0.00201271226729\\
18.91	0.00201271225833736\\
18.92	0.00201271224937771\\
18.93	0.00201271224041105\\
18.94	0.00201271223143738\\
18.95	0.00201271222245669\\
18.96	0.00201271221346898\\
18.97	0.00201271220447425\\
18.98	0.0020127121954725\\
18.99	0.00201271218646371\\
19	0.00201271217744788\\
19.01	0.00201271216842502\\
19.02	0.00201271215939511\\
19.03	0.00201271215035815\\
19.04	0.00201271214131415\\
19.05	0.00201271213226308\\
19.06	0.00201271212320497\\
19.07	0.00201271211413979\\
19.08	0.00201271210506754\\
19.09	0.00201271209598822\\
19.1	0.00201271208690183\\
19.11	0.00201271207780837\\
19.12	0.00201271206870782\\
19.13	0.00201271205960019\\
19.14	0.00201271205048546\\
19.15	0.00201271204136365\\
19.16	0.00201271203223474\\
19.17	0.00201271202309874\\
19.18	0.00201271201395563\\
19.19	0.00201271200480541\\
19.2	0.00201271199564808\\
19.21	0.00201271198648364\\
19.22	0.00201271197731208\\
19.23	0.0020127119681334\\
19.24	0.0020127119589476\\
19.25	0.00201271194975467\\
19.26	0.0020127119405546\\
19.27	0.0020127119313474\\
19.28	0.00201271192213306\\
19.29	0.00201271191291158\\
19.3	0.00201271190368295\\
19.31	0.00201271189444718\\
19.32	0.00201271188520425\\
19.33	0.00201271187595416\\
19.34	0.00201271186669691\\
19.35	0.0020127118574325\\
19.36	0.00201271184816092\\
19.37	0.00201271183888218\\
19.38	0.00201271182959625\\
19.39	0.00201271182030316\\
19.4	0.00201271181100287\\
19.41	0.00201271180169541\\
19.42	0.00201271179238076\\
19.43	0.00201271178305892\\
19.44	0.00201271177372988\\
19.45	0.00201271176439365\\
19.46	0.00201271175505021\\
19.47	0.00201271174569957\\
19.48	0.00201271173634173\\
19.49	0.00201271172697667\\
19.5	0.0020127117176044\\
19.51	0.00201271170822491\\
19.52	0.0020127116988382\\
19.53	0.00201271168944427\\
19.54	0.00201271168004311\\
19.55	0.00201271167063472\\
19.56	0.0020127116612191\\
19.57	0.00201271165179624\\
19.58	0.00201271164236614\\
19.59	0.0020127116329288\\
19.6	0.00201271162348421\\
19.61	0.00201271161403237\\
19.62	0.00201271160457328\\
19.63	0.00201271159510694\\
19.64	0.00201271158563333\\
19.65	0.00201271157615247\\
19.66	0.00201271156666434\\
19.67	0.00201271155716895\\
19.68	0.00201271154766628\\
19.69	0.00201271153815634\\
19.7	0.00201271152863912\\
19.71	0.00201271151911463\\
19.72	0.00201271150958285\\
19.73	0.00201271150004379\\
19.74	0.00201271149049744\\
19.75	0.0020127114809438\\
19.76	0.00201271147138286\\
19.77	0.00201271146181463\\
19.78	0.0020127114522391\\
19.79	0.00201271144265626\\
19.8	0.00201271143306612\\
19.81	0.00201271142346868\\
19.82	0.00201271141386392\\
19.83	0.00201271140425185\\
19.84	0.00201271139463246\\
19.85	0.00201271138500575\\
19.86	0.00201271137537173\\
19.87	0.00201271136573037\\
19.88	0.00201271135608169\\
19.89	0.00201271134642569\\
19.9	0.00201271133676235\\
19.91	0.00201271132709167\\
19.92	0.00201271131741366\\
19.93	0.00201271130772831\\
19.94	0.00201271129803562\\
19.95	0.00201271128833558\\
19.96	0.00201271127862819\\
19.97	0.00201271126891345\\
19.98	0.00201271125919137\\
19.99	0.00201271124946193\\
20	0.00201271123972513\\
20.01	0.00201271122998097\\
20.02	0.00201271122022945\\
20.03	0.00201271121047056\\
20.04	0.00201271120070431\\
20.05	0.00201271119093069\\
20.06	0.0020127111811497\\
20.07	0.00201271117136134\\
20.08	0.0020127111615656\\
20.09	0.00201271115176249\\
20.1	0.00201271114195199\\
20.11	0.00201271113213411\\
20.12	0.00201271112230885\\
20.13	0.0020127111124762\\
20.14	0.00201271110263617\\
20.15	0.00201271109278874\\
20.16	0.00201271108293392\\
20.17	0.00201271107307171\\
20.18	0.0020127110632021\\
20.19	0.00201271105332509\\
20.2	0.00201271104344069\\
20.21	0.00201271103354888\\
20.22	0.00201271102364967\\
20.23	0.00201271101374305\\
20.24	0.00201271100382902\\
20.25	0.00201271099390759\\
20.26	0.00201271098397874\\
20.27	0.00201271097404248\\
20.28	0.00201271096409881\\
20.29	0.00201271095414772\\
20.3	0.00201271094418921\\
20.31	0.00201271093422328\\
20.32	0.00201271092424994\\
20.33	0.00201271091426917\\
20.34	0.00201271090428097\\
20.35	0.00201271089428535\\
20.36	0.0020127108842823\\
20.37	0.00201271087427183\\
20.38	0.00201271086425392\\
20.39	0.00201271085422859\\
20.4	0.00201271084419582\\
20.41	0.00201271083415561\\
20.42	0.00201271082410798\\
20.43	0.0020127108140529\\
20.44	0.00201271080399039\\
20.45	0.00201271079392044\\
20.46	0.00201271078384304\\
20.47	0.00201271077375821\\
20.48	0.00201271076366593\\
20.49	0.00201271075356622\\
20.5	0.00201271074345905\\
20.51	0.00201271073334444\\
20.52	0.00201271072322239\\
20.53	0.00201271071309289\\
20.54	0.00201271070295594\\
20.55	0.00201271069281154\\
20.56	0.00201271068265969\\
20.57	0.00201271067250039\\
20.58	0.00201271066233364\\
20.59	0.00201271065215944\\
20.6	0.00201271064197778\\
20.61	0.00201271063178867\\
20.62	0.00201271062159211\\
20.63	0.0020127106113881\\
20.64	0.00201271060117662\\
20.65	0.0020127105909577\\
20.66	0.00201271058073132\\
20.67	0.00201271057049748\\
20.68	0.00201271056025618\\
20.69	0.00201271055000743\\
20.7	0.00201271053975122\\
20.71	0.00201271052948756\\
20.72	0.00201271051921643\\
20.73	0.00201271050893785\\
20.74	0.00201271049865182\\
20.75	0.00201271048835832\\
20.76	0.00201271047805737\\
20.77	0.00201271046774896\\
20.78	0.00201271045743309\\
20.79	0.00201271044710976\\
20.8	0.00201271043677898\\
20.81	0.00201271042644074\\
20.82	0.00201271041609504\\
20.83	0.00201271040574189\\
20.84	0.00201271039538128\\
20.85	0.00201271038501321\\
20.86	0.00201271037463769\\
20.87	0.00201271036425472\\
20.88	0.00201271035386429\\
20.89	0.0020127103434664\\
20.9	0.00201271033306107\\
20.91	0.00201271032264828\\
20.92	0.00201271031222804\\
20.93	0.00201271030180035\\
20.94	0.00201271029136521\\
20.95	0.00201271028092262\\
20.96	0.00201271027047258\\
20.97	0.0020127102600151\\
20.98	0.00201271024955017\\
20.99	0.0020127102390778\\
21	0.00201271022859798\\
21.01	0.00201271021811072\\
21.02	0.00201271020761602\\
21.03	0.00201271019711388\\
21.04	0.0020127101866043\\
21.05	0.00201271017608728\\
21.06	0.00201271016556283\\
21.07	0.00201271015503094\\
21.08	0.00201271014449162\\
21.09	0.00201271013394488\\
21.1	0.0020127101233907\\
21.11	0.00201271011282909\\
21.12	0.00201271010226006\\
21.13	0.00201271009168361\\
21.14	0.00201271008109974\\
21.15	0.00201271007050845\\
21.16	0.00201271005990974\\
21.17	0.00201271004930361\\
21.18	0.00201271003869007\\
21.19	0.00201271002806912\\
21.2	0.00201271001744077\\
21.21	0.00201271000680501\\
21.22	0.00201270999616184\\
21.23	0.00201270998551128\\
21.24	0.00201270997485331\\
21.25	0.00201270996418795\\
21.26	0.0020127099535152\\
21.27	0.00201270994283506\\
21.28	0.00201270993214753\\
21.29	0.00201270992145262\\
21.3	0.00201270991075033\\
21.31	0.00201270990004066\\
21.32	0.00201270988932362\\
21.33	0.0020127098785992\\
21.34	0.00201270986786742\\
21.35	0.00201270985712827\\
21.36	0.00201270984638176\\
21.37	0.00201270983562789\\
21.38	0.00201270982486667\\
21.39	0.0020127098140981\\
21.4	0.00201270980332219\\
21.41	0.00201270979253893\\
21.42	0.00201270978174833\\
21.43	0.0020127097709504\\
21.44	0.00201270976014513\\
21.45	0.00201270974933254\\
21.46	0.00201270973851263\\
21.47	0.0020127097276854\\
21.48	0.00201270971685085\\
21.49	0.002012709706009\\
21.5	0.00201270969515984\\
21.51	0.00201270968430339\\
21.52	0.00201270967343963\\
21.53	0.00201270966256859\\
21.54	0.00201270965169026\\
21.55	0.00201270964080465\\
21.56	0.00201270962991177\\
21.57	0.00201270961901162\\
21.58	0.0020127096081042\\
21.59	0.00201270959718952\\
21.6	0.00201270958626758\\
21.61	0.0020127095753384\\
21.62	0.00201270956440198\\
21.63	0.00201270955345831\\
21.64	0.00201270954250742\\
21.65	0.0020127095315493\\
21.66	0.00201270952058396\\
21.67	0.0020127095096114\\
21.68	0.00201270949863164\\
21.69	0.00201270948764468\\
21.7	0.00201270947665052\\
21.71	0.00201270946564917\\
21.72	0.00201270945464064\\
21.73	0.00201270944362494\\
21.74	0.00201270943260206\\
21.75	0.00201270942157203\\
21.76	0.00201270941053483\\
21.77	0.0020127093994905\\
21.78	0.00201270938843902\\
21.79	0.0020127093773804\\
21.8	0.00201270936631466\\
21.81	0.0020127093552418\\
21.82	0.00201270934416183\\
21.83	0.00201270933307475\\
21.84	0.00201270932198058\\
21.85	0.00201270931087932\\
21.86	0.00201270929977098\\
21.87	0.00201270928865557\\
21.88	0.00201270927753309\\
21.89	0.00201270926640356\\
21.9	0.00201270925526697\\
21.91	0.00201270924412335\\
21.92	0.0020127092329727\\
21.93	0.00201270922181503\\
21.94	0.00201270921065034\\
21.95	0.00201270919947865\\
21.96	0.00201270918829997\\
21.97	0.00201270917711429\\
21.98	0.00201270916592164\\
21.99	0.00201270915472203\\
22	0.00201270914351545\\
22.01	0.00201270913230193\\
22.02	0.00201270912108147\\
22.03	0.00201270910985408\\
22.04	0.00201270909861977\\
22.05	0.00201270908737855\\
22.06	0.00201270907613043\\
22.07	0.00201270906487542\\
22.08	0.00201270905361354\\
22.09	0.00201270904234479\\
22.1	0.00201270903106918\\
22.11	0.00201270901978672\\
22.12	0.00201270900849743\\
22.13	0.00201270899720131\\
22.14	0.00201270898589839\\
22.15	0.00201270897458866\\
22.16	0.00201270896327213\\
22.17	0.00201270895194883\\
22.18	0.00201270894061877\\
22.19	0.00201270892928194\\
22.2	0.00201270891793837\\
22.21	0.00201270890658807\\
22.22	0.00201270889523105\\
22.23	0.00201270888386733\\
22.24	0.0020127088724969\\
22.25	0.0020127088611198\\
22.26	0.00201270884973602\\
22.27	0.00201270883834559\\
22.28	0.00201270882694851\\
22.29	0.0020127088155448\\
22.3	0.00201270880413447\\
22.31	0.00201270879271754\\
22.32	0.00201270878129401\\
22.33	0.00201270876986391\\
22.34	0.00201270875842724\\
22.35	0.00201270874698402\\
22.36	0.00201270873553427\\
22.37	0.00201270872407799\\
22.38	0.0020127087126152\\
22.39	0.00201270870114592\\
22.4	0.00201270868967015\\
22.41	0.00201270867818793\\
22.42	0.00201270866669925\\
22.43	0.00201270865520413\\
22.44	0.0020127086437026\\
22.45	0.00201270863219465\\
22.46	0.00201270862068032\\
22.47	0.00201270860915962\\
22.48	0.00201270859763255\\
22.49	0.00201270858609913\\
22.5	0.0020127085745594\\
22.51	0.00201270856301334\\
22.52	0.00201270855146099\\
22.53	0.00201270853990236\\
22.54	0.00201270852833747\\
22.55	0.00201270851676633\\
22.56	0.00201270850518895\\
22.57	0.00201270849360537\\
22.58	0.00201270848201558\\
22.59	0.00201270847041961\\
22.6	0.00201270845881749\\
22.61	0.00201270844720921\\
22.62	0.00201270843559481\\
22.63	0.00201270842397429\\
22.64	0.00201270841234768\\
22.65	0.00201270840071499\\
22.66	0.00201270838907625\\
22.67	0.00201270837743146\\
22.68	0.00201270836578066\\
22.69	0.00201270835412385\\
22.7	0.00201270834246105\\
22.71	0.00201270833079228\\
22.72	0.00201270831911757\\
22.73	0.00201270830743693\\
22.74	0.00201270829575038\\
22.75	0.00201270828405794\\
22.76	0.00201270827235962\\
22.77	0.00201270826065545\\
22.78	0.00201270824894545\\
22.79	0.00201270823722963\\
22.8	0.00201270822550802\\
22.81	0.00201270821378063\\
22.82	0.00201270820204749\\
22.83	0.00201270819030862\\
22.84	0.00201270817856403\\
22.85	0.00201270816681374\\
22.86	0.00201270815505778\\
22.87	0.00201270814329617\\
22.88	0.00201270813152893\\
22.89	0.00201270811975607\\
22.9	0.00201270810797762\\
22.91	0.00201270809619361\\
22.92	0.00201270808440404\\
22.93	0.00201270807260895\\
22.94	0.00201270806080834\\
22.95	0.00201270804900226\\
22.96	0.00201270803719071\\
22.97	0.00201270802537371\\
22.98	0.0020127080135513\\
22.99	0.00201270800172349\\
23	0.0020127079898903\\
23.01	0.00201270797805176\\
23.02	0.00201270796620789\\
23.03	0.0020127079543587\\
23.04	0.00201270794250423\\
23.05	0.00201270793064449\\
23.06	0.00201270791877952\\
23.07	0.00201270790690932\\
23.08	0.00201270789503392\\
23.09	0.00201270788315334\\
23.1	0.00201270787126762\\
23.11	0.00201270785937677\\
23.12	0.00201270784748081\\
23.13	0.00201270783557977\\
23.14	0.00201270782367366\\
23.15	0.00201270781176252\\
23.16	0.00201270779984637\\
23.17	0.00201270778792523\\
23.18	0.00201270777599913\\
23.19	0.00201270776406808\\
23.2	0.00201270775213211\\
23.21	0.00201270774019124\\
23.22	0.0020127077282455\\
23.23	0.00201270771629492\\
23.24	0.00201270770433951\\
23.25	0.0020127076923793\\
23.26	0.00201270768041432\\
23.27	0.00201270766844458\\
23.28	0.00201270765647011\\
23.29	0.00201270764449094\\
23.3	0.00201270763250708\\
23.31	0.00201270762051857\\
23.32	0.00201270760852543\\
23.33	0.00201270759652768\\
23.34	0.00201270758452535\\
23.35	0.00201270757251845\\
23.36	0.00201270756050702\\
23.37	0.00201270754849107\\
23.38	0.00201270753647064\\
23.39	0.00201270752444575\\
23.4	0.00201270751241641\\
23.41	0.00201270750038266\\
23.42	0.00201270748834452\\
23.43	0.00201270747630201\\
23.44	0.00201270746425516\\
23.45	0.00201270745220398\\
23.46	0.00201270744014852\\
23.47	0.00201270742808878\\
23.48	0.00201270741602479\\
23.49	0.00201270740395658\\
23.5	0.00201270739188418\\
23.51	0.00201270737980759\\
23.52	0.00201270736772686\\
23.53	0.00201270735564199\\
23.54	0.00201270734355302\\
23.55	0.00201270733145997\\
23.56	0.00201270731936287\\
23.57	0.00201270730726172\\
23.58	0.00201270729515657\\
23.59	0.00201270728304743\\
23.6	0.00201270727093433\\
23.61	0.00201270725881728\\
23.62	0.00201270724669631\\
23.63	0.00201270723457145\\
23.64	0.00201270722244272\\
23.65	0.00201270721031013\\
23.66	0.00201270719817372\\
23.67	0.00201270718603349\\
23.68	0.00201270717388949\\
23.69	0.00201270716174172\\
23.7	0.00201270714959022\\
23.71	0.00201270713743499\\
23.72	0.00201270712527607\\
23.73	0.00201270711311347\\
23.74	0.00201270710094721\\
23.75	0.00201270708877733\\
23.76	0.00201270707660383\\
23.77	0.00201270706442673\\
23.78	0.00201270705224607\\
23.79	0.00201270704006185\\
23.8	0.0020127070278741\\
23.81	0.00201270701568284\\
23.82	0.00201270700348808\\
23.83	0.00201270699128985\\
23.84	0.00201270697908816\\
23.85	0.00201270696688304\\
23.86	0.0020127069546745\\
23.87	0.00201270694246256\\
23.88	0.00201270693024724\\
23.89	0.00201270691802855\\
23.9	0.00201270690580651\\
23.91	0.00201270689358115\\
23.92	0.00201270688135247\\
23.93	0.00201270686912049\\
23.94	0.00201270685688524\\
23.95	0.00201270684464671\\
23.96	0.00201270683240494\\
23.97	0.00201270682015993\\
23.98	0.0020127068079117\\
23.99	0.00201270679566026\\
24	0.00201270678340564\\
24.01	0.00201270677114783\\
24.02	0.00201270675888686\\
24.03	0.00201270674662273\\
24.04	0.00201270673435546\\
24.05	0.00201270672208507\\
24.06	0.00201270670981155\\
24.07	0.00201270669753494\\
24.08	0.00201270668525522\\
24.09	0.00201270667297242\\
24.1	0.00201270666068655\\
24.11	0.0020127066483976\\
24.12	0.00201270663610561\\
24.13	0.00201270662381056\\
24.14	0.00201270661151247\\
24.15	0.00201270659921135\\
24.16	0.0020127065869072\\
24.17	0.00201270657460003\\
24.18	0.00201270656228984\\
24.19	0.00201270654997665\\
24.2	0.00201270653766045\\
24.21	0.00201270652534125\\
24.22	0.00201270651301906\\
24.23	0.00201270650069387\\
24.24	0.0020127064883657\\
24.25	0.00201270647603453\\
24.26	0.00201270646370038\\
24.27	0.00201270645136324\\
24.28	0.00201270643902312\\
24.29	0.00201270642668001\\
24.3	0.00201270641433392\\
24.31	0.00201270640198483\\
24.32	0.00201270638963276\\
24.33	0.0020127063772777\\
24.34	0.00201270636491963\\
24.35	0.00201270635255857\\
24.36	0.00201270634019451\\
24.37	0.00201270632782743\\
24.38	0.00201270631545734\\
24.39	0.00201270630308423\\
24.4	0.00201270629070809\\
24.41	0.00201270627832891\\
24.42	0.00201270626594669\\
24.43	0.00201270625356141\\
24.44	0.00201270624117307\\
24.45	0.00201270622878165\\
24.46	0.00201270621638715\\
24.47	0.00201270620398956\\
24.48	0.00201270619158885\\
24.49	0.00201270617918501\\
24.5	0.00201270616677804\\
24.51	0.00201270615436792\\
24.52	0.00201270614195464\\
24.53	0.00201270612953817\\
24.54	0.0020127061171185\\
24.55	0.00201270610469561\\
24.56	0.00201270609226949\\
24.57	0.00201270607984012\\
24.58	0.00201270606740747\\
24.59	0.00201270605497153\\
24.6	0.00201270604253228\\
24.61	0.0020127060300897\\
24.62	0.00201270601764375\\
24.63	0.00201270600519443\\
24.64	0.00201270599274171\\
24.65	0.00201270598028556\\
24.66	0.00201270596782596\\
24.67	0.00201270595536288\\
24.68	0.0020127059428963\\
24.69	0.00201270593042619\\
24.7	0.00201270591795253\\
24.71	0.00201270590547528\\
24.72	0.00201270589299442\\
24.73	0.00201270588050992\\
24.74	0.00201270586802176\\
24.75	0.00201270585552989\\
24.76	0.00201270584303429\\
24.77	0.00201270583053493\\
24.78	0.00201270581803178\\
24.79	0.00201270580552481\\
24.8	0.00201270579301398\\
24.81	0.00201270578049926\\
24.82	0.00201270576798061\\
24.83	0.00201270575545801\\
24.84	0.00201270574293143\\
24.85	0.00201270573040081\\
24.86	0.00201270571786615\\
24.87	0.00201270570532738\\
24.88	0.00201270569278449\\
24.89	0.00201270568023744\\
24.9	0.00201270566768618\\
24.91	0.0020127056551307\\
24.92	0.00201270564257094\\
24.93	0.00201270563000688\\
24.94	0.00201270561743848\\
24.95	0.0020127056048657\\
24.96	0.00201270559228851\\
24.97	0.00201270557970686\\
24.98	0.00201270556712074\\
24.99	0.00201270555453009\\
25	0.00201270554193489\\
25.01	0.00201270552933509\\
25.02	0.00201270551673066\\
25.03	0.00201270550412157\\
25.04	0.00201270549150777\\
25.05	0.00201270547888924\\
25.06	0.00201270546626593\\
25.07	0.00201270545363781\\
25.08	0.00201270544100484\\
25.09	0.00201270542836699\\
25.1	0.00201270541572422\\
25.11	0.0020127054030765\\
25.12	0.00201270539042379\\
25.13	0.00201270537776605\\
25.14	0.00201270536510324\\
25.15	0.00201270535243534\\
25.16	0.00201270533976231\\
25.17	0.0020127053270841\\
25.18	0.00201270531440069\\
25.19	0.00201270530171205\\
25.2	0.00201270528901813\\
25.21	0.00201270527631889\\
25.22	0.00201270526361432\\
25.23	0.00201270525090437\\
25.24	0.00201270523818901\\
25.25	0.00201270522546819\\
25.26	0.0020127052127419\\
25.27	0.0020127052000101\\
25.28	0.00201270518727275\\
25.29	0.00201270517452982\\
25.3	0.00201270516178128\\
25.31	0.00201270514902709\\
25.32	0.00201270513626722\\
25.33	0.00201270512350165\\
25.34	0.00201270511073033\\
25.35	0.00201270509795325\\
25.36	0.00201270508517036\\
25.37	0.00201270507238164\\
25.38	0.00201270505958705\\
25.39	0.00201270504678657\\
25.4	0.00201270503398017\\
25.41	0.00201270502116782\\
25.42	0.00201270500834948\\
25.43	0.00201270499552514\\
25.44	0.00201270498269476\\
25.45	0.00201270496985831\\
25.46	0.00201270495701578\\
25.47	0.00201270494416713\\
25.48	0.00201270493131233\\
25.49	0.00201270491845136\\
25.5	0.00201270490558421\\
25.51	0.00201270489271083\\
25.52	0.0020127048798312\\
25.53	0.00201270486694531\\
25.54	0.00201270485405314\\
25.55	0.00201270484115464\\
25.56	0.00201270482824982\\
25.57	0.00201270481533864\\
25.58	0.00201270480242107\\
25.59	0.00201270478949712\\
25.6	0.00201270477656674\\
25.61	0.00201270476362993\\
25.62	0.00201270475068666\\
25.63	0.00201270473773691\\
25.64	0.00201270472478068\\
25.65	0.00201270471181793\\
25.66	0.00201270469884866\\
25.67	0.00201270468587284\\
25.68	0.00201270467289047\\
25.69	0.00201270465990152\\
25.7	0.00201270464690599\\
25.71	0.00201270463390386\\
25.72	0.00201270462089511\\
25.73	0.00201270460787973\\
25.74	0.00201270459485772\\
25.75	0.00201270458182905\\
25.76	0.00201270456879372\\
25.77	0.00201270455575172\\
25.78	0.00201270454270304\\
25.79	0.00201270452964766\\
25.8	0.00201270451658558\\
25.81	0.00201270450351679\\
25.82	0.00201270449044129\\
25.83	0.00201270447735906\\
25.84	0.00201270446427009\\
25.85	0.00201270445117438\\
25.86	0.00201270443807193\\
25.87	0.00201270442496273\\
25.88	0.00201270441184677\\
25.89	0.00201270439872404\\
25.9	0.00201270438559455\\
25.91	0.00201270437245828\\
25.92	0.00201270435931524\\
25.93	0.00201270434616541\\
25.94	0.0020127043330088\\
25.95	0.0020127043198454\\
25.96	0.00201270430667521\\
25.97	0.00201270429349822\\
25.98	0.00201270428031442\\
25.99	0.00201270426712383\\
26	0.00201270425392642\\
26.01	0.0020127042407222\\
26.02	0.00201270422751117\\
26.03	0.00201270421429332\\
26.04	0.00201270420106865\\
26.05	0.00201270418783715\\
26.06	0.00201270417459882\\
26.07	0.00201270416135365\\
26.08	0.00201270414810165\\
26.09	0.00201270413484281\\
26.1	0.00201270412157712\\
26.11	0.00201270410830458\\
26.12	0.00201270409502519\\
26.13	0.00201270408173894\\
26.14	0.00201270406844584\\
26.15	0.00201270405514587\\
26.16	0.00201270404183903\\
26.17	0.00201270402852533\\
26.18	0.00201270401520475\\
26.19	0.00201270400187729\\
26.2	0.00201270398854294\\
26.21	0.00201270397520172\\
26.22	0.0020127039618536\\
26.23	0.00201270394849859\\
26.24	0.00201270393513668\\
26.25	0.00201270392176788\\
26.26	0.00201270390839216\\
26.27	0.00201270389500955\\
26.28	0.00201270388162001\\
26.29	0.00201270386822357\\
26.3	0.0020127038548202\\
26.31	0.00201270384140991\\
26.32	0.00201270382799269\\
26.33	0.00201270381456854\\
26.34	0.00201270380113746\\
26.35	0.00201270378769944\\
26.36	0.00201270377425448\\
26.37	0.00201270376080257\\
26.38	0.00201270374734372\\
26.39	0.00201270373387791\\
26.4	0.00201270372040514\\
26.41	0.00201270370692541\\
26.42	0.00201270369343872\\
26.43	0.00201270367994506\\
26.44	0.00201270366644443\\
26.45	0.00201270365293682\\
26.46	0.00201270363942224\\
26.47	0.00201270362590066\\
26.48	0.00201270361237211\\
26.49	0.00201270359883656\\
26.5	0.00201270358529401\\
26.51	0.00201270357174447\\
26.52	0.00201270355818793\\
26.53	0.00201270354462438\\
26.54	0.00201270353105382\\
26.55	0.00201270351747624\\
26.56	0.00201270350389165\\
26.57	0.00201270349030004\\
26.58	0.0020127034767014\\
26.59	0.00201270346309573\\
26.6	0.00201270344948303\\
26.61	0.00201270343586329\\
26.62	0.00201270342223652\\
26.63	0.00201270340860269\\
26.64	0.00201270339496182\\
26.65	0.0020127033813139\\
26.66	0.00201270336765892\\
26.67	0.00201270335399688\\
26.68	0.00201270334032778\\
26.69	0.00201270332665161\\
26.7	0.00201270331296837\\
26.71	0.00201270329927806\\
26.72	0.00201270328558066\\
26.73	0.00201270327187619\\
26.74	0.00201270325816462\\
26.75	0.00201270324444597\\
26.76	0.00201270323072022\\
26.77	0.00201270321698737\\
26.78	0.00201270320324743\\
26.79	0.00201270318950037\\
26.8	0.00201270317574621\\
26.81	0.00201270316198492\\
26.82	0.00201270314821653\\
26.83	0.00201270313444101\\
26.84	0.00201270312065837\\
26.85	0.00201270310686859\\
26.86	0.00201270309307169\\
26.87	0.00201270307926764\\
26.88	0.00201270306545646\\
26.89	0.00201270305163813\\
26.9	0.00201270303781265\\
26.91	0.00201270302398002\\
26.92	0.00201270301014023\\
26.93	0.00201270299629329\\
26.94	0.00201270298243917\\
26.95	0.00201270296857789\\
26.96	0.00201270295470944\\
26.97	0.00201270294083381\\
26.98	0.002012702926951\\
26.99	0.002012702913061\\
27	0.00201270289916382\\
27.01	0.00201270288525945\\
27.02	0.00201270287134788\\
27.03	0.00201270285742911\\
27.04	0.00201270284350313\\
27.05	0.00201270282956995\\
27.06	0.00201270281562956\\
27.07	0.00201270280168195\\
27.08	0.00201270278772712\\
27.09	0.00201270277376507\\
27.1	0.00201270275979579\\
27.11	0.00201270274581927\\
27.12	0.00201270273183553\\
27.13	0.00201270271784454\\
27.14	0.00201270270384631\\
27.15	0.00201270268984083\\
27.16	0.0020127026758281\\
27.17	0.00201270266180812\\
27.18	0.00201270264778087\\
27.19	0.00201270263374636\\
27.2	0.00201270261970459\\
27.21	0.00201270260565554\\
27.22	0.00201270259159922\\
27.23	0.00201270257753562\\
27.24	0.00201270256346474\\
27.25	0.00201270254938657\\
27.26	0.0020127025353011\\
27.27	0.00201270252120834\\
27.28	0.00201270250710829\\
27.29	0.00201270249300092\\
27.3	0.00201270247888626\\
27.31	0.00201270246476428\\
27.32	0.00201270245063498\\
27.33	0.00201270243649836\\
27.34	0.00201270242235442\\
27.35	0.00201270240820316\\
27.36	0.00201270239404456\\
27.37	0.00201270237987862\\
27.38	0.00201270236570535\\
27.39	0.00201270235152473\\
27.4	0.00201270233733677\\
27.41	0.00201270232314145\\
27.42	0.00201270230893877\\
27.43	0.00201270229472874\\
27.44	0.00201270228051134\\
27.45	0.00201270226628658\\
27.46	0.00201270225205444\\
27.47	0.00201270223781492\\
27.48	0.00201270222356803\\
27.49	0.00201270220931375\\
27.5	0.00201270219505208\\
27.51	0.00201270218078302\\
27.52	0.00201270216650656\\
27.53	0.0020127021522227\\
27.54	0.00201270213793144\\
27.55	0.00201270212363277\\
27.56	0.00201270210932668\\
27.57	0.00201270209501318\\
27.58	0.00201270208069226\\
27.59	0.00201270206636391\\
27.6	0.00201270205202813\\
27.61	0.00201270203768492\\
27.62	0.00201270202333426\\
27.63	0.00201270200897617\\
27.64	0.00201270199461063\\
27.65	0.00201270198023765\\
27.66	0.0020127019658572\\
27.67	0.0020127019514693\\
27.68	0.00201270193707394\\
27.69	0.00201270192267111\\
27.7	0.00201270190826081\\
27.71	0.00201270189384303\\
27.72	0.00201270187941777\\
27.73	0.00201270186498504\\
27.74	0.00201270185054481\\
27.75	0.00201270183609709\\
27.76	0.00201270182164188\\
27.77	0.00201270180717916\\
27.78	0.00201270179270894\\
27.79	0.00201270177823122\\
27.8	0.00201270176374598\\
27.81	0.00201270174925322\\
27.82	0.00201270173475295\\
27.83	0.00201270172024515\\
27.84	0.00201270170572982\\
27.85	0.00201270169120695\\
27.86	0.00201270167667655\\
27.87	0.00201270166213861\\
27.88	0.00201270164759312\\
27.89	0.00201270163304008\\
27.9	0.00201270161847949\\
27.91	0.00201270160391133\\
27.92	0.00201270158933562\\
27.93	0.00201270157475233\\
27.94	0.00201270156016148\\
27.95	0.00201270154556305\\
27.96	0.00201270153095704\\
27.97	0.00201270151634345\\
27.98	0.00201270150172227\\
27.99	0.00201270148709349\\
28	0.00201270147245713\\
28.01	0.00201270145781315\\
28.02	0.00201270144316158\\
28.03	0.00201270142850239\\
28.04	0.00201270141383559\\
28.05	0.00201270139916118\\
28.06	0.00201270138447914\\
28.07	0.00201270136978947\\
28.08	0.00201270135509217\\
28.09	0.00201270134038724\\
28.1	0.00201270132567467\\
28.11	0.00201270131095446\\
28.12	0.0020127012962266\\
28.13	0.00201270128149108\\
28.14	0.00201270126674791\\
28.15	0.00201270125199708\\
28.16	0.00201270123723859\\
28.17	0.00201270122247242\\
28.18	0.00201270120769858\\
28.19	0.00201270119291707\\
28.2	0.00201270117812787\\
28.21	0.00201270116333098\\
28.22	0.00201270114852641\\
28.23	0.00201270113371414\\
28.24	0.00201270111889417\\
28.25	0.00201270110406649\\
28.26	0.00201270108923111\\
28.27	0.00201270107438802\\
28.28	0.00201270105953721\\
28.29	0.00201270104467867\\
28.3	0.00201270102981242\\
28.31	0.00201270101493843\\
28.32	0.00201270100005671\\
28.33	0.00201270098516725\\
28.34	0.00201270097027005\\
28.35	0.0020127009553651\\
28.36	0.0020127009404524\\
28.37	0.00201270092553194\\
28.38	0.00201270091060373\\
28.39	0.00201270089566774\\
28.4	0.00201270088072399\\
28.41	0.00201270086577247\\
28.42	0.00201270085081317\\
28.43	0.00201270083584609\\
28.44	0.00201270082087122\\
28.45	0.00201270080588856\\
28.46	0.0020127007908981\\
28.47	0.00201270077589985\\
28.48	0.00201270076089379\\
28.49	0.00201270074587992\\
28.5	0.00201270073085824\\
28.51	0.00201270071582874\\
28.52	0.00201270070079142\\
28.53	0.00201270068574627\\
28.54	0.0020127006706933\\
28.55	0.00201270065563249\\
28.56	0.00201270064056384\\
28.57	0.00201270062548735\\
28.58	0.002012700610403\\
28.59	0.00201270059531081\\
28.6	0.00201270058021076\\
28.61	0.00201270056510284\\
28.62	0.00201270054998706\\
28.63	0.00201270053486341\\
28.64	0.00201270051973189\\
28.65	0.00201270050459249\\
28.66	0.0020127004894452\\
28.67	0.00201270047429002\\
28.68	0.00201270045912695\\
28.69	0.00201270044395599\\
28.7	0.00201270042877712\\
28.71	0.00201270041359035\\
28.72	0.00201270039839566\\
28.73	0.00201270038319306\\
28.74	0.00201270036798254\\
28.75	0.0020127003527641\\
28.76	0.00201270033753773\\
28.77	0.00201270032230342\\
28.78	0.00201270030706118\\
28.79	0.002012700291811\\
28.8	0.00201270027655287\\
28.81	0.00201270026128678\\
28.82	0.00201270024601275\\
28.83	0.00201270023073075\\
28.84	0.00201270021544079\\
28.85	0.00201270020014285\\
28.86	0.00201270018483695\\
28.87	0.00201270016952306\\
28.88	0.0020127001542012\\
28.89	0.00201270013887134\\
28.9	0.0020127001235335\\
28.91	0.00201270010818766\\
28.92	0.00201270009283381\\
28.93	0.00201270007747197\\
28.94	0.00201270006210211\\
28.95	0.00201270004672423\\
28.96	0.00201270003133834\\
28.97	0.00201270001594442\\
28.98	0.00201270000054248\\
28.99	0.0020126999851325\\
29	0.00201269996971449\\
29.01	0.00201269995428843\\
29.02	0.00201269993885432\\
29.03	0.00201269992341217\\
29.04	0.00201269990796196\\
29.05	0.00201269989250369\\
29.06	0.00201269987703735\\
29.07	0.00201269986156294\\
29.08	0.00201269984608046\\
29.09	0.0020126998305899\\
29.1	0.00201269981509126\\
29.11	0.00201269979958453\\
29.12	0.0020126997840697\\
29.13	0.00201269976854678\\
29.14	0.00201269975301576\\
29.15	0.00201269973747663\\
29.16	0.00201269972192938\\
29.17	0.00201269970637403\\
29.18	0.00201269969081055\\
29.19	0.00201269967523894\\
29.2	0.00201269965965921\\
29.21	0.00201269964407134\\
29.22	0.00201269962847534\\
29.23	0.00201269961287118\\
29.24	0.00201269959725888\\
29.25	0.00201269958163843\\
29.26	0.00201269956600982\\
29.27	0.00201269955037305\\
29.28	0.00201269953472811\\
29.29	0.00201269951907499\\
29.3	0.0020126995034137\\
29.31	0.00201269948774424\\
29.32	0.00201269947206658\\
29.33	0.00201269945638073\\
29.34	0.00201269944068669\\
29.35	0.00201269942498444\\
29.36	0.002012699409274\\
29.37	0.00201269939355534\\
29.38	0.00201269937782847\\
29.39	0.00201269936209338\\
29.4	0.00201269934635006\\
29.41	0.00201269933059852\\
29.42	0.00201269931483874\\
29.43	0.00201269929907073\\
29.44	0.00201269928329447\\
29.45	0.00201269926750997\\
29.46	0.00201269925171721\\
29.47	0.0020126992359162\\
29.48	0.00201269922010692\\
29.49	0.00201269920428938\\
29.5	0.00201269918846357\\
29.51	0.00201269917262948\\
29.52	0.00201269915678711\\
29.53	0.00201269914093645\\
29.54	0.00201269912507751\\
29.55	0.00201269910921027\\
29.56	0.00201269909333473\\
29.57	0.00201269907745088\\
29.58	0.00201269906155873\\
29.59	0.00201269904565826\\
29.6	0.00201269902974947\\
29.61	0.00201269901383236\\
29.62	0.00201269899790692\\
29.63	0.00201269898197314\\
29.64	0.00201269896603103\\
29.65	0.00201269895008058\\
29.66	0.00201269893412177\\
29.67	0.00201269891815461\\
29.68	0.0020126989021791\\
29.69	0.00201269888619522\\
29.7	0.00201269887020298\\
29.71	0.00201269885420236\\
29.72	0.00201269883819337\\
29.73	0.00201269882217599\\
29.74	0.00201269880615023\\
29.75	0.00201269879011608\\
29.76	0.00201269877407353\\
29.77	0.00201269875802258\\
29.78	0.00201269874196322\\
29.79	0.00201269872589545\\
29.8	0.00201269870981927\\
29.81	0.00201269869373466\\
29.82	0.00201269867764163\\
29.83	0.00201269866154017\\
29.84	0.00201269864543027\\
29.85	0.00201269862931193\\
29.86	0.00201269861318515\\
29.87	0.00201269859704992\\
29.88	0.00201269858090623\\
29.89	0.00201269856475408\\
29.9	0.00201269854859347\\
29.91	0.00201269853242438\\
29.92	0.00201269851624683\\
29.93	0.00201269850006079\\
29.94	0.00201269848386627\\
29.95	0.00201269846766325\\
29.96	0.00201269845145175\\
29.97	0.00201269843523174\\
29.98	0.00201269841900323\\
29.99	0.00201269840276621\\
30	0.00201269838652068\\
30.01	0.00201269837026662\\
30.02	0.00201269835400405\\
30.03	0.00201269833773294\\
30.04	0.0020126983214533\\
30.05	0.00201269830516512\\
30.06	0.00201269828886839\\
30.07	0.00201269827256312\\
30.08	0.00201269825624929\\
30.09	0.0020126982399269\\
30.1	0.00201269822359595\\
30.11	0.00201269820725643\\
30.12	0.00201269819090834\\
30.13	0.00201269817455166\\
30.14	0.00201269815818641\\
30.15	0.00201269814181256\\
30.16	0.00201269812543012\\
30.17	0.00201269810903908\\
30.18	0.00201269809263943\\
30.19	0.00201269807623118\\
30.2	0.00201269805981431\\
30.21	0.00201269804338882\\
30.22	0.00201269802695471\\
30.23	0.00201269801051197\\
30.24	0.00201269799406059\\
30.25	0.00201269797760057\\
30.26	0.00201269796113191\\
30.27	0.0020126979446546\\
30.28	0.00201269792816864\\
30.29	0.00201269791167401\\
30.3	0.00201269789517072\\
30.31	0.00201269787865876\\
30.32	0.00201269786213813\\
30.33	0.00201269784560881\\
30.34	0.00201269782907081\\
30.35	0.00201269781252412\\
30.36	0.00201269779596874\\
30.37	0.00201269777940465\\
30.38	0.00201269776283186\\
30.39	0.00201269774625036\\
30.4	0.00201269772966014\\
30.41	0.0020126977130612\\
30.42	0.00201269769645353\\
30.43	0.00201269767983713\\
30.44	0.002012697663212\\
30.45	0.00201269764657812\\
30.46	0.0020126976299355\\
30.47	0.00201269761328413\\
30.48	0.002012697596624\\
30.49	0.0020126975799551\\
30.5	0.00201269756327744\\
30.51	0.00201269754659101\\
30.52	0.0020126975298958\\
30.53	0.00201269751319181\\
30.54	0.00201269749647902\\
30.55	0.00201269747975745\\
30.56	0.00201269746302708\\
30.57	0.0020126974462879\\
30.58	0.00201269742953992\\
30.59	0.00201269741278312\\
30.6	0.0020126973960175\\
30.61	0.00201269737924306\\
30.62	0.00201269736245979\\
30.63	0.00201269734566768\\
30.64	0.00201269732886673\\
30.65	0.00201269731205694\\
30.66	0.0020126972952383\\
30.67	0.0020126972784108\\
30.68	0.00201269726157445\\
30.69	0.00201269724472922\\
30.7	0.00201269722787512\\
30.71	0.00201269721101215\\
30.72	0.0020126971941403\\
30.73	0.00201269717725956\\
30.74	0.00201269716036992\\
30.75	0.0020126971434714\\
30.76	0.00201269712656396\\
30.77	0.00201269710964762\\
30.78	0.00201269709272237\\
30.79	0.00201269707578819\\
30.8	0.00201269705884509\\
30.81	0.00201269704189307\\
30.82	0.00201269702493211\\
30.83	0.00201269700796221\\
30.84	0.00201269699098336\\
30.85	0.00201269697399556\\
30.86	0.00201269695699881\\
30.87	0.0020126969399931\\
30.88	0.00201269692297842\\
30.89	0.00201269690595477\\
30.9	0.00201269688892214\\
30.91	0.00201269687188053\\
30.92	0.00201269685482993\\
30.93	0.00201269683777034\\
30.94	0.00201269682070176\\
30.95	0.00201269680362416\\
30.96	0.00201269678653756\\
30.97	0.00201269676944195\\
30.98	0.00201269675233731\\
30.99	0.00201269673522365\\
31	0.00201269671810096\\
31.01	0.00201269670096924\\
31.02	0.00201269668382847\\
31.03	0.00201269666667865\\
31.04	0.00201269664951979\\
31.05	0.00201269663235186\\
31.06	0.00201269661517488\\
31.07	0.00201269659798883\\
31.08	0.0020126965807937\\
31.09	0.00201269656358949\\
31.1	0.0020126965463762\\
31.11	0.00201269652915382\\
31.12	0.00201269651192235\\
31.13	0.00201269649468178\\
31.14	0.0020126964774321\\
31.15	0.0020126964601733\\
31.16	0.00201269644290539\\
31.17	0.00201269642562836\\
31.18	0.00201269640834221\\
31.19	0.00201269639104691\\
31.2	0.00201269637374248\\
31.21	0.00201269635642891\\
31.22	0.00201269633910619\\
31.23	0.00201269632177431\\
31.24	0.00201269630443327\\
31.25	0.00201269628708307\\
31.26	0.00201269626972369\\
31.27	0.00201269625235514\\
31.28	0.0020126962349774\\
31.29	0.00201269621759048\\
31.3	0.00201269620019437\\
31.31	0.00201269618278905\\
31.32	0.00201269616537453\\
31.33	0.00201269614795081\\
31.34	0.00201269613051786\\
31.35	0.0020126961130757\\
31.36	0.00201269609562431\\
31.37	0.00201269607816369\\
31.38	0.00201269606069383\\
31.39	0.00201269604321473\\
31.4	0.00201269602572638\\
31.41	0.00201269600822878\\
31.42	0.00201269599072191\\
31.43	0.00201269597320579\\
31.44	0.00201269595568039\\
31.45	0.00201269593814571\\
31.46	0.00201269592060176\\
31.47	0.00201269590304852\\
31.48	0.00201269588548598\\
31.49	0.00201269586791415\\
31.5	0.00201269585033301\\
31.51	0.00201269583274256\\
31.52	0.0020126958151428\\
31.53	0.00201269579753372\\
31.54	0.00201269577991531\\
31.55	0.00201269576228757\\
31.56	0.0020126957446505\\
31.57	0.00201269572700408\\
31.58	0.00201269570934831\\
31.59	0.00201269569168318\\
31.6	0.0020126956740087\\
31.61	0.00201269565632485\\
31.62	0.00201269563863163\\
31.63	0.00201269562092904\\
31.64	0.00201269560321706\\
31.65	0.0020126955854957\\
31.66	0.00201269556776494\\
31.67	0.00201269555002478\\
31.68	0.00201269553227522\\
31.69	0.00201269551451624\\
31.7	0.00201269549674785\\
31.71	0.00201269547897004\\
31.72	0.0020126954611828\\
31.73	0.00201269544338613\\
31.74	0.00201269542558002\\
31.75	0.00201269540776446\\
31.76	0.00201269538993946\\
31.77	0.002012695372105\\
31.78	0.00201269535426108\\
31.79	0.00201269533640769\\
31.8	0.00201269531854483\\
31.81	0.00201269530067249\\
31.82	0.00201269528279066\\
31.83	0.00201269526489935\\
31.84	0.00201269524699853\\
31.85	0.00201269522908822\\
31.86	0.0020126952111684\\
31.87	0.00201269519323907\\
31.88	0.00201269517530022\\
31.89	0.00201269515735184\\
31.9	0.00201269513939394\\
31.91	0.0020126951214265\\
31.92	0.00201269510344951\\
31.93	0.00201269508546298\\
31.94	0.0020126950674669\\
31.95	0.00201269504946126\\
31.96	0.00201269503144605\\
31.97	0.00201269501342127\\
31.98	0.00201269499538692\\
31.99	0.00201269497734298\\
32	0.00201269495928946\\
32.01	0.00201269494122635\\
32.02	0.00201269492315363\\
32.03	0.00201269490507131\\
32.04	0.00201269488697938\\
32.05	0.00201269486887783\\
32.06	0.00201269485076666\\
32.07	0.00201269483264586\\
32.08	0.00201269481451543\\
32.09	0.00201269479637535\\
32.1	0.00201269477822563\\
32.11	0.00201269476006626\\
32.12	0.00201269474189724\\
32.13	0.00201269472371854\\
32.14	0.00201269470553019\\
32.15	0.00201269468733215\\
32.16	0.00201269466912443\\
32.17	0.00201269465090703\\
32.18	0.00201269463267994\\
32.19	0.00201269461444314\\
32.2	0.00201269459619664\\
32.21	0.00201269457794044\\
32.22	0.00201269455967451\\
32.23	0.00201269454139887\\
32.24	0.0020126945231135\\
32.25	0.00201269450481839\\
32.26	0.00201269448651354\\
32.27	0.00201269446819895\\
32.28	0.00201269444987461\\
32.29	0.00201269443154051\\
32.3	0.00201269441319664\\
32.31	0.00201269439484301\\
32.32	0.00201269437647961\\
32.33	0.00201269435810642\\
32.34	0.00201269433972344\\
32.35	0.00201269432133067\\
32.36	0.00201269430292811\\
32.37	0.00201269428451574\\
32.38	0.00201269426609356\\
32.39	0.00201269424766156\\
32.4	0.00201269422921974\\
32.41	0.00201269421076809\\
32.42	0.00201269419230661\\
32.43	0.00201269417383528\\
32.44	0.00201269415535412\\
32.45	0.00201269413686309\\
32.46	0.00201269411836221\\
32.47	0.00201269409985147\\
32.48	0.00201269408133086\\
32.49	0.00201269406280037\\
32.5	0.00201269404425999\\
32.51	0.00201269402570973\\
32.52	0.00201269400714958\\
32.53	0.00201269398857952\\
32.54	0.00201269396999956\\
32.55	0.00201269395140969\\
32.56	0.0020126939328099\\
32.57	0.00201269391420019\\
32.58	0.00201269389558054\\
32.59	0.00201269387695096\\
32.6	0.00201269385831144\\
32.61	0.00201269383966197\\
32.62	0.00201269382100254\\
32.63	0.00201269380233315\\
32.64	0.0020126937836538\\
32.65	0.00201269376496447\\
32.66	0.00201269374626517\\
32.67	0.00201269372755588\\
32.68	0.0020126937088366\\
32.69	0.00201269369010733\\
32.7	0.00201269367136805\\
32.71	0.00201269365261876\\
32.72	0.00201269363385946\\
32.73	0.00201269361509014\\
32.74	0.00201269359631079\\
32.75	0.0020126935775214\\
32.76	0.00201269355872198\\
32.77	0.00201269353991251\\
32.78	0.00201269352109299\\
32.79	0.00201269350226342\\
32.8	0.00201269348342378\\
32.81	0.00201269346457407\\
32.82	0.00201269344571428\\
32.83	0.00201269342684441\\
32.84	0.00201269340796446\\
32.85	0.0020126933890744\\
32.86	0.00201269337017425\\
32.87	0.00201269335126399\\
32.88	0.00201269333234362\\
32.89	0.00201269331341313\\
32.9	0.00201269329447252\\
32.91	0.00201269327552177\\
32.92	0.00201269325656089\\
32.93	0.00201269323758986\\
32.94	0.00201269321860868\\
32.95	0.00201269319961735\\
32.96	0.00201269318061586\\
32.97	0.0020126931616042\\
32.98	0.00201269314258236\\
32.99	0.00201269312355034\\
33	0.00201269310450814\\
33.01	0.00201269308545574\\
33.02	0.00201269306639314\\
33.03	0.00201269304732034\\
33.04	0.00201269302823733\\
33.05	0.00201269300914409\\
33.06	0.00201269299004064\\
33.07	0.00201269297092695\\
33.08	0.00201269295180303\\
33.09	0.00201269293266887\\
33.1	0.00201269291352445\\
33.11	0.00201269289436978\\
33.12	0.00201269287520485\\
33.13	0.00201269285602966\\
33.14	0.00201269283684418\\
33.15	0.00201269281764843\\
33.16	0.00201269279844239\\
33.17	0.00201269277922606\\
33.18	0.00201269275999943\\
33.19	0.0020126927407625\\
33.2	0.00201269272151525\\
33.21	0.00201269270225768\\
33.22	0.0020126926829898\\
33.23	0.00201269266371158\\
33.24	0.00201269264442302\\
33.25	0.00201269262512412\\
33.26	0.00201269260581488\\
33.27	0.00201269258649527\\
33.28	0.00201269256716531\\
33.29	0.00201269254782498\\
33.3	0.00201269252847427\\
33.31	0.00201269250911318\\
33.32	0.00201269248974171\\
33.33	0.00201269247035984\\
33.34	0.00201269245096757\\
33.35	0.0020126924315649\\
33.36	0.00201269241215181\\
33.37	0.0020126923927283\\
33.38	0.00201269237329438\\
33.39	0.00201269235385001\\
33.4	0.00201269233439521\\
33.41	0.00201269231492997\\
33.42	0.00201269229545428\\
33.43	0.00201269227596813\\
33.44	0.00201269225647152\\
33.45	0.00201269223696443\\
33.46	0.00201269221744687\\
33.47	0.00201269219791883\\
33.48	0.0020126921783803\\
33.49	0.00201269215883128\\
33.5	0.00201269213927175\\
33.51	0.00201269211970172\\
33.52	0.00201269210012117\\
33.53	0.0020126920805301\\
33.54	0.00201269206092851\\
33.55	0.00201269204131638\\
33.56	0.00201269202169371\\
33.57	0.0020126920020605\\
33.58	0.00201269198241673\\
33.59	0.0020126919627624\\
33.6	0.00201269194309751\\
33.61	0.00201269192342205\\
33.62	0.002012691903736\\
33.63	0.00201269188403938\\
33.64	0.00201269186433216\\
33.65	0.00201269184461435\\
33.66	0.00201269182488593\\
33.67	0.0020126918051469\\
33.68	0.00201269178539725\\
33.69	0.00201269176563698\\
33.7	0.00201269174586608\\
33.71	0.00201269172608454\\
33.72	0.00201269170629236\\
33.73	0.00201269168648953\\
33.74	0.00201269166667605\\
33.75	0.0020126916468519\\
33.76	0.00201269162701709\\
33.77	0.0020126916071716\\
33.78	0.00201269158731543\\
33.79	0.00201269156744857\\
33.8	0.00201269154757101\\
33.81	0.00201269152768276\\
33.82	0.00201269150778379\\
33.83	0.00201269148787411\\
33.84	0.00201269146795371\\
33.85	0.00201269144802259\\
33.86	0.00201269142808073\\
33.87	0.00201269140812813\\
33.88	0.00201269138816478\\
33.89	0.00201269136819068\\
33.9	0.00201269134820582\\
33.91	0.0020126913282102\\
33.92	0.0020126913082038\\
33.93	0.00201269128818662\\
33.94	0.00201269126815866\\
33.95	0.0020126912481199\\
33.96	0.00201269122807034\\
33.97	0.00201269120800998\\
33.98	0.00201269118793881\\
33.99	0.00201269116785682\\
34	0.002012691147764\\
34.01	0.00201269112766035\\
34.02	0.00201269110754586\\
34.03	0.00201269108742053\\
34.04	0.00201269106728434\\
34.05	0.0020126910471373\\
34.06	0.00201269102697939\\
34.07	0.00201269100681062\\
34.08	0.00201269098663096\\
34.09	0.00201269096644042\\
34.1	0.00201269094623898\\
34.11	0.00201269092602665\\
34.12	0.00201269090580342\\
34.13	0.00201269088556927\\
34.14	0.00201269086532421\\
34.15	0.00201269084506822\\
34.16	0.00201269082480131\\
34.17	0.00201269080452345\\
34.18	0.00201269078423465\\
34.19	0.0020126907639349\\
34.2	0.00201269074362419\\
34.21	0.00201269072330252\\
34.22	0.00201269070296988\\
34.23	0.00201269068262626\\
34.24	0.00201269066227166\\
34.25	0.00201269064190606\\
34.26	0.00201269062152947\\
34.27	0.00201269060114188\\
34.28	0.00201269058074327\\
34.29	0.00201269056033365\\
34.3	0.002012690539913\\
34.31	0.00201269051948133\\
34.32	0.00201269049903861\\
34.33	0.00201269047858486\\
34.34	0.00201269045812005\\
34.35	0.00201269043764418\\
34.36	0.00201269041715725\\
34.37	0.00201269039665925\\
34.38	0.00201269037615018\\
34.39	0.00201269035563002\\
34.4	0.00201269033509876\\
34.41	0.00201269031455641\\
34.42	0.00201269029400296\\
34.43	0.0020126902734384\\
34.44	0.00201269025286272\\
34.45	0.00201269023227591\\
34.46	0.00201269021167798\\
34.47	0.0020126901910689\\
34.48	0.00201269017044869\\
34.49	0.00201269014981732\\
34.5	0.00201269012917479\\
34.51	0.0020126901085211\\
34.52	0.00201269008785624\\
34.53	0.0020126900671802\\
34.54	0.00201269004649297\\
34.55	0.00201269002579455\\
34.56	0.00201269000508494\\
34.57	0.00201268998436412\\
34.58	0.00201268996363209\\
34.59	0.00201268994288884\\
34.6	0.00201268992213437\\
34.61	0.00201268990136866\\
34.62	0.00201268988059171\\
34.63	0.00201268985980352\\
34.64	0.00201268983900408\\
34.65	0.00201268981819338\\
34.66	0.00201268979737141\\
34.67	0.00201268977653817\\
34.68	0.00201268975569365\\
34.69	0.00201268973483784\\
34.7	0.00201268971397074\\
34.71	0.00201268969309234\\
34.72	0.00201268967220264\\
34.73	0.00201268965130162\\
34.74	0.00201268963038928\\
34.75	0.00201268960946562\\
34.76	0.00201268958853062\\
34.77	0.00201268956758428\\
34.78	0.00201268954662659\\
34.79	0.00201268952565755\\
34.8	0.00201268950467715\\
34.81	0.00201268948368538\\
34.82	0.00201268946268223\\
34.83	0.00201268944166771\\
34.84	0.00201268942064179\\
34.85	0.00201268939960448\\
34.86	0.00201268937855577\\
34.87	0.00201268935749565\\
34.88	0.00201268933642412\\
34.89	0.00201268931534116\\
34.9	0.00201268929424677\\
34.91	0.00201268927314095\\
34.92	0.00201268925202369\\
34.93	0.00201268923089497\\
34.94	0.0020126892097548\\
34.95	0.00201268918860316\\
34.96	0.00201268916744006\\
34.97	0.00201268914626548\\
34.98	0.00201268912507941\\
34.99	0.00201268910388185\\
35	0.00201268908267279\\
35.01	0.00201268906145223\\
35.02	0.00201268904022016\\
35.03	0.00201268901897657\\
35.04	0.00201268899772145\\
35.05	0.00201268897645481\\
35.06	0.00201268895517662\\
35.07	0.00201268893388689\\
35.08	0.0020126889125856\\
35.09	0.00201268889127276\\
35.1	0.00201268886994834\\
35.11	0.00201268884861236\\
35.12	0.00201268882726479\\
35.13	0.00201268880590564\\
35.14	0.00201268878453489\\
35.15	0.00201268876315254\\
35.16	0.00201268874175859\\
35.17	0.00201268872035302\\
35.18	0.00201268869893582\\
35.19	0.002012688677507\\
35.2	0.00201268865606654\\
35.21	0.00201268863461444\\
35.22	0.00201268861315069\\
35.23	0.00201268859167529\\
35.24	0.00201268857018822\\
35.25	0.00201268854868948\\
35.26	0.00201268852717906\\
35.27	0.00201268850565696\\
35.28	0.00201268848412317\\
35.29	0.00201268846257768\\
35.3	0.00201268844102048\\
35.31	0.00201268841945157\\
35.32	0.00201268839787095\\
35.33	0.00201268837627859\\
35.34	0.00201268835467451\\
35.35	0.00201268833305869\\
35.36	0.00201268831143111\\
35.37	0.00201268828979179\\
35.38	0.0020126882681407\\
35.39	0.00201268824647785\\
35.4	0.00201268822480322\\
35.41	0.00201268820311681\\
35.42	0.00201268818141861\\
35.43	0.00201268815970862\\
35.44	0.00201268813798683\\
35.45	0.00201268811625322\\
35.46	0.0020126880945078\\
35.47	0.00201268807275055\\
35.48	0.00201268805098148\\
35.49	0.00201268802920057\\
35.5	0.00201268800740781\\
35.51	0.0020126879856032\\
35.52	0.00201268796378673\\
35.53	0.0020126879419584\\
35.54	0.0020126879201182\\
35.55	0.00201268789826611\\
35.56	0.00201268787640214\\
35.57	0.00201268785452628\\
35.58	0.00201268783263852\\
35.59	0.00201268781073885\\
35.6	0.00201268778882726\\
35.61	0.00201268776690375\\
35.62	0.00201268774496832\\
35.63	0.00201268772302095\\
35.64	0.00201268770106164\\
35.65	0.00201268767909038\\
35.66	0.00201268765710716\\
35.67	0.00201268763511198\\
35.68	0.00201268761310482\\
35.69	0.0020126875910857\\
35.7	0.00201268756905458\\
35.71	0.00201268754701147\\
35.72	0.00201268752495637\\
35.73	0.00201268750288926\\
35.74	0.00201268748081014\\
35.75	0.00201268745871899\\
35.76	0.00201268743661583\\
35.77	0.00201268741450062\\
35.78	0.00201268739237338\\
35.79	0.00201268737023409\\
35.8	0.00201268734808274\\
35.81	0.00201268732591933\\
35.82	0.00201268730374385\\
35.83	0.0020126872815563\\
35.84	0.00201268725935666\\
35.85	0.00201268723714494\\
35.86	0.00201268721492111\\
35.87	0.00201268719268518\\
35.88	0.00201268717043714\\
35.89	0.00201268714817698\\
35.9	0.0020126871259047\\
35.91	0.00201268710362028\\
35.92	0.00201268708132373\\
35.93	0.00201268705901502\\
35.94	0.00201268703669417\\
35.95	0.00201268701436115\\
35.96	0.00201268699201596\\
35.97	0.0020126869696586\\
35.98	0.00201268694728906\\
35.99	0.00201268692490732\\
36	0.0020126869025134\\
36.01	0.00201268688010726\\
36.02	0.00201268685768892\\
36.03	0.00201268683525836\\
36.04	0.00201268681281557\\
36.05	0.00201268679036056\\
36.06	0.0020126867678933\\
36.07	0.0020126867454138\\
36.08	0.00201268672292205\\
36.09	0.00201268670041803\\
36.1	0.00201268667790175\\
36.11	0.00201268665537319\\
36.12	0.00201268663283235\\
36.13	0.00201268661027923\\
36.14	0.0020126865877138\\
36.15	0.00201268656513608\\
36.16	0.00201268654254604\\
36.17	0.00201268651994369\\
36.18	0.00201268649732901\\
36.19	0.00201268647470201\\
36.2	0.00201268645206266\\
36.21	0.00201268642941097\\
36.22	0.00201268640674692\\
36.23	0.00201268638407052\\
36.24	0.00201268636138175\\
36.25	0.0020126863386806\\
36.26	0.00201268631596708\\
36.27	0.00201268629324116\\
36.28	0.00201268627050285\\
36.29	0.00201268624775214\\
36.3	0.00201268622498901\\
36.31	0.00201268620221347\\
36.32	0.00201268617942551\\
36.33	0.00201268615662512\\
36.34	0.00201268613381228\\
36.35	0.002012686110987\\
36.36	0.00201268608814927\\
36.37	0.00201268606529908\\
36.38	0.00201268604243642\\
36.39	0.00201268601956129\\
36.4	0.00201268599667368\\
36.41	0.00201268597377358\\
36.42	0.00201268595086098\\
36.43	0.00201268592793588\\
36.44	0.00201268590499827\\
36.45	0.00201268588204814\\
36.46	0.0020126858590855\\
36.47	0.00201268583611031\\
36.48	0.00201268581312259\\
36.49	0.00201268579012233\\
36.5	0.00201268576710951\\
36.51	0.00201268574408413\\
36.52	0.00201268572104618\\
36.53	0.00201268569799566\\
36.54	0.00201268567493256\\
36.55	0.00201268565185687\\
36.56	0.00201268562876859\\
36.57	0.0020126856056677\\
36.58	0.0020126855825542\\
36.59	0.00201268555942808\\
36.6	0.00201268553628934\\
36.61	0.00201268551313797\\
36.62	0.00201268548997396\\
36.63	0.0020126854667973\\
36.64	0.00201268544360799\\
36.65	0.00201268542040602\\
36.66	0.00201268539719138\\
36.67	0.00201268537396407\\
36.68	0.00201268535072407\\
36.69	0.00201268532747139\\
36.7	0.00201268530420601\\
36.71	0.00201268528092793\\
36.72	0.00201268525763713\\
36.73	0.00201268523433362\\
36.74	0.00201268521101738\\
36.75	0.00201268518768841\\
36.76	0.0020126851643467\\
36.77	0.00201268514099224\\
36.78	0.00201268511762503\\
36.79	0.00201268509424506\\
36.8	0.00201268507085232\\
36.81	0.00201268504744681\\
36.82	0.00201268502402851\\
36.83	0.00201268500059742\\
36.84	0.00201268497715353\\
36.85	0.00201268495369684\\
36.86	0.00201268493022734\\
36.87	0.00201268490674502\\
36.88	0.00201268488324987\\
36.89	0.0020126848597419\\
36.9	0.00201268483622108\\
36.91	0.00201268481268741\\
36.92	0.00201268478914089\\
36.93	0.0020126847655815\\
36.94	0.00201268474200925\\
36.95	0.00201268471842412\\
36.96	0.00201268469482611\\
36.97	0.0020126846712152\\
36.98	0.0020126846475914\\
36.99	0.00201268462395469\\
37	0.00201268460030507\\
37.01	0.00201268457664253\\
37.02	0.00201268455296706\\
37.03	0.00201268452927866\\
37.04	0.00201268450557732\\
37.05	0.00201268448186303\\
37.06	0.00201268445813578\\
37.07	0.00201268443439557\\
37.08	0.00201268441064239\\
37.09	0.00201268438687623\\
37.1	0.00201268436309708\\
37.11	0.00201268433930495\\
37.12	0.00201268431549981\\
37.13	0.00201268429168167\\
37.14	0.00201268426785052\\
37.15	0.00201268424400634\\
37.16	0.00201268422014914\\
37.17	0.0020126841962789\\
37.18	0.00201268417239561\\
37.19	0.00201268414849928\\
37.2	0.00201268412458989\\
37.21	0.00201268410066744\\
37.22	0.00201268407673191\\
37.23	0.00201268405278331\\
37.24	0.00201268402882162\\
37.25	0.00201268400484683\\
37.26	0.00201268398085895\\
37.27	0.00201268395685796\\
37.28	0.00201268393284385\\
37.29	0.00201268390881662\\
37.3	0.00201268388477626\\
37.31	0.00201268386072276\\
37.32	0.00201268383665612\\
37.33	0.00201268381257633\\
37.34	0.00201268378848338\\
37.35	0.00201268376437726\\
37.36	0.00201268374025797\\
37.37	0.0020126837161255\\
37.38	0.00201268369197985\\
37.39	0.002012683667821\\
37.4	0.00201268364364894\\
37.41	0.00201268361946368\\
37.42	0.0020126835952652\\
37.43	0.0020126835710535\\
37.44	0.00201268354682856\\
37.45	0.00201268352259039\\
37.46	0.00201268349833897\\
37.47	0.0020126834740743\\
37.48	0.00201268344979637\\
37.49	0.00201268342550517\\
37.5	0.0020126834012007\\
37.51	0.00201268337688294\\
37.52	0.0020126833525519\\
37.53	0.00201268332820756\\
37.54	0.00201268330384992\\
37.55	0.00201268327947896\\
37.56	0.00201268325509469\\
37.57	0.00201268323069709\\
37.58	0.00201268320628616\\
37.59	0.00201268318186189\\
37.6	0.00201268315742428\\
37.61	0.00201268313297331\\
37.62	0.00201268310850897\\
37.63	0.00201268308403127\\
37.64	0.0020126830595402\\
37.65	0.00201268303503573\\
37.66	0.00201268301051788\\
37.67	0.00201268298598663\\
37.68	0.00201268296144198\\
37.69	0.00201268293688391\\
37.7	0.00201268291231243\\
37.71	0.00201268288772751\\
37.72	0.00201268286312916\\
37.73	0.00201268283851738\\
37.74	0.00201268281389214\\
37.75	0.00201268278925344\\
37.76	0.00201268276460129\\
37.77	0.00201268273993566\\
37.78	0.00201268271525656\\
37.79	0.00201268269056397\\
37.8	0.00201268266585788\\
37.81	0.0020126826411383\\
37.82	0.00201268261640522\\
37.83	0.00201268259165861\\
37.84	0.00201268256689849\\
37.85	0.00201268254212484\\
37.86	0.00201268251733765\\
37.87	0.00201268249253692\\
37.88	0.00201268246772264\\
37.89	0.0020126824428948\\
37.9	0.0020126824180534\\
37.91	0.00201268239319843\\
37.92	0.00201268236832987\\
37.93	0.00201268234344773\\
37.94	0.002012682318552\\
37.95	0.00201268229364266\\
37.96	0.00201268226871972\\
37.97	0.00201268224378316\\
37.98	0.00201268221883298\\
37.99	0.00201268219386917\\
38	0.00201268216889172\\
38.01	0.00201268214390062\\
38.02	0.00201268211889588\\
38.03	0.00201268209387747\\
38.04	0.0020126820688454\\
38.05	0.00201268204379965\\
38.06	0.00201268201874023\\
38.07	0.00201268199366712\\
38.08	0.00201268196858031\\
38.09	0.0020126819434798\\
38.1	0.00201268191836558\\
38.11	0.00201268189323764\\
38.12	0.00201268186809597\\
38.13	0.00201268184294058\\
38.14	0.00201268181777144\\
38.15	0.00201268179258857\\
38.16	0.00201268176739193\\
38.17	0.00201268174218154\\
38.18	0.00201268171695738\\
38.19	0.00201268169171944\\
38.2	0.00201268166646772\\
38.21	0.00201268164120222\\
38.22	0.00201268161592291\\
38.23	0.0020126815906298\\
38.24	0.00201268156532288\\
38.25	0.00201268154000214\\
38.26	0.00201268151466758\\
38.27	0.00201268148931918\\
38.28	0.00201268146395694\\
38.29	0.00201268143858085\\
38.3	0.00201268141319091\\
38.31	0.00201268138778711\\
38.32	0.00201268136236944\\
38.33	0.00201268133693789\\
38.34	0.00201268131149246\\
38.35	0.00201268128603314\\
38.36	0.00201268126055991\\
38.37	0.00201268123507278\\
38.38	0.00201268120957175\\
38.39	0.00201268118405679\\
38.4	0.0020126811585279\\
38.41	0.00201268113298508\\
38.42	0.00201268110742831\\
38.43	0.0020126810818576\\
38.44	0.00201268105627293\\
38.45	0.0020126810306743\\
38.46	0.0020126810050617\\
38.47	0.00201268097943512\\
38.48	0.00201268095379455\\
38.49	0.00201268092814\\
38.5	0.00201268090247144\\
38.51	0.00201268087678888\\
38.52	0.0020126808510923\\
38.53	0.0020126808253817\\
38.54	0.00201268079965707\\
38.55	0.00201268077391841\\
38.56	0.0020126807481657\\
38.57	0.00201268072239895\\
38.58	0.00201268069661813\\
38.59	0.00201268067082325\\
38.6	0.0020126806450143\\
38.61	0.00201268061919128\\
38.62	0.00201268059335416\\
38.63	0.00201268056750295\\
38.64	0.00201268054163764\\
38.65	0.00201268051575823\\
38.66	0.00201268048986469\\
38.67	0.00201268046395704\\
38.68	0.00201268043803525\\
38.69	0.00201268041209933\\
38.7	0.00201268038614927\\
38.71	0.00201268036018505\\
38.72	0.00201268033420668\\
38.73	0.00201268030821414\\
38.74	0.00201268028220743\\
38.75	0.00201268025618653\\
38.76	0.00201268023015145\\
38.77	0.00201268020410218\\
38.78	0.0020126801780387\\
38.79	0.00201268015196102\\
38.8	0.00201268012586911\\
38.81	0.00201268009976299\\
38.82	0.00201268007364264\\
38.83	0.00201268004750805\\
38.84	0.00201268002135921\\
38.85	0.00201267999519612\\
38.86	0.00201267996901877\\
38.87	0.00201267994282716\\
38.88	0.00201267991662127\\
38.89	0.0020126798904011\\
38.9	0.00201267986416664\\
38.91	0.00201267983791789\\
38.92	0.00201267981165484\\
38.93	0.00201267978537747\\
38.94	0.00201267975908579\\
38.95	0.00201267973277979\\
38.96	0.00201267970645946\\
38.97	0.00201267968012478\\
38.98	0.00201267965377576\\
38.99	0.00201267962741239\\
39	0.00201267960103466\\
39.01	0.00201267957464256\\
39.02	0.00201267954823609\\
39.03	0.00201267952181524\\
39.04	0.00201267949537999\\
39.05	0.00201267946893036\\
39.06	0.00201267944246632\\
39.07	0.00201267941598787\\
39.08	0.002012679389495\\
39.09	0.00201267936298771\\
39.1	0.00201267933646598\\
39.11	0.00201267930992982\\
39.12	0.00201267928337921\\
39.13	0.00201267925681415\\
39.14	0.00201267923023463\\
39.15	0.00201267920364064\\
39.16	0.00201267917703218\\
39.17	0.00201267915040924\\
39.18	0.00201267912377181\\
39.19	0.00201267909711988\\
39.2	0.00201267907045345\\
39.21	0.00201267904377251\\
39.22	0.00201267901707705\\
39.23	0.00201267899036706\\
39.24	0.00201267896364255\\
39.25	0.0020126789369035\\
39.26	0.0020126789101499\\
39.27	0.00201267888338175\\
39.28	0.00201267885659904\\
39.29	0.00201267882980176\\
39.3	0.0020126788029899\\
39.31	0.00201267877616347\\
39.32	0.00201267874932244\\
39.33	0.00201267872246682\\
39.34	0.0020126786955966\\
39.35	0.00201267866871177\\
39.36	0.00201267864181232\\
39.37	0.00201267861489825\\
39.38	0.00201267858796954\\
39.39	0.0020126785610262\\
39.4	0.00201267853406821\\
39.41	0.00201267850709557\\
39.42	0.00201267848010827\\
39.43	0.0020126784531063\\
39.44	0.00201267842608966\\
39.45	0.00201267839905834\\
39.46	0.00201267837201233\\
39.47	0.00201267834495162\\
39.48	0.00201267831787621\\
39.49	0.0020126782907861\\
39.5	0.00201267826368126\\
39.51	0.0020126782365617\\
39.52	0.00201267820942742\\
39.53	0.00201267818227839\\
39.54	0.00201267815511462\\
39.55	0.00201267812793609\\
39.56	0.00201267810074281\\
39.57	0.00201267807353476\\
39.58	0.00201267804631194\\
39.59	0.00201267801907434\\
39.6	0.00201267799182195\\
39.61	0.00201267796455477\\
39.62	0.00201267793727278\\
39.63	0.00201267790997599\\
39.64	0.00201267788266438\\
39.65	0.00201267785533794\\
39.66	0.00201267782799668\\
39.67	0.00201267780064058\\
39.68	0.00201267777326964\\
39.69	0.00201267774588385\\
39.7	0.0020126777184832\\
39.71	0.00201267769106768\\
39.72	0.00201267766363729\\
39.73	0.00201267763619202\\
39.74	0.00201267760873187\\
39.75	0.00201267758125682\\
39.76	0.00201267755376688\\
39.77	0.00201267752626202\\
39.78	0.00201267749874225\\
39.79	0.00201267747120756\\
39.8	0.00201267744365794\\
39.81	0.00201267741609339\\
39.82	0.00201267738851389\\
39.83	0.00201267736091945\\
39.84	0.00201267733331005\\
39.85	0.00201267730568568\\
39.86	0.00201267727804634\\
39.87	0.00201267725039203\\
39.88	0.00201267722272273\\
39.89	0.00201267719503844\\
39.9	0.00201267716733915\\
39.91	0.00201267713962486\\
39.92	0.00201267711189555\\
39.93	0.00201267708415123\\
39.94	0.00201267705639188\\
39.95	0.00201267702861749\\
39.96	0.00201267700082806\\
39.97	0.00201267697302359\\
39.98	0.00201267694520406\\
39.99	0.00201267691736947\\
40	0.00201267688951981\\
40.01	0.00201267686165508\\
};
\addplot [color=blue,dashed,forget plot]
  table[row sep=crcr]{%
40.01	0.00201267686165508\\
40.02	0.00201267683377526\\
40.03	0.00201267680588036\\
40.04	0.00201267677797036\\
40.05	0.00201267675004525\\
40.06	0.00201267672210504\\
40.07	0.0020126766941497\\
40.08	0.00201267666617925\\
40.09	0.00201267663819366\\
40.1	0.00201267661019293\\
40.11	0.00201267658217706\\
40.12	0.00201267655414603\\
40.13	0.00201267652609985\\
40.14	0.0020126764980385\\
40.15	0.00201267646996198\\
40.16	0.00201267644187028\\
40.17	0.00201267641376339\\
40.18	0.00201267638564131\\
40.19	0.00201267635750403\\
40.2	0.00201267632935154\\
40.21	0.00201267630118383\\
40.22	0.00201267627300091\\
40.23	0.00201267624480275\\
40.24	0.00201267621658936\\
40.25	0.00201267618836073\\
40.26	0.00201267616011685\\
40.27	0.00201267613185771\\
40.28	0.00201267610358331\\
40.29	0.00201267607529364\\
40.3	0.00201267604698869\\
40.31	0.00201267601866846\\
40.32	0.00201267599033294\\
40.33	0.00201267596198212\\
40.34	0.002012675933616\\
40.35	0.00201267590523456\\
40.36	0.00201267587683781\\
40.37	0.00201267584842573\\
40.38	0.00201267581999832\\
40.39	0.00201267579155557\\
40.4	0.00201267576309748\\
40.41	0.00201267573462403\\
40.42	0.00201267570613523\\
40.43	0.00201267567763106\\
40.44	0.00201267564911151\\
40.45	0.00201267562057658\\
40.46	0.00201267559202627\\
40.47	0.00201267556346056\\
40.48	0.00201267553487946\\
40.49	0.00201267550628294\\
40.5	0.00201267547767101\\
40.51	0.00201267544904366\\
40.52	0.00201267542040088\\
40.53	0.00201267539174267\\
40.54	0.00201267536306902\\
40.55	0.00201267533437991\\
40.56	0.00201267530567536\\
40.57	0.00201267527695533\\
40.58	0.00201267524821984\\
40.59	0.00201267521946888\\
40.6	0.00201267519070243\\
40.61	0.00201267516192049\\
40.62	0.00201267513312306\\
40.63	0.00201267510431012\\
40.64	0.00201267507548167\\
40.65	0.00201267504663771\\
40.66	0.00201267501777822\\
40.67	0.0020126749889032\\
40.68	0.00201267496001264\\
40.69	0.00201267493110654\\
40.7	0.00201267490218489\\
40.71	0.00201267487324768\\
40.72	0.00201267484429491\\
40.73	0.00201267481532657\\
40.74	0.00201267478634265\\
40.75	0.00201267475734314\\
40.76	0.00201267472832804\\
40.77	0.00201267469929735\\
40.78	0.00201267467025105\\
40.79	0.00201267464118914\\
40.8	0.0020126746121116\\
40.81	0.00201267458301845\\
40.82	0.00201267455390966\\
40.83	0.00201267452478523\\
40.84	0.00201267449564516\\
40.85	0.00201267446648943\\
40.86	0.00201267443731805\\
40.87	0.002012674408131\\
40.88	0.00201267437892827\\
40.89	0.00201267434970987\\
40.9	0.00201267432047579\\
40.91	0.002012674291226\\
40.92	0.00201267426196052\\
40.93	0.00201267423267934\\
40.94	0.00201267420338244\\
40.95	0.00201267417406982\\
40.96	0.00201267414474147\\
40.97	0.00201267411539739\\
40.98	0.00201267408603757\\
40.99	0.00201267405666201\\
41	0.00201267402727069\\
41.01	0.00201267399786361\\
41.02	0.00201267396844077\\
41.03	0.00201267393900215\\
41.04	0.00201267390954775\\
41.05	0.00201267388007757\\
41.06	0.00201267385059159\\
41.07	0.00201267382108981\\
41.08	0.00201267379157222\\
41.09	0.00201267376203883\\
41.1	0.00201267373248961\\
41.11	0.00201267370292456\\
41.12	0.00201267367334368\\
41.13	0.00201267364374696\\
41.14	0.0020126736141344\\
41.15	0.00201267358450598\\
41.16	0.0020126735548617\\
41.17	0.00201267352520155\\
41.18	0.00201267349552554\\
41.19	0.00201267346583364\\
41.2	0.00201267343612585\\
41.21	0.00201267340640217\\
41.22	0.00201267337666259\\
41.23	0.0020126733469071\\
41.24	0.0020126733171357\\
41.25	0.00201267328734838\\
41.26	0.00201267325754514\\
41.27	0.00201267322772596\\
41.28	0.00201267319789084\\
41.29	0.00201267316803978\\
41.3	0.00201267313817276\\
41.31	0.00201267310828978\\
41.32	0.00201267307839084\\
41.33	0.00201267304847592\\
41.34	0.00201267301854503\\
41.35	0.00201267298859814\\
41.36	0.00201267295863527\\
41.37	0.0020126729286564\\
41.38	0.00201267289866152\\
41.39	0.00201267286865063\\
41.4	0.00201267283862372\\
41.41	0.00201267280858078\\
41.42	0.00201267277852181\\
41.43	0.0020126727484468\\
41.44	0.00201267271835575\\
41.45	0.00201267268824865\\
41.46	0.00201267265812549\\
41.47	0.00201267262798626\\
41.48	0.00201267259783096\\
41.49	0.00201267256765958\\
41.5	0.00201267253747212\\
41.51	0.00201267250726857\\
41.52	0.00201267247704892\\
41.53	0.00201267244681317\\
41.54	0.0020126724165613\\
41.55	0.00201267238629332\\
41.56	0.00201267235600921\\
41.57	0.00201267232570898\\
41.58	0.0020126722953926\\
41.59	0.00201267226506009\\
41.6	0.00201267223471142\\
41.61	0.0020126722043466\\
41.62	0.00201267217396561\\
41.63	0.00201267214356845\\
41.64	0.00201267211315512\\
41.65	0.00201267208272561\\
41.66	0.0020126720522799\\
41.67	0.002012672021818\\
41.68	0.0020126719913399\\
41.69	0.00201267196084559\\
41.7	0.00201267193033507\\
41.71	0.00201267189980832\\
41.72	0.00201267186926535\\
41.73	0.00201267183870614\\
41.74	0.00201267180813069\\
41.75	0.00201267177753899\\
41.76	0.00201267174693104\\
41.77	0.00201267171630683\\
41.78	0.00201267168566635\\
41.79	0.0020126716550096\\
41.8	0.00201267162433657\\
41.81	0.00201267159364725\\
41.82	0.00201267156294164\\
41.83	0.00201267153221973\\
41.84	0.00201267150148151\\
41.85	0.00201267147072699\\
41.86	0.00201267143995614\\
41.87	0.00201267140916897\\
41.88	0.00201267137836547\\
41.89	0.00201267134754562\\
41.9	0.00201267131670944\\
41.91	0.0020126712858569\\
41.92	0.00201267125498801\\
41.93	0.00201267122410275\\
41.94	0.00201267119320113\\
41.95	0.00201267116228313\\
41.96	0.00201267113134874\\
41.97	0.00201267110039796\\
41.98	0.00201267106943079\\
41.99	0.00201267103844722\\
42	0.00201267100744724\\
42.01	0.00201267097643084\\
42.02	0.00201267094539802\\
42.03	0.00201267091434877\\
42.04	0.0020126708832831\\
42.05	0.00201267085220097\\
42.06	0.00201267082110241\\
42.07	0.00201267078998738\\
42.08	0.0020126707588559\\
42.09	0.00201267072770796\\
42.1	0.00201267069654354\\
42.11	0.00201267066536264\\
42.12	0.00201267063416525\\
42.13	0.00201267060295138\\
42.14	0.002012670571721\\
42.15	0.00201267054047412\\
42.16	0.00201267050921074\\
42.17	0.00201267047793083\\
42.18	0.0020126704466344\\
42.19	0.00201267041532144\\
42.2	0.00201267038399194\\
42.21	0.0020126703526459\\
42.22	0.00201267032128332\\
42.23	0.00201267028990417\\
42.24	0.00201267025850847\\
42.25	0.00201267022709619\\
42.26	0.00201267019566735\\
42.27	0.00201267016422192\\
42.28	0.0020126701327599\\
42.29	0.00201267010128129\\
42.3	0.00201267006978609\\
42.31	0.00201267003827427\\
42.32	0.00201267000674584\\
42.33	0.0020126699752008\\
42.34	0.00201266994363913\\
42.35	0.00201266991206082\\
42.36	0.00201266988046588\\
42.37	0.0020126698488543\\
42.38	0.00201266981722606\\
42.39	0.00201266978558117\\
42.4	0.00201266975391962\\
42.41	0.00201266972224139\\
42.42	0.00201266969054649\\
42.43	0.00201266965883491\\
42.44	0.00201266962710664\\
42.45	0.00201266959536168\\
42.46	0.00201266956360001\\
42.47	0.00201266953182164\\
42.48	0.00201266950002655\\
42.49	0.00201266946821474\\
42.5	0.00201266943638621\\
42.51	0.00201266940454095\\
42.52	0.00201266937267894\\
42.53	0.0020126693408002\\
42.54	0.0020126693089047\\
42.55	0.00201266927699244\\
42.56	0.00201266924506342\\
42.57	0.00201266921311762\\
42.58	0.00201266918115506\\
42.59	0.0020126691491757\\
42.6	0.00201266911717956\\
42.61	0.00201266908516663\\
42.62	0.00201266905313689\\
42.63	0.00201266902109035\\
42.64	0.00201266898902699\\
42.65	0.00201266895694681\\
42.66	0.0020126689248498\\
42.67	0.00201266889273596\\
42.68	0.00201266886060528\\
42.69	0.00201266882845776\\
42.7	0.00201266879629338\\
42.71	0.00201266876411215\\
42.72	0.00201266873191405\\
42.73	0.00201266869969908\\
42.74	0.00201266866746723\\
42.75	0.0020126686352185\\
42.76	0.00201266860295289\\
42.77	0.00201266857067037\\
42.78	0.00201266853837096\\
42.79	0.00201266850605463\\
42.8	0.00201266847372139\\
42.81	0.00201266844137123\\
42.82	0.00201266840900415\\
42.83	0.00201266837662012\\
42.84	0.00201266834421917\\
42.85	0.00201266831180126\\
42.86	0.0020126682793664\\
42.87	0.00201266824691459\\
42.88	0.00201266821444581\\
42.89	0.00201266818196006\\
42.9	0.00201266814945733\\
42.91	0.00201266811693762\\
42.92	0.00201266808440092\\
42.93	0.00201266805184722\\
42.94	0.00201266801927653\\
42.95	0.00201266798668882\\
42.96	0.0020126679540841\\
42.97	0.00201266792146236\\
42.98	0.00201266788882359\\
42.99	0.00201266785616779\\
43	0.00201266782349495\\
43.01	0.00201266779080506\\
43.02	0.00201266775809812\\
43.03	0.00201266772537412\\
43.04	0.00201266769263306\\
43.05	0.00201266765987493\\
43.06	0.00201266762709972\\
43.07	0.00201266759430743\\
43.08	0.00201266756149805\\
43.09	0.00201266752867157\\
43.1	0.00201266749582799\\
43.11	0.00201266746296731\\
43.12	0.0020126674300895\\
43.13	0.00201266739719458\\
43.14	0.00201266736428253\\
43.15	0.00201266733135335\\
43.16	0.00201266729840703\\
43.17	0.00201266726544356\\
43.18	0.00201266723246295\\
43.19	0.00201266719946517\\
43.2	0.00201266716645023\\
43.21	0.00201266713341812\\
43.22	0.00201266710036883\\
43.23	0.00201266706730236\\
43.24	0.00201266703421869\\
43.25	0.00201266700111784\\
43.26	0.00201266696799978\\
43.27	0.00201266693486451\\
43.28	0.00201266690171203\\
43.29	0.00201266686854233\\
43.3	0.00201266683535541\\
43.31	0.00201266680215124\\
43.32	0.00201266676892984\\
43.33	0.0020126667356912\\
43.34	0.0020126667024353\\
43.35	0.00201266666916215\\
43.36	0.00201266663587173\\
43.37	0.00201266660256404\\
43.38	0.00201266656923907\\
43.39	0.00201266653589682\\
43.4	0.00201266650253728\\
43.41	0.00201266646916044\\
43.42	0.00201266643576631\\
43.43	0.00201266640235486\\
43.44	0.0020126663689261\\
43.45	0.00201266633548002\\
43.46	0.00201266630201662\\
43.47	0.00201266626853588\\
43.48	0.0020126662350378\\
43.49	0.00201266620152237\\
43.5	0.00201266616798959\\
43.51	0.00201266613443946\\
43.52	0.00201266610087196\\
43.53	0.00201266606728709\\
43.54	0.00201266603368484\\
43.55	0.00201266600006521\\
43.56	0.00201266596642818\\
43.57	0.00201266593277377\\
43.58	0.00201266589910195\\
43.59	0.00201266586541272\\
43.6	0.00201266583170607\\
43.61	0.00201266579798201\\
43.62	0.00201266576424052\\
43.63	0.00201266573048159\\
43.64	0.00201266569670523\\
43.65	0.00201266566291141\\
43.66	0.00201266562910015\\
43.67	0.00201266559527142\\
43.68	0.00201266556142523\\
43.69	0.00201266552756157\\
43.7	0.00201266549368043\\
43.71	0.00201266545978181\\
43.72	0.0020126654258657\\
43.73	0.00201266539193208\\
43.74	0.00201266535798097\\
43.75	0.00201266532401235\\
43.76	0.0020126652900262\\
43.77	0.00201266525602254\\
43.78	0.00201266522200135\\
43.79	0.00201266518796262\\
43.8	0.00201266515390636\\
43.81	0.00201266511983254\\
43.82	0.00201266508574117\\
43.83	0.00201266505163224\\
43.84	0.00201266501750575\\
43.85	0.00201266498336167\\
43.86	0.00201266494920002\\
43.87	0.00201266491502079\\
43.88	0.00201266488082396\\
43.89	0.00201266484660954\\
43.9	0.0020126648123775\\
43.91	0.00201266477812786\\
43.92	0.0020126647438606\\
43.93	0.00201266470957571\\
43.94	0.0020126646752732\\
43.95	0.00201266464095304\\
43.96	0.00201266460661525\\
43.97	0.0020126645722598\\
43.98	0.0020126645378867\\
43.99	0.00201266450349593\\
44	0.00201266446908749\\
44.01	0.00201266443466138\\
44.02	0.00201266440021759\\
44.03	0.00201266436575611\\
44.04	0.00201266433127693\\
44.05	0.00201266429678005\\
44.06	0.00201266426226547\\
44.07	0.00201266422773316\\
44.08	0.00201266419318314\\
44.09	0.00201266415861539\\
44.1	0.00201266412402991\\
44.11	0.00201266408942669\\
44.12	0.00201266405480572\\
44.13	0.00201266402016699\\
44.14	0.00201266398551051\\
44.15	0.00201266395083626\\
44.16	0.00201266391614424\\
44.17	0.00201266388143444\\
44.18	0.00201266384670685\\
44.19	0.00201266381196147\\
44.2	0.00201266377719829\\
44.21	0.0020126637424173\\
44.22	0.0020126637076185\\
44.23	0.00201266367280189\\
44.24	0.00201266363796745\\
44.25	0.00201266360311517\\
44.26	0.00201266356824506\\
44.27	0.0020126635333571\\
44.28	0.0020126634984513\\
44.29	0.00201266346352763\\
44.3	0.0020126634285861\\
44.31	0.0020126633936267\\
44.32	0.00201266335864941\\
44.33	0.00201266332365425\\
44.34	0.00201266328864119\\
44.35	0.00201266325361024\\
44.36	0.00201266321856138\\
44.37	0.0020126631834946\\
44.38	0.00201266314840992\\
44.39	0.0020126631133073\\
44.4	0.00201266307818676\\
44.41	0.00201266304304827\\
44.42	0.00201266300789185\\
44.43	0.00201266297271747\\
44.44	0.00201266293752513\\
44.45	0.00201266290231483\\
44.46	0.00201266286708656\\
44.47	0.0020126628318403\\
44.48	0.00201266279657607\\
44.49	0.00201266276129384\\
44.5	0.00201266272599361\\
44.51	0.00201266269067538\\
44.52	0.00201266265533913\\
44.53	0.00201266261998487\\
44.54	0.00201266258461258\\
44.55	0.00201266254922226\\
44.56	0.00201266251381389\\
44.57	0.00201266247838749\\
44.58	0.00201266244294303\\
44.59	0.0020126624074805\\
44.6	0.00201266237199992\\
44.61	0.00201266233650126\\
44.62	0.00201266230098452\\
44.63	0.00201266226544969\\
44.64	0.00201266222989676\\
44.65	0.00201266219432574\\
44.66	0.00201266215873661\\
44.67	0.00201266212312936\\
44.68	0.00201266208750399\\
44.69	0.0020126620518605\\
44.7	0.00201266201619886\\
44.71	0.00201266198051908\\
44.72	0.00201266194482116\\
44.73	0.00201266190910507\\
44.74	0.00201266187337083\\
44.75	0.00201266183761841\\
44.76	0.00201266180184782\\
44.77	0.00201266176605904\\
44.78	0.00201266173025206\\
44.79	0.0020126616944269\\
44.8	0.00201266165858352\\
44.81	0.00201266162272193\\
44.82	0.00201266158684212\\
44.83	0.00201266155094408\\
44.84	0.00201266151502781\\
44.85	0.0020126614790933\\
44.86	0.00201266144314054\\
44.87	0.00201266140716952\\
44.88	0.00201266137118024\\
44.89	0.00201266133517269\\
44.9	0.00201266129914687\\
44.91	0.00201266126310276\\
44.92	0.00201266122704035\\
44.93	0.00201266119095965\\
44.94	0.00201266115486064\\
44.95	0.00201266111874332\\
44.96	0.00201266108260768\\
44.97	0.00201266104645371\\
44.98	0.00201266101028141\\
44.99	0.00201266097409076\\
45	0.00201266093788176\\
45.01	0.00201266090165441\\
45.02	0.00201266086540869\\
45.03	0.0020126608291446\\
45.04	0.00201266079286213\\
45.05	0.00201266075656128\\
45.06	0.00201266072024203\\
45.07	0.00201266068390438\\
45.08	0.00201266064754832\\
45.09	0.00201266061117384\\
45.1	0.00201266057478094\\
45.11	0.00201266053836961\\
45.12	0.00201266050193985\\
45.13	0.00201266046549163\\
45.14	0.00201266042902496\\
45.15	0.00201266039253984\\
45.16	0.00201266035603624\\
45.17	0.00201266031951416\\
45.18	0.00201266028297361\\
45.19	0.00201266024641456\\
45.2	0.00201266020983701\\
45.21	0.00201266017324096\\
45.22	0.00201266013662639\\
45.23	0.0020126600999933\\
45.24	0.00201266006334168\\
45.25	0.00201266002667152\\
45.26	0.00201265998998282\\
45.27	0.00201265995327556\\
45.28	0.00201265991654975\\
45.29	0.00201265987980537\\
45.3	0.00201265984304241\\
45.31	0.00201265980626087\\
45.32	0.00201265976946074\\
45.33	0.002012659732642\\
45.34	0.00201265969580467\\
45.35	0.00201265965894871\\
45.36	0.00201265962207414\\
45.37	0.00201265958518093\\
45.38	0.00201265954826908\\
45.39	0.00201265951133859\\
45.4	0.00201265947438945\\
45.41	0.00201265943742164\\
45.42	0.00201265940043516\\
45.43	0.00201265936343\\
45.44	0.00201265932640616\\
45.45	0.00201265928936362\\
45.46	0.00201265925230238\\
45.47	0.00201265921522243\\
45.48	0.00201265917812376\\
45.49	0.00201265914100637\\
45.5	0.00201265910387024\\
45.51	0.00201265906671537\\
45.52	0.00201265902954175\\
45.53	0.00201265899234936\\
45.54	0.00201265895513821\\
45.55	0.00201265891790829\\
45.56	0.00201265888065958\\
45.57	0.00201265884339208\\
45.58	0.00201265880610578\\
45.59	0.00201265876880067\\
45.6	0.00201265873147675\\
45.61	0.00201265869413399\\
45.62	0.00201265865677241\\
45.63	0.00201265861939198\\
45.64	0.00201265858199271\\
45.65	0.00201265854457457\\
45.66	0.00201265850713757\\
45.67	0.00201265846968169\\
45.68	0.00201265843220693\\
45.69	0.00201265839471327\\
45.7	0.00201265835720072\\
45.71	0.00201265831966926\\
45.72	0.00201265828211887\\
45.73	0.00201265824454956\\
45.74	0.00201265820696131\\
45.75	0.00201265816935413\\
45.76	0.00201265813172798\\
45.77	0.00201265809408288\\
45.78	0.00201265805641881\\
45.79	0.00201265801873576\\
45.8	0.00201265798103372\\
45.81	0.00201265794331268\\
45.82	0.00201265790557264\\
45.83	0.00201265786781359\\
45.84	0.00201265783003551\\
45.85	0.00201265779223841\\
45.86	0.00201265775442226\\
45.87	0.00201265771658706\\
45.88	0.00201265767873281\\
45.89	0.00201265764085948\\
45.9	0.00201265760296709\\
45.91	0.0020126575650556\\
45.92	0.00201265752712503\\
45.93	0.00201265748917535\\
45.94	0.00201265745120656\\
45.95	0.00201265741321864\\
45.96	0.0020126573752116\\
45.97	0.00201265733718542\\
45.98	0.00201265729914009\\
45.99	0.0020126572610756\\
46	0.00201265722299194\\
46.01	0.00201265718488911\\
46.02	0.00201265714676709\\
46.03	0.00201265710862588\\
46.04	0.00201265707046547\\
46.05	0.00201265703228584\\
46.06	0.00201265699408699\\
46.07	0.0020126569558689\\
46.08	0.00201265691763158\\
46.09	0.002012656879375\\
46.1	0.00201265684109917\\
46.11	0.00201265680280406\\
46.12	0.00201265676448968\\
46.13	0.00201265672615601\\
46.14	0.00201265668780303\\
46.15	0.00201265664943076\\
46.16	0.00201265661103916\\
46.17	0.00201265657262823\\
46.18	0.00201265653419797\\
46.19	0.00201265649574837\\
46.2	0.0020126564572794\\
46.21	0.00201265641879107\\
46.22	0.00201265638028337\\
46.23	0.00201265634175628\\
46.24	0.00201265630320979\\
46.25	0.0020126562646439\\
46.26	0.00201265622605859\\
46.27	0.00201265618745386\\
46.28	0.00201265614882969\\
46.29	0.00201265611018609\\
46.3	0.00201265607152302\\
46.31	0.00201265603284049\\
46.32	0.00201265599413848\\
46.33	0.00201265595541699\\
46.34	0.00201265591667601\\
46.35	0.00201265587791552\\
46.36	0.00201265583913551\\
46.37	0.00201265580033598\\
46.38	0.00201265576151691\\
46.39	0.0020126557226783\\
46.4	0.00201265568382013\\
46.41	0.00201265564494239\\
46.42	0.00201265560604508\\
46.43	0.00201265556712818\\
46.44	0.00201265552819168\\
46.45	0.00201265548923558\\
46.46	0.00201265545025985\\
46.47	0.0020126554112645\\
46.48	0.00201265537224951\\
46.49	0.00201265533321487\\
46.5	0.00201265529416056\\
46.51	0.00201265525508659\\
46.52	0.00201265521599293\\
46.53	0.00201265517687959\\
46.54	0.00201265513774654\\
46.55	0.00201265509859377\\
46.56	0.00201265505942128\\
46.57	0.00201265502022906\\
46.58	0.00201265498101709\\
46.59	0.00201265494178536\\
46.6	0.00201265490253387\\
46.61	0.00201265486326259\\
46.62	0.00201265482397153\\
46.63	0.00201265478466067\\
46.64	0.00201265474532999\\
46.65	0.0020126547059795\\
46.66	0.00201265466660917\\
46.67	0.00201265462721899\\
46.68	0.00201265458780896\\
46.69	0.00201265454837906\\
46.7	0.00201265450892929\\
46.71	0.00201265446945962\\
46.72	0.00201265442997006\\
46.73	0.00201265439046058\\
46.74	0.00201265435093119\\
46.75	0.00201265431138185\\
46.76	0.00201265427181257\\
46.77	0.00201265423222334\\
46.78	0.00201265419261413\\
46.79	0.00201265415298495\\
46.8	0.00201265411333578\\
46.81	0.0020126540736666\\
46.82	0.00201265403397741\\
46.83	0.00201265399426819\\
46.84	0.00201265395453893\\
46.85	0.00201265391478963\\
46.86	0.00201265387502026\\
46.87	0.00201265383523083\\
46.88	0.0020126537954213\\
46.89	0.00201265375559169\\
46.9	0.00201265371574196\\
46.91	0.00201265367587212\\
46.92	0.00201265363598214\\
46.93	0.00201265359607202\\
46.94	0.00201265355614175\\
46.95	0.0020126535161913\\
46.96	0.00201265347622068\\
46.97	0.00201265343622987\\
46.98	0.00201265339621886\\
46.99	0.00201265335618763\\
47	0.00201265331613617\\
47.01	0.00201265327606447\\
47.02	0.00201265323597252\\
47.03	0.00201265319586031\\
47.04	0.00201265315572782\\
47.05	0.00201265311557504\\
47.06	0.00201265307540197\\
47.07	0.00201265303520857\\
47.08	0.00201265299499486\\
47.09	0.0020126529547608\\
47.1	0.0020126529145064\\
47.11	0.00201265287423163\\
47.12	0.00201265283393649\\
47.13	0.00201265279362096\\
47.14	0.00201265275328503\\
47.15	0.00201265271292869\\
47.16	0.00201265267255192\\
47.17	0.00201265263215471\\
47.18	0.00201265259173705\\
47.19	0.00201265255129893\\
47.2	0.00201265251084034\\
47.21	0.00201265247036125\\
47.22	0.00201265242986166\\
47.23	0.00201265238934156\\
47.24	0.00201265234880093\\
47.25	0.00201265230823976\\
47.26	0.00201265226765803\\
47.27	0.00201265222705574\\
47.28	0.00201265218643287\\
47.29	0.00201265214578941\\
47.3	0.00201265210512534\\
47.31	0.00201265206444065\\
47.32	0.00201265202373533\\
47.33	0.00201265198300937\\
47.34	0.00201265194226274\\
47.35	0.00201265190149545\\
47.36	0.00201265186070747\\
47.37	0.0020126518198988\\
47.38	0.00201265177906941\\
47.39	0.0020126517382193\\
47.4	0.00201265169734845\\
47.41	0.00201265165645684\\
47.42	0.00201265161554448\\
47.43	0.00201265157461133\\
47.44	0.00201265153365739\\
47.45	0.00201265149268265\\
47.46	0.00201265145168709\\
47.47	0.00201265141067069\\
47.48	0.00201265136963345\\
47.49	0.00201265132857535\\
47.5	0.00201265128749638\\
47.51	0.00201265124639652\\
47.52	0.00201265120527575\\
47.53	0.00201265116413408\\
47.54	0.00201265112297147\\
47.55	0.00201265108178792\\
47.56	0.00201265104058341\\
47.57	0.00201265099935794\\
47.58	0.00201265095811148\\
47.59	0.00201265091684402\\
47.6	0.00201265087555554\\
47.61	0.00201265083424605\\
47.62	0.00201265079291551\\
47.63	0.00201265075156391\\
47.64	0.00201265071019125\\
47.65	0.00201265066879751\\
47.66	0.00201265062738266\\
47.67	0.00201265058594671\\
47.68	0.00201265054448963\\
47.69	0.00201265050301141\\
47.7	0.00201265046151204\\
47.71	0.0020126504199915\\
47.72	0.00201265037844978\\
47.73	0.00201265033688685\\
47.74	0.00201265029530272\\
47.75	0.00201265025369736\\
47.76	0.00201265021207077\\
47.77	0.00201265017042291\\
47.78	0.00201265012875379\\
47.79	0.00201265008706338\\
47.8	0.00201265004535168\\
47.81	0.00201265000361866\\
47.82	0.00201264996186431\\
47.83	0.00201264992008862\\
47.84	0.00201264987829157\\
47.85	0.00201264983647315\\
47.86	0.00201264979463335\\
47.87	0.00201264975277214\\
47.88	0.00201264971088951\\
47.89	0.00201264966898546\\
47.9	0.00201264962705996\\
47.91	0.00201264958511299\\
47.92	0.00201264954314456\\
47.93	0.00201264950115463\\
47.94	0.00201264945914319\\
47.95	0.00201264941711024\\
47.96	0.00201264937505575\\
47.97	0.00201264933297971\\
47.98	0.0020126492908821\\
47.99	0.00201264924876291\\
48	0.00201264920662213\\
48.01	0.00201264916445973\\
48.02	0.00201264912227571\\
48.03	0.00201264908007004\\
48.04	0.00201264903784272\\
48.05	0.00201264899559373\\
48.06	0.00201264895332305\\
48.07	0.00201264891103067\\
48.08	0.00201264886871657\\
48.09	0.00201264882638073\\
48.1	0.00201264878402315\\
48.11	0.00201264874164381\\
48.12	0.00201264869924268\\
48.13	0.00201264865681976\\
48.14	0.00201264861437503\\
48.15	0.00201264857190847\\
48.16	0.00201264852942007\\
48.17	0.00201264848690982\\
48.18	0.00201264844437769\\
48.19	0.00201264840182368\\
48.2	0.00201264835924776\\
48.21	0.00201264831664992\\
48.22	0.00201264827403015\\
48.23	0.00201264823138843\\
48.24	0.00201264818872474\\
48.25	0.00201264814603907\\
48.26	0.0020126481033314\\
48.27	0.00201264806060172\\
48.28	0.00201264801785001\\
48.29	0.00201264797507626\\
48.3	0.00201264793228044\\
48.31	0.00201264788946255\\
48.32	0.00201264784662256\\
48.33	0.00201264780376047\\
48.34	0.00201264776087626\\
48.35	0.0020126477179699\\
48.36	0.00201264767504138\\
48.37	0.0020126476320907\\
48.38	0.00201264758911783\\
48.39	0.00201264754612275\\
48.4	0.00201264750310545\\
48.41	0.00201264746006592\\
48.42	0.00201264741700414\\
48.43	0.00201264737392008\\
48.44	0.00201264733081375\\
48.45	0.00201264728768511\\
48.46	0.00201264724453415\\
48.47	0.00201264720136087\\
48.48	0.00201264715816523\\
48.49	0.00201264711494723\\
48.5	0.00201264707170685\\
48.51	0.00201264702844407\\
48.52	0.00201264698515887\\
48.53	0.00201264694185125\\
48.54	0.00201264689852118\\
48.55	0.00201264685516865\\
48.56	0.00201264681179363\\
48.57	0.00201264676839613\\
48.58	0.00201264672497611\\
48.59	0.00201264668153356\\
48.6	0.00201264663806846\\
48.61	0.00201264659458081\\
48.62	0.00201264655107058\\
48.63	0.00201264650753775\\
48.64	0.00201264646398231\\
48.65	0.00201264642040425\\
48.66	0.00201264637680354\\
48.67	0.00201264633318017\\
48.68	0.00201264628953412\\
48.69	0.00201264624586538\\
48.7	0.00201264620217393\\
48.71	0.00201264615845976\\
48.72	0.00201264611472283\\
48.73	0.00201264607096315\\
48.74	0.00201264602718069\\
48.75	0.00201264598337544\\
48.76	0.00201264593954738\\
48.77	0.00201264589569649\\
48.78	0.00201264585182275\\
48.79	0.00201264580792615\\
48.8	0.00201264576400668\\
48.81	0.00201264572006431\\
48.82	0.00201264567609903\\
48.83	0.00201264563211082\\
48.84	0.00201264558809966\\
48.85	0.00201264554406554\\
48.86	0.00201264550000844\\
48.87	0.00201264545592835\\
48.88	0.00201264541182524\\
48.89	0.00201264536769909\\
48.9	0.0020126453235499\\
48.91	0.00201264527937765\\
48.92	0.00201264523518232\\
48.93	0.00201264519096388\\
48.94	0.00201264514672233\\
48.95	0.00201264510245764\\
48.96	0.0020126450581698\\
48.97	0.0020126450138588\\
48.98	0.00201264496952461\\
48.99	0.00201264492516721\\
49	0.0020126448807866\\
49.01	0.00201264483638275\\
49.02	0.00201264479195564\\
49.03	0.00201264474750526\\
49.04	0.00201264470303159\\
49.05	0.00201264465853462\\
49.06	0.00201264461401432\\
49.07	0.00201264456947067\\
49.08	0.00201264452490367\\
49.09	0.00201264448031329\\
49.1	0.00201264443569952\\
49.11	0.00201264439106233\\
49.12	0.00201264434640172\\
49.13	0.00201264430171766\\
49.14	0.00201264425701013\\
49.15	0.00201264421227912\\
49.16	0.00201264416752461\\
49.17	0.00201264412274658\\
49.18	0.00201264407794501\\
49.19	0.00201264403311989\\
49.2	0.0020126439882712\\
49.21	0.00201264394339892\\
49.22	0.00201264389850304\\
49.23	0.00201264385358352\\
49.24	0.00201264380864037\\
49.25	0.00201264376367355\\
49.26	0.00201264371868306\\
49.27	0.00201264367366887\\
49.28	0.00201264362863097\\
49.29	0.00201264358356933\\
49.3	0.00201264353848394\\
49.31	0.00201264349337479\\
49.32	0.00201264344824184\\
49.33	0.00201264340308509\\
49.34	0.00201264335790452\\
49.35	0.00201264331270011\\
49.36	0.00201264326747184\\
49.37	0.00201264322221969\\
49.38	0.00201264317694365\\
49.39	0.00201264313164369\\
49.4	0.0020126430863198\\
49.41	0.00201264304097196\\
49.42	0.00201264299560015\\
49.43	0.00201264295020435\\
49.44	0.00201264290478455\\
49.45	0.00201264285934072\\
49.46	0.00201264281387285\\
49.47	0.00201264276838092\\
49.48	0.00201264272286492\\
49.49	0.00201264267732481\\
49.5	0.00201264263176059\\
49.51	0.00201264258617223\\
49.52	0.00201264254055972\\
49.53	0.00201264249492304\\
49.54	0.00201264244926216\\
49.55	0.00201264240357708\\
49.56	0.00201264235786777\\
49.57	0.00201264231213421\\
49.58	0.00201264226637639\\
49.59	0.00201264222059428\\
49.6	0.00201264217478787\\
49.61	0.00201264212895714\\
49.62	0.00201264208310207\\
49.63	0.00201264203722264\\
49.64	0.00201264199131883\\
49.65	0.00201264194539062\\
49.66	0.002012641899438\\
49.67	0.00201264185346095\\
49.68	0.00201264180745943\\
49.69	0.00201264176143345\\
49.7	0.00201264171538297\\
49.71	0.00201264166930798\\
49.72	0.00201264162320846\\
49.73	0.0020126415770844\\
49.74	0.00201264153093576\\
49.75	0.00201264148476254\\
49.76	0.00201264143856471\\
49.77	0.00201264139234225\\
49.78	0.00201264134609515\\
49.79	0.00201264129982338\\
49.8	0.00201264125352693\\
49.81	0.00201264120720578\\
49.82	0.0020126411608599\\
49.83	0.00201264111448928\\
49.84	0.0020126410680939\\
49.85	0.00201264102167374\\
49.86	0.00201264097522878\\
49.87	0.002012640928759\\
49.88	0.00201264088226438\\
49.89	0.0020126408357449\\
49.9	0.00201264078920054\\
49.91	0.00201264074263128\\
49.92	0.00201264069603711\\
49.93	0.002012640649418\\
49.94	0.00201264060277393\\
49.95	0.00201264055610488\\
49.96	0.00201264050941084\\
49.97	0.00201264046269178\\
49.98	0.00201264041594769\\
49.99	0.00201264036917854\\
50	0.00201264032238431\\
50.01	0.00201264027556499\\
50.02	0.00201264022872055\\
50.03	0.00201264018185098\\
50.04	0.00201264013495625\\
50.05	0.00201264008803635\\
50.06	0.00201264004109125\\
50.07	0.00201263999412093\\
50.08	0.00201263994712538\\
50.09	0.00201263990010457\\
50.1	0.00201263985305849\\
50.11	0.00201263980598711\\
50.12	0.00201263975889042\\
50.13	0.00201263971176838\\
50.14	0.002012639664621\\
50.15	0.00201263961744823\\
50.16	0.00201263957025006\\
50.17	0.00201263952302648\\
50.18	0.00201263947577746\\
50.19	0.00201263942850298\\
50.2	0.00201263938120303\\
50.21	0.00201263933387757\\
50.22	0.0020126392865266\\
50.23	0.00201263923915008\\
50.24	0.002012639191748\\
50.25	0.00201263914432034\\
50.26	0.00201263909686708\\
50.27	0.00201263904938819\\
50.28	0.00201263900188367\\
50.29	0.00201263895435347\\
50.3	0.0020126389067976\\
50.31	0.00201263885921601\\
50.32	0.0020126388116087\\
50.33	0.00201263876397565\\
50.34	0.00201263871631682\\
50.35	0.0020126386686322\\
50.36	0.00201263862092178\\
50.37	0.00201263857318553\\
50.38	0.00201263852542342\\
50.39	0.00201263847763544\\
50.4	0.00201263842982156\\
50.41	0.00201263838198177\\
50.42	0.00201263833411605\\
50.43	0.00201263828622436\\
50.44	0.0020126382383067\\
50.45	0.00201263819036304\\
50.46	0.00201263814239336\\
50.47	0.00201263809439764\\
50.48	0.00201263804637585\\
50.49	0.00201263799832798\\
50.5	0.002012637950254\\
50.51	0.0020126379021539\\
50.52	0.00201263785402764\\
50.53	0.00201263780587522\\
50.54	0.0020126377576966\\
50.55	0.00201263770949177\\
50.56	0.00201263766126071\\
50.57	0.00201263761300339\\
50.58	0.00201263756471979\\
50.59	0.00201263751640989\\
50.6	0.00201263746807367\\
50.61	0.00201263741971111\\
50.62	0.00201263737132218\\
50.63	0.00201263732290687\\
50.64	0.00201263727446515\\
50.65	0.002012637225997\\
50.66	0.0020126371775024\\
50.67	0.00201263712898132\\
50.68	0.00201263708043375\\
50.69	0.00201263703185966\\
50.7	0.00201263698325903\\
50.71	0.00201263693463183\\
50.72	0.00201263688597806\\
50.73	0.00201263683729768\\
50.74	0.00201263678859067\\
50.75	0.00201263673985701\\
50.76	0.00201263669109668\\
50.77	0.00201263664230965\\
50.78	0.00201263659349591\\
50.79	0.00201263654465543\\
50.8	0.00201263649578819\\
50.81	0.00201263644689416\\
50.82	0.00201263639797333\\
50.83	0.00201263634902567\\
50.84	0.00201263630005115\\
50.85	0.00201263625104977\\
50.86	0.00201263620202149\\
50.87	0.00201263615296629\\
50.88	0.00201263610388415\\
50.89	0.00201263605477504\\
50.9	0.00201263600563895\\
50.91	0.00201263595647585\\
50.92	0.00201263590728572\\
50.93	0.00201263585806853\\
50.94	0.00201263580882427\\
50.95	0.0020126357595529\\
50.96	0.00201263571025442\\
50.97	0.00201263566092878\\
50.98	0.00201263561157598\\
50.99	0.00201263556219599\\
51	0.00201263551278878\\
51.01	0.00201263546335433\\
51.02	0.00201263541389263\\
51.03	0.00201263536440363\\
51.04	0.00201263531488734\\
51.05	0.00201263526534371\\
51.06	0.00201263521577272\\
51.07	0.00201263516617436\\
51.08	0.0020126351165486\\
51.09	0.00201263506689542\\
51.1	0.00201263501721479\\
51.11	0.00201263496750669\\
51.12	0.0020126349177711\\
51.13	0.00201263486800798\\
51.14	0.00201263481821733\\
51.15	0.00201263476839912\\
51.16	0.00201263471855331\\
51.17	0.0020126346686799\\
51.18	0.00201263461877885\\
51.19	0.00201263456885014\\
51.2	0.00201263451889374\\
51.21	0.00201263446890965\\
51.22	0.00201263441889782\\
51.23	0.00201263436885824\\
51.24	0.00201263431879088\\
51.25	0.00201263426869572\\
51.26	0.00201263421857273\\
51.27	0.0020126341684219\\
51.28	0.00201263411824319\\
51.29	0.00201263406803658\\
51.3	0.00201263401780205\\
51.31	0.00201263396753958\\
51.32	0.00201263391724914\\
51.33	0.0020126338669307\\
51.34	0.00201263381658424\\
51.35	0.00201263376620974\\
51.36	0.00201263371580717\\
51.37	0.00201263366537652\\
51.38	0.00201263361491774\\
51.39	0.00201263356443083\\
51.4	0.00201263351391575\\
51.41	0.00201263346337248\\
51.42	0.002012633412801\\
51.43	0.00201263336220128\\
51.44	0.0020126333115733\\
51.45	0.00201263326091703\\
51.46	0.00201263321023244\\
51.47	0.00201263315951952\\
51.48	0.00201263310877824\\
51.49	0.00201263305800858\\
51.5	0.00201263300721049\\
51.51	0.00201263295638398\\
51.52	0.002012632905529\\
51.53	0.00201263285464554\\
51.54	0.00201263280373356\\
51.55	0.00201263275279305\\
51.56	0.00201263270182398\\
51.57	0.00201263265082632\\
51.58	0.00201263259980005\\
51.59	0.00201263254874514\\
51.6	0.00201263249766157\\
51.61	0.00201263244654931\\
51.62	0.00201263239540834\\
51.63	0.00201263234423863\\
51.64	0.00201263229304016\\
51.65	0.0020126322418129\\
51.66	0.00201263219055682\\
51.67	0.0020126321392719\\
51.68	0.00201263208795812\\
51.69	0.00201263203661545\\
51.7	0.00201263198524386\\
51.71	0.00201263193384332\\
51.72	0.00201263188241382\\
51.73	0.00201263183095532\\
51.74	0.00201263177946781\\
51.75	0.00201263172795125\\
51.76	0.00201263167640561\\
51.77	0.00201263162483088\\
51.78	0.00201263157322703\\
51.79	0.00201263152159402\\
51.8	0.00201263146993184\\
51.81	0.00201263141824046\\
51.82	0.00201263136651984\\
51.83	0.00201263131476998\\
51.84	0.00201263126299083\\
51.85	0.00201263121118238\\
51.86	0.00201263115934459\\
51.87	0.00201263110747745\\
51.88	0.00201263105558092\\
51.89	0.00201263100365497\\
51.9	0.00201263095169959\\
51.91	0.00201263089971474\\
51.92	0.0020126308477004\\
51.93	0.00201263079565654\\
51.94	0.00201263074358313\\
51.95	0.00201263069148016\\
51.96	0.00201263063934758\\
51.97	0.00201263058718538\\
51.98	0.00201263053499352\\
51.99	0.00201263048277199\\
52	0.00201263043052075\\
52.01	0.00201263037823978\\
52.02	0.00201263032592905\\
52.03	0.00201263027358852\\
52.04	0.00201263022121819\\
52.05	0.00201263016881801\\
52.06	0.00201263011638797\\
52.07	0.00201263006392803\\
52.08	0.00201263001143816\\
52.09	0.00201262995891835\\
52.1	0.00201262990636855\\
52.11	0.00201262985378875\\
52.12	0.00201262980117892\\
52.13	0.00201262974853903\\
52.14	0.00201262969586905\\
52.15	0.00201262964316895\\
52.16	0.00201262959043871\\
52.17	0.0020126295376783\\
52.18	0.00201262948488769\\
52.19	0.00201262943206685\\
52.2	0.00201262937921576\\
52.21	0.00201262932633439\\
52.22	0.00201262927342271\\
52.23	0.00201262922048069\\
52.24	0.0020126291675083\\
52.25	0.00201262911450552\\
52.26	0.00201262906147232\\
52.27	0.00201262900840867\\
52.28	0.00201262895531453\\
52.29	0.0020126289021899\\
52.3	0.00201262884903472\\
52.31	0.00201262879584898\\
52.32	0.00201262874263266\\
52.33	0.00201262868938571\\
52.34	0.00201262863610811\\
52.35	0.00201262858279983\\
52.36	0.00201262852946085\\
52.37	0.00201262847609113\\
52.38	0.00201262842269065\\
52.39	0.00201262836925938\\
52.4	0.00201262831579729\\
52.41	0.00201262826230435\\
52.42	0.00201262820878052\\
52.43	0.0020126281552258\\
52.44	0.00201262810164013\\
52.45	0.0020126280480235\\
52.46	0.00201262799437588\\
52.47	0.00201262794069723\\
52.48	0.00201262788698753\\
52.49	0.00201262783324675\\
52.5	0.00201262777947486\\
52.51	0.00201262772567182\\
52.52	0.00201262767183762\\
52.53	0.00201262761797222\\
52.54	0.00201262756407559\\
52.55	0.0020126275101477\\
52.56	0.00201262745618852\\
52.57	0.00201262740219803\\
52.58	0.00201262734817618\\
52.59	0.00201262729412297\\
52.6	0.00201262724003834\\
52.61	0.00201262718592228\\
52.62	0.00201262713177476\\
52.63	0.00201262707759573\\
52.64	0.00201262702338518\\
52.65	0.00201262696914308\\
52.66	0.00201262691486939\\
52.67	0.00201262686056408\\
52.68	0.00201262680622713\\
52.69	0.0020126267518585\\
52.7	0.00201262669745816\\
52.71	0.00201262664302609\\
52.72	0.00201262658856225\\
52.73	0.00201262653406661\\
52.74	0.00201262647953914\\
52.75	0.00201262642497981\\
52.76	0.00201262637038859\\
52.77	0.00201262631576546\\
52.78	0.00201262626111037\\
52.79	0.0020126262064233\\
52.8	0.00201262615170422\\
52.81	0.0020126260969531\\
52.82	0.0020126260421699\\
52.83	0.00201262598735459\\
52.84	0.00201262593250716\\
52.85	0.00201262587762755\\
52.86	0.00201262582271574\\
52.87	0.00201262576777171\\
52.88	0.00201262571279541\\
52.89	0.00201262565778683\\
52.9	0.00201262560274592\\
52.91	0.00201262554767265\\
52.92	0.002012625492567\\
52.93	0.00201262543742893\\
52.94	0.00201262538225842\\
52.95	0.00201262532705542\\
52.96	0.00201262527181991\\
52.97	0.00201262521655186\\
52.98	0.00201262516125122\\
52.99	0.00201262510591799\\
53	0.00201262505055211\\
53.01	0.00201262499515356\\
53.02	0.00201262493972231\\
53.03	0.00201262488425832\\
53.04	0.00201262482876157\\
53.05	0.00201262477323201\\
53.06	0.00201262471766963\\
53.07	0.00201262466207438\\
53.08	0.00201262460644623\\
53.09	0.00201262455078516\\
53.1	0.00201262449509112\\
53.11	0.00201262443936409\\
53.12	0.00201262438360403\\
53.13	0.00201262432781092\\
53.14	0.00201262427198471\\
53.15	0.00201262421612538\\
53.16	0.00201262416023289\\
53.17	0.00201262410430721\\
53.18	0.00201262404834831\\
53.19	0.00201262399235615\\
53.2	0.00201262393633071\\
53.21	0.00201262388027194\\
53.22	0.00201262382417982\\
53.23	0.00201262376805431\\
53.24	0.00201262371189537\\
53.25	0.00201262365570299\\
53.26	0.00201262359947711\\
53.27	0.00201262354321772\\
53.28	0.00201262348692476\\
53.29	0.00201262343059823\\
53.3	0.00201262337423806\\
53.31	0.00201262331784425\\
53.32	0.00201262326141674\\
53.33	0.00201262320495551\\
53.34	0.00201262314846052\\
53.35	0.00201262309193175\\
53.36	0.00201262303536914\\
53.37	0.00201262297877268\\
53.38	0.00201262292214233\\
53.39	0.00201262286547804\\
53.4	0.0020126228087798\\
53.41	0.00201262275204756\\
53.42	0.00201262269528129\\
53.43	0.00201262263848096\\
53.44	0.00201262258164653\\
53.45	0.00201262252477796\\
53.46	0.00201262246787523\\
53.47	0.0020126224109383\\
53.48	0.00201262235396712\\
53.49	0.00201262229696168\\
53.5	0.00201262223992193\\
53.51	0.00201262218284783\\
53.52	0.00201262212573937\\
53.53	0.00201262206859649\\
53.54	0.00201262201141916\\
53.55	0.00201262195420735\\
53.56	0.00201262189696102\\
53.57	0.00201262183968015\\
53.58	0.00201262178236468\\
53.59	0.00201262172501459\\
53.6	0.00201262166762984\\
53.61	0.0020126216102104\\
53.62	0.00201262155275623\\
53.63	0.0020126214952673\\
53.64	0.00201262143774356\\
53.65	0.00201262138018499\\
53.66	0.00201262132259154\\
53.67	0.00201262126496319\\
53.68	0.00201262120729989\\
53.69	0.00201262114960161\\
53.7	0.00201262109186832\\
53.71	0.00201262103409997\\
53.72	0.00201262097629654\\
53.73	0.00201262091845798\\
53.74	0.00201262086058425\\
53.75	0.00201262080267534\\
53.76	0.00201262074473118\\
53.77	0.00201262068675176\\
53.78	0.00201262062873702\\
53.79	0.00201262057068695\\
53.8	0.00201262051260149\\
53.81	0.00201262045448061\\
53.82	0.00201262039632428\\
53.83	0.00201262033813246\\
53.84	0.00201262027990511\\
53.85	0.00201262022164219\\
53.86	0.00201262016334367\\
53.87	0.00201262010500951\\
53.88	0.00201262004663967\\
53.89	0.00201261998823412\\
53.9	0.00201261992979282\\
53.91	0.00201261987131572\\
53.92	0.0020126198128028\\
53.93	0.00201261975425401\\
53.94	0.00201261969566932\\
53.95	0.00201261963704869\\
53.96	0.00201261957839208\\
53.97	0.00201261951969946\\
53.98	0.00201261946097078\\
53.99	0.002012619402206\\
54	0.0020126193434051\\
54.01	0.00201261928456803\\
54.02	0.00201261922569475\\
54.03	0.00201261916678522\\
54.04	0.00201261910783942\\
54.05	0.00201261904885728\\
54.06	0.00201261898983879\\
54.07	0.0020126189307839\\
54.08	0.00201261887169257\\
54.09	0.00201261881256477\\
54.1	0.00201261875340045\\
54.11	0.00201261869419957\\
54.12	0.0020126186349621\\
54.13	0.00201261857568799\\
54.14	0.00201261851637722\\
54.15	0.00201261845702973\\
54.16	0.00201261839764549\\
54.17	0.00201261833822446\\
54.18	0.0020126182787666\\
54.19	0.00201261821927188\\
54.2	0.00201261815974024\\
54.21	0.00201261810017166\\
54.22	0.00201261804056609\\
54.23	0.00201261798092349\\
54.24	0.00201261792124383\\
54.25	0.00201261786152705\\
54.26	0.00201261780177313\\
54.27	0.00201261774198202\\
54.28	0.00201261768215369\\
54.29	0.00201261762228808\\
54.3	0.00201261756238517\\
54.31	0.0020126175024449\\
54.32	0.00201261744246725\\
54.33	0.00201261738245217\\
54.34	0.00201261732239962\\
54.35	0.00201261726230956\\
54.36	0.00201261720218194\\
54.37	0.00201261714201673\\
54.38	0.00201261708181389\\
54.39	0.00201261702157337\\
54.4	0.00201261696129514\\
54.41	0.00201261690097915\\
54.42	0.00201261684062536\\
54.43	0.00201261678023373\\
54.44	0.00201261671980422\\
54.45	0.00201261665933678\\
54.46	0.00201261659883138\\
54.47	0.00201261653828798\\
54.48	0.00201261647770653\\
54.49	0.00201261641708699\\
54.5	0.00201261635642931\\
54.51	0.00201261629573347\\
54.52	0.0020126162349994\\
54.53	0.00201261617422708\\
54.54	0.00201261611341646\\
54.55	0.0020126160525675\\
54.56	0.00201261599168016\\
54.57	0.00201261593075438\\
54.58	0.00201261586979014\\
54.59	0.00201261580878738\\
54.6	0.00201261574774607\\
54.61	0.00201261568666616\\
54.62	0.00201261562554761\\
54.63	0.00201261556439038\\
54.64	0.00201261550319442\\
54.65	0.00201261544195969\\
54.66	0.00201261538068615\\
54.67	0.00201261531937375\\
54.68	0.00201261525802245\\
54.69	0.00201261519663221\\
54.7	0.00201261513520298\\
54.71	0.00201261507373472\\
54.72	0.00201261501222739\\
54.73	0.00201261495068094\\
54.74	0.00201261488909533\\
54.75	0.00201261482747051\\
54.76	0.00201261476580644\\
54.77	0.00201261470410308\\
54.78	0.00201261464236038\\
54.79	0.0020126145805783\\
54.8	0.00201261451875679\\
54.81	0.00201261445689581\\
54.82	0.00201261439499532\\
54.83	0.00201261433305526\\
54.84	0.0020126142710756\\
54.85	0.00201261420905629\\
54.86	0.00201261414699729\\
54.87	0.00201261408489854\\
54.88	0.00201261402276001\\
54.89	0.00201261396058165\\
54.9	0.00201261389836341\\
54.91	0.00201261383610525\\
54.92	0.00201261377380713\\
54.93	0.00201261371146899\\
54.94	0.0020126136490908\\
54.95	0.0020126135866725\\
54.96	0.00201261352421405\\
54.97	0.00201261346171541\\
54.98	0.00201261339917653\\
54.99	0.00201261333659736\\
55	0.00201261327397786\\
55.01	0.00201261321131798\\
55.02	0.00201261314861767\\
55.03	0.00201261308587689\\
55.04	0.00201261302309559\\
55.05	0.00201261296027373\\
55.06	0.00201261289741125\\
55.07	0.00201261283450811\\
55.08	0.00201261277156427\\
55.09	0.00201261270857968\\
55.1	0.00201261264555428\\
55.11	0.00201261258248804\\
55.12	0.00201261251938091\\
55.13	0.00201261245623283\\
55.14	0.00201261239304376\\
55.15	0.00201261232981366\\
55.16	0.00201261226654247\\
55.17	0.00201261220323015\\
55.18	0.00201261213987665\\
55.19	0.00201261207648192\\
55.2	0.00201261201304591\\
55.21	0.00201261194956858\\
55.22	0.00201261188604988\\
55.23	0.00201261182248976\\
55.24	0.00201261175888817\\
55.25	0.00201261169524506\\
55.26	0.00201261163156038\\
55.27	0.00201261156783409\\
55.28	0.00201261150406614\\
55.29	0.00201261144025647\\
55.3	0.00201261137640504\\
55.31	0.0020126113125118\\
55.32	0.00201261124857669\\
55.33	0.00201261118459968\\
55.34	0.00201261112058071\\
55.35	0.00201261105651973\\
55.36	0.0020126109924167\\
55.37	0.00201261092827155\\
55.38	0.00201261086408425\\
55.39	0.00201261079985474\\
55.4	0.00201261073558297\\
55.41	0.0020126106712689\\
55.42	0.00201261060691246\\
55.43	0.00201261054251362\\
55.44	0.00201261047807232\\
55.45	0.00201261041358852\\
55.46	0.00201261034906215\\
55.47	0.00201261028449317\\
55.48	0.00201261021988153\\
55.49	0.00201261015522717\\
55.5	0.00201261009053006\\
55.51	0.00201261002579013\\
55.52	0.00201260996100733\\
55.53	0.00201260989618162\\
55.54	0.00201260983131293\\
55.55	0.00201260976640123\\
55.56	0.00201260970144646\\
55.57	0.00201260963644856\\
55.58	0.00201260957140748\\
55.59	0.00201260950632318\\
55.6	0.0020126094411956\\
55.61	0.00201260937602469\\
55.62	0.00201260931081039\\
55.63	0.00201260924555266\\
55.64	0.00201260918025144\\
55.65	0.00201260911490667\\
55.66	0.00201260904951831\\
55.67	0.00201260898408631\\
55.68	0.0020126089186106\\
55.69	0.00201260885309114\\
55.7	0.00201260878752788\\
55.71	0.00201260872192075\\
55.72	0.00201260865626971\\
55.73	0.0020126085905747\\
55.74	0.00201260852483568\\
55.75	0.00201260845905258\\
55.76	0.00201260839322535\\
55.77	0.00201260832735394\\
55.78	0.00201260826143829\\
55.79	0.00201260819547835\\
55.8	0.00201260812947407\\
55.81	0.00201260806342539\\
55.82	0.00201260799733225\\
55.83	0.00201260793119461\\
55.84	0.0020126078650124\\
55.85	0.00201260779878558\\
55.86	0.00201260773251408\\
55.87	0.00201260766619786\\
55.88	0.00201260759983685\\
55.89	0.00201260753343101\\
55.9	0.00201260746698027\\
55.91	0.00201260740048458\\
55.92	0.00201260733394388\\
55.93	0.00201260726735813\\
55.94	0.00201260720072725\\
55.95	0.00201260713405121\\
55.96	0.00201260706732993\\
55.97	0.00201260700056337\\
55.98	0.00201260693375147\\
55.99	0.00201260686689417\\
56	0.00201260679999142\\
56.01	0.00201260673304315\\
56.02	0.00201260666604931\\
56.03	0.00201260659900985\\
56.04	0.00201260653192471\\
56.05	0.00201260646479382\\
56.06	0.00201260639761714\\
56.07	0.00201260633039461\\
56.08	0.00201260626312616\\
56.09	0.00201260619581174\\
56.1	0.00201260612845129\\
56.11	0.00201260606104476\\
56.12	0.00201260599359208\\
56.13	0.0020126059260932\\
56.14	0.00201260585854806\\
56.15	0.0020126057909566\\
56.16	0.00201260572331876\\
56.17	0.00201260565563449\\
56.18	0.00201260558790372\\
56.19	0.0020126055201264\\
56.2	0.00201260545230246\\
56.21	0.00201260538443185\\
56.22	0.00201260531651451\\
56.23	0.00201260524855038\\
56.24	0.00201260518053939\\
56.25	0.0020126051124815\\
56.26	0.00201260504437663\\
56.27	0.00201260497622474\\
56.28	0.00201260490802575\\
56.29	0.00201260483977961\\
56.3	0.00201260477148626\\
56.31	0.00201260470314563\\
56.32	0.00201260463475768\\
56.33	0.00201260456632233\\
56.34	0.00201260449783953\\
56.35	0.00201260442930921\\
56.36	0.00201260436073131\\
56.37	0.00201260429210578\\
56.38	0.00201260422343254\\
56.39	0.00201260415471155\\
56.4	0.00201260408594273\\
56.41	0.00201260401712603\\
56.42	0.00201260394826138\\
56.43	0.00201260387934872\\
56.44	0.00201260381038799\\
56.45	0.00201260374137912\\
56.46	0.00201260367232206\\
56.47	0.00201260360321674\\
56.48	0.0020126035340631\\
56.49	0.00201260346486108\\
56.5	0.0020126033956106\\
56.51	0.00201260332631162\\
56.52	0.00201260325696406\\
56.53	0.00201260318756786\\
56.54	0.00201260311812297\\
56.55	0.0020126030486293\\
56.56	0.00201260297908681\\
56.57	0.00201260290949542\\
56.58	0.00201260283985508\\
56.59	0.00201260277016571\\
56.6	0.00201260270042726\\
56.61	0.00201260263063965\\
56.62	0.00201260256080283\\
56.63	0.00201260249091673\\
56.64	0.00201260242098128\\
56.65	0.00201260235099642\\
56.66	0.00201260228096209\\
56.67	0.00201260221087821\\
56.68	0.00201260214074472\\
56.69	0.00201260207056156\\
56.7	0.00201260200032866\\
56.71	0.00201260193004595\\
56.72	0.00201260185971337\\
56.73	0.00201260178933086\\
56.74	0.00201260171889833\\
56.75	0.00201260164841574\\
56.76	0.002012601577883\\
56.77	0.00201260150730006\\
56.78	0.00201260143666685\\
56.79	0.0020126013659833\\
56.8	0.00201260129524933\\
56.81	0.0020126012244649\\
56.82	0.00201260115362992\\
56.83	0.00201260108274432\\
56.84	0.00201260101180805\\
56.85	0.00201260094082103\\
56.86	0.00201260086978319\\
56.87	0.00201260079869447\\
56.88	0.0020126007275548\\
56.89	0.0020126006563641\\
56.9	0.00201260058512231\\
56.91	0.00201260051382936\\
56.92	0.00201260044248518\\
56.93	0.0020126003710897\\
56.94	0.00201260029964285\\
56.95	0.00201260022814456\\
56.96	0.00201260015659476\\
56.97	0.00201260008499339\\
56.98	0.00201260001334036\\
56.99	0.00201259994163561\\
57	0.00201259986987907\\
57.01	0.00201259979807067\\
57.02	0.00201259972621034\\
57.03	0.002012599654298\\
57.04	0.00201259958233359\\
57.05	0.00201259951031703\\
57.06	0.00201259943824825\\
57.07	0.00201259936612718\\
57.08	0.00201259929395375\\
57.09	0.00201259922172789\\
57.1	0.00201259914944951\\
57.11	0.00201259907711856\\
57.12	0.00201259900473496\\
57.13	0.00201259893229863\\
57.14	0.0020125988598095\\
57.15	0.0020125987872675\\
57.16	0.00201259871467256\\
57.17	0.0020125986420246\\
57.18	0.00201259856932355\\
57.19	0.00201259849656933\\
57.2	0.00201259842376187\\
57.21	0.0020125983509011\\
57.22	0.00201259827798694\\
57.23	0.00201259820501932\\
57.24	0.00201259813199815\\
57.25	0.00201259805892338\\
57.26	0.00201259798579492\\
57.27	0.00201259791261269\\
57.28	0.00201259783937663\\
57.29	0.00201259776608665\\
57.3	0.00201259769274268\\
57.31	0.00201259761934464\\
57.32	0.00201259754589247\\
57.33	0.00201259747238607\\
57.34	0.00201259739882538\\
57.35	0.00201259732521031\\
57.36	0.0020125972515408\\
57.37	0.00201259717781676\\
57.38	0.00201259710403811\\
57.39	0.00201259703020478\\
57.4	0.0020125969563167\\
57.41	0.00201259688237377\\
57.42	0.00201259680837594\\
57.43	0.00201259673432311\\
57.44	0.0020125966602152\\
57.45	0.00201259658605215\\
57.46	0.00201259651183386\\
57.47	0.00201259643756027\\
57.48	0.00201259636323129\\
57.49	0.00201259628884685\\
57.5	0.00201259621440685\\
57.51	0.00201259613991123\\
57.52	0.0020125960653599\\
57.53	0.00201259599075279\\
57.54	0.00201259591608981\\
57.55	0.00201259584137088\\
57.56	0.00201259576659592\\
57.57	0.00201259569176485\\
57.58	0.00201259561687759\\
57.59	0.00201259554193405\\
57.6	0.00201259546693417\\
57.61	0.00201259539187784\\
57.62	0.002012595316765\\
57.63	0.00201259524159556\\
57.64	0.00201259516636943\\
57.65	0.00201259509108654\\
57.66	0.0020125950157468\\
57.67	0.00201259494035013\\
57.68	0.00201259486489644\\
57.69	0.00201259478938566\\
57.7	0.00201259471381769\\
57.71	0.00201259463819246\\
57.72	0.00201259456250987\\
57.73	0.00201259448676985\\
57.74	0.00201259441097232\\
57.75	0.00201259433511717\\
57.76	0.00201259425920434\\
57.77	0.00201259418323374\\
57.78	0.00201259410720527\\
57.79	0.00201259403111886\\
57.8	0.00201259395497442\\
57.81	0.00201259387877186\\
57.82	0.00201259380251109\\
57.83	0.00201259372619204\\
57.84	0.00201259364981461\\
57.85	0.00201259357337871\\
57.86	0.00201259349688427\\
57.87	0.00201259342033118\\
57.88	0.00201259334371937\\
57.89	0.00201259326704874\\
57.9	0.00201259319031921\\
57.91	0.00201259311353069\\
57.92	0.00201259303668309\\
57.93	0.00201259295977632\\
57.94	0.00201259288281029\\
57.95	0.00201259280578492\\
57.96	0.00201259272870011\\
57.97	0.00201259265155577\\
57.98	0.00201259257435182\\
57.99	0.00201259249708816\\
58	0.00201259241976471\\
58.01	0.00201259234238136\\
58.02	0.00201259226493804\\
58.03	0.00201259218743465\\
58.04	0.0020125921098711\\
58.05	0.00201259203224729\\
58.06	0.00201259195456314\\
58.07	0.00201259187681855\\
58.08	0.00201259179901343\\
58.09	0.00201259172114769\\
58.1	0.00201259164322123\\
58.11	0.00201259156523397\\
58.12	0.0020125914871858\\
58.13	0.00201259140907664\\
58.14	0.00201259133090638\\
58.15	0.00201259125267495\\
58.16	0.00201259117438223\\
58.17	0.00201259109602814\\
58.18	0.00201259101761259\\
58.19	0.00201259093913547\\
58.2	0.00201259086059669\\
58.21	0.00201259078199616\\
58.22	0.00201259070333378\\
58.23	0.00201259062460945\\
58.24	0.00201259054582307\\
58.25	0.00201259046697456\\
58.26	0.00201259038806381\\
58.27	0.00201259030909072\\
58.28	0.0020125902300552\\
58.29	0.00201259015095716\\
58.3	0.00201259007179648\\
58.31	0.00201258999257308\\
58.32	0.00201258991328685\\
58.33	0.00201258983393769\\
58.34	0.00201258975452552\\
58.35	0.00201258967505021\\
58.36	0.00201258959551169\\
58.37	0.00201258951590984\\
58.38	0.00201258943624457\\
58.39	0.00201258935651577\\
58.4	0.00201258927672335\\
58.41	0.0020125891968672\\
58.42	0.00201258911694723\\
58.43	0.00201258903696332\\
58.44	0.00201258895691538\\
58.45	0.00201258887680331\\
58.46	0.00201258879662699\\
58.47	0.00201258871638634\\
58.48	0.00201258863608125\\
58.49	0.0020125885557116\\
58.5	0.00201258847527731\\
58.51	0.00201258839477826\\
58.52	0.00201258831421435\\
58.53	0.00201258823358548\\
58.54	0.00201258815289154\\
58.55	0.00201258807213242\\
58.56	0.00201258799130802\\
58.57	0.00201258791041823\\
58.58	0.00201258782946295\\
58.59	0.00201258774844207\\
58.6	0.00201258766735548\\
58.61	0.00201258758620308\\
58.62	0.00201258750498475\\
58.63	0.0020125874237004\\
58.64	0.00201258734234991\\
58.65	0.00201258726093317\\
58.66	0.00201258717945007\\
58.67	0.00201258709790052\\
58.68	0.00201258701628438\\
58.69	0.00201258693460156\\
58.7	0.00201258685285195\\
58.71	0.00201258677103543\\
58.72	0.0020125866891519\\
58.73	0.00201258660720124\\
58.74	0.00201258652518334\\
58.75	0.00201258644309809\\
58.76	0.00201258636094538\\
58.77	0.00201258627872509\\
58.78	0.00201258619643712\\
58.79	0.00201258611408134\\
58.8	0.00201258603165765\\
58.81	0.00201258594916593\\
58.82	0.00201258586660607\\
58.83	0.00201258578397794\\
58.84	0.00201258570128145\\
58.85	0.00201258561851647\\
58.86	0.00201258553568288\\
58.87	0.00201258545278057\\
58.88	0.00201258536980943\\
58.89	0.00201258528676933\\
58.9	0.00201258520366017\\
58.91	0.00201258512048181\\
58.92	0.00201258503723415\\
58.93	0.00201258495391707\\
58.94	0.00201258487053044\\
58.95	0.00201258478707415\\
58.96	0.00201258470354808\\
58.97	0.00201258461995211\\
58.98	0.00201258453628612\\
58.99	0.00201258445254998\\
59	0.00201258436874359\\
59.01	0.00201258428486681\\
59.02	0.00201258420091952\\
59.03	0.00201258411690161\\
59.04	0.00201258403281295\\
59.05	0.00201258394865341\\
59.06	0.00201258386442288\\
59.07	0.00201258378012122\\
59.08	0.00201258369574833\\
59.09	0.00201258361130406\\
59.1	0.00201258352678831\\
59.11	0.00201258344220093\\
59.12	0.00201258335754181\\
59.13	0.00201258327281082\\
59.14	0.00201258318800783\\
59.15	0.00201258310313271\\
59.16	0.00201258301818535\\
59.17	0.00201258293316561\\
59.18	0.00201258284807336\\
59.19	0.00201258276290847\\
59.2	0.00201258267767082\\
59.21	0.00201258259236027\\
59.22	0.0020125825069767\\
59.23	0.00201258242151998\\
59.24	0.00201258233598997\\
59.25	0.00201258225038654\\
59.26	0.00201258216470956\\
59.27	0.0020125820789589\\
59.28	0.00201258199313443\\
59.29	0.00201258190723602\\
59.3	0.00201258182126352\\
59.31	0.00201258173521681\\
59.32	0.00201258164909575\\
59.33	0.00201258156290021\\
59.34	0.00201258147663005\\
59.35	0.00201258139028513\\
59.36	0.00201258130386533\\
59.37	0.00201258121737049\\
59.38	0.0020125811308005\\
59.39	0.0020125810441552\\
59.4	0.00201258095743446\\
59.41	0.00201258087063814\\
59.42	0.0020125807837661\\
59.43	0.00201258069681821\\
59.44	0.00201258060979431\\
59.45	0.00201258052269428\\
59.46	0.00201258043551797\\
59.47	0.00201258034826524\\
59.48	0.00201258026093594\\
59.49	0.00201258017352994\\
59.5	0.00201258008604709\\
59.51	0.00201257999848724\\
59.52	0.00201257991085026\\
59.53	0.002012579823136\\
59.54	0.0020125797353443\\
59.55	0.00201257964747504\\
59.56	0.00201257955952806\\
59.57	0.00201257947150321\\
59.58	0.00201257938340035\\
59.59	0.00201257929521933\\
59.6	0.00201257920695999\\
59.61	0.0020125791186222\\
59.62	0.00201257903020581\\
59.63	0.00201257894171065\\
59.64	0.00201257885313658\\
59.65	0.00201257876448346\\
59.66	0.00201257867575112\\
59.67	0.00201257858693942\\
59.68	0.0020125784980482\\
59.69	0.00201257840907731\\
59.7	0.0020125783200266\\
59.71	0.00201257823089591\\
59.72	0.00201257814168509\\
59.73	0.00201257805239397\\
59.74	0.00201257796302242\\
59.75	0.00201257787357025\\
59.76	0.00201257778403733\\
59.77	0.00201257769442349\\
59.78	0.00201257760472857\\
59.79	0.00201257751495241\\
59.8	0.00201257742509486\\
59.81	0.00201257733515575\\
59.82	0.00201257724513492\\
59.83	0.00201257715503221\\
59.84	0.00201257706484746\\
59.85	0.0020125769745805\\
59.86	0.00201257688423117\\
59.87	0.00201257679379931\\
59.88	0.00201257670328475\\
59.89	0.00201257661268733\\
59.9	0.00201257652200687\\
59.91	0.00201257643124322\\
59.92	0.0020125763403962\\
59.93	0.00201257624946565\\
59.94	0.00201257615845139\\
59.95	0.00201257606735327\\
59.96	0.0020125759761711\\
59.97	0.00201257588490471\\
59.98	0.00201257579355395\\
59.99	0.00201257570211862\\
60	0.00201257561059856\\
60.01	0.0020125755189936\\
60.02	0.00201257542730356\\
60.03	0.00201257533552827\\
60.04	0.00201257524366754\\
60.05	0.00201257515172121\\
60.06	0.0020125750596891\\
60.07	0.00201257496757102\\
60.08	0.00201257487536681\\
60.09	0.00201257478307627\\
60.1	0.00201257469069923\\
60.11	0.00201257459823551\\
60.12	0.00201257450568493\\
60.13	0.00201257441304731\\
60.14	0.00201257432032245\\
60.15	0.00201257422751018\\
60.16	0.00201257413461031\\
60.17	0.00201257404162266\\
60.18	0.00201257394854704\\
60.19	0.00201257385538326\\
60.2	0.00201257376213114\\
60.21	0.00201257366879049\\
60.22	0.00201257357536111\\
60.23	0.00201257348184281\\
60.24	0.00201257338823542\\
60.25	0.00201257329453872\\
60.26	0.00201257320075254\\
60.27	0.00201257310687667\\
60.28	0.00201257301291093\\
60.29	0.00201257291885511\\
60.3	0.00201257282470902\\
60.31	0.00201257273047247\\
60.32	0.00201257263614525\\
60.33	0.00201257254172717\\
60.34	0.00201257244721803\\
60.35	0.00201257235261763\\
60.36	0.00201257225792575\\
60.37	0.00201257216314222\\
60.38	0.00201257206826681\\
60.39	0.00201257197329933\\
60.4	0.00201257187823957\\
60.41	0.00201257178308732\\
60.42	0.00201257168784239\\
60.43	0.00201257159250455\\
60.44	0.00201257149707361\\
60.45	0.00201257140154934\\
60.46	0.00201257130593155\\
60.47	0.00201257121022002\\
60.48	0.00201257111441454\\
60.49	0.00201257101851489\\
60.5	0.00201257092252086\\
60.51	0.00201257082643224\\
60.52	0.0020125707302488\\
60.53	0.00201257063397033\\
60.54	0.00201257053759662\\
60.55	0.00201257044112744\\
60.56	0.00201257034456257\\
60.57	0.00201257024790179\\
60.58	0.00201257015114488\\
60.59	0.00201257005429161\\
60.6	0.00201256995734177\\
60.61	0.00201256986029512\\
60.62	0.00201256976315144\\
60.63	0.0020125696659105\\
60.64	0.00201256956857207\\
60.65	0.00201256947113592\\
60.66	0.00201256937360183\\
60.67	0.00201256927596955\\
60.68	0.00201256917823886\\
60.69	0.00201256908040952\\
60.7	0.00201256898248129\\
60.71	0.00201256888445395\\
60.72	0.00201256878632724\\
60.73	0.00201256868810094\\
60.74	0.0020125685897748\\
60.75	0.00201256849134859\\
60.76	0.00201256839282205\\
60.77	0.00201256829419495\\
60.78	0.00201256819546704\\
60.79	0.00201256809663808\\
60.8	0.00201256799770782\\
60.81	0.002012567898676\\
60.82	0.00201256779954239\\
60.83	0.00201256770030673\\
60.84	0.00201256760096876\\
60.85	0.00201256750152824\\
60.86	0.00201256740198492\\
60.87	0.00201256730233852\\
60.88	0.0020125672025888\\
60.89	0.00201256710273551\\
60.9	0.00201256700277837\\
60.91	0.00201256690271713\\
60.92	0.00201256680255153\\
60.93	0.0020125667022813\\
60.94	0.00201256660190618\\
60.95	0.0020125665014259\\
60.96	0.0020125664008402\\
60.97	0.0020125663001488\\
60.98	0.00201256619935144\\
60.99	0.00201256609844784\\
61	0.00201256599743774\\
61.01	0.00201256589632085\\
61.02	0.0020125657950969\\
61.03	0.00201256569376561\\
61.04	0.0020125655923267\\
61.05	0.00201256549077991\\
61.06	0.00201256538912493\\
61.07	0.00201256528736149\\
61.08	0.00201256518548931\\
61.09	0.0020125650835081\\
61.1	0.00201256498141756\\
61.11	0.00201256487921742\\
61.12	0.00201256477690738\\
61.13	0.00201256467448715\\
61.14	0.00201256457195644\\
61.15	0.00201256446931495\\
61.16	0.00201256436656238\\
61.17	0.00201256426369843\\
61.18	0.00201256416072281\\
61.19	0.00201256405763522\\
61.2	0.00201256395443534\\
61.21	0.00201256385112288\\
61.22	0.00201256374769753\\
61.23	0.00201256364415898\\
61.24	0.00201256354050692\\
61.25	0.00201256343674104\\
61.26	0.00201256333286103\\
61.27	0.00201256322886657\\
61.28	0.00201256312475735\\
61.29	0.00201256302053304\\
61.3	0.00201256291619333\\
61.31	0.00201256281173789\\
61.32	0.00201256270716641\\
61.33	0.00201256260247855\\
61.34	0.00201256249767399\\
61.35	0.0020125623927524\\
61.36	0.00201256228771345\\
61.37	0.00201256218255681\\
61.38	0.00201256207728213\\
61.39	0.0020125619718891\\
61.4	0.00201256186637736\\
61.41	0.00201256176074658\\
61.42	0.00201256165499642\\
61.43	0.00201256154912652\\
61.44	0.00201256144313655\\
61.45	0.00201256133702617\\
61.46	0.00201256123079501\\
61.47	0.00201256112444272\\
61.48	0.00201256101796896\\
61.49	0.00201256091137336\\
61.5	0.00201256080465558\\
61.51	0.00201256069781524\\
61.52	0.00201256059085199\\
61.53	0.00201256048376546\\
61.54	0.0020125603765553\\
61.55	0.00201256026922112\\
61.56	0.00201256016176255\\
61.57	0.00201256005417924\\
61.58	0.0020125599464708\\
61.59	0.00201255983863685\\
61.6	0.00201255973067702\\
61.61	0.00201255962259092\\
61.62	0.00201255951437817\\
61.63	0.00201255940603839\\
61.64	0.00201255929757119\\
61.65	0.00201255918897618\\
61.66	0.00201255908025296\\
61.67	0.00201255897140114\\
61.68	0.00201255886242032\\
61.69	0.00201255875331011\\
61.7	0.0020125586440701\\
61.71	0.00201255853469989\\
61.72	0.00201255842519907\\
61.73	0.00201255831556724\\
61.74	0.00201255820580397\\
61.75	0.00201255809590886\\
61.76	0.00201255798588149\\
61.77	0.00201255787572145\\
61.78	0.0020125577654283\\
61.79	0.00201255765500163\\
61.8	0.00201255754444101\\
61.81	0.00201255743374601\\
61.82	0.0020125573229162\\
61.83	0.00201255721195114\\
61.84	0.00201255710085041\\
61.85	0.00201255698961355\\
61.86	0.00201255687824013\\
61.87	0.00201255676672969\\
61.88	0.00201255665508181\\
61.89	0.00201255654329601\\
61.9	0.00201255643137185\\
61.91	0.00201255631930888\\
61.92	0.00201255620710664\\
61.93	0.00201255609476465\\
61.94	0.00201255598228247\\
61.95	0.00201255586965962\\
61.96	0.00201255575689563\\
61.97	0.00201255564399003\\
61.98	0.00201255553094233\\
61.99	0.00201255541775208\\
62	0.00201255530441877\\
62.01	0.00201255519094193\\
62.02	0.00201255507732107\\
62.03	0.0020125549635557\\
62.04	0.00201255484964532\\
62.05	0.00201255473558943\\
62.06	0.00201255462138754\\
62.07	0.00201255450703914\\
62.08	0.00201255439254372\\
62.09	0.00201255427790078\\
62.1	0.0020125541631098\\
62.11	0.00201255404817027\\
62.12	0.00201255393308166\\
62.13	0.00201255381784345\\
62.14	0.00201255370245512\\
62.15	0.00201255358691614\\
62.16	0.00201255347122597\\
62.17	0.00201255335538407\\
62.18	0.00201255323938991\\
62.19	0.00201255312324294\\
62.2	0.00201255300694262\\
62.21	0.0020125528904884\\
62.22	0.00201255277387971\\
62.23	0.00201255265711601\\
62.24	0.00201255254019674\\
62.25	0.00201255242312132\\
62.26	0.00201255230588919\\
62.27	0.00201255218849977\\
62.28	0.0020125520709525\\
62.29	0.00201255195324679\\
62.3	0.00201255183538206\\
62.31	0.00201255171735772\\
62.32	0.00201255159917319\\
62.33	0.00201255148082785\\
62.34	0.00201255136232113\\
62.35	0.0020125512436524\\
62.36	0.00201255112482108\\
62.37	0.00201255100582654\\
62.38	0.00201255088666817\\
62.39	0.00201255076734536\\
62.4	0.00201255064785748\\
62.41	0.0020125505282039\\
62.42	0.00201255040838399\\
62.43	0.00201255028839713\\
62.44	0.00201255016824266\\
62.45	0.00201255004791994\\
62.46	0.00201254992742833\\
62.47	0.00201254980676718\\
62.48	0.00201254968593582\\
62.49	0.0020125495649336\\
62.5	0.00201254944375986\\
62.51	0.00201254932241391\\
62.52	0.00201254920089509\\
62.53	0.00201254907920272\\
62.54	0.00201254895733612\\
62.55	0.0020125488352946\\
62.56	0.00201254871307746\\
62.57	0.00201254859068401\\
62.58	0.00201254846811354\\
62.59	0.00201254834536536\\
62.6	0.00201254822243874\\
62.61	0.00201254809933298\\
62.62	0.00201254797604735\\
62.63	0.00201254785258112\\
62.64	0.00201254772893357\\
62.65	0.00201254760510396\\
62.66	0.00201254748109155\\
62.67	0.00201254735689559\\
62.68	0.00201254723251533\\
62.69	0.00201254710795002\\
62.7	0.0020125469831989\\
62.71	0.0020125468582612\\
62.72	0.00201254673313614\\
62.73	0.00201254660782296\\
62.74	0.00201254648232087\\
62.75	0.00201254635662908\\
62.76	0.0020125462307468\\
62.77	0.00201254610467323\\
62.78	0.00201254597840757\\
62.79	0.00201254585194901\\
62.8	0.00201254572529674\\
62.81	0.00201254559844993\\
62.82	0.00201254547140776\\
62.83	0.0020125453441694\\
62.84	0.00201254521673402\\
62.85	0.00201254508910075\\
62.86	0.00201254496126877\\
62.87	0.00201254483323722\\
62.88	0.00201254470500523\\
62.89	0.00201254457657193\\
62.9	0.00201254444793647\\
62.91	0.00201254431909795\\
62.92	0.00201254419005549\\
62.93	0.00201254406080821\\
62.94	0.0020125439313552\\
62.95	0.00201254380169557\\
62.96	0.0020125436718284\\
62.97	0.00201254354175278\\
62.98	0.00201254341146779\\
62.99	0.00201254328097249\\
63	0.00201254315026596\\
63.01	0.00201254301934725\\
63.02	0.00201254288821542\\
63.03	0.00201254275686951\\
63.04	0.00201254262530856\\
63.05	0.0020125424935316\\
63.06	0.00201254236153766\\
63.07	0.00201254222932576\\
63.08	0.00201254209689491\\
63.09	0.00201254196424411\\
63.1	0.00201254183137236\\
63.11	0.00201254169827866\\
63.12	0.00201254156496198\\
63.13	0.00201254143142131\\
63.14	0.00201254129765562\\
63.15	0.00201254116366387\\
63.16	0.002012541029445\\
63.17	0.00201254089499798\\
63.18	0.00201254076032174\\
63.19	0.00201254062541522\\
63.2	0.00201254049027734\\
63.21	0.00201254035490702\\
63.22	0.00201254021930317\\
63.23	0.00201254008346469\\
63.24	0.00201253994739048\\
63.25	0.00201253981107944\\
63.26	0.00201253967453042\\
63.27	0.00201253953774232\\
63.28	0.00201253940071399\\
63.29	0.00201253926344429\\
63.3	0.00201253912593206\\
63.31	0.00201253898817616\\
63.32	0.0020125388501754\\
63.33	0.00201253871192862\\
63.34	0.00201253857343463\\
63.35	0.00201253843469223\\
63.36	0.00201253829570022\\
63.37	0.0020125381564574\\
63.38	0.00201253801696255\\
63.39	0.00201253787721443\\
63.4	0.00201253773721182\\
63.41	0.00201253759695347\\
63.42	0.00201253745643813\\
63.43	0.00201253731566453\\
63.44	0.0020125371746314\\
63.45	0.00201253703333747\\
63.46	0.00201253689178144\\
63.47	0.00201253674996203\\
63.48	0.00201253660787791\\
63.49	0.00201253646552778\\
63.5	0.00201253632291031\\
63.51	0.00201253618002416\\
63.52	0.002012536036868\\
63.53	0.00201253589344046\\
63.54	0.00201253574974019\\
63.55	0.00201253560576581\\
63.56	0.00201253546151594\\
63.57	0.00201253531698919\\
63.58	0.00201253517218415\\
63.59	0.00201253502709942\\
63.6	0.00201253488173357\\
63.61	0.00201253473608517\\
63.62	0.00201253459015278\\
63.63	0.00201253444393495\\
63.64	0.00201253429743021\\
63.65	0.00201253415063709\\
63.66	0.00201253400355412\\
63.67	0.00201253385617979\\
63.68	0.0020125337085126\\
63.69	0.00201253356055104\\
63.7	0.00201253341229359\\
63.71	0.0020125332637387\\
63.72	0.00201253311488484\\
63.73	0.00201253296573044\\
63.74	0.00201253281627393\\
63.75	0.00201253266651374\\
63.76	0.00201253251644828\\
63.77	0.00201253236607595\\
63.78	0.00201253221539512\\
63.79	0.00201253206440419\\
63.8	0.0020125319131015\\
63.81	0.00201253176148543\\
63.82	0.0020125316095543\\
63.83	0.00201253145730644\\
63.84	0.00201253130474019\\
63.85	0.00201253115185383\\
63.86	0.00201253099864568\\
63.87	0.002012530845114\\
63.88	0.00201253069125707\\
63.89	0.00201253053707315\\
63.9	0.00201253038256049\\
63.91	0.00201253022771732\\
63.92	0.00201253007254186\\
63.93	0.00201252991703232\\
63.94	0.0020125297611869\\
63.95	0.00201252960500378\\
63.96	0.00201252944848113\\
63.97	0.00201252929161712\\
63.98	0.00201252913440989\\
63.99	0.00201252897685757\\
64	0.00201252881895828\\
64.01	0.00201252866071014\\
64.02	0.00201252850211123\\
64.03	0.00201252834315964\\
64.04	0.00201252818385343\\
64.05	0.00201252802419066\\
64.06	0.00201252786416937\\
64.07	0.00201252770378759\\
64.08	0.00201252754304333\\
64.09	0.00201252738193458\\
64.1	0.00201252722045934\\
64.11	0.00201252705861559\\
64.12	0.00201252689640126\\
64.13	0.00201252673381432\\
64.14	0.00201252657085269\\
64.15	0.00201252640751429\\
64.16	0.00201252624379702\\
64.17	0.00201252607969876\\
64.18	0.00201252591521739\\
64.19	0.00201252575035076\\
64.2	0.00201252558509672\\
64.21	0.00201252541945311\\
64.22	0.00201252525341772\\
64.23	0.00201252508698836\\
64.24	0.00201252492016282\\
64.25	0.00201252475293886\\
64.26	0.00201252458531423\\
64.27	0.00201252441728668\\
64.28	0.00201252424885393\\
64.29	0.00201252408001368\\
64.3	0.00201252391076362\\
64.31	0.00201252374110143\\
64.32	0.00201252357102478\\
64.33	0.0020125234005313\\
64.34	0.00201252322961863\\
64.35	0.00201252305828438\\
64.36	0.00201252288652613\\
64.37	0.00201252271434149\\
64.38	0.00201252254172799\\
64.39	0.00201252236868321\\
64.4	0.00201252219520466\\
64.41	0.00201252202128985\\
64.42	0.0020125218469363\\
64.43	0.00201252167214147\\
64.44	0.00201252149690284\\
64.45	0.00201252132121785\\
64.46	0.00201252114508393\\
64.47	0.00201252096849849\\
64.48	0.00201252079145893\\
64.49	0.00201252061396263\\
64.5	0.00201252043600695\\
64.51	0.00201252025758923\\
64.52	0.00201252007870679\\
64.53	0.00201251989935696\\
64.54	0.00201251971953701\\
64.55	0.00201251953924423\\
64.56	0.00201251935847586\\
64.57	0.00201251917722914\\
64.58	0.00201251899550129\\
64.59	0.00201251881328951\\
64.6	0.00201251863059099\\
64.61	0.00201251844740288\\
64.62	0.00201251826372233\\
64.63	0.00201251807954648\\
64.64	0.00201251789487242\\
64.65	0.00201251770969724\\
64.66	0.00201251752401802\\
64.67	0.00201251733783181\\
64.68	0.00201251715113564\\
64.69	0.00201251696392651\\
64.7	0.00201251677620144\\
64.71	0.00201251658795738\\
64.72	0.0020125163991913\\
64.73	0.00201251620990013\\
64.74	0.00201251602008079\\
64.75	0.00201251582973017\\
64.76	0.00201251563884515\\
64.77	0.00201251544742258\\
64.78	0.00201251525545931\\
64.79	0.00201251506295215\\
64.8	0.00201251486989789\\
64.81	0.00201251467629331\\
64.82	0.00201251448213517\\
64.83	0.0020125142874202\\
64.84	0.00201251409214512\\
64.85	0.00201251389630662\\
64.86	0.00201251369990138\\
64.87	0.00201251350292604\\
64.88	0.00201251330537725\\
64.89	0.0020125131072516\\
64.9	0.0020125129085457\\
64.91	0.0020125127092561\\
64.92	0.00201251250937937\\
64.93	0.00201251230891202\\
64.94	0.00201251210785055\\
64.95	0.00201251190619146\\
64.96	0.00201251170393121\\
64.97	0.00201251150106623\\
64.98	0.00201251129759294\\
64.99	0.00201251109350775\\
65	0.00201251088880703\\
65.01	0.00201251068348712\\
65.02	0.00201251047754437\\
65.03	0.00201251027097507\\
65.04	0.00201251006377553\\
65.05	0.002012509855942\\
65.06	0.00201250964747072\\
65.07	0.00201250943835792\\
65.08	0.00201250922859979\\
65.09	0.00201250901819251\\
65.1	0.00201250880713222\\
65.11	0.00201250859541507\\
65.12	0.00201250838303716\\
65.13	0.00201250816999457\\
65.14	0.00201250795628336\\
65.15	0.00201250774189957\\
65.16	0.00201250752683922\\
65.17	0.0020125073110983\\
65.18	0.00201250709467278\\
65.19	0.0020125068775586\\
65.2	0.00201250665975169\\
65.21	0.00201250644124795\\
65.22	0.00201250622204324\\
65.23	0.00201250600213343\\
65.24	0.00201250578151435\\
65.25	0.00201250556018179\\
65.26	0.00201250533813154\\
65.27	0.00201250511535936\\
65.28	0.00201250489186098\\
65.29	0.00201250466763212\\
65.3	0.00201250444266845\\
65.31	0.00201250421696565\\
65.32	0.00201250399051934\\
65.33	0.00201250376332516\\
65.34	0.00201250353537867\\
65.35	0.00201250330667546\\
65.36	0.00201250307721107\\
65.37	0.00201250284698101\\
65.38	0.00201250261598077\\
65.39	0.00201250238420583\\
65.4	0.00201250215165164\\
65.41	0.00201250191831361\\
65.42	0.00201250168418714\\
65.43	0.0020125014492676\\
65.44	0.00201250121355035\\
65.45	0.0020125009770307\\
65.46	0.00201250073970395\\
65.47	0.00201250050156539\\
65.48	0.00201250026261025\\
65.49	0.00201250002283377\\
65.5	0.00201249978223114\\
65.51	0.00201249954079755\\
65.52	0.00201249929852814\\
65.53	0.00201249905541805\\
65.54	0.00201249881146237\\
65.55	0.00201249856665619\\
65.56	0.00201249832099455\\
65.57	0.00201249807447249\\
65.58	0.00201249782708501\\
65.59	0.0020124975788271\\
65.6	0.0020124973296937\\
65.61	0.00201249707967975\\
65.62	0.00201249682878015\\
65.63	0.00201249657698979\\
65.64	0.00201249632430352\\
65.65	0.00201249607071618\\
65.66	0.00201249581622257\\
65.67	0.00201249556081748\\
65.68	0.00201249530449567\\
65.69	0.00201249504725187\\
65.7	0.00201249478908079\\
65.71	0.00201249452997712\\
65.72	0.00201249426993552\\
65.73	0.00201249400895062\\
65.74	0.00201249374701705\\
65.75	0.00201249348412939\\
65.76	0.0020124932202822\\
65.77	0.00201249295547003\\
65.78	0.00201249268968739\\
65.79	0.00201249242292878\\
65.8	0.00201249215518866\\
65.81	0.00201249188646148\\
65.82	0.00201249161674166\\
65.83	0.0020124913460236\\
65.84	0.00201249107430167\\
65.85	0.00201249080157023\\
65.86	0.0020124905278236\\
65.87	0.00201249025305608\\
65.88	0.00201248997726195\\
65.89	0.00201248970043548\\
65.9	0.00201248942257089\\
65.91	0.0020124891436624\\
65.92	0.0020124888637042\\
65.93	0.00201248858269044\\
65.94	0.00201248830061528\\
65.95	0.00201248801747284\\
65.96	0.0020124877332572\\
65.97	0.00201248744796246\\
65.98	0.00201248716158265\\
65.99	0.00201248687411182\\
66	0.00201248658554398\\
66.01	0.0020124862958731\\
66.02	0.00201248600509316\\
66.03	0.0020124857131981\\
66.04	0.00201248542018185\\
66.05	0.00201248512603831\\
66.06	0.00201248483076136\\
66.07	0.00201248453434486\\
66.08	0.00201248423678265\\
66.09	0.00201248393806855\\
66.1	0.00201248363819637\\
66.11	0.00201248333715988\\
66.12	0.00201248303495284\\
66.13	0.00201248273156899\\
66.14	0.00201248242700205\\
66.15	0.00201248212124573\\
66.16	0.00201248181429371\\
66.17	0.00201248150613964\\
66.18	0.00201248119677719\\
66.19	0.00201248088619997\\
66.2	0.00201248057440159\\
66.21	0.00201248026137565\\
66.22	0.00201247994711571\\
66.23	0.00201247963161535\\
66.24	0.00201247931486808\\
66.25	0.00201247899686745\\
66.26	0.00201247867760695\\
66.27	0.00201247835708008\\
66.28	0.0020124780352803\\
66.29	0.00201247771220108\\
66.3	0.00201247738783587\\
66.31	0.00201247706217808\\
66.32	0.00201247673522114\\
66.33	0.00201247640695844\\
66.34	0.00201247607738337\\
66.35	0.0020124757464893\\
66.36	0.00201247541426958\\
66.37	0.00201247508071757\\
66.38	0.0020124747458266\\
66.39	0.00201247440958998\\
66.4	0.00201247407200103\\
66.41	0.00201247373305304\\
66.42	0.0020124733927393\\
66.43	0.00201247305105308\\
66.44	0.00201247270798765\\
66.45	0.00201247236353626\\
66.46	0.00201247201769216\\
66.47	0.00201247167044858\\
66.48	0.00201247132179875\\
66.49	0.00201247097173589\\
66.5	0.0020124706202532\\
66.51	0.0020124702673439\\
66.52	0.00201246991300118\\
66.53	0.00201246955721823\\
66.54	0.00201246919998824\\
66.55	0.00201246884130438\\
66.56	0.00201246848115982\\
66.57	0.00201246811954773\\
66.58	0.00201246775646129\\
66.59	0.00201246739189365\\
66.6	0.00201246702583796\\
66.61	0.0020124666582874\\
66.62	0.0020124662892351\\
66.63	0.00201246591867422\\
66.64	0.00201246554659792\\
66.65	0.00201246517299934\\
66.66	0.00201246479787163\\
66.67	0.00201246442120796\\
66.68	0.00201246404300146\\
66.69	0.00201246366324531\\
66.7	0.00201246328193265\\
66.71	0.00201246289905665\\
66.72	0.00201246251461047\\
66.73	0.00201246212858729\\
66.74	0.00201246174098028\\
66.75	0.00201246135178261\\
66.76	0.00201246096098749\\
66.77	0.00201246056858809\\
66.78	0.00201246017457764\\
66.79	0.00201245977894932\\
66.8	0.00201245938169638\\
66.81	0.00201245898281202\\
66.82	0.00201245858228951\\
66.83	0.00201245818012208\\
66.84	0.002012457776303\\
66.85	0.00201245737082555\\
66.86	0.00201245696368302\\
66.87	0.00201245655486871\\
66.88	0.00201245614437594\\
66.89	0.00201245573219805\\
66.9	0.00201245531832839\\
66.91	0.00201245490276032\\
66.92	0.00201245448548724\\
66.93	0.00201245406650255\\
66.94	0.00201245364579968\\
66.95	0.00201245322337207\\
66.96	0.0020124527992132\\
66.97	0.00201245237331654\\
66.98	0.00201245194567563\\
66.99	0.00201245151628399\\
67	0.00201245108513519\\
67.01	0.00201245065222281\\
67.02	0.00201245021754048\\
67.03	0.00201244978108183\\
67.04	0.00201244934284054\\
67.05	0.00201244890281031\\
67.06	0.00201244846098488\\
67.07	0.00201244801735799\\
67.08	0.00201244757192346\\
67.09	0.0020124471246751\\
67.1	0.00201244667560678\\
67.11	0.0020124462247124\\
67.12	0.0020124457719859\\
67.13	0.00201244531742125\\
67.14	0.00201244486101244\\
67.15	0.00201244440275355\\
67.16	0.00201244394263864\\
67.17	0.00201244348066187\\
67.18	0.00201244301681739\\
67.19	0.00201244255109942\\
67.2	0.00201244208350223\\
67.21	0.00201244161402012\\
67.22	0.00201244114264745\\
67.23	0.00201244066937861\\
67.24	0.00201244019420805\\
67.25	0.00201243971713028\\
67.26	0.00201243923813985\\
67.27	0.00201243875723135\\
67.28	0.00201243827439943\\
67.29	0.00201243778963882\\
67.3	0.00201243730294428\\
67.31	0.00201243681431062\\
67.32	0.00201243632373273\\
67.33	0.00201243583120554\\
67.34	0.00201243533672406\\
67.35	0.00201243484028333\\
67.36	0.00201243434187849\\
67.37	0.00201243384150471\\
67.38	0.00201243333915725\\
67.39	0.00201243283483143\\
67.4	0.00201243232852261\\
67.41	0.00201243182022627\\
67.42	0.0020124313099379\\
67.43	0.00201243079765311\\
67.44	0.00201243028336756\\
67.45	0.00201242976707698\\
67.46	0.00201242924877718\\
67.47	0.00201242872846405\\
67.48	0.00201242820613355\\
67.49	0.00201242768178171\\
67.5	0.00201242715540466\\
67.51	0.0020124266269986\\
67.52	0.00201242609655979\\
67.53	0.00201242556408462\\
67.54	0.00201242502956951\\
67.55	0.00201242449301101\\
67.56	0.00201242395440573\\
67.57	0.00201242341375038\\
67.58	0.00201242287104175\\
67.59	0.00201242232627673\\
67.6	0.00201242177945228\\
67.61	0.00201242123056549\\
67.62	0.00201242067961349\\
67.63	0.00201242012659357\\
67.64	0.00201241957150306\\
67.65	0.00201241901433941\\
67.66	0.00201241845510017\\
67.67	0.00201241789378298\\
67.68	0.0020124173303856\\
67.69	0.00201241676490587\\
67.7	0.00201241619734173\\
67.71	0.00201241562769126\\
67.72	0.0020124150559526\\
67.73	0.00201241448212402\\
67.74	0.00201241390620389\\
67.75	0.0020124133281907\\
67.76	0.00201241274808304\\
67.77	0.0020124121658796\\
67.78	0.00201241158157921\\
67.79	0.00201241099518078\\
67.8	0.00201241040668335\\
67.81	0.00201240981608608\\
67.82	0.00201240922338824\\
67.83	0.0020124086285892\\
67.84	0.00201240803168848\\
67.85	0.00201240743268569\\
67.86	0.00201240683158057\\
67.87	0.00201240622837299\\
67.88	0.00201240562306291\\
67.89	0.00201240501565044\\
67.9	0.00201240440613581\\
67.91	0.00201240379451936\\
67.92	0.00201240318080156\\
67.93	0.002012402564983\\
67.94	0.00201240194706441\\
67.95	0.00201240132704663\\
67.96	0.00201240070493063\\
67.97	0.00201240008071751\\
67.98	0.00201239945440849\\
67.99	0.00201239882600494\\
68	0.00201239819550832\\
68.01	0.00201239756292024\\
68.02	0.00201239692824246\\
68.03	0.00201239629147682\\
68.04	0.00201239565262534\\
68.05	0.00201239501169013\\
68.06	0.00201239436867345\\
68.07	0.00201239372357768\\
68.08	0.00201239307640535\\
68.09	0.0020123924271591\\
68.1	0.00201239177584169\\
68.11	0.00201239112245603\\
68.12	0.00201239046700516\\
68.13	0.00201238980949224\\
68.14	0.00201238914992056\\
68.15	0.00201238848829353\\
68.16	0.0020123878246147\\
68.17	0.00201238715888774\\
68.18	0.00201238649111646\\
68.19	0.00201238582130477\\
68.2	0.00201238514945673\\
68.21	0.00201238447557651\\
68.22	0.0020123837996684\\
68.23	0.00201238312173683\\
68.24	0.00201238244178633\\
68.25	0.00201238175982156\\
68.26	0.0020123810758473\\
68.27	0.00201238038986845\\
68.28	0.00201237970189\\
68.29	0.00201237901191708\\
68.3	0.00201237831995492\\
68.31	0.00201237762600887\\
68.32	0.00201237693008438\\
68.33	0.00201237623218701\\
68.34	0.0020123755323224\\
68.35	0.00201237483049633\\
68.36	0.00201237412671466\\
68.37	0.00201237342098333\\
68.38	0.00201237271330839\\
68.39	0.002012372003696\\
68.4	0.00201237129215236\\
68.41	0.00201237057868379\\
68.42	0.00201236986329668\\
68.43	0.00201236914599749\\
68.44	0.00201236842679277\\
68.45	0.00201236770568913\\
68.46	0.00201236698269324\\
68.47	0.00201236625781184\\
68.48	0.00201236553105171\\
68.49	0.00201236480241972\\
68.5	0.00201236407192274\\
68.51	0.00201236333956772\\
68.52	0.00201236260536163\\
68.53	0.00201236186931148\\
68.54	0.0020123611314243\\
68.55	0.00201236039170714\\
68.56	0.00201235965016707\\
68.57	0.00201235890681117\\
68.58	0.00201235816164652\\
68.59	0.0020123574146802\\
68.6	0.00201235666591926\\
68.61	0.00201235591537077\\
68.62	0.00201235516304173\\
68.63	0.00201235440893913\\
68.64	0.00201235365306992\\
68.65	0.002012352895441\\
68.66	0.0020123521360592\\
68.67	0.0020123513749313\\
68.68	0.00201235061206399\\
68.69	0.00201234984746388\\
68.7	0.00201234908113748\\
68.71	0.00201234831309121\\
68.72	0.00201234754333136\\
68.73	0.0020123467718641\\
68.74	0.00201234599869547\\
68.75	0.00201234522383134\\
68.76	0.00201234444727744\\
68.77	0.00201234366903932\\
68.78	0.00201234288912236\\
68.79	0.00201234210753171\\
68.8	0.00201234132427235\\
68.81	0.002012340539349\\
68.82	0.00201233975276617\\
68.83	0.00201233896452809\\
68.84	0.00201233817463876\\
68.85	0.00201233738310186\\
68.86	0.0020123365899208\\
68.87	0.00201233579509865\\
68.88	0.00201233499863816\\
68.89	0.00201233420054175\\
68.9	0.00201233340081144\\
68.91	0.0020123325994489\\
68.92	0.00201233179645537\\
68.93	0.00201233099183168\\
68.94	0.00201233018557822\\
68.95	0.00201232937769492\\
68.96	0.00201232856818121\\
68.97	0.00201232775703604\\
68.98	0.00201232694425782\\
68.99	0.00201232612984441\\
69	0.00201232531379312\\
69.01	0.00201232449610062\\
69.02	0.00201232367676309\\
69.03	0.00201232285577674\\
69.04	0.00201232203313781\\
69.05	0.00201232120884264\\
69.06	0.00201232038288757\\
69.07	0.00201231955526902\\
69.08	0.00201231872598347\\
69.09	0.00201231789502744\\
69.1	0.00201231706239753\\
69.11	0.00201231622809036\\
69.12	0.00201231539210266\\
69.13	0.00201231455443119\\
69.14	0.00201231371507277\\
69.15	0.00201231287402429\\
69.16	0.00201231203128273\\
69.17	0.0020123111868451\\
69.18	0.00201231034070849\\
69.19	0.00201230949287008\\
69.2	0.00201230864332709\\
69.21	0.00201230779207684\\
69.22	0.00201230693911671\\
69.23	0.00201230608444415\\
69.24	0.00201230522805671\\
69.25	0.00201230436995199\\
69.26	0.00201230351012769\\
69.27	0.00201230264858159\\
69.28	0.00201230178531154\\
69.29	0.00201230092031549\\
69.3	0.00201230005359147\\
69.31	0.0020122991851376\\
69.32	0.00201229831495209\\
69.33	0.00201229744303325\\
69.34	0.00201229656937945\\
69.35	0.00201229569398921\\
69.36	0.00201229481686109\\
69.37	0.00201229393799378\\
69.38	0.00201229305738608\\
69.39	0.00201229217503685\\
69.4	0.00201229129094511\\
69.41	0.00201229040510993\\
69.42	0.00201228951753052\\
69.43	0.0020122886282062\\
69.44	0.00201228773713638\\
69.45	0.00201228684432061\\
69.46	0.00201228594975852\\
69.47	0.0020122850534499\\
69.48	0.00201228415539462\\
69.49	0.00201228325559269\\
69.5	0.00201228235404424\\
69.51	0.00201228145074953\\
69.52	0.00201228054570892\\
69.53	0.00201227963892293\\
69.54	0.0020122787303922\\
69.55	0.0020122778201175\\
69.56	0.00201227690809973\\
69.57	0.00201227599433994\\
69.58	0.0020122750788393\\
69.59	0.00201227416159913\\
69.6	0.00201227324262091\\
69.61	0.00201227232190623\\
69.62	0.00201227139945687\\
69.63	0.00201227047527471\\
69.64	0.00201226954936183\\
69.65	0.00201226862172042\\
69.66	0.00201226769235287\\
69.67	0.00201226676126169\\
69.68	0.00201226582844957\\
69.69	0.00201226489391936\\
69.7	0.00201226395767408\\
69.71	0.0020122630197169\\
69.72	0.00201226208005117\\
69.73	0.00201226113868042\\
69.74	0.00201226019560835\\
69.75	0.00201225925083883\\
69.76	0.00201225830437591\\
69.77	0.00201225735622383\\
69.78	0.00201225640638701\\
69.79	0.00201225545487004\\
69.8	0.00201225450167773\\
69.81	0.00201225354681505\\
69.82	0.00201225259028718\\
69.83	0.00201225163209951\\
69.84	0.00201225067225759\\
69.85	0.00201224971076721\\
69.86	0.00201224874763433\\
69.87	0.00201224778286515\\
69.88	0.00201224681646604\\
69.89	0.00201224584844362\\
69.9	0.00201224487880469\\
69.91	0.00201224390755629\\
69.92	0.00201224293470566\\
69.93	0.00201224196026028\\
69.94	0.00201224098422784\\
69.95	0.00201224000661625\\
69.96	0.00201223902743367\\
69.97	0.00201223804668848\\
69.98	0.00201223706438929\\
69.99	0.00201223608054495\\
70	0.00201223509516454\\
70.01	0.0020122341082574\\
70.02	0.0020122331198331\\
70.03	0.00201223212990147\\
70.04	0.00201223113847256\\
70.05	0.00201223014555671\\
70.06	0.00201222915116449\\
70.07	0.00201222815530674\\
70.08	0.00201222715799454\\
70.09	0.00201222615923925\\
70.1	0.0020122251590525\\
70.11	0.00201222415744616\\
70.12	0.0020122231544324\\
70.13	0.00201222215002364\\
70.14	0.0020122211442326\\
70.15	0.00201222013707224\\
70.16	0.00201221912855584\\
70.17	0.00201221811869693\\
70.18	0.00201221710750935\\
70.19	0.0020122160950072\\
70.2	0.00201221508120491\\
70.21	0.00201221406611717\\
70.22	0.00201221304975896\\
70.23	0.00201221203214558\\
70.24	0.00201221101329263\\
70.25	0.00201220999321599\\
70.26	0.00201220897193186\\
70.27	0.00201220794945674\\
70.28	0.00201220692580744\\
70.29	0.0020122059010011\\
70.3	0.00201220487505514\\
70.31	0.00201220384798732\\
70.32	0.0020122028198157\\
70.33	0.00201220179055868\\
70.34	0.00201220076023498\\
70.35	0.00201219972886362\\
70.36	0.00201219869646396\\
70.37	0.00201219766305571\\
70.38	0.00201219662865889\\
70.39	0.00201219559329383\\
70.4	0.00201219455698124\\
70.41	0.00201219351974213\\
70.42	0.00201219248159787\\
70.43	0.00201219144257015\\
70.44	0.00201219040268101\\
70.45	0.00201218936195285\\
70.46	0.00201218832040838\\
70.47	0.00201218727807067\\
70.48	0.00201218623496315\\
70.49	0.00201218519110959\\
70.5	0.0020121841465341\\
70.51	0.00201218310126115\\
70.52	0.00201218205531555\\
70.53	0.00201218100872249\\
70.54	0.00201217996150749\\
70.55	0.00201217891369643\\
70.56	0.00201217786531554\\
70.57	0.00201217681639143\\
70.58	0.00201217576695103\\
70.59	0.00201217471702167\\
70.6	0.002012173666631\\
70.61	0.00201217261580704\\
70.62	0.00201217156457819\\
70.63	0.00201217051297318\\
70.64	0.00201216946102111\\
70.65	0.00201216840875144\\
70.66	0.00201216735619399\\
70.67	0.00201216630337892\\
70.68	0.00201216525033678\\
70.69	0.00201216419709845\\
70.7	0.00201216314369517\\
70.71	0.00201216209015855\\
70.72	0.00201216103652054\\
70.73	0.00201215998281344\\
70.74	0.00201215892906992\\
70.75	0.00201215787532298\\
70.76	0.00201215682160597\\
70.77	0.00201215576795262\\
70.78	0.00201215471439695\\
70.79	0.00201215366097337\\
70.8	0.0020121526077166\\
70.81	0.00201215155466172\\
70.82	0.00201215050184412\\
70.83	0.00201214944929955\\
70.84	0.00201214839706407\\
70.85	0.00201214734517407\\
70.86	0.00201214629366627\\
70.87	0.00201214524257769\\
70.88	0.00201214419194567\\
70.89	0.00201214314180789\\
70.9	0.00201214209220229\\
70.91	0.00201214104316714\\
70.92	0.002012139994741\\
70.93	0.00201213894696272\\
70.94	0.00201213789987143\\
70.95	0.00201213685350654\\
70.96	0.00201213580790775\\
70.97	0.00201213476311501\\
70.98	0.00201213371916853\\
70.99	0.00201213267610879\\
71	0.0020121316339765\\
71.01	0.00201213059281261\\
71.02	0.00201212955265831\\
71.03	0.00201212851355501\\
71.04	0.00201212747554434\\
71.05	0.00201212643866812\\
71.06	0.00201212540296838\\
71.07	0.00201212436848733\\
71.08	0.00201212333526735\\
71.09	0.00201212230335101\\
71.1	0.00201212127278102\\
71.11	0.00201212024360022\\
71.12	0.00201211921585161\\
71.13	0.00201211818957829\\
71.14	0.00201211716482348\\
71.15	0.0020121161416305\\
71.16	0.00201211512004272\\
71.17	0.00201211410010363\\
71.18	0.00201211308185673\\
71.19	0.00201211206534558\\
71.2	0.00201211105061375\\
71.21	0.00201211003770483\\
71.22	0.00201210902666239\\
71.23	0.00201210801753\\
71.24	0.00201210701035115\\
71.25	0.0020121060051693\\
71.26	0.00201210500202781\\
71.27	0.00201210400096995\\
71.28	0.00201210300203886\\
71.29	0.00201210200527757\\
71.3	0.00201210101072893\\
71.31	0.00201210001843559\\
71.32	0.00201209902844003\\
71.33	0.00201209804078448\\
71.34	0.00201209705551092\\
71.35	0.00201209607266108\\
71.36	0.00201209509227635\\
71.37	0.00201209411439781\\
71.38	0.00201209313906621\\
71.39	0.00201209216632187\\
71.4	0.00201209119620475\\
71.41	0.00201209022875433\\
71.42	0.00201208926400965\\
71.43	0.00201208830200924\\
71.44	0.00201208734279109\\
71.45	0.00201208638639265\\
71.46	0.00201208543285075\\
71.47	0.0020120844822016\\
71.48	0.00201208353448074\\
71.49	0.002012082589723\\
71.5	0.00201208164796247\\
71.51	0.00201208070923249\\
71.52	0.00201207977356553\\
71.53	0.00201207884099325\\
71.54	0.00201207791154637\\
71.55	0.0020120769852547\\
71.56	0.00201207606214703\\
71.57	0.00201207514225116\\
71.58	0.00201207422559376\\
71.59	0.00201207331220042\\
71.6	0.00201207240209553\\
71.61	0.00201207149530227\\
71.62	0.00201207059184254\\
71.63	0.00201206969173694\\
71.64	0.00201206879500608\\
71.65	0.00201206790167053\\
71.66	0.00201206701175088\\
71.67	0.00201206612526767\\
71.68	0.00201206524224141\\
71.69	0.00201206436269262\\
71.7	0.00201206348664177\\
71.71	0.0020120626141093\\
71.72	0.00201206174511563\\
71.73	0.00201206087968113\\
71.74	0.00201206001782616\\
71.75	0.00201205915957102\\
71.76	0.00201205830493596\\
71.77	0.00201205745394122\\
71.78	0.00201205660660695\\
71.79	0.00201205576295328\\
71.8	0.00201205492300029\\
71.81	0.00201205408676798\\
71.82	0.00201205325427632\\
71.83	0.00201205242554519\\
71.84	0.00201205160059443\\
71.85	0.00201205077944381\\
71.86	0.00201204996211301\\
71.87	0.00201204914862166\\
71.88	0.00201204833898932\\
71.89	0.00201204753323544\\
71.9	0.0020120467313794\\
71.91	0.00201204593344053\\
71.92	0.00201204513943801\\
71.93	0.00201204434939098\\
71.94	0.00201204356331846\\
71.95	0.00201204278123938\\
71.96	0.00201204200317256\\
71.97	0.00201204122913673\\
71.98	0.00201204045915049\\
71.99	0.00201203969323235\\
72	0.0020120389314007\\
72.01	0.00201203817367379\\
72.02	0.00201203742006977\\
72.03	0.00201203667060666\\
72.04	0.00201203592530235\\
72.05	0.00201203518417459\\
72.06	0.00201203444724099\\
72.07	0.00201203371451904\\
72.08	0.00201203298602605\\
72.09	0.00201203226177922\\
72.1	0.00201203154179556\\
72.11	0.00201203082609195\\
72.12	0.00201203011468511\\
72.13	0.00201202940759157\\
72.14	0.00201202870482771\\
72.15	0.00201202800640973\\
72.16	0.00201202731235367\\
72.17	0.00201202662267536\\
72.18	0.00201202593739046\\
72.19	0.00201202525651445\\
72.2	0.00201202458006259\\
72.21	0.00201202390804997\\
72.22	0.00201202324049145\\
72.23	0.00201202257740171\\
72.24	0.00201202191879518\\
72.25	0.00201202126468612\\
72.26	0.00201202061508854\\
72.27	0.00201201997001622\\
72.28	0.00201201932948273\\
72.29	0.0020120186935014\\
72.3	0.00201201806208531\\
72.31	0.0020120174352473\\
72.32	0.00201201681299996\\
72.33	0.00201201619535564\\
72.34	0.00201201558232641\\
72.35	0.0020120149739241\\
72.36	0.00201201437016025\\
72.37	0.00201201377104614\\
72.38	0.00201201317659276\\
72.39	0.00201201258681084\\
72.4	0.0020120120017108\\
72.41	0.00201201142130276\\
72.42	0.00201201084559658\\
72.43	0.00201201027460178\\
72.44	0.00201200970832759\\
72.45	0.00201200914678292\\
72.46	0.00201200858997635\\
72.47	0.00201200803791617\\
72.48	0.0020120074906103\\
72.49	0.00201200694806637\\
72.5	0.00201200641029162\\
72.51	0.002012005877293\\
72.52	0.00201200534907705\\
72.53	0.00201200482565002\\
72.54	0.00201200430701775\\
72.55	0.00201200379318573\\
72.56	0.00201200328415909\\
72.57	0.00201200277994256\\
72.58	0.00201200228054052\\
72.59	0.00201200178595693\\
72.6	0.00201200129619539\\
72.61	0.00201200081125906\\
72.62	0.00201200033115075\\
72.63	0.00201199985587281\\
72.64	0.00201199938542722\\
72.65	0.00201199891981551\\
72.66	0.0020119984590388\\
72.67	0.00201199800309777\\
72.68	0.00201199755199269\\
72.69	0.00201199710572335\\
72.7	0.00201199666428913\\
72.71	0.00201199622768894\\
72.72	0.00201199579592124\\
72.73	0.00201199536898403\\
72.74	0.00201199494687483\\
72.75	0.00201199452959071\\
72.76	0.00201199411712825\\
72.77	0.00201199370948354\\
72.78	0.00201199330665221\\
72.79	0.00201199290862935\\
72.8	0.00201199251540962\\
72.81	0.00201199212698711\\
72.82	0.00201199174335545\\
72.83	0.00201199136450773\\
72.84	0.00201199099043654\\
72.85	0.00201199062113394\\
72.86	0.00201199025659147\\
72.87	0.00201198989680012\\
72.88	0.00201198954175038\\
72.89	0.00201198919143217\\
72.9	0.00201198884583486\\
72.91	0.00201198850494731\\
72.92	0.00201198816875778\\
72.93	0.002011987837254\\
72.94	0.00201198751042314\\
72.95	0.0020119871882518\\
72.96	0.00201198687072599\\
72.97	0.00201198655783118\\
72.98	0.00201198624955224\\
72.99	0.00201198594587347\\
73	0.00201198564677859\\
73.01	0.00201198535225071\\
73.02	0.00201198506227237\\
73.03	0.00201198477682552\\
73.04	0.0020119844958915\\
73.05	0.00201198421945106\\
73.06	0.00201198394748434\\
73.07	0.00201198367997088\\
73.08	0.00201198341688962\\
73.09	0.00201198315821888\\
73.1	0.00201198290393637\\
73.11	0.00201198265401919\\
73.12	0.00201198240844384\\
73.13	0.00201198216718619\\
73.14	0.00201198193022149\\
73.15	0.00201198169752438\\
73.16	0.00201198146906888\\
73.17	0.0020119812448284\\
73.18	0.00201198102477572\\
73.19	0.002011980808883\\
73.2	0.00201198059712179\\
73.21	0.00201198038946302\\
73.22	0.002011980185877\\
73.23	0.00201197998633342\\
73.24	0.00201197979080136\\
73.25	0.00201197959924929\\
73.26	0.00201197941164507\\
73.27	0.00201197922795594\\
73.28	0.00201197904814854\\
73.29	0.00201197887218891\\
73.3	0.00201197870004249\\
73.31	0.00201197853167411\\
73.32	0.00201197836704803\\
73.33	0.0020119782061279\\
73.34	0.00201197804887681\\
73.35	0.00201197789525725\\
73.36	0.00201197774523115\\
73.37	0.00201197759875986\\
73.38	0.00201197745580421\\
73.39	0.00201197731632442\\
73.4	0.0020119771802802\\
73.41	0.00201197704763073\\
73.42	0.00201197691833464\\
73.43	0.00201197679235004\\
73.44	0.00201197666963454\\
73.45	0.00201197655014525\\
73.46	0.0020119764338388\\
73.47	0.0020119763206713\\
73.48	0.00201197621059844\\
73.49	0.00201197610357544\\
73.5	0.00201197599955706\\
73.51	0.00201197589849765\\
73.52	0.00201197580035115\\
73.53	0.0020119757050711\\
73.54	0.00201197561261063\\
73.55	0.00201197552292254\\
73.56	0.00201197543595926\\
73.57	0.00201197535167289\\
73.58	0.00201197527001522\\
73.59	0.00201197519093775\\
73.6	0.00201197511439169\\
73.61	0.00201197504032801\\
73.62	0.00201197496869745\\
73.63	0.00201197489945053\\
73.64	0.0020119748325376\\
73.65	0.00201197476790884\\
73.66	0.00201197470551431\\
73.67	0.00201197464530393\\
73.68	0.00201197458722757\\
73.69	0.00201197453123505\\
73.7	0.00201197447727613\\
73.71	0.00201197442530062\\
73.72	0.00201197437525834\\
73.73	0.00201197432709921\\
73.74	0.00201197428077323\\
73.75	0.00201197423623056\\
73.76	0.00201197419342153\\
73.77	0.00201197415229668\\
73.78	0.00201197411280681\\
73.79	0.00201197407490303\\
73.8	0.00201197403853676\\
73.81	0.00201197400365982\\
73.82	0.00201197397022443\\
73.83	0.00201197393818331\\
73.84	0.00201197390748967\\
73.85	0.0020119738780973\\
73.86	0.0020119738499606\\
73.87	0.00201197382303464\\
73.88	0.0020119737972752\\
73.89	0.00201197377263885\\
73.9	0.00201197374908298\\
73.91	0.00201197372656588\\
73.92	0.00201197370504679\\
73.93	0.00201197368448597\\
73.94	0.00201197366484473\\
73.95	0.00201197364608554\\
73.96	0.00201197362817211\\
73.97	0.00201197361106937\\
73.98	0.00201197359474366\\
73.99	0.0020119735791627\\
74	0.00201197356429568\\
74.01	0.00201197355011188\\
74.02	0.00201197353658076\\
74.03	0.00201197352367196\\
74.04	0.00201197351135532\\
74.05	0.00201197349960091\\
74.06	0.00201197348837906\\
74.07	0.00201197347766039\\
74.08	0.00201197346741581\\
74.09	0.00201197345761659\\
74.1	0.00201197344823439\\
74.11	0.00201197343924122\\
74.12	0.00201197343060957\\
74.13	0.00201197342231237\\
74.14	0.00201197341432307\\
74.15	0.00201197340661565\\
74.16	0.00201197339916464\\
74.17	0.00201197339194521\\
74.18	0.00201197338493315\\
74.19	0.00201197337810496\\
74.2	0.00201197337143783\\
74.21	0.00201197336490976\\
74.22	0.00201197335849953\\
74.23	0.00201197335218677\\
74.24	0.00201197334595204\\
74.25	0.00201197333977682\\
74.26	0.00201197333364359\\
74.27	0.00201197332753587\\
74.28	0.00201197332143827\\
74.29	0.00201197331533657\\
74.3	0.00201197330921771\\
74.31	0.00201197330306991\\
74.32	0.00201197329688269\\
74.33	0.00201197329064695\\
74.34	0.002011973284355\\
74.35	0.00201197327800066\\
74.36	0.00201197327157927\\
74.37	0.00201197326508741\\
74.38	0.00201197325852225\\
74.39	0.00201197325188167\\
74.4	0.00201197324516419\\
74.41	0.00201197323836893\\
74.42	0.00201197323149498\\
74.43	0.00201197322454145\\
74.44	0.0020119732175074\\
74.45	0.00201197321039192\\
74.46	0.00201197320319406\\
74.47	0.00201197319591287\\
74.48	0.00201197318854738\\
74.49	0.00201197318109663\\
74.5	0.00201197317355963\\
74.51	0.00201197316593538\\
74.52	0.00201197315822287\\
74.53	0.00201197315042109\\
74.54	0.00201197314252901\\
74.55	0.00201197313454558\\
74.56	0.00201197312646974\\
74.57	0.00201197311830044\\
74.58	0.00201197311003659\\
74.59	0.00201197310167711\\
74.6	0.00201197309322088\\
74.61	0.00201197308466679\\
74.62	0.00201197307601371\\
74.63	0.00201197306726049\\
74.64	0.002011973058406\\
74.65	0.00201197304944904\\
74.66	0.00201197304038844\\
74.67	0.00201197303122301\\
74.68	0.00201197302195154\\
74.69	0.00201197301257279\\
74.7	0.00201197300308554\\
74.71	0.00201197299348853\\
74.72	0.00201197298378049\\
74.73	0.00201197297396015\\
74.74	0.0020119729640262\\
74.75	0.00201197295397734\\
74.76	0.00201197294381224\\
74.77	0.00201197293352956\\
74.78	0.00201197292312795\\
74.79	0.00201197291260602\\
74.8	0.0020119729019624\\
74.81	0.00201197289119568\\
74.82	0.00201197288030443\\
74.83	0.00201197286928723\\
74.84	0.00201197285814262\\
74.85	0.00201197284686913\\
74.86	0.00201197283546528\\
74.87	0.00201197282392955\\
74.88	0.00201197281226044\\
74.89	0.00201197280045639\\
74.9	0.00201197278851586\\
74.91	0.00201197277643727\\
74.92	0.00201197276421903\\
74.93	0.00201197275185953\\
74.94	0.00201197273935714\\
74.95	0.00201197272671021\\
74.96	0.00201197271391708\\
74.97	0.00201197270097605\\
74.98	0.00201197268788544\\
74.99	0.0020119726746435\\
75	0.0020119726612485\\
75.01	0.00201197264769867\\
75.02	0.00201197263399223\\
75.03	0.00201197262012737\\
75.04	0.00201197260610226\\
75.05	0.00201197259191506\\
75.06	0.0020119725775639\\
75.07	0.00201197256304689\\
75.08	0.00201197254836212\\
75.09	0.00201197253350766\\
75.1	0.00201197251848154\\
75.11	0.0020119725032818\\
75.12	0.00201197248790643\\
75.13	0.00201197247235341\\
75.14	0.00201197245662068\\
75.15	0.00201197244070618\\
75.16	0.00201197242460782\\
75.17	0.00201197240832348\\
75.18	0.00201197239185101\\
75.19	0.00201197237518824\\
75.2	0.00201197235833299\\
75.21	0.00201197234128304\\
75.22	0.00201197232403614\\
75.23	0.00201197230659003\\
75.24	0.0020119722889424\\
75.25	0.00201197227109095\\
75.26	0.00201197225303332\\
75.27	0.00201197223476713\\
75.28	0.00201197221628999\\
75.29	0.00201197219759946\\
75.3	0.0020119721786931\\
75.31	0.0020119721595684\\
75.32	0.00201197214022287\\
75.33	0.00201197212065394\\
75.34	0.00201197210085907\\
75.35	0.00201197208083563\\
75.36	0.002011972060581\\
75.37	0.00201197204009253\\
75.38	0.0020119720193675\\
75.39	0.00201197199840321\\
75.4	0.0020119719771969\\
75.41	0.00201197195574578\\
75.42	0.00201197193404703\\
75.43	0.00201197191209781\\
75.44	0.00201197188989523\\
75.45	0.00201197186743637\\
75.46	0.00201197184471829\\
75.47	0.00201197182173799\\
75.48	0.00201197179849247\\
75.49	0.00201197177497868\\
75.5	0.00201197175119351\\
75.51	0.00201197172713386\\
75.52	0.00201197170279656\\
75.53	0.00201197167817841\\
75.54	0.00201197165327619\\
75.55	0.00201197162808663\\
75.56	0.00201197160260643\\
75.57	0.00201197157683222\\
75.58	0.00201197155076065\\
75.59	0.00201197152438828\\
75.6	0.00201197149771165\\
75.61	0.00201197147072727\\
75.62	0.00201197144343159\\
75.63	0.00201197141582103\\
75.64	0.00201197138789198\\
75.65	0.00201197135964076\\
75.66	0.00201197133106368\\
75.67	0.00201197130215698\\
75.68	0.00201197127291688\\
75.69	0.00201197124333954\\
75.7	0.00201197121342109\\
75.71	0.00201197118315759\\
75.72	0.00201197115254509\\
75.73	0.00201197112157956\\
75.74	0.00201197109025696\\
75.75	0.00201197105857317\\
75.76	0.00201197102652405\\
75.77	0.00201197099410539\\
75.78	0.00201197096131294\\
75.79	0.00201197092814241\\
75.8	0.00201197089458946\\
75.81	0.00201197086064968\\
75.82	0.00201197082631863\\
75.83	0.00201197079159182\\
75.84	0.00201197075646469\\
75.85	0.00201197072093264\\
75.86	0.00201197068499102\\
75.87	0.00201197064863513\\
75.88	0.0020119706118602\\
75.89	0.00201197057466142\\
75.9	0.00201197053703392\\
75.91	0.00201197049897276\\
75.92	0.00201197046047298\\
75.93	0.00201197042152953\\
75.94	0.0020119703821373\\
75.95	0.00201197034229116\\
75.96	0.00201197030198588\\
75.97	0.00201197026121618\\
75.98	0.00201197021997674\\
75.99	0.00201197017826216\\
76	0.00201197013606697\\
76.01	0.00201197009338567\\
76.02	0.00201197005021266\\
76.03	0.0020119700065423\\
76.04	0.00201196996236888\\
76.05	0.00201196991768663\\
76.06	0.00201196987248969\\
76.07	0.00201196982677215\\
76.08	0.00201196978052805\\
76.09	0.00201196973375133\\
76.1	0.00201196968643588\\
76.11	0.00201196963857552\\
76.12	0.00201196959016398\\
76.13	0.00201196954119495\\
76.14	0.00201196949166202\\
76.15	0.00201196944155871\\
76.16	0.00201196939087848\\
76.17	0.00201196933961472\\
76.18	0.00201196928776071\\
76.19	0.00201196923530969\\
76.2	0.0020119691822548\\
76.21	0.00201196912858912\\
76.22	0.00201196907430564\\
76.23	0.00201196901939726\\
76.24	0.00201196896385681\\
76.25	0.00201196890767705\\
76.26	0.00201196885085063\\
76.27	0.00201196879337015\\
76.28	0.00201196873522808\\
76.29	0.00201196867641685\\
76.3	0.00201196861692878\\
76.31	0.0020119685567561\\
76.32	0.00201196849589095\\
76.33	0.00201196843432541\\
76.34	0.00201196837205142\\
76.35	0.00201196830906087\\
76.36	0.00201196824534555\\
76.37	0.00201196818089712\\
76.38	0.0020119681157072\\
76.39	0.00201196804976727\\
76.4	0.00201196798306873\\
76.41	0.0020119679156029\\
76.42	0.00201196784736096\\
76.43	0.00201196777833402\\
76.44	0.00201196770851307\\
76.45	0.00201196763788903\\
76.46	0.00201196756645267\\
76.47	0.00201196749419469\\
76.48	0.00201196742110567\\
76.49	0.00201196734717608\\
76.5	0.00201196727239628\\
76.51	0.00201196719675654\\
76.52	0.00201196712024699\\
76.53	0.00201196704285767\\
76.54	0.0020119669645785\\
76.55	0.00201196688539927\\
76.56	0.00201196680530967\\
76.57	0.00201196672429928\\
76.58	0.00201196664235753\\
76.59	0.00201196655947376\\
76.6	0.00201196647563718\\
76.61	0.00201196639083686\\
76.62	0.00201196630506177\\
76.63	0.00201196621830075\\
76.64	0.00201196613054249\\
76.65	0.00201196604177557\\
76.66	0.00201196595198844\\
76.67	0.00201196586116941\\
76.68	0.00201196576930667\\
76.69	0.00201196567638826\\
76.7	0.0020119655824021\\
76.71	0.00201196548733594\\
76.72	0.00201196539117743\\
76.73	0.00201196529391406\\
76.74	0.00201196519553317\\
76.75	0.00201196509602198\\
76.76	0.00201196499536753\\
76.77	0.00201196489355675\\
76.78	0.00201196479057638\\
76.79	0.00201196468641305\\
76.8	0.00201196458105321\\
76.81	0.00201196447448317\\
76.82	0.00201196436668906\\
76.83	0.00201196425765689\\
76.84	0.00201196414737248\\
76.85	0.0020119640358215\\
76.86	0.00201196392298945\\
76.87	0.00201196380886168\\
76.88	0.00201196369342336\\
76.89	0.00201196357665948\\
76.9	0.00201196345855489\\
76.91	0.00201196333909425\\
76.92	0.00201196321826203\\
76.93	0.00201196309604255\\
76.94	0.00201196297241994\\
76.95	0.00201196284737814\\
76.96	0.00201196272090092\\
76.97	0.00201196259297186\\
76.98	0.00201196246357436\\
76.99	0.00201196233269162\\
77	0.00201196220030666\\
77.01	0.00201196206640228\\
77.02	0.00201196193096113\\
77.03	0.00201196179396562\\
77.04	0.00201196165539797\\
77.05	0.00201196151524022\\
77.06	0.00201196137347419\\
77.07	0.00201196123008148\\
77.08	0.00201196108504349\\
77.09	0.00201196093834143\\
77.1	0.00201196078995626\\
77.11	0.00201196063986874\\
77.12	0.00201196048805942\\
77.13	0.00201196033450862\\
77.14	0.00201196017919643\\
77.15	0.00201196002210272\\
77.16	0.00201195986320713\\
77.17	0.00201195970248907\\
77.18	0.00201195953992772\\
77.19	0.002011959375502\\
77.2	0.00201195920919062\\
77.21	0.00201195904097203\\
77.22	0.00201195887082445\\
77.23	0.00201195869872583\\
77.24	0.00201195852465389\\
77.25	0.00201195834858607\\
77.26	0.0020119581704996\\
77.27	0.0020119579903714\\
77.28	0.00201195780817815\\
77.29	0.00201195762389628\\
77.3	0.00201195743750192\\
77.31	0.00201195724897095\\
77.32	0.00201195705827898\\
77.33	0.00201195686540132\\
77.34	0.00201195667031303\\
77.35	0.00201195647298884\\
77.36	0.00201195627340325\\
77.37	0.00201195607153043\\
77.38	0.00201195586734427\\
77.39	0.00201195566081835\\
77.4	0.00201195545192598\\
77.41	0.00201195524064014\\
77.42	0.00201195502693351\\
77.43	0.00201195481077846\\
77.44	0.00201195459214705\\
77.45	0.00201195437101101\\
77.46	0.00201195414734177\\
77.47	0.00201195392111041\\
77.48	0.0020119536922877\\
77.49	0.00201195346084408\\
77.5	0.00201195322674964\\
77.51	0.00201195298997414\\
77.52	0.00201195275048698\\
77.53	0.00201195250825725\\
77.54	0.00201195226325364\\
77.55	0.00201195201544453\\
77.56	0.00201195176479791\\
77.57	0.00201195151128141\\
77.58	0.00201195125486231\\
77.59	0.00201195099550751\\
77.6	0.00201195073318352\\
77.61	0.0020119504678565\\
77.62	0.00201195019949218\\
77.63	0.00201194992805596\\
77.64	0.00201194965351279\\
77.65	0.00201194937582727\\
77.66	0.00201194909496355\\
77.67	0.00201194881088543\\
77.68	0.00201194852355624\\
77.69	0.00201194823293895\\
77.7	0.00201194793899606\\
77.71	0.00201194764168968\\
77.72	0.00201194734098148\\
77.73	0.00201194703683267\\
77.74	0.00201194672920407\\
77.75	0.00201194641805602\\
77.76	0.0020119461033484\\
77.77	0.00201194578504068\\
77.78	0.00201194546309182\\
77.79	0.00201194513746036\\
77.8	0.00201194480810433\\
77.81	0.00201194447498131\\
77.82	0.00201194413804839\\
77.83	0.00201194379726218\\
77.84	0.00201194345257878\\
77.85	0.00201194310395382\\
77.86	0.0020119427513424\\
77.87	0.00201194239469913\\
77.88	0.0020119420339781\\
77.89	0.00201194166913288\\
77.9	0.0020119413001165\\
77.91	0.00201194092688148\\
77.92	0.00201194054937979\\
77.93	0.00201194016756285\\
77.94	0.00201193978138156\\
77.95	0.00201193939078622\\
77.96	0.00201193899572659\\
77.97	0.00201193859615187\\
77.98	0.00201193819201067\\
77.99	0.00201193778325102\\
78	0.00201193736982036\\
78.01	0.00201193695166555\\
78.02	0.00201193652873282\\
78.03	0.00201193610096781\\
78.04	0.00201193566831555\\
78.05	0.00201193523072044\\
78.06	0.00201193478812624\\
78.07	0.00201193434047609\\
78.08	0.00201193388771248\\
78.09	0.00201193342977725\\
78.1	0.00201193296661157\\
78.11	0.00201193249815597\\
78.12	0.00201193202435029\\
78.13	0.00201193154513368\\
78.14	0.00201193106044462\\
78.15	0.00201193057022088\\
78.16	0.00201193007439954\\
78.17	0.00201192957291696\\
78.18	0.00201192906570878\\
78.19	0.0020119285527099\\
78.2	0.00201192803385452\\
78.21	0.00201192750907605\\
78.22	0.00201192697830718\\
78.23	0.00201192644147983\\
78.24	0.00201192589852514\\
78.25	0.00201192534937348\\
78.26	0.00201192479395444\\
78.27	0.0020119242321968\\
78.28	0.00201192366402854\\
78.29	0.00201192308937683\\
78.3	0.00201192250816801\\
78.31	0.0020119219203276\\
78.32	0.00201192132578026\\
78.33	0.0020119207244498\\
78.34	0.0020119201162592\\
78.35	0.00201191950113053\\
78.36	0.002011918878985\\
78.37	0.00201191824974293\\
78.38	0.00201191761332373\\
78.39	0.0020119169696459\\
78.4	0.00201191631862704\\
78.41	0.00201191566018379\\
78.42	0.00201191499423187\\
78.43	0.00201191432068604\\
78.44	0.0020119136394601\\
78.45	0.00201191295046686\\
78.46	0.00201191225361818\\
78.47	0.0020119115488249\\
78.48	0.00201191083599686\\
78.49	0.00201191011504287\\
78.5	0.00201190938587073\\
78.51	0.00201190864838718\\
78.52	0.00201190790249794\\
78.53	0.00201190714810762\\
78.54	0.0020119063851198\\
78.55	0.00201190561343693\\
78.56	0.00201190483296039\\
78.57	0.00201190404359044\\
78.58	0.0020119032452262\\
78.59	0.00201190243776567\\
78.6	0.00201190162110568\\
78.61	0.00201190079514193\\
78.62	0.00201189995976891\\
78.63	0.00201189911487994\\
78.64	0.00201189826036711\\
78.65	0.00201189739612134\\
78.66	0.00201189652203227\\
78.67	0.00201189563798833\\
78.68	0.00201189474387668\\
78.69	0.00201189383958321\\
78.7	0.00201189292499251\\
78.71	0.0020118919999879\\
78.72	0.00201189106445136\\
78.73	0.00201189011826354\\
78.74	0.00201188916130375\\
78.75	0.00201188819344995\\
78.76	0.00201188721457872\\
78.77	0.00201188622456524\\
78.78	0.00201188522328328\\
78.79	0.00201188421060521\\
78.8	0.00201188318640196\\
78.81	0.00201188215054297\\
78.82	0.00201188110289626\\
78.83	0.00201188004332833\\
78.84	0.00201187897170418\\
78.85	0.00201187788788731\\
78.86	0.00201187679173966\\
78.87	0.00201187568312163\\
78.88	0.00201187456189205\\
78.89	0.00201187342790814\\
78.9	0.00201187228102554\\
78.91	0.00201187112109824\\
78.92	0.00201186994797861\\
78.93	0.00201186876151735\\
78.94	0.00201186756156348\\
78.95	0.00201186634796431\\
78.96	0.00201186512056544\\
78.97	0.00201186387921075\\
78.98	0.00201186262374234\\
78.99	0.00201186135400054\\
79	0.00201186006982389\\
79.01	0.00201185877104911\\
79.02	0.00201185745751109\\
79.03	0.00201185612904286\\
79.04	0.00201185478547556\\
79.05	0.00201185342663844\\
79.06	0.00201185205235885\\
79.07	0.00201185066246218\\
79.08	0.00201184925677184\\
79.09	0.0020118478351093\\
79.1	0.00201184639729398\\
79.11	0.00201184494314329\\
79.12	0.0020118434724726\\
79.13	0.00201184198509519\\
79.14	0.00201184048082224\\
79.15	0.00201183895946281\\
79.16	0.00201183742082382\\
79.17	0.00201183586471003\\
79.18	0.00201183429092399\\
79.19	0.00201183269926604\\
79.2	0.00201183108953429\\
79.21	0.00201182946152455\\
79.22	0.00201182781503037\\
79.23	0.00201182614984297\\
79.24	0.00201182446575123\\
79.25	0.00201182276254165\\
79.26	0.00201182103999835\\
79.27	0.002011819297903\\
79.28	0.00201181753603485\\
79.29	0.00201181575417067\\
79.3	0.0020118139520847\\
79.31	0.00201181212954866\\
79.32	0.00201181028633172\\
79.33	0.00201180842220045\\
79.34	0.00201180653691881\\
79.35	0.00201180463024809\\
79.36	0.00201180270194694\\
79.37	0.00201180075177126\\
79.38	0.00201179877947426\\
79.39	0.00201179678480633\\
79.4	0.00201179476751511\\
79.41	0.00201179272734539\\
79.42	0.00201179066403909\\
79.43	0.00201178857733526\\
79.44	0.00201178646696999\\
79.45	0.00201178433267646\\
79.46	0.00201178217418482\\
79.47	0.00201177999122222\\
79.48	0.00201177778351273\\
79.49	0.00201177555077735\\
79.5	0.00201177329273396\\
79.51	0.00201177100909725\\
79.52	0.00201176869957874\\
79.53	0.00201176636388671\\
79.54	0.00201176400172617\\
79.55	0.00201176161279884\\
79.56	0.00201175919680308\\
79.57	0.0020117567534339\\
79.58	0.00201175428238287\\
79.59	0.00201175178333811\\
79.6	0.00201174925598428\\
79.61	0.00201174670000246\\
79.62	0.00201174411507021\\
79.63	0.00201174150086146\\
79.64	0.00201173885704648\\
79.65	0.00201173618329188\\
79.66	0.00201173347926051\\
79.67	0.00201173074461147\\
79.68	0.00201172797900003\\
79.69	0.00201172518207763\\
79.7	0.00201172235349179\\
79.71	0.0020117194928861\\
79.72	0.00201171659990015\\
79.73	0.00201171367416952\\
79.74	0.0020117107153257\\
79.75	0.00201170772299608\\
79.76	0.00201170469680386\\
79.77	0.00201170163636805\\
79.78	0.00201169854130339\\
79.79	0.00201169541122033\\
79.8	0.00201169224572495\\
79.81	0.00201168904441893\\
79.82	0.00201168580689953\\
79.83	0.00201168253275946\\
79.84	0.00201167922158693\\
79.85	0.00201167587296552\\
79.86	0.00201167248647418\\
79.87	0.00201166906168715\\
79.88	0.0020116655981739\\
79.89	0.00201166209549913\\
79.9	0.00201165855322264\\
79.91	0.00201165497089935\\
79.92	0.0020116513480792\\
79.93	0.00201164768430711\\
79.94	0.00201164397912291\\
79.95	0.0020116402320613\\
79.96	0.00201163644265181\\
79.97	0.0020116326104187\\
79.98	0.00201162873488092\\
79.99	0.00201162481555207\\
80	0.00201162085194031\\
80.01	0.00201161684354834\\
};
\addplot [color=blue,dashed]
  table[row sep=crcr]{%
80.01	0.00201161684354834\\
80.02	0.00201161278987329\\
80.03	0.00201160869040669\\
80.04	0.0020116045446344\\
80.05	0.00201160035203657\\
80.06	0.00201159611208752\\
80.07	0.00201159182425574\\
80.08	0.00201158748800378\\
80.09	0.00201158310278822\\
80.1	0.00201157866805956\\
80.11	0.00201157418326219\\
80.12	0.00201156964783432\\
80.13	0.00201156506120789\\
80.14	0.00201156042280853\\
80.15	0.00201155573205544\\
80.16	0.00201155098836138\\
80.17	0.00201154619113258\\
80.18	0.00201154133976863\\
80.19	0.00201153643366246\\
80.2	0.00201153147220022\\
80.21	0.00201152645476126\\
80.22	0.00201152138071799\\
80.23	0.00201151624943585\\
80.24	0.00201151106027323\\
80.25	0.00201150581258137\\
80.26	0.00201150050570428\\
80.27	0.00201149513897871\\
80.28	0.002011489711734\\
80.29	0.00201148422329205\\
80.3	0.0020114786729672\\
80.31	0.00201147306006618\\
80.32	0.00201146738388802\\
80.33	0.00201146164372395\\
80.34	0.00201145583885732\\
80.35	0.0020114499685635\\
80.36	0.00201144403210984\\
80.37	0.00201143802875551\\
80.38	0.0020114319577515\\
80.39	0.00201142581834042\\
80.4	0.00201141960975651\\
80.41	0.00201141333122548\\
80.42	0.00201140698196446\\
80.43	0.00201140056118186\\
80.44	0.00201139406807733\\
80.45	0.00201138750184162\\
80.46	0.00201138086165648\\
80.47	0.0020113741466946\\
80.48	0.00201136735611948\\
80.49	0.00201136048908533\\
80.5	0.00201135354473699\\
80.51	0.00201134652220979\\
80.52	0.00201133942062949\\
80.53	0.00201133223911212\\
80.54	0.00201132497676394\\
80.55	0.00201131763268127\\
80.56	0.00201131020595043\\
80.57	0.00201130269564758\\
80.58	0.00201129510083867\\
80.59	0.00201128742057926\\
80.6	0.00201127965391448\\
80.61	0.00201127179987884\\
80.62	0.00201126385749616\\
80.63	0.00201125582577946\\
80.64	0.0020112477037308\\
80.65	0.0020112394903412\\
80.66	0.00201123118459049\\
80.67	0.00201122278544721\\
80.68	0.00201121429186847\\
80.69	0.00201120570279983\\
80.7	0.00201119701717517\\
80.71	0.00201118823391657\\
80.72	0.00201117935193419\\
80.73	0.0020111703701261\\
80.74	0.00201116128737819\\
80.75	0.00201115210256402\\
80.76	0.00201114281454468\\
80.77	0.00201113342216866\\
80.78	0.00201112392427171\\
80.79	0.00201111431967669\\
80.8	0.00201110460719346\\
80.81	0.00201109478561871\\
80.82	0.00201108485373581\\
80.83	0.0020110748103147\\
80.84	0.00201106465411169\\
80.85	0.00201105438386938\\
80.86	0.00201104399831644\\
80.87	0.00201103349616749\\
80.88	0.00201102287612297\\
80.89	0.00201101213686894\\
80.9	0.00201100127707695\\
80.91	0.00201099029540387\\
80.92	0.00201097919049173\\
80.93	0.00201096796096758\\
80.94	0.00201095660544329\\
80.95	0.00201094512251541\\
80.96	0.00201093351076498\\
80.97	0.00201092176875741\\
80.98	0.00201090989504224\\
80.99	0.00201089788815302\\
81	0.0020108857466071\\
81.01	0.0020108734689055\\
81.02	0.00201086105353267\\
81.03	0.00201084849895636\\
81.04	0.00201083580362741\\
81.05	0.00201082296597958\\
81.06	0.00201080998442936\\
81.07	0.00201079685737576\\
81.08	0.00201078358320017\\
81.09	0.00201077016026611\\
81.1	0.00201075658691907\\
81.11	0.00201074286148633\\
81.12	0.00201072898227669\\
81.13	0.00201071494758035\\
81.14	0.00201070075566868\\
81.15	0.00201068640479397\\
81.16	0.00201067189318929\\
81.17	0.00201065721906824\\
81.18	0.00201064238062476\\
81.19	0.0020106273760329\\
81.2	0.00201061220344659\\
81.21	0.00201059686099947\\
81.22	0.00201058134680461\\
81.23	0.00201056565895434\\
81.24	0.00201054979551997\\
81.25	0.00201053375455163\\
81.26	0.00201051753407796\\
81.27	0.00201050113210594\\
81.28	0.00201048454662062\\
81.29	0.0020104677755849\\
81.3	0.00201045081693929\\
81.31	0.00201043366860163\\
81.32	0.0020104163284669\\
81.33	0.00201039879440694\\
81.34	0.00201038106427019\\
81.35	0.00201036313588145\\
81.36	0.00201034500704163\\
81.37	0.00201032667552749\\
81.38	0.00201030813909136\\
81.39	0.00201028939546087\\
81.4	0.00201027044233872\\
81.41	0.00201025127740238\\
81.42	0.00201023189830383\\
81.43	0.00201021230266926\\
81.44	0.00201019248809881\\
81.45	0.00201017245216631\\
81.46	0.00201015219241893\\
81.47	0.00201013170637696\\
81.48	0.00201011099153348\\
81.49	0.00201009004535407\\
81.5	0.00201006886527652\\
81.51	0.00201004744871053\\
81.52	0.00201002579303739\\
81.53	0.0020100038956097\\
81.54	0.00200998175375102\\
81.55	0.0020099593647556\\
81.56	0.00200993672588803\\
81.57	0.00200991383438294\\
81.58	0.00200989068744465\\
81.59	0.00200986728224688\\
81.6	0.00200984361593237\\
81.61	0.00200981968561261\\
81.62	0.00200979548836743\\
81.63	0.00200977102124471\\
81.64	0.00200974628126001\\
81.65	0.00200972126539624\\
81.66	0.00200969597060327\\
81.67	0.00200967039379764\\
81.68	0.00200964453186212\\
81.69	0.0020096183816454\\
81.7	0.0020095919399617\\
81.71	0.00200956520359041\\
81.72	0.00200953816927571\\
81.73	0.00200951083372619\\
81.74	0.00200948319361444\\
81.75	0.00200945524557672\\
81.76	0.00200942698621251\\
81.77	0.00200939841208414\\
81.78	0.00200936951971641\\
81.79	0.00200934030559612\\
81.8	0.00200931076617175\\
81.81	0.00200928089785297\\
81.82	0.00200925069701026\\
81.83	0.00200922015997448\\
81.84	0.00200918928303644\\
81.85	0.00200915806244648\\
81.86	0.00200912649441402\\
81.87	0.00200909457510712\\
81.88	0.00200906230065206\\
81.89	0.00200902966713285\\
81.9	0.00200899667059082\\
81.91	0.0020089633070241\\
81.92	0.00200892957238724\\
81.93	0.00200889546259066\\
81.94	0.00200886097350023\\
81.95	0.00200882610093676\\
81.96	0.00200879084067554\\
81.97	0.00200875518844581\\
81.98	0.00200871913993033\\
81.99	0.00200868269076483\\
82	0.00200864583653751\\
82.01	0.00200860857278855\\
82.02	0.00200857089500959\\
82.03	0.00200853279864319\\
82.04	0.00200849427908232\\
82.05	0.00200845533166983\\
82.06	0.00200841595169792\\
82.07	0.00200837613440755\\
82.08	0.00200833587498795\\
82.09	0.00200829516857605\\
82.1	0.00200825401025588\\
82.11	0.00200821239505806\\
82.12	0.0020081703179592\\
82.13	0.0020081277738813\\
82.14	0.00200808475769121\\
82.15	0.00200804126420002\\
82.16	0.00200799728816245\\
82.17	0.00200795282427626\\
82.18	0.00200790786718165\\
82.19	0.00200786241146061\\
82.2	0.00200781645163635\\
82.21	0.00200776998217263\\
82.22	0.00200772299747312\\
82.23	0.00200767549188081\\
82.24	0.0020076274596773\\
82.25	0.00200757889508217\\
82.26	0.00200752979225234\\
82.27	0.00200748014528135\\
82.28	0.00200742994819875\\
82.29	0.00200737919496933\\
82.3	0.00200732787949253\\
82.31	0.00200727599560166\\
82.32	0.00200722353706323\\
82.33	0.00200717049757624\\
82.34	0.00200711687077146\\
82.35	0.00200706265021068\\
82.36	0.00200700782938599\\
82.37	0.00200695240171906\\
82.38	0.00200689636056036\\
82.39	0.0020068396991884\\
82.4	0.00200678241080897\\
82.41	0.0020067244885544\\
82.42	0.00200666592548272\\
82.43	0.00200660671457691\\
82.44	0.00200654684874408\\
82.45	0.00200648632081467\\
82.46	0.00200642512354166\\
82.47	0.00200636324959969\\
82.48	0.00200630069158428\\
82.49	0.00200623744201097\\
82.5	0.00200617349331446\\
82.51	0.00200610883784778\\
82.52	0.00200604346788137\\
82.53	0.00200597737560228\\
82.54	0.0020059105531132\\
82.55	0.00200584299243164\\
82.56	0.00200577468548897\\
82.57	0.00200570562412954\\
82.58	0.00200563580010974\\
82.59	0.00200556520509708\\
82.6	0.00200549383066925\\
82.61	0.00200542166831313\\
82.62	0.00200534870942389\\
82.63	0.00200527494530396\\
82.64	0.00200520036716209\\
82.65	0.00200512496611235\\
82.66	0.00200504873317308\\
82.67	0.00200497194232735\\
82.68	0.00200489471204742\\
82.69	0.00200481703781737\\
82.7	0.00200473891507041\\
82.71	0.00200466033918833\\
82.72	0.0020045813055009\\
82.73	0.00200450180928527\\
82.74	0.00200442184576537\\
82.75	0.00200434141011133\\
82.76	0.00200426049743881\\
82.77	0.00200417910280845\\
82.78	0.00200409722122516\\
82.79	0.00200401484763756\\
82.8	0.00200393197693729\\
82.81	0.00200384860395838\\
82.82	0.00200376472347659\\
82.83	0.00200368033020871\\
82.84	0.00200359541881195\\
82.85	0.00200350998388322\\
82.86	0.00200342401995844\\
82.87	0.00200333752151185\\
82.88	0.00200325048295532\\
82.89	0.00200316289863762\\
82.9	0.0020030747628437\\
82.91	0.00200298606979398\\
82.92	0.00200289681364357\\
82.93	0.00200280698848159\\
82.94	0.00200271658833037\\
82.95	0.00200262560714467\\
82.96	0.00200253403881098\\
82.97	0.00200244187714667\\
82.98	0.00200234911589924\\
82.99	0.0020022557487455\\
83	0.00200216176929078\\
83.01	0.0020020671710681\\
83.02	0.00200197194753738\\
83.03	0.00200187609208454\\
83.04	0.00200177959802071\\
83.05	0.00200168245858136\\
83.06	0.00200158466692543\\
83.07	0.00200148621613448\\
83.08	0.00200138709921175\\
83.09	0.00200128730908135\\
83.1	0.00200118683858727\\
83.11	0.00200108568049254\\
83.12	0.00200098382747824\\
83.13	0.00200088127214263\\
83.14	0.00200077800700016\\
83.15	0.0020006740244805\\
83.16	0.00200056931692764\\
83.17	0.00200046387659883\\
83.18	0.00200035769566366\\
83.19	0.00200025076620301\\
83.2	0.00200014308020807\\
83.21	0.00200003462957928\\
83.22	0.00199992540612534\\
83.23	0.00199981540156213\\
83.24	0.00199970460751167\\
83.25	0.00199959301550102\\
83.26	0.00199948061696124\\
83.27	0.00199936740322625\\
83.28	0.00199925336553175\\
83.29	0.00199913849501408\\
83.3	0.00199902278270913\\
83.31	0.00199890621955112\\
83.32	0.0019987887963715\\
83.33	0.00199867050389776\\
83.34	0.00199855133275223\\
83.35	0.00199843127345091\\
83.36	0.00199831031640222\\
83.37	0.00199818845190579\\
83.38	0.00199806567015124\\
83.39	0.00199794196121689\\
83.4	0.00199781731506849\\
83.41	0.00199769172155798\\
83.42	0.00199756517042215\\
83.43	0.00199743765128132\\
83.44	0.00199730915363806\\
83.45	0.00199717966687579\\
83.46	0.00199704918025745\\
83.47	0.00199691768292414\\
83.48	0.00199678516389369\\
83.49	0.00199665161205929\\
83.5	0.00199651701618805\\
83.51	0.00199638136491956\\
83.52	0.00199624464676445\\
83.53	0.0019961068501029\\
83.54	0.00199596796318318\\
83.55	0.0019958279741201\\
83.56	0.00199568687089354\\
83.57	0.00199554464134688\\
83.58	0.00199540127318543\\
83.59	0.0019952567539749\\
83.6	0.00199511107113978\\
83.61	0.00199496421196171\\
83.62	0.00199481616357789\\
83.63	0.00199466691297943\\
83.64	0.00199451644700962\\
83.65	0.00199436475236235\\
83.66	0.00199421181558032\\
83.67	0.00199405762305335\\
83.68	0.00199390216101663\\
83.69	0.00199374541554896\\
83.7	0.00199358737257097\\
83.71	0.00199342801784332\\
83.72	0.00199326733696484\\
83.73	0.00199310531537074\\
83.74	0.00199294193833072\\
83.75	0.00199277719094705\\
83.76	0.00199261105815272\\
83.77	0.00199244352470947\\
83.78	0.00199227457520585\\
83.79	0.00199210419405524\\
83.8	0.00199193236549388\\
83.81	0.0019917590735788\\
83.82	0.00199158430218581\\
83.83	0.00199140803500744\\
83.84	0.00199123025555084\\
83.85	0.00199105094713566\\
83.86	0.00199087009289192\\
83.87	0.00199068767575784\\
83.88	0.00199050367847769\\
83.89	0.00199031808359951\\
83.9	0.00199013087347295\\
83.91	0.00198994203024695\\
83.92	0.0019897515358675\\
83.93	0.00198955937207527\\
83.94	0.00198936552040332\\
83.95	0.00198916996217472\\
83.96	0.00198897267850016\\
83.97	0.00198877365027549\\
83.98	0.00198857285817933\\
83.99	0.00198837028267058\\
84	0.00198816590398587\\
84.01	0.00198795970213709\\
84.02	0.00198775165690879\\
84.03	0.00198754174785561\\
84.04	0.00198732995429964\\
84.05	0.00198711625532778\\
84.06	0.00198690062978907\\
84.07	0.00198668305629197\\
84.08	0.00198646351320161\\
84.09	0.00198624197863704\\
84.1	0.0019860184304684\\
84.11	0.00198579284631413\\
84.12	0.00198556520353802\\
84.13	0.00198533547924641\\
84.14	0.00198510365028516\\
84.15	0.00198486969323677\\
84.16	0.00198463358441731\\
84.17	0.00198439529987343\\
84.18	0.00198415481537927\\
84.19	0.00198391210643335\\
84.2	0.00198366714825545\\
84.21	0.00198341991578344\\
84.22	0.00198317038367002\\
84.23	0.00198291852627953\\
84.24	0.00198266431768465\\
84.25	0.00198240773166304\\
84.26	0.00198214874169403\\
84.27	0.00198188732095521\\
84.28	0.00198162344231895\\
84.29	0.00198135707834899\\
84.3	0.00198108820129689\\
84.31	0.00198081678309846\\
84.32	0.00198054279537021\\
84.33	0.00198026620940566\\
84.34	0.00197998699617169\\
84.35	0.00197970512630485\\
84.36	0.00197942057010751\\
84.37	0.00197913329754416\\
84.38	0.00197884327823749\\
84.39	0.00197855048146453\\
84.4	0.0019782548761527\\
84.41	0.00197795643087582\\
84.42	0.00197765511385011\\
84.43	0.00197735089293011\\
84.44	0.00197704373560456\\
84.45	0.00197673360899221\\
84.46	0.00197642047983764\\
84.47	0.001976104314507\\
84.48	0.00197578507898367\\
84.49	0.00197546273886393\\
84.5	0.00197513725935256\\
84.51	0.00197480860525833\\
84.52	0.00197447674098956\\
84.53	0.00197414163054952\\
84.54	0.00197380323753183\\
84.55	0.00197346152511577\\
84.56	0.00197311645606162\\
84.57	0.00197276799270583\\
84.58	0.00197241609695624\\
84.59	0.00197206073028717\\
84.6	0.00197170185373449\\
84.61	0.00197133942789065\\
84.62	0.0019709734128996\\
84.63	0.00197060376845172\\
84.64	0.00197023045377862\\
84.65	0.00196985342764794\\
84.66	0.00196947264835806\\
84.67	0.00196908807373277\\
84.68	0.00196869966111584\\
84.69	0.00196830736736559\\
84.7	0.00196791114884935\\
84.71	0.00196751096143785\\
84.72	0.00196710676049961\\
84.73	0.00196669850089517\\
84.74	0.00196628613697135\\
84.75	0.0019658696225554\\
84.76	0.00196544891094906\\
84.77	0.00196502395492258\\
84.78	0.0019645947067087\\
84.79	0.00196416111799652\\
84.8	0.00196372313992528\\
84.81	0.00196328072307813\\
84.82	0.00196283381747582\\
84.83	0.00196238237257025\\
84.84	0.00196192633723801\\
84.85	0.00196146565977387\\
84.86	0.0019610002878841\\
84.87	0.00196053016867981\\
84.88	0.00196005524867015\\
84.89	0.00195957547375548\\
84.9	0.00195909078922043\\
84.91	0.00195860113972688\\
84.92	0.0019581064693069\\
84.93	0.00195760672135553\\
84.94	0.0019571018386236\\
84.95	0.00195659176321033\\
84.96	0.00195607643655594\\
84.97	0.00195555579943416\\
84.98	0.00195502979194461\\
84.99	0.00195449835350517\\
85	0.00195396142284418\\
85.01	0.00195341893799259\\
85.02	0.00195287083627606\\
85.03	0.0019523170543069\\
85.04	0.00195175752797594\\
85.05	0.00195119219244435\\
85.06	0.00195062098213528\\
85.07	0.00195004383072554\\
85.08	0.001949460671137\\
85.09	0.00194887143552808\\
85.1	0.00194827605528502\\
85.11	0.00194767446101308\\
85.12	0.00194706658252768\\
85.13	0.00194645234884537\\
85.14	0.00194583168817475\\
85.15	0.00194520452790728\\
85.16	0.00194457079460797\\
85.17	0.00194393041400595\\
85.18	0.00194328331098496\\
85.19	0.00194262940957375\\
85.2	0.00194196863293631\\
85.21	0.00194130090336203\\
85.22	0.00194062614225575\\
85.23	0.00193994427012766\\
85.24	0.00193925520658315\\
85.25	0.00193855887031248\\
85.26	0.00193785517908034\\
85.27	0.00193714404971534\\
85.28	0.00193642539809934\\
85.29	0.00193569913915664\\
85.3	0.00193496518684312\\
85.31	0.00193422345413515\\
85.32	0.00193347385301846\\
85.33	0.00193271629447688\\
85.34	0.00193195068848087\\
85.35	0.00193117694397601\\
85.36	0.00193039496887133\\
85.37	0.00192960467002745\\
85.38	0.0019288059532447\\
85.39	0.00192799872325097\\
85.4	0.00192718288368956\\
85.41	0.00192635833710674\\
85.42	0.00192552498493933\\
85.43	0.00192468272750196\\
85.44	0.00192383146397438\\
85.45	0.00192297109238844\\
85.46	0.00192210150961506\\
85.47	0.00192122261135096\\
85.48	0.0019203342921053\\
85.49	0.00191943644518612\\
85.5	0.00191852896268664\\
85.51	0.00191761173547144\\
85.52	0.00191668465316241\\
85.53	0.00191574760412463\\
85.54	0.00191480178146727\\
85.55	0.00191384987596485\\
85.56	0.00191289182513978\\
85.57	0.00191192756580686\\
85.58	0.00191095703406505\\
85.59	0.00190998016528918\\
85.6	0.00190899689412164\\
85.61	0.00190800715446382\\
85.62	0.00190701087946759\\
85.63	0.00190600945949907\\
85.64	0.00190500301460894\\
85.65	0.00190399149098689\\
85.66	0.00190297483419296\\
85.67	0.00190195298915008\\
85.68	0.00190092590013648\\
85.69	0.00189989351077811\\
85.7	0.00189885576404078\\
85.71	0.00189781260222242\\
85.72	0.00189676396694508\\
85.73	0.00189570979914696\\
85.74	0.0018946500390742\\
85.75	0.00189358462627275\\
85.76	0.00189251349957998\\
85.77	0.0018914365971163\\
85.78	0.00189035385627661\\
85.79	0.0018892652137217\\
85.8	0.0018881706053695\\
85.81	0.00188706996638627\\
85.82	0.00188596323117765\\
85.83	0.00188485033337961\\
85.84	0.0018837312058493\\
85.85	0.00188260578065581\\
85.86	0.00188147398907074\\
85.87	0.00188033576155878\\
85.88	0.00187919102776806\\
85.89	0.00187803971652045\\
85.9	0.00187688175580174\\
85.91	0.00187571707275166\\
85.92	0.00187454559365385\\
85.93	0.00187336724392562\\
85.94	0.00187218194810767\\
85.95	0.00187098962985365\\
85.96	0.00186979021191959\\
85.97	0.00186858361615321\\
85.98	0.00186736976348309\\
85.99	0.00186614857390774\\
86	0.00186491996648453\\
86.01	0.00186368385931843\\
86.02	0.00186244016955071\\
86.03	0.00186118881334742\\
86.04	0.0018599297058878\\
86.05	0.00185866276135249\\
86.06	0.00185738789291162\\
86.07	0.00185610501271279\\
86.08	0.00185481403186887\\
86.09	0.00185351486044562\\
86.1	0.00185220740744925\\
86.11	0.00185089158081377\\
86.12	0.00184956728738816\\
86.13	0.00184823443292349\\
86.14	0.00184689292205975\\
86.15	0.00184554265831263\\
86.16	0.00184418354406008\\
86.17	0.00184281548052876\\
86.18	0.00184143836778023\\
86.19	0.00184005210469711\\
86.2	0.00183865658896894\\
86.21	0.00183725171707798\\
86.22	0.00183583738428473\\
86.23	0.00183441348461339\\
86.24	0.00183297991083704\\
86.25	0.0018315365544627\\
86.26	0.00183008330571624\\
86.27	0.00182862005352698\\
86.28	0.00182714668551225\\
86.29	0.0018256630879617\\
86.3	0.00182416914582138\\
86.31	0.0018226647426777\\
86.32	0.00182114976074115\\
86.33	0.00181962408082984\\
86.34	0.00181808758235285\\
86.35	0.00181654014329332\\
86.36	0.00181498164019144\\
86.37	0.00181341194812712\\
86.38	0.00181183094070253\\
86.39	0.0018102384900244\\
86.4	0.00180863446668609\\
86.41	0.00180701873974947\\
86.42	0.00180539117672658\\
86.43	0.001803751643561\\
86.44	0.00180210000460914\\
86.45	0.00180043612262111\\
86.46	0.00179875985872152\\
86.47	0.00179707107238998\\
86.48	0.00179536962144134\\
86.49	0.00179365536200573\\
86.5	0.00179192814850834\\
86.51	0.00179018783364899\\
86.52	0.00178843426838133\\
86.53	0.00178666730189198\\
86.54	0.00178488678157921\\
86.55	0.00178309255303155\\
86.56	0.00178128446000598\\
86.57	0.00177946234440593\\
86.58	0.00177762604625904\\
86.59	0.00177577540369457\\
86.6	0.00177391025292061\\
86.61	0.00177203042820092\\
86.62	0.00177013576183161\\
86.63	0.0017682260841174\\
86.64	0.00176630122334772\\
86.65	0.00176436100577241\\
86.66	0.00176240525557715\\
86.67	0.00176043379485866\\
86.68	0.0017584464435995\\
86.69	0.0017564430196426\\
86.7	0.00175442333866549\\
86.71	0.00175238721415419\\
86.72	0.00175033445737681\\
86.73	0.00174826487735677\\
86.74	0.00174617828084576\\
86.75	0.00174407447229632\\
86.76	0.00174195325383408\\
86.77	0.00173981442522972\\
86.78	0.00173765778387048\\
86.79	0.0017354831247314\\
86.8	0.00173329024034621\\
86.81	0.0017310789207778\\
86.82	0.00172884895358837\\
86.83	0.00172660012380917\\
86.84	0.00172433221390998\\
86.85	0.00172204500376802\\
86.86	0.00171973827063665\\
86.87	0.00171741178911363\\
86.88	0.0017150653311089\\
86.89	0.00171269866581212\\
86.9	0.00171031155965966\\
86.91	0.00170790377630128\\
86.92	0.00170547507656632\\
86.93	0.00170302521842955\\
86.94	0.00170056345832739\\
86.95	0.00169809020899514\\
86.96	0.00169560534287889\\
86.97	0.00169310873087927\\
86.98	0.00169060024233235\\
86.99	0.00168807974499035\\
87	0.00168554710500204\\
87.01	0.00168300218689294\\
87.02	0.00168044485354525\\
87.03	0.00167787496617754\\
87.04	0.00167529238432416\\
87.05	0.00167269696581444\\
87.06	0.00167008856675156\\
87.07	0.00166746704149123\\
87.08	0.00166483224262001\\
87.09	0.00166218402093346\\
87.1	0.00165952222541394\\
87.11	0.00165684670320814\\
87.12	0.00165415729960437\\
87.13	0.0016514538580095\\
87.14	0.00164873621992566\\
87.15	0.00164600422492662\\
87.16	0.00164325771063388\\
87.17	0.00164049651269244\\
87.18	0.00163772046474633\\
87.19	0.0016349293984137\\
87.2	0.00163212314326174\\
87.21	0.00162930152678117\\
87.22	0.0016264643743605\\
87.23	0.00162361150925986\\
87.24	0.0016207427525846\\
87.25	0.0016178579232585\\
87.26	0.0016149568379966\\
87.27	0.00161203931127783\\
87.28	0.00160910515531709\\
87.29	0.00160615418003715\\
87.3	0.00160318619304011\\
87.31	0.0016002009995785\\
87.32	0.00159719840252601\\
87.33	0.0015941782023479\\
87.34	0.00159114019707094\\
87.35	0.00158808418225306\\
87.36	0.00158500995095255\\
87.37	0.00158191729369691\\
87.38	0.00157880599845126\\
87.39	0.00157567585058635\\
87.4	0.00157252663284622\\
87.41	0.0015693581253154\\
87.42	0.00156617010538566\\
87.43	0.00156296234772238\\
87.44	0.00155973462423051\\
87.45	0.00155648670402002\\
87.46	0.00155321835337099\\
87.47	0.00154992933569815\\
87.48	0.00154661941151511\\
87.49	0.00154328833839795\\
87.5	0.00153993587094853\\
87.51	0.00153656176075717\\
87.52	0.00153316575636493\\
87.53	0.00152974760322542\\
87.54	0.00152630704366604\\
87.55	0.00152284381684883\\
87.56	0.00151935765873068\\
87.57	0.00151584830202315\\
87.58	0.00151231547615171\\
87.59	0.00150875890721445\\
87.6	0.00150517831794029\\
87.61	0.00150157342764662\\
87.62	0.00149794395219637\\
87.63	0.0014942896039546\\
87.64	0.00149061009174448\\
87.65	0.00148690512080265\\
87.66	0.0014831743927341\\
87.67	0.00147941760546638\\
87.68	0.00147563445320327\\
87.69	0.00147182462637782\\
87.7	0.00146798781160478\\
87.71	0.00146412369163241\\
87.72	0.00146023194529367\\
87.73	0.00145631224745677\\
87.74	0.00145239085706545\\
87.75	0.00144846855178409\\
87.76	0.00144454533384675\\
87.77	0.00144062120552747\\
87.78	0.00143669616914078\\
87.79	0.00143277022704227\\
87.8	0.00142884338162906\\
87.81	0.0014249156353404\\
87.82	0.00142098699065818\\
87.83	0.0014170574501075\\
87.84	0.00141312701625725\\
87.85	0.00140919569172062\\
87.86	0.00140526347915577\\
87.87	0.00140133038126631\\
87.88	0.00139739640080197\\
87.89	0.00139346154055918\\
87.9	0.00138952580338165\\
87.91	0.00138558919216101\\
87.92	0.00138165170983742\\
87.93	0.00137771335940021\\
87.94	0.00137377414388853\\
87.95	0.00136983406639194\\
87.96	0.00136589313005115\\
87.97	0.00136195133805861\\
87.98	0.00135800869365922\\
87.99	0.001354065200151\\
88	0.00135012086088577\\
88.01	0.00134617567926986\\
88.02	0.00134222965876482\\
88.03	0.00133828280288813\\
88.04	0.0013343351152139\\
88.05	0.00133038659937365\\
88.06	0.00132643725905702\\
88.07	0.00132248709801254\\
88.08	0.00131853612004838\\
88.09	0.00131458432903312\\
88.1	0.00131063172889655\\
88.11	0.00130667832363045\\
88.12	0.0013027241172894\\
88.13	0.00129876911399156\\
88.14	0.00129481331791955\\
88.15	0.00129085673332124\\
88.16	0.00128689936451059\\
88.17	0.00128294121586854\\
88.18	0.00127898229184386\\
88.19	0.001275022596954\\
88.2	0.00127106213578603\\
88.21	0.00126710091299747\\
88.22	0.0012631389333173\\
88.23	0.00125917620154675\\
88.24	0.00125521272256038\\
88.25	0.0012512485013069\\
88.26	0.00124728354281021\\
88.27	0.00124331785217033\\
88.28	0.0012393514345644\\
88.29	0.00123538429524767\\
88.3	0.00123141643955451\\
88.31	0.00122744787289944\\
88.32	0.00122347860077813\\
88.33	0.00121950862876849\\
88.34	0.00121553796253172\\
88.35	0.00121156660781336\\
88.36	0.0012075945704444\\
88.37	0.00120362185634239\\
88.38	0.00119964847151252\\
88.39	0.00119567442204879\\
88.4	0.00119169971413515\\
88.41	0.00118772435404663\\
88.42	0.00118374834815054\\
88.43	0.00117977170290766\\
88.44	0.00117579442487342\\
88.45	0.00117181652069915\\
88.46	0.00116783799713331\\
88.47	0.00116385886102273\\
88.48	0.00115987911931388\\
88.49	0.00115589877905415\\
88.5	0.00115191784739318\\
88.51	0.00114793633158413\\
88.52	0.00114395423898506\\
88.53	0.00113997157706025\\
88.54	0.00113598835338156\\
88.55	0.00113200457562987\\
88.56	0.00112802025159641\\
88.57	0.00112403538918426\\
88.58	0.00112004999640971\\
88.59	0.00111606408140377\\
88.6	0.00111207765241364\\
88.61	0.0011080907178042\\
88.62	0.00110410328605953\\
88.63	0.00110011536578441\\
88.64	0.00109612696570594\\
88.65	0.00109213809467506\\
88.66	0.00108814876166816\\
88.67	0.0010841589757887\\
88.68	0.00108016874626886\\
88.69	0.00107617808247115\\
88.7	0.00107218699389012\\
88.71	0.00106819549015409\\
88.72	0.0010642035810268\\
88.73	0.0010602112764092\\
88.74	0.00105621858634121\\
88.75	0.00105222552100351\\
88.76	0.00104823209071932\\
88.77	0.00104423830595629\\
88.78	0.00104024417732831\\
88.79	0.0010362497155974\\
88.8	0.00103225493167565\\
88.81	0.0010282598366271\\
88.82	0.00102426444166974\\
88.83	0.00102026875817745\\
88.84	0.00101627279768205\\
88.85	0.00101227657187529\\
88.86	0.00100828009261092\\
88.87	0.0010042833719068\\
88.88	0.00100028642194697\\
88.89	0.000996289255083821\\
88.9	0.000992291883840236\\
88.91	0.000988294320911801\\
88.92	0.000984296579169014\\
88.93	0.000980298671659545\\
88.94	0.000976300611610505\\
88.95	0.000972302412430761\\
88.96	0.00096830408771327\\
88.97	0.00096430565123745\\
88.98	0.000960307116971572\\
88.99	0.000956308499075196\\
89	0.000952309811901628\\
89.01	0.00094831107000041\\
89.02	0.000944312288119848\\
89.03	0.000940313481209567\\
89.04	0.000936314664423094\\
89.05	0.00093231585312049\\
89.06	0.000928317062871001\\
89.07	0.00092431830945575\\
89.08	0.000920319608870457\\
89.09	0.000916320977328208\\
89.1	0.000912322431262241\\
89.11	0.000908323987328787\\
89.12	0.000904325662409928\\
89.13	0.000900327473616508\\
89.14	0.000896329438291075\\
89.15	0.000892331574010856\\
89.16	0.000888333898590781\\
89.17	0.00088433643008654\\
89.18	0.000880339186797674\\
89.19	0.000876342187270717\\
89.2	0.000872345450302367\\
89.21	0.00086834899494271\\
89.22	0.000864352840498475\\
89.23	0.000860357006536337\\
89.24	0.000856361512886258\\
89.25	0.000852366379644875\\
89.26	0.000848371627178931\\
89.27	0.000844377276128745\\
89.28	0.000840383347411737\\
89.29	0.000836389862225988\\
89.3	0.000832396842053855\\
89.31	0.000828404308665615\\
89.32	0.000824412284123189\\
89.33	0.000820420790783873\\
89.34	0.000816429851304154\\
89.35	0.000812439488643547\\
89.36	0.000808449726068503\\
89.37	0.000804460587156346\\
89.38	0.000800472095799283\\
89.39	0.00079648427620845\\
89.4	0.000792497152918018\\
89.41	0.000788510750789343\\
89.42	0.000784525095015183\\
89.43	0.000780540211123951\\
89.44	0.000776556124984052\\
89.45	0.000772572862808235\\
89.46	0.000768590451158043\\
89.47	0.000764608916948286\\
89.48	0.000760628287451597\\
89.49	0.000756648590303031\\
89.5	0.000752669853504728\\
89.51	0.000748692105430644\\
89.52	0.000744715374831324\\
89.53	0.000740739690838758\\
89.54	0.000736765082971285\\
89.55	0.000732791581138566\\
89.56	0.000728819215646619\\
89.57	0.000724848017202922\\
89.58	0.000720878016921582\\
89.59	0.000716909246328558\\
89.6	0.000712941737366978\\
89.61	0.000708975522402492\\
89.62	0.000705010634228724\\
89.63	0.000701047106072768\\
89.64	0.000697084971600781\\
89.65	0.000693124264923626\\
89.66	0.000689165020602593\\
89.67	0.000685207273655208\\
89.68	0.000681251059561102\\
89.69	0.000677296414267951\\
89.7	0.00067334337419752\\
89.71	0.000669391976251744\\
89.72	0.000665442257818933\\
89.73	0.000661494256780013\\
89.74	0.00065754801151488\\
89.75	0.000653603560908827\\
89.76	0.00064966094435904\\
89.77	0.000645720201781202\\
89.78	0.000641781373616155\\
89.79	0.000637844500836674\\
89.8	0.000633909624954318\\
89.81	0.000629976788026353\\
89.82	0.000626046032662798\\
89.83	0.000622117402033523\\
89.84	0.000618190939875482\\
89.85	0.0006142666905\\
89.86	0.000610344698800168\\
89.87	0.00060642501025835\\
89.88	0.000602507670953762\\
89.89	0.000598592727570157\\
89.9	0.000594680227403622\\
89.91	0.000590770218370455\\
89.92	0.000586862749015154\\
89.93	0.000582957868518521\\
89.94	0.000579055626705838\\
89.95	0.000575156074055189\\
89.96	0.000571259261705852\\
89.97	0.000567365241466829\\
89.98	0.000563474065825466\\
89.99	0.000559585787956193\\
90	0.00055570046172938\\
90.01	0.000551818141720294\\
90.02	0.000547938883218189\\
90.03	0.000544062742235498\\
90.04	0.000540189775517152\\
90.05	0.000536320040550018\\
90.06	0.00053245359557246\\
90.07	0.000528590499584014\\
90.08	0.000524730812355203\\
90.09	0.000520874594437469\\
90.1	0.00051702190717323\\
90.11	0.00051317281270608\\
90.12	0.000509327373991101\\
90.13	0.00050548565480533\\
90.14	0.000501647719758344\\
90.15	0.000497813634302988\\
90.16	0.000493983464746247\\
90.17	0.000490157278260239\\
90.18	0.00048633514289338\\
90.19	0.000482517127581662\\
90.2	0.000478703302160098\\
90.21	0.000474893737374305\\
90.22	0.000471088504892235\\
90.23	0.00046728767731607\\
90.24	0.000463491328194246\\
90.25	0.000459699532033663\\
90.26	0.000455912364312024\\
90.27	0.000452129901490352\\
90.28	0.00044835222102566\\
90.29	0.000444579401383784\\
90.3	0.00044081152205239\\
90.31	0.000437048663554131\\
90.32	0.000433290907459994\\
90.33	0.000429538336402807\\
90.34	0.000425791034090918\\
90.35	0.000422049085322064\\
90.36	0.000418312575997401\\
90.37	0.000414581593135727\\
90.38	0.000410856224887882\\
90.39	0.000407136560551339\\
90.4	0.000403422690584977\\
90.41	0.00039971470662405\\
90.42	0.000396012701495339\\
90.43	0.000392316769232515\\
90.44	0.000388627005091683\\
90.45	0.000384943505567141\\
90.46	0.000381266368407322\\
90.47	0.000377595692630966\\
90.48	0.000373931578543485\\
90.49	0.000370274127753535\\
90.5	0.000366623443189811\\
90.51	0.000362979629118057\\
90.52	0.000359342791158292\\
90.53	0.000355713036302248\\
90.54	0.000352090472931058\\
90.55	0.000348475210833143\\
90.56	0.000344867361222342\\
90.57	0.000341267036756283\\
90.58	0.000337674351554971\\
90.59	0.00033408942121963\\
90.6	0.000330512362851783\\
90.61	0.000326943295072579\\
90.62	0.000323382338042352\\
90.63	0.000319829613480463\\
90.64	0.000316285244685365\\
90.65	0.000312749356554946\\
90.66	0.000309222075607123\\
90.67	0.000305703530000702\\
90.68	0.000302193849556511\\
90.69	0.000298693165778801\\
90.7	0.000295201611876914\\
90.71	0.000291719322787237\\
90.72	0.000288246435195443\\
90.73	0.000284783087559004\\
90.74	0.000281329420129994\\
90.75	0.000277885574978201\\
90.76	0.000274451696014512\\
90.77	0.00027102792901461\\
90.78	0.00026761442164298\\
90.79	0.000264211323477205\\
90.8	0.000260818786032587\\
90.81	0.000257436962787082\\
90.82	0.000254066009206546\\
90.83	0.000250706082770311\\
90.84	0.000247357342997084\\
90.85	0.000244019951471185\\
90.86	0.000240694071869104\\
90.87	0.000237379869986419\\
90.88	0.000234077513765035\\
90.89	0.000230787173320791\\
90.9	0.000227509020971408\\
90.91	0.000224243231264791\\
90.92	0.000220989981007707\\
90.93	0.000217749449294811\\
90.94	0.000214521817538053\\
90.95	0.000211307269496464\\
90.96	0.000208105991306298\\
90.97	0.000204918171511592\\
90.98	0.00020174400109509\\
90.99	0.000198583673509565\\
91	0.000195437384709556\\
91.01	0.000192305333183479\\
91.02	0.00018918771998618\\
91.03	0.000186084748771872\\
91.04	0.000182996625827513\\
91.05	0.000179923560106592\\
91.06	0.000176865763263355\\
91.07	0.000173823449687466\\
91.08	0.000170796836539092\\
91.09	0.000167786143784454\\
91.1	0.000164791594231817\\
91.11	0.000161813413567935\\
91.12	0.000158851830394964\\
91.13	0.000155907076267841\\
91.14	0.000152979385732135\\
91.15	0.000150068996362369\\
91.16	0.000147176148800847\\
91.17	0.000144301086796949\\
91.18	0.000141444057246939\\
91.19	0.000138605310234269\\
91.2	0.000135785099070388\\
91.21	0.000132983680336077\\
91.22	0.000130201313923295\\
91.23	0.000127438263077555\\
91.24	0.000124694794440843\\
91.25	0.00012197117809506\\
91.26	0.000119267687606027\\
91.27	0.000116584600068036\\
91.28	0.000113922196148962\\
91.29	0.000111280760135946\\
91.3	0.000108660579981636\\
91.31	0.000106061947351037\\
91.32	0.000103485157668925\\
91.33	0.000100930510167857\\
91.34	9.83983079367989e-05\\
91.35	9.58888579703446e-05\\
91.36	9.34024712185618e-05\\
91.37	9.09394626374578e-05\\
91.38	8.85001512400726e-05\\
91.39	8.60848601482202e-05\\
91.4	8.36939166448617e-05\\
91.41	8.13276522271447e-05\\
91.42	7.89864026600993e-05\\
91.43	7.66705080310001e-05\\
91.44	7.43803128044045e-05\\
91.45	7.21161658778832e-05\\
91.46	6.98784206384397e-05\\
91.47	6.76674350196249e-05\\
91.48	6.54835715593691e-05\\
91.49	6.33271974585237e-05\\
91.5	6.11986846401399e-05\\
91.51	5.90984098094664e-05\\
91.52	5.70267545147024e-05\\
91.53	5.49841052084939e-05\\
91.54	5.29708533101945e-05\\
91.55	5.09873952688894e-05\\
91.56	4.9034132627206e-05\\
91.57	4.71114720859006e-05\\
91.58	4.52198255692458e-05\\
91.59	4.33596102912184e-05\\
91.6	4.15312488225121e-05\\
91.61	3.97351691583513e-05\\
91.62	3.79718047871635e-05\\
91.63	3.62415947600721e-05\\
91.64	3.4544983761247e-05\\
91.65	3.28824221791212e-05\\
91.66	3.12543661784726e-05\\
91.67	2.96612777733827e-05\\
91.68	2.81036249010908e-05\\
91.69	2.65818814967449e-05\\
91.7	2.50965275690653e-05\\
91.71	2.36480492769257e-05\\
91.72	2.22369390068602e-05\\
91.73	2.08636954515239e-05\\
91.74	1.952882368908e-05\\
91.75	1.8232835263584e-05\\
91.76	1.69762482662948e-05\\
91.77	1.57595874180124e-05\\
91.78	1.45833841523734e-05\\
91.79	1.34481767001694e-05\\
91.8	1.2354510174676e-05\\
91.81	1.13029366580007e-05\\
91.82	1.02940152884728e-05\\
91.83	9.32831234908699e-06\\
91.84	8.40640135698484e-06\\
91.85	7.52886315403233e-06\\
91.86	6.6962859984563e-06\\
91.87	5.90926565758633e-06\\
91.88	5.16840550169803e-06\\
91.89	4.47431659896837e-06\\
91.9	3.82761781155851e-06\\
91.91	3.22893589284334e-06\\
91.92	2.67890558577899e-06\\
91.93	2.17816972244295e-06\\
91.94	1.7273793247452e-06\\
91.95	1.32719370632764e-06\\
91.96	9.78280575653426e-07\\
91.97	6.81316140322893e-07\\
91.98	4.3698521259676e-07\\
91.99	2.45981316161423e-07\\
92	1.0900679414845e-07\\
92.01	2.67729183944104e-08\\
92.02	0\\
92.03	0\\
92.04	0\\
92.05	0\\
92.06	0\\
92.07	0\\
92.08	0\\
92.09	0\\
92.1	0\\
92.11	0\\
92.12	0\\
92.13	0\\
92.14	0\\
92.15	0\\
92.16	0\\
92.17	0\\
92.18	0\\
92.19	0\\
92.2	0\\
92.21	0\\
92.22	0\\
92.23	0\\
92.24	0\\
92.25	0\\
92.26	0\\
92.27	0\\
92.28	0\\
92.29	0\\
92.3	0\\
92.31	0\\
92.32	0\\
92.33	0\\
92.34	0\\
92.35	0\\
92.36	0\\
92.37	0\\
92.38	0\\
92.39	0\\
92.4	0\\
92.41	0\\
92.42	0\\
92.43	0\\
92.44	0\\
92.45	0\\
92.46	0\\
92.47	0\\
92.48	0\\
92.49	0\\
92.5	0\\
92.51	0\\
92.52	0\\
92.53	0\\
92.54	0\\
92.55	0\\
92.56	0\\
92.57	0\\
92.58	0\\
92.59	0\\
92.6	0\\
92.61	0\\
92.62	0\\
92.63	0\\
92.64	0\\
92.65	0\\
92.66	0\\
92.67	0\\
92.68	0\\
92.69	0\\
92.7	0\\
92.71	0\\
92.72	0\\
92.73	0\\
92.74	0\\
92.75	0\\
92.76	0\\
92.77	0\\
92.78	0\\
92.79	0\\
92.8	0\\
92.81	0\\
92.82	0\\
92.83	0\\
92.84	0\\
92.85	0\\
92.86	0\\
92.87	0\\
92.88	0\\
92.89	0\\
92.9	0\\
92.91	0\\
92.92	0\\
92.93	0\\
92.94	0\\
92.95	0\\
92.96	0\\
92.97	0\\
92.98	0\\
92.99	0\\
93	0\\
93.01	0\\
93.02	0\\
93.03	0\\
93.04	0\\
93.05	0\\
93.06	0\\
93.07	0\\
93.08	0\\
93.09	0\\
93.1	0\\
93.11	0\\
93.12	0\\
93.13	0\\
93.14	0\\
93.15	0\\
93.16	0\\
93.17	0\\
93.18	0\\
93.19	0\\
93.2	0\\
93.21	0\\
93.22	0\\
93.23	0\\
93.24	0\\
93.25	0\\
93.26	0\\
93.27	0\\
93.28	0\\
93.29	0\\
93.3	0\\
93.31	0\\
93.32	0\\
93.33	0\\
93.34	0\\
93.35	0\\
93.36	0\\
93.37	0\\
93.38	0\\
93.39	0\\
93.4	0\\
93.41	0\\
93.42	0\\
93.43	0\\
93.44	0\\
93.45	0\\
93.46	0\\
93.47	0\\
93.48	0\\
93.49	0\\
93.5	0\\
93.51	0\\
93.52	0\\
93.53	0\\
93.54	0\\
93.55	0\\
93.56	0\\
93.57	0\\
93.58	0\\
93.59	0\\
93.6	0\\
93.61	0\\
93.62	0\\
93.63	0\\
93.64	0\\
93.65	0\\
93.66	0\\
93.67	0\\
93.68	0\\
93.69	0\\
93.7	0\\
93.71	0\\
93.72	0\\
93.73	0\\
93.74	0\\
93.75	0\\
93.76	0\\
93.77	0\\
93.78	0\\
93.79	0\\
93.8	0\\
93.81	0\\
93.82	0\\
93.83	0\\
93.84	0\\
93.85	0\\
93.86	0\\
93.87	0\\
93.88	0\\
93.89	0\\
93.9	0\\
93.91	0\\
93.92	0\\
93.93	0\\
93.94	0\\
93.95	0\\
93.96	0\\
93.97	0\\
93.98	0\\
93.99	0\\
94	0\\
94.01	0\\
94.02	0\\
94.03	0\\
94.04	0\\
94.05	0\\
94.06	0\\
94.07	0\\
94.08	0\\
94.09	0\\
94.1	0\\
94.11	0\\
94.12	0\\
94.13	0\\
94.14	0\\
94.15	0\\
94.16	0\\
94.17	0\\
94.18	0\\
94.19	0\\
94.2	0\\
94.21	0\\
94.22	0\\
94.23	0\\
94.24	0\\
94.25	0\\
94.26	0\\
94.27	0\\
94.28	0\\
94.29	0\\
94.3	0\\
94.31	0\\
94.32	0\\
94.33	0\\
94.34	0\\
94.35	0\\
94.36	0\\
94.37	0\\
94.38	0\\
94.39	0\\
94.4	0\\
94.41	0\\
94.42	0\\
94.43	0\\
94.44	0\\
94.45	0\\
94.46	0\\
94.47	0\\
94.48	0\\
94.49	0\\
94.5	0\\
94.51	0\\
94.52	0\\
94.53	0\\
94.54	0\\
94.55	0\\
94.56	0\\
94.57	0\\
94.58	0\\
94.59	0\\
94.6	0\\
94.61	0\\
94.62	0\\
94.63	0\\
94.64	0\\
94.65	0\\
94.66	0\\
94.67	0\\
94.68	0\\
94.69	0\\
94.7	0\\
94.71	0\\
94.72	0\\
94.73	0\\
94.74	0\\
94.75	0\\
94.76	0\\
94.77	0\\
94.78	0\\
94.79	0\\
94.8	0\\
94.81	0\\
94.82	0\\
94.83	0\\
94.84	0\\
94.85	0\\
94.86	0\\
94.87	0\\
94.88	0\\
94.89	0\\
94.9	0\\
94.91	0\\
94.92	0\\
94.93	0\\
94.94	0\\
94.95	0\\
94.96	0\\
94.97	0\\
94.98	0\\
94.99	0\\
95	0\\
95.01	0\\
95.02	0\\
95.03	0\\
95.04	0\\
95.05	0\\
95.06	0\\
95.07	0\\
95.08	0\\
95.09	0\\
95.1	0\\
95.11	0\\
95.12	0\\
95.13	0\\
95.14	0\\
95.15	0\\
95.16	0\\
95.17	0\\
95.18	0\\
95.19	0\\
95.2	0\\
95.21	0\\
95.22	0\\
95.23	0\\
95.24	0\\
95.25	0\\
95.26	0\\
95.27	0\\
95.28	0\\
95.29	0\\
95.3	0\\
95.31	0\\
95.32	0\\
95.33	0\\
95.34	0\\
95.35	0\\
95.36	0\\
95.37	0\\
95.38	0\\
95.39	0\\
95.4	0\\
95.41	0\\
95.42	0\\
95.43	0\\
95.44	0\\
95.45	0\\
95.46	0\\
95.47	0\\
95.48	0\\
95.49	0\\
95.5	0\\
95.51	0\\
95.52	0\\
95.53	0\\
95.54	0\\
95.55	0\\
95.56	0\\
95.57	0\\
95.58	0\\
95.59	0\\
95.6	0\\
95.61	0\\
95.62	0\\
95.63	0\\
95.64	0\\
95.65	0\\
95.66	0\\
95.67	0\\
95.68	0\\
95.69	0\\
95.7	0\\
95.71	0\\
95.72	0\\
95.73	0\\
95.74	0\\
95.75	0\\
95.76	0\\
95.77	0\\
95.78	0\\
95.79	0\\
95.8	0\\
95.81	0\\
95.82	0\\
95.83	0\\
95.84	0\\
95.85	0\\
95.86	0\\
95.87	0\\
95.88	0\\
95.89	0\\
95.9	0\\
95.91	0\\
95.92	0\\
95.93	0\\
95.94	0\\
95.95	0\\
95.96	0\\
95.97	0\\
95.98	0\\
95.99	0\\
96	0\\
96.01	0\\
96.02	0\\
96.03	0\\
96.04	0\\
96.05	0\\
96.06	0\\
96.07	0\\
96.08	0\\
96.09	0\\
96.1	0\\
96.11	0\\
96.12	0\\
96.13	0\\
96.14	0\\
96.15	0\\
96.16	0\\
96.17	0\\
96.18	0\\
96.19	0\\
96.2	0\\
96.21	0\\
96.22	0\\
96.23	0\\
96.24	0\\
96.25	0\\
96.26	0\\
96.27	0\\
96.28	0\\
96.29	0\\
96.3	0\\
96.31	0\\
96.32	0\\
96.33	0\\
96.34	0\\
96.35	0\\
96.36	0\\
96.37	0\\
96.38	0\\
96.39	0\\
96.4	0\\
96.41	0\\
96.42	0\\
96.43	0\\
96.44	0\\
96.45	0\\
96.46	0\\
96.47	0\\
96.48	0\\
96.49	0\\
96.5	0\\
96.51	0\\
96.52	0\\
96.53	0\\
96.54	0\\
96.55	0\\
96.56	0\\
96.57	0\\
96.58	0\\
96.59	0\\
96.6	0\\
96.61	0\\
96.62	0\\
96.63	0\\
96.64	0\\
96.65	0\\
96.66	0\\
96.67	0\\
96.68	0\\
96.69	0\\
96.7	0\\
96.71	0\\
96.72	0\\
96.73	0\\
96.74	0\\
96.75	0\\
96.76	0\\
96.77	0\\
96.78	0\\
96.79	0\\
96.8	0\\
96.81	0\\
96.82	0\\
96.83	0\\
96.84	0\\
96.85	0\\
96.86	0\\
96.87	0\\
96.88	0\\
96.89	0\\
96.9	0\\
96.91	0\\
96.92	0\\
96.93	0\\
96.94	0\\
96.95	0\\
96.96	0\\
96.97	0\\
96.98	0\\
96.99	0\\
97	0\\
97.01	0\\
97.02	0\\
97.03	0\\
97.04	0\\
97.05	0\\
97.06	0\\
97.07	0\\
97.08	0\\
97.09	0\\
97.1	0\\
97.11	0\\
97.12	0\\
97.13	0\\
97.14	0\\
97.15	0\\
97.16	0\\
97.17	0\\
97.18	0\\
97.19	0\\
97.2	0\\
97.21	0\\
97.22	0\\
97.23	0\\
97.24	0\\
97.25	0\\
97.26	0\\
97.27	0\\
97.28	0\\
97.29	0\\
97.3	0\\
97.31	0\\
97.32	0\\
97.33	0\\
97.34	0\\
97.35	0\\
97.36	0\\
97.37	0\\
97.38	0\\
97.39	0\\
97.4	0\\
97.41	0\\
97.42	0\\
97.43	0\\
97.44	0\\
97.45	0\\
97.46	0\\
97.47	0\\
97.48	0\\
97.49	0\\
97.5	0\\
97.51	0\\
97.52	0\\
97.53	0\\
97.54	0\\
97.55	0\\
97.56	0\\
97.57	0\\
97.58	0\\
97.59	0\\
97.6	0\\
97.61	0\\
97.62	0\\
97.63	0\\
97.64	0\\
97.65	0\\
97.66	0\\
97.67	0\\
97.68	0\\
97.69	0\\
97.7	0\\
97.71	0\\
97.72	0\\
97.73	0\\
97.74	0\\
97.75	0\\
97.76	0\\
97.77	0\\
97.78	0\\
97.79	0\\
97.8	0\\
97.81	0\\
97.82	0\\
97.83	0\\
97.84	0\\
97.85	0\\
97.86	0\\
97.87	0\\
97.88	0\\
97.89	0\\
97.9	0\\
97.91	0\\
97.92	0\\
97.93	0\\
97.94	0\\
97.95	0\\
97.96	0\\
97.97	0\\
97.98	0\\
97.99	0\\
98	0\\
98.01	0\\
98.02	0\\
98.03	0\\
98.04	0\\
98.05	0\\
98.06	0\\
98.07	0\\
98.08	0\\
98.09	0\\
98.1	0\\
98.11	0\\
98.12	0\\
98.13	0\\
98.14	0\\
98.15	0\\
98.16	0\\
98.17	0\\
98.18	0\\
98.19	0\\
98.2	0\\
98.21	0\\
98.22	0\\
98.23	0\\
98.24	0\\
98.25	0\\
98.26	0\\
98.27	0\\
98.28	0\\
98.29	0\\
98.3	0\\
98.31	0\\
98.32	0\\
98.33	0\\
98.34	0\\
98.35	0\\
98.36	0\\
98.37	0\\
98.38	0\\
98.39	0\\
98.4	0\\
98.41	0\\
98.42	0\\
98.43	0\\
98.44	0\\
98.45	0\\
98.46	0\\
98.47	0\\
98.48	0\\
98.49	0\\
98.5	0\\
98.51	0\\
98.52	0\\
98.53	0\\
98.54	0\\
98.55	0\\
98.56	0\\
98.57	0\\
98.58	0\\
98.59	0\\
98.6	0\\
98.61	0\\
98.62	0\\
98.63	0\\
98.64	0\\
98.65	0\\
98.66	0\\
98.67	0\\
98.68	0\\
98.69	0\\
98.7	0\\
98.71	0\\
98.72	0\\
98.73	0\\
98.74	0\\
98.75	0\\
98.76	0\\
98.77	0\\
98.78	0\\
98.79	0\\
98.8	0\\
98.81	0\\
98.82	0\\
98.83	0\\
98.84	0\\
98.85	0\\
98.86	0\\
98.87	0\\
98.88	0\\
98.89	0\\
98.9	0\\
98.91	0\\
98.92	0\\
98.93	0\\
98.94	0\\
98.95	0\\
98.96	0\\
98.97	0\\
98.98	0\\
98.99	0\\
99	0\\
99.01	0\\
99.02	0\\
99.03	0\\
99.04	0\\
99.05	0\\
99.06	0\\
99.07	0\\
99.08	0\\
99.09	0\\
99.1	0\\
99.11	0\\
99.12	0\\
99.13	0\\
99.14	0\\
99.15	0\\
99.16	0\\
99.17	0\\
99.18	0\\
99.19	0\\
99.2	0\\
99.21	0\\
99.22	0\\
99.23	0\\
99.24	0\\
99.25	0\\
99.26	0\\
99.27	0\\
99.28	0\\
99.29	0\\
99.3	0\\
99.31	0\\
99.32	0\\
99.33	0\\
99.34	0\\
99.35	0\\
99.36	0\\
99.37	0\\
99.38	0\\
99.39	0\\
99.4	0\\
99.41	0\\
99.42	0\\
99.43	0\\
99.44	0\\
99.45	0\\
99.46	0\\
99.47	0\\
99.48	0\\
99.49	0\\
99.5	0\\
99.51	0\\
99.52	0\\
99.53	0\\
99.54	0\\
99.55	0\\
99.56	0\\
99.57	0\\
99.58	0\\
99.59	0\\
99.6	0\\
99.61	0\\
99.62	0\\
99.63	0\\
99.64	0\\
99.65	0\\
99.66	0\\
99.67	0\\
99.68	0\\
99.69	0\\
99.7	0\\
99.71	0\\
99.72	0\\
99.73	0\\
99.74	0\\
99.75	0\\
99.76	0\\
99.77	0\\
99.78	0\\
99.79	0\\
99.8	0\\
99.81	0\\
99.82	0\\
99.83	0\\
99.84	0\\
99.85	0\\
99.86	0\\
99.87	0\\
99.88	0\\
99.89	0\\
99.9	0\\
99.91	0\\
99.92	0\\
99.93	0\\
99.94	0\\
99.95	0\\
99.96	0\\
99.97	0\\
99.98	0\\
99.99	0\\
100	0\\
};
\addlegendentry{$q=-1$};

\addplot [color=black,solid,forget plot]
  table[row sep=crcr]{%
0.01	0\\
0.02	0\\
0.03	0\\
0.04	0\\
0.05	0\\
0.06	0\\
0.07	0\\
0.08	0\\
0.09	0\\
0.1	0\\
0.11	0\\
0.12	0\\
0.13	0\\
0.14	0\\
0.15	0\\
0.16	0\\
0.17	0\\
0.18	0\\
0.19	0\\
0.2	0\\
0.21	0\\
0.22	0\\
0.23	0\\
0.24	0\\
0.25	0\\
0.26	0\\
0.27	0\\
0.28	0\\
0.29	0\\
0.3	0\\
0.31	0\\
0.32	0\\
0.33	0\\
0.34	0\\
0.35	0\\
0.36	0\\
0.37	0\\
0.38	0\\
0.39	0\\
0.4	0\\
0.41	0\\
0.42	0\\
0.43	0\\
0.44	0\\
0.45	0\\
0.46	0\\
0.47	0\\
0.48	0\\
0.49	0\\
0.5	0\\
0.51	0\\
0.52	0\\
0.53	0\\
0.54	0\\
0.55	0\\
0.56	0\\
0.57	0\\
0.58	0\\
0.59	0\\
0.6	0\\
0.61	0\\
0.62	0\\
0.63	0\\
0.64	0\\
0.65	0\\
0.66	0\\
0.67	0\\
0.68	0\\
0.69	0\\
0.7	0\\
0.71	0\\
0.72	0\\
0.73	0\\
0.74	0\\
0.75	0\\
0.76	0\\
0.77	0\\
0.78	0\\
0.79	0\\
0.8	0\\
0.81	0\\
0.82	0\\
0.83	0\\
0.84	0\\
0.85	0\\
0.86	0\\
0.87	0\\
0.88	0\\
0.89	0\\
0.9	0\\
0.91	0\\
0.92	0\\
0.93	0\\
0.94	0\\
0.95	0\\
0.96	0\\
0.97	0\\
0.98	0\\
0.99	0\\
1	0\\
1.01	0\\
1.02	0\\
1.03	0\\
1.04	0\\
1.05	0\\
1.06	0\\
1.07	0\\
1.08	0\\
1.09	0\\
1.1	0\\
1.11	0\\
1.12	0\\
1.13	0\\
1.14	0\\
1.15	0\\
1.16	0\\
1.17	0\\
1.18	0\\
1.19	0\\
1.2	0\\
1.21	0\\
1.22	0\\
1.23	0\\
1.24	0\\
1.25	0\\
1.26	0\\
1.27	0\\
1.28	0\\
1.29	0\\
1.3	0\\
1.31	0\\
1.32	0\\
1.33	0\\
1.34	0\\
1.35	0\\
1.36	0\\
1.37	0\\
1.38	0\\
1.39	0\\
1.4	0\\
1.41	0\\
1.42	0\\
1.43	0\\
1.44	0\\
1.45	0\\
1.46	0\\
1.47	0\\
1.48	0\\
1.49	0\\
1.5	0\\
1.51	0\\
1.52	0\\
1.53	0\\
1.54	0\\
1.55	0\\
1.56	0\\
1.57	0\\
1.58	0\\
1.59	0\\
1.6	0\\
1.61	0\\
1.62	0\\
1.63	0\\
1.64	0\\
1.65	0\\
1.66	0\\
1.67	0\\
1.68	0\\
1.69	0\\
1.7	0\\
1.71	0\\
1.72	0\\
1.73	0\\
1.74	0\\
1.75	0\\
1.76	0\\
1.77	0\\
1.78	0\\
1.79	0\\
1.8	0\\
1.81	0\\
1.82	0\\
1.83	0\\
1.84	0\\
1.85	0\\
1.86	0\\
1.87	0\\
1.88	0\\
1.89	0\\
1.9	0\\
1.91	0\\
1.92	0\\
1.93	0\\
1.94	0\\
1.95	0\\
1.96	0\\
1.97	0\\
1.98	0\\
1.99	0\\
2	0\\
2.01	0\\
2.02	0\\
2.03	0\\
2.04	0\\
2.05	0\\
2.06	0\\
2.07	0\\
2.08	0\\
2.09	0\\
2.1	0\\
2.11	0\\
2.12	0\\
2.13	0\\
2.14	0\\
2.15	0\\
2.16	0\\
2.17	0\\
2.18	0\\
2.19	0\\
2.2	0\\
2.21	0\\
2.22	0\\
2.23	0\\
2.24	0\\
2.25	0\\
2.26	0\\
2.27	0\\
2.28	0\\
2.29	0\\
2.3	0\\
2.31	0\\
2.32	0\\
2.33	0\\
2.34	0\\
2.35	0\\
2.36	0\\
2.37	0\\
2.38	0\\
2.39	0\\
2.4	0\\
2.41	0\\
2.42	0\\
2.43	0\\
2.44	0\\
2.45	0\\
2.46	0\\
2.47	0\\
2.48	0\\
2.49	0\\
2.5	0\\
2.51	0\\
2.52	0\\
2.53	0\\
2.54	0\\
2.55	0\\
2.56	0\\
2.57	0\\
2.58	0\\
2.59	0\\
2.6	0\\
2.61	0\\
2.62	0\\
2.63	0\\
2.64	0\\
2.65	0\\
2.66	0\\
2.67	0\\
2.68	0\\
2.69	0\\
2.7	0\\
2.71	0\\
2.72	0\\
2.73	0\\
2.74	0\\
2.75	0\\
2.76	0\\
2.77	0\\
2.78	0\\
2.79	0\\
2.8	0\\
2.81	0\\
2.82	0\\
2.83	0\\
2.84	0\\
2.85	0\\
2.86	0\\
2.87	0\\
2.88	0\\
2.89	0\\
2.9	0\\
2.91	0\\
2.92	0\\
2.93	0\\
2.94	0\\
2.95	0\\
2.96	0\\
2.97	0\\
2.98	0\\
2.99	0\\
3	0\\
3.01	0\\
3.02	0\\
3.03	0\\
3.04	0\\
3.05	0\\
3.06	0\\
3.07	0\\
3.08	0\\
3.09	0\\
3.1	0\\
3.11	0\\
3.12	0\\
3.13	0\\
3.14	0\\
3.15	0\\
3.16	0\\
3.17	0\\
3.18	0\\
3.19	0\\
3.2	0\\
3.21	0\\
3.22	0\\
3.23	0\\
3.24	0\\
3.25	0\\
3.26	0\\
3.27	0\\
3.28	0\\
3.29	0\\
3.3	0\\
3.31	0\\
3.32	0\\
3.33	0\\
3.34	0\\
3.35	0\\
3.36	0\\
3.37	0\\
3.38	0\\
3.39	0\\
3.4	0\\
3.41	0\\
3.42	0\\
3.43	0\\
3.44	0\\
3.45	0\\
3.46	0\\
3.47	0\\
3.48	0\\
3.49	0\\
3.5	0\\
3.51	0\\
3.52	0\\
3.53	0\\
3.54	0\\
3.55	0\\
3.56	0\\
3.57	0\\
3.58	0\\
3.59	0\\
3.6	0\\
3.61	0\\
3.62	0\\
3.63	0\\
3.64	0\\
3.65	0\\
3.66	0\\
3.67	0\\
3.68	0\\
3.69	0\\
3.7	0\\
3.71	0\\
3.72	0\\
3.73	0\\
3.74	0\\
3.75	0\\
3.76	0\\
3.77	0\\
3.78	0\\
3.79	0\\
3.8	0\\
3.81	0\\
3.82	0\\
3.83	0\\
3.84	0\\
3.85	0\\
3.86	0\\
3.87	0\\
3.88	0\\
3.89	0\\
3.9	0\\
3.91	0\\
3.92	0\\
3.93	0\\
3.94	0\\
3.95	0\\
3.96	0\\
3.97	0\\
3.98	0\\
3.99	0\\
4	0\\
4.01	0\\
4.02	0\\
4.03	0\\
4.04	0\\
4.05	0\\
4.06	0\\
4.07	0\\
4.08	0\\
4.09	0\\
4.1	0\\
4.11	0\\
4.12	0\\
4.13	0\\
4.14	0\\
4.15	0\\
4.16	0\\
4.17	0\\
4.18	0\\
4.19	0\\
4.2	0\\
4.21	0\\
4.22	0\\
4.23	0\\
4.24	0\\
4.25	0\\
4.26	0\\
4.27	0\\
4.28	0\\
4.29	0\\
4.3	0\\
4.31	0\\
4.32	0\\
4.33	0\\
4.34	0\\
4.35	0\\
4.36	0\\
4.37	0\\
4.38	0\\
4.39	0\\
4.4	0\\
4.41	0\\
4.42	0\\
4.43	0\\
4.44	0\\
4.45	0\\
4.46	0\\
4.47	0\\
4.48	0\\
4.49	0\\
4.5	0\\
4.51	0\\
4.52	0\\
4.53	0\\
4.54	0\\
4.55	0\\
4.56	0\\
4.57	0\\
4.58	0\\
4.59	0\\
4.6	0\\
4.61	0\\
4.62	0\\
4.63	0\\
4.64	0\\
4.65	0\\
4.66	0\\
4.67	0\\
4.68	0\\
4.69	0\\
4.7	0\\
4.71	0\\
4.72	0\\
4.73	0\\
4.74	0\\
4.75	0\\
4.76	0\\
4.77	0\\
4.78	0\\
4.79	0\\
4.8	0\\
4.81	0\\
4.82	0\\
4.83	0\\
4.84	0\\
4.85	0\\
4.86	0\\
4.87	0\\
4.88	0\\
4.89	0\\
4.9	0\\
4.91	0\\
4.92	0\\
4.93	0\\
4.94	0\\
4.95	0\\
4.96	0\\
4.97	0\\
4.98	0\\
4.99	0\\
5	0\\
5.01	0\\
5.02	0\\
5.03	0\\
5.04	0\\
5.05	0\\
5.06	0\\
5.07	0\\
5.08	0\\
5.09	0\\
5.1	0\\
5.11	0\\
5.12	0\\
5.13	0\\
5.14	0\\
5.15	0\\
5.16	0\\
5.17	0\\
5.18	0\\
5.19	0\\
5.2	0\\
5.21	0\\
5.22	0\\
5.23	0\\
5.24	0\\
5.25	0\\
5.26	0\\
5.27	0\\
5.28	0\\
5.29	0\\
5.3	0\\
5.31	0\\
5.32	0\\
5.33	0\\
5.34	0\\
5.35	0\\
5.36	0\\
5.37	0\\
5.38	0\\
5.39	0\\
5.4	0\\
5.41	0\\
5.42	0\\
5.43	0\\
5.44	0\\
5.45	0\\
5.46	0\\
5.47	0\\
5.48	0\\
5.49	0\\
5.5	0\\
5.51	0\\
5.52	0\\
5.53	0\\
5.54	0\\
5.55	0\\
5.56	0\\
5.57	0\\
5.58	0\\
5.59	0\\
5.6	0\\
5.61	0\\
5.62	0\\
5.63	0\\
5.64	0\\
5.65	0\\
5.66	0\\
5.67	0\\
5.68	0\\
5.69	0\\
5.7	0\\
5.71	0\\
5.72	0\\
5.73	0\\
5.74	0\\
5.75	0\\
5.76	0\\
5.77	0\\
5.78	0\\
5.79	0\\
5.8	0\\
5.81	0\\
5.82	0\\
5.83	0\\
5.84	0\\
5.85	0\\
5.86	0\\
5.87	0\\
5.88	0\\
5.89	0\\
5.9	0\\
5.91	0\\
5.92	0\\
5.93	0\\
5.94	0\\
5.95	0\\
5.96	0\\
5.97	0\\
5.98	0\\
5.99	0\\
6	0\\
6.01	0\\
6.02	0\\
6.03	0\\
6.04	0\\
6.05	0\\
6.06	0\\
6.07	0\\
6.08	0\\
6.09	0\\
6.1	0\\
6.11	0\\
6.12	0\\
6.13	0\\
6.14	0\\
6.15	0\\
6.16	0\\
6.17	0\\
6.18	0\\
6.19	0\\
6.2	0\\
6.21	0\\
6.22	0\\
6.23	0\\
6.24	0\\
6.25	0\\
6.26	0\\
6.27	0\\
6.28	0\\
6.29	0\\
6.3	0\\
6.31	0\\
6.32	0\\
6.33	0\\
6.34	0\\
6.35	0\\
6.36	0\\
6.37	0\\
6.38	0\\
6.39	0\\
6.4	0\\
6.41	0\\
6.42	0\\
6.43	0\\
6.44	0\\
6.45	0\\
6.46	0\\
6.47	0\\
6.48	0\\
6.49	0\\
6.5	0\\
6.51	0\\
6.52	0\\
6.53	0\\
6.54	0\\
6.55	0\\
6.56	0\\
6.57	0\\
6.58	0\\
6.59	0\\
6.6	0\\
6.61	0\\
6.62	0\\
6.63	0\\
6.64	0\\
6.65	0\\
6.66	0\\
6.67	0\\
6.68	0\\
6.69	0\\
6.7	0\\
6.71	0\\
6.72	0\\
6.73	0\\
6.74	0\\
6.75	0\\
6.76	0\\
6.77	0\\
6.78	0\\
6.79	0\\
6.8	0\\
6.81	0\\
6.82	0\\
6.83	0\\
6.84	0\\
6.85	0\\
6.86	0\\
6.87	0\\
6.88	0\\
6.89	0\\
6.9	0\\
6.91	0\\
6.92	0\\
6.93	0\\
6.94	0\\
6.95	0\\
6.96	0\\
6.97	0\\
6.98	0\\
6.99	0\\
7	0\\
7.01	0\\
7.02	0\\
7.03	0\\
7.04	0\\
7.05	0\\
7.06	0\\
7.07	0\\
7.08	0\\
7.09	0\\
7.1	0\\
7.11	0\\
7.12	0\\
7.13	0\\
7.14	0\\
7.15	0\\
7.16	0\\
7.17	0\\
7.18	0\\
7.19	0\\
7.2	0\\
7.21	0\\
7.22	0\\
7.23	0\\
7.24	0\\
7.25	0\\
7.26	0\\
7.27	0\\
7.28	0\\
7.29	0\\
7.3	0\\
7.31	0\\
7.32	0\\
7.33	0\\
7.34	0\\
7.35	0\\
7.36	0\\
7.37	0\\
7.38	0\\
7.39	0\\
7.4	0\\
7.41	0\\
7.42	0\\
7.43	0\\
7.44	0\\
7.45	0\\
7.46	0\\
7.47	0\\
7.48	0\\
7.49	0\\
7.5	0\\
7.51	0\\
7.52	0\\
7.53	0\\
7.54	0\\
7.55	0\\
7.56	0\\
7.57	0\\
7.58	0\\
7.59	0\\
7.6	0\\
7.61	0\\
7.62	0\\
7.63	0\\
7.64	0\\
7.65	0\\
7.66	0\\
7.67	0\\
7.68	0\\
7.69	0\\
7.7	0\\
7.71	0\\
7.72	0\\
7.73	0\\
7.74	0\\
7.75	0\\
7.76	0\\
7.77	0\\
7.78	0\\
7.79	0\\
7.8	0\\
7.81	0\\
7.82	0\\
7.83	0\\
7.84	0\\
7.85	0\\
7.86	0\\
7.87	0\\
7.88	0\\
7.89	0\\
7.9	0\\
7.91	0\\
7.92	0\\
7.93	0\\
7.94	0\\
7.95	0\\
7.96	0\\
7.97	0\\
7.98	0\\
7.99	0\\
8	0\\
8.01	0\\
8.02	0\\
8.03	0\\
8.04	0\\
8.05	0\\
8.06	0\\
8.07	0\\
8.08	0\\
8.09	0\\
8.1	0\\
8.11	0\\
8.12	0\\
8.13	0\\
8.14	0\\
8.15	0\\
8.16	0\\
8.17	0\\
8.18	0\\
8.19	0\\
8.2	0\\
8.21	0\\
8.22	0\\
8.23	0\\
8.24	0\\
8.25	0\\
8.26	0\\
8.27	0\\
8.28	0\\
8.29	0\\
8.3	0\\
8.31	0\\
8.32	0\\
8.33	0\\
8.34	0\\
8.35	0\\
8.36	0\\
8.37	0\\
8.38	0\\
8.39	0\\
8.4	0\\
8.41	0\\
8.42	0\\
8.43	0\\
8.44	0\\
8.45	0\\
8.46	0\\
8.47	0\\
8.48	0\\
8.49	0\\
8.5	0\\
8.51	0\\
8.52	0\\
8.53	0\\
8.54	0\\
8.55	0\\
8.56	0\\
8.57	0\\
8.58	0\\
8.59	0\\
8.6	0\\
8.61	0\\
8.62	0\\
8.63	0\\
8.64	0\\
8.65	0\\
8.66	0\\
8.67	0\\
8.68	0\\
8.69	0\\
8.7	0\\
8.71	0\\
8.72	0\\
8.73	0\\
8.74	0\\
8.75	0\\
8.76	0\\
8.77	0\\
8.78	0\\
8.79	0\\
8.8	0\\
8.81	0\\
8.82	0\\
8.83	0\\
8.84	0\\
8.85	0\\
8.86	0\\
8.87	0\\
8.88	0\\
8.89	0\\
8.9	0\\
8.91	0\\
8.92	0\\
8.93	0\\
8.94	0\\
8.95	0\\
8.96	0\\
8.97	0\\
8.98	0\\
8.99	0\\
9	0\\
9.01	0\\
9.02	0\\
9.03	0\\
9.04	0\\
9.05	0\\
9.06	0\\
9.07	0\\
9.08	0\\
9.09	0\\
9.1	0\\
9.11	0\\
9.12	0\\
9.13	0\\
9.14	0\\
9.15	0\\
9.16	0\\
9.17	0\\
9.18	0\\
9.19	0\\
9.2	0\\
9.21	0\\
9.22	0\\
9.23	0\\
9.24	0\\
9.25	0\\
9.26	0\\
9.27	0\\
9.28	0\\
9.29	0\\
9.3	0\\
9.31	0\\
9.32	0\\
9.33	0\\
9.34	0\\
9.35	0\\
9.36	0\\
9.37	0\\
9.38	0\\
9.39	0\\
9.4	0\\
9.41	0\\
9.42	0\\
9.43	0\\
9.44	0\\
9.45	0\\
9.46	0\\
9.47	0\\
9.48	0\\
9.49	0\\
9.5	0\\
9.51	0\\
9.52	0\\
9.53	0\\
9.54	0\\
9.55	0\\
9.56	0\\
9.57	0\\
9.58	0\\
9.59	0\\
9.6	0\\
9.61	0\\
9.62	0\\
9.63	0\\
9.64	0\\
9.65	0\\
9.66	0\\
9.67	0\\
9.68	0\\
9.69	0\\
9.7	0\\
9.71	0\\
9.72	0\\
9.73	0\\
9.74	0\\
9.75	0\\
9.76	0\\
9.77	0\\
9.78	0\\
9.79	0\\
9.8	0\\
9.81	0\\
9.82	0\\
9.83	0\\
9.84	0\\
9.85	0\\
9.86	0\\
9.87	0\\
9.88	0\\
9.89	0\\
9.9	0\\
9.91	0\\
9.92	0\\
9.93	0\\
9.94	0\\
9.95	0\\
9.96	0\\
9.97	0\\
9.98	0\\
9.99	0\\
10	0\\
10.01	0\\
10.02	0\\
10.03	0\\
10.04	0\\
10.05	0\\
10.06	0\\
10.07	0\\
10.08	0\\
10.09	0\\
10.1	0\\
10.11	0\\
10.12	0\\
10.13	0\\
10.14	0\\
10.15	0\\
10.16	0\\
10.17	0\\
10.18	0\\
10.19	0\\
10.2	0\\
10.21	0\\
10.22	0\\
10.23	0\\
10.24	0\\
10.25	0\\
10.26	0\\
10.27	0\\
10.28	0\\
10.29	0\\
10.3	0\\
10.31	0\\
10.32	0\\
10.33	0\\
10.34	0\\
10.35	0\\
10.36	0\\
10.37	0\\
10.38	0\\
10.39	0\\
10.4	0\\
10.41	0\\
10.42	0\\
10.43	0\\
10.44	0\\
10.45	0\\
10.46	0\\
10.47	0\\
10.48	0\\
10.49	0\\
10.5	0\\
10.51	0\\
10.52	0\\
10.53	0\\
10.54	0\\
10.55	0\\
10.56	0\\
10.57	0\\
10.58	0\\
10.59	0\\
10.6	0\\
10.61	0\\
10.62	0\\
10.63	0\\
10.64	0\\
10.65	0\\
10.66	0\\
10.67	0\\
10.68	0\\
10.69	0\\
10.7	0\\
10.71	0\\
10.72	0\\
10.73	0\\
10.74	0\\
10.75	0\\
10.76	0\\
10.77	0\\
10.78	0\\
10.79	0\\
10.8	0\\
10.81	0\\
10.82	0\\
10.83	0\\
10.84	0\\
10.85	0\\
10.86	0\\
10.87	0\\
10.88	0\\
10.89	0\\
10.9	0\\
10.91	0\\
10.92	0\\
10.93	0\\
10.94	0\\
10.95	0\\
10.96	0\\
10.97	0\\
10.98	0\\
10.99	0\\
11	0\\
11.01	0\\
11.02	0\\
11.03	0\\
11.04	0\\
11.05	0\\
11.06	0\\
11.07	0\\
11.08	0\\
11.09	0\\
11.1	0\\
11.11	0\\
11.12	0\\
11.13	0\\
11.14	0\\
11.15	0\\
11.16	0\\
11.17	0\\
11.18	0\\
11.19	0\\
11.2	0\\
11.21	0\\
11.22	0\\
11.23	0\\
11.24	0\\
11.25	0\\
11.26	0\\
11.27	0\\
11.28	0\\
11.29	0\\
11.3	0\\
11.31	0\\
11.32	0\\
11.33	0\\
11.34	0\\
11.35	0\\
11.36	0\\
11.37	0\\
11.38	0\\
11.39	0\\
11.4	0\\
11.41	0\\
11.42	0\\
11.43	0\\
11.44	0\\
11.45	0\\
11.46	0\\
11.47	0\\
11.48	0\\
11.49	0\\
11.5	0\\
11.51	0\\
11.52	0\\
11.53	0\\
11.54	0\\
11.55	0\\
11.56	0\\
11.57	0\\
11.58	0\\
11.59	0\\
11.6	0\\
11.61	0\\
11.62	0\\
11.63	0\\
11.64	0\\
11.65	0\\
11.66	0\\
11.67	0\\
11.68	0\\
11.69	0\\
11.7	0\\
11.71	0\\
11.72	0\\
11.73	0\\
11.74	0\\
11.75	0\\
11.76	0\\
11.77	0\\
11.78	0\\
11.79	0\\
11.8	0\\
11.81	0\\
11.82	0\\
11.83	0\\
11.84	0\\
11.85	0\\
11.86	0\\
11.87	0\\
11.88	0\\
11.89	0\\
11.9	0\\
11.91	0\\
11.92	0\\
11.93	0\\
11.94	0\\
11.95	0\\
11.96	0\\
11.97	0\\
11.98	0\\
11.99	0\\
12	0\\
12.01	0\\
12.02	0\\
12.03	0\\
12.04	0\\
12.05	0\\
12.06	0\\
12.07	0\\
12.08	0\\
12.09	0\\
12.1	0\\
12.11	0\\
12.12	0\\
12.13	0\\
12.14	0\\
12.15	0\\
12.16	0\\
12.17	0\\
12.18	0\\
12.19	0\\
12.2	0\\
12.21	0\\
12.22	0\\
12.23	0\\
12.24	0\\
12.25	0\\
12.26	0\\
12.27	0\\
12.28	0\\
12.29	0\\
12.3	0\\
12.31	0\\
12.32	0\\
12.33	0\\
12.34	0\\
12.35	0\\
12.36	0\\
12.37	0\\
12.38	0\\
12.39	0\\
12.4	0\\
12.41	0\\
12.42	0\\
12.43	0\\
12.44	0\\
12.45	0\\
12.46	0\\
12.47	0\\
12.48	0\\
12.49	0\\
12.5	0\\
12.51	0\\
12.52	0\\
12.53	0\\
12.54	0\\
12.55	0\\
12.56	0\\
12.57	0\\
12.58	0\\
12.59	0\\
12.6	0\\
12.61	0\\
12.62	0\\
12.63	0\\
12.64	0\\
12.65	0\\
12.66	0\\
12.67	0\\
12.68	0\\
12.69	0\\
12.7	0\\
12.71	0\\
12.72	0\\
12.73	0\\
12.74	0\\
12.75	0\\
12.76	0\\
12.77	0\\
12.78	0\\
12.79	0\\
12.8	0\\
12.81	0\\
12.82	0\\
12.83	0\\
12.84	0\\
12.85	0\\
12.86	0\\
12.87	0\\
12.88	0\\
12.89	0\\
12.9	0\\
12.91	0\\
12.92	0\\
12.93	0\\
12.94	0\\
12.95	0\\
12.96	0\\
12.97	0\\
12.98	0\\
12.99	0\\
13	0\\
13.01	0\\
13.02	0\\
13.03	0\\
13.04	0\\
13.05	0\\
13.06	0\\
13.07	0\\
13.08	0\\
13.09	0\\
13.1	0\\
13.11	0\\
13.12	0\\
13.13	0\\
13.14	0\\
13.15	0\\
13.16	0\\
13.17	0\\
13.18	0\\
13.19	0\\
13.2	0\\
13.21	0\\
13.22	0\\
13.23	0\\
13.24	0\\
13.25	0\\
13.26	0\\
13.27	0\\
13.28	0\\
13.29	0\\
13.3	0\\
13.31	0\\
13.32	0\\
13.33	0\\
13.34	0\\
13.35	0\\
13.36	0\\
13.37	0\\
13.38	0\\
13.39	0\\
13.4	0\\
13.41	0\\
13.42	0\\
13.43	0\\
13.44	0\\
13.45	0\\
13.46	0\\
13.47	0\\
13.48	0\\
13.49	0\\
13.5	0\\
13.51	0\\
13.52	0\\
13.53	0\\
13.54	0\\
13.55	0\\
13.56	0\\
13.57	0\\
13.58	0\\
13.59	0\\
13.6	0\\
13.61	0\\
13.62	0\\
13.63	0\\
13.64	0\\
13.65	0\\
13.66	0\\
13.67	0\\
13.68	0\\
13.69	0\\
13.7	0\\
13.71	0\\
13.72	0\\
13.73	0\\
13.74	0\\
13.75	0\\
13.76	0\\
13.77	0\\
13.78	0\\
13.79	0\\
13.8	0\\
13.81	0\\
13.82	0\\
13.83	0\\
13.84	0\\
13.85	0\\
13.86	0\\
13.87	0\\
13.88	0\\
13.89	0\\
13.9	0\\
13.91	0\\
13.92	0\\
13.93	0\\
13.94	0\\
13.95	0\\
13.96	0\\
13.97	0\\
13.98	0\\
13.99	0\\
14	0\\
14.01	0\\
14.02	0\\
14.03	0\\
14.04	0\\
14.05	0\\
14.06	0\\
14.07	0\\
14.08	0\\
14.09	0\\
14.1	0\\
14.11	0\\
14.12	0\\
14.13	0\\
14.14	0\\
14.15	0\\
14.16	0\\
14.17	0\\
14.18	0\\
14.19	0\\
14.2	0\\
14.21	0\\
14.22	0\\
14.23	0\\
14.24	0\\
14.25	0\\
14.26	0\\
14.27	0\\
14.28	0\\
14.29	0\\
14.3	0\\
14.31	0\\
14.32	0\\
14.33	0\\
14.34	0\\
14.35	0\\
14.36	0\\
14.37	0\\
14.38	0\\
14.39	0\\
14.4	0\\
14.41	0\\
14.42	0\\
14.43	0\\
14.44	0\\
14.45	0\\
14.46	0\\
14.47	0\\
14.48	0\\
14.49	0\\
14.5	0\\
14.51	0\\
14.52	0\\
14.53	0\\
14.54	0\\
14.55	0\\
14.56	0\\
14.57	0\\
14.58	0\\
14.59	0\\
14.6	0\\
14.61	0\\
14.62	0\\
14.63	0\\
14.64	0\\
14.65	0\\
14.66	0\\
14.67	0\\
14.68	0\\
14.69	0\\
14.7	0\\
14.71	0\\
14.72	0\\
14.73	0\\
14.74	0\\
14.75	0\\
14.76	0\\
14.77	0\\
14.78	0\\
14.79	0\\
14.8	0\\
14.81	0\\
14.82	0\\
14.83	0\\
14.84	0\\
14.85	0\\
14.86	0\\
14.87	0\\
14.88	0\\
14.89	0\\
14.9	0\\
14.91	0\\
14.92	0\\
14.93	0\\
14.94	0\\
14.95	0\\
14.96	0\\
14.97	0\\
14.98	0\\
14.99	0\\
15	0\\
15.01	0\\
15.02	0\\
15.03	0\\
15.04	0\\
15.05	0\\
15.06	0\\
15.07	0\\
15.08	0\\
15.09	0\\
15.1	0\\
15.11	0\\
15.12	0\\
15.13	0\\
15.14	0\\
15.15	0\\
15.16	0\\
15.17	0\\
15.18	0\\
15.19	0\\
15.2	0\\
15.21	0\\
15.22	0\\
15.23	0\\
15.24	0\\
15.25	0\\
15.26	0\\
15.27	0\\
15.28	0\\
15.29	0\\
15.3	0\\
15.31	0\\
15.32	0\\
15.33	0\\
15.34	0\\
15.35	0\\
15.36	0\\
15.37	0\\
15.38	0\\
15.39	0\\
15.4	0\\
15.41	0\\
15.42	0\\
15.43	0\\
15.44	0\\
15.45	0\\
15.46	0\\
15.47	0\\
15.48	0\\
15.49	0\\
15.5	0\\
15.51	0\\
15.52	0\\
15.53	0\\
15.54	0\\
15.55	0\\
15.56	0\\
15.57	0\\
15.58	0\\
15.59	0\\
15.6	0\\
15.61	0\\
15.62	0\\
15.63	0\\
15.64	0\\
15.65	0\\
15.66	0\\
15.67	0\\
15.68	0\\
15.69	0\\
15.7	0\\
15.71	0\\
15.72	0\\
15.73	0\\
15.74	0\\
15.75	0\\
15.76	0\\
15.77	0\\
15.78	0\\
15.79	0\\
15.8	0\\
15.81	0\\
15.82	0\\
15.83	0\\
15.84	0\\
15.85	0\\
15.86	0\\
15.87	0\\
15.88	0\\
15.89	0\\
15.9	0\\
15.91	0\\
15.92	0\\
15.93	0\\
15.94	0\\
15.95	0\\
15.96	0\\
15.97	0\\
15.98	0\\
15.99	0\\
16	0\\
16.01	0\\
16.02	0\\
16.03	0\\
16.04	0\\
16.05	0\\
16.06	0\\
16.07	0\\
16.08	0\\
16.09	0\\
16.1	0\\
16.11	0\\
16.12	0\\
16.13	0\\
16.14	0\\
16.15	0\\
16.16	0\\
16.17	0\\
16.18	0\\
16.19	0\\
16.2	0\\
16.21	0\\
16.22	0\\
16.23	0\\
16.24	0\\
16.25	0\\
16.26	0\\
16.27	0\\
16.28	0\\
16.29	0\\
16.3	0\\
16.31	0\\
16.32	0\\
16.33	0\\
16.34	0\\
16.35	0\\
16.36	0\\
16.37	0\\
16.38	0\\
16.39	0\\
16.4	0\\
16.41	0\\
16.42	0\\
16.43	0\\
16.44	0\\
16.45	0\\
16.46	0\\
16.47	0\\
16.48	0\\
16.49	0\\
16.5	0\\
16.51	0\\
16.52	0\\
16.53	0\\
16.54	0\\
16.55	0\\
16.56	0\\
16.57	0\\
16.58	0\\
16.59	0\\
16.6	0\\
16.61	0\\
16.62	0\\
16.63	0\\
16.64	0\\
16.65	0\\
16.66	0\\
16.67	0\\
16.68	0\\
16.69	0\\
16.7	0\\
16.71	0\\
16.72	0\\
16.73	0\\
16.74	0\\
16.75	0\\
16.76	0\\
16.77	0\\
16.78	0\\
16.79	0\\
16.8	0\\
16.81	0\\
16.82	0\\
16.83	0\\
16.84	0\\
16.85	0\\
16.86	0\\
16.87	0\\
16.88	0\\
16.89	0\\
16.9	0\\
16.91	0\\
16.92	0\\
16.93	0\\
16.94	0\\
16.95	0\\
16.96	0\\
16.97	0\\
16.98	0\\
16.99	0\\
17	0\\
17.01	0\\
17.02	0\\
17.03	0\\
17.04	0\\
17.05	0\\
17.06	0\\
17.07	0\\
17.08	0\\
17.09	0\\
17.1	0\\
17.11	0\\
17.12	0\\
17.13	0\\
17.14	0\\
17.15	0\\
17.16	0\\
17.17	0\\
17.18	0\\
17.19	0\\
17.2	0\\
17.21	0\\
17.22	0\\
17.23	0\\
17.24	0\\
17.25	0\\
17.26	0\\
17.27	0\\
17.28	0\\
17.29	0\\
17.3	0\\
17.31	0\\
17.32	0\\
17.33	0\\
17.34	0\\
17.35	0\\
17.36	0\\
17.37	0\\
17.38	0\\
17.39	0\\
17.4	0\\
17.41	0\\
17.42	0\\
17.43	0\\
17.44	0\\
17.45	0\\
17.46	0\\
17.47	0\\
17.48	0\\
17.49	0\\
17.5	0\\
17.51	0\\
17.52	0\\
17.53	0\\
17.54	0\\
17.55	0\\
17.56	0\\
17.57	0\\
17.58	0\\
17.59	0\\
17.6	0\\
17.61	0\\
17.62	0\\
17.63	0\\
17.64	0\\
17.65	0\\
17.66	0\\
17.67	0\\
17.68	0\\
17.69	0\\
17.7	0\\
17.71	0\\
17.72	0\\
17.73	0\\
17.74	0\\
17.75	0\\
17.76	0\\
17.77	0\\
17.78	0\\
17.79	0\\
17.8	0\\
17.81	0\\
17.82	0\\
17.83	0\\
17.84	0\\
17.85	0\\
17.86	0\\
17.87	0\\
17.88	0\\
17.89	0\\
17.9	0\\
17.91	0\\
17.92	0\\
17.93	0\\
17.94	0\\
17.95	0\\
17.96	0\\
17.97	0\\
17.98	0\\
17.99	0\\
18	0\\
18.01	0\\
18.02	0\\
18.03	0\\
18.04	0\\
18.05	0\\
18.06	0\\
18.07	0\\
18.08	0\\
18.09	0\\
18.1	0\\
18.11	0\\
18.12	0\\
18.13	0\\
18.14	0\\
18.15	0\\
18.16	0\\
18.17	0\\
18.18	0\\
18.19	0\\
18.2	0\\
18.21	0\\
18.22	0\\
18.23	0\\
18.24	0\\
18.25	0\\
18.26	0\\
18.27	0\\
18.28	0\\
18.29	0\\
18.3	0\\
18.31	0\\
18.32	0\\
18.33	0\\
18.34	0\\
18.35	0\\
18.36	0\\
18.37	0\\
18.38	0\\
18.39	0\\
18.4	0\\
18.41	0\\
18.42	0\\
18.43	0\\
18.44	0\\
18.45	0\\
18.46	0\\
18.47	0\\
18.48	0\\
18.49	0\\
18.5	0\\
18.51	0\\
18.52	0\\
18.53	0\\
18.54	0\\
18.55	0\\
18.56	0\\
18.57	0\\
18.58	0\\
18.59	0\\
18.6	0\\
18.61	0\\
18.62	0\\
18.63	0\\
18.64	0\\
18.65	0\\
18.66	0\\
18.67	0\\
18.68	0\\
18.69	0\\
18.7	0\\
18.71	0\\
18.72	0\\
18.73	0\\
18.74	0\\
18.75	0\\
18.76	0\\
18.77	0\\
18.78	0\\
18.79	0\\
18.8	0\\
18.81	0\\
18.82	0\\
18.83	0\\
18.84	0\\
18.85	0\\
18.86	0\\
18.87	0\\
18.88	0\\
18.89	0\\
18.9	0\\
18.91	0\\
18.92	0\\
18.93	0\\
18.94	0\\
18.95	0\\
18.96	0\\
18.97	0\\
18.98	0\\
18.99	0\\
19	0\\
19.01	0\\
19.02	0\\
19.03	0\\
19.04	0\\
19.05	0\\
19.06	0\\
19.07	0\\
19.08	0\\
19.09	0\\
19.1	0\\
19.11	0\\
19.12	0\\
19.13	0\\
19.14	0\\
19.15	0\\
19.16	0\\
19.17	0\\
19.18	0\\
19.19	0\\
19.2	0\\
19.21	0\\
19.22	0\\
19.23	0\\
19.24	0\\
19.25	0\\
19.26	0\\
19.27	0\\
19.28	0\\
19.29	0\\
19.3	0\\
19.31	0\\
19.32	0\\
19.33	0\\
19.34	0\\
19.35	0\\
19.36	0\\
19.37	0\\
19.38	0\\
19.39	0\\
19.4	0\\
19.41	0\\
19.42	0\\
19.43	0\\
19.44	0\\
19.45	0\\
19.46	0\\
19.47	0\\
19.48	0\\
19.49	0\\
19.5	0\\
19.51	0\\
19.52	0\\
19.53	0\\
19.54	0\\
19.55	0\\
19.56	0\\
19.57	0\\
19.58	0\\
19.59	0\\
19.6	0\\
19.61	0\\
19.62	0\\
19.63	0\\
19.64	0\\
19.65	0\\
19.66	0\\
19.67	0\\
19.68	0\\
19.69	0\\
19.7	0\\
19.71	0\\
19.72	0\\
19.73	0\\
19.74	0\\
19.75	0\\
19.76	0\\
19.77	0\\
19.78	0\\
19.79	0\\
19.8	0\\
19.81	0\\
19.82	0\\
19.83	0\\
19.84	0\\
19.85	0\\
19.86	0\\
19.87	0\\
19.88	0\\
19.89	0\\
19.9	0\\
19.91	0\\
19.92	0\\
19.93	0\\
19.94	0\\
19.95	0\\
19.96	0\\
19.97	0\\
19.98	0\\
19.99	0\\
20	0\\
20.01	0\\
20.02	0\\
20.03	0\\
20.04	0\\
20.05	0\\
20.06	0\\
20.07	0\\
20.08	0\\
20.09	0\\
20.1	0\\
20.11	0\\
20.12	0\\
20.13	0\\
20.14	0\\
20.15	0\\
20.16	0\\
20.17	0\\
20.18	0\\
20.19	0\\
20.2	0\\
20.21	0\\
20.22	0\\
20.23	0\\
20.24	0\\
20.25	0\\
20.26	0\\
20.27	0\\
20.28	0\\
20.29	0\\
20.3	0\\
20.31	0\\
20.32	0\\
20.33	0\\
20.34	0\\
20.35	0\\
20.36	0\\
20.37	0\\
20.38	0\\
20.39	0\\
20.4	0\\
20.41	0\\
20.42	0\\
20.43	0\\
20.44	0\\
20.45	0\\
20.46	0\\
20.47	0\\
20.48	0\\
20.49	0\\
20.5	0\\
20.51	0\\
20.52	0\\
20.53	0\\
20.54	0\\
20.55	0\\
20.56	0\\
20.57	0\\
20.58	0\\
20.59	0\\
20.6	0\\
20.61	0\\
20.62	0\\
20.63	0\\
20.64	0\\
20.65	0\\
20.66	0\\
20.67	0\\
20.68	0\\
20.69	0\\
20.7	0\\
20.71	0\\
20.72	0\\
20.73	0\\
20.74	0\\
20.75	0\\
20.76	0\\
20.77	0\\
20.78	0\\
20.79	0\\
20.8	0\\
20.81	0\\
20.82	0\\
20.83	0\\
20.84	0\\
20.85	0\\
20.86	0\\
20.87	0\\
20.88	0\\
20.89	0\\
20.9	0\\
20.91	0\\
20.92	0\\
20.93	0\\
20.94	0\\
20.95	0\\
20.96	0\\
20.97	0\\
20.98	0\\
20.99	0\\
21	0\\
21.01	0\\
21.02	0\\
21.03	0\\
21.04	0\\
21.05	0\\
21.06	0\\
21.07	0\\
21.08	0\\
21.09	0\\
21.1	0\\
21.11	0\\
21.12	0\\
21.13	0\\
21.14	0\\
21.15	0\\
21.16	0\\
21.17	0\\
21.18	0\\
21.19	0\\
21.2	0\\
21.21	0\\
21.22	0\\
21.23	0\\
21.24	0\\
21.25	0\\
21.26	0\\
21.27	0\\
21.28	0\\
21.29	0\\
21.3	0\\
21.31	0\\
21.32	0\\
21.33	0\\
21.34	0\\
21.35	0\\
21.36	0\\
21.37	0\\
21.38	0\\
21.39	0\\
21.4	0\\
21.41	0\\
21.42	0\\
21.43	0\\
21.44	0\\
21.45	0\\
21.46	0\\
21.47	0\\
21.48	0\\
21.49	0\\
21.5	0\\
21.51	0\\
21.52	0\\
21.53	0\\
21.54	0\\
21.55	0\\
21.56	0\\
21.57	0\\
21.58	0\\
21.59	0\\
21.6	0\\
21.61	0\\
21.62	0\\
21.63	0\\
21.64	0\\
21.65	0\\
21.66	0\\
21.67	0\\
21.68	0\\
21.69	0\\
21.7	0\\
21.71	0\\
21.72	0\\
21.73	0\\
21.74	0\\
21.75	0\\
21.76	0\\
21.77	0\\
21.78	0\\
21.79	0\\
21.8	0\\
21.81	0\\
21.82	0\\
21.83	0\\
21.84	0\\
21.85	0\\
21.86	0\\
21.87	0\\
21.88	0\\
21.89	0\\
21.9	0\\
21.91	0\\
21.92	0\\
21.93	0\\
21.94	0\\
21.95	0\\
21.96	0\\
21.97	0\\
21.98	0\\
21.99	0\\
22	0\\
22.01	0\\
22.02	0\\
22.03	0\\
22.04	0\\
22.05	0\\
22.06	0\\
22.07	0\\
22.08	0\\
22.09	0\\
22.1	0\\
22.11	0\\
22.12	0\\
22.13	0\\
22.14	0\\
22.15	0\\
22.16	0\\
22.17	0\\
22.18	0\\
22.19	0\\
22.2	0\\
22.21	0\\
22.22	0\\
22.23	0\\
22.24	0\\
22.25	0\\
22.26	0\\
22.27	0\\
22.28	0\\
22.29	0\\
22.3	0\\
22.31	0\\
22.32	0\\
22.33	0\\
22.34	0\\
22.35	0\\
22.36	0\\
22.37	0\\
22.38	0\\
22.39	0\\
22.4	0\\
22.41	0\\
22.42	0\\
22.43	0\\
22.44	0\\
22.45	0\\
22.46	0\\
22.47	0\\
22.48	0\\
22.49	0\\
22.5	0\\
22.51	0\\
22.52	0\\
22.53	0\\
22.54	0\\
22.55	0\\
22.56	0\\
22.57	0\\
22.58	0\\
22.59	0\\
22.6	0\\
22.61	0\\
22.62	0\\
22.63	0\\
22.64	0\\
22.65	0\\
22.66	0\\
22.67	0\\
22.68	0\\
22.69	0\\
22.7	0\\
22.71	0\\
22.72	0\\
22.73	0\\
22.74	0\\
22.75	0\\
22.76	0\\
22.77	0\\
22.78	0\\
22.79	0\\
22.8	0\\
22.81	0\\
22.82	0\\
22.83	0\\
22.84	0\\
22.85	0\\
22.86	0\\
22.87	0\\
22.88	0\\
22.89	0\\
22.9	0\\
22.91	0\\
22.92	0\\
22.93	0\\
22.94	0\\
22.95	0\\
22.96	0\\
22.97	0\\
22.98	0\\
22.99	0\\
23	0\\
23.01	0\\
23.02	0\\
23.03	0\\
23.04	0\\
23.05	0\\
23.06	0\\
23.07	0\\
23.08	0\\
23.09	0\\
23.1	0\\
23.11	0\\
23.12	0\\
23.13	0\\
23.14	0\\
23.15	0\\
23.16	0\\
23.17	0\\
23.18	0\\
23.19	0\\
23.2	0\\
23.21	0\\
23.22	0\\
23.23	0\\
23.24	0\\
23.25	0\\
23.26	0\\
23.27	0\\
23.28	0\\
23.29	0\\
23.3	0\\
23.31	0\\
23.32	0\\
23.33	0\\
23.34	0\\
23.35	0\\
23.36	0\\
23.37	0\\
23.38	0\\
23.39	0\\
23.4	0\\
23.41	0\\
23.42	0\\
23.43	0\\
23.44	0\\
23.45	0\\
23.46	0\\
23.47	0\\
23.48	0\\
23.49	0\\
23.5	0\\
23.51	0\\
23.52	0\\
23.53	0\\
23.54	0\\
23.55	0\\
23.56	0\\
23.57	0\\
23.58	0\\
23.59	0\\
23.6	0\\
23.61	0\\
23.62	0\\
23.63	0\\
23.64	0\\
23.65	0\\
23.66	0\\
23.67	0\\
23.68	0\\
23.69	0\\
23.7	0\\
23.71	0\\
23.72	0\\
23.73	0\\
23.74	0\\
23.75	0\\
23.76	0\\
23.77	0\\
23.78	0\\
23.79	0\\
23.8	0\\
23.81	0\\
23.82	0\\
23.83	0\\
23.84	0\\
23.85	0\\
23.86	0\\
23.87	0\\
23.88	0\\
23.89	0\\
23.9	0\\
23.91	0\\
23.92	0\\
23.93	0\\
23.94	0\\
23.95	0\\
23.96	0\\
23.97	0\\
23.98	0\\
23.99	0\\
24	0\\
24.01	0\\
24.02	0\\
24.03	0\\
24.04	0\\
24.05	0\\
24.06	0\\
24.07	0\\
24.08	0\\
24.09	0\\
24.1	0\\
24.11	0\\
24.12	0\\
24.13	0\\
24.14	0\\
24.15	0\\
24.16	0\\
24.17	0\\
24.18	0\\
24.19	0\\
24.2	0\\
24.21	0\\
24.22	0\\
24.23	0\\
24.24	0\\
24.25	0\\
24.26	0\\
24.27	0\\
24.28	0\\
24.29	0\\
24.3	0\\
24.31	0\\
24.32	0\\
24.33	0\\
24.34	0\\
24.35	0\\
24.36	0\\
24.37	0\\
24.38	0\\
24.39	0\\
24.4	0\\
24.41	0\\
24.42	0\\
24.43	0\\
24.44	0\\
24.45	0\\
24.46	0\\
24.47	0\\
24.48	0\\
24.49	0\\
24.5	0\\
24.51	0\\
24.52	0\\
24.53	0\\
24.54	0\\
24.55	0\\
24.56	0\\
24.57	0\\
24.58	0\\
24.59	0\\
24.6	0\\
24.61	0\\
24.62	0\\
24.63	0\\
24.64	0\\
24.65	0\\
24.66	0\\
24.67	0\\
24.68	0\\
24.69	0\\
24.7	0\\
24.71	0\\
24.72	0\\
24.73	0\\
24.74	0\\
24.75	0\\
24.76	0\\
24.77	0\\
24.78	0\\
24.79	0\\
24.8	0\\
24.81	0\\
24.82	0\\
24.83	0\\
24.84	0\\
24.85	0\\
24.86	0\\
24.87	0\\
24.88	0\\
24.89	0\\
24.9	0\\
24.91	0\\
24.92	0\\
24.93	0\\
24.94	0\\
24.95	0\\
24.96	0\\
24.97	0\\
24.98	0\\
24.99	0\\
25	0\\
25.01	0\\
25.02	0\\
25.03	0\\
25.04	0\\
25.05	0\\
25.06	0\\
25.07	0\\
25.08	0\\
25.09	0\\
25.1	0\\
25.11	0\\
25.12	0\\
25.13	0\\
25.14	0\\
25.15	0\\
25.16	0\\
25.17	0\\
25.18	0\\
25.19	0\\
25.2	0\\
25.21	0\\
25.22	0\\
25.23	0\\
25.24	0\\
25.25	0\\
25.26	0\\
25.27	0\\
25.28	0\\
25.29	0\\
25.3	0\\
25.31	0\\
25.32	0\\
25.33	0\\
25.34	0\\
25.35	0\\
25.36	0\\
25.37	0\\
25.38	0\\
25.39	0\\
25.4	0\\
25.41	0\\
25.42	0\\
25.43	0\\
25.44	0\\
25.45	0\\
25.46	0\\
25.47	0\\
25.48	0\\
25.49	0\\
25.5	0\\
25.51	0\\
25.52	0\\
25.53	0\\
25.54	0\\
25.55	0\\
25.56	0\\
25.57	0\\
25.58	0\\
25.59	0\\
25.6	0\\
25.61	0\\
25.62	0\\
25.63	0\\
25.64	0\\
25.65	0\\
25.66	0\\
25.67	0\\
25.68	0\\
25.69	0\\
25.7	0\\
25.71	0\\
25.72	0\\
25.73	0\\
25.74	0\\
25.75	0\\
25.76	0\\
25.77	0\\
25.78	0\\
25.79	0\\
25.8	0\\
25.81	0\\
25.82	0\\
25.83	0\\
25.84	0\\
25.85	0\\
25.86	0\\
25.87	0\\
25.88	0\\
25.89	0\\
25.9	0\\
25.91	0\\
25.92	0\\
25.93	0\\
25.94	0\\
25.95	0\\
25.96	0\\
25.97	0\\
25.98	0\\
25.99	0\\
26	0\\
26.01	0\\
26.02	0\\
26.03	0\\
26.04	0\\
26.05	0\\
26.06	0\\
26.07	0\\
26.08	0\\
26.09	0\\
26.1	0\\
26.11	0\\
26.12	0\\
26.13	0\\
26.14	0\\
26.15	0\\
26.16	0\\
26.17	0\\
26.18	0\\
26.19	0\\
26.2	0\\
26.21	0\\
26.22	0\\
26.23	0\\
26.24	0\\
26.25	0\\
26.26	0\\
26.27	0\\
26.28	0\\
26.29	0\\
26.3	0\\
26.31	0\\
26.32	0\\
26.33	0\\
26.34	0\\
26.35	0\\
26.36	0\\
26.37	0\\
26.38	0\\
26.39	0\\
26.4	0\\
26.41	0\\
26.42	0\\
26.43	0\\
26.44	0\\
26.45	0\\
26.46	0\\
26.47	0\\
26.48	0\\
26.49	0\\
26.5	0\\
26.51	0\\
26.52	0\\
26.53	0\\
26.54	0\\
26.55	0\\
26.56	0\\
26.57	0\\
26.58	0\\
26.59	0\\
26.6	0\\
26.61	0\\
26.62	0\\
26.63	0\\
26.64	0\\
26.65	0\\
26.66	0\\
26.67	0\\
26.68	0\\
26.69	0\\
26.7	0\\
26.71	0\\
26.72	0\\
26.73	0\\
26.74	0\\
26.75	0\\
26.76	0\\
26.77	0\\
26.78	0\\
26.79	0\\
26.8	0\\
26.81	0\\
26.82	0\\
26.83	0\\
26.84	0\\
26.85	0\\
26.86	0\\
26.87	0\\
26.88	0\\
26.89	0\\
26.9	0\\
26.91	0\\
26.92	0\\
26.93	0\\
26.94	0\\
26.95	0\\
26.96	0\\
26.97	0\\
26.98	0\\
26.99	0\\
27	0\\
27.01	0\\
27.02	0\\
27.03	0\\
27.04	0\\
27.05	0\\
27.06	0\\
27.07	0\\
27.08	0\\
27.09	0\\
27.1	0\\
27.11	0\\
27.12	0\\
27.13	0\\
27.14	0\\
27.15	0\\
27.16	0\\
27.17	0\\
27.18	0\\
27.19	0\\
27.2	0\\
27.21	0\\
27.22	0\\
27.23	0\\
27.24	0\\
27.25	0\\
27.26	0\\
27.27	0\\
27.28	0\\
27.29	0\\
27.3	0\\
27.31	0\\
27.32	0\\
27.33	0\\
27.34	0\\
27.35	0\\
27.36	0\\
27.37	0\\
27.38	0\\
27.39	0\\
27.4	0\\
27.41	0\\
27.42	0\\
27.43	0\\
27.44	0\\
27.45	0\\
27.46	0\\
27.47	0\\
27.48	0\\
27.49	0\\
27.5	0\\
27.51	0\\
27.52	0\\
27.53	0\\
27.54	0\\
27.55	0\\
27.56	0\\
27.57	0\\
27.58	0\\
27.59	0\\
27.6	0\\
27.61	0\\
27.62	0\\
27.63	0\\
27.64	0\\
27.65	0\\
27.66	0\\
27.67	0\\
27.68	0\\
27.69	0\\
27.7	0\\
27.71	0\\
27.72	0\\
27.73	0\\
27.74	0\\
27.75	0\\
27.76	0\\
27.77	0\\
27.78	0\\
27.79	0\\
27.8	0\\
27.81	0\\
27.82	0\\
27.83	0\\
27.84	0\\
27.85	0\\
27.86	0\\
27.87	0\\
27.88	0\\
27.89	0\\
27.9	0\\
27.91	0\\
27.92	0\\
27.93	0\\
27.94	0\\
27.95	0\\
27.96	0\\
27.97	0\\
27.98	0\\
27.99	0\\
28	0\\
28.01	0\\
28.02	0\\
28.03	0\\
28.04	0\\
28.05	0\\
28.06	0\\
28.07	0\\
28.08	0\\
28.09	0\\
28.1	0\\
28.11	0\\
28.12	0\\
28.13	0\\
28.14	0\\
28.15	0\\
28.16	0\\
28.17	0\\
28.18	0\\
28.19	0\\
28.2	0\\
28.21	0\\
28.22	0\\
28.23	0\\
28.24	0\\
28.25	0\\
28.26	0\\
28.27	0\\
28.28	0\\
28.29	0\\
28.3	0\\
28.31	0\\
28.32	0\\
28.33	0\\
28.34	0\\
28.35	0\\
28.36	0\\
28.37	0\\
28.38	0\\
28.39	0\\
28.4	0\\
28.41	0\\
28.42	0\\
28.43	0\\
28.44	0\\
28.45	0\\
28.46	0\\
28.47	0\\
28.48	0\\
28.49	0\\
28.5	0\\
28.51	0\\
28.52	0\\
28.53	0\\
28.54	0\\
28.55	0\\
28.56	0\\
28.57	0\\
28.58	0\\
28.59	0\\
28.6	0\\
28.61	0\\
28.62	0\\
28.63	0\\
28.64	0\\
28.65	0\\
28.66	0\\
28.67	0\\
28.68	0\\
28.69	0\\
28.7	0\\
28.71	0\\
28.72	0\\
28.73	0\\
28.74	0\\
28.75	0\\
28.76	0\\
28.77	0\\
28.78	0\\
28.79	0\\
28.8	0\\
28.81	0\\
28.82	0\\
28.83	0\\
28.84	0\\
28.85	0\\
28.86	0\\
28.87	0\\
28.88	0\\
28.89	0\\
28.9	0\\
28.91	0\\
28.92	0\\
28.93	0\\
28.94	0\\
28.95	0\\
28.96	0\\
28.97	0\\
28.98	0\\
28.99	0\\
29	0\\
29.01	0\\
29.02	0\\
29.03	0\\
29.04	0\\
29.05	0\\
29.06	0\\
29.07	0\\
29.08	0\\
29.09	0\\
29.1	0\\
29.11	0\\
29.12	0\\
29.13	0\\
29.14	0\\
29.15	0\\
29.16	0\\
29.17	0\\
29.18	0\\
29.19	0\\
29.2	0\\
29.21	0\\
29.22	0\\
29.23	0\\
29.24	0\\
29.25	0\\
29.26	0\\
29.27	0\\
29.28	0\\
29.29	0\\
29.3	0\\
29.31	0\\
29.32	0\\
29.33	0\\
29.34	0\\
29.35	0\\
29.36	0\\
29.37	0\\
29.38	0\\
29.39	0\\
29.4	0\\
29.41	0\\
29.42	0\\
29.43	0\\
29.44	0\\
29.45	0\\
29.46	0\\
29.47	0\\
29.48	0\\
29.49	0\\
29.5	0\\
29.51	0\\
29.52	0\\
29.53	0\\
29.54	0\\
29.55	0\\
29.56	0\\
29.57	0\\
29.58	0\\
29.59	0\\
29.6	0\\
29.61	0\\
29.62	0\\
29.63	0\\
29.64	0\\
29.65	0\\
29.66	0\\
29.67	0\\
29.68	0\\
29.69	0\\
29.7	0\\
29.71	0\\
29.72	0\\
29.73	0\\
29.74	0\\
29.75	0\\
29.76	0\\
29.77	0\\
29.78	0\\
29.79	0\\
29.8	0\\
29.81	0\\
29.82	0\\
29.83	0\\
29.84	0\\
29.85	0\\
29.86	0\\
29.87	0\\
29.88	0\\
29.89	0\\
29.9	0\\
29.91	0\\
29.92	0\\
29.93	0\\
29.94	0\\
29.95	0\\
29.96	0\\
29.97	0\\
29.98	0\\
29.99	0\\
30	0\\
30.01	0\\
30.02	0\\
30.03	0\\
30.04	0\\
30.05	0\\
30.06	0\\
30.07	0\\
30.08	0\\
30.09	0\\
30.1	0\\
30.11	0\\
30.12	0\\
30.13	0\\
30.14	0\\
30.15	0\\
30.16	0\\
30.17	0\\
30.18	0\\
30.19	0\\
30.2	0\\
30.21	0\\
30.22	0\\
30.23	0\\
30.24	0\\
30.25	0\\
30.26	0\\
30.27	0\\
30.28	0\\
30.29	0\\
30.3	0\\
30.31	0\\
30.32	0\\
30.33	0\\
30.34	0\\
30.35	0\\
30.36	0\\
30.37	0\\
30.38	0\\
30.39	0\\
30.4	0\\
30.41	0\\
30.42	0\\
30.43	0\\
30.44	0\\
30.45	0\\
30.46	0\\
30.47	0\\
30.48	0\\
30.49	0\\
30.5	0\\
30.51	0\\
30.52	0\\
30.53	0\\
30.54	0\\
30.55	0\\
30.56	0\\
30.57	0\\
30.58	0\\
30.59	0\\
30.6	0\\
30.61	0\\
30.62	0\\
30.63	0\\
30.64	0\\
30.65	0\\
30.66	0\\
30.67	0\\
30.68	0\\
30.69	0\\
30.7	0\\
30.71	0\\
30.72	0\\
30.73	0\\
30.74	0\\
30.75	0\\
30.76	0\\
30.77	0\\
30.78	0\\
30.79	0\\
30.8	0\\
30.81	0\\
30.82	0\\
30.83	0\\
30.84	0\\
30.85	0\\
30.86	0\\
30.87	0\\
30.88	0\\
30.89	0\\
30.9	0\\
30.91	0\\
30.92	0\\
30.93	0\\
30.94	0\\
30.95	0\\
30.96	0\\
30.97	0\\
30.98	0\\
30.99	0\\
31	0\\
31.01	0\\
31.02	0\\
31.03	0\\
31.04	0\\
31.05	0\\
31.06	0\\
31.07	0\\
31.08	0\\
31.09	0\\
31.1	0\\
31.11	0\\
31.12	0\\
31.13	0\\
31.14	0\\
31.15	0\\
31.16	0\\
31.17	0\\
31.18	0\\
31.19	0\\
31.2	0\\
31.21	0\\
31.22	0\\
31.23	0\\
31.24	0\\
31.25	0\\
31.26	0\\
31.27	0\\
31.28	0\\
31.29	0\\
31.3	0\\
31.31	0\\
31.32	0\\
31.33	0\\
31.34	0\\
31.35	0\\
31.36	0\\
31.37	0\\
31.38	0\\
31.39	0\\
31.4	0\\
31.41	0\\
31.42	0\\
31.43	0\\
31.44	0\\
31.45	0\\
31.46	0\\
31.47	0\\
31.48	0\\
31.49	0\\
31.5	0\\
31.51	0\\
31.52	0\\
31.53	0\\
31.54	0\\
31.55	0\\
31.56	0\\
31.57	0\\
31.58	0\\
31.59	0\\
31.6	0\\
31.61	0\\
31.62	0\\
31.63	0\\
31.64	0\\
31.65	0\\
31.66	0\\
31.67	0\\
31.68	0\\
31.69	0\\
31.7	0\\
31.71	0\\
31.72	0\\
31.73	0\\
31.74	0\\
31.75	0\\
31.76	0\\
31.77	0\\
31.78	0\\
31.79	0\\
31.8	0\\
31.81	0\\
31.82	0\\
31.83	0\\
31.84	0\\
31.85	0\\
31.86	0\\
31.87	0\\
31.88	0\\
31.89	0\\
31.9	0\\
31.91	0\\
31.92	0\\
31.93	0\\
31.94	0\\
31.95	0\\
31.96	0\\
31.97	0\\
31.98	0\\
31.99	0\\
32	0\\
32.01	0\\
32.02	0\\
32.03	0\\
32.04	0\\
32.05	0\\
32.06	0\\
32.07	0\\
32.08	0\\
32.09	0\\
32.1	0\\
32.11	0\\
32.12	0\\
32.13	0\\
32.14	0\\
32.15	0\\
32.16	0\\
32.17	0\\
32.18	0\\
32.19	0\\
32.2	0\\
32.21	0\\
32.22	0\\
32.23	0\\
32.24	0\\
32.25	0\\
32.26	0\\
32.27	0\\
32.28	0\\
32.29	0\\
32.3	0\\
32.31	0\\
32.32	0\\
32.33	0\\
32.34	0\\
32.35	0\\
32.36	0\\
32.37	0\\
32.38	0\\
32.39	0\\
32.4	0\\
32.41	0\\
32.42	0\\
32.43	0\\
32.44	0\\
32.45	0\\
32.46	0\\
32.47	0\\
32.48	0\\
32.49	0\\
32.5	0\\
32.51	0\\
32.52	0\\
32.53	0\\
32.54	0\\
32.55	0\\
32.56	0\\
32.57	0\\
32.58	0\\
32.59	0\\
32.6	0\\
32.61	0\\
32.62	0\\
32.63	0\\
32.64	0\\
32.65	0\\
32.66	0\\
32.67	0\\
32.68	0\\
32.69	0\\
32.7	0\\
32.71	0\\
32.72	0\\
32.73	0\\
32.74	0\\
32.75	0\\
32.76	0\\
32.77	0\\
32.78	0\\
32.79	0\\
32.8	0\\
32.81	0\\
32.82	0\\
32.83	0\\
32.84	0\\
32.85	0\\
32.86	0\\
32.87	0\\
32.88	0\\
32.89	0\\
32.9	0\\
32.91	0\\
32.92	0\\
32.93	0\\
32.94	0\\
32.95	0\\
32.96	0\\
32.97	0\\
32.98	0\\
32.99	0\\
33	0\\
33.01	0\\
33.02	0\\
33.03	0\\
33.04	0\\
33.05	0\\
33.06	0\\
33.07	0\\
33.08	0\\
33.09	0\\
33.1	0\\
33.11	0\\
33.12	0\\
33.13	0\\
33.14	0\\
33.15	0\\
33.16	0\\
33.17	0\\
33.18	0\\
33.19	0\\
33.2	0\\
33.21	0\\
33.22	0\\
33.23	0\\
33.24	0\\
33.25	0\\
33.26	0\\
33.27	0\\
33.28	0\\
33.29	0\\
33.3	0\\
33.31	0\\
33.32	0\\
33.33	0\\
33.34	0\\
33.35	0\\
33.36	0\\
33.37	0\\
33.38	0\\
33.39	0\\
33.4	0\\
33.41	0\\
33.42	0\\
33.43	0\\
33.44	0\\
33.45	0\\
33.46	0\\
33.47	0\\
33.48	0\\
33.49	0\\
33.5	0\\
33.51	0\\
33.52	0\\
33.53	0\\
33.54	0\\
33.55	0\\
33.56	0\\
33.57	0\\
33.58	0\\
33.59	0\\
33.6	0\\
33.61	0\\
33.62	0\\
33.63	0\\
33.64	0\\
33.65	0\\
33.66	0\\
33.67	0\\
33.68	0\\
33.69	0\\
33.7	0\\
33.71	0\\
33.72	0\\
33.73	0\\
33.74	0\\
33.75	0\\
33.76	0\\
33.77	0\\
33.78	0\\
33.79	0\\
33.8	0\\
33.81	0\\
33.82	0\\
33.83	0\\
33.84	0\\
33.85	0\\
33.86	0\\
33.87	0\\
33.88	0\\
33.89	0\\
33.9	0\\
33.91	0\\
33.92	0\\
33.93	0\\
33.94	0\\
33.95	0\\
33.96	0\\
33.97	0\\
33.98	0\\
33.99	0\\
34	0\\
34.01	0\\
34.02	0\\
34.03	0\\
34.04	0\\
34.05	0\\
34.06	0\\
34.07	0\\
34.08	0\\
34.09	0\\
34.1	0\\
34.11	0\\
34.12	0\\
34.13	0\\
34.14	0\\
34.15	0\\
34.16	0\\
34.17	0\\
34.18	0\\
34.19	0\\
34.2	0\\
34.21	0\\
34.22	0\\
34.23	0\\
34.24	0\\
34.25	0\\
34.26	0\\
34.27	0\\
34.28	0\\
34.29	0\\
34.3	0\\
34.31	0\\
34.32	0\\
34.33	0\\
34.34	0\\
34.35	0\\
34.36	0\\
34.37	0\\
34.38	0\\
34.39	0\\
34.4	0\\
34.41	0\\
34.42	0\\
34.43	0\\
34.44	0\\
34.45	0\\
34.46	0\\
34.47	0\\
34.48	0\\
34.49	0\\
34.5	0\\
34.51	0\\
34.52	0\\
34.53	0\\
34.54	0\\
34.55	0\\
34.56	0\\
34.57	0\\
34.58	0\\
34.59	0\\
34.6	0\\
34.61	0\\
34.62	0\\
34.63	0\\
34.64	0\\
34.65	0\\
34.66	0\\
34.67	0\\
34.68	0\\
34.69	0\\
34.7	0\\
34.71	0\\
34.72	0\\
34.73	0\\
34.74	0\\
34.75	0\\
34.76	0\\
34.77	0\\
34.78	0\\
34.79	0\\
34.8	0\\
34.81	0\\
34.82	0\\
34.83	0\\
34.84	0\\
34.85	0\\
34.86	0\\
34.87	0\\
34.88	0\\
34.89	0\\
34.9	0\\
34.91	0\\
34.92	0\\
34.93	0\\
34.94	0\\
34.95	0\\
34.96	0\\
34.97	0\\
34.98	0\\
34.99	0\\
35	0\\
35.01	0\\
35.02	0\\
35.03	0\\
35.04	0\\
35.05	0\\
35.06	0\\
35.07	0\\
35.08	0\\
35.09	0\\
35.1	0\\
35.11	0\\
35.12	0\\
35.13	0\\
35.14	0\\
35.15	0\\
35.16	0\\
35.17	0\\
35.18	0\\
35.19	0\\
35.2	0\\
35.21	0\\
35.22	0\\
35.23	0\\
35.24	0\\
35.25	0\\
35.26	0\\
35.27	0\\
35.28	0\\
35.29	0\\
35.3	0\\
35.31	0\\
35.32	0\\
35.33	0\\
35.34	0\\
35.35	0\\
35.36	0\\
35.37	0\\
35.38	0\\
35.39	0\\
35.4	0\\
35.41	0\\
35.42	0\\
35.43	0\\
35.44	0\\
35.45	0\\
35.46	0\\
35.47	0\\
35.48	0\\
35.49	0\\
35.5	0\\
35.51	0\\
35.52	0\\
35.53	0\\
35.54	0\\
35.55	0\\
35.56	0\\
35.57	0\\
35.58	0\\
35.59	0\\
35.6	0\\
35.61	0\\
35.62	0\\
35.63	0\\
35.64	0\\
35.65	0\\
35.66	0\\
35.67	0\\
35.68	0\\
35.69	0\\
35.7	0\\
35.71	0\\
35.72	0\\
35.73	0\\
35.74	0\\
35.75	0\\
35.76	0\\
35.77	0\\
35.78	0\\
35.79	0\\
35.8	0\\
35.81	0\\
35.82	0\\
35.83	0\\
35.84	0\\
35.85	0\\
35.86	0\\
35.87	0\\
35.88	0\\
35.89	0\\
35.9	0\\
35.91	0\\
35.92	0\\
35.93	0\\
35.94	0\\
35.95	0\\
35.96	0\\
35.97	0\\
35.98	0\\
35.99	0\\
36	0\\
36.01	0\\
36.02	0\\
36.03	0\\
36.04	0\\
36.05	0\\
36.06	0\\
36.07	0\\
36.08	0\\
36.09	0\\
36.1	0\\
36.11	0\\
36.12	0\\
36.13	0\\
36.14	0\\
36.15	0\\
36.16	0\\
36.17	0\\
36.18	0\\
36.19	0\\
36.2	0\\
36.21	0\\
36.22	0\\
36.23	0\\
36.24	0\\
36.25	0\\
36.26	0\\
36.27	0\\
36.28	0\\
36.29	0\\
36.3	0\\
36.31	0\\
36.32	0\\
36.33	0\\
36.34	0\\
36.35	0\\
36.36	0\\
36.37	0\\
36.38	0\\
36.39	0\\
36.4	0\\
36.41	0\\
36.42	0\\
36.43	0\\
36.44	0\\
36.45	0\\
36.46	0\\
36.47	0\\
36.48	0\\
36.49	0\\
36.5	0\\
36.51	0\\
36.52	0\\
36.53	0\\
36.54	0\\
36.55	0\\
36.56	0\\
36.57	0\\
36.58	0\\
36.59	0\\
36.6	0\\
36.61	0\\
36.62	0\\
36.63	0\\
36.64	0\\
36.65	0\\
36.66	0\\
36.67	0\\
36.68	0\\
36.69	0\\
36.7	0\\
36.71	0\\
36.72	0\\
36.73	0\\
36.74	0\\
36.75	0\\
36.76	0\\
36.77	0\\
36.78	0\\
36.79	0\\
36.8	0\\
36.81	0\\
36.82	0\\
36.83	0\\
36.84	0\\
36.85	0\\
36.86	0\\
36.87	0\\
36.88	0\\
36.89	0\\
36.9	0\\
36.91	0\\
36.92	0\\
36.93	0\\
36.94	0\\
36.95	0\\
36.96	0\\
36.97	0\\
36.98	0\\
36.99	0\\
37	0\\
37.01	0\\
37.02	0\\
37.03	0\\
37.04	0\\
37.05	0\\
37.06	0\\
37.07	0\\
37.08	0\\
37.09	0\\
37.1	0\\
37.11	0\\
37.12	0\\
37.13	0\\
37.14	0\\
37.15	0\\
37.16	0\\
37.17	0\\
37.18	0\\
37.19	0\\
37.2	0\\
37.21	0\\
37.22	0\\
37.23	0\\
37.24	0\\
37.25	0\\
37.26	0\\
37.27	0\\
37.28	0\\
37.29	0\\
37.3	0\\
37.31	0\\
37.32	0\\
37.33	0\\
37.34	0\\
37.35	0\\
37.36	0\\
37.37	0\\
37.38	0\\
37.39	0\\
37.4	0\\
37.41	0\\
37.42	0\\
37.43	0\\
37.44	0\\
37.45	0\\
37.46	0\\
37.47	0\\
37.48	0\\
37.49	0\\
37.5	0\\
37.51	0\\
37.52	0\\
37.53	0\\
37.54	0\\
37.55	0\\
37.56	0\\
37.57	0\\
37.58	0\\
37.59	0\\
37.6	0\\
37.61	0\\
37.62	0\\
37.63	0\\
37.64	0\\
37.65	0\\
37.66	0\\
37.67	0\\
37.68	0\\
37.69	0\\
37.7	0\\
37.71	0\\
37.72	0\\
37.73	0\\
37.74	0\\
37.75	0\\
37.76	0\\
37.77	0\\
37.78	0\\
37.79	0\\
37.8	0\\
37.81	0\\
37.82	0\\
37.83	0\\
37.84	0\\
37.85	0\\
37.86	0\\
37.87	0\\
37.88	0\\
37.89	0\\
37.9	0\\
37.91	0\\
37.92	0\\
37.93	0\\
37.94	0\\
37.95	0\\
37.96	0\\
37.97	0\\
37.98	0\\
37.99	0\\
38	0\\
38.01	0\\
38.02	0\\
38.03	0\\
38.04	0\\
38.05	0\\
38.06	0\\
38.07	0\\
38.08	0\\
38.09	0\\
38.1	0\\
38.11	0\\
38.12	0\\
38.13	0\\
38.14	0\\
38.15	0\\
38.16	0\\
38.17	0\\
38.18	0\\
38.19	0\\
38.2	0\\
38.21	0\\
38.22	0\\
38.23	0\\
38.24	0\\
38.25	0\\
38.26	0\\
38.27	0\\
38.28	0\\
38.29	0\\
38.3	0\\
38.31	0\\
38.32	0\\
38.33	0\\
38.34	0\\
38.35	0\\
38.36	0\\
38.37	0\\
38.38	0\\
38.39	0\\
38.4	0\\
38.41	0\\
38.42	0\\
38.43	0\\
38.44	0\\
38.45	0\\
38.46	0\\
38.47	0\\
38.48	0\\
38.49	0\\
38.5	0\\
38.51	0\\
38.52	0\\
38.53	0\\
38.54	0\\
38.55	0\\
38.56	0\\
38.57	0\\
38.58	0\\
38.59	0\\
38.6	0\\
38.61	0\\
38.62	0\\
38.63	0\\
38.64	0\\
38.65	0\\
38.66	0\\
38.67	0\\
38.68	0\\
38.69	0\\
38.7	0\\
38.71	0\\
38.72	0\\
38.73	0\\
38.74	0\\
38.75	0\\
38.76	0\\
38.77	0\\
38.78	0\\
38.79	0\\
38.8	0\\
38.81	0\\
38.82	0\\
38.83	0\\
38.84	0\\
38.85	0\\
38.86	0\\
38.87	0\\
38.88	0\\
38.89	0\\
38.9	0\\
38.91	0\\
38.92	0\\
38.93	0\\
38.94	0\\
38.95	0\\
38.96	0\\
38.97	0\\
38.98	0\\
38.99	0\\
39	0\\
39.01	0\\
39.02	0\\
39.03	0\\
39.04	0\\
39.05	0\\
39.06	0\\
39.07	0\\
39.08	0\\
39.09	0\\
39.1	0\\
39.11	0\\
39.12	0\\
39.13	0\\
39.14	0\\
39.15	0\\
39.16	0\\
39.17	0\\
39.18	0\\
39.19	0\\
39.2	0\\
39.21	0\\
39.22	0\\
39.23	0\\
39.24	0\\
39.25	0\\
39.26	0\\
39.27	0\\
39.28	0\\
39.29	0\\
39.3	0\\
39.31	0\\
39.32	0\\
39.33	0\\
39.34	0\\
39.35	0\\
39.36	0\\
39.37	0\\
39.38	0\\
39.39	0\\
39.4	0\\
39.41	0\\
39.42	0\\
39.43	0\\
39.44	0\\
39.45	0\\
39.46	0\\
39.47	0\\
39.48	0\\
39.49	0\\
39.5	0\\
39.51	0\\
39.52	0\\
39.53	0\\
39.54	0\\
39.55	0\\
39.56	0\\
39.57	0\\
39.58	0\\
39.59	0\\
39.6	0\\
39.61	0\\
39.62	0\\
39.63	0\\
39.64	0\\
39.65	0\\
39.66	0\\
39.67	0\\
39.68	0\\
39.69	0\\
39.7	0\\
39.71	0\\
39.72	0\\
39.73	0\\
39.74	0\\
39.75	0\\
39.76	0\\
39.77	0\\
39.78	0\\
39.79	0\\
39.8	0\\
39.81	0\\
39.82	0\\
39.83	0\\
39.84	0\\
39.85	0\\
39.86	0\\
39.87	0\\
39.88	0\\
39.89	0\\
39.9	0\\
39.91	0\\
39.92	0\\
39.93	0\\
39.94	0\\
39.95	0\\
39.96	0\\
39.97	0\\
39.98	0\\
39.99	0\\
40	0\\
40.01	0\\
};
\addplot [color=black,solid,forget plot]
  table[row sep=crcr]{%
40.01	0\\
40.02	0\\
40.03	0\\
40.04	0\\
40.05	0\\
40.06	0\\
40.07	0\\
40.08	0\\
40.09	0\\
40.1	0\\
40.11	0\\
40.12	0\\
40.13	0\\
40.14	0\\
40.15	0\\
40.16	0\\
40.17	0\\
40.18	0\\
40.19	0\\
40.2	0\\
40.21	0\\
40.22	0\\
40.23	0\\
40.24	0\\
40.25	0\\
40.26	0\\
40.27	0\\
40.28	0\\
40.29	0\\
40.3	0\\
40.31	0\\
40.32	0\\
40.33	0\\
40.34	0\\
40.35	0\\
40.36	0\\
40.37	0\\
40.38	0\\
40.39	0\\
40.4	0\\
40.41	0\\
40.42	0\\
40.43	0\\
40.44	0\\
40.45	0\\
40.46	0\\
40.47	0\\
40.48	0\\
40.49	0\\
40.5	0\\
40.51	0\\
40.52	0\\
40.53	0\\
40.54	0\\
40.55	0\\
40.56	0\\
40.57	0\\
40.58	0\\
40.59	0\\
40.6	0\\
40.61	0\\
40.62	0\\
40.63	0\\
40.64	0\\
40.65	0\\
40.66	0\\
40.67	0\\
40.68	0\\
40.69	0\\
40.7	0\\
40.71	0\\
40.72	0\\
40.73	0\\
40.74	0\\
40.75	0\\
40.76	0\\
40.77	0\\
40.78	0\\
40.79	0\\
40.8	0\\
40.81	0\\
40.82	0\\
40.83	0\\
40.84	0\\
40.85	0\\
40.86	0\\
40.87	0\\
40.88	0\\
40.89	0\\
40.9	0\\
40.91	0\\
40.92	0\\
40.93	0\\
40.94	0\\
40.95	0\\
40.96	0\\
40.97	0\\
40.98	0\\
40.99	0\\
41	0\\
41.01	0\\
41.02	0\\
41.03	0\\
41.04	0\\
41.05	0\\
41.06	0\\
41.07	0\\
41.08	0\\
41.09	0\\
41.1	0\\
41.11	0\\
41.12	0\\
41.13	0\\
41.14	0\\
41.15	0\\
41.16	0\\
41.17	0\\
41.18	0\\
41.19	0\\
41.2	0\\
41.21	0\\
41.22	0\\
41.23	0\\
41.24	0\\
41.25	0\\
41.26	0\\
41.27	0\\
41.28	0\\
41.29	0\\
41.3	0\\
41.31	0\\
41.32	0\\
41.33	0\\
41.34	0\\
41.35	0\\
41.36	0\\
41.37	0\\
41.38	0\\
41.39	0\\
41.4	0\\
41.41	0\\
41.42	0\\
41.43	0\\
41.44	0\\
41.45	0\\
41.46	0\\
41.47	0\\
41.48	0\\
41.49	0\\
41.5	0\\
41.51	0\\
41.52	0\\
41.53	0\\
41.54	0\\
41.55	0\\
41.56	0\\
41.57	0\\
41.58	0\\
41.59	0\\
41.6	0\\
41.61	0\\
41.62	0\\
41.63	0\\
41.64	0\\
41.65	0\\
41.66	0\\
41.67	0\\
41.68	0\\
41.69	0\\
41.7	0\\
41.71	0\\
41.72	0\\
41.73	0\\
41.74	0\\
41.75	0\\
41.76	0\\
41.77	0\\
41.78	0\\
41.79	0\\
41.8	0\\
41.81	0\\
41.82	0\\
41.83	0\\
41.84	0\\
41.85	0\\
41.86	0\\
41.87	0\\
41.88	0\\
41.89	0\\
41.9	0\\
41.91	0\\
41.92	0\\
41.93	0\\
41.94	0\\
41.95	0\\
41.96	0\\
41.97	0\\
41.98	0\\
41.99	0\\
42	0\\
42.01	0\\
42.02	0\\
42.03	0\\
42.04	0\\
42.05	0\\
42.06	0\\
42.07	0\\
42.08	0\\
42.09	0\\
42.1	0\\
42.11	0\\
42.12	0\\
42.13	0\\
42.14	0\\
42.15	0\\
42.16	0\\
42.17	0\\
42.18	0\\
42.19	0\\
42.2	0\\
42.21	0\\
42.22	0\\
42.23	0\\
42.24	0\\
42.25	0\\
42.26	0\\
42.27	0\\
42.28	0\\
42.29	0\\
42.3	0\\
42.31	0\\
42.32	0\\
42.33	0\\
42.34	0\\
42.35	0\\
42.36	0\\
42.37	0\\
42.38	0\\
42.39	0\\
42.4	0\\
42.41	0\\
42.42	0\\
42.43	0\\
42.44	0\\
42.45	0\\
42.46	0\\
42.47	0\\
42.48	0\\
42.49	0\\
42.5	0\\
42.51	0\\
42.52	0\\
42.53	0\\
42.54	0\\
42.55	0\\
42.56	0\\
42.57	0\\
42.58	0\\
42.59	0\\
42.6	0\\
42.61	0\\
42.62	0\\
42.63	0\\
42.64	0\\
42.65	0\\
42.66	0\\
42.67	0\\
42.68	0\\
42.69	0\\
42.7	0\\
42.71	0\\
42.72	0\\
42.73	0\\
42.74	0\\
42.75	0\\
42.76	0\\
42.77	0\\
42.78	0\\
42.79	0\\
42.8	0\\
42.81	0\\
42.82	0\\
42.83	0\\
42.84	0\\
42.85	0\\
42.86	0\\
42.87	0\\
42.88	0\\
42.89	0\\
42.9	0\\
42.91	0\\
42.92	0\\
42.93	0\\
42.94	0\\
42.95	0\\
42.96	0\\
42.97	0\\
42.98	0\\
42.99	0\\
43	0\\
43.01	0\\
43.02	0\\
43.03	0\\
43.04	0\\
43.05	0\\
43.06	0\\
43.07	0\\
43.08	0\\
43.09	0\\
43.1	0\\
43.11	0\\
43.12	0\\
43.13	0\\
43.14	0\\
43.15	0\\
43.16	0\\
43.17	0\\
43.18	0\\
43.19	0\\
43.2	0\\
43.21	0\\
43.22	0\\
43.23	0\\
43.24	0\\
43.25	0\\
43.26	0\\
43.27	0\\
43.28	0\\
43.29	0\\
43.3	0\\
43.31	0\\
43.32	0\\
43.33	0\\
43.34	0\\
43.35	0\\
43.36	0\\
43.37	0\\
43.38	0\\
43.39	0\\
43.4	0\\
43.41	0\\
43.42	0\\
43.43	0\\
43.44	0\\
43.45	0\\
43.46	0\\
43.47	0\\
43.48	0\\
43.49	0\\
43.5	0\\
43.51	0\\
43.52	0\\
43.53	0\\
43.54	0\\
43.55	0\\
43.56	0\\
43.57	0\\
43.58	0\\
43.59	0\\
43.6	0\\
43.61	0\\
43.62	0\\
43.63	0\\
43.64	0\\
43.65	0\\
43.66	0\\
43.67	0\\
43.68	0\\
43.69	0\\
43.7	0\\
43.71	0\\
43.72	0\\
43.73	0\\
43.74	0\\
43.75	0\\
43.76	0\\
43.77	0\\
43.78	0\\
43.79	0\\
43.8	0\\
43.81	0\\
43.82	0\\
43.83	0\\
43.84	0\\
43.85	0\\
43.86	0\\
43.87	0\\
43.88	0\\
43.89	0\\
43.9	0\\
43.91	0\\
43.92	0\\
43.93	0\\
43.94	0\\
43.95	0\\
43.96	0\\
43.97	0\\
43.98	0\\
43.99	0\\
44	0\\
44.01	0\\
44.02	0\\
44.03	0\\
44.04	0\\
44.05	0\\
44.06	0\\
44.07	0\\
44.08	0\\
44.09	0\\
44.1	0\\
44.11	0\\
44.12	0\\
44.13	0\\
44.14	0\\
44.15	0\\
44.16	0\\
44.17	0\\
44.18	0\\
44.19	0\\
44.2	0\\
44.21	0\\
44.22	0\\
44.23	0\\
44.24	0\\
44.25	0\\
44.26	0\\
44.27	0\\
44.28	0\\
44.29	0\\
44.3	0\\
44.31	0\\
44.32	0\\
44.33	0\\
44.34	0\\
44.35	0\\
44.36	0\\
44.37	0\\
44.38	0\\
44.39	0\\
44.4	0\\
44.41	0\\
44.42	0\\
44.43	0\\
44.44	0\\
44.45	0\\
44.46	0\\
44.47	0\\
44.48	0\\
44.49	0\\
44.5	0\\
44.51	0\\
44.52	0\\
44.53	0\\
44.54	0\\
44.55	0\\
44.56	0\\
44.57	0\\
44.58	0\\
44.59	0\\
44.6	0\\
44.61	0\\
44.62	0\\
44.63	0\\
44.64	0\\
44.65	0\\
44.66	0\\
44.67	0\\
44.68	0\\
44.69	0\\
44.7	0\\
44.71	0\\
44.72	0\\
44.73	0\\
44.74	0\\
44.75	0\\
44.76	0\\
44.77	0\\
44.78	0\\
44.79	0\\
44.8	0\\
44.81	0\\
44.82	0\\
44.83	0\\
44.84	0\\
44.85	0\\
44.86	0\\
44.87	0\\
44.88	0\\
44.89	0\\
44.9	0\\
44.91	0\\
44.92	0\\
44.93	0\\
44.94	0\\
44.95	0\\
44.96	0\\
44.97	0\\
44.98	0\\
44.99	0\\
45	0\\
45.01	0\\
45.02	0\\
45.03	0\\
45.04	0\\
45.05	0\\
45.06	0\\
45.07	0\\
45.08	0\\
45.09	0\\
45.1	0\\
45.11	0\\
45.12	0\\
45.13	0\\
45.14	0\\
45.15	0\\
45.16	0\\
45.17	0\\
45.18	0\\
45.19	0\\
45.2	0\\
45.21	0\\
45.22	0\\
45.23	0\\
45.24	0\\
45.25	0\\
45.26	0\\
45.27	0\\
45.28	0\\
45.29	0\\
45.3	0\\
45.31	0\\
45.32	0\\
45.33	0\\
45.34	0\\
45.35	0\\
45.36	0\\
45.37	0\\
45.38	0\\
45.39	0\\
45.4	0\\
45.41	0\\
45.42	0\\
45.43	0\\
45.44	0\\
45.45	0\\
45.46	0\\
45.47	0\\
45.48	0\\
45.49	0\\
45.5	0\\
45.51	0\\
45.52	0\\
45.53	0\\
45.54	0\\
45.55	0\\
45.56	0\\
45.57	0\\
45.58	0\\
45.59	0\\
45.6	0\\
45.61	0\\
45.62	0\\
45.63	0\\
45.64	0\\
45.65	0\\
45.66	0\\
45.67	0\\
45.68	0\\
45.69	0\\
45.7	0\\
45.71	0\\
45.72	0\\
45.73	0\\
45.74	0\\
45.75	0\\
45.76	0\\
45.77	0\\
45.78	0\\
45.79	0\\
45.8	0\\
45.81	0\\
45.82	0\\
45.83	0\\
45.84	0\\
45.85	0\\
45.86	0\\
45.87	0\\
45.88	0\\
45.89	0\\
45.9	0\\
45.91	0\\
45.92	0\\
45.93	0\\
45.94	0\\
45.95	0\\
45.96	0\\
45.97	0\\
45.98	0\\
45.99	0\\
46	0\\
46.01	0\\
46.02	0\\
46.03	0\\
46.04	0\\
46.05	0\\
46.06	0\\
46.07	0\\
46.08	0\\
46.09	0\\
46.1	0\\
46.11	0\\
46.12	0\\
46.13	0\\
46.14	0\\
46.15	0\\
46.16	0\\
46.17	0\\
46.18	0\\
46.19	0\\
46.2	0\\
46.21	0\\
46.22	0\\
46.23	0\\
46.24	0\\
46.25	0\\
46.26	0\\
46.27	0\\
46.28	0\\
46.29	0\\
46.3	0\\
46.31	0\\
46.32	0\\
46.33	0\\
46.34	0\\
46.35	0\\
46.36	0\\
46.37	0\\
46.38	0\\
46.39	0\\
46.4	0\\
46.41	0\\
46.42	0\\
46.43	0\\
46.44	0\\
46.45	0\\
46.46	0\\
46.47	0\\
46.48	0\\
46.49	0\\
46.5	0\\
46.51	0\\
46.52	0\\
46.53	0\\
46.54	0\\
46.55	0\\
46.56	0\\
46.57	0\\
46.58	0\\
46.59	0\\
46.6	0\\
46.61	0\\
46.62	0\\
46.63	0\\
46.64	0\\
46.65	0\\
46.66	0\\
46.67	0\\
46.68	0\\
46.69	0\\
46.7	0\\
46.71	0\\
46.72	0\\
46.73	0\\
46.74	0\\
46.75	0\\
46.76	0\\
46.77	0\\
46.78	0\\
46.79	0\\
46.8	0\\
46.81	0\\
46.82	0\\
46.83	0\\
46.84	0\\
46.85	0\\
46.86	0\\
46.87	0\\
46.88	0\\
46.89	0\\
46.9	0\\
46.91	0\\
46.92	0\\
46.93	0\\
46.94	0\\
46.95	0\\
46.96	0\\
46.97	0\\
46.98	0\\
46.99	0\\
47	0\\
47.01	0\\
47.02	0\\
47.03	0\\
47.04	0\\
47.05	0\\
47.06	0\\
47.07	0\\
47.08	0\\
47.09	0\\
47.1	0\\
47.11	0\\
47.12	0\\
47.13	0\\
47.14	0\\
47.15	0\\
47.16	0\\
47.17	0\\
47.18	0\\
47.19	0\\
47.2	0\\
47.21	0\\
47.22	0\\
47.23	0\\
47.24	0\\
47.25	0\\
47.26	0\\
47.27	0\\
47.28	0\\
47.29	0\\
47.3	0\\
47.31	0\\
47.32	0\\
47.33	0\\
47.34	0\\
47.35	0\\
47.36	0\\
47.37	0\\
47.38	0\\
47.39	0\\
47.4	0\\
47.41	0\\
47.42	0\\
47.43	0\\
47.44	0\\
47.45	0\\
47.46	0\\
47.47	0\\
47.48	0\\
47.49	0\\
47.5	0\\
47.51	0\\
47.52	0\\
47.53	0\\
47.54	0\\
47.55	0\\
47.56	0\\
47.57	0\\
47.58	0\\
47.59	0\\
47.6	0\\
47.61	0\\
47.62	0\\
47.63	0\\
47.64	0\\
47.65	0\\
47.66	0\\
47.67	0\\
47.68	0\\
47.69	0\\
47.7	0\\
47.71	0\\
47.72	0\\
47.73	0\\
47.74	0\\
47.75	0\\
47.76	0\\
47.77	0\\
47.78	0\\
47.79	0\\
47.8	0\\
47.81	0\\
47.82	0\\
47.83	0\\
47.84	0\\
47.85	0\\
47.86	0\\
47.87	0\\
47.88	0\\
47.89	0\\
47.9	0\\
47.91	0\\
47.92	0\\
47.93	0\\
47.94	0\\
47.95	0\\
47.96	0\\
47.97	0\\
47.98	0\\
47.99	0\\
48	0\\
48.01	0\\
48.02	0\\
48.03	0\\
48.04	0\\
48.05	0\\
48.06	0\\
48.07	0\\
48.08	0\\
48.09	0\\
48.1	0\\
48.11	0\\
48.12	0\\
48.13	0\\
48.14	0\\
48.15	0\\
48.16	0\\
48.17	0\\
48.18	0\\
48.19	0\\
48.2	0\\
48.21	0\\
48.22	0\\
48.23	0\\
48.24	0\\
48.25	0\\
48.26	0\\
48.27	0\\
48.28	0\\
48.29	0\\
48.3	0\\
48.31	0\\
48.32	0\\
48.33	0\\
48.34	0\\
48.35	0\\
48.36	0\\
48.37	0\\
48.38	0\\
48.39	0\\
48.4	0\\
48.41	0\\
48.42	0\\
48.43	0\\
48.44	0\\
48.45	0\\
48.46	0\\
48.47	0\\
48.48	0\\
48.49	0\\
48.5	0\\
48.51	0\\
48.52	0\\
48.53	0\\
48.54	0\\
48.55	0\\
48.56	0\\
48.57	0\\
48.58	0\\
48.59	0\\
48.6	0\\
48.61	0\\
48.62	0\\
48.63	0\\
48.64	0\\
48.65	0\\
48.66	0\\
48.67	0\\
48.68	0\\
48.69	0\\
48.7	0\\
48.71	0\\
48.72	0\\
48.73	0\\
48.74	0\\
48.75	0\\
48.76	0\\
48.77	0\\
48.78	0\\
48.79	0\\
48.8	0\\
48.81	0\\
48.82	0\\
48.83	0\\
48.84	0\\
48.85	0\\
48.86	0\\
48.87	0\\
48.88	0\\
48.89	0\\
48.9	0\\
48.91	0\\
48.92	0\\
48.93	0\\
48.94	0\\
48.95	0\\
48.96	0\\
48.97	0\\
48.98	0\\
48.99	0\\
49	0\\
49.01	0\\
49.02	0\\
49.03	0\\
49.04	0\\
49.05	0\\
49.06	0\\
49.07	0\\
49.08	0\\
49.09	0\\
49.1	0\\
49.11	0\\
49.12	0\\
49.13	0\\
49.14	0\\
49.15	0\\
49.16	0\\
49.17	0\\
49.18	0\\
49.19	0\\
49.2	0\\
49.21	0\\
49.22	0\\
49.23	0\\
49.24	0\\
49.25	0\\
49.26	0\\
49.27	0\\
49.28	0\\
49.29	0\\
49.3	0\\
49.31	0\\
49.32	0\\
49.33	0\\
49.34	0\\
49.35	0\\
49.36	0\\
49.37	0\\
49.38	0\\
49.39	0\\
49.4	0\\
49.41	0\\
49.42	0\\
49.43	0\\
49.44	0\\
49.45	0\\
49.46	0\\
49.47	0\\
49.48	0\\
49.49	0\\
49.5	0\\
49.51	0\\
49.52	0\\
49.53	0\\
49.54	0\\
49.55	0\\
49.56	0\\
49.57	0\\
49.58	0\\
49.59	0\\
49.6	0\\
49.61	0\\
49.62	0\\
49.63	0\\
49.64	0\\
49.65	0\\
49.66	0\\
49.67	0\\
49.68	0\\
49.69	0\\
49.7	0\\
49.71	0\\
49.72	0\\
49.73	0\\
49.74	0\\
49.75	0\\
49.76	0\\
49.77	0\\
49.78	0\\
49.79	0\\
49.8	0\\
49.81	0\\
49.82	0\\
49.83	0\\
49.84	0\\
49.85	0\\
49.86	0\\
49.87	0\\
49.88	0\\
49.89	0\\
49.9	0\\
49.91	0\\
49.92	0\\
49.93	0\\
49.94	0\\
49.95	0\\
49.96	0\\
49.97	0\\
49.98	0\\
49.99	0\\
50	0\\
50.01	0\\
50.02	0\\
50.03	0\\
50.04	0\\
50.05	0\\
50.06	0\\
50.07	0\\
50.08	0\\
50.09	0\\
50.1	0\\
50.11	0\\
50.12	0\\
50.13	0\\
50.14	0\\
50.15	0\\
50.16	0\\
50.17	0\\
50.18	0\\
50.19	0\\
50.2	0\\
50.21	0\\
50.22	0\\
50.23	0\\
50.24	0\\
50.25	0\\
50.26	0\\
50.27	0\\
50.28	0\\
50.29	0\\
50.3	0\\
50.31	0\\
50.32	0\\
50.33	0\\
50.34	0\\
50.35	0\\
50.36	0\\
50.37	0\\
50.38	0\\
50.39	0\\
50.4	0\\
50.41	0\\
50.42	0\\
50.43	0\\
50.44	0\\
50.45	0\\
50.46	0\\
50.47	0\\
50.48	0\\
50.49	0\\
50.5	0\\
50.51	0\\
50.52	0\\
50.53	0\\
50.54	0\\
50.55	0\\
50.56	0\\
50.57	0\\
50.58	0\\
50.59	0\\
50.6	0\\
50.61	0\\
50.62	0\\
50.63	0\\
50.64	0\\
50.65	0\\
50.66	0\\
50.67	0\\
50.68	0\\
50.69	0\\
50.7	0\\
50.71	0\\
50.72	0\\
50.73	0\\
50.74	0\\
50.75	0\\
50.76	0\\
50.77	0\\
50.78	0\\
50.79	0\\
50.8	0\\
50.81	0\\
50.82	0\\
50.83	0\\
50.84	0\\
50.85	0\\
50.86	0\\
50.87	0\\
50.88	0\\
50.89	0\\
50.9	0\\
50.91	0\\
50.92	0\\
50.93	0\\
50.94	0\\
50.95	0\\
50.96	0\\
50.97	0\\
50.98	0\\
50.99	0\\
51	0\\
51.01	0\\
51.02	0\\
51.03	0\\
51.04	0\\
51.05	0\\
51.06	0\\
51.07	0\\
51.08	0\\
51.09	0\\
51.1	0\\
51.11	0\\
51.12	0\\
51.13	0\\
51.14	0\\
51.15	0\\
51.16	0\\
51.17	0\\
51.18	0\\
51.19	0\\
51.2	0\\
51.21	0\\
51.22	0\\
51.23	0\\
51.24	0\\
51.25	0\\
51.26	0\\
51.27	0\\
51.28	0\\
51.29	0\\
51.3	0\\
51.31	0\\
51.32	0\\
51.33	0\\
51.34	0\\
51.35	0\\
51.36	0\\
51.37	0\\
51.38	0\\
51.39	0\\
51.4	0\\
51.41	0\\
51.42	0\\
51.43	0\\
51.44	0\\
51.45	0\\
51.46	0\\
51.47	0\\
51.48	0\\
51.49	0\\
51.5	0\\
51.51	0\\
51.52	0\\
51.53	0\\
51.54	0\\
51.55	0\\
51.56	0\\
51.57	0\\
51.58	0\\
51.59	0\\
51.6	0\\
51.61	0\\
51.62	0\\
51.63	0\\
51.64	0\\
51.65	0\\
51.66	0\\
51.67	0\\
51.68	0\\
51.69	0\\
51.7	0\\
51.71	0\\
51.72	0\\
51.73	0\\
51.74	0\\
51.75	0\\
51.76	0\\
51.77	0\\
51.78	0\\
51.79	0\\
51.8	0\\
51.81	0\\
51.82	0\\
51.83	0\\
51.84	0\\
51.85	0\\
51.86	0\\
51.87	0\\
51.88	0\\
51.89	0\\
51.9	0\\
51.91	0\\
51.92	0\\
51.93	0\\
51.94	0\\
51.95	0\\
51.96	0\\
51.97	0\\
51.98	0\\
51.99	0\\
52	0\\
52.01	0\\
52.02	0\\
52.03	0\\
52.04	0\\
52.05	0\\
52.06	0\\
52.07	0\\
52.08	0\\
52.09	0\\
52.1	0\\
52.11	0\\
52.12	0\\
52.13	0\\
52.14	0\\
52.15	0\\
52.16	0\\
52.17	0\\
52.18	0\\
52.19	0\\
52.2	0\\
52.21	0\\
52.22	0\\
52.23	0\\
52.24	0\\
52.25	0\\
52.26	0\\
52.27	0\\
52.28	0\\
52.29	0\\
52.3	0\\
52.31	0\\
52.32	0\\
52.33	0\\
52.34	0\\
52.35	0\\
52.36	0\\
52.37	0\\
52.38	0\\
52.39	0\\
52.4	0\\
52.41	0\\
52.42	0\\
52.43	0\\
52.44	0\\
52.45	0\\
52.46	0\\
52.47	0\\
52.48	0\\
52.49	0\\
52.5	0\\
52.51	0\\
52.52	0\\
52.53	0\\
52.54	0\\
52.55	0\\
52.56	0\\
52.57	0\\
52.58	0\\
52.59	0\\
52.6	0\\
52.61	0\\
52.62	0\\
52.63	0\\
52.64	0\\
52.65	0\\
52.66	0\\
52.67	0\\
52.68	0\\
52.69	0\\
52.7	0\\
52.71	0\\
52.72	0\\
52.73	0\\
52.74	0\\
52.75	0\\
52.76	0\\
52.77	0\\
52.78	0\\
52.79	0\\
52.8	0\\
52.81	0\\
52.82	0\\
52.83	0\\
52.84	0\\
52.85	0\\
52.86	0\\
52.87	0\\
52.88	0\\
52.89	0\\
52.9	0\\
52.91	0\\
52.92	0\\
52.93	0\\
52.94	0\\
52.95	0\\
52.96	0\\
52.97	0\\
52.98	0\\
52.99	0\\
53	0\\
53.01	0\\
53.02	0\\
53.03	0\\
53.04	0\\
53.05	0\\
53.06	0\\
53.07	0\\
53.08	0\\
53.09	0\\
53.1	0\\
53.11	0\\
53.12	0\\
53.13	0\\
53.14	0\\
53.15	0\\
53.16	0\\
53.17	0\\
53.18	0\\
53.19	0\\
53.2	0\\
53.21	0\\
53.22	0\\
53.23	0\\
53.24	0\\
53.25	0\\
53.26	0\\
53.27	0\\
53.28	0\\
53.29	0\\
53.3	0\\
53.31	0\\
53.32	0\\
53.33	0\\
53.34	0\\
53.35	0\\
53.36	0\\
53.37	0\\
53.38	0\\
53.39	0\\
53.4	0\\
53.41	0\\
53.42	0\\
53.43	0\\
53.44	0\\
53.45	0\\
53.46	0\\
53.47	0\\
53.48	0\\
53.49	0\\
53.5	0\\
53.51	0\\
53.52	0\\
53.53	0\\
53.54	0\\
53.55	0\\
53.56	0\\
53.57	0\\
53.58	0\\
53.59	0\\
53.6	0\\
53.61	0\\
53.62	0\\
53.63	0\\
53.64	0\\
53.65	0\\
53.66	0\\
53.67	0\\
53.68	0\\
53.69	0\\
53.7	0\\
53.71	0\\
53.72	0\\
53.73	0\\
53.74	0\\
53.75	0\\
53.76	0\\
53.77	0\\
53.78	0\\
53.79	0\\
53.8	0\\
53.81	0\\
53.82	0\\
53.83	0\\
53.84	0\\
53.85	0\\
53.86	0\\
53.87	0\\
53.88	0\\
53.89	0\\
53.9	0\\
53.91	0\\
53.92	0\\
53.93	0\\
53.94	0\\
53.95	0\\
53.96	0\\
53.97	0\\
53.98	0\\
53.99	0\\
54	0\\
54.01	0\\
54.02	0\\
54.03	0\\
54.04	0\\
54.05	0\\
54.06	0\\
54.07	0\\
54.08	0\\
54.09	0\\
54.1	0\\
54.11	0\\
54.12	0\\
54.13	0\\
54.14	0\\
54.15	0\\
54.16	0\\
54.17	0\\
54.18	0\\
54.19	0\\
54.2	0\\
54.21	0\\
54.22	0\\
54.23	0\\
54.24	0\\
54.25	0\\
54.26	0\\
54.27	0\\
54.28	0\\
54.29	0\\
54.3	0\\
54.31	0\\
54.32	0\\
54.33	0\\
54.34	0\\
54.35	0\\
54.36	0\\
54.37	0\\
54.38	0\\
54.39	0\\
54.4	0\\
54.41	0\\
54.42	0\\
54.43	0\\
54.44	0\\
54.45	0\\
54.46	0\\
54.47	0\\
54.48	0\\
54.49	0\\
54.5	0\\
54.51	0\\
54.52	0\\
54.53	0\\
54.54	0\\
54.55	0\\
54.56	0\\
54.57	0\\
54.58	0\\
54.59	0\\
54.6	0\\
54.61	0\\
54.62	0\\
54.63	0\\
54.64	0\\
54.65	0\\
54.66	0\\
54.67	0\\
54.68	0\\
54.69	0\\
54.7	0\\
54.71	0\\
54.72	0\\
54.73	0\\
54.74	0\\
54.75	0\\
54.76	0\\
54.77	0\\
54.78	0\\
54.79	0\\
54.8	0\\
54.81	0\\
54.82	0\\
54.83	0\\
54.84	0\\
54.85	0\\
54.86	0\\
54.87	0\\
54.88	0\\
54.89	0\\
54.9	0\\
54.91	0\\
54.92	0\\
54.93	0\\
54.94	0\\
54.95	0\\
54.96	0\\
54.97	0\\
54.98	0\\
54.99	0\\
55	0\\
55.01	0\\
55.02	0\\
55.03	0\\
55.04	0\\
55.05	0\\
55.06	0\\
55.07	0\\
55.08	0\\
55.09	0\\
55.1	0\\
55.11	0\\
55.12	0\\
55.13	0\\
55.14	0\\
55.15	0\\
55.16	0\\
55.17	0\\
55.18	0\\
55.19	0\\
55.2	0\\
55.21	0\\
55.22	0\\
55.23	0\\
55.24	0\\
55.25	0\\
55.26	0\\
55.27	0\\
55.28	0\\
55.29	0\\
55.3	0\\
55.31	0\\
55.32	0\\
55.33	0\\
55.34	0\\
55.35	0\\
55.36	0\\
55.37	0\\
55.38	0\\
55.39	0\\
55.4	0\\
55.41	0\\
55.42	0\\
55.43	0\\
55.44	0\\
55.45	0\\
55.46	0\\
55.47	0\\
55.48	0\\
55.49	0\\
55.5	0\\
55.51	0\\
55.52	0\\
55.53	0\\
55.54	0\\
55.55	0\\
55.56	0\\
55.57	0\\
55.58	0\\
55.59	0\\
55.6	0\\
55.61	0\\
55.62	0\\
55.63	0\\
55.64	0\\
55.65	0\\
55.66	0\\
55.67	0\\
55.68	0\\
55.69	0\\
55.7	0\\
55.71	0\\
55.72	0\\
55.73	0\\
55.74	0\\
55.75	0\\
55.76	0\\
55.77	0\\
55.78	0\\
55.79	0\\
55.8	0\\
55.81	0\\
55.82	0\\
55.83	0\\
55.84	0\\
55.85	0\\
55.86	0\\
55.87	0\\
55.88	0\\
55.89	0\\
55.9	0\\
55.91	0\\
55.92	0\\
55.93	0\\
55.94	0\\
55.95	0\\
55.96	0\\
55.97	0\\
55.98	0\\
55.99	0\\
56	0\\
56.01	0\\
56.02	0\\
56.03	0\\
56.04	0\\
56.05	0\\
56.06	0\\
56.07	0\\
56.08	0\\
56.09	0\\
56.1	0\\
56.11	0\\
56.12	0\\
56.13	0\\
56.14	0\\
56.15	0\\
56.16	0\\
56.17	0\\
56.18	0\\
56.19	0\\
56.2	0\\
56.21	0\\
56.22	0\\
56.23	0\\
56.24	0\\
56.25	0\\
56.26	0\\
56.27	0\\
56.28	0\\
56.29	0\\
56.3	0\\
56.31	0\\
56.32	0\\
56.33	0\\
56.34	0\\
56.35	0\\
56.36	0\\
56.37	0\\
56.38	0\\
56.39	0\\
56.4	0\\
56.41	0\\
56.42	0\\
56.43	0\\
56.44	0\\
56.45	0\\
56.46	0\\
56.47	0\\
56.48	0\\
56.49	0\\
56.5	0\\
56.51	0\\
56.52	0\\
56.53	0\\
56.54	0\\
56.55	0\\
56.56	0\\
56.57	0\\
56.58	0\\
56.59	0\\
56.6	0\\
56.61	0\\
56.62	0\\
56.63	0\\
56.64	0\\
56.65	0\\
56.66	0\\
56.67	0\\
56.68	0\\
56.69	0\\
56.7	0\\
56.71	0\\
56.72	0\\
56.73	0\\
56.74	0\\
56.75	0\\
56.76	0\\
56.77	0\\
56.78	0\\
56.79	0\\
56.8	0\\
56.81	0\\
56.82	0\\
56.83	0\\
56.84	0\\
56.85	0\\
56.86	0\\
56.87	0\\
56.88	0\\
56.89	0\\
56.9	0\\
56.91	0\\
56.92	0\\
56.93	0\\
56.94	0\\
56.95	0\\
56.96	0\\
56.97	0\\
56.98	0\\
56.99	0\\
57	0\\
57.01	0\\
57.02	0\\
57.03	0\\
57.04	0\\
57.05	0\\
57.06	0\\
57.07	0\\
57.08	0\\
57.09	0\\
57.1	0\\
57.11	0\\
57.12	0\\
57.13	0\\
57.14	0\\
57.15	0\\
57.16	0\\
57.17	0\\
57.18	0\\
57.19	0\\
57.2	0\\
57.21	0\\
57.22	0\\
57.23	0\\
57.24	0\\
57.25	0\\
57.26	0\\
57.27	0\\
57.28	0\\
57.29	0\\
57.3	0\\
57.31	0\\
57.32	0\\
57.33	0\\
57.34	0\\
57.35	0\\
57.36	0\\
57.37	0\\
57.38	0\\
57.39	0\\
57.4	0\\
57.41	0\\
57.42	0\\
57.43	0\\
57.44	0\\
57.45	0\\
57.46	0\\
57.47	0\\
57.48	0\\
57.49	0\\
57.5	0\\
57.51	0\\
57.52	0\\
57.53	0\\
57.54	0\\
57.55	0\\
57.56	0\\
57.57	0\\
57.58	0\\
57.59	0\\
57.6	0\\
57.61	0\\
57.62	0\\
57.63	0\\
57.64	0\\
57.65	0\\
57.66	0\\
57.67	0\\
57.68	0\\
57.69	0\\
57.7	0\\
57.71	0\\
57.72	0\\
57.73	0\\
57.74	0\\
57.75	0\\
57.76	0\\
57.77	0\\
57.78	0\\
57.79	0\\
57.8	0\\
57.81	0\\
57.82	0\\
57.83	0\\
57.84	0\\
57.85	0\\
57.86	0\\
57.87	0\\
57.88	0\\
57.89	0\\
57.9	0\\
57.91	0\\
57.92	0\\
57.93	0\\
57.94	0\\
57.95	0\\
57.96	0\\
57.97	0\\
57.98	0\\
57.99	0\\
58	0\\
58.01	0\\
58.02	0\\
58.03	0\\
58.04	0\\
58.05	0\\
58.06	0\\
58.07	0\\
58.08	0\\
58.09	0\\
58.1	0\\
58.11	0\\
58.12	0\\
58.13	0\\
58.14	0\\
58.15	0\\
58.16	0\\
58.17	0\\
58.18	0\\
58.19	0\\
58.2	0\\
58.21	0\\
58.22	0\\
58.23	0\\
58.24	0\\
58.25	0\\
58.26	0\\
58.27	0\\
58.28	0\\
58.29	0\\
58.3	0\\
58.31	0\\
58.32	0\\
58.33	0\\
58.34	0\\
58.35	0\\
58.36	0\\
58.37	0\\
58.38	0\\
58.39	0\\
58.4	0\\
58.41	0\\
58.42	0\\
58.43	0\\
58.44	0\\
58.45	0\\
58.46	0\\
58.47	0\\
58.48	0\\
58.49	0\\
58.5	0\\
58.51	0\\
58.52	0\\
58.53	0\\
58.54	0\\
58.55	0\\
58.56	0\\
58.57	0\\
58.58	0\\
58.59	0\\
58.6	0\\
58.61	0\\
58.62	0\\
58.63	0\\
58.64	0\\
58.65	0\\
58.66	0\\
58.67	0\\
58.68	0\\
58.69	0\\
58.7	0\\
58.71	0\\
58.72	0\\
58.73	0\\
58.74	0\\
58.75	0\\
58.76	0\\
58.77	0\\
58.78	0\\
58.79	0\\
58.8	0\\
58.81	0\\
58.82	0\\
58.83	0\\
58.84	0\\
58.85	0\\
58.86	0\\
58.87	0\\
58.88	0\\
58.89	0\\
58.9	0\\
58.91	0\\
58.92	0\\
58.93	0\\
58.94	0\\
58.95	0\\
58.96	0\\
58.97	0\\
58.98	0\\
58.99	0\\
59	0\\
59.01	0\\
59.02	0\\
59.03	0\\
59.04	0\\
59.05	0\\
59.06	0\\
59.07	0\\
59.08	0\\
59.09	0\\
59.1	0\\
59.11	0\\
59.12	0\\
59.13	0\\
59.14	0\\
59.15	0\\
59.16	0\\
59.17	0\\
59.18	0\\
59.19	0\\
59.2	0\\
59.21	0\\
59.22	0\\
59.23	0\\
59.24	0\\
59.25	0\\
59.26	0\\
59.27	0\\
59.28	0\\
59.29	0\\
59.3	0\\
59.31	0\\
59.32	0\\
59.33	0\\
59.34	0\\
59.35	0\\
59.36	0\\
59.37	0\\
59.38	0\\
59.39	0\\
59.4	0\\
59.41	0\\
59.42	0\\
59.43	0\\
59.44	0\\
59.45	0\\
59.46	0\\
59.47	0\\
59.48	0\\
59.49	0\\
59.5	0\\
59.51	0\\
59.52	0\\
59.53	0\\
59.54	0\\
59.55	0\\
59.56	0\\
59.57	0\\
59.58	0\\
59.59	0\\
59.6	0\\
59.61	0\\
59.62	0\\
59.63	0\\
59.64	0\\
59.65	0\\
59.66	0\\
59.67	0\\
59.68	0\\
59.69	0\\
59.7	0\\
59.71	0\\
59.72	0\\
59.73	0\\
59.74	0\\
59.75	0\\
59.76	0\\
59.77	0\\
59.78	0\\
59.79	0\\
59.8	0\\
59.81	0\\
59.82	0\\
59.83	0\\
59.84	0\\
59.85	0\\
59.86	0\\
59.87	0\\
59.88	0\\
59.89	0\\
59.9	0\\
59.91	0\\
59.92	0\\
59.93	0\\
59.94	0\\
59.95	0\\
59.96	0\\
59.97	0\\
59.98	0\\
59.99	0\\
60	0\\
60.01	0\\
60.02	0\\
60.03	0\\
60.04	0\\
60.05	0\\
60.06	0\\
60.07	0\\
60.08	0\\
60.09	0\\
60.1	0\\
60.11	0\\
60.12	0\\
60.13	0\\
60.14	0\\
60.15	0\\
60.16	0\\
60.17	0\\
60.18	0\\
60.19	0\\
60.2	0\\
60.21	0\\
60.22	0\\
60.23	0\\
60.24	0\\
60.25	0\\
60.26	0\\
60.27	0\\
60.28	0\\
60.29	0\\
60.3	0\\
60.31	0\\
60.32	0\\
60.33	0\\
60.34	0\\
60.35	0\\
60.36	0\\
60.37	0\\
60.38	0\\
60.39	0\\
60.4	0\\
60.41	0\\
60.42	0\\
60.43	0\\
60.44	0\\
60.45	0\\
60.46	0\\
60.47	0\\
60.48	0\\
60.49	0\\
60.5	0\\
60.51	0\\
60.52	0\\
60.53	0\\
60.54	0\\
60.55	0\\
60.56	0\\
60.57	0\\
60.58	0\\
60.59	0\\
60.6	0\\
60.61	0\\
60.62	0\\
60.63	0\\
60.64	0\\
60.65	0\\
60.66	0\\
60.67	0\\
60.68	0\\
60.69	0\\
60.7	0\\
60.71	0\\
60.72	0\\
60.73	0\\
60.74	0\\
60.75	0\\
60.76	0\\
60.77	0\\
60.78	0\\
60.79	0\\
60.8	0\\
60.81	0\\
60.82	0\\
60.83	0\\
60.84	0\\
60.85	0\\
60.86	0\\
60.87	0\\
60.88	0\\
60.89	0\\
60.9	0\\
60.91	0\\
60.92	0\\
60.93	0\\
60.94	0\\
60.95	0\\
60.96	0\\
60.97	0\\
60.98	0\\
60.99	0\\
61	0\\
61.01	0\\
61.02	0\\
61.03	0\\
61.04	0\\
61.05	0\\
61.06	0\\
61.07	0\\
61.08	0\\
61.09	0\\
61.1	0\\
61.11	0\\
61.12	0\\
61.13	0\\
61.14	0\\
61.15	0\\
61.16	0\\
61.17	0\\
61.18	0\\
61.19	0\\
61.2	0\\
61.21	0\\
61.22	0\\
61.23	0\\
61.24	0\\
61.25	0\\
61.26	0\\
61.27	0\\
61.28	0\\
61.29	0\\
61.3	0\\
61.31	0\\
61.32	0\\
61.33	0\\
61.34	0\\
61.35	0\\
61.36	0\\
61.37	0\\
61.38	0\\
61.39	0\\
61.4	0\\
61.41	0\\
61.42	0\\
61.43	0\\
61.44	0\\
61.45	0\\
61.46	0\\
61.47	0\\
61.48	0\\
61.49	0\\
61.5	0\\
61.51	0\\
61.52	0\\
61.53	0\\
61.54	0\\
61.55	0\\
61.56	0\\
61.57	0\\
61.58	0\\
61.59	0\\
61.6	0\\
61.61	0\\
61.62	0\\
61.63	0\\
61.64	0\\
61.65	0\\
61.66	0\\
61.67	0\\
61.68	0\\
61.69	0\\
61.7	0\\
61.71	0\\
61.72	0\\
61.73	0\\
61.74	0\\
61.75	0\\
61.76	0\\
61.77	0\\
61.78	0\\
61.79	0\\
61.8	0\\
61.81	0\\
61.82	0\\
61.83	0\\
61.84	0\\
61.85	0\\
61.86	0\\
61.87	0\\
61.88	0\\
61.89	0\\
61.9	0\\
61.91	0\\
61.92	0\\
61.93	0\\
61.94	0\\
61.95	0\\
61.96	0\\
61.97	0\\
61.98	0\\
61.99	0\\
62	0\\
62.01	0\\
62.02	0\\
62.03	0\\
62.04	0\\
62.05	0\\
62.06	0\\
62.07	0\\
62.08	0\\
62.09	0\\
62.1	0\\
62.11	0\\
62.12	0\\
62.13	0\\
62.14	0\\
62.15	0\\
62.16	0\\
62.17	0\\
62.18	0\\
62.19	0\\
62.2	0\\
62.21	0\\
62.22	0\\
62.23	0\\
62.24	0\\
62.25	0\\
62.26	0\\
62.27	0\\
62.28	0\\
62.29	0\\
62.3	0\\
62.31	0\\
62.32	0\\
62.33	0\\
62.34	0\\
62.35	0\\
62.36	0\\
62.37	0\\
62.38	0\\
62.39	0\\
62.4	0\\
62.41	0\\
62.42	0\\
62.43	0\\
62.44	0\\
62.45	0\\
62.46	0\\
62.47	0\\
62.48	0\\
62.49	0\\
62.5	0\\
62.51	0\\
62.52	0\\
62.53	0\\
62.54	0\\
62.55	0\\
62.56	0\\
62.57	0\\
62.58	0\\
62.59	0\\
62.6	0\\
62.61	0\\
62.62	0\\
62.63	0\\
62.64	0\\
62.65	0\\
62.66	0\\
62.67	0\\
62.68	0\\
62.69	0\\
62.7	0\\
62.71	0\\
62.72	0\\
62.73	0\\
62.74	0\\
62.75	0\\
62.76	0\\
62.77	0\\
62.78	0\\
62.79	0\\
62.8	0\\
62.81	0\\
62.82	0\\
62.83	0\\
62.84	0\\
62.85	0\\
62.86	0\\
62.87	0\\
62.88	0\\
62.89	0\\
62.9	0\\
62.91	0\\
62.92	0\\
62.93	0\\
62.94	0\\
62.95	0\\
62.96	0\\
62.97	0\\
62.98	0\\
62.99	0\\
63	0\\
63.01	0\\
63.02	0\\
63.03	0\\
63.04	0\\
63.05	0\\
63.06	0\\
63.07	0\\
63.08	0\\
63.09	0\\
63.1	0\\
63.11	0\\
63.12	0\\
63.13	0\\
63.14	0\\
63.15	0\\
63.16	0\\
63.17	0\\
63.18	0\\
63.19	0\\
63.2	0\\
63.21	0\\
63.22	0\\
63.23	0\\
63.24	0\\
63.25	0\\
63.26	0\\
63.27	0\\
63.28	0\\
63.29	0\\
63.3	0\\
63.31	0\\
63.32	0\\
63.33	0\\
63.34	0\\
63.35	0\\
63.36	0\\
63.37	0\\
63.38	0\\
63.39	0\\
63.4	0\\
63.41	0\\
63.42	0\\
63.43	0\\
63.44	0\\
63.45	0\\
63.46	0\\
63.47	0\\
63.48	0\\
63.49	0\\
63.5	0\\
63.51	0\\
63.52	0\\
63.53	0\\
63.54	0\\
63.55	0\\
63.56	0\\
63.57	0\\
63.58	0\\
63.59	0\\
63.6	0\\
63.61	0\\
63.62	0\\
63.63	0\\
63.64	0\\
63.65	0\\
63.66	0\\
63.67	0\\
63.68	0\\
63.69	0\\
63.7	0\\
63.71	0\\
63.72	0\\
63.73	0\\
63.74	0\\
63.75	0\\
63.76	0\\
63.77	0\\
63.78	0\\
63.79	0\\
63.8	0\\
63.81	0\\
63.82	0\\
63.83	0\\
63.84	0\\
63.85	0\\
63.86	0\\
63.87	0\\
63.88	0\\
63.89	0\\
63.9	0\\
63.91	0\\
63.92	0\\
63.93	0\\
63.94	0\\
63.95	0\\
63.96	0\\
63.97	0\\
63.98	0\\
63.99	0\\
64	0\\
64.01	0\\
64.02	0\\
64.03	0\\
64.04	0\\
64.05	0\\
64.06	0\\
64.07	0\\
64.08	0\\
64.09	0\\
64.1	0\\
64.11	0\\
64.12	0\\
64.13	0\\
64.14	0\\
64.15	0\\
64.16	0\\
64.17	0\\
64.18	0\\
64.19	0\\
64.2	0\\
64.21	0\\
64.22	0\\
64.23	0\\
64.24	0\\
64.25	0\\
64.26	0\\
64.27	0\\
64.28	0\\
64.29	0\\
64.3	0\\
64.31	0\\
64.32	0\\
64.33	0\\
64.34	0\\
64.35	0\\
64.36	0\\
64.37	0\\
64.38	0\\
64.39	0\\
64.4	0\\
64.41	0\\
64.42	0\\
64.43	0\\
64.44	0\\
64.45	0\\
64.46	0\\
64.47	0\\
64.48	0\\
64.49	0\\
64.5	0\\
64.51	0\\
64.52	0\\
64.53	0\\
64.54	0\\
64.55	0\\
64.56	0\\
64.57	0\\
64.58	0\\
64.59	0\\
64.6	0\\
64.61	0\\
64.62	0\\
64.63	0\\
64.64	0\\
64.65	0\\
64.66	0\\
64.67	0\\
64.68	0\\
64.69	0\\
64.7	0\\
64.71	0\\
64.72	0\\
64.73	0\\
64.74	0\\
64.75	0\\
64.76	0\\
64.77	0\\
64.78	0\\
64.79	0\\
64.8	0\\
64.81	0\\
64.82	0\\
64.83	0\\
64.84	0\\
64.85	0\\
64.86	0\\
64.87	0\\
64.88	0\\
64.89	0\\
64.9	0\\
64.91	0\\
64.92	0\\
64.93	0\\
64.94	0\\
64.95	0\\
64.96	0\\
64.97	0\\
64.98	0\\
64.99	0\\
65	0\\
65.01	0\\
65.02	0\\
65.03	0\\
65.04	0\\
65.05	0\\
65.06	0\\
65.07	0\\
65.08	0\\
65.09	0\\
65.1	0\\
65.11	0\\
65.12	0\\
65.13	0\\
65.14	0\\
65.15	0\\
65.16	0\\
65.17	0\\
65.18	0\\
65.19	0\\
65.2	0\\
65.21	0\\
65.22	0\\
65.23	0\\
65.24	0\\
65.25	0\\
65.26	0\\
65.27	0\\
65.28	0\\
65.29	0\\
65.3	0\\
65.31	0\\
65.32	0\\
65.33	0\\
65.34	0\\
65.35	0\\
65.36	0\\
65.37	0\\
65.38	0\\
65.39	0\\
65.4	0\\
65.41	0\\
65.42	0\\
65.43	0\\
65.44	0\\
65.45	0\\
65.46	0\\
65.47	0\\
65.48	0\\
65.49	0\\
65.5	0\\
65.51	0\\
65.52	0\\
65.53	0\\
65.54	0\\
65.55	0\\
65.56	0\\
65.57	0\\
65.58	0\\
65.59	0\\
65.6	0\\
65.61	0\\
65.62	0\\
65.63	0\\
65.64	0\\
65.65	0\\
65.66	0\\
65.67	0\\
65.68	0\\
65.69	0\\
65.7	0\\
65.71	0\\
65.72	0\\
65.73	0\\
65.74	0\\
65.75	0\\
65.76	0\\
65.77	0\\
65.78	0\\
65.79	0\\
65.8	0\\
65.81	0\\
65.82	0\\
65.83	0\\
65.84	0\\
65.85	0\\
65.86	0\\
65.87	0\\
65.88	0\\
65.89	0\\
65.9	0\\
65.91	0\\
65.92	0\\
65.93	0\\
65.94	0\\
65.95	0\\
65.96	0\\
65.97	0\\
65.98	0\\
65.99	0\\
66	0\\
66.01	0\\
66.02	0\\
66.03	0\\
66.04	0\\
66.05	0\\
66.06	0\\
66.07	0\\
66.08	0\\
66.09	0\\
66.1	0\\
66.11	0\\
66.12	0\\
66.13	0\\
66.14	0\\
66.15	0\\
66.16	0\\
66.17	0\\
66.18	0\\
66.19	0\\
66.2	0\\
66.21	0\\
66.22	0\\
66.23	0\\
66.24	0\\
66.25	0\\
66.26	0\\
66.27	0\\
66.28	0\\
66.29	0\\
66.3	0\\
66.31	0\\
66.32	0\\
66.33	0\\
66.34	0\\
66.35	0\\
66.36	0\\
66.37	0\\
66.38	0\\
66.39	0\\
66.4	0\\
66.41	0\\
66.42	0\\
66.43	0\\
66.44	0\\
66.45	0\\
66.46	0\\
66.47	0\\
66.48	0\\
66.49	0\\
66.5	0\\
66.51	0\\
66.52	0\\
66.53	0\\
66.54	0\\
66.55	0\\
66.56	0\\
66.57	0\\
66.58	0\\
66.59	0\\
66.6	0\\
66.61	0\\
66.62	0\\
66.63	0\\
66.64	0\\
66.65	0\\
66.66	0\\
66.67	0\\
66.68	0\\
66.69	0\\
66.7	0\\
66.71	0\\
66.72	0\\
66.73	0\\
66.74	0\\
66.75	0\\
66.76	0\\
66.77	0\\
66.78	0\\
66.79	0\\
66.8	0\\
66.81	0\\
66.82	0\\
66.83	0\\
66.84	0\\
66.85	0\\
66.86	0\\
66.87	0\\
66.88	0\\
66.89	0\\
66.9	0\\
66.91	0\\
66.92	0\\
66.93	0\\
66.94	0\\
66.95	0\\
66.96	0\\
66.97	0\\
66.98	0\\
66.99	0\\
67	0\\
67.01	0\\
67.02	0\\
67.03	0\\
67.04	0\\
67.05	0\\
67.06	0\\
67.07	0\\
67.08	0\\
67.09	0\\
67.1	0\\
67.11	0\\
67.12	0\\
67.13	0\\
67.14	0\\
67.15	0\\
67.16	0\\
67.17	0\\
67.18	0\\
67.19	0\\
67.2	0\\
67.21	0\\
67.22	0\\
67.23	0\\
67.24	0\\
67.25	0\\
67.26	0\\
67.27	0\\
67.28	0\\
67.29	0\\
67.3	0\\
67.31	0\\
67.32	0\\
67.33	0\\
67.34	0\\
67.35	0\\
67.36	0\\
67.37	0\\
67.38	0\\
67.39	0\\
67.4	0\\
67.41	0\\
67.42	0\\
67.43	0\\
67.44	0\\
67.45	0\\
67.46	0\\
67.47	0\\
67.48	0\\
67.49	0\\
67.5	0\\
67.51	0\\
67.52	0\\
67.53	0\\
67.54	0\\
67.55	0\\
67.56	0\\
67.57	0\\
67.58	0\\
67.59	0\\
67.6	0\\
67.61	0\\
67.62	0\\
67.63	0\\
67.64	0\\
67.65	0\\
67.66	0\\
67.67	0\\
67.68	0\\
67.69	0\\
67.7	0\\
67.71	0\\
67.72	0\\
67.73	0\\
67.74	0\\
67.75	0\\
67.76	0\\
67.77	0\\
67.78	0\\
67.79	0\\
67.8	0\\
67.81	0\\
67.82	0\\
67.83	0\\
67.84	0\\
67.85	0\\
67.86	0\\
67.87	0\\
67.88	0\\
67.89	0\\
67.9	0\\
67.91	0\\
67.92	0\\
67.93	0\\
67.94	0\\
67.95	0\\
67.96	0\\
67.97	0\\
67.98	0\\
67.99	0\\
68	0\\
68.01	0\\
68.02	0\\
68.03	0\\
68.04	0\\
68.05	0\\
68.06	0\\
68.07	0\\
68.08	0\\
68.09	0\\
68.1	0\\
68.11	0\\
68.12	0\\
68.13	0\\
68.14	0\\
68.15	0\\
68.16	0\\
68.17	0\\
68.18	0\\
68.19	0\\
68.2	0\\
68.21	0\\
68.22	0\\
68.23	0\\
68.24	0\\
68.25	0\\
68.26	0\\
68.27	0\\
68.28	0\\
68.29	0\\
68.3	0\\
68.31	0\\
68.32	0\\
68.33	0\\
68.34	0\\
68.35	0\\
68.36	0\\
68.37	0\\
68.38	0\\
68.39	0\\
68.4	0\\
68.41	0\\
68.42	0\\
68.43	0\\
68.44	0\\
68.45	0\\
68.46	0\\
68.47	0\\
68.48	0\\
68.49	0\\
68.5	0\\
68.51	0\\
68.52	0\\
68.53	0\\
68.54	0\\
68.55	0\\
68.56	0\\
68.57	0\\
68.58	0\\
68.59	0\\
68.6	0\\
68.61	0\\
68.62	0\\
68.63	0\\
68.64	0\\
68.65	0\\
68.66	0\\
68.67	0\\
68.68	0\\
68.69	0\\
68.7	0\\
68.71	0\\
68.72	0\\
68.73	0\\
68.74	0\\
68.75	0\\
68.76	0\\
68.77	0\\
68.78	0\\
68.79	0\\
68.8	0\\
68.81	0\\
68.82	0\\
68.83	0\\
68.84	0\\
68.85	0\\
68.86	0\\
68.87	0\\
68.88	0\\
68.89	0\\
68.9	0\\
68.91	0\\
68.92	0\\
68.93	0\\
68.94	0\\
68.95	0\\
68.96	0\\
68.97	0\\
68.98	0\\
68.99	0\\
69	0\\
69.01	0\\
69.02	0\\
69.03	0\\
69.04	0\\
69.05	0\\
69.06	0\\
69.07	0\\
69.08	0\\
69.09	0\\
69.1	0\\
69.11	0\\
69.12	0\\
69.13	0\\
69.14	0\\
69.15	0\\
69.16	0\\
69.17	0\\
69.18	0\\
69.19	0\\
69.2	0\\
69.21	0\\
69.22	0\\
69.23	0\\
69.24	0\\
69.25	0\\
69.26	0\\
69.27	0\\
69.28	0\\
69.29	0\\
69.3	0\\
69.31	0\\
69.32	0\\
69.33	0\\
69.34	0\\
69.35	0\\
69.36	0\\
69.37	0\\
69.38	0\\
69.39	0\\
69.4	0\\
69.41	0\\
69.42	0\\
69.43	0\\
69.44	0\\
69.45	0\\
69.46	0\\
69.47	0\\
69.48	0\\
69.49	0\\
69.5	0\\
69.51	0\\
69.52	0\\
69.53	0\\
69.54	0\\
69.55	0\\
69.56	0\\
69.57	0\\
69.58	0\\
69.59	0\\
69.6	0\\
69.61	0\\
69.62	0\\
69.63	0\\
69.64	0\\
69.65	0\\
69.66	0\\
69.67	0\\
69.68	0\\
69.69	0\\
69.7	0\\
69.71	0\\
69.72	0\\
69.73	0\\
69.74	0\\
69.75	0\\
69.76	0\\
69.77	0\\
69.78	0\\
69.79	0\\
69.8	0\\
69.81	0\\
69.82	0\\
69.83	0\\
69.84	0\\
69.85	0\\
69.86	0\\
69.87	0\\
69.88	0\\
69.89	0\\
69.9	0\\
69.91	0\\
69.92	0\\
69.93	0\\
69.94	0\\
69.95	0\\
69.96	0\\
69.97	0\\
69.98	0\\
69.99	0\\
70	0\\
70.01	0\\
70.02	0\\
70.03	0\\
70.04	0\\
70.05	0\\
70.06	0\\
70.07	0\\
70.08	0\\
70.09	0\\
70.1	0\\
70.11	0\\
70.12	0\\
70.13	0\\
70.14	0\\
70.15	0\\
70.16	0\\
70.17	0\\
70.18	0\\
70.19	0\\
70.2	0\\
70.21	0\\
70.22	0\\
70.23	0\\
70.24	0\\
70.25	0\\
70.26	0\\
70.27	0\\
70.28	0\\
70.29	0\\
70.3	0\\
70.31	0\\
70.32	0\\
70.33	0\\
70.34	0\\
70.35	0\\
70.36	0\\
70.37	0\\
70.38	0\\
70.39	0\\
70.4	0\\
70.41	0\\
70.42	0\\
70.43	0\\
70.44	0\\
70.45	0\\
70.46	0\\
70.47	0\\
70.48	0\\
70.49	0\\
70.5	0\\
70.51	0\\
70.52	0\\
70.53	0\\
70.54	0\\
70.55	0\\
70.56	0\\
70.57	0\\
70.58	0\\
70.59	0\\
70.6	0\\
70.61	0\\
70.62	0\\
70.63	0\\
70.64	0\\
70.65	0\\
70.66	0\\
70.67	0\\
70.68	0\\
70.69	0\\
70.7	0\\
70.71	0\\
70.72	0\\
70.73	0\\
70.74	0\\
70.75	0\\
70.76	0\\
70.77	0\\
70.78	0\\
70.79	0\\
70.8	0\\
70.81	0\\
70.82	0\\
70.83	0\\
70.84	0\\
70.85	0\\
70.86	0\\
70.87	0\\
70.88	0\\
70.89	0\\
70.9	0\\
70.91	0\\
70.92	0\\
70.93	0\\
70.94	0\\
70.95	0\\
70.96	0\\
70.97	0\\
70.98	0\\
70.99	0\\
71	0\\
71.01	0\\
71.02	0\\
71.03	0\\
71.04	0\\
71.05	0\\
71.06	0\\
71.07	0\\
71.08	0\\
71.09	0\\
71.1	0\\
71.11	0\\
71.12	0\\
71.13	0\\
71.14	0\\
71.15	0\\
71.16	0\\
71.17	0\\
71.18	0\\
71.19	0\\
71.2	0\\
71.21	0\\
71.22	0\\
71.23	0\\
71.24	0\\
71.25	0\\
71.26	0\\
71.27	0\\
71.28	0\\
71.29	0\\
71.3	0\\
71.31	0\\
71.32	0\\
71.33	0\\
71.34	0\\
71.35	0\\
71.36	0\\
71.37	0\\
71.38	0\\
71.39	0\\
71.4	0\\
71.41	0\\
71.42	0\\
71.43	0\\
71.44	0\\
71.45	0\\
71.46	0\\
71.47	0\\
71.48	0\\
71.49	0\\
71.5	0\\
71.51	0\\
71.52	0\\
71.53	0\\
71.54	0\\
71.55	0\\
71.56	0\\
71.57	0\\
71.58	0\\
71.59	0\\
71.6	0\\
71.61	0\\
71.62	0\\
71.63	0\\
71.64	0\\
71.65	0\\
71.66	0\\
71.67	0\\
71.68	0\\
71.69	0\\
71.7	0\\
71.71	0\\
71.72	0\\
71.73	0\\
71.74	0\\
71.75	0\\
71.76	0\\
71.77	0\\
71.78	0\\
71.79	0\\
71.8	0\\
71.81	0\\
71.82	0\\
71.83	0\\
71.84	0\\
71.85	0\\
71.86	0\\
71.87	0\\
71.88	0\\
71.89	0\\
71.9	0\\
71.91	0\\
71.92	0\\
71.93	0\\
71.94	0\\
71.95	0\\
71.96	0\\
71.97	0\\
71.98	0\\
71.99	0\\
72	0\\
72.01	0\\
72.02	0\\
72.03	0\\
72.04	0\\
72.05	0\\
72.06	0\\
72.07	0\\
72.08	0\\
72.09	0\\
72.1	0\\
72.11	0\\
72.12	0\\
72.13	0\\
72.14	0\\
72.15	0\\
72.16	0\\
72.17	0\\
72.18	0\\
72.19	0\\
72.2	0\\
72.21	0\\
72.22	0\\
72.23	0\\
72.24	0\\
72.25	0\\
72.26	0\\
72.27	0\\
72.28	0\\
72.29	0\\
72.3	0\\
72.31	0\\
72.32	0\\
72.33	0\\
72.34	0\\
72.35	0\\
72.36	0\\
72.37	0\\
72.38	0\\
72.39	0\\
72.4	0\\
72.41	0\\
72.42	0\\
72.43	0\\
72.44	0\\
72.45	0\\
72.46	0\\
72.47	0\\
72.48	0\\
72.49	0\\
72.5	0\\
72.51	0\\
72.52	0\\
72.53	0\\
72.54	0\\
72.55	0\\
72.56	0\\
72.57	0\\
72.58	0\\
72.59	0\\
72.6	0\\
72.61	0\\
72.62	0\\
72.63	0\\
72.64	0\\
72.65	0\\
72.66	0\\
72.67	0\\
72.68	0\\
72.69	0\\
72.7	0\\
72.71	0\\
72.72	0\\
72.73	0\\
72.74	0\\
72.75	0\\
72.76	0\\
72.77	0\\
72.78	0\\
72.79	0\\
72.8	0\\
72.81	0\\
72.82	0\\
72.83	0\\
72.84	0\\
72.85	0\\
72.86	0\\
72.87	0\\
72.88	0\\
72.89	0\\
72.9	0\\
72.91	0\\
72.92	0\\
72.93	0\\
72.94	0\\
72.95	0\\
72.96	0\\
72.97	0\\
72.98	0\\
72.99	0\\
73	0\\
73.01	0\\
73.02	0\\
73.03	0\\
73.04	0\\
73.05	0\\
73.06	0\\
73.07	0\\
73.08	0\\
73.09	0\\
73.1	0\\
73.11	0\\
73.12	0\\
73.13	0\\
73.14	0\\
73.15	0\\
73.16	0\\
73.17	0\\
73.18	0\\
73.19	0\\
73.2	0\\
73.21	0\\
73.22	0\\
73.23	0\\
73.24	0\\
73.25	0\\
73.26	0\\
73.27	0\\
73.28	0\\
73.29	0\\
73.3	0\\
73.31	0\\
73.32	0\\
73.33	0\\
73.34	0\\
73.35	0\\
73.36	0\\
73.37	0\\
73.38	0\\
73.39	0\\
73.4	0\\
73.41	0\\
73.42	0\\
73.43	0\\
73.44	0\\
73.45	0\\
73.46	0\\
73.47	0\\
73.48	0\\
73.49	0\\
73.5	0\\
73.51	0\\
73.52	0\\
73.53	0\\
73.54	0\\
73.55	0\\
73.56	0\\
73.57	0\\
73.58	0\\
73.59	0\\
73.6	0\\
73.61	0\\
73.62	0\\
73.63	0\\
73.64	0\\
73.65	0\\
73.66	0\\
73.67	0\\
73.68	0\\
73.69	0\\
73.7	0\\
73.71	0\\
73.72	0\\
73.73	0\\
73.74	0\\
73.75	0\\
73.76	0\\
73.77	0\\
73.78	0\\
73.79	0\\
73.8	0\\
73.81	0\\
73.82	0\\
73.83	0\\
73.84	0\\
73.85	0\\
73.86	0\\
73.87	0\\
73.88	0\\
73.89	0\\
73.9	0\\
73.91	0\\
73.92	0\\
73.93	0\\
73.94	0\\
73.95	0\\
73.96	0\\
73.97	0\\
73.98	0\\
73.99	0\\
74	0\\
74.01	0\\
74.02	0\\
74.03	0\\
74.04	0\\
74.05	0\\
74.06	0\\
74.07	0\\
74.08	0\\
74.09	0\\
74.1	0\\
74.11	0\\
74.12	0\\
74.13	0\\
74.14	0\\
74.15	0\\
74.16	0\\
74.17	0\\
74.18	0\\
74.19	0\\
74.2	0\\
74.21	0\\
74.22	0\\
74.23	0\\
74.24	0\\
74.25	0\\
74.26	0\\
74.27	0\\
74.28	0\\
74.29	0\\
74.3	0\\
74.31	0\\
74.32	0\\
74.33	0\\
74.34	0\\
74.35	0\\
74.36	0\\
74.37	0\\
74.38	0\\
74.39	0\\
74.4	0\\
74.41	0\\
74.42	0\\
74.43	0\\
74.44	0\\
74.45	0\\
74.46	0\\
74.47	0\\
74.48	0\\
74.49	0\\
74.5	0\\
74.51	0\\
74.52	0\\
74.53	0\\
74.54	0\\
74.55	0\\
74.56	0\\
74.57	0\\
74.58	0\\
74.59	0\\
74.6	0\\
74.61	0\\
74.62	0\\
74.63	0\\
74.64	0\\
74.65	0\\
74.66	0\\
74.67	0\\
74.68	0\\
74.69	0\\
74.7	0\\
74.71	0\\
74.72	0\\
74.73	0\\
74.74	0\\
74.75	0\\
74.76	0\\
74.77	0\\
74.78	0\\
74.79	0\\
74.8	0\\
74.81	0\\
74.82	0\\
74.83	0\\
74.84	0\\
74.85	0\\
74.86	0\\
74.87	0\\
74.88	0\\
74.89	0\\
74.9	0\\
74.91	0\\
74.92	0\\
74.93	0\\
74.94	0\\
74.95	0\\
74.96	0\\
74.97	0\\
74.98	0\\
74.99	0\\
75	0\\
75.01	0\\
75.02	0\\
75.03	0\\
75.04	0\\
75.05	0\\
75.06	0\\
75.07	0\\
75.08	0\\
75.09	0\\
75.1	0\\
75.11	0\\
75.12	0\\
75.13	0\\
75.14	0\\
75.15	0\\
75.16	0\\
75.17	0\\
75.18	0\\
75.19	0\\
75.2	0\\
75.21	0\\
75.22	0\\
75.23	0\\
75.24	0\\
75.25	0\\
75.26	0\\
75.27	0\\
75.28	0\\
75.29	0\\
75.3	0\\
75.31	0\\
75.32	0\\
75.33	0\\
75.34	0\\
75.35	0\\
75.36	0\\
75.37	0\\
75.38	0\\
75.39	0\\
75.4	0\\
75.41	0\\
75.42	0\\
75.43	0\\
75.44	0\\
75.45	0\\
75.46	0\\
75.47	0\\
75.48	0\\
75.49	0\\
75.5	0\\
75.51	0\\
75.52	0\\
75.53	0\\
75.54	0\\
75.55	0\\
75.56	0\\
75.57	0\\
75.58	0\\
75.59	0\\
75.6	0\\
75.61	0\\
75.62	0\\
75.63	0\\
75.64	0\\
75.65	0\\
75.66	0\\
75.67	0\\
75.68	0\\
75.69	0\\
75.7	0\\
75.71	0\\
75.72	0\\
75.73	0\\
75.74	0\\
75.75	0\\
75.76	0\\
75.77	0\\
75.78	0\\
75.79	0\\
75.8	0\\
75.81	0\\
75.82	0\\
75.83	0\\
75.84	0\\
75.85	0\\
75.86	0\\
75.87	0\\
75.88	0\\
75.89	0\\
75.9	0\\
75.91	0\\
75.92	0\\
75.93	0\\
75.94	0\\
75.95	0\\
75.96	0\\
75.97	0\\
75.98	0\\
75.99	0\\
76	0\\
76.01	0\\
76.02	0\\
76.03	0\\
76.04	0\\
76.05	0\\
76.06	0\\
76.07	0\\
76.08	0\\
76.09	0\\
76.1	0\\
76.11	0\\
76.12	0\\
76.13	0\\
76.14	0\\
76.15	0\\
76.16	0\\
76.17	0\\
76.18	0\\
76.19	0\\
76.2	0\\
76.21	0\\
76.22	0\\
76.23	0\\
76.24	0\\
76.25	0\\
76.26	0\\
76.27	0\\
76.28	0\\
76.29	0\\
76.3	0\\
76.31	0\\
76.32	0\\
76.33	0\\
76.34	0\\
76.35	0\\
76.36	0\\
76.37	0\\
76.38	0\\
76.39	0\\
76.4	0\\
76.41	0\\
76.42	0\\
76.43	0\\
76.44	0\\
76.45	0\\
76.46	0\\
76.47	0\\
76.48	0\\
76.49	0\\
76.5	0\\
76.51	0\\
76.52	0\\
76.53	0\\
76.54	0\\
76.55	0\\
76.56	0\\
76.57	0\\
76.58	0\\
76.59	0\\
76.6	0\\
76.61	0\\
76.62	0\\
76.63	0\\
76.64	0\\
76.65	0\\
76.66	0\\
76.67	0\\
76.68	0\\
76.69	0\\
76.7	0\\
76.71	0\\
76.72	0\\
76.73	0\\
76.74	0\\
76.75	0\\
76.76	0\\
76.77	0\\
76.78	0\\
76.79	0\\
76.8	0\\
76.81	0\\
76.82	0\\
76.83	0\\
76.84	0\\
76.85	0\\
76.86	0\\
76.87	0\\
76.88	0\\
76.89	0\\
76.9	0\\
76.91	0\\
76.92	0\\
76.93	0\\
76.94	0\\
76.95	0\\
76.96	0\\
76.97	0\\
76.98	0\\
76.99	0\\
77	0\\
77.01	0\\
77.02	0\\
77.03	0\\
77.04	0\\
77.05	0\\
77.06	0\\
77.07	0\\
77.08	0\\
77.09	0\\
77.1	0\\
77.11	0\\
77.12	0\\
77.13	0\\
77.14	0\\
77.15	0\\
77.16	0\\
77.17	0\\
77.18	0\\
77.19	0\\
77.2	0\\
77.21	0\\
77.22	0\\
77.23	0\\
77.24	0\\
77.25	0\\
77.26	0\\
77.27	0\\
77.28	0\\
77.29	0\\
77.3	0\\
77.31	0\\
77.32	0\\
77.33	0\\
77.34	0\\
77.35	0\\
77.36	0\\
77.37	0\\
77.38	0\\
77.39	0\\
77.4	0\\
77.41	0\\
77.42	0\\
77.43	0\\
77.44	0\\
77.45	0\\
77.46	0\\
77.47	0\\
77.48	0\\
77.49	0\\
77.5	0\\
77.51	0\\
77.52	0\\
77.53	0\\
77.54	0\\
77.55	0\\
77.56	0\\
77.57	0\\
77.58	0\\
77.59	0\\
77.6	0\\
77.61	0\\
77.62	0\\
77.63	0\\
77.64	0\\
77.65	0\\
77.66	0\\
77.67	0\\
77.68	0\\
77.69	0\\
77.7	0\\
77.71	0\\
77.72	0\\
77.73	0\\
77.74	0\\
77.75	0\\
77.76	0\\
77.77	0\\
77.78	0\\
77.79	0\\
77.8	0\\
77.81	0\\
77.82	0\\
77.83	0\\
77.84	0\\
77.85	0\\
77.86	0\\
77.87	0\\
77.88	0\\
77.89	0\\
77.9	0\\
77.91	0\\
77.92	0\\
77.93	0\\
77.94	0\\
77.95	0\\
77.96	0\\
77.97	0\\
77.98	0\\
77.99	0\\
78	0\\
78.01	0\\
78.02	0\\
78.03	0\\
78.04	0\\
78.05	0\\
78.06	0\\
78.07	0\\
78.08	0\\
78.09	0\\
78.1	0\\
78.11	0\\
78.12	0\\
78.13	0\\
78.14	0\\
78.15	0\\
78.16	0\\
78.17	0\\
78.18	0\\
78.19	0\\
78.2	0\\
78.21	0\\
78.22	0\\
78.23	0\\
78.24	0\\
78.25	0\\
78.26	0\\
78.27	0\\
78.28	0\\
78.29	0\\
78.3	0\\
78.31	0\\
78.32	0\\
78.33	0\\
78.34	0\\
78.35	0\\
78.36	0\\
78.37	0\\
78.38	0\\
78.39	0\\
78.4	0\\
78.41	0\\
78.42	0\\
78.43	0\\
78.44	0\\
78.45	0\\
78.46	0\\
78.47	0\\
78.48	0\\
78.49	0\\
78.5	0\\
78.51	0\\
78.52	0\\
78.53	0\\
78.54	0\\
78.55	0\\
78.56	0\\
78.57	0\\
78.58	0\\
78.59	0\\
78.6	0\\
78.61	0\\
78.62	0\\
78.63	0\\
78.64	0\\
78.65	0\\
78.66	0\\
78.67	0\\
78.68	0\\
78.69	0\\
78.7	0\\
78.71	0\\
78.72	0\\
78.73	0\\
78.74	0\\
78.75	0\\
78.76	0\\
78.77	0\\
78.78	0\\
78.79	0\\
78.8	0\\
78.81	0\\
78.82	0\\
78.83	0\\
78.84	0\\
78.85	0\\
78.86	0\\
78.87	0\\
78.88	0\\
78.89	0\\
78.9	0\\
78.91	0\\
78.92	0\\
78.93	0\\
78.94	0\\
78.95	0\\
78.96	0\\
78.97	0\\
78.98	0\\
78.99	0\\
79	0\\
79.01	0\\
79.02	0\\
79.03	0\\
79.04	0\\
79.05	0\\
79.06	0\\
79.07	0\\
79.08	0\\
79.09	0\\
79.1	0\\
79.11	0\\
79.12	0\\
79.13	0\\
79.14	0\\
79.15	0\\
79.16	0\\
79.17	0\\
79.18	0\\
79.19	0\\
79.2	0\\
79.21	0\\
79.22	0\\
79.23	0\\
79.24	0\\
79.25	0\\
79.26	0\\
79.27	0\\
79.28	0\\
79.29	0\\
79.3	0\\
79.31	0\\
79.32	0\\
79.33	0\\
79.34	0\\
79.35	0\\
79.36	0\\
79.37	0\\
79.38	0\\
79.39	0\\
79.4	0\\
79.41	0\\
79.42	0\\
79.43	0\\
79.44	0\\
79.45	0\\
79.46	0\\
79.47	0\\
79.48	0\\
79.49	0\\
79.5	0\\
79.51	0\\
79.52	0\\
79.53	0\\
79.54	0\\
79.55	0\\
79.56	0\\
79.57	0\\
79.58	0\\
79.59	0\\
79.6	0\\
79.61	0\\
79.62	0\\
79.63	0\\
79.64	0\\
79.65	0\\
79.66	0\\
79.67	0\\
79.68	0\\
79.69	0\\
79.7	0\\
79.71	0\\
79.72	0\\
79.73	0\\
79.74	0\\
79.75	0\\
79.76	0\\
79.77	0\\
79.78	0\\
79.79	0\\
79.8	0\\
79.81	0\\
79.82	0\\
79.83	0\\
79.84	0\\
79.85	0\\
79.86	0\\
79.87	0\\
79.88	0\\
79.89	0\\
79.9	0\\
79.91	0\\
79.92	0\\
79.93	0\\
79.94	0\\
79.95	0\\
79.96	0\\
79.97	0\\
79.98	0\\
79.99	0\\
80	0\\
80.01	0\\
};
\addplot [color=black,solid]
  table[row sep=crcr]{%
80.01	0\\
80.02	0\\
80.03	0\\
80.04	0\\
80.05	0\\
80.06	0\\
80.07	0\\
80.08	0\\
80.09	0\\
80.1	0\\
80.11	0\\
80.12	0\\
80.13	0\\
80.14	0\\
80.15	0\\
80.16	0\\
80.17	0\\
80.18	0\\
80.19	0\\
80.2	0\\
80.21	0\\
80.22	0\\
80.23	0\\
80.24	0\\
80.25	0\\
80.26	0\\
80.27	0\\
80.28	0\\
80.29	0\\
80.3	0\\
80.31	0\\
80.32	0\\
80.33	0\\
80.34	0\\
80.35	0\\
80.36	0\\
80.37	0\\
80.38	0\\
80.39	0\\
80.4	0\\
80.41	0\\
80.42	0\\
80.43	0\\
80.44	0\\
80.45	0\\
80.46	0\\
80.47	0\\
80.48	0\\
80.49	0\\
80.5	0\\
80.51	0\\
80.52	0\\
80.53	0\\
80.54	0\\
80.55	0\\
80.56	0\\
80.57	0\\
80.58	0\\
80.59	0\\
80.6	0\\
80.61	0\\
80.62	0\\
80.63	0\\
80.64	0\\
80.65	0\\
80.66	0\\
80.67	0\\
80.68	0\\
80.69	0\\
80.7	0\\
80.71	0\\
80.72	0\\
80.73	0\\
80.74	0\\
80.75	0\\
80.76	0\\
80.77	0\\
80.78	0\\
80.79	0\\
80.8	0\\
80.81	0\\
80.82	0\\
80.83	0\\
80.84	0\\
80.85	0\\
80.86	0\\
80.87	0\\
80.88	0\\
80.89	0\\
80.9	0\\
80.91	0\\
80.92	0\\
80.93	0\\
80.94	0\\
80.95	0\\
80.96	0\\
80.97	0\\
80.98	0\\
80.99	0\\
81	0\\
81.01	0\\
81.02	0\\
81.03	0\\
81.04	0\\
81.05	0\\
81.06	0\\
81.07	0\\
81.08	0\\
81.09	0\\
81.1	0\\
81.11	0\\
81.12	0\\
81.13	0\\
81.14	0\\
81.15	0\\
81.16	0\\
81.17	0\\
81.18	0\\
81.19	0\\
81.2	0\\
81.21	0\\
81.22	0\\
81.23	0\\
81.24	0\\
81.25	0\\
81.26	0\\
81.27	0\\
81.28	0\\
81.29	0\\
81.3	0\\
81.31	0\\
81.32	0\\
81.33	0\\
81.34	0\\
81.35	0\\
81.36	0\\
81.37	0\\
81.38	0\\
81.39	0\\
81.4	0\\
81.41	0\\
81.42	0\\
81.43	0\\
81.44	0\\
81.45	0\\
81.46	0\\
81.47	0\\
81.48	0\\
81.49	0\\
81.5	0\\
81.51	0\\
81.52	0\\
81.53	0\\
81.54	0\\
81.55	0\\
81.56	0\\
81.57	0\\
81.58	0\\
81.59	0\\
81.6	0\\
81.61	0\\
81.62	0\\
81.63	0\\
81.64	0\\
81.65	0\\
81.66	0\\
81.67	0\\
81.68	0\\
81.69	0\\
81.7	0\\
81.71	0\\
81.72	0\\
81.73	0\\
81.74	0\\
81.75	0\\
81.76	0\\
81.77	0\\
81.78	0\\
81.79	0\\
81.8	0\\
81.81	0\\
81.82	0\\
81.83	0\\
81.84	0\\
81.85	0\\
81.86	0\\
81.87	0\\
81.88	0\\
81.89	0\\
81.9	0\\
81.91	0\\
81.92	0\\
81.93	0\\
81.94	0\\
81.95	0\\
81.96	0\\
81.97	0\\
81.98	0\\
81.99	0\\
82	0\\
82.01	0\\
82.02	0\\
82.03	0\\
82.04	0\\
82.05	0\\
82.06	0\\
82.07	0\\
82.08	0\\
82.09	0\\
82.1	0\\
82.11	0\\
82.12	0\\
82.13	0\\
82.14	0\\
82.15	0\\
82.16	0\\
82.17	0\\
82.18	0\\
82.19	0\\
82.2	0\\
82.21	0\\
82.22	0\\
82.23	0\\
82.24	0\\
82.25	0\\
82.26	0\\
82.27	0\\
82.28	0\\
82.29	0\\
82.3	0\\
82.31	0\\
82.32	0\\
82.33	0\\
82.34	0\\
82.35	0\\
82.36	0\\
82.37	0\\
82.38	0\\
82.39	0\\
82.4	0\\
82.41	0\\
82.42	0\\
82.43	0\\
82.44	0\\
82.45	0\\
82.46	0\\
82.47	0\\
82.48	0\\
82.49	0\\
82.5	0\\
82.51	0\\
82.52	0\\
82.53	0\\
82.54	0\\
82.55	0\\
82.56	0\\
82.57	0\\
82.58	0\\
82.59	0\\
82.6	0\\
82.61	0\\
82.62	0\\
82.63	0\\
82.64	0\\
82.65	0\\
82.66	0\\
82.67	0\\
82.68	0\\
82.69	0\\
82.7	0\\
82.71	0\\
82.72	0\\
82.73	0\\
82.74	0\\
82.75	0\\
82.76	0\\
82.77	0\\
82.78	0\\
82.79	0\\
82.8	0\\
82.81	0\\
82.82	0\\
82.83	0\\
82.84	0\\
82.85	0\\
82.86	0\\
82.87	0\\
82.88	0\\
82.89	0\\
82.9	0\\
82.91	0\\
82.92	0\\
82.93	0\\
82.94	0\\
82.95	0\\
82.96	0\\
82.97	0\\
82.98	0\\
82.99	0\\
83	0\\
83.01	0\\
83.02	0\\
83.03	0\\
83.04	0\\
83.05	0\\
83.06	0\\
83.07	0\\
83.08	0\\
83.09	0\\
83.1	0\\
83.11	0\\
83.12	0\\
83.13	0\\
83.14	0\\
83.15	0\\
83.16	0\\
83.17	0\\
83.18	0\\
83.19	0\\
83.2	0\\
83.21	0\\
83.22	0\\
83.23	0\\
83.24	0\\
83.25	0\\
83.26	0\\
83.27	0\\
83.28	0\\
83.29	0\\
83.3	0\\
83.31	0\\
83.32	0\\
83.33	0\\
83.34	0\\
83.35	0\\
83.36	0\\
83.37	0\\
83.38	0\\
83.39	0\\
83.4	0\\
83.41	0\\
83.42	0\\
83.43	0\\
83.44	0\\
83.45	0\\
83.46	0\\
83.47	0\\
83.48	0\\
83.49	0\\
83.5	0\\
83.51	0\\
83.52	0\\
83.53	0\\
83.54	0\\
83.55	0\\
83.56	0\\
83.57	0\\
83.58	0\\
83.59	0\\
83.6	0\\
83.61	0\\
83.62	0\\
83.63	0\\
83.64	0\\
83.65	0\\
83.66	0\\
83.67	0\\
83.68	0\\
83.69	0\\
83.7	0\\
83.71	0\\
83.72	0\\
83.73	0\\
83.74	0\\
83.75	0\\
83.76	0\\
83.77	0\\
83.78	0\\
83.79	0\\
83.8	0\\
83.81	0\\
83.82	0\\
83.83	0\\
83.84	0\\
83.85	0\\
83.86	0\\
83.87	0\\
83.88	0\\
83.89	0\\
83.9	0\\
83.91	0\\
83.92	0\\
83.93	0\\
83.94	0\\
83.95	0\\
83.96	0\\
83.97	0\\
83.98	0\\
83.99	0\\
84	0\\
84.01	0\\
84.02	0\\
84.03	0\\
84.04	0\\
84.05	0\\
84.06	0\\
84.07	0\\
84.08	0\\
84.09	0\\
84.1	0\\
84.11	0\\
84.12	0\\
84.13	0\\
84.14	0\\
84.15	0\\
84.16	0\\
84.17	0\\
84.18	0\\
84.19	0\\
84.2	0\\
84.21	0\\
84.22	0\\
84.23	0\\
84.24	0\\
84.25	0\\
84.26	0\\
84.27	0\\
84.28	0\\
84.29	0\\
84.3	0\\
84.31	0\\
84.32	0\\
84.33	0\\
84.34	0\\
84.35	0\\
84.36	0\\
84.37	0\\
84.38	0\\
84.39	0\\
84.4	0\\
84.41	0\\
84.42	0\\
84.43	0\\
84.44	0\\
84.45	0\\
84.46	0\\
84.47	0\\
84.48	0\\
84.49	0\\
84.5	0\\
84.51	0\\
84.52	0\\
84.53	0\\
84.54	0\\
84.55	0\\
84.56	0\\
84.57	0\\
84.58	0\\
84.59	0\\
84.6	0\\
84.61	0\\
84.62	0\\
84.63	0\\
84.64	0\\
84.65	0\\
84.66	0\\
84.67	0\\
84.68	0\\
84.69	0\\
84.7	0\\
84.71	0\\
84.72	0\\
84.73	0\\
84.74	0\\
84.75	0\\
84.76	0\\
84.77	0\\
84.78	0\\
84.79	0\\
84.8	0\\
84.81	0\\
84.82	0\\
84.83	0\\
84.84	0\\
84.85	0\\
84.86	0\\
84.87	0\\
84.88	0\\
84.89	0\\
84.9	0\\
84.91	0\\
84.92	0\\
84.93	0\\
84.94	0\\
84.95	0\\
84.96	0\\
84.97	0\\
84.98	0\\
84.99	0\\
85	0\\
85.01	0\\
85.02	0\\
85.03	0\\
85.04	0\\
85.05	0\\
85.06	0\\
85.07	0\\
85.08	0\\
85.09	0\\
85.1	0\\
85.11	0\\
85.12	0\\
85.13	0\\
85.14	0\\
85.15	0\\
85.16	0\\
85.17	0\\
85.18	0\\
85.19	0\\
85.2	0\\
85.21	0\\
85.22	0\\
85.23	0\\
85.24	0\\
85.25	0\\
85.26	0\\
85.27	0\\
85.28	0\\
85.29	0\\
85.3	0\\
85.31	0\\
85.32	0\\
85.33	0\\
85.34	0\\
85.35	0\\
85.36	0\\
85.37	0\\
85.38	0\\
85.39	0\\
85.4	0\\
85.41	0\\
85.42	0\\
85.43	0\\
85.44	0\\
85.45	0\\
85.46	0\\
85.47	0\\
85.48	0\\
85.49	0\\
85.5	0\\
85.51	0\\
85.52	0\\
85.53	0\\
85.54	0\\
85.55	0\\
85.56	0\\
85.57	0\\
85.58	0\\
85.59	0\\
85.6	0\\
85.61	0\\
85.62	0\\
85.63	0\\
85.64	0\\
85.65	0\\
85.66	0\\
85.67	0\\
85.68	0\\
85.69	0\\
85.7	0\\
85.71	0\\
85.72	0\\
85.73	0\\
85.74	0\\
85.75	0\\
85.76	0\\
85.77	0\\
85.78	0\\
85.79	0\\
85.8	0\\
85.81	0\\
85.82	0\\
85.83	0\\
85.84	0\\
85.85	0\\
85.86	0\\
85.87	0\\
85.88	0\\
85.89	0\\
85.9	0\\
85.91	0\\
85.92	0\\
85.93	0\\
85.94	0\\
85.95	0\\
85.96	0\\
85.97	0\\
85.98	0\\
85.99	0\\
86	0\\
86.01	0\\
86.02	0\\
86.03	0\\
86.04	0\\
86.05	0\\
86.06	0\\
86.07	0\\
86.08	0\\
86.09	0\\
86.1	0\\
86.11	0\\
86.12	0\\
86.13	0\\
86.14	0\\
86.15	0\\
86.16	0\\
86.17	0\\
86.18	0\\
86.19	0\\
86.2	0\\
86.21	0\\
86.22	0\\
86.23	0\\
86.24	0\\
86.25	0\\
86.26	0\\
86.27	0\\
86.28	0\\
86.29	0\\
86.3	0\\
86.31	0\\
86.32	0\\
86.33	0\\
86.34	0\\
86.35	0\\
86.36	0\\
86.37	0\\
86.38	0\\
86.39	0\\
86.4	0\\
86.41	0\\
86.42	0\\
86.43	0\\
86.44	0\\
86.45	0\\
86.46	0\\
86.47	0\\
86.48	0\\
86.49	0\\
86.5	0\\
86.51	0\\
86.52	0\\
86.53	0\\
86.54	0\\
86.55	0\\
86.56	0\\
86.57	0\\
86.58	0\\
86.59	0\\
86.6	0\\
86.61	0\\
86.62	0\\
86.63	0\\
86.64	0\\
86.65	0\\
86.66	0\\
86.67	0\\
86.68	0\\
86.69	0\\
86.7	0\\
86.71	0\\
86.72	0\\
86.73	0\\
86.74	0\\
86.75	0\\
86.76	0\\
86.77	0\\
86.78	0\\
86.79	0\\
86.8	0\\
86.81	0\\
86.82	0\\
86.83	0\\
86.84	0\\
86.85	0\\
86.86	0\\
86.87	0\\
86.88	0\\
86.89	0\\
86.9	0\\
86.91	0\\
86.92	0\\
86.93	0\\
86.94	0\\
86.95	0\\
86.96	0\\
86.97	0\\
86.98	0\\
86.99	0\\
87	0\\
87.01	0\\
87.02	0\\
87.03	0\\
87.04	0\\
87.05	0\\
87.06	0\\
87.07	0\\
87.08	0\\
87.09	0\\
87.1	0\\
87.11	0\\
87.12	0\\
87.13	0\\
87.14	0\\
87.15	0\\
87.16	0\\
87.17	0\\
87.18	0\\
87.19	0\\
87.2	0\\
87.21	0\\
87.22	0\\
87.23	0\\
87.24	0\\
87.25	0\\
87.26	0\\
87.27	0\\
87.28	0\\
87.29	0\\
87.3	0\\
87.31	0\\
87.32	0\\
87.33	0\\
87.34	0\\
87.35	0\\
87.36	0\\
87.37	0\\
87.38	0\\
87.39	0\\
87.4	0\\
87.41	0\\
87.42	0\\
87.43	0\\
87.44	0\\
87.45	0\\
87.46	0\\
87.47	0\\
87.48	0\\
87.49	0\\
87.5	0\\
87.51	0\\
87.52	0\\
87.53	0\\
87.54	0\\
87.55	0\\
87.56	0\\
87.57	0\\
87.58	0\\
87.59	0\\
87.6	0\\
87.61	0\\
87.62	0\\
87.63	0\\
87.64	0\\
87.65	0\\
87.66	0\\
87.67	0\\
87.68	0\\
87.69	0\\
87.7	0\\
87.71	0\\
87.72	0\\
87.73	0\\
87.74	0\\
87.75	0\\
87.76	0\\
87.77	0\\
87.78	0\\
87.79	0\\
87.8	0\\
87.81	0\\
87.82	0\\
87.83	0\\
87.84	0\\
87.85	0\\
87.86	0\\
87.87	0\\
87.88	0\\
87.89	0\\
87.9	0\\
87.91	0\\
87.92	0\\
87.93	0\\
87.94	0\\
87.95	0\\
87.96	0\\
87.97	0\\
87.98	0\\
87.99	0\\
88	0\\
88.01	0\\
88.02	0\\
88.03	0\\
88.04	0\\
88.05	0\\
88.06	0\\
88.07	0\\
88.08	0\\
88.09	0\\
88.1	0\\
88.11	0\\
88.12	0\\
88.13	0\\
88.14	0\\
88.15	0\\
88.16	0\\
88.17	0\\
88.18	0\\
88.19	0\\
88.2	0\\
88.21	0\\
88.22	0\\
88.23	0\\
88.24	0\\
88.25	0\\
88.26	0\\
88.27	0\\
88.28	0\\
88.29	0\\
88.3	0\\
88.31	0\\
88.32	0\\
88.33	0\\
88.34	0\\
88.35	0\\
88.36	0\\
88.37	0\\
88.38	0\\
88.39	0\\
88.4	0\\
88.41	0\\
88.42	0\\
88.43	0\\
88.44	0\\
88.45	0\\
88.46	0\\
88.47	0\\
88.48	0\\
88.49	0\\
88.5	0\\
88.51	0\\
88.52	0\\
88.53	0\\
88.54	0\\
88.55	0\\
88.56	0\\
88.57	0\\
88.58	0\\
88.59	0\\
88.6	0\\
88.61	0\\
88.62	0\\
88.63	0\\
88.64	0\\
88.65	0\\
88.66	0\\
88.67	0\\
88.68	0\\
88.69	0\\
88.7	0\\
88.71	0\\
88.72	0\\
88.73	0\\
88.74	0\\
88.75	0\\
88.76	0\\
88.77	0\\
88.78	0\\
88.79	0\\
88.8	0\\
88.81	0\\
88.82	0\\
88.83	0\\
88.84	0\\
88.85	0\\
88.86	0\\
88.87	0\\
88.88	0\\
88.89	0\\
88.9	0\\
88.91	0\\
88.92	0\\
88.93	0\\
88.94	0\\
88.95	0\\
88.96	0\\
88.97	0\\
88.98	0\\
88.99	0\\
89	0\\
89.01	0\\
89.02	0\\
89.03	0\\
89.04	0\\
89.05	0\\
89.06	0\\
89.07	0\\
89.08	0\\
89.09	0\\
89.1	0\\
89.11	0\\
89.12	0\\
89.13	0\\
89.14	0\\
89.15	0\\
89.16	0\\
89.17	0\\
89.18	0\\
89.19	0\\
89.2	0\\
89.21	0\\
89.22	0\\
89.23	0\\
89.24	0\\
89.25	0\\
89.26	0\\
89.27	0\\
89.28	0\\
89.29	0\\
89.3	0\\
89.31	0\\
89.32	0\\
89.33	0\\
89.34	0\\
89.35	0\\
89.36	0\\
89.37	0\\
89.38	0\\
89.39	0\\
89.4	0\\
89.41	0\\
89.42	0\\
89.43	0\\
89.44	0\\
89.45	0\\
89.46	0\\
89.47	0\\
89.48	0\\
89.49	0\\
89.5	0\\
89.51	0\\
89.52	0\\
89.53	0\\
89.54	0\\
89.55	0\\
89.56	0\\
89.57	0\\
89.58	0\\
89.59	0\\
89.6	0\\
89.61	0\\
89.62	0\\
89.63	0\\
89.64	0\\
89.65	0\\
89.66	0\\
89.67	0\\
89.68	0\\
89.69	0\\
89.7	0\\
89.71	0\\
89.72	0\\
89.73	0\\
89.74	0\\
89.75	0\\
89.76	0\\
89.77	0\\
89.78	0\\
89.79	0\\
89.8	0\\
89.81	0\\
89.82	0\\
89.83	0\\
89.84	0\\
89.85	0\\
89.86	0\\
89.87	0\\
89.88	0\\
89.89	0\\
89.9	0\\
89.91	0\\
89.92	0\\
89.93	0\\
89.94	0\\
89.95	0\\
89.96	0\\
89.97	0\\
89.98	0\\
89.99	0\\
90	0\\
90.01	0\\
90.02	0\\
90.03	0\\
90.04	0\\
90.05	0\\
90.06	0\\
90.07	0\\
90.08	0\\
90.09	0\\
90.1	0\\
90.11	0\\
90.12	0\\
90.13	0\\
90.14	0\\
90.15	0\\
90.16	0\\
90.17	0\\
90.18	0\\
90.19	0\\
90.2	0\\
90.21	0\\
90.22	0\\
90.23	0\\
90.24	0\\
90.25	0\\
90.26	0\\
90.27	0\\
90.28	0\\
90.29	0\\
90.3	0\\
90.31	0\\
90.32	0\\
90.33	0\\
90.34	0\\
90.35	0\\
90.36	0\\
90.37	0\\
90.38	0\\
90.39	0\\
90.4	0\\
90.41	0\\
90.42	0\\
90.43	0\\
90.44	0\\
90.45	0\\
90.46	0\\
90.47	0\\
90.48	0\\
90.49	0\\
90.5	0\\
90.51	0\\
90.52	0\\
90.53	0\\
90.54	0\\
90.55	0\\
90.56	0\\
90.57	0\\
90.58	0\\
90.59	0\\
90.6	0\\
90.61	0\\
90.62	0\\
90.63	0\\
90.64	0\\
90.65	0\\
90.66	0\\
90.67	0\\
90.68	0\\
90.69	0\\
90.7	0\\
90.71	0\\
90.72	0\\
90.73	0\\
90.74	0\\
90.75	0\\
90.76	0\\
90.77	0\\
90.78	0\\
90.79	0\\
90.8	0\\
90.81	0\\
90.82	0\\
90.83	0\\
90.84	0\\
90.85	0\\
90.86	0\\
90.87	0\\
90.88	0\\
90.89	0\\
90.9	0\\
90.91	0\\
90.92	0\\
90.93	0\\
90.94	0\\
90.95	0\\
90.96	0\\
90.97	0\\
90.98	0\\
90.99	0\\
91	0\\
91.01	0\\
91.02	0\\
91.03	0\\
91.04	0\\
91.05	0\\
91.06	0\\
91.07	0\\
91.08	0\\
91.09	0\\
91.1	0\\
91.11	0\\
91.12	0\\
91.13	0\\
91.14	0\\
91.15	0\\
91.16	0\\
91.17	0\\
91.18	0\\
91.19	0\\
91.2	0\\
91.21	0\\
91.22	0\\
91.23	0\\
91.24	0\\
91.25	0\\
91.26	0\\
91.27	0\\
91.28	0\\
91.29	0\\
91.3	0\\
91.31	0\\
91.32	0\\
91.33	0\\
91.34	0\\
91.35	0\\
91.36	0\\
91.37	0\\
91.38	0\\
91.39	0\\
91.4	0\\
91.41	0\\
91.42	0\\
91.43	0\\
91.44	0\\
91.45	0\\
91.46	0\\
91.47	0\\
91.48	0\\
91.49	0\\
91.5	0\\
91.51	0\\
91.52	0\\
91.53	0\\
91.54	0\\
91.55	0\\
91.56	0\\
91.57	0\\
91.58	0\\
91.59	0\\
91.6	0\\
91.61	0\\
91.62	0\\
91.63	0\\
91.64	0\\
91.65	0\\
91.66	0\\
91.67	0\\
91.68	0\\
91.69	0\\
91.7	0\\
91.71	0\\
91.72	0\\
91.73	0\\
91.74	0\\
91.75	0\\
91.76	0\\
91.77	0\\
91.78	0\\
91.79	0\\
91.8	0\\
91.81	0\\
91.82	0\\
91.83	0\\
91.84	0\\
91.85	0\\
91.86	0\\
91.87	0\\
91.88	0\\
91.89	0\\
91.9	0\\
91.91	0\\
91.92	0\\
91.93	0\\
91.94	0\\
91.95	0\\
91.96	0\\
91.97	0\\
91.98	0\\
91.99	0\\
92	0\\
92.01	0\\
92.02	0\\
92.03	0\\
92.04	0\\
92.05	0\\
92.06	0\\
92.07	0\\
92.08	0\\
92.09	0\\
92.1	0\\
92.11	0\\
92.12	0\\
92.13	0\\
92.14	0\\
92.15	0\\
92.16	0\\
92.17	0\\
92.18	0\\
92.19	0\\
92.2	0\\
92.21	0\\
92.22	0\\
92.23	0\\
92.24	0\\
92.25	0\\
92.26	0\\
92.27	0\\
92.28	0\\
92.29	0\\
92.3	0\\
92.31	0\\
92.32	0\\
92.33	0\\
92.34	0\\
92.35	0\\
92.36	0\\
92.37	0\\
92.38	0\\
92.39	0\\
92.4	0\\
92.41	0\\
92.42	0\\
92.43	0\\
92.44	0\\
92.45	0\\
92.46	0\\
92.47	0\\
92.48	0\\
92.49	0\\
92.5	0\\
92.51	0\\
92.52	0\\
92.53	0\\
92.54	0\\
92.55	0\\
92.56	0\\
92.57	0\\
92.58	0\\
92.59	0\\
92.6	0\\
92.61	0\\
92.62	0\\
92.63	0\\
92.64	0\\
92.65	0\\
92.66	0\\
92.67	0\\
92.68	0\\
92.69	0\\
92.7	0\\
92.71	0\\
92.72	0\\
92.73	0\\
92.74	0\\
92.75	0\\
92.76	0\\
92.77	0\\
92.78	0\\
92.79	0\\
92.8	0\\
92.81	0\\
92.82	0\\
92.83	0\\
92.84	0\\
92.85	0\\
92.86	0\\
92.87	0\\
92.88	0\\
92.89	0\\
92.9	0\\
92.91	0\\
92.92	0\\
92.93	0\\
92.94	0\\
92.95	0\\
92.96	0\\
92.97	0\\
92.98	0\\
92.99	0\\
93	0\\
93.01	0\\
93.02	0\\
93.03	0\\
93.04	0\\
93.05	0\\
93.06	0\\
93.07	0\\
93.08	0\\
93.09	0\\
93.1	0\\
93.11	0\\
93.12	0\\
93.13	0\\
93.14	0\\
93.15	0\\
93.16	0\\
93.17	0\\
93.18	0\\
93.19	0\\
93.2	0\\
93.21	0\\
93.22	0\\
93.23	0\\
93.24	0\\
93.25	0\\
93.26	0\\
93.27	0\\
93.28	0\\
93.29	0\\
93.3	0\\
93.31	0\\
93.32	0\\
93.33	0\\
93.34	0\\
93.35	0\\
93.36	0\\
93.37	0\\
93.38	0\\
93.39	0\\
93.4	0\\
93.41	0\\
93.42	0\\
93.43	0\\
93.44	0\\
93.45	0\\
93.46	0\\
93.47	0\\
93.48	0\\
93.49	0\\
93.5	0\\
93.51	0\\
93.52	0\\
93.53	0\\
93.54	0\\
93.55	0\\
93.56	0\\
93.57	0\\
93.58	0\\
93.59	0\\
93.6	0\\
93.61	0\\
93.62	0\\
93.63	0\\
93.64	0\\
93.65	0\\
93.66	0\\
93.67	0\\
93.68	0\\
93.69	0\\
93.7	0\\
93.71	0\\
93.72	0\\
93.73	0\\
93.74	0\\
93.75	0\\
93.76	0\\
93.77	0\\
93.78	0\\
93.79	0\\
93.8	0\\
93.81	0\\
93.82	0\\
93.83	0\\
93.84	0\\
93.85	0\\
93.86	0\\
93.87	0\\
93.88	0\\
93.89	0\\
93.9	0\\
93.91	0\\
93.92	0\\
93.93	0\\
93.94	0\\
93.95	0\\
93.96	0\\
93.97	0\\
93.98	0\\
93.99	0\\
94	0\\
94.01	0\\
94.02	0\\
94.03	0\\
94.04	0\\
94.05	0\\
94.06	0\\
94.07	0\\
94.08	0\\
94.09	0\\
94.1	0\\
94.11	0\\
94.12	0\\
94.13	0\\
94.14	0\\
94.15	0\\
94.16	0\\
94.17	0\\
94.18	0\\
94.19	0\\
94.2	0\\
94.21	0\\
94.22	0\\
94.23	0\\
94.24	0\\
94.25	0\\
94.26	0\\
94.27	0\\
94.28	0\\
94.29	0\\
94.3	0\\
94.31	0\\
94.32	0\\
94.33	0\\
94.34	0\\
94.35	0\\
94.36	0\\
94.37	0\\
94.38	0\\
94.39	0\\
94.4	0\\
94.41	0\\
94.42	0\\
94.43	0\\
94.44	0\\
94.45	0\\
94.46	0\\
94.47	0\\
94.48	0\\
94.49	0\\
94.5	0\\
94.51	0\\
94.52	0\\
94.53	0\\
94.54	0\\
94.55	0\\
94.56	0\\
94.57	0\\
94.58	0\\
94.59	0\\
94.6	0\\
94.61	0\\
94.62	0\\
94.63	0\\
94.64	0\\
94.65	0\\
94.66	0\\
94.67	0\\
94.68	0\\
94.69	0\\
94.7	0\\
94.71	0\\
94.72	0\\
94.73	0\\
94.74	0\\
94.75	0\\
94.76	0\\
94.77	0\\
94.78	0\\
94.79	0\\
94.8	0\\
94.81	0\\
94.82	0\\
94.83	0\\
94.84	0\\
94.85	0\\
94.86	0\\
94.87	0\\
94.88	0\\
94.89	0\\
94.9	0\\
94.91	0\\
94.92	0\\
94.93	0\\
94.94	0\\
94.95	0\\
94.96	0\\
94.97	0\\
94.98	0\\
94.99	0\\
95	0\\
95.01	0\\
95.02	0\\
95.03	0\\
95.04	0\\
95.05	0\\
95.06	0\\
95.07	0\\
95.08	0\\
95.09	0\\
95.1	0\\
95.11	0\\
95.12	0\\
95.13	0\\
95.14	0\\
95.15	0\\
95.16	0\\
95.17	0\\
95.18	0\\
95.19	0\\
95.2	0\\
95.21	0\\
95.22	0\\
95.23	0\\
95.24	0\\
95.25	0\\
95.26	0\\
95.27	0\\
95.28	0\\
95.29	0\\
95.3	0\\
95.31	0\\
95.32	0\\
95.33	0\\
95.34	0\\
95.35	0\\
95.36	0\\
95.37	0\\
95.38	0\\
95.39	0\\
95.4	0\\
95.41	0\\
95.42	0\\
95.43	0\\
95.44	0\\
95.45	0\\
95.46	0\\
95.47	0\\
95.48	0\\
95.49	0\\
95.5	0\\
95.51	0\\
95.52	0\\
95.53	0\\
95.54	0\\
95.55	0\\
95.56	0\\
95.57	0\\
95.58	0\\
95.59	0\\
95.6	0\\
95.61	0\\
95.62	0\\
95.63	0\\
95.64	0\\
95.65	0\\
95.66	0\\
95.67	0\\
95.68	0\\
95.69	0\\
95.7	0\\
95.71	0\\
95.72	0\\
95.73	0\\
95.74	0\\
95.75	0\\
95.76	0\\
95.77	0\\
95.78	0\\
95.79	0\\
95.8	0\\
95.81	0\\
95.82	0\\
95.83	0\\
95.84	0\\
95.85	0\\
95.86	0\\
95.87	0\\
95.88	0\\
95.89	0\\
95.9	0\\
95.91	0\\
95.92	0\\
95.93	0\\
95.94	0\\
95.95	0\\
95.96	0\\
95.97	0\\
95.98	0\\
95.99	0\\
96	0\\
96.01	0\\
96.02	0\\
96.03	0\\
96.04	0\\
96.05	0\\
96.06	0\\
96.07	0\\
96.08	0\\
96.09	0\\
96.1	0\\
96.11	0\\
96.12	0\\
96.13	0\\
96.14	0\\
96.15	0\\
96.16	0\\
96.17	0\\
96.18	0\\
96.19	0\\
96.2	0\\
96.21	0\\
96.22	0\\
96.23	0\\
96.24	0\\
96.25	0\\
96.26	0\\
96.27	0\\
96.28	0\\
96.29	0\\
96.3	0\\
96.31	0\\
96.32	0\\
96.33	0\\
96.34	0\\
96.35	0\\
96.36	0\\
96.37	0\\
96.38	0\\
96.39	0\\
96.4	0\\
96.41	0\\
96.42	0\\
96.43	0\\
96.44	0\\
96.45	0\\
96.46	0\\
96.47	0\\
96.48	0\\
96.49	0\\
96.5	0\\
96.51	0\\
96.52	0\\
96.53	0\\
96.54	0\\
96.55	0\\
96.56	0\\
96.57	0\\
96.58	0\\
96.59	0\\
96.6	0\\
96.61	0\\
96.62	0\\
96.63	0\\
96.64	0\\
96.65	0\\
96.66	0\\
96.67	0\\
96.68	0\\
96.69	0\\
96.7	0\\
96.71	0\\
96.72	0\\
96.73	0\\
96.74	0\\
96.75	0\\
96.76	0\\
96.77	0\\
96.78	0\\
96.79	0\\
96.8	0\\
96.81	0\\
96.82	0\\
96.83	0\\
96.84	0\\
96.85	0\\
96.86	0\\
96.87	0\\
96.88	0\\
96.89	0\\
96.9	0\\
96.91	0\\
96.92	0\\
96.93	0\\
96.94	0\\
96.95	0\\
96.96	0\\
96.97	0\\
96.98	0\\
96.99	0\\
97	0\\
97.01	0\\
97.02	0\\
97.03	0\\
97.04	0\\
97.05	0\\
97.06	0\\
97.07	0\\
97.08	0\\
97.09	0\\
97.1	0\\
97.11	0\\
97.12	0\\
97.13	0\\
97.14	0\\
97.15	0\\
97.16	0\\
97.17	0\\
97.18	0\\
97.19	0\\
97.2	0\\
97.21	0\\
97.22	0\\
97.23	0\\
97.24	0\\
97.25	0\\
97.26	0\\
97.27	0\\
97.28	0\\
97.29	0\\
97.3	0\\
97.31	0\\
97.32	0\\
97.33	0\\
97.34	0\\
97.35	0\\
97.36	0\\
97.37	0\\
97.38	0\\
97.39	0\\
97.4	0\\
97.41	0\\
97.42	0\\
97.43	0\\
97.44	0\\
97.45	0\\
97.46	0\\
97.47	0\\
97.48	0\\
97.49	0\\
97.5	0\\
97.51	0\\
97.52	0\\
97.53	0\\
97.54	0\\
97.55	0\\
97.56	0\\
97.57	0\\
97.58	0\\
97.59	0\\
97.6	0\\
97.61	0\\
97.62	0\\
97.63	0\\
97.64	0\\
97.65	0\\
97.66	0\\
97.67	0\\
97.68	0\\
97.69	0\\
97.7	0\\
97.71	0\\
97.72	0\\
97.73	0\\
97.74	0\\
97.75	0\\
97.76	0\\
97.77	0\\
97.78	0\\
97.79	0\\
97.8	0\\
97.81	0\\
97.82	0\\
97.83	0\\
97.84	0\\
97.85	0\\
97.86	0\\
97.87	0\\
97.88	0\\
97.89	0\\
97.9	0\\
97.91	0\\
97.92	0\\
97.93	0\\
97.94	0\\
97.95	0\\
97.96	0\\
97.97	0\\
97.98	0\\
97.99	0\\
98	0\\
98.01	0\\
98.02	0\\
98.03	0\\
98.04	0\\
98.05	0\\
98.06	0\\
98.07	0\\
98.08	0\\
98.09	0\\
98.1	0\\
98.11	0\\
98.12	0\\
98.13	0\\
98.14	0\\
98.15	0\\
98.16	0\\
98.17	0\\
98.18	0\\
98.19	0\\
98.2	0\\
98.21	0\\
98.22	0\\
98.23	0\\
98.24	0\\
98.25	0\\
98.26	0\\
98.27	0\\
98.28	0\\
98.29	0\\
98.3	0\\
98.31	0\\
98.32	0\\
98.33	0\\
98.34	0\\
98.35	0\\
98.36	0\\
98.37	0\\
98.38	0\\
98.39	0\\
98.4	0\\
98.41	0\\
98.42	0\\
98.43	0\\
98.44	0\\
98.45	0\\
98.46	0\\
98.47	0\\
98.48	0\\
98.49	0\\
98.5	0\\
98.51	0\\
98.52	0\\
98.53	0\\
98.54	0\\
98.55	0\\
98.56	0\\
98.57	0\\
98.58	0\\
98.59	0\\
98.6	0\\
98.61	0\\
98.62	0\\
98.63	0\\
98.64	0\\
98.65	0\\
98.66	0\\
98.67	0\\
98.68	0\\
98.69	3.45443584492192e-05\\
98.7	8.03198772388932e-05\\
98.71	0.000126406080126856\\
98.72	0.000172805774561973\\
98.73	0.000219521789666057\\
98.74	0.000266554423900341\\
98.75	0.000313904475297423\\
98.76	0.000361574794340553\\
98.77	0.00040956826207514\\
98.78	0.000457887790512465\\
98.79	0.000506528840464412\\
98.8	0.000555492119411654\\
98.81	0.000604780473917106\\
98.82	0.000654396780925144\\
98.83	0.000704343948166776\\
98.84	0.00075462491457162\\
98.85	0.000805242650687451\\
98.86	0.000856200159106401\\
98.87	0.000907500474897953\\
98.88	0.000959146666048846\\
98.89	0.00101114183391\\
98.9	0.00106348911365062\\
98.91	0.00111619167471953\\
98.92	0.00116925272131401\\
98.93	0.00122267549285609\\
98.94	0.00127646326447658\\
98.95	0.00133061934750694\\
98.96	0.00138514708997907\\
98.97	0.00144004987713325\\
98.98	0.00149533113193433\\
98.99	0.00155099431559634\\
99	0.00160704292811567\\
99.01	0.00166348050881296\\
99.02	0.00172031063688394\\
99.03	0.00177753693195926\\
99.04	0.0018351630546736\\
99.05	0.00189319270724418\\
99.06	0.0019516296340588\\
99.07	0.00201047762227375\\
99.08	0.00206974050251821\\
99.09	0.00212942214959332\\
99.1	0.00218952648310286\\
99.11	0.00225005746809431\\
99.12	0.00231101911644198\\
99.13	0.00237241548697314\\
99.14	0.00243425068596524\\
99.15	0.00249652886783157\\
99.16	0.00255925423581837\\
99.17	0.0026224310427139\\
99.18	0.00268606359156946\\
99.19	0.00275015623643286\\
99.2	0.00281471338309431\\
99.21	0.00287973948984525\\
99.22	0.00294523906825022\\
99.23	0.00301121668393211\\
99.24	0.00307767695737102\\
99.25	0.00314462456471708\\
99.26	0.0032120642386175\\
99.27	0.00328000076905816\\
99.28	0.00334843900422\\
99.29	0.00341738385135061\\
99.3	0.00348684027765132\\
99.31	0.00355681331118016\\
99.32	0.00362730804177092\\
99.33	0.00369832962196888\\
99.34	0.00376988326798339\\
99.35	0.00384197426065781\\
99.36	0.00391460794645714\\
99.37	0.00398778973847378\\
99.38	0.00406152511745181\\
99.39	0.00413581963283032\\
99.4	0.00421067890380606\\
99.41	0.00428610862041612\\
99.42	0.00436211454464085\\
99.43	0.00443870251152773\\
99.44	0.00451587843033661\\
99.45	0.00459364828570688\\
99.46	0.004672018138847\\
99.47	0.00475099412874719\\
99.48	0.00483058247341566\\
99.49	0.00491078947113902\\
99.5	0.0049916215017676\\
99.51	0.00507308502802622\\
99.52	0.00515518659685116\\
99.53	0.0052379328407539\\
99.54	0.00532133047921257\\
99.55	0.0054053863200916\\
99.56	0.00549010726109058\\
99.57	0.00557550029122293\\
99.58	0.00566157249232538\\
99.59	0.00574833104059902\\
99.6	0.00583578320818287\\
99.61	0.00592393636476083\\
99.62	0.00601279797920297\\
99.63	0.00610237562124226\\
99.64	0.00619267696318778\\
99.65	0.00628370978167542\\
99.66	0.0063754819594573\\
99.67	0.006468001487231\\
99.68	0.00656127646550995\\
99.69	0.00665531510653603\\
99.7	0.00675012573623595\\
99.71	0.00684571679629706\\
99.72	0.00694209684622123\\
99.73	0.00703927456542089\\
99.74	0.00713725875536444\\
99.75	0.00723605834177294\\
99.76	0.00733568237686946\\
99.77	0.00743614004168323\\
99.78	0.00753744064841003\\
99.79	0.00763959364283112\\
99.8	0.00774260860679244\\
99.81	0.00784649526074633\\
99.82	0.00795126346635793\\
99.83	0.00805692322917855\\
99.84	0.00816348470138832\\
99.85	0.00827095818461085\\
99.86	0.00837935413280225\\
99.87	0.00848868315521739\\
99.88	0.00859895601945634\\
99.89	0.00871018365459377\\
99.9	0.0088223771543947\\
99.91	0.00893554778061969\\
99.92	0.00904970696642306\\
99.93	0.00916486631984768\\
99.94	0.00928103762742014\\
99.95	0.00939823285785025\\
99.96	0.00951646416583911\\
99.97	0.00963574389600006\\
99.98	0.00975608458689704\\
99.99	0.00987749897520541\\
100	0.01\\
};
\addlegendentry{$q=0$};

\addplot [color=blue,solid,forget plot]
  table[row sep=crcr]{%
0.01	0\\
0.02	0\\
0.03	0\\
0.04	0\\
0.05	0\\
0.06	0\\
0.07	0\\
0.08	0\\
0.09	0\\
0.1	0\\
0.11	0\\
0.12	0\\
0.13	0\\
0.14	0\\
0.15	0\\
0.16	0\\
0.17	0\\
0.18	0\\
0.19	0\\
0.2	0\\
0.21	0\\
0.22	0\\
0.23	0\\
0.24	0\\
0.25	0\\
0.26	0\\
0.27	0\\
0.28	0\\
0.29	0\\
0.3	0\\
0.31	0\\
0.32	0\\
0.33	0\\
0.34	0\\
0.35	0\\
0.36	0\\
0.37	0\\
0.38	0\\
0.39	0\\
0.4	0\\
0.41	0\\
0.42	0\\
0.43	0\\
0.44	0\\
0.45	0\\
0.46	0\\
0.47	0\\
0.48	0\\
0.49	0\\
0.5	0\\
0.51	0\\
0.52	0\\
0.53	0\\
0.54	0\\
0.55	0\\
0.56	0\\
0.57	0\\
0.58	0\\
0.59	0\\
0.6	0\\
0.61	0\\
0.62	0\\
0.63	0\\
0.64	0\\
0.65	0\\
0.66	0\\
0.67	0\\
0.68	0\\
0.69	0\\
0.7	0\\
0.71	0\\
0.72	0\\
0.73	0\\
0.74	0\\
0.75	0\\
0.76	0\\
0.77	0\\
0.78	0\\
0.79	0\\
0.8	0\\
0.81	0\\
0.82	0\\
0.83	0\\
0.84	0\\
0.85	0\\
0.86	0\\
0.87	0\\
0.88	0\\
0.89	0\\
0.9	0\\
0.91	0\\
0.92	0\\
0.93	0\\
0.94	0\\
0.95	0\\
0.96	0\\
0.97	0\\
0.98	0\\
0.99	0\\
1	0\\
1.01	0\\
1.02	0\\
1.03	0\\
1.04	0\\
1.05	0\\
1.06	0\\
1.07	0\\
1.08	0\\
1.09	0\\
1.1	0\\
1.11	0\\
1.12	0\\
1.13	0\\
1.14	0\\
1.15	0\\
1.16	0\\
1.17	0\\
1.18	0\\
1.19	0\\
1.2	0\\
1.21	0\\
1.22	0\\
1.23	0\\
1.24	0\\
1.25	0\\
1.26	0\\
1.27	0\\
1.28	0\\
1.29	0\\
1.3	0\\
1.31	0\\
1.32	0\\
1.33	0\\
1.34	0\\
1.35	0\\
1.36	0\\
1.37	0\\
1.38	0\\
1.39	0\\
1.4	0\\
1.41	0\\
1.42	0\\
1.43	0\\
1.44	0\\
1.45	0\\
1.46	0\\
1.47	0\\
1.48	0\\
1.49	0\\
1.5	0\\
1.51	0\\
1.52	0\\
1.53	0\\
1.54	0\\
1.55	0\\
1.56	0\\
1.57	0\\
1.58	0\\
1.59	0\\
1.6	0\\
1.61	0\\
1.62	0\\
1.63	0\\
1.64	0\\
1.65	0\\
1.66	0\\
1.67	0\\
1.68	0\\
1.69	0\\
1.7	0\\
1.71	0\\
1.72	0\\
1.73	0\\
1.74	0\\
1.75	0\\
1.76	0\\
1.77	0\\
1.78	0\\
1.79	0\\
1.8	0\\
1.81	0\\
1.82	0\\
1.83	0\\
1.84	0\\
1.85	0\\
1.86	0\\
1.87	0\\
1.88	0\\
1.89	0\\
1.9	0\\
1.91	0\\
1.92	0\\
1.93	0\\
1.94	0\\
1.95	0\\
1.96	0\\
1.97	0\\
1.98	0\\
1.99	0\\
2	0\\
2.01	0\\
2.02	0\\
2.03	0\\
2.04	0\\
2.05	0\\
2.06	0\\
2.07	0\\
2.08	0\\
2.09	0\\
2.1	0\\
2.11	0\\
2.12	0\\
2.13	0\\
2.14	0\\
2.15	0\\
2.16	0\\
2.17	0\\
2.18	0\\
2.19	0\\
2.2	0\\
2.21	0\\
2.22	0\\
2.23	0\\
2.24	0\\
2.25	0\\
2.26	0\\
2.27	0\\
2.28	0\\
2.29	0\\
2.3	0\\
2.31	0\\
2.32	0\\
2.33	0\\
2.34	0\\
2.35	0\\
2.36	0\\
2.37	0\\
2.38	0\\
2.39	0\\
2.4	0\\
2.41	0\\
2.42	0\\
2.43	0\\
2.44	0\\
2.45	0\\
2.46	0\\
2.47	0\\
2.48	0\\
2.49	0\\
2.5	0\\
2.51	0\\
2.52	0\\
2.53	0\\
2.54	0\\
2.55	0\\
2.56	0\\
2.57	0\\
2.58	0\\
2.59	0\\
2.6	0\\
2.61	0\\
2.62	0\\
2.63	0\\
2.64	0\\
2.65	0\\
2.66	0\\
2.67	0\\
2.68	0\\
2.69	0\\
2.7	0\\
2.71	0\\
2.72	0\\
2.73	0\\
2.74	0\\
2.75	0\\
2.76	0\\
2.77	0\\
2.78	0\\
2.79	0\\
2.8	0\\
2.81	0\\
2.82	0\\
2.83	0\\
2.84	0\\
2.85	0\\
2.86	0\\
2.87	0\\
2.88	0\\
2.89	0\\
2.9	0\\
2.91	0\\
2.92	0\\
2.93	0\\
2.94	0\\
2.95	0\\
2.96	0\\
2.97	0\\
2.98	0\\
2.99	0\\
3	0\\
3.01	0\\
3.02	0\\
3.03	0\\
3.04	0\\
3.05	0\\
3.06	0\\
3.07	0\\
3.08	0\\
3.09	0\\
3.1	0\\
3.11	0\\
3.12	0\\
3.13	0\\
3.14	0\\
3.15	0\\
3.16	0\\
3.17	0\\
3.18	0\\
3.19	0\\
3.2	0\\
3.21	0\\
3.22	0\\
3.23	0\\
3.24	0\\
3.25	0\\
3.26	0\\
3.27	0\\
3.28	0\\
3.29	0\\
3.3	0\\
3.31	0\\
3.32	0\\
3.33	0\\
3.34	0\\
3.35	0\\
3.36	0\\
3.37	0\\
3.38	0\\
3.39	0\\
3.4	0\\
3.41	0\\
3.42	0\\
3.43	0\\
3.44	0\\
3.45	0\\
3.46	0\\
3.47	0\\
3.48	0\\
3.49	0\\
3.5	0\\
3.51	0\\
3.52	0\\
3.53	0\\
3.54	0\\
3.55	0\\
3.56	0\\
3.57	0\\
3.58	0\\
3.59	0\\
3.6	0\\
3.61	0\\
3.62	0\\
3.63	0\\
3.64	0\\
3.65	0\\
3.66	0\\
3.67	0\\
3.68	0\\
3.69	0\\
3.7	0\\
3.71	0\\
3.72	0\\
3.73	0\\
3.74	0\\
3.75	0\\
3.76	0\\
3.77	0\\
3.78	0\\
3.79	0\\
3.8	0\\
3.81	0\\
3.82	0\\
3.83	0\\
3.84	0\\
3.85	0\\
3.86	0\\
3.87	0\\
3.88	0\\
3.89	0\\
3.9	0\\
3.91	0\\
3.92	0\\
3.93	0\\
3.94	0\\
3.95	0\\
3.96	0\\
3.97	0\\
3.98	0\\
3.99	0\\
4	0\\
4.01	0\\
4.02	0\\
4.03	0\\
4.04	0\\
4.05	0\\
4.06	0\\
4.07	0\\
4.08	0\\
4.09	0\\
4.1	0\\
4.11	0\\
4.12	0\\
4.13	0\\
4.14	0\\
4.15	0\\
4.16	0\\
4.17	0\\
4.18	0\\
4.19	0\\
4.2	0\\
4.21	0\\
4.22	0\\
4.23	0\\
4.24	0\\
4.25	0\\
4.26	0\\
4.27	0\\
4.28	0\\
4.29	0\\
4.3	0\\
4.31	0\\
4.32	0\\
4.33	0\\
4.34	0\\
4.35	0\\
4.36	0\\
4.37	0\\
4.38	0\\
4.39	0\\
4.4	0\\
4.41	0\\
4.42	0\\
4.43	0\\
4.44	0\\
4.45	0\\
4.46	0\\
4.47	0\\
4.48	0\\
4.49	0\\
4.5	0\\
4.51	0\\
4.52	0\\
4.53	0\\
4.54	0\\
4.55	0\\
4.56	0\\
4.57	0\\
4.58	0\\
4.59	0\\
4.6	0\\
4.61	0\\
4.62	0\\
4.63	0\\
4.64	0\\
4.65	0\\
4.66	0\\
4.67	0\\
4.68	0\\
4.69	0\\
4.7	0\\
4.71	0\\
4.72	0\\
4.73	0\\
4.74	0\\
4.75	0\\
4.76	0\\
4.77	0\\
4.78	0\\
4.79	0\\
4.8	0\\
4.81	0\\
4.82	0\\
4.83	0\\
4.84	0\\
4.85	0\\
4.86	0\\
4.87	0\\
4.88	0\\
4.89	0\\
4.9	0\\
4.91	0\\
4.92	0\\
4.93	0\\
4.94	0\\
4.95	0\\
4.96	0\\
4.97	0\\
4.98	0\\
4.99	0\\
5	0\\
5.01	0\\
5.02	0\\
5.03	0\\
5.04	0\\
5.05	0\\
5.06	0\\
5.07	0\\
5.08	0\\
5.09	0\\
5.1	0\\
5.11	0\\
5.12	0\\
5.13	0\\
5.14	0\\
5.15	0\\
5.16	0\\
5.17	0\\
5.18	0\\
5.19	0\\
5.2	0\\
5.21	0\\
5.22	0\\
5.23	0\\
5.24	0\\
5.25	0\\
5.26	0\\
5.27	0\\
5.28	0\\
5.29	0\\
5.3	0\\
5.31	0\\
5.32	0\\
5.33	0\\
5.34	0\\
5.35	0\\
5.36	0\\
5.37	0\\
5.38	0\\
5.39	0\\
5.4	0\\
5.41	0\\
5.42	0\\
5.43	0\\
5.44	0\\
5.45	0\\
5.46	0\\
5.47	0\\
5.48	0\\
5.49	0\\
5.5	0\\
5.51	0\\
5.52	0\\
5.53	0\\
5.54	0\\
5.55	0\\
5.56	0\\
5.57	0\\
5.58	0\\
5.59	0\\
5.6	0\\
5.61	0\\
5.62	0\\
5.63	0\\
5.64	0\\
5.65	0\\
5.66	0\\
5.67	0\\
5.68	0\\
5.69	0\\
5.7	0\\
5.71	0\\
5.72	0\\
5.73	0\\
5.74	0\\
5.75	0\\
5.76	0\\
5.77	0\\
5.78	0\\
5.79	0\\
5.8	0\\
5.81	0\\
5.82	0\\
5.83	0\\
5.84	0\\
5.85	0\\
5.86	0\\
5.87	0\\
5.88	0\\
5.89	0\\
5.9	0\\
5.91	0\\
5.92	0\\
5.93	0\\
5.94	0\\
5.95	0\\
5.96	0\\
5.97	0\\
5.98	0\\
5.99	0\\
6	0\\
6.01	0\\
6.02	0\\
6.03	0\\
6.04	0\\
6.05	0\\
6.06	0\\
6.07	0\\
6.08	0\\
6.09	0\\
6.1	0\\
6.11	0\\
6.12	0\\
6.13	0\\
6.14	0\\
6.15	0\\
6.16	0\\
6.17	0\\
6.18	0\\
6.19	0\\
6.2	0\\
6.21	0\\
6.22	0\\
6.23	0\\
6.24	0\\
6.25	0\\
6.26	0\\
6.27	0\\
6.28	0\\
6.29	0\\
6.3	0\\
6.31	0\\
6.32	0\\
6.33	0\\
6.34	0\\
6.35	0\\
6.36	0\\
6.37	0\\
6.38	0\\
6.39	0\\
6.4	0\\
6.41	0\\
6.42	0\\
6.43	0\\
6.44	0\\
6.45	0\\
6.46	0\\
6.47	0\\
6.48	0\\
6.49	0\\
6.5	0\\
6.51	0\\
6.52	0\\
6.53	0\\
6.54	0\\
6.55	0\\
6.56	0\\
6.57	0\\
6.58	0\\
6.59	0\\
6.6	0\\
6.61	0\\
6.62	0\\
6.63	0\\
6.64	0\\
6.65	0\\
6.66	0\\
6.67	0\\
6.68	0\\
6.69	0\\
6.7	0\\
6.71	0\\
6.72	0\\
6.73	0\\
6.74	0\\
6.75	0\\
6.76	0\\
6.77	0\\
6.78	0\\
6.79	0\\
6.8	0\\
6.81	0\\
6.82	0\\
6.83	0\\
6.84	0\\
6.85	0\\
6.86	0\\
6.87	0\\
6.88	0\\
6.89	0\\
6.9	0\\
6.91	0\\
6.92	0\\
6.93	0\\
6.94	0\\
6.95	0\\
6.96	0\\
6.97	0\\
6.98	0\\
6.99	0\\
7	0\\
7.01	0\\
7.02	0\\
7.03	0\\
7.04	0\\
7.05	0\\
7.06	0\\
7.07	0\\
7.08	0\\
7.09	0\\
7.1	0\\
7.11	0\\
7.12	0\\
7.13	0\\
7.14	0\\
7.15	0\\
7.16	0\\
7.17	0\\
7.18	0\\
7.19	0\\
7.2	0\\
7.21	0\\
7.22	0\\
7.23	0\\
7.24	0\\
7.25	0\\
7.26	0\\
7.27	0\\
7.28	0\\
7.29	0\\
7.3	0\\
7.31	0\\
7.32	0\\
7.33	0\\
7.34	0\\
7.35	0\\
7.36	0\\
7.37	0\\
7.38	0\\
7.39	0\\
7.4	0\\
7.41	0\\
7.42	0\\
7.43	0\\
7.44	0\\
7.45	0\\
7.46	0\\
7.47	0\\
7.48	0\\
7.49	0\\
7.5	0\\
7.51	0\\
7.52	0\\
7.53	0\\
7.54	0\\
7.55	0\\
7.56	0\\
7.57	0\\
7.58	0\\
7.59	0\\
7.6	0\\
7.61	0\\
7.62	0\\
7.63	0\\
7.64	0\\
7.65	0\\
7.66	0\\
7.67	0\\
7.68	0\\
7.69	0\\
7.7	0\\
7.71	0\\
7.72	0\\
7.73	0\\
7.74	0\\
7.75	0\\
7.76	0\\
7.77	0\\
7.78	0\\
7.79	0\\
7.8	0\\
7.81	0\\
7.82	0\\
7.83	0\\
7.84	0\\
7.85	0\\
7.86	0\\
7.87	0\\
7.88	0\\
7.89	0\\
7.9	0\\
7.91	0\\
7.92	0\\
7.93	0\\
7.94	0\\
7.95	0\\
7.96	0\\
7.97	0\\
7.98	0\\
7.99	0\\
8	0\\
8.01	0\\
8.02	0\\
8.03	0\\
8.04	0\\
8.05	0\\
8.06	0\\
8.07	0\\
8.08	0\\
8.09	0\\
8.1	0\\
8.11	0\\
8.12	0\\
8.13	0\\
8.14	0\\
8.15	0\\
8.16	0\\
8.17	0\\
8.18	0\\
8.19	0\\
8.2	0\\
8.21	0\\
8.22	0\\
8.23	0\\
8.24	0\\
8.25	0\\
8.26	0\\
8.27	0\\
8.28	0\\
8.29	0\\
8.3	0\\
8.31	0\\
8.32	0\\
8.33	0\\
8.34	0\\
8.35	0\\
8.36	0\\
8.37	0\\
8.38	0\\
8.39	0\\
8.4	0\\
8.41	0\\
8.42	0\\
8.43	0\\
8.44	0\\
8.45	0\\
8.46	0\\
8.47	0\\
8.48	0\\
8.49	0\\
8.5	0\\
8.51	0\\
8.52	0\\
8.53	0\\
8.54	0\\
8.55	0\\
8.56	0\\
8.57	0\\
8.58	0\\
8.59	0\\
8.6	0\\
8.61	0\\
8.62	0\\
8.63	0\\
8.64	0\\
8.65	0\\
8.66	0\\
8.67	0\\
8.68	0\\
8.69	0\\
8.7	0\\
8.71	0\\
8.72	0\\
8.73	0\\
8.74	0\\
8.75	0\\
8.76	0\\
8.77	0\\
8.78	0\\
8.79	0\\
8.8	0\\
8.81	0\\
8.82	0\\
8.83	0\\
8.84	0\\
8.85	0\\
8.86	0\\
8.87	0\\
8.88	0\\
8.89	0\\
8.9	0\\
8.91	0\\
8.92	0\\
8.93	0\\
8.94	0\\
8.95	0\\
8.96	0\\
8.97	0\\
8.98	0\\
8.99	0\\
9	0\\
9.01	0\\
9.02	0\\
9.03	0\\
9.04	0\\
9.05	0\\
9.06	0\\
9.07	0\\
9.08	0\\
9.09	0\\
9.1	0\\
9.11	0\\
9.12	0\\
9.13	0\\
9.14	0\\
9.15	0\\
9.16	0\\
9.17	0\\
9.18	0\\
9.19	0\\
9.2	0\\
9.21	0\\
9.22	0\\
9.23	0\\
9.24	0\\
9.25	0\\
9.26	0\\
9.27	0\\
9.28	0\\
9.29	0\\
9.3	0\\
9.31	0\\
9.32	0\\
9.33	0\\
9.34	0\\
9.35	0\\
9.36	0\\
9.37	0\\
9.38	0\\
9.39	0\\
9.4	0\\
9.41	0\\
9.42	0\\
9.43	0\\
9.44	0\\
9.45	0\\
9.46	0\\
9.47	0\\
9.48	0\\
9.49	0\\
9.5	0\\
9.51	0\\
9.52	0\\
9.53	0\\
9.54	0\\
9.55	0\\
9.56	0\\
9.57	0\\
9.58	0\\
9.59	0\\
9.6	0\\
9.61	0\\
9.62	0\\
9.63	0\\
9.64	0\\
9.65	0\\
9.66	0\\
9.67	0\\
9.68	0\\
9.69	0\\
9.7	0\\
9.71	0\\
9.72	0\\
9.73	0\\
9.74	0\\
9.75	0\\
9.76	0\\
9.77	0\\
9.78	0\\
9.79	0\\
9.8	0\\
9.81	0\\
9.82	0\\
9.83	0\\
9.84	0\\
9.85	0\\
9.86	0\\
9.87	0\\
9.88	0\\
9.89	0\\
9.9	0\\
9.91	0\\
9.92	0\\
9.93	0\\
9.94	0\\
9.95	0\\
9.96	0\\
9.97	0\\
9.98	0\\
9.99	0\\
10	0\\
10.01	0\\
10.02	0\\
10.03	0\\
10.04	0\\
10.05	0\\
10.06	0\\
10.07	0\\
10.08	0\\
10.09	0\\
10.1	0\\
10.11	0\\
10.12	0\\
10.13	0\\
10.14	0\\
10.15	0\\
10.16	0\\
10.17	0\\
10.18	0\\
10.19	0\\
10.2	0\\
10.21	0\\
10.22	0\\
10.23	0\\
10.24	0\\
10.25	0\\
10.26	0\\
10.27	0\\
10.28	0\\
10.29	0\\
10.3	0\\
10.31	0\\
10.32	0\\
10.33	0\\
10.34	0\\
10.35	0\\
10.36	0\\
10.37	0\\
10.38	0\\
10.39	0\\
10.4	0\\
10.41	0\\
10.42	0\\
10.43	0\\
10.44	0\\
10.45	0\\
10.46	0\\
10.47	0\\
10.48	0\\
10.49	0\\
10.5	0\\
10.51	0\\
10.52	0\\
10.53	0\\
10.54	0\\
10.55	0\\
10.56	0\\
10.57	0\\
10.58	0\\
10.59	0\\
10.6	0\\
10.61	0\\
10.62	0\\
10.63	0\\
10.64	0\\
10.65	0\\
10.66	0\\
10.67	0\\
10.68	0\\
10.69	0\\
10.7	0\\
10.71	0\\
10.72	0\\
10.73	0\\
10.74	0\\
10.75	0\\
10.76	0\\
10.77	0\\
10.78	0\\
10.79	0\\
10.8	0\\
10.81	0\\
10.82	0\\
10.83	0\\
10.84	0\\
10.85	0\\
10.86	0\\
10.87	0\\
10.88	0\\
10.89	0\\
10.9	0\\
10.91	0\\
10.92	0\\
10.93	0\\
10.94	0\\
10.95	0\\
10.96	0\\
10.97	0\\
10.98	0\\
10.99	0\\
11	0\\
11.01	0\\
11.02	0\\
11.03	0\\
11.04	0\\
11.05	0\\
11.06	0\\
11.07	0\\
11.08	0\\
11.09	0\\
11.1	0\\
11.11	0\\
11.12	0\\
11.13	0\\
11.14	0\\
11.15	0\\
11.16	0\\
11.17	0\\
11.18	0\\
11.19	0\\
11.2	0\\
11.21	0\\
11.22	0\\
11.23	0\\
11.24	0\\
11.25	0\\
11.26	0\\
11.27	0\\
11.28	0\\
11.29	0\\
11.3	0\\
11.31	0\\
11.32	0\\
11.33	0\\
11.34	0\\
11.35	0\\
11.36	0\\
11.37	0\\
11.38	0\\
11.39	0\\
11.4	0\\
11.41	0\\
11.42	0\\
11.43	0\\
11.44	0\\
11.45	0\\
11.46	0\\
11.47	0\\
11.48	0\\
11.49	0\\
11.5	0\\
11.51	0\\
11.52	0\\
11.53	0\\
11.54	0\\
11.55	0\\
11.56	0\\
11.57	0\\
11.58	0\\
11.59	0\\
11.6	0\\
11.61	0\\
11.62	0\\
11.63	0\\
11.64	0\\
11.65	0\\
11.66	0\\
11.67	0\\
11.68	0\\
11.69	0\\
11.7	0\\
11.71	0\\
11.72	0\\
11.73	0\\
11.74	0\\
11.75	0\\
11.76	0\\
11.77	0\\
11.78	0\\
11.79	0\\
11.8	0\\
11.81	0\\
11.82	0\\
11.83	0\\
11.84	0\\
11.85	0\\
11.86	0\\
11.87	0\\
11.88	0\\
11.89	0\\
11.9	0\\
11.91	0\\
11.92	0\\
11.93	0\\
11.94	0\\
11.95	0\\
11.96	0\\
11.97	0\\
11.98	0\\
11.99	0\\
12	0\\
12.01	0\\
12.02	0\\
12.03	0\\
12.04	0\\
12.05	0\\
12.06	0\\
12.07	0\\
12.08	0\\
12.09	0\\
12.1	0\\
12.11	0\\
12.12	0\\
12.13	0\\
12.14	0\\
12.15	0\\
12.16	0\\
12.17	0\\
12.18	0\\
12.19	0\\
12.2	0\\
12.21	0\\
12.22	0\\
12.23	0\\
12.24	0\\
12.25	0\\
12.26	0\\
12.27	0\\
12.28	0\\
12.29	0\\
12.3	0\\
12.31	0\\
12.32	0\\
12.33	0\\
12.34	0\\
12.35	0\\
12.36	0\\
12.37	0\\
12.38	0\\
12.39	0\\
12.4	0\\
12.41	0\\
12.42	0\\
12.43	0\\
12.44	0\\
12.45	0\\
12.46	0\\
12.47	0\\
12.48	0\\
12.49	0\\
12.5	0\\
12.51	0\\
12.52	0\\
12.53	0\\
12.54	0\\
12.55	0\\
12.56	0\\
12.57	0\\
12.58	0\\
12.59	0\\
12.6	0\\
12.61	0\\
12.62	0\\
12.63	0\\
12.64	0\\
12.65	0\\
12.66	0\\
12.67	0\\
12.68	0\\
12.69	0\\
12.7	0\\
12.71	0\\
12.72	0\\
12.73	0\\
12.74	0\\
12.75	0\\
12.76	0\\
12.77	0\\
12.78	0\\
12.79	0\\
12.8	0\\
12.81	0\\
12.82	0\\
12.83	0\\
12.84	0\\
12.85	0\\
12.86	0\\
12.87	0\\
12.88	0\\
12.89	0\\
12.9	0\\
12.91	0\\
12.92	0\\
12.93	0\\
12.94	0\\
12.95	0\\
12.96	0\\
12.97	0\\
12.98	0\\
12.99	0\\
13	0\\
13.01	0\\
13.02	0\\
13.03	0\\
13.04	0\\
13.05	0\\
13.06	0\\
13.07	0\\
13.08	0\\
13.09	0\\
13.1	0\\
13.11	0\\
13.12	0\\
13.13	0\\
13.14	0\\
13.15	0\\
13.16	0\\
13.17	0\\
13.18	0\\
13.19	0\\
13.2	0\\
13.21	0\\
13.22	0\\
13.23	0\\
13.24	0\\
13.25	0\\
13.26	0\\
13.27	0\\
13.28	0\\
13.29	0\\
13.3	0\\
13.31	0\\
13.32	0\\
13.33	0\\
13.34	0\\
13.35	0\\
13.36	0\\
13.37	0\\
13.38	0\\
13.39	0\\
13.4	0\\
13.41	0\\
13.42	0\\
13.43	0\\
13.44	0\\
13.45	0\\
13.46	0\\
13.47	0\\
13.48	0\\
13.49	0\\
13.5	0\\
13.51	0\\
13.52	0\\
13.53	0\\
13.54	0\\
13.55	0\\
13.56	0\\
13.57	0\\
13.58	0\\
13.59	0\\
13.6	0\\
13.61	0\\
13.62	0\\
13.63	0\\
13.64	0\\
13.65	0\\
13.66	0\\
13.67	0\\
13.68	0\\
13.69	0\\
13.7	0\\
13.71	0\\
13.72	0\\
13.73	0\\
13.74	0\\
13.75	0\\
13.76	0\\
13.77	0\\
13.78	0\\
13.79	0\\
13.8	0\\
13.81	0\\
13.82	0\\
13.83	0\\
13.84	0\\
13.85	0\\
13.86	0\\
13.87	0\\
13.88	0\\
13.89	0\\
13.9	0\\
13.91	0\\
13.92	0\\
13.93	0\\
13.94	0\\
13.95	0\\
13.96	0\\
13.97	0\\
13.98	0\\
13.99	0\\
14	0\\
14.01	0\\
14.02	0\\
14.03	0\\
14.04	0\\
14.05	0\\
14.06	0\\
14.07	0\\
14.08	0\\
14.09	0\\
14.1	0\\
14.11	0\\
14.12	0\\
14.13	0\\
14.14	0\\
14.15	0\\
14.16	0\\
14.17	0\\
14.18	0\\
14.19	0\\
14.2	0\\
14.21	0\\
14.22	0\\
14.23	0\\
14.24	0\\
14.25	0\\
14.26	0\\
14.27	0\\
14.28	0\\
14.29	0\\
14.3	0\\
14.31	0\\
14.32	0\\
14.33	0\\
14.34	0\\
14.35	0\\
14.36	0\\
14.37	0\\
14.38	0\\
14.39	0\\
14.4	0\\
14.41	0\\
14.42	0\\
14.43	0\\
14.44	0\\
14.45	0\\
14.46	0\\
14.47	0\\
14.48	0\\
14.49	0\\
14.5	0\\
14.51	0\\
14.52	0\\
14.53	0\\
14.54	0\\
14.55	0\\
14.56	0\\
14.57	0\\
14.58	0\\
14.59	0\\
14.6	0\\
14.61	0\\
14.62	0\\
14.63	0\\
14.64	0\\
14.65	0\\
14.66	0\\
14.67	0\\
14.68	0\\
14.69	0\\
14.7	0\\
14.71	0\\
14.72	0\\
14.73	0\\
14.74	0\\
14.75	0\\
14.76	0\\
14.77	0\\
14.78	0\\
14.79	0\\
14.8	0\\
14.81	0\\
14.82	0\\
14.83	0\\
14.84	0\\
14.85	0\\
14.86	0\\
14.87	0\\
14.88	0\\
14.89	0\\
14.9	0\\
14.91	0\\
14.92	0\\
14.93	0\\
14.94	0\\
14.95	0\\
14.96	0\\
14.97	0\\
14.98	0\\
14.99	0\\
15	0\\
15.01	0\\
15.02	0\\
15.03	0\\
15.04	0\\
15.05	0\\
15.06	0\\
15.07	0\\
15.08	0\\
15.09	0\\
15.1	0\\
15.11	0\\
15.12	0\\
15.13	0\\
15.14	0\\
15.15	0\\
15.16	0\\
15.17	0\\
15.18	0\\
15.19	0\\
15.2	0\\
15.21	0\\
15.22	0\\
15.23	0\\
15.24	0\\
15.25	0\\
15.26	0\\
15.27	0\\
15.28	0\\
15.29	0\\
15.3	0\\
15.31	0\\
15.32	0\\
15.33	0\\
15.34	0\\
15.35	0\\
15.36	0\\
15.37	0\\
15.38	0\\
15.39	0\\
15.4	0\\
15.41	0\\
15.42	0\\
15.43	0\\
15.44	0\\
15.45	0\\
15.46	0\\
15.47	0\\
15.48	0\\
15.49	0\\
15.5	0\\
15.51	0\\
15.52	0\\
15.53	0\\
15.54	0\\
15.55	0\\
15.56	0\\
15.57	0\\
15.58	0\\
15.59	0\\
15.6	0\\
15.61	0\\
15.62	0\\
15.63	0\\
15.64	0\\
15.65	0\\
15.66	0\\
15.67	0\\
15.68	0\\
15.69	0\\
15.7	0\\
15.71	0\\
15.72	0\\
15.73	0\\
15.74	0\\
15.75	0\\
15.76	0\\
15.77	0\\
15.78	0\\
15.79	0\\
15.8	0\\
15.81	0\\
15.82	0\\
15.83	0\\
15.84	0\\
15.85	0\\
15.86	0\\
15.87	0\\
15.88	0\\
15.89	0\\
15.9	0\\
15.91	0\\
15.92	0\\
15.93	0\\
15.94	0\\
15.95	0\\
15.96	0\\
15.97	0\\
15.98	0\\
15.99	0\\
16	0\\
16.01	0\\
16.02	0\\
16.03	0\\
16.04	0\\
16.05	0\\
16.06	0\\
16.07	0\\
16.08	0\\
16.09	0\\
16.1	0\\
16.11	0\\
16.12	0\\
16.13	0\\
16.14	0\\
16.15	0\\
16.16	0\\
16.17	0\\
16.18	0\\
16.19	0\\
16.2	0\\
16.21	0\\
16.22	0\\
16.23	0\\
16.24	0\\
16.25	0\\
16.26	0\\
16.27	0\\
16.28	0\\
16.29	0\\
16.3	0\\
16.31	0\\
16.32	0\\
16.33	0\\
16.34	0\\
16.35	0\\
16.36	0\\
16.37	0\\
16.38	0\\
16.39	0\\
16.4	0\\
16.41	0\\
16.42	0\\
16.43	0\\
16.44	0\\
16.45	0\\
16.46	0\\
16.47	0\\
16.48	0\\
16.49	0\\
16.5	0\\
16.51	0\\
16.52	0\\
16.53	0\\
16.54	0\\
16.55	0\\
16.56	0\\
16.57	0\\
16.58	0\\
16.59	0\\
16.6	0\\
16.61	0\\
16.62	0\\
16.63	0\\
16.64	0\\
16.65	0\\
16.66	0\\
16.67	0\\
16.68	0\\
16.69	0\\
16.7	0\\
16.71	0\\
16.72	0\\
16.73	0\\
16.74	0\\
16.75	0\\
16.76	0\\
16.77	0\\
16.78	0\\
16.79	0\\
16.8	0\\
16.81	0\\
16.82	0\\
16.83	0\\
16.84	0\\
16.85	0\\
16.86	0\\
16.87	0\\
16.88	0\\
16.89	0\\
16.9	0\\
16.91	0\\
16.92	0\\
16.93	0\\
16.94	0\\
16.95	0\\
16.96	0\\
16.97	0\\
16.98	0\\
16.99	0\\
17	0\\
17.01	0\\
17.02	0\\
17.03	0\\
17.04	0\\
17.05	0\\
17.06	0\\
17.07	0\\
17.08	0\\
17.09	0\\
17.1	0\\
17.11	0\\
17.12	0\\
17.13	0\\
17.14	0\\
17.15	0\\
17.16	0\\
17.17	0\\
17.18	0\\
17.19	0\\
17.2	0\\
17.21	0\\
17.22	0\\
17.23	0\\
17.24	0\\
17.25	0\\
17.26	0\\
17.27	0\\
17.28	0\\
17.29	0\\
17.3	0\\
17.31	0\\
17.32	0\\
17.33	0\\
17.34	0\\
17.35	0\\
17.36	0\\
17.37	0\\
17.38	0\\
17.39	0\\
17.4	0\\
17.41	0\\
17.42	0\\
17.43	0\\
17.44	0\\
17.45	0\\
17.46	0\\
17.47	0\\
17.48	0\\
17.49	0\\
17.5	0\\
17.51	0\\
17.52	0\\
17.53	0\\
17.54	0\\
17.55	0\\
17.56	0\\
17.57	0\\
17.58	0\\
17.59	0\\
17.6	0\\
17.61	0\\
17.62	0\\
17.63	0\\
17.64	0\\
17.65	0\\
17.66	0\\
17.67	0\\
17.68	0\\
17.69	0\\
17.7	0\\
17.71	0\\
17.72	0\\
17.73	0\\
17.74	0\\
17.75	0\\
17.76	0\\
17.77	0\\
17.78	0\\
17.79	0\\
17.8	0\\
17.81	0\\
17.82	0\\
17.83	0\\
17.84	0\\
17.85	0\\
17.86	0\\
17.87	0\\
17.88	0\\
17.89	0\\
17.9	0\\
17.91	0\\
17.92	0\\
17.93	0\\
17.94	0\\
17.95	0\\
17.96	0\\
17.97	0\\
17.98	0\\
17.99	0\\
18	0\\
18.01	0\\
18.02	0\\
18.03	0\\
18.04	0\\
18.05	0\\
18.06	0\\
18.07	0\\
18.08	0\\
18.09	0\\
18.1	0\\
18.11	0\\
18.12	0\\
18.13	0\\
18.14	0\\
18.15	0\\
18.16	0\\
18.17	0\\
18.18	0\\
18.19	0\\
18.2	0\\
18.21	0\\
18.22	0\\
18.23	0\\
18.24	0\\
18.25	0\\
18.26	0\\
18.27	0\\
18.28	0\\
18.29	0\\
18.3	0\\
18.31	0\\
18.32	0\\
18.33	0\\
18.34	0\\
18.35	0\\
18.36	0\\
18.37	0\\
18.38	0\\
18.39	0\\
18.4	0\\
18.41	0\\
18.42	0\\
18.43	0\\
18.44	0\\
18.45	0\\
18.46	0\\
18.47	0\\
18.48	0\\
18.49	0\\
18.5	0\\
18.51	0\\
18.52	0\\
18.53	0\\
18.54	0\\
18.55	0\\
18.56	0\\
18.57	0\\
18.58	0\\
18.59	0\\
18.6	0\\
18.61	0\\
18.62	0\\
18.63	0\\
18.64	0\\
18.65	0\\
18.66	0\\
18.67	0\\
18.68	0\\
18.69	0\\
18.7	0\\
18.71	0\\
18.72	0\\
18.73	0\\
18.74	0\\
18.75	0\\
18.76	0\\
18.77	0\\
18.78	0\\
18.79	0\\
18.8	0\\
18.81	0\\
18.82	0\\
18.83	0\\
18.84	0\\
18.85	0\\
18.86	0\\
18.87	0\\
18.88	0\\
18.89	0\\
18.9	0\\
18.91	0\\
18.92	0\\
18.93	0\\
18.94	0\\
18.95	0\\
18.96	0\\
18.97	0\\
18.98	0\\
18.99	0\\
19	0\\
19.01	0\\
19.02	0\\
19.03	0\\
19.04	0\\
19.05	0\\
19.06	0\\
19.07	0\\
19.08	0\\
19.09	0\\
19.1	0\\
19.11	0\\
19.12	0\\
19.13	0\\
19.14	0\\
19.15	0\\
19.16	0\\
19.17	0\\
19.18	0\\
19.19	0\\
19.2	0\\
19.21	0\\
19.22	0\\
19.23	0\\
19.24	0\\
19.25	0\\
19.26	0\\
19.27	0\\
19.28	0\\
19.29	0\\
19.3	0\\
19.31	0\\
19.32	0\\
19.33	0\\
19.34	0\\
19.35	0\\
19.36	0\\
19.37	0\\
19.38	0\\
19.39	0\\
19.4	0\\
19.41	0\\
19.42	0\\
19.43	0\\
19.44	0\\
19.45	0\\
19.46	0\\
19.47	0\\
19.48	0\\
19.49	0\\
19.5	0\\
19.51	0\\
19.52	0\\
19.53	0\\
19.54	0\\
19.55	0\\
19.56	0\\
19.57	0\\
19.58	0\\
19.59	0\\
19.6	0\\
19.61	0\\
19.62	0\\
19.63	0\\
19.64	0\\
19.65	0\\
19.66	0\\
19.67	0\\
19.68	0\\
19.69	0\\
19.7	0\\
19.71	0\\
19.72	0\\
19.73	0\\
19.74	0\\
19.75	0\\
19.76	0\\
19.77	0\\
19.78	0\\
19.79	0\\
19.8	0\\
19.81	0\\
19.82	0\\
19.83	0\\
19.84	0\\
19.85	0\\
19.86	0\\
19.87	0\\
19.88	0\\
19.89	0\\
19.9	0\\
19.91	0\\
19.92	0\\
19.93	0\\
19.94	0\\
19.95	0\\
19.96	0\\
19.97	0\\
19.98	0\\
19.99	0\\
20	0\\
20.01	0\\
20.02	0\\
20.03	0\\
20.04	0\\
20.05	0\\
20.06	0\\
20.07	0\\
20.08	0\\
20.09	0\\
20.1	0\\
20.11	0\\
20.12	0\\
20.13	0\\
20.14	0\\
20.15	0\\
20.16	0\\
20.17	0\\
20.18	0\\
20.19	0\\
20.2	0\\
20.21	0\\
20.22	0\\
20.23	0\\
20.24	0\\
20.25	0\\
20.26	0\\
20.27	0\\
20.28	0\\
20.29	0\\
20.3	0\\
20.31	0\\
20.32	0\\
20.33	0\\
20.34	0\\
20.35	0\\
20.36	0\\
20.37	0\\
20.38	0\\
20.39	0\\
20.4	0\\
20.41	0\\
20.42	0\\
20.43	0\\
20.44	0\\
20.45	0\\
20.46	0\\
20.47	0\\
20.48	0\\
20.49	0\\
20.5	0\\
20.51	0\\
20.52	0\\
20.53	0\\
20.54	0\\
20.55	0\\
20.56	0\\
20.57	0\\
20.58	0\\
20.59	0\\
20.6	0\\
20.61	0\\
20.62	0\\
20.63	0\\
20.64	0\\
20.65	0\\
20.66	0\\
20.67	0\\
20.68	0\\
20.69	0\\
20.7	0\\
20.71	0\\
20.72	0\\
20.73	0\\
20.74	0\\
20.75	0\\
20.76	0\\
20.77	0\\
20.78	0\\
20.79	0\\
20.8	0\\
20.81	0\\
20.82	0\\
20.83	0\\
20.84	0\\
20.85	0\\
20.86	0\\
20.87	0\\
20.88	0\\
20.89	0\\
20.9	0\\
20.91	0\\
20.92	0\\
20.93	0\\
20.94	0\\
20.95	0\\
20.96	0\\
20.97	0\\
20.98	0\\
20.99	0\\
21	0\\
21.01	0\\
21.02	0\\
21.03	0\\
21.04	0\\
21.05	0\\
21.06	0\\
21.07	0\\
21.08	0\\
21.09	0\\
21.1	0\\
21.11	0\\
21.12	0\\
21.13	0\\
21.14	0\\
21.15	0\\
21.16	0\\
21.17	0\\
21.18	0\\
21.19	0\\
21.2	0\\
21.21	0\\
21.22	0\\
21.23	0\\
21.24	0\\
21.25	0\\
21.26	0\\
21.27	0\\
21.28	0\\
21.29	0\\
21.3	0\\
21.31	0\\
21.32	0\\
21.33	0\\
21.34	0\\
21.35	0\\
21.36	0\\
21.37	0\\
21.38	0\\
21.39	0\\
21.4	0\\
21.41	0\\
21.42	0\\
21.43	0\\
21.44	0\\
21.45	0\\
21.46	0\\
21.47	0\\
21.48	0\\
21.49	0\\
21.5	0\\
21.51	0\\
21.52	0\\
21.53	0\\
21.54	0\\
21.55	0\\
21.56	0\\
21.57	0\\
21.58	0\\
21.59	0\\
21.6	0\\
21.61	0\\
21.62	0\\
21.63	0\\
21.64	0\\
21.65	0\\
21.66	0\\
21.67	0\\
21.68	0\\
21.69	0\\
21.7	0\\
21.71	0\\
21.72	0\\
21.73	0\\
21.74	0\\
21.75	0\\
21.76	0\\
21.77	0\\
21.78	0\\
21.79	0\\
21.8	0\\
21.81	0\\
21.82	0\\
21.83	0\\
21.84	0\\
21.85	0\\
21.86	0\\
21.87	0\\
21.88	0\\
21.89	0\\
21.9	0\\
21.91	0\\
21.92	0\\
21.93	0\\
21.94	0\\
21.95	0\\
21.96	0\\
21.97	0\\
21.98	0\\
21.99	0\\
22	0\\
22.01	0\\
22.02	0\\
22.03	0\\
22.04	0\\
22.05	0\\
22.06	0\\
22.07	0\\
22.08	0\\
22.09	0\\
22.1	0\\
22.11	0\\
22.12	0\\
22.13	0\\
22.14	0\\
22.15	0\\
22.16	0\\
22.17	0\\
22.18	0\\
22.19	0\\
22.2	0\\
22.21	0\\
22.22	0\\
22.23	0\\
22.24	0\\
22.25	0\\
22.26	0\\
22.27	0\\
22.28	0\\
22.29	0\\
22.3	0\\
22.31	0\\
22.32	0\\
22.33	0\\
22.34	0\\
22.35	0\\
22.36	0\\
22.37	0\\
22.38	0\\
22.39	0\\
22.4	0\\
22.41	0\\
22.42	0\\
22.43	0\\
22.44	0\\
22.45	0\\
22.46	0\\
22.47	0\\
22.48	0\\
22.49	0\\
22.5	0\\
22.51	0\\
22.52	0\\
22.53	0\\
22.54	0\\
22.55	0\\
22.56	0\\
22.57	0\\
22.58	0\\
22.59	0\\
22.6	0\\
22.61	0\\
22.62	0\\
22.63	0\\
22.64	0\\
22.65	0\\
22.66	0\\
22.67	0\\
22.68	0\\
22.69	0\\
22.7	0\\
22.71	0\\
22.72	0\\
22.73	0\\
22.74	0\\
22.75	0\\
22.76	0\\
22.77	0\\
22.78	0\\
22.79	0\\
22.8	0\\
22.81	0\\
22.82	0\\
22.83	0\\
22.84	0\\
22.85	0\\
22.86	0\\
22.87	0\\
22.88	0\\
22.89	0\\
22.9	0\\
22.91	0\\
22.92	0\\
22.93	0\\
22.94	0\\
22.95	0\\
22.96	0\\
22.97	0\\
22.98	0\\
22.99	0\\
23	0\\
23.01	0\\
23.02	0\\
23.03	0\\
23.04	0\\
23.05	0\\
23.06	0\\
23.07	0\\
23.08	0\\
23.09	0\\
23.1	0\\
23.11	0\\
23.12	0\\
23.13	0\\
23.14	0\\
23.15	0\\
23.16	0\\
23.17	0\\
23.18	0\\
23.19	0\\
23.2	0\\
23.21	0\\
23.22	0\\
23.23	0\\
23.24	0\\
23.25	0\\
23.26	0\\
23.27	0\\
23.28	0\\
23.29	0\\
23.3	0\\
23.31	0\\
23.32	0\\
23.33	0\\
23.34	0\\
23.35	0\\
23.36	0\\
23.37	0\\
23.38	0\\
23.39	0\\
23.4	0\\
23.41	0\\
23.42	0\\
23.43	0\\
23.44	0\\
23.45	0\\
23.46	0\\
23.47	0\\
23.48	0\\
23.49	0\\
23.5	0\\
23.51	0\\
23.52	0\\
23.53	0\\
23.54	0\\
23.55	0\\
23.56	0\\
23.57	0\\
23.58	0\\
23.59	0\\
23.6	0\\
23.61	0\\
23.62	0\\
23.63	0\\
23.64	0\\
23.65	0\\
23.66	0\\
23.67	0\\
23.68	0\\
23.69	0\\
23.7	0\\
23.71	0\\
23.72	0\\
23.73	0\\
23.74	0\\
23.75	0\\
23.76	0\\
23.77	0\\
23.78	0\\
23.79	0\\
23.8	0\\
23.81	0\\
23.82	0\\
23.83	0\\
23.84	0\\
23.85	0\\
23.86	0\\
23.87	0\\
23.88	0\\
23.89	0\\
23.9	0\\
23.91	0\\
23.92	0\\
23.93	0\\
23.94	0\\
23.95	0\\
23.96	0\\
23.97	0\\
23.98	0\\
23.99	0\\
24	0\\
24.01	0\\
24.02	0\\
24.03	0\\
24.04	0\\
24.05	0\\
24.06	0\\
24.07	0\\
24.08	0\\
24.09	0\\
24.1	0\\
24.11	0\\
24.12	0\\
24.13	0\\
24.14	0\\
24.15	0\\
24.16	0\\
24.17	0\\
24.18	0\\
24.19	0\\
24.2	0\\
24.21	0\\
24.22	0\\
24.23	0\\
24.24	0\\
24.25	0\\
24.26	0\\
24.27	0\\
24.28	0\\
24.29	0\\
24.3	0\\
24.31	0\\
24.32	0\\
24.33	0\\
24.34	0\\
24.35	0\\
24.36	0\\
24.37	0\\
24.38	0\\
24.39	0\\
24.4	0\\
24.41	0\\
24.42	0\\
24.43	0\\
24.44	0\\
24.45	0\\
24.46	0\\
24.47	0\\
24.48	0\\
24.49	0\\
24.5	0\\
24.51	0\\
24.52	0\\
24.53	0\\
24.54	0\\
24.55	0\\
24.56	0\\
24.57	0\\
24.58	0\\
24.59	0\\
24.6	0\\
24.61	0\\
24.62	0\\
24.63	0\\
24.64	0\\
24.65	0\\
24.66	0\\
24.67	0\\
24.68	0\\
24.69	0\\
24.7	0\\
24.71	0\\
24.72	0\\
24.73	0\\
24.74	0\\
24.75	0\\
24.76	0\\
24.77	0\\
24.78	0\\
24.79	0\\
24.8	0\\
24.81	0\\
24.82	0\\
24.83	0\\
24.84	0\\
24.85	0\\
24.86	0\\
24.87	0\\
24.88	0\\
24.89	0\\
24.9	0\\
24.91	0\\
24.92	0\\
24.93	0\\
24.94	0\\
24.95	0\\
24.96	0\\
24.97	0\\
24.98	0\\
24.99	0\\
25	0\\
25.01	0\\
25.02	0\\
25.03	0\\
25.04	0\\
25.05	0\\
25.06	0\\
25.07	0\\
25.08	0\\
25.09	0\\
25.1	0\\
25.11	0\\
25.12	0\\
25.13	0\\
25.14	0\\
25.15	0\\
25.16	0\\
25.17	0\\
25.18	0\\
25.19	0\\
25.2	0\\
25.21	0\\
25.22	0\\
25.23	0\\
25.24	0\\
25.25	0\\
25.26	0\\
25.27	0\\
25.28	0\\
25.29	0\\
25.3	0\\
25.31	0\\
25.32	0\\
25.33	0\\
25.34	0\\
25.35	0\\
25.36	0\\
25.37	0\\
25.38	0\\
25.39	0\\
25.4	0\\
25.41	0\\
25.42	0\\
25.43	0\\
25.44	0\\
25.45	0\\
25.46	0\\
25.47	0\\
25.48	0\\
25.49	0\\
25.5	0\\
25.51	0\\
25.52	0\\
25.53	0\\
25.54	0\\
25.55	0\\
25.56	0\\
25.57	0\\
25.58	0\\
25.59	0\\
25.6	0\\
25.61	0\\
25.62	0\\
25.63	0\\
25.64	0\\
25.65	0\\
25.66	0\\
25.67	0\\
25.68	0\\
25.69	0\\
25.7	0\\
25.71	0\\
25.72	0\\
25.73	0\\
25.74	0\\
25.75	0\\
25.76	0\\
25.77	0\\
25.78	0\\
25.79	0\\
25.8	0\\
25.81	0\\
25.82	0\\
25.83	0\\
25.84	0\\
25.85	0\\
25.86	0\\
25.87	0\\
25.88	0\\
25.89	0\\
25.9	0\\
25.91	0\\
25.92	0\\
25.93	0\\
25.94	0\\
25.95	0\\
25.96	0\\
25.97	0\\
25.98	0\\
25.99	0\\
26	0\\
26.01	0\\
26.02	0\\
26.03	0\\
26.04	0\\
26.05	0\\
26.06	0\\
26.07	0\\
26.08	0\\
26.09	0\\
26.1	0\\
26.11	0\\
26.12	0\\
26.13	0\\
26.14	0\\
26.15	0\\
26.16	0\\
26.17	0\\
26.18	0\\
26.19	0\\
26.2	0\\
26.21	0\\
26.22	0\\
26.23	0\\
26.24	0\\
26.25	0\\
26.26	0\\
26.27	0\\
26.28	0\\
26.29	0\\
26.3	0\\
26.31	0\\
26.32	0\\
26.33	0\\
26.34	0\\
26.35	0\\
26.36	0\\
26.37	0\\
26.38	0\\
26.39	0\\
26.4	0\\
26.41	0\\
26.42	0\\
26.43	0\\
26.44	0\\
26.45	0\\
26.46	0\\
26.47	0\\
26.48	0\\
26.49	0\\
26.5	0\\
26.51	0\\
26.52	0\\
26.53	0\\
26.54	0\\
26.55	0\\
26.56	0\\
26.57	0\\
26.58	0\\
26.59	0\\
26.6	0\\
26.61	0\\
26.62	0\\
26.63	0\\
26.64	0\\
26.65	0\\
26.66	0\\
26.67	0\\
26.68	0\\
26.69	0\\
26.7	0\\
26.71	0\\
26.72	0\\
26.73	0\\
26.74	0\\
26.75	0\\
26.76	0\\
26.77	0\\
26.78	0\\
26.79	0\\
26.8	0\\
26.81	0\\
26.82	0\\
26.83	0\\
26.84	0\\
26.85	0\\
26.86	0\\
26.87	0\\
26.88	0\\
26.89	0\\
26.9	0\\
26.91	0\\
26.92	0\\
26.93	0\\
26.94	0\\
26.95	0\\
26.96	0\\
26.97	0\\
26.98	0\\
26.99	0\\
27	0\\
27.01	0\\
27.02	0\\
27.03	0\\
27.04	0\\
27.05	0\\
27.06	0\\
27.07	0\\
27.08	0\\
27.09	0\\
27.1	0\\
27.11	0\\
27.12	0\\
27.13	0\\
27.14	0\\
27.15	0\\
27.16	0\\
27.17	0\\
27.18	0\\
27.19	0\\
27.2	0\\
27.21	0\\
27.22	0\\
27.23	0\\
27.24	0\\
27.25	0\\
27.26	0\\
27.27	0\\
27.28	0\\
27.29	0\\
27.3	0\\
27.31	0\\
27.32	0\\
27.33	0\\
27.34	0\\
27.35	0\\
27.36	0\\
27.37	0\\
27.38	0\\
27.39	0\\
27.4	0\\
27.41	0\\
27.42	0\\
27.43	0\\
27.44	0\\
27.45	0\\
27.46	0\\
27.47	0\\
27.48	0\\
27.49	0\\
27.5	0\\
27.51	0\\
27.52	0\\
27.53	0\\
27.54	0\\
27.55	0\\
27.56	0\\
27.57	0\\
27.58	0\\
27.59	0\\
27.6	0\\
27.61	0\\
27.62	0\\
27.63	0\\
27.64	0\\
27.65	0\\
27.66	0\\
27.67	0\\
27.68	0\\
27.69	0\\
27.7	0\\
27.71	0\\
27.72	0\\
27.73	0\\
27.74	0\\
27.75	0\\
27.76	0\\
27.77	0\\
27.78	0\\
27.79	0\\
27.8	0\\
27.81	0\\
27.82	0\\
27.83	0\\
27.84	0\\
27.85	0\\
27.86	0\\
27.87	0\\
27.88	0\\
27.89	0\\
27.9	0\\
27.91	0\\
27.92	0\\
27.93	0\\
27.94	0\\
27.95	0\\
27.96	0\\
27.97	0\\
27.98	0\\
27.99	0\\
28	0\\
28.01	0\\
28.02	0\\
28.03	0\\
28.04	0\\
28.05	0\\
28.06	0\\
28.07	0\\
28.08	0\\
28.09	0\\
28.1	0\\
28.11	0\\
28.12	0\\
28.13	0\\
28.14	0\\
28.15	0\\
28.16	0\\
28.17	0\\
28.18	0\\
28.19	0\\
28.2	0\\
28.21	0\\
28.22	0\\
28.23	0\\
28.24	0\\
28.25	0\\
28.26	0\\
28.27	0\\
28.28	0\\
28.29	0\\
28.3	0\\
28.31	0\\
28.32	0\\
28.33	0\\
28.34	0\\
28.35	0\\
28.36	0\\
28.37	0\\
28.38	0\\
28.39	0\\
28.4	0\\
28.41	0\\
28.42	0\\
28.43	0\\
28.44	0\\
28.45	0\\
28.46	0\\
28.47	0\\
28.48	0\\
28.49	0\\
28.5	0\\
28.51	0\\
28.52	0\\
28.53	0\\
28.54	0\\
28.55	0\\
28.56	0\\
28.57	0\\
28.58	0\\
28.59	0\\
28.6	0\\
28.61	0\\
28.62	0\\
28.63	0\\
28.64	0\\
28.65	0\\
28.66	0\\
28.67	0\\
28.68	0\\
28.69	0\\
28.7	0\\
28.71	0\\
28.72	0\\
28.73	0\\
28.74	0\\
28.75	0\\
28.76	0\\
28.77	0\\
28.78	0\\
28.79	0\\
28.8	0\\
28.81	0\\
28.82	0\\
28.83	0\\
28.84	0\\
28.85	0\\
28.86	0\\
28.87	0\\
28.88	0\\
28.89	0\\
28.9	0\\
28.91	0\\
28.92	0\\
28.93	0\\
28.94	0\\
28.95	0\\
28.96	0\\
28.97	0\\
28.98	0\\
28.99	0\\
29	0\\
29.01	0\\
29.02	0\\
29.03	0\\
29.04	0\\
29.05	0\\
29.06	0\\
29.07	0\\
29.08	0\\
29.09	0\\
29.1	0\\
29.11	0\\
29.12	0\\
29.13	0\\
29.14	0\\
29.15	0\\
29.16	0\\
29.17	0\\
29.18	0\\
29.19	0\\
29.2	0\\
29.21	0\\
29.22	0\\
29.23	0\\
29.24	0\\
29.25	0\\
29.26	0\\
29.27	0\\
29.28	0\\
29.29	0\\
29.3	0\\
29.31	0\\
29.32	0\\
29.33	0\\
29.34	0\\
29.35	0\\
29.36	0\\
29.37	0\\
29.38	0\\
29.39	0\\
29.4	0\\
29.41	0\\
29.42	0\\
29.43	0\\
29.44	0\\
29.45	0\\
29.46	0\\
29.47	0\\
29.48	0\\
29.49	0\\
29.5	0\\
29.51	0\\
29.52	0\\
29.53	0\\
29.54	0\\
29.55	0\\
29.56	0\\
29.57	0\\
29.58	0\\
29.59	0\\
29.6	0\\
29.61	0\\
29.62	0\\
29.63	0\\
29.64	0\\
29.65	0\\
29.66	0\\
29.67	0\\
29.68	0\\
29.69	0\\
29.7	0\\
29.71	0\\
29.72	0\\
29.73	0\\
29.74	0\\
29.75	0\\
29.76	0\\
29.77	0\\
29.78	0\\
29.79	0\\
29.8	0\\
29.81	0\\
29.82	0\\
29.83	0\\
29.84	0\\
29.85	0\\
29.86	0\\
29.87	0\\
29.88	0\\
29.89	0\\
29.9	0\\
29.91	0\\
29.92	0\\
29.93	0\\
29.94	0\\
29.95	0\\
29.96	0\\
29.97	0\\
29.98	0\\
29.99	0\\
30	0\\
30.01	0\\
30.02	0\\
30.03	0\\
30.04	0\\
30.05	0\\
30.06	0\\
30.07	0\\
30.08	0\\
30.09	0\\
30.1	0\\
30.11	0\\
30.12	0\\
30.13	0\\
30.14	0\\
30.15	0\\
30.16	0\\
30.17	0\\
30.18	0\\
30.19	0\\
30.2	0\\
30.21	0\\
30.22	0\\
30.23	0\\
30.24	0\\
30.25	0\\
30.26	0\\
30.27	0\\
30.28	0\\
30.29	0\\
30.3	0\\
30.31	0\\
30.32	0\\
30.33	0\\
30.34	0\\
30.35	0\\
30.36	0\\
30.37	0\\
30.38	0\\
30.39	0\\
30.4	0\\
30.41	0\\
30.42	0\\
30.43	0\\
30.44	0\\
30.45	0\\
30.46	0\\
30.47	0\\
30.48	0\\
30.49	0\\
30.5	0\\
30.51	0\\
30.52	0\\
30.53	0\\
30.54	0\\
30.55	0\\
30.56	0\\
30.57	0\\
30.58	0\\
30.59	0\\
30.6	0\\
30.61	0\\
30.62	0\\
30.63	0\\
30.64	0\\
30.65	0\\
30.66	0\\
30.67	0\\
30.68	0\\
30.69	0\\
30.7	0\\
30.71	0\\
30.72	0\\
30.73	0\\
30.74	0\\
30.75	0\\
30.76	0\\
30.77	0\\
30.78	0\\
30.79	0\\
30.8	0\\
30.81	0\\
30.82	0\\
30.83	0\\
30.84	0\\
30.85	0\\
30.86	0\\
30.87	0\\
30.88	0\\
30.89	0\\
30.9	0\\
30.91	0\\
30.92	0\\
30.93	0\\
30.94	0\\
30.95	0\\
30.96	0\\
30.97	0\\
30.98	0\\
30.99	0\\
31	0\\
31.01	0\\
31.02	0\\
31.03	0\\
31.04	0\\
31.05	0\\
31.06	0\\
31.07	0\\
31.08	0\\
31.09	0\\
31.1	0\\
31.11	0\\
31.12	0\\
31.13	0\\
31.14	0\\
31.15	0\\
31.16	0\\
31.17	0\\
31.18	0\\
31.19	0\\
31.2	0\\
31.21	0\\
31.22	0\\
31.23	0\\
31.24	0\\
31.25	0\\
31.26	0\\
31.27	0\\
31.28	0\\
31.29	0\\
31.3	0\\
31.31	0\\
31.32	0\\
31.33	0\\
31.34	0\\
31.35	0\\
31.36	0\\
31.37	0\\
31.38	0\\
31.39	0\\
31.4	0\\
31.41	0\\
31.42	0\\
31.43	0\\
31.44	0\\
31.45	0\\
31.46	0\\
31.47	0\\
31.48	0\\
31.49	0\\
31.5	0\\
31.51	0\\
31.52	0\\
31.53	0\\
31.54	0\\
31.55	0\\
31.56	0\\
31.57	0\\
31.58	0\\
31.59	0\\
31.6	0\\
31.61	0\\
31.62	0\\
31.63	0\\
31.64	0\\
31.65	0\\
31.66	0\\
31.67	0\\
31.68	0\\
31.69	0\\
31.7	0\\
31.71	0\\
31.72	0\\
31.73	0\\
31.74	0\\
31.75	0\\
31.76	0\\
31.77	0\\
31.78	0\\
31.79	0\\
31.8	0\\
31.81	0\\
31.82	0\\
31.83	0\\
31.84	0\\
31.85	0\\
31.86	0\\
31.87	0\\
31.88	0\\
31.89	0\\
31.9	0\\
31.91	0\\
31.92	0\\
31.93	0\\
31.94	0\\
31.95	0\\
31.96	0\\
31.97	0\\
31.98	0\\
31.99	0\\
32	0\\
32.01	0\\
32.02	0\\
32.03	0\\
32.04	0\\
32.05	0\\
32.06	0\\
32.07	0\\
32.08	0\\
32.09	0\\
32.1	0\\
32.11	0\\
32.12	0\\
32.13	0\\
32.14	0\\
32.15	0\\
32.16	0\\
32.17	0\\
32.18	0\\
32.19	0\\
32.2	0\\
32.21	0\\
32.22	0\\
32.23	0\\
32.24	0\\
32.25	0\\
32.26	0\\
32.27	0\\
32.28	0\\
32.29	0\\
32.3	0\\
32.31	0\\
32.32	0\\
32.33	0\\
32.34	0\\
32.35	0\\
32.36	0\\
32.37	0\\
32.38	0\\
32.39	0\\
32.4	0\\
32.41	0\\
32.42	0\\
32.43	0\\
32.44	0\\
32.45	0\\
32.46	0\\
32.47	0\\
32.48	0\\
32.49	0\\
32.5	0\\
32.51	0\\
32.52	0\\
32.53	0\\
32.54	0\\
32.55	0\\
32.56	0\\
32.57	0\\
32.58	0\\
32.59	0\\
32.6	0\\
32.61	0\\
32.62	0\\
32.63	0\\
32.64	0\\
32.65	0\\
32.66	0\\
32.67	0\\
32.68	0\\
32.69	0\\
32.7	0\\
32.71	0\\
32.72	0\\
32.73	0\\
32.74	0\\
32.75	0\\
32.76	0\\
32.77	0\\
32.78	0\\
32.79	0\\
32.8	0\\
32.81	0\\
32.82	0\\
32.83	0\\
32.84	0\\
32.85	0\\
32.86	0\\
32.87	0\\
32.88	0\\
32.89	0\\
32.9	0\\
32.91	0\\
32.92	0\\
32.93	0\\
32.94	0\\
32.95	0\\
32.96	0\\
32.97	0\\
32.98	0\\
32.99	0\\
33	0\\
33.01	0\\
33.02	0\\
33.03	0\\
33.04	0\\
33.05	0\\
33.06	0\\
33.07	0\\
33.08	0\\
33.09	0\\
33.1	0\\
33.11	0\\
33.12	0\\
33.13	0\\
33.14	0\\
33.15	0\\
33.16	0\\
33.17	0\\
33.18	0\\
33.19	0\\
33.2	0\\
33.21	0\\
33.22	0\\
33.23	0\\
33.24	0\\
33.25	0\\
33.26	0\\
33.27	0\\
33.28	0\\
33.29	0\\
33.3	0\\
33.31	0\\
33.32	0\\
33.33	0\\
33.34	0\\
33.35	0\\
33.36	0\\
33.37	0\\
33.38	0\\
33.39	0\\
33.4	0\\
33.41	0\\
33.42	0\\
33.43	0\\
33.44	0\\
33.45	0\\
33.46	0\\
33.47	0\\
33.48	0\\
33.49	0\\
33.5	0\\
33.51	0\\
33.52	0\\
33.53	0\\
33.54	0\\
33.55	0\\
33.56	0\\
33.57	0\\
33.58	0\\
33.59	0\\
33.6	0\\
33.61	0\\
33.62	0\\
33.63	0\\
33.64	0\\
33.65	0\\
33.66	0\\
33.67	0\\
33.68	0\\
33.69	0\\
33.7	0\\
33.71	0\\
33.72	0\\
33.73	0\\
33.74	0\\
33.75	0\\
33.76	0\\
33.77	0\\
33.78	0\\
33.79	0\\
33.8	0\\
33.81	0\\
33.82	0\\
33.83	0\\
33.84	0\\
33.85	0\\
33.86	0\\
33.87	0\\
33.88	0\\
33.89	0\\
33.9	0\\
33.91	0\\
33.92	0\\
33.93	0\\
33.94	0\\
33.95	0\\
33.96	0\\
33.97	0\\
33.98	0\\
33.99	0\\
34	0\\
34.01	0\\
34.02	0\\
34.03	0\\
34.04	0\\
34.05	0\\
34.06	0\\
34.07	0\\
34.08	0\\
34.09	0\\
34.1	0\\
34.11	0\\
34.12	0\\
34.13	0\\
34.14	0\\
34.15	0\\
34.16	0\\
34.17	0\\
34.18	0\\
34.19	0\\
34.2	0\\
34.21	0\\
34.22	0\\
34.23	0\\
34.24	0\\
34.25	0\\
34.26	0\\
34.27	0\\
34.28	0\\
34.29	0\\
34.3	0\\
34.31	0\\
34.32	0\\
34.33	0\\
34.34	0\\
34.35	0\\
34.36	0\\
34.37	0\\
34.38	0\\
34.39	0\\
34.4	0\\
34.41	0\\
34.42	0\\
34.43	0\\
34.44	0\\
34.45	0\\
34.46	0\\
34.47	0\\
34.48	0\\
34.49	0\\
34.5	0\\
34.51	0\\
34.52	0\\
34.53	0\\
34.54	0\\
34.55	0\\
34.56	0\\
34.57	0\\
34.58	0\\
34.59	0\\
34.6	0\\
34.61	0\\
34.62	0\\
34.63	0\\
34.64	0\\
34.65	0\\
34.66	0\\
34.67	0\\
34.68	0\\
34.69	0\\
34.7	0\\
34.71	0\\
34.72	0\\
34.73	0\\
34.74	0\\
34.75	0\\
34.76	0\\
34.77	0\\
34.78	0\\
34.79	0\\
34.8	0\\
34.81	0\\
34.82	0\\
34.83	0\\
34.84	0\\
34.85	0\\
34.86	0\\
34.87	0\\
34.88	0\\
34.89	0\\
34.9	0\\
34.91	0\\
34.92	0\\
34.93	0\\
34.94	0\\
34.95	0\\
34.96	0\\
34.97	0\\
34.98	0\\
34.99	0\\
35	0\\
35.01	0\\
35.02	0\\
35.03	0\\
35.04	0\\
35.05	0\\
35.06	0\\
35.07	0\\
35.08	0\\
35.09	0\\
35.1	0\\
35.11	0\\
35.12	0\\
35.13	0\\
35.14	0\\
35.15	0\\
35.16	0\\
35.17	0\\
35.18	0\\
35.19	0\\
35.2	0\\
35.21	0\\
35.22	0\\
35.23	0\\
35.24	0\\
35.25	0\\
35.26	0\\
35.27	0\\
35.28	0\\
35.29	0\\
35.3	0\\
35.31	0\\
35.32	0\\
35.33	0\\
35.34	0\\
35.35	0\\
35.36	0\\
35.37	0\\
35.38	0\\
35.39	0\\
35.4	0\\
35.41	0\\
35.42	0\\
35.43	0\\
35.44	0\\
35.45	0\\
35.46	0\\
35.47	0\\
35.48	0\\
35.49	0\\
35.5	0\\
35.51	0\\
35.52	0\\
35.53	0\\
35.54	0\\
35.55	0\\
35.56	0\\
35.57	0\\
35.58	0\\
35.59	0\\
35.6	0\\
35.61	0\\
35.62	0\\
35.63	0\\
35.64	0\\
35.65	0\\
35.66	0\\
35.67	0\\
35.68	0\\
35.69	0\\
35.7	0\\
35.71	0\\
35.72	0\\
35.73	0\\
35.74	0\\
35.75	0\\
35.76	0\\
35.77	0\\
35.78	0\\
35.79	0\\
35.8	0\\
35.81	0\\
35.82	0\\
35.83	0\\
35.84	0\\
35.85	0\\
35.86	0\\
35.87	0\\
35.88	0\\
35.89	0\\
35.9	0\\
35.91	0\\
35.92	0\\
35.93	0\\
35.94	0\\
35.95	0\\
35.96	0\\
35.97	0\\
35.98	0\\
35.99	0\\
36	0\\
36.01	0\\
36.02	0\\
36.03	0\\
36.04	0\\
36.05	0\\
36.06	0\\
36.07	0\\
36.08	0\\
36.09	0\\
36.1	0\\
36.11	0\\
36.12	0\\
36.13	0\\
36.14	0\\
36.15	0\\
36.16	0\\
36.17	0\\
36.18	0\\
36.19	0\\
36.2	0\\
36.21	0\\
36.22	0\\
36.23	0\\
36.24	0\\
36.25	0\\
36.26	0\\
36.27	0\\
36.28	0\\
36.29	0\\
36.3	0\\
36.31	0\\
36.32	0\\
36.33	0\\
36.34	0\\
36.35	0\\
36.36	0\\
36.37	0\\
36.38	0\\
36.39	0\\
36.4	0\\
36.41	0\\
36.42	0\\
36.43	0\\
36.44	0\\
36.45	0\\
36.46	0\\
36.47	0\\
36.48	0\\
36.49	0\\
36.5	0\\
36.51	0\\
36.52	0\\
36.53	0\\
36.54	0\\
36.55	0\\
36.56	0\\
36.57	0\\
36.58	0\\
36.59	0\\
36.6	0\\
36.61	0\\
36.62	0\\
36.63	0\\
36.64	0\\
36.65	0\\
36.66	0\\
36.67	0\\
36.68	0\\
36.69	0\\
36.7	0\\
36.71	0\\
36.72	0\\
36.73	0\\
36.74	0\\
36.75	0\\
36.76	0\\
36.77	0\\
36.78	0\\
36.79	0\\
36.8	0\\
36.81	0\\
36.82	0\\
36.83	0\\
36.84	0\\
36.85	0\\
36.86	0\\
36.87	0\\
36.88	0\\
36.89	0\\
36.9	0\\
36.91	0\\
36.92	0\\
36.93	0\\
36.94	0\\
36.95	0\\
36.96	0\\
36.97	0\\
36.98	0\\
36.99	0\\
37	0\\
37.01	0\\
37.02	0\\
37.03	0\\
37.04	0\\
37.05	0\\
37.06	0\\
37.07	0\\
37.08	0\\
37.09	0\\
37.1	0\\
37.11	0\\
37.12	0\\
37.13	0\\
37.14	0\\
37.15	0\\
37.16	0\\
37.17	0\\
37.18	0\\
37.19	0\\
37.2	0\\
37.21	0\\
37.22	0\\
37.23	0\\
37.24	0\\
37.25	0\\
37.26	0\\
37.27	0\\
37.28	0\\
37.29	0\\
37.3	0\\
37.31	0\\
37.32	0\\
37.33	0\\
37.34	0\\
37.35	0\\
37.36	0\\
37.37	0\\
37.38	0\\
37.39	0\\
37.4	0\\
37.41	0\\
37.42	0\\
37.43	0\\
37.44	0\\
37.45	0\\
37.46	0\\
37.47	0\\
37.48	0\\
37.49	0\\
37.5	0\\
37.51	0\\
37.52	0\\
37.53	0\\
37.54	0\\
37.55	0\\
37.56	0\\
37.57	0\\
37.58	0\\
37.59	0\\
37.6	0\\
37.61	0\\
37.62	0\\
37.63	0\\
37.64	0\\
37.65	0\\
37.66	0\\
37.67	0\\
37.68	0\\
37.69	0\\
37.7	0\\
37.71	0\\
37.72	0\\
37.73	0\\
37.74	0\\
37.75	0\\
37.76	0\\
37.77	0\\
37.78	0\\
37.79	0\\
37.8	0\\
37.81	0\\
37.82	0\\
37.83	0\\
37.84	0\\
37.85	0\\
37.86	0\\
37.87	0\\
37.88	0\\
37.89	0\\
37.9	0\\
37.91	0\\
37.92	0\\
37.93	0\\
37.94	0\\
37.95	0\\
37.96	0\\
37.97	0\\
37.98	0\\
37.99	0\\
38	0\\
38.01	0\\
38.02	0\\
38.03	0\\
38.04	0\\
38.05	0\\
38.06	0\\
38.07	0\\
38.08	0\\
38.09	0\\
38.1	0\\
38.11	0\\
38.12	0\\
38.13	0\\
38.14	0\\
38.15	0\\
38.16	0\\
38.17	0\\
38.18	0\\
38.19	0\\
38.2	0\\
38.21	0\\
38.22	0\\
38.23	0\\
38.24	0\\
38.25	0\\
38.26	0\\
38.27	0\\
38.28	0\\
38.29	0\\
38.3	0\\
38.31	0\\
38.32	0\\
38.33	0\\
38.34	0\\
38.35	0\\
38.36	0\\
38.37	0\\
38.38	0\\
38.39	0\\
38.4	0\\
38.41	0\\
38.42	0\\
38.43	0\\
38.44	0\\
38.45	0\\
38.46	0\\
38.47	0\\
38.48	0\\
38.49	0\\
38.5	0\\
38.51	0\\
38.52	0\\
38.53	0\\
38.54	0\\
38.55	0\\
38.56	0\\
38.57	0\\
38.58	0\\
38.59	0\\
38.6	0\\
38.61	0\\
38.62	0\\
38.63	0\\
38.64	0\\
38.65	0\\
38.66	0\\
38.67	0\\
38.68	0\\
38.69	0\\
38.7	0\\
38.71	0\\
38.72	0\\
38.73	0\\
38.74	0\\
38.75	0\\
38.76	0\\
38.77	0\\
38.78	0\\
38.79	0\\
38.8	0\\
38.81	0\\
38.82	0\\
38.83	0\\
38.84	0\\
38.85	0\\
38.86	0\\
38.87	0\\
38.88	0\\
38.89	0\\
38.9	0\\
38.91	0\\
38.92	0\\
38.93	0\\
38.94	0\\
38.95	0\\
38.96	0\\
38.97	0\\
38.98	0\\
38.99	0\\
39	0\\
39.01	0\\
39.02	0\\
39.03	0\\
39.04	0\\
39.05	0\\
39.06	0\\
39.07	0\\
39.08	0\\
39.09	0\\
39.1	0\\
39.11	0\\
39.12	0\\
39.13	0\\
39.14	0\\
39.15	0\\
39.16	0\\
39.17	0\\
39.18	0\\
39.19	0\\
39.2	0\\
39.21	0\\
39.22	0\\
39.23	0\\
39.24	0\\
39.25	0\\
39.26	0\\
39.27	0\\
39.28	0\\
39.29	0\\
39.3	0\\
39.31	0\\
39.32	0\\
39.33	0\\
39.34	0\\
39.35	0\\
39.36	0\\
39.37	0\\
39.38	0\\
39.39	0\\
39.4	0\\
39.41	0\\
39.42	0\\
39.43	0\\
39.44	0\\
39.45	0\\
39.46	0\\
39.47	0\\
39.48	0\\
39.49	0\\
39.5	0\\
39.51	0\\
39.52	0\\
39.53	0\\
39.54	0\\
39.55	0\\
39.56	0\\
39.57	0\\
39.58	0\\
39.59	0\\
39.6	0\\
39.61	0\\
39.62	0\\
39.63	0\\
39.64	0\\
39.65	0\\
39.66	0\\
39.67	0\\
39.68	0\\
39.69	0\\
39.7	0\\
39.71	0\\
39.72	0\\
39.73	0\\
39.74	0\\
39.75	0\\
39.76	0\\
39.77	0\\
39.78	0\\
39.79	0\\
39.8	0\\
39.81	0\\
39.82	0\\
39.83	0\\
39.84	0\\
39.85	0\\
39.86	0\\
39.87	0\\
39.88	0\\
39.89	0\\
39.9	0\\
39.91	0\\
39.92	0\\
39.93	0\\
39.94	0\\
39.95	0\\
39.96	0\\
39.97	0\\
39.98	0\\
39.99	0\\
40	0\\
40.01	0\\
};
\addplot [color=blue,solid,forget plot]
  table[row sep=crcr]{%
40.01	0\\
40.02	0\\
40.03	0\\
40.04	0\\
40.05	0\\
40.06	0\\
40.07	0\\
40.08	0\\
40.09	0\\
40.1	0\\
40.11	0\\
40.12	0\\
40.13	0\\
40.14	0\\
40.15	0\\
40.16	0\\
40.17	0\\
40.18	0\\
40.19	0\\
40.2	0\\
40.21	0\\
40.22	0\\
40.23	0\\
40.24	0\\
40.25	0\\
40.26	0\\
40.27	0\\
40.28	0\\
40.29	0\\
40.3	0\\
40.31	0\\
40.32	0\\
40.33	0\\
40.34	0\\
40.35	0\\
40.36	0\\
40.37	0\\
40.38	0\\
40.39	0\\
40.4	0\\
40.41	0\\
40.42	0\\
40.43	0\\
40.44	0\\
40.45	0\\
40.46	0\\
40.47	0\\
40.48	0\\
40.49	0\\
40.5	0\\
40.51	0\\
40.52	0\\
40.53	0\\
40.54	0\\
40.55	0\\
40.56	0\\
40.57	0\\
40.58	0\\
40.59	0\\
40.6	0\\
40.61	0\\
40.62	0\\
40.63	0\\
40.64	0\\
40.65	0\\
40.66	0\\
40.67	0\\
40.68	0\\
40.69	0\\
40.7	0\\
40.71	0\\
40.72	0\\
40.73	0\\
40.74	0\\
40.75	0\\
40.76	0\\
40.77	0\\
40.78	0\\
40.79	0\\
40.8	0\\
40.81	0\\
40.82	0\\
40.83	0\\
40.84	0\\
40.85	0\\
40.86	0\\
40.87	0\\
40.88	0\\
40.89	0\\
40.9	0\\
40.91	0\\
40.92	0\\
40.93	0\\
40.94	0\\
40.95	0\\
40.96	0\\
40.97	0\\
40.98	0\\
40.99	0\\
41	0\\
41.01	0\\
41.02	0\\
41.03	0\\
41.04	0\\
41.05	0\\
41.06	0\\
41.07	0\\
41.08	0\\
41.09	0\\
41.1	0\\
41.11	0\\
41.12	0\\
41.13	0\\
41.14	0\\
41.15	0\\
41.16	0\\
41.17	0\\
41.18	0\\
41.19	0\\
41.2	0\\
41.21	0\\
41.22	0\\
41.23	0\\
41.24	0\\
41.25	0\\
41.26	0\\
41.27	0\\
41.28	0\\
41.29	0\\
41.3	0\\
41.31	0\\
41.32	0\\
41.33	0\\
41.34	0\\
41.35	0\\
41.36	0\\
41.37	0\\
41.38	0\\
41.39	0\\
41.4	0\\
41.41	0\\
41.42	0\\
41.43	0\\
41.44	0\\
41.45	0\\
41.46	0\\
41.47	0\\
41.48	0\\
41.49	0\\
41.5	0\\
41.51	0\\
41.52	0\\
41.53	0\\
41.54	0\\
41.55	0\\
41.56	0\\
41.57	0\\
41.58	0\\
41.59	0\\
41.6	0\\
41.61	0\\
41.62	0\\
41.63	0\\
41.64	0\\
41.65	0\\
41.66	0\\
41.67	0\\
41.68	0\\
41.69	0\\
41.7	0\\
41.71	0\\
41.72	0\\
41.73	0\\
41.74	0\\
41.75	0\\
41.76	0\\
41.77	0\\
41.78	0\\
41.79	0\\
41.8	0\\
41.81	0\\
41.82	0\\
41.83	0\\
41.84	0\\
41.85	0\\
41.86	0\\
41.87	0\\
41.88	0\\
41.89	0\\
41.9	0\\
41.91	0\\
41.92	0\\
41.93	0\\
41.94	0\\
41.95	0\\
41.96	0\\
41.97	0\\
41.98	0\\
41.99	0\\
42	0\\
42.01	0\\
42.02	0\\
42.03	0\\
42.04	0\\
42.05	0\\
42.06	0\\
42.07	0\\
42.08	0\\
42.09	0\\
42.1	0\\
42.11	0\\
42.12	0\\
42.13	0\\
42.14	0\\
42.15	0\\
42.16	0\\
42.17	0\\
42.18	0\\
42.19	0\\
42.2	0\\
42.21	0\\
42.22	0\\
42.23	0\\
42.24	0\\
42.25	0\\
42.26	0\\
42.27	0\\
42.28	0\\
42.29	0\\
42.3	0\\
42.31	0\\
42.32	0\\
42.33	0\\
42.34	0\\
42.35	0\\
42.36	0\\
42.37	0\\
42.38	0\\
42.39	0\\
42.4	0\\
42.41	0\\
42.42	0\\
42.43	0\\
42.44	0\\
42.45	0\\
42.46	0\\
42.47	0\\
42.48	0\\
42.49	0\\
42.5	0\\
42.51	0\\
42.52	0\\
42.53	0\\
42.54	0\\
42.55	0\\
42.56	0\\
42.57	0\\
42.58	0\\
42.59	0\\
42.6	0\\
42.61	0\\
42.62	0\\
42.63	0\\
42.64	0\\
42.65	0\\
42.66	0\\
42.67	0\\
42.68	0\\
42.69	0\\
42.7	0\\
42.71	0\\
42.72	0\\
42.73	0\\
42.74	0\\
42.75	0\\
42.76	0\\
42.77	0\\
42.78	0\\
42.79	0\\
42.8	0\\
42.81	0\\
42.82	0\\
42.83	0\\
42.84	0\\
42.85	0\\
42.86	0\\
42.87	0\\
42.88	0\\
42.89	0\\
42.9	0\\
42.91	0\\
42.92	0\\
42.93	0\\
42.94	0\\
42.95	0\\
42.96	0\\
42.97	0\\
42.98	0\\
42.99	0\\
43	0\\
43.01	0\\
43.02	0\\
43.03	0\\
43.04	0\\
43.05	0\\
43.06	0\\
43.07	0\\
43.08	0\\
43.09	0\\
43.1	0\\
43.11	0\\
43.12	0\\
43.13	0\\
43.14	0\\
43.15	0\\
43.16	0\\
43.17	0\\
43.18	0\\
43.19	0\\
43.2	0\\
43.21	0\\
43.22	0\\
43.23	0\\
43.24	0\\
43.25	0\\
43.26	0\\
43.27	0\\
43.28	0\\
43.29	0\\
43.3	0\\
43.31	0\\
43.32	0\\
43.33	0\\
43.34	0\\
43.35	0\\
43.36	0\\
43.37	0\\
43.38	0\\
43.39	0\\
43.4	0\\
43.41	0\\
43.42	0\\
43.43	0\\
43.44	0\\
43.45	0\\
43.46	0\\
43.47	0\\
43.48	0\\
43.49	0\\
43.5	0\\
43.51	0\\
43.52	0\\
43.53	0\\
43.54	0\\
43.55	0\\
43.56	0\\
43.57	0\\
43.58	0\\
43.59	0\\
43.6	0\\
43.61	0\\
43.62	0\\
43.63	0\\
43.64	0\\
43.65	0\\
43.66	0\\
43.67	0\\
43.68	0\\
43.69	0\\
43.7	0\\
43.71	0\\
43.72	0\\
43.73	0\\
43.74	0\\
43.75	0\\
43.76	0\\
43.77	0\\
43.78	0\\
43.79	0\\
43.8	0\\
43.81	0\\
43.82	0\\
43.83	0\\
43.84	0\\
43.85	0\\
43.86	0\\
43.87	0\\
43.88	0\\
43.89	0\\
43.9	0\\
43.91	0\\
43.92	0\\
43.93	0\\
43.94	0\\
43.95	0\\
43.96	0\\
43.97	0\\
43.98	0\\
43.99	0\\
44	0\\
44.01	0\\
44.02	0\\
44.03	0\\
44.04	0\\
44.05	0\\
44.06	0\\
44.07	0\\
44.08	0\\
44.09	0\\
44.1	0\\
44.11	0\\
44.12	0\\
44.13	0\\
44.14	0\\
44.15	0\\
44.16	0\\
44.17	0\\
44.18	0\\
44.19	0\\
44.2	0\\
44.21	0\\
44.22	0\\
44.23	0\\
44.24	0\\
44.25	0\\
44.26	0\\
44.27	0\\
44.28	0\\
44.29	0\\
44.3	0\\
44.31	0\\
44.32	0\\
44.33	0\\
44.34	0\\
44.35	0\\
44.36	0\\
44.37	0\\
44.38	0\\
44.39	0\\
44.4	0\\
44.41	0\\
44.42	0\\
44.43	0\\
44.44	0\\
44.45	0\\
44.46	0\\
44.47	0\\
44.48	0\\
44.49	0\\
44.5	0\\
44.51	0\\
44.52	0\\
44.53	0\\
44.54	0\\
44.55	0\\
44.56	0\\
44.57	0\\
44.58	0\\
44.59	0\\
44.6	0\\
44.61	0\\
44.62	0\\
44.63	0\\
44.64	0\\
44.65	0\\
44.66	0\\
44.67	0\\
44.68	0\\
44.69	0\\
44.7	0\\
44.71	0\\
44.72	0\\
44.73	0\\
44.74	0\\
44.75	0\\
44.76	0\\
44.77	0\\
44.78	0\\
44.79	0\\
44.8	0\\
44.81	0\\
44.82	0\\
44.83	0\\
44.84	0\\
44.85	0\\
44.86	0\\
44.87	0\\
44.88	0\\
44.89	0\\
44.9	0\\
44.91	0\\
44.92	0\\
44.93	0\\
44.94	0\\
44.95	0\\
44.96	0\\
44.97	0\\
44.98	0\\
44.99	0\\
45	0\\
45.01	0\\
45.02	0\\
45.03	0\\
45.04	0\\
45.05	0\\
45.06	0\\
45.07	0\\
45.08	0\\
45.09	0\\
45.1	0\\
45.11	0\\
45.12	0\\
45.13	0\\
45.14	0\\
45.15	0\\
45.16	0\\
45.17	0\\
45.18	0\\
45.19	0\\
45.2	0\\
45.21	0\\
45.22	0\\
45.23	0\\
45.24	0\\
45.25	0\\
45.26	0\\
45.27	0\\
45.28	0\\
45.29	0\\
45.3	0\\
45.31	0\\
45.32	0\\
45.33	0\\
45.34	0\\
45.35	0\\
45.36	0\\
45.37	0\\
45.38	0\\
45.39	0\\
45.4	0\\
45.41	0\\
45.42	0\\
45.43	0\\
45.44	0\\
45.45	0\\
45.46	0\\
45.47	0\\
45.48	0\\
45.49	0\\
45.5	0\\
45.51	0\\
45.52	0\\
45.53	0\\
45.54	0\\
45.55	0\\
45.56	0\\
45.57	0\\
45.58	0\\
45.59	0\\
45.6	0\\
45.61	0\\
45.62	0\\
45.63	0\\
45.64	0\\
45.65	0\\
45.66	0\\
45.67	0\\
45.68	0\\
45.69	0\\
45.7	0\\
45.71	0\\
45.72	0\\
45.73	0\\
45.74	0\\
45.75	0\\
45.76	0\\
45.77	0\\
45.78	0\\
45.79	0\\
45.8	0\\
45.81	0\\
45.82	0\\
45.83	0\\
45.84	0\\
45.85	0\\
45.86	0\\
45.87	0\\
45.88	0\\
45.89	0\\
45.9	0\\
45.91	0\\
45.92	0\\
45.93	0\\
45.94	0\\
45.95	0\\
45.96	0\\
45.97	0\\
45.98	0\\
45.99	0\\
46	0\\
46.01	0\\
46.02	0\\
46.03	0\\
46.04	0\\
46.05	0\\
46.06	0\\
46.07	0\\
46.08	0\\
46.09	0\\
46.1	0\\
46.11	0\\
46.12	0\\
46.13	0\\
46.14	0\\
46.15	0\\
46.16	0\\
46.17	0\\
46.18	0\\
46.19	0\\
46.2	0\\
46.21	0\\
46.22	0\\
46.23	0\\
46.24	0\\
46.25	0\\
46.26	0\\
46.27	0\\
46.28	0\\
46.29	0\\
46.3	0\\
46.31	0\\
46.32	0\\
46.33	0\\
46.34	0\\
46.35	0\\
46.36	0\\
46.37	0\\
46.38	0\\
46.39	0\\
46.4	0\\
46.41	0\\
46.42	0\\
46.43	0\\
46.44	0\\
46.45	0\\
46.46	0\\
46.47	0\\
46.48	0\\
46.49	0\\
46.5	0\\
46.51	0\\
46.52	0\\
46.53	0\\
46.54	0\\
46.55	0\\
46.56	0\\
46.57	0\\
46.58	0\\
46.59	0\\
46.6	0\\
46.61	0\\
46.62	0\\
46.63	0\\
46.64	0\\
46.65	0\\
46.66	0\\
46.67	0\\
46.68	0\\
46.69	0\\
46.7	0\\
46.71	0\\
46.72	0\\
46.73	0\\
46.74	0\\
46.75	0\\
46.76	0\\
46.77	0\\
46.78	0\\
46.79	0\\
46.8	0\\
46.81	0\\
46.82	0\\
46.83	0\\
46.84	0\\
46.85	0\\
46.86	0\\
46.87	0\\
46.88	0\\
46.89	0\\
46.9	0\\
46.91	0\\
46.92	0\\
46.93	0\\
46.94	0\\
46.95	0\\
46.96	0\\
46.97	0\\
46.98	0\\
46.99	0\\
47	0\\
47.01	0\\
47.02	0\\
47.03	0\\
47.04	0\\
47.05	0\\
47.06	0\\
47.07	0\\
47.08	0\\
47.09	0\\
47.1	0\\
47.11	0\\
47.12	0\\
47.13	0\\
47.14	0\\
47.15	0\\
47.16	0\\
47.17	0\\
47.18	0\\
47.19	0\\
47.2	0\\
47.21	0\\
47.22	0\\
47.23	0\\
47.24	0\\
47.25	0\\
47.26	0\\
47.27	0\\
47.28	0\\
47.29	0\\
47.3	0\\
47.31	0\\
47.32	0\\
47.33	0\\
47.34	0\\
47.35	0\\
47.36	0\\
47.37	0\\
47.38	0\\
47.39	0\\
47.4	0\\
47.41	0\\
47.42	0\\
47.43	0\\
47.44	0\\
47.45	0\\
47.46	0\\
47.47	0\\
47.48	0\\
47.49	0\\
47.5	0\\
47.51	0\\
47.52	0\\
47.53	0\\
47.54	0\\
47.55	0\\
47.56	0\\
47.57	0\\
47.58	0\\
47.59	0\\
47.6	0\\
47.61	0\\
47.62	0\\
47.63	0\\
47.64	0\\
47.65	0\\
47.66	0\\
47.67	0\\
47.68	0\\
47.69	0\\
47.7	0\\
47.71	0\\
47.72	0\\
47.73	0\\
47.74	0\\
47.75	0\\
47.76	0\\
47.77	0\\
47.78	0\\
47.79	0\\
47.8	0\\
47.81	0\\
47.82	0\\
47.83	0\\
47.84	0\\
47.85	0\\
47.86	0\\
47.87	0\\
47.88	0\\
47.89	0\\
47.9	0\\
47.91	0\\
47.92	0\\
47.93	0\\
47.94	0\\
47.95	0\\
47.96	0\\
47.97	0\\
47.98	0\\
47.99	0\\
48	0\\
48.01	0\\
48.02	0\\
48.03	0\\
48.04	0\\
48.05	0\\
48.06	0\\
48.07	0\\
48.08	0\\
48.09	0\\
48.1	0\\
48.11	0\\
48.12	0\\
48.13	0\\
48.14	0\\
48.15	0\\
48.16	0\\
48.17	0\\
48.18	0\\
48.19	0\\
48.2	0\\
48.21	0\\
48.22	0\\
48.23	0\\
48.24	0\\
48.25	0\\
48.26	0\\
48.27	0\\
48.28	0\\
48.29	0\\
48.3	0\\
48.31	0\\
48.32	0\\
48.33	0\\
48.34	0\\
48.35	0\\
48.36	0\\
48.37	0\\
48.38	0\\
48.39	0\\
48.4	0\\
48.41	0\\
48.42	0\\
48.43	0\\
48.44	0\\
48.45	0\\
48.46	0\\
48.47	0\\
48.48	0\\
48.49	0\\
48.5	0\\
48.51	0\\
48.52	0\\
48.53	0\\
48.54	0\\
48.55	0\\
48.56	0\\
48.57	0\\
48.58	0\\
48.59	0\\
48.6	0\\
48.61	0\\
48.62	0\\
48.63	0\\
48.64	0\\
48.65	0\\
48.66	0\\
48.67	0\\
48.68	0\\
48.69	0\\
48.7	0\\
48.71	0\\
48.72	0\\
48.73	0\\
48.74	0\\
48.75	0\\
48.76	0\\
48.77	0\\
48.78	0\\
48.79	0\\
48.8	0\\
48.81	0\\
48.82	0\\
48.83	0\\
48.84	0\\
48.85	0\\
48.86	0\\
48.87	0\\
48.88	0\\
48.89	0\\
48.9	0\\
48.91	0\\
48.92	0\\
48.93	0\\
48.94	0\\
48.95	0\\
48.96	0\\
48.97	0\\
48.98	0\\
48.99	0\\
49	0\\
49.01	0\\
49.02	0\\
49.03	0\\
49.04	0\\
49.05	0\\
49.06	0\\
49.07	0\\
49.08	0\\
49.09	0\\
49.1	0\\
49.11	0\\
49.12	0\\
49.13	0\\
49.14	0\\
49.15	0\\
49.16	0\\
49.17	0\\
49.18	0\\
49.19	0\\
49.2	0\\
49.21	0\\
49.22	0\\
49.23	0\\
49.24	0\\
49.25	0\\
49.26	0\\
49.27	0\\
49.28	0\\
49.29	0\\
49.3	0\\
49.31	0\\
49.32	0\\
49.33	0\\
49.34	0\\
49.35	0\\
49.36	0\\
49.37	0\\
49.38	0\\
49.39	0\\
49.4	0\\
49.41	0\\
49.42	0\\
49.43	0\\
49.44	0\\
49.45	0\\
49.46	0\\
49.47	0\\
49.48	0\\
49.49	0\\
49.5	0\\
49.51	0\\
49.52	0\\
49.53	0\\
49.54	0\\
49.55	0\\
49.56	0\\
49.57	0\\
49.58	0\\
49.59	0\\
49.6	0\\
49.61	0\\
49.62	0\\
49.63	0\\
49.64	0\\
49.65	0\\
49.66	0\\
49.67	0\\
49.68	0\\
49.69	0\\
49.7	0\\
49.71	0\\
49.72	0\\
49.73	0\\
49.74	0\\
49.75	0\\
49.76	0\\
49.77	0\\
49.78	0\\
49.79	0\\
49.8	0\\
49.81	0\\
49.82	0\\
49.83	0\\
49.84	0\\
49.85	0\\
49.86	0\\
49.87	0\\
49.88	0\\
49.89	0\\
49.9	0\\
49.91	0\\
49.92	0\\
49.93	0\\
49.94	0\\
49.95	0\\
49.96	0\\
49.97	0\\
49.98	0\\
49.99	0\\
50	0\\
50.01	0\\
50.02	0\\
50.03	0\\
50.04	0\\
50.05	0\\
50.06	0\\
50.07	0\\
50.08	0\\
50.09	0\\
50.1	0\\
50.11	0\\
50.12	0\\
50.13	0\\
50.14	0\\
50.15	0\\
50.16	0\\
50.17	0\\
50.18	0\\
50.19	0\\
50.2	0\\
50.21	0\\
50.22	0\\
50.23	0\\
50.24	0\\
50.25	0\\
50.26	0\\
50.27	0\\
50.28	0\\
50.29	0\\
50.3	0\\
50.31	0\\
50.32	0\\
50.33	0\\
50.34	0\\
50.35	0\\
50.36	0\\
50.37	0\\
50.38	0\\
50.39	0\\
50.4	0\\
50.41	0\\
50.42	0\\
50.43	0\\
50.44	0\\
50.45	0\\
50.46	0\\
50.47	0\\
50.48	0\\
50.49	0\\
50.5	0\\
50.51	0\\
50.52	0\\
50.53	0\\
50.54	0\\
50.55	0\\
50.56	0\\
50.57	0\\
50.58	0\\
50.59	0\\
50.6	0\\
50.61	0\\
50.62	0\\
50.63	0\\
50.64	0\\
50.65	0\\
50.66	0\\
50.67	0\\
50.68	0\\
50.69	0\\
50.7	0\\
50.71	0\\
50.72	0\\
50.73	0\\
50.74	0\\
50.75	0\\
50.76	0\\
50.77	0\\
50.78	0\\
50.79	0\\
50.8	0\\
50.81	0\\
50.82	0\\
50.83	0\\
50.84	0\\
50.85	0\\
50.86	0\\
50.87	0\\
50.88	0\\
50.89	0\\
50.9	0\\
50.91	0\\
50.92	0\\
50.93	0\\
50.94	0\\
50.95	0\\
50.96	0\\
50.97	0\\
50.98	0\\
50.99	0\\
51	0\\
51.01	0\\
51.02	0\\
51.03	0\\
51.04	0\\
51.05	0\\
51.06	0\\
51.07	0\\
51.08	0\\
51.09	0\\
51.1	0\\
51.11	0\\
51.12	0\\
51.13	0\\
51.14	0\\
51.15	0\\
51.16	0\\
51.17	0\\
51.18	0\\
51.19	0\\
51.2	0\\
51.21	0\\
51.22	0\\
51.23	0\\
51.24	0\\
51.25	0\\
51.26	0\\
51.27	0\\
51.28	0\\
51.29	0\\
51.3	0\\
51.31	0\\
51.32	0\\
51.33	0\\
51.34	0\\
51.35	0\\
51.36	0\\
51.37	0\\
51.38	0\\
51.39	0\\
51.4	0\\
51.41	0\\
51.42	0\\
51.43	0\\
51.44	0\\
51.45	0\\
51.46	0\\
51.47	0\\
51.48	0\\
51.49	0\\
51.5	0\\
51.51	0\\
51.52	0\\
51.53	0\\
51.54	0\\
51.55	0\\
51.56	0\\
51.57	0\\
51.58	0\\
51.59	0\\
51.6	0\\
51.61	0\\
51.62	0\\
51.63	0\\
51.64	0\\
51.65	0\\
51.66	0\\
51.67	0\\
51.68	0\\
51.69	0\\
51.7	0\\
51.71	0\\
51.72	0\\
51.73	0\\
51.74	0\\
51.75	0\\
51.76	0\\
51.77	0\\
51.78	0\\
51.79	0\\
51.8	0\\
51.81	0\\
51.82	0\\
51.83	0\\
51.84	0\\
51.85	0\\
51.86	0\\
51.87	0\\
51.88	0\\
51.89	0\\
51.9	0\\
51.91	0\\
51.92	0\\
51.93	0\\
51.94	0\\
51.95	0\\
51.96	0\\
51.97	0\\
51.98	0\\
51.99	0\\
52	0\\
52.01	0\\
52.02	0\\
52.03	0\\
52.04	0\\
52.05	0\\
52.06	0\\
52.07	0\\
52.08	0\\
52.09	0\\
52.1	0\\
52.11	0\\
52.12	0\\
52.13	0\\
52.14	0\\
52.15	0\\
52.16	0\\
52.17	0\\
52.18	0\\
52.19	0\\
52.2	0\\
52.21	0\\
52.22	0\\
52.23	0\\
52.24	0\\
52.25	0\\
52.26	0\\
52.27	0\\
52.28	0\\
52.29	0\\
52.3	0\\
52.31	0\\
52.32	0\\
52.33	0\\
52.34	0\\
52.35	0\\
52.36	0\\
52.37	0\\
52.38	0\\
52.39	0\\
52.4	0\\
52.41	0\\
52.42	0\\
52.43	0\\
52.44	0\\
52.45	0\\
52.46	0\\
52.47	0\\
52.48	0\\
52.49	0\\
52.5	0\\
52.51	0\\
52.52	0\\
52.53	0\\
52.54	0\\
52.55	0\\
52.56	0\\
52.57	0\\
52.58	0\\
52.59	0\\
52.6	0\\
52.61	0\\
52.62	0\\
52.63	0\\
52.64	0\\
52.65	0\\
52.66	0\\
52.67	0\\
52.68	0\\
52.69	0\\
52.7	0\\
52.71	0\\
52.72	0\\
52.73	0\\
52.74	0\\
52.75	0\\
52.76	0\\
52.77	0\\
52.78	0\\
52.79	0\\
52.8	0\\
52.81	0\\
52.82	0\\
52.83	0\\
52.84	0\\
52.85	0\\
52.86	0\\
52.87	0\\
52.88	0\\
52.89	0\\
52.9	0\\
52.91	0\\
52.92	0\\
52.93	0\\
52.94	0\\
52.95	0\\
52.96	0\\
52.97	0\\
52.98	0\\
52.99	0\\
53	0\\
53.01	0\\
53.02	0\\
53.03	0\\
53.04	0\\
53.05	0\\
53.06	0\\
53.07	0\\
53.08	0\\
53.09	0\\
53.1	0\\
53.11	0\\
53.12	0\\
53.13	0\\
53.14	0\\
53.15	0\\
53.16	0\\
53.17	0\\
53.18	0\\
53.19	0\\
53.2	0\\
53.21	0\\
53.22	0\\
53.23	0\\
53.24	0\\
53.25	0\\
53.26	0\\
53.27	0\\
53.28	0\\
53.29	0\\
53.3	0\\
53.31	0\\
53.32	0\\
53.33	0\\
53.34	0\\
53.35	0\\
53.36	0\\
53.37	0\\
53.38	0\\
53.39	0\\
53.4	0\\
53.41	0\\
53.42	0\\
53.43	0\\
53.44	0\\
53.45	0\\
53.46	0\\
53.47	0\\
53.48	0\\
53.49	0\\
53.5	0\\
53.51	0\\
53.52	0\\
53.53	0\\
53.54	0\\
53.55	0\\
53.56	0\\
53.57	0\\
53.58	0\\
53.59	0\\
53.6	0\\
53.61	0\\
53.62	0\\
53.63	0\\
53.64	0\\
53.65	0\\
53.66	0\\
53.67	0\\
53.68	0\\
53.69	0\\
53.7	0\\
53.71	0\\
53.72	0\\
53.73	0\\
53.74	0\\
53.75	0\\
53.76	0\\
53.77	0\\
53.78	0\\
53.79	0\\
53.8	0\\
53.81	0\\
53.82	0\\
53.83	0\\
53.84	0\\
53.85	0\\
53.86	0\\
53.87	0\\
53.88	0\\
53.89	0\\
53.9	0\\
53.91	0\\
53.92	0\\
53.93	0\\
53.94	0\\
53.95	0\\
53.96	0\\
53.97	0\\
53.98	0\\
53.99	0\\
54	0\\
54.01	0\\
54.02	0\\
54.03	0\\
54.04	0\\
54.05	0\\
54.06	0\\
54.07	0\\
54.08	0\\
54.09	0\\
54.1	0\\
54.11	0\\
54.12	0\\
54.13	0\\
54.14	0\\
54.15	0\\
54.16	0\\
54.17	0\\
54.18	0\\
54.19	0\\
54.2	0\\
54.21	0\\
54.22	0\\
54.23	0\\
54.24	0\\
54.25	0\\
54.26	0\\
54.27	0\\
54.28	0\\
54.29	0\\
54.3	0\\
54.31	0\\
54.32	0\\
54.33	0\\
54.34	0\\
54.35	0\\
54.36	0\\
54.37	0\\
54.38	0\\
54.39	0\\
54.4	0\\
54.41	0\\
54.42	0\\
54.43	0\\
54.44	0\\
54.45	0\\
54.46	0\\
54.47	0\\
54.48	0\\
54.49	0\\
54.5	0\\
54.51	0\\
54.52	0\\
54.53	0\\
54.54	0\\
54.55	0\\
54.56	0\\
54.57	0\\
54.58	0\\
54.59	0\\
54.6	0\\
54.61	0\\
54.62	0\\
54.63	0\\
54.64	0\\
54.65	0\\
54.66	0\\
54.67	0\\
54.68	0\\
54.69	0\\
54.7	0\\
54.71	0\\
54.72	0\\
54.73	0\\
54.74	0\\
54.75	0\\
54.76	0\\
54.77	0\\
54.78	0\\
54.79	0\\
54.8	0\\
54.81	0\\
54.82	0\\
54.83	0\\
54.84	0\\
54.85	0\\
54.86	0\\
54.87	0\\
54.88	0\\
54.89	0\\
54.9	0\\
54.91	0\\
54.92	0\\
54.93	0\\
54.94	0\\
54.95	0\\
54.96	0\\
54.97	0\\
54.98	0\\
54.99	0\\
55	0\\
55.01	0\\
55.02	0\\
55.03	0\\
55.04	0\\
55.05	0\\
55.06	0\\
55.07	0\\
55.08	0\\
55.09	0\\
55.1	0\\
55.11	0\\
55.12	0\\
55.13	0\\
55.14	0\\
55.15	0\\
55.16	0\\
55.17	0\\
55.18	0\\
55.19	0\\
55.2	0\\
55.21	0\\
55.22	0\\
55.23	0\\
55.24	0\\
55.25	0\\
55.26	0\\
55.27	0\\
55.28	0\\
55.29	0\\
55.3	0\\
55.31	0\\
55.32	0\\
55.33	0\\
55.34	0\\
55.35	0\\
55.36	0\\
55.37	0\\
55.38	0\\
55.39	0\\
55.4	0\\
55.41	0\\
55.42	0\\
55.43	0\\
55.44	0\\
55.45	0\\
55.46	0\\
55.47	0\\
55.48	0\\
55.49	0\\
55.5	0\\
55.51	0\\
55.52	0\\
55.53	0\\
55.54	0\\
55.55	0\\
55.56	0\\
55.57	0\\
55.58	0\\
55.59	0\\
55.6	0\\
55.61	0\\
55.62	0\\
55.63	0\\
55.64	0\\
55.65	0\\
55.66	0\\
55.67	0\\
55.68	0\\
55.69	0\\
55.7	0\\
55.71	0\\
55.72	0\\
55.73	0\\
55.74	0\\
55.75	0\\
55.76	0\\
55.77	0\\
55.78	0\\
55.79	0\\
55.8	0\\
55.81	0\\
55.82	0\\
55.83	0\\
55.84	0\\
55.85	0\\
55.86	0\\
55.87	0\\
55.88	0\\
55.89	0\\
55.9	0\\
55.91	0\\
55.92	0\\
55.93	0\\
55.94	0\\
55.95	0\\
55.96	0\\
55.97	0\\
55.98	0\\
55.99	0\\
56	0\\
56.01	0\\
56.02	0\\
56.03	0\\
56.04	0\\
56.05	0\\
56.06	0\\
56.07	0\\
56.08	0\\
56.09	0\\
56.1	0\\
56.11	0\\
56.12	0\\
56.13	0\\
56.14	0\\
56.15	0\\
56.16	0\\
56.17	0\\
56.18	0\\
56.19	0\\
56.2	0\\
56.21	0\\
56.22	0\\
56.23	0\\
56.24	0\\
56.25	0\\
56.26	0\\
56.27	0\\
56.28	0\\
56.29	0\\
56.3	0\\
56.31	0\\
56.32	0\\
56.33	0\\
56.34	0\\
56.35	0\\
56.36	0\\
56.37	0\\
56.38	0\\
56.39	0\\
56.4	0\\
56.41	0\\
56.42	0\\
56.43	0\\
56.44	0\\
56.45	0\\
56.46	0\\
56.47	0\\
56.48	0\\
56.49	0\\
56.5	0\\
56.51	0\\
56.52	0\\
56.53	0\\
56.54	0\\
56.55	0\\
56.56	0\\
56.57	0\\
56.58	0\\
56.59	0\\
56.6	0\\
56.61	0\\
56.62	0\\
56.63	0\\
56.64	0\\
56.65	0\\
56.66	0\\
56.67	0\\
56.68	0\\
56.69	0\\
56.7	0\\
56.71	0\\
56.72	0\\
56.73	0\\
56.74	0\\
56.75	0\\
56.76	0\\
56.77	0\\
56.78	0\\
56.79	0\\
56.8	0\\
56.81	0\\
56.82	0\\
56.83	0\\
56.84	0\\
56.85	0\\
56.86	0\\
56.87	0\\
56.88	0\\
56.89	0\\
56.9	0\\
56.91	0\\
56.92	0\\
56.93	0\\
56.94	0\\
56.95	0\\
56.96	0\\
56.97	0\\
56.98	0\\
56.99	0\\
57	0\\
57.01	0\\
57.02	0\\
57.03	0\\
57.04	0\\
57.05	0\\
57.06	0\\
57.07	0\\
57.08	0\\
57.09	0\\
57.1	0\\
57.11	0\\
57.12	0\\
57.13	0\\
57.14	0\\
57.15	0\\
57.16	0\\
57.17	0\\
57.18	0\\
57.19	0\\
57.2	0\\
57.21	0\\
57.22	0\\
57.23	0\\
57.24	0\\
57.25	0\\
57.26	0\\
57.27	0\\
57.28	0\\
57.29	0\\
57.3	0\\
57.31	0\\
57.32	0\\
57.33	0\\
57.34	0\\
57.35	0\\
57.36	0\\
57.37	0\\
57.38	0\\
57.39	0\\
57.4	0\\
57.41	0\\
57.42	0\\
57.43	0\\
57.44	0\\
57.45	0\\
57.46	0\\
57.47	0\\
57.48	0\\
57.49	0\\
57.5	0\\
57.51	0\\
57.52	0\\
57.53	0\\
57.54	0\\
57.55	0\\
57.56	0\\
57.57	0\\
57.58	0\\
57.59	0\\
57.6	0\\
57.61	0\\
57.62	0\\
57.63	0\\
57.64	0\\
57.65	0\\
57.66	0\\
57.67	0\\
57.68	0\\
57.69	0\\
57.7	0\\
57.71	0\\
57.72	0\\
57.73	0\\
57.74	0\\
57.75	0\\
57.76	0\\
57.77	0\\
57.78	0\\
57.79	0\\
57.8	0\\
57.81	0\\
57.82	0\\
57.83	0\\
57.84	0\\
57.85	0\\
57.86	0\\
57.87	0\\
57.88	0\\
57.89	0\\
57.9	0\\
57.91	0\\
57.92	0\\
57.93	0\\
57.94	0\\
57.95	0\\
57.96	0\\
57.97	0\\
57.98	0\\
57.99	0\\
58	0\\
58.01	0\\
58.02	0\\
58.03	0\\
58.04	0\\
58.05	0\\
58.06	0\\
58.07	0\\
58.08	0\\
58.09	0\\
58.1	0\\
58.11	0\\
58.12	0\\
58.13	0\\
58.14	0\\
58.15	0\\
58.16	0\\
58.17	0\\
58.18	0\\
58.19	0\\
58.2	0\\
58.21	0\\
58.22	0\\
58.23	0\\
58.24	0\\
58.25	0\\
58.26	0\\
58.27	0\\
58.28	0\\
58.29	0\\
58.3	0\\
58.31	0\\
58.32	0\\
58.33	0\\
58.34	0\\
58.35	0\\
58.36	0\\
58.37	0\\
58.38	0\\
58.39	0\\
58.4	0\\
58.41	0\\
58.42	0\\
58.43	0\\
58.44	0\\
58.45	0\\
58.46	0\\
58.47	0\\
58.48	0\\
58.49	0\\
58.5	0\\
58.51	0\\
58.52	0\\
58.53	0\\
58.54	0\\
58.55	0\\
58.56	0\\
58.57	0\\
58.58	0\\
58.59	0\\
58.6	0\\
58.61	0\\
58.62	0\\
58.63	0\\
58.64	0\\
58.65	0\\
58.66	0\\
58.67	0\\
58.68	0\\
58.69	0\\
58.7	0\\
58.71	0\\
58.72	0\\
58.73	0\\
58.74	0\\
58.75	0\\
58.76	0\\
58.77	0\\
58.78	0\\
58.79	0\\
58.8	0\\
58.81	0\\
58.82	0\\
58.83	0\\
58.84	0\\
58.85	0\\
58.86	0\\
58.87	0\\
58.88	0\\
58.89	0\\
58.9	0\\
58.91	0\\
58.92	0\\
58.93	0\\
58.94	0\\
58.95	0\\
58.96	0\\
58.97	0\\
58.98	0\\
58.99	0\\
59	0\\
59.01	0\\
59.02	0\\
59.03	0\\
59.04	0\\
59.05	0\\
59.06	0\\
59.07	0\\
59.08	0\\
59.09	0\\
59.1	0\\
59.11	0\\
59.12	0\\
59.13	0\\
59.14	0\\
59.15	0\\
59.16	0\\
59.17	0\\
59.18	0\\
59.19	0\\
59.2	0\\
59.21	0\\
59.22	0\\
59.23	0\\
59.24	0\\
59.25	0\\
59.26	0\\
59.27	0\\
59.28	0\\
59.29	0\\
59.3	0\\
59.31	0\\
59.32	0\\
59.33	0\\
59.34	0\\
59.35	0\\
59.36	0\\
59.37	0\\
59.38	0\\
59.39	0\\
59.4	0\\
59.41	0\\
59.42	0\\
59.43	0\\
59.44	0\\
59.45	0\\
59.46	0\\
59.47	0\\
59.48	0\\
59.49	0\\
59.5	0\\
59.51	0\\
59.52	0\\
59.53	0\\
59.54	0\\
59.55	0\\
59.56	0\\
59.57	0\\
59.58	0\\
59.59	0\\
59.6	0\\
59.61	0\\
59.62	0\\
59.63	0\\
59.64	0\\
59.65	0\\
59.66	0\\
59.67	0\\
59.68	0\\
59.69	0\\
59.7	0\\
59.71	0\\
59.72	0\\
59.73	0\\
59.74	0\\
59.75	0\\
59.76	0\\
59.77	0\\
59.78	0\\
59.79	0\\
59.8	0\\
59.81	0\\
59.82	0\\
59.83	0\\
59.84	0\\
59.85	0\\
59.86	0\\
59.87	0\\
59.88	0\\
59.89	0\\
59.9	0\\
59.91	0\\
59.92	0\\
59.93	0\\
59.94	0\\
59.95	0\\
59.96	0\\
59.97	0\\
59.98	0\\
59.99	0\\
60	0\\
60.01	0\\
60.02	0\\
60.03	0\\
60.04	0\\
60.05	0\\
60.06	0\\
60.07	0\\
60.08	0\\
60.09	0\\
60.1	0\\
60.11	0\\
60.12	0\\
60.13	0\\
60.14	0\\
60.15	0\\
60.16	0\\
60.17	0\\
60.18	0\\
60.19	0\\
60.2	0\\
60.21	0\\
60.22	0\\
60.23	0\\
60.24	0\\
60.25	0\\
60.26	0\\
60.27	0\\
60.28	0\\
60.29	0\\
60.3	0\\
60.31	0\\
60.32	0\\
60.33	0\\
60.34	0\\
60.35	0\\
60.36	0\\
60.37	0\\
60.38	0\\
60.39	0\\
60.4	0\\
60.41	0\\
60.42	0\\
60.43	0\\
60.44	0\\
60.45	0\\
60.46	0\\
60.47	0\\
60.48	0\\
60.49	0\\
60.5	0\\
60.51	0\\
60.52	0\\
60.53	0\\
60.54	0\\
60.55	0\\
60.56	0\\
60.57	0\\
60.58	0\\
60.59	0\\
60.6	0\\
60.61	0\\
60.62	0\\
60.63	0\\
60.64	0\\
60.65	0\\
60.66	0\\
60.67	0\\
60.68	0\\
60.69	0\\
60.7	0\\
60.71	0\\
60.72	0\\
60.73	0\\
60.74	0\\
60.75	0\\
60.76	0\\
60.77	0\\
60.78	0\\
60.79	0\\
60.8	0\\
60.81	0\\
60.82	0\\
60.83	0\\
60.84	0\\
60.85	0\\
60.86	0\\
60.87	0\\
60.88	0\\
60.89	0\\
60.9	0\\
60.91	0\\
60.92	0\\
60.93	0\\
60.94	0\\
60.95	0\\
60.96	0\\
60.97	0\\
60.98	0\\
60.99	0\\
61	0\\
61.01	0\\
61.02	0\\
61.03	0\\
61.04	0\\
61.05	0\\
61.06	0\\
61.07	0\\
61.08	0\\
61.09	0\\
61.1	0\\
61.11	0\\
61.12	0\\
61.13	0\\
61.14	0\\
61.15	0\\
61.16	0\\
61.17	0\\
61.18	0\\
61.19	0\\
61.2	0\\
61.21	0\\
61.22	0\\
61.23	0\\
61.24	0\\
61.25	0\\
61.26	0\\
61.27	0\\
61.28	0\\
61.29	0\\
61.3	0\\
61.31	0\\
61.32	0\\
61.33	0\\
61.34	0\\
61.35	0\\
61.36	0\\
61.37	0\\
61.38	0\\
61.39	0\\
61.4	0\\
61.41	0\\
61.42	0\\
61.43	0\\
61.44	0\\
61.45	0\\
61.46	0\\
61.47	0\\
61.48	0\\
61.49	0\\
61.5	0\\
61.51	0\\
61.52	0\\
61.53	0\\
61.54	0\\
61.55	0\\
61.56	0\\
61.57	0\\
61.58	0\\
61.59	0\\
61.6	0\\
61.61	0\\
61.62	0\\
61.63	0\\
61.64	0\\
61.65	0\\
61.66	0\\
61.67	0\\
61.68	0\\
61.69	0\\
61.7	0\\
61.71	0\\
61.72	0\\
61.73	0\\
61.74	0\\
61.75	0\\
61.76	0\\
61.77	0\\
61.78	0\\
61.79	0\\
61.8	0\\
61.81	0\\
61.82	0\\
61.83	0\\
61.84	0\\
61.85	0\\
61.86	0\\
61.87	0\\
61.88	0\\
61.89	0\\
61.9	0\\
61.91	0\\
61.92	0\\
61.93	0\\
61.94	0\\
61.95	0\\
61.96	0\\
61.97	0\\
61.98	0\\
61.99	0\\
62	0\\
62.01	0\\
62.02	0\\
62.03	0\\
62.04	0\\
62.05	0\\
62.06	0\\
62.07	0\\
62.08	0\\
62.09	0\\
62.1	0\\
62.11	0\\
62.12	0\\
62.13	0\\
62.14	0\\
62.15	0\\
62.16	0\\
62.17	0\\
62.18	0\\
62.19	0\\
62.2	0\\
62.21	0\\
62.22	0\\
62.23	0\\
62.24	0\\
62.25	0\\
62.26	0\\
62.27	0\\
62.28	0\\
62.29	0\\
62.3	0\\
62.31	0\\
62.32	0\\
62.33	0\\
62.34	0\\
62.35	0\\
62.36	0\\
62.37	0\\
62.38	0\\
62.39	0\\
62.4	0\\
62.41	0\\
62.42	0\\
62.43	0\\
62.44	0\\
62.45	0\\
62.46	0\\
62.47	0\\
62.48	0\\
62.49	0\\
62.5	0\\
62.51	0\\
62.52	0\\
62.53	0\\
62.54	0\\
62.55	0\\
62.56	0\\
62.57	0\\
62.58	0\\
62.59	0\\
62.6	0\\
62.61	0\\
62.62	0\\
62.63	0\\
62.64	0\\
62.65	0\\
62.66	0\\
62.67	0\\
62.68	0\\
62.69	0\\
62.7	0\\
62.71	0\\
62.72	0\\
62.73	0\\
62.74	0\\
62.75	0\\
62.76	0\\
62.77	0\\
62.78	0\\
62.79	0\\
62.8	0\\
62.81	0\\
62.82	0\\
62.83	0\\
62.84	0\\
62.85	0\\
62.86	0\\
62.87	0\\
62.88	0\\
62.89	0\\
62.9	0\\
62.91	0\\
62.92	0\\
62.93	0\\
62.94	0\\
62.95	0\\
62.96	0\\
62.97	0\\
62.98	0\\
62.99	0\\
63	0\\
63.01	0\\
63.02	0\\
63.03	0\\
63.04	0\\
63.05	0\\
63.06	0\\
63.07	0\\
63.08	0\\
63.09	0\\
63.1	0\\
63.11	0\\
63.12	0\\
63.13	0\\
63.14	0\\
63.15	0\\
63.16	0\\
63.17	0\\
63.18	0\\
63.19	0\\
63.2	0\\
63.21	0\\
63.22	0\\
63.23	0\\
63.24	0\\
63.25	0\\
63.26	0\\
63.27	0\\
63.28	0\\
63.29	0\\
63.3	0\\
63.31	0\\
63.32	0\\
63.33	0\\
63.34	0\\
63.35	0\\
63.36	0\\
63.37	0\\
63.38	0\\
63.39	0\\
63.4	0\\
63.41	0\\
63.42	0\\
63.43	0\\
63.44	0\\
63.45	0\\
63.46	0\\
63.47	0\\
63.48	0\\
63.49	0\\
63.5	0\\
63.51	0\\
63.52	0\\
63.53	0\\
63.54	0\\
63.55	0\\
63.56	0\\
63.57	0\\
63.58	0\\
63.59	0\\
63.6	0\\
63.61	0\\
63.62	0\\
63.63	0\\
63.64	0\\
63.65	0\\
63.66	0\\
63.67	0\\
63.68	0\\
63.69	0\\
63.7	0\\
63.71	0\\
63.72	0\\
63.73	0\\
63.74	0\\
63.75	0\\
63.76	0\\
63.77	0\\
63.78	0\\
63.79	0\\
63.8	0\\
63.81	0\\
63.82	0\\
63.83	0\\
63.84	0\\
63.85	0\\
63.86	0\\
63.87	0\\
63.88	0\\
63.89	0\\
63.9	0\\
63.91	0\\
63.92	0\\
63.93	0\\
63.94	0\\
63.95	0\\
63.96	0\\
63.97	0\\
63.98	0\\
63.99	0\\
64	0\\
64.01	0\\
64.02	0\\
64.03	0\\
64.04	0\\
64.05	0\\
64.06	0\\
64.07	0\\
64.08	0\\
64.09	0\\
64.1	0\\
64.11	0\\
64.12	0\\
64.13	0\\
64.14	0\\
64.15	0\\
64.16	0\\
64.17	0\\
64.18	0\\
64.19	0\\
64.2	0\\
64.21	0\\
64.22	0\\
64.23	0\\
64.24	0\\
64.25	0\\
64.26	0\\
64.27	0\\
64.28	0\\
64.29	0\\
64.3	0\\
64.31	0\\
64.32	0\\
64.33	0\\
64.34	0\\
64.35	0\\
64.36	0\\
64.37	0\\
64.38	0\\
64.39	0\\
64.4	0\\
64.41	0\\
64.42	0\\
64.43	0\\
64.44	0\\
64.45	0\\
64.46	0\\
64.47	0\\
64.48	0\\
64.49	0\\
64.5	0\\
64.51	0\\
64.52	0\\
64.53	0\\
64.54	0\\
64.55	0\\
64.56	0\\
64.57	0\\
64.58	0\\
64.59	0\\
64.6	0\\
64.61	0\\
64.62	0\\
64.63	0\\
64.64	0\\
64.65	0\\
64.66	0\\
64.67	0\\
64.68	0\\
64.69	0\\
64.7	0\\
64.71	0\\
64.72	0\\
64.73	0\\
64.74	0\\
64.75	0\\
64.76	0\\
64.77	0\\
64.78	0\\
64.79	0\\
64.8	0\\
64.81	0\\
64.82	0\\
64.83	0\\
64.84	0\\
64.85	0\\
64.86	0\\
64.87	0\\
64.88	0\\
64.89	0\\
64.9	0\\
64.91	0\\
64.92	0\\
64.93	0\\
64.94	0\\
64.95	0\\
64.96	0\\
64.97	0\\
64.98	0\\
64.99	0\\
65	0\\
65.01	0\\
65.02	0\\
65.03	0\\
65.04	0\\
65.05	0\\
65.06	0\\
65.07	0\\
65.08	0\\
65.09	0\\
65.1	0\\
65.11	0\\
65.12	0\\
65.13	0\\
65.14	0\\
65.15	0\\
65.16	0\\
65.17	0\\
65.18	0\\
65.19	0\\
65.2	0\\
65.21	0\\
65.22	0\\
65.23	0\\
65.24	0\\
65.25	0\\
65.26	0\\
65.27	0\\
65.28	0\\
65.29	0\\
65.3	0\\
65.31	0\\
65.32	0\\
65.33	0\\
65.34	0\\
65.35	0\\
65.36	0\\
65.37	0\\
65.38	0\\
65.39	0\\
65.4	0\\
65.41	0\\
65.42	0\\
65.43	0\\
65.44	0\\
65.45	0\\
65.46	0\\
65.47	0\\
65.48	0\\
65.49	0\\
65.5	0\\
65.51	0\\
65.52	0\\
65.53	0\\
65.54	0\\
65.55	0\\
65.56	0\\
65.57	0\\
65.58	0\\
65.59	0\\
65.6	0\\
65.61	0\\
65.62	0\\
65.63	0\\
65.64	0\\
65.65	0\\
65.66	0\\
65.67	0\\
65.68	0\\
65.69	0\\
65.7	0\\
65.71	0\\
65.72	0\\
65.73	0\\
65.74	0\\
65.75	0\\
65.76	0\\
65.77	0\\
65.78	0\\
65.79	0\\
65.8	0\\
65.81	0\\
65.82	0\\
65.83	0\\
65.84	0\\
65.85	0\\
65.86	0\\
65.87	0\\
65.88	0\\
65.89	0\\
65.9	0\\
65.91	0\\
65.92	0\\
65.93	0\\
65.94	0\\
65.95	0\\
65.96	0\\
65.97	0\\
65.98	0\\
65.99	0\\
66	0\\
66.01	0\\
66.02	0\\
66.03	0\\
66.04	0\\
66.05	0\\
66.06	0\\
66.07	0\\
66.08	0\\
66.09	0\\
66.1	0\\
66.11	0\\
66.12	0\\
66.13	0\\
66.14	0\\
66.15	0\\
66.16	0\\
66.17	0\\
66.18	0\\
66.19	0\\
66.2	0\\
66.21	0\\
66.22	0\\
66.23	0\\
66.24	0\\
66.25	0\\
66.26	0\\
66.27	0\\
66.28	0\\
66.29	0\\
66.3	0\\
66.31	0\\
66.32	0\\
66.33	0\\
66.34	0\\
66.35	0\\
66.36	0\\
66.37	0\\
66.38	0\\
66.39	0\\
66.4	0\\
66.41	0\\
66.42	0\\
66.43	0\\
66.44	0\\
66.45	0\\
66.46	0\\
66.47	0\\
66.48	0\\
66.49	0\\
66.5	0\\
66.51	0\\
66.52	0\\
66.53	0\\
66.54	0\\
66.55	0\\
66.56	0\\
66.57	0\\
66.58	0\\
66.59	0\\
66.6	0\\
66.61	0\\
66.62	0\\
66.63	0\\
66.64	0\\
66.65	0\\
66.66	0\\
66.67	0\\
66.68	0\\
66.69	0\\
66.7	0\\
66.71	0\\
66.72	0\\
66.73	0\\
66.74	0\\
66.75	0\\
66.76	0\\
66.77	0\\
66.78	0\\
66.79	0\\
66.8	0\\
66.81	0\\
66.82	0\\
66.83	0\\
66.84	0\\
66.85	0\\
66.86	0\\
66.87	0\\
66.88	0\\
66.89	0\\
66.9	0\\
66.91	0\\
66.92	0\\
66.93	0\\
66.94	0\\
66.95	0\\
66.96	0\\
66.97	0\\
66.98	0\\
66.99	0\\
67	0\\
67.01	0\\
67.02	0\\
67.03	0\\
67.04	0\\
67.05	0\\
67.06	0\\
67.07	0\\
67.08	0\\
67.09	0\\
67.1	0\\
67.11	0\\
67.12	0\\
67.13	0\\
67.14	0\\
67.15	0\\
67.16	0\\
67.17	0\\
67.18	0\\
67.19	0\\
67.2	0\\
67.21	0\\
67.22	0\\
67.23	0\\
67.24	0\\
67.25	0\\
67.26	0\\
67.27	0\\
67.28	0\\
67.29	0\\
67.3	0\\
67.31	0\\
67.32	0\\
67.33	0\\
67.34	0\\
67.35	0\\
67.36	0\\
67.37	0\\
67.38	0\\
67.39	0\\
67.4	0\\
67.41	0\\
67.42	0\\
67.43	0\\
67.44	0\\
67.45	0\\
67.46	0\\
67.47	0\\
67.48	0\\
67.49	0\\
67.5	0\\
67.51	0\\
67.52	0\\
67.53	0\\
67.54	0\\
67.55	0\\
67.56	0\\
67.57	0\\
67.58	0\\
67.59	0\\
67.6	0\\
67.61	0\\
67.62	0\\
67.63	0\\
67.64	0\\
67.65	0\\
67.66	0\\
67.67	0\\
67.68	0\\
67.69	0\\
67.7	0\\
67.71	0\\
67.72	0\\
67.73	0\\
67.74	0\\
67.75	0\\
67.76	0\\
67.77	0\\
67.78	0\\
67.79	0\\
67.8	0\\
67.81	0\\
67.82	0\\
67.83	0\\
67.84	0\\
67.85	0\\
67.86	0\\
67.87	0\\
67.88	0\\
67.89	0\\
67.9	0\\
67.91	0\\
67.92	0\\
67.93	0\\
67.94	0\\
67.95	0\\
67.96	0\\
67.97	0\\
67.98	0\\
67.99	0\\
68	0\\
68.01	0\\
68.02	0\\
68.03	0\\
68.04	0\\
68.05	0\\
68.06	0\\
68.07	0\\
68.08	0\\
68.09	0\\
68.1	0\\
68.11	0\\
68.12	0\\
68.13	0\\
68.14	0\\
68.15	0\\
68.16	0\\
68.17	0\\
68.18	0\\
68.19	0\\
68.2	0\\
68.21	0\\
68.22	0\\
68.23	0\\
68.24	0\\
68.25	0\\
68.26	0\\
68.27	0\\
68.28	0\\
68.29	0\\
68.3	0\\
68.31	0\\
68.32	0\\
68.33	0\\
68.34	0\\
68.35	0\\
68.36	0\\
68.37	0\\
68.38	0\\
68.39	0\\
68.4	0\\
68.41	0\\
68.42	0\\
68.43	0\\
68.44	0\\
68.45	0\\
68.46	0\\
68.47	0\\
68.48	0\\
68.49	0\\
68.5	0\\
68.51	0\\
68.52	0\\
68.53	0\\
68.54	0\\
68.55	0\\
68.56	0\\
68.57	0\\
68.58	0\\
68.59	0\\
68.6	0\\
68.61	0\\
68.62	0\\
68.63	0\\
68.64	0\\
68.65	0\\
68.66	0\\
68.67	0\\
68.68	0\\
68.69	0\\
68.7	0\\
68.71	0\\
68.72	0\\
68.73	0\\
68.74	0\\
68.75	0\\
68.76	0\\
68.77	0\\
68.78	0\\
68.79	0\\
68.8	0\\
68.81	0\\
68.82	0\\
68.83	0\\
68.84	0\\
68.85	0\\
68.86	0\\
68.87	0\\
68.88	0\\
68.89	0\\
68.9	0\\
68.91	0\\
68.92	0\\
68.93	0\\
68.94	0\\
68.95	0\\
68.96	0\\
68.97	0\\
68.98	0\\
68.99	0\\
69	0\\
69.01	0\\
69.02	0\\
69.03	0\\
69.04	0\\
69.05	0\\
69.06	0\\
69.07	0\\
69.08	0\\
69.09	0\\
69.1	0\\
69.11	0\\
69.12	0\\
69.13	0\\
69.14	0\\
69.15	0\\
69.16	0\\
69.17	0\\
69.18	0\\
69.19	0\\
69.2	0\\
69.21	0\\
69.22	0\\
69.23	0\\
69.24	0\\
69.25	0\\
69.26	0\\
69.27	0\\
69.28	0\\
69.29	0\\
69.3	0\\
69.31	0\\
69.32	0\\
69.33	0\\
69.34	0\\
69.35	0\\
69.36	0\\
69.37	0\\
69.38	0\\
69.39	0\\
69.4	0\\
69.41	0\\
69.42	0\\
69.43	0\\
69.44	0\\
69.45	0\\
69.46	0\\
69.47	0\\
69.48	0\\
69.49	0\\
69.5	0\\
69.51	0\\
69.52	0\\
69.53	0\\
69.54	0\\
69.55	0\\
69.56	0\\
69.57	0\\
69.58	0\\
69.59	0\\
69.6	0\\
69.61	0\\
69.62	0\\
69.63	0\\
69.64	0\\
69.65	0\\
69.66	0\\
69.67	0\\
69.68	0\\
69.69	0\\
69.7	0\\
69.71	0\\
69.72	0\\
69.73	0\\
69.74	0\\
69.75	0\\
69.76	0\\
69.77	0\\
69.78	0\\
69.79	0\\
69.8	0\\
69.81	0\\
69.82	0\\
69.83	0\\
69.84	0\\
69.85	0\\
69.86	0\\
69.87	0\\
69.88	0\\
69.89	0\\
69.9	0\\
69.91	0\\
69.92	0\\
69.93	0\\
69.94	0\\
69.95	0\\
69.96	0\\
69.97	0\\
69.98	0\\
69.99	0\\
70	0\\
70.01	0\\
70.02	0\\
70.03	0\\
70.04	0\\
70.05	0\\
70.06	0\\
70.07	0\\
70.08	0\\
70.09	0\\
70.1	0\\
70.11	0\\
70.12	0\\
70.13	0\\
70.14	0\\
70.15	0\\
70.16	0\\
70.17	0\\
70.18	0\\
70.19	0\\
70.2	0\\
70.21	0\\
70.22	0\\
70.23	0\\
70.24	0\\
70.25	0\\
70.26	0\\
70.27	0\\
70.28	0\\
70.29	0\\
70.3	0\\
70.31	0\\
70.32	0\\
70.33	0\\
70.34	0\\
70.35	0\\
70.36	0\\
70.37	0\\
70.38	0\\
70.39	0\\
70.4	0\\
70.41	0\\
70.42	0\\
70.43	0\\
70.44	0\\
70.45	0\\
70.46	0\\
70.47	0\\
70.48	0\\
70.49	0\\
70.5	0\\
70.51	0\\
70.52	0\\
70.53	0\\
70.54	0\\
70.55	0\\
70.56	0\\
70.57	0\\
70.58	0\\
70.59	0\\
70.6	0\\
70.61	0\\
70.62	0\\
70.63	0\\
70.64	0\\
70.65	0\\
70.66	0\\
70.67	0\\
70.68	0\\
70.69	0\\
70.7	0\\
70.71	0\\
70.72	0\\
70.73	0\\
70.74	0\\
70.75	0\\
70.76	0\\
70.77	0\\
70.78	0\\
70.79	0\\
70.8	0\\
70.81	0\\
70.82	0\\
70.83	0\\
70.84	0\\
70.85	0\\
70.86	0\\
70.87	0\\
70.88	0\\
70.89	0\\
70.9	0\\
70.91	0\\
70.92	0\\
70.93	0\\
70.94	0\\
70.95	0\\
70.96	0\\
70.97	0\\
70.98	0\\
70.99	0\\
71	0\\
71.01	0\\
71.02	0\\
71.03	0\\
71.04	0\\
71.05	0\\
71.06	0\\
71.07	0\\
71.08	0\\
71.09	0\\
71.1	0\\
71.11	0\\
71.12	0\\
71.13	0\\
71.14	0\\
71.15	0\\
71.16	0\\
71.17	0\\
71.18	0\\
71.19	0\\
71.2	0\\
71.21	0\\
71.22	0\\
71.23	0\\
71.24	0\\
71.25	0\\
71.26	0\\
71.27	0\\
71.28	0\\
71.29	0\\
71.3	0\\
71.31	0\\
71.32	0\\
71.33	0\\
71.34	0\\
71.35	0\\
71.36	0\\
71.37	0\\
71.38	0\\
71.39	0\\
71.4	0\\
71.41	0\\
71.42	0\\
71.43	0\\
71.44	0\\
71.45	0\\
71.46	0\\
71.47	0\\
71.48	0\\
71.49	0\\
71.5	0\\
71.51	0\\
71.52	0\\
71.53	0\\
71.54	0\\
71.55	0\\
71.56	0\\
71.57	0\\
71.58	0\\
71.59	0\\
71.6	0\\
71.61	0\\
71.62	0\\
71.63	0\\
71.64	0\\
71.65	0\\
71.66	0\\
71.67	0\\
71.68	0\\
71.69	0\\
71.7	0\\
71.71	0\\
71.72	0\\
71.73	0\\
71.74	0\\
71.75	0\\
71.76	0\\
71.77	0\\
71.78	0\\
71.79	0\\
71.8	0\\
71.81	0\\
71.82	0\\
71.83	0\\
71.84	0\\
71.85	0\\
71.86	0\\
71.87	0\\
71.88	0\\
71.89	0\\
71.9	0\\
71.91	0\\
71.92	0\\
71.93	0\\
71.94	0\\
71.95	0\\
71.96	0\\
71.97	0\\
71.98	0\\
71.99	0\\
72	0\\
72.01	0\\
72.02	0\\
72.03	0\\
72.04	0\\
72.05	0\\
72.06	0\\
72.07	0\\
72.08	0\\
72.09	0\\
72.1	0\\
72.11	0\\
72.12	0\\
72.13	0\\
72.14	0\\
72.15	0\\
72.16	0\\
72.17	0\\
72.18	0\\
72.19	0\\
72.2	0\\
72.21	0\\
72.22	0\\
72.23	0\\
72.24	0\\
72.25	0\\
72.26	0\\
72.27	0\\
72.28	0\\
72.29	0\\
72.3	0\\
72.31	0\\
72.32	0\\
72.33	0\\
72.34	0\\
72.35	0\\
72.36	0\\
72.37	0\\
72.38	0\\
72.39	0\\
72.4	0\\
72.41	0\\
72.42	0\\
72.43	0\\
72.44	0\\
72.45	0\\
72.46	0\\
72.47	0\\
72.48	0\\
72.49	0\\
72.5	0\\
72.51	0\\
72.52	0\\
72.53	0\\
72.54	0\\
72.55	0\\
72.56	0\\
72.57	0\\
72.58	0\\
72.59	0\\
72.6	0\\
72.61	0\\
72.62	0\\
72.63	0\\
72.64	0\\
72.65	0\\
72.66	0\\
72.67	0\\
72.68	0\\
72.69	0\\
72.7	0\\
72.71	0\\
72.72	0\\
72.73	0\\
72.74	0\\
72.75	0\\
72.76	0\\
72.77	0\\
72.78	0\\
72.79	0\\
72.8	0\\
72.81	0\\
72.82	0\\
72.83	0\\
72.84	0\\
72.85	0\\
72.86	0\\
72.87	0\\
72.88	0\\
72.89	0\\
72.9	0\\
72.91	0\\
72.92	0\\
72.93	0\\
72.94	0\\
72.95	0\\
72.96	0\\
72.97	0\\
72.98	0\\
72.99	0\\
73	0\\
73.01	0\\
73.02	0\\
73.03	0\\
73.04	0\\
73.05	0\\
73.06	0\\
73.07	0\\
73.08	0\\
73.09	0\\
73.1	0\\
73.11	0\\
73.12	0\\
73.13	0\\
73.14	0\\
73.15	0\\
73.16	0\\
73.17	0\\
73.18	0\\
73.19	0\\
73.2	0\\
73.21	0\\
73.22	0\\
73.23	0\\
73.24	0\\
73.25	0\\
73.26	0\\
73.27	0\\
73.28	0\\
73.29	0\\
73.3	0\\
73.31	0\\
73.32	0\\
73.33	0\\
73.34	0\\
73.35	0\\
73.36	0\\
73.37	0\\
73.38	0\\
73.39	0\\
73.4	0\\
73.41	0\\
73.42	0\\
73.43	0\\
73.44	0\\
73.45	0\\
73.46	0\\
73.47	0\\
73.48	0\\
73.49	0\\
73.5	0\\
73.51	0\\
73.52	0\\
73.53	0\\
73.54	0\\
73.55	0\\
73.56	0\\
73.57	0\\
73.58	0\\
73.59	0\\
73.6	0\\
73.61	0\\
73.62	0\\
73.63	0\\
73.64	0\\
73.65	0\\
73.66	0\\
73.67	0\\
73.68	0\\
73.69	0\\
73.7	0\\
73.71	0\\
73.72	0\\
73.73	0\\
73.74	0\\
73.75	0\\
73.76	0\\
73.77	0\\
73.78	0\\
73.79	0\\
73.8	0\\
73.81	0\\
73.82	0\\
73.83	0\\
73.84	0\\
73.85	0\\
73.86	0\\
73.87	0\\
73.88	0\\
73.89	0\\
73.9	0\\
73.91	0\\
73.92	0\\
73.93	0\\
73.94	0\\
73.95	0\\
73.96	0\\
73.97	0\\
73.98	0\\
73.99	0\\
74	0\\
74.01	0\\
74.02	0\\
74.03	0\\
74.04	0\\
74.05	0\\
74.06	0\\
74.07	0\\
74.08	0\\
74.09	0\\
74.1	0\\
74.11	0\\
74.12	0\\
74.13	0\\
74.14	0\\
74.15	0\\
74.16	0\\
74.17	0\\
74.18	0\\
74.19	0\\
74.2	0\\
74.21	0\\
74.22	0\\
74.23	0\\
74.24	0\\
74.25	0\\
74.26	0\\
74.27	0\\
74.28	0\\
74.29	0\\
74.3	0\\
74.31	0\\
74.32	0\\
74.33	0\\
74.34	0\\
74.35	0\\
74.36	0\\
74.37	0\\
74.38	0\\
74.39	0\\
74.4	0\\
74.41	0\\
74.42	0\\
74.43	0\\
74.44	0\\
74.45	0\\
74.46	0\\
74.47	0\\
74.48	0\\
74.49	0\\
74.5	0\\
74.51	0\\
74.52	0\\
74.53	0\\
74.54	0\\
74.55	0\\
74.56	0\\
74.57	0\\
74.58	0\\
74.59	0\\
74.6	0\\
74.61	0\\
74.62	0\\
74.63	0\\
74.64	0\\
74.65	0\\
74.66	0\\
74.67	0\\
74.68	0\\
74.69	0\\
74.7	0\\
74.71	0\\
74.72	0\\
74.73	0\\
74.74	0\\
74.75	0\\
74.76	0\\
74.77	0\\
74.78	0\\
74.79	0\\
74.8	0\\
74.81	0\\
74.82	0\\
74.83	0\\
74.84	0\\
74.85	0\\
74.86	0\\
74.87	0\\
74.88	0\\
74.89	0\\
74.9	0\\
74.91	0\\
74.92	0\\
74.93	0\\
74.94	0\\
74.95	0\\
74.96	0\\
74.97	0\\
74.98	0\\
74.99	0\\
75	0\\
75.01	0\\
75.02	0\\
75.03	0\\
75.04	0\\
75.05	0\\
75.06	0\\
75.07	0\\
75.08	0\\
75.09	0\\
75.1	0\\
75.11	0\\
75.12	0\\
75.13	0\\
75.14	0\\
75.15	0\\
75.16	0\\
75.17	0\\
75.18	0\\
75.19	0\\
75.2	0\\
75.21	0\\
75.22	0\\
75.23	0\\
75.24	0\\
75.25	0\\
75.26	0\\
75.27	0\\
75.28	0\\
75.29	0\\
75.3	0\\
75.31	0\\
75.32	0\\
75.33	0\\
75.34	0\\
75.35	0\\
75.36	0\\
75.37	0\\
75.38	0\\
75.39	0\\
75.4	0\\
75.41	0\\
75.42	0\\
75.43	0\\
75.44	0\\
75.45	0\\
75.46	0\\
75.47	0\\
75.48	0\\
75.49	0\\
75.5	0\\
75.51	0\\
75.52	0\\
75.53	0\\
75.54	0\\
75.55	0\\
75.56	0\\
75.57	0\\
75.58	0\\
75.59	0\\
75.6	0\\
75.61	0\\
75.62	0\\
75.63	0\\
75.64	0\\
75.65	0\\
75.66	0\\
75.67	0\\
75.68	0\\
75.69	0\\
75.7	0\\
75.71	0\\
75.72	0\\
75.73	0\\
75.74	0\\
75.75	0\\
75.76	0\\
75.77	0\\
75.78	0\\
75.79	0\\
75.8	0\\
75.81	0\\
75.82	0\\
75.83	0\\
75.84	0\\
75.85	0\\
75.86	0\\
75.87	0\\
75.88	0\\
75.89	0\\
75.9	0\\
75.91	0\\
75.92	0\\
75.93	0\\
75.94	0\\
75.95	0\\
75.96	0\\
75.97	0\\
75.98	0\\
75.99	0\\
76	0\\
76.01	0\\
76.02	0\\
76.03	0\\
76.04	0\\
76.05	0\\
76.06	0\\
76.07	0\\
76.08	0\\
76.09	0\\
76.1	0\\
76.11	0\\
76.12	0\\
76.13	0\\
76.14	0\\
76.15	0\\
76.16	0\\
76.17	0\\
76.18	0\\
76.19	0\\
76.2	0\\
76.21	0\\
76.22	0\\
76.23	0\\
76.24	0\\
76.25	0\\
76.26	0\\
76.27	0\\
76.28	0\\
76.29	0\\
76.3	0\\
76.31	0\\
76.32	0\\
76.33	0\\
76.34	0\\
76.35	0\\
76.36	0\\
76.37	0\\
76.38	0\\
76.39	0\\
76.4	0\\
76.41	0\\
76.42	0\\
76.43	0\\
76.44	0\\
76.45	0\\
76.46	0\\
76.47	0\\
76.48	0\\
76.49	0\\
76.5	0\\
76.51	0\\
76.52	0\\
76.53	0\\
76.54	0\\
76.55	0\\
76.56	0\\
76.57	0\\
76.58	0\\
76.59	0\\
76.6	0\\
76.61	0\\
76.62	0\\
76.63	0\\
76.64	0\\
76.65	0\\
76.66	0\\
76.67	0\\
76.68	0\\
76.69	0\\
76.7	0\\
76.71	0\\
76.72	0\\
76.73	0\\
76.74	0\\
76.75	0\\
76.76	0\\
76.77	0\\
76.78	0\\
76.79	0\\
76.8	0\\
76.81	0\\
76.82	0\\
76.83	0\\
76.84	0\\
76.85	0\\
76.86	0\\
76.87	0\\
76.88	0\\
76.89	0\\
76.9	0\\
76.91	0\\
76.92	0\\
76.93	0\\
76.94	0\\
76.95	0\\
76.96	0\\
76.97	0\\
76.98	0\\
76.99	0\\
77	0\\
77.01	0\\
77.02	0\\
77.03	0\\
77.04	0\\
77.05	0\\
77.06	0\\
77.07	0\\
77.08	0\\
77.09	0\\
77.1	0\\
77.11	0\\
77.12	0\\
77.13	0\\
77.14	0\\
77.15	0\\
77.16	0\\
77.17	0\\
77.18	0\\
77.19	0\\
77.2	0\\
77.21	0\\
77.22	0\\
77.23	0\\
77.24	0\\
77.25	0\\
77.26	0\\
77.27	0\\
77.28	0\\
77.29	0\\
77.3	0\\
77.31	0\\
77.32	0\\
77.33	0\\
77.34	0\\
77.35	0\\
77.36	0\\
77.37	0\\
77.38	0\\
77.39	0\\
77.4	0\\
77.41	0\\
77.42	0\\
77.43	0\\
77.44	0\\
77.45	0\\
77.46	0\\
77.47	0\\
77.48	0\\
77.49	0\\
77.5	0\\
77.51	0\\
77.52	0\\
77.53	0\\
77.54	0\\
77.55	0\\
77.56	0\\
77.57	0\\
77.58	0\\
77.59	0\\
77.6	0\\
77.61	0\\
77.62	0\\
77.63	0\\
77.64	0\\
77.65	0\\
77.66	0\\
77.67	0\\
77.68	0\\
77.69	0\\
77.7	0\\
77.71	0\\
77.72	0\\
77.73	0\\
77.74	0\\
77.75	0\\
77.76	0\\
77.77	0\\
77.78	0\\
77.79	0\\
77.8	0\\
77.81	0\\
77.82	0\\
77.83	0\\
77.84	0\\
77.85	0\\
77.86	0\\
77.87	0\\
77.88	0\\
77.89	0\\
77.9	0\\
77.91	0\\
77.92	0\\
77.93	0\\
77.94	0\\
77.95	0\\
77.96	0\\
77.97	0\\
77.98	0\\
77.99	0\\
78	0\\
78.01	0\\
78.02	0\\
78.03	0\\
78.04	0\\
78.05	0\\
78.06	0\\
78.07	0\\
78.08	0\\
78.09	0\\
78.1	0\\
78.11	0\\
78.12	0\\
78.13	0\\
78.14	0\\
78.15	0\\
78.16	0\\
78.17	0\\
78.18	0\\
78.19	0\\
78.2	0\\
78.21	0\\
78.22	0\\
78.23	0\\
78.24	0\\
78.25	0\\
78.26	0\\
78.27	0\\
78.28	0\\
78.29	0\\
78.3	0\\
78.31	0\\
78.32	0\\
78.33	0\\
78.34	0\\
78.35	0\\
78.36	0\\
78.37	0\\
78.38	0\\
78.39	0\\
78.4	0\\
78.41	0\\
78.42	0\\
78.43	0\\
78.44	0\\
78.45	0\\
78.46	0\\
78.47	0\\
78.48	0\\
78.49	0\\
78.5	0\\
78.51	0\\
78.52	0\\
78.53	0\\
78.54	0\\
78.55	0\\
78.56	0\\
78.57	0\\
78.58	0\\
78.59	0\\
78.6	0\\
78.61	0\\
78.62	0\\
78.63	0\\
78.64	0\\
78.65	0\\
78.66	0\\
78.67	0\\
78.68	0\\
78.69	0\\
78.7	0\\
78.71	0\\
78.72	0\\
78.73	0\\
78.74	0\\
78.75	0\\
78.76	0\\
78.77	0\\
78.78	0\\
78.79	0\\
78.8	0\\
78.81	0\\
78.82	0\\
78.83	0\\
78.84	0\\
78.85	0\\
78.86	0\\
78.87	0\\
78.88	0\\
78.89	0\\
78.9	0\\
78.91	0\\
78.92	0\\
78.93	0\\
78.94	0\\
78.95	0\\
78.96	0\\
78.97	0\\
78.98	0\\
78.99	0\\
79	0\\
79.01	0\\
79.02	0\\
79.03	0\\
79.04	0\\
79.05	0\\
79.06	0\\
79.07	0\\
79.08	0\\
79.09	0\\
79.1	0\\
79.11	0\\
79.12	0\\
79.13	0\\
79.14	0\\
79.15	0\\
79.16	0\\
79.17	0\\
79.18	0\\
79.19	0\\
79.2	0\\
79.21	0\\
79.22	0\\
79.23	0\\
79.24	0\\
79.25	0\\
79.26	0\\
79.27	0\\
79.28	0\\
79.29	0\\
79.3	0\\
79.31	0\\
79.32	0\\
79.33	0\\
79.34	0\\
79.35	0\\
79.36	0\\
79.37	0\\
79.38	0\\
79.39	0\\
79.4	0\\
79.41	0\\
79.42	0\\
79.43	0\\
79.44	0\\
79.45	0\\
79.46	0\\
79.47	0\\
79.48	0\\
79.49	0\\
79.5	0\\
79.51	0\\
79.52	0\\
79.53	0\\
79.54	0\\
79.55	0\\
79.56	0\\
79.57	0\\
79.58	0\\
79.59	0\\
79.6	0\\
79.61	0\\
79.62	0\\
79.63	0\\
79.64	0\\
79.65	0\\
79.66	0\\
79.67	0\\
79.68	0\\
79.69	0\\
79.7	0\\
79.71	0\\
79.72	0\\
79.73	0\\
79.74	0\\
79.75	0\\
79.76	0\\
79.77	0\\
79.78	0\\
79.79	0\\
79.8	0\\
79.81	0\\
79.82	0\\
79.83	0\\
79.84	0\\
79.85	0\\
79.86	0\\
79.87	0\\
79.88	0\\
79.89	0\\
79.9	0\\
79.91	0\\
79.92	0\\
79.93	0\\
79.94	0\\
79.95	0\\
79.96	0\\
79.97	0\\
79.98	0\\
79.99	0\\
80	0\\
80.01	0\\
};
\addplot [color=blue,solid]
  table[row sep=crcr]{%
80.01	0\\
80.02	0\\
80.03	0\\
80.04	0\\
80.05	0\\
80.06	0\\
80.07	0\\
80.08	0\\
80.09	0\\
80.1	0\\
80.11	0\\
80.12	0\\
80.13	0\\
80.14	0\\
80.15	0\\
80.16	0\\
80.17	0\\
80.18	0\\
80.19	0\\
80.2	0\\
80.21	0\\
80.22	0\\
80.23	0\\
80.24	0\\
80.25	0\\
80.26	0\\
80.27	0\\
80.28	0\\
80.29	0\\
80.3	0\\
80.31	0\\
80.32	0\\
80.33	0\\
80.34	0\\
80.35	0\\
80.36	0\\
80.37	0\\
80.38	0\\
80.39	0\\
80.4	0\\
80.41	0\\
80.42	0\\
80.43	0\\
80.44	0\\
80.45	0\\
80.46	0\\
80.47	0\\
80.48	0\\
80.49	0\\
80.5	0\\
80.51	0\\
80.52	0\\
80.53	0\\
80.54	0\\
80.55	0\\
80.56	0\\
80.57	0\\
80.58	0\\
80.59	0\\
80.6	0\\
80.61	0\\
80.62	0\\
80.63	0\\
80.64	0\\
80.65	0\\
80.66	0\\
80.67	0\\
80.68	0\\
80.69	0\\
80.7	0\\
80.71	0\\
80.72	0\\
80.73	0\\
80.74	0\\
80.75	0\\
80.76	0\\
80.77	0\\
80.78	0\\
80.79	0\\
80.8	0\\
80.81	0\\
80.82	0\\
80.83	0\\
80.84	0\\
80.85	0\\
80.86	0\\
80.87	0\\
80.88	0\\
80.89	0\\
80.9	0\\
80.91	0\\
80.92	0\\
80.93	0\\
80.94	0\\
80.95	0\\
80.96	0\\
80.97	0\\
80.98	0\\
80.99	0\\
81	0\\
81.01	0\\
81.02	0\\
81.03	0\\
81.04	0\\
81.05	0\\
81.06	0\\
81.07	0\\
81.08	0\\
81.09	0\\
81.1	0\\
81.11	0\\
81.12	0\\
81.13	0\\
81.14	0\\
81.15	0\\
81.16	0\\
81.17	0\\
81.18	0\\
81.19	0\\
81.2	0\\
81.21	0\\
81.22	0\\
81.23	0\\
81.24	0\\
81.25	0\\
81.26	0\\
81.27	0\\
81.28	0\\
81.29	0\\
81.3	0\\
81.31	0\\
81.32	0\\
81.33	0\\
81.34	0\\
81.35	0\\
81.36	0\\
81.37	0\\
81.38	0\\
81.39	0\\
81.4	0\\
81.41	0\\
81.42	0\\
81.43	0\\
81.44	0\\
81.45	0\\
81.46	0\\
81.47	0\\
81.48	0\\
81.49	0\\
81.5	0\\
81.51	0\\
81.52	0\\
81.53	0\\
81.54	0\\
81.55	0\\
81.56	0\\
81.57	0\\
81.58	0\\
81.59	0\\
81.6	0\\
81.61	0\\
81.62	0\\
81.63	0\\
81.64	0\\
81.65	0\\
81.66	0\\
81.67	0\\
81.68	0\\
81.69	0\\
81.7	0\\
81.71	0\\
81.72	0\\
81.73	0\\
81.74	0\\
81.75	0\\
81.76	0\\
81.77	0\\
81.78	0\\
81.79	0\\
81.8	0\\
81.81	0\\
81.82	0\\
81.83	0\\
81.84	0\\
81.85	0\\
81.86	0\\
81.87	0\\
81.88	0\\
81.89	0\\
81.9	0\\
81.91	0\\
81.92	0\\
81.93	0\\
81.94	0\\
81.95	0\\
81.96	0\\
81.97	0\\
81.98	0\\
81.99	0\\
82	0\\
82.01	0\\
82.02	0\\
82.03	0\\
82.04	0\\
82.05	0\\
82.06	0\\
82.07	0\\
82.08	0\\
82.09	0\\
82.1	0\\
82.11	0\\
82.12	0\\
82.13	0\\
82.14	0\\
82.15	0\\
82.16	0\\
82.17	0\\
82.18	0\\
82.19	0\\
82.2	0\\
82.21	0\\
82.22	0\\
82.23	0\\
82.24	0\\
82.25	0\\
82.26	0\\
82.27	0\\
82.28	0\\
82.29	0\\
82.3	0\\
82.31	0\\
82.32	0\\
82.33	0\\
82.34	0\\
82.35	0\\
82.36	0\\
82.37	0\\
82.38	0\\
82.39	0\\
82.4	0\\
82.41	0\\
82.42	0\\
82.43	0\\
82.44	0\\
82.45	0\\
82.46	0\\
82.47	0\\
82.48	0\\
82.49	0\\
82.5	0\\
82.51	0\\
82.52	0\\
82.53	0\\
82.54	0\\
82.55	0\\
82.56	0\\
82.57	0\\
82.58	0\\
82.59	0\\
82.6	0\\
82.61	0\\
82.62	0\\
82.63	0\\
82.64	0\\
82.65	0\\
82.66	0\\
82.67	0\\
82.68	0\\
82.69	0\\
82.7	0\\
82.71	0\\
82.72	0\\
82.73	0\\
82.74	0\\
82.75	0\\
82.76	0\\
82.77	0\\
82.78	0\\
82.79	0\\
82.8	0\\
82.81	0\\
82.82	0\\
82.83	0\\
82.84	0\\
82.85	0\\
82.86	0\\
82.87	0\\
82.88	0\\
82.89	0\\
82.9	0\\
82.91	0\\
82.92	0\\
82.93	0\\
82.94	0\\
82.95	0\\
82.96	0\\
82.97	0\\
82.98	0\\
82.99	0\\
83	0\\
83.01	0\\
83.02	0\\
83.03	0\\
83.04	0\\
83.05	0\\
83.06	0\\
83.07	0\\
83.08	0\\
83.09	0\\
83.1	0\\
83.11	0\\
83.12	0\\
83.13	0\\
83.14	0\\
83.15	0\\
83.16	0\\
83.17	0\\
83.18	0\\
83.19	0\\
83.2	0\\
83.21	0\\
83.22	0\\
83.23	0\\
83.24	0\\
83.25	0\\
83.26	0\\
83.27	0\\
83.28	0\\
83.29	0\\
83.3	0\\
83.31	0\\
83.32	0\\
83.33	0\\
83.34	0\\
83.35	0\\
83.36	0\\
83.37	0\\
83.38	0\\
83.39	0\\
83.4	0\\
83.41	0\\
83.42	0\\
83.43	0\\
83.44	0\\
83.45	0\\
83.46	0\\
83.47	0\\
83.48	0\\
83.49	0\\
83.5	0\\
83.51	0\\
83.52	0\\
83.53	0\\
83.54	0\\
83.55	0\\
83.56	0\\
83.57	0\\
83.58	0\\
83.59	0\\
83.6	0\\
83.61	0\\
83.62	0\\
83.63	0\\
83.64	0\\
83.65	0\\
83.66	0\\
83.67	0\\
83.68	0\\
83.69	0\\
83.7	0\\
83.71	0\\
83.72	0\\
83.73	0\\
83.74	0\\
83.75	0\\
83.76	0\\
83.77	0\\
83.78	0\\
83.79	0\\
83.8	0\\
83.81	0\\
83.82	0\\
83.83	0\\
83.84	0\\
83.85	0\\
83.86	0\\
83.87	0\\
83.88	0\\
83.89	0\\
83.9	0\\
83.91	0\\
83.92	0\\
83.93	0\\
83.94	0\\
83.95	0\\
83.96	0\\
83.97	0\\
83.98	0\\
83.99	0\\
84	0\\
84.01	0\\
84.02	0\\
84.03	0\\
84.04	0\\
84.05	0\\
84.06	0\\
84.07	0\\
84.08	0\\
84.09	0\\
84.1	0\\
84.11	0\\
84.12	0\\
84.13	0\\
84.14	0\\
84.15	0\\
84.16	0\\
84.17	0\\
84.18	0\\
84.19	0\\
84.2	0\\
84.21	0\\
84.22	0\\
84.23	0\\
84.24	0\\
84.25	0\\
84.26	0\\
84.27	0\\
84.28	0\\
84.29	0\\
84.3	0\\
84.31	0\\
84.32	0\\
84.33	0\\
84.34	0\\
84.35	0\\
84.36	0\\
84.37	0\\
84.38	0\\
84.39	0\\
84.4	0\\
84.41	0\\
84.42	0\\
84.43	0\\
84.44	0\\
84.45	0\\
84.46	0\\
84.47	0\\
84.48	0\\
84.49	0\\
84.5	0\\
84.51	0\\
84.52	0\\
84.53	0\\
84.54	0\\
84.55	0\\
84.56	0\\
84.57	0\\
84.58	0\\
84.59	0\\
84.6	0\\
84.61	0\\
84.62	0\\
84.63	0\\
84.64	0\\
84.65	0\\
84.66	0\\
84.67	0\\
84.68	0\\
84.69	0\\
84.7	0\\
84.71	0\\
84.72	0\\
84.73	0\\
84.74	0\\
84.75	0\\
84.76	0\\
84.77	0\\
84.78	0\\
84.79	0\\
84.8	0\\
84.81	0\\
84.82	0\\
84.83	0\\
84.84	0\\
84.85	0\\
84.86	0\\
84.87	0\\
84.88	0\\
84.89	0\\
84.9	0\\
84.91	0\\
84.92	0\\
84.93	0\\
84.94	0\\
84.95	0\\
84.96	0\\
84.97	0\\
84.98	0\\
84.99	0\\
85	0\\
85.01	0\\
85.02	0\\
85.03	0\\
85.04	0\\
85.05	0\\
85.06	0\\
85.07	0\\
85.08	0\\
85.09	0\\
85.1	0\\
85.11	0\\
85.12	0\\
85.13	0\\
85.14	0\\
85.15	0\\
85.16	0\\
85.17	0\\
85.18	0\\
85.19	0\\
85.2	0\\
85.21	0\\
85.22	0\\
85.23	0\\
85.24	0\\
85.25	0\\
85.26	0\\
85.27	0\\
85.28	0\\
85.29	0\\
85.3	0\\
85.31	0\\
85.32	0\\
85.33	0\\
85.34	0\\
85.35	0\\
85.36	0\\
85.37	0\\
85.38	0\\
85.39	0\\
85.4	0\\
85.41	0\\
85.42	0\\
85.43	0\\
85.44	0\\
85.45	0\\
85.46	0\\
85.47	0\\
85.48	0\\
85.49	0\\
85.5	0\\
85.51	0\\
85.52	0\\
85.53	0\\
85.54	0\\
85.55	0\\
85.56	0\\
85.57	0\\
85.58	0\\
85.59	0\\
85.6	0\\
85.61	0\\
85.62	0\\
85.63	0\\
85.64	0\\
85.65	0\\
85.66	0\\
85.67	0\\
85.68	0\\
85.69	0\\
85.7	0\\
85.71	0\\
85.72	0\\
85.73	0\\
85.74	0\\
85.75	0\\
85.76	0\\
85.77	0\\
85.78	0\\
85.79	0\\
85.8	0\\
85.81	0\\
85.82	0\\
85.83	0\\
85.84	0\\
85.85	0\\
85.86	0\\
85.87	0\\
85.88	0\\
85.89	0\\
85.9	0\\
85.91	0\\
85.92	0\\
85.93	0\\
85.94	0\\
85.95	0\\
85.96	0\\
85.97	0\\
85.98	0\\
85.99	0\\
86	0\\
86.01	0\\
86.02	0\\
86.03	0\\
86.04	0\\
86.05	0\\
86.06	0\\
86.07	0\\
86.08	0\\
86.09	0\\
86.1	0\\
86.11	0\\
86.12	0\\
86.13	0\\
86.14	0\\
86.15	0\\
86.16	0\\
86.17	0\\
86.18	0\\
86.19	0\\
86.2	0\\
86.21	0\\
86.22	0\\
86.23	0\\
86.24	0\\
86.25	0\\
86.26	0\\
86.27	0\\
86.28	0\\
86.29	0\\
86.3	0\\
86.31	0\\
86.32	0\\
86.33	0\\
86.34	0\\
86.35	0\\
86.36	0\\
86.37	0\\
86.38	0\\
86.39	0\\
86.4	0\\
86.41	0\\
86.42	0\\
86.43	0\\
86.44	0\\
86.45	0\\
86.46	0\\
86.47	0\\
86.48	0\\
86.49	0\\
86.5	0\\
86.51	0\\
86.52	0\\
86.53	0\\
86.54	0\\
86.55	0\\
86.56	0\\
86.57	0\\
86.58	0\\
86.59	0\\
86.6	0\\
86.61	0\\
86.62	0\\
86.63	0\\
86.64	0\\
86.65	0\\
86.66	0\\
86.67	0\\
86.68	0\\
86.69	0\\
86.7	0\\
86.71	0\\
86.72	0\\
86.73	0\\
86.74	0\\
86.75	0\\
86.76	0\\
86.77	0\\
86.78	0\\
86.79	0\\
86.8	0\\
86.81	0\\
86.82	0\\
86.83	0\\
86.84	0\\
86.85	0\\
86.86	0\\
86.87	0\\
86.88	0\\
86.89	0\\
86.9	0\\
86.91	0\\
86.92	0\\
86.93	0\\
86.94	0\\
86.95	0\\
86.96	0\\
86.97	0\\
86.98	0\\
86.99	0\\
87	0\\
87.01	0\\
87.02	0\\
87.03	0\\
87.04	0\\
87.05	0\\
87.06	0\\
87.07	0\\
87.08	0\\
87.09	0\\
87.1	0\\
87.11	0\\
87.12	0\\
87.13	0\\
87.14	0\\
87.15	0\\
87.16	0\\
87.17	0\\
87.18	0\\
87.19	0\\
87.2	0\\
87.21	0\\
87.22	0\\
87.23	0\\
87.24	0\\
87.25	0\\
87.26	0\\
87.27	0\\
87.28	0\\
87.29	0\\
87.3	0\\
87.31	0\\
87.32	0\\
87.33	0\\
87.34	0\\
87.35	0\\
87.36	0\\
87.37	0\\
87.38	0\\
87.39	0\\
87.4	0\\
87.41	0\\
87.42	0\\
87.43	0\\
87.44	0\\
87.45	0\\
87.46	0\\
87.47	0\\
87.48	0\\
87.49	0\\
87.5	0\\
87.51	0\\
87.52	0\\
87.53	0\\
87.54	0\\
87.55	0\\
87.56	0\\
87.57	0\\
87.58	0\\
87.59	0\\
87.6	0\\
87.61	0\\
87.62	0\\
87.63	0\\
87.64	0\\
87.65	0\\
87.66	0\\
87.67	0\\
87.68	0\\
87.69	0\\
87.7	0\\
87.71	0\\
87.72	0\\
87.73	0\\
87.74	0\\
87.75	0\\
87.76	0\\
87.77	0\\
87.78	0\\
87.79	0\\
87.8	0\\
87.81	0\\
87.82	0\\
87.83	0\\
87.84	0\\
87.85	0\\
87.86	0\\
87.87	0\\
87.88	0\\
87.89	0\\
87.9	0\\
87.91	0\\
87.92	0\\
87.93	0\\
87.94	0\\
87.95	0\\
87.96	0\\
87.97	0\\
87.98	0\\
87.99	0\\
88	0\\
88.01	0\\
88.02	0\\
88.03	0\\
88.04	0\\
88.05	0\\
88.06	0\\
88.07	0\\
88.08	0\\
88.09	0\\
88.1	0\\
88.11	0\\
88.12	0\\
88.13	0\\
88.14	0\\
88.15	0\\
88.16	0\\
88.17	0\\
88.18	0\\
88.19	0\\
88.2	0\\
88.21	0\\
88.22	0\\
88.23	0\\
88.24	0\\
88.25	0\\
88.26	0\\
88.27	0\\
88.28	0\\
88.29	0\\
88.3	0\\
88.31	0\\
88.32	0\\
88.33	0\\
88.34	0\\
88.35	0\\
88.36	0\\
88.37	0\\
88.38	0\\
88.39	0\\
88.4	0\\
88.41	0\\
88.42	0\\
88.43	0\\
88.44	0\\
88.45	0\\
88.46	0\\
88.47	0\\
88.48	0\\
88.49	0\\
88.5	0\\
88.51	0\\
88.52	0\\
88.53	0\\
88.54	0\\
88.55	0\\
88.56	0\\
88.57	0\\
88.58	0\\
88.59	0\\
88.6	0\\
88.61	0\\
88.62	0\\
88.63	0\\
88.64	0\\
88.65	0\\
88.66	0\\
88.67	0\\
88.68	0\\
88.69	0\\
88.7	0\\
88.71	0\\
88.72	0\\
88.73	0\\
88.74	0\\
88.75	0\\
88.76	0\\
88.77	0\\
88.78	0\\
88.79	0\\
88.8	0\\
88.81	0\\
88.82	0\\
88.83	0\\
88.84	0\\
88.85	0\\
88.86	0\\
88.87	0\\
88.88	0\\
88.89	0\\
88.9	0\\
88.91	0\\
88.92	0\\
88.93	0\\
88.94	0\\
88.95	0\\
88.96	0\\
88.97	0\\
88.98	0\\
88.99	0\\
89	0\\
89.01	0\\
89.02	0\\
89.03	0\\
89.04	0\\
89.05	0\\
89.06	0\\
89.07	0\\
89.08	0\\
89.09	0\\
89.1	0\\
89.11	0\\
89.12	0\\
89.13	0\\
89.14	0\\
89.15	0\\
89.16	0\\
89.17	0\\
89.18	0\\
89.19	0\\
89.2	0\\
89.21	0\\
89.22	0\\
89.23	0\\
89.24	0\\
89.25	0\\
89.26	0\\
89.27	0\\
89.28	0\\
89.29	0\\
89.3	0\\
89.31	0\\
89.32	0\\
89.33	0\\
89.34	0\\
89.35	0\\
89.36	0\\
89.37	0\\
89.38	0\\
89.39	0\\
89.4	0\\
89.41	0\\
89.42	0\\
89.43	0\\
89.44	0\\
89.45	0\\
89.46	0\\
89.47	0\\
89.48	0\\
89.49	0\\
89.5	0\\
89.51	0\\
89.52	0\\
89.53	0\\
89.54	0\\
89.55	0\\
89.56	0\\
89.57	0\\
89.58	0\\
89.59	0\\
89.6	0\\
89.61	0\\
89.62	0\\
89.63	0\\
89.64	0\\
89.65	0\\
89.66	0\\
89.67	0\\
89.68	0\\
89.69	0\\
89.7	0\\
89.71	0\\
89.72	0\\
89.73	0\\
89.74	0\\
89.75	0\\
89.76	0\\
89.77	0\\
89.78	0\\
89.79	0\\
89.8	0\\
89.81	0\\
89.82	0\\
89.83	0\\
89.84	0\\
89.85	0\\
89.86	0\\
89.87	0\\
89.88	0\\
89.89	0\\
89.9	0\\
89.91	0\\
89.92	0\\
89.93	0\\
89.94	0\\
89.95	0\\
89.96	0\\
89.97	0\\
89.98	0\\
89.99	0\\
90	0\\
90.01	0\\
90.02	0\\
90.03	0\\
90.04	0\\
90.05	0\\
90.06	0\\
90.07	0\\
90.08	0\\
90.09	0\\
90.1	0\\
90.11	0\\
90.12	0\\
90.13	0\\
90.14	0\\
90.15	0\\
90.16	0\\
90.17	0\\
90.18	0\\
90.19	0\\
90.2	0\\
90.21	0\\
90.22	0\\
90.23	0\\
90.24	0\\
90.25	0\\
90.26	0\\
90.27	0\\
90.28	0\\
90.29	0\\
90.3	0\\
90.31	0\\
90.32	0\\
90.33	0\\
90.34	0\\
90.35	0\\
90.36	0\\
90.37	0\\
90.38	0\\
90.39	0\\
90.4	0\\
90.41	0\\
90.42	0\\
90.43	0\\
90.44	0\\
90.45	0\\
90.46	0\\
90.47	0\\
90.48	0\\
90.49	0\\
90.5	0\\
90.51	0\\
90.52	0\\
90.53	0\\
90.54	0\\
90.55	0\\
90.56	0\\
90.57	0\\
90.58	0\\
90.59	0\\
90.6	0\\
90.61	0\\
90.62	0\\
90.63	0\\
90.64	0\\
90.65	0\\
90.66	0\\
90.67	0\\
90.68	0\\
90.69	0\\
90.7	0\\
90.71	0\\
90.72	0\\
90.73	0\\
90.74	0\\
90.75	0\\
90.76	0\\
90.77	0\\
90.78	0\\
90.79	0\\
90.8	0\\
90.81	0\\
90.82	0\\
90.83	0\\
90.84	0\\
90.85	0\\
90.86	0\\
90.87	0\\
90.88	0\\
90.89	0\\
90.9	0\\
90.91	0\\
90.92	0\\
90.93	0\\
90.94	0\\
90.95	0\\
90.96	0\\
90.97	0\\
90.98	0\\
90.99	0\\
91	0\\
91.01	0\\
91.02	0\\
91.03	0\\
91.04	0\\
91.05	0\\
91.06	0\\
91.07	0\\
91.08	0\\
91.09	0\\
91.1	0\\
91.11	0\\
91.12	0\\
91.13	0\\
91.14	0\\
91.15	0\\
91.16	0\\
91.17	0\\
91.18	0\\
91.19	0\\
91.2	0\\
91.21	0\\
91.22	0\\
91.23	0\\
91.24	0\\
91.25	0\\
91.26	0\\
91.27	0\\
91.28	0\\
91.29	0\\
91.3	0\\
91.31	0\\
91.32	0\\
91.33	0\\
91.34	0\\
91.35	0\\
91.36	0\\
91.37	0\\
91.38	0\\
91.39	0\\
91.4	0\\
91.41	0\\
91.42	0\\
91.43	0\\
91.44	0\\
91.45	0\\
91.46	0\\
91.47	0\\
91.48	0\\
91.49	0\\
91.5	0\\
91.51	0\\
91.52	0\\
91.53	0\\
91.54	0\\
91.55	0\\
91.56	0\\
91.57	0\\
91.58	0\\
91.59	0\\
91.6	0\\
91.61	0\\
91.62	0\\
91.63	0\\
91.64	0\\
91.65	0\\
91.66	0\\
91.67	0\\
91.68	0\\
91.69	0\\
91.7	0\\
91.71	0\\
91.72	0\\
91.73	0\\
91.74	0\\
91.75	0\\
91.76	0\\
91.77	0\\
91.78	0\\
91.79	0\\
91.8	0\\
91.81	0\\
91.82	0\\
91.83	0\\
91.84	0\\
91.85	0\\
91.86	0\\
91.87	0\\
91.88	0\\
91.89	0\\
91.9	0\\
91.91	0\\
91.92	0\\
91.93	0\\
91.94	0\\
91.95	0\\
91.96	0\\
91.97	0\\
91.98	0\\
91.99	0\\
92	0\\
92.01	0\\
92.02	0\\
92.03	0\\
92.04	0\\
92.05	0\\
92.06	0\\
92.07	0\\
92.08	0\\
92.09	0\\
92.1	0\\
92.11	0\\
92.12	0\\
92.13	0\\
92.14	0\\
92.15	0\\
92.16	0\\
92.17	0\\
92.18	0\\
92.19	0\\
92.2	0\\
92.21	0\\
92.22	0\\
92.23	0\\
92.24	0\\
92.25	0\\
92.26	0\\
92.27	0\\
92.28	0\\
92.29	0\\
92.3	0\\
92.31	0\\
92.32	0\\
92.33	0\\
92.34	0\\
92.35	0\\
92.36	0\\
92.37	0\\
92.38	0\\
92.39	0\\
92.4	0\\
92.41	0\\
92.42	0\\
92.43	0\\
92.44	0\\
92.45	0\\
92.46	0\\
92.47	0\\
92.48	0\\
92.49	0\\
92.5	0\\
92.51	0\\
92.52	0\\
92.53	0\\
92.54	0\\
92.55	0\\
92.56	0\\
92.57	0\\
92.58	0\\
92.59	0\\
92.6	0\\
92.61	0\\
92.62	0\\
92.63	0\\
92.64	0\\
92.65	0\\
92.66	0\\
92.67	0\\
92.68	0\\
92.69	0\\
92.7	0\\
92.71	0\\
92.72	0\\
92.73	0\\
92.74	0\\
92.75	0\\
92.76	0\\
92.77	0\\
92.78	0\\
92.79	0\\
92.8	0\\
92.81	0\\
92.82	0\\
92.83	0\\
92.84	0\\
92.85	0\\
92.86	0\\
92.87	0\\
92.88	0\\
92.89	0\\
92.9	0\\
92.91	0\\
92.92	0\\
92.93	0\\
92.94	0\\
92.95	0\\
92.96	0\\
92.97	0\\
92.98	0\\
92.99	0\\
93	0\\
93.01	0\\
93.02	0\\
93.03	0\\
93.04	0\\
93.05	0\\
93.06	0\\
93.07	0\\
93.08	0\\
93.09	0\\
93.1	0\\
93.11	0\\
93.12	0\\
93.13	0\\
93.14	0\\
93.15	0\\
93.16	0\\
93.17	0\\
93.18	0\\
93.19	0\\
93.2	0\\
93.21	0\\
93.22	0\\
93.23	0\\
93.24	0\\
93.25	0\\
93.26	0\\
93.27	0\\
93.28	0\\
93.29	0\\
93.3	0\\
93.31	0\\
93.32	0\\
93.33	0\\
93.34	0\\
93.35	0\\
93.36	0\\
93.37	0\\
93.38	0\\
93.39	0\\
93.4	0\\
93.41	0\\
93.42	0\\
93.43	0\\
93.44	0\\
93.45	0\\
93.46	0\\
93.47	0\\
93.48	0\\
93.49	0\\
93.5	0\\
93.51	0\\
93.52	0\\
93.53	0\\
93.54	0\\
93.55	0\\
93.56	0\\
93.57	0\\
93.58	0\\
93.59	0\\
93.6	0\\
93.61	0\\
93.62	0\\
93.63	0\\
93.64	0\\
93.65	0\\
93.66	0\\
93.67	0\\
93.68	0\\
93.69	0\\
93.7	0\\
93.71	0\\
93.72	0\\
93.73	0\\
93.74	0\\
93.75	0\\
93.76	0\\
93.77	0\\
93.78	0\\
93.79	0\\
93.8	0\\
93.81	0\\
93.82	0\\
93.83	0\\
93.84	0\\
93.85	0\\
93.86	0\\
93.87	0\\
93.88	0\\
93.89	0\\
93.9	0\\
93.91	0\\
93.92	0\\
93.93	0\\
93.94	0\\
93.95	0\\
93.96	0\\
93.97	0\\
93.98	0\\
93.99	0\\
94	0\\
94.01	0\\
94.02	0\\
94.03	0\\
94.04	0\\
94.05	0\\
94.06	0\\
94.07	0\\
94.08	0\\
94.09	0\\
94.1	0\\
94.11	0\\
94.12	0\\
94.13	0\\
94.14	0\\
94.15	0\\
94.16	0\\
94.17	0\\
94.18	0\\
94.19	0\\
94.2	0\\
94.21	0\\
94.22	0\\
94.23	0\\
94.24	0\\
94.25	0\\
94.26	0\\
94.27	0\\
94.28	0\\
94.29	0\\
94.3	0\\
94.31	0\\
94.32	0\\
94.33	0\\
94.34	0\\
94.35	0\\
94.36	0\\
94.37	0\\
94.38	0\\
94.39	0\\
94.4	0\\
94.41	0\\
94.42	0\\
94.43	0\\
94.44	0\\
94.45	0\\
94.46	0\\
94.47	0\\
94.48	0\\
94.49	0\\
94.5	0\\
94.51	0\\
94.52	0\\
94.53	0\\
94.54	0\\
94.55	0\\
94.56	0\\
94.57	0\\
94.58	0\\
94.59	0\\
94.6	0\\
94.61	0\\
94.62	0\\
94.63	0\\
94.64	0\\
94.65	0\\
94.66	0\\
94.67	0\\
94.68	0\\
94.69	0\\
94.7	0\\
94.71	0\\
94.72	0\\
94.73	0\\
94.74	0\\
94.75	0\\
94.76	0\\
94.77	0\\
94.78	0\\
94.79	0\\
94.8	0\\
94.81	0\\
94.82	0\\
94.83	0\\
94.84	0\\
94.85	0\\
94.86	0\\
94.87	0\\
94.88	0\\
94.89	0\\
94.9	0\\
94.91	0\\
94.92	0\\
94.93	0\\
94.94	0\\
94.95	0\\
94.96	0\\
94.97	0\\
94.98	0\\
94.99	0\\
95	0\\
95.01	0\\
95.02	0\\
95.03	0\\
95.04	0\\
95.05	0\\
95.06	0\\
95.07	0\\
95.08	0\\
95.09	0\\
95.1	0\\
95.11	0\\
95.12	0\\
95.13	0\\
95.14	0\\
95.15	0\\
95.16	0\\
95.17	0\\
95.18	0\\
95.19	0\\
95.2	0\\
95.21	0\\
95.22	0\\
95.23	0\\
95.24	0\\
95.25	0\\
95.26	0\\
95.27	0\\
95.28	0\\
95.29	0\\
95.3	0\\
95.31	0\\
95.32	0\\
95.33	0\\
95.34	0\\
95.35	0\\
95.36	0\\
95.37	0\\
95.38	0\\
95.39	0\\
95.4	0\\
95.41	0\\
95.42	0\\
95.43	0\\
95.44	0\\
95.45	0\\
95.46	0\\
95.47	0\\
95.48	0\\
95.49	0\\
95.5	0\\
95.51	0\\
95.52	0\\
95.53	0\\
95.54	0\\
95.55	0\\
95.56	0\\
95.57	0\\
95.58	0\\
95.59	0\\
95.6	0\\
95.61	0\\
95.62	0\\
95.63	0\\
95.64	0\\
95.65	0\\
95.66	0\\
95.67	0\\
95.68	0\\
95.69	0\\
95.7	0\\
95.71	0\\
95.72	0\\
95.73	0\\
95.74	0\\
95.75	0\\
95.76	0\\
95.77	0\\
95.78	0\\
95.79	0\\
95.8	0\\
95.81	0\\
95.82	0\\
95.83	0\\
95.84	0\\
95.85	0\\
95.86	0\\
95.87	0\\
95.88	0\\
95.89	0\\
95.9	0\\
95.91	0\\
95.92	0\\
95.93	0\\
95.94	0\\
95.95	0\\
95.96	0\\
95.97	0\\
95.98	0\\
95.99	0\\
96	0\\
96.01	0\\
96.02	0\\
96.03	0\\
96.04	0\\
96.05	0\\
96.06	0\\
96.07	0\\
96.08	0\\
96.09	0\\
96.1	0\\
96.11	0\\
96.12	0\\
96.13	0\\
96.14	0\\
96.15	0\\
96.16	0\\
96.17	0\\
96.18	0\\
96.19	0\\
96.2	0\\
96.21	0\\
96.22	0\\
96.23	0\\
96.24	0\\
96.25	0\\
96.26	0\\
96.27	0\\
96.28	0\\
96.29	0\\
96.3	0\\
96.31	0\\
96.32	0\\
96.33	0\\
96.34	0\\
96.35	0\\
96.36	0\\
96.37	0\\
96.38	0\\
96.39	0\\
96.4	0\\
96.41	0\\
96.42	0\\
96.43	0\\
96.44	0\\
96.45	0\\
96.46	0\\
96.47	0\\
96.48	0\\
96.49	0\\
96.5	0\\
96.51	0\\
96.52	0\\
96.53	0\\
96.54	0\\
96.55	0\\
96.56	0\\
96.57	0\\
96.58	0\\
96.59	0\\
96.6	0\\
96.61	0\\
96.62	0\\
96.63	0\\
96.64	0\\
96.65	0\\
96.66	0\\
96.67	0\\
96.68	0\\
96.69	0\\
96.7	0\\
96.71	0\\
96.72	0\\
96.73	0\\
96.74	0\\
96.75	0\\
96.76	0\\
96.77	0\\
96.78	0\\
96.79	0\\
96.8	0\\
96.81	0\\
96.82	0\\
96.83	0\\
96.84	0\\
96.85	0\\
96.86	0\\
96.87	0\\
96.88	0\\
96.89	0\\
96.9	0\\
96.91	0\\
96.92	0\\
96.93	0\\
96.94	0\\
96.95	0\\
96.96	0\\
96.97	0\\
96.98	0\\
96.99	0\\
97	0\\
97.01	0\\
97.02	0\\
97.03	0\\
97.04	0\\
97.05	0\\
97.06	0\\
97.07	0\\
97.08	0\\
97.09	0\\
97.1	0\\
97.11	0\\
97.12	0\\
97.13	0\\
97.14	0\\
97.15	0\\
97.16	0\\
97.17	0\\
97.18	0\\
97.19	0\\
97.2	0\\
97.21	0\\
97.22	0\\
97.23	0\\
97.24	0\\
97.25	0\\
97.26	0\\
97.27	0\\
97.28	0\\
97.29	0\\
97.3	0\\
97.31	0\\
97.32	0\\
97.33	0\\
97.34	0\\
97.35	0\\
97.36	0\\
97.37	0\\
97.38	0\\
97.39	0\\
97.4	0\\
97.41	0\\
97.42	0\\
97.43	0\\
97.44	0\\
97.45	0\\
97.46	0\\
97.47	0\\
97.48	0\\
97.49	0\\
97.5	0\\
97.51	0\\
97.52	0\\
97.53	0\\
97.54	0\\
97.55	0\\
97.56	0\\
97.57	0\\
97.58	0\\
97.59	0\\
97.6	0\\
97.61	0\\
97.62	0\\
97.63	0\\
97.64	0\\
97.65	0\\
97.66	0\\
97.67	0\\
97.68	0\\
97.69	0\\
97.7	0\\
97.71	0\\
97.72	0\\
97.73	0\\
97.74	0\\
97.75	0\\
97.76	0\\
97.77	0\\
97.78	0\\
97.79	0\\
97.8	0\\
97.81	0\\
97.82	0\\
97.83	0\\
97.84	0\\
97.85	0\\
97.86	0\\
97.87	0\\
97.88	0\\
97.89	0\\
97.9	0\\
97.91	0\\
97.92	0\\
97.93	0\\
97.94	0\\
97.95	0\\
97.96	0\\
97.97	0\\
97.98	0\\
97.99	0\\
98	0\\
98.01	0\\
98.02	0\\
98.03	0\\
98.04	0\\
98.05	0\\
98.06	0\\
98.07	0\\
98.08	0\\
98.09	0\\
98.1	0\\
98.11	0\\
98.12	0\\
98.13	0\\
98.14	0\\
98.15	0\\
98.16	0\\
98.17	0\\
98.18	0\\
98.19	0\\
98.2	0\\
98.21	0\\
98.22	0\\
98.23	0\\
98.24	0\\
98.25	0\\
98.26	0\\
98.27	0\\
98.28	0\\
98.29	0\\
98.3	0\\
98.31	0\\
98.32	0\\
98.33	0\\
98.34	0\\
98.35	0\\
98.36	0\\
98.37	5.7998100325321e-05\\
98.38	0.000118506413573239\\
98.39	0.000179467720816638\\
98.4	0.000240886391479865\\
98.41	0.000302766842697373\\
98.42	0.000365113539938074\\
98.43	0.000427921760477259\\
98.44	0.000491193024538615\\
98.45	0.000554931731968803\\
98.46	0.000619142330284417\\
98.47	0.000683829315301882\\
98.48	0.00074899723177797\\
98.49	0.000814650674061151\\
98.5	0.000880794286754041\\
98.51	0.000947432765387158\\
98.52	0.00101457085710427\\
98.53	0.00108221336135956\\
98.54	0.00115036513062691\\
98.55	0.00121903107112158\\
98.56	0.00128821614353447\\
98.57	0.00135792536377948\\
98.58	0.00142816380375393\\
98.59	0.00149893659211269\\
98.6	0.00157024891505607\\
98.61	0.00164210601713198\\
98.62	0.00171451320205261\\
98.63	0.00178747583352593\\
98.64	0.0018609993361025\\
98.65	0.00193508919603787\\
98.66	0.00200975096216816\\
98.67	0.00208499024680375\\
98.68	0.0021608127266408\\
98.69	0.00220266311440171\\
98.7	0.00223385457855723\\
98.71	0.0022653127949539\\
98.72	0.00229704020679703\\
98.73	0.00232903928724017\\
98.74	0.00236131509615186\\
98.75	0.00239387224952333\\
98.76	0.00242671336449341\\
98.77	0.00245984108183159\\
98.78	0.00249325806613635\\
98.79	0.00252697449899538\\
98.8	0.00256099537475504\\
98.81	0.00259532360587348\\
98.82	0.00262996213227076\\
98.83	0.00266491392227347\\
98.84	0.00270018197272896\\
98.85	0.00273576930885851\\
98.86	0.002771678984517\\
98.87	0.00280791408245459\\
98.88	0.00284447771458059\\
98.89	0.00288137302222933\\
98.9	0.00291860317642809\\
98.91	0.00295617137816723\\
98.92	0.00299408085867234\\
98.93	0.00303233487967847\\
98.94	0.00307093673370654\\
98.95	0.00310988974434178\\
98.96	0.00314919726651431\\
98.97	0.00318886268678185\\
98.98	0.00322888942361441\\
98.99	0.00326928092768132\\
99	0.00331004068214008\\
99.01	0.00335117220292753\\
99.02	0.00339267903905301\\
99.03	0.00343456477289363\\
99.04	0.00347683302049166\\
99.05	0.00351948743185394\\
99.06	0.00356253169125339\\
99.07	0.00360596951753261\\
99.08	0.00364980462385488\\
99.09	0.00369404072412283\\
99.1	0.00373868156623795\\
99.11	0.00378373093239057\\
99.12	0.00382919263861687\\
99.13	0.00387507053562815\\
99.14	0.00392136850928232\\
99.15	0.00396809048088036\\
99.16	0.00401524040746425\\
99.17	0.00406282228211657\\
99.18	0.00411084013426162\\
99.19	0.00415929802996813\\
99.2	0.00420820007225344\\
99.21	0.00425755040138913\\
99.22	0.00430735319520808\\
99.23	0.0043576126694129\\
99.24	0.00440833307788572\\
99.25	0.0044595187129992\\
99.26	0.00451117390592883\\
99.27	0.00456330302696634\\
99.28	0.0046159104858343\\
99.29	0.00466900073200166\\
99.3	0.0047225782550004\\
99.31	0.00477664758474295\\
99.32	0.00483121329184059\\
99.33	0.00488627998792247\\
99.34	0.00494185232595538\\
99.35	0.00499793500056411\\
99.36	0.00505453274835219\\
99.37	0.0051116503482232\\
99.38	0.00516929262170212\\
99.39	0.00522746443325709\\
99.4	0.00528617069062104\\
99.41	0.0053454163451133\\
99.42	0.00540520639196104\\
99.43	0.0054655458706203\\
99.44	0.00552643986509657\\
99.45	0.00558789350426468\\
99.46	0.00564991196218789\\
99.47	0.00571250045843599\\
99.48	0.00577566425840224\\
99.49	0.00583940867361894\\
99.5	0.00590373906207135\\
99.51	0.00596866082851\\
99.52	0.00603417942476082\\
99.53	0.00610030035003307\\
99.54	0.00616702915122485\\
99.55	0.00623437142322573\\
99.56	0.00630233280921639\\
99.57	0.00637091900096489\\
99.58	0.00644013573911934\\
99.59	0.00650998881349655\\
99.6	0.0065804840633663\\
99.61	0.00665162737777062\\
99.62	0.00672342469581863\\
99.63	0.00679588200695593\\
99.64	0.00686900535122805\\
99.65	0.00694280081953754\\
99.66	0.00701727455389419\\
99.67	0.00709243274765783\\
99.68	0.00716828164577332\\
99.69	0.00724482754499706\\
99.7	0.0073220767941144\\
99.71	0.00740003575010869\\
99.72	0.0074787108180037\\
99.73	0.0075581084544986\\
99.74	0.00763823516812948\\
99.75	0.00771909751941802\\
99.76	0.00780070212100672\\
99.77	0.00788305563777966\\
99.78	0.00796616478696797\\
99.79	0.00805003633823904\\
99.8	0.00813467711376831\\
99.81	0.00822009398829281\\
99.82	0.0083062938891451\\
99.83	0.00839328379626649\\
99.84	0.00848107074219826\\
99.85	0.00856966181204955\\
99.86	0.00865906414344045\\
99.87	0.00874928492641887\\
99.88	0.00884033140334948\\
99.89	0.00893221086877317\\
99.9	0.00902493066923513\\
99.91	0.00911849820307978\\
99.92	0.0092129209202104\\
99.93	0.00930820632181158\\
99.94	0.00940436196003195\\
99.95	0.00950139543762514\\
99.96	0.00959931440754618\\
99.97	0.00969812657250083\\
99.98	0.00979783968444508\\
99.99	0.00989846154403157\\
100	0.01\\
};
\addlegendentry{$q=1$};

\addplot [color=red,solid,forget plot]
  table[row sep=crcr]{%
0.01	0\\
0.02	0\\
0.03	0\\
0.04	0\\
0.05	0\\
0.06	0\\
0.07	0\\
0.08	0\\
0.09	0\\
0.1	0\\
0.11	0\\
0.12	0\\
0.13	0\\
0.14	0\\
0.15	0\\
0.16	0\\
0.17	0\\
0.18	0\\
0.19	0\\
0.2	0\\
0.21	0\\
0.22	0\\
0.23	0\\
0.24	0\\
0.25	0\\
0.26	0\\
0.27	0\\
0.28	0\\
0.29	0\\
0.3	0\\
0.31	0\\
0.32	0\\
0.33	0\\
0.34	0\\
0.35	0\\
0.36	0\\
0.37	0\\
0.38	0\\
0.39	0\\
0.4	0\\
0.41	0\\
0.42	0\\
0.43	0\\
0.44	0\\
0.45	0\\
0.46	0\\
0.47	0\\
0.48	0\\
0.49	0\\
0.5	0\\
0.51	0\\
0.52	0\\
0.53	0\\
0.54	0\\
0.55	0\\
0.56	0\\
0.57	0\\
0.58	0\\
0.59	0\\
0.6	0\\
0.61	0\\
0.62	0\\
0.63	0\\
0.64	0\\
0.65	0\\
0.66	0\\
0.67	0\\
0.68	0\\
0.69	0\\
0.7	0\\
0.71	0\\
0.72	0\\
0.73	0\\
0.74	0\\
0.75	0\\
0.76	0\\
0.77	0\\
0.78	0\\
0.79	0\\
0.8	0\\
0.81	0\\
0.82	0\\
0.83	0\\
0.84	0\\
0.85	0\\
0.86	0\\
0.87	0\\
0.88	0\\
0.89	0\\
0.9	0\\
0.91	0\\
0.92	0\\
0.93	0\\
0.94	0\\
0.95	0\\
0.96	0\\
0.97	0\\
0.98	0\\
0.99	0\\
1	0\\
1.01	0\\
1.02	0\\
1.03	0\\
1.04	0\\
1.05	0\\
1.06	0\\
1.07	0\\
1.08	0\\
1.09	0\\
1.1	0\\
1.11	0\\
1.12	0\\
1.13	0\\
1.14	0\\
1.15	0\\
1.16	0\\
1.17	0\\
1.18	0\\
1.19	0\\
1.2	0\\
1.21	0\\
1.22	0\\
1.23	0\\
1.24	0\\
1.25	0\\
1.26	0\\
1.27	0\\
1.28	0\\
1.29	0\\
1.3	0\\
1.31	0\\
1.32	0\\
1.33	0\\
1.34	0\\
1.35	0\\
1.36	0\\
1.37	0\\
1.38	0\\
1.39	0\\
1.4	0\\
1.41	0\\
1.42	0\\
1.43	0\\
1.44	0\\
1.45	0\\
1.46	0\\
1.47	0\\
1.48	0\\
1.49	0\\
1.5	0\\
1.51	0\\
1.52	0\\
1.53	0\\
1.54	0\\
1.55	0\\
1.56	0\\
1.57	0\\
1.58	0\\
1.59	0\\
1.6	0\\
1.61	0\\
1.62	0\\
1.63	0\\
1.64	0\\
1.65	0\\
1.66	0\\
1.67	0\\
1.68	0\\
1.69	0\\
1.7	0\\
1.71	0\\
1.72	0\\
1.73	0\\
1.74	0\\
1.75	0\\
1.76	0\\
1.77	0\\
1.78	0\\
1.79	0\\
1.8	0\\
1.81	0\\
1.82	0\\
1.83	0\\
1.84	0\\
1.85	0\\
1.86	0\\
1.87	0\\
1.88	0\\
1.89	0\\
1.9	0\\
1.91	0\\
1.92	0\\
1.93	0\\
1.94	0\\
1.95	0\\
1.96	0\\
1.97	0\\
1.98	0\\
1.99	0\\
2	0\\
2.01	0\\
2.02	0\\
2.03	0\\
2.04	0\\
2.05	0\\
2.06	0\\
2.07	0\\
2.08	0\\
2.09	0\\
2.1	0\\
2.11	0\\
2.12	0\\
2.13	0\\
2.14	0\\
2.15	0\\
2.16	0\\
2.17	0\\
2.18	0\\
2.19	0\\
2.2	0\\
2.21	0\\
2.22	0\\
2.23	0\\
2.24	0\\
2.25	0\\
2.26	0\\
2.27	0\\
2.28	0\\
2.29	0\\
2.3	0\\
2.31	0\\
2.32	0\\
2.33	0\\
2.34	0\\
2.35	0\\
2.36	0\\
2.37	0\\
2.38	0\\
2.39	0\\
2.4	0\\
2.41	0\\
2.42	0\\
2.43	0\\
2.44	0\\
2.45	0\\
2.46	0\\
2.47	0\\
2.48	0\\
2.49	0\\
2.5	0\\
2.51	0\\
2.52	0\\
2.53	0\\
2.54	0\\
2.55	0\\
2.56	0\\
2.57	0\\
2.58	0\\
2.59	0\\
2.6	0\\
2.61	0\\
2.62	0\\
2.63	0\\
2.64	0\\
2.65	0\\
2.66	0\\
2.67	0\\
2.68	0\\
2.69	0\\
2.7	0\\
2.71	0\\
2.72	0\\
2.73	0\\
2.74	0\\
2.75	0\\
2.76	0\\
2.77	0\\
2.78	0\\
2.79	0\\
2.8	0\\
2.81	0\\
2.82	0\\
2.83	0\\
2.84	0\\
2.85	0\\
2.86	0\\
2.87	0\\
2.88	0\\
2.89	0\\
2.9	0\\
2.91	0\\
2.92	0\\
2.93	0\\
2.94	0\\
2.95	0\\
2.96	0\\
2.97	0\\
2.98	0\\
2.99	0\\
3	0\\
3.01	0\\
3.02	0\\
3.03	0\\
3.04	0\\
3.05	0\\
3.06	0\\
3.07	0\\
3.08	0\\
3.09	0\\
3.1	0\\
3.11	0\\
3.12	0\\
3.13	0\\
3.14	0\\
3.15	0\\
3.16	0\\
3.17	0\\
3.18	0\\
3.19	0\\
3.2	0\\
3.21	0\\
3.22	0\\
3.23	0\\
3.24	0\\
3.25	0\\
3.26	0\\
3.27	0\\
3.28	0\\
3.29	0\\
3.3	0\\
3.31	0\\
3.32	0\\
3.33	0\\
3.34	0\\
3.35	0\\
3.36	0\\
3.37	0\\
3.38	0\\
3.39	0\\
3.4	0\\
3.41	0\\
3.42	0\\
3.43	0\\
3.44	0\\
3.45	0\\
3.46	0\\
3.47	0\\
3.48	0\\
3.49	0\\
3.5	0\\
3.51	0\\
3.52	0\\
3.53	0\\
3.54	0\\
3.55	0\\
3.56	0\\
3.57	0\\
3.58	0\\
3.59	0\\
3.6	0\\
3.61	0\\
3.62	0\\
3.63	0\\
3.64	0\\
3.65	0\\
3.66	0\\
3.67	0\\
3.68	0\\
3.69	0\\
3.7	0\\
3.71	0\\
3.72	0\\
3.73	0\\
3.74	0\\
3.75	0\\
3.76	0\\
3.77	0\\
3.78	0\\
3.79	0\\
3.8	0\\
3.81	0\\
3.82	0\\
3.83	0\\
3.84	0\\
3.85	0\\
3.86	0\\
3.87	0\\
3.88	0\\
3.89	0\\
3.9	0\\
3.91	0\\
3.92	0\\
3.93	0\\
3.94	0\\
3.95	0\\
3.96	0\\
3.97	0\\
3.98	0\\
3.99	0\\
4	0\\
4.01	0\\
4.02	0\\
4.03	0\\
4.04	0\\
4.05	0\\
4.06	0\\
4.07	0\\
4.08	0\\
4.09	0\\
4.1	0\\
4.11	0\\
4.12	0\\
4.13	0\\
4.14	0\\
4.15	0\\
4.16	0\\
4.17	0\\
4.18	0\\
4.19	0\\
4.2	0\\
4.21	0\\
4.22	0\\
4.23	0\\
4.24	0\\
4.25	0\\
4.26	0\\
4.27	0\\
4.28	0\\
4.29	0\\
4.3	0\\
4.31	0\\
4.32	0\\
4.33	0\\
4.34	0\\
4.35	0\\
4.36	0\\
4.37	0\\
4.38	0\\
4.39	0\\
4.4	0\\
4.41	0\\
4.42	0\\
4.43	0\\
4.44	0\\
4.45	0\\
4.46	0\\
4.47	0\\
4.48	0\\
4.49	0\\
4.5	0\\
4.51	0\\
4.52	0\\
4.53	0\\
4.54	0\\
4.55	0\\
4.56	0\\
4.57	0\\
4.58	0\\
4.59	0\\
4.6	0\\
4.61	0\\
4.62	0\\
4.63	0\\
4.64	0\\
4.65	0\\
4.66	0\\
4.67	0\\
4.68	0\\
4.69	0\\
4.7	0\\
4.71	0\\
4.72	0\\
4.73	0\\
4.74	0\\
4.75	0\\
4.76	0\\
4.77	0\\
4.78	0\\
4.79	0\\
4.8	0\\
4.81	0\\
4.82	0\\
4.83	0\\
4.84	0\\
4.85	0\\
4.86	0\\
4.87	0\\
4.88	0\\
4.89	0\\
4.9	0\\
4.91	0\\
4.92	0\\
4.93	0\\
4.94	0\\
4.95	0\\
4.96	0\\
4.97	0\\
4.98	0\\
4.99	0\\
5	0\\
5.01	0\\
5.02	0\\
5.03	0\\
5.04	0\\
5.05	0\\
5.06	0\\
5.07	0\\
5.08	0\\
5.09	0\\
5.1	0\\
5.11	0\\
5.12	0\\
5.13	0\\
5.14	0\\
5.15	0\\
5.16	0\\
5.17	0\\
5.18	0\\
5.19	0\\
5.2	0\\
5.21	0\\
5.22	0\\
5.23	0\\
5.24	0\\
5.25	0\\
5.26	0\\
5.27	0\\
5.28	0\\
5.29	0\\
5.3	0\\
5.31	0\\
5.32	0\\
5.33	0\\
5.34	0\\
5.35	0\\
5.36	0\\
5.37	0\\
5.38	0\\
5.39	0\\
5.4	0\\
5.41	0\\
5.42	0\\
5.43	0\\
5.44	0\\
5.45	0\\
5.46	0\\
5.47	0\\
5.48	0\\
5.49	0\\
5.5	0\\
5.51	0\\
5.52	0\\
5.53	0\\
5.54	0\\
5.55	0\\
5.56	0\\
5.57	0\\
5.58	0\\
5.59	0\\
5.6	0\\
5.61	0\\
5.62	0\\
5.63	0\\
5.64	0\\
5.65	0\\
5.66	0\\
5.67	0\\
5.68	0\\
5.69	0\\
5.7	0\\
5.71	0\\
5.72	0\\
5.73	0\\
5.74	0\\
5.75	0\\
5.76	0\\
5.77	0\\
5.78	0\\
5.79	0\\
5.8	0\\
5.81	0\\
5.82	0\\
5.83	0\\
5.84	0\\
5.85	0\\
5.86	0\\
5.87	0\\
5.88	0\\
5.89	0\\
5.9	0\\
5.91	0\\
5.92	0\\
5.93	0\\
5.94	0\\
5.95	0\\
5.96	0\\
5.97	0\\
5.98	0\\
5.99	0\\
6	0\\
6.01	0\\
6.02	0\\
6.03	0\\
6.04	0\\
6.05	0\\
6.06	0\\
6.07	0\\
6.08	0\\
6.09	0\\
6.1	0\\
6.11	0\\
6.12	0\\
6.13	0\\
6.14	0\\
6.15	0\\
6.16	0\\
6.17	0\\
6.18	0\\
6.19	0\\
6.2	0\\
6.21	0\\
6.22	0\\
6.23	0\\
6.24	0\\
6.25	0\\
6.26	0\\
6.27	0\\
6.28	0\\
6.29	0\\
6.3	0\\
6.31	0\\
6.32	0\\
6.33	0\\
6.34	0\\
6.35	0\\
6.36	0\\
6.37	0\\
6.38	0\\
6.39	0\\
6.4	0\\
6.41	0\\
6.42	0\\
6.43	0\\
6.44	0\\
6.45	0\\
6.46	0\\
6.47	0\\
6.48	0\\
6.49	0\\
6.5	0\\
6.51	0\\
6.52	0\\
6.53	0\\
6.54	0\\
6.55	0\\
6.56	0\\
6.57	0\\
6.58	0\\
6.59	0\\
6.6	0\\
6.61	0\\
6.62	0\\
6.63	0\\
6.64	0\\
6.65	0\\
6.66	0\\
6.67	0\\
6.68	0\\
6.69	0\\
6.7	0\\
6.71	0\\
6.72	0\\
6.73	0\\
6.74	0\\
6.75	0\\
6.76	0\\
6.77	0\\
6.78	0\\
6.79	0\\
6.8	0\\
6.81	0\\
6.82	0\\
6.83	0\\
6.84	0\\
6.85	0\\
6.86	0\\
6.87	0\\
6.88	0\\
6.89	0\\
6.9	0\\
6.91	0\\
6.92	0\\
6.93	0\\
6.94	0\\
6.95	0\\
6.96	0\\
6.97	0\\
6.98	0\\
6.99	0\\
7	0\\
7.01	0\\
7.02	0\\
7.03	0\\
7.04	0\\
7.05	0\\
7.06	0\\
7.07	0\\
7.08	0\\
7.09	0\\
7.1	0\\
7.11	0\\
7.12	0\\
7.13	0\\
7.14	0\\
7.15	0\\
7.16	0\\
7.17	0\\
7.18	0\\
7.19	0\\
7.2	0\\
7.21	0\\
7.22	0\\
7.23	0\\
7.24	0\\
7.25	0\\
7.26	0\\
7.27	0\\
7.28	0\\
7.29	0\\
7.3	0\\
7.31	0\\
7.32	0\\
7.33	0\\
7.34	0\\
7.35	0\\
7.36	0\\
7.37	0\\
7.38	0\\
7.39	0\\
7.4	0\\
7.41	0\\
7.42	0\\
7.43	0\\
7.44	0\\
7.45	0\\
7.46	0\\
7.47	0\\
7.48	0\\
7.49	0\\
7.5	0\\
7.51	0\\
7.52	0\\
7.53	0\\
7.54	0\\
7.55	0\\
7.56	0\\
7.57	0\\
7.58	0\\
7.59	0\\
7.6	0\\
7.61	0\\
7.62	0\\
7.63	0\\
7.64	0\\
7.65	0\\
7.66	0\\
7.67	0\\
7.68	0\\
7.69	0\\
7.7	0\\
7.71	0\\
7.72	0\\
7.73	0\\
7.74	0\\
7.75	0\\
7.76	0\\
7.77	0\\
7.78	0\\
7.79	0\\
7.8	0\\
7.81	0\\
7.82	0\\
7.83	0\\
7.84	0\\
7.85	0\\
7.86	0\\
7.87	0\\
7.88	0\\
7.89	0\\
7.9	0\\
7.91	0\\
7.92	0\\
7.93	0\\
7.94	0\\
7.95	0\\
7.96	0\\
7.97	0\\
7.98	0\\
7.99	0\\
8	0\\
8.01	0\\
8.02	0\\
8.03	0\\
8.04	0\\
8.05	0\\
8.06	0\\
8.07	0\\
8.08	0\\
8.09	0\\
8.1	0\\
8.11	0\\
8.12	0\\
8.13	0\\
8.14	0\\
8.15	0\\
8.16	0\\
8.17	0\\
8.18	0\\
8.19	0\\
8.2	0\\
8.21	0\\
8.22	0\\
8.23	0\\
8.24	0\\
8.25	0\\
8.26	0\\
8.27	0\\
8.28	0\\
8.29	0\\
8.3	0\\
8.31	0\\
8.32	0\\
8.33	0\\
8.34	0\\
8.35	0\\
8.36	0\\
8.37	0\\
8.38	0\\
8.39	0\\
8.4	0\\
8.41	0\\
8.42	0\\
8.43	0\\
8.44	0\\
8.45	0\\
8.46	0\\
8.47	0\\
8.48	0\\
8.49	0\\
8.5	0\\
8.51	0\\
8.52	0\\
8.53	0\\
8.54	0\\
8.55	0\\
8.56	0\\
8.57	0\\
8.58	0\\
8.59	0\\
8.6	0\\
8.61	0\\
8.62	0\\
8.63	0\\
8.64	0\\
8.65	0\\
8.66	0\\
8.67	0\\
8.68	0\\
8.69	0\\
8.7	0\\
8.71	0\\
8.72	0\\
8.73	0\\
8.74	0\\
8.75	0\\
8.76	0\\
8.77	0\\
8.78	0\\
8.79	0\\
8.8	0\\
8.81	0\\
8.82	0\\
8.83	0\\
8.84	0\\
8.85	0\\
8.86	0\\
8.87	0\\
8.88	0\\
8.89	0\\
8.9	0\\
8.91	0\\
8.92	0\\
8.93	0\\
8.94	0\\
8.95	0\\
8.96	0\\
8.97	0\\
8.98	0\\
8.99	0\\
9	0\\
9.01	0\\
9.02	0\\
9.03	0\\
9.04	0\\
9.05	0\\
9.06	0\\
9.07	0\\
9.08	0\\
9.09	0\\
9.1	0\\
9.11	0\\
9.12	0\\
9.13	0\\
9.14	0\\
9.15	0\\
9.16	0\\
9.17	0\\
9.18	0\\
9.19	0\\
9.2	0\\
9.21	0\\
9.22	0\\
9.23	0\\
9.24	0\\
9.25	0\\
9.26	0\\
9.27	0\\
9.28	0\\
9.29	0\\
9.3	0\\
9.31	0\\
9.32	0\\
9.33	0\\
9.34	0\\
9.35	0\\
9.36	0\\
9.37	0\\
9.38	0\\
9.39	0\\
9.4	0\\
9.41	0\\
9.42	0\\
9.43	0\\
9.44	0\\
9.45	0\\
9.46	0\\
9.47	0\\
9.48	0\\
9.49	0\\
9.5	0\\
9.51	0\\
9.52	0\\
9.53	0\\
9.54	0\\
9.55	0\\
9.56	0\\
9.57	0\\
9.58	0\\
9.59	0\\
9.6	0\\
9.61	0\\
9.62	0\\
9.63	0\\
9.64	0\\
9.65	0\\
9.66	0\\
9.67	0\\
9.68	0\\
9.69	0\\
9.7	0\\
9.71	0\\
9.72	0\\
9.73	0\\
9.74	0\\
9.75	0\\
9.76	0\\
9.77	0\\
9.78	0\\
9.79	0\\
9.8	0\\
9.81	0\\
9.82	0\\
9.83	0\\
9.84	0\\
9.85	0\\
9.86	0\\
9.87	0\\
9.88	0\\
9.89	0\\
9.9	0\\
9.91	0\\
9.92	0\\
9.93	0\\
9.94	0\\
9.95	0\\
9.96	0\\
9.97	0\\
9.98	0\\
9.99	0\\
10	0\\
10.01	0\\
10.02	0\\
10.03	0\\
10.04	0\\
10.05	0\\
10.06	0\\
10.07	0\\
10.08	0\\
10.09	0\\
10.1	0\\
10.11	0\\
10.12	0\\
10.13	0\\
10.14	0\\
10.15	0\\
10.16	0\\
10.17	0\\
10.18	0\\
10.19	0\\
10.2	0\\
10.21	0\\
10.22	0\\
10.23	0\\
10.24	0\\
10.25	0\\
10.26	0\\
10.27	0\\
10.28	0\\
10.29	0\\
10.3	0\\
10.31	0\\
10.32	0\\
10.33	0\\
10.34	0\\
10.35	0\\
10.36	0\\
10.37	0\\
10.38	0\\
10.39	0\\
10.4	0\\
10.41	0\\
10.42	0\\
10.43	0\\
10.44	0\\
10.45	0\\
10.46	0\\
10.47	0\\
10.48	0\\
10.49	0\\
10.5	0\\
10.51	0\\
10.52	0\\
10.53	0\\
10.54	0\\
10.55	0\\
10.56	0\\
10.57	0\\
10.58	0\\
10.59	0\\
10.6	0\\
10.61	0\\
10.62	0\\
10.63	0\\
10.64	0\\
10.65	0\\
10.66	0\\
10.67	0\\
10.68	0\\
10.69	0\\
10.7	0\\
10.71	0\\
10.72	0\\
10.73	0\\
10.74	0\\
10.75	0\\
10.76	0\\
10.77	0\\
10.78	0\\
10.79	0\\
10.8	0\\
10.81	0\\
10.82	0\\
10.83	0\\
10.84	0\\
10.85	0\\
10.86	0\\
10.87	0\\
10.88	0\\
10.89	0\\
10.9	0\\
10.91	0\\
10.92	0\\
10.93	0\\
10.94	0\\
10.95	0\\
10.96	0\\
10.97	0\\
10.98	0\\
10.99	0\\
11	0\\
11.01	0\\
11.02	0\\
11.03	0\\
11.04	0\\
11.05	0\\
11.06	0\\
11.07	0\\
11.08	0\\
11.09	0\\
11.1	0\\
11.11	0\\
11.12	0\\
11.13	0\\
11.14	0\\
11.15	0\\
11.16	0\\
11.17	0\\
11.18	0\\
11.19	0\\
11.2	0\\
11.21	0\\
11.22	0\\
11.23	0\\
11.24	0\\
11.25	0\\
11.26	0\\
11.27	0\\
11.28	0\\
11.29	0\\
11.3	0\\
11.31	0\\
11.32	0\\
11.33	0\\
11.34	0\\
11.35	0\\
11.36	0\\
11.37	0\\
11.38	0\\
11.39	0\\
11.4	0\\
11.41	0\\
11.42	0\\
11.43	0\\
11.44	0\\
11.45	0\\
11.46	0\\
11.47	0\\
11.48	0\\
11.49	0\\
11.5	0\\
11.51	0\\
11.52	0\\
11.53	0\\
11.54	0\\
11.55	0\\
11.56	0\\
11.57	0\\
11.58	0\\
11.59	0\\
11.6	0\\
11.61	0\\
11.62	0\\
11.63	0\\
11.64	0\\
11.65	0\\
11.66	0\\
11.67	0\\
11.68	0\\
11.69	0\\
11.7	0\\
11.71	0\\
11.72	0\\
11.73	0\\
11.74	0\\
11.75	0\\
11.76	0\\
11.77	0\\
11.78	0\\
11.79	0\\
11.8	0\\
11.81	0\\
11.82	0\\
11.83	0\\
11.84	0\\
11.85	0\\
11.86	0\\
11.87	0\\
11.88	0\\
11.89	0\\
11.9	0\\
11.91	0\\
11.92	0\\
11.93	0\\
11.94	0\\
11.95	0\\
11.96	0\\
11.97	0\\
11.98	0\\
11.99	0\\
12	0\\
12.01	0\\
12.02	0\\
12.03	0\\
12.04	0\\
12.05	0\\
12.06	0\\
12.07	0\\
12.08	0\\
12.09	0\\
12.1	0\\
12.11	0\\
12.12	0\\
12.13	0\\
12.14	0\\
12.15	0\\
12.16	0\\
12.17	0\\
12.18	0\\
12.19	0\\
12.2	0\\
12.21	0\\
12.22	0\\
12.23	0\\
12.24	0\\
12.25	0\\
12.26	0\\
12.27	0\\
12.28	0\\
12.29	0\\
12.3	0\\
12.31	0\\
12.32	0\\
12.33	0\\
12.34	0\\
12.35	0\\
12.36	0\\
12.37	0\\
12.38	0\\
12.39	0\\
12.4	0\\
12.41	0\\
12.42	0\\
12.43	0\\
12.44	0\\
12.45	0\\
12.46	0\\
12.47	0\\
12.48	0\\
12.49	0\\
12.5	0\\
12.51	0\\
12.52	0\\
12.53	0\\
12.54	0\\
12.55	0\\
12.56	0\\
12.57	0\\
12.58	0\\
12.59	0\\
12.6	0\\
12.61	0\\
12.62	0\\
12.63	0\\
12.64	0\\
12.65	0\\
12.66	0\\
12.67	0\\
12.68	0\\
12.69	0\\
12.7	0\\
12.71	0\\
12.72	0\\
12.73	0\\
12.74	0\\
12.75	0\\
12.76	0\\
12.77	0\\
12.78	0\\
12.79	0\\
12.8	0\\
12.81	0\\
12.82	0\\
12.83	0\\
12.84	0\\
12.85	0\\
12.86	0\\
12.87	0\\
12.88	0\\
12.89	0\\
12.9	0\\
12.91	0\\
12.92	0\\
12.93	0\\
12.94	0\\
12.95	0\\
12.96	0\\
12.97	0\\
12.98	0\\
12.99	0\\
13	0\\
13.01	0\\
13.02	0\\
13.03	0\\
13.04	0\\
13.05	0\\
13.06	0\\
13.07	0\\
13.08	0\\
13.09	0\\
13.1	0\\
13.11	0\\
13.12	0\\
13.13	0\\
13.14	0\\
13.15	0\\
13.16	0\\
13.17	0\\
13.18	0\\
13.19	0\\
13.2	0\\
13.21	0\\
13.22	0\\
13.23	0\\
13.24	0\\
13.25	0\\
13.26	0\\
13.27	0\\
13.28	0\\
13.29	0\\
13.3	0\\
13.31	0\\
13.32	0\\
13.33	0\\
13.34	0\\
13.35	0\\
13.36	0\\
13.37	0\\
13.38	0\\
13.39	0\\
13.4	0\\
13.41	0\\
13.42	0\\
13.43	0\\
13.44	0\\
13.45	0\\
13.46	0\\
13.47	0\\
13.48	0\\
13.49	0\\
13.5	0\\
13.51	0\\
13.52	0\\
13.53	0\\
13.54	0\\
13.55	0\\
13.56	0\\
13.57	0\\
13.58	0\\
13.59	0\\
13.6	0\\
13.61	0\\
13.62	0\\
13.63	0\\
13.64	0\\
13.65	0\\
13.66	0\\
13.67	0\\
13.68	0\\
13.69	0\\
13.7	0\\
13.71	0\\
13.72	0\\
13.73	0\\
13.74	0\\
13.75	0\\
13.76	0\\
13.77	0\\
13.78	0\\
13.79	0\\
13.8	0\\
13.81	0\\
13.82	0\\
13.83	0\\
13.84	0\\
13.85	0\\
13.86	0\\
13.87	0\\
13.88	0\\
13.89	0\\
13.9	0\\
13.91	0\\
13.92	0\\
13.93	0\\
13.94	0\\
13.95	0\\
13.96	0\\
13.97	0\\
13.98	0\\
13.99	0\\
14	0\\
14.01	0\\
14.02	0\\
14.03	0\\
14.04	0\\
14.05	0\\
14.06	0\\
14.07	0\\
14.08	0\\
14.09	0\\
14.1	0\\
14.11	0\\
14.12	0\\
14.13	0\\
14.14	0\\
14.15	0\\
14.16	0\\
14.17	0\\
14.18	0\\
14.19	0\\
14.2	0\\
14.21	0\\
14.22	0\\
14.23	0\\
14.24	0\\
14.25	0\\
14.26	0\\
14.27	0\\
14.28	0\\
14.29	0\\
14.3	0\\
14.31	0\\
14.32	0\\
14.33	0\\
14.34	0\\
14.35	0\\
14.36	0\\
14.37	0\\
14.38	0\\
14.39	0\\
14.4	0\\
14.41	0\\
14.42	0\\
14.43	0\\
14.44	0\\
14.45	0\\
14.46	0\\
14.47	0\\
14.48	0\\
14.49	0\\
14.5	0\\
14.51	0\\
14.52	0\\
14.53	0\\
14.54	0\\
14.55	0\\
14.56	0\\
14.57	0\\
14.58	0\\
14.59	0\\
14.6	0\\
14.61	0\\
14.62	0\\
14.63	0\\
14.64	0\\
14.65	0\\
14.66	0\\
14.67	0\\
14.68	0\\
14.69	0\\
14.7	0\\
14.71	0\\
14.72	0\\
14.73	0\\
14.74	0\\
14.75	0\\
14.76	0\\
14.77	0\\
14.78	0\\
14.79	0\\
14.8	0\\
14.81	0\\
14.82	0\\
14.83	0\\
14.84	0\\
14.85	0\\
14.86	0\\
14.87	0\\
14.88	0\\
14.89	0\\
14.9	0\\
14.91	0\\
14.92	0\\
14.93	0\\
14.94	0\\
14.95	0\\
14.96	0\\
14.97	0\\
14.98	0\\
14.99	0\\
15	0\\
15.01	0\\
15.02	0\\
15.03	0\\
15.04	0\\
15.05	0\\
15.06	0\\
15.07	0\\
15.08	0\\
15.09	0\\
15.1	0\\
15.11	0\\
15.12	0\\
15.13	0\\
15.14	0\\
15.15	0\\
15.16	0\\
15.17	0\\
15.18	0\\
15.19	0\\
15.2	0\\
15.21	0\\
15.22	0\\
15.23	0\\
15.24	0\\
15.25	0\\
15.26	0\\
15.27	0\\
15.28	0\\
15.29	0\\
15.3	0\\
15.31	0\\
15.32	0\\
15.33	0\\
15.34	0\\
15.35	0\\
15.36	0\\
15.37	0\\
15.38	0\\
15.39	0\\
15.4	0\\
15.41	0\\
15.42	0\\
15.43	0\\
15.44	0\\
15.45	0\\
15.46	0\\
15.47	0\\
15.48	0\\
15.49	0\\
15.5	0\\
15.51	0\\
15.52	0\\
15.53	0\\
15.54	0\\
15.55	0\\
15.56	0\\
15.57	0\\
15.58	0\\
15.59	0\\
15.6	0\\
15.61	0\\
15.62	0\\
15.63	0\\
15.64	0\\
15.65	0\\
15.66	0\\
15.67	0\\
15.68	0\\
15.69	0\\
15.7	0\\
15.71	0\\
15.72	0\\
15.73	0\\
15.74	0\\
15.75	0\\
15.76	0\\
15.77	0\\
15.78	0\\
15.79	0\\
15.8	0\\
15.81	0\\
15.82	0\\
15.83	0\\
15.84	0\\
15.85	0\\
15.86	0\\
15.87	0\\
15.88	0\\
15.89	0\\
15.9	0\\
15.91	0\\
15.92	0\\
15.93	0\\
15.94	0\\
15.95	0\\
15.96	0\\
15.97	0\\
15.98	0\\
15.99	0\\
16	0\\
16.01	0\\
16.02	0\\
16.03	0\\
16.04	0\\
16.05	0\\
16.06	0\\
16.07	0\\
16.08	0\\
16.09	0\\
16.1	0\\
16.11	0\\
16.12	0\\
16.13	0\\
16.14	0\\
16.15	0\\
16.16	0\\
16.17	0\\
16.18	0\\
16.19	0\\
16.2	0\\
16.21	0\\
16.22	0\\
16.23	0\\
16.24	0\\
16.25	0\\
16.26	0\\
16.27	0\\
16.28	0\\
16.29	0\\
16.3	0\\
16.31	0\\
16.32	0\\
16.33	0\\
16.34	0\\
16.35	0\\
16.36	0\\
16.37	0\\
16.38	0\\
16.39	0\\
16.4	0\\
16.41	0\\
16.42	0\\
16.43	0\\
16.44	0\\
16.45	0\\
16.46	0\\
16.47	0\\
16.48	0\\
16.49	0\\
16.5	0\\
16.51	0\\
16.52	0\\
16.53	0\\
16.54	0\\
16.55	0\\
16.56	0\\
16.57	0\\
16.58	0\\
16.59	0\\
16.6	0\\
16.61	0\\
16.62	0\\
16.63	0\\
16.64	0\\
16.65	0\\
16.66	0\\
16.67	0\\
16.68	0\\
16.69	0\\
16.7	0\\
16.71	0\\
16.72	0\\
16.73	0\\
16.74	0\\
16.75	0\\
16.76	0\\
16.77	0\\
16.78	0\\
16.79	0\\
16.8	0\\
16.81	0\\
16.82	0\\
16.83	0\\
16.84	0\\
16.85	0\\
16.86	0\\
16.87	0\\
16.88	0\\
16.89	0\\
16.9	0\\
16.91	0\\
16.92	0\\
16.93	0\\
16.94	0\\
16.95	0\\
16.96	0\\
16.97	0\\
16.98	0\\
16.99	0\\
17	0\\
17.01	0\\
17.02	0\\
17.03	0\\
17.04	0\\
17.05	0\\
17.06	0\\
17.07	0\\
17.08	0\\
17.09	0\\
17.1	0\\
17.11	0\\
17.12	0\\
17.13	0\\
17.14	0\\
17.15	0\\
17.16	0\\
17.17	0\\
17.18	0\\
17.19	0\\
17.2	0\\
17.21	0\\
17.22	0\\
17.23	0\\
17.24	0\\
17.25	0\\
17.26	0\\
17.27	0\\
17.28	0\\
17.29	0\\
17.3	0\\
17.31	0\\
17.32	0\\
17.33	0\\
17.34	0\\
17.35	0\\
17.36	0\\
17.37	0\\
17.38	0\\
17.39	0\\
17.4	0\\
17.41	0\\
17.42	0\\
17.43	0\\
17.44	0\\
17.45	0\\
17.46	0\\
17.47	0\\
17.48	0\\
17.49	0\\
17.5	0\\
17.51	0\\
17.52	0\\
17.53	0\\
17.54	0\\
17.55	0\\
17.56	0\\
17.57	0\\
17.58	0\\
17.59	0\\
17.6	0\\
17.61	0\\
17.62	0\\
17.63	0\\
17.64	0\\
17.65	0\\
17.66	0\\
17.67	0\\
17.68	0\\
17.69	0\\
17.7	0\\
17.71	0\\
17.72	0\\
17.73	0\\
17.74	0\\
17.75	0\\
17.76	0\\
17.77	0\\
17.78	0\\
17.79	0\\
17.8	0\\
17.81	0\\
17.82	0\\
17.83	0\\
17.84	0\\
17.85	0\\
17.86	0\\
17.87	0\\
17.88	0\\
17.89	0\\
17.9	0\\
17.91	0\\
17.92	0\\
17.93	0\\
17.94	0\\
17.95	0\\
17.96	0\\
17.97	0\\
17.98	0\\
17.99	0\\
18	0\\
18.01	0\\
18.02	0\\
18.03	0\\
18.04	0\\
18.05	0\\
18.06	0\\
18.07	0\\
18.08	0\\
18.09	0\\
18.1	0\\
18.11	0\\
18.12	0\\
18.13	0\\
18.14	0\\
18.15	0\\
18.16	0\\
18.17	0\\
18.18	0\\
18.19	0\\
18.2	0\\
18.21	0\\
18.22	0\\
18.23	0\\
18.24	0\\
18.25	0\\
18.26	0\\
18.27	0\\
18.28	0\\
18.29	0\\
18.3	0\\
18.31	0\\
18.32	0\\
18.33	0\\
18.34	0\\
18.35	0\\
18.36	0\\
18.37	0\\
18.38	0\\
18.39	0\\
18.4	0\\
18.41	0\\
18.42	0\\
18.43	0\\
18.44	0\\
18.45	0\\
18.46	0\\
18.47	0\\
18.48	0\\
18.49	0\\
18.5	0\\
18.51	0\\
18.52	0\\
18.53	0\\
18.54	0\\
18.55	0\\
18.56	0\\
18.57	0\\
18.58	0\\
18.59	0\\
18.6	0\\
18.61	0\\
18.62	0\\
18.63	0\\
18.64	0\\
18.65	0\\
18.66	0\\
18.67	0\\
18.68	0\\
18.69	0\\
18.7	0\\
18.71	0\\
18.72	0\\
18.73	0\\
18.74	0\\
18.75	0\\
18.76	0\\
18.77	0\\
18.78	0\\
18.79	0\\
18.8	0\\
18.81	0\\
18.82	0\\
18.83	0\\
18.84	0\\
18.85	0\\
18.86	0\\
18.87	0\\
18.88	0\\
18.89	0\\
18.9	0\\
18.91	0\\
18.92	0\\
18.93	0\\
18.94	0\\
18.95	0\\
18.96	0\\
18.97	0\\
18.98	0\\
18.99	0\\
19	0\\
19.01	0\\
19.02	0\\
19.03	0\\
19.04	0\\
19.05	0\\
19.06	0\\
19.07	0\\
19.08	0\\
19.09	0\\
19.1	0\\
19.11	0\\
19.12	0\\
19.13	0\\
19.14	0\\
19.15	0\\
19.16	0\\
19.17	0\\
19.18	0\\
19.19	0\\
19.2	0\\
19.21	0\\
19.22	0\\
19.23	0\\
19.24	0\\
19.25	0\\
19.26	0\\
19.27	0\\
19.28	0\\
19.29	0\\
19.3	0\\
19.31	0\\
19.32	0\\
19.33	0\\
19.34	0\\
19.35	0\\
19.36	0\\
19.37	0\\
19.38	0\\
19.39	0\\
19.4	0\\
19.41	0\\
19.42	0\\
19.43	0\\
19.44	0\\
19.45	0\\
19.46	0\\
19.47	0\\
19.48	0\\
19.49	0\\
19.5	0\\
19.51	0\\
19.52	0\\
19.53	0\\
19.54	0\\
19.55	0\\
19.56	0\\
19.57	0\\
19.58	0\\
19.59	0\\
19.6	0\\
19.61	0\\
19.62	0\\
19.63	0\\
19.64	0\\
19.65	0\\
19.66	0\\
19.67	0\\
19.68	0\\
19.69	0\\
19.7	0\\
19.71	0\\
19.72	0\\
19.73	0\\
19.74	0\\
19.75	0\\
19.76	0\\
19.77	0\\
19.78	0\\
19.79	0\\
19.8	0\\
19.81	0\\
19.82	0\\
19.83	0\\
19.84	0\\
19.85	0\\
19.86	0\\
19.87	0\\
19.88	0\\
19.89	0\\
19.9	0\\
19.91	0\\
19.92	0\\
19.93	0\\
19.94	0\\
19.95	0\\
19.96	0\\
19.97	0\\
19.98	0\\
19.99	0\\
20	0\\
20.01	0\\
20.02	0\\
20.03	0\\
20.04	0\\
20.05	0\\
20.06	0\\
20.07	0\\
20.08	0\\
20.09	0\\
20.1	0\\
20.11	0\\
20.12	0\\
20.13	0\\
20.14	0\\
20.15	0\\
20.16	0\\
20.17	0\\
20.18	0\\
20.19	0\\
20.2	0\\
20.21	0\\
20.22	0\\
20.23	0\\
20.24	0\\
20.25	0\\
20.26	0\\
20.27	0\\
20.28	0\\
20.29	0\\
20.3	0\\
20.31	0\\
20.32	0\\
20.33	0\\
20.34	0\\
20.35	0\\
20.36	0\\
20.37	0\\
20.38	0\\
20.39	0\\
20.4	0\\
20.41	0\\
20.42	0\\
20.43	0\\
20.44	0\\
20.45	0\\
20.46	0\\
20.47	0\\
20.48	0\\
20.49	0\\
20.5	0\\
20.51	0\\
20.52	0\\
20.53	0\\
20.54	0\\
20.55	0\\
20.56	0\\
20.57	0\\
20.58	0\\
20.59	0\\
20.6	0\\
20.61	0\\
20.62	0\\
20.63	0\\
20.64	0\\
20.65	0\\
20.66	0\\
20.67	0\\
20.68	0\\
20.69	0\\
20.7	0\\
20.71	0\\
20.72	0\\
20.73	0\\
20.74	0\\
20.75	0\\
20.76	0\\
20.77	0\\
20.78	0\\
20.79	0\\
20.8	0\\
20.81	0\\
20.82	0\\
20.83	0\\
20.84	0\\
20.85	0\\
20.86	0\\
20.87	0\\
20.88	0\\
20.89	0\\
20.9	0\\
20.91	0\\
20.92	0\\
20.93	0\\
20.94	0\\
20.95	0\\
20.96	0\\
20.97	0\\
20.98	0\\
20.99	0\\
21	0\\
21.01	0\\
21.02	0\\
21.03	0\\
21.04	0\\
21.05	0\\
21.06	0\\
21.07	0\\
21.08	0\\
21.09	0\\
21.1	0\\
21.11	0\\
21.12	0\\
21.13	0\\
21.14	0\\
21.15	0\\
21.16	0\\
21.17	0\\
21.18	0\\
21.19	0\\
21.2	0\\
21.21	0\\
21.22	0\\
21.23	0\\
21.24	0\\
21.25	0\\
21.26	0\\
21.27	0\\
21.28	0\\
21.29	0\\
21.3	0\\
21.31	0\\
21.32	0\\
21.33	0\\
21.34	0\\
21.35	0\\
21.36	0\\
21.37	0\\
21.38	0\\
21.39	0\\
21.4	0\\
21.41	0\\
21.42	0\\
21.43	0\\
21.44	0\\
21.45	0\\
21.46	0\\
21.47	0\\
21.48	0\\
21.49	0\\
21.5	0\\
21.51	0\\
21.52	0\\
21.53	0\\
21.54	0\\
21.55	0\\
21.56	0\\
21.57	0\\
21.58	0\\
21.59	0\\
21.6	0\\
21.61	0\\
21.62	0\\
21.63	0\\
21.64	0\\
21.65	0\\
21.66	0\\
21.67	0\\
21.68	0\\
21.69	0\\
21.7	0\\
21.71	0\\
21.72	0\\
21.73	0\\
21.74	0\\
21.75	0\\
21.76	0\\
21.77	0\\
21.78	0\\
21.79	0\\
21.8	0\\
21.81	0\\
21.82	0\\
21.83	0\\
21.84	0\\
21.85	0\\
21.86	0\\
21.87	0\\
21.88	0\\
21.89	0\\
21.9	0\\
21.91	0\\
21.92	0\\
21.93	0\\
21.94	0\\
21.95	0\\
21.96	0\\
21.97	0\\
21.98	0\\
21.99	0\\
22	0\\
22.01	0\\
22.02	0\\
22.03	0\\
22.04	0\\
22.05	0\\
22.06	0\\
22.07	0\\
22.08	0\\
22.09	0\\
22.1	0\\
22.11	0\\
22.12	0\\
22.13	0\\
22.14	0\\
22.15	0\\
22.16	0\\
22.17	0\\
22.18	0\\
22.19	0\\
22.2	0\\
22.21	0\\
22.22	0\\
22.23	0\\
22.24	0\\
22.25	0\\
22.26	0\\
22.27	0\\
22.28	0\\
22.29	0\\
22.3	0\\
22.31	0\\
22.32	0\\
22.33	0\\
22.34	0\\
22.35	0\\
22.36	0\\
22.37	0\\
22.38	0\\
22.39	0\\
22.4	0\\
22.41	0\\
22.42	0\\
22.43	0\\
22.44	0\\
22.45	0\\
22.46	0\\
22.47	0\\
22.48	0\\
22.49	0\\
22.5	0\\
22.51	0\\
22.52	0\\
22.53	0\\
22.54	0\\
22.55	0\\
22.56	0\\
22.57	0\\
22.58	0\\
22.59	0\\
22.6	0\\
22.61	0\\
22.62	0\\
22.63	0\\
22.64	0\\
22.65	0\\
22.66	0\\
22.67	0\\
22.68	0\\
22.69	0\\
22.7	0\\
22.71	0\\
22.72	0\\
22.73	0\\
22.74	0\\
22.75	0\\
22.76	0\\
22.77	0\\
22.78	0\\
22.79	0\\
22.8	0\\
22.81	0\\
22.82	0\\
22.83	0\\
22.84	0\\
22.85	0\\
22.86	0\\
22.87	0\\
22.88	0\\
22.89	0\\
22.9	0\\
22.91	0\\
22.92	0\\
22.93	0\\
22.94	0\\
22.95	0\\
22.96	0\\
22.97	0\\
22.98	0\\
22.99	0\\
23	0\\
23.01	0\\
23.02	0\\
23.03	0\\
23.04	0\\
23.05	0\\
23.06	0\\
23.07	0\\
23.08	0\\
23.09	0\\
23.1	0\\
23.11	0\\
23.12	0\\
23.13	0\\
23.14	0\\
23.15	0\\
23.16	0\\
23.17	0\\
23.18	0\\
23.19	0\\
23.2	0\\
23.21	0\\
23.22	0\\
23.23	0\\
23.24	0\\
23.25	0\\
23.26	0\\
23.27	0\\
23.28	0\\
23.29	0\\
23.3	0\\
23.31	0\\
23.32	0\\
23.33	0\\
23.34	0\\
23.35	0\\
23.36	0\\
23.37	0\\
23.38	0\\
23.39	0\\
23.4	0\\
23.41	0\\
23.42	0\\
23.43	0\\
23.44	0\\
23.45	0\\
23.46	0\\
23.47	0\\
23.48	0\\
23.49	0\\
23.5	0\\
23.51	0\\
23.52	0\\
23.53	0\\
23.54	0\\
23.55	0\\
23.56	0\\
23.57	0\\
23.58	0\\
23.59	0\\
23.6	0\\
23.61	0\\
23.62	0\\
23.63	0\\
23.64	0\\
23.65	0\\
23.66	0\\
23.67	0\\
23.68	0\\
23.69	0\\
23.7	0\\
23.71	0\\
23.72	0\\
23.73	0\\
23.74	0\\
23.75	0\\
23.76	0\\
23.77	0\\
23.78	0\\
23.79	0\\
23.8	0\\
23.81	0\\
23.82	0\\
23.83	0\\
23.84	0\\
23.85	0\\
23.86	0\\
23.87	0\\
23.88	0\\
23.89	0\\
23.9	0\\
23.91	0\\
23.92	0\\
23.93	0\\
23.94	0\\
23.95	0\\
23.96	0\\
23.97	0\\
23.98	0\\
23.99	0\\
24	0\\
24.01	0\\
24.02	0\\
24.03	0\\
24.04	0\\
24.05	0\\
24.06	0\\
24.07	0\\
24.08	0\\
24.09	0\\
24.1	0\\
24.11	0\\
24.12	0\\
24.13	0\\
24.14	0\\
24.15	0\\
24.16	0\\
24.17	0\\
24.18	0\\
24.19	0\\
24.2	0\\
24.21	0\\
24.22	0\\
24.23	0\\
24.24	0\\
24.25	0\\
24.26	0\\
24.27	0\\
24.28	0\\
24.29	0\\
24.3	0\\
24.31	0\\
24.32	0\\
24.33	0\\
24.34	0\\
24.35	0\\
24.36	0\\
24.37	0\\
24.38	0\\
24.39	0\\
24.4	0\\
24.41	0\\
24.42	0\\
24.43	0\\
24.44	0\\
24.45	0\\
24.46	0\\
24.47	0\\
24.48	0\\
24.49	0\\
24.5	0\\
24.51	0\\
24.52	0\\
24.53	0\\
24.54	0\\
24.55	0\\
24.56	0\\
24.57	0\\
24.58	0\\
24.59	0\\
24.6	0\\
24.61	0\\
24.62	0\\
24.63	0\\
24.64	0\\
24.65	0\\
24.66	0\\
24.67	0\\
24.68	0\\
24.69	0\\
24.7	0\\
24.71	0\\
24.72	0\\
24.73	0\\
24.74	0\\
24.75	0\\
24.76	0\\
24.77	0\\
24.78	0\\
24.79	0\\
24.8	0\\
24.81	0\\
24.82	0\\
24.83	0\\
24.84	0\\
24.85	0\\
24.86	0\\
24.87	0\\
24.88	0\\
24.89	0\\
24.9	0\\
24.91	0\\
24.92	0\\
24.93	0\\
24.94	0\\
24.95	0\\
24.96	0\\
24.97	0\\
24.98	0\\
24.99	0\\
25	0\\
25.01	0\\
25.02	0\\
25.03	0\\
25.04	0\\
25.05	0\\
25.06	0\\
25.07	0\\
25.08	0\\
25.09	0\\
25.1	0\\
25.11	0\\
25.12	0\\
25.13	0\\
25.14	0\\
25.15	0\\
25.16	0\\
25.17	0\\
25.18	0\\
25.19	0\\
25.2	0\\
25.21	0\\
25.22	0\\
25.23	0\\
25.24	0\\
25.25	0\\
25.26	0\\
25.27	0\\
25.28	0\\
25.29	0\\
25.3	0\\
25.31	0\\
25.32	0\\
25.33	0\\
25.34	0\\
25.35	0\\
25.36	0\\
25.37	0\\
25.38	0\\
25.39	0\\
25.4	0\\
25.41	0\\
25.42	0\\
25.43	0\\
25.44	0\\
25.45	0\\
25.46	0\\
25.47	0\\
25.48	0\\
25.49	0\\
25.5	0\\
25.51	0\\
25.52	0\\
25.53	0\\
25.54	0\\
25.55	0\\
25.56	0\\
25.57	0\\
25.58	0\\
25.59	0\\
25.6	0\\
25.61	0\\
25.62	0\\
25.63	0\\
25.64	0\\
25.65	0\\
25.66	0\\
25.67	0\\
25.68	0\\
25.69	0\\
25.7	0\\
25.71	0\\
25.72	0\\
25.73	0\\
25.74	0\\
25.75	0\\
25.76	0\\
25.77	0\\
25.78	0\\
25.79	0\\
25.8	0\\
25.81	0\\
25.82	0\\
25.83	0\\
25.84	0\\
25.85	0\\
25.86	0\\
25.87	0\\
25.88	0\\
25.89	0\\
25.9	0\\
25.91	0\\
25.92	0\\
25.93	0\\
25.94	0\\
25.95	0\\
25.96	0\\
25.97	0\\
25.98	0\\
25.99	0\\
26	0\\
26.01	0\\
26.02	0\\
26.03	0\\
26.04	0\\
26.05	0\\
26.06	0\\
26.07	0\\
26.08	0\\
26.09	0\\
26.1	0\\
26.11	0\\
26.12	0\\
26.13	0\\
26.14	0\\
26.15	0\\
26.16	0\\
26.17	0\\
26.18	0\\
26.19	0\\
26.2	0\\
26.21	0\\
26.22	0\\
26.23	0\\
26.24	0\\
26.25	0\\
26.26	0\\
26.27	0\\
26.28	0\\
26.29	0\\
26.3	0\\
26.31	0\\
26.32	0\\
26.33	0\\
26.34	0\\
26.35	0\\
26.36	0\\
26.37	0\\
26.38	0\\
26.39	0\\
26.4	0\\
26.41	0\\
26.42	0\\
26.43	0\\
26.44	0\\
26.45	0\\
26.46	0\\
26.47	0\\
26.48	0\\
26.49	0\\
26.5	0\\
26.51	0\\
26.52	0\\
26.53	0\\
26.54	0\\
26.55	0\\
26.56	0\\
26.57	0\\
26.58	0\\
26.59	0\\
26.6	0\\
26.61	0\\
26.62	0\\
26.63	0\\
26.64	0\\
26.65	0\\
26.66	0\\
26.67	0\\
26.68	0\\
26.69	0\\
26.7	0\\
26.71	0\\
26.72	0\\
26.73	0\\
26.74	0\\
26.75	0\\
26.76	0\\
26.77	0\\
26.78	0\\
26.79	0\\
26.8	0\\
26.81	0\\
26.82	0\\
26.83	0\\
26.84	0\\
26.85	0\\
26.86	0\\
26.87	0\\
26.88	0\\
26.89	0\\
26.9	0\\
26.91	0\\
26.92	0\\
26.93	0\\
26.94	0\\
26.95	0\\
26.96	0\\
26.97	0\\
26.98	0\\
26.99	0\\
27	0\\
27.01	0\\
27.02	0\\
27.03	0\\
27.04	0\\
27.05	0\\
27.06	0\\
27.07	0\\
27.08	0\\
27.09	0\\
27.1	0\\
27.11	0\\
27.12	0\\
27.13	0\\
27.14	0\\
27.15	0\\
27.16	0\\
27.17	0\\
27.18	0\\
27.19	0\\
27.2	0\\
27.21	0\\
27.22	0\\
27.23	0\\
27.24	0\\
27.25	0\\
27.26	0\\
27.27	0\\
27.28	0\\
27.29	0\\
27.3	0\\
27.31	0\\
27.32	0\\
27.33	0\\
27.34	0\\
27.35	0\\
27.36	0\\
27.37	0\\
27.38	0\\
27.39	0\\
27.4	0\\
27.41	0\\
27.42	0\\
27.43	0\\
27.44	0\\
27.45	0\\
27.46	0\\
27.47	0\\
27.48	0\\
27.49	0\\
27.5	0\\
27.51	0\\
27.52	0\\
27.53	0\\
27.54	0\\
27.55	0\\
27.56	0\\
27.57	0\\
27.58	0\\
27.59	0\\
27.6	0\\
27.61	0\\
27.62	0\\
27.63	0\\
27.64	0\\
27.65	0\\
27.66	0\\
27.67	0\\
27.68	0\\
27.69	0\\
27.7	0\\
27.71	0\\
27.72	0\\
27.73	0\\
27.74	0\\
27.75	0\\
27.76	0\\
27.77	0\\
27.78	0\\
27.79	0\\
27.8	0\\
27.81	0\\
27.82	0\\
27.83	0\\
27.84	0\\
27.85	0\\
27.86	0\\
27.87	0\\
27.88	0\\
27.89	0\\
27.9	0\\
27.91	0\\
27.92	0\\
27.93	0\\
27.94	0\\
27.95	0\\
27.96	0\\
27.97	0\\
27.98	0\\
27.99	0\\
28	0\\
28.01	0\\
28.02	0\\
28.03	0\\
28.04	0\\
28.05	0\\
28.06	0\\
28.07	0\\
28.08	0\\
28.09	0\\
28.1	0\\
28.11	0\\
28.12	0\\
28.13	0\\
28.14	0\\
28.15	0\\
28.16	0\\
28.17	0\\
28.18	0\\
28.19	0\\
28.2	0\\
28.21	0\\
28.22	0\\
28.23	0\\
28.24	0\\
28.25	0\\
28.26	0\\
28.27	0\\
28.28	0\\
28.29	0\\
28.3	0\\
28.31	0\\
28.32	0\\
28.33	0\\
28.34	0\\
28.35	0\\
28.36	0\\
28.37	0\\
28.38	0\\
28.39	0\\
28.4	0\\
28.41	0\\
28.42	0\\
28.43	0\\
28.44	0\\
28.45	0\\
28.46	0\\
28.47	0\\
28.48	0\\
28.49	0\\
28.5	0\\
28.51	0\\
28.52	0\\
28.53	0\\
28.54	0\\
28.55	0\\
28.56	0\\
28.57	0\\
28.58	0\\
28.59	0\\
28.6	0\\
28.61	0\\
28.62	0\\
28.63	0\\
28.64	0\\
28.65	0\\
28.66	0\\
28.67	0\\
28.68	0\\
28.69	0\\
28.7	0\\
28.71	0\\
28.72	0\\
28.73	0\\
28.74	0\\
28.75	0\\
28.76	0\\
28.77	0\\
28.78	0\\
28.79	0\\
28.8	0\\
28.81	0\\
28.82	0\\
28.83	0\\
28.84	0\\
28.85	0\\
28.86	0\\
28.87	0\\
28.88	0\\
28.89	0\\
28.9	0\\
28.91	0\\
28.92	0\\
28.93	0\\
28.94	0\\
28.95	0\\
28.96	0\\
28.97	0\\
28.98	0\\
28.99	0\\
29	0\\
29.01	0\\
29.02	0\\
29.03	0\\
29.04	0\\
29.05	0\\
29.06	0\\
29.07	0\\
29.08	0\\
29.09	0\\
29.1	0\\
29.11	0\\
29.12	0\\
29.13	0\\
29.14	0\\
29.15	0\\
29.16	0\\
29.17	0\\
29.18	0\\
29.19	0\\
29.2	0\\
29.21	0\\
29.22	0\\
29.23	0\\
29.24	0\\
29.25	0\\
29.26	0\\
29.27	0\\
29.28	0\\
29.29	0\\
29.3	0\\
29.31	0\\
29.32	0\\
29.33	0\\
29.34	0\\
29.35	0\\
29.36	0\\
29.37	0\\
29.38	0\\
29.39	0\\
29.4	0\\
29.41	0\\
29.42	0\\
29.43	0\\
29.44	0\\
29.45	0\\
29.46	0\\
29.47	0\\
29.48	0\\
29.49	0\\
29.5	0\\
29.51	0\\
29.52	0\\
29.53	0\\
29.54	0\\
29.55	0\\
29.56	0\\
29.57	0\\
29.58	0\\
29.59	0\\
29.6	0\\
29.61	0\\
29.62	0\\
29.63	0\\
29.64	0\\
29.65	0\\
29.66	0\\
29.67	0\\
29.68	0\\
29.69	0\\
29.7	0\\
29.71	0\\
29.72	0\\
29.73	0\\
29.74	0\\
29.75	0\\
29.76	0\\
29.77	0\\
29.78	0\\
29.79	0\\
29.8	0\\
29.81	0\\
29.82	0\\
29.83	0\\
29.84	0\\
29.85	0\\
29.86	0\\
29.87	0\\
29.88	0\\
29.89	0\\
29.9	0\\
29.91	0\\
29.92	0\\
29.93	0\\
29.94	0\\
29.95	0\\
29.96	0\\
29.97	0\\
29.98	0\\
29.99	0\\
30	0\\
30.01	0\\
30.02	0\\
30.03	0\\
30.04	0\\
30.05	0\\
30.06	0\\
30.07	0\\
30.08	0\\
30.09	0\\
30.1	0\\
30.11	0\\
30.12	0\\
30.13	0\\
30.14	0\\
30.15	0\\
30.16	0\\
30.17	0\\
30.18	0\\
30.19	0\\
30.2	0\\
30.21	0\\
30.22	0\\
30.23	0\\
30.24	0\\
30.25	0\\
30.26	0\\
30.27	0\\
30.28	0\\
30.29	0\\
30.3	0\\
30.31	0\\
30.32	0\\
30.33	0\\
30.34	0\\
30.35	0\\
30.36	0\\
30.37	0\\
30.38	0\\
30.39	0\\
30.4	0\\
30.41	0\\
30.42	0\\
30.43	0\\
30.44	0\\
30.45	0\\
30.46	0\\
30.47	0\\
30.48	0\\
30.49	0\\
30.5	0\\
30.51	0\\
30.52	0\\
30.53	0\\
30.54	0\\
30.55	0\\
30.56	0\\
30.57	0\\
30.58	0\\
30.59	0\\
30.6	0\\
30.61	0\\
30.62	0\\
30.63	0\\
30.64	0\\
30.65	0\\
30.66	0\\
30.67	0\\
30.68	0\\
30.69	0\\
30.7	0\\
30.71	0\\
30.72	0\\
30.73	0\\
30.74	0\\
30.75	0\\
30.76	0\\
30.77	0\\
30.78	0\\
30.79	0\\
30.8	0\\
30.81	0\\
30.82	0\\
30.83	0\\
30.84	0\\
30.85	0\\
30.86	0\\
30.87	0\\
30.88	0\\
30.89	0\\
30.9	0\\
30.91	0\\
30.92	0\\
30.93	0\\
30.94	0\\
30.95	0\\
30.96	0\\
30.97	0\\
30.98	0\\
30.99	0\\
31	0\\
31.01	0\\
31.02	0\\
31.03	0\\
31.04	0\\
31.05	0\\
31.06	0\\
31.07	0\\
31.08	0\\
31.09	0\\
31.1	0\\
31.11	0\\
31.12	0\\
31.13	0\\
31.14	0\\
31.15	0\\
31.16	0\\
31.17	0\\
31.18	0\\
31.19	0\\
31.2	0\\
31.21	0\\
31.22	0\\
31.23	0\\
31.24	0\\
31.25	0\\
31.26	0\\
31.27	0\\
31.28	0\\
31.29	0\\
31.3	0\\
31.31	0\\
31.32	0\\
31.33	0\\
31.34	0\\
31.35	0\\
31.36	0\\
31.37	0\\
31.38	0\\
31.39	0\\
31.4	0\\
31.41	0\\
31.42	0\\
31.43	0\\
31.44	0\\
31.45	0\\
31.46	0\\
31.47	0\\
31.48	0\\
31.49	0\\
31.5	0\\
31.51	0\\
31.52	0\\
31.53	0\\
31.54	0\\
31.55	0\\
31.56	0\\
31.57	0\\
31.58	0\\
31.59	0\\
31.6	0\\
31.61	0\\
31.62	0\\
31.63	0\\
31.64	0\\
31.65	0\\
31.66	0\\
31.67	0\\
31.68	0\\
31.69	0\\
31.7	0\\
31.71	0\\
31.72	0\\
31.73	0\\
31.74	0\\
31.75	0\\
31.76	0\\
31.77	0\\
31.78	0\\
31.79	0\\
31.8	0\\
31.81	0\\
31.82	0\\
31.83	0\\
31.84	0\\
31.85	0\\
31.86	0\\
31.87	0\\
31.88	0\\
31.89	0\\
31.9	0\\
31.91	0\\
31.92	0\\
31.93	0\\
31.94	0\\
31.95	0\\
31.96	0\\
31.97	0\\
31.98	0\\
31.99	0\\
32	0\\
32.01	0\\
32.02	0\\
32.03	0\\
32.04	0\\
32.05	0\\
32.06	0\\
32.07	0\\
32.08	0\\
32.09	0\\
32.1	0\\
32.11	0\\
32.12	0\\
32.13	0\\
32.14	0\\
32.15	0\\
32.16	0\\
32.17	0\\
32.18	0\\
32.19	0\\
32.2	0\\
32.21	0\\
32.22	0\\
32.23	0\\
32.24	0\\
32.25	0\\
32.26	0\\
32.27	0\\
32.28	0\\
32.29	0\\
32.3	0\\
32.31	0\\
32.32	0\\
32.33	0\\
32.34	0\\
32.35	0\\
32.36	0\\
32.37	0\\
32.38	0\\
32.39	0\\
32.4	0\\
32.41	0\\
32.42	0\\
32.43	0\\
32.44	0\\
32.45	0\\
32.46	0\\
32.47	0\\
32.48	0\\
32.49	0\\
32.5	0\\
32.51	0\\
32.52	0\\
32.53	0\\
32.54	0\\
32.55	0\\
32.56	0\\
32.57	0\\
32.58	0\\
32.59	0\\
32.6	0\\
32.61	0\\
32.62	0\\
32.63	0\\
32.64	0\\
32.65	0\\
32.66	0\\
32.67	0\\
32.68	0\\
32.69	0\\
32.7	0\\
32.71	0\\
32.72	0\\
32.73	0\\
32.74	0\\
32.75	0\\
32.76	0\\
32.77	0\\
32.78	0\\
32.79	0\\
32.8	0\\
32.81	0\\
32.82	0\\
32.83	0\\
32.84	0\\
32.85	0\\
32.86	0\\
32.87	0\\
32.88	0\\
32.89	0\\
32.9	0\\
32.91	0\\
32.92	0\\
32.93	0\\
32.94	0\\
32.95	0\\
32.96	0\\
32.97	0\\
32.98	0\\
32.99	0\\
33	0\\
33.01	0\\
33.02	0\\
33.03	0\\
33.04	0\\
33.05	0\\
33.06	0\\
33.07	0\\
33.08	0\\
33.09	0\\
33.1	0\\
33.11	0\\
33.12	0\\
33.13	0\\
33.14	0\\
33.15	0\\
33.16	0\\
33.17	0\\
33.18	0\\
33.19	0\\
33.2	0\\
33.21	0\\
33.22	0\\
33.23	0\\
33.24	0\\
33.25	0\\
33.26	0\\
33.27	0\\
33.28	0\\
33.29	0\\
33.3	0\\
33.31	0\\
33.32	0\\
33.33	0\\
33.34	0\\
33.35	0\\
33.36	0\\
33.37	0\\
33.38	0\\
33.39	0\\
33.4	0\\
33.41	0\\
33.42	0\\
33.43	0\\
33.44	0\\
33.45	0\\
33.46	0\\
33.47	0\\
33.48	0\\
33.49	0\\
33.5	0\\
33.51	0\\
33.52	0\\
33.53	0\\
33.54	0\\
33.55	0\\
33.56	0\\
33.57	0\\
33.58	0\\
33.59	0\\
33.6	0\\
33.61	0\\
33.62	0\\
33.63	0\\
33.64	0\\
33.65	0\\
33.66	0\\
33.67	0\\
33.68	0\\
33.69	0\\
33.7	0\\
33.71	0\\
33.72	0\\
33.73	0\\
33.74	0\\
33.75	0\\
33.76	0\\
33.77	0\\
33.78	0\\
33.79	0\\
33.8	0\\
33.81	0\\
33.82	0\\
33.83	0\\
33.84	0\\
33.85	0\\
33.86	0\\
33.87	0\\
33.88	0\\
33.89	0\\
33.9	0\\
33.91	0\\
33.92	0\\
33.93	0\\
33.94	0\\
33.95	0\\
33.96	0\\
33.97	0\\
33.98	0\\
33.99	0\\
34	0\\
34.01	0\\
34.02	0\\
34.03	0\\
34.04	0\\
34.05	0\\
34.06	0\\
34.07	0\\
34.08	0\\
34.09	0\\
34.1	0\\
34.11	0\\
34.12	0\\
34.13	0\\
34.14	0\\
34.15	0\\
34.16	0\\
34.17	0\\
34.18	0\\
34.19	0\\
34.2	0\\
34.21	0\\
34.22	0\\
34.23	0\\
34.24	0\\
34.25	0\\
34.26	0\\
34.27	0\\
34.28	0\\
34.29	0\\
34.3	0\\
34.31	0\\
34.32	0\\
34.33	0\\
34.34	0\\
34.35	0\\
34.36	0\\
34.37	0\\
34.38	0\\
34.39	0\\
34.4	0\\
34.41	0\\
34.42	0\\
34.43	0\\
34.44	0\\
34.45	0\\
34.46	0\\
34.47	0\\
34.48	0\\
34.49	0\\
34.5	0\\
34.51	0\\
34.52	0\\
34.53	0\\
34.54	0\\
34.55	0\\
34.56	0\\
34.57	0\\
34.58	0\\
34.59	0\\
34.6	0\\
34.61	0\\
34.62	0\\
34.63	0\\
34.64	0\\
34.65	0\\
34.66	0\\
34.67	0\\
34.68	0\\
34.69	0\\
34.7	0\\
34.71	0\\
34.72	0\\
34.73	0\\
34.74	0\\
34.75	0\\
34.76	0\\
34.77	0\\
34.78	0\\
34.79	0\\
34.8	0\\
34.81	0\\
34.82	0\\
34.83	0\\
34.84	0\\
34.85	0\\
34.86	0\\
34.87	0\\
34.88	0\\
34.89	0\\
34.9	0\\
34.91	0\\
34.92	0\\
34.93	0\\
34.94	0\\
34.95	0\\
34.96	0\\
34.97	0\\
34.98	0\\
34.99	0\\
35	0\\
35.01	0\\
35.02	0\\
35.03	0\\
35.04	0\\
35.05	0\\
35.06	0\\
35.07	0\\
35.08	0\\
35.09	0\\
35.1	0\\
35.11	0\\
35.12	0\\
35.13	0\\
35.14	0\\
35.15	0\\
35.16	0\\
35.17	0\\
35.18	0\\
35.19	0\\
35.2	0\\
35.21	0\\
35.22	0\\
35.23	0\\
35.24	0\\
35.25	0\\
35.26	0\\
35.27	0\\
35.28	0\\
35.29	0\\
35.3	0\\
35.31	0\\
35.32	0\\
35.33	0\\
35.34	0\\
35.35	0\\
35.36	0\\
35.37	0\\
35.38	0\\
35.39	0\\
35.4	0\\
35.41	0\\
35.42	0\\
35.43	0\\
35.44	0\\
35.45	0\\
35.46	0\\
35.47	0\\
35.48	0\\
35.49	0\\
35.5	0\\
35.51	0\\
35.52	0\\
35.53	0\\
35.54	0\\
35.55	0\\
35.56	0\\
35.57	0\\
35.58	0\\
35.59	0\\
35.6	0\\
35.61	0\\
35.62	0\\
35.63	0\\
35.64	0\\
35.65	0\\
35.66	0\\
35.67	0\\
35.68	0\\
35.69	0\\
35.7	0\\
35.71	0\\
35.72	0\\
35.73	0\\
35.74	0\\
35.75	0\\
35.76	0\\
35.77	0\\
35.78	0\\
35.79	0\\
35.8	0\\
35.81	0\\
35.82	0\\
35.83	0\\
35.84	0\\
35.85	0\\
35.86	0\\
35.87	0\\
35.88	0\\
35.89	0\\
35.9	0\\
35.91	0\\
35.92	0\\
35.93	0\\
35.94	0\\
35.95	0\\
35.96	0\\
35.97	0\\
35.98	0\\
35.99	0\\
36	0\\
36.01	0\\
36.02	0\\
36.03	0\\
36.04	0\\
36.05	0\\
36.06	0\\
36.07	0\\
36.08	0\\
36.09	0\\
36.1	0\\
36.11	0\\
36.12	0\\
36.13	0\\
36.14	0\\
36.15	0\\
36.16	0\\
36.17	0\\
36.18	0\\
36.19	0\\
36.2	0\\
36.21	0\\
36.22	0\\
36.23	0\\
36.24	0\\
36.25	0\\
36.26	0\\
36.27	0\\
36.28	0\\
36.29	0\\
36.3	0\\
36.31	0\\
36.32	0\\
36.33	0\\
36.34	0\\
36.35	0\\
36.36	0\\
36.37	0\\
36.38	0\\
36.39	0\\
36.4	0\\
36.41	0\\
36.42	0\\
36.43	0\\
36.44	0\\
36.45	0\\
36.46	0\\
36.47	0\\
36.48	0\\
36.49	0\\
36.5	0\\
36.51	0\\
36.52	0\\
36.53	0\\
36.54	0\\
36.55	0\\
36.56	0\\
36.57	0\\
36.58	0\\
36.59	0\\
36.6	0\\
36.61	0\\
36.62	0\\
36.63	0\\
36.64	0\\
36.65	0\\
36.66	0\\
36.67	0\\
36.68	0\\
36.69	0\\
36.7	0\\
36.71	0\\
36.72	0\\
36.73	0\\
36.74	0\\
36.75	0\\
36.76	0\\
36.77	0\\
36.78	0\\
36.79	0\\
36.8	0\\
36.81	0\\
36.82	0\\
36.83	0\\
36.84	0\\
36.85	0\\
36.86	0\\
36.87	0\\
36.88	0\\
36.89	0\\
36.9	0\\
36.91	0\\
36.92	0\\
36.93	0\\
36.94	0\\
36.95	0\\
36.96	0\\
36.97	0\\
36.98	0\\
36.99	0\\
37	0\\
37.01	0\\
37.02	0\\
37.03	0\\
37.04	0\\
37.05	0\\
37.06	0\\
37.07	0\\
37.08	0\\
37.09	0\\
37.1	0\\
37.11	0\\
37.12	0\\
37.13	0\\
37.14	0\\
37.15	0\\
37.16	0\\
37.17	0\\
37.18	0\\
37.19	0\\
37.2	0\\
37.21	0\\
37.22	0\\
37.23	0\\
37.24	0\\
37.25	0\\
37.26	0\\
37.27	0\\
37.28	0\\
37.29	0\\
37.3	0\\
37.31	0\\
37.32	0\\
37.33	0\\
37.34	0\\
37.35	0\\
37.36	0\\
37.37	0\\
37.38	0\\
37.39	0\\
37.4	0\\
37.41	0\\
37.42	0\\
37.43	0\\
37.44	0\\
37.45	0\\
37.46	0\\
37.47	0\\
37.48	0\\
37.49	0\\
37.5	0\\
37.51	0\\
37.52	0\\
37.53	0\\
37.54	0\\
37.55	0\\
37.56	0\\
37.57	0\\
37.58	0\\
37.59	0\\
37.6	0\\
37.61	0\\
37.62	0\\
37.63	0\\
37.64	0\\
37.65	0\\
37.66	0\\
37.67	0\\
37.68	0\\
37.69	0\\
37.7	0\\
37.71	0\\
37.72	0\\
37.73	0\\
37.74	0\\
37.75	0\\
37.76	0\\
37.77	0\\
37.78	0\\
37.79	0\\
37.8	0\\
37.81	0\\
37.82	0\\
37.83	0\\
37.84	0\\
37.85	0\\
37.86	0\\
37.87	0\\
37.88	0\\
37.89	0\\
37.9	0\\
37.91	0\\
37.92	0\\
37.93	0\\
37.94	0\\
37.95	0\\
37.96	0\\
37.97	0\\
37.98	0\\
37.99	0\\
38	0\\
38.01	0\\
38.02	0\\
38.03	0\\
38.04	0\\
38.05	0\\
38.06	0\\
38.07	0\\
38.08	0\\
38.09	0\\
38.1	0\\
38.11	0\\
38.12	0\\
38.13	0\\
38.14	0\\
38.15	0\\
38.16	0\\
38.17	0\\
38.18	0\\
38.19	0\\
38.2	0\\
38.21	0\\
38.22	0\\
38.23	0\\
38.24	0\\
38.25	0\\
38.26	0\\
38.27	0\\
38.28	0\\
38.29	0\\
38.3	0\\
38.31	0\\
38.32	0\\
38.33	0\\
38.34	0\\
38.35	0\\
38.36	0\\
38.37	0\\
38.38	0\\
38.39	0\\
38.4	0\\
38.41	0\\
38.42	0\\
38.43	0\\
38.44	0\\
38.45	0\\
38.46	0\\
38.47	0\\
38.48	0\\
38.49	0\\
38.5	0\\
38.51	0\\
38.52	0\\
38.53	0\\
38.54	0\\
38.55	0\\
38.56	0\\
38.57	0\\
38.58	0\\
38.59	0\\
38.6	0\\
38.61	0\\
38.62	0\\
38.63	0\\
38.64	0\\
38.65	0\\
38.66	0\\
38.67	0\\
38.68	0\\
38.69	0\\
38.7	0\\
38.71	0\\
38.72	0\\
38.73	0\\
38.74	0\\
38.75	0\\
38.76	0\\
38.77	0\\
38.78	0\\
38.79	0\\
38.8	0\\
38.81	0\\
38.82	0\\
38.83	0\\
38.84	0\\
38.85	0\\
38.86	0\\
38.87	0\\
38.88	0\\
38.89	0\\
38.9	0\\
38.91	0\\
38.92	0\\
38.93	0\\
38.94	0\\
38.95	0\\
38.96	0\\
38.97	0\\
38.98	0\\
38.99	0\\
39	0\\
39.01	0\\
39.02	0\\
39.03	0\\
39.04	0\\
39.05	0\\
39.06	0\\
39.07	0\\
39.08	0\\
39.09	0\\
39.1	0\\
39.11	0\\
39.12	0\\
39.13	0\\
39.14	0\\
39.15	0\\
39.16	0\\
39.17	0\\
39.18	0\\
39.19	0\\
39.2	0\\
39.21	0\\
39.22	0\\
39.23	0\\
39.24	0\\
39.25	0\\
39.26	0\\
39.27	0\\
39.28	0\\
39.29	0\\
39.3	0\\
39.31	0\\
39.32	0\\
39.33	0\\
39.34	0\\
39.35	0\\
39.36	0\\
39.37	0\\
39.38	0\\
39.39	0\\
39.4	0\\
39.41	0\\
39.42	0\\
39.43	0\\
39.44	0\\
39.45	0\\
39.46	0\\
39.47	0\\
39.48	0\\
39.49	0\\
39.5	0\\
39.51	0\\
39.52	0\\
39.53	0\\
39.54	0\\
39.55	0\\
39.56	0\\
39.57	0\\
39.58	0\\
39.59	0\\
39.6	0\\
39.61	0\\
39.62	0\\
39.63	0\\
39.64	0\\
39.65	0\\
39.66	0\\
39.67	0\\
39.68	0\\
39.69	0\\
39.7	0\\
39.71	0\\
39.72	0\\
39.73	0\\
39.74	0\\
39.75	0\\
39.76	0\\
39.77	0\\
39.78	0\\
39.79	0\\
39.8	0\\
39.81	0\\
39.82	0\\
39.83	0\\
39.84	0\\
39.85	0\\
39.86	0\\
39.87	0\\
39.88	0\\
39.89	0\\
39.9	0\\
39.91	0\\
39.92	0\\
39.93	0\\
39.94	0\\
39.95	0\\
39.96	0\\
39.97	0\\
39.98	0\\
39.99	0\\
40	0\\
40.01	0\\
};
\addplot [color=red,solid,forget plot]
  table[row sep=crcr]{%
40.01	0\\
40.02	0\\
40.03	0\\
40.04	0\\
40.05	0\\
40.06	0\\
40.07	0\\
40.08	0\\
40.09	0\\
40.1	0\\
40.11	0\\
40.12	0\\
40.13	0\\
40.14	0\\
40.15	0\\
40.16	0\\
40.17	0\\
40.18	0\\
40.19	0\\
40.2	0\\
40.21	0\\
40.22	0\\
40.23	0\\
40.24	0\\
40.25	0\\
40.26	0\\
40.27	0\\
40.28	0\\
40.29	0\\
40.3	0\\
40.31	0\\
40.32	0\\
40.33	0\\
40.34	0\\
40.35	0\\
40.36	0\\
40.37	0\\
40.38	0\\
40.39	0\\
40.4	0\\
40.41	0\\
40.42	0\\
40.43	0\\
40.44	0\\
40.45	0\\
40.46	0\\
40.47	0\\
40.48	0\\
40.49	0\\
40.5	0\\
40.51	0\\
40.52	0\\
40.53	0\\
40.54	0\\
40.55	0\\
40.56	0\\
40.57	0\\
40.58	0\\
40.59	0\\
40.6	0\\
40.61	0\\
40.62	0\\
40.63	0\\
40.64	0\\
40.65	0\\
40.66	0\\
40.67	0\\
40.68	0\\
40.69	0\\
40.7	0\\
40.71	0\\
40.72	0\\
40.73	0\\
40.74	0\\
40.75	0\\
40.76	0\\
40.77	0\\
40.78	0\\
40.79	0\\
40.8	0\\
40.81	0\\
40.82	0\\
40.83	0\\
40.84	0\\
40.85	0\\
40.86	0\\
40.87	0\\
40.88	0\\
40.89	0\\
40.9	0\\
40.91	0\\
40.92	0\\
40.93	0\\
40.94	0\\
40.95	0\\
40.96	0\\
40.97	0\\
40.98	0\\
40.99	0\\
41	0\\
41.01	0\\
41.02	0\\
41.03	0\\
41.04	0\\
41.05	0\\
41.06	0\\
41.07	0\\
41.08	0\\
41.09	0\\
41.1	0\\
41.11	0\\
41.12	0\\
41.13	0\\
41.14	0\\
41.15	0\\
41.16	0\\
41.17	0\\
41.18	0\\
41.19	0\\
41.2	0\\
41.21	0\\
41.22	0\\
41.23	0\\
41.24	0\\
41.25	0\\
41.26	0\\
41.27	0\\
41.28	0\\
41.29	0\\
41.3	0\\
41.31	0\\
41.32	0\\
41.33	0\\
41.34	0\\
41.35	0\\
41.36	0\\
41.37	0\\
41.38	0\\
41.39	0\\
41.4	0\\
41.41	0\\
41.42	0\\
41.43	0\\
41.44	0\\
41.45	0\\
41.46	0\\
41.47	0\\
41.48	0\\
41.49	0\\
41.5	0\\
41.51	0\\
41.52	0\\
41.53	0\\
41.54	0\\
41.55	0\\
41.56	0\\
41.57	0\\
41.58	0\\
41.59	0\\
41.6	0\\
41.61	0\\
41.62	0\\
41.63	0\\
41.64	0\\
41.65	0\\
41.66	0\\
41.67	0\\
41.68	0\\
41.69	0\\
41.7	0\\
41.71	0\\
41.72	0\\
41.73	0\\
41.74	0\\
41.75	0\\
41.76	0\\
41.77	0\\
41.78	0\\
41.79	0\\
41.8	0\\
41.81	0\\
41.82	0\\
41.83	0\\
41.84	0\\
41.85	0\\
41.86	0\\
41.87	0\\
41.88	0\\
41.89	0\\
41.9	0\\
41.91	0\\
41.92	0\\
41.93	0\\
41.94	0\\
41.95	0\\
41.96	0\\
41.97	0\\
41.98	0\\
41.99	0\\
42	0\\
42.01	0\\
42.02	0\\
42.03	0\\
42.04	0\\
42.05	0\\
42.06	0\\
42.07	0\\
42.08	0\\
42.09	0\\
42.1	0\\
42.11	0\\
42.12	0\\
42.13	0\\
42.14	0\\
42.15	0\\
42.16	0\\
42.17	0\\
42.18	0\\
42.19	0\\
42.2	0\\
42.21	0\\
42.22	0\\
42.23	0\\
42.24	0\\
42.25	0\\
42.26	0\\
42.27	0\\
42.28	0\\
42.29	0\\
42.3	0\\
42.31	0\\
42.32	0\\
42.33	0\\
42.34	0\\
42.35	0\\
42.36	0\\
42.37	0\\
42.38	0\\
42.39	0\\
42.4	0\\
42.41	0\\
42.42	0\\
42.43	0\\
42.44	0\\
42.45	0\\
42.46	0\\
42.47	0\\
42.48	0\\
42.49	0\\
42.5	0\\
42.51	0\\
42.52	0\\
42.53	0\\
42.54	0\\
42.55	0\\
42.56	0\\
42.57	0\\
42.58	0\\
42.59	0\\
42.6	0\\
42.61	0\\
42.62	0\\
42.63	0\\
42.64	0\\
42.65	0\\
42.66	0\\
42.67	0\\
42.68	0\\
42.69	0\\
42.7	0\\
42.71	0\\
42.72	0\\
42.73	0\\
42.74	0\\
42.75	0\\
42.76	0\\
42.77	0\\
42.78	0\\
42.79	0\\
42.8	0\\
42.81	0\\
42.82	0\\
42.83	0\\
42.84	0\\
42.85	0\\
42.86	0\\
42.87	0\\
42.88	0\\
42.89	0\\
42.9	0\\
42.91	0\\
42.92	0\\
42.93	0\\
42.94	0\\
42.95	0\\
42.96	0\\
42.97	0\\
42.98	0\\
42.99	0\\
43	0\\
43.01	0\\
43.02	0\\
43.03	0\\
43.04	0\\
43.05	0\\
43.06	0\\
43.07	0\\
43.08	0\\
43.09	0\\
43.1	0\\
43.11	0\\
43.12	0\\
43.13	0\\
43.14	0\\
43.15	0\\
43.16	0\\
43.17	0\\
43.18	0\\
43.19	0\\
43.2	0\\
43.21	0\\
43.22	0\\
43.23	0\\
43.24	0\\
43.25	0\\
43.26	0\\
43.27	0\\
43.28	0\\
43.29	0\\
43.3	0\\
43.31	0\\
43.32	0\\
43.33	0\\
43.34	0\\
43.35	0\\
43.36	0\\
43.37	0\\
43.38	0\\
43.39	0\\
43.4	0\\
43.41	0\\
43.42	0\\
43.43	0\\
43.44	0\\
43.45	0\\
43.46	0\\
43.47	0\\
43.48	0\\
43.49	0\\
43.5	0\\
43.51	0\\
43.52	0\\
43.53	0\\
43.54	0\\
43.55	0\\
43.56	0\\
43.57	0\\
43.58	0\\
43.59	0\\
43.6	0\\
43.61	0\\
43.62	0\\
43.63	0\\
43.64	0\\
43.65	0\\
43.66	0\\
43.67	0\\
43.68	0\\
43.69	0\\
43.7	0\\
43.71	0\\
43.72	0\\
43.73	0\\
43.74	0\\
43.75	0\\
43.76	0\\
43.77	0\\
43.78	0\\
43.79	0\\
43.8	0\\
43.81	0\\
43.82	0\\
43.83	0\\
43.84	0\\
43.85	0\\
43.86	0\\
43.87	0\\
43.88	0\\
43.89	0\\
43.9	0\\
43.91	0\\
43.92	0\\
43.93	0\\
43.94	0\\
43.95	0\\
43.96	0\\
43.97	0\\
43.98	0\\
43.99	0\\
44	0\\
44.01	0\\
44.02	0\\
44.03	0\\
44.04	0\\
44.05	0\\
44.06	0\\
44.07	0\\
44.08	0\\
44.09	0\\
44.1	0\\
44.11	0\\
44.12	0\\
44.13	0\\
44.14	0\\
44.15	0\\
44.16	0\\
44.17	0\\
44.18	0\\
44.19	0\\
44.2	0\\
44.21	0\\
44.22	0\\
44.23	0\\
44.24	0\\
44.25	0\\
44.26	0\\
44.27	0\\
44.28	0\\
44.29	0\\
44.3	0\\
44.31	0\\
44.32	0\\
44.33	0\\
44.34	0\\
44.35	0\\
44.36	0\\
44.37	0\\
44.38	0\\
44.39	0\\
44.4	0\\
44.41	0\\
44.42	0\\
44.43	0\\
44.44	0\\
44.45	0\\
44.46	0\\
44.47	0\\
44.48	0\\
44.49	0\\
44.5	0\\
44.51	0\\
44.52	0\\
44.53	0\\
44.54	0\\
44.55	0\\
44.56	0\\
44.57	0\\
44.58	0\\
44.59	0\\
44.6	0\\
44.61	0\\
44.62	0\\
44.63	0\\
44.64	0\\
44.65	0\\
44.66	0\\
44.67	0\\
44.68	0\\
44.69	0\\
44.7	0\\
44.71	0\\
44.72	0\\
44.73	0\\
44.74	0\\
44.75	0\\
44.76	0\\
44.77	0\\
44.78	0\\
44.79	0\\
44.8	0\\
44.81	0\\
44.82	0\\
44.83	0\\
44.84	0\\
44.85	0\\
44.86	0\\
44.87	0\\
44.88	0\\
44.89	0\\
44.9	0\\
44.91	0\\
44.92	0\\
44.93	0\\
44.94	0\\
44.95	0\\
44.96	0\\
44.97	0\\
44.98	0\\
44.99	0\\
45	0\\
45.01	0\\
45.02	0\\
45.03	0\\
45.04	0\\
45.05	0\\
45.06	0\\
45.07	0\\
45.08	0\\
45.09	0\\
45.1	0\\
45.11	0\\
45.12	0\\
45.13	0\\
45.14	0\\
45.15	0\\
45.16	0\\
45.17	0\\
45.18	0\\
45.19	0\\
45.2	0\\
45.21	0\\
45.22	0\\
45.23	0\\
45.24	0\\
45.25	0\\
45.26	0\\
45.27	0\\
45.28	0\\
45.29	0\\
45.3	0\\
45.31	0\\
45.32	0\\
45.33	0\\
45.34	0\\
45.35	0\\
45.36	0\\
45.37	0\\
45.38	0\\
45.39	0\\
45.4	0\\
45.41	0\\
45.42	0\\
45.43	0\\
45.44	0\\
45.45	0\\
45.46	0\\
45.47	0\\
45.48	0\\
45.49	0\\
45.5	0\\
45.51	0\\
45.52	0\\
45.53	0\\
45.54	0\\
45.55	0\\
45.56	0\\
45.57	0\\
45.58	0\\
45.59	0\\
45.6	0\\
45.61	0\\
45.62	0\\
45.63	0\\
45.64	0\\
45.65	0\\
45.66	0\\
45.67	0\\
45.68	0\\
45.69	0\\
45.7	0\\
45.71	0\\
45.72	0\\
45.73	0\\
45.74	0\\
45.75	0\\
45.76	0\\
45.77	0\\
45.78	0\\
45.79	0\\
45.8	0\\
45.81	0\\
45.82	0\\
45.83	0\\
45.84	0\\
45.85	0\\
45.86	0\\
45.87	0\\
45.88	0\\
45.89	0\\
45.9	0\\
45.91	0\\
45.92	0\\
45.93	0\\
45.94	0\\
45.95	0\\
45.96	0\\
45.97	0\\
45.98	0\\
45.99	0\\
46	0\\
46.01	0\\
46.02	0\\
46.03	0\\
46.04	0\\
46.05	0\\
46.06	0\\
46.07	0\\
46.08	0\\
46.09	0\\
46.1	0\\
46.11	0\\
46.12	0\\
46.13	0\\
46.14	0\\
46.15	0\\
46.16	0\\
46.17	0\\
46.18	0\\
46.19	0\\
46.2	0\\
46.21	0\\
46.22	0\\
46.23	0\\
46.24	0\\
46.25	0\\
46.26	0\\
46.27	0\\
46.28	0\\
46.29	0\\
46.3	0\\
46.31	0\\
46.32	0\\
46.33	0\\
46.34	0\\
46.35	0\\
46.36	0\\
46.37	0\\
46.38	0\\
46.39	0\\
46.4	0\\
46.41	0\\
46.42	0\\
46.43	0\\
46.44	0\\
46.45	0\\
46.46	0\\
46.47	0\\
46.48	0\\
46.49	0\\
46.5	0\\
46.51	0\\
46.52	0\\
46.53	0\\
46.54	0\\
46.55	0\\
46.56	0\\
46.57	0\\
46.58	0\\
46.59	0\\
46.6	0\\
46.61	0\\
46.62	0\\
46.63	0\\
46.64	0\\
46.65	0\\
46.66	0\\
46.67	0\\
46.68	0\\
46.69	0\\
46.7	0\\
46.71	0\\
46.72	0\\
46.73	0\\
46.74	0\\
46.75	0\\
46.76	0\\
46.77	0\\
46.78	0\\
46.79	0\\
46.8	0\\
46.81	0\\
46.82	0\\
46.83	0\\
46.84	0\\
46.85	0\\
46.86	0\\
46.87	0\\
46.88	0\\
46.89	0\\
46.9	0\\
46.91	0\\
46.92	0\\
46.93	0\\
46.94	0\\
46.95	0\\
46.96	0\\
46.97	0\\
46.98	0\\
46.99	0\\
47	0\\
47.01	0\\
47.02	0\\
47.03	0\\
47.04	0\\
47.05	0\\
47.06	0\\
47.07	0\\
47.08	0\\
47.09	0\\
47.1	0\\
47.11	0\\
47.12	0\\
47.13	0\\
47.14	0\\
47.15	0\\
47.16	0\\
47.17	0\\
47.18	0\\
47.19	0\\
47.2	0\\
47.21	0\\
47.22	0\\
47.23	0\\
47.24	0\\
47.25	0\\
47.26	0\\
47.27	0\\
47.28	0\\
47.29	0\\
47.3	0\\
47.31	0\\
47.32	0\\
47.33	0\\
47.34	0\\
47.35	0\\
47.36	0\\
47.37	0\\
47.38	0\\
47.39	0\\
47.4	0\\
47.41	0\\
47.42	0\\
47.43	0\\
47.44	0\\
47.45	0\\
47.46	0\\
47.47	0\\
47.48	0\\
47.49	0\\
47.5	0\\
47.51	0\\
47.52	0\\
47.53	0\\
47.54	0\\
47.55	0\\
47.56	0\\
47.57	0\\
47.58	0\\
47.59	0\\
47.6	0\\
47.61	0\\
47.62	0\\
47.63	0\\
47.64	0\\
47.65	0\\
47.66	0\\
47.67	0\\
47.68	0\\
47.69	0\\
47.7	0\\
47.71	0\\
47.72	0\\
47.73	0\\
47.74	0\\
47.75	0\\
47.76	0\\
47.77	0\\
47.78	0\\
47.79	0\\
47.8	0\\
47.81	0\\
47.82	0\\
47.83	0\\
47.84	0\\
47.85	0\\
47.86	0\\
47.87	0\\
47.88	0\\
47.89	0\\
47.9	0\\
47.91	0\\
47.92	0\\
47.93	0\\
47.94	0\\
47.95	0\\
47.96	0\\
47.97	0\\
47.98	0\\
47.99	0\\
48	0\\
48.01	0\\
48.02	0\\
48.03	0\\
48.04	0\\
48.05	0\\
48.06	0\\
48.07	0\\
48.08	0\\
48.09	0\\
48.1	0\\
48.11	0\\
48.12	0\\
48.13	0\\
48.14	0\\
48.15	0\\
48.16	0\\
48.17	0\\
48.18	0\\
48.19	0\\
48.2	0\\
48.21	0\\
48.22	0\\
48.23	0\\
48.24	0\\
48.25	0\\
48.26	0\\
48.27	0\\
48.28	0\\
48.29	0\\
48.3	0\\
48.31	0\\
48.32	0\\
48.33	0\\
48.34	0\\
48.35	0\\
48.36	0\\
48.37	0\\
48.38	0\\
48.39	0\\
48.4	0\\
48.41	0\\
48.42	0\\
48.43	0\\
48.44	0\\
48.45	0\\
48.46	0\\
48.47	0\\
48.48	0\\
48.49	0\\
48.5	0\\
48.51	0\\
48.52	0\\
48.53	0\\
48.54	0\\
48.55	0\\
48.56	0\\
48.57	0\\
48.58	0\\
48.59	0\\
48.6	0\\
48.61	0\\
48.62	0\\
48.63	0\\
48.64	0\\
48.65	0\\
48.66	0\\
48.67	0\\
48.68	0\\
48.69	0\\
48.7	0\\
48.71	0\\
48.72	0\\
48.73	0\\
48.74	0\\
48.75	0\\
48.76	0\\
48.77	0\\
48.78	0\\
48.79	0\\
48.8	0\\
48.81	0\\
48.82	0\\
48.83	0\\
48.84	0\\
48.85	0\\
48.86	0\\
48.87	0\\
48.88	0\\
48.89	0\\
48.9	0\\
48.91	0\\
48.92	0\\
48.93	0\\
48.94	0\\
48.95	0\\
48.96	0\\
48.97	0\\
48.98	0\\
48.99	0\\
49	0\\
49.01	0\\
49.02	0\\
49.03	0\\
49.04	0\\
49.05	0\\
49.06	0\\
49.07	0\\
49.08	0\\
49.09	0\\
49.1	0\\
49.11	0\\
49.12	0\\
49.13	0\\
49.14	0\\
49.15	0\\
49.16	0\\
49.17	0\\
49.18	0\\
49.19	0\\
49.2	0\\
49.21	0\\
49.22	0\\
49.23	0\\
49.24	0\\
49.25	0\\
49.26	0\\
49.27	0\\
49.28	0\\
49.29	0\\
49.3	0\\
49.31	0\\
49.32	0\\
49.33	0\\
49.34	0\\
49.35	0\\
49.36	0\\
49.37	0\\
49.38	0\\
49.39	0\\
49.4	0\\
49.41	0\\
49.42	0\\
49.43	0\\
49.44	0\\
49.45	0\\
49.46	0\\
49.47	0\\
49.48	0\\
49.49	0\\
49.5	0\\
49.51	0\\
49.52	0\\
49.53	0\\
49.54	0\\
49.55	0\\
49.56	0\\
49.57	0\\
49.58	0\\
49.59	0\\
49.6	0\\
49.61	0\\
49.62	0\\
49.63	0\\
49.64	0\\
49.65	0\\
49.66	0\\
49.67	0\\
49.68	0\\
49.69	0\\
49.7	0\\
49.71	0\\
49.72	0\\
49.73	0\\
49.74	0\\
49.75	0\\
49.76	0\\
49.77	0\\
49.78	0\\
49.79	0\\
49.8	0\\
49.81	0\\
49.82	0\\
49.83	0\\
49.84	0\\
49.85	0\\
49.86	0\\
49.87	0\\
49.88	0\\
49.89	0\\
49.9	0\\
49.91	0\\
49.92	0\\
49.93	0\\
49.94	0\\
49.95	0\\
49.96	0\\
49.97	0\\
49.98	0\\
49.99	0\\
50	0\\
50.01	0\\
50.02	0\\
50.03	0\\
50.04	0\\
50.05	0\\
50.06	0\\
50.07	0\\
50.08	0\\
50.09	0\\
50.1	0\\
50.11	0\\
50.12	0\\
50.13	0\\
50.14	0\\
50.15	0\\
50.16	0\\
50.17	0\\
50.18	0\\
50.19	0\\
50.2	0\\
50.21	0\\
50.22	0\\
50.23	0\\
50.24	0\\
50.25	0\\
50.26	0\\
50.27	0\\
50.28	0\\
50.29	0\\
50.3	0\\
50.31	0\\
50.32	0\\
50.33	0\\
50.34	0\\
50.35	0\\
50.36	0\\
50.37	0\\
50.38	0\\
50.39	0\\
50.4	0\\
50.41	0\\
50.42	0\\
50.43	0\\
50.44	0\\
50.45	0\\
50.46	0\\
50.47	0\\
50.48	0\\
50.49	0\\
50.5	0\\
50.51	0\\
50.52	0\\
50.53	0\\
50.54	0\\
50.55	0\\
50.56	0\\
50.57	0\\
50.58	0\\
50.59	0\\
50.6	0\\
50.61	0\\
50.62	0\\
50.63	0\\
50.64	0\\
50.65	0\\
50.66	0\\
50.67	0\\
50.68	0\\
50.69	0\\
50.7	0\\
50.71	0\\
50.72	0\\
50.73	0\\
50.74	0\\
50.75	0\\
50.76	0\\
50.77	0\\
50.78	0\\
50.79	0\\
50.8	0\\
50.81	0\\
50.82	0\\
50.83	0\\
50.84	0\\
50.85	0\\
50.86	0\\
50.87	0\\
50.88	0\\
50.89	0\\
50.9	0\\
50.91	0\\
50.92	0\\
50.93	0\\
50.94	0\\
50.95	0\\
50.96	0\\
50.97	0\\
50.98	0\\
50.99	0\\
51	0\\
51.01	0\\
51.02	0\\
51.03	0\\
51.04	0\\
51.05	0\\
51.06	0\\
51.07	0\\
51.08	0\\
51.09	0\\
51.1	0\\
51.11	0\\
51.12	0\\
51.13	0\\
51.14	0\\
51.15	0\\
51.16	0\\
51.17	0\\
51.18	0\\
51.19	0\\
51.2	0\\
51.21	0\\
51.22	0\\
51.23	0\\
51.24	0\\
51.25	0\\
51.26	0\\
51.27	0\\
51.28	0\\
51.29	0\\
51.3	0\\
51.31	0\\
51.32	0\\
51.33	0\\
51.34	0\\
51.35	0\\
51.36	0\\
51.37	0\\
51.38	0\\
51.39	0\\
51.4	0\\
51.41	0\\
51.42	0\\
51.43	0\\
51.44	0\\
51.45	0\\
51.46	0\\
51.47	0\\
51.48	0\\
51.49	0\\
51.5	0\\
51.51	0\\
51.52	0\\
51.53	0\\
51.54	0\\
51.55	0\\
51.56	0\\
51.57	0\\
51.58	0\\
51.59	0\\
51.6	0\\
51.61	0\\
51.62	0\\
51.63	0\\
51.64	0\\
51.65	0\\
51.66	0\\
51.67	0\\
51.68	0\\
51.69	0\\
51.7	0\\
51.71	0\\
51.72	0\\
51.73	0\\
51.74	0\\
51.75	0\\
51.76	0\\
51.77	0\\
51.78	0\\
51.79	0\\
51.8	0\\
51.81	0\\
51.82	0\\
51.83	0\\
51.84	0\\
51.85	0\\
51.86	0\\
51.87	0\\
51.88	0\\
51.89	0\\
51.9	0\\
51.91	0\\
51.92	0\\
51.93	0\\
51.94	0\\
51.95	0\\
51.96	0\\
51.97	0\\
51.98	0\\
51.99	0\\
52	0\\
52.01	0\\
52.02	0\\
52.03	0\\
52.04	0\\
52.05	0\\
52.06	0\\
52.07	0\\
52.08	0\\
52.09	0\\
52.1	0\\
52.11	0\\
52.12	0\\
52.13	0\\
52.14	0\\
52.15	0\\
52.16	0\\
52.17	0\\
52.18	0\\
52.19	0\\
52.2	0\\
52.21	0\\
52.22	0\\
52.23	0\\
52.24	0\\
52.25	0\\
52.26	0\\
52.27	0\\
52.28	0\\
52.29	0\\
52.3	0\\
52.31	0\\
52.32	0\\
52.33	0\\
52.34	0\\
52.35	0\\
52.36	0\\
52.37	0\\
52.38	0\\
52.39	0\\
52.4	0\\
52.41	0\\
52.42	0\\
52.43	0\\
52.44	0\\
52.45	0\\
52.46	0\\
52.47	0\\
52.48	0\\
52.49	0\\
52.5	0\\
52.51	0\\
52.52	0\\
52.53	0\\
52.54	0\\
52.55	0\\
52.56	0\\
52.57	0\\
52.58	0\\
52.59	0\\
52.6	0\\
52.61	0\\
52.62	0\\
52.63	0\\
52.64	0\\
52.65	0\\
52.66	0\\
52.67	0\\
52.68	0\\
52.69	0\\
52.7	0\\
52.71	0\\
52.72	0\\
52.73	0\\
52.74	0\\
52.75	0\\
52.76	0\\
52.77	0\\
52.78	0\\
52.79	0\\
52.8	0\\
52.81	0\\
52.82	0\\
52.83	0\\
52.84	0\\
52.85	0\\
52.86	0\\
52.87	0\\
52.88	0\\
52.89	0\\
52.9	0\\
52.91	0\\
52.92	0\\
52.93	0\\
52.94	0\\
52.95	0\\
52.96	0\\
52.97	0\\
52.98	0\\
52.99	0\\
53	0\\
53.01	0\\
53.02	0\\
53.03	0\\
53.04	0\\
53.05	0\\
53.06	0\\
53.07	0\\
53.08	0\\
53.09	0\\
53.1	0\\
53.11	0\\
53.12	0\\
53.13	0\\
53.14	0\\
53.15	0\\
53.16	0\\
53.17	0\\
53.18	0\\
53.19	0\\
53.2	0\\
53.21	0\\
53.22	0\\
53.23	0\\
53.24	0\\
53.25	0\\
53.26	0\\
53.27	0\\
53.28	0\\
53.29	0\\
53.3	0\\
53.31	0\\
53.32	0\\
53.33	0\\
53.34	0\\
53.35	0\\
53.36	0\\
53.37	0\\
53.38	0\\
53.39	0\\
53.4	0\\
53.41	0\\
53.42	0\\
53.43	0\\
53.44	0\\
53.45	0\\
53.46	0\\
53.47	0\\
53.48	0\\
53.49	0\\
53.5	0\\
53.51	0\\
53.52	0\\
53.53	0\\
53.54	0\\
53.55	0\\
53.56	0\\
53.57	0\\
53.58	0\\
53.59	0\\
53.6	0\\
53.61	0\\
53.62	0\\
53.63	0\\
53.64	0\\
53.65	0\\
53.66	0\\
53.67	0\\
53.68	0\\
53.69	0\\
53.7	0\\
53.71	0\\
53.72	0\\
53.73	0\\
53.74	0\\
53.75	0\\
53.76	0\\
53.77	0\\
53.78	0\\
53.79	0\\
53.8	0\\
53.81	0\\
53.82	0\\
53.83	0\\
53.84	0\\
53.85	0\\
53.86	0\\
53.87	0\\
53.88	0\\
53.89	0\\
53.9	0\\
53.91	0\\
53.92	0\\
53.93	0\\
53.94	0\\
53.95	0\\
53.96	0\\
53.97	0\\
53.98	0\\
53.99	0\\
54	0\\
54.01	0\\
54.02	0\\
54.03	0\\
54.04	0\\
54.05	0\\
54.06	0\\
54.07	0\\
54.08	0\\
54.09	0\\
54.1	0\\
54.11	0\\
54.12	0\\
54.13	0\\
54.14	0\\
54.15	0\\
54.16	0\\
54.17	0\\
54.18	0\\
54.19	0\\
54.2	0\\
54.21	0\\
54.22	0\\
54.23	0\\
54.24	0\\
54.25	0\\
54.26	0\\
54.27	0\\
54.28	0\\
54.29	0\\
54.3	0\\
54.31	0\\
54.32	0\\
54.33	0\\
54.34	0\\
54.35	0\\
54.36	0\\
54.37	0\\
54.38	0\\
54.39	0\\
54.4	0\\
54.41	0\\
54.42	0\\
54.43	0\\
54.44	0\\
54.45	0\\
54.46	0\\
54.47	0\\
54.48	0\\
54.49	0\\
54.5	0\\
54.51	0\\
54.52	0\\
54.53	0\\
54.54	0\\
54.55	0\\
54.56	0\\
54.57	0\\
54.58	0\\
54.59	0\\
54.6	0\\
54.61	0\\
54.62	0\\
54.63	0\\
54.64	0\\
54.65	0\\
54.66	0\\
54.67	0\\
54.68	0\\
54.69	0\\
54.7	0\\
54.71	0\\
54.72	0\\
54.73	0\\
54.74	0\\
54.75	0\\
54.76	0\\
54.77	0\\
54.78	0\\
54.79	0\\
54.8	0\\
54.81	0\\
54.82	0\\
54.83	0\\
54.84	0\\
54.85	0\\
54.86	0\\
54.87	0\\
54.88	0\\
54.89	0\\
54.9	0\\
54.91	0\\
54.92	0\\
54.93	0\\
54.94	0\\
54.95	0\\
54.96	0\\
54.97	0\\
54.98	0\\
54.99	0\\
55	0\\
55.01	0\\
55.02	0\\
55.03	0\\
55.04	0\\
55.05	0\\
55.06	0\\
55.07	0\\
55.08	0\\
55.09	0\\
55.1	0\\
55.11	0\\
55.12	0\\
55.13	0\\
55.14	0\\
55.15	0\\
55.16	0\\
55.17	0\\
55.18	0\\
55.19	0\\
55.2	0\\
55.21	0\\
55.22	0\\
55.23	0\\
55.24	0\\
55.25	0\\
55.26	0\\
55.27	0\\
55.28	0\\
55.29	0\\
55.3	0\\
55.31	0\\
55.32	0\\
55.33	0\\
55.34	0\\
55.35	0\\
55.36	0\\
55.37	0\\
55.38	0\\
55.39	0\\
55.4	0\\
55.41	0\\
55.42	0\\
55.43	0\\
55.44	0\\
55.45	0\\
55.46	0\\
55.47	0\\
55.48	0\\
55.49	0\\
55.5	0\\
55.51	0\\
55.52	0\\
55.53	0\\
55.54	0\\
55.55	0\\
55.56	0\\
55.57	0\\
55.58	0\\
55.59	0\\
55.6	0\\
55.61	0\\
55.62	0\\
55.63	0\\
55.64	0\\
55.65	0\\
55.66	0\\
55.67	0\\
55.68	0\\
55.69	0\\
55.7	0\\
55.71	0\\
55.72	0\\
55.73	0\\
55.74	0\\
55.75	0\\
55.76	0\\
55.77	0\\
55.78	0\\
55.79	0\\
55.8	0\\
55.81	0\\
55.82	0\\
55.83	0\\
55.84	0\\
55.85	0\\
55.86	0\\
55.87	0\\
55.88	0\\
55.89	0\\
55.9	0\\
55.91	0\\
55.92	0\\
55.93	0\\
55.94	0\\
55.95	0\\
55.96	0\\
55.97	0\\
55.98	0\\
55.99	0\\
56	0\\
56.01	0\\
56.02	0\\
56.03	0\\
56.04	0\\
56.05	0\\
56.06	0\\
56.07	0\\
56.08	0\\
56.09	0\\
56.1	0\\
56.11	0\\
56.12	0\\
56.13	0\\
56.14	0\\
56.15	0\\
56.16	0\\
56.17	0\\
56.18	0\\
56.19	0\\
56.2	0\\
56.21	0\\
56.22	0\\
56.23	0\\
56.24	0\\
56.25	0\\
56.26	0\\
56.27	0\\
56.28	0\\
56.29	0\\
56.3	0\\
56.31	0\\
56.32	0\\
56.33	0\\
56.34	0\\
56.35	0\\
56.36	0\\
56.37	0\\
56.38	0\\
56.39	0\\
56.4	0\\
56.41	0\\
56.42	0\\
56.43	0\\
56.44	0\\
56.45	0\\
56.46	0\\
56.47	0\\
56.48	0\\
56.49	0\\
56.5	0\\
56.51	0\\
56.52	0\\
56.53	0\\
56.54	0\\
56.55	0\\
56.56	0\\
56.57	0\\
56.58	0\\
56.59	0\\
56.6	0\\
56.61	0\\
56.62	0\\
56.63	0\\
56.64	0\\
56.65	0\\
56.66	0\\
56.67	0\\
56.68	0\\
56.69	0\\
56.7	0\\
56.71	0\\
56.72	0\\
56.73	0\\
56.74	0\\
56.75	0\\
56.76	0\\
56.77	0\\
56.78	0\\
56.79	0\\
56.8	0\\
56.81	0\\
56.82	0\\
56.83	0\\
56.84	0\\
56.85	0\\
56.86	0\\
56.87	0\\
56.88	0\\
56.89	0\\
56.9	0\\
56.91	0\\
56.92	0\\
56.93	0\\
56.94	0\\
56.95	0\\
56.96	0\\
56.97	0\\
56.98	0\\
56.99	0\\
57	0\\
57.01	0\\
57.02	0\\
57.03	0\\
57.04	0\\
57.05	0\\
57.06	0\\
57.07	0\\
57.08	0\\
57.09	0\\
57.1	0\\
57.11	0\\
57.12	0\\
57.13	0\\
57.14	0\\
57.15	0\\
57.16	0\\
57.17	0\\
57.18	0\\
57.19	0\\
57.2	0\\
57.21	0\\
57.22	0\\
57.23	0\\
57.24	0\\
57.25	0\\
57.26	0\\
57.27	0\\
57.28	0\\
57.29	0\\
57.3	0\\
57.31	0\\
57.32	0\\
57.33	0\\
57.34	0\\
57.35	0\\
57.36	0\\
57.37	0\\
57.38	0\\
57.39	0\\
57.4	0\\
57.41	0\\
57.42	0\\
57.43	0\\
57.44	0\\
57.45	0\\
57.46	0\\
57.47	0\\
57.48	0\\
57.49	0\\
57.5	0\\
57.51	0\\
57.52	0\\
57.53	0\\
57.54	0\\
57.55	0\\
57.56	0\\
57.57	0\\
57.58	0\\
57.59	0\\
57.6	0\\
57.61	0\\
57.62	0\\
57.63	0\\
57.64	0\\
57.65	0\\
57.66	0\\
57.67	0\\
57.68	0\\
57.69	0\\
57.7	0\\
57.71	0\\
57.72	0\\
57.73	0\\
57.74	0\\
57.75	0\\
57.76	0\\
57.77	0\\
57.78	0\\
57.79	0\\
57.8	0\\
57.81	0\\
57.82	0\\
57.83	0\\
57.84	0\\
57.85	0\\
57.86	0\\
57.87	0\\
57.88	0\\
57.89	0\\
57.9	0\\
57.91	0\\
57.92	0\\
57.93	0\\
57.94	0\\
57.95	0\\
57.96	0\\
57.97	0\\
57.98	0\\
57.99	0\\
58	0\\
58.01	0\\
58.02	0\\
58.03	0\\
58.04	0\\
58.05	0\\
58.06	0\\
58.07	0\\
58.08	0\\
58.09	0\\
58.1	0\\
58.11	0\\
58.12	0\\
58.13	0\\
58.14	0\\
58.15	0\\
58.16	0\\
58.17	0\\
58.18	0\\
58.19	0\\
58.2	0\\
58.21	0\\
58.22	0\\
58.23	0\\
58.24	0\\
58.25	0\\
58.26	0\\
58.27	0\\
58.28	0\\
58.29	0\\
58.3	0\\
58.31	0\\
58.32	0\\
58.33	0\\
58.34	0\\
58.35	0\\
58.36	0\\
58.37	0\\
58.38	0\\
58.39	0\\
58.4	0\\
58.41	0\\
58.42	0\\
58.43	0\\
58.44	0\\
58.45	0\\
58.46	0\\
58.47	0\\
58.48	0\\
58.49	0\\
58.5	0\\
58.51	0\\
58.52	0\\
58.53	0\\
58.54	0\\
58.55	0\\
58.56	0\\
58.57	0\\
58.58	0\\
58.59	0\\
58.6	0\\
58.61	0\\
58.62	0\\
58.63	0\\
58.64	0\\
58.65	0\\
58.66	0\\
58.67	0\\
58.68	0\\
58.69	0\\
58.7	0\\
58.71	0\\
58.72	0\\
58.73	0\\
58.74	0\\
58.75	0\\
58.76	0\\
58.77	0\\
58.78	0\\
58.79	0\\
58.8	0\\
58.81	0\\
58.82	0\\
58.83	0\\
58.84	0\\
58.85	0\\
58.86	0\\
58.87	0\\
58.88	0\\
58.89	0\\
58.9	0\\
58.91	0\\
58.92	0\\
58.93	0\\
58.94	0\\
58.95	0\\
58.96	0\\
58.97	0\\
58.98	0\\
58.99	0\\
59	0\\
59.01	0\\
59.02	0\\
59.03	0\\
59.04	0\\
59.05	0\\
59.06	0\\
59.07	0\\
59.08	0\\
59.09	0\\
59.1	0\\
59.11	0\\
59.12	0\\
59.13	0\\
59.14	0\\
59.15	0\\
59.16	0\\
59.17	0\\
59.18	0\\
59.19	0\\
59.2	0\\
59.21	0\\
59.22	0\\
59.23	0\\
59.24	0\\
59.25	0\\
59.26	0\\
59.27	0\\
59.28	0\\
59.29	0\\
59.3	0\\
59.31	0\\
59.32	0\\
59.33	0\\
59.34	0\\
59.35	0\\
59.36	0\\
59.37	0\\
59.38	0\\
59.39	0\\
59.4	0\\
59.41	0\\
59.42	0\\
59.43	0\\
59.44	0\\
59.45	0\\
59.46	0\\
59.47	0\\
59.48	0\\
59.49	0\\
59.5	0\\
59.51	0\\
59.52	0\\
59.53	0\\
59.54	0\\
59.55	0\\
59.56	0\\
59.57	0\\
59.58	0\\
59.59	0\\
59.6	0\\
59.61	0\\
59.62	0\\
59.63	0\\
59.64	0\\
59.65	0\\
59.66	0\\
59.67	0\\
59.68	0\\
59.69	0\\
59.7	0\\
59.71	0\\
59.72	0\\
59.73	0\\
59.74	0\\
59.75	0\\
59.76	0\\
59.77	0\\
59.78	0\\
59.79	0\\
59.8	0\\
59.81	0\\
59.82	0\\
59.83	0\\
59.84	0\\
59.85	0\\
59.86	0\\
59.87	0\\
59.88	0\\
59.89	0\\
59.9	0\\
59.91	0\\
59.92	0\\
59.93	0\\
59.94	0\\
59.95	0\\
59.96	0\\
59.97	0\\
59.98	0\\
59.99	0\\
60	0\\
60.01	0\\
60.02	0\\
60.03	0\\
60.04	0\\
60.05	0\\
60.06	0\\
60.07	0\\
60.08	0\\
60.09	0\\
60.1	0\\
60.11	0\\
60.12	0\\
60.13	0\\
60.14	0\\
60.15	0\\
60.16	0\\
60.17	0\\
60.18	0\\
60.19	0\\
60.2	0\\
60.21	0\\
60.22	0\\
60.23	0\\
60.24	0\\
60.25	0\\
60.26	0\\
60.27	0\\
60.28	0\\
60.29	0\\
60.3	0\\
60.31	0\\
60.32	0\\
60.33	0\\
60.34	0\\
60.35	0\\
60.36	0\\
60.37	0\\
60.38	0\\
60.39	0\\
60.4	0\\
60.41	0\\
60.42	0\\
60.43	0\\
60.44	0\\
60.45	0\\
60.46	0\\
60.47	0\\
60.48	0\\
60.49	0\\
60.5	0\\
60.51	0\\
60.52	0\\
60.53	0\\
60.54	0\\
60.55	0\\
60.56	0\\
60.57	0\\
60.58	0\\
60.59	0\\
60.6	0\\
60.61	0\\
60.62	0\\
60.63	0\\
60.64	0\\
60.65	0\\
60.66	0\\
60.67	0\\
60.68	0\\
60.69	0\\
60.7	0\\
60.71	0\\
60.72	0\\
60.73	0\\
60.74	0\\
60.75	0\\
60.76	0\\
60.77	0\\
60.78	0\\
60.79	0\\
60.8	0\\
60.81	0\\
60.82	0\\
60.83	0\\
60.84	0\\
60.85	0\\
60.86	0\\
60.87	0\\
60.88	0\\
60.89	0\\
60.9	0\\
60.91	0\\
60.92	0\\
60.93	0\\
60.94	0\\
60.95	0\\
60.96	0\\
60.97	0\\
60.98	0\\
60.99	0\\
61	0\\
61.01	0\\
61.02	0\\
61.03	0\\
61.04	0\\
61.05	0\\
61.06	0\\
61.07	0\\
61.08	0\\
61.09	0\\
61.1	0\\
61.11	0\\
61.12	0\\
61.13	0\\
61.14	0\\
61.15	0\\
61.16	0\\
61.17	0\\
61.18	0\\
61.19	0\\
61.2	0\\
61.21	0\\
61.22	0\\
61.23	0\\
61.24	0\\
61.25	0\\
61.26	0\\
61.27	0\\
61.28	0\\
61.29	0\\
61.3	0\\
61.31	0\\
61.32	0\\
61.33	0\\
61.34	0\\
61.35	0\\
61.36	0\\
61.37	0\\
61.38	0\\
61.39	0\\
61.4	0\\
61.41	0\\
61.42	0\\
61.43	0\\
61.44	0\\
61.45	0\\
61.46	0\\
61.47	0\\
61.48	0\\
61.49	0\\
61.5	0\\
61.51	0\\
61.52	0\\
61.53	0\\
61.54	0\\
61.55	0\\
61.56	0\\
61.57	0\\
61.58	0\\
61.59	0\\
61.6	0\\
61.61	0\\
61.62	0\\
61.63	0\\
61.64	0\\
61.65	0\\
61.66	0\\
61.67	0\\
61.68	0\\
61.69	0\\
61.7	0\\
61.71	0\\
61.72	0\\
61.73	0\\
61.74	0\\
61.75	0\\
61.76	0\\
61.77	0\\
61.78	0\\
61.79	0\\
61.8	0\\
61.81	0\\
61.82	0\\
61.83	0\\
61.84	0\\
61.85	0\\
61.86	0\\
61.87	0\\
61.88	0\\
61.89	0\\
61.9	0\\
61.91	0\\
61.92	0\\
61.93	0\\
61.94	0\\
61.95	0\\
61.96	0\\
61.97	0\\
61.98	0\\
61.99	0\\
62	0\\
62.01	0\\
62.02	0\\
62.03	0\\
62.04	0\\
62.05	0\\
62.06	0\\
62.07	0\\
62.08	0\\
62.09	0\\
62.1	0\\
62.11	0\\
62.12	0\\
62.13	0\\
62.14	0\\
62.15	0\\
62.16	0\\
62.17	0\\
62.18	0\\
62.19	0\\
62.2	0\\
62.21	0\\
62.22	0\\
62.23	0\\
62.24	0\\
62.25	0\\
62.26	0\\
62.27	0\\
62.28	0\\
62.29	0\\
62.3	0\\
62.31	0\\
62.32	0\\
62.33	0\\
62.34	0\\
62.35	0\\
62.36	0\\
62.37	0\\
62.38	0\\
62.39	0\\
62.4	0\\
62.41	0\\
62.42	0\\
62.43	0\\
62.44	0\\
62.45	0\\
62.46	0\\
62.47	0\\
62.48	0\\
62.49	0\\
62.5	0\\
62.51	0\\
62.52	0\\
62.53	0\\
62.54	0\\
62.55	0\\
62.56	0\\
62.57	0\\
62.58	0\\
62.59	0\\
62.6	0\\
62.61	0\\
62.62	0\\
62.63	0\\
62.64	0\\
62.65	0\\
62.66	0\\
62.67	0\\
62.68	0\\
62.69	0\\
62.7	0\\
62.71	0\\
62.72	0\\
62.73	0\\
62.74	0\\
62.75	0\\
62.76	0\\
62.77	0\\
62.78	0\\
62.79	0\\
62.8	0\\
62.81	0\\
62.82	0\\
62.83	0\\
62.84	0\\
62.85	0\\
62.86	0\\
62.87	0\\
62.88	0\\
62.89	0\\
62.9	0\\
62.91	0\\
62.92	0\\
62.93	0\\
62.94	0\\
62.95	0\\
62.96	0\\
62.97	0\\
62.98	0\\
62.99	0\\
63	0\\
63.01	0\\
63.02	0\\
63.03	0\\
63.04	0\\
63.05	0\\
63.06	0\\
63.07	0\\
63.08	0\\
63.09	0\\
63.1	0\\
63.11	0\\
63.12	0\\
63.13	0\\
63.14	0\\
63.15	0\\
63.16	0\\
63.17	0\\
63.18	0\\
63.19	0\\
63.2	0\\
63.21	0\\
63.22	0\\
63.23	0\\
63.24	0\\
63.25	0\\
63.26	0\\
63.27	0\\
63.28	0\\
63.29	0\\
63.3	0\\
63.31	0\\
63.32	0\\
63.33	0\\
63.34	0\\
63.35	0\\
63.36	0\\
63.37	0\\
63.38	0\\
63.39	0\\
63.4	0\\
63.41	0\\
63.42	0\\
63.43	0\\
63.44	0\\
63.45	0\\
63.46	0\\
63.47	0\\
63.48	0\\
63.49	0\\
63.5	0\\
63.51	0\\
63.52	0\\
63.53	0\\
63.54	0\\
63.55	0\\
63.56	0\\
63.57	0\\
63.58	0\\
63.59	0\\
63.6	0\\
63.61	0\\
63.62	0\\
63.63	0\\
63.64	0\\
63.65	0\\
63.66	0\\
63.67	0\\
63.68	0\\
63.69	0\\
63.7	0\\
63.71	0\\
63.72	0\\
63.73	0\\
63.74	0\\
63.75	0\\
63.76	0\\
63.77	0\\
63.78	0\\
63.79	0\\
63.8	0\\
63.81	0\\
63.82	0\\
63.83	0\\
63.84	0\\
63.85	0\\
63.86	0\\
63.87	0\\
63.88	0\\
63.89	0\\
63.9	0\\
63.91	0\\
63.92	0\\
63.93	0\\
63.94	0\\
63.95	0\\
63.96	0\\
63.97	0\\
63.98	0\\
63.99	0\\
64	0\\
64.01	0\\
64.02	0\\
64.03	0\\
64.04	0\\
64.05	0\\
64.06	0\\
64.07	0\\
64.08	0\\
64.09	0\\
64.1	0\\
64.11	0\\
64.12	0\\
64.13	0\\
64.14	0\\
64.15	0\\
64.16	0\\
64.17	0\\
64.18	0\\
64.19	0\\
64.2	0\\
64.21	0\\
64.22	0\\
64.23	0\\
64.24	0\\
64.25	0\\
64.26	0\\
64.27	0\\
64.28	0\\
64.29	0\\
64.3	0\\
64.31	0\\
64.32	0\\
64.33	0\\
64.34	0\\
64.35	0\\
64.36	0\\
64.37	0\\
64.38	0\\
64.39	0\\
64.4	0\\
64.41	0\\
64.42	0\\
64.43	0\\
64.44	0\\
64.45	0\\
64.46	0\\
64.47	0\\
64.48	0\\
64.49	0\\
64.5	0\\
64.51	0\\
64.52	0\\
64.53	0\\
64.54	0\\
64.55	0\\
64.56	0\\
64.57	0\\
64.58	0\\
64.59	0\\
64.6	0\\
64.61	0\\
64.62	0\\
64.63	0\\
64.64	0\\
64.65	0\\
64.66	0\\
64.67	0\\
64.68	0\\
64.69	0\\
64.7	0\\
64.71	0\\
64.72	0\\
64.73	0\\
64.74	0\\
64.75	0\\
64.76	0\\
64.77	0\\
64.78	0\\
64.79	0\\
64.8	0\\
64.81	0\\
64.82	0\\
64.83	0\\
64.84	0\\
64.85	0\\
64.86	0\\
64.87	0\\
64.88	0\\
64.89	0\\
64.9	0\\
64.91	0\\
64.92	0\\
64.93	0\\
64.94	0\\
64.95	0\\
64.96	0\\
64.97	0\\
64.98	0\\
64.99	0\\
65	0\\
65.01	0\\
65.02	0\\
65.03	0\\
65.04	0\\
65.05	0\\
65.06	0\\
65.07	0\\
65.08	0\\
65.09	0\\
65.1	0\\
65.11	0\\
65.12	0\\
65.13	0\\
65.14	0\\
65.15	0\\
65.16	0\\
65.17	0\\
65.18	0\\
65.19	0\\
65.2	0\\
65.21	0\\
65.22	0\\
65.23	0\\
65.24	0\\
65.25	0\\
65.26	0\\
65.27	0\\
65.28	0\\
65.29	0\\
65.3	0\\
65.31	0\\
65.32	0\\
65.33	0\\
65.34	0\\
65.35	0\\
65.36	0\\
65.37	0\\
65.38	0\\
65.39	0\\
65.4	0\\
65.41	0\\
65.42	0\\
65.43	0\\
65.44	0\\
65.45	0\\
65.46	0\\
65.47	0\\
65.48	0\\
65.49	0\\
65.5	0\\
65.51	0\\
65.52	0\\
65.53	0\\
65.54	0\\
65.55	0\\
65.56	0\\
65.57	0\\
65.58	0\\
65.59	0\\
65.6	0\\
65.61	0\\
65.62	0\\
65.63	0\\
65.64	0\\
65.65	0\\
65.66	0\\
65.67	0\\
65.68	0\\
65.69	0\\
65.7	0\\
65.71	0\\
65.72	0\\
65.73	0\\
65.74	0\\
65.75	0\\
65.76	0\\
65.77	0\\
65.78	0\\
65.79	0\\
65.8	0\\
65.81	0\\
65.82	0\\
65.83	0\\
65.84	0\\
65.85	0\\
65.86	0\\
65.87	0\\
65.88	0\\
65.89	0\\
65.9	0\\
65.91	0\\
65.92	0\\
65.93	0\\
65.94	0\\
65.95	0\\
65.96	0\\
65.97	0\\
65.98	0\\
65.99	0\\
66	0\\
66.01	0\\
66.02	0\\
66.03	0\\
66.04	0\\
66.05	0\\
66.06	0\\
66.07	0\\
66.08	0\\
66.09	0\\
66.1	0\\
66.11	0\\
66.12	0\\
66.13	0\\
66.14	0\\
66.15	0\\
66.16	0\\
66.17	0\\
66.18	0\\
66.19	0\\
66.2	0\\
66.21	0\\
66.22	0\\
66.23	0\\
66.24	0\\
66.25	0\\
66.26	0\\
66.27	0\\
66.28	0\\
66.29	0\\
66.3	0\\
66.31	0\\
66.32	0\\
66.33	0\\
66.34	0\\
66.35	0\\
66.36	0\\
66.37	0\\
66.38	0\\
66.39	0\\
66.4	0\\
66.41	0\\
66.42	0\\
66.43	0\\
66.44	0\\
66.45	0\\
66.46	0\\
66.47	0\\
66.48	0\\
66.49	0\\
66.5	0\\
66.51	0\\
66.52	0\\
66.53	0\\
66.54	0\\
66.55	0\\
66.56	0\\
66.57	0\\
66.58	0\\
66.59	0\\
66.6	0\\
66.61	0\\
66.62	0\\
66.63	0\\
66.64	0\\
66.65	0\\
66.66	0\\
66.67	0\\
66.68	0\\
66.69	0\\
66.7	0\\
66.71	0\\
66.72	0\\
66.73	0\\
66.74	0\\
66.75	0\\
66.76	0\\
66.77	0\\
66.78	0\\
66.79	0\\
66.8	0\\
66.81	0\\
66.82	0\\
66.83	0\\
66.84	0\\
66.85	0\\
66.86	0\\
66.87	0\\
66.88	0\\
66.89	0\\
66.9	0\\
66.91	0\\
66.92	0\\
66.93	0\\
66.94	0\\
66.95	0\\
66.96	0\\
66.97	0\\
66.98	0\\
66.99	0\\
67	0\\
67.01	0\\
67.02	0\\
67.03	0\\
67.04	0\\
67.05	0\\
67.06	0\\
67.07	0\\
67.08	0\\
67.09	0\\
67.1	0\\
67.11	0\\
67.12	0\\
67.13	0\\
67.14	0\\
67.15	0\\
67.16	0\\
67.17	0\\
67.18	0\\
67.19	0\\
67.2	0\\
67.21	0\\
67.22	0\\
67.23	0\\
67.24	0\\
67.25	0\\
67.26	0\\
67.27	0\\
67.28	0\\
67.29	0\\
67.3	0\\
67.31	0\\
67.32	0\\
67.33	0\\
67.34	0\\
67.35	0\\
67.36	0\\
67.37	0\\
67.38	0\\
67.39	0\\
67.4	0\\
67.41	0\\
67.42	0\\
67.43	0\\
67.44	0\\
67.45	0\\
67.46	0\\
67.47	0\\
67.48	0\\
67.49	0\\
67.5	0\\
67.51	0\\
67.52	0\\
67.53	0\\
67.54	0\\
67.55	0\\
67.56	0\\
67.57	0\\
67.58	0\\
67.59	0\\
67.6	0\\
67.61	0\\
67.62	0\\
67.63	0\\
67.64	0\\
67.65	0\\
67.66	0\\
67.67	0\\
67.68	0\\
67.69	0\\
67.7	0\\
67.71	0\\
67.72	0\\
67.73	0\\
67.74	0\\
67.75	0\\
67.76	0\\
67.77	0\\
67.78	0\\
67.79	0\\
67.8	0\\
67.81	0\\
67.82	0\\
67.83	0\\
67.84	0\\
67.85	0\\
67.86	0\\
67.87	0\\
67.88	0\\
67.89	0\\
67.9	0\\
67.91	0\\
67.92	0\\
67.93	0\\
67.94	0\\
67.95	0\\
67.96	0\\
67.97	0\\
67.98	0\\
67.99	0\\
68	0\\
68.01	0\\
68.02	0\\
68.03	0\\
68.04	0\\
68.05	0\\
68.06	0\\
68.07	0\\
68.08	0\\
68.09	0\\
68.1	0\\
68.11	0\\
68.12	0\\
68.13	0\\
68.14	0\\
68.15	0\\
68.16	0\\
68.17	0\\
68.18	0\\
68.19	0\\
68.2	0\\
68.21	0\\
68.22	0\\
68.23	0\\
68.24	0\\
68.25	0\\
68.26	0\\
68.27	0\\
68.28	0\\
68.29	0\\
68.3	0\\
68.31	0\\
68.32	0\\
68.33	0\\
68.34	0\\
68.35	0\\
68.36	0\\
68.37	0\\
68.38	0\\
68.39	0\\
68.4	0\\
68.41	0\\
68.42	0\\
68.43	0\\
68.44	0\\
68.45	0\\
68.46	0\\
68.47	0\\
68.48	0\\
68.49	0\\
68.5	0\\
68.51	0\\
68.52	0\\
68.53	0\\
68.54	0\\
68.55	0\\
68.56	0\\
68.57	0\\
68.58	0\\
68.59	0\\
68.6	0\\
68.61	0\\
68.62	0\\
68.63	0\\
68.64	0\\
68.65	0\\
68.66	0\\
68.67	0\\
68.68	0\\
68.69	0\\
68.7	0\\
68.71	0\\
68.72	0\\
68.73	0\\
68.74	0\\
68.75	0\\
68.76	0\\
68.77	0\\
68.78	0\\
68.79	0\\
68.8	0\\
68.81	0\\
68.82	0\\
68.83	0\\
68.84	0\\
68.85	0\\
68.86	0\\
68.87	0\\
68.88	0\\
68.89	0\\
68.9	0\\
68.91	0\\
68.92	0\\
68.93	0\\
68.94	0\\
68.95	0\\
68.96	0\\
68.97	0\\
68.98	0\\
68.99	0\\
69	0\\
69.01	0\\
69.02	0\\
69.03	0\\
69.04	0\\
69.05	0\\
69.06	0\\
69.07	0\\
69.08	0\\
69.09	0\\
69.1	0\\
69.11	0\\
69.12	0\\
69.13	0\\
69.14	0\\
69.15	0\\
69.16	0\\
69.17	0\\
69.18	0\\
69.19	0\\
69.2	0\\
69.21	0\\
69.22	0\\
69.23	0\\
69.24	0\\
69.25	0\\
69.26	0\\
69.27	0\\
69.28	0\\
69.29	0\\
69.3	0\\
69.31	0\\
69.32	0\\
69.33	0\\
69.34	0\\
69.35	0\\
69.36	0\\
69.37	0\\
69.38	0\\
69.39	0\\
69.4	0\\
69.41	0\\
69.42	0\\
69.43	0\\
69.44	0\\
69.45	0\\
69.46	0\\
69.47	0\\
69.48	0\\
69.49	0\\
69.5	0\\
69.51	0\\
69.52	0\\
69.53	0\\
69.54	0\\
69.55	0\\
69.56	0\\
69.57	0\\
69.58	0\\
69.59	0\\
69.6	0\\
69.61	0\\
69.62	0\\
69.63	0\\
69.64	0\\
69.65	0\\
69.66	0\\
69.67	0\\
69.68	0\\
69.69	0\\
69.7	0\\
69.71	0\\
69.72	0\\
69.73	0\\
69.74	0\\
69.75	0\\
69.76	0\\
69.77	0\\
69.78	0\\
69.79	0\\
69.8	0\\
69.81	0\\
69.82	0\\
69.83	0\\
69.84	0\\
69.85	0\\
69.86	0\\
69.87	0\\
69.88	0\\
69.89	0\\
69.9	0\\
69.91	0\\
69.92	0\\
69.93	0\\
69.94	0\\
69.95	0\\
69.96	0\\
69.97	0\\
69.98	0\\
69.99	0\\
70	0\\
70.01	0\\
70.02	0\\
70.03	0\\
70.04	0\\
70.05	0\\
70.06	0\\
70.07	0\\
70.08	0\\
70.09	0\\
70.1	0\\
70.11	0\\
70.12	0\\
70.13	0\\
70.14	0\\
70.15	0\\
70.16	0\\
70.17	0\\
70.18	0\\
70.19	0\\
70.2	0\\
70.21	0\\
70.22	0\\
70.23	0\\
70.24	0\\
70.25	0\\
70.26	0\\
70.27	0\\
70.28	0\\
70.29	0\\
70.3	0\\
70.31	0\\
70.32	0\\
70.33	0\\
70.34	0\\
70.35	0\\
70.36	0\\
70.37	0\\
70.38	0\\
70.39	0\\
70.4	0\\
70.41	0\\
70.42	0\\
70.43	0\\
70.44	0\\
70.45	0\\
70.46	0\\
70.47	0\\
70.48	0\\
70.49	0\\
70.5	0\\
70.51	0\\
70.52	0\\
70.53	0\\
70.54	0\\
70.55	0\\
70.56	0\\
70.57	0\\
70.58	0\\
70.59	0\\
70.6	0\\
70.61	0\\
70.62	0\\
70.63	0\\
70.64	0\\
70.65	0\\
70.66	0\\
70.67	0\\
70.68	0\\
70.69	0\\
70.7	0\\
70.71	0\\
70.72	0\\
70.73	0\\
70.74	0\\
70.75	0\\
70.76	0\\
70.77	0\\
70.78	0\\
70.79	0\\
70.8	0\\
70.81	0\\
70.82	0\\
70.83	0\\
70.84	0\\
70.85	0\\
70.86	0\\
70.87	0\\
70.88	0\\
70.89	0\\
70.9	0\\
70.91	0\\
70.92	0\\
70.93	0\\
70.94	0\\
70.95	0\\
70.96	0\\
70.97	0\\
70.98	0\\
70.99	0\\
71	0\\
71.01	0\\
71.02	0\\
71.03	0\\
71.04	0\\
71.05	0\\
71.06	0\\
71.07	0\\
71.08	0\\
71.09	0\\
71.1	0\\
71.11	0\\
71.12	0\\
71.13	0\\
71.14	0\\
71.15	0\\
71.16	0\\
71.17	0\\
71.18	0\\
71.19	0\\
71.2	0\\
71.21	0\\
71.22	0\\
71.23	0\\
71.24	0\\
71.25	0\\
71.26	0\\
71.27	0\\
71.28	0\\
71.29	0\\
71.3	0\\
71.31	0\\
71.32	0\\
71.33	0\\
71.34	0\\
71.35	0\\
71.36	0\\
71.37	0\\
71.38	0\\
71.39	0\\
71.4	0\\
71.41	0\\
71.42	0\\
71.43	0\\
71.44	0\\
71.45	0\\
71.46	0\\
71.47	0\\
71.48	0\\
71.49	0\\
71.5	0\\
71.51	0\\
71.52	0\\
71.53	0\\
71.54	0\\
71.55	0\\
71.56	0\\
71.57	0\\
71.58	0\\
71.59	0\\
71.6	0\\
71.61	0\\
71.62	0\\
71.63	0\\
71.64	0\\
71.65	0\\
71.66	0\\
71.67	0\\
71.68	0\\
71.69	0\\
71.7	0\\
71.71	0\\
71.72	0\\
71.73	0\\
71.74	0\\
71.75	0\\
71.76	0\\
71.77	0\\
71.78	0\\
71.79	0\\
71.8	0\\
71.81	0\\
71.82	0\\
71.83	0\\
71.84	0\\
71.85	0\\
71.86	0\\
71.87	0\\
71.88	0\\
71.89	0\\
71.9	0\\
71.91	0\\
71.92	0\\
71.93	0\\
71.94	0\\
71.95	0\\
71.96	0\\
71.97	0\\
71.98	0\\
71.99	0\\
72	0\\
72.01	0\\
72.02	0\\
72.03	0\\
72.04	0\\
72.05	0\\
72.06	0\\
72.07	0\\
72.08	0\\
72.09	0\\
72.1	0\\
72.11	0\\
72.12	0\\
72.13	0\\
72.14	0\\
72.15	0\\
72.16	0\\
72.17	0\\
72.18	0\\
72.19	0\\
72.2	0\\
72.21	0\\
72.22	0\\
72.23	0\\
72.24	0\\
72.25	0\\
72.26	0\\
72.27	0\\
72.28	0\\
72.29	0\\
72.3	0\\
72.31	0\\
72.32	0\\
72.33	0\\
72.34	0\\
72.35	0\\
72.36	0\\
72.37	0\\
72.38	0\\
72.39	0\\
72.4	0\\
72.41	0\\
72.42	0\\
72.43	0\\
72.44	0\\
72.45	0\\
72.46	0\\
72.47	0\\
72.48	0\\
72.49	0\\
72.5	0\\
72.51	0\\
72.52	0\\
72.53	0\\
72.54	0\\
72.55	0\\
72.56	0\\
72.57	0\\
72.58	0\\
72.59	0\\
72.6	0\\
72.61	0\\
72.62	0\\
72.63	0\\
72.64	0\\
72.65	0\\
72.66	0\\
72.67	0\\
72.68	0\\
72.69	0\\
72.7	0\\
72.71	0\\
72.72	0\\
72.73	0\\
72.74	0\\
72.75	0\\
72.76	0\\
72.77	0\\
72.78	0\\
72.79	0\\
72.8	0\\
72.81	0\\
72.82	0\\
72.83	0\\
72.84	0\\
72.85	0\\
72.86	0\\
72.87	0\\
72.88	0\\
72.89	0\\
72.9	0\\
72.91	0\\
72.92	0\\
72.93	0\\
72.94	0\\
72.95	0\\
72.96	0\\
72.97	0\\
72.98	0\\
72.99	0\\
73	0\\
73.01	0\\
73.02	0\\
73.03	0\\
73.04	0\\
73.05	0\\
73.06	0\\
73.07	0\\
73.08	0\\
73.09	0\\
73.1	0\\
73.11	0\\
73.12	0\\
73.13	0\\
73.14	0\\
73.15	0\\
73.16	0\\
73.17	0\\
73.18	0\\
73.19	0\\
73.2	0\\
73.21	0\\
73.22	0\\
73.23	0\\
73.24	0\\
73.25	0\\
73.26	0\\
73.27	0\\
73.28	0\\
73.29	0\\
73.3	0\\
73.31	0\\
73.32	0\\
73.33	0\\
73.34	0\\
73.35	0\\
73.36	0\\
73.37	0\\
73.38	0\\
73.39	0\\
73.4	0\\
73.41	0\\
73.42	0\\
73.43	0\\
73.44	0\\
73.45	0\\
73.46	0\\
73.47	0\\
73.48	0\\
73.49	0\\
73.5	0\\
73.51	0\\
73.52	0\\
73.53	0\\
73.54	0\\
73.55	0\\
73.56	0\\
73.57	0\\
73.58	0\\
73.59	0\\
73.6	0\\
73.61	0\\
73.62	0\\
73.63	0\\
73.64	0\\
73.65	0\\
73.66	0\\
73.67	0\\
73.68	0\\
73.69	0\\
73.7	0\\
73.71	0\\
73.72	0\\
73.73	0\\
73.74	0\\
73.75	0\\
73.76	0\\
73.77	0\\
73.78	0\\
73.79	0\\
73.8	0\\
73.81	0\\
73.82	0\\
73.83	0\\
73.84	0\\
73.85	0\\
73.86	0\\
73.87	0\\
73.88	0\\
73.89	0\\
73.9	0\\
73.91	0\\
73.92	0\\
73.93	0\\
73.94	0\\
73.95	0\\
73.96	0\\
73.97	0\\
73.98	0\\
73.99	0\\
74	0\\
74.01	0\\
74.02	0\\
74.03	0\\
74.04	0\\
74.05	0\\
74.06	0\\
74.07	0\\
74.08	0\\
74.09	0\\
74.1	0\\
74.11	0\\
74.12	0\\
74.13	0\\
74.14	0\\
74.15	0\\
74.16	0\\
74.17	0\\
74.18	0\\
74.19	0\\
74.2	0\\
74.21	0\\
74.22	0\\
74.23	0\\
74.24	0\\
74.25	0\\
74.26	0\\
74.27	0\\
74.28	0\\
74.29	0\\
74.3	0\\
74.31	0\\
74.32	0\\
74.33	0\\
74.34	0\\
74.35	0\\
74.36	0\\
74.37	0\\
74.38	0\\
74.39	0\\
74.4	0\\
74.41	0\\
74.42	0\\
74.43	0\\
74.44	0\\
74.45	0\\
74.46	0\\
74.47	0\\
74.48	0\\
74.49	0\\
74.5	0\\
74.51	0\\
74.52	0\\
74.53	0\\
74.54	0\\
74.55	0\\
74.56	0\\
74.57	0\\
74.58	0\\
74.59	0\\
74.6	0\\
74.61	0\\
74.62	0\\
74.63	0\\
74.64	0\\
74.65	0\\
74.66	0\\
74.67	0\\
74.68	0\\
74.69	0\\
74.7	0\\
74.71	0\\
74.72	0\\
74.73	0\\
74.74	0\\
74.75	0\\
74.76	0\\
74.77	0\\
74.78	0\\
74.79	0\\
74.8	0\\
74.81	0\\
74.82	0\\
74.83	0\\
74.84	0\\
74.85	0\\
74.86	0\\
74.87	0\\
74.88	0\\
74.89	0\\
74.9	0\\
74.91	0\\
74.92	0\\
74.93	0\\
74.94	0\\
74.95	0\\
74.96	0\\
74.97	0\\
74.98	0\\
74.99	0\\
75	0\\
75.01	0\\
75.02	0\\
75.03	0\\
75.04	0\\
75.05	0\\
75.06	0\\
75.07	0\\
75.08	0\\
75.09	0\\
75.1	0\\
75.11	0\\
75.12	0\\
75.13	0\\
75.14	0\\
75.15	0\\
75.16	0\\
75.17	0\\
75.18	0\\
75.19	0\\
75.2	0\\
75.21	0\\
75.22	0\\
75.23	0\\
75.24	0\\
75.25	0\\
75.26	0\\
75.27	0\\
75.28	0\\
75.29	0\\
75.3	0\\
75.31	0\\
75.32	0\\
75.33	0\\
75.34	0\\
75.35	0\\
75.36	0\\
75.37	0\\
75.38	0\\
75.39	0\\
75.4	0\\
75.41	0\\
75.42	0\\
75.43	0\\
75.44	0\\
75.45	0\\
75.46	0\\
75.47	0\\
75.48	0\\
75.49	0\\
75.5	0\\
75.51	0\\
75.52	0\\
75.53	0\\
75.54	0\\
75.55	0\\
75.56	0\\
75.57	0\\
75.58	0\\
75.59	0\\
75.6	0\\
75.61	0\\
75.62	0\\
75.63	0\\
75.64	0\\
75.65	0\\
75.66	0\\
75.67	0\\
75.68	0\\
75.69	0\\
75.7	0\\
75.71	0\\
75.72	0\\
75.73	0\\
75.74	0\\
75.75	0\\
75.76	0\\
75.77	0\\
75.78	0\\
75.79	0\\
75.8	0\\
75.81	0\\
75.82	0\\
75.83	0\\
75.84	0\\
75.85	0\\
75.86	0\\
75.87	0\\
75.88	0\\
75.89	0\\
75.9	0\\
75.91	0\\
75.92	0\\
75.93	0\\
75.94	0\\
75.95	0\\
75.96	0\\
75.97	0\\
75.98	0\\
75.99	0\\
76	0\\
76.01	0\\
76.02	0\\
76.03	0\\
76.04	0\\
76.05	0\\
76.06	0\\
76.07	0\\
76.08	0\\
76.09	0\\
76.1	0\\
76.11	0\\
76.12	0\\
76.13	0\\
76.14	0\\
76.15	0\\
76.16	0\\
76.17	0\\
76.18	0\\
76.19	0\\
76.2	0\\
76.21	0\\
76.22	0\\
76.23	0\\
76.24	0\\
76.25	0\\
76.26	0\\
76.27	0\\
76.28	0\\
76.29	0\\
76.3	0\\
76.31	0\\
76.32	0\\
76.33	0\\
76.34	0\\
76.35	0\\
76.36	0\\
76.37	0\\
76.38	0\\
76.39	0\\
76.4	0\\
76.41	0\\
76.42	0\\
76.43	0\\
76.44	0\\
76.45	0\\
76.46	0\\
76.47	0\\
76.48	0\\
76.49	0\\
76.5	0\\
76.51	0\\
76.52	0\\
76.53	0\\
76.54	0\\
76.55	0\\
76.56	0\\
76.57	0\\
76.58	0\\
76.59	0\\
76.6	0\\
76.61	0\\
76.62	0\\
76.63	0\\
76.64	0\\
76.65	0\\
76.66	0\\
76.67	0\\
76.68	0\\
76.69	0\\
76.7	0\\
76.71	0\\
76.72	0\\
76.73	0\\
76.74	0\\
76.75	0\\
76.76	0\\
76.77	0\\
76.78	0\\
76.79	0\\
76.8	0\\
76.81	0\\
76.82	0\\
76.83	0\\
76.84	0\\
76.85	0\\
76.86	0\\
76.87	0\\
76.88	0\\
76.89	0\\
76.9	0\\
76.91	0\\
76.92	0\\
76.93	0\\
76.94	0\\
76.95	0\\
76.96	0\\
76.97	0\\
76.98	0\\
76.99	0\\
77	0\\
77.01	0\\
77.02	0\\
77.03	0\\
77.04	0\\
77.05	0\\
77.06	0\\
77.07	0\\
77.08	0\\
77.09	0\\
77.1	0\\
77.11	0\\
77.12	0\\
77.13	0\\
77.14	0\\
77.15	0\\
77.16	0\\
77.17	0\\
77.18	0\\
77.19	0\\
77.2	0\\
77.21	0\\
77.22	0\\
77.23	0\\
77.24	0\\
77.25	0\\
77.26	0\\
77.27	0\\
77.28	0\\
77.29	0\\
77.3	0\\
77.31	0\\
77.32	0\\
77.33	0\\
77.34	0\\
77.35	0\\
77.36	0\\
77.37	0\\
77.38	0\\
77.39	0\\
77.4	0\\
77.41	0\\
77.42	0\\
77.43	0\\
77.44	0\\
77.45	0\\
77.46	0\\
77.47	0\\
77.48	0\\
77.49	0\\
77.5	0\\
77.51	0\\
77.52	0\\
77.53	0\\
77.54	0\\
77.55	0\\
77.56	0\\
77.57	0\\
77.58	0\\
77.59	0\\
77.6	0\\
77.61	0\\
77.62	0\\
77.63	0\\
77.64	0\\
77.65	0\\
77.66	0\\
77.67	0\\
77.68	0\\
77.69	0\\
77.7	0\\
77.71	0\\
77.72	0\\
77.73	0\\
77.74	0\\
77.75	0\\
77.76	0\\
77.77	0\\
77.78	0\\
77.79	0\\
77.8	0\\
77.81	0\\
77.82	0\\
77.83	0\\
77.84	0\\
77.85	0\\
77.86	0\\
77.87	0\\
77.88	0\\
77.89	0\\
77.9	0\\
77.91	0\\
77.92	0\\
77.93	0\\
77.94	0\\
77.95	0\\
77.96	0\\
77.97	0\\
77.98	0\\
77.99	0\\
78	0\\
78.01	0\\
78.02	0\\
78.03	0\\
78.04	0\\
78.05	0\\
78.06	0\\
78.07	0\\
78.08	0\\
78.09	0\\
78.1	0\\
78.11	0\\
78.12	0\\
78.13	0\\
78.14	0\\
78.15	0\\
78.16	0\\
78.17	0\\
78.18	0\\
78.19	0\\
78.2	0\\
78.21	0\\
78.22	0\\
78.23	0\\
78.24	0\\
78.25	0\\
78.26	0\\
78.27	0\\
78.28	0\\
78.29	0\\
78.3	0\\
78.31	0\\
78.32	0\\
78.33	0\\
78.34	0\\
78.35	0\\
78.36	0\\
78.37	0\\
78.38	0\\
78.39	0\\
78.4	0\\
78.41	0\\
78.42	0\\
78.43	0\\
78.44	0\\
78.45	0\\
78.46	0\\
78.47	0\\
78.48	0\\
78.49	0\\
78.5	0\\
78.51	0\\
78.52	0\\
78.53	0\\
78.54	0\\
78.55	0\\
78.56	0\\
78.57	0\\
78.58	0\\
78.59	0\\
78.6	0\\
78.61	0\\
78.62	0\\
78.63	0\\
78.64	0\\
78.65	0\\
78.66	0\\
78.67	0\\
78.68	0\\
78.69	0\\
78.7	0\\
78.71	0\\
78.72	0\\
78.73	0\\
78.74	0\\
78.75	0\\
78.76	0\\
78.77	0\\
78.78	0\\
78.79	0\\
78.8	0\\
78.81	0\\
78.82	0\\
78.83	0\\
78.84	0\\
78.85	0\\
78.86	0\\
78.87	0\\
78.88	0\\
78.89	0\\
78.9	0\\
78.91	0\\
78.92	0\\
78.93	0\\
78.94	0\\
78.95	0\\
78.96	0\\
78.97	0\\
78.98	0\\
78.99	0\\
79	0\\
79.01	0\\
79.02	0\\
79.03	0\\
79.04	0\\
79.05	0\\
79.06	0\\
79.07	0\\
79.08	0\\
79.09	0\\
79.1	0\\
79.11	0\\
79.12	0\\
79.13	0\\
79.14	0\\
79.15	0\\
79.16	0\\
79.17	0\\
79.18	0\\
79.19	0\\
79.2	0\\
79.21	0\\
79.22	0\\
79.23	0\\
79.24	0\\
79.25	0\\
79.26	0\\
79.27	0\\
79.28	0\\
79.29	0\\
79.3	0\\
79.31	0\\
79.32	0\\
79.33	0\\
79.34	0\\
79.35	0\\
79.36	0\\
79.37	0\\
79.38	0\\
79.39	0\\
79.4	0\\
79.41	0\\
79.42	0\\
79.43	0\\
79.44	0\\
79.45	0\\
79.46	0\\
79.47	0\\
79.48	0\\
79.49	0\\
79.5	0\\
79.51	0\\
79.52	0\\
79.53	0\\
79.54	0\\
79.55	0\\
79.56	0\\
79.57	0\\
79.58	0\\
79.59	0\\
79.6	0\\
79.61	0\\
79.62	0\\
79.63	0\\
79.64	0\\
79.65	0\\
79.66	0\\
79.67	0\\
79.68	0\\
79.69	0\\
79.7	0\\
79.71	0\\
79.72	0\\
79.73	0\\
79.74	0\\
79.75	0\\
79.76	0\\
79.77	0\\
79.78	0\\
79.79	0\\
79.8	0\\
79.81	0\\
79.82	0\\
79.83	0\\
79.84	0\\
79.85	0\\
79.86	0\\
79.87	0\\
79.88	0\\
79.89	0\\
79.9	0\\
79.91	0\\
79.92	0\\
79.93	0\\
79.94	0\\
79.95	0\\
79.96	0\\
79.97	0\\
79.98	0\\
79.99	0\\
80	0\\
80.01	0\\
};
\addplot [color=red,solid]
  table[row sep=crcr]{%
80.01	0\\
80.02	0\\
80.03	0\\
80.04	0\\
80.05	0\\
80.06	0\\
80.07	0\\
80.08	0\\
80.09	0\\
80.1	0\\
80.11	0\\
80.12	0\\
80.13	0\\
80.14	0\\
80.15	0\\
80.16	0\\
80.17	0\\
80.18	0\\
80.19	0\\
80.2	0\\
80.21	0\\
80.22	0\\
80.23	0\\
80.24	0\\
80.25	0\\
80.26	0\\
80.27	0\\
80.28	0\\
80.29	0\\
80.3	0\\
80.31	0\\
80.32	0\\
80.33	0\\
80.34	0\\
80.35	0\\
80.36	0\\
80.37	0\\
80.38	0\\
80.39	0\\
80.4	0\\
80.41	0\\
80.42	0\\
80.43	0\\
80.44	0\\
80.45	0\\
80.46	0\\
80.47	0\\
80.48	0\\
80.49	0\\
80.5	0\\
80.51	0\\
80.52	0\\
80.53	0\\
80.54	0\\
80.55	0\\
80.56	0\\
80.57	0\\
80.58	0\\
80.59	0\\
80.6	0\\
80.61	0\\
80.62	0\\
80.63	0\\
80.64	0\\
80.65	0\\
80.66	0\\
80.67	0\\
80.68	0\\
80.69	0\\
80.7	0\\
80.71	0\\
80.72	0\\
80.73	0\\
80.74	0\\
80.75	0\\
80.76	0\\
80.77	0\\
80.78	0\\
80.79	0\\
80.8	0\\
80.81	0\\
80.82	0\\
80.83	0\\
80.84	0\\
80.85	0\\
80.86	0\\
80.87	0\\
80.88	0\\
80.89	0\\
80.9	0\\
80.91	0\\
80.92	0\\
80.93	0\\
80.94	0\\
80.95	0\\
80.96	0\\
80.97	0\\
80.98	0\\
80.99	0\\
81	0\\
81.01	0\\
81.02	0\\
81.03	0\\
81.04	0\\
81.05	0\\
81.06	0\\
81.07	0\\
81.08	0\\
81.09	0\\
81.1	0\\
81.11	0\\
81.12	0\\
81.13	0\\
81.14	0\\
81.15	0\\
81.16	0\\
81.17	0\\
81.18	0\\
81.19	0\\
81.2	0\\
81.21	0\\
81.22	0\\
81.23	0\\
81.24	0\\
81.25	0\\
81.26	0\\
81.27	0\\
81.28	0\\
81.29	0\\
81.3	0\\
81.31	0\\
81.32	0\\
81.33	0\\
81.34	0\\
81.35	0\\
81.36	0\\
81.37	0\\
81.38	0\\
81.39	0\\
81.4	0\\
81.41	0\\
81.42	0\\
81.43	0\\
81.44	0\\
81.45	0\\
81.46	0\\
81.47	0\\
81.48	0\\
81.49	0\\
81.5	0\\
81.51	0\\
81.52	0\\
81.53	0\\
81.54	0\\
81.55	0\\
81.56	0\\
81.57	0\\
81.58	0\\
81.59	0\\
81.6	0\\
81.61	0\\
81.62	0\\
81.63	0\\
81.64	0\\
81.65	0\\
81.66	0\\
81.67	0\\
81.68	0\\
81.69	0\\
81.7	0\\
81.71	0\\
81.72	0\\
81.73	0\\
81.74	0\\
81.75	0\\
81.76	0\\
81.77	0\\
81.78	0\\
81.79	0\\
81.8	0\\
81.81	0\\
81.82	0\\
81.83	0\\
81.84	0\\
81.85	0\\
81.86	0\\
81.87	0\\
81.88	0\\
81.89	0\\
81.9	0\\
81.91	0\\
81.92	0\\
81.93	0\\
81.94	0\\
81.95	0\\
81.96	0\\
81.97	0\\
81.98	0\\
81.99	0\\
82	0\\
82.01	0\\
82.02	0\\
82.03	0\\
82.04	0\\
82.05	0\\
82.06	0\\
82.07	0\\
82.08	0\\
82.09	0\\
82.1	0\\
82.11	0\\
82.12	0\\
82.13	0\\
82.14	0\\
82.15	0\\
82.16	0\\
82.17	0\\
82.18	0\\
82.19	0\\
82.2	0\\
82.21	0\\
82.22	0\\
82.23	0\\
82.24	0\\
82.25	0\\
82.26	0\\
82.27	0\\
82.28	0\\
82.29	0\\
82.3	0\\
82.31	0\\
82.32	0\\
82.33	0\\
82.34	0\\
82.35	0\\
82.36	0\\
82.37	0\\
82.38	0\\
82.39	0\\
82.4	0\\
82.41	0\\
82.42	0\\
82.43	0\\
82.44	0\\
82.45	0\\
82.46	0\\
82.47	0\\
82.48	0\\
82.49	0\\
82.5	0\\
82.51	0\\
82.52	0\\
82.53	0\\
82.54	0\\
82.55	0\\
82.56	0\\
82.57	0\\
82.58	0\\
82.59	0\\
82.6	0\\
82.61	0\\
82.62	0\\
82.63	0\\
82.64	0\\
82.65	0\\
82.66	0\\
82.67	0\\
82.68	0\\
82.69	0\\
82.7	0\\
82.71	0\\
82.72	0\\
82.73	0\\
82.74	0\\
82.75	0\\
82.76	0\\
82.77	0\\
82.78	0\\
82.79	0\\
82.8	0\\
82.81	0\\
82.82	0\\
82.83	0\\
82.84	0\\
82.85	0\\
82.86	0\\
82.87	0\\
82.88	0\\
82.89	0\\
82.9	0\\
82.91	0\\
82.92	0\\
82.93	0\\
82.94	0\\
82.95	0\\
82.96	0\\
82.97	0\\
82.98	0\\
82.99	0\\
83	0\\
83.01	0\\
83.02	0\\
83.03	0\\
83.04	0\\
83.05	0\\
83.06	0\\
83.07	0\\
83.08	0\\
83.09	0\\
83.1	0\\
83.11	0\\
83.12	0\\
83.13	0\\
83.14	0\\
83.15	0\\
83.16	0\\
83.17	0\\
83.18	0\\
83.19	0\\
83.2	0\\
83.21	0\\
83.22	0\\
83.23	0\\
83.24	0\\
83.25	0\\
83.26	0\\
83.27	0\\
83.28	0\\
83.29	0\\
83.3	0\\
83.31	0\\
83.32	0\\
83.33	0\\
83.34	0\\
83.35	0\\
83.36	0\\
83.37	0\\
83.38	0\\
83.39	0\\
83.4	0\\
83.41	0\\
83.42	0\\
83.43	0\\
83.44	0\\
83.45	0\\
83.46	0\\
83.47	0\\
83.48	0\\
83.49	0\\
83.5	0\\
83.51	0\\
83.52	0\\
83.53	0\\
83.54	0\\
83.55	0\\
83.56	0\\
83.57	0\\
83.58	0\\
83.59	0\\
83.6	0\\
83.61	0\\
83.62	0\\
83.63	0\\
83.64	0\\
83.65	0\\
83.66	0\\
83.67	0\\
83.68	0\\
83.69	0\\
83.7	0\\
83.71	0\\
83.72	0\\
83.73	0\\
83.74	0\\
83.75	0\\
83.76	0\\
83.77	0\\
83.78	0\\
83.79	0\\
83.8	0\\
83.81	0\\
83.82	0\\
83.83	0\\
83.84	0\\
83.85	0\\
83.86	0\\
83.87	0\\
83.88	0\\
83.89	0\\
83.9	0\\
83.91	0\\
83.92	0\\
83.93	0\\
83.94	0\\
83.95	0\\
83.96	0\\
83.97	0\\
83.98	0\\
83.99	0\\
84	0\\
84.01	0\\
84.02	0\\
84.03	0\\
84.04	0\\
84.05	0\\
84.06	0\\
84.07	0\\
84.08	0\\
84.09	0\\
84.1	0\\
84.11	0\\
84.12	0\\
84.13	0\\
84.14	0\\
84.15	0\\
84.16	0\\
84.17	0\\
84.18	0\\
84.19	0\\
84.2	0\\
84.21	0\\
84.22	0\\
84.23	0\\
84.24	0\\
84.25	0\\
84.26	0\\
84.27	0\\
84.28	0\\
84.29	0\\
84.3	0\\
84.31	0\\
84.32	0\\
84.33	0\\
84.34	0\\
84.35	0\\
84.36	0\\
84.37	0\\
84.38	0\\
84.39	0\\
84.4	0\\
84.41	0\\
84.42	0\\
84.43	0\\
84.44	0\\
84.45	0\\
84.46	0\\
84.47	0\\
84.48	0\\
84.49	0\\
84.5	0\\
84.51	0\\
84.52	0\\
84.53	0\\
84.54	0\\
84.55	0\\
84.56	0\\
84.57	0\\
84.58	0\\
84.59	0\\
84.6	0\\
84.61	0\\
84.62	0\\
84.63	0\\
84.64	0\\
84.65	0\\
84.66	0\\
84.67	0\\
84.68	0\\
84.69	0\\
84.7	0\\
84.71	0\\
84.72	0\\
84.73	0\\
84.74	0\\
84.75	0\\
84.76	0\\
84.77	0\\
84.78	0\\
84.79	0\\
84.8	0\\
84.81	0\\
84.82	0\\
84.83	0\\
84.84	0\\
84.85	0\\
84.86	0\\
84.87	0\\
84.88	0\\
84.89	0\\
84.9	0\\
84.91	0\\
84.92	0\\
84.93	0\\
84.94	0\\
84.95	0\\
84.96	0\\
84.97	0\\
84.98	0\\
84.99	0\\
85	0\\
85.01	0\\
85.02	0\\
85.03	0\\
85.04	0\\
85.05	0\\
85.06	0\\
85.07	0\\
85.08	0\\
85.09	0\\
85.1	0\\
85.11	0\\
85.12	0\\
85.13	0\\
85.14	0\\
85.15	0\\
85.16	0\\
85.17	0\\
85.18	0\\
85.19	0\\
85.2	0\\
85.21	0\\
85.22	0\\
85.23	0\\
85.24	0\\
85.25	0\\
85.26	0\\
85.27	0\\
85.28	0\\
85.29	0\\
85.3	0\\
85.31	0\\
85.32	0\\
85.33	0\\
85.34	0\\
85.35	0\\
85.36	0\\
85.37	0\\
85.38	0\\
85.39	0\\
85.4	0\\
85.41	0\\
85.42	0\\
85.43	0\\
85.44	0\\
85.45	0\\
85.46	0\\
85.47	0\\
85.48	0\\
85.49	0\\
85.5	0\\
85.51	0\\
85.52	0\\
85.53	0\\
85.54	0\\
85.55	0\\
85.56	0\\
85.57	0\\
85.58	0\\
85.59	0\\
85.6	0\\
85.61	0\\
85.62	0\\
85.63	0\\
85.64	0\\
85.65	0\\
85.66	0\\
85.67	0\\
85.68	0\\
85.69	0\\
85.7	0\\
85.71	0\\
85.72	0\\
85.73	0\\
85.74	0\\
85.75	0\\
85.76	0\\
85.77	0\\
85.78	0\\
85.79	0\\
85.8	0\\
85.81	0\\
85.82	0\\
85.83	0\\
85.84	0\\
85.85	0\\
85.86	0\\
85.87	0\\
85.88	0\\
85.89	0\\
85.9	0\\
85.91	0\\
85.92	0\\
85.93	0\\
85.94	0\\
85.95	0\\
85.96	0\\
85.97	0\\
85.98	0\\
85.99	0\\
86	0\\
86.01	0\\
86.02	0\\
86.03	0\\
86.04	0\\
86.05	0\\
86.06	0\\
86.07	0\\
86.08	0\\
86.09	0\\
86.1	0\\
86.11	0\\
86.12	0\\
86.13	0\\
86.14	0\\
86.15	0\\
86.16	0\\
86.17	0\\
86.18	0\\
86.19	0\\
86.2	0\\
86.21	0\\
86.22	0\\
86.23	0\\
86.24	0\\
86.25	0\\
86.26	0\\
86.27	0\\
86.28	0\\
86.29	0\\
86.3	0\\
86.31	0\\
86.32	0\\
86.33	0\\
86.34	0\\
86.35	0\\
86.36	0\\
86.37	0\\
86.38	0\\
86.39	0\\
86.4	0\\
86.41	0\\
86.42	0\\
86.43	0\\
86.44	0\\
86.45	0\\
86.46	0\\
86.47	0\\
86.48	0\\
86.49	0\\
86.5	0\\
86.51	0\\
86.52	0\\
86.53	0\\
86.54	0\\
86.55	0\\
86.56	0\\
86.57	0\\
86.58	0\\
86.59	0\\
86.6	0\\
86.61	0\\
86.62	0\\
86.63	0\\
86.64	0\\
86.65	0\\
86.66	0\\
86.67	0\\
86.68	0\\
86.69	0\\
86.7	0\\
86.71	0\\
86.72	0\\
86.73	0\\
86.74	0\\
86.75	0\\
86.76	0\\
86.77	0\\
86.78	0\\
86.79	0\\
86.8	0\\
86.81	0\\
86.82	0\\
86.83	0\\
86.84	0\\
86.85	0\\
86.86	0\\
86.87	0\\
86.88	0\\
86.89	0\\
86.9	0\\
86.91	0\\
86.92	0\\
86.93	0\\
86.94	0\\
86.95	0\\
86.96	0\\
86.97	0\\
86.98	0\\
86.99	0\\
87	0\\
87.01	0\\
87.02	0\\
87.03	0\\
87.04	0\\
87.05	0\\
87.06	0\\
87.07	0\\
87.08	0\\
87.09	0\\
87.1	0\\
87.11	0\\
87.12	0\\
87.13	0\\
87.14	0\\
87.15	0\\
87.16	0\\
87.17	0\\
87.18	0\\
87.19	0\\
87.2	0\\
87.21	0\\
87.22	0\\
87.23	0\\
87.24	0\\
87.25	0\\
87.26	0\\
87.27	0\\
87.28	0\\
87.29	0\\
87.3	0\\
87.31	0\\
87.32	0\\
87.33	0\\
87.34	0\\
87.35	0\\
87.36	0\\
87.37	0\\
87.38	0\\
87.39	0\\
87.4	0\\
87.41	0\\
87.42	0\\
87.43	0\\
87.44	0\\
87.45	0\\
87.46	0\\
87.47	0\\
87.48	0\\
87.49	0\\
87.5	0\\
87.51	0\\
87.52	0\\
87.53	0\\
87.54	0\\
87.55	0\\
87.56	0\\
87.57	0\\
87.58	0\\
87.59	0\\
87.6	0\\
87.61	0\\
87.62	0\\
87.63	0\\
87.64	0\\
87.65	0\\
87.66	0\\
87.67	0\\
87.68	0\\
87.69	0\\
87.7	0\\
87.71	0\\
87.72	0\\
87.73	0\\
87.74	0\\
87.75	0\\
87.76	0\\
87.77	0\\
87.78	0\\
87.79	0\\
87.8	0\\
87.81	0\\
87.82	0\\
87.83	0\\
87.84	0\\
87.85	0\\
87.86	0\\
87.87	0\\
87.88	0\\
87.89	0\\
87.9	0\\
87.91	0\\
87.92	0\\
87.93	0\\
87.94	0\\
87.95	0\\
87.96	0\\
87.97	0\\
87.98	0\\
87.99	0\\
88	0\\
88.01	0\\
88.02	0\\
88.03	0\\
88.04	0\\
88.05	0\\
88.06	0\\
88.07	0\\
88.08	0\\
88.09	0\\
88.1	0\\
88.11	0\\
88.12	0\\
88.13	0\\
88.14	0\\
88.15	0\\
88.16	0\\
88.17	0\\
88.18	0\\
88.19	0\\
88.2	0\\
88.21	0\\
88.22	0\\
88.23	0\\
88.24	0\\
88.25	0\\
88.26	0\\
88.27	0\\
88.28	0\\
88.29	0\\
88.3	0\\
88.31	0\\
88.32	0\\
88.33	0\\
88.34	0\\
88.35	0\\
88.36	0\\
88.37	0\\
88.38	0\\
88.39	0\\
88.4	0\\
88.41	0\\
88.42	0\\
88.43	0\\
88.44	0\\
88.45	0\\
88.46	0\\
88.47	0\\
88.48	0\\
88.49	0\\
88.5	0\\
88.51	0\\
88.52	0\\
88.53	0\\
88.54	0\\
88.55	0\\
88.56	0\\
88.57	0\\
88.58	0\\
88.59	0\\
88.6	0\\
88.61	0\\
88.62	0\\
88.63	0\\
88.64	0\\
88.65	0\\
88.66	0\\
88.67	0\\
88.68	0\\
88.69	0\\
88.7	0\\
88.71	0\\
88.72	0\\
88.73	0\\
88.74	0\\
88.75	0\\
88.76	0\\
88.77	0\\
88.78	0\\
88.79	0\\
88.8	0\\
88.81	0\\
88.82	0\\
88.83	0\\
88.84	0\\
88.85	0\\
88.86	0\\
88.87	0\\
88.88	0\\
88.89	0\\
88.9	0\\
88.91	0\\
88.92	0\\
88.93	0\\
88.94	0\\
88.95	0\\
88.96	0\\
88.97	0\\
88.98	0\\
88.99	0\\
89	0\\
89.01	0\\
89.02	0\\
89.03	0\\
89.04	0\\
89.05	0\\
89.06	0\\
89.07	0\\
89.08	0\\
89.09	0\\
89.1	0\\
89.11	0\\
89.12	0\\
89.13	0\\
89.14	0\\
89.15	0\\
89.16	0\\
89.17	0\\
89.18	0\\
89.19	0\\
89.2	0\\
89.21	0\\
89.22	0\\
89.23	0\\
89.24	0\\
89.25	0\\
89.26	0\\
89.27	0\\
89.28	0\\
89.29	0\\
89.3	0\\
89.31	0\\
89.32	0\\
89.33	0\\
89.34	0\\
89.35	0\\
89.36	0\\
89.37	0\\
89.38	0\\
89.39	0\\
89.4	0\\
89.41	0\\
89.42	0\\
89.43	0\\
89.44	0\\
89.45	0\\
89.46	0\\
89.47	0\\
89.48	0\\
89.49	0\\
89.5	0\\
89.51	0\\
89.52	0\\
89.53	0\\
89.54	0\\
89.55	0\\
89.56	0\\
89.57	0\\
89.58	0\\
89.59	0\\
89.6	0\\
89.61	0\\
89.62	0\\
89.63	0\\
89.64	0\\
89.65	0\\
89.66	0\\
89.67	0\\
89.68	0\\
89.69	0\\
89.7	0\\
89.71	0\\
89.72	0\\
89.73	0\\
89.74	0\\
89.75	0\\
89.76	0\\
89.77	0\\
89.78	0\\
89.79	0\\
89.8	0\\
89.81	0\\
89.82	0\\
89.83	0\\
89.84	0\\
89.85	0\\
89.86	0\\
89.87	0\\
89.88	0\\
89.89	0\\
89.9	0\\
89.91	0\\
89.92	0\\
89.93	0\\
89.94	0\\
89.95	0\\
89.96	0\\
89.97	0\\
89.98	0\\
89.99	0\\
90	0\\
90.01	0\\
90.02	0\\
90.03	0\\
90.04	0\\
90.05	0\\
90.06	0\\
90.07	0\\
90.08	0\\
90.09	0\\
90.1	0\\
90.11	0\\
90.12	0\\
90.13	0\\
90.14	0\\
90.15	0\\
90.16	0\\
90.17	0\\
90.18	0\\
90.19	0\\
90.2	0\\
90.21	0\\
90.22	0\\
90.23	0\\
90.24	0\\
90.25	0\\
90.26	0\\
90.27	0\\
90.28	0\\
90.29	0\\
90.3	0\\
90.31	0\\
90.32	0\\
90.33	0\\
90.34	0\\
90.35	0\\
90.36	0\\
90.37	0\\
90.38	0\\
90.39	0\\
90.4	0\\
90.41	0\\
90.42	0\\
90.43	0\\
90.44	0\\
90.45	0\\
90.46	0\\
90.47	0\\
90.48	0\\
90.49	0\\
90.5	0\\
90.51	0\\
90.52	0\\
90.53	0\\
90.54	0\\
90.55	0\\
90.56	0\\
90.57	0\\
90.58	0\\
90.59	0\\
90.6	0\\
90.61	0\\
90.62	0\\
90.63	0\\
90.64	0\\
90.65	0\\
90.66	0\\
90.67	0\\
90.68	0\\
90.69	0\\
90.7	0\\
90.71	0\\
90.72	0\\
90.73	0\\
90.74	0\\
90.75	0\\
90.76	0\\
90.77	0\\
90.78	0\\
90.79	0\\
90.8	0\\
90.81	0\\
90.82	0\\
90.83	0\\
90.84	0\\
90.85	0\\
90.86	0\\
90.87	0\\
90.88	0\\
90.89	0\\
90.9	0\\
90.91	0\\
90.92	0\\
90.93	0\\
90.94	0\\
90.95	0\\
90.96	0\\
90.97	0\\
90.98	0\\
90.99	0\\
91	0\\
91.01	0\\
91.02	0\\
91.03	0\\
91.04	0\\
91.05	0\\
91.06	0\\
91.07	0\\
91.08	0\\
91.09	0\\
91.1	0\\
91.11	0\\
91.12	0\\
91.13	0\\
91.14	0\\
91.15	0\\
91.16	0\\
91.17	0\\
91.18	0\\
91.19	0\\
91.2	0\\
91.21	0\\
91.22	0\\
91.23	0\\
91.24	0\\
91.25	0\\
91.26	0\\
91.27	0\\
91.28	0\\
91.29	0\\
91.3	0\\
91.31	0\\
91.32	0\\
91.33	0\\
91.34	0\\
91.35	0\\
91.36	0\\
91.37	0\\
91.38	0\\
91.39	0\\
91.4	0\\
91.41	0\\
91.42	0\\
91.43	0\\
91.44	0\\
91.45	0\\
91.46	0\\
91.47	0\\
91.48	0\\
91.49	0\\
91.5	0\\
91.51	0\\
91.52	0\\
91.53	0\\
91.54	0\\
91.55	0\\
91.56	0\\
91.57	0\\
91.58	0\\
91.59	0\\
91.6	0\\
91.61	0\\
91.62	0\\
91.63	0\\
91.64	0\\
91.65	0\\
91.66	0\\
91.67	0\\
91.68	0\\
91.69	0\\
91.7	0\\
91.71	0\\
91.72	0\\
91.73	0\\
91.74	0\\
91.75	0\\
91.76	0\\
91.77	0\\
91.78	0\\
91.79	0\\
91.8	0\\
91.81	0\\
91.82	0\\
91.83	0\\
91.84	0\\
91.85	0\\
91.86	0\\
91.87	0\\
91.88	0\\
91.89	0\\
91.9	0\\
91.91	0\\
91.92	0\\
91.93	0\\
91.94	0\\
91.95	0\\
91.96	0\\
91.97	0\\
91.98	0\\
91.99	0\\
92	0\\
92.01	0\\
92.02	0\\
92.03	0\\
92.04	0\\
92.05	0\\
92.06	0\\
92.07	0\\
92.08	0\\
92.09	0\\
92.1	0\\
92.11	0\\
92.12	0\\
92.13	0\\
92.14	0\\
92.15	0\\
92.16	0\\
92.17	0\\
92.18	0\\
92.19	0\\
92.2	0\\
92.21	0\\
92.22	0\\
92.23	0\\
92.24	0\\
92.25	0\\
92.26	0\\
92.27	0\\
92.28	0\\
92.29	0\\
92.3	0\\
92.31	0\\
92.32	0\\
92.33	0\\
92.34	0\\
92.35	0\\
92.36	0\\
92.37	0\\
92.38	0\\
92.39	0\\
92.4	0\\
92.41	0\\
92.42	0\\
92.43	0\\
92.44	0\\
92.45	0\\
92.46	0\\
92.47	0\\
92.48	0\\
92.49	0\\
92.5	0\\
92.51	0\\
92.52	0\\
92.53	0\\
92.54	0\\
92.55	0\\
92.56	0\\
92.57	0\\
92.58	0\\
92.59	0\\
92.6	0\\
92.61	0\\
92.62	0\\
92.63	0\\
92.64	0\\
92.65	0\\
92.66	0\\
92.67	0\\
92.68	0\\
92.69	0\\
92.7	0\\
92.71	0\\
92.72	0\\
92.73	0\\
92.74	0\\
92.75	0\\
92.76	0\\
92.77	0\\
92.78	0\\
92.79	0\\
92.8	0\\
92.81	0\\
92.82	0\\
92.83	0\\
92.84	0\\
92.85	0\\
92.86	0\\
92.87	0\\
92.88	0\\
92.89	0\\
92.9	0\\
92.91	0\\
92.92	0\\
92.93	0\\
92.94	0\\
92.95	0\\
92.96	0\\
92.97	0\\
92.98	0\\
92.99	0\\
93	0\\
93.01	0\\
93.02	0\\
93.03	0\\
93.04	0\\
93.05	0\\
93.06	0\\
93.07	0\\
93.08	0\\
93.09	0\\
93.1	0\\
93.11	0\\
93.12	0\\
93.13	0\\
93.14	0\\
93.15	0\\
93.16	0\\
93.17	0\\
93.18	0\\
93.19	0\\
93.2	0\\
93.21	0\\
93.22	0\\
93.23	0\\
93.24	0\\
93.25	0\\
93.26	0\\
93.27	0\\
93.28	0\\
93.29	0\\
93.3	0\\
93.31	0\\
93.32	0\\
93.33	0\\
93.34	0\\
93.35	0\\
93.36	0\\
93.37	0\\
93.38	0\\
93.39	0\\
93.4	0\\
93.41	0\\
93.42	0\\
93.43	0\\
93.44	0\\
93.45	0\\
93.46	0\\
93.47	0\\
93.48	0\\
93.49	0\\
93.5	0\\
93.51	0\\
93.52	0\\
93.53	0\\
93.54	0\\
93.55	0\\
93.56	0\\
93.57	0\\
93.58	0\\
93.59	0\\
93.6	0\\
93.61	0\\
93.62	0\\
93.63	0\\
93.64	0\\
93.65	0\\
93.66	0\\
93.67	0\\
93.68	0\\
93.69	0\\
93.7	0\\
93.71	0\\
93.72	0\\
93.73	0\\
93.74	0\\
93.75	0\\
93.76	0\\
93.77	0\\
93.78	0\\
93.79	0\\
93.8	0\\
93.81	0\\
93.82	0\\
93.83	0\\
93.84	0\\
93.85	0\\
93.86	0\\
93.87	0\\
93.88	0\\
93.89	0\\
93.9	0\\
93.91	0\\
93.92	0\\
93.93	0\\
93.94	0\\
93.95	0\\
93.96	0\\
93.97	0\\
93.98	0\\
93.99	0\\
94	0\\
94.01	0\\
94.02	0\\
94.03	0\\
94.04	0\\
94.05	0\\
94.06	0\\
94.07	0\\
94.08	0\\
94.09	0\\
94.1	0\\
94.11	0\\
94.12	0\\
94.13	0\\
94.14	0\\
94.15	0\\
94.16	0\\
94.17	0\\
94.18	0\\
94.19	0\\
94.2	0\\
94.21	0\\
94.22	0\\
94.23	0\\
94.24	0\\
94.25	0\\
94.26	0\\
94.27	0\\
94.28	0\\
94.29	0\\
94.3	0\\
94.31	0\\
94.32	0\\
94.33	0\\
94.34	0\\
94.35	0\\
94.36	0\\
94.37	0\\
94.38	0\\
94.39	0\\
94.4	0\\
94.41	0\\
94.42	0\\
94.43	0\\
94.44	0\\
94.45	0\\
94.46	0\\
94.47	0\\
94.48	0\\
94.49	0\\
94.5	0\\
94.51	0\\
94.52	0\\
94.53	0\\
94.54	0\\
94.55	0\\
94.56	0\\
94.57	0\\
94.58	0\\
94.59	0\\
94.6	0\\
94.61	0\\
94.62	0\\
94.63	0\\
94.64	0\\
94.65	0\\
94.66	0\\
94.67	0\\
94.68	0\\
94.69	0\\
94.7	0\\
94.71	0\\
94.72	0\\
94.73	0\\
94.74	0\\
94.75	0\\
94.76	0\\
94.77	0\\
94.78	0\\
94.79	0\\
94.8	0\\
94.81	0\\
94.82	0\\
94.83	0\\
94.84	0\\
94.85	0\\
94.86	0\\
94.87	0\\
94.88	0\\
94.89	0\\
94.9	0\\
94.91	0\\
94.92	0\\
94.93	0\\
94.94	0\\
94.95	0\\
94.96	0\\
94.97	0\\
94.98	0\\
94.99	0\\
95	0\\
95.01	0\\
95.02	0\\
95.03	0\\
95.04	0\\
95.05	0\\
95.06	0\\
95.07	0\\
95.08	0\\
95.09	0\\
95.1	0\\
95.11	0\\
95.12	0\\
95.13	0\\
95.14	0\\
95.15	0\\
95.16	0\\
95.17	0\\
95.18	0\\
95.19	0\\
95.2	0\\
95.21	0\\
95.22	0\\
95.23	0\\
95.24	0\\
95.25	0\\
95.26	0\\
95.27	0\\
95.28	0\\
95.29	0\\
95.3	0\\
95.31	0\\
95.32	0\\
95.33	0\\
95.34	0\\
95.35	0\\
95.36	0\\
95.37	0\\
95.38	0\\
95.39	0\\
95.4	0\\
95.41	0\\
95.42	0\\
95.43	0\\
95.44	0\\
95.45	0\\
95.46	0\\
95.47	0\\
95.48	0\\
95.49	0\\
95.5	0\\
95.51	0\\
95.52	0\\
95.53	0\\
95.54	0\\
95.55	0\\
95.56	0\\
95.57	0\\
95.58	0\\
95.59	0\\
95.6	0\\
95.61	0\\
95.62	0\\
95.63	0\\
95.64	0\\
95.65	0\\
95.66	0\\
95.67	0\\
95.68	0\\
95.69	0\\
95.7	0\\
95.71	0\\
95.72	0\\
95.73	0\\
95.74	0\\
95.75	0\\
95.76	0\\
95.77	0\\
95.78	0\\
95.79	0\\
95.8	0\\
95.81	0\\
95.82	0\\
95.83	0\\
95.84	0\\
95.85	0\\
95.86	0\\
95.87	0\\
95.88	0\\
95.89	0\\
95.9	0\\
95.91	0\\
95.92	0\\
95.93	0\\
95.94	0\\
95.95	0\\
95.96	0\\
95.97	0\\
95.98	0\\
95.99	0\\
96	0\\
96.01	0\\
96.02	0\\
96.03	0\\
96.04	0\\
96.05	0\\
96.06	0\\
96.07	0\\
96.08	0\\
96.09	0\\
96.1	0\\
96.11	0\\
96.12	0\\
96.13	0\\
96.14	0\\
96.15	0\\
96.16	0\\
96.17	0\\
96.18	0\\
96.19	0\\
96.2	0\\
96.21	0\\
96.22	0\\
96.23	0\\
96.24	0\\
96.25	0\\
96.26	0\\
96.27	0\\
96.28	0\\
96.29	0\\
96.3	0\\
96.31	0\\
96.32	0\\
96.33	0\\
96.34	0\\
96.35	0\\
96.36	0\\
96.37	0\\
96.38	0\\
96.39	0\\
96.4	0\\
96.41	0\\
96.42	0\\
96.43	0\\
96.44	0\\
96.45	0\\
96.46	0\\
96.47	0\\
96.48	0\\
96.49	0\\
96.5	0\\
96.51	0\\
96.52	0\\
96.53	0\\
96.54	0\\
96.55	0\\
96.56	0\\
96.57	0\\
96.58	0\\
96.59	0\\
96.6	0\\
96.61	0\\
96.62	0\\
96.63	0\\
96.64	0\\
96.65	0\\
96.66	0\\
96.67	0\\
96.68	0\\
96.69	0\\
96.7	0\\
96.71	0\\
96.72	0\\
96.73	0\\
96.74	0\\
96.75	0\\
96.76	0\\
96.77	0\\
96.78	0\\
96.79	0\\
96.8	0\\
96.81	0\\
96.82	0\\
96.83	0\\
96.84	0\\
96.85	0\\
96.86	0\\
96.87	0\\
96.88	0\\
96.89	0\\
96.9	0\\
96.91	0\\
96.92	0\\
96.93	0\\
96.94	0\\
96.95	0\\
96.96	0\\
96.97	0\\
96.98	0\\
96.99	0\\
97	0\\
97.01	0\\
97.02	0\\
97.03	0\\
97.04	0\\
97.05	0\\
97.06	0\\
97.07	0\\
97.08	0\\
97.09	0\\
97.1	0\\
97.11	0\\
97.12	0\\
97.13	0\\
97.14	0\\
97.15	0\\
97.16	0\\
97.17	0\\
97.18	0\\
97.19	0\\
97.2	0\\
97.21	0\\
97.22	0\\
97.23	0\\
97.24	0\\
97.25	0\\
97.26	0\\
97.27	0\\
97.28	0\\
97.29	0\\
97.3	0\\
97.31	0\\
97.32	0\\
97.33	0\\
97.34	0\\
97.35	0\\
97.36	0\\
97.37	0\\
97.38	0\\
97.39	0\\
97.4	0\\
97.41	0\\
97.42	0\\
97.43	0\\
97.44	0\\
97.45	0\\
97.46	0\\
97.47	0\\
97.48	0\\
97.49	0\\
97.5	0\\
97.51	0\\
97.52	0\\
97.53	0\\
97.54	0\\
97.55	0\\
97.56	0\\
97.57	0\\
97.58	0\\
97.59	0\\
97.6	0\\
97.61	0\\
97.62	0\\
97.63	0\\
97.64	0\\
97.65	0\\
97.66	0\\
97.67	0\\
97.68	0\\
97.69	0\\
97.7	0\\
97.71	0\\
97.72	0\\
97.73	0\\
97.74	0\\
97.75	0\\
97.76	0\\
97.77	0\\
97.78	0\\
97.79	0\\
97.8	0\\
97.81	0\\
97.82	0\\
97.83	0\\
97.84	0\\
97.85	0\\
97.86	0\\
97.87	0\\
97.88	0\\
97.89	0\\
97.9	0\\
97.91	0\\
97.92	0\\
97.93	0\\
97.94	0\\
97.95	0\\
97.96	0\\
97.97	0\\
97.98	0\\
97.99	0\\
98	0\\
98.01	0\\
98.02	0\\
98.03	0\\
98.04	0\\
98.05	0\\
98.06	0\\
98.07	0\\
98.08	0\\
98.09	0\\
98.1	0\\
98.11	0\\
98.12	0\\
98.13	0\\
98.14	0\\
98.15	0\\
98.16	0\\
98.17	0\\
98.18	0\\
98.19	5.87091993147215e-05\\
98.2	0.000131660668289388\\
98.21	0.000205183667401541\\
98.22	0.000279272272188956\\
98.23	0.000353932011134473\\
98.24	0.000429168473808594\\
98.25	0.000504987311700112\\
98.26	0.000581394239061801\\
98.27	0.000658395033771495\\
98.28	0.000735995538209001\\
98.29	0.000814201660149175\\
98.3	0.000893019375794994\\
98.31	0.000972454729098664\\
98.32	0.00105251383229726\\
98.33	0.00113320286707123\\
98.34	0.00121452808741002\\
98.35	0.00129649581862847\\
98.36	0.00137911245839192\\
98.37	0.00140436050083681\\
98.38	0.0014277326476487\\
98.39	0.00145129188951867\\
98.4	0.00147503968331663\\
98.41	0.00149897749327697\\
98.42	0.0015231067909245\\
98.43	0.00154743830384437\\
98.44	0.0015719765775428\\
98.45	0.00159672333583745\\
98.46	0.00162168031340307\\
98.47	0.00164684925574105\\
98.48	0.00167223191914474\\
98.49	0.00169783007066085\\
98.5	0.00172364548804649\\
98.51	0.00174967995972184\\
98.52	0.00177593528471809\\
98.53	0.00180241327262077\\
98.54	0.00182911574350748\\
98.55	0.00185604452787694\\
98.56	0.00188320146657662\\
98.57	0.00191058841072475\\
98.58	0.0019382072216267\\
98.59	0.0019660597706852\\
98.6	0.00199414793930437\\
98.61	0.00202247361878732\\
98.62	0.00205103871022702\\
98.63	0.00207984512439009\\
98.64	0.00210889478159352\\
98.65	0.00213818961157364\\
98.66	0.00216773155443861\\
98.67	0.00219752255983394\\
98.68	0.00222756458641881\\
98.69	0.00225787628061744\\
98.7	0.0022884654676135\\
98.71	0.00231933481963858\\
98.72	0.0023504870349779\\
98.73	0.00238192483822133\\
98.74	0.00241365097665809\\
98.75	0.00244566822147595\\
98.76	0.00247797937070885\\
98.77	0.00251058724950424\\
98.78	0.00254349471039308\\
98.79	0.00257670462316625\\
98.8	0.00261021988212598\\
98.81	0.002644043409222\\
98.82	0.0026781781543157\\
98.83	0.00271262709475195\\
98.84	0.00274739322467683\\
98.85	0.00278247952927148\\
98.86	0.00281788902175932\\
98.87	0.00285362474366859\\
98.88	0.00288968976509735\\
98.89	0.00292608718498091\\
98.9	0.00296282013136184\\
98.91	0.00299989176166234\\
98.92	0.00303730526295931\\
98.93	0.00307506385226188\\
98.94	0.00311317077679147\\
98.95	0.00315162931426461\\
98.96	0.00319044277317824\\
98.97	0.00322961449309772\\
98.98	0.0032691478449475\\
98.99	0.00330904623130455\\
99	0.00334931308669438\\
99.01	0.00338995187788997\\
99.02	0.00343096610421334\\
99.03	0.00347235929784001\\
99.04	0.00351413502410622\\
99.05	0.00355629688181904\\
99.06	0.00359884850356936\\
99.07	0.00364179355604775\\
99.08	0.00368513574038585\\
99.09	0.00372887879249803\\
99.1	0.00377302648340829\\
99.11	0.00381758261958013\\
99.12	0.00386255104325286\\
99.13	0.00390793563277871\\
99.14	0.00395374030296244\\
99.15	0.00399996900540418\\
99.16	0.00404662572884551\\
99.17	0.0040937144995188\\
99.18	0.00414123938149992\\
99.19	0.00418920447706429\\
99.2	0.0042376139270463\\
99.21	0.00428647191120219\\
99.22	0.00433578264857637\\
99.23	0.00438555039787123\\
99.24	0.00443577945782051\\
99.25	0.00448647416756618\\
99.26	0.00453763890703903\\
99.27	0.00458927809734277\\
99.28	0.00464139620114196\\
99.29	0.0046939977230535\\
99.3	0.00474708721004204\\
99.31	0.00480066925181908\\
99.32	0.00485474848124597\\
99.33	0.00490932957474073\\
99.34	0.00496441725268891\\
99.35	0.00502001627985832\\
99.36	0.00507613146581777\\
99.37	0.00513276766535998\\
99.38	0.00518992977892843\\
99.39	0.00524762275304852\\
99.4	0.00530585158076283\\
99.41	0.00536462130207064\\
99.42	0.00542393700437178\\
99.43	0.00548380382291478\\
99.44	0.00554422694124944\\
99.45	0.00560521159168388\\
99.46	0.00566676305574604\\
99.47	0.00572888666464975\\
99.48	0.00579158779976541\\
99.49	0.00585487189309542\\
99.5	0.0059187444277542\\
99.51	0.00598321093845309\\
99.52	0.00604827701199012\\
99.53	0.00611394828774462\\
99.54	0.00618023045817687\\
99.55	0.00624712926933275\\
99.56	0.00631465052135359\\
99.57	0.00638280006899107\\
99.58	0.0064515838221275\\
99.59	0.00652100774630139\\
99.6	0.00659107786323844\\
99.61	0.00666180022938315\\
99.62	0.00673318094678823\\
99.63	0.00680522617466349\\
99.64	0.00687794212991873\\
99.65	0.00695133508771244\\
99.66	0.00702541138200604\\
99.67	0.00710017740612412\\
99.68	0.00717563961332046\\
99.69	0.00725180451735017\\
99.7	0.00732867869304791\\
99.71	0.00740626877694791\\
99.72	0.00748458146787993\\
99.73	0.00756362352756889\\
99.74	0.0076434017812414\\
99.75	0.00772392311823898\\
99.76	0.0078051944926383\\
99.77	0.00788722292387853\\
99.78	0.00797001549739592\\
99.79	0.00805357936526582\\
99.8	0.00813792174685226\\
99.81	0.00822304992946529\\
99.82	0.00830897126902625\\
99.83	0.00839569319074109\\
99.84	0.00848322318978214\\
99.85	0.00857156883197824\\
99.86	0.00866073775451372\\
99.87	0.0087507376666363\\
99.88	0.00884157635037421\\
99.89	0.00893326166126276\\
99.9	0.00902580152908066\\
99.91	0.00911920395859638\\
99.92	0.00921347703032472\\
99.93	0.00930862890129417\\
99.94	0.00940466780582511\\
99.95	0.0095016020563194\\
99.96	0.00959944004406162\\
99.97	0.00969819024003246\\
99.98	0.0097978611957346\\
99.99	0.00989846154403157\\
100	0.01\\
};
\addlegendentry{$q=2$};

\addplot [color=mycolor1,solid,forget plot]
  table[row sep=crcr]{%
0.01	0\\
0.02	0\\
0.03	0\\
0.04	0\\
0.05	0\\
0.06	0\\
0.07	0\\
0.08	0\\
0.09	0\\
0.1	0\\
0.11	0\\
0.12	0\\
0.13	0\\
0.14	0\\
0.15	0\\
0.16	0\\
0.17	0\\
0.18	0\\
0.19	0\\
0.2	0\\
0.21	0\\
0.22	0\\
0.23	0\\
0.24	0\\
0.25	0\\
0.26	0\\
0.27	0\\
0.28	0\\
0.29	0\\
0.3	0\\
0.31	0\\
0.32	0\\
0.33	0\\
0.34	0\\
0.35	0\\
0.36	0\\
0.37	0\\
0.38	0\\
0.39	0\\
0.4	0\\
0.41	0\\
0.42	0\\
0.43	0\\
0.44	0\\
0.45	0\\
0.46	0\\
0.47	0\\
0.48	0\\
0.49	0\\
0.5	0\\
0.51	0\\
0.52	0\\
0.53	0\\
0.54	0\\
0.55	0\\
0.56	0\\
0.57	0\\
0.58	0\\
0.59	0\\
0.6	0\\
0.61	0\\
0.62	0\\
0.63	0\\
0.64	0\\
0.65	0\\
0.66	0\\
0.67	0\\
0.68	0\\
0.69	0\\
0.7	0\\
0.71	0\\
0.72	0\\
0.73	0\\
0.74	0\\
0.75	0\\
0.76	0\\
0.77	0\\
0.78	0\\
0.79	0\\
0.8	0\\
0.81	0\\
0.82	0\\
0.83	0\\
0.84	0\\
0.85	0\\
0.86	0\\
0.87	0\\
0.88	0\\
0.89	0\\
0.9	0\\
0.91	0\\
0.92	0\\
0.93	0\\
0.94	0\\
0.95	0\\
0.96	0\\
0.97	0\\
0.98	0\\
0.99	0\\
1	0\\
1.01	0\\
1.02	0\\
1.03	0\\
1.04	0\\
1.05	0\\
1.06	0\\
1.07	0\\
1.08	0\\
1.09	0\\
1.1	0\\
1.11	0\\
1.12	0\\
1.13	0\\
1.14	0\\
1.15	0\\
1.16	0\\
1.17	0\\
1.18	0\\
1.19	0\\
1.2	0\\
1.21	0\\
1.22	0\\
1.23	0\\
1.24	0\\
1.25	0\\
1.26	0\\
1.27	0\\
1.28	0\\
1.29	0\\
1.3	0\\
1.31	0\\
1.32	0\\
1.33	0\\
1.34	0\\
1.35	0\\
1.36	0\\
1.37	0\\
1.38	0\\
1.39	0\\
1.4	0\\
1.41	0\\
1.42	0\\
1.43	0\\
1.44	0\\
1.45	0\\
1.46	0\\
1.47	0\\
1.48	0\\
1.49	0\\
1.5	0\\
1.51	0\\
1.52	0\\
1.53	0\\
1.54	0\\
1.55	0\\
1.56	0\\
1.57	0\\
1.58	0\\
1.59	0\\
1.6	0\\
1.61	0\\
1.62	0\\
1.63	0\\
1.64	0\\
1.65	0\\
1.66	0\\
1.67	0\\
1.68	0\\
1.69	0\\
1.7	0\\
1.71	0\\
1.72	0\\
1.73	0\\
1.74	0\\
1.75	0\\
1.76	0\\
1.77	0\\
1.78	0\\
1.79	0\\
1.8	0\\
1.81	0\\
1.82	0\\
1.83	0\\
1.84	0\\
1.85	0\\
1.86	0\\
1.87	0\\
1.88	0\\
1.89	0\\
1.9	0\\
1.91	0\\
1.92	0\\
1.93	0\\
1.94	0\\
1.95	0\\
1.96	0\\
1.97	0\\
1.98	0\\
1.99	0\\
2	0\\
2.01	0\\
2.02	0\\
2.03	0\\
2.04	0\\
2.05	0\\
2.06	0\\
2.07	0\\
2.08	0\\
2.09	0\\
2.1	0\\
2.11	0\\
2.12	0\\
2.13	0\\
2.14	0\\
2.15	0\\
2.16	0\\
2.17	0\\
2.18	0\\
2.19	0\\
2.2	0\\
2.21	0\\
2.22	0\\
2.23	0\\
2.24	0\\
2.25	0\\
2.26	0\\
2.27	0\\
2.28	0\\
2.29	0\\
2.3	0\\
2.31	0\\
2.32	0\\
2.33	0\\
2.34	0\\
2.35	0\\
2.36	0\\
2.37	0\\
2.38	0\\
2.39	0\\
2.4	0\\
2.41	0\\
2.42	0\\
2.43	0\\
2.44	0\\
2.45	0\\
2.46	0\\
2.47	0\\
2.48	0\\
2.49	0\\
2.5	0\\
2.51	0\\
2.52	0\\
2.53	0\\
2.54	0\\
2.55	0\\
2.56	0\\
2.57	0\\
2.58	0\\
2.59	0\\
2.6	0\\
2.61	0\\
2.62	0\\
2.63	0\\
2.64	0\\
2.65	0\\
2.66	0\\
2.67	0\\
2.68	0\\
2.69	0\\
2.7	0\\
2.71	0\\
2.72	0\\
2.73	0\\
2.74	0\\
2.75	0\\
2.76	0\\
2.77	0\\
2.78	0\\
2.79	0\\
2.8	0\\
2.81	0\\
2.82	0\\
2.83	0\\
2.84	0\\
2.85	0\\
2.86	0\\
2.87	0\\
2.88	0\\
2.89	0\\
2.9	0\\
2.91	0\\
2.92	0\\
2.93	0\\
2.94	0\\
2.95	0\\
2.96	0\\
2.97	0\\
2.98	0\\
2.99	0\\
3	0\\
3.01	0\\
3.02	0\\
3.03	0\\
3.04	0\\
3.05	0\\
3.06	0\\
3.07	0\\
3.08	0\\
3.09	0\\
3.1	0\\
3.11	0\\
3.12	0\\
3.13	0\\
3.14	0\\
3.15	0\\
3.16	0\\
3.17	0\\
3.18	0\\
3.19	0\\
3.2	0\\
3.21	0\\
3.22	0\\
3.23	0\\
3.24	0\\
3.25	0\\
3.26	0\\
3.27	0\\
3.28	0\\
3.29	0\\
3.3	0\\
3.31	0\\
3.32	0\\
3.33	0\\
3.34	0\\
3.35	0\\
3.36	0\\
3.37	0\\
3.38	0\\
3.39	0\\
3.4	0\\
3.41	0\\
3.42	0\\
3.43	0\\
3.44	0\\
3.45	0\\
3.46	0\\
3.47	0\\
3.48	0\\
3.49	0\\
3.5	0\\
3.51	0\\
3.52	0\\
3.53	0\\
3.54	0\\
3.55	0\\
3.56	0\\
3.57	0\\
3.58	0\\
3.59	0\\
3.6	0\\
3.61	0\\
3.62	0\\
3.63	0\\
3.64	0\\
3.65	0\\
3.66	0\\
3.67	0\\
3.68	0\\
3.69	0\\
3.7	0\\
3.71	0\\
3.72	0\\
3.73	0\\
3.74	0\\
3.75	0\\
3.76	0\\
3.77	0\\
3.78	0\\
3.79	0\\
3.8	0\\
3.81	0\\
3.82	0\\
3.83	0\\
3.84	0\\
3.85	0\\
3.86	0\\
3.87	0\\
3.88	0\\
3.89	0\\
3.9	0\\
3.91	0\\
3.92	0\\
3.93	0\\
3.94	0\\
3.95	0\\
3.96	0\\
3.97	0\\
3.98	0\\
3.99	0\\
4	0\\
4.01	0\\
4.02	0\\
4.03	0\\
4.04	0\\
4.05	0\\
4.06	0\\
4.07	0\\
4.08	0\\
4.09	0\\
4.1	0\\
4.11	0\\
4.12	0\\
4.13	0\\
4.14	0\\
4.15	0\\
4.16	0\\
4.17	0\\
4.18	0\\
4.19	0\\
4.2	0\\
4.21	0\\
4.22	0\\
4.23	0\\
4.24	0\\
4.25	0\\
4.26	0\\
4.27	0\\
4.28	0\\
4.29	0\\
4.3	0\\
4.31	0\\
4.32	0\\
4.33	0\\
4.34	0\\
4.35	0\\
4.36	0\\
4.37	0\\
4.38	0\\
4.39	0\\
4.4	0\\
4.41	0\\
4.42	0\\
4.43	0\\
4.44	0\\
4.45	0\\
4.46	0\\
4.47	0\\
4.48	0\\
4.49	0\\
4.5	0\\
4.51	0\\
4.52	0\\
4.53	0\\
4.54	0\\
4.55	0\\
4.56	0\\
4.57	0\\
4.58	0\\
4.59	0\\
4.6	0\\
4.61	0\\
4.62	0\\
4.63	0\\
4.64	0\\
4.65	0\\
4.66	0\\
4.67	0\\
4.68	0\\
4.69	0\\
4.7	0\\
4.71	0\\
4.72	0\\
4.73	0\\
4.74	0\\
4.75	0\\
4.76	0\\
4.77	0\\
4.78	0\\
4.79	0\\
4.8	0\\
4.81	0\\
4.82	0\\
4.83	0\\
4.84	0\\
4.85	0\\
4.86	0\\
4.87	0\\
4.88	0\\
4.89	0\\
4.9	0\\
4.91	0\\
4.92	0\\
4.93	0\\
4.94	0\\
4.95	0\\
4.96	0\\
4.97	0\\
4.98	0\\
4.99	0\\
5	0\\
5.01	0\\
5.02	0\\
5.03	0\\
5.04	0\\
5.05	0\\
5.06	0\\
5.07	0\\
5.08	0\\
5.09	0\\
5.1	0\\
5.11	0\\
5.12	0\\
5.13	0\\
5.14	0\\
5.15	0\\
5.16	0\\
5.17	0\\
5.18	0\\
5.19	0\\
5.2	0\\
5.21	0\\
5.22	0\\
5.23	0\\
5.24	0\\
5.25	0\\
5.26	0\\
5.27	0\\
5.28	0\\
5.29	0\\
5.3	0\\
5.31	0\\
5.32	0\\
5.33	0\\
5.34	0\\
5.35	0\\
5.36	0\\
5.37	0\\
5.38	0\\
5.39	0\\
5.4	0\\
5.41	0\\
5.42	0\\
5.43	0\\
5.44	0\\
5.45	0\\
5.46	0\\
5.47	0\\
5.48	0\\
5.49	0\\
5.5	0\\
5.51	0\\
5.52	0\\
5.53	0\\
5.54	0\\
5.55	0\\
5.56	0\\
5.57	0\\
5.58	0\\
5.59	0\\
5.6	0\\
5.61	0\\
5.62	0\\
5.63	0\\
5.64	0\\
5.65	0\\
5.66	0\\
5.67	0\\
5.68	0\\
5.69	0\\
5.7	0\\
5.71	0\\
5.72	0\\
5.73	0\\
5.74	0\\
5.75	0\\
5.76	0\\
5.77	0\\
5.78	0\\
5.79	0\\
5.8	0\\
5.81	0\\
5.82	0\\
5.83	0\\
5.84	0\\
5.85	0\\
5.86	0\\
5.87	0\\
5.88	0\\
5.89	0\\
5.9	0\\
5.91	0\\
5.92	0\\
5.93	0\\
5.94	0\\
5.95	0\\
5.96	0\\
5.97	0\\
5.98	0\\
5.99	0\\
6	0\\
6.01	0\\
6.02	0\\
6.03	0\\
6.04	0\\
6.05	0\\
6.06	0\\
6.07	0\\
6.08	0\\
6.09	0\\
6.1	0\\
6.11	0\\
6.12	0\\
6.13	0\\
6.14	0\\
6.15	0\\
6.16	0\\
6.17	0\\
6.18	0\\
6.19	0\\
6.2	0\\
6.21	0\\
6.22	0\\
6.23	0\\
6.24	0\\
6.25	0\\
6.26	0\\
6.27	0\\
6.28	0\\
6.29	0\\
6.3	0\\
6.31	0\\
6.32	0\\
6.33	0\\
6.34	0\\
6.35	0\\
6.36	0\\
6.37	0\\
6.38	0\\
6.39	0\\
6.4	0\\
6.41	0\\
6.42	0\\
6.43	0\\
6.44	0\\
6.45	0\\
6.46	0\\
6.47	0\\
6.48	0\\
6.49	0\\
6.5	0\\
6.51	0\\
6.52	0\\
6.53	0\\
6.54	0\\
6.55	0\\
6.56	0\\
6.57	0\\
6.58	0\\
6.59	0\\
6.6	0\\
6.61	0\\
6.62	0\\
6.63	0\\
6.64	0\\
6.65	0\\
6.66	0\\
6.67	0\\
6.68	0\\
6.69	0\\
6.7	0\\
6.71	0\\
6.72	0\\
6.73	0\\
6.74	0\\
6.75	0\\
6.76	0\\
6.77	0\\
6.78	0\\
6.79	0\\
6.8	0\\
6.81	0\\
6.82	0\\
6.83	0\\
6.84	0\\
6.85	0\\
6.86	0\\
6.87	0\\
6.88	0\\
6.89	0\\
6.9	0\\
6.91	0\\
6.92	0\\
6.93	0\\
6.94	0\\
6.95	0\\
6.96	0\\
6.97	0\\
6.98	0\\
6.99	0\\
7	0\\
7.01	0\\
7.02	0\\
7.03	0\\
7.04	0\\
7.05	0\\
7.06	0\\
7.07	0\\
7.08	0\\
7.09	0\\
7.1	0\\
7.11	0\\
7.12	0\\
7.13	0\\
7.14	0\\
7.15	0\\
7.16	0\\
7.17	0\\
7.18	0\\
7.19	0\\
7.2	0\\
7.21	0\\
7.22	0\\
7.23	0\\
7.24	0\\
7.25	0\\
7.26	0\\
7.27	0\\
7.28	0\\
7.29	0\\
7.3	0\\
7.31	0\\
7.32	0\\
7.33	0\\
7.34	0\\
7.35	0\\
7.36	0\\
7.37	0\\
7.38	0\\
7.39	0\\
7.4	0\\
7.41	0\\
7.42	0\\
7.43	0\\
7.44	0\\
7.45	0\\
7.46	0\\
7.47	0\\
7.48	0\\
7.49	0\\
7.5	0\\
7.51	0\\
7.52	0\\
7.53	0\\
7.54	0\\
7.55	0\\
7.56	0\\
7.57	0\\
7.58	0\\
7.59	0\\
7.6	0\\
7.61	0\\
7.62	0\\
7.63	0\\
7.64	0\\
7.65	0\\
7.66	0\\
7.67	0\\
7.68	0\\
7.69	0\\
7.7	0\\
7.71	0\\
7.72	0\\
7.73	0\\
7.74	0\\
7.75	0\\
7.76	0\\
7.77	0\\
7.78	0\\
7.79	0\\
7.8	0\\
7.81	0\\
7.82	0\\
7.83	0\\
7.84	0\\
7.85	0\\
7.86	0\\
7.87	0\\
7.88	0\\
7.89	0\\
7.9	0\\
7.91	0\\
7.92	0\\
7.93	0\\
7.94	0\\
7.95	0\\
7.96	0\\
7.97	0\\
7.98	0\\
7.99	0\\
8	0\\
8.01	0\\
8.02	0\\
8.03	0\\
8.04	0\\
8.05	0\\
8.06	0\\
8.07	0\\
8.08	0\\
8.09	0\\
8.1	0\\
8.11	0\\
8.12	0\\
8.13	0\\
8.14	0\\
8.15	0\\
8.16	0\\
8.17	0\\
8.18	0\\
8.19	0\\
8.2	0\\
8.21	0\\
8.22	0\\
8.23	0\\
8.24	0\\
8.25	0\\
8.26	0\\
8.27	0\\
8.28	0\\
8.29	0\\
8.3	0\\
8.31	0\\
8.32	0\\
8.33	0\\
8.34	0\\
8.35	0\\
8.36	0\\
8.37	0\\
8.38	0\\
8.39	0\\
8.4	0\\
8.41	0\\
8.42	0\\
8.43	0\\
8.44	0\\
8.45	0\\
8.46	0\\
8.47	0\\
8.48	0\\
8.49	0\\
8.5	0\\
8.51	0\\
8.52	0\\
8.53	0\\
8.54	0\\
8.55	0\\
8.56	0\\
8.57	0\\
8.58	0\\
8.59	0\\
8.6	0\\
8.61	0\\
8.62	0\\
8.63	0\\
8.64	0\\
8.65	0\\
8.66	0\\
8.67	0\\
8.68	0\\
8.69	0\\
8.7	0\\
8.71	0\\
8.72	0\\
8.73	0\\
8.74	0\\
8.75	0\\
8.76	0\\
8.77	0\\
8.78	0\\
8.79	0\\
8.8	0\\
8.81	0\\
8.82	0\\
8.83	0\\
8.84	0\\
8.85	0\\
8.86	0\\
8.87	0\\
8.88	0\\
8.89	0\\
8.9	0\\
8.91	0\\
8.92	0\\
8.93	0\\
8.94	0\\
8.95	0\\
8.96	0\\
8.97	0\\
8.98	0\\
8.99	0\\
9	0\\
9.01	0\\
9.02	0\\
9.03	0\\
9.04	0\\
9.05	0\\
9.06	0\\
9.07	0\\
9.08	0\\
9.09	0\\
9.1	0\\
9.11	0\\
9.12	0\\
9.13	0\\
9.14	0\\
9.15	0\\
9.16	0\\
9.17	0\\
9.18	0\\
9.19	0\\
9.2	0\\
9.21	0\\
9.22	0\\
9.23	0\\
9.24	0\\
9.25	0\\
9.26	0\\
9.27	0\\
9.28	0\\
9.29	0\\
9.3	0\\
9.31	0\\
9.32	0\\
9.33	0\\
9.34	0\\
9.35	0\\
9.36	0\\
9.37	0\\
9.38	0\\
9.39	0\\
9.4	0\\
9.41	0\\
9.42	0\\
9.43	0\\
9.44	0\\
9.45	0\\
9.46	0\\
9.47	0\\
9.48	0\\
9.49	0\\
9.5	0\\
9.51	0\\
9.52	0\\
9.53	0\\
9.54	0\\
9.55	0\\
9.56	0\\
9.57	0\\
9.58	0\\
9.59	0\\
9.6	0\\
9.61	0\\
9.62	0\\
9.63	0\\
9.64	0\\
9.65	0\\
9.66	0\\
9.67	0\\
9.68	0\\
9.69	0\\
9.7	0\\
9.71	0\\
9.72	0\\
9.73	0\\
9.74	0\\
9.75	0\\
9.76	0\\
9.77	0\\
9.78	0\\
9.79	0\\
9.8	0\\
9.81	0\\
9.82	0\\
9.83	0\\
9.84	0\\
9.85	0\\
9.86	0\\
9.87	0\\
9.88	0\\
9.89	0\\
9.9	0\\
9.91	0\\
9.92	0\\
9.93	0\\
9.94	0\\
9.95	0\\
9.96	0\\
9.97	0\\
9.98	0\\
9.99	0\\
10	0\\
10.01	0\\
10.02	0\\
10.03	0\\
10.04	0\\
10.05	0\\
10.06	0\\
10.07	0\\
10.08	0\\
10.09	0\\
10.1	0\\
10.11	0\\
10.12	0\\
10.13	0\\
10.14	0\\
10.15	0\\
10.16	0\\
10.17	0\\
10.18	0\\
10.19	0\\
10.2	0\\
10.21	0\\
10.22	0\\
10.23	0\\
10.24	0\\
10.25	0\\
10.26	0\\
10.27	0\\
10.28	0\\
10.29	0\\
10.3	0\\
10.31	0\\
10.32	0\\
10.33	0\\
10.34	0\\
10.35	0\\
10.36	0\\
10.37	0\\
10.38	0\\
10.39	0\\
10.4	0\\
10.41	0\\
10.42	0\\
10.43	0\\
10.44	0\\
10.45	0\\
10.46	0\\
10.47	0\\
10.48	0\\
10.49	0\\
10.5	0\\
10.51	0\\
10.52	0\\
10.53	0\\
10.54	0\\
10.55	0\\
10.56	0\\
10.57	0\\
10.58	0\\
10.59	0\\
10.6	0\\
10.61	0\\
10.62	0\\
10.63	0\\
10.64	0\\
10.65	0\\
10.66	0\\
10.67	0\\
10.68	0\\
10.69	0\\
10.7	0\\
10.71	0\\
10.72	0\\
10.73	0\\
10.74	0\\
10.75	0\\
10.76	0\\
10.77	0\\
10.78	0\\
10.79	0\\
10.8	0\\
10.81	0\\
10.82	0\\
10.83	0\\
10.84	0\\
10.85	0\\
10.86	0\\
10.87	0\\
10.88	0\\
10.89	0\\
10.9	0\\
10.91	0\\
10.92	0\\
10.93	0\\
10.94	0\\
10.95	0\\
10.96	0\\
10.97	0\\
10.98	0\\
10.99	0\\
11	0\\
11.01	0\\
11.02	0\\
11.03	0\\
11.04	0\\
11.05	0\\
11.06	0\\
11.07	0\\
11.08	0\\
11.09	0\\
11.1	0\\
11.11	0\\
11.12	0\\
11.13	0\\
11.14	0\\
11.15	0\\
11.16	0\\
11.17	0\\
11.18	0\\
11.19	0\\
11.2	0\\
11.21	0\\
11.22	0\\
11.23	0\\
11.24	0\\
11.25	0\\
11.26	0\\
11.27	0\\
11.28	0\\
11.29	0\\
11.3	0\\
11.31	0\\
11.32	0\\
11.33	0\\
11.34	0\\
11.35	0\\
11.36	0\\
11.37	0\\
11.38	0\\
11.39	0\\
11.4	0\\
11.41	0\\
11.42	0\\
11.43	0\\
11.44	0\\
11.45	0\\
11.46	0\\
11.47	0\\
11.48	0\\
11.49	0\\
11.5	0\\
11.51	0\\
11.52	0\\
11.53	0\\
11.54	0\\
11.55	0\\
11.56	0\\
11.57	0\\
11.58	0\\
11.59	0\\
11.6	0\\
11.61	0\\
11.62	0\\
11.63	0\\
11.64	0\\
11.65	0\\
11.66	0\\
11.67	0\\
11.68	0\\
11.69	0\\
11.7	0\\
11.71	0\\
11.72	0\\
11.73	0\\
11.74	0\\
11.75	0\\
11.76	0\\
11.77	0\\
11.78	0\\
11.79	0\\
11.8	0\\
11.81	0\\
11.82	0\\
11.83	0\\
11.84	0\\
11.85	0\\
11.86	0\\
11.87	0\\
11.88	0\\
11.89	0\\
11.9	0\\
11.91	0\\
11.92	0\\
11.93	0\\
11.94	0\\
11.95	0\\
11.96	0\\
11.97	0\\
11.98	0\\
11.99	0\\
12	0\\
12.01	0\\
12.02	0\\
12.03	0\\
12.04	0\\
12.05	0\\
12.06	0\\
12.07	0\\
12.08	0\\
12.09	0\\
12.1	0\\
12.11	0\\
12.12	0\\
12.13	0\\
12.14	0\\
12.15	0\\
12.16	0\\
12.17	0\\
12.18	0\\
12.19	0\\
12.2	0\\
12.21	0\\
12.22	0\\
12.23	0\\
12.24	0\\
12.25	0\\
12.26	0\\
12.27	0\\
12.28	0\\
12.29	0\\
12.3	0\\
12.31	0\\
12.32	0\\
12.33	0\\
12.34	0\\
12.35	0\\
12.36	0\\
12.37	0\\
12.38	0\\
12.39	0\\
12.4	0\\
12.41	0\\
12.42	0\\
12.43	0\\
12.44	0\\
12.45	0\\
12.46	0\\
12.47	0\\
12.48	0\\
12.49	0\\
12.5	0\\
12.51	0\\
12.52	0\\
12.53	0\\
12.54	0\\
12.55	0\\
12.56	0\\
12.57	0\\
12.58	0\\
12.59	0\\
12.6	0\\
12.61	0\\
12.62	0\\
12.63	0\\
12.64	0\\
12.65	0\\
12.66	0\\
12.67	0\\
12.68	0\\
12.69	0\\
12.7	0\\
12.71	0\\
12.72	0\\
12.73	0\\
12.74	0\\
12.75	0\\
12.76	0\\
12.77	0\\
12.78	0\\
12.79	0\\
12.8	0\\
12.81	0\\
12.82	0\\
12.83	0\\
12.84	0\\
12.85	0\\
12.86	0\\
12.87	0\\
12.88	0\\
12.89	0\\
12.9	0\\
12.91	0\\
12.92	0\\
12.93	0\\
12.94	0\\
12.95	0\\
12.96	0\\
12.97	0\\
12.98	0\\
12.99	0\\
13	0\\
13.01	0\\
13.02	0\\
13.03	0\\
13.04	0\\
13.05	0\\
13.06	0\\
13.07	0\\
13.08	0\\
13.09	0\\
13.1	0\\
13.11	0\\
13.12	0\\
13.13	0\\
13.14	0\\
13.15	0\\
13.16	0\\
13.17	0\\
13.18	0\\
13.19	0\\
13.2	0\\
13.21	0\\
13.22	0\\
13.23	0\\
13.24	0\\
13.25	0\\
13.26	0\\
13.27	0\\
13.28	0\\
13.29	0\\
13.3	0\\
13.31	0\\
13.32	0\\
13.33	0\\
13.34	0\\
13.35	0\\
13.36	0\\
13.37	0\\
13.38	0\\
13.39	0\\
13.4	0\\
13.41	0\\
13.42	0\\
13.43	0\\
13.44	0\\
13.45	0\\
13.46	0\\
13.47	0\\
13.48	0\\
13.49	0\\
13.5	0\\
13.51	0\\
13.52	0\\
13.53	0\\
13.54	0\\
13.55	0\\
13.56	0\\
13.57	0\\
13.58	0\\
13.59	0\\
13.6	0\\
13.61	0\\
13.62	0\\
13.63	0\\
13.64	0\\
13.65	0\\
13.66	0\\
13.67	0\\
13.68	0\\
13.69	0\\
13.7	0\\
13.71	0\\
13.72	0\\
13.73	0\\
13.74	0\\
13.75	0\\
13.76	0\\
13.77	0\\
13.78	0\\
13.79	0\\
13.8	0\\
13.81	0\\
13.82	0\\
13.83	0\\
13.84	0\\
13.85	0\\
13.86	0\\
13.87	0\\
13.88	0\\
13.89	0\\
13.9	0\\
13.91	0\\
13.92	0\\
13.93	0\\
13.94	0\\
13.95	0\\
13.96	0\\
13.97	0\\
13.98	0\\
13.99	0\\
14	0\\
14.01	0\\
14.02	0\\
14.03	0\\
14.04	0\\
14.05	0\\
14.06	0\\
14.07	0\\
14.08	0\\
14.09	0\\
14.1	0\\
14.11	0\\
14.12	0\\
14.13	0\\
14.14	0\\
14.15	0\\
14.16	0\\
14.17	0\\
14.18	0\\
14.19	0\\
14.2	0\\
14.21	0\\
14.22	0\\
14.23	0\\
14.24	0\\
14.25	0\\
14.26	0\\
14.27	0\\
14.28	0\\
14.29	0\\
14.3	0\\
14.31	0\\
14.32	0\\
14.33	0\\
14.34	0\\
14.35	0\\
14.36	0\\
14.37	0\\
14.38	0\\
14.39	0\\
14.4	0\\
14.41	0\\
14.42	0\\
14.43	0\\
14.44	0\\
14.45	0\\
14.46	0\\
14.47	0\\
14.48	0\\
14.49	0\\
14.5	0\\
14.51	0\\
14.52	0\\
14.53	0\\
14.54	0\\
14.55	0\\
14.56	0\\
14.57	0\\
14.58	0\\
14.59	0\\
14.6	0\\
14.61	0\\
14.62	0\\
14.63	0\\
14.64	0\\
14.65	0\\
14.66	0\\
14.67	0\\
14.68	0\\
14.69	0\\
14.7	0\\
14.71	0\\
14.72	0\\
14.73	0\\
14.74	0\\
14.75	0\\
14.76	0\\
14.77	0\\
14.78	0\\
14.79	0\\
14.8	0\\
14.81	0\\
14.82	0\\
14.83	0\\
14.84	0\\
14.85	0\\
14.86	0\\
14.87	0\\
14.88	0\\
14.89	0\\
14.9	0\\
14.91	0\\
14.92	0\\
14.93	0\\
14.94	0\\
14.95	0\\
14.96	0\\
14.97	0\\
14.98	0\\
14.99	0\\
15	0\\
15.01	0\\
15.02	0\\
15.03	0\\
15.04	0\\
15.05	0\\
15.06	0\\
15.07	0\\
15.08	0\\
15.09	0\\
15.1	0\\
15.11	0\\
15.12	0\\
15.13	0\\
15.14	0\\
15.15	0\\
15.16	0\\
15.17	0\\
15.18	0\\
15.19	0\\
15.2	0\\
15.21	0\\
15.22	0\\
15.23	0\\
15.24	0\\
15.25	0\\
15.26	0\\
15.27	0\\
15.28	0\\
15.29	0\\
15.3	0\\
15.31	0\\
15.32	0\\
15.33	0\\
15.34	0\\
15.35	0\\
15.36	0\\
15.37	0\\
15.38	0\\
15.39	0\\
15.4	0\\
15.41	0\\
15.42	0\\
15.43	0\\
15.44	0\\
15.45	0\\
15.46	0\\
15.47	0\\
15.48	0\\
15.49	0\\
15.5	0\\
15.51	0\\
15.52	0\\
15.53	0\\
15.54	0\\
15.55	0\\
15.56	0\\
15.57	0\\
15.58	0\\
15.59	0\\
15.6	0\\
15.61	0\\
15.62	0\\
15.63	0\\
15.64	0\\
15.65	0\\
15.66	0\\
15.67	0\\
15.68	0\\
15.69	0\\
15.7	0\\
15.71	0\\
15.72	0\\
15.73	0\\
15.74	0\\
15.75	0\\
15.76	0\\
15.77	0\\
15.78	0\\
15.79	0\\
15.8	0\\
15.81	0\\
15.82	0\\
15.83	0\\
15.84	0\\
15.85	0\\
15.86	0\\
15.87	0\\
15.88	0\\
15.89	0\\
15.9	0\\
15.91	0\\
15.92	0\\
15.93	0\\
15.94	0\\
15.95	0\\
15.96	0\\
15.97	0\\
15.98	0\\
15.99	0\\
16	0\\
16.01	0\\
16.02	0\\
16.03	0\\
16.04	0\\
16.05	0\\
16.06	0\\
16.07	0\\
16.08	0\\
16.09	0\\
16.1	0\\
16.11	0\\
16.12	0\\
16.13	0\\
16.14	0\\
16.15	0\\
16.16	0\\
16.17	0\\
16.18	0\\
16.19	0\\
16.2	0\\
16.21	0\\
16.22	0\\
16.23	0\\
16.24	0\\
16.25	0\\
16.26	0\\
16.27	0\\
16.28	0\\
16.29	0\\
16.3	0\\
16.31	0\\
16.32	0\\
16.33	0\\
16.34	0\\
16.35	0\\
16.36	0\\
16.37	0\\
16.38	0\\
16.39	0\\
16.4	0\\
16.41	0\\
16.42	0\\
16.43	0\\
16.44	0\\
16.45	0\\
16.46	0\\
16.47	0\\
16.48	0\\
16.49	0\\
16.5	0\\
16.51	0\\
16.52	0\\
16.53	0\\
16.54	0\\
16.55	0\\
16.56	0\\
16.57	0\\
16.58	0\\
16.59	0\\
16.6	0\\
16.61	0\\
16.62	0\\
16.63	0\\
16.64	0\\
16.65	0\\
16.66	0\\
16.67	0\\
16.68	0\\
16.69	0\\
16.7	0\\
16.71	0\\
16.72	0\\
16.73	0\\
16.74	0\\
16.75	0\\
16.76	0\\
16.77	0\\
16.78	0\\
16.79	0\\
16.8	0\\
16.81	0\\
16.82	0\\
16.83	0\\
16.84	0\\
16.85	0\\
16.86	0\\
16.87	0\\
16.88	0\\
16.89	0\\
16.9	0\\
16.91	0\\
16.92	0\\
16.93	0\\
16.94	0\\
16.95	0\\
16.96	0\\
16.97	0\\
16.98	0\\
16.99	0\\
17	0\\
17.01	0\\
17.02	0\\
17.03	0\\
17.04	0\\
17.05	0\\
17.06	0\\
17.07	0\\
17.08	0\\
17.09	0\\
17.1	0\\
17.11	0\\
17.12	0\\
17.13	0\\
17.14	0\\
17.15	0\\
17.16	0\\
17.17	0\\
17.18	0\\
17.19	0\\
17.2	0\\
17.21	0\\
17.22	0\\
17.23	0\\
17.24	0\\
17.25	0\\
17.26	0\\
17.27	0\\
17.28	0\\
17.29	0\\
17.3	0\\
17.31	0\\
17.32	0\\
17.33	0\\
17.34	0\\
17.35	0\\
17.36	0\\
17.37	0\\
17.38	0\\
17.39	0\\
17.4	0\\
17.41	0\\
17.42	0\\
17.43	0\\
17.44	0\\
17.45	0\\
17.46	0\\
17.47	0\\
17.48	0\\
17.49	0\\
17.5	0\\
17.51	0\\
17.52	0\\
17.53	0\\
17.54	0\\
17.55	0\\
17.56	0\\
17.57	0\\
17.58	0\\
17.59	0\\
17.6	0\\
17.61	0\\
17.62	0\\
17.63	0\\
17.64	0\\
17.65	0\\
17.66	0\\
17.67	0\\
17.68	0\\
17.69	0\\
17.7	0\\
17.71	0\\
17.72	0\\
17.73	0\\
17.74	0\\
17.75	0\\
17.76	0\\
17.77	0\\
17.78	0\\
17.79	0\\
17.8	0\\
17.81	0\\
17.82	0\\
17.83	0\\
17.84	0\\
17.85	0\\
17.86	0\\
17.87	0\\
17.88	0\\
17.89	0\\
17.9	0\\
17.91	0\\
17.92	0\\
17.93	0\\
17.94	0\\
17.95	0\\
17.96	0\\
17.97	0\\
17.98	0\\
17.99	0\\
18	0\\
18.01	0\\
18.02	0\\
18.03	0\\
18.04	0\\
18.05	0\\
18.06	0\\
18.07	0\\
18.08	0\\
18.09	0\\
18.1	0\\
18.11	0\\
18.12	0\\
18.13	0\\
18.14	0\\
18.15	0\\
18.16	0\\
18.17	0\\
18.18	0\\
18.19	0\\
18.2	0\\
18.21	0\\
18.22	0\\
18.23	0\\
18.24	0\\
18.25	0\\
18.26	0\\
18.27	0\\
18.28	0\\
18.29	0\\
18.3	0\\
18.31	0\\
18.32	0\\
18.33	0\\
18.34	0\\
18.35	0\\
18.36	0\\
18.37	0\\
18.38	0\\
18.39	0\\
18.4	0\\
18.41	0\\
18.42	0\\
18.43	0\\
18.44	0\\
18.45	0\\
18.46	0\\
18.47	0\\
18.48	0\\
18.49	0\\
18.5	0\\
18.51	0\\
18.52	0\\
18.53	0\\
18.54	0\\
18.55	0\\
18.56	0\\
18.57	0\\
18.58	0\\
18.59	0\\
18.6	0\\
18.61	0\\
18.62	0\\
18.63	0\\
18.64	0\\
18.65	0\\
18.66	0\\
18.67	0\\
18.68	0\\
18.69	0\\
18.7	0\\
18.71	0\\
18.72	0\\
18.73	0\\
18.74	0\\
18.75	0\\
18.76	0\\
18.77	0\\
18.78	0\\
18.79	0\\
18.8	0\\
18.81	0\\
18.82	0\\
18.83	0\\
18.84	0\\
18.85	0\\
18.86	0\\
18.87	0\\
18.88	0\\
18.89	0\\
18.9	0\\
18.91	0\\
18.92	0\\
18.93	0\\
18.94	0\\
18.95	0\\
18.96	0\\
18.97	0\\
18.98	0\\
18.99	0\\
19	0\\
19.01	0\\
19.02	0\\
19.03	0\\
19.04	0\\
19.05	0\\
19.06	0\\
19.07	0\\
19.08	0\\
19.09	0\\
19.1	0\\
19.11	0\\
19.12	0\\
19.13	0\\
19.14	0\\
19.15	0\\
19.16	0\\
19.17	0\\
19.18	0\\
19.19	0\\
19.2	0\\
19.21	0\\
19.22	0\\
19.23	0\\
19.24	0\\
19.25	0\\
19.26	0\\
19.27	0\\
19.28	0\\
19.29	0\\
19.3	0\\
19.31	0\\
19.32	0\\
19.33	0\\
19.34	0\\
19.35	0\\
19.36	0\\
19.37	0\\
19.38	0\\
19.39	0\\
19.4	0\\
19.41	0\\
19.42	0\\
19.43	0\\
19.44	0\\
19.45	0\\
19.46	0\\
19.47	0\\
19.48	0\\
19.49	0\\
19.5	0\\
19.51	0\\
19.52	0\\
19.53	0\\
19.54	0\\
19.55	0\\
19.56	0\\
19.57	0\\
19.58	0\\
19.59	0\\
19.6	0\\
19.61	0\\
19.62	0\\
19.63	0\\
19.64	0\\
19.65	0\\
19.66	0\\
19.67	0\\
19.68	0\\
19.69	0\\
19.7	0\\
19.71	0\\
19.72	0\\
19.73	0\\
19.74	0\\
19.75	0\\
19.76	0\\
19.77	0\\
19.78	0\\
19.79	0\\
19.8	0\\
19.81	0\\
19.82	0\\
19.83	0\\
19.84	0\\
19.85	0\\
19.86	0\\
19.87	0\\
19.88	0\\
19.89	0\\
19.9	0\\
19.91	0\\
19.92	0\\
19.93	0\\
19.94	0\\
19.95	0\\
19.96	0\\
19.97	0\\
19.98	0\\
19.99	0\\
20	0\\
20.01	0\\
20.02	0\\
20.03	0\\
20.04	0\\
20.05	0\\
20.06	0\\
20.07	0\\
20.08	0\\
20.09	0\\
20.1	0\\
20.11	0\\
20.12	0\\
20.13	0\\
20.14	0\\
20.15	0\\
20.16	0\\
20.17	0\\
20.18	0\\
20.19	0\\
20.2	0\\
20.21	0\\
20.22	0\\
20.23	0\\
20.24	0\\
20.25	0\\
20.26	0\\
20.27	0\\
20.28	0\\
20.29	0\\
20.3	0\\
20.31	0\\
20.32	0\\
20.33	0\\
20.34	0\\
20.35	0\\
20.36	0\\
20.37	0\\
20.38	0\\
20.39	0\\
20.4	0\\
20.41	0\\
20.42	0\\
20.43	0\\
20.44	0\\
20.45	0\\
20.46	0\\
20.47	0\\
20.48	0\\
20.49	0\\
20.5	0\\
20.51	0\\
20.52	0\\
20.53	0\\
20.54	0\\
20.55	0\\
20.56	0\\
20.57	0\\
20.58	0\\
20.59	0\\
20.6	0\\
20.61	0\\
20.62	0\\
20.63	0\\
20.64	0\\
20.65	0\\
20.66	0\\
20.67	0\\
20.68	0\\
20.69	0\\
20.7	0\\
20.71	0\\
20.72	0\\
20.73	0\\
20.74	0\\
20.75	0\\
20.76	0\\
20.77	0\\
20.78	0\\
20.79	0\\
20.8	0\\
20.81	0\\
20.82	0\\
20.83	0\\
20.84	0\\
20.85	0\\
20.86	0\\
20.87	0\\
20.88	0\\
20.89	0\\
20.9	0\\
20.91	0\\
20.92	0\\
20.93	0\\
20.94	0\\
20.95	0\\
20.96	0\\
20.97	0\\
20.98	0\\
20.99	0\\
21	0\\
21.01	0\\
21.02	0\\
21.03	0\\
21.04	0\\
21.05	0\\
21.06	0\\
21.07	0\\
21.08	0\\
21.09	0\\
21.1	0\\
21.11	0\\
21.12	0\\
21.13	0\\
21.14	0\\
21.15	0\\
21.16	0\\
21.17	0\\
21.18	0\\
21.19	0\\
21.2	0\\
21.21	0\\
21.22	0\\
21.23	0\\
21.24	0\\
21.25	0\\
21.26	0\\
21.27	0\\
21.28	0\\
21.29	0\\
21.3	0\\
21.31	0\\
21.32	0\\
21.33	0\\
21.34	0\\
21.35	0\\
21.36	0\\
21.37	0\\
21.38	0\\
21.39	0\\
21.4	0\\
21.41	0\\
21.42	0\\
21.43	0\\
21.44	0\\
21.45	0\\
21.46	0\\
21.47	0\\
21.48	0\\
21.49	0\\
21.5	0\\
21.51	0\\
21.52	0\\
21.53	0\\
21.54	0\\
21.55	0\\
21.56	0\\
21.57	0\\
21.58	0\\
21.59	0\\
21.6	0\\
21.61	0\\
21.62	0\\
21.63	0\\
21.64	0\\
21.65	0\\
21.66	0\\
21.67	0\\
21.68	0\\
21.69	0\\
21.7	0\\
21.71	0\\
21.72	0\\
21.73	0\\
21.74	0\\
21.75	0\\
21.76	0\\
21.77	0\\
21.78	0\\
21.79	0\\
21.8	0\\
21.81	0\\
21.82	0\\
21.83	0\\
21.84	0\\
21.85	0\\
21.86	0\\
21.87	0\\
21.88	0\\
21.89	0\\
21.9	0\\
21.91	0\\
21.92	0\\
21.93	0\\
21.94	0\\
21.95	0\\
21.96	0\\
21.97	0\\
21.98	0\\
21.99	0\\
22	0\\
22.01	0\\
22.02	0\\
22.03	0\\
22.04	0\\
22.05	0\\
22.06	0\\
22.07	0\\
22.08	0\\
22.09	0\\
22.1	0\\
22.11	0\\
22.12	0\\
22.13	0\\
22.14	0\\
22.15	0\\
22.16	0\\
22.17	0\\
22.18	0\\
22.19	0\\
22.2	0\\
22.21	0\\
22.22	0\\
22.23	0\\
22.24	0\\
22.25	0\\
22.26	0\\
22.27	0\\
22.28	0\\
22.29	0\\
22.3	0\\
22.31	0\\
22.32	0\\
22.33	0\\
22.34	0\\
22.35	0\\
22.36	0\\
22.37	0\\
22.38	0\\
22.39	0\\
22.4	0\\
22.41	0\\
22.42	0\\
22.43	0\\
22.44	0\\
22.45	0\\
22.46	0\\
22.47	0\\
22.48	0\\
22.49	0\\
22.5	0\\
22.51	0\\
22.52	0\\
22.53	0\\
22.54	0\\
22.55	0\\
22.56	0\\
22.57	0\\
22.58	0\\
22.59	0\\
22.6	0\\
22.61	0\\
22.62	0\\
22.63	0\\
22.64	0\\
22.65	0\\
22.66	0\\
22.67	0\\
22.68	0\\
22.69	0\\
22.7	0\\
22.71	0\\
22.72	0\\
22.73	0\\
22.74	0\\
22.75	0\\
22.76	0\\
22.77	0\\
22.78	0\\
22.79	0\\
22.8	0\\
22.81	0\\
22.82	0\\
22.83	0\\
22.84	0\\
22.85	0\\
22.86	0\\
22.87	0\\
22.88	0\\
22.89	0\\
22.9	0\\
22.91	0\\
22.92	0\\
22.93	0\\
22.94	0\\
22.95	0\\
22.96	0\\
22.97	0\\
22.98	0\\
22.99	0\\
23	0\\
23.01	0\\
23.02	0\\
23.03	0\\
23.04	0\\
23.05	0\\
23.06	0\\
23.07	0\\
23.08	0\\
23.09	0\\
23.1	0\\
23.11	0\\
23.12	0\\
23.13	0\\
23.14	0\\
23.15	0\\
23.16	0\\
23.17	0\\
23.18	0\\
23.19	0\\
23.2	0\\
23.21	0\\
23.22	0\\
23.23	0\\
23.24	0\\
23.25	0\\
23.26	0\\
23.27	0\\
23.28	0\\
23.29	0\\
23.3	0\\
23.31	0\\
23.32	0\\
23.33	0\\
23.34	0\\
23.35	0\\
23.36	0\\
23.37	0\\
23.38	0\\
23.39	0\\
23.4	0\\
23.41	0\\
23.42	0\\
23.43	0\\
23.44	0\\
23.45	0\\
23.46	0\\
23.47	0\\
23.48	0\\
23.49	0\\
23.5	0\\
23.51	0\\
23.52	0\\
23.53	0\\
23.54	0\\
23.55	0\\
23.56	0\\
23.57	0\\
23.58	0\\
23.59	0\\
23.6	0\\
23.61	0\\
23.62	0\\
23.63	0\\
23.64	0\\
23.65	0\\
23.66	0\\
23.67	0\\
23.68	0\\
23.69	0\\
23.7	0\\
23.71	0\\
23.72	0\\
23.73	0\\
23.74	0\\
23.75	0\\
23.76	0\\
23.77	0\\
23.78	0\\
23.79	0\\
23.8	0\\
23.81	0\\
23.82	0\\
23.83	0\\
23.84	0\\
23.85	0\\
23.86	0\\
23.87	0\\
23.88	0\\
23.89	0\\
23.9	0\\
23.91	0\\
23.92	0\\
23.93	0\\
23.94	0\\
23.95	0\\
23.96	0\\
23.97	0\\
23.98	0\\
23.99	0\\
24	0\\
24.01	0\\
24.02	0\\
24.03	0\\
24.04	0\\
24.05	0\\
24.06	0\\
24.07	0\\
24.08	0\\
24.09	0\\
24.1	0\\
24.11	0\\
24.12	0\\
24.13	0\\
24.14	0\\
24.15	0\\
24.16	0\\
24.17	0\\
24.18	0\\
24.19	0\\
24.2	0\\
24.21	0\\
24.22	0\\
24.23	0\\
24.24	0\\
24.25	0\\
24.26	0\\
24.27	0\\
24.28	0\\
24.29	0\\
24.3	0\\
24.31	0\\
24.32	0\\
24.33	0\\
24.34	0\\
24.35	0\\
24.36	0\\
24.37	0\\
24.38	0\\
24.39	0\\
24.4	0\\
24.41	0\\
24.42	0\\
24.43	0\\
24.44	0\\
24.45	0\\
24.46	0\\
24.47	0\\
24.48	0\\
24.49	0\\
24.5	0\\
24.51	0\\
24.52	0\\
24.53	0\\
24.54	0\\
24.55	0\\
24.56	0\\
24.57	0\\
24.58	0\\
24.59	0\\
24.6	0\\
24.61	0\\
24.62	0\\
24.63	0\\
24.64	0\\
24.65	0\\
24.66	0\\
24.67	0\\
24.68	0\\
24.69	0\\
24.7	0\\
24.71	0\\
24.72	0\\
24.73	0\\
24.74	0\\
24.75	0\\
24.76	0\\
24.77	0\\
24.78	0\\
24.79	0\\
24.8	0\\
24.81	0\\
24.82	0\\
24.83	0\\
24.84	0\\
24.85	0\\
24.86	0\\
24.87	0\\
24.88	0\\
24.89	0\\
24.9	0\\
24.91	0\\
24.92	0\\
24.93	0\\
24.94	0\\
24.95	0\\
24.96	0\\
24.97	0\\
24.98	0\\
24.99	0\\
25	0\\
25.01	0\\
25.02	0\\
25.03	0\\
25.04	0\\
25.05	0\\
25.06	0\\
25.07	0\\
25.08	0\\
25.09	0\\
25.1	0\\
25.11	0\\
25.12	0\\
25.13	0\\
25.14	0\\
25.15	0\\
25.16	0\\
25.17	0\\
25.18	0\\
25.19	0\\
25.2	0\\
25.21	0\\
25.22	0\\
25.23	0\\
25.24	0\\
25.25	0\\
25.26	0\\
25.27	0\\
25.28	0\\
25.29	0\\
25.3	0\\
25.31	0\\
25.32	0\\
25.33	0\\
25.34	0\\
25.35	0\\
25.36	0\\
25.37	0\\
25.38	0\\
25.39	0\\
25.4	0\\
25.41	0\\
25.42	0\\
25.43	0\\
25.44	0\\
25.45	0\\
25.46	0\\
25.47	0\\
25.48	0\\
25.49	0\\
25.5	0\\
25.51	0\\
25.52	0\\
25.53	0\\
25.54	0\\
25.55	0\\
25.56	0\\
25.57	0\\
25.58	0\\
25.59	0\\
25.6	0\\
25.61	0\\
25.62	0\\
25.63	0\\
25.64	0\\
25.65	0\\
25.66	0\\
25.67	0\\
25.68	0\\
25.69	0\\
25.7	0\\
25.71	0\\
25.72	0\\
25.73	0\\
25.74	0\\
25.75	0\\
25.76	0\\
25.77	0\\
25.78	0\\
25.79	0\\
25.8	0\\
25.81	0\\
25.82	0\\
25.83	0\\
25.84	0\\
25.85	0\\
25.86	0\\
25.87	0\\
25.88	0\\
25.89	0\\
25.9	0\\
25.91	0\\
25.92	0\\
25.93	0\\
25.94	0\\
25.95	0\\
25.96	0\\
25.97	0\\
25.98	0\\
25.99	0\\
26	0\\
26.01	0\\
26.02	0\\
26.03	0\\
26.04	0\\
26.05	0\\
26.06	0\\
26.07	0\\
26.08	0\\
26.09	0\\
26.1	0\\
26.11	0\\
26.12	0\\
26.13	0\\
26.14	0\\
26.15	0\\
26.16	0\\
26.17	0\\
26.18	0\\
26.19	0\\
26.2	0\\
26.21	0\\
26.22	0\\
26.23	0\\
26.24	0\\
26.25	0\\
26.26	0\\
26.27	0\\
26.28	0\\
26.29	0\\
26.3	0\\
26.31	0\\
26.32	0\\
26.33	0\\
26.34	0\\
26.35	0\\
26.36	0\\
26.37	0\\
26.38	0\\
26.39	0\\
26.4	0\\
26.41	0\\
26.42	0\\
26.43	0\\
26.44	0\\
26.45	0\\
26.46	0\\
26.47	0\\
26.48	0\\
26.49	0\\
26.5	0\\
26.51	0\\
26.52	0\\
26.53	0\\
26.54	0\\
26.55	0\\
26.56	0\\
26.57	0\\
26.58	0\\
26.59	0\\
26.6	0\\
26.61	0\\
26.62	0\\
26.63	0\\
26.64	0\\
26.65	0\\
26.66	0\\
26.67	0\\
26.68	0\\
26.69	0\\
26.7	0\\
26.71	0\\
26.72	0\\
26.73	0\\
26.74	0\\
26.75	0\\
26.76	0\\
26.77	0\\
26.78	0\\
26.79	0\\
26.8	0\\
26.81	0\\
26.82	0\\
26.83	0\\
26.84	0\\
26.85	0\\
26.86	0\\
26.87	0\\
26.88	0\\
26.89	0\\
26.9	0\\
26.91	0\\
26.92	0\\
26.93	0\\
26.94	0\\
26.95	0\\
26.96	0\\
26.97	0\\
26.98	0\\
26.99	0\\
27	0\\
27.01	0\\
27.02	0\\
27.03	0\\
27.04	0\\
27.05	0\\
27.06	0\\
27.07	0\\
27.08	0\\
27.09	0\\
27.1	0\\
27.11	0\\
27.12	0\\
27.13	0\\
27.14	0\\
27.15	0\\
27.16	0\\
27.17	0\\
27.18	0\\
27.19	0\\
27.2	0\\
27.21	0\\
27.22	0\\
27.23	0\\
27.24	0\\
27.25	0\\
27.26	0\\
27.27	0\\
27.28	0\\
27.29	0\\
27.3	0\\
27.31	0\\
27.32	0\\
27.33	0\\
27.34	0\\
27.35	0\\
27.36	0\\
27.37	0\\
27.38	0\\
27.39	0\\
27.4	0\\
27.41	0\\
27.42	0\\
27.43	0\\
27.44	0\\
27.45	0\\
27.46	0\\
27.47	0\\
27.48	0\\
27.49	0\\
27.5	0\\
27.51	0\\
27.52	0\\
27.53	0\\
27.54	0\\
27.55	0\\
27.56	0\\
27.57	0\\
27.58	0\\
27.59	0\\
27.6	0\\
27.61	0\\
27.62	0\\
27.63	0\\
27.64	0\\
27.65	0\\
27.66	0\\
27.67	0\\
27.68	0\\
27.69	0\\
27.7	0\\
27.71	0\\
27.72	0\\
27.73	0\\
27.74	0\\
27.75	0\\
27.76	0\\
27.77	0\\
27.78	0\\
27.79	0\\
27.8	0\\
27.81	0\\
27.82	0\\
27.83	0\\
27.84	0\\
27.85	0\\
27.86	0\\
27.87	0\\
27.88	0\\
27.89	0\\
27.9	0\\
27.91	0\\
27.92	0\\
27.93	0\\
27.94	0\\
27.95	0\\
27.96	0\\
27.97	0\\
27.98	0\\
27.99	0\\
28	0\\
28.01	0\\
28.02	0\\
28.03	0\\
28.04	0\\
28.05	0\\
28.06	0\\
28.07	0\\
28.08	0\\
28.09	0\\
28.1	0\\
28.11	0\\
28.12	0\\
28.13	0\\
28.14	0\\
28.15	0\\
28.16	0\\
28.17	0\\
28.18	0\\
28.19	0\\
28.2	0\\
28.21	0\\
28.22	0\\
28.23	0\\
28.24	0\\
28.25	0\\
28.26	0\\
28.27	0\\
28.28	0\\
28.29	0\\
28.3	0\\
28.31	0\\
28.32	0\\
28.33	0\\
28.34	0\\
28.35	0\\
28.36	0\\
28.37	0\\
28.38	0\\
28.39	0\\
28.4	0\\
28.41	0\\
28.42	0\\
28.43	0\\
28.44	0\\
28.45	0\\
28.46	0\\
28.47	0\\
28.48	0\\
28.49	0\\
28.5	0\\
28.51	0\\
28.52	0\\
28.53	0\\
28.54	0\\
28.55	0\\
28.56	0\\
28.57	0\\
28.58	0\\
28.59	0\\
28.6	0\\
28.61	0\\
28.62	0\\
28.63	0\\
28.64	0\\
28.65	0\\
28.66	0\\
28.67	0\\
28.68	0\\
28.69	0\\
28.7	0\\
28.71	0\\
28.72	0\\
28.73	0\\
28.74	0\\
28.75	0\\
28.76	0\\
28.77	0\\
28.78	0\\
28.79	0\\
28.8	0\\
28.81	0\\
28.82	0\\
28.83	0\\
28.84	0\\
28.85	0\\
28.86	0\\
28.87	0\\
28.88	0\\
28.89	0\\
28.9	0\\
28.91	0\\
28.92	0\\
28.93	0\\
28.94	0\\
28.95	0\\
28.96	0\\
28.97	0\\
28.98	0\\
28.99	0\\
29	0\\
29.01	0\\
29.02	0\\
29.03	0\\
29.04	0\\
29.05	0\\
29.06	0\\
29.07	0\\
29.08	0\\
29.09	0\\
29.1	0\\
29.11	0\\
29.12	0\\
29.13	0\\
29.14	0\\
29.15	0\\
29.16	0\\
29.17	0\\
29.18	0\\
29.19	0\\
29.2	0\\
29.21	0\\
29.22	0\\
29.23	0\\
29.24	0\\
29.25	0\\
29.26	0\\
29.27	0\\
29.28	0\\
29.29	0\\
29.3	0\\
29.31	0\\
29.32	0\\
29.33	0\\
29.34	0\\
29.35	0\\
29.36	0\\
29.37	0\\
29.38	0\\
29.39	0\\
29.4	0\\
29.41	0\\
29.42	0\\
29.43	0\\
29.44	0\\
29.45	0\\
29.46	0\\
29.47	0\\
29.48	0\\
29.49	0\\
29.5	0\\
29.51	0\\
29.52	0\\
29.53	0\\
29.54	0\\
29.55	0\\
29.56	0\\
29.57	0\\
29.58	0\\
29.59	0\\
29.6	0\\
29.61	0\\
29.62	0\\
29.63	0\\
29.64	0\\
29.65	0\\
29.66	0\\
29.67	0\\
29.68	0\\
29.69	0\\
29.7	0\\
29.71	0\\
29.72	0\\
29.73	0\\
29.74	0\\
29.75	0\\
29.76	0\\
29.77	0\\
29.78	0\\
29.79	0\\
29.8	0\\
29.81	0\\
29.82	0\\
29.83	0\\
29.84	0\\
29.85	0\\
29.86	0\\
29.87	0\\
29.88	0\\
29.89	0\\
29.9	0\\
29.91	0\\
29.92	0\\
29.93	0\\
29.94	0\\
29.95	0\\
29.96	0\\
29.97	0\\
29.98	0\\
29.99	0\\
30	0\\
30.01	0\\
30.02	0\\
30.03	0\\
30.04	0\\
30.05	0\\
30.06	0\\
30.07	0\\
30.08	0\\
30.09	0\\
30.1	0\\
30.11	0\\
30.12	0\\
30.13	0\\
30.14	0\\
30.15	0\\
30.16	0\\
30.17	0\\
30.18	0\\
30.19	0\\
30.2	0\\
30.21	0\\
30.22	0\\
30.23	0\\
30.24	0\\
30.25	0\\
30.26	0\\
30.27	0\\
30.28	0\\
30.29	0\\
30.3	0\\
30.31	0\\
30.32	0\\
30.33	0\\
30.34	0\\
30.35	0\\
30.36	0\\
30.37	0\\
30.38	0\\
30.39	0\\
30.4	0\\
30.41	0\\
30.42	0\\
30.43	0\\
30.44	0\\
30.45	0\\
30.46	0\\
30.47	0\\
30.48	0\\
30.49	0\\
30.5	0\\
30.51	0\\
30.52	0\\
30.53	0\\
30.54	0\\
30.55	0\\
30.56	0\\
30.57	0\\
30.58	0\\
30.59	0\\
30.6	0\\
30.61	0\\
30.62	0\\
30.63	0\\
30.64	0\\
30.65	0\\
30.66	0\\
30.67	0\\
30.68	0\\
30.69	0\\
30.7	0\\
30.71	0\\
30.72	0\\
30.73	0\\
30.74	0\\
30.75	0\\
30.76	0\\
30.77	0\\
30.78	0\\
30.79	0\\
30.8	0\\
30.81	0\\
30.82	0\\
30.83	0\\
30.84	0\\
30.85	0\\
30.86	0\\
30.87	0\\
30.88	0\\
30.89	0\\
30.9	0\\
30.91	0\\
30.92	0\\
30.93	0\\
30.94	0\\
30.95	0\\
30.96	0\\
30.97	0\\
30.98	0\\
30.99	0\\
31	0\\
31.01	0\\
31.02	0\\
31.03	0\\
31.04	0\\
31.05	0\\
31.06	0\\
31.07	0\\
31.08	0\\
31.09	0\\
31.1	0\\
31.11	0\\
31.12	0\\
31.13	0\\
31.14	0\\
31.15	0\\
31.16	0\\
31.17	0\\
31.18	0\\
31.19	0\\
31.2	0\\
31.21	0\\
31.22	0\\
31.23	0\\
31.24	0\\
31.25	0\\
31.26	0\\
31.27	0\\
31.28	0\\
31.29	0\\
31.3	0\\
31.31	0\\
31.32	0\\
31.33	0\\
31.34	0\\
31.35	0\\
31.36	0\\
31.37	0\\
31.38	0\\
31.39	0\\
31.4	0\\
31.41	0\\
31.42	0\\
31.43	0\\
31.44	0\\
31.45	0\\
31.46	0\\
31.47	0\\
31.48	0\\
31.49	0\\
31.5	0\\
31.51	0\\
31.52	0\\
31.53	0\\
31.54	0\\
31.55	0\\
31.56	0\\
31.57	0\\
31.58	0\\
31.59	0\\
31.6	0\\
31.61	0\\
31.62	0\\
31.63	0\\
31.64	0\\
31.65	0\\
31.66	0\\
31.67	0\\
31.68	0\\
31.69	0\\
31.7	0\\
31.71	0\\
31.72	0\\
31.73	0\\
31.74	0\\
31.75	0\\
31.76	0\\
31.77	0\\
31.78	0\\
31.79	0\\
31.8	0\\
31.81	0\\
31.82	0\\
31.83	0\\
31.84	0\\
31.85	0\\
31.86	0\\
31.87	0\\
31.88	0\\
31.89	0\\
31.9	0\\
31.91	0\\
31.92	0\\
31.93	0\\
31.94	0\\
31.95	0\\
31.96	0\\
31.97	0\\
31.98	0\\
31.99	0\\
32	0\\
32.01	0\\
32.02	0\\
32.03	0\\
32.04	0\\
32.05	0\\
32.06	0\\
32.07	0\\
32.08	0\\
32.09	0\\
32.1	0\\
32.11	0\\
32.12	0\\
32.13	0\\
32.14	0\\
32.15	0\\
32.16	0\\
32.17	0\\
32.18	0\\
32.19	0\\
32.2	0\\
32.21	0\\
32.22	0\\
32.23	0\\
32.24	0\\
32.25	0\\
32.26	0\\
32.27	0\\
32.28	0\\
32.29	0\\
32.3	0\\
32.31	0\\
32.32	0\\
32.33	0\\
32.34	0\\
32.35	0\\
32.36	0\\
32.37	0\\
32.38	0\\
32.39	0\\
32.4	0\\
32.41	0\\
32.42	0\\
32.43	0\\
32.44	0\\
32.45	0\\
32.46	0\\
32.47	0\\
32.48	0\\
32.49	0\\
32.5	0\\
32.51	0\\
32.52	0\\
32.53	0\\
32.54	0\\
32.55	0\\
32.56	0\\
32.57	0\\
32.58	0\\
32.59	0\\
32.6	0\\
32.61	0\\
32.62	0\\
32.63	0\\
32.64	0\\
32.65	0\\
32.66	0\\
32.67	0\\
32.68	0\\
32.69	0\\
32.7	0\\
32.71	0\\
32.72	0\\
32.73	0\\
32.74	0\\
32.75	0\\
32.76	0\\
32.77	0\\
32.78	0\\
32.79	0\\
32.8	0\\
32.81	0\\
32.82	0\\
32.83	0\\
32.84	0\\
32.85	0\\
32.86	0\\
32.87	0\\
32.88	0\\
32.89	0\\
32.9	0\\
32.91	0\\
32.92	0\\
32.93	0\\
32.94	0\\
32.95	0\\
32.96	0\\
32.97	0\\
32.98	0\\
32.99	0\\
33	0\\
33.01	0\\
33.02	0\\
33.03	0\\
33.04	0\\
33.05	0\\
33.06	0\\
33.07	0\\
33.08	0\\
33.09	0\\
33.1	0\\
33.11	0\\
33.12	0\\
33.13	0\\
33.14	0\\
33.15	0\\
33.16	0\\
33.17	0\\
33.18	0\\
33.19	0\\
33.2	0\\
33.21	0\\
33.22	0\\
33.23	0\\
33.24	0\\
33.25	0\\
33.26	0\\
33.27	0\\
33.28	0\\
33.29	0\\
33.3	0\\
33.31	0\\
33.32	0\\
33.33	0\\
33.34	0\\
33.35	0\\
33.36	0\\
33.37	0\\
33.38	0\\
33.39	0\\
33.4	0\\
33.41	0\\
33.42	0\\
33.43	0\\
33.44	0\\
33.45	0\\
33.46	0\\
33.47	0\\
33.48	0\\
33.49	0\\
33.5	0\\
33.51	0\\
33.52	0\\
33.53	0\\
33.54	0\\
33.55	0\\
33.56	0\\
33.57	0\\
33.58	0\\
33.59	0\\
33.6	0\\
33.61	0\\
33.62	0\\
33.63	0\\
33.64	0\\
33.65	0\\
33.66	0\\
33.67	0\\
33.68	0\\
33.69	0\\
33.7	0\\
33.71	0\\
33.72	0\\
33.73	0\\
33.74	0\\
33.75	0\\
33.76	0\\
33.77	0\\
33.78	0\\
33.79	0\\
33.8	0\\
33.81	0\\
33.82	0\\
33.83	0\\
33.84	0\\
33.85	0\\
33.86	0\\
33.87	0\\
33.88	0\\
33.89	0\\
33.9	0\\
33.91	0\\
33.92	0\\
33.93	0\\
33.94	0\\
33.95	0\\
33.96	0\\
33.97	0\\
33.98	0\\
33.99	0\\
34	0\\
34.01	0\\
34.02	0\\
34.03	0\\
34.04	0\\
34.05	0\\
34.06	0\\
34.07	0\\
34.08	0\\
34.09	0\\
34.1	0\\
34.11	0\\
34.12	0\\
34.13	0\\
34.14	0\\
34.15	0\\
34.16	0\\
34.17	0\\
34.18	0\\
34.19	0\\
34.2	0\\
34.21	0\\
34.22	0\\
34.23	0\\
34.24	0\\
34.25	0\\
34.26	0\\
34.27	0\\
34.28	0\\
34.29	0\\
34.3	0\\
34.31	0\\
34.32	0\\
34.33	0\\
34.34	0\\
34.35	0\\
34.36	0\\
34.37	0\\
34.38	0\\
34.39	0\\
34.4	0\\
34.41	0\\
34.42	0\\
34.43	0\\
34.44	0\\
34.45	0\\
34.46	0\\
34.47	0\\
34.48	0\\
34.49	0\\
34.5	0\\
34.51	0\\
34.52	0\\
34.53	0\\
34.54	0\\
34.55	0\\
34.56	0\\
34.57	0\\
34.58	0\\
34.59	0\\
34.6	0\\
34.61	0\\
34.62	0\\
34.63	0\\
34.64	0\\
34.65	0\\
34.66	0\\
34.67	0\\
34.68	0\\
34.69	0\\
34.7	0\\
34.71	0\\
34.72	0\\
34.73	0\\
34.74	0\\
34.75	0\\
34.76	0\\
34.77	0\\
34.78	0\\
34.79	0\\
34.8	0\\
34.81	0\\
34.82	0\\
34.83	0\\
34.84	0\\
34.85	0\\
34.86	0\\
34.87	0\\
34.88	0\\
34.89	0\\
34.9	0\\
34.91	0\\
34.92	0\\
34.93	0\\
34.94	0\\
34.95	0\\
34.96	0\\
34.97	0\\
34.98	0\\
34.99	0\\
35	0\\
35.01	0\\
35.02	0\\
35.03	0\\
35.04	0\\
35.05	0\\
35.06	0\\
35.07	0\\
35.08	0\\
35.09	0\\
35.1	0\\
35.11	0\\
35.12	0\\
35.13	0\\
35.14	0\\
35.15	0\\
35.16	0\\
35.17	0\\
35.18	0\\
35.19	0\\
35.2	0\\
35.21	0\\
35.22	0\\
35.23	0\\
35.24	0\\
35.25	0\\
35.26	0\\
35.27	0\\
35.28	0\\
35.29	0\\
35.3	0\\
35.31	0\\
35.32	0\\
35.33	0\\
35.34	0\\
35.35	0\\
35.36	0\\
35.37	0\\
35.38	0\\
35.39	0\\
35.4	0\\
35.41	0\\
35.42	0\\
35.43	0\\
35.44	0\\
35.45	0\\
35.46	0\\
35.47	0\\
35.48	0\\
35.49	0\\
35.5	0\\
35.51	0\\
35.52	0\\
35.53	0\\
35.54	0\\
35.55	0\\
35.56	0\\
35.57	0\\
35.58	0\\
35.59	0\\
35.6	0\\
35.61	0\\
35.62	0\\
35.63	0\\
35.64	0\\
35.65	0\\
35.66	0\\
35.67	0\\
35.68	0\\
35.69	0\\
35.7	0\\
35.71	0\\
35.72	0\\
35.73	0\\
35.74	0\\
35.75	0\\
35.76	0\\
35.77	0\\
35.78	0\\
35.79	0\\
35.8	0\\
35.81	0\\
35.82	0\\
35.83	0\\
35.84	0\\
35.85	0\\
35.86	0\\
35.87	0\\
35.88	0\\
35.89	0\\
35.9	0\\
35.91	0\\
35.92	0\\
35.93	0\\
35.94	0\\
35.95	0\\
35.96	0\\
35.97	0\\
35.98	0\\
35.99	0\\
36	0\\
36.01	0\\
36.02	0\\
36.03	0\\
36.04	0\\
36.05	0\\
36.06	0\\
36.07	0\\
36.08	0\\
36.09	0\\
36.1	0\\
36.11	0\\
36.12	0\\
36.13	0\\
36.14	0\\
36.15	0\\
36.16	0\\
36.17	0\\
36.18	0\\
36.19	0\\
36.2	0\\
36.21	0\\
36.22	0\\
36.23	0\\
36.24	0\\
36.25	0\\
36.26	0\\
36.27	0\\
36.28	0\\
36.29	0\\
36.3	0\\
36.31	0\\
36.32	0\\
36.33	0\\
36.34	0\\
36.35	0\\
36.36	0\\
36.37	0\\
36.38	0\\
36.39	0\\
36.4	0\\
36.41	0\\
36.42	0\\
36.43	0\\
36.44	0\\
36.45	0\\
36.46	0\\
36.47	0\\
36.48	0\\
36.49	0\\
36.5	0\\
36.51	0\\
36.52	0\\
36.53	0\\
36.54	0\\
36.55	0\\
36.56	0\\
36.57	0\\
36.58	0\\
36.59	0\\
36.6	0\\
36.61	0\\
36.62	0\\
36.63	0\\
36.64	0\\
36.65	0\\
36.66	0\\
36.67	0\\
36.68	0\\
36.69	0\\
36.7	0\\
36.71	0\\
36.72	0\\
36.73	0\\
36.74	0\\
36.75	0\\
36.76	0\\
36.77	0\\
36.78	0\\
36.79	0\\
36.8	0\\
36.81	0\\
36.82	0\\
36.83	0\\
36.84	0\\
36.85	0\\
36.86	0\\
36.87	0\\
36.88	0\\
36.89	0\\
36.9	0\\
36.91	0\\
36.92	0\\
36.93	0\\
36.94	0\\
36.95	0\\
36.96	0\\
36.97	0\\
36.98	0\\
36.99	0\\
37	0\\
37.01	0\\
37.02	0\\
37.03	0\\
37.04	0\\
37.05	0\\
37.06	0\\
37.07	0\\
37.08	0\\
37.09	0\\
37.1	0\\
37.11	0\\
37.12	0\\
37.13	0\\
37.14	0\\
37.15	0\\
37.16	0\\
37.17	0\\
37.18	0\\
37.19	0\\
37.2	0\\
37.21	0\\
37.22	0\\
37.23	0\\
37.24	0\\
37.25	0\\
37.26	0\\
37.27	0\\
37.28	0\\
37.29	0\\
37.3	0\\
37.31	0\\
37.32	0\\
37.33	0\\
37.34	0\\
37.35	0\\
37.36	0\\
37.37	0\\
37.38	0\\
37.39	0\\
37.4	0\\
37.41	0\\
37.42	0\\
37.43	0\\
37.44	0\\
37.45	0\\
37.46	0\\
37.47	0\\
37.48	0\\
37.49	0\\
37.5	0\\
37.51	0\\
37.52	0\\
37.53	0\\
37.54	0\\
37.55	0\\
37.56	0\\
37.57	0\\
37.58	0\\
37.59	0\\
37.6	0\\
37.61	0\\
37.62	0\\
37.63	0\\
37.64	0\\
37.65	0\\
37.66	0\\
37.67	0\\
37.68	0\\
37.69	0\\
37.7	0\\
37.71	0\\
37.72	0\\
37.73	0\\
37.74	0\\
37.75	0\\
37.76	0\\
37.77	0\\
37.78	0\\
37.79	0\\
37.8	0\\
37.81	0\\
37.82	0\\
37.83	0\\
37.84	0\\
37.85	0\\
37.86	0\\
37.87	0\\
37.88	0\\
37.89	0\\
37.9	0\\
37.91	0\\
37.92	0\\
37.93	0\\
37.94	0\\
37.95	0\\
37.96	0\\
37.97	0\\
37.98	0\\
37.99	0\\
38	0\\
38.01	0\\
38.02	0\\
38.03	0\\
38.04	0\\
38.05	0\\
38.06	0\\
38.07	0\\
38.08	0\\
38.09	0\\
38.1	0\\
38.11	0\\
38.12	0\\
38.13	0\\
38.14	0\\
38.15	0\\
38.16	0\\
38.17	0\\
38.18	0\\
38.19	0\\
38.2	0\\
38.21	0\\
38.22	0\\
38.23	0\\
38.24	0\\
38.25	0\\
38.26	0\\
38.27	0\\
38.28	0\\
38.29	0\\
38.3	0\\
38.31	0\\
38.32	0\\
38.33	0\\
38.34	0\\
38.35	0\\
38.36	0\\
38.37	0\\
38.38	0\\
38.39	0\\
38.4	0\\
38.41	0\\
38.42	0\\
38.43	0\\
38.44	0\\
38.45	0\\
38.46	0\\
38.47	0\\
38.48	0\\
38.49	0\\
38.5	0\\
38.51	0\\
38.52	0\\
38.53	0\\
38.54	0\\
38.55	0\\
38.56	0\\
38.57	0\\
38.58	0\\
38.59	0\\
38.6	0\\
38.61	0\\
38.62	0\\
38.63	0\\
38.64	0\\
38.65	0\\
38.66	0\\
38.67	0\\
38.68	0\\
38.69	0\\
38.7	0\\
38.71	0\\
38.72	0\\
38.73	0\\
38.74	0\\
38.75	0\\
38.76	0\\
38.77	0\\
38.78	0\\
38.79	0\\
38.8	0\\
38.81	0\\
38.82	0\\
38.83	0\\
38.84	0\\
38.85	0\\
38.86	0\\
38.87	0\\
38.88	0\\
38.89	0\\
38.9	0\\
38.91	0\\
38.92	0\\
38.93	0\\
38.94	0\\
38.95	0\\
38.96	0\\
38.97	0\\
38.98	0\\
38.99	0\\
39	0\\
39.01	0\\
39.02	0\\
39.03	0\\
39.04	0\\
39.05	0\\
39.06	0\\
39.07	0\\
39.08	0\\
39.09	0\\
39.1	0\\
39.11	0\\
39.12	0\\
39.13	0\\
39.14	0\\
39.15	0\\
39.16	0\\
39.17	0\\
39.18	0\\
39.19	0\\
39.2	0\\
39.21	0\\
39.22	0\\
39.23	0\\
39.24	0\\
39.25	0\\
39.26	0\\
39.27	0\\
39.28	0\\
39.29	0\\
39.3	0\\
39.31	0\\
39.32	0\\
39.33	0\\
39.34	0\\
39.35	0\\
39.36	0\\
39.37	0\\
39.38	0\\
39.39	0\\
39.4	0\\
39.41	0\\
39.42	0\\
39.43	0\\
39.44	0\\
39.45	0\\
39.46	0\\
39.47	0\\
39.48	0\\
39.49	0\\
39.5	0\\
39.51	0\\
39.52	0\\
39.53	0\\
39.54	0\\
39.55	0\\
39.56	0\\
39.57	0\\
39.58	0\\
39.59	0\\
39.6	0\\
39.61	0\\
39.62	0\\
39.63	0\\
39.64	0\\
39.65	0\\
39.66	0\\
39.67	0\\
39.68	0\\
39.69	0\\
39.7	0\\
39.71	0\\
39.72	0\\
39.73	0\\
39.74	0\\
39.75	0\\
39.76	0\\
39.77	0\\
39.78	0\\
39.79	0\\
39.8	0\\
39.81	0\\
39.82	0\\
39.83	0\\
39.84	0\\
39.85	0\\
39.86	0\\
39.87	0\\
39.88	0\\
39.89	0\\
39.9	0\\
39.91	0\\
39.92	0\\
39.93	0\\
39.94	0\\
39.95	0\\
39.96	0\\
39.97	0\\
39.98	0\\
39.99	0\\
40	0\\
40.01	0\\
};
\addplot [color=mycolor1,solid,forget plot]
  table[row sep=crcr]{%
40.01	0\\
40.02	0\\
40.03	0\\
40.04	0\\
40.05	0\\
40.06	0\\
40.07	0\\
40.08	0\\
40.09	0\\
40.1	0\\
40.11	0\\
40.12	0\\
40.13	0\\
40.14	0\\
40.15	0\\
40.16	0\\
40.17	0\\
40.18	0\\
40.19	0\\
40.2	0\\
40.21	0\\
40.22	0\\
40.23	0\\
40.24	0\\
40.25	0\\
40.26	0\\
40.27	0\\
40.28	0\\
40.29	0\\
40.3	0\\
40.31	0\\
40.32	0\\
40.33	0\\
40.34	0\\
40.35	0\\
40.36	0\\
40.37	0\\
40.38	0\\
40.39	0\\
40.4	0\\
40.41	0\\
40.42	0\\
40.43	0\\
40.44	0\\
40.45	0\\
40.46	0\\
40.47	0\\
40.48	0\\
40.49	0\\
40.5	0\\
40.51	0\\
40.52	0\\
40.53	0\\
40.54	0\\
40.55	0\\
40.56	0\\
40.57	0\\
40.58	0\\
40.59	0\\
40.6	0\\
40.61	0\\
40.62	0\\
40.63	0\\
40.64	0\\
40.65	0\\
40.66	0\\
40.67	0\\
40.68	0\\
40.69	0\\
40.7	0\\
40.71	0\\
40.72	0\\
40.73	0\\
40.74	0\\
40.75	0\\
40.76	0\\
40.77	0\\
40.78	0\\
40.79	0\\
40.8	0\\
40.81	0\\
40.82	0\\
40.83	0\\
40.84	0\\
40.85	0\\
40.86	0\\
40.87	0\\
40.88	0\\
40.89	0\\
40.9	0\\
40.91	0\\
40.92	0\\
40.93	0\\
40.94	0\\
40.95	0\\
40.96	0\\
40.97	0\\
40.98	0\\
40.99	0\\
41	0\\
41.01	0\\
41.02	0\\
41.03	0\\
41.04	0\\
41.05	0\\
41.06	0\\
41.07	0\\
41.08	0\\
41.09	0\\
41.1	0\\
41.11	0\\
41.12	0\\
41.13	0\\
41.14	0\\
41.15	0\\
41.16	0\\
41.17	0\\
41.18	0\\
41.19	0\\
41.2	0\\
41.21	0\\
41.22	0\\
41.23	0\\
41.24	0\\
41.25	0\\
41.26	0\\
41.27	0\\
41.28	0\\
41.29	0\\
41.3	0\\
41.31	0\\
41.32	0\\
41.33	0\\
41.34	0\\
41.35	0\\
41.36	0\\
41.37	0\\
41.38	0\\
41.39	0\\
41.4	0\\
41.41	0\\
41.42	0\\
41.43	0\\
41.44	0\\
41.45	0\\
41.46	0\\
41.47	0\\
41.48	0\\
41.49	0\\
41.5	0\\
41.51	0\\
41.52	0\\
41.53	0\\
41.54	0\\
41.55	0\\
41.56	0\\
41.57	0\\
41.58	0\\
41.59	0\\
41.6	0\\
41.61	0\\
41.62	0\\
41.63	0\\
41.64	0\\
41.65	0\\
41.66	0\\
41.67	0\\
41.68	0\\
41.69	0\\
41.7	0\\
41.71	0\\
41.72	0\\
41.73	0\\
41.74	0\\
41.75	0\\
41.76	0\\
41.77	0\\
41.78	0\\
41.79	0\\
41.8	0\\
41.81	0\\
41.82	0\\
41.83	0\\
41.84	0\\
41.85	0\\
41.86	0\\
41.87	0\\
41.88	0\\
41.89	0\\
41.9	0\\
41.91	0\\
41.92	0\\
41.93	0\\
41.94	0\\
41.95	0\\
41.96	0\\
41.97	0\\
41.98	0\\
41.99	0\\
42	0\\
42.01	0\\
42.02	0\\
42.03	0\\
42.04	0\\
42.05	0\\
42.06	0\\
42.07	0\\
42.08	0\\
42.09	0\\
42.1	0\\
42.11	0\\
42.12	0\\
42.13	0\\
42.14	0\\
42.15	0\\
42.16	0\\
42.17	0\\
42.18	0\\
42.19	0\\
42.2	0\\
42.21	0\\
42.22	0\\
42.23	0\\
42.24	0\\
42.25	0\\
42.26	0\\
42.27	0\\
42.28	0\\
42.29	0\\
42.3	0\\
42.31	0\\
42.32	0\\
42.33	0\\
42.34	0\\
42.35	0\\
42.36	0\\
42.37	0\\
42.38	0\\
42.39	0\\
42.4	0\\
42.41	0\\
42.42	0\\
42.43	0\\
42.44	0\\
42.45	0\\
42.46	0\\
42.47	0\\
42.48	0\\
42.49	0\\
42.5	0\\
42.51	0\\
42.52	0\\
42.53	0\\
42.54	0\\
42.55	0\\
42.56	0\\
42.57	0\\
42.58	0\\
42.59	0\\
42.6	0\\
42.61	0\\
42.62	0\\
42.63	0\\
42.64	0\\
42.65	0\\
42.66	0\\
42.67	0\\
42.68	0\\
42.69	0\\
42.7	0\\
42.71	0\\
42.72	0\\
42.73	0\\
42.74	0\\
42.75	0\\
42.76	0\\
42.77	0\\
42.78	0\\
42.79	0\\
42.8	0\\
42.81	0\\
42.82	0\\
42.83	0\\
42.84	0\\
42.85	0\\
42.86	0\\
42.87	0\\
42.88	0\\
42.89	0\\
42.9	0\\
42.91	0\\
42.92	0\\
42.93	0\\
42.94	0\\
42.95	0\\
42.96	0\\
42.97	0\\
42.98	0\\
42.99	0\\
43	0\\
43.01	0\\
43.02	0\\
43.03	0\\
43.04	0\\
43.05	0\\
43.06	0\\
43.07	0\\
43.08	0\\
43.09	0\\
43.1	0\\
43.11	0\\
43.12	0\\
43.13	0\\
43.14	0\\
43.15	0\\
43.16	0\\
43.17	0\\
43.18	0\\
43.19	0\\
43.2	0\\
43.21	0\\
43.22	0\\
43.23	0\\
43.24	0\\
43.25	0\\
43.26	0\\
43.27	0\\
43.28	0\\
43.29	0\\
43.3	0\\
43.31	0\\
43.32	0\\
43.33	0\\
43.34	0\\
43.35	0\\
43.36	0\\
43.37	0\\
43.38	0\\
43.39	0\\
43.4	0\\
43.41	0\\
43.42	0\\
43.43	0\\
43.44	0\\
43.45	0\\
43.46	0\\
43.47	0\\
43.48	0\\
43.49	0\\
43.5	0\\
43.51	0\\
43.52	0\\
43.53	0\\
43.54	0\\
43.55	0\\
43.56	0\\
43.57	0\\
43.58	0\\
43.59	0\\
43.6	0\\
43.61	0\\
43.62	0\\
43.63	0\\
43.64	0\\
43.65	0\\
43.66	0\\
43.67	0\\
43.68	0\\
43.69	0\\
43.7	0\\
43.71	0\\
43.72	0\\
43.73	0\\
43.74	0\\
43.75	0\\
43.76	0\\
43.77	0\\
43.78	0\\
43.79	0\\
43.8	0\\
43.81	0\\
43.82	0\\
43.83	0\\
43.84	0\\
43.85	0\\
43.86	0\\
43.87	0\\
43.88	0\\
43.89	0\\
43.9	0\\
43.91	0\\
43.92	0\\
43.93	0\\
43.94	0\\
43.95	0\\
43.96	0\\
43.97	0\\
43.98	0\\
43.99	0\\
44	0\\
44.01	0\\
44.02	0\\
44.03	0\\
44.04	0\\
44.05	0\\
44.06	0\\
44.07	0\\
44.08	0\\
44.09	0\\
44.1	0\\
44.11	0\\
44.12	0\\
44.13	0\\
44.14	0\\
44.15	0\\
44.16	0\\
44.17	0\\
44.18	0\\
44.19	0\\
44.2	0\\
44.21	0\\
44.22	0\\
44.23	0\\
44.24	0\\
44.25	0\\
44.26	0\\
44.27	0\\
44.28	0\\
44.29	0\\
44.3	0\\
44.31	0\\
44.32	0\\
44.33	0\\
44.34	0\\
44.35	0\\
44.36	0\\
44.37	0\\
44.38	0\\
44.39	0\\
44.4	0\\
44.41	0\\
44.42	0\\
44.43	0\\
44.44	0\\
44.45	0\\
44.46	0\\
44.47	0\\
44.48	0\\
44.49	0\\
44.5	0\\
44.51	0\\
44.52	0\\
44.53	0\\
44.54	0\\
44.55	0\\
44.56	0\\
44.57	0\\
44.58	0\\
44.59	0\\
44.6	0\\
44.61	0\\
44.62	0\\
44.63	0\\
44.64	0\\
44.65	0\\
44.66	0\\
44.67	0\\
44.68	0\\
44.69	0\\
44.7	0\\
44.71	0\\
44.72	0\\
44.73	0\\
44.74	0\\
44.75	0\\
44.76	0\\
44.77	0\\
44.78	0\\
44.79	0\\
44.8	0\\
44.81	0\\
44.82	0\\
44.83	0\\
44.84	0\\
44.85	0\\
44.86	0\\
44.87	0\\
44.88	0\\
44.89	0\\
44.9	0\\
44.91	0\\
44.92	0\\
44.93	0\\
44.94	0\\
44.95	0\\
44.96	0\\
44.97	0\\
44.98	0\\
44.99	0\\
45	0\\
45.01	0\\
45.02	0\\
45.03	0\\
45.04	0\\
45.05	0\\
45.06	0\\
45.07	0\\
45.08	0\\
45.09	0\\
45.1	0\\
45.11	0\\
45.12	0\\
45.13	0\\
45.14	0\\
45.15	0\\
45.16	0\\
45.17	0\\
45.18	0\\
45.19	0\\
45.2	0\\
45.21	0\\
45.22	0\\
45.23	0\\
45.24	0\\
45.25	0\\
45.26	0\\
45.27	0\\
45.28	0\\
45.29	0\\
45.3	0\\
45.31	0\\
45.32	0\\
45.33	0\\
45.34	0\\
45.35	0\\
45.36	0\\
45.37	0\\
45.38	0\\
45.39	0\\
45.4	0\\
45.41	0\\
45.42	0\\
45.43	0\\
45.44	0\\
45.45	0\\
45.46	0\\
45.47	0\\
45.48	0\\
45.49	0\\
45.5	0\\
45.51	0\\
45.52	0\\
45.53	0\\
45.54	0\\
45.55	0\\
45.56	0\\
45.57	0\\
45.58	0\\
45.59	0\\
45.6	0\\
45.61	0\\
45.62	0\\
45.63	0\\
45.64	0\\
45.65	0\\
45.66	0\\
45.67	0\\
45.68	0\\
45.69	0\\
45.7	0\\
45.71	0\\
45.72	0\\
45.73	0\\
45.74	0\\
45.75	0\\
45.76	0\\
45.77	0\\
45.78	0\\
45.79	0\\
45.8	0\\
45.81	0\\
45.82	0\\
45.83	0\\
45.84	0\\
45.85	0\\
45.86	0\\
45.87	0\\
45.88	0\\
45.89	0\\
45.9	0\\
45.91	0\\
45.92	0\\
45.93	0\\
45.94	0\\
45.95	0\\
45.96	0\\
45.97	0\\
45.98	0\\
45.99	0\\
46	0\\
46.01	0\\
46.02	0\\
46.03	0\\
46.04	0\\
46.05	0\\
46.06	0\\
46.07	0\\
46.08	0\\
46.09	0\\
46.1	0\\
46.11	0\\
46.12	0\\
46.13	0\\
46.14	0\\
46.15	0\\
46.16	0\\
46.17	0\\
46.18	0\\
46.19	0\\
46.2	0\\
46.21	0\\
46.22	0\\
46.23	0\\
46.24	0\\
46.25	0\\
46.26	0\\
46.27	0\\
46.28	0\\
46.29	0\\
46.3	0\\
46.31	0\\
46.32	0\\
46.33	0\\
46.34	0\\
46.35	0\\
46.36	0\\
46.37	0\\
46.38	0\\
46.39	0\\
46.4	0\\
46.41	0\\
46.42	0\\
46.43	0\\
46.44	0\\
46.45	0\\
46.46	0\\
46.47	0\\
46.48	0\\
46.49	0\\
46.5	0\\
46.51	0\\
46.52	0\\
46.53	0\\
46.54	0\\
46.55	0\\
46.56	0\\
46.57	0\\
46.58	0\\
46.59	0\\
46.6	0\\
46.61	0\\
46.62	0\\
46.63	0\\
46.64	0\\
46.65	0\\
46.66	0\\
46.67	0\\
46.68	0\\
46.69	0\\
46.7	0\\
46.71	0\\
46.72	0\\
46.73	0\\
46.74	0\\
46.75	0\\
46.76	0\\
46.77	0\\
46.78	0\\
46.79	0\\
46.8	0\\
46.81	0\\
46.82	0\\
46.83	0\\
46.84	0\\
46.85	0\\
46.86	0\\
46.87	0\\
46.88	0\\
46.89	0\\
46.9	0\\
46.91	0\\
46.92	0\\
46.93	0\\
46.94	0\\
46.95	0\\
46.96	0\\
46.97	0\\
46.98	0\\
46.99	0\\
47	0\\
47.01	0\\
47.02	0\\
47.03	0\\
47.04	0\\
47.05	0\\
47.06	0\\
47.07	0\\
47.08	0\\
47.09	0\\
47.1	0\\
47.11	0\\
47.12	0\\
47.13	0\\
47.14	0\\
47.15	0\\
47.16	0\\
47.17	0\\
47.18	0\\
47.19	0\\
47.2	0\\
47.21	0\\
47.22	0\\
47.23	0\\
47.24	0\\
47.25	0\\
47.26	0\\
47.27	0\\
47.28	0\\
47.29	0\\
47.3	0\\
47.31	0\\
47.32	0\\
47.33	0\\
47.34	0\\
47.35	0\\
47.36	0\\
47.37	0\\
47.38	0\\
47.39	0\\
47.4	0\\
47.41	0\\
47.42	0\\
47.43	0\\
47.44	0\\
47.45	0\\
47.46	0\\
47.47	0\\
47.48	0\\
47.49	0\\
47.5	0\\
47.51	0\\
47.52	0\\
47.53	0\\
47.54	0\\
47.55	0\\
47.56	0\\
47.57	0\\
47.58	0\\
47.59	0\\
47.6	0\\
47.61	0\\
47.62	0\\
47.63	0\\
47.64	0\\
47.65	0\\
47.66	0\\
47.67	0\\
47.68	0\\
47.69	0\\
47.7	0\\
47.71	0\\
47.72	0\\
47.73	0\\
47.74	0\\
47.75	0\\
47.76	0\\
47.77	0\\
47.78	0\\
47.79	0\\
47.8	0\\
47.81	0\\
47.82	0\\
47.83	0\\
47.84	0\\
47.85	0\\
47.86	0\\
47.87	0\\
47.88	0\\
47.89	0\\
47.9	0\\
47.91	0\\
47.92	0\\
47.93	0\\
47.94	0\\
47.95	0\\
47.96	0\\
47.97	0\\
47.98	0\\
47.99	0\\
48	0\\
48.01	0\\
48.02	0\\
48.03	0\\
48.04	0\\
48.05	0\\
48.06	0\\
48.07	0\\
48.08	0\\
48.09	0\\
48.1	0\\
48.11	0\\
48.12	0\\
48.13	0\\
48.14	0\\
48.15	0\\
48.16	0\\
48.17	0\\
48.18	0\\
48.19	0\\
48.2	0\\
48.21	0\\
48.22	0\\
48.23	0\\
48.24	0\\
48.25	0\\
48.26	0\\
48.27	0\\
48.28	0\\
48.29	0\\
48.3	0\\
48.31	0\\
48.32	0\\
48.33	0\\
48.34	0\\
48.35	0\\
48.36	0\\
48.37	0\\
48.38	0\\
48.39	0\\
48.4	0\\
48.41	0\\
48.42	0\\
48.43	0\\
48.44	0\\
48.45	0\\
48.46	0\\
48.47	0\\
48.48	0\\
48.49	0\\
48.5	0\\
48.51	0\\
48.52	0\\
48.53	0\\
48.54	0\\
48.55	0\\
48.56	0\\
48.57	0\\
48.58	0\\
48.59	0\\
48.6	0\\
48.61	0\\
48.62	0\\
48.63	0\\
48.64	0\\
48.65	0\\
48.66	0\\
48.67	0\\
48.68	0\\
48.69	0\\
48.7	0\\
48.71	0\\
48.72	0\\
48.73	0\\
48.74	0\\
48.75	0\\
48.76	0\\
48.77	0\\
48.78	0\\
48.79	0\\
48.8	0\\
48.81	0\\
48.82	0\\
48.83	0\\
48.84	0\\
48.85	0\\
48.86	0\\
48.87	0\\
48.88	0\\
48.89	0\\
48.9	0\\
48.91	0\\
48.92	0\\
48.93	0\\
48.94	0\\
48.95	0\\
48.96	0\\
48.97	0\\
48.98	0\\
48.99	0\\
49	0\\
49.01	0\\
49.02	0\\
49.03	0\\
49.04	0\\
49.05	0\\
49.06	0\\
49.07	0\\
49.08	0\\
49.09	0\\
49.1	0\\
49.11	0\\
49.12	0\\
49.13	0\\
49.14	0\\
49.15	0\\
49.16	0\\
49.17	0\\
49.18	0\\
49.19	0\\
49.2	0\\
49.21	0\\
49.22	0\\
49.23	0\\
49.24	0\\
49.25	0\\
49.26	0\\
49.27	0\\
49.28	0\\
49.29	0\\
49.3	0\\
49.31	0\\
49.32	0\\
49.33	0\\
49.34	0\\
49.35	0\\
49.36	0\\
49.37	0\\
49.38	0\\
49.39	0\\
49.4	0\\
49.41	0\\
49.42	0\\
49.43	0\\
49.44	0\\
49.45	0\\
49.46	0\\
49.47	0\\
49.48	0\\
49.49	0\\
49.5	0\\
49.51	0\\
49.52	0\\
49.53	0\\
49.54	0\\
49.55	0\\
49.56	0\\
49.57	0\\
49.58	0\\
49.59	0\\
49.6	0\\
49.61	0\\
49.62	0\\
49.63	0\\
49.64	0\\
49.65	0\\
49.66	0\\
49.67	0\\
49.68	0\\
49.69	0\\
49.7	0\\
49.71	0\\
49.72	0\\
49.73	0\\
49.74	0\\
49.75	0\\
49.76	0\\
49.77	0\\
49.78	0\\
49.79	0\\
49.8	0\\
49.81	0\\
49.82	0\\
49.83	0\\
49.84	0\\
49.85	0\\
49.86	0\\
49.87	0\\
49.88	0\\
49.89	0\\
49.9	0\\
49.91	0\\
49.92	0\\
49.93	0\\
49.94	0\\
49.95	0\\
49.96	0\\
49.97	0\\
49.98	0\\
49.99	0\\
50	0\\
50.01	0\\
50.02	0\\
50.03	0\\
50.04	0\\
50.05	0\\
50.06	0\\
50.07	0\\
50.08	0\\
50.09	0\\
50.1	0\\
50.11	0\\
50.12	0\\
50.13	0\\
50.14	0\\
50.15	0\\
50.16	0\\
50.17	0\\
50.18	0\\
50.19	0\\
50.2	0\\
50.21	0\\
50.22	0\\
50.23	0\\
50.24	0\\
50.25	0\\
50.26	0\\
50.27	0\\
50.28	0\\
50.29	0\\
50.3	0\\
50.31	0\\
50.32	0\\
50.33	0\\
50.34	0\\
50.35	0\\
50.36	0\\
50.37	0\\
50.38	0\\
50.39	0\\
50.4	0\\
50.41	0\\
50.42	0\\
50.43	0\\
50.44	0\\
50.45	0\\
50.46	0\\
50.47	0\\
50.48	0\\
50.49	0\\
50.5	0\\
50.51	0\\
50.52	0\\
50.53	0\\
50.54	0\\
50.55	0\\
50.56	0\\
50.57	0\\
50.58	0\\
50.59	0\\
50.6	0\\
50.61	0\\
50.62	0\\
50.63	0\\
50.64	0\\
50.65	0\\
50.66	0\\
50.67	0\\
50.68	0\\
50.69	0\\
50.7	0\\
50.71	0\\
50.72	0\\
50.73	0\\
50.74	0\\
50.75	0\\
50.76	0\\
50.77	0\\
50.78	0\\
50.79	0\\
50.8	0\\
50.81	0\\
50.82	0\\
50.83	0\\
50.84	0\\
50.85	0\\
50.86	0\\
50.87	0\\
50.88	0\\
50.89	0\\
50.9	0\\
50.91	0\\
50.92	0\\
50.93	0\\
50.94	0\\
50.95	0\\
50.96	0\\
50.97	0\\
50.98	0\\
50.99	0\\
51	0\\
51.01	0\\
51.02	0\\
51.03	0\\
51.04	0\\
51.05	0\\
51.06	0\\
51.07	0\\
51.08	0\\
51.09	0\\
51.1	0\\
51.11	0\\
51.12	0\\
51.13	0\\
51.14	0\\
51.15	0\\
51.16	0\\
51.17	0\\
51.18	0\\
51.19	0\\
51.2	0\\
51.21	0\\
51.22	0\\
51.23	0\\
51.24	0\\
51.25	0\\
51.26	0\\
51.27	0\\
51.28	0\\
51.29	0\\
51.3	0\\
51.31	0\\
51.32	0\\
51.33	0\\
51.34	0\\
51.35	0\\
51.36	0\\
51.37	0\\
51.38	0\\
51.39	0\\
51.4	0\\
51.41	0\\
51.42	0\\
51.43	0\\
51.44	0\\
51.45	0\\
51.46	0\\
51.47	0\\
51.48	0\\
51.49	0\\
51.5	0\\
51.51	0\\
51.52	0\\
51.53	0\\
51.54	0\\
51.55	0\\
51.56	0\\
51.57	0\\
51.58	0\\
51.59	0\\
51.6	0\\
51.61	0\\
51.62	0\\
51.63	0\\
51.64	0\\
51.65	0\\
51.66	0\\
51.67	0\\
51.68	0\\
51.69	0\\
51.7	0\\
51.71	0\\
51.72	0\\
51.73	0\\
51.74	0\\
51.75	0\\
51.76	0\\
51.77	0\\
51.78	0\\
51.79	0\\
51.8	0\\
51.81	0\\
51.82	0\\
51.83	0\\
51.84	0\\
51.85	0\\
51.86	0\\
51.87	0\\
51.88	0\\
51.89	0\\
51.9	0\\
51.91	0\\
51.92	0\\
51.93	0\\
51.94	0\\
51.95	0\\
51.96	0\\
51.97	0\\
51.98	0\\
51.99	0\\
52	0\\
52.01	0\\
52.02	0\\
52.03	0\\
52.04	0\\
52.05	0\\
52.06	0\\
52.07	0\\
52.08	0\\
52.09	0\\
52.1	0\\
52.11	0\\
52.12	0\\
52.13	0\\
52.14	0\\
52.15	0\\
52.16	0\\
52.17	0\\
52.18	0\\
52.19	0\\
52.2	0\\
52.21	0\\
52.22	0\\
52.23	0\\
52.24	0\\
52.25	0\\
52.26	0\\
52.27	0\\
52.28	0\\
52.29	0\\
52.3	0\\
52.31	0\\
52.32	0\\
52.33	0\\
52.34	0\\
52.35	0\\
52.36	0\\
52.37	0\\
52.38	0\\
52.39	0\\
52.4	0\\
52.41	0\\
52.42	0\\
52.43	0\\
52.44	0\\
52.45	0\\
52.46	0\\
52.47	0\\
52.48	0\\
52.49	0\\
52.5	0\\
52.51	0\\
52.52	0\\
52.53	0\\
52.54	0\\
52.55	0\\
52.56	0\\
52.57	0\\
52.58	0\\
52.59	0\\
52.6	0\\
52.61	0\\
52.62	0\\
52.63	0\\
52.64	0\\
52.65	0\\
52.66	0\\
52.67	0\\
52.68	0\\
52.69	0\\
52.7	0\\
52.71	0\\
52.72	0\\
52.73	0\\
52.74	0\\
52.75	0\\
52.76	0\\
52.77	0\\
52.78	0\\
52.79	0\\
52.8	0\\
52.81	0\\
52.82	0\\
52.83	0\\
52.84	0\\
52.85	0\\
52.86	0\\
52.87	0\\
52.88	0\\
52.89	0\\
52.9	0\\
52.91	0\\
52.92	0\\
52.93	0\\
52.94	0\\
52.95	0\\
52.96	0\\
52.97	0\\
52.98	0\\
52.99	0\\
53	0\\
53.01	0\\
53.02	0\\
53.03	0\\
53.04	0\\
53.05	0\\
53.06	0\\
53.07	0\\
53.08	0\\
53.09	0\\
53.1	0\\
53.11	0\\
53.12	0\\
53.13	0\\
53.14	0\\
53.15	0\\
53.16	0\\
53.17	0\\
53.18	0\\
53.19	0\\
53.2	0\\
53.21	0\\
53.22	0\\
53.23	0\\
53.24	0\\
53.25	0\\
53.26	0\\
53.27	0\\
53.28	0\\
53.29	0\\
53.3	0\\
53.31	0\\
53.32	0\\
53.33	0\\
53.34	0\\
53.35	0\\
53.36	0\\
53.37	0\\
53.38	0\\
53.39	0\\
53.4	0\\
53.41	0\\
53.42	0\\
53.43	0\\
53.44	0\\
53.45	0\\
53.46	0\\
53.47	0\\
53.48	0\\
53.49	0\\
53.5	0\\
53.51	0\\
53.52	0\\
53.53	0\\
53.54	0\\
53.55	0\\
53.56	0\\
53.57	0\\
53.58	0\\
53.59	0\\
53.6	0\\
53.61	0\\
53.62	0\\
53.63	0\\
53.64	0\\
53.65	0\\
53.66	0\\
53.67	0\\
53.68	0\\
53.69	0\\
53.7	0\\
53.71	0\\
53.72	0\\
53.73	0\\
53.74	0\\
53.75	0\\
53.76	0\\
53.77	0\\
53.78	0\\
53.79	0\\
53.8	0\\
53.81	0\\
53.82	0\\
53.83	0\\
53.84	0\\
53.85	0\\
53.86	0\\
53.87	0\\
53.88	0\\
53.89	0\\
53.9	0\\
53.91	0\\
53.92	0\\
53.93	0\\
53.94	0\\
53.95	0\\
53.96	0\\
53.97	0\\
53.98	0\\
53.99	0\\
54	0\\
54.01	0\\
54.02	0\\
54.03	0\\
54.04	0\\
54.05	0\\
54.06	0\\
54.07	0\\
54.08	0\\
54.09	0\\
54.1	0\\
54.11	0\\
54.12	0\\
54.13	0\\
54.14	0\\
54.15	0\\
54.16	0\\
54.17	0\\
54.18	0\\
54.19	0\\
54.2	0\\
54.21	0\\
54.22	0\\
54.23	0\\
54.24	0\\
54.25	0\\
54.26	0\\
54.27	0\\
54.28	0\\
54.29	0\\
54.3	0\\
54.31	0\\
54.32	0\\
54.33	0\\
54.34	0\\
54.35	0\\
54.36	0\\
54.37	0\\
54.38	0\\
54.39	0\\
54.4	0\\
54.41	0\\
54.42	0\\
54.43	0\\
54.44	0\\
54.45	0\\
54.46	0\\
54.47	0\\
54.48	0\\
54.49	0\\
54.5	0\\
54.51	0\\
54.52	0\\
54.53	0\\
54.54	0\\
54.55	0\\
54.56	0\\
54.57	0\\
54.58	0\\
54.59	0\\
54.6	0\\
54.61	0\\
54.62	0\\
54.63	0\\
54.64	0\\
54.65	0\\
54.66	0\\
54.67	0\\
54.68	0\\
54.69	0\\
54.7	0\\
54.71	0\\
54.72	0\\
54.73	0\\
54.74	0\\
54.75	0\\
54.76	0\\
54.77	0\\
54.78	0\\
54.79	0\\
54.8	0\\
54.81	0\\
54.82	0\\
54.83	0\\
54.84	0\\
54.85	0\\
54.86	0\\
54.87	0\\
54.88	0\\
54.89	0\\
54.9	0\\
54.91	0\\
54.92	0\\
54.93	0\\
54.94	0\\
54.95	0\\
54.96	0\\
54.97	0\\
54.98	0\\
54.99	0\\
55	0\\
55.01	0\\
55.02	0\\
55.03	0\\
55.04	0\\
55.05	0\\
55.06	0\\
55.07	0\\
55.08	0\\
55.09	0\\
55.1	0\\
55.11	0\\
55.12	0\\
55.13	0\\
55.14	0\\
55.15	0\\
55.16	0\\
55.17	0\\
55.18	0\\
55.19	0\\
55.2	0\\
55.21	0\\
55.22	0\\
55.23	0\\
55.24	0\\
55.25	0\\
55.26	0\\
55.27	0\\
55.28	0\\
55.29	0\\
55.3	0\\
55.31	0\\
55.32	0\\
55.33	0\\
55.34	0\\
55.35	0\\
55.36	0\\
55.37	0\\
55.38	0\\
55.39	0\\
55.4	0\\
55.41	0\\
55.42	0\\
55.43	0\\
55.44	0\\
55.45	0\\
55.46	0\\
55.47	0\\
55.48	0\\
55.49	0\\
55.5	0\\
55.51	0\\
55.52	0\\
55.53	0\\
55.54	0\\
55.55	0\\
55.56	0\\
55.57	0\\
55.58	0\\
55.59	0\\
55.6	0\\
55.61	0\\
55.62	0\\
55.63	0\\
55.64	0\\
55.65	0\\
55.66	0\\
55.67	0\\
55.68	0\\
55.69	0\\
55.7	0\\
55.71	0\\
55.72	0\\
55.73	0\\
55.74	0\\
55.75	0\\
55.76	0\\
55.77	0\\
55.78	0\\
55.79	0\\
55.8	0\\
55.81	0\\
55.82	0\\
55.83	0\\
55.84	0\\
55.85	0\\
55.86	0\\
55.87	0\\
55.88	0\\
55.89	0\\
55.9	0\\
55.91	0\\
55.92	0\\
55.93	0\\
55.94	0\\
55.95	0\\
55.96	0\\
55.97	0\\
55.98	0\\
55.99	0\\
56	0\\
56.01	0\\
56.02	0\\
56.03	0\\
56.04	0\\
56.05	0\\
56.06	0\\
56.07	0\\
56.08	0\\
56.09	0\\
56.1	0\\
56.11	0\\
56.12	0\\
56.13	0\\
56.14	0\\
56.15	0\\
56.16	0\\
56.17	0\\
56.18	0\\
56.19	0\\
56.2	0\\
56.21	0\\
56.22	0\\
56.23	0\\
56.24	0\\
56.25	0\\
56.26	0\\
56.27	0\\
56.28	0\\
56.29	0\\
56.3	0\\
56.31	0\\
56.32	0\\
56.33	0\\
56.34	0\\
56.35	0\\
56.36	0\\
56.37	0\\
56.38	0\\
56.39	0\\
56.4	0\\
56.41	0\\
56.42	0\\
56.43	0\\
56.44	0\\
56.45	0\\
56.46	0\\
56.47	0\\
56.48	0\\
56.49	0\\
56.5	0\\
56.51	0\\
56.52	0\\
56.53	0\\
56.54	0\\
56.55	0\\
56.56	0\\
56.57	0\\
56.58	0\\
56.59	0\\
56.6	0\\
56.61	0\\
56.62	0\\
56.63	0\\
56.64	0\\
56.65	0\\
56.66	0\\
56.67	0\\
56.68	0\\
56.69	0\\
56.7	0\\
56.71	0\\
56.72	0\\
56.73	0\\
56.74	0\\
56.75	0\\
56.76	0\\
56.77	0\\
56.78	0\\
56.79	0\\
56.8	0\\
56.81	0\\
56.82	0\\
56.83	0\\
56.84	0\\
56.85	0\\
56.86	0\\
56.87	0\\
56.88	0\\
56.89	0\\
56.9	0\\
56.91	0\\
56.92	0\\
56.93	0\\
56.94	0\\
56.95	0\\
56.96	0\\
56.97	0\\
56.98	0\\
56.99	0\\
57	0\\
57.01	0\\
57.02	0\\
57.03	0\\
57.04	0\\
57.05	0\\
57.06	0\\
57.07	0\\
57.08	0\\
57.09	0\\
57.1	0\\
57.11	0\\
57.12	0\\
57.13	0\\
57.14	0\\
57.15	0\\
57.16	0\\
57.17	0\\
57.18	0\\
57.19	0\\
57.2	0\\
57.21	0\\
57.22	0\\
57.23	0\\
57.24	0\\
57.25	0\\
57.26	0\\
57.27	0\\
57.28	0\\
57.29	0\\
57.3	0\\
57.31	0\\
57.32	0\\
57.33	0\\
57.34	0\\
57.35	0\\
57.36	0\\
57.37	0\\
57.38	0\\
57.39	0\\
57.4	0\\
57.41	0\\
57.42	0\\
57.43	0\\
57.44	0\\
57.45	0\\
57.46	0\\
57.47	0\\
57.48	0\\
57.49	0\\
57.5	0\\
57.51	0\\
57.52	0\\
57.53	0\\
57.54	0\\
57.55	0\\
57.56	0\\
57.57	0\\
57.58	0\\
57.59	0\\
57.6	0\\
57.61	0\\
57.62	0\\
57.63	0\\
57.64	0\\
57.65	0\\
57.66	0\\
57.67	0\\
57.68	0\\
57.69	0\\
57.7	0\\
57.71	0\\
57.72	0\\
57.73	0\\
57.74	0\\
57.75	0\\
57.76	0\\
57.77	0\\
57.78	0\\
57.79	0\\
57.8	0\\
57.81	0\\
57.82	0\\
57.83	0\\
57.84	0\\
57.85	0\\
57.86	0\\
57.87	0\\
57.88	0\\
57.89	0\\
57.9	0\\
57.91	0\\
57.92	0\\
57.93	0\\
57.94	0\\
57.95	0\\
57.96	0\\
57.97	0\\
57.98	0\\
57.99	0\\
58	0\\
58.01	0\\
58.02	0\\
58.03	0\\
58.04	0\\
58.05	0\\
58.06	0\\
58.07	0\\
58.08	0\\
58.09	0\\
58.1	0\\
58.11	0\\
58.12	0\\
58.13	0\\
58.14	0\\
58.15	0\\
58.16	0\\
58.17	0\\
58.18	0\\
58.19	0\\
58.2	0\\
58.21	0\\
58.22	0\\
58.23	0\\
58.24	0\\
58.25	0\\
58.26	0\\
58.27	0\\
58.28	0\\
58.29	0\\
58.3	0\\
58.31	0\\
58.32	0\\
58.33	0\\
58.34	0\\
58.35	0\\
58.36	0\\
58.37	0\\
58.38	0\\
58.39	0\\
58.4	0\\
58.41	0\\
58.42	0\\
58.43	0\\
58.44	0\\
58.45	0\\
58.46	0\\
58.47	0\\
58.48	0\\
58.49	0\\
58.5	0\\
58.51	0\\
58.52	0\\
58.53	0\\
58.54	0\\
58.55	0\\
58.56	0\\
58.57	0\\
58.58	0\\
58.59	0\\
58.6	0\\
58.61	0\\
58.62	0\\
58.63	0\\
58.64	0\\
58.65	0\\
58.66	0\\
58.67	0\\
58.68	0\\
58.69	0\\
58.7	0\\
58.71	0\\
58.72	0\\
58.73	0\\
58.74	0\\
58.75	0\\
58.76	0\\
58.77	0\\
58.78	0\\
58.79	0\\
58.8	0\\
58.81	0\\
58.82	0\\
58.83	0\\
58.84	0\\
58.85	0\\
58.86	0\\
58.87	0\\
58.88	0\\
58.89	0\\
58.9	0\\
58.91	0\\
58.92	0\\
58.93	0\\
58.94	0\\
58.95	0\\
58.96	0\\
58.97	0\\
58.98	0\\
58.99	0\\
59	0\\
59.01	0\\
59.02	0\\
59.03	0\\
59.04	0\\
59.05	0\\
59.06	0\\
59.07	0\\
59.08	0\\
59.09	0\\
59.1	0\\
59.11	0\\
59.12	0\\
59.13	0\\
59.14	0\\
59.15	0\\
59.16	0\\
59.17	0\\
59.18	0\\
59.19	0\\
59.2	0\\
59.21	0\\
59.22	0\\
59.23	0\\
59.24	0\\
59.25	0\\
59.26	0\\
59.27	0\\
59.28	0\\
59.29	0\\
59.3	0\\
59.31	0\\
59.32	0\\
59.33	0\\
59.34	0\\
59.35	0\\
59.36	0\\
59.37	0\\
59.38	0\\
59.39	0\\
59.4	0\\
59.41	0\\
59.42	0\\
59.43	0\\
59.44	0\\
59.45	0\\
59.46	0\\
59.47	0\\
59.48	0\\
59.49	0\\
59.5	0\\
59.51	0\\
59.52	0\\
59.53	0\\
59.54	0\\
59.55	0\\
59.56	0\\
59.57	0\\
59.58	0\\
59.59	0\\
59.6	0\\
59.61	0\\
59.62	0\\
59.63	0\\
59.64	0\\
59.65	0\\
59.66	0\\
59.67	0\\
59.68	0\\
59.69	0\\
59.7	0\\
59.71	0\\
59.72	0\\
59.73	0\\
59.74	0\\
59.75	0\\
59.76	0\\
59.77	0\\
59.78	0\\
59.79	0\\
59.8	0\\
59.81	0\\
59.82	0\\
59.83	0\\
59.84	0\\
59.85	0\\
59.86	0\\
59.87	0\\
59.88	0\\
59.89	0\\
59.9	0\\
59.91	0\\
59.92	0\\
59.93	0\\
59.94	0\\
59.95	0\\
59.96	0\\
59.97	0\\
59.98	0\\
59.99	0\\
60	0\\
60.01	0\\
60.02	0\\
60.03	0\\
60.04	0\\
60.05	0\\
60.06	0\\
60.07	0\\
60.08	0\\
60.09	0\\
60.1	0\\
60.11	0\\
60.12	0\\
60.13	0\\
60.14	0\\
60.15	0\\
60.16	0\\
60.17	0\\
60.18	0\\
60.19	0\\
60.2	0\\
60.21	0\\
60.22	0\\
60.23	0\\
60.24	0\\
60.25	0\\
60.26	0\\
60.27	0\\
60.28	0\\
60.29	0\\
60.3	0\\
60.31	0\\
60.32	0\\
60.33	0\\
60.34	0\\
60.35	0\\
60.36	0\\
60.37	0\\
60.38	0\\
60.39	0\\
60.4	0\\
60.41	0\\
60.42	0\\
60.43	0\\
60.44	0\\
60.45	0\\
60.46	0\\
60.47	0\\
60.48	0\\
60.49	0\\
60.5	0\\
60.51	0\\
60.52	0\\
60.53	0\\
60.54	0\\
60.55	0\\
60.56	0\\
60.57	0\\
60.58	0\\
60.59	0\\
60.6	0\\
60.61	0\\
60.62	0\\
60.63	0\\
60.64	0\\
60.65	0\\
60.66	0\\
60.67	0\\
60.68	0\\
60.69	0\\
60.7	0\\
60.71	0\\
60.72	0\\
60.73	0\\
60.74	0\\
60.75	0\\
60.76	0\\
60.77	0\\
60.78	0\\
60.79	0\\
60.8	0\\
60.81	0\\
60.82	0\\
60.83	0\\
60.84	0\\
60.85	0\\
60.86	0\\
60.87	0\\
60.88	0\\
60.89	0\\
60.9	0\\
60.91	0\\
60.92	0\\
60.93	0\\
60.94	0\\
60.95	0\\
60.96	0\\
60.97	0\\
60.98	0\\
60.99	0\\
61	0\\
61.01	0\\
61.02	0\\
61.03	0\\
61.04	0\\
61.05	0\\
61.06	0\\
61.07	0\\
61.08	0\\
61.09	0\\
61.1	0\\
61.11	0\\
61.12	0\\
61.13	0\\
61.14	0\\
61.15	0\\
61.16	0\\
61.17	0\\
61.18	0\\
61.19	0\\
61.2	0\\
61.21	0\\
61.22	0\\
61.23	0\\
61.24	0\\
61.25	0\\
61.26	0\\
61.27	0\\
61.28	0\\
61.29	0\\
61.3	0\\
61.31	0\\
61.32	0\\
61.33	0\\
61.34	0\\
61.35	0\\
61.36	0\\
61.37	0\\
61.38	0\\
61.39	0\\
61.4	0\\
61.41	0\\
61.42	0\\
61.43	0\\
61.44	0\\
61.45	0\\
61.46	0\\
61.47	0\\
61.48	0\\
61.49	0\\
61.5	0\\
61.51	0\\
61.52	0\\
61.53	0\\
61.54	0\\
61.55	0\\
61.56	0\\
61.57	0\\
61.58	0\\
61.59	0\\
61.6	0\\
61.61	0\\
61.62	0\\
61.63	0\\
61.64	0\\
61.65	0\\
61.66	0\\
61.67	0\\
61.68	0\\
61.69	0\\
61.7	0\\
61.71	0\\
61.72	0\\
61.73	0\\
61.74	0\\
61.75	0\\
61.76	0\\
61.77	0\\
61.78	0\\
61.79	0\\
61.8	0\\
61.81	0\\
61.82	0\\
61.83	0\\
61.84	0\\
61.85	0\\
61.86	0\\
61.87	0\\
61.88	0\\
61.89	0\\
61.9	0\\
61.91	0\\
61.92	0\\
61.93	0\\
61.94	0\\
61.95	0\\
61.96	0\\
61.97	0\\
61.98	0\\
61.99	0\\
62	0\\
62.01	0\\
62.02	0\\
62.03	0\\
62.04	0\\
62.05	0\\
62.06	0\\
62.07	0\\
62.08	0\\
62.09	0\\
62.1	0\\
62.11	0\\
62.12	0\\
62.13	0\\
62.14	0\\
62.15	0\\
62.16	0\\
62.17	0\\
62.18	0\\
62.19	0\\
62.2	0\\
62.21	0\\
62.22	0\\
62.23	0\\
62.24	0\\
62.25	0\\
62.26	0\\
62.27	0\\
62.28	0\\
62.29	0\\
62.3	0\\
62.31	0\\
62.32	0\\
62.33	0\\
62.34	0\\
62.35	0\\
62.36	0\\
62.37	0\\
62.38	0\\
62.39	0\\
62.4	0\\
62.41	0\\
62.42	0\\
62.43	0\\
62.44	0\\
62.45	0\\
62.46	0\\
62.47	0\\
62.48	0\\
62.49	0\\
62.5	0\\
62.51	0\\
62.52	0\\
62.53	0\\
62.54	0\\
62.55	0\\
62.56	0\\
62.57	0\\
62.58	0\\
62.59	0\\
62.6	0\\
62.61	0\\
62.62	0\\
62.63	0\\
62.64	0\\
62.65	0\\
62.66	0\\
62.67	0\\
62.68	0\\
62.69	0\\
62.7	0\\
62.71	0\\
62.72	0\\
62.73	0\\
62.74	0\\
62.75	0\\
62.76	0\\
62.77	0\\
62.78	0\\
62.79	0\\
62.8	0\\
62.81	0\\
62.82	0\\
62.83	0\\
62.84	0\\
62.85	0\\
62.86	0\\
62.87	0\\
62.88	0\\
62.89	0\\
62.9	0\\
62.91	0\\
62.92	0\\
62.93	0\\
62.94	0\\
62.95	0\\
62.96	0\\
62.97	0\\
62.98	0\\
62.99	0\\
63	0\\
63.01	0\\
63.02	0\\
63.03	0\\
63.04	0\\
63.05	0\\
63.06	0\\
63.07	0\\
63.08	0\\
63.09	0\\
63.1	0\\
63.11	0\\
63.12	0\\
63.13	0\\
63.14	0\\
63.15	0\\
63.16	0\\
63.17	0\\
63.18	0\\
63.19	0\\
63.2	0\\
63.21	0\\
63.22	0\\
63.23	0\\
63.24	0\\
63.25	0\\
63.26	0\\
63.27	0\\
63.28	0\\
63.29	0\\
63.3	0\\
63.31	0\\
63.32	0\\
63.33	0\\
63.34	0\\
63.35	0\\
63.36	0\\
63.37	0\\
63.38	0\\
63.39	0\\
63.4	0\\
63.41	0\\
63.42	0\\
63.43	0\\
63.44	0\\
63.45	0\\
63.46	0\\
63.47	0\\
63.48	0\\
63.49	0\\
63.5	0\\
63.51	0\\
63.52	0\\
63.53	0\\
63.54	0\\
63.55	0\\
63.56	0\\
63.57	0\\
63.58	0\\
63.59	0\\
63.6	0\\
63.61	0\\
63.62	0\\
63.63	0\\
63.64	0\\
63.65	0\\
63.66	0\\
63.67	0\\
63.68	0\\
63.69	0\\
63.7	0\\
63.71	0\\
63.72	0\\
63.73	0\\
63.74	0\\
63.75	0\\
63.76	0\\
63.77	0\\
63.78	0\\
63.79	0\\
63.8	0\\
63.81	0\\
63.82	0\\
63.83	0\\
63.84	0\\
63.85	0\\
63.86	0\\
63.87	0\\
63.88	0\\
63.89	0\\
63.9	0\\
63.91	0\\
63.92	0\\
63.93	0\\
63.94	0\\
63.95	0\\
63.96	0\\
63.97	0\\
63.98	0\\
63.99	0\\
64	0\\
64.01	0\\
64.02	0\\
64.03	0\\
64.04	0\\
64.05	0\\
64.06	0\\
64.07	0\\
64.08	0\\
64.09	0\\
64.1	0\\
64.11	0\\
64.12	0\\
64.13	0\\
64.14	0\\
64.15	0\\
64.16	0\\
64.17	0\\
64.18	0\\
64.19	0\\
64.2	0\\
64.21	0\\
64.22	0\\
64.23	0\\
64.24	0\\
64.25	0\\
64.26	0\\
64.27	0\\
64.28	0\\
64.29	0\\
64.3	0\\
64.31	0\\
64.32	0\\
64.33	0\\
64.34	0\\
64.35	0\\
64.36	0\\
64.37	0\\
64.38	0\\
64.39	0\\
64.4	0\\
64.41	0\\
64.42	0\\
64.43	0\\
64.44	0\\
64.45	0\\
64.46	0\\
64.47	0\\
64.48	0\\
64.49	0\\
64.5	0\\
64.51	0\\
64.52	0\\
64.53	0\\
64.54	0\\
64.55	0\\
64.56	0\\
64.57	0\\
64.58	0\\
64.59	0\\
64.6	0\\
64.61	0\\
64.62	0\\
64.63	0\\
64.64	0\\
64.65	0\\
64.66	0\\
64.67	0\\
64.68	0\\
64.69	0\\
64.7	0\\
64.71	0\\
64.72	0\\
64.73	0\\
64.74	0\\
64.75	0\\
64.76	0\\
64.77	0\\
64.78	0\\
64.79	0\\
64.8	0\\
64.81	0\\
64.82	0\\
64.83	0\\
64.84	0\\
64.85	0\\
64.86	0\\
64.87	0\\
64.88	0\\
64.89	0\\
64.9	0\\
64.91	0\\
64.92	0\\
64.93	0\\
64.94	0\\
64.95	0\\
64.96	0\\
64.97	0\\
64.98	0\\
64.99	0\\
65	0\\
65.01	0\\
65.02	0\\
65.03	0\\
65.04	0\\
65.05	0\\
65.06	0\\
65.07	0\\
65.08	0\\
65.09	0\\
65.1	0\\
65.11	0\\
65.12	0\\
65.13	0\\
65.14	0\\
65.15	0\\
65.16	0\\
65.17	0\\
65.18	0\\
65.19	0\\
65.2	0\\
65.21	0\\
65.22	0\\
65.23	0\\
65.24	0\\
65.25	0\\
65.26	0\\
65.27	0\\
65.28	0\\
65.29	0\\
65.3	0\\
65.31	0\\
65.32	0\\
65.33	0\\
65.34	0\\
65.35	0\\
65.36	0\\
65.37	0\\
65.38	0\\
65.39	0\\
65.4	0\\
65.41	0\\
65.42	0\\
65.43	0\\
65.44	0\\
65.45	0\\
65.46	0\\
65.47	0\\
65.48	0\\
65.49	0\\
65.5	0\\
65.51	0\\
65.52	0\\
65.53	0\\
65.54	0\\
65.55	0\\
65.56	0\\
65.57	0\\
65.58	0\\
65.59	0\\
65.6	0\\
65.61	0\\
65.62	0\\
65.63	0\\
65.64	0\\
65.65	0\\
65.66	0\\
65.67	0\\
65.68	0\\
65.69	0\\
65.7	0\\
65.71	0\\
65.72	0\\
65.73	0\\
65.74	0\\
65.75	0\\
65.76	0\\
65.77	0\\
65.78	0\\
65.79	0\\
65.8	0\\
65.81	0\\
65.82	0\\
65.83	0\\
65.84	0\\
65.85	0\\
65.86	0\\
65.87	0\\
65.88	0\\
65.89	0\\
65.9	0\\
65.91	0\\
65.92	0\\
65.93	0\\
65.94	0\\
65.95	0\\
65.96	0\\
65.97	0\\
65.98	0\\
65.99	0\\
66	0\\
66.01	0\\
66.02	0\\
66.03	0\\
66.04	0\\
66.05	0\\
66.06	0\\
66.07	0\\
66.08	0\\
66.09	0\\
66.1	0\\
66.11	0\\
66.12	0\\
66.13	0\\
66.14	0\\
66.15	0\\
66.16	0\\
66.17	0\\
66.18	0\\
66.19	0\\
66.2	0\\
66.21	0\\
66.22	0\\
66.23	0\\
66.24	0\\
66.25	0\\
66.26	0\\
66.27	0\\
66.28	0\\
66.29	0\\
66.3	0\\
66.31	0\\
66.32	0\\
66.33	0\\
66.34	0\\
66.35	0\\
66.36	0\\
66.37	0\\
66.38	0\\
66.39	0\\
66.4	0\\
66.41	0\\
66.42	0\\
66.43	0\\
66.44	0\\
66.45	0\\
66.46	0\\
66.47	0\\
66.48	0\\
66.49	0\\
66.5	0\\
66.51	0\\
66.52	0\\
66.53	0\\
66.54	0\\
66.55	0\\
66.56	0\\
66.57	0\\
66.58	0\\
66.59	0\\
66.6	0\\
66.61	0\\
66.62	0\\
66.63	0\\
66.64	0\\
66.65	0\\
66.66	0\\
66.67	0\\
66.68	0\\
66.69	0\\
66.7	0\\
66.71	0\\
66.72	0\\
66.73	0\\
66.74	0\\
66.75	0\\
66.76	0\\
66.77	0\\
66.78	0\\
66.79	0\\
66.8	0\\
66.81	0\\
66.82	0\\
66.83	0\\
66.84	0\\
66.85	0\\
66.86	0\\
66.87	0\\
66.88	0\\
66.89	0\\
66.9	0\\
66.91	0\\
66.92	0\\
66.93	0\\
66.94	0\\
66.95	0\\
66.96	0\\
66.97	0\\
66.98	0\\
66.99	0\\
67	0\\
67.01	0\\
67.02	0\\
67.03	0\\
67.04	0\\
67.05	0\\
67.06	0\\
67.07	0\\
67.08	0\\
67.09	0\\
67.1	0\\
67.11	0\\
67.12	0\\
67.13	0\\
67.14	0\\
67.15	0\\
67.16	0\\
67.17	0\\
67.18	0\\
67.19	0\\
67.2	0\\
67.21	0\\
67.22	0\\
67.23	0\\
67.24	0\\
67.25	0\\
67.26	0\\
67.27	0\\
67.28	0\\
67.29	0\\
67.3	0\\
67.31	0\\
67.32	0\\
67.33	0\\
67.34	0\\
67.35	0\\
67.36	0\\
67.37	0\\
67.38	0\\
67.39	0\\
67.4	0\\
67.41	0\\
67.42	0\\
67.43	0\\
67.44	0\\
67.45	0\\
67.46	0\\
67.47	0\\
67.48	0\\
67.49	0\\
67.5	0\\
67.51	0\\
67.52	0\\
67.53	0\\
67.54	0\\
67.55	0\\
67.56	0\\
67.57	0\\
67.58	0\\
67.59	0\\
67.6	0\\
67.61	0\\
67.62	0\\
67.63	0\\
67.64	0\\
67.65	0\\
67.66	0\\
67.67	0\\
67.68	0\\
67.69	0\\
67.7	0\\
67.71	0\\
67.72	0\\
67.73	0\\
67.74	0\\
67.75	0\\
67.76	0\\
67.77	0\\
67.78	0\\
67.79	0\\
67.8	0\\
67.81	0\\
67.82	0\\
67.83	0\\
67.84	0\\
67.85	0\\
67.86	0\\
67.87	0\\
67.88	0\\
67.89	0\\
67.9	0\\
67.91	0\\
67.92	0\\
67.93	0\\
67.94	0\\
67.95	0\\
67.96	0\\
67.97	0\\
67.98	0\\
67.99	0\\
68	0\\
68.01	0\\
68.02	0\\
68.03	0\\
68.04	0\\
68.05	0\\
68.06	0\\
68.07	0\\
68.08	0\\
68.09	0\\
68.1	0\\
68.11	0\\
68.12	0\\
68.13	0\\
68.14	0\\
68.15	0\\
68.16	0\\
68.17	0\\
68.18	0\\
68.19	0\\
68.2	0\\
68.21	0\\
68.22	0\\
68.23	0\\
68.24	0\\
68.25	0\\
68.26	0\\
68.27	0\\
68.28	0\\
68.29	0\\
68.3	0\\
68.31	0\\
68.32	0\\
68.33	0\\
68.34	0\\
68.35	0\\
68.36	0\\
68.37	0\\
68.38	0\\
68.39	0\\
68.4	0\\
68.41	0\\
68.42	0\\
68.43	0\\
68.44	0\\
68.45	0\\
68.46	0\\
68.47	0\\
68.48	0\\
68.49	0\\
68.5	0\\
68.51	0\\
68.52	0\\
68.53	0\\
68.54	0\\
68.55	0\\
68.56	0\\
68.57	0\\
68.58	0\\
68.59	0\\
68.6	0\\
68.61	0\\
68.62	0\\
68.63	0\\
68.64	0\\
68.65	0\\
68.66	0\\
68.67	0\\
68.68	0\\
68.69	0\\
68.7	0\\
68.71	0\\
68.72	0\\
68.73	0\\
68.74	0\\
68.75	0\\
68.76	0\\
68.77	0\\
68.78	0\\
68.79	0\\
68.8	0\\
68.81	0\\
68.82	0\\
68.83	0\\
68.84	0\\
68.85	0\\
68.86	0\\
68.87	0\\
68.88	0\\
68.89	0\\
68.9	0\\
68.91	0\\
68.92	0\\
68.93	0\\
68.94	0\\
68.95	0\\
68.96	0\\
68.97	0\\
68.98	0\\
68.99	0\\
69	0\\
69.01	0\\
69.02	0\\
69.03	0\\
69.04	0\\
69.05	0\\
69.06	0\\
69.07	0\\
69.08	0\\
69.09	0\\
69.1	0\\
69.11	0\\
69.12	0\\
69.13	0\\
69.14	0\\
69.15	0\\
69.16	0\\
69.17	0\\
69.18	0\\
69.19	0\\
69.2	0\\
69.21	0\\
69.22	0\\
69.23	0\\
69.24	0\\
69.25	0\\
69.26	0\\
69.27	0\\
69.28	0\\
69.29	0\\
69.3	0\\
69.31	0\\
69.32	0\\
69.33	0\\
69.34	0\\
69.35	0\\
69.36	0\\
69.37	0\\
69.38	0\\
69.39	0\\
69.4	0\\
69.41	0\\
69.42	0\\
69.43	0\\
69.44	0\\
69.45	0\\
69.46	0\\
69.47	0\\
69.48	0\\
69.49	0\\
69.5	0\\
69.51	0\\
69.52	0\\
69.53	0\\
69.54	0\\
69.55	0\\
69.56	0\\
69.57	0\\
69.58	0\\
69.59	0\\
69.6	0\\
69.61	0\\
69.62	0\\
69.63	0\\
69.64	0\\
69.65	0\\
69.66	0\\
69.67	0\\
69.68	0\\
69.69	0\\
69.7	0\\
69.71	0\\
69.72	0\\
69.73	0\\
69.74	0\\
69.75	0\\
69.76	0\\
69.77	0\\
69.78	0\\
69.79	0\\
69.8	0\\
69.81	0\\
69.82	0\\
69.83	0\\
69.84	0\\
69.85	0\\
69.86	0\\
69.87	0\\
69.88	0\\
69.89	0\\
69.9	0\\
69.91	0\\
69.92	0\\
69.93	0\\
69.94	0\\
69.95	0\\
69.96	0\\
69.97	0\\
69.98	0\\
69.99	0\\
70	0\\
70.01	0\\
70.02	0\\
70.03	0\\
70.04	0\\
70.05	0\\
70.06	0\\
70.07	0\\
70.08	0\\
70.09	0\\
70.1	0\\
70.11	0\\
70.12	0\\
70.13	0\\
70.14	0\\
70.15	0\\
70.16	0\\
70.17	0\\
70.18	0\\
70.19	0\\
70.2	0\\
70.21	0\\
70.22	0\\
70.23	0\\
70.24	0\\
70.25	0\\
70.26	0\\
70.27	0\\
70.28	0\\
70.29	0\\
70.3	0\\
70.31	0\\
70.32	0\\
70.33	0\\
70.34	0\\
70.35	0\\
70.36	0\\
70.37	0\\
70.38	0\\
70.39	0\\
70.4	0\\
70.41	0\\
70.42	0\\
70.43	0\\
70.44	0\\
70.45	0\\
70.46	0\\
70.47	0\\
70.48	0\\
70.49	0\\
70.5	0\\
70.51	0\\
70.52	0\\
70.53	0\\
70.54	0\\
70.55	0\\
70.56	0\\
70.57	0\\
70.58	0\\
70.59	0\\
70.6	0\\
70.61	0\\
70.62	0\\
70.63	0\\
70.64	0\\
70.65	0\\
70.66	0\\
70.67	0\\
70.68	0\\
70.69	0\\
70.7	0\\
70.71	0\\
70.72	0\\
70.73	0\\
70.74	0\\
70.75	0\\
70.76	0\\
70.77	0\\
70.78	0\\
70.79	0\\
70.8	0\\
70.81	0\\
70.82	0\\
70.83	0\\
70.84	0\\
70.85	0\\
70.86	0\\
70.87	0\\
70.88	0\\
70.89	0\\
70.9	0\\
70.91	0\\
70.92	0\\
70.93	0\\
70.94	0\\
70.95	0\\
70.96	0\\
70.97	0\\
70.98	0\\
70.99	0\\
71	0\\
71.01	0\\
71.02	0\\
71.03	0\\
71.04	0\\
71.05	0\\
71.06	0\\
71.07	0\\
71.08	0\\
71.09	0\\
71.1	0\\
71.11	0\\
71.12	0\\
71.13	0\\
71.14	0\\
71.15	0\\
71.16	0\\
71.17	0\\
71.18	0\\
71.19	0\\
71.2	0\\
71.21	0\\
71.22	0\\
71.23	0\\
71.24	0\\
71.25	0\\
71.26	0\\
71.27	0\\
71.28	0\\
71.29	0\\
71.3	0\\
71.31	0\\
71.32	0\\
71.33	0\\
71.34	0\\
71.35	0\\
71.36	0\\
71.37	0\\
71.38	0\\
71.39	0\\
71.4	0\\
71.41	0\\
71.42	0\\
71.43	0\\
71.44	0\\
71.45	0\\
71.46	0\\
71.47	0\\
71.48	0\\
71.49	0\\
71.5	0\\
71.51	0\\
71.52	0\\
71.53	0\\
71.54	0\\
71.55	0\\
71.56	0\\
71.57	0\\
71.58	0\\
71.59	0\\
71.6	0\\
71.61	0\\
71.62	0\\
71.63	0\\
71.64	0\\
71.65	0\\
71.66	0\\
71.67	0\\
71.68	0\\
71.69	0\\
71.7	0\\
71.71	0\\
71.72	0\\
71.73	0\\
71.74	0\\
71.75	0\\
71.76	0\\
71.77	0\\
71.78	0\\
71.79	0\\
71.8	0\\
71.81	0\\
71.82	0\\
71.83	0\\
71.84	0\\
71.85	0\\
71.86	0\\
71.87	0\\
71.88	0\\
71.89	0\\
71.9	0\\
71.91	0\\
71.92	0\\
71.93	0\\
71.94	0\\
71.95	0\\
71.96	0\\
71.97	0\\
71.98	0\\
71.99	0\\
72	0\\
72.01	0\\
72.02	0\\
72.03	0\\
72.04	0\\
72.05	0\\
72.06	0\\
72.07	0\\
72.08	0\\
72.09	0\\
72.1	0\\
72.11	0\\
72.12	0\\
72.13	0\\
72.14	0\\
72.15	0\\
72.16	0\\
72.17	0\\
72.18	0\\
72.19	0\\
72.2	0\\
72.21	0\\
72.22	0\\
72.23	0\\
72.24	0\\
72.25	0\\
72.26	0\\
72.27	0\\
72.28	0\\
72.29	0\\
72.3	0\\
72.31	0\\
72.32	0\\
72.33	0\\
72.34	0\\
72.35	0\\
72.36	0\\
72.37	0\\
72.38	0\\
72.39	0\\
72.4	0\\
72.41	0\\
72.42	0\\
72.43	0\\
72.44	0\\
72.45	0\\
72.46	0\\
72.47	0\\
72.48	0\\
72.49	0\\
72.5	0\\
72.51	0\\
72.52	0\\
72.53	0\\
72.54	0\\
72.55	0\\
72.56	0\\
72.57	0\\
72.58	0\\
72.59	0\\
72.6	0\\
72.61	0\\
72.62	0\\
72.63	0\\
72.64	0\\
72.65	0\\
72.66	0\\
72.67	0\\
72.68	0\\
72.69	0\\
72.7	0\\
72.71	0\\
72.72	0\\
72.73	0\\
72.74	0\\
72.75	0\\
72.76	0\\
72.77	0\\
72.78	0\\
72.79	0\\
72.8	0\\
72.81	0\\
72.82	0\\
72.83	0\\
72.84	0\\
72.85	0\\
72.86	0\\
72.87	0\\
72.88	0\\
72.89	0\\
72.9	0\\
72.91	0\\
72.92	0\\
72.93	0\\
72.94	0\\
72.95	0\\
72.96	0\\
72.97	0\\
72.98	0\\
72.99	0\\
73	0\\
73.01	0\\
73.02	0\\
73.03	0\\
73.04	0\\
73.05	0\\
73.06	0\\
73.07	0\\
73.08	0\\
73.09	0\\
73.1	0\\
73.11	0\\
73.12	0\\
73.13	0\\
73.14	0\\
73.15	0\\
73.16	0\\
73.17	0\\
73.18	0\\
73.19	0\\
73.2	0\\
73.21	0\\
73.22	0\\
73.23	0\\
73.24	0\\
73.25	0\\
73.26	0\\
73.27	0\\
73.28	0\\
73.29	0\\
73.3	0\\
73.31	0\\
73.32	0\\
73.33	0\\
73.34	0\\
73.35	0\\
73.36	0\\
73.37	0\\
73.38	0\\
73.39	0\\
73.4	0\\
73.41	0\\
73.42	0\\
73.43	0\\
73.44	0\\
73.45	0\\
73.46	0\\
73.47	0\\
73.48	0\\
73.49	0\\
73.5	0\\
73.51	0\\
73.52	0\\
73.53	0\\
73.54	0\\
73.55	0\\
73.56	0\\
73.57	0\\
73.58	0\\
73.59	0\\
73.6	0\\
73.61	0\\
73.62	0\\
73.63	0\\
73.64	0\\
73.65	0\\
73.66	0\\
73.67	0\\
73.68	0\\
73.69	0\\
73.7	0\\
73.71	0\\
73.72	0\\
73.73	0\\
73.74	0\\
73.75	0\\
73.76	0\\
73.77	0\\
73.78	0\\
73.79	0\\
73.8	0\\
73.81	0\\
73.82	0\\
73.83	0\\
73.84	0\\
73.85	0\\
73.86	0\\
73.87	0\\
73.88	0\\
73.89	0\\
73.9	0\\
73.91	0\\
73.92	0\\
73.93	0\\
73.94	0\\
73.95	0\\
73.96	0\\
73.97	0\\
73.98	0\\
73.99	0\\
74	0\\
74.01	0\\
74.02	0\\
74.03	0\\
74.04	0\\
74.05	0\\
74.06	0\\
74.07	0\\
74.08	0\\
74.09	0\\
74.1	0\\
74.11	0\\
74.12	0\\
74.13	0\\
74.14	0\\
74.15	0\\
74.16	0\\
74.17	0\\
74.18	0\\
74.19	0\\
74.2	0\\
74.21	0\\
74.22	0\\
74.23	0\\
74.24	0\\
74.25	0\\
74.26	0\\
74.27	0\\
74.28	0\\
74.29	0\\
74.3	0\\
74.31	0\\
74.32	0\\
74.33	0\\
74.34	0\\
74.35	0\\
74.36	0\\
74.37	0\\
74.38	0\\
74.39	0\\
74.4	0\\
74.41	0\\
74.42	0\\
74.43	0\\
74.44	0\\
74.45	0\\
74.46	0\\
74.47	0\\
74.48	0\\
74.49	0\\
74.5	0\\
74.51	0\\
74.52	0\\
74.53	0\\
74.54	0\\
74.55	0\\
74.56	0\\
74.57	0\\
74.58	0\\
74.59	0\\
74.6	0\\
74.61	0\\
74.62	0\\
74.63	0\\
74.64	0\\
74.65	0\\
74.66	0\\
74.67	0\\
74.68	0\\
74.69	0\\
74.7	0\\
74.71	0\\
74.72	0\\
74.73	0\\
74.74	0\\
74.75	0\\
74.76	0\\
74.77	0\\
74.78	0\\
74.79	0\\
74.8	0\\
74.81	0\\
74.82	0\\
74.83	0\\
74.84	0\\
74.85	0\\
74.86	0\\
74.87	0\\
74.88	0\\
74.89	0\\
74.9	0\\
74.91	0\\
74.92	0\\
74.93	0\\
74.94	0\\
74.95	0\\
74.96	0\\
74.97	0\\
74.98	0\\
74.99	0\\
75	0\\
75.01	0\\
75.02	0\\
75.03	0\\
75.04	0\\
75.05	0\\
75.06	0\\
75.07	0\\
75.08	0\\
75.09	0\\
75.1	0\\
75.11	0\\
75.12	0\\
75.13	0\\
75.14	0\\
75.15	0\\
75.16	0\\
75.17	0\\
75.18	0\\
75.19	0\\
75.2	0\\
75.21	0\\
75.22	0\\
75.23	0\\
75.24	0\\
75.25	0\\
75.26	0\\
75.27	0\\
75.28	0\\
75.29	0\\
75.3	0\\
75.31	0\\
75.32	0\\
75.33	0\\
75.34	0\\
75.35	0\\
75.36	0\\
75.37	0\\
75.38	0\\
75.39	0\\
75.4	0\\
75.41	0\\
75.42	0\\
75.43	0\\
75.44	0\\
75.45	0\\
75.46	0\\
75.47	0\\
75.48	0\\
75.49	0\\
75.5	0\\
75.51	0\\
75.52	0\\
75.53	0\\
75.54	0\\
75.55	0\\
75.56	0\\
75.57	0\\
75.58	0\\
75.59	0\\
75.6	0\\
75.61	0\\
75.62	0\\
75.63	0\\
75.64	0\\
75.65	0\\
75.66	0\\
75.67	0\\
75.68	0\\
75.69	0\\
75.7	0\\
75.71	0\\
75.72	0\\
75.73	0\\
75.74	0\\
75.75	0\\
75.76	0\\
75.77	0\\
75.78	0\\
75.79	0\\
75.8	0\\
75.81	0\\
75.82	0\\
75.83	0\\
75.84	0\\
75.85	0\\
75.86	0\\
75.87	0\\
75.88	0\\
75.89	0\\
75.9	0\\
75.91	0\\
75.92	0\\
75.93	0\\
75.94	0\\
75.95	0\\
75.96	0\\
75.97	0\\
75.98	0\\
75.99	0\\
76	0\\
76.01	0\\
76.02	0\\
76.03	0\\
76.04	0\\
76.05	0\\
76.06	0\\
76.07	0\\
76.08	0\\
76.09	0\\
76.1	0\\
76.11	0\\
76.12	0\\
76.13	0\\
76.14	0\\
76.15	0\\
76.16	0\\
76.17	0\\
76.18	0\\
76.19	0\\
76.2	0\\
76.21	0\\
76.22	0\\
76.23	0\\
76.24	0\\
76.25	0\\
76.26	0\\
76.27	0\\
76.28	0\\
76.29	0\\
76.3	0\\
76.31	0\\
76.32	0\\
76.33	0\\
76.34	0\\
76.35	0\\
76.36	0\\
76.37	0\\
76.38	0\\
76.39	0\\
76.4	0\\
76.41	0\\
76.42	0\\
76.43	0\\
76.44	0\\
76.45	0\\
76.46	0\\
76.47	0\\
76.48	0\\
76.49	0\\
76.5	0\\
76.51	0\\
76.52	0\\
76.53	0\\
76.54	0\\
76.55	0\\
76.56	0\\
76.57	0\\
76.58	0\\
76.59	0\\
76.6	0\\
76.61	0\\
76.62	0\\
76.63	0\\
76.64	0\\
76.65	0\\
76.66	0\\
76.67	0\\
76.68	0\\
76.69	0\\
76.7	0\\
76.71	0\\
76.72	0\\
76.73	0\\
76.74	0\\
76.75	0\\
76.76	0\\
76.77	0\\
76.78	0\\
76.79	0\\
76.8	0\\
76.81	0\\
76.82	0\\
76.83	0\\
76.84	0\\
76.85	0\\
76.86	0\\
76.87	0\\
76.88	0\\
76.89	0\\
76.9	0\\
76.91	0\\
76.92	0\\
76.93	0\\
76.94	0\\
76.95	0\\
76.96	0\\
76.97	0\\
76.98	0\\
76.99	0\\
77	0\\
77.01	0\\
77.02	0\\
77.03	0\\
77.04	0\\
77.05	0\\
77.06	0\\
77.07	0\\
77.08	0\\
77.09	0\\
77.1	0\\
77.11	0\\
77.12	0\\
77.13	0\\
77.14	0\\
77.15	0\\
77.16	0\\
77.17	0\\
77.18	0\\
77.19	0\\
77.2	0\\
77.21	0\\
77.22	0\\
77.23	0\\
77.24	0\\
77.25	0\\
77.26	0\\
77.27	0\\
77.28	0\\
77.29	0\\
77.3	0\\
77.31	0\\
77.32	0\\
77.33	0\\
77.34	0\\
77.35	0\\
77.36	0\\
77.37	0\\
77.38	0\\
77.39	0\\
77.4	0\\
77.41	0\\
77.42	0\\
77.43	0\\
77.44	0\\
77.45	0\\
77.46	0\\
77.47	0\\
77.48	0\\
77.49	0\\
77.5	0\\
77.51	0\\
77.52	0\\
77.53	0\\
77.54	0\\
77.55	0\\
77.56	0\\
77.57	0\\
77.58	0\\
77.59	0\\
77.6	0\\
77.61	0\\
77.62	0\\
77.63	0\\
77.64	0\\
77.65	0\\
77.66	0\\
77.67	0\\
77.68	0\\
77.69	0\\
77.7	0\\
77.71	0\\
77.72	0\\
77.73	0\\
77.74	0\\
77.75	0\\
77.76	0\\
77.77	0\\
77.78	0\\
77.79	0\\
77.8	0\\
77.81	0\\
77.82	0\\
77.83	0\\
77.84	0\\
77.85	0\\
77.86	0\\
77.87	0\\
77.88	0\\
77.89	0\\
77.9	0\\
77.91	0\\
77.92	0\\
77.93	0\\
77.94	0\\
77.95	0\\
77.96	0\\
77.97	0\\
77.98	0\\
77.99	0\\
78	0\\
78.01	0\\
78.02	0\\
78.03	0\\
78.04	0\\
78.05	0\\
78.06	0\\
78.07	0\\
78.08	0\\
78.09	0\\
78.1	0\\
78.11	0\\
78.12	0\\
78.13	0\\
78.14	0\\
78.15	0\\
78.16	0\\
78.17	0\\
78.18	0\\
78.19	0\\
78.2	0\\
78.21	0\\
78.22	0\\
78.23	0\\
78.24	0\\
78.25	0\\
78.26	0\\
78.27	0\\
78.28	0\\
78.29	0\\
78.3	0\\
78.31	0\\
78.32	0\\
78.33	0\\
78.34	0\\
78.35	0\\
78.36	0\\
78.37	0\\
78.38	0\\
78.39	0\\
78.4	0\\
78.41	0\\
78.42	0\\
78.43	0\\
78.44	0\\
78.45	0\\
78.46	0\\
78.47	0\\
78.48	0\\
78.49	0\\
78.5	0\\
78.51	0\\
78.52	0\\
78.53	0\\
78.54	0\\
78.55	0\\
78.56	0\\
78.57	0\\
78.58	0\\
78.59	0\\
78.6	0\\
78.61	0\\
78.62	0\\
78.63	0\\
78.64	0\\
78.65	0\\
78.66	0\\
78.67	0\\
78.68	0\\
78.69	0\\
78.7	0\\
78.71	0\\
78.72	0\\
78.73	0\\
78.74	0\\
78.75	0\\
78.76	0\\
78.77	0\\
78.78	0\\
78.79	0\\
78.8	0\\
78.81	0\\
78.82	0\\
78.83	0\\
78.84	0\\
78.85	0\\
78.86	0\\
78.87	0\\
78.88	0\\
78.89	0\\
78.9	0\\
78.91	0\\
78.92	0\\
78.93	0\\
78.94	0\\
78.95	0\\
78.96	0\\
78.97	0\\
78.98	0\\
78.99	0\\
79	0\\
79.01	0\\
79.02	0\\
79.03	0\\
79.04	0\\
79.05	0\\
79.06	0\\
79.07	0\\
79.08	0\\
79.09	0\\
79.1	0\\
79.11	0\\
79.12	0\\
79.13	0\\
79.14	0\\
79.15	0\\
79.16	0\\
79.17	0\\
79.18	0\\
79.19	0\\
79.2	0\\
79.21	0\\
79.22	0\\
79.23	0\\
79.24	0\\
79.25	0\\
79.26	0\\
79.27	0\\
79.28	0\\
79.29	0\\
79.3	0\\
79.31	0\\
79.32	0\\
79.33	0\\
79.34	0\\
79.35	0\\
79.36	0\\
79.37	0\\
79.38	0\\
79.39	0\\
79.4	0\\
79.41	0\\
79.42	0\\
79.43	0\\
79.44	0\\
79.45	0\\
79.46	0\\
79.47	0\\
79.48	0\\
79.49	0\\
79.5	0\\
79.51	0\\
79.52	0\\
79.53	0\\
79.54	0\\
79.55	0\\
79.56	0\\
79.57	0\\
79.58	0\\
79.59	0\\
79.6	0\\
79.61	0\\
79.62	0\\
79.63	0\\
79.64	0\\
79.65	0\\
79.66	0\\
79.67	0\\
79.68	0\\
79.69	0\\
79.7	0\\
79.71	0\\
79.72	0\\
79.73	0\\
79.74	0\\
79.75	0\\
79.76	0\\
79.77	0\\
79.78	0\\
79.79	0\\
79.8	0\\
79.81	0\\
79.82	0\\
79.83	0\\
79.84	0\\
79.85	0\\
79.86	0\\
79.87	0\\
79.88	0\\
79.89	0\\
79.9	0\\
79.91	0\\
79.92	0\\
79.93	0\\
79.94	0\\
79.95	0\\
79.96	0\\
79.97	0\\
79.98	0\\
79.99	0\\
80	0\\
80.01	0\\
};
\addplot [color=mycolor1,solid]
  table[row sep=crcr]{%
80.01	0\\
80.02	0\\
80.03	0\\
80.04	0\\
80.05	0\\
80.06	0\\
80.07	0\\
80.08	0\\
80.09	0\\
80.1	0\\
80.11	0\\
80.12	0\\
80.13	0\\
80.14	0\\
80.15	0\\
80.16	0\\
80.17	0\\
80.18	0\\
80.19	0\\
80.2	0\\
80.21	0\\
80.22	0\\
80.23	0\\
80.24	0\\
80.25	0\\
80.26	0\\
80.27	0\\
80.28	0\\
80.29	0\\
80.3	0\\
80.31	0\\
80.32	0\\
80.33	0\\
80.34	0\\
80.35	0\\
80.36	0\\
80.37	0\\
80.38	0\\
80.39	0\\
80.4	0\\
80.41	0\\
80.42	0\\
80.43	0\\
80.44	0\\
80.45	0\\
80.46	0\\
80.47	0\\
80.48	0\\
80.49	0\\
80.5	0\\
80.51	0\\
80.52	0\\
80.53	0\\
80.54	0\\
80.55	0\\
80.56	0\\
80.57	0\\
80.58	0\\
80.59	0\\
80.6	0\\
80.61	0\\
80.62	0\\
80.63	0\\
80.64	0\\
80.65	0\\
80.66	0\\
80.67	0\\
80.68	0\\
80.69	0\\
80.7	0\\
80.71	0\\
80.72	0\\
80.73	0\\
80.74	0\\
80.75	0\\
80.76	0\\
80.77	0\\
80.78	0\\
80.79	0\\
80.8	0\\
80.81	0\\
80.82	0\\
80.83	0\\
80.84	0\\
80.85	0\\
80.86	0\\
80.87	0\\
80.88	0\\
80.89	0\\
80.9	0\\
80.91	0\\
80.92	0\\
80.93	0\\
80.94	0\\
80.95	0\\
80.96	0\\
80.97	0\\
80.98	0\\
80.99	0\\
81	0\\
81.01	0\\
81.02	0\\
81.03	0\\
81.04	0\\
81.05	0\\
81.06	0\\
81.07	0\\
81.08	0\\
81.09	0\\
81.1	0\\
81.11	0\\
81.12	0\\
81.13	0\\
81.14	0\\
81.15	0\\
81.16	0\\
81.17	0\\
81.18	0\\
81.19	0\\
81.2	0\\
81.21	0\\
81.22	0\\
81.23	0\\
81.24	0\\
81.25	0\\
81.26	0\\
81.27	0\\
81.28	0\\
81.29	0\\
81.3	0\\
81.31	0\\
81.32	0\\
81.33	0\\
81.34	0\\
81.35	0\\
81.36	0\\
81.37	0\\
81.38	0\\
81.39	0\\
81.4	0\\
81.41	0\\
81.42	0\\
81.43	0\\
81.44	0\\
81.45	0\\
81.46	0\\
81.47	0\\
81.48	0\\
81.49	0\\
81.5	0\\
81.51	0\\
81.52	0\\
81.53	0\\
81.54	0\\
81.55	0\\
81.56	0\\
81.57	0\\
81.58	0\\
81.59	0\\
81.6	0\\
81.61	0\\
81.62	0\\
81.63	0\\
81.64	0\\
81.65	0\\
81.66	0\\
81.67	0\\
81.68	0\\
81.69	0\\
81.7	0\\
81.71	0\\
81.72	0\\
81.73	0\\
81.74	0\\
81.75	0\\
81.76	0\\
81.77	0\\
81.78	0\\
81.79	0\\
81.8	0\\
81.81	0\\
81.82	0\\
81.83	0\\
81.84	0\\
81.85	0\\
81.86	0\\
81.87	0\\
81.88	0\\
81.89	0\\
81.9	0\\
81.91	0\\
81.92	0\\
81.93	0\\
81.94	0\\
81.95	0\\
81.96	0\\
81.97	0\\
81.98	0\\
81.99	0\\
82	0\\
82.01	0\\
82.02	0\\
82.03	0\\
82.04	0\\
82.05	0\\
82.06	0\\
82.07	0\\
82.08	0\\
82.09	0\\
82.1	0\\
82.11	0\\
82.12	0\\
82.13	0\\
82.14	0\\
82.15	0\\
82.16	0\\
82.17	0\\
82.18	0\\
82.19	0\\
82.2	0\\
82.21	0\\
82.22	0\\
82.23	0\\
82.24	0\\
82.25	0\\
82.26	0\\
82.27	0\\
82.28	0\\
82.29	0\\
82.3	0\\
82.31	0\\
82.32	0\\
82.33	0\\
82.34	0\\
82.35	0\\
82.36	0\\
82.37	0\\
82.38	0\\
82.39	0\\
82.4	0\\
82.41	0\\
82.42	0\\
82.43	0\\
82.44	0\\
82.45	0\\
82.46	0\\
82.47	0\\
82.48	0\\
82.49	0\\
82.5	0\\
82.51	0\\
82.52	0\\
82.53	0\\
82.54	0\\
82.55	0\\
82.56	0\\
82.57	0\\
82.58	0\\
82.59	0\\
82.6	0\\
82.61	0\\
82.62	0\\
82.63	0\\
82.64	0\\
82.65	0\\
82.66	0\\
82.67	0\\
82.68	0\\
82.69	0\\
82.7	0\\
82.71	0\\
82.72	0\\
82.73	0\\
82.74	0\\
82.75	0\\
82.76	0\\
82.77	0\\
82.78	0\\
82.79	0\\
82.8	0\\
82.81	0\\
82.82	0\\
82.83	0\\
82.84	0\\
82.85	0\\
82.86	0\\
82.87	0\\
82.88	0\\
82.89	0\\
82.9	0\\
82.91	0\\
82.92	0\\
82.93	0\\
82.94	0\\
82.95	0\\
82.96	0\\
82.97	0\\
82.98	0\\
82.99	0\\
83	0\\
83.01	0\\
83.02	0\\
83.03	0\\
83.04	0\\
83.05	0\\
83.06	0\\
83.07	0\\
83.08	0\\
83.09	0\\
83.1	0\\
83.11	0\\
83.12	0\\
83.13	0\\
83.14	0\\
83.15	0\\
83.16	0\\
83.17	0\\
83.18	0\\
83.19	0\\
83.2	0\\
83.21	0\\
83.22	0\\
83.23	0\\
83.24	0\\
83.25	0\\
83.26	0\\
83.27	0\\
83.28	0\\
83.29	0\\
83.3	0\\
83.31	0\\
83.32	0\\
83.33	0\\
83.34	0\\
83.35	0\\
83.36	0\\
83.37	0\\
83.38	0\\
83.39	0\\
83.4	0\\
83.41	0\\
83.42	0\\
83.43	0\\
83.44	0\\
83.45	0\\
83.46	0\\
83.47	0\\
83.48	0\\
83.49	0\\
83.5	0\\
83.51	0\\
83.52	0\\
83.53	0\\
83.54	0\\
83.55	0\\
83.56	0\\
83.57	0\\
83.58	0\\
83.59	0\\
83.6	0\\
83.61	0\\
83.62	0\\
83.63	0\\
83.64	0\\
83.65	0\\
83.66	0\\
83.67	0\\
83.68	0\\
83.69	0\\
83.7	0\\
83.71	0\\
83.72	0\\
83.73	0\\
83.74	0\\
83.75	0\\
83.76	0\\
83.77	0\\
83.78	0\\
83.79	0\\
83.8	0\\
83.81	0\\
83.82	0\\
83.83	0\\
83.84	0\\
83.85	0\\
83.86	0\\
83.87	0\\
83.88	0\\
83.89	0\\
83.9	0\\
83.91	0\\
83.92	0\\
83.93	0\\
83.94	0\\
83.95	0\\
83.96	0\\
83.97	0\\
83.98	0\\
83.99	0\\
84	0\\
84.01	0\\
84.02	0\\
84.03	0\\
84.04	0\\
84.05	0\\
84.06	0\\
84.07	0\\
84.08	0\\
84.09	0\\
84.1	0\\
84.11	0\\
84.12	0\\
84.13	0\\
84.14	0\\
84.15	0\\
84.16	0\\
84.17	0\\
84.18	0\\
84.19	0\\
84.2	0\\
84.21	0\\
84.22	0\\
84.23	0\\
84.24	0\\
84.25	0\\
84.26	0\\
84.27	0\\
84.28	0\\
84.29	0\\
84.3	0\\
84.31	0\\
84.32	0\\
84.33	0\\
84.34	0\\
84.35	0\\
84.36	0\\
84.37	0\\
84.38	0\\
84.39	0\\
84.4	0\\
84.41	0\\
84.42	0\\
84.43	0\\
84.44	0\\
84.45	0\\
84.46	0\\
84.47	0\\
84.48	0\\
84.49	0\\
84.5	0\\
84.51	0\\
84.52	0\\
84.53	0\\
84.54	0\\
84.55	0\\
84.56	0\\
84.57	0\\
84.58	0\\
84.59	0\\
84.6	0\\
84.61	0\\
84.62	0\\
84.63	0\\
84.64	0\\
84.65	0\\
84.66	0\\
84.67	0\\
84.68	0\\
84.69	0\\
84.7	0\\
84.71	0\\
84.72	0\\
84.73	0\\
84.74	0\\
84.75	0\\
84.76	0\\
84.77	0\\
84.78	0\\
84.79	0\\
84.8	0\\
84.81	0\\
84.82	0\\
84.83	0\\
84.84	0\\
84.85	0\\
84.86	0\\
84.87	0\\
84.88	0\\
84.89	0\\
84.9	0\\
84.91	0\\
84.92	0\\
84.93	0\\
84.94	0\\
84.95	0\\
84.96	0\\
84.97	0\\
84.98	0\\
84.99	0\\
85	0\\
85.01	0\\
85.02	0\\
85.03	0\\
85.04	0\\
85.05	0\\
85.06	0\\
85.07	0\\
85.08	0\\
85.09	0\\
85.1	0\\
85.11	0\\
85.12	0\\
85.13	0\\
85.14	0\\
85.15	0\\
85.16	0\\
85.17	0\\
85.18	0\\
85.19	0\\
85.2	0\\
85.21	0\\
85.22	0\\
85.23	0\\
85.24	0\\
85.25	0\\
85.26	0\\
85.27	0\\
85.28	0\\
85.29	0\\
85.3	0\\
85.31	0\\
85.32	0\\
85.33	0\\
85.34	0\\
85.35	0\\
85.36	0\\
85.37	0\\
85.38	0\\
85.39	0\\
85.4	0\\
85.41	0\\
85.42	0\\
85.43	0\\
85.44	0\\
85.45	0\\
85.46	0\\
85.47	0\\
85.48	0\\
85.49	0\\
85.5	0\\
85.51	0\\
85.52	0\\
85.53	0\\
85.54	0\\
85.55	0\\
85.56	0\\
85.57	0\\
85.58	0\\
85.59	0\\
85.6	0\\
85.61	0\\
85.62	0\\
85.63	0\\
85.64	0\\
85.65	0\\
85.66	0\\
85.67	0\\
85.68	0\\
85.69	0\\
85.7	0\\
85.71	0\\
85.72	0\\
85.73	0\\
85.74	0\\
85.75	0\\
85.76	0\\
85.77	0\\
85.78	0\\
85.79	0\\
85.8	0\\
85.81	0\\
85.82	0\\
85.83	0\\
85.84	0\\
85.85	0\\
85.86	0\\
85.87	0\\
85.88	0\\
85.89	0\\
85.9	0\\
85.91	0\\
85.92	0\\
85.93	0\\
85.94	0\\
85.95	0\\
85.96	0\\
85.97	0\\
85.98	0\\
85.99	0\\
86	0\\
86.01	0\\
86.02	0\\
86.03	0\\
86.04	0\\
86.05	0\\
86.06	0\\
86.07	0\\
86.08	0\\
86.09	0\\
86.1	0\\
86.11	0\\
86.12	0\\
86.13	0\\
86.14	0\\
86.15	0\\
86.16	0\\
86.17	0\\
86.18	0\\
86.19	0\\
86.2	0\\
86.21	0\\
86.22	0\\
86.23	0\\
86.24	0\\
86.25	0\\
86.26	0\\
86.27	0\\
86.28	0\\
86.29	0\\
86.3	0\\
86.31	0\\
86.32	0\\
86.33	0\\
86.34	0\\
86.35	0\\
86.36	0\\
86.37	0\\
86.38	0\\
86.39	0\\
86.4	0\\
86.41	0\\
86.42	0\\
86.43	0\\
86.44	0\\
86.45	0\\
86.46	0\\
86.47	0\\
86.48	0\\
86.49	0\\
86.5	0\\
86.51	0\\
86.52	0\\
86.53	0\\
86.54	0\\
86.55	0\\
86.56	0\\
86.57	0\\
86.58	0\\
86.59	0\\
86.6	0\\
86.61	0\\
86.62	0\\
86.63	0\\
86.64	0\\
86.65	0\\
86.66	0\\
86.67	0\\
86.68	0\\
86.69	0\\
86.7	0\\
86.71	0\\
86.72	0\\
86.73	0\\
86.74	0\\
86.75	0\\
86.76	0\\
86.77	0\\
86.78	0\\
86.79	0\\
86.8	0\\
86.81	0\\
86.82	0\\
86.83	0\\
86.84	0\\
86.85	0\\
86.86	0\\
86.87	0\\
86.88	0\\
86.89	0\\
86.9	0\\
86.91	0\\
86.92	0\\
86.93	0\\
86.94	0\\
86.95	0\\
86.96	0\\
86.97	0\\
86.98	0\\
86.99	0\\
87	0\\
87.01	0\\
87.02	0\\
87.03	0\\
87.04	0\\
87.05	0\\
87.06	0\\
87.07	0\\
87.08	0\\
87.09	0\\
87.1	0\\
87.11	0\\
87.12	0\\
87.13	0\\
87.14	0\\
87.15	0\\
87.16	0\\
87.17	0\\
87.18	0\\
87.19	0\\
87.2	0\\
87.21	0\\
87.22	0\\
87.23	0\\
87.24	0\\
87.25	0\\
87.26	0\\
87.27	0\\
87.28	0\\
87.29	0\\
87.3	0\\
87.31	0\\
87.32	0\\
87.33	0\\
87.34	0\\
87.35	0\\
87.36	0\\
87.37	0\\
87.38	0\\
87.39	0\\
87.4	0\\
87.41	0\\
87.42	0\\
87.43	0\\
87.44	0\\
87.45	0\\
87.46	0\\
87.47	0\\
87.48	0\\
87.49	0\\
87.5	0\\
87.51	0\\
87.52	0\\
87.53	0\\
87.54	0\\
87.55	0\\
87.56	0\\
87.57	0\\
87.58	0\\
87.59	0\\
87.6	0\\
87.61	0\\
87.62	0\\
87.63	0\\
87.64	0\\
87.65	0\\
87.66	0\\
87.67	0\\
87.68	0\\
87.69	0\\
87.7	0\\
87.71	0\\
87.72	0\\
87.73	0\\
87.74	0\\
87.75	0\\
87.76	0\\
87.77	0\\
87.78	0\\
87.79	0\\
87.8	0\\
87.81	0\\
87.82	0\\
87.83	0\\
87.84	0\\
87.85	0\\
87.86	0\\
87.87	0\\
87.88	0\\
87.89	0\\
87.9	0\\
87.91	0\\
87.92	0\\
87.93	0\\
87.94	0\\
87.95	0\\
87.96	0\\
87.97	0\\
87.98	0\\
87.99	0\\
88	0\\
88.01	0\\
88.02	0\\
88.03	0\\
88.04	0\\
88.05	0\\
88.06	0\\
88.07	0\\
88.08	0\\
88.09	0\\
88.1	0\\
88.11	0\\
88.12	0\\
88.13	0\\
88.14	0\\
88.15	0\\
88.16	0\\
88.17	0\\
88.18	0\\
88.19	0\\
88.2	0\\
88.21	0\\
88.22	0\\
88.23	0\\
88.24	0\\
88.25	0\\
88.26	0\\
88.27	0\\
88.28	0\\
88.29	0\\
88.3	0\\
88.31	0\\
88.32	0\\
88.33	0\\
88.34	0\\
88.35	0\\
88.36	0\\
88.37	0\\
88.38	0\\
88.39	0\\
88.4	0\\
88.41	0\\
88.42	0\\
88.43	0\\
88.44	0\\
88.45	0\\
88.46	0\\
88.47	0\\
88.48	0\\
88.49	0\\
88.5	0\\
88.51	0\\
88.52	0\\
88.53	0\\
88.54	0\\
88.55	0\\
88.56	0\\
88.57	0\\
88.58	0\\
88.59	0\\
88.6	0\\
88.61	0\\
88.62	0\\
88.63	0\\
88.64	0\\
88.65	0\\
88.66	0\\
88.67	0\\
88.68	0\\
88.69	0\\
88.7	0\\
88.71	0\\
88.72	0\\
88.73	0\\
88.74	0\\
88.75	0\\
88.76	0\\
88.77	0\\
88.78	0\\
88.79	0\\
88.8	0\\
88.81	0\\
88.82	0\\
88.83	0\\
88.84	0\\
88.85	0\\
88.86	0\\
88.87	0\\
88.88	0\\
88.89	0\\
88.9	0\\
88.91	0\\
88.92	0\\
88.93	0\\
88.94	0\\
88.95	0\\
88.96	0\\
88.97	0\\
88.98	0\\
88.99	0\\
89	0\\
89.01	0\\
89.02	0\\
89.03	0\\
89.04	0\\
89.05	0\\
89.06	0\\
89.07	0\\
89.08	0\\
89.09	0\\
89.1	0\\
89.11	0\\
89.12	0\\
89.13	0\\
89.14	0\\
89.15	0\\
89.16	0\\
89.17	0\\
89.18	0\\
89.19	0\\
89.2	0\\
89.21	0\\
89.22	0\\
89.23	0\\
89.24	0\\
89.25	0\\
89.26	0\\
89.27	0\\
89.28	0\\
89.29	0\\
89.3	0\\
89.31	0\\
89.32	0\\
89.33	0\\
89.34	0\\
89.35	0\\
89.36	0\\
89.37	0\\
89.38	0\\
89.39	0\\
89.4	0\\
89.41	0\\
89.42	0\\
89.43	0\\
89.44	0\\
89.45	0\\
89.46	0\\
89.47	0\\
89.48	0\\
89.49	0\\
89.5	0\\
89.51	0\\
89.52	0\\
89.53	0\\
89.54	0\\
89.55	0\\
89.56	0\\
89.57	0\\
89.58	0\\
89.59	0\\
89.6	0\\
89.61	0\\
89.62	0\\
89.63	0\\
89.64	0\\
89.65	0\\
89.66	0\\
89.67	0\\
89.68	0\\
89.69	0\\
89.7	0\\
89.71	0\\
89.72	0\\
89.73	0\\
89.74	0\\
89.75	0\\
89.76	0\\
89.77	0\\
89.78	0\\
89.79	0\\
89.8	0\\
89.81	0\\
89.82	0\\
89.83	0\\
89.84	0\\
89.85	0\\
89.86	0\\
89.87	0\\
89.88	0\\
89.89	0\\
89.9	0\\
89.91	0\\
89.92	0\\
89.93	0\\
89.94	0\\
89.95	0\\
89.96	0\\
89.97	0\\
89.98	0\\
89.99	0\\
90	0\\
90.01	0\\
90.02	0\\
90.03	0\\
90.04	0\\
90.05	0\\
90.06	0\\
90.07	0\\
90.08	0\\
90.09	0\\
90.1	0\\
90.11	0\\
90.12	0\\
90.13	0\\
90.14	0\\
90.15	0\\
90.16	0\\
90.17	0\\
90.18	0\\
90.19	0\\
90.2	0\\
90.21	0\\
90.22	0\\
90.23	0\\
90.24	0\\
90.25	0\\
90.26	0\\
90.27	0\\
90.28	0\\
90.29	0\\
90.3	0\\
90.31	0\\
90.32	0\\
90.33	0\\
90.34	0\\
90.35	0\\
90.36	0\\
90.37	0\\
90.38	0\\
90.39	0\\
90.4	0\\
90.41	0\\
90.42	0\\
90.43	0\\
90.44	0\\
90.45	0\\
90.46	0\\
90.47	0\\
90.48	0\\
90.49	0\\
90.5	0\\
90.51	0\\
90.52	0\\
90.53	0\\
90.54	0\\
90.55	0\\
90.56	0\\
90.57	0\\
90.58	0\\
90.59	0\\
90.6	0\\
90.61	0\\
90.62	0\\
90.63	0\\
90.64	0\\
90.65	0\\
90.66	0\\
90.67	0\\
90.68	0\\
90.69	0\\
90.7	0\\
90.71	0\\
90.72	0\\
90.73	0\\
90.74	0\\
90.75	0\\
90.76	0\\
90.77	0\\
90.78	0\\
90.79	0\\
90.8	0\\
90.81	0\\
90.82	0\\
90.83	0\\
90.84	0\\
90.85	0\\
90.86	0\\
90.87	0\\
90.88	0\\
90.89	0\\
90.9	0\\
90.91	0\\
90.92	0\\
90.93	0\\
90.94	0\\
90.95	0\\
90.96	0\\
90.97	0\\
90.98	0\\
90.99	0\\
91	0\\
91.01	0\\
91.02	0\\
91.03	0\\
91.04	0\\
91.05	0\\
91.06	0\\
91.07	0\\
91.08	0\\
91.09	0\\
91.1	0\\
91.11	0\\
91.12	0\\
91.13	0\\
91.14	0\\
91.15	0\\
91.16	0\\
91.17	0\\
91.18	0\\
91.19	0\\
91.2	0\\
91.21	0\\
91.22	0\\
91.23	0\\
91.24	0\\
91.25	0\\
91.26	0\\
91.27	0\\
91.28	0\\
91.29	0\\
91.3	0\\
91.31	0\\
91.32	0\\
91.33	0\\
91.34	0\\
91.35	0\\
91.36	0\\
91.37	0\\
91.38	0\\
91.39	0\\
91.4	0\\
91.41	0\\
91.42	0\\
91.43	0\\
91.44	0\\
91.45	0\\
91.46	0\\
91.47	0\\
91.48	0\\
91.49	0\\
91.5	0\\
91.51	0\\
91.52	0\\
91.53	0\\
91.54	0\\
91.55	0\\
91.56	0\\
91.57	0\\
91.58	0\\
91.59	0\\
91.6	0\\
91.61	0\\
91.62	0\\
91.63	0\\
91.64	0\\
91.65	0\\
91.66	0\\
91.67	0\\
91.68	0\\
91.69	0\\
91.7	0\\
91.71	0\\
91.72	0\\
91.73	0\\
91.74	0\\
91.75	0\\
91.76	0\\
91.77	0\\
91.78	0\\
91.79	0\\
91.8	0\\
91.81	0\\
91.82	0\\
91.83	0\\
91.84	0\\
91.85	0\\
91.86	0\\
91.87	0\\
91.88	0\\
91.89	0\\
91.9	0\\
91.91	0\\
91.92	0\\
91.93	0\\
91.94	0\\
91.95	0\\
91.96	0\\
91.97	0\\
91.98	0\\
91.99	0\\
92	0\\
92.01	0\\
92.02	0\\
92.03	0\\
92.04	0\\
92.05	0\\
92.06	0\\
92.07	0\\
92.08	0\\
92.09	0\\
92.1	0\\
92.11	0\\
92.12	0\\
92.13	0\\
92.14	0\\
92.15	0\\
92.16	0\\
92.17	0\\
92.18	0\\
92.19	0\\
92.2	0\\
92.21	0\\
92.22	0\\
92.23	0\\
92.24	0\\
92.25	0\\
92.26	0\\
92.27	0\\
92.28	0\\
92.29	0\\
92.3	0\\
92.31	0\\
92.32	0\\
92.33	0\\
92.34	0\\
92.35	0\\
92.36	0\\
92.37	0\\
92.38	0\\
92.39	0\\
92.4	0\\
92.41	0\\
92.42	0\\
92.43	0\\
92.44	0\\
92.45	0\\
92.46	0\\
92.47	0\\
92.48	0\\
92.49	0\\
92.5	0\\
92.51	0\\
92.52	0\\
92.53	0\\
92.54	0\\
92.55	0\\
92.56	0\\
92.57	0\\
92.58	0\\
92.59	0\\
92.6	0\\
92.61	0\\
92.62	0\\
92.63	0\\
92.64	0\\
92.65	0\\
92.66	0\\
92.67	0\\
92.68	0\\
92.69	0\\
92.7	0\\
92.71	0\\
92.72	0\\
92.73	0\\
92.74	0\\
92.75	0\\
92.76	0\\
92.77	0\\
92.78	0\\
92.79	0\\
92.8	0\\
92.81	0\\
92.82	0\\
92.83	0\\
92.84	0\\
92.85	0\\
92.86	0\\
92.87	0\\
92.88	0\\
92.89	0\\
92.9	0\\
92.91	0\\
92.92	0\\
92.93	0\\
92.94	0\\
92.95	0\\
92.96	0\\
92.97	0\\
92.98	0\\
92.99	0\\
93	0\\
93.01	0\\
93.02	0\\
93.03	0\\
93.04	0\\
93.05	0\\
93.06	0\\
93.07	0\\
93.08	0\\
93.09	0\\
93.1	0\\
93.11	0\\
93.12	0\\
93.13	0\\
93.14	0\\
93.15	0\\
93.16	0\\
93.17	0\\
93.18	0\\
93.19	0\\
93.2	0\\
93.21	0\\
93.22	0\\
93.23	0\\
93.24	0\\
93.25	0\\
93.26	0\\
93.27	0\\
93.28	0\\
93.29	0\\
93.3	0\\
93.31	0\\
93.32	0\\
93.33	0\\
93.34	0\\
93.35	0\\
93.36	0\\
93.37	0\\
93.38	0\\
93.39	0\\
93.4	0\\
93.41	0\\
93.42	0\\
93.43	0\\
93.44	0\\
93.45	0\\
93.46	0\\
93.47	0\\
93.48	0\\
93.49	0\\
93.5	0\\
93.51	0\\
93.52	0\\
93.53	0\\
93.54	0\\
93.55	0\\
93.56	0\\
93.57	0\\
93.58	0\\
93.59	0\\
93.6	0\\
93.61	0\\
93.62	0\\
93.63	0\\
93.64	0\\
93.65	0\\
93.66	0\\
93.67	0\\
93.68	0\\
93.69	0\\
93.7	0\\
93.71	0\\
93.72	0\\
93.73	0\\
93.74	0\\
93.75	0\\
93.76	0\\
93.77	0\\
93.78	0\\
93.79	0\\
93.8	0\\
93.81	0\\
93.82	0\\
93.83	0\\
93.84	0\\
93.85	0\\
93.86	0\\
93.87	0\\
93.88	0\\
93.89	0\\
93.9	0\\
93.91	0\\
93.92	0\\
93.93	0\\
93.94	0\\
93.95	0\\
93.96	0\\
93.97	0\\
93.98	0\\
93.99	0\\
94	0\\
94.01	0\\
94.02	0\\
94.03	0\\
94.04	0\\
94.05	0\\
94.06	0\\
94.07	0\\
94.08	0\\
94.09	0\\
94.1	0\\
94.11	0\\
94.12	0\\
94.13	0\\
94.14	0\\
94.15	0\\
94.16	0\\
94.17	0\\
94.18	0\\
94.19	0\\
94.2	0\\
94.21	0\\
94.22	0\\
94.23	0\\
94.24	0\\
94.25	0\\
94.26	0\\
94.27	0\\
94.28	0\\
94.29	0\\
94.3	0\\
94.31	0\\
94.32	0\\
94.33	0\\
94.34	0\\
94.35	0\\
94.36	0\\
94.37	0\\
94.38	0\\
94.39	0\\
94.4	0\\
94.41	0\\
94.42	0\\
94.43	0\\
94.44	0\\
94.45	0\\
94.46	0\\
94.47	0\\
94.48	0\\
94.49	0\\
94.5	0\\
94.51	0\\
94.52	0\\
94.53	0\\
94.54	0\\
94.55	0\\
94.56	0\\
94.57	0\\
94.58	0\\
94.59	0\\
94.6	0\\
94.61	0\\
94.62	0\\
94.63	0\\
94.64	0\\
94.65	0\\
94.66	0\\
94.67	0\\
94.68	0\\
94.69	0\\
94.7	0\\
94.71	0\\
94.72	0\\
94.73	0\\
94.74	0\\
94.75	0\\
94.76	0\\
94.77	0\\
94.78	0\\
94.79	0\\
94.8	0\\
94.81	0\\
94.82	0\\
94.83	0\\
94.84	0\\
94.85	0\\
94.86	0\\
94.87	0\\
94.88	0\\
94.89	0\\
94.9	0\\
94.91	0\\
94.92	0\\
94.93	0\\
94.94	0\\
94.95	0\\
94.96	0\\
94.97	0\\
94.98	0\\
94.99	0\\
95	0\\
95.01	0\\
95.02	0\\
95.03	0\\
95.04	0\\
95.05	0\\
95.06	0\\
95.07	0\\
95.08	0\\
95.09	0\\
95.1	0\\
95.11	0\\
95.12	0\\
95.13	0\\
95.14	0\\
95.15	0\\
95.16	0\\
95.17	0\\
95.18	0\\
95.19	0\\
95.2	0\\
95.21	0\\
95.22	0\\
95.23	0\\
95.24	0\\
95.25	0\\
95.26	0\\
95.27	0\\
95.28	0\\
95.29	0\\
95.3	0\\
95.31	0\\
95.32	0\\
95.33	0\\
95.34	0\\
95.35	0\\
95.36	0\\
95.37	0\\
95.38	0\\
95.39	0\\
95.4	0\\
95.41	0\\
95.42	0\\
95.43	0\\
95.44	0\\
95.45	0\\
95.46	0\\
95.47	0\\
95.48	0\\
95.49	0\\
95.5	0\\
95.51	0\\
95.52	0\\
95.53	0\\
95.54	0\\
95.55	0\\
95.56	0\\
95.57	0\\
95.58	0\\
95.59	0\\
95.6	0\\
95.61	0\\
95.62	0\\
95.63	0\\
95.64	0\\
95.65	0\\
95.66	0\\
95.67	0\\
95.68	0\\
95.69	0\\
95.7	0\\
95.71	0\\
95.72	0\\
95.73	0\\
95.74	0\\
95.75	0\\
95.76	0\\
95.77	0\\
95.78	0\\
95.79	0\\
95.8	0\\
95.81	0\\
95.82	0\\
95.83	0\\
95.84	0\\
95.85	0\\
95.86	0\\
95.87	0\\
95.88	0\\
95.89	0\\
95.9	0\\
95.91	0\\
95.92	0\\
95.93	0\\
95.94	0\\
95.95	0\\
95.96	0\\
95.97	0\\
95.98	0\\
95.99	0\\
96	0\\
96.01	0\\
96.02	0\\
96.03	0\\
96.04	0\\
96.05	0\\
96.06	0\\
96.07	0\\
96.08	0\\
96.09	0\\
96.1	0\\
96.11	0\\
96.12	0\\
96.13	0\\
96.14	0\\
96.15	0\\
96.16	0\\
96.17	0\\
96.18	0\\
96.19	0\\
96.2	0\\
96.21	0\\
96.22	0\\
96.23	0\\
96.24	0\\
96.25	0\\
96.26	0\\
96.27	0\\
96.28	0\\
96.29	0\\
96.3	0\\
96.31	0\\
96.32	0\\
96.33	0\\
96.34	0\\
96.35	0\\
96.36	0\\
96.37	0\\
96.38	0\\
96.39	0\\
96.4	0\\
96.41	0\\
96.42	0\\
96.43	0\\
96.44	0\\
96.45	0\\
96.46	0\\
96.47	0\\
96.48	0\\
96.49	0\\
96.5	0\\
96.51	0\\
96.52	0\\
96.53	0\\
96.54	0\\
96.55	0\\
96.56	0\\
96.57	0\\
96.58	0\\
96.59	0\\
96.6	0\\
96.61	0\\
96.62	0\\
96.63	0\\
96.64	0\\
96.65	0\\
96.66	0\\
96.67	0\\
96.68	0\\
96.69	0\\
96.7	0\\
96.71	0\\
96.72	0\\
96.73	0\\
96.74	0\\
96.75	0\\
96.76	0\\
96.77	0\\
96.78	0\\
96.79	0\\
96.8	0\\
96.81	0\\
96.82	0\\
96.83	0\\
96.84	0\\
96.85	0\\
96.86	0\\
96.87	0\\
96.88	0\\
96.89	0\\
96.9	0\\
96.91	0\\
96.92	0\\
96.93	0\\
96.94	0\\
96.95	0\\
96.96	0\\
96.97	0\\
96.98	0\\
96.99	0\\
97	0\\
97.01	0\\
97.02	0\\
97.03	0\\
97.04	0\\
97.05	0\\
97.06	0\\
97.07	0\\
97.08	0\\
97.09	0\\
97.1	0\\
97.11	0\\
97.12	0\\
97.13	0\\
97.14	0\\
97.15	0\\
97.16	0\\
97.17	0\\
97.18	0\\
97.19	0\\
97.2	0\\
97.21	0\\
97.22	0\\
97.23	0\\
97.24	0\\
97.25	0\\
97.26	0\\
97.27	0\\
97.28	0\\
97.29	0\\
97.3	0\\
97.31	0\\
97.32	0\\
97.33	0\\
97.34	0\\
97.35	0\\
97.36	0\\
97.37	0\\
97.38	0\\
97.39	0\\
97.4	0\\
97.41	0\\
97.42	0\\
97.43	0\\
97.44	0\\
97.45	0\\
97.46	0\\
97.47	0\\
97.48	0\\
97.49	0\\
97.5	0\\
97.51	0\\
97.52	0\\
97.53	0\\
97.54	0\\
97.55	0\\
97.56	0\\
97.57	0\\
97.58	0\\
97.59	0\\
97.6	0\\
97.61	0\\
97.62	0\\
97.63	0\\
97.64	0\\
97.65	0\\
97.66	0\\
97.67	0\\
97.68	0\\
97.69	0\\
97.7	0\\
97.71	0\\
97.72	0\\
97.73	0\\
97.74	0\\
97.75	0\\
97.76	0\\
97.77	0\\
97.78	0\\
97.79	0\\
97.8	0\\
97.81	0\\
97.82	0\\
97.83	0\\
97.84	0\\
97.85	0\\
97.86	0\\
97.87	0\\
97.88	0\\
97.89	0\\
97.9	0\\
97.91	0\\
97.92	0\\
97.93	0\\
97.94	0\\
97.95	0\\
97.96	0\\
97.97	0\\
97.98	0\\
97.99	0\\
98	0\\
98.01	0\\
98.02	0\\
98.03	0\\
98.04	0\\
98.05	0\\
98.06	0\\
98.07	3.32142083815171e-05\\
98.08	0.000117592884186023\\
98.09	0.000202637737334657\\
98.1	0.000288355406196129\\
98.11	0.000374752604359582\\
98.12	0.000461836121707214\\
98.13	0.000549612825508049\\
98.14	0.000638089661533441\\
98.15	0.000727273655194927\\
98.16	0.000817171912705021\\
98.17	0.000907791622261615\\
98.18	0.000999140055256649\\
98.19	0.00103249012756317\\
98.2	0.00105230944648016\\
98.21	0.00107227526507901\\
98.22	0.0010924000963525\\
98.23	0.00111268506241714\\
98.24	0.00113313128697955\\
98.25	0.00115373989512096\\
98.26	0.00117451201307303\\
98.27	0.00119544876798465\\
98.28	0.00121655128767936\\
98.29	0.00123782070040311\\
98.3	0.00125925813243572\\
98.31	0.00128086470942302\\
98.32	0.00130264155650186\\
98.33	0.00132458979780865\\
98.34	0.00134671055428548\\
98.35	0.00136900494534457\\
98.36	0.00139147408853117\\
98.37	0.00141414498729552\\
98.38	0.00143702049814452\\
98.39	0.00146010259331748\\
98.4	0.00148339326447661\\
98.41	0.00150689452291022\\
98.42	0.00153060839973829\\
98.43	0.00155453693442337\\
98.44	0.00157868218253603\\
98.45	0.00160304621947003\\
98.46	0.00162763114063745\\
98.47	0.00165243906166219\\
98.48	0.00167747211858005\\
98.49	0.00170273246804082\\
98.5	0.00172822228751266\\
98.51	0.00175394377548859\\
98.52	0.0017798991516953\\
98.53	0.00180609065730417\\
98.54	0.00183252055531797\\
98.55	0.00185919113216576\\
98.56	0.00188610469640969\\
98.57	0.00191326357896613\\
98.58	0.00194067013332923\\
98.59	0.00196832673579699\\
98.6	0.00199623578569987\\
98.61	0.00202439970563186\\
98.62	0.00205282094168419\\
98.63	0.00208150196368166\\
98.64	0.00211044526542164\\
98.65	0.00213965336491573\\
98.66	0.00216912880354172\\
98.67	0.00219887414697771\\
98.68	0.0022288919858204\\
98.69	0.00225918492775749\\
98.7	0.00228975560288559\\
98.71	0.00232060666640964\\
98.72	0.00235174079888165\\
98.73	0.00238316070644164\\
98.74	0.0024148691210655\\
98.75	0.00244686880081364\\
98.76	0.00247916253007931\\
98.77	0.00251175311983919\\
98.78	0.00254464340790631\\
98.79	0.00257783625919663\\
98.8	0.00261133456599068\\
98.81	0.00264514124819473\\
98.82	0.00267925925360437\\
98.83	0.00271369155817346\\
98.84	0.00274844114466484\\
98.85	0.0027835109979611\\
98.86	0.00281890413102645\\
98.87	0.00285462358517075\\
98.88	0.00289067243031597\\
98.89	0.00292705376526521\\
98.9	0.00296377071797415\\
98.91	0.0030008264458252\\
98.92	0.00303822413590406\\
98.93	0.00307596700527906\\
98.94	0.0031140583012829\\
98.95	0.00315250130179727\\
98.96	0.00319129931554\\
98.97	0.00323045568235492\\
98.98	0.00326997377350452\\
98.99	0.00330985699196527\\
99	0.00335010877272577\\
99.01	0.00339073258308769\\
99.02	0.00343173192296956\\
99.03	0.00347311032521334\\
99.04	0.00351487135589395\\
99.05	0.0035570186146317\\
99.06	0.00359955573490761\\
99.07	0.00364248638438174\\
99.08	0.00368581426521446\\
99.09	0.00372954311439079\\
99.1	0.00377367670404771\\
99.11	0.00381821884180458\\
99.12	0.0038631733710967\\
99.13	0.00390854417151193\\
99.14	0.00395433515913047\\
99.15	0.00400055028686791\\
99.16	0.00404719354482142\\
99.17	0.00409426896061924\\
99.18	0.00414178059977342\\
99.19	0.00418973256603592\\
99.2	0.00423812900175794\\
99.21	0.00428697408825284\\
99.22	0.0043362720461622\\
99.23	0.00438602713582556\\
99.24	0.00443624365765346\\
99.25	0.00448692595250408\\
99.26	0.00453807840206335\\
99.27	0.00458970542922874\\
99.28	0.0046418114984965\\
99.29	0.00469440111635265\\
99.3	0.00474747883166762\\
99.31	0.00480104923609458\\
99.32	0.00485511696447148\\
99.33	0.00490968669522694\\
99.34	0.0049647631507899\\
99.35	0.00502035109800311\\
99.36	0.00507645534854054\\
99.37	0.0051330807593287\\
99.38	0.00519023223297186\\
99.39	0.00524791471818133\\
99.4	0.00530613321020878\\
99.41	0.00536489275128353\\
99.42	0.00542419843105407\\
99.43	0.00548405538703363\\
99.44	0.00554446880504997\\
99.45	0.00560544391969941\\
99.46	0.00566698601480507\\
99.47	0.00572910042387948\\
99.48	0.00579179253059146\\
99.49	0.00585506776923741\\
99.5	0.00591893162521708\\
99.51	0.00598338963551369\\
99.52	0.00604844738917865\\
99.53	0.00611411052782078\\
99.54	0.00618038474610015\\
99.55	0.00624727579222653\\
99.56	0.0063147894684625\\
99.57	0.00638293163163133\\
99.58	0.00645170819362957\\
99.59	0.0065211251219445\\
99.6	0.00659118844017634\\
99.61	0.00666190420524143\\
99.62	0.00673327852119868\\
99.63	0.00680531754918415\\
99.64	0.00687802750794424\\
99.65	0.00695141467437383\\
99.66	0.00702548538405953\\
99.67	0.00710024603182786\\
99.68	0.00717570307229865\\
99.69	0.00725186302044349\\
99.7	0.00732873245214951\\
99.71	0.00740631800478823\\
99.72	0.0074846263777898\\
99.73	0.00756366433322253\\
99.74	0.00764343869637786\\
99.75	0.00772395635636067\\
99.76	0.00780522426668515\\
99.77	0.00788724944587618\\
99.78	0.00797003897807629\\
99.79	0.0080536000136582\\
99.8	0.00813793976984314\\
99.81	0.00822306553132476\\
99.82	0.00830898465089893\\
99.83	0.00839570455009931\\
99.84	0.00848323271983882\\
99.85	0.00857157672105694\\
99.86	0.00866074418537317\\
99.87	0.00875074281574629\\
99.88	0.00884158038713987\\
99.89	0.0089332647471938\\
99.9	0.009025803816902\\
99.91	0.00911920559129637\\
99.92	0.00921347814013702\\
99.93	0.00930862960860877\\
99.94	0.00940466821802401\\
99.95	0.00950160226653202\\
99.96	0.0095994401298347\\
99.97	0.00969819026190877\\
99.98	0.0097978611957346\\
99.99	0.00989846154403157\\
100	0.01\\
};
\addlegendentry{$q=3$};

\addplot [color=green,solid,forget plot]
  table[row sep=crcr]{%
0.01	0\\
0.02	0\\
0.03	0\\
0.04	0\\
0.05	0\\
0.06	0\\
0.07	0\\
0.08	0\\
0.09	0\\
0.1	0\\
0.11	0\\
0.12	0\\
0.13	0\\
0.14	0\\
0.15	0\\
0.16	0\\
0.17	0\\
0.18	0\\
0.19	0\\
0.2	0\\
0.21	0\\
0.22	0\\
0.23	0\\
0.24	0\\
0.25	0\\
0.26	0\\
0.27	0\\
0.28	0\\
0.29	0\\
0.3	0\\
0.31	0\\
0.32	0\\
0.33	0\\
0.34	0\\
0.35	0\\
0.36	0\\
0.37	0\\
0.38	0\\
0.39	0\\
0.4	0\\
0.41	0\\
0.42	0\\
0.43	0\\
0.44	0\\
0.45	0\\
0.46	0\\
0.47	0\\
0.48	0\\
0.49	0\\
0.5	0\\
0.51	0\\
0.52	0\\
0.53	0\\
0.54	0\\
0.55	0\\
0.56	0\\
0.57	0\\
0.58	0\\
0.59	0\\
0.6	0\\
0.61	0\\
0.62	0\\
0.63	0\\
0.64	0\\
0.65	0\\
0.66	0\\
0.67	0\\
0.68	0\\
0.69	0\\
0.7	0\\
0.71	0\\
0.72	0\\
0.73	0\\
0.74	0\\
0.75	0\\
0.76	0\\
0.77	0\\
0.78	0\\
0.79	0\\
0.8	0\\
0.81	0\\
0.82	0\\
0.83	0\\
0.84	0\\
0.85	0\\
0.86	0\\
0.87	0\\
0.88	0\\
0.89	0\\
0.9	0\\
0.91	0\\
0.92	0\\
0.93	0\\
0.94	0\\
0.95	0\\
0.96	0\\
0.97	0\\
0.98	0\\
0.99	0\\
1	0\\
1.01	0\\
1.02	0\\
1.03	0\\
1.04	0\\
1.05	0\\
1.06	0\\
1.07	0\\
1.08	0\\
1.09	0\\
1.1	0\\
1.11	0\\
1.12	0\\
1.13	0\\
1.14	0\\
1.15	0\\
1.16	0\\
1.17	0\\
1.18	0\\
1.19	0\\
1.2	0\\
1.21	0\\
1.22	0\\
1.23	0\\
1.24	0\\
1.25	0\\
1.26	0\\
1.27	0\\
1.28	0\\
1.29	0\\
1.3	0\\
1.31	0\\
1.32	0\\
1.33	0\\
1.34	0\\
1.35	0\\
1.36	0\\
1.37	0\\
1.38	0\\
1.39	0\\
1.4	0\\
1.41	0\\
1.42	0\\
1.43	0\\
1.44	0\\
1.45	0\\
1.46	0\\
1.47	0\\
1.48	0\\
1.49	0\\
1.5	0\\
1.51	0\\
1.52	0\\
1.53	0\\
1.54	0\\
1.55	0\\
1.56	0\\
1.57	0\\
1.58	0\\
1.59	0\\
1.6	0\\
1.61	0\\
1.62	0\\
1.63	0\\
1.64	0\\
1.65	0\\
1.66	0\\
1.67	0\\
1.68	0\\
1.69	0\\
1.7	0\\
1.71	0\\
1.72	0\\
1.73	0\\
1.74	0\\
1.75	0\\
1.76	0\\
1.77	0\\
1.78	0\\
1.79	0\\
1.8	0\\
1.81	0\\
1.82	0\\
1.83	0\\
1.84	0\\
1.85	0\\
1.86	0\\
1.87	0\\
1.88	0\\
1.89	0\\
1.9	0\\
1.91	0\\
1.92	0\\
1.93	0\\
1.94	0\\
1.95	0\\
1.96	0\\
1.97	0\\
1.98	0\\
1.99	0\\
2	0\\
2.01	0\\
2.02	0\\
2.03	0\\
2.04	0\\
2.05	0\\
2.06	0\\
2.07	0\\
2.08	0\\
2.09	0\\
2.1	0\\
2.11	0\\
2.12	0\\
2.13	0\\
2.14	0\\
2.15	0\\
2.16	0\\
2.17	0\\
2.18	0\\
2.19	0\\
2.2	0\\
2.21	0\\
2.22	0\\
2.23	0\\
2.24	0\\
2.25	0\\
2.26	0\\
2.27	0\\
2.28	0\\
2.29	0\\
2.3	0\\
2.31	0\\
2.32	0\\
2.33	0\\
2.34	0\\
2.35	0\\
2.36	0\\
2.37	0\\
2.38	0\\
2.39	0\\
2.4	0\\
2.41	0\\
2.42	0\\
2.43	0\\
2.44	0\\
2.45	0\\
2.46	0\\
2.47	0\\
2.48	0\\
2.49	0\\
2.5	0\\
2.51	0\\
2.52	0\\
2.53	0\\
2.54	0\\
2.55	0\\
2.56	0\\
2.57	0\\
2.58	0\\
2.59	0\\
2.6	0\\
2.61	0\\
2.62	0\\
2.63	0\\
2.64	0\\
2.65	0\\
2.66	0\\
2.67	0\\
2.68	0\\
2.69	0\\
2.7	0\\
2.71	0\\
2.72	0\\
2.73	0\\
2.74	0\\
2.75	0\\
2.76	0\\
2.77	0\\
2.78	0\\
2.79	0\\
2.8	0\\
2.81	0\\
2.82	0\\
2.83	0\\
2.84	0\\
2.85	0\\
2.86	0\\
2.87	0\\
2.88	0\\
2.89	0\\
2.9	0\\
2.91	0\\
2.92	0\\
2.93	0\\
2.94	0\\
2.95	0\\
2.96	0\\
2.97	0\\
2.98	0\\
2.99	0\\
3	0\\
3.01	0\\
3.02	0\\
3.03	0\\
3.04	0\\
3.05	0\\
3.06	0\\
3.07	0\\
3.08	0\\
3.09	0\\
3.1	0\\
3.11	0\\
3.12	0\\
3.13	0\\
3.14	0\\
3.15	0\\
3.16	0\\
3.17	0\\
3.18	0\\
3.19	0\\
3.2	0\\
3.21	0\\
3.22	0\\
3.23	0\\
3.24	0\\
3.25	0\\
3.26	0\\
3.27	0\\
3.28	0\\
3.29	0\\
3.3	0\\
3.31	0\\
3.32	0\\
3.33	0\\
3.34	0\\
3.35	0\\
3.36	0\\
3.37	0\\
3.38	0\\
3.39	0\\
3.4	0\\
3.41	0\\
3.42	0\\
3.43	0\\
3.44	0\\
3.45	0\\
3.46	0\\
3.47	0\\
3.48	0\\
3.49	0\\
3.5	0\\
3.51	0\\
3.52	0\\
3.53	0\\
3.54	0\\
3.55	0\\
3.56	0\\
3.57	0\\
3.58	0\\
3.59	0\\
3.6	0\\
3.61	0\\
3.62	0\\
3.63	0\\
3.64	0\\
3.65	0\\
3.66	0\\
3.67	0\\
3.68	0\\
3.69	0\\
3.7	0\\
3.71	0\\
3.72	0\\
3.73	0\\
3.74	0\\
3.75	0\\
3.76	0\\
3.77	0\\
3.78	0\\
3.79	0\\
3.8	0\\
3.81	0\\
3.82	0\\
3.83	0\\
3.84	0\\
3.85	0\\
3.86	0\\
3.87	0\\
3.88	0\\
3.89	0\\
3.9	0\\
3.91	0\\
3.92	0\\
3.93	0\\
3.94	0\\
3.95	0\\
3.96	0\\
3.97	0\\
3.98	0\\
3.99	0\\
4	0\\
4.01	0\\
4.02	0\\
4.03	0\\
4.04	0\\
4.05	0\\
4.06	0\\
4.07	0\\
4.08	0\\
4.09	0\\
4.1	0\\
4.11	0\\
4.12	0\\
4.13	0\\
4.14	0\\
4.15	0\\
4.16	0\\
4.17	0\\
4.18	0\\
4.19	0\\
4.2	0\\
4.21	0\\
4.22	0\\
4.23	0\\
4.24	0\\
4.25	0\\
4.26	0\\
4.27	0\\
4.28	0\\
4.29	0\\
4.3	0\\
4.31	0\\
4.32	0\\
4.33	0\\
4.34	0\\
4.35	0\\
4.36	0\\
4.37	0\\
4.38	0\\
4.39	0\\
4.4	0\\
4.41	0\\
4.42	0\\
4.43	0\\
4.44	0\\
4.45	0\\
4.46	0\\
4.47	0\\
4.48	0\\
4.49	0\\
4.5	0\\
4.51	0\\
4.52	0\\
4.53	0\\
4.54	0\\
4.55	0\\
4.56	0\\
4.57	0\\
4.58	0\\
4.59	0\\
4.6	0\\
4.61	0\\
4.62	0\\
4.63	0\\
4.64	0\\
4.65	0\\
4.66	0\\
4.67	0\\
4.68	0\\
4.69	0\\
4.7	0\\
4.71	0\\
4.72	0\\
4.73	0\\
4.74	0\\
4.75	0\\
4.76	0\\
4.77	0\\
4.78	0\\
4.79	0\\
4.8	0\\
4.81	0\\
4.82	0\\
4.83	0\\
4.84	0\\
4.85	0\\
4.86	0\\
4.87	0\\
4.88	0\\
4.89	0\\
4.9	0\\
4.91	0\\
4.92	0\\
4.93	0\\
4.94	0\\
4.95	0\\
4.96	0\\
4.97	0\\
4.98	0\\
4.99	0\\
5	0\\
5.01	0\\
5.02	0\\
5.03	0\\
5.04	0\\
5.05	0\\
5.06	0\\
5.07	0\\
5.08	0\\
5.09	0\\
5.1	0\\
5.11	0\\
5.12	0\\
5.13	0\\
5.14	0\\
5.15	0\\
5.16	0\\
5.17	0\\
5.18	0\\
5.19	0\\
5.2	0\\
5.21	0\\
5.22	0\\
5.23	0\\
5.24	0\\
5.25	0\\
5.26	0\\
5.27	0\\
5.28	0\\
5.29	0\\
5.3	0\\
5.31	0\\
5.32	0\\
5.33	0\\
5.34	0\\
5.35	0\\
5.36	0\\
5.37	0\\
5.38	0\\
5.39	0\\
5.4	0\\
5.41	0\\
5.42	0\\
5.43	0\\
5.44	0\\
5.45	0\\
5.46	0\\
5.47	0\\
5.48	0\\
5.49	0\\
5.5	0\\
5.51	0\\
5.52	0\\
5.53	0\\
5.54	0\\
5.55	0\\
5.56	0\\
5.57	0\\
5.58	0\\
5.59	0\\
5.6	0\\
5.61	0\\
5.62	0\\
5.63	0\\
5.64	0\\
5.65	0\\
5.66	0\\
5.67	0\\
5.68	0\\
5.69	0\\
5.7	0\\
5.71	0\\
5.72	0\\
5.73	0\\
5.74	0\\
5.75	0\\
5.76	0\\
5.77	0\\
5.78	0\\
5.79	0\\
5.8	0\\
5.81	0\\
5.82	0\\
5.83	0\\
5.84	0\\
5.85	0\\
5.86	0\\
5.87	0\\
5.88	0\\
5.89	0\\
5.9	0\\
5.91	0\\
5.92	0\\
5.93	0\\
5.94	0\\
5.95	0\\
5.96	0\\
5.97	0\\
5.98	0\\
5.99	0\\
6	0\\
6.01	0\\
6.02	0\\
6.03	0\\
6.04	0\\
6.05	0\\
6.06	0\\
6.07	0\\
6.08	0\\
6.09	0\\
6.1	0\\
6.11	0\\
6.12	0\\
6.13	0\\
6.14	0\\
6.15	0\\
6.16	0\\
6.17	0\\
6.18	0\\
6.19	0\\
6.2	0\\
6.21	0\\
6.22	0\\
6.23	0\\
6.24	0\\
6.25	0\\
6.26	0\\
6.27	0\\
6.28	0\\
6.29	0\\
6.3	0\\
6.31	0\\
6.32	0\\
6.33	0\\
6.34	0\\
6.35	0\\
6.36	0\\
6.37	0\\
6.38	0\\
6.39	0\\
6.4	0\\
6.41	0\\
6.42	0\\
6.43	0\\
6.44	0\\
6.45	0\\
6.46	0\\
6.47	0\\
6.48	0\\
6.49	0\\
6.5	0\\
6.51	0\\
6.52	0\\
6.53	0\\
6.54	0\\
6.55	0\\
6.56	0\\
6.57	0\\
6.58	0\\
6.59	0\\
6.6	0\\
6.61	0\\
6.62	0\\
6.63	0\\
6.64	0\\
6.65	0\\
6.66	0\\
6.67	0\\
6.68	0\\
6.69	0\\
6.7	0\\
6.71	0\\
6.72	0\\
6.73	0\\
6.74	0\\
6.75	0\\
6.76	0\\
6.77	0\\
6.78	0\\
6.79	0\\
6.8	0\\
6.81	0\\
6.82	0\\
6.83	0\\
6.84	0\\
6.85	0\\
6.86	0\\
6.87	0\\
6.88	0\\
6.89	0\\
6.9	0\\
6.91	0\\
6.92	0\\
6.93	0\\
6.94	0\\
6.95	0\\
6.96	0\\
6.97	0\\
6.98	0\\
6.99	0\\
7	0\\
7.01	0\\
7.02	0\\
7.03	0\\
7.04	0\\
7.05	0\\
7.06	0\\
7.07	0\\
7.08	0\\
7.09	0\\
7.1	0\\
7.11	0\\
7.12	0\\
7.13	0\\
7.14	0\\
7.15	0\\
7.16	0\\
7.17	0\\
7.18	0\\
7.19	0\\
7.2	0\\
7.21	0\\
7.22	0\\
7.23	0\\
7.24	0\\
7.25	0\\
7.26	0\\
7.27	0\\
7.28	0\\
7.29	0\\
7.3	0\\
7.31	0\\
7.32	0\\
7.33	0\\
7.34	0\\
7.35	0\\
7.36	0\\
7.37	0\\
7.38	0\\
7.39	0\\
7.4	0\\
7.41	0\\
7.42	0\\
7.43	0\\
7.44	0\\
7.45	0\\
7.46	0\\
7.47	0\\
7.48	0\\
7.49	0\\
7.5	0\\
7.51	0\\
7.52	0\\
7.53	0\\
7.54	0\\
7.55	0\\
7.56	0\\
7.57	0\\
7.58	0\\
7.59	0\\
7.6	0\\
7.61	0\\
7.62	0\\
7.63	0\\
7.64	0\\
7.65	0\\
7.66	0\\
7.67	0\\
7.68	0\\
7.69	0\\
7.7	0\\
7.71	0\\
7.72	0\\
7.73	0\\
7.74	0\\
7.75	0\\
7.76	0\\
7.77	0\\
7.78	0\\
7.79	0\\
7.8	0\\
7.81	0\\
7.82	0\\
7.83	0\\
7.84	0\\
7.85	0\\
7.86	0\\
7.87	0\\
7.88	0\\
7.89	0\\
7.9	0\\
7.91	0\\
7.92	0\\
7.93	0\\
7.94	0\\
7.95	0\\
7.96	0\\
7.97	0\\
7.98	0\\
7.99	0\\
8	0\\
8.01	0\\
8.02	0\\
8.03	0\\
8.04	0\\
8.05	0\\
8.06	0\\
8.07	0\\
8.08	0\\
8.09	0\\
8.1	0\\
8.11	0\\
8.12	0\\
8.13	0\\
8.14	0\\
8.15	0\\
8.16	0\\
8.17	0\\
8.18	0\\
8.19	0\\
8.2	0\\
8.21	0\\
8.22	0\\
8.23	0\\
8.24	0\\
8.25	0\\
8.26	0\\
8.27	0\\
8.28	0\\
8.29	0\\
8.3	0\\
8.31	0\\
8.32	0\\
8.33	0\\
8.34	0\\
8.35	0\\
8.36	0\\
8.37	0\\
8.38	0\\
8.39	0\\
8.4	0\\
8.41	0\\
8.42	0\\
8.43	0\\
8.44	0\\
8.45	0\\
8.46	0\\
8.47	0\\
8.48	0\\
8.49	0\\
8.5	0\\
8.51	0\\
8.52	0\\
8.53	0\\
8.54	0\\
8.55	0\\
8.56	0\\
8.57	0\\
8.58	0\\
8.59	0\\
8.6	0\\
8.61	0\\
8.62	0\\
8.63	0\\
8.64	0\\
8.65	0\\
8.66	0\\
8.67	0\\
8.68	0\\
8.69	0\\
8.7	0\\
8.71	0\\
8.72	0\\
8.73	0\\
8.74	0\\
8.75	0\\
8.76	0\\
8.77	0\\
8.78	0\\
8.79	0\\
8.8	0\\
8.81	0\\
8.82	0\\
8.83	0\\
8.84	0\\
8.85	0\\
8.86	0\\
8.87	0\\
8.88	0\\
8.89	0\\
8.9	0\\
8.91	0\\
8.92	0\\
8.93	0\\
8.94	0\\
8.95	0\\
8.96	0\\
8.97	0\\
8.98	0\\
8.99	0\\
9	0\\
9.01	0\\
9.02	0\\
9.03	0\\
9.04	0\\
9.05	0\\
9.06	0\\
9.07	0\\
9.08	0\\
9.09	0\\
9.1	0\\
9.11	0\\
9.12	0\\
9.13	0\\
9.14	0\\
9.15	0\\
9.16	0\\
9.17	0\\
9.18	0\\
9.19	0\\
9.2	0\\
9.21	0\\
9.22	0\\
9.23	0\\
9.24	0\\
9.25	0\\
9.26	0\\
9.27	0\\
9.28	0\\
9.29	0\\
9.3	0\\
9.31	0\\
9.32	0\\
9.33	0\\
9.34	0\\
9.35	0\\
9.36	0\\
9.37	0\\
9.38	0\\
9.39	0\\
9.4	0\\
9.41	0\\
9.42	0\\
9.43	0\\
9.44	0\\
9.45	0\\
9.46	0\\
9.47	0\\
9.48	0\\
9.49	0\\
9.5	0\\
9.51	0\\
9.52	0\\
9.53	0\\
9.54	0\\
9.55	0\\
9.56	0\\
9.57	0\\
9.58	0\\
9.59	0\\
9.6	0\\
9.61	0\\
9.62	0\\
9.63	0\\
9.64	0\\
9.65	0\\
9.66	0\\
9.67	0\\
9.68	0\\
9.69	0\\
9.7	0\\
9.71	0\\
9.72	0\\
9.73	0\\
9.74	0\\
9.75	0\\
9.76	0\\
9.77	0\\
9.78	0\\
9.79	0\\
9.8	0\\
9.81	0\\
9.82	0\\
9.83	0\\
9.84	0\\
9.85	0\\
9.86	0\\
9.87	0\\
9.88	0\\
9.89	0\\
9.9	0\\
9.91	0\\
9.92	0\\
9.93	0\\
9.94	0\\
9.95	0\\
9.96	0\\
9.97	0\\
9.98	0\\
9.99	0\\
10	0\\
10.01	0\\
10.02	0\\
10.03	0\\
10.04	0\\
10.05	0\\
10.06	0\\
10.07	0\\
10.08	0\\
10.09	0\\
10.1	0\\
10.11	0\\
10.12	0\\
10.13	0\\
10.14	0\\
10.15	0\\
10.16	0\\
10.17	0\\
10.18	0\\
10.19	0\\
10.2	0\\
10.21	0\\
10.22	0\\
10.23	0\\
10.24	0\\
10.25	0\\
10.26	0\\
10.27	0\\
10.28	0\\
10.29	0\\
10.3	0\\
10.31	0\\
10.32	0\\
10.33	0\\
10.34	0\\
10.35	0\\
10.36	0\\
10.37	0\\
10.38	0\\
10.39	0\\
10.4	0\\
10.41	0\\
10.42	0\\
10.43	0\\
10.44	0\\
10.45	0\\
10.46	0\\
10.47	0\\
10.48	0\\
10.49	0\\
10.5	0\\
10.51	0\\
10.52	0\\
10.53	0\\
10.54	0\\
10.55	0\\
10.56	0\\
10.57	0\\
10.58	0\\
10.59	0\\
10.6	0\\
10.61	0\\
10.62	0\\
10.63	0\\
10.64	0\\
10.65	0\\
10.66	0\\
10.67	0\\
10.68	0\\
10.69	0\\
10.7	0\\
10.71	0\\
10.72	0\\
10.73	0\\
10.74	0\\
10.75	0\\
10.76	0\\
10.77	0\\
10.78	0\\
10.79	0\\
10.8	0\\
10.81	0\\
10.82	0\\
10.83	0\\
10.84	0\\
10.85	0\\
10.86	0\\
10.87	0\\
10.88	0\\
10.89	0\\
10.9	0\\
10.91	0\\
10.92	0\\
10.93	0\\
10.94	0\\
10.95	0\\
10.96	0\\
10.97	0\\
10.98	0\\
10.99	0\\
11	0\\
11.01	0\\
11.02	0\\
11.03	0\\
11.04	0\\
11.05	0\\
11.06	0\\
11.07	0\\
11.08	0\\
11.09	0\\
11.1	0\\
11.11	0\\
11.12	0\\
11.13	0\\
11.14	0\\
11.15	0\\
11.16	0\\
11.17	0\\
11.18	0\\
11.19	0\\
11.2	0\\
11.21	0\\
11.22	0\\
11.23	0\\
11.24	0\\
11.25	0\\
11.26	0\\
11.27	0\\
11.28	0\\
11.29	0\\
11.3	0\\
11.31	0\\
11.32	0\\
11.33	0\\
11.34	0\\
11.35	0\\
11.36	0\\
11.37	0\\
11.38	0\\
11.39	0\\
11.4	0\\
11.41	0\\
11.42	0\\
11.43	0\\
11.44	0\\
11.45	0\\
11.46	0\\
11.47	0\\
11.48	0\\
11.49	0\\
11.5	0\\
11.51	0\\
11.52	0\\
11.53	0\\
11.54	0\\
11.55	0\\
11.56	0\\
11.57	0\\
11.58	0\\
11.59	0\\
11.6	0\\
11.61	0\\
11.62	0\\
11.63	0\\
11.64	0\\
11.65	0\\
11.66	0\\
11.67	0\\
11.68	0\\
11.69	0\\
11.7	0\\
11.71	0\\
11.72	0\\
11.73	0\\
11.74	0\\
11.75	0\\
11.76	0\\
11.77	0\\
11.78	0\\
11.79	0\\
11.8	0\\
11.81	0\\
11.82	0\\
11.83	0\\
11.84	0\\
11.85	0\\
11.86	0\\
11.87	0\\
11.88	0\\
11.89	0\\
11.9	0\\
11.91	0\\
11.92	0\\
11.93	0\\
11.94	0\\
11.95	0\\
11.96	0\\
11.97	0\\
11.98	0\\
11.99	0\\
12	0\\
12.01	0\\
12.02	0\\
12.03	0\\
12.04	0\\
12.05	0\\
12.06	0\\
12.07	0\\
12.08	0\\
12.09	0\\
12.1	0\\
12.11	0\\
12.12	0\\
12.13	0\\
12.14	0\\
12.15	0\\
12.16	0\\
12.17	0\\
12.18	0\\
12.19	0\\
12.2	0\\
12.21	0\\
12.22	0\\
12.23	0\\
12.24	0\\
12.25	0\\
12.26	0\\
12.27	0\\
12.28	0\\
12.29	0\\
12.3	0\\
12.31	0\\
12.32	0\\
12.33	0\\
12.34	0\\
12.35	0\\
12.36	0\\
12.37	0\\
12.38	0\\
12.39	0\\
12.4	0\\
12.41	0\\
12.42	0\\
12.43	0\\
12.44	0\\
12.45	0\\
12.46	0\\
12.47	0\\
12.48	0\\
12.49	0\\
12.5	0\\
12.51	0\\
12.52	0\\
12.53	0\\
12.54	0\\
12.55	0\\
12.56	0\\
12.57	0\\
12.58	0\\
12.59	0\\
12.6	0\\
12.61	0\\
12.62	0\\
12.63	0\\
12.64	0\\
12.65	0\\
12.66	0\\
12.67	0\\
12.68	0\\
12.69	0\\
12.7	0\\
12.71	0\\
12.72	0\\
12.73	0\\
12.74	0\\
12.75	0\\
12.76	0\\
12.77	0\\
12.78	0\\
12.79	0\\
12.8	0\\
12.81	0\\
12.82	0\\
12.83	0\\
12.84	0\\
12.85	0\\
12.86	0\\
12.87	0\\
12.88	0\\
12.89	0\\
12.9	0\\
12.91	0\\
12.92	0\\
12.93	0\\
12.94	0\\
12.95	0\\
12.96	0\\
12.97	0\\
12.98	0\\
12.99	0\\
13	0\\
13.01	0\\
13.02	0\\
13.03	0\\
13.04	0\\
13.05	0\\
13.06	0\\
13.07	0\\
13.08	0\\
13.09	0\\
13.1	0\\
13.11	0\\
13.12	0\\
13.13	0\\
13.14	0\\
13.15	0\\
13.16	0\\
13.17	0\\
13.18	0\\
13.19	0\\
13.2	0\\
13.21	0\\
13.22	0\\
13.23	0\\
13.24	0\\
13.25	0\\
13.26	0\\
13.27	0\\
13.28	0\\
13.29	0\\
13.3	0\\
13.31	0\\
13.32	0\\
13.33	0\\
13.34	0\\
13.35	0\\
13.36	0\\
13.37	0\\
13.38	0\\
13.39	0\\
13.4	0\\
13.41	0\\
13.42	0\\
13.43	0\\
13.44	0\\
13.45	0\\
13.46	0\\
13.47	0\\
13.48	0\\
13.49	0\\
13.5	0\\
13.51	0\\
13.52	0\\
13.53	0\\
13.54	0\\
13.55	0\\
13.56	0\\
13.57	0\\
13.58	0\\
13.59	0\\
13.6	0\\
13.61	0\\
13.62	0\\
13.63	0\\
13.64	0\\
13.65	0\\
13.66	0\\
13.67	0\\
13.68	0\\
13.69	0\\
13.7	0\\
13.71	0\\
13.72	0\\
13.73	0\\
13.74	0\\
13.75	0\\
13.76	0\\
13.77	0\\
13.78	0\\
13.79	0\\
13.8	0\\
13.81	0\\
13.82	0\\
13.83	0\\
13.84	0\\
13.85	0\\
13.86	0\\
13.87	0\\
13.88	0\\
13.89	0\\
13.9	0\\
13.91	0\\
13.92	0\\
13.93	0\\
13.94	0\\
13.95	0\\
13.96	0\\
13.97	0\\
13.98	0\\
13.99	0\\
14	0\\
14.01	0\\
14.02	0\\
14.03	0\\
14.04	0\\
14.05	0\\
14.06	0\\
14.07	0\\
14.08	0\\
14.09	0\\
14.1	0\\
14.11	0\\
14.12	0\\
14.13	0\\
14.14	0\\
14.15	0\\
14.16	0\\
14.17	0\\
14.18	0\\
14.19	0\\
14.2	0\\
14.21	0\\
14.22	0\\
14.23	0\\
14.24	0\\
14.25	0\\
14.26	0\\
14.27	0\\
14.28	0\\
14.29	0\\
14.3	0\\
14.31	0\\
14.32	0\\
14.33	0\\
14.34	0\\
14.35	0\\
14.36	0\\
14.37	0\\
14.38	0\\
14.39	0\\
14.4	0\\
14.41	0\\
14.42	0\\
14.43	0\\
14.44	0\\
14.45	0\\
14.46	0\\
14.47	0\\
14.48	0\\
14.49	0\\
14.5	0\\
14.51	0\\
14.52	0\\
14.53	0\\
14.54	0\\
14.55	0\\
14.56	0\\
14.57	0\\
14.58	0\\
14.59	0\\
14.6	0\\
14.61	0\\
14.62	0\\
14.63	0\\
14.64	0\\
14.65	0\\
14.66	0\\
14.67	0\\
14.68	0\\
14.69	0\\
14.7	0\\
14.71	0\\
14.72	0\\
14.73	0\\
14.74	0\\
14.75	0\\
14.76	0\\
14.77	0\\
14.78	0\\
14.79	0\\
14.8	0\\
14.81	0\\
14.82	0\\
14.83	0\\
14.84	0\\
14.85	0\\
14.86	0\\
14.87	0\\
14.88	0\\
14.89	0\\
14.9	0\\
14.91	0\\
14.92	0\\
14.93	0\\
14.94	0\\
14.95	0\\
14.96	0\\
14.97	0\\
14.98	0\\
14.99	0\\
15	0\\
15.01	0\\
15.02	0\\
15.03	0\\
15.04	0\\
15.05	0\\
15.06	0\\
15.07	0\\
15.08	0\\
15.09	0\\
15.1	0\\
15.11	0\\
15.12	0\\
15.13	0\\
15.14	0\\
15.15	0\\
15.16	0\\
15.17	0\\
15.18	0\\
15.19	0\\
15.2	0\\
15.21	0\\
15.22	0\\
15.23	0\\
15.24	0\\
15.25	0\\
15.26	0\\
15.27	0\\
15.28	0\\
15.29	0\\
15.3	0\\
15.31	0\\
15.32	0\\
15.33	0\\
15.34	0\\
15.35	0\\
15.36	0\\
15.37	0\\
15.38	0\\
15.39	0\\
15.4	0\\
15.41	0\\
15.42	0\\
15.43	0\\
15.44	0\\
15.45	0\\
15.46	0\\
15.47	0\\
15.48	0\\
15.49	0\\
15.5	0\\
15.51	0\\
15.52	0\\
15.53	0\\
15.54	0\\
15.55	0\\
15.56	0\\
15.57	0\\
15.58	0\\
15.59	0\\
15.6	0\\
15.61	0\\
15.62	0\\
15.63	0\\
15.64	0\\
15.65	0\\
15.66	0\\
15.67	0\\
15.68	0\\
15.69	0\\
15.7	0\\
15.71	0\\
15.72	0\\
15.73	0\\
15.74	0\\
15.75	0\\
15.76	0\\
15.77	0\\
15.78	0\\
15.79	0\\
15.8	0\\
15.81	0\\
15.82	0\\
15.83	0\\
15.84	0\\
15.85	0\\
15.86	0\\
15.87	0\\
15.88	0\\
15.89	0\\
15.9	0\\
15.91	0\\
15.92	0\\
15.93	0\\
15.94	0\\
15.95	0\\
15.96	0\\
15.97	0\\
15.98	0\\
15.99	0\\
16	0\\
16.01	0\\
16.02	0\\
16.03	0\\
16.04	0\\
16.05	0\\
16.06	0\\
16.07	0\\
16.08	0\\
16.09	0\\
16.1	0\\
16.11	0\\
16.12	0\\
16.13	0\\
16.14	0\\
16.15	0\\
16.16	0\\
16.17	0\\
16.18	0\\
16.19	0\\
16.2	0\\
16.21	0\\
16.22	0\\
16.23	0\\
16.24	0\\
16.25	0\\
16.26	0\\
16.27	0\\
16.28	0\\
16.29	0\\
16.3	0\\
16.31	0\\
16.32	0\\
16.33	0\\
16.34	0\\
16.35	0\\
16.36	0\\
16.37	0\\
16.38	0\\
16.39	0\\
16.4	0\\
16.41	0\\
16.42	0\\
16.43	0\\
16.44	0\\
16.45	0\\
16.46	0\\
16.47	0\\
16.48	0\\
16.49	0\\
16.5	0\\
16.51	0\\
16.52	0\\
16.53	0\\
16.54	0\\
16.55	0\\
16.56	0\\
16.57	0\\
16.58	0\\
16.59	0\\
16.6	0\\
16.61	0\\
16.62	0\\
16.63	0\\
16.64	0\\
16.65	0\\
16.66	0\\
16.67	0\\
16.68	0\\
16.69	0\\
16.7	0\\
16.71	0\\
16.72	0\\
16.73	0\\
16.74	0\\
16.75	0\\
16.76	0\\
16.77	0\\
16.78	0\\
16.79	0\\
16.8	0\\
16.81	0\\
16.82	0\\
16.83	0\\
16.84	0\\
16.85	0\\
16.86	0\\
16.87	0\\
16.88	0\\
16.89	0\\
16.9	0\\
16.91	0\\
16.92	0\\
16.93	0\\
16.94	0\\
16.95	0\\
16.96	0\\
16.97	0\\
16.98	0\\
16.99	0\\
17	0\\
17.01	0\\
17.02	0\\
17.03	0\\
17.04	0\\
17.05	0\\
17.06	0\\
17.07	0\\
17.08	0\\
17.09	0\\
17.1	0\\
17.11	0\\
17.12	0\\
17.13	0\\
17.14	0\\
17.15	0\\
17.16	0\\
17.17	0\\
17.18	0\\
17.19	0\\
17.2	0\\
17.21	0\\
17.22	0\\
17.23	0\\
17.24	0\\
17.25	0\\
17.26	0\\
17.27	0\\
17.28	0\\
17.29	0\\
17.3	0\\
17.31	0\\
17.32	0\\
17.33	0\\
17.34	0\\
17.35	0\\
17.36	0\\
17.37	0\\
17.38	0\\
17.39	0\\
17.4	0\\
17.41	0\\
17.42	0\\
17.43	0\\
17.44	0\\
17.45	0\\
17.46	0\\
17.47	0\\
17.48	0\\
17.49	0\\
17.5	0\\
17.51	0\\
17.52	0\\
17.53	0\\
17.54	0\\
17.55	0\\
17.56	0\\
17.57	0\\
17.58	0\\
17.59	0\\
17.6	0\\
17.61	0\\
17.62	0\\
17.63	0\\
17.64	0\\
17.65	0\\
17.66	0\\
17.67	0\\
17.68	0\\
17.69	0\\
17.7	0\\
17.71	0\\
17.72	0\\
17.73	0\\
17.74	0\\
17.75	0\\
17.76	0\\
17.77	0\\
17.78	0\\
17.79	0\\
17.8	0\\
17.81	0\\
17.82	0\\
17.83	0\\
17.84	0\\
17.85	0\\
17.86	0\\
17.87	0\\
17.88	0\\
17.89	0\\
17.9	0\\
17.91	0\\
17.92	0\\
17.93	0\\
17.94	0\\
17.95	0\\
17.96	0\\
17.97	0\\
17.98	0\\
17.99	0\\
18	0\\
18.01	0\\
18.02	0\\
18.03	0\\
18.04	0\\
18.05	0\\
18.06	0\\
18.07	0\\
18.08	0\\
18.09	0\\
18.1	0\\
18.11	0\\
18.12	0\\
18.13	0\\
18.14	0\\
18.15	0\\
18.16	0\\
18.17	0\\
18.18	0\\
18.19	0\\
18.2	0\\
18.21	0\\
18.22	0\\
18.23	0\\
18.24	0\\
18.25	0\\
18.26	0\\
18.27	0\\
18.28	0\\
18.29	0\\
18.3	0\\
18.31	0\\
18.32	0\\
18.33	0\\
18.34	0\\
18.35	0\\
18.36	0\\
18.37	0\\
18.38	0\\
18.39	0\\
18.4	0\\
18.41	0\\
18.42	0\\
18.43	0\\
18.44	0\\
18.45	0\\
18.46	0\\
18.47	0\\
18.48	0\\
18.49	0\\
18.5	0\\
18.51	0\\
18.52	0\\
18.53	0\\
18.54	0\\
18.55	0\\
18.56	0\\
18.57	0\\
18.58	0\\
18.59	0\\
18.6	0\\
18.61	0\\
18.62	0\\
18.63	0\\
18.64	0\\
18.65	0\\
18.66	0\\
18.67	0\\
18.68	0\\
18.69	0\\
18.7	0\\
18.71	0\\
18.72	0\\
18.73	0\\
18.74	0\\
18.75	0\\
18.76	0\\
18.77	0\\
18.78	0\\
18.79	0\\
18.8	0\\
18.81	0\\
18.82	0\\
18.83	0\\
18.84	0\\
18.85	0\\
18.86	0\\
18.87	0\\
18.88	0\\
18.89	0\\
18.9	0\\
18.91	0\\
18.92	0\\
18.93	0\\
18.94	0\\
18.95	0\\
18.96	0\\
18.97	0\\
18.98	0\\
18.99	0\\
19	0\\
19.01	0\\
19.02	0\\
19.03	0\\
19.04	0\\
19.05	0\\
19.06	0\\
19.07	0\\
19.08	0\\
19.09	0\\
19.1	0\\
19.11	0\\
19.12	0\\
19.13	0\\
19.14	0\\
19.15	0\\
19.16	0\\
19.17	0\\
19.18	0\\
19.19	0\\
19.2	0\\
19.21	0\\
19.22	0\\
19.23	0\\
19.24	0\\
19.25	0\\
19.26	0\\
19.27	0\\
19.28	0\\
19.29	0\\
19.3	0\\
19.31	0\\
19.32	0\\
19.33	0\\
19.34	0\\
19.35	0\\
19.36	0\\
19.37	0\\
19.38	0\\
19.39	0\\
19.4	0\\
19.41	0\\
19.42	0\\
19.43	0\\
19.44	0\\
19.45	0\\
19.46	0\\
19.47	0\\
19.48	0\\
19.49	0\\
19.5	0\\
19.51	0\\
19.52	0\\
19.53	0\\
19.54	0\\
19.55	0\\
19.56	0\\
19.57	0\\
19.58	0\\
19.59	0\\
19.6	0\\
19.61	0\\
19.62	0\\
19.63	0\\
19.64	0\\
19.65	0\\
19.66	0\\
19.67	0\\
19.68	0\\
19.69	0\\
19.7	0\\
19.71	0\\
19.72	0\\
19.73	0\\
19.74	0\\
19.75	0\\
19.76	0\\
19.77	0\\
19.78	0\\
19.79	0\\
19.8	0\\
19.81	0\\
19.82	0\\
19.83	0\\
19.84	0\\
19.85	0\\
19.86	0\\
19.87	0\\
19.88	0\\
19.89	0\\
19.9	0\\
19.91	0\\
19.92	0\\
19.93	0\\
19.94	0\\
19.95	0\\
19.96	0\\
19.97	0\\
19.98	0\\
19.99	0\\
20	0\\
20.01	0\\
20.02	0\\
20.03	0\\
20.04	0\\
20.05	0\\
20.06	0\\
20.07	0\\
20.08	0\\
20.09	0\\
20.1	0\\
20.11	0\\
20.12	0\\
20.13	0\\
20.14	0\\
20.15	0\\
20.16	0\\
20.17	0\\
20.18	0\\
20.19	0\\
20.2	0\\
20.21	0\\
20.22	0\\
20.23	0\\
20.24	0\\
20.25	0\\
20.26	0\\
20.27	0\\
20.28	0\\
20.29	0\\
20.3	0\\
20.31	0\\
20.32	0\\
20.33	0\\
20.34	0\\
20.35	0\\
20.36	0\\
20.37	0\\
20.38	0\\
20.39	0\\
20.4	0\\
20.41	0\\
20.42	0\\
20.43	0\\
20.44	0\\
20.45	0\\
20.46	0\\
20.47	0\\
20.48	0\\
20.49	0\\
20.5	0\\
20.51	0\\
20.52	0\\
20.53	0\\
20.54	0\\
20.55	0\\
20.56	0\\
20.57	0\\
20.58	0\\
20.59	0\\
20.6	0\\
20.61	0\\
20.62	0\\
20.63	0\\
20.64	0\\
20.65	0\\
20.66	0\\
20.67	0\\
20.68	0\\
20.69	0\\
20.7	0\\
20.71	0\\
20.72	0\\
20.73	0\\
20.74	0\\
20.75	0\\
20.76	0\\
20.77	0\\
20.78	0\\
20.79	0\\
20.8	0\\
20.81	0\\
20.82	0\\
20.83	0\\
20.84	0\\
20.85	0\\
20.86	0\\
20.87	0\\
20.88	0\\
20.89	0\\
20.9	0\\
20.91	0\\
20.92	0\\
20.93	0\\
20.94	0\\
20.95	0\\
20.96	0\\
20.97	0\\
20.98	0\\
20.99	0\\
21	0\\
21.01	0\\
21.02	0\\
21.03	0\\
21.04	0\\
21.05	0\\
21.06	0\\
21.07	0\\
21.08	0\\
21.09	0\\
21.1	0\\
21.11	0\\
21.12	0\\
21.13	0\\
21.14	0\\
21.15	0\\
21.16	0\\
21.17	0\\
21.18	0\\
21.19	0\\
21.2	0\\
21.21	0\\
21.22	0\\
21.23	0\\
21.24	0\\
21.25	0\\
21.26	0\\
21.27	0\\
21.28	0\\
21.29	0\\
21.3	0\\
21.31	0\\
21.32	0\\
21.33	0\\
21.34	0\\
21.35	0\\
21.36	0\\
21.37	0\\
21.38	0\\
21.39	0\\
21.4	0\\
21.41	0\\
21.42	0\\
21.43	0\\
21.44	0\\
21.45	0\\
21.46	0\\
21.47	0\\
21.48	0\\
21.49	0\\
21.5	0\\
21.51	0\\
21.52	0\\
21.53	0\\
21.54	0\\
21.55	0\\
21.56	0\\
21.57	0\\
21.58	0\\
21.59	0\\
21.6	0\\
21.61	0\\
21.62	0\\
21.63	0\\
21.64	0\\
21.65	0\\
21.66	0\\
21.67	0\\
21.68	0\\
21.69	0\\
21.7	0\\
21.71	0\\
21.72	0\\
21.73	0\\
21.74	0\\
21.75	0\\
21.76	0\\
21.77	0\\
21.78	0\\
21.79	0\\
21.8	0\\
21.81	0\\
21.82	0\\
21.83	0\\
21.84	0\\
21.85	0\\
21.86	0\\
21.87	0\\
21.88	0\\
21.89	0\\
21.9	0\\
21.91	0\\
21.92	0\\
21.93	0\\
21.94	0\\
21.95	0\\
21.96	0\\
21.97	0\\
21.98	0\\
21.99	0\\
22	0\\
22.01	0\\
22.02	0\\
22.03	0\\
22.04	0\\
22.05	0\\
22.06	0\\
22.07	0\\
22.08	0\\
22.09	0\\
22.1	0\\
22.11	0\\
22.12	0\\
22.13	0\\
22.14	0\\
22.15	0\\
22.16	0\\
22.17	0\\
22.18	0\\
22.19	0\\
22.2	0\\
22.21	0\\
22.22	0\\
22.23	0\\
22.24	0\\
22.25	0\\
22.26	0\\
22.27	0\\
22.28	0\\
22.29	0\\
22.3	0\\
22.31	0\\
22.32	0\\
22.33	0\\
22.34	0\\
22.35	0\\
22.36	0\\
22.37	0\\
22.38	0\\
22.39	0\\
22.4	0\\
22.41	0\\
22.42	0\\
22.43	0\\
22.44	0\\
22.45	0\\
22.46	0\\
22.47	0\\
22.48	0\\
22.49	0\\
22.5	0\\
22.51	0\\
22.52	0\\
22.53	0\\
22.54	0\\
22.55	0\\
22.56	0\\
22.57	0\\
22.58	0\\
22.59	0\\
22.6	0\\
22.61	0\\
22.62	0\\
22.63	0\\
22.64	0\\
22.65	0\\
22.66	0\\
22.67	0\\
22.68	0\\
22.69	0\\
22.7	0\\
22.71	0\\
22.72	0\\
22.73	0\\
22.74	0\\
22.75	0\\
22.76	0\\
22.77	0\\
22.78	0\\
22.79	0\\
22.8	0\\
22.81	0\\
22.82	0\\
22.83	0\\
22.84	0\\
22.85	0\\
22.86	0\\
22.87	0\\
22.88	0\\
22.89	0\\
22.9	0\\
22.91	0\\
22.92	0\\
22.93	0\\
22.94	0\\
22.95	0\\
22.96	0\\
22.97	0\\
22.98	0\\
22.99	0\\
23	0\\
23.01	0\\
23.02	0\\
23.03	0\\
23.04	0\\
23.05	0\\
23.06	0\\
23.07	0\\
23.08	0\\
23.09	0\\
23.1	0\\
23.11	0\\
23.12	0\\
23.13	0\\
23.14	0\\
23.15	0\\
23.16	0\\
23.17	0\\
23.18	0\\
23.19	0\\
23.2	0\\
23.21	0\\
23.22	0\\
23.23	0\\
23.24	0\\
23.25	0\\
23.26	0\\
23.27	0\\
23.28	0\\
23.29	0\\
23.3	0\\
23.31	0\\
23.32	0\\
23.33	0\\
23.34	0\\
23.35	0\\
23.36	0\\
23.37	0\\
23.38	0\\
23.39	0\\
23.4	0\\
23.41	0\\
23.42	0\\
23.43	0\\
23.44	0\\
23.45	0\\
23.46	0\\
23.47	0\\
23.48	0\\
23.49	0\\
23.5	0\\
23.51	0\\
23.52	0\\
23.53	0\\
23.54	0\\
23.55	0\\
23.56	0\\
23.57	0\\
23.58	0\\
23.59	0\\
23.6	0\\
23.61	0\\
23.62	0\\
23.63	0\\
23.64	0\\
23.65	0\\
23.66	0\\
23.67	0\\
23.68	0\\
23.69	0\\
23.7	0\\
23.71	0\\
23.72	0\\
23.73	0\\
23.74	0\\
23.75	0\\
23.76	0\\
23.77	0\\
23.78	0\\
23.79	0\\
23.8	0\\
23.81	0\\
23.82	0\\
23.83	0\\
23.84	0\\
23.85	0\\
23.86	0\\
23.87	0\\
23.88	0\\
23.89	0\\
23.9	0\\
23.91	0\\
23.92	0\\
23.93	0\\
23.94	0\\
23.95	0\\
23.96	0\\
23.97	0\\
23.98	0\\
23.99	0\\
24	0\\
24.01	0\\
24.02	0\\
24.03	0\\
24.04	0\\
24.05	0\\
24.06	0\\
24.07	0\\
24.08	0\\
24.09	0\\
24.1	0\\
24.11	0\\
24.12	0\\
24.13	0\\
24.14	0\\
24.15	0\\
24.16	0\\
24.17	0\\
24.18	0\\
24.19	0\\
24.2	0\\
24.21	0\\
24.22	0\\
24.23	0\\
24.24	0\\
24.25	0\\
24.26	0\\
24.27	0\\
24.28	0\\
24.29	0\\
24.3	0\\
24.31	0\\
24.32	0\\
24.33	0\\
24.34	0\\
24.35	0\\
24.36	0\\
24.37	0\\
24.38	0\\
24.39	0\\
24.4	0\\
24.41	0\\
24.42	0\\
24.43	0\\
24.44	0\\
24.45	0\\
24.46	0\\
24.47	0\\
24.48	0\\
24.49	0\\
24.5	0\\
24.51	0\\
24.52	0\\
24.53	0\\
24.54	0\\
24.55	0\\
24.56	0\\
24.57	0\\
24.58	0\\
24.59	0\\
24.6	0\\
24.61	0\\
24.62	0\\
24.63	0\\
24.64	0\\
24.65	0\\
24.66	0\\
24.67	0\\
24.68	0\\
24.69	0\\
24.7	0\\
24.71	0\\
24.72	0\\
24.73	0\\
24.74	0\\
24.75	0\\
24.76	0\\
24.77	0\\
24.78	0\\
24.79	0\\
24.8	0\\
24.81	0\\
24.82	0\\
24.83	0\\
24.84	0\\
24.85	0\\
24.86	0\\
24.87	0\\
24.88	0\\
24.89	0\\
24.9	0\\
24.91	0\\
24.92	0\\
24.93	0\\
24.94	0\\
24.95	0\\
24.96	0\\
24.97	0\\
24.98	0\\
24.99	0\\
25	0\\
25.01	0\\
25.02	0\\
25.03	0\\
25.04	0\\
25.05	0\\
25.06	0\\
25.07	0\\
25.08	0\\
25.09	0\\
25.1	0\\
25.11	0\\
25.12	0\\
25.13	0\\
25.14	0\\
25.15	0\\
25.16	0\\
25.17	0\\
25.18	0\\
25.19	0\\
25.2	0\\
25.21	0\\
25.22	0\\
25.23	0\\
25.24	0\\
25.25	0\\
25.26	0\\
25.27	0\\
25.28	0\\
25.29	0\\
25.3	0\\
25.31	0\\
25.32	0\\
25.33	0\\
25.34	0\\
25.35	0\\
25.36	0\\
25.37	0\\
25.38	0\\
25.39	0\\
25.4	0\\
25.41	0\\
25.42	0\\
25.43	0\\
25.44	0\\
25.45	0\\
25.46	0\\
25.47	0\\
25.48	0\\
25.49	0\\
25.5	0\\
25.51	0\\
25.52	0\\
25.53	0\\
25.54	0\\
25.55	0\\
25.56	0\\
25.57	0\\
25.58	0\\
25.59	0\\
25.6	0\\
25.61	0\\
25.62	0\\
25.63	0\\
25.64	0\\
25.65	0\\
25.66	0\\
25.67	0\\
25.68	0\\
25.69	0\\
25.7	0\\
25.71	0\\
25.72	0\\
25.73	0\\
25.74	0\\
25.75	0\\
25.76	0\\
25.77	0\\
25.78	0\\
25.79	0\\
25.8	0\\
25.81	0\\
25.82	0\\
25.83	0\\
25.84	0\\
25.85	0\\
25.86	0\\
25.87	0\\
25.88	0\\
25.89	0\\
25.9	0\\
25.91	0\\
25.92	0\\
25.93	0\\
25.94	0\\
25.95	0\\
25.96	0\\
25.97	0\\
25.98	0\\
25.99	0\\
26	0\\
26.01	0\\
26.02	0\\
26.03	0\\
26.04	0\\
26.05	0\\
26.06	0\\
26.07	0\\
26.08	0\\
26.09	0\\
26.1	0\\
26.11	0\\
26.12	0\\
26.13	0\\
26.14	0\\
26.15	0\\
26.16	0\\
26.17	0\\
26.18	0\\
26.19	0\\
26.2	0\\
26.21	0\\
26.22	0\\
26.23	0\\
26.24	0\\
26.25	0\\
26.26	0\\
26.27	0\\
26.28	0\\
26.29	0\\
26.3	0\\
26.31	0\\
26.32	0\\
26.33	0\\
26.34	0\\
26.35	0\\
26.36	0\\
26.37	0\\
26.38	0\\
26.39	0\\
26.4	0\\
26.41	0\\
26.42	0\\
26.43	0\\
26.44	0\\
26.45	0\\
26.46	0\\
26.47	0\\
26.48	0\\
26.49	0\\
26.5	0\\
26.51	0\\
26.52	0\\
26.53	0\\
26.54	0\\
26.55	0\\
26.56	0\\
26.57	0\\
26.58	0\\
26.59	0\\
26.6	0\\
26.61	0\\
26.62	0\\
26.63	0\\
26.64	0\\
26.65	0\\
26.66	0\\
26.67	0\\
26.68	0\\
26.69	0\\
26.7	0\\
26.71	0\\
26.72	0\\
26.73	0\\
26.74	0\\
26.75	0\\
26.76	0\\
26.77	0\\
26.78	0\\
26.79	0\\
26.8	0\\
26.81	0\\
26.82	0\\
26.83	0\\
26.84	0\\
26.85	0\\
26.86	0\\
26.87	0\\
26.88	0\\
26.89	0\\
26.9	0\\
26.91	0\\
26.92	0\\
26.93	0\\
26.94	0\\
26.95	0\\
26.96	0\\
26.97	0\\
26.98	0\\
26.99	0\\
27	0\\
27.01	0\\
27.02	0\\
27.03	0\\
27.04	0\\
27.05	0\\
27.06	0\\
27.07	0\\
27.08	0\\
27.09	0\\
27.1	0\\
27.11	0\\
27.12	0\\
27.13	0\\
27.14	0\\
27.15	0\\
27.16	0\\
27.17	0\\
27.18	0\\
27.19	0\\
27.2	0\\
27.21	0\\
27.22	0\\
27.23	0\\
27.24	0\\
27.25	0\\
27.26	0\\
27.27	0\\
27.28	0\\
27.29	0\\
27.3	0\\
27.31	0\\
27.32	0\\
27.33	0\\
27.34	0\\
27.35	0\\
27.36	0\\
27.37	0\\
27.38	0\\
27.39	0\\
27.4	0\\
27.41	0\\
27.42	0\\
27.43	0\\
27.44	0\\
27.45	0\\
27.46	0\\
27.47	0\\
27.48	0\\
27.49	0\\
27.5	0\\
27.51	0\\
27.52	0\\
27.53	0\\
27.54	0\\
27.55	0\\
27.56	0\\
27.57	0\\
27.58	0\\
27.59	0\\
27.6	0\\
27.61	0\\
27.62	0\\
27.63	0\\
27.64	0\\
27.65	0\\
27.66	0\\
27.67	0\\
27.68	0\\
27.69	0\\
27.7	0\\
27.71	0\\
27.72	0\\
27.73	0\\
27.74	0\\
27.75	0\\
27.76	0\\
27.77	0\\
27.78	0\\
27.79	0\\
27.8	0\\
27.81	0\\
27.82	0\\
27.83	0\\
27.84	0\\
27.85	0\\
27.86	0\\
27.87	0\\
27.88	0\\
27.89	0\\
27.9	0\\
27.91	0\\
27.92	0\\
27.93	0\\
27.94	0\\
27.95	0\\
27.96	0\\
27.97	0\\
27.98	0\\
27.99	0\\
28	0\\
28.01	0\\
28.02	0\\
28.03	0\\
28.04	0\\
28.05	0\\
28.06	0\\
28.07	0\\
28.08	0\\
28.09	0\\
28.1	0\\
28.11	0\\
28.12	0\\
28.13	0\\
28.14	0\\
28.15	0\\
28.16	0\\
28.17	0\\
28.18	0\\
28.19	0\\
28.2	0\\
28.21	0\\
28.22	0\\
28.23	0\\
28.24	0\\
28.25	0\\
28.26	0\\
28.27	0\\
28.28	0\\
28.29	0\\
28.3	0\\
28.31	0\\
28.32	0\\
28.33	0\\
28.34	0\\
28.35	0\\
28.36	0\\
28.37	0\\
28.38	0\\
28.39	0\\
28.4	0\\
28.41	0\\
28.42	0\\
28.43	0\\
28.44	0\\
28.45	0\\
28.46	0\\
28.47	0\\
28.48	0\\
28.49	0\\
28.5	0\\
28.51	0\\
28.52	0\\
28.53	0\\
28.54	0\\
28.55	0\\
28.56	0\\
28.57	0\\
28.58	0\\
28.59	0\\
28.6	0\\
28.61	0\\
28.62	0\\
28.63	0\\
28.64	0\\
28.65	0\\
28.66	0\\
28.67	0\\
28.68	0\\
28.69	0\\
28.7	0\\
28.71	0\\
28.72	0\\
28.73	0\\
28.74	0\\
28.75	0\\
28.76	0\\
28.77	0\\
28.78	0\\
28.79	0\\
28.8	0\\
28.81	0\\
28.82	0\\
28.83	0\\
28.84	0\\
28.85	0\\
28.86	0\\
28.87	0\\
28.88	0\\
28.89	0\\
28.9	0\\
28.91	0\\
28.92	0\\
28.93	0\\
28.94	0\\
28.95	0\\
28.96	0\\
28.97	0\\
28.98	0\\
28.99	0\\
29	0\\
29.01	0\\
29.02	0\\
29.03	0\\
29.04	0\\
29.05	0\\
29.06	0\\
29.07	0\\
29.08	0\\
29.09	0\\
29.1	0\\
29.11	0\\
29.12	0\\
29.13	0\\
29.14	0\\
29.15	0\\
29.16	0\\
29.17	0\\
29.18	0\\
29.19	0\\
29.2	0\\
29.21	0\\
29.22	0\\
29.23	0\\
29.24	0\\
29.25	0\\
29.26	0\\
29.27	0\\
29.28	0\\
29.29	0\\
29.3	0\\
29.31	0\\
29.32	0\\
29.33	0\\
29.34	0\\
29.35	0\\
29.36	0\\
29.37	0\\
29.38	0\\
29.39	0\\
29.4	0\\
29.41	0\\
29.42	0\\
29.43	0\\
29.44	0\\
29.45	0\\
29.46	0\\
29.47	0\\
29.48	0\\
29.49	0\\
29.5	0\\
29.51	0\\
29.52	0\\
29.53	0\\
29.54	0\\
29.55	0\\
29.56	0\\
29.57	0\\
29.58	0\\
29.59	0\\
29.6	0\\
29.61	0\\
29.62	0\\
29.63	0\\
29.64	0\\
29.65	0\\
29.66	0\\
29.67	0\\
29.68	0\\
29.69	0\\
29.7	0\\
29.71	0\\
29.72	0\\
29.73	0\\
29.74	0\\
29.75	0\\
29.76	0\\
29.77	0\\
29.78	0\\
29.79	0\\
29.8	0\\
29.81	0\\
29.82	0\\
29.83	0\\
29.84	0\\
29.85	0\\
29.86	0\\
29.87	0\\
29.88	0\\
29.89	0\\
29.9	0\\
29.91	0\\
29.92	0\\
29.93	0\\
29.94	0\\
29.95	0\\
29.96	0\\
29.97	0\\
29.98	0\\
29.99	0\\
30	0\\
30.01	0\\
30.02	0\\
30.03	0\\
30.04	0\\
30.05	0\\
30.06	0\\
30.07	0\\
30.08	0\\
30.09	0\\
30.1	0\\
30.11	0\\
30.12	0\\
30.13	0\\
30.14	0\\
30.15	0\\
30.16	0\\
30.17	0\\
30.18	0\\
30.19	0\\
30.2	0\\
30.21	0\\
30.22	0\\
30.23	0\\
30.24	0\\
30.25	0\\
30.26	0\\
30.27	0\\
30.28	0\\
30.29	0\\
30.3	0\\
30.31	0\\
30.32	0\\
30.33	0\\
30.34	0\\
30.35	0\\
30.36	0\\
30.37	0\\
30.38	0\\
30.39	0\\
30.4	0\\
30.41	0\\
30.42	0\\
30.43	0\\
30.44	0\\
30.45	0\\
30.46	0\\
30.47	0\\
30.48	0\\
30.49	0\\
30.5	0\\
30.51	0\\
30.52	0\\
30.53	0\\
30.54	0\\
30.55	0\\
30.56	0\\
30.57	0\\
30.58	0\\
30.59	0\\
30.6	0\\
30.61	0\\
30.62	0\\
30.63	0\\
30.64	0\\
30.65	0\\
30.66	0\\
30.67	0\\
30.68	0\\
30.69	0\\
30.7	0\\
30.71	0\\
30.72	0\\
30.73	0\\
30.74	0\\
30.75	0\\
30.76	0\\
30.77	0\\
30.78	0\\
30.79	0\\
30.8	0\\
30.81	0\\
30.82	0\\
30.83	0\\
30.84	0\\
30.85	0\\
30.86	0\\
30.87	0\\
30.88	0\\
30.89	0\\
30.9	0\\
30.91	0\\
30.92	0\\
30.93	0\\
30.94	0\\
30.95	0\\
30.96	0\\
30.97	0\\
30.98	0\\
30.99	0\\
31	0\\
31.01	0\\
31.02	0\\
31.03	0\\
31.04	0\\
31.05	0\\
31.06	0\\
31.07	0\\
31.08	0\\
31.09	0\\
31.1	0\\
31.11	0\\
31.12	0\\
31.13	0\\
31.14	0\\
31.15	0\\
31.16	0\\
31.17	0\\
31.18	0\\
31.19	0\\
31.2	0\\
31.21	0\\
31.22	0\\
31.23	0\\
31.24	0\\
31.25	0\\
31.26	0\\
31.27	0\\
31.28	0\\
31.29	0\\
31.3	0\\
31.31	0\\
31.32	0\\
31.33	0\\
31.34	0\\
31.35	0\\
31.36	0\\
31.37	0\\
31.38	0\\
31.39	0\\
31.4	0\\
31.41	0\\
31.42	0\\
31.43	0\\
31.44	0\\
31.45	0\\
31.46	0\\
31.47	0\\
31.48	0\\
31.49	0\\
31.5	0\\
31.51	0\\
31.52	0\\
31.53	0\\
31.54	0\\
31.55	0\\
31.56	0\\
31.57	0\\
31.58	0\\
31.59	0\\
31.6	0\\
31.61	0\\
31.62	0\\
31.63	0\\
31.64	0\\
31.65	0\\
31.66	0\\
31.67	0\\
31.68	0\\
31.69	0\\
31.7	0\\
31.71	0\\
31.72	0\\
31.73	0\\
31.74	0\\
31.75	0\\
31.76	0\\
31.77	0\\
31.78	0\\
31.79	0\\
31.8	0\\
31.81	0\\
31.82	0\\
31.83	0\\
31.84	0\\
31.85	0\\
31.86	0\\
31.87	0\\
31.88	0\\
31.89	0\\
31.9	0\\
31.91	0\\
31.92	0\\
31.93	0\\
31.94	0\\
31.95	0\\
31.96	0\\
31.97	0\\
31.98	0\\
31.99	0\\
32	0\\
32.01	0\\
32.02	0\\
32.03	0\\
32.04	0\\
32.05	0\\
32.06	0\\
32.07	0\\
32.08	0\\
32.09	0\\
32.1	0\\
32.11	0\\
32.12	0\\
32.13	0\\
32.14	0\\
32.15	0\\
32.16	0\\
32.17	0\\
32.18	0\\
32.19	0\\
32.2	0\\
32.21	0\\
32.22	0\\
32.23	0\\
32.24	0\\
32.25	0\\
32.26	0\\
32.27	0\\
32.28	0\\
32.29	0\\
32.3	0\\
32.31	0\\
32.32	0\\
32.33	0\\
32.34	0\\
32.35	0\\
32.36	0\\
32.37	0\\
32.38	0\\
32.39	0\\
32.4	0\\
32.41	0\\
32.42	0\\
32.43	0\\
32.44	0\\
32.45	0\\
32.46	0\\
32.47	0\\
32.48	0\\
32.49	0\\
32.5	0\\
32.51	0\\
32.52	0\\
32.53	0\\
32.54	0\\
32.55	0\\
32.56	0\\
32.57	0\\
32.58	0\\
32.59	0\\
32.6	0\\
32.61	0\\
32.62	0\\
32.63	0\\
32.64	0\\
32.65	0\\
32.66	0\\
32.67	0\\
32.68	0\\
32.69	0\\
32.7	0\\
32.71	0\\
32.72	0\\
32.73	0\\
32.74	0\\
32.75	0\\
32.76	0\\
32.77	0\\
32.78	0\\
32.79	0\\
32.8	0\\
32.81	0\\
32.82	0\\
32.83	0\\
32.84	0\\
32.85	0\\
32.86	0\\
32.87	0\\
32.88	0\\
32.89	0\\
32.9	0\\
32.91	0\\
32.92	0\\
32.93	0\\
32.94	0\\
32.95	0\\
32.96	0\\
32.97	0\\
32.98	0\\
32.99	0\\
33	0\\
33.01	0\\
33.02	0\\
33.03	0\\
33.04	0\\
33.05	0\\
33.06	0\\
33.07	0\\
33.08	0\\
33.09	0\\
33.1	0\\
33.11	0\\
33.12	0\\
33.13	0\\
33.14	0\\
33.15	0\\
33.16	0\\
33.17	0\\
33.18	0\\
33.19	0\\
33.2	0\\
33.21	0\\
33.22	0\\
33.23	0\\
33.24	0\\
33.25	0\\
33.26	0\\
33.27	0\\
33.28	0\\
33.29	0\\
33.3	0\\
33.31	0\\
33.32	0\\
33.33	0\\
33.34	0\\
33.35	0\\
33.36	0\\
33.37	0\\
33.38	0\\
33.39	0\\
33.4	0\\
33.41	0\\
33.42	0\\
33.43	0\\
33.44	0\\
33.45	0\\
33.46	0\\
33.47	0\\
33.48	0\\
33.49	0\\
33.5	0\\
33.51	0\\
33.52	0\\
33.53	0\\
33.54	0\\
33.55	0\\
33.56	0\\
33.57	0\\
33.58	0\\
33.59	0\\
33.6	0\\
33.61	0\\
33.62	0\\
33.63	0\\
33.64	0\\
33.65	0\\
33.66	0\\
33.67	0\\
33.68	0\\
33.69	0\\
33.7	0\\
33.71	0\\
33.72	0\\
33.73	0\\
33.74	0\\
33.75	0\\
33.76	0\\
33.77	0\\
33.78	0\\
33.79	0\\
33.8	0\\
33.81	0\\
33.82	0\\
33.83	0\\
33.84	0\\
33.85	0\\
33.86	0\\
33.87	0\\
33.88	0\\
33.89	0\\
33.9	0\\
33.91	0\\
33.92	0\\
33.93	0\\
33.94	0\\
33.95	0\\
33.96	0\\
33.97	0\\
33.98	0\\
33.99	0\\
34	0\\
34.01	0\\
34.02	0\\
34.03	0\\
34.04	0\\
34.05	0\\
34.06	0\\
34.07	0\\
34.08	0\\
34.09	0\\
34.1	0\\
34.11	0\\
34.12	0\\
34.13	0\\
34.14	0\\
34.15	0\\
34.16	0\\
34.17	0\\
34.18	0\\
34.19	0\\
34.2	0\\
34.21	0\\
34.22	0\\
34.23	0\\
34.24	0\\
34.25	0\\
34.26	0\\
34.27	0\\
34.28	0\\
34.29	0\\
34.3	0\\
34.31	0\\
34.32	0\\
34.33	0\\
34.34	0\\
34.35	0\\
34.36	0\\
34.37	0\\
34.38	0\\
34.39	0\\
34.4	0\\
34.41	0\\
34.42	0\\
34.43	0\\
34.44	0\\
34.45	0\\
34.46	0\\
34.47	0\\
34.48	0\\
34.49	0\\
34.5	0\\
34.51	0\\
34.52	0\\
34.53	0\\
34.54	0\\
34.55	0\\
34.56	0\\
34.57	0\\
34.58	0\\
34.59	0\\
34.6	0\\
34.61	0\\
34.62	0\\
34.63	0\\
34.64	0\\
34.65	0\\
34.66	0\\
34.67	0\\
34.68	0\\
34.69	0\\
34.7	0\\
34.71	0\\
34.72	0\\
34.73	0\\
34.74	0\\
34.75	0\\
34.76	0\\
34.77	0\\
34.78	0\\
34.79	0\\
34.8	0\\
34.81	0\\
34.82	0\\
34.83	0\\
34.84	0\\
34.85	0\\
34.86	0\\
34.87	0\\
34.88	0\\
34.89	0\\
34.9	0\\
34.91	0\\
34.92	0\\
34.93	0\\
34.94	0\\
34.95	0\\
34.96	0\\
34.97	0\\
34.98	0\\
34.99	0\\
35	0\\
35.01	0\\
35.02	0\\
35.03	0\\
35.04	0\\
35.05	0\\
35.06	0\\
35.07	0\\
35.08	0\\
35.09	0\\
35.1	0\\
35.11	0\\
35.12	0\\
35.13	0\\
35.14	0\\
35.15	0\\
35.16	0\\
35.17	0\\
35.18	0\\
35.19	0\\
35.2	0\\
35.21	0\\
35.22	0\\
35.23	0\\
35.24	0\\
35.25	0\\
35.26	0\\
35.27	0\\
35.28	0\\
35.29	0\\
35.3	0\\
35.31	0\\
35.32	0\\
35.33	0\\
35.34	0\\
35.35	0\\
35.36	0\\
35.37	0\\
35.38	0\\
35.39	0\\
35.4	0\\
35.41	0\\
35.42	0\\
35.43	0\\
35.44	0\\
35.45	0\\
35.46	0\\
35.47	0\\
35.48	0\\
35.49	0\\
35.5	0\\
35.51	0\\
35.52	0\\
35.53	0\\
35.54	0\\
35.55	0\\
35.56	0\\
35.57	0\\
35.58	0\\
35.59	0\\
35.6	0\\
35.61	0\\
35.62	0\\
35.63	0\\
35.64	0\\
35.65	0\\
35.66	0\\
35.67	0\\
35.68	0\\
35.69	0\\
35.7	0\\
35.71	0\\
35.72	0\\
35.73	0\\
35.74	0\\
35.75	0\\
35.76	0\\
35.77	0\\
35.78	0\\
35.79	0\\
35.8	0\\
35.81	0\\
35.82	0\\
35.83	0\\
35.84	0\\
35.85	0\\
35.86	0\\
35.87	0\\
35.88	0\\
35.89	0\\
35.9	0\\
35.91	0\\
35.92	0\\
35.93	0\\
35.94	0\\
35.95	0\\
35.96	0\\
35.97	0\\
35.98	0\\
35.99	0\\
36	0\\
36.01	0\\
36.02	0\\
36.03	0\\
36.04	0\\
36.05	0\\
36.06	0\\
36.07	0\\
36.08	0\\
36.09	0\\
36.1	0\\
36.11	0\\
36.12	0\\
36.13	0\\
36.14	0\\
36.15	0\\
36.16	0\\
36.17	0\\
36.18	0\\
36.19	0\\
36.2	0\\
36.21	0\\
36.22	0\\
36.23	0\\
36.24	0\\
36.25	0\\
36.26	0\\
36.27	0\\
36.28	0\\
36.29	0\\
36.3	0\\
36.31	0\\
36.32	0\\
36.33	0\\
36.34	0\\
36.35	0\\
36.36	0\\
36.37	0\\
36.38	0\\
36.39	0\\
36.4	0\\
36.41	0\\
36.42	0\\
36.43	0\\
36.44	0\\
36.45	0\\
36.46	0\\
36.47	0\\
36.48	0\\
36.49	0\\
36.5	0\\
36.51	0\\
36.52	0\\
36.53	0\\
36.54	0\\
36.55	0\\
36.56	0\\
36.57	0\\
36.58	0\\
36.59	0\\
36.6	0\\
36.61	0\\
36.62	0\\
36.63	0\\
36.64	0\\
36.65	0\\
36.66	0\\
36.67	0\\
36.68	0\\
36.69	0\\
36.7	0\\
36.71	0\\
36.72	0\\
36.73	0\\
36.74	0\\
36.75	0\\
36.76	0\\
36.77	0\\
36.78	0\\
36.79	0\\
36.8	0\\
36.81	0\\
36.82	0\\
36.83	0\\
36.84	0\\
36.85	0\\
36.86	0\\
36.87	0\\
36.88	0\\
36.89	0\\
36.9	0\\
36.91	0\\
36.92	0\\
36.93	0\\
36.94	0\\
36.95	0\\
36.96	0\\
36.97	0\\
36.98	0\\
36.99	0\\
37	0\\
37.01	0\\
37.02	0\\
37.03	0\\
37.04	0\\
37.05	0\\
37.06	0\\
37.07	0\\
37.08	0\\
37.09	0\\
37.1	0\\
37.11	0\\
37.12	0\\
37.13	0\\
37.14	0\\
37.15	0\\
37.16	0\\
37.17	0\\
37.18	0\\
37.19	0\\
37.2	0\\
37.21	0\\
37.22	0\\
37.23	0\\
37.24	0\\
37.25	0\\
37.26	0\\
37.27	0\\
37.28	0\\
37.29	0\\
37.3	0\\
37.31	0\\
37.32	0\\
37.33	0\\
37.34	0\\
37.35	0\\
37.36	0\\
37.37	0\\
37.38	0\\
37.39	0\\
37.4	0\\
37.41	0\\
37.42	0\\
37.43	0\\
37.44	0\\
37.45	0\\
37.46	0\\
37.47	0\\
37.48	0\\
37.49	0\\
37.5	0\\
37.51	0\\
37.52	0\\
37.53	0\\
37.54	0\\
37.55	0\\
37.56	0\\
37.57	0\\
37.58	0\\
37.59	0\\
37.6	0\\
37.61	0\\
37.62	0\\
37.63	0\\
37.64	0\\
37.65	0\\
37.66	0\\
37.67	0\\
37.68	0\\
37.69	0\\
37.7	0\\
37.71	0\\
37.72	0\\
37.73	0\\
37.74	0\\
37.75	0\\
37.76	0\\
37.77	0\\
37.78	0\\
37.79	0\\
37.8	0\\
37.81	0\\
37.82	0\\
37.83	0\\
37.84	0\\
37.85	0\\
37.86	0\\
37.87	0\\
37.88	0\\
37.89	0\\
37.9	0\\
37.91	0\\
37.92	0\\
37.93	0\\
37.94	0\\
37.95	0\\
37.96	0\\
37.97	0\\
37.98	0\\
37.99	0\\
38	0\\
38.01	0\\
38.02	0\\
38.03	0\\
38.04	0\\
38.05	0\\
38.06	0\\
38.07	0\\
38.08	0\\
38.09	0\\
38.1	0\\
38.11	0\\
38.12	0\\
38.13	0\\
38.14	0\\
38.15	0\\
38.16	0\\
38.17	0\\
38.18	0\\
38.19	0\\
38.2	0\\
38.21	0\\
38.22	0\\
38.23	0\\
38.24	0\\
38.25	0\\
38.26	0\\
38.27	0\\
38.28	0\\
38.29	0\\
38.3	0\\
38.31	0\\
38.32	0\\
38.33	0\\
38.34	0\\
38.35	0\\
38.36	0\\
38.37	0\\
38.38	0\\
38.39	0\\
38.4	0\\
38.41	0\\
38.42	0\\
38.43	0\\
38.44	0\\
38.45	0\\
38.46	0\\
38.47	0\\
38.48	0\\
38.49	0\\
38.5	0\\
38.51	0\\
38.52	0\\
38.53	0\\
38.54	0\\
38.55	0\\
38.56	0\\
38.57	0\\
38.58	0\\
38.59	0\\
38.6	0\\
38.61	0\\
38.62	0\\
38.63	0\\
38.64	0\\
38.65	0\\
38.66	0\\
38.67	0\\
38.68	0\\
38.69	0\\
38.7	0\\
38.71	0\\
38.72	0\\
38.73	0\\
38.74	0\\
38.75	0\\
38.76	0\\
38.77	0\\
38.78	0\\
38.79	0\\
38.8	0\\
38.81	0\\
38.82	0\\
38.83	0\\
38.84	0\\
38.85	0\\
38.86	0\\
38.87	0\\
38.88	0\\
38.89	0\\
38.9	0\\
38.91	0\\
38.92	0\\
38.93	0\\
38.94	0\\
38.95	0\\
38.96	0\\
38.97	0\\
38.98	0\\
38.99	0\\
39	0\\
39.01	0\\
39.02	0\\
39.03	0\\
39.04	0\\
39.05	0\\
39.06	0\\
39.07	0\\
39.08	0\\
39.09	0\\
39.1	0\\
39.11	0\\
39.12	0\\
39.13	0\\
39.14	0\\
39.15	0\\
39.16	0\\
39.17	0\\
39.18	0\\
39.19	0\\
39.2	0\\
39.21	0\\
39.22	0\\
39.23	0\\
39.24	0\\
39.25	0\\
39.26	0\\
39.27	0\\
39.28	0\\
39.29	0\\
39.3	0\\
39.31	0\\
39.32	0\\
39.33	0\\
39.34	0\\
39.35	0\\
39.36	0\\
39.37	0\\
39.38	0\\
39.39	0\\
39.4	0\\
39.41	0\\
39.42	0\\
39.43	0\\
39.44	0\\
39.45	0\\
39.46	0\\
39.47	0\\
39.48	0\\
39.49	0\\
39.5	0\\
39.51	0\\
39.52	0\\
39.53	0\\
39.54	0\\
39.55	0\\
39.56	0\\
39.57	0\\
39.58	0\\
39.59	0\\
39.6	0\\
39.61	0\\
39.62	0\\
39.63	0\\
39.64	0\\
39.65	0\\
39.66	0\\
39.67	0\\
39.68	0\\
39.69	0\\
39.7	0\\
39.71	0\\
39.72	0\\
39.73	0\\
39.74	0\\
39.75	0\\
39.76	0\\
39.77	0\\
39.78	0\\
39.79	0\\
39.8	0\\
39.81	0\\
39.82	0\\
39.83	0\\
39.84	0\\
39.85	0\\
39.86	0\\
39.87	0\\
39.88	0\\
39.89	0\\
39.9	0\\
39.91	0\\
39.92	0\\
39.93	0\\
39.94	0\\
39.95	0\\
39.96	0\\
39.97	0\\
39.98	0\\
39.99	0\\
40	0\\
40.01	0\\
};
\addplot [color=green,solid,forget plot]
  table[row sep=crcr]{%
40.01	0\\
40.02	0\\
40.03	0\\
40.04	0\\
40.05	0\\
40.06	0\\
40.07	0\\
40.08	0\\
40.09	0\\
40.1	0\\
40.11	0\\
40.12	0\\
40.13	0\\
40.14	0\\
40.15	0\\
40.16	0\\
40.17	0\\
40.18	0\\
40.19	0\\
40.2	0\\
40.21	0\\
40.22	0\\
40.23	0\\
40.24	0\\
40.25	0\\
40.26	0\\
40.27	0\\
40.28	0\\
40.29	0\\
40.3	0\\
40.31	0\\
40.32	0\\
40.33	0\\
40.34	0\\
40.35	0\\
40.36	0\\
40.37	0\\
40.38	0\\
40.39	0\\
40.4	0\\
40.41	0\\
40.42	0\\
40.43	0\\
40.44	0\\
40.45	0\\
40.46	0\\
40.47	0\\
40.48	0\\
40.49	0\\
40.5	0\\
40.51	0\\
40.52	0\\
40.53	0\\
40.54	0\\
40.55	0\\
40.56	0\\
40.57	0\\
40.58	0\\
40.59	0\\
40.6	0\\
40.61	0\\
40.62	0\\
40.63	0\\
40.64	0\\
40.65	0\\
40.66	0\\
40.67	0\\
40.68	0\\
40.69	0\\
40.7	0\\
40.71	0\\
40.72	0\\
40.73	0\\
40.74	0\\
40.75	0\\
40.76	0\\
40.77	0\\
40.78	0\\
40.79	0\\
40.8	0\\
40.81	0\\
40.82	0\\
40.83	0\\
40.84	0\\
40.85	0\\
40.86	0\\
40.87	0\\
40.88	0\\
40.89	0\\
40.9	0\\
40.91	0\\
40.92	0\\
40.93	0\\
40.94	0\\
40.95	0\\
40.96	0\\
40.97	0\\
40.98	0\\
40.99	0\\
41	0\\
41.01	0\\
41.02	0\\
41.03	0\\
41.04	0\\
41.05	0\\
41.06	0\\
41.07	0\\
41.08	0\\
41.09	0\\
41.1	0\\
41.11	0\\
41.12	0\\
41.13	0\\
41.14	0\\
41.15	0\\
41.16	0\\
41.17	0\\
41.18	0\\
41.19	0\\
41.2	0\\
41.21	0\\
41.22	0\\
41.23	0\\
41.24	0\\
41.25	0\\
41.26	0\\
41.27	0\\
41.28	0\\
41.29	0\\
41.3	0\\
41.31	0\\
41.32	0\\
41.33	0\\
41.34	0\\
41.35	0\\
41.36	0\\
41.37	0\\
41.38	0\\
41.39	0\\
41.4	0\\
41.41	0\\
41.42	0\\
41.43	0\\
41.44	0\\
41.45	0\\
41.46	0\\
41.47	0\\
41.48	0\\
41.49	0\\
41.5	0\\
41.51	0\\
41.52	0\\
41.53	0\\
41.54	0\\
41.55	0\\
41.56	0\\
41.57	0\\
41.58	0\\
41.59	0\\
41.6	0\\
41.61	0\\
41.62	0\\
41.63	0\\
41.64	0\\
41.65	0\\
41.66	0\\
41.67	0\\
41.68	0\\
41.69	0\\
41.7	0\\
41.71	0\\
41.72	0\\
41.73	0\\
41.74	0\\
41.75	0\\
41.76	0\\
41.77	0\\
41.78	0\\
41.79	0\\
41.8	0\\
41.81	0\\
41.82	0\\
41.83	0\\
41.84	0\\
41.85	0\\
41.86	0\\
41.87	0\\
41.88	0\\
41.89	0\\
41.9	0\\
41.91	0\\
41.92	0\\
41.93	0\\
41.94	0\\
41.95	0\\
41.96	0\\
41.97	0\\
41.98	0\\
41.99	0\\
42	0\\
42.01	0\\
42.02	0\\
42.03	0\\
42.04	0\\
42.05	0\\
42.06	0\\
42.07	0\\
42.08	0\\
42.09	0\\
42.1	0\\
42.11	0\\
42.12	0\\
42.13	0\\
42.14	0\\
42.15	0\\
42.16	0\\
42.17	0\\
42.18	0\\
42.19	0\\
42.2	0\\
42.21	0\\
42.22	0\\
42.23	0\\
42.24	0\\
42.25	0\\
42.26	0\\
42.27	0\\
42.28	0\\
42.29	0\\
42.3	0\\
42.31	0\\
42.32	0\\
42.33	0\\
42.34	0\\
42.35	0\\
42.36	0\\
42.37	0\\
42.38	0\\
42.39	0\\
42.4	0\\
42.41	0\\
42.42	0\\
42.43	0\\
42.44	0\\
42.45	0\\
42.46	0\\
42.47	0\\
42.48	0\\
42.49	0\\
42.5	0\\
42.51	0\\
42.52	0\\
42.53	0\\
42.54	0\\
42.55	0\\
42.56	0\\
42.57	0\\
42.58	0\\
42.59	0\\
42.6	0\\
42.61	0\\
42.62	0\\
42.63	0\\
42.64	0\\
42.65	0\\
42.66	0\\
42.67	0\\
42.68	0\\
42.69	0\\
42.7	0\\
42.71	0\\
42.72	0\\
42.73	0\\
42.74	0\\
42.75	0\\
42.76	0\\
42.77	0\\
42.78	0\\
42.79	0\\
42.8	0\\
42.81	0\\
42.82	0\\
42.83	0\\
42.84	0\\
42.85	0\\
42.86	0\\
42.87	0\\
42.88	0\\
42.89	0\\
42.9	0\\
42.91	0\\
42.92	0\\
42.93	0\\
42.94	0\\
42.95	0\\
42.96	0\\
42.97	0\\
42.98	0\\
42.99	0\\
43	0\\
43.01	0\\
43.02	0\\
43.03	0\\
43.04	0\\
43.05	0\\
43.06	0\\
43.07	0\\
43.08	0\\
43.09	0\\
43.1	0\\
43.11	0\\
43.12	0\\
43.13	0\\
43.14	0\\
43.15	0\\
43.16	0\\
43.17	0\\
43.18	0\\
43.19	0\\
43.2	0\\
43.21	0\\
43.22	0\\
43.23	0\\
43.24	0\\
43.25	0\\
43.26	0\\
43.27	0\\
43.28	0\\
43.29	0\\
43.3	0\\
43.31	0\\
43.32	0\\
43.33	0\\
43.34	0\\
43.35	0\\
43.36	0\\
43.37	0\\
43.38	0\\
43.39	0\\
43.4	0\\
43.41	0\\
43.42	0\\
43.43	0\\
43.44	0\\
43.45	0\\
43.46	0\\
43.47	0\\
43.48	0\\
43.49	0\\
43.5	0\\
43.51	0\\
43.52	0\\
43.53	0\\
43.54	0\\
43.55	0\\
43.56	0\\
43.57	0\\
43.58	0\\
43.59	0\\
43.6	0\\
43.61	0\\
43.62	0\\
43.63	0\\
43.64	0\\
43.65	0\\
43.66	0\\
43.67	0\\
43.68	0\\
43.69	0\\
43.7	0\\
43.71	0\\
43.72	0\\
43.73	0\\
43.74	0\\
43.75	0\\
43.76	0\\
43.77	0\\
43.78	0\\
43.79	0\\
43.8	0\\
43.81	0\\
43.82	0\\
43.83	0\\
43.84	0\\
43.85	0\\
43.86	0\\
43.87	0\\
43.88	0\\
43.89	0\\
43.9	0\\
43.91	0\\
43.92	0\\
43.93	0\\
43.94	0\\
43.95	0\\
43.96	0\\
43.97	0\\
43.98	0\\
43.99	0\\
44	0\\
44.01	0\\
44.02	0\\
44.03	0\\
44.04	0\\
44.05	0\\
44.06	0\\
44.07	0\\
44.08	0\\
44.09	0\\
44.1	0\\
44.11	0\\
44.12	0\\
44.13	0\\
44.14	0\\
44.15	0\\
44.16	0\\
44.17	0\\
44.18	0\\
44.19	0\\
44.2	0\\
44.21	0\\
44.22	0\\
44.23	0\\
44.24	0\\
44.25	0\\
44.26	0\\
44.27	0\\
44.28	0\\
44.29	0\\
44.3	0\\
44.31	0\\
44.32	0\\
44.33	0\\
44.34	0\\
44.35	0\\
44.36	0\\
44.37	0\\
44.38	0\\
44.39	0\\
44.4	0\\
44.41	0\\
44.42	0\\
44.43	0\\
44.44	0\\
44.45	0\\
44.46	0\\
44.47	0\\
44.48	0\\
44.49	0\\
44.5	0\\
44.51	0\\
44.52	0\\
44.53	0\\
44.54	0\\
44.55	0\\
44.56	0\\
44.57	0\\
44.58	0\\
44.59	0\\
44.6	0\\
44.61	0\\
44.62	0\\
44.63	0\\
44.64	0\\
44.65	0\\
44.66	0\\
44.67	0\\
44.68	0\\
44.69	0\\
44.7	0\\
44.71	0\\
44.72	0\\
44.73	0\\
44.74	0\\
44.75	0\\
44.76	0\\
44.77	0\\
44.78	0\\
44.79	0\\
44.8	0\\
44.81	0\\
44.82	0\\
44.83	0\\
44.84	0\\
44.85	0\\
44.86	0\\
44.87	0\\
44.88	0\\
44.89	0\\
44.9	0\\
44.91	0\\
44.92	0\\
44.93	0\\
44.94	0\\
44.95	0\\
44.96	0\\
44.97	0\\
44.98	0\\
44.99	0\\
45	0\\
45.01	0\\
45.02	0\\
45.03	0\\
45.04	0\\
45.05	0\\
45.06	0\\
45.07	0\\
45.08	0\\
45.09	0\\
45.1	0\\
45.11	0\\
45.12	0\\
45.13	0\\
45.14	0\\
45.15	0\\
45.16	0\\
45.17	0\\
45.18	0\\
45.19	0\\
45.2	0\\
45.21	0\\
45.22	0\\
45.23	0\\
45.24	0\\
45.25	0\\
45.26	0\\
45.27	0\\
45.28	0\\
45.29	0\\
45.3	0\\
45.31	0\\
45.32	0\\
45.33	0\\
45.34	0\\
45.35	0\\
45.36	0\\
45.37	0\\
45.38	0\\
45.39	0\\
45.4	0\\
45.41	0\\
45.42	0\\
45.43	0\\
45.44	0\\
45.45	0\\
45.46	0\\
45.47	0\\
45.48	0\\
45.49	0\\
45.5	0\\
45.51	0\\
45.52	0\\
45.53	0\\
45.54	0\\
45.55	0\\
45.56	0\\
45.57	0\\
45.58	0\\
45.59	0\\
45.6	0\\
45.61	0\\
45.62	0\\
45.63	0\\
45.64	0\\
45.65	0\\
45.66	0\\
45.67	0\\
45.68	0\\
45.69	0\\
45.7	0\\
45.71	0\\
45.72	0\\
45.73	0\\
45.74	0\\
45.75	0\\
45.76	0\\
45.77	0\\
45.78	0\\
45.79	0\\
45.8	0\\
45.81	0\\
45.82	0\\
45.83	0\\
45.84	0\\
45.85	0\\
45.86	0\\
45.87	0\\
45.88	0\\
45.89	0\\
45.9	0\\
45.91	0\\
45.92	0\\
45.93	0\\
45.94	0\\
45.95	0\\
45.96	0\\
45.97	0\\
45.98	0\\
45.99	0\\
46	0\\
46.01	0\\
46.02	0\\
46.03	0\\
46.04	0\\
46.05	0\\
46.06	0\\
46.07	0\\
46.08	0\\
46.09	0\\
46.1	0\\
46.11	0\\
46.12	0\\
46.13	0\\
46.14	0\\
46.15	0\\
46.16	0\\
46.17	0\\
46.18	0\\
46.19	0\\
46.2	0\\
46.21	0\\
46.22	0\\
46.23	0\\
46.24	0\\
46.25	0\\
46.26	0\\
46.27	0\\
46.28	0\\
46.29	0\\
46.3	0\\
46.31	0\\
46.32	0\\
46.33	0\\
46.34	0\\
46.35	0\\
46.36	0\\
46.37	0\\
46.38	0\\
46.39	0\\
46.4	0\\
46.41	0\\
46.42	0\\
46.43	0\\
46.44	0\\
46.45	0\\
46.46	0\\
46.47	0\\
46.48	0\\
46.49	0\\
46.5	0\\
46.51	0\\
46.52	0\\
46.53	0\\
46.54	0\\
46.55	0\\
46.56	0\\
46.57	0\\
46.58	0\\
46.59	0\\
46.6	0\\
46.61	0\\
46.62	0\\
46.63	0\\
46.64	0\\
46.65	0\\
46.66	0\\
46.67	0\\
46.68	0\\
46.69	0\\
46.7	0\\
46.71	0\\
46.72	0\\
46.73	0\\
46.74	0\\
46.75	0\\
46.76	0\\
46.77	0\\
46.78	0\\
46.79	0\\
46.8	0\\
46.81	0\\
46.82	0\\
46.83	0\\
46.84	0\\
46.85	0\\
46.86	0\\
46.87	0\\
46.88	0\\
46.89	0\\
46.9	0\\
46.91	0\\
46.92	0\\
46.93	0\\
46.94	0\\
46.95	0\\
46.96	0\\
46.97	0\\
46.98	0\\
46.99	0\\
47	0\\
47.01	0\\
47.02	0\\
47.03	0\\
47.04	0\\
47.05	0\\
47.06	0\\
47.07	0\\
47.08	0\\
47.09	0\\
47.1	0\\
47.11	0\\
47.12	0\\
47.13	0\\
47.14	0\\
47.15	0\\
47.16	0\\
47.17	0\\
47.18	0\\
47.19	0\\
47.2	0\\
47.21	0\\
47.22	0\\
47.23	0\\
47.24	0\\
47.25	0\\
47.26	0\\
47.27	0\\
47.28	0\\
47.29	0\\
47.3	0\\
47.31	0\\
47.32	0\\
47.33	0\\
47.34	0\\
47.35	0\\
47.36	0\\
47.37	0\\
47.38	0\\
47.39	0\\
47.4	0\\
47.41	0\\
47.42	0\\
47.43	0\\
47.44	0\\
47.45	0\\
47.46	0\\
47.47	0\\
47.48	0\\
47.49	0\\
47.5	0\\
47.51	0\\
47.52	0\\
47.53	0\\
47.54	0\\
47.55	0\\
47.56	0\\
47.57	0\\
47.58	0\\
47.59	0\\
47.6	0\\
47.61	0\\
47.62	0\\
47.63	0\\
47.64	0\\
47.65	0\\
47.66	0\\
47.67	0\\
47.68	0\\
47.69	0\\
47.7	0\\
47.71	0\\
47.72	0\\
47.73	0\\
47.74	0\\
47.75	0\\
47.76	0\\
47.77	0\\
47.78	0\\
47.79	0\\
47.8	0\\
47.81	0\\
47.82	0\\
47.83	0\\
47.84	0\\
47.85	0\\
47.86	0\\
47.87	0\\
47.88	0\\
47.89	0\\
47.9	0\\
47.91	0\\
47.92	0\\
47.93	0\\
47.94	0\\
47.95	0\\
47.96	0\\
47.97	0\\
47.98	0\\
47.99	0\\
48	0\\
48.01	0\\
48.02	0\\
48.03	0\\
48.04	0\\
48.05	0\\
48.06	0\\
48.07	0\\
48.08	0\\
48.09	0\\
48.1	0\\
48.11	0\\
48.12	0\\
48.13	0\\
48.14	0\\
48.15	0\\
48.16	0\\
48.17	0\\
48.18	0\\
48.19	0\\
48.2	0\\
48.21	0\\
48.22	0\\
48.23	0\\
48.24	0\\
48.25	0\\
48.26	0\\
48.27	0\\
48.28	0\\
48.29	0\\
48.3	0\\
48.31	0\\
48.32	0\\
48.33	0\\
48.34	0\\
48.35	0\\
48.36	0\\
48.37	0\\
48.38	0\\
48.39	0\\
48.4	0\\
48.41	0\\
48.42	0\\
48.43	0\\
48.44	0\\
48.45	0\\
48.46	0\\
48.47	0\\
48.48	0\\
48.49	0\\
48.5	0\\
48.51	0\\
48.52	0\\
48.53	0\\
48.54	0\\
48.55	0\\
48.56	0\\
48.57	0\\
48.58	0\\
48.59	0\\
48.6	0\\
48.61	0\\
48.62	0\\
48.63	0\\
48.64	0\\
48.65	0\\
48.66	0\\
48.67	0\\
48.68	0\\
48.69	0\\
48.7	0\\
48.71	0\\
48.72	0\\
48.73	0\\
48.74	0\\
48.75	0\\
48.76	0\\
48.77	0\\
48.78	0\\
48.79	0\\
48.8	0\\
48.81	0\\
48.82	0\\
48.83	0\\
48.84	0\\
48.85	0\\
48.86	0\\
48.87	0\\
48.88	0\\
48.89	0\\
48.9	0\\
48.91	0\\
48.92	0\\
48.93	0\\
48.94	0\\
48.95	0\\
48.96	0\\
48.97	0\\
48.98	0\\
48.99	0\\
49	0\\
49.01	0\\
49.02	0\\
49.03	0\\
49.04	0\\
49.05	0\\
49.06	0\\
49.07	0\\
49.08	0\\
49.09	0\\
49.1	0\\
49.11	0\\
49.12	0\\
49.13	0\\
49.14	0\\
49.15	0\\
49.16	0\\
49.17	0\\
49.18	0\\
49.19	0\\
49.2	0\\
49.21	0\\
49.22	0\\
49.23	0\\
49.24	0\\
49.25	0\\
49.26	0\\
49.27	0\\
49.28	0\\
49.29	0\\
49.3	0\\
49.31	0\\
49.32	0\\
49.33	0\\
49.34	0\\
49.35	0\\
49.36	0\\
49.37	0\\
49.38	0\\
49.39	0\\
49.4	0\\
49.41	0\\
49.42	0\\
49.43	0\\
49.44	0\\
49.45	0\\
49.46	0\\
49.47	0\\
49.48	0\\
49.49	0\\
49.5	0\\
49.51	0\\
49.52	0\\
49.53	0\\
49.54	0\\
49.55	0\\
49.56	0\\
49.57	0\\
49.58	0\\
49.59	0\\
49.6	0\\
49.61	0\\
49.62	0\\
49.63	0\\
49.64	0\\
49.65	0\\
49.66	0\\
49.67	0\\
49.68	0\\
49.69	0\\
49.7	0\\
49.71	0\\
49.72	0\\
49.73	0\\
49.74	0\\
49.75	0\\
49.76	0\\
49.77	0\\
49.78	0\\
49.79	0\\
49.8	0\\
49.81	0\\
49.82	0\\
49.83	0\\
49.84	0\\
49.85	0\\
49.86	0\\
49.87	0\\
49.88	0\\
49.89	0\\
49.9	0\\
49.91	0\\
49.92	0\\
49.93	0\\
49.94	0\\
49.95	0\\
49.96	0\\
49.97	0\\
49.98	0\\
49.99	0\\
50	0\\
50.01	0\\
50.02	0\\
50.03	0\\
50.04	0\\
50.05	0\\
50.06	0\\
50.07	0\\
50.08	0\\
50.09	0\\
50.1	0\\
50.11	0\\
50.12	0\\
50.13	0\\
50.14	0\\
50.15	0\\
50.16	0\\
50.17	0\\
50.18	0\\
50.19	0\\
50.2	0\\
50.21	0\\
50.22	0\\
50.23	0\\
50.24	0\\
50.25	0\\
50.26	0\\
50.27	0\\
50.28	0\\
50.29	0\\
50.3	0\\
50.31	0\\
50.32	0\\
50.33	0\\
50.34	0\\
50.35	0\\
50.36	0\\
50.37	0\\
50.38	0\\
50.39	0\\
50.4	0\\
50.41	0\\
50.42	0\\
50.43	0\\
50.44	0\\
50.45	0\\
50.46	0\\
50.47	0\\
50.48	0\\
50.49	0\\
50.5	0\\
50.51	0\\
50.52	0\\
50.53	0\\
50.54	0\\
50.55	0\\
50.56	0\\
50.57	0\\
50.58	0\\
50.59	0\\
50.6	0\\
50.61	0\\
50.62	0\\
50.63	0\\
50.64	0\\
50.65	0\\
50.66	0\\
50.67	0\\
50.68	0\\
50.69	0\\
50.7	0\\
50.71	0\\
50.72	0\\
50.73	0\\
50.74	0\\
50.75	0\\
50.76	0\\
50.77	0\\
50.78	0\\
50.79	0\\
50.8	0\\
50.81	0\\
50.82	0\\
50.83	0\\
50.84	0\\
50.85	0\\
50.86	0\\
50.87	0\\
50.88	0\\
50.89	0\\
50.9	0\\
50.91	0\\
50.92	0\\
50.93	0\\
50.94	0\\
50.95	0\\
50.96	0\\
50.97	0\\
50.98	0\\
50.99	0\\
51	0\\
51.01	0\\
51.02	0\\
51.03	0\\
51.04	0\\
51.05	0\\
51.06	0\\
51.07	0\\
51.08	0\\
51.09	0\\
51.1	0\\
51.11	0\\
51.12	0\\
51.13	0\\
51.14	0\\
51.15	0\\
51.16	0\\
51.17	0\\
51.18	0\\
51.19	0\\
51.2	0\\
51.21	0\\
51.22	0\\
51.23	0\\
51.24	0\\
51.25	0\\
51.26	0\\
51.27	0\\
51.28	0\\
51.29	0\\
51.3	0\\
51.31	0\\
51.32	0\\
51.33	0\\
51.34	0\\
51.35	0\\
51.36	0\\
51.37	0\\
51.38	0\\
51.39	0\\
51.4	0\\
51.41	0\\
51.42	0\\
51.43	0\\
51.44	0\\
51.45	0\\
51.46	0\\
51.47	0\\
51.48	0\\
51.49	0\\
51.5	0\\
51.51	0\\
51.52	0\\
51.53	0\\
51.54	0\\
51.55	0\\
51.56	0\\
51.57	0\\
51.58	0\\
51.59	0\\
51.6	0\\
51.61	0\\
51.62	0\\
51.63	0\\
51.64	0\\
51.65	0\\
51.66	0\\
51.67	0\\
51.68	0\\
51.69	0\\
51.7	0\\
51.71	0\\
51.72	0\\
51.73	0\\
51.74	0\\
51.75	0\\
51.76	0\\
51.77	0\\
51.78	0\\
51.79	0\\
51.8	0\\
51.81	0\\
51.82	0\\
51.83	0\\
51.84	0\\
51.85	0\\
51.86	0\\
51.87	0\\
51.88	0\\
51.89	0\\
51.9	0\\
51.91	0\\
51.92	0\\
51.93	0\\
51.94	0\\
51.95	0\\
51.96	0\\
51.97	0\\
51.98	0\\
51.99	0\\
52	0\\
52.01	0\\
52.02	0\\
52.03	0\\
52.04	0\\
52.05	0\\
52.06	0\\
52.07	0\\
52.08	0\\
52.09	0\\
52.1	0\\
52.11	0\\
52.12	0\\
52.13	0\\
52.14	0\\
52.15	0\\
52.16	0\\
52.17	0\\
52.18	0\\
52.19	0\\
52.2	0\\
52.21	0\\
52.22	0\\
52.23	0\\
52.24	0\\
52.25	0\\
52.26	0\\
52.27	0\\
52.28	0\\
52.29	0\\
52.3	0\\
52.31	0\\
52.32	0\\
52.33	0\\
52.34	0\\
52.35	0\\
52.36	0\\
52.37	0\\
52.38	0\\
52.39	0\\
52.4	0\\
52.41	0\\
52.42	0\\
52.43	0\\
52.44	0\\
52.45	0\\
52.46	0\\
52.47	0\\
52.48	0\\
52.49	0\\
52.5	0\\
52.51	0\\
52.52	0\\
52.53	0\\
52.54	0\\
52.55	0\\
52.56	0\\
52.57	0\\
52.58	0\\
52.59	0\\
52.6	0\\
52.61	0\\
52.62	0\\
52.63	0\\
52.64	0\\
52.65	0\\
52.66	0\\
52.67	0\\
52.68	0\\
52.69	0\\
52.7	0\\
52.71	0\\
52.72	0\\
52.73	0\\
52.74	0\\
52.75	0\\
52.76	0\\
52.77	0\\
52.78	0\\
52.79	0\\
52.8	0\\
52.81	0\\
52.82	0\\
52.83	0\\
52.84	0\\
52.85	0\\
52.86	0\\
52.87	0\\
52.88	0\\
52.89	0\\
52.9	0\\
52.91	0\\
52.92	0\\
52.93	0\\
52.94	0\\
52.95	0\\
52.96	0\\
52.97	0\\
52.98	0\\
52.99	0\\
53	0\\
53.01	0\\
53.02	0\\
53.03	0\\
53.04	0\\
53.05	0\\
53.06	0\\
53.07	0\\
53.08	0\\
53.09	0\\
53.1	0\\
53.11	0\\
53.12	0\\
53.13	0\\
53.14	0\\
53.15	0\\
53.16	0\\
53.17	0\\
53.18	0\\
53.19	0\\
53.2	0\\
53.21	0\\
53.22	0\\
53.23	0\\
53.24	0\\
53.25	0\\
53.26	0\\
53.27	0\\
53.28	0\\
53.29	0\\
53.3	0\\
53.31	0\\
53.32	0\\
53.33	0\\
53.34	0\\
53.35	0\\
53.36	0\\
53.37	0\\
53.38	0\\
53.39	0\\
53.4	0\\
53.41	0\\
53.42	0\\
53.43	0\\
53.44	0\\
53.45	0\\
53.46	0\\
53.47	0\\
53.48	0\\
53.49	0\\
53.5	0\\
53.51	0\\
53.52	0\\
53.53	0\\
53.54	0\\
53.55	0\\
53.56	0\\
53.57	0\\
53.58	0\\
53.59	0\\
53.6	0\\
53.61	0\\
53.62	0\\
53.63	0\\
53.64	0\\
53.65	0\\
53.66	0\\
53.67	0\\
53.68	0\\
53.69	0\\
53.7	0\\
53.71	0\\
53.72	0\\
53.73	0\\
53.74	0\\
53.75	0\\
53.76	0\\
53.77	0\\
53.78	0\\
53.79	0\\
53.8	0\\
53.81	0\\
53.82	0\\
53.83	0\\
53.84	0\\
53.85	0\\
53.86	0\\
53.87	0\\
53.88	0\\
53.89	0\\
53.9	0\\
53.91	0\\
53.92	0\\
53.93	0\\
53.94	0\\
53.95	0\\
53.96	0\\
53.97	0\\
53.98	0\\
53.99	0\\
54	0\\
54.01	0\\
54.02	0\\
54.03	0\\
54.04	0\\
54.05	0\\
54.06	0\\
54.07	0\\
54.08	0\\
54.09	0\\
54.1	0\\
54.11	0\\
54.12	0\\
54.13	0\\
54.14	0\\
54.15	0\\
54.16	0\\
54.17	0\\
54.18	0\\
54.19	0\\
54.2	0\\
54.21	0\\
54.22	0\\
54.23	0\\
54.24	0\\
54.25	0\\
54.26	0\\
54.27	0\\
54.28	0\\
54.29	0\\
54.3	0\\
54.31	0\\
54.32	0\\
54.33	0\\
54.34	0\\
54.35	0\\
54.36	0\\
54.37	0\\
54.38	0\\
54.39	0\\
54.4	0\\
54.41	0\\
54.42	0\\
54.43	0\\
54.44	0\\
54.45	0\\
54.46	0\\
54.47	0\\
54.48	0\\
54.49	0\\
54.5	0\\
54.51	0\\
54.52	0\\
54.53	0\\
54.54	0\\
54.55	0\\
54.56	0\\
54.57	0\\
54.58	0\\
54.59	0\\
54.6	0\\
54.61	0\\
54.62	0\\
54.63	0\\
54.64	0\\
54.65	0\\
54.66	0\\
54.67	0\\
54.68	0\\
54.69	0\\
54.7	0\\
54.71	0\\
54.72	0\\
54.73	0\\
54.74	0\\
54.75	0\\
54.76	0\\
54.77	0\\
54.78	0\\
54.79	0\\
54.8	0\\
54.81	0\\
54.82	0\\
54.83	0\\
54.84	0\\
54.85	0\\
54.86	0\\
54.87	0\\
54.88	0\\
54.89	0\\
54.9	0\\
54.91	0\\
54.92	0\\
54.93	0\\
54.94	0\\
54.95	0\\
54.96	0\\
54.97	0\\
54.98	0\\
54.99	0\\
55	0\\
55.01	0\\
55.02	0\\
55.03	0\\
55.04	0\\
55.05	0\\
55.06	0\\
55.07	0\\
55.08	0\\
55.09	0\\
55.1	0\\
55.11	0\\
55.12	0\\
55.13	0\\
55.14	0\\
55.15	0\\
55.16	0\\
55.17	0\\
55.18	0\\
55.19	0\\
55.2	0\\
55.21	0\\
55.22	0\\
55.23	0\\
55.24	0\\
55.25	0\\
55.26	0\\
55.27	0\\
55.28	0\\
55.29	0\\
55.3	0\\
55.31	0\\
55.32	0\\
55.33	0\\
55.34	0\\
55.35	0\\
55.36	0\\
55.37	0\\
55.38	0\\
55.39	0\\
55.4	0\\
55.41	0\\
55.42	0\\
55.43	0\\
55.44	0\\
55.45	0\\
55.46	0\\
55.47	0\\
55.48	0\\
55.49	0\\
55.5	0\\
55.51	0\\
55.52	0\\
55.53	0\\
55.54	0\\
55.55	0\\
55.56	0\\
55.57	0\\
55.58	0\\
55.59	0\\
55.6	0\\
55.61	0\\
55.62	0\\
55.63	0\\
55.64	0\\
55.65	0\\
55.66	0\\
55.67	0\\
55.68	0\\
55.69	0\\
55.7	0\\
55.71	0\\
55.72	0\\
55.73	0\\
55.74	0\\
55.75	0\\
55.76	0\\
55.77	0\\
55.78	0\\
55.79	0\\
55.8	0\\
55.81	0\\
55.82	0\\
55.83	0\\
55.84	0\\
55.85	0\\
55.86	0\\
55.87	0\\
55.88	0\\
55.89	0\\
55.9	0\\
55.91	0\\
55.92	0\\
55.93	0\\
55.94	0\\
55.95	0\\
55.96	0\\
55.97	0\\
55.98	0\\
55.99	0\\
56	0\\
56.01	0\\
56.02	0\\
56.03	0\\
56.04	0\\
56.05	0\\
56.06	0\\
56.07	0\\
56.08	0\\
56.09	0\\
56.1	0\\
56.11	0\\
56.12	0\\
56.13	0\\
56.14	0\\
56.15	0\\
56.16	0\\
56.17	0\\
56.18	0\\
56.19	0\\
56.2	0\\
56.21	0\\
56.22	0\\
56.23	0\\
56.24	0\\
56.25	0\\
56.26	0\\
56.27	0\\
56.28	0\\
56.29	0\\
56.3	0\\
56.31	0\\
56.32	0\\
56.33	0\\
56.34	0\\
56.35	0\\
56.36	0\\
56.37	0\\
56.38	0\\
56.39	0\\
56.4	0\\
56.41	0\\
56.42	0\\
56.43	0\\
56.44	0\\
56.45	0\\
56.46	0\\
56.47	0\\
56.48	0\\
56.49	0\\
56.5	0\\
56.51	0\\
56.52	0\\
56.53	0\\
56.54	0\\
56.55	0\\
56.56	0\\
56.57	0\\
56.58	0\\
56.59	0\\
56.6	0\\
56.61	0\\
56.62	0\\
56.63	0\\
56.64	0\\
56.65	0\\
56.66	0\\
56.67	0\\
56.68	0\\
56.69	0\\
56.7	0\\
56.71	0\\
56.72	0\\
56.73	0\\
56.74	0\\
56.75	0\\
56.76	0\\
56.77	0\\
56.78	0\\
56.79	0\\
56.8	0\\
56.81	0\\
56.82	0\\
56.83	0\\
56.84	0\\
56.85	0\\
56.86	0\\
56.87	0\\
56.88	0\\
56.89	0\\
56.9	0\\
56.91	0\\
56.92	0\\
56.93	0\\
56.94	0\\
56.95	0\\
56.96	0\\
56.97	0\\
56.98	0\\
56.99	0\\
57	0\\
57.01	0\\
57.02	0\\
57.03	0\\
57.04	0\\
57.05	0\\
57.06	0\\
57.07	0\\
57.08	0\\
57.09	0\\
57.1	0\\
57.11	0\\
57.12	0\\
57.13	0\\
57.14	0\\
57.15	0\\
57.16	0\\
57.17	0\\
57.18	0\\
57.19	0\\
57.2	0\\
57.21	0\\
57.22	0\\
57.23	0\\
57.24	0\\
57.25	0\\
57.26	0\\
57.27	0\\
57.28	0\\
57.29	0\\
57.3	0\\
57.31	0\\
57.32	0\\
57.33	0\\
57.34	0\\
57.35	0\\
57.36	0\\
57.37	0\\
57.38	0\\
57.39	0\\
57.4	0\\
57.41	0\\
57.42	0\\
57.43	0\\
57.44	0\\
57.45	0\\
57.46	0\\
57.47	0\\
57.48	0\\
57.49	0\\
57.5	0\\
57.51	0\\
57.52	0\\
57.53	0\\
57.54	0\\
57.55	0\\
57.56	0\\
57.57	0\\
57.58	0\\
57.59	0\\
57.6	0\\
57.61	0\\
57.62	0\\
57.63	0\\
57.64	0\\
57.65	0\\
57.66	0\\
57.67	0\\
57.68	0\\
57.69	0\\
57.7	0\\
57.71	0\\
57.72	0\\
57.73	0\\
57.74	0\\
57.75	0\\
57.76	0\\
57.77	0\\
57.78	0\\
57.79	0\\
57.8	0\\
57.81	0\\
57.82	0\\
57.83	0\\
57.84	0\\
57.85	0\\
57.86	0\\
57.87	0\\
57.88	0\\
57.89	0\\
57.9	0\\
57.91	0\\
57.92	0\\
57.93	0\\
57.94	0\\
57.95	0\\
57.96	0\\
57.97	0\\
57.98	0\\
57.99	0\\
58	0\\
58.01	0\\
58.02	0\\
58.03	0\\
58.04	0\\
58.05	0\\
58.06	0\\
58.07	0\\
58.08	0\\
58.09	0\\
58.1	0\\
58.11	0\\
58.12	0\\
58.13	0\\
58.14	0\\
58.15	0\\
58.16	0\\
58.17	0\\
58.18	0\\
58.19	0\\
58.2	0\\
58.21	0\\
58.22	0\\
58.23	0\\
58.24	0\\
58.25	0\\
58.26	0\\
58.27	0\\
58.28	0\\
58.29	0\\
58.3	0\\
58.31	0\\
58.32	0\\
58.33	0\\
58.34	0\\
58.35	0\\
58.36	0\\
58.37	0\\
58.38	0\\
58.39	0\\
58.4	0\\
58.41	0\\
58.42	0\\
58.43	0\\
58.44	0\\
58.45	0\\
58.46	0\\
58.47	0\\
58.48	0\\
58.49	0\\
58.5	0\\
58.51	0\\
58.52	0\\
58.53	0\\
58.54	0\\
58.55	0\\
58.56	0\\
58.57	0\\
58.58	0\\
58.59	0\\
58.6	0\\
58.61	0\\
58.62	0\\
58.63	0\\
58.64	0\\
58.65	0\\
58.66	0\\
58.67	0\\
58.68	0\\
58.69	0\\
58.7	0\\
58.71	0\\
58.72	0\\
58.73	0\\
58.74	0\\
58.75	0\\
58.76	0\\
58.77	0\\
58.78	0\\
58.79	0\\
58.8	0\\
58.81	0\\
58.82	0\\
58.83	0\\
58.84	0\\
58.85	0\\
58.86	0\\
58.87	0\\
58.88	0\\
58.89	0\\
58.9	0\\
58.91	0\\
58.92	0\\
58.93	0\\
58.94	0\\
58.95	0\\
58.96	0\\
58.97	0\\
58.98	0\\
58.99	0\\
59	0\\
59.01	0\\
59.02	0\\
59.03	0\\
59.04	0\\
59.05	0\\
59.06	0\\
59.07	0\\
59.08	0\\
59.09	0\\
59.1	0\\
59.11	0\\
59.12	0\\
59.13	0\\
59.14	0\\
59.15	0\\
59.16	0\\
59.17	0\\
59.18	0\\
59.19	0\\
59.2	0\\
59.21	0\\
59.22	0\\
59.23	0\\
59.24	0\\
59.25	0\\
59.26	0\\
59.27	0\\
59.28	0\\
59.29	0\\
59.3	0\\
59.31	0\\
59.32	0\\
59.33	0\\
59.34	0\\
59.35	0\\
59.36	0\\
59.37	0\\
59.38	0\\
59.39	0\\
59.4	0\\
59.41	0\\
59.42	0\\
59.43	0\\
59.44	0\\
59.45	0\\
59.46	0\\
59.47	0\\
59.48	0\\
59.49	0\\
59.5	0\\
59.51	0\\
59.52	0\\
59.53	0\\
59.54	0\\
59.55	0\\
59.56	0\\
59.57	0\\
59.58	0\\
59.59	0\\
59.6	0\\
59.61	0\\
59.62	0\\
59.63	0\\
59.64	0\\
59.65	0\\
59.66	0\\
59.67	0\\
59.68	0\\
59.69	0\\
59.7	0\\
59.71	0\\
59.72	0\\
59.73	0\\
59.74	0\\
59.75	0\\
59.76	0\\
59.77	0\\
59.78	0\\
59.79	0\\
59.8	0\\
59.81	0\\
59.82	0\\
59.83	0\\
59.84	0\\
59.85	0\\
59.86	0\\
59.87	0\\
59.88	0\\
59.89	0\\
59.9	0\\
59.91	0\\
59.92	0\\
59.93	0\\
59.94	0\\
59.95	0\\
59.96	0\\
59.97	0\\
59.98	0\\
59.99	0\\
60	0\\
60.01	0\\
60.02	0\\
60.03	0\\
60.04	0\\
60.05	0\\
60.06	0\\
60.07	0\\
60.08	0\\
60.09	0\\
60.1	0\\
60.11	0\\
60.12	0\\
60.13	0\\
60.14	0\\
60.15	0\\
60.16	0\\
60.17	0\\
60.18	0\\
60.19	0\\
60.2	0\\
60.21	0\\
60.22	0\\
60.23	0\\
60.24	0\\
60.25	0\\
60.26	0\\
60.27	0\\
60.28	0\\
60.29	0\\
60.3	0\\
60.31	0\\
60.32	0\\
60.33	0\\
60.34	0\\
60.35	0\\
60.36	0\\
60.37	0\\
60.38	0\\
60.39	0\\
60.4	0\\
60.41	0\\
60.42	0\\
60.43	0\\
60.44	0\\
60.45	0\\
60.46	0\\
60.47	0\\
60.48	0\\
60.49	0\\
60.5	0\\
60.51	0\\
60.52	0\\
60.53	0\\
60.54	0\\
60.55	0\\
60.56	0\\
60.57	0\\
60.58	0\\
60.59	0\\
60.6	0\\
60.61	0\\
60.62	0\\
60.63	0\\
60.64	0\\
60.65	0\\
60.66	0\\
60.67	0\\
60.68	0\\
60.69	0\\
60.7	0\\
60.71	0\\
60.72	0\\
60.73	0\\
60.74	0\\
60.75	0\\
60.76	0\\
60.77	0\\
60.78	0\\
60.79	0\\
60.8	0\\
60.81	0\\
60.82	0\\
60.83	0\\
60.84	0\\
60.85	0\\
60.86	0\\
60.87	0\\
60.88	0\\
60.89	0\\
60.9	0\\
60.91	0\\
60.92	0\\
60.93	0\\
60.94	0\\
60.95	0\\
60.96	0\\
60.97	0\\
60.98	0\\
60.99	0\\
61	0\\
61.01	0\\
61.02	0\\
61.03	0\\
61.04	0\\
61.05	0\\
61.06	0\\
61.07	0\\
61.08	0\\
61.09	0\\
61.1	0\\
61.11	0\\
61.12	0\\
61.13	0\\
61.14	0\\
61.15	0\\
61.16	0\\
61.17	0\\
61.18	0\\
61.19	0\\
61.2	0\\
61.21	0\\
61.22	0\\
61.23	0\\
61.24	0\\
61.25	0\\
61.26	0\\
61.27	0\\
61.28	0\\
61.29	0\\
61.3	0\\
61.31	0\\
61.32	0\\
61.33	0\\
61.34	0\\
61.35	0\\
61.36	0\\
61.37	0\\
61.38	0\\
61.39	0\\
61.4	0\\
61.41	0\\
61.42	0\\
61.43	0\\
61.44	0\\
61.45	0\\
61.46	0\\
61.47	0\\
61.48	0\\
61.49	0\\
61.5	0\\
61.51	0\\
61.52	0\\
61.53	0\\
61.54	0\\
61.55	0\\
61.56	0\\
61.57	0\\
61.58	0\\
61.59	0\\
61.6	0\\
61.61	0\\
61.62	0\\
61.63	0\\
61.64	0\\
61.65	0\\
61.66	0\\
61.67	0\\
61.68	0\\
61.69	0\\
61.7	0\\
61.71	0\\
61.72	0\\
61.73	0\\
61.74	0\\
61.75	0\\
61.76	0\\
61.77	0\\
61.78	0\\
61.79	0\\
61.8	0\\
61.81	0\\
61.82	0\\
61.83	0\\
61.84	0\\
61.85	0\\
61.86	0\\
61.87	0\\
61.88	0\\
61.89	0\\
61.9	0\\
61.91	0\\
61.92	0\\
61.93	0\\
61.94	0\\
61.95	0\\
61.96	0\\
61.97	0\\
61.98	0\\
61.99	0\\
62	0\\
62.01	0\\
62.02	0\\
62.03	0\\
62.04	0\\
62.05	0\\
62.06	0\\
62.07	0\\
62.08	0\\
62.09	0\\
62.1	0\\
62.11	0\\
62.12	0\\
62.13	0\\
62.14	0\\
62.15	0\\
62.16	0\\
62.17	0\\
62.18	0\\
62.19	0\\
62.2	0\\
62.21	0\\
62.22	0\\
62.23	0\\
62.24	0\\
62.25	0\\
62.26	0\\
62.27	0\\
62.28	0\\
62.29	0\\
62.3	0\\
62.31	0\\
62.32	0\\
62.33	0\\
62.34	0\\
62.35	0\\
62.36	0\\
62.37	0\\
62.38	0\\
62.39	0\\
62.4	0\\
62.41	0\\
62.42	0\\
62.43	0\\
62.44	0\\
62.45	0\\
62.46	0\\
62.47	0\\
62.48	0\\
62.49	0\\
62.5	0\\
62.51	0\\
62.52	0\\
62.53	0\\
62.54	0\\
62.55	0\\
62.56	0\\
62.57	0\\
62.58	0\\
62.59	0\\
62.6	0\\
62.61	0\\
62.62	0\\
62.63	0\\
62.64	0\\
62.65	0\\
62.66	0\\
62.67	0\\
62.68	0\\
62.69	0\\
62.7	0\\
62.71	0\\
62.72	0\\
62.73	0\\
62.74	0\\
62.75	0\\
62.76	0\\
62.77	0\\
62.78	0\\
62.79	0\\
62.8	0\\
62.81	0\\
62.82	0\\
62.83	0\\
62.84	0\\
62.85	0\\
62.86	0\\
62.87	0\\
62.88	0\\
62.89	0\\
62.9	0\\
62.91	0\\
62.92	0\\
62.93	0\\
62.94	0\\
62.95	0\\
62.96	0\\
62.97	0\\
62.98	0\\
62.99	0\\
63	0\\
63.01	0\\
63.02	0\\
63.03	0\\
63.04	0\\
63.05	0\\
63.06	0\\
63.07	0\\
63.08	0\\
63.09	0\\
63.1	0\\
63.11	0\\
63.12	0\\
63.13	0\\
63.14	0\\
63.15	0\\
63.16	0\\
63.17	0\\
63.18	0\\
63.19	0\\
63.2	0\\
63.21	0\\
63.22	0\\
63.23	0\\
63.24	0\\
63.25	0\\
63.26	0\\
63.27	0\\
63.28	0\\
63.29	0\\
63.3	0\\
63.31	0\\
63.32	0\\
63.33	0\\
63.34	0\\
63.35	0\\
63.36	0\\
63.37	0\\
63.38	0\\
63.39	0\\
63.4	0\\
63.41	0\\
63.42	0\\
63.43	0\\
63.44	0\\
63.45	0\\
63.46	0\\
63.47	0\\
63.48	0\\
63.49	0\\
63.5	0\\
63.51	0\\
63.52	0\\
63.53	0\\
63.54	0\\
63.55	0\\
63.56	0\\
63.57	0\\
63.58	0\\
63.59	0\\
63.6	0\\
63.61	0\\
63.62	0\\
63.63	0\\
63.64	0\\
63.65	0\\
63.66	0\\
63.67	0\\
63.68	0\\
63.69	0\\
63.7	0\\
63.71	0\\
63.72	0\\
63.73	0\\
63.74	0\\
63.75	0\\
63.76	0\\
63.77	0\\
63.78	0\\
63.79	0\\
63.8	0\\
63.81	0\\
63.82	0\\
63.83	0\\
63.84	0\\
63.85	0\\
63.86	0\\
63.87	0\\
63.88	0\\
63.89	0\\
63.9	0\\
63.91	0\\
63.92	0\\
63.93	0\\
63.94	0\\
63.95	0\\
63.96	0\\
63.97	0\\
63.98	0\\
63.99	0\\
64	0\\
64.01	0\\
64.02	0\\
64.03	0\\
64.04	0\\
64.05	0\\
64.06	0\\
64.07	0\\
64.08	0\\
64.09	0\\
64.1	0\\
64.11	0\\
64.12	0\\
64.13	0\\
64.14	0\\
64.15	0\\
64.16	0\\
64.17	0\\
64.18	0\\
64.19	0\\
64.2	0\\
64.21	0\\
64.22	0\\
64.23	0\\
64.24	0\\
64.25	0\\
64.26	0\\
64.27	0\\
64.28	0\\
64.29	0\\
64.3	0\\
64.31	0\\
64.32	0\\
64.33	0\\
64.34	0\\
64.35	0\\
64.36	0\\
64.37	0\\
64.38	0\\
64.39	0\\
64.4	0\\
64.41	0\\
64.42	0\\
64.43	0\\
64.44	0\\
64.45	0\\
64.46	0\\
64.47	0\\
64.48	0\\
64.49	0\\
64.5	0\\
64.51	0\\
64.52	0\\
64.53	0\\
64.54	0\\
64.55	0\\
64.56	0\\
64.57	0\\
64.58	0\\
64.59	0\\
64.6	0\\
64.61	0\\
64.62	0\\
64.63	0\\
64.64	0\\
64.65	0\\
64.66	0\\
64.67	0\\
64.68	0\\
64.69	0\\
64.7	0\\
64.71	0\\
64.72	0\\
64.73	0\\
64.74	0\\
64.75	0\\
64.76	0\\
64.77	0\\
64.78	0\\
64.79	0\\
64.8	0\\
64.81	0\\
64.82	0\\
64.83	0\\
64.84	0\\
64.85	0\\
64.86	0\\
64.87	0\\
64.88	0\\
64.89	0\\
64.9	0\\
64.91	0\\
64.92	0\\
64.93	0\\
64.94	0\\
64.95	0\\
64.96	0\\
64.97	0\\
64.98	0\\
64.99	0\\
65	0\\
65.01	0\\
65.02	0\\
65.03	0\\
65.04	0\\
65.05	0\\
65.06	0\\
65.07	0\\
65.08	0\\
65.09	0\\
65.1	0\\
65.11	0\\
65.12	0\\
65.13	0\\
65.14	0\\
65.15	0\\
65.16	0\\
65.17	0\\
65.18	0\\
65.19	0\\
65.2	0\\
65.21	0\\
65.22	0\\
65.23	0\\
65.24	0\\
65.25	0\\
65.26	0\\
65.27	0\\
65.28	0\\
65.29	0\\
65.3	0\\
65.31	0\\
65.32	0\\
65.33	0\\
65.34	0\\
65.35	0\\
65.36	0\\
65.37	0\\
65.38	0\\
65.39	0\\
65.4	0\\
65.41	0\\
65.42	0\\
65.43	0\\
65.44	0\\
65.45	0\\
65.46	0\\
65.47	0\\
65.48	0\\
65.49	0\\
65.5	0\\
65.51	0\\
65.52	0\\
65.53	0\\
65.54	0\\
65.55	0\\
65.56	0\\
65.57	0\\
65.58	0\\
65.59	0\\
65.6	0\\
65.61	0\\
65.62	0\\
65.63	0\\
65.64	0\\
65.65	0\\
65.66	0\\
65.67	0\\
65.68	0\\
65.69	0\\
65.7	0\\
65.71	0\\
65.72	0\\
65.73	0\\
65.74	0\\
65.75	0\\
65.76	0\\
65.77	0\\
65.78	0\\
65.79	0\\
65.8	0\\
65.81	0\\
65.82	0\\
65.83	0\\
65.84	0\\
65.85	0\\
65.86	0\\
65.87	0\\
65.88	0\\
65.89	0\\
65.9	0\\
65.91	0\\
65.92	0\\
65.93	0\\
65.94	0\\
65.95	0\\
65.96	0\\
65.97	0\\
65.98	0\\
65.99	0\\
66	0\\
66.01	0\\
66.02	0\\
66.03	0\\
66.04	0\\
66.05	0\\
66.06	0\\
66.07	0\\
66.08	0\\
66.09	0\\
66.1	0\\
66.11	0\\
66.12	0\\
66.13	0\\
66.14	0\\
66.15	0\\
66.16	0\\
66.17	0\\
66.18	0\\
66.19	0\\
66.2	0\\
66.21	0\\
66.22	0\\
66.23	0\\
66.24	0\\
66.25	0\\
66.26	0\\
66.27	0\\
66.28	0\\
66.29	0\\
66.3	0\\
66.31	0\\
66.32	0\\
66.33	0\\
66.34	0\\
66.35	0\\
66.36	0\\
66.37	0\\
66.38	0\\
66.39	0\\
66.4	0\\
66.41	0\\
66.42	0\\
66.43	0\\
66.44	0\\
66.45	0\\
66.46	0\\
66.47	0\\
66.48	0\\
66.49	0\\
66.5	0\\
66.51	0\\
66.52	0\\
66.53	0\\
66.54	0\\
66.55	0\\
66.56	0\\
66.57	0\\
66.58	0\\
66.59	0\\
66.6	0\\
66.61	0\\
66.62	0\\
66.63	0\\
66.64	0\\
66.65	0\\
66.66	0\\
66.67	0\\
66.68	0\\
66.69	0\\
66.7	0\\
66.71	0\\
66.72	0\\
66.73	0\\
66.74	0\\
66.75	0\\
66.76	0\\
66.77	0\\
66.78	0\\
66.79	0\\
66.8	0\\
66.81	0\\
66.82	0\\
66.83	0\\
66.84	0\\
66.85	0\\
66.86	0\\
66.87	0\\
66.88	0\\
66.89	0\\
66.9	0\\
66.91	0\\
66.92	0\\
66.93	0\\
66.94	0\\
66.95	0\\
66.96	0\\
66.97	0\\
66.98	0\\
66.99	0\\
67	0\\
67.01	0\\
67.02	0\\
67.03	0\\
67.04	0\\
67.05	0\\
67.06	0\\
67.07	0\\
67.08	0\\
67.09	0\\
67.1	0\\
67.11	0\\
67.12	0\\
67.13	0\\
67.14	0\\
67.15	0\\
67.16	0\\
67.17	0\\
67.18	0\\
67.19	0\\
67.2	0\\
67.21	0\\
67.22	0\\
67.23	0\\
67.24	0\\
67.25	0\\
67.26	0\\
67.27	0\\
67.28	0\\
67.29	0\\
67.3	0\\
67.31	0\\
67.32	0\\
67.33	0\\
67.34	0\\
67.35	0\\
67.36	0\\
67.37	0\\
67.38	0\\
67.39	0\\
67.4	0\\
67.41	0\\
67.42	0\\
67.43	0\\
67.44	0\\
67.45	0\\
67.46	0\\
67.47	0\\
67.48	0\\
67.49	0\\
67.5	0\\
67.51	0\\
67.52	0\\
67.53	0\\
67.54	0\\
67.55	0\\
67.56	0\\
67.57	0\\
67.58	0\\
67.59	0\\
67.6	0\\
67.61	0\\
67.62	0\\
67.63	0\\
67.64	0\\
67.65	0\\
67.66	0\\
67.67	0\\
67.68	0\\
67.69	0\\
67.7	0\\
67.71	0\\
67.72	0\\
67.73	0\\
67.74	0\\
67.75	0\\
67.76	0\\
67.77	0\\
67.78	0\\
67.79	0\\
67.8	0\\
67.81	0\\
67.82	0\\
67.83	0\\
67.84	0\\
67.85	0\\
67.86	0\\
67.87	0\\
67.88	0\\
67.89	0\\
67.9	0\\
67.91	0\\
67.92	0\\
67.93	0\\
67.94	0\\
67.95	0\\
67.96	0\\
67.97	0\\
67.98	0\\
67.99	0\\
68	0\\
68.01	0\\
68.02	0\\
68.03	0\\
68.04	0\\
68.05	0\\
68.06	0\\
68.07	0\\
68.08	0\\
68.09	0\\
68.1	0\\
68.11	0\\
68.12	0\\
68.13	0\\
68.14	0\\
68.15	0\\
68.16	0\\
68.17	0\\
68.18	0\\
68.19	0\\
68.2	0\\
68.21	0\\
68.22	0\\
68.23	0\\
68.24	0\\
68.25	0\\
68.26	0\\
68.27	0\\
68.28	0\\
68.29	0\\
68.3	0\\
68.31	0\\
68.32	0\\
68.33	0\\
68.34	0\\
68.35	0\\
68.36	0\\
68.37	0\\
68.38	0\\
68.39	0\\
68.4	0\\
68.41	0\\
68.42	0\\
68.43	0\\
68.44	0\\
68.45	0\\
68.46	0\\
68.47	0\\
68.48	0\\
68.49	0\\
68.5	0\\
68.51	0\\
68.52	0\\
68.53	0\\
68.54	0\\
68.55	0\\
68.56	0\\
68.57	0\\
68.58	0\\
68.59	0\\
68.6	0\\
68.61	0\\
68.62	0\\
68.63	0\\
68.64	0\\
68.65	0\\
68.66	0\\
68.67	0\\
68.68	0\\
68.69	0\\
68.7	0\\
68.71	0\\
68.72	0\\
68.73	0\\
68.74	0\\
68.75	0\\
68.76	0\\
68.77	0\\
68.78	0\\
68.79	0\\
68.8	0\\
68.81	0\\
68.82	0\\
68.83	0\\
68.84	0\\
68.85	0\\
68.86	0\\
68.87	0\\
68.88	0\\
68.89	0\\
68.9	0\\
68.91	0\\
68.92	0\\
68.93	0\\
68.94	0\\
68.95	0\\
68.96	0\\
68.97	0\\
68.98	0\\
68.99	0\\
69	0\\
69.01	0\\
69.02	0\\
69.03	0\\
69.04	0\\
69.05	0\\
69.06	0\\
69.07	0\\
69.08	0\\
69.09	0\\
69.1	0\\
69.11	0\\
69.12	0\\
69.13	0\\
69.14	0\\
69.15	0\\
69.16	0\\
69.17	0\\
69.18	0\\
69.19	0\\
69.2	0\\
69.21	0\\
69.22	0\\
69.23	0\\
69.24	0\\
69.25	0\\
69.26	0\\
69.27	0\\
69.28	0\\
69.29	0\\
69.3	0\\
69.31	0\\
69.32	0\\
69.33	0\\
69.34	0\\
69.35	0\\
69.36	0\\
69.37	0\\
69.38	0\\
69.39	0\\
69.4	0\\
69.41	0\\
69.42	0\\
69.43	0\\
69.44	0\\
69.45	0\\
69.46	0\\
69.47	0\\
69.48	0\\
69.49	0\\
69.5	0\\
69.51	0\\
69.52	0\\
69.53	0\\
69.54	0\\
69.55	0\\
69.56	0\\
69.57	0\\
69.58	0\\
69.59	0\\
69.6	0\\
69.61	0\\
69.62	0\\
69.63	0\\
69.64	0\\
69.65	0\\
69.66	0\\
69.67	0\\
69.68	0\\
69.69	0\\
69.7	0\\
69.71	0\\
69.72	0\\
69.73	0\\
69.74	0\\
69.75	0\\
69.76	0\\
69.77	0\\
69.78	0\\
69.79	0\\
69.8	0\\
69.81	0\\
69.82	0\\
69.83	0\\
69.84	0\\
69.85	0\\
69.86	0\\
69.87	0\\
69.88	0\\
69.89	0\\
69.9	0\\
69.91	0\\
69.92	0\\
69.93	0\\
69.94	0\\
69.95	0\\
69.96	0\\
69.97	0\\
69.98	0\\
69.99	0\\
70	0\\
70.01	0\\
70.02	0\\
70.03	0\\
70.04	0\\
70.05	0\\
70.06	0\\
70.07	0\\
70.08	0\\
70.09	0\\
70.1	0\\
70.11	0\\
70.12	0\\
70.13	0\\
70.14	0\\
70.15	0\\
70.16	0\\
70.17	0\\
70.18	0\\
70.19	0\\
70.2	0\\
70.21	0\\
70.22	0\\
70.23	0\\
70.24	0\\
70.25	0\\
70.26	0\\
70.27	0\\
70.28	0\\
70.29	0\\
70.3	0\\
70.31	0\\
70.32	0\\
70.33	0\\
70.34	0\\
70.35	0\\
70.36	0\\
70.37	0\\
70.38	0\\
70.39	0\\
70.4	0\\
70.41	0\\
70.42	0\\
70.43	0\\
70.44	0\\
70.45	0\\
70.46	0\\
70.47	0\\
70.48	0\\
70.49	0\\
70.5	0\\
70.51	0\\
70.52	0\\
70.53	0\\
70.54	0\\
70.55	0\\
70.56	0\\
70.57	0\\
70.58	0\\
70.59	0\\
70.6	0\\
70.61	0\\
70.62	0\\
70.63	0\\
70.64	0\\
70.65	0\\
70.66	0\\
70.67	0\\
70.68	0\\
70.69	0\\
70.7	0\\
70.71	0\\
70.72	0\\
70.73	0\\
70.74	0\\
70.75	0\\
70.76	0\\
70.77	0\\
70.78	0\\
70.79	0\\
70.8	0\\
70.81	0\\
70.82	0\\
70.83	0\\
70.84	0\\
70.85	0\\
70.86	0\\
70.87	0\\
70.88	0\\
70.89	0\\
70.9	0\\
70.91	0\\
70.92	0\\
70.93	0\\
70.94	0\\
70.95	0\\
70.96	0\\
70.97	0\\
70.98	0\\
70.99	0\\
71	0\\
71.01	0\\
71.02	0\\
71.03	0\\
71.04	0\\
71.05	0\\
71.06	0\\
71.07	0\\
71.08	0\\
71.09	0\\
71.1	0\\
71.11	0\\
71.12	0\\
71.13	0\\
71.14	0\\
71.15	0\\
71.16	0\\
71.17	0\\
71.18	0\\
71.19	0\\
71.2	0\\
71.21	0\\
71.22	0\\
71.23	0\\
71.24	0\\
71.25	0\\
71.26	0\\
71.27	0\\
71.28	0\\
71.29	0\\
71.3	0\\
71.31	0\\
71.32	0\\
71.33	0\\
71.34	0\\
71.35	0\\
71.36	0\\
71.37	0\\
71.38	0\\
71.39	0\\
71.4	0\\
71.41	0\\
71.42	0\\
71.43	0\\
71.44	0\\
71.45	0\\
71.46	0\\
71.47	0\\
71.48	0\\
71.49	0\\
71.5	0\\
71.51	0\\
71.52	0\\
71.53	0\\
71.54	0\\
71.55	0\\
71.56	0\\
71.57	0\\
71.58	0\\
71.59	0\\
71.6	0\\
71.61	0\\
71.62	0\\
71.63	0\\
71.64	0\\
71.65	0\\
71.66	0\\
71.67	0\\
71.68	0\\
71.69	0\\
71.7	0\\
71.71	0\\
71.72	0\\
71.73	0\\
71.74	0\\
71.75	0\\
71.76	0\\
71.77	0\\
71.78	0\\
71.79	0\\
71.8	0\\
71.81	0\\
71.82	0\\
71.83	0\\
71.84	0\\
71.85	0\\
71.86	0\\
71.87	0\\
71.88	0\\
71.89	0\\
71.9	0\\
71.91	0\\
71.92	0\\
71.93	0\\
71.94	0\\
71.95	0\\
71.96	0\\
71.97	0\\
71.98	0\\
71.99	0\\
72	0\\
72.01	0\\
72.02	0\\
72.03	0\\
72.04	0\\
72.05	0\\
72.06	0\\
72.07	0\\
72.08	0\\
72.09	0\\
72.1	0\\
72.11	0\\
72.12	0\\
72.13	0\\
72.14	0\\
72.15	0\\
72.16	0\\
72.17	0\\
72.18	0\\
72.19	0\\
72.2	0\\
72.21	0\\
72.22	0\\
72.23	0\\
72.24	0\\
72.25	0\\
72.26	0\\
72.27	0\\
72.28	0\\
72.29	0\\
72.3	0\\
72.31	0\\
72.32	0\\
72.33	0\\
72.34	0\\
72.35	0\\
72.36	0\\
72.37	0\\
72.38	0\\
72.39	0\\
72.4	0\\
72.41	0\\
72.42	0\\
72.43	0\\
72.44	0\\
72.45	0\\
72.46	0\\
72.47	0\\
72.48	0\\
72.49	0\\
72.5	0\\
72.51	0\\
72.52	0\\
72.53	0\\
72.54	0\\
72.55	0\\
72.56	0\\
72.57	0\\
72.58	0\\
72.59	0\\
72.6	0\\
72.61	0\\
72.62	0\\
72.63	0\\
72.64	0\\
72.65	0\\
72.66	0\\
72.67	0\\
72.68	0\\
72.69	0\\
72.7	0\\
72.71	0\\
72.72	0\\
72.73	0\\
72.74	0\\
72.75	0\\
72.76	0\\
72.77	0\\
72.78	0\\
72.79	0\\
72.8	0\\
72.81	0\\
72.82	0\\
72.83	0\\
72.84	0\\
72.85	0\\
72.86	0\\
72.87	0\\
72.88	0\\
72.89	0\\
72.9	0\\
72.91	0\\
72.92	0\\
72.93	0\\
72.94	0\\
72.95	0\\
72.96	0\\
72.97	0\\
72.98	0\\
72.99	0\\
73	0\\
73.01	0\\
73.02	0\\
73.03	0\\
73.04	0\\
73.05	0\\
73.06	0\\
73.07	0\\
73.08	0\\
73.09	0\\
73.1	0\\
73.11	0\\
73.12	0\\
73.13	0\\
73.14	0\\
73.15	0\\
73.16	0\\
73.17	0\\
73.18	0\\
73.19	0\\
73.2	0\\
73.21	0\\
73.22	0\\
73.23	0\\
73.24	0\\
73.25	0\\
73.26	0\\
73.27	0\\
73.28	0\\
73.29	0\\
73.3	0\\
73.31	0\\
73.32	0\\
73.33	0\\
73.34	0\\
73.35	0\\
73.36	0\\
73.37	0\\
73.38	0\\
73.39	0\\
73.4	0\\
73.41	0\\
73.42	0\\
73.43	0\\
73.44	0\\
73.45	0\\
73.46	0\\
73.47	0\\
73.48	0\\
73.49	0\\
73.5	0\\
73.51	0\\
73.52	0\\
73.53	0\\
73.54	0\\
73.55	0\\
73.56	0\\
73.57	0\\
73.58	0\\
73.59	0\\
73.6	0\\
73.61	0\\
73.62	0\\
73.63	0\\
73.64	0\\
73.65	0\\
73.66	0\\
73.67	0\\
73.68	0\\
73.69	0\\
73.7	0\\
73.71	0\\
73.72	0\\
73.73	0\\
73.74	0\\
73.75	0\\
73.76	0\\
73.77	0\\
73.78	0\\
73.79	0\\
73.8	0\\
73.81	0\\
73.82	0\\
73.83	0\\
73.84	0\\
73.85	0\\
73.86	0\\
73.87	0\\
73.88	0\\
73.89	0\\
73.9	0\\
73.91	0\\
73.92	0\\
73.93	0\\
73.94	0\\
73.95	0\\
73.96	0\\
73.97	0\\
73.98	0\\
73.99	0\\
74	0\\
74.01	0\\
74.02	0\\
74.03	0\\
74.04	0\\
74.05	0\\
74.06	0\\
74.07	0\\
74.08	0\\
74.09	0\\
74.1	0\\
74.11	0\\
74.12	0\\
74.13	0\\
74.14	0\\
74.15	0\\
74.16	0\\
74.17	0\\
74.18	0\\
74.19	0\\
74.2	0\\
74.21	0\\
74.22	0\\
74.23	0\\
74.24	0\\
74.25	0\\
74.26	0\\
74.27	0\\
74.28	0\\
74.29	0\\
74.3	0\\
74.31	0\\
74.32	0\\
74.33	0\\
74.34	0\\
74.35	0\\
74.36	0\\
74.37	0\\
74.38	0\\
74.39	0\\
74.4	0\\
74.41	0\\
74.42	0\\
74.43	0\\
74.44	0\\
74.45	0\\
74.46	0\\
74.47	0\\
74.48	0\\
74.49	0\\
74.5	0\\
74.51	0\\
74.52	0\\
74.53	0\\
74.54	0\\
74.55	0\\
74.56	0\\
74.57	0\\
74.58	0\\
74.59	0\\
74.6	0\\
74.61	0\\
74.62	0\\
74.63	0\\
74.64	0\\
74.65	0\\
74.66	0\\
74.67	0\\
74.68	0\\
74.69	0\\
74.7	0\\
74.71	0\\
74.72	0\\
74.73	0\\
74.74	0\\
74.75	0\\
74.76	0\\
74.77	0\\
74.78	0\\
74.79	0\\
74.8	0\\
74.81	0\\
74.82	0\\
74.83	0\\
74.84	0\\
74.85	0\\
74.86	0\\
74.87	0\\
74.88	0\\
74.89	0\\
74.9	0\\
74.91	0\\
74.92	0\\
74.93	0\\
74.94	0\\
74.95	0\\
74.96	0\\
74.97	0\\
74.98	0\\
74.99	0\\
75	0\\
75.01	0\\
75.02	0\\
75.03	0\\
75.04	0\\
75.05	0\\
75.06	0\\
75.07	0\\
75.08	0\\
75.09	0\\
75.1	0\\
75.11	0\\
75.12	0\\
75.13	0\\
75.14	0\\
75.15	0\\
75.16	0\\
75.17	0\\
75.18	0\\
75.19	0\\
75.2	0\\
75.21	0\\
75.22	0\\
75.23	0\\
75.24	0\\
75.25	0\\
75.26	0\\
75.27	0\\
75.28	0\\
75.29	0\\
75.3	0\\
75.31	0\\
75.32	0\\
75.33	0\\
75.34	0\\
75.35	0\\
75.36	0\\
75.37	0\\
75.38	0\\
75.39	0\\
75.4	0\\
75.41	0\\
75.42	0\\
75.43	0\\
75.44	0\\
75.45	0\\
75.46	0\\
75.47	0\\
75.48	0\\
75.49	0\\
75.5	0\\
75.51	0\\
75.52	0\\
75.53	0\\
75.54	0\\
75.55	0\\
75.56	0\\
75.57	0\\
75.58	0\\
75.59	0\\
75.6	0\\
75.61	0\\
75.62	0\\
75.63	0\\
75.64	0\\
75.65	0\\
75.66	0\\
75.67	0\\
75.68	0\\
75.69	0\\
75.7	0\\
75.71	0\\
75.72	0\\
75.73	0\\
75.74	0\\
75.75	0\\
75.76	0\\
75.77	0\\
75.78	0\\
75.79	0\\
75.8	0\\
75.81	0\\
75.82	0\\
75.83	0\\
75.84	0\\
75.85	0\\
75.86	0\\
75.87	0\\
75.88	0\\
75.89	0\\
75.9	0\\
75.91	0\\
75.92	0\\
75.93	0\\
75.94	0\\
75.95	0\\
75.96	0\\
75.97	0\\
75.98	0\\
75.99	0\\
76	0\\
76.01	0\\
76.02	0\\
76.03	0\\
76.04	0\\
76.05	0\\
76.06	0\\
76.07	0\\
76.08	0\\
76.09	0\\
76.1	0\\
76.11	0\\
76.12	0\\
76.13	0\\
76.14	0\\
76.15	0\\
76.16	0\\
76.17	0\\
76.18	0\\
76.19	0\\
76.2	0\\
76.21	0\\
76.22	0\\
76.23	0\\
76.24	0\\
76.25	0\\
76.26	0\\
76.27	0\\
76.28	0\\
76.29	0\\
76.3	0\\
76.31	0\\
76.32	0\\
76.33	0\\
76.34	0\\
76.35	0\\
76.36	0\\
76.37	0\\
76.38	0\\
76.39	0\\
76.4	0\\
76.41	0\\
76.42	0\\
76.43	0\\
76.44	0\\
76.45	0\\
76.46	0\\
76.47	0\\
76.48	0\\
76.49	0\\
76.5	0\\
76.51	0\\
76.52	0\\
76.53	0\\
76.54	0\\
76.55	0\\
76.56	0\\
76.57	0\\
76.58	0\\
76.59	0\\
76.6	0\\
76.61	0\\
76.62	0\\
76.63	0\\
76.64	0\\
76.65	0\\
76.66	0\\
76.67	0\\
76.68	0\\
76.69	0\\
76.7	0\\
76.71	0\\
76.72	0\\
76.73	0\\
76.74	0\\
76.75	0\\
76.76	0\\
76.77	0\\
76.78	0\\
76.79	0\\
76.8	0\\
76.81	0\\
76.82	0\\
76.83	0\\
76.84	0\\
76.85	0\\
76.86	0\\
76.87	0\\
76.88	0\\
76.89	0\\
76.9	0\\
76.91	0\\
76.92	0\\
76.93	0\\
76.94	0\\
76.95	0\\
76.96	0\\
76.97	0\\
76.98	0\\
76.99	0\\
77	0\\
77.01	0\\
77.02	0\\
77.03	0\\
77.04	0\\
77.05	0\\
77.06	0\\
77.07	0\\
77.08	0\\
77.09	0\\
77.1	0\\
77.11	0\\
77.12	0\\
77.13	0\\
77.14	0\\
77.15	0\\
77.16	0\\
77.17	0\\
77.18	0\\
77.19	0\\
77.2	0\\
77.21	0\\
77.22	0\\
77.23	0\\
77.24	0\\
77.25	0\\
77.26	0\\
77.27	0\\
77.28	0\\
77.29	0\\
77.3	0\\
77.31	0\\
77.32	0\\
77.33	0\\
77.34	0\\
77.35	0\\
77.36	0\\
77.37	0\\
77.38	0\\
77.39	0\\
77.4	0\\
77.41	0\\
77.42	0\\
77.43	0\\
77.44	0\\
77.45	0\\
77.46	0\\
77.47	0\\
77.48	0\\
77.49	0\\
77.5	0\\
77.51	0\\
77.52	0\\
77.53	0\\
77.54	0\\
77.55	0\\
77.56	0\\
77.57	0\\
77.58	0\\
77.59	0\\
77.6	0\\
77.61	0\\
77.62	0\\
77.63	0\\
77.64	0\\
77.65	0\\
77.66	0\\
77.67	0\\
77.68	0\\
77.69	0\\
77.7	0\\
77.71	0\\
77.72	0\\
77.73	0\\
77.74	0\\
77.75	0\\
77.76	0\\
77.77	0\\
77.78	0\\
77.79	0\\
77.8	0\\
77.81	0\\
77.82	0\\
77.83	0\\
77.84	0\\
77.85	0\\
77.86	0\\
77.87	0\\
77.88	0\\
77.89	0\\
77.9	0\\
77.91	0\\
77.92	0\\
77.93	0\\
77.94	0\\
77.95	0\\
77.96	0\\
77.97	0\\
77.98	0\\
77.99	0\\
78	0\\
78.01	0\\
78.02	0\\
78.03	0\\
78.04	0\\
78.05	0\\
78.06	0\\
78.07	0\\
78.08	0\\
78.09	0\\
78.1	0\\
78.11	0\\
78.12	0\\
78.13	0\\
78.14	0\\
78.15	0\\
78.16	0\\
78.17	0\\
78.18	0\\
78.19	0\\
78.2	0\\
78.21	0\\
78.22	0\\
78.23	0\\
78.24	0\\
78.25	0\\
78.26	0\\
78.27	0\\
78.28	0\\
78.29	0\\
78.3	0\\
78.31	0\\
78.32	0\\
78.33	0\\
78.34	0\\
78.35	0\\
78.36	0\\
78.37	0\\
78.38	0\\
78.39	0\\
78.4	0\\
78.41	0\\
78.42	0\\
78.43	0\\
78.44	0\\
78.45	0\\
78.46	0\\
78.47	0\\
78.48	0\\
78.49	0\\
78.5	0\\
78.51	0\\
78.52	0\\
78.53	0\\
78.54	0\\
78.55	0\\
78.56	0\\
78.57	0\\
78.58	0\\
78.59	0\\
78.6	0\\
78.61	0\\
78.62	0\\
78.63	0\\
78.64	0\\
78.65	0\\
78.66	0\\
78.67	0\\
78.68	0\\
78.69	0\\
78.7	0\\
78.71	0\\
78.72	0\\
78.73	0\\
78.74	0\\
78.75	0\\
78.76	0\\
78.77	0\\
78.78	0\\
78.79	0\\
78.8	0\\
78.81	0\\
78.82	0\\
78.83	0\\
78.84	0\\
78.85	0\\
78.86	0\\
78.87	0\\
78.88	0\\
78.89	0\\
78.9	0\\
78.91	0\\
78.92	0\\
78.93	0\\
78.94	0\\
78.95	0\\
78.96	0\\
78.97	0\\
78.98	0\\
78.99	0\\
79	0\\
79.01	0\\
79.02	0\\
79.03	0\\
79.04	0\\
79.05	0\\
79.06	0\\
79.07	0\\
79.08	0\\
79.09	0\\
79.1	0\\
79.11	0\\
79.12	0\\
79.13	0\\
79.14	0\\
79.15	0\\
79.16	0\\
79.17	0\\
79.18	0\\
79.19	0\\
79.2	0\\
79.21	0\\
79.22	0\\
79.23	0\\
79.24	0\\
79.25	0\\
79.26	0\\
79.27	0\\
79.28	0\\
79.29	0\\
79.3	0\\
79.31	0\\
79.32	0\\
79.33	0\\
79.34	0\\
79.35	0\\
79.36	0\\
79.37	0\\
79.38	0\\
79.39	0\\
79.4	0\\
79.41	0\\
79.42	0\\
79.43	0\\
79.44	0\\
79.45	0\\
79.46	0\\
79.47	0\\
79.48	0\\
79.49	0\\
79.5	0\\
79.51	0\\
79.52	0\\
79.53	0\\
79.54	0\\
79.55	0\\
79.56	0\\
79.57	0\\
79.58	0\\
79.59	0\\
79.6	0\\
79.61	0\\
79.62	0\\
79.63	0\\
79.64	0\\
79.65	0\\
79.66	0\\
79.67	0\\
79.68	0\\
79.69	0\\
79.7	0\\
79.71	0\\
79.72	0\\
79.73	0\\
79.74	0\\
79.75	0\\
79.76	0\\
79.77	0\\
79.78	0\\
79.79	0\\
79.8	0\\
79.81	0\\
79.82	0\\
79.83	0\\
79.84	0\\
79.85	0\\
79.86	0\\
79.87	0\\
79.88	0\\
79.89	0\\
79.9	0\\
79.91	0\\
79.92	0\\
79.93	0\\
79.94	0\\
79.95	0\\
79.96	0\\
79.97	0\\
79.98	0\\
79.99	0\\
80	0\\
80.01	0\\
};
\addplot [color=green,solid]
  table[row sep=crcr]{%
80.01	0\\
80.02	0\\
80.03	0\\
80.04	0\\
80.05	0\\
80.06	0\\
80.07	0\\
80.08	0\\
80.09	0\\
80.1	0\\
80.11	0\\
80.12	0\\
80.13	0\\
80.14	0\\
80.15	0\\
80.16	0\\
80.17	0\\
80.18	0\\
80.19	0\\
80.2	0\\
80.21	0\\
80.22	0\\
80.23	0\\
80.24	0\\
80.25	0\\
80.26	0\\
80.27	0\\
80.28	0\\
80.29	0\\
80.3	0\\
80.31	0\\
80.32	0\\
80.33	0\\
80.34	0\\
80.35	0\\
80.36	0\\
80.37	0\\
80.38	0\\
80.39	0\\
80.4	0\\
80.41	0\\
80.42	0\\
80.43	0\\
80.44	0\\
80.45	0\\
80.46	0\\
80.47	0\\
80.48	0\\
80.49	0\\
80.5	0\\
80.51	0\\
80.52	0\\
80.53	0\\
80.54	0\\
80.55	0\\
80.56	0\\
80.57	0\\
80.58	0\\
80.59	0\\
80.6	0\\
80.61	0\\
80.62	0\\
80.63	0\\
80.64	0\\
80.65	0\\
80.66	0\\
80.67	0\\
80.68	0\\
80.69	0\\
80.7	0\\
80.71	0\\
80.72	0\\
80.73	0\\
80.74	0\\
80.75	0\\
80.76	0\\
80.77	0\\
80.78	0\\
80.79	0\\
80.8	0\\
80.81	0\\
80.82	0\\
80.83	0\\
80.84	0\\
80.85	0\\
80.86	0\\
80.87	0\\
80.88	0\\
80.89	0\\
80.9	0\\
80.91	0\\
80.92	0\\
80.93	0\\
80.94	0\\
80.95	0\\
80.96	0\\
80.97	0\\
80.98	0\\
80.99	0\\
81	0\\
81.01	0\\
81.02	0\\
81.03	0\\
81.04	0\\
81.05	0\\
81.06	0\\
81.07	0\\
81.08	0\\
81.09	0\\
81.1	0\\
81.11	0\\
81.12	0\\
81.13	0\\
81.14	0\\
81.15	0\\
81.16	0\\
81.17	0\\
81.18	0\\
81.19	0\\
81.2	0\\
81.21	0\\
81.22	0\\
81.23	0\\
81.24	0\\
81.25	0\\
81.26	0\\
81.27	0\\
81.28	0\\
81.29	0\\
81.3	0\\
81.31	0\\
81.32	0\\
81.33	0\\
81.34	0\\
81.35	0\\
81.36	0\\
81.37	0\\
81.38	0\\
81.39	0\\
81.4	0\\
81.41	0\\
81.42	0\\
81.43	0\\
81.44	0\\
81.45	0\\
81.46	0\\
81.47	0\\
81.48	0\\
81.49	0\\
81.5	0\\
81.51	0\\
81.52	0\\
81.53	0\\
81.54	0\\
81.55	0\\
81.56	0\\
81.57	0\\
81.58	0\\
81.59	0\\
81.6	0\\
81.61	0\\
81.62	0\\
81.63	0\\
81.64	0\\
81.65	0\\
81.66	0\\
81.67	0\\
81.68	0\\
81.69	0\\
81.7	0\\
81.71	0\\
81.72	0\\
81.73	0\\
81.74	0\\
81.75	0\\
81.76	0\\
81.77	0\\
81.78	0\\
81.79	0\\
81.8	0\\
81.81	0\\
81.82	0\\
81.83	0\\
81.84	0\\
81.85	0\\
81.86	0\\
81.87	0\\
81.88	0\\
81.89	0\\
81.9	0\\
81.91	0\\
81.92	0\\
81.93	0\\
81.94	0\\
81.95	0\\
81.96	0\\
81.97	0\\
81.98	0\\
81.99	0\\
82	0\\
82.01	0\\
82.02	0\\
82.03	0\\
82.04	0\\
82.05	0\\
82.06	0\\
82.07	0\\
82.08	0\\
82.09	0\\
82.1	0\\
82.11	0\\
82.12	0\\
82.13	0\\
82.14	0\\
82.15	0\\
82.16	0\\
82.17	0\\
82.18	0\\
82.19	0\\
82.2	0\\
82.21	0\\
82.22	0\\
82.23	0\\
82.24	0\\
82.25	0\\
82.26	0\\
82.27	0\\
82.28	0\\
82.29	0\\
82.3	0\\
82.31	0\\
82.32	0\\
82.33	0\\
82.34	0\\
82.35	0\\
82.36	0\\
82.37	0\\
82.38	0\\
82.39	0\\
82.4	0\\
82.41	0\\
82.42	0\\
82.43	0\\
82.44	0\\
82.45	0\\
82.46	0\\
82.47	0\\
82.48	0\\
82.49	0\\
82.5	0\\
82.51	0\\
82.52	0\\
82.53	0\\
82.54	0\\
82.55	0\\
82.56	0\\
82.57	0\\
82.58	0\\
82.59	0\\
82.6	0\\
82.61	0\\
82.62	0\\
82.63	0\\
82.64	0\\
82.65	0\\
82.66	0\\
82.67	0\\
82.68	0\\
82.69	0\\
82.7	0\\
82.71	0\\
82.72	0\\
82.73	0\\
82.74	0\\
82.75	0\\
82.76	0\\
82.77	0\\
82.78	0\\
82.79	0\\
82.8	0\\
82.81	0\\
82.82	0\\
82.83	0\\
82.84	0\\
82.85	0\\
82.86	0\\
82.87	0\\
82.88	0\\
82.89	0\\
82.9	0\\
82.91	0\\
82.92	0\\
82.93	0\\
82.94	0\\
82.95	0\\
82.96	0\\
82.97	0\\
82.98	0\\
82.99	0\\
83	0\\
83.01	0\\
83.02	0\\
83.03	0\\
83.04	0\\
83.05	0\\
83.06	0\\
83.07	0\\
83.08	0\\
83.09	0\\
83.1	0\\
83.11	0\\
83.12	0\\
83.13	0\\
83.14	0\\
83.15	0\\
83.16	0\\
83.17	0\\
83.18	0\\
83.19	0\\
83.2	0\\
83.21	0\\
83.22	0\\
83.23	0\\
83.24	0\\
83.25	0\\
83.26	0\\
83.27	0\\
83.28	0\\
83.29	0\\
83.3	0\\
83.31	0\\
83.32	0\\
83.33	0\\
83.34	0\\
83.35	0\\
83.36	0\\
83.37	0\\
83.38	0\\
83.39	0\\
83.4	0\\
83.41	0\\
83.42	0\\
83.43	0\\
83.44	0\\
83.45	0\\
83.46	0\\
83.47	0\\
83.48	0\\
83.49	0\\
83.5	0\\
83.51	0\\
83.52	0\\
83.53	0\\
83.54	0\\
83.55	0\\
83.56	0\\
83.57	0\\
83.58	0\\
83.59	0\\
83.6	0\\
83.61	0\\
83.62	0\\
83.63	0\\
83.64	0\\
83.65	0\\
83.66	0\\
83.67	0\\
83.68	0\\
83.69	0\\
83.7	0\\
83.71	0\\
83.72	0\\
83.73	0\\
83.74	0\\
83.75	0\\
83.76	0\\
83.77	0\\
83.78	0\\
83.79	0\\
83.8	0\\
83.81	0\\
83.82	0\\
83.83	0\\
83.84	0\\
83.85	0\\
83.86	0\\
83.87	0\\
83.88	0\\
83.89	0\\
83.9	0\\
83.91	0\\
83.92	0\\
83.93	0\\
83.94	0\\
83.95	0\\
83.96	0\\
83.97	0\\
83.98	0\\
83.99	0\\
84	0\\
84.01	0\\
84.02	0\\
84.03	0\\
84.04	0\\
84.05	0\\
84.06	0\\
84.07	0\\
84.08	0\\
84.09	0\\
84.1	0\\
84.11	0\\
84.12	0\\
84.13	0\\
84.14	0\\
84.15	0\\
84.16	0\\
84.17	0\\
84.18	0\\
84.19	0\\
84.2	0\\
84.21	0\\
84.22	0\\
84.23	0\\
84.24	0\\
84.25	0\\
84.26	0\\
84.27	0\\
84.28	0\\
84.29	0\\
84.3	0\\
84.31	0\\
84.32	0\\
84.33	0\\
84.34	0\\
84.35	0\\
84.36	0\\
84.37	0\\
84.38	0\\
84.39	0\\
84.4	0\\
84.41	0\\
84.42	0\\
84.43	0\\
84.44	0\\
84.45	0\\
84.46	0\\
84.47	0\\
84.48	0\\
84.49	0\\
84.5	0\\
84.51	0\\
84.52	0\\
84.53	0\\
84.54	0\\
84.55	0\\
84.56	0\\
84.57	0\\
84.58	0\\
84.59	0\\
84.6	0\\
84.61	0\\
84.62	0\\
84.63	0\\
84.64	0\\
84.65	0\\
84.66	0\\
84.67	0\\
84.68	0\\
84.69	0\\
84.7	0\\
84.71	0\\
84.72	0\\
84.73	0\\
84.74	0\\
84.75	0\\
84.76	0\\
84.77	0\\
84.78	0\\
84.79	0\\
84.8	0\\
84.81	0\\
84.82	0\\
84.83	0\\
84.84	0\\
84.85	0\\
84.86	0\\
84.87	0\\
84.88	0\\
84.89	0\\
84.9	0\\
84.91	0\\
84.92	0\\
84.93	0\\
84.94	0\\
84.95	0\\
84.96	0\\
84.97	0\\
84.98	0\\
84.99	0\\
85	0\\
85.01	0\\
85.02	0\\
85.03	0\\
85.04	0\\
85.05	0\\
85.06	0\\
85.07	0\\
85.08	0\\
85.09	0\\
85.1	0\\
85.11	0\\
85.12	0\\
85.13	0\\
85.14	0\\
85.15	0\\
85.16	0\\
85.17	0\\
85.18	0\\
85.19	0\\
85.2	0\\
85.21	0\\
85.22	0\\
85.23	0\\
85.24	0\\
85.25	0\\
85.26	0\\
85.27	0\\
85.28	0\\
85.29	0\\
85.3	0\\
85.31	0\\
85.32	0\\
85.33	0\\
85.34	0\\
85.35	0\\
85.36	0\\
85.37	0\\
85.38	0\\
85.39	0\\
85.4	0\\
85.41	0\\
85.42	0\\
85.43	0\\
85.44	0\\
85.45	0\\
85.46	0\\
85.47	0\\
85.48	0\\
85.49	0\\
85.5	0\\
85.51	0\\
85.52	0\\
85.53	0\\
85.54	0\\
85.55	0\\
85.56	0\\
85.57	0\\
85.58	0\\
85.59	0\\
85.6	0\\
85.61	0\\
85.62	0\\
85.63	0\\
85.64	0\\
85.65	0\\
85.66	0\\
85.67	0\\
85.68	0\\
85.69	0\\
85.7	0\\
85.71	0\\
85.72	0\\
85.73	0\\
85.74	0\\
85.75	0\\
85.76	0\\
85.77	0\\
85.78	0\\
85.79	0\\
85.8	0\\
85.81	0\\
85.82	0\\
85.83	0\\
85.84	0\\
85.85	0\\
85.86	0\\
85.87	0\\
85.88	0\\
85.89	0\\
85.9	0\\
85.91	0\\
85.92	0\\
85.93	0\\
85.94	0\\
85.95	0\\
85.96	0\\
85.97	0\\
85.98	0\\
85.99	0\\
86	0\\
86.01	0\\
86.02	0\\
86.03	0\\
86.04	0\\
86.05	0\\
86.06	0\\
86.07	0\\
86.08	0\\
86.09	0\\
86.1	0\\
86.11	0\\
86.12	0\\
86.13	0\\
86.14	0\\
86.15	0\\
86.16	0\\
86.17	0\\
86.18	0\\
86.19	0\\
86.2	0\\
86.21	0\\
86.22	0\\
86.23	0\\
86.24	0\\
86.25	0\\
86.26	0\\
86.27	0\\
86.28	0\\
86.29	0\\
86.3	0\\
86.31	0\\
86.32	0\\
86.33	0\\
86.34	0\\
86.35	0\\
86.36	0\\
86.37	0\\
86.38	0\\
86.39	0\\
86.4	0\\
86.41	0\\
86.42	0\\
86.43	0\\
86.44	0\\
86.45	0\\
86.46	0\\
86.47	0\\
86.48	0\\
86.49	0\\
86.5	0\\
86.51	0\\
86.52	0\\
86.53	0\\
86.54	0\\
86.55	0\\
86.56	0\\
86.57	0\\
86.58	0\\
86.59	0\\
86.6	0\\
86.61	0\\
86.62	0\\
86.63	0\\
86.64	0\\
86.65	0\\
86.66	0\\
86.67	0\\
86.68	0\\
86.69	0\\
86.7	0\\
86.71	0\\
86.72	0\\
86.73	0\\
86.74	0\\
86.75	0\\
86.76	0\\
86.77	0\\
86.78	0\\
86.79	0\\
86.8	0\\
86.81	0\\
86.82	0\\
86.83	0\\
86.84	0\\
86.85	0\\
86.86	0\\
86.87	0\\
86.88	0\\
86.89	0\\
86.9	0\\
86.91	0\\
86.92	0\\
86.93	0\\
86.94	0\\
86.95	0\\
86.96	0\\
86.97	0\\
86.98	0\\
86.99	0\\
87	0\\
87.01	0\\
87.02	0\\
87.03	0\\
87.04	0\\
87.05	0\\
87.06	0\\
87.07	0\\
87.08	0\\
87.09	0\\
87.1	0\\
87.11	0\\
87.12	0\\
87.13	0\\
87.14	0\\
87.15	0\\
87.16	0\\
87.17	0\\
87.18	0\\
87.19	0\\
87.2	0\\
87.21	0\\
87.22	0\\
87.23	0\\
87.24	0\\
87.25	0\\
87.26	0\\
87.27	0\\
87.28	0\\
87.29	0\\
87.3	0\\
87.31	0\\
87.32	0\\
87.33	0\\
87.34	0\\
87.35	0\\
87.36	0\\
87.37	0\\
87.38	0\\
87.39	0\\
87.4	0\\
87.41	0\\
87.42	0\\
87.43	0\\
87.44	0\\
87.45	0\\
87.46	0\\
87.47	0\\
87.48	0\\
87.49	0\\
87.5	0\\
87.51	0\\
87.52	0\\
87.53	0\\
87.54	0\\
87.55	0\\
87.56	0\\
87.57	0\\
87.58	0\\
87.59	0\\
87.6	0\\
87.61	0\\
87.62	0\\
87.63	0\\
87.64	0\\
87.65	0\\
87.66	0\\
87.67	0\\
87.68	0\\
87.69	0\\
87.7	0\\
87.71	0\\
87.72	0\\
87.73	0\\
87.74	0\\
87.75	0\\
87.76	0\\
87.77	0\\
87.78	0\\
87.79	0\\
87.8	0\\
87.81	0\\
87.82	0\\
87.83	0\\
87.84	0\\
87.85	0\\
87.86	0\\
87.87	0\\
87.88	0\\
87.89	0\\
87.9	0\\
87.91	0\\
87.92	0\\
87.93	0\\
87.94	0\\
87.95	0\\
87.96	0\\
87.97	0\\
87.98	0\\
87.99	0\\
88	0\\
88.01	0\\
88.02	0\\
88.03	0\\
88.04	0\\
88.05	0\\
88.06	0\\
88.07	0\\
88.08	0\\
88.09	0\\
88.1	0\\
88.11	0\\
88.12	0\\
88.13	0\\
88.14	0\\
88.15	0\\
88.16	0\\
88.17	0\\
88.18	0\\
88.19	0\\
88.2	0\\
88.21	0\\
88.22	0\\
88.23	0\\
88.24	0\\
88.25	0\\
88.26	0\\
88.27	0\\
88.28	0\\
88.29	0\\
88.3	0\\
88.31	0\\
88.32	0\\
88.33	0\\
88.34	0\\
88.35	0\\
88.36	0\\
88.37	0\\
88.38	0\\
88.39	0\\
88.4	0\\
88.41	0\\
88.42	0\\
88.43	0\\
88.44	0\\
88.45	0\\
88.46	0\\
88.47	0\\
88.48	0\\
88.49	0\\
88.5	0\\
88.51	0\\
88.52	0\\
88.53	0\\
88.54	0\\
88.55	0\\
88.56	0\\
88.57	0\\
88.58	0\\
88.59	0\\
88.6	0\\
88.61	0\\
88.62	0\\
88.63	0\\
88.64	0\\
88.65	0\\
88.66	0\\
88.67	0\\
88.68	0\\
88.69	0\\
88.7	0\\
88.71	0\\
88.72	0\\
88.73	0\\
88.74	0\\
88.75	0\\
88.76	0\\
88.77	0\\
88.78	0\\
88.79	0\\
88.8	0\\
88.81	0\\
88.82	0\\
88.83	0\\
88.84	0\\
88.85	0\\
88.86	0\\
88.87	0\\
88.88	0\\
88.89	0\\
88.9	0\\
88.91	0\\
88.92	0\\
88.93	0\\
88.94	0\\
88.95	0\\
88.96	0\\
88.97	0\\
88.98	0\\
88.99	0\\
89	0\\
89.01	0\\
89.02	0\\
89.03	0\\
89.04	0\\
89.05	0\\
89.06	0\\
89.07	0\\
89.08	0\\
89.09	0\\
89.1	0\\
89.11	0\\
89.12	0\\
89.13	0\\
89.14	0\\
89.15	0\\
89.16	0\\
89.17	0\\
89.18	0\\
89.19	0\\
89.2	0\\
89.21	0\\
89.22	0\\
89.23	0\\
89.24	0\\
89.25	0\\
89.26	0\\
89.27	0\\
89.28	0\\
89.29	0\\
89.3	0\\
89.31	0\\
89.32	0\\
89.33	0\\
89.34	0\\
89.35	0\\
89.36	0\\
89.37	0\\
89.38	0\\
89.39	0\\
89.4	0\\
89.41	0\\
89.42	0\\
89.43	0\\
89.44	0\\
89.45	0\\
89.46	0\\
89.47	0\\
89.48	0\\
89.49	0\\
89.5	0\\
89.51	0\\
89.52	0\\
89.53	0\\
89.54	0\\
89.55	0\\
89.56	0\\
89.57	0\\
89.58	0\\
89.59	0\\
89.6	0\\
89.61	0\\
89.62	0\\
89.63	0\\
89.64	0\\
89.65	0\\
89.66	0\\
89.67	0\\
89.68	0\\
89.69	0\\
89.7	0\\
89.71	0\\
89.72	0\\
89.73	0\\
89.74	0\\
89.75	0\\
89.76	0\\
89.77	0\\
89.78	0\\
89.79	0\\
89.8	0\\
89.81	0\\
89.82	0\\
89.83	0\\
89.84	0\\
89.85	0\\
89.86	0\\
89.87	0\\
89.88	0\\
89.89	0\\
89.9	0\\
89.91	0\\
89.92	0\\
89.93	0\\
89.94	0\\
89.95	0\\
89.96	0\\
89.97	0\\
89.98	0\\
89.99	0\\
90	0\\
90.01	0\\
90.02	0\\
90.03	0\\
90.04	0\\
90.05	0\\
90.06	0\\
90.07	0\\
90.08	0\\
90.09	0\\
90.1	0\\
90.11	0\\
90.12	0\\
90.13	0\\
90.14	0\\
90.15	0\\
90.16	0\\
90.17	0\\
90.18	0\\
90.19	0\\
90.2	0\\
90.21	0\\
90.22	0\\
90.23	0\\
90.24	0\\
90.25	0\\
90.26	0\\
90.27	0\\
90.28	0\\
90.29	0\\
90.3	0\\
90.31	0\\
90.32	0\\
90.33	0\\
90.34	0\\
90.35	0\\
90.36	0\\
90.37	0\\
90.38	0\\
90.39	0\\
90.4	0\\
90.41	0\\
90.42	0\\
90.43	0\\
90.44	0\\
90.45	0\\
90.46	0\\
90.47	0\\
90.48	0\\
90.49	0\\
90.5	0\\
90.51	0\\
90.52	0\\
90.53	0\\
90.54	0\\
90.55	0\\
90.56	0\\
90.57	0\\
90.58	0\\
90.59	0\\
90.6	0\\
90.61	0\\
90.62	0\\
90.63	0\\
90.64	0\\
90.65	0\\
90.66	0\\
90.67	0\\
90.68	0\\
90.69	0\\
90.7	0\\
90.71	0\\
90.72	0\\
90.73	0\\
90.74	0\\
90.75	0\\
90.76	0\\
90.77	0\\
90.78	0\\
90.79	0\\
90.8	0\\
90.81	0\\
90.82	0\\
90.83	0\\
90.84	0\\
90.85	0\\
90.86	0\\
90.87	0\\
90.88	0\\
90.89	0\\
90.9	0\\
90.91	0\\
90.92	0\\
90.93	0\\
90.94	0\\
90.95	0\\
90.96	0\\
90.97	0\\
90.98	0\\
90.99	0\\
91	0\\
91.01	0\\
91.02	0\\
91.03	0\\
91.04	0\\
91.05	0\\
91.06	0\\
91.07	0\\
91.08	0\\
91.09	0\\
91.1	0\\
91.11	0\\
91.12	0\\
91.13	0\\
91.14	0\\
91.15	0\\
91.16	0\\
91.17	0\\
91.18	0\\
91.19	0\\
91.2	0\\
91.21	0\\
91.22	0\\
91.23	0\\
91.24	0\\
91.25	0\\
91.26	0\\
91.27	0\\
91.28	0\\
91.29	0\\
91.3	0\\
91.31	0\\
91.32	0\\
91.33	0\\
91.34	0\\
91.35	0\\
91.36	0\\
91.37	0\\
91.38	0\\
91.39	0\\
91.4	0\\
91.41	0\\
91.42	0\\
91.43	0\\
91.44	0\\
91.45	0\\
91.46	0\\
91.47	0\\
91.48	0\\
91.49	0\\
91.5	0\\
91.51	0\\
91.52	0\\
91.53	0\\
91.54	0\\
91.55	0\\
91.56	0\\
91.57	0\\
91.58	0\\
91.59	0\\
91.6	0\\
91.61	0\\
91.62	0\\
91.63	0\\
91.64	0\\
91.65	0\\
91.66	0\\
91.67	0\\
91.68	0\\
91.69	0\\
91.7	0\\
91.71	0\\
91.72	0\\
91.73	0\\
91.74	0\\
91.75	0\\
91.76	0\\
91.77	0\\
91.78	0\\
91.79	0\\
91.8	0\\
91.81	0\\
91.82	0\\
91.83	0\\
91.84	0\\
91.85	0\\
91.86	0\\
91.87	0\\
91.88	0\\
91.89	0\\
91.9	0\\
91.91	0\\
91.92	0\\
91.93	0\\
91.94	0\\
91.95	0\\
91.96	0\\
91.97	0\\
91.98	0\\
91.99	0\\
92	0\\
92.01	0\\
92.02	0\\
92.03	0\\
92.04	0\\
92.05	0\\
92.06	0\\
92.07	0\\
92.08	0\\
92.09	0\\
92.1	0\\
92.11	0\\
92.12	0\\
92.13	0\\
92.14	0\\
92.15	0\\
92.16	0\\
92.17	0\\
92.18	0\\
92.19	0\\
92.2	0\\
92.21	0\\
92.22	0\\
92.23	0\\
92.24	0\\
92.25	0\\
92.26	0\\
92.27	0\\
92.28	0\\
92.29	0\\
92.3	0\\
92.31	0\\
92.32	0\\
92.33	0\\
92.34	0\\
92.35	0\\
92.36	0\\
92.37	0\\
92.38	0\\
92.39	0\\
92.4	0\\
92.41	0\\
92.42	0\\
92.43	0\\
92.44	0\\
92.45	0\\
92.46	0\\
92.47	0\\
92.48	0\\
92.49	0\\
92.5	0\\
92.51	0\\
92.52	0\\
92.53	0\\
92.54	0\\
92.55	0\\
92.56	0\\
92.57	0\\
92.58	0\\
92.59	0\\
92.6	0\\
92.61	0\\
92.62	0\\
92.63	0\\
92.64	0\\
92.65	0\\
92.66	0\\
92.67	0\\
92.68	0\\
92.69	0\\
92.7	0\\
92.71	0\\
92.72	0\\
92.73	0\\
92.74	0\\
92.75	0\\
92.76	0\\
92.77	0\\
92.78	0\\
92.79	0\\
92.8	0\\
92.81	0\\
92.82	0\\
92.83	0\\
92.84	0\\
92.85	0\\
92.86	0\\
92.87	0\\
92.88	0\\
92.89	0\\
92.9	0\\
92.91	0\\
92.92	0\\
92.93	0\\
92.94	0\\
92.95	0\\
92.96	0\\
92.97	0\\
92.98	0\\
92.99	0\\
93	0\\
93.01	0\\
93.02	0\\
93.03	0\\
93.04	0\\
93.05	0\\
93.06	0\\
93.07	0\\
93.08	0\\
93.09	0\\
93.1	0\\
93.11	0\\
93.12	0\\
93.13	0\\
93.14	0\\
93.15	0\\
93.16	0\\
93.17	0\\
93.18	0\\
93.19	0\\
93.2	0\\
93.21	0\\
93.22	0\\
93.23	0\\
93.24	0\\
93.25	0\\
93.26	0\\
93.27	0\\
93.28	0\\
93.29	0\\
93.3	0\\
93.31	0\\
93.32	0\\
93.33	0\\
93.34	0\\
93.35	0\\
93.36	0\\
93.37	0\\
93.38	0\\
93.39	0\\
93.4	0\\
93.41	0\\
93.42	0\\
93.43	0\\
93.44	0\\
93.45	0\\
93.46	0\\
93.47	0\\
93.48	0\\
93.49	0\\
93.5	0\\
93.51	0\\
93.52	0\\
93.53	0\\
93.54	0\\
93.55	0\\
93.56	0\\
93.57	0\\
93.58	0\\
93.59	0\\
93.6	0\\
93.61	0\\
93.62	0\\
93.63	0\\
93.64	0\\
93.65	0\\
93.66	0\\
93.67	0\\
93.68	0\\
93.69	0\\
93.7	0\\
93.71	0\\
93.72	0\\
93.73	0\\
93.74	0\\
93.75	0\\
93.76	0\\
93.77	0\\
93.78	0\\
93.79	0\\
93.8	0\\
93.81	0\\
93.82	0\\
93.83	0\\
93.84	0\\
93.85	0\\
93.86	0\\
93.87	0\\
93.88	0\\
93.89	0\\
93.9	0\\
93.91	0\\
93.92	0\\
93.93	0\\
93.94	0\\
93.95	0\\
93.96	0\\
93.97	0\\
93.98	0\\
93.99	0\\
94	0\\
94.01	0\\
94.02	0\\
94.03	0\\
94.04	0\\
94.05	0\\
94.06	0\\
94.07	0\\
94.08	0\\
94.09	0\\
94.1	0\\
94.11	0\\
94.12	0\\
94.13	0\\
94.14	0\\
94.15	0\\
94.16	0\\
94.17	0\\
94.18	0\\
94.19	0\\
94.2	0\\
94.21	0\\
94.22	0\\
94.23	0\\
94.24	0\\
94.25	0\\
94.26	0\\
94.27	0\\
94.28	0\\
94.29	0\\
94.3	0\\
94.31	0\\
94.32	0\\
94.33	0\\
94.34	0\\
94.35	0\\
94.36	0\\
94.37	0\\
94.38	0\\
94.39	0\\
94.4	0\\
94.41	0\\
94.42	0\\
94.43	0\\
94.44	0\\
94.45	0\\
94.46	0\\
94.47	0\\
94.48	0\\
94.49	0\\
94.5	0\\
94.51	0\\
94.52	0\\
94.53	0\\
94.54	0\\
94.55	0\\
94.56	0\\
94.57	0\\
94.58	0\\
94.59	0\\
94.6	0\\
94.61	0\\
94.62	0\\
94.63	0\\
94.64	0\\
94.65	0\\
94.66	0\\
94.67	0\\
94.68	0\\
94.69	0\\
94.7	0\\
94.71	0\\
94.72	0\\
94.73	0\\
94.74	0\\
94.75	0\\
94.76	0\\
94.77	0\\
94.78	0\\
94.79	0\\
94.8	0\\
94.81	0\\
94.82	0\\
94.83	0\\
94.84	0\\
94.85	0\\
94.86	0\\
94.87	0\\
94.88	0\\
94.89	0\\
94.9	0\\
94.91	0\\
94.92	0\\
94.93	0\\
94.94	0\\
94.95	0\\
94.96	0\\
94.97	0\\
94.98	0\\
94.99	0\\
95	0\\
95.01	0\\
95.02	0\\
95.03	0\\
95.04	0\\
95.05	0\\
95.06	0\\
95.07	0\\
95.08	0\\
95.09	0\\
95.1	0\\
95.11	0\\
95.12	0\\
95.13	0\\
95.14	0\\
95.15	0\\
95.16	0\\
95.17	0\\
95.18	0\\
95.19	0\\
95.2	0\\
95.21	0\\
95.22	0\\
95.23	0\\
95.24	0\\
95.25	0\\
95.26	0\\
95.27	0\\
95.28	0\\
95.29	0\\
95.3	0\\
95.31	0\\
95.32	0\\
95.33	0\\
95.34	0\\
95.35	0\\
95.36	0\\
95.37	0\\
95.38	0\\
95.39	0\\
95.4	0\\
95.41	0\\
95.42	0\\
95.43	0\\
95.44	0\\
95.45	0\\
95.46	0\\
95.47	0\\
95.48	0\\
95.49	0\\
95.5	0\\
95.51	0\\
95.52	0\\
95.53	0\\
95.54	0\\
95.55	0\\
95.56	0\\
95.57	0\\
95.58	0\\
95.59	0\\
95.6	0\\
95.61	0\\
95.62	0\\
95.63	0\\
95.64	0\\
95.65	0\\
95.66	0\\
95.67	0\\
95.68	0\\
95.69	0\\
95.7	0\\
95.71	0\\
95.72	0\\
95.73	0\\
95.74	0\\
95.75	0\\
95.76	0\\
95.77	0\\
95.78	0\\
95.79	0\\
95.8	0\\
95.81	0\\
95.82	0\\
95.83	0\\
95.84	0\\
95.85	0\\
95.86	0\\
95.87	0\\
95.88	0\\
95.89	0\\
95.9	0\\
95.91	0\\
95.92	0\\
95.93	0\\
95.94	0\\
95.95	0\\
95.96	0\\
95.97	0\\
95.98	0\\
95.99	0\\
96	0\\
96.01	0\\
96.02	0\\
96.03	0\\
96.04	0\\
96.05	0\\
96.06	0\\
96.07	0\\
96.08	0\\
96.09	0\\
96.1	0\\
96.11	0\\
96.12	0\\
96.13	0\\
96.14	0\\
96.15	0\\
96.16	0\\
96.17	0\\
96.18	0\\
96.19	0\\
96.2	0\\
96.21	0\\
96.22	0\\
96.23	0\\
96.24	0\\
96.25	0\\
96.26	0\\
96.27	0\\
96.28	0\\
96.29	0\\
96.3	0\\
96.31	0\\
96.32	0\\
96.33	0\\
96.34	0\\
96.35	0\\
96.36	0\\
96.37	0\\
96.38	0\\
96.39	0\\
96.4	0\\
96.41	0\\
96.42	0\\
96.43	0\\
96.44	0\\
96.45	0\\
96.46	0\\
96.47	0\\
96.48	0\\
96.49	0\\
96.5	0\\
96.51	0\\
96.52	0\\
96.53	0\\
96.54	0\\
96.55	0\\
96.56	0\\
96.57	0\\
96.58	0\\
96.59	0\\
96.6	0\\
96.61	0\\
96.62	0\\
96.63	0\\
96.64	0\\
96.65	0\\
96.66	0\\
96.67	0\\
96.68	0\\
96.69	0\\
96.7	0\\
96.71	0\\
96.72	0\\
96.73	0\\
96.74	0\\
96.75	0\\
96.76	0\\
96.77	0\\
96.78	0\\
96.79	0\\
96.8	0\\
96.81	0\\
96.82	0\\
96.83	0\\
96.84	0\\
96.85	0\\
96.86	0\\
96.87	0\\
96.88	0\\
96.89	0\\
96.9	0\\
96.91	0\\
96.92	0\\
96.93	0\\
96.94	0\\
96.95	0\\
96.96	0\\
96.97	0\\
96.98	0\\
96.99	0\\
97	0\\
97.01	0\\
97.02	0\\
97.03	0\\
97.04	0\\
97.05	0\\
97.06	0\\
97.07	0\\
97.08	0\\
97.09	0\\
97.1	0\\
97.11	0\\
97.12	0\\
97.13	0\\
97.14	0\\
97.15	0\\
97.16	0\\
97.17	0\\
97.18	0\\
97.19	0\\
97.2	0\\
97.21	0\\
97.22	0\\
97.23	0\\
97.24	0\\
97.25	0\\
97.26	0\\
97.27	0\\
97.28	0\\
97.29	0\\
97.3	0\\
97.31	0\\
97.32	0\\
97.33	0\\
97.34	0\\
97.35	0\\
97.36	0\\
97.37	0\\
97.38	0\\
97.39	0\\
97.4	0\\
97.41	0\\
97.42	0\\
97.43	0\\
97.44	0\\
97.45	0\\
97.46	0\\
97.47	0\\
97.48	0\\
97.49	0\\
97.5	0\\
97.51	0\\
97.52	0\\
97.53	0\\
97.54	0\\
97.55	0\\
97.56	0\\
97.57	0\\
97.58	0\\
97.59	0\\
97.6	0\\
97.61	0\\
97.62	0\\
97.63	0\\
97.64	0\\
97.65	0\\
97.66	0\\
97.67	0\\
97.68	0\\
97.69	0\\
97.7	0\\
97.71	0\\
97.72	0\\
97.73	0\\
97.74	0\\
97.75	0\\
97.76	0\\
97.77	0\\
97.78	0\\
97.79	0\\
97.8	0\\
97.81	0\\
97.82	0\\
97.83	0\\
97.84	0\\
97.85	0\\
97.86	0\\
97.87	0\\
97.88	0\\
97.89	0\\
97.9	0\\
97.91	0\\
97.92	0\\
97.93	0\\
97.94	0\\
97.95	0\\
97.96	0\\
97.97	0\\
97.98	0\\
97.99	5.82904807695447e-05\\
98	0.00015409799186631\\
98.01	0.000250676523664825\\
98.02	0.000348033875184062\\
98.03	0.00044617793610998\\
98.04	0.00054511668815909\\
98.05	0.000644858206471345\\
98.06	0.000745410661033299\\
98.07	0.000813554165158825\\
98.08	0.000831324866783396\\
98.09	0.000849228356317379\\
98.1	0.000867265364232553\\
98.11	0.000885436614948054\\
98.12	0.000903742826422992\\
98.13	0.000922184709734451\\
98.14	0.000940762968640367\\
98.15	0.000959478299126647\\
98.16	0.000978331388938075\\
98.17	0.000997322917092383\\
98.18	0.00101645355337684\\
98.19	0.00103574920928199\\
98.2	0.0010552173123434\\
98.21	0.00107485953100661\\
98.22	0.00109467753629575\\
98.23	0.00111467301549213\\
98.24	0.0011348476722993\\
98.25	0.00115520322700986\\
98.26	0.00117574141667427\\
98.27	0.0011964639952716\\
98.28	0.00121737273388221\\
98.29	0.00123846942086252\\
98.3	0.00125975586202464\\
98.31	0.00128123388081602\\
98.32	0.00130290531850058\\
98.33	0.00132477203434225\\
98.34	0.00134683590579299\\
98.35	0.00136909882868069\\
98.36	0.00139156271739919\\
98.37	0.0014142294935727\\
98.38	0.00143710109677581\\
98.39	0.00146017948510352\\
98.4	0.00148346663534629\\
98.41	0.00150696454316288\\
98.42	0.00153067522325962\\
98.43	0.00155460070958288\\
98.44	0.00157874305550597\\
98.45	0.0016031043340145\\
98.46	0.00162768663821044\\
98.47	0.00165249208284321\\
98.48	0.00167752280288907\\
98.49	0.00170278095374452\\
98.5	0.00172826871142152\\
98.51	0.00175398827274457\\
98.52	0.00177994185554966\\
98.53	0.00180613169888516\\
98.54	0.00183256006304103\\
98.55	0.00185922922837249\\
98.56	0.0018861414970178\\
98.57	0.00191329919310589\\
98.58	0.00194070466296608\\
98.59	0.00196836027533968\\
98.6	0.00199626842159369\\
98.61	0.00202443151593641\\
98.62	0.00205285199563524\\
98.63	0.00208153232123635\\
98.64	0.00211047497678664\\
98.65	0.00213968247005763\\
98.66	0.00216915733277574\\
98.67	0.00219890212085219\\
98.68	0.0022289194146136\\
98.69	0.0022592118190388\\
98.7	0.00228978196399537\\
98.71	0.00232063250447734\\
98.72	0.00235176612084517\\
98.73	0.00238318551906794\\
98.74	0.00241489343096797\\
98.75	0.00244689261446764\\
98.76	0.00247918585383865\\
98.77	0.00251177595995369\\
98.78	0.00254466577054039\\
98.79	0.00257785815043781\\
98.8	0.00261135599185532\\
98.81	0.00264516221463393\\
98.82	0.00267927976651017\\
98.83	0.00271371162338246\\
98.84	0.0027484607678989\\
98.85	0.00278353018490016\\
98.86	0.00281892288730982\\
98.87	0.00285464191639842\\
98.88	0.00289069034204985\\
98.89	0.00292707126303038\\
98.9	0.00296378780726018\\
98.91	0.00300084313208736\\
98.92	0.00303824042456455\\
98.93	0.0030759829017282\\
98.94	0.00311407381088036\\
98.95	0.00315251642987323\\
98.96	0.00319131406739629\\
98.97	0.00323047006326616\\
98.98	0.0032699877887192\\
98.99	0.00330987064670684\\
99	0.00335012207219369\\
99.01	0.00339074553245839\\
99.02	0.00343174452739743\\
99.03	0.00347312258983165\\
99.04	0.00351488328581575\\
99.05	0.00355703021495063\\
99.06	0.00359956701069868\\
99.07	0.00364249734070207\\
99.08	0.00368582490710399\\
99.09	0.00372955344687287\\
99.1	0.00377368673212974\\
99.11	0.0038182285704786\\
99.12	0.00386318280533987\\
99.13	0.00390855331628703\\
99.14	0.00395434401938642\\
99.15	0.00400055886754013\\
99.16	0.00404720185083227\\
99.17	0.00409427699687832\\
99.18	0.00414178837117793\\
99.19	0.00418974007747085\\
99.2	0.00423813625809642\\
99.21	0.00428698109435617\\
99.22	0.00433627880688009\\
99.23	0.00438603365599616\\
99.24	0.00443624994210338\\
99.25	0.0044869320060484\\
99.26	0.00453808422950553\\
99.27	0.00458971103536046\\
99.28	0.00464181688809748\\
99.29	0.00469440629419043\\
99.3	0.0047474838024972\\
99.31	0.00480105400465803\\
99.32	0.00485512153549756\\
99.33	0.00490969107343051\\
99.34	0.00496476734087139\\
99.35	0.00502035510464784\\
99.36	0.00507645917641801\\
99.37	0.00513308441309174\\
99.38	0.00519023571725581\\
99.39	0.00524791803760305\\
99.4	0.0053061363693656\\
99.41	0.00536489575475216\\
99.42	0.00542420128338936\\
99.43	0.0054840580927673\\
99.44	0.00554447136868923\\
99.45	0.00560544634572549\\
99.46	0.00566698830767168\\
99.47	0.00572910258801115\\
99.48	0.0057917945703818\\
99.49	0.00585506968904733\\
99.5	0.00591893342937281\\
99.51	0.00598339132830483\\
99.52	0.00604844897485604\\
99.53	0.00611411201059433\\
99.54	0.00618038613013654\\
99.55	0.00624727708164684\\
99.56	0.00631479066733982\\
99.57	0.00638293274398813\\
99.58	0.00645170922343514\\
99.59	0.00652112607311221\\
99.6	0.00659118931656093\\
99.61	0.00666190501063383\\
99.62	0.00673327925932718\\
99.63	0.00680531822371098\\
99.64	0.00687802812246212\\
99.65	0.00695141523240246\\
99.66	0.007025485889042\\
99.67	0.007100246487127\\
99.68	0.0071757034811933\\
99.69	0.00725186338612484\\
99.7	0.00732873277771724\\
99.71	0.00740631829324684\\
99.72	0.00748462663204484\\
99.73	0.00756366455607695\\
99.74	0.00764343889052842\\
99.75	0.00772395652439445\\
99.76	0.0078052244110762\\
99.77	0.0078872495689823\\
99.78	0.00797003908213601\\
99.79	0.00805360010078805\\
99.8	0.00813793984203508\\
99.81	0.00822306559044401\\
99.82	0.00830898469868217\\
99.83	0.00839570458815322\\
99.84	0.00848323274963918\\
99.85	0.00857157674394827\\
99.86	0.00866074420256893\\
99.87	0.00875074282832986\\
99.88	0.00884158039606625\\
99.89	0.00893326475329224\\
99.9	0.00902580382087961\\
99.91	0.00911920559374282\\
99.92	0.00921347814153047\\
99.93	0.00930862960932319\\
99.94	0.00940466821833801\\
99.95	0.00950160226663938\\
99.96	0.00959944012985672\\
99.97	0.00969819026190876\\
99.98	0.0097978611957346\\
99.99	0.00989846154403157\\
100	0.01\\
};
\addlegendentry{$q=4$};

\end{axis}
\end{tikzpicture}%
  \caption{Continuous Time}
\end{subfigure}%
\hfill%
\begin{subfigure}{.45\linewidth}
  \centering
  \setlength\figureheight{\linewidth} 
  \setlength\figurewidth{\linewidth}
  \tikzsetnextfilename{testdp_dscr_nFPC_z1}
  % This file was created by matlab2tikz.
%
%The latest updates can be retrieved from
%  http://www.mathworks.com/matlabcentral/fileexchange/22022-matlab2tikz-matlab2tikz
%where you can also make suggestions and rate matlab2tikz.
%
\definecolor{mycolor1}{rgb}{1.00000,0.00000,1.00000}%
%
\begin{tikzpicture}[trim axis left, trim axis right]

\begin{axis}[%
width=\figurewidth,
height=\figureheight,
at={(0\figurewidth,0\figureheight)},
scale only axis,
every outer x axis line/.append style={black},
every x tick label/.append style={font=\color{black}},
xmin=0,
xmax=100,
%xlabel={Time},
every outer y axis line/.append style={black},
every y tick label/.append style={font=\color{black}},
ymin=0,
ymax=0.015,
%ylabel={Depth $\delta^+$},
axis background/.style={fill=white},
axis x line*=bottom,
axis y line*=left,
yticklabel style={
        /pgf/number format/fixed,
        /pgf/number format/precision=3
},
scaled y ticks=false,
legend style={legend cell align=left,align=left,draw=black,font=\footnotesize, at={(0.98,0.02)},anchor=south east},
every axis legend/.code={\renewcommand\addlegendentry[2][]{}}  %ignore legend locally
]
\addplot [color=green,dashed]
  table[row sep=crcr]{%
1	0.0133744051204268\\
2	0.0133732367419305\\
3	0.013372012621659\\
4	0.0133707296376077\\
5	0.0133693844416798\\
6	0.0133679734340833\\
7	0.0133664927327869\\
8	0.0133649381367293\\
9	0.0133633050810592\\
10	0.0133615885820692\\
11	0.0133597831684727\\
12	0.0133578827938262\\
13	0.0133558807212804\\
14	0.013353769364907\\
15	0.0133515400614031\\
16	0.0133491827478389\\
17	0.0133466856452729\\
18	0.0133440353793949\\
19	0.0133412158592881\\
20	0.0133382063050905\\
21	0.0133349788013253\\
22	0.0133314710416767\\
23	0.0133256307720385\\
24	0.0133187268005979\\
25	0.0133115014701138\\
26	0.0133039373776573\\
27	0.0132960163863587\\
28	0.0132877198362999\\
29	0.0132790285640418\\
30	0.0132699147596394\\
31	0.0132603500365029\\
32	0.0132503085665101\\
33	0.013239763717221\\
34	0.013228688850036\\
35	0.0132170595327755\\
36	0.0132048604286939\\
37	0.0131921086965202\\
38	0.0131797678054959\\
39	0.0131679889039777\\
40	0.0131555356694942\\
41	0.0131421149743497\\
42	0.0131258966933831\\
43	0.0131089245905023\\
44	0.0130911359964698\\
45	0.0130725022575923\\
46	0.0130529714764019\\
47	0.0130325185289378\\
48	0.0130109566518488\\
49	0.0129881140253643\\
50	0.0129638608645403\\
51	0.0129380481864874\\
52	0.0129105243784227\\
53	0.0128811922969727\\
54	0.0128598883764592\\
55	0.0128306068684299\\
56	0.0127753578860927\\
57	0.0127172376519649\\
58	0.0126560324730213\\
59	0.0125914737189533\\
60	0.0125230922061851\\
61	0.0124499912528344\\
62	0.0123547517648388\\
63	0.0122515452911363\\
64	0.0121426092756864\\
65	0.0120270049315931\\
66	0.011904636521389\\
67	0.0118179901678075\\
68	0.0115567838500847\\
69	0.0112828861431057\\
70	0.0109997672384112\\
71	0.0107067748385756\\
72	0.0104015375333005\\
73	0.0100837067018434\\
74	0.00992625215172344\\
75	0.00980546347874705\\
76	0.0096934048262946\\
77	0.00958319731487932\\
78	0.00947687363401329\\
79	0.00937730501186515\\
80	0.00928370229659566\\
81	0.00918927822085656\\
82	0.00909405347503245\\
83	0.00899741276828745\\
84	0.00889997132590081\\
85	0.00880131112469073\\
86	0.00870132857036005\\
87	0.00856351667487943\\
88	0.00841531445529877\\
89	0.00789039112455869\\
90	0.0072929319095366\\
91	0.00701078656667871\\
92	0.00685454740616991\\
93	0.00670384513780405\\
94	0.00653293313820345\\
95	0.00630986941934306\\
96	0.00595502673688305\\
97	0.00526543684781529\\
98	0.00372263221541378\\
99	0\\
100	0\\
};
\addlegendentry{$q=-4$};

\addplot [color=mycolor1,dashed]
  table[row sep=crcr]{%
1	0.0134701341523511\\
2	0.0134698838074164\\
3	0.0134696212177822\\
4	0.0134693456751211\\
5	0.0134690564181801\\
6	0.0134687526272248\\
7	0.0134684334176678\\
8	0.0134680978327024\\
9	0.0134677448346896\\
10	0.0134673732949176\\
11	0.0134669819811305\\
12	0.0134665695418827\\
13	0.0134661344863399\\
14	0.0134656751580115\\
15	0.0134651897020547\\
16	0.0134646760277116\\
17	0.0134641317446514\\
18	0.0134635539709976\\
19	0.0134629389695026\\
20	0.0134622813431573\\
21	0.0134615719323341\\
22	0.0134607952077426\\
23	0.0134597060197198\\
24	0.013458462219672\\
25	0.0134571525380902\\
26	0.0134557717215507\\
27	0.0134543141405631\\
28	0.0134527745610136\\
29	0.0134511515065982\\
30	0.0134496365673057\\
31	0.0134481644305644\\
32	0.0134466071367077\\
33	0.0134449552087326\\
34	0.0134431966828015\\
35	0.0134413152914504\\
36	0.0134392850437182\\
37	0.013437051065151\\
38	0.0134337372524226\\
39	0.0134291853479914\\
40	0.0134244200364384\\
41	0.0134194295918729\\
42	0.0134142082959354\\
43	0.0134087367591064\\
44	0.01340299405703\\
45	0.0133969592026743\\
46	0.0133906050693801\\
47	0.0133838975917666\\
48	0.0133767800324706\\
49	0.0133691814885951\\
50	0.0133610198062789\\
51	0.0133522072898686\\
52	0.0133426081295019\\
53	0.0133319378461673\\
54	0.0133112179468777\\
55	0.0132882827281586\\
56	0.013261208971274\\
57	0.0132327388726622\\
58	0.0132027370966733\\
59	0.0131710394515216\\
60	0.0131374476562211\\
61	0.0131017447763351\\
62	0.013063726819675\\
63	0.013023206509652\\
64	0.0129799585299569\\
65	0.0129330499650752\\
66	0.0128834617458436\\
67	0.0127935938377722\\
68	0.0126768342462107\\
69	0.0125531936284494\\
70	0.0124225798726219\\
71	0.0123021342240074\\
72	0.0121732761797577\\
73	0.0120320260918441\\
74	0.0117352022919444\\
75	0.0113987368258269\\
76	0.0110505695981884\\
77	0.010744369337832\\
78	0.0104242052773378\\
79	0.010087955721042\\
80	0.00980491057455578\\
81	0.00965453383350001\\
82	0.00950948033803455\\
83	0.00936543906702952\\
84	0.00922342535285324\\
85	0.009086406347184\\
86	0.00895613361887325\\
87	0.0088308107021435\\
88	0.00870221359062729\\
89	0.00853718400456589\\
90	0.00836484218109529\\
91	0.00789013764499167\\
92	0.00728526236144176\\
93	0.00678359870846127\\
94	0.00656524169251083\\
95	0.00631925648355607\\
96	0.00595502673688305\\
97	0.00526543684781529\\
98	0.00372263221541378\\
99	0\\
100	0\\
};
\addlegendentry{$q=-3$};

\addplot [color=red,dashed]
  table[row sep=crcr]{%
1	0.0135343238337338\\
2	0.0135343031881343\\
3	0.0135342815207839\\
4	0.0135342587715904\\
5	0.0135342348759616\\
6	0.0135342097643806\\
7	0.0135341833619359\\
8	0.0135341555877958\\
9	0.013534126354607\\
10	0.0135340955677864\\
11	0.0135340631246484\\
12	0.0135340289132894\\
13	0.0135339928111224\\
14	0.0135339546828592\\
15	0.0135339143771539\\
16	0.0135338717187688\\
17	0.0135338264893583\\
18	0.0135337783913148\\
19	0.0135337269846451\\
20	0.0135336716069749\\
21	0.0135336114424889\\
22	0.013533546147137\\
23	0.0135334767115275\\
24	0.0135334032420029\\
25	0.0135333252834157\\
26	0.0135332422065085\\
27	0.0135331529493006\\
28	0.0135330551495803\\
29	0.0135329420596284\\
30	0.0135326395983859\\
31	0.0135322006858445\\
32	0.0135317298818973\\
33	0.0135312245018903\\
34	0.0135306814296354\\
35	0.0135300964502921\\
36	0.0135294627018532\\
37	0.0135287685457008\\
38	0.0135279141300613\\
39	0.0135268924274728\\
40	0.0135258288117781\\
41	0.0135247210277558\\
42	0.0135235661358127\\
43	0.013522360922056\\
44	0.0135211017585016\\
45	0.0135197841377087\\
46	0.0135184023061363\\
47	0.0135169486771325\\
48	0.0135154138758027\\
49	0.0135137856729811\\
50	0.0135120498571668\\
51	0.0135101868306629\\
52	0.013508162323659\\
53	0.013505918626762\\
54	0.0135023997061433\\
55	0.0134986600639254\\
56	0.013494721941114\\
57	0.0134905629719759\\
58	0.0134861564652107\\
59	0.013481470725003\\
60	0.0134764688245645\\
61	0.0134711077681729\\
62	0.013465334094982\\
63	0.0134590822794882\\
64	0.0134522220916364\\
65	0.0134444536701181\\
66	0.0134335987208681\\
67	0.0134132997832804\\
68	0.0133915739776063\\
69	0.0133682069536464\\
70	0.013342634583863\\
71	0.0132990061590845\\
72	0.0132518006087606\\
73	0.0132003193723743\\
74	0.0131268930102856\\
75	0.0130450097086445\\
76	0.0129562647097305\\
77	0.0128108664130651\\
78	0.0126565085755826\\
79	0.012492602525595\\
80	0.0122592283844567\\
81	0.0118966548453714\\
82	0.0115172997662786\\
83	0.0111236241951391\\
84	0.0107356419355688\\
85	0.0103354256377107\\
86	0.00993918903635679\\
87	0.00959922542729119\\
88	0.00940838952857641\\
89	0.00921620342488413\\
90	0.00901950584871832\\
91	0.00879044486617429\\
92	0.00854778579999546\\
93	0.00817351423989826\\
94	0.00747809308056406\\
95	0.00666550905412058\\
96	0.00606998573569038\\
97	0.00526543684781529\\
98	0.00372263221541378\\
99	0\\
100	0\\
};
\addlegendentry{$q=-2$};

\addplot [color=blue,dashed]
  table[row sep=crcr]{%
1	0.0136879538880162\\
2	0.0136879532488779\\
3	0.0136879525779846\\
4	0.013687951873426\\
5	0.0136879511331346\\
6	0.0136879503548642\\
7	0.0136879495361634\\
8	0.0136879486743404\\
9	0.0136879477664158\\
10	0.0136879468090551\\
11	0.0136879457984725\\
12	0.0136879447302841\\
13	0.0136879435992629\\
14	0.0136879423988728\\
15	0.0136879411203181\\
16	0.0136879397507781\\
17	0.013687938270854\\
18	0.0136879366518943\\
19	0.0136879348564106\\
20	0.0136879328509975\\
21	0.0136879306392626\\
22	0.0136879282825246\\
23	0.0136879257882397\\
24	0.0136879231269934\\
25	0.0136879202409398\\
26	0.0136879170061582\\
27	0.0136879131477299\\
28	0.0136879080619715\\
29	0.0136879005023335\\
30	0.0136878696178941\\
31	0.0136878223730618\\
32	0.013687771622661\\
33	0.0136877170204215\\
34	0.0136876581134322\\
35	0.0136875942975557\\
36	0.0136875249024927\\
37	0.0136874497442301\\
38	0.0136873701750985\\
39	0.0136872875218237\\
40	0.0136872016321279\\
41	0.0136871123218062\\
42	0.0136870193922346\\
43	0.0136869226176787\\
44	0.0136868217165809\\
45	0.0136867163266599\\
46	0.0136866059707012\\
47	0.0136864900270535\\
48	0.0136863676198777\\
49	0.0136862376111408\\
50	0.0136860985730614\\
51	0.0136859483592565\\
52	0.0136857843863199\\
53	0.013685605759584\\
54	0.0136854174125948\\
55	0.013685218964755\\
56	0.0136850091454994\\
57	0.0136847864183317\\
58	0.0136845489189158\\
59	0.0136842943742768\\
60	0.0136840199107826\\
61	0.0136837213724977\\
62	0.0136833917845932\\
63	0.0136830170366707\\
64	0.0136825704135559\\
65	0.013682009593348\\
66	0.0136810910233284\\
67	0.0136796485810653\\
68	0.0136780824538643\\
69	0.0136763435950804\\
70	0.0136743481687335\\
71	0.0136702200007839\\
72	0.0136657586250234\\
73	0.0136609249111855\\
74	0.0136556928823626\\
75	0.0136498127711416\\
76	0.0136427952026973\\
77	0.0136289106894291\\
78	0.013613508218495\\
79	0.0135940830735077\\
80	0.0135655108209789\\
81	0.0135202758193455\\
82	0.0134719887720633\\
83	0.0134158625421176\\
84	0.0133340471324284\\
85	0.013236598858992\\
86	0.0131036152376525\\
87	0.0128978590642612\\
88	0.0125466906969927\\
89	0.0121802255208229\\
90	0.0117961179867617\\
91	0.0113917221775055\\
92	0.0109639819858097\\
93	0.0104985873627227\\
94	0.00997772646175004\\
95	0.009409852668374\\
96	0.0083273620799101\\
97	0.00667948821753289\\
98	0.00372263221541378\\
99	0\\
100	0\\
};
\addlegendentry{$q=-1$};

\addplot [color=black,solid]
  table[row sep=crcr]{%
1	0.00575426534597786\\
2	0.00575427712397154\\
3	0.00575428941830398\\
4	0.00575430226788118\\
5	0.00575431572062637\\
6	0.00575432983959859\\
7	0.00575434471210626\\
8	0.00575436046555295\\
9	0.0057543772894913\\
10	0.00575439545556883\\
11	0.00575441530725482\\
12	0.00575443716041685\\
13	0.0057544610569919\\
14	0.0057544865049941\\
15	0.00575451291759309\\
16	0.00575454058526589\\
17	0.00575457014517644\\
18	0.00575460300814864\\
19	0.00575464223581203\\
20	0.00575469417316846\\
21	0.00575477096272872\\
22	0.00575489277030884\\
23	0.00575508421830758\\
24	0.00575535102563675\\
25	0.00575563174449665\\
26	0.00575592019313621\\
27	0.00575621670372833\\
28	0.00575652161415754\\
29	0.00575683524586936\\
30	0.00575715796110782\\
31	0.00575748994627936\\
32	0.00575783115194502\\
33	0.00575818174225828\\
34	0.00575854250784988\\
35	0.00575891514749755\\
36	0.00575930142410048\\
37	0.00575970226635192\\
38	0.00576011873645547\\
39	0.00576055201622807\\
40	0.00576100358077005\\
41	0.0057614754293691\\
42	0.00576197052180693\\
43	0.00576249352917535\\
44	0.00576305190885719\\
45	0.0057636567912641\\
46	0.00576432157743081\\
47	0.00576505316688385\\
48	0.0057658319171062\\
49	0.00576662813007013\\
50	0.00576744284506347\\
51	0.00576828083451999\\
52	0.00576915522815688\\
53	0.00577010100550168\\
54	0.00577121050329697\\
55	0.00577272356675477\\
56	0.0057752337607018\\
57	0.00577983812218373\\
58	0.00578779052079711\\
59	0.00579697759927671\\
60	0.00580648992571076\\
61	0.00581635920251381\\
62	0.00582662714043398\\
63	0.00583734787811777\\
64	0.0058485837602797\\
65	0.00586047959393243\\
66	0.00587331682384705\\
67	0.00588753370203374\\
68	0.00590349855825579\\
69	0.00591978275470636\\
70	0.0059362553444193\\
71	0.0059528480472759\\
72	0.00596935970958119\\
73	0.00598534130282059\\
74	0.00600003305315891\\
75	0.00601509020883893\\
76	0.0060327308514258\\
77	0.00605632144821932\\
78	0.00609543669896921\\
79	0.00617595177027325\\
80	0.00637904248331634\\
81	0.00660012577470798\\
82	0.00684186381926437\\
83	0.0071192556232119\\
84	0.00745372701190976\\
85	0.00780802271269009\\
86	0.00819449115781853\\
87	0.00859420524903984\\
88	0.00899516773316457\\
89	0.00942483938052141\\
90	0.009880663115063\\
91	0.0103520113689121\\
92	0.0108143885853798\\
93	0.0112307011044688\\
94	0.0115299065647041\\
95	0.0118453772828423\\
96	0.0121711431019151\\
97	0.0125871619692242\\
98	0.0131599673541368\\
99	0\\
100	0\\
};
\addlegendentry{$q=0$};

\addplot [color=blue,solid]
  table[row sep=crcr]{%
1	0.00611195743992147\\
2	0.00611210127744183\\
3	0.00611225049062715\\
4	0.00611240611526335\\
5	0.00611256858988664\\
6	0.00611273842077946\\
7	0.00611291622794206\\
8	0.00611310280488133\\
9	0.0061132992563266\\
10	0.00611350726200461\\
11	0.00611372950913175\\
12	0.00611397019046641\\
13	0.00611423489675197\\
14	0.00611452790128542\\
15	0.00611484414413289\\
16	0.00611517048364539\\
17	0.0061155080513918\\
18	0.00611585924259172\\
19	0.00611622985442528\\
20	0.00611663477537872\\
21	0.00611711258970825\\
22	0.00611776069710894\\
23	0.00611880917260985\\
24	0.00612072631493327\\
25	0.00612410424085045\\
26	0.00612771283262292\\
27	0.00613142105590445\\
28	0.00613523341110056\\
29	0.00613915495187843\\
30	0.0061431912025746\\
31	0.00614734755506389\\
32	0.00615162737795241\\
33	0.00615602863792215\\
34	0.0061605488507114\\
35	0.006165191724798\\
36	0.00616997982339883\\
37	0.00617494386640447\\
38	0.00618009472629881\\
39	0.00618544441124828\\
40	0.00619100634842221\\
41	0.00619679593343876\\
42	0.00620283173448316\\
43	0.00620913826743095\\
44	0.00621575228835088\\
45	0.00622273639201471\\
46	0.00623020544272944\\
47	0.00623836346882385\\
48	0.0062474859011764\\
49	0.00625741208837532\\
50	0.00626755911935922\\
51	0.00627791631623346\\
52	0.00628848474721297\\
53	0.00629927068719561\\
54	0.00631028700434949\\
55	0.00632161216158944\\
56	0.00633368587905997\\
57	0.00635053227887613\\
58	0.006385735778517\\
59	0.00648331820795726\\
60	0.00660615967268541\\
61	0.006734745832982\\
62	0.00686961317024147\\
63	0.00701144230654091\\
64	0.00716118762100751\\
65	0.00731941956784894\\
66	0.00748741533415371\\
67	0.00766813855059602\\
68	0.00786880137596036\\
69	0.00811211653436205\\
70	0.00836752968578314\\
71	0.00863320622186124\\
72	0.00890958995293047\\
73	0.00919607014322709\\
74	0.00948783313999121\\
75	0.00974280190693537\\
76	0.00998260618842222\\
77	0.0102408042912283\\
78	0.0105124813587247\\
79	0.0107992798224028\\
80	0.0109815129779682\\
81	0.0111687040420122\\
82	0.0113668692677404\\
83	0.0115619946015561\\
84	0.0117158656319735\\
85	0.011872362666853\\
86	0.01201615239816\\
87	0.0121571154521932\\
88	0.0122993275565428\\
89	0.0124305953346559\\
90	0.0125648143575899\\
91	0.0127091240524227\\
92	0.0128618523447465\\
93	0.0130087734130237\\
94	0.0131401900614982\\
95	0.013274348821181\\
96	0.0134137237378342\\
97	0.0135415125049663\\
98	0.0137226322154138\\
99	0\\
100	0\\
};
\addlegendentry{$q=1$};

\addplot [color=red,solid]
  table[row sep=crcr]{%
1	0.00632895508506573\\
2	0.00633084145153905\\
3	0.0063327898353037\\
4	0.00633480574088444\\
5	0.00633690749891728\\
6	0.00633910078889653\\
7	0.00634139195517411\\
8	0.00634378841593545\\
9	0.00634629897419818\\
10	0.00634893477738024\\
11	0.00635171156688937\\
12	0.00635465462117146\\
13	0.00635780852672469\\
14	0.00636125159689001\\
15	0.00636509480871216\\
16	0.00636932404119802\\
17	0.00637368409169078\\
18	0.0063781818127259\\
19	0.00638282692695676\\
20	0.00638763790603132\\
21	0.0063926620419921\\
22	0.00639804469230451\\
23	0.0064042700138639\\
24	0.00641300324110871\\
25	0.00643005399373983\\
26	0.00647586057939858\\
27	0.00652601122097035\\
28	0.00657779307943962\\
29	0.00663129019834863\\
30	0.00668659735846028\\
31	0.00674382286306238\\
32	0.00680308934229339\\
33	0.00686451733197731\\
34	0.00692811829741432\\
35	0.00699384934535099\\
36	0.00706167063042404\\
37	0.00713190383396789\\
38	0.00720520453156834\\
39	0.00728179311926532\\
40	0.00736191397046987\\
41	0.00744583920216459\\
42	0.00753387381806631\\
43	0.0076263640728364\\
44	0.00772371474744643\\
45	0.00782643087124711\\
46	0.00793522919980522\\
47	0.0080513624552378\\
48	0.00817759912090745\\
49	0.00832182390107149\\
50	0.00848547917420062\\
51	0.00865556754041888\\
52	0.00883213841468531\\
53	0.00901545543800836\\
54	0.00920581918513799\\
55	0.00940305290874223\\
56	0.00960530765063492\\
57	0.00978760451252322\\
58	0.00993894072617349\\
59	0.0100459757563916\\
60	0.0101380258113839\\
61	0.0102340390513143\\
62	0.010334289505668\\
63	0.0104391142575754\\
64	0.0105490224994677\\
65	0.010675222322629\\
66	0.0108104263538037\\
67	0.0109487546910201\\
68	0.01109016555593\\
69	0.011204328632062\\
70	0.0113187943178877\\
71	0.011436122972429\\
72	0.0115570879763876\\
73	0.0116816137216073\\
74	0.0118146720086514\\
75	0.0119529324635412\\
76	0.0120722881679121\\
77	0.0121854604739172\\
78	0.0122965444770228\\
79	0.0124042048077948\\
80	0.0124983168303273\\
81	0.012586216201087\\
82	0.012671290989184\\
83	0.0127516837990635\\
84	0.0128238839241057\\
85	0.0128985199103956\\
86	0.0129683087063421\\
87	0.0130325766610747\\
88	0.0130961099206039\\
89	0.0131616966999991\\
90	0.0132309829015184\\
91	0.0132954532730901\\
92	0.0133510563634734\\
93	0.0134012434829163\\
94	0.0134449411682291\\
95	0.0134857845974406\\
96	0.013531098659188\\
97	0.0135945905484418\\
98	0.0137226322154138\\
99	0\\
100	0\\
};
\addlegendentry{$q=2$};

\addplot [color=mycolor1,solid]
  table[row sep=crcr]{%
1	0.00774242316317075\\
2	0.00777118496557083\\
3	0.00780100853035913\\
4	0.00783187552525477\\
5	0.00786378531598214\\
6	0.00789714489015466\\
7	0.00793205595309538\\
8	0.00796863099533649\\
9	0.00800699855745782\\
10	0.00804730429781573\\
11	0.00808971652244768\\
12	0.00813444044633872\\
13	0.00818175699480529\\
14	0.00823213582747507\\
15	0.00828657346284589\\
16	0.00834781258156791\\
17	0.00841762993618121\\
18	0.0084900885891944\\
19	0.00856533307147854\\
20	0.00864351013304426\\
21	0.00872474624884345\\
22	0.00880907110406114\\
23	0.00889616214417913\\
24	0.00898449243628432\\
25	0.00906847528986695\\
26	0.00912878115334637\\
27	0.00918840211423965\\
28	0.00925000266547761\\
29	0.0093136284707402\\
30	0.00937932565915868\\
31	0.00944715471556349\\
32	0.00951723756748327\\
33	0.00958990814091621\\
34	0.00966620147850951\\
35	0.00974420748530797\\
36	0.00981982208491264\\
37	0.00988499969562246\\
38	0.00993879301745489\\
39	0.00999439685913653\\
40	0.0100518746904371\\
41	0.0101112905981897\\
42	0.0101727087860881\\
43	0.0102361925886073\\
44	0.0103018009795629\\
45	0.0103695883173195\\
46	0.0104396182728121\\
47	0.0105119760223757\\
48	0.0105867235870087\\
49	0.0106571035795354\\
50	0.0107182318800902\\
51	0.0107828099100451\\
52	0.0108590729347671\\
53	0.0109408774427858\\
54	0.0110246576997461\\
55	0.0111102306182809\\
56	0.0111972095240246\\
57	0.0112850804328061\\
58	0.0113694380269561\\
59	0.0114487988228078\\
60	0.0115271342816182\\
61	0.011606879986735\\
62	0.011687876169651\\
63	0.0117698951856426\\
64	0.0118525851662363\\
65	0.0119255653130305\\
66	0.0119962295426003\\
67	0.012073892260441\\
68	0.0121504328153382\\
69	0.0122232501316895\\
70	0.0122952662059049\\
71	0.0123662888005724\\
72	0.0124357127132211\\
73	0.0125033780283189\\
74	0.0125632347167128\\
75	0.012617490066832\\
76	0.0126686278090238\\
77	0.0127179740493617\\
78	0.0127718015915979\\
79	0.012831249623831\\
80	0.0128925563906612\\
81	0.0129506770975516\\
82	0.0130004956983336\\
83	0.0130495479118441\\
84	0.0130985645731492\\
85	0.0131413761405173\\
86	0.0131849800381144\\
87	0.0132280361637285\\
88	0.013270860866495\\
89	0.0133101615301819\\
90	0.0133463040606936\\
91	0.013382197074282\\
92	0.0134133556601411\\
93	0.0134406278163612\\
94	0.0134672391046149\\
95	0.0134961142982946\\
96	0.0135332854097662\\
97	0.0135945905484418\\
98	0.0137226322154138\\
99	0\\
100	0\\
};
\addlegendentry{$q=3$};

\addplot [color=green,solid]
  table[row sep=crcr]{%
1	0.00978131038814559\\
2	0.00981637783802372\\
3	0.0098519579948201\\
4	0.00988725161043769\\
5	0.00992001432137705\\
6	0.00994439781645423\\
7	0.00996954322639673\\
8	0.00999543947896755\\
9	0.0100220435934237\\
10	0.010049343870241\\
11	0.0100773304180641\\
12	0.0101059866537501\\
13	0.0101352886140559\\
14	0.0101652053125276\\
15	0.0101956993675745\\
16	0.010224732170877\\
17	0.0102489349038127\\
18	0.0102738240953897\\
19	0.0102994033395041\\
20	0.0103256719693691\\
21	0.0103526222164455\\
22	0.0103802312855305\\
23	0.0104084365821289\\
24	0.0104370541520036\\
25	0.010465504381932\\
26	0.0104925425304635\\
27	0.0105204010627644\\
28	0.010549401409227\\
29	0.0105796321045768\\
30	0.0106111923906807\\
31	0.0106442033340564\\
32	0.0106788205221327\\
33	0.0107152763193217\\
34	0.010753946298309\\
35	0.0108003937513058\\
36	0.0108498679795693\\
37	0.0108992106234798\\
38	0.0109483004090331\\
39	0.0109985185663909\\
40	0.0110498343615476\\
41	0.0111022043669381\\
42	0.0111555706211653\\
43	0.011209860248281\\
44	0.011264991773661\\
45	0.0113207914424315\\
46	0.0113768696379083\\
47	0.0114328769696175\\
48	0.0114891674566955\\
49	0.011544946092546\\
50	0.0115998378016495\\
51	0.0116550428630709\\
52	0.0117029045664042\\
53	0.0117486353736254\\
54	0.011795319388958\\
55	0.0118429933566652\\
56	0.0118917177130443\\
57	0.011941610640322\\
58	0.011994539716604\\
59	0.0120529720239776\\
60	0.0121112138643652\\
61	0.0121690899593177\\
62	0.0122263408119997\\
63	0.0122826633947296\\
64	0.0123376954152201\\
65	0.012390153396452\\
66	0.012439404022673\\
67	0.0124814243912259\\
68	0.0125229229511473\\
69	0.0125639119594753\\
70	0.0126042857995569\\
71	0.0126438992303004\\
72	0.012682653040563\\
73	0.01272945580353\\
74	0.0127809022671538\\
75	0.0128345975476407\\
76	0.0128857948058951\\
77	0.0129355069189137\\
78	0.0129793137219755\\
79	0.0130147563130651\\
80	0.013044942059718\\
81	0.0130750041776383\\
82	0.0131068595515389\\
83	0.0131452874277376\\
84	0.0131834419894261\\
85	0.013220708759408\\
86	0.0132500695219427\\
87	0.0132795242628055\\
88	0.0133096673494054\\
89	0.0133412986932403\\
90	0.0133695116718789\\
91	0.0133948042804017\\
92	0.0134190923181857\\
93	0.013443060749888\\
94	0.0134680597695799\\
95	0.013496272944263\\
96	0.0135332854097662\\
97	0.0135945905484418\\
98	0.0137226322154138\\
99	0\\
100	0\\
};
\addlegendentry{$q=4$};

\end{axis}
\end{tikzpicture}%
 
  \caption{Discrete Time}
\end{subfigure}\\

\leavevmode\smash{\makebox[0pt]{\hspace{-7em}% HORIZONTAL POSITION           
  \rotatebox[origin=l]{90}{\hspace{20em}% VERTICAL POSITION
    Depth $\delta^+$}%
}}\hspace{0pt plus 1filll}\null

Time (s)

\vspace{1cm}
\begin{subfigure}{\linewidth}
  \centering
  \tikzsetnextfilename{altdeltalegend}
  \definecolor{mycolor1}{rgb}{0.00000,1.00000,0.14286}%
\definecolor{mycolor2}{rgb}{0.00000,1.00000,0.28571}%
\definecolor{mycolor3}{rgb}{0.00000,1.00000,0.42857}%
\definecolor{mycolor4}{rgb}{0.00000,1.00000,0.57143}%
\definecolor{mycolor5}{rgb}{0.00000,1.00000,0.71429}%
\definecolor{mycolor6}{rgb}{0.00000,1.00000,0.85714}%
\definecolor{mycolor7}{rgb}{0.00000,1.00000,1.00000}%
\definecolor{mycolor8}{rgb}{0.00000,0.87500,1.00000}%
\definecolor{mycolor9}{rgb}{0.00000,0.62500,1.00000}%
\definecolor{mycolor10}{rgb}{0.12500,0.00000,1.00000}%
\definecolor{mycolor11}{rgb}{0.25000,0.00000,1.00000}%
\definecolor{mycolor12}{rgb}{0.37500,0.00000,1.00000}%
\definecolor{mycolor13}{rgb}{0.50000,0.00000,1.00000}%
\definecolor{mycolor14}{rgb}{0.62500,0.00000,1.00000}%
\definecolor{mycolor15}{rgb}{0.75000,0.00000,1.00000}%
\definecolor{mycolor16}{rgb}{0.87500,0.00000,1.00000}%
\definecolor{mycolor17}{rgb}{1.00000,0.00000,1.00000}%
\definecolor{mycolor18}{rgb}{1.00000,0.00000,0.87500}%
\definecolor{mycolor19}{rgb}{1.00000,0.00000,0.62500}%
\definecolor{mycolor20}{rgb}{0.85714,0.00000,0.00000}%
\definecolor{mycolor21}{rgb}{0.71429,0.00000,0.00000}%
%[trim axis left, trim axis right]
\begin{tikzpicture}
\begin{axis}[%
    hide axis,
    scale only axis,
    height=0pt,
    width=0pt,
    point meta min=-19,
    point meta max=19,
    colormap={mymap}{[1pt] rgb(0pt)=(0,1,0); rgb(7pt)=(0,1,1); rgb(15pt)=(0,0,1); rgb(23pt)=(1,0,1); rgb(31pt)=(1,0,0); rgb(38pt)=(0,0,0)},
    colorbar horizontal,
    colorbar style={width=15cm,xtick={{-15},{-10},{-5},{0},{5},{10},{15}},ylabel={Inventory Level $Q$}, y label style={at={(axis description cs:0.5,-1)},rotate=-90,anchor=north}}
    %colorbar style={separate axis lines,every outer x axis line/.append style={black},every x tick label/.append style={font=\color{black}},every outer y axis line/.append style={black},every y tick label/.append style={font=\color{black}},yticklabels={{-19},{-17},{-15},{-13},{-11},{-9},{-7},{-5},{-3},{-1},{1},{3},{5},{7},{9},{11},{13},{15},{17},{19}},ylabel={Inventory Level $Q$}}
]%
    \addplot [draw=none] coordinates {(0,0)};
\end{axis}
\end{tikzpicture}
 
\end{subfigure}%
  \caption{Optimal buy depths $\delta^{+}$ for Markov state $Z=(\rho = -1, \Delta S = -1)$, implying heavy imbalance in favor of sell pressure, and having previously seen a downward price change. We expect the midprice to fall.}
  \label{fig:comp_dp_z1_test}
\end{figure}

\begin{figure}
\centering
\begin{subfigure}{.45\linewidth}
  \centering
  \setlength\figureheight{\linewidth} 
  \setlength\figurewidth{\linewidth}
  \tikzsetnextfilename{testdp_cts_z8}
  % This file was created by matlab2tikz.
%
%The latest updates can be retrieved from
%  http://www.mathworks.com/matlabcentral/fileexchange/22022-matlab2tikz-matlab2tikz
%where you can also make suggestions and rate matlab2tikz.
%
\definecolor{mycolor1}{rgb}{1.00000,0.00000,1.00000}%
%
\begin{tikzpicture}[trim axis left, trim axis right]

\begin{axis}[%
width=\figurewidth,
height=\figureheight,
at={(0\figurewidth,0\figureheight)},
scale only axis,
every outer x axis line/.append style={black},
every x tick label/.append style={font=\color{black}},
xmin=0,
xmax=100,
%xlabel={Time},
every outer y axis line/.append style={black},
every y tick label/.append style={font=\color{black}},
ymin=0,
ymax=0.015,
%ylabel={Depth $\delta^+$},
axis background/.style={fill=white},
axis x line*=bottom,
axis y line*=left,
yticklabel style={
        /pgf/number format/fixed,
        /pgf/number format/precision=3
},
scaled y ticks=false,
legend style={legend cell align=left,align=left,draw=black,font=\footnotesize, at={(0.98,0.02)},anchor=south east},
every axis legend/.code={\renewcommand\addlegendentry[2][]{}}  %ignore legend locally
]
\addplot [color=green,dashed]
  table[row sep=crcr]{%
0.01	0.00574337684382281\\
1.01	0.00573714257097391\\
2.01	0.00573060439045445\\
3.01	0.00572374696472981\\
4.01	0.00571655398577405\\
5.01	0.00570900807581416\\
6.01	0.00570109068452653\\
7.01	0.00569278199238071\\
8.01	0.00568406083785614\\
9.01	0.00567490469696307\\
10.01	0.00566528970019458\\
11.01	0.0056551905933952\\
12.01	0.00564458066781824\\
13.01	0.00563343168900105\\
14.01	0.00562171382645682\\
15.01	0.00560939558542931\\
16.01	0.00559644374225158\\
17.01	0.00558282328518515\\
18.01	0.00556849736298323\\
19.01	0.00555342724381153\\
20.01	0.00553757228754437\\
21.01	0.00552088993479424\\
22.01	0.00550333571625747\\
23.01	0.00548486328596039\\
24.01	0.00546542448160292\\
25.01	0.00544496941416772\\
26.01	0.00542344658691833\\
27.01	0.00540080304029837\\
28.01	0.00537698451325962\\
29.01	0.00535193560202306\\
30.01	0.00532559988252652\\
31.01	0.00529791994042451\\
32.01	0.00526883721902203\\
33.01	0.005238291546086\\
34.01	0.00520622012872673\\
35.01	0.0051725557061639\\
36.01	0.00513722339215991\\
37.01	0.00510013541902272\\
38.01	0.00506118484307974\\
39.01	0.00502025109539953\\
40.01	0.00497720620061289\\
41.01	0.00493191585711632\\
42.01	0.00488424099449774\\
43.01	0.00483404033902436\\
44.01	0.00478117440670781\\
45.01	0.00472551152196101\\
46.01	0.00466693694772569\\
47.01	0.00460536706918557\\
48.01	0.00454076724158745\\
49.01	0.00447317647854652\\
50.01	0.00440274318473162\\
51.01	0.00432977495393119\\
52.01	0.00425480766396775\\
53.01	0.00417875845854279\\
54.01	0.0041025512313847\\
55.01	0.00402649167987937\\
56.01	0.00395085846478194\\
57.01	0.0038759226263916\\
58.01	0.00380192181583351\\
59.01	0.00372902083263049\\
60.01	0.00365727270584941\\
61.01	0.00358678068417524\\
62.01	0.00351765955583304\\
63.01	0.003449865257216\\
64.01	0.0033830836898928\\
65.01	0.00331649676251478\\
66.01	0.00324921533936638\\
67.01	0.00318119667142671\\
68.01	0.00311243053015165\\
69.01	0.00304287603760009\\
70.01	0.00297245348148554\\
71.01	0.00290103585103694\\
72.01	0.00282844183185719\\
73.01	0.00275445855944312\\
74.01	0.00267886954662985\\
75.01	0.00260146359874958\\
76.01	0.00252208658155456\\
77.01	0.00244064040944975\\
78.01	0.00235702519206653\\
79.01	0.0022711388415078\\
80.01	0.00218288815641267\\
81.01	0.0020921920300596\\
82.01	0.0019989736781259\\
83.01	0.00190316156414087\\
84.01	0.00180469257861682\\
85.01	0.00170351474884668\\
86.01	0.00159958740458458\\
87.01	0.00149287866220315\\
88.01	0.00138336624866992\\
89.01	0.00127104122721427\\
90.01	0.00115591289641989\\
91.01	0.00103801510632268\\
92.01	0.000917414847049264\\
93.01	0.000794223772923664\\
94.01	0.000668613273120433\\
95.01	0.000540834094057197\\
96.01	0.000411241866149138\\
97.01	0.000280330070258007\\
98.01	0.000148772403259365\\
99.01	2.53717161561775e-05\\
99.02	2.46546563816945e-05\\
99.03	2.39601159352475e-05\\
99.04	2.32882766082222e-05\\
99.05	2.26246795302226e-05\\
99.06	2.1969188376313e-05\\
99.07	2.1321850006863e-05\\
99.08	2.06827099941027e-05\\
99.09	2.00518125263397e-05\\
99.1	1.94292049667238e-05\\
99.11	1.88149349641432e-05\\
99.12	1.82090488000525e-05\\
99.13	1.76115912866801e-05\\
99.14	1.70226057255845e-05\\
99.15	1.64421345168805e-05\\
99.16	1.58702182876998e-05\\
99.17	1.53068957691096e-05\\
99.18	1.47522130564508e-05\\
99.19	1.42062267174219e-05\\
99.2	1.36689923055655e-05\\
99.21	1.31405642745074e-05\\
99.22	1.26210036445577e-05\\
99.23	1.21103976940592e-05\\
99.24	1.16088345337613e-05\\
99.25	1.11164031182942e-05\\
99.26	1.06331932580306e-05\\
99.27	1.01592956314039e-05\\
99.28	9.6948017976814e-06\\
99.29	9.23980421022562e-06\\
99.3	8.79439623028763e-06\\
99.31	8.35867214133909e-06\\
99.32	7.93272716396315e-06\\
99.33	7.51665747139585e-06\\
99.34	7.11056020567293e-06\\
99.35	6.71453349447744e-06\\
99.36	6.32867646870733e-06\\
99.37	5.95308927810352e-06\\
99.38	5.58787308975114e-06\\
99.39	5.23313010598533e-06\\
99.4	4.88896356993541e-06\\
99.41	4.55547778330928e-06\\
99.42	4.23277812370594e-06\\
99.43	3.92097106434626e-06\\
99.44	3.62016419468321e-06\\
99.45	3.33046620878789e-06\\
99.46	3.05198690398953e-06\\
99.47	2.78483719876564e-06\\
99.48	2.52912911125232e-06\\
99.49	2.284975724548e-06\\
99.5	2.05249119692753e-06\\
99.51	1.83179077218112e-06\\
99.52	1.62299079003135e-06\\
99.53	1.42620869666987e-06\\
99.54	1.24156305541731e-06\\
99.55	1.06917355746118e-06\\
99.56	9.09161032741287e-07\\
99.57	7.61647460904499e-07\\
99.58	6.2675598241739e-07\\
99.59	5.04610909760408e-07\\
99.6	3.95337738760826e-07\\
99.61	2.99063160008953e-07\\
99.62	2.1591507044609e-07\\
99.63	1.46022585009731e-07\\
99.64	8.95160484556956e-08\\
99.65	4.65270472548712e-08\\
99.66	1.71884216391277e-08\\
99.67	1.63427776520009e-09\\
99.68	0\\
99.69	0\\
99.7	0\\
99.71	0\\
99.72	0\\
99.73	0\\
99.74	0\\
99.75	0\\
99.76	0\\
99.77	0\\
99.78	0\\
99.79	0\\
99.8	0\\
99.81	0\\
99.82	0\\
99.83	0\\
99.84	0\\
99.85	0\\
99.86	0\\
99.87	0\\
99.88	0\\
99.89	0\\
99.9	0\\
99.91	0\\
99.92	0\\
99.93	0\\
99.94	0\\
99.95	0\\
99.96	0\\
99.97	0\\
99.98	0\\
99.99	0\\
100	0\\
};
\addlegendentry{$q=-4$};

\addplot [color=mycolor1,dashed]
  table[row sep=crcr]{%
0.01	0.0070186501501409\\
1.01	0.00701068385965739\\
2.01	0.00700232779984816\\
3.01	0.006993562461304\\
4.01	0.00698436732943601\\
5.01	0.00697472083459523\\
6.01	0.00696460030094368\\
7.01	0.0069539818940995\\
8.01	0.00694284056726258\\
9.01	0.0069311500048925\\
10.01	0.00691888256247217\\
11.01	0.00690600920241112\\
12.01	0.00689249942714721\\
13.01	0.00687832120965122\\
14.01	0.00686344092132698\\
15.01	0.00684782325730622\\
16.01	0.00683143115915048\\
17.01	0.00681422573498196\\
18.01	0.0067961661770761\\
19.01	0.00677720967695822\\
20.01	0.00675731133805484\\
21.01	0.00673642408595915\\
22.01	0.00671449857637999\\
23.01	0.00669148310086328\\
24.01	0.00666732349040707\\
25.01	0.00664196301715866\\
26.01	0.00661534229450709\\
27.01	0.00658739917611283\\
28.01	0.00655806865481995\\
29.01	0.00652728276308499\\
30.01	0.00649497047770828\\
31.01	0.00646105763353036\\
32.01	0.0064254668537741\\
33.01	0.00638811750948853\\
34.01	0.00634892572806825\\
35.01	0.00630780448239685\\
36.01	0.00626466380683602\\
37.01	0.00621941120759202\\
38.01	0.00617195238109056\\
39.01	0.00612219215013244\\
40.01	0.00607003542773831\\
41.01	0.00601538837875547\\
42.01	0.00595815987264445\\
43.01	0.00589826325585382\\
44.01	0.00583561822829223\\
45.01	0.00577014956628841\\
46.01	0.00570174280475895\\
47.01	0.0056302266636106\\
48.01	0.00555554040879982\\
49.01	0.00547765856717402\\
50.01	0.00539658492007209\\
51.01	0.00531235803293239\\
52.01	0.00522504130473293\\
53.01	0.00513472583247616\\
54.01	0.00504163298858386\\
55.01	0.00494611778859616\\
56.01	0.00484871241380105\\
57.01	0.00475018608849895\\
58.01	0.00465162432451925\\
59.01	0.00455453349731493\\
60.01	0.00445886540992659\\
61.01	0.0043591992129649\\
62.01	0.00425526116606771\\
63.01	0.00414746353248364\\
64.01	0.00403653085849659\\
65.01	0.00392370339757333\\
66.01	0.00381027559922914\\
67.01	0.00369676154302832\\
68.01	0.00358371865321317\\
69.01	0.00347179165468971\\
70.01	0.00336171774942204\\
71.01	0.00325432753083376\\
72.01	0.00315028558852393\\
73.01	0.0030483274889385\\
74.01	0.00294799391403962\\
75.01	0.0028487361812953\\
76.01	0.00274930729110097\\
77.01	0.00264932492597471\\
78.01	0.00254867975966727\\
79.01	0.0024470542028248\\
80.01	0.00234391771000994\\
81.01	0.00223902237671757\\
82.01	0.00213225833422466\\
83.01	0.00202350108567132\\
84.01	0.00191261613460473\\
85.01	0.00179948973689244\\
86.01	0.00168408845874976\\
87.01	0.00156645972418092\\
88.01	0.00144666674474913\\
89.01	0.00132478455082746\\
90.01	0.00120091154714239\\
91.01	0.0010751732144873\\
92.01	0.000947717112789589\\
93.01	0.000818715889138384\\
94.01	0.000688377198913548\\
95.01	0.000556952354343623\\
96.01	0.000424750104796532\\
97.01	0.00029215908729637\\
98.01	0.000159680719535645\\
99.01	2.96570498975329e-05\\
99.02	2.85693280234946e-05\\
99.03	2.75032330208091e-05\\
99.04	2.64589315094269e-05\\
99.05	2.54499131151879e-05\\
99.06	2.44715387902536e-05\\
99.07	2.35240465486578e-05\\
99.08	2.2607677821718e-05\\
99.09	2.17226775639217e-05\\
99.1	2.08671890345388e-05\\
99.11	2.00407137099859e-05\\
99.12	1.92434815356798e-05\\
99.13	1.84757259127401e-05\\
99.14	1.7737653948394e-05\\
99.15	1.70291623237742e-05\\
99.16	1.63505144308868e-05\\
99.17	1.57019777590004e-05\\
99.18	1.50801239827813e-05\\
99.19	1.44804907248487e-05\\
99.2	1.39032743025295e-05\\
99.21	1.33486735977156e-05\\
99.22	1.28139271072334e-05\\
99.23	1.22890135552806e-05\\
99.24	1.17740031484286e-05\\
99.25	1.12689654332906e-05\\
99.26	1.07739692289735e-05\\
99.27	1.02890825562797e-05\\
99.28	9.81437256350297e-06\\
99.29	9.34990544862631e-06\\
99.3	8.89574637778952e-06\\
99.31	8.45195939980052e-06\\
99.32	8.01860735655247e-06\\
99.33	7.59575178907977e-06\\
99.34	7.18345283908656e-06\\
99.35	6.78176913591205e-06\\
99.36	6.39075768814855e-06\\
99.37	6.01047517948851e-06\\
99.38	5.64098714129178e-06\\
99.39	5.28235774042984e-06\\
99.4	4.93465549333653e-06\\
99.41	4.59794786000257e-06\\
99.42	4.27230170910176e-06\\
99.43	3.95778245825679e-06\\
99.44	3.65445395983724e-06\\
99.45	3.36239097627374e-06\\
99.46	3.08167491083755e-06\\
99.47	2.81238642052886e-06\\
99.48	2.55461999874342e-06\\
99.49	2.30848818367178e-06\\
99.5	2.07410454113217e-06\\
99.51	1.85158367354936e-06\\
99.52	1.64104122908626e-06\\
99.53	1.44259391091431e-06\\
99.54	1.25635948660871e-06\\
99.55	1.08245679775187e-06\\
99.56	9.21005769639188e-07\\
99.57	7.72127421229396e-07\\
99.58	6.35943875214665e-07\\
99.59	5.1257836833872e-07\\
99.6	4.02155261886722e-07\\
99.61	3.04800052421464e-07\\
99.62	2.20639382726012e-07\\
99.63	1.49801053020426e-07\\
99.64	9.24140324074646e-08\\
99.65	4.86084706270706e-08\\
99.66	1.85157100797417e-08\\
99.67	2.26829817602525e-09\\
99.68	0\\
99.69	0\\
99.7	0\\
99.71	0\\
99.72	0\\
99.73	0\\
99.74	0\\
99.75	0\\
99.76	0\\
99.77	0\\
99.78	0\\
99.79	0\\
99.8	0\\
99.81	0\\
99.82	0\\
99.83	0\\
99.84	0\\
99.85	0\\
99.86	0\\
99.87	0\\
99.88	0\\
99.89	0\\
99.9	0\\
99.91	0\\
99.92	0\\
99.93	0\\
99.94	0\\
99.95	0\\
99.96	0\\
99.97	0\\
99.98	0\\
99.99	0\\
100	0\\
};
\addlegendentry{$q=-3$};

\addplot [color=red,dashed]
  table[row sep=crcr]{%
0.01	0.00976935093545758\\
1.01	0.00976421015741647\\
2.01	0.0097588215886309\\
3.01	0.00975317302442196\\
4.01	0.00974725162725591\\
5.01	0.00974104388983768\\
6.01	0.00973453559548323\\
7.01	0.00972771177549045\\
8.01	0.00972055666319548\\
9.01	0.00971305364437342\\
10.01	0.00970518520362485\\
11.01	0.0096969328663622\\
12.01	0.00968827713594452\\
13.01	0.00967919742543643\\
14.01	0.00966967198339799\\
15.01	0.00965967781303261\\
16.01	0.00964919058392876\\
17.01	0.00963818453552645\\
18.01	0.00962663237131716\\
19.01	0.00961450514264473\\
20.01	0.0096017721208111\\
21.01	0.0095884006559997\\
22.01	0.00957435602130762\\
23.01	0.00955960123991775\\
24.01	0.00954409689313871\\
25.01	0.00952780090668435\\
26.01	0.00951066831214459\\
27.01	0.00949265098010407\\
28.01	0.00947369732077701\\
29.01	0.00945375194732381\\
30.01	0.00943275529617063\\
31.01	0.00941064319762986\\
32.01	0.0093873463888656\\
33.01	0.00936278995970001\\
34.01	0.00933689271983622\\
35.01	0.00930956647373386\\
36.01	0.00928071518658991\\
37.01	0.00925023401925399\\
38.01	0.00921800819881499\\
39.01	0.00918391170311225\\
40.01	0.00914780571916437\\
41.01	0.00910953682961546\\
42.01	0.00906893487004464\\
43.01	0.00902581036899637\\
44.01	0.00897995137612937\\
45.01	0.00893111902912432\\
46.01	0.00887904107466826\\
47.01	0.00882341174230693\\
48.01	0.00876388838872182\\
49.01	0.00870008027497436\\
50.01	0.00863153938499928\\
51.01	0.00855774912691809\\
52.01	0.00847810958725209\\
53.01	0.00839192360633966\\
54.01	0.00829837906931216\\
55.01	0.00819652145168077\\
56.01	0.00808522218423392\\
57.01	0.00796313916492984\\
58.01	0.00782866744072175\\
59.01	0.00767988038378758\\
60.01	0.00751656545295311\\
61.01	0.00734383749457317\\
62.01	0.00716163614187467\\
63.01	0.00696939652633905\\
64.01	0.0067665180071746\\
65.01	0.00655237162172994\\
66.01	0.00632629300831234\\
67.01	0.00608748509419098\\
68.01	0.0058350537733886\\
69.01	0.00556801218100254\\
70.01	0.00528534038425278\\
71.01	0.00498602044039237\\
72.01	0.00472967698385689\\
73.01	0.00454160306231707\\
74.01	0.0043497707016179\\
75.01	0.00415631635582206\\
76.01	0.0039643806323788\\
77.01	0.0037765873398881\\
78.01	0.00359581560290177\\
79.01	0.00342581610811684\\
80.01	0.0032641071483379\\
81.01	0.00310160122367578\\
82.01	0.00293840060970903\\
83.01	0.0027746067283078\\
84.01	0.00261009244484884\\
85.01	0.00244433973219168\\
86.01	0.00227619013466552\\
87.01	0.00210554807654228\\
88.01	0.00193279979337896\\
89.01	0.00175819184561082\\
90.01	0.00158171088518645\\
91.01	0.00140375913670724\\
92.01	0.00122509437063391\\
93.01	0.00104655139364088\\
94.01	0.000869090499221242\\
95.01	0.00069399932017805\\
96.01	0.000522799385888874\\
97.01	0.000357262206438121\\
98.01	0.000199424406982189\\
99.01	5.15714273097625e-05\\
99.02	5.01513467955263e-05\\
99.03	4.87324870754452e-05\\
99.04	4.73148484967465e-05\\
99.05	4.58997202600082e-05\\
99.06	4.44921994131667e-05\\
99.07	4.30923253333e-05\\
99.08	4.17001378667011e-05\\
99.09	4.03156773406757e-05\\
99.1	3.89410853922963e-05\\
99.11	3.75771478830951e-05\\
99.12	3.62239262758513e-05\\
99.13	3.48814827721682e-05\\
99.14	3.35499100302342e-05\\
99.15	3.22296239078142e-05\\
99.16	3.09206882554737e-05\\
99.17	2.96231677817966e-05\\
99.18	2.83408197653007e-05\\
99.19	2.70784264258875e-05\\
99.2	2.58361154154719e-05\\
99.21	2.4614016062012e-05\\
99.22	2.34152131222843e-05\\
99.23	2.22500227116482e-05\\
99.24	2.11186718924554e-05\\
99.25	2.00213908415521e-05\\
99.26	1.89584129430577e-05\\
99.27	1.79299748844716e-05\\
99.28	1.69363167562345e-05\\
99.29	1.59776821549128e-05\\
99.3	1.50543182901221e-05\\
99.31	1.4166476095397e-05\\
99.32	1.33144103431227e-05\\
99.33	1.24983797637635e-05\\
99.34	1.17186471695323e-05\\
99.35	1.09755131177357e-05\\
99.36	1.02692828861141e-05\\
99.37	9.59960440220786e-06\\
99.38	8.96184452921049e-06\\
99.39	8.35626969543889e-06\\
99.4	7.78080695248116e-06\\
99.41	7.2356194598059e-06\\
99.42	6.72064744043632e-06\\
99.43	6.23615102932421e-06\\
99.44	5.78239536754878e-06\\
99.45	5.35523345221965e-06\\
99.46	4.9522228167849e-06\\
99.47	4.57357835185676e-06\\
99.48	4.21454314757791e-06\\
99.49	3.86924434524726e-06\\
99.5	3.53780285034661e-06\\
99.51	3.22034032447863e-06\\
99.52	2.91697917155324e-06\\
99.53	2.62784252295402e-06\\
99.54	2.35305422169839e-06\\
99.55	2.09273880545299e-06\\
99.56	1.84702148846305e-06\\
99.57	1.61602814221368e-06\\
99.58	1.39988527489768e-06\\
99.59	1.19872000950774e-06\\
99.6	1.01266006058941e-06\\
99.61	8.41833709516043e-07\\
99.62	6.86369778247695e-07\\
99.63	5.46397601500909e-07\\
99.64	4.22046997228948e-07\\
99.65	3.13448235346458e-07\\
99.66	2.20732004603194e-07\\
99.67	1.44029377518312e-07\\
99.68	8.3471773252089e-08\\
99.69	3.91920583538846e-08\\
99.7	1.13235991017896e-08\\
99.71	0\\
99.72	0\\
99.73	0\\
99.74	0\\
99.75	0\\
99.76	0\\
99.77	0\\
99.78	0\\
99.79	0\\
99.8	0\\
99.81	0\\
99.82	0\\
99.83	0\\
99.84	0\\
99.85	0\\
99.86	0\\
99.87	0\\
99.88	0\\
99.89	0\\
99.9	0\\
99.91	0\\
99.92	0\\
99.93	0\\
99.94	0\\
99.95	0\\
99.96	0\\
99.97	0\\
99.98	0\\
99.99	0\\
100	0\\
};
\addlegendentry{$q=-2$};

\addplot [color=blue,dashed]
  table[row sep=crcr]{%
0.01	0.01\\
1.01	0.01\\
2.01	0.01\\
3.01	0.01\\
4.01	0.01\\
5.01	0.01\\
6.01	0.01\\
7.01	0.01\\
8.01	0.01\\
9.01	0.01\\
10.01	0.01\\
11.01	0.01\\
12.01	0.01\\
13.01	0.01\\
14.01	0.01\\
15.01	0.01\\
16.01	0.01\\
17.01	0.01\\
18.01	0.01\\
19.01	0.01\\
20.01	0.01\\
21.01	0.01\\
22.01	0.01\\
23.01	0.01\\
24.01	0.01\\
25.01	0.01\\
26.01	0.01\\
27.01	0.01\\
28.01	0.01\\
29.01	0.01\\
30.01	0.01\\
31.01	0.01\\
32.01	0.01\\
33.01	0.01\\
34.01	0.01\\
35.01	0.01\\
36.01	0.01\\
37.01	0.01\\
38.01	0.01\\
39.01	0.01\\
40.01	0.01\\
41.01	0.01\\
42.01	0.01\\
43.01	0.01\\
44.01	0.01\\
45.01	0.01\\
46.01	0.01\\
47.01	0.01\\
48.01	0.01\\
49.01	0.01\\
50.01	0.01\\
51.01	0.01\\
52.01	0.01\\
53.01	0.01\\
54.01	0.01\\
55.01	0.01\\
56.01	0.01\\
57.01	0.01\\
58.01	0.01\\
59.01	0.01\\
60.01	0.01\\
61.01	0.01\\
62.01	0.01\\
63.01	0.01\\
64.01	0.01\\
65.01	0.01\\
66.01	0.01\\
67.01	0.01\\
68.01	0.01\\
69.01	0.01\\
70.01	0.01\\
71.01	0.01\\
72.01	0.00993962270142486\\
73.01	0.0097947536949604\\
74.01	0.00963774812468075\\
75.01	0.00946688035474884\\
76.01	0.00928006638968593\\
77.01	0.00907478347136383\\
78.01	0.00884798695145539\\
79.01	0.00859600179908861\\
80.01	0.00832161547787609\\
81.01	0.00803388501022849\\
82.01	0.00773254790799095\\
83.01	0.00741738634259809\\
84.01	0.00708844456302999\\
85.01	0.00674616946310022\\
86.01	0.00639159211134382\\
87.01	0.00602461642880567\\
88.01	0.00564468650640655\\
89.01	0.00525122367465925\\
90.01	0.00484365353674161\\
91.01	0.00442144174622584\\
92.01	0.00398409351132958\\
93.01	0.00353116349211519\\
94.01	0.00306230474189923\\
95.01	0.00257731619586271\\
96.01	0.00207620154151122\\
97.01	0.00155927909521823\\
98.01	0.00102748243795499\\
99.01	0.000484002402631955\\
99.02	0.000478539426080221\\
99.03	0.000473076568376028\\
99.04	0.000467613860715534\\
99.05	0.000462151334831507\\
99.06	0.000456689022643036\\
99.07	0.000451226956759679\\
99.08	0.0004457651704979\\
99.09	0.000440303697897944\\
99.1	0.000434842573538527\\
99.11	0.000429381832685427\\
99.12	0.000423921511379667\\
99.13	0.000418461646456252\\
99.14	0.00041300227556062\\
99.15	0.000407543437137919\\
99.16	0.000402085170487329\\
99.17	0.00039662751578293\\
99.18	0.000391170513739127\\
99.19	0.00038571420553437\\
99.2	0.000380258633282306\\
99.21	0.000374803840055017\\
99.22	0.000369349869622499\\
99.23	0.000363896765781177\\
99.24	0.000358444573350736\\
99.25	0.000352993338199973\\
99.26	0.000347543107273404\\
99.27	0.000342093928618651\\
99.28	0.000336645851414607\\
99.29	0.000331198926000449\\
99.3	0.000325753203905484\\
99.31	0.000320308737879896\\
99.32	0.000314865581926406\\
99.33	0.000309423791332876\\
99.34	0.000303983422705911\\
99.35	0.000298544533995989\\
99.36	0.000293107184533013\\
99.37	0.000287672093871667\\
99.38	0.000282244243706301\\
99.39	0.000276823722011698\\
99.4	0.000271412954483807\\
99.41	0.000266012136295681\\
99.42	0.000260621683634504\\
99.43	0.000255241698840376\\
99.44	0.000249872285511276\\
99.45	0.000244517949873808\\
99.46	0.000239181489765374\\
99.47	0.000233863051317179\\
99.48	0.000228567739209068\\
99.49	0.000223301755739001\\
99.5	0.000218065313555817\\
99.51	0.000212858628749365\\
99.52	0.000207681920948452\\
99.53	0.000202535413421905\\
99.54	0.000197419333182835\\
99.55	0.000192333911096272\\
99.56	0.000187279381990244\\
99.57	0.000182255984770466\\
99.58	0.000177263962538754\\
99.59	0.00017230356271533\\
99.6	0.000167375037165131\\
99.61	0.000162478642328327\\
99.62	0.000157614639355172\\
99.63	0.000152783294245387\\
99.64	0.000147984877992261\\
99.65	0.000143219666731637\\
99.66	0.000138487941896021\\
99.67	0.000133789990373995\\
99.68	0.000129126104675206\\
99.69	0.000124496583100996\\
99.7	0.000119901729921192\\
99.71	0.000115341855557199\\
99.72	0.000110817277287002\\
99.73	0.000106328320030035\\
99.74	0.000101875315463656\\
99.75	9.74586022346059e-05\\
99.76	9.30785261783317e-05\\
99.77	8.87354405466265e-05\\
99.78	8.44297062438867e-05\\
99.79	8.01616920725161e-05\\
99.8	7.59317749878478e-05\\
99.81	7.17403403631186e-05\\
99.82	6.75877822649829e-05\\
99.83	6.34745037401234e-05\\
99.84	5.94009171135564e-05\\
99.85	5.53674442992213e-05\\
99.86	5.13745171235643e-05\\
99.87	4.74225776628031e-05\\
99.88	4.35120785946402e-05\\
99.89	3.96434835652562e-05\\
99.9	3.58172675724491e-05\\
99.91	3.20339186857307e-05\\
99.92	2.82939389100248e-05\\
99.93	2.45978428235222e-05\\
99.94	2.09461580345793e-05\\
99.95	1.73394256602066e-05\\
99.96	1.37782008275116e-05\\
99.97	1.02630531995645e-05\\
99.98	6.79456752728652e-06\\
99.99	3.37334422906808e-06\\
100	0\\
};
\addlegendentry{$q=-1$};

\addplot [color=black,solid]
  table[row sep=crcr]{%
0.01	0\\
1.01	0\\
2.01	0\\
3.01	0\\
4.01	0\\
5.01	0\\
6.01	0\\
7.01	0\\
8.01	0\\
9.01	0\\
10.01	0\\
11.01	0\\
12.01	0\\
13.01	0\\
14.01	0\\
15.01	0\\
16.01	0\\
17.01	0\\
18.01	0\\
19.01	0\\
20.01	0\\
21.01	0\\
22.01	0\\
23.01	0\\
24.01	0\\
25.01	0\\
26.01	0\\
27.01	0\\
28.01	0\\
29.01	0\\
30.01	0\\
31.01	0\\
32.01	0\\
33.01	0\\
34.01	0\\
35.01	0\\
36.01	0\\
37.01	0\\
38.01	0\\
39.01	0\\
40.01	0\\
41.01	0\\
42.01	0\\
43.01	0\\
44.01	0\\
45.01	0\\
46.01	0\\
47.01	0\\
48.01	0\\
49.01	0\\
50.01	0\\
51.01	0\\
52.01	0\\
53.01	0\\
54.01	0\\
55.01	0\\
56.01	0\\
57.01	0\\
58.01	0\\
59.01	0\\
60.01	0\\
61.01	0\\
62.01	0\\
63.01	0\\
64.01	0\\
65.01	0\\
66.01	0\\
67.01	0\\
68.01	0\\
69.01	0\\
70.01	0\\
71.01	0\\
72.01	0\\
73.01	0\\
74.01	0\\
75.01	0\\
76.01	0\\
77.01	0\\
78.01	9.04769124136298e-05\\
79.01	0.000333505182494108\\
80.01	0.000594652620758983\\
81.01	0.000876535109286314\\
82.01	0.00118230198421931\\
83.01	0.00151576720682256\\
84.01	0.00187945949896115\\
85.01	0.00226196939622701\\
86.01	0.00266118797422353\\
87.01	0.00307770745573711\\
88.01	0.00351192518159225\\
89.01	0.00396378885283169\\
90.01	0.00443228837130114\\
91.01	0.00491777491496314\\
92.01	0.00542085737592643\\
93.01	0.00594205496899957\\
94.01	0.00648174434477769\\
95.01	0.00704007988672598\\
96.01	0.00761688659932939\\
97.01	0.00821145768662725\\
98.01	0.00882212435172335\\
99.01	0.00944376771779301\\
99.02	0.00944999042923145\\
99.03	0.0094562121593535\\
99.04	0.00946243285656591\\
99.05	0.00946865246809523\\
99.06	0.0094748709399575\\
99.07	0.00948108821692697\\
99.08	0.00948730424250401\\
99.09	0.009493518958882\\
99.1	0.00949973230715658\\
99.11	0.00950594422709603\\
99.12	0.00951215465705893\\
99.13	0.00951836353395632\\
99.14	0.0095245707932135\\
99.15	0.00953077636873065\\
99.16	0.00953698019292056\\
99.17	0.00954318219709688\\
99.18	0.00954938231134147\\
99.19	0.0095555804640455\\
99.2	0.00956177658186387\\
99.21	0.00956797058966803\\
99.22	0.00957416241049743\\
99.23	0.00958035196550946\\
99.24	0.00958653917392775\\
99.25	0.00959272395298881\\
99.26	0.00959890621788724\\
99.27	0.00960508588171903\\
99.28	0.00961126285542295\\
99.29	0.00961743704771994\\
99.3	0.00962360836505055\\
99.31	0.00962977270185325\\
99.32	0.0096359270057073\\
99.33	0.00964207113242455\\
99.34	0.00964820493489091\\
99.35	0.00965432826298612\\
99.36	0.00966044096350087\\
99.37	0.00966654288005117\\
99.38	0.00967263241561898\\
99.39	0.00967870861727818\\
99.4	0.0096847713046778\\
99.41	0.0096908202937782\\
99.42	0.0096968554007452\\
99.43	0.00970287370872296\\
99.44	0.00970887190131976\\
99.45	0.00971484789971759\\
99.46	0.0097208014747299\\
99.47	0.00972673239257158\\
99.48	0.00973264041472378\\
99.49	0.00973852529779398\\
99.5	0.00974438679337096\\
99.51	0.0097502246478746\\
99.52	0.00975603860240025\\
99.53	0.00976182839255727\\
99.54	0.00976759374830175\\
99.55	0.00977333439376287\\
99.56	0.00977905004706274\\
99.57	0.00978474042012944\\
99.58	0.00979040521850278\\
99.59	0.00979604414113263\\
99.6	0.00980165688016925\\
99.61	0.00980724312074539\\
99.62	0.00981280254074964\\
99.63	0.00981833481059062\\
99.64	0.00982383959295151\\
99.65	0.00982931654253445\\
99.66	0.00983476530579426\\
99.67	0.00984018552066089\\
99.68	0.00984557681624994\\
99.69	0.00985093881256365\\
99.7	0.00985627112017661\\
99.71	0.00986157333990662\\
99.72	0.00986684506247061\\
99.73	0.009872085868125\\
99.74	0.00987729532628936\\
99.75	0.00988247299515249\\
99.76	0.00988761842125981\\
99.77	0.00989273113908086\\
99.78	0.00989781067055575\\
99.79	0.00990285652461904\\
99.8	0.00990786819669979\\
99.81	0.00991284516819603\\
99.82	0.00991778690592203\\
99.83	0.00992269286152642\\
99.84	0.00992756247087925\\
99.85	0.00993239515342571\\
99.86	0.00993719031150409\\
99.87	0.00994194732962551\\
99.88	0.00994666557371232\\
99.89	0.00995134439029238\\
99.9	0.00995598310564544\\
99.91	0.00996058102489816\\
99.92	0.00996513743106351\\
99.93	0.00996965158402005\\
99.94	0.00997412271942604\\
99.95	0.00997855004756298\\
99.96	0.00998293275210229\\
99.97	0.00998726998878853\\
99.98	0.00999156088403149\\
99.99	0.00999580453339885\\
100	0.01\\
};
\addlegendentry{$q=0$};

\addplot [color=blue,solid]
  table[row sep=crcr]{%
0.01	0\\
1.01	0\\
2.01	0\\
3.01	0\\
4.01	0\\
5.01	0\\
6.01	0\\
7.01	0\\
8.01	0\\
9.01	0\\
10.01	0\\
11.01	0\\
12.01	0\\
13.01	0\\
14.01	0\\
15.01	0\\
16.01	0\\
17.01	0\\
18.01	0\\
19.01	0\\
20.01	0\\
21.01	0\\
22.01	0\\
23.01	0\\
24.01	0\\
25.01	0\\
26.01	0\\
27.01	0\\
28.01	0\\
29.01	0\\
30.01	0\\
31.01	0\\
32.01	0\\
33.01	0\\
34.01	0\\
35.01	0\\
36.01	0\\
37.01	0\\
38.01	0\\
39.01	0\\
40.01	0\\
41.01	0\\
42.01	0\\
43.01	0\\
44.01	0\\
45.01	0\\
46.01	0\\
47.01	0\\
48.01	0\\
49.01	0\\
50.01	0\\
51.01	0\\
52.01	0\\
53.01	0\\
54.01	0\\
55.01	0\\
56.01	0\\
57.01	9.60420732154521e-06\\
58.01	0.000135229208699593\\
59.01	0.000267619698513074\\
60.01	0.000407377997348555\\
61.01	0.000555189437212708\\
62.01	0.000711837486274149\\
63.01	0.000878222137925787\\
64.01	0.00105538212964338\\
65.01	0.00124452076074874\\
66.01	0.00144703101524147\\
67.01	0.00166450581611247\\
68.01	0.00189879061589581\\
69.01	0.00215216061275922\\
70.01	0.00242734934410092\\
71.01	0.0027276214739467\\
72.01	0.0030553224765646\\
73.01	0.00340170047820055\\
74.01	0.00376536584380919\\
75.01	0.00414751595215838\\
76.01	0.00454941555165311\\
77.01	0.00497258007181894\\
78.01	0.00532787110404776\\
79.01	0.00555176416685463\\
80.01	0.00577864355419245\\
81.01	0.006006186134476\\
82.01	0.00623086932827717\\
83.01	0.00644742715515898\\
84.01	0.00665294230431784\\
85.01	0.00685876022147589\\
86.01	0.00706732051874468\\
87.01	0.00727830213096558\\
88.01	0.00749147483156968\\
89.01	0.00770693951350276\\
90.01	0.00792572199941495\\
91.01	0.00814741745533618\\
92.01	0.00837114551457187\\
93.01	0.00859584024134704\\
94.01	0.00882030457488775\\
95.01	0.00904314111888931\\
96.01	0.00926292729381968\\
97.01	0.00947785139077141\\
98.01	0.00968577898191207\\
99.01	0.00988448798209627\\
99.02	0.00988638323582952\\
99.03	0.00988826925410972\\
99.04	0.00989014597140742\\
99.05	0.00989201332144849\\
99.06	0.00989387123719858\\
99.07	0.00989571965084699\\
99.08	0.00989755849379011\\
99.09	0.00989938769661441\\
99.1	0.00990120495110082\\
99.11	0.00990300973914664\\
99.12	0.00990480196679558\\
99.13	0.00990658154135368\\
99.14	0.00990834836890682\\
99.15	0.00991010235429178\\
99.16	0.00991184267987425\\
99.17	0.00991356456016402\\
99.18	0.00991526401895467\\
99.19	0.00991694089850678\\
99.2	0.00991859503883935\\
99.21	0.00992022627766874\\
99.22	0.0099218344503455\\
99.23	0.00992341938978893\\
99.24	0.00992498092641946\\
99.25	0.00992651888808863\\
99.26	0.00992803309951846\\
99.27	0.00992952338219693\\
99.28	0.00993098955483227\\
99.29	0.00993243143327257\\
99.3	0.00993384883042234\\
99.31	0.00993524557604533\\
99.32	0.00993662444919607\\
99.33	0.00993798531576456\\
99.34	0.00993932803949575\\
99.35	0.0099406524819256\\
99.36	0.00994195850231464\\
99.37	0.00994324595757886\\
99.38	0.00994451614662083\\
99.39	0.00994576972259023\\
99.4	0.00994700656056504\\
99.41	0.00994822653365905\\
99.42	0.00994942951093435\\
99.43	0.00995061809883822\\
99.44	0.00995179531099282\\
99.45	0.00995296292686011\\
99.46	0.00995412087134393\\
99.47	0.00995526906889685\\
99.48	0.00995640744352824\\
99.49	0.00995753591881294\\
99.5	0.00995865441790052\\
99.51	0.0099597628635252\\
99.52	0.00996086117801654\\
99.53	0.00996194928331074\\
99.54	0.00996302710096279\\
99.55	0.00996409455215938\\
99.56	0.00996515155773268\\
99.57	0.00996619803817498\\
99.58	0.00996723391365427\\
99.59	0.00996825910403079\\
99.6	0.00996927352887465\\
99.61	0.00997027710748449\\
99.62	0.00997126975890726\\
99.63	0.00997225140195929\\
99.64	0.00997322195524853\\
99.65	0.0099741813371982\\
99.66	0.00997512946607508\\
99.67	0.00997606626001613\\
99.68	0.0099769916370565\\
99.69	0.00997790550391879\\
99.7	0.00997880776335109\\
99.71	0.00997969831793289\\
99.72	0.00998057707010968\\
99.73	0.00998144392222969\\
99.74	0.00998229877658283\\
99.75	0.00998314153544192\\
99.76	0.00998397210110645\\
99.77	0.00998479037594896\\
99.78	0.0099855962624641\\
99.79	0.00998638966332074\\
99.8	0.0099871704814171\\
99.81	0.00998793861993922\\
99.82	0.00998869398242292\\
99.83	0.00998943647281947\\
99.84	0.00999016599556515\\
99.85	0.0099908824556551\\
99.86	0.00999158575872152\\
99.87	0.00999227581111662\\
99.88	0.00999295252000064\\
99.89	0.00999361579343513\\
99.9	0.00999426554048208\\
99.91	0.00999490167130895\\
99.92	0.00999552409730028\\
99.93	0.00999613273117618\\
99.94	0.0099967274871182\\
99.95	0.00999730828090301\\
99.96	0.00999787503004458\\
99.97	0.00999842765394528\\
99.98	0.00999896607405659\\
99.99	0.0099994902140502\\
100	0.01\\
};
\addlegendentry{$q=1$};

\addplot [color=red,solid]
  table[row sep=crcr]{%
0.01	0\\
1.01	4.33087425613307e-06\\
2.01	2.05020285200704e-05\\
3.01	3.73173082565334e-05\\
4.01	5.48052763838687e-05\\
5.01	7.29960398933305e-05\\
6.01	9.19213584828459e-05\\
7.01	0.000111614762468255\\
8.01	0.000132111680702728\\
9.01	0.000153449579223666\\
10.01	0.000175668111308157\\
11.01	0.000198809279582709\\
12.01	0.000222917611182281\\
13.01	0.000248040351063565\\
14.01	0.000274227716331572\\
15.01	0.000301533540658074\\
16.01	0.000330016222390317\\
17.01	0.000359735894048408\\
18.01	0.000390755143170381\\
19.01	0.000423139955802711\\
20.01	0.000456959714356601\\
21.01	0.000492287091090202\\
22.01	0.000529197792254164\\
23.01	0.000567770088567635\\
24.01	0.00060808404356465\\
25.01	0.000650220318232357\\
26.01	0.000694258383589514\\
27.01	0.000740273899091044\\
28.01	0.00078833485226942\\
29.01	0.00083849544581216\\
30.01	0.000890785390867597\\
31.01	0.000945225797411579\\
32.01	0.00100188288591958\\
33.01	0.00106087247601426\\
34.01	0.00112232279011504\\
35.01	0.00118637312521186\\
36.01	0.00125317528135806\\
37.01	0.00132289522822918\\
38.01	0.001395715057432\\
39.01	0.00147183528004889\\
40.01	0.00155147754711322\\
41.01	0.00163488791603481\\
42.01	0.00172234103603849\\
43.01	0.00181414818239593\\
44.01	0.00191068769183668\\
45.01	0.00201238274759013\\
46.01	0.00211968589415343\\
47.01	0.00223311410368929\\
48.01	0.0023532618653983\\
49.01	0.00248081704155053\\
50.01	0.00261658124103317\\
51.01	0.00276149646863232\\
52.01	0.00291667055429176\\
53.01	0.00308341547440325\\
54.01	0.00326330360926503\\
55.01	0.00345827769959624\\
56.01	0.0036707692512595\\
57.01	0.00389108189510773\\
58.01	0.00400673064997862\\
59.01	0.00412657090339202\\
60.01	0.00425060839734533\\
61.01	0.00437880024151678\\
62.01	0.00451103965405149\\
63.01	0.00464713645209664\\
64.01	0.0047867920876785\\
65.01	0.004929567695006\\
66.01	0.00507484344691566\\
67.01	0.00522176948722644\\
68.01	0.00536920427941955\\
69.01	0.00551559391410557\\
70.01	0.00565883078017027\\
71.01	0.00579607275242026\\
72.01	0.00592505417474953\\
73.01	0.00605463702034386\\
74.01	0.00618674940302178\\
75.01	0.00631968360057137\\
76.01	0.00645164677137297\\
77.01	0.00658161168452686\\
78.01	0.00670883506929929\\
79.01	0.00683535875044596\\
80.01	0.00696168976889543\\
81.01	0.0070879286995167\\
82.01	0.00721477791450747\\
83.01	0.00734403669133708\\
84.01	0.00747646503390251\\
85.01	0.00761253205366901\\
86.01	0.00775234821559366\\
87.01	0.00789601354808364\\
88.01	0.00804363171240482\\
89.01	0.00819528512828527\\
90.01	0.00835096456057964\\
91.01	0.00851056933911956\\
92.01	0.00867396499933508\\
93.01	0.00884098409914399\\
94.01	0.00901142201835746\\
95.01	0.00918503015845018\\
96.01	0.00936150594239421\\
97.01	0.00954047723941063\\
98.01	0.00972148782114007\\
99.01	0.00990006106960213\\
99.02	0.00990157576503505\\
99.03	0.00990307369943541\\
99.04	0.00990455474622719\\
99.05	0.00990601877606795\\
99.06	0.00990746565753837\\
99.07	0.0099088952572433\\
99.08	0.00991030743975208\\
99.09	0.00991170206753619\\
99.1	0.0099130812465334\\
99.11	0.00991444529138604\\
99.12	0.00991579409007456\\
99.13	0.00991712752702963\\
99.14	0.00991844548509619\\
99.15	0.00991974784548744\\
99.16	0.00992103521124347\\
99.17	0.00992231216481521\\
99.18	0.00992358248643046\\
99.19	0.00992484613615709\\
99.2	0.00992610307459429\\
99.21	0.00992735326291633\\
99.22	0.00992859666291846\\
99.23	0.00992983323706484\\
99.24	0.00993106294853881\\
99.25	0.00993228576129553\\
99.26	0.00993350164011714\\
99.27	0.00993471055067064\\
99.28	0.00993591245956828\\
99.29	0.00993710733443088\\
99.3	0.0099382951439541\\
99.31	0.00993947584960176\\
99.32	0.00994064940747986\\
99.33	0.00994181577428223\\
99.34	0.00994297490734193\\
99.35	0.00994412676468527\\
99.36	0.00994527130508878\\
99.37	0.00994640848813914\\
99.38	0.00994753826862462\\
99.39	0.00994866059884041\\
99.4	0.00994977543158854\\
99.41	0.00995088272022702\\
99.42	0.00995198241872811\\
99.43	0.00995307447422267\\
99.44	0.00995415882356421\\
99.45	0.00995523539630366\\
99.46	0.00995630412136029\\
99.47	0.00995736492701623\\
99.48	0.00995841774091096\\
99.49	0.00995946249003573\\
99.5	0.00996049910072788\\
99.51	0.00996152749866509\\
99.52	0.00996254760885948\\
99.53	0.00996355935565166\\
99.54	0.00996456266270464\\
99.55	0.00996555745299768\\
99.56	0.00996654364881991\\
99.57	0.00996752117176395\\
99.58	0.00996848994271928\\
99.59	0.00996944988186555\\
99.6	0.00997040090866567\\
99.61	0.00997134294185881\\
99.62	0.00997227589945309\\
99.63	0.00997319969871828\\
99.64	0.00997411425617811\\
99.65	0.00997501948760265\\
99.66	0.00997591529512609\\
99.67	0.00997680157986543\\
99.68	0.00997767824200409\\
99.69	0.00997854518078455\\
99.7	0.00997940229449872\\
99.71	0.00998024948047705\\
99.72	0.0099810866350772\\
99.73	0.00998191365367248\\
99.74	0.00998273043063972\\
99.75	0.00998353685934682\\
99.76	0.00998433283213985\\
99.77	0.00998511824032956\\
99.78	0.00998589297417746\\
99.79	0.00998665692288131\\
99.8	0.00998740997456001\\
99.81	0.00998815201623783\\
99.82	0.00998888293382796\\
99.83	0.00998960261211535\\
99.84	0.00999031093473873\\
99.85	0.0099910077841718\\
99.86	0.00999169304170346\\
99.87	0.00999236658741715\\
99.88	0.00999302830016909\\
99.89	0.0099936780575654\\
99.9	0.00999431573593803\\
99.91	0.00999494121031937\\
99.92	0.00999555435441553\\
99.93	0.00999615504057801\\
99.94	0.0099967431397739\\
99.95	0.00999731852155427\\
99.96	0.00999788105402068\\
99.97	0.0099984306037898\\
99.98	0.00999896703595574\\
99.99	0.0099994902140502\\
100	0.01\\
};
\addlegendentry{$q=2$};

\addplot [color=mycolor1,solid]
  table[row sep=crcr]{%
0.01	0.00284497808044895\\
1.01	0.00286358053885133\\
2.01	0.00287129926155062\\
3.01	0.00287932108641342\\
4.01	0.00288766064546655\\
5.01	0.00289633368763364\\
6.01	0.00290535722157912\\
7.01	0.00291474968346587\\
8.01	0.00292453113476467\\
9.01	0.00293472349645438\\
10.01	0.0029453508274643\\
11.01	0.00295643965718092\\
12.01	0.00296801938486431\\
13.01	0.00298012276870255\\
14.01	0.00299278663179456\\
15.01	0.00300605504441553\\
16.01	0.00302000842328301\\
17.01	0.00303473814541508\\
18.01	0.00305032251621418\\
19.01	0.00306685293628916\\
20.01	0.00308443711580778\\
21.01	0.00310320283411347\\
22.01	0.0031233026949752\\
23.01	0.00314492016665325\\
24.01	0.00316827728551013\\
25.01	0.00319364452034619\\
26.01	0.00322135345120413\\
27.01	0.00325181312046181\\
28.01	0.00328553114811884\\
29.01	0.00332314053250383\\
30.01	0.00336502271506735\\
31.01	0.00340897577803881\\
32.01	0.0034545794415943\\
33.01	0.00350189297771827\\
34.01	0.00355097426539939\\
35.01	0.00360188083358857\\
36.01	0.0036546691968992\\
37.01	0.00370939401008315\\
38.01	0.00376610699494643\\
39.01	0.00382485558054767\\
40.01	0.00388568118090434\\
41.01	0.00394861701374206\\
42.01	0.0040136853418117\\
43.01	0.00408089396233934\\
44.01	0.00415023111434892\\
45.01	0.0042216585007195\\
46.01	0.00429510471647451\\
47.01	0.00437045591469198\\
48.01	0.00444754328204564\\
49.01	0.00452612729180118\\
50.01	0.00460588139141532\\
51.01	0.00468642205166135\\
52.01	0.00476725028192061\\
53.01	0.00484755663600465\\
54.01	0.00492625267494352\\
55.01	0.00500189379822647\\
56.01	0.00507256430410409\\
57.01	0.00513858890404529\\
58.01	0.00520639375284322\\
59.01	0.00527701442725958\\
60.01	0.00535052033884895\\
61.01	0.00542695153494704\\
62.01	0.00550630393992901\\
63.01	0.00558850915363026\\
64.01	0.00567340666506521\\
65.01	0.00576070572521015\\
66.01	0.00584993310939179\\
67.01	0.00594013192365608\\
68.01	0.00603005572768021\\
69.01	0.00611941606762184\\
70.01	0.00620806487750143\\
71.01	0.00629602283762802\\
72.01	0.00638358270879835\\
73.01	0.00647110577514324\\
74.01	0.00655889653940176\\
75.01	0.00664805220961663\\
76.01	0.00673960173277042\\
77.01	0.00683371260408346\\
78.01	0.00693060269721136\\
79.01	0.00703052089712495\\
80.01	0.00713372573133154\\
81.01	0.00724050405463523\\
82.01	0.00735114671509235\\
83.01	0.00746587608382042\\
84.01	0.00758485207437221\\
85.01	0.00770821135400273\\
86.01	0.00783607605931906\\
87.01	0.0079685553154049\\
88.01	0.00810573927843018\\
89.01	0.00824769268040544\\
90.01	0.00839444993526902\\
91.01	0.00854601215628517\\
92.01	0.00870233983786851\\
93.01	0.00886334196067576\\
94.01	0.00902886233730587\\
95.01	0.00919866259882805\\
96.01	0.00937240105551164\\
97.01	0.00954960660956304\\
98.01	0.00972964659424814\\
99.01	0.00990155492092998\\
99.02	0.00990290647092698\\
99.03	0.00990425247739058\\
99.04	0.00990559291149017\\
99.05	0.00990692774516513\\
99.06	0.00990825695118009\\
99.07	0.00990958050318291\\
99.08	0.00991089837576564\\
99.09	0.00991221054452871\\
99.1	0.00991351697823863\\
99.11	0.00991481764483132\\
99.12	0.00991611251291097\\
99.13	0.00991740155180701\\
99.14	0.00991868473162849\\
99.15	0.00991996202332142\\
99.16	0.00992123339633484\\
99.17	0.00992249880599432\\
99.18	0.00992375819814401\\
99.19	0.00992501151811266\\
99.2	0.00992625871070577\\
99.21	0.00992749972019735\\
99.22	0.00992873449032155\\
99.23	0.00992996296426372\\
99.24	0.00993118508465128\\
99.25	0.00993240079354419\\
99.26	0.00993361003242496\\
99.27	0.00993481274218827\\
99.28	0.00993600886313022\\
99.29	0.00993719833493701\\
99.3	0.00993838109667319\\
99.31	0.00993955708678515\\
99.32	0.00994072624310134\\
99.33	0.00994188850282124\\
99.34	0.00994304380250382\\
99.35	0.00994419207805562\\
99.36	0.00994533326471818\\
99.37	0.00994646729705494\\
99.38	0.00994759410895921\\
99.39	0.00994871363365381\\
99.4	0.00994982580367944\\
99.41	0.00995093055088268\\
99.42	0.00995202780640331\\
99.43	0.00995311750068041\\
99.44	0.0099541995634753\\
99.45	0.00995527392388831\\
99.46	0.00995634051035226\\
99.47	0.00995739925062604\\
99.48	0.00995845007178791\\
99.49	0.00995949290022894\\
99.5	0.00996052766164622\\
99.51	0.00996155428103607\\
99.52	0.00996257268268718\\
99.53	0.00996358279017364\\
99.54	0.00996458452634792\\
99.55	0.00996557781333376\\
99.56	0.00996656257251899\\
99.57	0.00996753872454828\\
99.58	0.00996850618931578\\
99.59	0.00996946488595774\\
99.6	0.00997041473284496\\
99.61	0.00997135564757522\\
99.62	0.00997228754696565\\
99.63	0.00997321034704495\\
99.64	0.00997412396304554\\
99.65	0.00997502830878446\\
99.66	0.00997592328496542\\
99.67	0.00997680879132069\\
99.68	0.00997768472660141\\
99.69	0.00997855098856782\\
99.7	0.00997940747397935\\
99.71	0.00998025407858468\\
99.72	0.00998109069711163\\
99.73	0.00998191722325695\\
99.74	0.00998273354967606\\
99.75	0.00998353956797263\\
99.76	0.00998433516868813\\
99.77	0.00998512024129114\\
99.78	0.00998589467416675\\
99.79	0.00998665835460569\\
99.8	0.00998741116879346\\
99.81	0.00998815300179934\\
99.82	0.00998888373756528\\
99.83	0.00998960325889476\\
99.84	0.00999031144744146\\
99.85	0.00999100818369797\\
99.86	0.00999169334698432\\
99.87	0.00999236681543648\\
99.88	0.00999302846599478\\
99.89	0.00999367817439227\\
99.9	0.00999431581514299\\
99.91	0.00999494126153029\\
99.92	0.00999555438559496\\
99.93	0.00999615505812347\\
99.94	0.00999674314863612\\
99.95	0.00999731852537521\\
99.96	0.00999788105529318\\
99.97	0.00999843060404087\\
99.98	0.00999896703595574\\
99.99	0.0099994902140502\\
100	0.01\\
};
\addlegendentry{$q=3$};

\addplot [color=green,solid]
  table[row sep=crcr]{%
0.01	0.00370052694073906\\
1.01	0.00371489680974314\\
2.01	0.003729733266011\\
3.01	0.00374510263360971\\
4.01	0.00376102247641767\\
5.01	0.00377751055051478\\
6.01	0.00379458471472968\\
7.01	0.00381226281754934\\
8.01	0.00383056255504445\\
9.01	0.00384950129320284\\
10.01	0.00386909584647858\\
11.01	0.00388936220235964\\
12.01	0.00391031517917854\\
13.01	0.00393196800074026\\
14.01	0.00395433176257054\\
15.01	0.00397741469037295\\
16.01	0.00400122010200923\\
17.01	0.00402574272849096\\
18.01	0.00405096843538435\\
19.01	0.00407687252174995\\
20.01	0.00410341674042718\\
21.01	0.00413054548883419\\
22.01	0.00415818092575786\\
23.01	0.00418621669149512\\
24.01	0.004214509805859\\
25.01	0.00424287018143341\\
26.01	0.00427104700640877\\
27.01	0.00429871100749294\\
28.01	0.00432543128517476\\
29.01	0.00435064506194685\\
30.01	0.00437402415580537\\
31.01	0.00439784060195186\\
32.01	0.00442262889131399\\
33.01	0.00444842897187869\\
34.01	0.00447528152896721\\
35.01	0.00450322776987547\\
36.01	0.00453230915337759\\
37.01	0.00456256706961049\\
38.01	0.00459404247426034\\
39.01	0.00462677548880118\\
40.01	0.00466080499106544\\
41.01	0.00469616823991596\\
42.01	0.00473290060754\\
43.01	0.00477103553798772\\
44.01	0.00481060492381193\\
45.01	0.00485164020926095\\
46.01	0.00489417462481386\\
47.01	0.00493824795988176\\
48.01	0.00498391133913707\\
49.01	0.00503121817438382\\
50.01	0.00508022057616017\\
51.01	0.00513096797342706\\
52.01	0.00518350491064937\\
53.01	0.00523787114789524\\
54.01	0.00529410167668842\\
55.01	0.00535222627147672\\
56.01	0.00541226976103684\\
57.01	0.00547422262344684\\
58.01	0.00553760852552886\\
59.01	0.00560149855493645\\
60.01	0.00566565758699211\\
61.01	0.00572990498322443\\
62.01	0.00579406766460756\\
63.01	0.00585799988988189\\
64.01	0.00592161366790277\\
65.01	0.0059849244496217\\
66.01	0.00604811864370441\\
67.01	0.00611188378999621\\
68.01	0.00617726638717013\\
69.01	0.00624444233280379\\
70.01	0.00631357221222928\\
71.01	0.00638484897151334\\
72.01	0.0064584920748115\\
73.01	0.00653473707203045\\
74.01	0.0066138304420014\\
75.01	0.00669600074530322\\
76.01	0.00678141400499415\\
77.01	0.00687022464081496\\
78.01	0.0069625913449022\\
79.01	0.00705867454162927\\
80.01	0.00715863382217472\\
81.01	0.00726262479044979\\
82.01	0.00737079598694848\\
83.01	0.00748328811973331\\
84.01	0.00760023549044121\\
85.01	0.00772176484420287\\
86.01	0.0078479928303368\\
87.01	0.00797902251869752\\
88.01	0.00811493918439277\\
89.01	0.00825580530776454\\
90.01	0.00840165453957058\\
91.01	0.00855248419770224\\
92.01	0.00870824597294096\\
93.01	0.00886883461799277\\
94.01	0.0090340742261738\\
95.01	0.00920370159746236\\
96.01	0.00937734608214109\\
97.01	0.00955450515810773\\
98.01	0.00973451482958923\\
99.01	0.00990158038381976\\
99.02	0.00990292910222302\\
99.03	0.00990427259288974\\
99.04	0.0099056108080794\\
99.05	0.00990694369956448\\
99.06	0.00990827121862094\\
99.07	0.00990959331601817\\
99.08	0.00991090994200862\\
99.09	0.00991222104631686\\
99.1	0.00991352657815434\\
99.11	0.009914826486215\\
99.12	0.00991612071866583\\
99.13	0.00991740922313704\\
99.14	0.00991869194671183\\
99.15	0.00991996883591571\\
99.16	0.00992123983671282\\
99.17	0.00992250489453981\\
99.18	0.00992376395432008\\
99.19	0.00992501696045874\\
99.2	0.00992626385683757\\
99.21	0.00992750458680995\\
99.22	0.00992873909319568\\
99.23	0.00992996731827578\\
99.24	0.00993118920378736\\
99.25	0.00993240469091825\\
99.26	0.00993361372030175\\
99.27	0.00993481623201131\\
99.28	0.0099360121655551\\
99.29	0.00993720145987065\\
99.3	0.00993838405331939\\
99.31	0.00993955988368111\\
99.32	0.00994072888814847\\
99.33	0.00994189100332133\\
99.34	0.00994304616520118\\
99.35	0.00994419430918545\\
99.36	0.00994533537006185\\
99.37	0.00994646928200261\\
99.38	0.00994759597855872\\
99.39	0.00994871539265403\\
99.4	0.00994982745657927\\
99.41	0.00995093210198614\\
99.42	0.0099520292598813\\
99.43	0.0099531188606203\\
99.44	0.00995420083390134\\
99.45	0.00995527510875895\\
99.46	0.0099563416135575\\
99.47	0.00995740027598479\\
99.48	0.00995845102304542\\
99.49	0.00995949378105424\\
99.5	0.00996052847562965\\
99.51	0.00996155503168684\\
99.52	0.00996257337343097\\
99.53	0.00996358342435036\\
99.54	0.00996458510720945\\
99.55	0.00996557834404188\\
99.56	0.00996656305614335\\
99.57	0.0099675391640645\\
99.58	0.00996850658760366\\
99.59	0.00996946524579963\\
99.6	0.00997041505692424\\
99.61	0.00997135593847498\\
99.62	0.00997228780716746\\
99.63	0.00997321057892787\\
99.64	0.00997412416888529\\
99.65	0.00997502849073425\\
99.66	0.00997592344507596\\
99.67	0.00997680893153984\\
99.68	0.00997768484877415\\
99.69	0.00997855109443635\\
99.7	0.00997940756518358\\
99.71	0.00998025415666283\\
99.72	0.00998109076350121\\
99.73	0.00998191727929598\\
99.74	0.00998273359660456\\
99.75	0.00998353960693446\\
99.76	0.00998433520073301\\
99.77	0.00998512026737716\\
99.78	0.00998589469516297\\
99.79	0.00998665837129522\\
99.8	0.00998741118187676\\
99.81	0.00998815301189781\\
99.82	0.00998888374522519\\
99.83	0.0099896032645914\\
99.84	0.00999031145158363\\
99.85	0.00999100818663261\\
99.86	0.00999169334900144\\
99.87	0.00999236681677426\\
99.88	0.00999302846684477\\
99.89	0.00999367817490475\\
99.9	0.00999431581543237\\
99.91	0.00999494126168042\\
99.92	0.00999555438566448\\
99.93	0.00999615505815084\\
99.94	0.00999674314864449\\
99.95	0.0099973185253768\\
99.96	0.00999788105529324\\
99.97	0.00999843060404087\\
99.98	0.00999896703595574\\
99.99	0.0099994902140502\\
100	0.01\\
};
\addlegendentry{$q=4$};

\end{axis}
\end{tikzpicture}%

  \caption{Continuous Time}
\end{subfigure}%
\hfill%
\begin{subfigure}{.45\linewidth}
  \centering
  \setlength\figureheight{\linewidth} 
  \setlength\figurewidth{\linewidth}
  \tikzsetnextfilename{testdp_dscr_z8}
  % This file was created by matlab2tikz.
%
%The latest updates can be retrieved from
%  http://www.mathworks.com/matlabcentral/fileexchange/22022-matlab2tikz-matlab2tikz
%where you can also make suggestions and rate matlab2tikz.
%
\definecolor{mycolor1}{rgb}{0.00000,1.00000,0.14286}%
\definecolor{mycolor2}{rgb}{0.00000,1.00000,0.28571}%
\definecolor{mycolor3}{rgb}{0.00000,1.00000,0.42857}%
\definecolor{mycolor4}{rgb}{0.00000,1.00000,0.57143}%
\definecolor{mycolor5}{rgb}{0.00000,1.00000,0.71429}%
\definecolor{mycolor6}{rgb}{0.00000,1.00000,0.85714}%
\definecolor{mycolor7}{rgb}{0.00000,1.00000,1.00000}%
\definecolor{mycolor8}{rgb}{0.00000,0.87500,1.00000}%
\definecolor{mycolor9}{rgb}{0.00000,0.62500,1.00000}%
\definecolor{mycolor10}{rgb}{0.12500,0.00000,1.00000}%
\definecolor{mycolor11}{rgb}{0.25000,0.00000,1.00000}%
\definecolor{mycolor12}{rgb}{0.37500,0.00000,1.00000}%
\definecolor{mycolor13}{rgb}{0.50000,0.00000,1.00000}%
\definecolor{mycolor14}{rgb}{0.62500,0.00000,1.00000}%
\definecolor{mycolor15}{rgb}{0.75000,0.00000,1.00000}%
\definecolor{mycolor16}{rgb}{0.87500,0.00000,1.00000}%
\definecolor{mycolor17}{rgb}{1.00000,0.00000,1.00000}%
\definecolor{mycolor18}{rgb}{1.00000,0.00000,0.87500}%
\definecolor{mycolor19}{rgb}{1.00000,0.00000,0.62500}%
\definecolor{mycolor20}{rgb}{0.85714,0.00000,0.00000}%
\definecolor{mycolor21}{rgb}{0.71429,0.00000,0.00000}%
%
\begin{tikzpicture}

\begin{axis}[%
width=4.1in,
height=3.803in,
at={(0.809in,0.513in)},
scale only axis,
point meta min=0,
point meta max=1,
every outer x axis line/.append style={black},
every x tick label/.append style={font=\color{black}},
xmin=0,
xmax=600,
every outer y axis line/.append style={black},
every y tick label/.append style={font=\color{black}},
ymin=0,
ymax=0.01,
axis background/.style={fill=white},
axis x line*=bottom,
axis y line*=left,
colormap={mymap}{[1pt] rgb(0pt)=(0,1,0); rgb(7pt)=(0,1,1); rgb(15pt)=(0,0,1); rgb(23pt)=(1,0,1); rgb(31pt)=(1,0,0); rgb(38pt)=(0,0,0)},
colorbar,
colorbar style={separate axis lines,every outer x axis line/.append style={black},every x tick label/.append style={font=\color{black}},every outer y axis line/.append style={black},every y tick label/.append style={font=\color{black}},yticklabels={{-19},{-17},{-15},{-13},{-11},{-9},{-7},{-5},{-3},{-1},{1},{3},{5},{7},{9},{11},{13},{15},{17},{19}}}
]
\addplot [color=green,solid,forget plot]
  table[row sep=crcr]{%
1	0.00547780019338487\\
2	0.00547779836203247\\
3	0.00547779649665558\\
4	0.00547779459661717\\
5	0.00547779266126814\\
6	0.00547779068994721\\
7	0.00547778868198065\\
8	0.00547778663668193\\
9	0.00547778455335156\\
10	0.00547778243127691\\
11	0.00547778026973184\\
12	0.00547777806797653\\
13	0.00547777582525721\\
14	0.0054777735408058\\
15	0.00547777121383979\\
16	0.00547776884356175\\
17	0.00547776642915928\\
18	0.00547776396980463\\
19	0.00547776146465437\\
20	0.00547775891284912\\
21	0.00547775631351337\\
22	0.00547775366575492\\
23	0.00547775096866472\\
24	0.00547774822131646\\
25	0.00547774542276647\\
26	0.00547774257205312\\
27	0.00547773966819658\\
28	0.00547773671019848\\
29	0.00547773369704161\\
30	0.00547773062768945\\
31	0.00547772750108587\\
32	0.00547772431615465\\
33	0.00547772107179931\\
34	0.00547771776690257\\
35	0.00547771440032607\\
36	0.00547771097090967\\
37	0.00547770747747142\\
38	0.00547770391880689\\
39	0.00547770029368877\\
40	0.00547769660086674\\
41	0.00547769283906642\\
42	0.00547768900698951\\
43	0.00547768510331311\\
44	0.005477681126689\\
45	0.00547767707574356\\
46	0.00547767294907713\\
47	0.00547766874526338\\
48	0.00547766446284898\\
49	0.005477660100353\\
50	0.00547765565626646\\
51	0.00547765112905163\\
52	0.00547764651714156\\
53	0.00547764181893974\\
54	0.00547763703281917\\
55	0.00547763215712194\\
56	0.00547762719015876\\
57	0.00547762213020816\\
58	0.00547761697551608\\
59	0.00547761172429486\\
60	0.00547760637472321\\
61	0.00547760092494489\\
62	0.00547759537306858\\
63	0.00547758971716693\\
64	0.00547758395527595\\
65	0.00547757808539432\\
66	0.00547757210548267\\
67	0.00547756601346271\\
68	0.0054775598072167\\
69	0.00547755348458673\\
70	0.00547754704337352\\
71	0.00547754048133619\\
72	0.00547753379619117\\
73	0.00547752698561129\\
74	0.00547752004722511\\
75	0.0054775129786161\\
76	0.00547750577732178\\
77	0.0054774984408325\\
78	0.00547749096659111\\
79	0.00547748335199146\\
80	0.00547747559437789\\
81	0.00547746769104432\\
82	0.00547745963923266\\
83	0.00547745143613261\\
84	0.00547744307888013\\
85	0.00547743456455674\\
86	0.00547742589018804\\
87	0.00547741705274312\\
88	0.00547740804913314\\
89	0.00547739887621048\\
90	0.00547738953076738\\
91	0.00547738000953478\\
92	0.00547737030918148\\
93	0.00547736042631254\\
94	0.00547735035746826\\
95	0.00547734009912296\\
96	0.00547732964768352\\
97	0.00547731899948825\\
98	0.00547730815080578\\
99	0.00547729709783312\\
100	0.00547728583669486\\
101	0.0054772743634416\\
102	0.00547726267404826\\
103	0.0054772507644131\\
104	0.00547723863035594\\
105	0.00547722626761666\\
106	0.00547721367185374\\
107	0.00547720083864259\\
108	0.00547718776347421\\
109	0.00547717444175333\\
110	0.00547716086879675\\
111	0.00547714703983176\\
112	0.00547713294999438\\
113	0.00547711859432751\\
114	0.00547710396777942\\
115	0.00547708906520158\\
116	0.00547707388134694\\
117	0.00547705841086813\\
118	0.00547704264831543\\
119	0.00547702658813468\\
120	0.00547701022466556\\
121	0.00547699355213936\\
122	0.00547697656467691\\
123	0.00547695925628634\\
124	0.00547694162086135\\
125	0.00547692365217836\\
126	0.00547690534389472\\
127	0.00547688668954638\\
128	0.00547686768254531\\
129	0.00547684831617732\\
130	0.00547682858359955\\
131	0.00547680847783804\\
132	0.00547678799178508\\
133	0.00547676711819685\\
134	0.00547674584969067\\
135	0.00547672417874215\\
136	0.00547670209768282\\
137	0.00547667959869716\\
138	0.00547665667381978\\
139	0.00547663331493242\\
140	0.00547660951376113\\
141	0.00547658526187342\\
142	0.00547656055067495\\
143	0.00547653537140648\\
144	0.00547650971514078\\
145	0.00547648357277933\\
146	0.0054764569350489\\
147	0.00547642979249845\\
148	0.00547640213549555\\
149	0.00547637395422284\\
150	0.00547634523867466\\
151	0.00547631597865313\\
152	0.00547628616376466\\
153	0.0054762557834163\\
154	0.00547622482681152\\
155	0.00547619328294673\\
156	0.00547616114060697\\
157	0.00547612838836183\\
158	0.0054760950145617\\
159	0.00547606100733317\\
160	0.00547602635457487\\
161	0.00547599104395312\\
162	0.00547595506289734\\
163	0.0054759183985957\\
164	0.00547588103799036\\
165	0.0054758429677727\\
166	0.00547580417437866\\
167	0.0054757646439838\\
168	0.00547572436249831\\
169	0.00547568331556171\\
170	0.00547564148853824\\
171	0.00547559886651095\\
172	0.00547555543427678\\
173	0.00547551117634084\\
174	0.00547546607691097\\
175	0.00547542011989205\\
176	0.00547537328888027\\
177	0.00547532556715714\\
178	0.00547527693768367\\
179	0.00547522738309414\\
180	0.00547517688568993\\
181	0.00547512542743321\\
182	0.00547507298994053\\
183	0.00547501955447631\\
184	0.00547496510194596\\
185	0.00547490961288938\\
186	0.00547485306747379\\
187	0.00547479544548693\\
188	0.00547473672632967\\
189	0.00547467688900864\\
190	0.00547461591212905\\
191	0.00547455377388681\\
192	0.0054744904520609\\
193	0.00547442592400535\\
194	0.00547436016664137\\
195	0.00547429315644886\\
196	0.00547422486945841\\
197	0.00547415528124248\\
198	0.0054740843669055\\
199	0.00547401210107523\\
200	0.00547393845789294\\
201	0.00547386341100487\\
202	0.00547378693355224\\
203	0.00547370899816192\\
204	0.00547362957693638\\
205	0.00547354864144361\\
206	0.00547346616270701\\
207	0.00547338211119487\\
208	0.0054732964568095\\
209	0.00547320916887667\\
210	0.00547312021613425\\
211	0.0054730295667209\\
212	0.00547293718816452\\
213	0.0054728430473704\\
214	0.00547274711060898\\
215	0.00547264934350355\\
216	0.00547254971101772\\
217	0.00547244817744224\\
218	0.00547234470638199\\
219	0.0054722392607422\\
220	0.00547213180271464\\
221	0.00547202229376333\\
222	0.00547191069460997\\
223	0.00547179696521914\\
224	0.0054716810647825\\
225	0.00547156295170338\\
226	0.0054714425835802\\
227	0.00547131991719004\\
228	0.00547119490847116\\
229	0.00547106751250553\\
230	0.00547093768350016\\
231	0.00547080537476862\\
232	0.00547067053871114\\
233	0.00547053312679474\\
234	0.00547039308953201\\
235	0.00547025037645972\\
236	0.00547010493611618\\
237	0.00546995671601781\\
238	0.00546980566263517\\
239	0.00546965172136738\\
240	0.00546949483651604\\
241	0.0054693349512578\\
242	0.00546917200761569\\
243	0.00546900594642939\\
244	0.00546883670732378\\
245	0.0054686642286767\\
246	0.00546848844758473\\
247	0.00546830929982718\\
248	0.0054681267198294\\
249	0.00546794064062302\\
250	0.00546775099380574\\
251	0.00546755770949818\\
252	0.00546736071629964\\
253	0.00546715994124117\\
254	0.00546695530973744\\
255	0.00546674674553633\\
256	0.00546653417066647\\
257	0.00546631750538349\\
258	0.00546609666811435\\
259	0.00546587157540045\\
260	0.00546564214183956\\
261	0.00546540828002758\\
262	0.00546516990049957\\
263	0.00546492691167163\\
264	0.00546467921978485\\
265	0.00546442672885071\\
266	0.00546416934060098\\
267	0.00546390695444281\\
268	0.00546363946742059\\
269	0.00546336677418728\\
270	0.0054630887669867\\
271	0.00546280533565021\\
272	0.00546251636760866\\
273	0.00546222174792165\\
274	0.00546192135932096\\
275	0.00546161508225876\\
276	0.0054613027949394\\
277	0.00546098437332235\\
278	0.00546065969126471\\
279	0.00546032862209576\\
280	0.00545999103929571\\
281	0.00545964681381874\\
282	0.00545929581404179\\
283	0.00545893790571285\\
284	0.00545857295189752\\
285	0.00545820081292533\\
286	0.005457821346334\\
287	0.00545743440681342\\
288	0.00545703984614783\\
289	0.00545663751315714\\
290	0.00545622725363689\\
291	0.00545580891029723\\
292	0.00545538232270017\\
293	0.005454947327196\\
294	0.005454503756858\\
295	0.00545405144141609\\
296	0.00545359020718901\\
297	0.0054531198770149\\
298	0.00545264027018071\\
299	0.0054521512023498\\
300	0.00545165248548837\\
301	0.00545114392779012\\
302	0.00545062533359921\\
303	0.00545009650333209\\
304	0.00544955723339684\\
305	0.00544900731611164\\
306	0.00544844653962099\\
307	0.00544787468781012\\
308	0.00544729154021798\\
309	0.00544669687194783\\
310	0.0054460904535762\\
311	0.00544547205105978\\
312	0.00544484142564023\\
313	0.0054441983337471\\
314	0.00544354252689835\\
315	0.00544287375159881\\
316	0.00544219174923633\\
317	0.00544149625597578\\
318	0.00544078700265033\\
319	0.00544006371465047\\
320	0.0054393261118108\\
321	0.00543857390829369\\
322	0.0054378068124707\\
323	0.00543702452680095\\
324	0.00543622674770699\\
325	0.00543541316544767\\
326	0.00543458346398793\\
327	0.00543373732086547\\
328	0.0054328744070547\\
329	0.00543199438682682\\
330	0.00543109691760711\\
331	0.0054301816498285\\
332	0.00542924822678164\\
333	0.00542829628446135\\
334	0.00542732545140942\\
335	0.00542633534855325\\
336	0.00542532558904098\\
337	0.00542429577807218\\
338	0.00542324551272441\\
339	0.00542217438177564\\
340	0.00542108196552196\\
341	0.00541996783559111\\
342	0.00541883155475096\\
343	0.00541767267671351\\
344	0.00541649074593372\\
345	0.00541528529740356\\
346	0.00541405585644107\\
347	0.00541280193847348\\
348	0.00541152304881607\\
349	0.00541021868244466\\
350	0.00540888832376318\\
351	0.00540753144636543\\
352	0.00540614751279144\\
353	0.00540473597427764\\
354	0.00540329627050202\\
355	0.00540182782932307\\
356	0.00540033006651343\\
357	0.00539880238548735\\
358	0.00539724417702318\\
359	0.00539565481897998\\
360	0.00539403367600851\\
361	0.00539238009925782\\
362	0.00539069342607582\\
363	0.00538897297970559\\
364	0.00538721806897695\\
365	0.00538542798799348\\
366	0.00538360201581567\\
367	0.00538173941614002\\
368	0.00537983943697468\\
369	0.00537790131031196\\
370	0.00537592425179702\\
371	0.00537390746039434\\
372	0.00537185011805059\\
373	0.00536975138935397\\
374	0.0053676104211907\\
375	0.00536542634239583\\
376	0.0053631982633991\\
377	0.00536092527586302\\
378	0.00535860645231078\\
379	0.0053562408457403\\
380	0.00535382748921942\\
381	0.00535136539545445\\
382	0.00534885355632346\\
383	0.00534629094236728\\
384	0.00534367650224867\\
385	0.00534100916223877\\
386	0.00533828782582163\\
387	0.00533551137291279\\
388	0.00533267865854607\\
389	0.00532978851220988\\
390	0.00532683973715953\\
391	0.00532383110970253\\
392	0.00532076137845693\\
393	0.00531762926357949\\
394	0.0053144334559635\\
395	0.00531117261640312\\
396	0.00530784537472329\\
397	0.00530445032887193\\
398	0.00530098604397334\\
399	0.00529745105133906\\
400	0.0052938438474343\\
401	0.00529016289279669\\
402	0.00528640661090413\\
403	0.00528257338698845\\
404	0.00527866156679076\\
405	0.00527466945525515\\
406	0.00527059531515573\\
407	0.00526643736565234\\
408	0.0052621937807702\\
409	0.00525786268779764\\
410	0.00525344216559556\\
411	0.00524893024281307\\
412	0.00524432489600113\\
413	0.00523962404761742\\
414	0.00523482556391416\\
415	0.0052299272526993\\
416	0.00522492686096276\\
417	0.00521982207235701\\
418	0.00521461050452102\\
419	0.00520928970623665\\
420	0.00520385715440529\\
421	0.00519831025083372\\
422	0.00519264631881968\\
423	0.00518686259952791\\
424	0.00518095624813875\\
425	0.00517492432971706\\
426	0.00516876381481588\\
427	0.00516247157484287\\
428	0.00515604437712275\\
429	0.00514947887963317\\
430	0.00514277162539182\\
431	0.00513591903646833\\
432	0.00512891740759394\\
433	0.00512176289934059\\
434	0.00511445153083856\\
435	0.005106979171999\\
436	0.00509934153520695\\
437	0.00509153416644783\\
438	0.00508355243582635\\
439	0.00507539152743719\\
440	0.0050670464285423\\
441	0.00505851191800767\\
442	0.00504978255395004\\
443	0.00504085266054141\\
444	0.00503171631391671\\
445	0.00502236732712819\\
446	0.00501279923408703\\
447	0.00500300527243186\\
448	0.00499297836526077\\
449	0.00498271110166135\\
450	0.00497219571597312\\
451	0.00496142406572608\\
452	0.00495038760818735\\
453	0.0049390773754565\\
454	0.00492748394805629\\
455	0.004915597426971\\
456	0.00490340740409602\\
457	0.00489090293107581\\
458	0.00487807248652442\\
459	0.0048649039416482\\
460	0.00485138452431932\\
461	0.00483750078168705\\
462	0.00482323854146321\\
463	0.00480858287207625\\
464	0.0047935180419613\\
465	0.00477802747833561\\
466	0.00476209372589299\\
467	0.00474569840590054\\
468	0.00472882217609919\\
469	0.00471144469131011\\
470	0.00469354456297209\\
471	0.00467509931101878\\
472	0.00465608528883763\\
473	0.0046364775315969\\
474	0.00461624941539552\\
475	0.00459537193149305\\
476	0.00457381089568\\
477	0.00455151205921459\\
478	0.00452843715885329\\
479	0.00450455081417934\\
480	0.00447981684303206\\
481	0.00445419803128292\\
482	0.00442765630872568\\
483	0.00440015300515289\\
484	0.00437164919286085\\
485	0.00434210613156405\\
486	0.00431148583748201\\
487	0.00427975180317468\\
488	0.00424686989968002\\
489	0.00421280949829499\\
490	0.00417754485596785\\
491	0.00414105681581911\\
492	0.00410333488287071\\
493	0.00406437974484788\\
494	0.00402420631952435\\
495	0.00398284742522583\\
496	0.00394035819421055\\
497	0.00389682138917866\\
498	0.00385235385275624\\
499	0.00380711438400887\\
500	0.00376131301138053\\
501	0.00371521852825999\\
502	0.00366915875495774\\
503	0.00362356040215765\\
504	0.00357912772682816\\
505	0.0035387837955858\\
506	0.00350347152895402\\
507	0.00347337562112852\\
508	0.0034477007156403\\
509	0.00342321915374529\\
510	0.0033995558606626\\
511	0.00337624383061699\\
512	0.00335267912856479\\
513	0.00332876510735656\\
514	0.00330442430717686\\
515	0.00327960483711814\\
516	0.00325427310653981\\
517	0.00322840785924007\\
518	0.00320198625777614\\
519	0.00317498308822855\\
520	0.00314738455198443\\
521	0.00311917684391942\\
522	0.00309034579133459\\
523	0.00306087673592382\\
524	0.00303075439888956\\
525	0.00299996271838372\\
526	0.00296848470340479\\
527	0.00293630226281886\\
528	0.00290339600729777\\
529	0.00286974502351109\\
530	0.00283532660238758\\
531	0.00280011576890602\\
532	0.00276406939748543\\
533	0.00272712121191253\\
534	0.0026911965166374\\
535	0.00265655106163712\\
536	0.00262214033891497\\
537	0.00258722058161971\\
538	0.00255170106648381\\
539	0.00251557462242589\\
540	0.00247883838157633\\
541	0.00244149232061441\\
542	0.00240353787403877\\
543	0.00236497792929109\\
544	0.00232581699903547\\
545	0.00228606137899959\\
546	0.00224571921278943\\
547	0.0022048002200509\\
548	0.00216331439627576\\
549	0.00212127465109859\\
550	0.0020787029667516\\
551	0.00203562649819421\\
552	0.00199207845654765\\
553	0.00194934483622062\\
554	0.0019076605974282\\
555	0.00186553102545023\\
556	0.00182295585535523\\
557	0.00177995162674704\\
558	0.00173653654153092\\
559	0.00169273037654859\\
560	0.00164855438448207\\
561	0.0016040311335613\\
562	0.00155918428001908\\
563	0.00151403828525429\\
564	0.0014686181370549\\
565	0.0014229492435685\\
566	0.00137720469839254\\
567	0.00133166599318628\\
568	0.0012858376589304\\
569	0.00123974504677254\\
570	0.00119341585869796\\
571	0.00114688025839211\\
572	0.00110017097212519\\
573	0.00105332337464662\\
574	0.00100637555373387\\
575	0.000959368345375567\\
576	0.00091234532951171\\
577	0.000865352773713771\\
578	0.000818439509057376\\
579	0.000771656718587056\\
580	0.000725057614040402\\
581	0.000678696970709246\\
582	0.000632630483283924\\
583	0.000586913897120322\\
584	0.000541601859672152\\
585	0.000496746426604043\\
586	0.000452395149029594\\
587	0.000408588670365815\\
588	0.000365357795666475\\
589	0.00032272012337526\\
590	0.000280676711614416\\
591	0.000239312817487405\\
592	0.000198788644961841\\
593	0.000159293651685586\\
594	0.000121079888727804\\
595	8.45520570083908e-05\\
596	5.05092148680373e-05\\
597	2.07908715710836e-05\\
598	0\\
599	0\\
600	0\\
};
\addplot [color=mycolor1,solid,forget plot]
  table[row sep=crcr]{%
1	0.00548095199615839\\
2	0.00548094992070687\\
3	0.00548094780715951\\
4	0.00548094565481181\\
5	0.00548094346294612\\
6	0.00548094123083137\\
7	0.00548093895772277\\
8	0.00548093664286176\\
9	0.00548093428547568\\
10	0.00548093188477729\\
11	0.00548092943996483\\
12	0.00548092695022149\\
13	0.00548092441471523\\
14	0.00548092183259856\\
15	0.00548091920300802\\
16	0.00548091652506419\\
17	0.00548091379787118\\
18	0.00548091102051634\\
19	0.00548090819207006\\
20	0.00548090531158525\\
21	0.00548090237809721\\
22	0.00548089939062314\\
23	0.005480896348162\\
24	0.00548089324969399\\
25	0.00548089009418024\\
26	0.00548088688056246\\
27	0.00548088360776255\\
28	0.00548088027468232\\
29	0.00548087688020299\\
30	0.00548087342318489\\
31	0.0054808699024671\\
32	0.00548086631686685\\
33	0.00548086266517931\\
34	0.00548085894617706\\
35	0.00548085515860977\\
36	0.0054808513012037\\
37	0.0054808473726612\\
38	0.00548084337166045\\
39	0.00548083929685487\\
40	0.00548083514687246\\
41	0.00548083092031579\\
42	0.00548082661576104\\
43	0.00548082223175787\\
44	0.00548081776682869\\
45	0.00548081321946827\\
46	0.00548080858814314\\
47	0.00548080387129106\\
48	0.00548079906732057\\
49	0.00548079417461039\\
50	0.00548078919150867\\
51	0.00548078411633274\\
52	0.00548077894736843\\
53	0.00548077368286925\\
54	0.00548076832105606\\
55	0.0054807628601163\\
56	0.00548075729820337\\
57	0.00548075163343604\\
58	0.00548074586389776\\
59	0.00548073998763612\\
60	0.00548073400266186\\
61	0.00548072790694861\\
62	0.00548072169843166\\
63	0.00548071537500777\\
64	0.00548070893453403\\
65	0.00548070237482737\\
66	0.00548069569366378\\
67	0.00548068888877726\\
68	0.00548068195785944\\
69	0.00548067489855835\\
70	0.00548066770847799\\
71	0.00548066038517733\\
72	0.00548065292616929\\
73	0.00548064532892022\\
74	0.00548063759084875\\
75	0.00548062970932491\\
76	0.00548062168166931\\
77	0.00548061350515221\\
78	0.00548060517699245\\
79	0.00548059669435663\\
80	0.00548058805435795\\
81	0.00548057925405533\\
82	0.00548057029045237\\
83	0.00548056116049621\\
84	0.00548055186107659\\
85	0.00548054238902466\\
86	0.00548053274111202\\
87	0.00548052291404934\\
88	0.0054805129044855\\
89	0.00548050270900618\\
90	0.00548049232413272\\
91	0.00548048174632115\\
92	0.00548047097196047\\
93	0.00548045999737188\\
94	0.00548044881880714\\
95	0.00548043743244753\\
96	0.00548042583440221\\
97	0.00548041402070723\\
98	0.00548040198732382\\
99	0.00548038973013709\\
100	0.0054803772449546\\
101	0.00548036452750497\\
102	0.00548035157343624\\
103	0.00548033837831449\\
104	0.00548032493762212\\
105	0.00548031124675644\\
106	0.00548029730102794\\
107	0.00548028309565878\\
108	0.00548026862578087\\
109	0.00548025388643441\\
110	0.00548023887256618\\
111	0.00548022357902757\\
112	0.0054802080005728\\
113	0.00548019213185738\\
114	0.00548017596743578\\
115	0.00548015950175985\\
116	0.00548014272917685\\
117	0.00548012564392743\\
118	0.00548010824014352\\
119	0.00548009051184649\\
120	0.00548007245294482\\
121	0.00548005405723204\\
122	0.00548003531838467\\
123	0.00548001622996009\\
124	0.00547999678539385\\
125	0.00547997697799801\\
126	0.00547995680095824\\
127	0.00547993624733152\\
128	0.00547991531004419\\
129	0.00547989398188898\\
130	0.00547987225552268\\
131	0.00547985012346352\\
132	0.00547982757808871\\
133	0.00547980461163155\\
134	0.00547978121617879\\
135	0.00547975738366809\\
136	0.00547973310588491\\
137	0.00547970837445984\\
138	0.00547968318086549\\
139	0.00547965751641371\\
140	0.00547963137225254\\
141	0.00547960473936312\\
142	0.00547957760855628\\
143	0.0054795499704698\\
144	0.00547952181556483\\
145	0.00547949313412261\\
146	0.00547946391624125\\
147	0.00547943415183217\\
148	0.00547940383061655\\
149	0.00547937294212185\\
150	0.00547934147567809\\
151	0.00547930942041437\\
152	0.00547927676525471\\
153	0.00547924349891452\\
154	0.00547920960989667\\
155	0.00547917508648734\\
156	0.00547913991675218\\
157	0.00547910408853202\\
158	0.00547906758943872\\
159	0.00547903040685087\\
160	0.0054789925279096\\
161	0.0054789539395139\\
162	0.00547891462831629\\
163	0.00547887458071825\\
164	0.00547883378286538\\
165	0.00547879222064283\\
166	0.0054787498796703\\
167	0.00547870674529745\\
168	0.00547866280259838\\
169	0.00547861803636722\\
170	0.00547857243111221\\
171	0.00547852597105114\\
172	0.0054784786401056\\
173	0.00547843042189563\\
174	0.00547838129973434\\
175	0.0054783312566222\\
176	0.00547828027524145\\
177	0.00547822833795021\\
178	0.00547817542677671\\
179	0.00547812152341338\\
180	0.00547806660921088\\
181	0.0054780106651719\\
182	0.00547795367194498\\
183	0.00547789560981841\\
184	0.00547783645871397\\
185	0.00547777619817999\\
186	0.00547771480738572\\
187	0.00547765226511421\\
188	0.00547758854975598\\
189	0.00547752363930238\\
190	0.00547745751133886\\
191	0.00547739014303824\\
192	0.00547732151115371\\
193	0.00547725159201235\\
194	0.00547718036150766\\
195	0.00547710779509384\\
196	0.00547703386778093\\
197	0.00547695855413523\\
198	0.00547688182828359\\
199	0.00547680366388242\\
200	0.00547672403410897\\
201	0.00547664291165308\\
202	0.00547656026870942\\
203	0.00547647607696921\\
204	0.00547639030761201\\
205	0.00547630293129755\\
206	0.00547621391815725\\
207	0.00547612323778595\\
208	0.00547603085923324\\
209	0.00547593675099499\\
210	0.00547584088100468\\
211	0.00547574321662483\\
212	0.00547564372463806\\
213	0.0054755423712384\\
214	0.00547543912202255\\
215	0.00547533394198096\\
216	0.00547522679548875\\
217	0.00547511764629692\\
218	0.00547500645752335\\
219	0.00547489319164394\\
220	0.00547477781048339\\
221	0.00547466027520632\\
222	0.00547454054630832\\
223	0.00547441858360674\\
224	0.00547429434623182\\
225	0.00547416779261774\\
226	0.00547403888049354\\
227	0.00547390756687405\\
228	0.00547377380805125\\
229	0.00547363755958502\\
230	0.00547349877629423\\
231	0.00547335741224781\\
232	0.005473213420756\\
233	0.00547306675436082\\
234	0.00547291736482763\\
235	0.0054727652031355\\
236	0.00547261021946829\\
237	0.00547245236320512\\
238	0.0054722915829106\\
239	0.00547212782632512\\
240	0.00547196104035452\\
241	0.00547179117105935\\
242	0.00547161816364347\\
243	0.00547144196244214\\
244	0.00547126251090899\\
245	0.00547107975160189\\
246	0.00547089362616763\\
247	0.00547070407532488\\
248	0.00547051103884505\\
249	0.00547031445553121\\
250	0.00547011426319354\\
251	0.00546991039862183\\
252	0.00546970279755407\\
253	0.00546949139464002\\
254	0.00546927612339942\\
255	0.00546905691617357\\
256	0.0054688337040694\\
257	0.00546860641689464\\
258	0.00546837498308231\\
259	0.00546813932960382\\
260	0.00546789938186764\\
261	0.00546765506360236\\
262	0.00546740629672147\\
263	0.00546715300116732\\
264	0.00546689509473118\\
265	0.00546663249284685\\
266	0.00546636510835429\\
267	0.00546609285122961\\
268	0.00546581562827825\\
269	0.00546553334278746\\
270	0.00546524589413507\\
271	0.00546495317735124\\
272	0.00546465508262966\\
273	0.0054643514947835\\
274	0.00546404229263197\\
275	0.00546372734826881\\
276	0.00546340652603965\\
277	0.0054630796806094\\
278	0.00546274665192147\\
279	0.00546240724915325\\
280	0.00546206128609614\\
281	0.00546170863564718\\
282	0.00546134916823953\\
283	0.00546098275179508\\
284	0.00546060925167596\\
285	0.00546022853063473\\
286	0.00545984044876465\\
287	0.00545944486344781\\
288	0.00545904162930304\\
289	0.00545863059813262\\
290	0.00545821161886802\\
291	0.00545778453751446\\
292	0.00545734919709457\\
293	0.00545690543759074\\
294	0.00545645309588682\\
295	0.00545599200570826\\
296	0.00545552199756122\\
297	0.00545504289867078\\
298	0.0054545545329177\\
299	0.0054540567207738\\
300	0.00545354927923665\\
301	0.00545303202176261\\
302	0.00545250475819857\\
303	0.00545196729471259\\
304	0.00545141943372308\\
305	0.00545086097382662\\
306	0.00545029170972414\\
307	0.00544971143214647\\
308	0.00544911992777732\\
309	0.00544851697917555\\
310	0.00544790236469576\\
311	0.00544727585840716\\
312	0.0054466372300107\\
313	0.00544598624475502\\
314	0.00544532266335037\\
315	0.00544464624188075\\
316	0.00544395673171442\\
317	0.00544325387941254\\
318	0.00544253742663585\\
319	0.00544180711004963\\
320	0.00544106266122603\\
321	0.00544030380654516\\
322	0.00543953026709337\\
323	0.00543874175855965\\
324	0.00543793799112967\\
325	0.00543711866937707\\
326	0.0054362834921529\\
327	0.00543543215247208\\
328	0.005434564337397\\
329	0.00543367972791881\\
330	0.00543277799883512\\
331	0.00543185881862527\\
332	0.0054309218493221\\
333	0.00542996674638014\\
334	0.00542899315854046\\
335	0.00542800072769201\\
336	0.00542698908872839\\
337	0.00542595786940121\\
338	0.0054249066901685\\
339	0.00542383516403881\\
340	0.00542274289641018\\
341	0.0054216294849042\\
342	0.00542049451919431\\
343	0.00541933758082834\\
344	0.00541815824304497\\
345	0.00541695607058338\\
346	0.00541573061948593\\
347	0.00541448143689398\\
348	0.00541320806083489\\
349	0.00541191002000126\\
350	0.00541058683352134\\
351	0.00540923801071954\\
352	0.00540786305086729\\
353	0.00540646144292367\\
354	0.00540503266526402\\
355	0.00540357618539771\\
356	0.00540209145967306\\
357	0.00540057793296957\\
358	0.0053990350383768\\
359	0.00539746219685888\\
360	0.00539585881690556\\
361	0.00539422429416723\\
362	0.00539255801107627\\
363	0.00539085933645289\\
364	0.00538912762509577\\
365	0.0053873622173592\\
366	0.00538556243871585\\
367	0.00538372759930749\\
368	0.00538185699348492\\
369	0.0053799498993387\\
370	0.00537800557822563\\
371	0.00537602327429234\\
372	0.0053740022140025\\
373	0.00537194160567334\\
374	0.00536984063902765\\
375	0.00536769848477218\\
376	0.00536551429421052\\
377	0.00536328719890381\\
378	0.00536101631039253\\
379	0.00535870071999319\\
380	0.00535633949868242\\
381	0.00535393169707442\\
382	0.00535147634547554\\
383	0.00534897245395576\\
384	0.00534641901230657\\
385	0.00534381498980157\\
386	0.00534115933575377\\
387	0.00533845098860062\\
388	0.00533568888069135\\
389	0.00533287192107836\\
390	0.00532999899488533\\
391	0.00532706896265217\\
392	0.00532408065965528\\
393	0.00532103289520269\\
394	0.00531792445190184\\
395	0.00531475408490083\\
396	0.00531152052110063\\
397	0.00530822245833811\\
398	0.00530485856453716\\
399	0.00530142747682666\\
400	0.00529792780062444\\
401	0.00529435810868445\\
402	0.00529071694010634\\
403	0.00528700279930482\\
404	0.00528321415493732\\
405	0.00527934943878715\\
406	0.00527540704460026\\
407	0.00527138532687276\\
408	0.00526728259958638\\
409	0.00526309713488923\\
410	0.00525882716171782\\
411	0.00525447086435603\\
412	0.0052500263809267\\
413	0.00524549180180995\\
414	0.00524086516798206\\
415	0.00523614446926672\\
416	0.00523132764248807\\
417	0.00522641256951514\\
418	0.00522139707518676\\
419	0.00521627892509709\\
420	0.00521105582321913\\
421	0.00520572540934116\\
422	0.00520028525629376\\
423	0.00519473286698471\\
424	0.0051890656713553\\
425	0.00518328102341414\\
426	0.00517737619710569\\
427	0.00517134838138553\\
428	0.0051651946761483\\
429	0.00515891208789345\\
430	0.00515249752510843\\
431	0.00514594779334966\\
432	0.00513925959000025\\
433	0.00513242949867982\\
434	0.00512545398328218\\
435	0.00511832938161421\\
436	0.00511105189860727\\
437	0.00510361759907029\\
438	0.00509602239995316\\
439	0.00508826206208469\\
440	0.00508033218134922\\
441	0.00507222817926275\\
442	0.00506394529290762\\
443	0.00505547856418199\\
444	0.005046822828319\\
445	0.00503797270162797\\
446	0.00502892256840949\\
447	0.00501966656699718\\
448	0.00501019857488201\\
449	0.00500051219286966\\
450	0.00499060072818567\\
451	0.0049804571764038\\
452	0.00497007420233899\\
453	0.00495944411973041\\
454	0.00494855886965451\\
455	0.00493740999761189\\
456	0.00492598862923287\\
457	0.00491428544455216\\
458	0.00490229065080857\\
459	0.00488999395373417\\
460	0.00487738452730974\\
461	0.00486445098197741\\
462	0.00485118133132173\\
463	0.00483756295725451\\
464	0.00482358257377012\\
465	0.00480922618937484\\
466	0.00479447906833696\\
467	0.00477932569094731\\
468	0.00476374971300784\\
469	0.00474773392474334\\
470	0.00473126020916734\\
471	0.00471430949936953\\
472	0.00469686173269478\\
473	0.00467889579597215\\
474	0.00466038944590183\\
475	0.0046413191586662\\
476	0.00462165983497246\\
477	0.0046013846366147\\
478	0.00458046289322046\\
479	0.00455885105141006\\
480	0.00453650300946854\\
481	0.00451338434121667\\
482	0.00448945977358611\\
483	0.00446469277837288\\
484	0.0044390455330161\\
485	0.00441247901676345\\
486	0.00438495317490672\\
487	0.00435642714567625\\
488	0.00432685956560797\\
489	0.00429620897197185\\
490	0.00426443432577805\\
491	0.00423149568450701\\
492	0.00419735506082228\\
493	0.00416197751252801\\
494	0.00412533252053638\\
495	0.00408739572641924\\
496	0.00404815112021496\\
497	0.00400759379329441\\
498	0.00396573339945054\\
499	0.00392259849595016\\
500	0.00387824197248343\\
501	0.00383274799840133\\
502	0.00378624154179356\\
503	0.00373890021273193\\
504	0.00369096392134772\\
505	0.00364273454391488\\
506	0.00359457071864176\\
507	0.00354702042024983\\
508	0.00350131809110213\\
509	0.00346061049916525\\
510	0.00342502183026241\\
511	0.00339439861452269\\
512	0.00336708322036617\\
513	0.00334086691234794\\
514	0.00331534730435857\\
515	0.00329008181911496\\
516	0.00326469884107946\\
517	0.00323892834194342\\
518	0.00321269198886212\\
519	0.00318593354574616\\
520	0.00315860607848959\\
521	0.00313068753007561\\
522	0.00310216113399003\\
523	0.00307301064759331\\
524	0.00304322031745423\\
525	0.00301277483353352\\
526	0.00298165833580482\\
527	0.00294985424466898\\
528	0.00291734506982389\\
529	0.00288411216856215\\
530	0.00285013547640853\\
531	0.00281539321418112\\
532	0.00277986195570541\\
533	0.00274351740193557\\
534	0.00270633319570252\\
535	0.00266822518666434\\
536	0.00263037485873978\\
537	0.0025936689644631\\
538	0.00255802540455848\\
539	0.00252190230147721\\
540	0.00248520925445124\\
541	0.00244790403567086\\
542	0.00240998217519735\\
543	0.00237144483485123\\
544	0.00233229481254299\\
545	0.00229253661599072\\
546	0.0022521764177109\\
547	0.00221122188822311\\
548	0.00216968132057519\\
549	0.00212756246687727\\
550	0.00208488317727384\\
551	0.00204166550220163\\
552	0.00199793669477371\\
553	0.0019537309880139\\
554	0.00190965922034648\\
555	0.00186720834870374\\
556	0.00182456308654889\\
557	0.00178148227067175\\
558	0.0017379831919425\\
559	0.0016940856717289\\
560	0.00164981119638615\\
561	0.00160518279977741\\
562	0.00156022488653992\\
563	0.00151496298230589\\
564	0.00146942342078429\\
565	0.00142363302680362\\
566	0.00137761899412085\\
567	0.00133166599321896\\
568	0.00128583765893305\\
569	0.0012397450467737\\
570	0.00119341585869848\\
571	0.00114688025839233\\
572	0.00110017097212528\\
573	0.00105332337464666\\
574	0.00100637555373387\\
575	0.000959368345375562\\
576	0.000912345329511703\\
577	0.000865352773713766\\
578	0.000818439509057374\\
579	0.00077165671858706\\
580	0.00072505761404041\\
581	0.000678696970709248\\
582	0.000632630483283932\\
583	0.000586913897120327\\
584	0.000541601859672158\\
585	0.00049674642660404\\
586	0.000452395149029591\\
587	0.000408588670365814\\
588	0.000365357795666474\\
589	0.000322720123375262\\
590	0.000280676711614417\\
591	0.000239312817487403\\
592	0.000198788644961841\\
593	0.000159293651685586\\
594	0.000121079888727805\\
595	8.45520570083909e-05\\
596	5.05092148680371e-05\\
597	2.07908715710836e-05\\
598	0\\
599	0\\
600	0\\
};
\addplot [color=mycolor2,solid,forget plot]
  table[row sep=crcr]{%
1	0.0054891968694182\\
2	0.00548919435537706\\
3	0.00548919179592685\\
4	0.00548918919024169\\
5	0.00548918653748061\\
6	0.00548918383678725\\
7	0.00548918108728961\\
8	0.00548917828809958\\
9	0.00548917543831285\\
10	0.00548917253700857\\
11	0.00548916958324894\\
12	0.00548916657607902\\
13	0.00548916351452631\\
14	0.00548916039760048\\
15	0.00548915722429305\\
16	0.00548915399357712\\
17	0.00548915070440682\\
18	0.00548914735571718\\
19	0.00548914394642359\\
20	0.0054891404754216\\
21	0.00548913694158649\\
22	0.00548913334377289\\
23	0.00548912968081434\\
24	0.00548912595152302\\
25	0.00548912215468925\\
26	0.00548911828908118\\
27	0.00548911435344429\\
28	0.00548911034650102\\
29	0.0054891062669503\\
30	0.00548910211346729\\
31	0.00548909788470254\\
32	0.005489093579282\\
33	0.00548908919580631\\
34	0.00548908473285036\\
35	0.0054890801889628\\
36	0.00548907556266564\\
37	0.00548907085245371\\
38	0.00548906605679403\\
39	0.00548906117412548\\
40	0.0054890562028583\\
41	0.00548905114137334\\
42	0.00548904598802173\\
43	0.00548904074112418\\
44	0.00548903539897048\\
45	0.00548902995981902\\
46	0.00548902442189594\\
47	0.00548901878339485\\
48	0.00548901304247607\\
49	0.00548900719726596\\
50	0.00548900124585643\\
51	0.00548899518630413\\
52	0.00548898901662991\\
53	0.0054889827348182\\
54	0.00548897633881622\\
55	0.00548896982653331\\
56	0.00548896319584039\\
57	0.00548895644456892\\
58	0.00548894957051057\\
59	0.005488942571416\\
60	0.00548893544499462\\
61	0.00548892818891339\\
62	0.00548892080079643\\
63	0.00548891327822374\\
64	0.00548890561873086\\
65	0.00548889781980776\\
66	0.00548888987889793\\
67	0.00548888179339792\\
68	0.00548887356065595\\
69	0.00548886517797149\\
70	0.00548885664259395\\
71	0.00548884795172195\\
72	0.0054888391025024\\
73	0.00548883009202948\\
74	0.00548882091734358\\
75	0.00548881157543053\\
76	0.0054888020632204\\
77	0.00548879237758641\\
78	0.00548878251534416\\
79	0.00548877247325024\\
80	0.00548876224800123\\
81	0.00548875183623264\\
82	0.00548874123451786\\
83	0.0054887304393668\\
84	0.00548871944722482\\
85	0.00548870825447156\\
86	0.00548869685741956\\
87	0.00548868525231328\\
88	0.00548867343532759\\
89	0.00548866140256658\\
90	0.00548864915006227\\
91	0.00548863667377318\\
92	0.00548862396958306\\
93	0.00548861103329949\\
94	0.00548859786065239\\
95	0.00548858444729253\\
96	0.00548857078879037\\
97	0.00548855688063409\\
98	0.00548854271822849\\
99	0.00548852829689328\\
100	0.0054885136118614\\
101	0.00548849865827747\\
102	0.00548848343119637\\
103	0.00548846792558115\\
104	0.00548845213630171\\
105	0.00548843605813288\\
106	0.0054884196857527\\
107	0.00548840301374054\\
108	0.00548838603657553\\
109	0.0054883687486344\\
110	0.00548835114418964\\
111	0.00548833321740775\\
112	0.00548831496234715\\
113	0.00548829637295601\\
114	0.00548827744307067\\
115	0.00548825816641306\\
116	0.00548823853658886\\
117	0.00548821854708523\\
118	0.00548819819126881\\
119	0.00548817746238323\\
120	0.00548815635354697\\
121	0.00548813485775122\\
122	0.00548811296785711\\
123	0.0054880906765936\\
124	0.00548806797655505\\
125	0.00548804486019857\\
126	0.00548802131984158\\
127	0.00548799734765942\\
128	0.00548797293568226\\
129	0.00548794807579285\\
130	0.00548792275972361\\
131	0.00548789697905397\\
132	0.00548787072520739\\
133	0.00548784398944858\\
134	0.00548781676288066\\
135	0.00548778903644202\\
136	0.00548776080090342\\
137	0.00548773204686483\\
138	0.00548770276475235\\
139	0.00548767294481489\\
140	0.00548764257712121\\
141	0.00548761165155607\\
142	0.00548758015781746\\
143	0.00548754808541279\\
144	0.00548751542365553\\
145	0.00548748216166155\\
146	0.00548744828834564\\
147	0.00548741379241772\\
148	0.00548737866237903\\
149	0.00548734288651851\\
150	0.00548730645290866\\
151	0.00548726934940173\\
152	0.00548723156362574\\
153	0.00548719308298007\\
154	0.00548715389463163\\
155	0.00548711398551041\\
156	0.00548707334230508\\
157	0.00548703195145868\\
158	0.00548698979916422\\
159	0.0054869468713599\\
160	0.00548690315372438\\
161	0.00548685863167226\\
162	0.0054868132903491\\
163	0.00548676711462645\\
164	0.00548672008909692\\
165	0.00548667219806894\\
166	0.00548662342556174\\
167	0.00548657375529968\\
168	0.00548652317070723\\
169	0.00548647165490311\\
170	0.00548641919069484\\
171	0.00548636576057296\\
172	0.00548631134670515\\
173	0.00548625593093042\\
174	0.00548619949475278\\
175	0.00548614201933525\\
176	0.00548608348549331\\
177	0.00548602387368881\\
178	0.00548596316402304\\
179	0.00548590133623029\\
180	0.00548583836967079\\
181	0.00548577424332394\\
182	0.00548570893578119\\
183	0.00548564242523852\\
184	0.00548557468948928\\
185	0.00548550570591667\\
186	0.00548543545148583\\
187	0.00548536390273596\\
188	0.00548529103577244\\
189	0.00548521682625827\\
190	0.00548514124940548\\
191	0.0054850642799659\\
192	0.00548498589222079\\
193	0.00548490605996795\\
194	0.00548482475650521\\
195	0.00548474195460491\\
196	0.00548465762647844\\
197	0.0054845717437398\\
198	0.00548448427744685\\
199	0.00548439519858112\\
200	0.0054843044776251\\
201	0.00548421208453498\\
202	0.00548411798873232\\
203	0.00548402215909528\\
204	0.00548392456395014\\
205	0.00548382517106224\\
206	0.00548372394762747\\
207	0.00548362086026317\\
208	0.0054835158749993\\
209	0.00548340895726937\\
210	0.00548330007190167\\
211	0.00548318918310984\\
212	0.0054830762544842\\
213	0.00548296124898257\\
214	0.00548284412892102\\
215	0.00548272485596526\\
216	0.00548260339112148\\
217	0.00548247969472741\\
218	0.0054823537264438\\
219	0.00548222544524543\\
220	0.00548209480941283\\
221	0.00548196177652399\\
222	0.00548182630344604\\
223	0.00548168834632761\\
224	0.00548154786059131\\
225	0.00548140480092645\\
226	0.00548125912128235\\
227	0.00548111077486217\\
228	0.00548095971411708\\
229	0.00548080589074135\\
230	0.00548064925566799\\
231	0.00548048975906534\\
232	0.00548032735033443\\
233	0.00548016197810786\\
234	0.00547999359024942\\
235	0.00547982213385538\\
236	0.00547964755525738\\
237	0.00547946980002722\\
238	0.00547928881298338\\
239	0.00547910453820034\\
240	0.00547891691901986\\
241	0.00547872589806595\\
242	0.00547853141726252\\
243	0.00547833341785528\\
244	0.00547813184043755\\
245	0.00547792662498099\\
246	0.00547771771087146\\
247	0.00547750503695109\\
248	0.00547728854156713\\
249	0.00547706816262813\\
250	0.00547684383766866\\
251	0.00547661550392378\\
252	0.00547638309841347\\
253	0.00547614655803969\\
254	0.00547590581969587\\
255	0.00547566082039141\\
256	0.0054754114973923\\
257	0.0054751577883796\\
258	0.00547489963162761\\
259	0.00547463696620386\\
260	0.00547436973219279\\
261	0.00547409787094485\\
262	0.00547382132535371\\
263	0.00547354004016329\\
264	0.00547325396230583\\
265	0.00547296304127366\\
266	0.00547266722952463\\
267	0.00547236648292219\\
268	0.00547206076120906\\
269	0.00547175002851286\\
270	0.00547143425387946\\
271	0.00547111341182841\\
272	0.00547078748292199\\
273	0.00547045645433792\\
274	0.00547012032043892\\
275	0.00546977908334958\\
276	0.00546943275361923\\
277	0.00546908135127342\\
278	0.00546872490828844\\
279	0.00546836347592409\\
280	0.00546799639134473\\
281	0.00546762240864155\\
282	0.00546724139884568\\
283	0.00546685323060331\\
284	0.00546645777013235\\
285	0.00546605488117901\\
286	0.00546564442497286\\
287	0.00546522626018168\\
288	0.00546480024286546\\
289	0.00546436622642938\\
290	0.00546392406157631\\
291	0.00546347359625828\\
292	0.00546301467562746\\
293	0.00546254714198608\\
294	0.0054620708347355\\
295	0.00546158559032467\\
296	0.00546109124219773\\
297	0.00546058762074055\\
298	0.00546007455322658\\
299	0.00545955186376227\\
300	0.00545901937323075\\
301	0.00545847689923528\\
302	0.00545792425604187\\
303	0.00545736125452067\\
304	0.00545678770208683\\
305	0.00545620340264023\\
306	0.00545560815650464\\
307	0.00545500176036559\\
308	0.0054543840072078\\
309	0.00545375468625155\\
310	0.0054531135828881\\
311	0.00545246047861439\\
312	0.00545179515096701\\
313	0.00545111737345481\\
314	0.00545042691549136\\
315	0.00544972354232614\\
316	0.00544900701497484\\
317	0.0054482770901493\\
318	0.00544753352018626\\
319	0.00544677605297551\\
320	0.00544600443188722\\
321	0.00544521839569857\\
322	0.00544441767851983\\
323	0.00544360200971929\\
324	0.00544277111384802\\
325	0.00544192471056379\\
326	0.00544106251455431\\
327	0.00544018423545993\\
328	0.00543928957779588\\
329	0.00543837824087373\\
330	0.00543744991872266\\
331	0.00543650430001013\\
332	0.00543554106796177\\
333	0.0054345599002812\\
334	0.00543356046906949\\
335	0.00543254244074374\\
336	0.00543150547595554\\
337	0.00543044922950901\\
338	0.00542937335027815\\
339	0.00542827748112374\\
340	0.00542716125880973\\
341	0.00542602431391878\\
342	0.00542486627076723\\
343	0.00542368674731854\\
344	0.00542248535509619\\
345	0.00542126169909461\\
346	0.00542001537768867\\
347	0.00541874598254035\\
348	0.00541745309850346\\
349	0.00541613630352446\\
350	0.00541479516853939\\
351	0.00541342925736604\\
352	0.00541203812658974\\
353	0.00541062132544212\\
354	0.00540917839567121\\
355	0.00540770887140096\\
356	0.00540621227897795\\
357	0.00540468813680368\\
358	0.00540313595514848\\
359	0.00540155523594473\\
360	0.00539994547255471\\
361	0.00539830614950943\\
362	0.00539663674221262\\
363	0.00539493671660397\\
364	0.00539320552877565\\
365	0.00539144262453276\\
366	0.00538964743889032\\
367	0.00538781939549552\\
368	0.0053859579059644\\
369	0.00538406236912007\\
370	0.00538213217011732\\
371	0.00538016667943867\\
372	0.00537816525174382\\
373	0.00537612722455306\\
374	0.00537405191674505\\
375	0.00537193862684648\\
376	0.00536978663109177\\
377	0.00536759518123094\\
378	0.00536536350206276\\
379	0.00536309078867397\\
380	0.00536077620336287\\
381	0.00535841887221443\\
382	0.00535601788124313\\
383	0.00535357227181194\\
384	0.00535108103429341\\
385	0.0053485430962889\\
386	0.00534595729230035\\
387	0.00534332226783449\\
388	0.00534063667519442\\
389	0.00533789953008658\\
390	0.00533510982741454\\
391	0.0053322665406314\\
392	0.00532936862105515\\
393	0.00532641499714516\\
394	0.00532340457373629\\
395	0.00532033623122087\\
396	0.00531720882469156\\
397	0.00531402118304865\\
398	0.00531077210806497\\
399	0.00530746037340651\\
400	0.00530408472360693\\
401	0.00530064387299334\\
402	0.0052971365045622\\
403	0.00529356126880294\\
404	0.00528991678246438\\
405	0.00528620162727035\\
406	0.00528241434857942\\
407	0.00527855345397782\\
408	0.00527461741181547\\
409	0.0052706046496879\\
410	0.00526651355286785\\
411	0.00526234246269237\\
412	0.00525808967491305\\
413	0.00525375343801949\\
414	0.00524933195154946\\
415	0.00524482336440346\\
416	0.00524022577318039\\
417	0.00523553722053481\\
418	0.00523075569352813\\
419	0.00522587912211942\\
420	0.00522090537774521\\
421	0.00521583227193822\\
422	0.00521065755485279\\
423	0.00520537891343096\\
424	0.00519999396908422\\
425	0.00519450027717447\\
426	0.00518889534523272\\
427	0.00518317663173477\\
428	0.00517734151593173\\
429	0.00517138729411216\\
430	0.00516531117563831\\
431	0.00515911027874116\\
432	0.00515278162605946\\
433	0.00514632213990524\\
434	0.00513972863723734\\
435	0.00513299782432283\\
436	0.00512612629106462\\
437	0.0051191105049721\\
438	0.00511194680474882\\
439	0.00510463139346928\\
440	0.00509716033131396\\
441	0.0050895295278284\\
442	0.00508173473366834\\
443	0.00507377153178841\\
444	0.00506563532802464\\
445	0.00505732134101474\\
446	0.00504882459138982\\
447	0.00504013989016983\\
448	0.00503126182632091\\
449	0.00502218475354018\\
450	0.00501290277652464\\
451	0.00500340973646137\\
452	0.0049936991921593\\
453	0.00498376440288637\\
454	0.00497359831009461\\
455	0.00496319351796841\\
456	0.00495254227273068\\
457	0.00494163644064148\\
458	0.00493046748462266\\
459	0.0049190264394442\\
460	0.00490730388540931\\
461	0.00489528992047967\\
462	0.00488297413078798\\
463	0.00487034555949376\\
464	0.00485739267395021\\
465	0.004844103331166\\
466	0.00483046474156964\\
467	0.00481646343111338\\
468	0.00480208520178408\\
469	0.00478731509058795\\
470	0.00477213732701297\\
471	0.00475653528923159\\
472	0.00474049145901888\\
473	0.00472398737499751\\
474	0.00470700358311135\\
475	0.00468951958190851\\
476	0.00467151375681195\\
477	0.00465296326746879\\
478	0.00463384386379001\\
479	0.00461412987626572\\
480	0.00459379379728429\\
481	0.00457280301484697\\
482	0.00455110433656644\\
483	0.00452865953063785\\
484	0.00450543286302288\\
485	0.00448138762584243\\
486	0.0044564855239955\\
487	0.00443068671594453\\
488	0.00440394987873579\\
489	0.00437623233356901\\
490	0.00434749022961062\\
491	0.00431767880121355\\
492	0.00428675271652006\\
493	0.00425466653997079\\
494	0.00422137533634864\\
495	0.00418683545021244\\
496	0.00415100550209224\\
497	0.00411384765185985\\
498	0.00407532919063057\\
499	0.00403542453652763\\
500	0.00399411772952133\\
501	0.00395140554229041\\
502	0.00390730132316084\\
503	0.00386183971978261\\
504	0.00381508261869486\\
505	0.00376712721209702\\
506	0.00371811743437225\\
507	0.0036682546917953\\
508	0.00361780571383835\\
509	0.00356710002107475\\
510	0.00351655779440234\\
511	0.00346685990014391\\
512	0.00342018933600302\\
513	0.00337865550221436\\
514	0.00334233331140309\\
515	0.00331092069354877\\
516	0.00328172789297185\\
517	0.00325356226126486\\
518	0.00322599612173834\\
519	0.00319856266008969\\
520	0.00317113359201397\\
521	0.00314330028321649\\
522	0.00311494865759424\\
523	0.00308604010602848\\
524	0.00305653561648579\\
525	0.00302639862942524\\
526	0.00299561110095096\\
527	0.00296415546625934\\
528	0.00293201443441139\\
529	0.00289917111057011\\
530	0.00286560835076743\\
531	0.00283130815884521\\
532	0.00279625143981421\\
533	0.0027604176819128\\
534	0.00272378461722965\\
535	0.00268632921308865\\
536	0.00264802742428661\\
537	0.00260881862072326\\
538	0.00256896387323768\\
539	0.00253012203665061\\
540	0.00249255684260618\\
541	0.00245517426589838\\
542	0.00241727949581605\\
543	0.00237877084369842\\
544	0.00233964426046532\\
545	0.00229989957797116\\
546	0.00225954118691839\\
547	0.00221857506323328\\
548	0.00217700799179227\\
549	0.00213484522472573\\
550	0.00209210022016971\\
551	0.00204879104734812\\
552	0.00200493976528336\\
553	0.00196057319704059\\
554	0.00191572431648802\\
555	0.00187042054078887\\
556	0.00182648374393835\\
557	0.00178330361132209\\
558	0.00173971373924418\\
559	0.00169571747444497\\
560	0.00165133624243625\\
561	0.00160659337207\\
562	0.00156151382896341\\
563	0.00151612401849032\\
564	0.00147045151570613\\
565	0.00142452473202305\\
566	0.00137837259118908\\
567	0.00133202429968801\\
568	0.00128583765932925\\
569	0.00123974504679409\\
570	0.00119341585870728\\
571	0.00114688025839642\\
572	0.0011001709721271\\
573	0.00105332337464741\\
574	0.00100637555373416\\
575	0.000959368345375665\\
576	0.000912345329511744\\
577	0.000865352773713785\\
578	0.000818439509057384\\
579	0.00077165671858706\\
580	0.000725057614040401\\
581	0.000678696970709242\\
582	0.000632630483283921\\
583	0.000586913897120319\\
584	0.000541601859672149\\
585	0.00049674642660404\\
586	0.000452395149029591\\
587	0.000408588670365815\\
588	0.000365357795666476\\
589	0.00032272012337526\\
590	0.000280676711614416\\
591	0.000239312817487405\\
592	0.000198788644961842\\
593	0.000159293651685586\\
594	0.000121079888727804\\
595	8.45520570083913e-05\\
596	5.05092148680373e-05\\
597	2.07908715710836e-05\\
598	0\\
599	0\\
600	0\\
};
\addplot [color=mycolor3,solid,forget plot]
  table[row sep=crcr]{%
1	0.00550477682330948\\
2	0.00550477425642561\\
3	0.0055047716434672\\
4	0.00550476898360203\\
5	0.00550476627598275\\
6	0.00550476351974651\\
7	0.00550476071401469\\
8	0.00550475785789298\\
9	0.00550475495047054\\
10	0.00550475199082008\\
11	0.00550474897799734\\
12	0.00550474591104091\\
13	0.00550474278897191\\
14	0.00550473961079362\\
15	0.0055047363754912\\
16	0.00550473308203119\\
17	0.0055047297293615\\
18	0.00550472631641071\\
19	0.00550472284208796\\
20	0.00550471930528255\\
21	0.00550471570486342\\
22	0.00550471203967905\\
23	0.00550470830855674\\
24	0.00550470451030261\\
25	0.00550470064370086\\
26	0.0055046967075136\\
27	0.00550469270048034\\
28	0.00550468862131753\\
29	0.00550468446871832\\
30	0.00550468024135183\\
31	0.00550467593786316\\
32	0.00550467155687248\\
33	0.00550466709697481\\
34	0.00550466255673955\\
35	0.00550465793470994\\
36	0.00550465322940267\\
37	0.00550464843930729\\
38	0.00550464356288576\\
39	0.00550463859857201\\
40	0.00550463354477119\\
41	0.00550462839985948\\
42	0.00550462316218334\\
43	0.00550461783005889\\
44	0.00550461240177163\\
45	0.00550460687557555\\
46	0.00550460124969282\\
47	0.00550459552231303\\
48	0.00550458969159265\\
49	0.00550458375565442\\
50	0.0055045777125868\\
51	0.0055045715604432\\
52	0.00550456529724142\\
53	0.005504558920963\\
54	0.00550455242955239\\
55	0.00550454582091652\\
56	0.00550453909292396\\
57	0.00550453224340419\\
58	0.00550452527014696\\
59	0.0055045181709016\\
60	0.00550451094337607\\
61	0.00550450358523637\\
62	0.00550449609410563\\
63	0.00550448846756352\\
64	0.0055044807031453\\
65	0.00550447279834096\\
66	0.00550446475059457\\
67	0.00550445655730317\\
68	0.00550444821581615\\
69	0.00550443972343412\\
70	0.00550443107740831\\
71	0.00550442227493934\\
72	0.00550441331317651\\
73	0.00550440418921677\\
74	0.00550439490010382\\
75	0.0055043854428269\\
76	0.00550437581432005\\
77	0.00550436601146093\\
78	0.00550435603106987\\
79	0.00550434586990864\\
80	0.00550433552467965\\
81	0.00550432499202459\\
82	0.00550431426852338\\
83	0.00550430335069311\\
84	0.00550429223498672\\
85	0.00550428091779194\\
86	0.00550426939543011\\
87	0.00550425766415471\\
88	0.00550424572015037\\
89	0.00550423355953152\\
90	0.00550422117834099\\
91	0.0055042085725488\\
92	0.00550419573805069\\
93	0.00550418267066679\\
94	0.00550416936614034\\
95	0.00550415582013608\\
96	0.0055041420282389\\
97	0.00550412798595239\\
98	0.00550411368869705\\
99	0.00550409913180917\\
100	0.00550408431053899\\
101	0.00550406922004924\\
102	0.0055040538554133\\
103	0.00550403821161376\\
104	0.00550402228354073\\
105	0.00550400606598996\\
106	0.00550398955366117\\
107	0.00550397274115646\\
108	0.00550395562297808\\
109	0.00550393819352694\\
110	0.00550392044710062\\
111	0.00550390237789137\\
112	0.0055038839799842\\
113	0.00550386524735506\\
114	0.00550384617386849\\
115	0.00550382675327583\\
116	0.00550380697921296\\
117	0.00550378684519831\\
118	0.0055037663446305\\
119	0.00550374547078629\\
120	0.00550372421681824\\
121	0.00550370257575223\\
122	0.00550368054048561\\
123	0.00550365810378426\\
124	0.00550363525828055\\
125	0.00550361199647067\\
126	0.00550358831071218\\
127	0.0055035641932214\\
128	0.00550353963607094\\
129	0.00550351463118682\\
130	0.005503489170346\\
131	0.00550346324517338\\
132	0.00550343684713932\\
133	0.00550340996755649\\
134	0.00550338259757716\\
135	0.00550335472819009\\
136	0.00550332635021771\\
137	0.00550329745431279\\
138	0.0055032680309557\\
139	0.00550323807045099\\
140	0.00550320756292402\\
141	0.00550317649831816\\
142	0.005503144866391\\
143	0.00550311265671116\\
144	0.00550307985865477\\
145	0.00550304646140206\\
146	0.00550301245393358\\
147	0.00550297782502669\\
148	0.00550294256325193\\
149	0.00550290665696907\\
150	0.00550287009432344\\
151	0.00550283286324184\\
152	0.0055027949514287\\
153	0.00550275634636214\\
154	0.00550271703528952\\
155	0.00550267700522361\\
156	0.00550263624293808\\
157	0.00550259473496338\\
158	0.00550255246758201\\
159	0.00550250942682436\\
160	0.00550246559846392\\
161	0.00550242096801271\\
162	0.00550237552071655\\
163	0.00550232924155014\\
164	0.00550228211521213\\
165	0.00550223412612033\\
166	0.00550218525840629\\
167	0.00550213549591032\\
168	0.00550208482217605\\
169	0.00550203322044512\\
170	0.00550198067365169\\
171	0.00550192716441669\\
172	0.00550187267504231\\
173	0.0055018171875059\\
174	0.00550176068345432\\
175	0.00550170314419761\\
176	0.00550164455070308\\
177	0.00550158488358875\\
178	0.00550152412311714\\
179	0.00550146224918863\\
180	0.00550139924133485\\
181	0.00550133507871183\\
182	0.00550126974009321\\
183	0.00550120320386315\\
184	0.00550113544800915\\
185	0.00550106645011499\\
186	0.00550099618735303\\
187	0.00550092463647722\\
188	0.00550085177381503\\
189	0.00550077757526\\
190	0.00550070201626403\\
191	0.00550062507182938\\
192	0.00550054671650088\\
193	0.00550046692435791\\
194	0.00550038566900639\\
195	0.00550030292357135\\
196	0.00550021866068941\\
197	0.00550013285250246\\
198	0.00550004547065237\\
199	0.00549995648627275\\
200	0.00549986586996504\\
201	0.00549977359178749\\
202	0.00549967962124463\\
203	0.00549958392727645\\
204	0.00549948647824725\\
205	0.00549938724193426\\
206	0.005499286185516\\
207	0.00549918327556026\\
208	0.00549907847801172\\
209	0.00549897175817939\\
210	0.00549886308072349\\
211	0.0054987524096421\\
212	0.0054986397082573\\
213	0.00549852493920084\\
214	0.0054984080643997\\
215	0.00549828904506062\\
216	0.00549816784165441\\
217	0.00549804441390017\\
218	0.00549791872074772\\
219	0.00549779072036074\\
220	0.00549766037009842\\
221	0.00549752762649653\\
222	0.00549739244524797\\
223	0.00549725478118225\\
224	0.00549711458824402\\
225	0.00549697181947082\\
226	0.00549682642696964\\
227	0.00549667836189246\\
228	0.00549652757441025\\
229	0.00549637401368581\\
230	0.00549621762784532\\
231	0.00549605836394773\\
232	0.00549589616795276\\
233	0.0054957309846868\\
234	0.00549556275780674\\
235	0.00549539142976125\\
236	0.00549521694174974\\
237	0.00549503923367794\\
238	0.00549485824411096\\
239	0.00549467391022222\\
240	0.00549448616773887\\
241	0.00549429495088275\\
242	0.00549410019230668\\
243	0.0054939018230254\\
244	0.00549369977234063\\
245	0.00549349396775976\\
246	0.00549328433490728\\
247	0.00549307079742838\\
248	0.00549285327688355\\
249	0.00549263169263335\\
250	0.00549240596171277\\
251	0.00549217599869314\\
252	0.00549194171553131\\
253	0.00549170302140385\\
254	0.00549145982252539\\
255	0.00549121202194953\\
256	0.0054909595193502\\
257	0.0054907022107821\\
258	0.00549043998841833\\
259	0.00549017274026306\\
260	0.00548990034983755\\
261	0.00548962269583773\\
262	0.00548933965176146\\
263	0.00548905108550404\\
264	0.00548875685892081\\
265	0.00548845682735624\\
266	0.0054881508391397\\
267	0.00548783873504886\\
268	0.00548752034774364\\
269	0.00548719550117455\\
270	0.00548686400997268\\
271	0.00548652567883073\\
272	0.0054861803018893\\
273	0.00548582766214855\\
274	0.00548546753093547\\
275	0.00548509966748068\\
276	0.00548472381871943\\
277	0.00548433971962728\\
278	0.00548394709507049\\
279	0.00548354566654718\\
280	0.00548313585038502\\
281	0.00548271859907244\\
282	0.00548229377981785\\
283	0.0054818612575201\\
284	0.00548142089472977\\
285	0.00548097255161\\
286	0.00548051608589664\\
287	0.00548005135285767\\
288	0.0054795782052522\\
289	0.00547909649328861\\
290	0.00547860606458215\\
291	0.00547810676411174\\
292	0.00547759843417632\\
293	0.00547708091435022\\
294	0.005476554041438\\
295	0.00547601764942866\\
296	0.00547547156944895\\
297	0.00547491562971612\\
298	0.00547434965548969\\
299	0.00547377346902287\\
300	0.00547318688951282\\
301	0.00547258973305047\\
302	0.00547198181256938\\
303	0.00547136293779389\\
304	0.00547073291518661\\
305	0.005470091547895\\
306	0.00546943863569711\\
307	0.00546877397494682\\
308	0.00546809735851799\\
309	0.00546740857574802\\
310	0.00546670741238051\\
311	0.00546599365050735\\
312	0.00546526706850985\\
313	0.00546452744099923\\
314	0.00546377453875629\\
315	0.00546300812867045\\
316	0.00546222797367823\\
317	0.00546143383270067\\
318	0.00546062546058052\\
319	0.00545980260801873\\
320	0.00545896502151018\\
321	0.00545811244327932\\
322	0.00545724461121499\\
323	0.00545636125880529\\
324	0.00545546211507171\\
325	0.00545454690450366\\
326	0.00545361534699242\\
327	0.0054526671577656\\
328	0.00545170204732146\\
329	0.00545071972136402\\
330	0.00544971988073853\\
331	0.00544870222136763\\
332	0.00544766643418897\\
333	0.00544661220509387\\
334	0.00544553921486785\\
335	0.00544444713913311\\
336	0.00544333564829348\\
337	0.00544220440748248\\
338	0.0054410530765146\\
339	0.00543988130984097\\
340	0.0054386887565096\\
341	0.00543747506013127\\
342	0.0054362398588518\\
343	0.00543498278533194\\
344	0.00543370346673545\\
345	0.00543240152472745\\
346	0.00543107657548345\\
347	0.00542972822971189\\
348	0.00542835609269072\\
349	0.00542695976432092\\
350	0.00542553883919896\\
351	0.00542409290671077\\
352	0.00542262155115011\\
353	0.00542112435186449\\
354	0.00541960088343242\\
355	0.00541805071587583\\
356	0.00541647341491245\\
357	0.00541486854225276\\
358	0.00541323565594737\\
359	0.00541157431079114\\
360	0.00540988405879056\\
361	0.0054081644497021\\
362	0.00540641503164977\\
363	0.00540463535183105\\
364	0.0054028249573207\\
365	0.00540098339598347\\
366	0.00539911021750652\\
367	0.00539720497456394\\
368	0.00539526722412573\\
369	0.00539329652892395\\
370	0.00539129245908901\\
371	0.00538925459396853\\
372	0.00538718252414036\\
373	0.00538507585362938\\
374	0.00538293420233504\\
375	0.00538075720867236\\
376	0.00537854453242384\\
377	0.00537629585779\\
378	0.00537401089661826\\
379	0.00537168939177402\\
380	0.00536933112060602\\
381	0.00536693589844766\\
382	0.00536450358211477\\
383	0.00536203407346954\\
384	0.00535952732353251\\
385	0.00535698333899241\\
386	0.00535440219739541\\
387	0.00535178409178646\\
388	0.00534912514988319\\
389	0.00534641761903743\\
390	0.00534366061596057\\
391	0.00534085324265133\\
392	0.00533799458610111\\
393	0.00533508371795533\\
394	0.00533211969415246\\
395	0.00532910155454309\\
396	0.00532602832211091\\
397	0.00532289900203223\\
398	0.00531971258080888\\
399	0.00531646802530294\\
400	0.00531316428168546\\
401	0.00530980027430732\\
402	0.00530637490449251\\
403	0.00530288704928074\\
404	0.00529933556009359\\
405	0.00529571926117679\\
406	0.00529203694800453\\
407	0.00528828738570286\\
408	0.00528446930703295\\
409	0.00528058141012162\\
410	0.00527662235590618\\
411	0.00527259076525703\\
412	0.00526848521573851\\
413	0.00526430423796721\\
414	0.00526004631153165\\
415	0.00525570986046398\\
416	0.00525129324831951\\
417	0.00524679477302037\\
418	0.00524221266141137\\
419	0.00523754506071366\\
420	0.00523279003052249\\
421	0.00522794553410434\\
422	0.0052230094281619\\
423	0.00521797944891619\\
424	0.00521285318686776\\
425	0.00520762802306626\\
426	0.00520230092936529\\
427	0.00519686899937804\\
428	0.00519132985949917\\
429	0.00518568105926974\\
430	0.00517992006754721\\
431	0.00517404426844573\\
432	0.00516805095702335\\
433	0.00516193733472785\\
434	0.00515570050460202\\
435	0.00514933746622776\\
436	0.00514284511040285\\
437	0.00513622021354724\\
438	0.00512945943183804\\
439	0.0051225592950757\\
440	0.00511551620027982\\
441	0.00510832640502191\\
442	0.00510098602050871\\
443	0.00509349100442846\\
444	0.00508583715357047\\
445	0.00507802009621341\\
446	0.00507003528423538\\
447	0.00506187798479317\\
448	0.00505354327122278\\
449	0.00504502601268046\\
450	0.00503632086333021\\
451	0.00502742226201914\\
452	0.00501832447056875\\
453	0.00500902149371753\\
454	0.00499950706396044\\
455	0.00498977462543773\\
456	0.00497981731679798\\
457	0.00496962795297015\\
458	0.00495919900577868\\
459	0.00494852258333314\\
460	0.00493759040812209\\
461	0.00492639379373232\\
462	0.00491492362011725\\
463	0.00490317030733283\\
464	0.00489112378765591\\
465	0.00487877347599587\\
466	0.00486610823851244\\
467	0.00485311635938098\\
468	0.00483978550576249\\
469	0.0048261026913033\\
470	0.00481205423849246\\
471	0.00479762573586072\\
472	0.00478280199444234\\
473	0.00476756700393987\\
474	0.00475190388728252\\
475	0.00473579485348131\\
476	0.00471922114823902\\
477	0.0047021630016523\\
478	0.00468459957115042\\
479	0.00466650885784214\\
480	0.0046478675515963\\
481	0.0046286508102488\\
482	0.00460883243742465\\
483	0.00458838400496198\\
484	0.00456726999448731\\
485	0.00454543469310754\\
486	0.00452284311214705\\
487	0.00449945871076504\\
488	0.00447524373166766\\
489	0.00445015868630056\\
490	0.00442416236753183\\
491	0.00439721188603949\\
492	0.00436926275841055\\
493	0.00434026904545963\\
494	0.00431018355398794\\
495	0.00427895811764693\\
496	0.00424654397658071\\
497	0.00421289228003259\\
498	0.00417795474171066\\
499	0.00414168448465006\\
500	0.0041040371207246\\
501	0.0040649721200045\\
502	0.00402445453844905\\
503	0.00398245719006864\\
504	0.00393896337156799\\
505	0.00389397025230831\\
506	0.00384749302944713\\
507	0.00379957011758446\\
508	0.00375026981833159\\
509	0.00369969955863891\\
510	0.0036480175997573\\
511	0.00359544297495342\\
512	0.0035422627542797\\
513	0.00348883084154451\\
514	0.0034356354815106\\
515	0.00338346436248934\\
516	0.00333529941397275\\
517	0.00329239584335947\\
518	0.00325479594604639\\
519	0.00322212145566726\\
520	0.00319081861398368\\
521	0.0031604552962268\\
522	0.00313060409131133\\
523	0.0031007791385642\\
524	0.00307090240615044\\
525	0.00304081945040074\\
526	0.00301018748031608\\
527	0.00297896731824318\\
528	0.0029471211890936\\
529	0.00291460839813878\\
530	0.00288140050348125\\
531	0.00284747823876823\\
532	0.00281282247445835\\
533	0.00277741396691225\\
534	0.00274123362774897\\
535	0.0027042612251743\\
536	0.00266647498145895\\
537	0.00262785203732332\\
538	0.00258836871834692\\
539	0.00254799039987889\\
540	0.00250663194236034\\
541	0.00246552657985651\\
542	0.00242551775773468\\
543	0.00238679174078739\\
544	0.00234758968952664\\
545	0.00230783756843862\\
546	0.00226746688232517\\
547	0.00222647633461372\\
548	0.00218486861377273\\
549	0.00214264741712259\\
550	0.00209982152218929\\
551	0.00205640582806659\\
552	0.00201241871448934\\
553	0.00196788261038474\\
554	0.001922824759329\\
555	0.00187727861552058\\
556	0.00183128530984319\\
557	0.00178571858457715\\
558	0.00174175370183844\\
559	0.00169765189645918\\
560	0.00165315928882859\\
561	0.00160829620918484\\
562	0.0015630878421284\\
563	0.00151756120705137\\
564	0.00147174485938914\\
565	0.00142566860063263\\
566	0.00137936312148276\\
567	0.00133285966646703\\
568	0.00128618971348929\\
569	0.00123974505182\\
570	0.00119341585886474\\
571	0.00114688025846023\\
572	0.00110017097215749\\
573	0.00105332337466137\\
574	0.00100637555374019\\
575	0.000959368345378089\\
576	0.00091234532951263\\
577	0.00086535277371407\\
578	0.00081843950905746\\
579	0.000771656718587084\\
580	0.000725057614040413\\
581	0.000678696970709247\\
582	0.000632630483283924\\
583	0.00058691389712032\\
584	0.000541601859672149\\
585	0.000496746426604037\\
586	0.000452395149029591\\
587	0.000408588670365813\\
588	0.000365357795666473\\
589	0.000322720123375262\\
590	0.000280676711614414\\
591	0.000239312817487402\\
592	0.000198788644961839\\
593	0.000159293651685585\\
594	0.000121079888727804\\
595	8.45520570083908e-05\\
596	5.05092148680372e-05\\
597	2.07908715710836e-05\\
598	0\\
599	0\\
600	0\\
};
\addplot [color=mycolor4,solid,forget plot]
  table[row sep=crcr]{%
1	0.00552236449378588\\
2	0.00552236161741465\\
3	0.00552235869004543\\
4	0.00552235571076934\\
5	0.00552235267866136\\
6	0.00552234959277986\\
7	0.00552234645216632\\
8	0.00552234325584501\\
9	0.0055223400028227\\
10	0.00552233669208844\\
11	0.0055223333226131\\
12	0.00552232989334907\\
13	0.00552232640323002\\
14	0.00552232285117041\\
15	0.00552231923606535\\
16	0.00552231555678995\\
17	0.00552231181219927\\
18	0.0055223080011278\\
19	0.0055223041223891\\
20	0.00552230017477544\\
21	0.0055222961570574\\
22	0.00552229206798352\\
23	0.00552228790627984\\
24	0.00552228367064954\\
25	0.00552227935977247\\
26	0.00552227497230478\\
27	0.00552227050687855\\
28	0.00552226596210119\\
29	0.00552226133655511\\
30	0.00552225662879733\\
31	0.00552225183735878\\
32	0.0055222469607441\\
33	0.00552224199743105\\
34	0.00552223694586996\\
35	0.00552223180448328\\
36	0.00552222657166512\\
37	0.00552222124578073\\
38	0.00552221582516592\\
39	0.00552221030812658\\
40	0.00552220469293806\\
41	0.00552219897784472\\
42	0.0055221931610593\\
43	0.00552218724076237\\
44	0.00552218121510173\\
45	0.00552217508219179\\
46	0.00552216884011311\\
47	0.00552216248691159\\
48	0.00552215602059788\\
49	0.00552214943914689\\
50	0.00552214274049698\\
51	0.00552213592254938\\
52	0.00552212898316735\\
53	0.00552212192017577\\
54	0.00552211473136027\\
55	0.00552210741446656\\
56	0.00552209996719957\\
57	0.00552209238722291\\
58	0.00552208467215801\\
59	0.00552207681958337\\
60	0.00552206882703378\\
61	0.0055220606919995\\
62	0.00552205241192548\\
63	0.00552204398421052\\
64	0.00552203540620642\\
65	0.00552202667521712\\
66	0.00552201778849787\\
67	0.00552200874325425\\
68	0.00552199953664139\\
69	0.00552199016576291\\
70	0.00552198062767006\\
71	0.00552197091936075\\
72	0.00552196103777858\\
73	0.00552195097981178\\
74	0.00552194074229234\\
75	0.00552193032199479\\
76	0.00552191971563531\\
77	0.00552190891987057\\
78	0.00552189793129664\\
79	0.00552188674644797\\
80	0.00552187536179611\\
81	0.00552186377374865\\
82	0.00552185197864809\\
83	0.00552183997277049\\
84	0.00552182775232437\\
85	0.00552181531344947\\
86	0.00552180265221537\\
87	0.00552178976462037\\
88	0.00552177664659001\\
89	0.00552176329397577\\
90	0.00552174970255381\\
91	0.00552173586802346\\
92	0.00552172178600583\\
93	0.00552170745204245\\
94	0.00552169286159359\\
95	0.00552167801003702\\
96	0.00552166289266629\\
97	0.0055216475046892\\
98	0.00552163184122631\\
99	0.00552161589730922\\
100	0.00552159966787898\\
101	0.00552158314778434\\
102	0.00552156633178012\\
103	0.0055215492145254\\
104	0.00552153179058173\\
105	0.00552151405441141\\
106	0.00552149600037559\\
107	0.0055214776227323\\
108	0.00552145891563467\\
109	0.00552143987312886\\
110	0.00552142048915216\\
111	0.00552140075753084\\
112	0.00552138067197822\\
113	0.00552136022609242\\
114	0.00552133941335427\\
115	0.00552131822712517\\
116	0.00552129666064476\\
117	0.00552127470702868\\
118	0.00552125235926623\\
119	0.00552122961021801\\
120	0.00552120645261356\\
121	0.00552118287904884\\
122	0.0055211588819837\\
123	0.00552113445373936\\
124	0.00552110958649586\\
125	0.00552108427228929\\
126	0.00552105850300921\\
127	0.00552103227039581\\
128	0.00552100556603709\\
129	0.00552097838136615\\
130	0.00552095070765808\\
131	0.00552092253602712\\
132	0.00552089385742359\\
133	0.0055208646626308\\
134	0.00552083494226198\\
135	0.00552080468675704\\
136	0.00552077388637924\\
137	0.00552074253121211\\
138	0.00552071061115585\\
139	0.00552067811592396\\
140	0.00552064503503989\\
141	0.00552061135783322\\
142	0.00552057707343636\\
143	0.00552054217078058\\
144	0.0055205066385925\\
145	0.00552047046539015\\
146	0.00552043363947916\\
147	0.00552039614894883\\
148	0.00552035798166811\\
149	0.00552031912528151\\
150	0.00552027956720514\\
151	0.00552023929462236\\
152	0.00552019829447949\\
153	0.00552015655348165\\
154	0.00552011405808831\\
155	0.00552007079450878\\
156	0.00552002674869782\\
157	0.00551998190635094\\
158	0.00551993625289978\\
159	0.00551988977350748\\
160	0.00551984245306379\\
161	0.00551979427618033\\
162	0.00551974522718557\\
163	0.00551969529011997\\
164	0.0055196444487308\\
165	0.00551959268646715\\
166	0.00551953998647461\\
167	0.00551948633159015\\
168	0.0055194317043366\\
169	0.0055193760869174\\
170	0.00551931946121092\\
171	0.00551926180876504\\
172	0.00551920311079126\\
173	0.00551914334815907\\
174	0.00551908250138998\\
175	0.00551902055065147\\
176	0.00551895747575083\\
177	0.00551889325612899\\
178	0.00551882787085396\\
179	0.0055187612986142\\
180	0.00551869351771199\\
181	0.00551862450605631\\
182	0.00551855424115581\\
183	0.00551848270011119\\
184	0.00551840985960772\\
185	0.00551833569590721\\
186	0.00551826018483992\\
187	0.00551818330179588\\
188	0.00551810502171628\\
189	0.00551802531908424\\
190	0.00551794416791541\\
191	0.00551786154174813\\
192	0.00551777741363352\\
193	0.00551769175612491\\
194	0.00551760454126753\\
195	0.00551751574058748\\
196	0.00551742532508085\\
197	0.00551733326520237\\
198	0.00551723953085419\\
199	0.00551714409137402\\
200	0.00551704691552371\\
201	0.00551694797147738\\
202	0.00551684722680929\\
203	0.00551674464848142\\
204	0.00551664020283073\\
205	0.00551653385555592\\
206	0.00551642557170416\\
207	0.0055163153156571\\
208	0.00551620305111673\\
209	0.00551608874109072\\
210	0.00551597234787748\\
211	0.00551585383305065\\
212	0.00551573315744316\\
213	0.00551561028113103\\
214	0.00551548516341644\\
215	0.00551535776281035\\
216	0.00551522803701485\\
217	0.00551509594290453\\
218	0.00551496143650786\\
219	0.00551482447298749\\
220	0.0055146850066202\\
221	0.00551454299077638\\
222	0.00551439837789852\\
223	0.00551425111947928\\
224	0.00551410116603904\\
225	0.00551394846710242\\
226	0.00551379297117448\\
227	0.00551363462571579\\
228	0.00551347337711732\\
229	0.00551330917067408\\
230	0.00551314195055836\\
231	0.00551297165979225\\
232	0.0055127982402193\\
233	0.00551262163247557\\
234	0.00551244177596003\\
235	0.00551225860880428\\
236	0.00551207206784177\\
237	0.00551188208857648\\
238	0.00551168860515112\\
239	0.00551149155031508\\
240	0.00551129085539215\\
241	0.00551108645024829\\
242	0.00551087826325943\\
243	0.00551066622127962\\
244	0.00551045024961022\\
245	0.00551023027196952\\
246	0.00551000621046425\\
247	0.0055097779855625\\
248	0.00550954551606927\\
249	0.00550930871910496\\
250	0.00550906751008725\\
251	0.00550882180271794\\
252	0.0055085715089747\\
253	0.00550831653910962\\
254	0.00550805680165543\\
255	0.00550779220344061\\
256	0.00550752264961524\\
257	0.00550724804368904\\
258	0.00550696828758372\\
259	0.00550668328170141\\
260	0.00550639292501186\\
261	0.00550609711516044\\
262	0.00550579574859972\\
263	0.00550548872074722\\
264	0.0055051759261724\\
265	0.00550485725881497\\
266	0.00550453261223783\\
267	0.00550420187991627\\
268	0.00550386495556571\\
269	0.00550352173350871\\
270	0.00550317210908153\\
271	0.00550281597907872\\
272	0.00550245324223288\\
273	0.00550208379972421\\
274	0.00550170755571362\\
275	0.00550132441789157\\
276	0.00550093429803821\\
277	0.00550053711259858\\
278	0.00550013278327558\\
279	0.0054997212375169\\
280	0.00549930240779239\\
281	0.00549887620663984\\
282	0.00549844250829231\\
283	0.00549800118495234\\
284	0.00549755210676195\\
285	0.00549709514177206\\
286	0.00549663015591132\\
287	0.00549615701295485\\
288	0.00549567557449205\\
289	0.00549518569989425\\
290	0.00549468724628167\\
291	0.00549418006848997\\
292	0.00549366401903615\\
293	0.00549313894808387\\
294	0.00549260470340838\\
295	0.00549206113036053\\
296	0.00549150807183033\\
297	0.00549094536820982\\
298	0.00549037285735526\\
299	0.00548979037454844\\
300	0.00548919775245742\\
301	0.00548859482109638\\
302	0.00548798140778461\\
303	0.00548735733710453\\
304	0.00548672243085905\\
305	0.00548607650802752\\
306	0.00548541938472111\\
307	0.00548475087413662\\
308	0.00548407078650947\\
309	0.00548337892906525\\
310	0.00548267510597001\\
311	0.00548195911827908\\
312	0.0054812307638845\\
313	0.0054804898374607\\
314	0.00547973613040869\\
315	0.00547896943079811\\
316	0.00547818952330758\\
317	0.0054773961891628\\
318	0.00547658920607254\\
319	0.00547576834816195\\
320	0.00547493338590362\\
321	0.00547408408604557\\
322	0.00547322021153632\\
323	0.00547234152144663\\
324	0.00547144777088796\\
325	0.00547053871092677\\
326	0.00546961408849504\\
327	0.00546867364629624\\
328	0.00546771712270641\\
329	0.00546674425167016\\
330	0.00546575476259082\\
331	0.00546474838021479\\
332	0.0054637248245088\\
333	0.0054626838105302\\
334	0.00546162504828915\\
335	0.00546054824260222\\
336	0.00545945309293683\\
337	0.00545833929324494\\
338	0.00545720653178605\\
339	0.00545605449093745\\
340	0.00545488284699143\\
341	0.00545369126993761\\
342	0.00545247942322917\\
343	0.00545124696353159\\
344	0.00544999354045185\\
345	0.00544871879624645\\
346	0.00544742236550625\\
347	0.00544610387481529\\
348	0.00544476294238176\\
349	0.00544339917763805\\
350	0.00544201218080623\\
351	0.0054406015424265\\
352	0.00543916684284406\\
353	0.00543770765165072\\
354	0.00543622352707626\\
355	0.00543471401532517\\
356	0.0054331786498525\\
357	0.00543161695057351\\
358	0.00543002842300023\\
359	0.00542841255729794\\
360	0.0054267688272538\\
361	0.00542509668914959\\
362	0.00542339558052954\\
363	0.00542166491885402\\
364	0.00541990410002906\\
365	0.00541811249680166\\
366	0.00541628945701016\\
367	0.00541443430167937\\
368	0.00541254632294991\\
369	0.00541062478183212\\
370	0.00540866890577628\\
371	0.00540667788605188\\
372	0.00540465087493226\\
373	0.00540258698268411\\
374	0.00540048527436669\\
375	0.00539834476645274\\
376	0.00539616442329166\\
377	0.00539394315344812\\
378	0.00539167980596377\\
379	0.00538937316660993\\
380	0.00538702195422511\\
381	0.00538462481726917\\
382	0.00538218033078822\\
383	0.00537968699412042\\
384	0.00537714323003581\\
385	0.00537454738714601\\
386	0.00537189775132708\\
387	0.00536919258590505\\
388	0.00536643411745236\\
389	0.00536362814878825\\
390	0.00536077380558616\\
391	0.00535787020184325\\
392	0.00535491644144934\\
393	0.00535191161956328\\
394	0.00534885482403642\\
395	0.0053457451376249\\
396	0.00534258164487129\\
397	0.00533936343356682\\
398	0.00533608959274483\\
399	0.00533275921349833\\
400	0.00532937138914247\\
401	0.00532592521468404\\
402	0.00532241978561147\\
403	0.00531885419614399\\
404	0.0053152275385854\\
405	0.0053115389049386\\
406	0.00530778738568527\\
407	0.00530397206919219\\
408	0.00530009204781425\\
409	0.00529614642026316\\
410	0.00529213429423897\\
411	0.00528805478932205\\
412	0.00528390704009951\\
413	0.00527969019945088\\
414	0.00527540344183438\\
415	0.00527104596630378\\
416	0.00526661699899993\\
417	0.00526211579574787\\
418	0.00525754165017734\\
419	0.00525289393257562\\
420	0.00524817207014708\\
421	0.00524337554790461\\
422	0.00523850391366673\\
423	0.00523355678417246\\
424	0.0052285338562323\\
425	0.00522343493609123\\
426	0.00521826003033973\\
427	0.00521299981013176\\
428	0.00520764154984799\\
429	0.00520218316507152\\
430	0.00519662250553933\\
431	0.00519095735143663\\
432	0.00518518540943985\\
433	0.00517930430786009\\
434	0.00517331159122543\\
435	0.00516720471464439\\
436	0.00516098103757801\\
437	0.00515463781694451\\
438	0.00514817219947595\\
439	0.0051415812132719\\
440	0.00513486175855821\\
441	0.00512801059742921\\
442	0.0051210243424309\\
443	0.00511389944394695\\
444	0.00510663217628548\\
445	0.00509921862234511\\
446	0.00509165465665163\\
447	0.00508393592620462\\
448	0.00507605782727588\\
449	0.00506801547165309\\
450	0.0050598036192859\\
451	0.00505141649510841\\
452	0.0050428475396761\\
453	0.00503409144145339\\
454	0.0050251426509999\\
455	0.00501599536655582\\
456	0.00500664351914084\\
457	0.00499708075683417\\
458	0.00498730042815378\\
459	0.00497729556449069\\
460	0.0049670588615872\\
461	0.00495658266011656\\
462	0.00494585892524127\\
463	0.0049348792251699\\
464	0.00492363470868343\\
465	0.00491211608156425\\
466	0.00490031358175842\\
467	0.00488821695288661\\
468	0.00487581541544252\\
469	0.00486309763559506\\
470	0.00485005169845369\\
471	0.0048366651328826\\
472	0.00482292483719772\\
473	0.00480881700809003\\
474	0.00479432709932754\\
475	0.00477943977858097\\
476	0.0047641388823978\\
477	0.00474840736927814\\
478	0.00473222727046199\\
479	0.00471557963788382\\
480	0.00469844448912965\\
481	0.00468080074667346\\
482	0.0046626261375146\\
483	0.00464389700735168\\
484	0.00462458811794882\\
485	0.0046046727571455\\
486	0.00458412118468098\\
487	0.00456289381669408\\
488	0.00454093863433154\\
489	0.00451821998828621\\
490	0.00449470064864536\\
491	0.00447034196370148\\
492	0.00444510340289893\\
493	0.00441894254081404\\
494	0.00439181506085209\\
495	0.00436367480329994\\
496	0.0043344738555302\\
497	0.00430416269614973\\
498	0.00427269040674317\\
499	0.00424000496854713\\
500	0.00420605366530414\\
501	0.00417078361847794\\
502	0.00413414248746009\\
503	0.00409607937443961\\
504	0.00405654598266416\\
505	0.00401549808831925\\
506	0.00397289740224313\\
507	0.00392871391491754\\
508	0.00388292883819276\\
509	0.00383553825499621\\
510	0.00378655763043521\\
511	0.00373602750432496\\
512	0.00368402082627437\\
513	0.00363065307189473\\
514	0.00357609408201811\\
515	0.00352058662871403\\
516	0.00346443837690163\\
517	0.00340801376574661\\
518	0.00335185318789953\\
519	0.00329683024981812\\
520	0.00324653985142069\\
521	0.00320161003127924\\
522	0.00316207202960114\\
523	0.00312749854380489\\
524	0.00309395292311824\\
525	0.00306108207870875\\
526	0.00302867931543196\\
527	0.00299627525953052\\
528	0.00296378678854382\\
529	0.00293115317034936\\
530	0.00289809271373717\\
531	0.00286441613276463\\
532	0.00283008516840576\\
533	0.00279506205124351\\
534	0.00275930312882306\\
535	0.00272278664427699\\
536	0.00268549207316304\\
537	0.00264739866875622\\
538	0.00260848525570127\\
539	0.00256873024649069\\
540	0.00252811165960377\\
541	0.00248660681497579\\
542	0.00244415586141974\\
543	0.00240076947053404\\
544	0.00235831976722801\\
545	0.00231705559205897\\
546	0.00227650183143643\\
547	0.00223546034490059\\
548	0.00219384736307803\\
549	0.00215161604661321\\
550	0.00210876510575541\\
551	0.00206530472400923\\
552	0.00202124965549764\\
553	0.00197661822767365\\
554	0.00193143270313793\\
555	0.00188571981291925\\
556	0.00183951158664442\\
557	0.00179284685067574\\
558	0.00174575367523509\\
559	0.00169999208744112\\
560	0.00165532117678849\\
561	0.00161033080979638\\
562	0.00156498746152975\\
563	0.00151931648608894\\
564	0.00147334703390769\\
565	0.00142710998207863\\
566	0.00138063755587401\\
567	0.00133396294340406\\
568	0.00128711993625959\\
569	0.00124014252333356\\
570	0.00119341592573713\\
571	0.00114688025971738\\
572	0.00110017097260895\\
573	0.0010533233748797\\
574	0.00100637555384331\\
575	0.000959368345424142\\
576	0.000912345329531775\\
577	0.000865352773721365\\
578	0.000818439509059961\\
579	0.000771656718587823\\
580	0.000725057614040593\\
581	0.000678696970709286\\
582	0.000632630483283935\\
583	0.000586913897120328\\
584	0.000541601859672155\\
585	0.000496746426604042\\
586	0.000452395149029594\\
587	0.000408588670365815\\
588	0.000365357795666473\\
589	0.000322720123375259\\
590	0.000280676711614414\\
591	0.000239312817487403\\
592	0.000198788644961842\\
593	0.000159293651685587\\
594	0.000121079888727805\\
595	8.45520570083912e-05\\
596	5.05092148680371e-05\\
597	2.07908715710836e-05\\
598	0\\
599	0\\
600	0\\
};
\addplot [color=mycolor5,solid,forget plot]
  table[row sep=crcr]{%
1	0.0055565662029982\\
2	0.0055565625234433\\
3	0.00555655877973134\\
4	0.00555655497074048\\
5	0.00555655109532921\\
6	0.00555654715233594\\
7	0.00555654314057877\\
8	0.00555653905885504\\
9	0.00555653490594107\\
10	0.0055565306805916\\
11	0.00555652638153959\\
12	0.00555652200749573\\
13	0.00555651755714814\\
14	0.00555651302916188\\
15	0.00555650842217858\\
16	0.0055565037348161\\
17	0.00555649896566798\\
18	0.00555649411330313\\
19	0.00555648917626525\\
20	0.00555648415307261\\
21	0.00555647904221737\\
22	0.00555647384216524\\
23	0.00555646855135501\\
24	0.00555646316819805\\
25	0.00555645769107788\\
26	0.00555645211834957\\
27	0.00555644644833935\\
28	0.00555644067934405\\
29	0.0055564348096306\\
30	0.00555642883743542\\
31	0.00555642276096407\\
32	0.00555641657839049\\
33	0.00555641028785659\\
34	0.00555640388747161\\
35	0.00555639737531164\\
36	0.00555639074941891\\
37	0.00555638400780123\\
38	0.00555637714843142\\
39	0.00555637016924665\\
40	0.00555636306814791\\
41	0.00555635584299926\\
42	0.00555634849162714\\
43	0.00555634101181982\\
44	0.00555633340132675\\
45	0.00555632565785775\\
46	0.00555631777908234\\
47	0.0055563097626291\\
48	0.00555630160608489\\
49	0.00555629330699411\\
50	0.00555628486285798\\
51	0.00555627627113374\\
52	0.00555626752923394\\
53	0.00555625863452551\\
54	0.00555624958432911\\
55	0.00555624037591822\\
56	0.00555623100651832\\
57	0.00555622147330605\\
58	0.00555621177340834\\
59	0.00555620190390153\\
60	0.00555619186181047\\
61	0.00555618164410758\\
62	0.005556171247712\\
63	0.00555616066948858\\
64	0.0055561499062469\\
65	0.00555613895474034\\
66	0.00555612781166499\\
67	0.00555611647365882\\
68	0.00555610493730038\\
69	0.00555609319910802\\
70	0.00555608125553862\\
71	0.00555606910298657\\
72	0.00555605673778266\\
73	0.00555604415619292\\
74	0.00555603135441743\\
75	0.00555601832858933\\
76	0.00555600507477335\\
77	0.0055559915889648\\
78	0.00555597786708824\\
79	0.00555596390499621\\
80	0.00555594969846801\\
81	0.00555593524320828\\
82	0.00555592053484576\\
83	0.00555590556893193\\
84	0.00555589034093952\\
85	0.00555587484626118\\
86	0.00555585908020805\\
87	0.00555584303800822\\
88	0.00555582671480538\\
89	0.00555581010565713\\
90	0.00555579320553354\\
91	0.00555577600931549\\
92	0.00555575851179316\\
93	0.00555574070766432\\
94	0.00555572259153268\\
95	0.00555570415790615\\
96	0.00555568540119518\\
97	0.00555566631571092\\
98	0.00555564689566349\\
99	0.00555562713516005\\
100	0.00555560702820297\\
101	0.00555558656868798\\
102	0.00555556575040211\\
103	0.0055555445670218\\
104	0.00555552301211084\\
105	0.00555550107911823\\
106	0.00555547876137623\\
107	0.00555545605209811\\
108	0.00555543294437596\\
109	0.00555540943117851\\
110	0.00555538550534878\\
111	0.00555536115960185\\
112	0.00555533638652235\\
113	0.00555531117856221\\
114	0.0055552855280381\\
115	0.00555525942712889\\
116	0.00555523286787309\\
117	0.00555520584216623\\
118	0.00555517834175824\\
119	0.00555515035825068\\
120	0.00555512188309391\\
121	0.00555509290758422\\
122	0.00555506342286103\\
123	0.00555503341990391\\
124	0.00555500288952936\\
125	0.0055549718223879\\
126	0.00555494020896089\\
127	0.00555490803955718\\
128	0.00555487530430996\\
129	0.00555484199317324\\
130	0.0055548080959185\\
131	0.00555477360213113\\
132	0.00555473850120685\\
133	0.005554702782348\\
134	0.00555466643455983\\
135	0.00555462944664663\\
136	0.00555459180720787\\
137	0.005554553504634\\
138	0.00555451452710263\\
139	0.00555447486257415\\
140	0.00555443449878744\\
141	0.00555439342325572\\
142	0.00555435162326185\\
143	0.00555430908585392\\
144	0.00555426579784046\\
145	0.00555422174578586\\
146	0.00555417691600541\\
147	0.00555413129456037\\
148	0.00555408486725289\\
149	0.00555403761962097\\
150	0.00555398953693303\\
151	0.0055539406041827\\
152	0.00555389080608334\\
153	0.00555384012706248\\
154	0.00555378855125606\\
155	0.00555373606250292\\
156	0.00555368264433869\\
157	0.0055536282799901\\
158	0.00555357295236878\\
159	0.00555351664406516\\
160	0.00555345933734259\\
161	0.00555340101413062\\
162	0.00555334165601911\\
163	0.00555328124425156\\
164	0.00555321975971893\\
165	0.00555315718295301\\
166	0.00555309349412002\\
167	0.00555302867301414\\
168	0.00555296269905097\\
169	0.00555289555126099\\
170	0.00555282720828328\\
171	0.00555275764835887\\
172	0.00555268684932458\\
173	0.00555261478860662\\
174	0.00555254144321438\\
175	0.00555246678973436\\
176	0.00555239080432414\\
177	0.00555231346270656\\
178	0.00555223474016394\\
179	0.00555215461153247\\
180	0.00555207305119675\\
181	0.00555199003308438\\
182	0.00555190553066068\\
183	0.00555181951692356\\
184	0.0055517319643983\\
185	0.00555164284513227\\
186	0.00555155213068955\\
187	0.00555145979214541\\
188	0.00555136580008011\\
189	0.00555127012457244\\
190	0.00555117273519229\\
191	0.00555107360099227\\
192	0.00555097269049799\\
193	0.00555086997169687\\
194	0.00555076541202491\\
195	0.00555065897835187\\
196	0.00555055063696405\\
197	0.00555044035354526\\
198	0.00555032809315718\\
199	0.00555021382022778\\
200	0.00555009749853958\\
201	0.00554997909121792\\
202	0.00554985856071892\\
203	0.00554973586881732\\
204	0.00554961097659401\\
205	0.00554948384442358\\
206	0.00554935443196139\\
207	0.00554922269813061\\
208	0.00554908860110916\\
209	0.00554895209831617\\
210	0.0055488131463986\\
211	0.00554867170121735\\
212	0.00554852771783352\\
213	0.00554838115049422\\
214	0.00554823195261837\\
215	0.0055480800767825\\
216	0.00554792547470603\\
217	0.00554776809723678\\
218	0.00554760789433616\\
219	0.00554744481506445\\
220	0.00554727880756594\\
221	0.00554710981905395\\
222	0.00554693779579595\\
223	0.0055467626830986\\
224	0.00554658442529286\\
225	0.00554640296571936\\
226	0.00554621824671349\\
227	0.00554603020959108\\
228	0.00554583879463392\\
229	0.00554564394107592\\
230	0.00554544558708915\\
231	0.0055452436697707\\
232	0.00554503812512965\\
233	0.00554482888807474\\
234	0.00554461589240254\\
235	0.00554439907078644\\
236	0.00554417835476615\\
237	0.00554395367473844\\
238	0.00554372495994837\\
239	0.00554349213848204\\
240	0.00554325513726026\\
241	0.0055430138820335\\
242	0.0055427682973783\\
243	0.00554251830669526\\
244	0.00554226383220837\\
245	0.00554200479496647\\
246	0.00554174111484607\\
247	0.0055414727105564\\
248	0.00554119949964632\\
249	0.00554092139851309\\
250	0.0055406383224134\\
251	0.00554035018547619\\
252	0.00554005690071743\\
253	0.00553975838005681\\
254	0.00553945453433581\\
255	0.00553914527333715\\
256	0.00553883050580524\\
257	0.00553851013946685\\
258	0.00553818408105177\\
259	0.00553785223631224\\
260	0.00553751451004054\\
261	0.00553717080608326\\
262	0.00553682102735139\\
263	0.005536465075824\\
264	0.00553610285254412\\
265	0.00553573425760475\\
266	0.00553535919012243\\
267	0.0055349775481963\\
268	0.00553458922884979\\
269	0.00553419412795238\\
270	0.00553379214011849\\
271	0.00553338315858182\\
272	0.00553296707504204\\
273	0.00553254377948372\\
274	0.00553211315996623\\
275	0.00553167510238606\\
276	0.00553122949021451\\
277	0.00553077620421378\\
278	0.00553031512213377\\
279	0.00552984611838764\\
280	0.0055293690637428\\
281	0.00552888382583964\\
282	0.00552839027033731\\
283	0.0055278882608906\\
284	0.0055273776591266\\
285	0.00552685832462144\\
286	0.00552633011487688\\
287	0.005525792885297\\
288	0.0055252464891646\\
289	0.00552469077761789\\
290	0.00552412559962685\\
291	0.00552355080196984\\
292	0.00552296622920996\\
293	0.00552237172367168\\
294	0.00552176712541721\\
295	0.00552115227222303\\
296	0.00552052699955636\\
297	0.00551989114055175\\
298	0.00551924452598748\\
299	0.00551858698426229\\
300	0.00551791834137189\\
301	0.00551723842088557\\
302	0.00551654704392292\\
303	0.0055158440291304\\
304	0.00551512919265814\\
305	0.00551440234813669\\
306	0.0055136633066536\\
307	0.00551291187673046\\
308	0.00551214786429946\\
309	0.00551137107268013\\
310	0.00551058130255629\\
311	0.0055097783519526\\
312	0.00550896201621118\\
313	0.00550813208796837\\
314	0.00550728835713096\\
315	0.00550643061085276\\
316	0.00550555863351065\\
317	0.00550467220668062\\
318	0.00550377110911339\\
319	0.00550285511670998\\
320	0.00550192400249677\\
321	0.00550097753660009\\
322	0.00550001548622049\\
323	0.00549903761560637\\
324	0.00549804368602693\\
325	0.00549703345574455\\
326	0.00549600667998613\\
327	0.00549496311091374\\
328	0.00549390249759393\\
329	0.00549282458596611\\
330	0.00549172911880951\\
331	0.0054906158357086\\
332	0.00548948447301702\\
333	0.00548833476381949\\
334	0.00548716643789195\\
335	0.00548597922165947\\
336	0.00548477283815144\\
337	0.00548354700695471\\
338	0.00548230144416329\\
339	0.00548103586232541\\
340	0.00547974997038695\\
341	0.0054784434736314\\
342	0.00547711607361598\\
343	0.00547576746810338\\
344	0.0054743973509896\\
345	0.00547300541222673\\
346	0.005471591337741\\
347	0.00547015480934577\\
348	0.00546869550464924\\
349	0.00546721309695638\\
350	0.00546570725516585\\
351	0.0054641776436606\\
352	0.00546262392219336\\
353	0.00546104574576619\\
354	0.00545944276450474\\
355	0.0054578146235275\\
356	0.00545616096281012\\
357	0.00545448141704604\\
358	0.00545277561550345\\
359	0.00545104318188011\\
360	0.00544928373415703\\
361	0.00544749688445256\\
362	0.00544568223887847\\
363	0.00544383939739993\\
364	0.00544196795370227\\
365	0.00544006749506626\\
366	0.00543813760225537\\
367	0.0054361778494178\\
368	0.00543418780400641\\
369	0.00543216702671998\\
370	0.00543011507146868\\
371	0.00542803148536672\\
372	0.00542591580875399\\
373	0.00542376757524806\\
374	0.00542158631182604\\
375	0.00541937153893374\\
376	0.00541712277061681\\
377	0.0054148395146646\\
378	0.00541252127275313\\
379	0.00541016754056764\\
380	0.00540777780787916\\
381	0.00540535155854355\\
382	0.00540288827038902\\
383	0.00540038741496609\\
384	0.00539784845716456\\
385	0.00539527085476927\\
386	0.00539265405805357\\
387	0.00538999750881576\\
388	0.00538730063262149\\
389	0.00538456269424022\\
390	0.00538178267450218\\
391	0.00537895948912136\\
392	0.00537609198820947\\
393	0.00537317895757003\\
394	0.00537021912229371\\
395	0.00536721115319308\\
396	0.00536415367653561\\
397	0.00536104528726815\\
398	0.00535788456594081\\
399	0.00535467009890918\\
400	0.00535140050032142\\
401	0.00534807443324338\\
402	0.00534469062506345\\
403	0.00534124786628992\\
404	0.00533774496506\\
405	0.00533418068466903\\
406	0.00533055374014991\\
407	0.00532686279465179\\
408	0.00532310645558164\\
409	0.00531928327028568\\
410	0.0053153917213833\\
411	0.0053114302217447\\
412	0.00530739710910915\\
413	0.00530329064035132\\
414	0.00529910898541757\\
415	0.00529485022097572\\
416	0.00529051232385152\\
417	0.00528609316436\\
418	0.00528159049964099\\
419	0.00527700196694627\\
420	0.00527232507615669\\
421	0.00526755720374208\\
422	0.00526269558811525\\
423	0.00525773732758168\\
424	0.0052526793845131\\
425	0.00524751860727432\\
426	0.00524225180987519\\
427	0.00523688478169057\\
428	0.00523142625771105\\
429	0.00522587457125705\\
430	0.00522022801988917\\
431	0.00521448486504337\\
432	0.00520864333315186\\
433	0.0052027016247223\\
434	0.00519665791541612\\
435	0.00519051035238131\\
436	0.00518425705670917\\
437	0.00517789612644616\\
438	0.005171425640055\\
439	0.00516484365994742\\
440	0.00515814823576862\\
441	0.00515133741063655\\
442	0.00514440922838081\\
443	0.00513736174019266\\
444	0.00513019301163666\\
445	0.00512290112979097\\
446	0.00511548421024411\\
447	0.00510794040379761\\
448	0.00510026790340343\\
449	0.00509246495436486\\
450	0.00508452987913126\\
451	0.00507646115502242\\
452	0.00506825492819732\\
453	0.00505988116691834\\
454	0.00505133523997187\\
455	0.00504261231016473\\
456	0.00503370731012738\\
457	0.00502461492443078\\
458	0.00501532957139311\\
459	0.00500584538340648\\
460	0.0049961561850006\\
461	0.00498625546799501\\
462	0.00497613636666769\\
463	0.0049657916299069\\
464	0.004955213591057\\
465	0.00494439413520791\\
466	0.00493332466344369\\
467	0.00492199605246461\\
468	0.00491039860409967\\
469	0.00489852196541588\\
470	0.00488635495113334\\
471	0.00487388502444875\\
472	0.00486110016872916\\
473	0.00484798846091057\\
474	0.00483453733968041\\
475	0.00482073356715673\\
476	0.00480656318852167\\
477	0.00479201148947316\\
478	0.00477706295186214\\
479	0.0047617012074535\\
480	0.00474590898969505\\
481	0.0047296680830029\\
482	0.00471295926900949\\
483	0.00469576226956938\\
484	0.00467805567959101\\
485	0.00465981685182614\\
486	0.00464102171153467\\
487	0.00462164470289859\\
488	0.00460165873227493\\
489	0.00458103299112202\\
490	0.00455972502689214\\
491	0.0045376851652642\\
492	0.00451487713810451\\
493	0.00449126302561625\\
494	0.00446680334652431\\
495	0.00444145660496162\\
496	0.00441517925139808\\
497	0.00438792565232686\\
498	0.00435964810079153\\
499	0.00433029686131218\\
500	0.0042998202602825\\
501	0.00426816483262271\\
502	0.00423527553509954\\
503	0.00420109605439452\\
504	0.0041655692281557\\
505	0.00412863760719055\\
506	0.00409024419353347\\
507	0.00405033339713849\\
508	0.00400885226367352\\
509	0.00396575203901821\\
510	0.0039209901516933\\
511	0.00387453271320261\\
512	0.00382635766291394\\
513	0.00377645868634219\\
514	0.00372485008048032\\
515	0.00367157241905168\\
516	0.0036167007092409\\
517	0.00356035948260369\\
518	0.00350273687447986\\
519	0.00344408652000935\\
520	0.00338472437097065\\
521	0.00332502096721129\\
522	0.00326555364965393\\
523	0.00320726580269645\\
524	0.00315410754475045\\
525	0.00310637335838345\\
526	0.00306411528959599\\
527	0.00302688819689068\\
528	0.00299070228713309\\
529	0.00295512081968544\\
530	0.00291990283029932\\
531	0.00288470889343574\\
532	0.00284941466422512\\
533	0.00281394805285373\\
534	0.00277826688557104\\
535	0.00274196152703712\\
536	0.00270497313438378\\
537	0.00266726288515937\\
538	0.0026287901198124\\
539	0.00258951542707469\\
540	0.0025494161112003\\
541	0.00250846903797176\\
542	0.00246665102411426\\
543	0.00242393945402507\\
544	0.00238030871636102\\
545	0.00233567897679858\\
546	0.00229069692309558\\
547	0.00224669736708306\\
548	0.00220394841017211\\
549	0.00216151286075968\\
550	0.00211858016627274\\
551	0.00207506500891316\\
552	0.00203094114340961\\
553	0.00198621792500414\\
554	0.00194091184632872\\
555	0.00189504487825128\\
556	0.00184864360533295\\
557	0.00180173986240179\\
558	0.00175437208562482\\
559	0.00170658633949408\\
560	0.00165892004054196\\
561	0.00161280548965377\\
562	0.00156727794385113\\
563	0.00152145163057755\\
564	0.00147531828604501\\
565	0.0014289079242206\\
566	0.00138225391875487\\
567	0.00133539122235288\\
568	0.00128835585617032\\
569	0.00124118449461087\\
570	0.00119391396189252\\
571	0.00114688118876282\\
572	0.00110017098334024\\
573	0.00105332337801152\\
574	0.00100637555536751\\
575	0.000959368346163701\\
576	0.000912345329873068\\
577	0.000865352773868407\\
578	0.00081843950911817\\
579	0.000771656718608598\\
580	0.000725057614047103\\
581	0.000678696970711006\\
582	0.000632630483284291\\
583	0.000586913897120376\\
584	0.000541601859672158\\
585	0.000496746426604039\\
586	0.000452395149029589\\
587	0.000408588670365811\\
588	0.000365357795666471\\
589	0.000322720123375259\\
590	0.000280676711614414\\
591	0.000239312817487403\\
592	0.00019878864496184\\
593	0.000159293651685586\\
594	0.000121079888727804\\
595	8.45520570083909e-05\\
596	5.05092148680371e-05\\
597	2.07908715710836e-05\\
598	0\\
599	0\\
600	0\\
};
\addplot [color=mycolor6,solid,forget plot]
  table[row sep=crcr]{%
1	0.00564414169708746\\
2	0.00564413625517687\\
3	0.00564413071987849\\
4	0.00564412508958927\\
5	0.00564411936267863\\
6	0.00564411353748807\\
7	0.00564410761233051\\
8	0.00564410158549004\\
9	0.00564409545522124\\
10	0.00564408921974873\\
11	0.00564408287726674\\
12	0.00564407642593847\\
13	0.0056440698638956\\
14	0.00564406318923781\\
15	0.00564405640003209\\
16	0.00564404949431236\\
17	0.00564404247007876\\
18	0.00564403532529714\\
19	0.00564402805789849\\
20	0.00564402066577821\\
21	0.00564401314679571\\
22	0.00564400549877365\\
23	0.00564399771949732\\
24	0.00564398980671406\\
25	0.00564398175813259\\
26	0.00564397357142226\\
27	0.00564396524421251\\
28	0.00564395677409218\\
29	0.00564394815860864\\
30	0.00564393939526732\\
31	0.00564393048153081\\
32	0.00564392141481821\\
33	0.00564391219250438\\
34	0.00564390281191916\\
35	0.00564389327034665\\
36	0.00564388356502433\\
37	0.00564387369314236\\
38	0.00564386365184279\\
39	0.00564385343821856\\
40	0.00564384304931287\\
41	0.00564383248211816\\
42	0.00564382173357545\\
43	0.00564381080057326\\
44	0.00564379967994678\\
45	0.00564378836847701\\
46	0.00564377686288969\\
47	0.00564376515985457\\
48	0.00564375325598425\\
49	0.00564374114783338\\
50	0.00564372883189747\\
51	0.00564371630461208\\
52	0.00564370356235165\\
53	0.00564369060142858\\
54	0.00564367741809204\\
55	0.0056436640085269\\
56	0.0056436503688528\\
57	0.00564363649512285\\
58	0.00564362238332254\\
59	0.00564360802936866\\
60	0.00564359342910799\\
61	0.00564357857831631\\
62	0.00564356347269692\\
63	0.0056435481078796\\
64	0.00564353247941936\\
65	0.00564351658279498\\
66	0.00564350041340793\\
67	0.00564348396658084\\
68	0.00564346723755637\\
69	0.00564345022149561\\
70	0.00564343291347689\\
71	0.00564341530849428\\
72	0.00564339740145613\\
73	0.00564337918718361\\
74	0.00564336066040927\\
75	0.00564334181577543\\
76	0.00564332264783279\\
77	0.0056433031510387\\
78	0.00564328331975563\\
79	0.0056432631482496\\
80	0.00564324263068842\\
81	0.00564322176114014\\
82	0.00564320053357121\\
83	0.0056431789418448\\
84	0.00564315697971908\\
85	0.00564313464084535\\
86	0.00564311191876622\\
87	0.00564308880691379\\
88	0.00564306529860771\\
89	0.00564304138705329\\
90	0.00564301706533944\\
91	0.00564299232643684\\
92	0.00564296716319575\\
93	0.00564294156834404\\
94	0.00564291553448503\\
95	0.00564288905409542\\
96	0.005642862119523\\
97	0.00564283472298454\\
98	0.00564280685656343\\
99	0.00564277851220748\\
100	0.00564274968172658\\
101	0.00564272035679014\\
102	0.00564269052892486\\
103	0.00564266018951221\\
104	0.00564262932978591\\
105	0.00564259794082932\\
106	0.00564256601357286\\
107	0.00564253353879135\\
108	0.00564250050710135\\
109	0.00564246690895835\\
110	0.00564243273465392\\
111	0.0056423979743129\\
112	0.00564236261789058\\
113	0.00564232665516947\\
114	0.00564229007575648\\
115	0.00564225286907973\\
116	0.00564221502438551\\
117	0.00564217653073491\\
118	0.00564213737700059\\
119	0.00564209755186348\\
120	0.00564205704380929\\
121	0.00564201584112511\\
122	0.00564197393189578\\
123	0.00564193130400025\\
124	0.0056418879451079\\
125	0.00564184384267478\\
126	0.00564179898393962\\
127	0.00564175335591998\\
128	0.0056417069454081\\
129	0.00564165973896684\\
130	0.00564161172292541\\
131	0.00564156288337497\\
132	0.00564151320616429\\
133	0.00564146267689518\\
134	0.00564141128091767\\
135	0.00564135900332546\\
136	0.00564130582895089\\
137	0.00564125174235987\\
138	0.00564119672784683\\
139	0.00564114076942943\\
140	0.00564108385084303\\
141	0.00564102595553514\\
142	0.00564096706665966\\
143	0.005640907167071\\
144	0.00564084623931801\\
145	0.00564078426563768\\
146	0.00564072122794868\\
147	0.00564065710784477\\
148	0.00564059188658799\\
149	0.00564052554510143\\
150	0.0056404580639621\\
151	0.0056403894233934\\
152	0.00564031960325727\\
153	0.00564024858304622\\
154	0.00564017634187505\\
155	0.00564010285847223\\
156	0.00564002811117116\\
157	0.00563995207790084\\
158	0.00563987473617662\\
159	0.00563979606309024\\
160	0.00563971603529977\\
161	0.00563963462901919\\
162	0.0056395518200075\\
163	0.00563946758355766\\
164	0.00563938189448501\\
165	0.00563929472711542\\
166	0.00563920605527308\\
167	0.00563911585226782\\
168	0.00563902409088234\\
169	0.00563893074335885\\
170	0.00563883578138554\\
171	0.00563873917608285\\
172	0.00563864089798935\\
173	0.00563854091704757\\
174	0.00563843920258979\\
175	0.00563833572332355\\
176	0.00563823044731771\\
177	0.0056381233419882\\
178	0.00563801437408465\\
179	0.00563790350967728\\
180	0.00563779071414467\\
181	0.00563767595216268\\
182	0.00563755918769461\\
183	0.00563744038398316\\
184	0.00563731950354441\\
185	0.00563719650816463\\
186	0.00563707135889997\\
187	0.00563694401607988\\
188	0.00563681443931492\\
189	0.00563668258750926\\
190	0.00563654841887897\\
191	0.0056364118909759\\
192	0.00563627296071807\\
193	0.00563613158442602\\
194	0.00563598771786444\\
195	0.00563584131628652\\
196	0.00563569233447492\\
197	0.00563554072676165\\
198	0.00563538644697248\\
199	0.00563522944814338\\
200	0.00563506968249567\\
201	0.00563490710142288\\
202	0.00563474165547733\\
203	0.00563457329435654\\
204	0.00563440196688957\\
205	0.00563422762102303\\
206	0.00563405020380734\\
207	0.00563386966138245\\
208	0.00563368593896365\\
209	0.00563349898082722\\
210	0.00563330873029588\\
211	0.00563311512972439\\
212	0.0056329181204847\\
213	0.00563271764295124\\
214	0.00563251363648603\\
215	0.00563230603942387\\
216	0.0056320947890571\\
217	0.00563187982162077\\
218	0.00563166107227743\\
219	0.00563143847510205\\
220	0.00563121196306666\\
221	0.00563098146802532\\
222	0.00563074692069891\\
223	0.00563050825065989\\
224	0.00563026538631715\\
225	0.00563001825490073\\
226	0.00562976678244683\\
227	0.0056295108937827\\
228	0.00562925051251158\\
229	0.00562898556099765\\
230	0.00562871596035131\\
231	0.0056284416304141\\
232	0.00562816248974406\\
233	0.00562787845560098\\
234	0.00562758944393172\\
235	0.0056272953693556\\
236	0.00562699614514988\\
237	0.00562669168323515\\
238	0.00562638189416079\\
239	0.0056260666870904\\
240	0.00562574596978704\\
241	0.00562541964859858\\
242	0.00562508762844261\\
243	0.0056247498127913\\
244	0.00562440610365575\\
245	0.00562405640157011\\
246	0.00562370060557519\\
247	0.00562333861320116\\
248	0.00562297032044985\\
249	0.00562259562177602\\
250	0.00562221441006732\\
251	0.00562182657662356\\
252	0.00562143201113391\\
253	0.00562103060165317\\
254	0.00562062223457563\\
255	0.00562020679460738\\
256	0.00561978416473611\\
257	0.00561935422619877\\
258	0.00561891685844633\\
259	0.0056184719391061\\
260	0.00561801934394091\\
261	0.00561755894680571\\
262	0.00561709061960082\\
263	0.00561661423222263\\
264	0.00561612965251152\\
265	0.00561563674619742\\
266	0.00561513537684337\\
267	0.00561462540578792\\
268	0.00561410669208684\\
269	0.00561357909245545\\
270	0.00561304246121241\\
271	0.00561249665022646\\
272	0.005611941508868\\
273	0.00561137688396609\\
274	0.00561080261977376\\
275	0.00561021855794219\\
276	0.00560962453750519\\
277	0.00560902039487498\\
278	0.00560840596384905\\
279	0.00560778107562823\\
280	0.00560714555884466\\
281	0.00560649923957224\\
282	0.00560584194129897\\
283	0.00560517348489965\\
284	0.00560449368860877\\
285	0.00560380236799386\\
286	0.00560309933592902\\
287	0.00560238440256894\\
288	0.0056016573753236\\
289	0.00560091805883311\\
290	0.00560016625494342\\
291	0.00559940176268257\\
292	0.00559862437823763\\
293	0.00559783389493254\\
294	0.00559703010320651\\
295	0.00559621279059369\\
296	0.0055953817417038\\
297	0.00559453673820359\\
298	0.00559367755879988\\
299	0.00559280397922369\\
300	0.00559191577221573\\
301	0.00559101270751374\\
302	0.00559009455184119\\
303	0.00558916106889795\\
304	0.00558821201935285\\
305	0.00558724716083838\\
306	0.00558626624794766\\
307	0.00558526903223386\\
308	0.00558425526221214\\
309	0.00558322468336484\\
310	0.00558217703814907\\
311	0.00558111206600827\\
312	0.00558002950338697\\
313	0.00557892908374936\\
314	0.00557781053760209\\
315	0.00557667359252137\\
316	0.00557551797318496\\
317	0.00557434340140912\\
318	0.00557314959619112\\
319	0.00557193627375752\\
320	0.00557070314761891\\
321	0.00556944992863151\\
322	0.00556817632506571\\
323	0.00556688204268277\\
324	0.0055655667848199\\
325	0.0055642302524841\\
326	0.00556287214445611\\
327	0.00556149215740428\\
328	0.00556008998601007\\
329	0.00555866532310511\\
330	0.0055572178598215\\
331	0.00555574728575555\\
332	0.00555425328914656\\
333	0.00555273555707142\\
334	0.00555119377565612\\
335	0.00554962763030557\\
336	0.00554803680595303\\
337	0.00554642098733027\\
338	0.00554477985926037\\
339	0.00554311310697427\\
340	0.00554142041645326\\
341	0.00553970147479857\\
342	0.00553795597063039\\
343	0.00553618359451826\\
344	0.00553438403944448\\
345	0.0055325570013031\\
346	0.00553070217943665\\
347	0.00552881927721283\\
348	0.00552690800264369\\
349	0.00552496806904948\\
350	0.00552299919577002\\
351	0.00552100110892575\\
352	0.00551897354223103\\
353	0.00551691623786212\\
354	0.00551482894738188\\
355	0.00551271143272327\\
356	0.00551056346723349\\
357	0.00550838483677973\\
358	0.00550617534091777\\
359	0.00550393479412282\\
360	0.00550166302708257\\
361	0.00549935988804942\\
362	0.00549702524424956\\
363	0.00549465898334295\\
364	0.00549226101492717\\
365	0.00548983127207504\\
366	0.00548736971289305\\
367	0.00548487632208314\\
368	0.00548235111248626\\
369	0.00547979412658058\\
370	0.00547720543789952\\
371	0.00547458515232831\\
372	0.00547193340922715\\
373	0.00546925038231827\\
374	0.0054665362802615\\
375	0.00546379134682676\\
376	0.00546101586055519\\
377	0.00545821013377916\\
378	0.00545537451084841\\
379	0.00545250936538243\\
380	0.00544961509633896\\
381	0.00544669212265489\\
382	0.00544374087617907\\
383	0.00544076179257658\\
384	0.0054377552998406\\
385	0.0054347218039998\\
386	0.0054316616715437\\
387	0.00542857520801726\\
388	0.00542546263241958\\
389	0.00542232405135473\\
390	0.00541915943173047\\
391	0.00541596856053574\\
392	0.00541275099852629\\
393	0.00540950602802968\\
394	0.00540623259578367\\
395	0.00540292925287511\\
396	0.00539959409575478\\
397	0.00539622471454812\\
398	0.00539281815841277\\
399	0.00538937093322477\\
400	0.00538587905482985\\
401	0.00538233819374411\\
402	0.00537874396997767\\
403	0.00537509250059577\\
404	0.00537138136169903\\
405	0.00536760963869151\\
406	0.00536377640670455\\
407	0.00535988073120857\\
408	0.00535592166875451\\
409	0.00535189826786303\\
410	0.00534780957007553\\
411	0.00534365461117891\\
412	0.0053394324226149\\
413	0.00533514203308011\\
414	0.00533078247031827\\
415	0.0053263527630979\\
416	0.00532185194335856\\
417	0.0053172790484941\\
418	0.0053126331237232\\
419	0.00530791322447705\\
420	0.00530311841874507\\
421	0.00529824778923973\\
422	0.00529330043525395\\
423	0.00528827547412149\\
424	0.00528317204227609\\
425	0.00527798929587917\\
426	0.00527272640940693\\
427	0.00526738255837269\\
428	0.00526195660210042\\
429	0.00525644697598015\\
430	0.00525085202010332\\
431	0.00524516996856736\\
432	0.00523939893776977\\
433	0.00523353691359344\\
434	0.00522758173720634\\
435	0.00522153108974944\\
436	0.00521538247611547\\
437	0.00520913320773026\\
438	0.00520278038449278\\
439	0.0051963208761216\\
440	0.00518975130328123\\
441	0.0051830680190074\\
442	0.00517626709101539\\
443	0.00516934428574013\\
444	0.00516229505523642\\
445	0.00515511452830036\\
446	0.00514779750747866\\
447	0.00514033847401999\\
448	0.00513273160350065\\
449	0.00512497079664492\\
450	0.00511704973592452\\
451	0.00510896200076135\\
452	0.00510070378862286\\
453	0.00509229594263602\\
454	0.00508373464752929\\
455	0.00507501624632246\\
456	0.00506613720024991\\
457	0.00505709388351448\\
458	0.00504788255170718\\
459	0.00503849933846256\\
460	0.00502894026034941\\
461	0.00501920122093163\\
462	0.00500927797509956\\
463	0.00499916613887364\\
464	0.00498886118326885\\
465	0.0049783584277492\\
466	0.0049676530324458\\
467	0.00495673999020768\\
468	0.00494561412153549\\
469	0.00493427008262226\\
470	0.00492270241860566\\
471	0.00491090576179196\\
472	0.00489884744474724\\
473	0.00488650547011184\\
474	0.00487386914060477\\
475	0.0048609271690671\\
476	0.00484766764446001\\
477	0.00483407799596508\\
478	0.00482014494740563\\
479	0.004805854470123\\
480	0.00479119173146963\\
481	0.00477614103689769\\
482	0.00476068576490874\\
483	0.00474480829222686\\
484	0.00472848990354356\\
485	0.00471171066811643\\
486	0.00469444922446338\\
487	0.00467668224998897\\
488	0.00465838538681553\\
489	0.00463953438165116\\
490	0.00462010347189625\\
491	0.00460006522725015\\
492	0.00457938821124362\\
493	0.00455802863850385\\
494	0.00453593713953987\\
495	0.00451307676840705\\
496	0.00448940875107804\\
497	0.00446489274774366\\
498	0.00443948628249789\\
499	0.00441314468988377\\
500	0.00438582105601528\\
501	0.00435746622138655\\
502	0.0043280288572434\\
503	0.00429745539131718\\
504	0.00426569009173831\\
505	0.00423267521069181\\
506	0.00419835120334439\\
507	0.00416265704191008\\
508	0.00412553064833676\\
509	0.00408690947531639\\
510	0.00404673127247722\\
511	0.0040049350833286\\
512	0.00396146252888379\\
513	0.00391625944696082\\
514	0.00386927797080916\\
515	0.00382047918193405\\
516	0.00376983645483958\\
517	0.00371733938793335\\
518	0.00366299856508242\\
519	0.0036068522353349\\
520	0.00354898254480159\\
521	0.00348952212899389\\
522	0.00342866087146759\\
523	0.00336665476859641\\
524	0.00330382152996528\\
525	0.00324053311458391\\
526	0.0031773849182677\\
527	0.00311536749133956\\
528	0.00305847479549083\\
529	0.00300703652234188\\
530	0.0029611558647825\\
531	0.00292045584447661\\
532	0.00288119041155718\\
533	0.00284263737455762\\
534	0.00280433019479108\\
535	0.00276613286193941\\
536	0.00272783883980301\\
537	0.00268936526453775\\
538	0.00265065306074691\\
539	0.00261155905477273\\
540	0.00257176095863499\\
541	0.0025312176445629\\
542	0.0024898888746985\\
543	0.00244772979898243\\
544	0.00240470570191919\\
545	0.00236079274626476\\
546	0.00231596758188079\\
547	0.0022701870295366\\
548	0.00222336772233065\\
549	0.0021766009736321\\
550	0.00213084111050196\\
551	0.00208636429435196\\
552	0.00204199580507713\\
553	0.00199714708603788\\
554	0.00195173779208016\\
555	0.00190574754438444\\
556	0.00185919235620512\\
557	0.0018120959308981\\
558	0.00176448917511599\\
559	0.00171640898605244\\
560	0.00166789981490248\\
561	0.00161900323277768\\
562	0.00157077130801954\\
563	0.00152412909216968\\
564	0.00147778367616564\\
565	0.00143118179454771\\
566	0.00138432726713373\\
567	0.00133725430873315\\
568	0.00129000081580688\\
569	0.00124260613446559\\
570	0.00119511043658036\\
571	0.0011475541064327\\
572	0.00110018398429605\\
573	0.00105332347969338\\
574	0.00100637557683123\\
575	0.000959368356499549\\
576	0.000912345335012991\\
577	0.000865352776319937\\
578	0.000818439510213676\\
579	0.000771656719059582\\
580	0.000725057614214845\\
581	0.000678696970765909\\
582	0.000632630483299512\\
583	0.000586913897123733\\
584	0.00054160185967268\\
585	0.000496746426604084\\
586	0.000452395149029592\\
587	0.000408588670365816\\
588	0.000365357795666476\\
589	0.000322720123375263\\
590	0.000280676711614417\\
591	0.000239312817487403\\
592	0.00019878864496184\\
593	0.000159293651685585\\
594	0.000121079888727804\\
595	8.45520570083909e-05\\
596	5.05092148680374e-05\\
597	2.07908715710837e-05\\
598	0\\
599	0\\
600	0\\
};
\addplot [color=mycolor7,solid,forget plot]
  table[row sep=crcr]{%
1	0.00588219123805133\\
2	0.00588218208726686\\
3	0.00588217278113544\\
4	0.00588216331702408\\
5	0.00588215369225527\\
6	0.00588214390410623\\
7	0.00588213394980824\\
8	0.00588212382654572\\
9	0.00588211353145555\\
10	0.00588210306162625\\
11	0.00588209241409704\\
12	0.00588208158585732\\
13	0.00588207057384544\\
14	0.00588205937494817\\
15	0.00588204798599965\\
16	0.00588203640378059\\
17	0.00588202462501728\\
18	0.00588201264638077\\
19	0.00588200046448591\\
20	0.00588198807589044\\
21	0.00588197547709392\\
22	0.00588196266453694\\
23	0.00588194963459998\\
24	0.00588193638360238\\
25	0.00588192290780144\\
26	0.00588190920339136\\
27	0.00588189526650207\\
28	0.00588188109319825\\
29	0.00588186667947823\\
30	0.00588185202127291\\
31	0.00588183711444447\\
32	0.0058818219547854\\
33	0.00588180653801729\\
34	0.00588179085978959\\
35	0.00588177491567842\\
36	0.00588175870118537\\
37	0.00588174221173632\\
38	0.00588172544267996\\
39	0.00588170838928676\\
40	0.00588169104674747\\
41	0.00588167341017188\\
42	0.00588165547458745\\
43	0.00588163723493796\\
44	0.00588161868608199\\
45	0.00588159982279169\\
46	0.00588158063975125\\
47	0.00588156113155537\\
48	0.00588154129270789\\
49	0.00588152111762016\\
50	0.00588150060060958\\
51	0.00588147973589798\\
52	0.00588145851761012\\
53	0.00588143693977191\\
54	0.00588141499630895\\
55	0.00588139268104478\\
56	0.00588136998769915\\
57	0.00588134690988632\\
58	0.0058813234411134\\
59	0.00588129957477834\\
60	0.00588127530416842\\
61	0.00588125062245813\\
62	0.00588122552270752\\
63	0.00588119999786017\\
64	0.00588117404074132\\
65	0.00588114764405584\\
66	0.00588112080038635\\
67	0.00588109350219113\\
68	0.00588106574180199\\
69	0.0058810375114224\\
70	0.00588100880312507\\
71	0.00588097960885001\\
72	0.00588094992040228\\
73	0.00588091972944976\\
74	0.00588088902752087\\
75	0.00588085780600227\\
76	0.00588082605613652\\
77	0.00588079376901975\\
78	0.00588076093559925\\
79	0.00588072754667089\\
80	0.00588069359287686\\
81	0.00588065906470292\\
82	0.00588062395247593\\
83	0.00588058824636128\\
84	0.00588055193636014\\
85	0.00588051501230688\\
86	0.00588047746386625\\
87	0.00588043928053062\\
88	0.00588040045161711\\
89	0.00588036096626489\\
90	0.00588032081343208\\
91	0.00588027998189295\\
92	0.00588023846023473\\
93	0.00588019623685472\\
94	0.00588015329995716\\
95	0.00588010963755001\\
96	0.00588006523744187\\
97	0.00588002008723857\\
98	0.00587997417434008\\
99	0.005879927485937\\
100	0.00587988000900718\\
101	0.0058798317303124\\
102	0.00587978263639473\\
103	0.00587973271357297\\
104	0.00587968194793907\\
105	0.00587963032535454\\
106	0.0058795778314466\\
107	0.00587952445160447\\
108	0.00587947017097546\\
109	0.00587941497446109\\
110	0.00587935884671325\\
111	0.00587930177212999\\
112	0.00587924373485153\\
113	0.00587918471875616\\
114	0.00587912470745592\\
115	0.00587906368429241\\
116	0.0058790016323324\\
117	0.00587893853436342\\
118	0.0058788743728894\\
119	0.00587880913012591\\
120	0.00587874278799578\\
121	0.00587867532812425\\
122	0.00587860673183428\\
123	0.00587853698014171\\
124	0.00587846605375035\\
125	0.00587839393304693\\
126	0.00587832059809621\\
127	0.00587824602863568\\
128	0.00587817020407035\\
129	0.0058780931034675\\
130	0.00587801470555129\\
131	0.0058779349886972\\
132	0.00587785393092651\\
133	0.00587777150990062\\
134	0.00587768770291526\\
135	0.00587760248689474\\
136	0.00587751583838573\\
137	0.00587742773355146\\
138	0.00587733814816538\\
139	0.00587724705760487\\
140	0.00587715443684479\\
141	0.00587706026045104\\
142	0.00587696450257372\\
143	0.00587686713694038\\
144	0.00587676813684901\\
145	0.00587666747516092\\
146	0.00587656512429337\\
147	0.00587646105621216\\
148	0.00587635524242384\\
149	0.00587624765396788\\
150	0.00587613826140854\\
151	0.00587602703482644\\
152	0.00587591394381001\\
153	0.00587579895744645\\
154	0.00587568204431256\\
155	0.00587556317246514\\
156	0.00587544230943093\\
157	0.00587531942219636\\
158	0.00587519447719651\\
159	0.00587506744030385\\
160	0.00587493827681619\\
161	0.00587480695144415\\
162	0.0058746734282978\\
163	0.0058745376708726\\
164	0.00587439964203436\\
165	0.00587425930400317\\
166	0.00587411661833645\\
167	0.00587397154591045\\
168	0.00587382404690054\\
169	0.00587367408075979\\
170	0.00587352160619589\\
171	0.00587336658114587\\
172	0.00587320896274888\\
173	0.00587304870731608\\
174	0.00587288577029793\\
175	0.00587272010624825\\
176	0.0058725516687845\\
177	0.0058723804105444\\
178	0.0058722062831377\\
179	0.00587202923709321\\
180	0.00587184922180032\\
181	0.00587166618544422\\
182	0.00587148007493482\\
183	0.005871290835828\\
184	0.0058710984122396\\
185	0.00587090274675077\\
186	0.00587070378030529\\
187	0.00587050145209808\\
188	0.00587029569945591\\
189	0.00587008645771111\\
190	0.0058698736600702\\
191	0.00586965723748165\\
192	0.00586943711850872\\
193	0.0058692132292203\\
194	0.00586898549312512\\
195	0.0058687538312082\\
196	0.00586851816221758\\
197	0.00586827840359564\\
198	0.00586803447412705\\
199	0.00586778630120221\\
200	0.00586753381139697\\
201	0.00586727693003613\\
202	0.0058670155811731\\
203	0.00586674968756943\\
204	0.00586647917067383\\
205	0.00586620395060098\\
206	0.00586592394611\\
207	0.00586563907458275\\
208	0.00586534925200168\\
209	0.00586505439292744\\
210	0.00586475441047631\\
211	0.00586444921629699\\
212	0.00586413872054743\\
213	0.00586382283187114\\
214	0.00586350145737323\\
215	0.00586317450259602\\
216	0.0058628418714946\\
217	0.00586250346641158\\
218	0.00586215918805206\\
219	0.00586180893545772\\
220	0.00586145260598089\\
221	0.00586109009525804\\
222	0.00586072129718302\\
223	0.00586034610387987\\
224	0.00585996440567515\\
225	0.00585957609106991\\
226	0.00585918104671132\\
227	0.00585877915736358\\
228	0.00585837030587866\\
229	0.00585795437316637\\
230	0.00585753123816403\\
231	0.00585710077780552\\
232	0.0058566628669898\\
233	0.00585621737854895\\
234	0.00585576418321546\\
235	0.005855303149589\\
236	0.00585483414410254\\
237	0.00585435703098766\\
238	0.00585387167223928\\
239	0.00585337792757956\\
240	0.00585287565442103\\
241	0.00585236470782888\\
242	0.00585184494048238\\
243	0.00585131620263546\\
244	0.00585077834207637\\
245	0.00585023120408629\\
246	0.00584967463139692\\
247	0.00584910846414732\\
248	0.00584853253983935\\
249	0.0058479466932923\\
250	0.00584735075659636\\
251	0.00584674455906493\\
252	0.00584612792718618\\
253	0.005845500684573\\
254	0.00584486265191258\\
255	0.0058442136469145\\
256	0.0058435534842581\\
257	0.00584288197553906\\
258	0.00584219892921499\\
259	0.00584150415055066\\
260	0.00584079744156258\\
261	0.00584007860096303\\
262	0.00583934742410421\\
263	0.00583860370292198\\
264	0.00583784722588\\
265	0.0058370777779139\\
266	0.00583629514037624\\
267	0.00583549909098182\\
268	0.00583468940375375\\
269	0.00583386584897045\\
270	0.00583302819311345\\
271	0.00583217619881607\\
272	0.00583130962481249\\
273	0.00583042822588784\\
274	0.00582953175282786\\
275	0.00582861995236874\\
276	0.00582769256714601\\
277	0.00582674933564253\\
278	0.00582578999213452\\
279	0.00582481426663568\\
280	0.00582382188483897\\
281	0.00582281256805669\\
282	0.00582178603316027\\
283	0.00582074199251977\\
284	0.00581968015394315\\
285	0.0058186002206152\\
286	0.00581750189103652\\
287	0.00581638485896217\\
288	0.00581524881334033\\
289	0.00581409343825085\\
290	0.0058129184128441\\
291	0.00581172341127963\\
292	0.00581050810266517\\
293	0.00580927215099572\\
294	0.00580801521509329\\
295	0.0058067369485468\\
296	0.00580543699965241\\
297	0.00580411501135488\\
298	0.00580277062118932\\
299	0.0058014034612239\\
300	0.00580001315800364\\
301	0.00579859933249527\\
302	0.00579716160003328\\
303	0.00579569957026767\\
304	0.00579421284711314\\
305	0.00579270102870023\\
306	0.00579116370732854\\
307	0.00578960046942194\\
308	0.00578801089548685\\
309	0.00578639456007275\\
310	0.00578475103173618\\
311	0.00578307987300787\\
312	0.00578138064036386\\
313	0.00577965288420035\\
314	0.00577789614881355\\
315	0.00577610997238415\\
316	0.00577429388696708\\
317	0.0057724474184875\\
318	0.00577057008674308\\
319	0.00576866140541335\\
320	0.00576672088207668\\
321	0.00576474801823573\\
322	0.00576274230935219\\
323	0.00576070324489125\\
324	0.00575863030837713\\
325	0.00575652297746055\\
326	0.00575438072399893\\
327	0.00575220301415095\\
328	0.00574998930848624\\
329	0.00574773906211215\\
330	0.00574545172481836\\
331	0.00574312674124184\\
332	0.00574076355105323\\
333	0.00573836158916711\\
334	0.00573592028597793\\
335	0.00573343906762434\\
336	0.00573091735628411\\
337	0.00572835457050286\\
338	0.00572575012555948\\
339	0.00572310343387171\\
340	0.00572041390544555\\
341	0.00571768094837294\\
342	0.00571490396938198\\
343	0.00571208237444478\\
344	0.00570921556944856\\
345	0.00570630296093632\\
346	0.00570334395692322\\
347	0.00570033796779702\\
348	0.00569728440731028\\
349	0.00569418269367389\\
350	0.00569103225076202\\
351	0.00568783250943989\\
352	0.00568458290902718\\
353	0.00568128289891095\\
354	0.00567793194032404\\
355	0.00567452950830646\\
356	0.00567107509386924\\
357	0.00566756820638286\\
358	0.00566400837621448\\
359	0.00566039515764186\\
360	0.00565672813207408\\
361	0.00565300691161411\\
362	0.00564923114300126\\
363	0.00564540051197693\\
364	0.00564151474812193\\
365	0.00563757363021977\\
366	0.00563357699220581\\
367	0.00562952472977064\\
368	0.00562541680769257\\
369	0.00562125326798283\\
370	0.00561703423893659\\
371	0.00561275994519176\\
372	0.0056084307189075\\
373	0.00560404701218465\\
374	0.00559960941085824\\
375	0.00559511864980179\\
376	0.00559057562988734\\
377	0.00558598143674747\\
378	0.00558133736148083\\
379	0.00557664492342907\\
380	0.00557190589512597\\
381	0.00556712232947278\\
382	0.00556229658911968\\
383	0.00555743137792129\\
384	0.00555252977416729\\
385	0.00554759526504962\\
386	0.00554263178149162\\
387	0.00553764373200467\\
388	0.00553263603363741\\
389	0.00552761413726744\\
390	0.00552258404403517\\
391	0.00551755230432516\\
392	0.00551252599174899\\
393	0.00550751264045885\\
394	0.00550252012886844\\
395	0.00549755648547518\\
396	0.0054926295808739\\
397	0.00548774667063801\\
398	0.00548291373303975\\
399	0.00547813451476601\\
400	0.00547340916938308\\
401	0.00546873232430192\\
402	0.00546409034107289\\
403	0.00545945746060361\\
404	0.00545479065452195\\
405	0.00545005260128847\\
406	0.00544524264586333\\
407	0.00544036015771497\\
408	0.00543540453319037\\
409	0.00543037519796206\\
410	0.00542527160953924\\
411	0.00542009325982682\\
412	0.00541483967771263\\
413	0.0054095104316578\\
414	0.00540410513226049\\
415	0.00539862343475789\\
416	0.0053930650414277\\
417	0.00538742970385511\\
418	0.00538171722506037\\
419	0.00537592746158062\\
420	0.00537006032535792\\
421	0.00536411578537901\\
422	0.00535809386903167\\
423	0.00535199466314406\\
424	0.00534581831466087\\
425	0.0053395650308661\\
426	0.00533323507899149\\
427	0.00532682878551841\\
428	0.00532034654460782\\
429	0.00531378882878005\\
430	0.00530715618751142\\
431	0.00530044924399701\\
432	0.00529366868970234\\
433	0.00528681527658234\\
434	0.00527988980632675\\
435	0.00527289311561445\\
436	0.00526582605648935\\
437	0.00525868947083078\\
438	0.00525148415773978\\
439	0.00524421083250835\\
440	0.0052368700757049\\
441	0.00522946227082878\\
442	0.00522198752898299\\
443	0.00521444559914102\\
444	0.00520683576294551\\
445	0.00519915671370076\\
446	0.00519140642050471\\
447	0.0051835819805921\\
448	0.00517567946634349\\
449	0.00516769377863308\\
450	0.00515961852592683\\
451	0.005151445958989\\
452	0.00514316699999214\\
453	0.0051347713274009\\
454	0.00512624707099367\\
455	0.00511758257154256\\
456	0.00510876815839775\\
457	0.00509979921378732\\
458	0.0050906715587827\\
459	0.00508138073314177\\
460	0.00507192196861216\\
461	0.00506229016033015\\
462	0.00505247983613435\\
463	0.00504248512515146\\
464	0.00503229972450469\\
465	0.00502191686506552\\
466	0.00501132927698835\\
467	0.00500052915644913\\
468	0.00498950813675828\\
469	0.00497825727254909\\
470	0.0049667670647291\\
471	0.00495502762238985\\
472	0.0049430541617577\\
473	0.00493085021822894\\
474	0.00491840729180021\\
475	0.00490571632057168\\
476	0.00489276766813617\\
477	0.00487955113536917\\
478	0.00486605606432649\\
479	0.00485227131562047\\
480	0.00483818530089091\\
481	0.00482378604549152\\
482	0.00480906125528886\\
483	0.00479399839257722\\
484	0.00477858473533605\\
485	0.00476280739141535\\
486	0.00474665320711469\\
487	0.00473010861408686\\
488	0.00471314021482741\\
489	0.00469570132323347\\
490	0.00467777053771616\\
491	0.00465932506538153\\
492	0.00464034054077538\\
493	0.00462079104619254\\
494	0.00460064903615265\\
495	0.00457988331032009\\
496	0.00455845203088011\\
497	0.00453630139290828\\
498	0.00451339335556852\\
499	0.00448968753890518\\
500	0.00446514170582074\\
501	0.00443971084097204\\
502	0.0044133469625095\\
503	0.0043860022132521\\
504	0.00435762631213689\\
505	0.00432816648945773\\
506	0.00429756746269201\\
507	0.00426577144806584\\
508	0.00423271823782212\\
509	0.0041983453407199\\
510	0.00416258819980435\\
511	0.00412538050824811\\
512	0.00408665464676958\\
513	0.0040463422770657\\
514	0.00400437516965353\\
515	0.00396068628906017\\
516	0.00391521098887378\\
517	0.00386788870705819\\
518	0.00381866515000677\\
519	0.00376749505137355\\
520	0.00371434520876362\\
521	0.0036591985190329\\
522	0.00360205971707007\\
523	0.00354297258640321\\
524	0.00348201672566389\\
525	0.00341931897600911\\
526	0.00335506276957203\\
527	0.00328949729187063\\
528	0.00322293267212333\\
529	0.00315573339194042\\
530	0.00308848525482472\\
531	0.00302213661827374\\
532	0.00296050629345285\\
533	0.00290430468982034\\
534	0.00285370670007441\\
535	0.00280842771725674\\
536	0.00276558846072333\\
537	0.00272363262543577\\
538	0.00268211738834541\\
539	0.00264064638487863\\
540	0.00259910961814965\\
541	0.00255741339571434\\
542	0.00251548061943308\\
543	0.00247326495309667\\
544	0.00243050165550552\\
545	0.00238698086892067\\
546	0.00234265994943839\\
547	0.00229749658658826\\
548	0.00225144245025375\\
549	0.0022044690067269\\
550	0.00215651952593421\\
551	0.00210750156462815\\
552	0.00205873951659417\\
553	0.0020109931608278\\
554	0.00196454992977493\\
555	0.00191823942837985\\
556	0.00187148664771676\\
557	0.00182422356319178\\
558	0.00177642326600128\\
559	0.0017281096045976\\
560	0.00167931427601956\\
561	0.00163007921572574\\
562	0.00158045429977808\\
563	0.00153047646411673\\
564	0.00148151315013013\\
565	0.00143416985390059\\
566	0.00138705937387004\\
567	0.00133975888995393\\
568	0.00129226939301679\\
569	0.0012446284113432\\
570	0.00119687824718637\\
571	0.0011490625211639\\
572	0.00110122520101772\\
573	0.0010534821435598\\
574	0.00100637667104439\\
575	0.000959368504257833\\
576	0.000912345403221071\\
577	0.000865352810900999\\
578	0.000818439527258546\\
579	0.000771656726968869\\
580	0.00072505761760655\\
581	0.000678696972083882\\
582	0.000632630483751538\\
583	0.000586913897255451\\
584	0.000541601859703357\\
585	0.000496746426609176\\
586	0.000452395149030046\\
587	0.000408588670365811\\
588	0.000365357795666471\\
589	0.000322720123375259\\
590	0.000280676711614415\\
591	0.000239312817487403\\
592	0.000198788644961841\\
593	0.000159293651685586\\
594	0.000121079888727804\\
595	8.45520570083908e-05\\
596	5.05092148680372e-05\\
597	2.07908715710836e-05\\
598	0\\
599	0\\
600	0\\
};
\addplot [color=mycolor8,solid,forget plot]
  table[row sep=crcr]{%
1	0.00647624004771694\\
2	0.00647622387146877\\
3	0.00647620742204131\\
4	0.00647619069483262\\
5	0.00647617368516355\\
6	0.00647615638827646\\
7	0.0064761387993338\\
8	0.0064761209134169\\
9	0.00647610272552456\\
10	0.00647608423057166\\
11	0.00647606542338788\\
12	0.00647604629871608\\
13	0.00647602685121104\\
14	0.00647600707543784\\
15	0.0064759869658705\\
16	0.00647596651689038\\
17	0.00647594572278466\\
18	0.00647592457774483\\
19	0.00647590307586496\\
20	0.0064758812111402\\
21	0.00647585897746516\\
22	0.00647583636863205\\
23	0.0064758133783292\\
24	0.00647579000013921\\
25	0.00647576622753725\\
26	0.00647574205388915\\
27	0.00647571747244984\\
28	0.00647569247636121\\
29	0.00647566705865046\\
30	0.00647564121222805\\
31	0.00647561492988589\\
32	0.00647558820429525\\
33	0.00647556102800488\\
34	0.00647553339343885\\
35	0.0064755052928946\\
36	0.00647547671854082\\
37	0.00647544766241516\\
38	0.00647541811642238\\
39	0.00647538807233187\\
40	0.00647535752177552\\
41	0.00647532645624541\\
42	0.00647529486709158\\
43	0.00647526274551955\\
44	0.0064752300825881\\
45	0.00647519686920668\\
46	0.00647516309613301\\
47	0.00647512875397063\\
48	0.00647509383316633\\
49	0.00647505832400752\\
50	0.0064750222166196\\
51	0.00647498550096336\\
52	0.00647494816683225\\
53	0.00647491020384958\\
54	0.00647487160146572\\
55	0.00647483234895532\\
56	0.00647479243541441\\
57	0.00647475184975742\\
58	0.00647471058071415\\
59	0.0064746686168269\\
60	0.00647462594644722\\
61	0.00647458255773297\\
62	0.00647453843864493\\
63	0.00647449357694374\\
64	0.0064744479601865\\
65	0.00647440157572358\\
66	0.00647435441069509\\
67	0.00647430645202757\\
68	0.00647425768643044\\
69	0.00647420810039235\\
70	0.00647415768017784\\
71	0.00647410641182344\\
72	0.00647405428113408\\
73	0.00647400127367924\\
74	0.00647394737478918\\
75	0.00647389256955111\\
76	0.00647383684280508\\
77	0.00647378017914003\\
78	0.00647372256288978\\
79	0.00647366397812885\\
80	0.00647360440866827\\
81	0.00647354383805118\\
82	0.00647348224954876\\
83	0.00647341962615564\\
84	0.00647335595058546\\
85	0.00647329120526639\\
86	0.00647322537233646\\
87	0.00647315843363897\\
88	0.00647309037071768\\
89	0.006473021164812\\
90	0.00647295079685217\\
91	0.0064728792474542\\
92	0.00647280649691494\\
93	0.00647273252520689\\
94	0.00647265731197301\\
95	0.00647258083652153\\
96	0.00647250307782057\\
97	0.00647242401449275\\
98	0.00647234362480961\\
99	0.00647226188668611\\
100	0.00647217877767497\\
101	0.0064720942749609\\
102	0.00647200835535479\\
103	0.00647192099528782\\
104	0.0064718321708054\\
105	0.00647174185756121\\
106	0.006471650030811\\
107	0.00647155666540625\\
108	0.00647146173578796\\
109	0.0064713652159802\\
110	0.00647126707958353\\
111	0.00647116729976853\\
112	0.00647106584926904\\
113	0.0064709627003754\\
114	0.0064708578249275\\
115	0.00647075119430803\\
116	0.00647064277943524\\
117	0.00647053255075598\\
118	0.00647042047823828\\
119	0.00647030653136422\\
120	0.00647019067912238\\
121	0.00647007289000054\\
122	0.00646995313197789\\
123	0.00646983137251749\\
124	0.00646970757855844\\
125	0.0064695817165082\\
126	0.00646945375223438\\
127	0.00646932365105696\\
128	0.00646919137774018\\
129	0.00646905689648426\\
130	0.00646892017091722\\
131	0.00646878116408651\\
132	0.00646863983845066\\
133	0.00646849615587075\\
134	0.00646835007760198\\
135	0.00646820156428492\\
136	0.00646805057593706\\
137	0.0064678970719439\\
138	0.00646774101105035\\
139	0.00646758235135188\\
140	0.00646742105028574\\
141	0.00646725706462213\\
142	0.00646709035045533\\
143	0.0064669208631949\\
144	0.00646674855755685\\
145	0.00646657338755481\\
146	0.00646639530649136\\
147	0.00646621426694923\\
148	0.00646603022078284\\
149	0.00646584311910966\\
150	0.00646565291230193\\
151	0.00646545954997842\\
152	0.00646526298099641\\
153	0.00646506315344396\\
154	0.00646486001463229\\
155	0.00646465351108868\\
156	0.0064644435885495\\
157	0.00646423019195383\\
158	0.00646401326543751\\
159	0.00646379275232764\\
160	0.00646356859513789\\
161	0.00646334073556431\\
162	0.00646310911448212\\
163	0.00646287367194317\\
164	0.00646263434717474\\
165	0.00646239107857922\\
166	0.00646214380373511\\
167	0.00646189245939954\\
168	0.00646163698151231\\
169	0.00646137730520183\\
170	0.00646111336479263\\
171	0.00646084509381557\\
172	0.00646057242501987\\
173	0.00646029529038804\\
174	0.00646001362115334\\
175	0.00645972734782021\\
176	0.00645943640018777\\
177	0.00645914070737652\\
178	0.00645884019785802\\
179	0.00645853479948794\\
180	0.00645822443954161\\
181	0.00645790904475235\\
182	0.00645758854135083\\
183	0.00645726285510531\\
184	0.00645693191136017\\
185	0.00645659563507051\\
186	0.006456253950829\\
187	0.00645590678287998\\
188	0.00645555405511283\\
189	0.00645519569102329\\
190	0.00645483161362372\\
191	0.00645446174526723\\
192	0.00645408600731315\\
193	0.00645370431946377\\
194	0.0064533165983436\\
195	0.0064529227541831\\
196	0.00645252268249495\\
197	0.00645211624207562\\
198	0.00645170319487064\\
199	0.00645128303787878\\
200	0.00645085563272166\\
201	0.0064504208551466\\
202	0.00644997857879551\\
203	0.00644952867516996\\
204	0.00644907101359553\\
205	0.00644860546118575\\
206	0.00644813188280542\\
207	0.00644765014103328\\
208	0.00644716009612396\\
209	0.00644666160596958\\
210	0.00644615452606039\\
211	0.00644563870944506\\
212	0.00644511400669005\\
213	0.0064445802658385\\
214	0.00644403733236834\\
215	0.00644348504914973\\
216	0.00644292325640165\\
217	0.00644235179164807\\
218	0.00644177048967304\\
219	0.00644117918247519\\
220	0.0064405776992215\\
221	0.00643996586620005\\
222	0.00643934350677232\\
223	0.0064387104413243\\
224	0.00643806648721708\\
225	0.00643741145873626\\
226	0.00643674516704073\\
227	0.00643606742011057\\
228	0.00643537802269375\\
229	0.00643467677625234\\
230	0.00643396347890729\\
231	0.00643323792538263\\
232	0.00643249990694855\\
233	0.00643174921136343\\
234	0.00643098562281489\\
235	0.00643020892185986\\
236	0.00642941888536348\\
237	0.00642861528643711\\
238	0.00642779789437499\\
239	0.00642696647459007\\
240	0.00642612078854862\\
241	0.00642526059370366\\
242	0.00642438564342734\\
243	0.00642349568694218\\
244	0.00642259046925103\\
245	0.00642166973106612\\
246	0.00642073320873672\\
247	0.00641978063417571\\
248	0.00641881173478507\\
249	0.00641782623338007\\
250	0.00641682384811246\\
251	0.00641580429239246\\
252	0.00641476727480949\\
253	0.00641371249905202\\
254	0.00641263966382602\\
255	0.00641154846277254\\
256	0.00641043858438417\\
257	0.00640930971192017\\
258	0.00640816152332094\\
259	0.00640699369112094\\
260	0.00640580588236107\\
261	0.00640459775849952\\
262	0.00640336897532169\\
263	0.00640211918284917\\
264	0.00640084802524742\\
265	0.00639955514073233\\
266	0.00639824016147552\\
267	0.00639690271350842\\
268	0.00639554241662507\\
269	0.00639415888428342\\
270	0.00639275172350507\\
271	0.00639132053477362\\
272	0.00638986491193117\\
273	0.00638838444207315\\
274	0.00638687870544136\\
275	0.00638534727531509\\
276	0.00638378971790053\\
277	0.00638220559221794\\
278	0.00638059444998737\\
279	0.00637895583551238\\
280	0.00637728928556208\\
281	0.00637559432925129\\
282	0.00637387048791925\\
283	0.00637211727500619\\
284	0.00637033419592853\\
285	0.00636852074795187\\
286	0.00636667642006241\\
287	0.00636480069283646\\
288	0.00636289303830807\\
289	0.0063609529198348\\
290	0.0063589797919614\\
291	0.00635697310028176\\
292	0.00635493228129887\\
293	0.00635285676228271\\
294	0.00635074596112604\\
295	0.00634859928619845\\
296	0.00634641613619815\\
297	0.00634419590000159\\
298	0.00634193795651116\\
299	0.00633964167450065\\
300	0.00633730641245856\\
301	0.00633493151842919\\
302	0.00633251632985156\\
303	0.00633006017339608\\
304	0.00632756236479882\\
305	0.00632502220869358\\
306	0.00632243899844158\\
307	0.00631981201595874\\
308	0.00631714053154054\\
309	0.00631442380368444\\
310	0.00631166107890984\\
311	0.00630885159157541\\
312	0.00630599456369383\\
313	0.00630308920474416\\
314	0.00630013471148108\\
315	0.00629713026774185\\
316	0.00629407504425036\\
317	0.00629096819841817\\
318	0.00628780887414304\\
319	0.00628459620160424\\
320	0.006281329297055\\
321	0.00627800726261194\\
322	0.00627462918604145\\
323	0.00627119414054285\\
324	0.00626770118452838\\
325	0.00626414936139994\\
326	0.00626053769932257\\
327	0.00625686521099455\\
328	0.00625313089341405\\
329	0.00624933372764253\\
330	0.00624547267856455\\
331	0.00624154669464419\\
332	0.00623755470767811\\
333	0.00623349563254495\\
334	0.00622936836695184\\
335	0.00622517179117711\\
336	0.00622090476781047\\
337	0.0062165661414898\\
338	0.00621215473863561\\
339	0.00620766936718299\\
340	0.00620310881631192\\
341	0.00619847185617595\\
342	0.00619375723763056\\
343	0.00618896369196163\\
344	0.00618408993061516\\
345	0.00617913464492978\\
346	0.00617409650587355\\
347	0.00616897416378686\\
348	0.00616376624813409\\
349	0.00615847136726692\\
350	0.00615308810820268\\
351	0.00614761503642237\\
352	0.00614205069569307\\
353	0.00613639360792149\\
354	0.0061306422730459\\
355	0.0061247951689757\\
356	0.00611885075158952\\
357	0.00611280745480499\\
358	0.00610666369073642\\
359	0.00610041784995903\\
360	0.00609406830190338\\
361	0.00608761339540712\\
362	0.00608105145945854\\
363	0.00607438080417158\\
364	0.00606759972204199\\
365	0.00606070648954351\\
366	0.00605369936913638\\
367	0.00604657661177525\\
368	0.00603933646002237\\
369	0.00603197715189596\\
370	0.00602449692561097\\
371	0.00601689402540503\\
372	0.00600916670868642\\
373	0.0060013132547934\\
374	0.00599333197572258\\
375	0.00598522122926663\\
376	0.00597697943510633\\
377	0.00596860509453244\\
378	0.0059600968146372\\
379	0.00595145333802245\\
380	0.0059426735793323\\
381	0.00593375667024947\\
382	0.00592470201501394\\
383	0.00591550935905417\\
384	0.00590617887399451\\
385	0.00589671126314147\\
386	0.00588710789256086\\
387	0.0058773709539363\\
388	0.00586750366609288\\
389	0.00585751052076736\\
390	0.00584739756964675\\
391	0.00583717286767827\\
392	0.00582684699909627\\
393	0.00581643373854259\\
394	0.00580595090300746\\
395	0.00579542146859583\\
396	0.00578487508013719\\
397	0.00577434973701437\\
398	0.00576389390029359\\
399	0.00575356936161976\\
400	0.00574345493461472\\
401	0.00573365119175744\\
402	0.00572428646084609\\
403	0.00571552405381711\\
404	0.00570756948526384\\
405	0.00570020151858036\\
406	0.00569271599057066\\
407	0.00568511155554668\\
408	0.00567738689906436\\
409	0.00566954074321656\\
410	0.00566157185251424\\
411	0.00565347904043678\\
412	0.00564526117674768\\
413	0.00563691719568569\\
414	0.00562844610515465\\
415	0.00561984699703459\\
416	0.00561111905869675\\
417	0.00560226158565433\\
418	0.00559327399484952\\
419	0.00558415583698435\\
420	0.00557490681356992\\
421	0.00556552679656875\\
422	0.00555601585045792\\
423	0.00554637425700223\\
424	0.0055366025430567\\
425	0.00552670151176883\\
426	0.00551667227767325\\
427	0.0055065163065078\\
428	0.00549623546132956\\
429	0.00548583205295491\\
430	0.0054753088941555\\
431	0.00546466935862506\\
432	0.0054539174425402\\
433	0.00544305782176041\\
434	0.00543209591680835\\
435	0.00542103797090364\\
436	0.0054098911331517\\
437	0.00539866354596811\\
438	0.00538736443515098\\
439	0.00537600420007305\\
440	0.00536459450004577\\
441	0.0053531483307078\\
442	0.00534168008187833\\
443	0.00533020556475153\\
444	0.00531874199109124\\
445	0.00530730788008513\\
446	0.00529592285891626\\
447	0.00528460730995834\\
448	0.00527338179949224\\
449	0.0052622661981414\\
450	0.00525127836917647\\
451	0.0052404322533988\\
452	0.0052297351121516\\
453	0.00521918359548272\\
454	0.00520875817253184\\
455	0.00519841518453535\\
456	0.00518807572061038\\
457	0.00517761143626688\\
458	0.0051670056819433\\
459	0.00515625832843616\\
460	0.00514536935435214\\
461	0.00513433882988236\\
462	0.00512316690726056\\
463	0.00511185380605984\\
464	0.00510039979199753\\
465	0.00508880514764454\\
466	0.00507707013315591\\
467	0.00506519493487367\\
468	0.00505317959946346\\
469	0.00504102395116322\\
470	0.00502872748930082\\
471	0.00501628925885373\\
472	0.00500370765351237\\
473	0.0049909793991094\\
474	0.00497809968549204\\
475	0.00496506227722912\\
476	0.00495185923263095\\
477	0.00493848060039775\\
478	0.00492491410648339\\
479	0.00491114485115891\\
480	0.00489715505779419\\
481	0.00488292393090441\\
482	0.00486842771334284\\
483	0.00485364008282515\\
484	0.00483853311333963\\
485	0.00482307919412716\\
486	0.00480725458589019\\
487	0.00479104299851726\\
488	0.00477444558520379\\
489	0.00475747358007265\\
490	0.00474011146481785\\
491	0.00472234262990966\\
492	0.00470414929833351\\
493	0.00468551243506386\\
494	0.00466641161320032\\
495	0.00464682485774645\\
496	0.00462672867982591\\
497	0.00460609842400275\\
498	0.00458490755158626\\
499	0.00456312446528458\\
500	0.00454069397053237\\
501	0.0045175760745551\\
502	0.00449372477696406\\
503	0.00446905426462991\\
504	0.00444351900407773\\
505	0.00441707207804545\\
506	0.00438966452483288\\
507	0.00436124514501922\\
508	0.00433175993730267\\
509	0.00430115198250383\\
510	0.00426936140119424\\
511	0.0042363253111851\\
512	0.00420197783560677\\
513	0.00416625008663321\\
514	0.00412906987623692\\
515	0.00409036252097511\\
516	0.00405005428318096\\
517	0.00400807037188265\\
518	0.00396433572973611\\
519	0.00391877605958476\\
520	0.00387131917511654\\
521	0.00382189674995734\\
522	0.00377044651661206\\
523	0.00371691446152693\\
524	0.00366125824878971\\
525	0.00360345349321673\\
526	0.00354350693164753\\
527	0.0034814518998612\\
528	0.00341735621285175\\
529	0.00335133319318523\\
530	0.00328355130589948\\
531	0.00321424345000754\\
532	0.00314370305522421\\
533	0.00307227580608966\\
534	0.00300050952196754\\
535	0.00292929211302619\\
536	0.00286178178697941\\
537	0.00279960223019647\\
538	0.00274303581167043\\
539	0.00269196434323486\\
540	0.00264490546750042\\
541	0.00259895181245598\\
542	0.00255370632164094\\
543	0.0025086849952371\\
544	0.00246359209360616\\
545	0.0024183959307389\\
546	0.00237300335466916\\
547	0.00232734162646493\\
548	0.00228137378200997\\
549	0.00223471812268854\\
550	0.00218726473464073\\
551	0.00213896836534945\\
552	0.00208977713704287\\
553	0.00203960532090274\\
554	0.00198836693274954\\
555	0.00193738925710515\\
556	0.00188742405447135\\
557	0.00183875487424147\\
558	0.00179049315303823\\
559	0.00174185184830918\\
560	0.00169278433015013\\
561	0.00164324280767594\\
562	0.0015932603719237\\
563	0.00154287867702145\\
564	0.00149215207571757\\
565	0.00144111678732867\\
566	0.00139123663602677\\
567	0.00134300099050608\\
568	0.00129517846879981\\
569	0.00124726049348067\\
570	0.00119922945498401\\
571	0.00115112545879614\\
572	0.00110299534370064\\
573	0.00105488697841206\\
574	0.00100684692524753\\
575	0.000959381521014027\\
576	0.000912346458482514\\
577	0.000865353251381383\\
578	0.000818439752416602\\
579	0.000771656841517049\\
580	0.000725057672851425\\
581	0.000678696996801308\\
582	0.000632630493807236\\
583	0.000586913900878002\\
584	0.000541601860815896\\
585	0.000496746426883291\\
586	0.000452395149078408\\
587	0.000408588670370421\\
588	0.000365357795666472\\
589	0.00032272012337526\\
590	0.000280676711614416\\
591	0.000239312817487404\\
592	0.000198788644961842\\
593	0.000159293651685588\\
594	0.000121079888727806\\
595	8.45520570083917e-05\\
596	5.05092148680373e-05\\
597	2.07908715710836e-05\\
598	0\\
599	0\\
600	0\\
};
\addplot [color=blue!25!mycolor7,solid,forget plot]
  table[row sep=crcr]{%
1	0.00682152599929207\\
2	0.00682151808048622\\
3	0.00682151002797638\\
4	0.00682150183951254\\
5	0.006821493512807\\
6	0.00682148504553382\\
7	0.00682147643532816\\
8	0.00682146767978563\\
9	0.00682145877646161\\
10	0.00682144972287067\\
11	0.00682144051648577\\
12	0.00682143115473767\\
13	0.00682142163501422\\
14	0.00682141195465958\\
15	0.00682140211097362\\
16	0.00682139210121112\\
17	0.00682138192258097\\
18	0.00682137157224552\\
19	0.00682136104731981\\
20	0.00682135034487067\\
21	0.00682133946191601\\
22	0.00682132839542409\\
23	0.00682131714231252\\
24	0.00682130569944763\\
25	0.00682129406364346\\
26	0.00682128223166108\\
27	0.00682127020020752\\
28	0.00682125796593495\\
29	0.00682124552543993\\
30	0.00682123287526229\\
31	0.00682122001188433\\
32	0.00682120693172984\\
33	0.0068211936311631\\
34	0.00682118010648805\\
35	0.00682116635394711\\
36	0.00682115236972032\\
37	0.00682113814992434\\
38	0.00682112369061126\\
39	0.00682110898776762\\
40	0.00682109403731346\\
41	0.00682107883510114\\
42	0.00682106337691419\\
43	0.00682104765846632\\
44	0.00682103167540016\\
45	0.00682101542328615\\
46	0.00682099889762152\\
47	0.00682098209382884\\
48	0.00682096500725499\\
49	0.00682094763316988\\
50	0.00682092996676526\\
51	0.00682091200315343\\
52	0.00682089373736587\\
53	0.00682087516435213\\
54	0.00682085627897833\\
55	0.00682083707602595\\
56	0.00682081755019037\\
57	0.00682079769607956\\
58	0.00682077750821269\\
59	0.00682075698101862\\
60	0.00682073610883456\\
61	0.00682071488590459\\
62	0.0068206933063781\\
63	0.00682067136430838\\
64	0.00682064905365105\\
65	0.00682062636826243\\
66	0.00682060330189814\\
67	0.00682057984821136\\
68	0.00682055600075125\\
69	0.00682053175296142\\
70	0.00682050709817799\\
71	0.00682048202962821\\
72	0.00682045654042859\\
73	0.00682043062358315\\
74	0.00682040427198172\\
75	0.00682037747839815\\
76	0.00682035023548839\\
77	0.00682032253578886\\
78	0.00682029437171437\\
79	0.00682026573555639\\
80	0.006820236619481\\
81	0.00682020701552719\\
82	0.00682017691560461\\
83	0.00682014631149174\\
84	0.0068201151948338\\
85	0.00682008355714087\\
86	0.00682005138978555\\
87	0.00682001868400113\\
88	0.00681998543087931\\
89	0.00681995162136805\\
90	0.00681991724626948\\
91	0.00681988229623762\\
92	0.00681984676177617\\
93	0.00681981063323628\\
94	0.00681977390081422\\
95	0.00681973655454912\\
96	0.0068196985843205\\
97	0.00681965997984607\\
98	0.00681962073067918\\
99	0.00681958082620655\\
100	0.0068195402556457\\
101	0.00681949900804251\\
102	0.00681945707226868\\
103	0.00681941443701926\\
104	0.00681937109081009\\
105	0.0068193270219751\\
106	0.00681928221866379\\
107	0.00681923666883861\\
108	0.00681919036027223\\
109	0.0068191432805449\\
110	0.00681909541704164\\
111	0.00681904675694962\\
112	0.00681899728725532\\
113	0.00681894699474166\\
114	0.00681889586598545\\
115	0.00681884388735419\\
116	0.00681879104500348\\
117	0.00681873732487395\\
118	0.00681868271268856\\
119	0.00681862719394947\\
120	0.00681857075393517\\
121	0.00681851337769744\\
122	0.00681845505005851\\
123	0.0068183957556079\\
124	0.00681833547869944\\
125	0.0068182742034482\\
126	0.00681821191372749\\
127	0.00681814859316567\\
128	0.00681808422514313\\
129	0.00681801879278911\\
130	0.00681795227897859\\
131	0.00681788466632918\\
132	0.00681781593719792\\
133	0.00681774607367807\\
134	0.00681767505759591\\
135	0.00681760287050757\\
136	0.00681752949369577\\
137	0.0068174549081666\\
138	0.00681737909464619\\
139	0.00681730203357736\\
140	0.00681722370511652\\
141	0.00681714408913013\\
142	0.00681706316519127\\
143	0.00681698091257642\\
144	0.00681689731026185\\
145	0.00681681233692014\\
146	0.00681672597091659\\
147	0.00681663819030566\\
148	0.00681654897282697\\
149	0.00681645829590186\\
150	0.00681636613662908\\
151	0.00681627247178097\\
152	0.00681617727779895\\
153	0.0068160805307893\\
154	0.00681598220651826\\
155	0.00681588228040731\\
156	0.00681578072752773\\
157	0.00681567752259502\\
158	0.00681557263996284\\
159	0.00681546605361657\\
160	0.00681535773716594\\
161	0.00681524766383737\\
162	0.00681513580646522\\
163	0.00681502213748237\\
164	0.00681490662890945\\
165	0.00681478925234313\\
166	0.00681466997894275\\
167	0.00681454877941545\\
168	0.00681442562399916\\
169	0.0068143004824438\\
170	0.00681417332398954\\
171	0.00681404411734251\\
172	0.00681391283064702\\
173	0.00681377943145406\\
174	0.00681364388668554\\
175	0.00681350616259372\\
176	0.00681336622471505\\
177	0.00681322403781779\\
178	0.00681307956584296\\
179	0.00681293277183706\\
180	0.00681278361787651\\
181	0.00681263206498246\\
182	0.00681247807302526\\
183	0.0068123216006177\\
184	0.00681216260499597\\
185	0.00681200104188856\\
186	0.00681183686537187\\
187	0.00681167002771319\\
188	0.00681150047920165\\
189	0.00681132816796832\\
190	0.00681115303979744\\
191	0.00681097503793067\\
192	0.00681079410286415\\
193	0.00681061017213128\\
194	0.0068104231800471\\
195	0.00681023305736157\\
196	0.00681003973076177\\
197	0.0068098431224084\\
198	0.00680964315129954\\
199	0.00680943974594813\\
200	0.00680923284747863\\
201	0.00680902239660873\\
202	0.00680880833307165\\
203	0.00680859059560025\\
204	0.00680836912191124\\
205	0.00680814384868866\\
206	0.00680791471156739\\
207	0.00680768164511646\\
208	0.00680744458282179\\
209	0.00680720345706904\\
210	0.00680695819912599\\
211	0.00680670873912447\\
212	0.00680645500604247\\
213	0.0068061969276855\\
214	0.00680593443066793\\
215	0.0068056674403939\\
216	0.0068053958810381\\
217	0.00680511967552597\\
218	0.00680483874551379\\
219	0.00680455301136853\\
220	0.00680426239214711\\
221	0.00680396680557559\\
222	0.00680366616802786\\
223	0.0068033603945041\\
224	0.00680304939860875\\
225	0.00680273309252847\\
226	0.00680241138700929\\
227	0.00680208419133379\\
228	0.00680175141329776\\
229	0.00680141295918634\\
230	0.00680106873375031\\
231	0.00680071864018141\\
232	0.0068003625800876\\
233	0.00680000045346802\\
234	0.0067996321586873\\
235	0.00679925759244981\\
236	0.00679887664977322\\
237	0.00679848922396195\\
238	0.00679809520658008\\
239	0.006797694487424\\
240	0.00679728695449443\\
241	0.00679687249396846\\
242	0.00679645099017089\\
243	0.00679602232554523\\
244	0.00679558638062453\\
245	0.00679514303400161\\
246	0.006794692162299\\
247	0.00679423364013861\\
248	0.00679376734011086\\
249	0.00679329313274358\\
250	0.00679281088647048\\
251	0.00679232046759924\\
252	0.00679182174027928\\
253	0.00679131456646923\\
254	0.00679079880590386\\
255	0.00679027431606076\\
256	0.00678974095212658\\
257	0.00678919856696295\\
258	0.00678864701107205\\
259	0.00678808613256156\\
260	0.00678751577710937\\
261	0.00678693578792792\\
262	0.00678634600572787\\
263	0.00678574626868148\\
264	0.0067851364123854\\
265	0.00678451626982306\\
266	0.00678388567132639\\
267	0.00678324444453717\\
268	0.00678259241436765\\
269	0.00678192940296065\\
270	0.00678125522964914\\
271	0.00678056971091495\\
272	0.00677987266034734\\
273	0.00677916388860037\\
274	0.00677844320335\\
275	0.00677771040925065\\
276	0.00677696530789062\\
277	0.00677620769774744\\
278	0.00677543737414224\\
279	0.00677465412919359\\
280	0.00677385775177052\\
281	0.00677304802744507\\
282	0.00677222473844398\\
283	0.00677138766359968\\
284	0.0067705365783005\\
285	0.00676967125444025\\
286	0.00676879146036689\\
287	0.00676789696083024\\
288	0.00676698751692908\\
289	0.00676606288605736\\
290	0.00676512282184922\\
291	0.00676416707412344\\
292	0.00676319538882644\\
293	0.00676220750797481\\
294	0.00676120316959627\\
295	0.00676018210766969\\
296	0.00675914405206401\\
297	0.00675808872847585\\
298	0.00675701585836574\\
299	0.00675592515889312\\
300	0.00675481634284983\\
301	0.00675368911859217\\
302	0.00675254318997126\\
303	0.00675137825626169\\
304	0.00675019401208859\\
305	0.00674899014735249\\
306	0.00674776634715239\\
307	0.00674652229170682\\
308	0.00674525765627203\\
309	0.00674397211105854\\
310	0.00674266532114452\\
311	0.00674133694638666\\
312	0.00673998664132796\\
313	0.00673861405510251\\
314	0.00673721883133672\\
315	0.00673580060804694\\
316	0.00673435901753325\\
317	0.00673289368626896\\
318	0.00673140423478562\\
319	0.0067298902775532\\
320	0.0067283514228549\\
321	0.00672678727265644\\
322	0.006725197422469\\
323	0.00672358146120589\\
324	0.00672193897103177\\
325	0.00672026952720406\\
326	0.00671857269790636\\
327	0.00671684804407221\\
328	0.00671509511919944\\
329	0.00671331346915329\\
330	0.00671150263195827\\
331	0.00670966213757684\\
332	0.00670779150767453\\
333	0.00670589025536974\\
334	0.00670395788496691\\
335	0.00670199389167166\\
336	0.00669999776128625\\
337	0.00669796896988296\\
338	0.00669590698345397\\
339	0.00669381125753483\\
340	0.00669168123679933\\
341	0.00668951635462255\\
342	0.00668731603260905\\
343	0.00668507968008221\\
344	0.0066828066935313\\
345	0.00668049645601082\\
346	0.0066781483364877\\
347	0.00667576168913012\\
348	0.00667333585253143\\
349	0.00667087014886187\\
350	0.00666836388293925\\
351	0.00666581634120933\\
352	0.00666322679062493\\
353	0.00666059447741077\\
354	0.0066579186257001\\
355	0.00665519843602668\\
356	0.00665243308365263\\
357	0.00664962171671082\\
358	0.00664676345413608\\
359	0.00664385738335635\\
360	0.00664090255770907\\
361	0.00663789799354379\\
362	0.006634842666964\\
363	0.00663173551015453\\
364	0.00662857540723024\\
365	0.00662536118953125\\
366	0.00662209163027598\\
367	0.00661876543846735\\
368	0.00661538125192753\\
369	0.00661193762931299\\
370	0.00660843304093303\\
371	0.00660486585816018\\
372	0.00660123434117727\\
373	0.00659753662475583\\
374	0.00659377070169413\\
375	0.00658993440346675\\
376	0.00658602537753923\\
377	0.0065820410606818\\
378	0.00657797864746696\\
379	0.00657383505295076\\
380	0.00656960686830603\\
381	0.00656529030789022\\
382	0.00656088114587788\\
383	0.00655637464017199\\
384	0.006551765440869\\
385	0.00654704748027803\\
386	0.00654221384207036\\
387	0.00653725661091762\\
388	0.00653216671896052\\
389	0.0065269338587564\\
390	0.00652154671652445\\
391	0.00651599095147768\\
392	0.00651024932391503\\
393	0.00650430118488542\\
394	0.00649812145499418\\
395	0.00649167893127105\\
396	0.00648493299469873\\
397	0.00647783680310319\\
398	0.00647033688183818\\
399	0.00646236779727463\\
400	0.00645384908975378\\
401	0.00644468139013384\\
402	0.00643474159157995\\
403	0.00642387719790623\\
404	0.00641190104780991\\
405	0.00639903657200411\\
406	0.00638596078425052\\
407	0.00637267027765513\\
408	0.00635916159825583\\
409	0.00634543124535949\\
410	0.00633147567191987\\
411	0.0063172912849489\\
412	0.00630287444596532\\
413	0.00628822147152613\\
414	0.00627332863398959\\
415	0.0062581921629295\\
416	0.00624280824829613\\
417	0.00622717304810649\\
418	0.00621128270763228\\
419	0.0061951334073186\\
420	0.00617872130624063\\
421	0.00616204253850911\\
422	0.00614509322036607\\
423	0.00612786945910534\\
424	0.0061103673640725\\
425	0.00609258305964295\\
426	0.00607451269871913\\
427	0.00605615247067725\\
428	0.00603749858282038\\
429	0.00601854731459214\\
430	0.00599929511417933\\
431	0.00597973868856861\\
432	0.00595987516664406\\
433	0.00593970243760798\\
434	0.00591921918218595\\
435	0.00589842481180852\\
436	0.0058773196493913\\
437	0.00585590515351625\\
438	0.00583418419697106\\
439	0.00581216141364589\\
440	0.00578984363321967\\
441	0.00576724043148619\\
442	0.00574436482077435\\
443	0.00572123411658393\\
444	0.00569787103223775\\
445	0.00567430506515952\\
446	0.00565057425709745\\
447	0.00562672743507575\\
448	0.00560282707190043\\
449	0.00557895294706662\\
450	0.00555520684395962\\
451	0.00553171859080939\\
452	0.00550865384334736\\
453	0.00548622411073188\\
454	0.00546469960050526\\
455	0.00544442531408781\\
456	0.0054258396714864\\
457	0.00540948983742446\\
458	0.00539323260504457\\
459	0.00537676311249133\\
460	0.00536008469878835\\
461	0.00534320198965536\\
462	0.00532612059192507\\
463	0.00530884720770051\\
464	0.00529138975884361\\
465	0.00527375752160495\\
466	0.00525596127074597\\
467	0.00523801343165346\\
468	0.00521992823775862\\
469	0.00520172188888132\\
470	0.00518341270376497\\
471	0.00516502125709558\\
472	0.0051465704860549\\
473	0.00512808576765448\\
474	0.0051095948999598\\
475	0.00509112794187953\\
476	0.00507271685943956\\
477	0.00505439488960137\\
478	0.00503619549586247\\
479	0.00501815076360548\\
480	0.0050002889997187\\
481	0.00498263121570112\\
482	0.00496518604990143\\
483	0.0049479424870087\\
484	0.00493085946334262\\
485	0.00491385109885935\\
486	0.00489676634709877\\
487	0.00487941046645583\\
488	0.00486177900139084\\
489	0.00484386633386189\\
490	0.00482566519217253\\
491	0.00480716726418363\\
492	0.00478836297873783\\
493	0.00476924124746257\\
494	0.00474978916359166\\
495	0.00472999165560429\\
496	0.00470983108666331\\
497	0.00468928677138484\\
498	0.0046683344028814\\
499	0.00464694545929378\\
500	0.00462508734092348\\
501	0.00460272350259559\\
502	0.00457982238022318\\
503	0.00455638069580096\\
504	0.00453233403861286\\
505	0.00450760479277492\\
506	0.00448213612080047\\
507	0.00445587381878527\\
508	0.00442877360691682\\
509	0.00440078970698855\\
510	0.00437187371198864\\
511	0.00434197524437358\\
512	0.00431104092335962\\
513	0.00427901432034292\\
514	0.00424583630966117\\
515	0.0042114290035927\\
516	0.00417567760871252\\
517	0.00413850706469992\\
518	0.00409983938562309\\
519	0.00405959407839367\\
520	0.00401768872169933\\
521	0.0039740397359373\\
522	0.00392856330490904\\
523	0.00388117643313738\\
524	0.00383179797078545\\
525	0.00378034940225419\\
526	0.0037267559301864\\
527	0.0036709496971085\\
528	0.00361287876217814\\
529	0.00355251646179829\\
530	0.00348985500560026\\
531	0.00342491096846352\\
532	0.00335773241447154\\
533	0.00328840965250917\\
534	0.00321708462727485\\
535	0.00314395991947484\\
536	0.00306929731610841\\
537	0.00299340958820004\\
538	0.0029167674290013\\
539	0.00284013624520599\\
540	0.00276555084315231\\
541	0.0026960778195408\\
542	0.00263212286836089\\
543	0.00257378503688891\\
544	0.00252085379215805\\
545	0.00247009708931051\\
546	0.00242037269702207\\
547	0.00237125755908936\\
548	0.00232224243204798\\
549	0.00227316453779023\\
550	0.00222398371594873\\
551	0.00217460623235058\\
552	0.00212495896211347\\
553	0.00207497488170052\\
554	0.00202421243490735\\
555	0.00197262615657274\\
556	0.0019201375163468\\
557	0.00186664188289284\\
558	0.00181318150960817\\
559	0.00176071773665684\\
560	0.00170951692853472\\
561	0.00165928064744219\\
562	0.00160875603182242\\
563	0.00155792315883206\\
564	0.00150671918854937\\
565	0.00145517384907741\\
566	0.00140334093225183\\
567	0.00135125632661799\\
568	0.00130024976398429\\
569	0.00125090910566476\\
570	0.00120241934549694\\
571	0.00115396094926996\\
572	0.00110548107662132\\
573	0.00105702283547423\\
574	0.00100863570447616\\
575	0.000960371942245695\\
576	0.000912495014597589\\
577	0.000865361483000015\\
578	0.000818442575994003\\
579	0.000771658262692563\\
580	0.000725058416501592\\
581	0.000678697369762489\\
582	0.000632630668187501\\
583	0.000586913975324799\\
584	0.000541601889067355\\
585	0.000496746436058626\\
586	0.000452395151479575\\
587	0.000408588670822602\\
588	0.000365357795712724\\
589	0.000322720123375259\\
590	0.000280676711614414\\
591	0.000239312817487402\\
592	0.000198788644961838\\
593	0.000159293651685583\\
594	0.000121079888727803\\
595	8.45520570083905e-05\\
596	5.05092148680371e-05\\
597	2.07908715710836e-05\\
598	0\\
599	0\\
600	0\\
};
\addplot [color=mycolor9,solid,forget plot]
  table[row sep=crcr]{%
1	0.00701212116330883\\
2	0.0070121158585532\\
3	0.00701211046433695\\
4	0.00701210497915749\\
5	0.00701209940148715\\
6	0.0070120937297728\\
7	0.00701208796243541\\
8	0.00701208209786965\\
9	0.00701207613444345\\
10	0.00701207007049751\\
11	0.00701206390434495\\
12	0.00701205763427078\\
13	0.00701205125853145\\
14	0.00701204477535449\\
15	0.00701203818293786\\
16	0.00701203147944953\\
17	0.00701202466302706\\
18	0.00701201773177704\\
19	0.00701201068377461\\
20	0.00701200351706295\\
21	0.00701199622965271\\
22	0.00701198881952148\\
23	0.00701198128461338\\
24	0.00701197362283837\\
25	0.00701196583207182\\
26	0.0070119579101537\\
27	0.00701194985488829\\
28	0.00701194166404353\\
29	0.00701193333535033\\
30	0.00701192486650201\\
31	0.00701191625515377\\
32	0.007011907498922\\
33	0.00701189859538364\\
34	0.00701188954207558\\
35	0.00701188033649402\\
36	0.00701187097609382\\
37	0.00701186145828771\\
38	0.00701185178044581\\
39	0.00701184193989485\\
40	0.00701183193391744\\
41	0.00701182175975135\\
42	0.00701181141458891\\
43	0.00701180089557619\\
44	0.00701179019981224\\
45	0.00701177932434842\\
46	0.00701176826618755\\
47	0.00701175702228318\\
48	0.00701174558953885\\
49	0.00701173396480721\\
50	0.00701172214488927\\
51	0.00701171012653347\\
52	0.0070116979064351\\
53	0.00701168548123514\\
54	0.00701167284751962\\
55	0.00701166000181868\\
56	0.00701164694060568\\
57	0.00701163366029628\\
58	0.00701162015724765\\
59	0.00701160642775744\\
60	0.00701159246806282\\
61	0.00701157827433967\\
62	0.00701156384270145\\
63	0.00701154916919833\\
64	0.00701153424981618\\
65	0.00701151908047565\\
66	0.00701150365703097\\
67	0.00701148797526907\\
68	0.00701147203090845\\
69	0.00701145581959819\\
70	0.00701143933691689\\
71	0.0070114225783714\\
72	0.00701140553939595\\
73	0.00701138821535091\\
74	0.00701137060152166\\
75	0.0070113526931174\\
76	0.00701133448527019\\
77	0.00701131597303348\\
78	0.00701129715138104\\
79	0.0070112780152059\\
80	0.00701125855931881\\
81	0.00701123877844722\\
82	0.00701121866723396\\
83	0.00701119822023592\\
84	0.00701117743192289\\
85	0.00701115629667599\\
86	0.00701113480878663\\
87	0.00701111296245492\\
88	0.00701109075178841\\
89	0.00701106817080076\\
90	0.00701104521341026\\
91	0.00701102187343835\\
92	0.00701099814460837\\
93	0.00701097402054388\\
94	0.00701094949476734\\
95	0.00701092456069852\\
96	0.00701089921165302\\
97	0.00701087344084075\\
98	0.00701084724136438\\
99	0.00701082060621768\\
100	0.00701079352828397\\
101	0.00701076600033452\\
102	0.00701073801502681\\
103	0.00701070956490305\\
104	0.00701068064238827\\
105	0.00701065123978876\\
106	0.00701062134929025\\
107	0.00701059096295618\\
108	0.00701056007272589\\
109	0.0070105286704128\\
110	0.0070104967477027\\
111	0.00701046429615158\\
112	0.00701043130718397\\
113	0.00701039777209098\\
114	0.00701036368202821\\
115	0.00701032902801386\\
116	0.00701029380092672\\
117	0.00701025799150396\\
118	0.0070102215903391\\
119	0.00701018458787989\\
120	0.00701014697442606\\
121	0.0070101087401271\\
122	0.00701006987497999\\
123	0.00701003036882673\\
124	0.00700999021135221\\
125	0.0070099493920815\\
126	0.00700990790037743\\
127	0.00700986572543801\\
128	0.00700982285629382\\
129	0.00700977928180526\\
130	0.00700973499065963\\
131	0.00700968997136834\\
132	0.00700964421226397\\
133	0.00700959770149705\\
134	0.00700955042703295\\
135	0.00700950237664855\\
136	0.00700945353792868\\
137	0.00700940389826273\\
138	0.00700935344484077\\
139	0.00700930216464976\\
140	0.00700925004446939\\
141	0.00700919707086783\\
142	0.00700914323019746\\
143	0.00700908850858996\\
144	0.00700903289195151\\
145	0.00700897636595761\\
146	0.00700891891604771\\
147	0.00700886052741928\\
148	0.00700880118502204\\
149	0.00700874087355126\\
150	0.00700867957744124\\
151	0.00700861728085784\\
152	0.0070085539676912\\
153	0.00700848962154737\\
154	0.00700842422573996\\
155	0.00700835776328083\\
156	0.0070082902168706\\
157	0.00700822156888834\\
158	0.00700815180138065\\
159	0.00700808089604997\\
160	0.00700800883424242\\
161	0.00700793559693475\\
162	0.00700786116472039\\
163	0.00700778551779494\\
164	0.00700770863594062\\
165	0.00700763049850998\\
166	0.00700755108440853\\
167	0.00700747037207685\\
168	0.00700738833947156\\
169	0.00700730496404541\\
170	0.00700722022272688\\
171	0.00700713409189867\\
172	0.0070070465473757\\
173	0.00700695756438277\\
174	0.0070068671175316\\
175	0.00700677518079791\\
176	0.0070066817274985\\
177	0.00700658673026886\\
178	0.00700649016104153\\
179	0.00700639199102593\\
180	0.00700629219068988\\
181	0.00700619072974375\\
182	0.00700608757712771\\
183	0.00700598270100331\\
184	0.00700587606875001\\
185	0.00700576764696783\\
186	0.00700565740148734\\
187	0.00700554529738814\\
188	0.007005431299027\\
189	0.00700531537007689\\
190	0.00700519747357758\\
191	0.00700507757199851\\
192	0.00700495562731353\\
193	0.00700483160108602\\
194	0.00700470545456245\\
195	0.00700457714877309\\
196	0.00700444664464774\\
197	0.0070043139031749\\
198	0.00700417888562068\\
199	0.00700404155337168\\
200	0.00700390186720293\\
201	0.00700375978723681\\
202	0.00700361527293231\\
203	0.00700346828307445\\
204	0.00700331877576297\\
205	0.00700316670840151\\
206	0.00700301203768601\\
207	0.00700285471959327\\
208	0.00700269470936933\\
209	0.0070025319615174\\
210	0.0070023664297858\\
211	0.00700219806715577\\
212	0.00700202682582877\\
213	0.00700185265721398\\
214	0.00700167551191515\\
215	0.0070014953397177\\
216	0.00700131208957524\\
217	0.00700112570959605\\
218	0.0070009361470293\\
219	0.00700074334825093\\
220	0.00700054725874964\\
221	0.00700034782311216\\
222	0.00700014498500878\\
223	0.0069999386871782\\
224	0.00699972887141246\\
225	0.00699951547854149\\
226	0.00699929844841746\\
227	0.00699907771989871\\
228	0.00699885323083372\\
229	0.00699862491804464\\
230	0.0069983927173104\\
231	0.0069981565633499\\
232	0.00699791638980476\\
233	0.00699767212922168\\
234	0.0069974237130348\\
235	0.00699717107154743\\
236	0.00699691413391379\\
237	0.00699665282812034\\
238	0.0069963870809668\\
239	0.00699611681804689\\
240	0.00699584196372884\\
241	0.0069955624411354\\
242	0.00699527817212372\\
243	0.0069949890772649\\
244	0.00699469507582308\\
245	0.00699439608573423\\
246	0.00699409202358484\\
247	0.00699378280458979\\
248	0.00699346834257033\\
249	0.00699314854993146\\
250	0.00699282333763886\\
251	0.00699249261519567\\
252	0.00699215629061865\\
253	0.006991814270414\\
254	0.00699146645955284\\
255	0.00699111276144603\\
256	0.00699075307791886\\
257	0.00699038730918494\\
258	0.00699001535381969\\
259	0.00698963710873354\\
260	0.00698925246914425\\
261	0.00698886132854892\\
262	0.00698846357869551\\
263	0.00698805910955341\\
264	0.0069876478092838\\
265	0.0069872295642091\\
266	0.00698680425878206\\
267	0.00698637177555383\\
268	0.00698593199514149\\
269	0.00698548479619506\\
270	0.00698503005536338\\
271	0.00698456764725966\\
272	0.00698409744442576\\
273	0.00698361931729606\\
274	0.00698313313416024\\
275	0.00698263876112519\\
276	0.00698213606207617\\
277	0.00698162489863678\\
278	0.00698110513012813\\
279	0.00698057661352673\\
280	0.00698003920342169\\
281	0.00697949275197032\\
282	0.00697893710885287\\
283	0.00697837212122592\\
284	0.00697779763367445\\
285	0.00697721348816269\\
286	0.00697661952398332\\
287	0.00697601557770547\\
288	0.00697540148312096\\
289	0.00697477707118888\\
290	0.00697414216997879\\
291	0.00697349660461161\\
292	0.00697284019719909\\
293	0.00697217276678098\\
294	0.00697149412926034\\
295	0.0069708040973366\\
296	0.00697010248043605\\
297	0.00696938908464039\\
298	0.00696866371261231\\
299	0.00696792616351857\\
300	0.00696717623295016\\
301	0.00696641371283952\\
302	0.00696563839137457\\
303	0.00696485005290942\\
304	0.00696404847787139\\
305	0.0069632334426646\\
306	0.00696240471956947\\
307	0.00696156207663779\\
308	0.006960705277584\\
309	0.00695983408167112\\
310	0.00695894824359223\\
311	0.00695804751334643\\
312	0.00695713163610941\\
313	0.00695620035209805\\
314	0.00695525339642884\\
315	0.00695429049896952\\
316	0.00695331138418405\\
317	0.00695231577096971\\
318	0.00695130337248641\\
319	0.00695027389597768\\
320	0.00694922704258255\\
321	0.00694816250713802\\
322	0.00694707997797123\\
323	0.00694597913668085\\
324	0.00694485965790696\\
325	0.00694372120908883\\
326	0.00694256345020909\\
327	0.00694138603352449\\
328	0.00694018860328137\\
329	0.0069389707954152\\
330	0.0069377322372331\\
331	0.00693647254707809\\
332	0.00693519133397384\\
333	0.0069338881972483\\
334	0.00693256272613508\\
335	0.00693121449935059\\
336	0.00692984308464523\\
337	0.00692844803832713\\
338	0.00692702890475551\\
339	0.00692558521580226\\
340	0.00692411649027866\\
341	0.00692262223332462\\
342	0.00692110193575778\\
343	0.00691955507337899\\
344	0.00691798110623043\\
345	0.00691637947780302\\
346	0.00691474961418832\\
347	0.00691309092317046\\
348	0.00691140279325312\\
349	0.00690968459261531\\
350	0.00690793566799061\\
351	0.00690615534346191\\
352	0.00690434291916476\\
353	0.00690249766989062\\
354	0.00690061884358072\\
355	0.00689870565969982\\
356	0.00689675730747893\\
357	0.00689477294401382\\
358	0.00689275169220508\\
359	0.00689069263852404\\
360	0.00688859483058697\\
361	0.00688645727451776\\
362	0.0068842789320776\\
363	0.00688205871753681\\
364	0.00687979549426217\\
365	0.00687748807098947\\
366	0.00687513519774751\\
367	0.00687273556139631\\
368	0.00687028778073843\\
369	0.006867790401157\\
370	0.00686524188873027\\
371	0.00686264062376682\\
372	0.00685998489370048\\
373	0.00685727288527866\\
374	0.00685450267597319\\
375	0.00685167222453705\\
376	0.00684877936062798\\
377	0.00684582177341692\\
378	0.0068427969991017\\
379	0.00683970240724956\\
380	0.0068365351859054\\
381	0.0068332923254236\\
382	0.00682997060101846\\
383	0.0068265665540891\\
384	0.00682307647247982\\
385	0.00681949637001059\\
386	0.00681582196590169\\
387	0.00681204866509814\\
388	0.00680817154044249\\
389	0.00680418531386084\\
390	0.00680008431010625\\
391	0.00679586243472893\\
392	0.00679151321162811\\
393	0.00678702981014269\\
394	0.00678240509034287\\
395	0.00677763170285273\\
396	0.00677270236258949\\
397	0.00676761019656155\\
398	0.00676234896814914\\
399	0.00675691344267549\\
400	0.00675130005788015\\
401	0.00674550787862871\\
402	0.00673953995275049\\
403	0.00673340522722306\\
404	0.00672712117799294\\
405	0.00672071625804746\\
406	0.00671421349619451\\
407	0.00670761146295683\\
408	0.00670090870046873\\
409	0.00669410372057442\\
410	0.00668719500258211\\
411	0.00668018099058624\\
412	0.00667306009025135\\
413	0.00666583066493078\\
414	0.0066584910309769\\
415	0.00665103945208354\\
416	0.00664347413248083\\
417	0.00663579320869548\\
418	0.00662799473913034\\
419	0.00662007668933002\\
420	0.00661203691626788\\
421	0.00660387315302666\\
422	0.00659558299162864\\
423	0.00658716386409457\\
424	0.00657861302397764\\
425	0.00656992753692295\\
426	0.00656110430995516\\
427	0.00655214026025386\\
428	0.00654303296443749\\
429	0.00653377909572568\\
430	0.00652437365427828\\
431	0.00651481037914996\\
432	0.00650508044441175\\
433	0.00649516897538018\\
434	0.00648506248822431\\
435	0.00647475124985164\\
436	0.0064642242365699\\
437	0.00645346889958397\\
438	0.00644247087869849\\
439	0.00643121364884752\\
440	0.00641967806504588\\
441	0.00640784181270222\\
442	0.00639567874715611\\
443	0.00638315807704225\\
444	0.00637024335089058\\
445	0.00635689119486041\\
446	0.00634304973467485\\
447	0.00632865661554167\\
448	0.00631363650875949\\
449	0.00629789796108431\\
450	0.00628132940073313\\
451	0.00626379406023011\\
452	0.00624512351136802\\
453	0.00622510944104158\\
454	0.00620349327436457\\
455	0.00617995345149095\\
456	0.00615409125076077\\
457	0.00612542075730385\\
458	0.00609601646693627\\
459	0.00606616241725746\\
460	0.00603584092572867\\
461	0.00600502899017431\\
462	0.00597372174869766\\
463	0.00594191505454013\\
464	0.00590960567751194\\
465	0.00587679155557354\\
466	0.0058434721069465\\
467	0.00580964861900482\\
468	0.00577532473337828\\
469	0.00574050705142841\\
470	0.00570520588925315\\
471	0.00566943621445993\\
472	0.0056332188542572\\
473	0.0055965820362056\\
474	0.00555956331100378\\
475	0.00552221199590847\\
476	0.00548459229280957\\
477	0.00544678728978496\\
478	0.00540890413927234\\
479	0.00537108040097315\\
480	0.00533349233870938\\
481	0.00529636564331368\\
482	0.00525998922738064\\
483	0.00522473313857186\\
484	0.00519107140812409\\
485	0.00515961030441914\\
486	0.00513111942658377\\
487	0.00510583050143825\\
488	0.00508025809679198\\
489	0.00505441758935908\\
490	0.00502832694705126\\
491	0.00500200695922378\\
492	0.00497548144923012\\
493	0.00494877744918957\\
494	0.0049219253078496\\
495	0.00489495868990997\\
496	0.00486791440817347\\
497	0.0048408320076661\\
498	0.00481375299114852\\
499	0.00478671953457748\\
500	0.00475977245909883\\
501	0.00473294813251668\\
502	0.00470627391279249\\
503	0.00467976119434337\\
504	0.00465339454787544\\
505	0.00462711890340711\\
506	0.00460081918406089\\
507	0.00457429104784732\\
508	0.00454720443578428\\
509	0.00451949920381706\\
510	0.00449112857439873\\
511	0.00446202316154678\\
512	0.00443214253187289\\
513	0.00440144139335974\\
514	0.00436986892632396\\
515	0.00433738351339268\\
516	0.00430396765942999\\
517	0.00426955412733077\\
518	0.00423406767181877\\
519	0.00419742583052031\\
520	0.00415953875905356\\
521	0.00412030917420201\\
522	0.00407963455826062\\
523	0.00403741149826289\\
524	0.00399354342326266\\
525	0.00394794794373237\\
526	0.00390054155404816\\
527	0.00385122407791419\\
528	0.0037998565790712\\
529	0.00374634929783748\\
530	0.00369061540493826\\
531	0.00363257431252558\\
532	0.00357216559047094\\
533	0.00350934272525963\\
534	0.00344407564145591\\
535	0.00337635592798696\\
536	0.00330620316315285\\
537	0.00323367434338437\\
538	0.00315887169747382\\
539	0.00308195591005351\\
540	0.00300314669175381\\
541	0.00292271428666918\\
542	0.00284102108688311\\
543	0.00275866609982855\\
544	0.00267662723432266\\
545	0.00259850455214081\\
546	0.00252561474725806\\
547	0.00245828060502498\\
548	0.00239657932141828\\
549	0.00233997269311372\\
550	0.00228472809978635\\
551	0.00223052266454992\\
552	0.00217693151171248\\
553	0.00212346363709513\\
554	0.00207004191030077\\
555	0.00201655731486821\\
556	0.00196290412541734\\
557	0.00190901167585809\\
558	0.00185475671516\\
559	0.00179972684135543\\
560	0.00174381965237555\\
561	0.00168751872449357\\
562	0.00163222103135658\\
563	0.00157815070953452\\
564	0.00152562817316902\\
565	0.00147318530183207\\
566	0.00142057936312707\\
567	0.00136778956338732\\
568	0.00131479124306664\\
569	0.0012616240709895\\
570	0.00120923369631625\\
571	0.00115852724686328\\
572	0.00110937816652107\\
573	0.00106041064718068\\
574	0.00101157345479646\\
575	0.000962869504412621\\
576	0.000914347675446341\\
577	0.000866058972727197\\
578	0.000818515558008548\\
579	0.000771676801487267\\
580	0.000725067167815385\\
581	0.000678702038941237\\
582	0.000632633102614749\\
583	0.000586915166466385\\
584	0.000541602424099719\\
585	0.000496746650654584\\
586	0.000452395225472513\\
587	0.000408588691486123\\
588	0.000365357799889246\\
589	0.000322720123837111\\
590	0.000280676711614414\\
591	0.000239312817487403\\
592	0.000198788644961841\\
593	0.000159293651685586\\
594	0.000121079888727805\\
595	8.45520570083913e-05\\
596	5.05092148680373e-05\\
597	2.07908715710836e-05\\
598	0\\
599	0\\
600	0\\
};
\addplot [color=blue!50!mycolor7,solid,forget plot]
  table[row sep=crcr]{%
1	0.0074068299566467\\
2	0.00740682441785876\\
3	0.00740681878582221\\
4	0.00740681305897183\\
5	0.00740680723571628\\
6	0.00740680131443758\\
7	0.00740679529349066\\
8	0.00740678917120309\\
9	0.00740678294587443\\
10	0.00740677661577581\\
11	0.00740677017914957\\
12	0.00740676363420853\\
13	0.0074067569791359\\
14	0.00740675021208434\\
15	0.00740674333117578\\
16	0.00740673633450079\\
17	0.00740672922011808\\
18	0.00740672198605392\\
19	0.00740671463030165\\
20	0.00740670715082115\\
21	0.00740669954553828\\
22	0.00740669181234433\\
23	0.00740668394909534\\
24	0.00740667595361162\\
25	0.00740666782367715\\
26	0.00740665955703898\\
27	0.00740665115140663\\
28	0.00740664260445139\\
29	0.00740663391380575\\
30	0.00740662507706283\\
31	0.00740661609177554\\
32	0.00740660695545612\\
33	0.00740659766557532\\
34	0.00740658821956183\\
35	0.00740657861480151\\
36	0.00740656884863661\\
37	0.00740655891836529\\
38	0.00740654882124067\\
39	0.00740653855447019\\
40	0.0074065281152148\\
41	0.0074065175005883\\
42	0.00740650670765637\\
43	0.007406495733436\\
44	0.00740648457489452\\
45	0.00740647322894892\\
46	0.00740646169246489\\
47	0.00740644996225608\\
48	0.00740643803508317\\
49	0.00740642590765303\\
50	0.00740641357661781\\
51	0.00740640103857418\\
52	0.00740638829006216\\
53	0.00740637532756442\\
54	0.00740636214750523\\
55	0.00740634874624953\\
56	0.00740633512010191\\
57	0.00740632126530572\\
58	0.00740630717804195\\
59	0.00740629285442831\\
60	0.00740627829051807\\
61	0.00740626348229912\\
62	0.00740624842569288\\
63	0.00740623311655316\\
64	0.00740621755066502\\
65	0.00740620172374375\\
66	0.00740618563143371\\
67	0.00740616926930701\\
68	0.00740615263286254\\
69	0.00740613571752465\\
70	0.00740611851864188\\
71	0.00740610103148595\\
72	0.00740608325125019\\
73	0.00740606517304846\\
74	0.00740604679191379\\
75	0.00740602810279702\\
76	0.00740600910056551\\
77	0.00740598978000174\\
78	0.00740597013580194\\
79	0.00740595016257458\\
80	0.00740592985483905\\
81	0.00740590920702419\\
82	0.00740588821346668\\
83	0.00740586686840964\\
84	0.00740584516600103\\
85	0.00740582310029207\\
86	0.00740580066523575\\
87	0.00740577785468505\\
88	0.00740575466239144\\
89	0.00740573108200305\\
90	0.0074057071070631\\
91	0.00740568273100809\\
92	0.00740565794716598\\
93	0.00740563274875449\\
94	0.00740560712887922\\
95	0.00740558108053179\\
96	0.00740555459658784\\
97	0.00740552766980521\\
98	0.00740550029282195\\
99	0.00740547245815421\\
100	0.00740544415819427\\
101	0.00740541538520844\\
102	0.00740538613133496\\
103	0.00740535638858165\\
104	0.00740532614882389\\
105	0.00740529540380227\\
106	0.00740526414512026\\
107	0.00740523236424184\\
108	0.00740520005248921\\
109	0.00740516720104009\\
110	0.00740513380092542\\
111	0.00740509984302671\\
112	0.00740506531807335\\
113	0.00740503021664007\\
114	0.00740499452914394\\
115	0.00740495824584187\\
116	0.00740492135682741\\
117	0.00740488385202798\\
118	0.00740484572120179\\
119	0.00740480695393475\\
120	0.00740476753963729\\
121	0.007404727467541\\
122	0.00740468672669541\\
123	0.00740464530596449\\
124	0.00740460319402304\\
125	0.00740456037935315\\
126	0.0074045168502405\\
127	0.00740447259477036\\
128	0.00740442760082377\\
129	0.00740438185607332\\
130	0.00740433534797922\\
131	0.00740428806378474\\
132	0.00740423999051178\\
133	0.00740419111495641\\
134	0.007404141423684\\
135	0.0074040909030245\\
136	0.00740403953906722\\
137	0.00740398731765573\\
138	0.00740393422438254\\
139	0.00740388024458343\\
140	0.00740382536333186\\
141	0.007403769565433\\
142	0.00740371283541765\\
143	0.00740365515753591\\
144	0.00740359651575079\\
145	0.00740353689373137\\
146	0.00740347627484593\\
147	0.00740341464215491\\
148	0.00740335197840341\\
149	0.00740328826601369\\
150	0.00740322348707738\\
151	0.00740315762334745\\
152	0.0074030906562299\\
153	0.00740302256677547\\
154	0.0074029533356708\\
155	0.00740288294322967\\
156	0.00740281136938397\\
157	0.00740273859367428\\
158	0.00740266459524067\\
159	0.00740258935281312\\
160	0.00740251284470192\\
161	0.00740243504878779\\
162	0.00740235594251239\\
163	0.00740227550286832\\
164	0.00740219370638955\\
165	0.00740211052914166\\
166	0.0074020259467126\\
167	0.00740193993420315\\
168	0.00740185246621825\\
169	0.00740176351685822\\
170	0.0074016730597107\\
171	0.00740158106784308\\
172	0.00740148751379556\\
173	0.00740139236957476\\
174	0.0074012956066485\\
175	0.0074011971959411\\
176	0.00740109710782996\\
177	0.00740099531214307\\
178	0.00740089177815762\\
179	0.00740078647459987\\
180	0.00740067936964604\\
181	0.00740057043092449\\
182	0.00740045962551893\\
183	0.00740034691997241\\
184	0.00740023228029238\\
185	0.00740011567195596\\
186	0.00739999705991545\\
187	0.0073998764086036\\
188	0.00739975368193796\\
189	0.00739962884332372\\
190	0.00739950185565448\\
191	0.00739937268130986\\
192	0.00739924128214912\\
193	0.00739910761950039\\
194	0.00739897165414423\\
195	0.0073988333462918\\
196	0.00739869265555768\\
197	0.00739854954092654\\
198	0.00739840396070977\\
199	0.00739825587250141\\
200	0.00739810523316437\\
201	0.00739795199881759\\
202	0.00739779612482297\\
203	0.0073976375657721\\
204	0.00739747627547275\\
205	0.00739731220693504\\
206	0.0073971453123574\\
207	0.00739697554311232\\
208	0.00739680284973175\\
209	0.00739662718189224\\
210	0.00739644848839992\\
211	0.007396266717175\\
212	0.00739608181523616\\
213	0.00739589372868453\\
214	0.0073957024026875\\
215	0.00739550778146202\\
216	0.00739530980825783\\
217	0.00739510842534015\\
218	0.00739490357397224\\
219	0.00739469519439751\\
220	0.00739448322582124\\
221	0.00739426760639218\\
222	0.00739404827318354\\
223	0.00739382516217373\\
224	0.00739359820822686\\
225	0.00739336734507254\\
226	0.00739313250528566\\
227	0.00739289362026561\\
228	0.00739265062021495\\
229	0.00739240343411795\\
230	0.00739215198971861\\
231	0.00739189621349811\\
232	0.007391636030652\\
233	0.00739137136506691\\
234	0.00739110213929665\\
235	0.00739082827453812\\
236	0.00739054969060642\\
237	0.00739026630590975\\
238	0.00738997803742359\\
239	0.00738968480066449\\
240	0.0073893865096632\\
241	0.00738908307693744\\
242	0.007388774413464\\
243	0.00738846042865016\\
244	0.00738814103030472\\
245	0.00738781612460839\\
246	0.00738748561608332\\
247	0.00738714940756228\\
248	0.00738680740015708\\
249	0.00738645949322613\\
250	0.00738610558434173\\
251	0.0073857455692561\\
252	0.00738537934186714\\
253	0.00738500679418307\\
254	0.00738462781628656\\
255	0.0073842422962979\\
256	0.0073838501203373\\
257	0.00738345117248648\\
258	0.00738304533474929\\
259	0.00738263248701129\\
260	0.00738221250699872\\
261	0.0073817852702362\\
262	0.00738135065000343\\
263	0.00738090851729121\\
264	0.00738045874075587\\
265	0.00738000118667315\\
266	0.0073795357188904\\
267	0.00737906219877818\\
268	0.00737858048518024\\
269	0.00737809043436237\\
270	0.00737759189996016\\
271	0.00737708473292503\\
272	0.00737656878146932\\
273	0.0073760438910098\\
274	0.00737550990410958\\
275	0.00737496666041871\\
276	0.00737441399661314\\
277	0.00737385174633213\\
278	0.00737327974011379\\
279	0.00737269780532916\\
280	0.00737210576611426\\
281	0.00737150344330047\\
282	0.00737089065434291\\
283	0.00737026721324675\\
284	0.00736963293049165\\
285	0.00736898761295384\\
286	0.00736833106382621\\
287	0.00736766308253593\\
288	0.00736698346465981\\
289	0.0073662920018371\\
290	0.00736558848167974\\
291	0.0073648726876801\\
292	0.00736414439911577\\
293	0.00736340339095163\\
294	0.00736264943373898\\
295	0.00736188229351134\\
296	0.00736110173167753\\
297	0.00736030750491094\\
298	0.00735949936503572\\
299	0.00735867705890934\\
300	0.00735784032830132\\
301	0.00735698890976823\\
302	0.00735612253452471\\
303	0.00735524092831032\\
304	0.00735434381125222\\
305	0.00735343089772323\\
306	0.00735250189619534\\
307	0.0073515565090885\\
308	0.00735059443261443\\
309	0.00734961535661524\\
310	0.00734861896439659\\
311	0.00734760493255559\\
312	0.00734657293080274\\
313	0.00734552262177798\\
314	0.00734445366086052\\
315	0.00734336569597255\\
316	0.00734225836737576\\
317	0.00734113130746157\\
318	0.00733998414053407\\
319	0.00733881648258547\\
320	0.00733762794106414\\
321	0.00733641811463464\\
322	0.00733518659292982\\
323	0.0073339329562943\\
324	0.00733265677551952\\
325	0.00733135761156964\\
326	0.00733003501529837\\
327	0.00732868852715621\\
328	0.00732731767688786\\
329	0.0073259219832197\\
330	0.00732450095353673\\
331	0.00732305408354902\\
332	0.00732158085694693\\
333	0.00732008074504546\\
334	0.00731855320641677\\
335	0.00731699768651116\\
336	0.00731541361726578\\
337	0.00731380041670108\\
338	0.00731215748850482\\
339	0.00731048422160312\\
340	0.00730877998971845\\
341	0.00730704415091459\\
342	0.00730527604712798\\
343	0.00730347500368556\\
344	0.00730164032880894\\
345	0.00729977131310471\\
346	0.00729786722904094\\
347	0.00729592733040972\\
348	0.00729395085177594\\
349	0.00729193700791217\\
350	0.00728988499322004\\
351	0.00728779398113814\\
352	0.00728566312353681\\
353	0.00728349155010023\\
354	0.00728127836769626\\
355	0.00727902265973467\\
356	0.00727672348551477\\
357	0.00727437987956289\\
358	0.00727199085096141\\
359	0.00726955538267042\\
360	0.00726707243084388\\
361	0.00726454092414224\\
362	0.007261959763044\\
363	0.00725932781915937\\
364	0.00725664393454936\\
365	0.00725390692105469\\
366	0.00725111555964022\\
367	0.00724826859976054\\
368	0.00724536475875515\\
369	0.00724240272128197\\
370	0.00723938113880114\\
371	0.00723629862912245\\
372	0.00723315377603356\\
373	0.0072299451290294\\
374	0.00722667120316761\\
375	0.00722333047908035\\
376	0.00721992140317944\\
377	0.00721644238809939\\
378	0.0072128918134324\\
379	0.00720926802682098\\
380	0.00720556934548708\\
381	0.0072017940582922\\
382	0.0071979404284423\\
383	0.00719400669697306\\
384	0.00718999108717504\\
385	0.00718589181014539\\
386	0.00718170707167325\\
387	0.00717743508066228\\
388	0.00717307405923942\\
389	0.00716862225468668\\
390	0.00716407795460443\\
391	0.00715943950628516\\
392	0.00715470533720593\\
393	0.00714987397793557\\
394	0.00714494408796095\\
395	0.00713991448424189\\
396	0.0071347841672599\\
397	0.00712955233752372\\
398	0.00712421840955929\\
399	0.00711878202108357\\
400	0.00711324302348959\\
401	0.00710760144058647\\
402	0.00710185737522207\\
403	0.00709601083121614\\
404	0.0070900613992831\\
405	0.00708400778027166\\
406	0.00707784782427909\\
407	0.00707157929726241\\
408	0.00706519989025344\\
409	0.00705870721664789\\
410	0.00705209880977829\\
411	0.00704537212084409\\
412	0.00703852451725753\\
413	0.00703155328144083\\
414	0.00702445561010178\\
415	0.00701722861410645\\
416	0.00700986931951129\\
417	0.00700237467173069\\
418	0.00699474154818311\\
419	0.00698696679845117\\
420	0.00697904714820445\\
421	0.00697097917369077\\
422	0.00696275928721133\\
423	0.00695438372097599\\
424	0.00694584850944636\\
425	0.0069371494704971\\
426	0.0069282821853252\\
427	0.00691924197185974\\
428	0.00691002381607948\\
429	0.00690062229619514\\
430	0.00689103157279853\\
431	0.0068812453785263\\
432	0.00687125700243555\\
433	0.00686105947035154\\
434	0.00685064572278754\\
435	0.00684000842741724\\
436	0.00682913979427269\\
437	0.00681803157463512\\
438	0.00680667509094791\\
439	0.00679506134536192\\
440	0.00678318167415435\\
441	0.00677102761914462\\
442	0.00675859057904313\\
443	0.00674586191458853\\
444	0.00673283310171506\\
445	0.00671949595028459\\
446	0.00670584291204887\\
447	0.00669186750977668\\
448	0.00667756493066131\\
449	0.00666293284232163\\
450	0.00664797251059849\\
451	0.00663269032774426\\
452	0.0066170999029803\\
453	0.00660122493712663\\
454	0.00658510323055887\\
455	0.00656879244459335\\
456	0.00655237888856633\\
457	0.00653598164334163\\
458	0.00651975971387527\\
459	0.00650381850604284\\
460	0.00648827189993001\\
461	0.00647325382423801\\
462	0.00645794555767362\\
463	0.00644233367387723\\
464	0.00642640309573541\\
465	0.00641013679445086\\
466	0.00639351544169336\\
467	0.00637651698950613\\
468	0.00635911616107912\\
469	0.00634128383119084\\
470	0.00632298626971447\\
471	0.0063041842149933\\
472	0.00628483173380949\\
473	0.00626487481204921\\
474	0.00624424960687539\\
475	0.00622288027233064\\
476	0.00620067624437277\\
477	0.00617752883802573\\
478	0.00615330696433514\\
479	0.00612785173133297\\
480	0.00610096961666709\\
481	0.00607242379360694\\
482	0.00604192306370216\\
483	0.00600910805190212\\
484	0.00597353391254416\\
485	0.00593464965252985\\
486	0.00589177709624734\\
487	0.0058447826614934\\
488	0.00579706219262404\\
489	0.00574862086705563\\
490	0.00569946786929883\\
491	0.00564961737971868\\
492	0.00559908980549789\\
493	0.00554791331535837\\
494	0.00549612575580097\\
495	0.00544377704720198\\
496	0.00539093218421757\\
497	0.00533767499801058\\
498	0.00528411287952206\\
499	0.00523038271546018\\
500	0.00517665835607147\\
501	0.0051231600238064\\
502	0.0050701661987078\\
503	0.00501802872194789\\
504	0.00496721929228177\\
505	0.00491837790758348\\
506	0.0048723283960852\\
507	0.0048301257606557\\
508	0.00479309668846519\\
509	0.00475625955239109\\
510	0.00471898918517486\\
511	0.00468131797440956\\
512	0.0046432840722822\\
513	0.00460493031991499\\
514	0.004566303879972\\
515	0.00452745570207619\\
516	0.00448844008980965\\
517	0.0044493097997433\\
518	0.00441008687171728\\
519	0.00437074144775284\\
520	0.00433122959196758\\
521	0.00429149784774394\\
522	0.00425142897253515\\
523	0.00421081080778946\\
524	0.00416929313380805\\
525	0.00412643381319458\\
526	0.00408215596336811\\
527	0.00403639275687767\\
528	0.00398910860078957\\
529	0.00394021841245417\\
530	0.00388963168405301\\
531	0.00383725090119883\\
532	0.00378297102842801\\
533	0.00372667994895513\\
534	0.00366826052558962\\
535	0.00360759499538104\\
536	0.00354458408089595\\
537	0.00347914048606964\\
538	0.00341120344786853\\
539	0.00334068582296219\\
540	0.00326757016964767\\
541	0.00319187772760867\\
542	0.0031136686903834\\
543	0.00303305075136004\\
544	0.00295018712023015\\
545	0.0028652920901408\\
546	0.00277862134303654\\
547	0.00269059274570993\\
548	0.00260190152765235\\
549	0.00251390181794879\\
550	0.00243061929734025\\
551	0.00235257346475307\\
552	0.00228003875241692\\
553	0.00221317257157155\\
554	0.00215112287025263\\
555	0.00209052503954314\\
556	0.00203108611377265\\
557	0.00197240575620483\\
558	0.00191405466630919\\
559	0.0018558391868882\\
560	0.001797667721791\\
561	0.00173945129258813\\
562	0.00168110421909231\\
563	0.00162243118677364\\
564	0.00156298187657357\\
565	0.00150437000468756\\
566	0.00144703158843678\\
567	0.00139122030448227\\
568	0.00133673591860897\\
569	0.00128228716789082\\
570	0.00122788747298109\\
571	0.00117350595556176\\
572	0.00111939032316793\\
573	0.00106700721614886\\
574	0.00101651208218248\\
575	0.000967001075013799\\
576	0.000917838202573339\\
577	0.000868968269894002\\
578	0.000820423355365856\\
579	0.000772347639745716\\
580	0.000725198081162504\\
581	0.000678755482625295\\
582	0.000632661594736207\\
583	0.000586930577994372\\
584	0.000541610321954773\\
585	0.00049675039409852\\
586	0.000452396819030694\\
587	0.000408589277753164\\
588	0.000365357975688249\\
589	0.000322720162279343\\
590	0.000280676716267942\\
591	0.000239312817487404\\
592	0.000198788644961841\\
593	0.000159293651685586\\
594	0.000121079888727805\\
595	8.45520570083912e-05\\
596	5.05092148680374e-05\\
597	2.07908715710836e-05\\
598	0\\
599	0\\
600	0\\
};
\addplot [color=blue!40!mycolor9,solid,forget plot]
  table[row sep=crcr]{%
1	0.0096074075755447\\
2	0.00960739244151761\\
3	0.00960737705294697\\
4	0.00960736140555528\\
5	0.00960734549499303\\
6	0.0096073293168376\\
7	0.00960731286659196\\
8	0.00960729613968336\\
9	0.00960727913146209\\
10	0.00960726183720023\\
11	0.00960724425209022\\
12	0.00960722637124369\\
13	0.00960720818968978\\
14	0.00960718970237404\\
15	0.0096071709041569\\
16	0.00960715178981218\\
17	0.00960713235402568\\
18	0.0096071125913937\\
19	0.00960709249642146\\
20	0.00960707206352154\\
21	0.00960705128701242\\
22	0.00960703016111673\\
23	0.00960700867995977\\
24	0.00960698683756772\\
25	0.00960696462786599\\
26	0.00960694204467757\\
27	0.00960691908172121\\
28	0.00960689573260967\\
29	0.00960687199084789\\
30	0.0096068478498312\\
31	0.00960682330284341\\
32	0.00960679834305482\\
33	0.00960677296352051\\
34	0.00960674715717811\\
35	0.00960672091684596\\
36	0.00960669423522108\\
37	0.00960666710487688\\
38	0.0096066395182613\\
39	0.00960661146769448\\
40	0.0096065829453667\\
41	0.00960655394333601\\
42	0.00960652445352608\\
43	0.00960649446772381\\
44	0.00960646397757705\\
45	0.0096064329745921\\
46	0.00960640145013138\\
47	0.00960636939541089\\
48	0.0096063368014977\\
49	0.00960630365930738\\
50	0.00960626995960139\\
51	0.00960623569298438\\
52	0.00960620084990154\\
53	0.00960616542063577\\
54	0.00960612939530495\\
55	0.00960609276385898\\
56	0.00960605551607695\\
57	0.00960601764156417\\
58	0.00960597912974904\\
59	0.0096059399698801\\
60	0.00960590015102288\\
61	0.00960585966205662\\
62	0.00960581849167114\\
63	0.00960577662836343\\
64	0.00960573406043442\\
65	0.00960569077598536\\
66	0.00960564676291445\\
67	0.00960560200891331\\
68	0.00960555650146327\\
69	0.00960551022783172\\
70	0.00960546317506837\\
71	0.00960541533000136\\
72	0.00960536667923342\\
73	0.00960531720913784\\
74	0.00960526690585453\\
75	0.00960521575528577\\
76	0.00960516374309205\\
77	0.00960511085468785\\
78	0.00960505707523722\\
79	0.00960500238964931\\
80	0.00960494678257397\\
81	0.00960489023839697\\
82	0.00960483274123542\\
83	0.00960477427493294\\
84	0.00960471482305481\\
85	0.00960465436888295\\
86	0.00960459289541087\\
87	0.00960453038533852\\
88	0.00960446682106704\\
89	0.00960440218469337\\
90	0.00960433645800474\\
91	0.0096042696224732\\
92	0.00960420165924989\\
93	0.00960413254915918\\
94	0.00960406227269294\\
95	0.0096039908100043\\
96	0.00960391814090174\\
97	0.00960384424484273\\
98	0.00960376910092729\\
99	0.00960369268789166\\
100	0.00960361498410159\\
101	0.00960353596754553\\
102	0.00960345561582784\\
103	0.00960337390616179\\
104	0.00960329081536231\\
105	0.00960320631983879\\
106	0.00960312039558761\\
107	0.00960303301818457\\
108	0.00960294416277709\\
109	0.00960285380407652\\
110	0.00960276191634992\\
111	0.00960266847341198\\
112	0.00960257344861666\\
113	0.00960247681484864\\
114	0.00960237854451474\\
115	0.00960227860953499\\
116	0.00960217698133369\\
117	0.00960207363083024\\
118	0.00960196852842973\\
119	0.00960186164401339\\
120	0.009601752946929\\
121	0.00960164240598088\\
122	0.00960152998941985\\
123	0.00960141566493296\\
124	0.00960129939963303\\
125	0.00960118116004798\\
126	0.00960106091210998\\
127	0.00960093862114444\\
128	0.00960081425185868\\
129	0.00960068776833064\\
130	0.009600559133997\\
131	0.00960042831164155\\
132	0.00960029526338298\\
133	0.00960015995066272\\
134	0.00960002233423241\\
135	0.00959988237414115\\
136	0.00959974002972281\\
137	0.00959959525958278\\
138	0.0095994480215847\\
139	0.00959929827283704\\
140	0.00959914596967927\\
141	0.00959899106766809\\
142	0.00959883352156315\\
143	0.00959867328531301\\
144	0.00959851031204037\\
145	0.00959834455402753\\
146	0.00959817596270155\\
147	0.0095980044886191\\
148	0.00959783008145125\\
149	0.00959765268996817\\
150	0.00959747226202343\\
151	0.00959728874453834\\
152	0.00959710208348609\\
153	0.0095969122238756\\
154	0.00959671910973556\\
155	0.00959652268409789\\
156	0.00959632288898141\\
157	0.00959611966537533\\
158	0.00959591295322249\\
159	0.00959570269140253\\
160	0.00959548881771502\\
161	0.00959527126886251\\
162	0.00959504998043319\\
163	0.00959482488688383\\
164	0.00959459592152232\\
165	0.00959436301649017\\
166	0.00959412610274484\\
167	0.00959388511004209\\
168	0.00959363996691783\\
169	0.00959339060067005\\
170	0.00959313693734044\\
171	0.00959287890169572\\
172	0.00959261641720862\\
173	0.00959234940603876\\
174	0.00959207778901272\\
175	0.00959180148560398\\
176	0.00959152041391221\\
177	0.00959123449064189\\
178	0.00959094363108033\\
179	0.00959064774907484\\
180	0.00959034675700905\\
181	0.0095900405657784\\
182	0.00958972908476431\\
183	0.00958941222180756\\
184	0.00958908988318017\\
185	0.00958876197355619\\
186	0.00958842839598119\\
187	0.00958808905184028\\
188	0.00958774384082494\\
189	0.00958739266089855\\
190	0.00958703540826065\\
191	0.00958667197731019\\
192	0.00958630226060791\\
193	0.00958592614883796\\
194	0.00958554353076913\\
195	0.00958515429321589\\
196	0.00958475832099956\\
197	0.00958435549690961\\
198	0.00958394570166566\\
199	0.00958352881387976\\
200	0.00958310471001807\\
201	0.00958267326436181\\
202	0.00958223434896743\\
203	0.00958178783362597\\
204	0.00958133358582178\\
205	0.00958087147069034\\
206	0.00958040135097536\\
207	0.00957992308698488\\
208	0.00957943653654678\\
209	0.00957894155496329\\
210	0.00957843799496454\\
211	0.00957792570666142\\
212	0.00957740453749741\\
213	0.00957687433219943\\
214	0.00957633493272788\\
215	0.00957578617822562\\
216	0.00957522790496593\\
217	0.00957465994629961\\
218	0.00957408213260086\\
219	0.00957349429121224\\
220	0.0095728962463885\\
221	0.00957228781923943\\
222	0.00957166882767131\\
223	0.00957103908632757\\
224	0.00957039840652806\\
225	0.00956974659620724\\
226	0.00956908345985101\\
227	0.00956840879843246\\
228	0.00956772240934639\\
229	0.00956702408634227\\
230	0.00956631361945617\\
231	0.00956559079494125\\
232	0.00956485539519678\\
233	0.00956410719869591\\
234	0.00956334597991188\\
235	0.00956257150924284\\
236	0.00956178355293522\\
237	0.0095609818730054\\
238	0.00956016622716001\\
239	0.00955933636871454\\
240	0.00955849204651045\\
241	0.00955763300483037\\
242	0.00955675898331186\\
243	0.0095558697168594\\
244	0.00955496493555446\\
245	0.00955404436456391\\
246	0.0095531077240466\\
247	0.00955215472905803\\
248	0.009551185089453\\
249	0.00955019850978655\\
250	0.00954919468921262\\
251	0.00954817332138089\\
252	0.00954713409433151\\
253	0.00954607669038755\\
254	0.00954500078604549\\
255	0.00954390605186338\\
256	0.00954279215234682\\
257	0.00954165874583254\\
258	0.00954050548436971\\
259	0.00953933201359898\\
260	0.00953813797262877\\
261	0.00953692299390935\\
262	0.00953568670310429\\
263	0.00953442871895927\\
264	0.00953314865316835\\
265	0.00953184611023756\\
266	0.00953052068734558\\
267	0.0095291719742018\\
268	0.00952779955290148\\
269	0.00952640299777804\\
270	0.00952498187525226\\
271	0.00952353574367869\\
272	0.0095220641531888\\
273	0.00952056664553099\\
274	0.00951904275390769\\
275	0.00951749200280888\\
276	0.00951591390784251\\
277	0.00951430797556143\\
278	0.00951267370328713\\
279	0.00951101057892972\\
280	0.00950931808080451\\
281	0.00950759567744505\\
282	0.00950584282741231\\
283	0.00950405897910038\\
284	0.00950224357053821\\
285	0.00950039602918757\\
286	0.00949851577173702\\
287	0.00949660220389208\\
288	0.00949465472016117\\
289	0.00949267270363763\\
290	0.00949065552577744\\
291	0.0094886025461728\\
292	0.00948651311232144\\
293	0.00948438655939151\\
294	0.00948222220998218\\
295	0.00948001937387975\\
296	0.00947777734780916\\
297	0.00947549541518103\\
298	0.00947317284583406\\
299	0.00947080889577266\\
300	0.00946840280689992\\
301	0.00946595380674576\\
302	0.00946346110819021\\
303	0.00946092390918188\\
304	0.00945834139245133\\
305	0.00945571272521959\\
306	0.00945303705890157\\
307	0.00945031352880442\\
308	0.00944754125382065\\
309	0.00944471933611625\\
310	0.0094418468608135\\
311	0.00943892289566851\\
312	0.00943594649074358\\
313	0.00943291667807412\\
314	0.0094298324713303\\
315	0.00942669286547325\\
316	0.00942349683640599\\
317	0.00942024334061878\\
318	0.00941693131482899\\
319	0.00941355967561562\\
320	0.00941012731904821\\
321	0.00940663312031021\\
322	0.00940307593331675\\
323	0.00939945459032695\\
324	0.00939576790155048\\
325	0.00939201465474842\\
326	0.00938819361482865\\
327	0.00938430352343525\\
328	0.0093803430985324\\
329	0.00937631103398228\\
330	0.00937220599911714\\
331	0.00936802663830553\\
332	0.00936377157051248\\
333	0.00935943938885354\\
334	0.00935502866014277\\
335	0.00935053792443428\\
336	0.00934596569455752\\
337	0.00934131045564591\\
338	0.00933657066465869\\
339	0.00933174474989592\\
340	0.00932683111050627\\
341	0.00932182811598741\\
342	0.00931673410567867\\
343	0.00931154738824574\\
344	0.00930626624115684\\
345	0.00930088891015026\\
346	0.00929541360869242\\
347	0.00928983851742644\\
348	0.00928416178361004\\
349	0.00927838152054283\\
350	0.00927249580698165\\
351	0.0092665026865438\\
352	0.00926040016709684\\
353	0.00925418622013451\\
354	0.00924785878013752\\
355	0.00924141574391836\\
356	0.00923485496994897\\
357	0.00922817427767003\\
358	0.00922137144678084\\
359	0.00921444421650832\\
360	0.00920739028485377\\
361	0.00920020730781615\\
362	0.00919289289859011\\
363	0.00918544462673757\\
364	0.00917786001733097\\
365	0.00917013655006702\\
366	0.0091622716583488\\
367	0.00915426272833516\\
368	0.00914610709795526\\
369	0.00913780205588702\\
370	0.00912934484049751\\
371	0.00912073263874376\\
372	0.00911196258503221\\
373	0.00910303176003477\\
374	0.00909393718945951\\
375	0.00908467584277375\\
376	0.00907524463187639\\
377	0.00906564040971642\\
378	0.00905585996885271\\
379	0.00904590003994982\\
380	0.00903575729020174\\
381	0.0090254283216738\\
382	0.00901490966954928\\
383	0.00900419780026284\\
384	0.00899328910949718\\
385	0.0089821799200117\\
386	0.00897086647926193\\
387	0.00895934495675723\\
388	0.0089476114410925\\
389	0.00893566193658549\\
390	0.00892349235939025\\
391	0.00891109853290295\\
392	0.00889847618237991\\
393	0.00888562092862454\\
394	0.00887252828055634\\
395	0.00885919362642484\\
396	0.0088456122235615\\
397	0.00883177918680417\\
398	0.00881768947542669\\
399	0.00880333787845024\\
400	0.00878871899876059\\
401	0.00877382723711889\\
402	0.00875865677849773\\
403	0.00874320158609038\\
404	0.00872745541512584\\
405	0.00871141187025567\\
406	0.00869506458123908\\
407	0.00867840695760704\\
408	0.00866143169002759\\
409	0.00864413122629998\\
410	0.00862649776265295\\
411	0.00860852323537416\\
412	0.00859019931259046\\
413	0.00857151738521099\\
414	0.00855246855384727\\
415	0.00853304360289888\\
416	0.0085132329392359\\
417	0.00849302644087801\\
418	0.00847241309164893\\
419	0.00845137969852336\\
420	0.00842991615910179\\
421	0.00840801263579948\\
422	0.00838565906355578\\
423	0.00836284514951068\\
424	0.00833956037385398\\
425	0.00831579399204311\\
426	0.00829153503852154\\
427	0.00826677233202398\\
428	0.00824149448407581\\
429	0.00821568991226852\\
430	0.00818934685571328\\
431	0.00816245339275658\\
432	0.00813499746363264\\
433	0.00810696689224114\\
434	0.00807834940349541\\
435	0.00804913265733362\\
436	0.00801930431272337\\
437	0.00798885212316119\\
438	0.00795776403013172\\
439	0.00792602794353914\\
440	0.00789361383292988\\
441	0.0078604925639844\\
442	0.00782664425161163\\
443	0.00779204802704287\\
444	0.00775668192794685\\
445	0.00772052276554722\\
446	0.00768354596203074\\
447	0.00764572534928105\\
448	0.00760703291693255\\
449	0.00756743849364348\\
450	0.0075269093399834\\
451	0.00748540962396113\\
452	0.00744289974033823\\
453	0.00739933542136051\\
454	0.00735466656663158\\
455	0.0073088356848356\\
456	0.00726177573290427\\
457	0.00721340736359842\\
458	0.00716363577891605\\
459	0.00711235074722052\\
460	0.00705942633796879\\
461	0.00700506357462199\\
462	0.0069897888830494\\
463	0.00697415177545719\\
464	0.00695813781822587\\
465	0.00694173199708212\\
466	0.0069249183178362\\
467	0.00690767973509524\\
468	0.0068899980807137\\
469	0.00687185399387683\\
470	0.00685322685424857\\
471	0.00683409471452101\\
472	0.00681443428282061\\
473	0.00679422095417364\\
474	0.0067734288573931\\
475	0.00675203096749532\\
476	0.00672999931676739\\
477	0.00670730535859358\\
478	0.00668392059335595\\
479	0.00665981715701593\\
480	0.00663496905067333\\
481	0.00660935439457848\\
482	0.00658296001510484\\
483	0.00655577770972172\\
484	0.00652780995818072\\
485	0.00649907640909796\\
486	0.00646961933516901\\
487	0.00643950390821288\\
488	0.00640880524000931\\
489	0.00637750451331813\\
490	0.00634558129860116\\
491	0.00631301334562971\\
492	0.00627977635446164\\
493	0.00624584372821354\\
494	0.00621118631284592\\
495	0.00617577213330074\\
496	0.00613956614119091\\
497	0.00610252999712457\\
498	0.00606462192026863\\
499	0.00602579664634754\\
500	0.00598600553441714\\
501	0.00594519683091755\\
502	0.00590331599908454\\
503	0.0058603058449187\\
504	0.00581513748167444\\
505	0.0057661407166878\\
506	0.00571256209853997\\
507	0.00565347260740609\\
508	0.00558774672291453\\
509	0.00552027502761627\\
510	0.00545171558099382\\
511	0.00538210725278225\\
512	0.00531150662435504\\
513	0.00523999271506372\\
514	0.00516767325099425\\
515	0.00509469267449204\\
516	0.00502124230610663\\
517	0.00494757317034403\\
518	0.004874013063796\\
519	0.00480099012756452\\
520	0.00472906273088725\\
521	0.00465895572787164\\
522	0.00459160707580159\\
523	0.00452822708277632\\
524	0.00447036487618838\\
525	0.0044184375290259\\
526	0.00436587397357472\\
527	0.0043126885610599\\
528	0.00425885476479472\\
529	0.00420427192507372\\
530	0.00414892816415185\\
531	0.00409285714575569\\
532	0.00403608410604617\\
533	0.00397861679527939\\
534	0.00392043183348241\\
535	0.0038614544566571\\
536	0.00380152811035017\\
537	0.00374037288718971\\
538	0.00367753023728315\\
539	0.00361247081341678\\
540	0.00354512799959542\\
541	0.00347543947168619\\
542	0.00340334641225652\\
543	0.0033287955821686\\
544	0.00325174240432155\\
545	0.00317215630959154\\
546	0.00309002880541974\\
547	0.00300538262148774\\
548	0.00291828382082769\\
549	0.00282884679169599\\
550	0.00273717375698181\\
551	0.00264351117793987\\
552	0.0025483095260621\\
553	0.00245231860330577\\
554	0.00235722079604287\\
555	0.00226672926843008\\
556	0.00218134599997664\\
557	0.00210146237082903\\
558	0.00202732470075\\
559	0.00195859464071399\\
560	0.00189147544706247\\
561	0.0018257439348815\\
562	0.00176104830254656\\
563	0.00169700040817372\\
564	0.0016334359172862\\
565	0.00157013355595559\\
566	0.0015069742191685\\
567	0.00144382934319225\\
568	0.00138100221627712\\
569	0.00131954993805358\\
570	0.0012596888460467\\
571	0.00120169720130309\\
572	0.00114504677832975\\
573	0.00108874011920619\\
574	0.00103277243214585\\
575	0.00097811364792387\\
576	0.000925437814226222\\
577	0.0008749039059792\\
578	0.000825214412511029\\
579	0.000776118395576129\\
580	0.000727556856375548\\
581	0.000679739350145835\\
582	0.00063299572419274\\
583	0.000587101587693234\\
584	0.000541705757972209\\
585	0.000496801587980551\\
586	0.000452422511210272\\
587	0.000408600951836586\\
588	0.000365362590765301\\
589	0.000322721662279121\\
590	0.000280677076068218\\
591	0.000239312865627478\\
592	0.00019878864496184\\
593	0.000159293651685586\\
594	0.000121079888727804\\
595	8.45520570083909e-05\\
596	5.0509214868037e-05\\
597	2.07908715710836e-05\\
598	0\\
599	0\\
600	0\\
};
\addplot [color=blue!75!mycolor7,solid,forget plot]
  table[row sep=crcr]{%
1	0.00983598579693763\\
2	0.00983598411159307\\
3	0.00983598239790594\\
4	0.00983598065539993\\
5	0.00983597888359073\\
6	0.00983597708198586\\
7	0.00983597525008457\\
8	0.00983597338737765\\
9	0.00983597149334735\\
10	0.00983596956746719\\
11	0.00983596760920181\\
12	0.00983596561800683\\
13	0.00983596359332872\\
14	0.00983596153460462\\
15	0.00983595944126217\\
16	0.00983595731271935\\
17	0.00983595514838439\\
18	0.00983595294765546\\
19	0.00983595070992064\\
20	0.00983594843455768\\
21	0.00983594612093382\\
22	0.00983594376840563\\
23	0.00983594137631882\\
24	0.00983593894400807\\
25	0.00983593647079682\\
26	0.00983593395599707\\
27	0.00983593139890923\\
28	0.00983592879882189\\
29	0.0098359261550116\\
30	0.00983592346674269\\
31	0.00983592073326708\\
32	0.00983591795382402\\
33	0.0098359151276399\\
34	0.00983591225392803\\
35	0.0098359093318884\\
36	0.00983590636070748\\
37	0.00983590333955796\\
38	0.00983590026759852\\
39	0.00983589714397359\\
40	0.00983589396781312\\
41	0.00983589073823231\\
42	0.00983588745433135\\
43	0.0098358841151952\\
44	0.00983588071989328\\
45	0.00983587726747922\\
46	0.00983587375699062\\
47	0.0098358701874487\\
48	0.0098358665578581\\
49	0.00983586286720652\\
50	0.00983585911446448\\
51	0.00983585529858498\\
52	0.00983585141850324\\
53	0.00983584747313635\\
54	0.00983584346138297\\
55	0.00983583938212302\\
56	0.00983583523421735\\
57	0.00983583101650739\\
58	0.00983582672781482\\
59	0.00983582236694124\\
60	0.0098358179326678\\
61	0.00983581342375487\\
62	0.00983580883894161\\
63	0.00983580417694569\\
64	0.00983579943646286\\
65	0.00983579461616655\\
66	0.00983578971470752\\
67	0.00983578473071342\\
68	0.00983577966278844\\
69	0.00983577450951282\\
70	0.00983576926944251\\
71	0.00983576394110868\\
72	0.00983575852301727\\
73	0.00983575301364863\\
74	0.00983574741145698\\
75	0.00983574171486998\\
76	0.00983573592228828\\
77	0.00983573003208498\\
78	0.0098357240426052\\
79	0.00983571795216557\\
80	0.00983571175905367\\
81	0.00983570546152761\\
82	0.00983569905781539\\
83	0.00983569254611445\\
84	0.00983568592459109\\
85	0.0098356791913799\\
86	0.00983567234458321\\
87	0.00983566538227052\\
88	0.00983565830247785\\
89	0.00983565110320724\\
90	0.00983564378242602\\
91	0.00983563633806631\\
92	0.00983562876802426\\
93	0.00983562107015949\\
94	0.00983561324229438\\
95	0.00983560528221342\\
96	0.00983559718766253\\
97	0.00983558895634832\\
98	0.00983558058593741\\
99	0.00983557207405573\\
100	0.00983556341828769\\
101	0.00983555461617552\\
102	0.00983554566521845\\
103	0.00983553656287193\\
104	0.00983552730654685\\
105	0.00983551789360869\\
106	0.00983550832137675\\
107	0.00983549858712324\\
108	0.00983548868807246\\
109	0.00983547862139991\\
110	0.0098354683842314\\
111	0.00983545797364213\\
112	0.00983544738665577\\
113	0.00983543662024351\\
114	0.00983542567132311\\
115	0.00983541453675792\\
116	0.00983540321335585\\
117	0.0098353916978684\\
118	0.00983537998698958\\
119	0.00983536807735489\\
120	0.00983535596554026\\
121	0.00983534364806088\\
122	0.0098353311213702\\
123	0.00983531838185871\\
124	0.00983530542585284\\
125	0.00983529224961377\\
126	0.00983527884933623\\
127	0.00983526522114734\\
128	0.00983525136110529\\
129	0.00983523726519819\\
130	0.00983522292934274\\
131	0.00983520834938293\\
132	0.00983519352108877\\
133	0.00983517844015494\\
134	0.0098351631021994\\
135	0.00983514750276208\\
136	0.00983513163730345\\
137	0.00983511550120311\\
138	0.00983509908975833\\
139	0.00983508239818266\\
140	0.00983506542160439\\
141	0.00983504815506511\\
142	0.00983503059351812\\
143	0.009835012731827\\
144	0.00983499456476397\\
145	0.00983497608700835\\
146	0.009834957293145\\
147	0.00983493817766266\\
148	0.00983491873495236\\
149	0.00983489895930575\\
150	0.0098348788449135\\
151	0.00983485838586352\\
152	0.00983483757613938\\
153	0.00983481640961852\\
154	0.00983479488007056\\
155	0.00983477298115552\\
156	0.00983475070642213\\
157	0.00983472804930598\\
158	0.00983470500312777\\
159	0.00983468156109151\\
160	0.00983465771628264\\
161	0.00983463346166626\\
162	0.00983460879008519\\
163	0.00983458369425816\\
164	0.00983455816677784\\
165	0.00983453220010894\\
166	0.00983450578658627\\
167	0.00983447891841271\\
168	0.00983445158765724\\
169	0.00983442378625285\\
170	0.00983439550599447\\
171	0.00983436673853685\\
172	0.00983433747539234\\
173	0.00983430770792866\\
174	0.00983427742736665\\
175	0.00983424662477785\\
176	0.00983421529108211\\
177	0.00983418341704506\\
178	0.00983415099327551\\
179	0.00983411801022278\\
180	0.00983408445817392\\
181	0.00983405032725079\\
182	0.0098340156074071\\
183	0.00983398028842531\\
184	0.00983394435991335\\
185	0.00983390781130135\\
186	0.00983387063183816\\
187	0.00983383281058781\\
188	0.00983379433642582\\
189	0.00983375519803545\\
190	0.00983371538390387\\
191	0.00983367488231827\\
192	0.00983363368136185\\
193	0.00983359176890989\\
194	0.00983354913262567\\
195	0.00983350575995649\\
196	0.00983346163812967\\
197	0.00983341675414844\\
198	0.00983337109478805\\
199	0.00983332464659158\\
200	0.0098332773958658\\
201	0.00983322932867695\\
202	0.00983318043084641\\
203	0.00983313068794629\\
204	0.00983308008529495\\
205	0.00983302860795242\\
206	0.00983297624071581\\
207	0.00983292296811446\\
208	0.00983286877440525\\
209	0.00983281364356758\\
210	0.00983275755929841\\
211	0.00983270050500715\\
212	0.00983264246381048\\
213	0.00983258341852703\\
214	0.00983252335167203\\
215	0.00983246224545177\\
216	0.00983240008175807\\
217	0.00983233684216253\\
218	0.00983227250791074\\
219	0.00983220705991638\\
220	0.00983214047875521\\
221	0.0098320727446589\\
222	0.00983200383750881\\
223	0.00983193373682962\\
224	0.00983186242178282\\
225	0.00983178987116015\\
226	0.00983171606337684\\
227	0.00983164097646476\\
228	0.00983156458806543\\
229	0.00983148687542292\\
230	0.00983140781537661\\
231	0.00983132738435378\\
232	0.0098312455583621\\
233	0.009831162312982\\
234	0.00983107762335881\\
235	0.00983099146419485\\
236	0.00983090380974132\\
237	0.00983081463379006\\
238	0.0098307239096651\\
239	0.00983063161021417\\
240	0.00983053770779993\\
241	0.00983044217429109\\
242	0.00983034498105338\\
243	0.00983024609894031\\
244	0.0098301454982838\\
245	0.0098300431488846\\
246	0.00982993902000258\\
247	0.00982983308034675\\
248	0.00982972529806523\\
249	0.00982961564073488\\
250	0.00982950407535088\\
251	0.00982939056831603\\
252	0.00982927508542986\\
253	0.00982915759187762\\
254	0.00982903805221893\\
255	0.00982891643037636\\
256	0.00982879268962371\\
257	0.00982866679257412\\
258	0.00982853870116795\\
259	0.00982840837666048\\
260	0.00982827577960929\\
261	0.00982814086986154\\
262	0.00982800360654092\\
263	0.00982786394803442\\
264	0.00982772185197886\\
265	0.00982757727524719\\
266	0.0098274301739345\\
267	0.00982728050334386\\
268	0.00982712821797182\\
269	0.00982697327149379\\
270	0.00982681561674903\\
271	0.00982665520572546\\
272	0.00982649198954422\\
273	0.00982632591844391\\
274	0.00982615694176462\\
275	0.00982598500793166\\
276	0.00982581006443908\\
277	0.0098256320578328\\
278	0.00982545093369359\\
279	0.00982526663661972\\
280	0.0098250791102093\\
281	0.00982488829704241\\
282	0.00982469413866293\\
283	0.00982449657556001\\
284	0.00982429554714937\\
285	0.00982409099175425\\
286	0.00982388284658605\\
287	0.00982367104772479\\
288	0.00982345553009914\\
289	0.0098232362274663\\
290	0.00982301307239145\\
291	0.00982278599622708\\
292	0.00982255492909187\\
293	0.00982231979984942\\
294	0.00982208053608657\\
295	0.00982183706409158\\
296	0.00982158930883191\\
297	0.00982133719393176\\
298	0.00982108064164936\\
299	0.00982081957285399\\
300	0.00982055390700268\\
301	0.00982028356211668\\
302	0.0098200084547577\\
303	0.00981972850000382\\
304	0.00981944361142525\\
305	0.00981915370105972\\
306	0.00981885867938778\\
307	0.00981855845530773\\
308	0.0098182529361104\\
309	0.00981794202745373\\
310	0.00981762563333706\\
311	0.00981730365607531\\
312	0.00981697599627289\\
313	0.0098166425527975\\
314	0.00981630322275366\\
315	0.00981595790145617\\
316	0.00981560648240336\\
317	0.00981524885725017\\
318	0.00981488491578119\\
319	0.00981451454588345\\
320	0.00981413763351916\\
321	0.00981375406269833\\
322	0.0098133637154513\\
323	0.00981296647180112\\
324	0.00981256220973592\\
325	0.00981215080518117\\
326	0.00981173213197181\\
327	0.00981130606182442\\
328	0.0098108724643092\\
329	0.00981043120682202\\
330	0.00980998215455628\\
331	0.00980952517047473\\
332	0.00980906011528128\\
333	0.00980858684739264\\
334	0.0098081052229099\\
335	0.00980761509558998\\
336	0.00980711631681693\\
337	0.00980660873557315\\
338	0.00980609219841025\\
339	0.00980556654941983\\
340	0.00980503163020391\\
341	0.00980448727984503\\
342	0.00980393333487598\\
343	0.00980336962924906\\
344	0.00980279599430481\\
345	0.0098022122587401\\
346	0.00980161824857549\\
347	0.00980101378712182\\
348	0.00980039869494577\\
349	0.00979977278983434\\
350	0.00979913588675813\\
351	0.00979848779783319\\
352	0.00979782833228123\\
353	0.00979715729638818\\
354	0.00979647449346058\\
355	0.00979577972377991\\
356	0.00979507278455434\\
357	0.00979435346986779\\
358	0.00979362157062594\\
359	0.009792876874499\\
360	0.00979211916586069\\
361	0.00979134822572343\\
362	0.00979056383166907\\
363	0.00978976575777495\\
364	0.00978895377453493\\
365	0.00978812764877493\\
366	0.00978728714356253\\
367	0.00978643201811032\\
368	0.0097855620276725\\
369	0.00978467692343428\\
370	0.00978377645239367\\
371	0.00978286035723521\\
372	0.00978192837619511\\
373	0.00978098024291748\\
374	0.00978001568630101\\
375	0.00977903443033571\\
376	0.00977803619392928\\
377	0.00977702069072246\\
378	0.00977598762889288\\
379	0.00977493671094685\\
380	0.00977386763349839\\
381	0.00977278008703469\\
382	0.00977167375566731\\
383	0.00977054831686783\\
384	0.0097694034411868\\
385	0.00976823879195447\\
386	0.00976705402496097\\
387	0.00976584878811385\\
388	0.00976462272106999\\
389	0.00976337545483856\\
390	0.00976210661134679\\
391	0.00976081580296672\\
392	0.00975950263199832\\
393	0.00975816669010216\\
394	0.00975680755767175\\
395	0.00975542480313566\\
396	0.009754017982191\\
397	0.00975258663695195\\
398	0.00975113029498769\\
399	0.0097496484682359\\
400	0.00974814065177603\\
401	0.0097466063224438\\
402	0.0097450449372837\\
403	0.00974345593202622\\
404	0.00974183872085087\\
405	0.00974019270298737\\
406	0.00973851726202865\\
407	0.00973681173897762\\
408	0.00973507542762574\\
409	0.00973330759298671\\
410	0.00973150746984965\\
411	0.00972967426148173\\
412	0.00972780713863298\\
413	0.00972590523909215\\
414	0.00972396766814842\\
415	0.0097219935001699\\
416	0.00971998177979321\\
417	0.00971793151220656\\
418	0.00971584159280819\\
419	0.00971371092465262\\
420	0.009711538524179\\
421	0.009709323390324\\
422	0.00970706448206493\\
423	0.00970476071462334\\
424	0.00970241095497063\\
425	0.00970001401646347\\
426	0.00969756865238058\\
427	0.00969507354811213\\
428	0.0096925273117392\\
429	0.00968992846244233\\
430	0.00968727541617079\\
431	0.00968456646808999\\
432	0.00968179977106074\\
433	0.00967897330958293\\
434	0.00967608487168244\\
435	0.00967313202846759\\
436	0.00967011215823784\\
437	0.00966702265224127\\
438	0.00966386179836363\\
439	0.00966063213953145\\
440	0.0096581438166461\\
441	0.00965618471398823\\
442	0.00965419083958201\\
443	0.0096521612180493\\
444	0.0096500947584985\\
445	0.00964799022959298\\
446	0.00964584622868649\\
447	0.00964366114348987\\
448	0.00964143310430262\\
449	0.00963915992428517\\
450	0.00963683902452495\\
451	0.00963446733971605\\
452	0.00963204119906393\\
453	0.00962955617537492\\
454	0.00962700689239675\\
455	0.00962438677223191\\
456	0.00962168769901288\\
457	0.00961889955818363\\
458	0.00961600960843108\\
459	0.00961300116654827\\
460	0.00960985036340646\\
461	0.0096061970184462\\
462	0.00956347884637286\\
463	0.00951990473209833\\
464	0.00947545229190016\\
465	0.00943009062697348\\
466	0.00938379558812048\\
467	0.00933654208400312\\
468	0.00928830404964439\\
469	0.00923905444750061\\
470	0.00918876536193582\\
471	0.00913740851543912\\
472	0.00908495400076351\\
473	0.00903136941236922\\
474	0.00897662092565435\\
475	0.00892067309233651\\
476	0.00886348839234452\\
477	0.00880502621294514\\
478	0.00874523943928828\\
479	0.00868408740739547\\
480	0.00862152976468595\\
481	0.00855752423148799\\
482	0.00849202612298176\\
483	0.00842498784328819\\
484	0.00835635834929183\\
485	0.00828608224596591\\
486	0.00821409847920065\\
487	0.0081403390213053\\
488	0.0080647285432332\\
489	0.00798718641478394\\
490	0.00790762535960176\\
491	0.0078259505817493\\
492	0.00774205873614955\\
493	0.0076558367082527\\
494	0.00756716015957302\\
495	0.00747589178508248\\
496	0.00738187921625173\\
497	0.00728495249213396\\
498	0.00718492101848958\\
499	0.00708156996690433\\
500	0.00697465620733147\\
501	0.00686390433814216\\
502	0.0067490048182002\\
503	0.00662962124945524\\
504	0.00654282147243754\\
505	0.00650416080218482\\
506	0.00646417271853324\\
507	0.00642286895962343\\
508	0.006380290798672\\
509	0.00633649164171396\\
510	0.00629140003101015\\
511	0.00624491050299828\\
512	0.00619690040501686\\
513	0.00614722597860324\\
514	0.00609571778460128\\
515	0.00604217500531807\\
516	0.0059863583248483\\
517	0.00592798101679594\\
518	0.00586669774697961\\
519	0.00580209039815476\\
520	0.00573365010006878\\
521	0.00566075463826947\\
522	0.0055826405375136\\
523	0.00549837040510512\\
524	0.00540680211042718\\
525	0.00530801300573983\\
526	0.00520877949944434\\
527	0.00510955840050037\\
528	0.0050109498653376\\
529	0.00491374327105498\\
530	0.00481743395479156\\
531	0.00472041401216693\\
532	0.0046230730277967\\
533	0.00452592907234172\\
534	0.00442966844522136\\
535	0.00433519904004857\\
536	0.00424372441285738\\
537	0.00415679393801028\\
538	0.00407632456672\\
539	0.00400303780149818\\
540	0.00392833741499244\\
541	0.00385228073333916\\
542	0.0037749313033303\\
543	0.00369635469524612\\
544	0.00361661131706036\\
545	0.00353574439772463\\
546	0.00345377407636575\\
547	0.0033706573271108\\
548	0.00328623817180589\\
549	0.00320020881435913\\
550	0.00311226168746383\\
551	0.00302174557633384\\
552	0.0029285721704536\\
553	0.00283284845112247\\
554	0.00273472165432939\\
555	0.0026343666970143\\
556	0.00253200294034252\\
557	0.00242802872893287\\
558	0.00232311566314357\\
559	0.00221851657094213\\
560	0.00211831907397233\\
561	0.00202299700719578\\
562	0.00193303203249592\\
563	0.00184882022133786\\
564	0.00177042797505062\\
565	0.00169520130901441\\
566	0.00162172309513774\\
567	0.00154972673714696\\
568	0.00147890019079677\\
569	0.00140915704425968\\
570	0.00133996649982656\\
571	0.00127121916543033\\
572	0.00120361310685485\\
573	0.00113828975820201\\
574	0.0010750352614193\\
575	0.00101406573453819\\
576	0.00095501450658787\\
577	0.000896782871369601\\
578	0.000840716170556225\\
579	0.0007870993751659\\
580	0.000735978059148706\\
581	0.000686067843775887\\
582	0.000637128970025702\\
583	0.000589268610066713\\
584	0.000542736806562125\\
585	0.000497389550407257\\
586	0.000452751633483821\\
587	0.000408776510568709\\
588	0.000365448460457315\\
589	0.000322758538660347\\
590	0.000280690270321202\\
591	0.000239316379010231\\
592	0.000198789174864552\\
593	0.000159293651685586\\
594	0.000121079888727805\\
595	8.45520570083912e-05\\
596	5.05092148680373e-05\\
597	2.07908715710836e-05\\
598	0\\
599	0\\
600	0\\
};
\addplot [color=blue!80!mycolor9,solid,forget plot]
  table[row sep=crcr]{%
1	0.00995024965342724\\
2	0.0099502494926092\\
3	0.00995024932908664\\
4	0.00995024916281409\\
5	0.0099502489937453\\
6	0.00995024882183327\\
7	0.00995024864703018\\
8	0.00995024846928742\\
9	0.00995024828855554\\
10	0.00995024810478429\\
11	0.00995024791792253\\
12	0.00995024772791829\\
13	0.00995024753471872\\
14	0.00995024733827005\\
15	0.00995024713851763\\
16	0.00995024693540588\\
17	0.00995024672887828\\
18	0.00995024651887735\\
19	0.00995024630534464\\
20	0.00995024608822072\\
21	0.00995024586744516\\
22	0.00995024564295647\\
23	0.00995024541469219\\
24	0.00995024518258872\\
25	0.00995024494658145\\
26	0.00995024470660464\\
27	0.00995024446259145\\
28	0.00995024421447391\\
29	0.0099502439621829\\
30	0.00995024370564812\\
31	0.00995024344479807\\
32	0.00995024317956006\\
33	0.00995024290986016\\
34	0.00995024263562317\\
35	0.00995024235677264\\
36	0.00995024207323081\\
37	0.00995024178491859\\
38	0.00995024149175555\\
39	0.0099502411936599\\
40	0.00995024089054847\\
41	0.00995024058233666\\
42	0.00995024026893843\\
43	0.00995023995026628\\
44	0.00995023962623122\\
45	0.00995023929674274\\
46	0.00995023896170881\\
47	0.00995023862103581\\
48	0.00995023827462851\\
49	0.0099502379223901\\
50	0.00995023756422208\\
51	0.00995023720002429\\
52	0.00995023682969484\\
53	0.00995023645313012\\
54	0.00995023607022472\\
55	0.00995023568087147\\
56	0.00995023528496134\\
57	0.00995023488238342\\
58	0.00995023447302493\\
59	0.00995023405677114\\
60	0.00995023363350536\\
61	0.0099502332031089\\
62	0.00995023276546102\\
63	0.00995023232043893\\
64	0.00995023186791771\\
65	0.00995023140777031\\
66	0.00995023093986748\\
67	0.00995023046407775\\
68	0.00995022998026739\\
69	0.00995022948830038\\
70	0.00995022898803832\\
71	0.00995022847934045\\
72	0.0099502279620636\\
73	0.00995022743606207\\
74	0.0099502269011877\\
75	0.00995022635728974\\
76	0.00995022580421483\\
77	0.00995022524180698\\
78	0.00995022466990747\\
79	0.00995022408835485\\
80	0.00995022349698486\\
81	0.00995022289563039\\
82	0.00995022228412143\\
83	0.00995022166228502\\
84	0.00995022102994519\\
85	0.00995022038692289\\
86	0.00995021973303601\\
87	0.00995021906809919\\
88	0.00995021839192389\\
89	0.00995021770431828\\
90	0.00995021700508715\\
91	0.0099502162940319\\
92	0.00995021557095047\\
93	0.00995021483563724\\
94	0.00995021408788301\\
95	0.0099502133274749\\
96	0.00995021255419632\\
97	0.00995021176782685\\
98	0.00995021096814223\\
99	0.00995021015491425\\
100	0.00995020932791068\\
101	0.00995020848689522\\
102	0.0099502076316274\\
103	0.00995020676186253\\
104	0.00995020587735159\\
105	0.00995020497784118\\
106	0.00995020406307343\\
107	0.00995020313278591\\
108	0.00995020218671155\\
109	0.00995020122457859\\
110	0.00995020024611043\\
111	0.00995019925102558\\
112	0.00995019823903758\\
113	0.00995019720985489\\
114	0.00995019616318081\\
115	0.00995019509871337\\
116	0.00995019401614525\\
117	0.00995019291516367\\
118	0.00995019179545031\\
119	0.00995019065668119\\
120	0.00995018949852657\\
121	0.00995018832065087\\
122	0.00995018712271253\\
123	0.00995018590436392\\
124	0.00995018466525123\\
125	0.00995018340501437\\
126	0.00995018212328684\\
127	0.00995018081969562\\
128	0.00995017949386106\\
129	0.00995017814539676\\
130	0.00995017677390944\\
131	0.00995017537899884\\
132	0.00995017396025759\\
133	0.00995017251727106\\
134	0.00995017104961728\\
135	0.00995016955686677\\
136	0.00995016803858243\\
137	0.00995016649431942\\
138	0.00995016492362499\\
139	0.00995016332603836\\
140	0.00995016170109063\\
141	0.00995016004830457\\
142	0.00995015836719451\\
143	0.00995015665726621\\
144	0.00995015491801671\\
145	0.00995015314893415\\
146	0.0099501513494977\\
147	0.00995014951917732\\
148	0.00995014765743366\\
149	0.00995014576371792\\
150	0.00995014383747166\\
151	0.00995014187812665\\
152	0.00995013988510474\\
153	0.00995013785781765\\
154	0.00995013579566686\\
155	0.00995013369804341\\
156	0.00995013156432773\\
157	0.00995012939388952\\
158	0.0099501271860875\\
159	0.00995012494026931\\
160	0.00995012265577127\\
161	0.00995012033191825\\
162	0.00995011796802346\\
163	0.00995011556338825\\
164	0.00995011311730196\\
165	0.00995011062904168\\
166	0.00995010809787208\\
167	0.00995010552304521\\
168	0.00995010290380026\\
169	0.00995010023936337\\
170	0.00995009752894742\\
171	0.00995009477175178\\
172	0.00995009196696209\\
173	0.00995008911375003\\
174	0.00995008621127306\\
175	0.00995008325867418\\
176	0.00995008025508167\\
177	0.00995007719960882\\
178	0.00995007409135364\\
179	0.00995007092939862\\
180	0.00995006771281036\\
181	0.00995006444063938\\
182	0.00995006111191971\\
183	0.00995005772566862\\
184	0.0099500542808863\\
185	0.00995005077655552\\
186	0.00995004721164129\\
187	0.00995004358509052\\
188	0.00995003989583166\\
189	0.00995003614277436\\
190	0.00995003232480913\\
191	0.00995002844080692\\
192	0.0099500244896188\\
193	0.0099500204700756\\
194	0.0099500163809875\\
195	0.00995001222114367\\
196	0.00995000798931191\\
197	0.00995000368423822\\
198	0.00994999930464644\\
199	0.00994999484923785\\
200	0.0099499903166907\\
201	0.00994998570565989\\
202	0.00994998101477646\\
203	0.00994997624264718\\
204	0.00994997138785415\\
205	0.00994996644895427\\
206	0.00994996142447887\\
207	0.00994995631293317\\
208	0.00994995111279585\\
209	0.00994994582251851\\
210	0.00994994044052526\\
211	0.00994993496521215\\
212	0.00994992939494666\\
213	0.0099499237280672\\
214	0.00994991796288258\\
215	0.00994991209767143\\
216	0.0099499061306817\\
217	0.00994990006013004\\
218	0.00994989388420129\\
219	0.0099498876010478\\
220	0.00994988120878893\\
221	0.0099498747055104\\
222	0.00994986808926366\\
223	0.00994986135806527\\
224	0.00994985450989627\\
225	0.00994984754270148\\
226	0.00994984045438891\\
227	0.00994983324282898\\
228	0.00994982590585392\\
229	0.00994981844125699\\
230	0.00994981084679182\\
231	0.00994980312017161\\
232	0.00994979525906845\\
233	0.00994978726111251\\
234	0.00994977912389129\\
235	0.00994977084494881\\
236	0.00994976242178481\\
237	0.00994975385185393\\
238	0.00994974513256489\\
239	0.00994973626127961\\
240	0.00994972723531233\\
241	0.0099497180519288\\
242	0.00994970870834526\\
243	0.00994969920172764\\
244	0.00994968952919054\\
245	0.00994967968779633\\
246	0.00994966967455414\\
247	0.00994965948641888\\
248	0.00994964912029027\\
249	0.00994963857301175\\
250	0.00994962784136949\\
251	0.00994961692209132\\
252	0.00994960581184558\\
253	0.00994959450724013\\
254	0.00994958300482111\\
255	0.00994957130107188\\
256	0.00994955939241183\\
257	0.00994954727519518\\
258	0.00994953494570976\\
259	0.00994952240017586\\
260	0.00994950963474487\\
261	0.00994949664549808\\
262	0.00994948342844539\\
263	0.00994946997952393\\
264	0.0099494562945968\\
265	0.00994944236945163\\
266	0.00994942819979924\\
267	0.00994941378127223\\
268	0.00994939910942352\\
269	0.0099493841797249\\
270	0.00994936898756556\\
271	0.00994935352825057\\
272	0.00994933779699935\\
273	0.00994932178894409\\
274	0.00994930549912823\\
275	0.00994928892250477\\
276	0.0099492720539347\\
277	0.00994925488818531\\
278	0.00994923741992849\\
279	0.00994921964373906\\
280	0.00994920155409301\\
281	0.00994918314536572\\
282	0.00994916441183021\\
283	0.00994914534765528\\
284	0.00994912594690369\\
285	0.0099491062035303\\
286	0.00994908611138015\\
287	0.00994906566418656\\
288	0.00994904485556918\\
289	0.00994902367903201\\
290	0.00994900212796138\\
291	0.00994898019562399\\
292	0.00994895787516478\\
293	0.0099489351596049\\
294	0.00994891204183962\\
295	0.00994888851463616\\
296	0.00994886457063157\\
297	0.00994884020233053\\
298	0.00994881540210321\\
299	0.00994879016218295\\
300	0.00994876447466411\\
301	0.00994873833149973\\
302	0.00994871172449927\\
303	0.00994868464532629\\
304	0.0099486570854961\\
305	0.00994862903637343\\
306	0.00994860048917002\\
307	0.00994857143494221\\
308	0.00994854186458856\\
309	0.00994851176884737\\
310	0.00994848113829424\\
311	0.00994844996333959\\
312	0.0099484182342261\\
313	0.00994838594102629\\
314	0.00994835307363989\\
315	0.00994831962179133\\
316	0.00994828557502712\\
317	0.0099482509227133\\
318	0.00994821565403277\\
319	0.00994817975798267\\
320	0.00994814322337174\\
321	0.00994810603881756\\
322	0.00994806819274393\\
323	0.00994802967337804\\
324	0.0099479904687478\\
325	0.00994795056667894\\
326	0.00994790995479227\\
327	0.00994786862050074\\
328	0.00994782655100661\\
329	0.00994778373329841\\
330	0.00994774015414801\\
331	0.00994769580010754\\
332	0.00994765065750626\\
333	0.0099476047124474\\
334	0.00994755795080488\\
335	0.00994751035821997\\
336	0.00994746192009786\\
337	0.00994741262160411\\
338	0.00994736244766097\\
339	0.00994731138294361\\
340	0.00994725941187616\\
341	0.00994720651862764\\
342	0.00994715268710765\\
343	0.00994709790096189\\
344	0.00994704214356749\\
345	0.00994698539802803\\
346	0.0099469276471684\\
347	0.00994686887352922\\
348	0.00994680905936108\\
349	0.00994674818661835\\
350	0.00994668623695262\\
351	0.00994662319170575\\
352	0.00994655903190239\\
353	0.00994649373824212\\
354	0.00994642729109096\\
355	0.00994635967047233\\
356	0.00994629085605737\\
357	0.00994622082715467\\
358	0.00994614956269912\\
359	0.00994607704124017\\
360	0.0099460032409291\\
361	0.00994592813950552\\
362	0.00994585171428288\\
363	0.00994577394213301\\
364	0.00994569479946957\\
365	0.00994561426223041\\
366	0.00994553230585879\\
367	0.00994544890528327\\
368	0.00994536403489642\\
369	0.00994527766853207\\
370	0.0099451897794412\\
371	0.00994510034026634\\
372	0.00994500932301441\\
373	0.0099449166990279\\
374	0.00994482243895451\\
375	0.00994472651271482\\
376	0.00994462888946838\\
377	0.00994452953757768\\
378	0.0099444284245702\\
379	0.00994432551709836\\
380	0.00994422078089722\\
381	0.00994411418073977\\
382	0.00994400568038974\\
383	0.00994389524255166\\
384	0.00994378282881784\\
385	0.00994366839961223\\
386	0.00994355191413059\\
387	0.00994343333027662\\
388	0.00994331260459367\\
389	0.00994318969219126\\
390	0.0099430645466659\\
391	0.00994293712001525\\
392	0.00994280736254448\\
393	0.00994267522276367\\
394	0.0099425406472743\\
395	0.00994240358064276\\
396	0.00994226396525818\\
397	0.00994212174117097\\
398	0.00994197684590766\\
399	0.00994182921425599\\
400	0.00994167877801249\\
401	0.00994152546568373\\
402	0.00994136920213928\\
403	0.00994120990824239\\
404	0.00994104750053987\\
405	0.00994088189096524\\
406	0.00994071298604871\\
407	0.00994054068635598\\
408	0.00994036488696237\\
409	0.00994018547708384\\
410	0.0099400023397103\\
411	0.00993981535125655\\
412	0.00993962438125417\\
413	0.00993942929211299\\
414	0.00993922993896757\\
415	0.0099390261695445\\
416	0.00993881782376216\\
417	0.00993860473254393\\
418	0.00993838671728155\\
419	0.00993816359336768\\
420	0.00993793516339354\\
421	0.00993770121470024\\
422	0.00993746151751359\\
423	0.00993721582279131\\
424	0.00993696385973073\\
425	0.00993670533287484\\
426	0.00993643991874117\\
427	0.00993616726188308\\
428	0.00993588697026948\\
429	0.00993559860983968\\
430	0.00993530169804049\\
431	0.00993499569604213\\
432	0.00993467999905065\\
433	0.00993435392334768\\
434	0.0099340166861986\\
435	0.00993366736666794\\
436	0.00993330480829898\\
437	0.00993292733222369\\
438	0.00993253180907329\\
439	0.00993211051292307\\
440	0.00993090909648561\\
441	0.00992913499893833\\
442	0.00992731780245534\\
443	0.00992545602322119\\
444	0.0099235481080655\\
445	0.00992159243265715\\
446	0.00991958730045363\\
447	0.00991753094274116\\
448	0.00991542152021376\\
449	0.00991325712668699\\
450	0.00991103579573368\\
451	0.00990875551127558\\
452	0.00990641422346764\\
453	0.00990400987154726\\
454	0.00990154041558171\\
455	0.0098990038789983\\
456	0.00989639840298146\\
457	0.00989372230993868\\
458	0.00989097416419154\\
459	0.00988815280269732\\
460	0.00988525731382974\\
461	0.00988227107081666\\
462	0.00987726586402313\\
463	0.00987217644234873\\
464	0.00986701232349152\\
465	0.00986207582144072\\
466	0.00985706019359379\\
467	0.0098519638048553\\
468	0.00984678493520374\\
469	0.00984152177448637\\
470	0.00983617244261739\\
471	0.00983073497250655\\
472	0.00982520722452214\\
473	0.00981958689007658\\
474	0.00981387154405269\\
475	0.00980805864121547\\
476	0.00980214548286862\\
477	0.00979612902455971\\
478	0.00979000614332297\\
479	0.0097837739541957\\
480	0.00977742945990819\\
481	0.00977096943423777\\
482	0.00976439038006545\\
483	0.00975768849293456\\
484	0.00975085960756883\\
485	0.00974389912357521\\
486	0.00973680192042395\\
487	0.00972956229998946\\
488	0.00972217402843945\\
489	0.00971463020511259\\
490	0.0097069231534983\\
491	0.00969904429054335\\
492	0.00969098396942824\\
493	0.00968273128975784\\
494	0.00967427386756375\\
495	0.00966559755544607\\
496	0.00965668610019438\\
497	0.00964752072027568\\
498	0.00963807957556147\\
499	0.00962833707685462\\
500	0.00961826291146477\\
501	0.00960782043746308\\
502	0.00959696336322506\\
503	0.00958562636004133\\
504	0.00953824077459935\\
505	0.0094428667253592\\
506	0.00934773776944827\\
507	0.00925021459251212\\
508	0.00915018834629194\\
509	0.00904753857441189\\
510	0.00894213426567151\\
511	0.00883382755758078\\
512	0.0087224673322712\\
513	0.00860790122505926\\
514	0.00848996516568321\\
515	0.00836848236237575\\
516	0.0082432622479476\\
517	0.00811409941774295\\
518	0.00798077260987465\\
519	0.00784304381493639\\
520	0.00770065766270751\\
521	0.00755334132295126\\
522	0.00740080524889721\\
523	0.00724274490892458\\
524	0.00707883679535215\\
525	0.00690873609418503\\
526	0.00673202972081162\\
527	0.00654799664263\\
528	0.00635577039827619\\
529	0.00615428802740013\\
530	0.00599822874844659\\
531	0.0059229785927858\\
532	0.00584393196698481\\
533	0.0057605063025082\\
534	0.00567197500609394\\
535	0.00557742595482749\\
536	0.00547570712171949\\
537	0.00536527523908269\\
538	0.00524427765188107\\
539	0.00511240149795213\\
540	0.00497826976213729\\
541	0.00484217428987451\\
542	0.00470452003423817\\
543	0.00456585911209548\\
544	0.00442693566006206\\
545	0.00428874727498958\\
546	0.00415262655058662\\
547	0.00402035366075448\\
548	0.00389417407788781\\
549	0.00377651184121573\\
550	0.00366528682586713\\
551	0.00356355257474437\\
552	0.00346211700310143\\
553	0.00335854913922197\\
554	0.00325299748522605\\
555	0.00314563583395675\\
556	0.00303666170785842\\
557	0.00292628271907932\\
558	0.00281469808724814\\
559	0.00270207149648971\\
560	0.00258855827107614\\
561	0.00247403834706246\\
562	0.00235818448456371\\
563	0.00224107956118409\\
564	0.00212364440435233\\
565	0.00200907458233774\\
566	0.00189926355595736\\
567	0.00179479801060058\\
568	0.00169626418907342\\
569	0.00160400293221254\\
570	0.00151783759986445\\
571	0.00143390318578877\\
572	0.00135203502647469\\
573	0.00127200640621873\\
574	0.00119387836966914\\
575	0.00111724688276997\\
576	0.0010422787258352\\
577	0.000970248500825921\\
578	0.000901376380147175\\
579	0.000835676581260966\\
580	0.000773155683964323\\
581	0.000713512969864073\\
582	0.000657039252267262\\
583	0.000603820761156831\\
584	0.000552908173124348\\
585	0.000503686775016135\\
586	0.0004564261874486\\
587	0.000410922180362966\\
588	0.000366669791782508\\
589	0.000323409636595769\\
590	0.000280998940489726\\
591	0.000239439664031353\\
592	0.000198826448456301\\
593	0.000159300178734201\\
594	0.000121079888727804\\
595	8.45520570083909e-05\\
596	5.05092148680373e-05\\
597	2.07908715710836e-05\\
598	0\\
599	0\\
600	0\\
};
\addplot [color=blue,solid,forget plot]
  table[row sep=crcr]{%
1	0.00996996602909277\\
2	0.00996996602247315\\
3	0.0099699660157422\\
4	0.00996996600889807\\
5	0.00996996600193883\\
6	0.00996996599486257\\
7	0.00996996598766729\\
8	0.00996996598035102\\
9	0.00996996597291171\\
10	0.00996996596534729\\
11	0.00996996595765566\\
12	0.00996996594983469\\
13	0.00996996594188218\\
14	0.00996996593379593\\
15	0.00996996592557369\\
16	0.00996996591721317\\
17	0.00996996590871205\\
18	0.00996996590006795\\
19	0.00996996589127847\\
20	0.00996996588234117\\
21	0.00996996587325356\\
22	0.0099699658640131\\
23	0.00996996585461723\\
24	0.00996996584506332\\
25	0.00996996583534872\\
26	0.00996996582547072\\
27	0.00996996581542656\\
28	0.00996996580521345\\
29	0.00996996579482855\\
30	0.00996996578426895\\
31	0.00996996577353172\\
32	0.00996996576261385\\
33	0.00996996575151232\\
34	0.00996996574022401\\
35	0.00996996572874579\\
36	0.00996996571707446\\
37	0.00996996570520674\\
38	0.00996996569313934\\
39	0.00996996568086888\\
40	0.00996996566839194\\
41	0.00996996565570504\\
42	0.00996996564280464\\
43	0.00996996562968712\\
44	0.00996996561634883\\
45	0.00996996560278604\\
46	0.00996996558899495\\
47	0.00996996557497172\\
48	0.00996996556071242\\
49	0.00996996554621307\\
50	0.0099699655314696\\
51	0.0099699655164779\\
52	0.00996996550123376\\
53	0.00996996548573293\\
54	0.00996996546997105\\
55	0.00996996545394371\\
56	0.00996996543764643\\
57	0.00996996542107463\\
58	0.00996996540422368\\
59	0.00996996538708884\\
60	0.00996996536966531\\
61	0.00996996535194819\\
62	0.00996996533393252\\
63	0.00996996531561324\\
64	0.0099699652969852\\
65	0.00996996527804317\\
66	0.00996996525878183\\
67	0.00996996523919575\\
68	0.00996996521927943\\
69	0.00996996519902727\\
70	0.00996996517843356\\
71	0.0099699651574925\\
72	0.0099699651361982\\
73	0.00996996511454465\\
74	0.00996996509252575\\
75	0.00996996507013529\\
76	0.00996996504736695\\
77	0.0099699650242143\\
78	0.0099699650006708\\
79	0.0099699649767298\\
80	0.00996996495238452\\
81	0.0099699649276281\\
82	0.00996996490245351\\
83	0.00996996487685363\\
84	0.00996996485082121\\
85	0.00996996482434888\\
86	0.00996996479742911\\
87	0.00996996477005428\\
88	0.00996996474221662\\
89	0.00996996471390822\\
90	0.00996996468512103\\
91	0.00996996465584687\\
92	0.00996996462607742\\
93	0.00996996459580418\\
94	0.00996996456501855\\
95	0.00996996453371174\\
96	0.00996996450187483\\
97	0.00996996446949874\\
98	0.00996996443657422\\
99	0.00996996440309185\\
100	0.00996996436904208\\
101	0.00996996433441515\\
102	0.00996996429920115\\
103	0.00996996426339\\
104	0.00996996422697143\\
105	0.009969964189935\\
106	0.00996996415227006\\
107	0.0099699641139658\\
108	0.00996996407501121\\
109	0.0099699640353951\\
110	0.00996996399510604\\
111	0.00996996395413243\\
112	0.00996996391246248\\
113	0.00996996387008414\\
114	0.0099699638269852\\
115	0.0099699637831532\\
116	0.00996996373857546\\
117	0.00996996369323909\\
118	0.00996996364713098\\
119	0.00996996360023775\\
120	0.00996996355254582\\
121	0.00996996350404134\\
122	0.00996996345471023\\
123	0.00996996340453816\\
124	0.00996996335351054\\
125	0.00996996330161252\\
126	0.009969963248829\\
127	0.00996996319514459\\
128	0.00996996314054364\\
129	0.00996996308501022\\
130	0.00996996302852812\\
131	0.00996996297108083\\
132	0.00996996291265156\\
133	0.00996996285322321\\
134	0.00996996279277839\\
135	0.00996996273129941\\
136	0.00996996266876823\\
137	0.00996996260516652\\
138	0.00996996254047562\\
139	0.00996996247467654\\
140	0.00996996240774995\\
141	0.00996996233967618\\
142	0.00996996227043522\\
143	0.0099699622000067\\
144	0.0099699621283699\\
145	0.00996996205550372\\
146	0.00996996198138669\\
147	0.00996996190599699\\
148	0.00996996182931239\\
149	0.00996996175131028\\
150	0.00996996167196766\\
151	0.00996996159126112\\
152	0.00996996150916683\\
153	0.00996996142566058\\
154	0.00996996134071771\\
155	0.00996996125431315\\
156	0.00996996116642136\\
157	0.00996996107701641\\
158	0.00996996098607187\\
159	0.0099699608935609\\
160	0.00996996079945615\\
161	0.00996996070372984\\
162	0.00996996060635368\\
163	0.00996996050729892\\
164	0.00996996040653629\\
165	0.00996996030403604\\
166	0.00996996019976788\\
167	0.00996996009370104\\
168	0.00996995998580419\\
169	0.00996995987604547\\
170	0.00996995976439248\\
171	0.00996995965081226\\
172	0.00996995953527128\\
173	0.00996995941773546\\
174	0.00996995929817009\\
175	0.0099699591765399\\
176	0.009969959052809\\
177	0.00996995892694088\\
178	0.0099699587988984\\
179	0.00996995866864378\\
180	0.00996995853613859\\
181	0.00996995840134372\\
182	0.00996995826421941\\
183	0.00996995812472517\\
184	0.00996995798281983\\
185	0.00996995783846149\\
186	0.00996995769160753\\
187	0.00996995754221457\\
188	0.00996995739023846\\
189	0.00996995723563431\\
190	0.00996995707835639\\
191	0.00996995691835821\\
192	0.00996995675559244\\
193	0.0099699565900109\\
194	0.0099699564215646\\
195	0.00996995625020365\\
196	0.00996995607587729\\
197	0.00996995589853388\\
198	0.00996995571812082\\
199	0.00996995553458463\\
200	0.00996995534787086\\
201	0.0099699551579241\\
202	0.00996995496468796\\
203	0.00996995476810505\\
204	0.00996995456811695\\
205	0.00996995436466423\\
206	0.00996995415768639\\
207	0.00996995394712184\\
208	0.00996995373290793\\
209	0.00996995351498086\\
210	0.00996995329327574\\
211	0.00996995306772648\\
212	0.00996995283826584\\
213	0.00996995260482538\\
214	0.00996995236733544\\
215	0.0099699521257251\\
216	0.00996995187992221\\
217	0.00996995162985329\\
218	0.0099699513754436\\
219	0.00996995111661702\\
220	0.00996995085329609\\
221	0.00996995058540196\\
222	0.00996995031285437\\
223	0.00996995003557163\\
224	0.00996994975347059\\
225	0.00996994946646659\\
226	0.00996994917447347\\
227	0.00996994887740352\\
228	0.00996994857516746\\
229	0.00996994826767441\\
230	0.00996994795483184\\
231	0.00996994763654557\\
232	0.00996994731271975\\
233	0.00996994698325675\\
234	0.00996994664805725\\
235	0.00996994630702009\\
236	0.0099699459600423\\
237	0.00996994560701908\\
238	0.00996994524784372\\
239	0.00996994488240757\\
240	0.00996994451060006\\
241	0.00996994413230859\\
242	0.00996994374741854\\
243	0.00996994335581322\\
244	0.00996994295737383\\
245	0.00996994255197941\\
246	0.00996994213950683\\
247	0.00996994171983073\\
248	0.00996994129282346\\
249	0.00996994085835508\\
250	0.0099699404162933\\
251	0.00996993996650341\\
252	0.00996993950884827\\
253	0.00996993904318826\\
254	0.00996993856938121\\
255	0.00996993808728239\\
256	0.00996993759674444\\
257	0.00996993709761731\\
258	0.00996993658974824\\
259	0.00996993607298169\\
260	0.00996993554715929\\
261	0.00996993501211981\\
262	0.00996993446769908\\
263	0.00996993391372993\\
264	0.00996993335004219\\
265	0.00996993277646257\\
266	0.00996993219281463\\
267	0.00996993159891875\\
268	0.009969930994592\\
269	0.00996993037964818\\
270	0.00996992975389767\\
271	0.00996992911714741\\
272	0.00996992846920084\\
273	0.00996992780985784\\
274	0.00996992713891465\\
275	0.00996992645616379\\
276	0.00996992576139404\\
277	0.00996992505439036\\
278	0.00996992433493378\\
279	0.00996992360280138\\
280	0.0099699228577662\\
281	0.00996992209959719\\
282	0.00996992132805907\\
283	0.00996992054291237\\
284	0.00996991974391325\\
285	0.00996991893081348\\
286	0.00996991810336036\\
287	0.00996991726129664\\
288	0.00996991640436042\\
289	0.00996991553228512\\
290	0.00996991464479934\\
291	0.00996991374162682\\
292	0.00996991282248637\\
293	0.00996991188709173\\
294	0.00996991093515156\\
295	0.0099699099663693\\
296	0.00996990898044311\\
297	0.00996990797706577\\
298	0.00996990695592461\\
299	0.00996990591670143\\
300	0.00996990485907237\\
301	0.00996990378270787\\
302	0.00996990268727253\\
303	0.00996990157242509\\
304	0.00996990043781826\\
305	0.00996989928309867\\
306	0.00996989810790679\\
307	0.0099698969118768\\
308	0.00996989569463652\\
309	0.00996989445580732\\
310	0.009969893195004\\
311	0.00996989191183472\\
312	0.00996989060590088\\
313	0.00996988927679704\\
314	0.00996988792411083\\
315	0.00996988654742281\\
316	0.00996988514630644\\
317	0.00996988372032789\\
318	0.00996988226904603\\
319	0.00996988079201227\\
320	0.00996987928877048\\
321	0.00996987775885688\\
322	0.00996987620179994\\
323	0.00996987461712028\\
324	0.00996987300433055\\
325	0.00996987136293531\\
326	0.009969869692431\\
327	0.0099698679923057\\
328	0.00996986626203913\\
329	0.0099698645011025\\
330	0.00996986270895836\\
331	0.00996986088506052\\
332	0.00996985902885392\\
333	0.0099698571397745\\
334	0.00996985521724908\\
335	0.0099698532606952\\
336	0.00996985126952101\\
337	0.00996984924312513\\
338	0.00996984718089648\\
339	0.00996984508221414\\
340	0.00996984294644718\\
341	0.0099698407729545\\
342	0.00996983856108466\\
343	0.00996983631017565\\
344	0.00996983401955474\\
345	0.00996983168853827\\
346	0.00996982931643136\\
347	0.00996982690252778\\
348	0.00996982444610959\\
349	0.00996982194644695\\
350	0.00996981940279781\\
351	0.00996981681440757\\
352	0.00996981418050881\\
353	0.0099698115003209\\
354	0.00996980877304962\\
355	0.00996980599788678\\
356	0.00996980317400977\\
357	0.00996980030058113\\
358	0.00996979737674798\\
359	0.00996979440164157\\
360	0.00996979137437667\\
361	0.00996978829405096\\
362	0.00996978515974438\\
363	0.00996978197051844\\
364	0.00996977872541548\\
365	0.00996977542345787\\
366	0.00996977206364713\\
367	0.00996976864496313\\
368	0.00996976516636301\\
369	0.00996976162678025\\
370	0.00996975802512356\\
371	0.00996975436027575\\
372	0.00996975063109251\\
373	0.00996974683640113\\
374	0.0099697429749992\\
375	0.00996973904565309\\
376	0.00996973504709655\\
377	0.00996973097802906\\
378	0.00996972683711417\\
379	0.00996972262297775\\
380	0.00996971833420612\\
381	0.00996971396934407\\
382	0.00996970952689278\\
383	0.00996970500530764\\
384	0.00996970040299582\\
385	0.00996969571831386\\
386	0.0099696909495649\\
387	0.00996968609499584\\
388	0.0099696811527942\\
389	0.00996967612108474\\
390	0.0099696709979258\\
391	0.00996966578130518\\
392	0.00996966046913576\\
393	0.00996965505925043\\
394	0.00996964954939657\\
395	0.00996964393722969\\
396	0.00996963822030625\\
397	0.00996963239607535\\
398	0.00996962646186914\\
399	0.00996962041489166\\
400	0.00996961425220605\\
401	0.00996960797072057\\
402	0.00996960156717452\\
403	0.00996959503812491\\
404	0.00996958837992792\\
405	0.00996958158870555\\
406	0.00996957466032922\\
407	0.00996956759043277\\
408	0.00996956037439588\\
409	0.00996955300732745\\
410	0.00996954548404943\\
411	0.00996953779908189\\
412	0.00996952994662935\\
413	0.0099695219205671\\
414	0.00996951371442238\\
415	0.00996950532134258\\
416	0.00996949673405511\\
417	0.00996948794487979\\
418	0.00996947894579393\\
419	0.00996946972817297\\
420	0.00996946028268595\\
421	0.00996945059920828\\
422	0.00996944066672143\\
423	0.00996943047319725\\
424	0.0099694200054641\\
425	0.00996940924905151\\
426	0.00996939818800903\\
427	0.00996938680469331\\
428	0.00996937507951417\\
429	0.00996936299062253\\
430	0.00996935051350271\\
431	0.00996933762037826\\
432	0.00996932427919697\\
433	0.00996931045158034\\
434	0.00996929608813639\\
435	0.00996928111709898\\
436	0.00996926541677184\\
437	0.00996924875236517\\
438	0.00996923065227521\\
439	0.00996921027240247\\
440	0.00996914722470959\\
441	0.00996905323791148\\
442	0.00996895700353201\\
443	0.00996885844626655\\
444	0.00996875748756669\\
445	0.00996865404562082\\
446	0.00996854803539017\\
447	0.00996843936872157\\
448	0.00996832795456464\\
449	0.0099682136993294\\
450	0.00996809650743038\\
451	0.00996797628207344\\
452	0.0099678529263493\\
453	0.00996772634469452\\
454	0.00996759644474977\\
455	0.00996746313955286\\
456	0.0099673263497839\\
457	0.00996718600542639\\
458	0.00996704204584876\\
459	0.00996689441745486\\
460	0.00996674306799945\\
461	0.0099665879168168\\
462	0.00996642858931903\\
463	0.00996626287390056\\
464	0.00996608086039723\\
465	0.00996560070924794\\
466	0.00996511141802948\\
467	0.00996461271605404\\
468	0.00996410431613691\\
469	0.00996358591377835\\
470	0.00996305718637086\\
471	0.00996251779086151\\
472	0.00996196736363268\\
473	0.00996140552254573\\
474	0.00996083186701568\\
475	0.0099602459774223\\
476	0.00995964741248944\\
477	0.0099590357087881\\
478	0.00995841039055374\\
479	0.00995777095424221\\
480	0.00995711686219497\\
481	0.00995644753924431\\
482	0.00995576236905725\\
483	0.00995506068982423\\
484	0.00995434178917141\\
485	0.00995360489833252\\
486	0.0099528491857619\\
487	0.00995207374943311\\
488	0.00995127760706633\\
489	0.00995045968546667\\
490	0.00994961880800696\\
491	0.00994875367986187\\
492	0.00994786287049847\\
493	0.00994694479278383\\
494	0.00994599767784186\\
495	0.00994501954436793\\
496	0.00994400816019456\\
497	0.00994296099166491\\
498	0.00994187513049822\\
499	0.00994074717187218\\
500	0.00993957297357205\\
501	0.00993834710585245\\
502	0.00993706147747492\\
503	0.00993570179724759\\
504	0.00993248398813303\\
505	0.00992458450554222\\
506	0.00991413653284674\\
507	0.00990352052705566\\
508	0.00989272929216558\\
509	0.00988175512175933\\
510	0.00987058941679399\\
511	0.0098592228462381\\
512	0.00984764596295923\\
513	0.00983584888235029\\
514	0.00982382077557695\\
515	0.00981154976657298\\
516	0.00979902281971108\\
517	0.00978622561948705\\
518	0.00977314244811886\\
519	0.00975975608224493\\
520	0.0097460477821495\\
521	0.00973199762945084\\
522	0.00971758611197099\\
523	0.00970280013215884\\
524	0.0096879112403344\\
525	0.00967331433680058\\
526	0.00965827364290924\\
527	0.0096427331866361\\
528	0.00962661829896217\\
529	0.00960982055197028\\
530	0.00953928920580111\\
531	0.009381629884998\\
532	0.00921843775153357\\
533	0.00904940266084434\\
534	0.00887420844912711\\
535	0.00869254825095905\\
536	0.00850414847764258\\
537	0.00831267373605276\\
538	0.00811927026652865\\
539	0.00791867309489294\\
540	0.00771042776501912\\
541	0.00749378093925573\\
542	0.00726785763693795\\
543	0.00703163999036617\\
544	0.00678393139256419\\
545	0.00652326853684657\\
546	0.00624786368936832\\
547	0.00595554554291631\\
548	0.00564387510515887\\
549	0.00532233053630014\\
550	0.00517288644773442\\
551	0.00500843477872852\\
552	0.00483831416493209\\
553	0.00466529191159948\\
554	0.00448978070172018\\
555	0.00431233796658526\\
556	0.00413367673565331\\
557	0.00395483603638486\\
558	0.00377714582892456\\
559	0.00360215714364037\\
560	0.00343164512815057\\
561	0.00326867563951183\\
562	0.00311755320915099\\
563	0.00297470414778135\\
564	0.00283388527610961\\
565	0.0026973744658845\\
566	0.00255897925763545\\
567	0.00241906878442113\\
568	0.002278315210554\\
569	0.00213766925746306\\
570	0.00199855167123961\\
571	0.00186534421375662\\
572	0.00173829019138898\\
573	0.00161722417034892\\
574	0.00150255682998357\\
575	0.00139486353128288\\
576	0.00129406798244129\\
577	0.0011966074677817\\
578	0.001102369942485\\
579	0.00101118175468484\\
580	0.00092283933127548\\
581	0.000838824294993086\\
582	0.000758927142791253\\
583	0.000683310297993498\\
584	0.00061327566441274\\
585	0.000548715386105192\\
586	0.000488688640616261\\
587	0.00043319358073894\\
588	0.000381096578785599\\
589	0.000332240260312541\\
590	0.000286210565888725\\
591	0.000242202348687757\\
592	0.000200085409992\\
593	0.000159745073714005\\
594	0.000121173935986304\\
595	8.45520570083912e-05\\
596	5.05092148680373e-05\\
597	2.07908715710836e-05\\
598	0\\
599	0\\
600	0\\
};
\addplot [color=mycolor10,solid,forget plot]
  table[row sep=crcr]{%
1	0.00997084548698741\\
2	0.00997084548691379\\
3	0.00997084548683892\\
4	0.0099708454867628\\
5	0.00997084548668539\\
6	0.00997084548660669\\
7	0.00997084548652666\\
8	0.00997084548644529\\
9	0.00997084548636254\\
10	0.00997084548627841\\
11	0.00997084548619286\\
12	0.00997084548610587\\
13	0.00997084548601742\\
14	0.00997084548592748\\
15	0.00997084548583603\\
16	0.00997084548574304\\
17	0.00997084548564849\\
18	0.00997084548555234\\
19	0.00997084548545458\\
20	0.00997084548535518\\
21	0.0099708454852541\\
22	0.00997084548515133\\
23	0.00997084548504682\\
24	0.00997084548494056\\
25	0.0099708454848325\\
26	0.00997084548472264\\
27	0.00997084548461093\\
28	0.00997084548449733\\
29	0.00997084548438182\\
30	0.00997084548426437\\
31	0.00997084548414495\\
32	0.00997084548402352\\
33	0.00997084548390004\\
34	0.00997084548377448\\
35	0.00997084548364682\\
36	0.009970845483517\\
37	0.00997084548338501\\
38	0.00997084548325079\\
39	0.00997084548311431\\
40	0.00997084548297553\\
41	0.00997084548283442\\
42	0.00997084548269094\\
43	0.00997084548254504\\
44	0.00997084548239668\\
45	0.00997084548224583\\
46	0.00997084548209243\\
47	0.00997084548193646\\
48	0.00997084548177786\\
49	0.00997084548161659\\
50	0.0099708454814526\\
51	0.00997084548128586\\
52	0.0099708454811163\\
53	0.00997084548094389\\
54	0.00997084548076858\\
55	0.00997084548059031\\
56	0.00997084548040904\\
57	0.00997084548022471\\
58	0.00997084548003728\\
59	0.00997084547984669\\
60	0.00997084547965289\\
61	0.00997084547945583\\
62	0.00997084547925545\\
63	0.00997084547905168\\
64	0.00997084547884449\\
65	0.00997084547863379\\
66	0.00997084547841955\\
67	0.0099708454782017\\
68	0.00997084547798016\\
69	0.0099708454777549\\
70	0.00997084547752584\\
71	0.00997084547729291\\
72	0.00997084547705604\\
73	0.00997084547681519\\
74	0.00997084547657027\\
75	0.00997084547632122\\
76	0.00997084547606796\\
77	0.00997084547581043\\
78	0.00997084547554855\\
79	0.00997084547528224\\
80	0.00997084547501144\\
81	0.00997084547473606\\
82	0.00997084547445604\\
83	0.00997084547417127\\
84	0.0099708454738817\\
85	0.00997084547358724\\
86	0.00997084547328779\\
87	0.00997084547298328\\
88	0.00997084547267362\\
89	0.00997084547235873\\
90	0.00997084547203851\\
91	0.00997084547171287\\
92	0.00997084547138171\\
93	0.00997084547104496\\
94	0.0099708454707025\\
95	0.00997084547035424\\
96	0.00997084547000008\\
97	0.00997084546963992\\
98	0.00997084546927367\\
99	0.0099708454689012\\
100	0.00997084546852242\\
101	0.00997084546813722\\
102	0.00997084546774548\\
103	0.0099708454673471\\
104	0.00997084546694196\\
105	0.00997084546652994\\
106	0.00997084546611093\\
107	0.00997084546568481\\
108	0.00997084546525145\\
109	0.00997084546481072\\
110	0.00997084546436251\\
111	0.00997084546390668\\
112	0.0099708454634431\\
113	0.00997084546297163\\
114	0.00997084546249215\\
115	0.00997084546200451\\
116	0.00997084546150856\\
117	0.00997084546100418\\
118	0.0099708454604912\\
119	0.00997084545996949\\
120	0.00997084545943889\\
121	0.00997084545889924\\
122	0.00997084545835039\\
123	0.00997084545779219\\
124	0.00997084545722446\\
125	0.00997084545664704\\
126	0.00997084545605977\\
127	0.00997084545546246\\
128	0.00997084545485496\\
129	0.00997084545423708\\
130	0.00997084545360864\\
131	0.00997084545296945\\
132	0.00997084545231933\\
133	0.0099708454516581\\
134	0.00997084545098555\\
135	0.00997084545030149\\
136	0.00997084544960571\\
137	0.00997084544889802\\
138	0.0099708454481782\\
139	0.00997084544744605\\
140	0.00997084544670135\\
141	0.00997084544594388\\
142	0.00997084544517342\\
143	0.00997084544438973\\
144	0.0099708454435926\\
145	0.00997084544278178\\
146	0.00997084544195704\\
147	0.00997084544111813\\
148	0.00997084544026481\\
149	0.00997084543939683\\
150	0.00997084543851392\\
151	0.00997084543761583\\
152	0.00997084543670229\\
153	0.00997084543577304\\
154	0.00997084543482779\\
155	0.00997084543386628\\
156	0.00997084543288821\\
157	0.0099708454318933\\
158	0.00997084543088126\\
159	0.00997084542985178\\
160	0.00997084542880455\\
161	0.00997084542773929\\
162	0.00997084542665566\\
163	0.00997084542555335\\
164	0.00997084542443202\\
165	0.00997084542329137\\
166	0.00997084542213103\\
167	0.00997084542095068\\
168	0.00997084541974996\\
169	0.00997084541852853\\
170	0.00997084541728601\\
171	0.00997084541602204\\
172	0.00997084541473626\\
173	0.00997084541342828\\
174	0.00997084541209771\\
175	0.00997084541074417\\
176	0.00997084540936725\\
177	0.00997084540796655\\
178	0.00997084540654165\\
179	0.00997084540509214\\
180	0.00997084540361759\\
181	0.00997084540211755\\
182	0.00997084540059161\\
183	0.00997084539903929\\
184	0.00997084539746014\\
185	0.0099708453958537\\
186	0.00997084539421949\\
187	0.00997084539255703\\
188	0.00997084539086584\\
189	0.00997084538914539\\
190	0.0099708453873952\\
191	0.00997084538561475\\
192	0.0099708453838035\\
193	0.00997084538196092\\
194	0.00997084538008647\\
195	0.00997084537817959\\
196	0.00997084537623971\\
197	0.00997084537426627\\
198	0.00997084537225868\\
199	0.00997084537021633\\
200	0.00997084536813864\\
201	0.00997084536602497\\
202	0.00997084536387471\\
203	0.00997084536168721\\
204	0.00997084535946183\\
205	0.0099708453571979\\
206	0.00997084535489475\\
207	0.0099708453525517\\
208	0.00997084535016805\\
209	0.00997084534774309\\
210	0.00997084534527609\\
211	0.00997084534276634\\
212	0.00997084534021306\\
213	0.00997084533761551\\
214	0.00997084533497291\\
215	0.00997084533228447\\
216	0.00997084532954939\\
217	0.00997084532676685\\
218	0.00997084532393602\\
219	0.00997084532105606\\
220	0.00997084531812609\\
221	0.00997084531514526\\
222	0.00997084531211265\\
223	0.00997084530902738\\
224	0.0099708453058885\\
225	0.00997084530269509\\
226	0.00997084529944617\\
227	0.00997084529614078\\
228	0.00997084529277792\\
229	0.00997084528935659\\
230	0.00997084528587576\\
231	0.00997084528233437\\
232	0.00997084527873137\\
233	0.00997084527506566\\
234	0.00997084527133614\\
235	0.0099708452675417\\
236	0.00997084526368117\\
237	0.00997084525975341\\
238	0.00997084525575722\\
239	0.0099708452516914\\
240	0.00997084524755472\\
241	0.00997084524334592\\
242	0.00997084523906373\\
243	0.00997084523470686\\
244	0.00997084523027398\\
245	0.00997084522576376\\
246	0.00997084522117482\\
247	0.00997084521650577\\
248	0.00997084521175519\\
249	0.00997084520692164\\
250	0.00997084520200365\\
251	0.00997084519699972\\
252	0.00997084519190834\\
253	0.00997084518672794\\
254	0.00997084518145695\\
255	0.00997084517609377\\
256	0.00997084517063675\\
257	0.00997084516508423\\
258	0.00997084515943451\\
259	0.00997084515368588\\
260	0.00997084514783656\\
261	0.00997084514188476\\
262	0.00997084513582868\\
263	0.00997084512966644\\
264	0.00997084512339617\\
265	0.00997084511701593\\
266	0.00997084511052378\\
267	0.00997084510391771\\
268	0.0099708450971957\\
269	0.00997084509035568\\
270	0.00997084508339555\\
271	0.00997084507631316\\
272	0.00997084506910634\\
273	0.00997084506177287\\
274	0.00997084505431048\\
275	0.00997084504671688\\
276	0.00997084503898972\\
277	0.00997084503112663\\
278	0.00997084502312516\\
279	0.00997084501498286\\
280	0.0099708450066972\\
281	0.00997084499826563\\
282	0.00997084498968554\\
283	0.00997084498095427\\
284	0.00997084497206913\\
285	0.00997084496302736\\
286	0.00997084495382617\\
287	0.00997084494446271\\
288	0.00997084493493408\\
289	0.00997084492523733\\
290	0.00997084491536945\\
291	0.00997084490532739\\
292	0.00997084489510803\\
293	0.00997084488470822\\
294	0.00997084487412471\\
295	0.00997084486335425\\
296	0.00997084485239349\\
297	0.00997084484123903\\
298	0.00997084482988741\\
299	0.00997084481833513\\
300	0.0099708448065786\\
301	0.00997084479461418\\
302	0.00997084478243817\\
303	0.00997084477004681\\
304	0.00997084475743624\\
305	0.00997084474460259\\
306	0.00997084473154187\\
307	0.00997084471825006\\
308	0.00997084470472304\\
309	0.00997084469095665\\
310	0.00997084467694664\\
311	0.00997084466268868\\
312	0.00997084464817839\\
313	0.00997084463341131\\
314	0.00997084461838288\\
315	0.00997084460308849\\
316	0.00997084458752345\\
317	0.00997084457168299\\
318	0.00997084455556225\\
319	0.00997084453915629\\
320	0.00997084452246011\\
321	0.0099708445054686\\
322	0.00997084448817659\\
323	0.00997084447057881\\
324	0.0099708444526699\\
325	0.00997084443444444\\
326	0.00997084441589689\\
327	0.00997084439702164\\
328	0.00997084437781298\\
329	0.00997084435826512\\
330	0.00997084433837215\\
331	0.00997084431812812\\
332	0.00997084429752692\\
333	0.00997084427656238\\
334	0.00997084425522824\\
335	0.00997084423351811\\
336	0.00997084421142552\\
337	0.00997084418894389\\
338	0.00997084416606655\\
339	0.00997084414278671\\
340	0.00997084411909746\\
341	0.00997084409499181\\
342	0.00997084407046265\\
343	0.00997084404550273\\
344	0.00997084402010473\\
345	0.00997084399426118\\
346	0.00997084396796449\\
347	0.00997084394120695\\
348	0.00997084391398074\\
349	0.0099708438862779\\
350	0.00997084385809033\\
351	0.0099708438294098\\
352	0.00997084380022794\\
353	0.00997084377053624\\
354	0.00997084374032603\\
355	0.00997084370958851\\
356	0.0099708436783147\\
357	0.00997084364649547\\
358	0.0099708436141215\\
359	0.00997084358118333\\
360	0.0099708435476713\\
361	0.00997084351357554\\
362	0.00997084347888602\\
363	0.00997084344359248\\
364	0.00997084340768447\\
365	0.0099708433711513\\
366	0.00997084333398205\\
367	0.00997084329616555\\
368	0.00997084325769042\\
369	0.00997084321854495\\
370	0.0099708431787172\\
371	0.00997084313819494\\
372	0.0099708430969656\\
373	0.00997084305501633\\
374	0.00997084301233393\\
375	0.00997084296890484\\
376	0.00997084292471516\\
377	0.00997084287975059\\
378	0.00997084283399642\\
379	0.00997084278743753\\
380	0.00997084274005834\\
381	0.00997084269184282\\
382	0.00997084264277444\\
383	0.00997084259283615\\
384	0.00997084254201035\\
385	0.00997084249027889\\
386	0.00997084243762299\\
387	0.00997084238402323\\
388	0.00997084232945953\\
389	0.00997084227391108\\
390	0.00997084221735631\\
391	0.00997084215977281\\
392	0.00997084210113733\\
393	0.00997084204142566\\
394	0.00997084198061258\\
395	0.00997084191867175\\
396	0.00997084185557563\\
397	0.00997084179129535\\
398	0.00997084172580054\\
399	0.0099708416590592\\
400	0.0099708415910376\\
401	0.00997084152170017\\
402	0.00997084145100952\\
403	0.00997084137892621\\
404	0.00997084130540826\\
405	0.00997084123041089\\
406	0.0099708411538868\\
407	0.00997084107578596\\
408	0.00997084099605557\\
409	0.00997084091463992\\
410	0.00997084083148032\\
411	0.00997084074651508\\
412	0.0099708406596792\\
413	0.00997084057090389\\
414	0.00997084048011545\\
415	0.00997084038723444\\
416	0.00997084029217644\\
417	0.00997084019485359\\
418	0.00997084009517133\\
419	0.00997083999302708\\
420	0.00997083988830929\\
421	0.0099708397808962\\
422	0.00997083967065452\\
423	0.00997083955743784\\
424	0.00997083944108465\\
425	0.00997083932141605\\
426	0.00997083919823265\\
427	0.00997083907131013\\
428	0.00997083894039184\\
429	0.0099708388051747\\
430	0.00997083866527911\\
431	0.00997083852018113\\
432	0.00997083836905726\\
433	0.0099708382104362\\
434	0.00997083804145571\\
435	0.00997083785641904\\
436	0.00997083764446549\\
437	0.0099708373873598\\
438	0.00997083706262313\\
439	0.00997083666448949\\
440	0.00997083623128341\\
441	0.00997083578821133\\
442	0.00997083533497104\\
443	0.00997083487124972\\
444	0.00997083439672429\\
445	0.00997083391106196\\
446	0.00997083341392111\\
447	0.0099708329049525\\
448	0.00997083238380088\\
449	0.0099708318501067\\
450	0.00997083130350745\\
451	0.00997083074363683\\
452	0.00997083017011757\\
453	0.00997082958253852\\
454	0.00997082898039505\\
455	0.00997082836294783\\
456	0.00997082772890378\\
457	0.00997082707570273\\
458	0.00997082639786802\\
459	0.00997082568288892\\
460	0.0099708249000621\\
461	0.00997082396978854\\
462	0.00997082269080592\\
463	0.00997082061716523\\
464	0.00997081682871155\\
465	0.00997079688302175\\
466	0.00997077655963949\\
467	0.00997075584701171\\
468	0.00997073473290042\\
469	0.00997071320433725\\
470	0.00997069124754055\\
471	0.00997066884793057\\
472	0.00997064599020059\\
473	0.00997062265830179\\
474	0.00997059883539591\\
475	0.00997057450379251\\
476	0.00997054964505322\\
477	0.00997052424024599\\
478	0.00997049826936458\\
479	0.00997047171110278\\
480	0.00997044454273607\\
481	0.00997041673999051\\
482	0.00997038827689002\\
483	0.0099703591255796\\
484	0.0099703292561263\\
485	0.00997029863629874\\
486	0.00997026723130749\\
487	0.00997023500349532\\
488	0.00997020191199549\\
489	0.00997016791233528\\
490	0.0099701329559731\\
491	0.00997009698975231\\
492	0.00997005995524691\\
493	0.0099700217879548\\
494	0.00996998241625147\\
495	0.00996994175991042\\
496	0.00996989972773478\\
497	0.00996985621320054\\
498	0.00996981108546361\\
499	0.00996976416950573\\
500	0.00996971520147044\\
501	0.0099696637305697\\
502	0.00996960891877847\\
503	0.00996954919283904\\
504	0.00996948186685571\\
505	0.00996929001369385\\
506	0.00996896204731497\\
507	0.0099686269049131\\
508	0.00996828423231784\\
509	0.00996793364018603\\
510	0.0099675747004568\\
511	0.00996720696758362\\
512	0.00996682997231482\\
513	0.00996644320053974\\
514	0.00996604608795879\\
515	0.00996563801379749\\
516	0.00996521829280044\\
517	0.00996478616328325\\
518	0.00996434076440996\\
519	0.00996388108095988\\
520	0.0099634057846906\\
521	0.00996291273618887\\
522	0.00996239734618991\\
523	0.0099618470340882\\
524	0.00996098584201838\\
525	0.00995942607081101\\
526	0.00995781208910305\\
527	0.00995613837293259\\
528	0.00995439742492012\\
529	0.00995257893419314\\
530	0.00994806151716236\\
531	0.00993936864397242\\
532	0.00993061374323875\\
533	0.00992179613327257\\
534	0.00991291338686796\\
535	0.00990395593179488\\
536	0.0098948917788269\\
537	0.00988209889720862\\
538	0.00986432247477629\\
539	0.00984622759948875\\
540	0.00982778327047255\\
541	0.00980895423599394\\
542	0.00978970060060886\\
543	0.00976997744200252\\
544	0.00974973268212212\\
545	0.00972890408895865\\
546	0.00970741598062481\\
547	0.00968517836419248\\
548	0.00966207204382259\\
549	0.00962617862383391\\
550	0.00940147602621652\\
551	0.00916775487436897\\
552	0.00892443714625995\\
553	0.0086706446418252\\
554	0.00840532921897203\\
555	0.00812729915250492\\
556	0.00783671321482161\\
557	0.00753063773729663\\
558	0.0072069709116426\\
559	0.00686361645109179\\
560	0.00650839030900701\\
561	0.00613192561547077\\
562	0.00572661969500788\\
563	0.0052961591784016\\
564	0.00484649400971892\\
565	0.00437548564659083\\
566	0.00415383614871475\\
567	0.00393383894687057\\
568	0.00371213006982874\\
569	0.00349020085871587\\
570	0.00326984259083614\\
571	0.00305370983431872\\
572	0.00284548539780111\\
573	0.00265016852435452\\
574	0.00245848661210091\\
575	0.00226662545412475\\
576	0.00207713061461257\\
577	0.00189587208535604\\
578	0.00172554605519975\\
579	0.00156853045100365\\
580	0.00142575417691301\\
581	0.00129100157684195\\
582	0.00116453672973894\\
583	0.00104320185244136\\
584	0.000925111851658451\\
585	0.000811581194832413\\
586	0.0007041690715661\\
587	0.000603390290013832\\
588	0.000510932493051772\\
589	0.00042719620397733\\
590	0.00035239839743971\\
591	0.000285859088514837\\
592	0.00022641344495182\\
593	0.000173745915387586\\
594	0.00012714455469262\\
595	8.61467099905492e-05\\
596	5.05092148680371e-05\\
597	2.07908715710836e-05\\
598	0\\
599	0\\
600	0\\
};
\addplot [color=mycolor11,solid,forget plot]
  table[row sep=crcr]{%
1	0.00997158250207167\\
2	0.00997158250206864\\
3	0.00997158250206557\\
4	0.00997158250206244\\
5	0.00997158250205925\\
6	0.00997158250205602\\
7	0.00997158250205273\\
8	0.00997158250204938\\
9	0.00997158250204598\\
10	0.00997158250204253\\
11	0.00997158250203901\\
12	0.00997158250203543\\
13	0.0099715825020318\\
14	0.0099715825020281\\
15	0.00997158250202434\\
16	0.00997158250202052\\
17	0.00997158250201663\\
18	0.00997158250201268\\
19	0.00997158250200866\\
20	0.00997158250200458\\
21	0.00997158250200042\\
22	0.0099715825019962\\
23	0.0099715825019919\\
24	0.00997158250198753\\
25	0.00997158250198309\\
26	0.00997158250197857\\
27	0.00997158250197398\\
28	0.00997158250196931\\
29	0.00997158250196457\\
30	0.00997158250195974\\
31	0.00997158250195483\\
32	0.00997158250194984\\
33	0.00997158250194476\\
34	0.0099715825019396\\
35	0.00997158250193435\\
36	0.00997158250192902\\
37	0.00997158250192359\\
38	0.00997158250191807\\
39	0.00997158250191246\\
40	0.00997158250190676\\
41	0.00997158250190096\\
42	0.00997158250189506\\
43	0.00997158250188906\\
44	0.00997158250188296\\
45	0.00997158250187676\\
46	0.00997158250187046\\
47	0.00997158250186404\\
48	0.00997158250185753\\
49	0.0099715825018509\\
50	0.00997158250184415\\
51	0.0099715825018373\\
52	0.00997158250183033\\
53	0.00997158250182324\\
54	0.00997158250181604\\
55	0.00997158250180871\\
56	0.00997158250180125\\
57	0.00997158250179368\\
58	0.00997158250178598\\
59	0.00997158250177814\\
60	0.00997158250177017\\
61	0.00997158250176207\\
62	0.00997158250175384\\
63	0.00997158250174546\\
64	0.00997158250173694\\
65	0.00997158250172828\\
66	0.00997158250171947\\
67	0.00997158250171052\\
68	0.00997158250170141\\
69	0.00997158250169215\\
70	0.00997158250168273\\
71	0.00997158250167316\\
72	0.00997158250166342\\
73	0.00997158250165352\\
74	0.00997158250164345\\
75	0.00997158250163321\\
76	0.0099715825016228\\
77	0.00997158250161221\\
78	0.00997158250160145\\
79	0.0099715825015905\\
80	0.00997158250157936\\
81	0.00997158250156805\\
82	0.00997158250155653\\
83	0.00997158250154483\\
84	0.00997158250153292\\
85	0.00997158250152082\\
86	0.0099715825015085\\
87	0.00997158250149599\\
88	0.00997158250148326\\
89	0.00997158250147031\\
90	0.00997158250145714\\
91	0.00997158250144376\\
92	0.00997158250143014\\
93	0.0099715825014163\\
94	0.00997158250140222\\
95	0.0099715825013879\\
96	0.00997158250137334\\
97	0.00997158250135853\\
98	0.00997158250134347\\
99	0.00997158250132816\\
100	0.00997158250131259\\
101	0.00997158250129675\\
102	0.00997158250128064\\
103	0.00997158250126427\\
104	0.00997158250124761\\
105	0.00997158250123067\\
106	0.00997158250121344\\
107	0.00997158250119592\\
108	0.0099715825011781\\
109	0.00997158250115998\\
110	0.00997158250114155\\
111	0.00997158250112281\\
112	0.00997158250110375\\
113	0.00997158250108436\\
114	0.00997158250106465\\
115	0.0099715825010446\\
116	0.0099715825010242\\
117	0.00997158250100346\\
118	0.00997158250098237\\
119	0.00997158250096092\\
120	0.0099715825009391\\
121	0.00997158250091691\\
122	0.00997158250089434\\
123	0.00997158250087139\\
124	0.00997158250084804\\
125	0.0099715825008243\\
126	0.00997158250080015\\
127	0.00997158250077559\\
128	0.0099715825007506\\
129	0.0099715825007252\\
130	0.00997158250069935\\
131	0.00997158250067307\\
132	0.00997158250064633\\
133	0.00997158250061914\\
134	0.00997158250059148\\
135	0.00997158250056335\\
136	0.00997158250053474\\
137	0.00997158250050563\\
138	0.00997158250047603\\
139	0.00997158250044592\\
140	0.0099715825004153\\
141	0.00997158250038414\\
142	0.00997158250035246\\
143	0.00997158250032023\\
144	0.00997158250028744\\
145	0.0099715825002541\\
146	0.00997158250022018\\
147	0.00997158250018568\\
148	0.00997158250015058\\
149	0.00997158250011488\\
150	0.00997158250007857\\
151	0.00997158250004163\\
152	0.00997158250000406\\
153	0.00997158249996584\\
154	0.00997158249992696\\
155	0.00997158249988742\\
156	0.00997158249984719\\
157	0.00997158249980627\\
158	0.00997158249976464\\
159	0.0099715824997223\\
160	0.00997158249967923\\
161	0.00997158249963541\\
162	0.00997158249959084\\
163	0.0099715824995455\\
164	0.00997158249949938\\
165	0.00997158249945247\\
166	0.00997158249940474\\
167	0.00997158249935619\\
168	0.0099715824993068\\
169	0.00997158249925656\\
170	0.00997158249920546\\
171	0.00997158249915346\\
172	0.00997158249910057\\
173	0.00997158249904678\\
174	0.00997158249899205\\
175	0.00997158249893637\\
176	0.00997158249887974\\
177	0.00997158249882212\\
178	0.00997158249876351\\
179	0.00997158249870389\\
180	0.00997158249864324\\
181	0.00997158249858154\\
182	0.00997158249851877\\
183	0.00997158249845492\\
184	0.00997158249838997\\
185	0.00997158249832389\\
186	0.00997158249825667\\
187	0.00997158249818829\\
188	0.00997158249811872\\
189	0.00997158249804795\\
190	0.00997158249797596\\
191	0.00997158249790273\\
192	0.00997158249782822\\
193	0.00997158249775243\\
194	0.00997158249767533\\
195	0.00997158249759689\\
196	0.00997158249751709\\
197	0.00997158249743592\\
198	0.00997158249735334\\
199	0.00997158249726933\\
200	0.00997158249718386\\
201	0.00997158249709692\\
202	0.00997158249700847\\
203	0.00997158249691849\\
204	0.00997158249682695\\
205	0.00997158249673382\\
206	0.00997158249663908\\
207	0.0099715824965427\\
208	0.00997158249644465\\
209	0.00997158249634489\\
210	0.00997158249624341\\
211	0.00997158249614017\\
212	0.00997158249603514\\
213	0.00997158249592829\\
214	0.00997158249581958\\
215	0.009971582495709\\
216	0.00997158249559648\\
217	0.00997158249548202\\
218	0.00997158249536557\\
219	0.0099715824952471\\
220	0.00997158249512657\\
221	0.00997158249500395\\
222	0.0099715824948792\\
223	0.00997158249475227\\
224	0.00997158249462315\\
225	0.00997158249449178\\
226	0.00997158249435813\\
227	0.00997158249422215\\
228	0.00997158249408381\\
229	0.00997158249394306\\
230	0.00997158249379987\\
231	0.00997158249365418\\
232	0.00997158249350596\\
233	0.00997158249335515\\
234	0.00997158249320173\\
235	0.00997158249304563\\
236	0.00997158249288681\\
237	0.00997158249272522\\
238	0.00997158249256082\\
239	0.00997158249239355\\
240	0.00997158249222337\\
241	0.00997158249205022\\
242	0.00997158249187405\\
243	0.00997158249169481\\
244	0.00997158249151243\\
245	0.00997158249132688\\
246	0.00997158249113808\\
247	0.00997158249094599\\
248	0.00997158249075055\\
249	0.00997158249055169\\
250	0.00997158249034935\\
251	0.00997158249014348\\
252	0.00997158248993401\\
253	0.00997158248972088\\
254	0.00997158248950401\\
255	0.00997158248928336\\
256	0.00997158248905884\\
257	0.00997158248883039\\
258	0.00997158248859794\\
259	0.00997158248836142\\
260	0.00997158248812076\\
261	0.00997158248787588\\
262	0.00997158248762671\\
263	0.00997158248737317\\
264	0.00997158248711518\\
265	0.00997158248685267\\
266	0.00997158248658555\\
267	0.00997158248631374\\
268	0.00997158248603716\\
269	0.00997158248575573\\
270	0.00997158248546934\\
271	0.00997158248517794\\
272	0.0099715824848814\\
273	0.00997158248457966\\
274	0.00997158248427261\\
275	0.00997158248396016\\
276	0.00997158248364221\\
277	0.00997158248331867\\
278	0.00997158248298943\\
279	0.0099715824826544\\
280	0.00997158248231347\\
281	0.00997158248196653\\
282	0.00997158248161348\\
283	0.00997158248125421\\
284	0.0099715824808886\\
285	0.00997158248051655\\
286	0.00997158248013794\\
287	0.00997158247975265\\
288	0.00997158247936056\\
289	0.00997158247896156\\
290	0.00997158247855551\\
291	0.0099715824781423\\
292	0.00997158247772179\\
293	0.00997158247729384\\
294	0.00997158247685835\\
295	0.00997158247641516\\
296	0.00997158247596413\\
297	0.00997158247550514\\
298	0.00997158247503804\\
299	0.00997158247456268\\
300	0.00997158247407891\\
301	0.00997158247358659\\
302	0.00997158247308556\\
303	0.00997158247257567\\
304	0.00997158247205676\\
305	0.00997158247152868\\
306	0.00997158247099126\\
307	0.00997158247044432\\
308	0.00997158246988771\\
309	0.00997158246932125\\
310	0.00997158246874477\\
311	0.00997158246815809\\
312	0.00997158246756103\\
313	0.00997158246695341\\
314	0.00997158246633504\\
315	0.00997158246570573\\
316	0.00997158246506528\\
317	0.00997158246441351\\
318	0.00997158246375021\\
319	0.00997158246307519\\
320	0.00997158246238822\\
321	0.00997158246168912\\
322	0.00997158246097766\\
323	0.00997158246025362\\
324	0.0099715824595168\\
325	0.00997158245876695\\
326	0.00997158245800387\\
327	0.00997158245722731\\
328	0.00997158245643705\\
329	0.00997158245563285\\
330	0.00997158245481445\\
331	0.00997158245398163\\
332	0.00997158245313413\\
333	0.0099715824522717\\
334	0.00997158245139408\\
335	0.00997158245050101\\
336	0.00997158244959222\\
337	0.00997158244866744\\
338	0.00997158244772641\\
339	0.00997158244676884\\
340	0.00997158244579445\\
341	0.00997158244480295\\
342	0.00997158244379405\\
343	0.00997158244276747\\
344	0.00997158244172288\\
345	0.00997158244066\\
346	0.0099715824395785\\
347	0.00997158243847807\\
348	0.0099715824373584\\
349	0.00997158243621915\\
350	0.00997158243505999\\
351	0.00997158243388059\\
352	0.0099715824326806\\
353	0.00997158243145967\\
354	0.00997158243021745\\
355	0.00997158242895357\\
356	0.00997158242766766\\
357	0.00997158242635936\\
358	0.00997158242502828\\
359	0.00997158242367402\\
360	0.00997158242229619\\
361	0.00997158242089439\\
362	0.00997158241946819\\
363	0.00997158241801718\\
364	0.00997158241654094\\
365	0.009971582415039\\
366	0.00997158241351092\\
367	0.00997158241195625\\
368	0.0099715824103745\\
369	0.00997158240876521\\
370	0.00997158240712786\\
371	0.00997158240546195\\
372	0.00997158240376696\\
373	0.00997158240204237\\
374	0.00997158240028761\\
375	0.00997158239850212\\
376	0.00997158239668534\\
377	0.00997158239483664\\
378	0.00997158239295544\\
379	0.00997158239104109\\
380	0.00997158238909294\\
381	0.00997158238711033\\
382	0.00997158238509255\\
383	0.00997158238303891\\
384	0.00997158238094865\\
385	0.00997158237882102\\
386	0.00997158237665522\\
387	0.00997158237445045\\
388	0.00997158237220584\\
389	0.00997158236992053\\
390	0.00997158236759361\\
391	0.00997158236522412\\
392	0.00997158236281107\\
393	0.00997158236035344\\
394	0.00997158235785015\\
395	0.00997158235530007\\
396	0.00997158235270203\\
397	0.00997158235005476\\
398	0.00997158234735696\\
399	0.00997158234460724\\
400	0.00997158234180415\\
401	0.00997158233894615\\
402	0.00997158233603162\\
403	0.00997158233305882\\
404	0.00997158233002592\\
405	0.00997158232693097\\
406	0.00997158232377192\\
407	0.00997158232054661\\
408	0.00997158231725276\\
409	0.00997158231388795\\
410	0.00997158231044967\\
411	0.00997158230693523\\
412	0.0099715823033418\\
413	0.00997158229966632\\
414	0.0099715822959055\\
415	0.0099715822920559\\
416	0.00997158228811384\\
417	0.00997158228407539\\
418	0.00997158227993627\\
419	0.00997158227569181\\
420	0.0099715822713369\\
421	0.00997158226686594\\
422	0.00997158226227277\\
423	0.00997158225755055\\
424	0.00997158225269167\\
425	0.00997158224768756\\
426	0.00997158224252837\\
427	0.00997158223720239\\
428	0.00997158223169493\\
429	0.00997158222598576\\
430	0.00997158222004395\\
431	0.00997158221381703\\
432	0.00997158220721015\\
433	0.00997158220004925\\
434	0.00997158219202694\\
435	0.0099715821826513\\
436	0.00997158217127756\\
437	0.00997158215739005\\
438	0.00997158214122857\\
439	0.00997158212401503\\
440	0.00997158210640971\\
441	0.00997158208840071\\
442	0.00997158206997567\\
443	0.00997158205112189\\
444	0.00997158203182631\\
445	0.00997158201207553\\
446	0.00997158199185592\\
447	0.00997158197115361\\
448	0.00997158194995454\\
449	0.00997158192824437\\
450	0.00997158190600811\\
451	0.00997158188322912\\
452	0.00997158185988651\\
453	0.00997158183594896\\
454	0.0099715818113606\\
455	0.00997158178600892\\
456	0.00997158175965185\\
457	0.00997158173175193\\
458	0.0099715817011006\\
459	0.00997158166499934\\
460	0.00997158161765242\\
461	0.00997158154774152\\
462	0.00997158143699461\\
463	0.00997158126622755\\
464	0.00997158104189281\\
465	0.00997158081343181\\
466	0.00997158058071974\\
467	0.00997158034362549\\
468	0.00997158010201132\\
469	0.00997157985573162\\
470	0.00997157960463319\\
471	0.00997157934855614\\
472	0.00997157908733262\\
473	0.0099715788207843\\
474	0.00997157854872028\\
475	0.0099715782709403\\
476	0.0099715779872406\\
477	0.00997157769740609\\
478	0.00997157740120733\\
479	0.00997157709839908\\
480	0.00997157678871876\\
481	0.00997157647188453\\
482	0.00997157614759318\\
483	0.00997157581551775\\
484	0.0099715754753051\\
485	0.00997157512657271\\
486	0.00997157476890483\\
487	0.00997157440184818\\
488	0.00997157402490682\\
489	0.00997157363753587\\
490	0.00997157323913343\\
491	0.00997157282902933\\
492	0.00997157240646753\\
493	0.00997157197057501\\
494	0.00997157152030013\\
495	0.00997157105428204\\
496	0.00997157057056529\\
497	0.00997157006597804\\
498	0.00997156953481639\\
499	0.00997156896621494\\
500	0.00997156833939104\\
501	0.00997156761655168\\
502	0.00997156673684214\\
503	0.00997156562584849\\
504	0.00997156425665817\\
505	0.00997156274604111\\
506	0.0099715612080005\\
507	0.00997155964131668\\
508	0.00997155804437759\\
509	0.00997155641517961\\
510	0.00997155475184757\\
511	0.00997155305262294\\
512	0.00997155131554636\\
513	0.00997154953838315\\
514	0.00997154771845011\\
515	0.00997154585217563\\
516	0.00997154393393788\\
517	0.00997154195295965\\
518	0.00997153988504464\\
519	0.00997153767096723\\
520	0.00997153516186568\\
521	0.00997153198982832\\
522	0.00997152729962883\\
523	0.00997151937369225\\
524	0.00997149318604075\\
525	0.00997142978152294\\
526	0.00997136434858027\\
527	0.00997129665617338\\
528	0.00997122642403624\\
529	0.00997115341161725\\
530	0.00997107758492663\\
531	0.00997099889826753\\
532	0.00997091691634657\\
533	0.00997083090031376\\
534	0.00997073943226779\\
535	0.00997063970133814\\
536	0.00997052669160612\\
537	0.0099702082159012\\
538	0.00996962328454001\\
539	0.00996902078936174\\
540	0.00996839922675705\\
541	0.00996775689666558\\
542	0.00996709187429563\\
543	0.00996640193383221\\
544	0.00996568437542882\\
545	0.00996493579503008\\
546	0.00996415169729546\\
547	0.00996332579684075\\
548	0.00996244843394521\\
549	0.00996092208856274\\
550	0.00995025740715425\\
551	0.00993968336057744\\
552	0.00992920641838787\\
553	0.00991882893957265\\
554	0.00990854606370902\\
555	0.00989832445380637\\
556	0.00988667401057027\\
557	0.00987483205160817\\
558	0.00986299957738175\\
559	0.00985111088440256\\
560	0.00982940086753967\\
561	0.00980373496254183\\
562	0.00977743457905058\\
563	0.00975058793269213\\
564	0.00972322653700024\\
565	0.00969510362383588\\
566	0.00940580285354155\\
567	0.00909740060809503\\
568	0.00877252040548564\\
569	0.00842896046466404\\
570	0.00806413126271723\\
571	0.00767497738991288\\
572	0.0072578893771046\\
573	0.00680862291380522\\
574	0.00633647419335779\\
575	0.00584531199636737\\
576	0.00533342520441997\\
577	0.00479868293791651\\
578	0.00423933523304903\\
579	0.00367550052531207\\
580	0.00317524304856927\\
581	0.00291649390778095\\
582	0.00266264787724582\\
583	0.0024212958895209\\
584	0.00219950504465807\\
585	0.00198101316635129\\
586	0.0017652741331012\\
587	0.00155340735392807\\
588	0.0013468096195542\\
589	0.00114664925084055\\
590	0.000951372357338311\\
591	0.000764343087165558\\
592	0.00058935041161929\\
593	0.000429546140184966\\
594	0.000289033664663948\\
595	0.000172252650723403\\
596	8.12550264159866e-05\\
597	2.07908715710836e-05\\
598	0\\
599	0\\
600	0\\
};
\addplot [color=mycolor12,solid,forget plot]
  table[row sep=crcr]{%
1	0.00998476612601138\\
2	0.00998476612601125\\
3	0.00998476612601112\\
4	0.00998476612601099\\
5	0.00998476612601085\\
6	0.00998476612601071\\
7	0.00998476612601057\\
8	0.00998476612601043\\
9	0.00998476612601028\\
10	0.00998476612601013\\
11	0.00998476612600998\\
12	0.00998476612600983\\
13	0.00998476612600967\\
14	0.00998476612600952\\
15	0.00998476612600935\\
16	0.00998476612600919\\
17	0.00998476612600903\\
18	0.00998476612600886\\
19	0.00998476612600869\\
20	0.00998476612600851\\
21	0.00998476612600833\\
22	0.00998476612600815\\
23	0.00998476612600797\\
24	0.00998476612600778\\
25	0.00998476612600759\\
26	0.0099847661260074\\
27	0.0099847661260072\\
28	0.009984766126007\\
29	0.0099847661260068\\
30	0.00998476612600659\\
31	0.00998476612600638\\
32	0.00998476612600617\\
33	0.00998476612600595\\
34	0.00998476612600573\\
35	0.00998476612600551\\
36	0.00998476612600528\\
37	0.00998476612600505\\
38	0.00998476612600481\\
39	0.00998476612600457\\
40	0.00998476612600433\\
41	0.00998476612600408\\
42	0.00998476612600383\\
43	0.00998476612600357\\
44	0.00998476612600331\\
45	0.00998476612600305\\
46	0.00998476612600278\\
47	0.0099847661260025\\
48	0.00998476612600223\\
49	0.00998476612600194\\
50	0.00998476612600165\\
51	0.00998476612600136\\
52	0.00998476612600106\\
53	0.00998476612600076\\
54	0.00998476612600045\\
55	0.00998476612600014\\
56	0.00998476612599982\\
57	0.0099847661259995\\
58	0.00998476612599917\\
59	0.00998476612599883\\
60	0.00998476612599849\\
61	0.00998476612599815\\
62	0.00998476612599779\\
63	0.00998476612599744\\
64	0.00998476612599707\\
65	0.0099847661259967\\
66	0.00998476612599633\\
67	0.00998476612599594\\
68	0.00998476612599555\\
69	0.00998476612599516\\
70	0.00998476612599476\\
71	0.00998476612599435\\
72	0.00998476612599393\\
73	0.00998476612599351\\
74	0.00998476612599307\\
75	0.00998476612599264\\
76	0.00998476612599219\\
77	0.00998476612599174\\
78	0.00998476612599128\\
79	0.00998476612599081\\
80	0.00998476612599034\\
81	0.00998476612598985\\
82	0.00998476612598936\\
83	0.00998476612598886\\
84	0.00998476612598835\\
85	0.00998476612598783\\
86	0.00998476612598731\\
87	0.00998476612598677\\
88	0.00998476612598623\\
89	0.00998476612598567\\
90	0.00998476612598511\\
91	0.00998476612598454\\
92	0.00998476612598396\\
93	0.00998476612598337\\
94	0.00998476612598276\\
95	0.00998476612598215\\
96	0.00998476612598153\\
97	0.00998476612598089\\
98	0.00998476612598025\\
99	0.0099847661259796\\
100	0.00998476612597893\\
101	0.00998476612597826\\
102	0.00998476612597757\\
103	0.00998476612597687\\
104	0.00998476612597616\\
105	0.00998476612597543\\
106	0.00998476612597469\\
107	0.00998476612597395\\
108	0.00998476612597318\\
109	0.00998476612597241\\
110	0.00998476612597162\\
111	0.00998476612597082\\
112	0.00998476612597\\
113	0.00998476612596918\\
114	0.00998476612596833\\
115	0.00998476612596747\\
116	0.0099847661259666\\
117	0.00998476612596572\\
118	0.00998476612596482\\
119	0.00998476612596389\\
120	0.00998476612596296\\
121	0.00998476612596201\\
122	0.00998476612596105\\
123	0.00998476612596007\\
124	0.00998476612595907\\
125	0.00998476612595805\\
126	0.00998476612595702\\
127	0.00998476612595597\\
128	0.0099847661259549\\
129	0.00998476612595382\\
130	0.00998476612595271\\
131	0.00998476612595159\\
132	0.00998476612595045\\
133	0.00998476612594928\\
134	0.0099847661259481\\
135	0.0099847661259469\\
136	0.00998476612594567\\
137	0.00998476612594443\\
138	0.00998476612594316\\
139	0.00998476612594187\\
140	0.00998476612594057\\
141	0.00998476612593923\\
142	0.00998476612593788\\
143	0.0099847661259365\\
144	0.0099847661259351\\
145	0.00998476612593367\\
146	0.00998476612593222\\
147	0.00998476612593075\\
148	0.00998476612592925\\
149	0.00998476612592772\\
150	0.00998476612592616\\
151	0.00998476612592459\\
152	0.00998476612592298\\
153	0.00998476612592135\\
154	0.00998476612591968\\
155	0.00998476612591799\\
156	0.00998476612591627\\
157	0.00998476612591452\\
158	0.00998476612591274\\
159	0.00998476612591093\\
160	0.00998476612590909\\
161	0.00998476612590721\\
162	0.00998476612590531\\
163	0.00998476612590336\\
164	0.0099847661259014\\
165	0.00998476612589939\\
166	0.00998476612589735\\
167	0.00998476612589527\\
168	0.00998476612589316\\
169	0.00998476612589101\\
170	0.00998476612588882\\
171	0.0099847661258866\\
172	0.00998476612588434\\
173	0.00998476612588204\\
174	0.0099847661258797\\
175	0.00998476612587731\\
176	0.00998476612587489\\
177	0.00998476612587243\\
178	0.00998476612586992\\
179	0.00998476612586737\\
180	0.00998476612586478\\
181	0.00998476612586214\\
182	0.00998476612585945\\
183	0.00998476612585672\\
184	0.00998476612585395\\
185	0.00998476612585112\\
186	0.00998476612584824\\
187	0.00998476612584532\\
188	0.00998476612584234\\
189	0.00998476612583932\\
190	0.00998476612583624\\
191	0.00998476612583311\\
192	0.00998476612582992\\
193	0.00998476612582668\\
194	0.00998476612582338\\
195	0.00998476612582003\\
196	0.00998476612581661\\
197	0.00998476612581314\\
198	0.00998476612580961\\
199	0.00998476612580602\\
200	0.00998476612580236\\
201	0.00998476612579864\\
202	0.00998476612579486\\
203	0.00998476612579101\\
204	0.00998476612578709\\
205	0.00998476612578311\\
206	0.00998476612577906\\
207	0.00998476612577493\\
208	0.00998476612577074\\
209	0.00998476612576647\\
210	0.00998476612576213\\
211	0.00998476612575772\\
212	0.00998476612575322\\
213	0.00998476612574865\\
214	0.009984766125744\\
215	0.00998476612573928\\
216	0.00998476612573446\\
217	0.00998476612572956\\
218	0.00998476612572458\\
219	0.00998476612571952\\
220	0.00998476612571436\\
221	0.00998476612570912\\
222	0.00998476612570378\\
223	0.00998476612569835\\
224	0.00998476612569283\\
225	0.00998476612568721\\
226	0.00998476612568149\\
227	0.00998476612567567\\
228	0.00998476612566975\\
229	0.00998476612566373\\
230	0.00998476612565761\\
231	0.00998476612565138\\
232	0.00998476612564504\\
233	0.00998476612563859\\
234	0.00998476612563202\\
235	0.00998476612562534\\
236	0.00998476612561855\\
237	0.00998476612561164\\
238	0.0099847661256046\\
239	0.00998476612559745\\
240	0.00998476612559017\\
241	0.00998476612558276\\
242	0.00998476612557522\\
243	0.00998476612556756\\
244	0.00998476612555975\\
245	0.00998476612555182\\
246	0.00998476612554374\\
247	0.00998476612553552\\
248	0.00998476612552716\\
249	0.00998476612551865\\
250	0.00998476612551\\
251	0.00998476612550119\\
252	0.00998476612549223\\
253	0.00998476612548311\\
254	0.00998476612547383\\
255	0.00998476612546439\\
256	0.00998476612545479\\
257	0.00998476612544501\\
258	0.00998476612543507\\
259	0.00998476612542495\\
260	0.00998476612541465\\
261	0.00998476612540418\\
262	0.00998476612539351\\
263	0.00998476612538267\\
264	0.00998476612537163\\
265	0.0099847661253604\\
266	0.00998476612534897\\
267	0.00998476612533734\\
268	0.00998476612532551\\
269	0.00998476612531347\\
270	0.00998476612530121\\
271	0.00998476612528874\\
272	0.00998476612527606\\
273	0.00998476612526314\\
274	0.00998476612525001\\
275	0.00998476612523664\\
276	0.00998476612522303\\
277	0.00998476612520919\\
278	0.00998476612519511\\
279	0.00998476612518077\\
280	0.00998476612516618\\
281	0.00998476612515134\\
282	0.00998476612513623\\
283	0.00998476612512086\\
284	0.00998476612510522\\
285	0.00998476612508929\\
286	0.0099847661250731\\
287	0.00998476612505661\\
288	0.00998476612503983\\
289	0.00998476612502276\\
290	0.00998476612500538\\
291	0.0099847661249877\\
292	0.00998476612496971\\
293	0.0099847661249514\\
294	0.00998476612493276\\
295	0.0099847661249138\\
296	0.0099847661248945\\
297	0.00998476612487486\\
298	0.00998476612485487\\
299	0.00998476612483453\\
300	0.00998476612481383\\
301	0.00998476612479277\\
302	0.00998476612477133\\
303	0.00998476612474951\\
304	0.00998476612472731\\
305	0.00998476612470471\\
306	0.00998476612468171\\
307	0.00998476612465831\\
308	0.00998476612463449\\
309	0.00998476612461025\\
310	0.00998476612458559\\
311	0.00998476612456048\\
312	0.00998476612453494\\
313	0.00998476612450894\\
314	0.00998476612448248\\
315	0.00998476612445555\\
316	0.00998476612442815\\
317	0.00998476612440026\\
318	0.00998476612437188\\
319	0.009984766124343\\
320	0.00998476612431361\\
321	0.0099847661242837\\
322	0.00998476612425326\\
323	0.00998476612422228\\
324	0.00998476612419075\\
325	0.00998476612415867\\
326	0.00998476612412602\\
327	0.00998476612409281\\
328	0.009984766124059\\
329	0.00998476612402459\\
330	0.00998476612398958\\
331	0.00998476612395395\\
332	0.0099847661239177\\
333	0.0099847661238808\\
334	0.00998476612384326\\
335	0.00998476612380506\\
336	0.00998476612376618\\
337	0.00998476612372663\\
338	0.00998476612368637\\
339	0.00998476612364541\\
340	0.00998476612360373\\
341	0.00998476612356133\\
342	0.00998476612351818\\
343	0.00998476612347427\\
344	0.00998476612342959\\
345	0.00998476612338413\\
346	0.00998476612333788\\
347	0.00998476612329082\\
348	0.00998476612324294\\
349	0.00998476612319422\\
350	0.00998476612314465\\
351	0.00998476612309421\\
352	0.0099847661230429\\
353	0.00998476612299069\\
354	0.00998476612293757\\
355	0.00998476612288353\\
356	0.00998476612282855\\
357	0.00998476612277261\\
358	0.00998476612271569\\
359	0.00998476612265779\\
360	0.00998476612259888\\
361	0.00998476612253895\\
362	0.00998476612247797\\
363	0.00998476612241593\\
364	0.00998476612235281\\
365	0.0099847661222886\\
366	0.00998476612222327\\
367	0.0099847661221568\\
368	0.00998476612208917\\
369	0.00998476612202037\\
370	0.00998476612195036\\
371	0.00998476612187914\\
372	0.00998476612180667\\
373	0.00998476612173293\\
374	0.0099847661216579\\
375	0.00998476612158155\\
376	0.00998476612150387\\
377	0.00998476612142482\\
378	0.00998476612134437\\
379	0.0099847661212625\\
380	0.00998476612117919\\
381	0.00998476612109439\\
382	0.00998476612100809\\
383	0.00998476612092025\\
384	0.00998476612083083\\
385	0.0099847661207398\\
386	0.00998476612064715\\
387	0.00998476612055281\\
388	0.00998476612045675\\
389	0.00998476612035895\\
390	0.00998476612025935\\
391	0.00998476612015792\\
392	0.00998476612005461\\
393	0.00998476611994938\\
394	0.00998476611984217\\
395	0.00998476611973294\\
396	0.00998476611962164\\
397	0.0099847661195082\\
398	0.00998476611939257\\
399	0.00998476611927468\\
400	0.00998476611915448\\
401	0.00998476611903188\\
402	0.00998476611890682\\
403	0.00998476611877922\\
404	0.00998476611864899\\
405	0.00998476611851605\\
406	0.00998476611838029\\
407	0.00998476611824164\\
408	0.00998476611809998\\
409	0.0099847661179552\\
410	0.0099847661178072\\
411	0.00998476611765584\\
412	0.00998476611750099\\
413	0.00998476611734252\\
414	0.00998476611718027\\
415	0.00998476611701409\\
416	0.0099847661168438\\
417	0.00998476611666922\\
418	0.00998476611649014\\
419	0.00998476611630634\\
420	0.00998476611611757\\
421	0.00998476611592355\\
422	0.00998476611572399\\
423	0.00998476611551854\\
424	0.0099847661153068\\
425	0.00998476611508829\\
426	0.00998476611486241\\
427	0.00998476611462834\\
428	0.00998476611438484\\
429	0.0099847661141299\\
430	0.0099847661138601\\
431	0.00998476611356959\\
432	0.00998476611324869\\
433	0.0099847661128826\\
434	0.00998476611245178\\
435	0.00998476611193696\\
436	0.00998476611133118\\
437	0.00998476611065367\\
438	0.00998476610994262\\
439	0.00998476610921538\\
440	0.00998476610847146\\
441	0.00998476610771034\\
442	0.00998476610693152\\
443	0.00998476610613445\\
444	0.00998476610531858\\
445	0.00998476610448337\\
446	0.00998476610362824\\
447	0.00998476610275261\\
448	0.00998476610185584\\
449	0.0099847661009372\\
450	0.00998476609999576\\
451	0.00998476609903\\
452	0.00998476609803725\\
453	0.00998476609701211\\
454	0.00998476609594338\\
455	0.00998476609480752\\
456	0.00998476609355561\\
457	0.00998476609208882\\
458	0.00998476609021732\\
459	0.00998476608760555\\
460	0.00998476608373925\\
461	0.00998476607801881\\
462	0.00998476607014061\\
463	0.00998476606073503\\
464	0.00998476605115606\\
465	0.00998476604139841\\
466	0.00998476603145654\\
467	0.00998476602132461\\
468	0.00998476601099649\\
469	0.00998476600046571\\
470	0.00998476598972551\\
471	0.00998476597876875\\
472	0.00998476596758783\\
473	0.00998476595617469\\
474	0.00998476594452093\\
475	0.00998476593261786\\
476	0.00998476592045632\\
477	0.0099847659080265\\
478	0.00998476589531791\\
479	0.00998476588231931\\
480	0.00998476586901861\\
481	0.0099847658554028\\
482	0.00998476584145781\\
483	0.00998476582716846\\
484	0.00998476581251822\\
485	0.00998476579748913\\
486	0.00998476578206153\\
487	0.00998476576621385\\
488	0.00998476574992225\\
489	0.00998476573316013\\
490	0.00998476571589736\\
491	0.0099847656980989\\
492	0.00998476567972207\\
493	0.00998476566071112\\
494	0.00998476564098614\\
495	0.00998476562042082\\
496	0.00998476559879951\\
497	0.0099847655757395\\
498	0.00998476555056474\\
499	0.00998476552213733\\
500	0.00998476548872632\\
501	0.00998476544816116\\
502	0.00998476539873531\\
503	0.00998476534112569\\
504	0.00998476527941104\\
505	0.00998476521657973\\
506	0.00998476515257641\\
507	0.00998476508733086\\
508	0.00998476502076274\\
509	0.00998476495279559\\
510	0.00998476488335207\\
511	0.00998476481234044\\
512	0.00998476473964256\\
513	0.00998476466508539\\
514	0.00998476458837248\\
515	0.0099847645089216\\
516	0.00998476442548878\\
517	0.00998476433532989\\
518	0.00998476423243209\\
519	0.00998476410410157\\
520	0.00998476392536108\\
521	0.00998476365277688\\
522	0.00998476322724478\\
523	0.00998476261074155\\
524	0.00998476185929397\\
525	0.00998476107951758\\
526	0.00998476026608737\\
527	0.00998475941368839\\
528	0.00998475851976169\\
529	0.00998475758590979\\
530	0.00998475660914964\\
531	0.00998475557133194\\
532	0.00998475443618455\\
533	0.00998475313530915\\
534	0.00998475155561873\\
535	0.00998474955614854\\
536	0.00998474707681101\\
537	0.00998474431322528\\
538	0.00998474148463942\\
539	0.00998473858561031\\
540	0.00998473560989778\\
541	0.00998473254958401\\
542	0.00998472939203223\\
543	0.0099847261133409\\
544	0.00998472266845293\\
545	0.00998471897603467\\
546	0.00998471489917038\\
547	0.00998471023443736\\
548	0.00998470478673389\\
549	0.009984698629892\\
550	0.0099846922355815\\
551	0.00998468539895788\\
552	0.00998467761634069\\
553	0.00998466766639784\\
554	0.00998465284881902\\
555	0.00998462839384316\\
556	0.00998451835371763\\
557	0.00998438713828728\\
558	0.00998424091014161\\
559	0.00998407048417542\\
560	0.00998337245231295\\
561	0.0099824607661268\\
562	0.00998150752370361\\
563	0.00998050497714601\\
564	0.00997944065250886\\
565	0.00997829668708648\\
566	0.00996445051854514\\
567	0.00995056165887344\\
568	0.00993686172377713\\
569	0.00992336079454392\\
570	0.00991007448540605\\
571	0.0098970254209339\\
572	0.0098842472847136\\
573	0.00987180223381704\\
574	0.00986010348666058\\
575	0.00984931761907658\\
576	0.00983942691884809\\
577	0.00983036670881545\\
578	0.00982193005125355\\
579	0.0097931048823283\\
580	0.00967661252408177\\
581	0.00930277543452092\\
582	0.00890533237643708\\
583	0.00848053657077329\\
584	0.00802397360009435\\
585	0.00755205906109305\\
586	0.00706559415643771\\
587	0.00656389548252311\\
588	0.00604646101757425\\
589	0.00551317570106527\\
590	0.00496674996040426\\
591	0.00440647859325226\\
592	0.00383129818654383\\
593	0.00323992338914098\\
594	0.0026306372037739\\
595	0.00200104994316786\\
596	0.00135174293718153\\
597	0.000683662792068022\\
598	0\\
599	0\\
600	0\\
};
\addplot [color=mycolor13,solid,forget plot]
  table[row sep=crcr]{%
1	0.000484240466903288\\
2	0.000484240466903338\\
3	0.000484240466903389\\
4	0.000484240466903441\\
5	0.000484240466903493\\
6	0.000484240466903547\\
7	0.000484240466903602\\
8	0.000484240466903656\\
9	0.000484240466903713\\
10	0.000484240466903771\\
11	0.000484240466903829\\
12	0.000484240466903889\\
13	0.00048424046690395\\
14	0.000484240466904013\\
15	0.000484240466904075\\
16	0.000484240466904139\\
17	0.000484240466904205\\
18	0.000484240466904272\\
19	0.000484240466904339\\
20	0.000484240466904408\\
21	0.000484240466904478\\
22	0.000484240466904549\\
23	0.000484240466904622\\
24	0.000484240466904695\\
25	0.000484240466904771\\
26	0.000484240466904848\\
27	0.000484240466904927\\
28	0.000484240466905006\\
29	0.000484240466905086\\
30	0.000484240466905168\\
31	0.000484240466905252\\
32	0.000484240466905337\\
33	0.000484240466905425\\
34	0.000484240466905513\\
35	0.000484240466905602\\
36	0.000484240466905693\\
37	0.000484240466905788\\
38	0.000484240466905882\\
39	0.000484240466905979\\
40	0.000484240466906078\\
41	0.000484240466906178\\
42	0.000484240466906279\\
43	0.000484240466906384\\
44	0.000484240466906488\\
45	0.000484240466906596\\
46	0.000484240466906706\\
47	0.000484240466906817\\
48	0.000484240466906931\\
49	0.000484240466907046\\
50	0.000484240466907165\\
51	0.000484240466907284\\
52	0.000484240466907405\\
53	0.000484240466907529\\
54	0.000484240466907655\\
55	0.000484240466907784\\
56	0.000484240466907914\\
57	0.000484240466908047\\
58	0.000484240466908183\\
59	0.000484240466908321\\
60	0.000484240466908462\\
61	0.000484240466908604\\
62	0.000484240466908751\\
63	0.000484240466908898\\
64	0.000484240466909049\\
65	0.000484240466909203\\
66	0.000484240466909359\\
67	0.000484240466909518\\
68	0.00048424046690968\\
69	0.000484240466909845\\
70	0.000484240466910012\\
71	0.000484240466910183\\
72	0.000484240466910357\\
73	0.000484240466910534\\
74	0.000484240466910714\\
75	0.000484240466910899\\
76	0.000484240466911085\\
77	0.000484240466911275\\
78	0.000484240466911469\\
79	0.000484240466911665\\
80	0.000484240466911866\\
81	0.00048424046691207\\
82	0.000484240466912277\\
83	0.000484240466912489\\
84	0.000484240466912703\\
85	0.000484240466912923\\
86	0.000484240466913147\\
87	0.000484240466913373\\
88	0.000484240466913604\\
89	0.00048424046691384\\
90	0.000484240466914079\\
91	0.000484240466914324\\
92	0.000484240466914572\\
93	0.000484240466914824\\
94	0.000484240466915081\\
95	0.000484240466915342\\
96	0.00048424046691561\\
97	0.00048424046691588\\
98	0.000484240466916157\\
99	0.000484240466916438\\
100	0.000484240466916723\\
101	0.000484240466917015\\
102	0.000484240466917311\\
103	0.000484240466917613\\
104	0.00048424046691792\\
105	0.000484240466918233\\
106	0.000484240466918552\\
107	0.000484240466918875\\
108	0.000484240466919205\\
109	0.00048424046691954\\
110	0.000484240466919883\\
111	0.000484240466920231\\
112	0.000484240466920584\\
113	0.000484240466920946\\
114	0.000484240466921311\\
115	0.000484240466921684\\
116	0.000484240466922066\\
117	0.000484240466922452\\
118	0.000484240466922847\\
119	0.000484240466923247\\
120	0.000484240466923655\\
121	0.000484240466924071\\
122	0.000484240466924494\\
123	0.000484240466924925\\
124	0.000484240466925363\\
125	0.000484240466925809\\
126	0.000484240466926262\\
127	0.000484240466926725\\
128	0.000484240466927195\\
129	0.000484240466927674\\
130	0.000484240466928163\\
131	0.000484240466928658\\
132	0.000484240466929163\\
133	0.000484240466929677\\
134	0.000484240466930202\\
135	0.000484240466930734\\
136	0.000484240466931277\\
137	0.000484240466931828\\
138	0.000484240466932391\\
139	0.000484240466932962\\
140	0.000484240466933544\\
141	0.000484240466934137\\
142	0.00048424046693474\\
143	0.000484240466935354\\
144	0.00048424046693598\\
145	0.000484240466936615\\
146	0.000484240466937264\\
147	0.000484240466937923\\
148	0.000484240466938593\\
149	0.000484240466939276\\
150	0.000484240466939972\\
151	0.00048424046694068\\
152	0.000484240466941401\\
153	0.000484240466942133\\
154	0.00048424046694288\\
155	0.000484240466943641\\
156	0.000484240466944414\\
157	0.000484240466945202\\
158	0.000484240466946003\\
159	0.000484240466946819\\
160	0.000484240466947649\\
161	0.000484240466948494\\
162	0.000484240466949355\\
163	0.000484240466950231\\
164	0.000484240466951122\\
165	0.00048424046695203\\
166	0.000484240466952954\\
167	0.000484240466953896\\
168	0.000484240466954853\\
169	0.000484240466955828\\
170	0.000484240466956819\\
171	0.000484240466957829\\
172	0.000484240466958857\\
173	0.000484240466959902\\
174	0.000484240466960968\\
175	0.000484240466962053\\
176	0.000484240466963156\\
177	0.000484240466964281\\
178	0.000484240466965423\\
179	0.000484240466966587\\
180	0.000484240466967772\\
181	0.000484240466968979\\
182	0.000484240466970208\\
183	0.000484240466971458\\
184	0.00048424046697273\\
185	0.000484240466974026\\
186	0.000484240466975344\\
187	0.000484240466976687\\
188	0.000484240466978053\\
189	0.000484240466979444\\
190	0.00048424046698086\\
191	0.000484240466982302\\
192	0.000484240466983769\\
193	0.000484240466985264\\
194	0.000484240466986784\\
195	0.000484240466988332\\
196	0.000484240466989907\\
197	0.000484240466991513\\
198	0.000484240466993146\\
199	0.000484240466994807\\
200	0.000484240466996501\\
201	0.000484240466998223\\
202	0.000484240466999977\\
203	0.000484240467001762\\
204	0.00048424046700358\\
205	0.00048424046700543\\
206	0.000484240467007315\\
207	0.000484240467009232\\
208	0.000484240467011185\\
209	0.000484240467013172\\
210	0.000484240467015197\\
211	0.000484240467017256\\
212	0.000484240467019352\\
213	0.000484240467021488\\
214	0.000484240467023662\\
215	0.000484240467025873\\
216	0.000484240467028126\\
217	0.00048424046703042\\
218	0.000484240467032754\\
219	0.000484240467035132\\
220	0.000484240467037552\\
221	0.000484240467040016\\
222	0.000484240467042524\\
223	0.000484240467045077\\
224	0.000484240467047678\\
225	0.000484240467050325\\
226	0.00048424046705302\\
227	0.000484240467055764\\
228	0.000484240467058556\\
229	0.0004842404670614\\
230	0.000484240467064296\\
231	0.000484240467067244\\
232	0.000484240467070247\\
233	0.000484240467073302\\
234	0.000484240467076414\\
235	0.000484240467079583\\
236	0.000484240467082807\\
237	0.000484240467086091\\
238	0.000484240467089436\\
239	0.00048424046709284\\
240	0.000484240467096307\\
241	0.000484240467099838\\
242	0.000484240467103431\\
243	0.000484240467107091\\
244	0.000484240467110817\\
245	0.000484240467114611\\
246	0.000484240467118475\\
247	0.000484240467122408\\
248	0.000484240467126413\\
249	0.000484240467130492\\
250	0.000484240467134645\\
251	0.000484240467138874\\
252	0.00048424046714318\\
253	0.000484240467147566\\
254	0.00048424046715203\\
255	0.000484240467156577\\
256	0.000484240467161208\\
257	0.000484240467165922\\
258	0.000484240467170724\\
259	0.000484240467175612\\
260	0.000484240467180592\\
261	0.000484240467185661\\
262	0.000484240467190824\\
263	0.000484240467196083\\
264	0.000484240467201438\\
265	0.00048424046720689\\
266	0.000484240467212444\\
267	0.000484240467218099\\
268	0.000484240467223858\\
269	0.000484240467229725\\
270	0.000484240467235697\\
271	0.000484240467241782\\
272	0.000484240467247977\\
273	0.000484240467254287\\
274	0.000484240467260715\\
275	0.000484240467267261\\
276	0.000484240467273926\\
277	0.000484240467280716\\
278	0.000484240467287631\\
279	0.000484240467294674\\
280	0.000484240467301846\\
281	0.000484240467309154\\
282	0.000484240467316596\\
283	0.000484240467324175\\
284	0.000484240467331895\\
285	0.000484240467339758\\
286	0.000484240467347768\\
287	0.000484240467355925\\
288	0.000484240467364235\\
289	0.000484240467372699\\
290	0.000484240467381321\\
291	0.000484240467390103\\
292	0.000484240467399049\\
293	0.000484240467408161\\
294	0.000484240467417442\\
295	0.000484240467426897\\
296	0.000484240467436528\\
297	0.000484240467446339\\
298	0.000484240467456333\\
299	0.000484240467466513\\
300	0.000484240467476884\\
301	0.000484240467487449\\
302	0.00048424046749821\\
303	0.000484240467509173\\
304	0.00048424046752034\\
305	0.000484240467531717\\
306	0.000484240467543307\\
307	0.000484240467555115\\
308	0.000484240467567143\\
309	0.000484240467579396\\
310	0.000484240467591879\\
311	0.000484240467604594\\
312	0.000484240467617549\\
313	0.000484240467630747\\
314	0.000484240467644192\\
315	0.00048424046765789\\
316	0.000484240467671845\\
317	0.00048424046768606\\
318	0.000484240467700543\\
319	0.000484240467715297\\
320	0.000484240467730329\\
321	0.000484240467745641\\
322	0.000484240467761242\\
323	0.000484240467777135\\
324	0.000484240467793326\\
325	0.000484240467809823\\
326	0.000484240467826628\\
327	0.000484240467843747\\
328	0.000484240467861189\\
329	0.000484240467878957\\
330	0.000484240467897058\\
331	0.0004842404679155\\
332	0.000484240467934286\\
333	0.000484240467953425\\
334	0.000484240467972922\\
335	0.000484240467992787\\
336	0.000484240468013021\\
337	0.000484240468033635\\
338	0.000484240468054636\\
339	0.000484240468076028\\
340	0.000484240468097822\\
341	0.000484240468120023\\
342	0.000484240468142638\\
343	0.000484240468165678\\
344	0.000484240468189149\\
345	0.000484240468213057\\
346	0.000484240468237412\\
347	0.000484240468262221\\
348	0.000484240468287495\\
349	0.000484240468313239\\
350	0.000484240468339464\\
351	0.000484240468366179\\
352	0.000484240468393391\\
353	0.000484240468421111\\
354	0.000484240468449347\\
355	0.00048424046847811\\
356	0.000484240468507407\\
357	0.000484240468537252\\
358	0.000484240468567653\\
359	0.000484240468598618\\
360	0.000484240468630161\\
361	0.000484240468662293\\
362	0.000484240468695021\\
363	0.000484240468728362\\
364	0.000484240468762321\\
365	0.000484240468796915\\
366	0.000484240468832154\\
367	0.00048424046886805\\
368	0.000484240468904617\\
369	0.000484240468941867\\
370	0.000484240468979814\\
371	0.000484240469018471\\
372	0.000484240469057852\\
373	0.000484240469097974\\
374	0.000484240469138849\\
375	0.000484240469180494\\
376	0.000484240469222925\\
377	0.00048424046926616\\
378	0.000484240469310213\\
379	0.000484240469355105\\
380	0.000484240469400852\\
381	0.000484240469447475\\
382	0.000484240469494994\\
383	0.000484240469543428\\
384	0.000484240469592801\\
385	0.000484240469643136\\
386	0.000484240469694458\\
387	0.000484240469746791\\
388	0.000484240469800161\\
389	0.000484240469854599\\
390	0.000484240469910134\\
391	0.000484240469966797\\
392	0.00048424047002462\\
393	0.00048424047008364\\
394	0.000484240470143896\\
395	0.000484240470205425\\
396	0.000484240470268271\\
397	0.000484240470332477\\
398	0.000484240470398088\\
399	0.000484240470465156\\
400	0.000484240470533727\\
401	0.000484240470603858\\
402	0.000484240470675606\\
403	0.00048424047074903\\
404	0.000484240470824197\\
405	0.000484240470901179\\
406	0.000484240470980048\\
407	0.000484240471060881\\
408	0.000484240471143761\\
409	0.000484240471228782\\
410	0.000484240471316043\\
411	0.000484240471405652\\
412	0.000484240471497727\\
413	0.000484240471592404\\
414	0.000484240471689826\\
415	0.000484240471790159\\
416	0.000484240471893589\\
417	0.000484240472000334\\
418	0.00048424047211065\\
419	0.000484240472224868\\
420	0.00048424047234344\\
421	0.000484240472467045\\
422	0.000484240472596777\\
423	0.000484240472734483\\
424	0.000484240472883296\\
425	0.000484240473048316\\
426	0.000484240473237145\\
427	0.000484240473459371\\
428	0.000484240473723402\\
429	0.000484240474029596\\
430	0.000484240474363645\\
431	0.000484240474704589\\
432	0.000484240475052085\\
433	0.000484240475405585\\
434	0.000484240475764482\\
435	0.000484240476128623\\
436	0.000484240476499017\\
437	0.000484240476877848\\
438	0.000484240477266483\\
439	0.000484240477665547\\
440	0.000484240478076063\\
441	0.000484240478499937\\
442	0.000484240478941073\\
443	0.000484240479407776\\
444	0.00048424047991768\\
445	0.000484240480507274\\
446	0.000484240481248529\\
447	0.000484240482273253\\
448	0.000484240483796312\\
449	0.000484240486103035\\
450	0.0004842404894243\\
451	0.000484240493635419\\
452	0.000484240498062313\\
453	0.000484240502573705\\
454	0.000484240507169916\\
455	0.000484240511850491\\
456	0.000484240516613272\\
457	0.000484240521452686\\
458	0.000484240526357104\\
459	0.000484240531306371\\
460	0.000484240536273769\\
461	0.000484240541240519\\
462	0.000484240546228037\\
463	0.000484240551311235\\
464	0.000484240556492951\\
465	0.000484240561776135\\
466	0.000484240567163889\\
467	0.000484240572659478\\
468	0.000484240578266315\\
469	0.000484240583988024\\
470	0.000484240589828514\\
471	0.000484240595792018\\
472	0.000484240601882934\\
473	0.000484240608105622\\
474	0.000484240614464737\\
475	0.000484240620965255\\
476	0.000484240627612496\\
477	0.000484240634412163\\
478	0.000484240641370363\\
479	0.000484240648493667\\
480	0.000484240655789144\\
481	0.000484240663264423\\
482	0.000484240670927752\\
483	0.000484240678788071\\
484	0.000484240686855094\\
485	0.00048424069513941\\
486	0.000484240703652612\\
487	0.000484240712407421\\
488	0.000484240721417891\\
489	0.000484240730699638\\
490	0.000484240740270217\\
491	0.000484240750149749\\
492	0.00048424076036209\\
493	0.000484240770937182\\
494	0.000484240781915921\\
495	0.00048424079336023\\
496	0.000484240805373368\\
497	0.000484240818138671\\
498	0.000484240831986899\\
499	0.000484240847495088\\
500	0.000484240865582196\\
501	0.000484240887462403\\
502	0.00048424091413874\\
503	0.000484240945128151\\
504	0.000484240977435586\\
505	0.000484241010360868\\
506	0.000484241043948233\\
507	0.000484241078241602\\
508	0.000484241113276388\\
509	0.000484241149094674\\
510	0.000484241185749511\\
511	0.000484241223317212\\
512	0.000484241261926852\\
513	0.000484241301830343\\
514	0.000484241343566185\\
515	0.000484241388329228\\
516	0.000484241438764512\\
517	0.000484241500542035\\
518	0.000484241585077113\\
519	0.000484241712968724\\
520	0.000484241914290421\\
521	0.000484242213187589\\
522	0.000484242580237095\\
523	0.000484242952253886\\
524	0.000484243336263891\\
525	0.000484243733975993\\
526	0.000484244147872262\\
527	0.000484244581210334\\
528	0.000484245037161741\\
529	0.000484245516419022\\
530	0.000484246016790978\\
531	0.000484246545701493\\
532	0.000484247119218224\\
533	0.000484247767406508\\
534	0.000484248541306853\\
535	0.000484249512882124\\
536	0.000484250731440822\\
537	0.000484252078883766\\
538	0.000484253458944545\\
539	0.000484254874471593\\
540	0.000484256328862317\\
541	0.000484257826378404\\
542	0.000484259373029604\\
543	0.000484260979213294\\
544	0.00048426266514276\\
545	0.000484264468980179\\
546	0.000484266455953145\\
547	0.000484268725433269\\
548	0.00048427138412627\\
549	0.000484274421921086\\
550	0.000484277630733646\\
551	0.000484281192722443\\
552	0.000484285569045597\\
553	0.000484291838485129\\
554	0.000484302055719857\\
555	0.000484318648444575\\
556	0.00048433717149597\\
557	0.000484358312049961\\
558	0.000484384433837813\\
559	0.000484419741214524\\
560	0.000484467972308084\\
561	0.000484518384319466\\
562	0.000484571833803974\\
563	0.00048463043821816\\
564	0.000484698204292027\\
565	0.000484780473870265\\
566	0.000484872232915441\\
567	0.000484965833244364\\
568	0.000485062002730881\\
569	0.00048516191722124\\
570	0.000485267787354841\\
571	0.000485383204690684\\
572	0.000485511922586414\\
573	0.000485642971766619\\
574	0.000485775812999213\\
575	0.000485914470174083\\
576	0.000486084229404065\\
577	0.000486346843596949\\
578	0.000486874060292007\\
579	0.000499332124294703\\
580	0.000549606817473229\\
581	0.00088240546943094\\
582	0.00124000461749179\\
583	0.00162646793862394\\
584	0.0020440485140991\\
585	0.00247787686417299\\
586	0.00292892988002479\\
587	0.00339798959748976\\
588	0.00388566154600113\\
589	0.00439214851677062\\
590	0.00491474520087469\\
591	0.00545443621534336\\
592	0.00601233132542992\\
593	0.00658977217077654\\
594	0.00718858827196951\\
595	0.00781118959413535\\
596	0.00845712341615153\\
597	0.00912599392411371\\
598	0.00981478197180822\\
599	0\\
600	0\\
};
\addplot [color=mycolor14,solid,forget plot]
  table[row sep=crcr]{%
1	2.808615381101e-05\\
2	2.80861538121573e-05\\
3	2.8086153813325e-05\\
4	2.80861538145132e-05\\
5	2.80861538157218e-05\\
6	2.80861538169526e-05\\
7	2.80861538182056e-05\\
8	2.80861538194807e-05\\
9	2.80861538207797e-05\\
10	2.80861538221009e-05\\
11	2.80861538234459e-05\\
12	2.8086153824813e-05\\
13	2.80861538262075e-05\\
14	2.80861538276258e-05\\
15	2.80861538290697e-05\\
16	2.80861538305392e-05\\
17	2.80861538320342e-05\\
18	2.80861538335565e-05\\
19	2.80861538351061e-05\\
20	2.80861538366846e-05\\
21	2.80861538382905e-05\\
22	2.80861538399236e-05\\
23	2.80861538415874e-05\\
24	2.80861538432801e-05\\
25	2.80861538450036e-05\\
26	2.80861538467577e-05\\
27	2.80861538485443e-05\\
28	2.80861538503615e-05\\
29	2.80861538522111e-05\\
30	2.80861538540948e-05\\
31	2.80861538560109e-05\\
32	2.80861538579611e-05\\
33	2.80861538599471e-05\\
34	2.80861538619689e-05\\
35	2.80861538640265e-05\\
36	2.80861538661198e-05\\
37	2.80861538682524e-05\\
38	2.80861538704208e-05\\
39	2.80861538726301e-05\\
40	2.80861538748769e-05\\
41	2.80861538771646e-05\\
42	2.80861538794933e-05\\
43	2.80861538818645e-05\\
44	2.80861538842767e-05\\
45	2.80861538867332e-05\\
46	2.80861538892323e-05\\
47	2.80861538917774e-05\\
48	2.80861538943669e-05\\
49	2.80861538970023e-05\\
50	2.80861538996855e-05\\
51	2.80861539024165e-05\\
52	2.80861539051969e-05\\
53	2.8086153908025e-05\\
54	2.80861539109059e-05\\
55	2.80861539138363e-05\\
56	2.80861539168213e-05\\
57	2.80861539198574e-05\\
58	2.8086153922948e-05\\
59	2.80861539260949e-05\\
60	2.80861539292981e-05\\
61	2.80861539325574e-05\\
62	2.80861539358748e-05\\
63	2.80861539392518e-05\\
64	2.80861539426902e-05\\
65	2.80861539461883e-05\\
66	2.80861539497494e-05\\
67	2.80861539533754e-05\\
68	2.80861539570643e-05\\
69	2.80861539608198e-05\\
70	2.80861539646418e-05\\
71	2.80861539685319e-05\\
72	2.80861539724919e-05\\
73	2.80861539765236e-05\\
74	2.80861539806268e-05\\
75	2.80861539848016e-05\\
76	2.80861539890532e-05\\
77	2.80861539933798e-05\\
78	2.8086153997783e-05\\
79	2.80861540022647e-05\\
80	2.80861540068265e-05\\
81	2.80861540114701e-05\\
82	2.80861540161973e-05\\
83	2.8086154021008e-05\\
84	2.80861540259039e-05\\
85	2.80861540308885e-05\\
86	2.80861540359617e-05\\
87	2.80861540411252e-05\\
88	2.80861540463809e-05\\
89	2.80861540517302e-05\\
90	2.80861540571751e-05\\
91	2.80861540627171e-05\\
92	2.8086154068358e-05\\
93	2.80861540740994e-05\\
94	2.80861540799432e-05\\
95	2.80861540858909e-05\\
96	2.8086154091946e-05\\
97	2.80861540981085e-05\\
98	2.80861541043802e-05\\
99	2.80861541107643e-05\\
100	2.80861541172626e-05\\
101	2.80861541238769e-05\\
102	2.80861541306088e-05\\
103	2.80861541374617e-05\\
104	2.80861541444357e-05\\
105	2.80861541515341e-05\\
106	2.80861541587604e-05\\
107	2.80861541661145e-05\\
108	2.80861541735999e-05\\
109	2.80861541812199e-05\\
110	2.80861541889746e-05\\
111	2.80861541968674e-05\\
112	2.80861542049017e-05\\
113	2.80861542130792e-05\\
114	2.80861542214033e-05\\
115	2.80861542298757e-05\\
116	2.80861542384981e-05\\
117	2.80861542472757e-05\\
118	2.80861542562083e-05\\
119	2.8086154265303e-05\\
120	2.80861542745578e-05\\
121	2.80861542839781e-05\\
122	2.80861542935653e-05\\
123	2.80861543033248e-05\\
124	2.80861543132598e-05\\
125	2.80861543233705e-05\\
126	2.80861543336618e-05\\
127	2.80861543441355e-05\\
128	2.80861543547967e-05\\
129	2.80861543656489e-05\\
130	2.80861543766955e-05\\
131	2.8086154387938e-05\\
132	2.808615439938e-05\\
133	2.80861544110283e-05\\
134	2.80861544228828e-05\\
135	2.80861544349505e-05\\
136	2.80861544472329e-05\\
137	2.80861544597335e-05\\
138	2.80861544724575e-05\\
139	2.80861544854098e-05\\
140	2.80861544985924e-05\\
141	2.80861545120101e-05\\
142	2.80861545256665e-05\\
143	2.80861545395684e-05\\
144	2.80861545537175e-05\\
145	2.80861545681189e-05\\
146	2.80861545827777e-05\\
147	2.80861545976973e-05\\
148	2.80861546128846e-05\\
149	2.80861546283429e-05\\
150	2.80861546440756e-05\\
151	2.80861546600896e-05\\
152	2.80861546763918e-05\\
153	2.8086154692982e-05\\
154	2.80861547098706e-05\\
155	2.80861547270591e-05\\
156	2.80861547445563e-05\\
157	2.80861547623654e-05\\
158	2.80861547804915e-05\\
159	2.80861547989416e-05\\
160	2.80861548177224e-05\\
161	2.80861548368372e-05\\
162	2.80861548562948e-05\\
163	2.80861548761001e-05\\
164	2.80861548962583e-05\\
165	2.80861549167762e-05\\
166	2.80861549376605e-05\\
167	2.808615495892e-05\\
168	2.80861549805578e-05\\
169	2.80861550025827e-05\\
170	2.80861550250013e-05\\
171	2.80861550478205e-05\\
172	2.80861550710472e-05\\
173	2.80861550946898e-05\\
174	2.80861551187552e-05\\
175	2.80861551432501e-05\\
176	2.80861551681832e-05\\
177	2.80861551935629e-05\\
178	2.80861552193961e-05\\
179	2.80861552456912e-05\\
180	2.80861552724568e-05\\
181	2.80861552997015e-05\\
182	2.80861553274337e-05\\
183	2.8086155355662e-05\\
184	2.80861553843966e-05\\
185	2.80861554136442e-05\\
186	2.80861554434153e-05\\
187	2.80861554737199e-05\\
188	2.80861555045666e-05\\
189	2.80861555359673e-05\\
190	2.80861555679289e-05\\
191	2.80861556004633e-05\\
192	2.80861556335807e-05\\
193	2.80861556672913e-05\\
194	2.80861557016054e-05\\
195	2.80861557365348e-05\\
196	2.80861557720917e-05\\
197	2.80861558082843e-05\\
198	2.80861558451265e-05\\
199	2.80861558826301e-05\\
200	2.80861559208071e-05\\
201	2.80861559596676e-05\\
202	2.80861559992254e-05\\
203	2.80861560394923e-05\\
204	2.80861560804838e-05\\
205	2.80861561222099e-05\\
206	2.80861561646862e-05\\
207	2.80861562079244e-05\\
208	2.808615625194e-05\\
209	2.80861562967465e-05\\
210	2.80861563423577e-05\\
211	2.80861563887889e-05\\
212	2.80861564360554e-05\\
213	2.80861564841725e-05\\
214	2.80861565331522e-05\\
215	2.80861565830149e-05\\
216	2.80861566337761e-05\\
217	2.80861566854492e-05\\
218	2.80861567380514e-05\\
219	2.80861567916031e-05\\
220	2.8086156846118e-05\\
221	2.80861569016166e-05\\
222	2.8086156958114e-05\\
223	2.80861570156292e-05\\
224	2.80861570741826e-05\\
225	2.80861571337911e-05\\
226	2.80861571944754e-05\\
227	2.8086157256254e-05\\
228	2.80861573191475e-05\\
229	2.80861573831781e-05\\
230	2.80861574483645e-05\\
231	2.80861575147271e-05\\
232	2.80861575822899e-05\\
233	2.80861576510749e-05\\
234	2.80861577211026e-05\\
235	2.8086157792397e-05\\
236	2.80861578649818e-05\\
237	2.80861579388793e-05\\
238	2.80861580141149e-05\\
239	2.80861580907143e-05\\
240	2.80861581686996e-05\\
241	2.80861582480999e-05\\
242	2.80861583289389e-05\\
243	2.8086158411244e-05\\
244	2.80861584950424e-05\\
245	2.80861585803614e-05\\
246	2.808615866723e-05\\
247	2.80861587556754e-05\\
248	2.80861588457284e-05\\
249	2.80861589374196e-05\\
250	2.80861590307763e-05\\
251	2.80861591258309e-05\\
252	2.80861592226158e-05\\
253	2.80861593211634e-05\\
254	2.80861594215043e-05\\
255	2.80861595236743e-05\\
256	2.80861596277059e-05\\
257	2.80861597336349e-05\\
258	2.80861598414953e-05\\
259	2.80861599513246e-05\\
260	2.80861600631604e-05\\
261	2.80861601770367e-05\\
262	2.80861602929962e-05\\
263	2.80861604110746e-05\\
264	2.80861605313111e-05\\
265	2.80861606537502e-05\\
266	2.80861607784293e-05\\
267	2.8086160905391e-05\\
268	2.80861610346796e-05\\
269	2.80861611663394e-05\\
270	2.80861613004132e-05\\
271	2.8086161436945e-05\\
272	2.80861615759846e-05\\
273	2.80861617175778e-05\\
274	2.80861618617707e-05\\
275	2.80861620086161e-05\\
276	2.80861621581618e-05\\
277	2.80861623104571e-05\\
278	2.80861624655585e-05\\
279	2.80861626235135e-05\\
280	2.80861627843802e-05\\
281	2.8086162948213e-05\\
282	2.80861631150649e-05\\
283	2.80861632849972e-05\\
284	2.80861634580662e-05\\
285	2.80861636343298e-05\\
286	2.80861638138511e-05\\
287	2.80861639966915e-05\\
288	2.80861641829124e-05\\
289	2.80861643725784e-05\\
290	2.80861645657562e-05\\
291	2.80861647625104e-05\\
292	2.80861649629093e-05\\
293	2.80861651670228e-05\\
294	2.80861653749207e-05\\
295	2.80861655866747e-05\\
296	2.80861658023598e-05\\
297	2.80861660220475e-05\\
298	2.80861662458163e-05\\
299	2.80861664737411e-05\\
300	2.80861667059039e-05\\
301	2.80861669423831e-05\\
302	2.80861671832603e-05\\
303	2.8086167428621e-05\\
304	2.80861676785469e-05\\
305	2.80861679331266e-05\\
306	2.80861681924488e-05\\
307	2.80861684566022e-05\\
308	2.80861687256771e-05\\
309	2.80861689997707e-05\\
310	2.80861692789733e-05\\
311	2.8086169563382e-05\\
312	2.80861698530975e-05\\
313	2.80861701482204e-05\\
314	2.80861704488495e-05\\
315	2.80861707550905e-05\\
316	2.80861710670491e-05\\
317	2.80861713848309e-05\\
318	2.80861717085486e-05\\
319	2.80861720383129e-05\\
320	2.80861723742346e-05\\
321	2.8086172716433e-05\\
322	2.80861730650224e-05\\
323	2.80861734201238e-05\\
324	2.80861737818582e-05\\
325	2.80861741503501e-05\\
326	2.80861745257256e-05\\
327	2.80861749081126e-05\\
328	2.80861752976423e-05\\
329	2.80861756944477e-05\\
330	2.808617609866e-05\\
331	2.80861765104226e-05\\
332	2.80861769298699e-05\\
333	2.8086177357147e-05\\
334	2.80861777923987e-05\\
335	2.80861782357716e-05\\
336	2.80861786874158e-05\\
337	2.8086179147483e-05\\
338	2.80861796161282e-05\\
339	2.80861800935101e-05\\
340	2.80861805797872e-05\\
341	2.80861810751247e-05\\
342	2.80861815796899e-05\\
343	2.8086182093648e-05\\
344	2.80861826171747e-05\\
345	2.80861831504437e-05\\
346	2.80861836936324e-05\\
347	2.8086184246925e-05\\
348	2.80861848105038e-05\\
349	2.80861853845562e-05\\
350	2.80861859692751e-05\\
351	2.80861865648563e-05\\
352	2.8086187171496e-05\\
353	2.80861877893987e-05\\
354	2.8086188418769e-05\\
355	2.80861890598182e-05\\
356	2.80861897127595e-05\\
357	2.80861903778127e-05\\
358	2.80861910551978e-05\\
359	2.80861917451431e-05\\
360	2.80861924478824e-05\\
361	2.80861931636507e-05\\
362	2.80861938926884e-05\\
363	2.80861946352462e-05\\
364	2.80861953915747e-05\\
365	2.80861961619311e-05\\
366	2.80861969465816e-05\\
367	2.80861977457971e-05\\
368	2.80861985598556e-05\\
369	2.80861993890399e-05\\
370	2.80862002336433e-05\\
371	2.80862010939642e-05\\
372	2.80862019703127e-05\\
373	2.80862028630044e-05\\
374	2.80862037723647e-05\\
375	2.80862046987312e-05\\
376	2.808620564245e-05\\
377	2.80862066038774e-05\\
378	2.80862075833849e-05\\
379	2.80862085813546e-05\\
380	2.80862095981818e-05\\
381	2.80862106342774e-05\\
382	2.8086211690071e-05\\
383	2.80862127660042e-05\\
384	2.80862138625407e-05\\
385	2.80862149801645e-05\\
386	2.80862161193821e-05\\
387	2.8086217280727e-05\\
388	2.80862184647584e-05\\
389	2.80862196720662e-05\\
390	2.80862209032726e-05\\
391	2.80862221590288e-05\\
392	2.80862234400219e-05\\
393	2.80862247469712e-05\\
394	2.80862260806475e-05\\
395	2.80862274418707e-05\\
396	2.80862288315067e-05\\
397	2.80862302504661e-05\\
398	2.80862316997153e-05\\
399	2.80862331802619e-05\\
400	2.80862346931698e-05\\
401	2.80862362395503e-05\\
402	2.80862378205697e-05\\
403	2.80862394374588e-05\\
404	2.8086241091532e-05\\
405	2.80862427842111e-05\\
406	2.80862445170288e-05\\
407	2.80862462915791e-05\\
408	2.80862481095242e-05\\
409	2.80862499726541e-05\\
410	2.80862518829092e-05\\
411	2.80862538424e-05\\
412	2.80862558534264e-05\\
413	2.8086257918505e-05\\
414	2.80862600403959e-05\\
415	2.80862622221511e-05\\
416	2.80862644671482e-05\\
417	2.80862667791621e-05\\
418	2.80862691624525e-05\\
419	2.80862716219223e-05\\
420	2.80862741634101e-05\\
421	2.80862767942748e-05\\
422	2.80862795246284e-05\\
423	2.80862823699214e-05\\
424	2.8086285356212e-05\\
425	2.8086288530267e-05\\
426	2.80862919770322e-05\\
427	2.80862958444663e-05\\
428	2.80863003639551e-05\\
429	2.80863058241207e-05\\
430	2.80863124129024e-05\\
431	2.80863198919898e-05\\
432	2.80863275459902e-05\\
433	2.80863353806416e-05\\
434	2.80863434019857e-05\\
435	2.80863516163779e-05\\
436	2.80863600305164e-05\\
437	2.80863686514935e-05\\
438	2.80863774868853e-05\\
439	2.8086386544978e-05\\
440	2.80863958353203e-05\\
441	2.80864053701281e-05\\
442	2.80864151678357e-05\\
443	2.80864252619863e-05\\
444	2.80864357231286e-05\\
445	2.8086446711237e-05\\
446	2.80864585968064e-05\\
447	2.80864722254694e-05\\
448	2.80864894474804e-05\\
449	2.80865140184409e-05\\
450	2.80865526116246e-05\\
451	2.80866141754087e-05\\
452	2.80867023239014e-05\\
453	2.80867965835178e-05\\
454	2.80868926758453e-05\\
455	2.80869906251254e-05\\
456	2.80870904593219e-05\\
457	2.80871922118918e-05\\
458	2.8087295922761e-05\\
459	2.80874016360984e-05\\
460	2.8087509391849e-05\\
461	2.80876192130461e-05\\
462	2.808773110846e-05\\
463	2.80878451184784e-05\\
464	2.80879613047944e-05\\
465	2.80880797319544e-05\\
466	2.80882004670643e-05\\
467	2.80883235804694e-05\\
468	2.80884491466583e-05\\
469	2.80885772429902e-05\\
470	2.80887079505498e-05\\
471	2.80888413562767e-05\\
472	2.80889775553677e-05\\
473	2.80891166493581e-05\\
474	2.80892587363316e-05\\
475	2.80894039207413e-05\\
476	2.80895523139946e-05\\
477	2.80897040350871e-05\\
478	2.8089859211319e-05\\
479	2.80900179791046e-05\\
480	2.80901804848662e-05\\
481	2.80903468860667e-05\\
482	2.80905173523679e-05\\
483	2.8090692066955e-05\\
484	2.80908712280712e-05\\
485	2.80910550508182e-05\\
486	2.80912437692606e-05\\
487	2.80914376388795e-05\\
488	2.80916369395449e-05\\
489	2.80918419790192e-05\\
490	2.80920530972932e-05\\
491	2.80922706722094e-05\\
492	2.80924951278461e-05\\
493	2.80927269487464e-05\\
494	2.80929667067045e-05\\
495	2.80932151172259e-05\\
496	2.80934731650202e-05\\
497	2.80937423869787e-05\\
498	2.80940254996891e-05\\
499	2.80943277284858e-05\\
500	2.80946593912327e-05\\
501	2.8095040150302e-05\\
502	2.80955034844796e-05\\
503	2.80960927022307e-05\\
504	2.80968231164462e-05\\
505	2.80975931042638e-05\\
506	2.80983776764648e-05\\
507	2.80991778365475e-05\\
508	2.80999946916829e-05\\
509	2.81008290390136e-05\\
510	2.81016817629114e-05\\
511	2.81025538510951e-05\\
512	2.8103446463201e-05\\
513	2.81043611010734e-05\\
514	2.81053000457361e-05\\
515	2.81062675292016e-05\\
516	2.81072728711379e-05\\
517	2.81083387599474e-05\\
518	2.81095225156801e-05\\
519	2.81109679733623e-05\\
520	2.81130200084217e-05\\
521	2.81164217137343e-05\\
522	2.8122372809078e-05\\
523	2.81307328451471e-05\\
524	2.81393374669202e-05\\
525	2.8148210089866e-05\\
526	2.81573853040879e-05\\
527	2.81669176615914e-05\\
528	2.81768901519144e-05\\
529	2.81874046473829e-05\\
530	2.81985071666067e-05\\
531	2.82100322058695e-05\\
532	2.82220592867267e-05\\
533	2.82348190650448e-05\\
534	2.82488249099873e-05\\
535	2.82651675017933e-05\\
536	2.82860473335031e-05\\
537	2.83145323942167e-05\\
538	2.83472442182761e-05\\
539	2.8380749358613e-05\\
540	2.84151133427996e-05\\
541	2.84504107378315e-05\\
542	2.8486726090546e-05\\
543	2.85241574117138e-05\\
544	2.85628504144372e-05\\
545	2.8603106863314e-05\\
546	2.86455858784958e-05\\
547	2.86916048197475e-05\\
548	2.87437943798303e-05\\
549	2.88064993319595e-05\\
550	2.88810108054738e-05\\
551	2.89574258384307e-05\\
552	2.90366136841843e-05\\
553	2.9121818143783e-05\\
554	2.92262971949298e-05\\
555	2.93947309519714e-05\\
556	3.11587393730253e-05\\
557	3.31624962695448e-05\\
558	3.52871923224828e-05\\
559	3.75823663834745e-05\\
560	4.1712988773412e-05\\
561	5.80622218823199e-05\\
562	7.5201694526181e-05\\
563	9.32340626589157e-05\\
564	0.000112306196431779\\
565	0.000132645131255909\\
566	0.00030164397021752\\
567	0.000607768803823587\\
568	0.000930822380175926\\
569	0.00127311620720435\\
570	0.00163736940748546\\
571	0.00202680492261195\\
572	0.00244524712103414\\
573	0.00289722057015775\\
574	0.0033697985617141\\
575	0.00386159633509131\\
576	0.00437431846692315\\
577	0.00491016184443514\\
578	0.00547110656601669\\
579	0.00604483533067061\\
580	0.00659986535839362\\
581	0.00686411121575563\\
582	0.00712289222269977\\
583	0.007367743343057\\
584	0.00759482125548611\\
585	0.00782073949169679\\
586	0.00804446638930281\\
587	0.00826483417025616\\
588	0.00848064034958728\\
589	0.00869044844652259\\
590	0.00889569740473372\\
591	0.00909380142313226\\
592	0.00928132122564964\\
593	0.00945516715155288\\
594	0.00961172798547802\\
595	0.00974741425137617\\
596	0.00986072654667908\\
597	0.00994757959173143\\
598	0.0099999191923403\\
599	0\\
600	0\\
};
\addplot [color=mycolor15,solid,forget plot]
  table[row sep=crcr]{%
1	2.90931793024638e-05\\
2	2.9093179327587e-05\\
3	2.90931793531601e-05\\
4	2.90931793791901e-05\\
5	2.90931794056854e-05\\
6	2.90931794326563e-05\\
7	2.90931794601061e-05\\
8	2.90931794880486e-05\\
9	2.90931795164905e-05\\
10	2.90931795454421e-05\\
11	2.90931795749101e-05\\
12	2.90931796049031e-05\\
13	2.90931796354348e-05\\
14	2.90931796665119e-05\\
15	2.9093179698143e-05\\
16	2.909317973034e-05\\
17	2.90931797631131e-05\\
18	2.90931797964726e-05\\
19	2.90931798304269e-05\\
20	2.90931798649881e-05\\
21	2.90931799001681e-05\\
22	2.90931799359752e-05\\
23	2.90931799724233e-05\\
24	2.90931800095226e-05\\
25	2.90931800472849e-05\\
26	2.90931800857222e-05\\
27	2.90931801248464e-05\\
28	2.90931801646695e-05\\
29	2.90931802052034e-05\\
30	2.90931802464617e-05\\
31	2.9093180288458e-05\\
32	2.90931803312044e-05\\
33	2.90931803747144e-05\\
34	2.90931804190017e-05\\
35	2.90931804640798e-05\\
36	2.90931805099642e-05\\
37	2.90931805566668e-05\\
38	2.90931806042046e-05\\
39	2.90931806525913e-05\\
40	2.90931807018439e-05\\
41	2.90931807519743e-05\\
42	2.90931808030013e-05\\
43	2.90931808549385e-05\\
44	2.90931809078047e-05\\
45	2.90931809616153e-05\\
46	2.90931810163854e-05\\
47	2.90931810721357e-05\\
48	2.90931811288797e-05\\
49	2.90931811866379e-05\\
50	2.90931812454274e-05\\
51	2.90931813052686e-05\\
52	2.90931813661768e-05\\
53	2.90931814281724e-05\\
54	2.90931814912744e-05\\
55	2.90931815555047e-05\\
56	2.90931816208822e-05\\
57	2.90931816874256e-05\\
58	2.9093181755157e-05\\
59	2.90931818240987e-05\\
60	2.90931818942728e-05\\
61	2.9093181965698e-05\\
62	2.90931820383982e-05\\
63	2.90931821123955e-05\\
64	2.90931821877155e-05\\
65	2.90931822643805e-05\\
66	2.90931823424124e-05\\
67	2.90931824218387e-05\\
68	2.90931825026832e-05\\
69	2.90931825849697e-05\\
70	2.90931826687255e-05\\
71	2.90931827539762e-05\\
72	2.9093182840749e-05\\
73	2.90931829290696e-05\\
74	2.90931830189685e-05\\
75	2.90931831104697e-05\\
76	2.90931832036056e-05\\
77	2.90931832984034e-05\\
78	2.90931833948938e-05\\
79	2.90931834931058e-05\\
80	2.909318359307e-05\\
81	2.90931836948189e-05\\
82	2.90931837983831e-05\\
83	2.90931839037968e-05\\
84	2.90931840110905e-05\\
85	2.90931841202984e-05\\
86	2.90931842314564e-05\\
87	2.90931843445984e-05\\
88	2.90931844597586e-05\\
89	2.90931845769727e-05\\
90	2.909318469628e-05\\
91	2.90931848177163e-05\\
92	2.90931849413191e-05\\
93	2.90931850671258e-05\\
94	2.90931851951791e-05\\
95	2.90931853255164e-05\\
96	2.90931854581805e-05\\
97	2.90931855932105e-05\\
98	2.90931857306489e-05\\
99	2.9093185870542e-05\\
100	2.90931860129287e-05\\
101	2.90931861578569e-05\\
102	2.90931863053709e-05\\
103	2.90931864555167e-05\\
104	2.90931866083403e-05\\
105	2.90931867638929e-05\\
106	2.90931869222187e-05\\
107	2.9093187083369e-05\\
108	2.90931872473948e-05\\
109	2.90931874143473e-05\\
110	2.90931875842777e-05\\
111	2.90931877572404e-05\\
112	2.909318793329e-05\\
113	2.90931881124794e-05\\
114	2.90931882948648e-05\\
115	2.90931884805041e-05\\
116	2.90931886694554e-05\\
117	2.90931888617782e-05\\
118	2.90931890575323e-05\\
119	2.9093189256779e-05\\
120	2.90931894595796e-05\\
121	2.9093189665999e-05\\
122	2.90931898761001e-05\\
123	2.90931900899495e-05\\
124	2.90931903076154e-05\\
125	2.90931905291642e-05\\
126	2.90931907546642e-05\\
127	2.90931909841902e-05\\
128	2.90931912178088e-05\\
129	2.90931914555968e-05\\
130	2.90931916976273e-05\\
131	2.90931919439771e-05\\
132	2.90931921947213e-05\\
133	2.90931924499398e-05\\
134	2.90931927097111e-05\\
135	2.90931929741188e-05\\
136	2.90931932432446e-05\\
137	2.90931935171721e-05\\
138	2.90931937959883e-05\\
139	2.90931940797799e-05\\
140	2.90931943686358e-05\\
141	2.90931946626461e-05\\
142	2.90931949619031e-05\\
143	2.9093195266502e-05\\
144	2.90931955765367e-05\\
145	2.90931958921044e-05\\
146	2.90931962133055e-05\\
147	2.9093196540239e-05\\
148	2.90931968730071e-05\\
149	2.90931972117172e-05\\
150	2.90931975564716e-05\\
151	2.90931979073828e-05\\
152	2.90931982645565e-05\\
153	2.90931986281085e-05\\
154	2.90931989981497e-05\\
155	2.90931993747994e-05\\
156	2.90931997581735e-05\\
157	2.9093200148393e-05\\
158	2.90932005455807e-05\\
159	2.90932009498626e-05\\
160	2.9093201361365e-05\\
161	2.90932017802156e-05\\
162	2.90932022065474e-05\\
163	2.90932026404951e-05\\
164	2.9093203082195e-05\\
165	2.90932035317853e-05\\
166	2.9093203989409e-05\\
167	2.90932044552076e-05\\
168	2.90932049293313e-05\\
169	2.90932054119247e-05\\
170	2.90932059031448e-05\\
171	2.90932064031433e-05\\
172	2.90932069120787e-05\\
173	2.90932074301129e-05\\
174	2.90932079574061e-05\\
175	2.90932084941271e-05\\
176	2.90932090404447e-05\\
177	2.90932095965309e-05\\
178	2.90932101625632e-05\\
179	2.90932107387169e-05\\
180	2.9093211325178e-05\\
181	2.90932119221289e-05\\
182	2.90932125297603e-05\\
183	2.90932131482633e-05\\
184	2.90932137778338e-05\\
185	2.90932144186731e-05\\
186	2.90932150709804e-05\\
187	2.90932157349655e-05\\
188	2.90932164108381e-05\\
189	2.90932170988112e-05\\
190	2.90932177991047e-05\\
191	2.90932185119401e-05\\
192	2.90932192375442e-05\\
193	2.90932199761454e-05\\
194	2.90932207279824e-05\\
195	2.90932214932903e-05\\
196	2.90932222723164e-05\\
197	2.90932230653042e-05\\
198	2.90932238725096e-05\\
199	2.90932246941883e-05\\
200	2.90932255306027e-05\\
201	2.90932263820186e-05\\
202	2.9093227248709e-05\\
203	2.90932281309498e-05\\
204	2.90932290290242e-05\\
205	2.90932299432148e-05\\
206	2.90932308738202e-05\\
207	2.90932318211317e-05\\
208	2.90932327854579e-05\\
209	2.9093233767104e-05\\
210	2.90932347663886e-05\\
211	2.9093235783627e-05\\
212	2.909323681915e-05\\
213	2.90932378732866e-05\\
214	2.90932389463793e-05\\
215	2.90932400387691e-05\\
216	2.90932411508105e-05\\
217	2.90932422828615e-05\\
218	2.90932434352834e-05\\
219	2.90932446084512e-05\\
220	2.90932458027433e-05\\
221	2.90932470185433e-05\\
222	2.90932482562432e-05\\
223	2.90932495162469e-05\\
224	2.90932507989602e-05\\
225	2.90932521047973e-05\\
226	2.90932534341825e-05\\
227	2.90932547875471e-05\\
228	2.90932561653293e-05\\
229	2.90932575679772e-05\\
230	2.90932589959478e-05\\
231	2.90932604497046e-05\\
232	2.90932619297197e-05\\
233	2.90932634364791e-05\\
234	2.90932649704702e-05\\
235	2.90932665321958e-05\\
236	2.90932681221656e-05\\
237	2.90932697409011e-05\\
238	2.90932713889309e-05\\
239	2.90932730667934e-05\\
240	2.9093274775041e-05\\
241	2.90932765142328e-05\\
242	2.90932782849433e-05\\
243	2.90932800877501e-05\\
244	2.90932819232483e-05\\
245	2.90932837920446e-05\\
246	2.90932856947527e-05\\
247	2.90932876320032e-05\\
248	2.90932896044337e-05\\
249	2.9093291612697e-05\\
250	2.90932936574579e-05\\
251	2.90932957393947e-05\\
252	2.90932978591995e-05\\
253	2.90933000175729e-05\\
254	2.90933022152342e-05\\
255	2.9093304452913e-05\\
256	2.90933067313574e-05\\
257	2.90933090513244e-05\\
258	2.90933114135895e-05\\
259	2.90933138189419e-05\\
260	2.90933162681879e-05\\
261	2.90933187621439e-05\\
262	2.90933213016469e-05\\
263	2.9093323887549e-05\\
264	2.90933265207198e-05\\
265	2.90933292020454e-05\\
266	2.90933319324242e-05\\
267	2.90933347127801e-05\\
268	2.90933375440488e-05\\
269	2.90933404271866e-05\\
270	2.90933433631668e-05\\
271	2.90933463529849e-05\\
272	2.90933493976515e-05\\
273	2.90933524981998e-05\\
274	2.90933556556813e-05\\
275	2.90933588711683e-05\\
276	2.9093362145755e-05\\
277	2.90933654805545e-05\\
278	2.90933688767037e-05\\
279	2.90933723353618e-05\\
280	2.90933758577065e-05\\
281	2.90933794449447e-05\\
282	2.90933830983003e-05\\
283	2.90933868190243e-05\\
284	2.90933906083918e-05\\
285	2.90933944677015e-05\\
286	2.90933983982781e-05\\
287	2.90934024014714e-05\\
288	2.90934064786569e-05\\
289	2.90934106312359e-05\\
290	2.90934148606385e-05\\
291	2.90934191683202e-05\\
292	2.90934235557693e-05\\
293	2.90934280244941e-05\\
294	2.90934325760407e-05\\
295	2.90934372119791e-05\\
296	2.90934419339097e-05\\
297	2.90934467434655e-05\\
298	2.90934516423083e-05\\
299	2.90934566321344e-05\\
300	2.9093461714672e-05\\
301	2.90934668916767e-05\\
302	2.90934721649436e-05\\
303	2.90934775362964e-05\\
304	2.90934830075999e-05\\
305	2.90934885807462e-05\\
306	2.90934942576664e-05\\
307	2.90935000403294e-05\\
308	2.90935059307362e-05\\
309	2.90935119309289e-05\\
310	2.90935180429871e-05\\
311	2.90935242690242e-05\\
312	2.90935306111966e-05\\
313	2.90935370717015e-05\\
314	2.90935436527717e-05\\
315	2.90935503566846e-05\\
316	2.90935571857566e-05\\
317	2.90935641423467e-05\\
318	2.90935712288566e-05\\
319	2.90935784477308e-05\\
320	2.90935858014594e-05\\
321	2.90935932925737e-05\\
322	2.90936009236545e-05\\
323	2.90936086973232e-05\\
324	2.90936166162525e-05\\
325	2.90936246831596e-05\\
326	2.90936329008091e-05\\
327	2.90936412720169e-05\\
328	2.90936497996484e-05\\
329	2.90936584866132e-05\\
330	2.90936673358806e-05\\
331	2.90936763504659e-05\\
332	2.90936855334373e-05\\
333	2.90936948879192e-05\\
334	2.90937044170888e-05\\
335	2.90937141241779e-05\\
336	2.90937240124763e-05\\
337	2.909373408533e-05\\
338	2.9093744346143e-05\\
339	2.90937547983787e-05\\
340	2.9093765445564e-05\\
341	2.90937762912835e-05\\
342	2.90937873391848e-05\\
343	2.90937985929839e-05\\
344	2.90938100564562e-05\\
345	2.9093821733449e-05\\
346	2.9093833627874e-05\\
347	2.90938457437149e-05\\
348	2.90938580850267e-05\\
349	2.90938706559326e-05\\
350	2.90938834606379e-05\\
351	2.90938965034156e-05\\
352	2.9093909788626e-05\\
353	2.90939233207027e-05\\
354	2.9093937104166e-05\\
355	2.90939511436199e-05\\
356	2.90939654437536e-05\\
357	2.90939800093517e-05\\
358	2.90939948452909e-05\\
359	2.909400995654e-05\\
360	2.90940253481715e-05\\
361	2.9094041025364e-05\\
362	2.90940569933997e-05\\
363	2.90940732576754e-05\\
364	2.90940898237051e-05\\
365	2.90941066971224e-05\\
366	2.90941238836903e-05\\
367	2.90941413893049e-05\\
368	2.90941592200037e-05\\
369	2.90941773819708e-05\\
370	2.90941958815436e-05\\
371	2.90942147252217e-05\\
372	2.90942339196784e-05\\
373	2.90942534717711e-05\\
374	2.90942733885413e-05\\
375	2.90942936772436e-05\\
376	2.90943143453424e-05\\
377	2.90943354005319e-05\\
378	2.90943568507555e-05\\
379	2.90943787042138e-05\\
380	2.90944009693871e-05\\
381	2.90944236550557e-05\\
382	2.90944467703202e-05\\
383	2.90944703246275e-05\\
384	2.9094494327796e-05\\
385	2.90945187900515e-05\\
386	2.9094543722063e-05\\
387	2.90945691349873e-05\\
388	2.90945950405195e-05\\
389	2.90946214509568e-05\\
390	2.9094648379261e-05\\
391	2.9094675839103e-05\\
392	2.90947038448594e-05\\
393	2.90947324115871e-05\\
394	2.9094761555059e-05\\
395	2.90947912921322e-05\\
396	2.9094821640816e-05\\
397	2.90948526201269e-05\\
398	2.90948842501369e-05\\
399	2.90949165520221e-05\\
400	2.90949495480855e-05\\
401	2.9094983261762e-05\\
402	2.90950177176112e-05\\
403	2.90950529413199e-05\\
404	2.9095088959847e-05\\
405	2.90951258017984e-05\\
406	2.90951634980899e-05\\
407	2.90952020822798e-05\\
408	2.90952415895099e-05\\
409	2.90952820557793e-05\\
410	2.90953235198916e-05\\
411	2.90953660238076e-05\\
412	2.9095409613039e-05\\
413	2.90954543371108e-05\\
414	2.90955002500981e-05\\
415	2.90955474112519e-05\\
416	2.90955958857304e-05\\
417	2.90956457454964e-05\\
418	2.90956970703915e-05\\
419	2.90957499495869e-05\\
420	2.90958044836598e-05\\
421	2.90958607880629e-05\\
422	2.90959189997387e-05\\
423	2.90959792911682e-05\\
424	2.90960419019107e-05\\
425	2.9096107210368e-05\\
426	2.90961758938051e-05\\
427	2.90962492668808e-05\\
428	2.90963299303535e-05\\
429	2.90964227872562e-05\\
430	2.90965358853274e-05\\
431	2.90966784828055e-05\\
432	2.90968501766455e-05\\
433	2.90970258792738e-05\\
434	2.90972057215506e-05\\
435	2.90973898409771e-05\\
436	2.90975783821051e-05\\
437	2.90977714969531e-05\\
438	2.90979693455173e-05\\
439	2.9098172096481e-05\\
440	2.90983799285325e-05\\
441	2.90985930334534e-05\\
442	2.90988116241307e-05\\
443	2.90990359564183e-05\\
444	2.9099266389285e-05\\
445	2.9099503550225e-05\\
446	2.90997487864562e-05\\
447	2.91000053791739e-05\\
448	2.91002817408142e-05\\
449	2.91005995366793e-05\\
450	2.91010130430726e-05\\
451	2.9101649662234e-05\\
452	2.91027682805687e-05\\
453	2.91047020870107e-05\\
454	2.91068190737854e-05\\
455	2.91089775935504e-05\\
456	2.91111781752481e-05\\
457	2.91134214173341e-05\\
458	2.91157080402251e-05\\
459	2.9118038945718e-05\\
460	2.91204152310904e-05\\
461	2.91228380245918e-05\\
462	2.91253079910998e-05\\
463	2.91278249038759e-05\\
464	2.91303894186689e-05\\
465	2.91330029185693e-05\\
466	2.91356668570307e-05\\
467	2.91383827440669e-05\\
468	2.91411521584032e-05\\
469	2.91439767852071e-05\\
470	2.91468583702275e-05\\
471	2.91497987253e-05\\
472	2.91527997675997e-05\\
473	2.91558636058589e-05\\
474	2.91589925727011e-05\\
475	2.91621888539417e-05\\
476	2.91654547756575e-05\\
477	2.91687928170124e-05\\
478	2.91722056241405e-05\\
479	2.91756960256899e-05\\
480	2.91792670502333e-05\\
481	2.91829219459032e-05\\
482	2.9186664202492e-05\\
483	2.91904975764392e-05\\
484	2.91944261191551e-05\\
485	2.91984542096033e-05\\
486	2.92025865924738e-05\\
487	2.92068284229079e-05\\
488	2.92111853178327e-05\\
489	2.92156634186546e-05\\
490	2.92202694634806e-05\\
491	2.92250108729358e-05\\
492	2.92298958493688e-05\\
493	2.92349335061599e-05\\
494	2.9240134058643e-05\\
495	2.92455091202278e-05\\
496	2.92510722606742e-05\\
497	2.92568402214188e-05\\
498	2.92628358171017e-05\\
499	2.92690951603317e-05\\
500	2.92756857450048e-05\\
501	2.92827506305816e-05\\
502	2.92906101766628e-05\\
503	2.92999676784904e-05\\
504	2.93121927531563e-05\\
505	2.93289709976244e-05\\
506	2.9346987131976e-05\\
507	2.93653466747014e-05\\
508	2.93840716575043e-05\\
509	2.94031900535272e-05\\
510	2.94227202101169e-05\\
511	2.94426823864227e-05\\
512	2.94630985001884e-05\\
513	2.9483992550852e-05\\
514	2.95053915277041e-05\\
515	2.95273272121389e-05\\
516	2.95498412891083e-05\\
517	2.95730008941682e-05\\
518	2.95969477534526e-05\\
519	2.96220575361834e-05\\
520	2.96494673391088e-05\\
521	2.96828537393044e-05\\
522	2.97345261078044e-05\\
523	3.01021053709156e-05\\
524	3.10823745284106e-05\\
525	3.20921802992323e-05\\
526	3.31334756897174e-05\\
527	3.42085687022956e-05\\
528	3.53203790258592e-05\\
529	3.64729006401969e-05\\
530	3.76717780476027e-05\\
531	3.89235042501472e-05\\
532	4.02246349365916e-05\\
533	4.15798328242147e-05\\
534	4.29965025234212e-05\\
535	4.44859984380255e-05\\
536	4.60685486942687e-05\\
537	4.77956073358063e-05\\
538	5.46164785367609e-05\\
539	6.53176031175611e-05\\
540	7.63862330248849e-05\\
541	8.78552200880326e-05\\
542	9.9762079827913e-05\\
543	0.000112149796931475\\
544	0.000125067565715275\\
545	0.000138572067831741\\
546	0.00015273059037222\\
547	0.000167626377004082\\
548	0.000183363584696547\\
549	0.000200089867442313\\
550	0.000259094817245577\\
551	0.000488015275408033\\
552	0.000726542020146932\\
553	0.000975629408224585\\
554	0.00123636589032539\\
555	0.00150999973544108\\
556	0.00179648290113427\\
557	0.00209887107636958\\
558	0.00241929322761919\\
559	0.00276003182189189\\
560	0.00312210450741178\\
561	0.00349818716082299\\
562	0.00390375972898349\\
563	0.00433457550024097\\
564	0.00478459859119485\\
565	0.00525596815664969\\
566	0.00560062013606284\\
567	0.00582740021304894\\
568	0.00605605899399815\\
569	0.00628510015539956\\
570	0.00651276544529824\\
571	0.00673634212359634\\
572	0.0069520380325136\\
573	0.0071547202173806\\
574	0.00735717499191341\\
575	0.00756064332953579\\
576	0.0077619994239701\\
577	0.00795658404551624\\
578	0.00814198173030071\\
579	0.00831546642819351\\
580	0.00847403078801572\\
581	0.00861655664322103\\
582	0.00875071443575307\\
583	0.00887968108874371\\
584	0.00900452361312009\\
585	0.00912444387945188\\
586	0.0092381478141785\\
587	0.00934511367152276\\
588	0.009443897894158\\
589	0.0095340619381977\\
590	0.00961580002830814\\
591	0.00968902107613021\\
592	0.00975458929242902\\
593	0.00981291410532457\\
594	0.00986442025097488\\
595	0.00990933460668555\\
596	0.00994779104554705\\
597	0.00997906286423442\\
598	0.0099999191923403\\
599	0\\
600	0\\
};
\addplot [color=mycolor16,solid,forget plot]
  table[row sep=crcr]{%
1	2.93043087117404e-05\\
2	2.93043092703793e-05\\
3	2.93043098390084e-05\\
4	2.93043104178049e-05\\
5	2.93043110069512e-05\\
6	2.93043116066297e-05\\
7	2.93043122170296e-05\\
8	2.93043128383435e-05\\
9	2.9304313470764e-05\\
10	2.93043141144905e-05\\
11	2.93043147697259e-05\\
12	2.93043154366748e-05\\
13	2.93043161155449e-05\\
14	2.93043168065513e-05\\
15	2.93043175099102e-05\\
16	2.93043182258399e-05\\
17	2.93043189545671e-05\\
18	2.93043196963202e-05\\
19	2.93043204513311e-05\\
20	2.93043212198348e-05\\
21	2.93043220020752e-05\\
22	2.93043227982943e-05\\
23	2.93043236087444e-05\\
24	2.9304324433676e-05\\
25	2.93043252733517e-05\\
26	2.93043261280323e-05\\
27	2.93043269979852e-05\\
28	2.93043278834833e-05\\
29	2.93043287848044e-05\\
30	2.93043297022315e-05\\
31	2.93043306360509e-05\\
32	2.9304331586554e-05\\
33	2.93043325540409e-05\\
34	2.93043335388133e-05\\
35	2.93043345411797e-05\\
36	2.93043355614537e-05\\
37	2.93043365999557e-05\\
38	2.93043376570097e-05\\
39	2.9304338732948e-05\\
40	2.93043398281064e-05\\
41	2.93043409428274e-05\\
42	2.93043420774606e-05\\
43	2.93043432323622e-05\\
44	2.93043444078918e-05\\
45	2.93043456044193e-05\\
46	2.93043468223162e-05\\
47	2.93043480619679e-05\\
48	2.93043493237578e-05\\
49	2.93043506080848e-05\\
50	2.93043519153494e-05\\
51	2.9304353245959e-05\\
52	2.93043546003329e-05\\
53	2.93043559788922e-05\\
54	2.93043573820698e-05\\
55	2.93043588103037e-05\\
56	2.93043602640422e-05\\
57	2.93043617437405e-05\\
58	2.93043632498588e-05\\
59	2.93043647828691e-05\\
60	2.93043663432522e-05\\
61	2.93043679314956e-05\\
62	2.9304369548097e-05\\
63	2.93043711935609e-05\\
64	2.93043728684021e-05\\
65	2.93043745731439e-05\\
66	2.93043763083197e-05\\
67	2.93043780744734e-05\\
68	2.93043798721572e-05\\
69	2.93043817019318e-05\\
70	2.930438356437e-05\\
71	2.93043854600548e-05\\
72	2.93043873895775e-05\\
73	2.93043893535434e-05\\
74	2.93043913525642e-05\\
75	2.93043933872674e-05\\
76	2.93043954582886e-05\\
77	2.93043975662756e-05\\
78	2.93043997118847e-05\\
79	2.9304401895789e-05\\
80	2.93044041186721e-05\\
81	2.93044063812278e-05\\
82	2.93044086841617e-05\\
83	2.93044110281965e-05\\
84	2.93044134140635e-05\\
85	2.93044158425092e-05\\
86	2.93044183142939e-05\\
87	2.93044208301863e-05\\
88	2.93044233909772e-05\\
89	2.93044259974645e-05\\
90	2.93044286504645e-05\\
91	2.9304431350804e-05\\
92	2.930443409933e-05\\
93	2.93044368969e-05\\
94	2.93044397443884e-05\\
95	2.93044426426867e-05\\
96	2.93044455926998e-05\\
97	2.93044485953501e-05\\
98	2.93044516515749e-05\\
99	2.93044547623304e-05\\
100	2.93044579285901e-05\\
101	2.93044611513407e-05\\
102	2.93044644315931e-05\\
103	2.93044677703717e-05\\
104	2.93044711687196e-05\\
105	2.93044746276987e-05\\
106	2.93044781483897e-05\\
107	2.93044817318955e-05\\
108	2.93044853793341e-05\\
109	2.93044890918477e-05\\
110	2.93044928705969e-05\\
111	2.93044967167614e-05\\
112	2.93045006315444e-05\\
113	2.93045046161699e-05\\
114	2.93045086718838e-05\\
115	2.93045127999542e-05\\
116	2.93045170016716e-05\\
117	2.93045212783518e-05\\
118	2.93045256313294e-05\\
119	2.93045300619681e-05\\
120	2.93045345716518e-05\\
121	2.9304539161792e-05\\
122	2.93045438338238e-05\\
123	2.93045485892098e-05\\
124	2.93045534294344e-05\\
125	2.93045583560166e-05\\
126	2.93045633704937e-05\\
127	2.93045684744355e-05\\
128	2.9304573669441e-05\\
129	2.93045789571328e-05\\
130	2.93045843391694e-05\\
131	2.93045898172315e-05\\
132	2.93045953930355e-05\\
133	2.93046010683268e-05\\
134	2.93046068448816e-05\\
135	2.93046127245083e-05\\
136	2.93046187090495e-05\\
137	2.93046248003767e-05\\
138	2.9304631000399e-05\\
139	2.93046373110579e-05\\
140	2.93046437343306e-05\\
141	2.93046502722301e-05\\
142	2.93046569268019e-05\\
143	2.93046637001356e-05\\
144	2.930467059435e-05\\
145	2.93046776116064e-05\\
146	2.9304684754107e-05\\
147	2.93046920240916e-05\\
148	2.93046994238357e-05\\
149	2.93047069556625e-05\\
150	2.93047146219363e-05\\
151	2.93047224250586e-05\\
152	2.93047303674788e-05\\
153	2.93047384516907e-05\\
154	2.93047466802305e-05\\
155	2.93047550556807e-05\\
156	2.93047635806729e-05\\
157	2.93047722578833e-05\\
158	2.93047810900391e-05\\
159	2.93047900799118e-05\\
160	2.93047992303276e-05\\
161	2.93048085441635e-05\\
162	2.93048180243463e-05\\
163	2.93048276738572e-05\\
164	2.93048374957319e-05\\
165	2.93048474930573e-05\\
166	2.93048576689833e-05\\
167	2.93048680267076e-05\\
168	2.93048785694943e-05\\
169	2.93048893006622e-05\\
170	2.93049002235912e-05\\
171	2.93049113417229e-05\\
172	2.93049226585599e-05\\
173	2.93049341776698e-05\\
174	2.93049459026849e-05\\
175	2.93049578373041e-05\\
176	2.93049699852926e-05\\
177	2.93049823504821e-05\\
178	2.93049949367794e-05\\
179	2.93050077481577e-05\\
180	2.93050207886653e-05\\
181	2.93050340624202e-05\\
182	2.93050475736206e-05\\
183	2.93050613265381e-05\\
184	2.93050753255261e-05\\
185	2.9305089575011e-05\\
186	2.93051040795048e-05\\
187	2.93051188436029e-05\\
188	2.93051338719807e-05\\
189	2.93051491694023e-05\\
190	2.93051647407171e-05\\
191	2.93051805908648e-05\\
192	2.93051967248771e-05\\
193	2.93052131478743e-05\\
194	2.93052298650723e-05\\
195	2.93052468817857e-05\\
196	2.93052642034196e-05\\
197	2.93052818354881e-05\\
198	2.93052997835972e-05\\
199	2.9305318053464e-05\\
200	2.93053366509077e-05\\
201	2.93053555818531e-05\\
202	2.93053748523359e-05\\
203	2.93053944685061e-05\\
204	2.93054144366209e-05\\
205	2.93054347630603e-05\\
206	2.93054554543167e-05\\
207	2.93054765170072e-05\\
208	2.93054979578662e-05\\
209	2.93055197837596e-05\\
210	2.93055420016758e-05\\
211	2.93055646187347e-05\\
212	2.93055876421871e-05\\
213	2.93056110794205e-05\\
214	2.93056349379604e-05\\
215	2.93056592254701e-05\\
216	2.93056839497564e-05\\
217	2.93057091187743e-05\\
218	2.93057347406236e-05\\
219	2.93057608235576e-05\\
220	2.93057873759846e-05\\
221	2.93058144064681e-05\\
222	2.93058419237318e-05\\
223	2.93058699366683e-05\\
224	2.93058984543284e-05\\
225	2.93059274859404e-05\\
226	2.9305957040903e-05\\
227	2.93059871287905e-05\\
228	2.93060177593578e-05\\
229	2.93060489425456e-05\\
230	2.93060806884822e-05\\
231	2.93061130074815e-05\\
232	2.93061459100605e-05\\
233	2.93061794069286e-05\\
234	2.93062135089981e-05\\
235	2.93062482273944e-05\\
236	2.93062835734457e-05\\
237	2.93063195586984e-05\\
238	2.93063561949187e-05\\
239	2.93063934940963e-05\\
240	2.93064314684473e-05\\
241	2.93064701304198e-05\\
242	2.93065094927023e-05\\
243	2.93065495682236e-05\\
244	2.9306590370156e-05\\
245	2.93066319119261e-05\\
246	2.93066742072177e-05\\
247	2.93067172699755e-05\\
248	2.930676111441e-05\\
249	2.93068057550044e-05\\
250	2.930685120652e-05\\
251	2.93068974840007e-05\\
252	2.93069446027787e-05\\
253	2.93069925784809e-05\\
254	2.93070414270343e-05\\
255	2.93070911646745e-05\\
256	2.93071418079455e-05\\
257	2.93071933737134e-05\\
258	2.93072458791667e-05\\
259	2.93072993418294e-05\\
260	2.9307353779558e-05\\
261	2.93074092105568e-05\\
262	2.93074656533828e-05\\
263	2.93075231269477e-05\\
264	2.93075816505345e-05\\
265	2.93076412437948e-05\\
266	2.93077019267599e-05\\
267	2.93077637198516e-05\\
268	2.93078266438872e-05\\
269	2.93078907200863e-05\\
270	2.93079559700773e-05\\
271	2.93080224159134e-05\\
272	2.93080900800703e-05\\
273	2.93081589854635e-05\\
274	2.93082291554465e-05\\
275	2.93083006138332e-05\\
276	2.93083733848958e-05\\
277	2.93084474933771e-05\\
278	2.93085229644989e-05\\
279	2.93085998239753e-05\\
280	2.93086780980152e-05\\
281	2.93087578133384e-05\\
282	2.93088389971796e-05\\
283	2.93089216773039e-05\\
284	2.93090058820111e-05\\
285	2.9309091640152e-05\\
286	2.93091789811311e-05\\
287	2.93092679349241e-05\\
288	2.93093585320843e-05\\
289	2.93094508037568e-05\\
290	2.93095447816828e-05\\
291	2.93096404982194e-05\\
292	2.93097379863436e-05\\
293	2.93098372796669e-05\\
294	2.93099384124448e-05\\
295	2.93100414195927e-05\\
296	2.93101463366906e-05\\
297	2.93102532000003e-05\\
298	2.93103620464757e-05\\
299	2.93104729137747e-05\\
300	2.93105858402726e-05\\
301	2.93107008650693e-05\\
302	2.93108180280113e-05\\
303	2.93109373696914e-05\\
304	2.93110589314767e-05\\
305	2.93111827555043e-05\\
306	2.93113088847092e-05\\
307	2.93114373628293e-05\\
308	2.93115682344172e-05\\
309	2.93117015448571e-05\\
310	2.93118373403806e-05\\
311	2.93119756680697e-05\\
312	2.93121165758844e-05\\
313	2.93122601126605e-05\\
314	2.93124063281393e-05\\
315	2.93125552729684e-05\\
316	2.93127069987215e-05\\
317	2.93128615579127e-05\\
318	2.93130190040092e-05\\
319	2.93131793914446e-05\\
320	2.93133427756377e-05\\
321	2.93135092130009e-05\\
322	2.93136787609609e-05\\
323	2.93138514779671e-05\\
324	2.93140274235173e-05\\
325	2.93142066581623e-05\\
326	2.93143892435288e-05\\
327	2.93145752423309e-05\\
328	2.93147647183921e-05\\
329	2.93149577366578e-05\\
330	2.9315154363217e-05\\
331	2.93153546653141e-05\\
332	2.93155587113752e-05\\
333	2.93157665710192e-05\\
334	2.9315978315082e-05\\
335	2.93161940156423e-05\\
336	2.93164137460263e-05\\
337	2.93166375808508e-05\\
338	2.93168655960277e-05\\
339	2.93170978687985e-05\\
340	2.93173344777514e-05\\
341	2.93175755028535e-05\\
342	2.93178210254676e-05\\
343	2.93180711283923e-05\\
344	2.93183258958727e-05\\
345	2.93185854136492e-05\\
346	2.93188497689687e-05\\
347	2.93191190506294e-05\\
348	2.93193933490078e-05\\
349	2.93196727560963e-05\\
350	2.93199573655373e-05\\
351	2.9320247272664e-05\\
352	2.93205425745397e-05\\
353	2.93208433700023e-05\\
354	2.93211497597115e-05\\
355	2.93214618461989e-05\\
356	2.93217797339203e-05\\
357	2.93221035293106e-05\\
358	2.93224333408535e-05\\
359	2.93227692791394e-05\\
360	2.93231114569356e-05\\
361	2.93234599892746e-05\\
362	2.93238149935226e-05\\
363	2.93241765894779e-05\\
364	2.93245448994635e-05\\
365	2.93249200484307e-05\\
366	2.93253021640719e-05\\
367	2.93256913769481e-05\\
368	2.93260878206101e-05\\
369	2.93264916317505e-05\\
370	2.9326902950348e-05\\
371	2.93273219198469e-05\\
372	2.9327748687329e-05\\
373	2.93281834037115e-05\\
374	2.93286262239667e-05\\
375	2.93290773073439e-05\\
376	2.93295368176299e-05\\
377	2.93300049234186e-05\\
378	2.93304817984106e-05\\
379	2.9330967621743e-05\\
380	2.93314625783413e-05\\
381	2.93319668593139e-05\\
382	2.93324806623867e-05\\
383	2.93330041923679e-05\\
384	2.93335376616909e-05\\
385	2.93340812909987e-05\\
386	2.93346353098155e-05\\
387	2.93351999573295e-05\\
388	2.93357754832963e-05\\
389	2.93363621491538e-05\\
390	2.9336960229354e-05\\
391	2.9337570012906e-05\\
392	2.93381918048516e-05\\
393	2.93388259268588e-05\\
394	2.9339472715923e-05\\
395	2.93401325229076e-05\\
396	2.93408057222732e-05\\
397	2.93414927158237e-05\\
398	2.93421939278154e-05\\
399	2.93429098063191e-05\\
400	2.9343640824428e-05\\
401	2.93443874811739e-05\\
402	2.93451503019373e-05\\
403	2.9345929838172e-05\\
404	2.93467266666941e-05\\
405	2.93475413900327e-05\\
406	2.93483746418173e-05\\
407	2.93492271025257e-05\\
408	2.93500995193379e-05\\
409	2.93509926816469e-05\\
410	2.93519073836872e-05\\
411	2.93528444802147e-05\\
412	2.93538048938463e-05\\
413	2.93547896235088e-05\\
414	2.93557997541904e-05\\
415	2.93568364682357e-05\\
416	2.93579010584453e-05\\
417	2.9358994943368e-05\\
418	2.93601196852197e-05\\
419	2.93612770111704e-05\\
420	2.93624688391245e-05\\
421	2.93636973104335e-05\\
422	2.93649648350278e-05\\
423	2.93662741629849e-05\\
424	2.93676285195353e-05\\
425	2.93690319025862e-05\\
426	2.93704898071103e-05\\
427	2.9372011068031e-05\\
428	2.93736125641227e-05\\
429	2.93753308720823e-05\\
430	2.93772491091177e-05\\
431	2.937954906549e-05\\
432	2.93825649121727e-05\\
433	2.93865647993176e-05\\
434	2.93906580715017e-05\\
435	2.93948477653989e-05\\
436	2.93991370711979e-05\\
437	2.94035293415676e-05\\
438	2.94080281007108e-05\\
439	2.94126370534127e-05\\
440	2.94173600941866e-05\\
441	2.94222013171684e-05\\
442	2.94271650294862e-05\\
443	2.94322557769839e-05\\
444	2.94374784106811e-05\\
445	2.94428382829006e-05\\
446	2.94483418539612e-05\\
447	2.94539986059908e-05\\
448	2.94598271691304e-05\\
449	2.9465875226017e-05\\
450	2.9472285212237e-05\\
451	2.94795147135854e-05\\
452	2.94890877018201e-05\\
453	2.95061946780885e-05\\
454	2.97137528650851e-05\\
455	2.99643177527541e-05\\
456	3.02198907130476e-05\\
457	3.04805329288714e-05\\
458	3.07463097120086e-05\\
459	3.10172934108331e-05\\
460	3.12935673858202e-05\\
461	3.15752306426986e-05\\
462	3.18623999270261e-05\\
463	3.21551991485585e-05\\
464	3.24537260405704e-05\\
465	3.2758119079863e-05\\
466	3.3068553554098e-05\\
467	3.33852152092756e-05\\
468	3.37083005507875e-05\\
469	3.40380176623664e-05\\
470	3.43745878109453e-05\\
471	3.47182432694875e-05\\
472	3.50692271046537e-05\\
473	3.54277931937303e-05\\
474	3.57942075380966e-05\\
475	3.61687510396599e-05\\
476	3.65517064958356e-05\\
477	3.69433747872977e-05\\
478	3.73440764709209e-05\\
479	3.77541535261776e-05\\
480	3.817397129973e-05\\
481	3.86039206799129e-05\\
482	3.90444205384928e-05\\
483	3.9495920483335e-05\\
484	3.99589039722751e-05\\
485	4.04338918478452e-05\\
486	4.09214463707476e-05\\
487	4.14221758634367e-05\\
488	4.19367400822611e-05\\
489	4.24658563771034e-05\\
490	4.30103069572825e-05\\
491	4.3570947296266e-05\\
492	4.41487160335469e-05\\
493	4.47446463784222e-05\\
494	4.53598795319635e-05\\
495	4.59956810291058e-05\\
496	4.66534600498183e-05\\
497	4.73347933366168e-05\\
498	4.80414565567155e-05\\
499	4.87754703249452e-05\\
500	4.9539181367542e-05\\
501	5.033544070285e-05\\
502	5.11680726456097e-05\\
503	5.20432562481787e-05\\
504	5.29738500783278e-05\\
505	5.39938867634016e-05\\
506	5.87673186764253e-05\\
507	6.4606815017149e-05\\
508	7.0590274701469e-05\\
509	7.67252049983633e-05\\
510	8.3019714326634e-05\\
511	8.9481993182314e-05\\
512	9.61209472369967e-05\\
513	0.000102946251576546\\
514	0.000109968443091134\\
515	0.000117199021355419\\
516	0.000124650541318546\\
517	0.000132336703579475\\
518	0.000140272419398281\\
519	0.000148473773838568\\
520	0.000156957625963504\\
521	0.000165739941223944\\
522	0.000174829676150907\\
523	0.000183950961874965\\
524	0.000192825752632742\\
525	0.000202059447372597\\
526	0.000211688618436699\\
527	0.000221755970778601\\
528	0.000232312204801424\\
529	0.000243420847026636\\
530	0.000255169619523371\\
531	0.000267702889893964\\
532	0.000412746538030032\\
533	0.000576195888250494\\
534	0.000745765199194563\\
535	0.000921788564508206\\
536	0.00110457820815963\\
537	0.0012943489400593\\
538	0.00148633721617432\\
539	0.00168191902455894\\
540	0.00188511074741496\\
541	0.00209666253199608\\
542	0.00231744598211593\\
543	0.00254847920196742\\
544	0.00279095733672176\\
545	0.00304628750794034\\
546	0.00331615021568388\\
547	0.0036026079092931\\
548	0.00390820819911889\\
549	0.00423590450669407\\
550	0.00454732922748713\\
551	0.00471530322788151\\
552	0.00489204389082255\\
553	0.00507209484026282\\
554	0.00525506600206778\\
555	0.00544042211683775\\
556	0.00562744957463521\\
557	0.00581514029179541\\
558	0.0060021940002762\\
559	0.00618708573928175\\
560	0.00636777611433338\\
561	0.00654145229614157\\
562	0.00670414805441769\\
563	0.00686023270116887\\
564	0.00701594208881994\\
565	0.00716944104891064\\
566	0.00731902377334951\\
567	0.00746707784675243\\
568	0.00761602253010009\\
569	0.00776477719611707\\
570	0.00791099048273329\\
571	0.00805137960956685\\
572	0.0081854601127217\\
573	0.00831318779882302\\
574	0.0084344350629761\\
575	0.00854874215821512\\
576	0.00865665372359101\\
577	0.00876045240599831\\
578	0.0088600806646344\\
579	0.00895569983895539\\
580	0.00904732770420615\\
581	0.00913419155627037\\
582	0.00921661474496478\\
583	0.00929486551170983\\
584	0.00936833250999321\\
585	0.00943643823988877\\
586	0.00949976142125213\\
587	0.00955824735515365\\
588	0.00961252025824673\\
589	0.00966324058718606\\
590	0.00971085603739681\\
591	0.00975602317557434\\
592	0.00979895351145169\\
593	0.00983980399718307\\
594	0.00987864202528269\\
595	0.00991535020799166\\
596	0.00994937493726701\\
597	0.00997906286423442\\
598	0.0099999191923403\\
599	0\\
600	0\\
};
\addplot [color=mycolor17,solid,forget plot]
  table[row sep=crcr]{%
1	4.85195228051361e-05\\
2	4.85195905472055e-05\\
3	4.85196595005982e-05\\
4	4.85197296869432e-05\\
5	4.85198011282528e-05\\
6	4.85198738469381e-05\\
7	4.85199478658032e-05\\
8	4.85200232080614e-05\\
9	4.85200998973416e-05\\
10	4.8520177957688e-05\\
11	4.85202574135813e-05\\
12	4.85203382899344e-05\\
13	4.85204206121018e-05\\
14	4.85205044058957e-05\\
15	4.85205896975854e-05\\
16	4.85206765139064e-05\\
17	4.8520764882073e-05\\
18	4.85208548297851e-05\\
19	4.85209463852382e-05\\
20	4.85210395771233e-05\\
21	4.85211344346477e-05\\
22	4.85212309875416e-05\\
23	4.85213292660615e-05\\
24	4.85214293010004e-05\\
25	4.85215311237068e-05\\
26	4.85216347660825e-05\\
27	4.85217402606018e-05\\
28	4.8521847640313e-05\\
29	4.85219569388589e-05\\
30	4.85220681904767e-05\\
31	4.85221814300167e-05\\
32	4.85222966929513e-05\\
33	4.85224140153793e-05\\
34	4.85225334340487e-05\\
35	4.85226549863615e-05\\
36	4.85227787103824e-05\\
37	4.85229046448591e-05\\
38	4.85230328292256e-05\\
39	4.85231633036214e-05\\
40	4.85232961088994e-05\\
41	4.85234312866416e-05\\
42	4.85235688791694e-05\\
43	4.85237089295637e-05\\
44	4.85238514816636e-05\\
45	4.85239965800966e-05\\
46	4.85241442702842e-05\\
47	4.85242945984534e-05\\
48	4.85244476116607e-05\\
49	4.85246033577975e-05\\
50	4.85247618856084e-05\\
51	4.85249232447049e-05\\
52	4.85250874855878e-05\\
53	4.85252546596556e-05\\
54	4.85254248192179e-05\\
55	4.85255980175213e-05\\
56	4.85257743087629e-05\\
57	4.85259537480985e-05\\
58	4.85261363916739e-05\\
59	4.85263222966311e-05\\
60	4.85265115211309e-05\\
61	4.85267041243712e-05\\
62	4.85269001666046e-05\\
63	4.85270997091599e-05\\
64	4.85273028144527e-05\\
65	4.85275095460144e-05\\
66	4.85277199685125e-05\\
67	4.85279341477591e-05\\
68	4.8528152150745e-05\\
69	4.85283740456517e-05\\
70	4.85285999018786e-05\\
71	4.852882979006e-05\\
72	4.85290637820906e-05\\
73	4.85293019511444e-05\\
74	4.85295443717018e-05\\
75	4.85297911195685e-05\\
76	4.85300422719043e-05\\
77	4.85302979072419e-05\\
78	4.85305581055176e-05\\
79	4.85308229480864e-05\\
80	4.85310925177616e-05\\
81	4.85313668988247e-05\\
82	4.85316461770697e-05\\
83	4.85319304398101e-05\\
84	4.85322197759229e-05\\
85	4.85325142758694e-05\\
86	4.85328140317205e-05\\
87	4.85331191371924e-05\\
88	4.85334296876743e-05\\
89	4.85337457802515e-05\\
90	4.85340675137454e-05\\
91	4.85343949887367e-05\\
92	4.85347283076049e-05\\
93	4.85350675745539e-05\\
94	4.8535412895644e-05\\
95	4.85357643788334e-05\\
96	4.85361221340029e-05\\
97	4.85364862729943e-05\\
98	4.85368569096487e-05\\
99	4.85372341598347e-05\\
100	4.85376181414916e-05\\
101	4.85380089746661e-05\\
102	4.85384067815443e-05\\
103	4.85388116864975e-05\\
104	4.85392238161132e-05\\
105	4.85396432992459e-05\\
106	4.85400702670465e-05\\
107	4.8540504853011e-05\\
108	4.85409471930187e-05\\
109	4.85413974253778e-05\\
110	4.85418556908663e-05\\
111	4.85423221327831e-05\\
112	4.85427968969806e-05\\
113	4.85432801319221e-05\\
114	4.85437719887268e-05\\
115	4.85442726212134e-05\\
116	4.854478218595e-05\\
117	4.85453008423083e-05\\
118	4.8545828752508e-05\\
119	4.85463660816713e-05\\
120	4.85469129978757e-05\\
121	4.85474696722051e-05\\
122	4.85480362788079e-05\\
123	4.85486129949458e-05\\
124	4.85492000010577e-05\\
125	4.85497974808099e-05\\
126	4.85504056211634e-05\\
127	4.85510246124225e-05\\
128	4.85516546483036e-05\\
129	4.85522959259959e-05\\
130	4.85529486462182e-05\\
131	4.85536130132935e-05\\
132	4.85542892352019e-05\\
133	4.8554977523657e-05\\
134	4.85556780941695e-05\\
135	4.85563911661148e-05\\
136	4.85571169628064e-05\\
137	4.85578557115676e-05\\
138	4.85586076437961e-05\\
139	4.85593729950541e-05\\
140	4.85601520051265e-05\\
141	4.85609449181111e-05\\
142	4.85617519824902e-05\\
143	4.85625734512119e-05\\
144	4.85634095817727e-05\\
145	4.85642606363002e-05\\
146	4.85651268816352e-05\\
147	4.85660085894219e-05\\
148	4.8566906036195e-05\\
149	4.85678195034628e-05\\
150	4.85687492778094e-05\\
151	4.85696956509753e-05\\
152	4.85706589199607e-05\\
153	4.85716393871228e-05\\
154	4.85726373602646e-05\\
155	4.85736531527433e-05\\
156	4.85746870835716e-05\\
157	4.85757394775177e-05\\
158	4.85768106652127e-05\\
159	4.85779009832611e-05\\
160	4.85790107743469e-05\\
161	4.8580140387351e-05\\
162	4.85812901774596e-05\\
163	4.85824605062826e-05\\
164	4.85836517419741e-05\\
165	4.85848642593533e-05\\
166	4.85860984400237e-05\\
167	4.85873546725047e-05\\
168	4.85886333523554e-05\\
169	4.8589934882306e-05\\
170	4.85912596723923e-05\\
171	4.85926081400904e-05\\
172	4.85939807104564e-05\\
173	4.85953778162622e-05\\
174	4.85967998981494e-05\\
175	4.85982474047637e-05\\
176	4.85997207929129e-05\\
177	4.86012205277192e-05\\
178	4.86027470827684e-05\\
179	4.86043009402722e-05\\
180	4.86058825912316e-05\\
181	4.86074925355984e-05\\
182	4.86091312824445e-05\\
183	4.86107993501316e-05\\
184	4.86124972664885e-05\\
185	4.86142255689852e-05\\
186	4.86159848049198e-05\\
187	4.86177755315993e-05\\
188	4.86195983165268e-05\\
189	4.86214537376046e-05\\
190	4.86233423833176e-05\\
191	4.86252648529382e-05\\
192	4.8627221756737e-05\\
193	4.8629213716181e-05\\
194	4.86312413641528e-05\\
195	4.8633305345169e-05\\
196	4.86354063155928e-05\\
197	4.86375449438675e-05\\
198	4.8639721910745e-05\\
199	4.86419379095203e-05\\
200	4.8644193646269e-05\\
201	4.86464898400934e-05\\
202	4.86488272233737e-05\\
203	4.86512065420214e-05\\
204	4.86536285557313e-05\\
205	4.86560940382629e-05\\
206	4.86586037776941e-05\\
207	4.86611585767051e-05\\
208	4.86637592528588e-05\\
209	4.86664066388893e-05\\
210	4.86691015829889e-05\\
211	4.86718449491142e-05\\
212	4.86746376172859e-05\\
213	4.8677480483906e-05\\
214	4.86803744620675e-05\\
215	4.86833204818831e-05\\
216	4.86863194908192e-05\\
217	4.86893724540281e-05\\
218	4.86924803546937e-05\\
219	4.86956441943839e-05\\
220	4.86988649934139e-05\\
221	4.87021437912051e-05\\
222	4.87054816466617e-05\\
223	4.87088796385576e-05\\
224	4.87123388659159e-05\\
225	4.87158604484162e-05\\
226	4.87194455267965e-05\\
227	4.87230952632638e-05\\
228	4.87268108419249e-05\\
229	4.87305934692125e-05\\
230	4.87344443743226e-05\\
231	4.87383648096746e-05\\
232	4.87423560513576e-05\\
233	4.87464193996125e-05\\
234	4.87505561792985e-05\\
235	4.87547677403838e-05\\
236	4.87590554584498e-05\\
237	4.87634207351953e-05\\
238	4.87678649989508e-05\\
239	4.87723897052186e-05\\
240	4.87769963372058e-05\\
241	4.87816864063762e-05\\
242	4.87864614530179e-05\\
243	4.87913230468166e-05\\
244	4.87962727874388e-05\\
245	4.88013123051327e-05\\
246	4.88064432613451e-05\\
247	4.88116673493306e-05\\
248	4.88169862948016e-05\\
249	4.88224018565728e-05\\
250	4.8827915827221e-05\\
251	4.88335300337648e-05\\
252	4.88392463383471e-05\\
253	4.8845066638951e-05\\
254	4.88509928701049e-05\\
255	4.88570270036215e-05\\
256	4.88631710493406e-05\\
257	4.88694270558925e-05\\
258	4.88757971114794e-05\\
259	4.88822833446537e-05\\
260	4.88888879251433e-05\\
261	4.88956130646577e-05\\
262	4.89024610177392e-05\\
263	4.89094340826172e-05\\
264	4.89165346020737e-05\\
265	4.89237649643376e-05\\
266	4.89311276039891e-05\\
267	4.89386250028848e-05\\
268	4.89462596910926e-05\\
269	4.89540342478559e-05\\
270	4.89619513025609e-05\\
271	4.89700135357422e-05\\
272	4.89782236800782e-05\\
273	4.89865845214382e-05\\
274	4.89950988999169e-05\\
275	4.90037697109065e-05\\
276	4.90125999061813e-05\\
277	4.90215924950037e-05\\
278	4.90307505452409e-05\\
279	4.90400771845188e-05\\
280	4.9049575601366e-05\\
281	4.90592490464081e-05\\
282	4.90691008335636e-05\\
283	4.9079134341253e-05\\
284	4.90893530136492e-05\\
285	4.90997603619277e-05\\
286	4.9110359965539e-05\\
287	4.91211554735117e-05\\
288	4.91321506057572e-05\\
289	4.91433491544102e-05\\
290	4.91547549851743e-05\\
291	4.91663720386933e-05\\
292	4.9178204331943e-05\\
293	4.91902559596341e-05\\
294	4.9202531095637e-05\\
295	4.92150339944272e-05\\
296	4.9227768992544e-05\\
297	4.92407405100684e-05\\
298	4.92539530521209e-05\\
299	4.92674112103702e-05\\
300	4.92811196645611e-05\\
301	4.92950831840634e-05\\
302	4.93093066294312e-05\\
303	4.93237949539777e-05\\
304	4.93385532053724e-05\\
305	4.93535865272458e-05\\
306	4.93689001608142e-05\\
307	4.93844994465169e-05\\
308	4.94003898256681e-05\\
309	4.941657684213e-05\\
310	4.94330661439883e-05\\
311	4.9449863485256e-05\\
312	4.94669747275817e-05\\
313	4.94844058419817e-05\\
314	4.95021629105794e-05\\
315	4.95202521283679e-05\\
316	4.9538679804987e-05\\
317	4.9557452366518e-05\\
318	4.95765763572965e-05\\
319	4.95960584417466e-05\\
320	4.96159054062375e-05\\
321	4.96361241609624e-05\\
322	4.9656721741848e-05\\
323	4.96777053124757e-05\\
324	4.96990821660607e-05\\
325	4.97208597274403e-05\\
326	4.97430455551105e-05\\
327	4.9765647343306e-05\\
328	4.97886729241193e-05\\
329	4.98121302696654e-05\\
330	4.98360274943216e-05\\
331	4.98603728569967e-05\\
332	4.98851747634897e-05\\
333	4.99104417689053e-05\\
334	4.99361825801509e-05\\
335	4.99624060585073e-05\\
336	4.9989121222297e-05\\
337	5.00163372496359e-05\\
338	5.00440634812877e-05\\
339	5.00723094236053e-05\\
340	5.01010847516042e-05\\
341	5.01303993121218e-05\\
342	5.01602631270978e-05\\
343	5.01906863969831e-05\\
344	5.02216795042634e-05\\
345	5.02532530171252e-05\\
346	5.02854176932574e-05\\
347	5.03181844838271e-05\\
348	5.03515645375995e-05\\
349	5.03855692052767e-05\\
350	5.04202100440215e-05\\
351	5.04554988222424e-05\\
352	5.04914475246349e-05\\
353	5.05280683575287e-05\\
354	5.05653737545819e-05\\
355	5.06033763828397e-05\\
356	5.06420891492121e-05\\
357	5.06815252073879e-05\\
358	5.07216979652491e-05\\
359	5.07626210928243e-05\\
360	5.08043085308115e-05\\
361	5.08467744997615e-05\\
362	5.08900335099431e-05\\
363	5.09341003719842e-05\\
364	5.09789902083251e-05\\
365	5.10247184655912e-05\\
366	5.10713009279394e-05\\
367	5.11187537314744e-05\\
368	5.11670933798314e-05\\
369	5.12163367610239e-05\\
370	5.12665011656658e-05\\
371	5.13176043066897e-05\\
372	5.13696643406997e-05\\
373	5.14226998910996e-05\\
374	5.14767300731561e-05\\
375	5.15317745211822e-05\\
376	5.15878534180362e-05\\
377	5.1644987527163e-05\\
378	5.17031982274314e-05\\
379	5.17625075510554e-05\\
380	5.18229382249418e-05\\
381	5.18845137158304e-05\\
382	5.19472582797052e-05\\
383	5.20111970159642e-05\\
384	5.20763559270271e-05\\
385	5.2142761984092e-05\\
386	5.22104431999679e-05\\
387	5.22794287101352e-05\\
388	5.23497488634706e-05\\
389	5.24214353245409e-05\\
390	5.24945211900758e-05\\
391	5.25690411231054e-05\\
392	5.26450315089373e-05\\
393	5.27225306351912e-05\\
394	5.28015788873556e-05\\
395	5.28822189220311e-05\\
396	5.29644957750493e-05\\
397	5.30484573727744e-05\\
398	5.31341547007723e-05\\
399	5.3221641628035e-05\\
400	5.33109751041869e-05\\
401	5.34022153600259e-05\\
402	5.34954261059868e-05\\
403	5.35906747198619e-05\\
404	5.36880324110592e-05\\
405	5.37875743466039e-05\\
406	5.38893797348472e-05\\
407	5.39935319156555e-05\\
408	5.41001186388268e-05\\
409	5.42092327617706e-05\\
410	5.4320971769912e-05\\
411	5.44354371495247e-05\\
412	5.45527380881619e-05\\
413	5.46729923971985e-05\\
414	5.47963275720986e-05\\
415	5.49228820134563e-05\\
416	5.50528064366033e-05\\
417	5.51862655029529e-05\\
418	5.53234397127437e-05\\
419	5.54645276068821e-05\\
420	5.56097483355481e-05\\
421	5.57593446642293e-05\\
422	5.5913586505215e-05\\
423	5.60727750903251e-05\\
424	5.6237247952476e-05\\
425	5.6407385006861e-05\\
426	5.65836163645633e-05\\
427	5.67664335585839e-05\\
428	5.69564092393166e-05\\
429	5.71542416273935e-05\\
430	5.73608781483743e-05\\
431	5.75779043230569e-05\\
432	5.78088449023424e-05\\
433	5.81528195365713e-05\\
434	5.944600411787e-05\\
435	6.07708016525669e-05\\
436	6.21282928284642e-05\\
437	6.35196126365089e-05\\
438	6.4945952967713e-05\\
439	6.64085650019074e-05\\
440	6.7908761246117e-05\\
441	6.94479170315175e-05\\
442	7.10274712130791e-05\\
443	7.26489257282991e-05\\
444	7.4313843546243e-05\\
445	7.60238443407499e-05\\
446	7.77805968532678e-05\\
447	7.95858060671009e-05\\
448	8.14411910082956e-05\\
449	8.33484419114667e-05\\
450	8.53091223231798e-05\\
451	8.7324403025868e-05\\
452	8.93942409883346e-05\\
453	9.15146481566733e-05\\
454	9.35035181938711e-05\\
455	9.55063189739746e-05\\
456	9.75619116404759e-05\\
457	9.96717437119328e-05\\
458	0.000101837158122837\\
459	0.000104059334890113\\
460	0.000106339220398147\\
461	0.000108677450218538\\
462	0.000111074278278899\\
463	0.000113529500974515\\
464	0.000116042112437598\\
465	0.000118607941916725\\
466	0.000121225878689891\\
467	0.000123897386685402\\
468	0.000126624013333797\\
469	0.000129407381553128\\
470	0.000132249194157849\\
471	0.000135151309566943\\
472	0.000138115684211088\\
473	0.000141144369207877\\
474	0.000144239505589599\\
475	0.000147403349520146\\
476	0.000150638493100372\\
477	0.000153947246360344\\
478	0.000157332063585502\\
479	0.000160795558711385\\
480	0.000164340522107563\\
481	0.000167969939549219\\
482	0.000171687013726675\\
483	0.00017549518871605\\
484	0.000179398177915557\\
485	0.000183399996029068\\
486	0.000187504995758643\\
487	0.000191717910133522\\
488	0.000196043902545426\\
489	0.00020048862747069\\
490	0.000205058300007985\\
491	0.000209759786026714\\
492	0.000214600706085218\\
493	0.000219589567095355\\
494	0.000224735913396954\\
495	0.000230050508790892\\
496	0.000235545583825447\\
497	0.000241235118228053\\
498	0.000247135186952318\\
499	0.00025326438473167\\
500	0.00025964434531677\\
501	0.000266300365657899\\
502	0.000273262109020108\\
503	0.000280564206036033\\
504	0.000288245964501537\\
505	0.000296341669277401\\
506	0.000308549128404662\\
507	0.000401301320030103\\
508	0.000496495979642583\\
509	0.00059424841886353\\
510	0.000694694016538126\\
511	0.000797978055842614\\
512	0.000904240215439014\\
513	0.00101363016259423\\
514	0.00112630667846547\\
515	0.00124243940537317\\
516	0.00136221071779773\\
517	0.00148581664666537\\
518	0.00161346774579206\\
519	0.0017453898123079\\
520	0.00188182432228271\\
521	0.00202302843785282\\
522	0.00216927474552162\\
523	0.00232085412302543\\
524	0.0024780776132213\\
525	0.00264125873975925\\
526	0.00281073686921825\\
527	0.00298714697226363\\
528	0.00317122809954563\\
529	0.00336384916330986\\
530	0.00356606087372385\\
531	0.00377911073125441\\
532	0.00387101444814595\\
533	0.00395496570181997\\
534	0.00404411861485679\\
535	0.0041393541562261\\
536	0.00424177554848268\\
537	0.0043527725456736\\
538	0.00447412533780575\\
539	0.00460796840269658\\
540	0.00474871272796527\\
541	0.00489195363004134\\
542	0.00503735684616973\\
543	0.00518445717205239\\
544	0.00533261860337901\\
545	0.00548098202337772\\
546	0.00562839635531924\\
547	0.00577331960216846\\
548	0.00591379893350599\\
549	0.00604742138754942\\
550	0.00617093021530702\\
551	0.00628179603343684\\
552	0.00639010919318062\\
553	0.00650075390556414\\
554	0.00661355826910363\\
555	0.00672831871640315\\
556	0.00684480418700696\\
557	0.00696276543345935\\
558	0.00708195602380279\\
559	0.00720220504478184\\
560	0.007323419518955\\
561	0.00744556801198611\\
562	0.00756874620780185\\
563	0.00769285619681815\\
564	0.00781677314052897\\
565	0.00793628932779018\\
566	0.00805088736053623\\
567	0.00816001629554882\\
568	0.00826313047911087\\
569	0.00835993460475097\\
570	0.00845129433017152\\
571	0.00853958906049813\\
572	0.00862495123378327\\
573	0.00870767807453262\\
574	0.00878809121736665\\
575	0.00886661500378345\\
576	0.00894286737083422\\
577	0.00901626113058295\\
578	0.0090866482592975\\
579	0.00915386945533433\\
580	0.00921808471957653\\
581	0.00927992511337708\\
582	0.00933845256431125\\
583	0.00939359549483182\\
584	0.0094456284309534\\
585	0.00949555685866655\\
586	0.00954333369779573\\
587	0.00958916916241128\\
588	0.00963370188398624\\
589	0.00967711629413547\\
590	0.00971958295016231\\
591	0.00976110413435579\\
592	0.00980161112056955\\
593	0.00984099058333643\\
594	0.00987905095617487\\
595	0.00991543381058343\\
596	0.00994937493726701\\
597	0.00997906286423442\\
598	0.0099999191923403\\
599	0\\
600	0\\
};
\addplot [color=mycolor18,solid,forget plot]
  table[row sep=crcr]{%
1	0.000169140775378202\\
2	0.000169141360870221\\
3	0.000169141956833748\\
4	0.000169142563455799\\
5	0.000169143180926731\\
6	0.000169143809440294\\
7	0.000169144449193693\\
8	0.00016914510038765\\
9	0.000169145763226465\\
10	0.000169146437918081\\
11	0.000169147124674152\\
12	0.0001691478237101\\
13	0.000169148535245194\\
14	0.000169149259502606\\
15	0.000169149996709493\\
16	0.000169150747097061\\
17	0.00016915151090063\\
18	0.000169152288359729\\
19	0.000169153079718147\\
20	0.000169153885224021\\
21	0.000169154705129912\\
22	0.000169155539692885\\
23	0.000169156389174587\\
24	0.000169157253841328\\
25	0.000169158133964165\\
26	0.000169159029818986\\
27	0.000169159941686601\\
28	0.00016916086985282\\
29	0.00016916181460855\\
30	0.000169162776249881\\
31	0.000169163755078186\\
32	0.000169164751400206\\
33	0.000169165765528146\\
34	0.000169166797779781\\
35	0.000169167848478545\\
36	0.000169168917953642\\
37	0.000169170006540139\\
38	0.000169171114579074\\
39	0.000169172242417566\\
40	0.000169173390408922\\
41	0.000169174558912741\\
42	0.000169175748295031\\
43	0.000169176958928325\\
44	0.000169178191191797\\
45	0.000169179445471374\\
46	0.000169180722159865\\
47	0.000169182021657074\\
48	0.000169183344369936\\
49	0.000169184690712634\\
50	0.000169186061106733\\
51	0.000169187455981313\\
52	0.000169188875773099\\
53	0.000169190320926595\\
54	0.000169191791894236\\
55	0.00016919328913651\\
56	0.00016919481312212\\
57	0.000169196364328115\\
58	0.00016919794324005\\
59	0.000169199550352132\\
60	0.000169201186167376\\
61	0.00016920285119776\\
62	0.00016920454596439\\
63	0.000169206270997654\\
64	0.000169208026837402\\
65	0.000169209814033093\\
66	0.000169211633143991\\
67	0.000169213484739319\\
68	0.000169215369398452\\
69	0.000169217287711087\\
70	0.000169219240277436\\
71	0.000169221227708407\\
72	0.000169223250625795\\
73	0.000169225309662487\\
74	0.000169227405462643\\
75	0.00016922953868191\\
76	0.000169231709987623\\
77	0.000169233920059015\\
78	0.000169236169587427\\
79	0.000169238459276524\\
80	0.00016924078984252\\
81	0.000169243162014397\\
82	0.000169245576534136\\
83	0.000169248034156946\\
84	0.000169250535651508\\
85	0.000169253081800205\\
86	0.000169255673399376\\
87	0.000169258311259563\\
88	0.000169260996205759\\
89	0.000169263729077678\\
90	0.000169266510730009\\
91	0.000169269342032687\\
92	0.000169272223871166\\
93	0.000169275157146695\\
94	0.000169278142776606\\
95	0.000169281181694597\\
96	0.000169284274851023\\
97	0.000169287423213203\\
98	0.000169290627765713\\
99	0.000169293889510708\\
100	0.000169297209468223\\
101	0.000169300588676507\\
102	0.000169304028192337\\
103	0.000169307529091361\\
104	0.000169311092468429\\
105	0.000169314719437943\\
106	0.000169318411134204\\
107	0.00016932216871177\\
108	0.000169325993345824\\
109	0.000169329886232535\\
110	0.000169333848589447\\
111	0.000169337881655853\\
112	0.000169341986693192\\
113	0.000169346164985444\\
114	0.000169350417839538\\
115	0.000169354746585762\\
116	0.000169359152578186\\
117	0.000169363637195088\\
118	0.000169368201839394\\
119	0.000169372847939113\\
120	0.000169377576947802\\
121	0.000169382390345014\\
122	0.000169387289636777\\
123	0.000169392276356063\\
124	0.000169397352063277\\
125	0.000169402518346758\\
126	0.000169407776823276\\
127	0.000169413129138547\\
128	0.00016941857696776\\
129	0.000169424122016103\\
130	0.000169429766019315\\
131	0.000169435510744231\\
132	0.000169441357989345\\
133	0.000169447309585389\\
134	0.000169453367395912\\
135	0.000169459533317877\\
136	0.000169465809282265\\
137	0.000169472197254695\\
138	0.000169478699236051\\
139	0.000169485317263119\\
140	0.000169492053409246\\
141	0.000169498909785002\\
142	0.000169505888538853\\
143	0.000169512991857853\\
144	0.000169520221968349\\
145	0.000169527581136695\\
146	0.000169535071669976\\
147	0.000169542695916759\\
148	0.000169550456267846\\
149	0.000169558355157043\\
150	0.000169566395061947\\
151	0.000169574578504747\\
152	0.000169582908053038\\
153	0.000169591386320653\\
154	0.000169600015968509\\
155	0.000169608799705471\\
156	0.000169617740289225\\
157	0.000169626840527181\\
158	0.00016963610327738\\
159	0.000169645531449434\\
160	0.000169655128005455\\
161	0.000169664895961041\\
162	0.000169674838386244\\
163	0.000169684958406587\\
164	0.000169695259204072\\
165	0.000169705744018236\\
166	0.000169716416147201\\
167	0.000169727278948765\\
168	0.000169738335841502\\
169	0.000169749590305881\\
170	0.000169761045885419\\
171	0.000169772706187853\\
172	0.000169784574886315\\
173	0.000169796655720564\\
174	0.000169808952498212\\
175	0.000169821469095988\\
176	0.00016983420946103\\
177	0.000169847177612184\\
178	0.000169860377641356\\
179	0.000169873813714861\\
180	0.000169887490074822\\
181	0.000169901411040579\\
182	0.000169915581010145\\
183	0.000169930004461665\\
184	0.000169944685954929\\
185	0.000169959630132898\\
186	0.00016997484172327\\
187	0.000169990325540069\\
188	0.000170006086485271\\
189	0.000170022129550458\\
190	0.000170038459818513\\
191	0.000170055082465341\\
192	0.000170072002761624\\
193	0.000170089226074614\\
194	0.000170106757869964\\
195	0.000170124603713592\\
196	0.000170142769273583\\
197	0.000170161260322129\\
198	0.00017018008273751\\
199	0.000170199242506112\\
200	0.000170218745724491\\
201	0.000170238598601468\\
202	0.000170258807460283\\
203	0.000170279378740776\\
204	0.000170300319001619\\
205	0.0001703216349226\\
206	0.00017034333330694\\
207	0.000170365421083671\\
208	0.00017038790531005\\
209	0.000170410793174038\\
210	0.000170434091996806\\
211	0.000170457809235325\\
212	0.000170481952484971\\
213	0.00017050652948222\\
214	0.000170531548107377\\
215	0.000170557016387361\\
216	0.000170582942498561\\
217	0.000170609334769738\\
218	0.000170636201684998\\
219	0.000170663551886816\\
220	0.000170691394179128\\
221	0.00017071973753049\\
222	0.000170748591077301\\
223	0.000170777964127089\\
224	0.000170807866161872\\
225	0.000170838306841585\\
226	0.00017086929600758\\
227	0.0001709008436862\\
228	0.000170932960092429\\
229	0.000170965655633617\\
230	0.000170998940913282\\
231	0.000171032826734991\\
232	0.000171067324106335\\
233	0.00017110244424297\\
234	0.000171138198572752\\
235	0.000171174598739968\\
236	0.000171211656609632\\
237	0.000171249384271903\\
238	0.000171287794046572\\
239	0.000171326898487654\\
240	0.000171366710388079\\
241	0.000171407242784477\\
242	0.00017144850896207\\
243	0.000171490522459664\\
244	0.000171533297074748\\
245	0.000171576846868694\\
246	0.000171621186172091\\
247	0.000171666329590157\\
248	0.000171712292008301\\
249	0.00017175908859777\\
250	0.000171806734821448\\
251	0.00017185524643975\\
252	0.000171904639516657\\
253	0.000171954930425878\\
254	0.000172006135857134\\
255	0.000172058272822577\\
256	0.000172111358663353\\
257	0.000172165411056294\\
258	0.000172220448020755\\
259	0.000172276487925589\\
260	0.000172333549496276\\
261	0.000172391651822199\\
262	0.000172450814364063\\
263	0.00017251105696149\\
264	0.000172572399840738\\
265	0.000172634863622617\\
266	0.000172698469330543\\
267	0.000172763238398767\\
268	0.000172829192680768\\
269	0.000172896354457827\\
270	0.000172964746447769\\
271	0.000173034391813874\\
272	0.00017310531417399\\
273	0.000173177537609807\\
274	0.000173251086676319\\
275	0.000173325986411495\\
276	0.000173402262346116\\
277	0.00017347994051381\\
278	0.000173559047461299\\
279	0.000173639610258825\\
280	0.000173721656510783\\
281	0.000173805214366563\\
282	0.000173890312531585\\
283	0.000173976980278556\\
284	0.000174065247458914\\
285	0.000174155144514509\\
286	0.000174246702489475\\
287	0.000174339953042325\\
288	0.000174434928458261\\
289	0.000174531661661694\\
290	0.000174630186228997\\
291	0.000174730536401446\\
292	0.000174832747098415\\
293	0.000174936853930769\\
294	0.000175042893214468\\
295	0.000175150901984418\\
296	0.000175260918008512\\
297	0.000175372979801904\\
298	0.0001754871266415\\
299	0.000175603398580655\\
300	0.000175721836464091\\
301	0.000175842481943035\\
302	0.000175965377490548\\
303	0.00017609056641709\\
304	0.000176218092886261\\
305	0.000176348001930775\\
306	0.000176480339468613\\
307	0.000176615152319388\\
308	0.000176752488220905\\
309	0.000176892395845903\\
310	0.000177034924819003\\
311	0.000177180125733827\\
312	0.000177328050170325\\
313	0.000177478750712254\\
314	0.000177632280964872\\
315	0.00017778869557278\\
316	0.000177948050237974\\
317	0.000178110401738062\\
318	0.000178275807944658\\
319	0.00017844432784198\\
320	0.000178616021545623\\
321	0.000178790950321544\\
322	0.000178969176605242\\
323	0.000179150764021163\\
324	0.000179335777402335\\
325	0.000179524282810263\\
326	0.000179716347555077\\
327	0.000179912040215995\\
328	0.00018011143066211\\
329	0.000180314590073516\\
330	0.000180521590962852\\
331	0.000180732507197275\\
332	0.000180947414020906\\
333	0.000181166388077831\\
334	0.000181389507435652\\
335	0.000181616851609706\\
336	0.000181848501587966\\
337	0.000182084539856667\\
338	0.000182325050426761\\
339	0.000182570118861175\\
340	0.00018281983230296\\
341	0.000183074279504315\\
342	0.000183333550856513\\
343	0.000183597738420683\\
344	0.000183866935959448\\
345	0.000184141238969377\\
346	0.000184420744714248\\
347	0.000184705552259115\\
348	0.000184995762505134\\
349	0.000185291478225168\\
350	0.000185592804100197\\
351	0.000185899846756542\\
352	0.000186212714804042\\
353	0.000186531518875363\\
354	0.000186856371666636\\
355	0.000187187387979784\\
356	0.000187524684766725\\
357	0.000187868381175616\\
358	0.000188218598599337\\
359	0.000188575460726427\\
360	0.000188939093594694\\
361	0.000189309625647759\\
362	0.000189687187794814\\
363	0.00019007191347385\\
364	0.000190463938718729\\
365	0.000190863402230402\\
366	0.000191270445452653\\
367	0.000191685212652803\\
368	0.000192107851007744\\
369	0.000192538510695839\\
370	0.000192977344995141\\
371	0.000193424510388485\\
372	0.000193880166676053\\
373	0.000194344477096035\\
374	0.000194817608454052\\
375	0.00019529973126214\\
376	0.000195791019888057\\
377	0.000196291652715871\\
378	0.0001968018123188\\
379	0.000197321685645456\\
380	0.000197851464220781\\
381	0.000198391344363156\\
382	0.000198941527419401\\
383	0.000199502220019748\\
384	0.000200073634355187\\
385	0.000200655988480237\\
386	0.000201249506644805\\
387	0.000201854419659851\\
388	0.000202470965303061\\
389	0.000203099388773038\\
390	0.000203739943204837\\
391	0.000204392890268078\\
392	0.000205058500885782\\
393	0.000205737056143686\\
394	0.000206428848497399\\
395	0.000207134183307596\\
396	0.000207853379823844\\
397	0.000208586765521004\\
398	0.00020933468860833\\
399	0.000210097524724835\\
400	0.000210875667369174\\
401	0.000211669529028189\\
402	0.000212479542287857\\
403	0.000213306160867271\\
404	0.000214149860468807\\
405	0.000215011139248253\\
406	0.000215890517559442\\
407	0.000216788536482469\\
408	0.000217705755060487\\
409	0.000218642749987742\\
410	0.000219600141048629\\
411	0.000220578570894087\\
412	0.000221578619178366\\
413	0.000222600889698149\\
414	0.00022364601233429\\
415	0.00022471464532483\\
416	0.000225807477920701\\
417	0.000226925233513964\\
418	0.000228068673351751\\
419	0.000229238600978757\\
420	0.000230435867590063\\
421	0.000231661378522649\\
422	0.000232916101182086\\
423	0.000234201074767163\\
424	0.000235517422258308\\
425	0.000236866365179842\\
426	0.000238249241555478\\
427	0.000239667526799369\\
428	0.000241122854473137\\
429	0.000242617023424316\\
430	0.000244151940366315\\
431	0.000245729312811729\\
432	0.000247349424765246\\
433	0.000248917206812989\\
434	0.000249596544315368\\
435	0.000250289392039495\\
436	0.000250995965316014\\
437	0.000251716480921113\\
438	0.00025245115865765\\
439	0.000253200223695467\\
440	0.00025396390992625\\
441	0.000254742464665859\\
442	0.000255536155136975\\
443	0.00025634527729162\\
444	0.000257170167690267\\
445	0.000258011219336887\\
446	0.000258868902549362\\
447	0.000259743792028783\\
448	0.000260636601032502\\
449	0.000261548222420013\\
450	0.000262479773609716\\
451	0.00026343263996428\\
452	0.000264408529489691\\
453	0.000265409712920401\\
454	0.000266440324185177\\
455	0.000267507390817175\\
456	0.000268615613600356\\
457	0.000269769795927175\\
458	0.00027097573957645\\
459	0.000272240500972844\\
460	0.00027357274626769\\
461	0.000274983305147449\\
462	0.000276486214505931\\
463	0.000278101230395547\\
464	0.000279861166060325\\
465	0.000307524637827358\\
466	0.000349969645631211\\
467	0.000393328689223382\\
468	0.000437627384730763\\
469	0.000482892213005026\\
470	0.000529150187553064\\
471	0.000576429124412446\\
472	0.0006247596323477\\
473	0.000674173667795424\\
474	0.000724704758816065\\
475	0.00077638852461577\\
476	0.000829263833541199\\
477	0.000883376879913921\\
478	0.000938764033621671\\
479	0.000995463115227499\\
480	0.00105351352101643\\
481	0.00111295636342444\\
482	0.00117383465337102\\
483	0.0012361935426771\\
484	0.00130008065241501\\
485	0.00136554652389328\\
486	0.00143264524472401\\
487	0.00150143532629143\\
488	0.00157198094668955\\
489	0.00164435372963181\\
490	0.00171863436274021\\
491	0.00179490999916231\\
492	0.00187327516151905\\
493	0.00195383234317257\\
494	0.00203669287158261\\
495	0.00212197728380178\\
496	0.00220981618775743\\
497	0.0023003531839373\\
498	0.00239374683710717\\
499	0.00249017303845386\\
500	0.00258982783074779\\
501	0.00269293074595561\\
502	0.00279972860438639\\
503	0.00291049941708504\\
504	0.00302555520514576\\
505	0.00314524552958512\\
506	0.00326263227805496\\
507	0.00330343466190086\\
508	0.00334563719849741\\
509	0.00338919372248778\\
510	0.00343401933015739\\
511	0.00348021493093214\\
512	0.00352790103705132\\
513	0.00357722775633646\\
514	0.0036283968779377\\
515	0.00368161901404434\\
516	0.00373712168333583\\
517	0.00379517532398462\\
518	0.00385610310005531\\
519	0.00392029325315597\\
520	0.0039882148043472\\
521	0.00406043735756756\\
522	0.00413765605181866\\
523	0.00422072260381219\\
524	0.00431068243548318\\
525	0.00440881296717743\\
526	0.00451553428790778\\
527	0.00462318184507296\\
528	0.00473134252956375\\
529	0.00483946652908471\\
530	0.00494682918447378\\
531	0.00505248079026667\\
532	0.00515562250187475\\
533	0.00525876642613349\\
534	0.00536161268074051\\
535	0.00546336034902462\\
536	0.00556294739333485\\
537	0.00565897620022263\\
538	0.00574971498516013\\
539	0.00583295559956373\\
540	0.00591379869531271\\
541	0.00599621185148525\\
542	0.00608011835818136\\
543	0.00616543130002583\\
544	0.00625205770730007\\
545	0.00633990646259004\\
546	0.00642888543393893\\
547	0.00651894091697697\\
548	0.00661009698930814\\
549	0.00670250472605515\\
550	0.0067965135636925\\
551	0.00689273545573893\\
552	0.00699161192885386\\
553	0.00709320002231934\\
554	0.00719740780199425\\
555	0.00730401456018862\\
556	0.0074126633149158\\
557	0.00752281634470193\\
558	0.00763362999187877\\
559	0.00774222432184838\\
560	0.00784624175801257\\
561	0.0079451173684526\\
562	0.00803836138267193\\
563	0.00812565253839744\\
564	0.00820701294573663\\
565	0.00828582198432094\\
566	0.00836215658040431\\
567	0.00843620586934715\\
568	0.00850832721502562\\
569	0.00857901013151536\\
570	0.00864906440997031\\
571	0.00871885032529967\\
572	0.00878725608766623\\
573	0.00885362539487147\\
574	0.00891781359239266\\
575	0.00897959328019709\\
576	0.00903975342437911\\
577	0.00909898426363027\\
578	0.0091567304087957\\
579	0.00921190939223783\\
580	0.00926446940119685\\
581	0.00931498012486593\\
582	0.00936431107260272\\
583	0.00941219245211477\\
584	0.00945866243584944\\
585	0.00950405575984369\\
586	0.00954865718510014\\
587	0.0095925304780904\\
588	0.00963572111751646\\
589	0.00967825788510043\\
590	0.00972017907935068\\
591	0.00976137872239597\\
592	0.00980171668827597\\
593	0.00984102099863122\\
594	0.00987905595166784\\
595	0.00991543381058343\\
596	0.00994937493726701\\
597	0.00997906286423442\\
598	0.0099999191923403\\
599	0\\
600	0\\
};
\addplot [color=red!25!mycolor17,solid,forget plot]
  table[row sep=crcr]{%
1	0.000375855361567015\\
2	0.000375864735362743\\
3	0.000375874276850767\\
4	0.000375883989026931\\
5	0.000375893874940512\\
6	0.000375903937695174\\
7	0.000375914180449942\\
8	0.000375924606420201\\
9	0.000375935218878671\\
10	0.000375946021156463\\
11	0.0003759570166441\\
12	0.000375968208792553\\
13	0.000375979601114359\\
14	0.000375991197184693\\
15	0.000376003000642513\\
16	0.000376015015191641\\
17	0.000376027244601968\\
18	0.000376039692710604\\
19	0.00037605236342309\\
20	0.000376065260714613\\
21	0.000376078388631249\\
22	0.000376091751291189\\
23	0.000376105352886097\\
24	0.000376119197682364\\
25	0.000376133290022439\\
26	0.000376147634326219\\
27	0.000376162235092381\\
28	0.000376177096899843\\
29	0.000376192224409123\\
30	0.000376207622363863\\
31	0.000376223295592248\\
32	0.000376239249008547\\
33	0.000376255487614668\\
34	0.000376272016501651\\
35	0.000376288840851316\\
36	0.000376305965937858\\
37	0.000376323397129494\\
38	0.000376341139890124\\
39	0.000376359199781053\\
40	0.000376377582462738\\
41	0.000376396293696532\\
42	0.000376415339346471\\
43	0.000376434725381129\\
44	0.000376454457875476\\
45	0.000376474543012747\\
46	0.000376494987086415\\
47	0.000376515796502134\\
48	0.000376536977779709\\
49	0.000376558537555178\\
50	0.00037658048258284\\
51	0.000376602819737422\\
52	0.000376625556016146\\
53	0.000376648698540977\\
54	0.000376672254560806\\
55	0.000376696231453725\\
56	0.000376720636729335\\
57	0.000376745478031095\\
58	0.000376770763138686\\
59	0.000376796499970441\\
60	0.000376822696585871\\
61	0.000376849361188093\\
62	0.000376876502126454\\
63	0.000376904127899108\\
64	0.000376932247155693\\
65	0.000376960868699994\\
66	0.000376990001492742\\
67	0.000377019654654379\\
68	0.000377049837467919\\
69	0.00037708055938181\\
70	0.000377111830012952\\
71	0.000377143659149655\\
72	0.000377176056754711\\
73	0.0003772090329685\\
74	0.000377242598112152\\
75	0.000377276762690801\\
76	0.000377311537396846\\
77	0.000377346933113272\\
78	0.000377382960917113\\
79	0.000377419632082857\\
80	0.000377456958085995\\
81	0.000377494950606607\\
82	0.000377533621533028\\
83	0.000377572982965524\\
84	0.000377613047220143\\
85	0.000377653826832503\\
86	0.000377695334561765\\
87	0.0003777375833946\\
88	0.000377780586549253\\
89	0.000377824357479701\\
90	0.00037786890987985\\
91	0.0003779142576878\\
92	0.000377960415090283\\
93	0.000378007396527034\\
94	0.000378055216695353\\
95	0.000378103890554693\\
96	0.000378153433331373\\
97	0.00037820386052334\\
98	0.000378255187905027\\
99	0.000378307431532289\\
100	0.00037836060774748\\
101	0.000378414733184553\\
102	0.00037846982477425\\
103	0.000378525899749498\\
104	0.000378582975650768\\
105	0.000378641070331584\\
106	0.000378700201964145\\
107	0.000378760389045049\\
108	0.000378821650401074\\
109	0.000378884005195147\\
110	0.000378947472932331\\
111	0.000379012073465995\\
112	0.000379077827004037\\
113	0.000379144754115263\\
114	0.000379212875735858\\
115	0.000379282213175986\\
116	0.000379352788126547\\
117	0.000379424622665918\\
118	0.000379497739267039\\
119	0.000379572160804407\\
120	0.000379647910561355\\
121	0.000379725012237393\\
122	0.000379803489955662\\
123	0.00037988336827062\\
124	0.000379964672175753\\
125	0.000380047427111495\\
126	0.000380131658973323\\
127	0.000380217394119883\\
128	0.000380304659381396\\
129	0.000380393482068136\\
130	0.000380483889979131\\
131	0.000380575911410879\\
132	0.000380669575166452\\
133	0.000380764910564579\\
134	0.000380861947448975\\
135	0.000380960716197793\\
136	0.000381061247733372\\
137	0.000381163573531995\\
138	0.00038126772563395\\
139	0.000381373736653725\\
140	0.000381481639790436\\
141	0.000381591468838373\\
142	0.000381703258197825\\
143	0.000381817042886077\\
144	0.00038193285854853\\
145	0.000382050741470206\\
146	0.000382170728587262\\
147	0.000382292857498925\\
148	0.000382417166479432\\
149	0.000382543694490394\\
150	0.000382672481193246\\
151	0.00038280356696203\\
152	0.000382936992896346\\
153	0.000383072800834601\\
154	0.000383211033367421\\
155	0.000383351733851479\\
156	0.00038349494642337\\
157	0.000383640716013898\\
158	0.000383789088362624\\
159	0.000383940110032628\\
160	0.000384093828425572\\
161	0.000384250291797077\\
162	0.00038440954927231\\
163	0.000384571650862024\\
164	0.000384736647478714\\
165	0.000384904590953235\\
166	0.000385075534051627\\
167	0.000385249530492313\\
168	0.000385426634963654\\
169	0.000385606903141773\\
170	0.000385790391708767\\
171	0.00038597715837124\\
172	0.000386167261879223\\
173	0.000386360762045411\\
174	0.000386557719764833\\
175	0.000386758197034831\\
176	0.00038696225697549\\
177	0.000387169963850431\\
178	0.000387381383087991\\
179	0.000387596581302871\\
180	0.000387815626318109\\
181	0.000388038587187575\\
182	0.000388265534218859\\
183	0.000388496538996624\\
184	0.000388731674406369\\
185	0.000388971014658706\\
186	0.000389214635314107\\
187	0.000389462613308146\\
188	0.000389715026977176\\
189	0.000389971956084584\\
190	0.000390233481847511\\
191	0.000390499686964138\\
192	0.00039077065564149\\
193	0.0003910464736238\\
194	0.000391327228221434\\
195	0.000391613008340391\\
196	0.000391903904512368\\
197	0.000392200008925458\\
198	0.000392501415455497\\
199	0.000392808219697883\\
200	0.000393120519000252\\
201	0.000393438412495624\\
202	0.000393762001136319\\
203	0.000394091387728522\\
204	0.000394426676967558\\
205	0.000394767975473868\\
206	0.000395115391829715\\
207	0.000395469036616637\\
208	0.000395829022453655\\
209	0.000396195464036341\\
210	0.000396568478176487\\
211	0.000396948183842786\\
212	0.000397334702202259\\
213	0.000397728156662532\\
214	0.000398128672915005\\
215	0.000398536378978847\\
216	0.000398951405246032\\
217	0.00039937388452712\\
218	0.000399803952098157\\
219	0.000400241745748425\\
220	0.000400687405829319\\
221	0.000401141075304038\\
222	0.000401602899798527\\
223	0.000402073027653355\\
224	0.000402551609976692\\
225	0.000403038800698443\\
226	0.000403534756625491\\
227	0.0004040396374981\\
228	0.000404553606047494\\
229	0.000405076828054671\\
230	0.00040560947241045\\
231	0.000406151711176854\\
232	0.000406703719649673\\
233	0.000407265676422422\\
234	0.000407837763451732\\
235	0.000408420166124007\\
236	0.000409013073323558\\
237	0.000409616677502251\\
238	0.000410231174750549\\
239	0.000410856764870171\\
240	0.000411493651448245\\
241	0.000412142041933094\\
242	0.000412802147711648\\
243	0.000413474184188537\\
244	0.000414158370866868\\
245	0.000414854931430798\\
246	0.000415564093829837\\
247	0.000416286090365026\\
248	0.000417021157777013\\
249	0.000417769537335968\\
250	0.000418531474933557\\
251	0.000419307221176814\\
252	0.000420097031484151\\
253	0.000420901166183438\\
254	0.000421719890612251\\
255	0.000422553475220229\\
256	0.000423402195673846\\
257	0.000424266332963319\\
258	0.000425146173511911\\
259	0.000426042009287706\\
260	0.000426954137917746\\
261	0.000427882862804754\\
262	0.000428828493246336\\
263	0.000429791344556945\\
264	0.000430771738192403\\
265	0.000431770001877172\\
266	0.000432786469734556\\
267	0.000433821482419583\\
268	0.000434875387254931\\
269	0.000435948538369868\\
270	0.000437041296842104\\
271	0.000438154030842877\\
272	0.000439287115785182\\
273	0.000440440934475291\\
274	0.000441615877267434\\
275	0.000442812342222102\\
276	0.00044403073526753\\
277	0.000445271470364928\\
278	0.000446534969677154\\
279	0.000447821663741041\\
280	0.000449131991643555\\
281	0.000450466401201604\\
282	0.000451825349145801\\
283	0.000453209301308104\\
284	0.000454618732813421\\
285	0.000456054128275351\\
286	0.00045751598199604\\
287	0.000459004798170102\\
288	0.000460521091092953\\
289	0.000462065385373391\\
290	0.000463638216150475\\
291	0.00046524012931502\\
292	0.000466871681735374\\
293	0.000468533441487851\\
294	0.000470225988091772\\
295	0.000471949912749025\\
296	0.000473705818588461\\
297	0.000475494320914845\\
298	0.000477316047462662\\
299	0.000479171638654685\\
300	0.000481061747865299\\
301	0.000482987041688683\\
302	0.000484948200211826\\
303	0.000486945917292301\\
304	0.000488980900840993\\
305	0.000491053873109538\\
306	0.000493165570982603\\
307	0.000495316746274937\\
308	0.000497508166033209\\
309	0.000499740612842413\\
310	0.000502014885137115\\
311	0.000504331797517111\\
312	0.000506692181067717\\
313	0.000509096883684502\\
314	0.000511546770402322\\
315	0.000514042723728773\\
316	0.000516585643981767\\
317	0.000519176449631222\\
318	0.00052181607764491\\
319	0.000524505483838112\\
320	0.00052724564322733\\
321	0.000530037550387755\\
322	0.00053288221981458\\
323	0.000535780686288141\\
324	0.000538734005242898\\
325	0.000541743253140319\\
326	0.00054480952784587\\
327	0.000547933949010139\\
328	0.000551117658454588\\
329	0.000554361820562086\\
330	0.000557667622672797\\
331	0.000561036275485951\\
332	0.000564469013468215\\
333	0.000567967095269487\\
334	0.000571531804147124\\
335	0.000575164448399749\\
336	0.000578866361811827\\
337	0.000582638904110663\\
338	0.000586483461437091\\
339	0.000590401446831726\\
340	0.000594394300738327\\
341	0.000598463491525742\\
342	0.000602610516029878\\
343	0.000606836900116855\\
344	0.000611144199267906\\
345	0.000615533999183235\\
346	0.000620007916404401\\
347	0.00062456759895429\\
348	0.0006292147269931\\
349	0.000633951013488447\\
350	0.000638778204896862\\
351	0.00064369808185397\\
352	0.000648712459870477\\
353	0.000653823190031695\\
354	0.00065903215969827\\
355	0.000664341293204633\\
356	0.000669752552572866\\
357	0.000675267938244582\\
358	0.000680889489831862\\
359	0.000686619286888755\\
360	0.000692459449704616\\
361	0.000698412140121118\\
362	0.000704479562374511\\
363	0.000710663963965251\\
364	0.000716967636556976\\
365	0.000723392916907245\\
366	0.000729942187832629\\
367	0.000736617879210691\\
368	0.000743422469022132\\
369	0.00075035848443618\\
370	0.000757428502942744\\
371	0.000764635153535262\\
372	0.000771981117948212\\
373	0.000779469131953892\\
374	0.000787101986723156\\
375	0.000794882530255428\\
376	0.000802813668883572\\
377	0.000810898368859764\\
378	0.000819139658029086\\
379	0.000827540627598236\\
380	0.000836104434007321\\
381	0.000844834300913916\\
382	0.000853733521299567\\
383	0.000862805459710351\\
384	0.000872053554644853\\
385	0.000881481321105537\\
386	0.000891092353332046\\
387	0.000900890327738918\\
388	0.000910879006084138\\
389	0.000921062238898888\\
390	0.000931443969208682\\
391	0.000942028236562785\\
392	0.000952819181333663\\
393	0.000963821049074246\\
394	0.000975038194216569\\
395	0.000986475081000435\\
396	0.00099813627578383\\
397	0.00101002641521094\\
398	0.00102215005977609\\
399	0.00103451205535606\\
400	0.00104711769728795\\
401	0.00105997245280715\\
402	0.00107308197031745\\
403	0.00108645208931192\\
404	0.00110008885116143\\
405	0.00111399851140646\\
406	0.00112818755533907\\
407	0.00114266272165957\\
408	0.00115743104644096\\
409	0.00117249995723704\\
410	0.00118787748593639\\
411	0.00120357290884931\\
412	0.00121959580838612\\
413	0.00123595364884107\\
414	0.00125265406994354\\
415	0.00126970487381734\\
416	0.0012871140242872\\
417	0.00130488964536087\\
418	0.00132304001870451\\
419	0.00134157357987641\\
420	0.00136049891310906\\
421	0.00137982474455152\\
422	0.00139955993346553\\
423	0.00141971346091733\\
424	0.00144029441416247\\
425	0.00146131196750383\\
426	0.00148277535892334\\
427	0.00150469386122702\\
428	0.00152707674509668\\
429	0.00154993322871793\\
430	0.00157327240515157\\
431	0.00159710314396872\\
432	0.00162143403281746\\
433	0.00164627419956294\\
434	0.00167164021505231\\
435	0.00169756119205294\\
436	0.00172405154377306\\
437	0.00175112621257999\\
438	0.00177880070542449\\
439	0.00180709113343772\\
440	0.00183601425639072\\
441	0.00186558753285368\\
442	0.0018958291770815\\
443	0.00192675822389243\\
444	0.00195839460311546\\
445	0.00199075922557841\\
446	0.00202387408312123\\
447	0.0020577623657961\\
448	0.002092448600366\\
449	0.00212795881569221\\
450	0.00216432074310062\\
451	0.00220156406288743\\
452	0.00223972071917512\\
453	0.0022788253353444\\
454	0.0023189156084635\\
455	0.00236003293586921\\
456	0.00240222330492044\\
457	0.00244553836795654\\
458	0.00249003681886893\\
459	0.00253578618256766\\
460	0.00258286512841324\\
461	0.00263136639903981\\
462	0.00268140027958883\\
463	0.00273309391864708\\
464	0.00278658892957238\\
465	0.0028160858344251\\
466	0.00283229364123304\\
467	0.00284891162233504\\
468	0.00286595707549019\\
469	0.00288344845123734\\
470	0.00290140543312626\\
471	0.00291984904478575\\
472	0.00293880188699185\\
473	0.00295828798435947\\
474	0.00297833269726963\\
475	0.00299896215397996\\
476	0.00302020306229267\\
477	0.00304208316074912\\
478	0.00306463106400083\\
479	0.00308787617611319\\
480	0.0031118482766772\\
481	0.0031365768954682\\
482	0.00316209039491253\\
483	0.00318841464993482\\
484	0.00321557117399093\\
485	0.00324357448262726\\
486	0.00327242840310902\\
487	0.00330212091969892\\
488	0.00333261699691251\\
489	0.00336384880815839\\
490	0.00339572914874768\\
491	0.00342827959547576\\
492	0.00346152324687747\\
493	0.00349549303595669\\
494	0.00353023096274955\\
495	0.00356579818865329\\
496	0.00360226589595257\\
497	0.00363968661879014\\
498	0.00367811966898888\\
499	0.00371763218444223\\
500	0.00375830031168834\\
501	0.00380021052783213\\
502	0.00384346111023267\\
503	0.00388816378050593\\
504	0.00393444559791164\\
505	0.00398245123033693\\
506	0.00403235920962256\\
507	0.00408472396073087\\
508	0.00414228410426581\\
509	0.00420609627069447\\
510	0.00427692741442519\\
511	0.00434901585665425\\
512	0.00442233189612585\\
513	0.00449682806344451\\
514	0.0045724331827926\\
515	0.00464904554710942\\
516	0.00472652549801599\\
517	0.00480468550780751\\
518	0.0048832768550998\\
519	0.00496197116355689\\
520	0.00504033495355125\\
521	0.00511780127935372\\
522	0.00519363221594222\\
523	0.00526686946768327\\
524	0.00533627214472673\\
525	0.00540024436774731\\
526	0.00545788363371357\\
527	0.00551627335090793\\
528	0.00557539167135386\\
529	0.00563527859893137\\
530	0.00569594513862986\\
531	0.00575739841589084\\
532	0.00581967989167963\\
533	0.00588276119286136\\
534	0.00594662385371704\\
535	0.0060112831455332\\
536	0.00607681286106642\\
537	0.00614337892796407\\
538	0.00621128615237045\\
539	0.00628104207503862\\
540	0.00635312254556396\\
541	0.0064277223001837\\
542	0.0065048984881829\\
543	0.00658470064598824\\
544	0.00666716561147777\\
545	0.00675231227196042\\
546	0.00684007090564941\\
547	0.00693038605348422\\
548	0.00702318517624303\\
549	0.00711841038944991\\
550	0.00721590978540083\\
551	0.00731534307738251\\
552	0.00741610746377254\\
553	0.00751723386195233\\
554	0.007615025649278\\
555	0.00770805129765681\\
556	0.00779580222598891\\
557	0.00787788560954736\\
558	0.00795385979339356\\
559	0.00802573665129025\\
560	0.00809530374867697\\
561	0.00816276262131606\\
562	0.00822843087997214\\
563	0.00829273829880889\\
564	0.00835656690377135\\
565	0.0084203077058799\\
566	0.0084841452589513\\
567	0.00854828119591742\\
568	0.00861187435009632\\
569	0.00867400404508977\\
570	0.00873448446449719\\
571	0.00879303057564046\\
572	0.00885059500516906\\
573	0.00890770640378388\\
574	0.00896437197501948\\
575	0.00902054698487434\\
576	0.00907467179264195\\
577	0.00912662893845292\\
578	0.00917681195253149\\
579	0.00922620839379051\\
580	0.00927487032649829\\
581	0.00932247740245366\\
582	0.0093690234246846\\
583	0.00941489058069364\\
584	0.00946020548959901\\
585	0.00950497683508797\\
586	0.009549192402668\\
587	0.00959283081692013\\
588	0.00963587774208467\\
589	0.0096783320190804\\
590	0.0097202097033969\\
591	0.00976138909231767\\
592	0.00980171925872612\\
593	0.00984102135019415\\
594	0.00987905595166784\\
595	0.00991543381058343\\
596	0.00994937493726701\\
597	0.00997906286423442\\
598	0.0099999191923403\\
599	0\\
600	0\\
};
\addplot [color=mycolor19,solid,forget plot]
  table[row sep=crcr]{%
1	0.00243322575185515\\
2	0.00243322940690112\\
3	0.002433233127433\\
4	0.00243323691462267\\
5	0.00243324076966301\\
6	0.00243324469376817\\
7	0.00243324868817403\\
8	0.00243325275413852\\
9	0.00243325689294207\\
10	0.00243326110588798\\
11	0.00243326539430283\\
12	0.00243326975953691\\
13	0.00243327420296463\\
14	0.00243327872598494\\
15	0.0024332833300218\\
16	0.0024332880165246\\
17	0.0024332927869686\\
18	0.00243329764285543\\
19	0.00243330258571356\\
20	0.00243330761709872\\
21	0.00243331273859443\\
22	0.00243331795181255\\
23	0.00243332325839364\\
24	0.00243332866000762\\
25	0.00243333415835421\\
26	0.00243333975516345\\
27	0.00243334545219633\\
28	0.00243335125124526\\
29	0.00243335715413463\\
30	0.00243336316272144\\
31	0.00243336927889581\\
32	0.00243337550458164\\
33	0.00243338184173715\\
34	0.00243338829235553\\
35	0.00243339485846553\\
36	0.00243340154213214\\
37	0.0024334083454572\\
38	0.00243341527058006\\
39	0.0024334223196783\\
40	0.00243342949496824\\
41	0.0024334367987059\\
42	0.00243344423318748\\
43	0.00243345180075018\\
44	0.0024334595037729\\
45	0.00243346734467702\\
46	0.00243347532592709\\
47	0.00243348345003162\\
48	0.00243349171954392\\
49	0.00243350013706283\\
50	0.00243350870523359\\
51	0.00243351742674857\\
52	0.00243352630434822\\
53	0.00243353534082182\\
54	0.00243354453900849\\
55	0.00243355390179791\\
56	0.00243356343213137\\
57	0.00243357313300256\\
58	0.00243358300745861\\
59	0.00243359305860097\\
60	0.00243360328958637\\
61	0.00243361370362788\\
62	0.00243362430399586\\
63	0.00243363509401895\\
64	0.00243364607708517\\
65	0.00243365725664295\\
66	0.0024336686362022\\
67	0.0024336802193354\\
68	0.00243369200967867\\
69	0.00243370401093303\\
70	0.00243371622686545\\
71	0.00243372866131001\\
72	0.00243374131816915\\
73	0.00243375420141487\\
74	0.00243376731508999\\
75	0.00243378066330931\\
76	0.00243379425026102\\
77	0.00243380808020794\\
78	0.0024338221574888\\
79	0.00243383648651969\\
80	0.00243385107179539\\
81	0.00243386591789073\\
82	0.00243388102946204\\
83	0.00243389641124867\\
84	0.00243391206807432\\
85	0.00243392800484867\\
86	0.00243394422656882\\
87	0.00243396073832091\\
88	0.00243397754528165\\
89	0.00243399465271997\\
90	0.00243401206599862\\
91	0.00243402979057587\\
92	0.00243404783200714\\
93	0.00243406619594686\\
94	0.00243408488815007\\
95	0.00243410391447433\\
96	0.00243412328088146\\
97	0.00243414299343945\\
98	0.00243416305832428\\
99	0.00243418348182193\\
100	0.00243420427033023\\
101	0.00243422543036089\\
102	0.00243424696854157\\
103	0.00243426889161785\\
104	0.00243429120645541\\
105	0.00243431392004206\\
106	0.00243433703949003\\
107	0.00243436057203805\\
108	0.00243438452505372\\
109	0.00243440890603569\\
110	0.00243443372261605\\
111	0.00243445898256268\\
112	0.00243448469378165\\
113	0.00243451086431968\\
114	0.00243453750236664\\
115	0.00243456461625812\\
116	0.00243459221447791\\
117	0.00243462030566075\\
118	0.00243464889859492\\
119	0.00243467800222503\\
120	0.00243470762565476\\
121	0.00243473777814967\\
122	0.00243476846914011\\
123	0.00243479970822408\\
124	0.00243483150517026\\
125	0.00243486386992106\\
126	0.0024348968125956\\
127	0.00243493034349291\\
128	0.00243496447309514\\
129	0.00243499921207075\\
130	0.00243503457127785\\
131	0.00243507056176759\\
132	0.00243510719478751\\
133	0.00243514448178507\\
134	0.00243518243441118\\
135	0.00243522106452384\\
136	0.00243526038419174\\
137	0.00243530040569808\\
138	0.00243534114154432\\
139	0.00243538260445403\\
140	0.00243542480737691\\
141	0.00243546776349272\\
142	0.00243551148621539\\
143	0.00243555598919714\\
144	0.0024356012863328\\
145	0.00243564739176398\\
146	0.00243569431988355\\
147	0.00243574208534001\\
148	0.00243579070304211\\
149	0.00243584018816341\\
150	0.00243589055614697\\
151	0.00243594182271013\\
152	0.00243599400384942\\
153	0.00243604711584545\\
154	0.00243610117526801\\
155	0.00243615619898108\\
156	0.00243621220414821\\
157	0.00243626920823775\\
158	0.00243632722902821\\
159	0.00243638628461381\\
160	0.0024364463934101\\
161	0.00243650757415964\\
162	0.00243656984593782\\
163	0.00243663322815867\\
164	0.002436697740581\\
165	0.00243676340331443\\
166	0.00243683023682566\\
167	0.00243689826194482\\
168	0.00243696749987184\\
169	0.00243703797218308\\
170	0.00243710970083805\\
171	0.00243718270818609\\
172	0.00243725701697337\\
173	0.00243733265035\\
174	0.00243740963187707\\
175	0.00243748798553408\\
176	0.00243756773572627\\
177	0.00243764890729223\\
178	0.00243773152551165\\
179	0.00243781561611305\\
180	0.0024379012052819\\
181	0.00243798831966862\\
182	0.00243807698639691\\
183	0.00243816723307218\\
184	0.00243825908779009\\
185	0.00243835257914535\\
186	0.00243844773624056\\
187	0.00243854458869515\\
188	0.00243864316665478\\
189	0.00243874350080061\\
190	0.00243884562235886\\
191	0.00243894956311053\\
192	0.00243905535540133\\
193	0.00243916303215172\\
194	0.00243927262686716\\
195	0.0024393841736486\\
196	0.00243949770720314\\
197	0.00243961326285478\\
198	0.00243973087655549\\
199	0.0024398505848965\\
200	0.00243997242511965\\
201	0.00244009643512912\\
202	0.00244022265350323\\
203	0.0024403511195066\\
204	0.00244048187310239\\
205	0.00244061495496486\\
206	0.00244075040649215\\
207	0.0024408882698193\\
208	0.00244102858783148\\
209	0.00244117140417743\\
210	0.00244131676328333\\
211	0.00244146471036672\\
212	0.0024416152914508\\
213	0.00244176855337893\\
214	0.00244192454382946\\
215	0.00244208331133091\\
216	0.00244224490527716\\
217	0.0024424093759433\\
218	0.0024425767745015\\
219	0.00244274715303737\\
220	0.00244292056456639\\
221	0.002443097063051\\
222	0.00244327670341765\\
223	0.00244345954157455\\
224	0.00244364563442941\\
225	0.00244383503990779\\
226	0.00244402781697168\\
227	0.00244422402563843\\
228	0.00244442372700015\\
229	0.00244462698324345\\
230	0.00244483385766953\\
231	0.00244504441471467\\
232	0.00244525871997119\\
233	0.00244547684020884\\
234	0.00244569884339648\\
235	0.00244592479872436\\
236	0.00244615477662682\\
237	0.00244638884880532\\
238	0.0024466270882522\\
239	0.00244686956927463\\
240	0.00244711636751926\\
241	0.00244736755999737\\
242	0.00244762322511045\\
243	0.00244788344267635\\
244	0.00244814829395612\\
245	0.00244841786168116\\
246	0.00244869223008125\\
247	0.00244897148491293\\
248	0.00244925571348868\\
249	0.00244954500470662\\
250	0.00244983944908089\\
251	0.0024501391387728\\
252	0.00245044416762254\\
253	0.0024507546311817\\
254	0.00245107062674643\\
255	0.00245139225339151\\
256	0.00245171961200502\\
257	0.00245205280532398\\
258	0.00245239193797073\\
259	0.00245273711649021\\
260	0.00245308844938809\\
261	0.00245344604716983\\
262	0.0024538100223808\\
263	0.00245418048964712\\
264	0.00245455756571778\\
265	0.00245494136950771\\
266	0.00245533202214183\\
267	0.00245572964700042\\
268	0.00245613436976553\\
269	0.00245654631846854\\
270	0.00245696562353919\\
271	0.00245739241785566\\
272	0.00245782683679614\\
273	0.00245826901829169\\
274	0.00245871910288073\\
275	0.00245917723376485\\
276	0.00245964355686627\\
277	0.00246011822088686\\
278	0.00246060137736899\\
279	0.002461093180758\\
280	0.00246159378846649\\
281	0.0024621033609407\\
282	0.00246262206172861\\
283	0.00246315005755033\\
284	0.00246368751837049\\
285	0.00246423461747296\\
286	0.00246479153153784\\
287	0.00246535844072099\\
288	0.00246593552873593\\
289	0.00246652298293855\\
290	0.00246712099441454\\
291	0.00246772975806956\\
292	0.00246834947272263\\
293	0.00246898034120261\\
294	0.0024696225704477\\
295	0.00247027637160877\\
296	0.00247094196015583\\
297	0.00247161955598855\\
298	0.00247230938355034\\
299	0.00247301167194677\\
300	0.00247372665506791\\
301	0.00247445457171518\\
302	0.00247519566573273\\
303	0.0024759501861435\\
304	0.00247671838729021\\
305	0.00247750052898146\\
306	0.00247829687664303\\
307	0.00247910770147479\\
308	0.00247993328061304\\
309	0.00248077389729889\\
310	0.00248162984105248\\
311	0.00248250140785344\\
312	0.00248338890032766\\
313	0.00248429262794039\\
314	0.00248521290719616\\
315	0.00248615006184504\\
316	0.00248710442309585\\
317	0.00248807632983603\\
318	0.00248906612885809\\
319	0.00249007417509303\\
320	0.0024911008318499\\
321	0.00249214647106188\\
322	0.00249321147353832\\
323	0.00249429622922244\\
324	0.0024954011374541\\
325	0.00249652660723728\\
326	0.00249767305751127\\
327	0.00249884091742482\\
328	0.00250003062661243\\
329	0.00250124263547115\\
330	0.00250247740543719\\
331	0.00250373540926026\\
332	0.00250501713127448\\
333	0.00250632306766386\\
334	0.00250765372672073\\
335	0.00250900962909522\\
336	0.0025103913080341\\
337	0.00251179930960745\\
338	0.00251323419292223\\
339	0.00251469653032239\\
340	0.00251618690757661\\
341	0.00251770592405601\\
342	0.00251925419290494\\
343	0.00252083234120518\\
344	0.0025224410101529\\
345	0.00252408085532325\\
346	0.00252575254696737\\
347	0.00252745677035158\\
348	0.00252919422614939\\
349	0.00253096563089699\\
350	0.00253277171752206\\
351	0.00253461323595275\\
352	0.00253649095380898\\
353	0.00253840565717714\\
354	0.00254035815148238\\
355	0.00254234926247657\\
356	0.00254437983683534\\
357	0.00254645074273674\\
358	0.00254856287045189\\
359	0.00255071713294689\\
360	0.00255291446649582\\
361	0.00255515583130394\\
362	0.00255744221214044\\
363	0.0025597746189799\\
364	0.00256215408765162\\
365	0.00256458168049537\\
366	0.00256705848702252\\
367	0.0025695856245811\\
368	0.00257216423902266\\
369	0.00257479550536971\\
370	0.0025774806284808\\
371	0.00258022084371129\\
372	0.00258301741756675\\
373	0.00258587164834578\\
374	0.00258878486676855\\
375	0.00259175843658691\\
376	0.00259479375517109\\
377	0.00259789225406755\\
378	0.00260105539952129\\
379	0.00260428469295505\\
380	0.00260758167139656\\
381	0.00261094790784304\\
382	0.00261438501155048\\
383	0.00261789462823249\\
384	0.00262147844015029\\
385	0.0026251381660719\\
386	0.00262887556107286\\
387	0.00263269241614548\\
388	0.00263659055757467\\
389	0.00264057184602923\\
390	0.0026446381753039\\
391	0.0026487914706317\\
392	0.00265303368646489\\
393	0.00265736680359865\\
394	0.00266179282548564\\
395	0.00266631377357758\\
396	0.00267093168157545\\
397	0.00267564858866405\\
398	0.00268046653230534\\
399	0.00268538753867355\\
400	0.00269041360363406\\
401	0.00269554667067543\\
402	0.00270078861131059\\
403	0.00270614120321678\\
404	0.00271160610671546\\
405	0.00271718484080559\\
406	0.00272287876090144\\
407	0.00272868904181218\\
408	0.00273461667146946\\
409	0.00274066246357437\\
410	0.00274682710084091\\
411	0.00275311122544019\\
412	0.00275951561264378\\
413	0.00276604151076391\\
414	0.0027726912044118\\
415	0.00277946746198407\\
416	0.00278637314424618\\
417	0.00279341120986781\\
418	0.00280058472191354\\
419	0.00280789685554711\\
420	0.00281535090508996\\
421	0.00282295028744672\\
422	0.00283069855175062\\
423	0.00283859939608647\\
424	0.00284665671753488\\
425	0.0028548746330132\\
426	0.00286325750122126\\
427	0.00287180994486809\\
428	0.00288053687188287\\
429	0.00288944349364547\\
430	0.00289853533701717\\
431	0.00290781824351838\\
432	0.00291729833865878\\
433	0.00292698192475248\\
434	0.00293687522341037\\
435	0.00294698438892954\\
436	0.00295731584817982\\
437	0.00296787631184095\\
438	0.00297867278412394\\
439	0.00298971257008401\\
440	0.00300100327931023\\
441	0.00301255282434944\\
442	0.00302436941164805\\
443	0.00303646152202934\\
444	0.00304883787669455\\
445	0.00306150738335215\\
446	0.00307447905521328\\
447	0.00308776189309536\\
448	0.00310136471771375\\
449	0.00311529593617701\\
450	0.00312956322845109\\
451	0.00314417316807397\\
452	0.0031591306794131\\
453	0.00317443806932127\\
454	0.00319009365038383\\
455	0.00320608971468185\\
456	0.00322241117922144\\
457	0.00323903284133796\\
458	0.00325591568031399\\
459	0.00327300186627504\\
460	0.00329020802013293\\
461	0.00330741618752921\\
462	0.00332446235535646\\
463	0.0033412318864999\\
464	0.00335758742685534\\
465	0.00337338760836127\\
466	0.00338918042814833\\
467	0.00340530655874068\\
468	0.00342178366260316\\
469	0.00343863177671142\\
470	0.00345587363475759\\
471	0.00347353476768834\\
472	0.0034916426578092\\
473	0.00351023430286914\\
474	0.00352935638325661\\
475	0.00354907225021517\\
476	0.00356945357161289\\
477	0.00359057112268743\\
478	0.00361250894201974\\
479	0.00363536731737758\\
480	0.00365926652319814\\
481	0.00368435151566849\\
482	0.00371079785126699\\
483	0.00373881917193254\\
484	0.00376867669766305\\
485	0.00380069128279977\\
486	0.0038352586996068\\
487	0.00387286879664353\\
488	0.0039141285910975\\
489	0.00395978649863041\\
490	0.00400999994379843\\
491	0.00406104221269004\\
492	0.00411290889303081\\
493	0.00416559058337188\\
494	0.00421907124212074\\
495	0.00427332600109312\\
496	0.00432831847470865\\
497	0.00438399836381636\\
498	0.0044402986351139\\
499	0.00449713119941296\\
500	0.00455438145906978\\
501	0.00461190142112322\\
502	0.00466950098744828\\
503	0.00472693692844685\\
504	0.00478389889952752\\
505	0.00483999165498312\\
506	0.00489471185943743\\
507	0.00494740805108051\\
508	0.00499718385826722\\
509	0.00504289057825752\\
510	0.00508359944924854\\
511	0.00512486917326703\\
512	0.0051666669189476\\
513	0.00520895418874585\\
514	0.00525168675647541\\
515	0.00529481481413926\\
516	0.00533828304376811\\
517	0.00538202974207427\\
518	0.00542599384801781\\
519	0.00547014462970504\\
520	0.00551450866905051\\
521	0.00555909764201829\\
522	0.00560393937996247\\
523	0.00564911930132595\\
524	0.00569481048431436\\
525	0.00574131614048781\\
526	0.0057890876497553\\
527	0.00583843272153866\\
528	0.00588943896141111\\
529	0.00594220010200307\\
530	0.00599681393333867\\
531	0.00605338332978803\\
532	0.00611201645377706\\
533	0.00617282830798182\\
534	0.00623593928816131\\
535	0.00630147018111279\\
536	0.00636949234990505\\
537	0.00644006104353074\\
538	0.00651325385171948\\
539	0.0065891142941634\\
540	0.00666763728431354\\
541	0.00674877479497815\\
542	0.00683243735517887\\
543	0.00691848790243981\\
544	0.00700678142595555\\
545	0.00709710460458226\\
546	0.00718914909923042\\
547	0.00728233011489576\\
548	0.00737567760376649\\
549	0.00746556707184357\\
550	0.00755055199800488\\
551	0.00763014193807071\\
552	0.00770374944381809\\
553	0.00777122664892783\\
554	0.00783558854786247\\
555	0.00789783794406934\\
556	0.00795825232183846\\
557	0.0080172196760148\\
558	0.00807556562364243\\
559	0.00813380648727646\\
560	0.00819215789604154\\
561	0.00825081058869847\\
562	0.00830995726729622\\
563	0.00836978176436801\\
564	0.00843009923217756\\
565	0.00848930666147047\\
566	0.00854719485207219\\
567	0.00860348792501928\\
568	0.00865888706570304\\
569	0.0087141507280509\\
570	0.00876926249803578\\
571	0.00882426452232611\\
572	0.00887920168507198\\
573	0.00893313527028819\\
574	0.00898517449966207\\
575	0.00903522942231062\\
576	0.00908474529495557\\
577	0.00913376135612379\\
578	0.00918233514225318\\
579	0.00923013735242427\\
580	0.00927705205247275\\
581	0.00932349070899567\\
582	0.00936953410212466\\
583	0.00941516998195261\\
584	0.00946036084630153\\
585	0.00950506346562379\\
586	0.00954923819722844\\
587	0.0095928530922261\\
588	0.00963588745565751\\
589	0.00967833566889083\\
590	0.0097202108171807\\
591	0.00976138933709939\\
592	0.00980171928775089\\
593	0.00984102135019415\\
594	0.00987905595166784\\
595	0.00991543381058343\\
596	0.00994937493726701\\
597	0.00997906286423442\\
598	0.0099999191923403\\
599	0\\
600	0\\
};
\addplot [color=red!50!mycolor17,solid,forget plot]
  table[row sep=crcr]{%
1	0.00283870092861365\\
2	0.00283870433655056\\
3	0.00283870780564254\\
4	0.00283871133698597\\
5	0.00283871493169682\\
6	0.0028387185909111\\
7	0.00283872231578512\\
8	0.00283872610749591\\
9	0.00283872996724156\\
10	0.00283873389624159\\
11	0.00283873789573735\\
12	0.00283874196699244\\
13	0.00283874611129304\\
14	0.00283875032994839\\
15	0.00283875462429111\\
16	0.00283875899567769\\
17	0.00283876344548891\\
18	0.00283876797513025\\
19	0.00283877258603231\\
20	0.00283877727965132\\
21	0.00283878205746957\\
22	0.00283878692099581\\
23	0.00283879187176586\\
24	0.00283879691134296\\
25	0.00283880204131834\\
26	0.0028388072633117\\
27	0.00283881257897167\\
28	0.00283881798997637\\
29	0.00283882349803399\\
30	0.00283882910488322\\
31	0.00283883481229385\\
32	0.00283884062206733\\
33	0.00283884653603731\\
34	0.00283885255607024\\
35	0.0028388586840659\\
36	0.0028388649219581\\
37	0.00283887127171516\\
38	0.00283887773534063\\
39	0.00283888431487382\\
40	0.00283889101239058\\
41	0.00283889783000374\\
42	0.00283890476986399\\
43	0.00283891183416043\\
44	0.00283891902512122\\
45	0.00283892634501442\\
46	0.00283893379614855\\
47	0.00283894138087341\\
48	0.00283894910158078\\
49	0.00283895696070514\\
50	0.00283896496072446\\
51	0.00283897310416097\\
52	0.00283898139358199\\
53	0.00283898983160067\\
54	0.00283899842087676\\
55	0.00283900716411757\\
56	0.00283901606407876\\
57	0.00283902512356513\\
58	0.00283903434543156\\
59	0.00283904373258396\\
60	0.00283905328798007\\
61	0.00283906301463037\\
62	0.00283907291559913\\
63	0.00283908299400531\\
64	0.00283909325302345\\
65	0.00283910369588481\\
66	0.00283911432587824\\
67	0.0028391251463513\\
68	0.00283913616071125\\
69	0.00283914737242611\\
70	0.00283915878502576\\
71	0.00283917040210304\\
72	0.00283918222731482\\
73	0.0028391942643832\\
74	0.0028392065170966\\
75	0.00283921898931104\\
76	0.00283923168495119\\
77	0.00283924460801173\\
78	0.00283925776255847\\
79	0.00283927115272973\\
80	0.00283928478273752\\
81	0.00283929865686889\\
82	0.00283931277948727\\
83	0.00283932715503378\\
84	0.00283934178802866\\
85	0.00283935668307262\\
86	0.0028393718448483\\
87	0.0028393872781217\\
88	0.00283940298774367\\
89	0.00283941897865134\\
90	0.00283943525586976\\
91	0.0028394518245134\\
92	0.00283946868978772\\
93	0.00283948585699075\\
94	0.00283950333151479\\
95	0.00283952111884801\\
96	0.0028395392245762\\
97	0.00283955765438446\\
98	0.00283957641405893\\
99	0.00283959550948862\\
100	0.00283961494666721\\
101	0.0028396347316949\\
102	0.00283965487078025\\
103	0.00283967537024211\\
104	0.00283969623651155\\
105	0.00283971747613387\\
106	0.0028397390957706\\
107	0.00283976110220152\\
108	0.00283978350232674\\
109	0.00283980630316888\\
110	0.00283982951187513\\
111	0.00283985313571945\\
112	0.00283987718210492\\
113	0.00283990165856587\\
114	0.00283992657277022\\
115	0.00283995193252184\\
116	0.0028399777457629\\
117	0.00284000402057639\\
118	0.00284003076518851\\
119	0.00284005798797113\\
120	0.00284008569744448\\
121	0.00284011390227963\\
122	0.00284014261130115\\
123	0.00284017183348989\\
124	0.00284020157798559\\
125	0.00284023185408969\\
126	0.00284026267126823\\
127	0.00284029403915462\\
128	0.00284032596755265\\
129	0.0028403584664394\\
130	0.00284039154596828\\
131	0.00284042521647216\\
132	0.00284045948846641\\
133	0.00284049437265214\\
134	0.00284052987991943\\
135	0.00284056602135055\\
136	0.00284060280822343\\
137	0.00284064025201492\\
138	0.00284067836440439\\
139	0.00284071715727713\\
140	0.00284075664272799\\
141	0.002840796833065\\
142	0.00284083774081305\\
143	0.00284087937871771\\
144	0.00284092175974898\\
145	0.00284096489710522\\
146	0.00284100880421708\\
147	0.00284105349475156\\
148	0.00284109898261603\\
149	0.00284114528196243\\
150	0.0028411924071915\\
151	0.00284124037295709\\
152	0.00284128919417041\\
153	0.00284133888600459\\
154	0.00284138946389919\\
155	0.00284144094356474\\
156	0.00284149334098739\\
157	0.00284154667243366\\
158	0.00284160095445534\\
159	0.00284165620389425\\
160	0.00284171243788735\\
161	0.00284176967387167\\
162	0.00284182792958957\\
163	0.00284188722309395\\
164	0.0028419475727535\\
165	0.00284200899725816\\
166	0.00284207151562455\\
167	0.00284213514720164\\
168	0.00284219991167635\\
169	0.00284226582907936\\
170	0.00284233291979086\\
171	0.00284240120454669\\
172	0.00284247070444422\\
173	0.00284254144094859\\
174	0.00284261343589891\\
175	0.00284268671151462\\
176	0.00284276129040196\\
177	0.0028428371955605\\
178	0.00284291445038981\\
179	0.00284299307869625\\
180	0.00284307310469975\\
181	0.00284315455304094\\
182	0.00284323744878815\\
183	0.00284332181744459\\
184	0.00284340768495583\\
185	0.00284349507771704\\
186	0.00284358402258066\\
187	0.00284367454686415\\
188	0.00284376667835761\\
189	0.00284386044533187\\
190	0.00284395587654643\\
191	0.00284405300125773\\
192	0.00284415184922738\\
193	0.00284425245073059\\
194	0.00284435483656481\\
195	0.0028444590380584\\
196	0.00284456508707942\\
197	0.00284467301604467\\
198	0.00284478285792881\\
199	0.00284489464627353\\
200	0.00284500841519708\\
201	0.00284512419940369\\
202	0.00284524203419337\\
203	0.00284536195547169\\
204	0.00284548399975977\\
205	0.00284560820420451\\
206	0.00284573460658881\\
207	0.00284586324534205\\
208	0.00284599415955079\\
209	0.00284612738896945\\
210	0.00284626297403133\\
211	0.00284640095585972\\
212	0.00284654137627915\\
213	0.00284668427782694\\
214	0.00284682970376474\\
215	0.00284697769809036\\
216	0.00284712830554981\\
217	0.00284728157164946\\
218	0.0028474375426683\\
219	0.00284759626567062\\
220	0.00284775778851866\\
221	0.00284792215988554\\
222	0.00284808942926841\\
223	0.00284825964700172\\
224	0.00284843286427077\\
225	0.00284860913312545\\
226	0.00284878850649403\\
227	0.00284897103819749\\
228	0.00284915678296373\\
229	0.00284934579644215\\
230	0.00284953813521847\\
231	0.00284973385682968\\
232	0.00284993301977929\\
233	0.00285013568355279\\
234	0.00285034190863333\\
235	0.00285055175651761\\
236	0.00285076528973209\\
237	0.00285098257184941\\
238	0.0028512036675049\\
239	0.00285142864241372\\
240	0.00285165756338779\\
241	0.00285189049835336\\
242	0.00285212751636863\\
243	0.00285236868764177\\
244	0.00285261408354902\\
245	0.00285286377665339\\
246	0.00285311784072332\\
247	0.00285337635075185\\
248	0.00285363938297592\\
249	0.00285390701489617\\
250	0.0028541793252969\\
251	0.00285445639426639\\
252	0.00285473830321754\\
253	0.00285502513490887\\
254	0.00285531697346588\\
255	0.00285561390440267\\
256	0.00285591601464397\\
257	0.0028562233925476\\
258	0.00285653612792724\\
259	0.00285685431207555\\
260	0.00285717803778778\\
261	0.00285750739938577\\
262	0.00285784249274233\\
263	0.00285818341530614\\
264	0.00285853026612701\\
265	0.00285888314588166\\
266	0.00285924215690008\\
267	0.00285960740319219\\
268	0.00285997899047519\\
269	0.00286035702620138\\
270	0.00286074161958659\\
271	0.0028611328816392\\
272	0.00286153092518973\\
273	0.00286193586492109\\
274	0.00286234781739955\\
275	0.00286276690110634\\
276	0.00286319323647004\\
277	0.00286362694589979\\
278	0.00286406815381909\\
279	0.00286451698670082\\
280	0.0028649735731028\\
281	0.0028654380437045\\
282	0.00286591053134476\\
283	0.00286639117106043\\
284	0.00286688010012632\\
285	0.00286737745809612\\
286	0.00286788338684473\\
287	0.0028683980306119\\
288	0.00286892153604727\\
289	0.00286945405225691\\
290	0.00286999573085146\\
291	0.00287054672599602\\
292	0.00287110719446185\\
293	0.00287167729567994\\
294	0.0028722571917969\\
295	0.00287284704773271\\
296	0.00287344703124124\\
297	0.00287405731297303\\
298	0.00287467806654102\\
299	0.00287530946858908\\
300	0.00287595169886375\\
301	0.00287660494028946\\
302	0.0028772693790472\\
303	0.00287794520465731\\
304	0.00287863261006642\\
305	0.00287933179173891\\
306	0.00288004294975335\\
307	0.00288076628790415\\
308	0.00288150201380895\\
309	0.00288225033902219\\
310	0.0028830114791553\\
311	0.002883785654004\\
312	0.00288457308768344\\
313	0.00288537400877162\\
314	0.00288618865046179\\
315	0.00288701725072479\\
316	0.00288786005248165\\
317	0.00288871730378771\\
318	0.00288958925802896\\
319	0.00289047617413145\\
320	0.0028913783167851\\
321	0.00289229595668272\\
322	0.00289322937077541\\
323	0.0028941788425457\\
324	0.00289514466229953\\
325	0.00289612712747834\\
326	0.0028971265429927\\
327	0.00289814322157858\\
328	0.00289917748417769\\
329	0.00290022966034285\\
330	0.00290130008866944\\
331	0.00290238911725369\\
332	0.00290349710417811\\
333	0.00290462441802381\\
334	0.00290577143840861\\
335	0.00290693855654915\\
336	0.00290812617584297\\
337	0.00290933471246505\\
338	0.00291056459596982\\
339	0.00291181626988715\\
340	0.00291309019229849\\
341	0.00291438683638664\\
342	0.0029157066909734\\
343	0.00291705026105032\\
344	0.00291841806776943\\
345	0.00291981064685911\\
346	0.0029212285484882\\
347	0.00292267233700952\\
348	0.00292414259058094\\
349	0.00292563990067231\\
350	0.00292716487147907\\
351	0.002928718119271\\
352	0.00293030027168571\\
353	0.00293191196688974\\
354	0.00293355385245548\\
355	0.0029352265849003\\
356	0.00293693084065037\\
357	0.00293866731783082\\
358	0.00294043673738679\\
359	0.0029422398442847\\
360	0.00294407740880065\\
361	0.00294595022790378\\
362	0.00294785912674309\\
363	0.00294980496024692\\
364	0.00295178861484537\\
365	0.00295381101032717\\
366	0.00295587310184337\\
367	0.00295797588207172\\
368	0.00296012038355706\\
369	0.00296230768124428\\
370	0.00296453889522294\\
371	0.00296681519370352\\
372	0.00296913779624821\\
373	0.00297150797728118\\
374	0.00297392706990552\\
375	0.0029763964700572\\
376	0.00297891764102881\\
377	0.00298149211839925\\
378	0.00298412151540846\\
379	0.00298680752881971\\
380	0.00298955194531476\\
381	0.00299235664847091\\
382	0.0029952236263704\\
383	0.00299815497989572\\
384	0.0030011529317642\\
385	0.00300421983635425\\
386	0.00300735819037249\\
387	0.00301057064440294\\
388	0.003013860015367\\
389	0.00301722929990288\\
390	0.00302068168864234\\
391	0.00302422058131805\\
392	0.00302784960257062\\
393	0.00303157261823363\\
394	0.00303539375174873\\
395	0.00303931740018923\\
396	0.00304334824912741\\
397	0.00304749128523309\\
398	0.00305175180497379\\
399	0.00305613541720487\\
400	0.00306064803684511\\
401	0.00306529586542185\\
402	0.00307008535243514\\
403	0.00307502312943827\\
404	0.00308011590570261\\
405	0.00308537031020121\\
406	0.00309079265898533\\
407	0.00309638861924857\\
408	0.0031021627305885\\
409	0.00310811772883947\\
410	0.00311425359633487\\
411	0.00312056623196952\\
412	0.00312704559457391\\
413	0.00313367314836349\\
414	0.00314041859936968\\
415	0.00314727139920554\\
416	0.00315423369614006\\
417	0.00316130776674751\\
418	0.0031684960276916\\
419	0.0031758010557658\\
420	0.00318322565622549\\
421	0.00319077296565129\\
422	0.00319844631403471\\
423	0.0032062490418238\\
424	0.003214183896715\\
425	0.00322225377701107\\
426	0.00323046174710021\\
427	0.00323881105488027\\
428	0.00324730515158386\\
429	0.00325594771461001\\
430	0.00326474267408322\\
431	0.00327369424394737\\
432	0.00328280695904933\\
433	0.00329208572344072\\
434	0.00330153588294981\\
435	0.00331116331548502\\
436	0.00332097451830581\\
437	0.00333097671219466\\
438	0.00334117796620837\\
439	0.00335158734756848\\
440	0.00336221510235997\\
441	0.00337307287410827\\
442	0.00338417396907542\\
443	0.00339553367935669\\
444	0.00340716967768397\\
445	0.0034191025013505\\
446	0.00343135614685067\\
447	0.0034439588011638\\
448	0.0034569437378115\\
449	0.00347035039750513\\
450	0.00348422562530908\\
451	0.00349862484951891\\
452	0.00351361518620998\\
453	0.00352928220505041\\
454	0.00354573489592987\\
455	0.0035631115355953\\
456	0.0035815654237004\\
457	0.00360128540371144\\
458	0.00362250512451155\\
459	0.00364551475566122\\
460	0.00367067542333025\\
461	0.00369843548734482\\
462	0.00372934297113895\\
463	0.00376095621363585\\
464	0.00379309105106504\\
465	0.00382577263716753\\
466	0.00385901484731637\\
467	0.00389282359395008\\
468	0.00392720376105065\\
469	0.00396215890131907\\
470	0.00399769086849669\\
471	0.00403379936012597\\
472	0.00407048133580928\\
473	0.00410773020443479\\
474	0.00414553463487091\\
475	0.00418387698318528\\
476	0.00422273139401959\\
477	0.00426206186955287\\
478	0.00430182020803441\\
479	0.00434194273055568\\
480	0.0043823461022077\\
481	0.00442292198487946\\
482	0.00446353017886259\\
483	0.00450398980586727\\
484	0.00454406795040861\\
485	0.00458346500642036\\
486	0.00462179578739872\\
487	0.00465856535452245\\
488	0.00469313896240186\\
489	0.00472470844453819\\
490	0.0047530155955093\\
491	0.00478167072697865\\
492	0.00481065613380342\\
493	0.00483995091478616\\
494	0.00486953070053233\\
495	0.0048993674315073\\
496	0.00492942922729178\\
497	0.00495968036195307\\
498	0.00499008137895352\\
499	0.00502058946022384\\
500	0.00505115916793818\\
501	0.00508174373103509\\
502	0.00511229716796209\\
503	0.00514277748512175\\
504	0.00517315143935459\\
505	0.00520340138261903\\
506	0.00523353332555396\\
507	0.00526359232558662\\
508	0.005293685216728\\
509	0.00532401058150215\\
510	0.00535487809816213\\
511	0.00538650320169562\\
512	0.00541896601711208\\
513	0.0054523077533292\\
514	0.00548657443382623\\
515	0.00552181785933801\\
516	0.00555809814807075\\
517	0.00559548305717446\\
518	0.00563404906956839\\
519	0.00567388228072107\\
520	0.00571507818832832\\
521	0.00575774030131697\\
522	0.00580198076408421\\
523	0.00584791935643933\\
524	0.00589567988748809\\
525	0.00594538341309121\\
526	0.00599709568295925\\
527	0.00605087853310064\\
528	0.00610682270070741\\
529	0.00616501674035226\\
530	0.00622554406078374\\
531	0.00628847479267532\\
532	0.00635385821429033\\
533	0.00642172861966806\\
534	0.00649209994009733\\
535	0.00656495654067599\\
536	0.006640281558746\\
537	0.00671802543277983\\
538	0.00679805038301329\\
539	0.00688017893851313\\
540	0.00696422364181383\\
541	0.00704985310511505\\
542	0.00713651715111697\\
543	0.0072233324844939\\
544	0.00730733886327977\\
545	0.00738638960302661\\
546	0.00745982539142606\\
547	0.00752713644055451\\
548	0.00758836051141061\\
549	0.00764653857858205\\
550	0.00770271529057265\\
551	0.00775720579591524\\
552	0.00781071095053047\\
553	0.00786393070041828\\
554	0.00791723492564974\\
555	0.00797081310434071\\
556	0.0080248179508191\\
557	0.00807939150536912\\
558	0.00813468278734686\\
559	0.00819081823622241\\
560	0.0082479087802371\\
561	0.00830477765193133\\
562	0.00836059325980605\\
563	0.00841510282450088\\
564	0.00846832031068026\\
565	0.00852161182516532\\
566	0.00857497379044075\\
567	0.00862842855953914\\
568	0.00868200623738066\\
569	0.00873571853626056\\
570	0.00878953117589831\\
571	0.0088418987105902\\
572	0.00889242540907162\\
573	0.00894187213284206\\
574	0.00899097596863721\\
575	0.00903979684709103\\
576	0.0090883736771921\\
577	0.00913644094746051\\
578	0.00918373619157009\\
579	0.00923062808335025\\
580	0.0092772493210446\\
581	0.00932358230216704\\
582	0.00936958177756908\\
583	0.00941519618323153\\
584	0.00946037485366822\\
585	0.00950507047593196\\
586	0.00954924140305753\\
587	0.00959285439625311\\
588	0.00963588790901514\\
589	0.0096783357954337\\
590	0.00972021084245027\\
591	0.00976138933978722\\
592	0.00980171928775089\\
593	0.00984102135019415\\
594	0.00987905595166784\\
595	0.00991543381058343\\
596	0.00994937493726701\\
597	0.00997906286423442\\
598	0.0099999191923403\\
599	0\\
600	0\\
};
\addplot [color=red!40!mycolor19,solid,forget plot]
  table[row sep=crcr]{%
1	0.0030112771484627\\
2	0.00301128212363723\\
3	0.00301128718813487\\
4	0.00301129234355798\\
5	0.00301129759153769\\
6	0.00301130293373427\\
7	0.0030113083718378\\
8	0.00301131390756862\\
9	0.00301131954267788\\
10	0.00301132527894816\\
11	0.00301133111819388\\
12	0.00301133706226203\\
13	0.00301134311303272\\
14	0.00301134927241964\\
15	0.00301135554237081\\
16	0.00301136192486916\\
17	0.00301136842193308\\
18	0.00301137503561719\\
19	0.00301138176801284\\
20	0.00301138862124889\\
21	0.00301139559749225\\
22	0.00301140269894872\\
23	0.00301140992786355\\
24	0.00301141728652219\\
25	0.00301142477725101\\
26	0.00301143240241807\\
27	0.00301144016443379\\
28	0.00301144806575179\\
29	0.00301145610886957\\
30	0.00301146429632934\\
31	0.00301147263071887\\
32	0.00301148111467217\\
33	0.00301148975087046\\
34	0.00301149854204294\\
35	0.00301150749096766\\
36	0.00301151660047236\\
37	0.00301152587343541\\
38	0.00301153531278668\\
39	0.00301154492150848\\
40	0.00301155470263645\\
41	0.00301156465926061\\
42	0.0030115747945262\\
43	0.00301158511163475\\
44	0.00301159561384511\\
45	0.00301160630447436\\
46	0.00301161718689893\\
47	0.00301162826455568\\
48	0.0030116395409429\\
49	0.0030116510196215\\
50	0.00301166270421601\\
51	0.00301167459841581\\
52	0.00301168670597626\\
53	0.00301169903071984\\
54	0.00301171157653739\\
55	0.00301172434738933\\
56	0.00301173734730686\\
57	0.00301175058039324\\
58	0.0030117640508251\\
59	0.00301177776285367\\
60	0.00301179172080623\\
61	0.00301180592908736\\
62	0.00301182039218036\\
63	0.0030118351146486\\
64	0.00301185010113706\\
65	0.00301186535637363\\
66	0.00301188088517075\\
67	0.00301189669242676\\
68	0.00301191278312753\\
69	0.00301192916234797\\
70	0.00301194583525363\\
71	0.0030119628071023\\
72	0.0030119800832457\\
73	0.00301199766913105\\
74	0.00301201557030285\\
75	0.00301203379240457\\
76	0.00301205234118046\\
77	0.00301207122247726\\
78	0.00301209044224609\\
79	0.00301211000654424\\
80	0.00301212992153708\\
81	0.00301215019350007\\
82	0.00301217082882056\\
83	0.00301219183399986\\
84	0.00301221321565528\\
85	0.00301223498052212\\
86	0.00301225713545579\\
87	0.003012279687434\\
88	0.00301230264355882\\
89	0.00301232601105897\\
90	0.00301234979729207\\
91	0.00301237400974681\\
92	0.0030123986560454\\
93	0.00301242374394588\\
94	0.0030124492813445\\
95	0.00301247527627825\\
96	0.00301250173692723\\
97	0.00301252867161732\\
98	0.00301255608882265\\
99	0.00301258399716825\\
100	0.00301261240543275\\
101	0.00301264132255103\\
102	0.00301267075761706\\
103	0.00301270071988666\\
104	0.00301273121878037\\
105	0.00301276226388639\\
106	0.00301279386496348\\
107	0.00301282603194397\\
108	0.00301285877493694\\
109	0.00301289210423115\\
110	0.00301292603029838\\
111	0.00301296056379656\\
112	0.00301299571557307\\
113	0.00301303149666809\\
114	0.00301306791831796\\
115	0.0030131049919587\\
116	0.00301314272922949\\
117	0.0030131811419762\\
118	0.00301322024225507\\
119	0.00301326004233647\\
120	0.00301330055470852\\
121	0.00301334179208103\\
122	0.00301338376738934\\
123	0.00301342649379836\\
124	0.00301346998470643\\
125	0.00301351425374962\\
126	0.00301355931480574\\
127	0.00301360518199869\\
128	0.00301365186970269\\
129	0.00301369939254675\\
130	0.00301374776541901\\
131	0.00301379700347138\\
132	0.00301384712212412\\
133	0.00301389813707056\\
134	0.0030139500642818\\
135	0.00301400292001168\\
136	0.00301405672080159\\
137	0.00301411148348558\\
138	0.00301416722519542\\
139	0.00301422396336584\\
140	0.00301428171573976\\
141	0.00301434050037368\\
142	0.00301440033564319\\
143	0.00301446124024844\\
144	0.00301452323321978\\
145	0.00301458633392366\\
146	0.00301465056206828\\
147	0.00301471593770963\\
148	0.0030147824812575\\
149	0.00301485021348162\\
150	0.00301491915551792\\
151	0.00301498932887487\\
152	0.00301506075543991\\
153	0.00301513345748602\\
154	0.00301520745767843\\
155	0.00301528277908132\\
156	0.00301535944516481\\
157	0.00301543747981191\\
158	0.00301551690732563\\
159	0.00301559775243631\\
160	0.00301568004030886\\
161	0.00301576379655038\\
162	0.00301584904721765\\
163	0.00301593581882494\\
164	0.00301602413835183\\
165	0.00301611403325125\\
166	0.00301620553145757\\
167	0.00301629866139481\\
168	0.00301639345198515\\
169	0.00301648993265734\\
170	0.00301658813335542\\
171	0.00301668808454757\\
172	0.00301678981723495\\
173	0.00301689336296089\\
174	0.00301699875382012\\
175	0.00301710602246815\\
176	0.00301721520213082\\
177	0.00301732632661396\\
178	0.00301743943031334\\
179	0.00301755454822457\\
180	0.00301767171595336\\
181	0.00301779096972585\\
182	0.00301791234639907\\
183	0.00301803588347162\\
184	0.00301816161909453\\
185	0.00301828959208228\\
186	0.00301841984192393\\
187	0.00301855240879453\\
188	0.00301868733356669\\
189	0.00301882465782218\\
190	0.00301896442386401\\
191	0.00301910667472834\\
192	0.00301925145419692\\
193	0.0030193988068095\\
194	0.00301954877787644\\
195	0.00301970141349166\\
196	0.00301985676054566\\
197	0.00302001486673878\\
198	0.0030201757805947\\
199	0.00302033955147407\\
200	0.00302050622958839\\
201	0.00302067586601417\\
202	0.00302084851270721\\
203	0.00302102422251701\\
204	0.00302120304920175\\
205	0.00302138504744302\\
206	0.00302157027286114\\
207	0.00302175878203055\\
208	0.00302195063249546\\
209	0.00302214588278573\\
210	0.00302234459243299\\
211	0.00302254682198701\\
212	0.00302275263303224\\
213	0.00302296208820471\\
214	0.0030231752512091\\
215	0.00302339218683602\\
216	0.0030236129609797\\
217	0.00302383764065569\\
218	0.00302406629401913\\
219	0.00302429899038295\\
220	0.00302453580023652\\
221	0.00302477679526461\\
222	0.00302502204836646\\
223	0.00302527163367521\\
224	0.0030255256265776\\
225	0.00302578410373393\\
226	0.00302604714309833\\
227	0.00302631482393923\\
228	0.00302658722686019\\
229	0.00302686443382102\\
230	0.00302714652815906\\
231	0.00302743359461096\\
232	0.00302772571933458\\
233	0.00302802298993118\\
234	0.00302832549546807\\
235	0.00302863332650143\\
236	0.0030289465750994\\
237	0.00302926533486556\\
238	0.00302958970096271\\
239	0.00302991977013685\\
240	0.00303025564074161\\
241	0.00303059741276294\\
242	0.00303094518784404\\
243	0.00303129906931069\\
244	0.00303165916219689\\
245	0.00303202557327074\\
246	0.0030323984110607\\
247	0.00303277778588221\\
248	0.00303316380986452\\
249	0.00303355659697799\\
250	0.00303395626306149\\
251	0.00303436292585038\\
252	0.0030347767050046\\
253	0.00303519772213726\\
254	0.0030356261008434\\
255	0.0030360619667292\\
256	0.00303650544744144\\
257	0.0030369566726973\\
258	0.00303741577431452\\
259	0.00303788288624183\\
260	0.00303835814458976\\
261	0.00303884168766177\\
262	0.00303933365598564\\
263	0.00303983419234523\\
264	0.00304034344181264\\
265	0.00304086155178052\\
266	0.00304138867199482\\
267	0.00304192495458787\\
268	0.00304247055411169\\
269	0.00304302562757174\\
270	0.00304359033446077\\
271	0.00304416483679328\\
272	0.00304474929914\\
273	0.00304534388866292\\
274	0.00304594877515046\\
275	0.00304656413105293\\
276	0.00304719013151849\\
277	0.00304782695442915\\
278	0.0030484747804373\\
279	0.0030491337930023\\
280	0.00304980417842765\\
281	0.00305048612589808\\
282	0.00305117982751735\\
283	0.00305188547834593\\
284	0.00305260327643924\\
285	0.00305333342288612\\
286	0.00305407612184748\\
287	0.00305483158059534\\
288	0.00305560000955211\\
289	0.00305638162233016\\
290	0.00305717663577175\\
291	0.00305798526998912\\
292	0.00305880774840501\\
293	0.00305964429779349\\
294	0.00306049514832087\\
295	0.00306136053358738\\
296	0.0030622406906688\\
297	0.0030631358601587\\
298	0.00306404628621093\\
299	0.00306497221658262\\
300	0.00306591390267762\\
301	0.00306687159959031\\
302	0.00306784556615013\\
303	0.00306883606496656\\
304	0.00306984336247477\\
305	0.00307086772898195\\
306	0.00307190943871459\\
307	0.00307296876986641\\
308	0.00307404600464744\\
309	0.0030751414293342\\
310	0.00307625533432112\\
311	0.00307738801417344\\
312	0.00307853976768165\\
313	0.00307971089791798\\
314	0.00308090171229491\\
315	0.00308211252262612\\
316	0.00308334364519038\\
317	0.00308459540079854\\
318	0.00308586811486459\\
319	0.00308716211748075\\
320	0.00308847774349799\\
321	0.00308981533261227\\
322	0.00309117522945808\\
323	0.00309255778370995\\
324	0.00309396335019373\\
325	0.00309539228900961\\
326	0.00309684496566861\\
327	0.00309832175124565\\
328	0.00309982302255223\\
329	0.00310134916233282\\
330	0.00310290055949007\\
331	0.00310447760934495\\
332	0.00310608071393943\\
333	0.00310771028239149\\
334	0.00310936673131448\\
335	0.0031110504853147\\
336	0.00311276197758505\\
337	0.00311450165061221\\
338	0.00311626995701265\\
339	0.00311806736049661\\
340	0.00311989433692556\\
341	0.00312175137539228\\
342	0.00312363897943671\\
343	0.003125557670164\\
344	0.0031275080020705\\
345	0.00312949059349016\\
346	0.00313150609232436\\
347	0.00313355517864743\\
348	0.00313563856738285\\
349	0.0031377570110202\\
350	0.00313991130239194\\
351	0.00314210227763664\\
352	0.00314433081954981\\
353	0.00314659786080227\\
354	0.00314890438113493\\
355	0.00315125136603214\\
356	0.00315363958019717\\
357	0.00315606978981859\\
358	0.00315854278111835\\
359	0.00316105936144412\\
360	0.00316362036046411\\
361	0.00316622663147663\\
362	0.00316887905284827\\
363	0.00317157852959675\\
364	0.00317432599513677\\
365	0.00317712241320975\\
366	0.00317996878002174\\
367	0.00318286612661715\\
368	0.00318581552152086\\
369	0.00318881807368557\\
370	0.00319187493578809\\
371	0.0031949873079246\\
372	0.00319815644176414\\
373	0.00320138364522874\\
374	0.0032046702877813\\
375	0.00320801780641599\\
376	0.00321142771246325\\
377	0.00321490159934192\\
378	0.00321844115141562\\
379	0.0032220481541401\\
380	0.00322572450572489\\
381	0.00322947223057536\\
382	0.003233293494836\\
383	0.00323719062441954\\
384	0.00324116612598733\\
385	0.0032452227114445\\
386	0.00324936332663442\\
387	0.00325359118506684\\
388	0.00325790980770063\\
389	0.00326232307003289\\
390	0.00326683525803616\\
391	0.00327145113484695\\
392	0.00327617602056441\\
393	0.00328101588809075\\
394	0.00328597747866928\\
395	0.00329106844169038\\
396	0.00329629750449335\\
397	0.00330167467934666\\
398	0.00330721151660432\\
399	0.00331292141582399\\
400	0.0033188200096813\\
401	0.00332492563927259\\
402	0.00333125994479976\\
403	0.00333784860241919\\
404	0.00334472224685625\\
405	0.00335191763085759\\
406	0.00335947908746879\\
407	0.00336746038041965\\
408	0.00337592705245904\\
409	0.00338495941124689\\
410	0.00339465632310242\\
411	0.00340513999615007\\
412	0.0034165618451223\\
413	0.00342910904764097\\
414	0.00344300944672684\\
415	0.00345755801444051\\
416	0.00347234515899185\\
417	0.00348737409996265\\
418	0.00350264800731524\\
419	0.00351816984114086\\
420	0.0035339418067238\\
421	0.00354996568105268\\
422	0.00356624732542498\\
423	0.00358279709431792\\
424	0.00359963373631932\\
425	0.00361676185217926\\
426	0.00363418607612658\\
427	0.00365191106702642\\
428	0.00366994149738172\\
429	0.00368828203972282\\
430	0.00370693734984093\\
431	0.00372591204624422\\
432	0.00374521068514317\\
433	0.00376483772985987\\
434	0.0037847975113347\\
435	0.00380509417336685\\
436	0.00382573160994369\\
437	0.00384671338743754\\
438	0.00386804264795385\\
439	0.00388972198915353\\
440	0.00391175331465244\\
441	0.0039341376475423\\
442	0.00395687489758588\\
443	0.00397996357008652\\
444	0.00400340040115312\\
445	0.00402717989986627\\
446	0.00405129377242333\\
447	0.00407573019637187\\
448	0.0041004729041679\\
449	0.00412550002427073\\
450	0.00415078261515364\\
451	0.00417628281527402\\
452	0.00420195147286436\\
453	0.00422772500064694\\
454	0.00425352122695221\\
455	0.00427923405601563\\
456	0.00430472705272688\\
457	0.00432982539830094\\
458	0.00435430509663297\\
459	0.00437787912462104\\
460	0.00440018001228618\\
461	0.00442073945979756\\
462	0.00443897091856598\\
463	0.00445725960360944\\
464	0.00447580055336444\\
465	0.00449458924557552\\
466	0.00451361983613378\\
467	0.00453288517536736\\
468	0.00455237665129548\\
469	0.00457208401945484\\
470	0.00459199521965731\\
471	0.00461209618082414\\
472	0.00463237061681782\\
473	0.00465279982181396\\
474	0.00467336248284928\\
475	0.00469403452956161\\
476	0.0047147890479902\\
477	0.00473559627477015\\
478	0.00475642369917354\\
479	0.00477723634335092\\
480	0.00479799731346455\\
481	0.00481866874103088\\
482	0.00483921327983798\\
483	0.00485959638845276\\
484	0.00487978971827209\\
485	0.00489977605615691\\
486	0.00491955646208567\\
487	0.00493916053055873\\
488	0.00495866108989554\\
489	0.00497819474011869\\
490	0.00499796145418809\\
491	0.0050180776608813\\
492	0.00503855184396394\\
493	0.00505939348956257\\
494	0.0050806133032376\\
495	0.00510222347856746\\
496	0.00512423801255947\\
497	0.00514667307048559\\
498	0.00516954748368211\\
499	0.00519288328983599\\
500	0.00521670633080219\\
501	0.00524104668724238\\
502	0.00526593805116734\\
503	0.00529141966021656\\
504	0.00531753775955137\\
505	0.00534435061159844\\
506	0.00537195477964907\\
507	0.0054004263733106\\
508	0.00542983977475685\\
509	0.00546026936773654\\
510	0.00549178321330305\\
511	0.00552444340442616\\
512	0.00555831555240889\\
513	0.00559346761621968\\
514	0.00562997120760527\\
515	0.00566789562440533\\
516	0.00570725572049871\\
517	0.00574812831483058\\
518	0.0057905930740293\\
519	0.0058347319712251\\
520	0.00588062854546756\\
521	0.0059283669909499\\
522	0.00597803091455968\\
523	0.00602970178255804\\
524	0.0060834574252944\\
525	0.00613936985774463\\
526	0.00619754555422584\\
527	0.00625807035375208\\
528	0.00632097800128714\\
529	0.00638628421195292\\
530	0.00645398122847025\\
531	0.00652402827496695\\
532	0.00659634488034116\\
533	0.00667080369120464\\
534	0.00674722577632755\\
535	0.00682543605429951\\
536	0.00690517067081394\\
537	0.006985979879753\\
538	0.0070671353684765\\
539	0.00714694473789928\\
540	0.00722171707236833\\
541	0.00729072305803338\\
542	0.00735364306074771\\
543	0.00741047186832885\\
544	0.00746351245621795\\
545	0.00751458096933631\\
546	0.0075642019222949\\
547	0.00761304925249949\\
548	0.00766168557552592\\
549	0.00771048316522664\\
550	0.00775963560327215\\
551	0.00780930056370565\\
552	0.00785961270762311\\
553	0.00791065670204328\\
554	0.00796248760120481\\
555	0.00801519227973258\\
556	0.00806886380586354\\
557	0.00812360765985678\\
558	0.0081776104431567\\
559	0.00823054447228591\\
560	0.00828212481938252\\
561	0.00833330018496336\\
562	0.00838468146777848\\
563	0.00843628943921433\\
564	0.0084881762700586\\
565	0.00854036375010845\\
566	0.00859280326717225\\
567	0.00864547347042305\\
568	0.00869831647947506\\
569	0.00874951291989383\\
570	0.0087989032234281\\
571	0.00884783920707862\\
572	0.00889659845102266\\
573	0.00894523117842181\\
574	0.0089937388115468\\
575	0.00904204227672353\\
576	0.00908965434686499\\
577	0.00913682916240663\\
578	0.0091838363449705\\
579	0.00923066223904429\\
580	0.00927726451044059\\
581	0.00932359040395203\\
582	0.00936958612525249\\
583	0.00941519840888499\\
584	0.00946037591227052\\
585	0.00950507093368084\\
586	0.00954924157818994\\
587	0.00959285445321423\\
588	0.00963588792379213\\
589	0.00967833579815023\\
590	0.00972021084271538\\
591	0.00976138933978722\\
592	0.00980171928775088\\
593	0.00984102135019415\\
594	0.00987905595166784\\
595	0.00991543381058343\\
596	0.00994937493726701\\
597	0.00997906286423442\\
598	0.0099999191923403\\
599	0\\
600	0\\
};
\addplot [color=red!75!mycolor17,solid,forget plot]
  table[row sep=crcr]{%
1	0.00330282272054889\\
2	0.00330283342163309\\
3	0.00330284431478508\\
4	0.00330285540344943\\
5	0.00330286669113244\\
6	0.00330287818140321\\
7	0.00330288987789482\\
8	0.0033029017843054\\
9	0.00330291390439932\\
10	0.00330292624200841\\
11	0.00330293880103316\\
12	0.00330295158544386\\
13	0.00330296459928195\\
14	0.00330297784666129\\
15	0.00330299133176942\\
16	0.00330300505886883\\
17	0.00330301903229842\\
18	0.00330303325647477\\
19	0.00330304773589353\\
20	0.00330306247513091\\
21	0.00330307747884505\\
22	0.00330309275177754\\
23	0.0033031082987548\\
24	0.00330312412468978\\
25	0.00330314023458331\\
26	0.00330315663352584\\
27	0.00330317332669894\\
28	0.00330319031937693\\
29	0.00330320761692861\\
30	0.00330322522481887\\
31	0.00330324314861042\\
32	0.00330326139396562\\
33	0.00330327996664815\\
34	0.0033032988725249\\
35	0.00330331811756777\\
36	0.00330333770785555\\
37	0.00330335764957587\\
38	0.00330337794902708\\
39	0.00330339861262027\\
40	0.0033034196468813\\
41	0.00330344105845283\\
42	0.00330346285409639\\
43	0.00330348504069449\\
44	0.00330350762525285\\
45	0.00330353061490255\\
46	0.00330355401690227\\
47	0.00330357783864058\\
48	0.00330360208763825\\
49	0.00330362677155063\\
50	0.00330365189817003\\
51	0.0033036774754282\\
52	0.00330370351139872\\
53	0.0033037300142997\\
54	0.00330375699249623\\
55	0.00330378445450304\\
56	0.00330381240898714\\
57	0.00330384086477062\\
58	0.00330386983083331\\
59	0.00330389931631573\\
60	0.0033039293305218\\
61	0.00330395988292187\\
62	0.00330399098315565\\
63	0.00330402264103528\\
64	0.00330405486654828\\
65	0.00330408766986083\\
66	0.00330412106132083\\
67	0.00330415505146126\\
68	0.00330418965100335\\
69	0.00330422487086005\\
70	0.0033042607221394\\
71	0.00330429721614801\\
72	0.00330433436439457\\
73	0.00330437217859355\\
74	0.00330441067066876\\
75	0.00330444985275714\\
76	0.00330448973721252\\
77	0.00330453033660952\\
78	0.00330457166374751\\
79	0.0033046137316545\\
80	0.00330465655359143\\
81	0.003304700143056\\
82	0.00330474451378721\\
83	0.00330478967976943\\
84	0.00330483565523691\\
85	0.00330488245467818\\
86	0.00330493009284055\\
87	0.00330497858473473\\
88	0.00330502794563957\\
89	0.00330507819110677\\
90	0.00330512933696575\\
91	0.00330518139932865\\
92	0.00330523439459526\\
93	0.00330528833945823\\
94	0.00330534325090826\\
95	0.00330539914623935\\
96	0.00330545604305425\\
97	0.00330551395926991\\
98	0.00330557291312314\\
99	0.00330563292317622\\
100	0.00330569400832267\\
101	0.00330575618779331\\
102	0.003305819481162\\
103	0.00330588390835195\\
104	0.00330594948964182\\
105	0.00330601624567204\\
106	0.0033060841974513\\
107	0.00330615336636303\\
108	0.00330622377417209\\
109	0.00330629544303151\\
110	0.00330636839548943\\
111	0.00330644265449607\\
112	0.00330651824341091\\
113	0.00330659518600989\\
114	0.00330667350649289\\
115	0.00330675322949116\\
116	0.00330683438007506\\
117	0.00330691698376181\\
118	0.00330700106652339\\
119	0.00330708665479459\\
120	0.00330717377548134\\
121	0.00330726245596888\\
122	0.00330735272413047\\
123	0.0033074446083358\\
124	0.00330753813745996\\
125	0.00330763334089236\\
126	0.0033077302485458\\
127	0.00330782889086571\\
128	0.00330792929883973\\
129	0.00330803150400712\\
130	0.00330813553846866\\
131	0.00330824143489649\\
132	0.00330834922654435\\
133	0.00330845894725767\\
134	0.00330857063148426\\
135	0.00330868431428477\\
136	0.00330880003134363\\
137	0.00330891781898009\\
138	0.00330903771415932\\
139	0.00330915975450399\\
140	0.00330928397830574\\
141	0.00330941042453704\\
142	0.00330953913286325\\
143	0.00330967014365478\\
144	0.00330980349799959\\
145	0.00330993923771584\\
146	0.00331007740536477\\
147	0.00331021804426372\\
148	0.00331036119849963\\
149	0.00331050691294246\\
150	0.00331065523325904\\
151	0.0033108062059271\\
152	0.00331095987824959\\
153	0.00331111629836915\\
154	0.00331127551528291\\
155	0.00331143757885755\\
156	0.00331160253984459\\
157	0.00331177044989592\\
158	0.00331194136157969\\
159	0.00331211532839636\\
160	0.0033122924047951\\
161	0.00331247264619046\\
162	0.00331265610897938\\
163	0.00331284285055829\\
164	0.00331303292934082\\
165	0.00331322640477553\\
166	0.00331342333736411\\
167	0.00331362378867986\\
168	0.00331382782138642\\
169	0.00331403549925693\\
170	0.00331424688719343\\
171	0.0033144620512466\\
172	0.00331468105863596\\
173	0.00331490397777025\\
174	0.00331513087826821\\
175	0.00331536183097977\\
176	0.00331559690800757\\
177	0.00331583618272889\\
178	0.00331607972981778\\
179	0.00331632762526781\\
180	0.0033165799464151\\
181	0.00331683677196163\\
182	0.00331709818199918\\
183	0.00331736425803347\\
184	0.00331763508300882\\
185	0.00331791074133321\\
186	0.00331819131890377\\
187	0.00331847690313258\\
188	0.00331876758297316\\
189	0.0033190634489472\\
190	0.00331936459317178\\
191	0.00331967110938712\\
192	0.00331998309298471\\
193	0.00332030064103599\\
194	0.00332062385232146\\
195	0.00332095282736034\\
196	0.00332128766844062\\
197	0.00332162847964977\\
198	0.0033219753669058\\
199	0.00332232843798895\\
200	0.00332268780257393\\
201	0.00332305357226255\\
202	0.00332342586061704\\
203	0.00332380478319396\\
204	0.00332419045757847\\
205	0.00332458300341937\\
206	0.00332498254246467\\
207	0.00332538919859769\\
208	0.00332580309787378\\
209	0.00332622436855776\\
210	0.00332665314116178\\
211	0.003327089548484\\
212	0.00332753372564779\\
213	0.00332798581014162\\
214	0.00332844594185967\\
215	0.0033289142631429\\
216	0.00332939091882108\\
217	0.0033298760562552\\
218	0.00333036982538093\\
219	0.00333087237875247\\
220	0.00333138387158732\\
221	0.00333190446181167\\
222	0.0033324343101066\\
223	0.00333297357995497\\
224	0.00333352243768914\\
225	0.00333408105253948\\
226	0.00333464959668346\\
227	0.00333522824529588\\
228	0.00333581717659961\\
229	0.00333641657191728\\
230	0.00333702661572385\\
231	0.0033376474956999\\
232	0.00333827940278585\\
233	0.0033389225312371\\
234	0.00333957707867991\\
235	0.00334024324616826\\
236	0.00334092123824171\\
237	0.00334161126298399\\
238	0.00334231353208262\\
239	0.00334302826088954\\
240	0.00334375566848259\\
241	0.00334449597772797\\
242	0.00334524941534377\\
243	0.0033460162119644\\
244	0.00334679660220614\\
245	0.0033475908247336\\
246	0.00334839912232735\\
247	0.00334922174195242\\
248	0.00335005893482809\\
249	0.00335091095649854\\
250	0.00335177806690484\\
251	0.00335266053045776\\
252	0.00335355861611198\\
253	0.00335447259744131\\
254	0.00335540275271499\\
255	0.00335634936497527\\
256	0.00335731272211621\\
257	0.0033582931169635\\
258	0.00335929084735563\\
259	0.00336030621622627\\
260	0.00336133953168787\\
261	0.00336239110711638\\
262	0.0033634612612376\\
263	0.00336455031821432\\
264	0.00336565860773516\\
265	0.0033667864651045\\
266	0.0033679342313338\\
267	0.00336910225323429\\
268	0.00337029088351086\\
269	0.00337150048085747\\
270	0.00337273141005392\\
271	0.00337398404206385\\
272	0.00337525875413432\\
273	0.00337655592989667\\
274	0.00337787595946882\\
275	0.00337921923955908\\
276	0.00338058617357127\\
277	0.00338197717171143\\
278	0.00338339265109577\\
279	0.00338483303586036\\
280	0.00338629875727196\\
281	0.00338779025384068\\
282	0.00338930797143373\\
283	0.00339085236339104\\
284	0.00339242389064208\\
285	0.00339402302182419\\
286	0.0033956502334026\\
287	0.00339730600979159\\
288	0.00339899084347746\\
289	0.00340070523514277\\
290	0.00340244969379195\\
291	0.00340422473687872\\
292	0.00340603089043444\\
293	0.00340786868919834\\
294	0.00340973867674884\\
295	0.0034116414056363\\
296	0.00341357743751731\\
297	0.00341554734328991\\
298	0.00341755170323048\\
299	0.00341959110713156\\
300	0.00342166615444105\\
301	0.00342377745440246\\
302	0.00342592562619619\\
303	0.0034281112990819\\
304	0.00343033511254174\\
305	0.00343259771642438\\
306	0.00343489977108982\\
307	0.00343724194755478\\
308	0.00343962492763856\\
309	0.00344204940410927\\
310	0.00344451608083008\\
311	0.00344702567290555\\
312	0.00344957890682761\\
313	0.00345217652062094\\
314	0.00345481926398756\\
315	0.00345750789845017\\
316	0.00346024319749367\\
317	0.00346302594670484\\
318	0.00346585694390895\\
319	0.00346873699930295\\
320	0.00347166693558459\\
321	0.00347464758807583\\
322	0.00347767980484001\\
323	0.00348076444679074\\
324	0.00348390238779121\\
325	0.00348709451474115\\
326	0.00349034172764914\\
327	0.00349364493968681\\
328	0.0034970050772207\\
329	0.0035004230798166\\
330	0.00350389990020994\\
331	0.00350743650423407\\
332	0.00351103387069592\\
333	0.00351469299118633\\
334	0.00351841486980706\\
335	0.00352220052279181\\
336	0.00352605097798613\\
337	0.00352996727412774\\
338	0.00353395045980457\\
339	0.00353800159178545\\
340	0.00354212173185045\\
341	0.00354631193944075\\
342	0.00355057325157021\\
343	0.00355490662209339\\
344	0.00355931272808235\\
345	0.00356379189166622\\
346	0.00356834515469216\\
347	0.00357297356679901\\
348	0.00357767818776288\\
349	0.00358246009188216\\
350	0.00358732037653211\\
351	0.00359226018023511\\
352	0.00359728072439287\\
353	0.00360238341730273\\
354	0.00360757012792604\\
355	0.00361284393236545\\
356	0.00361821119603823\\
357	0.00362367390982745\\
358	0.00362923375132823\\
359	0.00363489242754245\\
360	0.00364065167534496\\
361	0.00364651326193577\\
362	0.00365247898527317\\
363	0.00365855067448048\\
364	0.00366473019021816\\
365	0.00367101942501133\\
366	0.0036774203035199\\
367	0.00368393478273699\\
368	0.00369056485209686\\
369	0.003697312533471\\
370	0.00370417988102549\\
371	0.00371116898090746\\
372	0.00371828195072195\\
373	0.00372552093875149\\
374	0.00373288812286142\\
375	0.00374038570902078\\
376	0.00374801592935452\\
377	0.00375578103962323\\
378	0.00376368331600495\\
379	0.00377172505102493\\
380	0.00377990854844547\\
381	0.00378823611688516\\
382	0.0037967100618844\\
383	0.00380533267606928\\
384	0.00381410622698426\\
385	0.00382303294206365\\
386	0.00383211499008518\\
387	0.00384135445829135\\
388	0.00385075332416488\\
389	0.00386031342059559\\
390	0.00387003639285955\\
391	0.00387992364543375\\
392	0.0038899762761649\\
393	0.0039001949946742\\
394	0.00391058002108038\\
395	0.00392113096013745\\
396	0.00393184664472379\\
397	0.00394272494144054\\
398	0.00395376251048509\\
399	0.00396495449586735\\
400	0.00397629413257958\\
401	0.00398777225813496\\
402	0.00399937669498494\\
403	0.00401109146763229\\
404	0.00402289580756077\\
405	0.00403476288509919\\
406	0.00404665818896517\\
407	0.00405853745010709\\
408	0.00407034397486529\\
409	0.00408200521173058\\
410	0.0040934283265161\\
411	0.00410449451422482\\
412	0.00411505185245532\\
413	0.00412490688083491\\
414	0.00413381680457718\\
415	0.00414246359764626\\
416	0.00415125134065076\\
417	0.00416018176530774\\
418	0.00416925659035271\\
419	0.00417847751893866\\
420	0.00418784624035994\\
421	0.00419736444267878\\
422	0.00420703380198752\\
423	0.00421685584956833\\
424	0.00422683181202162\\
425	0.00423696240349508\\
426	0.00424724815917203\\
427	0.00425768941180764\\
428	0.00426828626551413\\
429	0.00427903856647297\\
430	0.00428994587021759\\
431	0.00430100740509318\\
432	0.00431222203145255\\
433	0.00432358819609439\\
434	0.00433510388148189\\
435	0.00434676654945613\\
436	0.00435857307901268\\
437	0.00437051969760218\\
438	0.00438260190559301\\
439	0.00439481439365558\\
440	0.00440715095302605\\
441	0.00441960437891222\\
442	0.00443216636776108\\
443	0.00444482740977402\\
444	0.00445757667901259\\
445	0.00447040192479705\\
446	0.00448328937001098\\
447	0.00449622362459359\\
448	0.00450918762620207\\
449	0.00452216262508911\\
450	0.00453512823714873\\
451	0.00454806259823822\\
452	0.00456094266652204\\
453	0.00457374474231479\\
454	0.00458644530390559\\
455	0.00459902229070082\\
456	0.00461145699868897\\
457	0.00462373681301679\\
458	0.00463585913954865\\
459	0.00464783703643459\\
460	0.00465970727157742\\
461	0.00467154172123509\\
462	0.00468346221395803\\
463	0.00469554963119461\\
464	0.00470781029851564\\
465	0.00472024469819738\\
466	0.00473285328017813\\
467	0.00474563649024712\\
468	0.00475859480719211\\
469	0.00477172879083086\\
470	0.00478503914318833\\
471	0.00479852678546889\\
472	0.00481219295390058\\
473	0.0048260393178048\\
474	0.00484006812319904\\
475	0.00485428236509176\\
476	0.0048686859910726\\
477	0.00488328413879259\\
478	0.00489808340948448\\
479	0.00491309217733719\\
480	0.00492832090057559\\
481	0.0049437824638965\\
482	0.00495949254182109\\
483	0.0049754699354219\\
484	0.00499173683140299\\
485	0.00500831890355531\\
486	0.00502524513177611\\
487	0.00504254713582935\\
488	0.00506025766691289\\
489	0.0050784076222585\\
490	0.00509702183527401\\
491	0.00511612026777347\\
492	0.00513572427144415\\
493	0.00515585814725386\\
494	0.00517654827885316\\
495	0.005197823043026\\
496	0.0052197129940454\\
497	0.00524225100489788\\
498	0.00526547092274689\\
499	0.00528940925192003\\
500	0.00531410682208839\\
501	0.00533961427475088\\
502	0.00536600662699488\\
503	0.00539333449201868\\
504	0.00542164384722157\\
505	0.00545093317709244\\
506	0.00548125584165916\\
507	0.00551266631417104\\
508	0.00554522086648375\\
509	0.00557897776063199\\
510	0.00561399747215752\\
511	0.00565034295582015\\
512	0.00568807961924574\\
513	0.00572727514130953\\
514	0.0057679989351695\\
515	0.00581032699449293\\
516	0.00585439423092452\\
517	0.00590027932527563\\
518	0.00594806052659907\\
519	0.00599781429473797\\
520	0.00604961359373404\\
521	0.00610352620502252\\
522	0.00615961228747102\\
523	0.00621792029876914\\
524	0.0062784771031797\\
525	0.0063412833477263\\
526	0.00640631260834304\\
527	0.00647349981735246\\
528	0.00654273540940123\\
529	0.00661386712290187\\
530	0.0066866853889961\\
531	0.00676099618491231\\
532	0.00683645871329546\\
533	0.00691251557173578\\
534	0.00698820960614814\\
535	0.0070600184570712\\
536	0.00712619182055303\\
537	0.00718635518837399\\
538	0.00724038537103614\\
539	0.0072891926067398\\
540	0.00733611246284593\\
541	0.00738175075357763\\
542	0.00742656905747836\\
543	0.00747111430771151\\
544	0.00751581455378255\\
545	0.00756089746622922\\
546	0.00760651824603813\\
547	0.00765279954637135\\
548	0.00769981995261393\\
549	0.00774763076508289\\
550	0.00779627233522115\\
551	0.00784578270503273\\
552	0.00789623511963474\\
553	0.00794771179525126\\
554	0.00800011671826545\\
555	0.00805165496701063\\
556	0.00810209881618632\\
557	0.0081511534873175\\
558	0.00820039187507906\\
559	0.00824993373640694\\
560	0.00829982916153586\\
561	0.00835013282549448\\
562	0.00840084161324492\\
563	0.00845191747075763\\
564	0.00850331152133759\\
565	0.0085549989209369\\
566	0.00860690126516819\\
567	0.00865718321445381\\
568	0.00870571147056739\\
569	0.00875412101410349\\
570	0.00880247830324436\\
571	0.00885082904818314\\
572	0.00889914614462142\\
573	0.00894739710985454\\
574	0.00899525183449353\\
575	0.00904253590053291\\
576	0.00908973605534923\\
577	0.0091368486117437\\
578	0.00918384152338781\\
579	0.00923066477496741\\
580	0.00927726586124061\\
581	0.00932359110588857\\
582	0.00936958646951639\\
583	0.0094151985649821\\
584	0.00946037597636478\\
585	0.00950507095687452\\
586	0.00954924158529672\\
587	0.0095928544549437\\
588	0.00963588792408897\\
589	0.00967833579817709\\
590	0.00972021084271539\\
591	0.00976138933978722\\
592	0.00980171928775089\\
593	0.00984102135019415\\
594	0.00987905595166785\\
595	0.00991543381058343\\
596	0.00994937493726701\\
597	0.00997906286423442\\
598	0.0099999191923403\\
599	0\\
600	0\\
};
\addplot [color=red!80!mycolor19,solid,forget plot]
  table[row sep=crcr]{%
1	0.00391309393964753\\
2	0.00391310035145049\\
3	0.00391310687863073\\
4	0.00391311352326403\\
5	0.0039131202874635\\
6	0.00391312717338031\\
7	0.00391313418320423\\
8	0.00391314131916451\\
9	0.00391314858353044\\
10	0.00391315597861217\\
11	0.00391316350676137\\
12	0.00391317117037203\\
13	0.00391317897188121\\
14	0.00391318691376976\\
15	0.00391319499856321\\
16	0.00391320322883247\\
17	0.00391321160719473\\
18	0.00391322013631422\\
19	0.0039132288189031\\
20	0.00391323765772234\\
21	0.00391324665558253\\
22	0.00391325581534482\\
23	0.00391326513992186\\
24	0.00391327463227862\\
25	0.00391328429543347\\
26	0.003913294132459\\
27	0.00391330414648312\\
28	0.00391331434068997\\
29	0.00391332471832095\\
30	0.00391333528267575\\
31	0.00391334603711346\\
32	0.00391335698505355\\
33	0.00391336812997696\\
34	0.00391337947542727\\
35	0.00391339102501179\\
36	0.00391340278240267\\
37	0.00391341475133813\\
38	0.00391342693562365\\
39	0.00391343933913308\\
40	0.00391345196580999\\
41	0.00391346481966886\\
42	0.00391347790479635\\
43	0.00391349122535264\\
44	0.00391350478557267\\
45	0.00391351858976759\\
46	0.0039135326423261\\
47	0.00391354694771575\\
48	0.00391356151048447\\
49	0.00391357633526195\\
50	0.00391359142676115\\
51	0.0039136067897798\\
52	0.00391362242920189\\
53	0.0039136383499992\\
54	0.00391365455723295\\
55	0.00391367105605535\\
56	0.00391368785171129\\
57	0.00391370494953988\\
58	0.00391372235497636\\
59	0.00391374007355357\\
60	0.00391375811090393\\
61	0.00391377647276107\\
62	0.00391379516496177\\
63	0.00391381419344772\\
64	0.00391383356426739\\
65	0.00391385328357808\\
66	0.00391387335764775\\
67	0.00391389379285705\\
68	0.00391391459570137\\
69	0.00391393577279287\\
70	0.00391395733086256\\
71	0.00391397927676248\\
72	0.00391400161746784\\
73	0.00391402436007925\\
74	0.00391404751182495\\
75	0.00391407108006316\\
76	0.00391409507228431\\
77	0.00391411949611354\\
78	0.00391414435931302\\
79	0.00391416966978445\\
80	0.00391419543557158\\
81	0.00391422166486273\\
82	0.00391424836599338\\
83	0.00391427554744888\\
84	0.00391430321786703\\
85	0.00391433138604094\\
86	0.00391436006092166\\
87	0.00391438925162124\\
88	0.00391441896741536\\
89	0.00391444921774651\\
90	0.00391448001222679\\
91	0.00391451136064106\\
92	0.00391454327295003\\
93	0.00391457575929336\\
94	0.00391460882999292\\
95	0.00391464249555611\\
96	0.00391467676667905\\
97	0.00391471165425006\\
98	0.0039147471693531\\
99	0.00391478332327129\\
100	0.00391482012749046\\
101	0.00391485759370267\\
102	0.00391489573381023\\
103	0.00391493455992908\\
104	0.00391497408439289\\
105	0.00391501431975687\\
106	0.00391505527880174\\
107	0.0039150969745378\\
108	0.00391513942020896\\
109	0.00391518262929707\\
110	0.00391522661552606\\
111	0.00391527139286634\\
112	0.00391531697553919\\
113	0.00391536337802129\\
114	0.00391541061504923\\
115	0.00391545870162422\\
116	0.00391550765301681\\
117	0.00391555748477168\\
118	0.00391560821271267\\
119	0.00391565985294758\\
120	0.00391571242187345\\
121	0.00391576593618165\\
122	0.00391582041286306\\
123	0.00391587586921361\\
124	0.0039159323228396\\
125	0.00391598979166334\\
126	0.00391604829392865\\
127	0.00391610784820688\\
128	0.00391616847340245\\
129	0.00391623018875904\\
130	0.00391629301386557\\
131	0.00391635696866237\\
132	0.00391642207344744\\
133	0.00391648834888292\\
134	0.0039165558160015\\
135	0.00391662449621307\\
136	0.0039166944113115\\
137	0.00391676558348142\\
138	0.0039168380353053\\
139	0.00391691178977044\\
140	0.00391698687027623\\
141	0.00391706330064156\\
142	0.0039171411051122\\
143	0.00391722030836855\\
144	0.00391730093553329\\
145	0.0039173830121792\\
146	0.00391746656433737\\
147	0.00391755161850521\\
148	0.00391763820165484\\
149	0.00391772634124151\\
150	0.00391781606521217\\
151	0.00391790740201436\\
152	0.00391800038060491\\
153	0.00391809503045919\\
154	0.00391819138158023\\
155	0.00391828946450815\\
156	0.00391838931032967\\
157	0.00391849095068785\\
158	0.00391859441779201\\
159	0.00391869974442765\\
160	0.00391880696396692\\
161	0.0039189161103788\\
162	0.0039190272182398\\
163	0.0039191403227448\\
164	0.00391925545971783\\
165	0.00391937266562344\\
166	0.0039194919775779\\
167	0.0039196134333608\\
168	0.00391973707142682\\
169	0.00391986293091758\\
170	0.003919991051674\\
171	0.00392012147424852\\
172	0.00392025423991759\\
173	0.0039203893906947\\
174	0.00392052696934325\\
175	0.00392066701938984\\
176	0.00392080958513773\\
177	0.00392095471168059\\
178	0.00392110244491633\\
179	0.00392125283156142\\
180	0.00392140591916523\\
181	0.00392156175612471\\
182	0.00392172039169928\\
183	0.00392188187602615\\
184	0.00392204626013556\\
185	0.00392221359596661\\
186	0.00392238393638313\\
187	0.00392255733519008\\
188	0.00392273384714987\\
189	0.00392291352799926\\
190	0.00392309643446646\\
191	0.0039232826242884\\
192	0.00392347215622846\\
193	0.00392366509009446\\
194	0.00392386148675686\\
195	0.00392406140816729\\
196	0.00392426491737753\\
197	0.00392447207855869\\
198	0.00392468295702064\\
199	0.00392489761923196\\
200	0.00392511613284003\\
201	0.00392533856669158\\
202	0.00392556499085358\\
203	0.00392579547663438\\
204	0.00392603009660527\\
205	0.00392626892462241\\
206	0.00392651203584904\\
207	0.00392675950677819\\
208	0.00392701141525566\\
209	0.00392726784050333\\
210	0.00392752886314305\\
211	0.00392779456522076\\
212	0.00392806503023093\\
213	0.00392834034314166\\
214	0.00392862059041988\\
215	0.00392890586005726\\
216	0.00392919624159628\\
217	0.00392949182615693\\
218	0.00392979270646361\\
219	0.00393009897687269\\
220	0.00393041073340037\\
221	0.00393072807375105\\
222	0.00393105109734605\\
223	0.00393137990535299\\
224	0.00393171460071542\\
225	0.00393205528818294\\
226	0.00393240207434204\\
227	0.00393275506764708\\
228	0.00393311437845195\\
229	0.00393348011904223\\
230	0.00393385240366776\\
231	0.00393423134857578\\
232	0.00393461707204458\\
233	0.00393500969441757\\
234	0.00393540933813807\\
235	0.00393581612778442\\
236	0.0039362301901057\\
237	0.0039366516540581\\
238	0.00393708065084169\\
239	0.00393751731393771\\
240	0.00393796177914665\\
241	0.0039384141846267\\
242	0.00393887467093278\\
243	0.00393934338105621\\
244	0.00393982046046497\\
245	0.00394030605714449\\
246	0.00394080032163905\\
247	0.00394130340709382\\
248	0.00394181546929747\\
249	0.00394233666672541\\
250	0.00394286716058364\\
251	0.00394340711485321\\
252	0.00394395669633537\\
253	0.00394451607469714\\
254	0.00394508542251787\\
255	0.00394566491533617\\
256	0.00394625473169753\\
257	0.00394685505320269\\
258	0.00394746606455656\\
259	0.00394808795361791\\
260	0.00394872091144956\\
261	0.00394936513236943\\
262	0.00395002081400208\\
263	0.00395068815733114\\
264	0.00395136736675211\\
265	0.0039520586501262\\
266	0.00395276221883456\\
267	0.00395347828783326\\
268	0.00395420707570917\\
269	0.0039549488047362\\
270	0.00395570370093242\\
271	0.00395647199411784\\
272	0.00395725391797283\\
273	0.0039580497100973\\
274	0.00395885961207044\\
275	0.00395968386951128\\
276	0.00396052273213981\\
277	0.00396137645383877\\
278	0.00396224529271628\\
279	0.00396312951116883\\
280	0.00396402937594523\\
281	0.00396494515821093\\
282	0.00396587713361319\\
283	0.00396682558234666\\
284	0.00396779078921978\\
285	0.00396877304372167\\
286	0.00396977264008943\\
287	0.00397078987737639\\
288	0.00397182505952052\\
289	0.00397287849541347\\
290	0.00397395049897024\\
291	0.00397504138919915\\
292	0.00397615149027226\\
293	0.00397728113159628\\
294	0.0039784306478837\\
295	0.00397960037922457\\
296	0.00398079067115801\\
297	0.00398200187474465\\
298	0.00398323434663867\\
299	0.00398448844916045\\
300	0.00398576455036895\\
301	0.00398706302413444\\
302	0.003988384250211\\
303	0.00398972861430908\\
304	0.00399109650816784\\
305	0.00399248832962729\\
306	0.0039939044827002\\
307	0.00399534537764357\\
308	0.00399681143102979\\
309	0.00399830306581716\\
310	0.00399982071141991\\
311	0.0040013648037776\\
312	0.00400293578542367\\
313	0.0040045341055533\\
314	0.00400616022009033\\
315	0.00400781459175329\\
316	0.00400949769012048\\
317	0.00401120999169397\\
318	0.00401295197996263\\
319	0.00401472414546422\\
320	0.00401652698584636\\
321	0.00401836100592681\\
322	0.00402022671775282\\
323	0.00402212464066014\\
324	0.00402405530133158\\
325	0.00402601923385591\\
326	0.00402801697978755\\
327	0.00403004908820754\\
328	0.00403211611578721\\
329	0.00403421862685557\\
330	0.00403635719347254\\
331	0.00403853239550991\\
332	0.00404074482074375\\
333	0.00404299506496178\\
334	0.00404528373209185\\
335	0.00404761143435832\\
336	0.00404997879247634\\
337	0.00405238643589678\\
338	0.00405483500311943\\
339	0.00405732514209878\\
340	0.00405985751078104\\
341	0.00406243277784526\\
342	0.00406505162382405\\
343	0.00406771474309292\\
344	0.00407042284814956\\
345	0.0040731766793176\\
346	0.00407597701430165\\
347	0.00407882465138177\\
348	0.00408172041067304\\
349	0.00408466513531158\\
350	0.00408765969244745\\
351	0.00409070497383421\\
352	0.00409380189560034\\
353	0.00409695139626494\\
354	0.00410015443066837\\
355	0.00410341195380954\\
356	0.00410672487967\\
357	0.00411009400125772\\
358	0.00411352010226292\\
359	0.00411700396460171\\
360	0.00412054636708436\\
361	0.00412414808393454\\
362	0.004127809883143\\
363	0.00413153252463749\\
364	0.0041353167582489\\
365	0.00413916332145065\\
366	0.0041430729368464\\
367	0.00414704630937751\\
368	0.0041510841232191\\
369	0.00415518703832901\\
370	0.00415935568661078\\
371	0.00416359066764624\\
372	0.00416789254394891\\
373	0.00417226183568325\\
374	0.00417669901478839\\
375	0.00418120449843865\\
376	0.00418577864176438\\
377	0.00419042172974946\\
378	0.00419513396821128\\
379	0.00419991547376057\\
380	0.00420476626262653\\
381	0.00420968623822302\\
382	0.00421467517731951\\
383	0.00421973271466958\\
384	0.00422485832593976\\
385	0.0042300513087723\\
386	0.0042353107618099\\
387	0.00424063556150867\\
388	0.00424602433657202\\
389	0.00425147543985365\\
390	0.00425698691761027\\
391	0.00426255647603806\\
392	0.00426818144511206\\
393	0.00427385873987608\\
394	0.00427958481952026\\
395	0.00428535564485324\\
396	0.00429116663515453\\
397	0.00429701262590292\\
398	0.00430288782952713\\
399	0.00430878580264344\\
400	0.00431469942497536\\
401	0.00432062089684156\\
402	0.00432654176479418\\
403	0.00433245298898998\\
404	0.00433834507115593\\
405	0.00434420826923577\\
406	0.00435003293470858\\
407	0.00435581002220494\\
408	0.00436153184002012\\
409	0.00436719313706548\\
410	0.00437279266132182\\
411	0.00437833538472884\\
412	0.00438383567545592\\
413	0.00438932177878079\\
414	0.00439484173526621\\
415	0.00440043693996867\\
416	0.00440611959258895\\
417	0.00441189048639354\\
418	0.00441775037617439\\
419	0.00442369997286494\\
420	0.00442973993741127\\
421	0.00443587087335271\\
422	0.00444209331871319\\
423	0.00444840774111909\\
424	0.00445481453771689\\
425	0.0044613140414695\\
426	0.00446790651697064\\
427	0.0044745921563337\\
428	0.00448137107521975\\
429	0.0044882433090829\\
430	0.00449520880974954\\
431	0.00450226744247634\\
432	0.00450941898366786\\
433	0.00451666311947818\\
434	0.00452399944557325\\
435	0.00453142746838399\\
436	0.00453894660824531\\
437	0.00454655620490727\\
438	0.00455425552600115\\
439	0.00456204377916066\\
440	0.00456992012862854\\
441	0.00457788371733072\\
442	0.00458593369556622\\
443	0.00459406925764097\\
444	0.00460228968796137\\
445	0.00461059441828207\\
446	0.00461898309795326\\
447	0.00462745567909243\\
448	0.00463601251855382\\
449	0.00464465449905457\\
450	0.00465338317019809\\
451	0.00466220090898406\\
452	0.00467111109790705\\
453	0.00468011831560007\\
454	0.00468922852994733\\
455	0.00469844927614577\\
456	0.00470778979203129\\
457	0.0047172610686496\\
458	0.00472687575038563\\
459	0.00473664778038361\\
460	0.00474659161162588\\
461	0.00475672065104738\\
462	0.00476704443458528\\
463	0.00477756856208024\\
464	0.00478829864180216\\
465	0.00479924063984959\\
466	0.00481040087052541\\
467	0.00482178602364282\\
468	0.00483340319515331\\
469	0.00484525992006689\\
470	0.00485736420771812\\
471	0.004869724579392\\
472	0.00488235010823822\\
473	0.00489525046131628\\
474	0.00490843594349121\\
475	0.00492191754271023\\
476	0.00493570697592564\\
477	0.00494981673455072\\
478	0.00496426012906006\\
479	0.00497905133959719\\
480	0.00499420641614639\\
481	0.00500974301389922\\
482	0.00502567987587271\\
483	0.00504203681804972\\
484	0.00505883469393796\\
485	0.00507609533954526\\
486	0.00509384150344459\\
487	0.0051120967723867\\
488	0.00513088551504203\\
489	0.00515023291574502\\
490	0.00517016521560303\\
491	0.00519070995746875\\
492	0.00521189574655456\\
493	0.00523370568230926\\
494	0.00525615797345135\\
495	0.00527928085048286\\
496	0.00530310526724121\\
497	0.00532766854668698\\
498	0.00535303735342734\\
499	0.00537925241866236\\
500	0.0054063556515977\\
501	0.00543439127309023\\
502	0.00546340557950195\\
503	0.00549344603574776\\
504	0.00552456889504518\\
505	0.00555688280214329\\
506	0.00559044812484436\\
507	0.00562532780659876\\
508	0.00566158726887005\\
509	0.00569929423683336\\
510	0.00573851846062438\\
511	0.00577933131156622\\
512	0.00582180523855337\\
513	0.0058660130791357\\
514	0.00591202733005102\\
515	0.00595991936941512\\
516	0.00600975518857396\\
517	0.00606159220849871\\
518	0.00611547879735186\\
519	0.00617145111727593\\
520	0.00622952827222556\\
521	0.00628970070576879\\
522	0.00635192430572301\\
523	0.00641612600430055\\
524	0.00648218785696528\\
525	0.00654993917002101\\
526	0.0066191621183003\\
527	0.00668963241587316\\
528	0.00676094353650896\\
529	0.00683211407028415\\
530	0.00690211112710542\\
531	0.00696671445942419\\
532	0.00702546965301922\\
533	0.00707812424877317\\
534	0.00712481587446659\\
535	0.00716849059922027\\
536	0.00721077696089307\\
537	0.00725210040069124\\
538	0.00729298041879735\\
539	0.00733394341899069\\
540	0.00737527202531301\\
541	0.00741711243910963\\
542	0.00745958050016931\\
543	0.00750275904121561\\
544	0.00754670141587454\\
545	0.00759144767135124\\
546	0.00763703054234583\\
547	0.00768347622976199\\
548	0.00773080823398957\\
549	0.00777909452938988\\
550	0.00782841534682607\\
551	0.00787862213957222\\
552	0.00792795218358609\\
553	0.00797616861463886\\
554	0.00802315876012582\\
555	0.00807047849253736\\
556	0.00811816536939713\\
557	0.0081662917589018\\
558	0.00821491588109375\\
559	0.00826402041714991\\
560	0.00831357679383578\\
561	0.00836354672564337\\
562	0.00841388288403202\\
563	0.00846455795573843\\
564	0.00851554903587025\\
565	0.00856510622084549\\
566	0.00861296704327435\\
567	0.00866079323060213\\
568	0.0087086623118578\\
569	0.00875662026486645\\
570	0.00880463250812337\\
571	0.0088526615516325\\
572	0.00890067654903162\\
573	0.00894813872224893\\
574	0.00899535238494104\\
575	0.00904255767963824\\
576	0.00908973840022921\\
577	0.00913684942824384\\
578	0.00918384194182773\\
579	0.00923066499370698\\
580	0.00927726597097645\\
581	0.00932359115751704\\
582	0.00936958649189013\\
583	0.00941519857373444\\
584	0.00946037597937255\\
585	0.00950507095774694\\
586	0.00954924158549703\\
587	0.00959285445497603\\
588	0.00963588792409171\\
589	0.00967833579817709\\
590	0.00972021084271539\\
591	0.00976138933978722\\
592	0.00980171928775089\\
593	0.00984102135019415\\
594	0.00987905595166784\\
595	0.00991543381058343\\
596	0.00994937493726701\\
597	0.00997906286423442\\
598	0.0099999191923403\\
599	0\\
600	0\\
};
\addplot [color=red,solid,forget plot]
  table[row sep=crcr]{%
1	0.00417141729594836\\
2	0.0041714212375612\\
3	0.00417142525044982\\
4	0.00417142933590415\\
5	0.00417143349523751\\
6	0.00417143772978697\\
7	0.00417144204091389\\
8	0.00417144643000426\\
9	0.00417145089846921\\
10	0.00417145544774538\\
11	0.00417146007929553\\
12	0.00417146479460882\\
13	0.00417146959520149\\
14	0.00417147448261719\\
15	0.00417147945842758\\
16	0.00417148452423281\\
17	0.004171489681662\\
18	0.00417149493237377\\
19	0.00417150027805687\\
20	0.0041715057204306\\
21	0.00417151126124547\\
22	0.00417151690228366\\
23	0.00417152264535971\\
24	0.00417152849232102\\
25	0.00417153444504848\\
26	0.00417154050545709\\
27	0.00417154667549652\\
28	0.0041715529571518\\
29	0.00417155935244397\\
30	0.00417156586343074\\
31	0.00417157249220702\\
32	0.00417157924090577\\
33	0.00417158611169864\\
34	0.00417159310679659\\
35	0.0041716002284507\\
36	0.0041716074789529\\
37	0.00417161486063662\\
38	0.00417162237587761\\
39	0.00417163002709472\\
40	0.00417163781675066\\
41	0.00417164574735278\\
42	0.00417165382145392\\
43	0.00417166204165321\\
44	0.00417167041059691\\
45	0.0041716789309793\\
46	0.00417168760554341\\
47	0.00417169643708219\\
48	0.00417170542843912\\
49	0.00417171458250933\\
50	0.00417172390224041\\
51	0.00417173339063343\\
52	0.00417174305074389\\
53	0.00417175288568275\\
54	0.00417176289861736\\
55	0.00417177309277255\\
56	0.00417178347143162\\
57	0.00417179403793749\\
58	0.00417180479569368\\
59	0.00417181574816551\\
60	0.00417182689888108\\
61	0.0041718382514326\\
62	0.00417184980947735\\
63	0.00417186157673903\\
64	0.00417187355700894\\
65	0.00417188575414711\\
66	0.00417189817208365\\
67	0.00417191081481996\\
68	0.00417192368643009\\
69	0.00417193679106193\\
70	0.00417195013293873\\
71	0.00417196371636035\\
72	0.0041719775457047\\
73	0.00417199162542913\\
74	0.00417200596007185\\
75	0.00417202055425349\\
76	0.00417203541267852\\
77	0.0041720505401368\\
78	0.00417206594150512\\
79	0.00417208162174879\\
80	0.00417209758592323\\
81	0.00417211383917564\\
82	0.00417213038674667\\
83	0.00417214723397208\\
84	0.00417216438628448\\
85	0.00417218184921508\\
86	0.00417219962839556\\
87	0.00417221772955976\\
88	0.00417223615854567\\
89	0.00417225492129723\\
90	0.00417227402386632\\
91	0.00417229347241466\\
92	0.00417231327321582\\
93	0.00417233343265732\\
94	0.00417235395724267\\
95	0.00417237485359332\\
96	0.00417239612845109\\
97	0.00417241778868015\\
98	0.00417243984126929\\
99	0.00417246229333417\\
100	0.00417248515211969\\
101	0.00417250842500232\\
102	0.00417253211949238\\
103	0.00417255624323666\\
104	0.00417258080402071\\
105	0.00417260580977149\\
106	0.0041726312685599\\
107	0.00417265718860336\\
108	0.00417268357826855\\
109	0.00417271044607403\\
110	0.00417273780069307\\
111	0.00417276565095637\\
112	0.00417279400585506\\
113	0.00417282287454349\\
114	0.00417285226634222\\
115	0.00417288219074112\\
116	0.00417291265740226\\
117	0.00417294367616329\\
118	0.00417297525704035\\
119	0.00417300741023158\\
120	0.00417304014612021\\
121	0.004173073475278\\
122	0.00417310740846868\\
123	0.00417314195665143\\
124	0.00417317713098439\\
125	0.00417321294282827\\
126	0.00417324940375013\\
127	0.00417328652552689\\
128	0.0041733243201494\\
129	0.00417336279982615\\
130	0.0041734019769873\\
131	0.00417344186428863\\
132	0.00417348247461566\\
133	0.00417352382108783\\
134	0.00417356591706271\\
135	0.00417360877614035\\
136	0.00417365241216765\\
137	0.00417369683924282\\
138	0.00417374207171994\\
139	0.00417378812421355\\
140	0.00417383501160351\\
141	0.00417388274903964\\
142	0.00417393135194666\\
143	0.00417398083602919\\
144	0.00417403121727675\\
145	0.00417408251196902\\
146	0.00417413473668098\\
147	0.00417418790828828\\
148	0.00417424204397271\\
149	0.00417429716122767\\
150	0.00417435327786387\\
151	0.00417441041201495\\
152	0.00417446858214343\\
153	0.0041745278070466\\
154	0.00417458810586256\\
155	0.00417464949807632\\
156	0.00417471200352611\\
157	0.00417477564240979\\
158	0.00417484043529119\\
159	0.00417490640310686\\
160	0.00417497356717269\\
161	0.00417504194919076\\
162	0.00417511157125627\\
163	0.00417518245586462\\
164	0.00417525462591869\\
165	0.00417532810473602\\
166	0.00417540291605636\\
167	0.00417547908404922\\
168	0.0041755566333216\\
169	0.00417563558892583\\
170	0.00417571597636753\\
171	0.00417579782161376\\
172	0.00417588115110138\\
173	0.00417596599174526\\
174	0.00417605237094702\\
175	0.00417614031660364\\
176	0.00417622985711639\\
177	0.00417632102139971\\
178	0.00417641383889057\\
179	0.00417650833955761\\
180	0.00417660455391075\\
181	0.00417670251301075\\
182	0.00417680224847913\\
183	0.004176903792508\\
184	0.00417700717787039\\
185	0.00417711243793043\\
186	0.00417721960665394\\
187	0.00417732871861905\\
188	0.00417743980902709\\
189	0.00417755291371368\\
190	0.00417766806915986\\
191	0.0041777853125036\\
192	0.00417790468155139\\
193	0.00417802621479001\\
194	0.00417814995139857\\
195	0.00417827593126074\\
196	0.00417840419497708\\
197	0.00417853478387773\\
198	0.00417866774003521\\
199	0.00417880310627744\\
200	0.00417894092620098\\
201	0.00417908124418459\\
202	0.00417922410540279\\
203	0.00417936955583983\\
204	0.00417951764230381\\
205	0.00417966841244109\\
206	0.00417982191475086\\
207	0.00417997819859995\\
208	0.00418013731423793\\
209	0.00418029931281241\\
210	0.00418046424638457\\
211	0.00418063216794502\\
212	0.00418080313142981\\
213	0.00418097719173671\\
214	0.00418115440474184\\
215	0.00418133482731642\\
216	0.00418151851734385\\
217	0.00418170553373708\\
218	0.00418189593645622\\
219	0.00418208978652635\\
220	0.00418228714605576\\
221	0.0041824880782543\\
222	0.0041826926474521\\
223	0.00418290091911855\\
224	0.00418311295988156\\
225	0.00418332883754708\\
226	0.00418354862111898\\
227	0.00418377238081907\\
228	0.00418400018810767\\
229	0.00418423211570418\\
230	0.00418446823760813\\
231	0.00418470862912047\\
232	0.00418495336686518\\
233	0.0041852025288112\\
234	0.00418545619429455\\
235	0.00418571444404094\\
236	0.00418597736018854\\
237	0.00418624502631107\\
238	0.00418651752744131\\
239	0.00418679495009489\\
240	0.00418707738229421\\
241	0.0041873649135929\\
242	0.00418765763510055\\
243	0.00418795563950764\\
244	0.00418825902111096\\
245	0.00418856787583913\\
246	0.00418888230127873\\
247	0.00418920239670044\\
248	0.0041895282630857\\
249	0.0041898600031536\\
250	0.00419019772138816\\
251	0.00419054152406586\\
252	0.0041908915192835\\
253	0.00419124781698645\\
254	0.00419161052899709\\
255	0.00419197976904371\\
256	0.00419235565278956\\
257	0.00419273829786239\\
258	0.00419312782388419\\
259	0.00419352435250128\\
260	0.00419392800741462\\
261	0.00419433891441072\\
262	0.00419475720139237\\
263	0.00419518299841011\\
264	0.00419561643769389\\
265	0.00419605765368487\\
266	0.00419650678306768\\
267	0.00419696396480297\\
268	0.00419742934016015\\
269	0.00419790305275047\\
270	0.0041983852485605\\
271	0.00419887607598572\\
272	0.00419937568586451\\
273	0.00419988423151231\\
274	0.00420040186875631\\
275	0.00420092875596998\\
276	0.00420146505410838\\
277	0.00420201092674335\\
278	0.00420256654009912\\
279	0.00420313206308821\\
280	0.00420370766734756\\
281	0.00420429352727493\\
282	0.00420488982006546\\
283	0.00420549672574871\\
284	0.00420611442722573\\
285	0.00420674311030644\\
286	0.00420738296374742\\
287	0.00420803417928973\\
288	0.00420869695169706\\
289	0.00420937147879415\\
290	0.00421005796150549\\
291	0.00421075660389411\\
292	0.00421146761320083\\
293	0.00421219119988342\\
294	0.00421292757765646\\
295	0.00421367696353093\\
296	0.00421443957785442\\
297	0.00421521564435124\\
298	0.00421600539016304\\
299	0.00421680904588942\\
300	0.00421762684562882\\
301	0.00421845902701961\\
302	0.00421930583128124\\
303	0.00422016750325577\\
304	0.0042210442914492\\
305	0.00422193644807332\\
306	0.00422284422908731\\
307	0.00422376789423958\\
308	0.00422470770710954\\
309	0.00422566393514946\\
310	0.00422663684972616\\
311	0.00422762672616268\\
312	0.00422863384377971\\
313	0.00422965848593693\\
314	0.0042307009400738\\
315	0.00423176149775028\\
316	0.00423284045468683\\
317	0.00423393811080408\\
318	0.00423505477026159\\
319	0.00423619074149609\\
320	0.00423734633725874\\
321	0.0042385218746514\\
322	0.00423971767516177\\
323	0.0042409340646973\\
324	0.00424217137361771\\
325	0.00424342993676598\\
326	0.00424471009349763\\
327	0.00424601218770806\\
328	0.00424733656785803\\
329	0.00424868358699681\\
330	0.00425005360278272\\
331	0.0042514469775011\\
332	0.00425286407807892\\
333	0.00425430527609568\\
334	0.00425577094779001\\
335	0.00425726147406045\\
336	0.0042587772404593\\
337	0.00426031863717721\\
338	0.00426188605901504\\
339	0.0042634799053389\\
340	0.00426510058001185\\
341	0.00426674849129432\\
342	0.00426842405170062\\
343	0.00427012767778342\\
344	0.00427185978975912\\
345	0.00427362081076799\\
346	0.00427541116576958\\
347	0.00427723128085833\\
348	0.00427908158246891\\
349	0.00428096249646347\\
350	0.0042828744470972\\
351	0.00428481785586465\\
352	0.00428679314023817\\
353	0.00428880071232879\\
354	0.00429084097756436\\
355	0.00429291433371471\\
356	0.0042950211712544\\
357	0.00429716187620316\\
358	0.00429933682967659\\
359	0.00430154640713625\\
360	0.00430379097758998\\
361	0.00430607090273958\\
362	0.00430838653607358\\
363	0.00431073822190244\\
364	0.00431312629433371\\
365	0.00431555107618505\\
366	0.00431801287783277\\
367	0.0043205119959941\\
368	0.00432304871244172\\
369	0.00432562329264953\\
370	0.00432823598436941\\
371	0.00433088701613951\\
372	0.00433357659572557\\
373	0.00433630490849874\\
374	0.00433907211575428\\
375	0.00434187835297895\\
376	0.0043447237280762\\
377	0.00434760831956328\\
378	0.0043505321747576\\
379	0.00435349530797551\\
380	0.00435649769877283\\
381	0.00435953929026478\\
382	0.00436261998757205\\
383	0.00436573965645217\\
384	0.00436889812218877\\
385	0.00437209516882886\\
386	0.00437533053887853\\
387	0.00437860393359149\\
388	0.00438191501401419\\
389	0.00438526340298479\\
390	0.0043886486883227\\
391	0.00439207042749119\\
392	0.00439552815406753\\
393	0.0043990213864128\\
394	0.00440254963899731\\
395	0.00440611243690426\\
396	0.00440970933410266\\
397	0.00441333993614392\\
398	0.00441700392799374\\
399	0.00442070110772882\\
400	0.00442443142677545\\
401	0.00442819503726806\\
402	0.00443199234687599\\
403	0.00443582408101143\\
404	0.00443969135159781\\
405	0.00444359573040631\\
406	0.00444753932316088\\
407	0.0044515248379043\\
408	0.00445555563713975\\
409	0.00445963575746238\\
410	0.0044637698717059\\
411	0.00446796315433237\\
412	0.00447222098358322\\
413	0.00447654835910707\\
414	0.00448094884530292\\
415	0.00448542420369106\\
416	0.00448997560015105\\
417	0.00449460422115946\\
418	0.00449931127504153\\
419	0.00450409799335963\\
420	0.00450896563245742\\
421	0.00451391547521329\\
422	0.00451894883305504\\
423	0.00452406704818085\\
424	0.00452927149581646\\
425	0.00453456358633038\\
426	0.00453994476757048\\
427	0.00454541652748043\\
428	0.00455098039742705\\
429	0.0045566379564885\\
430	0.00456239083633436\\
431	0.00456824072677151\\
432	0.00457418938203479\\
433	0.00458023862790652\\
434	0.00458639036975255\\
435	0.00459264660156371\\
436	0.00459900941609112\\
437	0.00460548101615734\\
438	0.00461206372721509\\
439	0.00461876001120466\\
440	0.00462557248173187\\
441	0.00463250392054477\\
442	0.00463955729522559\\
443	0.00464673577793208\\
444	0.00465404276491351\\
445	0.00466148189639597\\
446	0.0046690570762898\\
447	0.00467677249104362\\
448	0.00468463262666304\\
449	0.00469264226005057\\
450	0.00470080645466231\\
451	0.00470913056139528\\
452	0.00471762021278094\\
453	0.00472628130882881\\
454	0.00473511999293577\\
455	0.00474414261666887\\
456	0.00475335569311759\\
457	0.00476276583994915\\
458	0.00477237971546353\\
459	0.00478220395394506\\
460	0.00479224511277774\\
461	0.00480250966573844\\
462	0.0048130041410784\\
463	0.00482373531637173\\
464	0.00483471025564672\\
465	0.00484593634938176\\
466	0.00485742262810758\\
467	0.00486917868065496\\
468	0.0048812146512215\\
469	0.0048935412725203\\
470	0.00490616989892637\\
471	0.00491911253884136\\
472	0.00493238188599012\\
473	0.00494599134833053\\
474	0.00495995507289613\\
475	0.00497428796445237\\
476	0.00498900569361151\\
477	0.00500412468545673\\
478	0.00501966204991506\\
479	0.00503563534260202\\
480	0.00505203203850175\\
481	0.00506885035320858\\
482	0.00508610647732088\\
483	0.00510381737352443\\
484	0.00512200081946964\\
485	0.00514067545717838\\
486	0.00515986085050313\\
487	0.0051795775543532\\
488	0.0051998472524993\\
489	0.00522069291948455\\
490	0.00524213893576288\\
491	0.00526421036697117\\
492	0.0052869337710113\\
493	0.00531038708338049\\
494	0.00533463065977775\\
495	0.00535972170666477\\
496	0.00538570357433848\\
497	0.00541262217930969\\
498	0.00544052593988315\\
499	0.00546946488375426\\
500	0.0054994915755224\\
501	0.0055306611806234\\
502	0.00556303151660603\\
503	0.00559666327700001\\
504	0.00563162031738346\\
505	0.00566796721508761\\
506	0.00570576840772046\\
507	0.00574508942737241\\
508	0.00578599628828108\\
509	0.00582855467622107\\
510	0.00587282886678455\\
511	0.00591888013540978\\
512	0.00596676496370957\\
513	0.0060165327491921\\
514	0.00606822295732727\\
515	0.00612186193541779\\
516	0.00617745923946957\\
517	0.00623501086570649\\
518	0.0062944782534695\\
519	0.00635577778146541\\
520	0.0064187788043303\\
521	0.00648329513000825\\
522	0.00654909551745368\\
523	0.00661556809570386\\
524	0.00668225778840485\\
525	0.00674845423256209\\
526	0.00681265527495289\\
527	0.00687114274886836\\
528	0.00692366259412663\\
529	0.00697041755680642\\
530	0.00701168306378662\\
531	0.00705128720921838\\
532	0.00708973073153022\\
533	0.00712748667855347\\
534	0.00716510576662421\\
535	0.00720298425452938\\
536	0.00724131062521522\\
537	0.00728020567755491\\
538	0.0073197621304377\\
539	0.00736003760201659\\
540	0.00740107026007002\\
541	0.00744289189206673\\
542	0.00748552922793119\\
543	0.00752900542539635\\
544	0.00757334165858363\\
545	0.00761855794325576\\
546	0.0076647133488703\\
547	0.0077118861491722\\
548	0.00776011644032672\\
549	0.00780747688911366\\
550	0.00785373177371214\\
551	0.00789882269053892\\
552	0.00794427584846579\\
553	0.00799013221423805\\
554	0.00803647165748248\\
555	0.00808335531207435\\
556	0.00813077466079379\\
557	0.0081787090079221\\
558	0.00822712878473145\\
559	0.00827599893382302\\
560	0.00832527752852308\\
561	0.00837493010777871\\
562	0.00842494111645729\\
563	0.00847397024240702\\
564	0.00852129441202751\\
565	0.00856849529392683\\
566	0.00861580505555088\\
567	0.00866327323530527\\
568	0.00871086822438768\\
569	0.00875855220172137\\
570	0.0088062920664252\\
571	0.00885392879390287\\
572	0.00890103813867015\\
573	0.00894817183198183\\
574	0.00899535443329664\\
575	0.00904255798659048\\
576	0.00908973853069441\\
577	0.00913684949557439\\
578	0.0091838419761039\\
579	0.009230665010298\\
580	0.00927726597847558\\
581	0.00932359116063004\\
582	0.00936958649305349\\
583	0.00941519857411533\\
584	0.00946037597947751\\
585	0.00950507095776978\\
586	0.00954924158550052\\
587	0.00959285445497631\\
588	0.00963588792409171\\
589	0.00967833579817709\\
590	0.00972021084271539\\
591	0.00976138933978722\\
592	0.00980171928775089\\
593	0.00984102135019415\\
594	0.00987905595166784\\
595	0.00991543381058343\\
596	0.00994937493726701\\
597	0.00997906286423442\\
598	0.0099999191923403\\
599	0\\
600	0\\
};
\addplot [color=mycolor20,solid,forget plot]
  table[row sep=crcr]{%
1	0.00426663331528449\\
2	0.00426663605446198\\
3	0.00426663884350637\\
4	0.00426664168332735\\
5	0.00426664457485123\\
6	0.0042666475190213\\
7	0.00426665051679803\\
8	0.00426665356915952\\
9	0.00426665667710172\\
10	0.00426665984163881\\
11	0.00426666306380356\\
12	0.00426666634464759\\
13	0.00426666968524176\\
14	0.00426667308667659\\
15	0.00426667655006247\\
16	0.00426668007653015\\
17	0.00426668366723108\\
18	0.0042666873233378\\
19	0.00426669104604424\\
20	0.00426669483656624\\
21	0.00426669869614191\\
22	0.004266702626032\\
23	0.00426670662752032\\
24	0.0042667107019142\\
25	0.00426671485054495\\
26	0.00426671907476817\\
27	0.00426672337596439\\
28	0.00426672775553932\\
29	0.00426673221492451\\
30	0.00426673675557764\\
31	0.00426674137898315\\
32	0.00426674608665263\\
33	0.00426675088012538\\
34	0.00426675576096894\\
35	0.0042667607307795\\
36	0.00426676579118251\\
37	0.00426677094383328\\
38	0.00426677619041735\\
39	0.00426678153265121\\
40	0.00426678697228281\\
41	0.00426679251109208\\
42	0.00426679815089161\\
43	0.00426680389352723\\
44	0.0042668097408786\\
45	0.00426681569485981\\
46	0.00426682175742013\\
47	0.00426682793054448\\
48	0.00426683421625422\\
49	0.00426684061660772\\
50	0.00426684713370119\\
51	0.00426685376966924\\
52	0.00426686052668561\\
53	0.00426686740696394\\
54	0.00426687441275847\\
55	0.0042668815463648\\
56	0.00426688881012064\\
57	0.00426689620640662\\
58	0.004266903737647\\
59	0.00426691140631057\\
60	0.0042669192149114\\
61	0.00426692716600971\\
62	0.00426693526221271\\
63	0.00426694350617553\\
64	0.00426695190060191\\
65	0.00426696044824531\\
66	0.00426696915190976\\
67	0.00426697801445072\\
68	0.00426698703877609\\
69	0.00426699622784719\\
70	0.00426700558467973\\
71	0.00426701511234478\\
72	0.00426702481396984\\
73	0.00426703469273984\\
74	0.00426704475189825\\
75	0.00426705499474805\\
76	0.00426706542465298\\
77	0.00426707604503858\\
78	0.0042670868593933\\
79	0.0042670978712697\\
80	0.00426710908428565\\
81	0.00426712050212553\\
82	0.00426713212854137\\
83	0.00426714396735423\\
84	0.00426715602245535\\
85	0.00426716829780761\\
86	0.00426718079744667\\
87	0.00426719352548243\\
88	0.00426720648610035\\
89	0.00426721968356286\\
90	0.00426723312221078\\
91	0.00426724680646477\\
92	0.00426726074082681\\
93	0.00426727492988168\\
94	0.0042672893782985\\
95	0.00426730409083235\\
96	0.0042673190723257\\
97	0.00426733432771018\\
98	0.0042673498620082\\
99	0.00426736568033454\\
100	0.00426738178789813\\
101	0.00426739819000379\\
102	0.00426741489205398\\
103	0.00426743189955058\\
104	0.0042674492180968\\
105	0.00426746685339901\\
106	0.00426748481126861\\
107	0.00426750309762402\\
108	0.00426752171849266\\
109	0.00426754068001293\\
110	0.00426755998843633\\
111	0.00426757965012944\\
112	0.00426759967157611\\
113	0.00426762005937972\\
114	0.00426764082026519\\
115	0.00426766196108137\\
116	0.00426768348880335\\
117	0.00426770541053467\\
118	0.00426772773350981\\
119	0.00426775046509656\\
120	0.00426777361279847\\
121	0.0042677971842574\\
122	0.00426782118725611\\
123	0.00426784562972072\\
124	0.00426787051972356\\
125	0.00426789586548568\\
126	0.00426792167537974\\
127	0.00426794795793283\\
128	0.00426797472182915\\
129	0.00426800197591313\\
130	0.00426802972919219\\
131	0.00426805799083993\\
132	0.00426808677019914\\
133	0.00426811607678491\\
134	0.00426814592028779\\
135	0.00426817631057712\\
136	0.00426820725770426\\
137	0.00426823877190602\\
138	0.004268270863608\\
139	0.00426830354342821\\
140	0.0042683368221805\\
141	0.00426837071087827\\
142	0.00426840522073809\\
143	0.00426844036318351\\
144	0.00426847614984887\\
145	0.00426851259258323\\
146	0.00426854970345422\\
147	0.00426858749475229\\
148	0.00426862597899463\\
149	0.00426866516892941\\
150	0.00426870507754012\\
151	0.00426874571804986\\
152	0.00426878710392581\\
153	0.00426882924888365\\
154	0.00426887216689225\\
155	0.00426891587217831\\
156	0.00426896037923111\\
157	0.00426900570280735\\
158	0.00426905185793613\\
159	0.00426909885992393\\
160	0.00426914672435976\\
161	0.00426919546712036\\
162	0.00426924510437553\\
163	0.00426929565259355\\
164	0.00426934712854658\\
165	0.0042693995493164\\
166	0.00426945293230002\\
167	0.00426950729521554\\
168	0.00426956265610805\\
169	0.00426961903335571\\
170	0.00426967644567576\\
171	0.00426973491213084\\
172	0.00426979445213538\\
173	0.00426985508546202\\
174	0.00426991683224824\\
175	0.00426997971300302\\
176	0.00427004374861366\\
177	0.00427010896035288\\
178	0.00427017536988565\\
179	0.00427024299927665\\
180	0.00427031187099741\\
181	0.00427038200793391\\
182	0.00427045343339406\\
183	0.00427052617111558\\
184	0.00427060024527374\\
185	0.00427067568048948\\
186	0.00427075250183746\\
187	0.00427083073485445\\
188	0.0042709104055478\\
189	0.00427099154040393\\
190	0.00427107416639718\\
191	0.00427115831099867\\
192	0.00427124400218538\\
193	0.00427133126844935\\
194	0.00427142013880714\\
195	0.00427151064280929\\
196	0.00427160281055008\\
197	0.0042716966726774\\
198	0.00427179226040283\\
199	0.00427188960551184\\
200	0.00427198874037426\\
201	0.00427208969795479\\
202	0.00427219251182382\\
203	0.00427229721616838\\
204	0.00427240384580332\\
205	0.00427251243618258\\
206	0.00427262302341075\\
207	0.0042727356442548\\
208	0.004272850336156\\
209	0.00427296713724205\\
210	0.00427308608633941\\
211	0.00427320722298579\\
212	0.00427333058744295\\
213	0.00427345622070965\\
214	0.00427358416453478\\
215	0.00427371446143077\\
216	0.00427384715468722\\
217	0.00427398228838469\\
218	0.0042741199074088\\
219	0.00427426005746445\\
220	0.0042744027850904\\
221	0.00427454813767394\\
222	0.00427469616346598\\
223	0.00427484691159612\\
224	0.00427500043208823\\
225	0.00427515677587616\\
226	0.00427531599481954\\
227	0.00427547814172013\\
228	0.00427564327033817\\
229	0.00427581143540913\\
230	0.00427598269266059\\
231	0.00427615709882952\\
232	0.00427633471167965\\
233	0.00427651559001924\\
234	0.00427669979371907\\
235	0.00427688738373068\\
236	0.00427707842210479\\
237	0.00427727297201019\\
238	0.00427747109775273\\
239	0.00427767286479459\\
240	0.00427787833977384\\
241	0.00427808759052437\\
242	0.00427830068609593\\
243	0.00427851769677457\\
244	0.00427873869410317\\
245	0.00427896375090256\\
246	0.00427919294129249\\
247	0.00427942634071331\\
248	0.00427966402594756\\
249	0.00427990607514204\\
250	0.00428015256783011\\
251	0.00428040358495417\\
252	0.00428065920888856\\
253	0.00428091952346254\\
254	0.00428118461398376\\
255	0.00428145456726173\\
256	0.00428172947163182\\
257	0.00428200941697935\\
258	0.00428229449476391\\
259	0.00428258479804408\\
260	0.00428288042150233\\
261	0.00428318146147007\\
262	0.00428348801595315\\
263	0.00428380018465741\\
264	0.00428411806901457\\
265	0.00428444177220832\\
266	0.00428477139920069\\
267	0.00428510705675852\\
268	0.00428544885348034\\
269	0.00428579689982334\\
270	0.00428615130813052\\
271	0.00428651219265822\\
272	0.00428687966960365\\
273	0.00428725385713282\\
274	0.00428763487540848\\
275	0.00428802284661846\\
276	0.00428841789500398\\
277	0.00428882014688836\\
278	0.00428922973070582\\
279	0.00428964677703041\\
280	0.00429007141860525\\
281	0.00429050379037187\\
282	0.00429094402949989\\
283	0.00429139227541663\\
284	0.00429184866983709\\
285	0.00429231335679421\\
286	0.0042927864826692\\
287	0.00429326819622204\\
288	0.00429375864862242\\
289	0.00429425799348073\\
290	0.00429476638687946\\
291	0.00429528398740469\\
292	0.00429581095617808\\
293	0.00429634745688905\\
294	0.00429689365582736\\
295	0.00429744972191603\\
296	0.00429801582674476\\
297	0.00429859214460362\\
298	0.00429917885251751\\
299	0.00429977613028076\\
300	0.00430038416049273\\
301	0.00430100312859371\\
302	0.00430163322290172\\
303	0.00430227463464979\\
304	0.00430292755802435\\
305	0.00430359219020419\\
306	0.00430426873140046\\
307	0.0043049573848976\\
308	0.00430565835709522\\
309	0.00430637185755113\\
310	0.00430709809902548\\
311	0.00430783729752581\\
312	0.0043085896723536\\
313	0.00430935544615174\\
314	0.00431013484495313\\
315	0.00431092809823047\\
316	0.00431173543894718\\
317	0.00431255710360897\\
318	0.00431339333231663\\
319	0.00431424436881902\\
320	0.00431511046056672\\
321	0.00431599185876574\\
322	0.00431688881843094\\
323	0.00431780159843884\\
324	0.00431873046157911\\
325	0.00431967567460451\\
326	0.00432063750827823\\
327	0.0043216162374183\\
328	0.00432261214093769\\
329	0.00432362550187977\\
330	0.00432465660744775\\
331	0.00432570574902713\\
332	0.00432677322220024\\
333	0.00432785932675161\\
334	0.00432896436666306\\
335	0.00433008865009771\\
336	0.00433123248937182\\
337	0.00433239620091378\\
338	0.0043335801052101\\
339	0.00433478452673791\\
340	0.00433600979388494\\
341	0.00433725623885781\\
342	0.00433852419758035\\
343	0.00433981400958369\\
344	0.00434112601789247\\
345	0.00434246056892186\\
346	0.00434381801239764\\
347	0.00434519870128648\\
348	0.00434660299173398\\
349	0.00434803124301478\\
350	0.00434948381749918\\
351	0.00435096108064044\\
352	0.00435246340098642\\
353	0.00435399115021984\\
354	0.00435554470323253\\
355	0.00435712443823616\\
356	0.00435873073685524\\
357	0.00436036398412274\\
358	0.00436202456847492\\
359	0.00436371288175468\\
360	0.00436542931922488\\
361	0.00436717427959442\\
362	0.00436894816505918\\
363	0.00437075138136093\\
364	0.00437258433786724\\
365	0.00437444744767627\\
366	0.00437634112775016\\
367	0.00437826579908203\\
368	0.00438022188690141\\
369	0.00438220982092401\\
370	0.00438423003565191\\
371	0.00438628297073149\\
372	0.00438836907137671\\
373	0.00439048878886601\\
374	0.00439264258112255\\
375	0.00439483091338743\\
376	0.0043970542589972\\
377	0.00439931310027703\\
378	0.00440160792956237\\
379	0.00440393925036222\\
380	0.00440630757867798\\
381	0.00440871344449215\\
382	0.00441115739344192\\
383	0.00441363998869198\\
384	0.00441616181302129\\
385	0.00441872347113729\\
386	0.00442132559222974\\
387	0.00442396883277432\\
388	0.00442665387959276\\
389	0.00442938145317225\\
390	0.00443215231124075\\
391	0.00443496725258767\\
392	0.00443782712110927\\
393	0.0044407328100463\\
394	0.00444368526636549\\
395	0.00444668549521858\\
396	0.00444973456438823\\
397	0.00445283360860144\\
398	0.00445598383354885\\
399	0.00445918651955122\\
400	0.00446244302487293\\
401	0.0044657547879608\\
402	0.00446912332816642\\
403	0.00447255024442683\\
404	0.00447603721130479\\
405	0.00447958597174322\\
406	0.00448319832590475\\
407	0.004486876115589\\
408	0.0044906212040159\\
409	0.00449443545129221\\
410	0.00449832068667244\\
411	0.00450227867983526\\
412	0.00450631111553174\\
413	0.00451041958314296\\
414	0.00451460561420282\\
415	0.00451887075326674\\
416	0.00452321658618162\\
417	0.0045276447430362\\
418	0.0045321569014872\\
419	0.00453675479048509\\
420	0.00454144019441523\\
421	0.00454621495765489\\
422	0.00455108098952513\\
423	0.00455604026958924\\
424	0.00456109485322701\\
425	0.00456624687739753\\
426	0.00457149856649096\\
427	0.00457685223810688\\
428	0.00458231029759972\\
429	0.004587875226996\\
430	0.0045935495869901\\
431	0.00459933601879077\\
432	0.0046052372457871\\
433	0.00461125607500289\\
434	0.00461739539831013\\
435	0.00462365819337468\\
436	0.00463004752431581\\
437	0.00463656654207527\\
438	0.00464321848450911\\
439	0.0046500066762436\\
440	0.00465693452837336\\
441	0.00466400553812897\\
442	0.00467122328870246\\
443	0.00467859144948759\\
444	0.00468611377705478\\
445	0.00469379411721445\\
446	0.00470163640858135\\
447	0.00470964468879945\\
448	0.00471782311025046\\
449	0.00472617665871337\\
450	0.00473471089905645\\
451	0.00474343165936944\\
452	0.00475234504327954\\
453	0.00476145744292708\\
454	0.00477077555288365\\
455	0.00478030638543103\\
456	0.0047900572865475\\
457	0.00480003595306472\\
458	0.00481025045074162\\
459	0.00482070923265264\\
460	0.00483142115711013\\
461	0.00484239550330722\\
462	0.00485364197314648\\
463	0.00486517066339799\\
464	0.0048769919772247\\
465	0.00488911634640063\\
466	0.00490151423351223\\
467	0.00491419268996534\\
468	0.00492716026276567\\
469	0.0049404258593231\\
470	0.00495399877491506\\
471	0.00496788872794736\\
472	0.00498210588125886\\
473	0.00499666087416632\\
474	0.00501156486292523\\
475	0.005026829556054\\
476	0.00504246724598849\\
477	0.00505849076745665\\
478	0.00507491360227769\\
479	0.00509174979338502\\
480	0.00510904461151792\\
481	0.00512683513835379\\
482	0.00514514294413699\\
483	0.00516399086837915\\
484	0.00518340310039239\\
485	0.00520340532154351\\
486	0.00522402486784549\\
487	0.00524529086619148\\
488	0.00526723353178775\\
489	0.00528988555479908\\
490	0.00531328551978419\\
491	0.00533749217269366\\
492	0.00536255852004745\\
493	0.00538852951181075\\
494	0.00541545080092527\\
495	0.00544336957331949\\
496	0.00547233467708075\\
497	0.00550239710763532\\
498	0.00553360994472241\\
499	0.00556602827457043\\
500	0.00559970902539875\\
501	0.00563471072458662\\
502	0.00567109315732001\\
503	0.00570891688837819\\
504	0.00574824251891552\\
505	0.00578912967368312\\
506	0.00583163601149896\\
507	0.00587581585992625\\
508	0.0059217185016282\\
509	0.00596938601077104\\
510	0.00601885147197273\\
511	0.00607014279388628\\
512	0.00612327173424099\\
513	0.00617822896877258\\
514	0.00623497651937254\\
515	0.00629343208266863\\
516	0.00635345162141576\\
517	0.00641450018738264\\
518	0.00647630306501683\\
519	0.00653856759307269\\
520	0.00660083177351316\\
521	0.0066623734845899\\
522	0.00672148584669011\\
523	0.00677514287299271\\
524	0.00682314710642925\\
525	0.00686550063474744\\
526	0.00690303785427555\\
527	0.006939183141758\\
528	0.0069743533931855\\
529	0.00700904633276673\\
530	0.00704380336489818\\
531	0.00707891357147736\\
532	0.00711450782547904\\
533	0.00715069050118957\\
534	0.00718753256758812\\
535	0.00722507731585913\\
536	0.00726335660395916\\
537	0.00730239682932776\\
538	0.00734222033833203\\
539	0.00738284789789318\\
540	0.00742430017052845\\
541	0.00746659733277799\\
542	0.00750975936517875\\
543	0.00755383226995456\\
544	0.00759889240011881\\
545	0.00764503248014944\\
546	0.00769063100310164\\
547	0.00773515726105684\\
548	0.00777837421757474\\
549	0.00782197210231666\\
550	0.00786599669861675\\
551	0.00791053055057518\\
552	0.0079556338739196\\
553	0.00800129915646446\\
554	0.00804751442201892\\
555	0.00809426408080562\\
556	0.00814152538695643\\
557	0.00818926789268914\\
558	0.00823745553953724\\
559	0.00828604517572598\\
560	0.00833502635030947\\
561	0.0083836844121862\\
562	0.00843068273194334\\
563	0.00847720045237796\\
564	0.00852387037291765\\
565	0.0085707516393333\\
566	0.00861782349227564\\
567	0.00866505204116376\\
568	0.00871240258683382\\
569	0.00875984726743983\\
570	0.00880711445205011\\
571	0.00885403140777324\\
572	0.00890105211970034\\
573	0.00894817202601561\\
574	0.00899535447720905\\
575	0.00904255800719692\\
576	0.00908973854120521\\
577	0.00913684950076924\\
578	0.00918384197853101\\
579	0.00923066501135355\\
580	0.00927726597889613\\
581	0.00932359116078053\\
582	0.00936958649310055\\
583	0.00941519857412769\\
584	0.00946037597948007\\
585	0.00950507095777015\\
586	0.00954924158550054\\
587	0.00959285445497631\\
588	0.00963588792409171\\
589	0.00967833579817709\\
590	0.00972021084271539\\
591	0.00976138933978722\\
592	0.00980171928775089\\
593	0.00984102135019415\\
594	0.00987905595166784\\
595	0.00991543381058343\\
596	0.00994937493726701\\
597	0.00997906286423442\\
598	0.0099999191923403\\
599	0\\
600	0\\
};
\addplot [color=mycolor21,solid,forget plot]
  table[row sep=crcr]{%
1	0.00430148357374624\\
2	0.00430148576350371\\
3	0.00430148799339185\\
4	0.00430149026414835\\
5	0.00430149257652449\\
6	0.00430149493128541\\
7	0.00430149732921043\\
8	0.00430149977109314\\
9	0.00430150225774187\\
10	0.00430150478997984\\
11	0.00430150736864544\\
12	0.00430150999459258\\
13	0.00430151266869094\\
14	0.00430151539182618\\
15	0.00430151816490036\\
16	0.00430152098883223\\
17	0.00430152386455745\\
18	0.00430152679302899\\
19	0.00430152977521747\\
20	0.00430153281211134\\
21	0.00430153590471736\\
22	0.00430153905406089\\
23	0.00430154226118625\\
24	0.00430154552715704\\
25	0.0043015488530565\\
26	0.00430155223998792\\
27	0.00430155568907497\\
28	0.00430155920146212\\
29	0.00430156277831491\\
30	0.00430156642082051\\
31	0.00430157013018804\\
32	0.00430157390764895\\
33	0.0043015777544575\\
34	0.00430158167189106\\
35	0.00430158566125074\\
36	0.00430158972386165\\
37	0.0043015938610734\\
38	0.00430159807426065\\
39	0.00430160236482345\\
40	0.0043016067341877\\
41	0.00430161118380577\\
42	0.00430161571515693\\
43	0.00430162032974776\\
44	0.00430162502911277\\
45	0.00430162981481488\\
46	0.00430163468844592\\
47	0.00430163965162722\\
48	0.00430164470601016\\
49	0.00430164985327671\\
50	0.00430165509513993\\
51	0.00430166043334467\\
52	0.00430166586966809\\
53	0.00430167140592028\\
54	0.00430167704394488\\
55	0.00430168278561971\\
56	0.00430168863285737\\
57	0.0043016945876059\\
58	0.00430170065184949\\
59	0.00430170682760911\\
60	0.00430171311694324\\
61	0.00430171952194844\\
62	0.00430172604476025\\
63	0.00430173268755374\\
64	0.00430173945254442\\
65	0.00430174634198882\\
66	0.00430175335818536\\
67	0.00430176050347511\\
68	0.00430176778024267\\
69	0.00430177519091678\\
70	0.00430178273797132\\
71	0.00430179042392609\\
72	0.00430179825134768\\
73	0.00430180622285032\\
74	0.0043018143410968\\
75	0.00430182260879937\\
76	0.0043018310287207\\
77	0.0043018396036747\\
78	0.00430184833652761\\
79	0.00430185723019895\\
80	0.00430186628766246\\
81	0.00430187551194713\\
82	0.00430188490613833\\
83	0.00430189447337872\\
84	0.00430190421686942\\
85	0.00430191413987106\\
86	0.00430192424570493\\
87	0.00430193453775406\\
88	0.00430194501946442\\
89	0.00430195569434613\\
90	0.00430196656597458\\
91	0.00430197763799169\\
92	0.00430198891410719\\
93	0.00430200039809985\\
94	0.00430201209381875\\
95	0.00430202400518465\\
96	0.00430203613619134\\
97	0.00430204849090697\\
98	0.0043020610734754\\
99	0.00430207388811781\\
100	0.00430208693913393\\
101	0.00430210023090359\\
102	0.00430211376788829\\
103	0.00430212755463265\\
104	0.00430214159576601\\
105	0.00430215589600398\\
106	0.00430217046015014\\
107	0.00430218529309764\\
108	0.00430220039983083\\
109	0.0043022157854271\\
110	0.00430223145505849\\
111	0.00430224741399359\\
112	0.00430226366759927\\
113	0.0043022802213426\\
114	0.00430229708079265\\
115	0.00430231425162249\\
116	0.00430233173961106\\
117	0.00430234955064526\\
118	0.00430236769072188\\
119	0.00430238616594977\\
120	0.00430240498255189\\
121	0.00430242414686748\\
122	0.00430244366535419\\
123	0.00430246354459042\\
124	0.00430248379127747\\
125	0.00430250441224198\\
126	0.0043025254144382\\
127	0.00430254680495045\\
128	0.00430256859099555\\
129	0.00430259077992528\\
130	0.004302613379229\\
131	0.00430263639653623\\
132	0.00430265983961922\\
133	0.00430268371639573\\
134	0.00430270803493175\\
135	0.00430273280344428\\
136	0.00430275803030421\\
137	0.00430278372403919\\
138	0.00430280989333666\\
139	0.00430283654704676\\
140	0.00430286369418549\\
141	0.00430289134393785\\
142	0.00430291950566099\\
143	0.00430294818888749\\
144	0.00430297740332864\\
145	0.00430300715887788\\
146	0.00430303746561421\\
147	0.00430306833380574\\
148	0.00430309977391316\\
149	0.00430313179659353\\
150	0.00430316441270393\\
151	0.00430319763330525\\
152	0.00430323146966604\\
153	0.00430326593326646\\
154	0.00430330103580227\\
155	0.00430333678918899\\
156	0.00430337320556595\\
157	0.00430341029730062\\
158	0.00430344807699291\\
159	0.00430348655747963\\
160	0.00430352575183885\\
161	0.00430356567339461\\
162	0.00430360633572154\\
163	0.00430364775264963\\
164	0.00430368993826906\\
165	0.00430373290693513\\
166	0.00430377667327339\\
167	0.00430382125218463\\
168	0.00430386665885023\\
169	0.0043039129087374\\
170	0.00430396001760469\\
171	0.00430400800150752\\
172	0.00430405687680379\\
173	0.00430410666015964\\
174	0.0043041573685553\\
175	0.00430420901929114\\
176	0.00430426162999371\\
177	0.00430431521862189\\
178	0.00430436980347337\\
179	0.00430442540319092\\
180	0.00430448203676915\\
181	0.00430453972356102\\
182	0.00430459848328482\\
183	0.00430465833603102\\
184	0.00430471930226947\\
185	0.00430478140285652\\
186	0.00430484465904247\\
187	0.0043049090924791\\
188	0.00430497472522717\\
189	0.00430504157976439\\
190	0.00430510967899332\\
191	0.00430517904624945\\
192	0.00430524970530943\\
193	0.00430532168039956\\
194	0.00430539499620432\\
195	0.00430546967787514\\
196	0.00430554575103932\\
197	0.00430562324180901\\
198	0.00430570217679061\\
199	0.00430578258309413\\
200	0.00430586448834281\\
201	0.00430594792068292\\
202	0.00430603290879372\\
203	0.00430611948189766\\
204	0.00430620766977075\\
205	0.00430629750275308\\
206	0.00430638901175968\\
207	0.00430648222829146\\
208	0.00430657718444632\\
209	0.0043066739129306\\
210	0.00430677244707076\\
211	0.00430687282082509\\
212	0.0043069750687958\\
213	0.00430707922624141\\
214	0.00430718532908909\\
215	0.00430729341394761\\
216	0.00430740351812016\\
217	0.00430751567961771\\
218	0.00430762993717242\\
219	0.00430774633025148\\
220	0.004307864899071\\
221	0.00430798568461041\\
222	0.00430810872862685\\
223	0.00430823407367006\\
224	0.00430836176309745\\
225	0.00430849184108941\\
226	0.00430862435266503\\
227	0.00430875934369804\\
228	0.00430889686093293\\
229	0.00430903695200158\\
230	0.00430917966544005\\
231	0.00430932505070579\\
232	0.00430947315819503\\
233	0.00430962403926053\\
234	0.00430977774622978\\
235	0.00430993433242325\\
236	0.0043100938521734\\
237	0.00431025636084354\\
238	0.00431042191484736\\
239	0.0043105905716687\\
240	0.00431076238988176\\
241	0.00431093742917142\\
242	0.00431111575035422\\
243	0.00431129741539953\\
244	0.00431148248745111\\
245	0.00431167103084907\\
246	0.00431186311115226\\
247	0.00431205879516082\\
248	0.00431225815093942\\
249	0.00431246124784067\\
250	0.00431266815652895\\
251	0.00431287894900475\\
252	0.00431309369862925\\
253	0.00431331248014943\\
254	0.00431353536972344\\
255	0.00431376244494658\\
256	0.00431399378487743\\
257	0.00431422947006457\\
258	0.00431446958257367\\
259	0.00431471420601489\\
260	0.00431496342557085\\
261	0.00431521732802484\\
262	0.00431547600178957\\
263	0.00431573953693623\\
264	0.00431600802522396\\
265	0.00431628156012981\\
266	0.00431656023687893\\
267	0.00431684415247528\\
268	0.00431713340573277\\
269	0.00431742809730652\\
270	0.00431772832972479\\
271	0.00431803420742105\\
272	0.00431834583676664\\
273	0.00431866332610351\\
274	0.00431898678577741\\
275	0.00431931632817146\\
276	0.00431965206773999\\
277	0.00431999412104263\\
278	0.00432034260677865\\
279	0.00432069764582177\\
280	0.00432105936125497\\
281	0.00432142787840569\\
282	0.00432180332488109\\
283	0.00432218583060372\\
284	0.00432257552784727\\
285	0.00432297255127231\\
286	0.00432337703796235\\
287	0.00432378912746006\\
288	0.00432420896180337\\
289	0.00432463668556183\\
290	0.00432507244587283\\
291	0.0043255163924781\\
292	0.00432596867776004\\
293	0.00432642945677815\\
294	0.00432689888730534\\
295	0.00432737712986429\\
296	0.00432786434776395\\
297	0.00432836070713575\\
298	0.00432886637696996\\
299	0.00432938152915217\\
300	0.00432990633849959\\
301	0.00433044098279764\\
302	0.00433098564283659\\
303	0.00433154050244854\\
304	0.00433210574854453\\
305	0.00433268157115222\\
306	0.00433326816345408\\
307	0.00433386572182634\\
308	0.00433447444587875\\
309	0.00433509453849552\\
310	0.00433572620587763\\
311	0.00433636965758684\\
312	0.00433702510659156\\
313	0.00433769276931526\\
314	0.00433837286568757\\
315	0.00433906561919869\\
316	0.00433977125695748\\
317	0.00434049000975394\\
318	0.00434122211212661\\
319	0.00434196780243559\\
320	0.00434272732294176\\
321	0.004343500919893\\
322	0.00434428884361807\\
323	0.004345091348629\\
324	0.00434590869373256\\
325	0.00434674114215151\\
326	0.00434758896165619\\
327	0.00434845242470694\\
328	0.00434933180860762\\
329	0.00435022739567029\\
330	0.00435113947339081\\
331	0.00435206833463485\\
332	0.00435301427783332\\
333	0.00435397760718565\\
334	0.00435495863286911\\
335	0.00435595767125125\\
336	0.00435697504510243\\
337	0.00435801108380426\\
338	0.00435906612354941\\
339	0.00436014050752808\\
340	0.0043612345860951\\
341	0.0043623487169129\\
342	0.00436348326506536\\
343	0.00436463860313877\\
344	0.004365815111269\\
345	0.00436701317715537\\
346	0.00436823319604323\\
347	0.00436947557074711\\
348	0.0043707407118326\\
349	0.0043720290378046\\
350	0.00437334097530099\\
351	0.00437467695929117\\
352	0.00437603743327923\\
353	0.00437742284951051\\
354	0.00437883366918161\\
355	0.00438027036265104\\
356	0.00438173340965126\\
357	0.00438322329950548\\
358	0.00438474053134917\\
359	0.00438628561435644\\
360	0.00438785906797147\\
361	0.0043894614221449\\
362	0.00439109321757576\\
363	0.00439275500595847\\
364	0.00439444735023603\\
365	0.00439617082485916\\
366	0.00439792601605232\\
367	0.00439971352208733\\
368	0.00440153395356529\\
369	0.00440338793370829\\
370	0.00440527609866231\\
371	0.00440719909781317\\
372	0.0044091575941179\\
373	0.00441115226445455\\
374	0.00441318379999359\\
375	0.00441525290659536\\
376	0.00441736030523858\\
377	0.00441950673248561\\
378	0.00442169294099152\\
379	0.00442391970006501\\
380	0.00442618779629041\\
381	0.00442849803422117\\
382	0.00443085123715657\\
383	0.0044332482480148\\
384	0.00443568993031624\\
385	0.00443817716929183\\
386	0.004440710873132\\
387	0.00444329197439091\\
388	0.00444592143156016\\
389	0.00444860023082302\\
390	0.00445132938799669\\
391	0.00445410995066293\\
392	0.00445694300047828\\
393	0.00445982965564104\\
394	0.00446277107347559\\
395	0.00446576845307248\\
396	0.00446882303790564\\
397	0.004471936118343\\
398	0.00447510903394544\\
399	0.00447834317079437\\
400	0.00448163995012209\\
401	0.00448500082751085\\
402	0.00448842729181676\\
403	0.0044919208638358\\
404	0.00449548309475358\\
405	0.00449911556445268\\
406	0.00450281987978895\\
407	0.0045065976729882\\
408	0.00451045060034677\\
409	0.00451438034142587\\
410	0.00451838859889704\\
411	0.00452247709914652\\
412	0.00452664759381917\\
413	0.00453090186251676\\
414	0.0045352417147148\\
415	0.00453966898894157\\
416	0.0045441855506476\\
417	0.00454879328977864\\
418	0.0045534941180901\\
419	0.00455828996627101\\
420	0.00456318278098549\\
421	0.00456817452198913\\
422	0.00457326715952721\\
423	0.00457846267227382\\
424	0.00458376304609542\\
425	0.00458917027399014\\
426	0.00459468635791866\\
427	0.00460031331623116\\
428	0.00460605352979721\\
429	0.00461190990897305\\
430	0.00461788548949603\\
431	0.00462398344079992\\
432	0.00463020707486431\\
433	0.00463655985556917\\
434	0.0046430454085166\\
435	0.00464966753135385\\
436	0.00465643020445243\\
437	0.00466333760180588\\
438	0.00467039410194801\\
439	0.00467760429861155\\
440	0.00468497301074153\\
441	0.00469250529133314\\
442	0.00470020643436151\\
443	0.00470808197874832\\
444	0.00471613770768533\\
445	0.0047243796400989\\
446	0.00473281400643386\\
447	0.0047414471853645\\
448	0.0047502855217763\\
449	0.00475931359508267\\
450	0.00476852503652838\\
451	0.00477792411818879\\
452	0.00478751522745984\\
453	0.00479730286871586\\
454	0.00480729166239887\\
455	0.00481748633836206\\
456	0.00482789175655993\\
457	0.00483851291571316\\
458	0.00484935496444078\\
459	0.00486042321558921\\
460	0.00487172316447817\\
461	0.00488326051116053\\
462	0.00489504118541369\\
463	0.0049070713661606\\
464	0.0049193574708969\\
465	0.00493190604161025\\
466	0.00494476407015433\\
467	0.00495794401795656\\
468	0.00497145787261548\\
469	0.00498531831362211\\
470	0.00499953876316261\\
471	0.00501413343991552\\
472	0.00502911741621035\\
473	0.00504450667898729\\
474	0.00506031819431716\\
475	0.00507656997661637\\
476	0.00509328116762204\\
477	0.0051104721473742\\
478	0.0051281647361623\\
479	0.00514638269199349\\
480	0.00516515080193658\\
481	0.00518449389127852\\
482	0.0052044376264225\\
483	0.0052250091872611\\
484	0.00524623737495304\\
485	0.00526815187863189\\
486	0.00529078468564697\\
487	0.00531417357557784\\
488	0.00533837735097006\\
489	0.00536344414407026\\
490	0.0053894165439534\\
491	0.00541633916344982\\
492	0.00544425797132098\\
493	0.00547322049379098\\
494	0.00550327613013326\\
495	0.00553447607120864\\
496	0.00556687314828419\\
497	0.00560052158990553\\
498	0.00563547657178693\\
499	0.00567179360157377\\
500	0.00570952787723091\\
501	0.0057487334698313\\
502	0.00578946229353781\\
503	0.00583176282309453\\
504	0.00587568061430283\\
505	0.00592126098008407\\
506	0.00596854041180243\\
507	0.0060175431613355\\
508	0.00606827693345952\\
509	0.0061207277550409\\
510	0.0061748191704826\\
511	0.00623017318207606\\
512	0.00628662745140264\\
513	0.00634395484991725\\
514	0.00640187056429931\\
515	0.00646008917106267\\
516	0.00651816770353974\\
517	0.00657574688726379\\
518	0.00663128465556018\\
519	0.00668140030949382\\
520	0.0067258985133532\\
521	0.00676486664598173\\
522	0.00679941350434478\\
523	0.00683269133876112\\
524	0.00686511250829205\\
525	0.00689716691348905\\
526	0.00692935080768741\\
527	0.00696191476452744\\
528	0.00699497463284185\\
529	0.00702862070688934\\
530	0.00706290768641132\\
531	0.00709786900375971\\
532	0.00713353198173733\\
533	0.00716991911972827\\
534	0.00720705005775949\\
535	0.00724494357205851\\
536	0.00728361853919417\\
537	0.00732309453011616\\
538	0.00736339161534406\\
539	0.00740453024079764\\
540	0.00744653870738161\\
541	0.00748948972389208\\
542	0.00753347142049639\\
543	0.00757747381845585\\
544	0.00762044744386721\\
545	0.00766211550727995\\
546	0.00770387386088141\\
547	0.00774606208827088\\
548	0.00778876715738478\\
549	0.00783205956081914\\
550	0.00787593614332746\\
551	0.00792038862872709\\
552	0.00796540239658982\\
553	0.00801096018996022\\
554	0.00805704284211701\\
555	0.00810363228870382\\
556	0.00815069909831113\\
557	0.00819820719505273\\
558	0.00824613258604494\\
559	0.00829446918237396\\
560	0.00834131866989406\\
561	0.00838711839027356\\
562	0.00843309183235914\\
563	0.00847930989270675\\
564	0.00852577430147598\\
565	0.00857245512281394\\
566	0.00861932104340902\\
567	0.0086663403000851\\
568	0.00871348883275888\\
569	0.00876041921360522\\
570	0.00880715257596726\\
571	0.00885403364170714\\
572	0.00890105214066917\\
573	0.0089481720325389\\
574	0.00899535448039004\\
575	0.0090425580087869\\
576	0.00908973854196738\\
577	0.00913684950111323\\
578	0.00918384197867516\\
579	0.00923066501140877\\
580	0.0092772659789151\\
581	0.00932359116078622\\
582	0.00936958649310198\\
583	0.00941519857412797\\
584	0.00946037597948012\\
585	0.00950507095777015\\
586	0.00954924158550055\\
587	0.00959285445497632\\
588	0.00963588792409171\\
589	0.0096783357981771\\
590	0.00972021084271539\\
591	0.00976138933978722\\
592	0.00980171928775089\\
593	0.00984102135019415\\
594	0.00987905595166784\\
595	0.00991543381058343\\
596	0.00994937493726701\\
597	0.00997906286423442\\
598	0.0099999191923403\\
599	0\\
600	0\\
};
\addplot [color=black!20!mycolor21,solid,forget plot]
  table[row sep=crcr]{%
1	0.00431499694832056\\
2	0.00431499894077788\\
3	0.00431500096991465\\
4	0.00431500303640849\\
5	0.00431500514094959\\
6	0.00431500728424092\\
7	0.00431500946699855\\
8	0.00431501168995181\\
9	0.00431501395384356\\
10	0.00431501625943045\\
11	0.00431501860748317\\
12	0.00431502099878675\\
13	0.00431502343414068\\
14	0.00431502591435945\\
15	0.00431502844027261\\
16	0.00431503101272507\\
17	0.00431503363257744\\
18	0.00431503630070642\\
19	0.00431503901800483\\
20	0.00431504178538223\\
21	0.00431504460376497\\
22	0.00431504747409669\\
23	0.00431505039733851\\
24	0.00431505337446941\\
25	0.0043150564064866\\
26	0.00431505949440579\\
27	0.00431506263926162\\
28	0.0043150658421079\\
29	0.00431506910401811\\
30	0.00431507242608562\\
31	0.00431507580942415\\
32	0.0043150792551682\\
33	0.0043150827644733\\
34	0.0043150863385165\\
35	0.00431508997849671\\
36	0.00431509368563522\\
37	0.00431509746117602\\
38	0.0043151013063862\\
39	0.00431510522255646\\
40	0.0043151092110016\\
41	0.00431511327306083\\
42	0.00431511741009825\\
43	0.00431512162350343\\
44	0.00431512591469183\\
45	0.0043151302851052\\
46	0.0043151347362122\\
47	0.00431513926950884\\
48	0.00431514388651896\\
49	0.00431514858879484\\
50	0.0043151533779177\\
51	0.0043151582554982\\
52	0.00431516322317698\\
53	0.00431516828262534\\
54	0.00431517343554571\\
55	0.00431517868367225\\
56	0.00431518402877144\\
57	0.00431518947264272\\
58	0.00431519501711916\\
59	0.00431520066406793\\
60	0.00431520641539104\\
61	0.00431521227302606\\
62	0.00431521823894665\\
63	0.00431522431516331\\
64	0.00431523050372412\\
65	0.00431523680671531\\
66	0.00431524322626214\\
67	0.00431524976452953\\
68	0.00431525642372277\\
69	0.00431526320608839\\
70	0.00431527011391491\\
71	0.00431527714953353\\
72	0.00431528431531907\\
73	0.00431529161369067\\
74	0.00431529904711277\\
75	0.00431530661809576\\
76	0.00431531432919702\\
77	0.00431532218302178\\
78	0.00431533018222386\\
79	0.00431533832950687\\
80	0.00431534662762482\\
81	0.00431535507938334\\
82	0.00431536368764055\\
83	0.00431537245530801\\
84	0.00431538138535179\\
85	0.00431539048079349\\
86	0.00431539974471124\\
87	0.00431540918024076\\
88	0.00431541879057662\\
89	0.00431542857897303\\
90	0.00431543854874529\\
91	0.00431544870327075\\
92	0.00431545904599003\\
93	0.00431546958040823\\
94	0.00431548031009616\\
95	0.00431549123869157\\
96	0.00431550236990035\\
97	0.0043155137074979\\
98	0.00431552525533045\\
99	0.00431553701731626\\
100	0.00431554899744721\\
101	0.00431556119979001\\
102	0.0043155736284877\\
103	0.00431558628776109\\
104	0.00431559918191016\\
105	0.00431561231531569\\
106	0.00431562569244063\\
107	0.00431563931783176\\
108	0.00431565319612135\\
109	0.0043156673320286\\
110	0.00431568173036141\\
111	0.004315696396018\\
112	0.00431571133398872\\
113	0.00431572654935767\\
114	0.00431574204730453\\
115	0.00431575783310647\\
116	0.00431577391213986\\
117	0.00431579028988224\\
118	0.00431580697191418\\
119	0.00431582396392129\\
120	0.00431584127169621\\
121	0.00431585890114065\\
122	0.00431587685826738\\
123	0.0043158951492025\\
124	0.00431591378018745\\
125	0.0043159327575813\\
126	0.00431595208786295\\
127	0.0043159717776334\\
128	0.00431599183361809\\
129	0.00431601226266936\\
130	0.00431603307176863\\
131	0.00431605426802919\\
132	0.00431607585869843\\
133	0.00431609785116058\\
134	0.00431612025293926\\
135	0.00431614307170008\\
136	0.00431616631525346\\
137	0.00431618999155733\\
138	0.00431621410872\\
139	0.00431623867500296\\
140	0.00431626369882397\\
141	0.00431628918875977\\
142	0.00431631515354941\\
143	0.00431634160209725\\
144	0.00431636854347602\\
145	0.00431639598693029\\
146	0.00431642394187952\\
147	0.00431645241792153\\
148	0.00431648142483594\\
149	0.00431651097258764\\
150	0.00431654107133032\\
151	0.00431657173141015\\
152	0.00431660296336936\\
153	0.00431663477795024\\
154	0.00431666718609875\\
155	0.00431670019896855\\
156	0.00431673382792501\\
157	0.00431676808454926\\
158	0.00431680298064243\\
159	0.00431683852822968\\
160	0.00431687473956477\\
161	0.00431691162713437\\
162	0.0043169492036625\\
163	0.0043169874821152\\
164	0.00431702647570518\\
165	0.00431706619789658\\
166	0.00431710666240985\\
167	0.00431714788322675\\
168	0.00431718987459531\\
169	0.00431723265103512\\
170	0.00431727622734244\\
171	0.00431732061859577\\
172	0.00431736584016113\\
173	0.00431741190769776\\
174	0.00431745883716381\\
175	0.00431750664482215\\
176	0.00431755534724629\\
177	0.00431760496132637\\
178	0.00431765550427544\\
179	0.00431770699363569\\
180	0.00431775944728483\\
181	0.00431781288344283\\
182	0.00431786732067833\\
183	0.00431792277791573\\
184	0.00431797927444189\\
185	0.00431803682991343\\
186	0.00431809546436388\\
187	0.00431815519821102\\
188	0.00431821605226454\\
189	0.00431827804773369\\
190	0.00431834120623502\\
191	0.00431840554980058\\
192	0.00431847110088596\\
193	0.00431853788237864\\
194	0.0043186059176065\\
195	0.00431867523034657\\
196	0.00431874584483381\\
197	0.00431881778577029\\
198	0.00431889107833416\\
199	0.00431896574818935\\
200	0.004319041821495\\
201	0.0043191193249154\\
202	0.00431919828562992\\
203	0.00431927873134326\\
204	0.0043193606902959\\
205	0.0043194441912748\\
206	0.00431952926362411\\
207	0.00431961593725647\\
208	0.0043197042426642\\
209	0.00431979421093099\\
210	0.00431988587374356\\
211	0.00431997926340383\\
212	0.00432007441284126\\
213	0.00432017135562532\\
214	0.00432027012597845\\
215	0.00432037075878894\\
216	0.00432047328962461\\
217	0.00432057775474629\\
218	0.00432068419112179\\
219	0.00432079263644017\\
220	0.00432090312912635\\
221	0.00432101570835582\\
222	0.00432113041406999\\
223	0.00432124728699163\\
224	0.00432136636864066\\
225	0.00432148770135029\\
226	0.00432161132828377\\
227	0.00432173729345093\\
228	0.00432186564172562\\
229	0.00432199641886331\\
230	0.00432212967151902\\
231	0.00432226544726559\\
232	0.00432240379461269\\
233	0.00432254476302579\\
234	0.00432268840294584\\
235	0.00432283476580929\\
236	0.00432298390406842\\
237	0.0043231358712123\\
238	0.00432329072178821\\
239	0.00432344851142332\\
240	0.004323609296847\\
241	0.00432377313591361\\
242	0.00432394008762582\\
243	0.00432411021215819\\
244	0.00432428357088166\\
245	0.00432446022638823\\
246	0.00432464024251637\\
247	0.00432482368437691\\
248	0.0043250106183795\\
249	0.00432520111225972\\
250	0.00432539523510669\\
251	0.00432559305739126\\
252	0.0043257946509949\\
253	0.00432600008923925\\
254	0.00432620944691617\\
255	0.00432642280031853\\
256	0.00432664022727168\\
257	0.00432686180716552\\
258	0.00432708762098739\\
259	0.00432731775135558\\
260	0.00432755228255349\\
261	0.0043277913005647\\
262	0.00432803489310863\\
263	0.00432828314967698\\
264	0.00432853616157106\\
265	0.00432879402193971\\
266	0.00432905682581817\\
267	0.00432932467016762\\
268	0.00432959765391561\\
269	0.00432987587799733\\
270	0.00433015944539764\\
271	0.00433044846119397\\
272	0.00433074303260004\\
273	0.00433104326901036\\
274	0.00433134928204578\\
275	0.00433166118559969\\
276	0.00433197909588508\\
277	0.00433230313148254\\
278	0.00433263341338901\\
279	0.00433297006506743\\
280	0.00433331321249714\\
281	0.00433366298422504\\
282	0.00433401951141767\\
283	0.00433438292791386\\
284	0.00433475337027831\\
285	0.00433513097785564\\
286	0.00433551589282521\\
287	0.00433590826025656\\
288	0.00433630822816532\\
289	0.00433671594756963\\
290	0.00433713157254702\\
291	0.00433755526029147\\
292	0.00433798717117093\\
293	0.00433842746878474\\
294	0.00433887632002143\\
295	0.00433933389511596\\
296	0.00433980036770704\\
297	0.00434027591489401\\
298	0.00434076071729329\\
299	0.00434125495909364\\
300	0.00434175882811116\\
301	0.00434227251584266\\
302	0.00434279621751782\\
303	0.00434333013214985\\
304	0.00434387446258422\\
305	0.00434442941554523\\
306	0.0043449952016803\\
307	0.00434557203560147\\
308	0.00434616013592399\\
309	0.00434675972530172\\
310	0.00434737103045892\\
311	0.00434799428221822\\
312	0.00434862971552461\\
313	0.00434927756946509\\
314	0.00434993808728401\\
315	0.00435061151639384\\
316	0.00435129810838146\\
317	0.00435199811901004\\
318	0.00435271180821678\\
319	0.0043534394401069\\
320	0.00435418128294456\\
321	0.00435493760914164\\
322	0.00435570869524555\\
323	0.00435649482192759\\
324	0.0043572962739739\\
325	0.00435811334028154\\
326	0.00435894631386279\\
327	0.00435979549186134\\
328	0.00436066117558451\\
329	0.00436154367055733\\
330	0.0043624432866041\\
331	0.00436336033796421\\
332	0.00436429514345049\\
333	0.00436524802665853\\
334	0.00436621931623619\\
335	0.00436720934622373\\
336	0.00436821845647436\\
337	0.00436924699316567\\
338	0.00437029530941072\\
339	0.00437136376597633\\
340	0.00437245273211302\\
341	0.00437356258649627\\
342	0.00437469371827179\\
343	0.00437584652819098\\
344	0.00437702142981376\\
345	0.00437821885075589\\
346	0.00437943923394699\\
347	0.00438068303683586\\
348	0.0043819507271359\\
349	0.00438324278311871\\
350	0.00438455969391738\\
351	0.00438590195984028\\
352	0.00438727009269485\\
353	0.00438866461612229\\
354	0.00439008606594292\\
355	0.00439153499051145\\
356	0.00439301195107609\\
357	0.00439451752214569\\
358	0.00439605229186454\\
359	0.00439761686239407\\
360	0.00439921185030041\\
361	0.00440083788694632\\
362	0.00440249561888556\\
363	0.00440418570825825\\
364	0.00440590883318406\\
365	0.00440766568815095\\
366	0.00440945698439527\\
367	0.0044112834502697\\
368	0.00441314583159376\\
369	0.00441504489198145\\
370	0.00441698141313957\\
371	0.00441895619512892\\
372	0.00442097005658\\
373	0.00442302383485335\\
374	0.0044251183861332\\
375	0.00442725458544242\\
376	0.00442943332656388\\
377	0.00443165552185372\\
378	0.00443392210192886\\
379	0.00443623401521027\\
380	0.00443859222730249\\
381	0.00444099772018842\\
382	0.00444345149121776\\
383	0.00444595455186778\\
384	0.00444850792625582\\
385	0.00445111264938523\\
386	0.00445376976511048\\
387	0.00445648032381303\\
388	0.00445924537978965\\
389	0.00446206598836737\\
390	0.00446494320277788\\
391	0.00446787807084819\\
392	0.00447087163159433\\
393	0.00447392491184105\\
394	0.00447703892302631\\
395	0.00448021465837707\\
396	0.00448345309066581\\
397	0.00448675517093273\\
398	0.00449012182999356\\
399	0.00449355412269602\\
400	0.00449705347939348\\
401	0.00450062137353033\\
402	0.00450425932415012\\
403	0.00450796889867491\\
404	0.00451175171598503\\
405	0.00451560944982605\\
406	0.00451954383256431\\
407	0.00452355665930099\\
408	0.00452764979233868\\
409	0.00453182516598714\\
410	0.0045360847917572\\
411	0.00454043076433954\\
412	0.00454486526831602\\
413	0.00454939058452903\\
414	0.00455400909656978\\
415	0.00455872329750913\\
416	0.00456353579682076\\
417	0.00456844932732751\\
418	0.00457346675197657\\
419	0.00457859107016703\\
420	0.00458382542321563\\
421	0.00458917309840705\\
422	0.00459463753106349\\
423	0.00460022230356219\\
424	0.0046059311395983\\
425	0.00461176788800474\\
426	0.00461773648292344\\
427	0.00462384083473687\\
428	0.00463007454269391\\
429	0.00463642669478293\\
430	0.00464289947138564\\
431	0.00464949507804951\\
432	0.00465621574373993\\
433	0.00466306371942736\\
434	0.00467004127672492\\
435	0.00467715070359887\\
436	0.00468439430175754\\
437	0.00469177438409394\\
438	0.00469929327236011\\
439	0.0047069532953228\\
440	0.00471475678774227\\
441	0.00472270609062633\\
442	0.00473080355332531\\
443	0.00473905153808924\\
444	0.00474745242750651\\
445	0.00475600863417552\\
446	0.00476472260824388\\
447	0.00477359682704871\\
448	0.00478263371851145\\
449	0.00479185742060597\\
450	0.00480128388406291\\
451	0.00481091912788268\\
452	0.0048207694746903\\
453	0.00483084157395689\\
454	0.00484114242736373\\
455	0.00485167941703031\\
456	0.00486246033618055\\
457	0.00487349342185919\\
458	0.00488478739021973\\
459	0.00489635147453042\\
460	0.00490819546631836\\
461	0.00492032976123099\\
462	0.00493276541531882\\
463	0.00494551423176829\\
464	0.00495858894656423\\
465	0.00497200374370233\\
466	0.00498577276785185\\
467	0.00499990968924556\\
468	0.00501442887447785\\
469	0.00502934546358139\\
470	0.00504467541338836\\
471	0.00506043554226604\\
472	0.00507664357606649\\
473	0.00509331819510764\\
474	0.00511047908213691\\
475	0.00512814697166886\\
476	0.00514634370104921\\
477	0.00516509226166558\\
478	0.00518441684913845\\
479	0.00520434292710943\\
480	0.00522489733605872\\
481	0.00524610843829957\\
482	0.00526800545903931\\
483	0.00529061981117917\\
484	0.00531398865034318\\
485	0.0053381701578214\\
486	0.00536321129115987\\
487	0.00538915358114079\\
488	0.00541604042217931\\
489	0.0054439163752652\\
490	0.00547282747592509\\
491	0.00550282136238553\\
492	0.00553394694540943\\
493	0.0055662540850593\\
494	0.00559979322677082\\
495	0.00563461493277553\\
496	0.00567076928958796\\
497	0.00570830517638911\\
498	0.00574727309521228\\
499	0.00578772535756451\\
500	0.00582971021638085\\
501	0.00587326972706042\\
502	0.00591843697540275\\
503	0.00596523248465533\\
504	0.00601358359371006\\
505	0.00606322632772836\\
506	0.00611408959572197\\
507	0.00616606641138297\\
508	0.00621900369888828\\
509	0.00627268736679497\\
510	0.00632686701159178\\
511	0.00638156012032947\\
512	0.00643639486293027\\
513	0.00649077754730627\\
514	0.00654356250451543\\
515	0.00659100231749131\\
516	0.00663288736294835\\
517	0.00666934946135635\\
518	0.00670138719394604\\
519	0.00673220793885969\\
520	0.00676222399287684\\
521	0.00679192047183419\\
522	0.0068217634860095\\
523	0.00685198358162248\\
524	0.00688268742381229\\
525	0.00691395526121181\\
526	0.00694583429170407\\
527	0.0069783535437772\\
528	0.00701153670833231\\
529	0.00704540320024276\\
530	0.00707997042946935\\
531	0.00711525540894092\\
532	0.0071512752192859\\
533	0.0071880474870327\\
534	0.00722559083761262\\
535	0.0072639251102361\\
536	0.00730307067581672\\
537	0.00734304842914756\\
538	0.00738391399526787\\
539	0.00742574890892478\\
540	0.00746833575598918\\
541	0.00750995398553259\\
542	0.00755035018874908\\
543	0.00759032441223851\\
544	0.00763072324758412\\
545	0.00767162903141855\\
546	0.00771311969698628\\
547	0.00775520894382689\\
548	0.00779789355612366\\
549	0.00784116403536065\\
550	0.007885008292511\\
551	0.00792941148083906\\
552	0.00797435600912795\\
553	0.00801982170725804\\
554	0.00806578702538616\\
555	0.00811222764331581\\
556	0.0081591063939708\\
557	0.00820641693101756\\
558	0.00825328101622345\\
559	0.00829844873438657\\
560	0.00834367510390144\\
561	0.00838916204614102\\
562	0.00843494259992949\\
563	0.00848099134388042\\
564	0.00852727997864265\\
565	0.00857377962259063\\
566	0.008620461017887\\
567	0.00866730268255435\\
568	0.00871391830588996\\
569	0.00876044306491334\\
570	0.00880715274894682\\
571	0.00885403364426453\\
572	0.00890105214164188\\
573	0.00894817203301684\\
574	0.00899535448062322\\
575	0.00904255800889528\\
576	0.00908973854201466\\
577	0.00913684950113236\\
578	0.00918384197868222\\
579	0.0092306650114111\\
580	0.00927726597891577\\
581	0.00932359116078638\\
582	0.00936958649310202\\
583	0.00941519857412799\\
584	0.00946037597948012\\
585	0.00950507095777015\\
586	0.00954924158550055\\
587	0.00959285445497631\\
588	0.00963588792409171\\
589	0.00967833579817709\\
590	0.00972021084271539\\
591	0.00976138933978722\\
592	0.00980171928775089\\
593	0.00984102135019415\\
594	0.00987905595166784\\
595	0.00991543381058343\\
596	0.00994937493726701\\
597	0.00997906286423442\\
598	0.0099999191923403\\
599	0\\
600	0\\
};
\addplot [color=black!50!mycolor20,solid,forget plot]
  table[row sep=crcr]{%
1	0.00432156273776794\\
2	0.0043215647387563\\
3	0.00432156677664413\\
4	0.00432156885211437\\
5	0.00432157096586263\\
6	0.00432157311859742\\
7	0.00432157531104039\\
8	0.00432157754392669\\
9	0.00432157981800506\\
10	0.00432158213403821\\
11	0.00432158449280303\\
12	0.00432158689509076\\
13	0.00432158934170756\\
14	0.00432159183347438\\
15	0.0043215943712275\\
16	0.00432159695581882\\
17	0.00432159958811599\\
18	0.00432160226900288\\
19	0.00432160499937982\\
20	0.00432160778016382\\
21	0.00432161061228898\\
22	0.00432161349670679\\
23	0.00432161643438645\\
24	0.00432161942631524\\
25	0.00432162247349878\\
26	0.00432162557696142\\
27	0.00432162873774662\\
28	0.00432163195691715\\
29	0.00432163523555574\\
30	0.00432163857476522\\
31	0.00432164197566891\\
32	0.0043216454394111\\
33	0.00432164896715743\\
34	0.00432165256009517\\
35	0.00432165621943383\\
36	0.00432165994640539\\
37	0.00432166374226482\\
38	0.00432166760829047\\
39	0.00432167154578451\\
40	0.00432167555607341\\
41	0.00432167964050835\\
42	0.00432168380046576\\
43	0.00432168803734771\\
44	0.00432169235258235\\
45	0.0043216967476246\\
46	0.00432170122395643\\
47	0.00432170578308751\\
48	0.00432171042655564\\
49	0.0043217151559273\\
50	0.00432171997279828\\
51	0.00432172487879413\\
52	0.0043217298755707\\
53	0.00432173496481483\\
54	0.00432174014824476\\
55	0.00432174542761091\\
56	0.00432175080469635\\
57	0.00432175628131747\\
58	0.00432176185932452\\
59	0.00432176754060238\\
60	0.00432177332707118\\
61	0.00432177922068684\\
62	0.00432178522344185\\
63	0.00432179133736603\\
64	0.00432179756452703\\
65	0.00432180390703123\\
66	0.00432181036702437\\
67	0.00432181694669225\\
68	0.00432182364826164\\
69	0.00432183047400093\\
70	0.00432183742622088\\
71	0.0043218445072756\\
72	0.00432185171956308\\
73	0.00432185906552633\\
74	0.00432186654765398\\
75	0.00432187416848134\\
76	0.00432188193059106\\
77	0.00432188983661416\\
78	0.00432189788923104\\
79	0.00432190609117213\\
80	0.0043219144452191\\
81	0.00432192295420573\\
82	0.00432193162101884\\
83	0.00432194044859928\\
84	0.00432194943994308\\
85	0.00432195859810238\\
86	0.00432196792618659\\
87	0.00432197742736336\\
88	0.00432198710485967\\
89	0.00432199696196312\\
90	0.00432200700202283\\
91	0.00432201722845085\\
92	0.00432202764472316\\
93	0.00432203825438095\\
94	0.0043220490610319\\
95	0.00432206006835131\\
96	0.00432207128008351\\
97	0.00432208270004313\\
98	0.00432209433211635\\
99	0.00432210618026241\\
100	0.0043221182485148\\
101	0.00432213054098283\\
102	0.00432214306185296\\
103	0.00432215581539031\\
104	0.00432216880594014\\
105	0.00432218203792936\\
106	0.00432219551586814\\
107	0.00432220924435133\\
108	0.00432222322806028\\
109	0.00432223747176431\\
110	0.00432225198032239\\
111	0.00432226675868503\\
112	0.00432228181189574\\
113	0.00432229714509297\\
114	0.00432231276351192\\
115	0.00432232867248623\\
116	0.00432234487745005\\
117	0.0043223613839398\\
118	0.00432237819759618\\
119	0.00432239532416614\\
120	0.00432241276950487\\
121	0.00432243053957785\\
122	0.00432244864046308\\
123	0.00432246707835305\\
124	0.00432248585955702\\
125	0.00432250499050324\\
126	0.00432252447774112\\
127	0.00432254432794366\\
128	0.00432256454790976\\
129	0.00432258514456658\\
130	0.0043226061249721\\
131	0.00432262749631745\\
132	0.00432264926592956\\
133	0.00432267144127379\\
134	0.00432269402995643\\
135	0.0043227170397275\\
136	0.00432274047848354\\
137	0.00432276435427027\\
138	0.00432278867528554\\
139	0.00432281344988224\\
140	0.0043228386865711\\
141	0.00432286439402405\\
142	0.00432289058107703\\
143	0.00432291725673317\\
144	0.0043229444301661\\
145	0.00432297211072314\\
146	0.00432300030792858\\
147	0.00432302903148723\\
148	0.00432305829128777\\
149	0.00432308809740627\\
150	0.00432311846010989\\
151	0.00432314938986035\\
152	0.00432318089731803\\
153	0.00432321299334535\\
154	0.00432324568901105\\
155	0.00432327899559389\\
156	0.00432331292458686\\
157	0.00432334748770122\\
158	0.00432338269687067\\
159	0.00432341856425585\\
160	0.00432345510224846\\
161	0.00432349232347589\\
162	0.00432353024080571\\
163	0.0043235688673503\\
164	0.00432360821647165\\
165	0.00432364830178617\\
166	0.00432368913716954\\
167	0.0043237307367618\\
168	0.00432377311497254\\
169	0.00432381628648593\\
170	0.00432386026626634\\
171	0.00432390506956349\\
172	0.00432395071191814\\
173	0.00432399720916782\\
174	0.00432404457745243\\
175	0.00432409283322027\\
176	0.00432414199323388\\
177	0.00432419207457645\\
178	0.00432424309465783\\
179	0.00432429507122102\\
180	0.00432434802234867\\
181	0.00432440196646958\\
182	0.00432445692236575\\
183	0.00432451290917902\\
184	0.00432456994641837\\
185	0.00432462805396691\\
186	0.00432468725208936\\
187	0.00432474756143953\\
188	0.00432480900306796\\
189	0.00432487159842961\\
190	0.00432493536939208\\
191	0.0043250003382435\\
192	0.00432506652770101\\
193	0.00432513396091912\\
194	0.00432520266149843\\
195	0.00432527265349443\\
196	0.00432534396142653\\
197	0.00432541661028717\\
198	0.00432549062555138\\
199	0.00432556603318634\\
200	0.00432564285966097\\
201	0.00432572113195616\\
202	0.00432580087757486\\
203	0.00432588212455263\\
204	0.00432596490146808\\
205	0.00432604923745394\\
206	0.00432613516220803\\
207	0.00432622270600464\\
208	0.00432631189970619\\
209	0.00432640277477489\\
210	0.00432649536328497\\
211	0.00432658969793499\\
212	0.0043266858120604\\
213	0.00432678373964642\\
214	0.00432688351534124\\
215	0.00432698517446952\\
216	0.00432708875304603\\
217	0.00432719428778965\\
218	0.00432730181613792\\
219	0.00432741137626148\\
220	0.00432752300707917\\
221	0.00432763674827337\\
222	0.00432775264030556\\
223	0.00432787072443241\\
224	0.00432799104272207\\
225	0.00432811363807102\\
226	0.00432823855422097\\
227	0.00432836583577658\\
228	0.00432849552822319\\
229	0.00432862767794518\\
230	0.00432876233224459\\
231	0.00432889953936042\\
232	0.00432903934848804\\
233	0.00432918180979924\\
234	0.00432932697446275\\
235	0.00432947489466525\\
236	0.0043296256236327\\
237	0.0043297792156524\\
238	0.00432993572609534\\
239	0.00433009521143934\\
240	0.0043302577292924\\
241	0.004330423338417\\
242	0.00433059209875454\\
243	0.00433076407145074\\
244	0.00433093931888154\\
245	0.00433111790467953\\
246	0.00433129989376101\\
247	0.00433148535235388\\
248	0.00433167434802614\\
249	0.00433186694971496\\
250	0.00433206322775662\\
251	0.00433226325391725\\
252	0.00433246710142406\\
253	0.00433267484499761\\
254	0.00433288656088486\\
255	0.00433310232689295\\
256	0.00433332222242387\\
257	0.00433354632851018\\
258	0.00433377472785137\\
259	0.00433400750485138\\
260	0.00433424474565704\\
261	0.00433448653819752\\
262	0.00433473297222474\\
263	0.00433498413935495\\
264	0.00433524013311138\\
265	0.004335501048968\\
266	0.00433576698439459\\
267	0.00433603803890286\\
268	0.00433631431409392\\
269	0.00433659591370706\\
270	0.0043368829436698\\
271	0.00433717551214946\\
272	0.00433747372960603\\
273	0.00433777770884665\\
274	0.00433808756508153\\
275	0.00433840341598155\\
276	0.0043387253817374\\
277	0.00433905358512051\\
278	0.00433938815154574\\
279	0.00433972920913584\\
280	0.00434007688878776\\
281	0.00434043132424101\\
282	0.00434079265214799\\
283	0.00434116101214642\\
284	0.00434153654693376\\
285	0.004341919402344\\
286	0.00434230972742683\\
287	0.00434270767452879\\
288	0.00434311339937728\\
289	0.00434352706116669\\
290	0.00434394882264741\\
291	0.00434437885021715\\
292	0.00434481731401516\\
293	0.00434526438801898\\
294	0.00434572025014423\\
295	0.00434618508234692\\
296	0.00434665907072889\\
297	0.00434714240564587\\
298	0.00434763528181837\\
299	0.00434813789844575\\
300	0.00434865045932256\\
301	0.00434917317295794\\
302	0.00434970625269746\\
303	0.00435024991684725\\
304	0.00435080438880053\\
305	0.00435136989716599\\
306	0.00435194667589759\\
307	0.00435253496442565\\
308	0.00435313500778857\\
309	0.00435374705676422\\
310	0.00435437136800087\\
311	0.00435500820414609\\
312	0.00435565783397282\\
313	0.0043563205325014\\
314	0.0043569965811155\\
315	0.00435768626767073\\
316	0.00435838988659347\\
317	0.00435910773896732\\
318	0.0043598401326048\\
319	0.00436058738210031\\
320	0.0043613498088614\\
321	0.00436212774111298\\
322	0.00436292151387087\\
323	0.0043637314688778\\
324	0.00436455795449694\\
325	0.00436540132555528\\
326	0.00436626194312957\\
327	0.00436714017426647\\
328	0.00436803639162843\\
329	0.00436895097305518\\
330	0.00436988430103141\\
331	0.0043708367620511\\
332	0.00437180874586785\\
333	0.00437280064462305\\
334	0.00437381285184419\\
335	0.00437484576130809\\
336	0.0043758997657681\\
337	0.00437697525554858\\
338	0.00437807261701865\\
339	0.00437919223096592\\
340	0.00438033447090391\\
341	0.00438149970136107\\
342	0.00438268827621516\\
343	0.00438390053714678\\
344	0.00438513681229731\\
345	0.00438639741528997\\
346	0.00438768264538279\\
347	0.00438899284910283\\
348	0.00439032852186834\\
349	0.00439169016963117\\
350	0.00439307830916022\\
351	0.00439449346833624\\
352	0.00439593618645757\\
353	0.00439740701455486\\
354	0.00439890651571774\\
355	0.004400435265466\\
356	0.00440199385231873\\
357	0.00440358287824578\\
358	0.00440520295913732\\
359	0.00440685472530561\\
360	0.00440853882202268\\
361	0.00441025591009763\\
362	0.00441200666649775\\
363	0.0044137917850179\\
364	0.0044156119770036\\
365	0.00441746797213329\\
366	0.00441936051926643\\
367	0.00442129038736393\\
368	0.00442325836648904\\
369	0.00442526526889701\\
370	0.00442731193022267\\
371	0.0044293992107759\\
372	0.00443152799695614\\
373	0.00443369920279728\\
374	0.00443591377165537\\
375	0.00443817267805222\\
376	0.00444047692968832\\
377	0.0044428275696388\\
378	0.00444522567874573\\
379	0.00444767237822\\
380	0.00445016883246353\\
381	0.00445271625212137\\
382	0.00445531589736821\\
383	0.00445796908142958\\
384	0.00446067717432911\\
385	0.00446344160684308\\
386	0.00446626387462823\\
387	0.0044691455424694\\
388	0.00447208824856691\\
389	0.00447509370874885\\
390	0.00447816372044754\\
391	0.00448130016621814\\
392	0.00448450501649252\\
393	0.00448778033113468\\
394	0.00449112825913685\\
395	0.00449455103528687\\
396	0.00449805097120155\\
397	0.00450163043343578\\
398	0.00450529178476443\\
399	0.00450903312499081\\
400	0.00451284518155388\\
401	0.00451672918428431\\
402	0.00452068637426448\\
403	0.00452471800279135\\
404	0.0045288253302109\\
405	0.0045330096246486\\
406	0.00453727216070178\\
407	0.00454161421822174\\
408	0.00454603708133855\\
409	0.00455054203755669\\
410	0.00455513037477349\\
411	0.00455980336567712\\
412	0.0045645622530175\\
413	0.00456940826439893\\
414	0.00457434261038014\\
415	0.00457936647922931\\
416	0.00458448103142263\\
417	0.00458968739511137\\
418	0.00459498666169048\\
419	0.00460037988191109\\
420	0.0046058680639046\\
421	0.0046114521734614\\
422	0.00461713313071034\\
423	0.00462291180709291\\
424	0.00462878901320134\\
425	0.00463476551515811\\
426	0.00464084202686863\\
427	0.00464701916605228\\
428	0.00465330764650406\\
429	0.00465972294715053\\
430	0.00466626796949668\\
431	0.0046729457221835\\
432	0.00467975932905515\\
433	0.00468671203806247\\
434	0.00469380723103485\\
435	0.0047010484344903\\
436	0.0047084393316286\\
437	0.00471598377555301\\
438	0.00472368580381427\\
439	0.00473154965436415\\
440	0.004739579782992\\
441	0.00474778088230933\\
442	0.00475615790236735\\
443	0.00476471607315437\\
444	0.00477346092989589\\
445	0.00478239834456105\\
446	0.00479153457560125\\
447	0.00480087637727662\\
448	0.00481043130862172\\
449	0.00482020706581279\\
450	0.00483021084487188\\
451	0.00484044985114416\\
452	0.00485093163581722\\
453	0.00486166411637998\\
454	0.00487265559813321\\
455	0.00488391479673149\\
456	0.00489545086173582\\
457	0.00490727340119204\\
458	0.00491939250723813\\
459	0.00493181878277906\\
460	0.00494456336936779\\
461	0.00495763797663287\\
462	0.00497105491383159\\
463	0.0049848271235522\\
464	0.0049989682118395\\
465	0.0050134924396604\\
466	0.00502841475798565\\
467	0.00504375091022794\\
468	0.00505951747774275\\
469	0.00507573192480233\\
470	0.0050924126447223\\
471	0.00510957900694087\\
472	0.00512725140490503\\
473	0.00514545130482393\\
474	0.0051642012957185\\
475	0.0051835251411441\\
476	0.00520344785397583\\
477	0.00522399583307634\\
478	0.0052451969499694\\
479	0.00526708001994137\\
480	0.00528967582130181\\
481	0.00531302073050156\\
482	0.00533717128049394\\
483	0.00536217440954135\\
484	0.00538807060812187\\
485	0.00541490203822802\\
486	0.00544271156614922\\
487	0.00547154299835975\\
488	0.00550144111267412\\
489	0.00553245141674796\\
490	0.00556461982902692\\
491	0.00559799228696207\\
492	0.00563262021729175\\
493	0.00566855676452508\\
494	0.00570585375095233\\
495	0.00574456031946986\\
496	0.00578472113107927\\
497	0.00582637393627206\\
498	0.00586941332332261\\
499	0.00591367412563156\\
500	0.0059591312442232\\
501	0.00600573960009374\\
502	0.00605342869976447\\
503	0.00610209427571644\\
504	0.00615166353661207\\
505	0.00620218464799494\\
506	0.0062534348899334\\
507	0.00630517093441952\\
508	0.00635706597096826\\
509	0.00640860003944755\\
510	0.00645893456739127\\
511	0.0065044826325323\\
512	0.00654456380460978\\
513	0.00657922957752016\\
514	0.00660912384878171\\
515	0.00663781310127959\\
516	0.00666570035195224\\
517	0.00669325982984133\\
518	0.00672095204731527\\
519	0.0067490035419314\\
520	0.00677751497187324\\
521	0.00680656037357772\\
522	0.00683618266850091\\
523	0.00686640821868181\\
524	0.00689725841090671\\
525	0.0069287508654358\\
526	0.00696090155737811\\
527	0.00699372616758012\\
528	0.00702724047294954\\
529	0.00706146075026851\\
530	0.00709640408608181\\
531	0.00713208865138898\\
532	0.00716853402710913\\
533	0.00720576089941698\\
534	0.00724379067528484\\
535	0.00728265318316113\\
536	0.00732242355127311\\
537	0.00736319290804136\\
538	0.00740362713540518\\
539	0.00744294329917313\\
540	0.00748116645121112\\
541	0.00751979739156501\\
542	0.00755890721054429\\
543	0.00759857987564944\\
544	0.00763885345226277\\
545	0.00767972887409422\\
546	0.00772120151702209\\
547	0.00776326398074099\\
548	0.00780590660744546\\
549	0.00784911740722832\\
550	0.00789288179370365\\
551	0.00793718233108313\\
552	0.0079819985880279\\
553	0.00802730742623898\\
554	0.00807308398607837\\
555	0.00811930438610058\\
556	0.00816597124869282\\
557	0.00821131168944101\\
558	0.00825575379156772\\
559	0.00830045453103746\\
560	0.00834548332965024\\
561	0.00839082502862416\\
562	0.00843645437478884\\
563	0.00848234516615932\\
564	0.00852847072095564\\
565	0.00857480405939613\\
566	0.00862132528999987\\
567	0.00866764572460421\\
568	0.00871393573045342\\
569	0.00876044307747747\\
570	0.00880715274928313\\
571	0.0088540336444077\\
572	0.00890105214171166\\
573	0.00894817203305002\\
574	0.00899535448063817\\
575	0.00904255800890162\\
576	0.00908973854201715\\
577	0.00913684950113324\\
578	0.0091838419786825\\
579	0.00923066501141119\\
580	0.00927726597891579\\
581	0.00932359116078638\\
582	0.00936958649310201\\
583	0.00941519857412797\\
584	0.00946037597948012\\
585	0.00950507095777015\\
586	0.00954924158550055\\
587	0.00959285445497631\\
588	0.00963588792409171\\
589	0.00967833579817709\\
590	0.00972021084271538\\
591	0.00976138933978722\\
592	0.00980171928775089\\
593	0.00984102135019415\\
594	0.00987905595166784\\
595	0.00991543381058343\\
596	0.00994937493726701\\
597	0.00997906286423442\\
598	0.0099999191923403\\
599	0\\
600	0\\
};
\addplot [color=black!60!mycolor21,solid,forget plot]
  table[row sep=crcr]{%
1	0.00432695855327979\\
2	0.00432696069719906\\
3	0.00432696288063723\\
4	0.00432696510432514\\
5	0.00432696736900736\\
6	0.00432696967544223\\
7	0.0043269720244022\\
8	0.00432697441667397\\
9	0.004326976853059\\
10	0.00432697933437348\\
11	0.0043269818614489\\
12	0.00432698443513213\\
13	0.00432698705628577\\
14	0.00432698972578843\\
15	0.00432699244453511\\
16	0.0043269952134374\\
17	0.0043269980334239\\
18	0.00432700090544033\\
19	0.00432700383045009\\
20	0.00432700680943442\\
21	0.00432700984339293\\
22	0.00432701293334372\\
23	0.00432701608032385\\
24	0.00432701928538961\\
25	0.004327022549617\\
26	0.004327025874102\\
27	0.00432702925996095\\
28	0.00432703270833108\\
29	0.00432703622037062\\
30	0.0043270397972594\\
31	0.00432704344019931\\
32	0.0043270471504144\\
33	0.00432705092915171\\
34	0.00432705477768136\\
35	0.00432705869729714\\
36	0.00432706268931694\\
37	0.00432706675508314\\
38	0.00432707089596317\\
39	0.00432707511334992\\
40	0.00432707940866214\\
41	0.00432708378334511\\
42	0.00432708823887093\\
43	0.00432709277673917\\
44	0.00432709739847733\\
45	0.00432710210564138\\
46	0.00432710689981621\\
47	0.0043271117826163\\
48	0.00432711675568621\\
49	0.00432712182070122\\
50	0.00432712697936762\\
51	0.0043271322334237\\
52	0.00432713758464017\\
53	0.00432714303482059\\
54	0.00432714858580225\\
55	0.00432715423945664\\
56	0.00432715999769022\\
57	0.004327165862445\\
58	0.00432717183569916\\
59	0.00432717791946777\\
60	0.00432718411580356\\
61	0.0043271904267975\\
62	0.00432719685457959\\
63	0.00432720340131958\\
64	0.00432721006922769\\
65	0.00432721686055543\\
66	0.00432722377759633\\
67	0.0043272308226867\\
68	0.00432723799820648\\
69	0.00432724530658002\\
70	0.00432725275027702\\
71	0.00432726033181314\\
72	0.00432726805375125\\
73	0.00432727591870186\\
74	0.00432728392932442\\
75	0.00432729208832793\\
76	0.00432730039847212\\
77	0.0043273088625682\\
78	0.00432731748347992\\
79	0.0043273262641246\\
80	0.00432733520747408\\
81	0.00432734431655567\\
82	0.00432735359445329\\
83	0.00432736304430868\\
84	0.00432737266932216\\
85	0.00432738247275397\\
86	0.00432739245792525\\
87	0.00432740262821932\\
88	0.00432741298708276\\
89	0.00432742353802666\\
90	0.00432743428462778\\
91	0.00432744523052981\\
92	0.00432745637944468\\
93	0.00432746773515366\\
94	0.00432747930150892\\
95	0.00432749108243471\\
96	0.00432750308192877\\
97	0.00432751530406363\\
98	0.00432752775298816\\
99	0.00432754043292886\\
100	0.00432755334819144\\
101	0.00432756650316226\\
102	0.00432757990230986\\
103	0.00432759355018648\\
104	0.00432760745142971\\
105	0.00432762161076401\\
106	0.00432763603300237\\
107	0.00432765072304815\\
108	0.00432766568589645\\
109	0.00432768092663611\\
110	0.0043276964504514\\
111	0.00432771226262376\\
112	0.0043277283685338\\
113	0.00432774477366301\\
114	0.00432776148359568\\
115	0.00432777850402097\\
116	0.00432779584073471\\
117	0.00432781349964156\\
118	0.00432783148675699\\
119	0.00432784980820943\\
120	0.00432786847024234\\
121	0.00432788747921642\\
122	0.00432790684161185\\
123	0.00432792656403053\\
124	0.00432794665319832\\
125	0.00432796711596752\\
126	0.00432798795931918\\
127	0.00432800919036546\\
128	0.00432803081635228\\
129	0.00432805284466176\\
130	0.00432807528281481\\
131	0.00432809813847368\\
132	0.00432812141944487\\
133	0.00432814513368153\\
134	0.00432816928928657\\
135	0.00432819389451525\\
136	0.00432821895777821\\
137	0.00432824448764443\\
138	0.00432827049284404\\
139	0.00432829698227167\\
140	0.0043283239649894\\
141	0.00432835145022986\\
142	0.00432837944739962\\
143	0.00432840796608247\\
144	0.00432843701604282\\
145	0.00432846660722887\\
146	0.00432849674977659\\
147	0.00432852745401276\\
148	0.00432855873045901\\
149	0.0043285905898353\\
150	0.00432862304306379\\
151	0.00432865610127265\\
152	0.00432868977579992\\
153	0.00432872407819774\\
154	0.00432875902023616\\
155	0.00432879461390737\\
156	0.00432883087143009\\
157	0.00432886780525368\\
158	0.00432890542806266\\
159	0.00432894375278118\\
160	0.0043289827925775\\
161	0.0043290225608689\\
162	0.00432906307132606\\
163	0.00432910433787831\\
164	0.00432914637471825\\
165	0.00432918919630696\\
166	0.00432923281737907\\
167	0.00432927725294803\\
168	0.00432932251831144\\
169	0.0043293686290565\\
170	0.00432941560106546\\
171	0.0043294634505214\\
172	0.00432951219391403\\
173	0.00432956184804536\\
174	0.00432961243003595\\
175	0.00432966395733081\\
176	0.00432971644770586\\
177	0.00432976991927409\\
178	0.00432982439049204\\
179	0.00432987988016658\\
180	0.00432993640746147\\
181	0.00432999399190443\\
182	0.00433005265339393\\
183	0.00433011241220651\\
184	0.0043301732890039\\
185	0.00433023530484067\\
186	0.00433029848117162\\
187	0.00433036283985964\\
188	0.00433042840318345\\
189	0.00433049519384591\\
190	0.00433056323498191\\
191	0.00433063255016688\\
192	0.00433070316342541\\
193	0.00433077509923983\\
194	0.00433084838255915\\
195	0.00433092303880804\\
196	0.00433099909389616\\
197	0.0043310765742276\\
198	0.00433115550671049\\
199	0.00433123591876661\\
200	0.00433131783834173\\
201	0.00433140129391559\\
202	0.00433148631451246\\
203	0.00433157292971159\\
204	0.00433166116965828\\
205	0.00433175106507474\\
206	0.00433184264727161\\
207	0.00433193594815924\\
208	0.00433203100025962\\
209	0.00433212783671843\\
210	0.00433222649131709\\
211	0.00433232699848539\\
212	0.00433242939331433\\
213	0.00433253371156901\\
214	0.00433263998970192\\
215	0.00433274826486667\\
216	0.00433285857493167\\
217	0.00433297095849443\\
218	0.0043330854548959\\
219	0.00433320210423526\\
220	0.00433332094738493\\
221	0.00433344202600606\\
222	0.00433356538256405\\
223	0.00433369106034479\\
224	0.00433381910347088\\
225	0.00433394955691843\\
226	0.00433408246653407\\
227	0.00433421787905251\\
228	0.00433435584211419\\
229	0.00433449640428373\\
230	0.00433463961506829\\
231	0.00433478552493675\\
232	0.00433493418533899\\
233	0.00433508564872588\\
234	0.00433523996856943\\
235	0.00433539719938359\\
236	0.00433555739674543\\
237	0.00433572061731673\\
238	0.0043358869188662\\
239	0.00433605636029206\\
240	0.00433622900164514\\
241	0.00433640490415262\\
242	0.00433658413024219\\
243	0.00433676674356681\\
244	0.00433695280902993\\
245	0.00433714239281142\\
246	0.0043373355623941\\
247	0.00433753238659085\\
248	0.0043377329355722\\
249	0.00433793728089482\\
250	0.00433814549553049\\
251	0.00433835765389585\\
252	0.00433857383188282\\
253	0.00433879410688977\\
254	0.00433901855785339\\
255	0.00433924726528139\\
256	0.00433948031128586\\
257	0.00433971777961768\\
258	0.00433995975570167\\
259	0.00434020632667236\\
260	0.00434045758141111\\
261	0.0043407136105838\\
262	0.00434097450667962\\
263	0.00434124036405092\\
264	0.00434151127895385\\
265	0.00434178734959036\\
266	0.00434206867615101\\
267	0.00434235536085907\\
268	0.00434264750801582\\
269	0.00434294522404696\\
270	0.00434324861755036\\
271	0.00434355779934511\\
272	0.00434387288252188\\
273	0.00434419398249479\\
274	0.00434452121705471\\
275	0.00434485470642407\\
276	0.00434519457331332\\
277	0.00434554094297909\\
278	0.00434589394328401\\
279	0.00434625370475853\\
280	0.0043466203606644\\
281	0.00434699404706052\\
282	0.00434737490287053\\
283	0.00434776306995298\\
284	0.00434815869317363\\
285	0.00434856192048034\\
286	0.0043489729029806\\
287	0.00434939179502176\\
288	0.00434981875427422\\
289	0.00435025394181805\\
290	0.00435069752223251\\
291	0.00435114966368958\\
292	0.00435161053805094\\
293	0.00435208032096964\\
294	0.00435255919199538\\
295	0.00435304733468546\\
296	0.00435354493672011\\
297	0.00435405219002397\\
298	0.00435456929089321\\
299	0.00435509644012949\\
300	0.0043556338431809\\
301	0.00435618171029078\\
302	0.0043567402566551\\
303	0.00435730970258917\\
304	0.00435789027370451\\
305	0.0043584822010971\\
306	0.00435908572154804\\
307	0.00435970107773785\\
308	0.00436032851847588\\
309	0.0043609682989465\\
310	0.0043616206809735\\
311	0.00436228593330542\\
312	0.00436296433192296\\
313	0.00436365616037193\\
314	0.00436436171012386\\
315	0.00436508128096751\\
316	0.00436581518143451\\
317	0.00436656372926307\\
318	0.00436732725190336\\
319	0.00436810608706939\\
320	0.00436890058334199\\
321	0.00436971110082804\\
322	0.00437053801188148\\
323	0.0043713817018922\\
324	0.00437224257014867\\
325	0.0043731210307803\\
326	0.00437401751378662\\
327	0.00437493246615868\\
328	0.00437586635309825\\
329	0.0043768196593399\\
330	0.00437779289057866\\
331	0.00437878657500424\\
332	0.00437980126494006\\
333	0.00438083753858002\\
334	0.0043818960018099\\
335	0.00438297729009362\\
336	0.00438408207039105\\
337	0.0043852110430619\\
338	0.00438636494368967\\
339	0.00438754454473416\\
340	0.00438875065688568\\
341	0.00438998412993984\\
342	0.00439124585291623\\
343	0.00439253675292674\\
344	0.00439385779169215\\
345	0.00439520995665367\\
346	0.00439659423677513\\
347	0.00439800980650354\\
348	0.00439945288719747\\
349	0.00440092400728502\\
350	0.00440242370465192\\
351	0.00440395252680818\\
352	0.00440551103111611\\
353	0.00440709978509535\\
354	0.00440871936667319\\
355	0.00441037036353843\\
356	0.00441205336796628\\
357	0.00441376898130935\\
358	0.00441551781433659\\
359	0.00441730048719107\\
360	0.00441911762932339\\
361	0.00442096987939735\\
362	0.00442285788516469\\
363	0.00442478230330447\\
364	0.00442674379922321\\
365	0.0044287430468102\\
366	0.00443078072814295\\
367	0.00443285753313633\\
368	0.00443497415912887\\
369	0.00443713131039822\\
370	0.00443932969759767\\
371	0.00444157003710466\\
372	0.00444385305027066\\
373	0.00444617946256172\\
374	0.00444855000257832\\
375	0.00445096540094059\\
376	0.00445342638902633\\
377	0.00445593369754693\\
378	0.00445848805494721\\
379	0.00446109018561414\\
380	0.00446374080788017\\
381	0.00446644063180854\\
382	0.00446919035674984\\
383	0.00447199066866269\\
384	0.00447484223719717\\
385	0.00447774571254766\\
386	0.0044807017220934\\
387	0.00448371086686047\\
388	0.00448677371786047\\
389	0.00448989081239065\\
390	0.00449306265041508\\
391	0.00449628969119351\\
392	0.0044995723503721\\
393	0.00450291099778542\\
394	0.00450630595618257\\
395	0.00450975750081975\\
396	0.00451326585891238\\
397	0.00451683120515006\\
398	0.00452045364209266\\
399	0.00452413735820681\\
400	0.00452789394284761\\
401	0.00453172483387663\\
402	0.00453563150123215\\
403	0.00453961544838681\\
404	0.00454367821396767\\
405	0.00454782137356271\\
406	0.00455204654173728\\
407	0.00455635537427974\\
408	0.00456074957067161\\
409	0.00456523087672764\\
410	0.00456980108733492\\
411	0.0045744620497789\\
412	0.00457921566841893\\
413	0.00458406390985492\\
414	0.00458900880804196\\
415	0.00459405246996129\\
416	0.00459919708201783\\
417	0.00460444491720469\\
418	0.00460979834306404\\
419	0.00461525983049784\\
420	0.00462083196341752\\
421	0.00462651744911842\\
422	0.0046323191295337\\
423	0.00463823999399964\\
424	0.00464428319606497\\
425	0.00465045208049502\\
426	0.00465675024305133\\
427	0.00466318170211106\\
428	0.00466975068776097\\
429	0.00467646102269409\\
430	0.00468331626401164\\
431	0.00469032009822216\\
432	0.00469747634809923\\
433	0.00470478897990422\\
434	0.00471226211098159\\
435	0.00471990001772997\\
436	0.00472770714394326\\
437	0.00473568810951569\\
438	0.00474384771949964\\
439	0.00475219097350291\\
440	0.00476072307541075\\
441	0.0047694494434255\\
442	0.00477837572044345\\
443	0.00478750778485379\\
444	0.00479685176198382\\
445	0.0048064140365962\\
446	0.00481620126662683\\
447	0.00482622039523245\\
448	0.00483647864149911\\
449	0.004846983503795\\
450	0.00485774280977831\\
451	0.00486876475354069\\
452	0.00488005791751922\\
453	0.00489163129574941\\
454	0.00490349431852351\\
455	0.00491565687852088\\
456	0.00492812935847706\\
457	0.00494092266045717\\
458	0.00495404823679719\\
459	0.00496751812277545\\
460	0.00498134497106543\\
461	0.00499554208798379\\
462	0.00501012347143418\\
463	0.00502510385021868\\
464	0.00504049872440022\\
465	0.00505632440914279\\
466	0.00507259808091958\\
467	0.00508933782211824\\
468	0.0051065626661562\\
469	0.00512429264292472\\
470	0.00514254882450478\\
471	0.00516135337167488\\
472	0.00518072958309387\\
473	0.00520070195230599\\
474	0.00522129624785663\\
475	0.00524253966141403\\
476	0.00526446073178525\\
477	0.00528708936517332\\
478	0.00531046069907047\\
479	0.00533462727696618\\
480	0.00535963747295621\\
481	0.00538552963238027\\
482	0.00541234318243217\\
483	0.0054401178576874\\
484	0.00546889387848687\\
485	0.00549871271132115\\
486	0.00552962358923932\\
487	0.00556167658536844\\
488	0.00559492199409357\\
489	0.00562940946645967\\
490	0.00566518680759279\\
491	0.00570229826537744\\
492	0.00574057270663483\\
493	0.00577994483980594\\
494	0.00582041438361801\\
495	0.00586196977977204\\
496	0.00590458482482532\\
497	0.00594821370429102\\
498	0.0059929190006349\\
499	0.00603878586440403\\
500	0.00608571187789154\\
501	0.00613354908577968\\
502	0.00618209679919275\\
503	0.00623111975679493\\
504	0.0062803460795765\\
505	0.00632932117173365\\
506	0.00637730219787598\\
507	0.00642155840085762\\
508	0.00646045553143953\\
509	0.00649394778892431\\
510	0.00652243322939103\\
511	0.00654927054506817\\
512	0.00657526604504233\\
513	0.00660087923655145\\
514	0.00662658810284878\\
515	0.00665262797492567\\
516	0.00667909600829342\\
517	0.00670606353836837\\
518	0.00673357132505209\\
519	0.00676164372812885\\
520	0.0067903003794335\\
521	0.00681955743687604\\
522	0.00684942956662143\\
523	0.00687993114722263\\
524	0.00691107661826311\\
525	0.00694288081781127\\
526	0.00697535922344092\\
527	0.00700852813728366\\
528	0.00704240489215253\\
529	0.0070770080849942\\
530	0.00711235785255778\\
531	0.00714847517504492\\
532	0.0071853817036795\\
533	0.00722312753643244\\
534	0.00726179417464418\\
535	0.00730115804025518\\
536	0.00733950427705983\\
537	0.00737657424641395\\
538	0.007413491098651\\
539	0.00745084701001097\\
540	0.00748873240113763\\
541	0.00752721478980981\\
542	0.00756629878659511\\
543	0.00760598438697385\\
544	0.00764626809925852\\
545	0.00768714464315555\\
546	0.0077286069021184\\
547	0.00777064572670249\\
548	0.00781324967273179\\
549	0.00785640469595659\\
550	0.00790009381669526\\
551	0.00794429678697347\\
552	0.00798898986048007\\
553	0.00803414520393946\\
554	0.00807975463982383\\
555	0.00812542668900067\\
556	0.00816939338006928\\
557	0.00821329420027352\\
558	0.00825751363257453\\
559	0.00830208407653316\\
560	0.00834698368736043\\
561	0.00839218901342985\\
562	0.00843767581465922\\
563	0.00848341941476406\\
564	0.0085293949138881\\
565	0.00857558303702803\\
566	0.00862162837759405\\
567	0.00866765938100907\\
568	0.00871393573140663\\
569	0.00876044307752332\\
570	0.00880715274930379\\
571	0.00885403364441763\\
572	0.00890105214171626\\
573	0.00894817203305202\\
574	0.00899535448063901\\
575	0.00904255800890193\\
576	0.00908973854201726\\
577	0.00913684950113328\\
578	0.00918384197868252\\
579	0.00923066501141119\\
580	0.00927726597891579\\
581	0.00932359116078638\\
582	0.00936958649310202\\
583	0.00941519857412797\\
584	0.00946037597948011\\
585	0.00950507095777015\\
586	0.00954924158550055\\
587	0.00959285445497631\\
588	0.00963588792409171\\
589	0.00967833579817709\\
590	0.00972021084271538\\
591	0.00976138933978722\\
592	0.00980171928775089\\
593	0.00984102135019415\\
594	0.00987905595166784\\
595	0.00991543381058343\\
596	0.00994937493726701\\
597	0.00997906286423442\\
598	0.0099999191923403\\
599	0\\
600	0\\
};
\addplot [color=black!80!mycolor21,solid,forget plot]
  table[row sep=crcr]{%
1	0.00433439942612947\\
2	0.00433440181287078\\
3	0.00433440424353772\\
4	0.00433440671894131\\
5	0.00433440923990749\\
6	0.0043344118072775\\
7	0.00433441442190828\\
8	0.00433441708467256\\
9	0.00433441979645923\\
10	0.00433442255817373\\
11	0.00433442537073827\\
12	0.00433442823509211\\
13	0.00433443115219192\\
14	0.00433443412301212\\
15	0.0043344371485452\\
16	0.00433444022980203\\
17	0.00433444336781228\\
18	0.00433444656362468\\
19	0.00433444981830739\\
20	0.00433445313294842\\
21	0.00433445650865589\\
22	0.00433445994655854\\
23	0.00433446344780604\\
24	0.0043344670135694\\
25	0.0043344706450413\\
26	0.00433447434343657\\
27	0.00433447810999261\\
28	0.00433448194596979\\
29	0.00433448585265188\\
30	0.00433448983134642\\
31	0.0043344938833852\\
32	0.00433449801012487\\
33	0.00433450221294707\\
34	0.00433450649325921\\
35	0.00433451085249485\\
36	0.00433451529211415\\
37	0.00433451981360434\\
38	0.00433452441848027\\
39	0.00433452910828499\\
40	0.00433453388459013\\
41	0.00433453874899659\\
42	0.00433454370313495\\
43	0.00433454874866611\\
44	0.00433455388728183\\
45	0.00433455912070522\\
46	0.00433456445069154\\
47	0.00433456987902862\\
48	0.00433457540753754\\
49	0.00433458103807315\\
50	0.00433458677252486\\
51	0.00433459261281712\\
52	0.00433459856091015\\
53	0.0043346046188007\\
54	0.00433461078852263\\
55	0.00433461707214761\\
56	0.00433462347178577\\
57	0.00433462998958654\\
58	0.00433463662773934\\
59	0.0043346433884743\\
60	0.00433465027406297\\
61	0.00433465728681925\\
62	0.00433466442910008\\
63	0.00433467170330617\\
64	0.00433467911188301\\
65	0.00433468665732144\\
66	0.00433469434215876\\
67	0.00433470216897956\\
68	0.00433471014041645\\
69	0.00433471825915103\\
70	0.00433472652791482\\
71	0.00433473494949035\\
72	0.00433474352671167\\
73	0.00433475226246583\\
74	0.0043347611596935\\
75	0.00433477022139027\\
76	0.00433477945060735\\
77	0.00433478885045292\\
78	0.00433479842409299\\
79	0.00433480817475246\\
80	0.00433481810571649\\
81	0.00433482822033137\\
82	0.00433483852200576\\
83	0.00433484901421175\\
84	0.00433485970048622\\
85	0.00433487058443192\\
86	0.00433488166971885\\
87	0.00433489296008528\\
88	0.00433490445933922\\
89	0.00433491617135971\\
90	0.00433492810009815\\
91	0.00433494024957949\\
92	0.00433495262390377\\
93	0.00433496522724763\\
94	0.0043349780638655\\
95	0.00433499113809108\\
96	0.00433500445433906\\
97	0.00433501801710641\\
98	0.00433503183097401\\
99	0.00433504590060823\\
100	0.00433506023076251\\
101	0.00433507482627893\\
102	0.00433508969209002\\
103	0.0043351048332203\\
104	0.00433512025478811\\
105	0.00433513596200738\\
106	0.00433515196018933\\
107	0.00433516825474427\\
108	0.00433518485118366\\
109	0.00433520175512177\\
110	0.00433521897227784\\
111	0.00433523650847781\\
112	0.00433525436965644\\
113	0.00433527256185936\\
114	0.00433529109124517\\
115	0.00433530996408747\\
116	0.00433532918677717\\
117	0.00433534876582449\\
118	0.00433536870786138\\
119	0.00433538901964363\\
120	0.00433540970805338\\
121	0.00433543078010148\\
122	0.00433545224292965\\
123	0.00433547410381325\\
124	0.0043354963701636\\
125	0.00433551904953057\\
126	0.00433554214960529\\
127	0.00433556567822269\\
128	0.00433558964336422\\
129	0.00433561405316057\\
130	0.00433563891589457\\
131	0.00433566424000396\\
132	0.00433569003408434\\
133	0.00433571630689219\\
134	0.00433574306734774\\
135	0.00433577032453824\\
136	0.00433579808772094\\
137	0.00433582636632634\\
138	0.00433585516996155\\
139	0.00433588450841341\\
140	0.00433591439165201\\
141	0.00433594482983415\\
142	0.00433597583330677\\
143	0.00433600741261058\\
144	0.00433603957848358\\
145	0.00433607234186507\\
146	0.0043361057138991\\
147	0.00433613970593858\\
148	0.00433617432954894\\
149	0.00433620959651243\\
150	0.00433624551883205\\
151	0.00433628210873568\\
152	0.00433631937868039\\
153	0.00433635734135661\\
154	0.00433639600969271\\
155	0.0043364353968595\\
156	0.00433647551627451\\
157	0.0043365163816069\\
158	0.00433655800678218\\
159	0.00433660040598687\\
160	0.00433664359367367\\
161	0.00433668758456619\\
162	0.00433673239366435\\
163	0.00433677803624933\\
164	0.00433682452788904\\
165	0.00433687188444338\\
166	0.00433692012206996\\
167	0.00433696925722949\\
168	0.00433701930669161\\
169	0.00433707028754067\\
170	0.00433712221718179\\
171	0.0043371751133467\\
172	0.00433722899410006\\
173	0.00433728387784567\\
174	0.00433733978333297\\
175	0.00433739672966335\\
176	0.00433745473629706\\
177	0.00433751382305975\\
178	0.00433757401014956\\
179	0.00433763531814403\\
180	0.00433769776800725\\
181	0.00433776138109724\\
182	0.00433782617917334\\
183	0.00433789218440382\\
184	0.00433795941937356\\
185	0.00433802790709201\\
186	0.00433809767100114\\
187	0.00433816873498355\\
188	0.00433824112337095\\
189	0.00433831486095252\\
190	0.00433838997298366\\
191	0.00433846648519474\\
192	0.00433854442380009\\
193	0.00433862381550717\\
194	0.00433870468752586\\
195	0.00433878706757803\\
196	0.00433887098390719\\
197	0.00433895646528827\\
198	0.00433904354103794\\
199	0.00433913224102455\\
200	0.0043392225956787\\
201	0.00433931463600408\\
202	0.00433940839358797\\
203	0.0043395039006125\\
204	0.00433960118986595\\
205	0.00433970029475408\\
206	0.00433980124931195\\
207	0.00433990408821587\\
208	0.00434000884679544\\
209	0.004340115561046\\
210	0.00434022426764125\\
211	0.00434033500394618\\
212	0.00434044780802998\\
213	0.0043405627186797\\
214	0.00434067977541363\\
215	0.00434079901849514\\
216	0.00434092048894711\\
217	0.00434104422856615\\
218	0.00434117027993722\\
219	0.00434129868644876\\
220	0.00434142949230795\\
221	0.00434156274255614\\
222	0.00434169848308484\\
223	0.00434183676065168\\
224	0.00434197762289716\\
225	0.00434212111836126\\
226	0.00434226729650053\\
227	0.0043424162077056\\
228	0.00434256790331901\\
229	0.00434272243565309\\
230	0.00434287985800876\\
231	0.00434304022469402\\
232	0.00434320359104341\\
233	0.00434337001343736\\
234	0.00434353954932225\\
235	0.00434371225723068\\
236	0.00434388819680213\\
237	0.00434406742880416\\
238	0.00434425001515388\\
239	0.00434443601893982\\
240	0.00434462550444436\\
241	0.00434481853716655\\
242	0.00434501518384526\\
243	0.00434521551248277\\
244	0.00434541959236909\\
245	0.00434562749410651\\
246	0.00434583928963452\\
247	0.00434605505225551\\
248	0.00434627485666074\\
249	0.00434649877895702\\
250	0.00434672689669355\\
251	0.00434695928888968\\
252	0.0043471960360629\\
253	0.00434743722025764\\
254	0.00434768292507426\\
255	0.00434793323569891\\
256	0.004348188238934\\
257	0.00434844802322883\\
258	0.00434871267871118\\
259	0.00434898229721944\\
260	0.00434925697233537\\
261	0.00434953679941713\\
262	0.00434982187563352\\
263	0.00435011229999835\\
264	0.00435040817340566\\
265	0.00435070959866558\\
266	0.00435101668054085\\
267	0.00435132952578396\\
268	0.00435164824317491\\
269	0.00435197294355986\\
270	0.0043523037398901\\
271	0.00435264074726211\\
272	0.00435298408295787\\
273	0.00435333386648616\\
274	0.00435369021962445\\
275	0.00435405326646142\\
276	0.00435442313344009\\
277	0.00435479994940184\\
278	0.00435518384563063\\
279	0.00435557495589837\\
280	0.00435597341651037\\
281	0.00435637936635177\\
282	0.00435679294693444\\
283	0.00435721430244384\\
284	0.00435764357978722\\
285	0.00435808092864142\\
286	0.00435852650150141\\
287	0.00435898045372909\\
288	0.004359442943602\\
289	0.00435991413236213\\
290	0.00436039418426491\\
291	0.00436088326662764\\
292	0.0043613815498779\\
293	0.00436188920760098\\
294	0.00436240641658698\\
295	0.00436293335687661\\
296	0.0043634702118058\\
297	0.00436401716804851\\
298	0.00436457441565798\\
299	0.00436514214810526\\
300	0.00436572056231491\\
301	0.00436630985869759\\
302	0.00436691024117822\\
303	0.00436752191721976\\
304	0.00436814509784179\\
305	0.00436877999763203\\
306	0.00436942683475128\\
307	0.00437008583092947\\
308	0.00437075721145214\\
309	0.00437144120513579\\
310	0.00437213804429049\\
311	0.00437284796466764\\
312	0.0043735712053918\\
313	0.00437430800887281\\
314	0.00437505862069719\\
315	0.00437582328949509\\
316	0.00437660226677961\\
317	0.00437739580675559\\
318	0.00437820416609351\\
319	0.00437902760366431\\
320	0.00437986638023082\\
321	0.00438072075809017\\
322	0.0043815910006622\\
323	0.00438247737201782\\
324	0.00438338013634087\\
325	0.00438429955731751\\
326	0.00438523589744583\\
327	0.00438618941725928\\
328	0.00438716037445751\\
329	0.0043881490229384\\
330	0.0043891556117265\\
331	0.00439018038379473\\
332	0.00439122357477812\\
333	0.00439228541158204\\
334	0.00439336611089349\\
335	0.00439446587760887\\
336	0.00439558490320252\\
337	0.00439672336407297\\
338	0.00439788141991926\\
339	0.00439905921222226\\
340	0.00440025686292988\\
341	0.0044014744734705\\
342	0.00440271212422077\\
343	0.00440396987448398\\
344	0.00440524776275998\\
345	0.00440654580531069\\
346	0.00440786398489007\\
347	0.00440920402385365\\
348	0.00441057060935592\\
349	0.00441196426536201\\
350	0.00441338552577781\\
351	0.00441483493460724\\
352	0.00441631304610621\\
353	0.00441782042492295\\
354	0.0044193576462028\\
355	0.00442092529565036\\
356	0.00442252396978652\\
357	0.00442415427627943\\
358	0.00442581683414257\\
359	0.00442751227392775\\
360	0.00442924123792482\\
361	0.00443100438036846\\
362	0.00443280236765423\\
363	0.00443463587856507\\
364	0.00443650560451036\\
365	0.00443841224977974\\
366	0.00444035653181446\\
367	0.00444233918149978\\
368	0.00444436094348195\\
369	0.00444642257651406\\
370	0.00444852485383648\\
371	0.00445066856359745\\
372	0.00445285450932144\\
373	0.00445508351043292\\
374	0.00445735640284518\\
375	0.00445967403962522\\
376	0.00446203729174694\\
377	0.00446444704894753\\
378	0.00446690422070181\\
379	0.00446940973733432\\
380	0.00447196455128907\\
381	0.00447456963857968\\
382	0.00447722600044551\\
383	0.00447993466524188\\
384	0.0044826966905934\\
385	0.00448551316584321\\
386	0.0044883852148302\\
387	0.00449131399902813\\
388	0.00449430072107765\\
389	0.00449734662873975\\
390	0.00450045301929358\\
391	0.00450362124439519\\
392	0.00450685271541414\\
393	0.00451014890930527\\
394	0.00451351137525375\\
395	0.00451694174302952\\
396	0.0045204417364464\\
397	0.00452401320375389\\
398	0.00452765820524174\\
399	0.0045313789886601\\
400	0.00453517753936264\\
401	0.00453905557873385\\
402	0.0045430148741424\\
403	0.00454705724075805\\
404	0.00455118454346668\\
405	0.00455539869888527\\
406	0.00455970167748078\\
407	0.00456409550579306\\
408	0.00456858226876349\\
409	0.00457316411217185\\
410	0.00457784324519672\\
411	0.00458262194310348\\
412	0.00458750255001831\\
413	0.00459248748176176\\
414	0.00459757922875969\\
415	0.00460278035902185\\
416	0.00460809352116916\\
417	0.00461352144748655\\
418	0.00461906695697717\\
419	0.00462473295838905\\
420	0.00463052245318885\\
421	0.00463643853847783\\
422	0.0046424844098883\\
423	0.00464866336458066\\
424	0.00465497880455615\\
425	0.00466143424044806\\
426	0.00466803329447707\\
427	0.00467477969251494\\
428	0.00468167725361806\\
429	0.0046887299093657\\
430	0.00469594173033701\\
431	0.00470331693310338\\
432	0.00471085988765404\\
433	0.00471857512528687\\
434	0.00472646734699674\\
435	0.00473454143239654\\
436	0.00474280244921074\\
437	0.00475125566338333\\
438	0.00475990654984844\\
439	0.00476876080401437\\
440	0.00477782435402021\\
441	0.00478710337383024\\
442	0.00479660429724071\\
443	0.0048063338328793\\
444	0.00481629898027238\\
445	0.00482650704700604\\
446	0.00483696566688477\\
447	0.0048476828189708\\
448	0.0048586668488317\\
449	0.0048699264922354\\
450	0.00488147089918492\\
451	0.00489330965865071\\
452	0.00490545282471894\\
453	0.00491791094418666\\
454	0.00493069508562214\\
455	0.0049438168698928\\
456	0.00495728850214065\\
457	0.00497112280515963\\
458	0.00498533325409378\\
459	0.0049999340123333\\
460	0.00501493996843038\\
461	0.0050303667737904\\
462	0.00504623088081715\\
463	0.00506254958112901\\
464	0.00507934104342271\\
465	0.00509662435029209\\
466	0.0051144195332361\\
467	0.00513274760527605\\
468	0.00515163059078765\\
469	0.00517109155320189\\
470	0.00519115462429839\\
471	0.00521184504677305\\
472	0.00523318926021573\\
473	0.00525521509555144\\
474	0.00527795217275313\\
475	0.00530143265222818\\
476	0.00532569914663003\\
477	0.00535080864374956\\
478	0.00537679809147539\\
479	0.00540370560503476\\
480	0.00543157784115154\\
481	0.00546046220442261\\
482	0.00549040667370869\\
483	0.00552145904822377\\
484	0.00555366580937486\\
485	0.00558704762300578\\
486	0.00562138769391787\\
487	0.0056566986408806\\
488	0.00569298744985147\\
489	0.00573025396980975\\
490	0.00576848877383418\\
491	0.00580766931668448\\
492	0.00584796783373438\\
493	0.00588943448996365\\
494	0.0059320294334912\\
495	0.00597568959241816\\
496	0.00602032355118579\\
497	0.00606580644661143\\
498	0.00611196917249998\\
499	0.0061585824520033\\
500	0.00620541096103591\\
501	0.0062520924216836\\
502	0.00629802034333257\\
503	0.00634147450349103\\
504	0.00637971149189926\\
505	0.00641262592420366\\
506	0.00644048029382127\\
507	0.00646574725732371\\
508	0.00649010650081638\\
509	0.00651399622170918\\
510	0.00653789763179205\\
511	0.0065620801181893\\
512	0.00658665536465146\\
513	0.00661169510902583\\
514	0.006637239803336\\
515	0.00666331249094905\\
516	0.00668993163021992\\
517	0.00671711228438945\\
518	0.00674486803670351\\
519	0.00677321216013866\\
520	0.0068021579289145\\
521	0.00683171890503599\\
522	0.00686190912800818\\
523	0.00689274324833507\\
524	0.00692423666951791\\
525	0.00695640570141792\\
526	0.00698926773593825\\
527	0.00702284145500003\\
528	0.0070571470756616\\
529	0.00709220664141625\\
530	0.00712804440069717\\
531	0.00716472854050527\\
532	0.00720234880295647\\
533	0.00723985893619002\\
534	0.00727623466988971\\
535	0.00731151311131741\\
536	0.00734719353934899\\
537	0.0073833556106261\\
538	0.00742008300313997\\
539	0.00745740535755315\\
540	0.00749532669484875\\
541	0.00753384645106589\\
542	0.00757296262261381\\
543	0.00761267172446651\\
544	0.00765296869609351\\
545	0.00769384670138497\\
546	0.00773529689100766\\
547	0.00777730812572076\\
548	0.00781986665437991\\
549	0.00786295573820756\\
550	0.00790655521515773\\
551	0.00795064102098\\
552	0.00799518479699153\\
553	0.0080401859272667\\
554	0.00808464783813621\\
555	0.00812773859917388\\
556	0.00817111836314038\\
557	0.00821486580995575\\
558	0.00825897791658296\\
559	0.00830343402391313\\
560	0.00834821243243883\\
561	0.00839329070202066\\
562	0.00843864597211353\\
563	0.00848425526709758\\
564	0.00853009920226463\\
565	0.00857589049842109\\
566	0.00862164036385014\\
567	0.0086676593810865\\
568	0.00871393573141297\\
569	0.00876044307752625\\
570	0.00880715274930517\\
571	0.00885403364441827\\
572	0.00890105214171656\\
573	0.00894817203305216\\
574	0.00899535448063906\\
575	0.00904255800890195\\
576	0.00908973854201727\\
577	0.00913684950113329\\
578	0.00918384197868252\\
579	0.00923066501141119\\
580	0.00927726597891579\\
581	0.00932359116078638\\
582	0.00936958649310202\\
583	0.00941519857412798\\
584	0.00946037597948012\\
585	0.00950507095777016\\
586	0.00954924158550055\\
587	0.00959285445497631\\
588	0.00963588792409171\\
589	0.00967833579817709\\
590	0.00972021084271539\\
591	0.00976138933978722\\
592	0.00980171928775089\\
593	0.00984102135019415\\
594	0.00987905595166784\\
595	0.00991543381058343\\
596	0.00994937493726701\\
597	0.00997906286423442\\
598	0.0099999191923403\\
599	0\\
600	0\\
};
\addplot [color=black,solid,forget plot]
  table[row sep=crcr]{%
1	0.00434222595919898\\
2	0.00434222821860981\\
3	0.00434223051966135\\
4	0.00434223286312353\\
5	0.00434223524978059\\
6	0.00434223768043138\\
7	0.00434224015588943\\
8	0.00434224267698356\\
9	0.00434224524455786\\
10	0.00434224785947209\\
11	0.00434225052260194\\
12	0.00434225323483943\\
13	0.00434225599709309\\
14	0.00434225881028843\\
15	0.00434226167536802\\
16	0.00434226459329199\\
17	0.00434226756503826\\
18	0.00434227059160297\\
19	0.00434227367400076\\
20	0.00434227681326514\\
21	0.00434228001044872\\
22	0.00434228326662375\\
23	0.00434228658288232\\
24	0.00434228996033694\\
25	0.00434229340012076\\
26	0.00434229690338792\\
27	0.00434230047131408\\
28	0.0043423041050967\\
29	0.00434230780595548\\
30	0.0043423115751329\\
31	0.00434231541389449\\
32	0.0043423193235292\\
33	0.00434232330535016\\
34	0.00434232736069477\\
35	0.00434233149092533\\
36	0.00434233569742945\\
37	0.00434233998162055\\
38	0.00434234434493852\\
39	0.00434234878884976\\
40	0.00434235331484817\\
41	0.00434235792445522\\
42	0.00434236261922094\\
43	0.00434236740072407\\
44	0.00434237227057269\\
45	0.00434237723040493\\
46	0.00434238228188933\\
47	0.00434238742672547\\
48	0.00434239266664463\\
49	0.00434239800341021\\
50	0.00434240343881856\\
51	0.00434240897469947\\
52	0.00434241461291672\\
53	0.00434242035536886\\
54	0.00434242620398974\\
55	0.00434243216074926\\
56	0.00434243822765407\\
57	0.00434244440674809\\
58	0.00434245070011332\\
59	0.00434245710987066\\
60	0.00434246363818046\\
61	0.00434247028724327\\
62	0.00434247705930069\\
63	0.00434248395663615\\
64	0.00434249098157545\\
65	0.00434249813648803\\
66	0.00434250542378723\\
67	0.00434251284593148\\
68	0.00434252040542502\\
69	0.00434252810481881\\
70	0.00434253594671141\\
71	0.00434254393374964\\
72	0.00434255206862989\\
73	0.00434256035409873\\
74	0.00434256879295399\\
75	0.00434257738804563\\
76	0.00434258614227683\\
77	0.0043425950586049\\
78	0.00434260414004231\\
79	0.00434261338965775\\
80	0.00434262281057715\\
81	0.00434263240598475\\
82	0.00434264217912425\\
83	0.00434265213329983\\
84	0.00434266227187737\\
85	0.00434267259828563\\
86	0.00434268311601725\\
87	0.00434269382863009\\
88	0.00434270473974859\\
89	0.00434271585306465\\
90	0.00434272717233917\\
91	0.00434273870140327\\
92	0.00434275044415967\\
93	0.00434276240458394\\
94	0.00434277458672585\\
95	0.00434278699471084\\
96	0.00434279963274138\\
97	0.00434281250509854\\
98	0.00434282561614324\\
99	0.00434283897031795\\
100	0.00434285257214809\\
101	0.00434286642624372\\
102	0.00434288053730087\\
103	0.00434289491010356\\
104	0.00434290954952511\\
105	0.00434292446052985\\
106	0.00434293964817495\\
107	0.00434295511761211\\
108	0.00434297087408942\\
109	0.00434298692295295\\
110	0.00434300326964874\\
111	0.00434301991972472\\
112	0.00434303687883253\\
113	0.00434305415272947\\
114	0.00434307174728058\\
115	0.00434308966846045\\
116	0.00434310792235553\\
117	0.00434312651516604\\
118	0.00434314545320822\\
119	0.00434316474291647\\
120	0.00434318439084554\\
121	0.00434320440367275\\
122	0.00434322478820046\\
123	0.00434324555135824\\
124	0.00434326670020539\\
125	0.00434328824193327\\
126	0.00434331018386777\\
127	0.00434333253347188\\
128	0.00434335529834836\\
129	0.00434337848624228\\
130	0.0043434021050436\\
131	0.00434342616279007\\
132	0.00434345066766977\\
133	0.00434347562802419\\
134	0.00434350105235097\\
135	0.0043435269493068\\
136	0.00434355332771052\\
137	0.00434358019654606\\
138	0.0043436075649656\\
139	0.00434363544229271\\
140	0.0043436638380256\\
141	0.00434369276184034\\
142	0.00434372222359422\\
143	0.00434375223332918\\
144	0.00434378280127535\\
145	0.00434381393785427\\
146	0.00434384565368292\\
147	0.00434387795957702\\
148	0.00434391086655505\\
149	0.00434394438584184\\
150	0.00434397852887255\\
151	0.00434401330729659\\
152	0.00434404873298164\\
153	0.00434408481801783\\
154	0.00434412157472182\\
155	0.00434415901564095\\
156	0.00434419715355784\\
157	0.00434423600149462\\
158	0.00434427557271744\\
159	0.00434431588074112\\
160	0.00434435693933373\\
161	0.00434439876252149\\
162	0.00434444136459342\\
163	0.00434448476010633\\
164	0.00434452896389\\
165	0.00434457399105212\\
166	0.00434461985698352\\
167	0.00434466657736367\\
168	0.00434471416816581\\
169	0.00434476264566271\\
170	0.00434481202643204\\
171	0.0043448623273624\\
172	0.00434491356565887\\
173	0.00434496575884917\\
174	0.00434501892478946\\
175	0.00434507308167075\\
176	0.00434512824802498\\
177	0.00434518444273155\\
178	0.00434524168502374\\
179	0.00434529999449536\\
180	0.00434535939110761\\
181	0.00434541989519574\\
182	0.00434548152747634\\
183	0.00434554430905421\\
184	0.00434560826142982\\
185	0.00434567340650657\\
186	0.00434573976659853\\
187	0.00434580736443798\\
188	0.00434587622318318\\
189	0.00434594636642652\\
190	0.00434601781820248\\
191	0.00434609060299602\\
192	0.00434616474575091\\
193	0.00434624027187832\\
194	0.00434631720726564\\
195	0.00434639557828524\\
196	0.00434647541180373\\
197	0.00434655673519102\\
198	0.00434663957632959\\
199	0.00434672396362446\\
200	0.00434680992601267\\
201	0.00434689749297314\\
202	0.00434698669453694\\
203	0.00434707756129756\\
204	0.00434717012442129\\
205	0.00434726441565799\\
206	0.004347360467352\\
207	0.00434745831245296\\
208	0.00434755798452739\\
209	0.00434765951777001\\
210	0.00434776294701537\\
211	0.00434786830774987\\
212	0.00434797563612374\\
213	0.0043480849689634\\
214	0.00434819634378422\\
215	0.00434830979880302\\
216	0.00434842537295115\\
217	0.00434854310588775\\
218	0.00434866303801332\\
219	0.00434878521048335\\
220	0.0043489096652223\\
221	0.00434903644493781\\
222	0.00434916559313522\\
223	0.00434929715413229\\
224	0.00434943117307416\\
225	0.00434956769594852\\
226	0.00434970676960141\\
227	0.00434984844175275\\
228	0.0043499927610126\\
229	0.00435013977689738\\
230	0.0043502895398467\\
231	0.00435044210124015\\
232	0.00435059751341458\\
233	0.00435075582968171\\
234	0.00435091710434584\\
235	0.00435108139272219\\
236	0.00435124875115522\\
237	0.00435141923703743\\
238	0.00435159290882859\\
239	0.00435176982607499\\
240	0.00435195004942941\\
241	0.00435213364067103\\
242	0.00435232066272596\\
243	0.00435251117968812\\
244	0.0043527052568402\\
245	0.00435290296067521\\
246	0.00435310435891838\\
247	0.00435330952054942\\
248	0.00435351851582486\\
249	0.00435373141630114\\
250	0.00435394829485812\\
251	0.00435416922572236\\
252	0.0043543942844916\\
253	0.0043546235481589\\
254	0.00435485709513777\\
255	0.00435509500528721\\
256	0.00435533735993752\\
257	0.0043555842419162\\
258	0.00435583573557455\\
259	0.0043560919268145\\
260	0.00435635290311574\\
261	0.00435661875356367\\
262	0.00435688956887727\\
263	0.00435716544143785\\
264	0.0043574464653178\\
265	0.00435773273631016\\
266	0.00435802435195835\\
267	0.00435832141158653\\
268	0.00435862401633014\\
269	0.00435893226916712\\
270	0.00435924627494958\\
271	0.00435956614043557\\
272	0.00435989197432189\\
273	0.00436022388727688\\
274	0.00436056199197369\\
275	0.00436090640312437\\
276	0.00436125723751402\\
277	0.00436161461403568\\
278	0.00436197865372554\\
279	0.00436234947979873\\
280	0.00436272721768567\\
281	0.00436311199506868\\
282	0.00436350394191926\\
283	0.00436390319053608\\
284	0.00436430987558299\\
285	0.00436472413412836\\
286	0.00436514610568424\\
287	0.00436557593224635\\
288	0.00436601375833513\\
289	0.00436645973103691\\
290	0.00436691400004596\\
291	0.00436737671770755\\
292	0.00436784803906134\\
293	0.00436832812188603\\
294	0.0043688171267447\\
295	0.00436931521703131\\
296	0.00436982255901841\\
297	0.00437033932190591\\
298	0.00437086567787115\\
299	0.00437140180212078\\
300	0.00437194787294426\\
301	0.00437250407176894\\
302	0.00437307058321792\\
303	0.00437364759516981\\
304	0.00437423529882131\\
305	0.00437483388875333\\
306	0.0043754435630001\\
307	0.00437606452312268\\
308	0.00437669697428681\\
309	0.00437734112534621\\
310	0.00437799718893176\\
311	0.00437866538154727\\
312	0.0043793459236733\\
313	0.00438003903987961\\
314	0.00438074495894785\\
315	0.00438146391400548\\
316	0.00438219614267317\\
317	0.0043829418872268\\
318	0.00438370139477653\\
319	0.00438447491746537\\
320	0.00438526271268933\\
321	0.00438606504334291\\
322	0.00438688217809268\\
323	0.00438771439168233\\
324	0.0043885619652747\\
325	0.00438942518683345\\
326	0.00439030435155112\\
327	0.004391199762328\\
328	0.00439211173030835\\
329	0.00439304057548008\\
330	0.00439398662734469\\
331	0.00439495022566533\\
332	0.00439593172129862\\
333	0.00439693147711934\\
334	0.00439794986904292\\
335	0.00439898728715272\\
336	0.00440004413693626\\
337	0.00440112084063253\\
338	0.00440221783869206\\
339	0.00440333559135105\\
340	0.0044044745803329\\
341	0.00440563531073312\\
342	0.00440681831329135\\
343	0.00440802414769152\\
344	0.00440925340879721\\
345	0.00441050674127023\\
346	0.00441178487748705\\
347	0.00441308861159611\\
348	0.00441441859668612\\
349	0.00441577537239013\\
350	0.00441715949007091\\
351	0.00441857151312482\\
352	0.00442001201729709\\
353	0.00442148159100886\\
354	0.00442298083569901\\
355	0.00442451036618554\\
356	0.00442607081104137\\
357	0.00442766281297806\\
358	0.00442928702924466\\
359	0.00443094413204339\\
360	0.00443263480896246\\
361	0.00443435976342743\\
362	0.00443611971517158\\
363	0.00443791540072674\\
364	0.00443974757393486\\
365	0.00444161700648187\\
366	0.00444352448845459\\
367	0.00444547082892134\\
368	0.0044474568565372\\
369	0.0044494834201751\\
370	0.00445155138958286\\
371	0.00445366165606704\\
372	0.00445581513320393\\
373	0.00445801275757793\\
374	0.00446025548954678\\
375	0.00446254431403404\\
376	0.00446488024134691\\
377	0.00446726430801856\\
378	0.00446969757767323\\
379	0.00447218114191075\\
380	0.00447471612120721\\
381	0.00447730366582724\\
382	0.00447994495674243\\
383	0.0044826412065479\\
384	0.00448539366036974\\
385	0.00448820359675243\\
386	0.0044910723285143\\
387	0.00449400120355786\\
388	0.00449699160561887\\
389	0.00450004495493866\\
390	0.00450316270884218\\
391	0.00450634636220972\\
392	0.00450959744784129\\
393	0.00451291753673766\\
394	0.00451630823837839\\
395	0.00451977120114798\\
396	0.00452330811301538\\
397	0.00452692070170608\\
398	0.00453061072922442\\
399	0.00453437998069945\\
400	0.00453823027231605\\
401	0.00454216346841403\\
402	0.00454618148311681\\
403	0.00455028628203201\\
404	0.00455447988402687\\
405	0.00455876436308343\\
406	0.00456314185023713\\
407	0.004567614535605\\
408	0.00457218467050823\\
409	0.00457685456969654\\
410	0.00458162661367982\\
411	0.00458650325117443\\
412	0.0045914870016723\\
413	0.00459658045814428\\
414	0.00460178628988913\\
415	0.0046071072455413\\
416	0.00461254615625144\\
417	0.00461810593905845\\
418	0.00462378960047169\\
419	0.0046296002402883\\
420	0.00463554105567392\\
421	0.00464161534554071\\
422	0.0046478265152618\\
423	0.00465417808175562\\
424	0.00466067367894573\\
425	0.00466731706352769\\
426	0.00467411212094043\\
427	0.00468106287215399\\
428	0.00468817348186214\\
429	0.00469544826687676\\
430	0.00470289170423734\\
431	0.00471050843987783\\
432	0.00471830329789414\\
433	0.00472628129045722\\
434	0.00473444762842076\\
435	0.00474280773267571\\
436	0.00475136724630788\\
437	0.00476013204761824\\
438	0.00476910826406953\\
439	0.00477830228722649\\
440	0.0047877207887615\\
441	0.00479737073760044\\
442	0.00480725941828665\\
443	0.00481739445064306\\
444	0.00482778381081094\\
445	0.00483843585374646\\
446	0.00484935933727037\\
447	0.00486056344780547\\
448	0.00487205782786032\\
449	0.00488385260528553\\
450	0.00489595842442378\\
451	0.00490838647926189\\
452	0.00492114854865773\\
453	0.00493425703369915\\
454	0.00494772499723305\\
455	0.00496156620557198\\
456	0.00497579517234968\\
457	0.00499042720444371\\
458	0.00500547844981828\\
459	0.00502096594705182\\
460	0.005036907676203\\
461	0.00505332261052375\\
462	0.00507023076834851\\
463	0.00508765326425881\\
464	0.00510561235833505\\
465	0.00512413150199129\\
466	0.00514323537859757\\
467	0.0051629499370455\\
468	0.00518330241737538\\
469	0.00520432137194165\\
470	0.00522603670084335\\
471	0.0052484797701117\\
472	0.00527168384025851\\
473	0.00529568554045985\\
474	0.00532052975210367\\
475	0.00534628267470183\\
476	0.00537298517324241\\
477	0.00540067823671829\\
478	0.005429401877641\\
479	0.00545919466390937\\
480	0.00548981733005028\\
481	0.00552129955298564\\
482	0.00555367068432858\\
483	0.00558695979453451\\
484	0.00562119644582296\\
485	0.00565643318442142\\
486	0.00569293098925691\\
487	0.00573071750642869\\
488	0.00576981032505155\\
489	0.00581021341315327\\
490	0.00585191298560103\\
491	0.00589487354705255\\
492	0.00593902316472585\\
493	0.00598423743188394\\
494	0.00603033170296704\\
495	0.00607704681967084\\
496	0.00612402716256964\\
497	0.0061707902296417\\
498	0.00621667981092483\\
499	0.00626077931899491\\
500	0.00629986903605491\\
501	0.00633321627267928\\
502	0.00636114201732931\\
503	0.00638521509377619\\
504	0.00640821620985371\\
505	0.00643059378635421\\
506	0.00645283186455397\\
507	0.00647526889657669\\
508	0.00649805406722564\\
509	0.0065212637067607\\
510	0.00654494242954996\\
511	0.00656911485166403\\
512	0.00659379925237844\\
513	0.00661901035700388\\
514	0.00664476133373063\\
515	0.0066710650495984\\
516	0.00669793437851615\\
517	0.0067253824797126\\
518	0.00675342297778298\\
519	0.00678207008594943\\
520	0.00681133873692882\\
521	0.00684124472623264\\
522	0.00687180488058706\\
523	0.00690303726415001\\
524	0.00693496143280232\\
525	0.00696759874906755\\
526	0.00700097277136572\\
527	0.00703510972972449\\
528	0.00707003908897031\\
529	0.00710579415587524\\
530	0.00714241253002996\\
531	0.00717823387096147\\
532	0.00721273677147728\\
533	0.00724682109888564\\
534	0.00728132396461994\\
535	0.00731634289462122\\
536	0.00735194707806507\\
537	0.00738814406559645\\
538	0.00742493702546428\\
539	0.00746232678336369\\
540	0.00750031295800742\\
541	0.00753889396747831\\
542	0.00757806687614832\\
543	0.00761782721299181\\
544	0.00765816875730519\\
545	0.00769908328917873\\
546	0.00774056029745439\\
547	0.00778258663495929\\
548	0.00782514610529904\\
549	0.00786821895268939\\
550	0.00791178119219577\\
551	0.00795580361922691\\
552	0.00800025003764972\\
553	0.00804363618546738\\
554	0.00808620366337529\\
555	0.00812910677813416\\
556	0.00817240401823542\\
557	0.00821607813701697\\
558	0.00826010969053816\\
559	0.00830447822831884\\
560	0.0083491625665264\\
561	0.0083941411104164\\
562	0.00843939212066833\\
563	0.00848489356497313\\
564	0.00853045416626427\\
565	0.00857590279196591\\
566	0.0086216403638569\\
567	0.00866765938108735\\
568	0.00871393573141338\\
569	0.00876044307752645\\
570	0.00880715274930527\\
571	0.00885403364441831\\
572	0.00890105214171657\\
573	0.00894817203305217\\
574	0.00899535448063906\\
575	0.00904255800890197\\
576	0.00908973854201728\\
577	0.00913684950113329\\
578	0.00918384197868253\\
579	0.00923066501141121\\
580	0.00927726597891581\\
581	0.00932359116078639\\
582	0.00936958649310202\\
583	0.00941519857412798\\
584	0.00946037597948012\\
585	0.00950507095777015\\
586	0.00954924158550055\\
587	0.00959285445497631\\
588	0.00963588792409171\\
589	0.00967833579817709\\
590	0.00972021084271539\\
591	0.00976138933978722\\
592	0.00980171928775089\\
593	0.00984102135019415\\
594	0.00987905595166784\\
595	0.00991543381058343\\
596	0.00994937493726701\\
597	0.00997906286423442\\
598	0.0099999191923403\\
599	0\\
600	0\\
};
\end{axis}
\end{tikzpicture}% 
  \caption{Discrete Time}
\end{subfigure}\\
\vspace{1cm}
\begin{subfigure}{.45\linewidth}
  \centering
  \setlength\figureheight{\linewidth} 
  \setlength\figurewidth{\linewidth}
  \tikzsetnextfilename{testdp_cts_nFPC_z8}
  % This file was created by matlab2tikz.
%
%The latest updates can be retrieved from
%  http://www.mathworks.com/matlabcentral/fileexchange/22022-matlab2tikz-matlab2tikz
%where you can also make suggestions and rate matlab2tikz.
%
\definecolor{mycolor1}{rgb}{1.00000,0.00000,1.00000}%
%
\begin{tikzpicture}

\begin{axis}[%
width=4.564in,
height=3.803in,
at={(1.067in,0.513in)},
scale only axis,
every outer x axis line/.append style={black},
every x tick label/.append style={font=\color{black}},
xmin=0,
xmax=100,
xlabel={Time},
every outer y axis line/.append style={black},
every y tick label/.append style={font=\color{black}},
ymin=0,
ymax=0.01,
ylabel={Depth $\delta$},
axis background/.style={fill=white},
title={Z=8},
axis x line*=bottom,
axis y line*=left,
legend style={legend cell align=left,align=left,draw=black}
]
\addplot [color=green,dashed,forget plot]
  table[row sep=crcr]{%
0.01	0.00879904401680774\\
0.02	0.00879894761016513\\
0.03	0.0087988511565914\\
0.04	0.00879875465606243\\
0.05	0.00879865810855404\\
0.06	0.00879856151404209\\
0.07	0.00879846487250238\\
0.08	0.00879836818391074\\
0.09	0.00879827144824294\\
0.1	0.00879817466547478\\
0.11	0.00879807783558202\\
0.12	0.00879798095854041\\
0.13	0.0087978840343257\\
0.14	0.00879778706291362\\
0.15	0.00879769004427989\\
0.16	0.0087975929784002\\
0.17	0.00879749586525025\\
0.18	0.00879739870480571\\
0.19	0.00879730149704226\\
0.2	0.00879720424193552\\
0.21	0.00879710693946116\\
0.22	0.00879700958959477\\
0.23	0.00879691219231199\\
0.24	0.00879681474758841\\
0.25	0.00879671725539961\\
0.26	0.00879661971572116\\
0.27	0.00879652212852861\\
0.28	0.00879642449379753\\
0.29	0.00879632681150343\\
0.3	0.00879622908162183\\
0.31	0.00879613130412824\\
0.32	0.00879603347899816\\
0.33	0.00879593560620705\\
0.34	0.00879583768573038\\
0.35	0.00879573971754362\\
0.36	0.00879564170162219\\
0.37	0.00879554363794152\\
0.38	0.00879544552647702\\
0.39	0.0087953473672041\\
0.4	0.00879524916009813\\
0.41	0.00879515090513448\\
0.42	0.00879505260228853\\
0.43	0.00879495425153561\\
0.44	0.00879485585285105\\
0.45	0.00879475740621017\\
0.46	0.00879465891158828\\
0.47	0.00879456036896068\\
0.48	0.00879446177830263\\
0.49	0.0087943631395894\\
0.5	0.00879426445279625\\
0.51	0.00879416571789841\\
0.52	0.00879406693487111\\
0.53	0.00879396810368956\\
0.54	0.00879386922432896\\
0.55	0.00879377029676448\\
0.56	0.00879367132097131\\
0.57	0.0087935722969246\\
0.58	0.00879347322459949\\
0.59	0.00879337410397112\\
0.6	0.0087932749350146\\
0.61	0.00879317571770504\\
0.62	0.00879307645201752\\
0.63	0.00879297713792712\\
0.64	0.00879287777540891\\
0.65	0.00879277836443793\\
0.66	0.00879267890498922\\
0.67	0.0087925793970378\\
0.68	0.00879247984055869\\
0.69	0.00879238023552687\\
0.7	0.00879228058191732\\
0.71	0.00879218087970502\\
0.72	0.00879208112886493\\
0.73	0.00879198132937197\\
0.74	0.00879188148120108\\
0.75	0.00879178158432718\\
0.76	0.00879168163872515\\
0.77	0.00879158164436989\\
0.78	0.00879148160123627\\
0.79	0.00879138150929915\\
0.8	0.00879128136853338\\
0.81	0.00879118117891378\\
0.82	0.00879108094041517\\
0.83	0.00879098065301236\\
0.84	0.00879088031668014\\
0.85	0.00879077993139329\\
0.86	0.00879067949712657\\
0.87	0.00879057901385472\\
0.88	0.00879047848155248\\
0.89	0.00879037790019458\\
0.9	0.00879027726975573\\
0.91	0.00879017659021061\\
0.92	0.00879007586153391\\
0.93	0.00878997508370029\\
0.94	0.00878987425668441\\
0.95	0.00878977338046091\\
0.96	0.00878967245500441\\
0.97	0.00878957148028952\\
0.98	0.00878947045629085\\
0.99	0.00878936938298296\\
1	0.00878926826034045\\
1.01	0.00878916708833785\\
1.02	0.00878906586694972\\
1.03	0.00878896459615058\\
1.04	0.00878886327591495\\
1.05	0.00878876190621732\\
1.06	0.00878866048703218\\
1.07	0.00878855901833402\\
1.08	0.00878845750009728\\
1.09	0.00878835593229641\\
1.1	0.00878825431490583\\
1.11	0.00878815264789998\\
1.12	0.00878805093125325\\
1.13	0.00878794916494002\\
1.14	0.00878784734893469\\
1.15	0.0087877454832116\\
1.16	0.0087876435677451\\
1.17	0.00878754160250953\\
1.18	0.00878743958747921\\
1.19	0.00878733752262843\\
1.2	0.0087872354079315\\
1.21	0.00878713324336269\\
1.22	0.00878703102889626\\
1.23	0.00878692876450645\\
1.24	0.00878682645016752\\
1.25	0.00878672408585367\\
1.26	0.00878662167153911\\
1.27	0.00878651920719804\\
1.28	0.00878641669280464\\
1.29	0.00878631412833306\\
1.3	0.00878621151375746\\
1.31	0.00878610884905198\\
1.32	0.00878600613419073\\
1.33	0.00878590336914782\\
1.34	0.00878580055389736\\
1.35	0.00878569768841341\\
1.36	0.00878559477267004\\
1.37	0.00878549180664131\\
1.38	0.00878538879030125\\
1.39	0.00878528572362388\\
1.4	0.00878518260658322\\
1.41	0.00878507943915325\\
1.42	0.00878497622130796\\
1.43	0.00878487295302131\\
1.44	0.00878476963426725\\
1.45	0.00878466626501972\\
1.46	0.00878456284525264\\
1.47	0.00878445937493993\\
1.48	0.00878435585405547\\
1.49	0.00878425228257315\\
1.5	0.00878414866046683\\
1.51	0.00878404498771036\\
1.52	0.00878394126427758\\
1.53	0.00878383749014231\\
1.54	0.00878373366527837\\
1.55	0.00878362978965954\\
1.56	0.0087835258632596\\
1.57	0.00878342188605233\\
1.58	0.00878331785801147\\
1.59	0.00878321377911075\\
1.6	0.0087831096493239\\
1.61	0.00878300546862462\\
1.62	0.00878290123698662\\
1.63	0.00878279695438357\\
1.64	0.00878269262078912\\
1.65	0.00878258823617694\\
1.66	0.00878248380052066\\
1.67	0.0087823793137939\\
1.68	0.00878227477597025\\
1.69	0.00878217018702333\\
1.7	0.00878206554692671\\
1.71	0.00878196085565394\\
1.72	0.00878185611317857\\
1.73	0.00878175131947414\\
1.74	0.00878164647451417\\
1.75	0.00878154157827216\\
1.76	0.0087814366307216\\
1.77	0.00878133163183597\\
1.78	0.00878122658158873\\
1.79	0.00878112147995332\\
1.8	0.00878101632690319\\
1.81	0.00878091112241173\\
1.82	0.00878080586645236\\
1.83	0.00878070055899846\\
1.84	0.00878059520002341\\
1.85	0.00878048978950057\\
1.86	0.00878038432740327\\
1.87	0.00878027881370485\\
1.88	0.00878017324837863\\
1.89	0.0087800676313979\\
1.9	0.00877996196273594\\
1.91	0.00877985624236603\\
1.92	0.00877975047026142\\
1.93	0.00877964464639535\\
1.94	0.00877953877074106\\
1.95	0.00877943284327174\\
1.96	0.0087793268639606\\
1.97	0.00877922083278082\\
1.98	0.00877911474970557\\
1.99	0.00877900861470799\\
2	0.00877890242776123\\
2.01	0.0087787961888384\\
2.02	0.00877868989791262\\
2.03	0.00877858355495698\\
2.04	0.00877847715994455\\
2.05	0.00877837071284839\\
2.06	0.00877826421364157\\
2.07	0.00877815766229711\\
2.08	0.00877805105878802\\
2.09	0.00877794440308731\\
2.1	0.00877783769516798\\
2.11	0.00877773093500298\\
2.12	0.00877762412256529\\
2.13	0.00877751725782784\\
2.14	0.00877741034076357\\
2.15	0.00877730337134539\\
2.16	0.00877719634954619\\
2.17	0.00877708927533886\\
2.18	0.00877698214869627\\
2.19	0.00877687496959127\\
2.2	0.00877676773799671\\
2.21	0.0087766604538854\\
2.22	0.00877655311723016\\
2.23	0.00877644572800377\\
2.24	0.00877633828617902\\
2.25	0.00877623079172867\\
2.26	0.00877612324462546\\
2.27	0.00877601564484214\\
2.28	0.00877590799235141\\
2.29	0.00877580028712598\\
2.3	0.00877569252913855\\
2.31	0.00877558471836178\\
2.32	0.00877547685476832\\
2.33	0.00877536893833084\\
2.34	0.00877526096902194\\
2.35	0.00877515294681424\\
2.36	0.00877504487168034\\
2.37	0.00877493674359283\\
2.38	0.00877482856252425\\
2.39	0.00877472032844717\\
2.4	0.00877461204133413\\
2.41	0.00877450370115764\\
2.42	0.00877439530789021\\
2.43	0.00877428686150433\\
2.44	0.00877417836197248\\
2.45	0.00877406980926711\\
2.46	0.00877396120336067\\
2.47	0.00877385254422558\\
2.48	0.00877374383183426\\
2.49	0.00877363506615912\\
2.5	0.00877352624717252\\
2.51	0.00877341737484684\\
2.52	0.00877330844915444\\
2.53	0.00877319947006764\\
2.54	0.00877309043755877\\
2.55	0.00877298135160014\\
2.56	0.00877287221216403\\
2.57	0.00877276301922273\\
2.58	0.00877265377274848\\
2.59	0.00877254447271354\\
2.6	0.00877243511909013\\
2.61	0.00877232571185047\\
2.62	0.00877221625096675\\
2.63	0.00877210673641116\\
2.64	0.00877199716815586\\
2.65	0.008771887546173\\
2.66	0.00877177787043473\\
2.67	0.00877166814091315\\
2.68	0.00877155835758038\\
2.69	0.0087714485204085\\
2.7	0.0087713386293696\\
2.71	0.00877122868443571\\
2.72	0.0087711186855789\\
2.73	0.00877100863277117\\
2.74	0.00877089852598456\\
2.75	0.00877078836519103\\
2.76	0.0087706781503626\\
2.77	0.0087705678814712\\
2.78	0.0087704575584888\\
2.79	0.00877034718138732\\
2.8	0.00877023675013868\\
2.81	0.00877012626471479\\
2.82	0.00877001572508753\\
2.83	0.00876990513122876\\
2.84	0.00876979448311034\\
2.85	0.00876968378070412\\
2.86	0.00876957302398191\\
2.87	0.00876946221291552\\
2.88	0.00876935134747674\\
2.89	0.00876924042763735\\
2.9	0.0087691294533691\\
2.91	0.00876901842464374\\
2.92	0.008768907341433\\
2.93	0.00876879620370859\\
2.94	0.0087686850114422\\
2.95	0.00876857376460552\\
2.96	0.00876846246317021\\
2.97	0.00876835110710792\\
2.98	0.00876823969639027\\
2.99	0.00876812823098889\\
3	0.00876801671087538\\
3.01	0.00876790513602132\\
3.02	0.00876779350639828\\
3.03	0.00876768182197782\\
3.04	0.00876757008273146\\
3.05	0.00876745828863074\\
3.06	0.00876734643964715\\
3.07	0.00876723453575219\\
3.08	0.00876712257691732\\
3.09	0.008767010563114\\
3.1	0.00876689849431368\\
3.11	0.00876678637048777\\
3.12	0.0087666741916077\\
3.13	0.00876656195764485\\
3.14	0.00876644966857059\\
3.15	0.00876633732435628\\
3.16	0.00876622492497328\\
3.17	0.0087661124703929\\
3.18	0.00876599996058646\\
3.19	0.00876588739552526\\
3.2	0.00876577477518057\\
3.21	0.00876566209952366\\
3.22	0.00876554936852578\\
3.23	0.00876543658215814\\
3.24	0.00876532374039198\\
3.25	0.00876521084319849\\
3.26	0.00876509789054885\\
3.27	0.00876498488241422\\
3.28	0.00876487181876576\\
3.29	0.0087647586995746\\
3.3	0.00876464552481185\\
3.31	0.00876453229444862\\
3.32	0.00876441900845599\\
3.33	0.00876430566680503\\
3.34	0.00876419226946679\\
3.35	0.0087640788164123\\
3.36	0.00876396530761259\\
3.37	0.00876385174303865\\
3.38	0.00876373812266147\\
3.39	0.00876362444645203\\
3.4	0.00876351071438127\\
3.41	0.00876339692642013\\
3.42	0.00876328308253954\\
3.43	0.00876316918271038\\
3.44	0.00876305522690357\\
3.45	0.00876294121508995\\
3.46	0.0087628271472404\\
3.47	0.00876271302332574\\
3.48	0.0087625988433168\\
3.49	0.00876248460718437\\
3.5	0.00876237031489927\\
3.51	0.00876225596643224\\
3.52	0.00876214156175405\\
3.53	0.00876202710083544\\
3.54	0.00876191258364712\\
3.55	0.00876179801015981\\
3.56	0.00876168338034418\\
3.57	0.00876156869417092\\
3.58	0.00876145395161068\\
3.59	0.00876133915263409\\
3.6	0.00876122429721178\\
3.61	0.00876110938531435\\
3.62	0.00876099441691239\\
3.63	0.00876087939197648\\
3.64	0.00876076431047716\\
3.65	0.00876064917238499\\
3.66	0.00876053397767047\\
3.67	0.00876041872630411\\
3.68	0.0087603034182564\\
3.69	0.00876018805349781\\
3.7	0.0087600726319988\\
3.71	0.00875995715372981\\
3.72	0.00875984161866124\\
3.73	0.00875972602676351\\
3.74	0.00875961037800701\\
3.75	0.0087594946723621\\
3.76	0.00875937890979913\\
3.77	0.00875926309028845\\
3.78	0.00875914721380037\\
3.79	0.00875903128030519\\
3.8	0.0087589152897732\\
3.81	0.00875879924217467\\
3.82	0.00875868313747984\\
3.83	0.00875856697565896\\
3.84	0.00875845075668224\\
3.85	0.00875833448051988\\
3.86	0.00875821814714206\\
3.87	0.00875810175651894\\
3.88	0.00875798530862069\\
3.89	0.00875786880341743\\
3.9	0.00875775224087926\\
3.91	0.00875763562097631\\
3.92	0.00875751894367863\\
3.93	0.0087574022089563\\
3.94	0.00875728541677936\\
3.95	0.00875716856711784\\
3.96	0.00875705165994176\\
3.97	0.0087569346952211\\
3.98	0.00875681767292585\\
3.99	0.00875670059302596\\
4	0.00875658345549138\\
4.01	0.00875646626029204\\
4.02	0.00875634900739784\\
4.03	0.00875623169677867\\
4.04	0.0087561143284044\\
4.05	0.00875599690224491\\
4.06	0.00875587941827001\\
4.07	0.00875576187644954\\
4.08	0.0087556442767533\\
4.09	0.00875552661915108\\
4.1	0.00875540890361264\\
4.11	0.00875529113010774\\
4.12	0.00875517329860612\\
4.13	0.00875505540907749\\
4.14	0.00875493746149155\\
4.15	0.00875481945581798\\
4.16	0.00875470139202645\\
4.17	0.00875458327008661\\
4.18	0.00875446508996809\\
4.19	0.00875434685164049\\
4.2	0.00875422855507342\\
4.21	0.00875411020023646\\
4.22	0.00875399178709915\\
4.23	0.00875387331563105\\
4.24	0.00875375478580168\\
4.25	0.00875363619758055\\
4.26	0.00875351755093715\\
4.27	0.00875339884584094\\
4.28	0.00875328008226138\\
4.29	0.00875316126016791\\
4.3	0.00875304237952995\\
4.31	0.00875292344031691\\
4.32	0.00875280444249815\\
4.33	0.00875268538604305\\
4.34	0.00875256627092096\\
4.35	0.0087524470971012\\
4.36	0.00875232786455309\\
4.37	0.00875220857324594\\
4.38	0.008752089223149\\
4.39	0.00875196981423154\\
4.4	0.00875185034646281\\
4.41	0.00875173081981202\\
4.42	0.00875161123424838\\
4.43	0.00875149158974108\\
4.44	0.00875137188625928\\
4.45	0.00875125212377215\\
4.46	0.00875113230224881\\
4.47	0.00875101242165839\\
4.48	0.00875089248196997\\
4.49	0.00875077248315263\\
4.5	0.00875065242517546\\
4.51	0.00875053230800747\\
4.52	0.00875041213161771\\
4.53	0.00875029189597518\\
4.54	0.00875017160104886\\
4.55	0.00875005124680773\\
4.56	0.00874993083322075\\
4.57	0.00874981036025684\\
4.58	0.00874968982788494\\
4.59	0.00874956923607394\\
4.6	0.00874944858479272\\
4.61	0.00874932787401013\\
4.62	0.00874920710369504\\
4.63	0.00874908627381625\\
4.64	0.00874896538434259\\
4.65	0.00874884443524284\\
4.66	0.00874872342648578\\
4.67	0.00874860235804016\\
4.68	0.00874848122987471\\
4.69	0.00874836004195815\\
4.7	0.00874823879425919\\
4.71	0.0087481174867465\\
4.72	0.00874799611938874\\
4.73	0.00874787469215457\\
4.74	0.0087477532050126\\
4.75	0.00874763165793143\\
4.76	0.00874751005087967\\
4.77	0.00874738838382588\\
4.78	0.00874726665673861\\
4.79	0.0087471448695864\\
4.8	0.00874702302233776\\
4.81	0.00874690111496118\\
4.82	0.00874677914742515\\
4.83	0.00874665711969813\\
4.84	0.00874653503174856\\
4.85	0.00874641288354485\\
4.86	0.00874629067505542\\
4.87	0.00874616840624865\\
4.88	0.0087460460770929\\
4.89	0.00874592368755652\\
4.9	0.00874580123760785\\
4.91	0.00874567872721519\\
4.92	0.00874555615634684\\
4.93	0.00874543352497107\\
4.94	0.00874531083305613\\
4.95	0.00874518808057025\\
4.96	0.00874506526748167\\
4.97	0.00874494239375858\\
4.98	0.00874481945936915\\
4.99	0.00874469646428155\\
5	0.00874457340846392\\
5.01	0.00874445029188437\\
5.02	0.00874432711451103\\
5.03	0.00874420387631197\\
5.04	0.00874408057725526\\
5.05	0.00874395721730895\\
5.06	0.00874383379644106\\
5.07	0.00874371031461961\\
5.08	0.00874358677181259\\
5.09	0.00874346316798797\\
5.1	0.0087433395031137\\
5.11	0.00874321577715772\\
5.12	0.00874309199008795\\
5.13	0.00874296814187227\\
5.14	0.00874284423247857\\
5.15	0.00874272026187469\\
5.16	0.0087425962300285\\
5.17	0.00874247213690779\\
5.18	0.00874234798248038\\
5.19	0.00874222376671405\\
5.2	0.00874209948957655\\
5.21	0.00874197515103564\\
5.22	0.00874185075105902\\
5.23	0.00874172628961442\\
5.24	0.00874160176666952\\
5.25	0.00874147718219198\\
5.26	0.00874135253614944\\
5.27	0.00874122782850954\\
5.28	0.00874110305923989\\
5.29	0.00874097822830808\\
5.3	0.00874085333568167\\
5.31	0.00874072838132822\\
5.32	0.00874060336521525\\
5.33	0.00874047828731028\\
5.34	0.0087403531475808\\
5.35	0.00874022794599429\\
5.36	0.0087401026825182\\
5.37	0.00873997735711995\\
5.38	0.00873985196976697\\
5.39	0.00873972652042665\\
5.4	0.00873960100906637\\
5.41	0.00873947543565349\\
5.42	0.00873934980015533\\
5.43	0.00873922410253923\\
5.44	0.00873909834277246\\
5.45	0.00873897252082232\\
5.46	0.00873884663665605\\
5.47	0.00873872069024091\\
5.48	0.0087385946815441\\
5.49	0.00873846861053283\\
5.5	0.00873834247717428\\
5.51	0.0087382162814356\\
5.52	0.00873809002328394\\
5.53	0.00873796370268641\\
5.54	0.00873783731961012\\
5.55	0.00873771087402214\\
5.56	0.00873758436588955\\
5.57	0.00873745779517937\\
5.58	0.00873733116185863\\
5.59	0.00873720446589434\\
5.6	0.00873707770725346\\
5.61	0.00873695088590297\\
5.62	0.0087368240018098\\
5.63	0.00873669705494088\\
5.64	0.00873657004526311\\
5.65	0.00873644297274336\\
5.66	0.00873631583734851\\
5.67	0.00873618863904539\\
5.68	0.00873606137780083\\
5.69	0.00873593405358162\\
5.7	0.00873580666635455\\
5.71	0.00873567921608638\\
5.72	0.00873555170274385\\
5.73	0.00873542412629367\\
5.74	0.00873529648670256\\
5.75	0.00873516878393719\\
5.76	0.00873504101796422\\
5.77	0.0087349131887503\\
5.78	0.00873478529626204\\
5.79	0.00873465734046604\\
5.8	0.00873452932132888\\
5.81	0.00873440123881712\\
5.82	0.0087342730928973\\
5.83	0.00873414488353594\\
5.84	0.00873401661069953\\
5.85	0.00873388827435456\\
5.86	0.00873375987446747\\
5.87	0.00873363141100472\\
5.88	0.0087335028839327\\
5.89	0.00873337429321783\\
5.9	0.00873324563882647\\
5.91	0.00873311692072498\\
5.92	0.0087329881388797\\
5.93	0.00873285929325693\\
5.94	0.00873273038382298\\
5.95	0.00873260141054411\\
5.96	0.00873247237338657\\
5.97	0.00873234327231661\\
5.98	0.00873221410730042\\
5.99	0.0087320848783042\\
6	0.00873195558529411\\
6.01	0.00873182622823632\\
6.02	0.00873169680709694\\
6.03	0.00873156732184208\\
6.04	0.00873143777243783\\
6.05	0.00873130815885025\\
6.06	0.00873117848104538\\
6.07	0.00873104873898927\\
6.08	0.00873091893264789\\
6.09	0.00873078906198724\\
6.1	0.00873065912697328\\
6.11	0.00873052912757194\\
6.12	0.00873039906374916\\
6.13	0.00873026893547083\\
6.14	0.00873013874270282\\
6.15	0.00873000848541098\\
6.16	0.00872987816356116\\
6.17	0.00872974777711918\\
6.18	0.00872961732605082\\
6.19	0.00872948681032185\\
6.2	0.00872935622989803\\
6.21	0.00872922558474508\\
6.22	0.00872909487482872\\
6.23	0.00872896410011464\\
6.24	0.0087288332605685\\
6.25	0.00872870235615594\\
6.26	0.00872857138684258\\
6.27	0.00872844035259404\\
6.28	0.00872830925337589\\
6.29	0.0087281780891537\\
6.3	0.00872804685989299\\
6.31	0.00872791556555929\\
6.32	0.0087277842061181\\
6.33	0.00872765278153488\\
6.34	0.0087275212917751\\
6.35	0.00872738973680418\\
6.36	0.00872725811658753\\
6.37	0.00872712643109055\\
6.38	0.00872699468027859\\
6.39	0.00872686286411701\\
6.4	0.00872673098257113\\
6.41	0.00872659903560624\\
6.42	0.00872646702318764\\
6.43	0.00872633494528059\\
6.44	0.00872620280185031\\
6.45	0.00872607059286202\\
6.46	0.00872593831828093\\
6.47	0.00872580597807219\\
6.48	0.00872567357220097\\
6.49	0.00872554110063238\\
6.5	0.00872540856333155\\
6.51	0.00872527596026355\\
6.52	0.00872514329139344\\
6.53	0.00872501055668628\\
6.54	0.00872487775610707\\
6.55	0.00872474488962082\\
6.56	0.0087246119571925\\
6.57	0.00872447895878707\\
6.58	0.00872434589436945\\
6.59	0.00872421276390456\\
6.6	0.00872407956735729\\
6.61	0.0087239463046925\\
6.62	0.00872381297587503\\
6.63	0.00872367958086971\\
6.64	0.00872354611964134\\
6.65	0.00872341259215469\\
6.66	0.00872327899837452\\
6.67	0.00872314533826556\\
6.68	0.00872301161179252\\
6.69	0.0087228778189201\\
6.7	0.00872274395961296\\
6.71	0.00872261003383574\\
6.72	0.00872247604155306\\
6.73	0.00872234198272953\\
6.74	0.00872220785732971\\
6.75	0.00872207366531817\\
6.76	0.00872193940665944\\
6.77	0.00872180508131803\\
6.78	0.00872167068925842\\
6.79	0.00872153623044509\\
6.8	0.00872140170484247\\
6.81	0.00872126711241497\\
6.82	0.00872113245312701\\
6.83	0.00872099772694295\\
6.84	0.00872086293382715\\
6.85	0.00872072807374393\\
6.86	0.00872059314665761\\
6.87	0.00872045815253246\\
6.88	0.00872032309133275\\
6.89	0.00872018796302272\\
6.9	0.00872005276756659\\
6.91	0.00871991750492854\\
6.92	0.00871978217507275\\
6.93	0.00871964677796335\\
6.94	0.00871951131356449\\
6.95	0.00871937578184026\\
6.96	0.00871924018275474\\
6.97	0.00871910451627198\\
6.98	0.00871896878235603\\
6.99	0.00871883298097089\\
7	0.00871869711208053\\
7.01	0.00871856117564894\\
7.02	0.00871842517164005\\
7.03	0.00871828910001778\\
7.04	0.00871815296074603\\
7.05	0.00871801675378865\\
7.06	0.00871788047910951\\
7.07	0.00871774413667243\\
7.08	0.00871760772644121\\
7.09	0.00871747124837963\\
7.1	0.00871733470245145\\
7.11	0.00871719808862039\\
7.12	0.00871706140685016\\
7.13	0.00871692465710446\\
7.14	0.00871678783934695\\
7.15	0.00871665095354126\\
7.16	0.008716513999651\\
7.17	0.00871637697763978\\
7.18	0.00871623988747115\\
7.19	0.00871610272910867\\
7.2	0.00871596550251586\\
7.21	0.0087158282076562\\
7.22	0.00871569084449318\\
7.23	0.00871555341299026\\
7.24	0.00871541591311084\\
7.25	0.00871527834481835\\
7.26	0.00871514070807615\\
7.27	0.0087150030028476\\
7.28	0.00871486522909604\\
7.29	0.00871472738678478\\
7.3	0.0087145894758771\\
7.31	0.00871445149633626\\
7.32	0.00871431344812549\\
7.33	0.00871417533120801\\
7.34	0.00871403714554701\\
7.35	0.00871389889110566\\
7.36	0.00871376056784709\\
7.37	0.00871362217573443\\
7.38	0.00871348371473076\\
7.39	0.00871334518479915\\
7.4	0.00871320658590265\\
7.41	0.00871306791800428\\
7.42	0.00871292918106705\\
7.43	0.0087127903750539\\
7.44	0.00871265149992781\\
7.45	0.00871251255565169\\
7.46	0.00871237354218844\\
7.47	0.00871223445950095\\
7.48	0.00871209530755205\\
7.49	0.00871195608630458\\
7.5	0.00871181679572134\\
7.51	0.00871167743576511\\
7.52	0.00871153800639865\\
7.53	0.00871139850758467\\
7.54	0.00871125893928589\\
7.55	0.00871111930146498\\
7.56	0.00871097959408462\\
7.57	0.00871083981710741\\
7.58	0.00871069997049598\\
7.59	0.0087105600542129\\
7.6	0.00871042006822073\\
7.61	0.00871028001248201\\
7.62	0.00871013988695924\\
7.63	0.0087099996916149\\
7.64	0.00870985942641146\\
7.65	0.00870971909131134\\
7.66	0.00870957868627696\\
7.67	0.00870943821127071\\
7.68	0.00870929766625493\\
7.69	0.00870915705119197\\
7.7	0.00870901636604413\\
7.71	0.00870887561077371\\
7.72	0.00870873478534295\\
7.73	0.0087085938897141\\
7.74	0.00870845292384935\\
7.75	0.0087083118877109\\
7.76	0.00870817078126091\\
7.77	0.00870802960446151\\
7.78	0.00870788835727481\\
7.79	0.00870774703966289\\
7.8	0.00870760565158781\\
7.81	0.0087074641930116\\
7.82	0.00870732266389628\\
7.83	0.00870718106420382\\
7.84	0.00870703939389618\\
7.85	0.0087068976529353\\
7.86	0.00870675584128307\\
7.87	0.00870661395890138\\
7.88	0.00870647200575209\\
7.89	0.00870632998179702\\
7.9	0.00870618788699798\\
7.91	0.00870604572131675\\
7.92	0.00870590348471508\\
7.93	0.0087057611771547\\
7.94	0.00870561879859731\\
7.95	0.00870547634900459\\
7.96	0.00870533382833818\\
7.97	0.00870519123655971\\
7.98	0.00870504857363078\\
7.99	0.00870490583951297\\
8	0.00870476303416781\\
8.01	0.00870462015755683\\
8.02	0.00870447720964153\\
8.03	0.00870433419038337\\
8.04	0.00870419109974381\\
8.05	0.00870404793768424\\
8.06	0.00870390470416608\\
8.07	0.00870376139915068\\
8.08	0.00870361802259937\\
8.09	0.00870347457447348\\
8.1	0.00870333105473429\\
8.11	0.00870318746334306\\
8.12	0.00870304380026103\\
8.13	0.0087029000654494\\
8.14	0.00870275625886935\\
8.15	0.00870261238048204\\
8.16	0.0087024684302486\\
8.17	0.00870232440813013\\
8.18	0.00870218031408771\\
8.19	0.00870203614808238\\
8.2	0.00870189191007517\\
8.21	0.00870174760002707\\
8.22	0.00870160321789906\\
8.23	0.00870145876365208\\
8.24	0.00870131423724704\\
8.25	0.00870116963864484\\
8.26	0.00870102496780634\\
8.27	0.00870088022469237\\
8.28	0.00870073540926374\\
8.29	0.00870059052148124\\
8.3	0.00870044556130563\\
8.31	0.00870030052869762\\
8.32	0.00870015542361794\\
8.33	0.00870001024602724\\
8.34	0.00869986499588618\\
8.35	0.00869971967315538\\
8.36	0.00869957427779542\\
8.37	0.00869942880976689\\
8.38	0.00869928326903032\\
8.39	0.00869913765554622\\
8.4	0.00869899196927508\\
8.41	0.00869884621017736\\
8.42	0.00869870037821349\\
8.43	0.00869855447334386\\
8.44	0.00869840849552887\\
8.45	0.00869826244472885\\
8.46	0.00869811632090414\\
8.47	0.00869797012401501\\
8.48	0.00869782385402175\\
8.49	0.00869767751088459\\
8.5	0.00869753109456373\\
8.51	0.00869738460501938\\
8.52	0.00869723804221168\\
8.53	0.00869709140610076\\
8.54	0.00869694469664672\\
8.55	0.00869679791380964\\
8.56	0.00869665105754956\\
8.57	0.00869650412782651\\
8.58	0.00869635712460047\\
8.59	0.00869621004783139\\
8.6	0.00869606289747922\\
8.61	0.00869591567350387\\
8.62	0.00869576837586522\\
8.63	0.0086956210045231\\
8.64	0.00869547355943735\\
8.65	0.00869532604056776\\
8.66	0.00869517844787409\\
8.67	0.00869503078131609\\
8.68	0.00869488304085346\\
8.69	0.00869473522644589\\
8.7	0.00869458733805304\\
8.71	0.00869443937563451\\
8.72	0.00869429133914992\\
8.73	0.00869414322855883\\
8.74	0.00869399504382078\\
8.75	0.00869384678489528\\
8.76	0.00869369845174182\\
8.77	0.00869355004431984\\
8.78	0.00869340156258879\\
8.79	0.00869325300650804\\
8.8	0.00869310437603698\\
8.81	0.00869295567113494\\
8.82	0.00869280689176123\\
8.83	0.00869265803787514\\
8.84	0.00869250910943592\\
8.85	0.00869236010640279\\
8.86	0.00869221102873496\\
8.87	0.00869206187639158\\
8.88	0.0086919126493318\\
8.89	0.00869176334751473\\
8.9	0.00869161397089945\\
8.91	0.008691464519445\\
8.92	0.00869131499311041\\
8.93	0.00869116539185468\\
8.94	0.00869101571563677\\
8.95	0.00869086596441561\\
8.96	0.00869071613815011\\
8.97	0.00869056623679913\\
8.98	0.00869041626032154\\
8.99	0.00869026620867615\\
9	0.00869011608182174\\
9.01	0.00868996587971709\\
9.02	0.0086898156023209\\
9.03	0.00868966524959189\\
9.04	0.00868951482148872\\
9.05	0.00868936431797004\\
9.06	0.00868921373899445\\
9.07	0.00868906308452054\\
9.08	0.00868891235450686\\
9.09	0.00868876154891192\\
9.1	0.00868861066769423\\
9.11	0.00868845971081224\\
9.12	0.00868830867822439\\
9.13	0.00868815756988907\\
9.14	0.00868800638576466\\
9.15	0.00868785512580951\\
9.16	0.00868770378998192\\
9.17	0.00868755237824018\\
9.18	0.00868740089054254\\
9.19	0.00868724932684721\\
9.2	0.0086870976871124\\
9.21	0.00868694597129626\\
9.22	0.00868679417935692\\
9.23	0.00868664231125249\\
9.24	0.00868649036694104\\
9.25	0.0086863383463806\\
9.26	0.00868618624952919\\
9.27	0.00868603407634478\\
9.28	0.00868588182678532\\
9.29	0.00868572950080873\\
9.3	0.0086855770983729\\
9.31	0.00868542461943568\\
9.32	0.00868527206395491\\
9.33	0.00868511943188837\\
9.34	0.00868496672319382\\
9.35	0.00868481393782901\\
9.36	0.00868466107575163\\
9.37	0.00868450813691937\\
9.38	0.00868435512128985\\
9.39	0.00868420202882068\\
9.4	0.00868404885946945\\
9.41	0.0086838956131937\\
9.42	0.00868374228995096\\
9.43	0.0086835888896987\\
9.44	0.00868343541239439\\
9.45	0.00868328185799543\\
9.46	0.00868312822645923\\
9.47	0.00868297451774314\\
9.48	0.00868282073180449\\
9.49	0.00868266686860058\\
9.5	0.00868251292808867\\
9.51	0.00868235891022601\\
9.52	0.00868220481496979\\
9.53	0.00868205064227718\\
9.54	0.00868189639210532\\
9.55	0.00868174206441133\\
9.56	0.00868158765915227\\
9.57	0.0086814331762852\\
9.58	0.00868127861576711\\
9.59	0.008681123977555\\
9.6	0.00868096926160581\\
9.61	0.00868081446787646\\
9.62	0.00868065959632383\\
9.63	0.00868050464690478\\
9.64	0.00868034961957613\\
9.65	0.00868019451429465\\
9.66	0.00868003933101712\\
9.67	0.00867988406970025\\
9.68	0.00867972873030074\\
9.69	0.00867957331277525\\
9.7	0.00867941781708039\\
9.71	0.00867926224317278\\
9.72	0.00867910659100896\\
9.73	0.00867895086054547\\
9.74	0.0086787950517388\\
9.75	0.00867863916454542\\
9.76	0.00867848319892176\\
9.77	0.00867832715482422\\
9.78	0.00867817103220917\\
9.79	0.00867801483103294\\
9.8	0.00867785855125182\\
9.81	0.00867770219282209\\
9.82	0.00867754575569998\\
9.83	0.0086773892398417\\
9.84	0.00867723264520341\\
9.85	0.00867707597174124\\
9.86	0.00867691921941131\\
9.87	0.00867676238816968\\
9.88	0.00867660547797238\\
9.89	0.00867644848877541\\
9.9	0.00867629142053476\\
9.91	0.00867613427320636\\
9.92	0.0086759770467461\\
9.93	0.00867581974110985\\
9.94	0.00867566235625346\\
9.95	0.00867550489213272\\
9.96	0.00867534734870341\\
9.97	0.00867518972592126\\
9.98	0.00867503202374196\\
9.99	0.00867487424212119\\
10	0.00867471638101459\\
10.01	0.00867455844037774\\
10.02	0.00867440042016622\\
10.03	0.00867424232033556\\
10.04	0.00867408414084126\\
10.05	0.00867392588163878\\
10.06	0.00867376754268356\\
10.07	0.00867360912393098\\
10.08	0.00867345062533641\\
10.09	0.00867329204685517\\
10.1	0.00867313338844257\\
10.11	0.00867297465005386\\
10.12	0.00867281583164426\\
10.13	0.00867265693316897\\
10.14	0.00867249795458313\\
10.15	0.00867233889584188\\
10.16	0.00867217975690029\\
10.17	0.00867202053771342\\
10.18	0.00867186123823628\\
10.19	0.00867170185842387\\
10.2	0.00867154239823111\\
10.21	0.00867138285761294\\
10.22	0.00867122323652422\\
10.23	0.00867106353491981\\
10.24	0.0086709037527545\\
10.25	0.00867074388998307\\
10.26	0.00867058394656026\\
10.27	0.00867042392244077\\
10.28	0.00867026381757927\\
10.29	0.00867010363193039\\
10.3	0.00866994336544873\\
10.31	0.00866978301808885\\
10.32	0.00866962258980527\\
10.33	0.00866946208055249\\
10.34	0.00866930149028496\\
10.35	0.00866914081895711\\
10.36	0.00866898006652331\\
10.37	0.00866881923293791\\
10.38	0.00866865831815524\\
10.39	0.00866849732212955\\
10.4	0.00866833624481511\\
10.41	0.0086681750861661\\
10.42	0.0086680138461367\\
10.43	0.00866785252468105\\
10.44	0.00866769112175323\\
10.45	0.00866752963730732\\
10.46	0.00866736807129733\\
10.47	0.00866720642367727\\
10.48	0.00866704469440106\\
10.49	0.00866688288342265\\
10.5	0.0086667209906959\\
10.51	0.00866655901617466\\
10.52	0.00866639695981273\\
10.53	0.00866623482156389\\
10.54	0.00866607260138187\\
10.55	0.00866591029922036\\
10.56	0.00866574791503304\\
10.57	0.00866558544877352\\
10.58	0.00866542290039538\\
10.59	0.00866526026985219\\
10.6	0.00866509755709746\\
10.61	0.00866493476208465\\
10.62	0.00866477188476721\\
10.63	0.00866460892509855\\
10.64	0.00866444588303202\\
10.65	0.00866428275852097\\
10.66	0.00866411955151866\\
10.67	0.00866395626197837\\
10.68	0.00866379288985331\\
10.69	0.00866362943509665\\
10.7	0.00866346589766153\\
10.71	0.00866330227750107\\
10.72	0.00866313857456833\\
10.73	0.00866297478881633\\
10.74	0.00866281092019807\\
10.75	0.00866264696866651\\
10.76	0.00866248293417454\\
10.77	0.00866231881667506\\
10.78	0.00866215461612091\\
10.79	0.00866199033246489\\
10.8	0.00866182596565975\\
10.81	0.00866166151565823\\
10.82	0.00866149698241301\\
10.83	0.00866133236587675\\
10.84	0.00866116766600204\\
10.85	0.00866100288274148\\
10.86	0.00866083801604759\\
10.87	0.00866067306587286\\
10.88	0.00866050803216976\\
10.89	0.0086603429148907\\
10.9	0.00866017771398807\\
10.91	0.0086600124294142\\
10.92	0.0086598470611214\\
10.93	0.00865968160906194\\
10.94	0.00865951607318803\\
10.95	0.00865935045345187\\
10.96	0.0086591847498056\\
10.97	0.00865901896220134\\
10.98	0.00865885309059114\\
10.99	0.00865868713492705\\
11	0.00865852109516105\\
11.01	0.00865835497124509\\
11.02	0.00865818876313109\\
11.03	0.00865802247077093\\
11.04	0.00865785609411642\\
11.05	0.00865768963311938\\
11.06	0.00865752308773155\\
11.07	0.00865735645790465\\
11.08	0.00865718974359035\\
11.09	0.0086570229447403\\
11.1	0.00865685606130608\\
11.11	0.00865668909323925\\
11.12	0.00865652204049133\\
11.13	0.0086563549030138\\
11.14	0.00865618768075809\\
11.15	0.0086560203736756\\
11.16	0.00865585298171768\\
11.17	0.00865568550483565\\
11.18	0.00865551794298079\\
11.19	0.00865535029610432\\
11.2	0.00865518256415746\\
11.21	0.00865501474709134\\
11.22	0.00865484684485708\\
11.23	0.00865467885740576\\
11.24	0.0086545107846884\\
11.25	0.00865434262665601\\
11.26	0.00865417438325953\\
11.27	0.00865400605444987\\
11.28	0.0086538376401779\\
11.29	0.00865366914039446\\
11.3	0.00865350055505031\\
11.31	0.00865333188409623\\
11.32	0.0086531631274829\\
11.33	0.00865299428516099\\
11.34	0.00865282535708113\\
11.35	0.00865265634319389\\
11.36	0.00865248724344981\\
11.37	0.0086523180577994\\
11.38	0.0086521487861931\\
11.39	0.00865197942858135\\
11.4	0.0086518099849145\\
11.41	0.0086516404551429\\
11.42	0.00865147083921682\\
11.43	0.00865130113708652\\
11.44	0.00865113134870221\\
11.45	0.00865096147401405\\
11.46	0.00865079151297216\\
11.47	0.00865062146552662\\
11.48	0.00865045133162747\\
11.49	0.0086502811112247\\
11.5	0.00865011080426827\\
11.51	0.0086499404107081\\
11.52	0.00864976993049403\\
11.53	0.00864959936357592\\
11.54	0.00864942870990353\\
11.55	0.0086492579694266\\
11.56	0.00864908714209484\\
11.57	0.00864891622785791\\
11.58	0.0086487452266654\\
11.59	0.00864857413846691\\
11.6	0.00864840296321194\\
11.61	0.00864823170084998\\
11.62	0.00864806035133047\\
11.63	0.00864788891460282\\
11.64	0.00864771739061637\\
11.65	0.00864754577932043\\
11.66	0.00864737408066427\\
11.67	0.00864720229459713\\
11.68	0.00864703042106816\\
11.69	0.00864685846002652\\
11.7	0.0086466864114213\\
11.71	0.00864651427520154\\
11.72	0.00864634205131625\\
11.73	0.0086461697397144\\
11.74	0.0086459973403449\\
11.75	0.00864582485315661\\
11.76	0.00864565227809838\\
11.77	0.008645479615119\\
11.78	0.00864530686416719\\
11.79	0.00864513402519166\\
11.8	0.00864496109814106\\
11.81	0.008644788082964\\
11.82	0.00864461497960904\\
11.83	0.00864444178802471\\
11.84	0.00864426850815946\\
11.85	0.00864409513996175\\
11.86	0.00864392168337995\\
11.87	0.00864374813836241\\
11.88	0.00864357450485741\\
11.89	0.00864340078281321\\
11.9	0.00864322697217801\\
11.91	0.00864305307289998\\
11.92	0.00864287908492723\\
11.93	0.00864270500820783\\
11.94	0.0086425308426898\\
11.95	0.00864235658832113\\
11.96	0.00864218224504975\\
11.97	0.00864200781282355\\
11.98	0.00864183329159036\\
11.99	0.008641658681298\\
12	0.0086414839818942\\
12.01	0.00864130919332668\\
12.02	0.00864113431554308\\
12.03	0.00864095934849104\\
12.04	0.00864078429211811\\
12.05	0.00864060914637181\\
12.06	0.00864043391119963\\
12.07	0.008640258586549\\
12.08	0.00864008317236728\\
12.09	0.00863990766860182\\
12.1	0.00863973207519991\\
12.11	0.00863955639210881\\
12.12	0.00863938061927568\\
12.13	0.0086392047566477\\
12.14	0.00863902880417197\\
12.15	0.00863885276179553\\
12.16	0.0086386766294654\\
12.17	0.00863850040712854\\
12.18	0.00863832409473187\\
12.19	0.00863814769222227\\
12.2	0.00863797119954654\\
12.21	0.00863779461665146\\
12.22	0.00863761794348376\\
12.23	0.00863744117999012\\
12.24	0.00863726432611716\\
12.25	0.00863708738181148\\
12.26	0.00863691034701961\\
12.27	0.00863673322168805\\
12.28	0.00863655600576321\\
12.29	0.00863637869919152\\
12.3	0.0086362013019193\\
12.31	0.00863602381389286\\
12.32	0.00863584623505844\\
12.33	0.00863566856536224\\
12.34	0.00863549080475043\\
12.35	0.0086353129531691\\
12.36	0.0086351350105643\\
12.37	0.00863495697688206\\
12.38	0.00863477885206831\\
12.39	0.00863460063606898\\
12.4	0.00863442232882992\\
12.41	0.00863424393029695\\
12.42	0.00863406544041584\\
12.43	0.0086338868591323\\
12.44	0.00863370818639198\\
12.45	0.00863352942214052\\
12.46	0.00863335056632347\\
12.47	0.00863317161888636\\
12.48	0.00863299257977466\\
12.49	0.00863281344893378\\
12.5	0.0086326342263091\\
12.51	0.00863245491184593\\
12.52	0.00863227550548955\\
12.53	0.00863209600718518\\
12.54	0.00863191641687798\\
12.55	0.00863173673451309\\
12.56	0.00863155696003557\\
12.57	0.00863137709339045\\
12.58	0.00863119713452269\\
12.59	0.00863101708337721\\
12.6	0.0086308369398989\\
12.61	0.00863065670403256\\
12.62	0.00863047637572298\\
12.63	0.00863029595491486\\
12.64	0.00863011544155288\\
12.65	0.00862993483558166\\
12.66	0.00862975413694577\\
12.67	0.00862957334558971\\
12.68	0.00862939246145796\\
12.69	0.00862921148449494\\
12.7	0.008629030414645\\
12.71	0.00862884925185246\\
12.72	0.00862866799606158\\
12.73	0.00862848664721658\\
12.74	0.0086283052052616\\
12.75	0.00862812367014077\\
12.76	0.00862794204179814\\
12.77	0.0086277603201777\\
12.78	0.00862757850522341\\
12.79	0.00862739659687918\\
12.8	0.00862721459508884\\
12.81	0.00862703249979621\\
12.82	0.00862685031094501\\
12.83	0.00862666802847896\\
12.84	0.00862648565234169\\
12.85	0.00862630318247677\\
12.86	0.00862612061882776\\
12.87	0.00862593796133814\\
12.88	0.00862575520995134\\
12.89	0.00862557236461073\\
12.9	0.00862538942525965\\
12.91	0.00862520639184136\\
12.92	0.00862502326429908\\
12.93	0.00862484004257599\\
12.94	0.00862465672661519\\
12.95	0.00862447331635975\\
12.96	0.00862428981175267\\
12.97	0.00862410621273692\\
12.98	0.00862392251925539\\
12.99	0.00862373873125093\\
13	0.00862355484866633\\
13.01	0.00862337087144434\\
13.02	0.00862318679952764\\
13.03	0.00862300263285886\\
13.04	0.00862281837138059\\
13.05	0.00862263401503535\\
13.06	0.00862244956376562\\
13.07	0.0086222650175138\\
13.08	0.00862208037622227\\
13.09	0.00862189563983332\\
13.1	0.00862171080828923\\
13.11	0.00862152588153218\\
13.12	0.00862134085950433\\
13.13	0.00862115574214776\\
13.14	0.00862097052940451\\
13.15	0.00862078522121656\\
13.16	0.00862059981752584\\
13.17	0.00862041431827422\\
13.18	0.00862022872340352\\
13.19	0.0086200430328555\\
13.2	0.00861985724657186\\
13.21	0.00861967136449425\\
13.22	0.00861948538656428\\
13.23	0.00861929931272348\\
13.24	0.00861911314291333\\
13.25	0.00861892687707527\\
13.26	0.00861874051515067\\
13.27	0.00861855405708085\\
13.28	0.00861836750280706\\
13.29	0.00861818085227052\\
13.3	0.00861799410541238\\
13.31	0.00861780726217372\\
13.32	0.00861762032249559\\
13.33	0.00861743328631896\\
13.34	0.00861724615358475\\
13.35	0.00861705892423385\\
13.36	0.00861687159820705\\
13.37	0.00861668417544511\\
13.38	0.00861649665588874\\
13.39	0.00861630903947856\\
13.4	0.00861612132615517\\
13.41	0.00861593351585909\\
13.42	0.0086157456085308\\
13.43	0.0086155576041107\\
13.44	0.00861536950253915\\
13.45	0.00861518130375644\\
13.46	0.00861499300770283\\
13.47	0.00861480461431848\\
13.48	0.00861461612354354\\
13.49	0.00861442753531805\\
13.5	0.00861423884958204\\
13.51	0.00861405006627546\\
13.52	0.00861386118533819\\
13.53	0.00861367220671008\\
13.54	0.0086134831303309\\
13.55	0.00861329395614037\\
13.56	0.00861310468407815\\
13.57	0.00861291531408385\\
13.58	0.008612725846097\\
13.59	0.00861253628005709\\
13.6	0.00861234661590356\\
13.61	0.00861215685357576\\
13.62	0.00861196699301301\\
13.63	0.00861177703415456\\
13.64	0.00861158697693958\\
13.65	0.00861139682130723\\
13.66	0.00861120656719656\\
13.67	0.0086110162145466\\
13.68	0.00861082576329629\\
13.69	0.00861063521338453\\
13.7	0.00861044456475016\\
13.71	0.00861025381733195\\
13.72	0.0086100629710686\\
13.73	0.00860987202589878\\
13.74	0.00860968098176109\\
13.75	0.00860948983859405\\
13.76	0.00860929859633614\\
13.77	0.00860910725492578\\
13.78	0.00860891581430132\\
13.79	0.00860872427440105\\
13.8	0.00860853263516321\\
13.81	0.00860834089652597\\
13.82	0.00860814905842743\\
13.83	0.00860795712080566\\
13.84	0.00860776508359863\\
13.85	0.00860757294674429\\
13.86	0.00860738071018049\\
13.87	0.00860718837384505\\
13.88	0.0086069959376757\\
13.89	0.00860680340161014\\
13.9	0.00860661076558599\\
13.91	0.0086064180295408\\
13.92	0.00860622519341207\\
13.93	0.00860603225713726\\
13.94	0.00860583922065372\\
13.95	0.00860564608389877\\
13.96	0.00860545284680967\\
13.97	0.0086052595093236\\
13.98	0.00860506607137769\\
13.99	0.00860487253290901\\
14	0.00860467889385456\\
14.01	0.00860448515415128\\
14.02	0.00860429131373605\\
14.03	0.00860409737254568\\
14.04	0.00860390333051693\\
14.05	0.00860370918758648\\
14.06	0.00860351494369097\\
14.07	0.00860332059876696\\
14.08	0.00860312615275094\\
14.09	0.00860293160557935\\
14.1	0.00860273695718857\\
14.11	0.00860254220751491\\
14.12	0.00860234735649462\\
14.13	0.00860215240406387\\
14.14	0.00860195735015879\\
14.15	0.00860176219471544\\
14.16	0.0086015669376698\\
14.17	0.00860137157895781\\
14.18	0.00860117611851533\\
14.19	0.00860098055627815\\
14.2	0.00860078489218202\\
14.21	0.0086005891261626\\
14.22	0.00860039325815551\\
14.23	0.00860019728809628\\
14.24	0.0086000012159204\\
14.25	0.00859980504156327\\
14.26	0.00859960876496024\\
14.27	0.00859941238604661\\
14.28	0.00859921590475758\\
14.29	0.0085990193210283\\
14.3	0.00859882263479388\\
14.31	0.00859862584598932\\
14.32	0.00859842895454959\\
14.33	0.00859823196040958\\
14.34	0.00859803486350412\\
14.35	0.00859783766376796\\
14.36	0.00859764036113581\\
14.37	0.00859744295554228\\
14.38	0.00859724544692195\\
14.39	0.00859704783520931\\
14.4	0.00859685012033879\\
14.41	0.00859665230224476\\
14.42	0.00859645438086151\\
14.43	0.00859625635612328\\
14.44	0.00859605822796422\\
14.45	0.00859585999631845\\
14.46	0.00859566166111999\\
14.47	0.00859546322230281\\
14.48	0.00859526467980081\\
14.49	0.00859506603354782\\
14.5	0.0085948672834776\\
14.51	0.00859466842952384\\
14.52	0.00859446947162019\\
14.53	0.0085942704097002\\
14.54	0.00859407124369736\\
14.55	0.00859387197354512\\
14.56	0.00859367259917682\\
14.57	0.00859347312052576\\
14.58	0.00859327353752515\\
14.59	0.00859307385010817\\
14.6	0.00859287405820789\\
14.61	0.00859267416175733\\
14.62	0.00859247416068946\\
14.63	0.00859227405493714\\
14.64	0.00859207384443321\\
14.65	0.0085918735291104\\
14.66	0.00859167310890139\\
14.67	0.00859147258373879\\
14.68	0.00859127195355514\\
14.69	0.00859107121828293\\
14.7	0.00859087037785453\\
14.71	0.00859066943220231\\
14.72	0.0085904683812585\\
14.73	0.00859026722495531\\
14.74	0.00859006596322487\\
14.75	0.00858986459599923\\
14.76	0.00858966312321039\\
14.77	0.00858946154479025\\
14.78	0.00858925986067066\\
14.79	0.00858905807078341\\
14.8	0.0085888561750602\\
14.81	0.00858865417343267\\
14.82	0.00858845206583239\\
14.83	0.00858824985219085\\
14.84	0.00858804753243948\\
14.85	0.00858784510650963\\
14.86	0.0085876425743326\\
14.87	0.0085874399358396\\
14.88	0.00858723719096178\\
14.89	0.0085870343396302\\
14.9	0.00858683138177587\\
14.91	0.00858662831732973\\
14.92	0.00858642514622263\\
14.93	0.00858622186838537\\
14.94	0.00858601848374866\\
14.95	0.00858581499224315\\
14.96	0.00858561139379943\\
14.97	0.00858540768834798\\
14.98	0.00858520387581925\\
14.99	0.0085849999561436\\
15	0.00858479592925131\\
15.01	0.0085845917950726\\
15.02	0.00858438755353763\\
15.03	0.00858418320457645\\
15.04	0.00858397874811908\\
15.05	0.00858377418409544\\
15.06	0.00858356951243539\\
15.07	0.00858336473306871\\
15.08	0.00858315984592512\\
15.09	0.00858295485093425\\
15.1	0.00858274974802567\\
15.11	0.00858254453712888\\
15.12	0.00858233921817328\\
15.13	0.00858213379108824\\
15.14	0.00858192825580302\\
15.15	0.00858172261224682\\
15.16	0.00858151686034877\\
15.17	0.00858131100003793\\
15.18	0.00858110503124328\\
15.19	0.00858089895389372\\
15.2	0.00858069276791809\\
15.21	0.00858048647324515\\
15.22	0.00858028006980358\\
15.23	0.008580073557522\\
15.24	0.00857986693632893\\
15.25	0.00857966020615286\\
15.26	0.00857945336692215\\
15.27	0.00857924641856514\\
15.28	0.00857903936101006\\
15.29	0.00857883219418508\\
15.3	0.00857862491801829\\
15.31	0.0085784175324377\\
15.32	0.00857821003737126\\
15.33	0.00857800243274683\\
15.34	0.00857779471849221\\
15.35	0.00857758689453512\\
15.36	0.00857737896080319\\
15.37	0.00857717091722399\\
15.38	0.00857696276372502\\
15.39	0.00857675450023369\\
15.4	0.00857654612667733\\
15.41	0.00857633764298323\\
15.42	0.00857612904907856\\
15.43	0.00857592034489044\\
15.44	0.0085757115303459\\
15.45	0.00857550260537191\\
15.46	0.00857529356989536\\
15.47	0.00857508442384304\\
15.48	0.0085748751671417\\
15.49	0.008574665799718\\
15.5	0.0085744563214985\\
15.51	0.00857424673240973\\
15.52	0.00857403703237809\\
15.53	0.00857382722132996\\
15.54	0.00857361729919159\\
15.55	0.00857340726588919\\
15.56	0.00857319712134887\\
15.57	0.00857298686549668\\
15.58	0.00857277649825859\\
15.59	0.00857256601956048\\
15.6	0.00857235542932816\\
15.61	0.00857214472748738\\
15.62	0.00857193391396379\\
15.63	0.00857172298868295\\
15.64	0.00857151195157039\\
15.65	0.00857130080255152\\
15.66	0.00857108954155168\\
15.67	0.00857087816849615\\
15.68	0.00857066668331012\\
15.69	0.00857045508591869\\
15.7	0.0085702433762469\\
15.71	0.00857003155421971\\
15.72	0.008569819619762\\
15.73	0.00856960757279856\\
15.74	0.00856939541325411\\
15.75	0.0085691831410533\\
15.76	0.0085689707561207\\
15.77	0.00856875825838077\\
15.78	0.00856854564775794\\
15.79	0.00856833292417652\\
15.8	0.00856812008756076\\
15.81	0.00856790713783485\\
15.82	0.00856769407492286\\
15.83	0.00856748089874879\\
15.84	0.0085672676092366\\
15.85	0.00856705420631012\\
15.86	0.00856684068989313\\
15.87	0.00856662705990932\\
15.88	0.00856641331628231\\
15.89	0.00856619945893562\\
15.9	0.00856598548779272\\
15.91	0.00856577140277697\\
15.92	0.00856555720381167\\
15.93	0.00856534289082003\\
15.94	0.0085651284637252\\
15.95	0.00856491392245021\\
15.96	0.00856469926691805\\
15.97	0.0085644844970516\\
15.98	0.00856426961277368\\
15.99	0.00856405461400702\\
16	0.00856383950067427\\
16.01	0.008563624272698\\
16.02	0.0085634089300007\\
16.03	0.00856319347250479\\
16.04	0.00856297790013259\\
16.05	0.00856276221280634\\
16.06	0.00856254641044821\\
16.07	0.00856233049298029\\
16.08	0.00856211446032458\\
16.09	0.00856189831240301\\
16.1	0.00856168204913741\\
16.11	0.00856146567044954\\
16.12	0.00856124917626108\\
16.13	0.00856103256649363\\
16.14	0.00856081584106871\\
16.15	0.00856059899990774\\
16.16	0.00856038204293209\\
16.17	0.00856016497006301\\
16.18	0.00855994778122171\\
16.19	0.00855973047632927\\
16.2	0.00855951305530673\\
16.21	0.00855929551807503\\
16.22	0.00855907786455502\\
16.23	0.00855886009466749\\
16.24	0.00855864220833313\\
16.25	0.00855842420547255\\
16.26	0.00855820608600629\\
16.27	0.00855798784985477\\
16.28	0.00855776949693839\\
16.29	0.0085575510271774\\
16.3	0.00855733244049203\\
16.31	0.00855711373680238\\
16.32	0.00855689491602847\\
16.33	0.00855667597809028\\
16.34	0.00855645692290765\\
16.35	0.00855623775040038\\
16.36	0.00855601846048818\\
16.37	0.00855579905309065\\
16.38	0.00855557952812733\\
16.39	0.00855535988551767\\
16.4	0.00855514012518105\\
16.41	0.00855492024703674\\
16.42	0.00855470025100395\\
16.43	0.00855448013700179\\
16.44	0.0085542599049493\\
16.45	0.00855403955476543\\
16.46	0.00855381908636904\\
16.47	0.00855359849967893\\
16.48	0.00855337779461377\\
16.49	0.0085531569710922\\
16.5	0.00855293602903274\\
16.51	0.00855271496835384\\
16.52	0.00855249378897386\\
16.53	0.00855227249081108\\
16.54	0.00855205107378369\\
16.55	0.00855182953780981\\
16.56	0.00855160788280746\\
16.57	0.00855138610869457\\
16.58	0.00855116421538901\\
16.59	0.00855094220280856\\
16.6	0.00855072007087089\\
16.61	0.0085504978194936\\
16.62	0.00855027544859424\\
16.63	0.00855005295809021\\
16.64	0.00854983034789887\\
16.65	0.0085496076179375\\
16.66	0.00854938476812325\\
16.67	0.00854916179837323\\
16.68	0.00854893870860445\\
16.69	0.00854871549873384\\
16.7	0.00854849216867823\\
16.71	0.00854826871835438\\
16.72	0.00854804514767896\\
16.73	0.00854782145656855\\
16.74	0.00854759764493964\\
16.75	0.00854737371270866\\
16.76	0.00854714965979194\\
16.77	0.0085469254861057\\
16.78	0.00854670119156612\\
16.79	0.00854647677608927\\
16.8	0.00854625223959112\\
16.81	0.00854602758198759\\
16.82	0.00854580280319448\\
16.83	0.00854557790312753\\
16.84	0.00854535288170239\\
16.85	0.0085451277388346\\
16.86	0.00854490247443964\\
16.87	0.00854467708843291\\
16.88	0.0085444515807297\\
16.89	0.00854422595124523\\
16.9	0.00854400019989462\\
16.91	0.00854377432659293\\
16.92	0.0085435483312551\\
16.93	0.00854332221379602\\
16.94	0.00854309597413046\\
16.95	0.00854286961217313\\
16.96	0.00854264312783864\\
16.97	0.00854241652104152\\
16.98	0.00854218979169622\\
16.99	0.00854196293971708\\
17	0.00854173596501837\\
17.01	0.00854150886751429\\
17.02	0.00854128164711892\\
17.03	0.00854105430374629\\
17.04	0.0085408268373103\\
17.05	0.00854059924772481\\
17.06	0.00854037153490357\\
17.07	0.00854014369876023\\
17.08	0.00853991573920838\\
17.09	0.00853968765616151\\
17.1	0.00853945944953304\\
17.11	0.00853923111923627\\
17.12	0.00853900266518445\\
17.13	0.00853877408729072\\
17.14	0.00853854538546814\\
17.15	0.0085383165596297\\
17.16	0.00853808760968827\\
17.17	0.00853785853555666\\
17.18	0.00853762933714759\\
17.19	0.00853740001437368\\
17.2	0.00853717056714747\\
17.21	0.00853694099538143\\
17.22	0.00853671129898792\\
17.23	0.00853648147787922\\
17.24	0.00853625153196754\\
17.25	0.00853602146116497\\
17.26	0.00853579126538354\\
17.27	0.0085355609445352\\
17.28	0.00853533049853179\\
17.29	0.00853509992728507\\
17.3	0.00853486923070672\\
17.31	0.00853463840870834\\
17.32	0.00853440746120142\\
17.33	0.00853417638809739\\
17.34	0.00853394518930757\\
17.35	0.00853371386474321\\
17.36	0.00853348241431547\\
17.37	0.00853325083793542\\
17.38	0.00853301913551404\\
17.39	0.00853278730696224\\
17.4	0.00853255535219081\\
17.41	0.0085323232711105\\
17.42	0.00853209106363193\\
17.43	0.00853185872966567\\
17.44	0.00853162626912217\\
17.45	0.00853139368191182\\
17.46	0.00853116096794491\\
17.47	0.00853092812713165\\
17.48	0.00853069515938215\\
17.49	0.00853046206460646\\
17.5	0.00853022884271451\\
17.51	0.00852999549361618\\
17.52	0.00852976201722122\\
17.53	0.00852952841343935\\
17.54	0.00852929468218015\\
17.55	0.00852906082335314\\
17.56	0.00852882683686776\\
17.57	0.00852859272263335\\
17.58	0.00852835848055916\\
17.59	0.0085281241105544\\
17.6	0.00852788961252815\\
17.61	0.00852765498638945\\
17.62	0.00852742023204721\\
17.63	0.00852718534941031\\
17.64	0.00852695033838751\\
17.65	0.0085267151988875\\
17.66	0.00852647993081889\\
17.67	0.00852624453409021\\
17.68	0.00852600900860991\\
17.69	0.00852577335428634\\
17.7	0.00852553757102778\\
17.71	0.00852530165874244\\
17.72	0.00852506561733842\\
17.73	0.00852482944672375\\
17.74	0.00852459314680639\\
17.75	0.0085243567174942\\
17.76	0.00852412015869496\\
17.77	0.00852388347031638\\
17.78	0.00852364665226607\\
17.79	0.00852340970445155\\
17.8	0.00852317262678029\\
17.81	0.00852293541915966\\
17.82	0.00852269808149692\\
17.83	0.00852246061369928\\
17.84	0.00852222301567387\\
17.85	0.00852198528732771\\
17.86	0.00852174742856776\\
17.87	0.00852150943930088\\
17.88	0.00852127131943385\\
17.89	0.00852103306887336\\
17.9	0.00852079468752604\\
17.91	0.00852055617529841\\
17.92	0.00852031753209693\\
17.93	0.00852007875782795\\
17.94	0.00851983985239775\\
17.95	0.00851960081571252\\
17.96	0.00851936164767838\\
17.97	0.00851912234820135\\
17.98	0.00851888291718736\\
17.99	0.00851864335454228\\
18	0.00851840366017187\\
18.01	0.00851816383398183\\
18.02	0.00851792387587776\\
18.03	0.00851768378576516\\
18.04	0.00851744356354948\\
18.05	0.00851720320913606\\
18.06	0.00851696272243016\\
18.07	0.00851672210333696\\
18.08	0.00851648135176155\\
18.09	0.00851624046760894\\
18.1	0.00851599945078405\\
18.11	0.00851575830119171\\
18.12	0.00851551701873668\\
18.13	0.00851527560332362\\
18.14	0.0085150340548571\\
18.15	0.00851479237324163\\
18.16	0.00851455055838161\\
18.17	0.00851430861018136\\
18.18	0.00851406652854511\\
18.19	0.00851382431337703\\
18.2	0.00851358196458117\\
18.21	0.00851333948206151\\
18.22	0.00851309686572194\\
18.23	0.00851285411546626\\
18.24	0.00851261123119821\\
18.25	0.00851236821282141\\
18.26	0.0085121250602394\\
18.27	0.00851188177335565\\
18.28	0.00851163835207353\\
18.29	0.00851139479629632\\
18.3	0.00851115110592723\\
18.31	0.00851090728086937\\
18.32	0.00851066332102576\\
18.33	0.00851041922629935\\
18.34	0.00851017499659299\\
18.35	0.00850993063180945\\
18.36	0.00850968613185139\\
18.37	0.00850944149662142\\
18.38	0.00850919672602203\\
18.39	0.00850895181995565\\
18.4	0.00850870677832461\\
18.41	0.00850846160103113\\
18.42	0.00850821628797739\\
18.43	0.00850797083906543\\
18.44	0.00850772525419726\\
18.45	0.00850747953327474\\
18.46	0.00850723367619971\\
18.47	0.00850698768287385\\
18.48	0.00850674155319881\\
18.49	0.00850649528707611\\
18.5	0.00850624888440722\\
18.51	0.00850600234509349\\
18.52	0.00850575566903621\\
18.53	0.00850550885613655\\
18.54	0.00850526190629562\\
18.55	0.00850501481941443\\
18.56	0.0085047675953939\\
18.57	0.00850452023413485\\
18.58	0.00850427273553804\\
18.59	0.00850402509950413\\
18.6	0.00850377732593367\\
18.61	0.00850352941472715\\
18.62	0.00850328136578495\\
18.63	0.00850303317900738\\
18.64	0.00850278485429465\\
18.65	0.00850253639154688\\
18.66	0.0085022877906641\\
18.67	0.00850203905154625\\
18.68	0.0085017901740932\\
18.69	0.00850154115820471\\
18.7	0.00850129200378045\\
18.71	0.00850104271072001\\
18.72	0.00850079327892288\\
18.73	0.00850054370828848\\
18.74	0.00850029399871612\\
18.75	0.00850004415010503\\
18.76	0.00849979416235434\\
18.77	0.00849954403536311\\
18.78	0.00849929376903029\\
18.79	0.00849904336325476\\
18.8	0.00849879281793528\\
18.81	0.00849854213297056\\
18.82	0.00849829130825917\\
18.83	0.00849804034369965\\
18.84	0.0084977892391904\\
18.85	0.00849753799462974\\
18.86	0.00849728660991592\\
18.87	0.00849703508494708\\
18.88	0.00849678341962128\\
18.89	0.00849653161383647\\
18.9	0.00849627966749054\\
18.91	0.00849602758048127\\
18.92	0.00849577535270636\\
18.93	0.00849552298406339\\
18.94	0.00849527047444988\\
18.95	0.00849501782376326\\
18.96	0.00849476503190085\\
18.97	0.00849451209875989\\
18.98	0.00849425902423753\\
18.99	0.00849400580823081\\
19	0.0084937524506367\\
19.01	0.00849349895135208\\
19.02	0.00849324531027374\\
19.03	0.00849299152729834\\
19.04	0.0084927376023225\\
19.05	0.00849248353524272\\
19.06	0.00849222932595541\\
19.07	0.00849197497435689\\
19.08	0.00849172048034341\\
19.09	0.00849146584381109\\
19.1	0.00849121106465598\\
19.11	0.00849095614277404\\
19.12	0.00849070107806113\\
19.13	0.00849044587041301\\
19.14	0.00849019051972538\\
19.15	0.00848993502589382\\
19.16	0.00848967938881382\\
19.17	0.00848942360838077\\
19.18	0.00848916768449\\
19.19	0.00848891161703671\\
19.2	0.00848865540591603\\
19.21	0.00848839905102299\\
19.22	0.00848814255225253\\
19.23	0.00848788590949951\\
19.24	0.00848762912265865\\
19.25	0.00848737219162464\\
19.26	0.00848711511629204\\
19.27	0.00848685789655531\\
19.28	0.00848660053230885\\
19.29	0.00848634302344694\\
19.3	0.00848608536986378\\
19.31	0.00848582757145346\\
19.32	0.00848556962811001\\
19.33	0.00848531153972732\\
19.34	0.00848505330619923\\
19.35	0.00848479492741947\\
19.36	0.00848453640328166\\
19.37	0.00848427773367935\\
19.38	0.00848401891850599\\
19.39	0.00848375995765493\\
19.4	0.00848350085101944\\
19.41	0.00848324159849267\\
19.42	0.0084829821999677\\
19.43	0.00848272265533752\\
19.44	0.008482462964495\\
19.45	0.00848220312733294\\
19.46	0.00848194314374403\\
19.47	0.00848168301362088\\
19.48	0.008481422736856\\
19.49	0.00848116231334179\\
19.5	0.00848090174297059\\
19.51	0.00848064102563461\\
19.52	0.00848038016122599\\
19.53	0.00848011914963676\\
19.54	0.00847985799075888\\
19.55	0.00847959668448418\\
19.56	0.00847933523070441\\
19.57	0.00847907362931125\\
19.58	0.00847881188019624\\
19.59	0.00847854998325087\\
19.6	0.0084782879383665\\
19.61	0.00847802574543442\\
19.62	0.0084777634043458\\
19.63	0.00847750091499174\\
19.64	0.00847723827726324\\
19.65	0.00847697549105119\\
19.66	0.0084767125562464\\
19.67	0.00847644947273958\\
19.68	0.00847618624042133\\
19.69	0.00847592285918219\\
19.7	0.00847565932891257\\
19.71	0.0084753956495028\\
19.72	0.00847513182084311\\
19.73	0.00847486784282364\\
19.74	0.00847460371533443\\
19.75	0.00847433943826543\\
19.76	0.00847407501150649\\
19.77	0.00847381043494736\\
19.78	0.00847354570847771\\
19.79	0.00847328083198708\\
19.8	0.00847301580536496\\
19.81	0.00847275062850071\\
19.82	0.00847248530128361\\
19.83	0.00847221982360284\\
19.84	0.00847195419534747\\
19.85	0.00847168841640651\\
19.86	0.00847142248666883\\
19.87	0.00847115640602324\\
19.88	0.00847089017435843\\
19.89	0.008470623791563\\
19.9	0.00847035725752547\\
19.91	0.00847009057213424\\
19.92	0.00846982373527762\\
19.93	0.00846955674684383\\
19.94	0.008469289606721\\
19.95	0.00846902231479714\\
19.96	0.00846875487096019\\
19.97	0.00846848727509797\\
19.98	0.00846821952709822\\
19.99	0.00846795162684857\\
20	0.00846768357423657\\
20.01	0.00846741536914966\\
20.02	0.00846714701147518\\
20.03	0.00846687850110039\\
20.04	0.00846660983791244\\
20.05	0.00846634102179838\\
20.06	0.00846607205264518\\
20.07	0.00846580293033969\\
20.08	0.00846553365476868\\
20.09	0.00846526422581882\\
20.1	0.00846499464337668\\
20.11	0.00846472490732873\\
20.12	0.00846445501756136\\
20.13	0.00846418497396083\\
20.14	0.00846391477641333\\
20.15	0.00846364442480494\\
20.16	0.00846337391902164\\
20.17	0.00846310325894934\\
20.18	0.00846283244447381\\
20.19	0.00846256147548075\\
20.2	0.00846229035185577\\
20.21	0.00846201907348435\\
20.22	0.00846174764025191\\
20.23	0.00846147605204373\\
20.24	0.00846120430874503\\
20.25	0.00846093241024092\\
20.26	0.0084606603564164\\
20.27	0.00846038814715639\\
20.28	0.0084601157823457\\
20.29	0.00845984326186905\\
20.3	0.00845957058561105\\
20.31	0.00845929775345623\\
20.32	0.00845902476528902\\
20.33	0.00845875162099372\\
20.34	0.00845847832045458\\
20.35	0.00845820486355571\\
20.36	0.00845793125018115\\
20.37	0.00845765748021483\\
20.38	0.00845738355354058\\
20.39	0.00845710947004213\\
20.4	0.00845683522960313\\
20.41	0.00845656083210712\\
20.42	0.00845628627743752\\
20.43	0.00845601156547769\\
20.44	0.00845573669611087\\
20.45	0.00845546166922021\\
20.46	0.00845518648468874\\
20.47	0.00845491114239942\\
20.48	0.0084546356422351\\
20.49	0.00845435998407852\\
20.5	0.00845408416781234\\
20.51	0.00845380819331912\\
20.52	0.0084535320604813\\
20.53	0.00845325576918125\\
20.54	0.00845297931930122\\
20.55	0.00845270271072337\\
20.56	0.00845242594332976\\
20.57	0.00845214901700235\\
20.58	0.00845187193162301\\
20.59	0.00845159468707349\\
20.6	0.00845131728323546\\
20.61	0.0084510397199905\\
20.62	0.00845076199722006\\
20.63	0.00845048411480551\\
20.64	0.00845020607262812\\
20.65	0.00844992787056906\\
20.66	0.0084496495085094\\
20.67	0.00844937098633011\\
20.68	0.00844909230391208\\
20.69	0.00844881346113606\\
20.7	0.00844853445788274\\
20.71	0.00844825529403269\\
20.72	0.00844797596946639\\
20.73	0.0084476964840642\\
20.74	0.00844741683770643\\
20.75	0.00844713703027323\\
20.76	0.00844685706164469\\
20.77	0.0084465769317008\\
20.78	0.00844629664032142\\
20.79	0.00844601618738635\\
20.8	0.00844573557277527\\
20.81	0.00844545479636776\\
20.82	0.00844517385804329\\
20.83	0.00844489275768127\\
20.84	0.00844461149516097\\
20.85	0.00844433007036157\\
20.86	0.00844404848316217\\
20.87	0.00844376673344176\\
20.88	0.00844348482107921\\
20.89	0.00844320274595333\\
20.9	0.00844292050794278\\
20.91	0.00844263810692617\\
20.92	0.00844235554278199\\
20.93	0.00844207281538863\\
20.94	0.00844178992462437\\
20.95	0.00844150687036741\\
20.96	0.00844122365249583\\
20.97	0.00844094027088764\\
20.98	0.00844065672542072\\
20.99	0.00844037301597287\\
21	0.00844008914242178\\
21.01	0.00843980510464505\\
21.02	0.00843952090252017\\
21.03	0.00843923653592453\\
21.04	0.00843895200473544\\
21.05	0.00843866730883008\\
21.06	0.00843838244808555\\
21.07	0.00843809742237885\\
21.08	0.00843781223158689\\
21.09	0.00843752687558644\\
21.1	0.00843724135425422\\
21.11	0.00843695566746681\\
21.12	0.00843666981510072\\
21.13	0.00843638379703235\\
21.14	0.008436097613138\\
21.15	0.00843581126329386\\
21.16	0.00843552474737604\\
21.17	0.00843523806526055\\
21.18	0.00843495121682327\\
21.19	0.00843466420194\\
21.2	0.00843437702048646\\
21.21	0.00843408967233824\\
21.22	0.00843380215737085\\
21.23	0.00843351447545969\\
21.24	0.00843322662648006\\
21.25	0.00843293861030717\\
21.26	0.00843265042681612\\
21.27	0.00843236207588191\\
21.28	0.00843207355737946\\
21.29	0.00843178487118356\\
21.3	0.00843149601716892\\
21.31	0.00843120699521015\\
21.32	0.00843091780518175\\
21.33	0.00843062844695814\\
21.34	0.00843033892041362\\
21.35	0.00843004922542239\\
21.36	0.00842975936185857\\
21.37	0.00842946932959617\\
21.38	0.0084291791285091\\
21.39	0.00842888875847116\\
21.4	0.00842859821935607\\
21.41	0.00842830751103743\\
21.42	0.00842801663338877\\
21.43	0.00842772558628349\\
21.44	0.0084274343695949\\
21.45	0.00842714298319622\\
21.46	0.00842685142696056\\
21.47	0.00842655970076095\\
21.48	0.00842626780447029\\
21.49	0.0084259757379614\\
21.5	0.008425683501107\\
21.51	0.00842539109377971\\
21.52	0.00842509851585204\\
21.53	0.00842480576719641\\
21.54	0.00842451284768514\\
21.55	0.00842421975719046\\
21.56	0.00842392649558449\\
21.57	0.00842363306273925\\
21.58	0.00842333945852666\\
21.59	0.00842304568281854\\
21.6	0.00842275173548663\\
21.61	0.00842245761640255\\
21.62	0.00842216332543783\\
21.63	0.00842186886246389\\
21.64	0.00842157422735206\\
21.65	0.00842127941997358\\
21.66	0.00842098444019957\\
21.67	0.00842068928790107\\
21.68	0.00842039396294901\\
21.69	0.00842009846521423\\
21.7	0.00841980279456747\\
21.71	0.00841950695087935\\
21.72	0.00841921093402042\\
21.73	0.00841891474386113\\
21.74	0.0084186183802718\\
21.75	0.00841832184312268\\
21.76	0.00841802513228393\\
21.77	0.00841772824762558\\
21.78	0.00841743118901758\\
21.79	0.00841713395632978\\
21.8	0.00841683654943193\\
21.81	0.00841653896819368\\
21.82	0.00841624121248459\\
21.83	0.00841594328217411\\
21.84	0.0084156451771316\\
21.85	0.00841534689722632\\
21.86	0.00841504844232743\\
21.87	0.00841474981230398\\
21.88	0.00841445100702496\\
21.89	0.00841415202635921\\
21.9	0.00841385287017552\\
21.91	0.00841355353834254\\
21.92	0.00841325403072886\\
21.93	0.00841295434720295\\
21.94	0.00841265448763318\\
21.95	0.00841235445188785\\
21.96	0.00841205423983511\\
21.97	0.00841175385134308\\
21.98	0.00841145328627972\\
21.99	0.00841115254451292\\
22	0.00841085162591049\\
22.01	0.00841055053034011\\
22.02	0.00841024925766938\\
22.03	0.00840994780776579\\
22.04	0.00840964618049676\\
22.05	0.00840934437572959\\
22.06	0.00840904239333149\\
22.07	0.00840874023316955\\
22.08	0.00840843789511081\\
22.09	0.00840813537902217\\
22.1	0.00840783268477046\\
22.11	0.0084075298122224\\
22.12	0.00840722676124461\\
22.13	0.00840692353170364\\
22.14	0.0084066201234659\\
22.15	0.00840631653639774\\
22.16	0.0084060127703654\\
22.17	0.00840570882523502\\
22.18	0.00840540470087264\\
22.19	0.00840510039714423\\
22.2	0.00840479591391564\\
22.21	0.00840449125105262\\
22.22	0.00840418640842083\\
22.23	0.00840388138588585\\
22.24	0.00840357618331315\\
22.25	0.00840327080056809\\
22.26	0.00840296523751597\\
22.27	0.00840265949402196\\
22.28	0.00840235356995115\\
22.29	0.00840204746516854\\
22.3	0.00840174117953902\\
22.31	0.0084014347129274\\
22.32	0.00840112806519838\\
22.33	0.00840082123621656\\
22.34	0.00840051422584649\\
22.35	0.00840020703395256\\
22.36	0.00839989966039911\\
22.37	0.00839959210505037\\
22.38	0.00839928436777048\\
22.39	0.00839897644842348\\
22.4	0.00839866834687331\\
22.41	0.00839836006298384\\
22.42	0.00839805159661882\\
22.43	0.00839774294764192\\
22.44	0.0083974341159167\\
22.45	0.00839712510130664\\
22.46	0.00839681590367512\\
22.47	0.00839650652288545\\
22.48	0.00839619695880081\\
22.49	0.00839588721128429\\
22.5	0.00839557728019891\\
22.51	0.00839526716540758\\
22.52	0.00839495686677313\\
22.53	0.00839464638415828\\
22.54	0.00839433571742567\\
22.55	0.00839402486643783\\
22.56	0.00839371383105721\\
22.57	0.00839340261114619\\
22.58	0.008393091206567\\
22.59	0.00839277961718183\\
22.6	0.00839246784285276\\
22.61	0.00839215588344176\\
22.62	0.00839184373881073\\
22.63	0.00839153140882147\\
22.64	0.00839121889333569\\
22.65	0.00839090619221501\\
22.66	0.00839059330532096\\
22.67	0.00839028023251496\\
22.68	0.00838996697365836\\
22.69	0.0083896535286124\\
22.7	0.00838933989723826\\
22.71	0.00838902607939699\\
22.72	0.00838871207494958\\
22.73	0.00838839788375689\\
22.74	0.00838808350567975\\
22.75	0.00838776894057884\\
22.76	0.00838745418831479\\
22.77	0.00838713924874812\\
22.78	0.00838682412173925\\
22.79	0.00838650880714853\\
22.8	0.00838619330483622\\
22.81	0.00838587761466248\\
22.82	0.00838556173648737\\
22.83	0.00838524567017089\\
22.84	0.00838492941557293\\
22.85	0.0083846129725533\\
22.86	0.0083842963409717\\
22.87	0.00838397952068778\\
22.88	0.00838366251156106\\
22.89	0.00838334531345099\\
22.9	0.00838302792621694\\
22.91	0.00838271034971817\\
22.92	0.00838239258381388\\
22.93	0.00838207462836314\\
22.94	0.00838175648322499\\
22.95	0.00838143814825833\\
22.96	0.00838111962332199\\
22.97	0.00838080090827473\\
22.98	0.00838048200297519\\
22.99	0.00838016290728196\\
23	0.0083798436210535\\
23.01	0.00837952414414822\\
23.02	0.00837920447642443\\
23.03	0.00837888461774035\\
23.04	0.00837856456795411\\
23.05	0.00837824432692377\\
23.06	0.00837792389450729\\
23.07	0.00837760327056255\\
23.08	0.00837728245494734\\
23.09	0.00837696144751937\\
23.1	0.00837664024813625\\
23.11	0.00837631885665554\\
23.12	0.00837599727293468\\
23.13	0.00837567549683104\\
23.14	0.00837535352820189\\
23.15	0.00837503136690444\\
23.16	0.00837470901279579\\
23.17	0.00837438646573299\\
23.18	0.00837406372557297\\
23.19	0.0083737407921726\\
23.2	0.00837341766538865\\
23.21	0.00837309434507782\\
23.22	0.00837277083109672\\
23.23	0.00837244712330188\\
23.24	0.00837212322154974\\
23.25	0.00837179912569668\\
23.26	0.00837147483559896\\
23.27	0.00837115035111279\\
23.28	0.00837082567209429\\
23.29	0.00837050079839949\\
23.3	0.00837017572988435\\
23.31	0.00836985046640473\\
23.32	0.00836952500781644\\
23.33	0.00836919935397517\\
23.34	0.00836887350473657\\
23.35	0.00836854745995617\\
23.36	0.00836822121948945\\
23.37	0.0083678947831918\\
23.38	0.00836756815091852\\
23.39	0.00836724132252485\\
23.4	0.00836691429786593\\
23.41	0.00836658707679684\\
23.42	0.00836625965917257\\
23.43	0.00836593204484803\\
23.44	0.00836560423367805\\
23.45	0.0083652762255174\\
23.46	0.00836494802022074\\
23.47	0.00836461961764269\\
23.48	0.00836429101763776\\
23.49	0.0083639622200604\\
23.5	0.00836363322476497\\
23.51	0.00836330403160578\\
23.52	0.00836297464043703\\
23.53	0.00836264505111287\\
23.54	0.00836231526348735\\
23.55	0.00836198527741446\\
23.56	0.00836165509274811\\
23.57	0.00836132470934215\\
23.58	0.00836099412705032\\
23.59	0.00836066334572631\\
23.6	0.00836033236522375\\
23.61	0.00836000118539615\\
23.62	0.00835966980609699\\
23.63	0.00835933822717965\\
23.64	0.00835900644849746\\
23.65	0.00835867446990364\\
23.66	0.00835834229125138\\
23.67	0.00835800991239375\\
23.68	0.0083576773331838\\
23.69	0.00835734455347446\\
23.7	0.00835701157311862\\
23.71	0.00835667839196909\\
23.72	0.0083563450098786\\
23.73	0.00835601142669982\\
23.74	0.00835567764228534\\
23.75	0.00835534365648768\\
23.76	0.0083550094691593\\
23.77	0.00835467508015258\\
23.78	0.00835434048931984\\
23.79	0.00835400569651331\\
23.8	0.00835367070158518\\
23.81	0.00835333550438755\\
23.82	0.00835300010477246\\
23.83	0.00835266450259187\\
23.84	0.00835232869769769\\
23.85	0.00835199268994175\\
23.86	0.00835165647917582\\
23.87	0.00835132006525159\\
23.88	0.00835098344802071\\
23.89	0.00835064662733474\\
23.9	0.00835030960304517\\
23.91	0.00834997237500344\\
23.92	0.00834963494306093\\
23.93	0.00834929730706893\\
23.94	0.00834895946687869\\
23.95	0.00834862142234137\\
23.96	0.0083482831733081\\
23.97	0.00834794471962991\\
23.98	0.00834760606115779\\
23.99	0.00834726719774266\\
24	0.00834692812923538\\
24.01	0.00834658885548674\\
24.02	0.00834624937634748\\
24.03	0.00834590969166826\\
24.04	0.00834556980129969\\
24.05	0.00834522970509233\\
24.06	0.00834488940289666\\
24.07	0.00834454889456311\\
24.08	0.00834420817994205\\
24.09	0.00834386725888378\\
24.1	0.00834352613123855\\
24.11	0.00834318479685655\\
24.12	0.00834284325558792\\
24.13	0.00834250150728271\\
24.14	0.00834215955179096\\
24.15	0.00834181738896261\\
24.16	0.00834147501864757\\
24.17	0.00834113244069567\\
24.18	0.0083407896549567\\
24.19	0.0083404466612804\\
24.2	0.00834010345951644\\
24.21	0.00833976004951444\\
24.22	0.00833941643112397\\
24.23	0.00833907260419453\\
24.24	0.00833872856857557\\
24.25	0.00833838432411652\\
24.26	0.00833803987066671\\
24.27	0.00833769520807543\\
24.28	0.00833735033619194\\
24.29	0.00833700525486543\\
24.3	0.00833665996394503\\
24.31	0.00833631446327984\\
24.32	0.0083359687527189\\
24.33	0.00833562283211119\\
24.34	0.00833527670130564\\
24.35	0.00833493036015116\\
24.36	0.00833458380849656\\
24.37	0.00833423704619066\\
24.38	0.00833389007308218\\
24.39	0.00833354288901983\\
24.4	0.00833319549385225\\
24.41	0.00833284788742803\\
24.42	0.00833250006959573\\
24.43	0.00833215204020386\\
24.44	0.00833180379910088\\
24.45	0.0083314553461352\\
24.46	0.0083311066811552\\
24.47	0.0083307578040092\\
24.48	0.00833040871454549\\
24.49	0.00833005941261231\\
24.5	0.00832970989805784\\
24.51	0.00832936017073025\\
24.52	0.00832901023047765\\
24.53	0.0083286600771481\\
24.54	0.00832830971058964\\
24.55	0.00832795913065026\\
24.56	0.0083276083371779\\
24.57	0.00832725733002047\\
24.58	0.00832690610902584\\
24.59	0.00832655467404184\\
24.6	0.00832620302491628\\
24.61	0.00832585116149688\\
24.62	0.00832549908363139\\
24.63	0.00832514679116747\\
24.64	0.00832479428395278\\
24.65	0.00832444156183492\\
24.66	0.00832408862466146\\
24.67	0.00832373547227994\\
24.68	0.00832338210453788\\
24.69	0.00832302852128273\\
24.7	0.00832267472236194\\
24.71	0.00832232070762292\\
24.72	0.00832196647691304\\
24.73	0.00832161203007963\\
24.74	0.00832125736697002\\
24.75	0.00832090248743147\\
24.76	0.00832054739131124\\
24.77	0.00832019207845655\\
24.78	0.0083198365487146\\
24.79	0.00831948080193254\\
24.8	0.00831912483795751\\
24.81	0.00831876865663663\\
24.82	0.00831841225781696\\
24.83	0.00831805564134557\\
24.84	0.00831769880706949\\
24.85	0.00831734175483573\\
24.86	0.00831698448449126\\
24.87	0.00831662699588305\\
24.88	0.00831626928885802\\
24.89	0.00831591136326308\\
24.9	0.00831555321894514\\
24.91	0.00831519485575105\\
24.92	0.00831483627351357\\
24.93	0.00831447747203023\\
24.94	0.00831411845109824\\
24.95	0.00831375921051452\\
24.96	0.0083133997500757\\
24.97	0.00831304006957812\\
24.98	0.0083126801688178\\
24.99	0.00831232004759048\\
25	0.0083119597056916\\
25.01	0.00831159914291629\\
25.02	0.00831123835905939\\
25.03	0.00831087735391545\\
25.04	0.00831051612727869\\
25.05	0.00831015467894306\\
25.06	0.00830979300870219\\
25.07	0.00830943111634941\\
25.08	0.00830906900167775\\
25.09	0.00830870666447992\\
25.1	0.00830834410454836\\
25.11	0.00830798132167517\\
25.12	0.00830761831565217\\
25.13	0.00830725508627085\\
25.14	0.00830689163332241\\
25.15	0.00830652795659774\\
25.16	0.00830616405588743\\
25.17	0.00830579993098173\\
25.18	0.00830543558167063\\
25.19	0.00830507100774376\\
25.2	0.00830470620899048\\
25.21	0.00830434118519981\\
25.22	0.00830397593616047\\
25.23	0.00830361046166087\\
25.24	0.0083032447614891\\
25.25	0.00830287883543295\\
25.26	0.00830251268327989\\
25.27	0.00830214630481706\\
25.28	0.0083017796998313\\
25.29	0.00830141286810914\\
25.3	0.00830104580943677\\
25.31	0.00830067852360009\\
25.32	0.00830031101038466\\
25.33	0.00829994326957574\\
25.34	0.00829957530095825\\
25.35	0.00829920710431681\\
25.36	0.00829883867943571\\
25.37	0.00829847002609892\\
25.38	0.00829810114409009\\
25.39	0.00829773203319255\\
25.4	0.0082973626931893\\
25.41	0.00829699312386302\\
25.42	0.00829662332499607\\
25.43	0.00829625329637048\\
25.44	0.00829588303776796\\
25.45	0.00829551254896989\\
25.46	0.00829514182975732\\
25.47	0.00829477087991097\\
25.48	0.00829439969921125\\
25.49	0.00829402828743823\\
25.5	0.00829365664437164\\
25.51	0.0082932847697909\\
25.52	0.0082929126634751\\
25.53	0.00829254032520296\\
25.54	0.00829216775475292\\
25.55	0.00829179495190306\\
25.56	0.00829142191643113\\
25.57	0.00829104864811455\\
25.58	0.0082906751467304\\
25.59	0.00829030141205543\\
25.6	0.00828992744386605\\
25.61	0.00828955324193834\\
25.62	0.00828917880604804\\
25.63	0.00828880413597054\\
25.64	0.00828842923148092\\
25.65	0.00828805409235389\\
25.66	0.00828767871836385\\
25.67	0.00828730310928483\\
25.68	0.00828692726489053\\
25.69	0.00828655118495433\\
25.7	0.00828617486924923\\
25.71	0.00828579831754791\\
25.72	0.00828542152962271\\
25.73	0.00828504450524561\\
25.74	0.00828466724418825\\
25.75	0.00828428974622193\\
25.76	0.00828391201111761\\
25.77	0.00828353403864588\\
25.78	0.00828315582857701\\
25.79	0.00828277738068089\\
25.8	0.00828239869472709\\
25.81	0.00828201977048481\\
25.82	0.00828164060772292\\
25.83	0.00828126120620991\\
25.84	0.00828088156571395\\
25.85	0.00828050168600283\\
25.86	0.00828012156684401\\
25.87	0.00827974120800457\\
25.88	0.00827936060925127\\
25.89	0.00827897977035049\\
25.9	0.00827859869106824\\
25.91	0.00827821737117022\\
25.92	0.00827783581042173\\
25.93	0.00827745400858773\\
25.94	0.00827707196543281\\
25.95	0.00827668968072122\\
25.96	0.00827630715421684\\
25.97	0.00827592438568318\\
25.98	0.0082755413748834\\
25.99	0.00827515812158075\\
26	0.00827477462554393\\
26.01	0.00827439088654129\\
26.02	0.00827400690434086\\
26.03	0.0082736226787103\\
26.04	0.00827323820941694\\
26.05	0.00827285349622775\\
26.06	0.00827246853890938\\
26.07	0.0082720833372281\\
26.08	0.00827169789094987\\
26.09	0.00827131219984025\\
26.1	0.00827092626366451\\
26.11	0.00827054008218754\\
26.12	0.00827015365517386\\
26.13	0.00826976698238768\\
26.14	0.00826938006359284\\
26.15	0.00826899289855281\\
26.16	0.00826860548703074\\
26.17	0.00826821782878939\\
26.18	0.00826782992359119\\
26.19	0.00826744177119821\\
26.2	0.00826705337137217\\
26.21	0.00826666472387441\\
26.22	0.00826627582846593\\
26.23	0.00826588668490736\\
26.24	0.008265497292959\\
26.25	0.00826510765238075\\
26.26	0.00826471776293218\\
26.27	0.00826432762437247\\
26.28	0.00826393723646048\\
26.29	0.00826354659895466\\
26.3	0.00826315571161313\\
26.31	0.00826276457419362\\
26.32	0.00826237318645352\\
26.33	0.00826198154814983\\
26.34	0.00826158965903921\\
26.35	0.00826119751887793\\
26.36	0.0082608051274219\\
26.37	0.00826041248442666\\
26.38	0.00826001958964737\\
26.39	0.00825962644283884\\
26.4	0.00825923304375551\\
26.41	0.00825883939215141\\
26.42	0.00825844548778024\\
26.43	0.00825805133039531\\
26.44	0.00825765691974954\\
26.45	0.00825726225559552\\
26.46	0.00825686733768541\\
26.47	0.00825647216577102\\
26.48	0.00825607673960379\\
26.49	0.00825568105893476\\
26.5	0.00825528512351461\\
26.51	0.00825488893309363\\
26.52	0.00825449248742174\\
26.53	0.00825409578624845\\
26.54	0.00825369882932293\\
26.55	0.00825330161639393\\
26.56	0.00825290414720983\\
26.57	0.00825250642151863\\
26.58	0.00825210843906794\\
26.59	0.008251710199605\\
26.6	0.00825131170287662\\
26.61	0.00825091294862926\\
26.62	0.00825051393660899\\
26.63	0.00825011466656146\\
26.64	0.00824971513823197\\
26.65	0.0082493153513654\\
26.66	0.00824891530570626\\
26.67	0.00824851500099863\\
26.68	0.00824811443698624\\
26.69	0.0082477136134124\\
26.7	0.00824731253002002\\
26.71	0.00824691118655165\\
26.72	0.00824650958274939\\
26.73	0.00824610771835498\\
26.74	0.00824570559310976\\
26.75	0.00824530320675465\\
26.76	0.00824490055903019\\
26.77	0.00824449764967651\\
26.78	0.00824409447843332\\
26.79	0.00824369104503996\\
26.8	0.00824328734923535\\
26.81	0.00824288339075801\\
26.82	0.00824247916934604\\
26.83	0.00824207468473715\\
26.84	0.00824166993666864\\
26.85	0.0082412649248774\\
26.86	0.0082408596490999\\
26.87	0.00824045410907223\\
26.88	0.00824004830453003\\
26.89	0.00823964223520856\\
26.9	0.00823923590084266\\
26.91	0.00823882930116674\\
26.92	0.00823842243591482\\
26.93	0.00823801530482048\\
26.94	0.00823760790761691\\
26.95	0.00823720024403688\\
26.96	0.00823679231381271\\
26.97	0.00823638411667634\\
26.98	0.00823597565235927\\
26.99	0.00823556692059258\\
27	0.00823515792110695\\
27.01	0.00823474865363261\\
27.02	0.00823433911789938\\
27.03	0.00823392931363666\\
27.04	0.00823351924057342\\
27.05	0.0082331088984382\\
27.06	0.00823269828695911\\
27.07	0.00823228740586385\\
27.08	0.00823187625487967\\
27.09	0.00823146483373342\\
27.1	0.00823105314215149\\
27.11	0.00823064117985985\\
27.12	0.00823022894658404\\
27.13	0.00822981644204917\\
27.14	0.00822940366597991\\
27.15	0.00822899061810049\\
27.16	0.00822857729813471\\
27.17	0.00822816370580593\\
27.18	0.00822774984083709\\
27.19	0.00822733570295067\\
27.2	0.00822692129186871\\
27.21	0.00822650660731283\\
27.22	0.00822609164900418\\
27.23	0.0082256764166635\\
27.24	0.00822526091001106\\
27.25	0.00822484512876669\\
27.26	0.0082244290726498\\
27.27	0.0082240127413793\\
27.28	0.00822359613467372\\
27.29	0.00822317925225108\\
27.3	0.008222762093829\\
27.31	0.00822234465912461\\
27.32	0.00822192694785462\\
27.33	0.00822150895973527\\
27.34	0.00822109069448236\\
27.35	0.00822067215181122\\
27.36	0.00822025333143672\\
27.37	0.00821983423307331\\
27.38	0.00821941485643496\\
27.39	0.00821899520123517\\
27.4	0.00821857526718699\\
27.41	0.00821815505400302\\
27.42	0.00821773456139539\\
27.43	0.00821731378907577\\
27.44	0.00821689273675536\\
27.45	0.00821647140414492\\
27.46	0.00821604979095471\\
27.47	0.00821562789689455\\
27.48	0.00821520572167378\\
27.49	0.00821478326500128\\
27.5	0.00821436052658546\\
27.51	0.00821393750613426\\
27.52	0.00821351420335514\\
27.53	0.00821309061795509\\
27.54	0.00821266674964064\\
27.55	0.00821224259811783\\
27.56	0.00821181816309224\\
27.57	0.00821139344426895\\
27.58	0.00821096844135259\\
27.59	0.00821054315404731\\
27.6	0.00821011758205674\\
27.61	0.00820969172508408\\
27.62	0.00820926558283203\\
27.63	0.00820883915500281\\
27.64	0.00820841244129813\\
27.65	0.00820798544141925\\
27.66	0.00820755815506693\\
27.67	0.00820713058194145\\
27.68	0.00820670272174259\\
27.69	0.00820627457416965\\
27.7	0.00820584613892144\\
27.71	0.00820541741569628\\
27.72	0.00820498840419198\\
27.73	0.00820455910410589\\
27.74	0.00820412951513484\\
27.75	0.00820369963697516\\
27.76	0.00820326946932271\\
27.77	0.00820283901187283\\
27.78	0.00820240826432037\\
27.79	0.00820197722635969\\
27.8	0.00820154589768462\\
27.81	0.00820111427798851\\
27.82	0.00820068236696421\\
27.83	0.00820025016430406\\
27.84	0.00819981766969989\\
27.85	0.00819938488284302\\
27.86	0.00819895180342428\\
27.87	0.00819851843113397\\
27.88	0.0081980847656619\\
27.89	0.00819765080669736\\
27.9	0.00819721655392913\\
27.91	0.00819678200704546\\
27.92	0.00819634716573411\\
27.93	0.00819591202968231\\
27.94	0.00819547659857679\\
27.95	0.00819504087210375\\
27.96	0.00819460484994885\\
27.97	0.00819416853179727\\
27.98	0.00819373191733364\\
27.99	0.0081932950062421\\
28	0.00819285779820622\\
28.01	0.00819242029290907\\
28.02	0.00819198249003321\\
28.03	0.00819154438926064\\
28.04	0.00819110599027286\\
28.05	0.00819066729275082\\
28.06	0.00819022829637496\\
28.07	0.00818978900082516\\
28.08	0.00818934940578081\\
28.09	0.0081889095109207\\
28.1	0.00818846931592316\\
28.11	0.00818802882046593\\
28.12	0.00818758802422624\\
28.13	0.00818714692688076\\
28.14	0.00818670552810564\\
28.15	0.00818626382757647\\
28.16	0.00818582182496832\\
28.17	0.00818537951995569\\
28.18	0.00818493691221255\\
28.19	0.00818449400141234\\
28.2	0.00818405078722791\\
28.21	0.0081836072693316\\
28.22	0.00818316344739518\\
28.23	0.00818271932108988\\
28.24	0.00818227489008637\\
28.25	0.00818183015405477\\
28.26	0.00818138511266464\\
28.27	0.008180939765585\\
28.28	0.00818049411248428\\
28.29	0.0081800481530304\\
28.3	0.00817960188689067\\
28.31	0.00817915531373187\\
28.32	0.0081787084332202\\
28.33	0.00817826124502132\\
28.34	0.00817781374880031\\
28.35	0.00817736594422167\\
28.36	0.00817691783094935\\
28.37	0.00817646940864673\\
28.38	0.00817602067697663\\
28.39	0.00817557163560127\\
28.4	0.00817512228418232\\
28.41	0.00817467262238087\\
28.42	0.00817422264985744\\
28.43	0.00817377236627195\\
28.44	0.00817332177128379\\
28.45	0.00817287086455172\\
28.46	0.00817241964573395\\
28.47	0.0081719681144881\\
28.48	0.00817151627047121\\
28.49	0.00817106411333971\\
28.5	0.00817061164274949\\
28.51	0.00817015885835582\\
28.52	0.00816970575981339\\
28.53	0.0081692523467763\\
28.54	0.00816879861889806\\
28.55	0.0081683445758316\\
28.56	0.00816789021722923\\
28.57	0.00816743554274269\\
28.58	0.00816698055202311\\
28.59	0.00816652524472102\\
28.6	0.00816606962048636\\
28.61	0.00816561367896846\\
28.62	0.00816515741981606\\
28.63	0.00816470084267728\\
28.64	0.00816424394719966\\
28.65	0.0081637867330301\\
28.66	0.00816332919981493\\
28.67	0.00816287134719985\\
28.68	0.00816241317482994\\
28.69	0.0081619546823497\\
28.7	0.00816149586940298\\
28.71	0.00816103673563305\\
28.72	0.00816057728068254\\
28.73	0.00816011750419347\\
28.74	0.00815965740580724\\
28.75	0.00815919698516465\\
28.76	0.00815873624190584\\
28.77	0.00815827517567035\\
28.78	0.00815781378609711\\
28.79	0.0081573520728244\\
28.8	0.00815689003548988\\
28.81	0.00815642767373058\\
28.82	0.00815596498718291\\
28.83	0.00815550197548263\\
28.84	0.00815503863826488\\
28.85	0.00815457497516418\\
28.86	0.00815411098581438\\
28.87	0.00815364666984871\\
28.88	0.00815318202689976\\
28.89	0.0081527170565995\\
28.9	0.00815225175857922\\
28.91	0.00815178613246959\\
28.92	0.00815132017790064\\
28.93	0.00815085389450174\\
28.94	0.00815038728190162\\
28.95	0.00814992033972836\\
28.96	0.00814945306760937\\
28.97	0.00814898546517145\\
28.98	0.00814851753204071\\
28.99	0.00814804926784261\\
29	0.00814758067220196\\
29.01	0.00814711174474293\\
29.02	0.00814664248508899\\
29.03	0.00814617289286299\\
29.04	0.00814570296768709\\
29.05	0.00814523270918279\\
29.06	0.00814476211697092\\
29.07	0.00814429119067167\\
29.08	0.00814381992990452\\
29.09	0.00814334833428831\\
29.1	0.0081428764034412\\
29.11	0.00814240413698067\\
29.12	0.00814193153452353\\
29.13	0.00814145859568591\\
29.14	0.00814098532008326\\
29.15	0.00814051170733037\\
29.16	0.00814003775704132\\
29.17	0.00813956346882952\\
29.18	0.0081390888423077\\
29.19	0.00813861387708789\\
29.2	0.00813813857278145\\
29.21	0.00813766292899902\\
29.22	0.00813718694535059\\
29.23	0.00813671062144543\\
29.24	0.00813623395689212\\
29.25	0.00813575695129855\\
29.26	0.0081352796042719\\
29.27	0.00813480191541867\\
29.28	0.00813432388434464\\
29.29	0.0081338455106549\\
29.3	0.00813336679395382\\
29.31	0.0081328877338451\\
29.32	0.00813240832993167\\
29.33	0.00813192858181583\\
29.34	0.0081314484890991\\
29.35	0.00813096805138233\\
29.36	0.00813048726826563\\
29.37	0.0081300061393484\\
29.38	0.00812952466422935\\
29.39	0.00812904284250643\\
29.4	0.00812856067377689\\
29.41	0.00812807815763726\\
29.42	0.00812759529368334\\
29.43	0.0081271120815102\\
29.44	0.00812662852071219\\
29.45	0.00812614461088293\\
29.46	0.00812566035161529\\
29.47	0.00812517574250144\\
29.48	0.00812469078313279\\
29.49	0.00812420547310003\\
29.5	0.00812371981199309\\
29.51	0.00812323379940119\\
29.52	0.00812274743491279\\
29.53	0.0081222607181156\\
29.54	0.00812177364859659\\
29.55	0.008121286225942\\
29.56	0.0081207984497373\\
29.57	0.00812031031956723\\
29.58	0.00811982183501575\\
29.59	0.00811933299566609\\
29.6	0.00811884380110071\\
29.61	0.00811835425090132\\
29.62	0.00811786434464888\\
29.63	0.00811737408192356\\
29.64	0.00811688346230479\\
29.65	0.00811639248537124\\
29.66	0.0081159011507008\\
29.67	0.00811540945787059\\
29.68	0.00811491740645696\\
29.69	0.0081144249960355\\
29.7	0.00811393222618102\\
29.71	0.00811343909646754\\
29.72	0.00811294560646834\\
29.73	0.00811245175575588\\
29.74	0.00811195754390184\\
29.75	0.00811146297047716\\
29.76	0.00811096803505195\\
29.77	0.00811047273719554\\
29.78	0.0081099770764765\\
29.79	0.00810948105246259\\
29.8	0.00810898466472075\\
29.81	0.00810848791281718\\
29.82	0.00810799079631725\\
29.83	0.00810749331478552\\
29.84	0.00810699546778578\\
29.85	0.00810649725488102\\
29.86	0.00810599867563339\\
29.87	0.00810549972960426\\
29.88	0.0081050004163542\\
29.89	0.00810450073544294\\
29.9	0.00810400068642942\\
29.91	0.00810350026887177\\
29.92	0.0081029994823273\\
29.93	0.00810249832635248\\
29.94	0.00810199680050299\\
29.95	0.00810149490433368\\
29.96	0.00810099263739857\\
29.97	0.00810048999925085\\
29.98	0.00809998698944291\\
29.99	0.00809948360752627\\
30	0.00809897985305166\\
30.01	0.00809847572556894\\
30.02	0.00809797122462715\\
30.03	0.00809746634977451\\
30.04	0.00809696110055837\\
30.05	0.00809645547652525\\
30.06	0.00809594947722084\\
30.07	0.00809544310218996\\
30.08	0.0080949363509766\\
30.09	0.0080944292231239\\
30.1	0.00809392171817413\\
30.11	0.00809341383566873\\
30.12	0.00809290557514827\\
30.13	0.00809239693615246\\
30.14	0.00809188791822017\\
30.15	0.00809137852088937\\
30.16	0.00809086874369721\\
30.17	0.00809035858617995\\
30.18	0.00808984804787297\\
30.19	0.0080893371283108\\
30.2	0.0080888258270271\\
30.21	0.00808831414355463\\
30.22	0.0080878020774253\\
30.23	0.00808728962817012\\
30.24	0.00808677679531923\\
30.25	0.00808626357840189\\
30.26	0.00808574997694648\\
30.27	0.00808523599048047\\
30.28	0.00808472161853044\\
30.29	0.00808420686062211\\
30.3	0.00808369171628027\\
30.31	0.00808317618502885\\
30.32	0.00808266026639084\\
30.33	0.00808214395988836\\
30.34	0.00808162726504262\\
30.35	0.00808111018137391\\
30.36	0.00808059270840164\\
30.37	0.00808007484564429\\
30.38	0.00807955659261942\\
30.39	0.00807903794884371\\
30.4	0.00807851891383289\\
30.41	0.00807799948710179\\
30.42	0.00807747966816432\\
30.43	0.00807695945653345\\
30.44	0.00807643885172124\\
30.45	0.00807591785323882\\
30.46	0.00807539646059639\\
30.47	0.0080748746733032\\
30.48	0.00807435249086761\\
30.49	0.00807382991279698\\
30.5	0.0080733069385978\\
30.51	0.00807278356777557\\
30.52	0.00807225979983487\\
30.53	0.0080717356342793\\
30.54	0.00807121107061156\\
30.55	0.00807068610833338\\
30.56	0.00807016074694552\\
30.57	0.0080696349859478\\
30.58	0.00806910882483909\\
30.59	0.00806858226311729\\
30.6	0.00806805530027934\\
30.61	0.0080675279358212\\
30.62	0.00806700016923788\\
30.63	0.00806647200002343\\
30.64	0.0080659434276709\\
30.65	0.0080654144516724\\
30.66	0.00806488507151902\\
30.67	0.00806435528670092\\
30.68	0.00806382509670724\\
30.69	0.00806329450102615\\
30.7	0.00806276349914484\\
30.71	0.0080622320905495\\
30.72	0.00806170027472534\\
30.73	0.00806116805115656\\
30.74	0.00806063541932638\\
30.75	0.00806010237871701\\
30.76	0.00805956892880967\\
30.77	0.00805903506908456\\
30.78	0.00805850079902089\\
30.79	0.00805796611809686\\
30.8	0.00805743102578965\\
30.81	0.00805689552157542\\
30.82	0.00805635960492935\\
30.83	0.00805582327532556\\
30.84	0.00805528653223716\\
30.85	0.00805474937513625\\
30.86	0.00805421180349389\\
30.87	0.00805367381678011\\
30.88	0.00805313541446393\\
30.89	0.00805259659601331\\
30.9	0.00805205736089518\\
30.91	0.00805151770857544\\
30.92	0.00805097763851895\\
30.93	0.0080504371501895\\
30.94	0.00804989624304987\\
30.95	0.00804935491656175\\
30.96	0.00804881317018582\\
30.97	0.00804827100338167\\
30.98	0.00804772841560787\\
30.99	0.00804718540632188\\
31	0.00804664197498014\\
31.01	0.008046098121038\\
31.02	0.00804555384394976\\
31.03	0.00804500914316864\\
31.04	0.00804446401814678\\
31.05	0.00804391846833525\\
31.06	0.00804337249318405\\
31.07	0.00804282609214208\\
31.08	0.00804227926465717\\
31.09	0.00804173201017607\\
31.1	0.0080411843281444\\
31.11	0.00804063621800674\\
31.12	0.00804008767920655\\
31.13	0.00803953871118619\\
31.14	0.00803898931338692\\
31.15	0.0080384394852489\\
31.16	0.00803788922621118\\
31.17	0.00803733853571171\\
31.18	0.00803678741318732\\
31.19	0.00803623585807373\\
31.2	0.00803568386980553\\
31.21	0.0080351314478162\\
31.22	0.00803457859153811\\
31.23	0.00803402530040248\\
31.24	0.0080334715738394\\
31.25	0.00803291741127786\\
31.26	0.00803236281214569\\
31.27	0.00803180777586958\\
31.28	0.0080312523018751\\
31.29	0.00803069638958665\\
31.3	0.00803014003842751\\
31.31	0.00802958324781979\\
31.32	0.00802902601718447\\
31.33	0.00802846834594134\\
31.34	0.00802791023350908\\
31.35	0.00802735167930517\\
31.36	0.00802679268274595\\
31.37	0.00802623324324657\\
31.38	0.00802567336022104\\
31.39	0.00802511303308217\\
31.4	0.00802455226124162\\
31.41	0.00802399104410984\\
31.42	0.00802342938109613\\
31.43	0.00802286727160859\\
31.44	0.00802230471505414\\
31.45	0.00802174171083849\\
31.46	0.00802117825836619\\
31.47	0.00802061435704055\\
31.48	0.00802005000626373\\
31.49	0.00801948520543664\\
31.5	0.00801891995395902\\
31.51	0.00801835425122938\\
31.52	0.00801778809664504\\
31.53	0.00801722148960207\\
31.54	0.00801665442949536\\
31.55	0.00801608691571854\\
31.56	0.00801551894766406\\
31.57	0.00801495052472311\\
31.58	0.00801438164628565\\
31.59	0.00801381231174043\\
31.6	0.00801324252047494\\
31.61	0.00801267227187544\\
31.62	0.00801210156532694\\
31.63	0.00801153040021322\\
31.64	0.00801095877591679\\
31.65	0.00801038669181892\\
31.66	0.00800981414729961\\
31.67	0.00800924114173762\\
31.68	0.00800866767451044\\
31.69	0.00800809374499428\\
31.7	0.00800751935256411\\
31.71	0.0080069444965936\\
31.72	0.00800636917645516\\
31.73	0.00800579339151991\\
31.74	0.00800521714115771\\
31.75	0.00800464042473709\\
31.76	0.00800406324162535\\
31.77	0.00800348559118846\\
31.78	0.0080029074727911\\
31.79	0.00800232888579665\\
31.8	0.0080017498295672\\
31.81	0.00800117030346351\\
31.82	0.00800059030684507\\
31.83	0.00800000983907003\\
31.84	0.00799942889949523\\
31.85	0.00799884748747618\\
31.86	0.00799826560236709\\
31.87	0.00799768324352084\\
31.88	0.00799710041028896\\
31.89	0.00799651710202167\\
31.9	0.00799593331806783\\
31.91	0.007995349057775\\
31.92	0.00799476432048935\\
31.93	0.00799417910555574\\
31.94	0.00799359341231767\\
31.95	0.00799300724011726\\
31.96	0.00799242058829532\\
31.97	0.00799183345619126\\
31.98	0.00799124584314316\\
31.99	0.00799065774848769\\
32	0.00799006917156018\\
32.01	0.00798948011169459\\
32.02	0.00798889056822347\\
32.03	0.00798830054047802\\
32.04	0.00798771002778803\\
32.05	0.00798711902948193\\
32.06	0.00798652754488673\\
32.07	0.00798593557332806\\
32.08	0.00798534311413014\\
32.09	0.00798475016661578\\
32.1	0.00798415673010641\\
32.11	0.00798356280392203\\
32.12	0.00798296838738122\\
32.13	0.00798237347980115\\
32.14	0.00798177808049757\\
32.15	0.0079811821887848\\
32.16	0.00798058580397573\\
32.17	0.00797998892538183\\
32.18	0.00797939155231311\\
32.19	0.00797879368407815\\
32.2	0.00797819531998409\\
32.21	0.00797759645933662\\
32.22	0.00797699710143998\\
32.23	0.00797639724559695\\
32.24	0.00797579689110884\\
32.25	0.00797519603727552\\
32.26	0.00797459468339538\\
32.27	0.00797399282876533\\
32.28	0.00797339047268082\\
32.29	0.0079727876144358\\
32.3	0.00797218425332278\\
32.31	0.00797158038863274\\
32.32	0.00797097601965518\\
32.33	0.00797037114567812\\
32.34	0.00796976576598807\\
32.35	0.00796915987987004\\
32.36	0.00796855348660754\\
32.37	0.00796794658548254\\
32.38	0.00796733917577555\\
32.39	0.0079667312567655\\
32.4	0.00796612282772986\\
32.41	0.00796551388794451\\
32.42	0.00796490443668386\\
32.43	0.00796429447322074\\
32.44	0.00796368399682646\\
32.45	0.0079630730067708\\
32.46	0.00796246150232196\\
32.47	0.00796184948274663\\
32.48	0.00796123694730991\\
32.49	0.00796062389527536\\
32.5	0.00796001032590498\\
32.51	0.00795939623845919\\
32.52	0.00795878163219684\\
32.53	0.00795816650637522\\
32.54	0.00795755086025003\\
32.55	0.00795693469307537\\
32.56	0.00795631800410379\\
32.57	0.0079557007925862\\
32.58	0.00795508305777196\\
32.59	0.00795446479890881\\
32.6	0.00795384601524288\\
32.61	0.00795322670601869\\
32.62	0.00795260687047916\\
32.63	0.00795198650786557\\
32.64	0.00795136561741761\\
32.65	0.00795074419837331\\
32.66	0.00795012224996909\\
32.67	0.00794949977143975\\
32.68	0.0079488767620184\\
32.69	0.00794825322093656\\
32.7	0.00794762914742407\\
32.71	0.00794700454070912\\
32.72	0.00794637940001826\\
32.73	0.00794575372457637\\
32.74	0.00794512751360666\\
32.75	0.00794450076633067\\
32.76	0.00794387348196828\\
32.77	0.00794324565973766\\
32.78	0.00794261729885534\\
32.79	0.00794198839853611\\
32.8	0.00794135895799312\\
32.81	0.00794072897643778\\
32.82	0.00794009845307982\\
32.83	0.00793946738712728\\
32.84	0.00793883577778644\\
32.85	0.00793820362426191\\
32.86	0.00793757092575657\\
32.87	0.00793693768147156\\
32.88	0.0079363038906063\\
32.89	0.00793566955235849\\
32.9	0.00793503466592406\\
32.91	0.00793439923049723\\
32.92	0.00793376324527045\\
32.93	0.00793312670943442\\
32.94	0.0079324896221781\\
32.95	0.00793185198268867\\
32.96	0.00793121379015156\\
32.97	0.0079305750437504\\
32.98	0.00792993574266707\\
32.99	0.00792929588608166\\
33	0.00792865547317247\\
33.01	0.00792801450311603\\
33.02	0.00792737297508706\\
33.03	0.00792673088825846\\
33.04	0.00792608824180136\\
33.05	0.00792544503488505\\
33.06	0.00792480126667703\\
33.07	0.00792415693634297\\
33.08	0.00792351204304671\\
33.09	0.00792286658595028\\
33.1	0.00792222056421384\\
33.11	0.00792157397699574\\
33.12	0.00792092682345248\\
33.13	0.0079202791027387\\
33.14	0.00791963081400721\\
33.15	0.00791898195640893\\
33.16	0.00791833252909294\\
33.17	0.00791768253120642\\
33.18	0.00791703196189472\\
33.19	0.00791638082030127\\
33.2	0.00791572910556764\\
33.21	0.0079150768168335\\
33.22	0.00791442395323662\\
33.23	0.00791377051391288\\
33.24	0.00791311649799625\\
33.25	0.0079124619046188\\
33.26	0.00791180673291066\\
33.27	0.00791115098200007\\
33.28	0.00791049465101332\\
33.29	0.00790983773907477\\
33.3	0.00790918024530686\\
33.31	0.00790852216883008\\
33.32	0.00790786350876296\\
33.33	0.00790720426422209\\
33.34	0.00790654443432212\\
33.35	0.00790588401817569\\
33.36	0.00790522301489353\\
33.37	0.00790456142358434\\
33.38	0.00790389924335488\\
33.39	0.00790323647330991\\
33.4	0.0079025731125522\\
33.41	0.00790190916018253\\
33.42	0.00790124461529968\\
33.43	0.00790057947700041\\
33.44	0.00789991374437948\\
33.45	0.00789924741652964\\
33.46	0.0078985804925416\\
33.47	0.00789791297150405\\
33.48	0.00789724485250364\\
33.49	0.007896576134625\\
33.5	0.00789590681695069\\
33.51	0.00789523689856123\\
33.52	0.00789456637853509\\
33.53	0.00789389525594867\\
33.54	0.0078932235298763\\
33.55	0.00789255119939025\\
33.56	0.00789187826356069\\
33.57	0.00789120472145574\\
33.58	0.00789053057214138\\
33.59	0.00788985581468153\\
33.6	0.00788918044813801\\
33.61	0.00788850447157051\\
33.62	0.00788782788403662\\
33.63	0.00788715068459182\\
33.64	0.00788647287228944\\
33.65	0.00788579444618069\\
33.66	0.00788511540531466\\
33.67	0.00788443574873828\\
33.68	0.00788375547549633\\
33.69	0.00788307458463144\\
33.7	0.00788239307518408\\
33.71	0.00788171094619256\\
33.72	0.007881028196693\\
33.73	0.00788034482571935\\
33.74	0.0078796608323034\\
33.75	0.00787897621547471\\
33.76	0.00787829097426066\\
33.77	0.00787760510768643\\
33.78	0.007876918614775\\
33.79	0.00787623149454712\\
33.8	0.00787554374602131\\
33.81	0.0078748553682139\\
33.82	0.00787416636013894\\
33.83	0.00787347672080829\\
33.84	0.00787278644923151\\
33.85	0.00787209554441596\\
33.86	0.00787140400536671\\
33.87	0.00787071183108657\\
33.88	0.00787001902057609\\
33.89	0.00786932557283354\\
33.9	0.00786863148685489\\
33.91	0.00786793676163385\\
33.92	0.00786724139616181\\
33.93	0.00786654538942787\\
33.94	0.00786584874041881\\
33.95	0.0078651514481191\\
33.96	0.0078644535115109\\
33.97	0.00786375492957402\\
33.98	0.00786305570128597\\
33.99	0.00786235582562187\\
34	0.00786165530155454\\
34.01	0.00786095412805441\\
34.02	0.00786025230408957\\
34.03	0.00785954982862573\\
34.04	0.00785884670062625\\
34.05	0.00785814291905207\\
34.06	0.00785743848286178\\
34.07	0.00785673339101156\\
34.08	0.00785602764245519\\
34.09	0.00785532123614404\\
34.1	0.00785461417102708\\
34.11	0.00785390644605084\\
34.12	0.00785319806015943\\
34.13	0.00785248901229454\\
34.14	0.00785177930139539\\
34.15	0.00785106892639878\\
34.16	0.00785035788623904\\
34.17	0.00784964617984805\\
34.18	0.0078489338061552\\
34.19	0.00784822076408743\\
34.2	0.00784750705256918\\
34.21	0.00784679267052242\\
34.22	0.00784607761686658\\
34.23	0.00784536189051864\\
34.24	0.00784464549039304\\
34.25	0.00784392841540172\\
34.26	0.00784321066445407\\
34.27	0.00784249223645697\\
34.28	0.00784177313031476\\
34.29	0.00784105334492921\\
34.3	0.00784033287919957\\
34.31	0.00783961173202252\\
34.32	0.00783888990229216\\
34.33	0.00783816738890003\\
34.34	0.00783744419073506\\
34.35	0.00783672030668364\\
34.36	0.00783599573562952\\
34.37	0.00783527047645386\\
34.38	0.00783454452803521\\
34.39	0.00783381788924952\\
34.4	0.00783309055897007\\
34.41	0.00783236253606756\\
34.42	0.00783163381941\\
34.43	0.00783090440786278\\
34.44	0.00783017430028864\\
34.45	0.00782944349554763\\
34.46	0.00782871199249715\\
34.47	0.00782797978999192\\
34.48	0.00782724688688396\\
34.49	0.00782651328202262\\
34.5	0.00782577897425452\\
34.51	0.00782504396242359\\
34.52	0.00782430824537105\\
34.53	0.00782357182193538\\
34.54	0.00782283469095233\\
34.55	0.00782209685125492\\
34.56	0.00782135830167341\\
34.57	0.00782061904103532\\
34.58	0.00781987906816539\\
34.59	0.0078191383818856\\
34.6	0.00781839698101515\\
34.61	0.00781765486437046\\
34.62	0.00781691203076514\\
34.63	0.00781616847901\\
34.64	0.00781542420791305\\
34.65	0.00781467921627947\\
34.66	0.00781393350291163\\
34.67	0.00781318706660905\\
34.68	0.00781243990616841\\
34.69	0.00781169202038354\\
34.7	0.00781094340804541\\
34.71	0.00781019406794213\\
34.72	0.00780944399885893\\
34.73	0.00780869319957816\\
34.74	0.00780794166887927\\
34.75	0.0078071894055388\\
34.76	0.00780643640833042\\
34.77	0.00780568267602484\\
34.78	0.00780492820738989\\
34.79	0.00780417300119042\\
34.8	0.00780341705618835\\
34.81	0.00780266037114269\\
34.82	0.00780190294480945\\
34.83	0.00780114477594167\\
34.84	0.00780038586328944\\
34.85	0.00779962620559986\\
34.86	0.00779886580161703\\
34.87	0.00779810465008204\\
34.88	0.00779734274973298\\
34.89	0.00779658009930495\\
34.9	0.00779581669752996\\
34.91	0.00779505254313705\\
34.92	0.00779428763485216\\
34.93	0.00779352197139822\\
34.94	0.00779275555149507\\
34.95	0.00779198837385949\\
34.96	0.00779122043720518\\
34.97	0.00779045174024275\\
34.98	0.00778968228167973\\
34.99	0.00778891206022051\\
35	0.00778814107456638\\
35.01	0.00778736932341551\\
35.02	0.00778659680546295\\
35.03	0.00778582351940057\\
35.04	0.00778504946391712\\
35.05	0.00778427463769819\\
35.06	0.00778349903942618\\
35.07	0.00778272266778032\\
35.08	0.00778194552143667\\
35.09	0.00778116759906806\\
35.1	0.00778038889934413\\
35.11	0.00777960942093133\\
35.12	0.00777882916249283\\
35.13	0.00777804812268861\\
35.14	0.00777726630017539\\
35.15	0.00777648369360665\\
35.16	0.00777570030163258\\
35.17	0.00777491612290013\\
35.18	0.00777413115605295\\
35.19	0.0077733453997314\\
35.2	0.00777255885257254\\
35.21	0.00777177151321013\\
35.22	0.00777098338027461\\
35.23	0.00777019445239306\\
35.24	0.00776940472818928\\
35.25	0.00776861420628365\\
35.26	0.00776782288529326\\
35.27	0.00776703076383179\\
35.28	0.00776623784050954\\
35.29	0.00776544411393345\\
35.3	0.00776464958270705\\
35.31	0.00776385424543046\\
35.32	0.00776305810070037\\
35.33	0.00776226114711007\\
35.34	0.00776146338324939\\
35.35	0.00776066480770472\\
35.36	0.00775986541905902\\
35.37	0.00775906521589173\\
35.38	0.00775826419677884\\
35.39	0.00775746236029287\\
35.4	0.00775665970500281\\
35.41	0.00775585622947416\\
35.42	0.0077550519322689\\
35.43	0.00775424681194548\\
35.44	0.00775344086705882\\
35.45	0.00775263409616026\\
35.46	0.00775182649779763\\
35.47	0.00775101807051514\\
35.48	0.00775020881285345\\
35.49	0.00774939872334963\\
35.5	0.00774858780053713\\
35.51	0.00774777604294579\\
35.52	0.00774696344910186\\
35.53	0.00774615001752791\\
35.54	0.0077453357467429\\
35.55	0.00774452063526213\\
35.56	0.00774370468159723\\
35.57	0.00774288788425614\\
35.58	0.00774207024174313\\
35.59	0.00774125175255878\\
35.6	0.00774043241519994\\
35.61	0.00773961222815976\\
35.62	0.00773879118992765\\
35.63	0.00773796929898928\\
35.64	0.00773714655382658\\
35.65	0.00773632295291768\\
35.66	0.00773549849473698\\
35.67	0.00773467317775507\\
35.68	0.00773384700043876\\
35.69	0.00773301996125103\\
35.7	0.00773219205865105\\
35.71	0.00773136329109418\\
35.72	0.0077305336570319\\
35.73	0.00772970315491187\\
35.74	0.00772887178317788\\
35.75	0.00772803954026984\\
35.76	0.00772720642462375\\
35.77	0.00772637243467176\\
35.78	0.00772553756884207\\
35.79	0.00772470182555898\\
35.8	0.00772386520324284\\
35.81	0.00772302770031006\\
35.82	0.00772218931517312\\
35.83	0.00772135004624049\\
35.84	0.00772050989191669\\
35.85	0.00771966885060224\\
35.86	0.00771882692069365\\
35.87	0.00771798410058342\\
35.88	0.00771714038866002\\
35.89	0.00771629578330789\\
35.9	0.00771545028290741\\
35.91	0.00771460388583491\\
35.92	0.00771375659046261\\
35.93	0.00771290839515869\\
35.94	0.0077120592982872\\
35.95	0.00771120929820807\\
35.96	0.00771035839327716\\
35.97	0.00770950658184612\\
35.98	0.00770865386226251\\
35.99	0.00770780054274214\\
36	0.00770694676580507\\
36.01	0.00770609253118292\\
36.02	0.00770523783860718\\
36.03	0.00770438268780913\\
36.04	0.00770352707851992\\
36.05	0.00770267101047053\\
36.06	0.00770181448339179\\
36.07	0.00770095749701434\\
36.08	0.00770010005106869\\
36.09	0.00769924214528518\\
36.1	0.00769838377939396\\
36.11	0.00769752495312506\\
36.12	0.00769666566620831\\
36.13	0.00769580591837341\\
36.14	0.00769494570934986\\
36.15	0.00769408503886703\\
36.16	0.00769322390665412\\
36.17	0.00769236231244015\\
36.18	0.00769150025595399\\
36.19	0.00769063773692434\\
36.2	0.00768977475507973\\
36.21	0.00768891131014856\\
36.22	0.00768804740185902\\
36.23	0.00768718302993916\\
36.24	0.00768631819411687\\
36.25	0.00768545289411985\\
36.26	0.00768458712967566\\
36.27	0.00768372090051169\\
36.28	0.00768285420635514\\
36.29	0.00768198704693308\\
36.3	0.0076811194219724\\
36.31	0.00768025133119981\\
36.32	0.00767938277434188\\
36.33	0.00767851375112498\\
36.34	0.00767764426127535\\
36.35	0.00767677430451904\\
36.36	0.00767590388058194\\
36.37	0.00767503298918976\\
36.38	0.00767416163006807\\
36.39	0.00767328980294225\\
36.4	0.00767241750753751\\
36.41	0.00767154474357892\\
36.42	0.00767067151079134\\
36.43	0.00766979780889949\\
36.44	0.00766892363762792\\
36.45	0.007668048996701\\
36.46	0.00766717388584293\\
36.47	0.00766629830477777\\
36.48	0.00766542225322937\\
36.49	0.00766454573092143\\
36.5	0.00766366873757749\\
36.51	0.00766279127292089\\
36.52	0.00766191333667483\\
36.53	0.00766103492856234\\
36.54	0.00766015604830624\\
36.55	0.00765927669562923\\
36.56	0.0076583968702538\\
36.57	0.00765751657190228\\
36.58	0.00765663580029684\\
36.59	0.00765575455515947\\
36.6	0.00765487283621198\\
36.61	0.00765399064317603\\
36.62	0.00765310797577308\\
36.63	0.00765222483372444\\
36.64	0.00765134121675123\\
36.65	0.00765045712457441\\
36.66	0.00764957255691475\\
36.67	0.00764868751349287\\
36.68	0.0076478019940292\\
36.69	0.00764691599824399\\
36.7	0.00764602952585734\\
36.71	0.00764514257658916\\
36.72	0.00764425515015917\\
36.73	0.00764336724628695\\
36.74	0.00764247886469187\\
36.75	0.00764159000509315\\
36.76	0.00764070066720982\\
36.77	0.00763981085076074\\
36.78	0.00763892055546459\\
36.79	0.00763802978103988\\
36.8	0.00763713852720494\\
36.81	0.00763624679367792\\
36.82	0.00763535458017679\\
36.83	0.00763446188641937\\
36.84	0.00763356871212325\\
36.85	0.00763267505700589\\
36.86	0.00763178092078455\\
36.87	0.00763088630317632\\
36.88	0.0076299912038981\\
36.89	0.00762909562266663\\
36.9	0.00762819955919845\\
36.91	0.00762730301320993\\
36.92	0.00762640598441726\\
36.93	0.00762550847253645\\
36.94	0.00762461047728334\\
36.95	0.00762371199837356\\
36.96	0.00762281303552261\\
36.97	0.00762191358844575\\
36.98	0.00762101365685809\\
36.99	0.00762011324047457\\
37	0.00761921233900993\\
37.01	0.00761831095217872\\
37.02	0.00761740907969533\\
37.03	0.00761650672127395\\
37.04	0.0076156038766286\\
37.05	0.00761470054547311\\
37.06	0.00761379672752113\\
37.07	0.00761289242248611\\
37.08	0.00761198763008134\\
37.09	0.00761108235001992\\
37.1	0.00761017658201475\\
37.11	0.00760927032577857\\
37.12	0.0076083635810239\\
37.13	0.00760745634746312\\
37.14	0.00760654862480839\\
37.15	0.00760564041277168\\
37.16	0.00760473171106482\\
37.17	0.00760382251939939\\
37.18	0.00760291283748683\\
37.19	0.00760200266503838\\
37.2	0.00760109200176508\\
37.21	0.0076001808473778\\
37.22	0.00759926920158721\\
37.23	0.0075983570641038\\
37.24	0.00759744443463786\\
37.25	0.00759653131289952\\
37.26	0.00759561769859867\\
37.27	0.00759470359144506\\
37.28	0.00759378899114822\\
37.29	0.00759287389741751\\
37.3	0.00759195830996208\\
37.31	0.0075910422284909\\
37.32	0.00759012565271275\\
37.33	0.00758920858233622\\
37.34	0.00758829101706969\\
37.35	0.00758737295662138\\
37.36	0.00758645440069929\\
37.37	0.00758553534901124\\
37.38	0.00758461580126486\\
37.39	0.00758369575716757\\
37.4	0.00758277521642661\\
37.41	0.00758185417874904\\
37.42	0.00758093264384168\\
37.43	0.00758001061141121\\
37.44	0.00757908808116407\\
37.45	0.00757816505280654\\
37.46	0.00757724152604467\\
37.47	0.00757631750058435\\
37.48	0.00757539297613125\\
37.49	0.00757446795239086\\
37.5	0.00757354242906844\\
37.51	0.00757261640586909\\
37.52	0.00757168988249769\\
37.53	0.00757076285865893\\
37.54	0.00756983533405731\\
37.55	0.00756890730839711\\
37.56	0.00756797878138243\\
37.57	0.00756704975271716\\
37.58	0.00756612022210499\\
37.59	0.00756519018924942\\
37.6	0.00756425965385374\\
37.61	0.00756332861562104\\
37.62	0.00756239707425421\\
37.63	0.00756146502945594\\
37.64	0.00756053248092872\\
37.65	0.00755959942837482\\
37.66	0.00755866587149634\\
37.67	0.00755773180999514\\
37.68	0.0075567972435729\\
37.69	0.0075558621719311\\
37.7	0.00755492659477099\\
37.71	0.00755399051179365\\
37.72	0.00755305392269992\\
37.73	0.00755211682719045\\
37.74	0.00755117922496569\\
37.75	0.00755024111572589\\
37.76	0.00754930249917106\\
37.77	0.00754836337500104\\
37.78	0.00754742374291544\\
37.79	0.00754648360261368\\
37.8	0.00754554295379494\\
37.81	0.00754460179615822\\
37.82	0.00754366012940231\\
37.83	0.00754271795322578\\
37.84	0.007541775267327\\
37.85	0.00754083207140411\\
37.86	0.00753988836515506\\
37.87	0.00753894414827759\\
37.88	0.0075379994204692\\
37.89	0.00753705418142721\\
37.9	0.00753610843084871\\
37.91	0.00753516216843059\\
37.92	0.00753421539386951\\
37.93	0.00753326810686193\\
37.94	0.00753232030710408\\
37.95	0.00753137199429201\\
37.96	0.00753042316812151\\
37.97	0.00752947382828818\\
37.98	0.0075285239744874\\
37.99	0.00752757360641435\\
38	0.00752662272376396\\
38.01	0.00752567132623096\\
38.02	0.00752471941350987\\
38.03	0.00752376698529497\\
38.04	0.00752281404128034\\
38.05	0.00752186058115985\\
38.06	0.00752090660462711\\
38.07	0.00751995211137556\\
38.08	0.00751899710109838\\
38.09	0.00751804157348855\\
38.1	0.00751708552823883\\
38.11	0.00751612896504174\\
38.12	0.00751517188358959\\
38.13	0.00751421428357448\\
38.14	0.00751325616468825\\
38.15	0.00751229752662256\\
38.16	0.00751133836906881\\
38.17	0.0075103786917182\\
38.18	0.00750941849426168\\
38.19	0.00750845777639\\
38.2	0.00750749653779367\\
38.21	0.00750653477816298\\
38.22	0.00750557249718798\\
38.23	0.0075046096945585\\
38.24	0.00750364636996415\\
38.25	0.0075026825230943\\
38.26	0.00750171815363808\\
38.27	0.00750075326128443\\
38.28	0.00749978784572201\\
38.29	0.00749882190663928\\
38.3	0.00749785544372447\\
38.31	0.00749688845666555\\
38.32	0.0074959209451503\\
38.33	0.00749495290886623\\
38.34	0.00749398434750064\\
38.35	0.00749301526074057\\
38.36	0.00749204564827285\\
38.37	0.00749107550978407\\
38.38	0.00749010484496057\\
38.39	0.00748913365348848\\
38.4	0.00748816193505367\\
38.41	0.00748718968934177\\
38.42	0.0074862169160382\\
38.43	0.00748524361482811\\
38.44	0.00748426978539643\\
38.45	0.00748329542742785\\
38.46	0.00748232054060681\\
38.47	0.00748134512461751\\
38.48	0.00748036917914392\\
38.49	0.00747939270386976\\
38.5	0.00747841569847851\\
38.51	0.0074774381626534\\
38.52	0.00747646009607742\\
38.53	0.00747548149843332\\
38.54	0.00747450236940361\\
38.55	0.00747352270867054\\
38.56	0.00747254251591613\\
38.57	0.00747156179082213\\
38.58	0.00747058053307007\\
38.59	0.00746959874234121\\
38.6	0.00746861641831658\\
38.61	0.00746763356067695\\
38.62	0.00746665016910283\\
38.63	0.0074656662432745\\
38.64	0.00746468178287199\\
38.65	0.00746369678757506\\
38.66	0.00746271125706323\\
38.67	0.00746172519101576\\
38.68	0.00746073858911167\\
38.69	0.00745975145102972\\
38.7	0.00745876377644839\\
38.71	0.00745777556504595\\
38.72	0.00745678681650038\\
38.73	0.00745579753048941\\
38.74	0.00745480770669053\\
38.75	0.00745381734478094\\
38.76	0.00745282644443762\\
38.77	0.00745183500533725\\
38.78	0.00745084302715628\\
38.79	0.0074498505095709\\
38.8	0.00744885745225702\\
38.81	0.00744786385489029\\
38.82	0.00744686971714611\\
38.83	0.00744587503869961\\
38.84	0.00744487981922566\\
38.85	0.00744388405839885\\
38.86	0.00744288775589353\\
38.87	0.00744189091138375\\
38.88	0.00744089352454333\\
38.89	0.0074398955950458\\
38.9	0.00743889712256443\\
38.91	0.0074378981067722\\
38.92	0.00743689854734186\\
38.93	0.00743589844394585\\
38.94	0.00743489779625636\\
38.95	0.00743389660394531\\
38.96	0.00743289486668435\\
38.97	0.00743189258414482\\
38.98	0.00743088975599783\\
38.99	0.0074298863819142\\
39	0.00742888246156447\\
39.01	0.00742787799461891\\
39.02	0.0074268729807475\\
39.03	0.00742586741961995\\
39.04	0.00742486131090571\\
39.05	0.00742385465427392\\
39.06	0.00742284744939344\\
39.07	0.00742183969593289\\
39.08	0.00742083139356056\\
39.09	0.00741982254194448\\
39.1	0.00741881314075239\\
39.11	0.00741780318965176\\
39.12	0.00741679268830976\\
39.13	0.00741578163639327\\
39.14	0.00741477003356889\\
39.15	0.00741375787950295\\
39.16	0.00741274517386147\\
39.17	0.00741173191631017\\
39.18	0.00741071810651452\\
39.19	0.00740970374413966\\
39.2	0.00740868882885047\\
39.21	0.0074076733603115\\
39.22	0.00740665733818705\\
39.23	0.00740564076214109\\
39.24	0.00740462363183731\\
39.25	0.00740360594693912\\
39.26	0.0074025877071096\\
39.27	0.00740156891201156\\
39.28	0.00740054956130749\\
39.29	0.00739952965465961\\
39.3	0.0073985091917298\\
39.31	0.00739748817217967\\
39.32	0.00739646659567052\\
39.33	0.00739544446186334\\
39.34	0.00739442177041883\\
39.35	0.00739339852099737\\
39.36	0.00739237471325905\\
39.37	0.00739135034686364\\
39.38	0.00739032542147061\\
39.39	0.00738929993673911\\
39.4	0.00738827389232801\\
39.41	0.00738724728789583\\
39.42	0.00738622012310081\\
39.43	0.00738519239760087\\
39.44	0.00738416411105361\\
39.45	0.00738313526311632\\
39.46	0.00738210585344598\\
39.47	0.00738107588169924\\
39.48	0.00738004534753245\\
39.49	0.00737901425060163\\
39.5	0.0073779825905625\\
39.51	0.00737695036707044\\
39.52	0.00737591757978051\\
39.53	0.00737488422834746\\
39.54	0.00737385031242571\\
39.55	0.00737281583166936\\
39.56	0.00737178078573219\\
39.57	0.00737074517426763\\
39.58	0.00736970899692882\\
39.59	0.00736867225336855\\
39.6	0.00736763494323928\\
39.61	0.00736659706619315\\
39.62	0.00736555862188195\\
39.63	0.00736451960995717\\
39.64	0.00736348003006995\\
39.65	0.00736243988187108\\
39.66	0.00736139916501105\\
39.67	0.00736035787913998\\
39.68	0.00735931602390767\\
39.69	0.00735827359896359\\
39.7	0.00735723060395686\\
39.71	0.00735618703853625\\
39.72	0.0073551429023502\\
39.73	0.00735409819504681\\
39.74	0.00735305291627384\\
39.75	0.00735200706567868\\
39.76	0.00735096064290841\\
39.77	0.00734991364760973\\
39.78	0.00734886607942902\\
39.79	0.00734781793801229\\
39.8	0.00734676922300522\\
39.81	0.00734571993405311\\
39.82	0.00734467007080093\\
39.83	0.0073436196328933\\
39.84	0.00734256861997446\\
39.85	0.00734151703168832\\
39.86	0.00734046486767841\\
39.87	0.00733941212758793\\
39.88	0.0073383588110597\\
39.89	0.00733730491773617\\
39.9	0.00733625044725946\\
39.91	0.00733519539927129\\
39.92	0.00733413977341304\\
39.93	0.00733308356932572\\
39.94	0.00733202678664998\\
39.95	0.00733096942502609\\
39.96	0.00732991148409393\\
39.97	0.00732885296349307\\
39.98	0.00732779386286265\\
39.99	0.00732673418184146\\
40	0.00732567392006792\\
40.01	0.00732461307718008\\
};
\addplot [color=green,dashed,forget plot]
  table[row sep=crcr]{%
40.01	0.00732461307718008\\
40.02	0.00732355165281558\\
40.03	0.00732248964661172\\
40.04	0.0073214270582054\\
40.05	0.00732036388723315\\
40.06	0.0073193001333311\\
40.07	0.00731823579613502\\
40.08	0.00731717087528029\\
40.09	0.00731610537040189\\
40.1	0.00731503928113443\\
40.11	0.00731397260711211\\
40.12	0.00731290534796878\\
40.13	0.00731183750333785\\
40.14	0.00731076907285237\\
40.15	0.00730970005614499\\
40.16	0.00730863045284795\\
40.17	0.00730756026259312\\
40.18	0.00730648948501195\\
40.19	0.0073054181197355\\
40.2	0.00730434616639442\\
40.21	0.00730327362461898\\
40.22	0.00730220049403902\\
40.23	0.007301126774284\\
40.24	0.00730005246498294\\
40.25	0.0072989775657645\\
40.26	0.00729790207625688\\
40.27	0.0072968259960879\\
40.28	0.00729574932488497\\
40.29	0.00729467206227508\\
40.3	0.00729359420788479\\
40.31	0.00729251576134026\\
40.32	0.00729143672226724\\
40.33	0.00729035709029104\\
40.34	0.00728927686503656\\
40.35	0.00728819604612827\\
40.36	0.00728711463319024\\
40.37	0.00728603262584608\\
40.38	0.007284950023719\\
40.39	0.00728386682643177\\
40.4	0.00728278303360674\\
40.41	0.0072816986448658\\
40.42	0.00728061365983045\\
40.43	0.00727952807812173\\
40.44	0.00727844189936023\\
40.45	0.00727735512316614\\
40.46	0.00727626774915918\\
40.47	0.00727517977695864\\
40.48	0.00727409120618336\\
40.49	0.00727300203645175\\
40.5	0.00727191226738178\\
40.51	0.00727082189859094\\
40.52	0.0072697309296963\\
40.53	0.00726863936031447\\
40.54	0.00726754719006161\\
40.55	0.00726645441855341\\
40.56	0.00726536104540514\\
40.57	0.00726426707023159\\
40.58	0.00726317249264708\\
40.59	0.00726207731226549\\
40.6	0.00726098152870023\\
40.61	0.00725988514156424\\
40.62	0.00725878815047001\\
40.63	0.00725769055502954\\
40.64	0.00725659235485439\\
40.65	0.00725549354955562\\
40.66	0.00725439413874383\\
40.67	0.00725329412202917\\
40.68	0.00725219349902126\\
40.69	0.00725109226932929\\
40.7	0.00724999043256194\\
40.71	0.00724888798832744\\
40.72	0.0072477849362335\\
40.73	0.00724668127588737\\
40.74	0.0072455770068958\\
40.75	0.00724447212886507\\
40.76	0.00724336664140095\\
40.77	0.00724226054410872\\
40.78	0.00724115383659318\\
40.79	0.00724004651845861\\
40.8	0.00723893858930882\\
40.81	0.0072378300487471\\
40.82	0.00723672089637625\\
40.83	0.00723561113179855\\
40.84	0.00723450075461579\\
40.85	0.00723338976442926\\
40.86	0.00723227816083971\\
40.87	0.0072311659434474\\
40.88	0.00723005311185209\\
40.89	0.00722893966565299\\
40.9	0.00722782560444882\\
40.91	0.00722671092783777\\
40.92	0.00722559563541751\\
40.93	0.0072244797267852\\
40.94	0.00722336320153745\\
40.95	0.00722224605927036\\
40.96	0.00722112829957949\\
40.97	0.00722000992205988\\
40.98	0.00721889092630603\\
40.99	0.00721777131191191\\
41	0.00721665107847095\\
41.01	0.00721553022557604\\
41.02	0.00721440875281952\\
41.03	0.00721328665979319\\
41.04	0.00721216394608833\\
41.05	0.00721104061129563\\
41.06	0.00720991665500526\\
41.07	0.00720879207680682\\
41.08	0.00720766687628937\\
41.09	0.00720654105304141\\
41.1	0.00720541460665088\\
41.11	0.00720428753670514\\
41.12	0.00720315984279103\\
41.13	0.00720203152449479\\
41.14	0.00720090258140209\\
41.15	0.00719977301309806\\
41.16	0.00719864281916723\\
41.17	0.00719751199919357\\
41.18	0.00719638055276047\\
41.19	0.00719524847945073\\
41.2	0.0071941157788466\\
41.21	0.00719298245052971\\
41.22	0.00719184849408113\\
41.23	0.00719071390908133\\
41.24	0.0071895786951102\\
41.25	0.00718844285174703\\
41.26	0.00718730637857049\\
41.27	0.00718616927515871\\
41.28	0.00718503154108918\\
41.29	0.00718389317593878\\
41.3	0.00718275417928383\\
41.31	0.00718161455069999\\
41.32	0.00718047428976235\\
41.33	0.00717933339604538\\
41.34	0.00717819186912292\\
41.35	0.0071770497085682\\
41.36	0.00717590691395385\\
41.37	0.00717476348485184\\
41.38	0.00717361942083356\\
41.39	0.00717247472146974\\
41.4	0.00717132938633049\\
41.41	0.00717018341498532\\
41.42	0.00716903680700304\\
41.43	0.00716788956195188\\
41.44	0.00716674167939941\\
41.45	0.00716559315891256\\
41.46	0.00716444400005762\\
41.47	0.00716329420240022\\
41.48	0.00716214376550534\\
41.49	0.00716099268893734\\
41.5	0.00715984097225989\\
41.51	0.00715868861503601\\
41.52	0.00715753561682807\\
41.53	0.00715638197719777\\
41.54	0.00715522769570613\\
41.55	0.00715407277191354\\
41.56	0.00715291720537968\\
41.57	0.00715176099566358\\
41.58	0.00715060414232358\\
41.59	0.00714944664491735\\
41.6	0.00714828850300187\\
41.61	0.00714712971613343\\
41.62	0.00714597028386767\\
41.63	0.00714481020575948\\
41.64	0.00714364948136311\\
41.65	0.00714248811023207\\
41.66	0.00714132609191921\\
41.67	0.00714016342597666\\
41.68	0.00713900011195583\\
41.69	0.00713783614940745\\
41.7	0.00713667153788153\\
41.71	0.00713550627692736\\
41.72	0.00713434036609351\\
41.73	0.00713317380492785\\
41.74	0.00713200659297751\\
41.75	0.0071308387297889\\
41.76	0.0071296702149077\\
41.77	0.00712850104787886\\
41.78	0.0071273312282466\\
41.79	0.00712616075555439\\
41.8	0.00712498962934498\\
41.81	0.00712381784916035\\
41.82	0.00712264541454176\\
41.83	0.0071214723250297\\
41.84	0.00712029858016392\\
41.85	0.00711912417948341\\
41.86	0.00711794912252639\\
41.87	0.00711677340883033\\
41.88	0.00711559703793193\\
41.89	0.00711442000936713\\
41.9	0.00711324232267108\\
41.91	0.00711206397737819\\
41.92	0.00711088497302204\\
41.93	0.00710970530913546\\
41.94	0.00710852498525051\\
41.95	0.00710734400089842\\
41.96	0.00710616235560965\\
41.97	0.00710498004891389\\
41.98	0.00710379708033997\\
41.99	0.00710261344941599\\
42	0.00710142915566919\\
42.01	0.00710024419862604\\
42.02	0.00709905857781217\\
42.03	0.0070978722927524\\
42.04	0.00709668534297074\\
42.05	0.00709549772799038\\
42.06	0.00709430944733368\\
42.07	0.00709312050052216\\
42.08	0.00709193088707653\\
42.09	0.00709074060651664\\
42.1	0.00708954965836153\\
42.11	0.00708835804212936\\
42.12	0.00708716575733747\\
42.13	0.00708597280350235\\
42.14	0.00708477918013961\\
42.15	0.00708358488676402\\
42.16	0.00708238992288951\\
42.17	0.00708119428802912\\
42.18	0.007079997981695\\
42.19	0.00707880100339849\\
42.2	0.00707760335264999\\
42.21	0.00707640502895906\\
42.22	0.00707520603183436\\
42.23	0.00707400636078368\\
42.24	0.00707280601531388\\
42.25	0.00707160499493099\\
42.26	0.00707040329914007\\
42.27	0.00706920092744533\\
42.28	0.00706799787935004\\
42.29	0.00706679415435658\\
42.3	0.00706558975196641\\
42.31	0.00706438467168007\\
42.32	0.0070631789129972\\
42.33	0.00706197247541646\\
42.34	0.00706076535843565\\
42.35	0.00705955756155157\\
42.36	0.00705834908426013\\
42.37	0.00705713992605628\\
42.38	0.00705593008643402\\
42.39	0.0070547195648864\\
42.4	0.00705350836090554\\
42.41	0.00705229647398256\\
42.42	0.00705108390360766\\
42.43	0.00704987064927004\\
42.44	0.00704865671045795\\
42.45	0.00704744208665866\\
42.46	0.00704622677735846\\
42.47	0.00704501078204265\\
42.48	0.00704379410019555\\
42.49	0.0070425767313005\\
42.5	0.00704135867483981\\
42.51	0.00704013993029482\\
42.52	0.00703892049714586\\
42.53	0.00703770037487225\\
42.54	0.0070364795629523\\
42.55	0.00703525806086327\\
42.56	0.00703403586808144\\
42.57	0.00703281298408205\\
42.58	0.0070315894083393\\
42.59	0.00703036514032636\\
42.6	0.00702914017951536\\
42.61	0.00702791452537739\\
42.62	0.00702668817738248\\
42.63	0.00702546113499962\\
42.64	0.00702423339769673\\
42.65	0.00702300496494067\\
42.66	0.00702177583619724\\
42.67	0.00702054601093116\\
42.68	0.00701931548860607\\
42.69	0.00701808426868454\\
42.7	0.00701685235062805\\
42.71	0.00701561973389697\\
42.72	0.00701438641795061\\
42.73	0.00701315240224715\\
42.74	0.00701191768624367\\
42.75	0.00701068226939616\\
42.76	0.00700944615115946\\
42.77	0.00700820933098732\\
42.78	0.00700697180833236\\
42.79	0.00700573358264606\\
42.8	0.00700449465337876\\
42.81	0.00700325501997969\\
42.82	0.0070020146818969\\
42.83	0.00700077363857731\\
42.84	0.00699953188946667\\
42.85	0.00699828943400961\\
42.86	0.00699704627164954\\
42.87	0.00699580240182873\\
42.88	0.00699455782398829\\
42.89	0.00699331253756812\\
42.9	0.00699206654200694\\
42.91	0.00699081983674229\\
42.92	0.00698957242121051\\
42.93	0.00698832429484674\\
42.94	0.00698707545708491\\
42.95	0.00698582590735773\\
42.96	0.00698457564509671\\
42.97	0.00698332466973214\\
42.98	0.00698207298069306\\
42.99	0.00698082057740728\\
43	0.0069795674593014\\
43.01	0.00697831362580073\\
43.02	0.00697705907632938\\
43.03	0.00697580381031017\\
43.04	0.00697454782716467\\
43.05	0.00697329112631318\\
43.06	0.00697203370717475\\
43.07	0.00697077556916711\\
43.08	0.00696951671170674\\
43.09	0.00696825713420882\\
43.1	0.00696699683608724\\
43.11	0.00696573581675458\\
43.12	0.00696447407562213\\
43.13	0.00696321161209986\\
43.14	0.00696194842559642\\
43.15	0.00696068451551912\\
43.16	0.00695941988127397\\
43.17	0.00695815452226563\\
43.18	0.00695688843789743\\
43.19	0.00695562162757134\\
43.2	0.00695435409068798\\
43.21	0.0069530858266466\\
43.22	0.00695181683484512\\
43.23	0.00695054711468005\\
43.24	0.00694927666554653\\
43.25	0.00694800548683834\\
43.26	0.00694673357794785\\
43.27	0.00694546093826604\\
43.28	0.00694418756718247\\
43.29	0.00694291346408534\\
43.3	0.00694163862836137\\
43.31	0.00694036305939592\\
43.32	0.00693908675657288\\
43.33	0.00693780971927473\\
43.34	0.00693653194688249\\
43.35	0.00693525343877576\\
43.36	0.00693397419433267\\
43.37	0.00693269421292988\\
43.38	0.00693141349394262\\
43.39	0.0069301320367446\\
43.4	0.0069288498407081\\
43.41	0.00692756690520387\\
43.42	0.00692628322960121\\
43.43	0.00692499881326789\\
43.44	0.00692371365557017\\
43.45	0.00692242775587284\\
43.46	0.00692114111353912\\
43.47	0.00691985372793074\\
43.48	0.00691856559840788\\
43.49	0.00691727672432919\\
43.5	0.00691598710505176\\
43.51	0.00691469673993114\\
43.52	0.00691340562832132\\
43.53	0.00691211376957471\\
43.54	0.00691082116304216\\
43.55	0.00690952780807292\\
43.56	0.00690823370401468\\
43.57	0.00690693885021351\\
43.58	0.00690564324601388\\
43.59	0.00690434689075867\\
43.6	0.00690304978378911\\
43.61	0.00690175192444485\\
43.62	0.00690045331206386\\
43.63	0.0068991539459825\\
43.64	0.00689785382553548\\
43.65	0.00689655295005586\\
43.66	0.00689525131885908\\
43.67	0.00689394893125086\\
43.68	0.00689264578653586\\
43.69	0.00689134188401765\\
43.7	0.00689003722299874\\
43.71	0.00688873180278053\\
43.72	0.00688742562266336\\
43.73	0.00688611868194645\\
43.74	0.00688481097992796\\
43.75	0.00688350251590494\\
43.76	0.00688219328917336\\
43.77	0.00688088329902807\\
43.78	0.00687957254476283\\
43.79	0.00687826102567032\\
43.8	0.00687694874104206\\
43.81	0.00687563569016852\\
43.82	0.00687432187233902\\
43.83	0.00687300728684177\\
43.84	0.00687169193296389\\
43.85	0.00687037580999136\\
43.86	0.00686905891720903\\
43.87	0.00686774125390064\\
43.88	0.00686642281934881\\
43.89	0.00686510361283501\\
43.9	0.00686378363363959\\
43.91	0.00686246288104177\\
43.92	0.00686114135431961\\
43.93	0.00685981905275006\\
43.94	0.00685849597560891\\
43.95	0.0068571721221708\\
43.96	0.00685584749170923\\
43.97	0.00685452208349655\\
43.98	0.00685319589680396\\
43.99	0.00685186893090148\\
44	0.00685054118505799\\
44.01	0.00684921265854122\\
44.02	0.0068478833506177\\
44.03	0.00684655326055282\\
44.04	0.00684522238761078\\
44.05	0.00684389073105464\\
44.06	0.00684255829014625\\
44.07	0.00684122506414629\\
44.08	0.00683989105231426\\
44.09	0.00683855625390848\\
44.1	0.00683722066818608\\
44.11	0.006835884294403\\
44.12	0.00683454713181397\\
44.13	0.00683320917967256\\
44.14	0.00683187043723111\\
44.15	0.00683053090374077\\
44.16	0.00682919057845147\\
44.17	0.00682784946061197\\
44.18	0.00682650754946978\\
44.19	0.00682516484427122\\
44.2	0.00682382134426138\\
44.21	0.00682247704868414\\
44.22	0.00682113195678214\\
44.23	0.00681978606779683\\
44.24	0.00681843938096841\\
44.25	0.00681709189553583\\
44.26	0.00681574361073684\\
44.27	0.00681439452580794\\
44.28	0.00681304463998438\\
44.29	0.00681169395250019\\
44.3	0.00681034246258813\\
44.31	0.00680899016947971\\
44.32	0.00680763707240521\\
44.33	0.00680628317059364\\
44.34	0.00680492846327276\\
44.35	0.00680357294966905\\
44.36	0.00680221662900774\\
44.37	0.0068008595005128\\
44.38	0.00679950156340691\\
44.39	0.0067981428169115\\
44.4	0.00679678326024669\\
44.41	0.00679542289263136\\
44.42	0.00679406171328308\\
44.43	0.00679269972141813\\
44.44	0.00679133691625152\\
44.45	0.00678997329699695\\
44.46	0.00678860886286684\\
44.47	0.00678724361307229\\
44.48	0.00678587754682313\\
44.49	0.00678451066332786\\
44.5	0.00678314296179367\\
44.51	0.00678177444142644\\
44.52	0.00678040510143076\\
44.53	0.00677903494100987\\
44.54	0.0067776639593657\\
44.55	0.00677629215569885\\
44.56	0.00677491952920861\\
44.57	0.00677354607909293\\
44.58	0.0067721718045484\\
44.59	0.00677079670477031\\
44.6	0.00676942077895258\\
44.61	0.00676804402628781\\
44.62	0.00676666644596724\\
44.63	0.00676528803718075\\
44.64	0.00676390879911688\\
44.65	0.00676252873096281\\
44.66	0.00676114783190434\\
44.67	0.00675976610112593\\
44.68	0.00675838353781065\\
44.69	0.00675700014114023\\
44.7	0.00675561591029498\\
44.71	0.00675423084445387\\
44.72	0.00675284494279448\\
44.73	0.00675145820449298\\
44.74	0.00675007062872419\\
44.75	0.0067486822146615\\
44.76	0.00674729296147694\\
44.77	0.00674590286834111\\
44.78	0.00674451193442323\\
44.79	0.00674312015889109\\
44.8	0.0067417275409111\\
44.81	0.00674033407964822\\
44.82	0.00673893977426603\\
44.83	0.00673754462392667\\
44.84	0.00673614862779086\\
44.85	0.00673475178501788\\
44.86	0.0067333540947656\\
44.87	0.00673195555619043\\
44.88	0.00673055616844738\\
44.89	0.00672915593068998\\
44.9	0.00672775484207033\\
44.91	0.00672635290173907\\
44.92	0.00672495010884541\\
44.93	0.00672354646253709\\
44.94	0.00672214196196038\\
44.95	0.00672073660626008\\
44.96	0.00671933039457957\\
44.97	0.00671792332606069\\
44.98	0.00671651539984387\\
44.99	0.006715106615068\\
45	0.00671369697087054\\
45.01	0.00671228646638744\\
45.02	0.00671087510075314\\
45.03	0.00670946287310063\\
45.04	0.00670804978256136\\
45.05	0.0067066358282653\\
45.06	0.00670522100934092\\
45.07	0.00670380532491515\\
45.08	0.00670238877411345\\
45.09	0.00670097135605972\\
45.1	0.00669955306987637\\
45.11	0.00669813391468426\\
45.12	0.00669671388960275\\
45.13	0.00669529299374964\\
45.14	0.00669387122624121\\
45.15	0.00669244858619218\\
45.16	0.00669102507271576\\
45.17	0.00668960068492358\\
45.18	0.00668817542192572\\
45.19	0.00668674928283072\\
45.2	0.00668532226674555\\
45.21	0.00668389437277561\\
45.22	0.00668246560002472\\
45.23	0.00668103594759517\\
45.24	0.00667960541458763\\
45.25	0.00667817400010121\\
45.26	0.00667674170323344\\
45.27	0.00667530852308022\\
45.28	0.00667387445873592\\
45.29	0.00667243950929326\\
45.3	0.00667100367384339\\
45.31	0.00666956695147583\\
45.32	0.0066681293412785\\
45.33	0.00666669084233773\\
45.34	0.00666525145373818\\
45.35	0.00666381117456293\\
45.36	0.00666237000389343\\
45.37	0.00666092794080948\\
45.38	0.00665948498438925\\
45.39	0.00665804113370928\\
45.4	0.00665659638784447\\
45.41	0.00665515074586804\\
45.42	0.0066537042068516\\
45.43	0.00665225676986506\\
45.44	0.00665080843397673\\
45.45	0.00664935919825318\\
45.46	0.00664790906175936\\
45.47	0.00664645802355854\\
45.48	0.00664500608271229\\
45.49	0.00664355323828053\\
45.5	0.00664209948932144\\
45.51	0.00664064483489158\\
45.52	0.00663918927404576\\
45.53	0.00663773280583709\\
45.54	0.00663627542931701\\
45.55	0.00663481714353522\\
45.56	0.00663335794753972\\
45.57	0.00663189784037679\\
45.58	0.00663043682109097\\
45.59	0.0066289748887251\\
45.6	0.00662751204232026\\
45.61	0.00662604828091582\\
45.62	0.00662458360354938\\
45.63	0.00662311800925681\\
45.64	0.00662165149707224\\
45.65	0.00662018406602802\\
45.66	0.00661871571515475\\
45.67	0.00661724644348128\\
45.68	0.00661577625003467\\
45.69	0.0066143051338402\\
45.7	0.0066128330939214\\
45.71	0.00661136012929999\\
45.72	0.00660988623899592\\
45.73	0.00660841142202734\\
45.74	0.0066069356774106\\
45.75	0.00660545900416024\\
45.76	0.006603981401289\\
45.77	0.00660250286780781\\
45.78	0.00660102340272578\\
45.79	0.00659954300505021\\
45.8	0.00659806167378655\\
45.81	0.00659657940793843\\
45.82	0.00659509620650763\\
45.83	0.00659361206849413\\
45.84	0.00659212699289601\\
45.85	0.00659064097870953\\
45.86	0.00658915402492908\\
45.87	0.00658766613054721\\
45.88	0.00658617729455456\\
45.89	0.00658468751593995\\
45.9	0.00658319679369029\\
45.91	0.00658170512679061\\
45.92	0.00658021251422408\\
45.93	0.00657871895497193\\
45.94	0.00657722444801355\\
45.95	0.00657572899232639\\
45.96	0.00657423258688599\\
45.97	0.00657273523066599\\
45.98	0.00657123692263812\\
45.99	0.00656973766177218\\
46	0.00656823744703603\\
46.01	0.00656673627739562\\
46.02	0.00656523415181494\\
46.03	0.00656373106925605\\
46.04	0.00656222702867905\\
46.05	0.00656072202904209\\
46.06	0.00655921606930138\\
46.07	0.00655770914841112\\
46.08	0.00655620126532358\\
46.09	0.00655469241898904\\
46.1	0.0065531826083558\\
46.11	0.00655167183237016\\
46.12	0.00655016008997646\\
46.13	0.00654864738011702\\
46.14	0.00654713370173215\\
46.15	0.00654561905376017\\
46.16	0.00654410343513737\\
46.17	0.00654258684479803\\
46.18	0.0065410692816744\\
46.19	0.00653955074469672\\
46.2	0.00653803123279317\\
46.21	0.00653651074488989\\
46.22	0.00653498927991098\\
46.23	0.00653346683677849\\
46.24	0.0065319434144124\\
46.25	0.00653041901173063\\
46.26	0.00652889362764905\\
46.27	0.00652736726108141\\
46.28	0.00652583991093944\\
46.29	0.00652431157613272\\
46.3	0.00652278225556878\\
46.31	0.00652125194815304\\
46.32	0.0065197206527888\\
46.33	0.00651818836837727\\
46.34	0.00651665509381754\\
46.35	0.00651512082800657\\
46.36	0.00651358556983921\\
46.37	0.00651204931820814\\
46.38	0.00651051207200394\\
46.39	0.00650897383011503\\
46.4	0.00650743459142767\\
46.41	0.00650589435482597\\
46.42	0.00650435311919188\\
46.43	0.00650281088340518\\
46.44	0.00650126764634346\\
46.45	0.00649972340688215\\
46.46	0.00649817816389447\\
46.47	0.00649663191625147\\
46.48	0.00649508466282198\\
46.49	0.00649353640247263\\
46.5	0.00649198713406785\\
46.51	0.00649043685646984\\
46.52	0.00648888556853856\\
46.53	0.00648733326913179\\
46.54	0.00648577995710501\\
46.55	0.00648422563131149\\
46.56	0.00648267029060225\\
46.57	0.00648111393382605\\
46.58	0.00647955655982939\\
46.59	0.0064779981674565\\
46.6	0.00647643875554933\\
46.61	0.00647487832294756\\
46.62	0.00647331686848857\\
46.63	0.00647175439100746\\
46.64	0.00647019088933701\\
46.65	0.0064686263623077\\
46.66	0.00646706080874771\\
46.67	0.00646549422748289\\
46.68	0.00646392661733675\\
46.69	0.0064623579771305\\
46.7	0.00646078830568297\\
46.71	0.00645921760181067\\
46.72	0.00645764586432775\\
46.73	0.006456073092046\\
46.74	0.00645449928377484\\
46.75	0.00645292443832133\\
46.76	0.00645134855449013\\
46.77	0.00644977163108353\\
46.78	0.00644819366690142\\
46.79	0.00644661466074128\\
46.8	0.0064450346113982\\
46.81	0.00644345351766485\\
46.82	0.00644187137833148\\
46.83	0.0064402881921859\\
46.84	0.00643870395801351\\
46.85	0.00643711867459724\\
46.86	0.00643553234071759\\
46.87	0.0064339449551526\\
46.88	0.00643235651667785\\
46.89	0.00643076702406644\\
46.9	0.00642917647608901\\
46.91	0.00642758487151371\\
46.92	0.00642599220910618\\
46.93	0.0064243984876296\\
46.94	0.0064228037058446\\
46.95	0.00642120786250935\\
46.96	0.00641961095637944\\
46.97	0.00641801298620799\\
46.98	0.00641641395074554\\
46.99	0.00641481384874013\\
47	0.00641321267893722\\
47.01	0.00641161044007972\\
47.02	0.006410007130908\\
47.03	0.00640840275015982\\
47.04	0.00640679729657039\\
47.05	0.00640519076887233\\
47.06	0.00640358316579567\\
47.07	0.00640197448606782\\
47.08	0.00640036472841361\\
47.09	0.00639875389155523\\
47.1	0.00639714197421227\\
47.11	0.00639552897510167\\
47.12	0.00639391489293773\\
47.13	0.00639229972643211\\
47.14	0.00639068347429384\\
47.15	0.00638906613522926\\
47.16	0.00638744770794204\\
47.17	0.00638582819113318\\
47.18	0.00638420758350102\\
47.19	0.00638258588374116\\
47.2	0.00638096309054654\\
47.21	0.00637933920260737\\
47.22	0.00637771421861116\\
47.23	0.00637608813724267\\
47.24	0.00637446095718396\\
47.25	0.00637283267711434\\
47.26	0.00637120329571035\\
47.27	0.00636957281164581\\
47.28	0.00636794122359175\\
47.29	0.00636630853021644\\
47.3	0.00636467473018537\\
47.31	0.00636303982216122\\
47.32	0.00636140380480392\\
47.33	0.00635976667677055\\
47.34	0.00635812843671541\\
47.35	0.00635648908328995\\
47.36	0.0063548486151428\\
47.37	0.00635320703091978\\
47.38	0.00635156432926383\\
47.39	0.00634992050881505\\
47.4	0.00634827556821068\\
47.41	0.00634662950608507\\
47.42	0.00634498232106972\\
47.43	0.00634333401179322\\
47.44	0.00634168457688127\\
47.45	0.00634003401495666\\
47.46	0.00633838232463929\\
47.47	0.00633672950454611\\
47.48	0.00633507555329115\\
47.49	0.00633342046948549\\
47.5	0.00633176425173729\\
47.51	0.00633010689865172\\
47.52	0.00632844840883101\\
47.53	0.00632678878087439\\
47.54	0.00632512801337814\\
47.55	0.00632346610493552\\
47.56	0.0063218030541368\\
47.57	0.00632013885956924\\
47.58	0.00631847351981708\\
47.59	0.00631680703346153\\
47.6	0.00631513939908077\\
47.61	0.00631347061524992\\
47.62	0.00631180068054108\\
47.63	0.00631012959352323\\
47.64	0.00630845735276233\\
47.65	0.00630678395682123\\
47.66	0.0063051094042597\\
47.67	0.0063034336936344\\
47.68	0.00630175682349888\\
47.69	0.00630007879240358\\
47.7	0.00629839959889581\\
47.71	0.00629671924151974\\
47.72	0.00629503771881638\\
47.73	0.0062933550293236\\
47.74	0.00629167117157611\\
47.75	0.00628998614410542\\
47.76	0.00628829994543988\\
47.77	0.00628661257410463\\
47.78	0.00628492402862161\\
47.79	0.00628323430750955\\
47.8	0.00628154340928394\\
47.81	0.00627985133245707\\
47.82	0.00627815807553794\\
47.83	0.00627646363703235\\
47.84	0.00627476801544279\\
47.85	0.00627307120926852\\
47.86	0.00627137321700548\\
47.87	0.00626967403714635\\
47.88	0.00626797366818047\\
47.89	0.00626627210859392\\
47.9	0.0062645693568694\\
47.91	0.00626286541148633\\
47.92	0.00626116027092076\\
47.93	0.00625945393364539\\
47.94	0.00625774639812956\\
47.95	0.00625603766283924\\
47.96	0.00625432772623702\\
47.97	0.00625261658678209\\
47.98	0.00625090424293023\\
47.99	0.00624919069313384\\
48	0.00624747593584185\\
48.01	0.00624575996949978\\
48.02	0.00624404279254971\\
48.03	0.00624232440343026\\
48.04	0.00624060480057658\\
48.05	0.00623888398242035\\
48.06	0.00623716194738974\\
48.07	0.00623543869390946\\
48.08	0.00623371422040069\\
48.09	0.0062319885252811\\
48.1	0.00623026160696481\\
48.11	0.00622853346386243\\
48.12	0.00622680409438099\\
48.13	0.00622507349692399\\
48.14	0.00622334166989132\\
48.15	0.00622160861167931\\
48.16	0.00621987432068069\\
48.17	0.00621813879528459\\
48.18	0.00621640203387652\\
48.19	0.00621466403483835\\
48.2	0.00621292479654832\\
48.21	0.00621118431738103\\
48.22	0.0062094425957074\\
48.23	0.00620769962989468\\
48.24	0.00620595541830646\\
48.25	0.00620420995930261\\
48.26	0.00620246325123929\\
48.27	0.00620071529246896\\
48.28	0.00619896608134034\\
48.29	0.00619721561619842\\
48.3	0.00619546389538442\\
48.31	0.00619371091723581\\
48.32	0.00619195668008627\\
48.33	0.00619020118226571\\
48.34	0.00618844442210023\\
48.35	0.00618668639791213\\
48.36	0.00618492710801986\\
48.37	0.00618316655073806\\
48.38	0.00618140472437754\\
48.39	0.00617964162724521\\
48.4	0.00617787725764413\\
48.41	0.00617611161387348\\
48.42	0.00617434469422855\\
48.43	0.00617257649700071\\
48.44	0.00617080702047743\\
48.45	0.00616903626294222\\
48.46	0.00616726422267468\\
48.47	0.00616549089795044\\
48.48	0.00616371628704116\\
48.49	0.00616194038821453\\
48.5	0.00616016319973423\\
48.51	0.00615838471985995\\
48.52	0.00615660494684736\\
48.53	0.00615487450476265\\
48.54	0.00615395281727438\\
48.55	0.00615303064111654\\
48.56	0.00615210797600223\\
48.57	0.00615118482164451\\
48.58	0.00615026117775647\\
48.59	0.00614933704405116\\
48.6	0.00614841242024162\\
48.61	0.00614748730604091\\
48.62	0.00614656170116206\\
48.63	0.0061456356053181\\
48.64	0.00614470901822207\\
48.65	0.00614378193958699\\
48.66	0.00614285436912589\\
48.67	0.00614192630655178\\
48.68	0.00614099775157769\\
48.69	0.00614006870391664\\
48.7	0.00613913916328166\\
48.71	0.00613820912938578\\
48.72	0.00613727860194202\\
48.73	0.00613634758066343\\
48.74	0.00613541606526304\\
48.75	0.00613448405545392\\
48.76	0.00613355155094911\\
48.77	0.00613261855146168\\
48.78	0.0061316850567047\\
48.79	0.00613075106639128\\
48.8	0.0061298165802345\\
48.81	0.00612888159794747\\
48.82	0.00612794611924334\\
48.83	0.00612701014383523\\
48.84	0.00612607367143632\\
48.85	0.00612513670175977\\
48.86	0.00612419923451877\\
48.87	0.00612326126942656\\
48.88	0.00612232280619635\\
48.89	0.00612138384454142\\
48.9	0.00612044438417503\\
48.91	0.0061195044248105\\
48.92	0.00611856396616115\\
48.93	0.00611762300794034\\
48.94	0.00611668154986144\\
48.95	0.00611573959163788\\
48.96	0.00611479713298309\\
48.97	0.00611385417361055\\
48.98	0.00611291071323376\\
48.99	0.00611196675156626\\
49	0.00611102228832162\\
49.01	0.00611007732321344\\
49.02	0.00610913185595536\\
49.03	0.00610818588626109\\
49.04	0.00610723941384432\\
49.05	0.00610629243841881\\
49.06	0.00610534495969838\\
49.07	0.00610439697739686\\
49.08	0.00610344849122814\\
49.09	0.00610249950090616\\
49.1	0.00610155000614489\\
49.11	0.00610060000665836\\
49.12	0.00609964950216065\\
49.13	0.00609869849236587\\
49.14	0.0060977469769882\\
49.15	0.00609679495574188\\
49.16	0.00609584242834117\\
49.17	0.00609488939450041\\
49.18	0.006093935853934\\
49.19	0.00609298180635639\\
49.2	0.00609202725148208\\
49.21	0.00609107218902564\\
49.22	0.00609011661870169\\
49.23	0.00608916054022494\\
49.24	0.00608820395331012\\
49.25	0.00608724685767208\\
49.26	0.00608628925302568\\
49.27	0.00608533113908588\\
49.28	0.0060843725155677\\
49.29	0.00608341338218624\\
49.3	0.00608245373865666\\
49.31	0.00608149358469419\\
49.32	0.00608053292001416\\
49.33	0.00607957174433193\\
49.34	0.00607861005736299\\
49.35	0.00607764785882287\\
49.36	0.00607668514842718\\
49.37	0.00607572192589164\\
49.38	0.00607475819093202\\
49.39	0.00607379394326421\\
49.4	0.00607282918260414\\
49.41	0.00607186390866787\\
49.42	0.00607089812117152\\
49.43	0.00606993181983131\\
49.44	0.00606896500436355\\
49.45	0.00606799767448464\\
49.46	0.00606702982991107\\
49.47	0.00606606147035943\\
49.48	0.00606509259554642\\
49.49	0.00606412320518882\\
49.5	0.00606315329900351\\
49.51	0.00606218287670748\\
49.52	0.0060612119380178\\
49.53	0.0060602404826517\\
49.54	0.00605926851032644\\
49.55	0.00605829602075945\\
49.56	0.00605732301366822\\
49.57	0.00605634948877039\\
49.58	0.00605537544578368\\
49.59	0.00605440088442596\\
49.6	0.00605342580441517\\
49.61	0.00605245020546939\\
49.62	0.00605147408730682\\
49.63	0.00605049744964578\\
49.64	0.00604952029220469\\
49.65	0.00604854261470212\\
49.66	0.00604756441685675\\
49.67	0.00604658569838738\\
49.68	0.00604560645901294\\
49.69	0.0060446266984525\\
49.7	0.00604364641642526\\
49.71	0.00604266561265053\\
49.72	0.00604168428684777\\
49.73	0.00604070243873657\\
49.74	0.00603972006803667\\
49.75	0.00603873717446793\\
49.76	0.00603775375775035\\
49.77	0.00603676981760409\\
49.78	0.00603578535374944\\
49.79	0.00603480036590684\\
49.8	0.00603381485379686\\
49.81	0.00603282881714025\\
49.82	0.00603184225565788\\
49.83	0.00603085516907079\\
49.84	0.00602986755710015\\
49.85	0.00602887941946732\\
49.86	0.00602789075589379\\
49.87	0.00602690156610122\\
49.88	0.00602591184981144\\
49.89	0.0060249216067464\\
49.9	0.00602393083662827\\
49.91	0.00602293953917934\\
49.92	0.0060219477141221\\
49.93	0.00602095536117919\\
49.94	0.00601996248007342\\
49.95	0.00601896907052779\\
49.96	0.00601797513226545\\
49.97	0.00601698066500974\\
49.98	0.00601598566848419\\
49.99	0.00601499014241248\\
50	0.0060139940865185\\
50.01	0.00601299750052631\\
50.02	0.00601200038416016\\
50.03	0.00601100273714448\\
50.04	0.00601000455920389\\
50.05	0.00600900585006321\\
50.06	0.00600800660944745\\
50.07	0.00600700683708181\\
50.08	0.00600600653269169\\
50.09	0.00600500569600268\\
50.1	0.00600400432674059\\
50.11	0.00600300242463141\\
50.12	0.00600199998940135\\
50.13	0.00600099702077682\\
50.14	0.00599999351848445\\
50.15	0.00599898948225105\\
50.16	0.00599798491180368\\
50.17	0.00599697980686958\\
50.18	0.00599597416717623\\
50.19	0.00599496799245131\\
50.2	0.00599396128242275\\
50.21	0.00599295403681866\\
50.22	0.00599194625536741\\
50.23	0.00599093793779758\\
50.24	0.00598992908383798\\
50.25	0.00598891969321765\\
50.26	0.00598790976566586\\
50.27	0.00598689930091213\\
50.28	0.00598588829868618\\
50.29	0.00598487675871802\\
50.3	0.00598386468073786\\
50.31	0.00598285206447616\\
50.32	0.00598183890966364\\
50.33	0.00598082521603126\\
50.34	0.00597981098331021\\
50.35	0.00597879621123197\\
50.36	0.00597778089952824\\
50.37	0.00597676504793098\\
50.38	0.00597574865617242\\
50.39	0.00597473172398505\\
50.4	0.00597371425110161\\
50.41	0.00597269623725512\\
50.42	0.00597167768217885\\
50.43	0.00597065858560635\\
50.44	0.00596963894727142\\
50.45	0.00596861876690817\\
50.46	0.00596759804425096\\
50.47	0.00596657677903443\\
50.48	0.0059655549709935\\
50.49	0.00596453261986338\\
50.5	0.00596350972537956\\
50.51	0.0059624862872778\\
50.52	0.0059614623052942\\
50.53	0.00596043777916509\\
50.54	0.00595941270862712\\
50.55	0.00595838709341725\\
50.56	0.00595736093327272\\
50.57	0.00595633422793107\\
50.58	0.00595530697713015\\
50.59	0.00595427918060813\\
50.6	0.00595325083810345\\
50.61	0.00595222194935489\\
50.62	0.00595119251410154\\
50.63	0.00595016253208279\\
50.64	0.00594913200303837\\
50.65	0.00594810092670831\\
50.66	0.00594706930283298\\
50.67	0.00594603713115306\\
50.68	0.00594500441140955\\
50.69	0.00594397114334381\\
50.7	0.0059429373266975\\
50.71	0.00594190296121264\\
50.72	0.00594086804663157\\
50.73	0.00593983258269698\\
50.74	0.0059387965691519\\
50.75	0.00593776000573971\\
50.76	0.00593672289220411\\
50.77	0.00593568522828919\\
50.78	0.00593464701373937\\
50.79	0.00593360824829943\\
50.8	0.00593256893171452\\
50.81	0.00593152906373011\\
50.82	0.00593048864409208\\
50.83	0.00592944767254665\\
50.84	0.00592840614884043\\
50.85	0.00592736407272036\\
50.86	0.0059263214439338\\
50.87	0.00592527826222846\\
50.88	0.00592423452735243\\
50.89	0.00592319023905421\\
50.9	0.00592214539708263\\
50.91	0.00592110000118696\\
50.92	0.00592005405111683\\
50.93	0.00591900754662228\\
50.94	0.00591796048745374\\
50.95	0.00591691287336203\\
50.96	0.00591586470409837\\
50.97	0.0059148159794144\\
50.98	0.00591376669906215\\
50.99	0.00591271686279408\\
51	0.00591166647036303\\
51.01	0.00591061552152229\\
51.02	0.00590956401602555\\
51.03	0.00590851195362693\\
51.04	0.00590745933408095\\
51.05	0.00590640615714258\\
51.06	0.00590535242256723\\
51.07	0.00590429813011071\\
51.08	0.00590324327952928\\
51.09	0.00590218787057966\\
51.1	0.00590113190301897\\
51.11	0.00590007537660481\\
51.12	0.00589901829109521\\
51.13	0.00589796064624865\\
51.14	0.00589690244182408\\
51.15	0.00589584367758088\\
51.16	0.00589478435327892\\
51.17	0.00589372446867849\\
51.18	0.0058926640235404\\
51.19	0.00589160301762588\\
51.2	0.00589054145069667\\
51.21	0.00588947932251495\\
51.22	0.0058884166328434\\
51.23	0.00588735338144519\\
51.24	0.00588628956808394\\
51.25	0.00588522519252379\\
51.26	0.00588416025452936\\
51.27	0.00588309475386575\\
51.28	0.00588202869029858\\
51.29	0.00588096206359397\\
51.3	0.00587989487351851\\
51.31	0.00587882711983933\\
51.32	0.00587775880232406\\
51.33	0.00587668992074084\\
51.34	0.00587562047485834\\
51.35	0.00587455046444572\\
51.36	0.0058734798892727\\
51.37	0.0058724087491095\\
51.38	0.00587133704372689\\
51.39	0.00587026477289614\\
51.4	0.00586919193638909\\
51.41	0.00586811853397811\\
51.42	0.0058670445654361\\
51.43	0.00586597003053653\\
51.44	0.0058648949290534\\
51.45	0.00586381926076128\\
51.46	0.00586274302543528\\
51.47	0.00586166622285107\\
51.48	0.00586058885278491\\
51.49	0.00585951091501361\\
51.5	0.00585843240931454\\
51.51	0.00585735333546565\\
51.52	0.00585627369324549\\
51.53	0.00585519348243316\\
51.54	0.00585411270280838\\
51.55	0.00585303135415141\\
51.56	0.00585194943624316\\
51.57	0.00585086694886508\\
51.58	0.00584978389179926\\
51.59	0.00584870026482838\\
51.6	0.00584761606773573\\
51.61	0.00584653130030519\\
51.62	0.00584544596232129\\
51.63	0.00584436005356915\\
51.64	0.00584327357383453\\
51.65	0.00584218652290379\\
51.66	0.00584109890056395\\
51.67	0.00584001070660264\\
51.68	0.00583892194080814\\
51.69	0.00583783260296936\\
51.7	0.00583674269287586\\
51.71	0.00583565221031786\\
51.72	0.00583456115508621\\
51.73	0.00583346952697243\\
51.74	0.00583237732576869\\
51.75	0.00583128455126784\\
51.76	0.00583019120326338\\
51.77	0.00582909728154949\\
51.78	0.00582800278592103\\
51.79	0.00582690771617352\\
51.8	0.00582581207210318\\
51.81	0.00582471585350693\\
51.82	0.00582361906018236\\
51.83	0.00582252169192775\\
51.84	0.00582142374854211\\
51.85	0.00582032522982512\\
51.86	0.00581922613557719\\
51.87	0.00581812646559944\\
51.88	0.00581702621969368\\
51.89	0.00581592539766248\\
51.9	0.00581482399930911\\
51.91	0.00581372202443758\\
51.92	0.00581261947285261\\
51.93	0.00581151634435969\\
51.94	0.00581041263876502\\
51.95	0.00580930835587557\\
51.96	0.00580820349549904\\
51.97	0.0058070980574439\\
51.98	0.00580599204151936\\
51.99	0.00580488544753541\\
52	0.00580377827530281\\
52.01	0.00580267052463306\\
52.02	0.00580156219533847\\
52.03	0.00580045328723211\\
52.04	0.00579934380012785\\
52.05	0.00579823373384033\\
52.06	0.005797123088185\\
52.07	0.0057960118629781\\
52.08	0.00579490005803666\\
52.09	0.00579378767317853\\
52.1	0.00579267470822239\\
52.11	0.0057915611629877\\
52.12	0.00579044703729475\\
52.13	0.00578933233096468\\
52.14	0.00578821704381942\\
52.15	0.00578710117568176\\
52.16	0.00578598472637533\\
52.17	0.00578486769572459\\
52.18	0.00578375008355486\\
52.19	0.00578263188969229\\
52.2	0.00578151311396392\\
52.21	0.00578039375619763\\
52.22	0.00577927381622217\\
52.23	0.00577815329386717\\
52.24	0.00577703218896312\\
52.25	0.0057759105013414\\
52.26	0.0057747882308343\\
52.27	0.00577366537727496\\
52.28	0.00577254194049744\\
52.29	0.00577141792033668\\
52.3	0.00577029331662857\\
52.31	0.00576916812920985\\
52.32	0.00576804235791823\\
52.33	0.0057669160025923\\
52.34	0.0057657890630716\\
52.35	0.00576466153919659\\
52.36	0.00576353343080866\\
52.37	0.00576240473775015\\
52.38	0.00576127545986436\\
52.39	0.0057601455969955\\
52.4	0.00575901514898878\\
52.41	0.00575788411569034\\
52.42	0.00575675249694729\\
52.43	0.00575562029260774\\
52.44	0.00575448750252073\\
52.45	0.00575335412653632\\
52.46	0.00575222016450554\\
52.47	0.00575108561628041\\
52.48	0.00574995048171396\\
52.49	0.0057488147606602\\
52.5	0.00574767845297418\\
52.51	0.00574654155851193\\
52.52	0.00574540407713053\\
52.53	0.00574426600868807\\
52.54	0.00574312735304365\\
52.55	0.00574198811005744\\
52.56	0.00574084827959064\\
52.57	0.00573970786150547\\
52.58	0.00573856685566524\\
52.59	0.00573742526193429\\
52.6	0.00573628308017803\\
52.61	0.00573514031026295\\
52.62	0.0057339969520566\\
52.63	0.0057328530054276\\
52.64	0.00573170847024569\\
52.65	0.00573056334638166\\
52.66	0.00572941763370743\\
52.67	0.00572827133209599\\
52.68	0.00572712444142147\\
52.69	0.0057259769615591\\
52.7	0.00572482889238522\\
52.71	0.00572368023377731\\
52.72	0.00572253098561396\\
52.73	0.00572138114777493\\
52.74	0.00572023072014109\\
52.75	0.00571907970259449\\
52.76	0.0057179280950183\\
52.77	0.00571677589729688\\
52.78	0.00571562310931574\\
52.79	0.00571446973096157\\
52.8	0.00571331576212224\\
52.81	0.00571216120268679\\
52.82	0.00571100605254547\\
52.83	0.00570985031158972\\
52.84	0.00570869397971218\\
52.85	0.0057075370568067\\
52.86	0.00570637954276834\\
52.87	0.00570522143749341\\
52.88	0.00570406274087941\\
52.89	0.00570290345282508\\
52.9	0.00570174357323044\\
52.91	0.0057005831019967\\
52.92	0.00569942203902635\\
52.93	0.00569826038422316\\
52.94	0.00569709813749214\\
52.95	0.00569593529873956\\
52.96	0.005694771867873\\
52.97	0.00569360784480131\\
52.98	0.00569244322943463\\
52.99	0.00569127802168441\\
53	0.00569011222146338\\
53.01	0.00568894582868561\\
53.02	0.00568777884326646\\
53.03	0.00568661126512265\\
53.04	0.00568544309417219\\
53.05	0.00568427433033446\\
53.06	0.00568310497353017\\
53.07	0.00568193502368138\\
53.08	0.00568076448071152\\
53.09	0.00567959334454537\\
53.1	0.00567842161510909\\
53.11	0.00567724929233022\\
53.12	0.00567607637613767\\
53.13	0.00567490286646177\\
53.14	0.00567372876323423\\
53.15	0.00567255406638816\\
53.16	0.0056713787758581\\
53.17	0.00567020289158001\\
53.18	0.00566902641349125\\
53.19	0.00566784934153066\\
53.2	0.00566667167563847\\
53.21	0.00566549341575641\\
53.22	0.00566431456182762\\
53.23	0.00566313511379673\\
53.24	0.00566195507160984\\
53.25	0.00566077443521451\\
53.26	0.0056595932045598\\
53.27	0.00565841137959625\\
53.28	0.00565722896027593\\
53.29	0.00565604594655237\\
53.3	0.00565486233838064\\
53.31	0.00565367813571734\\
53.32	0.00565249333852057\\
53.33	0.00565130794675\\
53.34	0.00565012196036683\\
53.35	0.00564893537933379\\
53.36	0.0056477482036152\\
53.37	0.00564656043317695\\
53.38	0.00564537206798647\\
53.39	0.0056441831080128\\
53.4	0.00564299355322656\\
53.41	0.00564180340359997\\
53.42	0.00564061265910685\\
53.43	0.00563942131972264\\
53.44	0.00563822938542441\\
53.45	0.00563703685619084\\
53.46	0.00563584373200224\\
53.47	0.0056346500128406\\
53.48	0.00563345569868953\\
53.49	0.00563226078953432\\
53.5	0.00563106528536191\\
53.51	0.00562986918616093\\
53.52	0.00562867249192169\\
53.53	0.0056274752026362\\
53.54	0.00562627731829817\\
53.55	0.005625078838903\\
53.56	0.00562387976444783\\
53.57	0.00562268009493153\\
53.58	0.00562147983035468\\
53.59	0.00562027897071961\\
53.6	0.00561907751603041\\
53.61	0.00561787546629293\\
53.62	0.00561667282151478\\
53.63	0.00561546958170535\\
53.64	0.00561426574687581\\
53.65	0.00561306131703911\\
53.66	0.00561185629221005\\
53.67	0.00561065067240518\\
53.68	0.0056094444576429\\
53.69	0.00560823764794345\\
53.7	0.00560703024332887\\
53.71	0.00560582224382307\\
53.72	0.00560461364945182\\
53.73	0.00560340446024273\\
53.74	0.0056021946762253\\
53.75	0.0056009842974309\\
53.76	0.00559977332389278\\
53.77	0.00559856175564613\\
53.78	0.00559734959272798\\
53.79	0.00559613683517735\\
53.8	0.00559492348303512\\
53.81	0.00559370953634415\\
53.82	0.00559249499514922\\
53.83	0.00559127985949707\\
53.84	0.00559006412943641\\
53.85	0.0055888478050179\\
53.86	0.0055876308862942\\
53.87	0.00558641337331995\\
53.88	0.00558519526615179\\
53.89	0.00558397656484838\\
53.9	0.00558275726947038\\
53.91	0.00558153738008049\\
53.92	0.00558031689674344\\
53.93	0.00557909581952601\\
53.94	0.00557787414849703\\
53.95	0.00557665188372741\\
53.96	0.00557542902529014\\
53.97	0.00557420557326026\\
53.98	0.00557298152771493\\
53.99	0.00557175688873342\\
54	0.0055705316563971\\
54.01	0.00556930583078947\\
54.02	0.00556807941199618\\
54.03	0.00556685240010499\\
54.04	0.00556562479520584\\
54.05	0.00556439659739083\\
54.06	0.00556316780675424\\
54.07	0.00556193842339251\\
54.08	0.0055607084474043\\
54.09	0.00555947787889048\\
54.1	0.00555824671795409\\
54.11	0.00555701496470046\\
54.12	0.0055557826192371\\
54.13	0.00555454968167379\\
54.14	0.00555331615212258\\
54.15	0.00555208203069777\\
54.16	0.00555084731751592\\
54.17	0.00554961201269591\\
54.18	0.00554837611635891\\
54.19	0.00554713962862838\\
54.2	0.00554590254963013\\
54.21	0.00554466487949228\\
54.22	0.0055434266183453\\
54.23	0.005542187766322\\
54.24	0.00554094832355757\\
54.25	0.00553970829018955\\
54.26	0.00553846766635789\\
54.27	0.00553722645220493\\
54.28	0.0055359846478754\\
54.29	0.00553474225351647\\
54.3	0.00553349926927771\\
54.31	0.00553225569531117\\
54.32	0.00553101153177132\\
54.33	0.0055297667788151\\
54.34	0.00552852143660192\\
54.35	0.00552727550529369\\
54.36	0.00552602898505481\\
54.37	0.00552478187605216\\
54.38	0.00552353417845519\\
54.39	0.00552228589243583\\
54.4	0.00552103701816859\\
54.41	0.00551978755583052\\
54.42	0.00551853750560123\\
54.43	0.00551728686766292\\
54.44	0.00551603564220036\\
54.45	0.00551478382940094\\
54.46	0.00551353142945465\\
54.47	0.00551227844255411\\
54.48	0.00551102486889458\\
54.49	0.00550977070867396\\
54.5	0.0055085159620928\\
54.51	0.00550726062935436\\
54.52	0.00550600471066454\\
54.53	0.00550474820623198\\
54.54	0.00550349111626798\\
54.55	0.00550223344098661\\
54.56	0.00550097518060464\\
54.57	0.0054997163353416\\
54.58	0.00549845690541978\\
54.59	0.00549719689106422\\
54.6	0.00549593629250277\\
54.61	0.00549467510996607\\
54.62	0.00549341334368756\\
54.63	0.00549215099390351\\
54.64	0.00549088806085301\\
54.65	0.00548962454477801\\
54.66	0.00548836044592331\\
54.67	0.0054870957645366\\
54.68	0.00548583050086843\\
54.69	0.00548456465517227\\
54.7	0.00548329822770449\\
54.71	0.00548203121872439\\
54.72	0.00548076362849421\\
54.73	0.00547949545727912\\
54.74	0.00547822670534728\\
54.75	0.00547695737296981\\
54.76	0.00547568746042084\\
54.77	0.00547441696797747\\
54.78	0.00547314589591986\\
54.79	0.00547187424453118\\
54.8	0.00547060201409763\\
54.81	0.00546932920490849\\
54.82	0.00546805581725612\\
54.83	0.00546678185143593\\
54.84	0.00546550730774646\\
54.85	0.00546423218648935\\
54.86	0.00546295648796937\\
54.87	0.00546168021249444\\
54.88	0.00546040336037562\\
54.89	0.00545912593192714\\
54.9	0.00545784792746644\\
54.91	0.00545656934731412\\
54.92	0.00545529019179402\\
54.93	0.00545401046123318\\
54.94	0.00545273015596192\\
54.95	0.00545144927631376\\
54.96	0.00545016782262554\\
54.97	0.00544888579523737\\
54.98	0.00544760319449263\\
54.99	0.00544632002073805\\
55	0.00544503627432367\\
55.01	0.00544375195560288\\
55.02	0.00544246706493241\\
55.03	0.00544118160267239\\
55.04	0.00543989556918632\\
55.05	0.0054386089648411\\
55.06	0.00543732179000705\\
55.07	0.00543603404505793\\
55.08	0.00543474573037094\\
55.09	0.00543345684632674\\
55.1	0.00543216739330948\\
55.11	0.00543087737170679\\
55.12	0.00542958678190982\\
55.13	0.00542829562431324\\
55.14	0.00542700389931526\\
55.15	0.00542571160731765\\
55.16	0.00542441874872573\\
55.17	0.00542312532394844\\
55.18	0.0054218313333983\\
55.19	0.00542053677749145\\
55.2	0.00541924165664768\\
55.21	0.0054179459712904\\
55.22	0.00541664972184674\\
55.23	0.00541535290874745\\
55.24	0.00541405553242703\\
55.25	0.00541275759332365\\
55.26	0.00541145909187926\\
55.27	0.00541016002853952\\
55.28	0.00540886040375387\\
55.29	0.00540756021797553\\
55.3	0.00540625947166151\\
55.31	0.00540495816527265\\
55.32	0.00540365629927361\\
55.33	0.0054023538741329\\
55.34	0.00540105089032289\\
55.35	0.00539974734831983\\
55.36	0.00539844324860387\\
55.37	0.0053971385916591\\
55.38	0.0053958333779735\\
55.39	0.00539452760803903\\
55.4	0.0053932212823516\\
55.41	0.00539191440141112\\
55.42	0.0053906069657215\\
55.43	0.00538929897579065\\
55.44	0.00538799043213054\\
55.45	0.00538668133525718\\
55.46	0.00538537168569066\\
55.47	0.00538406148395514\\
55.48	0.00538275073057893\\
55.49	0.00538143942609442\\
55.5	0.00538012757103818\\
55.51	0.0053788151659509\\
55.52	0.0053775022113775\\
55.53	0.00537618870786706\\
55.54	0.00537487465597288\\
55.55	0.00537356005625252\\
55.56	0.00537224490926777\\
55.57	0.00537092921558469\\
55.58	0.00536961297577364\\
55.59	0.00536829619040929\\
55.6	0.00536697886007063\\
55.61	0.00536566098534099\\
55.62	0.00536434256680808\\
55.63	0.00536302360506399\\
55.64	0.0053617041007052\\
55.65	0.00536038405433263\\
55.66	0.00535906346655163\\
55.67	0.005357742337972\\
55.68	0.00535642066920805\\
55.69	0.00535509846087856\\
55.7	0.00535377571360685\\
55.71	0.00535245242802076\\
55.72	0.00535112860475269\\
55.73	0.00534980424443962\\
55.74	0.00534847934772314\\
55.75	0.00534715391524944\\
55.76	0.00534582794766936\\
55.77	0.00534450144563838\\
55.78	0.00534317440981667\\
55.79	0.00534184684086909\\
55.8	0.00534051873946522\\
55.81	0.00533919010627937\\
55.82	0.00533786094199063\\
55.83	0.00533653124728286\\
55.84	0.0053352010228447\\
55.85	0.00533387026936963\\
55.86	0.00533253898755596\\
55.87	0.00533120717810688\\
55.88	0.00532987484173043\\
55.89	0.00532854197913958\\
55.9	0.00532720859105222\\
55.91	0.00532587467819117\\
55.92	0.00532454024128424\\
55.93	0.00532320528106421\\
55.94	0.00532186979826887\\
55.95	0.00532053379364107\\
55.96	0.00531919726792867\\
55.97	0.00531786022188463\\
55.98	0.005316522656267\\
55.99	0.00531518457183896\\
56	0.00531384596936881\\
56.01	0.00531250684963002\\
56.02	0.00531116721340124\\
56.03	0.00530982706146634\\
56.04	0.00530848639461441\\
56.05	0.00530714521363979\\
56.06	0.00530580351934209\\
56.07	0.00530446131252623\\
56.08	0.00530311859400242\\
56.09	0.00530177536458625\\
56.1	0.00530043162509865\\
56.11	0.00529908737636595\\
56.12	0.00529774261921987\\
56.13	0.00529639735449758\\
56.14	0.00529505158304171\\
56.15	0.00529370530570035\\
56.16	0.00529235852332712\\
56.17	0.00529101123678115\\
56.18	0.00528966344692712\\
56.19	0.00528831515463528\\
56.2	0.0052869663607815\\
56.21	0.00528561706624724\\
56.22	0.00528426727191964\\
56.23	0.00528291697869149\\
56.24	0.00528156618746126\\
56.25	0.00528021489913318\\
56.26	0.00527886311461717\\
56.27	0.00527751083482897\\
56.28	0.00527615806069007\\
56.29	0.00527480479312778\\
56.3	0.00527345103307529\\
56.31	0.00527209678147161\\
56.32	0.00527074203926167\\
56.33	0.00526938680739629\\
56.34	0.00526803108683226\\
56.35	0.00526667487853231\\
56.36	0.00526531818346518\\
56.37	0.00526396100260562\\
56.38	0.00526260333693442\\
56.39	0.00526124518743843\\
56.4	0.00525988655511063\\
56.41	0.00525852744095007\\
56.42	0.00525716784596198\\
56.43	0.00525580777115774\\
56.44	0.00525444721755496\\
56.45	0.00525308618617743\\
56.46	0.00525172467805523\\
56.47	0.00525036269422469\\
56.48	0.00524900023572847\\
56.49	0.00524763730361552\\
56.5	0.00524627389894119\\
56.51	0.00524491002276719\\
56.52	0.00524354567616165\\
56.53	0.00524218086019912\\
56.54	0.00524081557596065\\
56.55	0.00523944982453375\\
56.56	0.00523808360701247\\
56.57	0.0052367169244974\\
56.58	0.00523534977809569\\
56.59	0.00523398216892113\\
56.6	0.00523261409809412\\
56.61	0.0052312455667417\\
56.62	0.00522987657599764\\
56.63	0.00522850712700238\\
56.64	0.00522713722090314\\
56.65	0.0052257668588539\\
56.66	0.00522439604201543\\
56.67	0.00522302477155533\\
56.68	0.00522165304864808\\
56.69	0.00522028087447504\\
56.7	0.00521890825022448\\
56.71	0.00521753517709159\\
56.72	0.00521616165627858\\
56.73	0.00521478768899465\\
56.74	0.00521341327645601\\
56.75	0.00521203841988596\\
56.76	0.00521066312051489\\
56.77	0.0052092873795803\\
56.78	0.00520791119832685\\
56.79	0.00520653457800638\\
56.8	0.00520515751987795\\
56.81	0.00520378002520786\\
56.82	0.00520240209526967\\
56.83	0.00520102373134427\\
56.84	0.00519964493471987\\
56.85	0.00519826570669205\\
56.86	0.00519688604856378\\
56.87	0.00519550596164547\\
56.88	0.00519412544725497\\
56.89	0.00519274450671765\\
56.9	0.00519136314136637\\
56.91	0.00518998135254156\\
56.92	0.00518859914159124\\
56.93	0.00518721650987103\\
56.94	0.00518583345874423\\
56.95	0.00518444998958179\\
56.96	0.00518306610376238\\
56.97	0.00518168180267242\\
56.98	0.00518029708770612\\
56.99	0.00517891196026548\\
57	0.00517752642176035\\
57.01	0.00517614047360847\\
57.02	0.00517475411723547\\
57.03	0.00517336735407493\\
57.04	0.0051719801855684\\
57.05	0.00517059261316546\\
57.06	0.00516920463832372\\
57.07	0.00516781626250886\\
57.08	0.00516642748719466\\
57.09	0.00516503831386308\\
57.1	0.00516364874400423\\
57.11	0.00516225877911645\\
57.12	0.00516086842070629\\
57.13	0.00515947767028863\\
57.14	0.00515808652938664\\
57.15	0.00515669499953184\\
57.16	0.00515530308226414\\
57.17	0.00515391077913187\\
57.18	0.00515251809169181\\
57.19	0.00515112502150925\\
57.2	0.00514973157015799\\
57.21	0.00514833773922039\\
57.22	0.00514694353028741\\
57.23	0.00514554894495865\\
57.24	0.00514415398484238\\
57.25	0.00514275865155558\\
57.26	0.00514136294672394\\
57.27	0.00513996687198198\\
57.28	0.005138570428973\\
57.29	0.00513717361934916\\
57.3	0.00513577644477151\\
57.31	0.00513437890691003\\
57.32	0.00513298100744365\\
57.33	0.00513158274806031\\
57.34	0.005130184130457\\
57.35	0.00512878515633977\\
57.36	0.00512738582742378\\
57.37	0.00512598614543335\\
57.38	0.005124586112102\\
57.39	0.00512318572917246\\
57.4	0.00512178499839673\\
57.41	0.00512038392153612\\
57.42	0.00511898250036128\\
57.43	0.00511758073665225\\
57.44	0.00511617863219849\\
57.45	0.00511477618879891\\
57.46	0.00511337340826192\\
57.47	0.00511197029240548\\
57.48	0.00511056684305713\\
57.49	0.00510916306205403\\
57.5	0.00510775895124298\\
57.51	0.00510635451248049\\
57.52	0.00510494974763284\\
57.53	0.00510354465857604\\
57.54	0.00510213924719596\\
57.55	0.00510073351538829\\
57.56	0.00509932746505866\\
57.57	0.00509792109812264\\
57.58	0.00509651441650577\\
57.59	0.00509510742214361\\
57.6	0.00509370011698181\\
57.61	0.00509229250297613\\
57.62	0.00509088458209246\\
57.63	0.00508947635630689\\
57.64	0.00508806782760575\\
57.65	0.00508665899798567\\
57.66	0.00508524986945356\\
57.67	0.00508384044402673\\
57.68	0.00508243072373286\\
57.69	0.00508102071061013\\
57.7	0.00507961040670716\\
57.71	0.00507819981408314\\
57.72	0.00507678893480784\\
57.73	0.00507537777096165\\
57.74	0.00507396632463562\\
57.75	0.00507255459793152\\
57.76	0.00507114259296187\\
57.77	0.00506973031185\\
57.78	0.00506831775673008\\
57.79	0.00506690492974718\\
57.8	0.0050654918330573\\
57.81	0.00506407846882742\\
57.82	0.00506266483923556\\
57.83	0.00506125094647079\\
57.84	0.00505983679273331\\
57.85	0.00505842238023449\\
57.86	0.00505700771119691\\
57.87	0.00505559278785441\\
57.88	0.00505417761245211\\
57.89	0.00505276218724651\\
57.9	0.00505134651450548\\
57.91	0.00504993059650836\\
57.92	0.00504851443554597\\
57.93	0.00504709803392067\\
57.94	0.00504568139394642\\
57.95	0.00504426451794878\\
57.96	0.00504284740826503\\
57.97	0.00504143006724417\\
57.98	0.00504001249724696\\
57.99	0.00503859470064602\\
58	0.00503717667982582\\
58.01	0.00503575843718278\\
58.02	0.00503433997512528\\
58.03	0.00503292129607373\\
58.04	0.00503150240246061\\
58.05	0.00503008329673052\\
58.06	0.00502866398134026\\
58.07	0.00502724445875884\\
58.08	0.00502582473146753\\
58.09	0.00502440480195994\\
58.1	0.00502298467274206\\
58.11	0.0050215643463323\\
58.12	0.00502014382526156\\
58.13	0.00501872311207326\\
58.14	0.0050173022093234\\
58.15	0.00501588111958062\\
58.16	0.00501445984542626\\
58.17	0.00501303838945438\\
58.18	0.00501161675427182\\
58.19	0.0050101949424983\\
58.2	0.00500877295676641\\
58.21	0.00500735079972169\\
58.22	0.0050059284740227\\
58.23	0.00500450598234104\\
58.24	0.00500308332736144\\
58.25	0.00500166051178177\\
58.26	0.00500023753831313\\
58.27	0.0049988144096799\\
58.28	0.00499739112861978\\
58.29	0.00499596769788386\\
58.3	0.00499454412023667\\
58.31	0.00499312039845622\\
58.32	0.0049916966234396\\
58.33	0.00499027280133458\\
58.34	0.00498884893394189\\
58.35	0.00498742502306905\\
58.36	0.0049860010705304\\
58.37	0.00498457707814716\\
58.38	0.00498315304774736\\
58.39	0.00498172898116597\\
58.4	0.00498030488024486\\
58.41	0.00497888074683282\\
58.42	0.00497745658278563\\
58.43	0.00497603238996604\\
58.44	0.00497460817024382\\
58.45	0.00497318392549577\\
58.46	0.00497175965760576\\
58.47	0.00497033536846473\\
58.48	0.00496891105997072\\
58.49	0.00496748673402894\\
58.5	0.00496606239255172\\
58.51	0.00496463803745859\\
58.52	0.00496321367067628\\
58.53	0.00496178929413876\\
58.54	0.00496036490978726\\
58.55	0.00495894051957026\\
58.56	0.00495751612544359\\
58.57	0.00495609172937039\\
58.58	0.00495466733332116\\
58.59	0.00495324293927378\\
58.6	0.00495181854921355\\
58.61	0.00495039416513318\\
58.62	0.00494896978903288\\
58.63	0.0049475454229203\\
58.64	0.00494612106881063\\
58.65	0.0049446967287266\\
58.66	0.00494327240469849\\
58.67	0.00494184809876416\\
58.68	0.00494042381296911\\
58.69	0.00493899954936648\\
58.7	0.00493757531001706\\
58.71	0.00493615109698933\\
58.72	0.00493472691235954\\
58.73	0.00493330275821163\\
58.74	0.00493187863663735\\
58.75	0.00493045454973624\\
58.76	0.00492903049961567\\
58.77	0.00492760648839087\\
58.78	0.00492618251818495\\
58.79	0.00492475859112893\\
58.8	0.00492333470936177\\
58.81	0.00492191087503041\\
58.82	0.00492048709028974\\
58.83	0.00491906335730272\\
58.84	0.00491763967824033\\
58.85	0.00491621605528164\\
58.86	0.00491479249061381\\
58.87	0.00491336898643215\\
58.88	0.00491194554494011\\
58.89	0.00491052216834935\\
58.9	0.00490909885887974\\
58.91	0.00490767561875938\\
58.92	0.00490625245022467\\
58.93	0.00490482935552029\\
58.94	0.00490340633689927\\
58.95	0.00490198339662299\\
58.96	0.00490056053696123\\
58.97	0.00489913776019217\\
58.98	0.00489771506860246\\
58.99	0.0048962924644872\\
59	0.00489486995015003\\
59.01	0.0048934475279031\\
59.02	0.00489202520006714\\
59.03	0.00489060296897146\\
59.04	0.00488918083695403\\
59.05	0.00488775880636142\\
59.06	0.00488633687954893\\
59.07	0.00488491505888056\\
59.08	0.00488349334672905\\
59.09	0.00488207174547594\\
59.1	0.00488065025751154\\
59.11	0.00487922888523503\\
59.12	0.00487780763105444\\
59.13	0.0048763864973867\\
59.14	0.00487496548665767\\
59.15	0.00487354460130218\\
59.16	0.00487212384376402\\
59.17	0.00487070321649606\\
59.18	0.00486928272196017\\
59.19	0.00486786236262733\\
59.2	0.00486644214097764\\
59.21	0.00486502205950032\\
59.22	0.00486360212069382\\
59.23	0.00486218232706577\\
59.24	0.00486076268113303\\
59.25	0.00485934318542178\\
59.26	0.00485792384246748\\
59.27	0.00485650465481492\\
59.28	0.0048550856250183\\
59.29	0.0048536667556412\\
59.3	0.00485224804925666\\
59.31	0.00485082950844715\\
59.32	0.0048494111358047\\
59.33	0.00484799293393085\\
59.34	0.0048465749054367\\
59.35	0.00484515705294298\\
59.36	0.00484373937908004\\
59.37	0.0048423218864879\\
59.38	0.00484090457781631\\
59.39	0.00483948745572473\\
59.4	0.00483807052288241\\
59.41	0.00483665378196839\\
59.42	0.00483523723567158\\
59.43	0.00483382088669074\\
59.44	0.00483240473773455\\
59.45	0.00483098879152164\\
59.46	0.0048295730507806\\
59.47	0.00482815751825005\\
59.48	0.00482674219667866\\
59.49	0.00482532708882517\\
59.5	0.00482391219745846\\
59.51	0.00482249752535755\\
59.52	0.00482108307531164\\
59.53	0.00481966885012017\\
59.54	0.00481825485259285\\
59.55	0.00481684108554967\\
59.56	0.00481542755182095\\
59.57	0.00481401425424739\\
59.58	0.00481260119568009\\
59.59	0.0048111883789806\\
59.6	0.00480977580702093\\
59.61	0.00480836348268362\\
59.62	0.00480695140886175\\
59.63	0.004805539588459\\
59.64	0.00480412802438966\\
59.65	0.00480271671957869\\
59.66	0.00480130567696176\\
59.67	0.00479989489948525\\
59.68	0.00479848439010634\\
59.69	0.00479707415179301\\
59.7	0.00479566418752408\\
59.71	0.00479425450028928\\
59.72	0.00479284509308924\\
59.73	0.00479143596893557\\
59.74	0.00479002713085089\\
59.75	0.00478861858186883\\
59.76	0.00478721032503414\\
59.77	0.00478580236340266\\
59.78	0.00478439470004138\\
59.79	0.0047829873380285\\
59.8	0.00478158028045346\\
59.81	0.00478017353041697\\
59.82	0.00477876709103104\\
59.83	0.00477736096541905\\
59.84	0.00477595515671577\\
59.85	0.00477454966806739\\
59.86	0.00477314450263159\\
59.87	0.00477173966357756\\
59.88	0.00477033515408602\\
59.89	0.00476893097734933\\
59.9	0.00476752713657144\\
59.91	0.004766123634968\\
59.92	0.00476472047576637\\
59.93	0.00476331766220565\\
59.94	0.00476191519753677\\
59.95	0.00476051308502248\\
59.96	0.00475911132793742\\
59.97	0.00475770992956814\\
59.98	0.00475630889321317\\
59.99	0.00475490822218303\\
60	0.00475350791980031\\
60.01	0.00475210798939968\\
60.02	0.00475070843432793\\
60.03	0.00474930925794404\\
60.04	0.00474791046361922\\
60.05	0.00474651205473691\\
60.06	0.00474511403469289\\
60.07	0.00474371640689527\\
60.08	0.00474231917476455\\
60.09	0.00474092234173366\\
60.1	0.00473952591124803\\
60.11	0.00473812988676558\\
60.12	0.00473673427175683\\
60.13	0.00473533906970489\\
60.14	0.00473394428410552\\
60.15	0.0047325499184672\\
60.16	0.00473115597631113\\
60.17	0.00472976246117132\\
60.18	0.0047283693765946\\
60.19	0.00472697672614067\\
60.2	0.00472558451338219\\
60.21	0.00472419274190474\\
60.22	0.00472280141530696\\
60.23	0.00472141053720052\\
60.24	0.0047200201112102\\
60.25	0.00471863014097395\\
60.26	0.0047172406301429\\
60.27	0.00471585158238144\\
60.28	0.00471446300136724\\
60.29	0.00471307489079132\\
60.3	0.00471168725435807\\
60.31	0.00471030009578534\\
60.32	0.00470891341880442\\
60.33	0.00470752722716017\\
60.34	0.00470614152461099\\
60.35	0.00470475631492893\\
60.36	0.00470337160189969\\
60.37	0.00470198738932271\\
60.38	0.00470060368101117\\
60.39	0.00469922048079209\\
60.4	0.00469783779250636\\
60.41	0.00469645562000875\\
60.42	0.00469507396716802\\
60.43	0.00469369283786693\\
60.44	0.00469231223600232\\
60.45	0.0046909321654851\\
60.46	0.0046895526302404\\
60.47	0.00468817363420751\\
60.48	0.00468679518133999\\
60.49	0.00468541727560574\\
60.5	0.00468403992098699\\
60.51	0.00468266312148038\\
60.52	0.00468128688109705\\
60.53	0.00467991120386262\\
60.54	0.00467853609381728\\
60.55	0.00467716155501585\\
60.56	0.0046757875915278\\
60.57	0.00467441420743735\\
60.58	0.00467304140684345\\
60.59	0.0046716691938599\\
60.6	0.00467029757261539\\
60.61	0.0046689265472535\\
60.62	0.00466755612193285\\
60.63	0.00466618630082702\\
60.64	0.00466481708812475\\
60.65	0.00466344848802987\\
60.66	0.00466208050476143\\
60.67	0.00466071314255372\\
60.68	0.00465934640565635\\
60.69	0.00465798029833425\\
60.7	0.0046566148248678\\
60.71	0.00465524998955283\\
60.72	0.00465388579670069\\
60.73	0.00465252225063829\\
60.74	0.0046511593557082\\
60.75	0.00464979711626866\\
60.76	0.00464843553669365\\
60.77	0.00464707462137295\\
60.78	0.0046457143747122\\
60.79	0.00464435480113293\\
60.8	0.00464299590507265\\
60.81	0.0046416376909849\\
60.82	0.00464028016333927\\
60.83	0.00463892332662152\\
60.84	0.00463756718533359\\
60.85	0.00463621174399364\\
60.86	0.00463485700713619\\
60.87	0.00463350297931209\\
60.88	0.00463214966508863\\
60.89	0.00463079706904958\\
60.9	0.00462944519579523\\
60.91	0.0046280940499425\\
60.92	0.00462674363612495\\
60.93	0.00462539395899287\\
60.94	0.0046240450232133\\
60.95	0.00462269683347015\\
60.96	0.00462134939446421\\
60.97	0.00462000271091321\\
60.98	0.00461865678755193\\
60.99	0.00461731162913219\\
61	0.00461596724042297\\
61.01	0.00461462362621046\\
61.02	0.00461328079129807\\
61.03	0.00461193874050655\\
61.04	0.00461059747867405\\
61.05	0.00460925701065614\\
61.06	0.0046079173413259\\
61.07	0.00460657847557399\\
61.08	0.00460524041830868\\
61.09	0.00460390317445594\\
61.1	0.00460256674895952\\
61.11	0.00460123114678095\\
61.12	0.00459989637289967\\
61.13	0.00459856243231306\\
61.14	0.0045972293300365\\
61.15	0.00459589707110345\\
61.16	0.00459456566056551\\
61.17	0.00459323510349248\\
61.18	0.00459190540497245\\
61.19	0.00459057657011179\\
61.2	0.00458924860403533\\
61.21	0.00458792151188631\\
61.22	0.00458659529882654\\
61.23	0.0045852699700364\\
61.24	0.00458394553071495\\
61.25	0.00458262198607996\\
61.26	0.004581299341368\\
61.27	0.0045799776018345\\
61.28	0.00457865677275384\\
61.29	0.00457733685941937\\
61.3	0.00457601786714351\\
61.31	0.00457469980125782\\
61.32	0.00457338266711305\\
61.33	0.00457206647007924\\
61.34	0.00457075121554574\\
61.35	0.00456943690892133\\
61.36	0.00456812355563424\\
61.37	0.00456681116113228\\
61.38	0.00456549973088283\\
61.39	0.004564189270373\\
61.4	0.00456287978510963\\
61.41	0.0045615712806194\\
61.42	0.00456026376244886\\
61.43	0.00455895723616457\\
61.44	0.00455765170735309\\
61.45	0.00455634718162111\\
61.46	0.0045550436645955\\
61.47	0.00455374116192339\\
61.48	0.00455243967927222\\
61.49	0.00455113922232985\\
61.5	0.0045498397968046\\
61.51	0.00454854140842534\\
61.52	0.00454724406294155\\
61.53	0.00454594776612343\\
61.54	0.00454465252376192\\
61.55	0.00454335834166881\\
61.56	0.0045420652256768\\
61.57	0.0045407731816396\\
61.58	0.00453948221543198\\
61.59	0.00453819233294982\\
61.6	0.00453690354011026\\
61.61	0.00453561584285172\\
61.62	0.00453432924713397\\
61.63	0.00453304375893825\\
61.64	0.0045317593842673\\
61.65	0.00453047612914549\\
61.66	0.00452919399961885\\
61.67	0.00452791300175516\\
61.68	0.00452663314164403\\
61.69	0.004525354425397\\
61.7	0.00452407685914758\\
61.71	0.00452280044905136\\
61.72	0.00452152520128609\\
61.73	0.00452025112205171\\
61.74	0.0045189782175705\\
61.75	0.00451770649408711\\
61.76	0.00451643595786866\\
61.77	0.00451516661520483\\
61.78	0.00451389847240793\\
61.79	0.00451263153581294\\
61.8	0.00451136581177768\\
61.81	0.00451010035659968\\
61.82	0.00450883463279916\\
61.83	0.00450756864084738\\
61.84	0.00450630238121686\\
61.85	0.0045050358543814\\
61.86	0.00450376906081604\\
61.87	0.00450250200099712\\
61.88	0.00450123467540221\\
61.89	0.00449996708451019\\
61.9	0.00449869922880116\\
61.91	0.00449743110875655\\
61.92	0.00449616272485904\\
61.93	0.00449489407759257\\
61.94	0.00449362516744238\\
61.95	0.00449235599489498\\
61.96	0.00449108656043816\\
61.97	0.00448981686456101\\
61.98	0.00448854690775388\\
61.99	0.00448727669050843\\
62	0.00448600621331759\\
62.01	0.00448473547667558\\
62.02	0.00448346448107793\\
62.03	0.00448219322702144\\
62.04	0.00448092171500421\\
62.05	0.00447964994552565\\
62.06	0.00447837791908646\\
62.07	0.00447710563618862\\
62.08	0.00447583309733545\\
62.09	0.00447456030303154\\
62.1	0.00447328725378279\\
62.11	0.00447201395009641\\
62.12	0.00447074039248091\\
62.13	0.00446946658144614\\
62.14	0.00446819251750323\\
62.15	0.00446691820116461\\
62.16	0.00446564363294405\\
62.17	0.00446436881335664\\
62.18	0.00446309374291876\\
62.19	0.00446181842214812\\
62.2	0.00446054285156377\\
62.21	0.00445926703168604\\
62.22	0.00445799096303662\\
62.23	0.0044567146461385\\
62.24	0.00445543808151602\\
62.25	0.00445416126969483\\
62.26	0.00445288421120191\\
62.27	0.00445160690656557\\
62.28	0.00445032935631547\\
62.29	0.00444905156098258\\
62.3	0.00444777352109922\\
62.31	0.00444649523719905\\
62.32	0.00444521670981704\\
62.33	0.00444393793948954\\
62.34	0.0044426589267542\\
62.35	0.00444137967215006\\
62.36	0.00444010017621746\\
62.37	0.00443882043949812\\
62.38	0.00443754046253509\\
62.39	0.00443626024587276\\
62.4	0.00443497979005689\\
62.41	0.00443369909563458\\
62.42	0.00443241816315429\\
62.43	0.00443113699316584\\
62.44	0.00442985558622039\\
62.45	0.00442857394287046\\
62.46	0.00442729206366994\\
62.47	0.00442600994917407\\
62.48	0.00442472759993947\\
62.49	0.00442344501652411\\
62.5	0.00442216219948732\\
62.51	0.0044208791493898\\
62.52	0.00441959586679363\\
62.53	0.00441831235226225\\
62.54	0.00441702860636047\\
62.55	0.00441574462965446\\
62.56	0.0044144604227118\\
62.57	0.00441317598610141\\
62.58	0.0044118913203936\\
62.59	0.00441060642616005\\
62.6	0.00440932130397383\\
62.61	0.00440803595440938\\
62.62	0.00440675037804254\\
62.63	0.00440546457545051\\
62.64	0.00440417854721189\\
62.65	0.00440289229390667\\
62.66	0.0044016058161162\\
62.67	0.00440031911442326\\
62.68	0.00439903218941198\\
62.69	0.00439774504166792\\
62.7	0.004396457671778\\
62.71	0.00439517008033055\\
62.72	0.00439388226791529\\
62.73	0.00439259423512334\\
62.74	0.00439130598254722\\
62.75	0.00439001751078085\\
62.76	0.00438872882041955\\
62.77	0.00438743991206003\\
62.78	0.00438615078630042\\
62.79	0.00438486144374025\\
62.8	0.00438357188498044\\
62.81	0.00438228211062336\\
62.82	0.00438099212127274\\
62.83	0.00437970191753373\\
62.84	0.00437841150001293\\
62.85	0.0043771208693183\\
62.86	0.00437583002605925\\
62.87	0.00437453897084658\\
62.88	0.00437324770429253\\
62.89	0.00437195622701073\\
62.9	0.00437066453961626\\
62.91	0.00436937264272559\\
62.92	0.00436808053695663\\
62.93	0.0043667882229287\\
62.94	0.00436549570126255\\
62.95	0.00436420297258036\\
62.96	0.00436291003750573\\
62.97	0.00436161689666367\\
62.98	0.00436032355068065\\
62.99	0.00435903000018455\\
63	0.00435773624580468\\
63.01	0.00435644228817179\\
63.02	0.00435514812791805\\
63.03	0.00435385376567707\\
63.04	0.0043525592020839\\
63.05	0.00435126443777502\\
63.06	0.00434996947338835\\
63.07	0.00434867430956324\\
63.08	0.0043473789469405\\
63.09	0.00434608338616235\\
63.1	0.00434478762787248\\
63.11	0.00434349167271599\\
63.12	0.00434219552133947\\
63.13	0.00434089917439092\\
63.14	0.00433960263251978\\
63.15	0.00433830589637695\\
63.16	0.00433700896661479\\
63.17	0.00433571184388708\\
63.18	0.00433441452884908\\
63.19	0.00433311702215748\\
63.2	0.00433181932447043\\
63.21	0.00433052143644752\\
63.22	0.00432922335874982\\
63.23	0.00432792509203983\\
63.24	0.00432662663698151\\
63.25	0.00432532799424028\\
63.26	0.00432402916448302\\
63.27	0.00432273014837808\\
63.28	0.00432143094659523\\
63.29	0.00432013155980573\\
63.3	0.0043188319886823\\
63.31	0.00431753223389913\\
63.32	0.00431623229613184\\
63.33	0.00431493217605755\\
63.34	0.00431363187435482\\
63.35	0.00431233139170369\\
63.36	0.00431103072878566\\
63.37	0.00430972988628368\\
63.38	0.0043084288648822\\
63.39	0.00430712766526713\\
63.4	0.00430582628812582\\
63.41	0.00430452473414714\\
63.42	0.00430322300402139\\
63.43	0.00430192109844037\\
63.44	0.00430061901809732\\
63.45	0.00429931676368698\\
63.46	0.00429801433590555\\
63.47	0.00429671173545072\\
63.48	0.00429540896302165\\
63.49	0.00429410601931896\\
63.5	0.00429280290504478\\
63.51	0.00429149962090267\\
63.52	0.00429019616759771\\
63.53	0.00428889254583645\\
63.54	0.0042875887563269\\
63.55	0.00428628479977858\\
63.56	0.00428498067690247\\
63.57	0.00428367638841103\\
63.58	0.00428237193501822\\
63.59	0.00428106731743946\\
63.6	0.00427976253639167\\
63.61	0.00427845759259326\\
63.62	0.00427715248676409\\
63.63	0.00427584721962555\\
63.64	0.00427454179190048\\
63.65	0.00427323620431324\\
63.66	0.00427193045758963\\
63.67	0.00427062455245698\\
63.68	0.00426931848964409\\
63.69	0.00426801226988125\\
63.7	0.00426670589390023\\
63.71	0.00426539936243431\\
63.72	0.00426409267621823\\
63.73	0.00426278583598826\\
63.74	0.00426147884248211\\
63.75	0.00426017169643903\\
63.76	0.00425886439859972\\
63.77	0.0042575569497064\\
63.78	0.00425624935050277\\
63.79	0.00425494160173403\\
63.8	0.00425363370414685\\
63.81	0.00425232565848943\\
63.82	0.00425101746551142\\
63.83	0.00424970912596401\\
63.84	0.00424840064059985\\
63.85	0.00424709201017308\\
63.86	0.00424578323543936\\
63.87	0.00424447431715584\\
63.88	0.00424316525608114\\
63.89	0.0042418560529754\\
63.9	0.00424054670860025\\
63.91	0.00423923722371881\\
63.92	0.0042379275990957\\
63.93	0.00423661783549704\\
63.94	0.00423530793369042\\
63.95	0.00423399789444496\\
63.96	0.00423268771853126\\
63.97	0.00423137740672142\\
63.98	0.00423006695978903\\
63.99	0.00422875637850917\\
64	0.00422744566365845\\
64.01	0.00422613481601493\\
64.02	0.00422482383635821\\
64.03	0.00422351272546934\\
64.04	0.00422220148413092\\
64.05	0.004220890113127\\
64.06	0.00421957861324315\\
64.07	0.00421826698526643\\
64.08	0.0042169552299854\\
64.09	0.00421564334819011\\
64.1	0.00421433134067211\\
64.11	0.00421301920822445\\
64.12	0.00421170695164166\\
64.13	0.00421039457171979\\
64.14	0.00420908206925636\\
64.15	0.0042077694450504\\
64.16	0.00420645669990245\\
64.17	0.0042051438346145\\
64.18	0.00420383084999008\\
64.19	0.00420251774683419\\
64.2	0.00420120452595332\\
64.21	0.00419989118815548\\
64.22	0.00419857773425014\\
64.23	0.0041972641650483\\
64.24	0.00419595048136242\\
64.25	0.00419463668400646\\
64.26	0.00419332277379589\\
64.27	0.00419200875154765\\
64.28	0.00419069461808019\\
64.29	0.00418938037421342\\
64.3	0.00418806602076878\\
64.31	0.00418675155856918\\
64.32	0.00418543698843902\\
64.33	0.00418412231120418\\
64.34	0.00418280752769203\\
64.35	0.00418149263873146\\
64.36	0.00418017764515279\\
64.37	0.00417886254778789\\
64.38	0.00417754734747007\\
64.39	0.00417623204503413\\
64.4	0.00417491664131637\\
64.41	0.00417360113715458\\
64.42	0.00417228553338801\\
64.43	0.00417096983085741\\
64.44	0.004169654030405\\
64.45	0.0041683381328745\\
64.46	0.00416702213911108\\
64.47	0.00416570604996141\\
64.48	0.00416438986627365\\
64.49	0.00416307358889741\\
64.5	0.00416175721868379\\
64.51	0.00416044075648537\\
64.52	0.00415912420315621\\
64.53	0.00415780755955182\\
64.54	0.00415649082652921\\
64.55	0.00415517400494685\\
64.56	0.00415385709566468\\
64.57	0.00415254009954413\\
64.58	0.00415122301744806\\
64.59	0.00414990585024084\\
64.6	0.00414858859878828\\
64.61	0.00414727126395766\\
64.62	0.00414595384661775\\
64.63	0.00414463634763874\\
64.64	0.00414331876789233\\
64.65	0.00414200110825164\\
64.66	0.00414068336959127\\
64.67	0.00413936555278729\\
64.68	0.00413804765871721\\
64.69	0.00413672968825999\\
64.7	0.00413541164229609\\
64.71	0.00413409352170736\\
64.72	0.00413277532737714\\
64.73	0.00413145706019023\\
64.74	0.00413013872103285\\
64.75	0.00412882031079269\\
64.76	0.00412750183035889\\
64.77	0.00412618328062201\\
64.78	0.00412486466247407\\
64.79	0.00412354597680854\\
64.8	0.00412222722452032\\
64.81	0.00412090840650575\\
64.82	0.00411958952366261\\
64.83	0.00411827057689012\\
64.84	0.00411695156708894\\
64.85	0.00411563249516113\\
64.86	0.00411431336201021\\
64.87	0.00411299416854114\\
64.88	0.00411167491566027\\
64.89	0.0041103556042754\\
64.9	0.00410903623529576\\
64.91	0.00410771680963199\\
64.92	0.00410639732819614\\
64.93	0.0041050777919017\\
64.94	0.00410375820166357\\
64.95	0.00410243855839805\\
64.96	0.00410111886302286\\
64.97	0.00409979911645715\\
64.98	0.00409847931962146\\
64.99	0.00409715947343772\\
65	0.0040958395788293\\
65.01	0.00409451963672095\\
65.02	0.00409319964803882\\
65.03	0.00409187961371045\\
65.04	0.00409055953466481\\
65.05	0.00408923941183223\\
65.06	0.00408791924614444\\
65.07	0.00408659903853456\\
65.08	0.00408527878993708\\
65.09	0.00408395850128791\\
65.1	0.00408263817352432\\
65.11	0.00408131780758495\\
65.12	0.00407999740440983\\
65.13	0.00407867696494035\\
65.14	0.00407735649011931\\
65.15	0.00407603598089083\\
65.16	0.00407471543820041\\
65.17	0.00407339486299495\\
65.18	0.00407207425622266\\
65.19	0.00407075361883313\\
65.2	0.00406943295177732\\
65.21	0.00406811225600752\\
65.22	0.00406679153247738\\
65.23	0.0040654707821419\\
65.24	0.00406415000595742\\
65.25	0.00406282920488163\\
65.26	0.00406150837987355\\
65.27	0.00406018753189352\\
65.28	0.00405886666190326\\
65.29	0.00405754577086577\\
65.3	0.00405622485974541\\
65.31	0.00405490392950784\\
65.32	0.00405358298112005\\
65.33	0.00405226201555036\\
65.34	0.00405094103376838\\
65.35	0.00404962003674507\\
65.36	0.00404829902545265\\
65.37	0.00404697800086466\\
65.38	0.00404565696395597\\
65.39	0.00404433591570271\\
65.4	0.00404301485708232\\
65.41	0.00404169378907355\\
65.42	0.00404037271265638\\
65.43	0.00403905162881214\\
65.44	0.00403773053852341\\
65.45	0.00403640944277404\\
65.46	0.00403508834254916\\
65.47	0.00403376723883518\\
65.48	0.00403244613261977\\
65.49	0.00403112502489184\\
65.5	0.00402980391664159\\
65.51	0.00402848280886047\\
65.52	0.00402716170254115\\
65.53	0.00402584059867759\\
65.54	0.00402451949826495\\
65.55	0.00402319840229967\\
65.56	0.00402187731177939\\
65.57	0.004020556227703\\
65.58	0.00401923515107061\\
65.59	0.00401791408288356\\
65.6	0.00401659302414439\\
65.61	0.00401527197585688\\
65.62	0.004013950939026\\
65.63	0.00401262991465792\\
65.64	0.00401130890376004\\
65.65	0.00400998790734092\\
65.66	0.00400866692641034\\
65.67	0.00400734596197926\\
65.68	0.00400602501505981\\
65.69	0.00400470408666533\\
65.7	0.0040033831778103\\
65.71	0.00400206228951039\\
65.72	0.00400074142278242\\
65.73	0.00399942057864438\\
65.74	0.00399809975811542\\
65.75	0.00399677896221584\\
65.76	0.00399545819196705\\
65.77	0.00399413744839165\\
65.78	0.00399281673251334\\
65.79	0.00399149604535698\\
65.8	0.00399017538794855\\
65.81	0.00398885476131511\\
65.82	0.00398753416648488\\
65.83	0.00398621360448719\\
65.84	0.00398489307635245\\
65.85	0.00398357258311218\\
65.86	0.003982252125799\\
65.87	0.0039809317054466\\
65.88	0.00397961132308978\\
65.89	0.00397829097976439\\
65.9	0.00397697067650738\\
65.91	0.00397565041435675\\
65.92	0.00397433019435157\\
65.93	0.00397301001753194\\
65.94	0.00397168988493904\\
65.95	0.00397036979761509\\
65.96	0.00396904975660334\\
65.97	0.00396772976294806\\
65.98	0.00396640981769457\\
65.99	0.0039650899218892\\
66	0.0039637700765793\\
66.01	0.00396245028281321\\
66.02	0.00396113054164029\\
66.03	0.00395981085411089\\
66.04	0.00395849122127635\\
66.05	0.00395717164418898\\
66.06	0.0039558521239021\\
66.07	0.00395453266146997\\
66.08	0.00395321325794783\\
66.09	0.00395189391439185\\
66.1	0.0039505746318592\\
66.11	0.00394925541140796\\
66.12	0.00394793625409714\\
66.13	0.00394661716098671\\
66.14	0.00394529813313755\\
66.15	0.00394397917161144\\
66.16	0.0039426602774711\\
66.17	0.00394134145178014\\
66.18	0.00394002269560306\\
66.19	0.00393870401000527\\
66.2	0.00393738539605304\\
66.21	0.00393606685481352\\
66.22	0.00393474838735472\\
66.23	0.00393342999474555\\
66.24	0.00393211167805572\\
66.25	0.00393079343835582\\
66.26	0.00392947527671727\\
66.27	0.00392815719421231\\
66.28	0.00392683919191402\\
66.29	0.00392552127089629\\
66.3	0.0039242034322338\\
66.31	0.00392288567700205\\
66.32	0.00392156800627734\\
66.33	0.00392025042113673\\
66.34	0.00391893292265806\\
66.35	0.00391761551191994\\
66.36	0.00391629819000178\\
66.37	0.00391498095798366\\
66.38	0.00391366381694648\\
66.39	0.00391234676797182\\
66.4	0.00391102981214204\\
66.41	0.00390971295054018\\
66.42	0.00390839618424998\\
66.43	0.00390707951435594\\
66.44	0.00390576294194317\\
66.45	0.00390444646809754\\
66.46	0.00390313009390556\\
66.47	0.0039018138204544\\
66.48	0.0039004976488319\\
66.49	0.00389918158012655\\
66.5	0.00389786561542747\\
66.51	0.00389654975582443\\
66.52	0.00389523400240779\\
66.53	0.00389391835626855\\
66.54	0.0038926028184983\\
66.55	0.00389128739018924\\
66.56	0.00388997207243412\\
66.57	0.00388865686632629\\
66.58	0.00388734177295966\\
66.59	0.00388602679342871\\
66.6	0.00388471192882844\\
66.61	0.00388339718025438\\
66.62	0.00388208254880263\\
66.63	0.00388076803556976\\
66.64	0.00387945364165287\\
66.65	0.00387813936814955\\
66.66	0.00387682521615788\\
66.67	0.0038755111867764\\
66.68	0.00387419728110413\\
66.69	0.00387288350024055\\
66.7	0.00387156984528557\\
66.71	0.00387025631733953\\
66.72	0.0038689429175032\\
66.73	0.00386762964687777\\
66.74	0.00386631650656483\\
66.75	0.00386500349766635\\
66.76	0.00386369062128468\\
66.77	0.00386237787852256\\
66.78	0.00386106527048305\\
66.79	0.0038597527982696\\
66.8	0.00385844046298595\\
66.81	0.00385712826573621\\
66.82	0.00385581620762477\\
66.83	0.00385450428975632\\
66.84	0.00385319251323586\\
66.85	0.00385188087916865\\
66.86	0.00385056938866024\\
66.87	0.00384925804281639\\
66.88	0.00384794684274316\\
66.89	0.00384663578954677\\
66.9	0.00384532488433373\\
66.91	0.00384401412821071\\
66.92	0.00384270352228459\\
66.93	0.00384139306766242\\
66.94	0.00384008276545143\\
66.95	0.00383877261675901\\
66.96	0.00383746262269268\\
66.97	0.00383615278436008\\
66.98	0.00383484310286901\\
66.99	0.00383353357932733\\
67	0.00383222421484303\\
67.01	0.00383091501052414\\
67.02	0.0038296059674788\\
67.03	0.00382829708681516\\
67.04	0.00382698836964144\\
67.05	0.00382567981706588\\
67.06	0.00382437143019671\\
67.07	0.0038230632101422\\
67.08	0.00382175515801055\\
67.09	0.00382044727490999\\
67.1	0.00381913956194867\\
67.11	0.00381783202023469\\
67.12	0.0038165246508761\\
67.13	0.00381521745498082\\
67.14	0.00381391043365673\\
67.15	0.00381260358801154\\
67.16	0.00381129691915287\\
67.17	0.00380999042818818\\
67.18	0.00380868411622478\\
67.19	0.00380737798436979\\
67.2	0.00380607203373019\\
67.21	0.00380476626541269\\
67.22	0.00380346068052385\\
67.23	0.00380215528016994\\
67.24	0.00380085006545702\\
67.25	0.00379954503749088\\
67.26	0.00379824019737702\\
67.27	0.00379693554622067\\
67.28	0.00379563108512671\\
67.29	0.00379432681519973\\
67.3	0.00379302273754396\\
67.31	0.00379171885326329\\
67.32	0.00379041516346122\\
67.33	0.00378911166924085\\
67.34	0.00378780837170491\\
67.35	0.00378650527195568\\
67.36	0.00378520237109499\\
67.37	0.00378389967022426\\
67.38	0.00378259717044439\\
67.39	0.0037812948728558\\
67.4	0.00377999277855842\\
67.41	0.00377869088865165\\
67.42	0.00377738920423435\\
67.43	0.0037760877264048\\
67.44	0.00377478645626073\\
67.45	0.00377348539489926\\
67.46	0.00377218454341693\\
67.47	0.0037708839029096\\
67.48	0.00376958347447252\\
67.49	0.00376828325920026\\
67.5	0.00376698325818672\\
67.51	0.00376568347252508\\
67.52	0.00376438390330782\\
67.53	0.00376308455162666\\
67.54	0.00376178541857257\\
67.55	0.00376048650523576\\
67.56	0.00375918781270562\\
67.57	0.00375788934207075\\
67.58	0.00375659109441889\\
67.59	0.00375529307083697\\
67.6	0.00375399527241098\\
67.61	0.00375269770022609\\
67.62	0.00375140035536651\\
67.63	0.00375010323891555\\
67.64	0.00374880635195554\\
67.65	0.00374750969556787\\
67.66	0.00374621327083292\\
67.67	0.00374491707883005\\
67.68	0.00374362112063762\\
67.69	0.00374232539733289\\
67.7	0.0037410299099921\\
67.71	0.00373973465969037\\
67.72	0.0037384396475017\\
67.73	0.00373714487449894\\
67.74	0.00373585034175384\\
67.75	0.0037345560503369\\
67.76	0.00373326200131747\\
67.77	0.00373196819576366\\
67.78	0.00373067463474234\\
67.79	0.0037293813193191\\
67.8	0.00372808825055826\\
67.81	0.00372679542952283\\
67.82	0.00372550285727448\\
67.83	0.00372421053487352\\
67.84	0.00372291846337889\\
67.85	0.00372162664384815\\
67.86	0.00372033507733738\\
67.87	0.00371904376490127\\
67.88	0.00371775270759303\\
67.89	0.00371646190646434\\
67.9	0.00371517136256541\\
67.91	0.00371388107694488\\
67.92	0.00371259105064985\\
67.93	0.0037113012847258\\
67.94	0.00371001178021664\\
67.95	0.00370872253816459\\
67.96	0.00370743355961026\\
67.97	0.00370614484559254\\
67.98	0.00370485639714863\\
67.99	0.00370356821531398\\
68	0.00370228030112229\\
68.01	0.00370099265560547\\
68.02	0.00369970527979364\\
68.03	0.00369841817471503\\
68.04	0.00369713134139607\\
68.05	0.00369584478086125\\
68.06	0.00369455849413319\\
68.07	0.00369327248223254\\
68.08	0.00369198674617799\\
68.09	0.00369070128698625\\
68.1	0.00368941610567199\\
68.11	0.00368813120324785\\
68.12	0.00368684658072438\\
68.13	0.00368556223911004\\
68.14	0.00368427817941116\\
68.15	0.00368299440263191\\
68.16	0.00368171090977429\\
68.17	0.00368042770183807\\
68.18	0.00367914477982078\\
68.19	0.00367786214471771\\
68.2	0.00367657979752182\\
68.21	0.00367529773922376\\
68.22	0.00367401597081182\\
68.23	0.00367273449327193\\
68.24	0.00367145330758758\\
68.25	0.00367017241473983\\
68.26	0.00366889181570725\\
68.27	0.00366761151146597\\
68.28	0.00366633150298951\\
68.29	0.00366505179124888\\
68.3	0.00366377237721249\\
68.31	0.00366249326184613\\
68.32	0.00366121444611292\\
68.33	0.00365993593097331\\
68.34	0.00365865771738505\\
68.35	0.00365737980630313\\
68.36	0.00365610219867977\\
68.37	0.00365482489546437\\
68.38	0.0036535478976035\\
68.39	0.00365227120604087\\
68.4	0.00365099482171726\\
68.41	0.00364971874557053\\
68.42	0.00364844297853558\\
68.43	0.00364716752154429\\
68.44	0.00364589237552552\\
68.45	0.00364461754140504\\
68.46	0.00364334302010555\\
68.47	0.00364206881254659\\
68.48	0.00364079491964454\\
68.49	0.00363952134231258\\
68.5	0.00363824808146066\\
68.51	0.00363697513799544\\
68.52	0.00363570251282028\\
68.53	0.00363443020683521\\
68.54	0.00363315822093687\\
68.55	0.00363188655601851\\
68.56	0.0036306152129699\\
68.57	0.00362934419267736\\
68.58	0.00362807349602366\\
68.59	0.00362680312388806\\
68.6	0.00362553307714617\\
68.61	0.00362426335667001\\
68.62	0.00362299396332794\\
68.63	0.0036217248979846\\
68.64	0.00362045616150089\\
68.65	0.00361918775473394\\
68.66	0.00361791967853708\\
68.67	0.00361665193375975\\
68.68	0.00361538452124754\\
68.69	0.00361411744184208\\
68.7	0.00361285069638105\\
68.71	0.0036115842856981\\
68.72	0.00361031821062287\\
68.73	0.00360905247198089\\
68.74	0.00360778707059356\\
68.75	0.00360652200727811\\
68.76	0.00360525728284761\\
68.77	0.00360399289811082\\
68.78	0.00360272885387225\\
68.79	0.00360146515093208\\
68.8	0.00360020179008611\\
68.81	0.00359893877212573\\
68.82	0.0035976760978379\\
68.83	0.00359641376800505\\
68.84	0.00359515178340509\\
68.85	0.00359389014481136\\
68.86	0.00359262885299257\\
68.87	0.00359136790871276\\
68.88	0.00359010731273127\\
68.89	0.00358884706580268\\
68.9	0.00358758716867678\\
68.91	0.00358632762209852\\
68.92	0.00358506842680796\\
68.93	0.00358380958354025\\
68.94	0.00358255109302553\\
68.95	0.00358129295598896\\
68.96	0.00358003517315061\\
68.97	0.00357877774522545\\
68.98	0.00357752067292329\\
68.99	0.00357626395694874\\
69	0.00357500759800117\\
69.01	0.00357375159677462\\
69.02	0.0035724959540176\\
69.03	0.00357124067091168\\
69.04	0.00356998574863691\\
69.05	0.00356873118837174\\
69.06	0.0035674769912931\\
69.07	0.00356622315857623\\
69.08	0.0035649696913948\\
69.09	0.00356371659092076\\
69.1	0.00356246385832442\\
69.11	0.00356121149477436\\
69.12	0.00355995950143741\\
69.13	0.00355870787947865\\
69.14	0.00355745663006138\\
69.15	0.00355620575434706\\
69.16	0.00355495525349534\\
69.17	0.00355370512866397\\
69.18	0.00355245538100886\\
69.19	0.00355120601168394\\
69.2	0.00354995702184123\\
69.21	0.00354870841263077\\
69.22	0.00354746018520061\\
69.23	0.00354621234069674\\
69.24	0.00354496488026315\\
69.25	0.0035437178050417\\
69.26	0.00354247111617218\\
69.27	0.0035412248147922\\
69.28	0.00353997890203727\\
69.29	0.00353873337904064\\
69.3	0.00353748824693339\\
69.31	0.00353624350684431\\
69.32	0.00353499915989996\\
69.33	0.00353375520722456\\
69.34	0.00353251164994\\
69.35	0.00353126848916581\\
69.36	0.00353002572601914\\
69.37	0.00352878336161471\\
69.38	0.00352754139706476\\
69.39	0.0035262998334791\\
69.4	0.00352505867196498\\
69.41	0.00352381791362714\\
69.42	0.00352257743880314\\
69.43	0.00352133724319583\\
69.44	0.00352009732923924\\
69.45	0.00351885769937576\\
69.46	0.00351761835605609\\
69.47	0.00351637930173932\\
69.48	0.0035151405388929\\
69.49	0.00351390206999276\\
69.5	0.00351266389752323\\
69.51	0.00351142602397715\\
69.52	0.0035101884518559\\
69.53	0.00350895118366936\\
69.54	0.00350771422193601\\
69.55	0.00350647756918293\\
69.56	0.00350524122794585\\
69.57	0.00350400520076913\\
69.58	0.00350276949020587\\
69.59	0.00350153409881785\\
69.6	0.00350029902917564\\
69.61	0.0034990642838586\\
69.62	0.00349782986545488\\
69.63	0.00349659577656149\\
69.64	0.00349536201978434\\
69.65	0.00349412859773824\\
69.66	0.00349289551304693\\
69.67	0.00349166276834314\\
69.68	0.0034904303662686\\
69.69	0.00348919830947408\\
69.7	0.00348796660061941\\
69.71	0.00348673524237354\\
69.72	0.00348550423741454\\
69.73	0.00348427358842963\\
69.74	0.00348304329811527\\
69.75	0.00348181336917711\\
69.76	0.00348058380433008\\
69.77	0.0034793546062984\\
69.78	0.00347812577781562\\
69.79	0.00347689732162466\\
69.8	0.00347566924047781\\
69.81	0.0034744415371368\\
69.82	0.00347321421437284\\
69.83	0.0034719872749666\\
69.84	0.00347076072170829\\
69.85	0.00346953455739767\\
69.86	0.00346830878484414\\
69.87	0.00346708340686665\\
69.88	0.00346585842629388\\
69.89	0.00346463384596417\\
69.9	0.00346340966872561\\
69.91	0.00346218589743604\\
69.92	0.00346096253496309\\
69.93	0.00345973958418424\\
69.94	0.00345851704798685\\
69.95	0.00345729492926817\\
69.96	0.00345607323093537\\
69.97	0.00345485195590561\\
69.98	0.00345363110710606\\
69.99	0.00345241068747394\\
70	0.00345119069995654\\
70.01	0.00344997114751124\\
70.02	0.00344875203310563\\
70.03	0.00344753335971744\\
70.04	0.00344631513033461\\
70.05	0.00344509734795541\\
70.06	0.00344388001558832\\
70.07	0.00344266313625221\\
70.08	0.00344144671297628\\
70.09	0.00344023074880018\\
70.1	0.00343901524677394\\
70.11	0.00343780020995814\\
70.12	0.0034365856414238\\
70.13	0.00343537154425256\\
70.14	0.00343415792153661\\
70.15	0.00343294477637878\\
70.16	0.00343173211189257\\
70.17	0.00343051993120217\\
70.18	0.00342930823744252\\
70.19	0.00342809703375934\\
70.2	0.00342688632330915\\
70.21	0.00342567610925937\\
70.22	0.00342446639478826\\
70.23	0.00342325718308504\\
70.24	0.00342204847734992\\
70.25	0.00342084028079408\\
70.26	0.00341963259663977\\
70.27	0.00341842542812033\\
70.28	0.00341721877848023\\
70.29	0.00341601265097511\\
70.3	0.0034148070488718\\
70.31	0.00341360197544839\\
70.32	0.00341239743399425\\
70.33	0.00341119342781009\\
70.34	0.00340998996020798\\
70.35	0.00340878703451139\\
70.36	0.00340758465405524\\
70.37	0.00340638282218596\\
70.38	0.00340518154226147\\
70.39	0.0034039808176513\\
70.4	0.00340278065173657\\
70.41	0.00340158104791005\\
70.42	0.00340038200957622\\
70.43	0.00339918354015128\\
70.44	0.00339798564306321\\
70.45	0.00339678832175183\\
70.46	0.00339559157966878\\
70.47	0.00339439542027765\\
70.48	0.00339319984705394\\
70.49	0.00339200486348516\\
70.5	0.00339081047307083\\
70.51	0.00338961667932256\\
70.52	0.00338842348576408\\
70.53	0.00338723089593126\\
70.54	0.00338603891337219\\
70.55	0.00338484754164719\\
70.56	0.00338365678432889\\
70.57	0.00338246664500223\\
70.58	0.00338127712726455\\
70.59	0.00338008823472559\\
70.6	0.00337889997100757\\
70.61	0.0033777123397452\\
70.62	0.00337652534458578\\
70.63	0.00337533898918918\\
70.64	0.00337415327722792\\
70.65	0.00337296821238719\\
70.66	0.00337178379836496\\
70.67	0.00337060003887192\\
70.68	0.00336941693763164\\
70.69	0.00336823449838051\\
70.7	0.00336705272486788\\
70.71	0.00336587162085603\\
70.72	0.00336469119012025\\
70.73	0.00336351143644891\\
70.74	0.00336233236364344\\
70.75	0.00336115397551844\\
70.76	0.0033599762759017\\
70.77	0.00335879926863426\\
70.78	0.00335762295757043\\
70.79	0.00335644734657786\\
70.8	0.00335527243953758\\
70.81	0.00335409824034406\\
70.82	0.00335292475290524\\
70.83	0.00335175198114257\\
70.84	0.0033505799289911\\
70.85	0.00334940860039948\\
70.86	0.00334823799933003\\
70.87	0.00334706812975882\\
70.88	0.00334589899567564\\
70.89	0.00334473060108411\\
70.9	0.00334356295000174\\
70.91	0.00334239604645992\\
70.92	0.00334122989450402\\
70.93	0.00334006449819343\\
70.94	0.00333889986160158\\
70.95	0.00333773598881604\\
70.96	0.00333657288393851\\
70.97	0.00333541055108493\\
70.98	0.00333424899438549\\
70.99	0.0033330882179847\\
71	0.00333192822604142\\
71.01	0.00333076902272894\\
71.02	0.00332961061223501\\
71.03	0.0033284529987619\\
71.04	0.00332729618652645\\
71.05	0.00332614017976011\\
71.06	0.00332498498270902\\
71.07	0.00332383059963403\\
71.08	0.00332267703481077\\
71.09	0.0033215242925297\\
71.1	0.00332037237709616\\
71.11	0.00331922129283043\\
71.12	0.00331807104406775\\
71.13	0.00331692163515845\\
71.14	0.0033157730704679\\
71.15	0.00331462535437664\\
71.16	0.00331347849128042\\
71.17	0.00331233248559022\\
71.18	0.00331118734173234\\
71.19	0.00331004306414845\\
71.2	0.00330889965729562\\
71.21	0.00330775712564638\\
71.22	0.00330661547368882\\
71.23	0.00330547470592657\\
71.24	0.00330433482687892\\
71.25	0.00330319584108085\\
71.26	0.00330205775308306\\
71.27	0.00330092056745206\\
71.28	0.00329978428877024\\
71.29	0.00329864892163588\\
71.3	0.00329751447066321\\
71.31	0.00329638094048252\\
71.32	0.00329524833574015\\
71.33	0.00329411666109859\\
71.34	0.00329298592123653\\
71.35	0.00329185612084889\\
71.36	0.00329072726464692\\
71.37	0.0032895993573582\\
71.38	0.00328847240372678\\
71.39	0.00328734640851314\\
71.4	0.00328622137649433\\
71.41	0.00328509731246398\\
71.42	0.00328397422123238\\
71.43	0.00328285210762653\\
71.44	0.00328173097649018\\
71.45	0.00328061083268395\\
71.46	0.0032794916810853\\
71.47	0.00327837352658868\\
71.48	0.0032772563741055\\
71.49	0.00327614022856428\\
71.5	0.00327502509491063\\
71.51	0.00327391097810736\\
71.52	0.00327279788313451\\
71.53	0.00327168581498945\\
71.54	0.00327057477868688\\
71.55	0.00326946477925895\\
71.56	0.00326835582175529\\
71.57	0.00326724791124308\\
71.58	0.00326614105280708\\
71.59	0.00326503525154974\\
71.6	0.00326393051259125\\
71.61	0.00326282684106956\\
71.62	0.0032617242421405\\
71.63	0.00326062269007743\\
71.64	0.0032595211927667\\
71.65	0.00325841975071102\\
71.66	0.00325731836441211\\
71.67	0.00325621703437073\\
71.68	0.00325511576108665\\
71.69	0.00325401454505862\\
71.7	0.00325291338678441\\
71.71	0.00325181228676074\\
71.72	0.0032507112454833\\
71.73	0.00324961026344674\\
71.74	0.00324850934114464\\
71.75	0.00324740847906952\\
71.76	0.0032463076777128\\
71.77	0.00324520693756483\\
71.78	0.00324410625911483\\
71.79	0.00324300564285091\\
71.8	0.00324190508926003\\
71.81	0.00324080459882805\\
71.82	0.00323970417203962\\
71.83	0.00323860380937825\\
71.84	0.00323750351132626\\
71.85	0.00323640327836478\\
71.86	0.00323530311097372\\
71.87	0.00323420300963179\\
71.88	0.00323310297481644\\
71.89	0.00323200300700391\\
71.9	0.00323090310666915\\
71.91	0.00322980327428585\\
71.92	0.00322870351032641\\
71.93	0.00322760381526193\\
71.94	0.00322650418956221\\
71.95	0.00322540463369571\\
71.96	0.00322430514812955\\
71.97	0.00322320573332952\\
71.98	0.00322210638976002\\
71.99	0.00322100711788408\\
72	0.00321990791816335\\
72.01	0.00321880879105804\\
72.02	0.00321770973702696\\
72.03	0.0032166107565275\\
72.04	0.00321551185001558\\
72.05	0.00321441301794564\\
72.06	0.00321331426077069\\
72.07	0.00321221557894221\\
72.08	0.00321111697291018\\
72.09	0.00321001844312308\\
72.1	0.00320891999002782\\
72.11	0.0032078216140698\\
72.12	0.00320672331569283\\
72.13	0.00320562509533913\\
72.14	0.00320452695344937\\
72.15	0.00320342889046257\\
72.16	0.00320233090681614\\
72.17	0.00320123300294587\\
72.18	0.00320013517928584\\
72.19	0.00319903743626854\\
72.2	0.0031979397743247\\
72.21	0.0031968421938834\\
72.22	0.00319574469537198\\
72.23	0.00319464727921606\\
72.24	0.00319354994583951\\
72.25	0.00319245269566444\\
72.26	0.00319135552911117\\
72.27	0.00319025844659824\\
72.28	0.00318916144854236\\
72.29	0.00318806453535845\\
72.3	0.00318696770745954\\
72.31	0.00318587096525685\\
72.32	0.00318477430915966\\
72.33	0.00318367773957543\\
72.34	0.00318258125690968\\
72.35	0.00318148486156597\\
72.36	0.00318038855394599\\
72.37	0.00317929233444941\\
72.38	0.00317819620347396\\
72.39	0.00317710016141534\\
72.4	0.0031760042086673\\
72.41	0.0031749083456215\\
72.42	0.00317381257266759\\
72.43	0.00317271689019317\\
72.44	0.00317162129858372\\
72.45	0.00317052579822265\\
72.46	0.00316943038949127\\
72.47	0.00316833507276872\\
72.48	0.00316723984843202\\
72.49	0.003166144716856\\
72.5	0.00316504967841333\\
72.51	0.00316395473347446\\
72.52	0.00316285988240761\\
72.53	0.00316176512557878\\
72.54	0.00316067046335169\\
72.55	0.0031595758960878\\
72.56	0.00315848142414627\\
72.57	0.00315738704788393\\
72.58	0.00315629276765528\\
72.59	0.00315519858381249\\
72.6	0.00315410449670534\\
72.61	0.00315301050668122\\
72.62	0.00315191661408509\\
72.63	0.00315082281925954\\
72.64	0.00314972912254464\\
72.65	0.00314863552427804\\
72.66	0.00314754202479487\\
72.67	0.00314644862442778\\
72.68	0.00314535532350687\\
72.69	0.0031442621223597\\
72.7	0.00314316902131126\\
72.71	0.00314207602068396\\
72.72	0.00314098312079759\\
72.73	0.00313989032196932\\
72.74	0.00313879762451366\\
72.75	0.00313770502874246\\
72.76	0.00313661253496488\\
72.77	0.00313552014348734\\
72.78	0.00313442785461358\\
72.79	0.00313333566864453\\
72.8	0.00313224358587838\\
72.81	0.00313115160661052\\
72.82	0.00313005973113349\\
72.83	0.00312896795973704\\
72.84	0.00312787629270803\\
72.85	0.00312678473033044\\
72.86	0.00312569327288533\\
72.87	0.00312460192065087\\
72.88	0.00312351067390225\\
72.89	0.00312241953291169\\
72.9	0.00312132849794844\\
72.91	0.0031202375692787\\
72.92	0.00311914674716566\\
72.93	0.00311805603186943\\
72.94	0.00311696542364705\\
72.95	0.00311587492275244\\
72.96	0.0031147845294364\\
72.97	0.00311369424394656\\
72.98	0.00311260406652738\\
72.99	0.00311151399742013\\
73	0.00311042403686285\\
73.01	0.00310933418509031\\
73.02	0.00310824444233405\\
73.03	0.00310715480882227\\
73.04	0.00310606528477989\\
73.05	0.00310497587042845\\
73.06	0.00310388656598613\\
73.07	0.00310279737166775\\
73.08	0.00310170828768466\\
73.09	0.0031006193142448\\
73.1	0.00309953045155264\\
73.11	0.00309844169980915\\
73.12	0.00309735305921177\\
73.13	0.00309626452995442\\
73.14	0.00309517611222745\\
73.15	0.00309408780621759\\
73.16	0.00309299961210797\\
73.17	0.00309191153007808\\
73.18	0.00309082356030372\\
73.19	0.003089735702957\\
73.2	0.00308864795820631\\
73.21	0.00308756032621628\\
73.22	0.00308647280714777\\
73.23	0.00308538540115782\\
73.24	0.00308429810839967\\
73.25	0.00308321092902268\\
73.26	0.00308212386317231\\
73.27	0.00308103691099014\\
73.28	0.00307995007261379\\
73.29	0.00307886334817692\\
73.3	0.0030777767378092\\
73.31	0.00307669024163625\\
73.32	0.00307560385977966\\
73.33	0.00307451759235695\\
73.34	0.00307343143948152\\
73.35	0.00307234540126261\\
73.36	0.00307125947780534\\
73.37	0.00307017366921059\\
73.38	0.00306908797557508\\
73.39	0.0030680023969912\\
73.4	0.00306691693354711\\
73.41	0.00306583158532665\\
73.42	0.00306474635240932\\
73.43	0.00306366123487023\\
73.44	0.00306257623278012\\
73.45	0.00306149134620527\\
73.46	0.00306040657520753\\
73.47	0.00305932191984425\\
73.48	0.00305823738016823\\
73.49	0.00305715295622775\\
73.5	0.0030560686480665\\
73.51	0.00305498445572355\\
73.52	0.00305390037923334\\
73.53	0.0030528164186256\\
73.54	0.00305173257392538\\
73.55	0.00305064884515299\\
73.56	0.00304956523232394\\
73.57	0.00304848173544898\\
73.58	0.00304739835453399\\
73.59	0.00304631508957997\\
73.6	0.00304523194058308\\
73.61	0.00304414890753446\\
73.62	0.00304306599042035\\
73.63	0.00304198318922196\\
73.64	0.00304090050391548\\
73.65	0.00303981793447201\\
73.66	0.00303873548085757\\
73.67	0.00303765314303304\\
73.68	0.00303657092095412\\
73.69	0.00303548881457132\\
73.7	0.00303440682382992\\
73.71	0.00303332494866989\\
73.72	0.00303224318902593\\
73.73	0.00303116154482739\\
73.74	0.00303008001599822\\
73.75	0.00302899860245699\\
73.76	0.00302791730411679\\
73.77	0.00302683612088526\\
73.78	0.00302575505266448\\
73.79	0.003024674099351\\
73.8	0.00302359326083577\\
73.81	0.00302251253700411\\
73.82	0.00302143192773568\\
73.83	0.00302035143290441\\
73.84	0.00301927105237853\\
73.85	0.00301819078602046\\
73.86	0.00301711063368683\\
73.87	0.00301603059522837\\
73.88	0.00301495067048998\\
73.89	0.00301387085931058\\
73.9	0.00301279116152316\\
73.91	0.00301171157695466\\
73.92	0.00301063210542602\\
73.93	0.00300955274675207\\
73.94	0.0030084735007415\\
73.95	0.00300739436719687\\
73.96	0.00300631534591452\\
73.97	0.00300523643668456\\
73.98	0.00300415763929078\\
73.99	0.00300307895351067\\
74	0.00300200037911537\\
74.01	0.00300092191586959\\
74.02	0.00299984356353162\\
74.03	0.00299876532185323\\
74.04	0.00299768719057971\\
74.05	0.00299660916944975\\
74.06	0.00299553125819545\\
74.07	0.00299445345654227\\
74.08	0.00299337576420897\\
74.09	0.0029922981809076\\
74.1	0.00299122070634344\\
74.11	0.00299014334021492\\
74.12	0.00298906608221368\\
74.13	0.00298798893202443\\
74.14	0.00298691188932494\\
74.15	0.00298583495378603\\
74.16	0.00298475812507148\\
74.17	0.002983681402838\\
74.18	0.00298260478673521\\
74.19	0.00298152827640557\\
74.2	0.00298045187148437\\
74.21	0.00297937557159963\\
74.22	0.0029782993763721\\
74.23	0.00297722328541523\\
74.24	0.00297614729833508\\
74.25	0.00297507141473029\\
74.26	0.00297399563419207\\
74.27	0.0029729199563041\\
74.28	0.00297184438064255\\
74.29	0.00297076890677594\\
74.3	0.0029696935342652\\
74.31	0.00296861826266357\\
74.32	0.00296754309151654\\
74.33	0.00296646802036184\\
74.34	0.00296539304872937\\
74.35	0.00296431817614117\\
74.36	0.00296324340211135\\
74.37	0.00296216872614608\\
74.38	0.00296109414774349\\
74.39	0.00296001966639368\\
74.4	0.00295894528157861\\
74.41	0.00295787099277213\\
74.42	0.00295679679943986\\
74.43	0.00295572270103916\\
74.44	0.00295464869701913\\
74.45	0.00295357478682048\\
74.46	0.00295250096987556\\
74.47	0.00295142724560825\\
74.48	0.00295035361343394\\
74.49	0.0029492800727595\\
74.5	0.00294820662298316\\
74.51	0.00294713326349455\\
74.52	0.00294605999367459\\
74.53	0.00294498681289546\\
74.54	0.00294391372052052\\
74.55	0.00294284071590431\\
74.56	0.00294176779839247\\
74.57	0.00294069496732168\\
74.58	0.00293962222201962\\
74.59	0.00293854956180493\\
74.6	0.00293747698598713\\
74.61	0.00293640449386659\\
74.62	0.00293533208473446\\
74.63	0.00293425975787264\\
74.64	0.00293318751255369\\
74.65	0.00293211534804084\\
74.66	0.00293104326358786\\
74.67	0.00292997125843905\\
74.68	0.00292889933182918\\
74.69	0.00292782748298343\\
74.7	0.00292675571111735\\
74.71	0.00292568401543678\\
74.72	0.00292461239513782\\
74.73	0.00292354084940673\\
74.74	0.00292246937741994\\
74.75	0.00292139797834395\\
74.76	0.00292032665133528\\
74.77	0.00291925539554042\\
74.78	0.00291818421009578\\
74.79	0.00291711309412759\\
74.8	0.00291604204675191\\
74.81	0.00291497106707452\\
74.82	0.00291390015419088\\
74.83	0.00291282930718608\\
74.84	0.00291175852513477\\
74.85	0.00291068780710109\\
74.86	0.00290961715213864\\
74.87	0.00290854655929041\\
74.88	0.00290747602758868\\
74.89	0.00290640555605504\\
74.9	0.00290533514370025\\
74.91	0.00290426478952425\\
74.92	0.00290319449251602\\
74.93	0.00290212425165359\\
74.94	0.00290105406590396\\
74.95	0.00289998393422301\\
74.96	0.00289891385555545\\
74.97	0.00289784382883479\\
74.98	0.00289677385298324\\
74.99	0.00289570392691166\\
75	0.00289463404951948\\
75.01	0.00289356421969468\\
75.02	0.00289249443631369\\
75.03	0.00289142469824132\\
75.04	0.00289035500433074\\
75.05	0.00288928535342335\\
75.06	0.00288821574434879\\
75.07	0.00288714617592479\\
75.08	0.00288607664695719\\
75.09	0.00288500715623981\\
75.1	0.00288393770255441\\
75.11	0.00288286828467064\\
75.12	0.00288179890134594\\
75.13	0.00288072955132548\\
75.14	0.0028796602333421\\
75.15	0.00287859094611627\\
75.16	0.00287752168835595\\
75.17	0.00287645245875661\\
75.18	0.00287538325600107\\
75.19	0.00287431407875952\\
75.2	0.00287324492568937\\
75.21	0.00287217579543526\\
75.22	0.00287110668662891\\
75.23	0.0028700375978891\\
75.24	0.00286896852782159\\
75.25	0.00286789947501905\\
75.26	0.00286683043806097\\
75.27	0.00286576141551362\\
75.28	0.00286469240592996\\
75.29	0.00286362340784953\\
75.3	0.00286255441979848\\
75.31	0.00286148544028938\\
75.32	0.00286041646782122\\
75.33	0.0028593475008793\\
75.34	0.00285827853793519\\
75.35	0.00285720957744664\\
75.36	0.00285614061785748\\
75.37	0.00285507165759757\\
75.38	0.00285400269508274\\
75.39	0.00285293372871468\\
75.4	0.00285186475688087\\
75.41	0.00285079577795454\\
75.42	0.00284972679029454\\
75.43	0.0028486577922453\\
75.44	0.00284758878213673\\
75.45	0.00284651975828417\\
75.46	0.00284545071898827\\
75.47	0.00284438166253496\\
75.48	0.00284331258719533\\
75.49	0.00284224349122557\\
75.5	0.00284117437286689\\
75.51	0.00284010523034544\\
75.52	0.00283903606187221\\
75.53	0.00283796686564297\\
75.54	0.00283689763983821\\
75.55	0.00283582838262299\\
75.56	0.00283475909214693\\
75.57	0.00283368976654408\\
75.58	0.00283262040393286\\
75.59	0.00283155100241599\\
75.6	0.00283048156008034\\
75.61	0.00282941207499695\\
75.62	0.00282834254522085\\
75.63	0.00282727296879103\\
75.64	0.00282620334373033\\
75.65	0.00282513366804536\\
75.66	0.00282406393972645\\
75.67	0.00282299415674748\\
75.68	0.00282192431706589\\
75.69	0.00282085441862251\\
75.7	0.00281978445934153\\
75.71	0.00281871443713041\\
75.72	0.00281764434987971\\
75.73	0.00281657419546314\\
75.74	0.00281550397173734\\
75.75	0.00281443367654186\\
75.76	0.00281336330769907\\
75.77	0.00281229286301404\\
75.78	0.00281122234027447\\
75.79	0.00281015173725058\\
75.8	0.00280908105169505\\
75.81	0.00280801028134288\\
75.82	0.00280693942391137\\
75.83	0.00280586847709993\\
75.84	0.00280479743859008\\
75.85	0.00280372630604529\\
75.86	0.00280265507711094\\
75.87	0.00280158374941417\\
75.88	0.0028005123205638\\
75.89	0.00279944078815029\\
75.9	0.00279836914974556\\
75.91	0.00279729740290296\\
75.92	0.00279622554515712\\
75.93	0.00279515357402389\\
75.94	0.00279408148700024\\
75.95	0.00279300928394473\\
75.96	0.00279193696484594\\
75.97	0.00279086452969186\\
75.98	0.00278979197846995\\
75.99	0.00278871931116706\\
76	0.00278764652776951\\
76.01	0.00278657362826304\\
76.02	0.00278550061263281\\
76.03	0.00278442748086342\\
76.04	0.00278335423293894\\
76.05	0.00278228086884283\\
76.06	0.00278120738855803\\
76.07	0.00278013379206689\\
76.08	0.00277906007935124\\
76.09	0.0027779862503923\\
76.1	0.00277691230517079\\
76.11	0.00277583824366684\\
76.12	0.00277476406586006\\
76.13	0.00277368977172948\\
76.14	0.00277261536125359\\
76.15	0.00277154083441034\\
76.16	0.00277046619117712\\
76.17	0.00276939143153081\\
76.18	0.0027683165554477\\
76.19	0.00276724156290358\\
76.2	0.00276616645387367\\
76.21	0.00276509122833268\\
76.22	0.00276401588625476\\
76.23	0.00276294042761356\\
76.24	0.00276186485238216\\
76.25	0.00276078916053313\\
76.26	0.00275971335203852\\
76.27	0.00275863742686983\\
76.28	0.00275756138499806\\
76.29	0.00275648522639368\\
76.3	0.00275540895102663\\
76.31	0.00275433255886636\\
76.32	0.00275325604988176\\
76.33	0.00275217942404126\\
76.34	0.00275110268131273\\
76.35	0.00275002582166358\\
76.36	0.00274894884506066\\
76.37	0.00274787175147036\\
76.38	0.00274679454085854\\
76.39	0.00274571721319058\\
76.4	0.00274463976843136\\
76.41	0.00274356220654524\\
76.42	0.00274248452749613\\
76.43	0.00274140673124742\\
76.44	0.00274032881776203\\
76.45	0.00273925078700237\\
76.46	0.0027381726389304\\
76.47	0.00273709437350759\\
76.48	0.00273601599069493\\
76.49	0.00273493749045294\\
76.5	0.00273385887274167\\
76.51	0.0027327801375207\\
76.52	0.00273170128474914\\
76.53	0.00273062231438565\\
76.54	0.00272954322638843\\
76.55	0.00272846402071522\\
76.56	0.0027273846973233\\
76.57	0.00272630525616952\\
76.58	0.00272522569721027\\
76.59	0.0027241460204015\\
76.6	0.00272306622569872\\
76.61	0.00272198631305701\\
76.62	0.002720906282431\\
76.63	0.0027198261337749\\
76.64	0.00271874586704251\\
76.65	0.00271766548218719\\
76.66	0.00271658497916188\\
76.67	0.0027155043579191\\
76.68	0.00271442361841098\\
76.69	0.00271334276058923\\
76.7	0.00271226178440513\\
76.71	0.00271118068980961\\
76.72	0.00271009947675316\\
76.73	0.0027090181451859\\
76.74	0.00270793669505756\\
76.75	0.00270685512631747\\
76.76	0.00270577343891457\\
76.77	0.00270469163279748\\
76.78	0.00270360970791437\\
76.79	0.00270252766421311\\
76.8	0.00270144550164114\\
76.81	0.0027003632201456\\
76.82	0.00269928081967324\\
76.83	0.00269819830017046\\
76.84	0.00269711566158332\\
76.85	0.00269603290385754\\
76.86	0.0026949500269385\\
76.87	0.00269386703077123\\
76.88	0.00269278391530046\\
76.89	0.00269170068047058\\
76.9	0.00269061732622565\\
76.91	0.00268953385250943\\
76.92	0.00268845025926537\\
76.93	0.0026873665464366\\
76.94	0.00268628271396598\\
76.95	0.00268519876179603\\
76.96	0.00268411468986903\\
76.97	0.00268303049812693\\
76.98	0.00268194618651143\\
76.99	0.00268086175496395\\
77	0.00267977720342564\\
77.01	0.00267869253183736\\
77.02	0.00267760774013976\\
77.03	0.00267652282827321\\
77.04	0.00267543779617782\\
77.05	0.00267435264379348\\
77.06	0.00267326737105984\\
77.07	0.00267218197791631\\
77.08	0.00267109646430209\\
77.09	0.00267001083015614\\
77.1	0.00266892507541722\\
77.11	0.0026678392000239\\
77.12	0.00266675320391452\\
77.13	0.00266566708702724\\
77.14	0.00266458084930002\\
77.15	0.00266349449067066\\
77.16	0.00266240801107676\\
77.17	0.00266132141045577\\
77.18	0.00266023468874495\\
77.19	0.00265914784588143\\
77.2	0.00265806088180219\\
77.21	0.00265697379644403\\
77.22	0.00265588658974367\\
77.23	0.00265479926163765\\
77.24	0.0026537118120624\\
77.25	0.00265262424095424\\
77.26	0.0026515365482494\\
77.27	0.00265044873388396\\
77.28	0.00264936079779393\\
77.29	0.00264827273991526\\
77.3	0.00264718456018377\\
77.31	0.00264609625853523\\
77.32	0.00264500783490534\\
77.33	0.00264391928922974\\
77.34	0.00264283062144403\\
77.35	0.00264174183148375\\
77.36	0.00264065291928441\\
77.37	0.00263956388478149\\
77.38	0.00263847472791045\\
77.39	0.00263738544860675\\
77.4	0.00263629604680582\\
77.41	0.00263520652244311\\
77.42	0.00263411687545409\\
77.43	0.00263302710577422\\
77.44	0.00263193721333902\\
77.45	0.00263084719808402\\
77.46	0.00262975705994481\\
77.47	0.00262866679885704\\
77.48	0.00262757641475641\\
77.49	0.00262648590757867\\
77.5	0.0026253952772597\\
77.51	0.00262430452373543\\
77.52	0.00262321364694188\\
77.53	0.00262212264681521\\
77.54	0.00262103152329168\\
77.55	0.00261994027630766\\
77.56	0.00261884890579966\\
77.57	0.00261775741170435\\
77.58	0.00261666579395854\\
77.59	0.0026155740524992\\
77.6	0.00261448218726347\\
77.61	0.00261339019818867\\
77.62	0.00261229808521232\\
77.63	0.00261120584827213\\
77.64	0.00261011348730604\\
77.65	0.00260902100225218\\
77.66	0.00260792839304894\\
77.67	0.00260683565963492\\
77.68	0.002605742801949\\
77.69	0.00260464981993031\\
77.7	0.00260355671351825\\
77.71	0.00260246348265251\\
77.72	0.00260137012727306\\
77.73	0.00260027664732019\\
77.74	0.00259918304273449\\
77.75	0.0025980893134569\\
77.76	0.00259699545942865\\
77.77	0.00259590148059138\\
77.78	0.00259480737688702\\
77.79	0.00259371314825795\\
77.8	0.00259261879464685\\
77.81	0.00259152431599687\\
77.82	0.0025904297122515\\
77.83	0.00258933498335468\\
77.84	0.00258824012925079\\
77.85	0.00258714514988461\\
77.86	0.00258605004520142\\
77.87	0.00258495481514693\\
77.88	0.00258385945966734\\
77.89	0.00258276397870933\\
77.9	0.00258166837222011\\
77.91	0.00258057264014736\\
77.92	0.00257947678243933\\
77.93	0.00257838079904478\\
77.94	0.00257728468991304\\
77.95	0.002576188454994\\
77.96	0.00257509209423812\\
77.97	0.00257399560759646\\
77.98	0.00257289899502069\\
77.99	0.00257180225646308\\
78	0.00257070539187656\\
78.01	0.00256960840121469\\
78.02	0.00256851128443168\\
78.03	0.00256741404148242\\
78.04	0.00256631667232251\\
78.05	0.00256521917690821\\
78.06	0.00256412155519652\\
78.07	0.00256302380714518\\
78.08	0.00256192593271265\\
78.09	0.00256082793185816\\
78.1	0.0025597298045417\\
78.11	0.00255863155072408\\
78.12	0.0025575331703669\\
78.13	0.00255643466343254\\
78.14	0.00255533602988427\\
78.15	0.00255423726968617\\
78.16	0.00255313838280321\\
78.17	0.00255203936920121\\
78.18	0.00255094022884691\\
78.19	0.00254984096170795\\
78.2	0.0025487415677529\\
78.21	0.00254764204695127\\
78.22	0.00254654239927352\\
78.23	0.0025454426246911\\
78.24	0.00254434272317644\\
78.25	0.002543242694703\\
78.26	0.00254214253924522\\
78.27	0.00254104225677862\\
78.28	0.00253994184727977\\
78.29	0.00253884131072631\\
78.3	0.00253774064709698\\
78.31	0.0025366398563716\\
78.32	0.00253553893853115\\
78.33	0.00253443789355775\\
78.34	0.00253333672143467\\
78.35	0.00253223542214638\\
78.36	0.0025311339956785\\
78.37	0.00253003244201794\\
78.38	0.00252893076115277\\
78.39	0.00252782895307236\\
78.4	0.00252672701776734\\
78.41	0.00252562495522963\\
78.42	0.00252452276545245\\
78.43	0.00252342044843035\\
78.44	0.00252231800415925\\
78.45	0.00252121543263639\\
78.46	0.00252011273386044\\
78.47	0.00251900990783145\\
78.48	0.00251790695455091\\
78.49	0.00251680387402174\\
78.5	0.00251570066624831\\
78.51	0.00251459733123653\\
78.52	0.00251349386899375\\
78.53	0.00251239027952889\\
78.54	0.0025112865628524\\
78.55	0.00251018271897629\\
78.56	0.00250907874791417\\
78.57	0.00250797464968125\\
78.58	0.00250687042429437\\
78.59	0.00250576607177203\\
78.6	0.00250466159213439\\
78.61	0.00250355698540331\\
78.62	0.00250245225160238\\
78.63	0.0025013473907569\\
78.64	0.00250024240289394\\
78.65	0.00249913728804238\\
78.66	0.00249803204623286\\
78.67	0.00249692667749788\\
78.68	0.00249582118187179\\
78.69	0.00249471555939078\\
78.7	0.00249360981009298\\
78.71	0.00249250393401842\\
78.72	0.00249139793120907\\
78.73	0.00249029180170888\\
78.74	0.00248918554556378\\
78.75	0.00248807916282171\\
78.76	0.00248697265353265\\
78.77	0.00248586601774868\\
78.78	0.00248475925552392\\
78.79	0.00248365236691463\\
78.8	0.00248254535197919\\
78.81	0.00248143821077817\\
78.82	0.00248033094337429\\
78.83	0.00247922354983252\\
78.84	0.00247811603022005\\
78.85	0.00247700838460633\\
78.86	0.00247590061306313\\
78.87	0.00247479271566449\\
78.88	0.00247368469248685\\
78.89	0.00247257654360897\\
78.9	0.00247146826911203\\
78.91	0.00247035986907964\\
78.92	0.00246925134362118\\
78.93	0.00246814269299274\\
78.94	0.0024670339174503\\
78.95	0.0024659250172498\\
78.96	0.0024648159926471\\
78.97	0.00246370684389801\\
78.98	0.00246259757125824\\
78.99	0.00246148817498344\\
79	0.00246037865532918\\
79.01	0.00245926901255094\\
79.02	0.00245815924690416\\
79.03	0.00245704935864414\\
79.04	0.00245593934802612\\
79.05	0.00245482921530527\\
79.06	0.00245371896073665\\
79.07	0.0024526085845752\\
79.08	0.00245149808707583\\
79.09	0.00245038746849329\\
79.1	0.00244927672908227\\
79.11	0.00244816586909734\\
79.12	0.00244705488879298\\
79.13	0.00244594378842354\\
79.14	0.00244483256824327\\
79.15	0.00244372122850634\\
79.16	0.00244260976946675\\
79.17	0.00244149819137844\\
79.18	0.00244038649449518\\
79.19	0.00243927467907066\\
79.2	0.00243816274535843\\
79.21	0.00243705069361191\\
79.22	0.00243593852408439\\
79.23	0.00243482623702905\\
79.24	0.00243371383269891\\
79.25	0.00243260131134687\\
79.26	0.00243148867322569\\
79.27	0.00243037591858799\\
79.28	0.00242926304768625\\
79.29	0.00242815006077279\\
79.3	0.0024270369580998\\
79.31	0.0024259237399193\\
79.32	0.00242481040648317\\
79.33	0.00242369695804314\\
79.34	0.00242258339485076\\
79.35	0.00242146971715745\\
79.36	0.00242035592521442\\
79.37	0.00241924201927278\\
79.38	0.0024181279995834\\
79.39	0.00241701386639703\\
79.4	0.00241589961996422\\
79.41	0.00241478526053535\\
79.42	0.00241367078836063\\
79.43	0.00241255620369007\\
79.44	0.00241144150677351\\
79.45	0.00241032669786059\\
79.46	0.00240921177720078\\
79.47	0.00240809674504334\\
79.48	0.00240698160163733\\
79.49	0.00240586634723163\\
79.5	0.0024047509820749\\
79.51	0.0024036355064156\\
79.52	0.00240251992050201\\
79.53	0.00240140422458215\\
79.54	0.00240028841890388\\
79.55	0.0023991725037148\\
79.56	0.00239805647926232\\
79.57	0.00239694034579362\\
79.58	0.00239582410355567\\
79.59	0.00239470775279518\\
79.6	0.00239359129375868\\
79.61	0.0023924747266924\\
79.62	0.00239135805184241\\
79.63	0.00239024126945448\\
79.64	0.00238912437977419\\
79.65	0.00238800738304682\\
79.66	0.00238689027951744\\
79.67	0.00238577306943087\\
79.68	0.00238465575303167\\
79.69	0.00238353833056414\\
79.7	0.00238242080227233\\
79.71	0.00238130316840001\\
79.72	0.00238018542919069\\
79.73	0.00237906758488764\\
79.74	0.00237794963573382\\
79.75	0.00237683158197194\\
79.76	0.00237571342384442\\
79.77	0.00237459516159341\\
79.78	0.00237347679546077\\
79.79	0.00237235832568806\\
79.8	0.00237123975251658\\
79.81	0.00237012107618731\\
79.82	0.00236900229694093\\
79.83	0.00236788341501786\\
79.84	0.00236676443065818\\
79.85	0.00236564534410168\\
79.86	0.00236452615558781\\
79.87	0.00236340686535576\\
79.88	0.00236228747364437\\
79.89	0.00236116798069216\\
79.9	0.00236004838673735\\
79.91	0.0023589286920178\\
79.92	0.00235780889677108\\
79.93	0.00235668900123439\\
79.94	0.00235556900564463\\
79.95	0.00235444891023833\\
79.96	0.0023533287152517\\
79.97	0.00235220842092058\\
79.98	0.00235108802748049\\
79.99	0.00234996753516659\\
80	0.00234884694421366\\
80.01	0.00234772625485614\\
};
\addplot [color=green,dashed]
  table[row sep=crcr]{%
80.01	0.00234772625485614\\
80.02	0.00234660546732811\\
80.03	0.00234548458186328\\
80.04	0.00234436359869498\\
80.05	0.00234324251805617\\
80.06	0.00234212134017945\\
80.07	0.00234100006529703\\
80.08	0.00233987869364071\\
80.09	0.00233875722544194\\
80.1	0.00233763566093177\\
80.11	0.00233651400034083\\
80.12	0.00233539224389938\\
80.13	0.00233427039183727\\
80.14	0.00233314844438394\\
80.15	0.00233202640176842\\
80.16	0.00233090426421934\\
80.17	0.00232978203196488\\
80.18	0.00232865970523285\\
80.19	0.00232753728425058\\
80.2	0.00232641476924501\\
80.21	0.00232529216044265\\
80.22	0.00232416945806955\\
80.23	0.00232304666235132\\
80.24	0.00232192377351317\\
80.25	0.00232080079177981\\
80.26	0.00231967771737552\\
80.27	0.00231855455052415\\
80.28	0.00231743129144905\\
80.29	0.00231630794037312\\
80.3	0.00231518449751881\\
80.31	0.00231406096310809\\
80.32	0.00231293733736246\\
80.33	0.00231181362050293\\
80.34	0.00231068981275003\\
80.35	0.00230956591432381\\
80.36	0.00230844192544384\\
80.37	0.00230731784632918\\
80.38	0.00230619367719838\\
80.39	0.00230506941826954\\
80.4	0.00230394506976017\\
80.41	0.00230282063188736\\
80.42	0.00230169610486762\\
80.43	0.00230057148891697\\
80.44	0.00229944678425091\\
80.45	0.00229832199108439\\
80.46	0.00229719710963185\\
80.47	0.00229607214010719\\
80.48	0.00229494708272377\\
80.49	0.0022938219376944\\
80.5	0.00229269670523134\\
80.51	0.00229157138554632\\
80.52	0.0022904459788505\\
80.53	0.00228932048535447\\
80.54	0.00228819490526825\\
80.55	0.00228706923880132\\
80.56	0.00228594348616257\\
80.57	0.00228481764756031\\
80.58	0.00228369172320225\\
80.59	0.00228256571329555\\
80.6	0.00228143961804676\\
80.61	0.00228031343766183\\
80.62	0.00227918717234611\\
80.63	0.00227806082230436\\
80.64	0.0022769343877407\\
80.65	0.00227580786885868\\
80.66	0.0022746812658612\\
80.67	0.00227355457895053\\
80.68	0.00227242780832835\\
80.69	0.00227130095419567\\
80.7	0.00227017401675287\\
80.71	0.00226904699619973\\
80.72	0.00226791989273531\\
80.73	0.0022667927065581\\
80.74	0.00226566543786586\\
80.75	0.00226453808685576\\
80.76	0.00226341065372425\\
80.77	0.00226228313866713\\
80.78	0.00226115554187953\\
80.79	0.0022600278635559\\
80.8	0.00225890010389001\\
80.81	0.00225777226307493\\
80.82	0.00225664434130303\\
80.83	0.00225551633876602\\
80.84	0.00225438825565485\\
80.85	0.00225326009215981\\
80.86	0.00225213184847046\\
80.87	0.00225100352477563\\
80.88	0.00224987512126345\\
80.89	0.0022487466381213\\
80.9	0.00224761807553585\\
80.91	0.00224648943369299\\
80.92	0.00224536071277793\\
80.93	0.00224423191297509\\
80.94	0.00224310303446812\\
80.95	0.00224197407743996\\
80.96	0.00224084504207275\\
80.97	0.00223971592854789\\
80.98	0.00223858673704598\\
80.99	0.00223745746774684\\
81	0.00223632812082953\\
81.01	0.00223519869647231\\
81.02	0.00223406919485264\\
81.03	0.00223293961614717\\
81.04	0.00223180996053177\\
81.05	0.00223068022818149\\
81.06	0.00222955041927055\\
81.07	0.00222842053397237\\
81.08	0.00222729057245952\\
81.09	0.00222616053490376\\
81.1	0.00222503042147601\\
81.11	0.00222390023234633\\
81.12	0.00222276996768394\\
81.13	0.0022216396276572\\
81.14	0.00222050921243364\\
81.15	0.00221937872217989\\
81.16	0.00221824815706173\\
81.17	0.00221711751724404\\
81.18	0.00221598680289086\\
81.19	0.00221485601416529\\
81.2	0.00221372515122959\\
81.21	0.00221259421424507\\
81.22	0.00221146320337218\\
81.23	0.00221033211877043\\
81.24	0.00220920096059843\\
81.25	0.00220806972901385\\
81.26	0.00220693842417347\\
81.27	0.0022058070462331\\
81.28	0.00220467559534761\\
81.29	0.00220354407167095\\
81.3	0.00220241247535611\\
81.31	0.00220128080655511\\
81.32	0.00220014906541903\\
81.33	0.00219901725209796\\
81.34	0.00219788536674103\\
81.35	0.00219675340949637\\
81.36	0.00219562138051115\\
81.37	0.00219448927993154\\
81.38	0.00219335710790268\\
81.39	0.00219222486456874\\
81.4	0.00219109255007288\\
81.41	0.00218996016455723\\
81.42	0.00218882770816289\\
81.43	0.00218769518102993\\
81.44	0.00218656258329741\\
81.45	0.00218542991510332\\
81.46	0.00218429717658461\\
81.47	0.00218316436787719\\
81.48	0.00218203148911588\\
81.49	0.00218089854043446\\
81.5	0.00217976552196563\\
81.51	0.00217863243384098\\
81.52	0.00217749927619106\\
81.53	0.0021763660491453\\
81.54	0.00217523275283204\\
81.55	0.00217409938737849\\
81.56	0.0021729659529108\\
81.57	0.00217183244955394\\
81.58	0.00217069887743179\\
81.59	0.00216956523666709\\
81.6	0.00216843152738144\\
81.61	0.00216729774969531\\
81.62	0.00216616390372798\\
81.63	0.0021650299895976\\
81.64	0.00216389600742115\\
81.65	0.00216276195731444\\
81.66	0.0021616278393921\\
81.67	0.00216049365376755\\
81.68	0.00215935940055305\\
81.69	0.00215822507985966\\
81.7	0.00215709069179719\\
81.71	0.00215595623647429\\
81.72	0.00215482171399836\\
81.73	0.00215368712447557\\
81.74	0.00215255246801087\\
81.75	0.00215141774470796\\
81.76	0.00215028295466929\\
81.77	0.00214914809799607\\
81.78	0.00214801317478821\\
81.79	0.0021468781851444\\
81.8	0.00214574312916201\\
81.81	0.00214460800693714\\
81.82	0.00214347281856461\\
81.83	0.00214233756413793\\
81.84	0.00214120224374928\\
81.85	0.00214006685748959\\
81.86	0.0021389314054484\\
81.87	0.00213779588771396\\
81.88	0.00213666030437317\\
81.89	0.0021355246555116\\
81.9	0.00213438894121346\\
81.91	0.00213325316156159\\
81.92	0.0021321173166375\\
81.93	0.00213098140652128\\
81.94	0.00212984543129167\\
81.95	0.00212870939102603\\
81.96	0.00212757328580029\\
81.97	0.00212643711568899\\
81.98	0.00212530088076527\\
81.99	0.00212416458110084\\
82	0.002123028216766\\
82.01	0.00212189178782957\\
82.02	0.00212075529435899\\
82.03	0.0021196187364202\\
82.04	0.0021184821140777\\
82.05	0.00211734542739453\\
82.06	0.00211620867643224\\
82.07	0.0021150718612509\\
82.08	0.00211393498190912\\
82.09	0.00211279803846396\\
82.1	0.00211166103097101\\
82.11	0.00211052395948434\\
82.12	0.00210938682405649\\
82.13	0.00210824962473847\\
82.14	0.00210711236157976\\
82.15	0.00210597503462828\\
82.16	0.0021048376439304\\
82.17	0.00210370018953091\\
82.18	0.00210256267147308\\
82.19	0.00210142508979854\\
82.2	0.00210028744454735\\
82.21	0.00209914973575799\\
82.22	0.00209801196346731\\
82.23	0.00209687412771057\\
82.24	0.00209573622852138\\
82.25	0.00209459826593174\\
82.26	0.002093460239972\\
82.27	0.00209232215067087\\
82.28	0.00209118399805538\\
82.29	0.00209004578215092\\
82.3	0.0020889075029812\\
82.31	0.00208776916056823\\
82.32	0.00208663075493235\\
82.33	0.00208549228609219\\
82.34	0.00208435375406465\\
82.35	0.00208321515886495\\
82.36	0.00208207650050655\\
82.37	0.00208093777900119\\
82.38	0.00207979899435886\\
82.39	0.00207866014658778\\
82.4	0.00207752123569444\\
82.41	0.00207638226168354\\
82.42	0.00207524322455797\\
82.43	0.00207410412431886\\
82.44	0.00207296496096556\\
82.45	0.00207182573449555\\
82.46	0.00207068644490455\\
82.47	0.00206954709218641\\
82.48	0.00206840767633317\\
82.49	0.002067268197335\\
82.5	0.00206612865518024\\
82.51	0.00206498904985533\\
82.52	0.00206384938134487\\
82.53	0.00206270964963155\\
82.54	0.00206156985469618\\
82.55	0.00206042999651766\\
82.56	0.00205929007507297\\
82.57	0.00205815009033718\\
82.58	0.00205701004228342\\
82.59	0.00205586993088287\\
82.6	0.00205472975610477\\
82.61	0.00205358951791639\\
82.62	0.00205244921628302\\
82.63	0.002051308851168\\
82.64	0.00205016842253264\\
82.65	0.00204902793033626\\
82.66	0.00204788737453618\\
82.67	0.00204674675508765\\
82.68	0.00204560607194384\\
82.69	0.00204446532505582\\
82.7	0.00204332451437256\\
82.71	0.00204218363984093\\
82.72	0.00204104270140564\\
82.73	0.0020399016990093\\
82.74	0.00203876063259235\\
82.75	0.00203761950209309\\
82.76	0.00203647830744762\\
82.77	0.00203533704858991\\
82.78	0.00203419572545171\\
82.79	0.00203305433796256\\
82.8	0.00203191288604984\\
82.81	0.00203077136963865\\
82.82	0.00202962978865189\\
82.83	0.00202848814301022\\
82.84	0.00202734643263204\\
82.85	0.00202620465743346\\
82.86	0.00202506281732836\\
82.87	0.00202392091222831\\
82.88	0.00202277894204258\\
82.89	0.00202163690667813\\
82.9	0.00202049480603962\\
82.91	0.00201935264002935\\
82.92	0.0020182104085473\\
82.93	0.00201706811149109\\
82.94	0.00201592574875597\\
82.95	0.00201478332023482\\
82.96	0.00201364082581813\\
82.97	0.00201249826539401\\
82.98	0.00201135563884811\\
82.99	0.00201021294606371\\
83	0.00200907018692164\\
83.01	0.00200792736130027\\
83.02	0.00200678446907552\\
83.03	0.00200564151012087\\
83.04	0.00200449848430728\\
83.05	0.00200335539150324\\
83.06	0.00200221223157474\\
83.07	0.00200106900438523\\
83.08	0.00199992570979565\\
83.09	0.00199878234766442\\
83.1	0.00199763891784737\\
83.11	0.0019964954201978\\
83.12	0.0019953518545664\\
83.13	0.00199420822080131\\
83.14	0.00199306451874805\\
83.15	0.00199192074824954\\
83.16	0.00199077690914606\\
83.17	0.00198963300127526\\
83.18	0.00198848902447216\\
83.19	0.00198734497856908\\
83.2	0.00198620086339571\\
83.21	0.00198505667877903\\
83.22	0.00198391242454332\\
83.23	0.00198276810051016\\
83.24	0.0019816237064984\\
83.25	0.00198047924232417\\
83.26	0.00197933470780083\\
83.27	0.00197819010273898\\
83.28	0.00197704542694647\\
83.29	0.00197590068022834\\
83.3	0.00197475586238684\\
83.31	0.00197361097322139\\
83.32	0.00197246601252861\\
83.33	0.00197132098010228\\
83.34	0.0019701758757333\\
83.35	0.00196903069920974\\
83.36	0.00196788545031678\\
83.37	0.00196674012883671\\
83.38	0.00196559473454889\\
83.39	0.0019644492672298\\
83.4	0.00196330372665298\\
83.41	0.00196215811258901\\
83.42	0.00196101242480554\\
83.43	0.00195986666306722\\
83.44	0.00195872082713573\\
83.45	0.00195757491676976\\
83.46	0.00195642893172497\\
83.47	0.00195528287175401\\
83.48	0.00195413673660648\\
83.49	0.00195299052602894\\
83.5	0.00195184423976488\\
83.51	0.0019506978775547\\
83.52	0.00194955143913571\\
83.53	0.00194840492424213\\
83.54	0.00194725833260504\\
83.55	0.00194611166395238\\
83.56	0.00194496491800895\\
83.57	0.00194381809449638\\
83.58	0.00194267119313314\\
83.59	0.00194152421363449\\
83.6	0.00194037715571248\\
83.61	0.00193923001907594\\
83.62	0.0019380828034305\\
83.63	0.00193693550847848\\
83.64	0.00193578813391897\\
83.65	0.00193464067944779\\
83.66	0.00193349314475745\\
83.67	0.00193234552953715\\
83.68	0.00193119783347279\\
83.69	0.0019300500562469\\
83.7	0.00192890219753867\\
83.71	0.00192775425702395\\
83.72	0.00192660623437518\\
83.73	0.00192545812926139\\
83.74	0.00192430994134823\\
83.75	0.00192316167029791\\
83.76	0.0019220133157692\\
83.77	0.00192086487741741\\
83.78	0.00191971635489438\\
83.79	0.00191856774784847\\
83.8	0.00191741905592452\\
83.81	0.00191627027876386\\
83.82	0.0019151214160043\\
83.83	0.00191397246728008\\
83.84	0.00191282343222189\\
83.85	0.00191167431045684\\
83.86	0.00191052510160843\\
83.87	0.00190937580529657\\
83.88	0.00190822642113752\\
83.89	0.00190707694874392\\
83.9	0.00190592738772473\\
83.91	0.00190477773768527\\
83.92	0.00190362799822712\\
83.93	0.00190247816894819\\
83.94	0.00190132824944267\\
83.95	0.00190017823930099\\
83.96	0.00189902813810983\\
83.97	0.00189787794545213\\
83.98	0.00189672766090699\\
83.99	0.00189557728404975\\
84	0.00189442681445193\\
84.01	0.00189327625168117\\
84.02	0.00189212559530129\\
84.03	0.00189097484487227\\
84.04	0.00188982399995015\\
84.05	0.00188867306008709\\
84.06	0.00188752202483135\\
84.07	0.00188637089372721\\
84.08	0.00188521966631505\\
84.09	0.00188406834213124\\
84.1	0.00188291692070818\\
84.11	0.00188176540157428\\
84.12	0.0018806137842539\\
84.13	0.00187946206826739\\
84.14	0.00187831025313102\\
84.15	0.00187715833835701\\
84.16	0.00187600632345349\\
84.17	0.00187485420792447\\
84.18	0.00187370199126984\\
84.19	0.00187254967298537\\
84.2	0.00187139725256264\\
84.21	0.00187024472948907\\
84.22	0.00186909210324789\\
84.23	0.00186793937331811\\
84.24	0.00186678653917451\\
84.25	0.00186563360028764\\
84.26	0.00186448055612376\\
84.27	0.00186332740614487\\
84.28	0.00186217414980866\\
84.29	0.00186102078656849\\
84.3	0.0018598673158734\\
84.31	0.00185871373716808\\
84.32	0.00185756004989283\\
84.33	0.00185640625348357\\
84.34	0.0018552523473718\\
84.35	0.0018540983309846\\
84.36	0.0018529442037446\\
84.37	0.00185178996506997\\
84.38	0.0018506356143744\\
84.39	0.00184948115106707\\
84.4	0.00184832657455263\\
84.41	0.00184717188423121\\
84.42	0.00184601707949838\\
84.43	0.00184486215974511\\
84.44	0.00184370712435781\\
84.45	0.00184255197271824\\
84.46	0.00184139670420355\\
84.47	0.00184024131818621\\
84.48	0.00183908581403407\\
84.49	0.00183793019111022\\
84.5	0.0018367744487731\\
84.51	0.00183561858637637\\
84.52	0.00183446260326897\\
84.53	0.00183330649879508\\
84.54	0.00183215027229406\\
84.55	0.00183099392310048\\
84.56	0.00182983745054408\\
84.57	0.00182868085394975\\
84.58	0.00182752413263751\\
84.59	0.0018263672859225\\
84.6	0.00182521031311495\\
84.61	0.00182405321352016\\
84.62	0.00182289598643849\\
84.63	0.00182173863116532\\
84.64	0.00182058114699105\\
84.65	0.00181942353320108\\
84.66	0.00181826578907575\\
84.67	0.0018171079138904\\
84.68	0.00181594990691526\\
84.69	0.00181479176741549\\
84.7	0.00181363349465114\\
84.71	0.00181247508787712\\
84.72	0.00181131654634318\\
84.73	0.00181015786929394\\
84.74	0.00180899905596878\\
84.75	0.00180784010560188\\
84.76	0.0018066810174222\\
84.77	0.00180552179065342\\
84.78	0.00180436242451397\\
84.79	0.00180320291821695\\
84.8	0.00180204327097018\\
84.81	0.0018008834819761\\
84.82	0.00179972355043181\\
84.83	0.00179856347552902\\
84.84	0.00179740325645404\\
84.85	0.00179624289238775\\
84.86	0.00179508238250558\\
84.87	0.00179392172597749\\
84.88	0.00179276092196795\\
84.89	0.00179159996963592\\
84.9	0.00179043886813481\\
84.91	0.00178927761661249\\
84.92	0.00178811621421125\\
84.93	0.00178695466006777\\
84.94	0.0017857929533131\\
84.95	0.00178463109307265\\
84.96	0.00178346907846618\\
84.97	0.00178230690860774\\
84.98	0.00178114458260565\\
84.99	0.00177998209956254\\
85	0.00177881945857525\\
85.01	0.00177765665873484\\
85.02	0.00177649369912656\\
85.03	0.00177533057882987\\
85.04	0.00177416729691835\\
85.05	0.00177300385245971\\
85.06	0.00177184024451577\\
85.07	0.00177067647214242\\
85.08	0.00176951253438964\\
85.09	0.00176834843030141\\
85.1	0.00176718415891574\\
85.11	0.00176601971926463\\
85.12	0.00176485511037404\\
85.13	0.00176369033126387\\
85.14	0.00176252538094794\\
85.15	0.00176136025843398\\
85.16	0.00176019496272355\\
85.17	0.00175902949281211\\
85.18	0.0017578638476889\\
85.19	0.00175669802633699\\
85.2	0.00175553202773321\\
85.21	0.00175436585084814\\
85.22	0.00175319949464609\\
85.23	0.00175203295808509\\
85.24	0.00175086624011681\\
85.25	0.00174969933968662\\
85.26	0.00174853225573349\\
85.27	0.00174736498719\\
85.28	0.00174619753298231\\
85.29	0.00174502989203015\\
85.3	0.00174386206324678\\
85.31	0.00174269404553895\\
85.32	0.00174152583780691\\
85.33	0.00174035743894435\\
85.34	0.00173918884783841\\
85.35	0.00173802006336963\\
85.36	0.00173685108441192\\
85.37	0.00173568190983258\\
85.38	0.0017345125384922\\
85.39	0.0017333429692447\\
85.4	0.00173217320093729\\
85.41	0.00173100323241041\\
85.42	0.00172983306249775\\
85.43	0.0017286626900262\\
85.44	0.00172749211381581\\
85.45	0.00172632133267981\\
85.46	0.00172515034542454\\
85.47	0.00172397915084945\\
85.48	0.00172280774774707\\
85.49	0.00172163613490295\\
85.5	0.0017204643110957\\
85.51	0.00171929227509689\\
85.52	0.00171812002567111\\
85.53	0.00171694756157584\\
85.54	0.00171577488189922\\
85.55	0.00171460198644673\\
85.56	0.00171342887502622\\
85.57	0.00171225554744797\\
85.58	0.00171108200352467\\
85.59	0.00170990824307149\\
85.6	0.00170873426590615\\
85.61	0.00170756007184885\\
85.62	0.0017063856607224\\
85.63	0.00170521103235054\\
85.64	0.00170403618655775\\
85.65	0.00170286112316926\\
85.66	0.00170168584201108\\
85.67	0.00170051034291\\
85.68	0.00169933462569362\\
85.69	0.00169815869019037\\
85.7	0.00169698253622949\\
85.71	0.00169580616364111\\
85.72	0.00169462957225618\\
85.73	0.00169345276190657\\
85.74	0.00169227573242504\\
85.75	0.00169109848364525\\
85.76	0.00168992101540181\\
85.77	0.00168874332753027\\
85.78	0.00168756541986714\\
85.79	0.00168638729224994\\
85.8	0.00168520894451715\\
85.81	0.00168403037650829\\
85.82	0.00168285158806391\\
85.83	0.0016816725790256\\
85.84	0.00168049334923603\\
85.85	0.00167931389853894\\
85.86	0.0016781342267792\\
85.87	0.00167695433380278\\
85.88	0.00167577421945679\\
85.89	0.0016745938835895\\
85.9	0.00167341332605036\\
85.91	0.00167223254669\\
85.92	0.00167105154536028\\
85.93	0.00166987032191429\\
85.94	0.00166868887620634\\
85.95	0.00166750720809205\\
85.96	0.0016663253174283\\
85.97	0.0016651432040733\\
85.98	0.00166396086788655\\
85.99	0.00166277830872892\\
86	0.00166159552646266\\
86.01	0.00166041252095138\\
86.02	0.0016592292920601\\
86.03	0.00165804583965527\\
86.04	0.00165686216360478\\
86.05	0.001655678263778\\
86.06	0.00165449414004576\\
86.07	0.00165330979228044\\
86.08	0.0016521252203559\\
86.09	0.00165094042414758\\
86.1	0.00164975540353248\\
86.11	0.00164857015838918\\
86.12	0.00164738468859791\\
86.13	0.00164619899404049\\
86.14	0.00164501307460043\\
86.15	0.00164382693016289\\
86.16	0.00164264056061476\\
86.17	0.00164145396584463\\
86.18	0.00164026714574284\\
86.19	0.0016390801002015\\
86.2	0.00163789282911452\\
86.21	0.00163670533237761\\
86.22	0.00163551760988832\\
86.23	0.00163432966154605\\
86.24	0.0016331414872521\\
86.25	0.00163195308690968\\
86.26	0.00163076446042391\\
86.27	0.00162957560770187\\
86.28	0.00162838652865262\\
86.29	0.00162719722318724\\
86.3	0.0016260076912188\\
86.31	0.00162481793266246\\
86.32	0.00162362794743543\\
86.33	0.00162243773545704\\
86.34	0.00162124729664872\\
86.35	0.00162005663093408\\
86.36	0.00161886573823888\\
86.37	0.00161767461849112\\
86.38	0.00161648327162099\\
86.39	0.00161529169756095\\
86.4	0.00161409989624576\\
86.41	0.00161290786761245\\
86.42	0.0016117156116004\\
86.43	0.00161052312815136\\
86.44	0.00160933041720946\\
86.45	0.00160813747872124\\
86.46	0.00160694431263569\\
86.47	0.00160575091890426\\
86.48	0.00160455729748091\\
86.49	0.00160336344832211\\
86.5	0.00160216937138689\\
86.51	0.00160097506663685\\
86.52	0.00159978053403624\\
86.53	0.00159858577355188\\
86.54	0.00159739078515333\\
86.55	0.0015961955688128\\
86.56	0.00159500012450522\\
86.57	0.00159380445220831\\
86.58	0.00159260855190256\\
86.59	0.00159141242357126\\
86.6	0.00159021606720056\\
86.61	0.00158901948277947\\
86.62	0.00158782267029994\\
86.63	0.0015866256297568\\
86.64	0.00158542836114789\\
86.65	0.00158423086447405\\
86.66	0.00158303313973911\\
86.67	0.00158183518694999\\
86.68	0.00158063700611671\\
86.69	0.0015794385972524\\
86.7	0.00157823996037334\\
86.71	0.00157704109549901\\
86.72	0.00157584200265211\\
86.73	0.00157464268185859\\
86.74	0.00157344313314769\\
86.75	0.00157224335655197\\
86.76	0.00157104335210735\\
86.77	0.00156984311985313\\
86.78	0.00156864265983203\\
86.79	0.00156744197209023\\
86.8	0.00156624105667741\\
86.81	0.00156503991364674\\
86.82	0.001563838543055\\
86.83	0.00156263694496253\\
86.84	0.00156143511943331\\
86.85	0.00156023306653499\\
86.86	0.00155903078633889\\
86.87	0.00155782827892013\\
86.88	0.00155662554435754\\
86.89	0.0015554225827338\\
86.9	0.00155421939413541\\
86.91	0.00155301597865279\\
86.92	0.00155181233638025\\
86.93	0.00155060846741605\\
86.94	0.00154940437185914\\
86.95	0.00154820004980893\\
86.96	0.00154699550136527\\
86.97	0.00154579072662851\\
86.98	0.00154458572569944\\
86.99	0.00154338049867934\\
87	0.00154217504566994\\
87.01	0.00154096936677342\\
87.02	0.00153976346209245\\
87.03	0.00153855733173015\\
87.04	0.0015373509757901\\
87.05	0.00153614439437634\\
87.06	0.00153493758759337\\
87.07	0.00153373055554615\\
87.08	0.00153252329834009\\
87.09	0.00153131581608108\\
87.1	0.00153010810887544\\
87.11	0.00152890017682994\\
87.12	0.00152769202005184\\
87.13	0.00152648363864881\\
87.14	0.001525275032729\\
87.15	0.001524066202401\\
87.16	0.00152285714777385\\
87.17	0.00152164786895704\\
87.18	0.00152043836606049\\
87.19	0.00151922863919457\\
87.2	0.00151801868847013\\
87.21	0.0015168085139984\\
87.22	0.00151559811589111\\
87.23	0.00151438749426037\\
87.24	0.00151317664921879\\
87.25	0.00151196558087936\\
87.26	0.00151075428935554\\
87.27	0.00150954277476121\\
87.28	0.00150833103721068\\
87.29	0.0015071190768187\\
87.3	0.00150590689370042\\
87.31	0.00150469448797145\\
87.32	0.00150348185974779\\
87.33	0.0015022690091459\\
87.34	0.00150105593628263\\
87.35	0.00149984264127525\\
87.36	0.00149862912424146\\
87.37	0.00149741538529936\\
87.38	0.00149620142456747\\
87.39	0.00149498724216472\\
87.4	0.00149377283821043\\
87.41	0.00149255821282436\\
87.42	0.00149134336612665\\
87.43	0.00149012829823782\\
87.44	0.00148891300927883\\
87.45	0.00148769749937101\\
87.46	0.00148648176863609\\
87.47	0.00148526581719618\\
87.48	0.00148404964517379\\
87.49	0.00148283325269183\\
87.5	0.00148161663987354\\
87.51	0.00148039980684261\\
87.52	0.00147918275372305\\
87.53	0.00147796548063927\\
87.54	0.00147674798771605\\
87.55	0.00147553027507853\\
87.56	0.00147431234285223\\
87.57	0.00147309419116301\\
87.58	0.00147187582013712\\
87.59	0.00147065722990113\\
87.6	0.00146943842058199\\
87.61	0.001468219392307\\
87.62	0.00146700014520379\\
87.63	0.00146578067940035\\
87.64	0.00146456099502498\\
87.65	0.00146334109220636\\
87.66	0.00146212097107346\\
87.67	0.00146090063175562\\
87.68	0.00145968007438247\\
87.69	0.00145845929908398\\
87.7	0.00145723830599044\\
87.71	0.00145601709523245\\
87.72	0.0014547956669409\\
87.73	0.00145357402124703\\
87.74	0.00145235215828215\\
87.75	0.0014511300781777\\
87.76	0.00144990778106522\\
87.77	0.00144868526707635\\
87.78	0.00144746253634284\\
87.79	0.00144623958899654\\
87.8	0.00144501642516941\\
87.81	0.00144379304499352\\
87.82	0.00144256944860104\\
87.83	0.00144134563612422\\
87.84	0.00144012160769547\\
87.85	0.00143889736344724\\
87.86	0.00143767290351214\\
87.87	0.00143644822802284\\
87.88	0.00143522333711216\\
87.89	0.00143399823091297\\
87.9	0.00143277290955829\\
87.91	0.00143154737318121\\
87.92	0.00143032162191496\\
87.93	0.00142909565589283\\
87.94	0.00142786947524824\\
87.95	0.00142664308011472\\
87.96	0.00142541647062588\\
87.97	0.00142418964691545\\
87.98	0.00142296260911725\\
87.99	0.00142173535736522\\
88	0.00142050789179339\\
88.01	0.00141928021253589\\
88.02	0.00141805231972695\\
88.03	0.00141682421350093\\
88.04	0.00141559589399225\\
88.05	0.00141436736133546\\
88.06	0.00141313861566521\\
88.07	0.00141190965711625\\
88.08	0.00141068048582342\\
88.09	0.00140945110192167\\
88.1	0.00140822150554605\\
88.11	0.00140699169683171\\
88.12	0.00140576167591392\\
88.13	0.00140453144292802\\
88.14	0.00140330099800947\\
88.15	0.00140207034129382\\
88.16	0.00140083947291673\\
88.17	0.00139960839301397\\
88.18	0.00139837710172139\\
88.19	0.00139714559917494\\
88.2	0.00139591388551069\\
88.21	0.00139468196086479\\
88.22	0.0013934498253735\\
88.23	0.00139221747917319\\
88.24	0.00139098492240031\\
88.25	0.00138975215519141\\
88.26	0.00138851917768316\\
88.27	0.00138728599001232\\
88.28	0.00138605259231574\\
88.29	0.00138481898473038\\
88.3	0.00138358516739328\\
88.31	0.00138235114044162\\
88.32	0.00138111690401263\\
88.33	0.00137988245824368\\
88.34	0.0013786478032722\\
88.35	0.00137741293923576\\
88.36	0.001376177866272\\
88.37	0.00137494258451866\\
88.38	0.00137370709411359\\
88.39	0.00137247139519474\\
88.4	0.00137123548790014\\
88.41	0.00136999937236792\\
88.42	0.00136876304873635\\
88.43	0.00136752651714373\\
88.44	0.0013662897777285\\
88.45	0.00136505283062921\\
88.46	0.00136381567598447\\
88.47	0.001362578313933\\
88.48	0.00136134074461364\\
88.49	0.00136010296816529\\
88.5	0.00135886498472698\\
88.51	0.00135762679443781\\
88.52	0.00135638839743701\\
88.53	0.00135514979386388\\
88.54	0.00135391098385781\\
88.55	0.00135267196755831\\
88.56	0.00135143274510498\\
88.57	0.00135019331663751\\
88.58	0.0013489536822957\\
88.59	0.00134771384221941\\
88.6	0.00134647379654864\\
88.61	0.00134523354542347\\
88.62	0.00134399308898408\\
88.63	0.00134275242737072\\
88.64	0.00134151156072377\\
88.65	0.00134027048918369\\
88.66	0.00133902921289103\\
88.67	0.00133778773198646\\
88.68	0.00133654604661071\\
88.69	0.00133530415690462\\
88.7	0.00133406206300915\\
88.71	0.00133281976506533\\
88.72	0.00133157726321427\\
88.73	0.00133033455759721\\
88.74	0.00132909164835547\\
88.75	0.00132784853563046\\
88.76	0.00132660521956368\\
88.77	0.00132536170029676\\
88.78	0.00132411797797137\\
88.79	0.00132287405272931\\
88.8	0.00132162992471247\\
88.81	0.00132038559406284\\
88.82	0.00131914106092249\\
88.83	0.00131789632543358\\
88.84	0.00131665138773839\\
88.85	0.00131540624797927\\
88.86	0.00131416090629867\\
88.87	0.00131291536283915\\
88.88	0.00131166961774333\\
88.89	0.00131042367115396\\
88.9	0.00130917752321386\\
88.91	0.00130793117406596\\
88.92	0.00130668462385328\\
88.93	0.00130543787271891\\
88.94	0.00130419092080606\\
88.95	0.00130294376825803\\
88.96	0.00130169641521821\\
88.97	0.00130044886183007\\
88.98	0.0012992011082372\\
88.99	0.00129795315458327\\
89	0.00129670500101203\\
89.01	0.00129545664766734\\
89.02	0.00129420809469315\\
89.03	0.0012929593422335\\
89.04	0.00129171039043252\\
89.05	0.00129046123943445\\
89.06	0.00128921188938359\\
89.07	0.00128796234042437\\
89.08	0.00128671259270128\\
89.09	0.00128546264635892\\
89.1	0.00128421250154199\\
89.11	0.00128296215839527\\
89.12	0.00128171161706363\\
89.13	0.00128046087769205\\
89.14	0.00127920994042557\\
89.15	0.00127795880540935\\
89.16	0.00127670747278864\\
89.17	0.00127545594270877\\
89.18	0.00127420421531518\\
89.19	0.00127295229075338\\
89.2	0.00127170016916898\\
89.21	0.0012704478507077\\
89.22	0.00126919533551533\\
89.23	0.00126794262373776\\
89.24	0.00126668971552097\\
89.25	0.00126543661101103\\
89.26	0.00126418331035412\\
89.27	0.00126292981369648\\
89.28	0.00126167612118447\\
89.29	0.00126042223296453\\
89.3	0.00125916814918318\\
89.31	0.00125791386998707\\
89.32	0.00125665939552289\\
89.33	0.00125540472593747\\
89.34	0.0012541498613777\\
89.35	0.00125289480199057\\
89.36	0.00125163954792316\\
89.37	0.00125038409932265\\
89.38	0.00124912845633632\\
89.39	0.0012478726191115\\
89.4	0.00124661658779567\\
89.41	0.00124536036253635\\
89.42	0.00124410394348118\\
89.43	0.00124284733077788\\
89.44	0.00124159052457428\\
89.45	0.00124033352501828\\
89.46	0.00123907633225787\\
89.47	0.00123781894644116\\
89.48	0.00123656136771632\\
89.49	0.00123530359623162\\
89.5	0.00123404563213544\\
89.51	0.00123278747557623\\
89.52	0.00123152912670254\\
89.53	0.001230270585663\\
89.54	0.00122901185260636\\
89.55	0.00122775292768144\\
89.56	0.00122649381103715\\
89.57	0.00122523450282249\\
89.58	0.00122397500318657\\
89.59	0.00122271531227858\\
89.6	0.0012214554302478\\
89.61	0.0012201953572436\\
89.62	0.00121893509341545\\
89.63	0.0012176746389129\\
89.64	0.00121641399388561\\
89.65	0.0012151531584833\\
89.66	0.00121389213285583\\
89.67	0.0012126309171531\\
89.68	0.00121136951152515\\
89.69	0.00121010791612206\\
89.7	0.00120884613109405\\
89.71	0.00120758415659141\\
89.72	0.00120632199276452\\
89.73	0.00120505963976385\\
89.74	0.00120379709773997\\
89.75	0.00120253436684355\\
89.76	0.00120127144722533\\
89.77	0.00120000833903616\\
89.78	0.00119874504242698\\
89.79	0.00119748155754881\\
89.8	0.00119621788455278\\
89.81	0.0011949540235901\\
89.82	0.00119368997481206\\
89.83	0.00119242573837009\\
89.84	0.00119116131441565\\
89.85	0.00118989670310035\\
89.86	0.00118863190457585\\
89.87	0.00118736691899392\\
89.88	0.00118610174650643\\
89.89	0.00118483638726532\\
89.9	0.00118357084142266\\
89.91	0.00118230510913057\\
89.92	0.00118103919054129\\
89.93	0.00117977308580715\\
89.94	0.00117850679508056\\
89.95	0.00117724031851406\\
89.96	0.00117597365626022\\
89.97	0.00117470680847177\\
89.98	0.00117343977530149\\
89.99	0.00117217255690227\\
90	0.00117090515342709\\
90.01	0.00116963756502902\\
90.02	0.00116836979186125\\
90.03	0.00116710183407702\\
90.04	0.0011658336918297\\
90.05	0.00116456536527273\\
90.06	0.00116329685455967\\
90.07	0.00116202815984415\\
90.08	0.00116075928127991\\
90.09	0.00115949021902077\\
90.1	0.00115822097322067\\
90.11	0.00115695154403362\\
90.12	0.00115568193161373\\
90.13	0.00115441213611522\\
90.14	0.00115314215769238\\
90.15	0.00115187199649963\\
90.16	0.00115060165269145\\
90.17	0.00114933112642243\\
90.18	0.00114806041784726\\
90.19	0.00114678952712073\\
90.2	0.0011455184543977\\
90.21	0.00114424719983317\\
90.22	0.00114297576358219\\
90.23	0.00114170414579993\\
90.24	0.00114043234664167\\
90.25	0.00113916036626276\\
90.26	0.00113788820481865\\
90.27	0.0011366158624649\\
90.28	0.00113534333935717\\
90.29	0.0011340706356512\\
90.3	0.00113279775150284\\
90.31	0.00113152468706803\\
90.32	0.00113025144250282\\
90.33	0.00112897801796334\\
90.34	0.00112770441360584\\
90.35	0.00112643062958665\\
90.36	0.00112515666606221\\
90.37	0.00112388252318904\\
90.38	0.00112260820112379\\
90.39	0.00112133370002318\\
90.4	0.00112005902004405\\
90.41	0.00111878416134333\\
90.42	0.00111750912407804\\
90.43	0.00111623390840531\\
90.44	0.00111495851448239\\
90.45	0.00111368294246659\\
90.46	0.00111240719251534\\
90.47	0.0011111312647862\\
90.48	0.00110985515943676\\
90.49	0.00110857887662478\\
90.5	0.00110730241650809\\
90.51	0.00110602577924463\\
90.52	0.00110474896499243\\
90.53	0.00110347197390963\\
90.54	0.00110219480615448\\
90.55	0.00110091746188531\\
90.56	0.00109963994126058\\
90.57	0.00109836224443884\\
90.58	0.00109708437157873\\
90.59	0.00109580632283902\\
90.6	0.00109452809837856\\
90.61	0.00109324969835633\\
90.62	0.00109197112293137\\
90.63	0.00109069237226288\\
90.64	0.00108941344651013\\
90.65	0.00108813434583249\\
90.66	0.00108685507038945\\
90.67	0.00108557562034061\\
90.68	0.00108429599584567\\
90.69	0.00108301619706442\\
90.7	0.00108173622415679\\
90.71	0.00108045607728278\\
90.72	0.00107917575660252\\
90.73	0.00107789526227624\\
90.74	0.00107661459446428\\
90.75	0.00107533375332708\\
90.76	0.0010740527390252\\
90.77	0.0010727715517193\\
90.78	0.00107149019157015\\
90.79	0.00107020865873862\\
90.8	0.00106892695338572\\
90.81	0.00106764507567254\\
90.82	0.00106636302576028\\
90.83	0.00106508080381026\\
90.84	0.00106379840998392\\
90.85	0.00106251584444279\\
90.86	0.00106123310734852\\
90.87	0.00105995019886288\\
90.88	0.00105866711914773\\
90.89	0.00105738386836508\\
90.9	0.001056100446677\\
90.91	0.00105481685424573\\
90.92	0.00105353309123357\\
90.93	0.00105224915780298\\
90.94	0.0010509650541165\\
90.95	0.0010496807803368\\
90.96	0.00104839633662666\\
90.97	0.00104711172314897\\
90.98	0.00104582694006677\\
90.99	0.00104454198754316\\
91	0.00104325686574139\\
91.01	0.00104197157482483\\
91.02	0.00104068611495695\\
91.03	0.00103940048630135\\
91.04	0.00103811468902174\\
91.05	0.00103682872328196\\
91.06	0.00103554258924595\\
91.07	0.00103425628707778\\
91.08	0.00103296981694164\\
91.09	0.00103168317900185\\
91.1	0.00103039637342283\\
91.11	0.00102910940036912\\
91.12	0.00102782226000541\\
91.13	0.00102653495249648\\
91.14	0.00102524747800724\\
91.15	0.00102395983670275\\
91.16	0.00102267202874815\\
91.17	0.00102138405430873\\
91.18	0.0010200959135499\\
91.19	0.00101880760663718\\
91.2	0.00101751913373625\\
91.21	0.00101623049501288\\
91.22	0.00101494169063298\\
91.23	0.00101365272076257\\
91.24	0.00101236358556784\\
91.25	0.00101107428521505\\
91.26	0.00100978481987064\\
91.27	0.00100849518970114\\
91.28	0.00100720539487322\\
91.29	0.0010059154355537\\
91.3	0.0010046253119095\\
91.31	0.00100333502410769\\
91.32	0.00100204457231546\\
91.33	0.00100075395670013\\
91.34	0.000999463177429175\\
91.35	0.000998172234670167\\
91.36	0.000996881128590837\\
91.37	0.00099558985935904\\
91.38	0.000994298427142765\\
91.39	0.000993006832110141\\
91.4	0.000991715074429425\\
91.41	0.000990423154269015\\
91.42	0.000989131071797449\\
91.43	0.000987838827183387\\
91.44	0.000986546420595646\\
91.45	0.000985253852203166\\
91.46	0.000983961122175032\\
91.47	0.000982668230680474\\
91.48	0.000981375177888847\\
91.49	0.000980081963969662\\
91.5	0.000978788589092562\\
91.51	0.000977495053427337\\
91.52	0.000976201357143913\\
91.53	0.000974907500412368\\
91.54	0.000973613483402913\\
91.55	0.000972319306285914\\
91.56	0.000971024969231878\\
91.57	0.000969730472411454\\
91.58	0.000968435815995445\\
91.59	0.000967141000154791\\
91.6	0.000965846025060587\\
91.61	0.000964550890884071\\
91.62	0.000963255597796638\\
91.63	0.000961960145969831\\
91.64	0.000960664535575334\\
91.65	0.000959368766784994\\
91.66	0.000958072839770804\\
91.67	0.000956776754704906\\
91.68	0.000955480511759602\\
91.69	0.000954184111107344\\
91.7	0.000952887552920745\\
91.71	0.000951590837372564\\
91.72	0.000950293964635722\\
91.73	0.000948996934883295\\
91.74	0.000947699748288518\\
91.75	0.000946402405024785\\
91.76	0.000945104905265644\\
91.77	0.000943807249184811\\
91.78	0.000942509436956158\\
91.79	0.000941211468753717\\
91.8	0.000939913344751686\\
91.81	0.000938615065124427\\
91.82	0.000937316630046462\\
91.83	0.000936018039692477\\
91.84	0.000934719294237328\\
91.85	0.000933420393856035\\
91.86	0.000932121338723784\\
91.87	0.000930822129015935\\
91.88	0.000929522764908002\\
91.89	0.000928223246575691\\
91.9	0.000926923574194861\\
91.91	0.000925623747941544\\
91.92	0.000924323767991956\\
91.93	0.000923023634522471\\
91.94	0.000921723347709649\\
91.95	0.000920422907730213\\
91.96	0.00091912231476108\\
91.97	0.000917821568979326\\
91.98	0.000916520670562209\\
91.99	0.000915219619687173\\
92	0.000913918416531835\\
92.01	0.000912617061273989\\
92.02	0.000911315554091616\\
92.03	0.000910013895162883\\
92.04	0.000908712084666129\\
92.05	0.000907410122779886\\
92.06	0.000906108009682867\\
92.07	0.000904805745553973\\
92.08	0.00090350333057229\\
92.09	0.000902200764917096\\
92.1	0.000900898048767852\\
92.11	0.000899595182304218\\
92.12	0.000898292165706029\\
92.13	0.000896988999153327\\
92.14	0.000895685682826344\\
92.15	0.000894382216905501\\
92.16	0.00089307860157142\\
92.17	0.000891774837004915\\
92.18	0.000890470923386994\\
92.19	0.000889166860898875\\
92.2	0.000887862649721963\\
92.21	0.000886558290037868\\
92.22	0.000885253782028399\\
92.23	0.000883949125875579\\
92.24	0.000882644321761617\\
92.25	0.000881339369868934\\
92.26	0.000880034270380163\\
92.27	0.000878729023478134\\
92.28	0.000877423629345899\\
92.29	0.000876118088166695\\
92.3	0.000874812400123997\\
92.31	0.000873506565401472\\
92.32	0.000872200584183008\\
92.33	0.000870894456652707\\
92.34	0.000869588182994875\\
92.35	0.000868281763394051\\
92.36	0.000866975198034976\\
92.37	0.000865668487102624\\
92.38	0.000864361630782176\\
92.39	0.000863054629259039\\
92.4	0.000861747482718844\\
92.41	0.00086044019134744\\
92.42	0.000859132755330907\\
92.43	0.000857825174855546\\
92.44	0.000856517450107888\\
92.45	0.000855209581274687\\
92.46	0.000853901568542934\\
92.47	0.000852593412099844\\
92.48	0.000851285112132873\\
92.49	0.000849976668829697\\
92.5	0.000848668082378238\\
92.51	0.000847359352966651\\
92.52	0.000846050480783327\\
92.53	0.000844741466016896\\
92.54	0.00084343230885623\\
92.55	0.000842123009490438\\
92.56	0.000840813568108879\\
92.57	0.000839503984901149\\
92.58	0.000838194260057094\\
92.59	0.000836884393766802\\
92.6	0.000835574386220613\\
92.61	0.000834264237609116\\
92.62	0.000832953948123152\\
92.63	0.000831643517953813\\
92.64	0.000830332947292442\\
92.65	0.000829022236330643\\
92.66	0.000827711385260275\\
92.67	0.000826400394273454\\
92.68	0.000825089263562557\\
92.69	0.000823777993320217\\
92.7	0.000822466583739341\\
92.71	0.000821155035013092\\
92.72	0.000819843347334894\\
92.73	0.00081853152089845\\
92.74	0.000817219555897728\\
92.75	0.000815907452526956\\
92.76	0.000814595210980647\\
92.77	0.00081328283145358\\
92.78	0.000811970314140816\\
92.79	0.000810657659237683\\
92.8	0.000809344866939791\\
92.81	0.000808031937443032\\
92.82	0.000806718870943576\\
92.83	0.000805405667637883\\
92.84	0.000804092327722689\\
92.85	0.000802778851395012\\
92.86	0.000801465238852179\\
92.87	0.00080015149029178\\
92.88	0.000798837605911715\\
92.89	0.000797523585910171\\
92.9	0.000796209430485625\\
92.91	0.000794895139836851\\
92.92	0.000793580714162934\\
92.93	0.00079226615366324\\
92.94	0.000790951458537446\\
92.95	0.00078963662898553\\
92.96	0.000788321665207785\\
92.97	0.00078700656740479\\
92.98	0.000785691335777449\\
92.99	0.000784375970526972\\
93	0.000783060471854872\\
93.01	0.000781744839962994\\
93.02	0.000780429075053482\\
93.03	0.000779113177328802\\
93.04	0.000777797146991744\\
93.05	0.000776480984245419\\
93.06	0.000775164689293249\\
93.07	0.000773848262339\\
93.08	0.000772531703586743\\
93.09	0.0007712150132409\\
93.1	0.000769898191506203\\
93.11	0.000768581238587734\\
93.12	0.000767264154690896\\
93.13	0.000765946940021434\\
93.14	0.000764629594785435\\
93.15	0.000763312119189315\\
93.16	0.000761994513439846\\
93.17	0.000760676777744137\\
93.18	0.000759358912309643\\
93.19	0.000758040917344167\\
93.2	0.000756722793055864\\
93.21	0.00075540453965324\\
93.22	0.000754086157345157\\
93.23	0.000752767646340833\\
93.24	0.000751449006849841\\
93.25	0.000750130239082124\\
93.26	0.000748811343247972\\
93.27	0.000747492319558056\\
93.28	0.000746173168223407\\
93.29	0.000744853889455426\\
93.3	0.000743534483465881\\
93.31	0.000742214950466917\\
93.32	0.000740895290671053\\
93.33	0.000739575504291193\\
93.34	0.000738255591540614\\
93.35	0.00073693555263297\\
93.36	0.00073561538778231\\
93.37	0.00073429509720307\\
93.38	0.000732974681110063\\
93.39	0.000731654139718506\\
93.4	0.000730333473244003\\
93.41	0.000729012681902548\\
93.42	0.000727691765910549\\
93.43	0.000726370725484803\\
93.44	0.000725049560842512\\
93.45	0.00072372827220128\\
93.46	0.000722406859779126\\
93.47	0.000721085323794472\\
93.48	0.000719763664466159\\
93.49	0.00071844188201343\\
93.5	0.000717119976655961\\
93.51	0.000715797948613836\\
93.52	0.00071447579810757\\
93.53	0.000713153525358097\\
93.54	0.000711831130586768\\
93.55	0.000710508614015387\\
93.56	0.000709185975866168\\
93.57	0.000707863216361774\\
93.58	0.000706540335725292\\
93.59	0.000705217334180263\\
93.6	0.000703894211950655\\
93.61	0.000702570969260895\\
93.62	0.00070124760633584\\
93.63	0.00069992412340082\\
93.64	0.000698600520681599\\
93.65	0.0006972767984044\\
93.66	0.000695952956795908\\
93.67	0.000694628996083262\\
93.68	0.000693304916494069\\
93.69	0.000691980718256402\\
93.7	0.000690656401598806\\
93.71	0.000689331966750284\\
93.72	0.00068800741394032\\
93.73	0.000686682743398887\\
93.74	0.000685357955356418\\
93.75	0.000684033050043839\\
93.76	0.000682708027692555\\
93.77	0.000681382888534471\\
93.78	0.000680057632801967\\
93.79	0.000678732260727928\\
93.8	0.000677406772545729\\
93.81	0.000676081168489244\\
93.82	0.000674755448792858\\
93.83	0.000673429613691452\\
93.84	0.000672103663420418\\
93.85	0.00067077759821566\\
93.86	0.000669451418313595\\
93.87	0.000668125123951155\\
93.88	0.000666798715365792\\
93.89	0.000665472192795488\\
93.9	0.000664145556478741\\
93.91	0.000662818806654588\\
93.92	0.000661491943562588\\
93.93	0.000660164967442841\\
93.94	0.000658837878535984\\
93.95	0.000657510677083197\\
93.96	0.0006561833633262\\
93.97	0.000654855937507267\\
93.98	0.000653528399869216\\
93.99	0.000652200750655424\\
94	0.000650872990109819\\
94.01	0.0006495451184769\\
94.02	0.00064821713600172\\
94.03	0.000646889042929906\\
94.04	0.000645560839507642\\
94.05	0.000644232525981704\\
94.06	0.000642904102599427\\
94.07	0.000641575569608739\\
94.08	0.000640246927258144\\
94.09	0.000638918175796737\\
94.1	0.0006375893154742\\
94.11	0.000636260346540808\\
94.12	0.000634931269247437\\
94.13	0.000633602083845561\\
94.14	0.000632272790587253\\
94.15	0.000630943389725197\\
94.16	0.000629613881512694\\
94.17	0.000628284266203651\\
94.18	0.000626954544052594\\
94.19	0.000625624715314669\\
94.2	0.000624294780245649\\
94.21	0.000622964739101941\\
94.22	0.00062163459214057\\
94.23	0.000620304339619208\\
94.24	0.000618973981796165\\
94.25	0.000617643518930388\\
94.26	0.000616312951281478\\
94.27	0.000614982279109678\\
94.28	0.000613651502675889\\
94.29	0.000612320622241673\\
94.3	0.000610989638069244\\
94.31	0.000609658550421497\\
94.32	0.000608327359561978\\
94.33	0.000606996065754919\\
94.34	0.000605664669265217\\
94.35	0.000604333170358464\\
94.36	0.000603001569300923\\
94.37	0.000601669866359547\\
94.38	0.000600338061801995\\
94.39	0.0005990061558966\\
94.4	0.000597674148912415\\
94.41	0.000596342041119182\\
94.42	0.000595009832787362\\
94.43	0.000593677524188123\\
94.44	0.000592345115593347\\
94.45	0.000591012607275633\\
94.46	0.000589679999508324\\
94.47	0.000588347292565463\\
94.48	0.000587014486721845\\
94.49	0.000585681582252995\\
94.5	0.000584348579435179\\
94.51	0.000583015478545406\\
94.52	0.000581682279861434\\
94.53	0.000580348983661784\\
94.54	0.000579015590225725\\
94.55	0.00057768209983328\\
94.56	0.000576348512765256\\
94.57	0.000575014829303217\\
94.58	0.00057368104972951\\
94.59	0.00057234717432726\\
94.6	0.000571013203380363\\
94.61	0.000569679137173517\\
94.62	0.000568344975992211\\
94.63	0.00056701072012272\\
94.64	0.000565676369852138\\
94.65	0.000564341925468342\\
94.66	0.000563007387260037\\
94.67	0.000561672755516739\\
94.68	0.000560338030528773\\
94.69	0.000559003212587299\\
94.7	0.000557668301984303\\
94.71	0.0005563332990126\\
94.72	0.000554998203965843\\
94.73	0.000553663017138535\\
94.74	0.000552327738826014\\
94.75	0.000550992369324479\\
94.76	0.000549656908930976\\
94.77	0.000548321357943424\\
94.78	0.0005469857166606\\
94.79	0.000545649985382151\\
94.8	0.000544314164408604\\
94.81	0.000542978254041359\\
94.82	0.000541642254582711\\
94.83	0.000540306166335837\\
94.84	0.000538969989604813\\
94.85	0.000537633724694614\\
94.86	0.000536297371911116\\
94.87	0.000534960931561114\\
94.88	0.000533624403952311\\
94.89	0.000532287789393323\\
94.9	0.000530951088193704\\
94.91	0.000529614300663938\\
94.92	0.000528277427115429\\
94.93	0.000526940467860537\\
94.94	0.000525603423212563\\
94.95	0.000524266293485754\\
94.96	0.000522929078995312\\
94.97	0.000521591780057409\\
94.98	0.000520254396989174\\
94.99	0.000518916930108715\\
95	0.000517579379735112\\
95.01	0.000516241746188426\\
95.02	0.000514904029789714\\
95.03	0.000513566230861009\\
95.04	0.000512228349725358\\
95.05	0.000510890386706807\\
95.06	0.000509552342130412\\
95.07	0.000508214216322243\\
95.08	0.000506876009609384\\
95.09	0.000505537722319951\\
95.1	0.000504199354783092\\
95.11	0.000502860907328989\\
95.12	0.000501522380288867\\
95.13	0.000500183773995001\\
95.14	0.000498845088780715\\
95.15	0.000497506324980397\\
95.16	0.000496167482929499\\
95.17	0.000494828562964539\\
95.18	0.00049348956542312\\
95.19	0.000492150490643923\\
95.2	0.000490811338966715\\
95.21	0.000489472110732359\\
95.22	0.000488132806282822\\
95.23	0.000486793425961167\\
95.24	0.000485453970111583\\
95.25	0.000484114439079357\\
95.26	0.000482774833210922\\
95.27	0.000481435152853826\\
95.28	0.000480095398356755\\
95.29	0.000478755570069538\\
95.3	0.000477415668343155\\
95.31	0.000476075693529735\\
95.32	0.000474735645982568\\
95.33	0.000473395526056119\\
95.34	0.000472055334106007\\
95.35	0.00047071507048905\\
95.36	0.000469374735563237\\
95.37	0.000468034329687757\\
95.38	0.000466693853222988\\
95.39	0.000465353306530523\\
95.4	0.000464012689973155\\
95.41	0.000462672003914902\\
95.42	0.000461331248720994\\
95.43	0.000459990424757906\\
95.44	0.000458649532393338\\
95.45	0.000457308571996233\\
95.46	0.000455967543936788\\
95.47	0.000454626448586454\\
95.48	0.000453285286317943\\
95.49	0.000451944057505241\\
95.5	0.000450602762523603\\
95.51	0.000449261401749569\\
95.52	0.000447919975560969\\
95.53	0.000446578484336929\\
95.54	0.00044523692845788\\
95.55	0.000443895308305549\\
95.56	0.000442553624263002\\
95.57	0.000441211876714613\\
95.58	0.000439870066046085\\
95.59	0.000438528192644469\\
95.6	0.000437186256898146\\
95.61	0.000435844259196861\\
95.62	0.000434502199931708\\
95.63	0.000433160079495157\\
95.64	0.000431817898281036\\
95.65	0.000430475656684562\\
95.66	0.000429133355102336\\
95.67	0.000427790993932352\\
95.68	0.000426448573574007\\
95.69	0.000425106094428104\\
95.7	0.000423763556896866\\
95.71	0.000422420961383929\\
95.72	0.000421078308294373\\
95.73	0.000419735598034699\\
95.74	0.00041839283101287\\
95.75	0.000417050007638286\\
95.76	0.000415707128321821\\
95.77	0.000414364193475807\\
95.78	0.000413021203514054\\
95.79	0.000411678158851852\\
95.8	0.000410335059905986\\
95.81	0.000408991907094735\\
95.82	0.000407648700837883\\
95.83	0.00040630544155673\\
95.84	0.000404962129674094\\
95.85	0.000403618765614325\\
95.86	0.000402275349803307\\
95.87	0.000400931882668465\\
95.88	0.000399588364638782\\
95.89	0.000398244796144802\\
95.9	0.000396901177618633\\
95.91	0.000395557509493957\\
95.92	0.000394213792206046\\
95.93	0.000392870026191759\\
95.94	0.000391526211889559\\
95.95	0.000390182349739516\\
95.96	0.000388838440183312\\
95.97	0.000387494483664262\\
95.98	0.000386150480627305\\
95.99	0.00038480643151903\\
96	0.000383462336787667\\
96.01	0.000382118196883106\\
96.02	0.000380774012256909\\
96.03	0.000379429783362304\\
96.04	0.000378085510654208\\
96.05	0.000376741194589223\\
96.06	0.000375396835625657\\
96.07	0.000374052434223527\\
96.08	0.000372707990844556\\
96.09	0.000371363505952205\\
96.1	0.000370018980011667\\
96.11	0.000368674413489872\\
96.12	0.000367329806855502\\
96.13	0.000365985160579009\\
96.14	0.000364640475132601\\
96.15	0.000363295750990275\\
96.16	0.000361950988627804\\
96.17	0.000360606188522768\\
96.18	0.000359261351154544\\
96.19	0.00035791647700432\\
96.2	0.00035657156655512\\
96.21	0.000355226620291785\\
96.22	0.000353881638701002\\
96.23	0.000352536622271313\\
96.24	0.00035119157149311\\
96.25	0.00034984648685866\\
96.26	0.000348501368862102\\
96.27	0.000347156217999467\\
96.28	0.000345811034768681\\
96.29	0.000344465819669569\\
96.3	0.000343120573203879\\
96.31	0.00034177529587528\\
96.32	0.000340429988189372\\
96.33	0.000339084650653703\\
96.34	0.000337739283777772\\
96.35	0.000336393888073038\\
96.36	0.000335048464052934\\
96.37	0.00033370301223287\\
96.38	0.000332357533130257\\
96.39	0.0003310120272645\\
96.4	0.00032966649515701\\
96.41	0.000328320937331228\\
96.42	0.000326975354312623\\
96.43	0.000325629746628702\\
96.44	0.000324284114809024\\
96.45	0.000322938459385207\\
96.46	0.000321592780890936\\
96.47	0.000320247079861986\\
96.48	0.00031890135683621\\
96.49	0.00031755561235357\\
96.5	0.000316209846956137\\
96.51	0.000314864061188106\\
96.52	0.000313518255595792\\
96.53	0.000312172430727663\\
96.54	0.000310826587134334\\
96.55	0.000309480725368584\\
96.56	0.000308134845985361\\
96.57	0.0003067889495418\\
96.58	0.000305443036597226\\
96.59	0.000304097107713171\\
96.6	0.000302751163453385\\
96.61	0.000301405204383832\\
96.62	0.00030005923107272\\
96.63	0.000298713244090509\\
96.64	0.000297367244009902\\
96.65	0.000296021231405883\\
96.66	0.000294675206855713\\
96.67	0.000293329170938933\\
96.68	0.000291983124237399\\
96.69	0.000290637067335267\\
96.7	0.000289291000819024\\
96.71	0.00028794492527749\\
96.72	0.000286598841301824\\
96.73	0.000285252749485545\\
96.74	0.000283906650424539\\
96.75	0.000282560544717075\\
96.76	0.000281214432963807\\
96.77	0.000279868315767789\\
96.78	0.000278522193734488\\
96.79	0.000277176067471801\\
96.8	0.000275829937590051\\
96.81	0.000274483804702014\\
96.82	0.000273137669422926\\
96.83	0.000271791532370487\\
96.84	0.000270445394164881\\
96.85	0.000269099255428786\\
96.86	0.000267753116787383\\
96.87	0.000266406978868376\\
96.88	0.000265060842301986\\
96.89	0.000263714707720988\\
96.9	0.000262368575760698\\
96.91	0.000261022447059002\\
96.92	0.000259676322256359\\
96.93	0.000258330201995819\\
96.94	0.000256984086923031\\
96.95	0.000255637977686256\\
96.96	0.000254291874936382\\
96.97	0.000252945779326927\\
96.98	0.000251599691514066\\
96.99	0.000250253612156632\\
97	0.000248907541916128\\
97.01	0.00024756148145675\\
97.02	0.000246215431445389\\
97.03	0.000244869392551644\\
97.04	0.000243523365447843\\
97.05	0.000242177350809045\\
97.06	0.000240831349313063\\
97.07	0.000239485361640467\\
97.08	0.000238139388474602\\
97.09	0.0002367934305016\\
97.1	0.000235447488410397\\
97.11	0.000234101562892733\\
97.12	0.000232755654643178\\
97.13	0.000231409764359144\\
97.14	0.000230063892740889\\
97.15	0.000228718040491536\\
97.16	0.000227372208317089\\
97.17	0.000226026396926439\\
97.18	0.000224680607031381\\
97.19	0.000223334839346633\\
97.2	0.000221989094589835\\
97.21	0.000220643373481575\\
97.22	0.000219297676745401\\
97.23	0.000217952005107826\\
97.24	0.000216606359298351\\
97.25	0.000215260740049471\\
97.26	0.000213915148096695\\
97.27	0.000212569584178561\\
97.28	0.000211224049036635\\
97.29	0.000209878543415545\\
97.3	0.000208533068062981\\
97.31	0.00020718762372971\\
97.32	0.000205842211169599\\
97.33	0.000204496831139619\\
97.34	0.000203151484399863\\
97.35	0.000201806171713561\\
97.36	0.000200460893847093\\
97.37	0.000199115651569999\\
97.38	0.000197770445655004\\
97.39	0.000196425276878019\\
97.4	0.000195080146018163\\
97.41	0.00019373505385778\\
97.42	0.000192390001182447\\
97.43	0.000191044988780987\\
97.44	0.000189700017445494\\
97.45	0.000188355087971336\\
97.46	0.000187010201157177\\
97.47	0.000185665357804989\\
97.48	0.000184320558720063\\
97.49	0.000182975804711034\\
97.5	0.000181631096589883\\
97.51	0.000180286435171964\\
97.52	0.000178941821276007\\
97.53	0.000177597255724143\\
97.54	0.000176252739341919\\
97.55	0.000174908272958298\\
97.56	0.000173563857405695\\
97.57	0.000172219493519978\\
97.58	0.000170875182140489\\
97.59	0.000169530924110059\\
97.6	0.000168186720275017\\
97.61	0.000166842571485219\\
97.62	0.000165498478594045\\
97.63	0.000164154442458434\\
97.64	0.000162810463938883\\
97.65	0.000161466543899471\\
97.66	0.000160122683207876\\
97.67	0.000158778882735385\\
97.68	0.000157435143356914\\
97.69	0.000156091465951018\\
97.7	0.000154747851400036\\
97.71	0.000153404300590123\\
97.72	0.000152060814411268\\
97.73	0.000150717393757304\\
97.74	0.000149374039525923\\
97.75	0.000148030752618688\\
97.76	0.000146687533941052\\
97.77	0.000145344384402352\\
97.78	0.000144001304915846\\
97.79	0.000142658296398705\\
97.8	0.000141315359479272\\
97.81	0.000139972494572746\\
97.82	0.000138629702101208\\
97.83	0.000137286982493707\\
97.84	0.000135944336186347\\
97.85	0.000134601763622377\\
97.86	0.000133259265252277\\
97.87	0.000131916841533856\\
97.88	0.000130574492932344\\
97.89	0.000129232219776799\\
97.9	0.000127890020957828\\
97.91	0.000126547893017973\\
97.92	0.000125205832460567\\
97.93	0.000123864910660196\\
97.94	0.00012252581358818\\
97.95	0.000121188545859499\\
97.96	0.000119853112104362\\
97.97	0.000118519516967454\\
97.98	0.000117187765057444\\
97.99	0.000115857861002487\\
98	0.000114529809450571\\
98.01	0.000113204229188223\\
98.02	0.000111886377497135\\
98.03	0.000110583684521723\\
98.04	0.000109296289321601\\
98.05	0.000108024410716714\\
98.06	0.000106768291156864\\
98.07	0.000105528070912258\\
98.08	0.000104303891723391\\
98.09	0.000103095896819935\\
98.1	0.000101903528003319\\
98.11	0.000100726032038069\\
98.12	9.95635478075879e-05\\
98.13	9.84162156899785e-05\\
98.14	9.72841775775096e-05\\
98.15	9.6167576896438e-05\\
98.16	9.50664536770018e-05\\
98.17	9.39803675914178e-05\\
98.18	9.29055674737875e-05\\
98.19	9.18347434802728e-05\\
98.2	9.07679165073217e-05\\
98.21	8.97051061932285e-05\\
98.22	8.86463321764525e-05\\
98.23	8.75916140914493e-05\\
98.24	8.65409715643823e-05\\
98.25	8.54944242086788e-05\\
98.26	8.44519916204187e-05\\
98.27	8.3413693373573e-05\\
98.28	8.23795490150665e-05\\
98.29	8.13495780596866e-05\\
98.3	8.03237999847991e-05\\
98.31	7.93022342248807e-05\\
98.32	7.82849001658825e-05\\
98.33	7.72718172501775e-05\\
98.34	7.62630048716673e-05\\
98.35	7.52584855767992e-05\\
98.36	7.42582830585113e-05\\
98.37	7.32624210060418e-05\\
98.38	7.22709230996654e-05\\
98.39	7.1283813005198e-05\\
98.4	7.0301114875302e-05\\
98.41	6.93228548944904e-05\\
98.42	6.83490723662316e-05\\
98.43	6.73798069644026e-05\\
98.44	6.64150987370205e-05\\
98.45	6.54549881100221e-05\\
98.46	6.44995158910876e-05\\
98.47	6.35487232735054e-05\\
98.48	6.26026518400821e-05\\
98.49	6.16613435671133e-05\\
98.5	6.07248408283894e-05\\
98.51	5.97931863992492e-05\\
98.52	5.88664234606965e-05\\
98.53	5.79445956035533e-05\\
98.54	5.70277468326767e-05\\
98.55	5.61159215712106e-05\\
98.56	5.52091646643139e-05\\
98.57	5.43075213789105e-05\\
98.58	5.34110374078942e-05\\
98.59	5.25197588743774e-05\\
98.6	5.16337323359928e-05\\
98.61	5.07530047892269e-05\\
98.62	4.98776236738472e-05\\
98.63	4.9007636877315e-05\\
98.64	4.81430927392525e-05\\
98.65	4.72840400559827e-05\\
98.66	4.64305280851538e-05\\
98.67	4.55826065503939e-05\\
98.68	4.4740325646048e-05\\
98.69	4.39037360419389e-05\\
98.7	4.30728888882181e-05\\
98.71	4.22478358202572e-05\\
98.72	4.142862896361e-05\\
98.73	4.06153209390244e-05\\
98.74	3.98079648666269e-05\\
98.75	3.90066143700751e-05\\
98.76	3.82113235817064e-05\\
98.77	3.74221471477563e-05\\
98.78	3.66391402333039e-05\\
98.79	3.58623585261005e-05\\
98.8	3.50918582272593e-05\\
98.81	3.43276960560156e-05\\
98.82	3.35699292545559e-05\\
98.83	3.28186155928757e-05\\
98.84	3.20738133736918e-05\\
98.85	3.13355814373934e-05\\
98.86	3.06039791670466e-05\\
98.87	2.9879066493434e-05\\
98.88	2.91609039001581e-05\\
98.89	2.84495524287675e-05\\
98.9	2.77450736839522e-05\\
98.91	2.70475298387793e-05\\
98.92	2.63569836399854e-05\\
98.93	2.56734984133162e-05\\
98.94	2.49971380689353e-05\\
98.95	2.43279671068505e-05\\
98.96	2.36660506224232e-05\\
98.97	2.30114543119109e-05\\
98.98	2.23642444780736e-05\\
98.99	2.17244880358122e-05\\
99	2.10922525178785e-05\\
99.01	2.04676060806437e-05\\
99.02	1.98506175098923e-05\\
99.03	1.92413562266989e-05\\
99.04	1.86398922933494e-05\\
99.05	1.804629641931e-05\\
99.06	1.74606399672653e-05\\
99.07	1.68829949592029e-05\\
99.08	1.6313434082562e-05\\
99.09	1.57520306964323e-05\\
99.1	1.51988588378213e-05\\
99.11	1.46539932279633e-05\\
99.12	1.41175092787053e-05\\
99.13	1.35894830989459e-05\\
99.14	1.30699915011235e-05\\
99.15	1.25591120077907e-05\\
99.16	1.20569228582185e-05\\
99.17	1.15635030150888e-05\\
99.18	1.10789321712389e-05\\
99.19	1.06032981187169e-05\\
99.2	1.01366899886816e-05\\
99.21	9.67919775133295e-06\\
99.22	9.23091222374516e-06\\
99.23	8.79192507778105e-06\\
99.24	8.36232884808032e-06\\
99.25	7.94221694011034e-06\\
99.26	7.53168363830208e-06\\
99.27	7.13082411427438e-06\\
99.28	6.73973443509994e-06\\
99.29	6.3585115716875e-06\\
99.3	5.98725340720911e-06\\
99.31	5.62605874563499e-06\\
99.32	5.27502732032906e-06\\
99.33	4.93425980272778e-06\\
99.34	4.60385781110052e-06\\
99.35	4.28392391940528e-06\\
99.36	3.97456166620347e-06\\
99.37	3.67587556367177e-06\\
99.38	3.38797110669906e-06\\
99.39	3.11095478206652e-06\\
99.4	2.84493407771459e-06\\
99.41	2.59001749209654e-06\\
99.42	2.34631454362755e-06\\
99.43	2.11393578021697e-06\\
99.44	1.89299278887875e-06\\
99.45	1.68359820545798e-06\\
99.46	1.48586572441996e-06\\
99.47	1.29991010874159e-06\\
99.48	1.12584719991031e-06\\
99.49	9.63793927985512e-07\\
99.5	8.13868321781347e-07\\
99.51	6.76189519131093e-07\\
99.52	5.50877777248659e-07\\
99.53	4.38054483194172e-07\\
99.54	3.37842164419358e-07\\
99.55	2.50364499436093e-07\\
99.56	1.7574632856128e-07\\
99.57	1.14113664783158e-07\\
99.58	6.55937047039368e-08\\
99.59	3.03148396090663e-08\\
99.6	8.40666662844936e-09\\
99.61	0\\
99.62	0\\
99.63	0\\
99.64	0\\
99.65	0\\
99.66	0\\
99.67	0\\
99.68	0\\
99.69	0\\
99.7	0\\
99.71	0\\
99.72	0\\
99.73	0\\
99.74	0\\
99.75	0\\
99.76	0\\
99.77	0\\
99.78	0\\
99.79	0\\
99.8	0\\
99.81	0\\
99.82	0\\
99.83	0\\
99.84	0\\
99.85	0\\
99.86	0\\
99.87	0\\
99.88	0\\
99.89	0\\
99.9	0\\
99.91	0\\
99.92	0\\
99.93	0\\
99.94	0\\
99.95	0\\
99.96	0\\
99.97	0\\
99.98	0\\
99.99	0\\
100	0\\
};
\addlegendentry{$q=-4$};

\addplot [color=mycolor1,dashed,forget plot]
  table[row sep=crcr]{%
0.01	0.01\\
0.02	0.01\\
0.03	0.01\\
0.04	0.01\\
0.05	0.01\\
0.06	0.01\\
0.07	0.01\\
0.08	0.01\\
0.09	0.01\\
0.1	0.01\\
0.11	0.01\\
0.12	0.01\\
0.13	0.01\\
0.14	0.01\\
0.15	0.01\\
0.16	0.01\\
0.17	0.01\\
0.18	0.01\\
0.19	0.01\\
0.2	0.01\\
0.21	0.01\\
0.22	0.01\\
0.23	0.01\\
0.24	0.01\\
0.25	0.01\\
0.26	0.01\\
0.27	0.01\\
0.28	0.01\\
0.29	0.01\\
0.3	0.01\\
0.31	0.01\\
0.32	0.01\\
0.33	0.01\\
0.34	0.01\\
0.35	0.01\\
0.36	0.01\\
0.37	0.01\\
0.38	0.01\\
0.39	0.01\\
0.4	0.01\\
0.41	0.01\\
0.42	0.01\\
0.43	0.01\\
0.44	0.01\\
0.45	0.01\\
0.46	0.01\\
0.47	0.01\\
0.48	0.01\\
0.49	0.01\\
0.5	0.01\\
0.51	0.01\\
0.52	0.01\\
0.53	0.01\\
0.54	0.01\\
0.55	0.01\\
0.56	0.01\\
0.57	0.01\\
0.58	0.01\\
0.59	0.01\\
0.6	0.01\\
0.61	0.01\\
0.62	0.01\\
0.63	0.01\\
0.64	0.01\\
0.65	0.01\\
0.66	0.01\\
0.67	0.01\\
0.68	0.01\\
0.69	0.01\\
0.7	0.01\\
0.71	0.01\\
0.72	0.01\\
0.73	0.01\\
0.74	0.01\\
0.75	0.01\\
0.76	0.01\\
0.77	0.01\\
0.78	0.01\\
0.79	0.01\\
0.8	0.01\\
0.81	0.01\\
0.82	0.01\\
0.83	0.01\\
0.84	0.01\\
0.85	0.01\\
0.86	0.01\\
0.87	0.01\\
0.88	0.01\\
0.89	0.01\\
0.9	0.01\\
0.91	0.01\\
0.92	0.01\\
0.93	0.01\\
0.94	0.01\\
0.95	0.01\\
0.96	0.01\\
0.97	0.01\\
0.98	0.01\\
0.99	0.01\\
1	0.01\\
1.01	0.01\\
1.02	0.01\\
1.03	0.01\\
1.04	0.01\\
1.05	0.01\\
1.06	0.01\\
1.07	0.01\\
1.08	0.01\\
1.09	0.01\\
1.1	0.01\\
1.11	0.01\\
1.12	0.01\\
1.13	0.01\\
1.14	0.01\\
1.15	0.01\\
1.16	0.01\\
1.17	0.01\\
1.18	0.01\\
1.19	0.01\\
1.2	0.01\\
1.21	0.01\\
1.22	0.01\\
1.23	0.01\\
1.24	0.01\\
1.25	0.01\\
1.26	0.01\\
1.27	0.01\\
1.28	0.01\\
1.29	0.01\\
1.3	0.01\\
1.31	0.01\\
1.32	0.01\\
1.33	0.01\\
1.34	0.01\\
1.35	0.01\\
1.36	0.01\\
1.37	0.01\\
1.38	0.01\\
1.39	0.01\\
1.4	0.01\\
1.41	0.01\\
1.42	0.01\\
1.43	0.01\\
1.44	0.01\\
1.45	0.01\\
1.46	0.01\\
1.47	0.01\\
1.48	0.01\\
1.49	0.01\\
1.5	0.01\\
1.51	0.01\\
1.52	0.01\\
1.53	0.01\\
1.54	0.01\\
1.55	0.01\\
1.56	0.01\\
1.57	0.01\\
1.58	0.01\\
1.59	0.01\\
1.6	0.01\\
1.61	0.01\\
1.62	0.01\\
1.63	0.01\\
1.64	0.01\\
1.65	0.01\\
1.66	0.01\\
1.67	0.01\\
1.68	0.01\\
1.69	0.01\\
1.7	0.01\\
1.71	0.01\\
1.72	0.01\\
1.73	0.01\\
1.74	0.01\\
1.75	0.01\\
1.76	0.01\\
1.77	0.01\\
1.78	0.01\\
1.79	0.01\\
1.8	0.01\\
1.81	0.01\\
1.82	0.01\\
1.83	0.01\\
1.84	0.01\\
1.85	0.01\\
1.86	0.01\\
1.87	0.01\\
1.88	0.01\\
1.89	0.01\\
1.9	0.01\\
1.91	0.01\\
1.92	0.01\\
1.93	0.01\\
1.94	0.01\\
1.95	0.01\\
1.96	0.01\\
1.97	0.01\\
1.98	0.01\\
1.99	0.01\\
2	0.01\\
2.01	0.01\\
2.02	0.01\\
2.03	0.01\\
2.04	0.01\\
2.05	0.01\\
2.06	0.01\\
2.07	0.01\\
2.08	0.01\\
2.09	0.01\\
2.1	0.01\\
2.11	0.01\\
2.12	0.01\\
2.13	0.01\\
2.14	0.01\\
2.15	0.01\\
2.16	0.01\\
2.17	0.01\\
2.18	0.01\\
2.19	0.01\\
2.2	0.01\\
2.21	0.01\\
2.22	0.01\\
2.23	0.01\\
2.24	0.01\\
2.25	0.01\\
2.26	0.01\\
2.27	0.01\\
2.28	0.01\\
2.29	0.01\\
2.3	0.01\\
2.31	0.01\\
2.32	0.01\\
2.33	0.01\\
2.34	0.01\\
2.35	0.01\\
2.36	0.01\\
2.37	0.01\\
2.38	0.01\\
2.39	0.01\\
2.4	0.01\\
2.41	0.01\\
2.42	0.01\\
2.43	0.01\\
2.44	0.01\\
2.45	0.01\\
2.46	0.01\\
2.47	0.01\\
2.48	0.01\\
2.49	0.01\\
2.5	0.01\\
2.51	0.01\\
2.52	0.01\\
2.53	0.01\\
2.54	0.01\\
2.55	0.01\\
2.56	0.01\\
2.57	0.01\\
2.58	0.01\\
2.59	0.01\\
2.6	0.01\\
2.61	0.01\\
2.62	0.01\\
2.63	0.01\\
2.64	0.01\\
2.65	0.01\\
2.66	0.01\\
2.67	0.01\\
2.68	0.01\\
2.69	0.01\\
2.7	0.01\\
2.71	0.01\\
2.72	0.01\\
2.73	0.01\\
2.74	0.01\\
2.75	0.01\\
2.76	0.01\\
2.77	0.01\\
2.78	0.01\\
2.79	0.01\\
2.8	0.01\\
2.81	0.01\\
2.82	0.01\\
2.83	0.01\\
2.84	0.01\\
2.85	0.01\\
2.86	0.01\\
2.87	0.01\\
2.88	0.01\\
2.89	0.01\\
2.9	0.01\\
2.91	0.01\\
2.92	0.01\\
2.93	0.01\\
2.94	0.01\\
2.95	0.01\\
2.96	0.01\\
2.97	0.01\\
2.98	0.01\\
2.99	0.01\\
3	0.01\\
3.01	0.01\\
3.02	0.01\\
3.03	0.01\\
3.04	0.01\\
3.05	0.01\\
3.06	0.01\\
3.07	0.01\\
3.08	0.01\\
3.09	0.01\\
3.1	0.01\\
3.11	0.01\\
3.12	0.01\\
3.13	0.01\\
3.14	0.01\\
3.15	0.01\\
3.16	0.01\\
3.17	0.01\\
3.18	0.01\\
3.19	0.01\\
3.2	0.01\\
3.21	0.01\\
3.22	0.01\\
3.23	0.01\\
3.24	0.01\\
3.25	0.01\\
3.26	0.01\\
3.27	0.01\\
3.28	0.01\\
3.29	0.01\\
3.3	0.01\\
3.31	0.01\\
3.32	0.01\\
3.33	0.01\\
3.34	0.01\\
3.35	0.01\\
3.36	0.01\\
3.37	0.01\\
3.38	0.01\\
3.39	0.01\\
3.4	0.01\\
3.41	0.01\\
3.42	0.01\\
3.43	0.01\\
3.44	0.01\\
3.45	0.01\\
3.46	0.01\\
3.47	0.01\\
3.48	0.01\\
3.49	0.01\\
3.5	0.01\\
3.51	0.01\\
3.52	0.01\\
3.53	0.01\\
3.54	0.01\\
3.55	0.01\\
3.56	0.01\\
3.57	0.01\\
3.58	0.01\\
3.59	0.01\\
3.6	0.01\\
3.61	0.01\\
3.62	0.01\\
3.63	0.01\\
3.64	0.01\\
3.65	0.01\\
3.66	0.01\\
3.67	0.01\\
3.68	0.01\\
3.69	0.01\\
3.7	0.01\\
3.71	0.01\\
3.72	0.01\\
3.73	0.01\\
3.74	0.01\\
3.75	0.01\\
3.76	0.01\\
3.77	0.01\\
3.78	0.01\\
3.79	0.01\\
3.8	0.01\\
3.81	0.01\\
3.82	0.01\\
3.83	0.01\\
3.84	0.01\\
3.85	0.01\\
3.86	0.01\\
3.87	0.01\\
3.88	0.01\\
3.89	0.01\\
3.9	0.01\\
3.91	0.01\\
3.92	0.01\\
3.93	0.01\\
3.94	0.01\\
3.95	0.01\\
3.96	0.01\\
3.97	0.01\\
3.98	0.01\\
3.99	0.01\\
4	0.01\\
4.01	0.01\\
4.02	0.01\\
4.03	0.01\\
4.04	0.01\\
4.05	0.01\\
4.06	0.01\\
4.07	0.01\\
4.08	0.01\\
4.09	0.01\\
4.1	0.01\\
4.11	0.01\\
4.12	0.01\\
4.13	0.01\\
4.14	0.01\\
4.15	0.01\\
4.16	0.01\\
4.17	0.01\\
4.18	0.01\\
4.19	0.01\\
4.2	0.01\\
4.21	0.01\\
4.22	0.01\\
4.23	0.01\\
4.24	0.01\\
4.25	0.01\\
4.26	0.01\\
4.27	0.01\\
4.28	0.01\\
4.29	0.01\\
4.3	0.01\\
4.31	0.01\\
4.32	0.01\\
4.33	0.01\\
4.34	0.01\\
4.35	0.01\\
4.36	0.01\\
4.37	0.01\\
4.38	0.01\\
4.39	0.01\\
4.4	0.01\\
4.41	0.01\\
4.42	0.01\\
4.43	0.01\\
4.44	0.01\\
4.45	0.01\\
4.46	0.01\\
4.47	0.01\\
4.48	0.01\\
4.49	0.01\\
4.5	0.01\\
4.51	0.01\\
4.52	0.01\\
4.53	0.01\\
4.54	0.01\\
4.55	0.01\\
4.56	0.01\\
4.57	0.01\\
4.58	0.01\\
4.59	0.01\\
4.6	0.01\\
4.61	0.01\\
4.62	0.01\\
4.63	0.01\\
4.64	0.01\\
4.65	0.01\\
4.66	0.01\\
4.67	0.01\\
4.68	0.01\\
4.69	0.01\\
4.7	0.01\\
4.71	0.01\\
4.72	0.01\\
4.73	0.01\\
4.74	0.01\\
4.75	0.01\\
4.76	0.01\\
4.77	0.01\\
4.78	0.01\\
4.79	0.01\\
4.8	0.01\\
4.81	0.01\\
4.82	0.01\\
4.83	0.01\\
4.84	0.01\\
4.85	0.01\\
4.86	0.01\\
4.87	0.01\\
4.88	0.01\\
4.89	0.01\\
4.9	0.01\\
4.91	0.01\\
4.92	0.01\\
4.93	0.01\\
4.94	0.01\\
4.95	0.01\\
4.96	0.01\\
4.97	0.01\\
4.98	0.01\\
4.99	0.01\\
5	0.01\\
5.01	0.01\\
5.02	0.01\\
5.03	0.01\\
5.04	0.01\\
5.05	0.01\\
5.06	0.01\\
5.07	0.01\\
5.08	0.01\\
5.09	0.01\\
5.1	0.01\\
5.11	0.01\\
5.12	0.01\\
5.13	0.01\\
5.14	0.01\\
5.15	0.01\\
5.16	0.01\\
5.17	0.01\\
5.18	0.01\\
5.19	0.01\\
5.2	0.01\\
5.21	0.01\\
5.22	0.01\\
5.23	0.01\\
5.24	0.01\\
5.25	0.01\\
5.26	0.01\\
5.27	0.01\\
5.28	0.01\\
5.29	0.01\\
5.3	0.01\\
5.31	0.01\\
5.32	0.01\\
5.33	0.01\\
5.34	0.01\\
5.35	0.01\\
5.36	0.01\\
5.37	0.01\\
5.38	0.01\\
5.39	0.01\\
5.4	0.01\\
5.41	0.01\\
5.42	0.01\\
5.43	0.01\\
5.44	0.01\\
5.45	0.01\\
5.46	0.01\\
5.47	0.01\\
5.48	0.01\\
5.49	0.01\\
5.5	0.01\\
5.51	0.01\\
5.52	0.01\\
5.53	0.01\\
5.54	0.01\\
5.55	0.01\\
5.56	0.01\\
5.57	0.01\\
5.58	0.01\\
5.59	0.01\\
5.6	0.01\\
5.61	0.01\\
5.62	0.01\\
5.63	0.01\\
5.64	0.01\\
5.65	0.01\\
5.66	0.01\\
5.67	0.01\\
5.68	0.01\\
5.69	0.01\\
5.7	0.01\\
5.71	0.01\\
5.72	0.01\\
5.73	0.01\\
5.74	0.01\\
5.75	0.01\\
5.76	0.01\\
5.77	0.01\\
5.78	0.01\\
5.79	0.01\\
5.8	0.01\\
5.81	0.01\\
5.82	0.01\\
5.83	0.01\\
5.84	0.01\\
5.85	0.01\\
5.86	0.01\\
5.87	0.01\\
5.88	0.01\\
5.89	0.01\\
5.9	0.01\\
5.91	0.01\\
5.92	0.01\\
5.93	0.01\\
5.94	0.01\\
5.95	0.01\\
5.96	0.01\\
5.97	0.01\\
5.98	0.01\\
5.99	0.01\\
6	0.01\\
6.01	0.01\\
6.02	0.01\\
6.03	0.01\\
6.04	0.01\\
6.05	0.01\\
6.06	0.01\\
6.07	0.01\\
6.08	0.01\\
6.09	0.01\\
6.1	0.01\\
6.11	0.01\\
6.12	0.01\\
6.13	0.01\\
6.14	0.01\\
6.15	0.01\\
6.16	0.01\\
6.17	0.01\\
6.18	0.01\\
6.19	0.01\\
6.2	0.01\\
6.21	0.01\\
6.22	0.01\\
6.23	0.01\\
6.24	0.01\\
6.25	0.01\\
6.26	0.01\\
6.27	0.01\\
6.28	0.01\\
6.29	0.01\\
6.3	0.01\\
6.31	0.01\\
6.32	0.01\\
6.33	0.01\\
6.34	0.01\\
6.35	0.01\\
6.36	0.01\\
6.37	0.01\\
6.38	0.01\\
6.39	0.01\\
6.4	0.01\\
6.41	0.01\\
6.42	0.01\\
6.43	0.01\\
6.44	0.01\\
6.45	0.01\\
6.46	0.01\\
6.47	0.01\\
6.48	0.01\\
6.49	0.01\\
6.5	0.01\\
6.51	0.01\\
6.52	0.01\\
6.53	0.01\\
6.54	0.01\\
6.55	0.01\\
6.56	0.01\\
6.57	0.01\\
6.58	0.01\\
6.59	0.01\\
6.6	0.01\\
6.61	0.01\\
6.62	0.01\\
6.63	0.01\\
6.64	0.01\\
6.65	0.01\\
6.66	0.01\\
6.67	0.01\\
6.68	0.01\\
6.69	0.01\\
6.7	0.01\\
6.71	0.01\\
6.72	0.01\\
6.73	0.01\\
6.74	0.01\\
6.75	0.01\\
6.76	0.01\\
6.77	0.01\\
6.78	0.01\\
6.79	0.01\\
6.8	0.01\\
6.81	0.01\\
6.82	0.01\\
6.83	0.01\\
6.84	0.01\\
6.85	0.01\\
6.86	0.01\\
6.87	0.01\\
6.88	0.01\\
6.89	0.01\\
6.9	0.01\\
6.91	0.01\\
6.92	0.01\\
6.93	0.01\\
6.94	0.01\\
6.95	0.01\\
6.96	0.01\\
6.97	0.01\\
6.98	0.01\\
6.99	0.01\\
7	0.01\\
7.01	0.01\\
7.02	0.01\\
7.03	0.01\\
7.04	0.01\\
7.05	0.01\\
7.06	0.01\\
7.07	0.01\\
7.08	0.01\\
7.09	0.01\\
7.1	0.01\\
7.11	0.01\\
7.12	0.01\\
7.13	0.01\\
7.14	0.01\\
7.15	0.01\\
7.16	0.01\\
7.17	0.01\\
7.18	0.01\\
7.19	0.01\\
7.2	0.01\\
7.21	0.01\\
7.22	0.01\\
7.23	0.01\\
7.24	0.01\\
7.25	0.01\\
7.26	0.01\\
7.27	0.01\\
7.28	0.01\\
7.29	0.01\\
7.3	0.01\\
7.31	0.01\\
7.32	0.01\\
7.33	0.01\\
7.34	0.01\\
7.35	0.01\\
7.36	0.01\\
7.37	0.01\\
7.38	0.01\\
7.39	0.01\\
7.4	0.01\\
7.41	0.01\\
7.42	0.01\\
7.43	0.01\\
7.44	0.01\\
7.45	0.01\\
7.46	0.01\\
7.47	0.01\\
7.48	0.01\\
7.49	0.01\\
7.5	0.01\\
7.51	0.01\\
7.52	0.01\\
7.53	0.01\\
7.54	0.01\\
7.55	0.01\\
7.56	0.01\\
7.57	0.01\\
7.58	0.01\\
7.59	0.01\\
7.6	0.01\\
7.61	0.01\\
7.62	0.01\\
7.63	0.01\\
7.64	0.01\\
7.65	0.01\\
7.66	0.01\\
7.67	0.01\\
7.68	0.01\\
7.69	0.01\\
7.7	0.01\\
7.71	0.01\\
7.72	0.01\\
7.73	0.01\\
7.74	0.01\\
7.75	0.01\\
7.76	0.01\\
7.77	0.01\\
7.78	0.01\\
7.79	0.01\\
7.8	0.01\\
7.81	0.01\\
7.82	0.01\\
7.83	0.01\\
7.84	0.01\\
7.85	0.01\\
7.86	0.01\\
7.87	0.01\\
7.88	0.01\\
7.89	0.01\\
7.9	0.01\\
7.91	0.01\\
7.92	0.01\\
7.93	0.01\\
7.94	0.01\\
7.95	0.01\\
7.96	0.01\\
7.97	0.01\\
7.98	0.01\\
7.99	0.01\\
8	0.01\\
8.01	0.01\\
8.02	0.01\\
8.03	0.01\\
8.04	0.01\\
8.05	0.01\\
8.06	0.01\\
8.07	0.01\\
8.08	0.01\\
8.09	0.01\\
8.1	0.01\\
8.11	0.01\\
8.12	0.01\\
8.13	0.01\\
8.14	0.01\\
8.15	0.01\\
8.16	0.01\\
8.17	0.01\\
8.18	0.01\\
8.19	0.01\\
8.2	0.01\\
8.21	0.01\\
8.22	0.01\\
8.23	0.01\\
8.24	0.01\\
8.25	0.01\\
8.26	0.01\\
8.27	0.01\\
8.28	0.01\\
8.29	0.01\\
8.3	0.01\\
8.31	0.01\\
8.32	0.01\\
8.33	0.01\\
8.34	0.01\\
8.35	0.01\\
8.36	0.01\\
8.37	0.01\\
8.38	0.01\\
8.39	0.01\\
8.4	0.01\\
8.41	0.01\\
8.42	0.01\\
8.43	0.01\\
8.44	0.01\\
8.45	0.01\\
8.46	0.01\\
8.47	0.01\\
8.48	0.01\\
8.49	0.01\\
8.5	0.01\\
8.51	0.01\\
8.52	0.01\\
8.53	0.01\\
8.54	0.01\\
8.55	0.01\\
8.56	0.01\\
8.57	0.01\\
8.58	0.01\\
8.59	0.01\\
8.6	0.01\\
8.61	0.01\\
8.62	0.01\\
8.63	0.01\\
8.64	0.01\\
8.65	0.01\\
8.66	0.01\\
8.67	0.01\\
8.68	0.01\\
8.69	0.01\\
8.7	0.01\\
8.71	0.01\\
8.72	0.01\\
8.73	0.01\\
8.74	0.01\\
8.75	0.01\\
8.76	0.01\\
8.77	0.01\\
8.78	0.01\\
8.79	0.01\\
8.8	0.01\\
8.81	0.01\\
8.82	0.01\\
8.83	0.01\\
8.84	0.01\\
8.85	0.01\\
8.86	0.01\\
8.87	0.01\\
8.88	0.01\\
8.89	0.01\\
8.9	0.01\\
8.91	0.01\\
8.92	0.01\\
8.93	0.01\\
8.94	0.01\\
8.95	0.01\\
8.96	0.01\\
8.97	0.01\\
8.98	0.01\\
8.99	0.01\\
9	0.01\\
9.01	0.01\\
9.02	0.01\\
9.03	0.01\\
9.04	0.01\\
9.05	0.01\\
9.06	0.01\\
9.07	0.01\\
9.08	0.01\\
9.09	0.01\\
9.1	0.01\\
9.11	0.01\\
9.12	0.01\\
9.13	0.01\\
9.14	0.01\\
9.15	0.01\\
9.16	0.01\\
9.17	0.01\\
9.18	0.01\\
9.19	0.01\\
9.2	0.01\\
9.21	0.01\\
9.22	0.01\\
9.23	0.01\\
9.24	0.01\\
9.25	0.01\\
9.26	0.01\\
9.27	0.01\\
9.28	0.01\\
9.29	0.01\\
9.3	0.01\\
9.31	0.01\\
9.32	0.01\\
9.33	0.01\\
9.34	0.01\\
9.35	0.01\\
9.36	0.01\\
9.37	0.01\\
9.38	0.01\\
9.39	0.01\\
9.4	0.01\\
9.41	0.01\\
9.42	0.01\\
9.43	0.01\\
9.44	0.01\\
9.45	0.01\\
9.46	0.01\\
9.47	0.01\\
9.48	0.01\\
9.49	0.01\\
9.5	0.01\\
9.51	0.01\\
9.52	0.01\\
9.53	0.01\\
9.54	0.01\\
9.55	0.01\\
9.56	0.01\\
9.57	0.01\\
9.58	0.01\\
9.59	0.01\\
9.6	0.01\\
9.61	0.01\\
9.62	0.01\\
9.63	0.01\\
9.64	0.01\\
9.65	0.01\\
9.66	0.01\\
9.67	0.01\\
9.68	0.01\\
9.69	0.01\\
9.7	0.01\\
9.71	0.01\\
9.72	0.01\\
9.73	0.01\\
9.74	0.01\\
9.75	0.01\\
9.76	0.01\\
9.77	0.01\\
9.78	0.01\\
9.79	0.01\\
9.8	0.01\\
9.81	0.01\\
9.82	0.01\\
9.83	0.01\\
9.84	0.01\\
9.85	0.01\\
9.86	0.01\\
9.87	0.01\\
9.88	0.01\\
9.89	0.01\\
9.9	0.01\\
9.91	0.01\\
9.92	0.01\\
9.93	0.01\\
9.94	0.01\\
9.95	0.01\\
9.96	0.01\\
9.97	0.01\\
9.98	0.01\\
9.99	0.01\\
10	0.01\\
10.01	0.01\\
10.02	0.01\\
10.03	0.01\\
10.04	0.01\\
10.05	0.01\\
10.06	0.01\\
10.07	0.01\\
10.08	0.01\\
10.09	0.01\\
10.1	0.01\\
10.11	0.01\\
10.12	0.01\\
10.13	0.01\\
10.14	0.01\\
10.15	0.01\\
10.16	0.01\\
10.17	0.01\\
10.18	0.01\\
10.19	0.01\\
10.2	0.01\\
10.21	0.01\\
10.22	0.01\\
10.23	0.01\\
10.24	0.01\\
10.25	0.01\\
10.26	0.01\\
10.27	0.01\\
10.28	0.01\\
10.29	0.01\\
10.3	0.01\\
10.31	0.01\\
10.32	0.01\\
10.33	0.01\\
10.34	0.01\\
10.35	0.01\\
10.36	0.01\\
10.37	0.01\\
10.38	0.01\\
10.39	0.01\\
10.4	0.01\\
10.41	0.01\\
10.42	0.01\\
10.43	0.01\\
10.44	0.01\\
10.45	0.01\\
10.46	0.01\\
10.47	0.01\\
10.48	0.01\\
10.49	0.01\\
10.5	0.01\\
10.51	0.01\\
10.52	0.01\\
10.53	0.01\\
10.54	0.01\\
10.55	0.01\\
10.56	0.01\\
10.57	0.01\\
10.58	0.01\\
10.59	0.01\\
10.6	0.01\\
10.61	0.01\\
10.62	0.01\\
10.63	0.01\\
10.64	0.01\\
10.65	0.01\\
10.66	0.01\\
10.67	0.01\\
10.68	0.01\\
10.69	0.01\\
10.7	0.01\\
10.71	0.01\\
10.72	0.01\\
10.73	0.01\\
10.74	0.01\\
10.75	0.01\\
10.76	0.01\\
10.77	0.01\\
10.78	0.01\\
10.79	0.01\\
10.8	0.01\\
10.81	0.01\\
10.82	0.01\\
10.83	0.01\\
10.84	0.01\\
10.85	0.01\\
10.86	0.01\\
10.87	0.01\\
10.88	0.01\\
10.89	0.01\\
10.9	0.01\\
10.91	0.01\\
10.92	0.01\\
10.93	0.01\\
10.94	0.01\\
10.95	0.01\\
10.96	0.01\\
10.97	0.01\\
10.98	0.01\\
10.99	0.01\\
11	0.01\\
11.01	0.01\\
11.02	0.01\\
11.03	0.01\\
11.04	0.01\\
11.05	0.01\\
11.06	0.01\\
11.07	0.01\\
11.08	0.01\\
11.09	0.01\\
11.1	0.01\\
11.11	0.01\\
11.12	0.01\\
11.13	0.01\\
11.14	0.01\\
11.15	0.01\\
11.16	0.01\\
11.17	0.01\\
11.18	0.01\\
11.19	0.01\\
11.2	0.01\\
11.21	0.01\\
11.22	0.01\\
11.23	0.01\\
11.24	0.01\\
11.25	0.01\\
11.26	0.01\\
11.27	0.01\\
11.28	0.01\\
11.29	0.01\\
11.3	0.01\\
11.31	0.01\\
11.32	0.01\\
11.33	0.01\\
11.34	0.01\\
11.35	0.01\\
11.36	0.01\\
11.37	0.01\\
11.38	0.01\\
11.39	0.01\\
11.4	0.01\\
11.41	0.01\\
11.42	0.01\\
11.43	0.01\\
11.44	0.01\\
11.45	0.01\\
11.46	0.01\\
11.47	0.01\\
11.48	0.01\\
11.49	0.01\\
11.5	0.01\\
11.51	0.01\\
11.52	0.01\\
11.53	0.01\\
11.54	0.01\\
11.55	0.01\\
11.56	0.01\\
11.57	0.01\\
11.58	0.01\\
11.59	0.01\\
11.6	0.01\\
11.61	0.01\\
11.62	0.01\\
11.63	0.01\\
11.64	0.01\\
11.65	0.01\\
11.66	0.01\\
11.67	0.01\\
11.68	0.01\\
11.69	0.01\\
11.7	0.01\\
11.71	0.01\\
11.72	0.01\\
11.73	0.01\\
11.74	0.01\\
11.75	0.01\\
11.76	0.01\\
11.77	0.01\\
11.78	0.01\\
11.79	0.01\\
11.8	0.01\\
11.81	0.01\\
11.82	0.01\\
11.83	0.01\\
11.84	0.01\\
11.85	0.01\\
11.86	0.01\\
11.87	0.01\\
11.88	0.01\\
11.89	0.01\\
11.9	0.01\\
11.91	0.01\\
11.92	0.01\\
11.93	0.01\\
11.94	0.01\\
11.95	0.01\\
11.96	0.01\\
11.97	0.01\\
11.98	0.01\\
11.99	0.01\\
12	0.01\\
12.01	0.01\\
12.02	0.01\\
12.03	0.01\\
12.04	0.01\\
12.05	0.01\\
12.06	0.01\\
12.07	0.01\\
12.08	0.01\\
12.09	0.01\\
12.1	0.01\\
12.11	0.01\\
12.12	0.01\\
12.13	0.01\\
12.14	0.01\\
12.15	0.01\\
12.16	0.01\\
12.17	0.01\\
12.18	0.01\\
12.19	0.01\\
12.2	0.01\\
12.21	0.01\\
12.22	0.01\\
12.23	0.01\\
12.24	0.01\\
12.25	0.01\\
12.26	0.01\\
12.27	0.01\\
12.28	0.01\\
12.29	0.01\\
12.3	0.01\\
12.31	0.01\\
12.32	0.01\\
12.33	0.01\\
12.34	0.01\\
12.35	0.01\\
12.36	0.01\\
12.37	0.01\\
12.38	0.01\\
12.39	0.01\\
12.4	0.01\\
12.41	0.01\\
12.42	0.01\\
12.43	0.01\\
12.44	0.01\\
12.45	0.01\\
12.46	0.01\\
12.47	0.01\\
12.48	0.01\\
12.49	0.01\\
12.5	0.01\\
12.51	0.01\\
12.52	0.01\\
12.53	0.01\\
12.54	0.01\\
12.55	0.01\\
12.56	0.01\\
12.57	0.01\\
12.58	0.01\\
12.59	0.01\\
12.6	0.01\\
12.61	0.01\\
12.62	0.01\\
12.63	0.01\\
12.64	0.01\\
12.65	0.01\\
12.66	0.01\\
12.67	0.01\\
12.68	0.01\\
12.69	0.01\\
12.7	0.01\\
12.71	0.01\\
12.72	0.01\\
12.73	0.01\\
12.74	0.01\\
12.75	0.01\\
12.76	0.01\\
12.77	0.01\\
12.78	0.01\\
12.79	0.01\\
12.8	0.01\\
12.81	0.01\\
12.82	0.01\\
12.83	0.01\\
12.84	0.01\\
12.85	0.01\\
12.86	0.01\\
12.87	0.01\\
12.88	0.01\\
12.89	0.01\\
12.9	0.01\\
12.91	0.01\\
12.92	0.01\\
12.93	0.01\\
12.94	0.01\\
12.95	0.01\\
12.96	0.01\\
12.97	0.01\\
12.98	0.01\\
12.99	0.01\\
13	0.01\\
13.01	0.01\\
13.02	0.01\\
13.03	0.01\\
13.04	0.01\\
13.05	0.01\\
13.06	0.01\\
13.07	0.01\\
13.08	0.01\\
13.09	0.01\\
13.1	0.01\\
13.11	0.01\\
13.12	0.01\\
13.13	0.01\\
13.14	0.01\\
13.15	0.01\\
13.16	0.01\\
13.17	0.01\\
13.18	0.01\\
13.19	0.01\\
13.2	0.01\\
13.21	0.01\\
13.22	0.01\\
13.23	0.01\\
13.24	0.01\\
13.25	0.01\\
13.26	0.01\\
13.27	0.01\\
13.28	0.01\\
13.29	0.01\\
13.3	0.01\\
13.31	0.01\\
13.32	0.01\\
13.33	0.01\\
13.34	0.01\\
13.35	0.01\\
13.36	0.01\\
13.37	0.01\\
13.38	0.01\\
13.39	0.01\\
13.4	0.01\\
13.41	0.01\\
13.42	0.01\\
13.43	0.01\\
13.44	0.01\\
13.45	0.01\\
13.46	0.01\\
13.47	0.01\\
13.48	0.01\\
13.49	0.01\\
13.5	0.01\\
13.51	0.01\\
13.52	0.01\\
13.53	0.01\\
13.54	0.01\\
13.55	0.01\\
13.56	0.01\\
13.57	0.01\\
13.58	0.01\\
13.59	0.01\\
13.6	0.01\\
13.61	0.01\\
13.62	0.01\\
13.63	0.01\\
13.64	0.01\\
13.65	0.01\\
13.66	0.01\\
13.67	0.01\\
13.68	0.01\\
13.69	0.01\\
13.7	0.01\\
13.71	0.01\\
13.72	0.01\\
13.73	0.01\\
13.74	0.01\\
13.75	0.01\\
13.76	0.01\\
13.77	0.01\\
13.78	0.01\\
13.79	0.01\\
13.8	0.01\\
13.81	0.01\\
13.82	0.01\\
13.83	0.01\\
13.84	0.01\\
13.85	0.01\\
13.86	0.01\\
13.87	0.01\\
13.88	0.01\\
13.89	0.01\\
13.9	0.01\\
13.91	0.01\\
13.92	0.01\\
13.93	0.01\\
13.94	0.01\\
13.95	0.01\\
13.96	0.01\\
13.97	0.01\\
13.98	0.01\\
13.99	0.01\\
14	0.01\\
14.01	0.01\\
14.02	0.01\\
14.03	0.01\\
14.04	0.01\\
14.05	0.01\\
14.06	0.01\\
14.07	0.01\\
14.08	0.01\\
14.09	0.01\\
14.1	0.01\\
14.11	0.01\\
14.12	0.01\\
14.13	0.01\\
14.14	0.01\\
14.15	0.01\\
14.16	0.01\\
14.17	0.01\\
14.18	0.01\\
14.19	0.01\\
14.2	0.01\\
14.21	0.01\\
14.22	0.01\\
14.23	0.01\\
14.24	0.01\\
14.25	0.01\\
14.26	0.01\\
14.27	0.01\\
14.28	0.01\\
14.29	0.01\\
14.3	0.01\\
14.31	0.01\\
14.32	0.01\\
14.33	0.01\\
14.34	0.01\\
14.35	0.01\\
14.36	0.01\\
14.37	0.01\\
14.38	0.01\\
14.39	0.01\\
14.4	0.01\\
14.41	0.01\\
14.42	0.01\\
14.43	0.01\\
14.44	0.01\\
14.45	0.01\\
14.46	0.01\\
14.47	0.01\\
14.48	0.01\\
14.49	0.01\\
14.5	0.01\\
14.51	0.01\\
14.52	0.01\\
14.53	0.01\\
14.54	0.01\\
14.55	0.01\\
14.56	0.01\\
14.57	0.01\\
14.58	0.01\\
14.59	0.01\\
14.6	0.01\\
14.61	0.01\\
14.62	0.01\\
14.63	0.01\\
14.64	0.01\\
14.65	0.01\\
14.66	0.01\\
14.67	0.01\\
14.68	0.01\\
14.69	0.01\\
14.7	0.01\\
14.71	0.01\\
14.72	0.01\\
14.73	0.01\\
14.74	0.01\\
14.75	0.01\\
14.76	0.01\\
14.77	0.01\\
14.78	0.01\\
14.79	0.01\\
14.8	0.01\\
14.81	0.01\\
14.82	0.01\\
14.83	0.01\\
14.84	0.01\\
14.85	0.01\\
14.86	0.01\\
14.87	0.01\\
14.88	0.01\\
14.89	0.01\\
14.9	0.01\\
14.91	0.01\\
14.92	0.01\\
14.93	0.01\\
14.94	0.01\\
14.95	0.01\\
14.96	0.01\\
14.97	0.01\\
14.98	0.01\\
14.99	0.01\\
15	0.01\\
15.01	0.01\\
15.02	0.01\\
15.03	0.01\\
15.04	0.01\\
15.05	0.01\\
15.06	0.01\\
15.07	0.01\\
15.08	0.01\\
15.09	0.01\\
15.1	0.01\\
15.11	0.01\\
15.12	0.01\\
15.13	0.01\\
15.14	0.01\\
15.15	0.01\\
15.16	0.01\\
15.17	0.01\\
15.18	0.01\\
15.19	0.01\\
15.2	0.01\\
15.21	0.01\\
15.22	0.01\\
15.23	0.01\\
15.24	0.01\\
15.25	0.01\\
15.26	0.01\\
15.27	0.01\\
15.28	0.01\\
15.29	0.01\\
15.3	0.01\\
15.31	0.01\\
15.32	0.01\\
15.33	0.01\\
15.34	0.01\\
15.35	0.01\\
15.36	0.01\\
15.37	0.01\\
15.38	0.01\\
15.39	0.01\\
15.4	0.01\\
15.41	0.01\\
15.42	0.01\\
15.43	0.01\\
15.44	0.01\\
15.45	0.01\\
15.46	0.01\\
15.47	0.01\\
15.48	0.01\\
15.49	0.01\\
15.5	0.01\\
15.51	0.01\\
15.52	0.01\\
15.53	0.01\\
15.54	0.01\\
15.55	0.01\\
15.56	0.01\\
15.57	0.01\\
15.58	0.01\\
15.59	0.01\\
15.6	0.01\\
15.61	0.01\\
15.62	0.01\\
15.63	0.01\\
15.64	0.01\\
15.65	0.01\\
15.66	0.01\\
15.67	0.01\\
15.68	0.01\\
15.69	0.01\\
15.7	0.01\\
15.71	0.01\\
15.72	0.01\\
15.73	0.01\\
15.74	0.01\\
15.75	0.01\\
15.76	0.01\\
15.77	0.01\\
15.78	0.01\\
15.79	0.01\\
15.8	0.01\\
15.81	0.01\\
15.82	0.01\\
15.83	0.01\\
15.84	0.01\\
15.85	0.01\\
15.86	0.01\\
15.87	0.01\\
15.88	0.01\\
15.89	0.01\\
15.9	0.01\\
15.91	0.01\\
15.92	0.01\\
15.93	0.01\\
15.94	0.01\\
15.95	0.01\\
15.96	0.01\\
15.97	0.01\\
15.98	0.01\\
15.99	0.01\\
16	0.01\\
16.01	0.01\\
16.02	0.01\\
16.03	0.01\\
16.04	0.01\\
16.05	0.01\\
16.06	0.01\\
16.07	0.01\\
16.08	0.01\\
16.09	0.01\\
16.1	0.01\\
16.11	0.01\\
16.12	0.01\\
16.13	0.01\\
16.14	0.01\\
16.15	0.01\\
16.16	0.01\\
16.17	0.01\\
16.18	0.01\\
16.19	0.01\\
16.2	0.01\\
16.21	0.01\\
16.22	0.01\\
16.23	0.01\\
16.24	0.01\\
16.25	0.01\\
16.26	0.01\\
16.27	0.01\\
16.28	0.01\\
16.29	0.01\\
16.3	0.01\\
16.31	0.01\\
16.32	0.01\\
16.33	0.01\\
16.34	0.01\\
16.35	0.01\\
16.36	0.01\\
16.37	0.01\\
16.38	0.01\\
16.39	0.01\\
16.4	0.01\\
16.41	0.01\\
16.42	0.01\\
16.43	0.01\\
16.44	0.01\\
16.45	0.01\\
16.46	0.01\\
16.47	0.01\\
16.48	0.01\\
16.49	0.01\\
16.5	0.01\\
16.51	0.01\\
16.52	0.01\\
16.53	0.01\\
16.54	0.01\\
16.55	0.01\\
16.56	0.01\\
16.57	0.01\\
16.58	0.01\\
16.59	0.01\\
16.6	0.01\\
16.61	0.01\\
16.62	0.01\\
16.63	0.01\\
16.64	0.01\\
16.65	0.01\\
16.66	0.01\\
16.67	0.01\\
16.68	0.01\\
16.69	0.01\\
16.7	0.01\\
16.71	0.01\\
16.72	0.01\\
16.73	0.01\\
16.74	0.01\\
16.75	0.01\\
16.76	0.01\\
16.77	0.01\\
16.78	0.01\\
16.79	0.01\\
16.8	0.01\\
16.81	0.01\\
16.82	0.01\\
16.83	0.01\\
16.84	0.01\\
16.85	0.01\\
16.86	0.01\\
16.87	0.01\\
16.88	0.01\\
16.89	0.01\\
16.9	0.01\\
16.91	0.01\\
16.92	0.01\\
16.93	0.01\\
16.94	0.01\\
16.95	0.01\\
16.96	0.01\\
16.97	0.01\\
16.98	0.01\\
16.99	0.01\\
17	0.01\\
17.01	0.01\\
17.02	0.01\\
17.03	0.01\\
17.04	0.01\\
17.05	0.01\\
17.06	0.01\\
17.07	0.01\\
17.08	0.01\\
17.09	0.01\\
17.1	0.01\\
17.11	0.01\\
17.12	0.01\\
17.13	0.01\\
17.14	0.01\\
17.15	0.01\\
17.16	0.01\\
17.17	0.01\\
17.18	0.01\\
17.19	0.01\\
17.2	0.01\\
17.21	0.01\\
17.22	0.01\\
17.23	0.01\\
17.24	0.01\\
17.25	0.01\\
17.26	0.01\\
17.27	0.01\\
17.28	0.01\\
17.29	0.01\\
17.3	0.01\\
17.31	0.01\\
17.32	0.01\\
17.33	0.01\\
17.34	0.01\\
17.35	0.01\\
17.36	0.01\\
17.37	0.01\\
17.38	0.01\\
17.39	0.01\\
17.4	0.01\\
17.41	0.01\\
17.42	0.01\\
17.43	0.01\\
17.44	0.01\\
17.45	0.01\\
17.46	0.01\\
17.47	0.01\\
17.48	0.01\\
17.49	0.01\\
17.5	0.01\\
17.51	0.01\\
17.52	0.01\\
17.53	0.01\\
17.54	0.01\\
17.55	0.01\\
17.56	0.01\\
17.57	0.01\\
17.58	0.01\\
17.59	0.01\\
17.6	0.01\\
17.61	0.01\\
17.62	0.01\\
17.63	0.01\\
17.64	0.01\\
17.65	0.01\\
17.66	0.01\\
17.67	0.01\\
17.68	0.01\\
17.69	0.01\\
17.7	0.01\\
17.71	0.01\\
17.72	0.01\\
17.73	0.01\\
17.74	0.01\\
17.75	0.01\\
17.76	0.01\\
17.77	0.01\\
17.78	0.01\\
17.79	0.01\\
17.8	0.01\\
17.81	0.01\\
17.82	0.01\\
17.83	0.01\\
17.84	0.01\\
17.85	0.01\\
17.86	0.01\\
17.87	0.01\\
17.88	0.01\\
17.89	0.01\\
17.9	0.01\\
17.91	0.01\\
17.92	0.01\\
17.93	0.01\\
17.94	0.01\\
17.95	0.01\\
17.96	0.01\\
17.97	0.01\\
17.98	0.01\\
17.99	0.01\\
18	0.01\\
18.01	0.01\\
18.02	0.01\\
18.03	0.01\\
18.04	0.01\\
18.05	0.01\\
18.06	0.01\\
18.07	0.01\\
18.08	0.01\\
18.09	0.01\\
18.1	0.01\\
18.11	0.01\\
18.12	0.01\\
18.13	0.01\\
18.14	0.01\\
18.15	0.01\\
18.16	0.01\\
18.17	0.01\\
18.18	0.01\\
18.19	0.01\\
18.2	0.01\\
18.21	0.01\\
18.22	0.01\\
18.23	0.01\\
18.24	0.01\\
18.25	0.01\\
18.26	0.01\\
18.27	0.01\\
18.28	0.01\\
18.29	0.01\\
18.3	0.01\\
18.31	0.01\\
18.32	0.01\\
18.33	0.01\\
18.34	0.01\\
18.35	0.01\\
18.36	0.01\\
18.37	0.01\\
18.38	0.01\\
18.39	0.01\\
18.4	0.01\\
18.41	0.01\\
18.42	0.01\\
18.43	0.01\\
18.44	0.01\\
18.45	0.01\\
18.46	0.01\\
18.47	0.01\\
18.48	0.01\\
18.49	0.01\\
18.5	0.01\\
18.51	0.01\\
18.52	0.01\\
18.53	0.01\\
18.54	0.01\\
18.55	0.01\\
18.56	0.01\\
18.57	0.01\\
18.58	0.01\\
18.59	0.01\\
18.6	0.01\\
18.61	0.01\\
18.62	0.01\\
18.63	0.01\\
18.64	0.01\\
18.65	0.01\\
18.66	0.01\\
18.67	0.01\\
18.68	0.01\\
18.69	0.01\\
18.7	0.01\\
18.71	0.01\\
18.72	0.01\\
18.73	0.01\\
18.74	0.01\\
18.75	0.01\\
18.76	0.01\\
18.77	0.01\\
18.78	0.01\\
18.79	0.01\\
18.8	0.01\\
18.81	0.01\\
18.82	0.01\\
18.83	0.01\\
18.84	0.01\\
18.85	0.01\\
18.86	0.01\\
18.87	0.01\\
18.88	0.01\\
18.89	0.01\\
18.9	0.01\\
18.91	0.01\\
18.92	0.01\\
18.93	0.01\\
18.94	0.01\\
18.95	0.01\\
18.96	0.01\\
18.97	0.01\\
18.98	0.01\\
18.99	0.01\\
19	0.01\\
19.01	0.01\\
19.02	0.01\\
19.03	0.01\\
19.04	0.01\\
19.05	0.01\\
19.06	0.01\\
19.07	0.01\\
19.08	0.01\\
19.09	0.01\\
19.1	0.01\\
19.11	0.01\\
19.12	0.01\\
19.13	0.01\\
19.14	0.01\\
19.15	0.01\\
19.16	0.01\\
19.17	0.01\\
19.18	0.01\\
19.19	0.01\\
19.2	0.01\\
19.21	0.01\\
19.22	0.01\\
19.23	0.01\\
19.24	0.01\\
19.25	0.01\\
19.26	0.01\\
19.27	0.01\\
19.28	0.01\\
19.29	0.01\\
19.3	0.01\\
19.31	0.01\\
19.32	0.01\\
19.33	0.01\\
19.34	0.01\\
19.35	0.01\\
19.36	0.01\\
19.37	0.01\\
19.38	0.01\\
19.39	0.01\\
19.4	0.01\\
19.41	0.01\\
19.42	0.01\\
19.43	0.01\\
19.44	0.01\\
19.45	0.01\\
19.46	0.01\\
19.47	0.01\\
19.48	0.01\\
19.49	0.01\\
19.5	0.01\\
19.51	0.01\\
19.52	0.01\\
19.53	0.01\\
19.54	0.01\\
19.55	0.01\\
19.56	0.01\\
19.57	0.01\\
19.58	0.01\\
19.59	0.01\\
19.6	0.01\\
19.61	0.01\\
19.62	0.01\\
19.63	0.01\\
19.64	0.01\\
19.65	0.01\\
19.66	0.01\\
19.67	0.01\\
19.68	0.01\\
19.69	0.01\\
19.7	0.01\\
19.71	0.01\\
19.72	0.01\\
19.73	0.01\\
19.74	0.01\\
19.75	0.01\\
19.76	0.01\\
19.77	0.01\\
19.78	0.01\\
19.79	0.01\\
19.8	0.01\\
19.81	0.01\\
19.82	0.01\\
19.83	0.01\\
19.84	0.01\\
19.85	0.01\\
19.86	0.01\\
19.87	0.01\\
19.88	0.01\\
19.89	0.01\\
19.9	0.01\\
19.91	0.01\\
19.92	0.01\\
19.93	0.01\\
19.94	0.01\\
19.95	0.01\\
19.96	0.01\\
19.97	0.01\\
19.98	0.01\\
19.99	0.01\\
20	0.01\\
20.01	0.01\\
20.02	0.01\\
20.03	0.01\\
20.04	0.01\\
20.05	0.01\\
20.06	0.01\\
20.07	0.01\\
20.08	0.01\\
20.09	0.01\\
20.1	0.01\\
20.11	0.01\\
20.12	0.01\\
20.13	0.01\\
20.14	0.01\\
20.15	0.01\\
20.16	0.01\\
20.17	0.01\\
20.18	0.01\\
20.19	0.01\\
20.2	0.01\\
20.21	0.01\\
20.22	0.01\\
20.23	0.01\\
20.24	0.01\\
20.25	0.01\\
20.26	0.01\\
20.27	0.01\\
20.28	0.01\\
20.29	0.01\\
20.3	0.01\\
20.31	0.01\\
20.32	0.01\\
20.33	0.01\\
20.34	0.01\\
20.35	0.01\\
20.36	0.01\\
20.37	0.01\\
20.38	0.01\\
20.39	0.01\\
20.4	0.01\\
20.41	0.01\\
20.42	0.01\\
20.43	0.01\\
20.44	0.01\\
20.45	0.01\\
20.46	0.01\\
20.47	0.01\\
20.48	0.01\\
20.49	0.01\\
20.5	0.01\\
20.51	0.01\\
20.52	0.01\\
20.53	0.01\\
20.54	0.01\\
20.55	0.01\\
20.56	0.01\\
20.57	0.01\\
20.58	0.01\\
20.59	0.01\\
20.6	0.01\\
20.61	0.01\\
20.62	0.01\\
20.63	0.01\\
20.64	0.01\\
20.65	0.01\\
20.66	0.01\\
20.67	0.01\\
20.68	0.01\\
20.69	0.01\\
20.7	0.01\\
20.71	0.01\\
20.72	0.01\\
20.73	0.01\\
20.74	0.01\\
20.75	0.01\\
20.76	0.01\\
20.77	0.01\\
20.78	0.01\\
20.79	0.01\\
20.8	0.01\\
20.81	0.01\\
20.82	0.01\\
20.83	0.01\\
20.84	0.01\\
20.85	0.01\\
20.86	0.01\\
20.87	0.01\\
20.88	0.01\\
20.89	0.01\\
20.9	0.01\\
20.91	0.01\\
20.92	0.01\\
20.93	0.01\\
20.94	0.01\\
20.95	0.01\\
20.96	0.01\\
20.97	0.01\\
20.98	0.01\\
20.99	0.01\\
21	0.01\\
21.01	0.01\\
21.02	0.01\\
21.03	0.01\\
21.04	0.01\\
21.05	0.01\\
21.06	0.01\\
21.07	0.01\\
21.08	0.01\\
21.09	0.01\\
21.1	0.01\\
21.11	0.01\\
21.12	0.01\\
21.13	0.01\\
21.14	0.01\\
21.15	0.01\\
21.16	0.01\\
21.17	0.01\\
21.18	0.01\\
21.19	0.01\\
21.2	0.01\\
21.21	0.01\\
21.22	0.01\\
21.23	0.01\\
21.24	0.01\\
21.25	0.01\\
21.26	0.01\\
21.27	0.01\\
21.28	0.01\\
21.29	0.01\\
21.3	0.01\\
21.31	0.01\\
21.32	0.01\\
21.33	0.01\\
21.34	0.01\\
21.35	0.01\\
21.36	0.01\\
21.37	0.01\\
21.38	0.01\\
21.39	0.01\\
21.4	0.01\\
21.41	0.01\\
21.42	0.01\\
21.43	0.01\\
21.44	0.01\\
21.45	0.01\\
21.46	0.01\\
21.47	0.01\\
21.48	0.01\\
21.49	0.01\\
21.5	0.01\\
21.51	0.01\\
21.52	0.01\\
21.53	0.01\\
21.54	0.01\\
21.55	0.01\\
21.56	0.01\\
21.57	0.01\\
21.58	0.01\\
21.59	0.01\\
21.6	0.01\\
21.61	0.01\\
21.62	0.01\\
21.63	0.01\\
21.64	0.01\\
21.65	0.01\\
21.66	0.01\\
21.67	0.01\\
21.68	0.01\\
21.69	0.01\\
21.7	0.01\\
21.71	0.01\\
21.72	0.01\\
21.73	0.01\\
21.74	0.01\\
21.75	0.01\\
21.76	0.01\\
21.77	0.01\\
21.78	0.01\\
21.79	0.01\\
21.8	0.01\\
21.81	0.01\\
21.82	0.01\\
21.83	0.01\\
21.84	0.01\\
21.85	0.01\\
21.86	0.01\\
21.87	0.01\\
21.88	0.01\\
21.89	0.01\\
21.9	0.01\\
21.91	0.01\\
21.92	0.01\\
21.93	0.01\\
21.94	0.01\\
21.95	0.01\\
21.96	0.01\\
21.97	0.01\\
21.98	0.01\\
21.99	0.01\\
22	0.01\\
22.01	0.01\\
22.02	0.01\\
22.03	0.01\\
22.04	0.01\\
22.05	0.01\\
22.06	0.01\\
22.07	0.01\\
22.08	0.01\\
22.09	0.01\\
22.1	0.01\\
22.11	0.01\\
22.12	0.01\\
22.13	0.01\\
22.14	0.01\\
22.15	0.01\\
22.16	0.01\\
22.17	0.01\\
22.18	0.01\\
22.19	0.01\\
22.2	0.01\\
22.21	0.01\\
22.22	0.01\\
22.23	0.01\\
22.24	0.01\\
22.25	0.01\\
22.26	0.01\\
22.27	0.01\\
22.28	0.01\\
22.29	0.01\\
22.3	0.01\\
22.31	0.01\\
22.32	0.01\\
22.33	0.01\\
22.34	0.01\\
22.35	0.01\\
22.36	0.01\\
22.37	0.01\\
22.38	0.01\\
22.39	0.01\\
22.4	0.01\\
22.41	0.01\\
22.42	0.01\\
22.43	0.01\\
22.44	0.01\\
22.45	0.01\\
22.46	0.01\\
22.47	0.01\\
22.48	0.01\\
22.49	0.01\\
22.5	0.01\\
22.51	0.01\\
22.52	0.01\\
22.53	0.01\\
22.54	0.01\\
22.55	0.01\\
22.56	0.01\\
22.57	0.01\\
22.58	0.01\\
22.59	0.01\\
22.6	0.01\\
22.61	0.01\\
22.62	0.01\\
22.63	0.01\\
22.64	0.01\\
22.65	0.01\\
22.66	0.01\\
22.67	0.01\\
22.68	0.01\\
22.69	0.01\\
22.7	0.01\\
22.71	0.01\\
22.72	0.01\\
22.73	0.01\\
22.74	0.01\\
22.75	0.01\\
22.76	0.01\\
22.77	0.01\\
22.78	0.01\\
22.79	0.01\\
22.8	0.01\\
22.81	0.01\\
22.82	0.01\\
22.83	0.01\\
22.84	0.01\\
22.85	0.01\\
22.86	0.01\\
22.87	0.01\\
22.88	0.01\\
22.89	0.01\\
22.9	0.01\\
22.91	0.01\\
22.92	0.01\\
22.93	0.01\\
22.94	0.01\\
22.95	0.01\\
22.96	0.01\\
22.97	0.01\\
22.98	0.01\\
22.99	0.01\\
23	0.01\\
23.01	0.01\\
23.02	0.01\\
23.03	0.01\\
23.04	0.01\\
23.05	0.01\\
23.06	0.01\\
23.07	0.01\\
23.08	0.01\\
23.09	0.01\\
23.1	0.01\\
23.11	0.01\\
23.12	0.01\\
23.13	0.01\\
23.14	0.01\\
23.15	0.01\\
23.16	0.01\\
23.17	0.01\\
23.18	0.01\\
23.19	0.01\\
23.2	0.01\\
23.21	0.01\\
23.22	0.01\\
23.23	0.01\\
23.24	0.01\\
23.25	0.01\\
23.26	0.01\\
23.27	0.01\\
23.28	0.01\\
23.29	0.01\\
23.3	0.01\\
23.31	0.01\\
23.32	0.01\\
23.33	0.01\\
23.34	0.01\\
23.35	0.01\\
23.36	0.01\\
23.37	0.01\\
23.38	0.01\\
23.39	0.01\\
23.4	0.01\\
23.41	0.01\\
23.42	0.01\\
23.43	0.01\\
23.44	0.01\\
23.45	0.01\\
23.46	0.01\\
23.47	0.01\\
23.48	0.01\\
23.49	0.01\\
23.5	0.01\\
23.51	0.01\\
23.52	0.01\\
23.53	0.01\\
23.54	0.01\\
23.55	0.01\\
23.56	0.01\\
23.57	0.01\\
23.58	0.01\\
23.59	0.01\\
23.6	0.01\\
23.61	0.01\\
23.62	0.01\\
23.63	0.01\\
23.64	0.01\\
23.65	0.01\\
23.66	0.01\\
23.67	0.01\\
23.68	0.01\\
23.69	0.01\\
23.7	0.01\\
23.71	0.01\\
23.72	0.01\\
23.73	0.01\\
23.74	0.01\\
23.75	0.01\\
23.76	0.01\\
23.77	0.01\\
23.78	0.01\\
23.79	0.01\\
23.8	0.01\\
23.81	0.01\\
23.82	0.01\\
23.83	0.01\\
23.84	0.01\\
23.85	0.01\\
23.86	0.01\\
23.87	0.01\\
23.88	0.01\\
23.89	0.01\\
23.9	0.01\\
23.91	0.01\\
23.92	0.01\\
23.93	0.01\\
23.94	0.01\\
23.95	0.01\\
23.96	0.01\\
23.97	0.01\\
23.98	0.01\\
23.99	0.01\\
24	0.01\\
24.01	0.01\\
24.02	0.01\\
24.03	0.01\\
24.04	0.01\\
24.05	0.01\\
24.06	0.01\\
24.07	0.01\\
24.08	0.01\\
24.09	0.01\\
24.1	0.01\\
24.11	0.01\\
24.12	0.01\\
24.13	0.01\\
24.14	0.01\\
24.15	0.01\\
24.16	0.01\\
24.17	0.01\\
24.18	0.01\\
24.19	0.01\\
24.2	0.01\\
24.21	0.01\\
24.22	0.01\\
24.23	0.01\\
24.24	0.01\\
24.25	0.01\\
24.26	0.01\\
24.27	0.01\\
24.28	0.01\\
24.29	0.01\\
24.3	0.01\\
24.31	0.01\\
24.32	0.01\\
24.33	0.01\\
24.34	0.01\\
24.35	0.01\\
24.36	0.01\\
24.37	0.01\\
24.38	0.01\\
24.39	0.01\\
24.4	0.01\\
24.41	0.01\\
24.42	0.01\\
24.43	0.01\\
24.44	0.01\\
24.45	0.01\\
24.46	0.01\\
24.47	0.01\\
24.48	0.01\\
24.49	0.01\\
24.5	0.01\\
24.51	0.01\\
24.52	0.01\\
24.53	0.01\\
24.54	0.01\\
24.55	0.01\\
24.56	0.01\\
24.57	0.01\\
24.58	0.01\\
24.59	0.01\\
24.6	0.01\\
24.61	0.01\\
24.62	0.01\\
24.63	0.01\\
24.64	0.01\\
24.65	0.01\\
24.66	0.01\\
24.67	0.01\\
24.68	0.01\\
24.69	0.01\\
24.7	0.01\\
24.71	0.01\\
24.72	0.01\\
24.73	0.01\\
24.74	0.01\\
24.75	0.01\\
24.76	0.01\\
24.77	0.01\\
24.78	0.01\\
24.79	0.01\\
24.8	0.01\\
24.81	0.01\\
24.82	0.01\\
24.83	0.01\\
24.84	0.01\\
24.85	0.01\\
24.86	0.01\\
24.87	0.01\\
24.88	0.01\\
24.89	0.01\\
24.9	0.01\\
24.91	0.01\\
24.92	0.01\\
24.93	0.01\\
24.94	0.01\\
24.95	0.01\\
24.96	0.01\\
24.97	0.01\\
24.98	0.01\\
24.99	0.01\\
25	0.01\\
25.01	0.01\\
25.02	0.01\\
25.03	0.01\\
25.04	0.01\\
25.05	0.01\\
25.06	0.01\\
25.07	0.01\\
25.08	0.01\\
25.09	0.01\\
25.1	0.01\\
25.11	0.01\\
25.12	0.01\\
25.13	0.01\\
25.14	0.01\\
25.15	0.01\\
25.16	0.01\\
25.17	0.01\\
25.18	0.01\\
25.19	0.01\\
25.2	0.01\\
25.21	0.01\\
25.22	0.01\\
25.23	0.01\\
25.24	0.01\\
25.25	0.01\\
25.26	0.01\\
25.27	0.01\\
25.28	0.01\\
25.29	0.01\\
25.3	0.01\\
25.31	0.01\\
25.32	0.01\\
25.33	0.01\\
25.34	0.01\\
25.35	0.01\\
25.36	0.01\\
25.37	0.01\\
25.38	0.01\\
25.39	0.01\\
25.4	0.01\\
25.41	0.01\\
25.42	0.01\\
25.43	0.01\\
25.44	0.01\\
25.45	0.01\\
25.46	0.01\\
25.47	0.01\\
25.48	0.01\\
25.49	0.01\\
25.5	0.01\\
25.51	0.01\\
25.52	0.01\\
25.53	0.01\\
25.54	0.01\\
25.55	0.01\\
25.56	0.01\\
25.57	0.01\\
25.58	0.01\\
25.59	0.01\\
25.6	0.01\\
25.61	0.01\\
25.62	0.01\\
25.63	0.01\\
25.64	0.01\\
25.65	0.01\\
25.66	0.01\\
25.67	0.01\\
25.68	0.01\\
25.69	0.01\\
25.7	0.01\\
25.71	0.01\\
25.72	0.01\\
25.73	0.01\\
25.74	0.01\\
25.75	0.01\\
25.76	0.01\\
25.77	0.01\\
25.78	0.01\\
25.79	0.01\\
25.8	0.01\\
25.81	0.01\\
25.82	0.01\\
25.83	0.01\\
25.84	0.01\\
25.85	0.01\\
25.86	0.01\\
25.87	0.01\\
25.88	0.01\\
25.89	0.01\\
25.9	0.01\\
25.91	0.01\\
25.92	0.01\\
25.93	0.01\\
25.94	0.01\\
25.95	0.01\\
25.96	0.01\\
25.97	0.01\\
25.98	0.01\\
25.99	0.01\\
26	0.01\\
26.01	0.01\\
26.02	0.01\\
26.03	0.01\\
26.04	0.01\\
26.05	0.01\\
26.06	0.01\\
26.07	0.01\\
26.08	0.01\\
26.09	0.01\\
26.1	0.01\\
26.11	0.01\\
26.12	0.01\\
26.13	0.01\\
26.14	0.01\\
26.15	0.01\\
26.16	0.01\\
26.17	0.01\\
26.18	0.01\\
26.19	0.01\\
26.2	0.01\\
26.21	0.01\\
26.22	0.01\\
26.23	0.01\\
26.24	0.01\\
26.25	0.01\\
26.26	0.01\\
26.27	0.01\\
26.28	0.01\\
26.29	0.01\\
26.3	0.01\\
26.31	0.01\\
26.32	0.01\\
26.33	0.01\\
26.34	0.01\\
26.35	0.01\\
26.36	0.01\\
26.37	0.01\\
26.38	0.01\\
26.39	0.01\\
26.4	0.01\\
26.41	0.01\\
26.42	0.01\\
26.43	0.01\\
26.44	0.01\\
26.45	0.01\\
26.46	0.01\\
26.47	0.01\\
26.48	0.01\\
26.49	0.01\\
26.5	0.01\\
26.51	0.01\\
26.52	0.01\\
26.53	0.01\\
26.54	0.01\\
26.55	0.01\\
26.56	0.01\\
26.57	0.01\\
26.58	0.01\\
26.59	0.01\\
26.6	0.01\\
26.61	0.01\\
26.62	0.01\\
26.63	0.01\\
26.64	0.01\\
26.65	0.01\\
26.66	0.01\\
26.67	0.01\\
26.68	0.01\\
26.69	0.01\\
26.7	0.01\\
26.71	0.01\\
26.72	0.01\\
26.73	0.01\\
26.74	0.01\\
26.75	0.01\\
26.76	0.01\\
26.77	0.01\\
26.78	0.01\\
26.79	0.01\\
26.8	0.01\\
26.81	0.01\\
26.82	0.01\\
26.83	0.01\\
26.84	0.01\\
26.85	0.01\\
26.86	0.01\\
26.87	0.01\\
26.88	0.01\\
26.89	0.01\\
26.9	0.01\\
26.91	0.01\\
26.92	0.01\\
26.93	0.01\\
26.94	0.01\\
26.95	0.01\\
26.96	0.01\\
26.97	0.01\\
26.98	0.01\\
26.99	0.01\\
27	0.01\\
27.01	0.01\\
27.02	0.01\\
27.03	0.01\\
27.04	0.01\\
27.05	0.01\\
27.06	0.01\\
27.07	0.01\\
27.08	0.01\\
27.09	0.01\\
27.1	0.01\\
27.11	0.01\\
27.12	0.01\\
27.13	0.01\\
27.14	0.01\\
27.15	0.01\\
27.16	0.01\\
27.17	0.01\\
27.18	0.01\\
27.19	0.01\\
27.2	0.01\\
27.21	0.01\\
27.22	0.01\\
27.23	0.01\\
27.24	0.01\\
27.25	0.01\\
27.26	0.01\\
27.27	0.01\\
27.28	0.01\\
27.29	0.01\\
27.3	0.01\\
27.31	0.01\\
27.32	0.01\\
27.33	0.01\\
27.34	0.01\\
27.35	0.01\\
27.36	0.01\\
27.37	0.01\\
27.38	0.01\\
27.39	0.01\\
27.4	0.01\\
27.41	0.01\\
27.42	0.01\\
27.43	0.01\\
27.44	0.01\\
27.45	0.01\\
27.46	0.01\\
27.47	0.01\\
27.48	0.01\\
27.49	0.01\\
27.5	0.01\\
27.51	0.01\\
27.52	0.01\\
27.53	0.01\\
27.54	0.01\\
27.55	0.01\\
27.56	0.01\\
27.57	0.01\\
27.58	0.01\\
27.59	0.01\\
27.6	0.01\\
27.61	0.01\\
27.62	0.01\\
27.63	0.01\\
27.64	0.01\\
27.65	0.01\\
27.66	0.01\\
27.67	0.01\\
27.68	0.01\\
27.69	0.01\\
27.7	0.01\\
27.71	0.01\\
27.72	0.01\\
27.73	0.01\\
27.74	0.01\\
27.75	0.01\\
27.76	0.01\\
27.77	0.01\\
27.78	0.01\\
27.79	0.01\\
27.8	0.01\\
27.81	0.01\\
27.82	0.01\\
27.83	0.01\\
27.84	0.01\\
27.85	0.01\\
27.86	0.01\\
27.87	0.01\\
27.88	0.01\\
27.89	0.01\\
27.9	0.01\\
27.91	0.01\\
27.92	0.01\\
27.93	0.01\\
27.94	0.01\\
27.95	0.01\\
27.96	0.01\\
27.97	0.01\\
27.98	0.01\\
27.99	0.01\\
28	0.01\\
28.01	0.01\\
28.02	0.01\\
28.03	0.01\\
28.04	0.01\\
28.05	0.01\\
28.06	0.01\\
28.07	0.01\\
28.08	0.01\\
28.09	0.01\\
28.1	0.01\\
28.11	0.01\\
28.12	0.01\\
28.13	0.01\\
28.14	0.01\\
28.15	0.01\\
28.16	0.01\\
28.17	0.01\\
28.18	0.01\\
28.19	0.01\\
28.2	0.01\\
28.21	0.01\\
28.22	0.01\\
28.23	0.01\\
28.24	0.01\\
28.25	0.01\\
28.26	0.01\\
28.27	0.01\\
28.28	0.01\\
28.29	0.01\\
28.3	0.01\\
28.31	0.01\\
28.32	0.01\\
28.33	0.01\\
28.34	0.01\\
28.35	0.01\\
28.36	0.01\\
28.37	0.01\\
28.38	0.01\\
28.39	0.01\\
28.4	0.01\\
28.41	0.01\\
28.42	0.01\\
28.43	0.01\\
28.44	0.01\\
28.45	0.01\\
28.46	0.01\\
28.47	0.01\\
28.48	0.01\\
28.49	0.01\\
28.5	0.01\\
28.51	0.01\\
28.52	0.01\\
28.53	0.01\\
28.54	0.01\\
28.55	0.01\\
28.56	0.01\\
28.57	0.01\\
28.58	0.01\\
28.59	0.01\\
28.6	0.01\\
28.61	0.01\\
28.62	0.01\\
28.63	0.01\\
28.64	0.01\\
28.65	0.01\\
28.66	0.01\\
28.67	0.01\\
28.68	0.01\\
28.69	0.01\\
28.7	0.01\\
28.71	0.01\\
28.72	0.01\\
28.73	0.01\\
28.74	0.01\\
28.75	0.01\\
28.76	0.01\\
28.77	0.01\\
28.78	0.01\\
28.79	0.01\\
28.8	0.01\\
28.81	0.01\\
28.82	0.01\\
28.83	0.01\\
28.84	0.01\\
28.85	0.01\\
28.86	0.01\\
28.87	0.01\\
28.88	0.01\\
28.89	0.01\\
28.9	0.01\\
28.91	0.01\\
28.92	0.01\\
28.93	0.01\\
28.94	0.01\\
28.95	0.01\\
28.96	0.01\\
28.97	0.01\\
28.98	0.01\\
28.99	0.01\\
29	0.01\\
29.01	0.01\\
29.02	0.01\\
29.03	0.01\\
29.04	0.01\\
29.05	0.01\\
29.06	0.01\\
29.07	0.01\\
29.08	0.01\\
29.09	0.01\\
29.1	0.01\\
29.11	0.01\\
29.12	0.01\\
29.13	0.01\\
29.14	0.01\\
29.15	0.01\\
29.16	0.01\\
29.17	0.01\\
29.18	0.01\\
29.19	0.01\\
29.2	0.01\\
29.21	0.01\\
29.22	0.01\\
29.23	0.01\\
29.24	0.01\\
29.25	0.01\\
29.26	0.01\\
29.27	0.01\\
29.28	0.01\\
29.29	0.01\\
29.3	0.01\\
29.31	0.01\\
29.32	0.01\\
29.33	0.01\\
29.34	0.01\\
29.35	0.01\\
29.36	0.01\\
29.37	0.01\\
29.38	0.01\\
29.39	0.01\\
29.4	0.01\\
29.41	0.01\\
29.42	0.01\\
29.43	0.01\\
29.44	0.01\\
29.45	0.01\\
29.46	0.01\\
29.47	0.01\\
29.48	0.01\\
29.49	0.01\\
29.5	0.01\\
29.51	0.01\\
29.52	0.01\\
29.53	0.01\\
29.54	0.01\\
29.55	0.01\\
29.56	0.01\\
29.57	0.01\\
29.58	0.01\\
29.59	0.01\\
29.6	0.01\\
29.61	0.01\\
29.62	0.01\\
29.63	0.01\\
29.64	0.01\\
29.65	0.01\\
29.66	0.01\\
29.67	0.01\\
29.68	0.01\\
29.69	0.01\\
29.7	0.01\\
29.71	0.01\\
29.72	0.01\\
29.73	0.01\\
29.74	0.01\\
29.75	0.01\\
29.76	0.01\\
29.77	0.01\\
29.78	0.01\\
29.79	0.01\\
29.8	0.01\\
29.81	0.01\\
29.82	0.01\\
29.83	0.01\\
29.84	0.01\\
29.85	0.01\\
29.86	0.01\\
29.87	0.01\\
29.88	0.01\\
29.89	0.01\\
29.9	0.01\\
29.91	0.01\\
29.92	0.01\\
29.93	0.01\\
29.94	0.01\\
29.95	0.01\\
29.96	0.01\\
29.97	0.01\\
29.98	0.01\\
29.99	0.01\\
30	0.01\\
30.01	0.01\\
30.02	0.01\\
30.03	0.01\\
30.04	0.01\\
30.05	0.01\\
30.06	0.01\\
30.07	0.01\\
30.08	0.01\\
30.09	0.01\\
30.1	0.01\\
30.11	0.01\\
30.12	0.01\\
30.13	0.01\\
30.14	0.01\\
30.15	0.01\\
30.16	0.01\\
30.17	0.01\\
30.18	0.01\\
30.19	0.01\\
30.2	0.01\\
30.21	0.01\\
30.22	0.01\\
30.23	0.01\\
30.24	0.01\\
30.25	0.01\\
30.26	0.01\\
30.27	0.01\\
30.28	0.01\\
30.29	0.01\\
30.3	0.01\\
30.31	0.01\\
30.32	0.01\\
30.33	0.01\\
30.34	0.01\\
30.35	0.01\\
30.36	0.01\\
30.37	0.01\\
30.38	0.01\\
30.39	0.01\\
30.4	0.01\\
30.41	0.01\\
30.42	0.01\\
30.43	0.01\\
30.44	0.01\\
30.45	0.01\\
30.46	0.01\\
30.47	0.01\\
30.48	0.01\\
30.49	0.01\\
30.5	0.01\\
30.51	0.01\\
30.52	0.01\\
30.53	0.01\\
30.54	0.01\\
30.55	0.01\\
30.56	0.01\\
30.57	0.01\\
30.58	0.01\\
30.59	0.01\\
30.6	0.01\\
30.61	0.01\\
30.62	0.01\\
30.63	0.01\\
30.64	0.01\\
30.65	0.01\\
30.66	0.01\\
30.67	0.01\\
30.68	0.01\\
30.69	0.01\\
30.7	0.01\\
30.71	0.01\\
30.72	0.01\\
30.73	0.01\\
30.74	0.01\\
30.75	0.01\\
30.76	0.01\\
30.77	0.01\\
30.78	0.01\\
30.79	0.01\\
30.8	0.01\\
30.81	0.01\\
30.82	0.01\\
30.83	0.01\\
30.84	0.01\\
30.85	0.01\\
30.86	0.01\\
30.87	0.01\\
30.88	0.01\\
30.89	0.01\\
30.9	0.01\\
30.91	0.01\\
30.92	0.01\\
30.93	0.01\\
30.94	0.01\\
30.95	0.01\\
30.96	0.01\\
30.97	0.01\\
30.98	0.01\\
30.99	0.01\\
31	0.01\\
31.01	0.01\\
31.02	0.01\\
31.03	0.01\\
31.04	0.01\\
31.05	0.01\\
31.06	0.01\\
31.07	0.01\\
31.08	0.01\\
31.09	0.01\\
31.1	0.01\\
31.11	0.01\\
31.12	0.01\\
31.13	0.01\\
31.14	0.01\\
31.15	0.01\\
31.16	0.01\\
31.17	0.01\\
31.18	0.01\\
31.19	0.01\\
31.2	0.01\\
31.21	0.01\\
31.22	0.01\\
31.23	0.01\\
31.24	0.01\\
31.25	0.01\\
31.26	0.01\\
31.27	0.01\\
31.28	0.01\\
31.29	0.01\\
31.3	0.01\\
31.31	0.01\\
31.32	0.01\\
31.33	0.01\\
31.34	0.01\\
31.35	0.01\\
31.36	0.01\\
31.37	0.01\\
31.38	0.01\\
31.39	0.01\\
31.4	0.01\\
31.41	0.01\\
31.42	0.01\\
31.43	0.01\\
31.44	0.01\\
31.45	0.01\\
31.46	0.01\\
31.47	0.01\\
31.48	0.01\\
31.49	0.01\\
31.5	0.01\\
31.51	0.01\\
31.52	0.01\\
31.53	0.01\\
31.54	0.01\\
31.55	0.01\\
31.56	0.01\\
31.57	0.01\\
31.58	0.01\\
31.59	0.01\\
31.6	0.01\\
31.61	0.01\\
31.62	0.01\\
31.63	0.01\\
31.64	0.01\\
31.65	0.01\\
31.66	0.01\\
31.67	0.01\\
31.68	0.01\\
31.69	0.01\\
31.7	0.01\\
31.71	0.01\\
31.72	0.01\\
31.73	0.01\\
31.74	0.01\\
31.75	0.01\\
31.76	0.01\\
31.77	0.01\\
31.78	0.01\\
31.79	0.01\\
31.8	0.01\\
31.81	0.01\\
31.82	0.01\\
31.83	0.01\\
31.84	0.01\\
31.85	0.01\\
31.86	0.01\\
31.87	0.01\\
31.88	0.01\\
31.89	0.01\\
31.9	0.01\\
31.91	0.01\\
31.92	0.01\\
31.93	0.01\\
31.94	0.01\\
31.95	0.01\\
31.96	0.01\\
31.97	0.01\\
31.98	0.01\\
31.99	0.01\\
32	0.01\\
32.01	0.01\\
32.02	0.01\\
32.03	0.01\\
32.04	0.01\\
32.05	0.01\\
32.06	0.01\\
32.07	0.01\\
32.08	0.01\\
32.09	0.01\\
32.1	0.01\\
32.11	0.01\\
32.12	0.01\\
32.13	0.01\\
32.14	0.01\\
32.15	0.01\\
32.16	0.01\\
32.17	0.01\\
32.18	0.01\\
32.19	0.01\\
32.2	0.01\\
32.21	0.01\\
32.22	0.01\\
32.23	0.01\\
32.24	0.01\\
32.25	0.01\\
32.26	0.01\\
32.27	0.01\\
32.28	0.01\\
32.29	0.01\\
32.3	0.01\\
32.31	0.01\\
32.32	0.01\\
32.33	0.01\\
32.34	0.01\\
32.35	0.01\\
32.36	0.01\\
32.37	0.01\\
32.38	0.01\\
32.39	0.01\\
32.4	0.01\\
32.41	0.01\\
32.42	0.01\\
32.43	0.01\\
32.44	0.01\\
32.45	0.01\\
32.46	0.01\\
32.47	0.01\\
32.48	0.01\\
32.49	0.01\\
32.5	0.01\\
32.51	0.01\\
32.52	0.01\\
32.53	0.01\\
32.54	0.01\\
32.55	0.01\\
32.56	0.01\\
32.57	0.01\\
32.58	0.01\\
32.59	0.01\\
32.6	0.01\\
32.61	0.01\\
32.62	0.01\\
32.63	0.01\\
32.64	0.01\\
32.65	0.01\\
32.66	0.01\\
32.67	0.01\\
32.68	0.01\\
32.69	0.01\\
32.7	0.01\\
32.71	0.01\\
32.72	0.01\\
32.73	0.01\\
32.74	0.01\\
32.75	0.01\\
32.76	0.01\\
32.77	0.01\\
32.78	0.01\\
32.79	0.01\\
32.8	0.01\\
32.81	0.01\\
32.82	0.01\\
32.83	0.01\\
32.84	0.01\\
32.85	0.01\\
32.86	0.01\\
32.87	0.01\\
32.88	0.01\\
32.89	0.01\\
32.9	0.01\\
32.91	0.01\\
32.92	0.01\\
32.93	0.01\\
32.94	0.01\\
32.95	0.01\\
32.96	0.01\\
32.97	0.01\\
32.98	0.01\\
32.99	0.01\\
33	0.01\\
33.01	0.01\\
33.02	0.01\\
33.03	0.01\\
33.04	0.01\\
33.05	0.01\\
33.06	0.01\\
33.07	0.01\\
33.08	0.01\\
33.09	0.01\\
33.1	0.01\\
33.11	0.01\\
33.12	0.01\\
33.13	0.01\\
33.14	0.01\\
33.15	0.01\\
33.16	0.01\\
33.17	0.01\\
33.18	0.01\\
33.19	0.01\\
33.2	0.01\\
33.21	0.01\\
33.22	0.01\\
33.23	0.01\\
33.24	0.01\\
33.25	0.01\\
33.26	0.01\\
33.27	0.01\\
33.28	0.01\\
33.29	0.01\\
33.3	0.01\\
33.31	0.01\\
33.32	0.01\\
33.33	0.01\\
33.34	0.01\\
33.35	0.01\\
33.36	0.01\\
33.37	0.01\\
33.38	0.01\\
33.39	0.01\\
33.4	0.01\\
33.41	0.01\\
33.42	0.01\\
33.43	0.01\\
33.44	0.01\\
33.45	0.01\\
33.46	0.01\\
33.47	0.01\\
33.48	0.01\\
33.49	0.01\\
33.5	0.01\\
33.51	0.01\\
33.52	0.01\\
33.53	0.01\\
33.54	0.01\\
33.55	0.01\\
33.56	0.01\\
33.57	0.01\\
33.58	0.01\\
33.59	0.01\\
33.6	0.01\\
33.61	0.01\\
33.62	0.01\\
33.63	0.01\\
33.64	0.01\\
33.65	0.01\\
33.66	0.01\\
33.67	0.01\\
33.68	0.01\\
33.69	0.01\\
33.7	0.01\\
33.71	0.01\\
33.72	0.01\\
33.73	0.01\\
33.74	0.01\\
33.75	0.01\\
33.76	0.01\\
33.77	0.01\\
33.78	0.01\\
33.79	0.01\\
33.8	0.01\\
33.81	0.01\\
33.82	0.01\\
33.83	0.01\\
33.84	0.01\\
33.85	0.01\\
33.86	0.01\\
33.87	0.01\\
33.88	0.01\\
33.89	0.01\\
33.9	0.01\\
33.91	0.01\\
33.92	0.01\\
33.93	0.01\\
33.94	0.01\\
33.95	0.01\\
33.96	0.01\\
33.97	0.01\\
33.98	0.01\\
33.99	0.01\\
34	0.01\\
34.01	0.01\\
34.02	0.01\\
34.03	0.01\\
34.04	0.01\\
34.05	0.01\\
34.06	0.01\\
34.07	0.01\\
34.08	0.01\\
34.09	0.01\\
34.1	0.01\\
34.11	0.01\\
34.12	0.01\\
34.13	0.01\\
34.14	0.01\\
34.15	0.01\\
34.16	0.01\\
34.17	0.01\\
34.18	0.01\\
34.19	0.01\\
34.2	0.01\\
34.21	0.01\\
34.22	0.01\\
34.23	0.01\\
34.24	0.01\\
34.25	0.01\\
34.26	0.01\\
34.27	0.01\\
34.28	0.01\\
34.29	0.01\\
34.3	0.01\\
34.31	0.01\\
34.32	0.01\\
34.33	0.01\\
34.34	0.01\\
34.35	0.01\\
34.36	0.01\\
34.37	0.01\\
34.38	0.01\\
34.39	0.01\\
34.4	0.01\\
34.41	0.01\\
34.42	0.01\\
34.43	0.01\\
34.44	0.01\\
34.45	0.01\\
34.46	0.01\\
34.47	0.01\\
34.48	0.01\\
34.49	0.01\\
34.5	0.01\\
34.51	0.01\\
34.52	0.01\\
34.53	0.01\\
34.54	0.01\\
34.55	0.01\\
34.56	0.01\\
34.57	0.01\\
34.58	0.01\\
34.59	0.01\\
34.6	0.01\\
34.61	0.01\\
34.62	0.01\\
34.63	0.01\\
34.64	0.01\\
34.65	0.01\\
34.66	0.01\\
34.67	0.01\\
34.68	0.01\\
34.69	0.01\\
34.7	0.01\\
34.71	0.01\\
34.72	0.01\\
34.73	0.01\\
34.74	0.01\\
34.75	0.01\\
34.76	0.01\\
34.77	0.01\\
34.78	0.01\\
34.79	0.01\\
34.8	0.01\\
34.81	0.01\\
34.82	0.01\\
34.83	0.01\\
34.84	0.01\\
34.85	0.01\\
34.86	0.01\\
34.87	0.01\\
34.88	0.01\\
34.89	0.01\\
34.9	0.01\\
34.91	0.01\\
34.92	0.01\\
34.93	0.01\\
34.94	0.01\\
34.95	0.01\\
34.96	0.01\\
34.97	0.01\\
34.98	0.01\\
34.99	0.01\\
35	0.01\\
35.01	0.01\\
35.02	0.01\\
35.03	0.01\\
35.04	0.01\\
35.05	0.01\\
35.06	0.01\\
35.07	0.01\\
35.08	0.01\\
35.09	0.01\\
35.1	0.01\\
35.11	0.01\\
35.12	0.01\\
35.13	0.01\\
35.14	0.01\\
35.15	0.01\\
35.16	0.01\\
35.17	0.01\\
35.18	0.01\\
35.19	0.01\\
35.2	0.01\\
35.21	0.01\\
35.22	0.01\\
35.23	0.01\\
35.24	0.01\\
35.25	0.01\\
35.26	0.01\\
35.27	0.01\\
35.28	0.01\\
35.29	0.01\\
35.3	0.01\\
35.31	0.01\\
35.32	0.01\\
35.33	0.01\\
35.34	0.01\\
35.35	0.01\\
35.36	0.01\\
35.37	0.01\\
35.38	0.01\\
35.39	0.01\\
35.4	0.01\\
35.41	0.01\\
35.42	0.01\\
35.43	0.01\\
35.44	0.01\\
35.45	0.01\\
35.46	0.01\\
35.47	0.01\\
35.48	0.01\\
35.49	0.01\\
35.5	0.01\\
35.51	0.01\\
35.52	0.01\\
35.53	0.01\\
35.54	0.01\\
35.55	0.01\\
35.56	0.01\\
35.57	0.01\\
35.58	0.01\\
35.59	0.01\\
35.6	0.01\\
35.61	0.01\\
35.62	0.01\\
35.63	0.01\\
35.64	0.01\\
35.65	0.01\\
35.66	0.01\\
35.67	0.01\\
35.68	0.01\\
35.69	0.01\\
35.7	0.01\\
35.71	0.01\\
35.72	0.01\\
35.73	0.01\\
35.74	0.01\\
35.75	0.01\\
35.76	0.01\\
35.77	0.01\\
35.78	0.01\\
35.79	0.01\\
35.8	0.01\\
35.81	0.01\\
35.82	0.01\\
35.83	0.01\\
35.84	0.01\\
35.85	0.01\\
35.86	0.01\\
35.87	0.01\\
35.88	0.01\\
35.89	0.01\\
35.9	0.01\\
35.91	0.01\\
35.92	0.01\\
35.93	0.01\\
35.94	0.01\\
35.95	0.01\\
35.96	0.01\\
35.97	0.01\\
35.98	0.01\\
35.99	0.01\\
36	0.01\\
36.01	0.01\\
36.02	0.01\\
36.03	0.01\\
36.04	0.01\\
36.05	0.01\\
36.06	0.01\\
36.07	0.01\\
36.08	0.01\\
36.09	0.01\\
36.1	0.01\\
36.11	0.01\\
36.12	0.01\\
36.13	0.01\\
36.14	0.01\\
36.15	0.01\\
36.16	0.01\\
36.17	0.01\\
36.18	0.01\\
36.19	0.01\\
36.2	0.01\\
36.21	0.01\\
36.22	0.01\\
36.23	0.01\\
36.24	0.01\\
36.25	0.01\\
36.26	0.01\\
36.27	0.01\\
36.28	0.01\\
36.29	0.01\\
36.3	0.01\\
36.31	0.01\\
36.32	0.01\\
36.33	0.01\\
36.34	0.01\\
36.35	0.01\\
36.36	0.01\\
36.37	0.01\\
36.38	0.01\\
36.39	0.01\\
36.4	0.01\\
36.41	0.01\\
36.42	0.01\\
36.43	0.01\\
36.44	0.01\\
36.45	0.01\\
36.46	0.01\\
36.47	0.01\\
36.48	0.01\\
36.49	0.01\\
36.5	0.01\\
36.51	0.01\\
36.52	0.01\\
36.53	0.01\\
36.54	0.01\\
36.55	0.01\\
36.56	0.01\\
36.57	0.01\\
36.58	0.01\\
36.59	0.01\\
36.6	0.01\\
36.61	0.01\\
36.62	0.01\\
36.63	0.01\\
36.64	0.01\\
36.65	0.01\\
36.66	0.01\\
36.67	0.01\\
36.68	0.01\\
36.69	0.01\\
36.7	0.01\\
36.71	0.01\\
36.72	0.01\\
36.73	0.01\\
36.74	0.01\\
36.75	0.01\\
36.76	0.01\\
36.77	0.01\\
36.78	0.01\\
36.79	0.01\\
36.8	0.01\\
36.81	0.01\\
36.82	0.01\\
36.83	0.01\\
36.84	0.01\\
36.85	0.01\\
36.86	0.01\\
36.87	0.01\\
36.88	0.01\\
36.89	0.01\\
36.9	0.01\\
36.91	0.01\\
36.92	0.01\\
36.93	0.01\\
36.94	0.01\\
36.95	0.01\\
36.96	0.01\\
36.97	0.01\\
36.98	0.01\\
36.99	0.01\\
37	0.01\\
37.01	0.01\\
37.02	0.01\\
37.03	0.01\\
37.04	0.01\\
37.05	0.01\\
37.06	0.01\\
37.07	0.01\\
37.08	0.01\\
37.09	0.01\\
37.1	0.01\\
37.11	0.01\\
37.12	0.01\\
37.13	0.01\\
37.14	0.01\\
37.15	0.01\\
37.16	0.01\\
37.17	0.01\\
37.18	0.01\\
37.19	0.01\\
37.2	0.01\\
37.21	0.01\\
37.22	0.01\\
37.23	0.01\\
37.24	0.01\\
37.25	0.01\\
37.26	0.01\\
37.27	0.01\\
37.28	0.01\\
37.29	0.01\\
37.3	0.01\\
37.31	0.01\\
37.32	0.01\\
37.33	0.01\\
37.34	0.01\\
37.35	0.01\\
37.36	0.01\\
37.37	0.01\\
37.38	0.01\\
37.39	0.01\\
37.4	0.01\\
37.41	0.01\\
37.42	0.01\\
37.43	0.01\\
37.44	0.01\\
37.45	0.01\\
37.46	0.01\\
37.47	0.01\\
37.48	0.01\\
37.49	0.01\\
37.5	0.01\\
37.51	0.01\\
37.52	0.01\\
37.53	0.01\\
37.54	0.01\\
37.55	0.01\\
37.56	0.01\\
37.57	0.01\\
37.58	0.01\\
37.59	0.01\\
37.6	0.01\\
37.61	0.01\\
37.62	0.01\\
37.63	0.01\\
37.64	0.01\\
37.65	0.01\\
37.66	0.01\\
37.67	0.01\\
37.68	0.01\\
37.69	0.01\\
37.7	0.01\\
37.71	0.01\\
37.72	0.01\\
37.73	0.01\\
37.74	0.01\\
37.75	0.01\\
37.76	0.01\\
37.77	0.01\\
37.78	0.01\\
37.79	0.01\\
37.8	0.01\\
37.81	0.01\\
37.82	0.01\\
37.83	0.01\\
37.84	0.01\\
37.85	0.01\\
37.86	0.01\\
37.87	0.01\\
37.88	0.01\\
37.89	0.01\\
37.9	0.01\\
37.91	0.01\\
37.92	0.01\\
37.93	0.01\\
37.94	0.01\\
37.95	0.01\\
37.96	0.01\\
37.97	0.01\\
37.98	0.01\\
37.99	0.01\\
38	0.01\\
38.01	0.01\\
38.02	0.01\\
38.03	0.01\\
38.04	0.01\\
38.05	0.01\\
38.06	0.01\\
38.07	0.01\\
38.08	0.01\\
38.09	0.01\\
38.1	0.01\\
38.11	0.01\\
38.12	0.01\\
38.13	0.01\\
38.14	0.01\\
38.15	0.01\\
38.16	0.01\\
38.17	0.01\\
38.18	0.01\\
38.19	0.01\\
38.2	0.01\\
38.21	0.01\\
38.22	0.01\\
38.23	0.01\\
38.24	0.01\\
38.25	0.01\\
38.26	0.01\\
38.27	0.01\\
38.28	0.01\\
38.29	0.01\\
38.3	0.01\\
38.31	0.01\\
38.32	0.01\\
38.33	0.01\\
38.34	0.01\\
38.35	0.01\\
38.36	0.01\\
38.37	0.01\\
38.38	0.01\\
38.39	0.01\\
38.4	0.01\\
38.41	0.01\\
38.42	0.01\\
38.43	0.01\\
38.44	0.01\\
38.45	0.01\\
38.46	0.01\\
38.47	0.01\\
38.48	0.01\\
38.49	0.01\\
38.5	0.01\\
38.51	0.01\\
38.52	0.01\\
38.53	0.01\\
38.54	0.01\\
38.55	0.01\\
38.56	0.01\\
38.57	0.01\\
38.58	0.01\\
38.59	0.01\\
38.6	0.01\\
38.61	0.01\\
38.62	0.01\\
38.63	0.01\\
38.64	0.01\\
38.65	0.01\\
38.66	0.01\\
38.67	0.01\\
38.68	0.01\\
38.69	0.01\\
38.7	0.01\\
38.71	0.01\\
38.72	0.01\\
38.73	0.01\\
38.74	0.01\\
38.75	0.01\\
38.76	0.01\\
38.77	0.01\\
38.78	0.01\\
38.79	0.01\\
38.8	0.01\\
38.81	0.01\\
38.82	0.01\\
38.83	0.01\\
38.84	0.01\\
38.85	0.01\\
38.86	0.01\\
38.87	0.01\\
38.88	0.01\\
38.89	0.01\\
38.9	0.01\\
38.91	0.01\\
38.92	0.01\\
38.93	0.01\\
38.94	0.01\\
38.95	0.01\\
38.96	0.01\\
38.97	0.01\\
38.98	0.01\\
38.99	0.01\\
39	0.01\\
39.01	0.01\\
39.02	0.01\\
39.03	0.01\\
39.04	0.01\\
39.05	0.01\\
39.06	0.01\\
39.07	0.01\\
39.08	0.01\\
39.09	0.01\\
39.1	0.01\\
39.11	0.01\\
39.12	0.01\\
39.13	0.01\\
39.14	0.01\\
39.15	0.01\\
39.16	0.01\\
39.17	0.01\\
39.18	0.01\\
39.19	0.01\\
39.2	0.01\\
39.21	0.01\\
39.22	0.01\\
39.23	0.01\\
39.24	0.01\\
39.25	0.01\\
39.26	0.01\\
39.27	0.01\\
39.28	0.01\\
39.29	0.01\\
39.3	0.01\\
39.31	0.01\\
39.32	0.01\\
39.33	0.01\\
39.34	0.01\\
39.35	0.01\\
39.36	0.01\\
39.37	0.01\\
39.38	0.01\\
39.39	0.01\\
39.4	0.01\\
39.41	0.01\\
39.42	0.01\\
39.43	0.01\\
39.44	0.01\\
39.45	0.01\\
39.46	0.01\\
39.47	0.01\\
39.48	0.01\\
39.49	0.01\\
39.5	0.01\\
39.51	0.01\\
39.52	0.01\\
39.53	0.01\\
39.54	0.01\\
39.55	0.01\\
39.56	0.01\\
39.57	0.01\\
39.58	0.01\\
39.59	0.01\\
39.6	0.01\\
39.61	0.01\\
39.62	0.01\\
39.63	0.01\\
39.64	0.01\\
39.65	0.01\\
39.66	0.01\\
39.67	0.01\\
39.68	0.01\\
39.69	0.01\\
39.7	0.01\\
39.71	0.01\\
39.72	0.01\\
39.73	0.01\\
39.74	0.01\\
39.75	0.01\\
39.76	0.01\\
39.77	0.01\\
39.78	0.01\\
39.79	0.01\\
39.8	0.01\\
39.81	0.01\\
39.82	0.01\\
39.83	0.01\\
39.84	0.01\\
39.85	0.01\\
39.86	0.01\\
39.87	0.01\\
39.88	0.01\\
39.89	0.01\\
39.9	0.01\\
39.91	0.01\\
39.92	0.01\\
39.93	0.01\\
39.94	0.01\\
39.95	0.01\\
39.96	0.01\\
39.97	0.01\\
39.98	0.01\\
39.99	0.01\\
40	0.01\\
40.01	0.01\\
};
\addplot [color=mycolor1,dashed,forget plot]
  table[row sep=crcr]{%
40.01	0.01\\
40.02	0.01\\
40.03	0.01\\
40.04	0.01\\
40.05	0.01\\
40.06	0.01\\
40.07	0.01\\
40.08	0.01\\
40.09	0.01\\
40.1	0.01\\
40.11	0.01\\
40.12	0.01\\
40.13	0.01\\
40.14	0.01\\
40.15	0.01\\
40.16	0.01\\
40.17	0.01\\
40.18	0.01\\
40.19	0.01\\
40.2	0.01\\
40.21	0.01\\
40.22	0.01\\
40.23	0.01\\
40.24	0.01\\
40.25	0.01\\
40.26	0.01\\
40.27	0.01\\
40.28	0.01\\
40.29	0.01\\
40.3	0.01\\
40.31	0.01\\
40.32	0.01\\
40.33	0.01\\
40.34	0.01\\
40.35	0.01\\
40.36	0.01\\
40.37	0.01\\
40.38	0.01\\
40.39	0.01\\
40.4	0.01\\
40.41	0.01\\
40.42	0.01\\
40.43	0.01\\
40.44	0.01\\
40.45	0.01\\
40.46	0.01\\
40.47	0.01\\
40.48	0.01\\
40.49	0.01\\
40.5	0.01\\
40.51	0.01\\
40.52	0.01\\
40.53	0.01\\
40.54	0.01\\
40.55	0.01\\
40.56	0.01\\
40.57	0.01\\
40.58	0.01\\
40.59	0.01\\
40.6	0.01\\
40.61	0.01\\
40.62	0.01\\
40.63	0.01\\
40.64	0.01\\
40.65	0.01\\
40.66	0.01\\
40.67	0.01\\
40.68	0.01\\
40.69	0.01\\
40.7	0.01\\
40.71	0.01\\
40.72	0.01\\
40.73	0.01\\
40.74	0.01\\
40.75	0.01\\
40.76	0.01\\
40.77	0.01\\
40.78	0.01\\
40.79	0.01\\
40.8	0.01\\
40.81	0.01\\
40.82	0.01\\
40.83	0.01\\
40.84	0.01\\
40.85	0.01\\
40.86	0.01\\
40.87	0.01\\
40.88	0.01\\
40.89	0.01\\
40.9	0.01\\
40.91	0.01\\
40.92	0.01\\
40.93	0.01\\
40.94	0.01\\
40.95	0.01\\
40.96	0.01\\
40.97	0.01\\
40.98	0.01\\
40.99	0.01\\
41	0.01\\
41.01	0.01\\
41.02	0.01\\
41.03	0.01\\
41.04	0.01\\
41.05	0.01\\
41.06	0.01\\
41.07	0.01\\
41.08	0.01\\
41.09	0.01\\
41.1	0.01\\
41.11	0.01\\
41.12	0.01\\
41.13	0.01\\
41.14	0.01\\
41.15	0.01\\
41.16	0.01\\
41.17	0.01\\
41.18	0.01\\
41.19	0.01\\
41.2	0.01\\
41.21	0.01\\
41.22	0.01\\
41.23	0.01\\
41.24	0.01\\
41.25	0.01\\
41.26	0.01\\
41.27	0.01\\
41.28	0.01\\
41.29	0.01\\
41.3	0.01\\
41.31	0.01\\
41.32	0.01\\
41.33	0.01\\
41.34	0.01\\
41.35	0.01\\
41.36	0.01\\
41.37	0.01\\
41.38	0.01\\
41.39	0.01\\
41.4	0.01\\
41.41	0.01\\
41.42	0.01\\
41.43	0.01\\
41.44	0.01\\
41.45	0.01\\
41.46	0.01\\
41.47	0.01\\
41.48	0.01\\
41.49	0.01\\
41.5	0.01\\
41.51	0.01\\
41.52	0.01\\
41.53	0.01\\
41.54	0.01\\
41.55	0.01\\
41.56	0.01\\
41.57	0.01\\
41.58	0.01\\
41.59	0.01\\
41.6	0.01\\
41.61	0.01\\
41.62	0.01\\
41.63	0.01\\
41.64	0.01\\
41.65	0.01\\
41.66	0.01\\
41.67	0.01\\
41.68	0.01\\
41.69	0.01\\
41.7	0.01\\
41.71	0.01\\
41.72	0.01\\
41.73	0.01\\
41.74	0.01\\
41.75	0.01\\
41.76	0.01\\
41.77	0.01\\
41.78	0.01\\
41.79	0.01\\
41.8	0.01\\
41.81	0.01\\
41.82	0.01\\
41.83	0.01\\
41.84	0.01\\
41.85	0.01\\
41.86	0.01\\
41.87	0.01\\
41.88	0.01\\
41.89	0.01\\
41.9	0.01\\
41.91	0.01\\
41.92	0.01\\
41.93	0.01\\
41.94	0.01\\
41.95	0.01\\
41.96	0.01\\
41.97	0.01\\
41.98	0.01\\
41.99	0.01\\
42	0.01\\
42.01	0.01\\
42.02	0.01\\
42.03	0.01\\
42.04	0.01\\
42.05	0.01\\
42.06	0.01\\
42.07	0.01\\
42.08	0.01\\
42.09	0.01\\
42.1	0.01\\
42.11	0.01\\
42.12	0.01\\
42.13	0.01\\
42.14	0.01\\
42.15	0.01\\
42.16	0.01\\
42.17	0.01\\
42.18	0.01\\
42.19	0.01\\
42.2	0.01\\
42.21	0.01\\
42.22	0.01\\
42.23	0.01\\
42.24	0.01\\
42.25	0.01\\
42.26	0.01\\
42.27	0.01\\
42.28	0.01\\
42.29	0.01\\
42.3	0.01\\
42.31	0.01\\
42.32	0.01\\
42.33	0.01\\
42.34	0.01\\
42.35	0.01\\
42.36	0.01\\
42.37	0.01\\
42.38	0.01\\
42.39	0.01\\
42.4	0.01\\
42.41	0.01\\
42.42	0.01\\
42.43	0.01\\
42.44	0.01\\
42.45	0.01\\
42.46	0.01\\
42.47	0.01\\
42.48	0.01\\
42.49	0.01\\
42.5	0.01\\
42.51	0.01\\
42.52	0.01\\
42.53	0.01\\
42.54	0.01\\
42.55	0.01\\
42.56	0.01\\
42.57	0.01\\
42.58	0.01\\
42.59	0.01\\
42.6	0.01\\
42.61	0.01\\
42.62	0.01\\
42.63	0.01\\
42.64	0.01\\
42.65	0.01\\
42.66	0.01\\
42.67	0.01\\
42.68	0.01\\
42.69	0.01\\
42.7	0.01\\
42.71	0.01\\
42.72	0.01\\
42.73	0.01\\
42.74	0.01\\
42.75	0.01\\
42.76	0.01\\
42.77	0.01\\
42.78	0.01\\
42.79	0.01\\
42.8	0.01\\
42.81	0.01\\
42.82	0.01\\
42.83	0.01\\
42.84	0.01\\
42.85	0.01\\
42.86	0.01\\
42.87	0.01\\
42.88	0.01\\
42.89	0.01\\
42.9	0.01\\
42.91	0.01\\
42.92	0.01\\
42.93	0.01\\
42.94	0.01\\
42.95	0.01\\
42.96	0.01\\
42.97	0.01\\
42.98	0.01\\
42.99	0.01\\
43	0.01\\
43.01	0.01\\
43.02	0.01\\
43.03	0.01\\
43.04	0.01\\
43.05	0.01\\
43.06	0.01\\
43.07	0.01\\
43.08	0.01\\
43.09	0.01\\
43.1	0.01\\
43.11	0.01\\
43.12	0.01\\
43.13	0.01\\
43.14	0.01\\
43.15	0.01\\
43.16	0.01\\
43.17	0.01\\
43.18	0.01\\
43.19	0.01\\
43.2	0.01\\
43.21	0.01\\
43.22	0.01\\
43.23	0.01\\
43.24	0.01\\
43.25	0.01\\
43.26	0.01\\
43.27	0.01\\
43.28	0.01\\
43.29	0.01\\
43.3	0.01\\
43.31	0.01\\
43.32	0.01\\
43.33	0.01\\
43.34	0.01\\
43.35	0.01\\
43.36	0.01\\
43.37	0.01\\
43.38	0.01\\
43.39	0.01\\
43.4	0.01\\
43.41	0.01\\
43.42	0.01\\
43.43	0.01\\
43.44	0.01\\
43.45	0.01\\
43.46	0.01\\
43.47	0.01\\
43.48	0.01\\
43.49	0.01\\
43.5	0.01\\
43.51	0.01\\
43.52	0.01\\
43.53	0.01\\
43.54	0.01\\
43.55	0.01\\
43.56	0.01\\
43.57	0.01\\
43.58	0.01\\
43.59	0.01\\
43.6	0.01\\
43.61	0.01\\
43.62	0.01\\
43.63	0.01\\
43.64	0.01\\
43.65	0.01\\
43.66	0.01\\
43.67	0.01\\
43.68	0.01\\
43.69	0.01\\
43.7	0.01\\
43.71	0.01\\
43.72	0.01\\
43.73	0.01\\
43.74	0.01\\
43.75	0.01\\
43.76	0.01\\
43.77	0.01\\
43.78	0.01\\
43.79	0.01\\
43.8	0.01\\
43.81	0.01\\
43.82	0.01\\
43.83	0.01\\
43.84	0.01\\
43.85	0.01\\
43.86	0.01\\
43.87	0.01\\
43.88	0.01\\
43.89	0.01\\
43.9	0.01\\
43.91	0.01\\
43.92	0.01\\
43.93	0.01\\
43.94	0.01\\
43.95	0.01\\
43.96	0.01\\
43.97	0.01\\
43.98	0.01\\
43.99	0.01\\
44	0.01\\
44.01	0.01\\
44.02	0.01\\
44.03	0.01\\
44.04	0.01\\
44.05	0.01\\
44.06	0.01\\
44.07	0.01\\
44.08	0.01\\
44.09	0.01\\
44.1	0.01\\
44.11	0.01\\
44.12	0.01\\
44.13	0.01\\
44.14	0.01\\
44.15	0.01\\
44.16	0.01\\
44.17	0.01\\
44.18	0.01\\
44.19	0.01\\
44.2	0.01\\
44.21	0.01\\
44.22	0.01\\
44.23	0.01\\
44.24	0.01\\
44.25	0.01\\
44.26	0.01\\
44.27	0.01\\
44.28	0.01\\
44.29	0.01\\
44.3	0.01\\
44.31	0.01\\
44.32	0.01\\
44.33	0.01\\
44.34	0.01\\
44.35	0.01\\
44.36	0.01\\
44.37	0.01\\
44.38	0.01\\
44.39	0.01\\
44.4	0.01\\
44.41	0.01\\
44.42	0.01\\
44.43	0.01\\
44.44	0.01\\
44.45	0.01\\
44.46	0.01\\
44.47	0.01\\
44.48	0.01\\
44.49	0.01\\
44.5	0.01\\
44.51	0.01\\
44.52	0.01\\
44.53	0.01\\
44.54	0.01\\
44.55	0.01\\
44.56	0.01\\
44.57	0.01\\
44.58	0.01\\
44.59	0.01\\
44.6	0.01\\
44.61	0.01\\
44.62	0.01\\
44.63	0.01\\
44.64	0.01\\
44.65	0.01\\
44.66	0.01\\
44.67	0.01\\
44.68	0.01\\
44.69	0.01\\
44.7	0.01\\
44.71	0.01\\
44.72	0.01\\
44.73	0.01\\
44.74	0.01\\
44.75	0.01\\
44.76	0.01\\
44.77	0.01\\
44.78	0.01\\
44.79	0.01\\
44.8	0.01\\
44.81	0.01\\
44.82	0.01\\
44.83	0.01\\
44.84	0.01\\
44.85	0.01\\
44.86	0.01\\
44.87	0.01\\
44.88	0.01\\
44.89	0.01\\
44.9	0.01\\
44.91	0.01\\
44.92	0.01\\
44.93	0.01\\
44.94	0.01\\
44.95	0.01\\
44.96	0.01\\
44.97	0.01\\
44.98	0.01\\
44.99	0.01\\
45	0.01\\
45.01	0.01\\
45.02	0.01\\
45.03	0.01\\
45.04	0.01\\
45.05	0.01\\
45.06	0.01\\
45.07	0.01\\
45.08	0.01\\
45.09	0.01\\
45.1	0.01\\
45.11	0.01\\
45.12	0.01\\
45.13	0.01\\
45.14	0.01\\
45.15	0.01\\
45.16	0.01\\
45.17	0.01\\
45.18	0.01\\
45.19	0.01\\
45.2	0.01\\
45.21	0.01\\
45.22	0.01\\
45.23	0.01\\
45.24	0.01\\
45.25	0.01\\
45.26	0.01\\
45.27	0.01\\
45.28	0.01\\
45.29	0.01\\
45.3	0.01\\
45.31	0.01\\
45.32	0.01\\
45.33	0.01\\
45.34	0.01\\
45.35	0.01\\
45.36	0.01\\
45.37	0.01\\
45.38	0.01\\
45.39	0.01\\
45.4	0.01\\
45.41	0.01\\
45.42	0.01\\
45.43	0.01\\
45.44	0.01\\
45.45	0.01\\
45.46	0.01\\
45.47	0.01\\
45.48	0.01\\
45.49	0.01\\
45.5	0.01\\
45.51	0.01\\
45.52	0.01\\
45.53	0.01\\
45.54	0.01\\
45.55	0.01\\
45.56	0.01\\
45.57	0.01\\
45.58	0.01\\
45.59	0.01\\
45.6	0.01\\
45.61	0.01\\
45.62	0.01\\
45.63	0.01\\
45.64	0.01\\
45.65	0.01\\
45.66	0.01\\
45.67	0.01\\
45.68	0.01\\
45.69	0.01\\
45.7	0.01\\
45.71	0.01\\
45.72	0.01\\
45.73	0.01\\
45.74	0.01\\
45.75	0.01\\
45.76	0.01\\
45.77	0.01\\
45.78	0.01\\
45.79	0.01\\
45.8	0.01\\
45.81	0.01\\
45.82	0.01\\
45.83	0.01\\
45.84	0.01\\
45.85	0.01\\
45.86	0.01\\
45.87	0.01\\
45.88	0.01\\
45.89	0.01\\
45.9	0.01\\
45.91	0.01\\
45.92	0.01\\
45.93	0.01\\
45.94	0.01\\
45.95	0.01\\
45.96	0.01\\
45.97	0.01\\
45.98	0.01\\
45.99	0.01\\
46	0.01\\
46.01	0.01\\
46.02	0.01\\
46.03	0.01\\
46.04	0.01\\
46.05	0.01\\
46.06	0.01\\
46.07	0.01\\
46.08	0.01\\
46.09	0.01\\
46.1	0.01\\
46.11	0.01\\
46.12	0.01\\
46.13	0.01\\
46.14	0.01\\
46.15	0.01\\
46.16	0.01\\
46.17	0.01\\
46.18	0.01\\
46.19	0.01\\
46.2	0.01\\
46.21	0.01\\
46.22	0.01\\
46.23	0.01\\
46.24	0.01\\
46.25	0.01\\
46.26	0.01\\
46.27	0.01\\
46.28	0.01\\
46.29	0.01\\
46.3	0.01\\
46.31	0.01\\
46.32	0.01\\
46.33	0.01\\
46.34	0.01\\
46.35	0.01\\
46.36	0.01\\
46.37	0.01\\
46.38	0.01\\
46.39	0.01\\
46.4	0.01\\
46.41	0.01\\
46.42	0.01\\
46.43	0.01\\
46.44	0.01\\
46.45	0.01\\
46.46	0.01\\
46.47	0.01\\
46.48	0.01\\
46.49	0.01\\
46.5	0.01\\
46.51	0.01\\
46.52	0.01\\
46.53	0.01\\
46.54	0.01\\
46.55	0.01\\
46.56	0.01\\
46.57	0.01\\
46.58	0.01\\
46.59	0.01\\
46.6	0.01\\
46.61	0.01\\
46.62	0.01\\
46.63	0.01\\
46.64	0.01\\
46.65	0.01\\
46.66	0.01\\
46.67	0.01\\
46.68	0.01\\
46.69	0.01\\
46.7	0.01\\
46.71	0.01\\
46.72	0.01\\
46.73	0.01\\
46.74	0.01\\
46.75	0.01\\
46.76	0.01\\
46.77	0.01\\
46.78	0.01\\
46.79	0.01\\
46.8	0.01\\
46.81	0.01\\
46.82	0.01\\
46.83	0.01\\
46.84	0.01\\
46.85	0.01\\
46.86	0.01\\
46.87	0.01\\
46.88	0.01\\
46.89	0.01\\
46.9	0.01\\
46.91	0.01\\
46.92	0.01\\
46.93	0.01\\
46.94	0.01\\
46.95	0.01\\
46.96	0.01\\
46.97	0.01\\
46.98	0.01\\
46.99	0.01\\
47	0.01\\
47.01	0.01\\
47.02	0.01\\
47.03	0.01\\
47.04	0.01\\
47.05	0.01\\
47.06	0.01\\
47.07	0.01\\
47.08	0.01\\
47.09	0.01\\
47.1	0.01\\
47.11	0.01\\
47.12	0.01\\
47.13	0.01\\
47.14	0.01\\
47.15	0.01\\
47.16	0.01\\
47.17	0.01\\
47.18	0.01\\
47.19	0.01\\
47.2	0.01\\
47.21	0.01\\
47.22	0.01\\
47.23	0.01\\
47.24	0.01\\
47.25	0.01\\
47.26	0.01\\
47.27	0.01\\
47.28	0.01\\
47.29	0.01\\
47.3	0.01\\
47.31	0.01\\
47.32	0.01\\
47.33	0.01\\
47.34	0.01\\
47.35	0.01\\
47.36	0.01\\
47.37	0.01\\
47.38	0.01\\
47.39	0.01\\
47.4	0.01\\
47.41	0.01\\
47.42	0.01\\
47.43	0.01\\
47.44	0.01\\
47.45	0.01\\
47.46	0.01\\
47.47	0.01\\
47.48	0.01\\
47.49	0.01\\
47.5	0.01\\
47.51	0.01\\
47.52	0.01\\
47.53	0.01\\
47.54	0.01\\
47.55	0.01\\
47.56	0.01\\
47.57	0.01\\
47.58	0.01\\
47.59	0.01\\
47.6	0.01\\
47.61	0.01\\
47.62	0.01\\
47.63	0.01\\
47.64	0.01\\
47.65	0.01\\
47.66	0.01\\
47.67	0.01\\
47.68	0.01\\
47.69	0.01\\
47.7	0.01\\
47.71	0.01\\
47.72	0.01\\
47.73	0.01\\
47.74	0.01\\
47.75	0.01\\
47.76	0.01\\
47.77	0.01\\
47.78	0.01\\
47.79	0.01\\
47.8	0.01\\
47.81	0.01\\
47.82	0.01\\
47.83	0.01\\
47.84	0.01\\
47.85	0.01\\
47.86	0.01\\
47.87	0.01\\
47.88	0.01\\
47.89	0.01\\
47.9	0.01\\
47.91	0.01\\
47.92	0.01\\
47.93	0.01\\
47.94	0.01\\
47.95	0.01\\
47.96	0.01\\
47.97	0.01\\
47.98	0.01\\
47.99	0.01\\
48	0.01\\
48.01	0.01\\
48.02	0.01\\
48.03	0.01\\
48.04	0.01\\
48.05	0.01\\
48.06	0.01\\
48.07	0.01\\
48.08	0.01\\
48.09	0.01\\
48.1	0.01\\
48.11	0.01\\
48.12	0.01\\
48.13	0.01\\
48.14	0.01\\
48.15	0.01\\
48.16	0.01\\
48.17	0.01\\
48.18	0.01\\
48.19	0.01\\
48.2	0.01\\
48.21	0.01\\
48.22	0.01\\
48.23	0.01\\
48.24	0.01\\
48.25	0.01\\
48.26	0.01\\
48.27	0.01\\
48.28	0.01\\
48.29	0.01\\
48.3	0.01\\
48.31	0.01\\
48.32	0.01\\
48.33	0.01\\
48.34	0.01\\
48.35	0.01\\
48.36	0.01\\
48.37	0.01\\
48.38	0.01\\
48.39	0.01\\
48.4	0.01\\
48.41	0.01\\
48.42	0.01\\
48.43	0.01\\
48.44	0.01\\
48.45	0.01\\
48.46	0.01\\
48.47	0.01\\
48.48	0.01\\
48.49	0.01\\
48.5	0.01\\
48.51	0.01\\
48.52	0.01\\
48.53	0.00999994939160281\\
48.54	0.00999908902815979\\
48.55	0.00999822815220618\\
48.56	0.00999736676329754\\
48.57	0.00999650486098884\\
48.58	0.00999564244483448\\
48.59	0.00999477951438828\\
48.6	0.00999391606920347\\
48.61	0.00999305210883271\\
48.62	0.00999218763282806\\
48.63	0.009991322640741\\
48.64	0.00999045713212242\\
48.65	0.00998959110652264\\
48.66	0.00998872456349135\\
48.67	0.00998785750257772\\
48.68	0.00998698992333025\\
48.69	0.0099861218252969\\
48.7	0.00998525320802502\\
48.71	0.00998438407106136\\
48.72	0.0099835144139521\\
48.73	0.00998264423624279\\
48.74	0.00998177353747841\\
48.75	0.00998090231720331\\
48.76	0.00998003057496127\\
48.77	0.00997915831029547\\
48.78	0.00997828552274846\\
48.79	0.0099774122118622\\
48.8	0.00997653837717807\\
48.81	0.00997566401823681\\
48.82	0.00997478913457856\\
48.83	0.00997391372574287\\
48.84	0.00997303779126867\\
48.85	0.00997216133069428\\
48.86	0.00997128434355739\\
48.87	0.00997040682939513\\
48.88	0.00996952878774396\\
48.89	0.00996865021813976\\
48.9	0.00996777112011778\\
48.91	0.00996689149321265\\
48.92	0.00996601133695839\\
48.93	0.00996513065088841\\
48.94	0.00996424943453549\\
48.95	0.00996336768743177\\
48.96	0.0099624854091088\\
48.97	0.00996160259909748\\
48.98	0.0099607192569281\\
48.99	0.00995983538213033\\
49	0.00995895097423318\\
49.01	0.00995806603276508\\
49.02	0.00995718055725377\\
49.03	0.00995629454722643\\
49.04	0.00995540800220955\\
49.05	0.00995452092172901\\
49.06	0.00995363330531005\\
49.07	0.00995274515247729\\
49.08	0.00995185646275469\\
49.09	0.00995096723566558\\
49.1	0.00995007747073266\\
49.11	0.00994918716747798\\
49.12	0.00994829632542295\\
49.13	0.00994740494408833\\
49.14	0.00994651302299425\\
49.15	0.0099456205616602\\
49.16	0.00994472755960499\\
49.17	0.00994383401634682\\
49.18	0.00994293993140321\\
49.19	0.00994204530429105\\
49.2	0.00994115013452656\\
49.21	0.00994025442162535\\
49.22	0.00993935816510231\\
49.23	0.00993846136447174\\
49.24	0.00993756401924723\\
49.25	0.00993666612894175\\
49.26	0.00993576769306758\\
49.27	0.00993486871113637\\
49.28	0.0099339691826591\\
49.29	0.00993306910714606\\
49.3	0.0099321684841069\\
49.31	0.00993126731305063\\
49.32	0.00993036559348553\\
49.33	0.00992946332491926\\
49.34	0.0099285605068588\\
49.35	0.00992765713881045\\
49.36	0.00992675322027986\\
49.37	0.00992584875077197\\
49.38	0.00992494372979109\\
49.39	0.00992403815684082\\
49.4	0.00992313203142409\\
49.41	0.00992222535304316\\
49.42	0.00992131812119962\\
49.43	0.00992041033539434\\
49.44	0.00991950199512755\\
49.45	0.00991859309989877\\
49.46	0.00991768364920686\\
49.47	0.00991677364254996\\
49.48	0.00991586307942555\\
49.49	0.00991495195933041\\
49.5	0.00991404028176063\\
49.51	0.00991312804621162\\
49.52	0.00991221525217807\\
49.53	0.009911301899154\\
49.54	0.00991038798663274\\
49.55	0.00990947351410688\\
49.56	0.00990855848106837\\
49.57	0.00990764288700842\\
49.58	0.00990672673141755\\
49.59	0.00990581001378558\\
49.6	0.00990489273360162\\
49.61	0.00990397489035409\\
49.62	0.00990305648353068\\
49.63	0.0099021375126184\\
49.64	0.00990121797710351\\
49.65	0.00990029787647161\\
49.66	0.00989937721020756\\
49.67	0.0098984559777955\\
49.68	0.00989753417871886\\
49.69	0.00989661181246037\\
49.7	0.00989568887850202\\
49.71	0.00989476537632509\\
49.72	0.00989384130541016\\
49.73	0.00989291666523705\\
49.74	0.00989199145528488\\
49.75	0.00989106567503205\\
49.76	0.0098901393239562\\
49.77	0.00988921240153429\\
49.78	0.00988828490724252\\
49.79	0.00988735684055637\\
49.8	0.00988642820095057\\
49.81	0.00988549898789915\\
49.82	0.00988456920087538\\
49.83	0.0098836388393518\\
49.84	0.00988270790280022\\
49.85	0.00988177639069169\\
49.86	0.00988084430249655\\
49.87	0.00987991163768437\\
49.88	0.00987897839572399\\
49.89	0.0098780445760835\\
49.9	0.00987711017823025\\
49.91	0.00987617520163084\\
49.92	0.00987523964575111\\
49.93	0.00987430351005616\\
49.94	0.00987336679401034\\
49.95	0.00987242949707724\\
49.96	0.00987149161871969\\
49.97	0.00987055315839977\\
49.98	0.00986961411557881\\
49.99	0.00986867448971735\\
50	0.00986773428027521\\
50.01	0.00986679348671141\\
50.02	0.00986585210848423\\
50.03	0.00986491014505117\\
50.04	0.00986396759586897\\
50.05	0.00986302446039359\\
50.06	0.00986208073808025\\
50.07	0.00986113642838335\\
50.08	0.00986019153075655\\
50.09	0.00985924604465273\\
50.1	0.00985829996952399\\
50.11	0.00985735330482164\\
50.12	0.00985640604999624\\
50.13	0.00985545820449753\\
50.14	0.0098545097677745\\
50.15	0.00985356073927533\\
50.16	0.00985261111844743\\
50.17	0.00985166090473743\\
50.18	0.00985071009759114\\
50.19	0.0098497586964536\\
50.2	0.00984880670076907\\
50.21	0.00984785410998098\\
50.22	0.00984690092353199\\
50.23	0.00984594714086397\\
50.24	0.00984499276141795\\
50.25	0.00984403778463422\\
50.26	0.00984308220995221\\
50.27	0.00984212603681059\\
50.28	0.00984116926464719\\
50.29	0.00984021189289905\\
50.3	0.00983925392100241\\
50.31	0.00983829534839268\\
50.32	0.00983733617450448\\
50.33	0.00983637639877158\\
50.34	0.00983541602062698\\
50.35	0.00983445503950284\\
50.36	0.00983349345483048\\
50.37	0.00983253126604045\\
50.38	0.00983156847256243\\
50.39	0.0098306050738253\\
50.4	0.00982964106925712\\
50.41	0.0098286764582851\\
50.42	0.00982771124033566\\
50.43	0.00982674541483433\\
50.44	0.00982577898120587\\
50.45	0.00982481193887416\\
50.46	0.00982384428726227\\
50.47	0.00982287602579244\\
50.48	0.00982190715388604\\
50.49	0.00982093767096362\\
50.5	0.00981996757644488\\
50.51	0.00981899686974869\\
50.52	0.00981802555029305\\
50.53	0.00981705361749514\\
50.54	0.00981608107077127\\
50.55	0.00981510790953691\\
50.56	0.00981413413320668\\
50.57	0.00981315974119431\\
50.58	0.00981218473291273\\
50.59	0.00981120910777398\\
50.6	0.00981023286518923\\
50.61	0.00980925600456881\\
50.62	0.00980827852532218\\
50.63	0.00980730042685794\\
50.64	0.00980632170858381\\
50.65	0.00980534236990665\\
50.66	0.00980436241023245\\
50.67	0.00980338182896632\\
50.68	0.0098024006255125\\
50.69	0.00980141879927435\\
50.7	0.00980043634965438\\
50.71	0.00979945327605418\\
50.72	0.00979846957787447\\
50.73	0.00979748525451511\\
50.74	0.00979650030537504\\
50.75	0.00979551472985235\\
50.76	0.00979452852734421\\
50.77	0.00979354169724691\\
50.78	0.00979255423895586\\
50.79	0.00979156615186555\\
50.8	0.00979057743536959\\
50.81	0.00978958808886071\\
50.82	0.00978859811173069\\
50.83	0.00978760750337047\\
50.84	0.00978661626317003\\
50.85	0.00978562439051848\\
50.86	0.00978463188480401\\
50.87	0.00978363874541389\\
50.88	0.00978264497173452\\
50.89	0.00978165056315133\\
50.9	0.00978065551904887\\
50.91	0.00977965983881078\\
50.92	0.00977866352181975\\
50.93	0.00977766656745756\\
50.94	0.00977666897510509\\
50.95	0.00977567074414228\\
50.96	0.00977467187394813\\
50.97	0.00977367236390072\\
50.98	0.00977267221337721\\
50.99	0.00977167142175382\\
51	0.00977066998840583\\
51.01	0.0097696679127076\\
51.02	0.00976866519403252\\
51.03	0.00976766183175308\\
51.04	0.00976665782524079\\
51.05	0.00976565317386625\\
51.06	0.00976464787699908\\
51.07	0.00976364193400798\\
51.08	0.00976263534426068\\
51.09	0.00976162810712397\\
51.1	0.00976062022196368\\
51.11	0.00975961168814468\\
51.12	0.00975860250503089\\
51.13	0.00975759267198526\\
51.14	0.00975658218836979\\
51.15	0.0097555710535455\\
51.16	0.00975455926687246\\
51.17	0.00975354682770976\\
51.18	0.00975253373541552\\
51.19	0.00975151998934689\\
51.2	0.00975050558886003\\
51.21	0.00974949053331016\\
51.22	0.00974847482205149\\
51.23	0.00974745845443725\\
51.24	0.00974644142981969\\
51.25	0.0097454237475501\\
51.26	0.00974440540697874\\
51.27	0.0097433864074549\\
51.28	0.00974236674832689\\
51.29	0.00974134642894202\\
51.3	0.00974032544864658\\
51.31	0.0097393038067859\\
51.32	0.00973828150270428\\
51.33	0.00973725853574504\\
51.34	0.00973623490525047\\
51.35	0.00973521061056188\\
51.36	0.00973418565101955\\
51.37	0.00973316002596276\\
51.38	0.00973213373472977\\
51.39	0.00973110677665784\\
51.4	0.0097300791510832\\
51.41	0.00972905085734106\\
51.42	0.00972802189476561\\
51.43	0.00972699226269001\\
51.44	0.00972596196044641\\
51.45	0.0097249309873659\\
51.46	0.00972389934277859\\
51.47	0.00972286702601351\\
51.48	0.00972183403639867\\
51.49	0.00972080037326104\\
51.5	0.00971976603592657\\
51.51	0.00971873102372015\\
51.52	0.00971769533596562\\
51.53	0.00971665897198578\\
51.54	0.0097156219311024\\
51.55	0.00971458421263617\\
51.56	0.00971354581590673\\
51.57	0.00971250674023269\\
51.58	0.00971146698493157\\
51.59	0.00971042654931987\\
51.6	0.00970938543271297\\
51.61	0.00970834363442523\\
51.62	0.00970730115376994\\
51.63	0.0097062579900593\\
51.64	0.00970521414260446\\
51.65	0.00970416961071547\\
51.66	0.00970312439370132\\
51.67	0.00970207849086993\\
51.68	0.00970103190152813\\
51.69	0.00969998462498164\\
51.7	0.00969893666053515\\
51.71	0.00969788800749222\\
51.72	0.00969683866515532\\
51.73	0.00969578863282584\\
51.74	0.00969473790980407\\
51.75	0.00969368649538921\\
51.76	0.00969263438887935\\
51.77	0.00969158158957147\\
51.78	0.00969052809676147\\
51.79	0.00968947390974412\\
51.8	0.00968841902781309\\
51.81	0.00968736345026093\\
51.82	0.00968630717637907\\
51.83	0.00968525020545786\\
51.84	0.00968419253678648\\
51.85	0.00968313416965302\\
51.86	0.00968207510334442\\
51.87	0.00968101533714652\\
51.88	0.00967995487034402\\
51.89	0.00967889370222048\\
51.9	0.00967783183205834\\
51.91	0.00967676925913888\\
51.92	0.00967570598274227\\
51.93	0.0096746420021475\\
51.94	0.00967357731663246\\
51.95	0.00967251192547385\\
51.96	0.00967144582794725\\
51.97	0.00967037902332708\\
51.98	0.00966931151088659\\
51.99	0.00966824328989789\\
52	0.00966717435963193\\
52.01	0.00966610471935848\\
52.02	0.00966503436834615\\
52.03	0.00966396330586241\\
52.04	0.00966289153117351\\
52.05	0.00966181904354458\\
52.06	0.00966074584223953\\
52.07	0.00965967192652112\\
52.08	0.0096585972956509\\
52.09	0.00965752194888928\\
52.1	0.00965644588549545\\
52.11	0.0096553691047274\\
52.12	0.00965429160584197\\
52.13	0.00965321338809477\\
52.14	0.00965213445074023\\
52.15	0.00965105479303159\\
52.16	0.00964997441422085\\
52.17	0.00964889331355884\\
52.18	0.00964781149029517\\
52.19	0.00964672894367824\\
52.2	0.00964564567295523\\
52.21	0.00964456167737212\\
52.22	0.00964347695617367\\
52.23	0.00964239150860339\\
52.24	0.00964130533390361\\
52.25	0.00964021843131539\\
52.26	0.00963913080007858\\
52.27	0.00963804243943181\\
52.28	0.00963695334861245\\
52.29	0.00963586352685666\\
52.3	0.00963477297339932\\
52.31	0.0096336816874741\\
52.32	0.00963258966831341\\
52.33	0.00963149691514842\\
52.34	0.00963040342720904\\
52.35	0.00962930920372392\\
52.36	0.00962821424392046\\
52.37	0.0096271185470248\\
52.38	0.00962602211226181\\
52.39	0.0096249249388551\\
52.4	0.00962382702602699\\
52.41	0.00962272837299858\\
52.42	0.00962162897898963\\
52.43	0.00962052884321866\\
52.44	0.00961942796490291\\
52.45	0.00961832634325833\\
52.46	0.00961722397749957\\
52.47	0.00961612086683999\\
52.48	0.00961501701049169\\
52.49	0.00961391240766545\\
52.5	0.00961280705757074\\
52.51	0.00961170095941575\\
52.52	0.00961059411240736\\
52.53	0.00960948651575113\\
52.54	0.00960837816865134\\
52.55	0.00960726907031092\\
52.56	0.0096061592199315\\
52.57	0.00960504861671339\\
52.58	0.00960393725985559\\
52.59	0.00960282514855574\\
52.6	0.0096017122820102\\
52.61	0.00960059865941394\\
52.62	0.00959948427996065\\
52.63	0.00959836914284265\\
52.64	0.00959725324725093\\
52.65	0.00959613659237513\\
52.66	0.00959501917740355\\
52.67	0.00959390100152314\\
52.68	0.00959278206391947\\
52.69	0.00959166236377681\\
52.7	0.00959054190027801\\
52.71	0.0095894206726046\\
52.72	0.00958829867993672\\
52.73	0.00958717592145315\\
52.74	0.0095860523963313\\
52.75	0.00958492810374719\\
52.76	0.00958380304287549\\
52.77	0.00958267721288945\\
52.78	0.00958155061296097\\
52.79	0.00958042324226055\\
52.8	0.00957929509995728\\
52.81	0.00957816618521889\\
52.82	0.00957703649721167\\
52.83	0.00957590603510055\\
52.84	0.00957477479804904\\
52.85	0.00957364278521924\\
52.86	0.00957250999577182\\
52.87	0.00957137642886606\\
52.88	0.00957024208365983\\
52.89	0.00956910695930955\\
52.9	0.00956797105497025\\
52.91	0.0095668343697955\\
52.92	0.00956569690293746\\
52.93	0.00956455865354684\\
52.94	0.00956341962077293\\
52.95	0.00956227980376357\\
52.96	0.00956113920166514\\
52.97	0.00955999781362261\\
52.98	0.00955885563877947\\
52.99	0.00955771267627776\\
53	0.00955656892525806\\
53.01	0.0095554243848595\\
53.02	0.00955427905421974\\
53.03	0.00955313293247497\\
53.04	0.0095519860187599\\
53.05	0.00955083831220778\\
53.06	0.00954968981195037\\
53.07	0.00954854051711795\\
53.08	0.00954739042683933\\
53.09	0.00954623954024181\\
53.1	0.00954508785645119\\
53.11	0.00954393537459182\\
53.12	0.0095427820937865\\
53.13	0.00954162801315654\\
53.14	0.00954047313182176\\
53.15	0.00953931744890046\\
53.16	0.00953816096350941\\
53.17	0.00953700367476389\\
53.18	0.00953584558177765\\
53.19	0.0095346866836629\\
53.2	0.00953352697953033\\
53.21	0.00953236646848911\\
53.22	0.00953120514964688\\
53.23	0.0095300430221097\\
53.24	0.00952888008498211\\
53.25	0.00952771633736713\\
53.26	0.0095265517783662\\
53.27	0.0095253864070792\\
53.28	0.00952422022260447\\
53.29	0.00952305322403878\\
53.3	0.00952188541047735\\
53.31	0.00952071678101381\\
53.32	0.00951954733474022\\
53.33	0.00951837707074709\\
53.34	0.0095172059881233\\
53.35	0.00951603408595621\\
53.36	0.00951486136333154\\
53.37	0.00951368781933343\\
53.38	0.00951251345304444\\
53.39	0.00951133826354553\\
53.4	0.00951016224991603\\
53.41	0.0095089854112337\\
53.42	0.00950780774657467\\
53.43	0.00950662925501345\\
53.44	0.00950544993562294\\
53.45	0.00950426978747441\\
53.46	0.00950308880963753\\
53.47	0.00950190700118031\\
53.48	0.00950072436116913\\
53.49	0.00949954088866874\\
53.5	0.00949835658274225\\
53.51	0.00949717144245112\\
53.52	0.00949598546685516\\
53.53	0.00949479865501253\\
53.54	0.00949361100597972\\
53.55	0.00949242251881158\\
53.56	0.00949123319256126\\
53.57	0.00949004302628028\\
53.58	0.00948885201901846\\
53.59	0.00948766016982395\\
53.6	0.00948646747774321\\
53.61	0.00948527394182103\\
53.62	0.0094840795611005\\
53.63	0.00948288433462301\\
53.64	0.00948168826142827\\
53.65	0.00948049134055426\\
53.66	0.00947929357103727\\
53.67	0.00947809495191189\\
53.68	0.00947689548221097\\
53.69	0.00947569516096566\\
53.7	0.00947449398720538\\
53.71	0.00947329195995781\\
53.72	0.00947208907824894\\
53.73	0.00947088534110296\\
53.74	0.00946968074754237\\
53.75	0.00946847529658792\\
53.76	0.00946726898725859\\
53.77	0.00946606181857161\\
53.78	0.00946485378954247\\
53.79	0.00946364489918488\\
53.8	0.00946243514651081\\
53.81	0.00946122453053043\\
53.82	0.00946001305025214\\
53.83	0.00945880070468259\\
53.84	0.00945758749282661\\
53.85	0.00945637341368727\\
53.86	0.00945515846626582\\
53.87	0.00945394264956174\\
53.88	0.0094527259625727\\
53.89	0.00945150840429454\\
53.9	0.00945028997372135\\
53.91	0.00944907066984535\\
53.92	0.00944785049165695\\
53.93	0.00944662943814477\\
53.94	0.00944540750829556\\
53.95	0.00944418470109427\\
53.96	0.00944296101552398\\
53.97	0.00944173645056597\\
53.98	0.00944051100519965\\
53.99	0.00943928467840257\\
54	0.00943805746915045\\
54.01	0.00943682937641714\\
54.02	0.00943560039917462\\
54.03	0.009434370536393\\
54.04	0.00943313978704051\\
54.05	0.00943190815008355\\
54.06	0.00943067562448657\\
54.07	0.00942944220921218\\
54.08	0.00942820790322107\\
54.09	0.00942697270547206\\
54.1	0.00942573661492203\\
54.11	0.009424499630526\\
54.12	0.00942326175123704\\
54.13	0.00942202297600633\\
54.14	0.00942078330378312\\
54.15	0.00941954273351472\\
54.16	0.00941830126414654\\
54.17	0.00941705889462203\\
54.18	0.00941581562388272\\
54.19	0.00941457145086818\\
54.2	0.00941332637451603\\
54.21	0.00941208039376196\\
54.22	0.00941083350753965\\
54.23	0.00940958571478087\\
54.24	0.0094083370144154\\
54.25	0.00940708740537104\\
54.26	0.00940583688657361\\
54.27	0.00940458545694696\\
54.28	0.00940333311541295\\
54.29	0.00940207986089142\\
54.3	0.00940082569230024\\
54.31	0.00939957060855528\\
54.32	0.00939831460857037\\
54.33	0.00939705769125735\\
54.34	0.00939579985552603\\
54.35	0.0093945411002842\\
54.36	0.00939328142443764\\
54.37	0.00939202082689006\\
54.38	0.00939075930654314\\
54.39	0.00938949686229654\\
54.4	0.00938823349304785\\
54.41	0.0093869691976926\\
54.42	0.00938570397512427\\
54.43	0.00938443782423429\\
54.44	0.00938317074391197\\
54.45	0.0093819027330446\\
54.46	0.00938063379051736\\
54.47	0.00937936391521334\\
54.48	0.00937809310601357\\
54.49	0.00937682136179693\\
54.5	0.00937554868144025\\
54.51	0.00937427506381822\\
54.52	0.00937300050780344\\
54.53	0.00937172501226635\\
54.54	0.00937044857607533\\
54.55	0.00936917119809657\\
54.56	0.00936789287719417\\
54.57	0.00936661361223005\\
54.58	0.00936533340206403\\
54.59	0.00936405224555375\\
54.6	0.00936277014155468\\
54.61	0.00936148708892017\\
54.62	0.00936020308650137\\
54.63	0.00935891813314726\\
54.64	0.00935763222770466\\
54.65	0.0093563453690182\\
54.66	0.00935505755593029\\
54.67	0.00935376878728119\\
54.68	0.00935247906190893\\
54.69	0.00935118837864934\\
54.7	0.00934989673633605\\
54.71	0.00934860413380045\\
54.72	0.00934731056987172\\
54.73	0.00934601604337682\\
54.74	0.00934472055314046\\
54.75	0.0093434240979851\\
54.76	0.00934212667673098\\
54.77	0.00934082828819607\\
54.78	0.0093395289311961\\
54.79	0.0093382286045445\\
54.8	0.00933692730705247\\
54.81	0.00933562503752891\\
54.82	0.00933432179478045\\
54.83	0.00933301757761144\\
54.84	0.0093317123848239\\
54.85	0.00933040621521759\\
54.86	0.00932909906758996\\
54.87	0.00932779094073614\\
54.88	0.00932648183344892\\
54.89	0.00932517174451882\\
54.9	0.00932386067273399\\
54.91	0.00932254861688025\\
54.92	0.0093212355757411\\
54.93	0.00931992154809766\\
54.94	0.00931860653272874\\
54.95	0.00931729052841076\\
54.96	0.00931597353391776\\
54.97	0.00931465554802144\\
54.98	0.00931333656949113\\
54.99	0.00931201659709373\\
55	0.0093106956295938\\
55.01	0.00930937366575347\\
55.02	0.00930805070433247\\
55.03	0.00930672674408814\\
55.04	0.0093054017837754\\
55.05	0.00930407582214674\\
55.06	0.00930274885795221\\
55.07	0.00930142088993946\\
55.08	0.00930009191685367\\
55.09	0.00929876193743759\\
55.1	0.0092974309504315\\
55.11	0.00929609895457325\\
55.12	0.00929476594859819\\
55.13	0.00929343193123922\\
55.14	0.00929209690122674\\
55.15	0.0092907608572887\\
55.16	0.00928942379815052\\
55.17	0.00928808572253514\\
55.18	0.009286746629163\\
55.19	0.00928540651675201\\
55.2	0.00928406538401758\\
55.21	0.00928272322967259\\
55.22	0.00928138005242738\\
55.23	0.00928003585098976\\
55.24	0.00927869062406498\\
55.25	0.00927734437035578\\
55.26	0.00927599708856229\\
55.27	0.0092746487773821\\
55.28	0.00927329943551024\\
55.29	0.00927194906163914\\
55.3	0.00927059765445865\\
55.31	0.00926924521265603\\
55.32	0.00926789173491595\\
55.33	0.00926653721992045\\
55.34	0.00926518166634898\\
55.35	0.00926382507287836\\
55.36	0.0092624674381828\\
55.37	0.00926110876093385\\
55.38	0.00925974903980043\\
55.39	0.00925838827344882\\
55.4	0.00925702646054263\\
55.41	0.00925566359974283\\
55.42	0.0092542996897077\\
55.43	0.00925293472909286\\
55.44	0.00925156871655123\\
55.45	0.00925020165073306\\
55.46	0.00924883353028589\\
55.47	0.00924746435385457\\
55.48	0.0092460941200812\\
55.49	0.00924472282760521\\
55.5	0.00924335047506328\\
55.51	0.00924197706108935\\
55.52	0.00924060258431462\\
55.53	0.00923922704336758\\
55.54	0.00923785043687391\\
55.55	0.00923647276345656\\
55.56	0.0092350940217357\\
55.57	0.00923371421032872\\
55.58	0.00923233332785025\\
55.59	0.00923095137291208\\
55.6	0.00922956834412324\\
55.61	0.00922818424008995\\
55.62	0.00922679905941559\\
55.63	0.00922541280070074\\
55.64	0.00922402546254314\\
55.65	0.0092226370435377\\
55.66	0.00922124754227647\\
55.67	0.00921985695734868\\
55.68	0.00921846528734065\\
55.69	0.00921707253083588\\
55.7	0.00921567868641496\\
55.71	0.00921428375265561\\
55.72	0.00921288772813267\\
55.73	0.00921149061141806\\
55.74	0.00921009240108079\\
55.75	0.00920869309568699\\
55.76	0.00920729269379983\\
55.77	0.00920589119397956\\
55.78	0.00920448859478351\\
55.79	0.00920308489476603\\
55.8	0.00920168009247855\\
55.81	0.00920027418646952\\
55.82	0.00919886717528441\\
55.83	0.00919745905746573\\
55.84	0.009196049831553\\
55.85	0.00919463949608273\\
55.86	0.00919322804958844\\
55.87	0.00919181549060065\\
55.88	0.00919040181764685\\
55.89	0.00918898702925148\\
55.9	0.00918757112393598\\
55.91	0.00918615410021873\\
55.92	0.00918473595661506\\
55.93	0.00918331669163725\\
55.94	0.00918189630379448\\
55.95	0.00918047479159289\\
55.96	0.00917905215353552\\
55.97	0.00917762838812231\\
55.98	0.0091762034938501\\
55.99	0.00917477746921262\\
56	0.00917335031270049\\
56.01	0.00917192202280119\\
56.02	0.00917049259799907\\
56.03	0.00916906203677533\\
56.04	0.00916763033760803\\
56.05	0.00916619749897204\\
56.06	0.0091647635193391\\
56.07	0.00916332839717775\\
56.08	0.00916189213095331\\
56.09	0.00916045471912796\\
56.1	0.00915901616016064\\
56.11	0.00915757645250708\\
56.12	0.0091561355946198\\
56.13	0.00915469358494807\\
56.14	0.00915325042193792\\
56.15	0.00915180610403215\\
56.16	0.00915036062967028\\
56.17	0.00914891399728857\\
56.18	0.00914746620532002\\
56.19	0.0091460172521943\\
56.2	0.00914456713633784\\
56.21	0.00914311585617373\\
56.22	0.00914166341012175\\
56.23	0.00914020979659837\\
56.24	0.00913875501401672\\
56.25	0.00913729906078658\\
56.26	0.00913584193531443\\
56.27	0.00913438363600331\\
56.28	0.00913292416125296\\
56.29	0.00913146350945972\\
56.3	0.00913000167901652\\
56.31	0.00912853866831294\\
56.32	0.0091270744757351\\
56.33	0.00912560909966575\\
56.34	0.0091241425384842\\
56.35	0.00912267479056632\\
56.36	0.00912120585428454\\
56.37	0.00911973572800784\\
56.38	0.00911826441010173\\
56.39	0.00911679189892826\\
56.4	0.00911531819284597\\
56.41	0.00911384329020995\\
56.42	0.00911236718937176\\
56.43	0.00911088988867946\\
56.44	0.00910941138647757\\
56.45	0.00910793168110711\\
56.46	0.00910645077090552\\
56.47	0.00910496865420673\\
56.48	0.00910348532934109\\
56.49	0.00910200079463538\\
56.5	0.00910051504841278\\
56.51	0.00909902808899293\\
56.52	0.00909753991469183\\
56.53	0.00909605052382187\\
56.54	0.00909455991469184\\
56.55	0.00909306808560687\\
56.56	0.00909157503486849\\
56.57	0.00909008076077453\\
56.58	0.00908858526161921\\
56.59	0.00908708853569303\\
56.6	0.00908559058128284\\
56.61	0.00908409139667179\\
56.62	0.00908259098013932\\
56.63	0.00908108932996116\\
56.64	0.00907958644440933\\
56.65	0.00907808232175209\\
56.66	0.00907657696025398\\
56.67	0.00907507035817577\\
56.68	0.00907356251377447\\
56.69	0.00907205342530332\\
56.7	0.00907054309101176\\
56.71	0.00906903150914544\\
56.72	0.00906751867794619\\
56.73	0.00906600459565205\\
56.74	0.00906448926049719\\
56.75	0.00906297267071199\\
56.76	0.00906145482452292\\
56.77	0.00905993572015264\\
56.78	0.00905841535581989\\
56.79	0.00905689372973956\\
56.8	0.00905537084012265\\
56.81	0.00905384668517622\\
56.82	0.00905232126310343\\
56.83	0.00905079457210352\\
56.84	0.00904926661037177\\
56.85	0.00904773737609953\\
56.86	0.00904620686747419\\
56.87	0.00904467508267913\\
56.88	0.0090431420198938\\
56.89	0.0090416076772936\\
56.9	0.00904007205304996\\
56.91	0.00903853514533029\\
56.92	0.00903699695229794\\
56.93	0.00903545747211224\\
56.94	0.00903391670292848\\
56.95	0.00903237464289785\\
56.96	0.00903083129016751\\
56.97	0.00902928664288048\\
56.98	0.00902774069917571\\
56.99	0.00902619345718805\\
57	0.0090246449150482\\
57.01	0.00902309507088274\\
57.02	0.00902154392281409\\
57.03	0.00901999146896053\\
57.04	0.00901843770743617\\
57.05	0.00901688263635092\\
57.06	0.0090153262538105\\
57.07	0.00901376855791644\\
57.08	0.00901220954676603\\
57.09	0.00901064921845233\\
57.1	0.00900908757106419\\
57.11	0.00900752460268617\\
57.12	0.00900596031139857\\
57.13	0.00900439469527742\\
57.14	0.00900282775239444\\
57.15	0.00900125948081708\\
57.16	0.00899968987860845\\
57.17	0.00899811894382732\\
57.18	0.00899654667452814\\
57.19	0.00899497306876101\\
57.2	0.00899339812457163\\
57.21	0.00899182184000136\\
57.22	0.00899024421308714\\
57.23	0.00898866524186154\\
57.24	0.00898708492435266\\
57.25	0.00898550325858422\\
57.26	0.00898392024257548\\
57.27	0.00898233587434124\\
57.28	0.00898075015189182\\
57.29	0.00897916307323309\\
57.3	0.0089775746363664\\
57.31	0.00897598483928862\\
57.32	0.00897439367999205\\
57.33	0.00897280115646452\\
57.34	0.00897120726668925\\
57.35	0.00896961200864496\\
57.36	0.00896801538030575\\
57.37	0.00896641737964116\\
57.38	0.00896481800461612\\
57.39	0.00896321725319096\\
57.4	0.00896161512332136\\
57.41	0.00896001161295837\\
57.42	0.0089584067200484\\
57.43	0.00895680044253318\\
57.44	0.00895519277834976\\
57.45	0.0089535837254305\\
57.46	0.00895197328170305\\
57.47	0.00895036144509034\\
57.48	0.00894874821351057\\
57.49	0.00894713358487716\\
57.5	0.00894551755709882\\
57.51	0.00894390012807943\\
57.52	0.00894228129571812\\
57.53	0.00894066105790919\\
57.54	0.00893903941254212\\
57.55	0.00893741635750158\\
57.56	0.00893579189066737\\
57.57	0.00893416600991444\\
57.58	0.00893253871311286\\
57.59	0.0089309099981278\\
57.6	0.00892927986281956\\
57.61	0.00892764830504346\\
57.62	0.00892601532264996\\
57.63	0.00892438091348452\\
57.64	0.00892274507538765\\
57.65	0.00892110780619489\\
57.66	0.00891946910373679\\
57.67	0.00891782896583888\\
57.68	0.00891618739032168\\
57.69	0.00891454437500066\\
57.7	0.00891289991768627\\
57.71	0.00891125401618388\\
57.72	0.00890960666829375\\
57.73	0.00890795787181108\\
57.74	0.00890630762452595\\
57.75	0.00890465592422332\\
57.76	0.00890300276868298\\
57.77	0.00890134815567961\\
57.78	0.00889969208298268\\
57.79	0.00889803454835649\\
57.8	0.00889637554956014\\
57.81	0.00889471508434751\\
57.82	0.00889305315046724\\
57.83	0.00889138974566272\\
57.84	0.00888972486767211\\
57.85	0.00888805851422823\\
57.86	0.00888639068305866\\
57.87	0.00888472137188565\\
57.88	0.00888305057842611\\
57.89	0.00888137830039163\\
57.9	0.00887970453548842\\
57.91	0.00887802928141733\\
57.92	0.00887635253587381\\
57.93	0.00887467429654792\\
57.94	0.00887299456112428\\
57.95	0.00887131332728208\\
57.96	0.00886963059269505\\
57.97	0.00886794635503146\\
57.98	0.00886626061195407\\
57.99	0.00886457336112016\\
58	0.00886288460018149\\
58.01	0.00886119432678425\\
58.02	0.00885950253856912\\
58.03	0.00885780923317118\\
58.04	0.00885611440821995\\
58.05	0.0088544180613393\\
58.06	0.00885272019014754\\
58.07	0.0088510207922573\\
58.08	0.00884931986527557\\
58.09	0.00884761740680367\\
58.1	0.00884591341443722\\
58.11	0.00884420788576617\\
58.12	0.00884250081837471\\
58.13	0.0088407922098413\\
58.14	0.00883908205773866\\
58.15	0.00883737035963371\\
58.16	0.00883565711308761\\
58.17	0.00883394231565568\\
58.18	0.00883222596488745\\
58.19	0.00883050805832656\\
58.2	0.00882878859351083\\
58.21	0.00882706756797219\\
58.22	0.00882534497923665\\
58.23	0.00882362082482435\\
58.24	0.00882189510224945\\
58.25	0.00882016780902019\\
58.26	0.00881843894263885\\
58.27	0.00881670850060168\\
58.28	0.00881497648039898\\
58.29	0.00881324287951499\\
58.3	0.00881150769542792\\
58.31	0.00880977092560993\\
58.32	0.00880803256748838\\
58.33	0.00880629261848303\\
58.34	0.00880455107600795\\
58.35	0.00880280793747143\\
58.36	0.00880106320027607\\
58.37	0.00879931686181868\\
58.38	0.00879756891949031\\
58.39	0.00879581937067621\\
58.4	0.00879406821275585\\
58.41	0.00879231544310285\\
58.42	0.00879056105908503\\
58.43	0.00878880505806433\\
58.44	0.00878704743739684\\
58.45	0.00878528819443278\\
58.46	0.00878352732651645\\
58.47	0.00878176483098627\\
58.48	0.0087800007051747\\
58.49	0.00877823494640829\\
58.5	0.0087764675520076\\
58.51	0.00877469851928725\\
58.52	0.00877292784555584\\
58.53	0.008771155528116\\
58.54	0.0087693815642643\\
58.55	0.0087676059512913\\
58.56	0.00876582868648151\\
58.57	0.00876404976711335\\
58.58	0.00876226919045917\\
58.59	0.00876048695378522\\
58.6	0.00875870305435163\\
58.61	0.0087569174894124\\
58.62	0.00875513025621537\\
58.63	0.00875334135200224\\
58.64	0.00875155077400851\\
58.65	0.00874975851946347\\
58.66	0.00874796458559022\\
58.67	0.00874616896960561\\
58.68	0.00874437166872027\\
58.69	0.00874257268013854\\
58.7	0.00874077200105849\\
58.71	0.00873896962867189\\
58.72	0.0087371655601642\\
58.73	0.00873535979271454\\
58.74	0.00873355232349569\\
58.75	0.00873174314967408\\
58.76	0.00872993226840973\\
58.77	0.0087281196768563\\
58.78	0.00872630537216098\\
58.79	0.00872448935146459\\
58.8	0.00872267161190145\\
58.81	0.00872085215059945\\
58.82	0.00871903096467997\\
58.83	0.00871720805125791\\
58.84	0.00871538340744163\\
58.85	0.00871355703033299\\
58.86	0.00871172891702725\\
58.87	0.00870989906461315\\
58.88	0.00870806747017279\\
58.89	0.00870623413078172\\
58.9	0.00870439904350881\\
58.91	0.00870256220541635\\
58.92	0.0087007236135599\\
58.93	0.00869888326498842\\
58.94	0.00869704115674413\\
58.95	0.00869519728586253\\
58.96	0.00869335164937242\\
58.97	0.00869150424429585\\
58.98	0.00868965506764808\\
58.99	0.00868780411643761\\
59	0.00868595138766612\\
59.01	0.00868409687832849\\
59.02	0.00868224058541274\\
59.03	0.00868038250590004\\
59.04	0.00867852263676469\\
59.05	0.0086766609749741\\
59.06	0.00867479751748876\\
59.07	0.00867293226126221\\
59.08	0.00867106520324109\\
59.09	0.00866919634036503\\
59.1	0.00866732566956669\\
59.11	0.00866545318777171\\
59.12	0.00866357889189872\\
59.13	0.00866170277885931\\
59.14	0.00865982484555799\\
59.15	0.00865794508889221\\
59.16	0.0086560635057523\\
59.17	0.00865418009302149\\
59.18	0.00865229484757585\\
59.19	0.00865040776628431\\
59.2	0.00864851884600861\\
59.21	0.00864662808360331\\
59.22	0.00864473547591573\\
59.23	0.00864284101978598\\
59.24	0.00864094471204689\\
59.25	0.00863904654952404\\
59.26	0.0086371465290357\\
59.27	0.00863524464739283\\
59.28	0.00863334090139906\\
59.29	0.00863143528785066\\
59.3	0.00862952780353653\\
59.31	0.00862761844523818\\
59.32	0.0086257072097297\\
59.33	0.00862379409377775\\
59.34	0.00862187909414154\\
59.35	0.0086199622075728\\
59.36	0.00861804343081576\\
59.37	0.00861612276060716\\
59.38	0.00861420019367618\\
59.39	0.00861227572674446\\
59.4	0.00861034935652605\\
59.41	0.00860842107972741\\
59.42	0.0086064908930474\\
59.43	0.00860455879317721\\
59.44	0.00860262477680041\\
59.45	0.00860068884059286\\
59.46	0.00859875098122272\\
59.47	0.00859681119535046\\
59.48	0.00859486947962876\\
59.49	0.00859292583070259\\
59.5	0.0085909802452091\\
59.51	0.00858903271977763\\
59.52	0.00858708325102973\\
59.53	0.00858513183557907\\
59.54	0.00858317847003145\\
59.55	0.0085812231509848\\
59.56	0.00857926587502912\\
59.57	0.00857730663874648\\
59.58	0.00857534543871101\\
59.59	0.00857338227148883\\
59.6	0.00857141713363808\\
59.61	0.0085694500217089\\
59.62	0.00856748093224333\\
59.63	0.00856550986177541\\
59.64	0.00856353680683104\\
59.65	0.00856156176392805\\
59.66	0.0085595847295761\\
59.67	0.00855760570027673\\
59.68	0.0085556246725233\\
59.69	0.00855364164280094\\
59.7	0.00855165660758658\\
59.71	0.00854966956334892\\
59.72	0.00854768050654839\\
59.73	0.00854568943363709\\
59.74	0.00854369634105886\\
59.75	0.00854170122524918\\
59.76	0.00853970408263517\\
59.77	0.00853770490963558\\
59.78	0.00853570370266075\\
59.79	0.0085337004581126\\
59.8	0.00853169517238459\\
59.81	0.00852968784186172\\
59.82	0.00852767846292047\\
59.83	0.00852566703192883\\
59.84	0.00852365354524621\\
59.85	0.00852163799922349\\
59.86	0.00851962039020295\\
59.87	0.00851760071451823\\
59.88	0.00851557896849436\\
59.89	0.0085135551484477\\
59.9	0.00851152925068591\\
59.91	0.00850950127150796\\
59.92	0.00850747120720409\\
59.93	0.00850543905405574\\
59.94	0.00850340480833562\\
59.95	0.0085013684663076\\
59.96	0.00849933002422674\\
59.97	0.00849728947833921\\
59.98	0.00849524682488235\\
59.99	0.00849320206008454\\
60	0.00849115518016527\\
60.01	0.00848910618133506\\
60.02	0.00848705505979546\\
60.03	0.00848500181173901\\
60.04	0.00848294643334921\\
60.05	0.00848088892080053\\
60.06	0.00847882927025834\\
60.07	0.00847676747787891\\
60.08	0.00847470353980938\\
60.09	0.00847263745218775\\
60.1	0.00847056921114281\\
60.11	0.00846849881279418\\
60.12	0.0084664262532522\\
60.13	0.008464351528618\\
60.14	0.0084622746349834\\
60.15	0.00846019556843092\\
60.16	0.00845811432503373\\
60.17	0.00845603090085566\\
60.18	0.00845394529195113\\
60.19	0.00845185749436518\\
60.2	0.00844976750413337\\
60.21	0.00844767531728182\\
60.22	0.00844558092982714\\
60.23	0.00844348433777644\\
60.24	0.00844138553712726\\
60.25	0.00843928452386759\\
60.26	0.00843718129397581\\
60.27	0.00843507584342067\\
60.28	0.00843296816816127\\
60.29	0.00843085826414703\\
60.3	0.00842874612731766\\
60.31	0.00842663175360315\\
60.32	0.0084245151389237\\
60.33	0.00842239627918974\\
60.34	0.00842027517030188\\
60.35	0.00841815180815088\\
60.36	0.00841602618861764\\
60.37	0.00841389830757314\\
60.38	0.00841176816087845\\
60.39	0.00840963574438468\\
60.4	0.00840750105393294\\
60.41	0.00840536408535435\\
60.42	0.00840322483446999\\
60.43	0.00840108329709084\\
60.44	0.00839893946901783\\
60.45	0.00839679334604173\\
60.46	0.00839464492394318\\
60.47	0.0083924941984926\\
60.48	0.00839034116545025\\
60.49	0.00838818582056611\\
60.5	0.0083860281595799\\
60.51	0.00838386817822106\\
60.52	0.00838170587220867\\
60.53	0.0083795412372515\\
60.54	0.00837737426904789\\
60.55	0.00837520496328579\\
60.56	0.00837303331564269\\
60.57	0.00837085932178563\\
60.58	0.00836868297737112\\
60.59	0.00836650427804515\\
60.6	0.00836432321944315\\
60.61	0.00836213979718995\\
60.62	0.00835995400689975\\
60.63	0.00835776584417611\\
60.64	0.00835557530461191\\
60.65	0.00835338238378929\\
60.66	0.00835118707727968\\
60.67	0.00834898938064372\\
60.68	0.00834678928943123\\
60.69	0.00834458679918121\\
60.7	0.00834238190542179\\
60.71	0.00834017460367021\\
60.72	0.00833796488943277\\
60.73	0.0083357527582048\\
60.74	0.00833353820547067\\
60.75	0.00833132122670369\\
60.76	0.00832910181736615\\
60.77	0.00832687997290923\\
60.78	0.00832465568877301\\
60.79	0.0083224289603864\\
60.8	0.00832019978316715\\
60.81	0.00831796815252179\\
60.82	0.0083157340638456\\
60.83	0.00831349751252259\\
60.84	0.00831125849392546\\
60.85	0.00830901700341555\\
60.86	0.00830677303634287\\
60.87	0.00830452658804598\\
60.88	0.00830227765385202\\
60.89	0.00830002622907665\\
60.9	0.00829777230902404\\
60.91	0.0082955158889868\\
60.92	0.00829325696424599\\
60.93	0.00829099553007105\\
60.94	0.0082887315817198\\
60.95	0.00828646511443837\\
60.96	0.00828419612346119\\
60.97	0.00828192460401097\\
60.98	0.00827965055129863\\
60.99	0.00827737396052328\\
61	0.00827509482687221\\
61.01	0.00827281314552082\\
61.02	0.00827052891163261\\
61.03	0.00826824212035913\\
61.04	0.00826595276683996\\
61.05	0.00826366084620268\\
61.06	0.00826136635356281\\
61.07	0.00825906928402378\\
61.08	0.00825676963267692\\
61.09	0.00825446739460141\\
61.1	0.00825216256486424\\
61.11	0.00824985513852019\\
61.12	0.00824754511061176\\
61.13	0.00824523247616918\\
61.14	0.00824291723021034\\
61.15	0.00824059936774078\\
61.16	0.00823827888375365\\
61.17	0.00823595577322962\\
61.18	0.00823363003113695\\
61.19	0.00823130165243135\\
61.2	0.00822897063205601\\
61.21	0.00822663696494153\\
61.22	0.0082243006460059\\
61.23	0.00822196167015446\\
61.24	0.00821962003227985\\
61.25	0.00821727572726199\\
61.26	0.00821492874996805\\
61.27	0.00821257909525239\\
61.28	0.00821022675795652\\
61.29	0.00820787173290909\\
61.3	0.00820551401492585\\
61.31	0.00820315359880958\\
61.32	0.00820079047935009\\
61.33	0.00819842465132413\\
61.34	0.00819605610949544\\
61.35	0.00819368484861462\\
61.36	0.00819131086341915\\
61.37	0.00818893414863332\\
61.38	0.00818655469896821\\
61.39	0.00818417250912166\\
61.4	0.00818178757377818\\
61.41	0.00817939988760899\\
61.42	0.00817700944527191\\
61.43	0.00817461624141136\\
61.44	0.00817222027065831\\
61.45	0.00816982152763024\\
61.46	0.00816742000693112\\
61.47	0.00816501570315132\\
61.48	0.00816260861086761\\
61.49	0.00816019872464315\\
61.5	0.00815778603902736\\
61.51	0.00815537054855596\\
61.52	0.00815295224775091\\
61.53	0.00815053113112034\\
61.54	0.00814810719315855\\
61.55	0.00814568042834594\\
61.56	0.00814325083114897\\
61.57	0.00814081839602015\\
61.58	0.00813838311739798\\
61.59	0.00813594498970687\\
61.6	0.00813350400735718\\
61.61	0.0081310601647451\\
61.62	0.00812861345625267\\
61.63	0.00812616387624768\\
61.64	0.00812371141908369\\
61.65	0.00812125607909993\\
61.66	0.00811879785062128\\
61.67	0.00811633672795828\\
61.68	0.00811387270540698\\
61.69	0.00811140577724899\\
61.7	0.00810893593775139\\
61.71	0.00810646318116671\\
61.72	0.00810398750173287\\
61.73	0.00810150889367315\\
61.74	0.00809902735119615\\
61.75	0.00809654286849571\\
61.76	0.00809405543975092\\
61.77	0.00809156505912604\\
61.78	0.00808907172077047\\
61.79	0.00808657541881871\\
61.8	0.00808407614739028\\
61.81	0.00808157484985784\\
61.82	0.00807907206291085\\
61.83	0.0080765677852878\\
61.84	0.00807406201572598\\
61.85	0.00807155475296149\\
61.86	0.00806904599572924\\
61.87	0.00806653574276295\\
61.88	0.00806402399279513\\
61.89	0.00806151074455713\\
61.9	0.00805899599677908\\
61.91	0.00805647974818992\\
61.92	0.00805396199751742\\
61.93	0.00805144274348812\\
61.94	0.00804892198482739\\
61.95	0.00804639972025939\\
61.96	0.00804387594850711\\
61.97	0.00804135066829231\\
61.98	0.00803882387833559\\
61.99	0.00803629557735632\\
62	0.0080337657640727\\
62.01	0.00803123443720172\\
62.02	0.00802870159545917\\
62.03	0.00802616723755966\\
62.04	0.00802363136221658\\
62.05	0.00802109396814215\\
62.06	0.00801855505404737\\
62.07	0.00801601461864203\\
62.08	0.00801347266063477\\
62.09	0.00801092917873297\\
62.1	0.00800838417164287\\
62.11	0.00800583763806945\\
62.12	0.00800328957671655\\
62.13	0.00800073998628677\\
62.14	0.00799818886548153\\
62.15	0.00799563621300103\\
62.16	0.0079930820275443\\
62.17	0.00799052630780914\\
62.18	0.00798796905249215\\
62.19	0.00798541026028877\\
62.2	0.00798284992989317\\
62.21	0.0079802880599984\\
62.22	0.00797772464929622\\
62.23	0.00797515969647727\\
62.24	0.00797259320023094\\
62.25	0.00797002515924542\\
62.26	0.00796745557220772\\
62.27	0.00796488443780362\\
62.28	0.00796231175471773\\
62.29	0.00795973752163343\\
62.3	0.00795716173723291\\
62.31	0.00795458440019715\\
62.32	0.00795200550920592\\
62.33	0.00794942506293782\\
62.34	0.00794684306007021\\
62.35	0.00794425949927926\\
62.36	0.00794167437923994\\
62.37	0.00793908769862601\\
62.38	0.00793649945611004\\
62.39	0.00793390965036337\\
62.4	0.00793131828005616\\
62.41	0.00792872534385736\\
62.42	0.0079261308404347\\
62.43	0.00792353476845474\\
62.44	0.0079209371265828\\
62.45	0.00791833791348301\\
62.46	0.00791573712781832\\
62.47	0.00791313476825042\\
62.48	0.00791053083343984\\
62.49	0.00790792532204591\\
62.5	0.00790531823272672\\
62.51	0.00790270956413918\\
62.52	0.007900099314939\\
62.53	0.00789748748378066\\
62.54	0.00789487406931746\\
62.55	0.0078922590702015\\
62.56	0.00788964248508365\\
62.57	0.0078870243126136\\
62.58	0.00788440455143981\\
62.59	0.00788178320020957\\
62.6	0.00787916025756893\\
62.61	0.00787653572216276\\
62.62	0.00787390959263473\\
62.63	0.00787128186762728\\
62.64	0.00786865254578167\\
62.65	0.00786602162573795\\
62.66	0.00786338910613495\\
62.67	0.00786075498561033\\
62.68	0.00785811926280053\\
62.69	0.00785548193634076\\
62.7	0.00785284300486507\\
62.71	0.00785020246700629\\
62.72	0.00784756032139604\\
62.73	0.00784491656666474\\
62.74	0.00784227120144162\\
62.75	0.00783962422435469\\
62.76	0.00783697563403077\\
62.77	0.00783432542909548\\
62.78	0.00783167360817322\\
62.79	0.00782902016988721\\
62.8	0.00782636511285946\\
62.81	0.00782370843571078\\
62.82	0.00782105013706078\\
62.83	0.00781839021552785\\
62.84	0.00781572866972922\\
62.85	0.00781306549828087\\
62.86	0.00781040069979764\\
62.87	0.00780773427289311\\
62.88	0.00780506621617969\\
62.89	0.0078023965282686\\
62.9	0.00779972520776985\\
62.91	0.00779705225329224\\
62.92	0.00779437766344339\\
62.93	0.00779170143682972\\
62.94	0.00778902357205644\\
62.95	0.00778634406772757\\
62.96	0.00778366292244594\\
62.97	0.00778098013481317\\
62.98	0.0077782957034297\\
62.99	0.00777560962689475\\
63	0.00777292190380638\\
63.01	0.00777023253276142\\
63.02	0.00776754151235552\\
63.03	0.00776484884118315\\
63.04	0.00776215451783755\\
63.05	0.00775945854091081\\
63.06	0.0077567609089938\\
63.07	0.00775406162067619\\
63.08	0.00775136067454649\\
63.09	0.00774865806919198\\
63.1	0.00774595380319878\\
63.11	0.00774324787515182\\
63.12	0.00774054028363481\\
63.13	0.00773783102723029\\
63.14	0.0077351201045196\\
63.15	0.00773240751408293\\
63.16	0.00772969325449924\\
63.17	0.00772697732434629\\
63.18	0.00772425972220071\\
63.19	0.0077215404466379\\
63.2	0.00771881949623208\\
63.21	0.00771609686955629\\
63.22	0.00771337256518239\\
63.23	0.00771064658168105\\
63.24	0.00770791891762176\\
63.25	0.00770518957157282\\
63.26	0.00770245854210136\\
63.27	0.00769972582777331\\
63.28	0.00769699142715344\\
63.29	0.00769425533880533\\
63.3	0.00769151756129137\\
63.31	0.0076887780931728\\
63.32	0.00768603693300964\\
63.33	0.00768329407936077\\
63.34	0.00768054953078388\\
63.35	0.00767780328583546\\
63.36	0.00767505534307088\\
63.37	0.00767230570104428\\
63.38	0.00766955435830865\\
63.39	0.00766680131341582\\
63.4	0.00766404656491643\\
63.41	0.00766129011135993\\
63.42	0.00765853195129464\\
63.43	0.00765577208326769\\
63.44	0.00765301050582504\\
63.45	0.00765024721751149\\
63.46	0.00764748221687065\\
63.47	0.007644715502445\\
63.48	0.00764194707277581\\
63.49	0.00763917692640323\\
63.5	0.00763640506186622\\
63.51	0.00763363147770259\\
63.52	0.00763085617244896\\
63.53	0.00762807914464083\\
63.54	0.0076253003928125\\
63.55	0.00762251991549716\\
63.56	0.00761973771122679\\
63.57	0.00761695377853223\\
63.58	0.00761416811594318\\
63.59	0.00761138072198818\\
63.6	0.00760859159519459\\
63.61	0.00760580073408865\\
63.62	0.00760300813719544\\
63.63	0.00760021380303886\\
63.64	0.0075974177301417\\
63.65	0.00759461991702558\\
63.66	0.00759182036221098\\
63.67	0.00758901906421722\\
63.68	0.0075862160215625\\
63.69	0.00758341123276384\\
63.7	0.00758060469633714\\
63.71	0.00757779641079717\\
63.72	0.00757498637465753\\
63.73	0.00757217458643069\\
63.74	0.007569361044628\\
63.75	0.00756654574775964\\
63.76	0.00756372869433468\\
63.77	0.00756090988286104\\
63.78	0.00755808931184553\\
63.79	0.0075552669797938\\
63.8	0.00755244288521037\\
63.81	0.00754961702659866\\
63.82	0.00754678940246092\\
63.83	0.00754396001129832\\
63.84	0.00754112885161085\\
63.85	0.00753829592189743\\
63.86	0.0075354612206558\\
63.87	0.00753262474638265\\
63.88	0.00752978649757348\\
63.89	0.00752694647272272\\
63.9	0.00752410467032365\\
63.91	0.00752126108886845\\
63.92	0.00751841572684819\\
63.93	0.00751556858275282\\
63.94	0.00751271965507117\\
63.95	0.00750986894229098\\
63.96	0.00750701644289885\\
63.97	0.00750416215538033\\
63.98	0.0075013060782198\\
63.99	0.00749844820990057\\
64	0.00749558854890485\\
64.01	0.00749272709371373\\
64.02	0.00748986384280723\\
64.03	0.00748699879466425\\
64.04	0.0074841319477626\\
64.05	0.00748126330057899\\
64.06	0.00747839285158905\\
64.07	0.00747552059926732\\
64.08	0.00747264654208724\\
64.09	0.00746977067852117\\
64.1	0.00746689300704038\\
64.11	0.00746401352611505\\
64.12	0.00746113223421431\\
64.13	0.00745824912980617\\
64.14	0.00745536421135759\\
64.15	0.00745247747733443\\
64.16	0.00744958892620151\\
64.17	0.00744669855642253\\
64.18	0.00744380636646017\\
64.19	0.00744091235477601\\
64.2	0.00743801651983056\\
64.21	0.00743511886008329\\
64.22	0.00743221937399258\\
64.23	0.00742931806001577\\
64.24	0.00742641491660913\\
64.25	0.00742350994222789\\
64.26	0.00742060313532618\\
64.27	0.00741769449435713\\
64.28	0.0074147840177728\\
64.29	0.00741187170402419\\
64.3	0.00740895755156127\\
64.31	0.00740604155883295\\
64.32	0.00740312372428711\\
64.33	0.00740020404637059\\
64.34	0.0073972825235292\\
64.35	0.00739435915420767\\
64.36	0.00739143393684976\\
64.37	0.00738850686989815\\
64.38	0.00738557795179452\\
64.39	0.00738264718097951\\
64.4	0.00737971455589274\\
64.41	0.0073767800749728\\
64.42	0.00737384373665728\\
64.43	0.00737090553938273\\
64.44	0.0073679654815847\\
64.45	0.00736502356169772\\
64.46	0.00736207977815532\\
64.47	0.00735913412939001\\
64.48	0.0073561866138333\\
64.49	0.00735323722991571\\
64.5	0.00735028597606673\\
64.51	0.00734733285071488\\
64.52	0.00734437785228768\\
64.53	0.00734142097921166\\
64.54	0.00733846222991233\\
64.55	0.00733550160281427\\
64.56	0.00733253909634102\\
64.57	0.00732957470891516\\
64.58	0.0073266084389583\\
64.59	0.00732364028489107\\
64.6	0.00732067024513312\\
64.61	0.00731769831810313\\
64.62	0.00731472450221881\\
64.63	0.00731174879589692\\
64.64	0.00730877119755323\\
64.65	0.00730579170560259\\
64.66	0.00730281031845885\\
64.67	0.00729982703453494\\
64.68	0.00729684185224281\\
64.69	0.0072938547699935\\
64.7	0.00729086578619707\\
64.71	0.00728787489926266\\
64.72	0.00728488210759846\\
64.73	0.00728188740961173\\
64.74	0.00727889080370878\\
64.75	0.00727589228829502\\
64.76	0.00727289186177493\\
64.77	0.00726988952255204\\
64.78	0.00726688526902897\\
64.79	0.00726387909960745\\
64.8	0.00726087101268825\\
64.81	0.00725786100667129\\
64.82	0.00725484907995552\\
64.83	0.00725183523093903\\
64.84	0.00724881945801899\\
64.85	0.00724580175959167\\
64.86	0.00724278213405247\\
64.87	0.00723976057979587\\
64.88	0.00723673709521548\\
64.89	0.00723371167870403\\
64.9	0.00723068432865335\\
64.91	0.00722765504345443\\
64.92	0.00722462382149735\\
64.93	0.00722159066117135\\
64.94	0.00721855556086477\\
64.95	0.00721551851896512\\
64.96	0.00721247953385905\\
64.97	0.00720943860393234\\
64.98	0.00720639572756992\\
64.99	0.00720335090315588\\
65	0.00720030412907347\\
65.01	0.0071972554037051\\
65.02	0.00719420472543233\\
65.03	0.00719115209263592\\
65.04	0.00718809750369575\\
65.05	0.00718504095699095\\
65.06	0.00718198245089975\\
65.07	0.00717892198379963\\
65.08	0.00717585955406721\\
65.09	0.00717279516007834\\
65.1	0.00716972880020804\\
65.11	0.00716666047283056\\
65.12	0.0071635901763193\\
65.13	0.00716051790904694\\
65.14	0.00715744366938531\\
65.15	0.0071543674557055\\
65.16	0.00715128926637781\\
65.17	0.00714820909977175\\
65.18	0.00714512695425608\\
65.19	0.00714204282819879\\
65.2	0.00713895671996711\\
65.21	0.0071358686279275\\
65.22	0.00713277855044569\\
65.23	0.00712968648588664\\
65.24	0.0071265924326146\\
65.25	0.00712349638899304\\
65.26	0.00712039835338472\\
65.27	0.00711729832415168\\
65.28	0.00711419629965522\\
65.29	0.00711109227825592\\
65.3	0.00710798625831366\\
65.31	0.00710487823818759\\
65.32	0.00710176821623618\\
65.33	0.00709865619081718\\
65.34	0.00709554216028765\\
65.35	0.00709242612300395\\
65.36	0.00708930807732179\\
65.37	0.00708618802159615\\
65.38	0.00708306595418138\\
65.39	0.00707994187343113\\
65.4	0.0070768157776984\\
65.41	0.00707368766533552\\
65.42	0.00707055753469419\\
65.43	0.00706742538412543\\
65.44	0.00706429121197962\\
65.45	0.00706115501660653\\
65.46	0.00705801679635526\\
65.47	0.00705487654957432\\
65.48	0.00705173427461156\\
65.49	0.00704858996981424\\
65.5	0.007045443633529\\
65.51	0.00704229526410187\\
65.52	0.0070391448598783\\
65.53	0.00703599241920311\\
65.54	0.00703283794042058\\
65.55	0.00702968142187436\\
65.56	0.00702652286190754\\
65.57	0.00702336225886266\\
65.58	0.00702019961108167\\
65.59	0.00701703491690597\\
65.6	0.0070138681746764\\
65.61	0.00701069938273327\\
65.62	0.00700752853941632\\
65.63	0.00700435564306479\\
65.64	0.00700118069201736\\
65.65	0.00699800368461222\\
65.66	0.006994824619187\\
65.67	0.00699164349407887\\
65.68	0.00698846030762446\\
65.69	0.00698527505815992\\
65.7	0.0069820877440209\\
65.71	0.00697889836354257\\
65.72	0.00697570691505962\\
65.73	0.00697251339690628\\
65.74	0.00696931780741632\\
65.75	0.00696612014492301\\
65.76	0.00696292040775922\\
65.77	0.00695971859425735\\
65.78	0.00695651470274936\\
65.79	0.0069533087315668\\
65.8	0.00695010067904079\\
65.81	0.00694689054350201\\
65.82	0.00694367832328076\\
65.83	0.00694046401670693\\
65.84	0.00693724762210999\\
65.85	0.00693402913781908\\
65.86	0.0069308085621629\\
65.87	0.0069275858934698\\
65.88	0.00692436113006779\\
65.89	0.00692113427028446\\
65.9	0.00691790531244712\\
65.91	0.00691467425488268\\
65.92	0.00691144109591775\\
65.93	0.0069082058338786\\
65.94	0.00690496846709117\\
65.95	0.00690172899388112\\
65.96	0.00689848741257376\\
65.97	0.00689524372149414\\
65.98	0.00689199791896699\\
65.99	0.00688875000331681\\
66	0.00688549997286777\\
66.01	0.0068822478259438\\
66.02	0.00687899356086859\\
66.03	0.00687573717596555\\
66.04	0.00687247866955789\\
66.05	0.00686921803996854\\
66.06	0.00686595528552026\\
66.07	0.00686269040453555\\
66.08	0.00685942339533673\\
66.09	0.00685615425624593\\
66.1	0.00685288298558506\\
66.11	0.00684960958167589\\
66.12	0.00684633404283998\\
66.13	0.00684305636739876\\
66.14	0.0068397765536735\\
66.15	0.00683649459998532\\
66.16	0.0068332105046552\\
66.17	0.00682992426600402\\
66.18	0.00682663588235252\\
66.19	0.00682334535202134\\
66.2	0.00682005267333102\\
66.21	0.00681675784460203\\
66.22	0.00681346086415474\\
66.23	0.00681016173030947\\
66.24	0.00680686044138645\\
66.25	0.00680355699570591\\
66.26	0.006800251391588\\
66.27	0.00679694362735287\\
66.28	0.00679363370132063\\
66.29	0.00679032161181137\\
66.3	0.00678700735714523\\
66.31	0.00678369093564231\\
66.32	0.00678037234562276\\
66.33	0.00677705158540675\\
66.34	0.0067737286533145\\
66.35	0.0067704035476663\\
66.36	0.00676707626678246\\
66.37	0.00676374680898341\\
66.38	0.00676041517258963\\
66.39	0.00675708135592172\\
66.4	0.00675374535730038\\
66.41	0.00675040717504643\\
66.42	0.00674706680748081\\
66.43	0.00674372425292463\\
66.44	0.00674037950969911\\
66.45	0.00673703257612566\\
66.46	0.00673368345052588\\
66.47	0.00673033213122151\\
66.48	0.00672697861653453\\
66.49	0.00672362290478712\\
66.5	0.00672026499430167\\
66.51	0.00671690488340083\\
66.52	0.00671354257040746\\
66.53	0.0067101780536447\\
66.54	0.00670681133143597\\
66.55	0.00670344240210497\\
66.56	0.00670007126397566\\
66.57	0.00669669791537236\\
66.58	0.00669332235461967\\
66.59	0.00668994458004256\\
66.6	0.0066865645899663\\
66.61	0.00668318238271657\\
66.62	0.00667979795661939\\
66.63	0.00667641131000116\\
66.64	0.00667302244118872\\
66.65	0.00666963134850927\\
66.66	0.00666623803029048\\
66.67	0.00666284248486043\\
66.68	0.00665944471054766\\
66.69	0.00665604470568119\\
66.7	0.00665264246859051\\
66.71	0.0066492379976056\\
66.72	0.00664583129105695\\
66.73	0.00664242234727559\\
66.74	0.00663901116459307\\
66.75	0.00663559774134151\\
66.76	0.00663218207585357\\
66.77	0.0066287641664625\\
66.78	0.00662534401150218\\
66.79	0.00662192160930704\\
66.8	0.0066184969582122\\
66.81	0.00661507005655338\\
66.82	0.00661164090266697\\
66.83	0.00660820949489003\\
66.84	0.00660477583156031\\
66.85	0.00660133991101627\\
66.86	0.00659790173159708\\
66.87	0.00659446129164264\\
66.88	0.00659101858949362\\
66.89	0.00658757362349146\\
66.9	0.00658412639197835\\
66.91	0.00658067689329733\\
66.92	0.00657722512579221\\
66.93	0.00657377108780767\\
66.94	0.00657031477768922\\
66.95	0.00656685619378325\\
66.96	0.00656339533443705\\
66.97	0.00655993219799877\\
66.98	0.00655646678281751\\
66.99	0.00655299908724331\\
67	0.00654952910962717\\
67.01	0.00654605684832105\\
67.02	0.00654258230167789\\
67.03	0.00653910546805168\\
67.04	0.00653562634579741\\
67.05	0.00653214493327112\\
67.06	0.00652866122882992\\
67.07	0.006525175230832\\
67.08	0.00652168693763667\\
67.09	0.00651819634760435\\
67.1	0.00651470345909661\\
67.11	0.00651120827047617\\
67.12	0.00650771078010694\\
67.13	0.00650421098635403\\
67.14	0.00650070888758378\\
67.15	0.00649720448216376\\
67.16	0.00649369776846283\\
67.17	0.00649018874485108\\
67.18	0.00648667740969994\\
67.19	0.00648316376138218\\
67.2	0.0064796477982719\\
67.21	0.00647612951874454\\
67.22	0.00647260892117696\\
67.23	0.00646908600394743\\
67.24	0.00646556076543565\\
67.25	0.00646203320402275\\
67.26	0.00645850331809138\\
67.27	0.00645497110602566\\
67.28	0.00645143656621123\\
67.29	0.00644789969703527\\
67.3	0.00644436049688657\\
67.31	0.00644081896415546\\
67.32	0.00643727509723392\\
67.33	0.00643372889451554\\
67.34	0.00643018035439559\\
67.35	0.00642662947527101\\
67.36	0.00642307625554048\\
67.37	0.00641952069360439\\
67.38	0.00641596278786489\\
67.39	0.00641240253672592\\
67.4	0.00640883993859324\\
67.41	0.00640527499187442\\
67.42	0.00640170769497892\\
67.43	0.00639813804631806\\
67.44	0.0063945660443051\\
67.45	0.0063909916873552\\
67.46	0.00638741497388553\\
67.47	0.00638383590231523\\
67.48	0.00638025447106545\\
67.49	0.00637667067855941\\
67.5	0.00637308452322239\\
67.51	0.00636949600348176\\
67.52	0.00636590511776706\\
67.53	0.00636231186450995\\
67.54	0.00635871624214429\\
67.55	0.00635511824910616\\
67.56	0.00635151788383387\\
67.57	0.00634791514476804\\
67.58	0.00634431003035155\\
67.59	0.00634070253902964\\
67.6	0.00633709266924992\\
67.61	0.00633348041946238\\
67.62	0.00632986578811943\\
67.63	0.00632624877367595\\
67.64	0.00632262937458931\\
67.65	0.00631900758931938\\
67.66	0.00631538341632861\\
67.67	0.00631175685408202\\
67.68	0.00630812790104724\\
67.69	0.00630449655569455\\
67.7	0.00630086281649694\\
67.71	0.00629722668193006\\
67.72	0.00629358815047237\\
67.73	0.00628994722060508\\
67.74	0.00628630389081222\\
67.75	0.00628265815958069\\
67.76	0.00627901002540026\\
67.77	0.00627535948676363\\
67.78	0.00627170654216646\\
67.79	0.00626805119010741\\
67.8	0.00626439342908817\\
67.81	0.00626073325761349\\
67.82	0.00625707067419124\\
67.83	0.00625340567733242\\
67.84	0.00624973826555122\\
67.85	0.00624606843736506\\
67.86	0.0062423961912946\\
67.87	0.0062387215258638\\
67.88	0.00623504443959997\\
67.89	0.00623136493103378\\
67.9	0.00622768299869933\\
67.91	0.00622399864113417\\
67.92	0.00622031185687936\\
67.93	0.00621662264447949\\
67.94	0.00621293100248274\\
67.95	0.00620923692944089\\
67.96	0.00620554042390942\\
67.97	0.00620184148444751\\
67.98	0.00619814010961807\\
67.99	0.00619443629798783\\
68	0.00619073004812737\\
68.01	0.00618702135861113\\
68.02	0.0061833102280175\\
68.03	0.00617959665492885\\
68.04	0.00617588063793155\\
68.05	0.00617216217561606\\
68.06	0.00616844126657697\\
68.07	0.006164717909413\\
68.08	0.00616099210272711\\
68.09	0.0061572638451265\\
68.1	0.00615353313522272\\
68.11	0.00614979997163163\\
68.12	0.00614606435297353\\
68.13	0.00614232627787318\\
68.14	0.00613858574495984\\
68.15	0.00613484275286734\\
68.16	0.00613109730023412\\
68.17	0.00612734938570331\\
68.18	0.00612359900792274\\
68.19	0.00611984616554502\\
68.2	0.00611609085722758\\
68.21	0.00611233308163276\\
68.22	0.00610857283742781\\
68.23	0.006104810123285\\
68.24	0.00610104493788165\\
68.25	0.00609727727990017\\
68.26	0.00609350714802816\\
68.27	0.00608973454095842\\
68.28	0.00608595945738908\\
68.29	0.00608218189602357\\
68.3	0.00607840185557073\\
68.31	0.00607461933474491\\
68.32	0.00607083433226594\\
68.33	0.00606704684685925\\
68.34	0.00606325687725594\\
68.35	0.0060594644221928\\
68.36	0.00605566948041242\\
68.37	0.00605187205066325\\
68.38	0.0060480721316996\\
68.39	0.00604426972228181\\
68.4	0.00604046482117625\\
68.41	0.00603665742715537\\
68.42	0.00603284753899783\\
68.43	0.00602903515548855\\
68.44	0.00602522027541874\\
68.45	0.00602140289758603\\
68.46	0.00601758302079448\\
68.47	0.00601376064385471\\
68.48	0.00600993576558394\\
68.49	0.00600610838480606\\
68.5	0.00600227850035173\\
68.51	0.00599844611105844\\
68.52	0.00599461121577056\\
68.53	0.00599077381333949\\
68.54	0.00598693390262365\\
68.55	0.00598309148248863\\
68.56	0.00597924655180724\\
68.57	0.00597539910945955\\
68.58	0.00597154915433307\\
68.59	0.00596769668532274\\
68.6	0.00596384170133105\\
68.61	0.00595998420126813\\
68.62	0.00595612418405184\\
68.63	0.00595226164860782\\
68.64	0.0059483965938696\\
68.65	0.00594452901877871\\
68.66	0.00594065892228473\\
68.67	0.00593678630334542\\
68.68	0.00593291116092673\\
68.69	0.00592903349400304\\
68.7	0.00592515330155708\\
68.71	0.00592127058258015\\
68.72	0.00591738533607217\\
68.73	0.00591349756104176\\
68.74	0.00590960725650639\\
68.75	0.0059057144214924\\
68.76	0.00590181905503518\\
68.77	0.00589792115617921\\
68.78	0.00589402072397822\\
68.79	0.0058901177574952\\
68.8	0.00588621225580262\\
68.81	0.00588230421798246\\
68.82	0.00587839364312633\\
68.83	0.00587448053033558\\
68.84	0.00587056487872142\\
68.85	0.00586664668740502\\
68.86	0.00586272595551762\\
68.87	0.00585880268220065\\
68.88	0.00585487686660584\\
68.89	0.00585094850789534\\
68.9	0.00584701760524182\\
68.91	0.00584308415782859\\
68.92	0.00583914816484979\\
68.93	0.00583520962551036\\
68.94	0.00583126853902634\\
68.95	0.00582732490462485\\
68.96	0.0058233787215443\\
68.97	0.00581942998903446\\
68.98	0.00581547870635666\\
68.99	0.00581152487278383\\
69	0.00580756848760071\\
69.01	0.00580360955010393\\
69.02	0.00579964805948012\\
69.03	0.00579568401403238\\
69.04	0.00579171741206535\\
69.05	0.00578774825188526\\
69.06	0.00578377653179986\\
69.07	0.00577980225011854\\
69.08	0.0057758254051523\\
69.09	0.00577184599521378\\
69.1	0.00576786401861729\\
69.11	0.0057638794736788\\
69.12	0.00575989235871601\\
69.13	0.00575590267204832\\
69.14	0.0057519104119969\\
69.15	0.00574791557688468\\
69.16	0.00574391816503638\\
69.17	0.00573991817477853\\
69.18	0.00573591560443951\\
69.19	0.00573191045234954\\
69.2	0.00572790271684074\\
69.21	0.00572389239624712\\
69.22	0.00571987948890463\\
69.23	0.00571586399315115\\
69.24	0.00571184590732657\\
69.25	0.00570782522977273\\
69.26	0.00570380195883353\\
69.27	0.0056997760928549\\
69.28	0.00569574763018485\\
69.29	0.00569171656917347\\
69.3	0.00568768290817298\\
69.31	0.00568364664553775\\
69.32	0.00567960777962429\\
69.33	0.00567556630879134\\
69.34	0.00567152223139984\\
69.35	0.00566747554581297\\
69.36	0.00566342625039621\\
69.37	0.00565937434351729\\
69.38	0.0056553198235463\\
69.39	0.00565126268885568\\
69.4	0.00564720293782021\\
69.41	0.00564314056881713\\
69.42	0.00563907570087051\\
69.43	0.00563500833763439\\
69.44	0.00563093847603353\\
69.45	0.00562686611298631\\
69.46	0.00562279124540478\\
69.47	0.00561871387019461\\
69.48	0.00561463398425509\\
69.49	0.00561055158447911\\
69.5	0.00560646666775315\\
69.51	0.0056023792309573\\
69.52	0.00559828927096517\\
69.53	0.00559419678464398\\
69.54	0.00559010176885448\\
69.55	0.00558600422045094\\
69.56	0.00558190413628117\\
69.57	0.00557780151318651\\
69.58	0.00557369634800177\\
69.59	0.00556958863755528\\
69.6	0.00556547837866883\\
69.61	0.0055613655681577\\
69.62	0.00555725020283062\\
69.63	0.00555313227948976\\
69.64	0.00554901179493074\\
69.65	0.0055448887459426\\
69.66	0.00554076312930781\\
69.67	0.00553663494180224\\
69.68	0.00553250418019514\\
69.69	0.00552837084124918\\
69.7	0.00552423492172037\\
69.71	0.00552009641835812\\
69.72	0.00551595532790517\\
69.73	0.00551181164709762\\
69.74	0.00550766537266491\\
69.75	0.0055035165013298\\
69.76	0.00549936502980836\\
69.77	0.00549521095480999\\
69.78	0.00549105427303738\\
69.79	0.00548689498118652\\
69.8	0.00548273307594665\\
69.81	0.00547856855400032\\
69.82	0.00547440141202332\\
69.83	0.00547023164668472\\
69.84	0.00546605925464681\\
69.85	0.00546188423256514\\
69.86	0.00545770657708847\\
69.87	0.00545352628485882\\
69.88	0.00544934335251137\\
69.89	0.00544515777667457\\
69.9	0.00544096955397002\\
69.91	0.00543677868101254\\
69.92	0.00543258515441012\\
69.93	0.00542838897076393\\
69.94	0.00542419012666834\\
69.95	0.00541998861871082\\
69.96	0.00541578444347208\\
69.97	0.00541157759752592\\
69.98	0.00540736807743931\\
69.99	0.00540315587977235\\
70	0.00539894100107829\\
70.01	0.00539472343790348\\
70.02	0.00539050318678742\\
70.03	0.00538628024426271\\
70.04	0.00538205460685507\\
70.05	0.00537782627108332\\
70.06	0.0053735952334594\\
70.07	0.00536936149048832\\
70.08	0.0053651250386682\\
70.09	0.00536088587449026\\
70.1	0.00535664399443878\\
70.11	0.00535239939499114\\
70.12	0.0053481520726178\\
70.13	0.00534390202378228\\
70.14	0.00533964924494119\\
70.15	0.00533539373254419\\
70.16	0.00533113548303403\\
70.17	0.00532687449284651\\
70.18	0.00532261075841051\\
70.19	0.00531834427614795\\
70.2	0.00531407504247383\\
70.21	0.00530980305379619\\
70.22	0.00530552830651615\\
70.23	0.00530125079702786\\
70.24	0.00529697052171856\\
70.25	0.00529268747696852\\
70.26	0.0052884016591511\\
70.27	0.00528411306463267\\
70.28	0.00527982168977271\\
70.29	0.00527552753092373\\
70.3	0.00527123058443131\\
70.31	0.0052669308466341\\
70.32	0.0052626283138638\\
70.33	0.0052583229824452\\
70.34	0.00525401484869614\\
70.35	0.00524970390892756\\
70.36	0.00524539015944345\\
70.37	0.0052410735965409\\
70.38	0.00523675421651009\\
70.39	0.00523243201563427\\
70.4	0.0052281069901898\\
70.41	0.00522377913644615\\
70.42	0.00521944845066587\\
70.43	0.00521511492910466\\
70.44	0.00521077856801131\\
70.45	0.00520643936362774\\
70.46	0.00520209731218902\\
70.47	0.00519775240992334\\
70.48	0.00519340465305207\\
70.49	0.00518905403778971\\
70.5	0.00518470056034393\\
70.51	0.0051803442169156\\
70.52	0.00517598500369875\\
70.53	0.00517162291688062\\
70.54	0.00516725795264167\\
70.55	0.00516289010715556\\
70.56	0.00515851937658918\\
70.57	0.0051541457571027\\
70.58	0.00514976924484949\\
70.59	0.00514538983597625\\
70.6	0.00514100752662291\\
70.61	0.00513662231292274\\
70.62	0.00513223419100231\\
70.63	0.0051278431569815\\
70.64	0.00512344920697357\\
70.65	0.00511905233708512\\
70.66	0.00511465254341613\\
70.67	0.00511024982205997\\
70.68	0.00510584416910346\\
70.69	0.0051014355806268\\
70.7	0.0050970240527037\\
70.71	0.00509260958140131\\
70.72	0.00508819216278027\\
70.73	0.00508377179289476\\
70.74	0.00507934846779247\\
70.75	0.0050749221835147\\
70.76	0.00507049293609628\\
70.77	0.00506606072156568\\
70.78	0.00506162553594498\\
70.79	0.00505718737524995\\
70.8	0.00505274623549003\\
70.81	0.00504830211266836\\
70.82	0.00504385500278185\\
70.83	0.00503940490182115\\
70.84	0.00503495180577072\\
70.85	0.00503049571060884\\
70.86	0.00502603661230767\\
70.87	0.00502157450683323\\
70.88	0.0050171093901455\\
70.89	0.00501264125819837\\
70.9	0.00500817010693976\\
70.91	0.00500369593231159\\
70.92	0.00499921873024986\\
70.93	0.00499473849668464\\
70.94	0.00499025522754017\\
70.95	0.00498576891873483\\
70.96	0.00498127956618124\\
70.97	0.00497678716578625\\
70.98	0.00497229171345102\\
70.99	0.00496779320507105\\
71	0.00496329163653619\\
71.01	0.00495878700373076\\
71.02	0.0049542793025335\\
71.03	0.0049497685288177\\
71.04	0.00494525467845121\\
71.05	0.00494073774729647\\
71.06	0.0049362177312106\\
71.07	0.00493169462604542\\
71.08	0.00492716842764751\\
71.09	0.00492263913185829\\
71.1	0.00491810673451401\\
71.11	0.00491357123144588\\
71.12	0.00490903261848006\\
71.13	0.00490449089143779\\
71.14	0.00489994604613535\\
71.15	0.00489539807838421\\
71.16	0.00489084698399106\\
71.17	0.00488629275875786\\
71.18	0.00488173539848191\\
71.19	0.00487717489895593\\
71.2	0.00487261125596811\\
71.21	0.00486804446530217\\
71.22	0.00486347452273746\\
71.23	0.004858901424049\\
71.24	0.00485432516500757\\
71.25	0.00484974574137978\\
71.26	0.00484516314892812\\
71.27	0.00484057738341108\\
71.28	0.00483598844058321\\
71.29	0.00483139631619518\\
71.3	0.00482680100599388\\
71.31	0.00482220250572249\\
71.32	0.00481760081112059\\
71.33	0.00481299591792421\\
71.34	0.00480838782186597\\
71.35	0.00480377651867507\\
71.36	0.00479916200407751\\
71.37	0.00479454427379608\\
71.38	0.0047899233235505\\
71.39	0.00478529914905751\\
71.4	0.00478067174603098\\
71.41	0.00477604111018195\\
71.42	0.00477140723721882\\
71.43	0.00476677012284739\\
71.44	0.00476212976277098\\
71.45	0.00475748615269057\\
71.46	0.00475283928830483\\
71.47	0.00474818916531033\\
71.48	0.0047435357794016\\
71.49	0.00473887912627121\\
71.5	0.00473421920160997\\
71.51	0.00472955600110699\\
71.52	0.00472488952044982\\
71.53	0.00472021975532458\\
71.54	0.00471554670141608\\
71.55	0.00471087035440792\\
71.56	0.00470619070998268\\
71.57	0.00470150776382202\\
71.58	0.00469682151160679\\
71.59	0.00469213194901722\\
71.6	0.00468743907173303\\
71.61	0.00468274287543357\\
71.62	0.00467804335579797\\
71.63	0.00467341792590884\\
71.64	0.00467128668778365\\
71.65	0.00466915529691223\\
71.66	0.00466702375563885\\
71.67	0.00466489206631688\\
71.68	0.00466276023130885\\
71.69	0.00466062825298646\\
71.7	0.00465849613373064\\
71.71	0.00465636387593153\\
71.72	0.00465423148198858\\
71.73	0.00465209895431056\\
71.74	0.00464996629531557\\
71.75	0.00464783350743109\\
71.76	0.00464570059309402\\
71.77	0.0046435675547507\\
71.78	0.00464143439485698\\
71.79	0.00463930111587822\\
71.8	0.00463716772028931\\
71.81	0.00463503421057475\\
71.82	0.00463290058922867\\
71.83	0.00463076685875483\\
71.84	0.0046286330216667\\
71.85	0.00462649908048751\\
71.86	0.00462436503775019\\
71.87	0.00462223089599753\\
71.88	0.00462009665778211\\
71.89	0.00461796232566642\\
71.9	0.00461582790222284\\
71.91	0.0046136933900337\\
71.92	0.0046115587916913\\
71.93	0.00460942410979798\\
71.94	0.00460728934696612\\
71.95	0.00460515450581818\\
71.96	0.00460301958898678\\
71.97	0.00460088459911469\\
71.98	0.00459874953885488\\
71.99	0.00459661441087056\\
72	0.00459447921783524\\
72.01	0.00459234396243271\\
72.02	0.00459020864735715\\
72.03	0.00458807327531313\\
72.04	0.00458593784901561\\
72.05	0.0045838023711901\\
72.06	0.00458166684457253\\
72.07	0.00457953127190945\\
72.08	0.00457739565595794\\
72.09	0.00457525999948576\\
72.1	0.0045731243052713\\
72.11	0.00457098857610366\\
72.12	0.00456885281478271\\
72.13	0.00456671702411907\\
72.14	0.00456458120693422\\
72.15	0.00456244536606047\\
72.16	0.00456030950434106\\
72.17	0.00455817362463018\\
72.18	0.00455603772979301\\
72.19	0.00455390182270572\\
72.2	0.0045517659062556\\
72.21	0.00454962998334104\\
72.22	0.00454749405687155\\
72.23	0.00454535812976788\\
72.24	0.00454322220496199\\
72.25	0.00454108628539712\\
72.26	0.00453895037402785\\
72.27	0.0045368144738201\\
72.28	0.00453467858775122\\
72.29	0.00453254271881\\
72.3	0.00453040686999672\\
72.31	0.00452827104432318\\
72.32	0.00452613524481279\\
72.33	0.00452399947450059\\
72.34	0.00452186373643324\\
72.35	0.00451972803366915\\
72.36	0.00451759236927849\\
72.37	0.0045154567463432\\
72.38	0.00451332116795708\\
72.39	0.00451118563722584\\
72.4	0.00450905015726709\\
72.41	0.00450691473121043\\
72.42	0.00450477936219753\\
72.43	0.00450264405338206\\
72.44	0.00450050880792986\\
72.45	0.00449837362901892\\
72.46	0.00449623851983943\\
72.47	0.00449410348359385\\
72.48	0.00449196852349695\\
72.49	0.00448983364277582\\
72.5	0.00448769884466999\\
72.51	0.00448556413243142\\
72.52	0.00448342950932453\\
72.53	0.00448129497862634\\
72.54	0.00447916054362643\\
72.55	0.00447702620762701\\
72.56	0.004474891973943\\
72.57	0.00447275784590204\\
72.58	0.00447062382684455\\
72.59	0.00446848992012382\\
72.6	0.00446635612910598\\
72.61	0.00446422245717012\\
72.62	0.0044620889077083\\
72.63	0.00445995548412564\\
72.64	0.00445782218984031\\
72.65	0.00445568902828364\\
72.66	0.00445355600290015\\
72.67	0.00445142311714759\\
72.68	0.004449290374497\\
72.69	0.00444715777843277\\
72.7	0.00444502533245266\\
72.71	0.00444289304006792\\
72.72	0.00444076090480326\\
72.73	0.00443862893019696\\
72.74	0.0044364971198009\\
72.75	0.00443436547718064\\
72.76	0.00443223400591542\\
72.77	0.00443010270959826\\
72.78	0.00442797159183601\\
72.79	0.00442584065624939\\
72.8	0.00442370990647304\\
72.81	0.00442157934615561\\
72.82	0.00441944897895976\\
72.83	0.00441731880856228\\
72.84	0.00441518883865408\\
72.85	0.00441305907294031\\
72.86	0.00441092951514035\\
72.87	0.00440880016898793\\
72.88	0.00440667103823116\\
72.89	0.00440454212663257\\
72.9	0.00440241343796919\\
72.91	0.00440028497603263\\
72.92	0.00439815674462904\\
72.93	0.00439602874757933\\
72.94	0.00439390098871906\\
72.95	0.00439177347189864\\
72.96	0.00438964620098327\\
72.97	0.00438751917985312\\
72.98	0.00438539241240327\\
72.99	0.00438326590254385\\
73	0.0043811396542001\\
73.01	0.00437901367131237\\
73.02	0.00437688795783625\\
73.03	0.00437476251774259\\
73.04	0.00437263735501759\\
73.05	0.00437051247366284\\
73.06	0.00436838787769538\\
73.07	0.0043662635711478\\
73.08	0.00436413955806823\\
73.09	0.00436201584252052\\
73.1	0.00435989242858418\\
73.11	0.00435776932035452\\
73.12	0.0043556465219427\\
73.13	0.00435352403747579\\
73.14	0.00435140187109682\\
73.15	0.00434928002696488\\
73.16	0.00434715850925517\\
73.17	0.00434503732215906\\
73.18	0.00434291646988416\\
73.19	0.00434079595665438\\
73.2	0.00433867578671005\\
73.21	0.0043365559643079\\
73.22	0.0043344364937212\\
73.23	0.0043323173792398\\
73.24	0.00433019862517021\\
73.25	0.00432808023583565\\
73.26	0.00432596221557614\\
73.27	0.00432384456874859\\
73.28	0.0043217272997268\\
73.29	0.00431961041290162\\
73.3	0.00431749391268095\\
73.31	0.00431537780348987\\
73.32	0.00431326208977066\\
73.33	0.00431114677598291\\
73.34	0.00430903186660359\\
73.35	0.0043069173661271\\
73.36	0.00430480327906538\\
73.37	0.00430268960994795\\
73.38	0.004300576363322\\
73.39	0.00429846354375249\\
73.4	0.00429635115582218\\
73.41	0.00429423920413174\\
73.42	0.00429212769329981\\
73.43	0.00429001662796311\\
73.44	0.00428790601277647\\
73.45	0.00428579585241294\\
73.46	0.00428368615156386\\
73.47	0.00428157691493896\\
73.48	0.00427946814726639\\
73.49	0.00427735985329286\\
73.5	0.00427525203778367\\
73.51	0.00427314470552285\\
73.52	0.00427103786131317\\
73.53	0.00426893150997629\\
73.54	0.00426682565635281\\
73.55	0.00426472030530233\\
73.56	0.0042626154617036\\
73.57	0.00426051113045454\\
73.58	0.00425840731647237\\
73.59	0.00425630402469367\\
73.6	0.00425420126007448\\
73.61	0.00425209902759037\\
73.62	0.00424999733223656\\
73.63	0.00424789617902797\\
73.64	0.00424579557299933\\
73.65	0.00424369551920528\\
73.66	0.00424159602272043\\
73.67	0.00423949708863947\\
73.68	0.00423739872207726\\
73.69	0.00423530092816894\\
73.7	0.00423320371206996\\
73.71	0.00423110707895625\\
73.72	0.00422901103402426\\
73.73	0.00422691558249108\\
73.74	0.00422482072959452\\
73.75	0.00422272648059324\\
73.76	0.00422063284076678\\
73.77	0.00421853981541572\\
73.78	0.00421644740986177\\
73.79	0.0042143556294478\\
73.8	0.00421226447953803\\
73.81	0.0042101739655181\\
73.82	0.00420808409279512\\
73.83	0.00420599486679785\\
73.84	0.00420390629297674\\
73.85	0.00420181837680406\\
73.86	0.004199731123774\\
73.87	0.00419764453940277\\
73.88	0.00419555862922871\\
73.89	0.0041934733988124\\
73.9	0.00419138885373673\\
73.91	0.00418930499960705\\
73.92	0.00418722184205127\\
73.93	0.00418513938671994\\
73.94	0.00418305763928641\\
73.95	0.00418097660544688\\
73.96	0.00417889629092055\\
73.97	0.00417681670144971\\
73.98	0.00417473784279988\\
73.99	0.0041726597207599\\
74	0.00417058234114195\\
74.01	0.00416850570978012\\
74.02	0.00416642983253049\\
74.03	0.00416435471527117\\
74.04	0.00416228036390246\\
74.05	0.00416020678434687\\
74.06	0.00415813398254925\\
74.07	0.00415606196447683\\
74.08	0.00415399073611938\\
74.09	0.0041519203034892\\
74.1	0.00414985067262129\\
74.11	0.00414778184957342\\
74.12	0.00414571384042615\\
74.13	0.00414364665128303\\
74.14	0.00414158028827061\\
74.15	0.00413951475753853\\
74.16	0.00413745006525966\\
74.17	0.00413538621763017\\
74.18	0.00413332322086957\\
74.19	0.00413126108122088\\
74.2	0.00412919980495067\\
74.21	0.00412713939834917\\
74.22	0.00412507986773036\\
74.23	0.00412302121943206\\
74.24	0.00412096345981602\\
74.25	0.00411890659526803\\
74.26	0.004116850632198\\
74.27	0.00411479557704006\\
74.28	0.00411274143625263\\
74.29	0.00411068821631858\\
74.3	0.00410863592374524\\
74.31	0.00410658456506459\\
74.32	0.00410453414683328\\
74.33	0.00410248467563273\\
74.34	0.0041004361580693\\
74.35	0.00409838860077434\\
74.36	0.00409634201040425\\
74.37	0.00409429639364065\\
74.38	0.0040922517571904\\
74.39	0.00409020810778572\\
74.4	0.00408816545218431\\
74.41	0.00408612379716928\\
74.42	0.00408408314954874\\
74.43	0.00408204351615586\\
74.44	0.004080004903849\\
74.45	0.00407796731951173\\
74.46	0.004075930770053\\
74.47	0.00407389526240719\\
74.48	0.00407186080353419\\
74.49	0.00406982740041954\\
74.5	0.00406779506007448\\
74.51	0.00406576378953605\\
74.52	0.0040637335958672\\
74.53	0.00406170448615688\\
74.54	0.00405967646752011\\
74.55	0.00405764954709812\\
74.56	0.00405562373205838\\
74.57	0.00405359902959477\\
74.58	0.00405157544692763\\
74.59	0.00404955299130384\\
74.6	0.00404753166999697\\
74.61	0.00404551149030732\\
74.62	0.00404349245956209\\
74.63	0.00404147458511538\\
74.64	0.00403945787434837\\
74.65	0.00403744233466937\\
74.66	0.00403542797351396\\
74.67	0.00403341479834504\\
74.68	0.00403140281665297\\
74.69	0.00402939203595564\\
74.7	0.0040273824637986\\
74.71	0.00402537410775514\\
74.72	0.00402336697542637\\
74.73	0.00402136107444139\\
74.74	0.00401935641245732\\
74.75	0.00401735299715942\\
74.76	0.00401535083626123\\
74.77	0.00401334993750462\\
74.78	0.00401135030865992\\
74.79	0.00400935195752604\\
74.8	0.00400735489193054\\
74.81	0.00400535911972974\\
74.82	0.00400336464880885\\
74.83	0.00400137148708205\\
74.84	0.00399937964249261\\
74.85	0.00399738912301296\\
74.86	0.00399539993664488\\
74.87	0.00399341209141951\\
74.88	0.0039914255953975\\
74.89	0.00398944045666915\\
74.9	0.00398745668335444\\
74.91	0.00398547428360322\\
74.92	0.00398349326559525\\
74.93	0.00398151363754038\\
74.94	0.00397953540767857\\
74.95	0.00397755858428008\\
74.96	0.00397558317564556\\
74.97	0.00397360919010613\\
74.98	0.00397163663602351\\
74.99	0.00396966552179014\\
75	0.0039676958558293\\
75.01	0.00396572764659518\\
75.02	0.00396376090257304\\
75.03	0.00396179563227929\\
75.04	0.00395983184426163\\
75.05	0.00395786954709915\\
75.06	0.00395590874940243\\
75.07	0.00395394945981369\\
75.08	0.00395199168700689\\
75.09	0.00395003543968782\\
75.1	0.00394808072659426\\
75.11	0.00394612755649605\\
75.12	0.00394417593819526\\
75.13	0.00394222588052626\\
75.14	0.00394027739235588\\
75.15	0.00393833048258348\\
75.16	0.00393638516014111\\
75.17	0.00393444143399362\\
75.18	0.00393249931313876\\
75.19	0.00393055880660734\\
75.2	0.00392861992346328\\
75.21	0.00392668267280385\\
75.22	0.00392474706375964\\
75.23	0.00392281310549481\\
75.24	0.00392088080720715\\
75.25	0.00391895017812823\\
75.26	0.00391702122752348\\
75.27	0.00391509396469238\\
75.28	0.00391316839896853\\
75.29	0.00391124453971979\\
75.3	0.00390932239634843\\
75.31	0.0039074019782912\\
75.32	0.00390548329501953\\
75.33	0.00390356635603961\\
75.34	0.00390165117089251\\
75.35	0.00389973774915433\\
75.36	0.00389782610043634\\
75.37	0.00389591623438506\\
75.38	0.00389400816068246\\
75.39	0.00389210188904601\\
75.4	0.00389019742922889\\
75.41	0.00388829479102005\\
75.42	0.00388639398424439\\
75.43	0.00388449501876288\\
75.44	0.00388259790447266\\
75.45	0.00388070265130724\\
75.46	0.00387880926923656\\
75.47	0.00387691776826717\\
75.48	0.00387502815844237\\
75.49	0.00387314044984229\\
75.5	0.00387125465258409\\
75.51	0.00386937077682206\\
75.52	0.00386748883274776\\
75.53	0.00386560883059018\\
75.54	0.00386373078061583\\
75.55	0.00386185469312894\\
75.56	0.00385998057847155\\
75.57	0.00385810844702364\\
75.58	0.00385623830920335\\
75.59	0.00385437017546702\\
75.6	0.00385250405630938\\
75.61	0.00385063996226371\\
75.62	0.00384877790390193\\
75.63	0.00384691789183481\\
75.64	0.00384505993671202\\
75.65	0.00384320404922239\\
75.66	0.00384135024009392\\
75.67	0.00383949852009406\\
75.68	0.00383764890002976\\
75.69	0.00383580139074765\\
75.7	0.00383395600313417\\
75.71	0.00383211274811575\\
75.72	0.00383027163665894\\
75.73	0.00382843267977053\\
75.74	0.00382659588849774\\
75.75	0.00382476127392835\\
75.76	0.00382292884719084\\
75.77	0.00382109861945459\\
75.78	0.00381927060192993\\
75.79	0.00381744480586841\\
75.8	0.00381562124256286\\
75.81	0.00381379992334762\\
75.82	0.00381198085959859\\
75.83	0.0038101640627335\\
75.84	0.00380834954421198\\
75.85	0.00380653731553576\\
75.86	0.00380472738824879\\
75.87	0.00380291977393745\\
75.88	0.00380111448423063\\
75.89	0.00379931153079996\\
75.9	0.00379751092535992\\
75.91	0.00379571267966803\\
75.92	0.00379391680552499\\
75.93	0.00379212331477484\\
75.94	0.00379033221930514\\
75.95	0.00378854099534318\\
75.96	0.0037867495040187\\
75.97	0.00378495774595762\\
75.98	0.00378316572178552\\
75.99	0.00378137343212763\\
76	0.00377958087760879\\
76.01	0.00377778805885348\\
76.02	0.00377599497648576\\
76.03	0.0037742016311293\\
76.04	0.00377240802340734\\
76.05	0.00377061415394268\\
76.06	0.00376882002335768\\
76.07	0.00376702563227425\\
76.08	0.00376523098131381\\
76.09	0.00376343607109731\\
76.1	0.0037616409022452\\
76.11	0.00375984547537741\\
76.12	0.00375804979111336\\
76.13	0.00375625385007193\\
76.14	0.00375445765287143\\
76.15	0.00375266120012966\\
76.16	0.00375086449246378\\
76.17	0.00374906753049038\\
76.18	0.00374727031482548\\
76.19	0.00374547284608445\\
76.2	0.00374367512488202\\
76.21	0.00374187715183231\\
76.22	0.00374007892754875\\
76.23	0.00373828045264411\\
76.24	0.00373648172773048\\
76.25	0.00373468275341924\\
76.26	0.00373288353032105\\
76.27	0.00373108405904584\\
76.28	0.00372928434020281\\
76.29	0.0037274843744004\\
76.3	0.00372568416224626\\
76.31	0.00372388370434726\\
76.32	0.00372208300130947\\
76.33	0.00372028205373814\\
76.34	0.00371848086223769\\
76.35	0.00371667942741169\\
76.36	0.00371487774986284\\
76.37	0.00371307583019296\\
76.38	0.00371127366900301\\
76.39	0.00370947126689298\\
76.4	0.003707668624462\\
76.41	0.0037058657423082\\
76.42	0.0037040626210288\\
76.43	0.00370225926122003\\
76.44	0.00370045566347711\\
76.45	0.0036986518283943\\
76.46	0.00369684775656481\\
76.47	0.00369504344858081\\
76.48	0.00369323890503343\\
76.49	0.00369143412651273\\
76.5	0.00368962911360768\\
76.51	0.00368782386690614\\
76.52	0.00368601838699485\\
76.53	0.00368421267445944\\
76.54	0.00368240672988436\\
76.55	0.00368060055385289\\
76.56	0.00367879414694713\\
76.57	0.00367698750974797\\
76.58	0.00367518064283509\\
76.59	0.00367337354678691\\
76.6	0.00367156622218061\\
76.61	0.00366975866959209\\
76.62	0.00366795088959595\\
76.63	0.00366614288276548\\
76.64	0.00366433464967264\\
76.65	0.00366252619088806\\
76.66	0.00366071750698099\\
76.67	0.00365890859851928\\
76.68	0.00365709946606941\\
76.69	0.00365529011019642\\
76.7	0.00365348053146391\\
76.71	0.00365167073043403\\
76.72	0.00364986070766743\\
76.73	0.00364805046372331\\
76.74	0.00364623999915931\\
76.75	0.00364442931453155\\
76.76	0.0036426184103946\\
76.77	0.00364080728730143\\
76.78	0.00363899594580347\\
76.79	0.0036371843864505\\
76.8	0.00363537260979066\\
76.81	0.00363356061637045\\
76.82	0.0036317484067347\\
76.83	0.00362993598142654\\
76.84	0.0036281233409874\\
76.85	0.00362631048595695\\
76.86	0.00362449741687313\\
76.87	0.00362268413427209\\
76.88	0.00362087063868819\\
76.89	0.00361905693065397\\
76.9	0.00361724301070013\\
76.91	0.00361542887935552\\
76.92	0.0036136145371471\\
76.93	0.00361179998459993\\
76.94	0.00360998522223715\\
76.95	0.00360817025057995\\
76.96	0.00360635507014754\\
76.97	0.00360453968145718\\
76.98	0.00360272408502408\\
76.99	0.00360090828136143\\
77	0.00359909227098037\\
77.01	0.00359727605438995\\
77.02	0.00359545963209713\\
77.03	0.00359364300460673\\
77.04	0.00359182617242146\\
77.05	0.0035900091360418\\
77.06	0.00358819189596611\\
77.07	0.00358637445269047\\
77.08	0.00358455680670875\\
77.09	0.00358273895851258\\
77.1	0.00358092090859125\\
77.11	0.00357910265743177\\
77.12	0.00357728420551884\\
77.13	0.00357546555333475\\
77.14	0.00357364670135944\\
77.15	0.00357182765007045\\
77.16	0.00357000839994284\\
77.17	0.00356818895144929\\
77.18	0.00356636930505993\\
77.19	0.00356454946124242\\
77.2	0.00356272942046188\\
77.21	0.00356090918318087\\
77.22	0.00355908874985939\\
77.23	0.0035572681209548\\
77.24	0.00355544729692185\\
77.25	0.00355362627821263\\
77.26	0.00355180506527653\\
77.27	0.00354998365856024\\
77.28	0.00354816205850773\\
77.29	0.00354634026556016\\
77.3	0.00354451828015597\\
77.31	0.00354269610273071\\
77.32	0.00354087373371713\\
77.33	0.00353905117354512\\
77.34	0.00353722842264162\\
77.35	0.0035354054814307\\
77.36	0.00353358235033346\\
77.37	0.003531759029768\\
77.38	0.00352993552014943\\
77.39	0.00352811182188984\\
77.4	0.00352628793539821\\
77.41	0.00352446386108048\\
77.42	0.00352263959933944\\
77.43	0.00352081515057473\\
77.44	0.00351899051518282\\
77.45	0.00351716569355698\\
77.46	0.00351534068608724\\
77.47	0.00351351549316035\\
77.48	0.00351169011515979\\
77.49	0.00350986455246568\\
77.5	0.00350803880545484\\
77.51	0.00350621287450066\\
77.52	0.00350438675997312\\
77.53	0.00350256046223878\\
77.54	0.00350073398166072\\
77.55	0.00349890731859849\\
77.56	0.00349708047340814\\
77.57	0.00349525344644213\\
77.58	0.00349342623804932\\
77.59	0.00349159884857496\\
77.6	0.00348977127836063\\
77.61	0.0034879435277442\\
77.62	0.00348611559705985\\
77.63	0.00348428748663798\\
77.64	0.00348245919680521\\
77.65	0.00348063072788433\\
77.66	0.00347880208019428\\
77.67	0.00347697325405014\\
77.68	0.00347514424976304\\
77.69	0.00347331506764016\\
77.7	0.00347148570798471\\
77.71	0.00346965617109588\\
77.72	0.0034678264572688\\
77.73	0.00346599656679452\\
77.74	0.00346416649995996\\
77.75	0.00346233625704791\\
77.76	0.00346050583833693\\
77.77	0.0034586752441014\\
77.78	0.00345684447461143\\
77.79	0.00345501353013282\\
77.8	0.00345318241092706\\
77.81	0.00345135111725126\\
77.82	0.00344951964935815\\
77.83	0.00344768800749603\\
77.84	0.0034458561919087\\
77.85	0.00344402420283549\\
77.86	0.00344219204051115\\
77.87	0.00344035970516587\\
77.88	0.00343852719702523\\
77.89	0.00343669451631016\\
77.9	0.00343486166323686\\
77.91	0.00343302863801686\\
77.92	0.00343119544085689\\
77.93	0.00342936207195888\\
77.94	0.00342752853151993\\
77.95	0.00342569481973224\\
77.96	0.00342386093678313\\
77.97	0.00342202688285492\\
77.98	0.00342019265812498\\
77.99	0.0034183582627656\\
78	0.00341652369694404\\
78.01	0.00341468896082242\\
78.02	0.00341285405455771\\
78.03	0.0034110189783017\\
78.04	0.00340918373220094\\
78.05	0.00340734831639672\\
78.06	0.00340551273102499\\
78.07	0.00340367697621637\\
78.08	0.0034018410520961\\
78.09	0.00340000495878394\\
78.1	0.0033981686963942\\
78.11	0.00339633226503567\\
78.12	0.00339449566481159\\
78.13	0.00339265889581956\\
78.14	0.00339082195815158\\
78.15	0.00338898485189393\\
78.16	0.00338714757712718\\
78.17	0.00338531013392611\\
78.18	0.0033834725223597\\
78.19	0.00338163474249106\\
78.2	0.00337979679437739\\
78.21	0.00337795867806996\\
78.22	0.00337612039361402\\
78.23	0.00337428194104883\\
78.24	0.00337244332040752\\
78.25	0.00337060453171712\\
78.26	0.00336876557499848\\
78.27	0.00336692645026624\\
78.28	0.00336508715752878\\
78.29	0.00336324769678816\\
78.3	0.00336140806804009\\
78.31	0.00335956827127388\\
78.32	0.00335772830647241\\
78.33	0.00335588817361203\\
78.34	0.00335404787266259\\
78.35	0.00335220740358731\\
78.36	0.00335036676634281\\
78.37	0.003348525960879\\
78.38	0.00334668498713908\\
78.39	0.00334484384505945\\
78.4	0.00334300253456968\\
78.41	0.00334116105559248\\
78.42	0.00333931940804361\\
78.43	0.00333747759183189\\
78.44	0.00333563560685906\\
78.45	0.00333379345301982\\
78.46	0.00333195113020174\\
78.47	0.0033301086382852\\
78.48	0.00332826597714336\\
78.49	0.0033264231466421\\
78.5	0.00332458014663997\\
78.51	0.00332273697698814\\
78.52	0.00332089363753033\\
78.53	0.00331905012810279\\
78.54	0.00331720644853424\\
78.55	0.00331536259864578\\
78.56	0.00331351857825087\\
78.57	0.00331167438715531\\
78.58	0.0033098300251571\\
78.59	0.00330798549204647\\
78.6	0.00330614078760577\\
78.61	0.00330429591160944\\
78.62	0.00330245086382396\\
78.63	0.00330060564400777\\
78.64	0.00329876025191126\\
78.65	0.00329691468727665\\
78.66	0.00329506894983799\\
78.67	0.00329322303932109\\
78.68	0.00329137695544344\\
78.69	0.00328953069791418\\
78.7	0.00328768426643403\\
78.71	0.00328583766069524\\
78.72	0.00328399088038152\\
78.73	0.003282143925168\\
78.74	0.00328029679472114\\
78.75	0.00327844948869875\\
78.76	0.00327660200674979\\
78.77	0.00327475434851447\\
78.78	0.00327290651362406\\
78.79	0.00327105850170093\\
78.8	0.00326921031235843\\
78.81	0.00326736194520083\\
78.82	0.00326551339982328\\
78.83	0.00326366467581177\\
78.84	0.003261815772743\\
78.85	0.00325996669018439\\
78.86	0.00325811742769398\\
78.87	0.00325626798482038\\
78.88	0.00325441836110268\\
78.89	0.00325256855607045\\
78.9	0.00325071856924359\\
78.91	0.00324886840013234\\
78.92	0.00324701802581425\\
78.93	0.0032451673072342\\
78.94	0.00324331624555393\\
78.95	0.00324146484193895\\
78.96	0.00323961309755856\\
78.97	0.00323776101358589\\
78.98	0.00323590859119788\\
78.99	0.00323405583157529\\
79	0.00323220273590273\\
79.01	0.00323034930536865\\
79.02	0.00322849554116538\\
79.03	0.00322664144448908\\
79.04	0.00322478701653983\\
79.05	0.00322293225852158\\
79.06	0.00322107717164219\\
79.07	0.00321922175711343\\
79.08	0.00321736601615098\\
79.09	0.00321550994997446\\
79.1	0.00321365355980746\\
79.11	0.00321179684687747\\
79.12	0.00320993981241599\\
79.13	0.00320808245765847\\
79.14	0.00320622478384436\\
79.15	0.0032043667922171\\
79.16	0.00320250848402414\\
79.17	0.00320064986051694\\
79.18	0.00319879092295099\\
79.19	0.00319693167258584\\
79.2	0.00319507211068505\\
79.21	0.00319321223851629\\
79.22	0.00319135205735127\\
79.23	0.00318949156846578\\
79.24	0.00318763077313972\\
79.25	0.00318576967265709\\
79.26	0.00318390826830602\\
79.27	0.00318204656137874\\
79.28	0.00318018455317163\\
79.29	0.00317832224498523\\
79.3	0.00317645963812422\\
79.31	0.00317459673389747\\
79.32	0.00317273353361804\\
79.33	0.00317087003860315\\
79.34	0.00316900625017426\\
79.35	0.00316714216965702\\
79.36	0.00316527779838135\\
79.37	0.00316341313768135\\
79.38	0.00316154818889544\\
79.39	0.00315968295336623\\
79.4	0.00315781743244065\\
79.41	0.00315595162746993\\
79.42	0.00315408553980954\\
79.43	0.00315221917081931\\
79.44	0.00315035252186337\\
79.45	0.00314848559431019\\
79.46	0.00314661838953256\\
79.47	0.00314475090890767\\
79.48	0.00314288315381703\\
79.49	0.00314101512564655\\
79.5	0.00313914682578656\\
79.51	0.00313727825563173\\
79.52	0.0031354094165812\\
79.53	0.0031335403100385\\
79.54	0.00313167093741162\\
79.55	0.00312980130011299\\
79.56	0.00312793139955953\\
79.57	0.00312606123717258\\
79.58	0.00312419081437801\\
79.59	0.00312232013260618\\
79.6	0.00312044919329195\\
79.61	0.00311857799787472\\
79.62	0.00311670654779842\\
79.63	0.00311483484451151\\
79.64	0.00311296288946703\\
79.65	0.00311109068412259\\
79.66	0.00310921822994038\\
79.67	0.00310734552838718\\
79.68	0.0031054725809344\\
79.69	0.00310359938905805\\
79.7	0.00310172595423879\\
79.71	0.00309985227796192\\
79.72	0.00309797836171739\\
79.73	0.00309610420699985\\
79.74	0.0030942298153086\\
79.75	0.00309235518814768\\
79.76	0.0030904803270258\\
79.77	0.00308860523345641\\
79.78	0.0030867299089577\\
79.79	0.0030848543550526\\
79.8	0.00308297857326881\\
79.81	0.00308110256513878\\
79.82	0.0030792263321998\\
79.83	0.00307734987599391\\
79.84	0.00307547319806798\\
79.85	0.00307359629997371\\
79.86	0.00307171918326764\\
79.87	0.00306984184951116\\
79.88	0.00306796430027053\\
79.89	0.00306608653711688\\
79.9	0.00306420856162624\\
79.91	0.00306233037537954\\
79.92	0.00306045197996264\\
79.93	0.00305857337696632\\
79.94	0.00305669456798631\\
79.95	0.00305481555462331\\
79.96	0.00305293633848297\\
79.97	0.00305105692117595\\
79.98	0.00304917730431789\\
79.99	0.00304729748952945\\
80	0.00304541747843634\\
80.01	0.00304353727266926\\
};
\addplot [color=mycolor1,dashed]
  table[row sep=crcr]{%
80.01	0.00304353727266926\\
80.02	0.00304165687386401\\
80.03	0.00303977628366145\\
80.04	0.00303789550370749\\
80.05	0.00303601453565319\\
80.06	0.00303413338115466\\
80.07	0.00303225204187319\\
80.08	0.00303037051947516\\
80.09	0.00302848881563213\\
80.1	0.00302660693202082\\
80.11	0.00302472487032312\\
80.12	0.00302284263222613\\
80.13	0.00302096021942213\\
80.14	0.00301907763360865\\
80.15	0.00301719487648845\\
80.16	0.00301531194976953\\
80.17	0.00301342885516517\\
80.18	0.00301154559439391\\
80.19	0.0030096621691796\\
80.2	0.00300777858125138\\
80.21	0.00300589483234375\\
80.22	0.00300401092419649\\
80.23	0.00300212685855479\\
80.24	0.00300024263716915\\
80.25	0.00299835826179551\\
80.26	0.00299647373419514\\
80.27	0.00299458905613478\\
80.28	0.00299270422938656\\
80.29	0.00299081925572806\\
80.3	0.00298893413694232\\
80.31	0.00298704887481783\\
80.32	0.00298516347114858\\
80.33	0.00298327792773406\\
80.34	0.00298139224637927\\
80.35	0.00297950642889473\\
80.36	0.00297762047709653\\
80.37	0.0029757343928063\\
80.38	0.00297384817785124\\
80.39	0.00297196183406415\\
80.4	0.00297007536328345\\
80.41	0.00296818876735313\\
80.42	0.00296630204812289\\
80.43	0.00296441520744802\\
80.44	0.00296252824718949\\
80.45	0.00296064116921398\\
80.46	0.00295875397539385\\
80.47	0.00295686666760715\\
80.48	0.00295497924773771\\
80.49	0.00295309171767505\\
80.5	0.00295120407931448\\
80.51	0.00294931633455711\\
80.52	0.00294742848530978\\
80.53	0.00294554053348519\\
80.54	0.00294365248100185\\
80.55	0.0029417643297841\\
80.56	0.00293987608176214\\
80.57	0.00293798773887204\\
80.58	0.00293609930305577\\
80.59	0.0029342107762612\\
80.6	0.0029323221604421\\
80.61	0.00293043345755821\\
80.62	0.00292854466957518\\
80.63	0.00292665579846467\\
80.64	0.00292476684620432\\
80.65	0.00292287781477774\\
80.66	0.00292098870617459\\
80.67	0.00291909952239054\\
80.68	0.00291721026542733\\
80.69	0.00291532093729277\\
80.7	0.00291343154000073\\
80.71	0.00291154207557123\\
80.72	0.00290965254603034\\
80.73	0.00290776295341032\\
80.74	0.00290587329974955\\
80.75	0.00290398358709259\\
80.76	0.0029020938174902\\
80.77	0.00290020399299931\\
80.78	0.00289831411568309\\
80.79	0.00289642418761094\\
80.8	0.00289453421085853\\
80.81	0.00289264418750777\\
80.82	0.00289075411964688\\
80.83	0.00288886400937039\\
80.84	0.00288697385877912\\
80.85	0.00288508366998026\\
80.86	0.00288319344508736\\
80.87	0.00288130318622032\\
80.88	0.00287941289550544\\
80.89	0.00287752257507544\\
80.9	0.00287563222706947\\
80.91	0.00287374185363311\\
80.92	0.0028718514569184\\
80.93	0.00286996103908389\\
80.94	0.00286807060229459\\
80.95	0.00286618014872204\\
80.96	0.00286428968054433\\
80.97	0.00286239919994609\\
80.98	0.00286050870911851\\
80.99	0.00285861821025938\\
81	0.00285672770557309\\
81.01	0.00285483719727066\\
81.02	0.00285294668756975\\
81.03	0.00285105617869468\\
81.04	0.00284916567287645\\
81.05	0.00284727517235275\\
81.06	0.002845384679368\\
81.07	0.00284349419617334\\
81.08	0.00284160372502667\\
81.09	0.00283971326819266\\
81.1	0.00283782282794278\\
81.11	0.00283593240655528\\
81.12	0.00283404200631526\\
81.13	0.00283215162951468\\
81.14	0.00283026127845233\\
81.15	0.0028283709554339\\
81.16	0.00282648066277199\\
81.17	0.00282459040278611\\
81.18	0.00282270017780273\\
81.19	0.00282080999015524\\
81.2	0.00281891984218405\\
81.21	0.00281702973623655\\
81.22	0.00281513967466715\\
81.23	0.0028132496598373\\
81.24	0.0028113596941155\\
81.25	0.00280946977987733\\
81.26	0.00280757991950546\\
81.27	0.00280569011538968\\
81.28	0.00280380036992691\\
81.29	0.00280191068552124\\
81.3	0.0028000210645839\\
81.31	0.00279813150953334\\
81.32	0.00279624202279521\\
81.33	0.00279435260680241\\
81.34	0.00279246326399507\\
81.35	0.00279057399682061\\
81.36	0.00278868480773372\\
81.37	0.00278679569919644\\
81.38	0.00278490667367812\\
81.39	0.00278301773365544\\
81.4	0.0027811288816125\\
81.41	0.00277924012004077\\
81.42	0.00277735145143912\\
81.43	0.00277546287831388\\
81.44	0.00277357440317882\\
81.45	0.0027716860285552\\
81.46	0.00276979775697175\\
81.47	0.00276790959096474\\
81.48	0.00276602153307797\\
81.49	0.00276413358586279\\
81.5	0.00276224575187813\\
81.51	0.00276035803369053\\
81.52	0.00275847043387414\\
81.53	0.00275658295501077\\
81.54	0.00275469559968986\\
81.55	0.00275280837050856\\
81.56	0.00275092127007172\\
81.57	0.00274903430099191\\
81.58	0.00274714746588943\\
81.59	0.00274526076739239\\
81.6	0.00274337420813666\\
81.61	0.00274148779076592\\
81.62	0.00273960151793169\\
81.63	0.00273771539229336\\
81.64	0.00273582941651816\\
81.65	0.00273394359328124\\
81.66	0.00273205792526566\\
81.67	0.00273017241516245\\
81.68	0.00272828706567055\\
81.69	0.00272640187949693\\
81.7	0.00272451685935655\\
81.71	0.00272263200797238\\
81.72	0.00272074732807547\\
81.73	0.00271886282240493\\
81.74	0.00271697849370796\\
81.75	0.00271509434473988\\
81.76	0.00271321037826415\\
81.77	0.00271132659705241\\
81.78	0.00270944300388443\\
81.79	0.00270755960154826\\
81.8	0.00270567639284011\\
81.81	0.00270379338056448\\
81.82	0.00270191056753414\\
81.83	0.00270002795657016\\
81.84	0.0026981455505019\\
81.85	0.0026962633521671\\
81.86	0.00269438136441184\\
81.87	0.00269249959009061\\
81.88	0.00269061803206628\\
81.89	0.00268873669321017\\
81.9	0.00268685557640208\\
81.91	0.00268497468453025\\
81.92	0.00268309402049145\\
81.93	0.00268121358719096\\
81.94	0.00267933338754263\\
81.95	0.00267745342446887\\
81.96	0.00267557370090067\\
81.97	0.00267369421977768\\
81.98	0.00267181498404818\\
81.99	0.00266993599666908\\
82	0.00266805726060605\\
82.01	0.00266617877883341\\
82.02	0.00266430055433425\\
82.03	0.00266242259010042\\
82.04	0.00266054488913255\\
82.05	0.0026586674544401\\
82.06	0.00265679028904133\\
82.07	0.0026549133959634\\
82.08	0.00265303677824231\\
82.09	0.002651160438923\\
82.1	0.00264928438105933\\
82.11	0.00264740860771411\\
82.12	0.00264553312195914\\
82.13	0.00264365792687521\\
82.14	0.00264178302555216\\
82.15	0.00263990842108888\\
82.16	0.00263803411659331\\
82.17	0.00263616011518252\\
82.18	0.00263428641998271\\
82.19	0.00263241303412921\\
82.2	0.00263053996076654\\
82.21	0.00262866720304844\\
82.22	0.00262679476413785\\
82.23	0.00262492264720697\\
82.24	0.00262305085543727\\
82.25	0.00262117939201955\\
82.26	0.00261930826015391\\
82.27	0.00261743746304982\\
82.28	0.00261556700392611\\
82.29	0.00261369688601102\\
82.3	0.00261182711254224\\
82.31	0.00260995768676689\\
82.32	0.00260808861194157\\
82.33	0.00260621989133241\\
82.34	0.00260435152821504\\
82.35	0.00260248352587466\\
82.36	0.00260061588760607\\
82.37	0.00259874861671366\\
82.38	0.00259688171651145\\
82.39	0.00259501519032313\\
82.4	0.00259314904148209\\
82.41	0.00259128327333141\\
82.42	0.00258941788922391\\
82.43	0.0025875528925222\\
82.44	0.00258568828659864\\
82.45	0.00258382407483545\\
82.46	0.00258196026062466\\
82.47	0.00258009684736821\\
82.48	0.00257823383847789\\
82.49	0.00257637123737543\\
82.5	0.00257450904749254\\
82.51	0.00257264727227086\\
82.52	0.00257078591516208\\
82.53	0.00256892497962787\\
82.54	0.00256706446914\\
82.55	0.00256520438718029\\
82.56	0.0025633447372407\\
82.57	0.0025614855228233\\
82.58	0.00255962674744034\\
82.59	0.00255776841461427\\
82.6	0.00255591052787775\\
82.61	0.00255405309077367\\
82.62	0.0025521961068552\\
82.63	0.00255033957968584\\
82.64	0.00254848351283938\\
82.65	0.00254662790989999\\
82.66	0.0025447727744622\\
82.67	0.00254291811017059\\
82.68	0.0025410639206978\\
82.69	0.00253921020972722\\
82.7	0.00253735698095301\\
82.71	0.00253550423808015\\
82.72	0.00253365198482446\\
82.73	0.00253180022491263\\
82.74	0.00252994896208225\\
82.75	0.00252809820008186\\
82.76	0.00252624794267094\\
82.77	0.00252439819361996\\
82.78	0.00252254895671042\\
82.79	0.00252070023573489\\
82.8	0.00251885203449698\\
82.81	0.00251700435681145\\
82.82	0.0025151572065042\\
82.83	0.00251331058741228\\
82.84	0.00251146450338398\\
82.85	0.0025096189582788\\
82.86	0.00250777395596752\\
82.87	0.00250592950033223\\
82.88	0.00250408559526632\\
82.89	0.00250224224467455\\
82.9	0.0025003994524731\\
82.91	0.00249855722258955\\
82.92	0.00249671555896294\\
82.93	0.00249487446554377\\
82.94	0.00249303394629412\\
82.95	0.00249119400518756\\
82.96	0.00248935464620928\\
82.97	0.00248751587335607\\
82.98	0.00248567769063636\\
82.99	0.00248384010207026\\
83	0.00248200311168961\\
83.01	0.00248016672353796\\
83.02	0.00247833094167067\\
83.03	0.00247649577015488\\
83.04	0.00247466121306958\\
83.05	0.00247282727450564\\
83.06	0.00247099395856581\\
83.07	0.00246916126936481\\
83.08	0.00246732921102931\\
83.09	0.00246549778769798\\
83.1	0.00246366700352155\\
83.11	0.0024618368626628\\
83.12	0.00246000736929661\\
83.13	0.00245817852761002\\
83.14	0.00245635034180221\\
83.15	0.00245452281608459\\
83.16	0.00245269595468078\\
83.17	0.0024508697618267\\
83.18	0.00244904424177054\\
83.19	0.00244721939877286\\
83.2	0.00244539523710658\\
83.21	0.00244357176105701\\
83.22	0.00244174897492192\\
83.23	0.00243992688301155\\
83.24	0.00243810548964865\\
83.25	0.0024362847991685\\
83.26	0.00243446481591898\\
83.27	0.00243264554426054\\
83.28	0.00243082698856633\\
83.29	0.00242900915322214\\
83.3	0.00242719204262649\\
83.31	0.00242537566119067\\
83.32	0.00242356001333871\\
83.33	0.0024217451035075\\
83.34	0.00241993093614677\\
83.35	0.00241811751571915\\
83.36	0.00241630484670017\\
83.37	0.00241449293357837\\
83.38	0.00241268178085524\\
83.39	0.00241087139304533\\
83.4	0.00240906177467625\\
83.41	0.00240725293028871\\
83.42	0.00240544486443657\\
83.43	0.00240363758168687\\
83.44	0.00240183108661983\\
83.45	0.00240002538382897\\
83.46	0.00239822047792105\\
83.47	0.00239641637351617\\
83.48	0.00239461307524779\\
83.49	0.00239281058776275\\
83.5	0.00239100891572134\\
83.51	0.00238920806379731\\
83.52	0.0023874080366779\\
83.53	0.00238560883906391\\
83.54	0.00238381047566971\\
83.55	0.00238201295122327\\
83.56	0.00238021627046625\\
83.57	0.00237842043815396\\
83.58	0.00237662545905546\\
83.59	0.00237483133795356\\
83.6	0.00237303807964488\\
83.61	0.00237124568893989\\
83.62	0.0023694541706629\\
83.63	0.00236766352965218\\
83.64	0.00236587377075993\\
83.65	0.00236408489885232\\
83.66	0.0023622969188096\\
83.67	0.00236050983552604\\
83.68	0.00235872365391003\\
83.69	0.00235693837888411\\
83.7	0.002355154015385\\
83.71	0.00235337056836363\\
83.72	0.00235158804278521\\
83.73	0.00234980644362922\\
83.74	0.00234802577588949\\
83.75	0.00234624604457424\\
83.76	0.00234446725470607\\
83.77	0.00234268941132205\\
83.78	0.00234091251947377\\
83.79	0.0023391365842273\\
83.8	0.00233736161066333\\
83.81	0.00233558760387712\\
83.82	0.0023338145689786\\
83.83	0.0023320425110924\\
83.84	0.00233027143535785\\
83.85	0.00232850134692908\\
83.86	0.00232673225097501\\
83.87	0.0023249641526794\\
83.88	0.00232319705724093\\
83.89	0.00232143096987319\\
83.9	0.00231966589580473\\
83.91	0.00231790184027913\\
83.92	0.002316138808555\\
83.93	0.00231437680590606\\
83.94	0.00231261583762115\\
83.95	0.00231085590900428\\
83.96	0.00230909702537469\\
83.97	0.00230733919206685\\
83.98	0.00230558241443053\\
83.99	0.00230382669783086\\
84	0.0023020720476483\\
84.01	0.00230031846927878\\
84.02	0.00229856596813366\\
84.03	0.0022968145496398\\
84.04	0.0022950642192396\\
84.05	0.00229331498239107\\
84.06	0.00229156684456781\\
84.07	0.00228981981125912\\
84.08	0.00228807388796998\\
84.09	0.00228632908022113\\
84.1	0.00228458539354913\\
84.11	0.00228284283350632\\
84.12	0.00228110140566098\\
84.13	0.00227936111559726\\
84.14	0.0022776219689153\\
84.15	0.00227588397123123\\
84.16	0.00227414712817724\\
84.17	0.00227241144540159\\
84.18	0.00227067692856869\\
84.19	0.00226894358335912\\
84.2	0.00226721141546968\\
84.21	0.00226548043061342\\
84.22	0.0022637506345197\\
84.23	0.00226202203293424\\
84.24	0.00226029463161913\\
84.25	0.0022585684363529\\
84.26	0.00225684345293057\\
84.27	0.00225511968716365\\
84.28	0.00225339714488026\\
84.29	0.00225167583192508\\
84.3	0.00224995575415948\\
84.31	0.00224823691746149\\
84.32	0.0022465193277259\\
84.33	0.00224480299086428\\
84.34	0.00224308791280503\\
84.35	0.00224137409949341\\
84.36	0.0022396615568916\\
84.37	0.00223795029097873\\
84.38	0.00223624030775097\\
84.39	0.0022345316132215\\
84.4	0.0022328242134206\\
84.41	0.00223111811439569\\
84.42	0.00222941332221137\\
84.43	0.00222770984294947\\
84.44	0.00222600768270911\\
84.45	0.00222430684760668\\
84.46	0.00222260734377597\\
84.47	0.00222090917736815\\
84.48	0.00221921235455187\\
84.49	0.00221751688151324\\
84.5	0.00221582276445595\\
84.51	0.00221413000960125\\
84.52	0.00221243862318802\\
84.53	0.00221074861147283\\
84.54	0.00220905998072997\\
84.55	0.0022073727372515\\
84.56	0.00220568688734729\\
84.57	0.00220400243734506\\
84.58	0.00220231939359044\\
84.59	0.00220063776244705\\
84.6	0.00219895755029643\\
84.61	0.00219727876353823\\
84.62	0.00219560140859017\\
84.63	0.00219392549188808\\
84.64	0.00219225101988599\\
84.65	0.00219057799905618\\
84.66	0.00218890643588915\\
84.67	0.00218723633689376\\
84.68	0.00218556770859724\\
84.69	0.0021839005575452\\
84.7	0.00218223489030173\\
84.71	0.00218057071344942\\
84.72	0.00217890803358942\\
84.73	0.00217724685734146\\
84.74	0.00217558719134394\\
84.75	0.00217392904225392\\
84.76	0.00217227241674722\\
84.77	0.00217061732151845\\
84.78	0.00216896376328104\\
84.79	0.00216731174876729\\
84.8	0.00216566128472845\\
84.81	0.00216401237793472\\
84.82	0.00216236503517533\\
84.83	0.00216071926325858\\
84.84	0.00215907506901188\\
84.85	0.0021574324592818\\
84.86	0.00215579144093411\\
84.87	0.00215415202085385\\
84.88	0.00215251420594534\\
84.89	0.00215087800313229\\
84.9	0.00214924341935776\\
84.91	0.00214761046158429\\
84.92	0.00214597913679389\\
84.93	0.00214434945198812\\
84.94	0.00214272141418812\\
84.95	0.00214109503043466\\
84.96	0.0021394703077882\\
84.97	0.00213784725332892\\
84.98	0.00213622587415679\\
84.99	0.00213460617739158\\
85	0.00213298817017296\\
85.01	0.00213137185966049\\
85.02	0.00212975725303372\\
85.03	0.00212814435749219\\
85.04	0.00212653318025552\\
85.05	0.00212492372856342\\
85.06	0.00212331600967577\\
85.07	0.00212171003087266\\
85.08	0.00212010579945441\\
85.09	0.00211850332274166\\
85.1	0.00211690260807536\\
85.11	0.00211530366281691\\
85.12	0.0021137064943481\\
85.13	0.00211211111007123\\
85.14	0.00211051751740912\\
85.15	0.00210892572380521\\
85.16	0.00210733573672353\\
85.17	0.00210574756364883\\
85.18	0.00210416121208652\\
85.19	0.00210257668956286\\
85.2	0.00210099400362488\\
85.21	0.00209941316184049\\
85.22	0.00209783417179852\\
85.23	0.00209625704110876\\
85.24	0.00209468177740201\\
85.25	0.00209310838833011\\
85.26	0.00209153688156605\\
85.27	0.00208996726480391\\
85.28	0.00208839954575901\\
85.29	0.00208683373216791\\
85.3	0.00208526983178843\\
85.31	0.00208370785239978\\
85.32	0.00208214780180252\\
85.33	0.00208058968781865\\
85.34	0.00207903351829165\\
85.35	0.00207747930108653\\
85.36	0.00207592704408986\\
85.37	0.00207437675520985\\
85.38	0.00207282844237635\\
85.39	0.00207128211354095\\
85.4	0.00206973777667697\\
85.41	0.00206819543977955\\
85.42	0.00206665511086567\\
85.43	0.00206511679797422\\
85.44	0.00206358050916603\\
85.45	0.0020620462525239\\
85.46	0.00206051403615268\\
85.47	0.0020589838681793\\
85.48	0.00205745575675281\\
85.49	0.00205592971004443\\
85.5	0.00205440573624759\\
85.51	0.00205288384357798\\
85.52	0.00205136404027362\\
85.53	0.00204984633459484\\
85.54	0.00204833008408384\\
85.55	0.0020468139033471\\
85.56	0.00204529778880708\\
85.57	0.00204378173685461\\
85.58	0.00204226574384867\\
85.59	0.00204074980611624\\
85.6	0.00203923391995215\\
85.61	0.00203771808161887\\
85.62	0.00203620228734638\\
85.63	0.00203468653565361\\
85.64	0.00203317082532171\\
85.65	0.00203165515511737\\
85.66	0.00203013952379279\\
85.67	0.00202862393008553\\
85.68	0.00202710837271848\\
85.69	0.00202559285039976\\
85.7	0.00202407736182261\\
85.71	0.00202256190566536\\
85.72	0.00202104648059128\\
85.73	0.00201953108524853\\
85.74	0.0020180157182701\\
85.75	0.00201650037827366\\
85.76	0.00201498506386151\\
85.77	0.00201346977362051\\
85.78	0.00201195450612195\\
85.79	0.00201043925992148\\
85.8	0.00200892403355904\\
85.81	0.00200740882555875\\
85.82	0.00200589363442883\\
85.83	0.00200437845866147\\
85.84	0.00200286329673284\\
85.85	0.00200134814710286\\
85.86	0.00199983300821525\\
85.87	0.00199831787849733\\
85.88	0.00199680275635998\\
85.89	0.00199528764019755\\
85.9	0.00199377252838773\\
85.91	0.0019922574192915\\
85.92	0.00199074231125302\\
85.93	0.00198922720259953\\
85.94	0.00198771209164123\\
85.95	0.00198619697667126\\
85.96	0.00198468185596554\\
85.97	0.00198316672778267\\
85.98	0.0019816515903639\\
85.99	0.00198013644193294\\
86	0.00197862128069595\\
86.01	0.0019771061048414\\
86.02	0.00197559091253995\\
86.03	0.00197407570194441\\
86.04	0.0019725604711896\\
86.05	0.00197104521839224\\
86.06	0.00196952994165089\\
86.07	0.00196801463904582\\
86.08	0.00196649930863893\\
86.09	0.00196498394847363\\
86.1	0.00196346855657473\\
86.11	0.00196195313094838\\
86.12	0.00196043766958193\\
86.13	0.00195892217044384\\
86.14	0.00195740663148355\\
86.15	0.00195589105063144\\
86.16	0.00195437542579864\\
86.17	0.00195285975487702\\
86.18	0.001951344035739\\
86.19	0.00194982826623749\\
86.2	0.00194831244420576\\
86.21	0.00194679656745737\\
86.22	0.00194528063378601\\
86.23	0.00194376464096545\\
86.24	0.00194224858674937\\
86.25	0.00194073246887131\\
86.26	0.00193921628504451\\
86.27	0.00193770003296185\\
86.28	0.00193618371029569\\
86.29	0.0019346673146978\\
86.3	0.00193315084379922\\
86.31	0.00193163429521015\\
86.32	0.00193011766651988\\
86.33	0.0019286009552966\\
86.34	0.00192708415908738\\
86.35	0.00192556727541796\\
86.36	0.00192405030179271\\
86.37	0.0019225332356945\\
86.38	0.00192101607458454\\
86.39	0.00191949881590231\\
86.4	0.00191798145706544\\
86.41	0.00191646399546956\\
86.42	0.00191494642848825\\
86.43	0.00191342875347283\\
86.44	0.00191191096775232\\
86.45	0.00191039306863327\\
86.46	0.00190887505339968\\
86.47	0.00190735691931285\\
86.48	0.00190583866361128\\
86.49	0.00190432028351053\\
86.5	0.0019028017762031\\
86.51	0.00190128313885835\\
86.52	0.00189976436862231\\
86.53	0.00189824546261761\\
86.54	0.00189672641794331\\
86.55	0.00189520723167483\\
86.56	0.00189368790086379\\
86.57	0.00189216842253788\\
86.58	0.00189064879370077\\
86.59	0.00188912901133193\\
86.6	0.00188760907238656\\
86.61	0.00188608897379541\\
86.62	0.00188456871246469\\
86.63	0.00188304828527594\\
86.64	0.00188152768908586\\
86.65	0.00188000692072623\\
86.66	0.00187848597700376\\
86.67	0.00187696485469994\\
86.68	0.00187544355057096\\
86.69	0.00187392206134751\\
86.7	0.00187240038373469\\
86.71	0.00187087851441188\\
86.72	0.00186935645003259\\
86.73	0.00186783418722434\\
86.74	0.00186631172258848\\
86.75	0.00186478905270014\\
86.76	0.00186326617410802\\
86.77	0.00186174308333427\\
86.78	0.00186021977687437\\
86.79	0.00185869625119699\\
86.8	0.00185717250274383\\
86.81	0.00185564852792951\\
86.82	0.0018541243231414\\
86.83	0.00185259988473952\\
86.84	0.00185107520905636\\
86.85	0.00184955029239674\\
86.86	0.00184802513103772\\
86.87	0.00184649972122839\\
86.88	0.00184497405918975\\
86.89	0.00184344814111461\\
86.9	0.00184192196316737\\
86.91	0.00184039552148395\\
86.92	0.00183886881217158\\
86.93	0.00183734183130869\\
86.94	0.00183581457875494\\
86.95	0.00183428705458884\\
86.96	0.00183275925889077\\
86.97	0.00183123119174291\\
86.98	0.00182970285322933\\
86.99	0.00182817424343597\\
87	0.00182664536245068\\
87.01	0.00182511621036325\\
87.02	0.0018235867872654\\
87.03	0.00182205709325086\\
87.04	0.00182052712841532\\
87.05	0.00181899689285651\\
87.06	0.00181746638667421\\
87.07	0.00181593560997025\\
87.08	0.00181440456284857\\
87.09	0.00181287324541521\\
87.1	0.00181134165777834\\
87.11	0.00180980980004833\\
87.12	0.00180827767233768\\
87.13	0.00180674527476116\\
87.14	0.00180521260743573\\
87.15	0.00180367967048063\\
87.16	0.00180214646401738\\
87.17	0.00180061298816982\\
87.18	0.00179907924306411\\
87.19	0.00179754522882879\\
87.2	0.00179601094559476\\
87.21	0.00179447639349536\\
87.22	0.00179294157266635\\
87.23	0.00179140648324595\\
87.24	0.0017898711253749\\
87.25	0.00178833549919643\\
87.26	0.00178679960485631\\
87.27	0.00178526344250292\\
87.28	0.00178372701228719\\
87.29	0.00178219031436271\\
87.3	0.00178065334888571\\
87.31	0.0017791161160151\\
87.32	0.0017775786159125\\
87.33	0.00177604084874228\\
87.34	0.00177450281467155\\
87.35	0.00177296451387023\\
87.36	0.00177142594651107\\
87.37	0.00176988711276967\\
87.38	0.00176834801282448\\
87.39	0.00176680864685691\\
87.4	0.00176526901505126\\
87.41	0.00176372911759485\\
87.42	0.00176218895467795\\
87.43	0.0017606485264939\\
87.44	0.00175910783323908\\
87.45	0.00175756687511297\\
87.46	0.00175602565231815\\
87.47	0.00175448416506038\\
87.48	0.0017529424135486\\
87.49	0.00175140039799495\\
87.5	0.00174985811861484\\
87.51	0.00174831557562693\\
87.52	0.00174677276925322\\
87.53	0.00174522969971903\\
87.54	0.00174368636725308\\
87.55	0.00174214277208748\\
87.56	0.0017405989144578\\
87.57	0.00173905479460306\\
87.58	0.00173751041276582\\
87.59	0.00173596576919216\\
87.6	0.00173442086413175\\
87.61	0.00173287569783786\\
87.62	0.00173133027056742\\
87.63	0.00172978458258103\\
87.64	0.00172823863414301\\
87.65	0.00172669242552142\\
87.66	0.00172514595698811\\
87.67	0.00172359922881878\\
87.68	0.00172205224129295\\
87.69	0.00172050499469403\\
87.7	0.00171895748930941\\
87.71	0.00171740972543039\\
87.72	0.00171586170335231\\
87.73	0.00171431342337453\\
87.74	0.00171276488600161\\
87.75	0.00171121609174509\\
87.76	0.0017096670411175\\
87.77	0.00170811773463236\\
87.78	0.00170656817280413\\
87.79	0.0017050183561483\\
87.8	0.00170346828518129\\
87.81	0.00170191796042053\\
87.82	0.00170036738238442\\
87.83	0.00169881655159234\\
87.84	0.00169726546856465\\
87.85	0.00169571413382271\\
87.86	0.00169416254788885\\
87.87	0.00169261071128638\\
87.88	0.0016910586245396\\
87.89	0.00168950628817381\\
87.9	0.00168795370271527\\
87.91	0.00168640086869125\\
87.92	0.00168484778662998\\
87.93	0.00168329445706073\\
87.94	0.0016817408805137\\
87.95	0.00168018705752012\\
87.96	0.0016786329886122\\
87.97	0.00167707867432313\\
87.98	0.00167552411518711\\
87.99	0.00167396931173932\\
88	0.00167241426451595\\
88.01	0.00167085897405416\\
88.02	0.00166930344089214\\
88.03	0.00166774766556904\\
88.04	0.00166619164862501\\
88.05	0.00166463539060124\\
88.06	0.00166307889203987\\
88.07	0.00166152215348405\\
88.08	0.00165996517547795\\
88.09	0.00165840795856672\\
88.1	0.00165685050329652\\
88.11	0.00165529281021449\\
88.12	0.00165373487986882\\
88.13	0.00165217671280865\\
88.14	0.00165061830958415\\
88.15	0.0016490596707465\\
88.16	0.00164750079684787\\
88.17	0.00164594168844144\\
88.18	0.0016443823460814\\
88.19	0.00164282277032295\\
88.2	0.00164126296172228\\
88.21	0.00163970292083661\\
88.22	0.00163814264822416\\
88.23	0.00163658214444415\\
88.24	0.00163502141005682\\
88.25	0.00163346044562343\\
88.26	0.00163189925170623\\
88.27	0.0016303378288685\\
88.28	0.00162877617767453\\
88.29	0.00162721429868961\\
88.3	0.00162565219248005\\
88.31	0.00162408985961319\\
88.32	0.00162252730065737\\
88.33	0.00162096451618195\\
88.34	0.00161940150675731\\
88.35	0.00161783827295483\\
88.36	0.00161627481534692\\
88.37	0.00161471113450704\\
88.38	0.0016131472310096\\
88.39	0.00161158310543008\\
88.4	0.00161001875834498\\
88.41	0.0016084541903318\\
88.42	0.00160688940196906\\
88.43	0.00160532439383633\\
88.44	0.00160375916651418\\
88.45	0.00160219372058421\\
88.46	0.00160062805662903\\
88.47	0.00159906217523229\\
88.48	0.00159749607697868\\
88.49	0.00159592976245387\\
88.5	0.0015943632322446\\
88.51	0.0015927964869386\\
88.52	0.00159122952712467\\
88.53	0.00158966235339261\\
88.54	0.00158809496633324\\
88.55	0.00158652736653842\\
88.56	0.00158495955460107\\
88.57	0.00158339153111507\\
88.58	0.0015818232966754\\
88.59	0.00158025485187803\\
88.6	0.00157868619731998\\
88.61	0.0015771173335993\\
88.62	0.00157554826131506\\
88.63	0.00157397898106738\\
88.64	0.00157240949345739\\
88.65	0.00157083979908729\\
88.66	0.00156926989856029\\
88.67	0.00156769979248064\\
88.68	0.00156612948145362\\
88.69	0.00156455896608555\\
88.7	0.0015629882469838\\
88.71	0.00156141732475676\\
88.72	0.00155984620001387\\
88.73	0.0015582748733656\\
88.74	0.00155670334542346\\
88.75	0.0015551316168\\
88.76	0.00155355968810881\\
88.77	0.00155198755996452\\
88.78	0.00155041523298281\\
88.79	0.00154884270778039\\
88.8	0.001547269984975\\
88.81	0.00154569706518546\\
88.82	0.00154412394903159\\
88.83	0.00154255063713428\\
88.84	0.00154097713011545\\
88.85	0.00153940342859808\\
88.86	0.00153782953320617\\
88.87	0.0015362554445648\\
88.88	0.00153468116330005\\
88.89	0.00153310669003908\\
88.9	0.00153153202541008\\
88.91	0.00152995717004231\\
88.92	0.00152838212456604\\
88.93	0.00152680688961261\\
88.94	0.00152523146581441\\
88.95	0.00152365585380487\\
88.96	0.00152208005421847\\
88.97	0.00152050406769074\\
88.98	0.00151892789485826\\
88.99	0.00151735153635866\\
89	0.00151577499283061\\
89.01	0.00151419826491386\\
89.02	0.00151262135324917\\
89.03	0.00151104425847838\\
89.04	0.00150946698124437\\
89.05	0.00150788952219108\\
89.06	0.0015063118819635\\
89.07	0.00150473406120766\\
89.08	0.00150315606057067\\
89.09	0.00150157788070066\\
89.1	0.00149999952224684\\
89.11	0.00149842098585946\\
89.12	0.00149684227218983\\
89.13	0.00149526338189033\\
89.14	0.00149368431561435\\
89.15	0.00149210507401638\\
89.16	0.00149052565775195\\
89.17	0.00148894606747764\\
89.18	0.00148736630385111\\
89.19	0.00148578636753103\\
89.2	0.00148420625917718\\
89.21	0.00148262597945036\\
89.22	0.00148104552901245\\
89.23	0.00147946490852637\\
89.24	0.00147788411865611\\
89.25	0.00147630316006671\\
89.26	0.00147472203342429\\
89.27	0.00147314073939598\\
89.28	0.00147155927865002\\
89.29	0.0014699776518557\\
89.3	0.00146839585968334\\
89.31	0.00146681390280435\\
89.32	0.00146523178189118\\
89.33	0.00146364949761737\\
89.34	0.00146206705065748\\
89.35	0.00146048444168716\\
89.36	0.00145890167138312\\
89.37	0.0014573187404231\\
89.38	0.00145573564948594\\
89.39	0.00145415239925152\\
89.4	0.0014525689904008\\
89.41	0.00145098542361577\\
89.42	0.00144940169957952\\
89.43	0.00144781781897617\\
89.44	0.00144623378249092\\
89.45	0.00144464959081004\\
89.46	0.00144306524462083\\
89.47	0.0014414807446117\\
89.48	0.00143989609147208\\
89.49	0.00143831128589248\\
89.5	0.00143672632856448\\
89.51	0.00143514122018072\\
89.52	0.00143355596143489\\
89.53	0.00143197055302177\\
89.54	0.00143038499563718\\
89.55	0.00142879928997802\\
89.56	0.00142721343674223\\
89.57	0.00142562743662885\\
89.58	0.00142404129033795\\
89.59	0.00142245499857069\\
89.6	0.00142086856202929\\
89.61	0.00141928198141701\\
89.62	0.0014176952574382\\
89.63	0.00141610839079828\\
89.64	0.0014145213822037\\
89.65	0.00141293423236202\\
89.66	0.00141134694198182\\
89.67	0.00140975951177278\\
89.68	0.00140817194244564\\
89.69	0.00140658423471218\\
89.7	0.00140499638928526\\
89.71	0.00140340840687882\\
89.72	0.00140182028820784\\
89.73	0.00140023203398838\\
89.74	0.00139864364493757\\
89.75	0.00139705512177357\\
89.76	0.00139546646521566\\
89.77	0.00139387767598415\\
89.78	0.0013922887548004\\
89.79	0.00139069970238687\\
89.8	0.00138911051946708\\
89.81	0.00138752120676558\\
89.82	0.00138593176500804\\
89.83	0.00138434219492114\\
89.84	0.00138275249723265\\
89.85	0.00138116267267141\\
89.86	0.00137957272196733\\
89.87	0.00137798264585134\\
89.88	0.00137639244505549\\
89.89	0.00137480212031287\\
89.9	0.00137321167235763\\
89.91	0.00137162110192497\\
89.92	0.0013700304097512\\
89.93	0.00136843959657365\\
89.94	0.00136684866313073\\
89.95	0.00136525761016191\\
89.96	0.00136366643840773\\
89.97	0.00136207514860979\\
89.98	0.00136048374151075\\
89.99	0.00135889221785432\\
90	0.0013573005783853\\
90.01	0.00135570882384954\\
90.02	0.00135411695499394\\
90.03	0.00135252497256649\\
90.04	0.00135093287731621\\
90.05	0.00134934066999319\\
90.06	0.0013477483513486\\
90.07	0.00134615592213466\\
90.08	0.00134456338310465\\
90.09	0.00134297073501289\\
90.1	0.0013413779786148\\
90.11	0.00133978511466683\\
90.12	0.00133819214392651\\
90.13	0.00133659906715241\\
90.14	0.00133500588510417\\
90.15	0.00133341259854249\\
90.16	0.00133181920822912\\
90.17	0.00133022571492689\\
90.18	0.00132863211939965\\
90.19	0.00132703842241236\\
90.2	0.00132544462473098\\
90.21	0.00132385072712257\\
90.22	0.00132225673035522\\
90.23	0.0013206626351981\\
90.24	0.00131906844242142\\
90.25	0.00131747415279646\\
90.26	0.00131587976709552\\
90.27	0.001314285286092\\
90.28	0.00131269071056033\\
90.29	0.001311096041276\\
90.3	0.00130950127901554\\
90.31	0.00130790642455656\\
90.32	0.0013063114786777\\
90.33	0.00130471644215867\\
90.34	0.00130312131578022\\
90.35	0.00130152610032414\\
90.36	0.0012999307965733\\
90.37	0.00129833540531161\\
90.38	0.00129673992732401\\
90.39	0.00129514436339652\\
90.4	0.00129354871431618\\
90.41	0.00129195298087111\\
90.42	0.00129035716385045\\
90.43	0.00128876126404441\\
90.44	0.00128716528224422\\
90.45	0.00128556921924219\\
90.46	0.00128397307583165\\
90.47	0.001282376852807\\
90.48	0.00128078055096364\\
90.49	0.00127918417109807\\
90.5	0.00127758771400779\\
90.51	0.00127599118049137\\
90.52	0.00127439457134841\\
90.53	0.00127279788737956\\
90.54	0.00127120112938651\\
90.55	0.00126960429817198\\
90.56	0.00126800739453974\\
90.57	0.00126641041929459\\
90.58	0.00126481337324239\\
90.59	0.00126321625719001\\
90.6	0.00126161907194538\\
90.61	0.00126002181831746\\
90.62	0.00125842449711624\\
90.63	0.00125682710915275\\
90.64	0.00125522965523905\\
90.65	0.00125363213618825\\
90.66	0.00125203455281447\\
90.67	0.00125043690593289\\
90.68	0.00124883919635969\\
90.69	0.0012472414249121\\
90.7	0.00124564359240839\\
90.71	0.00124404569966783\\
90.72	0.00124244774751074\\
90.73	0.00124084973675848\\
90.74	0.0012392516682334\\
90.75	0.0012376535427589\\
90.76	0.00123605536115941\\
90.77	0.00123445712426037\\
90.78	0.00123285883288826\\
90.79	0.00123126048787057\\
90.8	0.00122966209003581\\
90.81	0.00122806364021352\\
90.82	0.00122646513923426\\
90.83	0.0012248665879296\\
90.84	0.00122326798713214\\
90.85	0.00122166933767549\\
90.86	0.00122007064039428\\
90.87	0.00121847189612416\\
90.88	0.00121687310570179\\
90.89	0.00121527426996483\\
90.9	0.00121367538975197\\
90.91	0.00121207646590292\\
90.92	0.00121047749925838\\
90.93	0.00120887849066007\\
90.94	0.00120727944095072\\
90.95	0.00120568035097406\\
90.96	0.00120408122157483\\
90.97	0.00120248205359879\\
90.98	0.00120088284789268\\
90.99	0.00119928360530426\\
91	0.0011976843266823\\
91.01	0.00119608501287654\\
91.02	0.00119448566473775\\
91.03	0.00119288628311769\\
91.04	0.00119128686886911\\
91.05	0.00118968742284577\\
91.06	0.00118808794590243\\
91.07	0.00118648843889482\\
91.08	0.00118488890267968\\
91.09	0.00118328933811475\\
91.1	0.00118168974605874\\
91.11	0.00118009012737137\\
91.12	0.00117849048291334\\
91.13	0.00117689081354632\\
91.14	0.00117529112013301\\
91.15	0.00117369140353706\\
91.16	0.00117209166462309\\
91.17	0.00117049190425676\\
91.18	0.00116889212330465\\
91.19	0.00116729232263435\\
91.2	0.00116569250311443\\
91.21	0.00116409266561444\\
91.22	0.00116249281100487\\
91.23	0.00116089294015724\\
91.24	0.00115929305394399\\
91.25	0.00115769315323857\\
91.26	0.00115609323891539\\
91.27	0.00115449331184981\\
91.28	0.00115289337291818\\
91.29	0.00115129342299782\\
91.3	0.00114969346296698\\
91.31	0.00114809349370492\\
91.32	0.00114649351609183\\
91.33	0.00114489353100886\\
91.34	0.00114329353933814\\
91.35	0.00114169354196274\\
91.36	0.00114009353976669\\
91.37	0.00113849353363498\\
91.38	0.00113689352445353\\
91.39	0.00113529351310925\\
91.4	0.00113369350048996\\
91.41	0.00113209348748446\\
91.42	0.00113049347498248\\
91.43	0.00112889346387469\\
91.44	0.00112729345505272\\
91.45	0.00112569344940911\\
91.46	0.00112409344783739\\
91.47	0.00112249345123199\\
91.48	0.00112089346048828\\
91.49	0.00111929347650257\\
91.5	0.00111769350017211\\
91.51	0.00111609353239508\\
91.52	0.00111449357407057\\
91.53	0.00111289362609863\\
91.54	0.00111129368938021\\
91.55	0.0011096937648172\\
91.56	0.0011080938533124\\
91.57	0.00110649395576954\\
91.58	0.00110489407309326\\
91.59	0.00110329420618914\\
91.6	0.00110169435596364\\
91.61	0.00110009452332416\\
91.62	0.001098494709179\\
91.63	0.00109689491443738\\
91.64	0.00109529514000942\\
91.65	0.00109369538680614\\
91.66	0.00109209565573948\\
91.67	0.00109049594772227\\
91.68	0.00108889626366824\\
91.69	0.00108729660449202\\
91.7	0.00108569697110914\\
91.71	0.00108409736443602\\
91.72	0.00108249778538996\\
91.73	0.00108089823488919\\
91.74	0.00107929871385278\\
91.75	0.00107769922320072\\
91.76	0.00107609976385385\\
91.77	0.00107450033673393\\
91.78	0.00107290094276358\\
91.79	0.0010713015828663\\
91.8	0.00106970225796645\\
91.81	0.00106810296898929\\
91.82	0.00106650371686094\\
91.83	0.00106490450250838\\
91.84	0.00106330532685946\\
91.85	0.00106170619084291\\
91.86	0.0010601070953883\\
91.87	0.00105850804142607\\
91.88	0.00105690902988753\\
91.89	0.00105531006170481\\
91.9	0.00105371113781093\\
91.91	0.00105211225913975\\
91.92	0.00105051342662596\\
91.93	0.00104891464120512\\
91.94	0.00104731590381361\\
91.95	0.00104571721538868\\
91.96	0.00104411857686841\\
91.97	0.00104251998919168\\
91.98	0.00104092145329825\\
91.99	0.0010393229701287\\
92	0.00103772454062443\\
92.01	0.00103612616572766\\
92.02	0.00103452784638146\\
92.03	0.0010329295835297\\
92.04	0.00103133137811706\\
92.05	0.00102973323108907\\
92.06	0.00102813514339205\\
92.07	0.00102653711597312\\
92.08	0.00102493914978024\\
92.09	0.00102334124576216\\
92.1	0.00102174340486843\\
92.11	0.0010201456280494\\
92.12	0.00101854791625624\\
92.13	0.00101695027044089\\
92.14	0.00101535269155609\\
92.15	0.00101375518055538\\
92.16	0.00101215773839308\\
92.17	0.0010105603660243\\
92.18	0.00100896306440491\\
92.19	0.00100736583449161\\
92.2	0.00100576867724181\\
92.21	0.00100417159361375\\
92.22	0.00100257458456642\\
92.23	0.00100097765105957\\
92.24	0.000999380794053732\\
92.25	0.00099778401451019\\
92.26	0.000996187313390989\\
92.27	0.000994590691658935\\
92.28	0.000992994150277586\\
92.29	0.000991397690211244\\
92.3	0.000989801312424973\\
92.31	0.00098820501788457\\
92.32	0.000986608807556582\\
92.33	0.000985012682408293\\
92.34	0.000983416643407715\\
92.35	0.000981820691523611\\
92.36	0.000980224827725454\\
92.37	0.000978629052983454\\
92.38	0.00097703336826855\\
92.39	0.000975437774552383\\
92.4	0.000973842272807333\\
92.41	0.000972246864006482\\
92.42	0.000970651549123623\\
92.43	0.00096905632913326\\
92.44	0.000967461205010603\\
92.45	0.000965866177731556\\
92.46	0.00096427124827273\\
92.47	0.000962676417611429\\
92.48	0.000961081686725644\\
92.49	0.000959487056594053\\
92.5	0.000957892528196029\\
92.51	0.000956298102511608\\
92.52	0.000954703780521522\\
92.53	0.000953109563207167\\
92.54	0.000951515451550618\\
92.55	0.000949921446534607\\
92.56	0.000948327549142534\\
92.57	0.000946733760358467\\
92.58	0.00094514008116712\\
92.59	0.000943546512553867\\
92.6	0.000941953055504732\\
92.61	0.000940359711006379\\
92.62	0.000938766480046115\\
92.63	0.000937173363611899\\
92.64	0.000935580362692307\\
92.65	0.000933987478276561\\
92.66	0.000932394711354494\\
92.67	0.000930802062916588\\
92.68	0.000929209533953918\\
92.69	0.000927617125458196\\
92.7	0.000926024838421731\\
92.71	0.000924432673837452\\
92.72	0.000922840632698893\\
92.73	0.000921248716000179\\
92.74	0.000919656924736039\\
92.75	0.000918065259901799\\
92.76	0.00091647372249336\\
92.77	0.000914882313507228\\
92.78	0.000913291033940467\\
92.79	0.000911699884790741\\
92.8	0.000910108867056275\\
92.81	0.000908517981735866\\
92.82	0.000906927229828868\\
92.83	0.000905336612335201\\
92.84	0.000903746130255354\\
92.85	0.000902155784590344\\
92.86	0.00090056557634176\\
92.87	0.000898975506511719\\
92.88	0.000897385576102883\\
92.89	0.000895795786118448\\
92.9	0.000894206137562139\\
92.91	0.000892616631438218\\
92.92	0.000891027268751461\\
92.93	0.000889438050507155\\
92.94	0.000887848977711119\\
92.95	0.000886260051369664\\
92.96	0.000884671272489613\\
92.97	0.000883082642078287\\
92.98	0.000881494161143507\\
92.99	0.000879905830693582\\
93	0.000878317651737305\\
93.01	0.000876729625283948\\
93.02	0.000875141752343276\\
93.03	0.000873554033925509\\
93.04	0.000871966471041343\\
93.05	0.000870379064701934\\
93.06	0.000868791815918897\\
93.07	0.000867204725704307\\
93.08	0.000865617795070675\\
93.09	0.00086403102503097\\
93.1	0.000862444416598589\\
93.11	0.000860857970787372\\
93.12	0.000859271688611581\\
93.13	0.000857685571085901\\
93.14	0.000856099619225449\\
93.15	0.000854513834045743\\
93.16	0.00085292821656271\\
93.17	0.000851342767792699\\
93.18	0.000849757488752432\\
93.19	0.000848172380459046\\
93.2	0.000846587443930062\\
93.21	0.000845002680183374\\
93.22	0.000843418090237271\\
93.23	0.0008418336751104\\
93.24	0.000840249435821784\\
93.25	0.000838665373390814\\
93.26	0.000837081488837222\\
93.27	0.000835497783181113\\
93.28	0.000833914257442918\\
93.29	0.00083233091264342\\
93.3	0.000830747749803734\\
93.31	0.000829164769945316\\
93.32	0.000827581974089933\\
93.33	0.000825999363259671\\
93.34	0.000824416938476938\\
93.35	0.00082283470076445\\
93.36	0.000821252651145217\\
93.37	0.000819670790642548\\
93.38	0.000818089120280053\\
93.39	0.000816507641081611\\
93.4	0.000814926354071389\\
93.41	0.000813345260273829\\
93.42	0.000811764360713637\\
93.43	0.000810183656415782\\
93.44	0.000808603148405487\\
93.45	0.000807022837708228\\
93.46	0.000805442725349723\\
93.47	0.000803862812355923\\
93.48	0.000802283099753024\\
93.49	0.000800703588567436\\
93.5	0.000799124279825784\\
93.51	0.000797545174554923\\
93.52	0.000795966273781904\\
93.53	0.00079438757853398\\
93.54	0.000792809089838606\\
93.55	0.000791230808723405\\
93.56	0.00078965273621621\\
93.57	0.000788074873345005\\
93.58	0.000786497221137957\\
93.59	0.000784919780623399\\
93.6	0.000783342552829806\\
93.61	0.000781765538785813\\
93.62	0.000780188739520194\\
93.63	0.000778612156061865\\
93.64	0.000777035789439872\\
93.65	0.000775459640683382\\
93.66	0.000773883710821689\\
93.67	0.000772308000884184\\
93.68	0.000770732511900388\\
93.69	0.00076915724489989\\
93.7	0.000767582200912402\\
93.71	0.000766007380967706\\
93.72	0.000764432786095663\\
93.73	0.000762858417326219\\
93.74	0.00076128427568937\\
93.75	0.000759710362215187\\
93.76	0.000758136677933785\\
93.77	0.000756563223875331\\
93.78	0.000754990001070028\\
93.79	0.000753417010548115\\
93.8	0.000751844253339854\\
93.81	0.000750271730475524\\
93.82	0.000748699442985428\\
93.83	0.000747127391899855\\
93.84	0.000745555578249108\\
93.85	0.000743984003063481\\
93.86	0.000742412667373236\\
93.87	0.000740841572208628\\
93.88	0.000739270718599874\\
93.89	0.00073770010757716\\
93.9	0.00073612974017062\\
93.91	0.000734559617410334\\
93.92	0.000732989740326331\\
93.93	0.000731420109948568\\
93.94	0.000729850727306931\\
93.95	0.000728281593431213\\
93.96	0.000726712709351133\\
93.97	0.000725144076096303\\
93.98	0.00072357569469623\\
93.99	0.00072200756618032\\
94	0.000720439691577838\\
94.01	0.000718872071917943\\
94.02	0.000717304708229644\\
94.03	0.000715737601541812\\
94.04	0.000714170752883164\\
94.05	0.000712604163282258\\
94.06	0.000711037833767489\\
94.07	0.00070947176536707\\
94.08	0.000707905959109035\\
94.09	0.000706340416021219\\
94.1	0.000704775137131269\\
94.11	0.000703210123466611\\
94.12	0.000701645376054464\\
94.13	0.000700080895921816\\
94.14	0.000698516684095421\\
94.15	0.000696952741601796\\
94.16	0.000695389069467209\\
94.17	0.000693825668717658\\
94.18	0.000692262540378885\\
94.19	0.000690699685476347\\
94.2	0.000689137105035225\\
94.21	0.000687574800080399\\
94.22	0.000686012771636447\\
94.23	0.000684451020727639\\
94.24	0.000682889548377924\\
94.25	0.000681328355610912\\
94.26	0.000679767443449886\\
94.27	0.000678206812917777\\
94.28	0.000676646465037156\\
94.29	0.000675086400830237\\
94.3	0.000673526621318846\\
94.31	0.000671967127524426\\
94.32	0.000670407920468036\\
94.33	0.000668849001170325\\
94.34	0.000667290370651523\\
94.35	0.000665732029931441\\
94.36	0.000664173980029459\\
94.37	0.000662616221964516\\
94.38	0.00066105875675509\\
94.39	0.000659501585419201\\
94.4	0.000657944708974402\\
94.41	0.000656388128437759\\
94.42	0.000654831844825846\\
94.43	0.000653275859154747\\
94.44	0.000651720172440032\\
94.45	0.000650164785696752\\
94.46	0.000648609699939446\\
94.47	0.000647054916182102\\
94.48	0.000645500435438179\\
94.49	0.000643946258720574\\
94.5	0.000642392387041635\\
94.51	0.00064083882141313\\
94.52	0.000639285562846252\\
94.53	0.000637732612351607\\
94.54	0.000636179970939203\\
94.55	0.00063462763961844\\
94.56	0.000633075619398106\\
94.57	0.000631523911286366\\
94.58	0.000629972516290747\\
94.59	0.000628421435418134\\
94.6	0.000626870669674763\\
94.61	0.000625320220066198\\
94.62	0.000623770087597354\\
94.63	0.000622220273272437\\
94.64	0.000620670778094985\\
94.65	0.000619121603067829\\
94.66	0.000617572749193088\\
94.67	0.000616024217472168\\
94.68	0.000614476008905745\\
94.69	0.000612928124493756\\
94.7	0.000611380565235393\\
94.71	0.000609833332129087\\
94.72	0.000608286426172512\\
94.73	0.000606739848362554\\
94.74	0.00060519359969532\\
94.75	0.000603647681166121\\
94.76	0.000602102093769459\\
94.77	0.000600556838499026\\
94.78	0.000599011916347685\\
94.79	0.000597467328307464\\
94.8	0.000595923075369546\\
94.81	0.000594379158524264\\
94.82	0.000592835578761077\\
94.83	0.000591292337068576\\
94.84	0.000589749434434466\\
94.85	0.000588206871845553\\
94.86	0.000586664650287739\\
94.87	0.000585122770746016\\
94.88	0.000583581234204442\\
94.89	0.000582040041646142\\
94.9	0.000580499194053298\\
94.91	0.000578958692407129\\
94.92	0.000577418537687894\\
94.93	0.000575878730874864\\
94.94	0.000574339272946338\\
94.95	0.000572800164879605\\
94.96	0.000571261407650943\\
94.97	0.000569723002235616\\
94.98	0.000568184949607862\\
94.99	0.000566647250740867\\
95	0.000565109906606775\\
95.01	0.000563572918176667\\
95.02	0.000562036286420544\\
95.03	0.000560500012307329\\
95.04	0.000558964096804856\\
95.05	0.000557428540879845\\
95.06	0.000555893345497901\\
95.07	0.000554358511623511\\
95.08	0.000552824040220007\\
95.09	0.000551289932249587\\
95.1	0.000549756188673286\\
95.11	0.000548222810450959\\
95.12	0.0005466897985413\\
95.13	0.000545157153901783\\
95.14	0.000543624877488698\\
95.15	0.000542092970257107\\
95.16	0.000540561433160851\\
95.17	0.00053903026715253\\
95.18	0.000537499473183505\\
95.19	0.000535969052203858\\
95.2	0.000534439005162413\\
95.21	0.000532909333006707\\
95.22	0.000531380036682979\\
95.23	0.00052985111713616\\
95.24	0.000528322575309869\\
95.25	0.000526794412146389\\
95.26	0.000525266628586669\\
95.27	0.000523739225570299\\
95.28	0.000522212204035503\\
95.29	0.000520685564919136\\
95.3	0.000519159309156655\\
95.31	0.000517633437682128\\
95.32	0.000516107951428204\\
95.33	0.000514582851326105\\
95.34	0.000513058138305626\\
95.35	0.000511533813295105\\
95.36	0.000510009877221435\\
95.37	0.000508486331010022\\
95.38	0.000506963175584792\\
95.39	0.000505440411868183\\
95.4	0.000503918040781115\\
95.41	0.000502396063242995\\
95.42	0.000500874480171692\\
95.43	0.000499353292483537\\
95.44	0.000497832501093297\\
95.45	0.000496312106914174\\
95.46	0.000494792110857787\\
95.47	0.00049327251383416\\
95.48	0.000491753316751718\\
95.49	0.000490234520517253\\
95.5	0.000488716126035937\\
95.51	0.000487198134211294\\
95.52	0.000485680545945189\\
95.53	0.000484163362137824\\
95.54	0.000482646583687714\\
95.55	0.000481130211491676\\
95.56	0.000479614246444831\\
95.57	0.00047809868944057\\
95.58	0.00047658354137055\\
95.59	0.000475068803124692\\
95.6	0.000473554475591146\\
95.61	0.000472040559656297\\
95.62	0.000470527056204746\\
95.63	0.000469013966119293\\
95.64	0.000467501290280932\\
95.65	0.000465989029568825\\
95.66	0.000464477184860302\\
95.67	0.000462965757030845\\
95.68	0.00046145474695407\\
95.69	0.000459944155501711\\
95.7	0.000458433983543621\\
95.71	0.000456924231947747\\
95.72	0.000455414901580114\\
95.73	0.000453905993304821\\
95.74	0.000452397507984027\\
95.75	0.000450889446477927\\
95.76	0.000449381809644749\\
95.77	0.000447874598340739\\
95.78	0.000446367813420142\\
95.79	0.000444861455735189\\
95.8	0.000443355526136094\\
95.81	0.000441850025471027\\
95.82	0.000440344954586105\\
95.83	0.000438840314325386\\
95.84	0.000437336105530832\\
95.85	0.000435832329042332\\
95.86	0.000434328985697651\\
95.87	0.000432826076332436\\
95.88	0.000431323601780206\\
95.89	0.00042982156287232\\
95.9	0.000428319960437981\\
95.91	0.000426818795304209\\
95.92	0.000425318068295833\\
95.93	0.000423817780235479\\
95.94	0.00042231793194355\\
95.95	0.000420818524238213\\
95.96	0.000419319557935392\\
95.97	0.00041782103384874\\
95.98	0.000416322952789641\\
95.99	0.000414825315567179\\
96	0.000413328122988136\\
96.01	0.000411831375856974\\
96.02	0.000410335074975815\\
96.03	0.000408839221144439\\
96.04	0.000407343815160255\\
96.05	0.000405848857818291\\
96.06	0.000404354349911191\\
96.07	0.000402860292229185\\
96.08	0.000401366685560076\\
96.09	0.000399873530689238\\
96.1	0.00039838082839958\\
96.11	0.000396888579471557\\
96.12	0.000395396784683139\\
96.13	0.000393905444809788\\
96.14	0.000392414560624462\\
96.15	0.000390924132897597\\
96.16	0.000389434162397074\\
96.17	0.000387944649888229\\
96.18	0.000386455596133817\\
96.19	0.000384967001894007\\
96.2	0.000383478867926374\\
96.21	0.000381991194985857\\
96.22	0.000380503983824785\\
96.23	0.000379017235192821\\
96.24	0.000377530949836967\\
96.25	0.000376045128501553\\
96.26	0.000374559771928209\\
96.27	0.000373074880855856\\
96.28	0.000371590456020693\\
96.29	0.000370106498156169\\
96.3	0.000368623007992988\\
96.31	0.000367139986259077\\
96.32	0.000365657433679575\\
96.33	0.000364175350976817\\
96.34	0.000362693738870318\\
96.35	0.000361212598076765\\
96.36	0.000359731929309989\\
96.37	0.000358251733280952\\
96.38	0.000356772010697746\\
96.39	0.000355292762265555\\
96.4	0.00035381398868665\\
96.41	0.000352335690660373\\
96.42	0.000350857868883123\\
96.43	0.000349380524048339\\
96.44	0.000347903656846478\\
96.45	0.000346427267965002\\
96.46	0.000344951358088371\\
96.47	0.000343475927898011\\
96.48	0.000342000978072307\\
96.49	0.000340526509286595\\
96.5	0.000339052522213125\\
96.51	0.000337579017521062\\
96.52	0.000336105995876461\\
96.53	0.000334633457942256\\
96.54	0.000333161404378243\\
96.55	0.000331689835841054\\
96.56	0.000330218752984163\\
96.57	0.000328748156457837\\
96.58	0.000327278046909149\\
96.59	0.000325808424981946\\
96.6	0.000324339291316839\\
96.61	0.000322870646551185\\
96.62	0.00032140249131906\\
96.63	0.000319934826251261\\
96.64	0.000318467651975275\\
96.65	0.00031700096911527\\
96.66	0.000315534778292072\\
96.67	0.000314069080123155\\
96.68	0.000312603875222615\\
96.69	0.000311139164201165\\
96.7	0.000309674947666105\\
96.71	0.00030821122622132\\
96.72	0.00030674800046725\\
96.73	0.000305285271000879\\
96.74	0.000303823038415717\\
96.75	0.000302361303301779\\
96.76	0.000300900066245578\\
96.77	0.000299439327830102\\
96.78	0.000297979088634789\\
96.79	0.000296519349235524\\
96.8	0.000295060110204614\\
96.81	0.000293601372110774\\
96.82	0.000292143135519099\\
96.83	0.000290685400991064\\
96.84	0.000289228169084501\\
96.85	0.000287771440353564\\
96.86	0.000286315215348741\\
96.87	0.000284859494616816\\
96.88	0.00028340427870086\\
96.89	0.000281949568140209\\
96.9	0.000280495363470445\\
96.91	0.000279041665223393\\
96.92	0.000277588473927079\\
96.93	0.000276135790105735\\
96.94	0.000274683614279772\\
96.95	0.000273231946965756\\
96.96	0.000271780788676408\\
96.97	0.000270330139920567\\
96.98	0.000268880001203181\\
96.99	0.00026743037302529\\
97	0.000265981255884012\\
97.01	0.000264532650272513\\
97.02	0.000263084556680001\\
97.03	0.000261636975591705\\
97.04	0.000260189907488851\\
97.05	0.00025874335284865\\
97.06	0.000257297312144286\\
97.07	0.000255851785844882\\
97.08	0.000254406774415497\\
97.09	0.000252962278317099\\
97.1	0.000251518298006551\\
97.11	0.000250074833936595\\
97.12	0.000248631886555828\\
97.13	0.000247189456308685\\
97.14	0.000245747543635431\\
97.15	0.000244306148972124\\
97.16	0.000242865272750616\\
97.17	0.000241424915398523\\
97.18	0.000239985077339211\\
97.19	0.000238545758991779\\
97.2	0.000237106960771033\\
97.21	0.00023566868308748\\
97.22	0.000234230926347299\\
97.23	0.000232793690952332\\
97.24	0.000231356977300054\\
97.25	0.000229920785783565\\
97.26	0.000228485116791568\\
97.27	0.000227049970708354\\
97.28	0.000225615347913774\\
97.29	0.000224181248783229\\
97.3	0.000222747673687651\\
97.31	0.000221314622993478\\
97.32	0.000219882097062652\\
97.33	0.000218450096252574\\
97.34	0.000217018620916113\\
97.35	0.000215587671401569\\
97.36	0.00021415724805266\\
97.37	0.000212727351208506\\
97.38	0.000211297981203605\\
97.39	0.000209869138367827\\
97.4	0.000208440823026375\\
97.41	0.000207013035499783\\
97.42	0.000205585776103889\\
97.43	0.000204159045149825\\
97.44	0.00020273284294399\\
97.45	0.00020130716978803\\
97.46	0.000199882025978827\\
97.47	0.000198457411808479\\
97.48	0.000197033327564275\\
97.49	0.000195609773528678\\
97.5	0.000194186749979319\\
97.51	0.000192764257188956\\
97.52	0.000191342295425471\\
97.53	0.000189920864951851\\
97.54	0.000188499966026162\\
97.55	0.000187079598901536\\
97.56	0.000185659763826144\\
97.57	0.000184240461043194\\
97.58	0.000182821690790889\\
97.59	0.00018140345330243\\
97.6	0.000179985748805987\\
97.61	0.000178568577524673\\
97.62	0.000177151939676542\\
97.63	0.000175735835474559\\
97.64	0.000174320265126579\\
97.65	0.000172905228835341\\
97.66	0.000171490726798432\\
97.67	0.000170076759208283\\
97.68	0.000168663326252141\\
97.69	0.000167250428112055\\
97.7	0.000165838064964854\\
97.71	0.000164426236982132\\
97.72	0.000163014944330226\\
97.73	0.00016160418717019\\
97.74	0.0001601939656578\\
97.75	0.000158784279943503\\
97.76	0.000157375130172418\\
97.77	0.000155966516484321\\
97.78	0.000154558439013609\\
97.79	0.000153150897889297\\
97.8	0.000151743893235065\\
97.81	0.000150337425169305\\
97.82	0.000148931493805094\\
97.83	0.000147526099250173\\
97.84	0.000146121241606926\\
97.85	0.000144716920972355\\
97.86	0.000143313137438069\\
97.87	0.000141909891090248\\
97.88	0.000140507182009619\\
97.89	0.000139105010271498\\
97.9	0.000137703375946119\\
97.91	0.000136302279099288\\
97.92	0.000134901719792349\\
97.93	0.000133501697794601\\
97.94	0.000132102212688756\\
97.95	0.000130703264054717\\
97.96	0.000129304851469616\\
97.97	0.000127906974507885\\
97.98	0.000126509632741294\\
97.99	0.000125112825739012\\
98	0.000123716553067654\\
98.01	0.000122320814127108\\
98.02	0.000120925606898312\\
98.03	0.000119530927353886\\
98.04	0.000118136771421792\\
98.05	0.000116743134964492\\
98.06	0.000115350013773462\\
98.07	0.00011395740359667\\
98.08	0.000112565300138151\\
98.09	0.000111173699057597\\
98.1	0.000109783298154441\\
98.11	0.000108394990769643\\
98.12	0.000107008779600988\\
98.13	0.000105624667346543\\
98.14	0.000104242656704808\\
98.15	0.000102862750374865\\
98.16	0.000101485055938421\\
98.17	0.000100110162105732\\
98.18	9.87419653347837e-05\\
98.19	9.73879133488714e-05\\
98.2	9.60481236720965e-05\\
98.21	9.47227150263439e-05\\
98.22	9.34118073461897e-05\\
98.23	9.21155217940544e-05\\
98.24	9.08339807756172e-05\\
98.25	8.95673079554585e-05\\
98.26	8.83156282729721e-05\\
98.27	8.70790679585402e-05\\
98.28	8.58577545499572e-05\\
98.29	8.46518169091651e-05\\
98.3	8.34613852392189e-05\\
98.31	8.22865911015783e-05\\
98.32	8.11275674336698e-05\\
98.33	7.99843285393433e-05\\
98.34	7.88569989661515e-05\\
98.35	7.77443087035621e-05\\
98.36	7.66463885000131e-05\\
98.37	7.55633711730444e-05\\
98.38	7.44953909389782e-05\\
98.39	7.34425834309806e-05\\
98.4	7.240479691842e-05\\
98.41	7.13808484974505e-05\\
98.42	7.03621857074555e-05\\
98.43	6.93488396395318e-05\\
98.44	6.83408415566283e-05\\
98.45	6.73382228926957e-05\\
98.46	6.63410152517627e-05\\
98.47	6.53492504069479e-05\\
98.48	6.43629602993835e-05\\
98.49	6.33821770370628e-05\\
98.5	6.24069328936123e-05\\
98.51	6.14372603069801e-05\\
98.52	6.04731918780309e-05\\
98.53	5.9514760369056e-05\\
98.54	5.85619987021856e-05\\
98.55	5.76149399577237e-05\\
98.56	5.66736175449424e-05\\
98.57	5.57380663403235e-05\\
98.58	5.4808321408719e-05\\
98.59	5.38844180020277e-05\\
98.6	5.29663915577877e-05\\
98.61	5.20542776976576e-05\\
98.62	5.11481122257972e-05\\
98.63	5.02479311374291e-05\\
98.64	4.93537706873076e-05\\
98.65	4.84656673069971e-05\\
98.66	4.75836576027955e-05\\
98.67	4.67077783535674e-05\\
98.68	4.58380665084041e-05\\
98.69	4.49745591842039e-05\\
98.7	4.41172936624448e-05\\
98.71	4.32663073863318e-05\\
98.72	4.24216379579217e-05\\
98.73	4.15833231351189e-05\\
98.74	4.07514016388874e-05\\
98.75	3.9925913233603e-05\\
98.76	3.91068978568884e-05\\
98.77	3.82943956165217e-05\\
98.78	3.74884470218211e-05\\
98.79	3.66890938656855e-05\\
98.8	3.58963901455397e-05\\
98.81	3.51103903651184e-05\\
98.82	3.43311495393447e-05\\
98.83	3.35587231993055e-05\\
98.84	3.27931673972433e-05\\
98.85	3.2034538711612e-05\\
98.86	3.12828942521919e-05\\
98.87	3.05382916652355e-05\\
98.88	2.98007891386867e-05\\
98.89	2.90704454074479e-05\\
98.9	2.83473197586898e-05\\
98.91	2.76314720372376e-05\\
98.92	2.69229626470337e-05\\
98.93	2.62218525559875e-05\\
98.94	2.5528203301298e-05\\
98.95	2.48420769948503e-05\\
98.96	2.41635363286679e-05\\
98.97	2.34926445803856e-05\\
98.98	2.28294656188094e-05\\
98.99	2.21740639095094e-05\\
99	2.15265045204697e-05\\
99.01	2.08868531277918e-05\\
99.02	2.02551760214512e-05\\
99.03	1.96315401111164e-05\\
99.04	1.90160129319994e-05\\
99.05	1.84086626507998e-05\\
99.06	1.78095580716676e-05\\
99.07	1.7218768642243e-05\\
99.08	1.66363644597527e-05\\
99.09	1.60624162771558e-05\\
99.1	1.54969955093523e-05\\
99.11	1.49401742394594e-05\\
99.12	1.43920252251226e-05\\
99.13	1.38526219049216e-05\\
99.14	1.33220384048085e-05\\
99.15	1.28003495446158e-05\\
99.16	1.22876308446349e-05\\
99.17	1.1783958532248e-05\\
99.18	1.12894095486218e-05\\
99.19	1.08040673361641e-05\\
99.2	1.03280183761673e-05\\
99.21	9.86134997679744e-06\\
99.22	9.40415028095366e-06\\
99.23	8.95650827419971e-06\\
99.24	8.51851379275563e-06\\
99.25	8.09025753160421e-06\\
99.26	7.67183105262158e-06\\
99.27	7.26332679283626e-06\\
99.28	6.86483807273673e-06\\
99.29	6.47645910466059e-06\\
99.3	6.09828500128598e-06\\
99.31	5.73041178416993e-06\\
99.32	5.37293639239766e-06\\
99.33	5.02595669130655e-06\\
99.34	4.68957148128807e-06\\
99.35	4.36388050667827e-06\\
99.36	4.04898446473498e-06\\
99.37	3.74498501470345e-06\\
99.38	3.45198478696705e-06\\
99.39	3.17008739210423e-06\\
99.4	2.89939742239079e-06\\
99.41	2.64002046114487e-06\\
99.42	2.39206309212044e-06\\
99.43	2.15563290903095e-06\\
99.44	1.93083852514583e-06\\
99.45	1.71778958298931e-06\\
99.46	1.51659676413639e-06\\
99.47	1.32737179910601e-06\\
99.48	1.15022747734263e-06\\
99.49	9.85277657314029e-07\\
99.5	8.32637276701118e-07\\
99.51	6.92422362690015e-07\\
99.52	5.64750042368264e-07\\
99.53	4.49738553228579e-07\\
99.54	3.47507253785351e-07\\
99.55	2.58176634277893e-07\\
99.56	1.81868327514198e-07\\
99.57	1.18705119791021e-07\\
99.58	6.88109619700894e-08\\
99.59	3.23109806184968e-08\\
99.6	9.33148930348793e-09\\
99.61	0\\
99.62	0\\
99.63	0\\
99.64	0\\
99.65	0\\
99.66	0\\
99.67	0\\
99.68	0\\
99.69	0\\
99.7	0\\
99.71	0\\
99.72	0\\
99.73	0\\
99.74	0\\
99.75	0\\
99.76	0\\
99.77	0\\
99.78	0\\
99.79	0\\
99.8	0\\
99.81	0\\
99.82	0\\
99.83	0\\
99.84	0\\
99.85	0\\
99.86	0\\
99.87	0\\
99.88	0\\
99.89	0\\
99.9	0\\
99.91	0\\
99.92	0\\
99.93	0\\
99.94	0\\
99.95	0\\
99.96	0\\
99.97	0\\
99.98	0\\
99.99	0\\
100	0\\
};
\addlegendentry{$q=-3$};

\addplot [color=red,dashed,forget plot]
  table[row sep=crcr]{%
0.01	0.01\\
0.02	0.01\\
0.03	0.01\\
0.04	0.01\\
0.05	0.01\\
0.06	0.01\\
0.07	0.01\\
0.08	0.01\\
0.09	0.01\\
0.1	0.01\\
0.11	0.01\\
0.12	0.01\\
0.13	0.01\\
0.14	0.01\\
0.15	0.01\\
0.16	0.01\\
0.17	0.01\\
0.18	0.01\\
0.19	0.01\\
0.2	0.01\\
0.21	0.01\\
0.22	0.01\\
0.23	0.01\\
0.24	0.01\\
0.25	0.01\\
0.26	0.01\\
0.27	0.01\\
0.28	0.01\\
0.29	0.01\\
0.3	0.01\\
0.31	0.01\\
0.32	0.01\\
0.33	0.01\\
0.34	0.01\\
0.35	0.01\\
0.36	0.01\\
0.37	0.01\\
0.38	0.01\\
0.39	0.01\\
0.4	0.01\\
0.41	0.01\\
0.42	0.01\\
0.43	0.01\\
0.44	0.01\\
0.45	0.01\\
0.46	0.01\\
0.47	0.01\\
0.48	0.01\\
0.49	0.01\\
0.5	0.01\\
0.51	0.01\\
0.52	0.01\\
0.53	0.01\\
0.54	0.01\\
0.55	0.01\\
0.56	0.01\\
0.57	0.01\\
0.58	0.01\\
0.59	0.01\\
0.6	0.01\\
0.61	0.01\\
0.62	0.01\\
0.63	0.01\\
0.64	0.01\\
0.65	0.01\\
0.66	0.01\\
0.67	0.01\\
0.68	0.01\\
0.69	0.01\\
0.7	0.01\\
0.71	0.01\\
0.72	0.01\\
0.73	0.01\\
0.74	0.01\\
0.75	0.01\\
0.76	0.01\\
0.77	0.01\\
0.78	0.01\\
0.79	0.01\\
0.8	0.01\\
0.81	0.01\\
0.82	0.01\\
0.83	0.01\\
0.84	0.01\\
0.85	0.01\\
0.86	0.01\\
0.87	0.01\\
0.88	0.01\\
0.89	0.01\\
0.9	0.01\\
0.91	0.01\\
0.92	0.01\\
0.93	0.01\\
0.94	0.01\\
0.95	0.01\\
0.96	0.01\\
0.97	0.01\\
0.98	0.01\\
0.99	0.01\\
1	0.01\\
1.01	0.01\\
1.02	0.01\\
1.03	0.01\\
1.04	0.01\\
1.05	0.01\\
1.06	0.01\\
1.07	0.01\\
1.08	0.01\\
1.09	0.01\\
1.1	0.01\\
1.11	0.01\\
1.12	0.01\\
1.13	0.01\\
1.14	0.01\\
1.15	0.01\\
1.16	0.01\\
1.17	0.01\\
1.18	0.01\\
1.19	0.01\\
1.2	0.01\\
1.21	0.01\\
1.22	0.01\\
1.23	0.01\\
1.24	0.01\\
1.25	0.01\\
1.26	0.01\\
1.27	0.01\\
1.28	0.01\\
1.29	0.01\\
1.3	0.01\\
1.31	0.01\\
1.32	0.01\\
1.33	0.01\\
1.34	0.01\\
1.35	0.01\\
1.36	0.01\\
1.37	0.01\\
1.38	0.01\\
1.39	0.01\\
1.4	0.01\\
1.41	0.01\\
1.42	0.01\\
1.43	0.01\\
1.44	0.01\\
1.45	0.01\\
1.46	0.01\\
1.47	0.01\\
1.48	0.01\\
1.49	0.01\\
1.5	0.01\\
1.51	0.01\\
1.52	0.01\\
1.53	0.01\\
1.54	0.01\\
1.55	0.01\\
1.56	0.01\\
1.57	0.01\\
1.58	0.01\\
1.59	0.01\\
1.6	0.01\\
1.61	0.01\\
1.62	0.01\\
1.63	0.01\\
1.64	0.01\\
1.65	0.01\\
1.66	0.01\\
1.67	0.01\\
1.68	0.01\\
1.69	0.01\\
1.7	0.01\\
1.71	0.01\\
1.72	0.01\\
1.73	0.01\\
1.74	0.01\\
1.75	0.01\\
1.76	0.01\\
1.77	0.01\\
1.78	0.01\\
1.79	0.01\\
1.8	0.01\\
1.81	0.01\\
1.82	0.01\\
1.83	0.01\\
1.84	0.01\\
1.85	0.01\\
1.86	0.01\\
1.87	0.01\\
1.88	0.01\\
1.89	0.01\\
1.9	0.01\\
1.91	0.01\\
1.92	0.01\\
1.93	0.01\\
1.94	0.01\\
1.95	0.01\\
1.96	0.01\\
1.97	0.01\\
1.98	0.01\\
1.99	0.01\\
2	0.01\\
2.01	0.01\\
2.02	0.01\\
2.03	0.01\\
2.04	0.01\\
2.05	0.01\\
2.06	0.01\\
2.07	0.01\\
2.08	0.01\\
2.09	0.01\\
2.1	0.01\\
2.11	0.01\\
2.12	0.01\\
2.13	0.01\\
2.14	0.01\\
2.15	0.01\\
2.16	0.01\\
2.17	0.01\\
2.18	0.01\\
2.19	0.01\\
2.2	0.01\\
2.21	0.01\\
2.22	0.01\\
2.23	0.01\\
2.24	0.01\\
2.25	0.01\\
2.26	0.01\\
2.27	0.01\\
2.28	0.01\\
2.29	0.01\\
2.3	0.01\\
2.31	0.01\\
2.32	0.01\\
2.33	0.01\\
2.34	0.01\\
2.35	0.01\\
2.36	0.01\\
2.37	0.01\\
2.38	0.01\\
2.39	0.01\\
2.4	0.01\\
2.41	0.01\\
2.42	0.01\\
2.43	0.01\\
2.44	0.01\\
2.45	0.01\\
2.46	0.01\\
2.47	0.01\\
2.48	0.01\\
2.49	0.01\\
2.5	0.01\\
2.51	0.01\\
2.52	0.01\\
2.53	0.01\\
2.54	0.01\\
2.55	0.01\\
2.56	0.01\\
2.57	0.01\\
2.58	0.01\\
2.59	0.01\\
2.6	0.01\\
2.61	0.01\\
2.62	0.01\\
2.63	0.01\\
2.64	0.01\\
2.65	0.01\\
2.66	0.01\\
2.67	0.01\\
2.68	0.01\\
2.69	0.01\\
2.7	0.01\\
2.71	0.01\\
2.72	0.01\\
2.73	0.01\\
2.74	0.01\\
2.75	0.01\\
2.76	0.01\\
2.77	0.01\\
2.78	0.01\\
2.79	0.01\\
2.8	0.01\\
2.81	0.01\\
2.82	0.01\\
2.83	0.01\\
2.84	0.01\\
2.85	0.01\\
2.86	0.01\\
2.87	0.01\\
2.88	0.01\\
2.89	0.01\\
2.9	0.01\\
2.91	0.01\\
2.92	0.01\\
2.93	0.01\\
2.94	0.01\\
2.95	0.01\\
2.96	0.01\\
2.97	0.01\\
2.98	0.01\\
2.99	0.01\\
3	0.01\\
3.01	0.01\\
3.02	0.01\\
3.03	0.01\\
3.04	0.01\\
3.05	0.01\\
3.06	0.01\\
3.07	0.01\\
3.08	0.01\\
3.09	0.01\\
3.1	0.01\\
3.11	0.01\\
3.12	0.01\\
3.13	0.01\\
3.14	0.01\\
3.15	0.01\\
3.16	0.01\\
3.17	0.01\\
3.18	0.01\\
3.19	0.01\\
3.2	0.01\\
3.21	0.01\\
3.22	0.01\\
3.23	0.01\\
3.24	0.01\\
3.25	0.01\\
3.26	0.01\\
3.27	0.01\\
3.28	0.01\\
3.29	0.01\\
3.3	0.01\\
3.31	0.01\\
3.32	0.01\\
3.33	0.01\\
3.34	0.01\\
3.35	0.01\\
3.36	0.01\\
3.37	0.01\\
3.38	0.01\\
3.39	0.01\\
3.4	0.01\\
3.41	0.01\\
3.42	0.01\\
3.43	0.01\\
3.44	0.01\\
3.45	0.01\\
3.46	0.01\\
3.47	0.01\\
3.48	0.01\\
3.49	0.01\\
3.5	0.01\\
3.51	0.01\\
3.52	0.01\\
3.53	0.01\\
3.54	0.01\\
3.55	0.01\\
3.56	0.01\\
3.57	0.01\\
3.58	0.01\\
3.59	0.01\\
3.6	0.01\\
3.61	0.01\\
3.62	0.01\\
3.63	0.01\\
3.64	0.01\\
3.65	0.01\\
3.66	0.01\\
3.67	0.01\\
3.68	0.01\\
3.69	0.01\\
3.7	0.01\\
3.71	0.01\\
3.72	0.01\\
3.73	0.01\\
3.74	0.01\\
3.75	0.01\\
3.76	0.01\\
3.77	0.01\\
3.78	0.01\\
3.79	0.01\\
3.8	0.01\\
3.81	0.01\\
3.82	0.01\\
3.83	0.01\\
3.84	0.01\\
3.85	0.01\\
3.86	0.01\\
3.87	0.01\\
3.88	0.01\\
3.89	0.01\\
3.9	0.01\\
3.91	0.01\\
3.92	0.01\\
3.93	0.01\\
3.94	0.01\\
3.95	0.01\\
3.96	0.01\\
3.97	0.01\\
3.98	0.01\\
3.99	0.01\\
4	0.01\\
4.01	0.01\\
4.02	0.01\\
4.03	0.01\\
4.04	0.01\\
4.05	0.01\\
4.06	0.01\\
4.07	0.01\\
4.08	0.01\\
4.09	0.01\\
4.1	0.01\\
4.11	0.01\\
4.12	0.01\\
4.13	0.01\\
4.14	0.01\\
4.15	0.01\\
4.16	0.01\\
4.17	0.01\\
4.18	0.01\\
4.19	0.01\\
4.2	0.01\\
4.21	0.01\\
4.22	0.01\\
4.23	0.01\\
4.24	0.01\\
4.25	0.01\\
4.26	0.01\\
4.27	0.01\\
4.28	0.01\\
4.29	0.01\\
4.3	0.01\\
4.31	0.01\\
4.32	0.01\\
4.33	0.01\\
4.34	0.01\\
4.35	0.01\\
4.36	0.01\\
4.37	0.01\\
4.38	0.01\\
4.39	0.01\\
4.4	0.01\\
4.41	0.01\\
4.42	0.01\\
4.43	0.01\\
4.44	0.01\\
4.45	0.01\\
4.46	0.01\\
4.47	0.01\\
4.48	0.01\\
4.49	0.01\\
4.5	0.01\\
4.51	0.01\\
4.52	0.01\\
4.53	0.01\\
4.54	0.01\\
4.55	0.01\\
4.56	0.01\\
4.57	0.01\\
4.58	0.01\\
4.59	0.01\\
4.6	0.01\\
4.61	0.01\\
4.62	0.01\\
4.63	0.01\\
4.64	0.01\\
4.65	0.01\\
4.66	0.01\\
4.67	0.01\\
4.68	0.01\\
4.69	0.01\\
4.7	0.01\\
4.71	0.01\\
4.72	0.01\\
4.73	0.01\\
4.74	0.01\\
4.75	0.01\\
4.76	0.01\\
4.77	0.01\\
4.78	0.01\\
4.79	0.01\\
4.8	0.01\\
4.81	0.01\\
4.82	0.01\\
4.83	0.01\\
4.84	0.01\\
4.85	0.01\\
4.86	0.01\\
4.87	0.01\\
4.88	0.01\\
4.89	0.01\\
4.9	0.01\\
4.91	0.01\\
4.92	0.01\\
4.93	0.01\\
4.94	0.01\\
4.95	0.01\\
4.96	0.01\\
4.97	0.01\\
4.98	0.01\\
4.99	0.01\\
5	0.01\\
5.01	0.01\\
5.02	0.01\\
5.03	0.01\\
5.04	0.01\\
5.05	0.01\\
5.06	0.01\\
5.07	0.01\\
5.08	0.01\\
5.09	0.01\\
5.1	0.01\\
5.11	0.01\\
5.12	0.01\\
5.13	0.01\\
5.14	0.01\\
5.15	0.01\\
5.16	0.01\\
5.17	0.01\\
5.18	0.01\\
5.19	0.01\\
5.2	0.01\\
5.21	0.01\\
5.22	0.01\\
5.23	0.01\\
5.24	0.01\\
5.25	0.01\\
5.26	0.01\\
5.27	0.01\\
5.28	0.01\\
5.29	0.01\\
5.3	0.01\\
5.31	0.01\\
5.32	0.01\\
5.33	0.01\\
5.34	0.01\\
5.35	0.01\\
5.36	0.01\\
5.37	0.01\\
5.38	0.01\\
5.39	0.01\\
5.4	0.01\\
5.41	0.01\\
5.42	0.01\\
5.43	0.01\\
5.44	0.01\\
5.45	0.01\\
5.46	0.01\\
5.47	0.01\\
5.48	0.01\\
5.49	0.01\\
5.5	0.01\\
5.51	0.01\\
5.52	0.01\\
5.53	0.01\\
5.54	0.01\\
5.55	0.01\\
5.56	0.01\\
5.57	0.01\\
5.58	0.01\\
5.59	0.01\\
5.6	0.01\\
5.61	0.01\\
5.62	0.01\\
5.63	0.01\\
5.64	0.01\\
5.65	0.01\\
5.66	0.01\\
5.67	0.01\\
5.68	0.01\\
5.69	0.01\\
5.7	0.01\\
5.71	0.01\\
5.72	0.01\\
5.73	0.01\\
5.74	0.01\\
5.75	0.01\\
5.76	0.01\\
5.77	0.01\\
5.78	0.01\\
5.79	0.01\\
5.8	0.01\\
5.81	0.01\\
5.82	0.01\\
5.83	0.01\\
5.84	0.01\\
5.85	0.01\\
5.86	0.01\\
5.87	0.01\\
5.88	0.01\\
5.89	0.01\\
5.9	0.01\\
5.91	0.01\\
5.92	0.01\\
5.93	0.01\\
5.94	0.01\\
5.95	0.01\\
5.96	0.01\\
5.97	0.01\\
5.98	0.01\\
5.99	0.01\\
6	0.01\\
6.01	0.01\\
6.02	0.01\\
6.03	0.01\\
6.04	0.01\\
6.05	0.01\\
6.06	0.01\\
6.07	0.01\\
6.08	0.01\\
6.09	0.01\\
6.1	0.01\\
6.11	0.01\\
6.12	0.01\\
6.13	0.01\\
6.14	0.01\\
6.15	0.01\\
6.16	0.01\\
6.17	0.01\\
6.18	0.01\\
6.19	0.01\\
6.2	0.01\\
6.21	0.01\\
6.22	0.01\\
6.23	0.01\\
6.24	0.01\\
6.25	0.01\\
6.26	0.01\\
6.27	0.01\\
6.28	0.01\\
6.29	0.01\\
6.3	0.01\\
6.31	0.01\\
6.32	0.01\\
6.33	0.01\\
6.34	0.01\\
6.35	0.01\\
6.36	0.01\\
6.37	0.01\\
6.38	0.01\\
6.39	0.01\\
6.4	0.01\\
6.41	0.01\\
6.42	0.01\\
6.43	0.01\\
6.44	0.01\\
6.45	0.01\\
6.46	0.01\\
6.47	0.01\\
6.48	0.01\\
6.49	0.01\\
6.5	0.01\\
6.51	0.01\\
6.52	0.01\\
6.53	0.01\\
6.54	0.01\\
6.55	0.01\\
6.56	0.01\\
6.57	0.01\\
6.58	0.01\\
6.59	0.01\\
6.6	0.01\\
6.61	0.01\\
6.62	0.01\\
6.63	0.01\\
6.64	0.01\\
6.65	0.01\\
6.66	0.01\\
6.67	0.01\\
6.68	0.01\\
6.69	0.01\\
6.7	0.01\\
6.71	0.01\\
6.72	0.01\\
6.73	0.01\\
6.74	0.01\\
6.75	0.01\\
6.76	0.01\\
6.77	0.01\\
6.78	0.01\\
6.79	0.01\\
6.8	0.01\\
6.81	0.01\\
6.82	0.01\\
6.83	0.01\\
6.84	0.01\\
6.85	0.01\\
6.86	0.01\\
6.87	0.01\\
6.88	0.01\\
6.89	0.01\\
6.9	0.01\\
6.91	0.01\\
6.92	0.01\\
6.93	0.01\\
6.94	0.01\\
6.95	0.01\\
6.96	0.01\\
6.97	0.01\\
6.98	0.01\\
6.99	0.01\\
7	0.01\\
7.01	0.01\\
7.02	0.01\\
7.03	0.01\\
7.04	0.01\\
7.05	0.01\\
7.06	0.01\\
7.07	0.01\\
7.08	0.01\\
7.09	0.01\\
7.1	0.01\\
7.11	0.01\\
7.12	0.01\\
7.13	0.01\\
7.14	0.01\\
7.15	0.01\\
7.16	0.01\\
7.17	0.01\\
7.18	0.01\\
7.19	0.01\\
7.2	0.01\\
7.21	0.01\\
7.22	0.01\\
7.23	0.01\\
7.24	0.01\\
7.25	0.01\\
7.26	0.01\\
7.27	0.01\\
7.28	0.01\\
7.29	0.01\\
7.3	0.01\\
7.31	0.01\\
7.32	0.01\\
7.33	0.01\\
7.34	0.01\\
7.35	0.01\\
7.36	0.01\\
7.37	0.01\\
7.38	0.01\\
7.39	0.01\\
7.4	0.01\\
7.41	0.01\\
7.42	0.01\\
7.43	0.01\\
7.44	0.01\\
7.45	0.01\\
7.46	0.01\\
7.47	0.01\\
7.48	0.01\\
7.49	0.01\\
7.5	0.01\\
7.51	0.01\\
7.52	0.01\\
7.53	0.01\\
7.54	0.01\\
7.55	0.01\\
7.56	0.01\\
7.57	0.01\\
7.58	0.01\\
7.59	0.01\\
7.6	0.01\\
7.61	0.01\\
7.62	0.01\\
7.63	0.01\\
7.64	0.01\\
7.65	0.01\\
7.66	0.01\\
7.67	0.01\\
7.68	0.01\\
7.69	0.01\\
7.7	0.01\\
7.71	0.01\\
7.72	0.01\\
7.73	0.01\\
7.74	0.01\\
7.75	0.01\\
7.76	0.01\\
7.77	0.01\\
7.78	0.01\\
7.79	0.01\\
7.8	0.01\\
7.81	0.01\\
7.82	0.01\\
7.83	0.01\\
7.84	0.01\\
7.85	0.01\\
7.86	0.01\\
7.87	0.01\\
7.88	0.01\\
7.89	0.01\\
7.9	0.01\\
7.91	0.01\\
7.92	0.01\\
7.93	0.01\\
7.94	0.01\\
7.95	0.01\\
7.96	0.01\\
7.97	0.01\\
7.98	0.01\\
7.99	0.01\\
8	0.01\\
8.01	0.01\\
8.02	0.01\\
8.03	0.01\\
8.04	0.01\\
8.05	0.01\\
8.06	0.01\\
8.07	0.01\\
8.08	0.01\\
8.09	0.01\\
8.1	0.01\\
8.11	0.01\\
8.12	0.01\\
8.13	0.01\\
8.14	0.01\\
8.15	0.01\\
8.16	0.01\\
8.17	0.01\\
8.18	0.01\\
8.19	0.01\\
8.2	0.01\\
8.21	0.01\\
8.22	0.01\\
8.23	0.01\\
8.24	0.01\\
8.25	0.01\\
8.26	0.01\\
8.27	0.01\\
8.28	0.01\\
8.29	0.01\\
8.3	0.01\\
8.31	0.01\\
8.32	0.01\\
8.33	0.01\\
8.34	0.01\\
8.35	0.01\\
8.36	0.01\\
8.37	0.01\\
8.38	0.01\\
8.39	0.01\\
8.4	0.01\\
8.41	0.01\\
8.42	0.01\\
8.43	0.01\\
8.44	0.01\\
8.45	0.01\\
8.46	0.01\\
8.47	0.01\\
8.48	0.01\\
8.49	0.01\\
8.5	0.01\\
8.51	0.01\\
8.52	0.01\\
8.53	0.01\\
8.54	0.01\\
8.55	0.01\\
8.56	0.01\\
8.57	0.01\\
8.58	0.01\\
8.59	0.01\\
8.6	0.01\\
8.61	0.01\\
8.62	0.01\\
8.63	0.01\\
8.64	0.01\\
8.65	0.01\\
8.66	0.01\\
8.67	0.01\\
8.68	0.01\\
8.69	0.01\\
8.7	0.01\\
8.71	0.01\\
8.72	0.01\\
8.73	0.01\\
8.74	0.01\\
8.75	0.01\\
8.76	0.01\\
8.77	0.01\\
8.78	0.01\\
8.79	0.01\\
8.8	0.01\\
8.81	0.01\\
8.82	0.01\\
8.83	0.01\\
8.84	0.01\\
8.85	0.01\\
8.86	0.01\\
8.87	0.01\\
8.88	0.01\\
8.89	0.01\\
8.9	0.01\\
8.91	0.01\\
8.92	0.01\\
8.93	0.01\\
8.94	0.01\\
8.95	0.01\\
8.96	0.01\\
8.97	0.01\\
8.98	0.01\\
8.99	0.01\\
9	0.01\\
9.01	0.01\\
9.02	0.01\\
9.03	0.01\\
9.04	0.01\\
9.05	0.01\\
9.06	0.01\\
9.07	0.01\\
9.08	0.01\\
9.09	0.01\\
9.1	0.01\\
9.11	0.01\\
9.12	0.01\\
9.13	0.01\\
9.14	0.01\\
9.15	0.01\\
9.16	0.01\\
9.17	0.01\\
9.18	0.01\\
9.19	0.01\\
9.2	0.01\\
9.21	0.01\\
9.22	0.01\\
9.23	0.01\\
9.24	0.01\\
9.25	0.01\\
9.26	0.01\\
9.27	0.01\\
9.28	0.01\\
9.29	0.01\\
9.3	0.01\\
9.31	0.01\\
9.32	0.01\\
9.33	0.01\\
9.34	0.01\\
9.35	0.01\\
9.36	0.01\\
9.37	0.01\\
9.38	0.01\\
9.39	0.01\\
9.4	0.01\\
9.41	0.01\\
9.42	0.01\\
9.43	0.01\\
9.44	0.01\\
9.45	0.01\\
9.46	0.01\\
9.47	0.01\\
9.48	0.01\\
9.49	0.01\\
9.5	0.01\\
9.51	0.01\\
9.52	0.01\\
9.53	0.01\\
9.54	0.01\\
9.55	0.01\\
9.56	0.01\\
9.57	0.01\\
9.58	0.01\\
9.59	0.01\\
9.6	0.01\\
9.61	0.01\\
9.62	0.01\\
9.63	0.01\\
9.64	0.01\\
9.65	0.01\\
9.66	0.01\\
9.67	0.01\\
9.68	0.01\\
9.69	0.01\\
9.7	0.01\\
9.71	0.01\\
9.72	0.01\\
9.73	0.01\\
9.74	0.01\\
9.75	0.01\\
9.76	0.01\\
9.77	0.01\\
9.78	0.01\\
9.79	0.01\\
9.8	0.01\\
9.81	0.01\\
9.82	0.01\\
9.83	0.01\\
9.84	0.01\\
9.85	0.01\\
9.86	0.01\\
9.87	0.01\\
9.88	0.01\\
9.89	0.01\\
9.9	0.01\\
9.91	0.01\\
9.92	0.01\\
9.93	0.01\\
9.94	0.01\\
9.95	0.01\\
9.96	0.01\\
9.97	0.01\\
9.98	0.01\\
9.99	0.01\\
10	0.01\\
10.01	0.01\\
10.02	0.01\\
10.03	0.01\\
10.04	0.01\\
10.05	0.01\\
10.06	0.01\\
10.07	0.01\\
10.08	0.01\\
10.09	0.01\\
10.1	0.01\\
10.11	0.01\\
10.12	0.01\\
10.13	0.01\\
10.14	0.01\\
10.15	0.01\\
10.16	0.01\\
10.17	0.01\\
10.18	0.01\\
10.19	0.01\\
10.2	0.01\\
10.21	0.01\\
10.22	0.01\\
10.23	0.01\\
10.24	0.01\\
10.25	0.01\\
10.26	0.01\\
10.27	0.01\\
10.28	0.01\\
10.29	0.01\\
10.3	0.01\\
10.31	0.01\\
10.32	0.01\\
10.33	0.01\\
10.34	0.01\\
10.35	0.01\\
10.36	0.01\\
10.37	0.01\\
10.38	0.01\\
10.39	0.01\\
10.4	0.01\\
10.41	0.01\\
10.42	0.01\\
10.43	0.01\\
10.44	0.01\\
10.45	0.01\\
10.46	0.01\\
10.47	0.01\\
10.48	0.01\\
10.49	0.01\\
10.5	0.01\\
10.51	0.01\\
10.52	0.01\\
10.53	0.01\\
10.54	0.01\\
10.55	0.01\\
10.56	0.01\\
10.57	0.01\\
10.58	0.01\\
10.59	0.01\\
10.6	0.01\\
10.61	0.01\\
10.62	0.01\\
10.63	0.01\\
10.64	0.01\\
10.65	0.01\\
10.66	0.01\\
10.67	0.01\\
10.68	0.01\\
10.69	0.01\\
10.7	0.01\\
10.71	0.01\\
10.72	0.01\\
10.73	0.01\\
10.74	0.01\\
10.75	0.01\\
10.76	0.01\\
10.77	0.01\\
10.78	0.01\\
10.79	0.01\\
10.8	0.01\\
10.81	0.01\\
10.82	0.01\\
10.83	0.01\\
10.84	0.01\\
10.85	0.01\\
10.86	0.01\\
10.87	0.01\\
10.88	0.01\\
10.89	0.01\\
10.9	0.01\\
10.91	0.01\\
10.92	0.01\\
10.93	0.01\\
10.94	0.01\\
10.95	0.01\\
10.96	0.01\\
10.97	0.01\\
10.98	0.01\\
10.99	0.01\\
11	0.01\\
11.01	0.01\\
11.02	0.01\\
11.03	0.01\\
11.04	0.01\\
11.05	0.01\\
11.06	0.01\\
11.07	0.01\\
11.08	0.01\\
11.09	0.01\\
11.1	0.01\\
11.11	0.01\\
11.12	0.01\\
11.13	0.01\\
11.14	0.01\\
11.15	0.01\\
11.16	0.01\\
11.17	0.01\\
11.18	0.01\\
11.19	0.01\\
11.2	0.01\\
11.21	0.01\\
11.22	0.01\\
11.23	0.01\\
11.24	0.01\\
11.25	0.01\\
11.26	0.01\\
11.27	0.01\\
11.28	0.01\\
11.29	0.01\\
11.3	0.01\\
11.31	0.01\\
11.32	0.01\\
11.33	0.01\\
11.34	0.01\\
11.35	0.01\\
11.36	0.01\\
11.37	0.01\\
11.38	0.01\\
11.39	0.01\\
11.4	0.01\\
11.41	0.01\\
11.42	0.01\\
11.43	0.01\\
11.44	0.01\\
11.45	0.01\\
11.46	0.01\\
11.47	0.01\\
11.48	0.01\\
11.49	0.01\\
11.5	0.01\\
11.51	0.01\\
11.52	0.01\\
11.53	0.01\\
11.54	0.01\\
11.55	0.01\\
11.56	0.01\\
11.57	0.01\\
11.58	0.01\\
11.59	0.01\\
11.6	0.01\\
11.61	0.01\\
11.62	0.01\\
11.63	0.01\\
11.64	0.01\\
11.65	0.01\\
11.66	0.01\\
11.67	0.01\\
11.68	0.01\\
11.69	0.01\\
11.7	0.01\\
11.71	0.01\\
11.72	0.01\\
11.73	0.01\\
11.74	0.01\\
11.75	0.01\\
11.76	0.01\\
11.77	0.01\\
11.78	0.01\\
11.79	0.01\\
11.8	0.01\\
11.81	0.01\\
11.82	0.01\\
11.83	0.01\\
11.84	0.01\\
11.85	0.01\\
11.86	0.01\\
11.87	0.01\\
11.88	0.01\\
11.89	0.01\\
11.9	0.01\\
11.91	0.01\\
11.92	0.01\\
11.93	0.01\\
11.94	0.01\\
11.95	0.01\\
11.96	0.01\\
11.97	0.01\\
11.98	0.01\\
11.99	0.01\\
12	0.01\\
12.01	0.01\\
12.02	0.01\\
12.03	0.01\\
12.04	0.01\\
12.05	0.01\\
12.06	0.01\\
12.07	0.01\\
12.08	0.01\\
12.09	0.01\\
12.1	0.01\\
12.11	0.01\\
12.12	0.01\\
12.13	0.01\\
12.14	0.01\\
12.15	0.01\\
12.16	0.01\\
12.17	0.01\\
12.18	0.01\\
12.19	0.01\\
12.2	0.01\\
12.21	0.01\\
12.22	0.01\\
12.23	0.01\\
12.24	0.01\\
12.25	0.01\\
12.26	0.01\\
12.27	0.01\\
12.28	0.01\\
12.29	0.01\\
12.3	0.01\\
12.31	0.01\\
12.32	0.01\\
12.33	0.01\\
12.34	0.01\\
12.35	0.01\\
12.36	0.01\\
12.37	0.01\\
12.38	0.01\\
12.39	0.01\\
12.4	0.01\\
12.41	0.01\\
12.42	0.01\\
12.43	0.01\\
12.44	0.01\\
12.45	0.01\\
12.46	0.01\\
12.47	0.01\\
12.48	0.01\\
12.49	0.01\\
12.5	0.01\\
12.51	0.01\\
12.52	0.01\\
12.53	0.01\\
12.54	0.01\\
12.55	0.01\\
12.56	0.01\\
12.57	0.01\\
12.58	0.01\\
12.59	0.01\\
12.6	0.01\\
12.61	0.01\\
12.62	0.01\\
12.63	0.01\\
12.64	0.01\\
12.65	0.01\\
12.66	0.01\\
12.67	0.01\\
12.68	0.01\\
12.69	0.01\\
12.7	0.01\\
12.71	0.01\\
12.72	0.01\\
12.73	0.01\\
12.74	0.01\\
12.75	0.01\\
12.76	0.01\\
12.77	0.01\\
12.78	0.01\\
12.79	0.01\\
12.8	0.01\\
12.81	0.01\\
12.82	0.01\\
12.83	0.01\\
12.84	0.01\\
12.85	0.01\\
12.86	0.01\\
12.87	0.01\\
12.88	0.01\\
12.89	0.01\\
12.9	0.01\\
12.91	0.01\\
12.92	0.01\\
12.93	0.01\\
12.94	0.01\\
12.95	0.01\\
12.96	0.01\\
12.97	0.01\\
12.98	0.01\\
12.99	0.01\\
13	0.01\\
13.01	0.01\\
13.02	0.01\\
13.03	0.01\\
13.04	0.01\\
13.05	0.01\\
13.06	0.01\\
13.07	0.01\\
13.08	0.01\\
13.09	0.01\\
13.1	0.01\\
13.11	0.01\\
13.12	0.01\\
13.13	0.01\\
13.14	0.01\\
13.15	0.01\\
13.16	0.01\\
13.17	0.01\\
13.18	0.01\\
13.19	0.01\\
13.2	0.01\\
13.21	0.01\\
13.22	0.01\\
13.23	0.01\\
13.24	0.01\\
13.25	0.01\\
13.26	0.01\\
13.27	0.01\\
13.28	0.01\\
13.29	0.01\\
13.3	0.01\\
13.31	0.01\\
13.32	0.01\\
13.33	0.01\\
13.34	0.01\\
13.35	0.01\\
13.36	0.01\\
13.37	0.01\\
13.38	0.01\\
13.39	0.01\\
13.4	0.01\\
13.41	0.01\\
13.42	0.01\\
13.43	0.01\\
13.44	0.01\\
13.45	0.01\\
13.46	0.01\\
13.47	0.01\\
13.48	0.01\\
13.49	0.01\\
13.5	0.01\\
13.51	0.01\\
13.52	0.01\\
13.53	0.01\\
13.54	0.01\\
13.55	0.01\\
13.56	0.01\\
13.57	0.01\\
13.58	0.01\\
13.59	0.01\\
13.6	0.01\\
13.61	0.01\\
13.62	0.01\\
13.63	0.01\\
13.64	0.01\\
13.65	0.01\\
13.66	0.01\\
13.67	0.01\\
13.68	0.01\\
13.69	0.01\\
13.7	0.01\\
13.71	0.01\\
13.72	0.01\\
13.73	0.01\\
13.74	0.01\\
13.75	0.01\\
13.76	0.01\\
13.77	0.01\\
13.78	0.01\\
13.79	0.01\\
13.8	0.01\\
13.81	0.01\\
13.82	0.01\\
13.83	0.01\\
13.84	0.01\\
13.85	0.01\\
13.86	0.01\\
13.87	0.01\\
13.88	0.01\\
13.89	0.01\\
13.9	0.01\\
13.91	0.01\\
13.92	0.01\\
13.93	0.01\\
13.94	0.01\\
13.95	0.01\\
13.96	0.01\\
13.97	0.01\\
13.98	0.01\\
13.99	0.01\\
14	0.01\\
14.01	0.01\\
14.02	0.01\\
14.03	0.01\\
14.04	0.01\\
14.05	0.01\\
14.06	0.01\\
14.07	0.01\\
14.08	0.01\\
14.09	0.01\\
14.1	0.01\\
14.11	0.01\\
14.12	0.01\\
14.13	0.01\\
14.14	0.01\\
14.15	0.01\\
14.16	0.01\\
14.17	0.01\\
14.18	0.01\\
14.19	0.01\\
14.2	0.01\\
14.21	0.01\\
14.22	0.01\\
14.23	0.01\\
14.24	0.01\\
14.25	0.01\\
14.26	0.01\\
14.27	0.01\\
14.28	0.01\\
14.29	0.01\\
14.3	0.01\\
14.31	0.01\\
14.32	0.01\\
14.33	0.01\\
14.34	0.01\\
14.35	0.01\\
14.36	0.01\\
14.37	0.01\\
14.38	0.01\\
14.39	0.01\\
14.4	0.01\\
14.41	0.01\\
14.42	0.01\\
14.43	0.01\\
14.44	0.01\\
14.45	0.01\\
14.46	0.01\\
14.47	0.01\\
14.48	0.01\\
14.49	0.01\\
14.5	0.01\\
14.51	0.01\\
14.52	0.01\\
14.53	0.01\\
14.54	0.01\\
14.55	0.01\\
14.56	0.01\\
14.57	0.01\\
14.58	0.01\\
14.59	0.01\\
14.6	0.01\\
14.61	0.01\\
14.62	0.01\\
14.63	0.01\\
14.64	0.01\\
14.65	0.01\\
14.66	0.01\\
14.67	0.01\\
14.68	0.01\\
14.69	0.01\\
14.7	0.01\\
14.71	0.01\\
14.72	0.01\\
14.73	0.01\\
14.74	0.01\\
14.75	0.01\\
14.76	0.01\\
14.77	0.01\\
14.78	0.01\\
14.79	0.01\\
14.8	0.01\\
14.81	0.01\\
14.82	0.01\\
14.83	0.01\\
14.84	0.01\\
14.85	0.01\\
14.86	0.01\\
14.87	0.01\\
14.88	0.01\\
14.89	0.01\\
14.9	0.01\\
14.91	0.01\\
14.92	0.01\\
14.93	0.01\\
14.94	0.01\\
14.95	0.01\\
14.96	0.01\\
14.97	0.01\\
14.98	0.01\\
14.99	0.01\\
15	0.01\\
15.01	0.01\\
15.02	0.01\\
15.03	0.01\\
15.04	0.01\\
15.05	0.01\\
15.06	0.01\\
15.07	0.01\\
15.08	0.01\\
15.09	0.01\\
15.1	0.01\\
15.11	0.01\\
15.12	0.01\\
15.13	0.01\\
15.14	0.01\\
15.15	0.01\\
15.16	0.01\\
15.17	0.01\\
15.18	0.01\\
15.19	0.01\\
15.2	0.01\\
15.21	0.01\\
15.22	0.01\\
15.23	0.01\\
15.24	0.01\\
15.25	0.01\\
15.26	0.01\\
15.27	0.01\\
15.28	0.01\\
15.29	0.01\\
15.3	0.01\\
15.31	0.01\\
15.32	0.01\\
15.33	0.01\\
15.34	0.01\\
15.35	0.01\\
15.36	0.01\\
15.37	0.01\\
15.38	0.01\\
15.39	0.01\\
15.4	0.01\\
15.41	0.01\\
15.42	0.01\\
15.43	0.01\\
15.44	0.01\\
15.45	0.01\\
15.46	0.01\\
15.47	0.01\\
15.48	0.01\\
15.49	0.01\\
15.5	0.01\\
15.51	0.01\\
15.52	0.01\\
15.53	0.01\\
15.54	0.01\\
15.55	0.01\\
15.56	0.01\\
15.57	0.01\\
15.58	0.01\\
15.59	0.01\\
15.6	0.01\\
15.61	0.01\\
15.62	0.01\\
15.63	0.01\\
15.64	0.01\\
15.65	0.01\\
15.66	0.01\\
15.67	0.01\\
15.68	0.01\\
15.69	0.01\\
15.7	0.01\\
15.71	0.01\\
15.72	0.01\\
15.73	0.01\\
15.74	0.01\\
15.75	0.01\\
15.76	0.01\\
15.77	0.01\\
15.78	0.01\\
15.79	0.01\\
15.8	0.01\\
15.81	0.01\\
15.82	0.01\\
15.83	0.01\\
15.84	0.01\\
15.85	0.01\\
15.86	0.01\\
15.87	0.01\\
15.88	0.01\\
15.89	0.01\\
15.9	0.01\\
15.91	0.01\\
15.92	0.01\\
15.93	0.01\\
15.94	0.01\\
15.95	0.01\\
15.96	0.01\\
15.97	0.01\\
15.98	0.01\\
15.99	0.01\\
16	0.01\\
16.01	0.01\\
16.02	0.01\\
16.03	0.01\\
16.04	0.01\\
16.05	0.01\\
16.06	0.01\\
16.07	0.01\\
16.08	0.01\\
16.09	0.01\\
16.1	0.01\\
16.11	0.01\\
16.12	0.01\\
16.13	0.01\\
16.14	0.01\\
16.15	0.01\\
16.16	0.01\\
16.17	0.01\\
16.18	0.01\\
16.19	0.01\\
16.2	0.01\\
16.21	0.01\\
16.22	0.01\\
16.23	0.01\\
16.24	0.01\\
16.25	0.01\\
16.26	0.01\\
16.27	0.01\\
16.28	0.01\\
16.29	0.01\\
16.3	0.01\\
16.31	0.01\\
16.32	0.01\\
16.33	0.01\\
16.34	0.01\\
16.35	0.01\\
16.36	0.01\\
16.37	0.01\\
16.38	0.01\\
16.39	0.01\\
16.4	0.01\\
16.41	0.01\\
16.42	0.01\\
16.43	0.01\\
16.44	0.01\\
16.45	0.01\\
16.46	0.01\\
16.47	0.01\\
16.48	0.01\\
16.49	0.01\\
16.5	0.01\\
16.51	0.01\\
16.52	0.01\\
16.53	0.01\\
16.54	0.01\\
16.55	0.01\\
16.56	0.01\\
16.57	0.01\\
16.58	0.01\\
16.59	0.01\\
16.6	0.01\\
16.61	0.01\\
16.62	0.01\\
16.63	0.01\\
16.64	0.01\\
16.65	0.01\\
16.66	0.01\\
16.67	0.01\\
16.68	0.01\\
16.69	0.01\\
16.7	0.01\\
16.71	0.01\\
16.72	0.01\\
16.73	0.01\\
16.74	0.01\\
16.75	0.01\\
16.76	0.01\\
16.77	0.01\\
16.78	0.01\\
16.79	0.01\\
16.8	0.01\\
16.81	0.01\\
16.82	0.01\\
16.83	0.01\\
16.84	0.01\\
16.85	0.01\\
16.86	0.01\\
16.87	0.01\\
16.88	0.01\\
16.89	0.01\\
16.9	0.01\\
16.91	0.01\\
16.92	0.01\\
16.93	0.01\\
16.94	0.01\\
16.95	0.01\\
16.96	0.01\\
16.97	0.01\\
16.98	0.01\\
16.99	0.01\\
17	0.01\\
17.01	0.01\\
17.02	0.01\\
17.03	0.01\\
17.04	0.01\\
17.05	0.01\\
17.06	0.01\\
17.07	0.01\\
17.08	0.01\\
17.09	0.01\\
17.1	0.01\\
17.11	0.01\\
17.12	0.01\\
17.13	0.01\\
17.14	0.01\\
17.15	0.01\\
17.16	0.01\\
17.17	0.01\\
17.18	0.01\\
17.19	0.01\\
17.2	0.01\\
17.21	0.01\\
17.22	0.01\\
17.23	0.01\\
17.24	0.01\\
17.25	0.01\\
17.26	0.01\\
17.27	0.01\\
17.28	0.01\\
17.29	0.01\\
17.3	0.01\\
17.31	0.01\\
17.32	0.01\\
17.33	0.01\\
17.34	0.01\\
17.35	0.01\\
17.36	0.01\\
17.37	0.01\\
17.38	0.01\\
17.39	0.01\\
17.4	0.01\\
17.41	0.01\\
17.42	0.01\\
17.43	0.01\\
17.44	0.01\\
17.45	0.01\\
17.46	0.01\\
17.47	0.01\\
17.48	0.01\\
17.49	0.01\\
17.5	0.01\\
17.51	0.01\\
17.52	0.01\\
17.53	0.01\\
17.54	0.01\\
17.55	0.01\\
17.56	0.01\\
17.57	0.01\\
17.58	0.01\\
17.59	0.01\\
17.6	0.01\\
17.61	0.01\\
17.62	0.01\\
17.63	0.01\\
17.64	0.01\\
17.65	0.01\\
17.66	0.01\\
17.67	0.01\\
17.68	0.01\\
17.69	0.01\\
17.7	0.01\\
17.71	0.01\\
17.72	0.01\\
17.73	0.01\\
17.74	0.01\\
17.75	0.01\\
17.76	0.01\\
17.77	0.01\\
17.78	0.01\\
17.79	0.01\\
17.8	0.01\\
17.81	0.01\\
17.82	0.01\\
17.83	0.01\\
17.84	0.01\\
17.85	0.01\\
17.86	0.01\\
17.87	0.01\\
17.88	0.01\\
17.89	0.01\\
17.9	0.01\\
17.91	0.01\\
17.92	0.01\\
17.93	0.01\\
17.94	0.01\\
17.95	0.01\\
17.96	0.01\\
17.97	0.01\\
17.98	0.01\\
17.99	0.01\\
18	0.01\\
18.01	0.01\\
18.02	0.01\\
18.03	0.01\\
18.04	0.01\\
18.05	0.01\\
18.06	0.01\\
18.07	0.01\\
18.08	0.01\\
18.09	0.01\\
18.1	0.01\\
18.11	0.01\\
18.12	0.01\\
18.13	0.01\\
18.14	0.01\\
18.15	0.01\\
18.16	0.01\\
18.17	0.01\\
18.18	0.01\\
18.19	0.01\\
18.2	0.01\\
18.21	0.01\\
18.22	0.01\\
18.23	0.01\\
18.24	0.01\\
18.25	0.01\\
18.26	0.01\\
18.27	0.01\\
18.28	0.01\\
18.29	0.01\\
18.3	0.01\\
18.31	0.01\\
18.32	0.01\\
18.33	0.01\\
18.34	0.01\\
18.35	0.01\\
18.36	0.01\\
18.37	0.01\\
18.38	0.01\\
18.39	0.01\\
18.4	0.01\\
18.41	0.01\\
18.42	0.01\\
18.43	0.01\\
18.44	0.01\\
18.45	0.01\\
18.46	0.01\\
18.47	0.01\\
18.48	0.01\\
18.49	0.01\\
18.5	0.01\\
18.51	0.01\\
18.52	0.01\\
18.53	0.01\\
18.54	0.01\\
18.55	0.01\\
18.56	0.01\\
18.57	0.01\\
18.58	0.01\\
18.59	0.01\\
18.6	0.01\\
18.61	0.01\\
18.62	0.01\\
18.63	0.01\\
18.64	0.01\\
18.65	0.01\\
18.66	0.01\\
18.67	0.01\\
18.68	0.01\\
18.69	0.01\\
18.7	0.01\\
18.71	0.01\\
18.72	0.01\\
18.73	0.01\\
18.74	0.01\\
18.75	0.01\\
18.76	0.01\\
18.77	0.01\\
18.78	0.01\\
18.79	0.01\\
18.8	0.01\\
18.81	0.01\\
18.82	0.01\\
18.83	0.01\\
18.84	0.01\\
18.85	0.01\\
18.86	0.01\\
18.87	0.01\\
18.88	0.01\\
18.89	0.01\\
18.9	0.01\\
18.91	0.01\\
18.92	0.01\\
18.93	0.01\\
18.94	0.01\\
18.95	0.01\\
18.96	0.01\\
18.97	0.01\\
18.98	0.01\\
18.99	0.01\\
19	0.01\\
19.01	0.01\\
19.02	0.01\\
19.03	0.01\\
19.04	0.01\\
19.05	0.01\\
19.06	0.01\\
19.07	0.01\\
19.08	0.01\\
19.09	0.01\\
19.1	0.01\\
19.11	0.01\\
19.12	0.01\\
19.13	0.01\\
19.14	0.01\\
19.15	0.01\\
19.16	0.01\\
19.17	0.01\\
19.18	0.01\\
19.19	0.01\\
19.2	0.01\\
19.21	0.01\\
19.22	0.01\\
19.23	0.01\\
19.24	0.01\\
19.25	0.01\\
19.26	0.01\\
19.27	0.01\\
19.28	0.01\\
19.29	0.01\\
19.3	0.01\\
19.31	0.01\\
19.32	0.01\\
19.33	0.01\\
19.34	0.01\\
19.35	0.01\\
19.36	0.01\\
19.37	0.01\\
19.38	0.01\\
19.39	0.01\\
19.4	0.01\\
19.41	0.01\\
19.42	0.01\\
19.43	0.01\\
19.44	0.01\\
19.45	0.01\\
19.46	0.01\\
19.47	0.01\\
19.48	0.01\\
19.49	0.01\\
19.5	0.01\\
19.51	0.01\\
19.52	0.01\\
19.53	0.01\\
19.54	0.01\\
19.55	0.01\\
19.56	0.01\\
19.57	0.01\\
19.58	0.01\\
19.59	0.01\\
19.6	0.01\\
19.61	0.01\\
19.62	0.01\\
19.63	0.01\\
19.64	0.01\\
19.65	0.01\\
19.66	0.01\\
19.67	0.01\\
19.68	0.01\\
19.69	0.01\\
19.7	0.01\\
19.71	0.01\\
19.72	0.01\\
19.73	0.01\\
19.74	0.01\\
19.75	0.01\\
19.76	0.01\\
19.77	0.01\\
19.78	0.01\\
19.79	0.01\\
19.8	0.01\\
19.81	0.01\\
19.82	0.01\\
19.83	0.01\\
19.84	0.01\\
19.85	0.01\\
19.86	0.01\\
19.87	0.01\\
19.88	0.01\\
19.89	0.01\\
19.9	0.01\\
19.91	0.01\\
19.92	0.01\\
19.93	0.01\\
19.94	0.01\\
19.95	0.01\\
19.96	0.01\\
19.97	0.01\\
19.98	0.01\\
19.99	0.01\\
20	0.01\\
20.01	0.01\\
20.02	0.01\\
20.03	0.01\\
20.04	0.01\\
20.05	0.01\\
20.06	0.01\\
20.07	0.01\\
20.08	0.01\\
20.09	0.01\\
20.1	0.01\\
20.11	0.01\\
20.12	0.01\\
20.13	0.01\\
20.14	0.01\\
20.15	0.01\\
20.16	0.01\\
20.17	0.01\\
20.18	0.01\\
20.19	0.01\\
20.2	0.01\\
20.21	0.01\\
20.22	0.01\\
20.23	0.01\\
20.24	0.01\\
20.25	0.01\\
20.26	0.01\\
20.27	0.01\\
20.28	0.01\\
20.29	0.01\\
20.3	0.01\\
20.31	0.01\\
20.32	0.01\\
20.33	0.01\\
20.34	0.01\\
20.35	0.01\\
20.36	0.01\\
20.37	0.01\\
20.38	0.01\\
20.39	0.01\\
20.4	0.01\\
20.41	0.01\\
20.42	0.01\\
20.43	0.01\\
20.44	0.01\\
20.45	0.01\\
20.46	0.01\\
20.47	0.01\\
20.48	0.01\\
20.49	0.01\\
20.5	0.01\\
20.51	0.01\\
20.52	0.01\\
20.53	0.01\\
20.54	0.01\\
20.55	0.01\\
20.56	0.01\\
20.57	0.01\\
20.58	0.01\\
20.59	0.01\\
20.6	0.01\\
20.61	0.01\\
20.62	0.01\\
20.63	0.01\\
20.64	0.01\\
20.65	0.01\\
20.66	0.01\\
20.67	0.01\\
20.68	0.01\\
20.69	0.01\\
20.7	0.01\\
20.71	0.01\\
20.72	0.01\\
20.73	0.01\\
20.74	0.01\\
20.75	0.01\\
20.76	0.01\\
20.77	0.01\\
20.78	0.01\\
20.79	0.01\\
20.8	0.01\\
20.81	0.01\\
20.82	0.01\\
20.83	0.01\\
20.84	0.01\\
20.85	0.01\\
20.86	0.01\\
20.87	0.01\\
20.88	0.01\\
20.89	0.01\\
20.9	0.01\\
20.91	0.01\\
20.92	0.01\\
20.93	0.01\\
20.94	0.01\\
20.95	0.01\\
20.96	0.01\\
20.97	0.01\\
20.98	0.01\\
20.99	0.01\\
21	0.01\\
21.01	0.01\\
21.02	0.01\\
21.03	0.01\\
21.04	0.01\\
21.05	0.01\\
21.06	0.01\\
21.07	0.01\\
21.08	0.01\\
21.09	0.01\\
21.1	0.01\\
21.11	0.01\\
21.12	0.01\\
21.13	0.01\\
21.14	0.01\\
21.15	0.01\\
21.16	0.01\\
21.17	0.01\\
21.18	0.01\\
21.19	0.01\\
21.2	0.01\\
21.21	0.01\\
21.22	0.01\\
21.23	0.01\\
21.24	0.01\\
21.25	0.01\\
21.26	0.01\\
21.27	0.01\\
21.28	0.01\\
21.29	0.01\\
21.3	0.01\\
21.31	0.01\\
21.32	0.01\\
21.33	0.01\\
21.34	0.01\\
21.35	0.01\\
21.36	0.01\\
21.37	0.01\\
21.38	0.01\\
21.39	0.01\\
21.4	0.01\\
21.41	0.01\\
21.42	0.01\\
21.43	0.01\\
21.44	0.01\\
21.45	0.01\\
21.46	0.01\\
21.47	0.01\\
21.48	0.01\\
21.49	0.01\\
21.5	0.01\\
21.51	0.01\\
21.52	0.01\\
21.53	0.01\\
21.54	0.01\\
21.55	0.01\\
21.56	0.01\\
21.57	0.01\\
21.58	0.01\\
21.59	0.01\\
21.6	0.01\\
21.61	0.01\\
21.62	0.01\\
21.63	0.01\\
21.64	0.01\\
21.65	0.01\\
21.66	0.01\\
21.67	0.01\\
21.68	0.01\\
21.69	0.01\\
21.7	0.01\\
21.71	0.01\\
21.72	0.01\\
21.73	0.01\\
21.74	0.01\\
21.75	0.01\\
21.76	0.01\\
21.77	0.01\\
21.78	0.01\\
21.79	0.01\\
21.8	0.01\\
21.81	0.01\\
21.82	0.01\\
21.83	0.01\\
21.84	0.01\\
21.85	0.01\\
21.86	0.01\\
21.87	0.01\\
21.88	0.01\\
21.89	0.01\\
21.9	0.01\\
21.91	0.01\\
21.92	0.01\\
21.93	0.01\\
21.94	0.01\\
21.95	0.01\\
21.96	0.01\\
21.97	0.01\\
21.98	0.01\\
21.99	0.01\\
22	0.01\\
22.01	0.01\\
22.02	0.01\\
22.03	0.01\\
22.04	0.01\\
22.05	0.01\\
22.06	0.01\\
22.07	0.01\\
22.08	0.01\\
22.09	0.01\\
22.1	0.01\\
22.11	0.01\\
22.12	0.01\\
22.13	0.01\\
22.14	0.01\\
22.15	0.01\\
22.16	0.01\\
22.17	0.01\\
22.18	0.01\\
22.19	0.01\\
22.2	0.01\\
22.21	0.01\\
22.22	0.01\\
22.23	0.01\\
22.24	0.01\\
22.25	0.01\\
22.26	0.01\\
22.27	0.01\\
22.28	0.01\\
22.29	0.01\\
22.3	0.01\\
22.31	0.01\\
22.32	0.01\\
22.33	0.01\\
22.34	0.01\\
22.35	0.01\\
22.36	0.01\\
22.37	0.01\\
22.38	0.01\\
22.39	0.01\\
22.4	0.01\\
22.41	0.01\\
22.42	0.01\\
22.43	0.01\\
22.44	0.01\\
22.45	0.01\\
22.46	0.01\\
22.47	0.01\\
22.48	0.01\\
22.49	0.01\\
22.5	0.01\\
22.51	0.01\\
22.52	0.01\\
22.53	0.01\\
22.54	0.01\\
22.55	0.01\\
22.56	0.01\\
22.57	0.01\\
22.58	0.01\\
22.59	0.01\\
22.6	0.01\\
22.61	0.01\\
22.62	0.01\\
22.63	0.01\\
22.64	0.01\\
22.65	0.01\\
22.66	0.01\\
22.67	0.01\\
22.68	0.01\\
22.69	0.01\\
22.7	0.01\\
22.71	0.01\\
22.72	0.01\\
22.73	0.01\\
22.74	0.01\\
22.75	0.01\\
22.76	0.01\\
22.77	0.01\\
22.78	0.01\\
22.79	0.01\\
22.8	0.01\\
22.81	0.01\\
22.82	0.01\\
22.83	0.01\\
22.84	0.01\\
22.85	0.01\\
22.86	0.01\\
22.87	0.01\\
22.88	0.01\\
22.89	0.01\\
22.9	0.01\\
22.91	0.01\\
22.92	0.01\\
22.93	0.01\\
22.94	0.01\\
22.95	0.01\\
22.96	0.01\\
22.97	0.01\\
22.98	0.01\\
22.99	0.01\\
23	0.01\\
23.01	0.01\\
23.02	0.01\\
23.03	0.01\\
23.04	0.01\\
23.05	0.01\\
23.06	0.01\\
23.07	0.01\\
23.08	0.01\\
23.09	0.01\\
23.1	0.01\\
23.11	0.01\\
23.12	0.01\\
23.13	0.01\\
23.14	0.01\\
23.15	0.01\\
23.16	0.01\\
23.17	0.01\\
23.18	0.01\\
23.19	0.01\\
23.2	0.01\\
23.21	0.01\\
23.22	0.01\\
23.23	0.01\\
23.24	0.01\\
23.25	0.01\\
23.26	0.01\\
23.27	0.01\\
23.28	0.01\\
23.29	0.01\\
23.3	0.01\\
23.31	0.01\\
23.32	0.01\\
23.33	0.01\\
23.34	0.01\\
23.35	0.01\\
23.36	0.01\\
23.37	0.01\\
23.38	0.01\\
23.39	0.01\\
23.4	0.01\\
23.41	0.01\\
23.42	0.01\\
23.43	0.01\\
23.44	0.01\\
23.45	0.01\\
23.46	0.01\\
23.47	0.01\\
23.48	0.01\\
23.49	0.01\\
23.5	0.01\\
23.51	0.01\\
23.52	0.01\\
23.53	0.01\\
23.54	0.01\\
23.55	0.01\\
23.56	0.01\\
23.57	0.01\\
23.58	0.01\\
23.59	0.01\\
23.6	0.01\\
23.61	0.01\\
23.62	0.01\\
23.63	0.01\\
23.64	0.01\\
23.65	0.01\\
23.66	0.01\\
23.67	0.01\\
23.68	0.01\\
23.69	0.01\\
23.7	0.01\\
23.71	0.01\\
23.72	0.01\\
23.73	0.01\\
23.74	0.01\\
23.75	0.01\\
23.76	0.01\\
23.77	0.01\\
23.78	0.01\\
23.79	0.01\\
23.8	0.01\\
23.81	0.01\\
23.82	0.01\\
23.83	0.01\\
23.84	0.01\\
23.85	0.01\\
23.86	0.01\\
23.87	0.01\\
23.88	0.01\\
23.89	0.01\\
23.9	0.01\\
23.91	0.01\\
23.92	0.01\\
23.93	0.01\\
23.94	0.01\\
23.95	0.01\\
23.96	0.01\\
23.97	0.01\\
23.98	0.01\\
23.99	0.01\\
24	0.01\\
24.01	0.01\\
24.02	0.01\\
24.03	0.01\\
24.04	0.01\\
24.05	0.01\\
24.06	0.01\\
24.07	0.01\\
24.08	0.01\\
24.09	0.01\\
24.1	0.01\\
24.11	0.01\\
24.12	0.01\\
24.13	0.01\\
24.14	0.01\\
24.15	0.01\\
24.16	0.01\\
24.17	0.01\\
24.18	0.01\\
24.19	0.01\\
24.2	0.01\\
24.21	0.01\\
24.22	0.01\\
24.23	0.01\\
24.24	0.01\\
24.25	0.01\\
24.26	0.01\\
24.27	0.01\\
24.28	0.01\\
24.29	0.01\\
24.3	0.01\\
24.31	0.01\\
24.32	0.01\\
24.33	0.01\\
24.34	0.01\\
24.35	0.01\\
24.36	0.01\\
24.37	0.01\\
24.38	0.01\\
24.39	0.01\\
24.4	0.01\\
24.41	0.01\\
24.42	0.01\\
24.43	0.01\\
24.44	0.01\\
24.45	0.01\\
24.46	0.01\\
24.47	0.01\\
24.48	0.01\\
24.49	0.01\\
24.5	0.01\\
24.51	0.01\\
24.52	0.01\\
24.53	0.01\\
24.54	0.01\\
24.55	0.01\\
24.56	0.01\\
24.57	0.01\\
24.58	0.01\\
24.59	0.01\\
24.6	0.01\\
24.61	0.01\\
24.62	0.01\\
24.63	0.01\\
24.64	0.01\\
24.65	0.01\\
24.66	0.01\\
24.67	0.01\\
24.68	0.01\\
24.69	0.01\\
24.7	0.01\\
24.71	0.01\\
24.72	0.01\\
24.73	0.01\\
24.74	0.01\\
24.75	0.01\\
24.76	0.01\\
24.77	0.01\\
24.78	0.01\\
24.79	0.01\\
24.8	0.01\\
24.81	0.01\\
24.82	0.01\\
24.83	0.01\\
24.84	0.01\\
24.85	0.01\\
24.86	0.01\\
24.87	0.01\\
24.88	0.01\\
24.89	0.01\\
24.9	0.01\\
24.91	0.01\\
24.92	0.01\\
24.93	0.01\\
24.94	0.01\\
24.95	0.01\\
24.96	0.01\\
24.97	0.01\\
24.98	0.01\\
24.99	0.01\\
25	0.01\\
25.01	0.01\\
25.02	0.01\\
25.03	0.01\\
25.04	0.01\\
25.05	0.01\\
25.06	0.01\\
25.07	0.01\\
25.08	0.01\\
25.09	0.01\\
25.1	0.01\\
25.11	0.01\\
25.12	0.01\\
25.13	0.01\\
25.14	0.01\\
25.15	0.01\\
25.16	0.01\\
25.17	0.01\\
25.18	0.01\\
25.19	0.01\\
25.2	0.01\\
25.21	0.01\\
25.22	0.01\\
25.23	0.01\\
25.24	0.01\\
25.25	0.01\\
25.26	0.01\\
25.27	0.01\\
25.28	0.01\\
25.29	0.01\\
25.3	0.01\\
25.31	0.01\\
25.32	0.01\\
25.33	0.01\\
25.34	0.01\\
25.35	0.01\\
25.36	0.01\\
25.37	0.01\\
25.38	0.01\\
25.39	0.01\\
25.4	0.01\\
25.41	0.01\\
25.42	0.01\\
25.43	0.01\\
25.44	0.01\\
25.45	0.01\\
25.46	0.01\\
25.47	0.01\\
25.48	0.01\\
25.49	0.01\\
25.5	0.01\\
25.51	0.01\\
25.52	0.01\\
25.53	0.01\\
25.54	0.01\\
25.55	0.01\\
25.56	0.01\\
25.57	0.01\\
25.58	0.01\\
25.59	0.01\\
25.6	0.01\\
25.61	0.01\\
25.62	0.01\\
25.63	0.01\\
25.64	0.01\\
25.65	0.01\\
25.66	0.01\\
25.67	0.01\\
25.68	0.01\\
25.69	0.01\\
25.7	0.01\\
25.71	0.01\\
25.72	0.01\\
25.73	0.01\\
25.74	0.01\\
25.75	0.01\\
25.76	0.01\\
25.77	0.01\\
25.78	0.01\\
25.79	0.01\\
25.8	0.01\\
25.81	0.01\\
25.82	0.01\\
25.83	0.01\\
25.84	0.01\\
25.85	0.01\\
25.86	0.01\\
25.87	0.01\\
25.88	0.01\\
25.89	0.01\\
25.9	0.01\\
25.91	0.01\\
25.92	0.01\\
25.93	0.01\\
25.94	0.01\\
25.95	0.01\\
25.96	0.01\\
25.97	0.01\\
25.98	0.01\\
25.99	0.01\\
26	0.01\\
26.01	0.01\\
26.02	0.01\\
26.03	0.01\\
26.04	0.01\\
26.05	0.01\\
26.06	0.01\\
26.07	0.01\\
26.08	0.01\\
26.09	0.01\\
26.1	0.01\\
26.11	0.01\\
26.12	0.01\\
26.13	0.01\\
26.14	0.01\\
26.15	0.01\\
26.16	0.01\\
26.17	0.01\\
26.18	0.01\\
26.19	0.01\\
26.2	0.01\\
26.21	0.01\\
26.22	0.01\\
26.23	0.01\\
26.24	0.01\\
26.25	0.01\\
26.26	0.01\\
26.27	0.01\\
26.28	0.01\\
26.29	0.01\\
26.3	0.01\\
26.31	0.01\\
26.32	0.01\\
26.33	0.01\\
26.34	0.01\\
26.35	0.01\\
26.36	0.01\\
26.37	0.01\\
26.38	0.01\\
26.39	0.01\\
26.4	0.01\\
26.41	0.01\\
26.42	0.01\\
26.43	0.01\\
26.44	0.01\\
26.45	0.01\\
26.46	0.01\\
26.47	0.01\\
26.48	0.01\\
26.49	0.01\\
26.5	0.01\\
26.51	0.01\\
26.52	0.01\\
26.53	0.01\\
26.54	0.01\\
26.55	0.01\\
26.56	0.01\\
26.57	0.01\\
26.58	0.01\\
26.59	0.01\\
26.6	0.01\\
26.61	0.01\\
26.62	0.01\\
26.63	0.01\\
26.64	0.01\\
26.65	0.01\\
26.66	0.01\\
26.67	0.01\\
26.68	0.01\\
26.69	0.01\\
26.7	0.01\\
26.71	0.01\\
26.72	0.01\\
26.73	0.01\\
26.74	0.01\\
26.75	0.01\\
26.76	0.01\\
26.77	0.01\\
26.78	0.01\\
26.79	0.01\\
26.8	0.01\\
26.81	0.01\\
26.82	0.01\\
26.83	0.01\\
26.84	0.01\\
26.85	0.01\\
26.86	0.01\\
26.87	0.01\\
26.88	0.01\\
26.89	0.01\\
26.9	0.01\\
26.91	0.01\\
26.92	0.01\\
26.93	0.01\\
26.94	0.01\\
26.95	0.01\\
26.96	0.01\\
26.97	0.01\\
26.98	0.01\\
26.99	0.01\\
27	0.01\\
27.01	0.01\\
27.02	0.01\\
27.03	0.01\\
27.04	0.01\\
27.05	0.01\\
27.06	0.01\\
27.07	0.01\\
27.08	0.01\\
27.09	0.01\\
27.1	0.01\\
27.11	0.01\\
27.12	0.01\\
27.13	0.01\\
27.14	0.01\\
27.15	0.01\\
27.16	0.01\\
27.17	0.01\\
27.18	0.01\\
27.19	0.01\\
27.2	0.01\\
27.21	0.01\\
27.22	0.01\\
27.23	0.01\\
27.24	0.01\\
27.25	0.01\\
27.26	0.01\\
27.27	0.01\\
27.28	0.01\\
27.29	0.01\\
27.3	0.01\\
27.31	0.01\\
27.32	0.01\\
27.33	0.01\\
27.34	0.01\\
27.35	0.01\\
27.36	0.01\\
27.37	0.01\\
27.38	0.01\\
27.39	0.01\\
27.4	0.01\\
27.41	0.01\\
27.42	0.01\\
27.43	0.01\\
27.44	0.01\\
27.45	0.01\\
27.46	0.01\\
27.47	0.01\\
27.48	0.01\\
27.49	0.01\\
27.5	0.01\\
27.51	0.01\\
27.52	0.01\\
27.53	0.01\\
27.54	0.01\\
27.55	0.01\\
27.56	0.01\\
27.57	0.01\\
27.58	0.01\\
27.59	0.01\\
27.6	0.01\\
27.61	0.01\\
27.62	0.01\\
27.63	0.01\\
27.64	0.01\\
27.65	0.01\\
27.66	0.01\\
27.67	0.01\\
27.68	0.01\\
27.69	0.01\\
27.7	0.01\\
27.71	0.01\\
27.72	0.01\\
27.73	0.01\\
27.74	0.01\\
27.75	0.01\\
27.76	0.01\\
27.77	0.01\\
27.78	0.01\\
27.79	0.01\\
27.8	0.01\\
27.81	0.01\\
27.82	0.01\\
27.83	0.01\\
27.84	0.01\\
27.85	0.01\\
27.86	0.01\\
27.87	0.01\\
27.88	0.01\\
27.89	0.01\\
27.9	0.01\\
27.91	0.01\\
27.92	0.01\\
27.93	0.01\\
27.94	0.01\\
27.95	0.01\\
27.96	0.01\\
27.97	0.01\\
27.98	0.01\\
27.99	0.01\\
28	0.01\\
28.01	0.01\\
28.02	0.01\\
28.03	0.01\\
28.04	0.01\\
28.05	0.01\\
28.06	0.01\\
28.07	0.01\\
28.08	0.01\\
28.09	0.01\\
28.1	0.01\\
28.11	0.01\\
28.12	0.01\\
28.13	0.01\\
28.14	0.01\\
28.15	0.01\\
28.16	0.01\\
28.17	0.01\\
28.18	0.01\\
28.19	0.01\\
28.2	0.01\\
28.21	0.01\\
28.22	0.01\\
28.23	0.01\\
28.24	0.01\\
28.25	0.01\\
28.26	0.01\\
28.27	0.01\\
28.28	0.01\\
28.29	0.01\\
28.3	0.01\\
28.31	0.01\\
28.32	0.01\\
28.33	0.01\\
28.34	0.01\\
28.35	0.01\\
28.36	0.01\\
28.37	0.01\\
28.38	0.01\\
28.39	0.01\\
28.4	0.01\\
28.41	0.01\\
28.42	0.01\\
28.43	0.01\\
28.44	0.01\\
28.45	0.01\\
28.46	0.01\\
28.47	0.01\\
28.48	0.01\\
28.49	0.01\\
28.5	0.01\\
28.51	0.01\\
28.52	0.01\\
28.53	0.01\\
28.54	0.01\\
28.55	0.01\\
28.56	0.01\\
28.57	0.01\\
28.58	0.01\\
28.59	0.01\\
28.6	0.01\\
28.61	0.01\\
28.62	0.01\\
28.63	0.01\\
28.64	0.01\\
28.65	0.01\\
28.66	0.01\\
28.67	0.01\\
28.68	0.01\\
28.69	0.01\\
28.7	0.01\\
28.71	0.01\\
28.72	0.01\\
28.73	0.01\\
28.74	0.01\\
28.75	0.01\\
28.76	0.01\\
28.77	0.01\\
28.78	0.01\\
28.79	0.01\\
28.8	0.01\\
28.81	0.01\\
28.82	0.01\\
28.83	0.01\\
28.84	0.01\\
28.85	0.01\\
28.86	0.01\\
28.87	0.01\\
28.88	0.01\\
28.89	0.01\\
28.9	0.01\\
28.91	0.01\\
28.92	0.01\\
28.93	0.01\\
28.94	0.01\\
28.95	0.01\\
28.96	0.01\\
28.97	0.01\\
28.98	0.01\\
28.99	0.01\\
29	0.01\\
29.01	0.01\\
29.02	0.01\\
29.03	0.01\\
29.04	0.01\\
29.05	0.01\\
29.06	0.01\\
29.07	0.01\\
29.08	0.01\\
29.09	0.01\\
29.1	0.01\\
29.11	0.01\\
29.12	0.01\\
29.13	0.01\\
29.14	0.01\\
29.15	0.01\\
29.16	0.01\\
29.17	0.01\\
29.18	0.01\\
29.19	0.01\\
29.2	0.01\\
29.21	0.01\\
29.22	0.01\\
29.23	0.01\\
29.24	0.01\\
29.25	0.01\\
29.26	0.01\\
29.27	0.01\\
29.28	0.01\\
29.29	0.01\\
29.3	0.01\\
29.31	0.01\\
29.32	0.01\\
29.33	0.01\\
29.34	0.01\\
29.35	0.01\\
29.36	0.01\\
29.37	0.01\\
29.38	0.01\\
29.39	0.01\\
29.4	0.01\\
29.41	0.01\\
29.42	0.01\\
29.43	0.01\\
29.44	0.01\\
29.45	0.01\\
29.46	0.01\\
29.47	0.01\\
29.48	0.01\\
29.49	0.01\\
29.5	0.01\\
29.51	0.01\\
29.52	0.01\\
29.53	0.01\\
29.54	0.01\\
29.55	0.01\\
29.56	0.01\\
29.57	0.01\\
29.58	0.01\\
29.59	0.01\\
29.6	0.01\\
29.61	0.01\\
29.62	0.01\\
29.63	0.01\\
29.64	0.01\\
29.65	0.01\\
29.66	0.01\\
29.67	0.01\\
29.68	0.01\\
29.69	0.01\\
29.7	0.01\\
29.71	0.01\\
29.72	0.01\\
29.73	0.01\\
29.74	0.01\\
29.75	0.01\\
29.76	0.01\\
29.77	0.01\\
29.78	0.01\\
29.79	0.01\\
29.8	0.01\\
29.81	0.01\\
29.82	0.01\\
29.83	0.01\\
29.84	0.01\\
29.85	0.01\\
29.86	0.01\\
29.87	0.01\\
29.88	0.01\\
29.89	0.01\\
29.9	0.01\\
29.91	0.01\\
29.92	0.01\\
29.93	0.01\\
29.94	0.01\\
29.95	0.01\\
29.96	0.01\\
29.97	0.01\\
29.98	0.01\\
29.99	0.01\\
30	0.01\\
30.01	0.01\\
30.02	0.01\\
30.03	0.01\\
30.04	0.01\\
30.05	0.01\\
30.06	0.01\\
30.07	0.01\\
30.08	0.01\\
30.09	0.01\\
30.1	0.01\\
30.11	0.01\\
30.12	0.01\\
30.13	0.01\\
30.14	0.01\\
30.15	0.01\\
30.16	0.01\\
30.17	0.01\\
30.18	0.01\\
30.19	0.01\\
30.2	0.01\\
30.21	0.01\\
30.22	0.01\\
30.23	0.01\\
30.24	0.01\\
30.25	0.01\\
30.26	0.01\\
30.27	0.01\\
30.28	0.01\\
30.29	0.01\\
30.3	0.01\\
30.31	0.01\\
30.32	0.01\\
30.33	0.01\\
30.34	0.01\\
30.35	0.01\\
30.36	0.01\\
30.37	0.01\\
30.38	0.01\\
30.39	0.01\\
30.4	0.01\\
30.41	0.01\\
30.42	0.01\\
30.43	0.01\\
30.44	0.01\\
30.45	0.01\\
30.46	0.01\\
30.47	0.01\\
30.48	0.01\\
30.49	0.01\\
30.5	0.01\\
30.51	0.01\\
30.52	0.01\\
30.53	0.01\\
30.54	0.01\\
30.55	0.01\\
30.56	0.01\\
30.57	0.01\\
30.58	0.01\\
30.59	0.01\\
30.6	0.01\\
30.61	0.01\\
30.62	0.01\\
30.63	0.01\\
30.64	0.01\\
30.65	0.01\\
30.66	0.01\\
30.67	0.01\\
30.68	0.01\\
30.69	0.01\\
30.7	0.01\\
30.71	0.01\\
30.72	0.01\\
30.73	0.01\\
30.74	0.01\\
30.75	0.01\\
30.76	0.01\\
30.77	0.01\\
30.78	0.01\\
30.79	0.01\\
30.8	0.01\\
30.81	0.01\\
30.82	0.01\\
30.83	0.01\\
30.84	0.01\\
30.85	0.01\\
30.86	0.01\\
30.87	0.01\\
30.88	0.01\\
30.89	0.01\\
30.9	0.01\\
30.91	0.01\\
30.92	0.01\\
30.93	0.01\\
30.94	0.01\\
30.95	0.01\\
30.96	0.01\\
30.97	0.01\\
30.98	0.01\\
30.99	0.01\\
31	0.01\\
31.01	0.01\\
31.02	0.01\\
31.03	0.01\\
31.04	0.01\\
31.05	0.01\\
31.06	0.01\\
31.07	0.01\\
31.08	0.01\\
31.09	0.01\\
31.1	0.01\\
31.11	0.01\\
31.12	0.01\\
31.13	0.01\\
31.14	0.01\\
31.15	0.01\\
31.16	0.01\\
31.17	0.01\\
31.18	0.01\\
31.19	0.01\\
31.2	0.01\\
31.21	0.01\\
31.22	0.01\\
31.23	0.01\\
31.24	0.01\\
31.25	0.01\\
31.26	0.01\\
31.27	0.01\\
31.28	0.01\\
31.29	0.01\\
31.3	0.01\\
31.31	0.01\\
31.32	0.01\\
31.33	0.01\\
31.34	0.01\\
31.35	0.01\\
31.36	0.01\\
31.37	0.01\\
31.38	0.01\\
31.39	0.01\\
31.4	0.01\\
31.41	0.01\\
31.42	0.01\\
31.43	0.01\\
31.44	0.01\\
31.45	0.01\\
31.46	0.01\\
31.47	0.01\\
31.48	0.01\\
31.49	0.01\\
31.5	0.01\\
31.51	0.01\\
31.52	0.01\\
31.53	0.01\\
31.54	0.01\\
31.55	0.01\\
31.56	0.01\\
31.57	0.01\\
31.58	0.01\\
31.59	0.01\\
31.6	0.01\\
31.61	0.01\\
31.62	0.01\\
31.63	0.01\\
31.64	0.01\\
31.65	0.01\\
31.66	0.01\\
31.67	0.01\\
31.68	0.01\\
31.69	0.01\\
31.7	0.01\\
31.71	0.01\\
31.72	0.01\\
31.73	0.01\\
31.74	0.01\\
31.75	0.01\\
31.76	0.01\\
31.77	0.01\\
31.78	0.01\\
31.79	0.01\\
31.8	0.01\\
31.81	0.01\\
31.82	0.01\\
31.83	0.01\\
31.84	0.01\\
31.85	0.01\\
31.86	0.01\\
31.87	0.01\\
31.88	0.01\\
31.89	0.01\\
31.9	0.01\\
31.91	0.01\\
31.92	0.01\\
31.93	0.01\\
31.94	0.01\\
31.95	0.01\\
31.96	0.01\\
31.97	0.01\\
31.98	0.01\\
31.99	0.01\\
32	0.01\\
32.01	0.01\\
32.02	0.01\\
32.03	0.01\\
32.04	0.01\\
32.05	0.01\\
32.06	0.01\\
32.07	0.01\\
32.08	0.01\\
32.09	0.01\\
32.1	0.01\\
32.11	0.01\\
32.12	0.01\\
32.13	0.01\\
32.14	0.01\\
32.15	0.01\\
32.16	0.01\\
32.17	0.01\\
32.18	0.01\\
32.19	0.01\\
32.2	0.01\\
32.21	0.01\\
32.22	0.01\\
32.23	0.01\\
32.24	0.01\\
32.25	0.01\\
32.26	0.01\\
32.27	0.01\\
32.28	0.01\\
32.29	0.01\\
32.3	0.01\\
32.31	0.01\\
32.32	0.01\\
32.33	0.01\\
32.34	0.01\\
32.35	0.01\\
32.36	0.01\\
32.37	0.01\\
32.38	0.01\\
32.39	0.01\\
32.4	0.01\\
32.41	0.01\\
32.42	0.01\\
32.43	0.01\\
32.44	0.01\\
32.45	0.01\\
32.46	0.01\\
32.47	0.01\\
32.48	0.01\\
32.49	0.01\\
32.5	0.01\\
32.51	0.01\\
32.52	0.01\\
32.53	0.01\\
32.54	0.01\\
32.55	0.01\\
32.56	0.01\\
32.57	0.01\\
32.58	0.01\\
32.59	0.01\\
32.6	0.01\\
32.61	0.01\\
32.62	0.01\\
32.63	0.01\\
32.64	0.01\\
32.65	0.01\\
32.66	0.01\\
32.67	0.01\\
32.68	0.01\\
32.69	0.01\\
32.7	0.01\\
32.71	0.01\\
32.72	0.01\\
32.73	0.01\\
32.74	0.01\\
32.75	0.01\\
32.76	0.01\\
32.77	0.01\\
32.78	0.01\\
32.79	0.01\\
32.8	0.01\\
32.81	0.01\\
32.82	0.01\\
32.83	0.01\\
32.84	0.01\\
32.85	0.01\\
32.86	0.01\\
32.87	0.01\\
32.88	0.01\\
32.89	0.01\\
32.9	0.01\\
32.91	0.01\\
32.92	0.01\\
32.93	0.01\\
32.94	0.01\\
32.95	0.01\\
32.96	0.01\\
32.97	0.01\\
32.98	0.01\\
32.99	0.01\\
33	0.01\\
33.01	0.01\\
33.02	0.01\\
33.03	0.01\\
33.04	0.01\\
33.05	0.01\\
33.06	0.01\\
33.07	0.01\\
33.08	0.01\\
33.09	0.01\\
33.1	0.01\\
33.11	0.01\\
33.12	0.01\\
33.13	0.01\\
33.14	0.01\\
33.15	0.01\\
33.16	0.01\\
33.17	0.01\\
33.18	0.01\\
33.19	0.01\\
33.2	0.01\\
33.21	0.01\\
33.22	0.01\\
33.23	0.01\\
33.24	0.01\\
33.25	0.01\\
33.26	0.01\\
33.27	0.01\\
33.28	0.01\\
33.29	0.01\\
33.3	0.01\\
33.31	0.01\\
33.32	0.01\\
33.33	0.01\\
33.34	0.01\\
33.35	0.01\\
33.36	0.01\\
33.37	0.01\\
33.38	0.01\\
33.39	0.01\\
33.4	0.01\\
33.41	0.01\\
33.42	0.01\\
33.43	0.01\\
33.44	0.01\\
33.45	0.01\\
33.46	0.01\\
33.47	0.01\\
33.48	0.01\\
33.49	0.01\\
33.5	0.01\\
33.51	0.01\\
33.52	0.01\\
33.53	0.01\\
33.54	0.01\\
33.55	0.01\\
33.56	0.01\\
33.57	0.01\\
33.58	0.01\\
33.59	0.01\\
33.6	0.01\\
33.61	0.01\\
33.62	0.01\\
33.63	0.01\\
33.64	0.01\\
33.65	0.01\\
33.66	0.01\\
33.67	0.01\\
33.68	0.01\\
33.69	0.01\\
33.7	0.01\\
33.71	0.01\\
33.72	0.01\\
33.73	0.01\\
33.74	0.01\\
33.75	0.01\\
33.76	0.01\\
33.77	0.01\\
33.78	0.01\\
33.79	0.01\\
33.8	0.01\\
33.81	0.01\\
33.82	0.01\\
33.83	0.01\\
33.84	0.01\\
33.85	0.01\\
33.86	0.01\\
33.87	0.01\\
33.88	0.01\\
33.89	0.01\\
33.9	0.01\\
33.91	0.01\\
33.92	0.01\\
33.93	0.01\\
33.94	0.01\\
33.95	0.01\\
33.96	0.01\\
33.97	0.01\\
33.98	0.01\\
33.99	0.01\\
34	0.01\\
34.01	0.01\\
34.02	0.01\\
34.03	0.01\\
34.04	0.01\\
34.05	0.01\\
34.06	0.01\\
34.07	0.01\\
34.08	0.01\\
34.09	0.01\\
34.1	0.01\\
34.11	0.01\\
34.12	0.01\\
34.13	0.01\\
34.14	0.01\\
34.15	0.01\\
34.16	0.01\\
34.17	0.01\\
34.18	0.01\\
34.19	0.01\\
34.2	0.01\\
34.21	0.01\\
34.22	0.01\\
34.23	0.01\\
34.24	0.01\\
34.25	0.01\\
34.26	0.01\\
34.27	0.01\\
34.28	0.01\\
34.29	0.01\\
34.3	0.01\\
34.31	0.01\\
34.32	0.01\\
34.33	0.01\\
34.34	0.01\\
34.35	0.01\\
34.36	0.01\\
34.37	0.01\\
34.38	0.01\\
34.39	0.01\\
34.4	0.01\\
34.41	0.01\\
34.42	0.01\\
34.43	0.01\\
34.44	0.01\\
34.45	0.01\\
34.46	0.01\\
34.47	0.01\\
34.48	0.01\\
34.49	0.01\\
34.5	0.01\\
34.51	0.01\\
34.52	0.01\\
34.53	0.01\\
34.54	0.01\\
34.55	0.01\\
34.56	0.01\\
34.57	0.01\\
34.58	0.01\\
34.59	0.01\\
34.6	0.01\\
34.61	0.01\\
34.62	0.01\\
34.63	0.01\\
34.64	0.01\\
34.65	0.01\\
34.66	0.01\\
34.67	0.01\\
34.68	0.01\\
34.69	0.01\\
34.7	0.01\\
34.71	0.01\\
34.72	0.01\\
34.73	0.01\\
34.74	0.01\\
34.75	0.01\\
34.76	0.01\\
34.77	0.01\\
34.78	0.01\\
34.79	0.01\\
34.8	0.01\\
34.81	0.01\\
34.82	0.01\\
34.83	0.01\\
34.84	0.01\\
34.85	0.01\\
34.86	0.01\\
34.87	0.01\\
34.88	0.01\\
34.89	0.01\\
34.9	0.01\\
34.91	0.01\\
34.92	0.01\\
34.93	0.01\\
34.94	0.01\\
34.95	0.01\\
34.96	0.01\\
34.97	0.01\\
34.98	0.01\\
34.99	0.01\\
35	0.01\\
35.01	0.01\\
35.02	0.01\\
35.03	0.01\\
35.04	0.01\\
35.05	0.01\\
35.06	0.01\\
35.07	0.01\\
35.08	0.01\\
35.09	0.01\\
35.1	0.01\\
35.11	0.01\\
35.12	0.01\\
35.13	0.01\\
35.14	0.01\\
35.15	0.01\\
35.16	0.01\\
35.17	0.01\\
35.18	0.01\\
35.19	0.01\\
35.2	0.01\\
35.21	0.01\\
35.22	0.01\\
35.23	0.01\\
35.24	0.01\\
35.25	0.01\\
35.26	0.01\\
35.27	0.01\\
35.28	0.01\\
35.29	0.01\\
35.3	0.01\\
35.31	0.01\\
35.32	0.01\\
35.33	0.01\\
35.34	0.01\\
35.35	0.01\\
35.36	0.01\\
35.37	0.01\\
35.38	0.01\\
35.39	0.01\\
35.4	0.01\\
35.41	0.01\\
35.42	0.01\\
35.43	0.01\\
35.44	0.01\\
35.45	0.01\\
35.46	0.01\\
35.47	0.01\\
35.48	0.01\\
35.49	0.01\\
35.5	0.01\\
35.51	0.01\\
35.52	0.01\\
35.53	0.01\\
35.54	0.01\\
35.55	0.01\\
35.56	0.01\\
35.57	0.01\\
35.58	0.01\\
35.59	0.01\\
35.6	0.01\\
35.61	0.01\\
35.62	0.01\\
35.63	0.01\\
35.64	0.01\\
35.65	0.01\\
35.66	0.01\\
35.67	0.01\\
35.68	0.01\\
35.69	0.01\\
35.7	0.01\\
35.71	0.01\\
35.72	0.01\\
35.73	0.01\\
35.74	0.01\\
35.75	0.01\\
35.76	0.01\\
35.77	0.01\\
35.78	0.01\\
35.79	0.01\\
35.8	0.01\\
35.81	0.01\\
35.82	0.01\\
35.83	0.01\\
35.84	0.01\\
35.85	0.01\\
35.86	0.01\\
35.87	0.01\\
35.88	0.01\\
35.89	0.01\\
35.9	0.01\\
35.91	0.01\\
35.92	0.01\\
35.93	0.01\\
35.94	0.01\\
35.95	0.01\\
35.96	0.01\\
35.97	0.01\\
35.98	0.01\\
35.99	0.01\\
36	0.01\\
36.01	0.01\\
36.02	0.01\\
36.03	0.01\\
36.04	0.01\\
36.05	0.01\\
36.06	0.01\\
36.07	0.01\\
36.08	0.01\\
36.09	0.01\\
36.1	0.01\\
36.11	0.01\\
36.12	0.01\\
36.13	0.01\\
36.14	0.01\\
36.15	0.01\\
36.16	0.01\\
36.17	0.01\\
36.18	0.01\\
36.19	0.01\\
36.2	0.01\\
36.21	0.01\\
36.22	0.01\\
36.23	0.01\\
36.24	0.01\\
36.25	0.01\\
36.26	0.01\\
36.27	0.01\\
36.28	0.01\\
36.29	0.01\\
36.3	0.01\\
36.31	0.01\\
36.32	0.01\\
36.33	0.01\\
36.34	0.01\\
36.35	0.01\\
36.36	0.01\\
36.37	0.01\\
36.38	0.01\\
36.39	0.01\\
36.4	0.01\\
36.41	0.01\\
36.42	0.01\\
36.43	0.01\\
36.44	0.01\\
36.45	0.01\\
36.46	0.01\\
36.47	0.01\\
36.48	0.01\\
36.49	0.01\\
36.5	0.01\\
36.51	0.01\\
36.52	0.01\\
36.53	0.01\\
36.54	0.01\\
36.55	0.01\\
36.56	0.01\\
36.57	0.01\\
36.58	0.01\\
36.59	0.01\\
36.6	0.01\\
36.61	0.01\\
36.62	0.01\\
36.63	0.01\\
36.64	0.01\\
36.65	0.01\\
36.66	0.01\\
36.67	0.01\\
36.68	0.01\\
36.69	0.01\\
36.7	0.01\\
36.71	0.01\\
36.72	0.01\\
36.73	0.01\\
36.74	0.01\\
36.75	0.01\\
36.76	0.01\\
36.77	0.01\\
36.78	0.01\\
36.79	0.01\\
36.8	0.01\\
36.81	0.01\\
36.82	0.01\\
36.83	0.01\\
36.84	0.01\\
36.85	0.01\\
36.86	0.01\\
36.87	0.01\\
36.88	0.01\\
36.89	0.01\\
36.9	0.01\\
36.91	0.01\\
36.92	0.01\\
36.93	0.01\\
36.94	0.01\\
36.95	0.01\\
36.96	0.01\\
36.97	0.01\\
36.98	0.01\\
36.99	0.01\\
37	0.01\\
37.01	0.01\\
37.02	0.01\\
37.03	0.01\\
37.04	0.01\\
37.05	0.01\\
37.06	0.01\\
37.07	0.01\\
37.08	0.01\\
37.09	0.01\\
37.1	0.01\\
37.11	0.01\\
37.12	0.01\\
37.13	0.01\\
37.14	0.01\\
37.15	0.01\\
37.16	0.01\\
37.17	0.01\\
37.18	0.01\\
37.19	0.01\\
37.2	0.01\\
37.21	0.01\\
37.22	0.01\\
37.23	0.01\\
37.24	0.01\\
37.25	0.01\\
37.26	0.01\\
37.27	0.01\\
37.28	0.01\\
37.29	0.01\\
37.3	0.01\\
37.31	0.01\\
37.32	0.01\\
37.33	0.01\\
37.34	0.01\\
37.35	0.01\\
37.36	0.01\\
37.37	0.01\\
37.38	0.01\\
37.39	0.01\\
37.4	0.01\\
37.41	0.01\\
37.42	0.01\\
37.43	0.01\\
37.44	0.01\\
37.45	0.01\\
37.46	0.01\\
37.47	0.01\\
37.48	0.01\\
37.49	0.01\\
37.5	0.01\\
37.51	0.01\\
37.52	0.01\\
37.53	0.01\\
37.54	0.01\\
37.55	0.01\\
37.56	0.01\\
37.57	0.01\\
37.58	0.01\\
37.59	0.01\\
37.6	0.01\\
37.61	0.01\\
37.62	0.01\\
37.63	0.01\\
37.64	0.01\\
37.65	0.01\\
37.66	0.01\\
37.67	0.01\\
37.68	0.01\\
37.69	0.01\\
37.7	0.01\\
37.71	0.01\\
37.72	0.01\\
37.73	0.01\\
37.74	0.01\\
37.75	0.01\\
37.76	0.01\\
37.77	0.01\\
37.78	0.01\\
37.79	0.01\\
37.8	0.01\\
37.81	0.01\\
37.82	0.01\\
37.83	0.01\\
37.84	0.01\\
37.85	0.01\\
37.86	0.01\\
37.87	0.01\\
37.88	0.01\\
37.89	0.01\\
37.9	0.01\\
37.91	0.01\\
37.92	0.01\\
37.93	0.01\\
37.94	0.01\\
37.95	0.01\\
37.96	0.01\\
37.97	0.01\\
37.98	0.01\\
37.99	0.01\\
38	0.01\\
38.01	0.01\\
38.02	0.01\\
38.03	0.01\\
38.04	0.01\\
38.05	0.01\\
38.06	0.01\\
38.07	0.01\\
38.08	0.01\\
38.09	0.01\\
38.1	0.01\\
38.11	0.01\\
38.12	0.01\\
38.13	0.01\\
38.14	0.01\\
38.15	0.01\\
38.16	0.01\\
38.17	0.01\\
38.18	0.01\\
38.19	0.01\\
38.2	0.01\\
38.21	0.01\\
38.22	0.01\\
38.23	0.01\\
38.24	0.01\\
38.25	0.01\\
38.26	0.01\\
38.27	0.01\\
38.28	0.01\\
38.29	0.01\\
38.3	0.01\\
38.31	0.01\\
38.32	0.01\\
38.33	0.01\\
38.34	0.01\\
38.35	0.01\\
38.36	0.01\\
38.37	0.01\\
38.38	0.01\\
38.39	0.01\\
38.4	0.01\\
38.41	0.01\\
38.42	0.01\\
38.43	0.01\\
38.44	0.01\\
38.45	0.01\\
38.46	0.01\\
38.47	0.01\\
38.48	0.01\\
38.49	0.01\\
38.5	0.01\\
38.51	0.01\\
38.52	0.01\\
38.53	0.01\\
38.54	0.01\\
38.55	0.01\\
38.56	0.01\\
38.57	0.01\\
38.58	0.01\\
38.59	0.01\\
38.6	0.01\\
38.61	0.01\\
38.62	0.01\\
38.63	0.01\\
38.64	0.01\\
38.65	0.01\\
38.66	0.01\\
38.67	0.01\\
38.68	0.01\\
38.69	0.01\\
38.7	0.01\\
38.71	0.01\\
38.72	0.01\\
38.73	0.01\\
38.74	0.01\\
38.75	0.01\\
38.76	0.01\\
38.77	0.01\\
38.78	0.01\\
38.79	0.01\\
38.8	0.01\\
38.81	0.01\\
38.82	0.01\\
38.83	0.01\\
38.84	0.01\\
38.85	0.01\\
38.86	0.01\\
38.87	0.01\\
38.88	0.01\\
38.89	0.01\\
38.9	0.01\\
38.91	0.01\\
38.92	0.01\\
38.93	0.01\\
38.94	0.01\\
38.95	0.01\\
38.96	0.01\\
38.97	0.01\\
38.98	0.01\\
38.99	0.01\\
39	0.01\\
39.01	0.01\\
39.02	0.01\\
39.03	0.01\\
39.04	0.01\\
39.05	0.01\\
39.06	0.01\\
39.07	0.01\\
39.08	0.01\\
39.09	0.01\\
39.1	0.01\\
39.11	0.01\\
39.12	0.01\\
39.13	0.01\\
39.14	0.01\\
39.15	0.01\\
39.16	0.01\\
39.17	0.01\\
39.18	0.01\\
39.19	0.01\\
39.2	0.01\\
39.21	0.01\\
39.22	0.01\\
39.23	0.01\\
39.24	0.01\\
39.25	0.01\\
39.26	0.01\\
39.27	0.01\\
39.28	0.01\\
39.29	0.01\\
39.3	0.01\\
39.31	0.01\\
39.32	0.01\\
39.33	0.01\\
39.34	0.01\\
39.35	0.01\\
39.36	0.01\\
39.37	0.01\\
39.38	0.01\\
39.39	0.01\\
39.4	0.01\\
39.41	0.01\\
39.42	0.01\\
39.43	0.01\\
39.44	0.01\\
39.45	0.01\\
39.46	0.01\\
39.47	0.01\\
39.48	0.01\\
39.49	0.01\\
39.5	0.01\\
39.51	0.01\\
39.52	0.01\\
39.53	0.01\\
39.54	0.01\\
39.55	0.01\\
39.56	0.01\\
39.57	0.01\\
39.58	0.01\\
39.59	0.01\\
39.6	0.01\\
39.61	0.01\\
39.62	0.01\\
39.63	0.01\\
39.64	0.01\\
39.65	0.01\\
39.66	0.01\\
39.67	0.01\\
39.68	0.01\\
39.69	0.01\\
39.7	0.01\\
39.71	0.01\\
39.72	0.01\\
39.73	0.01\\
39.74	0.01\\
39.75	0.01\\
39.76	0.01\\
39.77	0.01\\
39.78	0.01\\
39.79	0.01\\
39.8	0.01\\
39.81	0.01\\
39.82	0.01\\
39.83	0.01\\
39.84	0.01\\
39.85	0.01\\
39.86	0.01\\
39.87	0.01\\
39.88	0.01\\
39.89	0.01\\
39.9	0.01\\
39.91	0.01\\
39.92	0.01\\
39.93	0.01\\
39.94	0.01\\
39.95	0.01\\
39.96	0.01\\
39.97	0.01\\
39.98	0.01\\
39.99	0.01\\
40	0.01\\
40.01	0.01\\
};
\addplot [color=red,dashed,forget plot]
  table[row sep=crcr]{%
40.01	0.01\\
40.02	0.01\\
40.03	0.01\\
40.04	0.01\\
40.05	0.01\\
40.06	0.01\\
40.07	0.01\\
40.08	0.01\\
40.09	0.01\\
40.1	0.01\\
40.11	0.01\\
40.12	0.01\\
40.13	0.01\\
40.14	0.01\\
40.15	0.01\\
40.16	0.01\\
40.17	0.01\\
40.18	0.01\\
40.19	0.01\\
40.2	0.01\\
40.21	0.01\\
40.22	0.01\\
40.23	0.01\\
40.24	0.01\\
40.25	0.01\\
40.26	0.01\\
40.27	0.01\\
40.28	0.01\\
40.29	0.01\\
40.3	0.01\\
40.31	0.01\\
40.32	0.01\\
40.33	0.01\\
40.34	0.01\\
40.35	0.01\\
40.36	0.01\\
40.37	0.01\\
40.38	0.01\\
40.39	0.01\\
40.4	0.01\\
40.41	0.01\\
40.42	0.01\\
40.43	0.01\\
40.44	0.01\\
40.45	0.01\\
40.46	0.01\\
40.47	0.01\\
40.48	0.01\\
40.49	0.01\\
40.5	0.01\\
40.51	0.01\\
40.52	0.01\\
40.53	0.01\\
40.54	0.01\\
40.55	0.01\\
40.56	0.01\\
40.57	0.01\\
40.58	0.01\\
40.59	0.01\\
40.6	0.01\\
40.61	0.01\\
40.62	0.01\\
40.63	0.01\\
40.64	0.01\\
40.65	0.01\\
40.66	0.01\\
40.67	0.01\\
40.68	0.01\\
40.69	0.01\\
40.7	0.01\\
40.71	0.01\\
40.72	0.01\\
40.73	0.01\\
40.74	0.01\\
40.75	0.01\\
40.76	0.01\\
40.77	0.01\\
40.78	0.01\\
40.79	0.01\\
40.8	0.01\\
40.81	0.01\\
40.82	0.01\\
40.83	0.01\\
40.84	0.01\\
40.85	0.01\\
40.86	0.01\\
40.87	0.01\\
40.88	0.01\\
40.89	0.01\\
40.9	0.01\\
40.91	0.01\\
40.92	0.01\\
40.93	0.01\\
40.94	0.01\\
40.95	0.01\\
40.96	0.01\\
40.97	0.01\\
40.98	0.01\\
40.99	0.01\\
41	0.01\\
41.01	0.01\\
41.02	0.01\\
41.03	0.01\\
41.04	0.01\\
41.05	0.01\\
41.06	0.01\\
41.07	0.01\\
41.08	0.01\\
41.09	0.01\\
41.1	0.01\\
41.11	0.01\\
41.12	0.01\\
41.13	0.01\\
41.14	0.01\\
41.15	0.01\\
41.16	0.01\\
41.17	0.01\\
41.18	0.01\\
41.19	0.01\\
41.2	0.01\\
41.21	0.01\\
41.22	0.01\\
41.23	0.01\\
41.24	0.01\\
41.25	0.01\\
41.26	0.01\\
41.27	0.01\\
41.28	0.01\\
41.29	0.01\\
41.3	0.01\\
41.31	0.01\\
41.32	0.01\\
41.33	0.01\\
41.34	0.01\\
41.35	0.01\\
41.36	0.01\\
41.37	0.01\\
41.38	0.01\\
41.39	0.01\\
41.4	0.01\\
41.41	0.01\\
41.42	0.01\\
41.43	0.01\\
41.44	0.01\\
41.45	0.01\\
41.46	0.01\\
41.47	0.01\\
41.48	0.01\\
41.49	0.01\\
41.5	0.01\\
41.51	0.01\\
41.52	0.01\\
41.53	0.01\\
41.54	0.01\\
41.55	0.01\\
41.56	0.01\\
41.57	0.01\\
41.58	0.01\\
41.59	0.01\\
41.6	0.01\\
41.61	0.01\\
41.62	0.01\\
41.63	0.01\\
41.64	0.01\\
41.65	0.01\\
41.66	0.01\\
41.67	0.01\\
41.68	0.01\\
41.69	0.01\\
41.7	0.01\\
41.71	0.01\\
41.72	0.01\\
41.73	0.01\\
41.74	0.01\\
41.75	0.01\\
41.76	0.01\\
41.77	0.01\\
41.78	0.01\\
41.79	0.01\\
41.8	0.01\\
41.81	0.01\\
41.82	0.01\\
41.83	0.01\\
41.84	0.01\\
41.85	0.01\\
41.86	0.01\\
41.87	0.01\\
41.88	0.01\\
41.89	0.01\\
41.9	0.01\\
41.91	0.01\\
41.92	0.01\\
41.93	0.01\\
41.94	0.01\\
41.95	0.01\\
41.96	0.01\\
41.97	0.01\\
41.98	0.01\\
41.99	0.01\\
42	0.01\\
42.01	0.01\\
42.02	0.01\\
42.03	0.01\\
42.04	0.01\\
42.05	0.01\\
42.06	0.01\\
42.07	0.01\\
42.08	0.01\\
42.09	0.01\\
42.1	0.01\\
42.11	0.01\\
42.12	0.01\\
42.13	0.01\\
42.14	0.01\\
42.15	0.01\\
42.16	0.01\\
42.17	0.01\\
42.18	0.01\\
42.19	0.01\\
42.2	0.01\\
42.21	0.01\\
42.22	0.01\\
42.23	0.01\\
42.24	0.01\\
42.25	0.01\\
42.26	0.01\\
42.27	0.01\\
42.28	0.01\\
42.29	0.01\\
42.3	0.01\\
42.31	0.01\\
42.32	0.01\\
42.33	0.01\\
42.34	0.01\\
42.35	0.01\\
42.36	0.01\\
42.37	0.01\\
42.38	0.01\\
42.39	0.01\\
42.4	0.01\\
42.41	0.01\\
42.42	0.01\\
42.43	0.01\\
42.44	0.01\\
42.45	0.01\\
42.46	0.01\\
42.47	0.01\\
42.48	0.01\\
42.49	0.01\\
42.5	0.01\\
42.51	0.01\\
42.52	0.01\\
42.53	0.01\\
42.54	0.01\\
42.55	0.01\\
42.56	0.01\\
42.57	0.01\\
42.58	0.01\\
42.59	0.01\\
42.6	0.01\\
42.61	0.01\\
42.62	0.01\\
42.63	0.01\\
42.64	0.01\\
42.65	0.01\\
42.66	0.01\\
42.67	0.01\\
42.68	0.01\\
42.69	0.01\\
42.7	0.01\\
42.71	0.01\\
42.72	0.01\\
42.73	0.01\\
42.74	0.01\\
42.75	0.01\\
42.76	0.01\\
42.77	0.01\\
42.78	0.01\\
42.79	0.01\\
42.8	0.01\\
42.81	0.01\\
42.82	0.01\\
42.83	0.01\\
42.84	0.01\\
42.85	0.01\\
42.86	0.01\\
42.87	0.01\\
42.88	0.01\\
42.89	0.01\\
42.9	0.01\\
42.91	0.01\\
42.92	0.01\\
42.93	0.01\\
42.94	0.01\\
42.95	0.01\\
42.96	0.01\\
42.97	0.01\\
42.98	0.01\\
42.99	0.01\\
43	0.01\\
43.01	0.01\\
43.02	0.01\\
43.03	0.01\\
43.04	0.01\\
43.05	0.01\\
43.06	0.01\\
43.07	0.01\\
43.08	0.01\\
43.09	0.01\\
43.1	0.01\\
43.11	0.01\\
43.12	0.01\\
43.13	0.01\\
43.14	0.01\\
43.15	0.01\\
43.16	0.01\\
43.17	0.01\\
43.18	0.01\\
43.19	0.01\\
43.2	0.01\\
43.21	0.01\\
43.22	0.01\\
43.23	0.01\\
43.24	0.01\\
43.25	0.01\\
43.26	0.01\\
43.27	0.01\\
43.28	0.01\\
43.29	0.01\\
43.3	0.01\\
43.31	0.01\\
43.32	0.01\\
43.33	0.01\\
43.34	0.01\\
43.35	0.01\\
43.36	0.01\\
43.37	0.01\\
43.38	0.01\\
43.39	0.01\\
43.4	0.01\\
43.41	0.01\\
43.42	0.01\\
43.43	0.01\\
43.44	0.01\\
43.45	0.01\\
43.46	0.01\\
43.47	0.01\\
43.48	0.01\\
43.49	0.01\\
43.5	0.01\\
43.51	0.01\\
43.52	0.01\\
43.53	0.01\\
43.54	0.01\\
43.55	0.01\\
43.56	0.01\\
43.57	0.01\\
43.58	0.01\\
43.59	0.01\\
43.6	0.01\\
43.61	0.01\\
43.62	0.01\\
43.63	0.01\\
43.64	0.01\\
43.65	0.01\\
43.66	0.01\\
43.67	0.01\\
43.68	0.01\\
43.69	0.01\\
43.7	0.01\\
43.71	0.01\\
43.72	0.01\\
43.73	0.01\\
43.74	0.01\\
43.75	0.01\\
43.76	0.01\\
43.77	0.01\\
43.78	0.01\\
43.79	0.01\\
43.8	0.01\\
43.81	0.01\\
43.82	0.01\\
43.83	0.01\\
43.84	0.01\\
43.85	0.01\\
43.86	0.01\\
43.87	0.01\\
43.88	0.01\\
43.89	0.01\\
43.9	0.01\\
43.91	0.01\\
43.92	0.01\\
43.93	0.01\\
43.94	0.01\\
43.95	0.01\\
43.96	0.01\\
43.97	0.01\\
43.98	0.01\\
43.99	0.01\\
44	0.01\\
44.01	0.01\\
44.02	0.01\\
44.03	0.01\\
44.04	0.01\\
44.05	0.01\\
44.06	0.01\\
44.07	0.01\\
44.08	0.01\\
44.09	0.01\\
44.1	0.01\\
44.11	0.01\\
44.12	0.01\\
44.13	0.01\\
44.14	0.01\\
44.15	0.01\\
44.16	0.01\\
44.17	0.01\\
44.18	0.01\\
44.19	0.01\\
44.2	0.01\\
44.21	0.01\\
44.22	0.01\\
44.23	0.01\\
44.24	0.01\\
44.25	0.01\\
44.26	0.01\\
44.27	0.01\\
44.28	0.01\\
44.29	0.01\\
44.3	0.01\\
44.31	0.01\\
44.32	0.01\\
44.33	0.01\\
44.34	0.01\\
44.35	0.01\\
44.36	0.01\\
44.37	0.01\\
44.38	0.01\\
44.39	0.01\\
44.4	0.01\\
44.41	0.01\\
44.42	0.01\\
44.43	0.01\\
44.44	0.01\\
44.45	0.01\\
44.46	0.01\\
44.47	0.01\\
44.48	0.01\\
44.49	0.01\\
44.5	0.01\\
44.51	0.01\\
44.52	0.01\\
44.53	0.01\\
44.54	0.01\\
44.55	0.01\\
44.56	0.01\\
44.57	0.01\\
44.58	0.01\\
44.59	0.01\\
44.6	0.01\\
44.61	0.01\\
44.62	0.01\\
44.63	0.01\\
44.64	0.01\\
44.65	0.01\\
44.66	0.01\\
44.67	0.01\\
44.68	0.01\\
44.69	0.01\\
44.7	0.01\\
44.71	0.01\\
44.72	0.01\\
44.73	0.01\\
44.74	0.01\\
44.75	0.01\\
44.76	0.01\\
44.77	0.01\\
44.78	0.01\\
44.79	0.01\\
44.8	0.01\\
44.81	0.01\\
44.82	0.01\\
44.83	0.01\\
44.84	0.01\\
44.85	0.01\\
44.86	0.01\\
44.87	0.01\\
44.88	0.01\\
44.89	0.01\\
44.9	0.01\\
44.91	0.01\\
44.92	0.01\\
44.93	0.01\\
44.94	0.01\\
44.95	0.01\\
44.96	0.01\\
44.97	0.01\\
44.98	0.01\\
44.99	0.01\\
45	0.01\\
45.01	0.01\\
45.02	0.01\\
45.03	0.01\\
45.04	0.01\\
45.05	0.01\\
45.06	0.01\\
45.07	0.01\\
45.08	0.01\\
45.09	0.01\\
45.1	0.01\\
45.11	0.01\\
45.12	0.01\\
45.13	0.01\\
45.14	0.01\\
45.15	0.01\\
45.16	0.01\\
45.17	0.01\\
45.18	0.01\\
45.19	0.01\\
45.2	0.01\\
45.21	0.01\\
45.22	0.01\\
45.23	0.01\\
45.24	0.01\\
45.25	0.01\\
45.26	0.01\\
45.27	0.01\\
45.28	0.01\\
45.29	0.01\\
45.3	0.01\\
45.31	0.01\\
45.32	0.01\\
45.33	0.01\\
45.34	0.01\\
45.35	0.01\\
45.36	0.01\\
45.37	0.01\\
45.38	0.01\\
45.39	0.01\\
45.4	0.01\\
45.41	0.01\\
45.42	0.01\\
45.43	0.01\\
45.44	0.01\\
45.45	0.01\\
45.46	0.01\\
45.47	0.01\\
45.48	0.01\\
45.49	0.01\\
45.5	0.01\\
45.51	0.01\\
45.52	0.01\\
45.53	0.01\\
45.54	0.01\\
45.55	0.01\\
45.56	0.01\\
45.57	0.01\\
45.58	0.01\\
45.59	0.01\\
45.6	0.01\\
45.61	0.01\\
45.62	0.01\\
45.63	0.01\\
45.64	0.01\\
45.65	0.01\\
45.66	0.01\\
45.67	0.01\\
45.68	0.01\\
45.69	0.01\\
45.7	0.01\\
45.71	0.01\\
45.72	0.01\\
45.73	0.01\\
45.74	0.01\\
45.75	0.01\\
45.76	0.01\\
45.77	0.01\\
45.78	0.01\\
45.79	0.01\\
45.8	0.01\\
45.81	0.01\\
45.82	0.01\\
45.83	0.01\\
45.84	0.01\\
45.85	0.01\\
45.86	0.01\\
45.87	0.01\\
45.88	0.01\\
45.89	0.01\\
45.9	0.01\\
45.91	0.01\\
45.92	0.01\\
45.93	0.01\\
45.94	0.01\\
45.95	0.01\\
45.96	0.01\\
45.97	0.01\\
45.98	0.01\\
45.99	0.01\\
46	0.01\\
46.01	0.01\\
46.02	0.01\\
46.03	0.01\\
46.04	0.01\\
46.05	0.01\\
46.06	0.01\\
46.07	0.01\\
46.08	0.01\\
46.09	0.01\\
46.1	0.01\\
46.11	0.01\\
46.12	0.01\\
46.13	0.01\\
46.14	0.01\\
46.15	0.01\\
46.16	0.01\\
46.17	0.01\\
46.18	0.01\\
46.19	0.01\\
46.2	0.01\\
46.21	0.01\\
46.22	0.01\\
46.23	0.01\\
46.24	0.01\\
46.25	0.01\\
46.26	0.01\\
46.27	0.01\\
46.28	0.01\\
46.29	0.01\\
46.3	0.01\\
46.31	0.01\\
46.32	0.01\\
46.33	0.01\\
46.34	0.01\\
46.35	0.01\\
46.36	0.01\\
46.37	0.01\\
46.38	0.01\\
46.39	0.01\\
46.4	0.01\\
46.41	0.01\\
46.42	0.01\\
46.43	0.01\\
46.44	0.01\\
46.45	0.01\\
46.46	0.01\\
46.47	0.01\\
46.48	0.01\\
46.49	0.01\\
46.5	0.01\\
46.51	0.01\\
46.52	0.01\\
46.53	0.01\\
46.54	0.01\\
46.55	0.01\\
46.56	0.01\\
46.57	0.01\\
46.58	0.01\\
46.59	0.01\\
46.6	0.01\\
46.61	0.01\\
46.62	0.01\\
46.63	0.01\\
46.64	0.01\\
46.65	0.01\\
46.66	0.01\\
46.67	0.01\\
46.68	0.01\\
46.69	0.01\\
46.7	0.01\\
46.71	0.01\\
46.72	0.01\\
46.73	0.01\\
46.74	0.01\\
46.75	0.01\\
46.76	0.01\\
46.77	0.01\\
46.78	0.01\\
46.79	0.01\\
46.8	0.01\\
46.81	0.01\\
46.82	0.01\\
46.83	0.01\\
46.84	0.01\\
46.85	0.01\\
46.86	0.01\\
46.87	0.01\\
46.88	0.01\\
46.89	0.01\\
46.9	0.01\\
46.91	0.01\\
46.92	0.01\\
46.93	0.01\\
46.94	0.01\\
46.95	0.01\\
46.96	0.01\\
46.97	0.01\\
46.98	0.01\\
46.99	0.01\\
47	0.01\\
47.01	0.01\\
47.02	0.01\\
47.03	0.01\\
47.04	0.01\\
47.05	0.01\\
47.06	0.01\\
47.07	0.01\\
47.08	0.01\\
47.09	0.01\\
47.1	0.01\\
47.11	0.01\\
47.12	0.01\\
47.13	0.01\\
47.14	0.01\\
47.15	0.01\\
47.16	0.01\\
47.17	0.01\\
47.18	0.01\\
47.19	0.01\\
47.2	0.01\\
47.21	0.01\\
47.22	0.01\\
47.23	0.01\\
47.24	0.01\\
47.25	0.01\\
47.26	0.01\\
47.27	0.01\\
47.28	0.01\\
47.29	0.01\\
47.3	0.01\\
47.31	0.01\\
47.32	0.01\\
47.33	0.01\\
47.34	0.01\\
47.35	0.01\\
47.36	0.01\\
47.37	0.01\\
47.38	0.01\\
47.39	0.01\\
47.4	0.01\\
47.41	0.01\\
47.42	0.01\\
47.43	0.01\\
47.44	0.01\\
47.45	0.01\\
47.46	0.01\\
47.47	0.01\\
47.48	0.01\\
47.49	0.01\\
47.5	0.01\\
47.51	0.01\\
47.52	0.01\\
47.53	0.01\\
47.54	0.01\\
47.55	0.01\\
47.56	0.01\\
47.57	0.01\\
47.58	0.01\\
47.59	0.01\\
47.6	0.01\\
47.61	0.01\\
47.62	0.01\\
47.63	0.01\\
47.64	0.01\\
47.65	0.01\\
47.66	0.01\\
47.67	0.01\\
47.68	0.01\\
47.69	0.01\\
47.7	0.01\\
47.71	0.01\\
47.72	0.01\\
47.73	0.01\\
47.74	0.01\\
47.75	0.01\\
47.76	0.01\\
47.77	0.01\\
47.78	0.01\\
47.79	0.01\\
47.8	0.01\\
47.81	0.01\\
47.82	0.01\\
47.83	0.01\\
47.84	0.01\\
47.85	0.01\\
47.86	0.01\\
47.87	0.01\\
47.88	0.01\\
47.89	0.01\\
47.9	0.01\\
47.91	0.01\\
47.92	0.01\\
47.93	0.01\\
47.94	0.01\\
47.95	0.01\\
47.96	0.01\\
47.97	0.01\\
47.98	0.01\\
47.99	0.01\\
48	0.01\\
48.01	0.01\\
48.02	0.01\\
48.03	0.01\\
48.04	0.01\\
48.05	0.01\\
48.06	0.01\\
48.07	0.01\\
48.08	0.01\\
48.09	0.01\\
48.1	0.01\\
48.11	0.01\\
48.12	0.01\\
48.13	0.01\\
48.14	0.01\\
48.15	0.01\\
48.16	0.01\\
48.17	0.01\\
48.18	0.01\\
48.19	0.01\\
48.2	0.01\\
48.21	0.01\\
48.22	0.01\\
48.23	0.01\\
48.24	0.01\\
48.25	0.01\\
48.26	0.01\\
48.27	0.01\\
48.28	0.01\\
48.29	0.01\\
48.3	0.01\\
48.31	0.01\\
48.32	0.01\\
48.33	0.01\\
48.34	0.01\\
48.35	0.01\\
48.36	0.01\\
48.37	0.01\\
48.38	0.01\\
48.39	0.01\\
48.4	0.01\\
48.41	0.01\\
48.42	0.01\\
48.43	0.01\\
48.44	0.01\\
48.45	0.01\\
48.46	0.01\\
48.47	0.01\\
48.48	0.01\\
48.49	0.01\\
48.5	0.01\\
48.51	0.01\\
48.52	0.01\\
48.53	0.01\\
48.54	0.01\\
48.55	0.01\\
48.56	0.01\\
48.57	0.01\\
48.58	0.01\\
48.59	0.01\\
48.6	0.01\\
48.61	0.01\\
48.62	0.01\\
48.63	0.01\\
48.64	0.01\\
48.65	0.01\\
48.66	0.01\\
48.67	0.01\\
48.68	0.01\\
48.69	0.01\\
48.7	0.01\\
48.71	0.01\\
48.72	0.01\\
48.73	0.01\\
48.74	0.01\\
48.75	0.01\\
48.76	0.01\\
48.77	0.01\\
48.78	0.01\\
48.79	0.01\\
48.8	0.01\\
48.81	0.01\\
48.82	0.01\\
48.83	0.01\\
48.84	0.01\\
48.85	0.01\\
48.86	0.01\\
48.87	0.01\\
48.88	0.01\\
48.89	0.01\\
48.9	0.01\\
48.91	0.01\\
48.92	0.01\\
48.93	0.01\\
48.94	0.01\\
48.95	0.01\\
48.96	0.01\\
48.97	0.01\\
48.98	0.01\\
48.99	0.01\\
49	0.01\\
49.01	0.01\\
49.02	0.01\\
49.03	0.01\\
49.04	0.01\\
49.05	0.01\\
49.06	0.01\\
49.07	0.01\\
49.08	0.01\\
49.09	0.01\\
49.1	0.01\\
49.11	0.01\\
49.12	0.01\\
49.13	0.01\\
49.14	0.01\\
49.15	0.01\\
49.16	0.01\\
49.17	0.01\\
49.18	0.01\\
49.19	0.01\\
49.2	0.01\\
49.21	0.01\\
49.22	0.01\\
49.23	0.01\\
49.24	0.01\\
49.25	0.01\\
49.26	0.01\\
49.27	0.01\\
49.28	0.01\\
49.29	0.01\\
49.3	0.01\\
49.31	0.01\\
49.32	0.01\\
49.33	0.01\\
49.34	0.01\\
49.35	0.01\\
49.36	0.01\\
49.37	0.01\\
49.38	0.01\\
49.39	0.01\\
49.4	0.01\\
49.41	0.01\\
49.42	0.01\\
49.43	0.01\\
49.44	0.01\\
49.45	0.01\\
49.46	0.01\\
49.47	0.01\\
49.48	0.01\\
49.49	0.01\\
49.5	0.01\\
49.51	0.01\\
49.52	0.01\\
49.53	0.01\\
49.54	0.01\\
49.55	0.01\\
49.56	0.01\\
49.57	0.01\\
49.58	0.01\\
49.59	0.01\\
49.6	0.01\\
49.61	0.01\\
49.62	0.01\\
49.63	0.01\\
49.64	0.01\\
49.65	0.01\\
49.66	0.01\\
49.67	0.01\\
49.68	0.01\\
49.69	0.01\\
49.7	0.01\\
49.71	0.01\\
49.72	0.01\\
49.73	0.01\\
49.74	0.01\\
49.75	0.01\\
49.76	0.01\\
49.77	0.01\\
49.78	0.01\\
49.79	0.01\\
49.8	0.01\\
49.81	0.01\\
49.82	0.01\\
49.83	0.01\\
49.84	0.01\\
49.85	0.01\\
49.86	0.01\\
49.87	0.01\\
49.88	0.01\\
49.89	0.01\\
49.9	0.01\\
49.91	0.01\\
49.92	0.01\\
49.93	0.01\\
49.94	0.01\\
49.95	0.01\\
49.96	0.01\\
49.97	0.01\\
49.98	0.01\\
49.99	0.01\\
50	0.01\\
50.01	0.01\\
50.02	0.01\\
50.03	0.01\\
50.04	0.01\\
50.05	0.01\\
50.06	0.01\\
50.07	0.01\\
50.08	0.01\\
50.09	0.01\\
50.1	0.01\\
50.11	0.01\\
50.12	0.01\\
50.13	0.01\\
50.14	0.01\\
50.15	0.01\\
50.16	0.01\\
50.17	0.01\\
50.18	0.01\\
50.19	0.01\\
50.2	0.01\\
50.21	0.01\\
50.22	0.01\\
50.23	0.01\\
50.24	0.01\\
50.25	0.01\\
50.26	0.01\\
50.27	0.01\\
50.28	0.01\\
50.29	0.01\\
50.3	0.01\\
50.31	0.01\\
50.32	0.01\\
50.33	0.01\\
50.34	0.01\\
50.35	0.01\\
50.36	0.01\\
50.37	0.01\\
50.38	0.01\\
50.39	0.01\\
50.4	0.01\\
50.41	0.01\\
50.42	0.01\\
50.43	0.01\\
50.44	0.01\\
50.45	0.01\\
50.46	0.01\\
50.47	0.01\\
50.48	0.01\\
50.49	0.01\\
50.5	0.01\\
50.51	0.01\\
50.52	0.01\\
50.53	0.01\\
50.54	0.01\\
50.55	0.01\\
50.56	0.01\\
50.57	0.01\\
50.58	0.01\\
50.59	0.01\\
50.6	0.01\\
50.61	0.01\\
50.62	0.01\\
50.63	0.01\\
50.64	0.01\\
50.65	0.01\\
50.66	0.01\\
50.67	0.01\\
50.68	0.01\\
50.69	0.01\\
50.7	0.01\\
50.71	0.01\\
50.72	0.01\\
50.73	0.01\\
50.74	0.01\\
50.75	0.01\\
50.76	0.01\\
50.77	0.01\\
50.78	0.01\\
50.79	0.01\\
50.8	0.01\\
50.81	0.01\\
50.82	0.01\\
50.83	0.01\\
50.84	0.01\\
50.85	0.01\\
50.86	0.01\\
50.87	0.01\\
50.88	0.01\\
50.89	0.01\\
50.9	0.01\\
50.91	0.01\\
50.92	0.01\\
50.93	0.01\\
50.94	0.01\\
50.95	0.01\\
50.96	0.01\\
50.97	0.01\\
50.98	0.01\\
50.99	0.01\\
51	0.01\\
51.01	0.01\\
51.02	0.01\\
51.03	0.01\\
51.04	0.01\\
51.05	0.01\\
51.06	0.01\\
51.07	0.01\\
51.08	0.01\\
51.09	0.01\\
51.1	0.01\\
51.11	0.01\\
51.12	0.01\\
51.13	0.01\\
51.14	0.01\\
51.15	0.01\\
51.16	0.01\\
51.17	0.01\\
51.18	0.01\\
51.19	0.01\\
51.2	0.01\\
51.21	0.01\\
51.22	0.01\\
51.23	0.01\\
51.24	0.01\\
51.25	0.01\\
51.26	0.01\\
51.27	0.01\\
51.28	0.01\\
51.29	0.01\\
51.3	0.01\\
51.31	0.01\\
51.32	0.01\\
51.33	0.01\\
51.34	0.01\\
51.35	0.01\\
51.36	0.01\\
51.37	0.01\\
51.38	0.01\\
51.39	0.01\\
51.4	0.01\\
51.41	0.01\\
51.42	0.01\\
51.43	0.01\\
51.44	0.01\\
51.45	0.01\\
51.46	0.01\\
51.47	0.01\\
51.48	0.01\\
51.49	0.01\\
51.5	0.01\\
51.51	0.01\\
51.52	0.01\\
51.53	0.01\\
51.54	0.01\\
51.55	0.01\\
51.56	0.01\\
51.57	0.01\\
51.58	0.01\\
51.59	0.01\\
51.6	0.01\\
51.61	0.01\\
51.62	0.01\\
51.63	0.01\\
51.64	0.01\\
51.65	0.01\\
51.66	0.01\\
51.67	0.01\\
51.68	0.01\\
51.69	0.01\\
51.7	0.01\\
51.71	0.01\\
51.72	0.01\\
51.73	0.01\\
51.74	0.01\\
51.75	0.01\\
51.76	0.01\\
51.77	0.01\\
51.78	0.01\\
51.79	0.01\\
51.8	0.01\\
51.81	0.01\\
51.82	0.01\\
51.83	0.01\\
51.84	0.01\\
51.85	0.01\\
51.86	0.01\\
51.87	0.01\\
51.88	0.01\\
51.89	0.01\\
51.9	0.01\\
51.91	0.01\\
51.92	0.01\\
51.93	0.01\\
51.94	0.01\\
51.95	0.01\\
51.96	0.01\\
51.97	0.01\\
51.98	0.01\\
51.99	0.01\\
52	0.01\\
52.01	0.01\\
52.02	0.01\\
52.03	0.01\\
52.04	0.01\\
52.05	0.01\\
52.06	0.01\\
52.07	0.01\\
52.08	0.01\\
52.09	0.01\\
52.1	0.01\\
52.11	0.01\\
52.12	0.01\\
52.13	0.01\\
52.14	0.01\\
52.15	0.01\\
52.16	0.01\\
52.17	0.01\\
52.18	0.01\\
52.19	0.01\\
52.2	0.01\\
52.21	0.01\\
52.22	0.01\\
52.23	0.01\\
52.24	0.01\\
52.25	0.01\\
52.26	0.01\\
52.27	0.01\\
52.28	0.01\\
52.29	0.01\\
52.3	0.01\\
52.31	0.01\\
52.32	0.01\\
52.33	0.01\\
52.34	0.01\\
52.35	0.01\\
52.36	0.01\\
52.37	0.01\\
52.38	0.01\\
52.39	0.01\\
52.4	0.01\\
52.41	0.01\\
52.42	0.01\\
52.43	0.01\\
52.44	0.01\\
52.45	0.01\\
52.46	0.01\\
52.47	0.01\\
52.48	0.01\\
52.49	0.01\\
52.5	0.01\\
52.51	0.01\\
52.52	0.01\\
52.53	0.01\\
52.54	0.01\\
52.55	0.01\\
52.56	0.01\\
52.57	0.01\\
52.58	0.01\\
52.59	0.01\\
52.6	0.01\\
52.61	0.01\\
52.62	0.01\\
52.63	0.01\\
52.64	0.01\\
52.65	0.01\\
52.66	0.01\\
52.67	0.01\\
52.68	0.01\\
52.69	0.01\\
52.7	0.01\\
52.71	0.01\\
52.72	0.01\\
52.73	0.01\\
52.74	0.01\\
52.75	0.01\\
52.76	0.01\\
52.77	0.01\\
52.78	0.01\\
52.79	0.01\\
52.8	0.01\\
52.81	0.01\\
52.82	0.01\\
52.83	0.01\\
52.84	0.01\\
52.85	0.01\\
52.86	0.01\\
52.87	0.01\\
52.88	0.01\\
52.89	0.01\\
52.9	0.01\\
52.91	0.01\\
52.92	0.01\\
52.93	0.01\\
52.94	0.01\\
52.95	0.01\\
52.96	0.01\\
52.97	0.01\\
52.98	0.01\\
52.99	0.01\\
53	0.01\\
53.01	0.01\\
53.02	0.01\\
53.03	0.01\\
53.04	0.01\\
53.05	0.01\\
53.06	0.01\\
53.07	0.01\\
53.08	0.01\\
53.09	0.01\\
53.1	0.01\\
53.11	0.01\\
53.12	0.01\\
53.13	0.01\\
53.14	0.01\\
53.15	0.01\\
53.16	0.01\\
53.17	0.01\\
53.18	0.01\\
53.19	0.01\\
53.2	0.01\\
53.21	0.01\\
53.22	0.01\\
53.23	0.01\\
53.24	0.01\\
53.25	0.01\\
53.26	0.01\\
53.27	0.01\\
53.28	0.01\\
53.29	0.01\\
53.3	0.01\\
53.31	0.01\\
53.32	0.01\\
53.33	0.01\\
53.34	0.01\\
53.35	0.01\\
53.36	0.01\\
53.37	0.01\\
53.38	0.01\\
53.39	0.01\\
53.4	0.01\\
53.41	0.01\\
53.42	0.01\\
53.43	0.01\\
53.44	0.01\\
53.45	0.01\\
53.46	0.01\\
53.47	0.01\\
53.48	0.01\\
53.49	0.01\\
53.5	0.01\\
53.51	0.01\\
53.52	0.01\\
53.53	0.01\\
53.54	0.01\\
53.55	0.01\\
53.56	0.01\\
53.57	0.01\\
53.58	0.01\\
53.59	0.01\\
53.6	0.01\\
53.61	0.01\\
53.62	0.01\\
53.63	0.01\\
53.64	0.01\\
53.65	0.01\\
53.66	0.01\\
53.67	0.01\\
53.68	0.01\\
53.69	0.01\\
53.7	0.01\\
53.71	0.01\\
53.72	0.01\\
53.73	0.01\\
53.74	0.01\\
53.75	0.01\\
53.76	0.01\\
53.77	0.01\\
53.78	0.01\\
53.79	0.01\\
53.8	0.01\\
53.81	0.01\\
53.82	0.01\\
53.83	0.01\\
53.84	0.01\\
53.85	0.01\\
53.86	0.01\\
53.87	0.01\\
53.88	0.01\\
53.89	0.01\\
53.9	0.01\\
53.91	0.01\\
53.92	0.01\\
53.93	0.01\\
53.94	0.01\\
53.95	0.01\\
53.96	0.01\\
53.97	0.01\\
53.98	0.01\\
53.99	0.01\\
54	0.01\\
54.01	0.01\\
54.02	0.01\\
54.03	0.01\\
54.04	0.01\\
54.05	0.01\\
54.06	0.01\\
54.07	0.01\\
54.08	0.01\\
54.09	0.01\\
54.1	0.01\\
54.11	0.01\\
54.12	0.01\\
54.13	0.01\\
54.14	0.01\\
54.15	0.01\\
54.16	0.01\\
54.17	0.01\\
54.18	0.01\\
54.19	0.01\\
54.2	0.01\\
54.21	0.01\\
54.22	0.01\\
54.23	0.01\\
54.24	0.01\\
54.25	0.01\\
54.26	0.01\\
54.27	0.01\\
54.28	0.01\\
54.29	0.01\\
54.3	0.01\\
54.31	0.01\\
54.32	0.01\\
54.33	0.01\\
54.34	0.01\\
54.35	0.01\\
54.36	0.01\\
54.37	0.01\\
54.38	0.01\\
54.39	0.01\\
54.4	0.01\\
54.41	0.01\\
54.42	0.01\\
54.43	0.01\\
54.44	0.01\\
54.45	0.01\\
54.46	0.01\\
54.47	0.01\\
54.48	0.01\\
54.49	0.01\\
54.5	0.01\\
54.51	0.01\\
54.52	0.01\\
54.53	0.01\\
54.54	0.01\\
54.55	0.01\\
54.56	0.01\\
54.57	0.01\\
54.58	0.01\\
54.59	0.01\\
54.6	0.01\\
54.61	0.01\\
54.62	0.01\\
54.63	0.01\\
54.64	0.01\\
54.65	0.01\\
54.66	0.01\\
54.67	0.01\\
54.68	0.01\\
54.69	0.01\\
54.7	0.01\\
54.71	0.01\\
54.72	0.01\\
54.73	0.01\\
54.74	0.01\\
54.75	0.01\\
54.76	0.01\\
54.77	0.01\\
54.78	0.01\\
54.79	0.01\\
54.8	0.01\\
54.81	0.01\\
54.82	0.01\\
54.83	0.01\\
54.84	0.01\\
54.85	0.01\\
54.86	0.01\\
54.87	0.01\\
54.88	0.01\\
54.89	0.01\\
54.9	0.01\\
54.91	0.01\\
54.92	0.01\\
54.93	0.01\\
54.94	0.01\\
54.95	0.01\\
54.96	0.01\\
54.97	0.01\\
54.98	0.01\\
54.99	0.01\\
55	0.01\\
55.01	0.01\\
55.02	0.01\\
55.03	0.01\\
55.04	0.01\\
55.05	0.01\\
55.06	0.01\\
55.07	0.01\\
55.08	0.01\\
55.09	0.01\\
55.1	0.01\\
55.11	0.01\\
55.12	0.01\\
55.13	0.01\\
55.14	0.01\\
55.15	0.01\\
55.16	0.01\\
55.17	0.01\\
55.18	0.01\\
55.19	0.01\\
55.2	0.01\\
55.21	0.01\\
55.22	0.01\\
55.23	0.01\\
55.24	0.01\\
55.25	0.01\\
55.26	0.01\\
55.27	0.01\\
55.28	0.01\\
55.29	0.01\\
55.3	0.01\\
55.31	0.01\\
55.32	0.01\\
55.33	0.01\\
55.34	0.01\\
55.35	0.01\\
55.36	0.01\\
55.37	0.01\\
55.38	0.01\\
55.39	0.01\\
55.4	0.01\\
55.41	0.01\\
55.42	0.01\\
55.43	0.01\\
55.44	0.01\\
55.45	0.01\\
55.46	0.01\\
55.47	0.01\\
55.48	0.01\\
55.49	0.01\\
55.5	0.01\\
55.51	0.01\\
55.52	0.01\\
55.53	0.01\\
55.54	0.01\\
55.55	0.01\\
55.56	0.01\\
55.57	0.01\\
55.58	0.01\\
55.59	0.01\\
55.6	0.01\\
55.61	0.01\\
55.62	0.01\\
55.63	0.01\\
55.64	0.01\\
55.65	0.01\\
55.66	0.01\\
55.67	0.01\\
55.68	0.01\\
55.69	0.01\\
55.7	0.01\\
55.71	0.01\\
55.72	0.01\\
55.73	0.01\\
55.74	0.01\\
55.75	0.01\\
55.76	0.01\\
55.77	0.01\\
55.78	0.01\\
55.79	0.01\\
55.8	0.01\\
55.81	0.01\\
55.82	0.01\\
55.83	0.01\\
55.84	0.01\\
55.85	0.01\\
55.86	0.01\\
55.87	0.01\\
55.88	0.01\\
55.89	0.01\\
55.9	0.01\\
55.91	0.01\\
55.92	0.01\\
55.93	0.01\\
55.94	0.01\\
55.95	0.01\\
55.96	0.01\\
55.97	0.01\\
55.98	0.01\\
55.99	0.01\\
56	0.01\\
56.01	0.01\\
56.02	0.01\\
56.03	0.01\\
56.04	0.01\\
56.05	0.01\\
56.06	0.01\\
56.07	0.01\\
56.08	0.01\\
56.09	0.01\\
56.1	0.01\\
56.11	0.01\\
56.12	0.01\\
56.13	0.01\\
56.14	0.01\\
56.15	0.01\\
56.16	0.01\\
56.17	0.01\\
56.18	0.01\\
56.19	0.01\\
56.2	0.01\\
56.21	0.01\\
56.22	0.01\\
56.23	0.01\\
56.24	0.01\\
56.25	0.01\\
56.26	0.01\\
56.27	0.01\\
56.28	0.01\\
56.29	0.01\\
56.3	0.01\\
56.31	0.01\\
56.32	0.01\\
56.33	0.01\\
56.34	0.01\\
56.35	0.01\\
56.36	0.01\\
56.37	0.01\\
56.38	0.01\\
56.39	0.01\\
56.4	0.01\\
56.41	0.01\\
56.42	0.01\\
56.43	0.01\\
56.44	0.01\\
56.45	0.01\\
56.46	0.01\\
56.47	0.01\\
56.48	0.01\\
56.49	0.01\\
56.5	0.01\\
56.51	0.01\\
56.52	0.01\\
56.53	0.01\\
56.54	0.01\\
56.55	0.01\\
56.56	0.01\\
56.57	0.01\\
56.58	0.01\\
56.59	0.01\\
56.6	0.01\\
56.61	0.01\\
56.62	0.01\\
56.63	0.01\\
56.64	0.01\\
56.65	0.01\\
56.66	0.01\\
56.67	0.01\\
56.68	0.01\\
56.69	0.01\\
56.7	0.01\\
56.71	0.01\\
56.72	0.01\\
56.73	0.01\\
56.74	0.01\\
56.75	0.01\\
56.76	0.01\\
56.77	0.01\\
56.78	0.01\\
56.79	0.01\\
56.8	0.01\\
56.81	0.01\\
56.82	0.01\\
56.83	0.01\\
56.84	0.01\\
56.85	0.01\\
56.86	0.01\\
56.87	0.01\\
56.88	0.01\\
56.89	0.01\\
56.9	0.01\\
56.91	0.01\\
56.92	0.01\\
56.93	0.01\\
56.94	0.01\\
56.95	0.01\\
56.96	0.01\\
56.97	0.01\\
56.98	0.01\\
56.99	0.01\\
57	0.01\\
57.01	0.01\\
57.02	0.01\\
57.03	0.01\\
57.04	0.01\\
57.05	0.01\\
57.06	0.01\\
57.07	0.01\\
57.08	0.01\\
57.09	0.01\\
57.1	0.01\\
57.11	0.01\\
57.12	0.01\\
57.13	0.01\\
57.14	0.01\\
57.15	0.01\\
57.16	0.01\\
57.17	0.01\\
57.18	0.01\\
57.19	0.01\\
57.2	0.01\\
57.21	0.01\\
57.22	0.01\\
57.23	0.01\\
57.24	0.01\\
57.25	0.01\\
57.26	0.01\\
57.27	0.01\\
57.28	0.01\\
57.29	0.01\\
57.3	0.01\\
57.31	0.01\\
57.32	0.01\\
57.33	0.01\\
57.34	0.01\\
57.35	0.01\\
57.36	0.01\\
57.37	0.01\\
57.38	0.01\\
57.39	0.01\\
57.4	0.01\\
57.41	0.01\\
57.42	0.01\\
57.43	0.01\\
57.44	0.01\\
57.45	0.01\\
57.46	0.01\\
57.47	0.01\\
57.48	0.01\\
57.49	0.01\\
57.5	0.01\\
57.51	0.01\\
57.52	0.01\\
57.53	0.01\\
57.54	0.01\\
57.55	0.01\\
57.56	0.01\\
57.57	0.01\\
57.58	0.01\\
57.59	0.01\\
57.6	0.01\\
57.61	0.01\\
57.62	0.01\\
57.63	0.01\\
57.64	0.01\\
57.65	0.01\\
57.66	0.01\\
57.67	0.01\\
57.68	0.01\\
57.69	0.01\\
57.7	0.01\\
57.71	0.01\\
57.72	0.01\\
57.73	0.01\\
57.74	0.01\\
57.75	0.01\\
57.76	0.01\\
57.77	0.01\\
57.78	0.01\\
57.79	0.01\\
57.8	0.01\\
57.81	0.01\\
57.82	0.01\\
57.83	0.01\\
57.84	0.01\\
57.85	0.01\\
57.86	0.01\\
57.87	0.01\\
57.88	0.01\\
57.89	0.01\\
57.9	0.01\\
57.91	0.01\\
57.92	0.01\\
57.93	0.01\\
57.94	0.01\\
57.95	0.01\\
57.96	0.01\\
57.97	0.01\\
57.98	0.01\\
57.99	0.01\\
58	0.01\\
58.01	0.01\\
58.02	0.01\\
58.03	0.01\\
58.04	0.01\\
58.05	0.01\\
58.06	0.01\\
58.07	0.01\\
58.08	0.01\\
58.09	0.01\\
58.1	0.01\\
58.11	0.01\\
58.12	0.01\\
58.13	0.01\\
58.14	0.01\\
58.15	0.01\\
58.16	0.01\\
58.17	0.01\\
58.18	0.01\\
58.19	0.01\\
58.2	0.01\\
58.21	0.01\\
58.22	0.01\\
58.23	0.01\\
58.24	0.01\\
58.25	0.01\\
58.26	0.01\\
58.27	0.01\\
58.28	0.01\\
58.29	0.01\\
58.3	0.01\\
58.31	0.01\\
58.32	0.01\\
58.33	0.01\\
58.34	0.01\\
58.35	0.01\\
58.36	0.01\\
58.37	0.01\\
58.38	0.01\\
58.39	0.01\\
58.4	0.01\\
58.41	0.01\\
58.42	0.01\\
58.43	0.01\\
58.44	0.01\\
58.45	0.01\\
58.46	0.01\\
58.47	0.01\\
58.48	0.01\\
58.49	0.01\\
58.5	0.01\\
58.51	0.01\\
58.52	0.01\\
58.53	0.01\\
58.54	0.01\\
58.55	0.01\\
58.56	0.01\\
58.57	0.01\\
58.58	0.01\\
58.59	0.01\\
58.6	0.01\\
58.61	0.01\\
58.62	0.01\\
58.63	0.01\\
58.64	0.01\\
58.65	0.01\\
58.66	0.01\\
58.67	0.01\\
58.68	0.01\\
58.69	0.01\\
58.7	0.01\\
58.71	0.01\\
58.72	0.01\\
58.73	0.01\\
58.74	0.01\\
58.75	0.01\\
58.76	0.01\\
58.77	0.01\\
58.78	0.01\\
58.79	0.01\\
58.8	0.01\\
58.81	0.01\\
58.82	0.01\\
58.83	0.01\\
58.84	0.01\\
58.85	0.01\\
58.86	0.01\\
58.87	0.01\\
58.88	0.01\\
58.89	0.01\\
58.9	0.01\\
58.91	0.01\\
58.92	0.01\\
58.93	0.01\\
58.94	0.01\\
58.95	0.01\\
58.96	0.01\\
58.97	0.01\\
58.98	0.01\\
58.99	0.01\\
59	0.01\\
59.01	0.01\\
59.02	0.01\\
59.03	0.01\\
59.04	0.01\\
59.05	0.01\\
59.06	0.01\\
59.07	0.01\\
59.08	0.01\\
59.09	0.01\\
59.1	0.01\\
59.11	0.01\\
59.12	0.01\\
59.13	0.01\\
59.14	0.01\\
59.15	0.01\\
59.16	0.01\\
59.17	0.01\\
59.18	0.01\\
59.19	0.01\\
59.2	0.01\\
59.21	0.01\\
59.22	0.01\\
59.23	0.01\\
59.24	0.01\\
59.25	0.01\\
59.26	0.01\\
59.27	0.01\\
59.28	0.01\\
59.29	0.01\\
59.3	0.01\\
59.31	0.01\\
59.32	0.01\\
59.33	0.01\\
59.34	0.01\\
59.35	0.01\\
59.36	0.01\\
59.37	0.01\\
59.38	0.01\\
59.39	0.01\\
59.4	0.01\\
59.41	0.01\\
59.42	0.01\\
59.43	0.01\\
59.44	0.01\\
59.45	0.01\\
59.46	0.01\\
59.47	0.01\\
59.48	0.01\\
59.49	0.01\\
59.5	0.01\\
59.51	0.01\\
59.52	0.01\\
59.53	0.01\\
59.54	0.01\\
59.55	0.01\\
59.56	0.01\\
59.57	0.01\\
59.58	0.01\\
59.59	0.01\\
59.6	0.01\\
59.61	0.01\\
59.62	0.01\\
59.63	0.01\\
59.64	0.01\\
59.65	0.01\\
59.66	0.01\\
59.67	0.01\\
59.68	0.01\\
59.69	0.01\\
59.7	0.01\\
59.71	0.01\\
59.72	0.01\\
59.73	0.01\\
59.74	0.01\\
59.75	0.01\\
59.76	0.01\\
59.77	0.01\\
59.78	0.01\\
59.79	0.01\\
59.8	0.01\\
59.81	0.01\\
59.82	0.01\\
59.83	0.01\\
59.84	0.01\\
59.85	0.01\\
59.86	0.01\\
59.87	0.01\\
59.88	0.01\\
59.89	0.01\\
59.9	0.01\\
59.91	0.01\\
59.92	0.01\\
59.93	0.01\\
59.94	0.01\\
59.95	0.01\\
59.96	0.01\\
59.97	0.01\\
59.98	0.01\\
59.99	0.01\\
60	0.01\\
60.01	0.01\\
60.02	0.01\\
60.03	0.01\\
60.04	0.01\\
60.05	0.01\\
60.06	0.01\\
60.07	0.01\\
60.08	0.01\\
60.09	0.01\\
60.1	0.01\\
60.11	0.01\\
60.12	0.01\\
60.13	0.01\\
60.14	0.01\\
60.15	0.01\\
60.16	0.01\\
60.17	0.01\\
60.18	0.01\\
60.19	0.01\\
60.2	0.01\\
60.21	0.01\\
60.22	0.01\\
60.23	0.01\\
60.24	0.01\\
60.25	0.01\\
60.26	0.01\\
60.27	0.01\\
60.28	0.01\\
60.29	0.01\\
60.3	0.01\\
60.31	0.01\\
60.32	0.01\\
60.33	0.01\\
60.34	0.01\\
60.35	0.01\\
60.36	0.01\\
60.37	0.01\\
60.38	0.01\\
60.39	0.01\\
60.4	0.01\\
60.41	0.01\\
60.42	0.01\\
60.43	0.01\\
60.44	0.01\\
60.45	0.01\\
60.46	0.01\\
60.47	0.01\\
60.48	0.01\\
60.49	0.01\\
60.5	0.01\\
60.51	0.01\\
60.52	0.01\\
60.53	0.01\\
60.54	0.01\\
60.55	0.01\\
60.56	0.01\\
60.57	0.01\\
60.58	0.01\\
60.59	0.01\\
60.6	0.01\\
60.61	0.01\\
60.62	0.01\\
60.63	0.01\\
60.64	0.01\\
60.65	0.01\\
60.66	0.01\\
60.67	0.01\\
60.68	0.01\\
60.69	0.01\\
60.7	0.01\\
60.71	0.01\\
60.72	0.01\\
60.73	0.01\\
60.74	0.01\\
60.75	0.01\\
60.76	0.01\\
60.77	0.01\\
60.78	0.01\\
60.79	0.01\\
60.8	0.01\\
60.81	0.01\\
60.82	0.01\\
60.83	0.01\\
60.84	0.01\\
60.85	0.01\\
60.86	0.01\\
60.87	0.01\\
60.88	0.01\\
60.89	0.01\\
60.9	0.01\\
60.91	0.01\\
60.92	0.01\\
60.93	0.01\\
60.94	0.01\\
60.95	0.01\\
60.96	0.01\\
60.97	0.01\\
60.98	0.01\\
60.99	0.01\\
61	0.01\\
61.01	0.01\\
61.02	0.01\\
61.03	0.01\\
61.04	0.01\\
61.05	0.01\\
61.06	0.01\\
61.07	0.01\\
61.08	0.01\\
61.09	0.01\\
61.1	0.01\\
61.11	0.01\\
61.12	0.01\\
61.13	0.01\\
61.14	0.01\\
61.15	0.01\\
61.16	0.01\\
61.17	0.01\\
61.18	0.01\\
61.19	0.01\\
61.2	0.01\\
61.21	0.01\\
61.22	0.01\\
61.23	0.01\\
61.24	0.01\\
61.25	0.01\\
61.26	0.01\\
61.27	0.01\\
61.28	0.01\\
61.29	0.01\\
61.3	0.01\\
61.31	0.01\\
61.32	0.01\\
61.33	0.01\\
61.34	0.01\\
61.35	0.01\\
61.36	0.01\\
61.37	0.01\\
61.38	0.01\\
61.39	0.01\\
61.4	0.01\\
61.41	0.01\\
61.42	0.01\\
61.43	0.01\\
61.44	0.01\\
61.45	0.01\\
61.46	0.01\\
61.47	0.01\\
61.48	0.01\\
61.49	0.01\\
61.5	0.01\\
61.51	0.01\\
61.52	0.01\\
61.53	0.01\\
61.54	0.01\\
61.55	0.01\\
61.56	0.01\\
61.57	0.01\\
61.58	0.01\\
61.59	0.01\\
61.6	0.01\\
61.61	0.01\\
61.62	0.01\\
61.63	0.01\\
61.64	0.01\\
61.65	0.01\\
61.66	0.01\\
61.67	0.01\\
61.68	0.01\\
61.69	0.01\\
61.7	0.01\\
61.71	0.01\\
61.72	0.01\\
61.73	0.01\\
61.74	0.01\\
61.75	0.01\\
61.76	0.01\\
61.77	0.01\\
61.78	0.01\\
61.79	0.01\\
61.8	0.01\\
61.81	0.01\\
61.82	0.01\\
61.83	0.01\\
61.84	0.01\\
61.85	0.01\\
61.86	0.01\\
61.87	0.01\\
61.88	0.01\\
61.89	0.01\\
61.9	0.01\\
61.91	0.01\\
61.92	0.01\\
61.93	0.01\\
61.94	0.01\\
61.95	0.01\\
61.96	0.01\\
61.97	0.01\\
61.98	0.01\\
61.99	0.01\\
62	0.01\\
62.01	0.01\\
62.02	0.01\\
62.03	0.01\\
62.04	0.01\\
62.05	0.01\\
62.06	0.01\\
62.07	0.01\\
62.08	0.01\\
62.09	0.01\\
62.1	0.01\\
62.11	0.01\\
62.12	0.01\\
62.13	0.01\\
62.14	0.01\\
62.15	0.01\\
62.16	0.01\\
62.17	0.01\\
62.18	0.01\\
62.19	0.01\\
62.2	0.01\\
62.21	0.01\\
62.22	0.01\\
62.23	0.01\\
62.24	0.01\\
62.25	0.01\\
62.26	0.01\\
62.27	0.01\\
62.28	0.01\\
62.29	0.01\\
62.3	0.01\\
62.31	0.01\\
62.32	0.01\\
62.33	0.01\\
62.34	0.01\\
62.35	0.01\\
62.36	0.01\\
62.37	0.01\\
62.38	0.01\\
62.39	0.01\\
62.4	0.01\\
62.41	0.01\\
62.42	0.01\\
62.43	0.01\\
62.44	0.01\\
62.45	0.01\\
62.46	0.01\\
62.47	0.01\\
62.48	0.01\\
62.49	0.01\\
62.5	0.01\\
62.51	0.01\\
62.52	0.01\\
62.53	0.01\\
62.54	0.01\\
62.55	0.01\\
62.56	0.01\\
62.57	0.01\\
62.58	0.01\\
62.59	0.01\\
62.6	0.01\\
62.61	0.01\\
62.62	0.01\\
62.63	0.01\\
62.64	0.01\\
62.65	0.01\\
62.66	0.01\\
62.67	0.01\\
62.68	0.01\\
62.69	0.01\\
62.7	0.01\\
62.71	0.01\\
62.72	0.01\\
62.73	0.01\\
62.74	0.01\\
62.75	0.01\\
62.76	0.01\\
62.77	0.01\\
62.78	0.01\\
62.79	0.01\\
62.8	0.01\\
62.81	0.01\\
62.82	0.01\\
62.83	0.01\\
62.84	0.01\\
62.85	0.01\\
62.86	0.01\\
62.87	0.01\\
62.88	0.01\\
62.89	0.01\\
62.9	0.01\\
62.91	0.01\\
62.92	0.01\\
62.93	0.01\\
62.94	0.01\\
62.95	0.01\\
62.96	0.01\\
62.97	0.01\\
62.98	0.01\\
62.99	0.01\\
63	0.01\\
63.01	0.01\\
63.02	0.01\\
63.03	0.01\\
63.04	0.01\\
63.05	0.01\\
63.06	0.01\\
63.07	0.01\\
63.08	0.01\\
63.09	0.01\\
63.1	0.01\\
63.11	0.01\\
63.12	0.01\\
63.13	0.01\\
63.14	0.01\\
63.15	0.01\\
63.16	0.01\\
63.17	0.01\\
63.18	0.01\\
63.19	0.01\\
63.2	0.01\\
63.21	0.01\\
63.22	0.01\\
63.23	0.01\\
63.24	0.01\\
63.25	0.01\\
63.26	0.01\\
63.27	0.01\\
63.28	0.01\\
63.29	0.01\\
63.3	0.01\\
63.31	0.01\\
63.32	0.01\\
63.33	0.01\\
63.34	0.01\\
63.35	0.01\\
63.36	0.01\\
63.37	0.01\\
63.38	0.01\\
63.39	0.01\\
63.4	0.01\\
63.41	0.01\\
63.42	0.01\\
63.43	0.01\\
63.44	0.01\\
63.45	0.01\\
63.46	0.01\\
63.47	0.01\\
63.48	0.01\\
63.49	0.01\\
63.5	0.01\\
63.51	0.01\\
63.52	0.01\\
63.53	0.01\\
63.54	0.01\\
63.55	0.01\\
63.56	0.01\\
63.57	0.01\\
63.58	0.01\\
63.59	0.01\\
63.6	0.01\\
63.61	0.01\\
63.62	0.01\\
63.63	0.01\\
63.64	0.01\\
63.65	0.01\\
63.66	0.01\\
63.67	0.01\\
63.68	0.01\\
63.69	0.01\\
63.7	0.01\\
63.71	0.01\\
63.72	0.01\\
63.73	0.01\\
63.74	0.01\\
63.75	0.01\\
63.76	0.01\\
63.77	0.01\\
63.78	0.01\\
63.79	0.01\\
63.8	0.01\\
63.81	0.01\\
63.82	0.01\\
63.83	0.01\\
63.84	0.01\\
63.85	0.01\\
63.86	0.01\\
63.87	0.01\\
63.88	0.01\\
63.89	0.01\\
63.9	0.01\\
63.91	0.01\\
63.92	0.01\\
63.93	0.01\\
63.94	0.01\\
63.95	0.01\\
63.96	0.01\\
63.97	0.01\\
63.98	0.01\\
63.99	0.01\\
64	0.01\\
64.01	0.01\\
64.02	0.01\\
64.03	0.01\\
64.04	0.01\\
64.05	0.01\\
64.06	0.01\\
64.07	0.01\\
64.08	0.01\\
64.09	0.01\\
64.1	0.01\\
64.11	0.01\\
64.12	0.01\\
64.13	0.01\\
64.14	0.01\\
64.15	0.01\\
64.16	0.01\\
64.17	0.01\\
64.18	0.01\\
64.19	0.01\\
64.2	0.01\\
64.21	0.01\\
64.22	0.01\\
64.23	0.01\\
64.24	0.01\\
64.25	0.01\\
64.26	0.01\\
64.27	0.01\\
64.28	0.01\\
64.29	0.01\\
64.3	0.01\\
64.31	0.01\\
64.32	0.01\\
64.33	0.01\\
64.34	0.01\\
64.35	0.01\\
64.36	0.01\\
64.37	0.01\\
64.38	0.01\\
64.39	0.01\\
64.4	0.01\\
64.41	0.01\\
64.42	0.01\\
64.43	0.01\\
64.44	0.01\\
64.45	0.01\\
64.46	0.01\\
64.47	0.01\\
64.48	0.01\\
64.49	0.01\\
64.5	0.01\\
64.51	0.01\\
64.52	0.01\\
64.53	0.01\\
64.54	0.01\\
64.55	0.01\\
64.56	0.01\\
64.57	0.01\\
64.58	0.01\\
64.59	0.01\\
64.6	0.01\\
64.61	0.01\\
64.62	0.01\\
64.63	0.01\\
64.64	0.01\\
64.65	0.01\\
64.66	0.01\\
64.67	0.01\\
64.68	0.01\\
64.69	0.01\\
64.7	0.01\\
64.71	0.01\\
64.72	0.01\\
64.73	0.01\\
64.74	0.01\\
64.75	0.01\\
64.76	0.01\\
64.77	0.01\\
64.78	0.01\\
64.79	0.01\\
64.8	0.01\\
64.81	0.01\\
64.82	0.01\\
64.83	0.01\\
64.84	0.01\\
64.85	0.01\\
64.86	0.01\\
64.87	0.01\\
64.88	0.01\\
64.89	0.01\\
64.9	0.01\\
64.91	0.01\\
64.92	0.01\\
64.93	0.01\\
64.94	0.01\\
64.95	0.01\\
64.96	0.01\\
64.97	0.01\\
64.98	0.01\\
64.99	0.01\\
65	0.01\\
65.01	0.01\\
65.02	0.01\\
65.03	0.01\\
65.04	0.01\\
65.05	0.01\\
65.06	0.01\\
65.07	0.01\\
65.08	0.01\\
65.09	0.01\\
65.1	0.01\\
65.11	0.01\\
65.12	0.01\\
65.13	0.01\\
65.14	0.01\\
65.15	0.01\\
65.16	0.01\\
65.17	0.01\\
65.18	0.01\\
65.19	0.01\\
65.2	0.01\\
65.21	0.01\\
65.22	0.01\\
65.23	0.01\\
65.24	0.01\\
65.25	0.01\\
65.26	0.01\\
65.27	0.01\\
65.28	0.01\\
65.29	0.01\\
65.3	0.01\\
65.31	0.01\\
65.32	0.01\\
65.33	0.01\\
65.34	0.01\\
65.35	0.01\\
65.36	0.01\\
65.37	0.01\\
65.38	0.01\\
65.39	0.01\\
65.4	0.01\\
65.41	0.01\\
65.42	0.01\\
65.43	0.01\\
65.44	0.01\\
65.45	0.01\\
65.46	0.01\\
65.47	0.01\\
65.48	0.01\\
65.49	0.01\\
65.5	0.01\\
65.51	0.01\\
65.52	0.01\\
65.53	0.01\\
65.54	0.01\\
65.55	0.01\\
65.56	0.01\\
65.57	0.01\\
65.58	0.01\\
65.59	0.01\\
65.6	0.01\\
65.61	0.01\\
65.62	0.01\\
65.63	0.01\\
65.64	0.01\\
65.65	0.01\\
65.66	0.01\\
65.67	0.01\\
65.68	0.01\\
65.69	0.01\\
65.7	0.01\\
65.71	0.01\\
65.72	0.01\\
65.73	0.01\\
65.74	0.01\\
65.75	0.01\\
65.76	0.01\\
65.77	0.01\\
65.78	0.01\\
65.79	0.01\\
65.8	0.01\\
65.81	0.01\\
65.82	0.01\\
65.83	0.01\\
65.84	0.01\\
65.85	0.01\\
65.86	0.01\\
65.87	0.01\\
65.88	0.01\\
65.89	0.01\\
65.9	0.01\\
65.91	0.01\\
65.92	0.01\\
65.93	0.01\\
65.94	0.01\\
65.95	0.01\\
65.96	0.01\\
65.97	0.01\\
65.98	0.01\\
65.99	0.01\\
66	0.01\\
66.01	0.01\\
66.02	0.01\\
66.03	0.01\\
66.04	0.01\\
66.05	0.01\\
66.06	0.01\\
66.07	0.01\\
66.08	0.01\\
66.09	0.01\\
66.1	0.01\\
66.11	0.01\\
66.12	0.01\\
66.13	0.01\\
66.14	0.01\\
66.15	0.01\\
66.16	0.01\\
66.17	0.01\\
66.18	0.01\\
66.19	0.01\\
66.2	0.01\\
66.21	0.01\\
66.22	0.01\\
66.23	0.01\\
66.24	0.01\\
66.25	0.01\\
66.26	0.01\\
66.27	0.01\\
66.28	0.01\\
66.29	0.01\\
66.3	0.01\\
66.31	0.01\\
66.32	0.01\\
66.33	0.01\\
66.34	0.01\\
66.35	0.01\\
66.36	0.01\\
66.37	0.01\\
66.38	0.01\\
66.39	0.01\\
66.4	0.01\\
66.41	0.01\\
66.42	0.01\\
66.43	0.01\\
66.44	0.01\\
66.45	0.01\\
66.46	0.01\\
66.47	0.01\\
66.48	0.01\\
66.49	0.01\\
66.5	0.01\\
66.51	0.01\\
66.52	0.01\\
66.53	0.01\\
66.54	0.01\\
66.55	0.01\\
66.56	0.01\\
66.57	0.01\\
66.58	0.01\\
66.59	0.01\\
66.6	0.01\\
66.61	0.01\\
66.62	0.01\\
66.63	0.01\\
66.64	0.01\\
66.65	0.01\\
66.66	0.01\\
66.67	0.01\\
66.68	0.01\\
66.69	0.01\\
66.7	0.01\\
66.71	0.01\\
66.72	0.01\\
66.73	0.01\\
66.74	0.01\\
66.75	0.01\\
66.76	0.01\\
66.77	0.01\\
66.78	0.01\\
66.79	0.01\\
66.8	0.01\\
66.81	0.01\\
66.82	0.01\\
66.83	0.01\\
66.84	0.01\\
66.85	0.01\\
66.86	0.01\\
66.87	0.01\\
66.88	0.01\\
66.89	0.01\\
66.9	0.01\\
66.91	0.01\\
66.92	0.01\\
66.93	0.01\\
66.94	0.01\\
66.95	0.01\\
66.96	0.01\\
66.97	0.01\\
66.98	0.01\\
66.99	0.01\\
67	0.01\\
67.01	0.01\\
67.02	0.01\\
67.03	0.01\\
67.04	0.01\\
67.05	0.01\\
67.06	0.01\\
67.07	0.01\\
67.08	0.01\\
67.09	0.01\\
67.1	0.01\\
67.11	0.01\\
67.12	0.01\\
67.13	0.01\\
67.14	0.01\\
67.15	0.01\\
67.16	0.01\\
67.17	0.01\\
67.18	0.01\\
67.19	0.01\\
67.2	0.01\\
67.21	0.01\\
67.22	0.01\\
67.23	0.01\\
67.24	0.01\\
67.25	0.01\\
67.26	0.01\\
67.27	0.01\\
67.28	0.01\\
67.29	0.01\\
67.3	0.01\\
67.31	0.01\\
67.32	0.01\\
67.33	0.01\\
67.34	0.01\\
67.35	0.01\\
67.36	0.01\\
67.37	0.01\\
67.38	0.01\\
67.39	0.01\\
67.4	0.01\\
67.41	0.01\\
67.42	0.01\\
67.43	0.01\\
67.44	0.01\\
67.45	0.01\\
67.46	0.01\\
67.47	0.01\\
67.48	0.01\\
67.49	0.01\\
67.5	0.01\\
67.51	0.01\\
67.52	0.01\\
67.53	0.01\\
67.54	0.01\\
67.55	0.01\\
67.56	0.01\\
67.57	0.01\\
67.58	0.01\\
67.59	0.01\\
67.6	0.01\\
67.61	0.01\\
67.62	0.01\\
67.63	0.01\\
67.64	0.01\\
67.65	0.01\\
67.66	0.01\\
67.67	0.01\\
67.68	0.01\\
67.69	0.01\\
67.7	0.01\\
67.71	0.01\\
67.72	0.01\\
67.73	0.01\\
67.74	0.01\\
67.75	0.01\\
67.76	0.01\\
67.77	0.01\\
67.78	0.01\\
67.79	0.01\\
67.8	0.01\\
67.81	0.01\\
67.82	0.01\\
67.83	0.01\\
67.84	0.01\\
67.85	0.01\\
67.86	0.01\\
67.87	0.01\\
67.88	0.01\\
67.89	0.01\\
67.9	0.01\\
67.91	0.01\\
67.92	0.01\\
67.93	0.01\\
67.94	0.01\\
67.95	0.01\\
67.96	0.01\\
67.97	0.01\\
67.98	0.01\\
67.99	0.01\\
68	0.01\\
68.01	0.01\\
68.02	0.01\\
68.03	0.01\\
68.04	0.01\\
68.05	0.01\\
68.06	0.01\\
68.07	0.01\\
68.08	0.01\\
68.09	0.01\\
68.1	0.01\\
68.11	0.01\\
68.12	0.01\\
68.13	0.01\\
68.14	0.01\\
68.15	0.01\\
68.16	0.01\\
68.17	0.01\\
68.18	0.01\\
68.19	0.01\\
68.2	0.01\\
68.21	0.01\\
68.22	0.01\\
68.23	0.01\\
68.24	0.01\\
68.25	0.01\\
68.26	0.01\\
68.27	0.01\\
68.28	0.01\\
68.29	0.01\\
68.3	0.01\\
68.31	0.01\\
68.32	0.01\\
68.33	0.01\\
68.34	0.01\\
68.35	0.01\\
68.36	0.01\\
68.37	0.01\\
68.38	0.01\\
68.39	0.01\\
68.4	0.01\\
68.41	0.01\\
68.42	0.01\\
68.43	0.01\\
68.44	0.01\\
68.45	0.01\\
68.46	0.01\\
68.47	0.01\\
68.48	0.01\\
68.49	0.01\\
68.5	0.01\\
68.51	0.01\\
68.52	0.01\\
68.53	0.01\\
68.54	0.01\\
68.55	0.01\\
68.56	0.01\\
68.57	0.01\\
68.58	0.01\\
68.59	0.01\\
68.6	0.01\\
68.61	0.01\\
68.62	0.01\\
68.63	0.01\\
68.64	0.01\\
68.65	0.01\\
68.66	0.01\\
68.67	0.01\\
68.68	0.01\\
68.69	0.01\\
68.7	0.01\\
68.71	0.01\\
68.72	0.01\\
68.73	0.01\\
68.74	0.01\\
68.75	0.01\\
68.76	0.01\\
68.77	0.01\\
68.78	0.01\\
68.79	0.01\\
68.8	0.01\\
68.81	0.01\\
68.82	0.01\\
68.83	0.01\\
68.84	0.01\\
68.85	0.01\\
68.86	0.01\\
68.87	0.01\\
68.88	0.01\\
68.89	0.01\\
68.9	0.01\\
68.91	0.01\\
68.92	0.01\\
68.93	0.01\\
68.94	0.01\\
68.95	0.01\\
68.96	0.01\\
68.97	0.01\\
68.98	0.01\\
68.99	0.01\\
69	0.01\\
69.01	0.01\\
69.02	0.01\\
69.03	0.01\\
69.04	0.01\\
69.05	0.01\\
69.06	0.01\\
69.07	0.01\\
69.08	0.01\\
69.09	0.01\\
69.1	0.01\\
69.11	0.01\\
69.12	0.01\\
69.13	0.01\\
69.14	0.01\\
69.15	0.01\\
69.16	0.01\\
69.17	0.01\\
69.18	0.01\\
69.19	0.01\\
69.2	0.01\\
69.21	0.01\\
69.22	0.01\\
69.23	0.01\\
69.24	0.01\\
69.25	0.01\\
69.26	0.01\\
69.27	0.01\\
69.28	0.01\\
69.29	0.01\\
69.3	0.01\\
69.31	0.01\\
69.32	0.01\\
69.33	0.01\\
69.34	0.01\\
69.35	0.01\\
69.36	0.01\\
69.37	0.01\\
69.38	0.01\\
69.39	0.01\\
69.4	0.01\\
69.41	0.01\\
69.42	0.01\\
69.43	0.01\\
69.44	0.01\\
69.45	0.01\\
69.46	0.01\\
69.47	0.01\\
69.48	0.01\\
69.49	0.01\\
69.5	0.01\\
69.51	0.01\\
69.52	0.01\\
69.53	0.01\\
69.54	0.01\\
69.55	0.01\\
69.56	0.01\\
69.57	0.01\\
69.58	0.01\\
69.59	0.01\\
69.6	0.01\\
69.61	0.01\\
69.62	0.01\\
69.63	0.01\\
69.64	0.01\\
69.65	0.01\\
69.66	0.01\\
69.67	0.01\\
69.68	0.01\\
69.69	0.01\\
69.7	0.01\\
69.71	0.01\\
69.72	0.01\\
69.73	0.01\\
69.74	0.01\\
69.75	0.01\\
69.76	0.01\\
69.77	0.01\\
69.78	0.01\\
69.79	0.01\\
69.8	0.01\\
69.81	0.01\\
69.82	0.01\\
69.83	0.01\\
69.84	0.01\\
69.85	0.01\\
69.86	0.01\\
69.87	0.01\\
69.88	0.01\\
69.89	0.01\\
69.9	0.01\\
69.91	0.01\\
69.92	0.01\\
69.93	0.01\\
69.94	0.01\\
69.95	0.01\\
69.96	0.01\\
69.97	0.01\\
69.98	0.01\\
69.99	0.01\\
70	0.01\\
70.01	0.01\\
70.02	0.01\\
70.03	0.01\\
70.04	0.01\\
70.05	0.01\\
70.06	0.01\\
70.07	0.01\\
70.08	0.01\\
70.09	0.01\\
70.1	0.01\\
70.11	0.01\\
70.12	0.01\\
70.13	0.01\\
70.14	0.01\\
70.15	0.01\\
70.16	0.01\\
70.17	0.01\\
70.18	0.01\\
70.19	0.01\\
70.2	0.01\\
70.21	0.01\\
70.22	0.01\\
70.23	0.01\\
70.24	0.01\\
70.25	0.01\\
70.26	0.01\\
70.27	0.01\\
70.28	0.01\\
70.29	0.01\\
70.3	0.01\\
70.31	0.01\\
70.32	0.01\\
70.33	0.01\\
70.34	0.01\\
70.35	0.01\\
70.36	0.01\\
70.37	0.01\\
70.38	0.01\\
70.39	0.01\\
70.4	0.01\\
70.41	0.01\\
70.42	0.01\\
70.43	0.01\\
70.44	0.01\\
70.45	0.01\\
70.46	0.01\\
70.47	0.01\\
70.48	0.01\\
70.49	0.01\\
70.5	0.01\\
70.51	0.01\\
70.52	0.01\\
70.53	0.01\\
70.54	0.01\\
70.55	0.01\\
70.56	0.01\\
70.57	0.01\\
70.58	0.01\\
70.59	0.01\\
70.6	0.01\\
70.61	0.01\\
70.62	0.01\\
70.63	0.01\\
70.64	0.01\\
70.65	0.01\\
70.66	0.01\\
70.67	0.01\\
70.68	0.01\\
70.69	0.01\\
70.7	0.01\\
70.71	0.01\\
70.72	0.01\\
70.73	0.01\\
70.74	0.01\\
70.75	0.01\\
70.76	0.01\\
70.77	0.01\\
70.78	0.01\\
70.79	0.01\\
70.8	0.01\\
70.81	0.01\\
70.82	0.01\\
70.83	0.01\\
70.84	0.01\\
70.85	0.01\\
70.86	0.01\\
70.87	0.01\\
70.88	0.01\\
70.89	0.01\\
70.9	0.01\\
70.91	0.01\\
70.92	0.01\\
70.93	0.01\\
70.94	0.01\\
70.95	0.01\\
70.96	0.01\\
70.97	0.01\\
70.98	0.01\\
70.99	0.01\\
71	0.01\\
71.01	0.01\\
71.02	0.01\\
71.03	0.01\\
71.04	0.01\\
71.05	0.01\\
71.06	0.01\\
71.07	0.01\\
71.08	0.01\\
71.09	0.01\\
71.1	0.01\\
71.11	0.01\\
71.12	0.01\\
71.13	0.01\\
71.14	0.01\\
71.15	0.01\\
71.16	0.01\\
71.17	0.01\\
71.18	0.01\\
71.19	0.01\\
71.2	0.01\\
71.21	0.01\\
71.22	0.01\\
71.23	0.01\\
71.24	0.01\\
71.25	0.01\\
71.26	0.01\\
71.27	0.01\\
71.28	0.01\\
71.29	0.01\\
71.3	0.01\\
71.31	0.01\\
71.32	0.01\\
71.33	0.01\\
71.34	0.01\\
71.35	0.01\\
71.36	0.01\\
71.37	0.01\\
71.38	0.01\\
71.39	0.01\\
71.4	0.01\\
71.41	0.01\\
71.42	0.01\\
71.43	0.01\\
71.44	0.01\\
71.45	0.01\\
71.46	0.01\\
71.47	0.01\\
71.48	0.01\\
71.49	0.01\\
71.5	0.01\\
71.51	0.01\\
71.52	0.01\\
71.53	0.01\\
71.54	0.01\\
71.55	0.01\\
71.56	0.01\\
71.57	0.01\\
71.58	0.01\\
71.59	0.01\\
71.6	0.01\\
71.61	0.01\\
71.62	0.01\\
71.63	0.00999992261345002\\
71.64	0.00999734872975903\\
71.65	0.00999477269224675\\
71.66	0.00999219449737882\\
71.67	0.00998961414161215\\
71.68	0.00998703162139485\\
71.69	0.00998444693316621\\
71.7	0.00998186007335671\\
71.71	0.0099792710383879\\
71.72	0.00997667982467244\\
71.73	0.00997408642861399\\
71.74	0.00997149084660726\\
71.75	0.00996889307503789\\
71.76	0.00996629311028248\\
71.77	0.0099636909487085\\
71.78	0.00996108658667428\\
71.79	0.00995848002052897\\
71.8	0.0099558712466125\\
71.81	0.00995326026125554\\
71.82	0.00995064706077944\\
71.83	0.00994803164149624\\
71.84	0.00994541399970859\\
71.85	0.00994279413170971\\
71.86	0.00994017203378336\\
71.87	0.00993754770220384\\
71.88	0.00993492113323585\\
71.89	0.00993229232313455\\
71.9	0.00992966126814547\\
71.91	0.00992702796450446\\
71.92	0.00992439240843766\\
71.93	0.00992175459616149\\
71.94	0.00991911452388256\\
71.95	0.00991647218779762\\
71.96	0.00991382758409356\\
71.97	0.00991118070894736\\
71.98	0.00990853155852601\\
71.99	0.0099058801289865\\
72	0.00990322641647573\\
72.01	0.00990057041713054\\
72.02	0.00989791212707759\\
72.03	0.00989525154243336\\
72.04	0.00989258865930407\\
72.05	0.00988992347378565\\
72.06	0.00988725598196371\\
72.07	0.00988458617991347\\
72.08	0.00988191406369969\\
72.09	0.00987923962937668\\
72.1	0.0098765628729882\\
72.11	0.00987388379056743\\
72.12	0.00987120237813691\\
72.13	0.00986851863170853\\
72.14	0.00986583254728339\\
72.15	0.00986314412085186\\
72.16	0.00986045334839344\\
72.17	0.00985776022587675\\
72.18	0.00985506474925948\\
72.19	0.00985236691448831\\
72.2	0.00984966671749887\\
72.21	0.00984696415421569\\
72.22	0.00984425922055216\\
72.23	0.00984155191241042\\
72.24	0.00983884222568136\\
72.25	0.00983613015624457\\
72.26	0.00983341569996822\\
72.27	0.00983069885270906\\
72.28	0.00982797961031234\\
72.29	0.00982525796861174\\
72.3	0.00982253392342936\\
72.31	0.00981980747057559\\
72.32	0.00981707860584911\\
72.33	0.0098143473250368\\
72.34	0.00981161362391368\\
72.35	0.00980887749824287\\
72.36	0.00980613894377549\\
72.37	0.00980339795625062\\
72.38	0.00980065453139525\\
72.39	0.00979790866492419\\
72.4	0.00979516035254002\\
72.41	0.00979240958993303\\
72.42	0.00978965637278111\\
72.43	0.00978690069674976\\
72.44	0.00978414255749197\\
72.45	0.00978138195064814\\
72.46	0.00977861887184607\\
72.47	0.00977585331670083\\
72.48	0.00977308528081473\\
72.49	0.00977031475977722\\
72.5	0.00976754174916486\\
72.51	0.0097647662445412\\
72.52	0.00976198824145673\\
72.53	0.00975920773544882\\
72.54	0.00975642472204163\\
72.55	0.00975363919674603\\
72.56	0.00975085115505953\\
72.57	0.00974806059246622\\
72.58	0.00974526750443666\\
72.59	0.00974247188642784\\
72.6	0.00973967373388307\\
72.61	0.00973687304223191\\
72.62	0.00973406980689012\\
72.63	0.0097312640232595\\
72.64	0.00972845568672791\\
72.65	0.0097256447926691\\
72.66	0.0097228313364427\\
72.67	0.00972001531339407\\
72.68	0.00971719671885425\\
72.69	0.00971437554813986\\
72.7	0.00971155179655305\\
72.71	0.00970872545938135\\
72.72	0.00970589653189762\\
72.73	0.00970306500935998\\
72.74	0.00970023088701165\\
72.75	0.00969739416008093\\
72.76	0.00969455482378108\\
72.77	0.00969171287331022\\
72.78	0.00968886830385121\\
72.79	0.00968602111057165\\
72.8	0.00968317128862364\\
72.81	0.00968031883314383\\
72.82	0.00967746373925321\\
72.83	0.00967460600205706\\
72.84	0.00967174561664485\\
72.85	0.00966888257809012\\
72.86	0.00966601688145038\\
72.87	0.00966314852176703\\
72.88	0.00966027749406522\\
72.89	0.00965740379335378\\
72.9	0.00965452741462506\\
72.91	0.00965164835285488\\
72.92	0.0096487666030024\\
72.93	0.00964588216000997\\
72.94	0.00964299501880307\\
72.95	0.00964010517429019\\
72.96	0.00963721262136268\\
72.97	0.00963431735489466\\
72.98	0.00963141936974291\\
72.99	0.00962851866074672\\
73	0.00962561522272781\\
73.01	0.00962270905049018\\
73.02	0.00961980013881999\\
73.03	0.00961688848248545\\
73.04	0.00961397407623669\\
73.05	0.00961105691480561\\
73.06	0.0096081369929058\\
73.07	0.00960521430523234\\
73.08	0.00960228884646176\\
73.09	0.00959936061125183\\
73.1	0.00959642959424145\\
73.11	0.00959349579005052\\
73.12	0.00959055919327981\\
73.13	0.00958761979851081\\
73.14	0.00958467760030558\\
73.15	0.00958173259320663\\
73.16	0.00957878477173677\\
73.17	0.00957583413039894\\
73.18	0.00957288066367611\\
73.19	0.00956992436603109\\
73.2	0.0095669652319064\\
73.21	0.00956400325572409\\
73.22	0.00956103843188565\\
73.23	0.00955807075477178\\
73.24	0.00955510021874229\\
73.25	0.0095521268181359\\
73.26	0.00954915054727009\\
73.27	0.00954617140044097\\
73.28	0.00954318937192305\\
73.29	0.00954020445596915\\
73.3	0.00953721664681017\\
73.31	0.00953422593865494\\
73.32	0.00953123232569006\\
73.33	0.00952823580207971\\
73.34	0.00952523636196547\\
73.35	0.00952223399946616\\
73.36	0.00951922870867762\\
73.37	0.00951622048367261\\
73.38	0.00951320931850051\\
73.39	0.00951019520718722\\
73.4	0.00950717814373494\\
73.41	0.00950415812212197\\
73.42	0.00950113513630253\\
73.43	0.00949810918020657\\
73.44	0.00949508024773953\\
73.45	0.0094920483327822\\
73.46	0.00948901342919046\\
73.47	0.0094859755307951\\
73.48	0.00948293463140163\\
73.49	0.00947989072479002\\
73.5	0.00947684380471453\\
73.51	0.00947379386490348\\
73.52	0.009470740899059\\
73.53	0.00946768490085687\\
73.54	0.00946462586394625\\
73.55	0.00946156378194946\\
73.56	0.00945849864846175\\
73.57	0.00945543045705109\\
73.58	0.00945235920125789\\
73.59	0.00944928487459482\\
73.6	0.0094462074705465\\
73.61	0.00944312698256932\\
73.62	0.00944004340409116\\
73.63	0.00943695672851112\\
73.64	0.00943386694919934\\
73.65	0.00943077405949663\\
73.66	0.00942767805271433\\
73.67	0.00942457892213396\\
73.68	0.009421476661007\\
73.69	0.00941837126255459\\
73.7	0.00941526271996728\\
73.71	0.00941215102640476\\
73.72	0.00940903617499554\\
73.73	0.00940591815883672\\
73.74	0.00940279697099366\\
73.75	0.00939967260449971\\
73.76	0.0093965450523559\\
73.77	0.00939341430753069\\
73.78	0.00939028036295958\\
73.79	0.00938714321154492\\
73.8	0.00938400284615549\\
73.81	0.00938085925962626\\
73.82	0.00937771244475804\\
73.83	0.00937456239431718\\
73.84	0.00937140910103523\\
73.85	0.00936825255760859\\
73.86	0.00936509275669826\\
73.87	0.00936192969092938\\
73.88	0.00935876335289101\\
73.89	0.00935559373513568\\
73.9	0.00935242083017913\\
73.91	0.00934924463049988\\
73.92	0.00934606512853892\\
73.93	0.00934288231669933\\
73.94	0.00933969618734592\\
73.95	0.00933650673280483\\
73.96	0.00933331394536319\\
73.97	0.0093301178172687\\
73.98	0.00932691834072927\\
73.99	0.00932371550791263\\
74	0.00932050931096307\\
74.01	0.00931729974228617\\
74.02	0.00931408679425123\\
74.03	0.00931087045919091\\
74.04	0.00930765072940085\\
74.05	0.00930442759713939\\
74.06	0.00930120105462711\\
74.07	0.00929797109404659\\
74.08	0.00929473770754195\\
74.09	0.00929150088721853\\
74.1	0.00928826062514249\\
74.11	0.00928501691334047\\
74.12	0.00928176974379915\\
74.13	0.00927851910846491\\
74.14	0.0092752649992434\\
74.15	0.00927200740799919\\
74.16	0.00926874632655531\\
74.17	0.00926548174669287\\
74.18	0.00926221366015065\\
74.19	0.00925894205862468\\
74.2	0.00925566693376778\\
74.21	0.00925238827718919\\
74.22	0.00924910608045409\\
74.23	0.00924582033508317\\
74.24	0.00924253103255218\\
74.25	0.00923923816429148\\
74.26	0.00923594172168561\\
74.27	0.00923264169607275\\
74.28	0.00922933807874434\\
74.29	0.00922603086094453\\
74.3	0.00922272003386974\\
74.31	0.00921940558866816\\
74.32	0.00921608751643925\\
74.33	0.00921276580823322\\
74.34	0.00920944045505058\\
74.35	0.00920611144784156\\
74.36	0.00920277877750562\\
74.37	0.0091994424349401\\
74.38	0.00919610241106353\\
74.39	0.00919275869674249\\
74.4	0.00918941128279104\\
74.41	0.00918606016039872\\
74.42	0.00918270532253261\\
74.43	0.00917934676214259\\
74.44	0.00917598447216132\\
74.45	0.00917261844550419\\
74.46	0.00916924867506927\\
74.47	0.00916587515373725\\
74.48	0.0091624978743714\\
74.49	0.00915911682981751\\
74.5	0.00915573201290387\\
74.51	0.00915234341644115\\
74.52	0.00914895103322246\\
74.53	0.00914555485602319\\
74.54	0.00914215487760102\\
74.55	0.00913875109069584\\
74.56	0.00913534348802975\\
74.57	0.00913193206230691\\
74.58	0.00912851680621362\\
74.59	0.00912509771241815\\
74.6	0.00912167477357076\\
74.61	0.00911824798230361\\
74.62	0.00911481733123074\\
74.63	0.00911138281294797\\
74.64	0.00910794442003292\\
74.65	0.00910450214504488\\
74.66	0.00910105598052482\\
74.67	0.00909760591899529\\
74.68	0.00909415195296039\\
74.69	0.00909069407490571\\
74.7	0.0090872322772983\\
74.71	0.00908376655258657\\
74.72	0.00908029689320027\\
74.73	0.00907682329155045\\
74.74	0.00907334574002934\\
74.75	0.00906986423101038\\
74.76	0.00906637875684811\\
74.77	0.00906288930987812\\
74.78	0.00905939588241701\\
74.79	0.00905589846676235\\
74.8	0.00905239705519257\\
74.81	0.00904889163996696\\
74.82	0.00904538221332559\\
74.83	0.00904186876748927\\
74.84	0.00903835129465945\\
74.85	0.00903482978701822\\
74.86	0.00903130423672822\\
74.87	0.0090277746359326\\
74.88	0.00902424097675494\\
74.89	0.00902070325129922\\
74.9	0.00901716145164974\\
74.91	0.00901361556987109\\
74.92	0.00901006559800807\\
74.93	0.00900651152808563\\
74.94	0.00900295335210881\\
74.95	0.00899939106206273\\
74.96	0.00899582464991244\\
74.97	0.00899225410760298\\
74.98	0.00898867942705919\\
74.99	0.00898510060018577\\
75	0.00898151761886713\\
75.01	0.0089779304749674\\
75.02	0.0089743391603303\\
75.03	0.00897074366677916\\
75.04	0.0089671439861168\\
75.05	0.00896354011012549\\
75.06	0.00895993203056689\\
75.07	0.00895631973918198\\
75.08	0.00895270322769102\\
75.09	0.00894908248779347\\
75.1	0.00894545751116794\\
75.11	0.00894182828947211\\
75.12	0.00893819481434271\\
75.13	0.0089345570773954\\
75.14	0.00893091507022476\\
75.15	0.00892726878440418\\
75.16	0.00892361821148585\\
75.17	0.00891996334300067\\
75.18	0.00891630417045817\\
75.19	0.00891264068534647\\
75.2	0.00890897287913222\\
75.21	0.00890530074326053\\
75.22	0.00890162426915489\\
75.23	0.00889794344821713\\
75.24	0.00889425827182734\\
75.25	0.00889056873134383\\
75.26	0.00888687481810301\\
75.27	0.00888317652341941\\
75.28	0.00887947383858553\\
75.29	0.00887576675487183\\
75.3	0.00887205526352663\\
75.31	0.00886833935577608\\
75.32	0.00886461902282407\\
75.33	0.00886089425585215\\
75.34	0.00885716504601952\\
75.35	0.00885343138446287\\
75.36	0.00884969326229644\\
75.37	0.00884595067061181\\
75.38	0.00884220360047795\\
75.39	0.00883845204294109\\
75.4	0.00883469598902468\\
75.41	0.00883093542972929\\
75.42	0.00882717035603258\\
75.43	0.00882340075888921\\
75.44	0.00881962662923078\\
75.45	0.00881584795796574\\
75.46	0.00881206473597934\\
75.47	0.00880827695413359\\
75.48	0.00880448460326711\\
75.49	0.00880068767419514\\
75.5	0.00879688615770943\\
75.51	0.00879308004457818\\
75.52	0.00878926932554595\\
75.53	0.00878545399133362\\
75.54	0.00878163403263831\\
75.55	0.0087778094401333\\
75.56	0.00877398020446794\\
75.57	0.00877014631626763\\
75.58	0.00876630776613371\\
75.59	0.00876246454464338\\
75.6	0.00875861664234966\\
75.61	0.00875476404978128\\
75.62	0.00875090675744265\\
75.63	0.00874704475581374\\
75.64	0.00874317803535005\\
75.65	0.00873930658648251\\
75.66	0.00873543039961738\\
75.67	0.00873154946513628\\
75.68	0.00872766377339595\\
75.69	0.00872377331472834\\
75.7	0.00871987807944043\\
75.71	0.00871597805781418\\
75.72	0.00871207324010649\\
75.73	0.00870816361654906\\
75.74	0.00870424917734837\\
75.75	0.00870032991268557\\
75.76	0.00869640581271644\\
75.77	0.00869247686757125\\
75.78	0.00868854306735475\\
75.79	0.00868460440214604\\
75.8	0.00868066086199854\\
75.81	0.00867671243693986\\
75.82	0.00867275911697178\\
75.83	0.0086688008920701\\
75.84	0.00866483775218463\\
75.85	0.00866086968723907\\
75.86	0.00865689668713096\\
75.87	0.00865291874173154\\
75.88	0.00864893584088576\\
75.89	0.00864494797441212\\
75.9	0.00864095513210262\\
75.91	0.0086369573037227\\
75.92	0.00863295447901113\\
75.93	0.00862894664767992\\
75.94	0.00862493379941427\\
75.95	0.00862091845719767\\
75.96	0.00861690075898092\\
75.97	0.00861288070321891\\
75.98	0.00860885828836722\\
75.99	0.00860483351288209\\
76	0.00860080637522044\\
76.01	0.00859677687383986\\
76.02	0.00859274500719869\\
76.03	0.00858871077375595\\
76.04	0.0085846741719714\\
76.05	0.00858063520030555\\
76.06	0.0085765938572196\\
76.07	0.00857255014117558\\
76.08	0.00856850405063624\\
76.09	0.00856445558406512\\
76.1	0.00856040473992655\\
76.11	0.00855635151668567\\
76.12	0.00855229591280843\\
76.13	0.00854823792676159\\
76.14	0.00854417755701276\\
76.15	0.00854011480203038\\
76.16	0.00853604966028376\\
76.17	0.00853198213024308\\
76.18	0.00852791221037938\\
76.19	0.00852383989916462\\
76.2	0.00851976519507163\\
76.21	0.0085156880965742\\
76.22	0.00851160860214698\\
76.23	0.00850752671026563\\
76.24	0.0085034424194067\\
76.25	0.00849935572804774\\
76.26	0.00849526663466725\\
76.27	0.00849117513774472\\
76.28	0.00848708123576066\\
76.29	0.00848298492719655\\
76.3	0.00847888621053492\\
76.31	0.00847478508425932\\
76.32	0.00847068154685435\\
76.33	0.00846657559680567\\
76.34	0.0084624672326\\
76.35	0.00845835645272515\\
76.36	0.00845424325567004\\
76.37	0.00845012763992465\\
76.38	0.00844600960398014\\
76.39	0.00844188914632876\\
76.4	0.00843776626546394\\
76.41	0.00843364095988022\\
76.42	0.00842951322807335\\
76.43	0.00842538306854025\\
76.44	0.00842125047977905\\
76.45	0.00841711546028907\\
76.46	0.00841297800857086\\
76.47	0.00840883812312621\\
76.48	0.00840469580245816\\
76.49	0.00840055104507101\\
76.5	0.00839640384947033\\
76.51	0.008392254214163\\
76.52	0.00838810213765717\\
76.53	0.00838394761846232\\
76.54	0.00837979065508929\\
76.55	0.00837563124605021\\
76.56	0.0083714693898586\\
76.57	0.00836730508502937\\
76.58	0.00836313833007876\\
76.59	0.00835896912352445\\
76.6	0.00835479746388552\\
76.61	0.00835062334968248\\
76.62	0.00834644677943728\\
76.63	0.00834226775167333\\
76.64	0.0083380862649155\\
76.65	0.00833390231769014\\
76.66	0.00832971590852511\\
76.67	0.00832552703594979\\
76.68	0.00832133569849506\\
76.69	0.00831714189469338\\
76.7	0.00831294562307872\\
76.71	0.00830874688218666\\
76.72	0.00830454567055435\\
76.73	0.00830034198672054\\
76.74	0.0082961358292256\\
76.75	0.00829192719661152\\
76.76	0.00828771608742197\\
76.77	0.00828350250020224\\
76.78	0.00827928643349932\\
76.79	0.00827506788586188\\
76.8	0.00827084685584031\\
76.81	0.00826662334198671\\
76.82	0.00826239734285492\\
76.83	0.00825816885700054\\
76.84	0.00825393788298096\\
76.85	0.0082497044193553\\
76.86	0.00824546846468453\\
76.87	0.00824123001753143\\
76.88	0.00823698907646059\\
76.89	0.00823274564003847\\
76.9	0.00822849970683339\\
76.91	0.00822425127541556\\
76.92	0.00822000034435706\\
76.93	0.00821574691223192\\
76.94	0.00821149097761608\\
76.95	0.00820723253908743\\
76.96	0.00820297159522583\\
76.97	0.00819870814461311\\
76.98	0.00819444218583311\\
76.99	0.00819017371747168\\
77	0.0081859027381167\\
77.01	0.00818162924635808\\
77.02	0.00817735324078785\\
77.03	0.00817307472000006\\
77.04	0.0081687936825909\\
77.05	0.00816451012715865\\
77.06	0.00816022405230375\\
77.07	0.00815593545662878\\
77.08	0.0081516443387385\\
77.09	0.00814735069723982\\
77.1	0.00814305453074191\\
77.11	0.00813875583785613\\
77.12	0.00813445461719609\\
77.13	0.00813015086737765\\
77.14	0.00812584458701896\\
77.15	0.00812153577474045\\
77.16	0.0081172244291649\\
77.17	0.00811291054891737\\
77.18	0.0081085941326253\\
77.19	0.0081042751789185\\
77.2	0.00809995368642917\\
77.21	0.0080956296537919\\
77.22	0.00809130307964372\\
77.23	0.00808697396262409\\
77.24	0.00808264230137495\\
77.25	0.00807830809454072\\
77.26	0.0080739713407683\\
77.27	0.00806963203870714\\
77.28	0.00806529018700921\\
77.29	0.00806094578432905\\
77.3	0.00805659882932378\\
77.31	0.00805224932065309\\
77.32	0.00804789725697934\\
77.33	0.00804354263696748\\
77.34	0.00803918545928515\\
77.35	0.00803482572260265\\
77.36	0.00803046342559298\\
77.37	0.00802609856693187\\
77.38	0.00802173114529778\\
77.39	0.00801736115937193\\
77.4	0.00801298860783831\\
77.41	0.00800861348938373\\
77.42	0.0080042358026978\\
77.43	0.00799985554647299\\
77.44	0.00799547271940462\\
77.45	0.0079910873201909\\
77.46	0.00798669934753294\\
77.47	0.00798230880013477\\
77.48	0.00797791567670338\\
77.49	0.00797351997594872\\
77.5	0.00796912169658374\\
77.51	0.00796472083732437\\
77.52	0.00796031739688961\\
77.53	0.0079559113740015\\
77.54	0.00795150276738516\\
77.55	0.00794709157576878\\
77.56	0.00794267779788373\\
77.57	0.00793826143246446\\
77.58	0.00793384247824862\\
77.59	0.00792942093397704\\
77.6	0.00792499679839377\\
77.61	0.00792057007024606\\
77.62	0.00791614074828446\\
77.63	0.00791170883126275\\
77.64	0.00790727431793806\\
77.65	0.0079028372070708\\
77.66	0.00789839749742475\\
77.67	0.00789395518776707\\
77.68	0.00788951027686828\\
77.69	0.00788506276350235\\
77.7	0.00788061264644668\\
77.71	0.00787615992448213\\
77.72	0.00787170459639303\\
77.73	0.00786724666096727\\
77.74	0.00786278611699623\\
77.75	0.00785832296327487\\
77.76	0.00785385719860175\\
77.77	0.00784938882177899\\
77.78	0.0078449178316124\\
77.79	0.0078404442269114\\
77.8	0.00783596800648914\\
77.81	0.00783148916916243\\
77.82	0.00782700771375185\\
77.83	0.0078225236390817\\
77.84	0.00781803694398011\\
77.85	0.00781354762727897\\
77.86	0.00780905568781403\\
77.87	0.0078045611244249\\
77.88	0.00780006393595505\\
77.89	0.00779556412125189\\
77.9	0.00779106167916675\\
77.91	0.00778655660855494\\
77.92	0.00778204890827571\\
77.93	0.00777753857719238\\
77.94	0.00777302561417229\\
77.95	0.00776851001808682\\
77.96	0.0077639917878115\\
77.97	0.00775947092222593\\
77.98	0.00775494742021388\\
77.99	0.00775042128066329\\
78	0.00774589250246631\\
78.01	0.00774136108451931\\
78.02	0.0077368270257229\\
78.03	0.007732290324982\\
78.04	0.00772775098120585\\
78.05	0.00772320899330798\\
78.06	0.00771866436020634\\
78.07	0.00771411708082325\\
78.08	0.00770956715408543\\
78.09	0.00770501457892411\\
78.1	0.00770045935427496\\
78.11	0.00769590147907814\\
78.12	0.00769134095227841\\
78.13	0.00768677777282503\\
78.14	0.00768221193967188\\
78.15	0.00767764345177748\\
78.16	0.00767307230810498\\
78.17	0.00766849850762223\\
78.18	0.00766392204930178\\
78.19	0.00765934293212093\\
78.2	0.00765476115506173\\
78.21	0.00765017671711107\\
78.22	0.00764558961726062\\
78.23	0.00764099985450696\\
78.24	0.00763640742785155\\
78.25	0.00763181233630073\\
78.26	0.00762721457886587\\
78.27	0.00762261415456325\\
78.28	0.00761801106241419\\
78.29	0.0076134053014451\\
78.3	0.00760879687068739\\
78.31	0.00760418576917762\\
78.32	0.0075995719959575\\
78.33	0.00759495555007389\\
78.34	0.00759033643057886\\
78.35	0.00758571463652971\\
78.36	0.00758109016698901\\
78.37	0.00757646302102464\\
78.38	0.00757183319770979\\
78.39	0.00756720069612304\\
78.4	0.00756256551534835\\
78.41	0.00755792765447511\\
78.42	0.00755328711259817\\
78.43	0.00754864388881788\\
78.44	0.00754399798224014\\
78.45	0.00753934939197638\\
78.46	0.00753469811714364\\
78.47	0.0075300441568646\\
78.48	0.00752538751026757\\
78.49	0.0075207281764866\\
78.5	0.00751606615466145\\
78.51	0.00751140144393766\\
78.52	0.00750673404346655\\
78.53	0.00750206395240528\\
78.54	0.0074973911699169\\
78.55	0.00749271569517036\\
78.56	0.00748803752734054\\
78.57	0.00748335666560829\\
78.58	0.0074786731091605\\
78.59	0.00747398685719009\\
78.6	0.00746929790889606\\
78.61	0.00746460626348354\\
78.62	0.00745991192016381\\
78.63	0.00745521487815435\\
78.64	0.00745051513667885\\
78.65	0.00744581269496729\\
78.66	0.00744110755225593\\
78.67	0.00743639970778739\\
78.68	0.00743168916081066\\
78.69	0.00742697591058113\\
78.7	0.00742225995636067\\
78.71	0.00741754129741761\\
78.72	0.00741281993302682\\
78.73	0.00740809586246975\\
78.74	0.00740336908503443\\
78.75	0.00739863960001556\\
78.76	0.00739390740671448\\
78.77	0.0073891725044393\\
78.78	0.00738443489250486\\
78.79	0.00737969457023279\\
78.8	0.00737495153695158\\
78.81	0.00737020579199659\\
78.82	0.00736545733471009\\
78.83	0.00736070616444132\\
78.84	0.00735595228054652\\
78.85	0.00735119568238894\\
78.86	0.00734643636933893\\
78.87	0.00734167434077395\\
78.88	0.00733690959607865\\
78.89	0.00733214213464483\\
78.9	0.00732737195587157\\
78.91	0.00732259905916523\\
78.92	0.0073178234663391\\
78.93	0.00731304531519263\\
78.94	0.00730826460330868\\
78.95	0.00730348132826657\\
78.96	0.00729869548764205\\
78.97	0.00729390707900735\\
78.98	0.00728911609993109\\
78.99	0.00728432254797836\\
79	0.00727952642071066\\
79.01	0.00727472771568591\\
79.02	0.00726992643045844\\
79.03	0.00726512256257898\\
79.04	0.00726031610959469\\
79.05	0.00725550706904908\\
79.06	0.00725069543848207\\
79.07	0.00724588121542998\\
79.08	0.00724106439742547\\
79.09	0.00723624498199759\\
79.1	0.00723142296667175\\
79.11	0.00722659834896971\\
79.12	0.00722177112640958\\
79.13	0.00721694129650582\\
79.14	0.00721210885676923\\
79.15	0.00720727380470693\\
79.16	0.00720243613782237\\
79.17	0.00719759585361533\\
79.18	0.00719275294958189\\
79.19	0.00718790742321443\\
79.2	0.00718305927200164\\
79.21	0.00717820849342851\\
79.22	0.0071733550849763\\
79.23	0.00716849904412258\\
79.24	0.00716364036834115\\
79.25	0.00715877905510212\\
79.26	0.00715391510187183\\
79.27	0.00714904850611291\\
79.28	0.0071441792652842\\
79.29	0.00713930737684082\\
79.3	0.00713443283823411\\
79.31	0.00712955564691163\\
79.32	0.00712467580031719\\
79.33	0.00711979329589079\\
79.34	0.00711490813106867\\
79.35	0.00711002030328324\\
79.36	0.00710512980996314\\
79.37	0.0071002366485332\\
79.38	0.00709534081641442\\
79.39	0.00709044231102397\\
79.4	0.00708554112977525\\
79.41	0.00708063727007776\\
79.42	0.00707573072933721\\
79.43	0.00707082150495542\\
79.44	0.0070659095943304\\
79.45	0.00706099499485629\\
79.46	0.00705607770392335\\
79.47	0.00705115771891798\\
79.48	0.00704623503722272\\
79.49	0.00704130965621619\\
79.5	0.00703638157327315\\
79.51	0.00703145078576446\\
79.52	0.00702651729105706\\
79.53	0.00702158108651401\\
79.54	0.00701664216949444\\
79.55	0.00701170053735355\\
79.56	0.00700675618744262\\
79.57	0.00700180911710901\\
79.58	0.00699685932369613\\
79.59	0.00699190680454343\\
79.6	0.00698695155698642\\
79.61	0.00698199357835667\\
79.62	0.00697703286598175\\
79.63	0.00697206941718529\\
79.64	0.00696710322928691\\
79.65	0.00696213429960228\\
79.66	0.00695716262544306\\
79.67	0.00695218820411692\\
79.68	0.00694721103292753\\
79.69	0.00694223110917453\\
79.7	0.00693724843015359\\
79.71	0.00693226299315631\\
79.72	0.0069272747954703\\
79.73	0.00692228383437912\\
79.74	0.00691729010716229\\
79.75	0.00691229361109527\\
79.76	0.00690729434344951\\
79.77	0.00690229230149237\\
79.78	0.00689728748248715\\
79.79	0.00689227988369308\\
79.8	0.00688726950236531\\
79.81	0.00688225633575493\\
79.82	0.00687724038110892\\
79.83	0.00687222163567015\\
79.84	0.00686720009667744\\
79.85	0.00686217576136545\\
79.86	0.00685714862696476\\
79.87	0.00685211869070182\\
79.88	0.00684708594979895\\
79.89	0.00684205040147436\\
79.9	0.00683701204294209\\
79.91	0.00683197087141206\\
79.92	0.00682692688409003\\
79.93	0.00682188007817763\\
79.94	0.00681683045087229\\
79.95	0.00681177799936731\\
79.96	0.00680672272085179\\
79.97	0.00680166461251067\\
79.98	0.00679660367152469\\
79.99	0.00679153989507041\\
80	0.0067864732803202\\
80.01	0.00678140382444221\\
};
\addplot [color=red,dashed]
  table[row sep=crcr]{%
80.01	0.00678140382444221\\
80.02	0.0067763315246004\\
80.03	0.00677125637795449\\
80.04	0.00676617838166003\\
80.05	0.00676109753286829\\
80.06	0.00675601382872634\\
80.07	0.00675092726637702\\
80.08	0.0067458378429589\\
80.09	0.00674074555560633\\
80.1	0.00673565040144938\\
80.11	0.00673055237761389\\
80.12	0.00672545148122141\\
80.13	0.00672034770938924\\
80.14	0.00671524105923039\\
80.15	0.0067101315278536\\
80.16	0.00670501911236331\\
80.17	0.0066999038098597\\
80.18	0.00669478561743859\\
80.19	0.00668966453219157\\
80.2	0.00668454055120587\\
80.21	0.00667941367156443\\
80.22	0.00667428389034586\\
80.23	0.00666915120462446\\
80.24	0.0066640156114702\\
80.25	0.00665887710794869\\
80.26	0.00665373569112123\\
80.27	0.00664859135804476\\
80.28	0.00664344410577187\\
80.29	0.0066382939313508\\
80.3	0.00663314083182544\\
80.31	0.00662798480423529\\
80.32	0.00662282584561549\\
80.33	0.00661766395299683\\
80.34	0.00661249912340568\\
80.35	0.00660733135386406\\
80.36	0.00660216064138956\\
80.37	0.00659698698299543\\
80.38	0.00659181037569048\\
80.39	0.00658663081647911\\
80.4	0.00658144830236136\\
80.41	0.0065762628303328\\
80.42	0.00657107439738463\\
80.43	0.0065658830005036\\
80.44	0.00656068863667204\\
80.45	0.00655549130286784\\
80.46	0.00655029099606448\\
80.47	0.00654508771323098\\
80.48	0.00653988145133192\\
80.49	0.00653467220732743\\
80.5	0.00652945997817319\\
80.51	0.00652424476082043\\
80.52	0.00651902655221592\\
80.53	0.00651380534930194\\
80.54	0.00650858114901634\\
80.55	0.00650335394829246\\
80.56	0.00649812374405919\\
80.57	0.00649289053324095\\
80.58	0.00648765431275763\\
80.59	0.00648241507952467\\
80.6	0.00647717283045302\\
80.61	0.00647192756244911\\
80.62	0.0064666792724149\\
80.63	0.00646142795724783\\
80.64	0.00645617361384084\\
80.65	0.00645091623908236\\
80.66	0.00644565582985631\\
80.67	0.0064403923830421\\
80.68	0.00643512589551462\\
80.69	0.00642985636414423\\
80.7	0.00642458378579678\\
80.71	0.00641930815733358\\
80.72	0.00641402947561143\\
80.73	0.00640874773748257\\
80.74	0.00640346293979474\\
80.75	0.0063981750793911\\
80.76	0.00639288415311031\\
80.77	0.00638759015778646\\
80.78	0.00638229309024911\\
80.79	0.00637699294732328\\
80.8	0.00637168972582941\\
80.81	0.00636638342258343\\
80.82	0.00636107403439668\\
80.83	0.00635576155807596\\
80.84	0.00635044599042353\\
80.85	0.00634512732823705\\
80.86	0.00633980556830966\\
80.87	0.00633448070742992\\
80.88	0.00632915274238183\\
80.89	0.00632382166994482\\
80.9	0.00631848748689376\\
80.91	0.00631315018999895\\
80.92	0.00630780977602613\\
80.93	0.00630246624173645\\
80.94	0.00629711958388651\\
80.95	0.00629176979922833\\
80.96	0.00628641688450936\\
80.97	0.00628106083647248\\
80.98	0.006275701651856\\
80.99	0.00627033932739365\\
81	0.00626497385981459\\
81.01	0.0062596052458434\\
81.02	0.00625423348220011\\
81.03	0.00624885856560015\\
81.04	0.0062434804927544\\
81.05	0.00623809926036913\\
81.06	0.0062327148651461\\
81.07	0.00622732730378243\\
81.08	0.00622193657297073\\
81.09	0.00621654266939899\\
81.1	0.00621114558975067\\
81.11	0.00620574533070464\\
81.12	0.00620034188893523\\
81.13	0.00619493526111217\\
81.14	0.00618952544390064\\
81.15	0.00618411243396128\\
81.16	0.00617869622795014\\
81.17	0.00617327682251874\\
81.18	0.00616785421431402\\
81.19	0.00616242839997839\\
81.2	0.00615699937614969\\
81.21	0.00615156713946121\\
81.22	0.00614613168654173\\
81.23	0.00614069301401545\\
81.24	0.00613525111850204\\
81.25	0.00612980599661666\\
81.26	0.0061243576449699\\
81.27	0.00611890606016783\\
81.28	0.00611345123881202\\
81.29	0.0061079931774995\\
81.3	0.00610253187282277\\
81.31	0.00609706732136985\\
81.32	0.00609159951972423\\
81.33	0.00608612846446491\\
81.34	0.00608065415216637\\
81.35	0.00607517657939862\\
81.36	0.00606969574272716\\
81.37	0.00606421163871303\\
81.38	0.00605872426391278\\
81.39	0.00605323361487849\\
81.4	0.00604773968815777\\
81.41	0.00604224248029378\\
81.42	0.00603674198782522\\
81.43	0.00603123820728636\\
81.44	0.00602573113520701\\
81.45	0.00602022076811256\\
81.46	0.00601470710252397\\
81.47	0.00600919013495779\\
81.48	0.00600366986192615\\
81.49	0.00599814627993678\\
81.5	0.00599261938549303\\
81.51	0.00598708917509386\\
81.52	0.00598155564523385\\
81.53	0.00597601879240321\\
81.54	0.00597047861308779\\
81.55	0.00596493510376912\\
81.56	0.00595938826092435\\
81.57	0.00595383808102633\\
81.58	0.00594828456054358\\
81.59	0.00594272769594033\\
81.6	0.00593716748367648\\
81.61	0.00593160392020766\\
81.62	0.00592603700198525\\
81.63	0.00592046672545632\\
81.64	0.00591489308706374\\
81.65	0.00590931608324609\\
81.66	0.00590373571043776\\
81.67	0.00589815196506891\\
81.68	0.00589256484356551\\
81.69	0.00588697434234934\\
81.7	0.00588138045783799\\
81.71	0.00587578318644491\\
81.72	0.00587018252457939\\
81.73	0.00586457846864661\\
81.74	0.00585897101504761\\
81.75	0.00585336016017935\\
81.76	0.00584774590043468\\
81.77	0.00584212823220241\\
81.78	0.00583650715186729\\
81.79	0.00583088265581001\\
81.8	0.00582525474040727\\
81.81	0.00581962340203176\\
81.82	0.00581398863705218\\
81.83	0.00580835044183326\\
81.84	0.00580270881273579\\
81.85	0.00579706374611665\\
81.86	0.00579141523832878\\
81.87	0.00578576328572124\\
81.88	0.00578010788463923\\
81.89	0.0057744490314241\\
81.9	0.00576878672241335\\
81.91	0.0057631209539407\\
81.92	0.00575745172233606\\
81.93	0.0057517790239256\\
81.94	0.00574610285503175\\
81.95	0.00574042321197319\\
81.96	0.00573474009106494\\
81.97	0.00572905348861832\\
81.98	0.00572336340094103\\
81.99	0.00571766982433714\\
82	0.00571197275510711\\
82.01	0.00570627218954784\\
82.02	0.00570056812395268\\
82.03	0.00569486055461147\\
82.04	0.00568914947781056\\
82.05	0.00568343488983282\\
82.06	0.00567771678695771\\
82.07	0.00567199516546125\\
82.08	0.00566627002161611\\
82.09	0.00566054135169159\\
82.1	0.0056548091519537\\
82.11	0.00564907341866514\\
82.12	0.00564333414808535\\
82.13	0.00563759133647057\\
82.14	0.00563184498007381\\
82.15	0.00562609507514498\\
82.16	0.0056203416179308\\
82.17	0.00561458460467494\\
82.18	0.00560882403161799\\
82.19	0.00560305989499754\\
82.2	0.00559729219104817\\
82.21	0.00559152091600153\\
82.22	0.00558574606608636\\
82.23	0.0055799676375285\\
82.24	0.00557418562655098\\
82.25	0.00556840002937401\\
82.26	0.00556261084221508\\
82.27	0.00555681806128891\\
82.28	0.00555102168280759\\
82.29	0.00554522170298054\\
82.3	0.00553941811801463\\
82.31	0.00553361092411413\\
82.32	0.00552780011748084\\
82.33	0.00552198569431409\\
82.34	0.00551616765081078\\
82.35	0.00551034598316547\\
82.36	0.00550452068757037\\
82.37	0.00549869176021542\\
82.38	0.00549285919728834\\
82.39	0.00548702299497467\\
82.4	0.00548118314945781\\
82.41	0.0054753396569191\\
82.42	0.00546949251353783\\
82.43	0.00546364171549133\\
82.44	0.005457787258955\\
82.45	0.00545192914010239\\
82.46	0.00544606735510521\\
82.47	0.00544020190013344\\
82.48	0.00543433277135532\\
82.49	0.00542845996493749\\
82.5	0.00542258347704499\\
82.51	0.0054167033038413\\
82.52	0.00541081944148847\\
82.53	0.00540493188614714\\
82.54	0.00539904063397659\\
82.55	0.00539314568113484\\
82.56	0.00538724702377867\\
82.57	0.00538134465806372\\
82.58	0.00537543858014454\\
82.59	0.00536952878617465\\
82.6	0.00536361527230664\\
82.61	0.0053576980346922\\
82.62	0.00535177706948222\\
82.63	0.00534585237282682\\
82.64	0.00533992394087547\\
82.65	0.00533399176977705\\
82.66	0.0053280558556799\\
82.67	0.00532211619438163\\
82.68	0.00531617278152109\\
82.69	0.00531022561273023\\
82.7	0.0053042746836342\\
82.71	0.00529831998985126\\
82.72	0.00529236152699282\\
82.73	0.00528639929066343\\
82.74	0.00528043327646079\\
82.75	0.00527446347997571\\
82.76	0.00526848989679214\\
82.77	0.00526251252248718\\
82.78	0.00525653135263102\\
82.79	0.00525054638278702\\
82.8	0.00524455760851162\\
82.81	0.00523856502535443\\
82.82	0.00523256862885814\\
82.83	0.00522656841455859\\
82.84	0.00522056437798473\\
82.85	0.00521455651465863\\
82.86	0.00520854482009547\\
82.87	0.00520252928980357\\
82.88	0.00519650991928436\\
82.89	0.00519048670403237\\
82.9	0.00518445963953528\\
82.91	0.00517842872127387\\
82.92	0.00517239394472205\\
82.93	0.00516635530534685\\
82.94	0.00516031279860841\\
82.95	0.00515426641996\\
82.96	0.00514821616484803\\
82.97	0.00514216202871202\\
82.98	0.00513610400698462\\
82.99	0.00513004209509163\\
83	0.00512397628845194\\
83.01	0.00511790658247763\\
83.02	0.00511183297257387\\
83.03	0.00510575545413901\\
83.04	0.00509967402256452\\
83.05	0.00509358867323502\\
83.06	0.0050874994015283\\
83.07	0.00508140620281529\\
83.08	0.00507530907246009\\
83.09	0.00506920800581995\\
83.1	0.0050631029982453\\
83.11	0.00505699404507976\\
83.12	0.00505088114166009\\
83.13	0.00504476428331628\\
83.14	0.00503864346537147\\
83.15	0.00503251868314204\\
83.16	0.00502638993193754\\
83.17	0.00502025720706075\\
83.18	0.00501412050380765\\
83.19	0.00500797981746745\\
83.2	0.00500183514332262\\
83.21	0.00499568647664883\\
83.22	0.00498953381271502\\
83.23	0.00498337714678338\\
83.24	0.00497721647410938\\
83.25	0.00497105178994173\\
83.26	0.00496488308952247\\
83.27	0.0049587103680869\\
83.28	0.00495253362086363\\
83.29	0.00494635284307459\\
83.3	0.00494016802993505\\
83.31	0.00493397917665358\\
83.32	0.00492778627843213\\
83.33	0.004921589330466\\
83.34	0.00491538832794384\\
83.35	0.00490918326604773\\
83.36	0.00490297413995311\\
83.37	0.00489676094482885\\
83.38	0.00489054367583722\\
83.39	0.00488432232813396\\
83.4	0.00487809689686824\\
83.41	0.00487186737718271\\
83.42	0.00486563376421349\\
83.43	0.00485939605309022\\
83.44	0.00485315423893604\\
83.45	0.00484690831686761\\
83.46	0.00484065828199517\\
83.47	0.0048344041294225\\
83.48	0.00482814585424696\\
83.49	0.00482188345155955\\
83.5	0.00481561691644484\\
83.51	0.00480934624398107\\
83.52	0.00480307142924015\\
83.53	0.00479679246728763\\
83.54	0.0047905093531828\\
83.55	0.00478422208197864\\
83.56	0.00477793064872191\\
83.57	0.00477163504845308\\
83.58	0.00476533527620647\\
83.59	0.00475903132701018\\
83.6	0.00475272319588612\\
83.61	0.00474641087785011\\
83.62	0.00474009436791182\\
83.63	0.00473377366107483\\
83.64	0.00472744875233667\\
83.65	0.00472111963668882\\
83.66	0.00471478630911673\\
83.67	0.00470844876459992\\
83.68	0.00470210699811188\\
83.69	0.00469576100462023\\
83.7	0.00468941077908666\\
83.71	0.00468305631646701\\
83.72	0.00467669761171126\\
83.73	0.0046703346597636\\
83.74	0.00466396745556243\\
83.75	0.00465759599404043\\
83.76	0.00465122027012454\\
83.77	0.00464484027873604\\
83.78	0.00463845601479056\\
83.79	0.00463206747319813\\
83.8	0.00462567464886321\\
83.81	0.00461927753668471\\
83.82	0.00461287613155604\\
83.83	0.00460647042836517\\
83.84	0.00460006042199464\\
83.85	0.0045936461073216\\
83.86	0.00458722747921785\\
83.87	0.0045808045325499\\
83.88	0.00457437726217901\\
83.89	0.00456794566296119\\
83.9	0.0045615097297473\\
83.91	0.00455506945738305\\
83.92	0.00454862484070908\\
83.93	0.00454217587456096\\
83.94	0.00453572255376931\\
83.95	0.00452926487315976\\
83.96	0.00452280282755306\\
83.97	0.00451633641176509\\
83.98	0.00450986562060694\\
83.99	0.00450339044888495\\
84	0.00449691089140076\\
84.01	0.00449042694295134\\
84.02	0.00448393859832911\\
84.03	0.0044774458523219\\
84.04	0.00447094869971308\\
84.05	0.00446444713528158\\
84.06	0.00445794115380198\\
84.07	0.00445143075004451\\
84.08	0.00444491591877517\\
84.09	0.00443839665475575\\
84.1	0.00443187295274391\\
84.11	0.00442534480749325\\
84.12	0.00441881221375334\\
84.13	0.0044122751662698\\
84.14	0.0044057336597844\\
84.15	0.00439918768903507\\
84.16	0.004392637248756\\
84.17	0.00438608233367769\\
84.18	0.00437952293852707\\
84.19	0.00437295905802748\\
84.2	0.00436639068689884\\
84.21	0.00435981781985764\\
84.22	0.00435324045161707\\
84.23	0.00434665857688709\\
84.24	0.00434007219037446\\
84.25	0.00433348128678288\\
84.26	0.00432688586081302\\
84.27	0.00432028590716265\\
84.28	0.00431368142052666\\
84.29	0.0043070723955972\\
84.3	0.00430045882706373\\
84.31	0.00429384070961311\\
84.32	0.00428721803792973\\
84.33	0.00428059080669551\\
84.34	0.00427395901059008\\
84.35	0.00426732264429082\\
84.36	0.00426068170247295\\
84.37	0.00425403617980969\\
84.38	0.00424738607097225\\
84.39	0.00424073137063001\\
84.4	0.00423407207345061\\
84.41	0.00422740817409998\\
84.42	0.00422073966724255\\
84.43	0.00421406654754126\\
84.44	0.00420738880965773\\
84.45	0.00420070644825231\\
84.46	0.00419401945798424\\
84.47	0.00418732783351173\\
84.48	0.00418063156949208\\
84.49	0.00417393066058179\\
84.5	0.00416722510143666\\
84.51	0.00416051488671196\\
84.52	0.00415380001106248\\
84.53	0.0041470804691427\\
84.54	0.00414035625560688\\
84.55	0.00413362736510922\\
84.56	0.00412689379230395\\
84.57	0.00412015553184546\\
84.58	0.00411341257838846\\
84.59	0.00410666492658809\\
84.6	0.00409991257110005\\
84.61	0.00409315550658074\\
84.62	0.00408639372768741\\
84.63	0.00407962722907829\\
84.64	0.00407285600541271\\
84.65	0.00406608005135128\\
84.66	0.00405929936155603\\
84.67	0.00405251393069052\\
84.68	0.00404572375342004\\
84.69	0.00403892882441174\\
84.7	0.00403212913833478\\
84.71	0.00402532468986048\\
84.72	0.00401851547366252\\
84.73	0.00401170148441705\\
84.74	0.00400488271680288\\
84.75	0.00399805916550166\\
84.76	0.003991230825198\\
84.77	0.00398439769057969\\
84.78	0.00397755975633789\\
84.79	0.00397071701716721\\
84.8	0.00396386946776598\\
84.81	0.00395701710283643\\
84.82	0.00395015991708481\\
84.83	0.00394329790522163\\
84.84	0.00393643106196181\\
84.85	0.00392955938202494\\
84.86	0.00392268286013537\\
84.87	0.00391580149102254\\
84.88	0.00390891526942103\\
84.89	0.00390202419007089\\
84.9	0.00389512824771779\\
84.91	0.00388822743711323\\
84.92	0.00388132175301474\\
84.93	0.00387441119018616\\
84.94	0.00386749574339776\\
84.95	0.00386057540742655\\
84.96	0.00385365017705644\\
84.97	0.00384672004707852\\
84.98	0.00383978501229125\\
84.99	0.00383284506750072\\
85	0.00382590020752085\\
85.01	0.00381895042717371\\
85.02	0.00381199572128966\\
85.03	0.00380503608470768\\
85.04	0.00379807151227559\\
85.05	0.00379110199885029\\
85.06	0.00378412753929805\\
85.07	0.00377714812849476\\
85.08	0.00377016376132618\\
85.09	0.00376317443268822\\
85.1	0.00375618013748722\\
85.11	0.00374918087064023\\
85.12	0.00374217662707528\\
85.13	0.00373516740173165\\
85.14	0.00372815318956021\\
85.15	0.00372113398552364\\
85.16	0.0037141097845968\\
85.17	0.00370708058176697\\
85.18	0.00370004637203422\\
85.19	0.00369300715041164\\
85.2	0.00368596291192571\\
85.21	0.00367891365161663\\
85.22	0.00367185936453859\\
85.23	0.00366480004576013\\
85.24	0.00365773569036448\\
85.25	0.00365066629344985\\
85.26	0.00364359185012984\\
85.27	0.00363651235553373\\
85.28	0.00362942780480686\\
85.29	0.00362233819311097\\
85.3	0.00361524351562456\\
85.31	0.00360814376754328\\
85.32	0.00360103894408026\\
85.33	0.00359392904046652\\
85.34	0.00358681405195133\\
85.35	0.00357969397380262\\
85.36	0.00357256880130733\\
85.37	0.00356543852977187\\
85.38	0.00355830315452243\\
85.39	0.0035511626709055\\
85.4	0.00354401707428819\\
85.41	0.00353686636005869\\
85.42	0.00352971052362669\\
85.43	0.00352254956042382\\
85.44	0.00351538346590407\\
85.45	0.00350821223554424\\
85.46	0.00350103586484438\\
85.47	0.00349385434932828\\
85.48	0.00348666768454386\\
85.49	0.00347947586606373\\
85.5	0.00347227888948558\\
85.51	0.00346507675043272\\
85.52	0.00345786944455453\\
85.53	0.00345065696752697\\
85.54	0.00344488106945515\\
85.55	0.00344218099965369\\
85.56	0.0034394855100355\\
85.57	0.00343679462172208\\
85.58	0.00343410835590736\\
85.59	0.00343142673385796\\
85.6	0.00342874977691339\\
85.61	0.00342607750648627\\
85.62	0.00342340994406259\\
85.63	0.00342074436059817\\
85.64	0.00341808043756742\\
85.65	0.00341541818512362\\
85.66	0.00341275761345693\\
85.67	0.00341009873279459\\
85.68	0.00340744155340098\\
85.69	0.00340478608557778\\
85.7	0.00340213233966408\\
85.71	0.0033994803260365\\
85.72	0.00339683005510936\\
85.73	0.00339418153733473\\
85.74	0.00339153478320262\\
85.75	0.00338888980324109\\
85.76	0.00338624660801637\\
85.77	0.00338360520813297\\
85.78	0.00338096561423386\\
85.79	0.00337832783700055\\
85.8	0.00337569188715323\\
85.81	0.00337305777545092\\
85.82	0.00337042551269156\\
85.83	0.00336779510971218\\
85.84	0.00336516657738901\\
85.85	0.00336253992663759\\
85.86	0.00335991516841297\\
85.87	0.00335729231370973\\
85.88	0.00335467137356222\\
85.89	0.00335205235904463\\
85.9	0.00334943528127114\\
85.91	0.00334682015139603\\
85.92	0.00334420698061387\\
85.93	0.00334159578015956\\
85.94	0.00333898656130856\\
85.95	0.00333637933537696\\
85.96	0.00333377411372163\\
85.97	0.00333117090774036\\
85.98	0.00332856972887198\\
85.99	0.00332597058859652\\
86	0.00332337349843531\\
86.01	0.00332077846995116\\
86.02	0.00331818551474843\\
86.03	0.00331559464447323\\
86.04	0.00331300587081352\\
86.05	0.00331041920549925\\
86.06	0.00330783466030252\\
86.07	0.00330525224703768\\
86.08	0.00330267197756149\\
86.09	0.00330009386377323\\
86.1	0.0032975179176149\\
86.11	0.00329494415107128\\
86.12	0.00329237257617012\\
86.13	0.00328980320498225\\
86.14	0.00328723604962175\\
86.15	0.00328467112224605\\
86.16	0.00328210843505609\\
86.17	0.00327954800029646\\
86.18	0.00327698983025554\\
86.19	0.00327443393726562\\
86.2	0.00327188033370307\\
86.21	0.00326932903198846\\
86.22	0.00326678004458671\\
86.23	0.00326423338400721\\
86.24	0.003261689062804\\
86.25	0.00325914709357586\\
86.26	0.00325660748896651\\
86.27	0.0032540702616647\\
86.28	0.00325153542440436\\
86.29	0.00324900298996479\\
86.3	0.00324647297117074\\
86.31	0.00324394538089257\\
86.32	0.00324142023204643\\
86.33	0.00323889753759435\\
86.34	0.00323637731054441\\
86.35	0.00323385956395089\\
86.36	0.0032313443109144\\
86.37	0.003228831564582\\
86.38	0.0032263213381474\\
86.39	0.00322381364485108\\
86.4	0.00322130849798039\\
86.41	0.00321880591086976\\
86.42	0.00321630589690082\\
86.43	0.00321380846950252\\
86.44	0.0032113136421513\\
86.45	0.00320882142837124\\
86.46	0.00320633184173419\\
86.47	0.00320384489585993\\
86.48	0.00320136060441628\\
86.49	0.00319887898111929\\
86.5	0.00319640003973337\\
86.51	0.00319392379407142\\
86.52	0.00319145025799499\\
86.53	0.00318897944541442\\
86.54	0.00318651137028901\\
86.55	0.00318404604662712\\
86.56	0.00318158348848636\\
86.57	0.0031791237099737\\
86.58	0.00317666672524565\\
86.59	0.00317421254850839\\
86.6	0.00317176119401791\\
86.61	0.00316931267608017\\
86.62	0.00316686700905124\\
86.63	0.00316442420733744\\
86.64	0.00316198428539551\\
86.65	0.00315954725773271\\
86.66	0.00315711313890705\\
86.67	0.00315468194352733\\
86.68	0.00315225368625337\\
86.69	0.00314982838179614\\
86.7	0.00314740604491786\\
86.71	0.00314498669043221\\
86.72	0.00314257033320445\\
86.73	0.00314015698815154\\
86.74	0.00313774667024235\\
86.75	0.00313533939449774\\
86.76	0.00313293517599075\\
86.77	0.00313053402984673\\
86.78	0.0031281359712435\\
86.79	0.00312574101541148\\
86.8	0.00312334917763384\\
86.81	0.00312096047324665\\
86.82	0.00311857491763905\\
86.83	0.00311619252625333\\
86.84	0.00311381331458515\\
86.85	0.00311143729818366\\
86.86	0.00310906449265163\\
86.87	0.0031066949136456\\
86.88	0.00310432857687605\\
86.89	0.00310196549810751\\
86.9	0.00309960569315875\\
86.91	0.00309724917790287\\
86.92	0.00309489596826751\\
86.93	0.00309254608023492\\
86.94	0.00309019580124057\\
86.95	0.00308784491742243\\
86.96	0.00308549342798923\\
86.97	0.00308314133214113\\
86.98	0.0030807886290697\\
86.99	0.00307843531795785\\
87	0.00307608139797979\\
87.01	0.00307372686830099\\
87.02	0.00307137172807808\\
87.03	0.00306901597645881\\
87.04	0.00306665961258203\\
87.05	0.0030643026355776\\
87.06	0.00306194504456634\\
87.07	0.00305958683865997\\
87.08	0.00305722801696107\\
87.09	0.003054868578563\\
87.1	0.00305250852254986\\
87.11	0.00305014784799643\\
87.12	0.0030477865539681\\
87.13	0.00304542463952084\\
87.14	0.00304306210370109\\
87.15	0.00304069894554578\\
87.16	0.00303833516408217\\
87.17	0.00303597075832789\\
87.18	0.00303360572729081\\
87.19	0.00303124006996902\\
87.2	0.00302887378535076\\
87.21	0.00302650687241433\\
87.22	0.00302413933012808\\
87.23	0.00302177115745031\\
87.24	0.00301940235332923\\
87.25	0.00301703291670287\\
87.26	0.00301466284649908\\
87.27	0.00301229214163538\\
87.28	0.00300992080101896\\
87.29	0.00300754882354661\\
87.3	0.00300517620810464\\
87.31	0.00300280295356881\\
87.32	0.0030004290588043\\
87.33	0.0029980545226656\\
87.34	0.00299567934399651\\
87.35	0.00299330352162999\\
87.36	0.00299092705438816\\
87.37	0.00298854994108222\\
87.38	0.00298617218051239\\
87.39	0.00298379377146778\\
87.4	0.00298141471272644\\
87.41	0.00297903500305519\\
87.42	0.0029766546412096\\
87.43	0.00297427362593393\\
87.44	0.00297189195596101\\
87.45	0.00296950963001225\\
87.46	0.0029671266467975\\
87.47	0.00296474300501502\\
87.48	0.00296235870335141\\
87.49	0.00295997374048152\\
87.5	0.0029575881150684\\
87.51	0.00295520182576321\\
87.52	0.00295281487120518\\
87.53	0.0029504272500215\\
87.54	0.00294803896082729\\
87.55	0.00294565000222548\\
87.56	0.00294326037280679\\
87.57	0.00294087007114963\\
87.58	0.00293847909582001\\
87.59	0.00293608744537151\\
87.6	0.00293369511834517\\
87.61	0.00293130211326944\\
87.62	0.00292890842866009\\
87.63	0.00292651406302013\\
87.64	0.00292411901483978\\
87.65	0.00292172328259632\\
87.66	0.00291932686475408\\
87.67	0.00291692975976433\\
87.68	0.00291453196606522\\
87.69	0.0029121334820817\\
87.7	0.00290973430622542\\
87.71	0.0029073344368947\\
87.72	0.00290493387247439\\
87.73	0.00290253261133586\\
87.74	0.00290013046945741\\
87.75	0.00289772744173341\\
87.76	0.00289532352850149\\
87.77	0.00289291873010137\\
87.78	0.00289051304687493\\
87.79	0.00288810647916613\\
87.8	0.0028856990273211\\
87.81	0.00288329069168808\\
87.82	0.00288088147261747\\
87.83	0.00287847137046183\\
87.84	0.00287606038557585\\
87.85	0.0028736485183164\\
87.86	0.00287123576904251\\
87.87	0.00286882213811539\\
87.88	0.00286640762589842\\
87.89	0.00286399223275719\\
87.9	0.00286157595905945\\
87.91	0.00285915880517517\\
87.92	0.00285674077147652\\
87.93	0.00285432185833788\\
87.94	0.00285190206613584\\
87.95	0.00284948139524924\\
87.96	0.00284705984605911\\
87.97	0.00284463741894874\\
87.98	0.00284221411430366\\
87.99	0.00283978993251165\\
88	0.00283736487396273\\
88.01	0.00283493893904919\\
88.02	0.0028325121281656\\
88.03	0.00283008444170878\\
88.04	0.00282765588007784\\
88.05	0.00282522644367417\\
88.06	0.00282279613290147\\
88.07	0.00282036494816573\\
88.08	0.00281793288987523\\
88.09	0.00281549995844058\\
88.1	0.0028130661542747\\
88.11	0.00281063147779284\\
88.12	0.00280819592941258\\
88.13	0.00280575950955381\\
88.14	0.00280332221863882\\
88.15	0.0028008840570922\\
88.16	0.00279844502534093\\
88.17	0.00279600512381433\\
88.18	0.00279356435294409\\
88.19	0.0027911227131643\\
88.2	0.00278868020491142\\
88.21	0.00278623682862429\\
88.22	0.00278379258474416\\
88.23	0.00278134747371469\\
88.24	0.00277890149598192\\
88.25	0.00277645465199434\\
88.26	0.00277400694220284\\
88.27	0.00277155836706075\\
88.28	0.00276910892702384\\
88.29	0.00276665862255032\\
88.3	0.00276420745410085\\
88.31	0.00276175542213854\\
88.32	0.00275930252712898\\
88.33	0.00275684876954021\\
88.34	0.00275439414984277\\
88.35	0.00275193866850968\\
88.36	0.00274948232601643\\
88.37	0.00274702512284103\\
88.38	0.00274456705946399\\
88.39	0.00274210813636834\\
88.4	0.0027396483540396\\
88.41	0.00273718771296586\\
88.42	0.00273472621363771\\
88.43	0.00273226385654829\\
88.44	0.00272980064219328\\
88.45	0.00272733657107094\\
88.46	0.00272487164368207\\
88.47	0.00272240586053003\\
88.48	0.00271993922212077\\
88.49	0.00271747172896284\\
88.5	0.00271500338156733\\
88.51	0.00271253418044799\\
88.52	0.00271006412612112\\
88.53	0.00270759321910566\\
88.54	0.00270512145992317\\
88.55	0.00270264884909782\\
88.56	0.00270017538715642\\
88.57	0.00269770107462843\\
88.58	0.00269522591204595\\
88.59	0.00269274989994373\\
88.6	0.0026902730388592\\
88.61	0.00268779532933243\\
88.62	0.0026853167719062\\
88.63	0.00268283736712596\\
88.64	0.00268035711553984\\
88.65	0.0026778760176987\\
88.66	0.00267539407415609\\
88.67	0.00267291128546825\\
88.68	0.00267042765219419\\
88.69	0.00266794317489562\\
88.7	0.00266545785413699\\
88.71	0.0026629716904855\\
88.72	0.00266048468451109\\
88.73	0.00265799683678649\\
88.74	0.00265550814788717\\
88.75	0.00265301861839137\\
88.76	0.00265052824888015\\
88.77	0.00264803703993732\\
88.78	0.0026455449921495\\
88.79	0.00264305210610614\\
88.8	0.00264055838239947\\
88.81	0.00263806382162456\\
88.82	0.00263556842437931\\
88.83	0.00263307219126445\\
88.84	0.00263057512288356\\
88.85	0.00262807721984307\\
88.86	0.00262557848275227\\
88.87	0.00262307891222332\\
88.88	0.00262057850887126\\
88.89	0.00261807727331401\\
88.9	0.00261557520617238\\
88.91	0.0026130723080701\\
88.92	0.00261056857963378\\
88.93	0.00260806402149296\\
88.94	0.00260555863428011\\
88.95	0.00260305241863062\\
88.96	0.00260054537518284\\
88.97	0.00259803750457804\\
88.98	0.00259552880746047\\
88.99	0.00259301928447734\\
89	0.00259050893627883\\
89.01	0.00258799776351812\\
89.02	0.00258548576685134\\
89.03	0.00258297294693765\\
89.04	0.00258045930443921\\
89.05	0.00257794484002119\\
89.06	0.00257542955435179\\
89.07	0.00257291344810223\\
89.08	0.00257039652194678\\
89.09	0.00256787877656276\\
89.1	0.00256536021263052\\
89.11	0.00256284083083351\\
89.12	0.00256032063185824\\
89.13	0.0025577996163943\\
89.14	0.00255527778513437\\
89.15	0.00255275513877422\\
89.16	0.00255023167801274\\
89.17	0.00254770740355194\\
89.18	0.00254518231609694\\
89.19	0.002542656416356\\
89.2	0.00254012970504053\\
89.21	0.00253760218286507\\
89.22	0.00253507385054734\\
89.23	0.00253254470880821\\
89.24	0.00253001475837174\\
89.25	0.00252748399996517\\
89.26	0.00252495243431894\\
89.27	0.00252242006216667\\
89.28	0.00251988688424523\\
89.29	0.00251735290129467\\
89.3	0.00251481811405831\\
89.31	0.00251228252328266\\
89.32	0.00250974612971753\\
89.33	0.00250720893411594\\
89.34	0.00250467093723421\\
89.35	0.00250213213983191\\
89.36	0.0024995925426719\\
89.37	0.00249705214652034\\
89.38	0.00249451095214668\\
89.39	0.00249196896032369\\
89.4	0.00248942617182746\\
89.41	0.00248688258743739\\
89.42	0.00248433820793625\\
89.43	0.00248179303411012\\
89.44	0.00247924706674846\\
89.45	0.0024767003066441\\
89.46	0.00247415275459322\\
89.47	0.00247160441139541\\
89.48	0.00246905527785363\\
89.49	0.00246650535477426\\
89.5	0.00246395464296709\\
89.51	0.00246140314324531\\
89.52	0.00245885085642559\\
89.53	0.00245629778332798\\
89.54	0.00245374392477601\\
89.55	0.00245118928159668\\
89.56	0.00244863385462044\\
89.57	0.00244607764468123\\
89.58	0.00244352065261646\\
89.59	0.00244096287926706\\
89.6	0.00243840432547745\\
89.61	0.00243584499209556\\
89.62	0.00243328487997289\\
89.63	0.00243072398996442\\
89.64	0.00242816232292872\\
89.65	0.00242559987972789\\
89.66	0.00242303666122759\\
89.67	0.00242047266829708\\
89.68	0.0024179079018092\\
89.69	0.00241534236264036\\
89.7	0.0024127760516706\\
89.71	0.00241020896978357\\
89.72	0.00240764111786654\\
89.73	0.00240507249681041\\
89.74	0.00240250310750975\\
89.75	0.00239993295086274\\
89.76	0.00239736202777127\\
89.77	0.00239479033914088\\
89.78	0.0023922178858808\\
89.79	0.00238964466890395\\
89.8	0.00238707068912697\\
89.81	0.00238449594747021\\
89.82	0.00238192044485773\\
89.83	0.00237934418221735\\
89.84	0.00237676716048061\\
89.85	0.00237418938058284\\
89.86	0.0023716108434631\\
89.87	0.00236903155006426\\
89.88	0.00236645150133295\\
89.89	0.00236387069821963\\
89.9	0.00236128914167854\\
89.91	0.00235870683266774\\
89.92	0.00235612377214914\\
89.93	0.00235353996108849\\
89.94	0.00235095540045536\\
89.95	0.0023483700912232\\
89.96	0.00234578403436936\\
89.97	0.00234319723087501\\
89.98	0.00234060968172529\\
89.99	0.00233802138790917\\
90	0.00233543235041959\\
90.01	0.00233284257025338\\
90.02	0.00233025204841133\\
90.03	0.00232766078589817\\
90.04	0.00232506878372258\\
90.05	0.00232247604289722\\
90.06	0.00231988256443873\\
90.07	0.00231728834936773\\
90.08	0.00231469339870885\\
90.09	0.00231209771349074\\
90.1	0.00230950129474606\\
90.11	0.00230690414351151\\
90.12	0.00230430626082784\\
90.13	0.00230170764773986\\
90.14	0.00229910830529645\\
90.15	0.00229650823455056\\
90.16	0.00229390743655922\\
90.17	0.0022913059123836\\
90.18	0.00228870366308896\\
90.19	0.00228610068974468\\
90.2	0.00228349699342429\\
90.21	0.00228089257520546\\
90.22	0.00227828743617002\\
90.23	0.00227568157740398\\
90.24	0.00227307499999752\\
90.25	0.00227046770504503\\
90.26	0.00226785969364509\\
90.27	0.0022652509669005\\
90.28	0.00226264152591828\\
90.29	0.00226003137180973\\
90.3	0.00225742050569035\\
90.31	0.00225480892867994\\
90.32	0.00225219664190257\\
90.33	0.00224958364648658\\
90.34	0.00224696994356463\\
90.35	0.00224435553427369\\
90.36	0.00224174041975503\\
90.37	0.00223912460115428\\
90.38	0.00223650807962141\\
90.39	0.00223389085631075\\
90.4	0.00223127293238101\\
90.41	0.00222865430899525\\
90.42	0.00222603498732096\\
90.43	0.00222341496853004\\
90.44	0.00222079425379879\\
90.45	0.00221817284430794\\
90.46	0.00221555074124268\\
90.47	0.00221292794579265\\
90.48	0.00221030445915197\\
90.49	0.00220768028251922\\
90.5	0.00220505541709749\\
90.51	0.00220242986409437\\
90.52	0.00219980362472197\\
90.53	0.00219717670019694\\
90.54	0.00219454909174046\\
90.55	0.00219192080057828\\
90.56	0.00218929182794071\\
90.57	0.00218666217506264\\
90.58	0.00218403184318356\\
90.59	0.00218140083354757\\
90.6	0.00217876914740338\\
90.61	0.00217613678600436\\
90.62	0.00217350375060848\\
90.63	0.00217087004247842\\
90.64	0.00216823566288148\\
90.65	0.0021656006130897\\
90.66	0.00216296489437977\\
90.67	0.00216032850803311\\
90.68	0.00215769145533588\\
90.69	0.00215505373757894\\
90.7	0.00215241535605794\\
90.71	0.00214977631207326\\
90.72	0.00214713660693009\\
90.73	0.00214449624193838\\
90.74	0.0021418552184129\\
90.75	0.00213921353767322\\
90.76	0.00213657120104377\\
90.77	0.0021339282098538\\
90.78	0.00213128456543742\\
90.79	0.00212864026913361\\
90.8	0.00212599532228625\\
90.81	0.0021233497262441\\
90.82	0.00212070348236085\\
90.83	0.00211805659199508\\
90.84	0.00211540905651035\\
90.85	0.00211276087727515\\
90.86	0.00211011205566294\\
90.87	0.00210746259305217\\
90.88	0.00210481249082627\\
90.89	0.0021021617503737\\
90.9	0.0020995103730879\\
90.91	0.00209685836036741\\
90.92	0.00209420571361576\\
90.93	0.00209155243424158\\
90.94	0.00208889852365856\\
90.95	0.00208624398328551\\
90.96	0.00208358881454632\\
90.97	0.00208093301887002\\
90.98	0.00207827659769075\\
90.99	0.00207561955244784\\
91	0.00207296188458575\\
91.01	0.00207030359555413\\
91.02	0.00206764468680785\\
91.03	0.00206498515980695\\
91.04	0.0020623250160167\\
91.05	0.00205966425690764\\
91.06	0.00205700288395552\\
91.07	0.0020543408986414\\
91.08	0.00205167830245158\\
91.09	0.00204901509687769\\
91.1	0.00204635128341665\\
91.11	0.00204368686357072\\
91.12	0.00204102183884751\\
91.13	0.00203835621075996\\
91.14	0.00203568998082642\\
91.15	0.00203302315057058\\
91.16	0.00203035572152157\\
91.17	0.00202768769521393\\
91.18	0.00202501907318763\\
91.19	0.00202234985698809\\
91.2	0.00201968004816619\\
91.21	0.00201700964827829\\
91.22	0.00201433865888625\\
91.23	0.00201166708155744\\
91.24	0.00200899491786475\\
91.25	0.00200632216938662\\
91.26	0.00200364883770704\\
91.27	0.00200097492441558\\
91.28	0.00199830043110739\\
91.29	0.00199562535938324\\
91.3	0.00199294971084952\\
91.31	0.00199027348711823\\
91.32	0.00198759668980705\\
91.33	0.00198491932053933\\
91.34	0.00198224138094409\\
91.35	0.00197956287265606\\
91.36	0.00197688379731568\\
91.37	0.00197420415656915\\
91.38	0.00197152395206837\\
91.39	0.00196884318547106\\
91.4	0.0019661618584407\\
91.41	0.00196347997264657\\
91.42	0.00196079752976376\\
91.43	0.0019581145314732\\
91.44	0.00195543097946168\\
91.45	0.00195274687542184\\
91.46	0.00195006222105221\\
91.47	0.00194737701805722\\
91.48	0.00194469126814722\\
91.49	0.00194200497303849\\
91.5	0.00193931813445327\\
91.51	0.00193663075411976\\
91.52	0.00193394283377214\\
91.53	0.0019312543751506\\
91.54	0.00192856538000135\\
91.55	0.00192587585007665\\
91.56	0.00192318578713479\\
91.57	0.00192049519294015\\
91.58	0.00191780406926319\\
91.59	0.00191511241788048\\
91.6	0.00191242024057473\\
91.61	0.00190972753913477\\
91.62	0.00190703431535561\\
91.63	0.00190434057103841\\
91.64	0.00190164630799057\\
91.65	0.00189895152802567\\
91.66	0.00189625623296353\\
91.67	0.00189356042463024\\
91.68	0.00189086410485813\\
91.69	0.00188816727548583\\
91.7	0.00188546993835828\\
91.71	0.00188277209532673\\
91.72	0.0018800737482488\\
91.73	0.00187737489898844\\
91.74	0.001874675549416\\
91.75	0.00187197570140821\\
91.76	0.00186927535684824\\
91.77	0.00186657451762568\\
91.78	0.00186387318563658\\
91.79	0.00186117136278346\\
91.8	0.00185846905097533\\
91.81	0.00185576625212773\\
91.82	0.0018530629681627\\
91.83	0.00185035920100886\\
91.84	0.00184765495260138\\
91.85	0.00184495022488203\\
91.86	0.00184224501979918\\
91.87	0.00183953933930784\\
91.88	0.00183683318536965\\
91.89	0.00183412655995294\\
91.9	0.0018314194650327\\
91.91	0.00182871190259065\\
91.92	0.00182600387461523\\
91.93	0.00182329538310162\\
91.94	0.00182058643005176\\
91.95	0.00181787701747442\\
91.96	0.00181516714738512\\
91.97	0.00181245682180624\\
91.98	0.00180974604276701\\
91.99	0.00180703481230352\\
92	0.00180432313245874\\
92.01	0.00180161100528257\\
92.02	0.00179889843283183\\
92.03	0.00179618541717022\\
92.04	0.00179347196036831\\
92.05	0.0017907580645035\\
92.06	0.00178804373166012\\
92.07	0.00178532896392937\\
92.08	0.00178261376340941\\
92.09	0.00177989813220532\\
92.1	0.00177718207242918\\
92.11	0.00177446558620003\\
92.12	0.00177174867564394\\
92.13	0.00176903134289401\\
92.14	0.00176631359009039\\
92.15	0.00176359541938031\\
92.16	0.00176087683291809\\
92.17	0.00175815783286517\\
92.18	0.00175543842139012\\
92.19	0.00175271860066868\\
92.2	0.00174999837288376\\
92.21	0.00174727774022549\\
92.22	0.00174455670489121\\
92.23	0.00174183526908552\\
92.24	0.00173911343502026\\
92.25	0.0017363912049146\\
92.26	0.001733668580995\\
92.27	0.00173094556549524\\
92.28	0.00172822216065649\\
92.29	0.00172549836872728\\
92.3	0.00172277419196354\\
92.31	0.00172004963262863\\
92.32	0.00171732469299335\\
92.33	0.00171459937533599\\
92.34	0.00171187368194231\\
92.35	0.0017091476151056\\
92.36	0.00170642117712667\\
92.37	0.00170369437031391\\
92.38	0.0017009671969833\\
92.39	0.0016982396594584\\
92.4	0.00169551176007044\\
92.41	0.00169278350115826\\
92.42	0.00169005488506843\\
92.43	0.00168732591415519\\
92.44	0.0016845965907805\\
92.45	0.00168186691731411\\
92.46	0.0016791368961335\\
92.47	0.00167640652962399\\
92.48	0.00167367582017869\\
92.49	0.00167094477019858\\
92.5	0.0016682133820925\\
92.51	0.00166548165827719\\
92.52	0.00166274960117732\\
92.53	0.00166001721322552\\
92.54	0.00165728449686234\\
92.55	0.00165455145453638\\
92.56	0.00165181808870424\\
92.57	0.00164908440183059\\
92.58	0.00164635039638813\\
92.59	0.0016436160748577\\
92.6	0.00164088143972826\\
92.61	0.00163814649349689\\
92.62	0.00163541123866889\\
92.63	0.00163267567775773\\
92.64	0.00162993981328513\\
92.65	0.00162720364778104\\
92.66	0.00162446718378373\\
92.67	0.00162173042383975\\
92.68	0.001618993370504\\
92.69	0.00161625602633973\\
92.7	0.00161351839391859\\
92.71	0.00161078047582065\\
92.72	0.00160804227463441\\
92.73	0.00160530379295686\\
92.74	0.00160256503339348\\
92.75	0.00159982599855827\\
92.76	0.0015970866910738\\
92.77	0.00159434711357122\\
92.78	0.00159160726869029\\
92.79	0.00158886715907939\\
92.8	0.00158612678739561\\
92.81	0.0015833861563047\\
92.82	0.00158064526848115\\
92.83	0.00157790412660821\\
92.84	0.00157516273337792\\
92.85	0.0015724210914911\\
92.86	0.00156967920365744\\
92.87	0.00156693707259551\\
92.88	0.00156419470103276\\
92.89	0.00156145209170558\\
92.9	0.00155870924735933\\
92.91	0.00155596617074834\\
92.92	0.001553222864636\\
92.93	0.00155047933179472\\
92.94	0.001547735575006\\
92.95	0.00154499159706047\\
92.96	0.00154224740075789\\
92.97	0.00153950298890721\\
92.98	0.00153675836432658\\
92.99	0.0015340135298434\\
93	0.00153126848829432\\
93.01	0.00152852324252531\\
93.02	0.0015257777953917\\
93.03	0.00152303214975814\\
93.04	0.00152028630849872\\
93.05	0.00151754027449695\\
93.06	0.00151479405064579\\
93.07	0.00151204763984774\\
93.08	0.0015093010450148\\
93.09	0.00150655426906855\\
93.1	0.00150380731494017\\
93.11	0.00150106018557047\\
93.12	0.00149831288390993\\
93.13	0.00149556541291874\\
93.14	0.00149281777556683\\
93.15	0.00149006997483388\\
93.16	0.0014873220137094\\
93.17	0.00148457389519275\\
93.18	0.00148182562229313\\
93.19	0.00147907719802969\\
93.2	0.00147632862543152\\
93.21	0.00147357990753768\\
93.22	0.00147083104739728\\
93.23	0.00146808204806947\\
93.24	0.00146533291262349\\
93.25	0.00146258364413873\\
93.26	0.00145983424570474\\
93.27	0.00145708472042127\\
93.28	0.00145433507139834\\
93.29	0.00145158530175621\\
93.3	0.0014488354146255\\
93.31	0.00144608541314718\\
93.32	0.0014433353004726\\
93.33	0.00144058507976358\\
93.34	0.00143783475419238\\
93.35	0.0014350843269418\\
93.36	0.00143233380120518\\
93.37	0.00142958318018649\\
93.38	0.00142683246710028\\
93.39	0.00142408166517183\\
93.4	0.0014213307776371\\
93.41	0.00141857980774284\\
93.42	0.00141582875874656\\
93.43	0.00141307763391666\\
93.44	0.00141032643653238\\
93.45	0.00140757516988392\\
93.46	0.00140482383727242\\
93.47	0.00140207244201006\\
93.48	0.00139932098742005\\
93.49	0.00139656947683671\\
93.5	0.00139381791360551\\
93.51	0.00139106630108308\\
93.52	0.00138831464263731\\
93.53	0.00138556294164735\\
93.54	0.00138281120150368\\
93.55	0.00138005942560812\\
93.56	0.00137730761737395\\
93.57	0.00137455578022586\\
93.58	0.00137180391760006\\
93.59	0.00136905203294432\\
93.6	0.00136630012971801\\
93.61	0.00136354821139167\\
93.62	0.00136079628144688\\
93.63	0.00135804434337625\\
93.64	0.00135529240068351\\
93.65	0.00135254045688353\\
93.66	0.00134978851550232\\
93.67	0.00134703658007717\\
93.68	0.00134428465415661\\
93.69	0.00134153274130048\\
93.7	0.00133878084507997\\
93.71	0.00133602896907769\\
93.72	0.00133327711688766\\
93.73	0.00133052529211541\\
93.74	0.00132777349837799\\
93.75	0.00132502173930403\\
93.76	0.00132227001853379\\
93.77	0.00131951833971918\\
93.78	0.00131676670652384\\
93.79	0.00131401512262316\\
93.8	0.00131126359170436\\
93.81	0.0013085121174665\\
93.82	0.00130576070362053\\
93.83	0.00130300935388939\\
93.84	0.00130025807200799\\
93.85	0.0012975068617233\\
93.86	0.00129475572679439\\
93.87	0.0012920046709925\\
93.88	0.00128925369810102\\
93.89	0.00128650281191566\\
93.9	0.00128375201624438\\
93.91	0.00128100131490752\\
93.92	0.00127825071173781\\
93.93	0.00127550021058047\\
93.94	0.00127274981529319\\
93.95	0.00126999952974627\\
93.96	0.0012672493578226\\
93.97	0.00126449930341777\\
93.98	0.00126174937044008\\
93.99	0.00125899956281063\\
94	0.00125624988446335\\
94.01	0.00125350033934508\\
94.02	0.00125075093141561\\
94.03	0.00124800166464777\\
94.04	0.00124525254302741\\
94.05	0.00124250357055356\\
94.06	0.00123975475123841\\
94.07	0.00123700608910741\\
94.08	0.00123425758819934\\
94.09	0.00123150925256632\\
94.1	0.00122876108627392\\
94.11	0.0012260130934012\\
94.12	0.00122326527804079\\
94.13	0.00122051764429893\\
94.14	0.00121777019629553\\
94.15	0.0012150229381643\\
94.16	0.00121227587405271\\
94.17	0.00120952900812214\\
94.18	0.00120678234454792\\
94.19	0.00120403588751938\\
94.2	0.00120128964123994\\
94.21	0.00119854360992717\\
94.22	0.00119579779781287\\
94.23	0.00119305220914311\\
94.24	0.00119030684817835\\
94.25	0.00118756171919345\\
94.26	0.00118481682647779\\
94.27	0.00118207217433532\\
94.28	0.00117932776708466\\
94.29	0.00117658360905912\\
94.3	0.00117383970460685\\
94.31	0.00117109605809083\\
94.32	0.00116835267388903\\
94.33	0.00116560955639443\\
94.34	0.00116286671001511\\
94.35	0.00116012413917436\\
94.36	0.00115738184831071\\
94.37	0.00115463984187806\\
94.38	0.00115189812434571\\
94.39	0.00114915670019849\\
94.4	0.00114641557393682\\
94.41	0.00114367475007458\\
94.42	0.00114093423313898\\
94.43	0.00113819402767061\\
94.44	0.00113545413822342\\
94.45	0.00113271456936483\\
94.46	0.00112997532567568\\
94.47	0.00112723641175035\\
94.48	0.00112449783219672\\
94.49	0.00112175959163623\\
94.5	0.00111902169470394\\
94.51	0.00111628414604853\\
94.52	0.00111354695033234\\
94.53	0.00111081011223142\\
94.54	0.00110807363643554\\
94.55	0.00110533752764826\\
94.56	0.00110260179058692\\
94.57	0.00109986642998269\\
94.58	0.00109713145058066\\
94.59	0.00109439685713976\\
94.6	0.0010916626544329\\
94.61	0.00108892884724695\\
94.62	0.0010861954403828\\
94.63	0.00108346243865537\\
94.64	0.00108072984689367\\
94.65	0.0010779976699408\\
94.66	0.00107526591265404\\
94.67	0.00107253457990483\\
94.68	0.00106980367657883\\
94.69	0.00106707320757587\\
94.7	0.00106434317780998\\
94.71	0.00106161359220937\\
94.72	0.00105888445571649\\
94.73	0.00105615577328809\\
94.74	0.00105342754989522\\
94.75	0.00105069979052326\\
94.76	0.001047972500172\\
94.77	0.00104524568385565\\
94.78	0.00104251934660285\\
94.79	0.00103979349345675\\
94.8	0.00103706812947502\\
94.81	0.00103434325972991\\
94.82	0.00103161888930824\\
94.83	0.00102889502331148\\
94.84	0.00102617166685578\\
94.85	0.00102344882507198\\
94.86	0.00102072650310568\\
94.87	0.00101800470611725\\
94.88	0.00101528343928187\\
94.89	0.0010125627077896\\
94.9	0.00100984251684536\\
94.91	0.00100712287166901\\
94.92	0.00100440377749539\\
94.93	0.0010016852395743\\
94.94	0.000998967263170622\\
94.95	0.000996249853564285\\
94.96	0.000993533016050334\\
94.97	0.000990816755938971\\
94.98	0.000988101078555579\\
94.99	0.000985385989240762\\
95	0.000982671493350394\\
95.01	0.000979957596255637\\
95.02	0.000977244303342993\\
95.03	0.000974531620014343\\
95.04	0.000971819551686971\\
95.05	0.000969108103793619\\
95.06	0.00096639728178251\\
95.07	0.000963687091117394\\
95.08	0.000960977537277582\\
95.09	0.000958268625757994\\
95.1	0.00095556036206918\\
95.11	0.000952852751737374\\
95.12	0.000950145800304529\\
95.13	0.000947439513328341\\
95.14	0.00094473389638231\\
95.15	0.000942028955055766\\
95.16	0.000939324694953909\\
95.17	0.000936621121697843\\
95.18	0.000933918240924628\\
95.19	0.000931216058287306\\
95.2	0.000928514579454946\\
95.21	0.000925813810112685\\
95.22	0.00092311375596176\\
95.23	0.00092041442271955\\
95.24	0.000917715816119621\\
95.25	0.000915017941911763\\
95.26	0.000912320805862022\\
95.27	0.000909624413752747\\
95.28	0.000906928771382627\\
95.29	0.000904233884566735\\
95.3	0.000901539759136556\\
95.31	0.000898846400940045\\
95.32	0.000896153815841648\\
95.33	0.000893462009722358\\
95.34	0.000890770988479742\\
95.35	0.000888080758027987\\
95.36	0.000885391324297947\\
95.37	0.00088270269323717\\
95.38	0.00088001487080995\\
95.39	0.000877327862997357\\
95.4	0.000874641675797283\\
95.41	0.000871956315224489\\
95.42	0.000869271787310633\\
95.43	0.00086658809810432\\
95.44	0.000863905253671138\\
95.45	0.0008612232600937\\
95.46	0.00085854212347169\\
95.47	0.000855861849921896\\
95.48	0.000853182445578258\\
95.49	0.0008505039165919\\
95.5	0.000847826269131187\\
95.51	0.000845149509381747\\
95.52	0.00084247364354653\\
95.53	0.000839798677845841\\
95.54	0.000837124618517374\\
95.55	0.000834451471816275\\
95.56	0.000831779244015163\\
95.57	0.000829107941404183\\
95.58	0.000826437570291041\\
95.59	0.000823768137001054\\
95.6	0.000821099647877186\\
95.61	0.000818432109280093\\
95.62	0.000815765527588164\\
95.63	0.000813099909197562\\
95.64	0.00081043526052227\\
95.65	0.000807771587994133\\
95.66	0.000805108898062895\\
95.67	0.000802447197196247\\
95.68	0.000799786491879872\\
95.69	0.000797126788617481\\
95.7	0.000794468093930854\\
95.71	0.000791810414359899\\
95.72	0.000789153756462678\\
95.73	0.000786498126815454\\
95.74	0.00078384353201274\\
95.75	0.000781189978667338\\
95.76	0.000778537473410381\\
95.77	0.000775886022891378\\
95.78	0.00077323563377826\\
95.79	0.000770586312757419\\
95.8	0.000767938066533757\\
95.81	0.000765290901830723\\
95.82	0.00076264482539036\\
95.83	0.000759999843973357\\
95.84	0.000757355964359073\\
95.85	0.000754713193345599\\
95.86	0.000752071537749801\\
95.87	0.000749431004407351\\
95.88	0.000746791600172785\\
95.89	0.00074415333191954\\
95.9	0.000741516206539998\\
95.91	0.000738880230945538\\
95.92	0.000736245412066575\\
95.93	0.000733611756852595\\
95.94	0.000730979272272224\\
95.95	0.000728347965313243\\
95.96	0.000725717842982665\\
95.97	0.000723088912306749\\
95.98	0.000720461180331066\\
95.99	0.000717834654120534\\
96	0.000715209340759472\\
96.01	0.000712585247351627\\
96.02	0.00070996238102025\\
96.03	0.000707340748908105\\
96.04	0.00070472035817754\\
96.05	0.000702101216010531\\
96.06	0.000699483329608713\\
96.07	0.000696866706193436\\
96.08	0.000694251353005814\\
96.09	0.000691637277306754\\
96.1	0.000689024486377027\\
96.11	0.000686412987517294\\
96.12	0.00068380278804816\\
96.13	0.000681193895310213\\
96.14	0.000678586316664087\\
96.15	0.000675980059490487\\
96.16	0.000673375131190248\\
96.17	0.00067077153918438\\
96.18	0.000668169290914115\\
96.19	0.000665568393840946\\
96.2	0.000662968855446687\\
96.21	0.000660370683233501\\
96.22	0.000657773884723968\\
96.23	0.000655178467461116\\
96.24	0.000652584439008477\\
96.25	0.000649991806950122\\
96.26	0.000647400578890726\\
96.27	0.0006448107624556\\
96.28	0.000642222365290745\\
96.29	0.000639635395062895\\
96.3	0.000637049859459568\\
96.31	0.00063446576618912\\
96.32	0.000631883122980769\\
96.33	0.000629301937584674\\
96.34	0.000626722217771958\\
96.35	0.000624143971334762\\
96.36	0.000621567206086302\\
96.37	0.000618991929860912\\
96.38	0.000616418150514076\\
96.39	0.000613845875922501\\
96.4	0.000611275113984153\\
96.41	0.000608705872618302\\
96.42	0.000606138159765573\\
96.43	0.000603571983388\\
96.44	0.000601007351469063\\
96.45	0.000598444272013747\\
96.46	0.000595882753048581\\
96.47	0.000593322802621693\\
96.48	0.000590764428802862\\
96.49	0.000588207639683555\\
96.5	0.00058565244337698\\
96.51	0.000583098848018141\\
96.52	0.00058054686176388\\
96.53	0.00057799649279293\\
96.54	0.000575447749305961\\
96.55	0.00057290063952563\\
96.56	0.000570355171696625\\
96.57	0.000567811354085727\\
96.58	0.000565269194981845\\
96.59	0.000562728702696073\\
96.6	0.000560189885561739\\
96.61	0.000557652751934449\\
96.62	0.000555117310192142\\
96.63	0.000552583568735139\\
96.64	0.000550051535986187\\
96.65	0.000547521220390515\\
96.66	0.000544992630415884\\
96.67	0.000542465774552628\\
96.68	0.00053994066131371\\
96.69	0.000537417299234776\\
96.7	0.000534895696874198\\
96.71	0.000532375862813123\\
96.72	0.000529857805655527\\
96.73	0.000527341534028265\\
96.74	0.00052482705658112\\
96.75	0.000522314381986855\\
96.76	0.000519803518941254\\
96.77	0.000517294476163187\\
96.78	0.00051478726239465\\
96.79	0.000512281886400819\\
96.8	0.0005097783569701\\
96.81	0.000507276682914179\\
96.82	0.00050477687306808\\
96.83	0.000502278936290202\\
96.84	0.000499782881462378\\
96.85	0.000497288717489929\\
96.86	0.000494796453301713\\
96.87	0.000492306097850169\\
96.88	0.000489817660111372\\
96.89	0.000487331149085096\\
96.9	0.000484846573794842\\
96.91	0.000482363943287909\\
96.92	0.000479883266635438\\
96.93	0.000477404552932464\\
96.94	0.000474927811297963\\
96.95	0.000472453050874911\\
96.96	0.00046998028083033\\
96.97	0.000467509510355341\\
96.98	0.000465040748665222\\
96.99	0.000462574004999444\\
97	0.00046010928862174\\
97.01	0.000457646608820146\\
97.02	0.000455185974907058\\
97.03	0.000452727396219278\\
97.04	0.000450270882118072\\
97.05	0.000447816441989221\\
97.06	0.000445364085243073\\
97.07	0.000442913821314585\\
97.08	0.000440465659663393\\
97.09	0.000438019609773853\\
97.1	0.000435575681155089\\
97.11	0.000433133883341057\\
97.12	0.000430694225890593\\
97.13	0.000428256718387459\\
97.14	0.000425821370440399\\
97.15	0.000423388191683201\\
97.16	0.000420957191774733\\
97.17	0.000418528380399004\\
97.18	0.000416101767265219\\
97.19	0.000413677362107824\\
97.2	0.000411255174686572\\
97.21	0.000408835214786556\\
97.22	0.000406417492218277\\
97.23	0.000404002016817695\\
97.24	0.000401588798446271\\
97.25	0.000399177846991035\\
97.26	0.000396769172364624\\
97.27	0.000394362784505352\\
97.28	0.000391958693377241\\
97.29	0.000389556908970096\\
97.3	0.000387157441299543\\
97.31	0.000384760300407086\\
97.32	0.000382365496360162\\
97.33	0.000379973039252194\\
97.34	0.000377582939202642\\
97.35	0.000375195206357055\\
97.36	0.00037280985088713\\
97.37	0.000370426882990755\\
97.38	0.000368046312892077\\
97.39	0.00036566815084154\\
97.4	0.000363292407115945\\
97.41	0.000360919092018503\\
97.42	0.000358548215878895\\
97.43	0.000356179789053309\\
97.44	0.000353813821924509\\
97.45	0.000351450324901881\\
97.46	0.000349089308421487\\
97.47	0.000346730782946118\\
97.48	0.000344374758965349\\
97.49	0.000342021246995595\\
97.5	0.000339670257580157\\
97.51	0.000337321801289281\\
97.52	0.000334975888720212\\
97.53	0.000332632530497244\\
97.54	0.000330291737271775\\
97.55	0.00032795351972236\\
97.56	0.000325617888554766\\
97.57	0.000323284854502028\\
97.58	0.000320954428324496\\
97.59	0.00031862662080989\\
97.6	0.000316301442773361\\
97.61	0.000313978905057534\\
97.62	0.00031165901853257\\
97.63	0.000309341794096212\\
97.64	0.000307027242673851\\
97.65	0.000304715375218562\\
97.66	0.000302406202711172\\
97.67	0.00030009973616031\\
97.68	0.000297795986602458\\
97.69	0.000295494965102002\\
97.7	0.000293196682751296\\
97.71	0.000290901150670705\\
97.72	0.000288608380008664\\
97.73	0.000286318381941731\\
97.74	0.000284031167674642\\
97.75	0.000281746748440359\\
97.76	0.000279465135500128\\
97.77	0.000277186340143534\\
97.78	0.00027491037368855\\
97.79	0.000272637247481593\\
97.8	0.000270366972897583\\
97.81	0.000268099561339984\\
97.82	0.000265835024240869\\
97.83	0.000263573373060967\\
97.84	0.000261314619289719\\
97.85	0.000259058774445335\\
97.86	0.000256805850074837\\
97.87	0.000254555857754126\\
97.88	0.000252308809088024\\
97.89	0.000250064715710331\\
97.9	0.000247823589283885\\
97.91	0.000245585441500601\\
97.92	0.000243350284081543\\
97.93	0.000241118128777033\\
97.94	0.000238888987366764\\
97.95	0.000236662871659852\\
97.96	0.00023443979349488\\
97.97	0.000232219764739949\\
97.98	0.00023000279729273\\
97.99	0.000227788903080513\\
98	0.000225578094060265\\
98.01	0.00022337038221871\\
98.02	0.000221165779572775\\
98.03	0.000218964298170168\\
98.04	0.000216765950089454\\
98.05	0.000214570747440119\\
98.06	0.000212378702362634\\
98.07	0.000210189827028522\\
98.08	0.000208004133640411\\
98.09	0.000205821634432111\\
98.1	0.000203642341481064\\
98.11	0.000201466266657718\\
98.12	0.000199293421865464\\
98.13	0.000197123819040758\\
98.14	0.000194957470153214\\
98.15	0.000192794387205738\\
98.16	0.000190634582207314\\
98.17	0.000188478067045202\\
98.18	0.000186324852593203\\
98.19	0.00018417494775237\\
98.2	0.000182028361422695\\
98.21	0.000179885102502837\\
98.22	0.000177745179889843\\
98.23	0.000175608602478861\\
98.24	0.000173475379162853\\
98.25	0.0001713455188323\\
98.26	0.0001692190303749\\
98.27	0.000167095922675274\\
98.28	0.000164976204614647\\
98.29	0.000162859885070541\\
98.3	0.000160746972916457\\
98.31	0.000158637477021552\\
98.32	0.000156531406250314\\
98.33	0.000154428889410309\\
98.34	0.000152329946462073\\
98.35	0.000150235990568359\\
98.36	0.000148147035042449\\
98.37	0.000146063093212499\\
98.38	0.000143984178421541\\
98.39	0.000141910304027489\\
98.4	0.000139841771773151\\
98.41	0.00013777991434433\\
98.42	0.000135733424685895\\
98.43	0.000133702413337858\\
98.44	0.000131686991825733\\
98.45	0.000129687272671966\\
98.46	0.00012770336940757\\
98.47	0.000125735396583913\\
98.48	0.000123783469784707\\
98.49	0.000121847705638169\\
98.5	0.000119928221829378\\
98.51	0.000118025137112818\\
98.52	0.000116138571325128\\
98.53	0.000114268645398031\\
98.54	0.000112415481371494\\
98.55	0.000110579202407077\\
98.56	0.000108759932629461\\
98.57	0.000106957796004338\\
98.58	0.000105172917629471\\
98.59	0.000103405423748352\\
98.6	0.000101655441764102\\
98.61	9.99231002535763e-05\\
98.62	9.82085289817029e-05\\
98.63	9.6511848141816e-05\\
98.64	9.48331057032663e-05\\
98.65	9.31724359905829e-05\\
98.66	9.15299745840991e-05\\
98.67	8.99058583355688e-05\\
98.68	8.83002253840651e-05\\
98.69	8.67132151721045e-05\\
98.7	8.51449694648683e-05\\
98.71	8.35956315950563e-05\\
98.72	8.20653462532666e-05\\
98.73	8.05542595052854e-05\\
98.74	7.90619610721213e-05\\
98.75	7.75879928146608e-05\\
98.76	7.61324997723516e-05\\
98.77	7.4695628389158e-05\\
98.78	7.32774004500343e-05\\
98.79	7.18773164142782e-05\\
98.8	7.04879875739993e-05\\
98.81	6.91094829359008e-05\\
98.82	6.77418720203167e-05\\
98.83	6.63852248652241e-05\\
98.84	6.50396120302644e-05\\
98.85	6.37051046008009e-05\\
98.86	6.23817741919903e-05\\
98.87	6.10696929528778e-05\\
98.88	5.97689335705091e-05\\
98.89	5.8479569274076e-05\\
98.9	5.72016738390625e-05\\
98.91	5.59353215914375e-05\\
98.92	5.46805884606635e-05\\
98.93	5.34375510873788e-05\\
98.94	5.22062867035281e-05\\
98.95	5.09868731372199e-05\\
98.96	4.97793888175994e-05\\
98.97	4.85839127797802e-05\\
98.98	4.74005246697794e-05\\
98.99	4.62293047495052e-05\\
99	4.50703339017599e-05\\
99.01	4.39236936352807e-05\\
99.02	4.27894660898227e-05\\
99.03	4.16677340412432e-05\\
99.04	4.05585809066453e-05\\
99.05	3.94620907495404e-05\\
99.06	3.83783482850279e-05\\
99.07	3.73074388850345e-05\\
99.08	3.62494485835439e-05\\
99.09	3.52044640818897e-05\\
99.1	3.41725727540567e-05\\
99.11	3.31538626520134e-05\\
99.12	3.21484225110687e-05\\
99.13	3.11563417552743e-05\\
99.14	3.01777105028277e-05\\
99.15	2.92126195715094e-05\\
99.16	2.82611604841623e-05\\
99.17	2.73234254741638e-05\\
99.18	2.63995074909502e-05\\
99.19	2.54895002014359e-05\\
99.2	2.45934979939289e-05\\
99.21	2.37115959836436e-05\\
99.22	2.28438900182325e-05\\
99.23	2.19904766833416e-05\\
99.24	2.11514533081832e-05\\
99.25	2.03269179711221e-05\\
99.26	1.95169695052736e-05\\
99.27	1.87217075041362e-05\\
99.28	1.79412323272118e-05\\
99.29	1.71756451056593e-05\\
99.3	1.64250477479484e-05\\
99.31	1.56895429455264e-05\\
99.32	1.49692341784944e-05\\
99.33	1.42642257212871e-05\\
99.34	1.35746226483605e-05\\
99.35	1.29005308398839e-05\\
99.36	1.22420569874312e-05\\
99.37	1.15993085996729e-05\\
99.38	1.09723940080689e-05\\
99.39	1.03614226602163e-05\\
99.4	9.76651714497574e-06\\
99.41	9.18780102560947e-06\\
99.42	8.62539884664143e-06\\
99.43	8.07943614066463e-06\\
99.44	7.55003943518281e-06\\
99.45	7.03733625940541e-06\\
99.46	6.5414551510528e-06\\
99.47	6.06252566313741e-06\\
99.48	5.60067837072735e-06\\
99.49	5.15604487768065e-06\\
99.5	4.72875782336381e-06\\
99.51	4.3189508893253e-06\\
99.52	3.92675880595342e-06\\
99.53	3.5523173590752e-06\\
99.54	3.19576339652752e-06\\
99.55	2.85723483467448e-06\\
99.56	2.53687066486923e-06\\
99.57	2.23481095987417e-06\\
99.58	1.95119688019101e-06\\
99.59	1.68617068034213e-06\\
99.6	1.4398757150879e-06\\
99.61	1.2124564455207e-06\\
99.62	1.00405963195105e-06\\
99.63	8.14835682361875e-07\\
99.64	6.449361741271e-07\\
99.65	4.94513860235801e-07\\
99.66	3.63722675407116e-07\\
99.67	2.52717742073305e-07\\
99.68	1.61655376246586e-07\\
99.69	9.069309325066e-08\\
99.7	3.99896132857042e-08\\
99.71	9.70486685111793e-09\\
99.72	0\\
99.73	0\\
99.74	0\\
99.75	0\\
99.76	0\\
99.77	0\\
99.78	0\\
99.79	0\\
99.8	0\\
99.81	0\\
99.82	0\\
99.83	0\\
99.84	0\\
99.85	0\\
99.86	0\\
99.87	0\\
99.88	0\\
99.89	0\\
99.9	0\\
99.91	0\\
99.92	0\\
99.93	0\\
99.94	0\\
99.95	0\\
99.96	0\\
99.97	0\\
99.98	0\\
99.99	0\\
100	0\\
};
\addlegendentry{$q=-2$};

\addplot [color=blue,dashed,forget plot]
  table[row sep=crcr]{%
0.01	0.01\\
0.02	0.01\\
0.03	0.01\\
0.04	0.01\\
0.05	0.01\\
0.06	0.01\\
0.07	0.01\\
0.08	0.01\\
0.09	0.01\\
0.1	0.01\\
0.11	0.01\\
0.12	0.01\\
0.13	0.01\\
0.14	0.01\\
0.15	0.01\\
0.16	0.01\\
0.17	0.01\\
0.18	0.01\\
0.19	0.01\\
0.2	0.01\\
0.21	0.01\\
0.22	0.01\\
0.23	0.01\\
0.24	0.01\\
0.25	0.01\\
0.26	0.01\\
0.27	0.01\\
0.28	0.01\\
0.29	0.01\\
0.3	0.01\\
0.31	0.01\\
0.32	0.01\\
0.33	0.01\\
0.34	0.01\\
0.35	0.01\\
0.36	0.01\\
0.37	0.01\\
0.38	0.01\\
0.39	0.01\\
0.4	0.01\\
0.41	0.01\\
0.42	0.01\\
0.43	0.01\\
0.44	0.01\\
0.45	0.01\\
0.46	0.01\\
0.47	0.01\\
0.48	0.01\\
0.49	0.01\\
0.5	0.01\\
0.51	0.01\\
0.52	0.01\\
0.53	0.01\\
0.54	0.01\\
0.55	0.01\\
0.56	0.01\\
0.57	0.01\\
0.58	0.01\\
0.59	0.01\\
0.6	0.01\\
0.61	0.01\\
0.62	0.01\\
0.63	0.01\\
0.64	0.01\\
0.65	0.01\\
0.66	0.01\\
0.67	0.01\\
0.68	0.01\\
0.69	0.01\\
0.7	0.01\\
0.71	0.01\\
0.72	0.01\\
0.73	0.01\\
0.74	0.01\\
0.75	0.01\\
0.76	0.01\\
0.77	0.01\\
0.78	0.01\\
0.79	0.01\\
0.8	0.01\\
0.81	0.01\\
0.82	0.01\\
0.83	0.01\\
0.84	0.01\\
0.85	0.01\\
0.86	0.01\\
0.87	0.01\\
0.88	0.01\\
0.89	0.01\\
0.9	0.01\\
0.91	0.01\\
0.92	0.01\\
0.93	0.01\\
0.94	0.01\\
0.95	0.01\\
0.96	0.01\\
0.97	0.01\\
0.98	0.01\\
0.99	0.01\\
1	0.01\\
1.01	0.01\\
1.02	0.01\\
1.03	0.01\\
1.04	0.01\\
1.05	0.01\\
1.06	0.01\\
1.07	0.01\\
1.08	0.01\\
1.09	0.01\\
1.1	0.01\\
1.11	0.01\\
1.12	0.01\\
1.13	0.01\\
1.14	0.01\\
1.15	0.01\\
1.16	0.01\\
1.17	0.01\\
1.18	0.01\\
1.19	0.01\\
1.2	0.01\\
1.21	0.01\\
1.22	0.01\\
1.23	0.01\\
1.24	0.01\\
1.25	0.01\\
1.26	0.01\\
1.27	0.01\\
1.28	0.01\\
1.29	0.01\\
1.3	0.01\\
1.31	0.01\\
1.32	0.01\\
1.33	0.01\\
1.34	0.01\\
1.35	0.01\\
1.36	0.01\\
1.37	0.01\\
1.38	0.01\\
1.39	0.01\\
1.4	0.01\\
1.41	0.01\\
1.42	0.01\\
1.43	0.01\\
1.44	0.01\\
1.45	0.01\\
1.46	0.01\\
1.47	0.01\\
1.48	0.01\\
1.49	0.01\\
1.5	0.01\\
1.51	0.01\\
1.52	0.01\\
1.53	0.01\\
1.54	0.01\\
1.55	0.01\\
1.56	0.01\\
1.57	0.01\\
1.58	0.01\\
1.59	0.01\\
1.6	0.01\\
1.61	0.01\\
1.62	0.01\\
1.63	0.01\\
1.64	0.01\\
1.65	0.01\\
1.66	0.01\\
1.67	0.01\\
1.68	0.01\\
1.69	0.01\\
1.7	0.01\\
1.71	0.01\\
1.72	0.01\\
1.73	0.01\\
1.74	0.01\\
1.75	0.01\\
1.76	0.01\\
1.77	0.01\\
1.78	0.01\\
1.79	0.01\\
1.8	0.01\\
1.81	0.01\\
1.82	0.01\\
1.83	0.01\\
1.84	0.01\\
1.85	0.01\\
1.86	0.01\\
1.87	0.01\\
1.88	0.01\\
1.89	0.01\\
1.9	0.01\\
1.91	0.01\\
1.92	0.01\\
1.93	0.01\\
1.94	0.01\\
1.95	0.01\\
1.96	0.01\\
1.97	0.01\\
1.98	0.01\\
1.99	0.01\\
2	0.01\\
2.01	0.01\\
2.02	0.01\\
2.03	0.01\\
2.04	0.01\\
2.05	0.01\\
2.06	0.01\\
2.07	0.01\\
2.08	0.01\\
2.09	0.01\\
2.1	0.01\\
2.11	0.01\\
2.12	0.01\\
2.13	0.01\\
2.14	0.01\\
2.15	0.01\\
2.16	0.01\\
2.17	0.01\\
2.18	0.01\\
2.19	0.01\\
2.2	0.01\\
2.21	0.01\\
2.22	0.01\\
2.23	0.01\\
2.24	0.01\\
2.25	0.01\\
2.26	0.01\\
2.27	0.01\\
2.28	0.01\\
2.29	0.01\\
2.3	0.01\\
2.31	0.01\\
2.32	0.01\\
2.33	0.01\\
2.34	0.01\\
2.35	0.01\\
2.36	0.01\\
2.37	0.01\\
2.38	0.01\\
2.39	0.01\\
2.4	0.01\\
2.41	0.01\\
2.42	0.01\\
2.43	0.01\\
2.44	0.01\\
2.45	0.01\\
2.46	0.01\\
2.47	0.01\\
2.48	0.01\\
2.49	0.01\\
2.5	0.01\\
2.51	0.01\\
2.52	0.01\\
2.53	0.01\\
2.54	0.01\\
2.55	0.01\\
2.56	0.01\\
2.57	0.01\\
2.58	0.01\\
2.59	0.01\\
2.6	0.01\\
2.61	0.01\\
2.62	0.01\\
2.63	0.01\\
2.64	0.01\\
2.65	0.01\\
2.66	0.01\\
2.67	0.01\\
2.68	0.01\\
2.69	0.01\\
2.7	0.01\\
2.71	0.01\\
2.72	0.01\\
2.73	0.01\\
2.74	0.01\\
2.75	0.01\\
2.76	0.01\\
2.77	0.01\\
2.78	0.01\\
2.79	0.01\\
2.8	0.01\\
2.81	0.01\\
2.82	0.01\\
2.83	0.01\\
2.84	0.01\\
2.85	0.01\\
2.86	0.01\\
2.87	0.01\\
2.88	0.01\\
2.89	0.01\\
2.9	0.01\\
2.91	0.01\\
2.92	0.01\\
2.93	0.01\\
2.94	0.01\\
2.95	0.01\\
2.96	0.01\\
2.97	0.01\\
2.98	0.01\\
2.99	0.01\\
3	0.01\\
3.01	0.01\\
3.02	0.01\\
3.03	0.01\\
3.04	0.01\\
3.05	0.01\\
3.06	0.01\\
3.07	0.01\\
3.08	0.01\\
3.09	0.01\\
3.1	0.01\\
3.11	0.01\\
3.12	0.01\\
3.13	0.01\\
3.14	0.01\\
3.15	0.01\\
3.16	0.01\\
3.17	0.01\\
3.18	0.01\\
3.19	0.01\\
3.2	0.01\\
3.21	0.01\\
3.22	0.01\\
3.23	0.01\\
3.24	0.01\\
3.25	0.01\\
3.26	0.01\\
3.27	0.01\\
3.28	0.01\\
3.29	0.01\\
3.3	0.01\\
3.31	0.01\\
3.32	0.01\\
3.33	0.01\\
3.34	0.01\\
3.35	0.01\\
3.36	0.01\\
3.37	0.01\\
3.38	0.01\\
3.39	0.01\\
3.4	0.01\\
3.41	0.01\\
3.42	0.01\\
3.43	0.01\\
3.44	0.01\\
3.45	0.01\\
3.46	0.01\\
3.47	0.01\\
3.48	0.01\\
3.49	0.01\\
3.5	0.01\\
3.51	0.01\\
3.52	0.01\\
3.53	0.01\\
3.54	0.01\\
3.55	0.01\\
3.56	0.01\\
3.57	0.01\\
3.58	0.01\\
3.59	0.01\\
3.6	0.01\\
3.61	0.01\\
3.62	0.01\\
3.63	0.01\\
3.64	0.01\\
3.65	0.01\\
3.66	0.01\\
3.67	0.01\\
3.68	0.01\\
3.69	0.01\\
3.7	0.01\\
3.71	0.01\\
3.72	0.01\\
3.73	0.01\\
3.74	0.01\\
3.75	0.01\\
3.76	0.01\\
3.77	0.01\\
3.78	0.01\\
3.79	0.01\\
3.8	0.01\\
3.81	0.01\\
3.82	0.01\\
3.83	0.01\\
3.84	0.01\\
3.85	0.01\\
3.86	0.01\\
3.87	0.01\\
3.88	0.01\\
3.89	0.01\\
3.9	0.01\\
3.91	0.01\\
3.92	0.01\\
3.93	0.01\\
3.94	0.01\\
3.95	0.01\\
3.96	0.01\\
3.97	0.01\\
3.98	0.01\\
3.99	0.01\\
4	0.01\\
4.01	0.01\\
4.02	0.01\\
4.03	0.01\\
4.04	0.01\\
4.05	0.01\\
4.06	0.01\\
4.07	0.01\\
4.08	0.01\\
4.09	0.01\\
4.1	0.01\\
4.11	0.01\\
4.12	0.01\\
4.13	0.01\\
4.14	0.01\\
4.15	0.01\\
4.16	0.01\\
4.17	0.01\\
4.18	0.01\\
4.19	0.01\\
4.2	0.01\\
4.21	0.01\\
4.22	0.01\\
4.23	0.01\\
4.24	0.01\\
4.25	0.01\\
4.26	0.01\\
4.27	0.01\\
4.28	0.01\\
4.29	0.01\\
4.3	0.01\\
4.31	0.01\\
4.32	0.01\\
4.33	0.01\\
4.34	0.01\\
4.35	0.01\\
4.36	0.01\\
4.37	0.01\\
4.38	0.01\\
4.39	0.01\\
4.4	0.01\\
4.41	0.01\\
4.42	0.01\\
4.43	0.01\\
4.44	0.01\\
4.45	0.01\\
4.46	0.01\\
4.47	0.01\\
4.48	0.01\\
4.49	0.01\\
4.5	0.01\\
4.51	0.01\\
4.52	0.01\\
4.53	0.01\\
4.54	0.01\\
4.55	0.01\\
4.56	0.01\\
4.57	0.01\\
4.58	0.01\\
4.59	0.01\\
4.6	0.01\\
4.61	0.01\\
4.62	0.01\\
4.63	0.01\\
4.64	0.01\\
4.65	0.01\\
4.66	0.01\\
4.67	0.01\\
4.68	0.01\\
4.69	0.01\\
4.7	0.01\\
4.71	0.01\\
4.72	0.01\\
4.73	0.01\\
4.74	0.01\\
4.75	0.01\\
4.76	0.01\\
4.77	0.01\\
4.78	0.01\\
4.79	0.01\\
4.8	0.01\\
4.81	0.01\\
4.82	0.01\\
4.83	0.01\\
4.84	0.01\\
4.85	0.01\\
4.86	0.01\\
4.87	0.01\\
4.88	0.01\\
4.89	0.01\\
4.9	0.01\\
4.91	0.01\\
4.92	0.01\\
4.93	0.01\\
4.94	0.01\\
4.95	0.01\\
4.96	0.01\\
4.97	0.01\\
4.98	0.01\\
4.99	0.01\\
5	0.01\\
5.01	0.01\\
5.02	0.01\\
5.03	0.01\\
5.04	0.01\\
5.05	0.01\\
5.06	0.01\\
5.07	0.01\\
5.08	0.01\\
5.09	0.01\\
5.1	0.01\\
5.11	0.01\\
5.12	0.01\\
5.13	0.01\\
5.14	0.01\\
5.15	0.01\\
5.16	0.01\\
5.17	0.01\\
5.18	0.01\\
5.19	0.01\\
5.2	0.01\\
5.21	0.01\\
5.22	0.01\\
5.23	0.01\\
5.24	0.01\\
5.25	0.01\\
5.26	0.01\\
5.27	0.01\\
5.28	0.01\\
5.29	0.01\\
5.3	0.01\\
5.31	0.01\\
5.32	0.01\\
5.33	0.01\\
5.34	0.01\\
5.35	0.01\\
5.36	0.01\\
5.37	0.01\\
5.38	0.01\\
5.39	0.01\\
5.4	0.01\\
5.41	0.01\\
5.42	0.01\\
5.43	0.01\\
5.44	0.01\\
5.45	0.01\\
5.46	0.01\\
5.47	0.01\\
5.48	0.01\\
5.49	0.01\\
5.5	0.01\\
5.51	0.01\\
5.52	0.01\\
5.53	0.01\\
5.54	0.01\\
5.55	0.01\\
5.56	0.01\\
5.57	0.01\\
5.58	0.01\\
5.59	0.01\\
5.6	0.01\\
5.61	0.01\\
5.62	0.01\\
5.63	0.01\\
5.64	0.01\\
5.65	0.01\\
5.66	0.01\\
5.67	0.01\\
5.68	0.01\\
5.69	0.01\\
5.7	0.01\\
5.71	0.01\\
5.72	0.01\\
5.73	0.01\\
5.74	0.01\\
5.75	0.01\\
5.76	0.01\\
5.77	0.01\\
5.78	0.01\\
5.79	0.01\\
5.8	0.01\\
5.81	0.01\\
5.82	0.01\\
5.83	0.01\\
5.84	0.01\\
5.85	0.01\\
5.86	0.01\\
5.87	0.01\\
5.88	0.01\\
5.89	0.01\\
5.9	0.01\\
5.91	0.01\\
5.92	0.01\\
5.93	0.01\\
5.94	0.01\\
5.95	0.01\\
5.96	0.01\\
5.97	0.01\\
5.98	0.01\\
5.99	0.01\\
6	0.01\\
6.01	0.01\\
6.02	0.01\\
6.03	0.01\\
6.04	0.01\\
6.05	0.01\\
6.06	0.01\\
6.07	0.01\\
6.08	0.01\\
6.09	0.01\\
6.1	0.01\\
6.11	0.01\\
6.12	0.01\\
6.13	0.01\\
6.14	0.01\\
6.15	0.01\\
6.16	0.01\\
6.17	0.01\\
6.18	0.01\\
6.19	0.01\\
6.2	0.01\\
6.21	0.01\\
6.22	0.01\\
6.23	0.01\\
6.24	0.01\\
6.25	0.01\\
6.26	0.01\\
6.27	0.01\\
6.28	0.01\\
6.29	0.01\\
6.3	0.01\\
6.31	0.01\\
6.32	0.01\\
6.33	0.01\\
6.34	0.01\\
6.35	0.01\\
6.36	0.01\\
6.37	0.01\\
6.38	0.01\\
6.39	0.01\\
6.4	0.01\\
6.41	0.01\\
6.42	0.01\\
6.43	0.01\\
6.44	0.01\\
6.45	0.01\\
6.46	0.01\\
6.47	0.01\\
6.48	0.01\\
6.49	0.01\\
6.5	0.01\\
6.51	0.01\\
6.52	0.01\\
6.53	0.01\\
6.54	0.01\\
6.55	0.01\\
6.56	0.01\\
6.57	0.01\\
6.58	0.01\\
6.59	0.01\\
6.6	0.01\\
6.61	0.01\\
6.62	0.01\\
6.63	0.01\\
6.64	0.01\\
6.65	0.01\\
6.66	0.01\\
6.67	0.01\\
6.68	0.01\\
6.69	0.01\\
6.7	0.01\\
6.71	0.01\\
6.72	0.01\\
6.73	0.01\\
6.74	0.01\\
6.75	0.01\\
6.76	0.01\\
6.77	0.01\\
6.78	0.01\\
6.79	0.01\\
6.8	0.01\\
6.81	0.01\\
6.82	0.01\\
6.83	0.01\\
6.84	0.01\\
6.85	0.01\\
6.86	0.01\\
6.87	0.01\\
6.88	0.01\\
6.89	0.01\\
6.9	0.01\\
6.91	0.01\\
6.92	0.01\\
6.93	0.01\\
6.94	0.01\\
6.95	0.01\\
6.96	0.01\\
6.97	0.01\\
6.98	0.01\\
6.99	0.01\\
7	0.01\\
7.01	0.01\\
7.02	0.01\\
7.03	0.01\\
7.04	0.01\\
7.05	0.01\\
7.06	0.01\\
7.07	0.01\\
7.08	0.01\\
7.09	0.01\\
7.1	0.01\\
7.11	0.01\\
7.12	0.01\\
7.13	0.01\\
7.14	0.01\\
7.15	0.01\\
7.16	0.01\\
7.17	0.01\\
7.18	0.01\\
7.19	0.01\\
7.2	0.01\\
7.21	0.01\\
7.22	0.01\\
7.23	0.01\\
7.24	0.01\\
7.25	0.01\\
7.26	0.01\\
7.27	0.01\\
7.28	0.01\\
7.29	0.01\\
7.3	0.01\\
7.31	0.01\\
7.32	0.01\\
7.33	0.01\\
7.34	0.01\\
7.35	0.01\\
7.36	0.01\\
7.37	0.01\\
7.38	0.01\\
7.39	0.01\\
7.4	0.01\\
7.41	0.01\\
7.42	0.01\\
7.43	0.01\\
7.44	0.01\\
7.45	0.01\\
7.46	0.01\\
7.47	0.01\\
7.48	0.01\\
7.49	0.01\\
7.5	0.01\\
7.51	0.01\\
7.52	0.01\\
7.53	0.01\\
7.54	0.01\\
7.55	0.01\\
7.56	0.01\\
7.57	0.01\\
7.58	0.01\\
7.59	0.01\\
7.6	0.01\\
7.61	0.01\\
7.62	0.01\\
7.63	0.01\\
7.64	0.01\\
7.65	0.01\\
7.66	0.01\\
7.67	0.01\\
7.68	0.01\\
7.69	0.01\\
7.7	0.01\\
7.71	0.01\\
7.72	0.01\\
7.73	0.01\\
7.74	0.01\\
7.75	0.01\\
7.76	0.01\\
7.77	0.01\\
7.78	0.01\\
7.79	0.01\\
7.8	0.01\\
7.81	0.01\\
7.82	0.01\\
7.83	0.01\\
7.84	0.01\\
7.85	0.01\\
7.86	0.01\\
7.87	0.01\\
7.88	0.01\\
7.89	0.01\\
7.9	0.01\\
7.91	0.01\\
7.92	0.01\\
7.93	0.01\\
7.94	0.01\\
7.95	0.01\\
7.96	0.01\\
7.97	0.01\\
7.98	0.01\\
7.99	0.01\\
8	0.01\\
8.01	0.01\\
8.02	0.01\\
8.03	0.01\\
8.04	0.01\\
8.05	0.01\\
8.06	0.01\\
8.07	0.01\\
8.08	0.01\\
8.09	0.01\\
8.1	0.01\\
8.11	0.01\\
8.12	0.01\\
8.13	0.01\\
8.14	0.01\\
8.15	0.01\\
8.16	0.01\\
8.17	0.01\\
8.18	0.01\\
8.19	0.01\\
8.2	0.01\\
8.21	0.01\\
8.22	0.01\\
8.23	0.01\\
8.24	0.01\\
8.25	0.01\\
8.26	0.01\\
8.27	0.01\\
8.28	0.01\\
8.29	0.01\\
8.3	0.01\\
8.31	0.01\\
8.32	0.01\\
8.33	0.01\\
8.34	0.01\\
8.35	0.01\\
8.36	0.01\\
8.37	0.01\\
8.38	0.01\\
8.39	0.01\\
8.4	0.01\\
8.41	0.01\\
8.42	0.01\\
8.43	0.01\\
8.44	0.01\\
8.45	0.01\\
8.46	0.01\\
8.47	0.01\\
8.48	0.01\\
8.49	0.01\\
8.5	0.01\\
8.51	0.01\\
8.52	0.01\\
8.53	0.01\\
8.54	0.01\\
8.55	0.01\\
8.56	0.01\\
8.57	0.01\\
8.58	0.01\\
8.59	0.01\\
8.6	0.01\\
8.61	0.01\\
8.62	0.01\\
8.63	0.01\\
8.64	0.01\\
8.65	0.01\\
8.66	0.01\\
8.67	0.01\\
8.68	0.01\\
8.69	0.01\\
8.7	0.01\\
8.71	0.01\\
8.72	0.01\\
8.73	0.01\\
8.74	0.01\\
8.75	0.01\\
8.76	0.01\\
8.77	0.01\\
8.78	0.01\\
8.79	0.01\\
8.8	0.01\\
8.81	0.01\\
8.82	0.01\\
8.83	0.01\\
8.84	0.01\\
8.85	0.01\\
8.86	0.01\\
8.87	0.01\\
8.88	0.01\\
8.89	0.01\\
8.9	0.01\\
8.91	0.01\\
8.92	0.01\\
8.93	0.01\\
8.94	0.01\\
8.95	0.01\\
8.96	0.01\\
8.97	0.01\\
8.98	0.01\\
8.99	0.01\\
9	0.01\\
9.01	0.01\\
9.02	0.01\\
9.03	0.01\\
9.04	0.01\\
9.05	0.01\\
9.06	0.01\\
9.07	0.01\\
9.08	0.01\\
9.09	0.01\\
9.1	0.01\\
9.11	0.01\\
9.12	0.01\\
9.13	0.01\\
9.14	0.01\\
9.15	0.01\\
9.16	0.01\\
9.17	0.01\\
9.18	0.01\\
9.19	0.01\\
9.2	0.01\\
9.21	0.01\\
9.22	0.01\\
9.23	0.01\\
9.24	0.01\\
9.25	0.01\\
9.26	0.01\\
9.27	0.01\\
9.28	0.01\\
9.29	0.01\\
9.3	0.01\\
9.31	0.01\\
9.32	0.01\\
9.33	0.01\\
9.34	0.01\\
9.35	0.01\\
9.36	0.01\\
9.37	0.01\\
9.38	0.01\\
9.39	0.01\\
9.4	0.01\\
9.41	0.01\\
9.42	0.01\\
9.43	0.01\\
9.44	0.01\\
9.45	0.01\\
9.46	0.01\\
9.47	0.01\\
9.48	0.01\\
9.49	0.01\\
9.5	0.01\\
9.51	0.01\\
9.52	0.01\\
9.53	0.01\\
9.54	0.01\\
9.55	0.01\\
9.56	0.01\\
9.57	0.01\\
9.58	0.01\\
9.59	0.01\\
9.6	0.01\\
9.61	0.01\\
9.62	0.01\\
9.63	0.01\\
9.64	0.01\\
9.65	0.01\\
9.66	0.01\\
9.67	0.01\\
9.68	0.01\\
9.69	0.01\\
9.7	0.01\\
9.71	0.01\\
9.72	0.01\\
9.73	0.01\\
9.74	0.01\\
9.75	0.01\\
9.76	0.01\\
9.77	0.01\\
9.78	0.01\\
9.79	0.01\\
9.8	0.01\\
9.81	0.01\\
9.82	0.01\\
9.83	0.01\\
9.84	0.01\\
9.85	0.01\\
9.86	0.01\\
9.87	0.01\\
9.88	0.01\\
9.89	0.01\\
9.9	0.01\\
9.91	0.01\\
9.92	0.01\\
9.93	0.01\\
9.94	0.01\\
9.95	0.01\\
9.96	0.01\\
9.97	0.01\\
9.98	0.01\\
9.99	0.01\\
10	0.01\\
10.01	0.01\\
10.02	0.01\\
10.03	0.01\\
10.04	0.01\\
10.05	0.01\\
10.06	0.01\\
10.07	0.01\\
10.08	0.01\\
10.09	0.01\\
10.1	0.01\\
10.11	0.01\\
10.12	0.01\\
10.13	0.01\\
10.14	0.01\\
10.15	0.01\\
10.16	0.01\\
10.17	0.01\\
10.18	0.01\\
10.19	0.01\\
10.2	0.01\\
10.21	0.01\\
10.22	0.01\\
10.23	0.01\\
10.24	0.01\\
10.25	0.01\\
10.26	0.01\\
10.27	0.01\\
10.28	0.01\\
10.29	0.01\\
10.3	0.01\\
10.31	0.01\\
10.32	0.01\\
10.33	0.01\\
10.34	0.01\\
10.35	0.01\\
10.36	0.01\\
10.37	0.01\\
10.38	0.01\\
10.39	0.01\\
10.4	0.01\\
10.41	0.01\\
10.42	0.01\\
10.43	0.01\\
10.44	0.01\\
10.45	0.01\\
10.46	0.01\\
10.47	0.01\\
10.48	0.01\\
10.49	0.01\\
10.5	0.01\\
10.51	0.01\\
10.52	0.01\\
10.53	0.01\\
10.54	0.01\\
10.55	0.01\\
10.56	0.01\\
10.57	0.01\\
10.58	0.01\\
10.59	0.01\\
10.6	0.01\\
10.61	0.01\\
10.62	0.01\\
10.63	0.01\\
10.64	0.01\\
10.65	0.01\\
10.66	0.01\\
10.67	0.01\\
10.68	0.01\\
10.69	0.01\\
10.7	0.01\\
10.71	0.01\\
10.72	0.01\\
10.73	0.01\\
10.74	0.01\\
10.75	0.01\\
10.76	0.01\\
10.77	0.01\\
10.78	0.01\\
10.79	0.01\\
10.8	0.01\\
10.81	0.01\\
10.82	0.01\\
10.83	0.01\\
10.84	0.01\\
10.85	0.01\\
10.86	0.01\\
10.87	0.01\\
10.88	0.01\\
10.89	0.01\\
10.9	0.01\\
10.91	0.01\\
10.92	0.01\\
10.93	0.01\\
10.94	0.01\\
10.95	0.01\\
10.96	0.01\\
10.97	0.01\\
10.98	0.01\\
10.99	0.01\\
11	0.01\\
11.01	0.01\\
11.02	0.01\\
11.03	0.01\\
11.04	0.01\\
11.05	0.01\\
11.06	0.01\\
11.07	0.01\\
11.08	0.01\\
11.09	0.01\\
11.1	0.01\\
11.11	0.01\\
11.12	0.01\\
11.13	0.01\\
11.14	0.01\\
11.15	0.01\\
11.16	0.01\\
11.17	0.01\\
11.18	0.01\\
11.19	0.01\\
11.2	0.01\\
11.21	0.01\\
11.22	0.01\\
11.23	0.01\\
11.24	0.01\\
11.25	0.01\\
11.26	0.01\\
11.27	0.01\\
11.28	0.01\\
11.29	0.01\\
11.3	0.01\\
11.31	0.01\\
11.32	0.01\\
11.33	0.01\\
11.34	0.01\\
11.35	0.01\\
11.36	0.01\\
11.37	0.01\\
11.38	0.01\\
11.39	0.01\\
11.4	0.01\\
11.41	0.01\\
11.42	0.01\\
11.43	0.01\\
11.44	0.01\\
11.45	0.01\\
11.46	0.01\\
11.47	0.01\\
11.48	0.01\\
11.49	0.01\\
11.5	0.01\\
11.51	0.01\\
11.52	0.01\\
11.53	0.01\\
11.54	0.01\\
11.55	0.01\\
11.56	0.01\\
11.57	0.01\\
11.58	0.01\\
11.59	0.01\\
11.6	0.01\\
11.61	0.01\\
11.62	0.01\\
11.63	0.01\\
11.64	0.01\\
11.65	0.01\\
11.66	0.01\\
11.67	0.01\\
11.68	0.01\\
11.69	0.01\\
11.7	0.01\\
11.71	0.01\\
11.72	0.01\\
11.73	0.01\\
11.74	0.01\\
11.75	0.01\\
11.76	0.01\\
11.77	0.01\\
11.78	0.01\\
11.79	0.01\\
11.8	0.01\\
11.81	0.01\\
11.82	0.01\\
11.83	0.01\\
11.84	0.01\\
11.85	0.01\\
11.86	0.01\\
11.87	0.01\\
11.88	0.01\\
11.89	0.01\\
11.9	0.01\\
11.91	0.01\\
11.92	0.01\\
11.93	0.01\\
11.94	0.01\\
11.95	0.01\\
11.96	0.01\\
11.97	0.01\\
11.98	0.01\\
11.99	0.01\\
12	0.01\\
12.01	0.01\\
12.02	0.01\\
12.03	0.01\\
12.04	0.01\\
12.05	0.01\\
12.06	0.01\\
12.07	0.01\\
12.08	0.01\\
12.09	0.01\\
12.1	0.01\\
12.11	0.01\\
12.12	0.01\\
12.13	0.01\\
12.14	0.01\\
12.15	0.01\\
12.16	0.01\\
12.17	0.01\\
12.18	0.01\\
12.19	0.01\\
12.2	0.01\\
12.21	0.01\\
12.22	0.01\\
12.23	0.01\\
12.24	0.01\\
12.25	0.01\\
12.26	0.01\\
12.27	0.01\\
12.28	0.01\\
12.29	0.01\\
12.3	0.01\\
12.31	0.01\\
12.32	0.01\\
12.33	0.01\\
12.34	0.01\\
12.35	0.01\\
12.36	0.01\\
12.37	0.01\\
12.38	0.01\\
12.39	0.01\\
12.4	0.01\\
12.41	0.01\\
12.42	0.01\\
12.43	0.01\\
12.44	0.01\\
12.45	0.01\\
12.46	0.01\\
12.47	0.01\\
12.48	0.01\\
12.49	0.01\\
12.5	0.01\\
12.51	0.01\\
12.52	0.01\\
12.53	0.01\\
12.54	0.01\\
12.55	0.01\\
12.56	0.01\\
12.57	0.01\\
12.58	0.01\\
12.59	0.01\\
12.6	0.01\\
12.61	0.01\\
12.62	0.01\\
12.63	0.01\\
12.64	0.01\\
12.65	0.01\\
12.66	0.01\\
12.67	0.01\\
12.68	0.01\\
12.69	0.01\\
12.7	0.01\\
12.71	0.01\\
12.72	0.01\\
12.73	0.01\\
12.74	0.01\\
12.75	0.01\\
12.76	0.01\\
12.77	0.01\\
12.78	0.01\\
12.79	0.01\\
12.8	0.01\\
12.81	0.01\\
12.82	0.01\\
12.83	0.01\\
12.84	0.01\\
12.85	0.01\\
12.86	0.01\\
12.87	0.01\\
12.88	0.01\\
12.89	0.01\\
12.9	0.01\\
12.91	0.01\\
12.92	0.01\\
12.93	0.01\\
12.94	0.01\\
12.95	0.01\\
12.96	0.01\\
12.97	0.01\\
12.98	0.01\\
12.99	0.01\\
13	0.01\\
13.01	0.01\\
13.02	0.01\\
13.03	0.01\\
13.04	0.01\\
13.05	0.01\\
13.06	0.01\\
13.07	0.01\\
13.08	0.01\\
13.09	0.01\\
13.1	0.01\\
13.11	0.01\\
13.12	0.01\\
13.13	0.01\\
13.14	0.01\\
13.15	0.01\\
13.16	0.01\\
13.17	0.01\\
13.18	0.01\\
13.19	0.01\\
13.2	0.01\\
13.21	0.01\\
13.22	0.01\\
13.23	0.01\\
13.24	0.01\\
13.25	0.01\\
13.26	0.01\\
13.27	0.01\\
13.28	0.01\\
13.29	0.01\\
13.3	0.01\\
13.31	0.01\\
13.32	0.01\\
13.33	0.01\\
13.34	0.01\\
13.35	0.01\\
13.36	0.01\\
13.37	0.01\\
13.38	0.01\\
13.39	0.01\\
13.4	0.01\\
13.41	0.01\\
13.42	0.01\\
13.43	0.01\\
13.44	0.01\\
13.45	0.01\\
13.46	0.01\\
13.47	0.01\\
13.48	0.01\\
13.49	0.01\\
13.5	0.01\\
13.51	0.01\\
13.52	0.01\\
13.53	0.01\\
13.54	0.01\\
13.55	0.01\\
13.56	0.01\\
13.57	0.01\\
13.58	0.01\\
13.59	0.01\\
13.6	0.01\\
13.61	0.01\\
13.62	0.01\\
13.63	0.01\\
13.64	0.01\\
13.65	0.01\\
13.66	0.01\\
13.67	0.01\\
13.68	0.01\\
13.69	0.01\\
13.7	0.01\\
13.71	0.01\\
13.72	0.01\\
13.73	0.01\\
13.74	0.01\\
13.75	0.01\\
13.76	0.01\\
13.77	0.01\\
13.78	0.01\\
13.79	0.01\\
13.8	0.01\\
13.81	0.01\\
13.82	0.01\\
13.83	0.01\\
13.84	0.01\\
13.85	0.01\\
13.86	0.01\\
13.87	0.01\\
13.88	0.01\\
13.89	0.01\\
13.9	0.01\\
13.91	0.01\\
13.92	0.01\\
13.93	0.01\\
13.94	0.01\\
13.95	0.01\\
13.96	0.01\\
13.97	0.01\\
13.98	0.01\\
13.99	0.01\\
14	0.01\\
14.01	0.01\\
14.02	0.01\\
14.03	0.01\\
14.04	0.01\\
14.05	0.01\\
14.06	0.01\\
14.07	0.01\\
14.08	0.01\\
14.09	0.01\\
14.1	0.01\\
14.11	0.01\\
14.12	0.01\\
14.13	0.01\\
14.14	0.01\\
14.15	0.01\\
14.16	0.01\\
14.17	0.01\\
14.18	0.01\\
14.19	0.01\\
14.2	0.01\\
14.21	0.01\\
14.22	0.01\\
14.23	0.01\\
14.24	0.01\\
14.25	0.01\\
14.26	0.01\\
14.27	0.01\\
14.28	0.01\\
14.29	0.01\\
14.3	0.01\\
14.31	0.01\\
14.32	0.01\\
14.33	0.01\\
14.34	0.01\\
14.35	0.01\\
14.36	0.01\\
14.37	0.01\\
14.38	0.01\\
14.39	0.01\\
14.4	0.01\\
14.41	0.01\\
14.42	0.01\\
14.43	0.01\\
14.44	0.01\\
14.45	0.01\\
14.46	0.01\\
14.47	0.01\\
14.48	0.01\\
14.49	0.01\\
14.5	0.01\\
14.51	0.01\\
14.52	0.01\\
14.53	0.01\\
14.54	0.01\\
14.55	0.01\\
14.56	0.01\\
14.57	0.01\\
14.58	0.01\\
14.59	0.01\\
14.6	0.01\\
14.61	0.01\\
14.62	0.01\\
14.63	0.01\\
14.64	0.01\\
14.65	0.01\\
14.66	0.01\\
14.67	0.01\\
14.68	0.01\\
14.69	0.01\\
14.7	0.01\\
14.71	0.01\\
14.72	0.01\\
14.73	0.01\\
14.74	0.01\\
14.75	0.01\\
14.76	0.01\\
14.77	0.01\\
14.78	0.01\\
14.79	0.01\\
14.8	0.01\\
14.81	0.01\\
14.82	0.01\\
14.83	0.01\\
14.84	0.01\\
14.85	0.01\\
14.86	0.01\\
14.87	0.01\\
14.88	0.01\\
14.89	0.01\\
14.9	0.01\\
14.91	0.01\\
14.92	0.01\\
14.93	0.01\\
14.94	0.01\\
14.95	0.01\\
14.96	0.01\\
14.97	0.01\\
14.98	0.01\\
14.99	0.01\\
15	0.01\\
15.01	0.01\\
15.02	0.01\\
15.03	0.01\\
15.04	0.01\\
15.05	0.01\\
15.06	0.01\\
15.07	0.01\\
15.08	0.01\\
15.09	0.01\\
15.1	0.01\\
15.11	0.01\\
15.12	0.01\\
15.13	0.01\\
15.14	0.01\\
15.15	0.01\\
15.16	0.01\\
15.17	0.01\\
15.18	0.01\\
15.19	0.01\\
15.2	0.01\\
15.21	0.01\\
15.22	0.01\\
15.23	0.01\\
15.24	0.01\\
15.25	0.01\\
15.26	0.01\\
15.27	0.01\\
15.28	0.01\\
15.29	0.01\\
15.3	0.01\\
15.31	0.01\\
15.32	0.01\\
15.33	0.01\\
15.34	0.01\\
15.35	0.01\\
15.36	0.01\\
15.37	0.01\\
15.38	0.01\\
15.39	0.01\\
15.4	0.01\\
15.41	0.01\\
15.42	0.01\\
15.43	0.01\\
15.44	0.01\\
15.45	0.01\\
15.46	0.01\\
15.47	0.01\\
15.48	0.01\\
15.49	0.01\\
15.5	0.01\\
15.51	0.01\\
15.52	0.01\\
15.53	0.01\\
15.54	0.01\\
15.55	0.01\\
15.56	0.01\\
15.57	0.01\\
15.58	0.01\\
15.59	0.01\\
15.6	0.01\\
15.61	0.01\\
15.62	0.01\\
15.63	0.01\\
15.64	0.01\\
15.65	0.01\\
15.66	0.01\\
15.67	0.01\\
15.68	0.01\\
15.69	0.01\\
15.7	0.01\\
15.71	0.01\\
15.72	0.01\\
15.73	0.01\\
15.74	0.01\\
15.75	0.01\\
15.76	0.01\\
15.77	0.01\\
15.78	0.01\\
15.79	0.01\\
15.8	0.01\\
15.81	0.01\\
15.82	0.01\\
15.83	0.01\\
15.84	0.01\\
15.85	0.01\\
15.86	0.01\\
15.87	0.01\\
15.88	0.01\\
15.89	0.01\\
15.9	0.01\\
15.91	0.01\\
15.92	0.01\\
15.93	0.01\\
15.94	0.01\\
15.95	0.01\\
15.96	0.01\\
15.97	0.01\\
15.98	0.01\\
15.99	0.01\\
16	0.01\\
16.01	0.01\\
16.02	0.01\\
16.03	0.01\\
16.04	0.01\\
16.05	0.01\\
16.06	0.01\\
16.07	0.01\\
16.08	0.01\\
16.09	0.01\\
16.1	0.01\\
16.11	0.01\\
16.12	0.01\\
16.13	0.01\\
16.14	0.01\\
16.15	0.01\\
16.16	0.01\\
16.17	0.01\\
16.18	0.01\\
16.19	0.01\\
16.2	0.01\\
16.21	0.01\\
16.22	0.01\\
16.23	0.01\\
16.24	0.01\\
16.25	0.01\\
16.26	0.01\\
16.27	0.01\\
16.28	0.01\\
16.29	0.01\\
16.3	0.01\\
16.31	0.01\\
16.32	0.01\\
16.33	0.01\\
16.34	0.01\\
16.35	0.01\\
16.36	0.01\\
16.37	0.01\\
16.38	0.01\\
16.39	0.01\\
16.4	0.01\\
16.41	0.01\\
16.42	0.01\\
16.43	0.01\\
16.44	0.01\\
16.45	0.01\\
16.46	0.01\\
16.47	0.01\\
16.48	0.01\\
16.49	0.01\\
16.5	0.01\\
16.51	0.01\\
16.52	0.01\\
16.53	0.01\\
16.54	0.01\\
16.55	0.01\\
16.56	0.01\\
16.57	0.01\\
16.58	0.01\\
16.59	0.01\\
16.6	0.01\\
16.61	0.01\\
16.62	0.01\\
16.63	0.01\\
16.64	0.01\\
16.65	0.01\\
16.66	0.01\\
16.67	0.01\\
16.68	0.01\\
16.69	0.01\\
16.7	0.01\\
16.71	0.01\\
16.72	0.01\\
16.73	0.01\\
16.74	0.01\\
16.75	0.01\\
16.76	0.01\\
16.77	0.01\\
16.78	0.01\\
16.79	0.01\\
16.8	0.01\\
16.81	0.01\\
16.82	0.01\\
16.83	0.01\\
16.84	0.01\\
16.85	0.01\\
16.86	0.01\\
16.87	0.01\\
16.88	0.01\\
16.89	0.01\\
16.9	0.01\\
16.91	0.01\\
16.92	0.01\\
16.93	0.01\\
16.94	0.01\\
16.95	0.01\\
16.96	0.01\\
16.97	0.01\\
16.98	0.01\\
16.99	0.01\\
17	0.01\\
17.01	0.01\\
17.02	0.01\\
17.03	0.01\\
17.04	0.01\\
17.05	0.01\\
17.06	0.01\\
17.07	0.01\\
17.08	0.01\\
17.09	0.01\\
17.1	0.01\\
17.11	0.01\\
17.12	0.01\\
17.13	0.01\\
17.14	0.01\\
17.15	0.01\\
17.16	0.01\\
17.17	0.01\\
17.18	0.01\\
17.19	0.01\\
17.2	0.01\\
17.21	0.01\\
17.22	0.01\\
17.23	0.01\\
17.24	0.01\\
17.25	0.01\\
17.26	0.01\\
17.27	0.01\\
17.28	0.01\\
17.29	0.01\\
17.3	0.01\\
17.31	0.01\\
17.32	0.01\\
17.33	0.01\\
17.34	0.01\\
17.35	0.01\\
17.36	0.01\\
17.37	0.01\\
17.38	0.01\\
17.39	0.01\\
17.4	0.01\\
17.41	0.01\\
17.42	0.01\\
17.43	0.01\\
17.44	0.01\\
17.45	0.01\\
17.46	0.01\\
17.47	0.01\\
17.48	0.01\\
17.49	0.01\\
17.5	0.01\\
17.51	0.01\\
17.52	0.01\\
17.53	0.01\\
17.54	0.01\\
17.55	0.01\\
17.56	0.01\\
17.57	0.01\\
17.58	0.01\\
17.59	0.01\\
17.6	0.01\\
17.61	0.01\\
17.62	0.01\\
17.63	0.01\\
17.64	0.01\\
17.65	0.01\\
17.66	0.01\\
17.67	0.01\\
17.68	0.01\\
17.69	0.01\\
17.7	0.01\\
17.71	0.01\\
17.72	0.01\\
17.73	0.01\\
17.74	0.01\\
17.75	0.01\\
17.76	0.01\\
17.77	0.01\\
17.78	0.01\\
17.79	0.01\\
17.8	0.01\\
17.81	0.01\\
17.82	0.01\\
17.83	0.01\\
17.84	0.01\\
17.85	0.01\\
17.86	0.01\\
17.87	0.01\\
17.88	0.01\\
17.89	0.01\\
17.9	0.01\\
17.91	0.01\\
17.92	0.01\\
17.93	0.01\\
17.94	0.01\\
17.95	0.01\\
17.96	0.01\\
17.97	0.01\\
17.98	0.01\\
17.99	0.01\\
18	0.01\\
18.01	0.01\\
18.02	0.01\\
18.03	0.01\\
18.04	0.01\\
18.05	0.01\\
18.06	0.01\\
18.07	0.01\\
18.08	0.01\\
18.09	0.01\\
18.1	0.01\\
18.11	0.01\\
18.12	0.01\\
18.13	0.01\\
18.14	0.01\\
18.15	0.01\\
18.16	0.01\\
18.17	0.01\\
18.18	0.01\\
18.19	0.01\\
18.2	0.01\\
18.21	0.01\\
18.22	0.01\\
18.23	0.01\\
18.24	0.01\\
18.25	0.01\\
18.26	0.01\\
18.27	0.01\\
18.28	0.01\\
18.29	0.01\\
18.3	0.01\\
18.31	0.01\\
18.32	0.01\\
18.33	0.01\\
18.34	0.01\\
18.35	0.01\\
18.36	0.01\\
18.37	0.01\\
18.38	0.01\\
18.39	0.01\\
18.4	0.01\\
18.41	0.01\\
18.42	0.01\\
18.43	0.01\\
18.44	0.01\\
18.45	0.01\\
18.46	0.01\\
18.47	0.01\\
18.48	0.01\\
18.49	0.01\\
18.5	0.01\\
18.51	0.01\\
18.52	0.01\\
18.53	0.01\\
18.54	0.01\\
18.55	0.01\\
18.56	0.01\\
18.57	0.01\\
18.58	0.01\\
18.59	0.01\\
18.6	0.01\\
18.61	0.01\\
18.62	0.01\\
18.63	0.01\\
18.64	0.01\\
18.65	0.01\\
18.66	0.01\\
18.67	0.01\\
18.68	0.01\\
18.69	0.01\\
18.7	0.01\\
18.71	0.01\\
18.72	0.01\\
18.73	0.01\\
18.74	0.01\\
18.75	0.01\\
18.76	0.01\\
18.77	0.01\\
18.78	0.01\\
18.79	0.01\\
18.8	0.01\\
18.81	0.01\\
18.82	0.01\\
18.83	0.01\\
18.84	0.01\\
18.85	0.01\\
18.86	0.01\\
18.87	0.01\\
18.88	0.01\\
18.89	0.01\\
18.9	0.01\\
18.91	0.01\\
18.92	0.01\\
18.93	0.01\\
18.94	0.01\\
18.95	0.01\\
18.96	0.01\\
18.97	0.01\\
18.98	0.01\\
18.99	0.01\\
19	0.01\\
19.01	0.01\\
19.02	0.01\\
19.03	0.01\\
19.04	0.01\\
19.05	0.01\\
19.06	0.01\\
19.07	0.01\\
19.08	0.01\\
19.09	0.01\\
19.1	0.01\\
19.11	0.01\\
19.12	0.01\\
19.13	0.01\\
19.14	0.01\\
19.15	0.01\\
19.16	0.01\\
19.17	0.01\\
19.18	0.01\\
19.19	0.01\\
19.2	0.01\\
19.21	0.01\\
19.22	0.01\\
19.23	0.01\\
19.24	0.01\\
19.25	0.01\\
19.26	0.01\\
19.27	0.01\\
19.28	0.01\\
19.29	0.01\\
19.3	0.01\\
19.31	0.01\\
19.32	0.01\\
19.33	0.01\\
19.34	0.01\\
19.35	0.01\\
19.36	0.01\\
19.37	0.01\\
19.38	0.01\\
19.39	0.01\\
19.4	0.01\\
19.41	0.01\\
19.42	0.01\\
19.43	0.01\\
19.44	0.01\\
19.45	0.01\\
19.46	0.01\\
19.47	0.01\\
19.48	0.01\\
19.49	0.01\\
19.5	0.01\\
19.51	0.01\\
19.52	0.01\\
19.53	0.01\\
19.54	0.01\\
19.55	0.01\\
19.56	0.01\\
19.57	0.01\\
19.58	0.01\\
19.59	0.01\\
19.6	0.01\\
19.61	0.01\\
19.62	0.01\\
19.63	0.01\\
19.64	0.01\\
19.65	0.01\\
19.66	0.01\\
19.67	0.01\\
19.68	0.01\\
19.69	0.01\\
19.7	0.01\\
19.71	0.01\\
19.72	0.01\\
19.73	0.01\\
19.74	0.01\\
19.75	0.01\\
19.76	0.01\\
19.77	0.01\\
19.78	0.01\\
19.79	0.01\\
19.8	0.01\\
19.81	0.01\\
19.82	0.01\\
19.83	0.01\\
19.84	0.01\\
19.85	0.01\\
19.86	0.01\\
19.87	0.01\\
19.88	0.01\\
19.89	0.01\\
19.9	0.01\\
19.91	0.01\\
19.92	0.01\\
19.93	0.01\\
19.94	0.01\\
19.95	0.01\\
19.96	0.01\\
19.97	0.01\\
19.98	0.01\\
19.99	0.01\\
20	0.01\\
20.01	0.01\\
20.02	0.01\\
20.03	0.01\\
20.04	0.01\\
20.05	0.01\\
20.06	0.01\\
20.07	0.01\\
20.08	0.01\\
20.09	0.01\\
20.1	0.01\\
20.11	0.01\\
20.12	0.01\\
20.13	0.01\\
20.14	0.01\\
20.15	0.01\\
20.16	0.01\\
20.17	0.01\\
20.18	0.01\\
20.19	0.01\\
20.2	0.01\\
20.21	0.01\\
20.22	0.01\\
20.23	0.01\\
20.24	0.01\\
20.25	0.01\\
20.26	0.01\\
20.27	0.01\\
20.28	0.01\\
20.29	0.01\\
20.3	0.01\\
20.31	0.01\\
20.32	0.01\\
20.33	0.01\\
20.34	0.01\\
20.35	0.01\\
20.36	0.01\\
20.37	0.01\\
20.38	0.01\\
20.39	0.01\\
20.4	0.01\\
20.41	0.01\\
20.42	0.01\\
20.43	0.01\\
20.44	0.01\\
20.45	0.01\\
20.46	0.01\\
20.47	0.01\\
20.48	0.01\\
20.49	0.01\\
20.5	0.01\\
20.51	0.01\\
20.52	0.01\\
20.53	0.01\\
20.54	0.01\\
20.55	0.01\\
20.56	0.01\\
20.57	0.01\\
20.58	0.01\\
20.59	0.01\\
20.6	0.01\\
20.61	0.01\\
20.62	0.01\\
20.63	0.01\\
20.64	0.01\\
20.65	0.01\\
20.66	0.01\\
20.67	0.01\\
20.68	0.01\\
20.69	0.01\\
20.7	0.01\\
20.71	0.01\\
20.72	0.01\\
20.73	0.01\\
20.74	0.01\\
20.75	0.01\\
20.76	0.01\\
20.77	0.01\\
20.78	0.01\\
20.79	0.01\\
20.8	0.01\\
20.81	0.01\\
20.82	0.01\\
20.83	0.01\\
20.84	0.01\\
20.85	0.01\\
20.86	0.01\\
20.87	0.01\\
20.88	0.01\\
20.89	0.01\\
20.9	0.01\\
20.91	0.01\\
20.92	0.01\\
20.93	0.01\\
20.94	0.01\\
20.95	0.01\\
20.96	0.01\\
20.97	0.01\\
20.98	0.01\\
20.99	0.01\\
21	0.01\\
21.01	0.01\\
21.02	0.01\\
21.03	0.01\\
21.04	0.01\\
21.05	0.01\\
21.06	0.01\\
21.07	0.01\\
21.08	0.01\\
21.09	0.01\\
21.1	0.01\\
21.11	0.01\\
21.12	0.01\\
21.13	0.01\\
21.14	0.01\\
21.15	0.01\\
21.16	0.01\\
21.17	0.01\\
21.18	0.01\\
21.19	0.01\\
21.2	0.01\\
21.21	0.01\\
21.22	0.01\\
21.23	0.01\\
21.24	0.01\\
21.25	0.01\\
21.26	0.01\\
21.27	0.01\\
21.28	0.01\\
21.29	0.01\\
21.3	0.01\\
21.31	0.01\\
21.32	0.01\\
21.33	0.01\\
21.34	0.01\\
21.35	0.01\\
21.36	0.01\\
21.37	0.01\\
21.38	0.01\\
21.39	0.01\\
21.4	0.01\\
21.41	0.01\\
21.42	0.01\\
21.43	0.01\\
21.44	0.01\\
21.45	0.01\\
21.46	0.01\\
21.47	0.01\\
21.48	0.01\\
21.49	0.01\\
21.5	0.01\\
21.51	0.01\\
21.52	0.01\\
21.53	0.01\\
21.54	0.01\\
21.55	0.01\\
21.56	0.01\\
21.57	0.01\\
21.58	0.01\\
21.59	0.01\\
21.6	0.01\\
21.61	0.01\\
21.62	0.01\\
21.63	0.01\\
21.64	0.01\\
21.65	0.01\\
21.66	0.01\\
21.67	0.01\\
21.68	0.01\\
21.69	0.01\\
21.7	0.01\\
21.71	0.01\\
21.72	0.01\\
21.73	0.01\\
21.74	0.01\\
21.75	0.01\\
21.76	0.01\\
21.77	0.01\\
21.78	0.01\\
21.79	0.01\\
21.8	0.01\\
21.81	0.01\\
21.82	0.01\\
21.83	0.01\\
21.84	0.01\\
21.85	0.01\\
21.86	0.01\\
21.87	0.01\\
21.88	0.01\\
21.89	0.01\\
21.9	0.01\\
21.91	0.01\\
21.92	0.01\\
21.93	0.01\\
21.94	0.01\\
21.95	0.01\\
21.96	0.01\\
21.97	0.01\\
21.98	0.01\\
21.99	0.01\\
22	0.01\\
22.01	0.01\\
22.02	0.01\\
22.03	0.01\\
22.04	0.01\\
22.05	0.01\\
22.06	0.01\\
22.07	0.01\\
22.08	0.01\\
22.09	0.01\\
22.1	0.01\\
22.11	0.01\\
22.12	0.01\\
22.13	0.01\\
22.14	0.01\\
22.15	0.01\\
22.16	0.01\\
22.17	0.01\\
22.18	0.01\\
22.19	0.01\\
22.2	0.01\\
22.21	0.01\\
22.22	0.01\\
22.23	0.01\\
22.24	0.01\\
22.25	0.01\\
22.26	0.01\\
22.27	0.01\\
22.28	0.01\\
22.29	0.01\\
22.3	0.01\\
22.31	0.01\\
22.32	0.01\\
22.33	0.01\\
22.34	0.01\\
22.35	0.01\\
22.36	0.01\\
22.37	0.01\\
22.38	0.01\\
22.39	0.01\\
22.4	0.01\\
22.41	0.01\\
22.42	0.01\\
22.43	0.01\\
22.44	0.01\\
22.45	0.01\\
22.46	0.01\\
22.47	0.01\\
22.48	0.01\\
22.49	0.01\\
22.5	0.01\\
22.51	0.01\\
22.52	0.01\\
22.53	0.01\\
22.54	0.01\\
22.55	0.01\\
22.56	0.01\\
22.57	0.01\\
22.58	0.01\\
22.59	0.01\\
22.6	0.01\\
22.61	0.01\\
22.62	0.01\\
22.63	0.01\\
22.64	0.01\\
22.65	0.01\\
22.66	0.01\\
22.67	0.01\\
22.68	0.01\\
22.69	0.01\\
22.7	0.01\\
22.71	0.01\\
22.72	0.01\\
22.73	0.01\\
22.74	0.01\\
22.75	0.01\\
22.76	0.01\\
22.77	0.01\\
22.78	0.01\\
22.79	0.01\\
22.8	0.01\\
22.81	0.01\\
22.82	0.01\\
22.83	0.01\\
22.84	0.01\\
22.85	0.01\\
22.86	0.01\\
22.87	0.01\\
22.88	0.01\\
22.89	0.01\\
22.9	0.01\\
22.91	0.01\\
22.92	0.01\\
22.93	0.01\\
22.94	0.01\\
22.95	0.01\\
22.96	0.01\\
22.97	0.01\\
22.98	0.01\\
22.99	0.01\\
23	0.01\\
23.01	0.01\\
23.02	0.01\\
23.03	0.01\\
23.04	0.01\\
23.05	0.01\\
23.06	0.01\\
23.07	0.01\\
23.08	0.01\\
23.09	0.01\\
23.1	0.01\\
23.11	0.01\\
23.12	0.01\\
23.13	0.01\\
23.14	0.01\\
23.15	0.01\\
23.16	0.01\\
23.17	0.01\\
23.18	0.01\\
23.19	0.01\\
23.2	0.01\\
23.21	0.01\\
23.22	0.01\\
23.23	0.01\\
23.24	0.01\\
23.25	0.01\\
23.26	0.01\\
23.27	0.01\\
23.28	0.01\\
23.29	0.01\\
23.3	0.01\\
23.31	0.01\\
23.32	0.01\\
23.33	0.01\\
23.34	0.01\\
23.35	0.01\\
23.36	0.01\\
23.37	0.01\\
23.38	0.01\\
23.39	0.01\\
23.4	0.01\\
23.41	0.01\\
23.42	0.01\\
23.43	0.01\\
23.44	0.01\\
23.45	0.01\\
23.46	0.01\\
23.47	0.01\\
23.48	0.01\\
23.49	0.01\\
23.5	0.01\\
23.51	0.01\\
23.52	0.01\\
23.53	0.01\\
23.54	0.01\\
23.55	0.01\\
23.56	0.01\\
23.57	0.01\\
23.58	0.01\\
23.59	0.01\\
23.6	0.01\\
23.61	0.01\\
23.62	0.01\\
23.63	0.01\\
23.64	0.01\\
23.65	0.01\\
23.66	0.01\\
23.67	0.01\\
23.68	0.01\\
23.69	0.01\\
23.7	0.01\\
23.71	0.01\\
23.72	0.01\\
23.73	0.01\\
23.74	0.01\\
23.75	0.01\\
23.76	0.01\\
23.77	0.01\\
23.78	0.01\\
23.79	0.01\\
23.8	0.01\\
23.81	0.01\\
23.82	0.01\\
23.83	0.01\\
23.84	0.01\\
23.85	0.01\\
23.86	0.01\\
23.87	0.01\\
23.88	0.01\\
23.89	0.01\\
23.9	0.01\\
23.91	0.01\\
23.92	0.01\\
23.93	0.01\\
23.94	0.01\\
23.95	0.01\\
23.96	0.01\\
23.97	0.01\\
23.98	0.01\\
23.99	0.01\\
24	0.01\\
24.01	0.01\\
24.02	0.01\\
24.03	0.01\\
24.04	0.01\\
24.05	0.01\\
24.06	0.01\\
24.07	0.01\\
24.08	0.01\\
24.09	0.01\\
24.1	0.01\\
24.11	0.01\\
24.12	0.01\\
24.13	0.01\\
24.14	0.01\\
24.15	0.01\\
24.16	0.01\\
24.17	0.01\\
24.18	0.01\\
24.19	0.01\\
24.2	0.01\\
24.21	0.01\\
24.22	0.01\\
24.23	0.01\\
24.24	0.01\\
24.25	0.01\\
24.26	0.01\\
24.27	0.01\\
24.28	0.01\\
24.29	0.01\\
24.3	0.01\\
24.31	0.01\\
24.32	0.01\\
24.33	0.01\\
24.34	0.01\\
24.35	0.01\\
24.36	0.01\\
24.37	0.01\\
24.38	0.01\\
24.39	0.01\\
24.4	0.01\\
24.41	0.01\\
24.42	0.01\\
24.43	0.01\\
24.44	0.01\\
24.45	0.01\\
24.46	0.01\\
24.47	0.01\\
24.48	0.01\\
24.49	0.01\\
24.5	0.01\\
24.51	0.01\\
24.52	0.01\\
24.53	0.01\\
24.54	0.01\\
24.55	0.01\\
24.56	0.01\\
24.57	0.01\\
24.58	0.01\\
24.59	0.01\\
24.6	0.01\\
24.61	0.01\\
24.62	0.01\\
24.63	0.01\\
24.64	0.01\\
24.65	0.01\\
24.66	0.01\\
24.67	0.01\\
24.68	0.01\\
24.69	0.01\\
24.7	0.01\\
24.71	0.01\\
24.72	0.01\\
24.73	0.01\\
24.74	0.01\\
24.75	0.01\\
24.76	0.01\\
24.77	0.01\\
24.78	0.01\\
24.79	0.01\\
24.8	0.01\\
24.81	0.01\\
24.82	0.01\\
24.83	0.01\\
24.84	0.01\\
24.85	0.01\\
24.86	0.01\\
24.87	0.01\\
24.88	0.01\\
24.89	0.01\\
24.9	0.01\\
24.91	0.01\\
24.92	0.01\\
24.93	0.01\\
24.94	0.01\\
24.95	0.01\\
24.96	0.01\\
24.97	0.01\\
24.98	0.01\\
24.99	0.01\\
25	0.01\\
25.01	0.01\\
25.02	0.01\\
25.03	0.01\\
25.04	0.01\\
25.05	0.01\\
25.06	0.01\\
25.07	0.01\\
25.08	0.01\\
25.09	0.01\\
25.1	0.01\\
25.11	0.01\\
25.12	0.01\\
25.13	0.01\\
25.14	0.01\\
25.15	0.01\\
25.16	0.01\\
25.17	0.01\\
25.18	0.01\\
25.19	0.01\\
25.2	0.01\\
25.21	0.01\\
25.22	0.01\\
25.23	0.01\\
25.24	0.01\\
25.25	0.01\\
25.26	0.01\\
25.27	0.01\\
25.28	0.01\\
25.29	0.01\\
25.3	0.01\\
25.31	0.01\\
25.32	0.01\\
25.33	0.01\\
25.34	0.01\\
25.35	0.01\\
25.36	0.01\\
25.37	0.01\\
25.38	0.01\\
25.39	0.01\\
25.4	0.01\\
25.41	0.01\\
25.42	0.01\\
25.43	0.01\\
25.44	0.01\\
25.45	0.01\\
25.46	0.01\\
25.47	0.01\\
25.48	0.01\\
25.49	0.01\\
25.5	0.01\\
25.51	0.01\\
25.52	0.01\\
25.53	0.01\\
25.54	0.01\\
25.55	0.01\\
25.56	0.01\\
25.57	0.01\\
25.58	0.01\\
25.59	0.01\\
25.6	0.01\\
25.61	0.01\\
25.62	0.01\\
25.63	0.01\\
25.64	0.01\\
25.65	0.01\\
25.66	0.01\\
25.67	0.01\\
25.68	0.01\\
25.69	0.01\\
25.7	0.01\\
25.71	0.01\\
25.72	0.01\\
25.73	0.01\\
25.74	0.01\\
25.75	0.01\\
25.76	0.01\\
25.77	0.01\\
25.78	0.01\\
25.79	0.01\\
25.8	0.01\\
25.81	0.01\\
25.82	0.01\\
25.83	0.01\\
25.84	0.01\\
25.85	0.01\\
25.86	0.01\\
25.87	0.01\\
25.88	0.01\\
25.89	0.01\\
25.9	0.01\\
25.91	0.01\\
25.92	0.01\\
25.93	0.01\\
25.94	0.01\\
25.95	0.01\\
25.96	0.01\\
25.97	0.01\\
25.98	0.01\\
25.99	0.01\\
26	0.01\\
26.01	0.01\\
26.02	0.01\\
26.03	0.01\\
26.04	0.01\\
26.05	0.01\\
26.06	0.01\\
26.07	0.01\\
26.08	0.01\\
26.09	0.01\\
26.1	0.01\\
26.11	0.01\\
26.12	0.01\\
26.13	0.01\\
26.14	0.01\\
26.15	0.01\\
26.16	0.01\\
26.17	0.01\\
26.18	0.01\\
26.19	0.01\\
26.2	0.01\\
26.21	0.01\\
26.22	0.01\\
26.23	0.01\\
26.24	0.01\\
26.25	0.01\\
26.26	0.01\\
26.27	0.01\\
26.28	0.01\\
26.29	0.01\\
26.3	0.01\\
26.31	0.01\\
26.32	0.01\\
26.33	0.01\\
26.34	0.01\\
26.35	0.01\\
26.36	0.01\\
26.37	0.01\\
26.38	0.01\\
26.39	0.01\\
26.4	0.01\\
26.41	0.01\\
26.42	0.01\\
26.43	0.01\\
26.44	0.01\\
26.45	0.01\\
26.46	0.01\\
26.47	0.01\\
26.48	0.01\\
26.49	0.01\\
26.5	0.01\\
26.51	0.01\\
26.52	0.01\\
26.53	0.01\\
26.54	0.01\\
26.55	0.01\\
26.56	0.01\\
26.57	0.01\\
26.58	0.01\\
26.59	0.01\\
26.6	0.01\\
26.61	0.01\\
26.62	0.01\\
26.63	0.01\\
26.64	0.01\\
26.65	0.01\\
26.66	0.01\\
26.67	0.01\\
26.68	0.01\\
26.69	0.01\\
26.7	0.01\\
26.71	0.01\\
26.72	0.01\\
26.73	0.01\\
26.74	0.01\\
26.75	0.01\\
26.76	0.01\\
26.77	0.01\\
26.78	0.01\\
26.79	0.01\\
26.8	0.01\\
26.81	0.01\\
26.82	0.01\\
26.83	0.01\\
26.84	0.01\\
26.85	0.01\\
26.86	0.01\\
26.87	0.01\\
26.88	0.01\\
26.89	0.01\\
26.9	0.01\\
26.91	0.01\\
26.92	0.01\\
26.93	0.01\\
26.94	0.01\\
26.95	0.01\\
26.96	0.01\\
26.97	0.01\\
26.98	0.01\\
26.99	0.01\\
27	0.01\\
27.01	0.01\\
27.02	0.01\\
27.03	0.01\\
27.04	0.01\\
27.05	0.01\\
27.06	0.01\\
27.07	0.01\\
27.08	0.01\\
27.09	0.01\\
27.1	0.01\\
27.11	0.01\\
27.12	0.01\\
27.13	0.01\\
27.14	0.01\\
27.15	0.01\\
27.16	0.01\\
27.17	0.01\\
27.18	0.01\\
27.19	0.01\\
27.2	0.01\\
27.21	0.01\\
27.22	0.01\\
27.23	0.01\\
27.24	0.01\\
27.25	0.01\\
27.26	0.01\\
27.27	0.01\\
27.28	0.01\\
27.29	0.01\\
27.3	0.01\\
27.31	0.01\\
27.32	0.01\\
27.33	0.01\\
27.34	0.01\\
27.35	0.01\\
27.36	0.01\\
27.37	0.01\\
27.38	0.01\\
27.39	0.01\\
27.4	0.01\\
27.41	0.01\\
27.42	0.01\\
27.43	0.01\\
27.44	0.01\\
27.45	0.01\\
27.46	0.01\\
27.47	0.01\\
27.48	0.01\\
27.49	0.01\\
27.5	0.01\\
27.51	0.01\\
27.52	0.01\\
27.53	0.01\\
27.54	0.01\\
27.55	0.01\\
27.56	0.01\\
27.57	0.01\\
27.58	0.01\\
27.59	0.01\\
27.6	0.01\\
27.61	0.01\\
27.62	0.01\\
27.63	0.01\\
27.64	0.01\\
27.65	0.01\\
27.66	0.01\\
27.67	0.01\\
27.68	0.01\\
27.69	0.01\\
27.7	0.01\\
27.71	0.01\\
27.72	0.01\\
27.73	0.01\\
27.74	0.01\\
27.75	0.01\\
27.76	0.01\\
27.77	0.01\\
27.78	0.01\\
27.79	0.01\\
27.8	0.01\\
27.81	0.01\\
27.82	0.01\\
27.83	0.01\\
27.84	0.01\\
27.85	0.01\\
27.86	0.01\\
27.87	0.01\\
27.88	0.01\\
27.89	0.01\\
27.9	0.01\\
27.91	0.01\\
27.92	0.01\\
27.93	0.01\\
27.94	0.01\\
27.95	0.01\\
27.96	0.01\\
27.97	0.01\\
27.98	0.01\\
27.99	0.01\\
28	0.01\\
28.01	0.01\\
28.02	0.01\\
28.03	0.01\\
28.04	0.01\\
28.05	0.01\\
28.06	0.01\\
28.07	0.01\\
28.08	0.01\\
28.09	0.01\\
28.1	0.01\\
28.11	0.01\\
28.12	0.01\\
28.13	0.01\\
28.14	0.01\\
28.15	0.01\\
28.16	0.01\\
28.17	0.01\\
28.18	0.01\\
28.19	0.01\\
28.2	0.01\\
28.21	0.01\\
28.22	0.01\\
28.23	0.01\\
28.24	0.01\\
28.25	0.01\\
28.26	0.01\\
28.27	0.01\\
28.28	0.01\\
28.29	0.01\\
28.3	0.01\\
28.31	0.01\\
28.32	0.01\\
28.33	0.01\\
28.34	0.01\\
28.35	0.01\\
28.36	0.01\\
28.37	0.01\\
28.38	0.01\\
28.39	0.01\\
28.4	0.01\\
28.41	0.01\\
28.42	0.01\\
28.43	0.01\\
28.44	0.01\\
28.45	0.01\\
28.46	0.01\\
28.47	0.01\\
28.48	0.01\\
28.49	0.01\\
28.5	0.01\\
28.51	0.01\\
28.52	0.01\\
28.53	0.01\\
28.54	0.01\\
28.55	0.01\\
28.56	0.01\\
28.57	0.01\\
28.58	0.01\\
28.59	0.01\\
28.6	0.01\\
28.61	0.01\\
28.62	0.01\\
28.63	0.01\\
28.64	0.01\\
28.65	0.01\\
28.66	0.01\\
28.67	0.01\\
28.68	0.01\\
28.69	0.01\\
28.7	0.01\\
28.71	0.01\\
28.72	0.01\\
28.73	0.01\\
28.74	0.01\\
28.75	0.01\\
28.76	0.01\\
28.77	0.01\\
28.78	0.01\\
28.79	0.01\\
28.8	0.01\\
28.81	0.01\\
28.82	0.01\\
28.83	0.01\\
28.84	0.01\\
28.85	0.01\\
28.86	0.01\\
28.87	0.01\\
28.88	0.01\\
28.89	0.01\\
28.9	0.01\\
28.91	0.01\\
28.92	0.01\\
28.93	0.01\\
28.94	0.01\\
28.95	0.01\\
28.96	0.01\\
28.97	0.01\\
28.98	0.01\\
28.99	0.01\\
29	0.01\\
29.01	0.01\\
29.02	0.01\\
29.03	0.01\\
29.04	0.01\\
29.05	0.01\\
29.06	0.01\\
29.07	0.01\\
29.08	0.01\\
29.09	0.01\\
29.1	0.01\\
29.11	0.01\\
29.12	0.01\\
29.13	0.01\\
29.14	0.01\\
29.15	0.01\\
29.16	0.01\\
29.17	0.01\\
29.18	0.01\\
29.19	0.01\\
29.2	0.01\\
29.21	0.01\\
29.22	0.01\\
29.23	0.01\\
29.24	0.01\\
29.25	0.01\\
29.26	0.01\\
29.27	0.01\\
29.28	0.01\\
29.29	0.01\\
29.3	0.01\\
29.31	0.01\\
29.32	0.01\\
29.33	0.01\\
29.34	0.01\\
29.35	0.01\\
29.36	0.01\\
29.37	0.01\\
29.38	0.01\\
29.39	0.01\\
29.4	0.01\\
29.41	0.01\\
29.42	0.01\\
29.43	0.01\\
29.44	0.01\\
29.45	0.01\\
29.46	0.01\\
29.47	0.01\\
29.48	0.01\\
29.49	0.01\\
29.5	0.01\\
29.51	0.01\\
29.52	0.01\\
29.53	0.01\\
29.54	0.01\\
29.55	0.01\\
29.56	0.01\\
29.57	0.01\\
29.58	0.01\\
29.59	0.01\\
29.6	0.01\\
29.61	0.01\\
29.62	0.01\\
29.63	0.01\\
29.64	0.01\\
29.65	0.01\\
29.66	0.01\\
29.67	0.01\\
29.68	0.01\\
29.69	0.01\\
29.7	0.01\\
29.71	0.01\\
29.72	0.01\\
29.73	0.01\\
29.74	0.01\\
29.75	0.01\\
29.76	0.01\\
29.77	0.01\\
29.78	0.01\\
29.79	0.01\\
29.8	0.01\\
29.81	0.01\\
29.82	0.01\\
29.83	0.01\\
29.84	0.01\\
29.85	0.01\\
29.86	0.01\\
29.87	0.01\\
29.88	0.01\\
29.89	0.01\\
29.9	0.01\\
29.91	0.01\\
29.92	0.01\\
29.93	0.01\\
29.94	0.01\\
29.95	0.01\\
29.96	0.01\\
29.97	0.01\\
29.98	0.01\\
29.99	0.01\\
30	0.01\\
30.01	0.01\\
30.02	0.01\\
30.03	0.01\\
30.04	0.01\\
30.05	0.01\\
30.06	0.01\\
30.07	0.01\\
30.08	0.01\\
30.09	0.01\\
30.1	0.01\\
30.11	0.01\\
30.12	0.01\\
30.13	0.01\\
30.14	0.01\\
30.15	0.01\\
30.16	0.01\\
30.17	0.01\\
30.18	0.01\\
30.19	0.01\\
30.2	0.01\\
30.21	0.01\\
30.22	0.01\\
30.23	0.01\\
30.24	0.01\\
30.25	0.01\\
30.26	0.01\\
30.27	0.01\\
30.28	0.01\\
30.29	0.01\\
30.3	0.01\\
30.31	0.01\\
30.32	0.01\\
30.33	0.01\\
30.34	0.01\\
30.35	0.01\\
30.36	0.01\\
30.37	0.01\\
30.38	0.01\\
30.39	0.01\\
30.4	0.01\\
30.41	0.01\\
30.42	0.01\\
30.43	0.01\\
30.44	0.01\\
30.45	0.01\\
30.46	0.01\\
30.47	0.01\\
30.48	0.01\\
30.49	0.01\\
30.5	0.01\\
30.51	0.01\\
30.52	0.01\\
30.53	0.01\\
30.54	0.01\\
30.55	0.01\\
30.56	0.01\\
30.57	0.01\\
30.58	0.01\\
30.59	0.01\\
30.6	0.01\\
30.61	0.01\\
30.62	0.01\\
30.63	0.01\\
30.64	0.01\\
30.65	0.01\\
30.66	0.01\\
30.67	0.01\\
30.68	0.01\\
30.69	0.01\\
30.7	0.01\\
30.71	0.01\\
30.72	0.01\\
30.73	0.01\\
30.74	0.01\\
30.75	0.01\\
30.76	0.01\\
30.77	0.01\\
30.78	0.01\\
30.79	0.01\\
30.8	0.01\\
30.81	0.01\\
30.82	0.01\\
30.83	0.01\\
30.84	0.01\\
30.85	0.01\\
30.86	0.01\\
30.87	0.01\\
30.88	0.01\\
30.89	0.01\\
30.9	0.01\\
30.91	0.01\\
30.92	0.01\\
30.93	0.01\\
30.94	0.01\\
30.95	0.01\\
30.96	0.01\\
30.97	0.01\\
30.98	0.01\\
30.99	0.01\\
31	0.01\\
31.01	0.01\\
31.02	0.01\\
31.03	0.01\\
31.04	0.01\\
31.05	0.01\\
31.06	0.01\\
31.07	0.01\\
31.08	0.01\\
31.09	0.01\\
31.1	0.01\\
31.11	0.01\\
31.12	0.01\\
31.13	0.01\\
31.14	0.01\\
31.15	0.01\\
31.16	0.01\\
31.17	0.01\\
31.18	0.01\\
31.19	0.01\\
31.2	0.01\\
31.21	0.01\\
31.22	0.01\\
31.23	0.01\\
31.24	0.01\\
31.25	0.01\\
31.26	0.01\\
31.27	0.01\\
31.28	0.01\\
31.29	0.01\\
31.3	0.01\\
31.31	0.01\\
31.32	0.01\\
31.33	0.01\\
31.34	0.01\\
31.35	0.01\\
31.36	0.01\\
31.37	0.01\\
31.38	0.01\\
31.39	0.01\\
31.4	0.01\\
31.41	0.01\\
31.42	0.01\\
31.43	0.01\\
31.44	0.01\\
31.45	0.01\\
31.46	0.01\\
31.47	0.01\\
31.48	0.01\\
31.49	0.01\\
31.5	0.01\\
31.51	0.01\\
31.52	0.01\\
31.53	0.01\\
31.54	0.01\\
31.55	0.01\\
31.56	0.01\\
31.57	0.01\\
31.58	0.01\\
31.59	0.01\\
31.6	0.01\\
31.61	0.01\\
31.62	0.01\\
31.63	0.01\\
31.64	0.01\\
31.65	0.01\\
31.66	0.01\\
31.67	0.01\\
31.68	0.01\\
31.69	0.01\\
31.7	0.01\\
31.71	0.01\\
31.72	0.01\\
31.73	0.01\\
31.74	0.01\\
31.75	0.01\\
31.76	0.01\\
31.77	0.01\\
31.78	0.01\\
31.79	0.01\\
31.8	0.01\\
31.81	0.01\\
31.82	0.01\\
31.83	0.01\\
31.84	0.01\\
31.85	0.01\\
31.86	0.01\\
31.87	0.01\\
31.88	0.01\\
31.89	0.01\\
31.9	0.01\\
31.91	0.01\\
31.92	0.01\\
31.93	0.01\\
31.94	0.01\\
31.95	0.01\\
31.96	0.01\\
31.97	0.01\\
31.98	0.01\\
31.99	0.01\\
32	0.01\\
32.01	0.01\\
32.02	0.01\\
32.03	0.01\\
32.04	0.01\\
32.05	0.01\\
32.06	0.01\\
32.07	0.01\\
32.08	0.01\\
32.09	0.01\\
32.1	0.01\\
32.11	0.01\\
32.12	0.01\\
32.13	0.01\\
32.14	0.01\\
32.15	0.01\\
32.16	0.01\\
32.17	0.01\\
32.18	0.01\\
32.19	0.01\\
32.2	0.01\\
32.21	0.01\\
32.22	0.01\\
32.23	0.01\\
32.24	0.01\\
32.25	0.01\\
32.26	0.01\\
32.27	0.01\\
32.28	0.01\\
32.29	0.01\\
32.3	0.01\\
32.31	0.01\\
32.32	0.01\\
32.33	0.01\\
32.34	0.01\\
32.35	0.01\\
32.36	0.01\\
32.37	0.01\\
32.38	0.01\\
32.39	0.01\\
32.4	0.01\\
32.41	0.01\\
32.42	0.01\\
32.43	0.01\\
32.44	0.01\\
32.45	0.01\\
32.46	0.01\\
32.47	0.01\\
32.48	0.01\\
32.49	0.01\\
32.5	0.01\\
32.51	0.01\\
32.52	0.01\\
32.53	0.01\\
32.54	0.01\\
32.55	0.01\\
32.56	0.01\\
32.57	0.01\\
32.58	0.01\\
32.59	0.01\\
32.6	0.01\\
32.61	0.01\\
32.62	0.01\\
32.63	0.01\\
32.64	0.01\\
32.65	0.01\\
32.66	0.01\\
32.67	0.01\\
32.68	0.01\\
32.69	0.01\\
32.7	0.01\\
32.71	0.01\\
32.72	0.01\\
32.73	0.01\\
32.74	0.01\\
32.75	0.01\\
32.76	0.01\\
32.77	0.01\\
32.78	0.01\\
32.79	0.01\\
32.8	0.01\\
32.81	0.01\\
32.82	0.01\\
32.83	0.01\\
32.84	0.01\\
32.85	0.01\\
32.86	0.01\\
32.87	0.01\\
32.88	0.01\\
32.89	0.01\\
32.9	0.01\\
32.91	0.01\\
32.92	0.01\\
32.93	0.01\\
32.94	0.01\\
32.95	0.01\\
32.96	0.01\\
32.97	0.01\\
32.98	0.01\\
32.99	0.01\\
33	0.01\\
33.01	0.01\\
33.02	0.01\\
33.03	0.01\\
33.04	0.01\\
33.05	0.01\\
33.06	0.01\\
33.07	0.01\\
33.08	0.01\\
33.09	0.01\\
33.1	0.01\\
33.11	0.01\\
33.12	0.01\\
33.13	0.01\\
33.14	0.01\\
33.15	0.01\\
33.16	0.01\\
33.17	0.01\\
33.18	0.01\\
33.19	0.01\\
33.2	0.01\\
33.21	0.01\\
33.22	0.01\\
33.23	0.01\\
33.24	0.01\\
33.25	0.01\\
33.26	0.01\\
33.27	0.01\\
33.28	0.01\\
33.29	0.01\\
33.3	0.01\\
33.31	0.01\\
33.32	0.01\\
33.33	0.01\\
33.34	0.01\\
33.35	0.01\\
33.36	0.01\\
33.37	0.01\\
33.38	0.01\\
33.39	0.01\\
33.4	0.01\\
33.41	0.01\\
33.42	0.01\\
33.43	0.01\\
33.44	0.01\\
33.45	0.01\\
33.46	0.01\\
33.47	0.01\\
33.48	0.01\\
33.49	0.01\\
33.5	0.01\\
33.51	0.01\\
33.52	0.01\\
33.53	0.01\\
33.54	0.01\\
33.55	0.01\\
33.56	0.01\\
33.57	0.01\\
33.58	0.01\\
33.59	0.01\\
33.6	0.01\\
33.61	0.01\\
33.62	0.01\\
33.63	0.01\\
33.64	0.01\\
33.65	0.01\\
33.66	0.01\\
33.67	0.01\\
33.68	0.01\\
33.69	0.01\\
33.7	0.01\\
33.71	0.01\\
33.72	0.01\\
33.73	0.01\\
33.74	0.01\\
33.75	0.01\\
33.76	0.01\\
33.77	0.01\\
33.78	0.01\\
33.79	0.01\\
33.8	0.01\\
33.81	0.01\\
33.82	0.01\\
33.83	0.01\\
33.84	0.01\\
33.85	0.01\\
33.86	0.01\\
33.87	0.01\\
33.88	0.01\\
33.89	0.01\\
33.9	0.01\\
33.91	0.01\\
33.92	0.01\\
33.93	0.01\\
33.94	0.01\\
33.95	0.01\\
33.96	0.01\\
33.97	0.01\\
33.98	0.01\\
33.99	0.01\\
34	0.01\\
34.01	0.01\\
34.02	0.01\\
34.03	0.01\\
34.04	0.01\\
34.05	0.01\\
34.06	0.01\\
34.07	0.01\\
34.08	0.01\\
34.09	0.01\\
34.1	0.01\\
34.11	0.01\\
34.12	0.01\\
34.13	0.01\\
34.14	0.01\\
34.15	0.01\\
34.16	0.01\\
34.17	0.01\\
34.18	0.01\\
34.19	0.01\\
34.2	0.01\\
34.21	0.01\\
34.22	0.01\\
34.23	0.01\\
34.24	0.01\\
34.25	0.01\\
34.26	0.01\\
34.27	0.01\\
34.28	0.01\\
34.29	0.01\\
34.3	0.01\\
34.31	0.01\\
34.32	0.01\\
34.33	0.01\\
34.34	0.01\\
34.35	0.01\\
34.36	0.01\\
34.37	0.01\\
34.38	0.01\\
34.39	0.01\\
34.4	0.01\\
34.41	0.01\\
34.42	0.01\\
34.43	0.01\\
34.44	0.01\\
34.45	0.01\\
34.46	0.01\\
34.47	0.01\\
34.48	0.01\\
34.49	0.01\\
34.5	0.01\\
34.51	0.01\\
34.52	0.01\\
34.53	0.01\\
34.54	0.01\\
34.55	0.01\\
34.56	0.01\\
34.57	0.01\\
34.58	0.01\\
34.59	0.01\\
34.6	0.01\\
34.61	0.01\\
34.62	0.01\\
34.63	0.01\\
34.64	0.01\\
34.65	0.01\\
34.66	0.01\\
34.67	0.01\\
34.68	0.01\\
34.69	0.01\\
34.7	0.01\\
34.71	0.01\\
34.72	0.01\\
34.73	0.01\\
34.74	0.01\\
34.75	0.01\\
34.76	0.01\\
34.77	0.01\\
34.78	0.01\\
34.79	0.01\\
34.8	0.01\\
34.81	0.01\\
34.82	0.01\\
34.83	0.01\\
34.84	0.01\\
34.85	0.01\\
34.86	0.01\\
34.87	0.01\\
34.88	0.01\\
34.89	0.01\\
34.9	0.01\\
34.91	0.01\\
34.92	0.01\\
34.93	0.01\\
34.94	0.01\\
34.95	0.01\\
34.96	0.01\\
34.97	0.01\\
34.98	0.01\\
34.99	0.01\\
35	0.01\\
35.01	0.01\\
35.02	0.01\\
35.03	0.01\\
35.04	0.01\\
35.05	0.01\\
35.06	0.01\\
35.07	0.01\\
35.08	0.01\\
35.09	0.01\\
35.1	0.01\\
35.11	0.01\\
35.12	0.01\\
35.13	0.01\\
35.14	0.01\\
35.15	0.01\\
35.16	0.01\\
35.17	0.01\\
35.18	0.01\\
35.19	0.01\\
35.2	0.01\\
35.21	0.01\\
35.22	0.01\\
35.23	0.01\\
35.24	0.01\\
35.25	0.01\\
35.26	0.01\\
35.27	0.01\\
35.28	0.01\\
35.29	0.01\\
35.3	0.01\\
35.31	0.01\\
35.32	0.01\\
35.33	0.01\\
35.34	0.01\\
35.35	0.01\\
35.36	0.01\\
35.37	0.01\\
35.38	0.01\\
35.39	0.01\\
35.4	0.01\\
35.41	0.01\\
35.42	0.01\\
35.43	0.01\\
35.44	0.01\\
35.45	0.01\\
35.46	0.01\\
35.47	0.01\\
35.48	0.01\\
35.49	0.01\\
35.5	0.01\\
35.51	0.01\\
35.52	0.01\\
35.53	0.01\\
35.54	0.01\\
35.55	0.01\\
35.56	0.01\\
35.57	0.01\\
35.58	0.01\\
35.59	0.01\\
35.6	0.01\\
35.61	0.01\\
35.62	0.01\\
35.63	0.01\\
35.64	0.01\\
35.65	0.01\\
35.66	0.01\\
35.67	0.01\\
35.68	0.01\\
35.69	0.01\\
35.7	0.01\\
35.71	0.01\\
35.72	0.01\\
35.73	0.01\\
35.74	0.01\\
35.75	0.01\\
35.76	0.01\\
35.77	0.01\\
35.78	0.01\\
35.79	0.01\\
35.8	0.01\\
35.81	0.01\\
35.82	0.01\\
35.83	0.01\\
35.84	0.01\\
35.85	0.01\\
35.86	0.01\\
35.87	0.01\\
35.88	0.01\\
35.89	0.01\\
35.9	0.01\\
35.91	0.01\\
35.92	0.01\\
35.93	0.01\\
35.94	0.01\\
35.95	0.01\\
35.96	0.01\\
35.97	0.01\\
35.98	0.01\\
35.99	0.01\\
36	0.01\\
36.01	0.01\\
36.02	0.01\\
36.03	0.01\\
36.04	0.01\\
36.05	0.01\\
36.06	0.01\\
36.07	0.01\\
36.08	0.01\\
36.09	0.01\\
36.1	0.01\\
36.11	0.01\\
36.12	0.01\\
36.13	0.01\\
36.14	0.01\\
36.15	0.01\\
36.16	0.01\\
36.17	0.01\\
36.18	0.01\\
36.19	0.01\\
36.2	0.01\\
36.21	0.01\\
36.22	0.01\\
36.23	0.01\\
36.24	0.01\\
36.25	0.01\\
36.26	0.01\\
36.27	0.01\\
36.28	0.01\\
36.29	0.01\\
36.3	0.01\\
36.31	0.01\\
36.32	0.01\\
36.33	0.01\\
36.34	0.01\\
36.35	0.01\\
36.36	0.01\\
36.37	0.01\\
36.38	0.01\\
36.39	0.01\\
36.4	0.01\\
36.41	0.01\\
36.42	0.01\\
36.43	0.01\\
36.44	0.01\\
36.45	0.01\\
36.46	0.01\\
36.47	0.01\\
36.48	0.01\\
36.49	0.01\\
36.5	0.01\\
36.51	0.01\\
36.52	0.01\\
36.53	0.01\\
36.54	0.01\\
36.55	0.01\\
36.56	0.01\\
36.57	0.01\\
36.58	0.01\\
36.59	0.01\\
36.6	0.01\\
36.61	0.01\\
36.62	0.01\\
36.63	0.01\\
36.64	0.01\\
36.65	0.01\\
36.66	0.01\\
36.67	0.01\\
36.68	0.01\\
36.69	0.01\\
36.7	0.01\\
36.71	0.01\\
36.72	0.01\\
36.73	0.01\\
36.74	0.01\\
36.75	0.01\\
36.76	0.01\\
36.77	0.01\\
36.78	0.01\\
36.79	0.01\\
36.8	0.01\\
36.81	0.01\\
36.82	0.01\\
36.83	0.01\\
36.84	0.01\\
36.85	0.01\\
36.86	0.01\\
36.87	0.01\\
36.88	0.01\\
36.89	0.01\\
36.9	0.01\\
36.91	0.01\\
36.92	0.01\\
36.93	0.01\\
36.94	0.01\\
36.95	0.01\\
36.96	0.01\\
36.97	0.01\\
36.98	0.01\\
36.99	0.01\\
37	0.01\\
37.01	0.01\\
37.02	0.01\\
37.03	0.01\\
37.04	0.01\\
37.05	0.01\\
37.06	0.01\\
37.07	0.01\\
37.08	0.01\\
37.09	0.01\\
37.1	0.01\\
37.11	0.01\\
37.12	0.01\\
37.13	0.01\\
37.14	0.01\\
37.15	0.01\\
37.16	0.01\\
37.17	0.01\\
37.18	0.01\\
37.19	0.01\\
37.2	0.01\\
37.21	0.01\\
37.22	0.01\\
37.23	0.01\\
37.24	0.01\\
37.25	0.01\\
37.26	0.01\\
37.27	0.01\\
37.28	0.01\\
37.29	0.01\\
37.3	0.01\\
37.31	0.01\\
37.32	0.01\\
37.33	0.01\\
37.34	0.01\\
37.35	0.01\\
37.36	0.01\\
37.37	0.01\\
37.38	0.01\\
37.39	0.01\\
37.4	0.01\\
37.41	0.01\\
37.42	0.01\\
37.43	0.01\\
37.44	0.01\\
37.45	0.01\\
37.46	0.01\\
37.47	0.01\\
37.48	0.01\\
37.49	0.01\\
37.5	0.01\\
37.51	0.01\\
37.52	0.01\\
37.53	0.01\\
37.54	0.01\\
37.55	0.01\\
37.56	0.01\\
37.57	0.01\\
37.58	0.01\\
37.59	0.01\\
37.6	0.01\\
37.61	0.01\\
37.62	0.01\\
37.63	0.01\\
37.64	0.01\\
37.65	0.01\\
37.66	0.01\\
37.67	0.01\\
37.68	0.01\\
37.69	0.01\\
37.7	0.01\\
37.71	0.01\\
37.72	0.01\\
37.73	0.01\\
37.74	0.01\\
37.75	0.01\\
37.76	0.01\\
37.77	0.01\\
37.78	0.01\\
37.79	0.01\\
37.8	0.01\\
37.81	0.01\\
37.82	0.01\\
37.83	0.01\\
37.84	0.01\\
37.85	0.01\\
37.86	0.01\\
37.87	0.01\\
37.88	0.01\\
37.89	0.01\\
37.9	0.01\\
37.91	0.01\\
37.92	0.01\\
37.93	0.01\\
37.94	0.01\\
37.95	0.01\\
37.96	0.01\\
37.97	0.01\\
37.98	0.01\\
37.99	0.01\\
38	0.01\\
38.01	0.01\\
38.02	0.01\\
38.03	0.01\\
38.04	0.01\\
38.05	0.01\\
38.06	0.01\\
38.07	0.01\\
38.08	0.01\\
38.09	0.01\\
38.1	0.01\\
38.11	0.01\\
38.12	0.01\\
38.13	0.01\\
38.14	0.01\\
38.15	0.01\\
38.16	0.01\\
38.17	0.01\\
38.18	0.01\\
38.19	0.01\\
38.2	0.01\\
38.21	0.01\\
38.22	0.01\\
38.23	0.01\\
38.24	0.01\\
38.25	0.01\\
38.26	0.01\\
38.27	0.01\\
38.28	0.01\\
38.29	0.01\\
38.3	0.01\\
38.31	0.01\\
38.32	0.01\\
38.33	0.01\\
38.34	0.01\\
38.35	0.01\\
38.36	0.01\\
38.37	0.01\\
38.38	0.01\\
38.39	0.01\\
38.4	0.01\\
38.41	0.01\\
38.42	0.01\\
38.43	0.01\\
38.44	0.01\\
38.45	0.01\\
38.46	0.01\\
38.47	0.01\\
38.48	0.01\\
38.49	0.01\\
38.5	0.01\\
38.51	0.01\\
38.52	0.01\\
38.53	0.01\\
38.54	0.01\\
38.55	0.01\\
38.56	0.01\\
38.57	0.01\\
38.58	0.01\\
38.59	0.01\\
38.6	0.01\\
38.61	0.01\\
38.62	0.01\\
38.63	0.01\\
38.64	0.01\\
38.65	0.01\\
38.66	0.01\\
38.67	0.01\\
38.68	0.01\\
38.69	0.01\\
38.7	0.01\\
38.71	0.01\\
38.72	0.01\\
38.73	0.01\\
38.74	0.01\\
38.75	0.01\\
38.76	0.01\\
38.77	0.01\\
38.78	0.01\\
38.79	0.01\\
38.8	0.01\\
38.81	0.01\\
38.82	0.01\\
38.83	0.01\\
38.84	0.01\\
38.85	0.01\\
38.86	0.01\\
38.87	0.01\\
38.88	0.01\\
38.89	0.01\\
38.9	0.01\\
38.91	0.01\\
38.92	0.01\\
38.93	0.01\\
38.94	0.01\\
38.95	0.01\\
38.96	0.01\\
38.97	0.01\\
38.98	0.01\\
38.99	0.01\\
39	0.01\\
39.01	0.01\\
39.02	0.01\\
39.03	0.01\\
39.04	0.01\\
39.05	0.01\\
39.06	0.01\\
39.07	0.01\\
39.08	0.01\\
39.09	0.01\\
39.1	0.01\\
39.11	0.01\\
39.12	0.01\\
39.13	0.01\\
39.14	0.01\\
39.15	0.01\\
39.16	0.01\\
39.17	0.01\\
39.18	0.01\\
39.19	0.01\\
39.2	0.01\\
39.21	0.01\\
39.22	0.01\\
39.23	0.01\\
39.24	0.01\\
39.25	0.01\\
39.26	0.01\\
39.27	0.01\\
39.28	0.01\\
39.29	0.01\\
39.3	0.01\\
39.31	0.01\\
39.32	0.01\\
39.33	0.01\\
39.34	0.01\\
39.35	0.01\\
39.36	0.01\\
39.37	0.01\\
39.38	0.01\\
39.39	0.01\\
39.4	0.01\\
39.41	0.01\\
39.42	0.01\\
39.43	0.01\\
39.44	0.01\\
39.45	0.01\\
39.46	0.01\\
39.47	0.01\\
39.48	0.01\\
39.49	0.01\\
39.5	0.01\\
39.51	0.01\\
39.52	0.01\\
39.53	0.01\\
39.54	0.01\\
39.55	0.01\\
39.56	0.01\\
39.57	0.01\\
39.58	0.01\\
39.59	0.01\\
39.6	0.01\\
39.61	0.01\\
39.62	0.01\\
39.63	0.01\\
39.64	0.01\\
39.65	0.01\\
39.66	0.01\\
39.67	0.01\\
39.68	0.01\\
39.69	0.01\\
39.7	0.01\\
39.71	0.01\\
39.72	0.01\\
39.73	0.01\\
39.74	0.01\\
39.75	0.01\\
39.76	0.01\\
39.77	0.01\\
39.78	0.01\\
39.79	0.01\\
39.8	0.01\\
39.81	0.01\\
39.82	0.01\\
39.83	0.01\\
39.84	0.01\\
39.85	0.01\\
39.86	0.01\\
39.87	0.01\\
39.88	0.01\\
39.89	0.01\\
39.9	0.01\\
39.91	0.01\\
39.92	0.01\\
39.93	0.01\\
39.94	0.01\\
39.95	0.01\\
39.96	0.01\\
39.97	0.01\\
39.98	0.01\\
39.99	0.01\\
40	0.01\\
40.01	0.01\\
};
\addplot [color=blue,dashed,forget plot]
  table[row sep=crcr]{%
40.01	0.01\\
40.02	0.01\\
40.03	0.01\\
40.04	0.01\\
40.05	0.01\\
40.06	0.01\\
40.07	0.01\\
40.08	0.01\\
40.09	0.01\\
40.1	0.01\\
40.11	0.01\\
40.12	0.01\\
40.13	0.01\\
40.14	0.01\\
40.15	0.01\\
40.16	0.01\\
40.17	0.01\\
40.18	0.01\\
40.19	0.01\\
40.2	0.01\\
40.21	0.01\\
40.22	0.01\\
40.23	0.01\\
40.24	0.01\\
40.25	0.01\\
40.26	0.01\\
40.27	0.01\\
40.28	0.01\\
40.29	0.01\\
40.3	0.01\\
40.31	0.01\\
40.32	0.01\\
40.33	0.01\\
40.34	0.01\\
40.35	0.01\\
40.36	0.01\\
40.37	0.01\\
40.38	0.01\\
40.39	0.01\\
40.4	0.01\\
40.41	0.01\\
40.42	0.01\\
40.43	0.01\\
40.44	0.01\\
40.45	0.01\\
40.46	0.01\\
40.47	0.01\\
40.48	0.01\\
40.49	0.01\\
40.5	0.01\\
40.51	0.01\\
40.52	0.01\\
40.53	0.01\\
40.54	0.01\\
40.55	0.01\\
40.56	0.01\\
40.57	0.01\\
40.58	0.01\\
40.59	0.01\\
40.6	0.01\\
40.61	0.01\\
40.62	0.01\\
40.63	0.01\\
40.64	0.01\\
40.65	0.01\\
40.66	0.01\\
40.67	0.01\\
40.68	0.01\\
40.69	0.01\\
40.7	0.01\\
40.71	0.01\\
40.72	0.01\\
40.73	0.01\\
40.74	0.01\\
40.75	0.01\\
40.76	0.01\\
40.77	0.01\\
40.78	0.01\\
40.79	0.01\\
40.8	0.01\\
40.81	0.01\\
40.82	0.01\\
40.83	0.01\\
40.84	0.01\\
40.85	0.01\\
40.86	0.01\\
40.87	0.01\\
40.88	0.01\\
40.89	0.01\\
40.9	0.01\\
40.91	0.01\\
40.92	0.01\\
40.93	0.01\\
40.94	0.01\\
40.95	0.01\\
40.96	0.01\\
40.97	0.01\\
40.98	0.01\\
40.99	0.01\\
41	0.01\\
41.01	0.01\\
41.02	0.01\\
41.03	0.01\\
41.04	0.01\\
41.05	0.01\\
41.06	0.01\\
41.07	0.01\\
41.08	0.01\\
41.09	0.01\\
41.1	0.01\\
41.11	0.01\\
41.12	0.01\\
41.13	0.01\\
41.14	0.01\\
41.15	0.01\\
41.16	0.01\\
41.17	0.01\\
41.18	0.01\\
41.19	0.01\\
41.2	0.01\\
41.21	0.01\\
41.22	0.01\\
41.23	0.01\\
41.24	0.01\\
41.25	0.01\\
41.26	0.01\\
41.27	0.01\\
41.28	0.01\\
41.29	0.01\\
41.3	0.01\\
41.31	0.01\\
41.32	0.01\\
41.33	0.01\\
41.34	0.01\\
41.35	0.01\\
41.36	0.01\\
41.37	0.01\\
41.38	0.01\\
41.39	0.01\\
41.4	0.01\\
41.41	0.01\\
41.42	0.01\\
41.43	0.01\\
41.44	0.01\\
41.45	0.01\\
41.46	0.01\\
41.47	0.01\\
41.48	0.01\\
41.49	0.01\\
41.5	0.01\\
41.51	0.01\\
41.52	0.01\\
41.53	0.01\\
41.54	0.01\\
41.55	0.01\\
41.56	0.01\\
41.57	0.01\\
41.58	0.01\\
41.59	0.01\\
41.6	0.01\\
41.61	0.01\\
41.62	0.01\\
41.63	0.01\\
41.64	0.01\\
41.65	0.01\\
41.66	0.01\\
41.67	0.01\\
41.68	0.01\\
41.69	0.01\\
41.7	0.01\\
41.71	0.01\\
41.72	0.01\\
41.73	0.01\\
41.74	0.01\\
41.75	0.01\\
41.76	0.01\\
41.77	0.01\\
41.78	0.01\\
41.79	0.01\\
41.8	0.01\\
41.81	0.01\\
41.82	0.01\\
41.83	0.01\\
41.84	0.01\\
41.85	0.01\\
41.86	0.01\\
41.87	0.01\\
41.88	0.01\\
41.89	0.01\\
41.9	0.01\\
41.91	0.01\\
41.92	0.01\\
41.93	0.01\\
41.94	0.01\\
41.95	0.01\\
41.96	0.01\\
41.97	0.01\\
41.98	0.01\\
41.99	0.01\\
42	0.01\\
42.01	0.01\\
42.02	0.01\\
42.03	0.01\\
42.04	0.01\\
42.05	0.01\\
42.06	0.01\\
42.07	0.01\\
42.08	0.01\\
42.09	0.01\\
42.1	0.01\\
42.11	0.01\\
42.12	0.01\\
42.13	0.01\\
42.14	0.01\\
42.15	0.01\\
42.16	0.01\\
42.17	0.01\\
42.18	0.01\\
42.19	0.01\\
42.2	0.01\\
42.21	0.01\\
42.22	0.01\\
42.23	0.01\\
42.24	0.01\\
42.25	0.01\\
42.26	0.01\\
42.27	0.01\\
42.28	0.01\\
42.29	0.01\\
42.3	0.01\\
42.31	0.01\\
42.32	0.01\\
42.33	0.01\\
42.34	0.01\\
42.35	0.01\\
42.36	0.01\\
42.37	0.01\\
42.38	0.01\\
42.39	0.01\\
42.4	0.01\\
42.41	0.01\\
42.42	0.01\\
42.43	0.01\\
42.44	0.01\\
42.45	0.01\\
42.46	0.01\\
42.47	0.01\\
42.48	0.01\\
42.49	0.01\\
42.5	0.01\\
42.51	0.01\\
42.52	0.01\\
42.53	0.01\\
42.54	0.01\\
42.55	0.01\\
42.56	0.01\\
42.57	0.01\\
42.58	0.01\\
42.59	0.01\\
42.6	0.01\\
42.61	0.01\\
42.62	0.01\\
42.63	0.01\\
42.64	0.01\\
42.65	0.01\\
42.66	0.01\\
42.67	0.01\\
42.68	0.01\\
42.69	0.01\\
42.7	0.01\\
42.71	0.01\\
42.72	0.01\\
42.73	0.01\\
42.74	0.01\\
42.75	0.01\\
42.76	0.01\\
42.77	0.01\\
42.78	0.01\\
42.79	0.01\\
42.8	0.01\\
42.81	0.01\\
42.82	0.01\\
42.83	0.01\\
42.84	0.01\\
42.85	0.01\\
42.86	0.01\\
42.87	0.01\\
42.88	0.01\\
42.89	0.01\\
42.9	0.01\\
42.91	0.01\\
42.92	0.01\\
42.93	0.01\\
42.94	0.01\\
42.95	0.01\\
42.96	0.01\\
42.97	0.01\\
42.98	0.01\\
42.99	0.01\\
43	0.01\\
43.01	0.01\\
43.02	0.01\\
43.03	0.01\\
43.04	0.01\\
43.05	0.01\\
43.06	0.01\\
43.07	0.01\\
43.08	0.01\\
43.09	0.01\\
43.1	0.01\\
43.11	0.01\\
43.12	0.01\\
43.13	0.01\\
43.14	0.01\\
43.15	0.01\\
43.16	0.01\\
43.17	0.01\\
43.18	0.01\\
43.19	0.01\\
43.2	0.01\\
43.21	0.01\\
43.22	0.01\\
43.23	0.01\\
43.24	0.01\\
43.25	0.01\\
43.26	0.01\\
43.27	0.01\\
43.28	0.01\\
43.29	0.01\\
43.3	0.01\\
43.31	0.01\\
43.32	0.01\\
43.33	0.01\\
43.34	0.01\\
43.35	0.01\\
43.36	0.01\\
43.37	0.01\\
43.38	0.01\\
43.39	0.01\\
43.4	0.01\\
43.41	0.01\\
43.42	0.01\\
43.43	0.01\\
43.44	0.01\\
43.45	0.01\\
43.46	0.01\\
43.47	0.01\\
43.48	0.01\\
43.49	0.01\\
43.5	0.01\\
43.51	0.01\\
43.52	0.01\\
43.53	0.01\\
43.54	0.01\\
43.55	0.01\\
43.56	0.01\\
43.57	0.01\\
43.58	0.01\\
43.59	0.01\\
43.6	0.01\\
43.61	0.01\\
43.62	0.01\\
43.63	0.01\\
43.64	0.01\\
43.65	0.01\\
43.66	0.01\\
43.67	0.01\\
43.68	0.01\\
43.69	0.01\\
43.7	0.01\\
43.71	0.01\\
43.72	0.01\\
43.73	0.01\\
43.74	0.01\\
43.75	0.01\\
43.76	0.01\\
43.77	0.01\\
43.78	0.01\\
43.79	0.01\\
43.8	0.01\\
43.81	0.01\\
43.82	0.01\\
43.83	0.01\\
43.84	0.01\\
43.85	0.01\\
43.86	0.01\\
43.87	0.01\\
43.88	0.01\\
43.89	0.01\\
43.9	0.01\\
43.91	0.01\\
43.92	0.01\\
43.93	0.01\\
43.94	0.01\\
43.95	0.01\\
43.96	0.01\\
43.97	0.01\\
43.98	0.01\\
43.99	0.01\\
44	0.01\\
44.01	0.01\\
44.02	0.01\\
44.03	0.01\\
44.04	0.01\\
44.05	0.01\\
44.06	0.01\\
44.07	0.01\\
44.08	0.01\\
44.09	0.01\\
44.1	0.01\\
44.11	0.01\\
44.12	0.01\\
44.13	0.01\\
44.14	0.01\\
44.15	0.01\\
44.16	0.01\\
44.17	0.01\\
44.18	0.01\\
44.19	0.01\\
44.2	0.01\\
44.21	0.01\\
44.22	0.01\\
44.23	0.01\\
44.24	0.01\\
44.25	0.01\\
44.26	0.01\\
44.27	0.01\\
44.28	0.01\\
44.29	0.01\\
44.3	0.01\\
44.31	0.01\\
44.32	0.01\\
44.33	0.01\\
44.34	0.01\\
44.35	0.01\\
44.36	0.01\\
44.37	0.01\\
44.38	0.01\\
44.39	0.01\\
44.4	0.01\\
44.41	0.01\\
44.42	0.01\\
44.43	0.01\\
44.44	0.01\\
44.45	0.01\\
44.46	0.01\\
44.47	0.01\\
44.48	0.01\\
44.49	0.01\\
44.5	0.01\\
44.51	0.01\\
44.52	0.01\\
44.53	0.01\\
44.54	0.01\\
44.55	0.01\\
44.56	0.01\\
44.57	0.01\\
44.58	0.01\\
44.59	0.01\\
44.6	0.01\\
44.61	0.01\\
44.62	0.01\\
44.63	0.01\\
44.64	0.01\\
44.65	0.01\\
44.66	0.01\\
44.67	0.01\\
44.68	0.01\\
44.69	0.01\\
44.7	0.01\\
44.71	0.01\\
44.72	0.01\\
44.73	0.01\\
44.74	0.01\\
44.75	0.01\\
44.76	0.01\\
44.77	0.01\\
44.78	0.01\\
44.79	0.01\\
44.8	0.01\\
44.81	0.01\\
44.82	0.01\\
44.83	0.01\\
44.84	0.01\\
44.85	0.01\\
44.86	0.01\\
44.87	0.01\\
44.88	0.01\\
44.89	0.01\\
44.9	0.01\\
44.91	0.01\\
44.92	0.01\\
44.93	0.01\\
44.94	0.01\\
44.95	0.01\\
44.96	0.01\\
44.97	0.01\\
44.98	0.01\\
44.99	0.01\\
45	0.01\\
45.01	0.01\\
45.02	0.01\\
45.03	0.01\\
45.04	0.01\\
45.05	0.01\\
45.06	0.01\\
45.07	0.01\\
45.08	0.01\\
45.09	0.01\\
45.1	0.01\\
45.11	0.01\\
45.12	0.01\\
45.13	0.01\\
45.14	0.01\\
45.15	0.01\\
45.16	0.01\\
45.17	0.01\\
45.18	0.01\\
45.19	0.01\\
45.2	0.01\\
45.21	0.01\\
45.22	0.01\\
45.23	0.01\\
45.24	0.01\\
45.25	0.01\\
45.26	0.01\\
45.27	0.01\\
45.28	0.01\\
45.29	0.01\\
45.3	0.01\\
45.31	0.01\\
45.32	0.01\\
45.33	0.01\\
45.34	0.01\\
45.35	0.01\\
45.36	0.01\\
45.37	0.01\\
45.38	0.01\\
45.39	0.01\\
45.4	0.01\\
45.41	0.01\\
45.42	0.01\\
45.43	0.01\\
45.44	0.01\\
45.45	0.01\\
45.46	0.01\\
45.47	0.01\\
45.48	0.01\\
45.49	0.01\\
45.5	0.01\\
45.51	0.01\\
45.52	0.01\\
45.53	0.01\\
45.54	0.01\\
45.55	0.01\\
45.56	0.01\\
45.57	0.01\\
45.58	0.01\\
45.59	0.01\\
45.6	0.01\\
45.61	0.01\\
45.62	0.01\\
45.63	0.01\\
45.64	0.01\\
45.65	0.01\\
45.66	0.01\\
45.67	0.01\\
45.68	0.01\\
45.69	0.01\\
45.7	0.01\\
45.71	0.01\\
45.72	0.01\\
45.73	0.01\\
45.74	0.01\\
45.75	0.01\\
45.76	0.01\\
45.77	0.01\\
45.78	0.01\\
45.79	0.01\\
45.8	0.01\\
45.81	0.01\\
45.82	0.01\\
45.83	0.01\\
45.84	0.01\\
45.85	0.01\\
45.86	0.01\\
45.87	0.01\\
45.88	0.01\\
45.89	0.01\\
45.9	0.01\\
45.91	0.01\\
45.92	0.01\\
45.93	0.01\\
45.94	0.01\\
45.95	0.01\\
45.96	0.01\\
45.97	0.01\\
45.98	0.01\\
45.99	0.01\\
46	0.01\\
46.01	0.01\\
46.02	0.01\\
46.03	0.01\\
46.04	0.01\\
46.05	0.01\\
46.06	0.01\\
46.07	0.01\\
46.08	0.01\\
46.09	0.01\\
46.1	0.01\\
46.11	0.01\\
46.12	0.01\\
46.13	0.01\\
46.14	0.01\\
46.15	0.01\\
46.16	0.01\\
46.17	0.01\\
46.18	0.01\\
46.19	0.01\\
46.2	0.01\\
46.21	0.01\\
46.22	0.01\\
46.23	0.01\\
46.24	0.01\\
46.25	0.01\\
46.26	0.01\\
46.27	0.01\\
46.28	0.01\\
46.29	0.01\\
46.3	0.01\\
46.31	0.01\\
46.32	0.01\\
46.33	0.01\\
46.34	0.01\\
46.35	0.01\\
46.36	0.01\\
46.37	0.01\\
46.38	0.01\\
46.39	0.01\\
46.4	0.01\\
46.41	0.01\\
46.42	0.01\\
46.43	0.01\\
46.44	0.01\\
46.45	0.01\\
46.46	0.01\\
46.47	0.01\\
46.48	0.01\\
46.49	0.01\\
46.5	0.01\\
46.51	0.01\\
46.52	0.01\\
46.53	0.01\\
46.54	0.01\\
46.55	0.01\\
46.56	0.01\\
46.57	0.01\\
46.58	0.01\\
46.59	0.01\\
46.6	0.01\\
46.61	0.01\\
46.62	0.01\\
46.63	0.01\\
46.64	0.01\\
46.65	0.01\\
46.66	0.01\\
46.67	0.01\\
46.68	0.01\\
46.69	0.01\\
46.7	0.01\\
46.71	0.01\\
46.72	0.01\\
46.73	0.01\\
46.74	0.01\\
46.75	0.01\\
46.76	0.01\\
46.77	0.01\\
46.78	0.01\\
46.79	0.01\\
46.8	0.01\\
46.81	0.01\\
46.82	0.01\\
46.83	0.01\\
46.84	0.01\\
46.85	0.01\\
46.86	0.01\\
46.87	0.01\\
46.88	0.01\\
46.89	0.01\\
46.9	0.01\\
46.91	0.01\\
46.92	0.01\\
46.93	0.01\\
46.94	0.01\\
46.95	0.01\\
46.96	0.01\\
46.97	0.01\\
46.98	0.01\\
46.99	0.01\\
47	0.01\\
47.01	0.01\\
47.02	0.01\\
47.03	0.01\\
47.04	0.01\\
47.05	0.01\\
47.06	0.01\\
47.07	0.01\\
47.08	0.01\\
47.09	0.01\\
47.1	0.01\\
47.11	0.01\\
47.12	0.01\\
47.13	0.01\\
47.14	0.01\\
47.15	0.01\\
47.16	0.01\\
47.17	0.01\\
47.18	0.01\\
47.19	0.01\\
47.2	0.01\\
47.21	0.01\\
47.22	0.01\\
47.23	0.01\\
47.24	0.01\\
47.25	0.01\\
47.26	0.01\\
47.27	0.01\\
47.28	0.01\\
47.29	0.01\\
47.3	0.01\\
47.31	0.01\\
47.32	0.01\\
47.33	0.01\\
47.34	0.01\\
47.35	0.01\\
47.36	0.01\\
47.37	0.01\\
47.38	0.01\\
47.39	0.01\\
47.4	0.01\\
47.41	0.01\\
47.42	0.01\\
47.43	0.01\\
47.44	0.01\\
47.45	0.01\\
47.46	0.01\\
47.47	0.01\\
47.48	0.01\\
47.49	0.01\\
47.5	0.01\\
47.51	0.01\\
47.52	0.01\\
47.53	0.01\\
47.54	0.01\\
47.55	0.01\\
47.56	0.01\\
47.57	0.01\\
47.58	0.01\\
47.59	0.01\\
47.6	0.01\\
47.61	0.01\\
47.62	0.01\\
47.63	0.01\\
47.64	0.01\\
47.65	0.01\\
47.66	0.01\\
47.67	0.01\\
47.68	0.01\\
47.69	0.01\\
47.7	0.01\\
47.71	0.01\\
47.72	0.01\\
47.73	0.01\\
47.74	0.01\\
47.75	0.01\\
47.76	0.01\\
47.77	0.01\\
47.78	0.01\\
47.79	0.01\\
47.8	0.01\\
47.81	0.01\\
47.82	0.01\\
47.83	0.01\\
47.84	0.01\\
47.85	0.01\\
47.86	0.01\\
47.87	0.01\\
47.88	0.01\\
47.89	0.01\\
47.9	0.01\\
47.91	0.01\\
47.92	0.01\\
47.93	0.01\\
47.94	0.01\\
47.95	0.01\\
47.96	0.01\\
47.97	0.01\\
47.98	0.01\\
47.99	0.01\\
48	0.01\\
48.01	0.01\\
48.02	0.01\\
48.03	0.01\\
48.04	0.01\\
48.05	0.01\\
48.06	0.01\\
48.07	0.01\\
48.08	0.01\\
48.09	0.01\\
48.1	0.01\\
48.11	0.01\\
48.12	0.01\\
48.13	0.01\\
48.14	0.01\\
48.15	0.01\\
48.16	0.01\\
48.17	0.01\\
48.18	0.01\\
48.19	0.01\\
48.2	0.01\\
48.21	0.01\\
48.22	0.01\\
48.23	0.01\\
48.24	0.01\\
48.25	0.01\\
48.26	0.01\\
48.27	0.01\\
48.28	0.01\\
48.29	0.01\\
48.3	0.01\\
48.31	0.01\\
48.32	0.01\\
48.33	0.01\\
48.34	0.01\\
48.35	0.01\\
48.36	0.01\\
48.37	0.01\\
48.38	0.01\\
48.39	0.01\\
48.4	0.01\\
48.41	0.01\\
48.42	0.01\\
48.43	0.01\\
48.44	0.01\\
48.45	0.01\\
48.46	0.01\\
48.47	0.01\\
48.48	0.01\\
48.49	0.01\\
48.5	0.01\\
48.51	0.01\\
48.52	0.01\\
48.53	0.01\\
48.54	0.01\\
48.55	0.01\\
48.56	0.01\\
48.57	0.01\\
48.58	0.01\\
48.59	0.01\\
48.6	0.01\\
48.61	0.01\\
48.62	0.01\\
48.63	0.01\\
48.64	0.01\\
48.65	0.01\\
48.66	0.01\\
48.67	0.01\\
48.68	0.01\\
48.69	0.01\\
48.7	0.01\\
48.71	0.01\\
48.72	0.01\\
48.73	0.01\\
48.74	0.01\\
48.75	0.01\\
48.76	0.01\\
48.77	0.01\\
48.78	0.01\\
48.79	0.01\\
48.8	0.01\\
48.81	0.01\\
48.82	0.01\\
48.83	0.01\\
48.84	0.01\\
48.85	0.01\\
48.86	0.01\\
48.87	0.01\\
48.88	0.01\\
48.89	0.01\\
48.9	0.01\\
48.91	0.01\\
48.92	0.01\\
48.93	0.01\\
48.94	0.01\\
48.95	0.01\\
48.96	0.01\\
48.97	0.01\\
48.98	0.01\\
48.99	0.01\\
49	0.01\\
49.01	0.01\\
49.02	0.01\\
49.03	0.01\\
49.04	0.01\\
49.05	0.01\\
49.06	0.01\\
49.07	0.01\\
49.08	0.01\\
49.09	0.01\\
49.1	0.01\\
49.11	0.01\\
49.12	0.01\\
49.13	0.01\\
49.14	0.01\\
49.15	0.01\\
49.16	0.01\\
49.17	0.01\\
49.18	0.01\\
49.19	0.01\\
49.2	0.01\\
49.21	0.01\\
49.22	0.01\\
49.23	0.01\\
49.24	0.01\\
49.25	0.01\\
49.26	0.01\\
49.27	0.01\\
49.28	0.01\\
49.29	0.01\\
49.3	0.01\\
49.31	0.01\\
49.32	0.01\\
49.33	0.01\\
49.34	0.01\\
49.35	0.01\\
49.36	0.01\\
49.37	0.01\\
49.38	0.01\\
49.39	0.01\\
49.4	0.01\\
49.41	0.01\\
49.42	0.01\\
49.43	0.01\\
49.44	0.01\\
49.45	0.01\\
49.46	0.01\\
49.47	0.01\\
49.48	0.01\\
49.49	0.01\\
49.5	0.01\\
49.51	0.01\\
49.52	0.01\\
49.53	0.01\\
49.54	0.01\\
49.55	0.01\\
49.56	0.01\\
49.57	0.01\\
49.58	0.01\\
49.59	0.01\\
49.6	0.01\\
49.61	0.01\\
49.62	0.01\\
49.63	0.01\\
49.64	0.01\\
49.65	0.01\\
49.66	0.01\\
49.67	0.01\\
49.68	0.01\\
49.69	0.01\\
49.7	0.01\\
49.71	0.01\\
49.72	0.01\\
49.73	0.01\\
49.74	0.01\\
49.75	0.01\\
49.76	0.01\\
49.77	0.01\\
49.78	0.01\\
49.79	0.01\\
49.8	0.01\\
49.81	0.01\\
49.82	0.01\\
49.83	0.01\\
49.84	0.01\\
49.85	0.01\\
49.86	0.01\\
49.87	0.01\\
49.88	0.01\\
49.89	0.01\\
49.9	0.01\\
49.91	0.01\\
49.92	0.01\\
49.93	0.01\\
49.94	0.01\\
49.95	0.01\\
49.96	0.01\\
49.97	0.01\\
49.98	0.01\\
49.99	0.01\\
50	0.01\\
50.01	0.01\\
50.02	0.01\\
50.03	0.01\\
50.04	0.01\\
50.05	0.01\\
50.06	0.01\\
50.07	0.01\\
50.08	0.01\\
50.09	0.01\\
50.1	0.01\\
50.11	0.01\\
50.12	0.01\\
50.13	0.01\\
50.14	0.01\\
50.15	0.01\\
50.16	0.01\\
50.17	0.01\\
50.18	0.01\\
50.19	0.01\\
50.2	0.01\\
50.21	0.01\\
50.22	0.01\\
50.23	0.01\\
50.24	0.01\\
50.25	0.01\\
50.26	0.01\\
50.27	0.01\\
50.28	0.01\\
50.29	0.01\\
50.3	0.01\\
50.31	0.01\\
50.32	0.01\\
50.33	0.01\\
50.34	0.01\\
50.35	0.01\\
50.36	0.01\\
50.37	0.01\\
50.38	0.01\\
50.39	0.01\\
50.4	0.01\\
50.41	0.01\\
50.42	0.01\\
50.43	0.01\\
50.44	0.01\\
50.45	0.01\\
50.46	0.01\\
50.47	0.01\\
50.48	0.01\\
50.49	0.01\\
50.5	0.01\\
50.51	0.01\\
50.52	0.01\\
50.53	0.01\\
50.54	0.01\\
50.55	0.01\\
50.56	0.01\\
50.57	0.01\\
50.58	0.01\\
50.59	0.01\\
50.6	0.01\\
50.61	0.01\\
50.62	0.01\\
50.63	0.01\\
50.64	0.01\\
50.65	0.01\\
50.66	0.01\\
50.67	0.01\\
50.68	0.01\\
50.69	0.01\\
50.7	0.01\\
50.71	0.01\\
50.72	0.01\\
50.73	0.01\\
50.74	0.01\\
50.75	0.01\\
50.76	0.01\\
50.77	0.01\\
50.78	0.01\\
50.79	0.01\\
50.8	0.01\\
50.81	0.01\\
50.82	0.01\\
50.83	0.01\\
50.84	0.01\\
50.85	0.01\\
50.86	0.01\\
50.87	0.01\\
50.88	0.01\\
50.89	0.01\\
50.9	0.01\\
50.91	0.01\\
50.92	0.01\\
50.93	0.01\\
50.94	0.01\\
50.95	0.01\\
50.96	0.01\\
50.97	0.01\\
50.98	0.01\\
50.99	0.01\\
51	0.01\\
51.01	0.01\\
51.02	0.01\\
51.03	0.01\\
51.04	0.01\\
51.05	0.01\\
51.06	0.01\\
51.07	0.01\\
51.08	0.01\\
51.09	0.01\\
51.1	0.01\\
51.11	0.01\\
51.12	0.01\\
51.13	0.01\\
51.14	0.01\\
51.15	0.01\\
51.16	0.01\\
51.17	0.01\\
51.18	0.01\\
51.19	0.01\\
51.2	0.01\\
51.21	0.01\\
51.22	0.01\\
51.23	0.01\\
51.24	0.01\\
51.25	0.01\\
51.26	0.01\\
51.27	0.01\\
51.28	0.01\\
51.29	0.01\\
51.3	0.01\\
51.31	0.01\\
51.32	0.01\\
51.33	0.01\\
51.34	0.01\\
51.35	0.01\\
51.36	0.01\\
51.37	0.01\\
51.38	0.01\\
51.39	0.01\\
51.4	0.01\\
51.41	0.01\\
51.42	0.01\\
51.43	0.01\\
51.44	0.01\\
51.45	0.01\\
51.46	0.01\\
51.47	0.01\\
51.48	0.01\\
51.49	0.01\\
51.5	0.01\\
51.51	0.01\\
51.52	0.01\\
51.53	0.01\\
51.54	0.01\\
51.55	0.01\\
51.56	0.01\\
51.57	0.01\\
51.58	0.01\\
51.59	0.01\\
51.6	0.01\\
51.61	0.01\\
51.62	0.01\\
51.63	0.01\\
51.64	0.01\\
51.65	0.01\\
51.66	0.01\\
51.67	0.01\\
51.68	0.01\\
51.69	0.01\\
51.7	0.01\\
51.71	0.01\\
51.72	0.01\\
51.73	0.01\\
51.74	0.01\\
51.75	0.01\\
51.76	0.01\\
51.77	0.01\\
51.78	0.01\\
51.79	0.01\\
51.8	0.01\\
51.81	0.01\\
51.82	0.01\\
51.83	0.01\\
51.84	0.01\\
51.85	0.01\\
51.86	0.01\\
51.87	0.01\\
51.88	0.01\\
51.89	0.01\\
51.9	0.01\\
51.91	0.01\\
51.92	0.01\\
51.93	0.01\\
51.94	0.01\\
51.95	0.01\\
51.96	0.01\\
51.97	0.01\\
51.98	0.01\\
51.99	0.01\\
52	0.01\\
52.01	0.01\\
52.02	0.01\\
52.03	0.01\\
52.04	0.01\\
52.05	0.01\\
52.06	0.01\\
52.07	0.01\\
52.08	0.01\\
52.09	0.01\\
52.1	0.01\\
52.11	0.01\\
52.12	0.01\\
52.13	0.01\\
52.14	0.01\\
52.15	0.01\\
52.16	0.01\\
52.17	0.01\\
52.18	0.01\\
52.19	0.01\\
52.2	0.01\\
52.21	0.01\\
52.22	0.01\\
52.23	0.01\\
52.24	0.01\\
52.25	0.01\\
52.26	0.01\\
52.27	0.01\\
52.28	0.01\\
52.29	0.01\\
52.3	0.01\\
52.31	0.01\\
52.32	0.01\\
52.33	0.01\\
52.34	0.01\\
52.35	0.01\\
52.36	0.01\\
52.37	0.01\\
52.38	0.01\\
52.39	0.01\\
52.4	0.01\\
52.41	0.01\\
52.42	0.01\\
52.43	0.01\\
52.44	0.01\\
52.45	0.01\\
52.46	0.01\\
52.47	0.01\\
52.48	0.01\\
52.49	0.01\\
52.5	0.01\\
52.51	0.01\\
52.52	0.01\\
52.53	0.01\\
52.54	0.01\\
52.55	0.01\\
52.56	0.01\\
52.57	0.01\\
52.58	0.01\\
52.59	0.01\\
52.6	0.01\\
52.61	0.01\\
52.62	0.01\\
52.63	0.01\\
52.64	0.01\\
52.65	0.01\\
52.66	0.01\\
52.67	0.01\\
52.68	0.01\\
52.69	0.01\\
52.7	0.01\\
52.71	0.01\\
52.72	0.01\\
52.73	0.01\\
52.74	0.01\\
52.75	0.01\\
52.76	0.01\\
52.77	0.01\\
52.78	0.01\\
52.79	0.01\\
52.8	0.01\\
52.81	0.01\\
52.82	0.01\\
52.83	0.01\\
52.84	0.01\\
52.85	0.01\\
52.86	0.01\\
52.87	0.01\\
52.88	0.01\\
52.89	0.01\\
52.9	0.01\\
52.91	0.01\\
52.92	0.01\\
52.93	0.01\\
52.94	0.01\\
52.95	0.01\\
52.96	0.01\\
52.97	0.01\\
52.98	0.01\\
52.99	0.01\\
53	0.01\\
53.01	0.01\\
53.02	0.01\\
53.03	0.01\\
53.04	0.01\\
53.05	0.01\\
53.06	0.01\\
53.07	0.01\\
53.08	0.01\\
53.09	0.01\\
53.1	0.01\\
53.11	0.01\\
53.12	0.01\\
53.13	0.01\\
53.14	0.01\\
53.15	0.01\\
53.16	0.01\\
53.17	0.01\\
53.18	0.01\\
53.19	0.01\\
53.2	0.01\\
53.21	0.01\\
53.22	0.01\\
53.23	0.01\\
53.24	0.01\\
53.25	0.01\\
53.26	0.01\\
53.27	0.01\\
53.28	0.01\\
53.29	0.01\\
53.3	0.01\\
53.31	0.01\\
53.32	0.01\\
53.33	0.01\\
53.34	0.01\\
53.35	0.01\\
53.36	0.01\\
53.37	0.01\\
53.38	0.01\\
53.39	0.01\\
53.4	0.01\\
53.41	0.01\\
53.42	0.01\\
53.43	0.01\\
53.44	0.01\\
53.45	0.01\\
53.46	0.01\\
53.47	0.01\\
53.48	0.01\\
53.49	0.01\\
53.5	0.01\\
53.51	0.01\\
53.52	0.01\\
53.53	0.01\\
53.54	0.01\\
53.55	0.01\\
53.56	0.01\\
53.57	0.01\\
53.58	0.01\\
53.59	0.01\\
53.6	0.01\\
53.61	0.01\\
53.62	0.01\\
53.63	0.01\\
53.64	0.01\\
53.65	0.01\\
53.66	0.01\\
53.67	0.01\\
53.68	0.01\\
53.69	0.01\\
53.7	0.01\\
53.71	0.01\\
53.72	0.01\\
53.73	0.01\\
53.74	0.01\\
53.75	0.01\\
53.76	0.01\\
53.77	0.01\\
53.78	0.01\\
53.79	0.01\\
53.8	0.01\\
53.81	0.01\\
53.82	0.01\\
53.83	0.01\\
53.84	0.01\\
53.85	0.01\\
53.86	0.01\\
53.87	0.01\\
53.88	0.01\\
53.89	0.01\\
53.9	0.01\\
53.91	0.01\\
53.92	0.01\\
53.93	0.01\\
53.94	0.01\\
53.95	0.01\\
53.96	0.01\\
53.97	0.01\\
53.98	0.01\\
53.99	0.01\\
54	0.01\\
54.01	0.01\\
54.02	0.01\\
54.03	0.01\\
54.04	0.01\\
54.05	0.01\\
54.06	0.01\\
54.07	0.01\\
54.08	0.01\\
54.09	0.01\\
54.1	0.01\\
54.11	0.01\\
54.12	0.01\\
54.13	0.01\\
54.14	0.01\\
54.15	0.01\\
54.16	0.01\\
54.17	0.01\\
54.18	0.01\\
54.19	0.01\\
54.2	0.01\\
54.21	0.01\\
54.22	0.01\\
54.23	0.01\\
54.24	0.01\\
54.25	0.01\\
54.26	0.01\\
54.27	0.01\\
54.28	0.01\\
54.29	0.01\\
54.3	0.01\\
54.31	0.01\\
54.32	0.01\\
54.33	0.01\\
54.34	0.01\\
54.35	0.01\\
54.36	0.01\\
54.37	0.01\\
54.38	0.01\\
54.39	0.01\\
54.4	0.01\\
54.41	0.01\\
54.42	0.01\\
54.43	0.01\\
54.44	0.01\\
54.45	0.01\\
54.46	0.01\\
54.47	0.01\\
54.48	0.01\\
54.49	0.01\\
54.5	0.01\\
54.51	0.01\\
54.52	0.01\\
54.53	0.01\\
54.54	0.01\\
54.55	0.01\\
54.56	0.01\\
54.57	0.01\\
54.58	0.01\\
54.59	0.01\\
54.6	0.01\\
54.61	0.01\\
54.62	0.01\\
54.63	0.01\\
54.64	0.01\\
54.65	0.01\\
54.66	0.01\\
54.67	0.01\\
54.68	0.01\\
54.69	0.01\\
54.7	0.01\\
54.71	0.01\\
54.72	0.01\\
54.73	0.01\\
54.74	0.01\\
54.75	0.01\\
54.76	0.01\\
54.77	0.01\\
54.78	0.01\\
54.79	0.01\\
54.8	0.01\\
54.81	0.01\\
54.82	0.01\\
54.83	0.01\\
54.84	0.01\\
54.85	0.01\\
54.86	0.01\\
54.87	0.01\\
54.88	0.01\\
54.89	0.01\\
54.9	0.01\\
54.91	0.01\\
54.92	0.01\\
54.93	0.01\\
54.94	0.01\\
54.95	0.01\\
54.96	0.01\\
54.97	0.01\\
54.98	0.01\\
54.99	0.01\\
55	0.01\\
55.01	0.01\\
55.02	0.01\\
55.03	0.01\\
55.04	0.01\\
55.05	0.01\\
55.06	0.01\\
55.07	0.01\\
55.08	0.01\\
55.09	0.01\\
55.1	0.01\\
55.11	0.01\\
55.12	0.01\\
55.13	0.01\\
55.14	0.01\\
55.15	0.01\\
55.16	0.01\\
55.17	0.01\\
55.18	0.01\\
55.19	0.01\\
55.2	0.01\\
55.21	0.01\\
55.22	0.01\\
55.23	0.01\\
55.24	0.01\\
55.25	0.01\\
55.26	0.01\\
55.27	0.01\\
55.28	0.01\\
55.29	0.01\\
55.3	0.01\\
55.31	0.01\\
55.32	0.01\\
55.33	0.01\\
55.34	0.01\\
55.35	0.01\\
55.36	0.01\\
55.37	0.01\\
55.38	0.01\\
55.39	0.01\\
55.4	0.01\\
55.41	0.01\\
55.42	0.01\\
55.43	0.01\\
55.44	0.01\\
55.45	0.01\\
55.46	0.01\\
55.47	0.01\\
55.48	0.01\\
55.49	0.01\\
55.5	0.01\\
55.51	0.01\\
55.52	0.01\\
55.53	0.01\\
55.54	0.01\\
55.55	0.01\\
55.56	0.01\\
55.57	0.01\\
55.58	0.01\\
55.59	0.01\\
55.6	0.01\\
55.61	0.01\\
55.62	0.01\\
55.63	0.01\\
55.64	0.01\\
55.65	0.01\\
55.66	0.01\\
55.67	0.01\\
55.68	0.01\\
55.69	0.01\\
55.7	0.01\\
55.71	0.01\\
55.72	0.01\\
55.73	0.01\\
55.74	0.01\\
55.75	0.01\\
55.76	0.01\\
55.77	0.01\\
55.78	0.01\\
55.79	0.01\\
55.8	0.01\\
55.81	0.01\\
55.82	0.01\\
55.83	0.01\\
55.84	0.01\\
55.85	0.01\\
55.86	0.01\\
55.87	0.01\\
55.88	0.01\\
55.89	0.01\\
55.9	0.01\\
55.91	0.01\\
55.92	0.01\\
55.93	0.01\\
55.94	0.01\\
55.95	0.01\\
55.96	0.01\\
55.97	0.01\\
55.98	0.01\\
55.99	0.01\\
56	0.01\\
56.01	0.01\\
56.02	0.01\\
56.03	0.01\\
56.04	0.01\\
56.05	0.01\\
56.06	0.01\\
56.07	0.01\\
56.08	0.01\\
56.09	0.01\\
56.1	0.01\\
56.11	0.01\\
56.12	0.01\\
56.13	0.01\\
56.14	0.01\\
56.15	0.01\\
56.16	0.01\\
56.17	0.01\\
56.18	0.01\\
56.19	0.01\\
56.2	0.01\\
56.21	0.01\\
56.22	0.01\\
56.23	0.01\\
56.24	0.01\\
56.25	0.01\\
56.26	0.01\\
56.27	0.01\\
56.28	0.01\\
56.29	0.01\\
56.3	0.01\\
56.31	0.01\\
56.32	0.01\\
56.33	0.01\\
56.34	0.01\\
56.35	0.01\\
56.36	0.01\\
56.37	0.01\\
56.38	0.01\\
56.39	0.01\\
56.4	0.01\\
56.41	0.01\\
56.42	0.01\\
56.43	0.01\\
56.44	0.01\\
56.45	0.01\\
56.46	0.01\\
56.47	0.01\\
56.48	0.01\\
56.49	0.01\\
56.5	0.01\\
56.51	0.01\\
56.52	0.01\\
56.53	0.01\\
56.54	0.01\\
56.55	0.01\\
56.56	0.01\\
56.57	0.01\\
56.58	0.01\\
56.59	0.01\\
56.6	0.01\\
56.61	0.01\\
56.62	0.01\\
56.63	0.01\\
56.64	0.01\\
56.65	0.01\\
56.66	0.01\\
56.67	0.01\\
56.68	0.01\\
56.69	0.01\\
56.7	0.01\\
56.71	0.01\\
56.72	0.01\\
56.73	0.01\\
56.74	0.01\\
56.75	0.01\\
56.76	0.01\\
56.77	0.01\\
56.78	0.01\\
56.79	0.01\\
56.8	0.01\\
56.81	0.01\\
56.82	0.01\\
56.83	0.01\\
56.84	0.01\\
56.85	0.01\\
56.86	0.01\\
56.87	0.01\\
56.88	0.01\\
56.89	0.01\\
56.9	0.01\\
56.91	0.01\\
56.92	0.01\\
56.93	0.01\\
56.94	0.01\\
56.95	0.01\\
56.96	0.01\\
56.97	0.01\\
56.98	0.01\\
56.99	0.01\\
57	0.01\\
57.01	0.01\\
57.02	0.01\\
57.03	0.01\\
57.04	0.01\\
57.05	0.01\\
57.06	0.01\\
57.07	0.01\\
57.08	0.01\\
57.09	0.01\\
57.1	0.01\\
57.11	0.01\\
57.12	0.01\\
57.13	0.01\\
57.14	0.01\\
57.15	0.01\\
57.16	0.01\\
57.17	0.01\\
57.18	0.01\\
57.19	0.01\\
57.2	0.01\\
57.21	0.01\\
57.22	0.01\\
57.23	0.01\\
57.24	0.01\\
57.25	0.01\\
57.26	0.01\\
57.27	0.01\\
57.28	0.01\\
57.29	0.01\\
57.3	0.01\\
57.31	0.01\\
57.32	0.01\\
57.33	0.01\\
57.34	0.01\\
57.35	0.01\\
57.36	0.01\\
57.37	0.01\\
57.38	0.01\\
57.39	0.01\\
57.4	0.01\\
57.41	0.01\\
57.42	0.01\\
57.43	0.01\\
57.44	0.01\\
57.45	0.01\\
57.46	0.01\\
57.47	0.01\\
57.48	0.01\\
57.49	0.01\\
57.5	0.01\\
57.51	0.01\\
57.52	0.01\\
57.53	0.01\\
57.54	0.01\\
57.55	0.01\\
57.56	0.01\\
57.57	0.01\\
57.58	0.01\\
57.59	0.01\\
57.6	0.01\\
57.61	0.01\\
57.62	0.01\\
57.63	0.01\\
57.64	0.01\\
57.65	0.01\\
57.66	0.01\\
57.67	0.01\\
57.68	0.01\\
57.69	0.01\\
57.7	0.01\\
57.71	0.01\\
57.72	0.01\\
57.73	0.01\\
57.74	0.01\\
57.75	0.01\\
57.76	0.01\\
57.77	0.01\\
57.78	0.01\\
57.79	0.01\\
57.8	0.01\\
57.81	0.01\\
57.82	0.01\\
57.83	0.01\\
57.84	0.01\\
57.85	0.01\\
57.86	0.01\\
57.87	0.01\\
57.88	0.01\\
57.89	0.01\\
57.9	0.01\\
57.91	0.01\\
57.92	0.01\\
57.93	0.01\\
57.94	0.01\\
57.95	0.01\\
57.96	0.01\\
57.97	0.01\\
57.98	0.01\\
57.99	0.01\\
58	0.01\\
58.01	0.01\\
58.02	0.01\\
58.03	0.01\\
58.04	0.01\\
58.05	0.01\\
58.06	0.01\\
58.07	0.01\\
58.08	0.01\\
58.09	0.01\\
58.1	0.01\\
58.11	0.01\\
58.12	0.01\\
58.13	0.01\\
58.14	0.01\\
58.15	0.01\\
58.16	0.01\\
58.17	0.01\\
58.18	0.01\\
58.19	0.01\\
58.2	0.01\\
58.21	0.01\\
58.22	0.01\\
58.23	0.01\\
58.24	0.01\\
58.25	0.01\\
58.26	0.01\\
58.27	0.01\\
58.28	0.01\\
58.29	0.01\\
58.3	0.01\\
58.31	0.01\\
58.32	0.01\\
58.33	0.01\\
58.34	0.01\\
58.35	0.01\\
58.36	0.01\\
58.37	0.01\\
58.38	0.01\\
58.39	0.01\\
58.4	0.01\\
58.41	0.01\\
58.42	0.01\\
58.43	0.01\\
58.44	0.01\\
58.45	0.01\\
58.46	0.01\\
58.47	0.01\\
58.48	0.01\\
58.49	0.01\\
58.5	0.01\\
58.51	0.01\\
58.52	0.01\\
58.53	0.01\\
58.54	0.01\\
58.55	0.01\\
58.56	0.01\\
58.57	0.01\\
58.58	0.01\\
58.59	0.01\\
58.6	0.01\\
58.61	0.01\\
58.62	0.01\\
58.63	0.01\\
58.64	0.01\\
58.65	0.01\\
58.66	0.01\\
58.67	0.01\\
58.68	0.01\\
58.69	0.01\\
58.7	0.01\\
58.71	0.01\\
58.72	0.01\\
58.73	0.01\\
58.74	0.01\\
58.75	0.01\\
58.76	0.01\\
58.77	0.01\\
58.78	0.01\\
58.79	0.01\\
58.8	0.01\\
58.81	0.01\\
58.82	0.01\\
58.83	0.01\\
58.84	0.01\\
58.85	0.01\\
58.86	0.01\\
58.87	0.01\\
58.88	0.01\\
58.89	0.01\\
58.9	0.01\\
58.91	0.01\\
58.92	0.01\\
58.93	0.01\\
58.94	0.01\\
58.95	0.01\\
58.96	0.01\\
58.97	0.01\\
58.98	0.01\\
58.99	0.01\\
59	0.01\\
59.01	0.01\\
59.02	0.01\\
59.03	0.01\\
59.04	0.01\\
59.05	0.01\\
59.06	0.01\\
59.07	0.01\\
59.08	0.01\\
59.09	0.01\\
59.1	0.01\\
59.11	0.01\\
59.12	0.01\\
59.13	0.01\\
59.14	0.01\\
59.15	0.01\\
59.16	0.01\\
59.17	0.01\\
59.18	0.01\\
59.19	0.01\\
59.2	0.01\\
59.21	0.01\\
59.22	0.01\\
59.23	0.01\\
59.24	0.01\\
59.25	0.01\\
59.26	0.01\\
59.27	0.01\\
59.28	0.01\\
59.29	0.01\\
59.3	0.01\\
59.31	0.01\\
59.32	0.01\\
59.33	0.01\\
59.34	0.01\\
59.35	0.01\\
59.36	0.01\\
59.37	0.01\\
59.38	0.01\\
59.39	0.01\\
59.4	0.01\\
59.41	0.01\\
59.42	0.01\\
59.43	0.01\\
59.44	0.01\\
59.45	0.01\\
59.46	0.01\\
59.47	0.01\\
59.48	0.01\\
59.49	0.01\\
59.5	0.01\\
59.51	0.01\\
59.52	0.01\\
59.53	0.01\\
59.54	0.01\\
59.55	0.01\\
59.56	0.01\\
59.57	0.01\\
59.58	0.01\\
59.59	0.01\\
59.6	0.01\\
59.61	0.01\\
59.62	0.01\\
59.63	0.01\\
59.64	0.01\\
59.65	0.01\\
59.66	0.01\\
59.67	0.01\\
59.68	0.01\\
59.69	0.01\\
59.7	0.01\\
59.71	0.01\\
59.72	0.01\\
59.73	0.01\\
59.74	0.01\\
59.75	0.01\\
59.76	0.01\\
59.77	0.01\\
59.78	0.01\\
59.79	0.01\\
59.8	0.01\\
59.81	0.01\\
59.82	0.01\\
59.83	0.01\\
59.84	0.01\\
59.85	0.01\\
59.86	0.01\\
59.87	0.01\\
59.88	0.01\\
59.89	0.01\\
59.9	0.01\\
59.91	0.01\\
59.92	0.01\\
59.93	0.01\\
59.94	0.01\\
59.95	0.01\\
59.96	0.01\\
59.97	0.01\\
59.98	0.01\\
59.99	0.01\\
60	0.01\\
60.01	0.01\\
60.02	0.01\\
60.03	0.01\\
60.04	0.01\\
60.05	0.01\\
60.06	0.01\\
60.07	0.01\\
60.08	0.01\\
60.09	0.01\\
60.1	0.01\\
60.11	0.01\\
60.12	0.01\\
60.13	0.01\\
60.14	0.01\\
60.15	0.01\\
60.16	0.01\\
60.17	0.01\\
60.18	0.01\\
60.19	0.01\\
60.2	0.01\\
60.21	0.01\\
60.22	0.01\\
60.23	0.01\\
60.24	0.01\\
60.25	0.01\\
60.26	0.01\\
60.27	0.01\\
60.28	0.01\\
60.29	0.01\\
60.3	0.01\\
60.31	0.01\\
60.32	0.01\\
60.33	0.01\\
60.34	0.01\\
60.35	0.01\\
60.36	0.01\\
60.37	0.01\\
60.38	0.01\\
60.39	0.01\\
60.4	0.01\\
60.41	0.01\\
60.42	0.01\\
60.43	0.01\\
60.44	0.01\\
60.45	0.01\\
60.46	0.01\\
60.47	0.01\\
60.48	0.01\\
60.49	0.01\\
60.5	0.01\\
60.51	0.01\\
60.52	0.01\\
60.53	0.01\\
60.54	0.01\\
60.55	0.01\\
60.56	0.01\\
60.57	0.01\\
60.58	0.01\\
60.59	0.01\\
60.6	0.01\\
60.61	0.01\\
60.62	0.01\\
60.63	0.01\\
60.64	0.01\\
60.65	0.01\\
60.66	0.01\\
60.67	0.01\\
60.68	0.01\\
60.69	0.01\\
60.7	0.01\\
60.71	0.01\\
60.72	0.01\\
60.73	0.01\\
60.74	0.01\\
60.75	0.01\\
60.76	0.01\\
60.77	0.01\\
60.78	0.01\\
60.79	0.01\\
60.8	0.01\\
60.81	0.01\\
60.82	0.01\\
60.83	0.01\\
60.84	0.01\\
60.85	0.01\\
60.86	0.01\\
60.87	0.01\\
60.88	0.01\\
60.89	0.01\\
60.9	0.01\\
60.91	0.01\\
60.92	0.01\\
60.93	0.01\\
60.94	0.01\\
60.95	0.01\\
60.96	0.01\\
60.97	0.01\\
60.98	0.01\\
60.99	0.01\\
61	0.01\\
61.01	0.01\\
61.02	0.01\\
61.03	0.01\\
61.04	0.01\\
61.05	0.01\\
61.06	0.01\\
61.07	0.01\\
61.08	0.01\\
61.09	0.01\\
61.1	0.01\\
61.11	0.01\\
61.12	0.01\\
61.13	0.01\\
61.14	0.01\\
61.15	0.01\\
61.16	0.01\\
61.17	0.01\\
61.18	0.01\\
61.19	0.01\\
61.2	0.01\\
61.21	0.01\\
61.22	0.01\\
61.23	0.01\\
61.24	0.01\\
61.25	0.01\\
61.26	0.01\\
61.27	0.01\\
61.28	0.01\\
61.29	0.01\\
61.3	0.01\\
61.31	0.01\\
61.32	0.01\\
61.33	0.01\\
61.34	0.01\\
61.35	0.01\\
61.36	0.01\\
61.37	0.01\\
61.38	0.01\\
61.39	0.01\\
61.4	0.01\\
61.41	0.01\\
61.42	0.01\\
61.43	0.01\\
61.44	0.01\\
61.45	0.01\\
61.46	0.01\\
61.47	0.01\\
61.48	0.01\\
61.49	0.01\\
61.5	0.01\\
61.51	0.01\\
61.52	0.01\\
61.53	0.01\\
61.54	0.01\\
61.55	0.01\\
61.56	0.01\\
61.57	0.01\\
61.58	0.01\\
61.59	0.01\\
61.6	0.01\\
61.61	0.01\\
61.62	0.01\\
61.63	0.01\\
61.64	0.01\\
61.65	0.01\\
61.66	0.01\\
61.67	0.01\\
61.68	0.01\\
61.69	0.01\\
61.7	0.01\\
61.71	0.01\\
61.72	0.01\\
61.73	0.01\\
61.74	0.01\\
61.75	0.01\\
61.76	0.01\\
61.77	0.01\\
61.78	0.01\\
61.79	0.01\\
61.8	0.01\\
61.81	0.01\\
61.82	0.01\\
61.83	0.01\\
61.84	0.01\\
61.85	0.01\\
61.86	0.01\\
61.87	0.01\\
61.88	0.01\\
61.89	0.01\\
61.9	0.01\\
61.91	0.01\\
61.92	0.01\\
61.93	0.01\\
61.94	0.01\\
61.95	0.01\\
61.96	0.01\\
61.97	0.01\\
61.98	0.01\\
61.99	0.01\\
62	0.01\\
62.01	0.01\\
62.02	0.01\\
62.03	0.01\\
62.04	0.01\\
62.05	0.01\\
62.06	0.01\\
62.07	0.01\\
62.08	0.01\\
62.09	0.01\\
62.1	0.01\\
62.11	0.01\\
62.12	0.01\\
62.13	0.01\\
62.14	0.01\\
62.15	0.01\\
62.16	0.01\\
62.17	0.01\\
62.18	0.01\\
62.19	0.01\\
62.2	0.01\\
62.21	0.01\\
62.22	0.01\\
62.23	0.01\\
62.24	0.01\\
62.25	0.01\\
62.26	0.01\\
62.27	0.01\\
62.28	0.01\\
62.29	0.01\\
62.3	0.01\\
62.31	0.01\\
62.32	0.01\\
62.33	0.01\\
62.34	0.01\\
62.35	0.01\\
62.36	0.01\\
62.37	0.01\\
62.38	0.01\\
62.39	0.01\\
62.4	0.01\\
62.41	0.01\\
62.42	0.01\\
62.43	0.01\\
62.44	0.01\\
62.45	0.01\\
62.46	0.01\\
62.47	0.01\\
62.48	0.01\\
62.49	0.01\\
62.5	0.01\\
62.51	0.01\\
62.52	0.01\\
62.53	0.01\\
62.54	0.01\\
62.55	0.01\\
62.56	0.01\\
62.57	0.01\\
62.58	0.01\\
62.59	0.01\\
62.6	0.01\\
62.61	0.01\\
62.62	0.01\\
62.63	0.01\\
62.64	0.01\\
62.65	0.01\\
62.66	0.01\\
62.67	0.01\\
62.68	0.01\\
62.69	0.01\\
62.7	0.01\\
62.71	0.01\\
62.72	0.01\\
62.73	0.01\\
62.74	0.01\\
62.75	0.01\\
62.76	0.01\\
62.77	0.01\\
62.78	0.01\\
62.79	0.01\\
62.8	0.01\\
62.81	0.01\\
62.82	0.01\\
62.83	0.01\\
62.84	0.01\\
62.85	0.01\\
62.86	0.01\\
62.87	0.01\\
62.88	0.01\\
62.89	0.01\\
62.9	0.01\\
62.91	0.01\\
62.92	0.01\\
62.93	0.01\\
62.94	0.01\\
62.95	0.01\\
62.96	0.01\\
62.97	0.01\\
62.98	0.01\\
62.99	0.01\\
63	0.01\\
63.01	0.01\\
63.02	0.01\\
63.03	0.01\\
63.04	0.01\\
63.05	0.01\\
63.06	0.01\\
63.07	0.01\\
63.08	0.01\\
63.09	0.01\\
63.1	0.01\\
63.11	0.01\\
63.12	0.01\\
63.13	0.01\\
63.14	0.01\\
63.15	0.01\\
63.16	0.01\\
63.17	0.01\\
63.18	0.01\\
63.19	0.01\\
63.2	0.01\\
63.21	0.01\\
63.22	0.01\\
63.23	0.01\\
63.24	0.01\\
63.25	0.01\\
63.26	0.01\\
63.27	0.01\\
63.28	0.01\\
63.29	0.01\\
63.3	0.01\\
63.31	0.01\\
63.32	0.01\\
63.33	0.01\\
63.34	0.01\\
63.35	0.01\\
63.36	0.01\\
63.37	0.01\\
63.38	0.01\\
63.39	0.01\\
63.4	0.01\\
63.41	0.01\\
63.42	0.01\\
63.43	0.01\\
63.44	0.01\\
63.45	0.01\\
63.46	0.01\\
63.47	0.01\\
63.48	0.01\\
63.49	0.01\\
63.5	0.01\\
63.51	0.01\\
63.52	0.01\\
63.53	0.01\\
63.54	0.01\\
63.55	0.01\\
63.56	0.01\\
63.57	0.01\\
63.58	0.01\\
63.59	0.01\\
63.6	0.01\\
63.61	0.01\\
63.62	0.01\\
63.63	0.01\\
63.64	0.01\\
63.65	0.01\\
63.66	0.01\\
63.67	0.01\\
63.68	0.01\\
63.69	0.01\\
63.7	0.01\\
63.71	0.01\\
63.72	0.01\\
63.73	0.01\\
63.74	0.01\\
63.75	0.01\\
63.76	0.01\\
63.77	0.01\\
63.78	0.01\\
63.79	0.01\\
63.8	0.01\\
63.81	0.01\\
63.82	0.01\\
63.83	0.01\\
63.84	0.01\\
63.85	0.01\\
63.86	0.01\\
63.87	0.01\\
63.88	0.01\\
63.89	0.01\\
63.9	0.01\\
63.91	0.01\\
63.92	0.01\\
63.93	0.01\\
63.94	0.01\\
63.95	0.01\\
63.96	0.01\\
63.97	0.01\\
63.98	0.01\\
63.99	0.01\\
64	0.01\\
64.01	0.01\\
64.02	0.01\\
64.03	0.01\\
64.04	0.01\\
64.05	0.01\\
64.06	0.01\\
64.07	0.01\\
64.08	0.01\\
64.09	0.01\\
64.1	0.01\\
64.11	0.01\\
64.12	0.01\\
64.13	0.01\\
64.14	0.01\\
64.15	0.01\\
64.16	0.01\\
64.17	0.01\\
64.18	0.01\\
64.19	0.01\\
64.2	0.01\\
64.21	0.01\\
64.22	0.01\\
64.23	0.01\\
64.24	0.01\\
64.25	0.01\\
64.26	0.01\\
64.27	0.01\\
64.28	0.01\\
64.29	0.01\\
64.3	0.01\\
64.31	0.01\\
64.32	0.01\\
64.33	0.01\\
64.34	0.01\\
64.35	0.01\\
64.36	0.01\\
64.37	0.01\\
64.38	0.01\\
64.39	0.01\\
64.4	0.01\\
64.41	0.01\\
64.42	0.01\\
64.43	0.01\\
64.44	0.01\\
64.45	0.01\\
64.46	0.01\\
64.47	0.01\\
64.48	0.01\\
64.49	0.01\\
64.5	0.01\\
64.51	0.01\\
64.52	0.01\\
64.53	0.01\\
64.54	0.01\\
64.55	0.01\\
64.56	0.01\\
64.57	0.01\\
64.58	0.01\\
64.59	0.01\\
64.6	0.01\\
64.61	0.01\\
64.62	0.01\\
64.63	0.01\\
64.64	0.01\\
64.65	0.01\\
64.66	0.01\\
64.67	0.01\\
64.68	0.01\\
64.69	0.01\\
64.7	0.01\\
64.71	0.01\\
64.72	0.01\\
64.73	0.01\\
64.74	0.01\\
64.75	0.01\\
64.76	0.01\\
64.77	0.01\\
64.78	0.01\\
64.79	0.01\\
64.8	0.01\\
64.81	0.01\\
64.82	0.01\\
64.83	0.01\\
64.84	0.01\\
64.85	0.01\\
64.86	0.01\\
64.87	0.01\\
64.88	0.01\\
64.89	0.01\\
64.9	0.01\\
64.91	0.01\\
64.92	0.01\\
64.93	0.01\\
64.94	0.01\\
64.95	0.01\\
64.96	0.01\\
64.97	0.01\\
64.98	0.01\\
64.99	0.01\\
65	0.01\\
65.01	0.01\\
65.02	0.01\\
65.03	0.01\\
65.04	0.01\\
65.05	0.01\\
65.06	0.01\\
65.07	0.01\\
65.08	0.01\\
65.09	0.01\\
65.1	0.01\\
65.11	0.01\\
65.12	0.01\\
65.13	0.01\\
65.14	0.01\\
65.15	0.01\\
65.16	0.01\\
65.17	0.01\\
65.18	0.01\\
65.19	0.01\\
65.2	0.01\\
65.21	0.01\\
65.22	0.01\\
65.23	0.01\\
65.24	0.01\\
65.25	0.01\\
65.26	0.01\\
65.27	0.01\\
65.28	0.01\\
65.29	0.01\\
65.3	0.01\\
65.31	0.01\\
65.32	0.01\\
65.33	0.01\\
65.34	0.01\\
65.35	0.01\\
65.36	0.01\\
65.37	0.01\\
65.38	0.01\\
65.39	0.01\\
65.4	0.01\\
65.41	0.01\\
65.42	0.01\\
65.43	0.01\\
65.44	0.01\\
65.45	0.01\\
65.46	0.01\\
65.47	0.01\\
65.48	0.01\\
65.49	0.01\\
65.5	0.01\\
65.51	0.01\\
65.52	0.01\\
65.53	0.01\\
65.54	0.01\\
65.55	0.01\\
65.56	0.01\\
65.57	0.01\\
65.58	0.01\\
65.59	0.01\\
65.6	0.01\\
65.61	0.01\\
65.62	0.01\\
65.63	0.01\\
65.64	0.01\\
65.65	0.01\\
65.66	0.01\\
65.67	0.01\\
65.68	0.01\\
65.69	0.01\\
65.7	0.01\\
65.71	0.01\\
65.72	0.01\\
65.73	0.01\\
65.74	0.01\\
65.75	0.01\\
65.76	0.01\\
65.77	0.01\\
65.78	0.01\\
65.79	0.01\\
65.8	0.01\\
65.81	0.01\\
65.82	0.01\\
65.83	0.01\\
65.84	0.01\\
65.85	0.01\\
65.86	0.01\\
65.87	0.01\\
65.88	0.01\\
65.89	0.01\\
65.9	0.01\\
65.91	0.01\\
65.92	0.01\\
65.93	0.01\\
65.94	0.01\\
65.95	0.01\\
65.96	0.01\\
65.97	0.01\\
65.98	0.01\\
65.99	0.01\\
66	0.01\\
66.01	0.01\\
66.02	0.01\\
66.03	0.01\\
66.04	0.01\\
66.05	0.01\\
66.06	0.01\\
66.07	0.01\\
66.08	0.01\\
66.09	0.01\\
66.1	0.01\\
66.11	0.01\\
66.12	0.01\\
66.13	0.01\\
66.14	0.01\\
66.15	0.01\\
66.16	0.01\\
66.17	0.01\\
66.18	0.01\\
66.19	0.01\\
66.2	0.01\\
66.21	0.01\\
66.22	0.01\\
66.23	0.01\\
66.24	0.01\\
66.25	0.01\\
66.26	0.01\\
66.27	0.01\\
66.28	0.01\\
66.29	0.01\\
66.3	0.01\\
66.31	0.01\\
66.32	0.01\\
66.33	0.01\\
66.34	0.01\\
66.35	0.01\\
66.36	0.01\\
66.37	0.01\\
66.38	0.01\\
66.39	0.01\\
66.4	0.01\\
66.41	0.01\\
66.42	0.01\\
66.43	0.01\\
66.44	0.01\\
66.45	0.01\\
66.46	0.01\\
66.47	0.01\\
66.48	0.01\\
66.49	0.01\\
66.5	0.01\\
66.51	0.01\\
66.52	0.01\\
66.53	0.01\\
66.54	0.01\\
66.55	0.01\\
66.56	0.01\\
66.57	0.01\\
66.58	0.01\\
66.59	0.01\\
66.6	0.01\\
66.61	0.01\\
66.62	0.01\\
66.63	0.01\\
66.64	0.01\\
66.65	0.01\\
66.66	0.01\\
66.67	0.01\\
66.68	0.01\\
66.69	0.01\\
66.7	0.01\\
66.71	0.01\\
66.72	0.01\\
66.73	0.01\\
66.74	0.01\\
66.75	0.01\\
66.76	0.01\\
66.77	0.01\\
66.78	0.01\\
66.79	0.01\\
66.8	0.01\\
66.81	0.01\\
66.82	0.01\\
66.83	0.01\\
66.84	0.01\\
66.85	0.01\\
66.86	0.01\\
66.87	0.01\\
66.88	0.01\\
66.89	0.01\\
66.9	0.01\\
66.91	0.01\\
66.92	0.01\\
66.93	0.01\\
66.94	0.01\\
66.95	0.01\\
66.96	0.01\\
66.97	0.01\\
66.98	0.01\\
66.99	0.01\\
67	0.01\\
67.01	0.01\\
67.02	0.01\\
67.03	0.01\\
67.04	0.01\\
67.05	0.01\\
67.06	0.01\\
67.07	0.01\\
67.08	0.01\\
67.09	0.01\\
67.1	0.01\\
67.11	0.01\\
67.12	0.01\\
67.13	0.01\\
67.14	0.01\\
67.15	0.01\\
67.16	0.01\\
67.17	0.01\\
67.18	0.01\\
67.19	0.01\\
67.2	0.01\\
67.21	0.01\\
67.22	0.01\\
67.23	0.01\\
67.24	0.01\\
67.25	0.01\\
67.26	0.01\\
67.27	0.01\\
67.28	0.01\\
67.29	0.01\\
67.3	0.01\\
67.31	0.01\\
67.32	0.01\\
67.33	0.01\\
67.34	0.01\\
67.35	0.01\\
67.36	0.01\\
67.37	0.01\\
67.38	0.01\\
67.39	0.01\\
67.4	0.01\\
67.41	0.01\\
67.42	0.01\\
67.43	0.01\\
67.44	0.01\\
67.45	0.01\\
67.46	0.01\\
67.47	0.01\\
67.48	0.01\\
67.49	0.01\\
67.5	0.01\\
67.51	0.01\\
67.52	0.01\\
67.53	0.01\\
67.54	0.01\\
67.55	0.01\\
67.56	0.01\\
67.57	0.01\\
67.58	0.01\\
67.59	0.01\\
67.6	0.01\\
67.61	0.01\\
67.62	0.01\\
67.63	0.01\\
67.64	0.01\\
67.65	0.01\\
67.66	0.01\\
67.67	0.01\\
67.68	0.01\\
67.69	0.01\\
67.7	0.01\\
67.71	0.01\\
67.72	0.01\\
67.73	0.01\\
67.74	0.01\\
67.75	0.01\\
67.76	0.01\\
67.77	0.01\\
67.78	0.01\\
67.79	0.01\\
67.8	0.01\\
67.81	0.01\\
67.82	0.01\\
67.83	0.01\\
67.84	0.01\\
67.85	0.01\\
67.86	0.01\\
67.87	0.01\\
67.88	0.01\\
67.89	0.01\\
67.9	0.01\\
67.91	0.01\\
67.92	0.01\\
67.93	0.01\\
67.94	0.01\\
67.95	0.01\\
67.96	0.01\\
67.97	0.01\\
67.98	0.01\\
67.99	0.01\\
68	0.01\\
68.01	0.01\\
68.02	0.01\\
68.03	0.01\\
68.04	0.01\\
68.05	0.01\\
68.06	0.01\\
68.07	0.01\\
68.08	0.01\\
68.09	0.01\\
68.1	0.01\\
68.11	0.01\\
68.12	0.01\\
68.13	0.01\\
68.14	0.01\\
68.15	0.01\\
68.16	0.01\\
68.17	0.01\\
68.18	0.01\\
68.19	0.01\\
68.2	0.01\\
68.21	0.01\\
68.22	0.01\\
68.23	0.01\\
68.24	0.01\\
68.25	0.01\\
68.26	0.01\\
68.27	0.01\\
68.28	0.01\\
68.29	0.01\\
68.3	0.01\\
68.31	0.01\\
68.32	0.01\\
68.33	0.01\\
68.34	0.01\\
68.35	0.01\\
68.36	0.01\\
68.37	0.01\\
68.38	0.01\\
68.39	0.01\\
68.4	0.01\\
68.41	0.01\\
68.42	0.01\\
68.43	0.01\\
68.44	0.01\\
68.45	0.01\\
68.46	0.01\\
68.47	0.01\\
68.48	0.01\\
68.49	0.01\\
68.5	0.01\\
68.51	0.01\\
68.52	0.01\\
68.53	0.01\\
68.54	0.01\\
68.55	0.01\\
68.56	0.01\\
68.57	0.01\\
68.58	0.01\\
68.59	0.01\\
68.6	0.01\\
68.61	0.01\\
68.62	0.01\\
68.63	0.01\\
68.64	0.01\\
68.65	0.01\\
68.66	0.01\\
68.67	0.01\\
68.68	0.01\\
68.69	0.01\\
68.7	0.01\\
68.71	0.01\\
68.72	0.01\\
68.73	0.01\\
68.74	0.01\\
68.75	0.01\\
68.76	0.01\\
68.77	0.01\\
68.78	0.01\\
68.79	0.01\\
68.8	0.01\\
68.81	0.01\\
68.82	0.01\\
68.83	0.01\\
68.84	0.01\\
68.85	0.01\\
68.86	0.01\\
68.87	0.01\\
68.88	0.01\\
68.89	0.01\\
68.9	0.01\\
68.91	0.01\\
68.92	0.01\\
68.93	0.01\\
68.94	0.01\\
68.95	0.01\\
68.96	0.01\\
68.97	0.01\\
68.98	0.01\\
68.99	0.01\\
69	0.01\\
69.01	0.01\\
69.02	0.01\\
69.03	0.01\\
69.04	0.01\\
69.05	0.01\\
69.06	0.01\\
69.07	0.01\\
69.08	0.01\\
69.09	0.01\\
69.1	0.01\\
69.11	0.01\\
69.12	0.01\\
69.13	0.01\\
69.14	0.01\\
69.15	0.01\\
69.16	0.01\\
69.17	0.01\\
69.18	0.01\\
69.19	0.01\\
69.2	0.01\\
69.21	0.01\\
69.22	0.01\\
69.23	0.01\\
69.24	0.01\\
69.25	0.01\\
69.26	0.01\\
69.27	0.01\\
69.28	0.01\\
69.29	0.01\\
69.3	0.01\\
69.31	0.01\\
69.32	0.01\\
69.33	0.01\\
69.34	0.01\\
69.35	0.01\\
69.36	0.01\\
69.37	0.01\\
69.38	0.01\\
69.39	0.01\\
69.4	0.01\\
69.41	0.01\\
69.42	0.01\\
69.43	0.01\\
69.44	0.01\\
69.45	0.01\\
69.46	0.01\\
69.47	0.01\\
69.48	0.01\\
69.49	0.01\\
69.5	0.01\\
69.51	0.01\\
69.52	0.01\\
69.53	0.01\\
69.54	0.01\\
69.55	0.01\\
69.56	0.01\\
69.57	0.01\\
69.58	0.01\\
69.59	0.01\\
69.6	0.01\\
69.61	0.01\\
69.62	0.01\\
69.63	0.01\\
69.64	0.01\\
69.65	0.01\\
69.66	0.01\\
69.67	0.01\\
69.68	0.01\\
69.69	0.01\\
69.7	0.01\\
69.71	0.01\\
69.72	0.01\\
69.73	0.01\\
69.74	0.01\\
69.75	0.01\\
69.76	0.01\\
69.77	0.01\\
69.78	0.01\\
69.79	0.01\\
69.8	0.01\\
69.81	0.01\\
69.82	0.01\\
69.83	0.01\\
69.84	0.01\\
69.85	0.01\\
69.86	0.01\\
69.87	0.01\\
69.88	0.01\\
69.89	0.01\\
69.9	0.01\\
69.91	0.01\\
69.92	0.01\\
69.93	0.01\\
69.94	0.01\\
69.95	0.01\\
69.96	0.01\\
69.97	0.01\\
69.98	0.01\\
69.99	0.01\\
70	0.01\\
70.01	0.01\\
70.02	0.01\\
70.03	0.01\\
70.04	0.01\\
70.05	0.01\\
70.06	0.01\\
70.07	0.01\\
70.08	0.01\\
70.09	0.01\\
70.1	0.01\\
70.11	0.01\\
70.12	0.01\\
70.13	0.01\\
70.14	0.01\\
70.15	0.01\\
70.16	0.01\\
70.17	0.01\\
70.18	0.01\\
70.19	0.01\\
70.2	0.01\\
70.21	0.01\\
70.22	0.01\\
70.23	0.01\\
70.24	0.01\\
70.25	0.01\\
70.26	0.01\\
70.27	0.01\\
70.28	0.01\\
70.29	0.01\\
70.3	0.01\\
70.31	0.01\\
70.32	0.01\\
70.33	0.01\\
70.34	0.01\\
70.35	0.01\\
70.36	0.01\\
70.37	0.01\\
70.38	0.01\\
70.39	0.01\\
70.4	0.01\\
70.41	0.01\\
70.42	0.01\\
70.43	0.01\\
70.44	0.01\\
70.45	0.01\\
70.46	0.01\\
70.47	0.01\\
70.48	0.01\\
70.49	0.01\\
70.5	0.01\\
70.51	0.01\\
70.52	0.01\\
70.53	0.01\\
70.54	0.01\\
70.55	0.01\\
70.56	0.01\\
70.57	0.01\\
70.58	0.01\\
70.59	0.01\\
70.6	0.01\\
70.61	0.01\\
70.62	0.01\\
70.63	0.01\\
70.64	0.01\\
70.65	0.01\\
70.66	0.01\\
70.67	0.01\\
70.68	0.01\\
70.69	0.01\\
70.7	0.01\\
70.71	0.01\\
70.72	0.01\\
70.73	0.01\\
70.74	0.01\\
70.75	0.01\\
70.76	0.01\\
70.77	0.01\\
70.78	0.01\\
70.79	0.01\\
70.8	0.01\\
70.81	0.01\\
70.82	0.01\\
70.83	0.01\\
70.84	0.01\\
70.85	0.01\\
70.86	0.01\\
70.87	0.01\\
70.88	0.01\\
70.89	0.01\\
70.9	0.01\\
70.91	0.01\\
70.92	0.01\\
70.93	0.01\\
70.94	0.01\\
70.95	0.01\\
70.96	0.01\\
70.97	0.01\\
70.98	0.01\\
70.99	0.01\\
71	0.01\\
71.01	0.01\\
71.02	0.01\\
71.03	0.01\\
71.04	0.01\\
71.05	0.01\\
71.06	0.01\\
71.07	0.01\\
71.08	0.01\\
71.09	0.01\\
71.1	0.01\\
71.11	0.01\\
71.12	0.01\\
71.13	0.01\\
71.14	0.01\\
71.15	0.01\\
71.16	0.01\\
71.17	0.01\\
71.18	0.01\\
71.19	0.01\\
71.2	0.01\\
71.21	0.01\\
71.22	0.01\\
71.23	0.01\\
71.24	0.01\\
71.25	0.01\\
71.26	0.01\\
71.27	0.01\\
71.28	0.01\\
71.29	0.01\\
71.3	0.01\\
71.31	0.01\\
71.32	0.01\\
71.33	0.01\\
71.34	0.01\\
71.35	0.01\\
71.36	0.01\\
71.37	0.01\\
71.38	0.01\\
71.39	0.01\\
71.4	0.01\\
71.41	0.01\\
71.42	0.01\\
71.43	0.01\\
71.44	0.01\\
71.45	0.01\\
71.46	0.01\\
71.47	0.01\\
71.48	0.01\\
71.49	0.01\\
71.5	0.01\\
71.51	0.01\\
71.52	0.01\\
71.53	0.01\\
71.54	0.01\\
71.55	0.01\\
71.56	0.01\\
71.57	0.01\\
71.58	0.01\\
71.59	0.01\\
71.6	0.01\\
71.61	0.01\\
71.62	0.01\\
71.63	0.01\\
71.64	0.01\\
71.65	0.01\\
71.66	0.01\\
71.67	0.01\\
71.68	0.01\\
71.69	0.01\\
71.7	0.01\\
71.71	0.01\\
71.72	0.01\\
71.73	0.01\\
71.74	0.01\\
71.75	0.01\\
71.76	0.01\\
71.77	0.01\\
71.78	0.01\\
71.79	0.01\\
71.8	0.01\\
71.81	0.01\\
71.82	0.01\\
71.83	0.01\\
71.84	0.01\\
71.85	0.01\\
71.86	0.01\\
71.87	0.01\\
71.88	0.01\\
71.89	0.01\\
71.9	0.01\\
71.91	0.01\\
71.92	0.01\\
71.93	0.01\\
71.94	0.01\\
71.95	0.01\\
71.96	0.01\\
71.97	0.01\\
71.98	0.01\\
71.99	0.01\\
72	0.01\\
72.01	0.01\\
72.02	0.01\\
72.03	0.01\\
72.04	0.01\\
72.05	0.01\\
72.06	0.01\\
72.07	0.01\\
72.08	0.01\\
72.09	0.01\\
72.1	0.01\\
72.11	0.01\\
72.12	0.01\\
72.13	0.01\\
72.14	0.01\\
72.15	0.01\\
72.16	0.01\\
72.17	0.01\\
72.18	0.01\\
72.19	0.01\\
72.2	0.01\\
72.21	0.01\\
72.22	0.01\\
72.23	0.01\\
72.24	0.01\\
72.25	0.01\\
72.26	0.01\\
72.27	0.01\\
72.28	0.01\\
72.29	0.01\\
72.3	0.01\\
72.31	0.01\\
72.32	0.01\\
72.33	0.01\\
72.34	0.01\\
72.35	0.01\\
72.36	0.01\\
72.37	0.01\\
72.38	0.01\\
72.39	0.01\\
72.4	0.01\\
72.41	0.01\\
72.42	0.01\\
72.43	0.01\\
72.44	0.01\\
72.45	0.01\\
72.46	0.01\\
72.47	0.01\\
72.48	0.01\\
72.49	0.01\\
72.5	0.01\\
72.51	0.01\\
72.52	0.01\\
72.53	0.01\\
72.54	0.01\\
72.55	0.01\\
72.56	0.01\\
72.57	0.01\\
72.58	0.01\\
72.59	0.01\\
72.6	0.01\\
72.61	0.01\\
72.62	0.01\\
72.63	0.01\\
72.64	0.01\\
72.65	0.01\\
72.66	0.01\\
72.67	0.01\\
72.68	0.01\\
72.69	0.01\\
72.7	0.01\\
72.71	0.01\\
72.72	0.01\\
72.73	0.01\\
72.74	0.01\\
72.75	0.01\\
72.76	0.01\\
72.77	0.01\\
72.78	0.01\\
72.79	0.01\\
72.8	0.01\\
72.81	0.01\\
72.82	0.01\\
72.83	0.01\\
72.84	0.01\\
72.85	0.01\\
72.86	0.01\\
72.87	0.01\\
72.88	0.01\\
72.89	0.01\\
72.9	0.01\\
72.91	0.01\\
72.92	0.01\\
72.93	0.01\\
72.94	0.01\\
72.95	0.01\\
72.96	0.01\\
72.97	0.01\\
72.98	0.01\\
72.99	0.01\\
73	0.01\\
73.01	0.01\\
73.02	0.01\\
73.03	0.01\\
73.04	0.01\\
73.05	0.01\\
73.06	0.01\\
73.07	0.01\\
73.08	0.01\\
73.09	0.01\\
73.1	0.01\\
73.11	0.01\\
73.12	0.01\\
73.13	0.01\\
73.14	0.01\\
73.15	0.01\\
73.16	0.01\\
73.17	0.01\\
73.18	0.01\\
73.19	0.01\\
73.2	0.01\\
73.21	0.01\\
73.22	0.01\\
73.23	0.01\\
73.24	0.01\\
73.25	0.01\\
73.26	0.01\\
73.27	0.01\\
73.28	0.01\\
73.29	0.01\\
73.3	0.01\\
73.31	0.01\\
73.32	0.01\\
73.33	0.01\\
73.34	0.01\\
73.35	0.01\\
73.36	0.01\\
73.37	0.01\\
73.38	0.01\\
73.39	0.01\\
73.4	0.01\\
73.41	0.01\\
73.42	0.01\\
73.43	0.01\\
73.44	0.01\\
73.45	0.01\\
73.46	0.01\\
73.47	0.01\\
73.48	0.01\\
73.49	0.01\\
73.5	0.01\\
73.51	0.01\\
73.52	0.01\\
73.53	0.01\\
73.54	0.01\\
73.55	0.01\\
73.56	0.01\\
73.57	0.01\\
73.58	0.01\\
73.59	0.01\\
73.6	0.01\\
73.61	0.01\\
73.62	0.01\\
73.63	0.01\\
73.64	0.01\\
73.65	0.01\\
73.66	0.01\\
73.67	0.01\\
73.68	0.01\\
73.69	0.01\\
73.7	0.01\\
73.71	0.01\\
73.72	0.01\\
73.73	0.01\\
73.74	0.01\\
73.75	0.01\\
73.76	0.01\\
73.77	0.01\\
73.78	0.01\\
73.79	0.01\\
73.8	0.01\\
73.81	0.01\\
73.82	0.01\\
73.83	0.01\\
73.84	0.01\\
73.85	0.01\\
73.86	0.01\\
73.87	0.01\\
73.88	0.01\\
73.89	0.01\\
73.9	0.01\\
73.91	0.01\\
73.92	0.01\\
73.93	0.01\\
73.94	0.01\\
73.95	0.01\\
73.96	0.01\\
73.97	0.01\\
73.98	0.01\\
73.99	0.01\\
74	0.01\\
74.01	0.01\\
74.02	0.01\\
74.03	0.01\\
74.04	0.01\\
74.05	0.01\\
74.06	0.01\\
74.07	0.01\\
74.08	0.01\\
74.09	0.01\\
74.1	0.01\\
74.11	0.01\\
74.12	0.01\\
74.13	0.01\\
74.14	0.01\\
74.15	0.01\\
74.16	0.01\\
74.17	0.01\\
74.18	0.01\\
74.19	0.01\\
74.2	0.01\\
74.21	0.01\\
74.22	0.01\\
74.23	0.01\\
74.24	0.01\\
74.25	0.01\\
74.26	0.01\\
74.27	0.01\\
74.28	0.01\\
74.29	0.01\\
74.3	0.01\\
74.31	0.01\\
74.32	0.01\\
74.33	0.01\\
74.34	0.01\\
74.35	0.01\\
74.36	0.01\\
74.37	0.01\\
74.38	0.01\\
74.39	0.01\\
74.4	0.01\\
74.41	0.01\\
74.42	0.01\\
74.43	0.01\\
74.44	0.01\\
74.45	0.01\\
74.46	0.01\\
74.47	0.01\\
74.48	0.01\\
74.49	0.01\\
74.5	0.01\\
74.51	0.01\\
74.52	0.01\\
74.53	0.01\\
74.54	0.01\\
74.55	0.01\\
74.56	0.01\\
74.57	0.01\\
74.58	0.01\\
74.59	0.01\\
74.6	0.01\\
74.61	0.01\\
74.62	0.01\\
74.63	0.01\\
74.64	0.01\\
74.65	0.01\\
74.66	0.01\\
74.67	0.01\\
74.68	0.01\\
74.69	0.01\\
74.7	0.01\\
74.71	0.01\\
74.72	0.01\\
74.73	0.01\\
74.74	0.01\\
74.75	0.01\\
74.76	0.01\\
74.77	0.01\\
74.78	0.01\\
74.79	0.01\\
74.8	0.01\\
74.81	0.01\\
74.82	0.01\\
74.83	0.01\\
74.84	0.01\\
74.85	0.01\\
74.86	0.01\\
74.87	0.01\\
74.88	0.01\\
74.89	0.01\\
74.9	0.01\\
74.91	0.01\\
74.92	0.01\\
74.93	0.01\\
74.94	0.01\\
74.95	0.01\\
74.96	0.01\\
74.97	0.01\\
74.98	0.01\\
74.99	0.01\\
75	0.01\\
75.01	0.01\\
75.02	0.01\\
75.03	0.01\\
75.04	0.01\\
75.05	0.01\\
75.06	0.01\\
75.07	0.01\\
75.08	0.01\\
75.09	0.01\\
75.1	0.01\\
75.11	0.01\\
75.12	0.01\\
75.13	0.01\\
75.14	0.01\\
75.15	0.01\\
75.16	0.01\\
75.17	0.01\\
75.18	0.01\\
75.19	0.01\\
75.2	0.01\\
75.21	0.01\\
75.22	0.01\\
75.23	0.01\\
75.24	0.01\\
75.25	0.01\\
75.26	0.01\\
75.27	0.01\\
75.28	0.01\\
75.29	0.01\\
75.3	0.01\\
75.31	0.01\\
75.32	0.01\\
75.33	0.01\\
75.34	0.01\\
75.35	0.01\\
75.36	0.01\\
75.37	0.01\\
75.38	0.01\\
75.39	0.01\\
75.4	0.01\\
75.41	0.01\\
75.42	0.01\\
75.43	0.01\\
75.44	0.01\\
75.45	0.01\\
75.46	0.01\\
75.47	0.01\\
75.48	0.01\\
75.49	0.01\\
75.5	0.01\\
75.51	0.01\\
75.52	0.01\\
75.53	0.01\\
75.54	0.01\\
75.55	0.01\\
75.56	0.01\\
75.57	0.01\\
75.58	0.01\\
75.59	0.01\\
75.6	0.01\\
75.61	0.01\\
75.62	0.01\\
75.63	0.01\\
75.64	0.01\\
75.65	0.01\\
75.66	0.01\\
75.67	0.01\\
75.68	0.01\\
75.69	0.01\\
75.7	0.01\\
75.71	0.01\\
75.72	0.01\\
75.73	0.01\\
75.74	0.01\\
75.75	0.01\\
75.76	0.01\\
75.77	0.01\\
75.78	0.01\\
75.79	0.01\\
75.8	0.01\\
75.81	0.01\\
75.82	0.01\\
75.83	0.01\\
75.84	0.01\\
75.85	0.01\\
75.86	0.01\\
75.87	0.01\\
75.88	0.01\\
75.89	0.01\\
75.9	0.01\\
75.91	0.01\\
75.92	0.01\\
75.93	0.01\\
75.94	0.01\\
75.95	0.01\\
75.96	0.01\\
75.97	0.01\\
75.98	0.01\\
75.99	0.01\\
76	0.01\\
76.01	0.01\\
76.02	0.01\\
76.03	0.01\\
76.04	0.01\\
76.05	0.01\\
76.06	0.01\\
76.07	0.01\\
76.08	0.01\\
76.09	0.01\\
76.1	0.01\\
76.11	0.01\\
76.12	0.01\\
76.13	0.01\\
76.14	0.01\\
76.15	0.01\\
76.16	0.01\\
76.17	0.01\\
76.18	0.01\\
76.19	0.01\\
76.2	0.01\\
76.21	0.01\\
76.22	0.01\\
76.23	0.01\\
76.24	0.01\\
76.25	0.01\\
76.26	0.01\\
76.27	0.01\\
76.28	0.01\\
76.29	0.01\\
76.3	0.01\\
76.31	0.01\\
76.32	0.01\\
76.33	0.01\\
76.34	0.01\\
76.35	0.01\\
76.36	0.01\\
76.37	0.01\\
76.38	0.01\\
76.39	0.01\\
76.4	0.01\\
76.41	0.01\\
76.42	0.01\\
76.43	0.01\\
76.44	0.01\\
76.45	0.01\\
76.46	0.01\\
76.47	0.01\\
76.48	0.01\\
76.49	0.01\\
76.5	0.01\\
76.51	0.01\\
76.52	0.01\\
76.53	0.01\\
76.54	0.01\\
76.55	0.01\\
76.56	0.01\\
76.57	0.01\\
76.58	0.01\\
76.59	0.01\\
76.6	0.01\\
76.61	0.01\\
76.62	0.01\\
76.63	0.01\\
76.64	0.01\\
76.65	0.01\\
76.66	0.01\\
76.67	0.01\\
76.68	0.01\\
76.69	0.01\\
76.7	0.01\\
76.71	0.01\\
76.72	0.01\\
76.73	0.01\\
76.74	0.01\\
76.75	0.01\\
76.76	0.01\\
76.77	0.01\\
76.78	0.01\\
76.79	0.01\\
76.8	0.01\\
76.81	0.01\\
76.82	0.01\\
76.83	0.01\\
76.84	0.01\\
76.85	0.01\\
76.86	0.01\\
76.87	0.01\\
76.88	0.01\\
76.89	0.01\\
76.9	0.01\\
76.91	0.01\\
76.92	0.01\\
76.93	0.01\\
76.94	0.01\\
76.95	0.01\\
76.96	0.01\\
76.97	0.01\\
76.98	0.01\\
76.99	0.01\\
77	0.01\\
77.01	0.01\\
77.02	0.01\\
77.03	0.01\\
77.04	0.01\\
77.05	0.01\\
77.06	0.01\\
77.07	0.01\\
77.08	0.01\\
77.09	0.01\\
77.1	0.01\\
77.11	0.01\\
77.12	0.01\\
77.13	0.01\\
77.14	0.01\\
77.15	0.01\\
77.16	0.01\\
77.17	0.01\\
77.18	0.01\\
77.19	0.01\\
77.2	0.01\\
77.21	0.01\\
77.22	0.01\\
77.23	0.01\\
77.24	0.01\\
77.25	0.01\\
77.26	0.01\\
77.27	0.01\\
77.28	0.01\\
77.29	0.01\\
77.3	0.01\\
77.31	0.01\\
77.32	0.01\\
77.33	0.01\\
77.34	0.01\\
77.35	0.01\\
77.36	0.01\\
77.37	0.01\\
77.38	0.01\\
77.39	0.01\\
77.4	0.01\\
77.41	0.01\\
77.42	0.01\\
77.43	0.01\\
77.44	0.01\\
77.45	0.01\\
77.46	0.01\\
77.47	0.01\\
77.48	0.01\\
77.49	0.01\\
77.5	0.01\\
77.51	0.01\\
77.52	0.01\\
77.53	0.01\\
77.54	0.01\\
77.55	0.01\\
77.56	0.01\\
77.57	0.01\\
77.58	0.01\\
77.59	0.01\\
77.6	0.01\\
77.61	0.01\\
77.62	0.01\\
77.63	0.01\\
77.64	0.01\\
77.65	0.01\\
77.66	0.01\\
77.67	0.01\\
77.68	0.01\\
77.69	0.01\\
77.7	0.01\\
77.71	0.01\\
77.72	0.01\\
77.73	0.01\\
77.74	0.01\\
77.75	0.01\\
77.76	0.01\\
77.77	0.01\\
77.78	0.01\\
77.79	0.01\\
77.8	0.01\\
77.81	0.01\\
77.82	0.01\\
77.83	0.01\\
77.84	0.01\\
77.85	0.01\\
77.86	0.01\\
77.87	0.01\\
77.88	0.01\\
77.89	0.01\\
77.9	0.01\\
77.91	0.01\\
77.92	0.01\\
77.93	0.01\\
77.94	0.01\\
77.95	0.01\\
77.96	0.01\\
77.97	0.01\\
77.98	0.01\\
77.99	0.01\\
78	0.01\\
78.01	0.01\\
78.02	0.01\\
78.03	0.01\\
78.04	0.01\\
78.05	0.01\\
78.06	0.01\\
78.07	0.01\\
78.08	0.01\\
78.09	0.01\\
78.1	0.01\\
78.11	0.01\\
78.12	0.01\\
78.13	0.01\\
78.14	0.01\\
78.15	0.01\\
78.16	0.01\\
78.17	0.01\\
78.18	0.01\\
78.19	0.01\\
78.2	0.01\\
78.21	0.01\\
78.22	0.01\\
78.23	0.01\\
78.24	0.01\\
78.25	0.01\\
78.26	0.01\\
78.27	0.01\\
78.28	0.01\\
78.29	0.01\\
78.3	0.01\\
78.31	0.01\\
78.32	0.01\\
78.33	0.01\\
78.34	0.01\\
78.35	0.01\\
78.36	0.01\\
78.37	0.01\\
78.38	0.01\\
78.39	0.01\\
78.4	0.01\\
78.41	0.01\\
78.42	0.01\\
78.43	0.01\\
78.44	0.01\\
78.45	0.01\\
78.46	0.01\\
78.47	0.01\\
78.48	0.01\\
78.49	0.01\\
78.5	0.01\\
78.51	0.01\\
78.52	0.01\\
78.53	0.01\\
78.54	0.01\\
78.55	0.01\\
78.56	0.01\\
78.57	0.01\\
78.58	0.01\\
78.59	0.01\\
78.6	0.01\\
78.61	0.01\\
78.62	0.01\\
78.63	0.01\\
78.64	0.01\\
78.65	0.01\\
78.66	0.01\\
78.67	0.01\\
78.68	0.01\\
78.69	0.01\\
78.7	0.01\\
78.71	0.01\\
78.72	0.01\\
78.73	0.01\\
78.74	0.01\\
78.75	0.01\\
78.76	0.01\\
78.77	0.01\\
78.78	0.01\\
78.79	0.01\\
78.8	0.01\\
78.81	0.01\\
78.82	0.01\\
78.83	0.01\\
78.84	0.01\\
78.85	0.01\\
78.86	0.01\\
78.87	0.01\\
78.88	0.01\\
78.89	0.01\\
78.9	0.01\\
78.91	0.01\\
78.92	0.01\\
78.93	0.01\\
78.94	0.01\\
78.95	0.01\\
78.96	0.01\\
78.97	0.01\\
78.98	0.01\\
78.99	0.01\\
79	0.01\\
79.01	0.01\\
79.02	0.01\\
79.03	0.01\\
79.04	0.01\\
79.05	0.01\\
79.06	0.01\\
79.07	0.01\\
79.08	0.01\\
79.09	0.01\\
79.1	0.01\\
79.11	0.01\\
79.12	0.01\\
79.13	0.01\\
79.14	0.01\\
79.15	0.01\\
79.16	0.01\\
79.17	0.01\\
79.18	0.01\\
79.19	0.01\\
79.2	0.01\\
79.21	0.01\\
79.22	0.01\\
79.23	0.01\\
79.24	0.01\\
79.25	0.01\\
79.26	0.01\\
79.27	0.01\\
79.28	0.01\\
79.29	0.01\\
79.3	0.01\\
79.31	0.01\\
79.32	0.01\\
79.33	0.01\\
79.34	0.01\\
79.35	0.01\\
79.36	0.01\\
79.37	0.01\\
79.38	0.01\\
79.39	0.01\\
79.4	0.01\\
79.41	0.01\\
79.42	0.01\\
79.43	0.01\\
79.44	0.01\\
79.45	0.01\\
79.46	0.01\\
79.47	0.01\\
79.48	0.01\\
79.49	0.01\\
79.5	0.01\\
79.51	0.01\\
79.52	0.01\\
79.53	0.01\\
79.54	0.01\\
79.55	0.01\\
79.56	0.01\\
79.57	0.01\\
79.58	0.01\\
79.59	0.01\\
79.6	0.01\\
79.61	0.01\\
79.62	0.01\\
79.63	0.01\\
79.64	0.01\\
79.65	0.01\\
79.66	0.01\\
79.67	0.01\\
79.68	0.01\\
79.69	0.01\\
79.7	0.01\\
79.71	0.01\\
79.72	0.01\\
79.73	0.01\\
79.74	0.01\\
79.75	0.01\\
79.76	0.01\\
79.77	0.01\\
79.78	0.01\\
79.79	0.01\\
79.8	0.01\\
79.81	0.01\\
79.82	0.01\\
79.83	0.01\\
79.84	0.01\\
79.85	0.01\\
79.86	0.01\\
79.87	0.01\\
79.88	0.01\\
79.89	0.01\\
79.9	0.01\\
79.91	0.01\\
79.92	0.01\\
79.93	0.01\\
79.94	0.01\\
79.95	0.01\\
79.96	0.01\\
79.97	0.01\\
79.98	0.01\\
79.99	0.01\\
80	0.01\\
80.01	0.01\\
};
\addplot [color=blue,dashed]
  table[row sep=crcr]{%
80.01	0.01\\
80.02	0.01\\
80.03	0.01\\
80.04	0.01\\
80.05	0.01\\
80.06	0.01\\
80.07	0.01\\
80.08	0.01\\
80.09	0.01\\
80.1	0.01\\
80.11	0.01\\
80.12	0.01\\
80.13	0.01\\
80.14	0.01\\
80.15	0.01\\
80.16	0.01\\
80.17	0.01\\
80.18	0.01\\
80.19	0.01\\
80.2	0.01\\
80.21	0.01\\
80.22	0.01\\
80.23	0.01\\
80.24	0.01\\
80.25	0.01\\
80.26	0.01\\
80.27	0.01\\
80.28	0.01\\
80.29	0.01\\
80.3	0.01\\
80.31	0.01\\
80.32	0.01\\
80.33	0.01\\
80.34	0.01\\
80.35	0.01\\
80.36	0.01\\
80.37	0.01\\
80.38	0.01\\
80.39	0.01\\
80.4	0.01\\
80.41	0.01\\
80.42	0.01\\
80.43	0.01\\
80.44	0.01\\
80.45	0.01\\
80.46	0.01\\
80.47	0.01\\
80.48	0.01\\
80.49	0.01\\
80.5	0.01\\
80.51	0.01\\
80.52	0.01\\
80.53	0.01\\
80.54	0.01\\
80.55	0.01\\
80.56	0.01\\
80.57	0.01\\
80.58	0.01\\
80.59	0.01\\
80.6	0.01\\
80.61	0.01\\
80.62	0.01\\
80.63	0.01\\
80.64	0.01\\
80.65	0.01\\
80.66	0.01\\
80.67	0.01\\
80.68	0.01\\
80.69	0.01\\
80.7	0.01\\
80.71	0.01\\
80.72	0.01\\
80.73	0.01\\
80.74	0.01\\
80.75	0.01\\
80.76	0.01\\
80.77	0.01\\
80.78	0.01\\
80.79	0.01\\
80.8	0.01\\
80.81	0.01\\
80.82	0.01\\
80.83	0.01\\
80.84	0.01\\
80.85	0.01\\
80.86	0.01\\
80.87	0.01\\
80.88	0.01\\
80.89	0.01\\
80.9	0.01\\
80.91	0.01\\
80.92	0.01\\
80.93	0.01\\
80.94	0.01\\
80.95	0.01\\
80.96	0.01\\
80.97	0.01\\
80.98	0.01\\
80.99	0.01\\
81	0.01\\
81.01	0.01\\
81.02	0.01\\
81.03	0.01\\
81.04	0.01\\
81.05	0.01\\
81.06	0.01\\
81.07	0.01\\
81.08	0.01\\
81.09	0.01\\
81.1	0.01\\
81.11	0.01\\
81.12	0.01\\
81.13	0.01\\
81.14	0.01\\
81.15	0.01\\
81.16	0.01\\
81.17	0.01\\
81.18	0.01\\
81.19	0.01\\
81.2	0.01\\
81.21	0.01\\
81.22	0.01\\
81.23	0.01\\
81.24	0.01\\
81.25	0.01\\
81.26	0.01\\
81.27	0.01\\
81.28	0.01\\
81.29	0.01\\
81.3	0.01\\
81.31	0.01\\
81.32	0.01\\
81.33	0.01\\
81.34	0.01\\
81.35	0.01\\
81.36	0.01\\
81.37	0.01\\
81.38	0.01\\
81.39	0.01\\
81.4	0.01\\
81.41	0.01\\
81.42	0.01\\
81.43	0.01\\
81.44	0.01\\
81.45	0.01\\
81.46	0.01\\
81.47	0.01\\
81.48	0.01\\
81.49	0.01\\
81.5	0.01\\
81.51	0.01\\
81.52	0.01\\
81.53	0.01\\
81.54	0.01\\
81.55	0.01\\
81.56	0.01\\
81.57	0.01\\
81.58	0.01\\
81.59	0.01\\
81.6	0.01\\
81.61	0.01\\
81.62	0.01\\
81.63	0.01\\
81.64	0.01\\
81.65	0.01\\
81.66	0.01\\
81.67	0.01\\
81.68	0.01\\
81.69	0.01\\
81.7	0.01\\
81.71	0.01\\
81.72	0.01\\
81.73	0.01\\
81.74	0.01\\
81.75	0.01\\
81.76	0.01\\
81.77	0.01\\
81.78	0.01\\
81.79	0.01\\
81.8	0.01\\
81.81	0.01\\
81.82	0.01\\
81.83	0.01\\
81.84	0.01\\
81.85	0.01\\
81.86	0.01\\
81.87	0.01\\
81.88	0.01\\
81.89	0.01\\
81.9	0.01\\
81.91	0.01\\
81.92	0.01\\
81.93	0.01\\
81.94	0.01\\
81.95	0.01\\
81.96	0.01\\
81.97	0.01\\
81.98	0.01\\
81.99	0.01\\
82	0.01\\
82.01	0.01\\
82.02	0.01\\
82.03	0.01\\
82.04	0.01\\
82.05	0.01\\
82.06	0.01\\
82.07	0.01\\
82.08	0.01\\
82.09	0.01\\
82.1	0.01\\
82.11	0.01\\
82.12	0.01\\
82.13	0.01\\
82.14	0.01\\
82.15	0.01\\
82.16	0.01\\
82.17	0.01\\
82.18	0.01\\
82.19	0.01\\
82.2	0.01\\
82.21	0.01\\
82.22	0.01\\
82.23	0.01\\
82.24	0.01\\
82.25	0.01\\
82.26	0.01\\
82.27	0.01\\
82.28	0.01\\
82.29	0.01\\
82.3	0.01\\
82.31	0.01\\
82.32	0.01\\
82.33	0.01\\
82.34	0.01\\
82.35	0.01\\
82.36	0.01\\
82.37	0.01\\
82.38	0.01\\
82.39	0.01\\
82.4	0.01\\
82.41	0.01\\
82.42	0.01\\
82.43	0.01\\
82.44	0.01\\
82.45	0.01\\
82.46	0.01\\
82.47	0.01\\
82.48	0.01\\
82.49	0.01\\
82.5	0.01\\
82.51	0.01\\
82.52	0.01\\
82.53	0.01\\
82.54	0.01\\
82.55	0.01\\
82.56	0.01\\
82.57	0.01\\
82.58	0.01\\
82.59	0.01\\
82.6	0.01\\
82.61	0.01\\
82.62	0.01\\
82.63	0.01\\
82.64	0.01\\
82.65	0.01\\
82.66	0.01\\
82.67	0.01\\
82.68	0.01\\
82.69	0.01\\
82.7	0.01\\
82.71	0.01\\
82.72	0.01\\
82.73	0.01\\
82.74	0.01\\
82.75	0.01\\
82.76	0.01\\
82.77	0.01\\
82.78	0.01\\
82.79	0.01\\
82.8	0.01\\
82.81	0.01\\
82.82	0.01\\
82.83	0.01\\
82.84	0.01\\
82.85	0.01\\
82.86	0.01\\
82.87	0.01\\
82.88	0.01\\
82.89	0.01\\
82.9	0.01\\
82.91	0.01\\
82.92	0.01\\
82.93	0.01\\
82.94	0.01\\
82.95	0.01\\
82.96	0.01\\
82.97	0.01\\
82.98	0.01\\
82.99	0.01\\
83	0.01\\
83.01	0.01\\
83.02	0.01\\
83.03	0.01\\
83.04	0.01\\
83.05	0.01\\
83.06	0.01\\
83.07	0.01\\
83.08	0.01\\
83.09	0.01\\
83.1	0.01\\
83.11	0.01\\
83.12	0.01\\
83.13	0.01\\
83.14	0.01\\
83.15	0.01\\
83.16	0.01\\
83.17	0.01\\
83.18	0.01\\
83.19	0.01\\
83.2	0.01\\
83.21	0.01\\
83.22	0.01\\
83.23	0.01\\
83.24	0.01\\
83.25	0.01\\
83.26	0.01\\
83.27	0.01\\
83.28	0.01\\
83.29	0.01\\
83.3	0.01\\
83.31	0.01\\
83.32	0.01\\
83.33	0.01\\
83.34	0.01\\
83.35	0.01\\
83.36	0.01\\
83.37	0.01\\
83.38	0.01\\
83.39	0.01\\
83.4	0.01\\
83.41	0.01\\
83.42	0.01\\
83.43	0.01\\
83.44	0.01\\
83.45	0.01\\
83.46	0.01\\
83.47	0.01\\
83.48	0.01\\
83.49	0.01\\
83.5	0.01\\
83.51	0.01\\
83.52	0.01\\
83.53	0.01\\
83.54	0.01\\
83.55	0.01\\
83.56	0.01\\
83.57	0.01\\
83.58	0.01\\
83.59	0.01\\
83.6	0.01\\
83.61	0.01\\
83.62	0.01\\
83.63	0.01\\
83.64	0.01\\
83.65	0.01\\
83.66	0.01\\
83.67	0.01\\
83.68	0.01\\
83.69	0.01\\
83.7	0.01\\
83.71	0.01\\
83.72	0.01\\
83.73	0.01\\
83.74	0.01\\
83.75	0.01\\
83.76	0.01\\
83.77	0.01\\
83.78	0.01\\
83.79	0.01\\
83.8	0.01\\
83.81	0.01\\
83.82	0.01\\
83.83	0.01\\
83.84	0.01\\
83.85	0.01\\
83.86	0.01\\
83.87	0.01\\
83.88	0.01\\
83.89	0.01\\
83.9	0.01\\
83.91	0.01\\
83.92	0.01\\
83.93	0.01\\
83.94	0.01\\
83.95	0.01\\
83.96	0.01\\
83.97	0.01\\
83.98	0.01\\
83.99	0.01\\
84	0.01\\
84.01	0.01\\
84.02	0.01\\
84.03	0.01\\
84.04	0.01\\
84.05	0.01\\
84.06	0.01\\
84.07	0.01\\
84.08	0.01\\
84.09	0.01\\
84.1	0.01\\
84.11	0.01\\
84.12	0.01\\
84.13	0.01\\
84.14	0.01\\
84.15	0.01\\
84.16	0.01\\
84.17	0.01\\
84.18	0.01\\
84.19	0.01\\
84.2	0.01\\
84.21	0.01\\
84.22	0.01\\
84.23	0.01\\
84.24	0.01\\
84.25	0.01\\
84.26	0.01\\
84.27	0.01\\
84.28	0.01\\
84.29	0.01\\
84.3	0.01\\
84.31	0.01\\
84.32	0.01\\
84.33	0.01\\
84.34	0.01\\
84.35	0.01\\
84.36	0.01\\
84.37	0.01\\
84.38	0.01\\
84.39	0.01\\
84.4	0.01\\
84.41	0.01\\
84.42	0.01\\
84.43	0.01\\
84.44	0.01\\
84.45	0.01\\
84.46	0.01\\
84.47	0.01\\
84.48	0.01\\
84.49	0.01\\
84.5	0.01\\
84.51	0.01\\
84.52	0.01\\
84.53	0.01\\
84.54	0.01\\
84.55	0.01\\
84.56	0.01\\
84.57	0.01\\
84.58	0.01\\
84.59	0.01\\
84.6	0.01\\
84.61	0.01\\
84.62	0.01\\
84.63	0.01\\
84.64	0.01\\
84.65	0.01\\
84.66	0.01\\
84.67	0.01\\
84.68	0.01\\
84.69	0.01\\
84.7	0.01\\
84.71	0.01\\
84.72	0.01\\
84.73	0.01\\
84.74	0.01\\
84.75	0.01\\
84.76	0.01\\
84.77	0.01\\
84.78	0.01\\
84.79	0.01\\
84.8	0.01\\
84.81	0.01\\
84.82	0.01\\
84.83	0.01\\
84.84	0.01\\
84.85	0.01\\
84.86	0.01\\
84.87	0.01\\
84.88	0.01\\
84.89	0.01\\
84.9	0.01\\
84.91	0.01\\
84.92	0.01\\
84.93	0.01\\
84.94	0.01\\
84.95	0.01\\
84.96	0.01\\
84.97	0.01\\
84.98	0.01\\
84.99	0.01\\
85	0.01\\
85.01	0.01\\
85.02	0.01\\
85.03	0.01\\
85.04	0.01\\
85.05	0.01\\
85.06	0.01\\
85.07	0.01\\
85.08	0.01\\
85.09	0.01\\
85.1	0.01\\
85.11	0.01\\
85.12	0.01\\
85.13	0.01\\
85.14	0.01\\
85.15	0.01\\
85.16	0.01\\
85.17	0.01\\
85.18	0.01\\
85.19	0.01\\
85.2	0.01\\
85.21	0.01\\
85.22	0.01\\
85.23	0.01\\
85.24	0.01\\
85.25	0.01\\
85.26	0.01\\
85.27	0.01\\
85.28	0.01\\
85.29	0.01\\
85.3	0.01\\
85.31	0.01\\
85.32	0.01\\
85.33	0.01\\
85.34	0.01\\
85.35	0.01\\
85.36	0.01\\
85.37	0.01\\
85.38	0.01\\
85.39	0.01\\
85.4	0.01\\
85.41	0.01\\
85.42	0.01\\
85.43	0.01\\
85.44	0.01\\
85.45	0.01\\
85.46	0.01\\
85.47	0.01\\
85.48	0.01\\
85.49	0.01\\
85.5	0.01\\
85.51	0.01\\
85.52	0.01\\
85.53	0.01\\
85.54	0.00999855889538515\\
85.55	0.00999403882326935\\
85.56	0.00998951103636577\\
85.57	0.00998497551822438\\
85.58	0.00998043225236838\\
85.59	0.00997588122229432\\
85.6	0.00997132241147221\\
85.61	0.0099667558033456\\
85.62	0.0099621813813317\\
85.63	0.00995760187610974\\
85.64	0.00995301760746407\\
85.65	0.00994842856651868\\
85.66	0.00994383474437918\\
85.67	0.00993923613213287\\
85.68	0.00993463272084862\\
85.69	0.00993002450157701\\
85.7	0.00992541146535017\\
85.71	0.0099207936031819\\
85.72	0.00991617090606758\\
85.73	0.00991154336498421\\
85.74	0.00990691097089038\\
85.75	0.00990227371472629\\
85.76	0.00989763158741372\\
85.77	0.00989298457985604\\
85.78	0.00988833268293821\\
85.79	0.00988367588752678\\
85.8	0.00987901418446988\\
85.81	0.00987434756459721\\
85.82	0.00986967601872006\\
85.83	0.00986499953763133\\
85.84	0.00986031811210546\\
85.85	0.0098556317328985\\
85.86	0.0098509403907481\\
85.87	0.00984624407637348\\
85.88	0.00984154278047547\\
85.89	0.00983683649373649\\
85.9	0.00983212520682059\\
85.91	0.00982740891037341\\
85.92	0.00982268759502222\\
85.93	0.00981796125137592\\
85.94	0.00981322987002503\\
85.95	0.00980849344154174\\
85.96	0.00980375195647988\\
85.97	0.00979900540537494\\
85.98	0.00979425377874412\\
85.99	0.00978949706708627\\
86	0.00978473526088197\\
86.01	0.00977996835059351\\
86.02	0.00977519632666492\\
86.03	0.00977041917952197\\
86.04	0.0097656368995722\\
86.05	0.00976084947720494\\
86.06	0.00975605690279131\\
86.07	0.00975125916668428\\
86.08	0.00974645625921862\\
86.09	0.009741648170711\\
86.1	0.00973683489145998\\
86.11	0.00973201641174599\\
86.12	0.00972719272183145\\
86.13	0.00972236381196069\\
86.14	0.00971752967236008\\
86.15	0.00971269029323796\\
86.16	0.00970784566478474\\
86.17	0.0097029957771729\\
86.18	0.00969814062055701\\
86.19	0.00969328018507382\\
86.2	0.00968841446084219\\
86.21	0.00968354343796325\\
86.22	0.00967866710652031\\
86.23	0.009673785456579\\
86.24	0.00966889847818726\\
86.25	0.00966400616137538\\
86.26	0.00965910849615603\\
86.27	0.00965420547252435\\
86.28	0.00964929708045793\\
86.29	0.00964438330991691\\
86.3	0.00963946415084398\\
86.31	0.00963453959316446\\
86.32	0.00962960962678634\\
86.33	0.00962467424160032\\
86.34	0.00961973342747986\\
86.35	0.00961478717428127\\
86.36	0.00960983547184369\\
86.37	0.00960487830998922\\
86.38	0.00959991567852294\\
86.39	0.00959494756723298\\
86.4	0.00958997396589058\\
86.41	0.00958499486425011\\
86.42	0.00958001025204922\\
86.43	0.00957502011900884\\
86.44	0.00957002445483324\\
86.45	0.00956502324921014\\
86.46	0.00956001649181076\\
86.47	0.00955500417228988\\
86.48	0.00954998628028595\\
86.49	0.0095449628054211\\
86.5	0.00953993373730129\\
86.51	0.00953489906551634\\
86.52	0.00952985877964002\\
86.53	0.00952481286923016\\
86.54	0.00951976132382868\\
86.55	0.00951470413296173\\
86.56	0.00950964128613974\\
86.57	0.00950457277285755\\
86.58	0.00949949858259444\\
86.59	0.0094944187048143\\
86.6	0.00948933312896566\\
86.61	0.00948424184448182\\
86.62	0.00947914484078096\\
86.63	0.00947404210726619\\
86.64	0.00946893363332574\\
86.65	0.00946381940833298\\
86.66	0.00945869942164659\\
86.67	0.00945357366261063\\
86.68	0.00944844212055469\\
86.69	0.00944330478479397\\
86.7	0.00943816164462942\\
86.71	0.00943301268934786\\
86.72	0.00942785790822211\\
86.73	0.00942269729051107\\
86.74	0.00941753082545991\\
86.75	0.00941235850230016\\
86.76	0.00940718031024987\\
86.77	0.0094019962385137\\
86.78	0.00939680627628312\\
86.79	0.0093916104127365\\
86.8	0.00938640863703929\\
86.81	0.00938120093834413\\
86.82	0.00937598730579101\\
86.83	0.00937076772850748\\
86.84	0.0093655421956087\\
86.85	0.00936031069619768\\
86.86	0.0093550732193654\\
86.87	0.00934982975419101\\
86.88	0.00934458028974196\\
86.89	0.00933932481507417\\
86.9	0.00933406331923223\\
86.91	0.00932879579124956\\
86.92	0.00932352222014861\\
86.93	0.009318242594941\\
86.94	0.00931296062792505\\
86.95	0.00930767653239375\\
86.96	0.00930239030856869\\
86.97	0.00929710195667923\\
86.98	0.00929181147696262\\
86.99	0.00928651886966399\\
87	0.0092812241350364\\
87.01	0.0092759272733409\\
87.02	0.00927062828484655\\
87.03	0.00926532716983047\\
87.04	0.00926002392857791\\
87.05	0.00925471856138223\\
87.06	0.00924941106854499\\
87.07	0.009244101450376\\
87.08	0.00923878970719332\\
87.09	0.00923347583932335\\
87.1	0.00922815984710081\\
87.11	0.00922284173086888\\
87.12	0.00921752149097914\\
87.13	0.0092121991277917\\
87.14	0.00920687464167518\\
87.15	0.00920154803300678\\
87.16	0.00919621930217235\\
87.17	0.00919088844956639\\
87.18	0.00918555547559213\\
87.19	0.00918022038066155\\
87.2	0.00917488316519543\\
87.21	0.00916954382962343\\
87.22	0.00916420237438408\\
87.23	0.00915885879992488\\
87.24	0.0091535131067023\\
87.25	0.00914816529518186\\
87.26	0.00914281536583817\\
87.27	0.00913746331915497\\
87.28	0.00913210915562518\\
87.29	0.00912675287575093\\
87.3	0.00912139448004367\\
87.31	0.00911603396902413\\
87.32	0.00911067134322244\\
87.33	0.00910530660317814\\
87.34	0.00909993974944025\\
87.35	0.00909457078256731\\
87.36	0.00908919970312743\\
87.37	0.00908382651169832\\
87.38	0.0090784512088674\\
87.39	0.00907307379523177\\
87.4	0.00906769427139833\\
87.41	0.00906231263798379\\
87.42	0.00905692889561473\\
87.43	0.00905154304492766\\
87.44	0.00904615508656906\\
87.45	0.00904076502119545\\
87.46	0.00903537284947342\\
87.47	0.00902997857207971\\
87.48	0.0090245821897012\\
87.49	0.00901918370303508\\
87.5	0.00901378311278876\\
87.51	0.00900838041968005\\
87.52	0.00900297562443712\\
87.53	0.00899756872779863\\
87.54	0.00899215973051371\\
87.55	0.00898674863334209\\
87.56	0.0089813354370541\\
87.57	0.00897592014243073\\
87.58	0.00897050275026372\\
87.59	0.00896508326135558\\
87.6	0.00895966167651969\\
87.61	0.00895423799658029\\
87.62	0.0089488122223726\\
87.63	0.00894338435474283\\
87.64	0.00893795439454829\\
87.65	0.00893252234265739\\
87.66	0.00892708819994972\\
87.67	0.00892165196731616\\
87.68	0.00891621364565883\\
87.69	0.00891077323589125\\
87.7	0.00890533073893836\\
87.71	0.00889988615573656\\
87.72	0.00889443948723381\\
87.73	0.00888899073438965\\
87.74	0.00888354008006392\\
87.75	0.00887808752818598\\
87.76	0.00887263307724207\\
87.77	0.00886717672571643\\
87.78	0.00886171847209137\\
87.79	0.00885625831484717\\
87.8	0.00885079625246215\\
87.81	0.00884533228341264\\
87.82	0.00883986640617296\\
87.83	0.00883439861921546\\
87.84	0.00882892892101046\\
87.85	0.00882345731002631\\
87.86	0.00881798378472933\\
87.87	0.00881250834358382\\
87.88	0.0088070309850521\\
87.89	0.00880155170759444\\
87.9	0.00879607050966912\\
87.91	0.00879058738973237\\
87.92	0.00878510234623841\\
87.93	0.00877961537763941\\
87.94	0.00877412648238552\\
87.95	0.00876863565892484\\
87.96	0.00876314290570345\\
87.97	0.00875764822116535\\
87.98	0.00875215160375254\\
87.99	0.0087466530519049\\
88	0.00874115256406031\\
88.01	0.00873565013865457\\
88.02	0.00873014577412142\\
88.03	0.00872463946889251\\
88.04	0.00871913122139745\\
88.05	0.00871362103006376\\
88.06	0.00870810889331687\\
88.07	0.00870259480958016\\
88.08	0.00869707877727489\\
88.09	0.00869156079482024\\
88.1	0.00868604086063332\\
88.11	0.00868051897312909\\
88.12	0.00867499513072047\\
88.13	0.00866946933181822\\
88.14	0.00866394157483103\\
88.15	0.00865841185816546\\
88.16	0.00865288018022596\\
88.17	0.00864734653941483\\
88.18	0.00864181093413229\\
88.19	0.00863627336277641\\
88.2	0.00863073382374313\\
88.21	0.00862519231542623\\
88.22	0.0086196488362174\\
88.23	0.00861410338450612\\
88.24	0.00860855595867979\\
88.25	0.00860300655712361\\
88.26	0.00859745517822064\\
88.27	0.00859190182035177\\
88.28	0.00858634648189574\\
88.29	0.00858078916122912\\
88.3	0.00857522985672628\\
88.31	0.00856966856675945\\
88.32	0.00856410528969866\\
88.33	0.00855854002391175\\
88.34	0.00855297276776437\\
88.35	0.00854740351962\\
88.36	0.00854183227783989\\
88.37	0.00853625904078312\\
88.38	0.00853068380680653\\
88.39	0.00852510657426477\\
88.4	0.00851952734151027\\
88.41	0.00851394610689324\\
88.42	0.00850836286876167\\
88.43	0.00850277762546132\\
88.44	0.00849719037533572\\
88.45	0.00849160111672615\\
88.46	0.00848600984797166\\
88.47	0.00848041656740906\\
88.48	0.00847482127337289\\
88.49	0.00846922396419547\\
88.5	0.00846362463820682\\
88.51	0.00845802329373473\\
88.52	0.0084524199291047\\
88.53	0.00844681454263997\\
88.54	0.00844120713266149\\
88.55	0.00843559769748794\\
88.56	0.00842998623543571\\
88.57	0.00842437274481891\\
88.58	0.00841875722394933\\
88.59	0.00841313967113647\\
88.6	0.00840752008468755\\
88.61	0.00840189846290743\\
88.62	0.0083962748040987\\
88.63	0.00839064910656161\\
88.64	0.00838502136859409\\
88.65	0.00837939158849175\\
88.66	0.00837375976454785\\
88.67	0.00836812589505332\\
88.68	0.00836248997829673\\
88.69	0.00835685201256435\\
88.7	0.00835121199614003\\
88.71	0.00834556992730531\\
88.72	0.00833992580433935\\
88.73	0.00833427962551895\\
88.74	0.00832863138911852\\
88.75	0.0083229810934101\\
88.76	0.00831732873666335\\
88.77	0.00831167431714555\\
88.78	0.00830601783312155\\
88.79	0.00830035928285384\\
88.8	0.0082946986646025\\
88.81	0.00828903597662517\\
88.82	0.00828337121717711\\
88.83	0.00827770438451114\\
88.84	0.00827203547687768\\
88.85	0.00826636449252466\\
88.86	0.00826069142969765\\
88.87	0.00825501628663972\\
88.88	0.00824933906159153\\
88.89	0.00824365975279126\\
88.9	0.00823797835847464\\
88.91	0.00823229487687495\\
88.92	0.00822660930622299\\
88.93	0.00822092164474708\\
88.94	0.00821523189067307\\
88.95	0.00820954004222431\\
88.96	0.00820384609762167\\
88.97	0.00819815005508352\\
88.98	0.00819245191282572\\
88.99	0.00818675166906163\\
89	0.0081810493220021\\
89.01	0.00817534486985543\\
89.02	0.00816963831082743\\
89.03	0.00816392964312135\\
89.04	0.0081582188649379\\
89.05	0.00815250597447528\\
89.06	0.00814679096992909\\
89.07	0.00814107384949241\\
89.08	0.00813535461135575\\
89.09	0.00812963325370703\\
89.1	0.00812390977473161\\
89.11	0.00811818417261227\\
89.12	0.00811245644552919\\
89.13	0.00810672659165998\\
89.14	0.00810099460917962\\
89.15	0.00809526049626049\\
89.16	0.00808952425107238\\
89.17	0.00808378587178241\\
89.18	0.00807804535655514\\
89.19	0.00807230270355242\\
89.2	0.00806655791093353\\
89.21	0.00806081097685506\\
89.22	0.00805506189947095\\
89.23	0.0080493106769325\\
89.24	0.00804355730738833\\
89.25	0.00803780178898438\\
89.26	0.00803204411986391\\
89.27	0.00802628429816751\\
89.28	0.00802052232203306\\
89.29	0.00801475818959573\\
89.3	0.008008991898988\\
89.31	0.00800322344833962\\
89.32	0.00799745283577762\\
89.33	0.0079916800594263\\
89.34	0.00798590511740722\\
89.35	0.00798012800783919\\
89.36	0.00797434872883829\\
89.37	0.0079685672785178\\
89.38	0.00796278365498827\\
89.39	0.00795699785635746\\
89.4	0.00795120988073034\\
89.41	0.00794541972620908\\
89.42	0.00793962739089309\\
89.43	0.00793383287287894\\
89.44	0.00792803617026038\\
89.45	0.00792223728112838\\
89.46	0.00791643620357103\\
89.47	0.00791063293567361\\
89.48	0.00790482747551855\\
89.49	0.00789901982118543\\
89.5	0.00789320997075095\\
89.51	0.00788739792228895\\
89.52	0.0078815836738704\\
89.53	0.00787576722356336\\
89.54	0.007869948569433\\
89.55	0.00786412770954162\\
89.56	0.00785830464194855\\
89.57	0.00785247936471024\\
89.58	0.00784665187588018\\
89.59	0.00784082217350895\\
89.6	0.00783499025564416\\
89.61	0.00782915612033047\\
89.62	0.00782331976560958\\
89.63	0.00781748118952019\\
89.64	0.00781164039009804\\
89.65	0.00780579736537587\\
89.66	0.00779995211338341\\
89.67	0.00779410463214738\\
89.68	0.00778825491969149\\
89.69	0.0077824029740364\\
89.7	0.00777654879319973\\
89.71	0.00777069237519607\\
89.72	0.00776483371803692\\
89.73	0.00775897281973073\\
89.74	0.00775310967828285\\
89.75	0.00774724429169557\\
89.76	0.00774137665796805\\
89.77	0.00773550677509635\\
89.78	0.00772963464107341\\
89.79	0.00772376025388903\\
89.8	0.00771788361152989\\
89.81	0.00771200471197948\\
89.82	0.00770612355321816\\
89.83	0.00770024013322309\\
89.84	0.00769435444996828\\
89.85	0.00768846650142451\\
89.86	0.00768257628555936\\
89.87	0.0076766838003372\\
89.88	0.00767078904371917\\
89.89	0.00766489201366316\\
89.9	0.00765899270812383\\
89.91	0.00765309112505254\\
89.92	0.00764718726239741\\
89.93	0.00764128111810326\\
89.94	0.0076353726901116\\
89.95	0.00762946197636064\\
89.96	0.00762354897478528\\
89.97	0.00761763368331706\\
89.98	0.00761171609988418\\
89.99	0.00760579622241149\\
90	0.00759987404882046\\
90.01	0.00759394957702919\\
90.02	0.00758802280495235\\
90.03	0.00758209373050124\\
90.04	0.0075761623515837\\
90.05	0.00757022866610416\\
90.06	0.00756429267196359\\
90.07	0.0075583543670595\\
90.08	0.00755241374928592\\
90.09	0.00754647081653339\\
90.1	0.00754052556668895\\
90.11	0.00753457799763612\\
90.12	0.0075286281072549\\
90.13	0.00752267589342172\\
90.14	0.00751672135400947\\
90.15	0.00751076448688747\\
90.16	0.00750480528992144\\
90.17	0.00749884376097349\\
90.18	0.00749287989790213\\
90.19	0.00748691369856224\\
90.2	0.00748094516080501\\
90.21	0.00747497428247803\\
90.22	0.00746900106142516\\
90.23	0.00746302549548659\\
90.24	0.00745704758249881\\
90.25	0.00745106732029455\\
90.26	0.00744508470670283\\
90.27	0.0074390997395489\\
90.28	0.00743311241665424\\
90.29	0.00742712273583653\\
90.3	0.00742113069490966\\
90.31	0.00741513629168369\\
90.32	0.00740913952396482\\
90.33	0.00740314038955543\\
90.34	0.00739713888625398\\
90.35	0.00739113501185508\\
90.36	0.00738512876414941\\
90.37	0.00737912014092371\\
90.38	0.0073731091399608\\
90.39	0.00736709575903951\\
90.4	0.00736107999593472\\
90.41	0.00735506184841727\\
90.42	0.00734904131425401\\
90.43	0.00734301839120773\\
90.44	0.00733699307703719\\
90.45	0.00733096536949704\\
90.46	0.00732493526633786\\
90.47	0.0073189027653061\\
90.48	0.00731286786414407\\
90.49	0.00730683056058994\\
90.5	0.00730079085237768\\
90.51	0.00729474873723708\\
90.52	0.0072887042128937\\
90.53	0.00728265727706888\\
90.54	0.00727660792747968\\
90.55	0.00727055616183888\\
90.56	0.00726450197785497\\
90.57	0.00725844537323211\\
90.58	0.00725238634567008\\
90.59	0.00724632489286435\\
90.6	0.00724026101250595\\
90.61	0.0072341947022815\\
90.62	0.00722812595987321\\
90.63	0.0072220547829588\\
90.64	0.00721598116921151\\
90.65	0.00720990511630007\\
90.66	0.0072038266218887\\
90.67	0.00719774568363701\\
90.68	0.00719166229920008\\
90.69	0.00718557646622837\\
90.7	0.00717948818236767\\
90.71	0.00717339744525917\\
90.72	0.00716730425253933\\
90.73	0.00716120860183992\\
90.74	0.00715511049078797\\
90.75	0.00714900991700576\\
90.76	0.00714290687811076\\
90.77	0.00713680137171562\\
90.78	0.00713069339542819\\
90.79	0.00712458294685138\\
90.8	0.00711847002358325\\
90.81	0.00711235462321693\\
90.82	0.00710623674334056\\
90.83	0.00710011638153732\\
90.84	0.00709399353538538\\
90.85	0.00708786820245783\\
90.86	0.00708174038032273\\
90.87	0.00707561006654301\\
90.88	0.00706947725867645\\
90.89	0.00706334195427569\\
90.9	0.00705720415088816\\
90.91	0.00705106384605606\\
90.92	0.00704492103731632\\
90.93	0.0070387757222006\\
90.94	0.00703262789823519\\
90.95	0.00702647756294107\\
90.96	0.00702032471383378\\
90.97	0.00701416934842345\\
90.98	0.00700801146421477\\
90.99	0.00700185105870688\\
91	0.00699568812939345\\
91.01	0.00698952267376253\\
91.02	0.00698335468929659\\
91.03	0.00697718417347246\\
91.04	0.00697101112376128\\
91.05	0.00696483553762849\\
91.06	0.00695865741253378\\
91.07	0.00695247674593103\\
91.08	0.00694629353526831\\
91.09	0.00694010777798782\\
91.1	0.00693391947152584\\
91.11	0.00692772861331272\\
91.12	0.00692153520077282\\
91.13	0.00691533923132446\\
91.14	0.00690914070237992\\
91.15	0.00690293961134533\\
91.16	0.00689673595562069\\
91.17	0.00689052973259982\\
91.18	0.00688432093967026\\
91.19	0.0068781095742133\\
91.2	0.0068718956336039\\
91.21	0.00686567911521064\\
91.22	0.00685946001639569\\
91.23	0.00685323833451475\\
91.24	0.00684701406691702\\
91.25	0.00684078721094514\\
91.26	0.00683455776393515\\
91.27	0.00682832572321644\\
91.28	0.00682209108611169\\
91.29	0.00681585384993686\\
91.3	0.00680961401200106\\
91.31	0.00680337156960662\\
91.32	0.0067971265200489\\
91.33	0.00679087886061638\\
91.34	0.00678462858859047\\
91.35	0.00677837570124557\\
91.36	0.00677212019584896\\
91.37	0.00676586206966076\\
91.38	0.00675960131993385\\
91.39	0.00675333794391386\\
91.4	0.0067470719388391\\
91.41	0.00674080330194046\\
91.42	0.00673453203044142\\
91.43	0.00672825812155795\\
91.44	0.00672198157249846\\
91.45	0.00671570238046373\\
91.46	0.00670942054264688\\
91.47	0.00670313605623329\\
91.48	0.00669684891840052\\
91.49	0.00669055912631829\\
91.5	0.00668426667714838\\
91.51	0.00667797156804458\\
91.52	0.00667167379615263\\
91.53	0.00666537335861016\\
91.54	0.0066590702525466\\
91.55	0.00665276447508314\\
91.56	0.00664645602333262\\
91.57	0.00664014489439954\\
91.58	0.00663383108537989\\
91.59	0.00662751459336118\\
91.6	0.00662119541542228\\
91.61	0.00661487354863341\\
91.62	0.00660854899005605\\
91.63	0.00660222173674285\\
91.64	0.00659589178573757\\
91.65	0.00658955913407501\\
91.66	0.00658322377878092\\
91.67	0.00657688571687192\\
91.68	0.00657054494535545\\
91.69	0.00656420146122965\\
91.7	0.00655785526148332\\
91.71	0.00655150634309581\\
91.72	0.00654515470303694\\
91.73	0.00653880033826695\\
91.74	0.00653244324573636\\
91.75	0.00652608342238592\\
91.76	0.00651972086514655\\
91.77	0.00651335557093919\\
91.78	0.00650698753667474\\
91.79	0.006500616759254\\
91.8	0.00649424323556753\\
91.81	0.00648786696249559\\
91.82	0.00648148793690804\\
91.83	0.00647510615566426\\
91.84	0.00646872161561299\\
91.85	0.00646233431359235\\
91.86	0.00645594424642962\\
91.87	0.00644955141094123\\
91.88	0.00644315580393261\\
91.89	0.00643675742219812\\
91.9	0.00643035626252093\\
91.91	0.00642395232167291\\
91.92	0.00641754559641455\\
91.93	0.00641113608349483\\
91.94	0.00640472377965113\\
91.95	0.0063983086816091\\
91.96	0.00639189078608258\\
91.97	0.00638547008977347\\
91.98	0.00637904658937159\\
91.99	0.00637262028155464\\
92	0.006366191162988\\
92.01	0.00635975923032467\\
92.02	0.00635332448020514\\
92.03	0.00634688690945161\\
92.04	0.00634044651505435\\
92.05	0.00633400329399381\\
92.06	0.00632755724324056\\
92.07	0.00632110835975515\\
92.08	0.00631465664048807\\
92.09	0.00630820208237962\\
92.1	0.00630174468235988\\
92.11	0.0062952844373485\\
92.12	0.00628882134425473\\
92.13	0.00628235539997724\\
92.14	0.00627588660140406\\
92.15	0.00626941494541246\\
92.16	0.00626294042886886\\
92.17	0.00625646304862873\\
92.18	0.00624998280153649\\
92.19	0.00624349968442537\\
92.2	0.00623701369411737\\
92.21	0.00623052482742308\\
92.22	0.00622403308114163\\
92.23	0.00621753845206054\\
92.24	0.00621104093695563\\
92.25	0.00620454053259091\\
92.26	0.00619803723571843\\
92.27	0.00619153104307821\\
92.28	0.0061850219513981\\
92.29	0.00617850995739366\\
92.3	0.00617199505776803\\
92.31	0.00616547724921186\\
92.32	0.00615895652840309\\
92.33	0.00615243289200693\\
92.34	0.00614590633667566\\
92.35	0.00613937685904852\\
92.36	0.0061328444557516\\
92.37	0.00612630912339769\\
92.38	0.00611977085858614\\
92.39	0.00611322965790273\\
92.4	0.00610668551791956\\
92.41	0.00610013843519487\\
92.42	0.0060935884062729\\
92.43	0.0060870354276838\\
92.44	0.00608047949594341\\
92.45	0.00607392060755319\\
92.46	0.006067358759\\
92.47	0.006060793946756\\
92.48	0.00605422616727848\\
92.49	0.0060476554170097\\
92.5	0.00604108169237674\\
92.51	0.00603450498979135\\
92.52	0.00602792530564978\\
92.53	0.0060213426363326\\
92.54	0.00601475697820458\\
92.55	0.00600816832761447\\
92.56	0.00600157668089489\\
92.57	0.00599498203436208\\
92.58	0.00598838438431581\\
92.59	0.00598178372703916\\
92.6	0.00597518005879834\\
92.61	0.00596857337584251\\
92.62	0.00596196367440361\\
92.63	0.00595535095069619\\
92.64	0.00594873520091718\\
92.65	0.0059421164212457\\
92.66	0.00593549460784294\\
92.67	0.00592886975685187\\
92.68	0.00592224186439711\\
92.69	0.00591561092658467\\
92.7	0.00590897693950182\\
92.71	0.00590233989921682\\
92.72	0.00589569980177874\\
92.73	0.00588905664321726\\
92.74	0.00588241041954241\\
92.75	0.00587576112674441\\
92.76	0.0058691087607934\\
92.77	0.00586245331763927\\
92.78	0.00585579479321139\\
92.79	0.00584913318341837\\
92.8	0.00584246848414789\\
92.81	0.00583580069126642\\
92.82	0.00582912980061897\\
92.83	0.0058224558080289\\
92.84	0.00581577870929762\\
92.85	0.00580909850020437\\
92.86	0.00580241517650598\\
92.87	0.0057957287339366\\
92.88	0.00578903916820742\\
92.89	0.00578234647500646\\
92.9	0.00577565064999828\\
92.91	0.00576895168882367\\
92.92	0.00576224958709948\\
92.93	0.00575554434041822\\
92.94	0.00574883594434789\\
92.95	0.00574212439443163\\
92.96	0.00573540968618745\\
92.97	0.00572869181510798\\
92.98	0.00572197077666009\\
92.99	0.00571524656628467\\
93	0.0057085191793963\\
93.01	0.00570178861138295\\
93.02	0.00569505485760563\\
93.03	0.00568831791339816\\
93.04	0.00568157777406675\\
93.05	0.00567483443488977\\
93.06	0.00566808789111736\\
93.07	0.00566133813797112\\
93.08	0.00565458517064378\\
93.09	0.00564782898429884\\
93.1	0.00564106957407026\\
93.11	0.00563430693506209\\
93.12	0.00562754106234808\\
93.13	0.00562077195097139\\
93.14	0.00561399959594419\\
93.15	0.00560722399224728\\
93.16	0.00560044513482974\\
93.17	0.00559366301860854\\
93.18	0.00558687763846815\\
93.19	0.00558008898926018\\
93.2	0.00557329706580295\\
93.21	0.00556650186288111\\
93.22	0.00555970337524524\\
93.23	0.00555290159761143\\
93.24	0.00554609652466086\\
93.25	0.0055392881510394\\
93.26	0.00553247647135717\\
93.27	0.0055256614801881\\
93.28	0.00551884317206951\\
93.29	0.00551202154150165\\
93.3	0.00550519658294727\\
93.31	0.00549836829083112\\
93.32	0.00549153665953958\\
93.33	0.00548470168342008\\
93.34	0.00547786335678069\\
93.35	0.00547102167388965\\
93.36	0.00546417662897484\\
93.37	0.00545732821622333\\
93.38	0.00545047642978083\\
93.39	0.00544362126375123\\
93.4	0.00543676271219607\\
93.41	0.00542990076913402\\
93.42	0.00542303542854035\\
93.43	0.00541616668434639\\
93.44	0.005409294530439\\
93.45	0.00540241896066003\\
93.46	0.00539553996880575\\
93.47	0.00538865754862628\\
93.48	0.00538177169382503\\
93.49	0.00537488239805814\\
93.5	0.00536798965493386\\
93.51	0.00536109345801197\\
93.52	0.00535419380080319\\
93.53	0.00534729067676855\\
93.54	0.00534038407931881\\
93.55	0.00533347400181378\\
93.56	0.00532656043756174\\
93.57	0.00531964337981873\\
93.58	0.005312722821788\\
93.59	0.00530579875661924\\
93.6	0.00529887117740799\\
93.61	0.00529194007841584\\
93.62	0.00528500545444815\\
93.63	0.00527806730027387\\
93.64	0.00527112561062511\\
93.65	0.00526418038019663\\
93.66	0.00525723160364533\\
93.67	0.00525027927558981\\
93.68	0.00524332339060986\\
93.69	0.00523636394324591\\
93.7	0.00522940092799855\\
93.71	0.00522243433932801\\
93.72	0.00521546417165362\\
93.73	0.0052084904193533\\
93.74	0.00520151307676297\\
93.75	0.00519453213817603\\
93.76	0.00518754759784282\\
93.77	0.00518055944997\\
93.78	0.00517356768872003\\
93.79	0.00516657230821055\\
93.8	0.00515957330251383\\
93.81	0.00515257066565612\\
93.82	0.00514556439161709\\
93.83	0.00513855447432919\\
93.84	0.00513154090767703\\
93.85	0.00512452368549676\\
93.86	0.00511750280157541\\
93.87	0.00511047824965026\\
93.88	0.00510345002340816\\
93.89	0.00509641811648491\\
93.9	0.00508938252246449\\
93.91	0.00508234323487848\\
93.92	0.00507530024720528\\
93.93	0.00506825355286948\\
93.94	0.00506120314524107\\
93.95	0.00505414901763477\\
93.96	0.00504709116330927\\
93.97	0.00504002957546647\\
93.98	0.00503296424725077\\
93.99	0.00502589517174828\\
94	0.00501882234198602\\
94.01	0.00501174575093118\\
94.02	0.00500466539149026\\
94.03	0.00499758125650835\\
94.04	0.0049904933387682\\
94.05	0.0049834016309895\\
94.06	0.00497630612582794\\
94.07	0.00496920681587443\\
94.08	0.00496210369365419\\
94.09	0.00495499675162588\\
94.1	0.00494788598218072\\
94.11	0.00494077137764159\\
94.12	0.00493365293026209\\
94.13	0.00492653063222566\\
94.14	0.00491940447564458\\
94.15	0.00491227445255907\\
94.16	0.00490514055493629\\
94.17	0.00489800277466938\\
94.18	0.00489086110357646\\
94.19	0.00488371553339963\\
94.2	0.00487656605580393\\
94.21	0.00486941266237634\\
94.22	0.00486225534462472\\
94.23	0.00485509409397671\\
94.24	0.00484792890177873\\
94.25	0.00484075975929481\\
94.26	0.00483358665770555\\
94.27	0.00482640958810694\\
94.28	0.00481922854150927\\
94.29	0.00481204350883597\\
94.3	0.00480485448092243\\
94.31	0.0047976614485148\\
94.32	0.00479046440226885\\
94.33	0.00478326333274869\\
94.34	0.00477605823042558\\
94.35	0.00476884908567667\\
94.36	0.00476163588878373\\
94.37	0.00475441862993188\\
94.38	0.00474719729920827\\
94.39	0.0047399718866008\\
94.4	0.00473274238199673\\
94.41	0.00472550878074039\\
94.42	0.0047182710786135\\
94.43	0.00471102927139192\\
94.44	0.00470378335484564\\
94.45	0.0046965333247388\\
94.46	0.00468927917682961\\
94.47	0.00468202090687045\\
94.48	0.00467475851060777\\
94.49	0.00466749198378214\\
94.5	0.00466022132212823\\
94.51	0.00465294652137478\\
94.52	0.00464566757724465\\
94.53	0.00463838448545476\\
94.54	0.00463109724171612\\
94.55	0.00462380584173379\\
94.56	0.00461651028120693\\
94.57	0.00460921055582874\\
94.58	0.00460190666128647\\
94.59	0.00459459859326145\\
94.6	0.00458728634742903\\
94.61	0.00457996991945862\\
94.62	0.00457264930501367\\
94.63	0.00456532449975164\\
94.64	0.00455799549932404\\
94.65	0.00455066229937642\\
94.66	0.00454332489554831\\
94.67	0.0045359832834733\\
94.68	0.00452863745877897\\
94.69	0.00452128741727185\\
94.7	0.00451393315485705\\
94.71	0.00450657466743472\\
94.72	0.00449921195089999\\
94.73	0.00449184500114304\\
94.74	0.00448447381404904\\
94.75	0.00447709838549819\\
94.76	0.00446971871136567\\
94.77	0.00446233478752166\\
94.78	0.00445494660983133\\
94.79	0.00444755417415486\\
94.8	0.0044401574763474\\
94.81	0.00443275651225907\\
94.82	0.00442535127773499\\
94.83	0.00441794176861524\\
94.84	0.00441052798073486\\
94.85	0.00440310990992386\\
94.86	0.00439568755200722\\
94.87	0.00438826090280486\\
94.88	0.00438082995813166\\
94.89	0.00437339471379744\\
94.9	0.00436595516560698\\
94.91	0.00435851130935997\\
94.92	0.00435106314085105\\
94.93	0.00434361065586981\\
94.94	0.00433615385020074\\
94.95	0.00432869271962326\\
94.96	0.00432122725991171\\
94.97	0.00431375746683534\\
94.98	0.00430628333615831\\
94.99	0.00429880486363971\\
95	0.0042913220450335\\
95.01	0.00428383487608855\\
95.02	0.00427634335254864\\
95.03	0.00426884747015242\\
95.04	0.00426134722463344\\
95.05	0.00425384261172012\\
95.06	0.00424633362713577\\
95.07	0.00423882026659858\\
95.08	0.0042313025258216\\
95.09	0.00422378040051275\\
95.1	0.00421625388637482\\
95.11	0.00420872297910546\\
95.12	0.00420118767439716\\
95.13	0.00419364796793729\\
95.14	0.00418610385540806\\
95.15	0.0041785553324865\\
95.16	0.00417100239484452\\
95.17	0.00416344503814884\\
95.18	0.00415588325806103\\
95.19	0.00414831705023748\\
95.2	0.00414074641032943\\
95.21	0.00413317133398291\\
95.22	0.00412559181683879\\
95.23	0.00411800785453276\\
95.24	0.00411041944269531\\
95.25	0.00410282657695175\\
95.26	0.00409522925292219\\
95.27	0.00408762746622156\\
95.28	0.00408002121245956\\
95.29	0.0040724104872407\\
95.3	0.0040647952861643\\
95.31	0.00405717560482444\\
95.32	0.00404955143881\\
95.33	0.00404192278370465\\
95.34	0.00403428963508684\\
95.35	0.00402665198852976\\
95.36	0.00401900983960143\\
95.37	0.00401136318386459\\
95.38	0.00400371201687679\\
95.39	0.0039960563341903\\
95.4	0.00398839613135218\\
95.41	0.00398073140390425\\
95.42	0.00397306214738305\\
95.43	0.00396538835731993\\
95.44	0.00395771002924092\\
95.45	0.00395002715866684\\
95.46	0.00394233974111325\\
95.47	0.00393464777209044\\
95.48	0.00392695124710342\\
95.49	0.00391925016165197\\
95.5	0.00391154451123058\\
95.51	0.00390383429132846\\
95.52	0.00389611949742956\\
95.53	0.00388840012501254\\
95.54	0.00388067616955081\\
95.55	0.00387294762651245\\
95.56	0.0038652144913603\\
95.57	0.00385747675955188\\
95.58	0.00384973442653943\\
95.59	0.00384198748776991\\
95.6	0.00383423593868495\\
95.61	0.00382647977472092\\
95.62	0.00381871899130887\\
95.63	0.00381095358387455\\
95.64	0.0038031835478384\\
95.65	0.00379540887861556\\
95.66	0.00378762957161586\\
95.67	0.0037798456222438\\
95.68	0.00377205702589859\\
95.69	0.00376426377797411\\
95.7	0.00375646587385892\\
95.71	0.00374866330893626\\
95.72	0.00374085607858406\\
95.73	0.0037330441781749\\
95.74	0.00372522760307604\\
95.75	0.00371740634864943\\
95.76	0.00370958041025166\\
95.77	0.00370174978323402\\
95.78	0.00369391446294243\\
95.79	0.00368607444471748\\
95.8	0.00367822972389445\\
95.81	0.00367038029580324\\
95.82	0.00366252615576844\\
95.83	0.00365466729910927\\
95.84	0.00364680372113963\\
95.85	0.00363893541716805\\
95.86	0.00363106238249771\\
95.87	0.00362318461242647\\
95.88	0.0036153021022468\\
95.89	0.00360741484724584\\
95.9	0.00359952284270538\\
95.91	0.00359162608390182\\
95.92	0.00358372456610625\\
95.93	0.00357581828458436\\
95.94	0.00356790723459651\\
95.95	0.00355999141139768\\
95.96	0.00355207081023749\\
95.97	0.00354414542636021\\
95.98	0.00353621525500474\\
95.99	0.0035282802914046\\
96	0.00352034053078797\\
96.01	0.00351239596837763\\
96.02	0.00350444659939104\\
96.03	0.00349649241904023\\
96.04	0.00348853342253192\\
96.05	0.00348056960506742\\
96.06	0.0034726009618427\\
96.07	0.00346462748804832\\
96.08	0.0034566491788695\\
96.09	0.00344866602948609\\
96.1	0.00344067803507254\\
96.11	0.00343268519079795\\
96.12	0.00342468749182604\\
96.13	0.00341668493331517\\
96.14	0.0034086775104183\\
96.15	0.00340066521828302\\
96.16	0.00339264805205159\\
96.17	0.00338462600686083\\
96.18	0.00337659907784224\\
96.19	0.00336856726012192\\
96.2	0.0033605305488206\\
96.21	0.00335248893905364\\
96.22	0.00334444242593103\\
96.23	0.00333639100455739\\
96.24	0.00332833467003195\\
96.25	0.00332027341744859\\
96.26	0.0033122072418958\\
96.27	0.00330413613845672\\
96.28	0.00329606010220911\\
96.29	0.00328797912822536\\
96.3	0.00327989321157249\\
96.31	0.00327180234731215\\
96.32	0.00326370653050064\\
96.33	0.00325560575618888\\
96.34	0.00324750001942242\\
96.35	0.00323938931524148\\
96.36	0.00323127363868087\\
96.37	0.00322315298477007\\
96.38	0.00321502734853321\\
96.39	0.00320689672498903\\
96.4	0.00319876110915094\\
96.41	0.00319062049602698\\
96.42	0.00318247488061984\\
96.43	0.00317432425792688\\
96.44	0.00316616862294007\\
96.45	0.00315800797064607\\
96.46	0.00314984229602617\\
96.47	0.00314167159405633\\
96.48	0.00313349585970716\\
96.49	0.00312531508794394\\
96.5	0.0031171292737266\\
96.51	0.00310893841200975\\
96.52	0.00310074249774266\\
96.53	0.00309254152586927\\
96.54	0.00308433549132819\\
96.55	0.00307612438905271\\
96.56	0.00306790821397081\\
96.57	0.00305968696100513\\
96.58	0.00305146062507301\\
96.59	0.00304322920108647\\
96.6	0.00303499268395221\\
96.61	0.00302675106857165\\
96.62	0.00301850434984089\\
96.63	0.00301025252265073\\
96.64	0.00300199558188669\\
96.65	0.00299373352242897\\
96.66	0.0029854663391525\\
96.67	0.00297719402692693\\
96.68	0.00296891658061661\\
96.69	0.00296063399508063\\
96.7	0.00295234626517281\\
96.71	0.00294405338574168\\
96.72	0.00293575535163052\\
96.73	0.00292745215767735\\
96.74	0.00291914379871494\\
96.75	0.0029108302695708\\
96.76	0.00290251156506715\\
96.77	0.0028941876800207\\
96.78	0.00288585860924267\\
96.79	0.00287752434753879\\
96.8	0.00286918488970926\\
96.81	0.00286084023054882\\
96.82	0.00285249036484668\\
96.83	0.00284413528738658\\
96.84	0.00283577499294675\\
96.85	0.00282740947629995\\
96.86	0.00281903873221342\\
96.87	0.00281066275544894\\
96.88	0.00280228154076279\\
96.89	0.00279389508290576\\
96.9	0.00278550337662318\\
96.91	0.00277710641665487\\
96.92	0.00276870419773519\\
96.93	0.00276029671459302\\
96.94	0.00275188396195176\\
96.95	0.00274346593452935\\
96.96	0.00273504262703824\\
96.97	0.00272661403418543\\
96.98	0.00271818015067244\\
96.99	0.00270974097119535\\
97	0.00270129649044475\\
97.01	0.00269284670310579\\
97.02	0.00268439160385817\\
97.03	0.00267593118737612\\
97.04	0.00266746544832843\\
97.05	0.00265899438137845\\
97.06	0.00265051798118407\\
97.07	0.00264203624239776\\
97.08	0.00263354915966653\\
97.09	0.00262505672763197\\
97.1	0.00261655894093024\\
97.11	0.00260805579419207\\
97.12	0.00259954728204275\\
97.13	0.00259103339910218\\
97.14	0.00258251413998481\\
97.15	0.00257398949929971\\
97.16	0.0025654594716505\\
97.17	0.00255692405163544\\
97.18	0.00254838323384734\\
97.19	0.00253983701287365\\
97.2	0.00253128538329641\\
97.21	0.00252272833969227\\
97.22	0.0025141658766325\\
97.23	0.00250559798868298\\
97.24	0.00249702467040422\\
97.25	0.00248844591635134\\
97.26	0.00247986172107413\\
97.27	0.00247127207911698\\
97.28	0.00246267698501893\\
97.29	0.00245407643331368\\
97.3	0.00244547041852957\\
97.31	0.00243685893518959\\
97.32	0.0024282419778114\\
97.33	0.00241961954090733\\
97.34	0.00241099161898438\\
97.35	0.00240235820654421\\
97.36	0.00239371929808318\\
97.37	0.00238507488809233\\
97.38	0.0023764249710574\\
97.39	0.00236776954145883\\
97.4	0.00235910859377175\\
97.41	0.00235044212246602\\
97.42	0.0023417701220062\\
97.43	0.00233309258685159\\
97.44	0.0023244095114562\\
97.45	0.00231572089026879\\
97.46	0.00230702671773286\\
97.47	0.00229832698828666\\
97.48	0.00228962169636317\\
97.49	0.00228091083639017\\
97.5	0.00227219440279019\\
97.51	0.00226347238998052\\
97.52	0.00225474479237327\\
97.53	0.00224601160437531\\
97.54	0.00223727282038831\\
97.55	0.00222852843480874\\
97.56	0.00221977844202791\\
97.57	0.00221102283643192\\
97.58	0.0022022616124017\\
97.59	0.00219349476431302\\
97.6	0.0021847222865365\\
97.61	0.0021759441734376\\
97.62	0.00216716041937663\\
97.63	0.00215837101870879\\
97.64	0.00214957596578414\\
97.65	0.00214077525494762\\
97.66	0.00213196888053906\\
97.67	0.00212315683689321\\
97.68	0.00211433911833971\\
97.69	0.00210551571920312\\
97.7	0.00209668663380295\\
97.71	0.0020878518564536\\
97.72	0.00207901138146446\\
97.73	0.00207016520313986\\
97.74	0.00206131331577909\\
97.75	0.00205245571367641\\
97.76	0.00204359239112108\\
97.77	0.00203472334239733\\
97.78	0.00202584856178441\\
97.79	0.0020169680435566\\
97.8	0.00200808178198316\\
97.81	0.00199918977132842\\
97.82	0.00199029200585175\\
97.83	0.00198138847980756\\
97.84	0.00197247918744535\\
97.85	0.00196356412300966\\
97.86	0.00195464328074016\\
97.87	0.00194571665487159\\
97.88	0.00193678423963382\\
97.89	0.00192784602925183\\
97.9	0.00191890201794574\\
97.91	0.0019099521999308\\
97.92	0.00190099656941745\\
97.93	0.00189203512061126\\
97.94	0.00188306784771302\\
97.95	0.00187409474491867\\
97.96	0.0018651158064194\\
97.97	0.0018561310264016\\
97.98	0.00184714039904687\\
97.99	0.0018381439185321\\
98	0.0018291415790294\\
98.01	0.00182013337470615\\
98.02	0.00181111929972505\\
98.03	0.00180209934824406\\
98.04	0.00179307351441646\\
98.05	0.00178404179239085\\
98.06	0.00177500417631119\\
98.07	0.00176596066031677\\
98.08	0.00175691123854224\\
98.09	0.00174785590511765\\
98.1	0.00173879465416847\\
98.11	0.00172972747981571\\
98.12	0.0017206543761759\\
98.13	0.00171157533736114\\
98.14	0.00170249035747905\\
98.15	0.00169339943063285\\
98.16	0.00168430255092135\\
98.17	0.00167519971243899\\
98.18	0.00166609090927615\\
98.19	0.00165697613551974\\
98.2	0.00164785538525315\\
98.21	0.00163872865255636\\
98.22	0.00162959593150587\\
98.23	0.00162045721617481\\
98.24	0.00161131250063292\\
98.25	0.00160216177894657\\
98.26	0.0015930050451788\\
98.27	0.00158384229338934\\
98.28	0.00157467351763463\\
98.29	0.00156549871196786\\
98.3	0.00155631787043897\\
98.31	0.0015471309870947\\
98.32	0.0015379380559786\\
98.33	0.00152873907109969\\
98.34	0.00151953402646197\\
98.35	0.00151032291569093\\
98.36	0.00150110573241224\\
98.37	0.00149188247025179\\
98.38	0.00148265312283583\\
98.39	0.00147341768379099\\
98.4	0.00146417614666711\\
98.41	0.00145492850465893\\
98.42	0.00144567474861469\\
98.43	0.00143641486935398\\
98.44	0.00142714885766748\\
98.45	0.00141787670431677\\
98.46	0.00140859840003403\\
98.47	0.00139931393552186\\
98.48	0.00139002330145297\\
98.49	0.00138072648846998\\
98.5	0.00137142348718512\\
98.51	0.00136211428818003\\
98.52	0.00135279888200543\\
98.53	0.00134347725918092\\
98.54	0.00133414941019467\\
98.55	0.00132481532550315\\
98.56	0.00131547499553094\\
98.57	0.00130612841067081\\
98.58	0.00129677556128345\\
98.59	0.00128741643769718\\
98.6	0.00127805103020772\\
98.61	0.00126867932907785\\
98.62	0.00125930132453712\\
98.63	0.0012499170175457\\
98.64	0.00124052648239183\\
98.65	0.00123112970820827\\
98.66	0.00122172668408664\\
98.67	0.00121231739907713\\
98.68	0.00120290184218811\\
98.69	0.00119348000238578\\
98.7	0.00118405186859308\\
98.71	0.0011746174296891\\
98.72	0.00116517667450878\\
98.73	0.00115572959184246\\
98.74	0.00114627672733929\\
98.75	0.00113681867495569\\
98.76	0.00112735542682799\\
98.77	0.00111788697506406\\
98.78	0.00110841343759048\\
98.79	0.0010989354539535\\
98.8	0.00108946054226941\\
98.81	0.00107998876405688\\
98.82	0.00107052018144894\\
98.83	0.00106105485720021\\
98.84	0.00105159285469419\\
98.85	0.0010421342379507\\
98.86	0.00103267907163337\\
98.87	0.00102322742105735\\
98.88	0.00101377935219699\\
98.89	0.00100433493169381\\
98.9	0.000994894226864423\\
98.91	0.000985457305708684\\
98.92	0.000976024235873082\\
98.93	0.000966595085556268\\
98.94	0.000957169923641867\\
98.95	0.000947748819706719\\
98.96	0.000938331844029227\\
98.97	0.000928919067597863\\
98.98	0.000919510562119788\\
98.99	0.000910106400029611\\
99	0.00090070665449829\\
99.01	0.000891311399442173\\
99.02	0.000881920709532172\\
99.03	0.0008725346602031\\
99.04	0.000863153327663143\\
99.05	0.000853776788903485\\
99.06	0.000844405121708091\\
99.07	0.000835038404663648\\
99.08	0.000825676717169661\\
99.09	0.000816320139448719\\
99.1	0.000806968752556916\\
99.11	0.000797622638394458\\
99.12	0.000788281879716437\\
99.13	0.000778946560143766\\
99.14	0.000769616764174317\\
99.15	0.000760292577194233\\
99.16	0.000750974085489424\\
99.17	0.000741661376257259\\
99.18	0.000732354537618451\\
99.19	0.000723053658629137\\
99.2	0.000713758829293178\\
99.21	0.000704470140574644\\
99.22	0.000695187684410533\\
99.23	0.000685911553723684\\
99.24	0.000676641842435923\\
99.25	0.000667378645481434\\
99.26	0.00065812205882035\\
99.27	0.000648872179452586\\
99.28	0.000639629105431912\\
99.29	0.000630392935880257\\
99.3	0.000621163771002279\\
99.31	0.000611941712100177\\
99.32	0.000602726861588772\\
99.33	0.000593519323010844\\
99.34	0.000584319201052751\\
99.35	0.000575126601560311\\
99.36	0.000565941631554984\\
99.37	0.00055676439925033\\
99.38	0.00054759501406877\\
99.39	0.000538433586658444\\
99.4	0.000529280228901823\\
99.41	0.000520135053933318\\
99.42	0.000510998176157183\\
99.43	0.000501869711265774\\
99.44	0.000492749776258126\\
99.45	0.000483638489458892\\
99.46	0.000474535970537619\\
99.47	0.000465442340528399\\
99.48	0.000456357721849881\\
99.49	0.000447282238325672\\
99.5	0.000438216015205108\\
99.51	0.000429159179184449\\
99.52	0.000420111858428454\\
99.53	0.000411074182592397\\
99.54	0.000402046282844485\\
99.55	0.000393028291888746\\
99.56	0.000384020343988333\\
99.57	0.000375022574989311\\
99.58	0.000366035122344904\\
99.59	0.000357058125140244\\
99.6	0.000348091724117582\\
99.61	0.000339136061702052\\
99.62	0.000330191282027605\\
99.63	0.000321257530963142\\
99.64	0.000312334956139852\\
99.65	0.000303423706979116\\
99.66	0.000294523934720993\\
99.67	0.000285635792453304\\
99.68	0.000276759435141329\\
99.69	0.000267895019658133\\
99.7	0.000259042704815544\\
99.71	0.000250202651395793\\
99.72	0.000241375022183844\\
99.73	0.000232559984946454\\
99.74	0.000223757712161201\\
99.75	0.000214968378467508\\
99.76	0.000206192160702477\\
99.77	0.000197429237937504\\
99.78	0.000188679791515737\\
99.79	0.00017994400509036\\
99.8	0.000171222064663754\\
99.81	0.000162514158627553\\
99.82	0.000153820477803632\\
99.83	0.000145141215486051\\
99.84	0.000136476567483968\\
99.85	0.00012782673216559\\
99.86	0.000119191910503162\\
99.87	0.000110572306119041\\
99.88	0.000101968125332889\\
99.89	9.33795772100343e-05\\
99.9	8.4806873611008e-05\\
99.91	7.62502292423421e-05\\
99.92	6.77098617086289e-05\\
99.93	5.91859915659125e-05\\
99.94	5.06788423764483e-05\\
99.95	4.21886407648911e-05\\
99.96	3.37156164759312e-05\\
99.97	2.52600024334866e-05\\
99.98	1.68220348014392e-05\\
99.99	8.40195304604129e-06\\
100	0\\
};
\addlegendentry{$q=-1$};

\addplot [color=black,solid,forget plot]
  table[row sep=crcr]{%
0.01	0\\
0.02	0\\
0.03	0\\
0.04	0\\
0.05	0\\
0.06	0\\
0.07	0\\
0.08	0\\
0.09	0\\
0.1	0\\
0.11	0\\
0.12	0\\
0.13	0\\
0.14	0\\
0.15	0\\
0.16	0\\
0.17	0\\
0.18	0\\
0.19	0\\
0.2	0\\
0.21	0\\
0.22	0\\
0.23	0\\
0.24	0\\
0.25	0\\
0.26	0\\
0.27	0\\
0.28	0\\
0.29	0\\
0.3	0\\
0.31	0\\
0.32	0\\
0.33	0\\
0.34	0\\
0.35	0\\
0.36	0\\
0.37	0\\
0.38	0\\
0.39	0\\
0.4	0\\
0.41	0\\
0.42	0\\
0.43	0\\
0.44	0\\
0.45	0\\
0.46	0\\
0.47	0\\
0.48	0\\
0.49	0\\
0.5	0\\
0.51	0\\
0.52	0\\
0.53	0\\
0.54	0\\
0.55	0\\
0.56	0\\
0.57	0\\
0.58	0\\
0.59	0\\
0.6	0\\
0.61	0\\
0.62	0\\
0.63	0\\
0.64	0\\
0.65	0\\
0.66	0\\
0.67	0\\
0.68	0\\
0.69	0\\
0.7	0\\
0.71	0\\
0.72	0\\
0.73	0\\
0.74	0\\
0.75	0\\
0.76	0\\
0.77	0\\
0.78	0\\
0.79	0\\
0.8	0\\
0.81	0\\
0.82	0\\
0.83	0\\
0.84	0\\
0.85	0\\
0.86	0\\
0.87	0\\
0.88	0\\
0.89	0\\
0.9	0\\
0.91	0\\
0.92	0\\
0.93	0\\
0.94	0\\
0.95	0\\
0.96	0\\
0.97	0\\
0.98	0\\
0.99	0\\
1	0\\
1.01	0\\
1.02	0\\
1.03	0\\
1.04	0\\
1.05	0\\
1.06	0\\
1.07	0\\
1.08	0\\
1.09	0\\
1.1	0\\
1.11	0\\
1.12	0\\
1.13	0\\
1.14	0\\
1.15	0\\
1.16	0\\
1.17	0\\
1.18	0\\
1.19	0\\
1.2	0\\
1.21	0\\
1.22	0\\
1.23	0\\
1.24	0\\
1.25	0\\
1.26	0\\
1.27	0\\
1.28	0\\
1.29	0\\
1.3	0\\
1.31	0\\
1.32	0\\
1.33	0\\
1.34	0\\
1.35	0\\
1.36	0\\
1.37	0\\
1.38	0\\
1.39	0\\
1.4	0\\
1.41	0\\
1.42	0\\
1.43	0\\
1.44	0\\
1.45	0\\
1.46	0\\
1.47	0\\
1.48	0\\
1.49	0\\
1.5	0\\
1.51	0\\
1.52	0\\
1.53	0\\
1.54	0\\
1.55	0\\
1.56	0\\
1.57	0\\
1.58	0\\
1.59	0\\
1.6	0\\
1.61	0\\
1.62	0\\
1.63	0\\
1.64	0\\
1.65	0\\
1.66	0\\
1.67	0\\
1.68	0\\
1.69	0\\
1.7	0\\
1.71	0\\
1.72	0\\
1.73	0\\
1.74	0\\
1.75	0\\
1.76	0\\
1.77	0\\
1.78	0\\
1.79	0\\
1.8	0\\
1.81	0\\
1.82	0\\
1.83	0\\
1.84	0\\
1.85	0\\
1.86	0\\
1.87	0\\
1.88	0\\
1.89	0\\
1.9	0\\
1.91	0\\
1.92	0\\
1.93	0\\
1.94	0\\
1.95	0\\
1.96	0\\
1.97	0\\
1.98	0\\
1.99	0\\
2	0\\
2.01	0\\
2.02	0\\
2.03	0\\
2.04	0\\
2.05	0\\
2.06	0\\
2.07	0\\
2.08	0\\
2.09	0\\
2.1	0\\
2.11	0\\
2.12	0\\
2.13	0\\
2.14	0\\
2.15	0\\
2.16	0\\
2.17	0\\
2.18	0\\
2.19	0\\
2.2	0\\
2.21	0\\
2.22	0\\
2.23	0\\
2.24	0\\
2.25	0\\
2.26	0\\
2.27	0\\
2.28	0\\
2.29	0\\
2.3	0\\
2.31	0\\
2.32	0\\
2.33	0\\
2.34	0\\
2.35	0\\
2.36	0\\
2.37	0\\
2.38	0\\
2.39	0\\
2.4	0\\
2.41	0\\
2.42	0\\
2.43	0\\
2.44	0\\
2.45	0\\
2.46	0\\
2.47	0\\
2.48	0\\
2.49	0\\
2.5	0\\
2.51	0\\
2.52	0\\
2.53	0\\
2.54	0\\
2.55	0\\
2.56	0\\
2.57	0\\
2.58	0\\
2.59	0\\
2.6	0\\
2.61	0\\
2.62	0\\
2.63	0\\
2.64	0\\
2.65	0\\
2.66	0\\
2.67	0\\
2.68	0\\
2.69	0\\
2.7	0\\
2.71	0\\
2.72	0\\
2.73	0\\
2.74	0\\
2.75	0\\
2.76	0\\
2.77	0\\
2.78	0\\
2.79	0\\
2.8	0\\
2.81	0\\
2.82	0\\
2.83	0\\
2.84	0\\
2.85	0\\
2.86	0\\
2.87	0\\
2.88	0\\
2.89	0\\
2.9	0\\
2.91	0\\
2.92	0\\
2.93	0\\
2.94	0\\
2.95	0\\
2.96	0\\
2.97	0\\
2.98	0\\
2.99	0\\
3	0\\
3.01	0\\
3.02	0\\
3.03	0\\
3.04	0\\
3.05	0\\
3.06	0\\
3.07	0\\
3.08	0\\
3.09	0\\
3.1	0\\
3.11	0\\
3.12	0\\
3.13	0\\
3.14	0\\
3.15	0\\
3.16	0\\
3.17	0\\
3.18	0\\
3.19	0\\
3.2	0\\
3.21	0\\
3.22	0\\
3.23	0\\
3.24	0\\
3.25	0\\
3.26	0\\
3.27	0\\
3.28	0\\
3.29	0\\
3.3	0\\
3.31	0\\
3.32	0\\
3.33	0\\
3.34	0\\
3.35	0\\
3.36	0\\
3.37	0\\
3.38	0\\
3.39	0\\
3.4	0\\
3.41	0\\
3.42	0\\
3.43	0\\
3.44	0\\
3.45	0\\
3.46	0\\
3.47	0\\
3.48	0\\
3.49	0\\
3.5	0\\
3.51	0\\
3.52	0\\
3.53	0\\
3.54	0\\
3.55	0\\
3.56	0\\
3.57	0\\
3.58	0\\
3.59	0\\
3.6	0\\
3.61	0\\
3.62	0\\
3.63	0\\
3.64	0\\
3.65	0\\
3.66	0\\
3.67	0\\
3.68	0\\
3.69	0\\
3.7	0\\
3.71	0\\
3.72	0\\
3.73	0\\
3.74	0\\
3.75	0\\
3.76	0\\
3.77	0\\
3.78	0\\
3.79	0\\
3.8	0\\
3.81	0\\
3.82	0\\
3.83	0\\
3.84	0\\
3.85	0\\
3.86	0\\
3.87	0\\
3.88	0\\
3.89	0\\
3.9	0\\
3.91	0\\
3.92	0\\
3.93	0\\
3.94	0\\
3.95	0\\
3.96	0\\
3.97	0\\
3.98	0\\
3.99	0\\
4	0\\
4.01	0\\
4.02	0\\
4.03	0\\
4.04	0\\
4.05	0\\
4.06	0\\
4.07	0\\
4.08	0\\
4.09	0\\
4.1	0\\
4.11	0\\
4.12	0\\
4.13	0\\
4.14	0\\
4.15	0\\
4.16	0\\
4.17	0\\
4.18	0\\
4.19	0\\
4.2	0\\
4.21	0\\
4.22	0\\
4.23	0\\
4.24	0\\
4.25	0\\
4.26	0\\
4.27	0\\
4.28	0\\
4.29	0\\
4.3	0\\
4.31	0\\
4.32	0\\
4.33	0\\
4.34	0\\
4.35	0\\
4.36	0\\
4.37	0\\
4.38	0\\
4.39	0\\
4.4	0\\
4.41	0\\
4.42	0\\
4.43	0\\
4.44	0\\
4.45	0\\
4.46	0\\
4.47	0\\
4.48	0\\
4.49	0\\
4.5	0\\
4.51	0\\
4.52	0\\
4.53	0\\
4.54	0\\
4.55	0\\
4.56	0\\
4.57	0\\
4.58	0\\
4.59	0\\
4.6	0\\
4.61	0\\
4.62	0\\
4.63	0\\
4.64	0\\
4.65	0\\
4.66	0\\
4.67	0\\
4.68	0\\
4.69	0\\
4.7	0\\
4.71	0\\
4.72	0\\
4.73	0\\
4.74	0\\
4.75	0\\
4.76	0\\
4.77	0\\
4.78	0\\
4.79	0\\
4.8	0\\
4.81	0\\
4.82	0\\
4.83	0\\
4.84	0\\
4.85	0\\
4.86	0\\
4.87	0\\
4.88	0\\
4.89	0\\
4.9	0\\
4.91	0\\
4.92	0\\
4.93	0\\
4.94	0\\
4.95	0\\
4.96	0\\
4.97	0\\
4.98	0\\
4.99	0\\
5	0\\
5.01	0\\
5.02	0\\
5.03	0\\
5.04	0\\
5.05	0\\
5.06	0\\
5.07	0\\
5.08	0\\
5.09	0\\
5.1	0\\
5.11	0\\
5.12	0\\
5.13	0\\
5.14	0\\
5.15	0\\
5.16	0\\
5.17	0\\
5.18	0\\
5.19	0\\
5.2	0\\
5.21	0\\
5.22	0\\
5.23	0\\
5.24	0\\
5.25	0\\
5.26	0\\
5.27	0\\
5.28	0\\
5.29	0\\
5.3	0\\
5.31	0\\
5.32	0\\
5.33	0\\
5.34	0\\
5.35	0\\
5.36	0\\
5.37	0\\
5.38	0\\
5.39	0\\
5.4	0\\
5.41	0\\
5.42	0\\
5.43	0\\
5.44	0\\
5.45	0\\
5.46	0\\
5.47	0\\
5.48	0\\
5.49	0\\
5.5	0\\
5.51	0\\
5.52	0\\
5.53	0\\
5.54	0\\
5.55	0\\
5.56	0\\
5.57	0\\
5.58	0\\
5.59	0\\
5.6	0\\
5.61	0\\
5.62	0\\
5.63	0\\
5.64	0\\
5.65	0\\
5.66	0\\
5.67	0\\
5.68	0\\
5.69	0\\
5.7	0\\
5.71	0\\
5.72	0\\
5.73	0\\
5.74	0\\
5.75	0\\
5.76	0\\
5.77	0\\
5.78	0\\
5.79	0\\
5.8	0\\
5.81	0\\
5.82	0\\
5.83	0\\
5.84	0\\
5.85	0\\
5.86	0\\
5.87	0\\
5.88	0\\
5.89	0\\
5.9	0\\
5.91	0\\
5.92	0\\
5.93	0\\
5.94	0\\
5.95	0\\
5.96	0\\
5.97	0\\
5.98	0\\
5.99	0\\
6	0\\
6.01	0\\
6.02	0\\
6.03	0\\
6.04	0\\
6.05	0\\
6.06	0\\
6.07	0\\
6.08	0\\
6.09	0\\
6.1	0\\
6.11	0\\
6.12	0\\
6.13	0\\
6.14	0\\
6.15	0\\
6.16	0\\
6.17	0\\
6.18	0\\
6.19	0\\
6.2	0\\
6.21	0\\
6.22	0\\
6.23	0\\
6.24	0\\
6.25	0\\
6.26	0\\
6.27	0\\
6.28	0\\
6.29	0\\
6.3	0\\
6.31	0\\
6.32	0\\
6.33	0\\
6.34	0\\
6.35	0\\
6.36	0\\
6.37	0\\
6.38	0\\
6.39	0\\
6.4	0\\
6.41	0\\
6.42	0\\
6.43	0\\
6.44	0\\
6.45	0\\
6.46	0\\
6.47	0\\
6.48	0\\
6.49	0\\
6.5	0\\
6.51	0\\
6.52	0\\
6.53	0\\
6.54	0\\
6.55	0\\
6.56	0\\
6.57	0\\
6.58	0\\
6.59	0\\
6.6	0\\
6.61	0\\
6.62	0\\
6.63	0\\
6.64	0\\
6.65	0\\
6.66	0\\
6.67	0\\
6.68	0\\
6.69	0\\
6.7	0\\
6.71	0\\
6.72	0\\
6.73	0\\
6.74	0\\
6.75	0\\
6.76	0\\
6.77	0\\
6.78	0\\
6.79	0\\
6.8	0\\
6.81	0\\
6.82	0\\
6.83	0\\
6.84	0\\
6.85	0\\
6.86	0\\
6.87	0\\
6.88	0\\
6.89	0\\
6.9	0\\
6.91	0\\
6.92	0\\
6.93	0\\
6.94	0\\
6.95	0\\
6.96	0\\
6.97	0\\
6.98	0\\
6.99	0\\
7	0\\
7.01	0\\
7.02	0\\
7.03	0\\
7.04	0\\
7.05	0\\
7.06	0\\
7.07	0\\
7.08	0\\
7.09	0\\
7.1	0\\
7.11	0\\
7.12	0\\
7.13	0\\
7.14	0\\
7.15	0\\
7.16	0\\
7.17	0\\
7.18	0\\
7.19	0\\
7.2	0\\
7.21	0\\
7.22	0\\
7.23	0\\
7.24	0\\
7.25	0\\
7.26	0\\
7.27	0\\
7.28	0\\
7.29	0\\
7.3	0\\
7.31	0\\
7.32	0\\
7.33	0\\
7.34	0\\
7.35	0\\
7.36	0\\
7.37	0\\
7.38	0\\
7.39	0\\
7.4	0\\
7.41	0\\
7.42	0\\
7.43	0\\
7.44	0\\
7.45	0\\
7.46	0\\
7.47	0\\
7.48	0\\
7.49	0\\
7.5	0\\
7.51	0\\
7.52	0\\
7.53	0\\
7.54	0\\
7.55	0\\
7.56	0\\
7.57	0\\
7.58	0\\
7.59	0\\
7.6	0\\
7.61	0\\
7.62	0\\
7.63	0\\
7.64	0\\
7.65	0\\
7.66	0\\
7.67	0\\
7.68	0\\
7.69	0\\
7.7	0\\
7.71	0\\
7.72	0\\
7.73	0\\
7.74	0\\
7.75	0\\
7.76	0\\
7.77	0\\
7.78	0\\
7.79	0\\
7.8	0\\
7.81	0\\
7.82	0\\
7.83	0\\
7.84	0\\
7.85	0\\
7.86	0\\
7.87	0\\
7.88	0\\
7.89	0\\
7.9	0\\
7.91	0\\
7.92	0\\
7.93	0\\
7.94	0\\
7.95	0\\
7.96	0\\
7.97	0\\
7.98	0\\
7.99	0\\
8	0\\
8.01	0\\
8.02	0\\
8.03	0\\
8.04	0\\
8.05	0\\
8.06	0\\
8.07	0\\
8.08	0\\
8.09	0\\
8.1	0\\
8.11	0\\
8.12	0\\
8.13	0\\
8.14	0\\
8.15	0\\
8.16	0\\
8.17	0\\
8.18	0\\
8.19	0\\
8.2	0\\
8.21	0\\
8.22	0\\
8.23	0\\
8.24	0\\
8.25	0\\
8.26	0\\
8.27	0\\
8.28	0\\
8.29	0\\
8.3	0\\
8.31	0\\
8.32	0\\
8.33	0\\
8.34	0\\
8.35	0\\
8.36	0\\
8.37	0\\
8.38	0\\
8.39	0\\
8.4	0\\
8.41	0\\
8.42	0\\
8.43	0\\
8.44	0\\
8.45	0\\
8.46	0\\
8.47	0\\
8.48	0\\
8.49	0\\
8.5	0\\
8.51	0\\
8.52	0\\
8.53	0\\
8.54	0\\
8.55	0\\
8.56	0\\
8.57	0\\
8.58	0\\
8.59	0\\
8.6	0\\
8.61	0\\
8.62	0\\
8.63	0\\
8.64	0\\
8.65	0\\
8.66	0\\
8.67	0\\
8.68	0\\
8.69	0\\
8.7	0\\
8.71	0\\
8.72	0\\
8.73	0\\
8.74	0\\
8.75	0\\
8.76	0\\
8.77	0\\
8.78	0\\
8.79	0\\
8.8	0\\
8.81	0\\
8.82	0\\
8.83	0\\
8.84	0\\
8.85	0\\
8.86	0\\
8.87	0\\
8.88	0\\
8.89	0\\
8.9	0\\
8.91	0\\
8.92	0\\
8.93	0\\
8.94	0\\
8.95	0\\
8.96	0\\
8.97	0\\
8.98	0\\
8.99	0\\
9	0\\
9.01	0\\
9.02	0\\
9.03	0\\
9.04	0\\
9.05	0\\
9.06	0\\
9.07	0\\
9.08	0\\
9.09	0\\
9.1	0\\
9.11	0\\
9.12	0\\
9.13	0\\
9.14	0\\
9.15	0\\
9.16	0\\
9.17	0\\
9.18	0\\
9.19	0\\
9.2	0\\
9.21	0\\
9.22	0\\
9.23	0\\
9.24	0\\
9.25	0\\
9.26	0\\
9.27	0\\
9.28	0\\
9.29	0\\
9.3	0\\
9.31	0\\
9.32	0\\
9.33	0\\
9.34	0\\
9.35	0\\
9.36	0\\
9.37	0\\
9.38	0\\
9.39	0\\
9.4	0\\
9.41	0\\
9.42	0\\
9.43	0\\
9.44	0\\
9.45	0\\
9.46	0\\
9.47	0\\
9.48	0\\
9.49	0\\
9.5	0\\
9.51	0\\
9.52	0\\
9.53	0\\
9.54	0\\
9.55	0\\
9.56	0\\
9.57	0\\
9.58	0\\
9.59	0\\
9.6	0\\
9.61	0\\
9.62	0\\
9.63	0\\
9.64	0\\
9.65	0\\
9.66	0\\
9.67	0\\
9.68	0\\
9.69	0\\
9.7	0\\
9.71	0\\
9.72	0\\
9.73	0\\
9.74	0\\
9.75	0\\
9.76	0\\
9.77	0\\
9.78	0\\
9.79	0\\
9.8	0\\
9.81	0\\
9.82	0\\
9.83	0\\
9.84	0\\
9.85	0\\
9.86	0\\
9.87	0\\
9.88	0\\
9.89	0\\
9.9	0\\
9.91	0\\
9.92	0\\
9.93	0\\
9.94	0\\
9.95	0\\
9.96	0\\
9.97	0\\
9.98	0\\
9.99	0\\
10	0\\
10.01	0\\
10.02	0\\
10.03	0\\
10.04	0\\
10.05	0\\
10.06	0\\
10.07	0\\
10.08	0\\
10.09	0\\
10.1	0\\
10.11	0\\
10.12	0\\
10.13	0\\
10.14	0\\
10.15	0\\
10.16	0\\
10.17	0\\
10.18	0\\
10.19	0\\
10.2	0\\
10.21	0\\
10.22	0\\
10.23	0\\
10.24	0\\
10.25	0\\
10.26	0\\
10.27	0\\
10.28	0\\
10.29	0\\
10.3	0\\
10.31	0\\
10.32	0\\
10.33	0\\
10.34	0\\
10.35	0\\
10.36	0\\
10.37	0\\
10.38	0\\
10.39	0\\
10.4	0\\
10.41	0\\
10.42	0\\
10.43	0\\
10.44	0\\
10.45	0\\
10.46	0\\
10.47	0\\
10.48	0\\
10.49	0\\
10.5	0\\
10.51	0\\
10.52	0\\
10.53	0\\
10.54	0\\
10.55	0\\
10.56	0\\
10.57	0\\
10.58	0\\
10.59	0\\
10.6	0\\
10.61	0\\
10.62	0\\
10.63	0\\
10.64	0\\
10.65	0\\
10.66	0\\
10.67	0\\
10.68	0\\
10.69	0\\
10.7	0\\
10.71	0\\
10.72	0\\
10.73	0\\
10.74	0\\
10.75	0\\
10.76	0\\
10.77	0\\
10.78	0\\
10.79	0\\
10.8	0\\
10.81	0\\
10.82	0\\
10.83	0\\
10.84	0\\
10.85	0\\
10.86	0\\
10.87	0\\
10.88	0\\
10.89	0\\
10.9	0\\
10.91	0\\
10.92	0\\
10.93	0\\
10.94	0\\
10.95	0\\
10.96	0\\
10.97	0\\
10.98	0\\
10.99	0\\
11	0\\
11.01	0\\
11.02	0\\
11.03	0\\
11.04	0\\
11.05	0\\
11.06	0\\
11.07	0\\
11.08	0\\
11.09	0\\
11.1	0\\
11.11	0\\
11.12	0\\
11.13	0\\
11.14	0\\
11.15	0\\
11.16	0\\
11.17	0\\
11.18	0\\
11.19	0\\
11.2	0\\
11.21	0\\
11.22	0\\
11.23	0\\
11.24	0\\
11.25	0\\
11.26	0\\
11.27	0\\
11.28	0\\
11.29	0\\
11.3	0\\
11.31	0\\
11.32	0\\
11.33	0\\
11.34	0\\
11.35	0\\
11.36	0\\
11.37	0\\
11.38	0\\
11.39	0\\
11.4	0\\
11.41	0\\
11.42	0\\
11.43	0\\
11.44	0\\
11.45	0\\
11.46	0\\
11.47	0\\
11.48	0\\
11.49	0\\
11.5	0\\
11.51	0\\
11.52	0\\
11.53	0\\
11.54	0\\
11.55	0\\
11.56	0\\
11.57	0\\
11.58	0\\
11.59	0\\
11.6	0\\
11.61	0\\
11.62	0\\
11.63	0\\
11.64	0\\
11.65	0\\
11.66	0\\
11.67	0\\
11.68	0\\
11.69	0\\
11.7	0\\
11.71	0\\
11.72	0\\
11.73	0\\
11.74	0\\
11.75	0\\
11.76	0\\
11.77	0\\
11.78	0\\
11.79	0\\
11.8	0\\
11.81	0\\
11.82	0\\
11.83	0\\
11.84	0\\
11.85	0\\
11.86	0\\
11.87	0\\
11.88	0\\
11.89	0\\
11.9	0\\
11.91	0\\
11.92	0\\
11.93	0\\
11.94	0\\
11.95	0\\
11.96	0\\
11.97	0\\
11.98	0\\
11.99	0\\
12	0\\
12.01	0\\
12.02	0\\
12.03	0\\
12.04	0\\
12.05	0\\
12.06	0\\
12.07	0\\
12.08	0\\
12.09	0\\
12.1	0\\
12.11	0\\
12.12	0\\
12.13	0\\
12.14	0\\
12.15	0\\
12.16	0\\
12.17	0\\
12.18	0\\
12.19	0\\
12.2	0\\
12.21	0\\
12.22	0\\
12.23	0\\
12.24	0\\
12.25	0\\
12.26	0\\
12.27	0\\
12.28	0\\
12.29	0\\
12.3	0\\
12.31	0\\
12.32	0\\
12.33	0\\
12.34	0\\
12.35	0\\
12.36	0\\
12.37	0\\
12.38	0\\
12.39	0\\
12.4	0\\
12.41	0\\
12.42	0\\
12.43	0\\
12.44	0\\
12.45	0\\
12.46	0\\
12.47	0\\
12.48	0\\
12.49	0\\
12.5	0\\
12.51	0\\
12.52	0\\
12.53	0\\
12.54	0\\
12.55	0\\
12.56	0\\
12.57	0\\
12.58	0\\
12.59	0\\
12.6	0\\
12.61	0\\
12.62	0\\
12.63	0\\
12.64	0\\
12.65	0\\
12.66	0\\
12.67	0\\
12.68	0\\
12.69	0\\
12.7	0\\
12.71	0\\
12.72	0\\
12.73	0\\
12.74	0\\
12.75	0\\
12.76	0\\
12.77	0\\
12.78	0\\
12.79	0\\
12.8	0\\
12.81	0\\
12.82	0\\
12.83	0\\
12.84	0\\
12.85	0\\
12.86	0\\
12.87	0\\
12.88	0\\
12.89	0\\
12.9	0\\
12.91	0\\
12.92	0\\
12.93	0\\
12.94	0\\
12.95	0\\
12.96	0\\
12.97	0\\
12.98	0\\
12.99	0\\
13	0\\
13.01	0\\
13.02	0\\
13.03	0\\
13.04	0\\
13.05	0\\
13.06	0\\
13.07	0\\
13.08	0\\
13.09	0\\
13.1	0\\
13.11	0\\
13.12	0\\
13.13	0\\
13.14	0\\
13.15	0\\
13.16	0\\
13.17	0\\
13.18	0\\
13.19	0\\
13.2	0\\
13.21	0\\
13.22	0\\
13.23	0\\
13.24	0\\
13.25	0\\
13.26	0\\
13.27	0\\
13.28	0\\
13.29	0\\
13.3	0\\
13.31	0\\
13.32	0\\
13.33	0\\
13.34	0\\
13.35	0\\
13.36	0\\
13.37	0\\
13.38	0\\
13.39	0\\
13.4	0\\
13.41	0\\
13.42	0\\
13.43	0\\
13.44	0\\
13.45	0\\
13.46	0\\
13.47	0\\
13.48	0\\
13.49	0\\
13.5	0\\
13.51	0\\
13.52	0\\
13.53	0\\
13.54	0\\
13.55	0\\
13.56	0\\
13.57	0\\
13.58	0\\
13.59	0\\
13.6	0\\
13.61	0\\
13.62	0\\
13.63	0\\
13.64	0\\
13.65	0\\
13.66	0\\
13.67	0\\
13.68	0\\
13.69	0\\
13.7	0\\
13.71	0\\
13.72	0\\
13.73	0\\
13.74	0\\
13.75	0\\
13.76	0\\
13.77	0\\
13.78	0\\
13.79	0\\
13.8	0\\
13.81	0\\
13.82	0\\
13.83	0\\
13.84	0\\
13.85	0\\
13.86	0\\
13.87	0\\
13.88	0\\
13.89	0\\
13.9	0\\
13.91	0\\
13.92	0\\
13.93	0\\
13.94	0\\
13.95	0\\
13.96	0\\
13.97	0\\
13.98	0\\
13.99	0\\
14	0\\
14.01	0\\
14.02	0\\
14.03	0\\
14.04	0\\
14.05	0\\
14.06	0\\
14.07	0\\
14.08	0\\
14.09	0\\
14.1	0\\
14.11	0\\
14.12	0\\
14.13	0\\
14.14	0\\
14.15	0\\
14.16	0\\
14.17	0\\
14.18	0\\
14.19	0\\
14.2	0\\
14.21	0\\
14.22	0\\
14.23	0\\
14.24	0\\
14.25	0\\
14.26	0\\
14.27	0\\
14.28	0\\
14.29	0\\
14.3	0\\
14.31	0\\
14.32	0\\
14.33	0\\
14.34	0\\
14.35	0\\
14.36	0\\
14.37	0\\
14.38	0\\
14.39	0\\
14.4	0\\
14.41	0\\
14.42	0\\
14.43	0\\
14.44	0\\
14.45	0\\
14.46	0\\
14.47	0\\
14.48	0\\
14.49	0\\
14.5	0\\
14.51	0\\
14.52	0\\
14.53	0\\
14.54	0\\
14.55	0\\
14.56	0\\
14.57	0\\
14.58	0\\
14.59	0\\
14.6	0\\
14.61	0\\
14.62	0\\
14.63	0\\
14.64	0\\
14.65	0\\
14.66	0\\
14.67	0\\
14.68	0\\
14.69	0\\
14.7	0\\
14.71	0\\
14.72	0\\
14.73	0\\
14.74	0\\
14.75	0\\
14.76	0\\
14.77	0\\
14.78	0\\
14.79	0\\
14.8	0\\
14.81	0\\
14.82	0\\
14.83	0\\
14.84	0\\
14.85	0\\
14.86	0\\
14.87	0\\
14.88	0\\
14.89	0\\
14.9	0\\
14.91	0\\
14.92	0\\
14.93	0\\
14.94	0\\
14.95	0\\
14.96	0\\
14.97	0\\
14.98	0\\
14.99	0\\
15	0\\
15.01	0\\
15.02	0\\
15.03	0\\
15.04	0\\
15.05	0\\
15.06	0\\
15.07	0\\
15.08	0\\
15.09	0\\
15.1	0\\
15.11	0\\
15.12	0\\
15.13	0\\
15.14	0\\
15.15	0\\
15.16	0\\
15.17	0\\
15.18	0\\
15.19	0\\
15.2	0\\
15.21	0\\
15.22	0\\
15.23	0\\
15.24	0\\
15.25	0\\
15.26	0\\
15.27	0\\
15.28	0\\
15.29	0\\
15.3	0\\
15.31	0\\
15.32	0\\
15.33	0\\
15.34	0\\
15.35	0\\
15.36	0\\
15.37	0\\
15.38	0\\
15.39	0\\
15.4	0\\
15.41	0\\
15.42	0\\
15.43	0\\
15.44	0\\
15.45	0\\
15.46	0\\
15.47	0\\
15.48	0\\
15.49	0\\
15.5	0\\
15.51	0\\
15.52	0\\
15.53	0\\
15.54	0\\
15.55	0\\
15.56	0\\
15.57	0\\
15.58	0\\
15.59	0\\
15.6	0\\
15.61	0\\
15.62	0\\
15.63	0\\
15.64	0\\
15.65	0\\
15.66	0\\
15.67	0\\
15.68	0\\
15.69	0\\
15.7	0\\
15.71	0\\
15.72	0\\
15.73	0\\
15.74	0\\
15.75	0\\
15.76	0\\
15.77	0\\
15.78	0\\
15.79	0\\
15.8	0\\
15.81	0\\
15.82	0\\
15.83	0\\
15.84	0\\
15.85	0\\
15.86	0\\
15.87	0\\
15.88	0\\
15.89	0\\
15.9	0\\
15.91	0\\
15.92	0\\
15.93	0\\
15.94	0\\
15.95	0\\
15.96	0\\
15.97	0\\
15.98	0\\
15.99	0\\
16	0\\
16.01	0\\
16.02	0\\
16.03	0\\
16.04	0\\
16.05	0\\
16.06	0\\
16.07	0\\
16.08	0\\
16.09	0\\
16.1	0\\
16.11	0\\
16.12	0\\
16.13	0\\
16.14	0\\
16.15	0\\
16.16	0\\
16.17	0\\
16.18	0\\
16.19	0\\
16.2	0\\
16.21	0\\
16.22	0\\
16.23	0\\
16.24	0\\
16.25	0\\
16.26	0\\
16.27	0\\
16.28	0\\
16.29	0\\
16.3	0\\
16.31	0\\
16.32	0\\
16.33	0\\
16.34	0\\
16.35	0\\
16.36	0\\
16.37	0\\
16.38	0\\
16.39	0\\
16.4	0\\
16.41	0\\
16.42	0\\
16.43	0\\
16.44	0\\
16.45	0\\
16.46	0\\
16.47	0\\
16.48	0\\
16.49	0\\
16.5	0\\
16.51	0\\
16.52	0\\
16.53	0\\
16.54	0\\
16.55	0\\
16.56	0\\
16.57	0\\
16.58	0\\
16.59	0\\
16.6	0\\
16.61	0\\
16.62	0\\
16.63	0\\
16.64	0\\
16.65	0\\
16.66	0\\
16.67	0\\
16.68	0\\
16.69	0\\
16.7	0\\
16.71	0\\
16.72	0\\
16.73	0\\
16.74	0\\
16.75	0\\
16.76	0\\
16.77	0\\
16.78	0\\
16.79	0\\
16.8	0\\
16.81	0\\
16.82	0\\
16.83	0\\
16.84	0\\
16.85	0\\
16.86	0\\
16.87	0\\
16.88	0\\
16.89	0\\
16.9	0\\
16.91	0\\
16.92	0\\
16.93	0\\
16.94	0\\
16.95	0\\
16.96	0\\
16.97	0\\
16.98	0\\
16.99	0\\
17	0\\
17.01	0\\
17.02	0\\
17.03	0\\
17.04	0\\
17.05	0\\
17.06	0\\
17.07	0\\
17.08	0\\
17.09	0\\
17.1	0\\
17.11	0\\
17.12	0\\
17.13	0\\
17.14	0\\
17.15	0\\
17.16	0\\
17.17	0\\
17.18	0\\
17.19	0\\
17.2	0\\
17.21	0\\
17.22	0\\
17.23	0\\
17.24	0\\
17.25	0\\
17.26	0\\
17.27	0\\
17.28	0\\
17.29	0\\
17.3	0\\
17.31	0\\
17.32	0\\
17.33	0\\
17.34	0\\
17.35	0\\
17.36	0\\
17.37	0\\
17.38	0\\
17.39	0\\
17.4	0\\
17.41	0\\
17.42	0\\
17.43	0\\
17.44	0\\
17.45	0\\
17.46	0\\
17.47	0\\
17.48	0\\
17.49	0\\
17.5	0\\
17.51	0\\
17.52	0\\
17.53	0\\
17.54	0\\
17.55	0\\
17.56	0\\
17.57	0\\
17.58	0\\
17.59	0\\
17.6	0\\
17.61	0\\
17.62	0\\
17.63	0\\
17.64	0\\
17.65	0\\
17.66	0\\
17.67	0\\
17.68	0\\
17.69	0\\
17.7	0\\
17.71	0\\
17.72	0\\
17.73	0\\
17.74	0\\
17.75	0\\
17.76	0\\
17.77	0\\
17.78	0\\
17.79	0\\
17.8	0\\
17.81	0\\
17.82	0\\
17.83	0\\
17.84	0\\
17.85	0\\
17.86	0\\
17.87	0\\
17.88	0\\
17.89	0\\
17.9	0\\
17.91	0\\
17.92	0\\
17.93	0\\
17.94	0\\
17.95	0\\
17.96	0\\
17.97	0\\
17.98	0\\
17.99	0\\
18	0\\
18.01	0\\
18.02	0\\
18.03	0\\
18.04	0\\
18.05	0\\
18.06	0\\
18.07	0\\
18.08	0\\
18.09	0\\
18.1	0\\
18.11	0\\
18.12	0\\
18.13	0\\
18.14	0\\
18.15	0\\
18.16	0\\
18.17	0\\
18.18	0\\
18.19	0\\
18.2	0\\
18.21	0\\
18.22	0\\
18.23	0\\
18.24	0\\
18.25	0\\
18.26	0\\
18.27	0\\
18.28	0\\
18.29	0\\
18.3	0\\
18.31	0\\
18.32	0\\
18.33	0\\
18.34	0\\
18.35	0\\
18.36	0\\
18.37	0\\
18.38	0\\
18.39	0\\
18.4	0\\
18.41	0\\
18.42	0\\
18.43	0\\
18.44	0\\
18.45	0\\
18.46	0\\
18.47	0\\
18.48	0\\
18.49	0\\
18.5	0\\
18.51	0\\
18.52	0\\
18.53	0\\
18.54	0\\
18.55	0\\
18.56	0\\
18.57	0\\
18.58	0\\
18.59	0\\
18.6	0\\
18.61	0\\
18.62	0\\
18.63	0\\
18.64	0\\
18.65	0\\
18.66	0\\
18.67	0\\
18.68	0\\
18.69	0\\
18.7	0\\
18.71	0\\
18.72	0\\
18.73	0\\
18.74	0\\
18.75	0\\
18.76	0\\
18.77	0\\
18.78	0\\
18.79	0\\
18.8	0\\
18.81	0\\
18.82	0\\
18.83	0\\
18.84	0\\
18.85	0\\
18.86	0\\
18.87	0\\
18.88	0\\
18.89	0\\
18.9	0\\
18.91	0\\
18.92	0\\
18.93	0\\
18.94	0\\
18.95	0\\
18.96	0\\
18.97	0\\
18.98	0\\
18.99	0\\
19	0\\
19.01	0\\
19.02	0\\
19.03	0\\
19.04	0\\
19.05	0\\
19.06	0\\
19.07	0\\
19.08	0\\
19.09	0\\
19.1	0\\
19.11	0\\
19.12	0\\
19.13	0\\
19.14	0\\
19.15	0\\
19.16	0\\
19.17	0\\
19.18	0\\
19.19	0\\
19.2	0\\
19.21	0\\
19.22	0\\
19.23	0\\
19.24	0\\
19.25	0\\
19.26	0\\
19.27	0\\
19.28	0\\
19.29	0\\
19.3	0\\
19.31	0\\
19.32	0\\
19.33	0\\
19.34	0\\
19.35	0\\
19.36	0\\
19.37	0\\
19.38	0\\
19.39	0\\
19.4	0\\
19.41	0\\
19.42	0\\
19.43	0\\
19.44	0\\
19.45	0\\
19.46	0\\
19.47	0\\
19.48	0\\
19.49	0\\
19.5	0\\
19.51	0\\
19.52	0\\
19.53	0\\
19.54	0\\
19.55	0\\
19.56	0\\
19.57	0\\
19.58	0\\
19.59	0\\
19.6	0\\
19.61	0\\
19.62	0\\
19.63	0\\
19.64	0\\
19.65	0\\
19.66	0\\
19.67	0\\
19.68	0\\
19.69	0\\
19.7	0\\
19.71	0\\
19.72	0\\
19.73	0\\
19.74	0\\
19.75	0\\
19.76	0\\
19.77	0\\
19.78	0\\
19.79	0\\
19.8	0\\
19.81	0\\
19.82	0\\
19.83	0\\
19.84	0\\
19.85	0\\
19.86	0\\
19.87	0\\
19.88	0\\
19.89	0\\
19.9	0\\
19.91	0\\
19.92	0\\
19.93	0\\
19.94	0\\
19.95	0\\
19.96	0\\
19.97	0\\
19.98	0\\
19.99	0\\
20	0\\
20.01	0\\
20.02	0\\
20.03	0\\
20.04	0\\
20.05	0\\
20.06	0\\
20.07	0\\
20.08	0\\
20.09	0\\
20.1	0\\
20.11	0\\
20.12	0\\
20.13	0\\
20.14	0\\
20.15	0\\
20.16	0\\
20.17	0\\
20.18	0\\
20.19	0\\
20.2	0\\
20.21	0\\
20.22	0\\
20.23	0\\
20.24	0\\
20.25	0\\
20.26	0\\
20.27	0\\
20.28	0\\
20.29	0\\
20.3	0\\
20.31	0\\
20.32	0\\
20.33	0\\
20.34	0\\
20.35	0\\
20.36	0\\
20.37	0\\
20.38	0\\
20.39	0\\
20.4	0\\
20.41	0\\
20.42	0\\
20.43	0\\
20.44	0\\
20.45	0\\
20.46	0\\
20.47	0\\
20.48	0\\
20.49	0\\
20.5	0\\
20.51	0\\
20.52	0\\
20.53	0\\
20.54	0\\
20.55	0\\
20.56	0\\
20.57	0\\
20.58	0\\
20.59	0\\
20.6	0\\
20.61	0\\
20.62	0\\
20.63	0\\
20.64	0\\
20.65	0\\
20.66	0\\
20.67	0\\
20.68	0\\
20.69	0\\
20.7	0\\
20.71	0\\
20.72	0\\
20.73	0\\
20.74	0\\
20.75	0\\
20.76	0\\
20.77	0\\
20.78	0\\
20.79	0\\
20.8	0\\
20.81	0\\
20.82	0\\
20.83	0\\
20.84	0\\
20.85	0\\
20.86	0\\
20.87	0\\
20.88	0\\
20.89	0\\
20.9	0\\
20.91	0\\
20.92	0\\
20.93	0\\
20.94	0\\
20.95	0\\
20.96	0\\
20.97	0\\
20.98	0\\
20.99	0\\
21	0\\
21.01	0\\
21.02	0\\
21.03	0\\
21.04	0\\
21.05	0\\
21.06	0\\
21.07	0\\
21.08	0\\
21.09	0\\
21.1	0\\
21.11	0\\
21.12	0\\
21.13	0\\
21.14	0\\
21.15	0\\
21.16	0\\
21.17	0\\
21.18	0\\
21.19	0\\
21.2	0\\
21.21	0\\
21.22	0\\
21.23	0\\
21.24	0\\
21.25	0\\
21.26	0\\
21.27	0\\
21.28	0\\
21.29	0\\
21.3	0\\
21.31	0\\
21.32	0\\
21.33	0\\
21.34	0\\
21.35	0\\
21.36	0\\
21.37	0\\
21.38	0\\
21.39	0\\
21.4	0\\
21.41	0\\
21.42	0\\
21.43	0\\
21.44	0\\
21.45	0\\
21.46	0\\
21.47	0\\
21.48	0\\
21.49	0\\
21.5	0\\
21.51	0\\
21.52	0\\
21.53	0\\
21.54	0\\
21.55	0\\
21.56	0\\
21.57	0\\
21.58	0\\
21.59	0\\
21.6	0\\
21.61	0\\
21.62	0\\
21.63	0\\
21.64	0\\
21.65	0\\
21.66	0\\
21.67	0\\
21.68	0\\
21.69	0\\
21.7	0\\
21.71	0\\
21.72	0\\
21.73	0\\
21.74	0\\
21.75	0\\
21.76	0\\
21.77	0\\
21.78	0\\
21.79	0\\
21.8	0\\
21.81	0\\
21.82	0\\
21.83	0\\
21.84	0\\
21.85	0\\
21.86	0\\
21.87	0\\
21.88	0\\
21.89	0\\
21.9	0\\
21.91	0\\
21.92	0\\
21.93	0\\
21.94	0\\
21.95	0\\
21.96	0\\
21.97	0\\
21.98	0\\
21.99	0\\
22	0\\
22.01	0\\
22.02	0\\
22.03	0\\
22.04	0\\
22.05	0\\
22.06	0\\
22.07	0\\
22.08	0\\
22.09	0\\
22.1	0\\
22.11	0\\
22.12	0\\
22.13	0\\
22.14	0\\
22.15	0\\
22.16	0\\
22.17	0\\
22.18	0\\
22.19	0\\
22.2	0\\
22.21	0\\
22.22	0\\
22.23	0\\
22.24	0\\
22.25	0\\
22.26	0\\
22.27	0\\
22.28	0\\
22.29	0\\
22.3	0\\
22.31	0\\
22.32	0\\
22.33	0\\
22.34	0\\
22.35	0\\
22.36	0\\
22.37	0\\
22.38	0\\
22.39	0\\
22.4	0\\
22.41	0\\
22.42	0\\
22.43	0\\
22.44	0\\
22.45	0\\
22.46	0\\
22.47	0\\
22.48	0\\
22.49	0\\
22.5	0\\
22.51	0\\
22.52	0\\
22.53	0\\
22.54	0\\
22.55	0\\
22.56	0\\
22.57	0\\
22.58	0\\
22.59	0\\
22.6	0\\
22.61	0\\
22.62	0\\
22.63	0\\
22.64	0\\
22.65	0\\
22.66	0\\
22.67	0\\
22.68	0\\
22.69	0\\
22.7	0\\
22.71	0\\
22.72	0\\
22.73	0\\
22.74	0\\
22.75	0\\
22.76	0\\
22.77	0\\
22.78	0\\
22.79	0\\
22.8	0\\
22.81	0\\
22.82	0\\
22.83	0\\
22.84	0\\
22.85	0\\
22.86	0\\
22.87	0\\
22.88	0\\
22.89	0\\
22.9	0\\
22.91	0\\
22.92	0\\
22.93	0\\
22.94	0\\
22.95	0\\
22.96	0\\
22.97	0\\
22.98	0\\
22.99	0\\
23	0\\
23.01	0\\
23.02	0\\
23.03	0\\
23.04	0\\
23.05	0\\
23.06	0\\
23.07	0\\
23.08	0\\
23.09	0\\
23.1	0\\
23.11	0\\
23.12	0\\
23.13	0\\
23.14	0\\
23.15	0\\
23.16	0\\
23.17	0\\
23.18	0\\
23.19	0\\
23.2	0\\
23.21	0\\
23.22	0\\
23.23	0\\
23.24	0\\
23.25	0\\
23.26	0\\
23.27	0\\
23.28	0\\
23.29	0\\
23.3	0\\
23.31	0\\
23.32	0\\
23.33	0\\
23.34	0\\
23.35	0\\
23.36	0\\
23.37	0\\
23.38	0\\
23.39	0\\
23.4	0\\
23.41	0\\
23.42	0\\
23.43	0\\
23.44	0\\
23.45	0\\
23.46	0\\
23.47	0\\
23.48	0\\
23.49	0\\
23.5	0\\
23.51	0\\
23.52	0\\
23.53	0\\
23.54	0\\
23.55	0\\
23.56	0\\
23.57	0\\
23.58	0\\
23.59	0\\
23.6	0\\
23.61	0\\
23.62	0\\
23.63	0\\
23.64	0\\
23.65	0\\
23.66	0\\
23.67	0\\
23.68	0\\
23.69	0\\
23.7	0\\
23.71	0\\
23.72	0\\
23.73	0\\
23.74	0\\
23.75	0\\
23.76	0\\
23.77	0\\
23.78	0\\
23.79	0\\
23.8	0\\
23.81	0\\
23.82	0\\
23.83	0\\
23.84	0\\
23.85	0\\
23.86	0\\
23.87	0\\
23.88	0\\
23.89	0\\
23.9	0\\
23.91	0\\
23.92	0\\
23.93	0\\
23.94	0\\
23.95	0\\
23.96	0\\
23.97	0\\
23.98	0\\
23.99	0\\
24	0\\
24.01	0\\
24.02	0\\
24.03	0\\
24.04	0\\
24.05	0\\
24.06	0\\
24.07	0\\
24.08	0\\
24.09	0\\
24.1	0\\
24.11	0\\
24.12	0\\
24.13	0\\
24.14	0\\
24.15	0\\
24.16	0\\
24.17	0\\
24.18	0\\
24.19	0\\
24.2	0\\
24.21	0\\
24.22	0\\
24.23	0\\
24.24	0\\
24.25	0\\
24.26	0\\
24.27	0\\
24.28	0\\
24.29	0\\
24.3	0\\
24.31	0\\
24.32	0\\
24.33	0\\
24.34	0\\
24.35	0\\
24.36	0\\
24.37	0\\
24.38	0\\
24.39	0\\
24.4	0\\
24.41	0\\
24.42	0\\
24.43	0\\
24.44	0\\
24.45	0\\
24.46	0\\
24.47	0\\
24.48	0\\
24.49	0\\
24.5	0\\
24.51	0\\
24.52	0\\
24.53	0\\
24.54	0\\
24.55	0\\
24.56	0\\
24.57	0\\
24.58	0\\
24.59	0\\
24.6	0\\
24.61	0\\
24.62	0\\
24.63	0\\
24.64	0\\
24.65	0\\
24.66	0\\
24.67	0\\
24.68	0\\
24.69	0\\
24.7	0\\
24.71	0\\
24.72	0\\
24.73	0\\
24.74	0\\
24.75	0\\
24.76	0\\
24.77	0\\
24.78	0\\
24.79	0\\
24.8	0\\
24.81	0\\
24.82	0\\
24.83	0\\
24.84	0\\
24.85	0\\
24.86	0\\
24.87	0\\
24.88	0\\
24.89	0\\
24.9	0\\
24.91	0\\
24.92	0\\
24.93	0\\
24.94	0\\
24.95	0\\
24.96	0\\
24.97	0\\
24.98	0\\
24.99	0\\
25	0\\
25.01	0\\
25.02	0\\
25.03	0\\
25.04	0\\
25.05	0\\
25.06	0\\
25.07	0\\
25.08	0\\
25.09	0\\
25.1	0\\
25.11	0\\
25.12	0\\
25.13	0\\
25.14	0\\
25.15	0\\
25.16	0\\
25.17	0\\
25.18	0\\
25.19	0\\
25.2	0\\
25.21	0\\
25.22	0\\
25.23	0\\
25.24	0\\
25.25	0\\
25.26	0\\
25.27	0\\
25.28	0\\
25.29	0\\
25.3	0\\
25.31	0\\
25.32	0\\
25.33	0\\
25.34	0\\
25.35	0\\
25.36	0\\
25.37	0\\
25.38	0\\
25.39	0\\
25.4	0\\
25.41	0\\
25.42	0\\
25.43	0\\
25.44	0\\
25.45	0\\
25.46	0\\
25.47	0\\
25.48	0\\
25.49	0\\
25.5	0\\
25.51	0\\
25.52	0\\
25.53	0\\
25.54	0\\
25.55	0\\
25.56	0\\
25.57	0\\
25.58	0\\
25.59	0\\
25.6	0\\
25.61	0\\
25.62	0\\
25.63	0\\
25.64	0\\
25.65	0\\
25.66	0\\
25.67	0\\
25.68	0\\
25.69	0\\
25.7	0\\
25.71	0\\
25.72	0\\
25.73	0\\
25.74	0\\
25.75	0\\
25.76	0\\
25.77	0\\
25.78	0\\
25.79	0\\
25.8	0\\
25.81	0\\
25.82	0\\
25.83	0\\
25.84	0\\
25.85	0\\
25.86	0\\
25.87	0\\
25.88	0\\
25.89	0\\
25.9	0\\
25.91	0\\
25.92	0\\
25.93	0\\
25.94	0\\
25.95	0\\
25.96	0\\
25.97	0\\
25.98	0\\
25.99	0\\
26	0\\
26.01	0\\
26.02	0\\
26.03	0\\
26.04	0\\
26.05	0\\
26.06	0\\
26.07	0\\
26.08	0\\
26.09	0\\
26.1	0\\
26.11	0\\
26.12	0\\
26.13	0\\
26.14	0\\
26.15	0\\
26.16	0\\
26.17	0\\
26.18	0\\
26.19	0\\
26.2	0\\
26.21	0\\
26.22	0\\
26.23	0\\
26.24	0\\
26.25	0\\
26.26	0\\
26.27	0\\
26.28	0\\
26.29	0\\
26.3	0\\
26.31	0\\
26.32	0\\
26.33	0\\
26.34	0\\
26.35	0\\
26.36	0\\
26.37	0\\
26.38	0\\
26.39	0\\
26.4	0\\
26.41	0\\
26.42	0\\
26.43	0\\
26.44	0\\
26.45	0\\
26.46	0\\
26.47	0\\
26.48	0\\
26.49	0\\
26.5	0\\
26.51	0\\
26.52	0\\
26.53	0\\
26.54	0\\
26.55	0\\
26.56	0\\
26.57	0\\
26.58	0\\
26.59	0\\
26.6	0\\
26.61	0\\
26.62	0\\
26.63	0\\
26.64	0\\
26.65	0\\
26.66	0\\
26.67	0\\
26.68	0\\
26.69	0\\
26.7	0\\
26.71	0\\
26.72	0\\
26.73	0\\
26.74	0\\
26.75	0\\
26.76	0\\
26.77	0\\
26.78	0\\
26.79	0\\
26.8	0\\
26.81	0\\
26.82	0\\
26.83	0\\
26.84	0\\
26.85	0\\
26.86	0\\
26.87	0\\
26.88	0\\
26.89	0\\
26.9	0\\
26.91	0\\
26.92	0\\
26.93	0\\
26.94	0\\
26.95	0\\
26.96	0\\
26.97	0\\
26.98	0\\
26.99	0\\
27	0\\
27.01	0\\
27.02	0\\
27.03	0\\
27.04	0\\
27.05	0\\
27.06	0\\
27.07	0\\
27.08	0\\
27.09	0\\
27.1	0\\
27.11	0\\
27.12	0\\
27.13	0\\
27.14	0\\
27.15	0\\
27.16	0\\
27.17	0\\
27.18	0\\
27.19	0\\
27.2	0\\
27.21	0\\
27.22	0\\
27.23	0\\
27.24	0\\
27.25	0\\
27.26	0\\
27.27	0\\
27.28	0\\
27.29	0\\
27.3	0\\
27.31	0\\
27.32	0\\
27.33	0\\
27.34	0\\
27.35	0\\
27.36	0\\
27.37	0\\
27.38	0\\
27.39	0\\
27.4	0\\
27.41	0\\
27.42	0\\
27.43	0\\
27.44	0\\
27.45	0\\
27.46	0\\
27.47	0\\
27.48	0\\
27.49	0\\
27.5	0\\
27.51	0\\
27.52	0\\
27.53	0\\
27.54	0\\
27.55	0\\
27.56	0\\
27.57	0\\
27.58	0\\
27.59	0\\
27.6	0\\
27.61	0\\
27.62	0\\
27.63	0\\
27.64	0\\
27.65	0\\
27.66	0\\
27.67	0\\
27.68	0\\
27.69	0\\
27.7	0\\
27.71	0\\
27.72	0\\
27.73	0\\
27.74	0\\
27.75	0\\
27.76	0\\
27.77	0\\
27.78	0\\
27.79	0\\
27.8	0\\
27.81	0\\
27.82	0\\
27.83	0\\
27.84	0\\
27.85	0\\
27.86	0\\
27.87	0\\
27.88	0\\
27.89	0\\
27.9	0\\
27.91	0\\
27.92	0\\
27.93	0\\
27.94	0\\
27.95	0\\
27.96	0\\
27.97	0\\
27.98	0\\
27.99	0\\
28	0\\
28.01	0\\
28.02	0\\
28.03	0\\
28.04	0\\
28.05	0\\
28.06	0\\
28.07	0\\
28.08	0\\
28.09	0\\
28.1	0\\
28.11	0\\
28.12	0\\
28.13	0\\
28.14	0\\
28.15	0\\
28.16	0\\
28.17	0\\
28.18	0\\
28.19	0\\
28.2	0\\
28.21	0\\
28.22	0\\
28.23	0\\
28.24	0\\
28.25	0\\
28.26	0\\
28.27	0\\
28.28	0\\
28.29	0\\
28.3	0\\
28.31	0\\
28.32	0\\
28.33	0\\
28.34	0\\
28.35	0\\
28.36	0\\
28.37	0\\
28.38	0\\
28.39	0\\
28.4	0\\
28.41	0\\
28.42	0\\
28.43	0\\
28.44	0\\
28.45	0\\
28.46	0\\
28.47	0\\
28.48	0\\
28.49	0\\
28.5	0\\
28.51	0\\
28.52	0\\
28.53	0\\
28.54	0\\
28.55	0\\
28.56	0\\
28.57	0\\
28.58	0\\
28.59	0\\
28.6	0\\
28.61	0\\
28.62	0\\
28.63	0\\
28.64	0\\
28.65	0\\
28.66	0\\
28.67	0\\
28.68	0\\
28.69	0\\
28.7	0\\
28.71	0\\
28.72	0\\
28.73	0\\
28.74	0\\
28.75	0\\
28.76	0\\
28.77	0\\
28.78	0\\
28.79	0\\
28.8	0\\
28.81	0\\
28.82	0\\
28.83	0\\
28.84	0\\
28.85	0\\
28.86	0\\
28.87	0\\
28.88	0\\
28.89	0\\
28.9	0\\
28.91	0\\
28.92	0\\
28.93	0\\
28.94	0\\
28.95	0\\
28.96	0\\
28.97	0\\
28.98	0\\
28.99	0\\
29	0\\
29.01	0\\
29.02	0\\
29.03	0\\
29.04	0\\
29.05	0\\
29.06	0\\
29.07	0\\
29.08	0\\
29.09	0\\
29.1	0\\
29.11	0\\
29.12	0\\
29.13	0\\
29.14	0\\
29.15	0\\
29.16	0\\
29.17	0\\
29.18	0\\
29.19	0\\
29.2	0\\
29.21	0\\
29.22	0\\
29.23	0\\
29.24	0\\
29.25	0\\
29.26	0\\
29.27	0\\
29.28	0\\
29.29	0\\
29.3	0\\
29.31	0\\
29.32	0\\
29.33	0\\
29.34	0\\
29.35	0\\
29.36	0\\
29.37	0\\
29.38	0\\
29.39	0\\
29.4	0\\
29.41	0\\
29.42	0\\
29.43	0\\
29.44	0\\
29.45	0\\
29.46	0\\
29.47	0\\
29.48	0\\
29.49	0\\
29.5	0\\
29.51	0\\
29.52	0\\
29.53	0\\
29.54	0\\
29.55	0\\
29.56	0\\
29.57	0\\
29.58	0\\
29.59	0\\
29.6	0\\
29.61	0\\
29.62	0\\
29.63	0\\
29.64	0\\
29.65	0\\
29.66	0\\
29.67	0\\
29.68	0\\
29.69	0\\
29.7	0\\
29.71	0\\
29.72	0\\
29.73	0\\
29.74	0\\
29.75	0\\
29.76	0\\
29.77	0\\
29.78	0\\
29.79	0\\
29.8	0\\
29.81	0\\
29.82	0\\
29.83	0\\
29.84	0\\
29.85	0\\
29.86	0\\
29.87	0\\
29.88	0\\
29.89	0\\
29.9	0\\
29.91	0\\
29.92	0\\
29.93	0\\
29.94	0\\
29.95	0\\
29.96	0\\
29.97	0\\
29.98	0\\
29.99	0\\
30	0\\
30.01	0\\
30.02	0\\
30.03	0\\
30.04	0\\
30.05	0\\
30.06	0\\
30.07	0\\
30.08	0\\
30.09	0\\
30.1	0\\
30.11	0\\
30.12	0\\
30.13	0\\
30.14	0\\
30.15	0\\
30.16	0\\
30.17	0\\
30.18	0\\
30.19	0\\
30.2	0\\
30.21	0\\
30.22	0\\
30.23	0\\
30.24	0\\
30.25	0\\
30.26	0\\
30.27	0\\
30.28	0\\
30.29	0\\
30.3	0\\
30.31	0\\
30.32	0\\
30.33	0\\
30.34	0\\
30.35	0\\
30.36	0\\
30.37	0\\
30.38	0\\
30.39	0\\
30.4	0\\
30.41	0\\
30.42	0\\
30.43	0\\
30.44	0\\
30.45	0\\
30.46	0\\
30.47	0\\
30.48	0\\
30.49	0\\
30.5	0\\
30.51	0\\
30.52	0\\
30.53	0\\
30.54	0\\
30.55	0\\
30.56	0\\
30.57	0\\
30.58	0\\
30.59	0\\
30.6	0\\
30.61	0\\
30.62	0\\
30.63	0\\
30.64	0\\
30.65	0\\
30.66	0\\
30.67	0\\
30.68	0\\
30.69	0\\
30.7	0\\
30.71	0\\
30.72	0\\
30.73	0\\
30.74	0\\
30.75	0\\
30.76	0\\
30.77	0\\
30.78	0\\
30.79	0\\
30.8	0\\
30.81	0\\
30.82	0\\
30.83	0\\
30.84	0\\
30.85	0\\
30.86	0\\
30.87	0\\
30.88	0\\
30.89	0\\
30.9	0\\
30.91	0\\
30.92	0\\
30.93	0\\
30.94	0\\
30.95	0\\
30.96	0\\
30.97	0\\
30.98	0\\
30.99	0\\
31	0\\
31.01	0\\
31.02	0\\
31.03	0\\
31.04	0\\
31.05	0\\
31.06	0\\
31.07	0\\
31.08	0\\
31.09	0\\
31.1	0\\
31.11	0\\
31.12	0\\
31.13	0\\
31.14	0\\
31.15	0\\
31.16	0\\
31.17	0\\
31.18	0\\
31.19	0\\
31.2	0\\
31.21	0\\
31.22	0\\
31.23	0\\
31.24	0\\
31.25	0\\
31.26	0\\
31.27	0\\
31.28	0\\
31.29	0\\
31.3	0\\
31.31	0\\
31.32	0\\
31.33	0\\
31.34	0\\
31.35	0\\
31.36	0\\
31.37	0\\
31.38	0\\
31.39	0\\
31.4	0\\
31.41	0\\
31.42	0\\
31.43	0\\
31.44	0\\
31.45	0\\
31.46	0\\
31.47	0\\
31.48	0\\
31.49	0\\
31.5	0\\
31.51	0\\
31.52	0\\
31.53	0\\
31.54	0\\
31.55	0\\
31.56	0\\
31.57	0\\
31.58	0\\
31.59	0\\
31.6	0\\
31.61	0\\
31.62	0\\
31.63	0\\
31.64	0\\
31.65	0\\
31.66	0\\
31.67	0\\
31.68	0\\
31.69	0\\
31.7	0\\
31.71	0\\
31.72	0\\
31.73	0\\
31.74	0\\
31.75	0\\
31.76	0\\
31.77	0\\
31.78	0\\
31.79	0\\
31.8	0\\
31.81	0\\
31.82	0\\
31.83	0\\
31.84	0\\
31.85	0\\
31.86	0\\
31.87	0\\
31.88	0\\
31.89	0\\
31.9	0\\
31.91	0\\
31.92	0\\
31.93	0\\
31.94	0\\
31.95	0\\
31.96	0\\
31.97	0\\
31.98	0\\
31.99	0\\
32	0\\
32.01	0\\
32.02	0\\
32.03	0\\
32.04	0\\
32.05	0\\
32.06	0\\
32.07	0\\
32.08	0\\
32.09	0\\
32.1	0\\
32.11	0\\
32.12	0\\
32.13	0\\
32.14	0\\
32.15	0\\
32.16	0\\
32.17	0\\
32.18	0\\
32.19	0\\
32.2	0\\
32.21	0\\
32.22	0\\
32.23	0\\
32.24	0\\
32.25	0\\
32.26	0\\
32.27	0\\
32.28	0\\
32.29	0\\
32.3	0\\
32.31	0\\
32.32	0\\
32.33	0\\
32.34	0\\
32.35	0\\
32.36	0\\
32.37	0\\
32.38	0\\
32.39	0\\
32.4	0\\
32.41	0\\
32.42	0\\
32.43	0\\
32.44	0\\
32.45	0\\
32.46	0\\
32.47	0\\
32.48	0\\
32.49	0\\
32.5	0\\
32.51	0\\
32.52	0\\
32.53	0\\
32.54	0\\
32.55	0\\
32.56	0\\
32.57	0\\
32.58	0\\
32.59	0\\
32.6	0\\
32.61	0\\
32.62	0\\
32.63	0\\
32.64	0\\
32.65	0\\
32.66	0\\
32.67	0\\
32.68	0\\
32.69	0\\
32.7	0\\
32.71	0\\
32.72	0\\
32.73	0\\
32.74	0\\
32.75	0\\
32.76	0\\
32.77	0\\
32.78	0\\
32.79	0\\
32.8	0\\
32.81	0\\
32.82	0\\
32.83	0\\
32.84	0\\
32.85	0\\
32.86	0\\
32.87	0\\
32.88	0\\
32.89	0\\
32.9	0\\
32.91	0\\
32.92	0\\
32.93	0\\
32.94	0\\
32.95	0\\
32.96	0\\
32.97	0\\
32.98	0\\
32.99	0\\
33	0\\
33.01	0\\
33.02	0\\
33.03	0\\
33.04	0\\
33.05	0\\
33.06	0\\
33.07	0\\
33.08	0\\
33.09	0\\
33.1	0\\
33.11	0\\
33.12	0\\
33.13	0\\
33.14	0\\
33.15	0\\
33.16	0\\
33.17	0\\
33.18	0\\
33.19	0\\
33.2	0\\
33.21	0\\
33.22	0\\
33.23	0\\
33.24	0\\
33.25	0\\
33.26	0\\
33.27	0\\
33.28	0\\
33.29	0\\
33.3	0\\
33.31	0\\
33.32	0\\
33.33	0\\
33.34	0\\
33.35	0\\
33.36	0\\
33.37	0\\
33.38	0\\
33.39	0\\
33.4	0\\
33.41	0\\
33.42	0\\
33.43	0\\
33.44	0\\
33.45	0\\
33.46	0\\
33.47	0\\
33.48	0\\
33.49	0\\
33.5	0\\
33.51	0\\
33.52	0\\
33.53	0\\
33.54	0\\
33.55	0\\
33.56	0\\
33.57	0\\
33.58	0\\
33.59	0\\
33.6	0\\
33.61	0\\
33.62	0\\
33.63	0\\
33.64	0\\
33.65	0\\
33.66	0\\
33.67	0\\
33.68	0\\
33.69	0\\
33.7	0\\
33.71	0\\
33.72	0\\
33.73	0\\
33.74	0\\
33.75	0\\
33.76	0\\
33.77	0\\
33.78	0\\
33.79	0\\
33.8	0\\
33.81	0\\
33.82	0\\
33.83	0\\
33.84	0\\
33.85	0\\
33.86	0\\
33.87	0\\
33.88	0\\
33.89	0\\
33.9	0\\
33.91	0\\
33.92	0\\
33.93	0\\
33.94	0\\
33.95	0\\
33.96	0\\
33.97	0\\
33.98	0\\
33.99	0\\
34	0\\
34.01	0\\
34.02	0\\
34.03	0\\
34.04	0\\
34.05	0\\
34.06	0\\
34.07	0\\
34.08	0\\
34.09	0\\
34.1	0\\
34.11	0\\
34.12	0\\
34.13	0\\
34.14	0\\
34.15	0\\
34.16	0\\
34.17	0\\
34.18	0\\
34.19	0\\
34.2	0\\
34.21	0\\
34.22	0\\
34.23	0\\
34.24	0\\
34.25	0\\
34.26	0\\
34.27	0\\
34.28	0\\
34.29	0\\
34.3	0\\
34.31	0\\
34.32	0\\
34.33	0\\
34.34	0\\
34.35	0\\
34.36	0\\
34.37	0\\
34.38	0\\
34.39	0\\
34.4	0\\
34.41	0\\
34.42	0\\
34.43	0\\
34.44	0\\
34.45	0\\
34.46	0\\
34.47	0\\
34.48	0\\
34.49	0\\
34.5	0\\
34.51	0\\
34.52	0\\
34.53	0\\
34.54	0\\
34.55	0\\
34.56	0\\
34.57	0\\
34.58	0\\
34.59	0\\
34.6	0\\
34.61	0\\
34.62	0\\
34.63	0\\
34.64	0\\
34.65	0\\
34.66	0\\
34.67	0\\
34.68	0\\
34.69	0\\
34.7	0\\
34.71	0\\
34.72	0\\
34.73	0\\
34.74	0\\
34.75	0\\
34.76	0\\
34.77	0\\
34.78	0\\
34.79	0\\
34.8	0\\
34.81	0\\
34.82	0\\
34.83	0\\
34.84	0\\
34.85	0\\
34.86	0\\
34.87	0\\
34.88	0\\
34.89	0\\
34.9	0\\
34.91	0\\
34.92	0\\
34.93	0\\
34.94	0\\
34.95	0\\
34.96	0\\
34.97	0\\
34.98	0\\
34.99	0\\
35	0\\
35.01	0\\
35.02	0\\
35.03	0\\
35.04	0\\
35.05	0\\
35.06	0\\
35.07	0\\
35.08	0\\
35.09	0\\
35.1	0\\
35.11	0\\
35.12	0\\
35.13	0\\
35.14	0\\
35.15	0\\
35.16	0\\
35.17	0\\
35.18	0\\
35.19	0\\
35.2	0\\
35.21	0\\
35.22	0\\
35.23	0\\
35.24	0\\
35.25	0\\
35.26	0\\
35.27	0\\
35.28	0\\
35.29	0\\
35.3	0\\
35.31	0\\
35.32	0\\
35.33	0\\
35.34	0\\
35.35	0\\
35.36	0\\
35.37	0\\
35.38	0\\
35.39	0\\
35.4	0\\
35.41	0\\
35.42	0\\
35.43	0\\
35.44	0\\
35.45	0\\
35.46	0\\
35.47	0\\
35.48	0\\
35.49	0\\
35.5	0\\
35.51	0\\
35.52	0\\
35.53	0\\
35.54	0\\
35.55	0\\
35.56	0\\
35.57	0\\
35.58	0\\
35.59	0\\
35.6	0\\
35.61	0\\
35.62	0\\
35.63	0\\
35.64	0\\
35.65	0\\
35.66	0\\
35.67	0\\
35.68	0\\
35.69	0\\
35.7	0\\
35.71	0\\
35.72	0\\
35.73	0\\
35.74	0\\
35.75	0\\
35.76	0\\
35.77	0\\
35.78	0\\
35.79	0\\
35.8	0\\
35.81	0\\
35.82	0\\
35.83	0\\
35.84	0\\
35.85	0\\
35.86	0\\
35.87	0\\
35.88	0\\
35.89	0\\
35.9	0\\
35.91	0\\
35.92	0\\
35.93	0\\
35.94	0\\
35.95	0\\
35.96	0\\
35.97	0\\
35.98	0\\
35.99	0\\
36	0\\
36.01	0\\
36.02	0\\
36.03	0\\
36.04	0\\
36.05	0\\
36.06	0\\
36.07	0\\
36.08	0\\
36.09	0\\
36.1	0\\
36.11	0\\
36.12	0\\
36.13	0\\
36.14	0\\
36.15	0\\
36.16	0\\
36.17	0\\
36.18	0\\
36.19	0\\
36.2	0\\
36.21	0\\
36.22	0\\
36.23	0\\
36.24	0\\
36.25	0\\
36.26	0\\
36.27	0\\
36.28	0\\
36.29	0\\
36.3	0\\
36.31	0\\
36.32	0\\
36.33	0\\
36.34	0\\
36.35	0\\
36.36	0\\
36.37	0\\
36.38	0\\
36.39	0\\
36.4	0\\
36.41	0\\
36.42	0\\
36.43	0\\
36.44	0\\
36.45	0\\
36.46	0\\
36.47	0\\
36.48	0\\
36.49	0\\
36.5	0\\
36.51	0\\
36.52	0\\
36.53	0\\
36.54	0\\
36.55	0\\
36.56	0\\
36.57	0\\
36.58	0\\
36.59	0\\
36.6	0\\
36.61	0\\
36.62	0\\
36.63	0\\
36.64	0\\
36.65	0\\
36.66	0\\
36.67	0\\
36.68	0\\
36.69	0\\
36.7	0\\
36.71	0\\
36.72	0\\
36.73	0\\
36.74	0\\
36.75	0\\
36.76	0\\
36.77	0\\
36.78	0\\
36.79	0\\
36.8	0\\
36.81	0\\
36.82	0\\
36.83	0\\
36.84	0\\
36.85	0\\
36.86	0\\
36.87	0\\
36.88	0\\
36.89	0\\
36.9	0\\
36.91	0\\
36.92	0\\
36.93	0\\
36.94	0\\
36.95	0\\
36.96	0\\
36.97	0\\
36.98	0\\
36.99	0\\
37	0\\
37.01	0\\
37.02	0\\
37.03	0\\
37.04	0\\
37.05	0\\
37.06	0\\
37.07	0\\
37.08	0\\
37.09	0\\
37.1	0\\
37.11	0\\
37.12	0\\
37.13	0\\
37.14	0\\
37.15	0\\
37.16	0\\
37.17	0\\
37.18	0\\
37.19	0\\
37.2	0\\
37.21	0\\
37.22	0\\
37.23	0\\
37.24	0\\
37.25	0\\
37.26	0\\
37.27	0\\
37.28	0\\
37.29	0\\
37.3	0\\
37.31	0\\
37.32	0\\
37.33	0\\
37.34	0\\
37.35	0\\
37.36	0\\
37.37	0\\
37.38	0\\
37.39	0\\
37.4	0\\
37.41	0\\
37.42	0\\
37.43	0\\
37.44	0\\
37.45	0\\
37.46	0\\
37.47	0\\
37.48	0\\
37.49	0\\
37.5	0\\
37.51	0\\
37.52	0\\
37.53	0\\
37.54	0\\
37.55	0\\
37.56	0\\
37.57	0\\
37.58	0\\
37.59	0\\
37.6	0\\
37.61	0\\
37.62	0\\
37.63	0\\
37.64	0\\
37.65	0\\
37.66	0\\
37.67	0\\
37.68	0\\
37.69	0\\
37.7	0\\
37.71	0\\
37.72	0\\
37.73	0\\
37.74	0\\
37.75	0\\
37.76	0\\
37.77	0\\
37.78	0\\
37.79	0\\
37.8	0\\
37.81	0\\
37.82	0\\
37.83	0\\
37.84	0\\
37.85	0\\
37.86	0\\
37.87	0\\
37.88	0\\
37.89	0\\
37.9	0\\
37.91	0\\
37.92	0\\
37.93	0\\
37.94	0\\
37.95	0\\
37.96	0\\
37.97	0\\
37.98	0\\
37.99	0\\
38	0\\
38.01	0\\
38.02	0\\
38.03	0\\
38.04	0\\
38.05	0\\
38.06	0\\
38.07	0\\
38.08	0\\
38.09	0\\
38.1	0\\
38.11	0\\
38.12	0\\
38.13	0\\
38.14	0\\
38.15	0\\
38.16	0\\
38.17	0\\
38.18	0\\
38.19	0\\
38.2	0\\
38.21	0\\
38.22	0\\
38.23	0\\
38.24	0\\
38.25	0\\
38.26	0\\
38.27	0\\
38.28	0\\
38.29	0\\
38.3	0\\
38.31	0\\
38.32	0\\
38.33	0\\
38.34	0\\
38.35	0\\
38.36	0\\
38.37	0\\
38.38	0\\
38.39	0\\
38.4	0\\
38.41	0\\
38.42	0\\
38.43	0\\
38.44	0\\
38.45	0\\
38.46	0\\
38.47	0\\
38.48	0\\
38.49	0\\
38.5	0\\
38.51	0\\
38.52	0\\
38.53	0\\
38.54	0\\
38.55	0\\
38.56	0\\
38.57	0\\
38.58	0\\
38.59	0\\
38.6	0\\
38.61	0\\
38.62	0\\
38.63	0\\
38.64	0\\
38.65	0\\
38.66	0\\
38.67	0\\
38.68	0\\
38.69	0\\
38.7	0\\
38.71	0\\
38.72	0\\
38.73	0\\
38.74	0\\
38.75	0\\
38.76	0\\
38.77	0\\
38.78	0\\
38.79	0\\
38.8	0\\
38.81	0\\
38.82	0\\
38.83	0\\
38.84	0\\
38.85	0\\
38.86	0\\
38.87	0\\
38.88	0\\
38.89	0\\
38.9	0\\
38.91	0\\
38.92	0\\
38.93	0\\
38.94	0\\
38.95	0\\
38.96	0\\
38.97	0\\
38.98	0\\
38.99	0\\
39	0\\
39.01	0\\
39.02	0\\
39.03	0\\
39.04	0\\
39.05	0\\
39.06	0\\
39.07	0\\
39.08	0\\
39.09	0\\
39.1	0\\
39.11	0\\
39.12	0\\
39.13	0\\
39.14	0\\
39.15	0\\
39.16	0\\
39.17	0\\
39.18	0\\
39.19	0\\
39.2	0\\
39.21	0\\
39.22	0\\
39.23	0\\
39.24	0\\
39.25	0\\
39.26	0\\
39.27	0\\
39.28	0\\
39.29	0\\
39.3	0\\
39.31	0\\
39.32	0\\
39.33	0\\
39.34	0\\
39.35	0\\
39.36	0\\
39.37	0\\
39.38	0\\
39.39	0\\
39.4	0\\
39.41	0\\
39.42	0\\
39.43	0\\
39.44	0\\
39.45	0\\
39.46	0\\
39.47	0\\
39.48	0\\
39.49	0\\
39.5	0\\
39.51	0\\
39.52	0\\
39.53	0\\
39.54	0\\
39.55	0\\
39.56	0\\
39.57	0\\
39.58	0\\
39.59	0\\
39.6	0\\
39.61	0\\
39.62	0\\
39.63	0\\
39.64	0\\
39.65	0\\
39.66	0\\
39.67	0\\
39.68	0\\
39.69	0\\
39.7	0\\
39.71	0\\
39.72	0\\
39.73	0\\
39.74	0\\
39.75	0\\
39.76	0\\
39.77	0\\
39.78	0\\
39.79	0\\
39.8	0\\
39.81	0\\
39.82	0\\
39.83	0\\
39.84	0\\
39.85	0\\
39.86	0\\
39.87	0\\
39.88	0\\
39.89	0\\
39.9	0\\
39.91	0\\
39.92	0\\
39.93	0\\
39.94	0\\
39.95	0\\
39.96	0\\
39.97	0\\
39.98	0\\
39.99	0\\
40	0\\
40.01	0\\
};
\addplot [color=black,solid,forget plot]
  table[row sep=crcr]{%
40.01	0\\
40.02	0\\
40.03	0\\
40.04	0\\
40.05	0\\
40.06	0\\
40.07	0\\
40.08	0\\
40.09	0\\
40.1	0\\
40.11	0\\
40.12	0\\
40.13	0\\
40.14	0\\
40.15	0\\
40.16	0\\
40.17	0\\
40.18	0\\
40.19	0\\
40.2	0\\
40.21	0\\
40.22	0\\
40.23	0\\
40.24	0\\
40.25	0\\
40.26	0\\
40.27	0\\
40.28	0\\
40.29	0\\
40.3	0\\
40.31	0\\
40.32	0\\
40.33	0\\
40.34	0\\
40.35	0\\
40.36	0\\
40.37	0\\
40.38	0\\
40.39	0\\
40.4	0\\
40.41	0\\
40.42	0\\
40.43	0\\
40.44	0\\
40.45	0\\
40.46	0\\
40.47	0\\
40.48	0\\
40.49	0\\
40.5	0\\
40.51	0\\
40.52	0\\
40.53	0\\
40.54	0\\
40.55	0\\
40.56	0\\
40.57	0\\
40.58	0\\
40.59	0\\
40.6	0\\
40.61	0\\
40.62	0\\
40.63	0\\
40.64	0\\
40.65	0\\
40.66	0\\
40.67	0\\
40.68	0\\
40.69	0\\
40.7	0\\
40.71	0\\
40.72	0\\
40.73	0\\
40.74	0\\
40.75	0\\
40.76	0\\
40.77	0\\
40.78	0\\
40.79	0\\
40.8	0\\
40.81	0\\
40.82	0\\
40.83	0\\
40.84	0\\
40.85	0\\
40.86	0\\
40.87	0\\
40.88	0\\
40.89	0\\
40.9	0\\
40.91	0\\
40.92	0\\
40.93	0\\
40.94	0\\
40.95	0\\
40.96	0\\
40.97	0\\
40.98	0\\
40.99	0\\
41	0\\
41.01	0\\
41.02	0\\
41.03	0\\
41.04	0\\
41.05	0\\
41.06	0\\
41.07	0\\
41.08	0\\
41.09	0\\
41.1	0\\
41.11	0\\
41.12	0\\
41.13	0\\
41.14	0\\
41.15	0\\
41.16	0\\
41.17	0\\
41.18	0\\
41.19	0\\
41.2	0\\
41.21	0\\
41.22	0\\
41.23	0\\
41.24	0\\
41.25	0\\
41.26	0\\
41.27	0\\
41.28	0\\
41.29	0\\
41.3	0\\
41.31	0\\
41.32	0\\
41.33	0\\
41.34	0\\
41.35	0\\
41.36	0\\
41.37	0\\
41.38	0\\
41.39	0\\
41.4	0\\
41.41	0\\
41.42	0\\
41.43	0\\
41.44	0\\
41.45	0\\
41.46	0\\
41.47	0\\
41.48	0\\
41.49	0\\
41.5	0\\
41.51	0\\
41.52	0\\
41.53	0\\
41.54	0\\
41.55	0\\
41.56	0\\
41.57	0\\
41.58	0\\
41.59	0\\
41.6	0\\
41.61	0\\
41.62	0\\
41.63	0\\
41.64	0\\
41.65	0\\
41.66	0\\
41.67	0\\
41.68	0\\
41.69	0\\
41.7	0\\
41.71	0\\
41.72	0\\
41.73	0\\
41.74	0\\
41.75	0\\
41.76	0\\
41.77	0\\
41.78	0\\
41.79	0\\
41.8	0\\
41.81	0\\
41.82	0\\
41.83	0\\
41.84	0\\
41.85	0\\
41.86	0\\
41.87	0\\
41.88	0\\
41.89	0\\
41.9	0\\
41.91	0\\
41.92	0\\
41.93	0\\
41.94	0\\
41.95	0\\
41.96	0\\
41.97	0\\
41.98	0\\
41.99	0\\
42	0\\
42.01	0\\
42.02	0\\
42.03	0\\
42.04	0\\
42.05	0\\
42.06	0\\
42.07	0\\
42.08	0\\
42.09	0\\
42.1	0\\
42.11	0\\
42.12	0\\
42.13	0\\
42.14	0\\
42.15	0\\
42.16	0\\
42.17	0\\
42.18	0\\
42.19	0\\
42.2	0\\
42.21	0\\
42.22	0\\
42.23	0\\
42.24	0\\
42.25	0\\
42.26	0\\
42.27	0\\
42.28	0\\
42.29	0\\
42.3	0\\
42.31	0\\
42.32	0\\
42.33	0\\
42.34	0\\
42.35	0\\
42.36	0\\
42.37	0\\
42.38	0\\
42.39	0\\
42.4	0\\
42.41	0\\
42.42	0\\
42.43	0\\
42.44	0\\
42.45	0\\
42.46	0\\
42.47	0\\
42.48	0\\
42.49	0\\
42.5	0\\
42.51	0\\
42.52	0\\
42.53	0\\
42.54	0\\
42.55	0\\
42.56	0\\
42.57	0\\
42.58	0\\
42.59	0\\
42.6	0\\
42.61	0\\
42.62	0\\
42.63	0\\
42.64	0\\
42.65	0\\
42.66	0\\
42.67	0\\
42.68	0\\
42.69	0\\
42.7	0\\
42.71	0\\
42.72	0\\
42.73	0\\
42.74	0\\
42.75	0\\
42.76	0\\
42.77	0\\
42.78	0\\
42.79	0\\
42.8	0\\
42.81	0\\
42.82	0\\
42.83	0\\
42.84	0\\
42.85	0\\
42.86	0\\
42.87	0\\
42.88	0\\
42.89	0\\
42.9	0\\
42.91	0\\
42.92	0\\
42.93	0\\
42.94	0\\
42.95	0\\
42.96	0\\
42.97	0\\
42.98	0\\
42.99	0\\
43	0\\
43.01	0\\
43.02	0\\
43.03	0\\
43.04	0\\
43.05	0\\
43.06	0\\
43.07	0\\
43.08	0\\
43.09	0\\
43.1	0\\
43.11	0\\
43.12	0\\
43.13	0\\
43.14	0\\
43.15	0\\
43.16	0\\
43.17	0\\
43.18	0\\
43.19	0\\
43.2	0\\
43.21	0\\
43.22	0\\
43.23	0\\
43.24	0\\
43.25	0\\
43.26	0\\
43.27	0\\
43.28	0\\
43.29	0\\
43.3	0\\
43.31	0\\
43.32	0\\
43.33	0\\
43.34	0\\
43.35	0\\
43.36	0\\
43.37	0\\
43.38	0\\
43.39	0\\
43.4	0\\
43.41	0\\
43.42	0\\
43.43	0\\
43.44	0\\
43.45	0\\
43.46	0\\
43.47	0\\
43.48	0\\
43.49	0\\
43.5	0\\
43.51	0\\
43.52	0\\
43.53	0\\
43.54	0\\
43.55	0\\
43.56	0\\
43.57	0\\
43.58	0\\
43.59	0\\
43.6	0\\
43.61	0\\
43.62	0\\
43.63	0\\
43.64	0\\
43.65	0\\
43.66	0\\
43.67	0\\
43.68	0\\
43.69	0\\
43.7	0\\
43.71	0\\
43.72	0\\
43.73	0\\
43.74	0\\
43.75	0\\
43.76	0\\
43.77	0\\
43.78	0\\
43.79	0\\
43.8	0\\
43.81	0\\
43.82	0\\
43.83	0\\
43.84	0\\
43.85	0\\
43.86	0\\
43.87	0\\
43.88	0\\
43.89	0\\
43.9	0\\
43.91	0\\
43.92	0\\
43.93	0\\
43.94	0\\
43.95	0\\
43.96	0\\
43.97	0\\
43.98	0\\
43.99	0\\
44	0\\
44.01	0\\
44.02	0\\
44.03	0\\
44.04	0\\
44.05	0\\
44.06	0\\
44.07	0\\
44.08	0\\
44.09	0\\
44.1	0\\
44.11	0\\
44.12	0\\
44.13	0\\
44.14	0\\
44.15	0\\
44.16	0\\
44.17	0\\
44.18	0\\
44.19	0\\
44.2	0\\
44.21	0\\
44.22	0\\
44.23	0\\
44.24	0\\
44.25	0\\
44.26	0\\
44.27	0\\
44.28	0\\
44.29	0\\
44.3	0\\
44.31	0\\
44.32	0\\
44.33	0\\
44.34	0\\
44.35	0\\
44.36	0\\
44.37	0\\
44.38	0\\
44.39	0\\
44.4	0\\
44.41	0\\
44.42	0\\
44.43	0\\
44.44	0\\
44.45	0\\
44.46	0\\
44.47	0\\
44.48	0\\
44.49	0\\
44.5	0\\
44.51	0\\
44.52	0\\
44.53	0\\
44.54	0\\
44.55	0\\
44.56	0\\
44.57	0\\
44.58	0\\
44.59	0\\
44.6	0\\
44.61	0\\
44.62	0\\
44.63	0\\
44.64	0\\
44.65	0\\
44.66	0\\
44.67	0\\
44.68	0\\
44.69	0\\
44.7	0\\
44.71	0\\
44.72	0\\
44.73	0\\
44.74	0\\
44.75	0\\
44.76	0\\
44.77	0\\
44.78	0\\
44.79	0\\
44.8	0\\
44.81	0\\
44.82	0\\
44.83	0\\
44.84	0\\
44.85	0\\
44.86	0\\
44.87	0\\
44.88	0\\
44.89	0\\
44.9	0\\
44.91	0\\
44.92	0\\
44.93	0\\
44.94	0\\
44.95	0\\
44.96	0\\
44.97	0\\
44.98	0\\
44.99	0\\
45	0\\
45.01	0\\
45.02	0\\
45.03	0\\
45.04	0\\
45.05	0\\
45.06	0\\
45.07	0\\
45.08	0\\
45.09	0\\
45.1	0\\
45.11	0\\
45.12	0\\
45.13	0\\
45.14	0\\
45.15	0\\
45.16	0\\
45.17	0\\
45.18	0\\
45.19	0\\
45.2	0\\
45.21	0\\
45.22	0\\
45.23	0\\
45.24	0\\
45.25	0\\
45.26	0\\
45.27	0\\
45.28	0\\
45.29	0\\
45.3	0\\
45.31	0\\
45.32	0\\
45.33	0\\
45.34	0\\
45.35	0\\
45.36	0\\
45.37	0\\
45.38	0\\
45.39	0\\
45.4	0\\
45.41	0\\
45.42	0\\
45.43	0\\
45.44	0\\
45.45	0\\
45.46	0\\
45.47	0\\
45.48	0\\
45.49	0\\
45.5	0\\
45.51	0\\
45.52	0\\
45.53	0\\
45.54	0\\
45.55	0\\
45.56	0\\
45.57	0\\
45.58	0\\
45.59	0\\
45.6	0\\
45.61	0\\
45.62	0\\
45.63	0\\
45.64	0\\
45.65	0\\
45.66	0\\
45.67	0\\
45.68	0\\
45.69	0\\
45.7	0\\
45.71	0\\
45.72	0\\
45.73	0\\
45.74	0\\
45.75	0\\
45.76	0\\
45.77	0\\
45.78	0\\
45.79	0\\
45.8	0\\
45.81	0\\
45.82	0\\
45.83	0\\
45.84	0\\
45.85	0\\
45.86	0\\
45.87	0\\
45.88	0\\
45.89	0\\
45.9	0\\
45.91	0\\
45.92	0\\
45.93	0\\
45.94	0\\
45.95	0\\
45.96	0\\
45.97	0\\
45.98	0\\
45.99	0\\
46	0\\
46.01	0\\
46.02	0\\
46.03	0\\
46.04	0\\
46.05	0\\
46.06	0\\
46.07	0\\
46.08	0\\
46.09	0\\
46.1	0\\
46.11	0\\
46.12	0\\
46.13	0\\
46.14	0\\
46.15	0\\
46.16	0\\
46.17	0\\
46.18	0\\
46.19	0\\
46.2	0\\
46.21	0\\
46.22	0\\
46.23	0\\
46.24	0\\
46.25	0\\
46.26	0\\
46.27	0\\
46.28	0\\
46.29	0\\
46.3	0\\
46.31	0\\
46.32	0\\
46.33	0\\
46.34	0\\
46.35	0\\
46.36	0\\
46.37	0\\
46.38	0\\
46.39	0\\
46.4	0\\
46.41	0\\
46.42	0\\
46.43	0\\
46.44	0\\
46.45	0\\
46.46	0\\
46.47	0\\
46.48	0\\
46.49	0\\
46.5	0\\
46.51	0\\
46.52	0\\
46.53	0\\
46.54	0\\
46.55	0\\
46.56	0\\
46.57	0\\
46.58	0\\
46.59	0\\
46.6	0\\
46.61	0\\
46.62	0\\
46.63	0\\
46.64	0\\
46.65	0\\
46.66	0\\
46.67	0\\
46.68	0\\
46.69	0\\
46.7	0\\
46.71	0\\
46.72	0\\
46.73	0\\
46.74	0\\
46.75	0\\
46.76	0\\
46.77	0\\
46.78	0\\
46.79	0\\
46.8	0\\
46.81	0\\
46.82	0\\
46.83	0\\
46.84	0\\
46.85	0\\
46.86	0\\
46.87	0\\
46.88	0\\
46.89	0\\
46.9	0\\
46.91	0\\
46.92	0\\
46.93	0\\
46.94	0\\
46.95	0\\
46.96	0\\
46.97	0\\
46.98	0\\
46.99	0\\
47	0\\
47.01	0\\
47.02	0\\
47.03	0\\
47.04	0\\
47.05	0\\
47.06	0\\
47.07	0\\
47.08	0\\
47.09	0\\
47.1	0\\
47.11	0\\
47.12	0\\
47.13	0\\
47.14	0\\
47.15	0\\
47.16	0\\
47.17	0\\
47.18	0\\
47.19	0\\
47.2	0\\
47.21	0\\
47.22	0\\
47.23	0\\
47.24	0\\
47.25	0\\
47.26	0\\
47.27	0\\
47.28	0\\
47.29	0\\
47.3	0\\
47.31	0\\
47.32	0\\
47.33	0\\
47.34	0\\
47.35	0\\
47.36	0\\
47.37	0\\
47.38	0\\
47.39	0\\
47.4	0\\
47.41	0\\
47.42	0\\
47.43	0\\
47.44	0\\
47.45	0\\
47.46	0\\
47.47	0\\
47.48	0\\
47.49	0\\
47.5	0\\
47.51	0\\
47.52	0\\
47.53	0\\
47.54	0\\
47.55	0\\
47.56	0\\
47.57	0\\
47.58	0\\
47.59	0\\
47.6	0\\
47.61	0\\
47.62	0\\
47.63	0\\
47.64	0\\
47.65	0\\
47.66	0\\
47.67	0\\
47.68	0\\
47.69	0\\
47.7	0\\
47.71	0\\
47.72	0\\
47.73	0\\
47.74	0\\
47.75	0\\
47.76	0\\
47.77	0\\
47.78	0\\
47.79	0\\
47.8	0\\
47.81	0\\
47.82	0\\
47.83	0\\
47.84	0\\
47.85	0\\
47.86	0\\
47.87	0\\
47.88	0\\
47.89	0\\
47.9	0\\
47.91	0\\
47.92	0\\
47.93	0\\
47.94	0\\
47.95	0\\
47.96	0\\
47.97	0\\
47.98	0\\
47.99	0\\
48	0\\
48.01	0\\
48.02	0\\
48.03	0\\
48.04	0\\
48.05	0\\
48.06	0\\
48.07	0\\
48.08	0\\
48.09	0\\
48.1	0\\
48.11	0\\
48.12	0\\
48.13	0\\
48.14	0\\
48.15	0\\
48.16	0\\
48.17	0\\
48.18	0\\
48.19	0\\
48.2	0\\
48.21	0\\
48.22	0\\
48.23	0\\
48.24	0\\
48.25	0\\
48.26	0\\
48.27	0\\
48.28	0\\
48.29	0\\
48.3	0\\
48.31	0\\
48.32	0\\
48.33	0\\
48.34	0\\
48.35	0\\
48.36	0\\
48.37	0\\
48.38	0\\
48.39	0\\
48.4	0\\
48.41	0\\
48.42	0\\
48.43	0\\
48.44	0\\
48.45	0\\
48.46	0\\
48.47	0\\
48.48	0\\
48.49	0\\
48.5	0\\
48.51	0\\
48.52	0\\
48.53	0\\
48.54	0\\
48.55	0\\
48.56	0\\
48.57	0\\
48.58	0\\
48.59	0\\
48.6	0\\
48.61	0\\
48.62	0\\
48.63	0\\
48.64	0\\
48.65	0\\
48.66	0\\
48.67	0\\
48.68	0\\
48.69	0\\
48.7	0\\
48.71	0\\
48.72	0\\
48.73	0\\
48.74	0\\
48.75	0\\
48.76	0\\
48.77	0\\
48.78	0\\
48.79	0\\
48.8	0\\
48.81	0\\
48.82	0\\
48.83	0\\
48.84	0\\
48.85	0\\
48.86	0\\
48.87	0\\
48.88	0\\
48.89	0\\
48.9	0\\
48.91	0\\
48.92	0\\
48.93	0\\
48.94	0\\
48.95	0\\
48.96	0\\
48.97	0\\
48.98	0\\
48.99	0\\
49	0\\
49.01	0\\
49.02	0\\
49.03	0\\
49.04	0\\
49.05	0\\
49.06	0\\
49.07	0\\
49.08	0\\
49.09	0\\
49.1	0\\
49.11	0\\
49.12	0\\
49.13	0\\
49.14	0\\
49.15	0\\
49.16	0\\
49.17	0\\
49.18	0\\
49.19	0\\
49.2	0\\
49.21	0\\
49.22	0\\
49.23	0\\
49.24	0\\
49.25	0\\
49.26	0\\
49.27	0\\
49.28	0\\
49.29	0\\
49.3	0\\
49.31	0\\
49.32	0\\
49.33	0\\
49.34	0\\
49.35	0\\
49.36	0\\
49.37	0\\
49.38	0\\
49.39	0\\
49.4	0\\
49.41	0\\
49.42	0\\
49.43	0\\
49.44	0\\
49.45	0\\
49.46	0\\
49.47	0\\
49.48	0\\
49.49	0\\
49.5	0\\
49.51	0\\
49.52	0\\
49.53	0\\
49.54	0\\
49.55	0\\
49.56	0\\
49.57	0\\
49.58	0\\
49.59	0\\
49.6	0\\
49.61	0\\
49.62	0\\
49.63	0\\
49.64	0\\
49.65	0\\
49.66	0\\
49.67	0\\
49.68	0\\
49.69	0\\
49.7	0\\
49.71	0\\
49.72	0\\
49.73	0\\
49.74	0\\
49.75	0\\
49.76	0\\
49.77	0\\
49.78	0\\
49.79	0\\
49.8	0\\
49.81	0\\
49.82	0\\
49.83	0\\
49.84	0\\
49.85	0\\
49.86	0\\
49.87	0\\
49.88	0\\
49.89	0\\
49.9	0\\
49.91	0\\
49.92	0\\
49.93	0\\
49.94	0\\
49.95	0\\
49.96	0\\
49.97	0\\
49.98	0\\
49.99	0\\
50	0\\
50.01	0\\
50.02	0\\
50.03	0\\
50.04	0\\
50.05	0\\
50.06	0\\
50.07	0\\
50.08	0\\
50.09	0\\
50.1	0\\
50.11	0\\
50.12	0\\
50.13	0\\
50.14	0\\
50.15	0\\
50.16	0\\
50.17	0\\
50.18	0\\
50.19	0\\
50.2	0\\
50.21	0\\
50.22	0\\
50.23	0\\
50.24	0\\
50.25	0\\
50.26	0\\
50.27	0\\
50.28	0\\
50.29	0\\
50.3	0\\
50.31	0\\
50.32	0\\
50.33	0\\
50.34	0\\
50.35	0\\
50.36	0\\
50.37	0\\
50.38	0\\
50.39	0\\
50.4	0\\
50.41	0\\
50.42	0\\
50.43	0\\
50.44	0\\
50.45	0\\
50.46	0\\
50.47	0\\
50.48	0\\
50.49	0\\
50.5	0\\
50.51	0\\
50.52	0\\
50.53	0\\
50.54	0\\
50.55	0\\
50.56	0\\
50.57	0\\
50.58	0\\
50.59	0\\
50.6	0\\
50.61	0\\
50.62	0\\
50.63	0\\
50.64	0\\
50.65	0\\
50.66	0\\
50.67	0\\
50.68	0\\
50.69	0\\
50.7	0\\
50.71	0\\
50.72	0\\
50.73	0\\
50.74	0\\
50.75	0\\
50.76	0\\
50.77	0\\
50.78	0\\
50.79	0\\
50.8	0\\
50.81	0\\
50.82	0\\
50.83	0\\
50.84	0\\
50.85	0\\
50.86	0\\
50.87	0\\
50.88	0\\
50.89	0\\
50.9	0\\
50.91	0\\
50.92	0\\
50.93	0\\
50.94	0\\
50.95	0\\
50.96	0\\
50.97	0\\
50.98	0\\
50.99	0\\
51	0\\
51.01	0\\
51.02	0\\
51.03	0\\
51.04	0\\
51.05	0\\
51.06	0\\
51.07	0\\
51.08	0\\
51.09	0\\
51.1	0\\
51.11	0\\
51.12	0\\
51.13	0\\
51.14	0\\
51.15	0\\
51.16	0\\
51.17	0\\
51.18	0\\
51.19	0\\
51.2	0\\
51.21	0\\
51.22	0\\
51.23	0\\
51.24	0\\
51.25	0\\
51.26	0\\
51.27	0\\
51.28	0\\
51.29	0\\
51.3	0\\
51.31	0\\
51.32	0\\
51.33	0\\
51.34	0\\
51.35	0\\
51.36	0\\
51.37	0\\
51.38	0\\
51.39	0\\
51.4	0\\
51.41	0\\
51.42	0\\
51.43	0\\
51.44	0\\
51.45	0\\
51.46	0\\
51.47	0\\
51.48	0\\
51.49	0\\
51.5	0\\
51.51	0\\
51.52	0\\
51.53	0\\
51.54	0\\
51.55	0\\
51.56	0\\
51.57	0\\
51.58	0\\
51.59	0\\
51.6	0\\
51.61	0\\
51.62	0\\
51.63	0\\
51.64	0\\
51.65	0\\
51.66	0\\
51.67	0\\
51.68	0\\
51.69	0\\
51.7	0\\
51.71	0\\
51.72	0\\
51.73	0\\
51.74	0\\
51.75	0\\
51.76	0\\
51.77	0\\
51.78	0\\
51.79	0\\
51.8	0\\
51.81	0\\
51.82	0\\
51.83	0\\
51.84	0\\
51.85	0\\
51.86	0\\
51.87	0\\
51.88	0\\
51.89	0\\
51.9	0\\
51.91	0\\
51.92	0\\
51.93	0\\
51.94	0\\
51.95	0\\
51.96	0\\
51.97	0\\
51.98	0\\
51.99	0\\
52	0\\
52.01	0\\
52.02	0\\
52.03	0\\
52.04	0\\
52.05	0\\
52.06	0\\
52.07	0\\
52.08	0\\
52.09	0\\
52.1	0\\
52.11	0\\
52.12	0\\
52.13	0\\
52.14	0\\
52.15	0\\
52.16	0\\
52.17	0\\
52.18	0\\
52.19	0\\
52.2	0\\
52.21	0\\
52.22	0\\
52.23	0\\
52.24	0\\
52.25	0\\
52.26	0\\
52.27	0\\
52.28	0\\
52.29	0\\
52.3	0\\
52.31	0\\
52.32	0\\
52.33	0\\
52.34	0\\
52.35	0\\
52.36	0\\
52.37	0\\
52.38	0\\
52.39	0\\
52.4	0\\
52.41	0\\
52.42	0\\
52.43	0\\
52.44	0\\
52.45	0\\
52.46	0\\
52.47	0\\
52.48	0\\
52.49	0\\
52.5	0\\
52.51	0\\
52.52	0\\
52.53	0\\
52.54	0\\
52.55	0\\
52.56	0\\
52.57	0\\
52.58	0\\
52.59	0\\
52.6	0\\
52.61	0\\
52.62	0\\
52.63	0\\
52.64	0\\
52.65	0\\
52.66	0\\
52.67	0\\
52.68	0\\
52.69	0\\
52.7	0\\
52.71	0\\
52.72	0\\
52.73	0\\
52.74	0\\
52.75	0\\
52.76	0\\
52.77	0\\
52.78	0\\
52.79	0\\
52.8	0\\
52.81	0\\
52.82	0\\
52.83	0\\
52.84	0\\
52.85	0\\
52.86	0\\
52.87	0\\
52.88	0\\
52.89	0\\
52.9	0\\
52.91	0\\
52.92	0\\
52.93	0\\
52.94	0\\
52.95	0\\
52.96	0\\
52.97	0\\
52.98	0\\
52.99	0\\
53	0\\
53.01	0\\
53.02	0\\
53.03	0\\
53.04	0\\
53.05	0\\
53.06	0\\
53.07	0\\
53.08	0\\
53.09	0\\
53.1	0\\
53.11	0\\
53.12	0\\
53.13	0\\
53.14	0\\
53.15	0\\
53.16	0\\
53.17	0\\
53.18	0\\
53.19	0\\
53.2	0\\
53.21	0\\
53.22	0\\
53.23	0\\
53.24	0\\
53.25	0\\
53.26	0\\
53.27	0\\
53.28	0\\
53.29	0\\
53.3	0\\
53.31	0\\
53.32	0\\
53.33	0\\
53.34	0\\
53.35	0\\
53.36	0\\
53.37	0\\
53.38	0\\
53.39	0\\
53.4	0\\
53.41	0\\
53.42	0\\
53.43	0\\
53.44	0\\
53.45	0\\
53.46	0\\
53.47	0\\
53.48	0\\
53.49	0\\
53.5	0\\
53.51	0\\
53.52	0\\
53.53	0\\
53.54	0\\
53.55	0\\
53.56	0\\
53.57	0\\
53.58	0\\
53.59	0\\
53.6	0\\
53.61	0\\
53.62	0\\
53.63	0\\
53.64	0\\
53.65	0\\
53.66	0\\
53.67	0\\
53.68	0\\
53.69	0\\
53.7	0\\
53.71	0\\
53.72	0\\
53.73	0\\
53.74	0\\
53.75	0\\
53.76	0\\
53.77	0\\
53.78	0\\
53.79	0\\
53.8	0\\
53.81	0\\
53.82	0\\
53.83	0\\
53.84	0\\
53.85	0\\
53.86	0\\
53.87	0\\
53.88	0\\
53.89	0\\
53.9	0\\
53.91	0\\
53.92	0\\
53.93	0\\
53.94	0\\
53.95	0\\
53.96	0\\
53.97	0\\
53.98	0\\
53.99	0\\
54	0\\
54.01	0\\
54.02	0\\
54.03	0\\
54.04	0\\
54.05	0\\
54.06	0\\
54.07	0\\
54.08	0\\
54.09	0\\
54.1	0\\
54.11	0\\
54.12	0\\
54.13	0\\
54.14	0\\
54.15	0\\
54.16	0\\
54.17	0\\
54.18	0\\
54.19	0\\
54.2	0\\
54.21	0\\
54.22	0\\
54.23	0\\
54.24	0\\
54.25	0\\
54.26	0\\
54.27	0\\
54.28	0\\
54.29	0\\
54.3	0\\
54.31	0\\
54.32	0\\
54.33	0\\
54.34	0\\
54.35	0\\
54.36	0\\
54.37	0\\
54.38	0\\
54.39	0\\
54.4	0\\
54.41	0\\
54.42	0\\
54.43	0\\
54.44	0\\
54.45	0\\
54.46	0\\
54.47	0\\
54.48	0\\
54.49	0\\
54.5	0\\
54.51	0\\
54.52	0\\
54.53	0\\
54.54	0\\
54.55	0\\
54.56	0\\
54.57	0\\
54.58	0\\
54.59	0\\
54.6	0\\
54.61	0\\
54.62	0\\
54.63	0\\
54.64	0\\
54.65	0\\
54.66	0\\
54.67	0\\
54.68	0\\
54.69	0\\
54.7	0\\
54.71	0\\
54.72	0\\
54.73	0\\
54.74	0\\
54.75	0\\
54.76	0\\
54.77	0\\
54.78	0\\
54.79	0\\
54.8	0\\
54.81	0\\
54.82	0\\
54.83	0\\
54.84	0\\
54.85	0\\
54.86	0\\
54.87	0\\
54.88	0\\
54.89	0\\
54.9	0\\
54.91	0\\
54.92	0\\
54.93	0\\
54.94	0\\
54.95	0\\
54.96	0\\
54.97	0\\
54.98	0\\
54.99	0\\
55	0\\
55.01	0\\
55.02	0\\
55.03	0\\
55.04	0\\
55.05	0\\
55.06	0\\
55.07	0\\
55.08	0\\
55.09	0\\
55.1	0\\
55.11	0\\
55.12	0\\
55.13	0\\
55.14	0\\
55.15	0\\
55.16	0\\
55.17	0\\
55.18	0\\
55.19	0\\
55.2	0\\
55.21	0\\
55.22	0\\
55.23	0\\
55.24	0\\
55.25	0\\
55.26	0\\
55.27	0\\
55.28	0\\
55.29	0\\
55.3	0\\
55.31	0\\
55.32	0\\
55.33	0\\
55.34	0\\
55.35	0\\
55.36	0\\
55.37	0\\
55.38	0\\
55.39	0\\
55.4	0\\
55.41	0\\
55.42	0\\
55.43	0\\
55.44	0\\
55.45	0\\
55.46	0\\
55.47	0\\
55.48	0\\
55.49	0\\
55.5	0\\
55.51	0\\
55.52	0\\
55.53	0\\
55.54	0\\
55.55	0\\
55.56	0\\
55.57	0\\
55.58	0\\
55.59	0\\
55.6	0\\
55.61	0\\
55.62	0\\
55.63	0\\
55.64	0\\
55.65	0\\
55.66	0\\
55.67	0\\
55.68	0\\
55.69	0\\
55.7	0\\
55.71	0\\
55.72	0\\
55.73	0\\
55.74	0\\
55.75	0\\
55.76	0\\
55.77	0\\
55.78	0\\
55.79	0\\
55.8	0\\
55.81	0\\
55.82	0\\
55.83	0\\
55.84	0\\
55.85	0\\
55.86	0\\
55.87	0\\
55.88	0\\
55.89	0\\
55.9	0\\
55.91	0\\
55.92	0\\
55.93	0\\
55.94	0\\
55.95	0\\
55.96	0\\
55.97	0\\
55.98	0\\
55.99	0\\
56	0\\
56.01	0\\
56.02	0\\
56.03	0\\
56.04	0\\
56.05	0\\
56.06	0\\
56.07	0\\
56.08	0\\
56.09	0\\
56.1	0\\
56.11	0\\
56.12	0\\
56.13	0\\
56.14	0\\
56.15	0\\
56.16	0\\
56.17	0\\
56.18	0\\
56.19	0\\
56.2	0\\
56.21	0\\
56.22	0\\
56.23	0\\
56.24	0\\
56.25	0\\
56.26	0\\
56.27	0\\
56.28	0\\
56.29	0\\
56.3	0\\
56.31	0\\
56.32	0\\
56.33	0\\
56.34	0\\
56.35	0\\
56.36	0\\
56.37	0\\
56.38	0\\
56.39	0\\
56.4	0\\
56.41	0\\
56.42	0\\
56.43	0\\
56.44	0\\
56.45	0\\
56.46	0\\
56.47	0\\
56.48	0\\
56.49	0\\
56.5	0\\
56.51	0\\
56.52	0\\
56.53	0\\
56.54	0\\
56.55	0\\
56.56	0\\
56.57	0\\
56.58	0\\
56.59	0\\
56.6	0\\
56.61	0\\
56.62	0\\
56.63	0\\
56.64	0\\
56.65	0\\
56.66	0\\
56.67	0\\
56.68	0\\
56.69	0\\
56.7	0\\
56.71	0\\
56.72	0\\
56.73	0\\
56.74	0\\
56.75	0\\
56.76	0\\
56.77	0\\
56.78	0\\
56.79	0\\
56.8	0\\
56.81	0\\
56.82	0\\
56.83	0\\
56.84	0\\
56.85	0\\
56.86	0\\
56.87	0\\
56.88	0\\
56.89	0\\
56.9	0\\
56.91	0\\
56.92	0\\
56.93	0\\
56.94	0\\
56.95	0\\
56.96	0\\
56.97	0\\
56.98	0\\
56.99	0\\
57	0\\
57.01	0\\
57.02	0\\
57.03	0\\
57.04	0\\
57.05	0\\
57.06	0\\
57.07	0\\
57.08	0\\
57.09	0\\
57.1	0\\
57.11	0\\
57.12	0\\
57.13	0\\
57.14	0\\
57.15	0\\
57.16	0\\
57.17	0\\
57.18	0\\
57.19	0\\
57.2	0\\
57.21	0\\
57.22	0\\
57.23	0\\
57.24	0\\
57.25	0\\
57.26	0\\
57.27	0\\
57.28	0\\
57.29	0\\
57.3	0\\
57.31	0\\
57.32	0\\
57.33	0\\
57.34	0\\
57.35	0\\
57.36	0\\
57.37	0\\
57.38	0\\
57.39	0\\
57.4	0\\
57.41	0\\
57.42	0\\
57.43	0\\
57.44	0\\
57.45	0\\
57.46	0\\
57.47	0\\
57.48	0\\
57.49	0\\
57.5	0\\
57.51	0\\
57.52	0\\
57.53	0\\
57.54	0\\
57.55	0\\
57.56	0\\
57.57	0\\
57.58	0\\
57.59	0\\
57.6	0\\
57.61	0\\
57.62	0\\
57.63	0\\
57.64	0\\
57.65	0\\
57.66	0\\
57.67	0\\
57.68	0\\
57.69	0\\
57.7	0\\
57.71	0\\
57.72	0\\
57.73	0\\
57.74	0\\
57.75	0\\
57.76	0\\
57.77	0\\
57.78	0\\
57.79	0\\
57.8	0\\
57.81	0\\
57.82	0\\
57.83	0\\
57.84	0\\
57.85	0\\
57.86	0\\
57.87	0\\
57.88	0\\
57.89	0\\
57.9	0\\
57.91	0\\
57.92	0\\
57.93	0\\
57.94	0\\
57.95	0\\
57.96	0\\
57.97	0\\
57.98	0\\
57.99	0\\
58	0\\
58.01	0\\
58.02	0\\
58.03	0\\
58.04	0\\
58.05	0\\
58.06	0\\
58.07	0\\
58.08	0\\
58.09	0\\
58.1	0\\
58.11	0\\
58.12	0\\
58.13	0\\
58.14	0\\
58.15	0\\
58.16	0\\
58.17	0\\
58.18	0\\
58.19	0\\
58.2	0\\
58.21	0\\
58.22	0\\
58.23	0\\
58.24	0\\
58.25	0\\
58.26	0\\
58.27	0\\
58.28	0\\
58.29	0\\
58.3	0\\
58.31	0\\
58.32	0\\
58.33	0\\
58.34	0\\
58.35	0\\
58.36	0\\
58.37	0\\
58.38	0\\
58.39	0\\
58.4	0\\
58.41	0\\
58.42	0\\
58.43	0\\
58.44	0\\
58.45	0\\
58.46	0\\
58.47	0\\
58.48	0\\
58.49	0\\
58.5	0\\
58.51	0\\
58.52	0\\
58.53	0\\
58.54	0\\
58.55	0\\
58.56	0\\
58.57	0\\
58.58	0\\
58.59	0\\
58.6	0\\
58.61	0\\
58.62	0\\
58.63	0\\
58.64	0\\
58.65	0\\
58.66	0\\
58.67	0\\
58.68	0\\
58.69	0\\
58.7	0\\
58.71	0\\
58.72	0\\
58.73	0\\
58.74	0\\
58.75	0\\
58.76	0\\
58.77	0\\
58.78	0\\
58.79	0\\
58.8	0\\
58.81	0\\
58.82	0\\
58.83	0\\
58.84	0\\
58.85	0\\
58.86	0\\
58.87	0\\
58.88	0\\
58.89	0\\
58.9	0\\
58.91	0\\
58.92	0\\
58.93	0\\
58.94	0\\
58.95	0\\
58.96	0\\
58.97	0\\
58.98	0\\
58.99	0\\
59	0\\
59.01	0\\
59.02	0\\
59.03	0\\
59.04	0\\
59.05	0\\
59.06	0\\
59.07	0\\
59.08	0\\
59.09	0\\
59.1	0\\
59.11	0\\
59.12	0\\
59.13	0\\
59.14	0\\
59.15	0\\
59.16	0\\
59.17	0\\
59.18	0\\
59.19	0\\
59.2	0\\
59.21	0\\
59.22	0\\
59.23	0\\
59.24	0\\
59.25	0\\
59.26	0\\
59.27	0\\
59.28	0\\
59.29	0\\
59.3	0\\
59.31	0\\
59.32	0\\
59.33	0\\
59.34	0\\
59.35	0\\
59.36	0\\
59.37	0\\
59.38	0\\
59.39	0\\
59.4	0\\
59.41	0\\
59.42	0\\
59.43	0\\
59.44	0\\
59.45	0\\
59.46	0\\
59.47	0\\
59.48	0\\
59.49	0\\
59.5	0\\
59.51	0\\
59.52	0\\
59.53	0\\
59.54	0\\
59.55	0\\
59.56	0\\
59.57	0\\
59.58	0\\
59.59	0\\
59.6	0\\
59.61	0\\
59.62	0\\
59.63	0\\
59.64	0\\
59.65	0\\
59.66	0\\
59.67	0\\
59.68	0\\
59.69	0\\
59.7	0\\
59.71	0\\
59.72	0\\
59.73	0\\
59.74	0\\
59.75	0\\
59.76	0\\
59.77	0\\
59.78	0\\
59.79	0\\
59.8	0\\
59.81	0\\
59.82	0\\
59.83	0\\
59.84	0\\
59.85	0\\
59.86	0\\
59.87	0\\
59.88	0\\
59.89	0\\
59.9	0\\
59.91	0\\
59.92	0\\
59.93	0\\
59.94	0\\
59.95	0\\
59.96	0\\
59.97	0\\
59.98	0\\
59.99	0\\
60	0\\
60.01	0\\
60.02	0\\
60.03	0\\
60.04	0\\
60.05	0\\
60.06	0\\
60.07	0\\
60.08	0\\
60.09	0\\
60.1	0\\
60.11	0\\
60.12	0\\
60.13	0\\
60.14	0\\
60.15	0\\
60.16	0\\
60.17	0\\
60.18	0\\
60.19	0\\
60.2	0\\
60.21	0\\
60.22	0\\
60.23	0\\
60.24	0\\
60.25	0\\
60.26	0\\
60.27	0\\
60.28	0\\
60.29	0\\
60.3	0\\
60.31	0\\
60.32	0\\
60.33	0\\
60.34	0\\
60.35	0\\
60.36	0\\
60.37	0\\
60.38	0\\
60.39	0\\
60.4	0\\
60.41	0\\
60.42	0\\
60.43	0\\
60.44	0\\
60.45	0\\
60.46	0\\
60.47	0\\
60.48	0\\
60.49	0\\
60.5	0\\
60.51	0\\
60.52	0\\
60.53	0\\
60.54	0\\
60.55	0\\
60.56	0\\
60.57	0\\
60.58	0\\
60.59	0\\
60.6	0\\
60.61	0\\
60.62	0\\
60.63	0\\
60.64	0\\
60.65	0\\
60.66	0\\
60.67	0\\
60.68	0\\
60.69	0\\
60.7	0\\
60.71	0\\
60.72	0\\
60.73	0\\
60.74	0\\
60.75	0\\
60.76	0\\
60.77	0\\
60.78	0\\
60.79	0\\
60.8	0\\
60.81	0\\
60.82	0\\
60.83	0\\
60.84	0\\
60.85	0\\
60.86	0\\
60.87	0\\
60.88	0\\
60.89	0\\
60.9	0\\
60.91	0\\
60.92	0\\
60.93	0\\
60.94	0\\
60.95	0\\
60.96	0\\
60.97	0\\
60.98	0\\
60.99	0\\
61	0\\
61.01	0\\
61.02	0\\
61.03	0\\
61.04	0\\
61.05	0\\
61.06	0\\
61.07	0\\
61.08	0\\
61.09	0\\
61.1	0\\
61.11	0\\
61.12	0\\
61.13	0\\
61.14	0\\
61.15	0\\
61.16	0\\
61.17	0\\
61.18	0\\
61.19	0\\
61.2	0\\
61.21	0\\
61.22	0\\
61.23	0\\
61.24	0\\
61.25	0\\
61.26	0\\
61.27	0\\
61.28	0\\
61.29	0\\
61.3	0\\
61.31	0\\
61.32	0\\
61.33	0\\
61.34	0\\
61.35	0\\
61.36	0\\
61.37	0\\
61.38	0\\
61.39	0\\
61.4	0\\
61.41	0\\
61.42	0\\
61.43	0\\
61.44	0\\
61.45	0\\
61.46	0\\
61.47	0\\
61.48	0\\
61.49	0\\
61.5	0\\
61.51	0\\
61.52	0\\
61.53	0\\
61.54	0\\
61.55	0\\
61.56	0\\
61.57	0\\
61.58	0\\
61.59	0\\
61.6	0\\
61.61	0\\
61.62	0\\
61.63	0\\
61.64	0\\
61.65	0\\
61.66	0\\
61.67	0\\
61.68	0\\
61.69	0\\
61.7	0\\
61.71	0\\
61.72	0\\
61.73	0\\
61.74	0\\
61.75	0\\
61.76	0\\
61.77	0\\
61.78	0\\
61.79	0\\
61.8	0\\
61.81	0\\
61.82	0\\
61.83	0\\
61.84	0\\
61.85	0\\
61.86	0\\
61.87	0\\
61.88	0\\
61.89	0\\
61.9	0\\
61.91	0\\
61.92	0\\
61.93	0\\
61.94	0\\
61.95	0\\
61.96	0\\
61.97	0\\
61.98	0\\
61.99	0\\
62	0\\
62.01	0\\
62.02	0\\
62.03	0\\
62.04	0\\
62.05	0\\
62.06	0\\
62.07	0\\
62.08	0\\
62.09	0\\
62.1	0\\
62.11	0\\
62.12	0\\
62.13	0\\
62.14	0\\
62.15	0\\
62.16	0\\
62.17	0\\
62.18	0\\
62.19	0\\
62.2	0\\
62.21	0\\
62.22	0\\
62.23	0\\
62.24	0\\
62.25	0\\
62.26	0\\
62.27	0\\
62.28	0\\
62.29	0\\
62.3	0\\
62.31	0\\
62.32	0\\
62.33	0\\
62.34	0\\
62.35	0\\
62.36	0\\
62.37	0\\
62.38	0\\
62.39	0\\
62.4	0\\
62.41	0\\
62.42	0\\
62.43	0\\
62.44	0\\
62.45	0\\
62.46	0\\
62.47	0\\
62.48	0\\
62.49	0\\
62.5	0\\
62.51	0\\
62.52	0\\
62.53	0\\
62.54	0\\
62.55	0\\
62.56	0\\
62.57	0\\
62.58	0\\
62.59	0\\
62.6	0\\
62.61	0\\
62.62	0\\
62.63	0\\
62.64	0\\
62.65	0\\
62.66	0\\
62.67	0\\
62.68	0\\
62.69	0\\
62.7	0\\
62.71	0\\
62.72	0\\
62.73	0\\
62.74	0\\
62.75	0\\
62.76	0\\
62.77	0\\
62.78	0\\
62.79	0\\
62.8	0\\
62.81	0\\
62.82	0\\
62.83	0\\
62.84	0\\
62.85	0\\
62.86	0\\
62.87	0\\
62.88	0\\
62.89	0\\
62.9	0\\
62.91	0\\
62.92	0\\
62.93	0\\
62.94	0\\
62.95	0\\
62.96	0\\
62.97	0\\
62.98	0\\
62.99	0\\
63	0\\
63.01	0\\
63.02	0\\
63.03	0\\
63.04	0\\
63.05	0\\
63.06	0\\
63.07	0\\
63.08	0\\
63.09	0\\
63.1	0\\
63.11	0\\
63.12	0\\
63.13	0\\
63.14	0\\
63.15	0\\
63.16	0\\
63.17	0\\
63.18	0\\
63.19	0\\
63.2	0\\
63.21	0\\
63.22	0\\
63.23	0\\
63.24	0\\
63.25	0\\
63.26	0\\
63.27	0\\
63.28	0\\
63.29	0\\
63.3	0\\
63.31	0\\
63.32	0\\
63.33	0\\
63.34	0\\
63.35	0\\
63.36	0\\
63.37	0\\
63.38	0\\
63.39	0\\
63.4	0\\
63.41	0\\
63.42	0\\
63.43	0\\
63.44	0\\
63.45	0\\
63.46	0\\
63.47	0\\
63.48	0\\
63.49	0\\
63.5	0\\
63.51	0\\
63.52	0\\
63.53	0\\
63.54	0\\
63.55	0\\
63.56	0\\
63.57	0\\
63.58	0\\
63.59	0\\
63.6	0\\
63.61	0\\
63.62	0\\
63.63	0\\
63.64	0\\
63.65	0\\
63.66	0\\
63.67	0\\
63.68	0\\
63.69	0\\
63.7	0\\
63.71	0\\
63.72	0\\
63.73	0\\
63.74	0\\
63.75	0\\
63.76	0\\
63.77	0\\
63.78	0\\
63.79	0\\
63.8	0\\
63.81	0\\
63.82	0\\
63.83	0\\
63.84	0\\
63.85	0\\
63.86	0\\
63.87	0\\
63.88	0\\
63.89	0\\
63.9	0\\
63.91	0\\
63.92	0\\
63.93	0\\
63.94	0\\
63.95	0\\
63.96	0\\
63.97	0\\
63.98	0\\
63.99	0\\
64	0\\
64.01	0\\
64.02	0\\
64.03	0\\
64.04	0\\
64.05	0\\
64.06	0\\
64.07	0\\
64.08	0\\
64.09	0\\
64.1	0\\
64.11	0\\
64.12	0\\
64.13	0\\
64.14	0\\
64.15	0\\
64.16	0\\
64.17	0\\
64.18	0\\
64.19	0\\
64.2	0\\
64.21	0\\
64.22	0\\
64.23	0\\
64.24	0\\
64.25	0\\
64.26	0\\
64.27	0\\
64.28	0\\
64.29	0\\
64.3	0\\
64.31	0\\
64.32	0\\
64.33	0\\
64.34	0\\
64.35	0\\
64.36	0\\
64.37	0\\
64.38	0\\
64.39	0\\
64.4	0\\
64.41	0\\
64.42	0\\
64.43	0\\
64.44	0\\
64.45	0\\
64.46	0\\
64.47	0\\
64.48	0\\
64.49	0\\
64.5	0\\
64.51	0\\
64.52	0\\
64.53	0\\
64.54	0\\
64.55	0\\
64.56	0\\
64.57	0\\
64.58	0\\
64.59	0\\
64.6	0\\
64.61	0\\
64.62	0\\
64.63	0\\
64.64	0\\
64.65	0\\
64.66	0\\
64.67	0\\
64.68	0\\
64.69	0\\
64.7	0\\
64.71	0\\
64.72	0\\
64.73	0\\
64.74	0\\
64.75	0\\
64.76	0\\
64.77	0\\
64.78	0\\
64.79	0\\
64.8	0\\
64.81	0\\
64.82	0\\
64.83	0\\
64.84	0\\
64.85	0\\
64.86	0\\
64.87	0\\
64.88	0\\
64.89	0\\
64.9	0\\
64.91	0\\
64.92	0\\
64.93	0\\
64.94	0\\
64.95	0\\
64.96	0\\
64.97	0\\
64.98	0\\
64.99	0\\
65	0\\
65.01	0\\
65.02	0\\
65.03	0\\
65.04	0\\
65.05	0\\
65.06	0\\
65.07	0\\
65.08	0\\
65.09	0\\
65.1	0\\
65.11	0\\
65.12	0\\
65.13	0\\
65.14	0\\
65.15	0\\
65.16	0\\
65.17	0\\
65.18	0\\
65.19	0\\
65.2	0\\
65.21	0\\
65.22	0\\
65.23	0\\
65.24	0\\
65.25	0\\
65.26	0\\
65.27	0\\
65.28	0\\
65.29	0\\
65.3	0\\
65.31	0\\
65.32	0\\
65.33	0\\
65.34	0\\
65.35	0\\
65.36	0\\
65.37	0\\
65.38	0\\
65.39	0\\
65.4	0\\
65.41	0\\
65.42	0\\
65.43	0\\
65.44	0\\
65.45	0\\
65.46	0\\
65.47	0\\
65.48	0\\
65.49	0\\
65.5	0\\
65.51	0\\
65.52	0\\
65.53	0\\
65.54	0\\
65.55	0\\
65.56	0\\
65.57	0\\
65.58	0\\
65.59	0\\
65.6	0\\
65.61	0\\
65.62	0\\
65.63	0\\
65.64	0\\
65.65	0\\
65.66	0\\
65.67	0\\
65.68	0\\
65.69	0\\
65.7	0\\
65.71	0\\
65.72	0\\
65.73	0\\
65.74	0\\
65.75	0\\
65.76	0\\
65.77	0\\
65.78	0\\
65.79	0\\
65.8	0\\
65.81	0\\
65.82	0\\
65.83	0\\
65.84	0\\
65.85	0\\
65.86	0\\
65.87	0\\
65.88	0\\
65.89	0\\
65.9	0\\
65.91	0\\
65.92	0\\
65.93	0\\
65.94	0\\
65.95	0\\
65.96	0\\
65.97	0\\
65.98	0\\
65.99	0\\
66	0\\
66.01	0\\
66.02	0\\
66.03	0\\
66.04	0\\
66.05	0\\
66.06	0\\
66.07	0\\
66.08	0\\
66.09	0\\
66.1	0\\
66.11	0\\
66.12	0\\
66.13	0\\
66.14	0\\
66.15	0\\
66.16	0\\
66.17	0\\
66.18	0\\
66.19	0\\
66.2	0\\
66.21	0\\
66.22	0\\
66.23	0\\
66.24	0\\
66.25	0\\
66.26	0\\
66.27	0\\
66.28	0\\
66.29	0\\
66.3	0\\
66.31	0\\
66.32	0\\
66.33	0\\
66.34	0\\
66.35	0\\
66.36	0\\
66.37	0\\
66.38	0\\
66.39	0\\
66.4	0\\
66.41	0\\
66.42	0\\
66.43	0\\
66.44	0\\
66.45	0\\
66.46	0\\
66.47	0\\
66.48	0\\
66.49	0\\
66.5	0\\
66.51	0\\
66.52	0\\
66.53	0\\
66.54	0\\
66.55	0\\
66.56	0\\
66.57	0\\
66.58	0\\
66.59	0\\
66.6	0\\
66.61	0\\
66.62	0\\
66.63	0\\
66.64	0\\
66.65	0\\
66.66	0\\
66.67	0\\
66.68	0\\
66.69	0\\
66.7	0\\
66.71	0\\
66.72	0\\
66.73	0\\
66.74	0\\
66.75	0\\
66.76	0\\
66.77	0\\
66.78	0\\
66.79	0\\
66.8	0\\
66.81	0\\
66.82	0\\
66.83	0\\
66.84	0\\
66.85	0\\
66.86	0\\
66.87	0\\
66.88	0\\
66.89	0\\
66.9	0\\
66.91	0\\
66.92	0\\
66.93	0\\
66.94	0\\
66.95	0\\
66.96	0\\
66.97	0\\
66.98	0\\
66.99	0\\
67	0\\
67.01	0\\
67.02	0\\
67.03	0\\
67.04	0\\
67.05	0\\
67.06	0\\
67.07	0\\
67.08	0\\
67.09	0\\
67.1	0\\
67.11	0\\
67.12	0\\
67.13	0\\
67.14	0\\
67.15	0\\
67.16	0\\
67.17	0\\
67.18	0\\
67.19	0\\
67.2	0\\
67.21	0\\
67.22	0\\
67.23	0\\
67.24	0\\
67.25	0\\
67.26	0\\
67.27	0\\
67.28	0\\
67.29	0\\
67.3	0\\
67.31	0\\
67.32	0\\
67.33	0\\
67.34	0\\
67.35	0\\
67.36	0\\
67.37	0\\
67.38	0\\
67.39	0\\
67.4	0\\
67.41	0\\
67.42	0\\
67.43	0\\
67.44	0\\
67.45	0\\
67.46	0\\
67.47	0\\
67.48	0\\
67.49	0\\
67.5	0\\
67.51	0\\
67.52	0\\
67.53	0\\
67.54	0\\
67.55	0\\
67.56	0\\
67.57	0\\
67.58	0\\
67.59	0\\
67.6	0\\
67.61	0\\
67.62	0\\
67.63	0\\
67.64	0\\
67.65	0\\
67.66	0\\
67.67	0\\
67.68	0\\
67.69	0\\
67.7	0\\
67.71	0\\
67.72	0\\
67.73	0\\
67.74	0\\
67.75	0\\
67.76	0\\
67.77	0\\
67.78	0\\
67.79	0\\
67.8	0\\
67.81	0\\
67.82	0\\
67.83	0\\
67.84	0\\
67.85	0\\
67.86	0\\
67.87	0\\
67.88	0\\
67.89	0\\
67.9	0\\
67.91	0\\
67.92	0\\
67.93	0\\
67.94	0\\
67.95	0\\
67.96	0\\
67.97	0\\
67.98	0\\
67.99	0\\
68	0\\
68.01	0\\
68.02	0\\
68.03	0\\
68.04	0\\
68.05	0\\
68.06	0\\
68.07	0\\
68.08	0\\
68.09	0\\
68.1	0\\
68.11	0\\
68.12	0\\
68.13	0\\
68.14	0\\
68.15	0\\
68.16	0\\
68.17	0\\
68.18	0\\
68.19	0\\
68.2	0\\
68.21	0\\
68.22	0\\
68.23	0\\
68.24	0\\
68.25	0\\
68.26	0\\
68.27	0\\
68.28	0\\
68.29	0\\
68.3	0\\
68.31	0\\
68.32	0\\
68.33	0\\
68.34	0\\
68.35	0\\
68.36	0\\
68.37	0\\
68.38	0\\
68.39	0\\
68.4	0\\
68.41	0\\
68.42	0\\
68.43	0\\
68.44	0\\
68.45	0\\
68.46	0\\
68.47	0\\
68.48	0\\
68.49	0\\
68.5	0\\
68.51	0\\
68.52	0\\
68.53	0\\
68.54	0\\
68.55	0\\
68.56	0\\
68.57	0\\
68.58	0\\
68.59	0\\
68.6	0\\
68.61	0\\
68.62	0\\
68.63	0\\
68.64	0\\
68.65	0\\
68.66	0\\
68.67	0\\
68.68	0\\
68.69	0\\
68.7	0\\
68.71	0\\
68.72	0\\
68.73	0\\
68.74	0\\
68.75	0\\
68.76	0\\
68.77	0\\
68.78	0\\
68.79	0\\
68.8	0\\
68.81	0\\
68.82	0\\
68.83	0\\
68.84	0\\
68.85	0\\
68.86	0\\
68.87	0\\
68.88	0\\
68.89	0\\
68.9	0\\
68.91	0\\
68.92	0\\
68.93	0\\
68.94	0\\
68.95	0\\
68.96	0\\
68.97	0\\
68.98	0\\
68.99	0\\
69	0\\
69.01	0\\
69.02	0\\
69.03	0\\
69.04	0\\
69.05	0\\
69.06	0\\
69.07	0\\
69.08	0\\
69.09	0\\
69.1	0\\
69.11	0\\
69.12	0\\
69.13	0\\
69.14	0\\
69.15	0\\
69.16	0\\
69.17	0\\
69.18	0\\
69.19	0\\
69.2	0\\
69.21	0\\
69.22	0\\
69.23	0\\
69.24	0\\
69.25	0\\
69.26	0\\
69.27	0\\
69.28	0\\
69.29	0\\
69.3	0\\
69.31	0\\
69.32	0\\
69.33	0\\
69.34	0\\
69.35	0\\
69.36	0\\
69.37	0\\
69.38	0\\
69.39	0\\
69.4	0\\
69.41	0\\
69.42	0\\
69.43	0\\
69.44	0\\
69.45	0\\
69.46	0\\
69.47	0\\
69.48	0\\
69.49	0\\
69.5	0\\
69.51	0\\
69.52	0\\
69.53	0\\
69.54	0\\
69.55	0\\
69.56	0\\
69.57	0\\
69.58	0\\
69.59	0\\
69.6	0\\
69.61	0\\
69.62	0\\
69.63	0\\
69.64	0\\
69.65	0\\
69.66	0\\
69.67	0\\
69.68	0\\
69.69	0\\
69.7	0\\
69.71	0\\
69.72	0\\
69.73	0\\
69.74	0\\
69.75	0\\
69.76	0\\
69.77	0\\
69.78	0\\
69.79	0\\
69.8	0\\
69.81	0\\
69.82	0\\
69.83	0\\
69.84	0\\
69.85	0\\
69.86	0\\
69.87	0\\
69.88	0\\
69.89	0\\
69.9	0\\
69.91	0\\
69.92	0\\
69.93	0\\
69.94	0\\
69.95	0\\
69.96	0\\
69.97	0\\
69.98	0\\
69.99	0\\
70	0\\
70.01	0\\
70.02	0\\
70.03	0\\
70.04	0\\
70.05	0\\
70.06	0\\
70.07	0\\
70.08	0\\
70.09	0\\
70.1	0\\
70.11	0\\
70.12	0\\
70.13	0\\
70.14	0\\
70.15	0\\
70.16	0\\
70.17	0\\
70.18	0\\
70.19	0\\
70.2	0\\
70.21	0\\
70.22	0\\
70.23	0\\
70.24	0\\
70.25	0\\
70.26	0\\
70.27	0\\
70.28	0\\
70.29	0\\
70.3	0\\
70.31	0\\
70.32	0\\
70.33	0\\
70.34	0\\
70.35	0\\
70.36	0\\
70.37	0\\
70.38	0\\
70.39	0\\
70.4	0\\
70.41	0\\
70.42	0\\
70.43	0\\
70.44	0\\
70.45	0\\
70.46	0\\
70.47	0\\
70.48	0\\
70.49	0\\
70.5	0\\
70.51	0\\
70.52	0\\
70.53	0\\
70.54	0\\
70.55	0\\
70.56	0\\
70.57	0\\
70.58	0\\
70.59	0\\
70.6	0\\
70.61	0\\
70.62	0\\
70.63	0\\
70.64	0\\
70.65	0\\
70.66	0\\
70.67	0\\
70.68	0\\
70.69	0\\
70.7	0\\
70.71	0\\
70.72	0\\
70.73	0\\
70.74	0\\
70.75	0\\
70.76	0\\
70.77	0\\
70.78	0\\
70.79	0\\
70.8	0\\
70.81	0\\
70.82	0\\
70.83	0\\
70.84	0\\
70.85	0\\
70.86	0\\
70.87	0\\
70.88	0\\
70.89	0\\
70.9	0\\
70.91	0\\
70.92	0\\
70.93	0\\
70.94	0\\
70.95	0\\
70.96	0\\
70.97	0\\
70.98	0\\
70.99	0\\
71	0\\
71.01	0\\
71.02	0\\
71.03	0\\
71.04	0\\
71.05	0\\
71.06	0\\
71.07	0\\
71.08	0\\
71.09	0\\
71.1	0\\
71.11	0\\
71.12	0\\
71.13	0\\
71.14	0\\
71.15	0\\
71.16	0\\
71.17	0\\
71.18	0\\
71.19	0\\
71.2	0\\
71.21	0\\
71.22	0\\
71.23	0\\
71.24	0\\
71.25	0\\
71.26	0\\
71.27	0\\
71.28	0\\
71.29	0\\
71.3	0\\
71.31	0\\
71.32	0\\
71.33	0\\
71.34	0\\
71.35	0\\
71.36	0\\
71.37	0\\
71.38	0\\
71.39	0\\
71.4	0\\
71.41	0\\
71.42	0\\
71.43	0\\
71.44	0\\
71.45	0\\
71.46	0\\
71.47	0\\
71.48	0\\
71.49	0\\
71.5	0\\
71.51	0\\
71.52	0\\
71.53	0\\
71.54	0\\
71.55	0\\
71.56	0\\
71.57	0\\
71.58	0\\
71.59	0\\
71.6	0\\
71.61	0\\
71.62	0\\
71.63	0\\
71.64	0\\
71.65	0\\
71.66	0\\
71.67	0\\
71.68	0\\
71.69	0\\
71.7	0\\
71.71	0\\
71.72	0\\
71.73	0\\
71.74	0\\
71.75	0\\
71.76	0\\
71.77	0\\
71.78	0\\
71.79	0\\
71.8	0\\
71.81	0\\
71.82	0\\
71.83	0\\
71.84	0\\
71.85	0\\
71.86	0\\
71.87	0\\
71.88	0\\
71.89	0\\
71.9	0\\
71.91	0\\
71.92	0\\
71.93	0\\
71.94	0\\
71.95	0\\
71.96	0\\
71.97	0\\
71.98	0\\
71.99	0\\
72	0\\
72.01	0\\
72.02	0\\
72.03	0\\
72.04	0\\
72.05	0\\
72.06	0\\
72.07	0\\
72.08	0\\
72.09	0\\
72.1	0\\
72.11	0\\
72.12	0\\
72.13	0\\
72.14	0\\
72.15	0\\
72.16	0\\
72.17	0\\
72.18	0\\
72.19	0\\
72.2	0\\
72.21	0\\
72.22	0\\
72.23	0\\
72.24	0\\
72.25	0\\
72.26	0\\
72.27	0\\
72.28	0\\
72.29	0\\
72.3	0\\
72.31	0\\
72.32	0\\
72.33	0\\
72.34	0\\
72.35	0\\
72.36	0\\
72.37	0\\
72.38	0\\
72.39	0\\
72.4	0\\
72.41	0\\
72.42	0\\
72.43	0\\
72.44	0\\
72.45	0\\
72.46	0\\
72.47	0\\
72.48	0\\
72.49	0\\
72.5	0\\
72.51	0\\
72.52	0\\
72.53	0\\
72.54	0\\
72.55	0\\
72.56	0\\
72.57	0\\
72.58	0\\
72.59	0\\
72.6	0\\
72.61	0\\
72.62	0\\
72.63	0\\
72.64	0\\
72.65	0\\
72.66	0\\
72.67	0\\
72.68	0\\
72.69	0\\
72.7	0\\
72.71	0\\
72.72	0\\
72.73	0\\
72.74	0\\
72.75	0\\
72.76	0\\
72.77	0\\
72.78	0\\
72.79	0\\
72.8	0\\
72.81	0\\
72.82	0\\
72.83	0\\
72.84	0\\
72.85	0\\
72.86	0\\
72.87	0\\
72.88	0\\
72.89	0\\
72.9	0\\
72.91	0\\
72.92	0\\
72.93	0\\
72.94	0\\
72.95	0\\
72.96	0\\
72.97	0\\
72.98	0\\
72.99	0\\
73	0\\
73.01	0\\
73.02	0\\
73.03	0\\
73.04	0\\
73.05	0\\
73.06	0\\
73.07	0\\
73.08	0\\
73.09	0\\
73.1	0\\
73.11	0\\
73.12	0\\
73.13	0\\
73.14	0\\
73.15	0\\
73.16	0\\
73.17	0\\
73.18	0\\
73.19	0\\
73.2	0\\
73.21	0\\
73.22	0\\
73.23	0\\
73.24	0\\
73.25	0\\
73.26	0\\
73.27	0\\
73.28	0\\
73.29	0\\
73.3	0\\
73.31	0\\
73.32	0\\
73.33	0\\
73.34	0\\
73.35	0\\
73.36	0\\
73.37	0\\
73.38	0\\
73.39	0\\
73.4	0\\
73.41	0\\
73.42	0\\
73.43	0\\
73.44	0\\
73.45	0\\
73.46	0\\
73.47	0\\
73.48	0\\
73.49	0\\
73.5	0\\
73.51	0\\
73.52	0\\
73.53	0\\
73.54	0\\
73.55	0\\
73.56	0\\
73.57	0\\
73.58	0\\
73.59	0\\
73.6	0\\
73.61	0\\
73.62	0\\
73.63	0\\
73.64	0\\
73.65	0\\
73.66	0\\
73.67	0\\
73.68	0\\
73.69	0\\
73.7	0\\
73.71	0\\
73.72	0\\
73.73	0\\
73.74	0\\
73.75	0\\
73.76	0\\
73.77	0\\
73.78	0\\
73.79	0\\
73.8	0\\
73.81	0\\
73.82	0\\
73.83	0\\
73.84	0\\
73.85	0\\
73.86	0\\
73.87	0\\
73.88	0\\
73.89	0\\
73.9	0\\
73.91	0\\
73.92	0\\
73.93	0\\
73.94	0\\
73.95	0\\
73.96	0\\
73.97	0\\
73.98	0\\
73.99	0\\
74	0\\
74.01	0\\
74.02	0\\
74.03	0\\
74.04	0\\
74.05	0\\
74.06	0\\
74.07	0\\
74.08	0\\
74.09	0\\
74.1	0\\
74.11	0\\
74.12	0\\
74.13	0\\
74.14	0\\
74.15	0\\
74.16	0\\
74.17	0\\
74.18	0\\
74.19	0\\
74.2	0\\
74.21	0\\
74.22	0\\
74.23	0\\
74.24	0\\
74.25	0\\
74.26	0\\
74.27	0\\
74.28	0\\
74.29	0\\
74.3	0\\
74.31	0\\
74.32	0\\
74.33	0\\
74.34	0\\
74.35	0\\
74.36	0\\
74.37	0\\
74.38	0\\
74.39	0\\
74.4	0\\
74.41	0\\
74.42	0\\
74.43	0\\
74.44	0\\
74.45	0\\
74.46	0\\
74.47	0\\
74.48	0\\
74.49	0\\
74.5	0\\
74.51	0\\
74.52	0\\
74.53	0\\
74.54	0\\
74.55	0\\
74.56	0\\
74.57	0\\
74.58	0\\
74.59	0\\
74.6	0\\
74.61	0\\
74.62	0\\
74.63	0\\
74.64	0\\
74.65	0\\
74.66	0\\
74.67	0\\
74.68	0\\
74.69	0\\
74.7	0\\
74.71	0\\
74.72	0\\
74.73	0\\
74.74	0\\
74.75	0\\
74.76	0\\
74.77	0\\
74.78	0\\
74.79	0\\
74.8	0\\
74.81	0\\
74.82	0\\
74.83	0\\
74.84	0\\
74.85	0\\
74.86	0\\
74.87	0\\
74.88	0\\
74.89	0\\
74.9	0\\
74.91	0\\
74.92	0\\
74.93	0\\
74.94	0\\
74.95	0\\
74.96	0\\
74.97	0\\
74.98	0\\
74.99	0\\
75	0\\
75.01	0\\
75.02	0\\
75.03	0\\
75.04	0\\
75.05	0\\
75.06	0\\
75.07	0\\
75.08	0\\
75.09	0\\
75.1	0\\
75.11	0\\
75.12	0\\
75.13	0\\
75.14	0\\
75.15	0\\
75.16	0\\
75.17	0\\
75.18	0\\
75.19	0\\
75.2	0\\
75.21	0\\
75.22	0\\
75.23	0\\
75.24	0\\
75.25	0\\
75.26	0\\
75.27	0\\
75.28	0\\
75.29	0\\
75.3	0\\
75.31	0\\
75.32	0\\
75.33	0\\
75.34	0\\
75.35	0\\
75.36	0\\
75.37	0\\
75.38	0\\
75.39	0\\
75.4	0\\
75.41	0\\
75.42	0\\
75.43	0\\
75.44	0\\
75.45	0\\
75.46	0\\
75.47	0\\
75.48	0\\
75.49	0\\
75.5	0\\
75.51	0\\
75.52	0\\
75.53	0\\
75.54	0\\
75.55	0\\
75.56	0\\
75.57	0\\
75.58	0\\
75.59	0\\
75.6	0\\
75.61	0\\
75.62	0\\
75.63	0\\
75.64	0\\
75.65	0\\
75.66	0\\
75.67	0\\
75.68	0\\
75.69	0\\
75.7	0\\
75.71	0\\
75.72	0\\
75.73	0\\
75.74	0\\
75.75	0\\
75.76	0\\
75.77	0\\
75.78	0\\
75.79	0\\
75.8	0\\
75.81	0\\
75.82	0\\
75.83	0\\
75.84	0\\
75.85	0\\
75.86	0\\
75.87	0\\
75.88	0\\
75.89	0\\
75.9	0\\
75.91	0\\
75.92	0\\
75.93	0\\
75.94	0\\
75.95	0\\
75.96	0\\
75.97	0\\
75.98	0\\
75.99	0\\
76	0\\
76.01	0\\
76.02	0\\
76.03	0\\
76.04	0\\
76.05	0\\
76.06	0\\
76.07	0\\
76.08	0\\
76.09	0\\
76.1	0\\
76.11	0\\
76.12	0\\
76.13	0\\
76.14	0\\
76.15	0\\
76.16	0\\
76.17	0\\
76.18	0\\
76.19	0\\
76.2	0\\
76.21	0\\
76.22	0\\
76.23	0\\
76.24	0\\
76.25	0\\
76.26	0\\
76.27	0\\
76.28	0\\
76.29	0\\
76.3	0\\
76.31	0\\
76.32	0\\
76.33	0\\
76.34	0\\
76.35	0\\
76.36	0\\
76.37	0\\
76.38	0\\
76.39	0\\
76.4	0\\
76.41	0\\
76.42	0\\
76.43	0\\
76.44	0\\
76.45	0\\
76.46	0\\
76.47	0\\
76.48	0\\
76.49	0\\
76.5	0\\
76.51	0\\
76.52	0\\
76.53	0\\
76.54	0\\
76.55	0\\
76.56	0\\
76.57	0\\
76.58	0\\
76.59	0\\
76.6	0\\
76.61	0\\
76.62	0\\
76.63	0\\
76.64	0\\
76.65	0\\
76.66	0\\
76.67	0\\
76.68	0\\
76.69	0\\
76.7	0\\
76.71	0\\
76.72	0\\
76.73	0\\
76.74	0\\
76.75	0\\
76.76	0\\
76.77	0\\
76.78	0\\
76.79	0\\
76.8	0\\
76.81	0\\
76.82	0\\
76.83	0\\
76.84	0\\
76.85	0\\
76.86	0\\
76.87	0\\
76.88	0\\
76.89	0\\
76.9	0\\
76.91	0\\
76.92	0\\
76.93	0\\
76.94	0\\
76.95	0\\
76.96	0\\
76.97	0\\
76.98	0\\
76.99	0\\
77	0\\
77.01	0\\
77.02	0\\
77.03	0\\
77.04	0\\
77.05	0\\
77.06	0\\
77.07	0\\
77.08	0\\
77.09	0\\
77.1	0\\
77.11	0\\
77.12	0\\
77.13	0\\
77.14	0\\
77.15	0\\
77.16	0\\
77.17	0\\
77.18	0\\
77.19	0\\
77.2	0\\
77.21	0\\
77.22	0\\
77.23	0\\
77.24	0\\
77.25	0\\
77.26	0\\
77.27	0\\
77.28	0\\
77.29	0\\
77.3	0\\
77.31	0\\
77.32	0\\
77.33	0\\
77.34	0\\
77.35	0\\
77.36	0\\
77.37	0\\
77.38	0\\
77.39	0\\
77.4	0\\
77.41	0\\
77.42	0\\
77.43	0\\
77.44	0\\
77.45	0\\
77.46	0\\
77.47	0\\
77.48	0\\
77.49	0\\
77.5	0\\
77.51	0\\
77.52	0\\
77.53	0\\
77.54	0\\
77.55	0\\
77.56	0\\
77.57	0\\
77.58	0\\
77.59	0\\
77.6	0\\
77.61	0\\
77.62	0\\
77.63	0\\
77.64	0\\
77.65	0\\
77.66	0\\
77.67	0\\
77.68	0\\
77.69	0\\
77.7	0\\
77.71	0\\
77.72	0\\
77.73	0\\
77.74	0\\
77.75	0\\
77.76	0\\
77.77	0\\
77.78	0\\
77.79	0\\
77.8	0\\
77.81	0\\
77.82	0\\
77.83	0\\
77.84	0\\
77.85	0\\
77.86	0\\
77.87	0\\
77.88	0\\
77.89	0\\
77.9	0\\
77.91	0\\
77.92	0\\
77.93	0\\
77.94	0\\
77.95	0\\
77.96	0\\
77.97	0\\
77.98	0\\
77.99	0\\
78	0\\
78.01	0\\
78.02	0\\
78.03	0\\
78.04	0\\
78.05	0\\
78.06	0\\
78.07	0\\
78.08	0\\
78.09	0\\
78.1	0\\
78.11	0\\
78.12	0\\
78.13	0\\
78.14	0\\
78.15	0\\
78.16	0\\
78.17	0\\
78.18	0\\
78.19	0\\
78.2	0\\
78.21	0\\
78.22	0\\
78.23	0\\
78.24	0\\
78.25	0\\
78.26	0\\
78.27	0\\
78.28	0\\
78.29	0\\
78.3	0\\
78.31	0\\
78.32	0\\
78.33	0\\
78.34	0\\
78.35	0\\
78.36	0\\
78.37	0\\
78.38	0\\
78.39	0\\
78.4	0\\
78.41	0\\
78.42	0\\
78.43	0\\
78.44	0\\
78.45	0\\
78.46	0\\
78.47	0\\
78.48	0\\
78.49	0\\
78.5	0\\
78.51	0\\
78.52	0\\
78.53	0\\
78.54	0\\
78.55	0\\
78.56	0\\
78.57	0\\
78.58	0\\
78.59	0\\
78.6	0\\
78.61	0\\
78.62	0\\
78.63	0\\
78.64	0\\
78.65	0\\
78.66	0\\
78.67	0\\
78.68	0\\
78.69	0\\
78.7	0\\
78.71	0\\
78.72	0\\
78.73	0\\
78.74	0\\
78.75	0\\
78.76	0\\
78.77	0\\
78.78	0\\
78.79	0\\
78.8	0\\
78.81	0\\
78.82	0\\
78.83	0\\
78.84	0\\
78.85	0\\
78.86	0\\
78.87	0\\
78.88	0\\
78.89	0\\
78.9	0\\
78.91	0\\
78.92	0\\
78.93	0\\
78.94	0\\
78.95	0\\
78.96	0\\
78.97	0\\
78.98	0\\
78.99	0\\
79	0\\
79.01	0\\
79.02	0\\
79.03	0\\
79.04	0\\
79.05	0\\
79.06	0\\
79.07	0\\
79.08	0\\
79.09	0\\
79.1	0\\
79.11	0\\
79.12	0\\
79.13	0\\
79.14	0\\
79.15	0\\
79.16	0\\
79.17	0\\
79.18	0\\
79.19	0\\
79.2	0\\
79.21	0\\
79.22	0\\
79.23	0\\
79.24	0\\
79.25	0\\
79.26	0\\
79.27	0\\
79.28	0\\
79.29	0\\
79.3	0\\
79.31	0\\
79.32	0\\
79.33	0\\
79.34	0\\
79.35	0\\
79.36	0\\
79.37	0\\
79.38	0\\
79.39	0\\
79.4	0\\
79.41	0\\
79.42	0\\
79.43	0\\
79.44	0\\
79.45	0\\
79.46	0\\
79.47	0\\
79.48	0\\
79.49	0\\
79.5	0\\
79.51	0\\
79.52	0\\
79.53	0\\
79.54	0\\
79.55	0\\
79.56	0\\
79.57	0\\
79.58	0\\
79.59	0\\
79.6	0\\
79.61	0\\
79.62	0\\
79.63	0\\
79.64	0\\
79.65	0\\
79.66	0\\
79.67	0\\
79.68	0\\
79.69	0\\
79.7	0\\
79.71	0\\
79.72	0\\
79.73	0\\
79.74	0\\
79.75	0\\
79.76	0\\
79.77	0\\
79.78	0\\
79.79	0\\
79.8	0\\
79.81	0\\
79.82	0\\
79.83	0\\
79.84	0\\
79.85	0\\
79.86	0\\
79.87	0\\
79.88	0\\
79.89	0\\
79.9	0\\
79.91	0\\
79.92	0\\
79.93	0\\
79.94	0\\
79.95	0\\
79.96	0\\
79.97	0\\
79.98	0\\
79.99	0\\
80	0\\
80.01	0\\
};
\addplot [color=black,solid]
  table[row sep=crcr]{%
80.01	0\\
80.02	0\\
80.03	0\\
80.04	0\\
80.05	0\\
80.06	0\\
80.07	0\\
80.08	0\\
80.09	0\\
80.1	0\\
80.11	0\\
80.12	0\\
80.13	0\\
80.14	0\\
80.15	0\\
80.16	0\\
80.17	0\\
80.18	0\\
80.19	0\\
80.2	0\\
80.21	0\\
80.22	0\\
80.23	0\\
80.24	0\\
80.25	0\\
80.26	0\\
80.27	0\\
80.28	0\\
80.29	0\\
80.3	0\\
80.31	0\\
80.32	0\\
80.33	0\\
80.34	0\\
80.35	0\\
80.36	0\\
80.37	0\\
80.38	0\\
80.39	0\\
80.4	0\\
80.41	0\\
80.42	0\\
80.43	0\\
80.44	0\\
80.45	0\\
80.46	0\\
80.47	0\\
80.48	0\\
80.49	0\\
80.5	0\\
80.51	0\\
80.52	0\\
80.53	0\\
80.54	0\\
80.55	0\\
80.56	0\\
80.57	0\\
80.58	0\\
80.59	0\\
80.6	0\\
80.61	0\\
80.62	0\\
80.63	0\\
80.64	0\\
80.65	0\\
80.66	0\\
80.67	0\\
80.68	0\\
80.69	0\\
80.7	0\\
80.71	0\\
80.72	0\\
80.73	0\\
80.74	0\\
80.75	0\\
80.76	0\\
80.77	0\\
80.78	0\\
80.79	0\\
80.8	0\\
80.81	0\\
80.82	0\\
80.83	0\\
80.84	0\\
80.85	0\\
80.86	0\\
80.87	0\\
80.88	0\\
80.89	0\\
80.9	0\\
80.91	0\\
80.92	0\\
80.93	0\\
80.94	0\\
80.95	0\\
80.96	0\\
80.97	0\\
80.98	0\\
80.99	0\\
81	0\\
81.01	0\\
81.02	0\\
81.03	0\\
81.04	0\\
81.05	0\\
81.06	0\\
81.07	0\\
81.08	0\\
81.09	0\\
81.1	0\\
81.11	0\\
81.12	0\\
81.13	0\\
81.14	0\\
81.15	0\\
81.16	0\\
81.17	0\\
81.18	0\\
81.19	0\\
81.2	0\\
81.21	0\\
81.22	0\\
81.23	0\\
81.24	0\\
81.25	0\\
81.26	0\\
81.27	0\\
81.28	0\\
81.29	0\\
81.3	0\\
81.31	0\\
81.32	0\\
81.33	0\\
81.34	0\\
81.35	0\\
81.36	0\\
81.37	0\\
81.38	0\\
81.39	0\\
81.4	0\\
81.41	0\\
81.42	0\\
81.43	0\\
81.44	0\\
81.45	0\\
81.46	0\\
81.47	0\\
81.48	0\\
81.49	0\\
81.5	0\\
81.51	0\\
81.52	0\\
81.53	0\\
81.54	0\\
81.55	0\\
81.56	0\\
81.57	0\\
81.58	0\\
81.59	0\\
81.6	0\\
81.61	0\\
81.62	0\\
81.63	0\\
81.64	0\\
81.65	0\\
81.66	0\\
81.67	0\\
81.68	0\\
81.69	0\\
81.7	0\\
81.71	0\\
81.72	0\\
81.73	0\\
81.74	0\\
81.75	0\\
81.76	0\\
81.77	0\\
81.78	0\\
81.79	0\\
81.8	0\\
81.81	0\\
81.82	0\\
81.83	0\\
81.84	0\\
81.85	0\\
81.86	0\\
81.87	0\\
81.88	0\\
81.89	0\\
81.9	0\\
81.91	0\\
81.92	0\\
81.93	0\\
81.94	0\\
81.95	0\\
81.96	0\\
81.97	0\\
81.98	0\\
81.99	0\\
82	0\\
82.01	0\\
82.02	0\\
82.03	0\\
82.04	0\\
82.05	0\\
82.06	0\\
82.07	0\\
82.08	0\\
82.09	0\\
82.1	0\\
82.11	0\\
82.12	0\\
82.13	0\\
82.14	0\\
82.15	0\\
82.16	0\\
82.17	0\\
82.18	0\\
82.19	0\\
82.2	0\\
82.21	0\\
82.22	0\\
82.23	0\\
82.24	0\\
82.25	0\\
82.26	0\\
82.27	0\\
82.28	0\\
82.29	0\\
82.3	0\\
82.31	0\\
82.32	0\\
82.33	0\\
82.34	0\\
82.35	0\\
82.36	0\\
82.37	0\\
82.38	0\\
82.39	0\\
82.4	0\\
82.41	0\\
82.42	0\\
82.43	0\\
82.44	0\\
82.45	0\\
82.46	0\\
82.47	0\\
82.48	0\\
82.49	0\\
82.5	0\\
82.51	0\\
82.52	0\\
82.53	0\\
82.54	0\\
82.55	0\\
82.56	0\\
82.57	0\\
82.58	0\\
82.59	0\\
82.6	0\\
82.61	0\\
82.62	0\\
82.63	0\\
82.64	0\\
82.65	0\\
82.66	0\\
82.67	0\\
82.68	0\\
82.69	0\\
82.7	0\\
82.71	0\\
82.72	0\\
82.73	0\\
82.74	0\\
82.75	0\\
82.76	0\\
82.77	0\\
82.78	0\\
82.79	0\\
82.8	0\\
82.81	0\\
82.82	0\\
82.83	0\\
82.84	0\\
82.85	0\\
82.86	0\\
82.87	0\\
82.88	0\\
82.89	0\\
82.9	0\\
82.91	0\\
82.92	0\\
82.93	0\\
82.94	0\\
82.95	0\\
82.96	0\\
82.97	0\\
82.98	0\\
82.99	0\\
83	0\\
83.01	0\\
83.02	0\\
83.03	0\\
83.04	0\\
83.05	0\\
83.06	0\\
83.07	0\\
83.08	0\\
83.09	0\\
83.1	0\\
83.11	0\\
83.12	0\\
83.13	0\\
83.14	0\\
83.15	0\\
83.16	0\\
83.17	0\\
83.18	0\\
83.19	0\\
83.2	0\\
83.21	0\\
83.22	0\\
83.23	0\\
83.24	0\\
83.25	0\\
83.26	0\\
83.27	0\\
83.28	0\\
83.29	0\\
83.3	0\\
83.31	0\\
83.32	0\\
83.33	0\\
83.34	0\\
83.35	0\\
83.36	0\\
83.37	0\\
83.38	0\\
83.39	0\\
83.4	0\\
83.41	0\\
83.42	0\\
83.43	0\\
83.44	0\\
83.45	0\\
83.46	0\\
83.47	0\\
83.48	0\\
83.49	0\\
83.5	0\\
83.51	0\\
83.52	0\\
83.53	0\\
83.54	0\\
83.55	0\\
83.56	0\\
83.57	0\\
83.58	0\\
83.59	0\\
83.6	0\\
83.61	0\\
83.62	0\\
83.63	0\\
83.64	0\\
83.65	0\\
83.66	0\\
83.67	0\\
83.68	0\\
83.69	0\\
83.7	0\\
83.71	0\\
83.72	0\\
83.73	0\\
83.74	0\\
83.75	0\\
83.76	0\\
83.77	0\\
83.78	0\\
83.79	0\\
83.8	0\\
83.81	0\\
83.82	0\\
83.83	0\\
83.84	0\\
83.85	0\\
83.86	0\\
83.87	0\\
83.88	0\\
83.89	0\\
83.9	0\\
83.91	0\\
83.92	0\\
83.93	0\\
83.94	0\\
83.95	0\\
83.96	0\\
83.97	0\\
83.98	0\\
83.99	0\\
84	0\\
84.01	0\\
84.02	0\\
84.03	0\\
84.04	0\\
84.05	0\\
84.06	0\\
84.07	0\\
84.08	0\\
84.09	0\\
84.1	0\\
84.11	0\\
84.12	0\\
84.13	0\\
84.14	0\\
84.15	0\\
84.16	0\\
84.17	0\\
84.18	0\\
84.19	0\\
84.2	0\\
84.21	0\\
84.22	0\\
84.23	0\\
84.24	0\\
84.25	0\\
84.26	0\\
84.27	0\\
84.28	0\\
84.29	0\\
84.3	0\\
84.31	0\\
84.32	0\\
84.33	0\\
84.34	0\\
84.35	0\\
84.36	0\\
84.37	0\\
84.38	0\\
84.39	0\\
84.4	0\\
84.41	0\\
84.42	0\\
84.43	0\\
84.44	0\\
84.45	0\\
84.46	0\\
84.47	0\\
84.48	0\\
84.49	0\\
84.5	0\\
84.51	0\\
84.52	0\\
84.53	0\\
84.54	0\\
84.55	0\\
84.56	0\\
84.57	0\\
84.58	0\\
84.59	0\\
84.6	0\\
84.61	0\\
84.62	0\\
84.63	0\\
84.64	0\\
84.65	0\\
84.66	0\\
84.67	0\\
84.68	0\\
84.69	0\\
84.7	0\\
84.71	0\\
84.72	0\\
84.73	0\\
84.74	0\\
84.75	0\\
84.76	0\\
84.77	0\\
84.78	0\\
84.79	0\\
84.8	0\\
84.81	0\\
84.82	0\\
84.83	0\\
84.84	0\\
84.85	0\\
84.86	0\\
84.87	0\\
84.88	0\\
84.89	0\\
84.9	0\\
84.91	0\\
84.92	0\\
84.93	0\\
84.94	0\\
84.95	0\\
84.96	0\\
84.97	0\\
84.98	0\\
84.99	0\\
85	0\\
85.01	0\\
85.02	0\\
85.03	0\\
85.04	0\\
85.05	0\\
85.06	0\\
85.07	0\\
85.08	0\\
85.09	0\\
85.1	0\\
85.11	0\\
85.12	0\\
85.13	0\\
85.14	0\\
85.15	0\\
85.16	0\\
85.17	0\\
85.18	0\\
85.19	0\\
85.2	0\\
85.21	0\\
85.22	0\\
85.23	0\\
85.24	0\\
85.25	0\\
85.26	0\\
85.27	0\\
85.28	0\\
85.29	0\\
85.3	0\\
85.31	0\\
85.32	0\\
85.33	0\\
85.34	0\\
85.35	0\\
85.36	0\\
85.37	0\\
85.38	0\\
85.39	0\\
85.4	0\\
85.41	0\\
85.42	0\\
85.43	0\\
85.44	0\\
85.45	0\\
85.46	0\\
85.47	0\\
85.48	0\\
85.49	0\\
85.5	0\\
85.51	0\\
85.52	0\\
85.53	0\\
85.54	0\\
85.55	0\\
85.56	0\\
85.57	0\\
85.58	0\\
85.59	0\\
85.6	0\\
85.61	0\\
85.62	0\\
85.63	0\\
85.64	0\\
85.65	0\\
85.66	0\\
85.67	0\\
85.68	0\\
85.69	0\\
85.7	0\\
85.71	0\\
85.72	0\\
85.73	0\\
85.74	0\\
85.75	0\\
85.76	0\\
85.77	0\\
85.78	0\\
85.79	0\\
85.8	0\\
85.81	0\\
85.82	0\\
85.83	0\\
85.84	0\\
85.85	0\\
85.86	0\\
85.87	0\\
85.88	0\\
85.89	0\\
85.9	0\\
85.91	0\\
85.92	0\\
85.93	0\\
85.94	0\\
85.95	0\\
85.96	0\\
85.97	0\\
85.98	0\\
85.99	0\\
86	0\\
86.01	0\\
86.02	0\\
86.03	0\\
86.04	0\\
86.05	0\\
86.06	0\\
86.07	0\\
86.08	0\\
86.09	0\\
86.1	0\\
86.11	0\\
86.12	0\\
86.13	0\\
86.14	0\\
86.15	0\\
86.16	0\\
86.17	0\\
86.18	0\\
86.19	0\\
86.2	0\\
86.21	0\\
86.22	0\\
86.23	0\\
86.24	0\\
86.25	0\\
86.26	0\\
86.27	0\\
86.28	0\\
86.29	0\\
86.3	0\\
86.31	0\\
86.32	0\\
86.33	0\\
86.34	0\\
86.35	0\\
86.36	0\\
86.37	0\\
86.38	0\\
86.39	0\\
86.4	0\\
86.41	0\\
86.42	0\\
86.43	0\\
86.44	0\\
86.45	0\\
86.46	0\\
86.47	0\\
86.48	0\\
86.49	0\\
86.5	0\\
86.51	0\\
86.52	0\\
86.53	0\\
86.54	0\\
86.55	0\\
86.56	0\\
86.57	0\\
86.58	0\\
86.59	0\\
86.6	0\\
86.61	0\\
86.62	0\\
86.63	0\\
86.64	0\\
86.65	0\\
86.66	0\\
86.67	0\\
86.68	0\\
86.69	0\\
86.7	0\\
86.71	0\\
86.72	0\\
86.73	0\\
86.74	0\\
86.75	0\\
86.76	0\\
86.77	0\\
86.78	0\\
86.79	0\\
86.8	0\\
86.81	0\\
86.82	3.27970578975663e-06\\
86.83	7.9811883659086e-06\\
86.84	1.26868625115548e-05\\
86.85	1.73967356682046e-05\\
86.86	2.21108152934513e-05\\
86.87	2.68291088610009e-05\\
86.88	3.15516238606952e-05\\
86.89	3.62783677985336e-05\\
86.9	4.10093481966987e-05\\
86.91	4.5744572593584e-05\\
86.92	5.0484048543813e-05\\
86.93	5.52277836182655e-05\\
86.94	5.9975785404098e-05\\
86.95	6.47280615047718e-05\\
86.96	6.94846195400719e-05\\
86.97	7.42454671461331e-05\\
86.98	7.9010611975459e-05\\
86.99	8.37800616969499e-05\\
87	8.85538239959183e-05\\
87.01	9.33319065741164e-05\\
87.02	9.8114317149756e-05\\
87.03	0.000102901063457532\\
87.04	0.000107692153248637\\
87.05	0.000112487594290792\\
87.06	0.000117287394368261\\
87.07	0.000122091561281873\\
87.08	0.000126900102849044\\
87.09	0.000131713026903795\\
87.1	0.000136530341296773\\
87.11	0.000141352053895271\\
87.12	0.000146178172583247\\
87.13	0.000151008705261342\\
87.14	0.0001558436598469\\
87.15	0.000160683044273989\\
87.16	0.000165526866493413\\
87.17	0.000170375134472734\\
87.18	0.000175227856196291\\
87.19	0.000180085039665216\\
87.2	0.000184946692897448\\
87.21	0.000189812823927755\\
87.22	0.000194683440807747\\
87.23	0.000199558551605892\\
87.24	0.000204438164407532\\
87.25	0.000209322287314902\\
87.26	0.00021421092844714\\
87.27	0.000219104095940302\\
87.28	0.000224001797947381\\
87.29	0.00022890404263832\\
87.3	0.000233810838200019\\
87.31	0.000238722192836355\\
87.32	0.000243638114768192\\
87.33	0.000248558612233397\\
87.34	0.000253483693486848\\
87.35	0.000258413366800447\\
87.36	0.000263347640463129\\
87.37	0.000268286522780881\\
87.38	0.000273230022076741\\
87.39	0.000278178146690818\\
87.4	0.000283130904980295\\
87.41	0.000288088305319439\\
87.42	0.000293050356099615\\
87.43	0.000298017065729286\\
87.44	0.000302988442634029\\
87.45	0.000307964495256535\\
87.46	0.000312945232056619\\
87.47	0.00031793066151123\\
87.48	0.000322920792114451\\
87.49	0.000327915632377506\\
87.5	0.000332915190828766\\
87.51	0.00033791947601375\\
87.52	0.000342928496495136\\
87.53	0.000347942260852755\\
87.54	0.000352960777683599\\
87.55	0.000357984055601824\\
87.56	0.000363012103238747\\
87.57	0.000368044929242852\\
87.58	0.000373082542279787\\
87.59	0.000378124951032363\\
87.6	0.000383172164200555\\
87.61	0.000388224190501503\\
87.62	0.000393281038669506\\
87.63	0.000398342717456018\\
87.64	0.000403409235629648\\
87.65	0.000408480601976159\\
87.66	0.000413556825298454\\
87.67	0.000418637914416578\\
87.68	0.000423723878167713\\
87.69	0.000428814725406164\\
87.7	0.000433910465003354\\
87.71	0.000439011105847827\\
87.72	0.000444116656845218\\
87.73	0.000449227126918264\\
87.74	0.000454342525006779\\
87.75	0.000459462860067653\\
87.76	0.000464588141074833\\
87.77	0.000469718377019313\\
87.78	0.00047485357690912\\
87.79	0.000479993749769304\\
87.8	0.000485138904641913\\
87.81	0.000490289050585988\\
87.82	0.00049544419667754\\
87.83	0.000500604352009531\\
87.84	0.000505769525691859\\
87.85	0.000510939726851341\\
87.86	0.000516114964631685\\
87.87	0.000521295248193474\\
87.88	0.000526480586714144\\
87.89	0.000531670989387957\\
87.9	0.000536866465425983\\
87.91	0.00054206702405607\\
87.92	0.000547272674522819\\
87.93	0.000552483426087559\\
87.94	0.000557699288028317\\
87.95	0.000562920269639786\\
87.96	0.000568146380233302\\
87.97	0.000573377629136807\\
87.98	0.000578614025694821\\
87.99	0.000583855579268405\\
88	0.000589102299235127\\
88.01	0.00059435419498903\\
88.02	0.00059961127594059\\
88.03	0.000604873551516682\\
88.04	0.000610141031160539\\
88.05	0.000615413724331715\\
88.06	0.000620691640506037\\
88.07	0.00062597478917557\\
88.08	0.000631263179848568\\
88.09	0.00063655682204943\\
88.1	0.000641855725318654\\
88.11	0.000647159899212789\\
88.12	0.000652469353304385\\
88.13	0.000657784097181946\\
88.14	0.000663104140449873\\
88.15	0.000668429492728415\\
88.16	0.000673760163653612\\
88.17	0.000679096162877242\\
88.18	0.000684437500066758\\
88.19	0.000689784184905236\\
88.2	0.000695136227091308\\
88.21	0.000700493636339108\\
88.22	0.0007058564223782\\
88.23	0.000711224594953517\\
88.24	0.000716598163825298\\
88.25	0.000721977138769012\\
88.26	0.000727361529575298\\
88.27	0.000732751346049883\\
88.28	0.000738146598013521\\
88.29	0.000743547295301903\\
88.3	0.000748953447765597\\
88.31	0.000754365065269959\\
88.32	0.000759782157695057\\
88.33	0.000765204734935587\\
88.34	0.000770632806900791\\
88.35	0.000776066383514375\\
88.36	0.000781505474714417\\
88.37	0.000786950090453278\\
88.38	0.000792400240697516\\
88.39	0.000797855935427786\\
88.4	0.00080331718463875\\
88.41	0.000808783998338981\\
88.42	0.00081425638655086\\
88.43	0.000819734359310476\\
88.44	0.000825217926667527\\
88.45	0.000830707098685211\\
88.46	0.000836201885440117\\
88.47	0.000841702297022122\\
88.48	0.000847208343534273\\
88.49	0.000852720035092675\\
88.5	0.000858237381826377\\
88.51	0.000863760393877252\\
88.52	0.000869289081399873\\
88.53	0.000874823454561399\\
88.54	0.000880363523541438\\
88.55	0.000885909298531927\\
88.56	0.000891460789736998\\
88.57	0.000897018007372845\\
88.58	0.000902580961667593\\
88.59	0.000908149662861152\\
88.6	0.000913724121205085\\
88.61	0.00091930434696246\\
88.62	0.000924890350407701\\
88.63	0.000930482141826446\\
88.64	0.000936079731515393\\
88.65	0.000941683129782142\\
88.66	0.000947292346945044\\
88.67	0.000952907393333039\\
88.68	0.000958528279285489\\
88.69	0.000964155015152018\\
88.7	0.000969787611292339\\
88.71	0.000975426078076087\\
88.72	0.000981070425882639\\
88.73	0.000986720665100936\\
88.74	0.000992376806129309\\
88.75	0.000998038859375284\\
88.76	0.0010037068352554\\
88.77	0.00100938074419503\\
88.78	0.00101506059662815\\
88.79	0.00102074640299719\\
88.8	0.00102643817375279\\
88.81	0.00103213591935363\\
88.82	0.0010378396502662\\
88.83	0.00104354937696459\\
88.84	0.00104926510993028\\
88.85	0.00105498685965191\\
88.86	0.0010607146366251\\
88.87	0.00106644845135216\\
88.88	0.00107218831434189\\
88.89	0.00107793423610936\\
88.9	0.00108368622717565\\
88.91	0.00108944429806762\\
88.92	0.00109520845931765\\
88.93	0.0011009787214634\\
88.94	0.00110675509504756\\
88.95	0.00111253759061758\\
88.96	0.00111832621872541\\
88.97	0.00112412098992724\\
88.98	0.00112992191478323\\
88.99	0.00113572900385719\\
89	0.00114154226771638\\
89.01	0.00114736171693115\\
89.02	0.00115318736207468\\
89.03	0.00115901921372272\\
89.04	0.0011648572824532\\
89.05	0.00117070157884602\\
89.06	0.00117655211348268\\
89.07	0.00118240889694601\\
89.08	0.00118827193981979\\
89.09	0.0011941412526885\\
89.1	0.00120001684613694\\
89.11	0.00120589873074988\\
89.12	0.00121178691711179\\
89.13	0.00121768141580642\\
89.14	0.00122358223741649\\
89.15	0.00122948939252332\\
89.16	0.00123540289170648\\
89.17	0.0012413227455434\\
89.18	0.00124724896460901\\
89.19	0.00125318029646694\\
89.2	0.00125911453504934\\
89.21	0.00126505168240045\\
89.22	0.00127099174056352\\
89.23	0.00127693471158071\\
89.24	0.00128288059749313\\
89.25	0.00128882940034078\\
89.26	0.00129478112216256\\
89.27	0.00130073576499621\\
89.28	0.00130669333087835\\
89.29	0.00131265382184441\\
89.3	0.00131861723992862\\
89.31	0.00132458358716402\\
89.32	0.00133055286558238\\
89.33	0.00133652507721424\\
89.34	0.00134250022408888\\
89.35	0.00134847830823424\\
89.36	0.00135445933167698\\
89.37	0.0013604432964424\\
89.38	0.00136643020455446\\
89.39	0.00137242005803573\\
89.4	0.00137841285890736\\
89.41	0.00138440860918911\\
89.42	0.00139040731089926\\
89.43	0.00139640896605465\\
89.44	0.00140241357667062\\
89.45	0.00140842114476098\\
89.46	0.00141443167233804\\
89.47	0.00142044516141252\\
89.48	0.00142646161399358\\
89.49	0.00143248103208877\\
89.5	0.00143850341770401\\
89.51	0.00144452877284358\\
89.52	0.00145055709951008\\
89.53	0.00145658839970441\\
89.54	0.00146262267542577\\
89.55	0.00146865992867159\\
89.56	0.00147470016143755\\
89.57	0.00148074337571753\\
89.58	0.0014867895735036\\
89.59	0.00149283875678597\\
89.6	0.00149889092755302\\
89.61	0.0015049460877912\\
89.62	0.00151100423948508\\
89.63	0.00151706538461725\\
89.64	0.00152312952516837\\
89.65	0.00152919666311709\\
89.66	0.00153526680044005\\
89.67	0.00154133993911184\\
89.68	0.00154741608110497\\
89.69	0.00155349522838989\\
89.7	0.00155957738293488\\
89.71	0.00156566254670612\\
89.72	0.00157175072166757\\
89.73	0.00157784190978101\\
89.74	0.00158393611300599\\
89.75	0.00159003333329979\\
89.76	0.00159613357261743\\
89.77	0.00160223683291159\\
89.78	0.00160834311613261\\
89.79	0.00161445242422848\\
89.8	0.00162056475914478\\
89.81	0.00162668012282467\\
89.82	0.00163279851720885\\
89.83	0.00163891994423553\\
89.84	0.00164504440584042\\
89.85	0.00165117190395668\\
89.86	0.00165730244051489\\
89.87	0.00166343601744305\\
89.88	0.00166957263666651\\
89.89	0.00167571230010797\\
89.9	0.00168185500968741\\
89.91	0.00168800076732213\\
89.92	0.00169414957492664\\
89.93	0.00170030143441267\\
89.94	0.00170645634768916\\
89.95	0.00171261431666216\\
89.96	0.00171877534323488\\
89.97	0.00172493942930759\\
89.98	0.00173110657677763\\
89.99	0.00173727678753935\\
90	0.00174345006348412\\
90.01	0.00174962640650023\\
90.02	0.00175580581847291\\
90.03	0.00176198830128429\\
90.04	0.00176817385681336\\
90.05	0.00177436248693591\\
90.06	0.00178055419352454\\
90.07	0.00178674897844859\\
90.08	0.00179294684357414\\
90.09	0.00179914779076395\\
90.1	0.00180535182187742\\
90.11	0.00181155893877057\\
90.12	0.001817769143296\\
90.13	0.00182398243730286\\
90.14	0.00183019882263679\\
90.15	0.00183641830113994\\
90.16	0.00184264087465084\\
90.17	0.00184886654500447\\
90.18	0.00185509531403213\\
90.19	0.00186132718356148\\
90.2	0.00186756215541643\\
90.21	0.00187380023141715\\
90.22	0.00188004141338004\\
90.23	0.00188628570311763\\
90.24	0.00189253310243863\\
90.25	0.00189878361314781\\
90.26	0.00190503723704599\\
90.27	0.00191129397593003\\
90.28	0.00191755383159273\\
90.29	0.00192381680582287\\
90.3	0.00193008290040506\\
90.31	0.00193635211711983\\
90.32	0.00194262445774347\\
90.33	0.00194889992404806\\
90.34	0.0019551785178014\\
90.35	0.00196146024076699\\
90.36	0.00196774509470396\\
90.37	0.00197403308136704\\
90.38	0.00198032420250652\\
90.39	0.00198661845986821\\
90.4	0.00199291585519339\\
90.41	0.00199921639021876\\
90.42	0.00200552006667641\\
90.43	0.00201182688629375\\
90.44	0.0020181368507935\\
90.45	0.00202444996189363\\
90.46	0.0020307662213073\\
90.47	0.00203708563074283\\
90.48	0.00204340819190365\\
90.49	0.00204973390648825\\
90.5	0.00205606277619015\\
90.51	0.00206239480269782\\
90.52	0.00206872998769465\\
90.53	0.00207506833285893\\
90.54	0.00208140983986374\\
90.55	0.00208775451037696\\
90.56	0.00209410234606119\\
90.57	0.00210045334857369\\
90.58	0.00210680751956637\\
90.59	0.00211316486068572\\
90.6	0.00211952537357272\\
90.61	0.00212588905986287\\
90.62	0.00213225592118607\\
90.63	0.00213862595916659\\
90.64	0.00214499917542303\\
90.65	0.00215137557156824\\
90.66	0.00215775514920931\\
90.67	0.00216413790994748\\
90.68	0.00217052385537807\\
90.69	0.00217691298709051\\
90.7	0.00218330530666816\\
90.71	0.00218970081568839\\
90.72	0.0021960995157224\\
90.73	0.00220250140833527\\
90.74	0.00220890649508581\\
90.75	0.00221531477752658\\
90.76	0.0022217262572038\\
90.77	0.00222814079385458\\
90.78	0.00223455818222918\\
90.79	0.00224097842531869\\
90.8	0.0022474015261192\\
90.81	0.00225382748763176\\
90.82	0.00226025631286246\\
90.83	0.00226668800482241\\
90.84	0.00227312256652777\\
90.85	0.00227956000099976\\
90.86	0.0022860003112647\\
90.87	0.002292443500354\\
90.88	0.0022988895713042\\
90.89	0.00230533852715696\\
90.9	0.00231179037095911\\
90.91	0.00231824510576266\\
90.92	0.0023247027346248\\
90.93	0.00233116326060795\\
90.94	0.00233762668677974\\
90.95	0.00234409301621307\\
90.96	0.0023505622519861\\
90.97	0.00235703439718229\\
90.98	0.0023635094548904\\
90.99	0.00236998742820452\\
91	0.00237646832022409\\
91.01	0.00238295213405392\\
91.02	0.00238943887280421\\
91.03	0.00239592853959057\\
91.04	0.00240242113753404\\
91.05	0.0024089166697611\\
91.06	0.00241541513940374\\
91.07	0.00242191654959939\\
91.08	0.00242842090349103\\
91.09	0.00243492820422717\\
91.1	0.00244143845496186\\
91.11	0.00244795165885477\\
91.12	0.00245446781907112\\
91.13	0.00246098693878179\\
91.14	0.0024675090211633\\
91.15	0.00247403406939781\\
91.16	0.00248056208667322\\
91.17	0.00248709307618311\\
91.18	0.00249362704112679\\
91.19	0.00250016398470937\\
91.2	0.0025067039101417\\
91.21	0.00251324682064046\\
91.22	0.00251979271942817\\
91.23	0.00252634160973318\\
91.24	0.00253289349478974\\
91.25	0.002539448377838\\
91.26	0.00254600626212403\\
91.27	0.00255256715089987\\
91.28	0.00255913104742352\\
91.29	0.002565697954959\\
91.3	0.00257226787677635\\
91.31	0.00257884081615167\\
91.32	0.00258541677636715\\
91.33	0.00259199576071106\\
91.34	0.00259857777247784\\
91.35	0.00260516281496807\\
91.36	0.00261175089148851\\
91.37	0.00261834200535216\\
91.38	0.00262493615987823\\
91.39	0.00263153335839224\\
91.4	0.00263813360422596\\
91.41	0.00264473690071753\\
91.42	0.00265134325121141\\
91.43	0.00265795265905846\\
91.44	0.00266456512761595\\
91.45	0.00267118066024758\\
91.46	0.00267779926032352\\
91.47	0.00268442093122045\\
91.48	0.00269104567632157\\
91.49	0.00269767349901664\\
91.5	0.002704304402702\\
91.51	0.00271093839078063\\
91.52	0.00271757546666214\\
91.53	0.00272421563376283\\
91.54	0.0027308588955057\\
91.55	0.00273750525532051\\
91.56	0.00274415471664378\\
91.57	0.00275080728291886\\
91.58	0.00275746295759592\\
91.59	0.00276412174413199\\
91.6	0.00277078364599105\\
91.61	0.00277744866664398\\
91.62	0.00278411680956864\\
91.63	0.0027907880782499\\
91.64	0.00279746247617968\\
91.65	0.00280414000685696\\
91.66	0.00281082067378783\\
91.67	0.00281750448048554\\
91.68	0.00282419143047049\\
91.69	0.00283088152727033\\
91.7	0.00283757477441993\\
91.71	0.00284427117546147\\
91.72	0.00285097073394445\\
91.73	0.00285767345342571\\
91.74	0.00286437933746951\\
91.75	0.00287108838964754\\
91.76	0.00287780061353896\\
91.77	0.00288451601273045\\
91.78	0.00289123459081622\\
91.79	0.00289795635139808\\
91.8	0.00290468129808548\\
91.81	0.00291140943449552\\
91.82	0.00291814076425301\\
91.83	0.0029248752909905\\
91.84	0.00293161301834836\\
91.85	0.00293835394997475\\
91.86	0.00294509808952571\\
91.87	0.00295184544066521\\
91.88	0.00295859600706515\\
91.89	0.00296534979240543\\
91.9	0.00297210680037401\\
91.91	0.0029788670346669\\
91.92	0.00298563049898825\\
91.93	0.00299239719705038\\
91.94	0.00299916713257381\\
91.95	0.00300594030928735\\
91.96	0.00301271673092806\\
91.97	0.0030194964012414\\
91.98	0.00302627932398118\\
91.99	0.00303306550290968\\
92	0.00303985494179765\\
92.01	0.00304664764442438\\
92.02	0.00305344361457773\\
92.03	0.00306024285605419\\
92.04	0.00306704537265895\\
92.05	0.00307385116820589\\
92.06	0.00308066024651768\\
92.07	0.00308747261142581\\
92.08	0.00309428826677064\\
92.09	0.00310110721640146\\
92.1	0.00310792946417653\\
92.11	0.00311475501396314\\
92.12	0.00312158386963764\\
92.13	0.00312841603508553\\
92.14	0.00313525151420147\\
92.15	0.00314209031088937\\
92.16	0.00314893242906241\\
92.17	0.00315577787264314\\
92.18	0.00316262664556347\\
92.19	0.00316947875176478\\
92.2	0.00317633419519836\\
92.21	0.00318319297982619\\
92.22	0.00319005510962103\\
92.23	0.00319692058856655\\
92.24	0.00320378942065734\\
92.25	0.00321066160989907\\
92.26	0.0032175371603085\\
92.27	0.00322441607591367\\
92.28	0.00323129836075389\\
92.29	0.00323818401887989\\
92.3	0.00324507305435391\\
92.31	0.00325196547124976\\
92.32	0.00325886127365294\\
92.33	0.00326576046566075\\
92.34	0.00327266305138236\\
92.35	0.00327956903493891\\
92.36	0.00328647842046362\\
92.37	0.00329339121210189\\
92.38	0.00330030741401141\\
92.39	0.00330722703036223\\
92.4	0.00331415006533692\\
92.41	0.00332107652313061\\
92.42	0.00332800640795116\\
92.43	0.00333493972401921\\
92.44	0.00334187647556835\\
92.45	0.00334881666684519\\
92.46	0.00335576030210946\\
92.47	0.00336270738563417\\
92.48	0.00336965792170572\\
92.49	0.00337661191462395\\
92.5	0.00338356936870236\\
92.51	0.00339053028826814\\
92.52	0.00339749467766235\\
92.53	0.00340446254124001\\
92.54	0.00341143388337023\\
92.55	0.00341840870843637\\
92.56	0.00342538702083612\\
92.57	0.00343236882498164\\
92.58	0.00343935412529971\\
92.59	0.00344634292623185\\
92.6	0.00345333523223447\\
92.61	0.00346033104777896\\
92.62	0.00346733037735189\\
92.63	0.00347433322545509\\
92.64	0.00348133959660584\\
92.65	0.00348834949533698\\
92.66	0.00349536292619707\\
92.67	0.00350237989375053\\
92.68	0.00350940040257779\\
92.69	0.00351642445727544\\
92.7	0.00352345206245638\\
92.71	0.00353048322274998\\
92.72	0.00353751794280222\\
92.73	0.00354455622727588\\
92.74	0.00355159808085066\\
92.75	0.00355864350822336\\
92.76	0.00356569251410805\\
92.77	0.00357274510323624\\
92.78	0.00357980128035701\\
92.79	0.00358686105023724\\
92.8	0.00359392441766172\\
92.81	0.00360099138743337\\
92.82	0.00360806196437341\\
92.83	0.00361513615332151\\
92.84	0.00362221395913599\\
92.85	0.00362929538667364\\
92.86	0.00363638044069388\\
92.87	0.00364346912597721\\
92.88	0.00365056144732537\\
92.89	0.0036576574095616\\
92.9	0.00366475701753088\\
92.91	0.00367186027610016\\
92.92	0.0036789671901586\\
92.93	0.00368607776461783\\
92.94	0.00369319200441218\\
92.95	0.00370030991449897\\
92.96	0.00370743149985873\\
92.97	0.00371455676549545\\
92.98	0.0037216857164369\\
92.99	0.00372881835773484\\
93	0.00373595469446534\\
93.01	0.00374309473172899\\
93.02	0.00375023847465124\\
93.03	0.00375738592838267\\
93.04	0.00376453709809923\\
93.05	0.0037716919890026\\
93.06	0.00377885060632044\\
93.07	0.00378601295530667\\
93.08	0.00379317904124182\\
93.09	0.00380034886943333\\
93.1	0.00380752244521581\\
93.11	0.0038146997739514\\
93.12	0.00382188086103009\\
93.13	0.00382906571187001\\
93.14	0.00383625433191781\\
93.15	0.00384344672664892\\
93.16	0.00385064290156795\\
93.17	0.003857842862209\\
93.18	0.00386504661413603\\
93.19	0.00387225416294316\\
93.2	0.0038794655142551\\
93.21	0.00388668067372743\\
93.22	0.00389389964704702\\
93.23	0.00390112243993241\\
93.24	0.00390834905813412\\
93.25	0.00391557950743509\\
93.26	0.00392281379365105\\
93.27	0.00393005192263089\\
93.28	0.0039372939002571\\
93.29	0.00394453973244611\\
93.3	0.00395178942514877\\
93.31	0.00395904298435068\\
93.32	0.0039663004160727\\
93.33	0.0039735617263713\\
93.34	0.00398082692133902\\
93.35	0.00398809600710491\\
93.36	0.00399536898983496\\
93.37	0.00400264587573256\\
93.38	0.00400992667103894\\
93.39	0.00401721138203364\\
93.4	0.00402450001503498\\
93.41	0.00403179257640052\\
93.42	0.00403908907252755\\
93.43	0.00404638950985358\\
93.44	0.00405369389485682\\
93.45	0.00406100223405669\\
93.46	0.00406831453401432\\
93.47	0.00407563080133307\\
93.48	0.00408295104265903\\
93.49	0.0040902752646816\\
93.5	0.00409760347413396\\
93.51	0.00410493567779366\\
93.52	0.00411227188248312\\
93.53	0.00411961209507027\\
93.54	0.00412695632246903\\
93.55	0.00413430457163991\\
93.56	0.00414165684959064\\
93.57	0.00414901316300528\\
93.58	0.00415637351846089\\
93.59	0.00416373792258096\\
93.6	0.0041711063820361\\
93.61	0.00417847890354457\\
93.62	0.00418585549387297\\
93.63	0.00419323615983681\\
93.64	0.00420062090830116\\
93.65	0.00420800974618131\\
93.66	0.00421540268044339\\
93.67	0.00422279971810507\\
93.68	0.00423020086623619\\
93.69	0.00423760613195947\\
93.7	0.00424501552245117\\
93.71	0.00425242904494182\\
93.72	0.00425984670671691\\
93.73	0.00426726851511759\\
93.74	0.00427469447754145\\
93.75	0.00428212460144319\\
93.76	0.00428955889433544\\
93.77	0.00429699736378944\\
93.78	0.00430444001743589\\
93.79	0.00431188686296565\\
93.8	0.00431933790813061\\
93.81	0.00432679316074442\\
93.82	0.00433425262868335\\
93.83	0.00434171631988708\\
93.84	0.00434918424235958\\
93.85	0.0043566564041699\\
93.86	0.00436413281345308\\
93.87	0.00437161347841098\\
93.88	0.00437909840731319\\
93.89	0.0043865876084979\\
93.9	0.00439408109037284\\
93.91	0.00440157886141618\\
93.92	0.00440908093017744\\
93.93	0.00441658730527849\\
93.94	0.00442409799541444\\
93.95	0.00443161300935467\\
93.96	0.00443913235594379\\
93.97	0.00444665604410264\\
93.98	0.0044541840828293\\
93.99	0.00446171648120012\\
94	0.00446925324837075\\
94.01	0.00447679439357722\\
94.02	0.00448433992613697\\
94.03	0.00449188985544999\\
94.04	0.00449944419099987\\
94.05	0.00450700294235491\\
94.06	0.00451456611916929\\
94.07	0.00452213373118421\\
94.08	0.004529705788229\\
94.09	0.00453728230022235\\
94.1	0.0045448632771735\\
94.11	0.00455244872918341\\
94.12	0.00456003866644602\\
94.13	0.00456763309924948\\
94.14	0.00457523203797742\\
94.15	0.00458283549311022\\
94.16	0.00459044347522631\\
94.17	0.00459805599500346\\
94.18	0.00460567306322015\\
94.19	0.00461329469075688\\
94.2	0.00462092088859757\\
94.21	0.00462855166783092\\
94.22	0.00463618703965184\\
94.23	0.00464382701536284\\
94.24	0.00465147160637551\\
94.25	0.00465912082421196\\
94.26	0.00466677468050632\\
94.27	0.00467443318700623\\
94.28	0.00468209635557437\\
94.29	0.00468976419819001\\
94.3	0.00469743672695055\\
94.31	0.00470511395407316\\
94.32	0.00471279589189633\\
94.33	0.00472048255288156\\
94.34	0.00472817394961495\\
94.35	0.00473587009480893\\
94.36	0.00474357100130393\\
94.37	0.00475127668207011\\
94.38	0.0047589871502091\\
94.39	0.00476670241895579\\
94.4	0.00477442250168007\\
94.41	0.00478214741188873\\
94.42	0.00478987716322721\\
94.43	0.00479761176948155\\
94.44	0.00480535124458021\\
94.45	0.00481309560259604\\
94.46	0.0048208448577482\\
94.47	0.00482859902440414\\
94.48	0.00483635811708157\\
94.49	0.00484412215045054\\
94.5	0.0048518911393354\\
94.51	0.00485966509871699\\
94.52	0.00486744404373465\\
94.53	0.00487522798968839\\
94.54	0.00488301695204107\\
94.55	0.00489081094642058\\
94.56	0.00489860998862204\\
94.57	0.00490641409461008\\
94.58	0.00491422328052111\\
94.59	0.00492203756266562\\
94.6	0.00492985695753054\\
94.61	0.00493768148178161\\
94.62	0.00494551115226578\\
94.63	0.00495334598601365\\
94.64	0.00496118600024193\\
94.65	0.00496903121235598\\
94.66	0.00497688163577178\\
94.67	0.0049847372760841\\
94.68	0.00499259813879543\\
94.69	0.0050004642294144\\
94.7	0.00500833555345587\\
94.71	0.00501621211644086\\
94.72	0.00502409392389658\\
94.73	0.00503198098135646\\
94.74	0.00503987329436007\\
94.75	0.00504777086845321\\
94.76	0.00505567370918784\\
94.77	0.00506358182212211\\
94.78	0.00507149521282038\\
94.79	0.00507941388685317\\
94.8	0.0050873378497972\\
94.81	0.00509526710723537\\
94.82	0.00510320166475675\\
94.83	0.00511114152795662\\
94.84	0.00511908670243642\\
94.85	0.0051270371938038\\
94.86	0.00513499300767255\\
94.87	0.00514295414966267\\
94.88	0.00515092062540034\\
94.89	0.00515889244051789\\
94.9	0.00516686960065387\\
94.91	0.00517485211145295\\
94.92	0.00518283997856602\\
94.93	0.00519083320765012\\
94.94	0.00519883180436847\\
94.95	0.00520683577439044\\
94.96	0.00521484512339161\\
94.97	0.00522285985705367\\
94.98	0.00523087998106452\\
94.99	0.0052389055011182\\
95	0.00524693642291491\\
95.01	0.00525497275216101\\
95.02	0.00526301449456903\\
95.03	0.00527106165585764\\
95.04	0.00527911424175167\\
95.05	0.00528717225798208\\
95.06	0.00529523571028601\\
95.07	0.00530330460440671\\
95.08	0.0053113789460936\\
95.09	0.00531945874110222\\
95.1	0.00532754399519427\\
95.11	0.00533563471413756\\
95.12	0.00534373090370604\\
95.13	0.0053518325696798\\
95.14	0.00535993971784504\\
95.15	0.00536805235399409\\
95.16	0.0053761704839254\\
95.17	0.00538429411344352\\
95.18	0.00539242324835915\\
95.19	0.00540055789448906\\
95.2	0.00540869805765614\\
95.21	0.0054168437436894\\
95.22	0.00542499495842392\\
95.23	0.00543315170770089\\
95.24	0.0054413139973676\\
95.25	0.0054494818332774\\
95.26	0.00545765522128975\\
95.27	0.00546583416727018\\
95.28	0.00547401867709029\\
95.29	0.00548220875662775\\
95.3	0.00549040441176632\\
95.31	0.00549860564839578\\
95.32	0.00550681247241201\\
95.33	0.00551502488971691\\
95.34	0.00552324290621844\\
95.35	0.00553146652783061\\
95.36	0.00553969576047347\\
95.37	0.00554793061007308\\
95.38	0.00555617108256156\\
95.39	0.00556441718387702\\
95.4	0.00557266891996362\\
95.41	0.0055809262967715\\
95.42	0.00558918932025684\\
95.43	0.0055974579963818\\
95.44	0.00560573233111453\\
95.45	0.00561401233042919\\
95.46	0.0056222980003059\\
95.47	0.00563058934673079\\
95.48	0.00563888637569592\\
95.49	0.00564718909319936\\
95.5	0.0056554975052451\\
95.51	0.0056638116178431\\
95.52	0.00567213143700928\\
95.53	0.00568045696876546\\
95.54	0.00568878821913943\\
95.55	0.0056971251941649\\
95.56	0.00570546789988148\\
95.57	0.0057138163423347\\
95.58	0.00572217052757601\\
95.59	0.00573053046166273\\
95.6	0.0057388961506581\\
95.61	0.00574726760063121\\
95.62	0.00575564481765705\\
95.63	0.00576402780781646\\
95.64	0.00577241657719615\\
95.65	0.00578081113188868\\
95.66	0.00578921147799245\\
95.67	0.00579761762161168\\
95.68	0.00580602956885645\\
95.69	0.00581444732584262\\
95.7	0.00582287089869188\\
95.71	0.00583130029353173\\
95.72	0.00583973551649543\\
95.73	0.00584817657372204\\
95.74	0.00585662347135641\\
95.75	0.00586507621554912\\
95.76	0.00587353481245653\\
95.77	0.00588199926824073\\
95.78	0.00589046958906955\\
95.79	0.00589894578111656\\
95.8	0.00590742785056102\\
95.81	0.00591591580358792\\
95.82	0.00592440964638792\\
95.83	0.00593290938515739\\
95.84	0.00594141502609836\\
95.85	0.00594992657541854\\
95.86	0.00595844403933126\\
95.87	0.00596696742405552\\
95.88	0.00597549673581596\\
95.89	0.00598403198084282\\
95.9	0.00599257316537195\\
95.91	0.0060011202956448\\
95.92	0.00600967337790842\\
95.93	0.00601823241841542\\
95.94	0.00602679742342398\\
95.95	0.00603536839919781\\
95.96	0.0060439453520062\\
95.97	0.00605252828812393\\
95.98	0.0060611172138313\\
95.99	0.00606971213541413\\
96	0.0060783130591637\\
96.01	0.00608691999137679\\
96.02	0.00609553293835562\\
96.03	0.00610415190640789\\
96.04	0.00611277690184671\\
96.05	0.00612140793099061\\
96.06	0.00613004500016354\\
96.07	0.00613868811569485\\
96.08	0.00614733728391926\\
96.09	0.00615599251117686\\
96.1	0.00616465380381308\\
96.11	0.00617332116817871\\
96.12	0.00618199461062984\\
96.13	0.00619067413752788\\
96.14	0.00619935975523954\\
96.15	0.00620805147013679\\
96.16	0.00621674928859688\\
96.17	0.00622545321700228\\
96.18	0.00623416326174073\\
96.19	0.00624287942920514\\
96.2	0.00625160172579366\\
96.21	0.00626033015790959\\
96.22	0.00626906473196142\\
96.23	0.00627780545436278\\
96.24	0.00628655233153242\\
96.25	0.00629530536989424\\
96.26	0.00630406457587719\\
96.27	0.00631282995591534\\
96.28	0.00632160151644781\\
96.29	0.00633037926391877\\
96.3	0.00633916320477741\\
96.31	0.00634795334547793\\
96.32	0.00635674969247953\\
96.33	0.00636555225224637\\
96.34	0.00637436103124759\\
96.35	0.00638317603595722\\
96.36	0.00639199727285426\\
96.37	0.00640082474842255\\
96.38	0.00640965846915086\\
96.39	0.00641849844153278\\
96.4	0.00642734467206676\\
96.41	0.00643619716725605\\
96.42	0.0064450559336087\\
96.43	0.00645392097763755\\
96.44	0.00646279230586019\\
96.45	0.00647166992479893\\
96.46	0.0064805538409808\\
96.47	0.00648944406093753\\
96.48	0.00649834059120552\\
96.49	0.00650724343832581\\
96.5	0.00651615260884407\\
96.51	0.00652506810931057\\
96.52	0.00653398994628017\\
96.53	0.00654291812631227\\
96.54	0.00655185265597083\\
96.55	0.00656079354182431\\
96.56	0.00656974079044565\\
96.57	0.00657869440841226\\
96.58	0.00658765440230601\\
96.59	0.00659662077871316\\
96.6	0.00660559354422437\\
96.61	0.00661457270543468\\
96.62	0.00662355826894346\\
96.63	0.0066325502413544\\
96.64	0.00664154862927548\\
96.65	0.00665055343931895\\
96.66	0.0066595646781013\\
96.67	0.00666858235224324\\
96.68	0.00667760646836966\\
96.69	0.00668663703310961\\
96.7	0.00669567405309628\\
96.71	0.00670471753496696\\
96.72	0.00671376748536304\\
96.73	0.00672282391092993\\
96.74	0.00673188681831709\\
96.75	0.00674095621417796\\
96.76	0.00675003210516996\\
96.77	0.00675911449795443\\
96.78	0.00676820339919665\\
96.79	0.00677729881556575\\
96.8	0.00678640075373473\\
96.81	0.00679550922038039\\
96.82	0.00680462422218334\\
96.83	0.00681374576582794\\
96.84	0.00682287385800228\\
96.85	0.00683200850539815\\
96.86	0.00684114971471101\\
96.87	0.00685029749263994\\
96.88	0.00685945184588765\\
96.89	0.00686861278116039\\
96.9	0.00687778030516798\\
96.91	0.0068869544246237\\
96.92	0.00689613514624436\\
96.93	0.00690532247675017\\
96.94	0.00691451642286475\\
96.95	0.00692371699131509\\
96.96	0.00693292418883154\\
96.97	0.00694213802214773\\
96.98	0.00695135849800055\\
96.99	0.00696058562313014\\
97	0.00696981940427983\\
97.01	0.00697905984819611\\
97.02	0.00698830696162859\\
97.03	0.00699756075132997\\
97.04	0.00700682122405601\\
97.05	0.00701608838656546\\
97.06	0.00702536224562006\\
97.07	0.00703464280798449\\
97.08	0.00704393008042632\\
97.09	0.00705322406971599\\
97.1	0.00706252478262675\\
97.11	0.00707183222593463\\
97.12	0.00708114640641842\\
97.13	0.0070904673308596\\
97.14	0.00709979500604231\\
97.15	0.00710912943875332\\
97.16	0.00711847063578195\\
97.17	0.00712781860392011\\
97.18	0.00713717334996215\\
97.19	0.00714653488070492\\
97.2	0.00715590320294765\\
97.21	0.00716527832349196\\
97.22	0.00717466024914177\\
97.23	0.00718404898670331\\
97.24	0.00719344454298502\\
97.25	0.00720284692479755\\
97.26	0.00721225613895369\\
97.27	0.00722167219226832\\
97.28	0.00723109509155841\\
97.29	0.00724052484364291\\
97.3	0.00724996145534274\\
97.31	0.00725940493348073\\
97.32	0.0072688552848816\\
97.33	0.00727831251637188\\
97.34	0.00728777663477986\\
97.35	0.00729724764693557\\
97.36	0.00730672555967071\\
97.37	0.0073162103798186\\
97.38	0.00732570211421415\\
97.39	0.00733520076969379\\
97.4	0.00734470635309541\\
97.41	0.00735421887125835\\
97.42	0.0073637383310233\\
97.43	0.00737326473923226\\
97.44	0.00738279810272854\\
97.45	0.00739233842835663\\
97.46	0.00740188572296218\\
97.47	0.00741143999339196\\
97.48	0.00742100124649378\\
97.49	0.00743056948911647\\
97.5	0.00744014472810977\\
97.51	0.00744972697032433\\
97.52	0.00745931622261162\\
97.53	0.00746891249182389\\
97.54	0.00747851578481409\\
97.55	0.00748812610843585\\
97.56	0.00749774346954338\\
97.57	0.00750736787499146\\
97.58	0.0075169993316353\\
97.59	0.00752663784633059\\
97.6	0.00753628342593335\\
97.61	0.0075459360772999\\
97.62	0.00755559580728682\\
97.63	0.00756526262275084\\
97.64	0.00757493653054882\\
97.65	0.00758461753753769\\
97.66	0.00759430565057435\\
97.67	0.00760400087651561\\
97.68	0.00761370322221818\\
97.69	0.00762341269453854\\
97.7	0.00763312930033291\\
97.71	0.00764285304645716\\
97.72	0.00765258393976677\\
97.73	0.00766232198711677\\
97.74	0.00767206719536161\\
97.75	0.00768181957135516\\
97.76	0.00769157912195062\\
97.77	0.00770134585400044\\
97.78	0.00771111977435625\\
97.79	0.0077209008898688\\
97.8	0.00773068920738788\\
97.81	0.00774048473376227\\
97.82	0.00775028747583962\\
97.83	0.00776009744046643\\
97.84	0.00776991463448794\\
97.85	0.00777973906474809\\
97.86	0.00778957073808939\\
97.87	0.00779940966135291\\
97.88	0.00780925584137815\\
97.89	0.00781910928500301\\
97.9	0.00782896999906365\\
97.91	0.00783883799039448\\
97.92	0.00784871326582805\\
97.93	0.00785859583219497\\
97.94	0.00786848569632381\\
97.95	0.00787838286504107\\
97.96	0.00788828734517107\\
97.97	0.00789819914353585\\
97.98	0.00790811826695511\\
97.99	0.00791804472224614\\
98	0.00792797851622369\\
98.01	0.00793791965569993\\
98.02	0.00794786814748433\\
98.03	0.00795782399838358\\
98.04	0.00796778721520148\\
98.05	0.00797775780473889\\
98.06	0.00798773577379363\\
98.07	0.00799772112916038\\
98.08	0.00800771387763031\\
98.09	0.0080177140259909\\
98.1	0.00802772158102585\\
98.11	0.00803773654951499\\
98.12	0.00804775893823414\\
98.13	0.00805778875395509\\
98.14	0.00806782600344543\\
98.15	0.00807787069346848\\
98.16	0.0080879228307832\\
98.17	0.00809798242214404\\
98.18	0.00810804947430092\\
98.19	0.00811812399399903\\
98.2	0.00812820598797874\\
98.21	0.00813829546297551\\
98.22	0.00814839242571964\\
98.23	0.00815849688293618\\
98.24	0.00816860884134483\\
98.25	0.00817872830765984\\
98.26	0.00818885528858987\\
98.27	0.0081989897908379\\
98.28	0.00820913182110115\\
98.29	0.00821928138607091\\
98.3	0.00822943849248156\\
98.31	0.0082396031470706\\
98.32	0.00824977535656853\\
98.33	0.008259955127815\\
98.34	0.00827014246888159\\
98.35	0.00828033738784738\\
98.36	0.00829053989279898\\
98.37	0.00830074999192238\\
98.38	0.00831096769341426\\
98.39	0.00832119300547873\\
98.4	0.00833142593632734\\
98.41	0.00834166649417914\\
98.42	0.0083519146872607\\
98.43	0.00836217052413896\\
98.44	0.00837243401349226\\
98.45	0.0083827051640066\\
98.46	0.00839298398437564\\
98.47	0.00840327048330072\\
98.48	0.00841356466949091\\
98.49	0.00842386655166301\\
98.5	0.00843417613854159\\
98.51	0.00844449343885901\\
98.52	0.00845481846135549\\
98.53	0.00846515121477906\\
98.54	0.00847549170788566\\
98.55	0.00848583994943916\\
98.56	0.00849619594821136\\
98.57	0.00850655971298206\\
98.58	0.00851693125253908\\
98.59	0.00852731057567829\\
98.6	0.00853769769120368\\
98.61	0.00854809260792734\\
98.62	0.00855849533466957\\
98.63	0.00856890588025885\\
98.64	0.00857932425353194\\
98.65	0.0085897504633339\\
98.66	0.00860018451851813\\
98.67	0.00861062642794642\\
98.68	0.008621076200489\\
98.69	0.00863153361642958\\
98.7	0.00864199861111742\\
98.71	0.00865247119243095\\
98.72	0.00866295136824614\\
98.73	0.00867343914566332\\
98.74	0.00868393372853303\\
98.75	0.00869443110019272\\
98.76	0.00870493122910945\\
98.77	0.00871543408338785\\
98.78	0.00872593963076603\\
98.79	0.0087364467118356\\
98.8	0.00874695496642904\\
98.81	0.00875746435560937\\
98.82	0.00876797484002521\\
98.83	0.00877848637990586\\
98.84	0.00878899893505648\\
98.85	0.00879951246485317\\
98.86	0.00881002692823806\\
98.87	0.00882054228371423\\
98.88	0.00883105848934069\\
98.89	0.00884157550272716\\
98.9	0.00885209328102884\\
98.91	0.0088626117809411\\
98.92	0.00887313095869405\\
98.93	0.00888365077004706\\
98.94	0.00889417117028323\\
98.95	0.00890469211420367\\
98.96	0.00891521355612183\\
98.97	0.00892573544985762\\
98.98	0.00893625774873154\\
98.99	0.00894678040555864\\
99	0.00895730337264244\\
99.01	0.00896782660176875\\
99.02	0.00897835004419935\\
99.03	0.00898887365066564\\
99.04	0.00899939737136215\\
99.05	0.00900992115593992\\
99.06	0.00902044495349986\\
99.07	0.00903096871258595\\
99.08	0.00904149238117846\\
99.09	0.00905201590668712\\
99.1	0.009062539235944\\
99.11	0.00907306231519623\\
99.12	0.0090835850903821\\
99.13	0.00909410750691204\\
99.14	0.00910462950959158\\
99.15	0.00911515104261371\\
99.16	0.00912567204955099\\
99.17	0.00913619247334767\\
99.18	0.00914671225631155\\
99.19	0.00915723134010582\\
99.2	0.00916774966574069\\
99.21	0.00917826717356491\\
99.22	0.00918878380325717\\
99.23	0.00919929949381728\\
99.24	0.00920981418355733\\
99.25	0.00922032781009259\\
99.26	0.0092308403103323\\
99.27	0.00924135162047032\\
99.28	0.00925186167597559\\
99.29	0.00926237041158246\\
99.3	0.00927287776128084\\
99.31	0.00928338365830616\\
99.32	0.00929388803512921\\
99.33	0.00930439082344577\\
99.34	0.00931489195416604\\
99.35	0.00932539135740396\\
99.36	0.00933588896246625\\
99.37	0.00934638469784135\\
99.38	0.00935687849118809\\
99.39	0.00936737026932421\\
99.4	0.00937785995821467\\
99.41	0.00938834748295972\\
99.42	0.00939883276778278\\
99.43	0.00940931573601812\\
99.44	0.00941979631009832\\
99.45	0.00943027441154143\\
99.46	0.00944074996093798\\
99.47	0.00945122287793775\\
99.48	0.00946169308123622\\
99.49	0.00947216048856088\\
99.5	0.0094826250166572\\
99.51	0.00949308658127434\\
99.52	0.00950354509715069\\
99.53	0.00951400047799902\\
99.54	0.00952445263649142\\
99.55	0.00953490148424391\\
99.56	0.00954534693180085\\
99.57	0.00955578888861889\\
99.58	0.00956622726305078\\
99.59	0.00957666196232875\\
99.6	0.00958709289254763\\
99.61	0.0095975199586476\\
99.62	0.00960794306439661\\
99.63	0.00961836211237247\\
99.64	0.00962877700394458\\
99.65	0.00963918763925525\\
99.66	0.00964959391720069\\
99.67	0.00965999573541164\\
99.68	0.00967039299023348\\
99.69	0.0096807855767061\\
99.7	0.00969117338854325\\
99.71	0.00970155631812073\\
99.72	0.0097119342564558\\
99.73	0.00972230709318531\\
99.74	0.00973267471654341\\
99.75	0.00974303701333878\\
99.76	0.00975339386893135\\
99.77	0.00976374516720854\\
99.78	0.00977409079056095\\
99.79	0.00978443061985756\\
99.8	0.0097947645344203\\
99.81	0.00980509241199814\\
99.82	0.00981541412874054\\
99.83	0.00982572955917033\\
99.84	0.00983603857615593\\
99.85	0.00984634105088294\\
99.86	0.00985663685282512\\
99.87	0.00986692584971462\\
99.88	0.00987720790751152\\
99.89	0.0098874828903727\\
99.9	0.00989775066061986\\
99.91	0.00990801107870688\\
99.92	0.00991826400318625\\
99.93	0.00992850929067483\\
99.94	0.00993874679581858\\
99.95	0.00994897637125655\\
99.96	0.00995919786758387\\
99.97	0.0099694111333138\\
99.98	0.00997961601483888\\
99.99	0.0099898123563909\\
100	0.01\\
};
\addlegendentry{$q=0$};

\addplot [color=blue,solid,forget plot]
  table[row sep=crcr]{%
0.01	0\\
0.02	0\\
0.03	0\\
0.04	0\\
0.05	0\\
0.06	0\\
0.07	0\\
0.08	0\\
0.09	0\\
0.1	0\\
0.11	0\\
0.12	0\\
0.13	0\\
0.14	0\\
0.15	0\\
0.16	0\\
0.17	0\\
0.18	0\\
0.19	0\\
0.2	0\\
0.21	0\\
0.22	0\\
0.23	0\\
0.24	0\\
0.25	0\\
0.26	0\\
0.27	0\\
0.28	0\\
0.29	0\\
0.3	0\\
0.31	0\\
0.32	0\\
0.33	0\\
0.34	0\\
0.35	0\\
0.36	0\\
0.37	0\\
0.38	0\\
0.39	0\\
0.4	0\\
0.41	0\\
0.42	0\\
0.43	0\\
0.44	0\\
0.45	0\\
0.46	0\\
0.47	0\\
0.48	0\\
0.49	0\\
0.5	0\\
0.51	0\\
0.52	0\\
0.53	0\\
0.54	0\\
0.55	0\\
0.56	0\\
0.57	0\\
0.58	0\\
0.59	0\\
0.6	0\\
0.61	0\\
0.62	0\\
0.63	0\\
0.64	0\\
0.65	0\\
0.66	0\\
0.67	0\\
0.68	0\\
0.69	0\\
0.7	0\\
0.71	0\\
0.72	0\\
0.73	0\\
0.74	0\\
0.75	0\\
0.76	0\\
0.77	0\\
0.78	0\\
0.79	0\\
0.8	0\\
0.81	0\\
0.82	0\\
0.83	0\\
0.84	0\\
0.85	0\\
0.86	0\\
0.87	0\\
0.88	0\\
0.89	0\\
0.9	0\\
0.91	0\\
0.92	0\\
0.93	0\\
0.94	0\\
0.95	0\\
0.96	0\\
0.97	0\\
0.98	0\\
0.99	0\\
1	0\\
1.01	0\\
1.02	0\\
1.03	0\\
1.04	0\\
1.05	0\\
1.06	0\\
1.07	0\\
1.08	0\\
1.09	0\\
1.1	0\\
1.11	0\\
1.12	0\\
1.13	0\\
1.14	0\\
1.15	0\\
1.16	0\\
1.17	0\\
1.18	0\\
1.19	0\\
1.2	0\\
1.21	0\\
1.22	0\\
1.23	0\\
1.24	0\\
1.25	0\\
1.26	0\\
1.27	0\\
1.28	0\\
1.29	0\\
1.3	0\\
1.31	0\\
1.32	0\\
1.33	0\\
1.34	0\\
1.35	0\\
1.36	0\\
1.37	0\\
1.38	0\\
1.39	0\\
1.4	0\\
1.41	0\\
1.42	0\\
1.43	0\\
1.44	0\\
1.45	0\\
1.46	0\\
1.47	0\\
1.48	0\\
1.49	0\\
1.5	0\\
1.51	0\\
1.52	0\\
1.53	0\\
1.54	0\\
1.55	0\\
1.56	0\\
1.57	0\\
1.58	0\\
1.59	0\\
1.6	0\\
1.61	0\\
1.62	0\\
1.63	0\\
1.64	0\\
1.65	0\\
1.66	0\\
1.67	0\\
1.68	0\\
1.69	0\\
1.7	0\\
1.71	0\\
1.72	0\\
1.73	0\\
1.74	0\\
1.75	0\\
1.76	0\\
1.77	0\\
1.78	0\\
1.79	0\\
1.8	0\\
1.81	0\\
1.82	0\\
1.83	0\\
1.84	0\\
1.85	0\\
1.86	0\\
1.87	0\\
1.88	0\\
1.89	0\\
1.9	0\\
1.91	0\\
1.92	0\\
1.93	0\\
1.94	0\\
1.95	0\\
1.96	0\\
1.97	0\\
1.98	0\\
1.99	0\\
2	0\\
2.01	0\\
2.02	0\\
2.03	0\\
2.04	0\\
2.05	0\\
2.06	0\\
2.07	0\\
2.08	0\\
2.09	0\\
2.1	0\\
2.11	0\\
2.12	0\\
2.13	0\\
2.14	0\\
2.15	0\\
2.16	0\\
2.17	0\\
2.18	0\\
2.19	0\\
2.2	0\\
2.21	0\\
2.22	0\\
2.23	0\\
2.24	0\\
2.25	0\\
2.26	0\\
2.27	0\\
2.28	0\\
2.29	0\\
2.3	0\\
2.31	0\\
2.32	0\\
2.33	0\\
2.34	0\\
2.35	0\\
2.36	0\\
2.37	0\\
2.38	0\\
2.39	0\\
2.4	0\\
2.41	0\\
2.42	0\\
2.43	0\\
2.44	0\\
2.45	0\\
2.46	0\\
2.47	0\\
2.48	0\\
2.49	0\\
2.5	0\\
2.51	0\\
2.52	0\\
2.53	0\\
2.54	0\\
2.55	0\\
2.56	0\\
2.57	0\\
2.58	0\\
2.59	0\\
2.6	0\\
2.61	0\\
2.62	0\\
2.63	0\\
2.64	0\\
2.65	0\\
2.66	0\\
2.67	0\\
2.68	0\\
2.69	0\\
2.7	0\\
2.71	0\\
2.72	0\\
2.73	0\\
2.74	0\\
2.75	0\\
2.76	0\\
2.77	0\\
2.78	0\\
2.79	0\\
2.8	0\\
2.81	0\\
2.82	0\\
2.83	0\\
2.84	0\\
2.85	0\\
2.86	0\\
2.87	0\\
2.88	0\\
2.89	0\\
2.9	0\\
2.91	0\\
2.92	0\\
2.93	0\\
2.94	0\\
2.95	0\\
2.96	0\\
2.97	0\\
2.98	0\\
2.99	0\\
3	0\\
3.01	0\\
3.02	0\\
3.03	0\\
3.04	0\\
3.05	0\\
3.06	0\\
3.07	0\\
3.08	0\\
3.09	0\\
3.1	0\\
3.11	0\\
3.12	0\\
3.13	0\\
3.14	0\\
3.15	0\\
3.16	0\\
3.17	0\\
3.18	0\\
3.19	0\\
3.2	0\\
3.21	0\\
3.22	0\\
3.23	0\\
3.24	0\\
3.25	0\\
3.26	0\\
3.27	0\\
3.28	0\\
3.29	0\\
3.3	0\\
3.31	0\\
3.32	0\\
3.33	0\\
3.34	0\\
3.35	0\\
3.36	0\\
3.37	0\\
3.38	0\\
3.39	0\\
3.4	0\\
3.41	0\\
3.42	0\\
3.43	0\\
3.44	0\\
3.45	0\\
3.46	0\\
3.47	0\\
3.48	0\\
3.49	0\\
3.5	0\\
3.51	0\\
3.52	0\\
3.53	0\\
3.54	0\\
3.55	0\\
3.56	0\\
3.57	0\\
3.58	0\\
3.59	0\\
3.6	0\\
3.61	0\\
3.62	0\\
3.63	0\\
3.64	0\\
3.65	0\\
3.66	0\\
3.67	0\\
3.68	0\\
3.69	0\\
3.7	0\\
3.71	0\\
3.72	0\\
3.73	0\\
3.74	0\\
3.75	0\\
3.76	0\\
3.77	0\\
3.78	0\\
3.79	0\\
3.8	0\\
3.81	0\\
3.82	0\\
3.83	0\\
3.84	0\\
3.85	0\\
3.86	0\\
3.87	0\\
3.88	0\\
3.89	0\\
3.9	0\\
3.91	0\\
3.92	0\\
3.93	0\\
3.94	0\\
3.95	0\\
3.96	0\\
3.97	0\\
3.98	0\\
3.99	0\\
4	0\\
4.01	0\\
4.02	0\\
4.03	0\\
4.04	0\\
4.05	0\\
4.06	0\\
4.07	0\\
4.08	0\\
4.09	0\\
4.1	0\\
4.11	0\\
4.12	0\\
4.13	0\\
4.14	0\\
4.15	0\\
4.16	0\\
4.17	0\\
4.18	0\\
4.19	0\\
4.2	0\\
4.21	0\\
4.22	0\\
4.23	0\\
4.24	0\\
4.25	0\\
4.26	0\\
4.27	0\\
4.28	0\\
4.29	0\\
4.3	0\\
4.31	0\\
4.32	0\\
4.33	0\\
4.34	0\\
4.35	0\\
4.36	0\\
4.37	0\\
4.38	0\\
4.39	0\\
4.4	0\\
4.41	0\\
4.42	0\\
4.43	0\\
4.44	0\\
4.45	0\\
4.46	0\\
4.47	0\\
4.48	0\\
4.49	0\\
4.5	0\\
4.51	0\\
4.52	0\\
4.53	0\\
4.54	0\\
4.55	0\\
4.56	0\\
4.57	0\\
4.58	0\\
4.59	0\\
4.6	0\\
4.61	0\\
4.62	0\\
4.63	0\\
4.64	0\\
4.65	0\\
4.66	0\\
4.67	0\\
4.68	0\\
4.69	0\\
4.7	0\\
4.71	0\\
4.72	0\\
4.73	0\\
4.74	0\\
4.75	0\\
4.76	0\\
4.77	0\\
4.78	0\\
4.79	0\\
4.8	0\\
4.81	0\\
4.82	0\\
4.83	0\\
4.84	0\\
4.85	0\\
4.86	0\\
4.87	0\\
4.88	0\\
4.89	0\\
4.9	0\\
4.91	0\\
4.92	0\\
4.93	0\\
4.94	0\\
4.95	0\\
4.96	0\\
4.97	0\\
4.98	0\\
4.99	0\\
5	0\\
5.01	0\\
5.02	0\\
5.03	0\\
5.04	0\\
5.05	0\\
5.06	0\\
5.07	0\\
5.08	0\\
5.09	0\\
5.1	0\\
5.11	0\\
5.12	0\\
5.13	0\\
5.14	0\\
5.15	0\\
5.16	0\\
5.17	0\\
5.18	0\\
5.19	0\\
5.2	0\\
5.21	0\\
5.22	0\\
5.23	0\\
5.24	0\\
5.25	0\\
5.26	0\\
5.27	0\\
5.28	0\\
5.29	0\\
5.3	0\\
5.31	0\\
5.32	0\\
5.33	0\\
5.34	0\\
5.35	0\\
5.36	0\\
5.37	0\\
5.38	0\\
5.39	0\\
5.4	0\\
5.41	0\\
5.42	0\\
5.43	0\\
5.44	0\\
5.45	0\\
5.46	0\\
5.47	0\\
5.48	0\\
5.49	0\\
5.5	0\\
5.51	0\\
5.52	0\\
5.53	0\\
5.54	0\\
5.55	0\\
5.56	0\\
5.57	0\\
5.58	0\\
5.59	0\\
5.6	0\\
5.61	0\\
5.62	0\\
5.63	0\\
5.64	0\\
5.65	0\\
5.66	0\\
5.67	0\\
5.68	0\\
5.69	0\\
5.7	0\\
5.71	0\\
5.72	0\\
5.73	0\\
5.74	0\\
5.75	0\\
5.76	0\\
5.77	0\\
5.78	0\\
5.79	0\\
5.8	0\\
5.81	0\\
5.82	0\\
5.83	0\\
5.84	0\\
5.85	0\\
5.86	0\\
5.87	0\\
5.88	0\\
5.89	0\\
5.9	0\\
5.91	0\\
5.92	0\\
5.93	0\\
5.94	0\\
5.95	0\\
5.96	0\\
5.97	0\\
5.98	0\\
5.99	0\\
6	0\\
6.01	0\\
6.02	0\\
6.03	0\\
6.04	0\\
6.05	0\\
6.06	0\\
6.07	0\\
6.08	0\\
6.09	0\\
6.1	0\\
6.11	0\\
6.12	0\\
6.13	0\\
6.14	0\\
6.15	0\\
6.16	0\\
6.17	0\\
6.18	0\\
6.19	0\\
6.2	0\\
6.21	0\\
6.22	0\\
6.23	0\\
6.24	0\\
6.25	0\\
6.26	0\\
6.27	0\\
6.28	0\\
6.29	0\\
6.3	0\\
6.31	0\\
6.32	0\\
6.33	0\\
6.34	0\\
6.35	0\\
6.36	0\\
6.37	0\\
6.38	0\\
6.39	0\\
6.4	0\\
6.41	0\\
6.42	0\\
6.43	0\\
6.44	0\\
6.45	0\\
6.46	0\\
6.47	0\\
6.48	0\\
6.49	0\\
6.5	0\\
6.51	0\\
6.52	0\\
6.53	0\\
6.54	0\\
6.55	0\\
6.56	0\\
6.57	0\\
6.58	0\\
6.59	0\\
6.6	0\\
6.61	0\\
6.62	0\\
6.63	0\\
6.64	0\\
6.65	0\\
6.66	0\\
6.67	0\\
6.68	0\\
6.69	0\\
6.7	0\\
6.71	0\\
6.72	0\\
6.73	0\\
6.74	0\\
6.75	0\\
6.76	0\\
6.77	0\\
6.78	0\\
6.79	0\\
6.8	0\\
6.81	0\\
6.82	0\\
6.83	0\\
6.84	0\\
6.85	0\\
6.86	0\\
6.87	0\\
6.88	0\\
6.89	0\\
6.9	0\\
6.91	0\\
6.92	0\\
6.93	0\\
6.94	0\\
6.95	0\\
6.96	0\\
6.97	0\\
6.98	0\\
6.99	0\\
7	0\\
7.01	0\\
7.02	0\\
7.03	0\\
7.04	0\\
7.05	0\\
7.06	0\\
7.07	0\\
7.08	0\\
7.09	0\\
7.1	0\\
7.11	0\\
7.12	0\\
7.13	0\\
7.14	0\\
7.15	0\\
7.16	0\\
7.17	0\\
7.18	0\\
7.19	0\\
7.2	0\\
7.21	0\\
7.22	0\\
7.23	0\\
7.24	0\\
7.25	0\\
7.26	0\\
7.27	0\\
7.28	0\\
7.29	0\\
7.3	0\\
7.31	0\\
7.32	0\\
7.33	0\\
7.34	0\\
7.35	0\\
7.36	0\\
7.37	0\\
7.38	0\\
7.39	0\\
7.4	0\\
7.41	0\\
7.42	0\\
7.43	0\\
7.44	0\\
7.45	0\\
7.46	0\\
7.47	0\\
7.48	0\\
7.49	0\\
7.5	0\\
7.51	0\\
7.52	0\\
7.53	0\\
7.54	0\\
7.55	0\\
7.56	0\\
7.57	0\\
7.58	0\\
7.59	0\\
7.6	0\\
7.61	0\\
7.62	0\\
7.63	0\\
7.64	0\\
7.65	0\\
7.66	0\\
7.67	0\\
7.68	0\\
7.69	0\\
7.7	0\\
7.71	0\\
7.72	0\\
7.73	0\\
7.74	0\\
7.75	0\\
7.76	0\\
7.77	0\\
7.78	0\\
7.79	0\\
7.8	0\\
7.81	0\\
7.82	0\\
7.83	0\\
7.84	0\\
7.85	0\\
7.86	0\\
7.87	0\\
7.88	0\\
7.89	0\\
7.9	0\\
7.91	0\\
7.92	0\\
7.93	0\\
7.94	0\\
7.95	0\\
7.96	0\\
7.97	0\\
7.98	0\\
7.99	0\\
8	0\\
8.01	0\\
8.02	0\\
8.03	0\\
8.04	0\\
8.05	0\\
8.06	0\\
8.07	0\\
8.08	0\\
8.09	0\\
8.1	0\\
8.11	0\\
8.12	0\\
8.13	0\\
8.14	0\\
8.15	0\\
8.16	0\\
8.17	0\\
8.18	0\\
8.19	0\\
8.2	0\\
8.21	0\\
8.22	0\\
8.23	0\\
8.24	0\\
8.25	0\\
8.26	0\\
8.27	0\\
8.28	0\\
8.29	0\\
8.3	0\\
8.31	0\\
8.32	0\\
8.33	0\\
8.34	0\\
8.35	0\\
8.36	0\\
8.37	0\\
8.38	0\\
8.39	0\\
8.4	0\\
8.41	0\\
8.42	0\\
8.43	0\\
8.44	0\\
8.45	0\\
8.46	0\\
8.47	0\\
8.48	0\\
8.49	0\\
8.5	0\\
8.51	0\\
8.52	0\\
8.53	0\\
8.54	0\\
8.55	0\\
8.56	0\\
8.57	0\\
8.58	0\\
8.59	0\\
8.6	0\\
8.61	0\\
8.62	0\\
8.63	0\\
8.64	0\\
8.65	0\\
8.66	0\\
8.67	0\\
8.68	0\\
8.69	0\\
8.7	0\\
8.71	0\\
8.72	0\\
8.73	0\\
8.74	0\\
8.75	0\\
8.76	0\\
8.77	0\\
8.78	0\\
8.79	0\\
8.8	0\\
8.81	0\\
8.82	0\\
8.83	0\\
8.84	0\\
8.85	0\\
8.86	0\\
8.87	0\\
8.88	0\\
8.89	0\\
8.9	0\\
8.91	0\\
8.92	0\\
8.93	0\\
8.94	0\\
8.95	0\\
8.96	0\\
8.97	0\\
8.98	0\\
8.99	0\\
9	0\\
9.01	0\\
9.02	0\\
9.03	0\\
9.04	0\\
9.05	0\\
9.06	0\\
9.07	0\\
9.08	0\\
9.09	0\\
9.1	0\\
9.11	0\\
9.12	0\\
9.13	0\\
9.14	0\\
9.15	0\\
9.16	0\\
9.17	0\\
9.18	0\\
9.19	0\\
9.2	0\\
9.21	0\\
9.22	0\\
9.23	0\\
9.24	0\\
9.25	0\\
9.26	0\\
9.27	0\\
9.28	0\\
9.29	0\\
9.3	0\\
9.31	0\\
9.32	0\\
9.33	0\\
9.34	0\\
9.35	0\\
9.36	0\\
9.37	0\\
9.38	0\\
9.39	0\\
9.4	0\\
9.41	0\\
9.42	0\\
9.43	0\\
9.44	0\\
9.45	0\\
9.46	0\\
9.47	0\\
9.48	0\\
9.49	0\\
9.5	0\\
9.51	0\\
9.52	0\\
9.53	0\\
9.54	0\\
9.55	0\\
9.56	0\\
9.57	0\\
9.58	0\\
9.59	0\\
9.6	0\\
9.61	0\\
9.62	0\\
9.63	0\\
9.64	0\\
9.65	0\\
9.66	0\\
9.67	0\\
9.68	0\\
9.69	0\\
9.7	0\\
9.71	0\\
9.72	0\\
9.73	0\\
9.74	0\\
9.75	0\\
9.76	0\\
9.77	0\\
9.78	0\\
9.79	0\\
9.8	0\\
9.81	0\\
9.82	0\\
9.83	0\\
9.84	0\\
9.85	0\\
9.86	0\\
9.87	0\\
9.88	0\\
9.89	0\\
9.9	0\\
9.91	0\\
9.92	0\\
9.93	0\\
9.94	0\\
9.95	0\\
9.96	0\\
9.97	0\\
9.98	0\\
9.99	0\\
10	0\\
10.01	0\\
10.02	0\\
10.03	0\\
10.04	0\\
10.05	0\\
10.06	0\\
10.07	0\\
10.08	0\\
10.09	0\\
10.1	0\\
10.11	0\\
10.12	0\\
10.13	0\\
10.14	0\\
10.15	0\\
10.16	0\\
10.17	0\\
10.18	0\\
10.19	0\\
10.2	0\\
10.21	0\\
10.22	0\\
10.23	0\\
10.24	0\\
10.25	0\\
10.26	0\\
10.27	0\\
10.28	0\\
10.29	0\\
10.3	0\\
10.31	0\\
10.32	0\\
10.33	0\\
10.34	0\\
10.35	0\\
10.36	0\\
10.37	0\\
10.38	0\\
10.39	0\\
10.4	0\\
10.41	0\\
10.42	0\\
10.43	0\\
10.44	0\\
10.45	0\\
10.46	0\\
10.47	0\\
10.48	0\\
10.49	0\\
10.5	0\\
10.51	0\\
10.52	0\\
10.53	0\\
10.54	0\\
10.55	0\\
10.56	0\\
10.57	0\\
10.58	0\\
10.59	0\\
10.6	0\\
10.61	0\\
10.62	0\\
10.63	0\\
10.64	0\\
10.65	0\\
10.66	0\\
10.67	0\\
10.68	0\\
10.69	0\\
10.7	0\\
10.71	0\\
10.72	0\\
10.73	0\\
10.74	0\\
10.75	0\\
10.76	0\\
10.77	0\\
10.78	0\\
10.79	0\\
10.8	0\\
10.81	0\\
10.82	0\\
10.83	0\\
10.84	0\\
10.85	0\\
10.86	0\\
10.87	0\\
10.88	0\\
10.89	0\\
10.9	0\\
10.91	0\\
10.92	0\\
10.93	0\\
10.94	0\\
10.95	0\\
10.96	0\\
10.97	0\\
10.98	0\\
10.99	0\\
11	0\\
11.01	0\\
11.02	0\\
11.03	0\\
11.04	0\\
11.05	0\\
11.06	0\\
11.07	0\\
11.08	0\\
11.09	0\\
11.1	0\\
11.11	0\\
11.12	0\\
11.13	0\\
11.14	0\\
11.15	0\\
11.16	0\\
11.17	0\\
11.18	0\\
11.19	0\\
11.2	0\\
11.21	0\\
11.22	0\\
11.23	0\\
11.24	0\\
11.25	0\\
11.26	0\\
11.27	0\\
11.28	0\\
11.29	0\\
11.3	0\\
11.31	0\\
11.32	0\\
11.33	0\\
11.34	0\\
11.35	0\\
11.36	0\\
11.37	0\\
11.38	0\\
11.39	0\\
11.4	0\\
11.41	0\\
11.42	0\\
11.43	0\\
11.44	0\\
11.45	0\\
11.46	0\\
11.47	0\\
11.48	0\\
11.49	0\\
11.5	0\\
11.51	0\\
11.52	0\\
11.53	0\\
11.54	0\\
11.55	0\\
11.56	0\\
11.57	0\\
11.58	0\\
11.59	0\\
11.6	0\\
11.61	0\\
11.62	0\\
11.63	0\\
11.64	0\\
11.65	0\\
11.66	0\\
11.67	0\\
11.68	0\\
11.69	0\\
11.7	0\\
11.71	0\\
11.72	0\\
11.73	0\\
11.74	0\\
11.75	0\\
11.76	0\\
11.77	0\\
11.78	0\\
11.79	0\\
11.8	0\\
11.81	0\\
11.82	0\\
11.83	0\\
11.84	0\\
11.85	0\\
11.86	0\\
11.87	0\\
11.88	0\\
11.89	0\\
11.9	0\\
11.91	0\\
11.92	0\\
11.93	0\\
11.94	0\\
11.95	0\\
11.96	0\\
11.97	0\\
11.98	0\\
11.99	0\\
12	0\\
12.01	0\\
12.02	0\\
12.03	0\\
12.04	0\\
12.05	0\\
12.06	0\\
12.07	0\\
12.08	0\\
12.09	0\\
12.1	0\\
12.11	0\\
12.12	0\\
12.13	0\\
12.14	0\\
12.15	0\\
12.16	0\\
12.17	0\\
12.18	0\\
12.19	0\\
12.2	0\\
12.21	0\\
12.22	0\\
12.23	0\\
12.24	0\\
12.25	0\\
12.26	0\\
12.27	0\\
12.28	0\\
12.29	0\\
12.3	0\\
12.31	0\\
12.32	0\\
12.33	0\\
12.34	0\\
12.35	0\\
12.36	0\\
12.37	0\\
12.38	0\\
12.39	0\\
12.4	0\\
12.41	0\\
12.42	0\\
12.43	0\\
12.44	0\\
12.45	0\\
12.46	0\\
12.47	0\\
12.48	0\\
12.49	0\\
12.5	0\\
12.51	0\\
12.52	0\\
12.53	0\\
12.54	0\\
12.55	0\\
12.56	0\\
12.57	0\\
12.58	0\\
12.59	0\\
12.6	0\\
12.61	0\\
12.62	0\\
12.63	0\\
12.64	0\\
12.65	0\\
12.66	0\\
12.67	0\\
12.68	0\\
12.69	0\\
12.7	0\\
12.71	0\\
12.72	0\\
12.73	0\\
12.74	0\\
12.75	0\\
12.76	0\\
12.77	0\\
12.78	0\\
12.79	0\\
12.8	0\\
12.81	0\\
12.82	0\\
12.83	0\\
12.84	0\\
12.85	0\\
12.86	0\\
12.87	0\\
12.88	0\\
12.89	0\\
12.9	0\\
12.91	0\\
12.92	0\\
12.93	0\\
12.94	0\\
12.95	0\\
12.96	0\\
12.97	0\\
12.98	0\\
12.99	0\\
13	0\\
13.01	0\\
13.02	0\\
13.03	0\\
13.04	0\\
13.05	0\\
13.06	0\\
13.07	0\\
13.08	0\\
13.09	0\\
13.1	0\\
13.11	0\\
13.12	0\\
13.13	0\\
13.14	0\\
13.15	0\\
13.16	0\\
13.17	0\\
13.18	0\\
13.19	0\\
13.2	0\\
13.21	0\\
13.22	0\\
13.23	0\\
13.24	0\\
13.25	0\\
13.26	0\\
13.27	0\\
13.28	0\\
13.29	0\\
13.3	0\\
13.31	0\\
13.32	0\\
13.33	0\\
13.34	0\\
13.35	0\\
13.36	0\\
13.37	0\\
13.38	0\\
13.39	0\\
13.4	0\\
13.41	0\\
13.42	0\\
13.43	0\\
13.44	0\\
13.45	0\\
13.46	0\\
13.47	0\\
13.48	0\\
13.49	0\\
13.5	0\\
13.51	0\\
13.52	0\\
13.53	0\\
13.54	0\\
13.55	0\\
13.56	0\\
13.57	0\\
13.58	0\\
13.59	0\\
13.6	0\\
13.61	0\\
13.62	0\\
13.63	0\\
13.64	0\\
13.65	0\\
13.66	0\\
13.67	0\\
13.68	0\\
13.69	0\\
13.7	0\\
13.71	0\\
13.72	0\\
13.73	0\\
13.74	0\\
13.75	0\\
13.76	0\\
13.77	0\\
13.78	0\\
13.79	0\\
13.8	0\\
13.81	0\\
13.82	0\\
13.83	0\\
13.84	0\\
13.85	0\\
13.86	0\\
13.87	0\\
13.88	0\\
13.89	0\\
13.9	0\\
13.91	0\\
13.92	0\\
13.93	0\\
13.94	0\\
13.95	0\\
13.96	0\\
13.97	0\\
13.98	0\\
13.99	0\\
14	0\\
14.01	0\\
14.02	0\\
14.03	0\\
14.04	0\\
14.05	0\\
14.06	0\\
14.07	0\\
14.08	0\\
14.09	0\\
14.1	0\\
14.11	0\\
14.12	0\\
14.13	0\\
14.14	0\\
14.15	0\\
14.16	0\\
14.17	0\\
14.18	0\\
14.19	0\\
14.2	0\\
14.21	0\\
14.22	0\\
14.23	0\\
14.24	0\\
14.25	0\\
14.26	0\\
14.27	0\\
14.28	0\\
14.29	0\\
14.3	0\\
14.31	0\\
14.32	0\\
14.33	0\\
14.34	0\\
14.35	0\\
14.36	0\\
14.37	0\\
14.38	0\\
14.39	0\\
14.4	0\\
14.41	0\\
14.42	0\\
14.43	0\\
14.44	0\\
14.45	0\\
14.46	0\\
14.47	0\\
14.48	0\\
14.49	0\\
14.5	0\\
14.51	0\\
14.52	0\\
14.53	0\\
14.54	0\\
14.55	0\\
14.56	0\\
14.57	0\\
14.58	0\\
14.59	0\\
14.6	0\\
14.61	0\\
14.62	0\\
14.63	0\\
14.64	0\\
14.65	0\\
14.66	0\\
14.67	0\\
14.68	0\\
14.69	0\\
14.7	0\\
14.71	0\\
14.72	0\\
14.73	0\\
14.74	0\\
14.75	0\\
14.76	0\\
14.77	0\\
14.78	0\\
14.79	0\\
14.8	0\\
14.81	0\\
14.82	0\\
14.83	0\\
14.84	0\\
14.85	0\\
14.86	0\\
14.87	0\\
14.88	0\\
14.89	0\\
14.9	0\\
14.91	0\\
14.92	0\\
14.93	0\\
14.94	0\\
14.95	0\\
14.96	0\\
14.97	0\\
14.98	0\\
14.99	0\\
15	0\\
15.01	0\\
15.02	0\\
15.03	0\\
15.04	0\\
15.05	0\\
15.06	0\\
15.07	0\\
15.08	0\\
15.09	0\\
15.1	0\\
15.11	0\\
15.12	0\\
15.13	0\\
15.14	0\\
15.15	0\\
15.16	0\\
15.17	0\\
15.18	0\\
15.19	0\\
15.2	0\\
15.21	0\\
15.22	0\\
15.23	0\\
15.24	0\\
15.25	0\\
15.26	0\\
15.27	0\\
15.28	0\\
15.29	0\\
15.3	0\\
15.31	0\\
15.32	0\\
15.33	0\\
15.34	0\\
15.35	0\\
15.36	0\\
15.37	0\\
15.38	0\\
15.39	0\\
15.4	0\\
15.41	0\\
15.42	0\\
15.43	0\\
15.44	0\\
15.45	0\\
15.46	0\\
15.47	0\\
15.48	0\\
15.49	0\\
15.5	0\\
15.51	0\\
15.52	0\\
15.53	0\\
15.54	0\\
15.55	0\\
15.56	0\\
15.57	0\\
15.58	0\\
15.59	0\\
15.6	0\\
15.61	0\\
15.62	0\\
15.63	0\\
15.64	0\\
15.65	0\\
15.66	0\\
15.67	0\\
15.68	0\\
15.69	0\\
15.7	0\\
15.71	0\\
15.72	0\\
15.73	0\\
15.74	0\\
15.75	0\\
15.76	0\\
15.77	0\\
15.78	0\\
15.79	0\\
15.8	0\\
15.81	0\\
15.82	0\\
15.83	0\\
15.84	0\\
15.85	0\\
15.86	0\\
15.87	0\\
15.88	0\\
15.89	0\\
15.9	0\\
15.91	0\\
15.92	0\\
15.93	0\\
15.94	0\\
15.95	0\\
15.96	0\\
15.97	0\\
15.98	0\\
15.99	0\\
16	0\\
16.01	0\\
16.02	0\\
16.03	0\\
16.04	0\\
16.05	0\\
16.06	0\\
16.07	0\\
16.08	0\\
16.09	0\\
16.1	0\\
16.11	0\\
16.12	0\\
16.13	0\\
16.14	0\\
16.15	0\\
16.16	0\\
16.17	0\\
16.18	0\\
16.19	0\\
16.2	0\\
16.21	0\\
16.22	0\\
16.23	0\\
16.24	0\\
16.25	0\\
16.26	0\\
16.27	0\\
16.28	0\\
16.29	0\\
16.3	0\\
16.31	0\\
16.32	0\\
16.33	0\\
16.34	0\\
16.35	0\\
16.36	0\\
16.37	0\\
16.38	0\\
16.39	0\\
16.4	0\\
16.41	0\\
16.42	0\\
16.43	0\\
16.44	0\\
16.45	0\\
16.46	0\\
16.47	0\\
16.48	0\\
16.49	0\\
16.5	0\\
16.51	0\\
16.52	0\\
16.53	0\\
16.54	0\\
16.55	0\\
16.56	0\\
16.57	0\\
16.58	0\\
16.59	0\\
16.6	0\\
16.61	0\\
16.62	0\\
16.63	0\\
16.64	0\\
16.65	0\\
16.66	0\\
16.67	0\\
16.68	0\\
16.69	0\\
16.7	0\\
16.71	0\\
16.72	0\\
16.73	0\\
16.74	0\\
16.75	0\\
16.76	0\\
16.77	0\\
16.78	0\\
16.79	0\\
16.8	0\\
16.81	0\\
16.82	0\\
16.83	0\\
16.84	0\\
16.85	0\\
16.86	0\\
16.87	0\\
16.88	0\\
16.89	0\\
16.9	0\\
16.91	0\\
16.92	0\\
16.93	0\\
16.94	0\\
16.95	0\\
16.96	0\\
16.97	0\\
16.98	0\\
16.99	0\\
17	0\\
17.01	0\\
17.02	0\\
17.03	0\\
17.04	0\\
17.05	0\\
17.06	0\\
17.07	0\\
17.08	0\\
17.09	0\\
17.1	0\\
17.11	0\\
17.12	0\\
17.13	0\\
17.14	0\\
17.15	0\\
17.16	0\\
17.17	0\\
17.18	0\\
17.19	0\\
17.2	0\\
17.21	0\\
17.22	0\\
17.23	0\\
17.24	0\\
17.25	0\\
17.26	0\\
17.27	0\\
17.28	0\\
17.29	0\\
17.3	0\\
17.31	0\\
17.32	0\\
17.33	0\\
17.34	0\\
17.35	0\\
17.36	0\\
17.37	0\\
17.38	0\\
17.39	0\\
17.4	0\\
17.41	0\\
17.42	0\\
17.43	0\\
17.44	0\\
17.45	0\\
17.46	0\\
17.47	0\\
17.48	0\\
17.49	0\\
17.5	0\\
17.51	0\\
17.52	0\\
17.53	0\\
17.54	0\\
17.55	0\\
17.56	0\\
17.57	0\\
17.58	0\\
17.59	0\\
17.6	0\\
17.61	0\\
17.62	0\\
17.63	0\\
17.64	0\\
17.65	0\\
17.66	0\\
17.67	0\\
17.68	0\\
17.69	0\\
17.7	0\\
17.71	0\\
17.72	0\\
17.73	0\\
17.74	0\\
17.75	0\\
17.76	0\\
17.77	0\\
17.78	0\\
17.79	0\\
17.8	0\\
17.81	0\\
17.82	0\\
17.83	0\\
17.84	0\\
17.85	0\\
17.86	0\\
17.87	0\\
17.88	0\\
17.89	0\\
17.9	0\\
17.91	0\\
17.92	0\\
17.93	0\\
17.94	0\\
17.95	0\\
17.96	0\\
17.97	0\\
17.98	0\\
17.99	0\\
18	0\\
18.01	0\\
18.02	0\\
18.03	0\\
18.04	0\\
18.05	0\\
18.06	0\\
18.07	0\\
18.08	0\\
18.09	0\\
18.1	0\\
18.11	0\\
18.12	0\\
18.13	0\\
18.14	0\\
18.15	0\\
18.16	0\\
18.17	0\\
18.18	0\\
18.19	0\\
18.2	0\\
18.21	0\\
18.22	0\\
18.23	0\\
18.24	0\\
18.25	0\\
18.26	0\\
18.27	0\\
18.28	0\\
18.29	0\\
18.3	0\\
18.31	0\\
18.32	0\\
18.33	0\\
18.34	0\\
18.35	0\\
18.36	0\\
18.37	0\\
18.38	0\\
18.39	0\\
18.4	0\\
18.41	0\\
18.42	0\\
18.43	0\\
18.44	0\\
18.45	0\\
18.46	0\\
18.47	0\\
18.48	0\\
18.49	0\\
18.5	0\\
18.51	0\\
18.52	0\\
18.53	0\\
18.54	0\\
18.55	0\\
18.56	0\\
18.57	0\\
18.58	0\\
18.59	0\\
18.6	0\\
18.61	0\\
18.62	0\\
18.63	0\\
18.64	0\\
18.65	0\\
18.66	0\\
18.67	0\\
18.68	0\\
18.69	0\\
18.7	0\\
18.71	0\\
18.72	0\\
18.73	0\\
18.74	0\\
18.75	0\\
18.76	0\\
18.77	0\\
18.78	0\\
18.79	0\\
18.8	0\\
18.81	0\\
18.82	0\\
18.83	0\\
18.84	0\\
18.85	0\\
18.86	0\\
18.87	0\\
18.88	0\\
18.89	0\\
18.9	0\\
18.91	0\\
18.92	0\\
18.93	0\\
18.94	0\\
18.95	0\\
18.96	0\\
18.97	0\\
18.98	0\\
18.99	0\\
19	0\\
19.01	0\\
19.02	0\\
19.03	0\\
19.04	0\\
19.05	0\\
19.06	0\\
19.07	0\\
19.08	0\\
19.09	0\\
19.1	0\\
19.11	0\\
19.12	0\\
19.13	0\\
19.14	0\\
19.15	0\\
19.16	0\\
19.17	0\\
19.18	0\\
19.19	0\\
19.2	0\\
19.21	0\\
19.22	0\\
19.23	0\\
19.24	0\\
19.25	0\\
19.26	0\\
19.27	0\\
19.28	0\\
19.29	0\\
19.3	0\\
19.31	0\\
19.32	0\\
19.33	0\\
19.34	0\\
19.35	0\\
19.36	0\\
19.37	0\\
19.38	0\\
19.39	0\\
19.4	0\\
19.41	0\\
19.42	0\\
19.43	0\\
19.44	0\\
19.45	0\\
19.46	0\\
19.47	0\\
19.48	0\\
19.49	0\\
19.5	0\\
19.51	0\\
19.52	0\\
19.53	0\\
19.54	0\\
19.55	0\\
19.56	0\\
19.57	0\\
19.58	0\\
19.59	0\\
19.6	0\\
19.61	0\\
19.62	0\\
19.63	0\\
19.64	0\\
19.65	0\\
19.66	0\\
19.67	0\\
19.68	0\\
19.69	0\\
19.7	0\\
19.71	0\\
19.72	0\\
19.73	0\\
19.74	0\\
19.75	0\\
19.76	0\\
19.77	0\\
19.78	0\\
19.79	0\\
19.8	0\\
19.81	0\\
19.82	0\\
19.83	0\\
19.84	0\\
19.85	0\\
19.86	0\\
19.87	0\\
19.88	0\\
19.89	0\\
19.9	0\\
19.91	0\\
19.92	0\\
19.93	0\\
19.94	0\\
19.95	0\\
19.96	0\\
19.97	0\\
19.98	0\\
19.99	0\\
20	0\\
20.01	0\\
20.02	0\\
20.03	0\\
20.04	0\\
20.05	0\\
20.06	0\\
20.07	0\\
20.08	0\\
20.09	0\\
20.1	0\\
20.11	0\\
20.12	0\\
20.13	0\\
20.14	0\\
20.15	0\\
20.16	0\\
20.17	0\\
20.18	0\\
20.19	0\\
20.2	0\\
20.21	0\\
20.22	0\\
20.23	0\\
20.24	0\\
20.25	0\\
20.26	0\\
20.27	0\\
20.28	0\\
20.29	0\\
20.3	0\\
20.31	0\\
20.32	0\\
20.33	0\\
20.34	0\\
20.35	0\\
20.36	0\\
20.37	0\\
20.38	0\\
20.39	0\\
20.4	0\\
20.41	0\\
20.42	0\\
20.43	0\\
20.44	0\\
20.45	0\\
20.46	0\\
20.47	0\\
20.48	0\\
20.49	0\\
20.5	0\\
20.51	0\\
20.52	0\\
20.53	0\\
20.54	0\\
20.55	0\\
20.56	0\\
20.57	0\\
20.58	0\\
20.59	0\\
20.6	0\\
20.61	0\\
20.62	0\\
20.63	0\\
20.64	0\\
20.65	0\\
20.66	0\\
20.67	0\\
20.68	0\\
20.69	0\\
20.7	0\\
20.71	0\\
20.72	0\\
20.73	0\\
20.74	0\\
20.75	0\\
20.76	0\\
20.77	0\\
20.78	0\\
20.79	0\\
20.8	0\\
20.81	0\\
20.82	0\\
20.83	0\\
20.84	0\\
20.85	0\\
20.86	0\\
20.87	0\\
20.88	0\\
20.89	0\\
20.9	0\\
20.91	0\\
20.92	0\\
20.93	0\\
20.94	0\\
20.95	0\\
20.96	0\\
20.97	0\\
20.98	0\\
20.99	0\\
21	0\\
21.01	0\\
21.02	0\\
21.03	0\\
21.04	0\\
21.05	0\\
21.06	0\\
21.07	0\\
21.08	0\\
21.09	0\\
21.1	0\\
21.11	0\\
21.12	0\\
21.13	0\\
21.14	0\\
21.15	0\\
21.16	0\\
21.17	0\\
21.18	0\\
21.19	0\\
21.2	0\\
21.21	0\\
21.22	0\\
21.23	0\\
21.24	0\\
21.25	0\\
21.26	0\\
21.27	0\\
21.28	0\\
21.29	0\\
21.3	0\\
21.31	0\\
21.32	0\\
21.33	0\\
21.34	0\\
21.35	0\\
21.36	0\\
21.37	0\\
21.38	0\\
21.39	0\\
21.4	0\\
21.41	0\\
21.42	0\\
21.43	0\\
21.44	0\\
21.45	0\\
21.46	0\\
21.47	0\\
21.48	0\\
21.49	0\\
21.5	0\\
21.51	0\\
21.52	0\\
21.53	0\\
21.54	0\\
21.55	0\\
21.56	0\\
21.57	0\\
21.58	0\\
21.59	0\\
21.6	0\\
21.61	0\\
21.62	0\\
21.63	0\\
21.64	0\\
21.65	0\\
21.66	0\\
21.67	0\\
21.68	0\\
21.69	0\\
21.7	0\\
21.71	0\\
21.72	0\\
21.73	0\\
21.74	0\\
21.75	0\\
21.76	0\\
21.77	0\\
21.78	0\\
21.79	0\\
21.8	0\\
21.81	0\\
21.82	0\\
21.83	0\\
21.84	0\\
21.85	0\\
21.86	0\\
21.87	0\\
21.88	0\\
21.89	0\\
21.9	0\\
21.91	0\\
21.92	0\\
21.93	0\\
21.94	0\\
21.95	0\\
21.96	0\\
21.97	0\\
21.98	0\\
21.99	0\\
22	0\\
22.01	0\\
22.02	0\\
22.03	0\\
22.04	0\\
22.05	0\\
22.06	0\\
22.07	0\\
22.08	0\\
22.09	0\\
22.1	0\\
22.11	0\\
22.12	0\\
22.13	0\\
22.14	0\\
22.15	0\\
22.16	0\\
22.17	0\\
22.18	0\\
22.19	0\\
22.2	0\\
22.21	0\\
22.22	0\\
22.23	0\\
22.24	0\\
22.25	0\\
22.26	0\\
22.27	0\\
22.28	0\\
22.29	0\\
22.3	0\\
22.31	0\\
22.32	0\\
22.33	0\\
22.34	0\\
22.35	0\\
22.36	0\\
22.37	0\\
22.38	0\\
22.39	0\\
22.4	0\\
22.41	0\\
22.42	0\\
22.43	0\\
22.44	0\\
22.45	0\\
22.46	0\\
22.47	0\\
22.48	0\\
22.49	0\\
22.5	0\\
22.51	0\\
22.52	0\\
22.53	0\\
22.54	0\\
22.55	0\\
22.56	0\\
22.57	0\\
22.58	0\\
22.59	0\\
22.6	0\\
22.61	0\\
22.62	0\\
22.63	0\\
22.64	0\\
22.65	0\\
22.66	0\\
22.67	0\\
22.68	0\\
22.69	0\\
22.7	0\\
22.71	0\\
22.72	0\\
22.73	0\\
22.74	0\\
22.75	0\\
22.76	0\\
22.77	0\\
22.78	0\\
22.79	0\\
22.8	0\\
22.81	0\\
22.82	0\\
22.83	0\\
22.84	0\\
22.85	0\\
22.86	0\\
22.87	0\\
22.88	0\\
22.89	0\\
22.9	0\\
22.91	0\\
22.92	0\\
22.93	0\\
22.94	0\\
22.95	0\\
22.96	0\\
22.97	0\\
22.98	0\\
22.99	0\\
23	0\\
23.01	0\\
23.02	0\\
23.03	0\\
23.04	0\\
23.05	0\\
23.06	0\\
23.07	0\\
23.08	0\\
23.09	0\\
23.1	0\\
23.11	0\\
23.12	0\\
23.13	0\\
23.14	0\\
23.15	0\\
23.16	0\\
23.17	0\\
23.18	0\\
23.19	0\\
23.2	0\\
23.21	0\\
23.22	0\\
23.23	0\\
23.24	0\\
23.25	0\\
23.26	0\\
23.27	0\\
23.28	0\\
23.29	0\\
23.3	0\\
23.31	0\\
23.32	0\\
23.33	0\\
23.34	0\\
23.35	0\\
23.36	0\\
23.37	0\\
23.38	0\\
23.39	0\\
23.4	0\\
23.41	0\\
23.42	0\\
23.43	0\\
23.44	0\\
23.45	0\\
23.46	0\\
23.47	0\\
23.48	0\\
23.49	0\\
23.5	0\\
23.51	0\\
23.52	0\\
23.53	0\\
23.54	0\\
23.55	0\\
23.56	0\\
23.57	0\\
23.58	0\\
23.59	0\\
23.6	0\\
23.61	0\\
23.62	0\\
23.63	0\\
23.64	0\\
23.65	0\\
23.66	0\\
23.67	0\\
23.68	0\\
23.69	0\\
23.7	0\\
23.71	0\\
23.72	0\\
23.73	0\\
23.74	0\\
23.75	0\\
23.76	0\\
23.77	0\\
23.78	0\\
23.79	0\\
23.8	0\\
23.81	0\\
23.82	0\\
23.83	0\\
23.84	0\\
23.85	0\\
23.86	0\\
23.87	0\\
23.88	0\\
23.89	0\\
23.9	0\\
23.91	0\\
23.92	0\\
23.93	0\\
23.94	0\\
23.95	0\\
23.96	0\\
23.97	0\\
23.98	0\\
23.99	0\\
24	0\\
24.01	0\\
24.02	0\\
24.03	0\\
24.04	0\\
24.05	0\\
24.06	0\\
24.07	0\\
24.08	0\\
24.09	0\\
24.1	0\\
24.11	0\\
24.12	0\\
24.13	0\\
24.14	0\\
24.15	0\\
24.16	0\\
24.17	0\\
24.18	0\\
24.19	0\\
24.2	0\\
24.21	0\\
24.22	0\\
24.23	0\\
24.24	0\\
24.25	0\\
24.26	0\\
24.27	0\\
24.28	0\\
24.29	0\\
24.3	0\\
24.31	0\\
24.32	0\\
24.33	0\\
24.34	0\\
24.35	0\\
24.36	0\\
24.37	0\\
24.38	0\\
24.39	0\\
24.4	0\\
24.41	0\\
24.42	0\\
24.43	0\\
24.44	0\\
24.45	0\\
24.46	0\\
24.47	0\\
24.48	0\\
24.49	0\\
24.5	0\\
24.51	0\\
24.52	0\\
24.53	0\\
24.54	0\\
24.55	0\\
24.56	0\\
24.57	0\\
24.58	0\\
24.59	0\\
24.6	0\\
24.61	0\\
24.62	0\\
24.63	0\\
24.64	0\\
24.65	0\\
24.66	0\\
24.67	0\\
24.68	0\\
24.69	0\\
24.7	0\\
24.71	0\\
24.72	0\\
24.73	0\\
24.74	0\\
24.75	0\\
24.76	0\\
24.77	0\\
24.78	0\\
24.79	0\\
24.8	0\\
24.81	0\\
24.82	0\\
24.83	0\\
24.84	0\\
24.85	0\\
24.86	0\\
24.87	0\\
24.88	0\\
24.89	0\\
24.9	0\\
24.91	0\\
24.92	0\\
24.93	0\\
24.94	0\\
24.95	0\\
24.96	0\\
24.97	0\\
24.98	0\\
24.99	0\\
25	0\\
25.01	0\\
25.02	0\\
25.03	0\\
25.04	0\\
25.05	0\\
25.06	0\\
25.07	0\\
25.08	0\\
25.09	0\\
25.1	0\\
25.11	0\\
25.12	0\\
25.13	0\\
25.14	0\\
25.15	0\\
25.16	0\\
25.17	0\\
25.18	0\\
25.19	0\\
25.2	0\\
25.21	0\\
25.22	0\\
25.23	0\\
25.24	0\\
25.25	0\\
25.26	0\\
25.27	0\\
25.28	0\\
25.29	0\\
25.3	0\\
25.31	0\\
25.32	0\\
25.33	0\\
25.34	0\\
25.35	0\\
25.36	0\\
25.37	0\\
25.38	0\\
25.39	0\\
25.4	0\\
25.41	0\\
25.42	0\\
25.43	0\\
25.44	0\\
25.45	0\\
25.46	0\\
25.47	0\\
25.48	0\\
25.49	0\\
25.5	0\\
25.51	0\\
25.52	0\\
25.53	0\\
25.54	0\\
25.55	0\\
25.56	0\\
25.57	0\\
25.58	0\\
25.59	0\\
25.6	0\\
25.61	0\\
25.62	0\\
25.63	0\\
25.64	0\\
25.65	0\\
25.66	0\\
25.67	0\\
25.68	0\\
25.69	0\\
25.7	0\\
25.71	0\\
25.72	0\\
25.73	0\\
25.74	0\\
25.75	0\\
25.76	0\\
25.77	0\\
25.78	0\\
25.79	0\\
25.8	0\\
25.81	0\\
25.82	0\\
25.83	0\\
25.84	0\\
25.85	0\\
25.86	0\\
25.87	0\\
25.88	0\\
25.89	0\\
25.9	0\\
25.91	0\\
25.92	0\\
25.93	0\\
25.94	0\\
25.95	0\\
25.96	0\\
25.97	0\\
25.98	0\\
25.99	0\\
26	0\\
26.01	0\\
26.02	0\\
26.03	0\\
26.04	0\\
26.05	0\\
26.06	0\\
26.07	0\\
26.08	0\\
26.09	0\\
26.1	0\\
26.11	0\\
26.12	0\\
26.13	0\\
26.14	0\\
26.15	0\\
26.16	0\\
26.17	0\\
26.18	0\\
26.19	0\\
26.2	0\\
26.21	0\\
26.22	0\\
26.23	0\\
26.24	0\\
26.25	0\\
26.26	0\\
26.27	0\\
26.28	0\\
26.29	0\\
26.3	0\\
26.31	0\\
26.32	0\\
26.33	0\\
26.34	0\\
26.35	0\\
26.36	0\\
26.37	0\\
26.38	0\\
26.39	0\\
26.4	0\\
26.41	0\\
26.42	0\\
26.43	0\\
26.44	0\\
26.45	0\\
26.46	0\\
26.47	0\\
26.48	0\\
26.49	0\\
26.5	0\\
26.51	0\\
26.52	0\\
26.53	0\\
26.54	0\\
26.55	0\\
26.56	0\\
26.57	0\\
26.58	0\\
26.59	0\\
26.6	0\\
26.61	0\\
26.62	0\\
26.63	0\\
26.64	0\\
26.65	0\\
26.66	0\\
26.67	0\\
26.68	0\\
26.69	0\\
26.7	0\\
26.71	0\\
26.72	0\\
26.73	0\\
26.74	0\\
26.75	0\\
26.76	0\\
26.77	0\\
26.78	0\\
26.79	0\\
26.8	0\\
26.81	0\\
26.82	0\\
26.83	0\\
26.84	0\\
26.85	0\\
26.86	0\\
26.87	0\\
26.88	0\\
26.89	0\\
26.9	0\\
26.91	0\\
26.92	0\\
26.93	0\\
26.94	0\\
26.95	0\\
26.96	0\\
26.97	0\\
26.98	0\\
26.99	0\\
27	0\\
27.01	0\\
27.02	0\\
27.03	0\\
27.04	0\\
27.05	0\\
27.06	0\\
27.07	0\\
27.08	0\\
27.09	0\\
27.1	0\\
27.11	0\\
27.12	0\\
27.13	0\\
27.14	0\\
27.15	0\\
27.16	0\\
27.17	0\\
27.18	0\\
27.19	0\\
27.2	0\\
27.21	0\\
27.22	0\\
27.23	0\\
27.24	0\\
27.25	0\\
27.26	0\\
27.27	0\\
27.28	0\\
27.29	0\\
27.3	0\\
27.31	0\\
27.32	0\\
27.33	0\\
27.34	0\\
27.35	0\\
27.36	0\\
27.37	0\\
27.38	0\\
27.39	0\\
27.4	0\\
27.41	0\\
27.42	0\\
27.43	0\\
27.44	0\\
27.45	0\\
27.46	0\\
27.47	0\\
27.48	0\\
27.49	0\\
27.5	0\\
27.51	0\\
27.52	0\\
27.53	0\\
27.54	0\\
27.55	0\\
27.56	0\\
27.57	0\\
27.58	0\\
27.59	0\\
27.6	0\\
27.61	0\\
27.62	0\\
27.63	0\\
27.64	0\\
27.65	0\\
27.66	0\\
27.67	0\\
27.68	0\\
27.69	0\\
27.7	0\\
27.71	0\\
27.72	0\\
27.73	0\\
27.74	0\\
27.75	0\\
27.76	0\\
27.77	0\\
27.78	0\\
27.79	0\\
27.8	0\\
27.81	0\\
27.82	0\\
27.83	0\\
27.84	0\\
27.85	0\\
27.86	0\\
27.87	0\\
27.88	0\\
27.89	0\\
27.9	0\\
27.91	0\\
27.92	0\\
27.93	0\\
27.94	0\\
27.95	0\\
27.96	0\\
27.97	0\\
27.98	0\\
27.99	0\\
28	0\\
28.01	0\\
28.02	0\\
28.03	0\\
28.04	0\\
28.05	0\\
28.06	0\\
28.07	0\\
28.08	0\\
28.09	0\\
28.1	0\\
28.11	0\\
28.12	0\\
28.13	0\\
28.14	0\\
28.15	0\\
28.16	0\\
28.17	0\\
28.18	0\\
28.19	0\\
28.2	0\\
28.21	0\\
28.22	0\\
28.23	0\\
28.24	0\\
28.25	0\\
28.26	0\\
28.27	0\\
28.28	0\\
28.29	0\\
28.3	0\\
28.31	0\\
28.32	0\\
28.33	0\\
28.34	0\\
28.35	0\\
28.36	0\\
28.37	0\\
28.38	0\\
28.39	0\\
28.4	0\\
28.41	0\\
28.42	0\\
28.43	0\\
28.44	0\\
28.45	0\\
28.46	0\\
28.47	0\\
28.48	0\\
28.49	0\\
28.5	0\\
28.51	0\\
28.52	0\\
28.53	0\\
28.54	0\\
28.55	0\\
28.56	0\\
28.57	0\\
28.58	0\\
28.59	0\\
28.6	0\\
28.61	0\\
28.62	0\\
28.63	0\\
28.64	0\\
28.65	0\\
28.66	0\\
28.67	0\\
28.68	0\\
28.69	0\\
28.7	0\\
28.71	0\\
28.72	0\\
28.73	0\\
28.74	0\\
28.75	0\\
28.76	0\\
28.77	0\\
28.78	0\\
28.79	0\\
28.8	0\\
28.81	0\\
28.82	0\\
28.83	0\\
28.84	0\\
28.85	0\\
28.86	0\\
28.87	0\\
28.88	0\\
28.89	0\\
28.9	0\\
28.91	0\\
28.92	0\\
28.93	0\\
28.94	0\\
28.95	0\\
28.96	0\\
28.97	0\\
28.98	0\\
28.99	0\\
29	0\\
29.01	0\\
29.02	0\\
29.03	0\\
29.04	0\\
29.05	0\\
29.06	0\\
29.07	0\\
29.08	0\\
29.09	0\\
29.1	0\\
29.11	0\\
29.12	0\\
29.13	0\\
29.14	0\\
29.15	0\\
29.16	0\\
29.17	0\\
29.18	0\\
29.19	0\\
29.2	0\\
29.21	0\\
29.22	0\\
29.23	0\\
29.24	0\\
29.25	0\\
29.26	0\\
29.27	0\\
29.28	0\\
29.29	0\\
29.3	0\\
29.31	0\\
29.32	0\\
29.33	0\\
29.34	0\\
29.35	0\\
29.36	0\\
29.37	0\\
29.38	0\\
29.39	0\\
29.4	0\\
29.41	0\\
29.42	0\\
29.43	0\\
29.44	0\\
29.45	0\\
29.46	0\\
29.47	0\\
29.48	0\\
29.49	0\\
29.5	0\\
29.51	0\\
29.52	0\\
29.53	0\\
29.54	0\\
29.55	0\\
29.56	0\\
29.57	0\\
29.58	0\\
29.59	0\\
29.6	0\\
29.61	0\\
29.62	0\\
29.63	0\\
29.64	0\\
29.65	0\\
29.66	0\\
29.67	0\\
29.68	0\\
29.69	0\\
29.7	0\\
29.71	0\\
29.72	0\\
29.73	0\\
29.74	0\\
29.75	0\\
29.76	0\\
29.77	0\\
29.78	0\\
29.79	0\\
29.8	0\\
29.81	0\\
29.82	0\\
29.83	0\\
29.84	0\\
29.85	0\\
29.86	0\\
29.87	0\\
29.88	0\\
29.89	0\\
29.9	0\\
29.91	0\\
29.92	0\\
29.93	0\\
29.94	0\\
29.95	0\\
29.96	0\\
29.97	0\\
29.98	0\\
29.99	0\\
30	0\\
30.01	0\\
30.02	0\\
30.03	0\\
30.04	0\\
30.05	0\\
30.06	0\\
30.07	0\\
30.08	0\\
30.09	0\\
30.1	0\\
30.11	0\\
30.12	0\\
30.13	0\\
30.14	0\\
30.15	0\\
30.16	0\\
30.17	0\\
30.18	0\\
30.19	0\\
30.2	0\\
30.21	0\\
30.22	0\\
30.23	0\\
30.24	0\\
30.25	0\\
30.26	0\\
30.27	0\\
30.28	0\\
30.29	0\\
30.3	0\\
30.31	0\\
30.32	0\\
30.33	0\\
30.34	0\\
30.35	0\\
30.36	0\\
30.37	0\\
30.38	0\\
30.39	0\\
30.4	0\\
30.41	0\\
30.42	0\\
30.43	0\\
30.44	0\\
30.45	0\\
30.46	0\\
30.47	0\\
30.48	0\\
30.49	0\\
30.5	0\\
30.51	0\\
30.52	0\\
30.53	0\\
30.54	0\\
30.55	0\\
30.56	0\\
30.57	0\\
30.58	0\\
30.59	0\\
30.6	0\\
30.61	0\\
30.62	0\\
30.63	0\\
30.64	0\\
30.65	0\\
30.66	0\\
30.67	0\\
30.68	0\\
30.69	0\\
30.7	0\\
30.71	0\\
30.72	0\\
30.73	0\\
30.74	0\\
30.75	0\\
30.76	0\\
30.77	0\\
30.78	0\\
30.79	0\\
30.8	0\\
30.81	0\\
30.82	0\\
30.83	0\\
30.84	0\\
30.85	0\\
30.86	0\\
30.87	0\\
30.88	0\\
30.89	0\\
30.9	0\\
30.91	0\\
30.92	0\\
30.93	0\\
30.94	0\\
30.95	0\\
30.96	0\\
30.97	0\\
30.98	0\\
30.99	0\\
31	0\\
31.01	0\\
31.02	0\\
31.03	0\\
31.04	0\\
31.05	0\\
31.06	0\\
31.07	0\\
31.08	0\\
31.09	0\\
31.1	0\\
31.11	0\\
31.12	0\\
31.13	0\\
31.14	0\\
31.15	0\\
31.16	0\\
31.17	0\\
31.18	0\\
31.19	0\\
31.2	0\\
31.21	0\\
31.22	0\\
31.23	0\\
31.24	0\\
31.25	0\\
31.26	0\\
31.27	0\\
31.28	0\\
31.29	0\\
31.3	0\\
31.31	0\\
31.32	0\\
31.33	0\\
31.34	0\\
31.35	0\\
31.36	0\\
31.37	0\\
31.38	0\\
31.39	0\\
31.4	0\\
31.41	0\\
31.42	0\\
31.43	0\\
31.44	0\\
31.45	0\\
31.46	0\\
31.47	0\\
31.48	0\\
31.49	0\\
31.5	0\\
31.51	0\\
31.52	0\\
31.53	0\\
31.54	0\\
31.55	0\\
31.56	0\\
31.57	0\\
31.58	0\\
31.59	0\\
31.6	0\\
31.61	0\\
31.62	0\\
31.63	0\\
31.64	0\\
31.65	0\\
31.66	0\\
31.67	0\\
31.68	0\\
31.69	0\\
31.7	0\\
31.71	0\\
31.72	0\\
31.73	0\\
31.74	0\\
31.75	0\\
31.76	0\\
31.77	0\\
31.78	0\\
31.79	0\\
31.8	0\\
31.81	0\\
31.82	0\\
31.83	0\\
31.84	0\\
31.85	0\\
31.86	0\\
31.87	0\\
31.88	0\\
31.89	0\\
31.9	0\\
31.91	0\\
31.92	0\\
31.93	0\\
31.94	0\\
31.95	0\\
31.96	0\\
31.97	0\\
31.98	0\\
31.99	0\\
32	0\\
32.01	0\\
32.02	0\\
32.03	0\\
32.04	0\\
32.05	0\\
32.06	0\\
32.07	0\\
32.08	0\\
32.09	0\\
32.1	0\\
32.11	0\\
32.12	0\\
32.13	0\\
32.14	0\\
32.15	0\\
32.16	0\\
32.17	0\\
32.18	0\\
32.19	0\\
32.2	0\\
32.21	0\\
32.22	0\\
32.23	0\\
32.24	0\\
32.25	0\\
32.26	0\\
32.27	0\\
32.28	0\\
32.29	0\\
32.3	0\\
32.31	0\\
32.32	0\\
32.33	0\\
32.34	0\\
32.35	0\\
32.36	0\\
32.37	0\\
32.38	0\\
32.39	0\\
32.4	0\\
32.41	0\\
32.42	0\\
32.43	0\\
32.44	0\\
32.45	0\\
32.46	0\\
32.47	0\\
32.48	0\\
32.49	0\\
32.5	0\\
32.51	0\\
32.52	0\\
32.53	0\\
32.54	0\\
32.55	0\\
32.56	0\\
32.57	0\\
32.58	0\\
32.59	0\\
32.6	0\\
32.61	0\\
32.62	0\\
32.63	0\\
32.64	0\\
32.65	0\\
32.66	0\\
32.67	0\\
32.68	0\\
32.69	0\\
32.7	0\\
32.71	0\\
32.72	0\\
32.73	0\\
32.74	0\\
32.75	0\\
32.76	0\\
32.77	0\\
32.78	0\\
32.79	0\\
32.8	0\\
32.81	0\\
32.82	0\\
32.83	0\\
32.84	0\\
32.85	0\\
32.86	0\\
32.87	0\\
32.88	0\\
32.89	0\\
32.9	0\\
32.91	0\\
32.92	0\\
32.93	0\\
32.94	0\\
32.95	0\\
32.96	0\\
32.97	0\\
32.98	0\\
32.99	0\\
33	0\\
33.01	0\\
33.02	0\\
33.03	0\\
33.04	0\\
33.05	0\\
33.06	0\\
33.07	0\\
33.08	0\\
33.09	0\\
33.1	0\\
33.11	0\\
33.12	0\\
33.13	0\\
33.14	0\\
33.15	0\\
33.16	0\\
33.17	0\\
33.18	0\\
33.19	0\\
33.2	0\\
33.21	0\\
33.22	0\\
33.23	0\\
33.24	0\\
33.25	0\\
33.26	0\\
33.27	0\\
33.28	0\\
33.29	0\\
33.3	0\\
33.31	0\\
33.32	0\\
33.33	0\\
33.34	0\\
33.35	0\\
33.36	0\\
33.37	0\\
33.38	0\\
33.39	0\\
33.4	0\\
33.41	0\\
33.42	0\\
33.43	0\\
33.44	0\\
33.45	0\\
33.46	0\\
33.47	0\\
33.48	0\\
33.49	0\\
33.5	0\\
33.51	0\\
33.52	0\\
33.53	0\\
33.54	0\\
33.55	0\\
33.56	0\\
33.57	0\\
33.58	0\\
33.59	0\\
33.6	0\\
33.61	0\\
33.62	0\\
33.63	0\\
33.64	0\\
33.65	0\\
33.66	0\\
33.67	0\\
33.68	0\\
33.69	0\\
33.7	0\\
33.71	0\\
33.72	0\\
33.73	0\\
33.74	0\\
33.75	0\\
33.76	0\\
33.77	0\\
33.78	0\\
33.79	0\\
33.8	0\\
33.81	0\\
33.82	0\\
33.83	0\\
33.84	0\\
33.85	0\\
33.86	0\\
33.87	0\\
33.88	0\\
33.89	0\\
33.9	0\\
33.91	0\\
33.92	0\\
33.93	0\\
33.94	0\\
33.95	0\\
33.96	0\\
33.97	0\\
33.98	0\\
33.99	0\\
34	0\\
34.01	0\\
34.02	0\\
34.03	0\\
34.04	0\\
34.05	0\\
34.06	0\\
34.07	0\\
34.08	0\\
34.09	0\\
34.1	0\\
34.11	0\\
34.12	0\\
34.13	0\\
34.14	0\\
34.15	0\\
34.16	0\\
34.17	0\\
34.18	0\\
34.19	0\\
34.2	0\\
34.21	0\\
34.22	0\\
34.23	0\\
34.24	0\\
34.25	0\\
34.26	0\\
34.27	0\\
34.28	0\\
34.29	0\\
34.3	0\\
34.31	0\\
34.32	0\\
34.33	0\\
34.34	0\\
34.35	0\\
34.36	0\\
34.37	0\\
34.38	0\\
34.39	0\\
34.4	0\\
34.41	0\\
34.42	0\\
34.43	0\\
34.44	0\\
34.45	0\\
34.46	0\\
34.47	0\\
34.48	0\\
34.49	0\\
34.5	0\\
34.51	0\\
34.52	0\\
34.53	0\\
34.54	0\\
34.55	0\\
34.56	0\\
34.57	0\\
34.58	0\\
34.59	0\\
34.6	0\\
34.61	0\\
34.62	0\\
34.63	0\\
34.64	0\\
34.65	0\\
34.66	0\\
34.67	0\\
34.68	0\\
34.69	0\\
34.7	0\\
34.71	0\\
34.72	0\\
34.73	0\\
34.74	0\\
34.75	0\\
34.76	0\\
34.77	0\\
34.78	0\\
34.79	0\\
34.8	0\\
34.81	0\\
34.82	0\\
34.83	0\\
34.84	0\\
34.85	0\\
34.86	0\\
34.87	0\\
34.88	0\\
34.89	0\\
34.9	0\\
34.91	0\\
34.92	0\\
34.93	0\\
34.94	0\\
34.95	0\\
34.96	0\\
34.97	0\\
34.98	0\\
34.99	0\\
35	0\\
35.01	0\\
35.02	0\\
35.03	0\\
35.04	0\\
35.05	0\\
35.06	0\\
35.07	0\\
35.08	0\\
35.09	0\\
35.1	0\\
35.11	0\\
35.12	0\\
35.13	0\\
35.14	0\\
35.15	0\\
35.16	0\\
35.17	0\\
35.18	0\\
35.19	0\\
35.2	0\\
35.21	0\\
35.22	0\\
35.23	0\\
35.24	0\\
35.25	0\\
35.26	0\\
35.27	0\\
35.28	0\\
35.29	0\\
35.3	0\\
35.31	0\\
35.32	0\\
35.33	0\\
35.34	0\\
35.35	0\\
35.36	0\\
35.37	0\\
35.38	0\\
35.39	0\\
35.4	0\\
35.41	0\\
35.42	0\\
35.43	0\\
35.44	0\\
35.45	0\\
35.46	0\\
35.47	0\\
35.48	0\\
35.49	0\\
35.5	0\\
35.51	0\\
35.52	0\\
35.53	0\\
35.54	0\\
35.55	0\\
35.56	0\\
35.57	0\\
35.58	0\\
35.59	0\\
35.6	0\\
35.61	0\\
35.62	0\\
35.63	0\\
35.64	0\\
35.65	0\\
35.66	0\\
35.67	0\\
35.68	0\\
35.69	0\\
35.7	0\\
35.71	0\\
35.72	0\\
35.73	0\\
35.74	0\\
35.75	0\\
35.76	0\\
35.77	0\\
35.78	0\\
35.79	0\\
35.8	0\\
35.81	0\\
35.82	0\\
35.83	0\\
35.84	0\\
35.85	0\\
35.86	0\\
35.87	0\\
35.88	0\\
35.89	0\\
35.9	0\\
35.91	0\\
35.92	0\\
35.93	0\\
35.94	0\\
35.95	0\\
35.96	0\\
35.97	0\\
35.98	0\\
35.99	0\\
36	0\\
36.01	0\\
36.02	0\\
36.03	0\\
36.04	0\\
36.05	0\\
36.06	0\\
36.07	0\\
36.08	0\\
36.09	0\\
36.1	0\\
36.11	0\\
36.12	0\\
36.13	0\\
36.14	0\\
36.15	0\\
36.16	0\\
36.17	0\\
36.18	0\\
36.19	0\\
36.2	0\\
36.21	0\\
36.22	0\\
36.23	0\\
36.24	0\\
36.25	0\\
36.26	0\\
36.27	0\\
36.28	0\\
36.29	0\\
36.3	0\\
36.31	0\\
36.32	0\\
36.33	0\\
36.34	0\\
36.35	0\\
36.36	0\\
36.37	0\\
36.38	0\\
36.39	0\\
36.4	0\\
36.41	0\\
36.42	0\\
36.43	0\\
36.44	0\\
36.45	0\\
36.46	0\\
36.47	0\\
36.48	0\\
36.49	0\\
36.5	0\\
36.51	0\\
36.52	0\\
36.53	0\\
36.54	0\\
36.55	0\\
36.56	0\\
36.57	0\\
36.58	0\\
36.59	0\\
36.6	0\\
36.61	0\\
36.62	0\\
36.63	0\\
36.64	0\\
36.65	0\\
36.66	0\\
36.67	0\\
36.68	0\\
36.69	0\\
36.7	0\\
36.71	0\\
36.72	0\\
36.73	0\\
36.74	0\\
36.75	0\\
36.76	0\\
36.77	0\\
36.78	0\\
36.79	0\\
36.8	0\\
36.81	0\\
36.82	0\\
36.83	0\\
36.84	0\\
36.85	0\\
36.86	0\\
36.87	0\\
36.88	0\\
36.89	0\\
36.9	0\\
36.91	0\\
36.92	0\\
36.93	0\\
36.94	0\\
36.95	0\\
36.96	0\\
36.97	0\\
36.98	0\\
36.99	0\\
37	0\\
37.01	0\\
37.02	0\\
37.03	0\\
37.04	0\\
37.05	0\\
37.06	0\\
37.07	0\\
37.08	0\\
37.09	0\\
37.1	0\\
37.11	0\\
37.12	0\\
37.13	0\\
37.14	0\\
37.15	0\\
37.16	0\\
37.17	0\\
37.18	0\\
37.19	0\\
37.2	0\\
37.21	0\\
37.22	0\\
37.23	0\\
37.24	0\\
37.25	0\\
37.26	0\\
37.27	0\\
37.28	0\\
37.29	0\\
37.3	0\\
37.31	0\\
37.32	0\\
37.33	0\\
37.34	0\\
37.35	0\\
37.36	0\\
37.37	0\\
37.38	0\\
37.39	0\\
37.4	0\\
37.41	0\\
37.42	0\\
37.43	0\\
37.44	0\\
37.45	0\\
37.46	0\\
37.47	0\\
37.48	0\\
37.49	0\\
37.5	0\\
37.51	0\\
37.52	0\\
37.53	0\\
37.54	0\\
37.55	0\\
37.56	0\\
37.57	0\\
37.58	0\\
37.59	0\\
37.6	0\\
37.61	0\\
37.62	0\\
37.63	0\\
37.64	0\\
37.65	0\\
37.66	0\\
37.67	0\\
37.68	0\\
37.69	0\\
37.7	0\\
37.71	0\\
37.72	0\\
37.73	0\\
37.74	0\\
37.75	0\\
37.76	0\\
37.77	0\\
37.78	0\\
37.79	0\\
37.8	0\\
37.81	0\\
37.82	0\\
37.83	0\\
37.84	0\\
37.85	0\\
37.86	0\\
37.87	0\\
37.88	0\\
37.89	0\\
37.9	0\\
37.91	0\\
37.92	0\\
37.93	0\\
37.94	0\\
37.95	0\\
37.96	0\\
37.97	0\\
37.98	0\\
37.99	0\\
38	0\\
38.01	0\\
38.02	0\\
38.03	0\\
38.04	0\\
38.05	0\\
38.06	0\\
38.07	0\\
38.08	0\\
38.09	0\\
38.1	0\\
38.11	0\\
38.12	0\\
38.13	0\\
38.14	0\\
38.15	0\\
38.16	0\\
38.17	0\\
38.18	0\\
38.19	0\\
38.2	0\\
38.21	0\\
38.22	0\\
38.23	0\\
38.24	0\\
38.25	0\\
38.26	0\\
38.27	0\\
38.28	0\\
38.29	0\\
38.3	0\\
38.31	0\\
38.32	0\\
38.33	0\\
38.34	0\\
38.35	0\\
38.36	0\\
38.37	0\\
38.38	0\\
38.39	0\\
38.4	0\\
38.41	0\\
38.42	0\\
38.43	0\\
38.44	0\\
38.45	0\\
38.46	0\\
38.47	0\\
38.48	0\\
38.49	0\\
38.5	0\\
38.51	0\\
38.52	0\\
38.53	0\\
38.54	0\\
38.55	0\\
38.56	0\\
38.57	0\\
38.58	0\\
38.59	0\\
38.6	0\\
38.61	0\\
38.62	0\\
38.63	0\\
38.64	0\\
38.65	0\\
38.66	0\\
38.67	0\\
38.68	0\\
38.69	0\\
38.7	0\\
38.71	0\\
38.72	0\\
38.73	0\\
38.74	0\\
38.75	0\\
38.76	0\\
38.77	0\\
38.78	0\\
38.79	0\\
38.8	0\\
38.81	0\\
38.82	0\\
38.83	0\\
38.84	0\\
38.85	0\\
38.86	0\\
38.87	0\\
38.88	0\\
38.89	0\\
38.9	0\\
38.91	0\\
38.92	0\\
38.93	0\\
38.94	0\\
38.95	0\\
38.96	0\\
38.97	0\\
38.98	0\\
38.99	0\\
39	0\\
39.01	0\\
39.02	0\\
39.03	0\\
39.04	0\\
39.05	0\\
39.06	0\\
39.07	0\\
39.08	0\\
39.09	0\\
39.1	0\\
39.11	0\\
39.12	0\\
39.13	0\\
39.14	0\\
39.15	0\\
39.16	0\\
39.17	0\\
39.18	0\\
39.19	0\\
39.2	0\\
39.21	0\\
39.22	0\\
39.23	0\\
39.24	0\\
39.25	0\\
39.26	0\\
39.27	0\\
39.28	0\\
39.29	0\\
39.3	0\\
39.31	0\\
39.32	0\\
39.33	0\\
39.34	0\\
39.35	0\\
39.36	0\\
39.37	0\\
39.38	0\\
39.39	0\\
39.4	0\\
39.41	0\\
39.42	0\\
39.43	0\\
39.44	0\\
39.45	0\\
39.46	0\\
39.47	0\\
39.48	0\\
39.49	0\\
39.5	0\\
39.51	0\\
39.52	0\\
39.53	0\\
39.54	0\\
39.55	0\\
39.56	0\\
39.57	0\\
39.58	0\\
39.59	0\\
39.6	0\\
39.61	0\\
39.62	0\\
39.63	0\\
39.64	0\\
39.65	0\\
39.66	0\\
39.67	0\\
39.68	0\\
39.69	0\\
39.7	0\\
39.71	0\\
39.72	0\\
39.73	0\\
39.74	0\\
39.75	0\\
39.76	0\\
39.77	0\\
39.78	0\\
39.79	0\\
39.8	0\\
39.81	0\\
39.82	0\\
39.83	0\\
39.84	0\\
39.85	0\\
39.86	0\\
39.87	0\\
39.88	0\\
39.89	0\\
39.9	0\\
39.91	0\\
39.92	0\\
39.93	0\\
39.94	0\\
39.95	0\\
39.96	0\\
39.97	0\\
39.98	0\\
39.99	0\\
40	0\\
40.01	0\\
};
\addplot [color=blue,solid,forget plot]
  table[row sep=crcr]{%
40.01	0\\
40.02	0\\
40.03	0\\
40.04	0\\
40.05	0\\
40.06	0\\
40.07	0\\
40.08	0\\
40.09	0\\
40.1	0\\
40.11	0\\
40.12	0\\
40.13	0\\
40.14	0\\
40.15	0\\
40.16	0\\
40.17	0\\
40.18	0\\
40.19	0\\
40.2	0\\
40.21	0\\
40.22	0\\
40.23	0\\
40.24	0\\
40.25	0\\
40.26	0\\
40.27	0\\
40.28	0\\
40.29	0\\
40.3	0\\
40.31	0\\
40.32	0\\
40.33	0\\
40.34	0\\
40.35	0\\
40.36	0\\
40.37	0\\
40.38	0\\
40.39	0\\
40.4	0\\
40.41	0\\
40.42	0\\
40.43	0\\
40.44	0\\
40.45	0\\
40.46	0\\
40.47	0\\
40.48	0\\
40.49	0\\
40.5	0\\
40.51	0\\
40.52	0\\
40.53	0\\
40.54	0\\
40.55	0\\
40.56	0\\
40.57	0\\
40.58	0\\
40.59	0\\
40.6	0\\
40.61	0\\
40.62	0\\
40.63	0\\
40.64	0\\
40.65	0\\
40.66	0\\
40.67	0\\
40.68	0\\
40.69	0\\
40.7	0\\
40.71	0\\
40.72	0\\
40.73	0\\
40.74	0\\
40.75	0\\
40.76	0\\
40.77	0\\
40.78	0\\
40.79	0\\
40.8	0\\
40.81	0\\
40.82	0\\
40.83	0\\
40.84	0\\
40.85	0\\
40.86	0\\
40.87	0\\
40.88	0\\
40.89	0\\
40.9	0\\
40.91	0\\
40.92	0\\
40.93	0\\
40.94	0\\
40.95	0\\
40.96	0\\
40.97	0\\
40.98	0\\
40.99	0\\
41	0\\
41.01	0\\
41.02	0\\
41.03	0\\
41.04	0\\
41.05	0\\
41.06	0\\
41.07	0\\
41.08	0\\
41.09	0\\
41.1	0\\
41.11	0\\
41.12	0\\
41.13	0\\
41.14	0\\
41.15	0\\
41.16	0\\
41.17	0\\
41.18	0\\
41.19	0\\
41.2	0\\
41.21	0\\
41.22	0\\
41.23	0\\
41.24	0\\
41.25	0\\
41.26	0\\
41.27	0\\
41.28	0\\
41.29	0\\
41.3	0\\
41.31	0\\
41.32	0\\
41.33	0\\
41.34	0\\
41.35	0\\
41.36	0\\
41.37	0\\
41.38	0\\
41.39	0\\
41.4	0\\
41.41	0\\
41.42	0\\
41.43	0\\
41.44	0\\
41.45	0\\
41.46	0\\
41.47	0\\
41.48	0\\
41.49	0\\
41.5	0\\
41.51	0\\
41.52	0\\
41.53	0\\
41.54	0\\
41.55	0\\
41.56	0\\
41.57	0\\
41.58	0\\
41.59	0\\
41.6	0\\
41.61	0\\
41.62	0\\
41.63	0\\
41.64	0\\
41.65	0\\
41.66	0\\
41.67	0\\
41.68	0\\
41.69	0\\
41.7	0\\
41.71	0\\
41.72	0\\
41.73	0\\
41.74	0\\
41.75	0\\
41.76	0\\
41.77	0\\
41.78	0\\
41.79	0\\
41.8	0\\
41.81	0\\
41.82	0\\
41.83	0\\
41.84	0\\
41.85	0\\
41.86	0\\
41.87	0\\
41.88	0\\
41.89	0\\
41.9	0\\
41.91	0\\
41.92	0\\
41.93	0\\
41.94	0\\
41.95	0\\
41.96	0\\
41.97	0\\
41.98	0\\
41.99	0\\
42	0\\
42.01	0\\
42.02	0\\
42.03	0\\
42.04	0\\
42.05	0\\
42.06	0\\
42.07	0\\
42.08	0\\
42.09	0\\
42.1	0\\
42.11	0\\
42.12	0\\
42.13	0\\
42.14	0\\
42.15	0\\
42.16	0\\
42.17	0\\
42.18	0\\
42.19	0\\
42.2	0\\
42.21	0\\
42.22	0\\
42.23	0\\
42.24	0\\
42.25	0\\
42.26	0\\
42.27	0\\
42.28	0\\
42.29	0\\
42.3	0\\
42.31	0\\
42.32	0\\
42.33	0\\
42.34	0\\
42.35	0\\
42.36	0\\
42.37	0\\
42.38	0\\
42.39	0\\
42.4	0\\
42.41	0\\
42.42	0\\
42.43	0\\
42.44	0\\
42.45	0\\
42.46	0\\
42.47	0\\
42.48	0\\
42.49	0\\
42.5	0\\
42.51	0\\
42.52	0\\
42.53	0\\
42.54	0\\
42.55	0\\
42.56	0\\
42.57	0\\
42.58	0\\
42.59	0\\
42.6	0\\
42.61	0\\
42.62	0\\
42.63	0\\
42.64	0\\
42.65	0\\
42.66	0\\
42.67	0\\
42.68	0\\
42.69	0\\
42.7	0\\
42.71	0\\
42.72	0\\
42.73	0\\
42.74	0\\
42.75	0\\
42.76	0\\
42.77	0\\
42.78	0\\
42.79	0\\
42.8	0\\
42.81	0\\
42.82	0\\
42.83	0\\
42.84	0\\
42.85	0\\
42.86	0\\
42.87	0\\
42.88	0\\
42.89	0\\
42.9	0\\
42.91	0\\
42.92	0\\
42.93	0\\
42.94	0\\
42.95	0\\
42.96	0\\
42.97	0\\
42.98	0\\
42.99	0\\
43	0\\
43.01	0\\
43.02	0\\
43.03	0\\
43.04	0\\
43.05	0\\
43.06	0\\
43.07	0\\
43.08	0\\
43.09	0\\
43.1	0\\
43.11	0\\
43.12	0\\
43.13	0\\
43.14	0\\
43.15	0\\
43.16	0\\
43.17	0\\
43.18	0\\
43.19	0\\
43.2	0\\
43.21	0\\
43.22	0\\
43.23	0\\
43.24	0\\
43.25	0\\
43.26	0\\
43.27	0\\
43.28	0\\
43.29	0\\
43.3	0\\
43.31	0\\
43.32	0\\
43.33	0\\
43.34	0\\
43.35	0\\
43.36	0\\
43.37	0\\
43.38	0\\
43.39	0\\
43.4	0\\
43.41	0\\
43.42	0\\
43.43	0\\
43.44	0\\
43.45	0\\
43.46	0\\
43.47	0\\
43.48	0\\
43.49	0\\
43.5	0\\
43.51	0\\
43.52	0\\
43.53	0\\
43.54	0\\
43.55	0\\
43.56	0\\
43.57	0\\
43.58	0\\
43.59	0\\
43.6	0\\
43.61	0\\
43.62	0\\
43.63	0\\
43.64	0\\
43.65	0\\
43.66	0\\
43.67	0\\
43.68	0\\
43.69	0\\
43.7	0\\
43.71	0\\
43.72	0\\
43.73	0\\
43.74	0\\
43.75	0\\
43.76	0\\
43.77	0\\
43.78	0\\
43.79	0\\
43.8	0\\
43.81	0\\
43.82	0\\
43.83	0\\
43.84	0\\
43.85	0\\
43.86	0\\
43.87	0\\
43.88	0\\
43.89	0\\
43.9	0\\
43.91	0\\
43.92	0\\
43.93	0\\
43.94	0\\
43.95	0\\
43.96	0\\
43.97	0\\
43.98	0\\
43.99	0\\
44	0\\
44.01	0\\
44.02	0\\
44.03	0\\
44.04	0\\
44.05	0\\
44.06	0\\
44.07	0\\
44.08	0\\
44.09	0\\
44.1	0\\
44.11	0\\
44.12	0\\
44.13	0\\
44.14	0\\
44.15	0\\
44.16	0\\
44.17	0\\
44.18	0\\
44.19	0\\
44.2	0\\
44.21	0\\
44.22	0\\
44.23	0\\
44.24	0\\
44.25	0\\
44.26	0\\
44.27	0\\
44.28	0\\
44.29	0\\
44.3	0\\
44.31	0\\
44.32	0\\
44.33	0\\
44.34	0\\
44.35	0\\
44.36	0\\
44.37	0\\
44.38	0\\
44.39	0\\
44.4	0\\
44.41	0\\
44.42	0\\
44.43	0\\
44.44	0\\
44.45	0\\
44.46	0\\
44.47	0\\
44.48	0\\
44.49	0\\
44.5	0\\
44.51	0\\
44.52	0\\
44.53	0\\
44.54	0\\
44.55	0\\
44.56	0\\
44.57	0\\
44.58	0\\
44.59	0\\
44.6	0\\
44.61	0\\
44.62	0\\
44.63	0\\
44.64	0\\
44.65	0\\
44.66	0\\
44.67	0\\
44.68	0\\
44.69	0\\
44.7	0\\
44.71	0\\
44.72	0\\
44.73	0\\
44.74	0\\
44.75	0\\
44.76	0\\
44.77	0\\
44.78	0\\
44.79	0\\
44.8	0\\
44.81	0\\
44.82	0\\
44.83	0\\
44.84	0\\
44.85	0\\
44.86	0\\
44.87	0\\
44.88	0\\
44.89	0\\
44.9	0\\
44.91	0\\
44.92	0\\
44.93	0\\
44.94	0\\
44.95	0\\
44.96	0\\
44.97	0\\
44.98	0\\
44.99	0\\
45	0\\
45.01	0\\
45.02	0\\
45.03	0\\
45.04	0\\
45.05	0\\
45.06	0\\
45.07	0\\
45.08	0\\
45.09	0\\
45.1	0\\
45.11	0\\
45.12	0\\
45.13	0\\
45.14	0\\
45.15	0\\
45.16	0\\
45.17	0\\
45.18	0\\
45.19	0\\
45.2	0\\
45.21	0\\
45.22	0\\
45.23	0\\
45.24	0\\
45.25	0\\
45.26	0\\
45.27	0\\
45.28	0\\
45.29	0\\
45.3	0\\
45.31	0\\
45.32	0\\
45.33	0\\
45.34	0\\
45.35	0\\
45.36	0\\
45.37	0\\
45.38	0\\
45.39	0\\
45.4	0\\
45.41	0\\
45.42	0\\
45.43	0\\
45.44	0\\
45.45	0\\
45.46	0\\
45.47	0\\
45.48	0\\
45.49	0\\
45.5	0\\
45.51	0\\
45.52	0\\
45.53	0\\
45.54	0\\
45.55	0\\
45.56	0\\
45.57	0\\
45.58	0\\
45.59	0\\
45.6	0\\
45.61	0\\
45.62	0\\
45.63	0\\
45.64	0\\
45.65	0\\
45.66	0\\
45.67	0\\
45.68	0\\
45.69	0\\
45.7	0\\
45.71	0\\
45.72	0\\
45.73	0\\
45.74	0\\
45.75	0\\
45.76	0\\
45.77	0\\
45.78	0\\
45.79	0\\
45.8	0\\
45.81	0\\
45.82	0\\
45.83	0\\
45.84	0\\
45.85	0\\
45.86	0\\
45.87	0\\
45.88	0\\
45.89	0\\
45.9	0\\
45.91	0\\
45.92	0\\
45.93	0\\
45.94	0\\
45.95	0\\
45.96	0\\
45.97	0\\
45.98	0\\
45.99	0\\
46	0\\
46.01	0\\
46.02	0\\
46.03	0\\
46.04	0\\
46.05	0\\
46.06	0\\
46.07	0\\
46.08	0\\
46.09	0\\
46.1	0\\
46.11	0\\
46.12	0\\
46.13	0\\
46.14	0\\
46.15	0\\
46.16	0\\
46.17	0\\
46.18	0\\
46.19	0\\
46.2	0\\
46.21	0\\
46.22	0\\
46.23	0\\
46.24	0\\
46.25	0\\
46.26	0\\
46.27	0\\
46.28	0\\
46.29	0\\
46.3	0\\
46.31	0\\
46.32	0\\
46.33	0\\
46.34	0\\
46.35	0\\
46.36	0\\
46.37	0\\
46.38	0\\
46.39	0\\
46.4	0\\
46.41	0\\
46.42	0\\
46.43	0\\
46.44	0\\
46.45	0\\
46.46	0\\
46.47	0\\
46.48	0\\
46.49	0\\
46.5	0\\
46.51	0\\
46.52	0\\
46.53	0\\
46.54	0\\
46.55	0\\
46.56	0\\
46.57	0\\
46.58	0\\
46.59	0\\
46.6	0\\
46.61	0\\
46.62	0\\
46.63	0\\
46.64	0\\
46.65	0\\
46.66	0\\
46.67	0\\
46.68	0\\
46.69	0\\
46.7	0\\
46.71	0\\
46.72	0\\
46.73	0\\
46.74	0\\
46.75	0\\
46.76	0\\
46.77	0\\
46.78	0\\
46.79	0\\
46.8	0\\
46.81	0\\
46.82	0\\
46.83	0\\
46.84	0\\
46.85	0\\
46.86	0\\
46.87	0\\
46.88	0\\
46.89	0\\
46.9	0\\
46.91	0\\
46.92	0\\
46.93	0\\
46.94	0\\
46.95	0\\
46.96	0\\
46.97	0\\
46.98	0\\
46.99	0\\
47	0\\
47.01	0\\
47.02	0\\
47.03	0\\
47.04	0\\
47.05	0\\
47.06	0\\
47.07	0\\
47.08	0\\
47.09	0\\
47.1	0\\
47.11	0\\
47.12	0\\
47.13	0\\
47.14	0\\
47.15	0\\
47.16	0\\
47.17	0\\
47.18	0\\
47.19	0\\
47.2	0\\
47.21	0\\
47.22	0\\
47.23	0\\
47.24	0\\
47.25	0\\
47.26	0\\
47.27	0\\
47.28	0\\
47.29	0\\
47.3	0\\
47.31	0\\
47.32	0\\
47.33	0\\
47.34	0\\
47.35	0\\
47.36	0\\
47.37	0\\
47.38	0\\
47.39	0\\
47.4	0\\
47.41	0\\
47.42	0\\
47.43	0\\
47.44	0\\
47.45	0\\
47.46	0\\
47.47	0\\
47.48	0\\
47.49	0\\
47.5	0\\
47.51	0\\
47.52	0\\
47.53	0\\
47.54	0\\
47.55	0\\
47.56	0\\
47.57	0\\
47.58	0\\
47.59	0\\
47.6	0\\
47.61	0\\
47.62	0\\
47.63	0\\
47.64	0\\
47.65	0\\
47.66	0\\
47.67	0\\
47.68	0\\
47.69	0\\
47.7	0\\
47.71	0\\
47.72	0\\
47.73	0\\
47.74	0\\
47.75	0\\
47.76	0\\
47.77	0\\
47.78	0\\
47.79	0\\
47.8	0\\
47.81	0\\
47.82	0\\
47.83	0\\
47.84	0\\
47.85	0\\
47.86	0\\
47.87	0\\
47.88	0\\
47.89	0\\
47.9	0\\
47.91	0\\
47.92	0\\
47.93	0\\
47.94	0\\
47.95	0\\
47.96	0\\
47.97	0\\
47.98	0\\
47.99	0\\
48	0\\
48.01	0\\
48.02	0\\
48.03	0\\
48.04	0\\
48.05	0\\
48.06	0\\
48.07	0\\
48.08	0\\
48.09	0\\
48.1	0\\
48.11	0\\
48.12	0\\
48.13	0\\
48.14	0\\
48.15	0\\
48.16	0\\
48.17	0\\
48.18	0\\
48.19	0\\
48.2	0\\
48.21	0\\
48.22	0\\
48.23	0\\
48.24	0\\
48.25	0\\
48.26	0\\
48.27	0\\
48.28	0\\
48.29	0\\
48.3	0\\
48.31	0\\
48.32	0\\
48.33	0\\
48.34	0\\
48.35	0\\
48.36	0\\
48.37	0\\
48.38	0\\
48.39	0\\
48.4	0\\
48.41	0\\
48.42	0\\
48.43	0\\
48.44	0\\
48.45	0\\
48.46	0\\
48.47	0\\
48.48	0\\
48.49	0\\
48.5	0\\
48.51	0\\
48.52	0\\
48.53	0\\
48.54	0\\
48.55	0\\
48.56	0\\
48.57	0\\
48.58	0\\
48.59	0\\
48.6	0\\
48.61	0\\
48.62	0\\
48.63	0\\
48.64	0\\
48.65	0\\
48.66	0\\
48.67	0\\
48.68	0\\
48.69	0\\
48.7	0\\
48.71	0\\
48.72	0\\
48.73	0\\
48.74	0\\
48.75	0\\
48.76	0\\
48.77	0\\
48.78	0\\
48.79	0\\
48.8	0\\
48.81	0\\
48.82	0\\
48.83	0\\
48.84	0\\
48.85	0\\
48.86	0\\
48.87	0\\
48.88	0\\
48.89	0\\
48.9	0\\
48.91	0\\
48.92	0\\
48.93	0\\
48.94	0\\
48.95	0\\
48.96	0\\
48.97	0\\
48.98	0\\
48.99	0\\
49	0\\
49.01	0\\
49.02	0\\
49.03	0\\
49.04	0\\
49.05	0\\
49.06	0\\
49.07	0\\
49.08	0\\
49.09	0\\
49.1	0\\
49.11	0\\
49.12	0\\
49.13	0\\
49.14	0\\
49.15	0\\
49.16	0\\
49.17	0\\
49.18	0\\
49.19	0\\
49.2	0\\
49.21	0\\
49.22	0\\
49.23	0\\
49.24	0\\
49.25	0\\
49.26	0\\
49.27	0\\
49.28	0\\
49.29	0\\
49.3	0\\
49.31	0\\
49.32	0\\
49.33	0\\
49.34	0\\
49.35	0\\
49.36	0\\
49.37	0\\
49.38	0\\
49.39	0\\
49.4	0\\
49.41	0\\
49.42	0\\
49.43	0\\
49.44	0\\
49.45	0\\
49.46	0\\
49.47	0\\
49.48	0\\
49.49	0\\
49.5	0\\
49.51	0\\
49.52	0\\
49.53	0\\
49.54	0\\
49.55	0\\
49.56	0\\
49.57	0\\
49.58	0\\
49.59	0\\
49.6	0\\
49.61	0\\
49.62	0\\
49.63	0\\
49.64	0\\
49.65	0\\
49.66	0\\
49.67	0\\
49.68	0\\
49.69	0\\
49.7	0\\
49.71	0\\
49.72	0\\
49.73	0\\
49.74	0\\
49.75	0\\
49.76	0\\
49.77	0\\
49.78	0\\
49.79	0\\
49.8	0\\
49.81	0\\
49.82	0\\
49.83	0\\
49.84	0\\
49.85	0\\
49.86	0\\
49.87	0\\
49.88	0\\
49.89	0\\
49.9	0\\
49.91	0\\
49.92	0\\
49.93	0\\
49.94	0\\
49.95	0\\
49.96	0\\
49.97	0\\
49.98	0\\
49.99	0\\
50	0\\
50.01	0\\
50.02	0\\
50.03	0\\
50.04	0\\
50.05	0\\
50.06	0\\
50.07	0\\
50.08	0\\
50.09	0\\
50.1	0\\
50.11	0\\
50.12	0\\
50.13	0\\
50.14	0\\
50.15	0\\
50.16	0\\
50.17	0\\
50.18	0\\
50.19	0\\
50.2	0\\
50.21	0\\
50.22	0\\
50.23	0\\
50.24	0\\
50.25	0\\
50.26	0\\
50.27	0\\
50.28	0\\
50.29	0\\
50.3	0\\
50.31	0\\
50.32	0\\
50.33	0\\
50.34	0\\
50.35	0\\
50.36	0\\
50.37	0\\
50.38	0\\
50.39	0\\
50.4	0\\
50.41	0\\
50.42	0\\
50.43	0\\
50.44	0\\
50.45	0\\
50.46	0\\
50.47	0\\
50.48	0\\
50.49	0\\
50.5	0\\
50.51	0\\
50.52	0\\
50.53	0\\
50.54	0\\
50.55	0\\
50.56	0\\
50.57	0\\
50.58	0\\
50.59	0\\
50.6	0\\
50.61	0\\
50.62	0\\
50.63	0\\
50.64	0\\
50.65	0\\
50.66	0\\
50.67	0\\
50.68	0\\
50.69	0\\
50.7	0\\
50.71	0\\
50.72	0\\
50.73	0\\
50.74	0\\
50.75	0\\
50.76	0\\
50.77	0\\
50.78	0\\
50.79	0\\
50.8	0\\
50.81	0\\
50.82	0\\
50.83	0\\
50.84	0\\
50.85	0\\
50.86	0\\
50.87	0\\
50.88	0\\
50.89	0\\
50.9	0\\
50.91	0\\
50.92	0\\
50.93	0\\
50.94	0\\
50.95	0\\
50.96	0\\
50.97	0\\
50.98	0\\
50.99	0\\
51	0\\
51.01	0\\
51.02	0\\
51.03	0\\
51.04	0\\
51.05	0\\
51.06	0\\
51.07	0\\
51.08	0\\
51.09	0\\
51.1	0\\
51.11	0\\
51.12	0\\
51.13	0\\
51.14	0\\
51.15	0\\
51.16	0\\
51.17	0\\
51.18	0\\
51.19	0\\
51.2	0\\
51.21	0\\
51.22	0\\
51.23	0\\
51.24	0\\
51.25	0\\
51.26	0\\
51.27	0\\
51.28	0\\
51.29	0\\
51.3	0\\
51.31	0\\
51.32	0\\
51.33	0\\
51.34	0\\
51.35	0\\
51.36	0\\
51.37	0\\
51.38	0\\
51.39	0\\
51.4	0\\
51.41	0\\
51.42	0\\
51.43	0\\
51.44	0\\
51.45	0\\
51.46	0\\
51.47	0\\
51.48	0\\
51.49	0\\
51.5	0\\
51.51	0\\
51.52	0\\
51.53	0\\
51.54	0\\
51.55	0\\
51.56	0\\
51.57	0\\
51.58	0\\
51.59	0\\
51.6	0\\
51.61	0\\
51.62	0\\
51.63	0\\
51.64	0\\
51.65	0\\
51.66	0\\
51.67	0\\
51.68	0\\
51.69	0\\
51.7	0\\
51.71	0\\
51.72	0\\
51.73	0\\
51.74	0\\
51.75	0\\
51.76	0\\
51.77	0\\
51.78	0\\
51.79	0\\
51.8	0\\
51.81	0\\
51.82	0\\
51.83	0\\
51.84	0\\
51.85	0\\
51.86	0\\
51.87	0\\
51.88	0\\
51.89	0\\
51.9	0\\
51.91	0\\
51.92	0\\
51.93	0\\
51.94	0\\
51.95	0\\
51.96	0\\
51.97	0\\
51.98	0\\
51.99	0\\
52	0\\
52.01	0\\
52.02	0\\
52.03	0\\
52.04	0\\
52.05	0\\
52.06	0\\
52.07	0\\
52.08	0\\
52.09	0\\
52.1	0\\
52.11	0\\
52.12	0\\
52.13	0\\
52.14	0\\
52.15	0\\
52.16	0\\
52.17	0\\
52.18	0\\
52.19	0\\
52.2	0\\
52.21	0\\
52.22	0\\
52.23	0\\
52.24	0\\
52.25	0\\
52.26	0\\
52.27	0\\
52.28	0\\
52.29	0\\
52.3	0\\
52.31	0\\
52.32	0\\
52.33	0\\
52.34	0\\
52.35	0\\
52.36	0\\
52.37	0\\
52.38	0\\
52.39	0\\
52.4	0\\
52.41	0\\
52.42	0\\
52.43	0\\
52.44	0\\
52.45	0\\
52.46	0\\
52.47	0\\
52.48	0\\
52.49	0\\
52.5	0\\
52.51	0\\
52.52	0\\
52.53	0\\
52.54	0\\
52.55	0\\
52.56	0\\
52.57	0\\
52.58	0\\
52.59	0\\
52.6	0\\
52.61	0\\
52.62	0\\
52.63	0\\
52.64	0\\
52.65	0\\
52.66	0\\
52.67	0\\
52.68	0\\
52.69	0\\
52.7	0\\
52.71	0\\
52.72	0\\
52.73	0\\
52.74	0\\
52.75	0\\
52.76	0\\
52.77	0\\
52.78	0\\
52.79	0\\
52.8	0\\
52.81	0\\
52.82	0\\
52.83	0\\
52.84	0\\
52.85	0\\
52.86	0\\
52.87	0\\
52.88	0\\
52.89	0\\
52.9	0\\
52.91	0\\
52.92	0\\
52.93	0\\
52.94	0\\
52.95	0\\
52.96	0\\
52.97	0\\
52.98	0\\
52.99	0\\
53	0\\
53.01	0\\
53.02	0\\
53.03	0\\
53.04	0\\
53.05	0\\
53.06	0\\
53.07	0\\
53.08	0\\
53.09	0\\
53.1	0\\
53.11	0\\
53.12	0\\
53.13	0\\
53.14	0\\
53.15	0\\
53.16	0\\
53.17	0\\
53.18	0\\
53.19	0\\
53.2	0\\
53.21	0\\
53.22	0\\
53.23	0\\
53.24	0\\
53.25	0\\
53.26	0\\
53.27	0\\
53.28	0\\
53.29	0\\
53.3	0\\
53.31	0\\
53.32	0\\
53.33	0\\
53.34	0\\
53.35	0\\
53.36	0\\
53.37	0\\
53.38	0\\
53.39	0\\
53.4	0\\
53.41	0\\
53.42	0\\
53.43	0\\
53.44	0\\
53.45	0\\
53.46	0\\
53.47	0\\
53.48	0\\
53.49	0\\
53.5	0\\
53.51	0\\
53.52	0\\
53.53	0\\
53.54	0\\
53.55	0\\
53.56	0\\
53.57	0\\
53.58	0\\
53.59	0\\
53.6	0\\
53.61	0\\
53.62	0\\
53.63	0\\
53.64	0\\
53.65	0\\
53.66	0\\
53.67	0\\
53.68	0\\
53.69	0\\
53.7	0\\
53.71	0\\
53.72	0\\
53.73	0\\
53.74	0\\
53.75	0\\
53.76	0\\
53.77	0\\
53.78	0\\
53.79	0\\
53.8	0\\
53.81	0\\
53.82	0\\
53.83	0\\
53.84	0\\
53.85	0\\
53.86	0\\
53.87	0\\
53.88	0\\
53.89	0\\
53.9	0\\
53.91	0\\
53.92	0\\
53.93	0\\
53.94	0\\
53.95	0\\
53.96	0\\
53.97	0\\
53.98	0\\
53.99	0\\
54	0\\
54.01	0\\
54.02	0\\
54.03	0\\
54.04	0\\
54.05	0\\
54.06	0\\
54.07	0\\
54.08	0\\
54.09	0\\
54.1	0\\
54.11	0\\
54.12	0\\
54.13	0\\
54.14	0\\
54.15	0\\
54.16	0\\
54.17	0\\
54.18	0\\
54.19	0\\
54.2	0\\
54.21	0\\
54.22	0\\
54.23	0\\
54.24	0\\
54.25	0\\
54.26	0\\
54.27	0\\
54.28	0\\
54.29	0\\
54.3	0\\
54.31	0\\
54.32	0\\
54.33	0\\
54.34	0\\
54.35	0\\
54.36	0\\
54.37	0\\
54.38	0\\
54.39	0\\
54.4	0\\
54.41	0\\
54.42	0\\
54.43	0\\
54.44	0\\
54.45	0\\
54.46	0\\
54.47	0\\
54.48	0\\
54.49	0\\
54.5	0\\
54.51	0\\
54.52	0\\
54.53	0\\
54.54	0\\
54.55	0\\
54.56	0\\
54.57	0\\
54.58	0\\
54.59	0\\
54.6	0\\
54.61	0\\
54.62	0\\
54.63	0\\
54.64	0\\
54.65	0\\
54.66	0\\
54.67	0\\
54.68	0\\
54.69	0\\
54.7	0\\
54.71	0\\
54.72	0\\
54.73	0\\
54.74	0\\
54.75	0\\
54.76	0\\
54.77	0\\
54.78	0\\
54.79	0\\
54.8	0\\
54.81	0\\
54.82	0\\
54.83	0\\
54.84	0\\
54.85	0\\
54.86	0\\
54.87	0\\
54.88	0\\
54.89	0\\
54.9	0\\
54.91	0\\
54.92	0\\
54.93	0\\
54.94	0\\
54.95	0\\
54.96	0\\
54.97	0\\
54.98	0\\
54.99	0\\
55	0\\
55.01	0\\
55.02	0\\
55.03	0\\
55.04	0\\
55.05	0\\
55.06	0\\
55.07	0\\
55.08	0\\
55.09	0\\
55.1	0\\
55.11	0\\
55.12	0\\
55.13	0\\
55.14	0\\
55.15	0\\
55.16	0\\
55.17	0\\
55.18	0\\
55.19	0\\
55.2	0\\
55.21	0\\
55.22	0\\
55.23	0\\
55.24	0\\
55.25	0\\
55.26	0\\
55.27	0\\
55.28	0\\
55.29	0\\
55.3	0\\
55.31	0\\
55.32	0\\
55.33	0\\
55.34	0\\
55.35	0\\
55.36	0\\
55.37	0\\
55.38	0\\
55.39	0\\
55.4	0\\
55.41	0\\
55.42	0\\
55.43	0\\
55.44	0\\
55.45	0\\
55.46	0\\
55.47	0\\
55.48	0\\
55.49	0\\
55.5	0\\
55.51	0\\
55.52	0\\
55.53	0\\
55.54	0\\
55.55	0\\
55.56	0\\
55.57	0\\
55.58	0\\
55.59	0\\
55.6	0\\
55.61	0\\
55.62	0\\
55.63	0\\
55.64	0\\
55.65	0\\
55.66	0\\
55.67	0\\
55.68	0\\
55.69	0\\
55.7	0\\
55.71	0\\
55.72	0\\
55.73	0\\
55.74	0\\
55.75	0\\
55.76	0\\
55.77	0\\
55.78	0\\
55.79	0\\
55.8	0\\
55.81	0\\
55.82	0\\
55.83	0\\
55.84	0\\
55.85	0\\
55.86	0\\
55.87	0\\
55.88	0\\
55.89	0\\
55.9	0\\
55.91	0\\
55.92	0\\
55.93	0\\
55.94	0\\
55.95	0\\
55.96	0\\
55.97	0\\
55.98	0\\
55.99	0\\
56	0\\
56.01	0\\
56.02	0\\
56.03	0\\
56.04	0\\
56.05	0\\
56.06	0\\
56.07	0\\
56.08	0\\
56.09	0\\
56.1	0\\
56.11	0\\
56.12	0\\
56.13	0\\
56.14	0\\
56.15	0\\
56.16	0\\
56.17	0\\
56.18	0\\
56.19	0\\
56.2	0\\
56.21	0\\
56.22	0\\
56.23	0\\
56.24	0\\
56.25	0\\
56.26	0\\
56.27	0\\
56.28	0\\
56.29	0\\
56.3	0\\
56.31	0\\
56.32	0\\
56.33	0\\
56.34	0\\
56.35	0\\
56.36	0\\
56.37	0\\
56.38	0\\
56.39	0\\
56.4	0\\
56.41	0\\
56.42	0\\
56.43	0\\
56.44	0\\
56.45	0\\
56.46	0\\
56.47	0\\
56.48	0\\
56.49	0\\
56.5	0\\
56.51	0\\
56.52	0\\
56.53	0\\
56.54	0\\
56.55	0\\
56.56	0\\
56.57	0\\
56.58	0\\
56.59	0\\
56.6	0\\
56.61	0\\
56.62	0\\
56.63	0\\
56.64	0\\
56.65	0\\
56.66	0\\
56.67	0\\
56.68	0\\
56.69	0\\
56.7	0\\
56.71	0\\
56.72	0\\
56.73	0\\
56.74	0\\
56.75	0\\
56.76	0\\
56.77	0\\
56.78	0\\
56.79	0\\
56.8	0\\
56.81	0\\
56.82	0\\
56.83	0\\
56.84	0\\
56.85	0\\
56.86	0\\
56.87	0\\
56.88	0\\
56.89	0\\
56.9	0\\
56.91	0\\
56.92	0\\
56.93	0\\
56.94	0\\
56.95	0\\
56.96	0\\
56.97	0\\
56.98	0\\
56.99	0\\
57	0\\
57.01	0\\
57.02	0\\
57.03	0\\
57.04	0\\
57.05	0\\
57.06	0\\
57.07	0\\
57.08	0\\
57.09	0\\
57.1	0\\
57.11	0\\
57.12	0\\
57.13	0\\
57.14	0\\
57.15	0\\
57.16	0\\
57.17	0\\
57.18	0\\
57.19	0\\
57.2	0\\
57.21	0\\
57.22	0\\
57.23	0\\
57.24	0\\
57.25	0\\
57.26	0\\
57.27	0\\
57.28	0\\
57.29	0\\
57.3	0\\
57.31	0\\
57.32	0\\
57.33	0\\
57.34	0\\
57.35	0\\
57.36	0\\
57.37	0\\
57.38	0\\
57.39	0\\
57.4	0\\
57.41	0\\
57.42	0\\
57.43	0\\
57.44	0\\
57.45	0\\
57.46	0\\
57.47	0\\
57.48	0\\
57.49	0\\
57.5	0\\
57.51	0\\
57.52	0\\
57.53	0\\
57.54	0\\
57.55	0\\
57.56	0\\
57.57	0\\
57.58	0\\
57.59	0\\
57.6	0\\
57.61	0\\
57.62	0\\
57.63	0\\
57.64	0\\
57.65	0\\
57.66	0\\
57.67	0\\
57.68	0\\
57.69	0\\
57.7	0\\
57.71	0\\
57.72	0\\
57.73	0\\
57.74	0\\
57.75	0\\
57.76	0\\
57.77	0\\
57.78	0\\
57.79	0\\
57.8	0\\
57.81	0\\
57.82	0\\
57.83	0\\
57.84	0\\
57.85	0\\
57.86	0\\
57.87	0\\
57.88	0\\
57.89	0\\
57.9	0\\
57.91	0\\
57.92	0\\
57.93	0\\
57.94	0\\
57.95	0\\
57.96	0\\
57.97	0\\
57.98	0\\
57.99	0\\
58	0\\
58.01	0\\
58.02	0\\
58.03	0\\
58.04	0\\
58.05	0\\
58.06	0\\
58.07	0\\
58.08	0\\
58.09	0\\
58.1	0\\
58.11	0\\
58.12	0\\
58.13	0\\
58.14	0\\
58.15	0\\
58.16	0\\
58.17	0\\
58.18	0\\
58.19	0\\
58.2	0\\
58.21	0\\
58.22	0\\
58.23	0\\
58.24	0\\
58.25	0\\
58.26	0\\
58.27	0\\
58.28	0\\
58.29	0\\
58.3	0\\
58.31	0\\
58.32	0\\
58.33	0\\
58.34	0\\
58.35	0\\
58.36	0\\
58.37	0\\
58.38	0\\
58.39	0\\
58.4	0\\
58.41	0\\
58.42	0\\
58.43	0\\
58.44	0\\
58.45	0\\
58.46	0\\
58.47	0\\
58.48	0\\
58.49	0\\
58.5	0\\
58.51	0\\
58.52	0\\
58.53	0\\
58.54	0\\
58.55	0\\
58.56	0\\
58.57	0\\
58.58	0\\
58.59	0\\
58.6	0\\
58.61	0\\
58.62	0\\
58.63	0\\
58.64	0\\
58.65	0\\
58.66	0\\
58.67	0\\
58.68	0\\
58.69	0\\
58.7	0\\
58.71	0\\
58.72	0\\
58.73	0\\
58.74	0\\
58.75	0\\
58.76	0\\
58.77	0\\
58.78	0\\
58.79	0\\
58.8	0\\
58.81	0\\
58.82	0\\
58.83	0\\
58.84	0\\
58.85	0\\
58.86	0\\
58.87	0\\
58.88	0\\
58.89	0\\
58.9	0\\
58.91	0\\
58.92	0\\
58.93	0\\
58.94	0\\
58.95	0\\
58.96	0\\
58.97	0\\
58.98	0\\
58.99	0\\
59	0\\
59.01	0\\
59.02	0\\
59.03	0\\
59.04	0\\
59.05	0\\
59.06	0\\
59.07	0\\
59.08	0\\
59.09	0\\
59.1	0\\
59.11	0\\
59.12	0\\
59.13	0\\
59.14	0\\
59.15	0\\
59.16	0\\
59.17	0\\
59.18	0\\
59.19	0\\
59.2	0\\
59.21	0\\
59.22	0\\
59.23	0\\
59.24	0\\
59.25	0\\
59.26	0\\
59.27	0\\
59.28	0\\
59.29	0\\
59.3	0\\
59.31	0\\
59.32	0\\
59.33	0\\
59.34	0\\
59.35	0\\
59.36	0\\
59.37	0\\
59.38	0\\
59.39	0\\
59.4	0\\
59.41	0\\
59.42	0\\
59.43	0\\
59.44	0\\
59.45	0\\
59.46	0\\
59.47	0\\
59.48	0\\
59.49	0\\
59.5	0\\
59.51	0\\
59.52	0\\
59.53	0\\
59.54	0\\
59.55	0\\
59.56	0\\
59.57	0\\
59.58	0\\
59.59	0\\
59.6	0\\
59.61	0\\
59.62	0\\
59.63	0\\
59.64	0\\
59.65	0\\
59.66	0\\
59.67	0\\
59.68	0\\
59.69	0\\
59.7	0\\
59.71	0\\
59.72	0\\
59.73	0\\
59.74	0\\
59.75	0\\
59.76	0\\
59.77	0\\
59.78	0\\
59.79	0\\
59.8	0\\
59.81	0\\
59.82	0\\
59.83	0\\
59.84	0\\
59.85	0\\
59.86	0\\
59.87	0\\
59.88	0\\
59.89	0\\
59.9	0\\
59.91	0\\
59.92	0\\
59.93	0\\
59.94	0\\
59.95	0\\
59.96	0\\
59.97	0\\
59.98	0\\
59.99	0\\
60	0\\
60.01	0\\
60.02	0\\
60.03	0\\
60.04	0\\
60.05	0\\
60.06	0\\
60.07	0\\
60.08	0\\
60.09	0\\
60.1	0\\
60.11	0\\
60.12	0\\
60.13	0\\
60.14	0\\
60.15	0\\
60.16	0\\
60.17	0\\
60.18	0\\
60.19	0\\
60.2	0\\
60.21	0\\
60.22	0\\
60.23	0\\
60.24	0\\
60.25	0\\
60.26	0\\
60.27	0\\
60.28	0\\
60.29	0\\
60.3	0\\
60.31	0\\
60.32	0\\
60.33	0\\
60.34	0\\
60.35	0\\
60.36	0\\
60.37	0\\
60.38	0\\
60.39	0\\
60.4	0\\
60.41	0\\
60.42	0\\
60.43	0\\
60.44	0\\
60.45	0\\
60.46	0\\
60.47	0\\
60.48	0\\
60.49	0\\
60.5	0\\
60.51	0\\
60.52	0\\
60.53	0\\
60.54	0\\
60.55	0\\
60.56	0\\
60.57	0\\
60.58	0\\
60.59	0\\
60.6	0\\
60.61	0\\
60.62	0\\
60.63	0\\
60.64	0\\
60.65	0\\
60.66	0\\
60.67	0\\
60.68	0\\
60.69	0\\
60.7	0\\
60.71	0\\
60.72	0\\
60.73	0\\
60.74	0\\
60.75	0\\
60.76	0\\
60.77	0\\
60.78	0\\
60.79	0\\
60.8	0\\
60.81	0\\
60.82	0\\
60.83	0\\
60.84	0\\
60.85	0\\
60.86	0\\
60.87	0\\
60.88	0\\
60.89	0\\
60.9	0\\
60.91	0\\
60.92	0\\
60.93	0\\
60.94	0\\
60.95	0\\
60.96	0\\
60.97	0\\
60.98	0\\
60.99	0\\
61	0\\
61.01	0\\
61.02	0\\
61.03	0\\
61.04	0\\
61.05	0\\
61.06	0\\
61.07	0\\
61.08	0\\
61.09	0\\
61.1	0\\
61.11	0\\
61.12	0\\
61.13	0\\
61.14	0\\
61.15	0\\
61.16	0\\
61.17	0\\
61.18	0\\
61.19	0\\
61.2	0\\
61.21	0\\
61.22	0\\
61.23	0\\
61.24	0\\
61.25	0\\
61.26	0\\
61.27	0\\
61.28	0\\
61.29	0\\
61.3	0\\
61.31	0\\
61.32	0\\
61.33	0\\
61.34	0\\
61.35	0\\
61.36	0\\
61.37	0\\
61.38	0\\
61.39	0\\
61.4	0\\
61.41	0\\
61.42	0\\
61.43	0\\
61.44	0\\
61.45	0\\
61.46	0\\
61.47	0\\
61.48	0\\
61.49	0\\
61.5	0\\
61.51	0\\
61.52	0\\
61.53	0\\
61.54	0\\
61.55	0\\
61.56	0\\
61.57	0\\
61.58	0\\
61.59	0\\
61.6	0\\
61.61	0\\
61.62	0\\
61.63	0\\
61.64	0\\
61.65	0\\
61.66	0\\
61.67	0\\
61.68	0\\
61.69	0\\
61.7	0\\
61.71	0\\
61.72	0\\
61.73	0\\
61.74	0\\
61.75	0\\
61.76	0\\
61.77	0\\
61.78	0\\
61.79	0\\
61.8	0\\
61.81	0\\
61.82	0\\
61.83	0\\
61.84	0\\
61.85	0\\
61.86	0\\
61.87	0\\
61.88	0\\
61.89	0\\
61.9	0\\
61.91	0\\
61.92	0\\
61.93	0\\
61.94	0\\
61.95	0\\
61.96	0\\
61.97	0\\
61.98	0\\
61.99	0\\
62	0\\
62.01	0\\
62.02	0\\
62.03	0\\
62.04	0\\
62.05	0\\
62.06	0\\
62.07	0\\
62.08	0\\
62.09	0\\
62.1	0\\
62.11	0\\
62.12	0\\
62.13	0\\
62.14	0\\
62.15	0\\
62.16	0\\
62.17	0\\
62.18	0\\
62.19	0\\
62.2	0\\
62.21	0\\
62.22	0\\
62.23	0\\
62.24	0\\
62.25	0\\
62.26	0\\
62.27	0\\
62.28	0\\
62.29	0\\
62.3	0\\
62.31	0\\
62.32	0\\
62.33	0\\
62.34	0\\
62.35	0\\
62.36	0\\
62.37	0\\
62.38	0\\
62.39	0\\
62.4	0\\
62.41	0\\
62.42	0\\
62.43	0\\
62.44	0\\
62.45	0\\
62.46	0\\
62.47	0\\
62.48	0\\
62.49	0\\
62.5	0\\
62.51	0\\
62.52	0\\
62.53	0\\
62.54	0\\
62.55	0\\
62.56	0\\
62.57	0\\
62.58	0\\
62.59	0\\
62.6	0\\
62.61	0\\
62.62	0\\
62.63	0\\
62.64	0\\
62.65	0\\
62.66	0\\
62.67	0\\
62.68	0\\
62.69	0\\
62.7	0\\
62.71	0\\
62.72	0\\
62.73	0\\
62.74	0\\
62.75	0\\
62.76	0\\
62.77	0\\
62.78	0\\
62.79	0\\
62.8	0\\
62.81	0\\
62.82	0\\
62.83	0\\
62.84	0\\
62.85	0\\
62.86	0\\
62.87	0\\
62.88	0\\
62.89	0\\
62.9	0\\
62.91	0\\
62.92	0\\
62.93	0\\
62.94	0\\
62.95	0\\
62.96	0\\
62.97	0\\
62.98	0\\
62.99	0\\
63	0\\
63.01	0\\
63.02	0\\
63.03	0\\
63.04	0\\
63.05	0\\
63.06	0\\
63.07	0\\
63.08	0\\
63.09	0\\
63.1	0\\
63.11	0\\
63.12	0\\
63.13	0\\
63.14	0\\
63.15	0\\
63.16	0\\
63.17	0\\
63.18	0\\
63.19	0\\
63.2	0\\
63.21	0\\
63.22	0\\
63.23	0\\
63.24	0\\
63.25	0\\
63.26	0\\
63.27	0\\
63.28	0\\
63.29	0\\
63.3	0\\
63.31	0\\
63.32	0\\
63.33	0\\
63.34	0\\
63.35	0\\
63.36	0\\
63.37	0\\
63.38	0\\
63.39	0\\
63.4	0\\
63.41	0\\
63.42	0\\
63.43	0\\
63.44	0\\
63.45	0\\
63.46	0\\
63.47	0\\
63.48	0\\
63.49	0\\
63.5	0\\
63.51	0\\
63.52	0\\
63.53	0\\
63.54	0\\
63.55	0\\
63.56	0\\
63.57	0\\
63.58	0\\
63.59	0\\
63.6	0\\
63.61	0\\
63.62	0\\
63.63	0\\
63.64	0\\
63.65	0\\
63.66	0\\
63.67	0\\
63.68	0\\
63.69	0\\
63.7	0\\
63.71	0\\
63.72	0\\
63.73	0\\
63.74	0\\
63.75	0\\
63.76	0\\
63.77	0\\
63.78	0\\
63.79	0\\
63.8	0\\
63.81	0\\
63.82	0\\
63.83	0\\
63.84	0\\
63.85	0\\
63.86	0\\
63.87	0\\
63.88	0\\
63.89	0\\
63.9	0\\
63.91	0\\
63.92	0\\
63.93	0\\
63.94	0\\
63.95	0\\
63.96	0\\
63.97	0\\
63.98	0\\
63.99	0\\
64	0\\
64.01	0\\
64.02	0\\
64.03	0\\
64.04	0\\
64.05	0\\
64.06	0\\
64.07	0\\
64.08	0\\
64.09	0\\
64.1	0\\
64.11	0\\
64.12	0\\
64.13	0\\
64.14	0\\
64.15	0\\
64.16	0\\
64.17	0\\
64.18	0\\
64.19	0\\
64.2	0\\
64.21	0\\
64.22	0\\
64.23	0\\
64.24	0\\
64.25	0\\
64.26	0\\
64.27	0\\
64.28	0\\
64.29	0\\
64.3	0\\
64.31	0\\
64.32	0\\
64.33	0\\
64.34	0\\
64.35	0\\
64.36	0\\
64.37	0\\
64.38	0\\
64.39	0\\
64.4	0\\
64.41	0\\
64.42	0\\
64.43	0\\
64.44	0\\
64.45	0\\
64.46	0\\
64.47	0\\
64.48	0\\
64.49	0\\
64.5	0\\
64.51	0\\
64.52	0\\
64.53	0\\
64.54	0\\
64.55	0\\
64.56	0\\
64.57	0\\
64.58	0\\
64.59	0\\
64.6	0\\
64.61	0\\
64.62	0\\
64.63	0\\
64.64	0\\
64.65	0\\
64.66	0\\
64.67	0\\
64.68	0\\
64.69	0\\
64.7	0\\
64.71	0\\
64.72	0\\
64.73	0\\
64.74	0\\
64.75	0\\
64.76	0\\
64.77	0\\
64.78	0\\
64.79	0\\
64.8	0\\
64.81	0\\
64.82	0\\
64.83	0\\
64.84	0\\
64.85	0\\
64.86	0\\
64.87	0\\
64.88	0\\
64.89	0\\
64.9	0\\
64.91	0\\
64.92	0\\
64.93	0\\
64.94	0\\
64.95	0\\
64.96	0\\
64.97	0\\
64.98	0\\
64.99	0\\
65	0\\
65.01	0\\
65.02	0\\
65.03	0\\
65.04	0\\
65.05	0\\
65.06	0\\
65.07	0\\
65.08	0\\
65.09	0\\
65.1	0\\
65.11	0\\
65.12	0\\
65.13	0\\
65.14	0\\
65.15	0\\
65.16	0\\
65.17	0\\
65.18	0\\
65.19	0\\
65.2	0\\
65.21	0\\
65.22	0\\
65.23	0\\
65.24	0\\
65.25	0\\
65.26	0\\
65.27	0\\
65.28	0\\
65.29	0\\
65.3	0\\
65.31	0\\
65.32	0\\
65.33	0\\
65.34	0\\
65.35	0\\
65.36	0\\
65.37	0\\
65.38	0\\
65.39	0\\
65.4	0\\
65.41	0\\
65.42	0\\
65.43	0\\
65.44	0\\
65.45	0\\
65.46	0\\
65.47	0\\
65.48	0\\
65.49	0\\
65.5	0\\
65.51	0\\
65.52	0\\
65.53	0\\
65.54	0\\
65.55	0\\
65.56	0\\
65.57	0\\
65.58	0\\
65.59	0\\
65.6	0\\
65.61	0\\
65.62	0\\
65.63	0\\
65.64	0\\
65.65	0\\
65.66	0\\
65.67	0\\
65.68	0\\
65.69	0\\
65.7	0\\
65.71	0\\
65.72	0\\
65.73	0\\
65.74	0\\
65.75	0\\
65.76	0\\
65.77	0\\
65.78	0\\
65.79	0\\
65.8	0\\
65.81	0\\
65.82	0\\
65.83	0\\
65.84	0\\
65.85	0\\
65.86	0\\
65.87	0\\
65.88	0\\
65.89	0\\
65.9	0\\
65.91	0\\
65.92	0\\
65.93	0\\
65.94	0\\
65.95	0\\
65.96	0\\
65.97	0\\
65.98	0\\
65.99	0\\
66	0\\
66.01	0\\
66.02	0\\
66.03	0\\
66.04	0\\
66.05	0\\
66.06	0\\
66.07	0\\
66.08	0\\
66.09	0\\
66.1	0\\
66.11	0\\
66.12	0\\
66.13	0\\
66.14	0\\
66.15	0\\
66.16	0\\
66.17	0\\
66.18	0\\
66.19	0\\
66.2	0\\
66.21	0\\
66.22	0\\
66.23	0\\
66.24	0\\
66.25	0\\
66.26	0\\
66.27	0\\
66.28	0\\
66.29	0\\
66.3	0\\
66.31	0\\
66.32	0\\
66.33	0\\
66.34	0\\
66.35	0\\
66.36	0\\
66.37	0\\
66.38	0\\
66.39	0\\
66.4	0\\
66.41	0\\
66.42	0\\
66.43	0\\
66.44	0\\
66.45	0\\
66.46	0\\
66.47	0\\
66.48	0\\
66.49	0\\
66.5	0\\
66.51	0\\
66.52	0\\
66.53	0\\
66.54	0\\
66.55	0\\
66.56	0\\
66.57	0\\
66.58	0\\
66.59	0\\
66.6	0\\
66.61	0\\
66.62	0\\
66.63	0\\
66.64	0\\
66.65	0\\
66.66	0\\
66.67	0\\
66.68	0\\
66.69	0\\
66.7	0\\
66.71	0\\
66.72	0\\
66.73	0\\
66.74	0\\
66.75	0\\
66.76	0\\
66.77	0\\
66.78	0\\
66.79	0\\
66.8	0\\
66.81	0\\
66.82	0\\
66.83	0\\
66.84	0\\
66.85	0\\
66.86	0\\
66.87	0\\
66.88	0\\
66.89	0\\
66.9	0\\
66.91	0\\
66.92	0\\
66.93	0\\
66.94	0\\
66.95	0\\
66.96	0\\
66.97	0\\
66.98	0\\
66.99	0\\
67	0\\
67.01	0\\
67.02	0\\
67.03	0\\
67.04	0\\
67.05	0\\
67.06	0\\
67.07	0\\
67.08	0\\
67.09	0\\
67.1	0\\
67.11	0\\
67.12	0\\
67.13	0\\
67.14	0\\
67.15	0\\
67.16	0\\
67.17	0\\
67.18	0\\
67.19	0\\
67.2	0\\
67.21	0\\
67.22	0\\
67.23	0\\
67.24	0\\
67.25	0\\
67.26	0\\
67.27	0\\
67.28	0\\
67.29	0\\
67.3	0\\
67.31	0\\
67.32	0\\
67.33	0\\
67.34	0\\
67.35	0\\
67.36	0\\
67.37	0\\
67.38	0\\
67.39	0\\
67.4	0\\
67.41	0\\
67.42	0\\
67.43	0\\
67.44	0\\
67.45	0\\
67.46	0\\
67.47	0\\
67.48	0\\
67.49	0\\
67.5	0\\
67.51	0\\
67.52	0\\
67.53	0\\
67.54	0\\
67.55	0\\
67.56	0\\
67.57	0\\
67.58	0\\
67.59	0\\
67.6	0\\
67.61	0\\
67.62	0\\
67.63	0\\
67.64	0\\
67.65	0\\
67.66	0\\
67.67	0\\
67.68	0\\
67.69	0\\
67.7	0\\
67.71	0\\
67.72	0\\
67.73	0\\
67.74	0\\
67.75	0\\
67.76	0\\
67.77	0\\
67.78	0\\
67.79	0\\
67.8	0\\
67.81	0\\
67.82	0\\
67.83	0\\
67.84	0\\
67.85	0\\
67.86	0\\
67.87	0\\
67.88	0\\
67.89	0\\
67.9	0\\
67.91	0\\
67.92	0\\
67.93	0\\
67.94	0\\
67.95	0\\
67.96	0\\
67.97	0\\
67.98	0\\
67.99	0\\
68	0\\
68.01	0\\
68.02	0\\
68.03	0\\
68.04	0\\
68.05	0\\
68.06	0\\
68.07	0\\
68.08	0\\
68.09	0\\
68.1	0\\
68.11	0\\
68.12	0\\
68.13	0\\
68.14	0\\
68.15	0\\
68.16	0\\
68.17	0\\
68.18	0\\
68.19	0\\
68.2	0\\
68.21	0\\
68.22	0\\
68.23	0\\
68.24	0\\
68.25	0\\
68.26	0\\
68.27	0\\
68.28	0\\
68.29	0\\
68.3	0\\
68.31	0\\
68.32	0\\
68.33	0\\
68.34	0\\
68.35	0\\
68.36	0\\
68.37	0\\
68.38	0\\
68.39	0\\
68.4	0\\
68.41	0\\
68.42	0\\
68.43	0\\
68.44	0\\
68.45	0\\
68.46	0\\
68.47	0\\
68.48	0\\
68.49	0\\
68.5	0\\
68.51	0\\
68.52	0\\
68.53	0\\
68.54	0\\
68.55	0\\
68.56	0\\
68.57	0\\
68.58	0\\
68.59	0\\
68.6	0\\
68.61	0\\
68.62	0\\
68.63	0\\
68.64	0\\
68.65	0\\
68.66	0\\
68.67	0\\
68.68	0\\
68.69	0\\
68.7	0\\
68.71	0\\
68.72	0\\
68.73	0\\
68.74	0\\
68.75	0\\
68.76	0\\
68.77	0\\
68.78	0\\
68.79	0\\
68.8	0\\
68.81	0\\
68.82	0\\
68.83	0\\
68.84	0\\
68.85	0\\
68.86	0\\
68.87	0\\
68.88	0\\
68.89	0\\
68.9	0\\
68.91	0\\
68.92	0\\
68.93	0\\
68.94	0\\
68.95	0\\
68.96	0\\
68.97	0\\
68.98	0\\
68.99	0\\
69	0\\
69.01	0\\
69.02	0\\
69.03	0\\
69.04	0\\
69.05	0\\
69.06	0\\
69.07	0\\
69.08	0\\
69.09	0\\
69.1	0\\
69.11	0\\
69.12	0\\
69.13	0\\
69.14	0\\
69.15	0\\
69.16	0\\
69.17	0\\
69.18	0\\
69.19	0\\
69.2	0\\
69.21	0\\
69.22	0\\
69.23	0\\
69.24	0\\
69.25	0\\
69.26	0\\
69.27	0\\
69.28	0\\
69.29	0\\
69.3	0\\
69.31	0\\
69.32	0\\
69.33	0\\
69.34	0\\
69.35	0\\
69.36	0\\
69.37	0\\
69.38	0\\
69.39	0\\
69.4	0\\
69.41	0\\
69.42	0\\
69.43	0\\
69.44	0\\
69.45	0\\
69.46	0\\
69.47	0\\
69.48	0\\
69.49	0\\
69.5	0\\
69.51	0\\
69.52	0\\
69.53	0\\
69.54	0\\
69.55	0\\
69.56	0\\
69.57	0\\
69.58	0\\
69.59	0\\
69.6	0\\
69.61	0\\
69.62	0\\
69.63	0\\
69.64	0\\
69.65	0\\
69.66	0\\
69.67	0\\
69.68	0\\
69.69	0\\
69.7	0\\
69.71	0\\
69.72	0\\
69.73	0\\
69.74	0\\
69.75	0\\
69.76	0\\
69.77	0\\
69.78	0\\
69.79	0\\
69.8	0\\
69.81	0\\
69.82	0\\
69.83	0\\
69.84	0\\
69.85	0\\
69.86	0\\
69.87	0\\
69.88	0\\
69.89	0\\
69.9	0\\
69.91	0\\
69.92	0\\
69.93	0\\
69.94	0\\
69.95	0\\
69.96	0\\
69.97	0\\
69.98	0\\
69.99	0\\
70	0\\
70.01	0\\
70.02	0\\
70.03	0\\
70.04	0\\
70.05	0\\
70.06	0\\
70.07	0\\
70.08	0\\
70.09	0\\
70.1	0\\
70.11	0\\
70.12	0\\
70.13	0\\
70.14	0\\
70.15	0\\
70.16	0\\
70.17	0\\
70.18	0\\
70.19	0\\
70.2	0\\
70.21	0\\
70.22	0\\
70.23	0\\
70.24	0\\
70.25	0\\
70.26	0\\
70.27	0\\
70.28	0\\
70.29	0\\
70.3	0\\
70.31	0\\
70.32	0\\
70.33	0\\
70.34	0\\
70.35	0\\
70.36	0\\
70.37	0\\
70.38	0\\
70.39	0\\
70.4	0\\
70.41	0\\
70.42	0\\
70.43	0\\
70.44	0\\
70.45	0\\
70.46	0\\
70.47	0\\
70.48	0\\
70.49	0\\
70.5	0\\
70.51	0\\
70.52	0\\
70.53	0\\
70.54	0\\
70.55	0\\
70.56	0\\
70.57	0\\
70.58	0\\
70.59	0\\
70.6	0\\
70.61	0\\
70.62	0\\
70.63	0\\
70.64	0\\
70.65	0\\
70.66	0\\
70.67	0\\
70.68	0\\
70.69	0\\
70.7	0\\
70.71	0\\
70.72	0\\
70.73	0\\
70.74	0\\
70.75	0\\
70.76	0\\
70.77	0\\
70.78	0\\
70.79	0\\
70.8	0\\
70.81	0\\
70.82	0\\
70.83	0\\
70.84	0\\
70.85	0\\
70.86	0\\
70.87	0\\
70.88	0\\
70.89	0\\
70.9	0\\
70.91	0\\
70.92	0\\
70.93	0\\
70.94	0\\
70.95	0\\
70.96	0\\
70.97	0\\
70.98	0\\
70.99	0\\
71	0\\
71.01	0\\
71.02	0\\
71.03	0\\
71.04	0\\
71.05	0\\
71.06	0\\
71.07	0\\
71.08	0\\
71.09	0\\
71.1	0\\
71.11	0\\
71.12	0\\
71.13	0\\
71.14	0\\
71.15	0\\
71.16	0\\
71.17	0\\
71.18	0\\
71.19	0\\
71.2	0\\
71.21	0\\
71.22	0\\
71.23	0\\
71.24	0\\
71.25	0\\
71.26	0\\
71.27	0\\
71.28	0\\
71.29	0\\
71.3	0\\
71.31	0\\
71.32	0\\
71.33	0\\
71.34	0\\
71.35	0\\
71.36	0\\
71.37	0\\
71.38	0\\
71.39	0\\
71.4	0\\
71.41	0\\
71.42	0\\
71.43	0\\
71.44	0\\
71.45	0\\
71.46	0\\
71.47	0\\
71.48	0\\
71.49	0\\
71.5	0\\
71.51	0\\
71.52	0\\
71.53	0\\
71.54	0\\
71.55	0\\
71.56	0\\
71.57	0\\
71.58	0\\
71.59	0\\
71.6	0\\
71.61	0\\
71.62	0\\
71.63	0\\
71.64	0\\
71.65	0\\
71.66	0\\
71.67	0\\
71.68	0\\
71.69	0\\
71.7	0\\
71.71	0\\
71.72	0\\
71.73	0\\
71.74	0\\
71.75	0\\
71.76	0\\
71.77	0\\
71.78	0\\
71.79	0\\
71.8	0\\
71.81	0\\
71.82	0\\
71.83	0\\
71.84	0\\
71.85	0\\
71.86	0\\
71.87	0\\
71.88	0\\
71.89	0\\
71.9	0\\
71.91	0\\
71.92	0\\
71.93	0\\
71.94	0\\
71.95	0\\
71.96	0\\
71.97	0\\
71.98	0\\
71.99	0\\
72	0\\
72.01	0\\
72.02	0\\
72.03	0\\
72.04	0\\
72.05	0\\
72.06	0\\
72.07	0\\
72.08	0\\
72.09	0\\
72.1	0\\
72.11	0\\
72.12	0\\
72.13	0\\
72.14	0\\
72.15	0\\
72.16	0\\
72.17	0\\
72.18	0\\
72.19	0\\
72.2	0\\
72.21	0\\
72.22	0\\
72.23	0\\
72.24	0\\
72.25	0\\
72.26	0\\
72.27	0\\
72.28	0\\
72.29	0\\
72.3	0\\
72.31	0\\
72.32	0\\
72.33	0\\
72.34	0\\
72.35	0\\
72.36	0\\
72.37	0\\
72.38	0\\
72.39	0\\
72.4	0\\
72.41	0\\
72.42	0\\
72.43	0\\
72.44	0\\
72.45	0\\
72.46	0\\
72.47	0\\
72.48	0\\
72.49	0\\
72.5	0\\
72.51	0\\
72.52	0\\
72.53	0\\
72.54	0\\
72.55	0\\
72.56	0\\
72.57	0\\
72.58	0\\
72.59	0\\
72.6	0\\
72.61	0\\
72.62	0\\
72.63	0\\
72.64	0\\
72.65	0\\
72.66	0\\
72.67	0\\
72.68	0\\
72.69	0\\
72.7	0\\
72.71	0\\
72.72	0\\
72.73	0\\
72.74	0\\
72.75	0\\
72.76	0\\
72.77	0\\
72.78	0\\
72.79	0\\
72.8	0\\
72.81	0\\
72.82	0\\
72.83	0\\
72.84	0\\
72.85	0\\
72.86	0\\
72.87	0\\
72.88	0\\
72.89	0\\
72.9	0\\
72.91	0\\
72.92	0\\
72.93	0\\
72.94	0\\
72.95	0\\
72.96	0\\
72.97	0\\
72.98	0\\
72.99	0\\
73	0\\
73.01	0\\
73.02	0\\
73.03	0\\
73.04	0\\
73.05	0\\
73.06	0\\
73.07	0\\
73.08	0\\
73.09	0\\
73.1	0\\
73.11	0\\
73.12	0\\
73.13	0\\
73.14	0\\
73.15	0\\
73.16	0\\
73.17	0\\
73.18	0\\
73.19	0\\
73.2	0\\
73.21	0\\
73.22	0\\
73.23	0\\
73.24	0\\
73.25	0\\
73.26	0\\
73.27	0\\
73.28	0\\
73.29	0\\
73.3	0\\
73.31	0\\
73.32	0\\
73.33	0\\
73.34	0\\
73.35	0\\
73.36	0\\
73.37	0\\
73.38	0\\
73.39	0\\
73.4	0\\
73.41	0\\
73.42	0\\
73.43	0\\
73.44	0\\
73.45	0\\
73.46	0\\
73.47	0\\
73.48	0\\
73.49	0\\
73.5	0\\
73.51	0\\
73.52	0\\
73.53	0\\
73.54	0\\
73.55	0\\
73.56	0\\
73.57	0\\
73.58	0\\
73.59	0\\
73.6	0\\
73.61	0\\
73.62	0\\
73.63	0\\
73.64	0\\
73.65	0\\
73.66	0\\
73.67	0\\
73.68	0\\
73.69	0\\
73.7	0\\
73.71	0\\
73.72	0\\
73.73	0\\
73.74	0\\
73.75	0\\
73.76	0\\
73.77	0\\
73.78	0\\
73.79	0\\
73.8	0\\
73.81	0\\
73.82	0\\
73.83	0\\
73.84	0\\
73.85	0\\
73.86	0\\
73.87	0\\
73.88	0\\
73.89	0\\
73.9	0\\
73.91	0\\
73.92	0\\
73.93	0\\
73.94	0\\
73.95	0\\
73.96	0\\
73.97	0\\
73.98	0\\
73.99	0\\
74	0\\
74.01	0\\
74.02	0\\
74.03	0\\
74.04	0\\
74.05	0\\
74.06	0\\
74.07	0\\
74.08	0\\
74.09	0\\
74.1	0\\
74.11	0\\
74.12	0\\
74.13	0\\
74.14	0\\
74.15	0\\
74.16	0\\
74.17	0\\
74.18	0\\
74.19	0\\
74.2	0\\
74.21	0\\
74.22	0\\
74.23	0\\
74.24	0\\
74.25	0\\
74.26	0\\
74.27	0\\
74.28	0\\
74.29	0\\
74.3	0\\
74.31	0\\
74.32	0\\
74.33	0\\
74.34	0\\
74.35	0\\
74.36	0\\
74.37	0\\
74.38	0\\
74.39	0\\
74.4	0\\
74.41	0\\
74.42	0\\
74.43	0\\
74.44	0\\
74.45	0\\
74.46	0\\
74.47	0\\
74.48	0\\
74.49	0\\
74.5	0\\
74.51	0\\
74.52	0\\
74.53	0\\
74.54	0\\
74.55	0\\
74.56	0\\
74.57	0\\
74.58	0\\
74.59	0\\
74.6	0\\
74.61	0\\
74.62	0\\
74.63	0\\
74.64	0\\
74.65	0\\
74.66	0\\
74.67	0\\
74.68	0\\
74.69	0\\
74.7	0\\
74.71	0\\
74.72	0\\
74.73	0\\
74.74	0\\
74.75	0\\
74.76	0\\
74.77	0\\
74.78	0\\
74.79	0\\
74.8	0\\
74.81	0\\
74.82	0\\
74.83	0\\
74.84	0\\
74.85	0\\
74.86	0\\
74.87	0\\
74.88	0\\
74.89	0\\
74.9	0\\
74.91	0\\
74.92	0\\
74.93	0\\
74.94	0\\
74.95	0\\
74.96	0\\
74.97	0\\
74.98	0\\
74.99	0\\
75	0\\
75.01	0\\
75.02	0\\
75.03	0\\
75.04	0\\
75.05	0\\
75.06	0\\
75.07	0\\
75.08	0\\
75.09	0\\
75.1	0\\
75.11	0\\
75.12	0\\
75.13	0\\
75.14	0\\
75.15	0\\
75.16	0\\
75.17	0\\
75.18	0\\
75.19	0\\
75.2	0\\
75.21	0\\
75.22	0\\
75.23	0\\
75.24	0\\
75.25	0\\
75.26	0\\
75.27	0\\
75.28	0\\
75.29	0\\
75.3	0\\
75.31	0\\
75.32	0\\
75.33	0\\
75.34	0\\
75.35	0\\
75.36	0\\
75.37	0\\
75.38	0\\
75.39	0\\
75.4	0\\
75.41	0\\
75.42	0\\
75.43	0\\
75.44	0\\
75.45	0\\
75.46	0\\
75.47	0\\
75.48	0\\
75.49	0\\
75.5	0\\
75.51	0\\
75.52	1.83244008483338e-06\\
75.53	4.95691426126779e-06\\
75.54	8.08391395710692e-06\\
75.55	1.1213442883462e-05\\
75.56	1.43455047594103e-05\\
75.57	1.74801033120328e-05\\
75.58	2.06172422764286e-05\\
75.59	2.3756925395739e-05\\
75.6	2.68991564211681e-05\\
75.61	3.00439391120143e-05\\
75.62	3.31912772356911e-05\\
75.63	3.63411745677444e-05\\
75.64	3.94936348918837e-05\\
75.65	4.26486620000062e-05\\
75.66	4.58062596922146e-05\\
75.67	4.89664317768514e-05\\
75.68	5.21291820705129e-05\\
75.69	5.52945143980771e-05\\
75.7	5.84624325927346e-05\\
75.71	6.16329404960064e-05\\
75.72	6.48060419577712e-05\\
75.73	6.79817408362864e-05\\
75.74	7.11600409982194e-05\\
75.75	7.43409463186684e-05\\
75.76	7.75244606811901e-05\\
75.77	8.07105879778239e-05\\
75.78	8.38993321091198e-05\\
75.79	8.70906969841559e-05\\
75.8	9.0284686520576e-05\\
75.81	9.34813046446076e-05\\
75.82	9.66805552910892e-05\\
75.83	9.98824424034983e-05\\
75.84	0.000103086969933976\\
75.85	0.000106294141843353\\
75.86	0.00010950396210118\\
75.87	0.000112716434685748\\
75.88	0.00011593156358412\\
75.89	0.000119149352792156\\
75.9	0.00012236980631454\\
75.91	0.00012559292816481\\
75.92	0.000128818722365379\\
75.93	0.000132047192947571\\
75.94	0.000135278343951637\\
75.95	0.000138512179426797\\
75.96	0.000141748703431251\\
75.97	0.000144987920032227\\
75.98	0.000148229833305986\\
75.99	0.000151474447337877\\
76	0.000154721766222339\\
76.01	0.000157971794062944\\
76.02	0.000161224534972429\\
76.03	0.000164479993072713\\
76.04	0.000167738172494932\\
76.05	0.000170999077379469\\
76.06	0.000174262711875987\\
76.07	0.000177529080143447\\
76.08	0.000180798186350143\\
76.09	0.000184070034673741\\
76.1	0.000187344629301298\\
76.11	0.000190621974429289\\
76.12	0.000193902074263647\\
76.13	0.000197184933019789\\
76.14	0.000200470554922651\\
76.15	0.000203758944206708\\
76.16	0.000207050105116013\\
76.17	0.000210344041904228\\
76.18	0.000213640758834652\\
76.19	0.000216940260180255\\
76.2	0.000220242550223706\\
76.21	0.000223547633257409\\
76.22	0.000226855513583531\\
76.23	0.000230166195514039\\
76.24	0.000233479683370726\\
76.25	0.000236795981485245\\
76.26	0.000240115094199144\\
76.27	0.000243437025863897\\
76.28	0.000246761780840939\\
76.29	0.000250089363501696\\
76.3	0.000253419778227619\\
76.31	0.000256753029410219\\
76.32	0.000260089121451101\\
76.33	0.000263428058761995\\
76.34	0.000266769845764788\\
76.35	0.000270114486891571\\
76.36	0.000273461986584653\\
76.37	0.000276812349296617\\
76.38	0.000280165579490337\\
76.39	0.000283521681639021\\
76.4	0.000286880660226248\\
76.41	0.000290242519746006\\
76.42	0.000293607264702712\\
76.43	0.000296974899611266\\
76.44	0.000300345428997078\\
76.45	0.000303718857396103\\
76.46	0.000307095189354886\\
76.47	0.000310474429430586\\
76.48	0.000313856582191026\\
76.49	0.000317241652214719\\
76.5	0.000320629644090912\\
76.51	0.000324020562419626\\
76.52	0.00032741441181168\\
76.53	0.000330811196888749\\
76.54	0.000334210922283387\\
76.55	0.000337613592639067\\
76.56	0.000341019212610229\\
76.57	0.000344427786862313\\
76.58	0.000347839320071795\\
76.59	0.000351253816926227\\
76.6	0.000354671282124287\\
76.61	0.000358091720375806\\
76.62	0.000361515136401813\\
76.63	0.000364941534934583\\
76.64	0.000368370920717662\\
76.65	0.000371803298505922\\
76.66	0.000375238673065596\\
76.67	0.000378677049174318\\
76.68	0.00038211843162117\\
76.69	0.000385562825206721\\
76.7	0.000389010234743067\\
76.71	0.000392460665053877\\
76.72	0.000395914120974439\\
76.73	0.000399370607351693\\
76.74	0.000402830129044286\\
76.75	0.000406292690922606\\
76.76	0.000409758297868833\\
76.77	0.000413226954776982\\
76.78	0.000416698666552941\\
76.79	0.000420173438186528\\
76.8	0.000423651274694466\\
76.81	0.000427132181105395\\
76.82	0.000430616162459919\\
76.83	0.000434103223810649\\
76.84	0.000437593370222245\\
76.85	0.000441086606771467\\
76.86	0.000444582938547211\\
76.87	0.000448082370650561\\
76.88	0.00045158490819483\\
76.89	0.000455090556305614\\
76.9	0.00045859932012083\\
76.91	0.000462111204790762\\
76.92	0.000465626215478115\\
76.93	0.000469144357358061\\
76.94	0.000472665635618277\\
76.95	0.00047619005545901\\
76.96	0.000479717622093107\\
76.97	0.000483248340746079\\
76.98	0.000486782216656139\\
76.99	0.000490319255074256\\
77	0.000493859461264203\\
77.01	0.000497402840502613\\
77.02	0.000500949398079025\\
77.03	0.000504499139295931\\
77.04	0.000508052069468828\\
77.05	0.000511608193926285\\
77.06	0.000515167518009969\\
77.07	0.000518730047074718\\
77.08	0.000522295786488589\\
77.09	0.000525864741632905\\
77.1	0.000529436917902314\\
77.11	0.000533012320704841\\
77.12	0.000536590955461946\\
77.13	0.000540172827608574\\
77.14	0.000543757942593211\\
77.15	0.000547346305877942\\
77.16	0.000550937922938508\\
77.17	0.000554532799264359\\
77.18	0.000558130940358718\\
77.19	0.000561732351738623\\
77.2	0.000565337038935007\\
77.21	0.000568945007492733\\
77.22	0.000572556262970678\\
77.23	0.000576170810941766\\
77.24	0.000579788656993051\\
77.25	0.000583409806725763\\
77.26	0.000587034265755369\\
77.27	0.000590662039711646\\
77.28	0.00059429313423873\\
77.29	0.000597927554995187\\
77.3	0.000601565307654071\\
77.31	0.000605206397902985\\
77.32	0.000608850831444154\\
77.33	0.000612498613994483\\
77.34	0.00061614975128562\\
77.35	0.000619804249064028\\
77.36	0.000623462113091045\\
77.37	0.000627123349142954\\
77.38	0.00063078796301105\\
77.39	0.000634455960501708\\
77.4	0.00063812734743645\\
77.41	0.000641802129652012\\
77.42	0.000645480313000418\\
77.43	0.000649161903349051\\
77.44	0.000652846906580715\\
77.45	0.000656535328593714\\
77.46	0.000660227175301929\\
77.47	0.000663922452634871\\
77.48	0.00066762116653778\\
77.49	0.000671323322971682\\
77.5	0.000675028927913462\\
77.51	0.000678737987355953\\
77.52	0.000682450507307995\\
77.53	0.000686166493794528\\
77.54	0.000689885952856659\\
77.55	0.000693608890551745\\
77.56	0.000697335312953466\\
77.57	0.000701065226151903\\
77.58	0.000704798636253636\\
77.59	0.0007085355493818\\
77.6	0.000712275971676182\\
77.61	0.000716019909293302\\
77.62	0.000719767368406492\\
77.63	0.000723518355205979\\
77.64	0.000727272875898976\\
77.65	0.000731030936709762\\
77.66	0.000734792543879769\\
77.67	0.000738557703667674\\
77.68	0.00074232642234948\\
77.69	0.000746098706218605\\
77.7	0.00074987456158598\\
77.71	0.000753653994780133\\
77.72	0.000757437012147276\\
77.73	0.000761223620051404\\
77.74	0.000765013824874389\\
77.75	0.000768807633016066\\
77.76	0.00077260505089434\\
77.77	0.000776406084945266\\
77.78	0.000780210741623162\\
77.79	0.000784019027400697\\
77.8	0.000787830948768987\\
77.81	0.000791646512237704\\
77.82	0.000795465724335167\\
77.83	0.000799288591608455\\
77.84	0.000803115120623497\\
77.85	0.000806945317965182\\
77.86	0.000810779190237466\\
77.87	0.000814616744063471\\
77.88	0.000818457986085597\\
77.89	0.000822302922965625\\
77.9	0.000826151561384832\\
77.91	0.000830003908044094\\
77.92	0.000833859969664008\\
77.93	0.000837719752984988\\
77.94	0.000841583264767392\\
77.95	0.000845450511791631\\
77.96	0.000849321500858282\\
77.97	0.000853196238788215\\
77.98	0.000857074732422698\\
77.99	0.000860956988623523\\
78	0.000864843014273128\\
78.01	0.000868732816274718\\
78.02	0.000872626401552383\\
78.03	0.000876523777051223\\
78.04	0.000880424949737484\\
78.05	0.000884329926598672\\
78.06	0.000888238714643691\\
78.07	0.00089215132086555\\
78.08	0.000896067751923985\\
78.09	0.000899988014494167\\
78.1	0.000903912115266737\\
78.11	0.000907840060947851\\
78.12	0.000911771858259219\\
78.13	0.000915707513938154\\
78.14	0.000919647034737611\\
78.15	0.000923590427426228\\
78.16	0.000927537698788377\\
78.17	0.000931488855624194\\
78.18	0.000935443904749641\\
78.19	0.00093940285299653\\
78.2	0.000943365707212582\\
78.21	0.000947332474261459\\
78.22	0.000951303161022814\\
78.23	0.000955277774392339\\
78.24	0.000959256321281798\\
78.25	0.000963238808619083\\
78.26	0.000967225243348253\\
78.27	0.000971215632429575\\
78.28	0.000975209982839578\\
78.29	0.000979208301571086\\
78.3	0.000983210595633277\\
78.31	0.000987216872051715\\
78.32	0.000991227137868401\\
78.33	0.000995241400141823\\
78.34	0.000999259665946994\\
78.35	0.0010032819423755\\
78.36	0.00100730823653555\\
78.37	0.00101133855555201\\
78.38	0.00101537290656647\\
78.39	0.00101941129673727\\
78.4	0.00102345373323956\\
78.41	0.00102750022326532\\
78.42	0.00103155077402347\\
78.43	0.00103560539273983\\
78.44	0.00103966408665724\\
78.45	0.00104372686303557\\
78.46	0.00104779372915178\\
78.47	0.00105186469229994\\
78.48	0.00105593975979134\\
78.49	0.00106001893895446\\
78.5	0.00106410223713509\\
78.51	0.00106818966169633\\
78.52	0.00107228122001864\\
78.53	0.00107637691949993\\
78.54	0.00108047676755555\\
78.55	0.00108458077161838\\
78.56	0.00108868893913888\\
78.57	0.00109280127758511\\
78.58	0.00109691779444279\\
78.59	0.00110103849721538\\
78.6	0.00110516339342407\\
78.61	0.00110929249060788\\
78.62	0.00111342579632371\\
78.63	0.00111756331814633\\
78.64	0.00112170506366852\\
78.65	0.00112585104050104\\
78.66	0.00113000125627273\\
78.67	0.00113415571863052\\
78.68	0.00113831443523954\\
78.69	0.0011424774137831\\
78.7	0.0011466446619628\\
78.71	0.00115081618749854\\
78.72	0.00115499199812859\\
78.73	0.00115917210160967\\
78.74	0.00116335650571692\\
78.75	0.00116754521824404\\
78.76	0.00117173824700329\\
78.77	0.00117593559982556\\
78.78	0.00118013728456042\\
78.79	0.00118434330907616\\
78.8	0.00118855368125987\\
78.81	0.00119276840901747\\
78.82	0.00119698750027375\\
78.83	0.00120121096297248\\
78.84	0.00120543880507639\\
78.85	0.00120967103456729\\
78.86	0.00121390765944605\\
78.87	0.00121814868773274\\
78.88	0.00122239412746661\\
78.89	0.00122664398670618\\
78.9	0.0012308982735293\\
78.91	0.00123515699603318\\
78.92	0.00123942016233444\\
78.93	0.00124368778056922\\
78.94	0.00124795985889316\\
78.95	0.00125223640548151\\
78.96	0.00125651742852917\\
78.97	0.00126080293625071\\
78.98	0.00126509293688049\\
78.99	0.00126938743867267\\
79	0.00127368644990128\\
79.01	0.00127798997886026\\
79.02	0.00128229803386356\\
79.03	0.00128661062324513\\
79.04	0.00129092775535904\\
79.05	0.00129524943857949\\
79.06	0.00129957568130092\\
79.07	0.00130390649193799\\
79.08	0.0013082418789257\\
79.09	0.00131258185071945\\
79.1	0.00131692641579503\\
79.11	0.00132127558264875\\
79.12	0.00132562935979749\\
79.13	0.00132998775577869\\
79.14	0.00133435077915051\\
79.15	0.00133871843849179\\
79.16	0.00134309074240218\\
79.17	0.00134746769950218\\
79.18	0.00135184931843317\\
79.19	0.00135623560785751\\
79.2	0.00136062657645857\\
79.21	0.00136502223294079\\
79.22	0.00136942258602978\\
79.23	0.00137382764447231\\
79.24	0.00137823741703646\\
79.25	0.00138265191251158\\
79.26	0.00138707113970843\\
79.27	0.00139149510745919\\
79.28	0.00139592382461757\\
79.29	0.00140035730005882\\
79.3	0.00140479554267982\\
79.31	0.00140923856139914\\
79.32	0.00141368636515708\\
79.33	0.00141813896291576\\
79.34	0.00142259636365918\\
79.35	0.00142705857639326\\
79.36	0.00143152561014591\\
79.37	0.0014359974739671\\
79.38	0.00144047417692892\\
79.39	0.00144495572812566\\
79.4	0.00144944213667383\\
79.41	0.00145393341171225\\
79.42	0.00145842956240213\\
79.43	0.00146293059792709\\
79.44	0.00146743652749328\\
79.45	0.00147194736032938\\
79.46	0.00147646310568673\\
79.47	0.00148098377283933\\
79.48	0.00148550937108395\\
79.49	0.0014900399097402\\
79.5	0.00149457539815053\\
79.51	0.00149911584568039\\
79.52	0.00150366126171821\\
79.53	0.00150821165567551\\
79.54	0.00151276703698697\\
79.55	0.00151732741511047\\
79.56	0.00152189279952718\\
79.57	0.00152646319974159\\
79.58	0.00153103862528164\\
79.59	0.00153561908569871\\
79.6	0.00154020459056775\\
79.61	0.00154479514948732\\
79.62	0.00154939077207965\\
79.63	0.00155399146799072\\
79.64	0.00155859724689033\\
79.65	0.00156320811847215\\
79.66	0.00156782409245381\\
79.67	0.00157244517857697\\
79.68	0.00157707138660734\\
79.69	0.00158170272633483\\
79.7	0.00158633920757354\\
79.71	0.00159098084016189\\
79.72	0.00159562763396264\\
79.73	0.00160027959886299\\
79.74	0.00160493674477465\\
79.75	0.00160959908163389\\
79.76	0.00161426661940164\\
79.77	0.00161893936806351\\
79.78	0.00162361733762992\\
79.79	0.00162830053813614\\
79.8	0.00163298897964236\\
79.81	0.00163768267223374\\
79.82	0.00164238162602055\\
79.83	0.00164708585113818\\
79.84	0.00165179535774721\\
79.85	0.00165651015603352\\
79.86	0.00166123025620835\\
79.87	0.00166595566850836\\
79.88	0.00167068640319569\\
79.89	0.00167542247055807\\
79.9	0.00168016388090887\\
79.91	0.00168491064458717\\
79.92	0.00168966277195785\\
79.93	0.00169442027341166\\
79.94	0.00169918315936526\\
79.95	0.00170395144026135\\
79.96	0.0017087251265687\\
79.97	0.00171350422878224\\
79.98	0.00171828875742314\\
79.99	0.00172307872303887\\
80	0.00172787413620331\\
80.01	0.00173267500751677\\
};
\addplot [color=blue,solid]
  table[row sep=crcr]{%
80.01	0.00173267500751677\\
80.02	0.0017374813476061\\
80.03	0.00174229316712478\\
80.04	0.00174711047675296\\
80.05	0.00175193328719754\\
80.06	0.00175676160919229\\
80.07	0.00176159545349788\\
80.08	0.00176643483090196\\
80.09	0.00177127975221927\\
80.1	0.00177613022829169\\
80.11	0.0017809862699883\\
80.12	0.00178584788820552\\
80.13	0.00179071509386712\\
80.14	0.00179558789792433\\
80.15	0.00180046631135592\\
80.16	0.00180535034516829\\
80.17	0.00181024001039549\\
80.18	0.00181513531809937\\
80.19	0.00182003627936964\\
80.2	0.0018249429053239\\
80.21	0.0018298552071078\\
80.22	0.00183477319589504\\
80.23	0.00183969688288753\\
80.24	0.00184462627931538\\
80.25	0.00184956139643707\\
80.26	0.00185450224553945\\
80.27	0.00185944883793791\\
80.28	0.00186440118497635\\
80.29	0.00186935929802737\\
80.3	0.00187432318849228\\
80.31	0.00187929286780121\\
80.32	0.00188426834741318\\
80.33	0.0018892496388162\\
80.34	0.00189423675352733\\
80.35	0.00189922970309278\\
80.36	0.00190422849908798\\
80.37	0.00190923315311767\\
80.38	0.00191424367681599\\
80.39	0.00191926008184654\\
80.4	0.00192428237990249\\
80.41	0.00192931058270664\\
80.42	0.0019343431849562\\
80.43	0.00193937914030812\\
80.44	0.00194441845210473\\
80.45	0.00194946112369128\\
80.46	0.00195450715841599\\
80.47	0.00195955655963\\
80.48	0.00196460933068741\\
80.49	0.00196966547494521\\
80.5	0.00197472499576333\\
80.51	0.00197978789650457\\
80.52	0.00198485418053465\\
80.53	0.00198992385122214\\
80.54	0.00199499691193851\\
80.55	0.00200007336605808\\
80.56	0.00200515321695801\\
80.57	0.00201023646801831\\
80.58	0.0020153231226218\\
80.59	0.00202041318415414\\
80.6	0.00202550665600379\\
80.61	0.00203060354156197\\
80.62	0.00203570384422274\\
80.63	0.00204080756738287\\
80.64	0.00204591471444192\\
80.65	0.0020510252888022\\
80.66	0.00205613929386872\\
80.67	0.00206125673304925\\
80.68	0.00206637760975423\\
80.69	0.00207150192739683\\
80.7	0.00207662968939287\\
80.71	0.00208176089916084\\
80.72	0.00208689556012191\\
80.73	0.00209203367569986\\
80.74	0.00209717524932111\\
80.75	0.00210232028441468\\
80.76	0.00210746878441221\\
80.77	0.00211262075274789\\
80.78	0.0021177761928585\\
80.79	0.00212293510818336\\
80.8	0.00212809750216434\\
80.81	0.00213326337824581\\
80.82	0.00213843273987467\\
80.83	0.00214360559050028\\
80.84	0.00214878193357449\\
80.85	0.00215396177255161\\
80.86	0.00215914511088839\\
80.87	0.00216433195204398\\
80.88	0.00216952229947996\\
80.89	0.00217471615666029\\
80.9	0.00217991352705129\\
80.91	0.00218511441412165\\
80.92	0.00219031882134237\\
80.93	0.00219552675218679\\
80.94	0.00220073821013054\\
80.95	0.00220595319865151\\
80.96	0.00221117172122986\\
80.97	0.002216393781348\\
80.98	0.00222161938249054\\
80.99	0.0022268485281443\\
81	0.00223208122179828\\
81.01	0.00223731746694362\\
81.02	0.00224255726707362\\
81.03	0.00224780062568368\\
81.04	0.0022530475462713\\
81.05	0.00225829803233605\\
81.06	0.00226355208737955\\
81.07	0.00226880971490545\\
81.08	0.0022740709184194\\
81.09	0.00227933570142903\\
81.1	0.00228460406744393\\
81.11	0.00228987601997565\\
81.12	0.00229515156253761\\
81.13	0.00230043069864514\\
81.14	0.00230571343181543\\
81.15	0.0023109997655675\\
81.16	0.00231628970342218\\
81.17	0.00232158324890211\\
81.18	0.00232688040553165\\
81.19	0.00233218117683692\\
81.2	0.00233748556634576\\
81.21	0.00234279357758764\\
81.22	0.00234810521409375\\
81.23	0.00235342047939684\\
81.24	0.0023587393770313\\
81.25	0.00236406191053308\\
81.26	0.00236938808343965\\
81.27	0.00237471789929002\\
81.28	0.00238005136162465\\
81.29	0.00238538847398546\\
81.3	0.00239072923991582\\
81.31	0.00239607366296044\\
81.32	0.00240142174666542\\
81.33	0.00240677349457818\\
81.34	0.00241212891024743\\
81.35	0.00241748799722314\\
81.36	0.00242285075905653\\
81.37	0.00242821719929999\\
81.38	0.00243358732150707\\
81.39	0.00243896112923248\\
81.4	0.00244433862603199\\
81.41	0.00244971981546244\\
81.42	0.00245510470108169\\
81.43	0.00246049328644861\\
81.44	0.00246588557512298\\
81.45	0.00247128157066552\\
81.46	0.00247668127663782\\
81.47	0.00248208469660231\\
81.48	0.00248749183412222\\
81.49	0.00249290269276153\\
81.5	0.00249831727608496\\
81.51	0.0025037355876579\\
81.52	0.00250915763104638\\
81.53	0.00251458340981704\\
81.54	0.00252001292753707\\
81.55	0.00252544618777418\\
81.56	0.00253088319409657\\
81.57	0.00253632395007284\\
81.58	0.00254176845927201\\
81.59	0.00254721672526342\\
81.6	0.00255266875161674\\
81.61	0.00255812454190187\\
81.62	0.00256358409968892\\
81.63	0.00256904742854817\\
81.64	0.00257451453205002\\
81.65	0.00257998541376493\\
81.66	0.00258546007726339\\
81.67	0.00259093852611584\\
81.68	0.00259642076389268\\
81.69	0.00260190679416414\\
81.7	0.0026073966205003\\
81.71	0.002612890246471\\
81.72	0.00261838767564581\\
81.73	0.00262388891159394\\
81.74	0.00262939395788426\\
81.75	0.00263490281808514\\
81.76	0.00264041549576451\\
81.77	0.00264593199448972\\
81.78	0.00265145231782752\\
81.79	0.00265697646934399\\
81.8	0.00266250445260451\\
81.81	0.00266803627117367\\
81.82	0.00267357192861522\\
81.83	0.00267911142849203\\
81.84	0.002684654774366\\
81.85	0.00269020196979801\\
81.86	0.00269575301834789\\
81.87	0.0027013079235743\\
81.88	0.00270686668903471\\
81.89	0.00271242931828532\\
81.9	0.002717995814881\\
81.91	0.00272356618237522\\
81.92	0.00272914042431999\\
81.93	0.00273471854426579\\
81.94	0.0027403005457615\\
81.95	0.00274588643235433\\
81.96	0.00275147620758974\\
81.97	0.00275706987501142\\
81.98	0.00276266743816115\\
81.99	0.00276826890057877\\
82	0.00277387426580209\\
82.01	0.00277948353736682\\
82.02	0.00278509671880652\\
82.03	0.00279071381365248\\
82.04	0.00279633482543366\\
82.05	0.00280195975767663\\
82.06	0.00280758861390548\\
82.07	0.00281322139764173\\
82.08	0.00281885811240426\\
82.09	0.00282449876170924\\
82.1	0.00283014334907001\\
82.11	0.00283579187799704\\
82.12	0.00284144435199781\\
82.13	0.00284710077457677\\
82.14	0.0028527611492352\\
82.15	0.00285842547947114\\
82.16	0.00286409376877932\\
82.17	0.00286976602065107\\
82.18	0.0028754422385742\\
82.19	0.00288112242603292\\
82.2	0.00288680658650777\\
82.21	0.00289249472347549\\
82.22	0.00289818684040896\\
82.23	0.00290388294077707\\
82.24	0.00290958302804465\\
82.25	0.00291528710567234\\
82.26	0.00292099517711654\\
82.27	0.00292670724582925\\
82.28	0.00293242331525801\\
82.29	0.00293814338884578\\
82.3	0.00294386747003082\\
82.31	0.00294959556224663\\
82.32	0.00295532766892178\\
82.33	0.00296106379347985\\
82.34	0.00296680393933932\\
82.35	0.0029725481099134\\
82.36	0.00297829630861\\
82.37	0.00298404853950593\\
82.38	0.00298980480669384\\
82.39	0.00299556511426892\\
82.4	0.00300132946632891\\
82.41	0.00300709786697403\\
82.42	0.00301287032030701\\
82.43	0.00301864683043302\\
82.44	0.00302442740145972\\
82.45	0.00303021203749715\\
82.46	0.00303600074265779\\
82.47	0.00304179352105649\\
82.48	0.00304759037681047\\
82.49	0.00305339131403929\\
82.5	0.00305919633686482\\
82.51	0.00306500544941125\\
82.52	0.00307081865580504\\
82.53	0.0030766359601749\\
82.54	0.00308245736665176\\
82.55	0.00308828287936878\\
82.56	0.0030941125024613\\
82.57	0.00309994624006681\\
82.58	0.00310578409632495\\
82.59	0.00311162607537746\\
82.6	0.00311747218136819\\
82.61	0.00312332241844303\\
82.62	0.00312917679074992\\
82.63	0.00313503530243884\\
82.64	0.0031408979576617\\
82.65	0.00314676476057243\\
82.66	0.00315263571532686\\
82.67	0.00315851082608274\\
82.68	0.00316439009699972\\
82.69	0.00317027353223928\\
82.7	0.00317616113596473\\
82.71	0.00318205291234121\\
82.72	0.00318794886553559\\
82.73	0.00319384899971653\\
82.74	0.00319975331905436\\
82.75	0.00320566182772112\\
82.76	0.0032115745298905\\
82.77	0.00321749142973784\\
82.78	0.00322341253144003\\
82.79	0.00322933783917557\\
82.8	0.00323526735712447\\
82.81	0.00324120108946825\\
82.82	0.00324713904038991\\
82.83	0.00325308121407388\\
82.84	0.003259027614706\\
82.85	0.00326497824647349\\
82.86	0.00327093311356491\\
82.87	0.00327689222017012\\
82.88	0.00328285557048028\\
82.89	0.00328882316868775\\
82.9	0.00329479501898614\\
82.91	0.0033007711255702\\
82.92	0.00330675149263583\\
82.93	0.00331273612438003\\
82.94	0.00331872502500085\\
82.95	0.00332471819869738\\
82.96	0.00333071564966969\\
82.97	0.00333671738211882\\
82.98	0.0033427234002467\\
82.99	0.00334873370825615\\
83	0.00335474831035082\\
83.01	0.00336076721073516\\
83.02	0.00336679041361439\\
83.03	0.00337281792319442\\
83.04	0.00337884974368186\\
83.05	0.00338488587928394\\
83.06	0.0033909263342085\\
83.07	0.00339697111266392\\
83.08	0.00340302021885908\\
83.09	0.00340907365700336\\
83.1	0.00341513143130652\\
83.11	0.00342119354597872\\
83.12	0.00342726000523046\\
83.13	0.00343333081327251\\
83.14	0.00343940597431591\\
83.15	0.00344548549257186\\
83.16	0.00345156937225174\\
83.17	0.00345765761756701\\
83.18	0.00346375023272921\\
83.19	0.00346984722194987\\
83.2	0.00347594858944048\\
83.21	0.00348205433941243\\
83.22	0.00348816447607699\\
83.23	0.00349427900364522\\
83.24	0.00350039792632793\\
83.25	0.00350652124833567\\
83.26	0.0035126489738786\\
83.27	0.0035187811071665\\
83.28	0.0035249176524087\\
83.29	0.00353105861381401\\
83.3	0.00353720399559068\\
83.31	0.00354335380194635\\
83.32	0.00354950803708797\\
83.33	0.00355566670522177\\
83.34	0.00356182981055318\\
83.35	0.0035679973572868\\
83.36	0.00357416934962632\\
83.37	0.00358034579177445\\
83.38	0.00358652668793288\\
83.39	0.00359271204230224\\
83.4	0.00359890185908198\\
83.41	0.00360509614247035\\
83.42	0.00361129489666434\\
83.43	0.00361749812585958\\
83.44	0.00362370583425033\\
83.45	0.00362991802602937\\
83.46	0.00363613470538794\\
83.47	0.00364235587651569\\
83.48	0.00364858154360062\\
83.49	0.00365481171082897\\
83.5	0.0036610463823852\\
83.51	0.00366728556245188\\
83.52	0.00367352925520967\\
83.53	0.00367977746483717\\
83.54	0.00368603019551094\\
83.55	0.00369228745140535\\
83.56	0.00369854923669256\\
83.57	0.0037048155555424\\
83.58	0.00371108641212235\\
83.59	0.00371736181059739\\
83.6	0.00372364175513001\\
83.61	0.00372992624988005\\
83.62	0.00373621529900467\\
83.63	0.00374250890665827\\
83.64	0.00374880707699237\\
83.65	0.00375510981415559\\
83.66	0.00376141712229349\\
83.67	0.00376772900554856\\
83.68	0.0037740454680601\\
83.69	0.00378036651396413\\
83.7	0.0037866921473933\\
83.71	0.00379302237247685\\
83.72	0.00379935719334045\\
83.73	0.00380569661410616\\
83.74	0.00381204063889233\\
83.75	0.0038183892718135\\
83.76	0.0038247425169803\\
83.77	0.00383110037849939\\
83.78	0.00383746286047331\\
83.79	0.00384382996700045\\
83.8	0.0038502017021749\\
83.81	0.00385657807008638\\
83.82	0.00386295907482011\\
83.83	0.00386934472045676\\
83.84	0.00387573501107231\\
83.85	0.00388212995073794\\
83.86	0.00388852954351997\\
83.87	0.0038949337934797\\
83.88	0.00390134270467335\\
83.89	0.00390775628115192\\
83.9	0.00391417452696111\\
83.91	0.00392059744614118\\
83.92	0.00392702504272686\\
83.93	0.00393345732074723\\
83.94	0.00393989428422559\\
83.95	0.00394633593717938\\
83.96	0.00395278228362005\\
83.97	0.00395923332755291\\
83.98	0.00396568907297705\\
83.99	0.00397214952388521\\
84	0.00397861468426364\\
84.01	0.00398508455809201\\
84.02	0.00399155914934324\\
84.03	0.00399803846198341\\
84.04	0.00400452249997163\\
84.05	0.00401101126725989\\
84.06	0.00401750476779292\\
84.07	0.00402400300550813\\
84.08	0.00403050598433536\\
84.09	0.00403701370819685\\
84.1	0.00404352618100706\\
84.11	0.00405004340667249\\
84.12	0.00405656524916951\\
84.13	0.00406309155378942\\
84.14	0.00406962232515156\\
84.15	0.00407615756787363\\
84.16	0.00408269728657157\\
84.17	0.00408924148585945\\
84.18	0.00409579017034934\\
84.19	0.00410234334465123\\
84.2	0.00410890101337289\\
84.21	0.00411546318111974\\
84.22	0.00412202985249477\\
84.23	0.00412860103209834\\
84.24	0.00413517672452817\\
84.25	0.00414175693437912\\
84.26	0.0041483416662431\\
84.27	0.00415493092470892\\
84.28	0.0041615247143622\\
84.29	0.00416812303978521\\
84.3	0.00417472590555673\\
84.31	0.00418133331625192\\
84.32	0.0041879452764422\\
84.33	0.00419456179069509\\
84.34	0.00420118286357406\\
84.35	0.00420780849963841\\
84.36	0.00421443870344312\\
84.37	0.0042210734795387\\
84.38	0.00422771283247101\\
84.39	0.00423435676678117\\
84.4	0.00424100528700535\\
84.41	0.00424765839767464\\
84.42	0.00425431610331489\\
84.43	0.00426097840844655\\
84.44	0.0042676453175845\\
84.45	0.0042743168352379\\
84.46	0.00428099296591002\\
84.47	0.00428767371409804\\
84.48	0.00429435908429294\\
84.49	0.00430104908097926\\
84.5	0.00430774370863498\\
84.51	0.0043144429717313\\
84.52	0.0043211468747325\\
84.53	0.0043278554220957\\
84.54	0.00433456861827074\\
84.55	0.00434128646769996\\
84.56	0.00434800897481798\\
84.57	0.00435473614405158\\
84.58	0.00436146797981943\\
84.59	0.00436820448653193\\
84.6	0.00437494566859101\\
84.61	0.00438169153038993\\
84.62	0.00438844207631304\\
84.63	0.0043951973107356\\
84.64	0.00440195723802358\\
84.65	0.00440872186253341\\
84.66	0.00441549118861179\\
84.67	0.00442226522059545\\
84.68	0.00442904396281094\\
84.69	0.00443582741957442\\
84.7	0.00444261559519138\\
84.71	0.00444940849395645\\
84.72	0.00445620612015317\\
84.73	0.00446300847805372\\
84.74	0.00446981557191868\\
84.75	0.00447662740599682\\
84.76	0.00448344398452482\\
84.77	0.00449026531172703\\
84.78	0.00449709139181521\\
84.79	0.00450392222898828\\
84.8	0.00451075782743204\\
84.81	0.00451759819131893\\
84.82	0.00452444332480776\\
84.83	0.0045312932320434\\
84.84	0.00453814791715654\\
84.85	0.00454500738426341\\
84.86	0.00455187163746549\\
84.87	0.00455874068084921\\
84.88	0.00456561451848568\\
84.89	0.00457249315443037\\
84.9	0.00457937659272284\\
84.91	0.00458626483738643\\
84.92	0.00459315789507986\\
84.93	0.0046000557728471\\
84.94	0.00460695847774704\\
84.95	0.00461386601685348\\
84.96	0.00462077839725518\\
84.97	0.00462769562605585\\
84.98	0.00463461771037419\\
84.99	0.00464154465734389\\
85	0.00464847647411363\\
85.01	0.00465541316784714\\
85.02	0.00466235474572317\\
85.03	0.00466930121493551\\
85.04	0.00467625258269305\\
85.05	0.00468320885621974\\
85.06	0.00469017004275461\\
85.07	0.00469713614955181\\
85.08	0.0047041071838806\\
85.09	0.00471108315302539\\
85.1	0.00471806406428569\\
85.11	0.0047250499249762\\
85.12	0.00473204074242676\\
85.13	0.00473903652398239\\
85.14	0.00474603727700328\\
85.15	0.00475304300886484\\
85.16	0.00476005372695766\\
85.17	0.00476706943868753\\
85.18	0.00477409015147547\\
85.19	0.00478111587275772\\
85.2	0.00478814660998576\\
85.21	0.0047951823706263\\
85.22	0.00480222316216129\\
85.23	0.00480926899208792\\
85.24	0.00481631986791867\\
85.25	0.00482337579718123\\
85.26	0.00483043678741858\\
85.27	0.00483750284618898\\
85.28	0.00484457398106592\\
85.29	0.00485165019963818\\
85.3	0.00485873150950981\\
85.31	0.00486581791830015\\
85.32	0.00487290943364378\\
85.33	0.00488000606319056\\
85.34	0.00488710781460566\\
85.35	0.00489421469556947\\
85.36	0.00490132671377768\\
85.37	0.00490844387694122\\
85.38	0.00491556619278632\\
85.39	0.00492269366905444\\
85.4	0.00492982631350229\\
85.41	0.00493696413390185\\
85.42	0.00494410713804032\\
85.43	0.00495125533372015\\
85.44	0.00495840872875902\\
85.45	0.00496556733098981\\
85.46	0.00497273114826064\\
85.47	0.00497990018843479\\
85.48	0.00498707445939076\\
85.49	0.00499425396902222\\
85.5	0.00500143872523801\\
85.51	0.00500862873596212\\
85.52	0.00501582400913367\\
85.53	0.00502302455270692\\
85.54	0.00503023037465123\\
85.55	0.00503744148295105\\
85.56	0.00504465788560591\\
85.57	0.0050518795906304\\
85.58	0.00505910660605415\\
85.59	0.0050663389399218\\
85.6	0.00507357660029298\\
85.61	0.00508081959524231\\
85.62	0.00508806793285937\\
85.63	0.00509532162124864\\
85.64	0.00510258066852953\\
85.65	0.00510984508283632\\
85.66	0.00511711487231813\\
85.67	0.00512439004513892\\
85.68	0.00513167060947744\\
85.69	0.0051389565735272\\
85.7	0.00514624794549645\\
85.71	0.00515354473360812\\
85.72	0.00516084694609984\\
85.73	0.00516815459122386\\
85.74	0.00517546767724701\\
85.75	0.00518278621245071\\
85.76	0.00519011020513089\\
85.77	0.00519743966359795\\
85.78	0.00520477459617676\\
85.79	0.00521211501120658\\
85.8	0.00521946091704101\\
85.81	0.00522681232204799\\
85.82	0.00523416923460973\\
85.83	0.00524153166312265\\
85.84	0.00524889961599734\\
85.85	0.00525627310165852\\
85.86	0.00526365212854499\\
85.87	0.00527103670510956\\
85.88	0.00527842683981903\\
85.89	0.00528582254115407\\
85.9	0.00529322381760925\\
85.91	0.00530063067769291\\
85.92	0.00530804312992714\\
85.93	0.0053154611828477\\
85.94	0.00532288484500396\\
85.95	0.00533031412495885\\
85.96	0.00533774903128877\\
85.97	0.00534518957258356\\
85.98	0.00535263575744638\\
85.99	0.00536008759449367\\
86	0.00536754509235508\\
86.01	0.0053750082596734\\
86.02	0.00538247710510444\\
86.03	0.00538995163731701\\
86.04	0.00539743186499281\\
86.05	0.00540491779682635\\
86.06	0.00541240944152486\\
86.07	0.00541990680780824\\
86.08	0.00542740990440893\\
86.09	0.00543491874007184\\
86.1	0.00544243332355425\\
86.11	0.00544995366362573\\
86.12	0.00545747976906806\\
86.13	0.00546501164867508\\
86.14	0.00547254931125263\\
86.15	0.00548009276561845\\
86.16	0.00548764202060207\\
86.17	0.00549519708504469\\
86.18	0.0055027579677991\\
86.19	0.00551032467772953\\
86.2	0.00551789722371158\\
86.21	0.00552547561463209\\
86.22	0.005533059859389\\
86.23	0.00554064996689127\\
86.24	0.00554824594605873\\
86.25	0.00555584780582197\\
86.26	0.00556345555512219\\
86.27	0.00557106920291112\\
86.28	0.00557868875815082\\
86.29	0.00558631422981361\\
86.3	0.00559394562688189\\
86.31	0.00560158295834803\\
86.32	0.00560922623321419\\
86.33	0.00561687546049221\\
86.34	0.00562453064920347\\
86.35	0.00563219180837868\\
86.36	0.00563985894705779\\
86.37	0.00564753207428982\\
86.38	0.00565521119913267\\
86.39	0.00566289633065297\\
86.4	0.00567058747792595\\
86.41	0.00567828465003522\\
86.42	0.00568598785607264\\
86.43	0.00569369710513812\\
86.44	0.00570141240633944\\
86.45	0.00570913376879209\\
86.46	0.00571686120161907\\
86.47	0.00572459471395072\\
86.48	0.00573233431492449\\
86.49	0.00574008001368478\\
86.5	0.00574783181938275\\
86.51	0.00575558974117608\\
86.52	0.00576335378822878\\
86.53	0.00577112396971103\\
86.54	0.00577890029479886\\
86.55	0.00578668277267406\\
86.56	0.00579447141252385\\
86.57	0.00580226622354074\\
86.58	0.00581006721492224\\
86.59	0.00581787439587067\\
86.6	0.00582568777559292\\
86.61	0.00583350736330016\\
86.62	0.00584133316820768\\
86.63	0.00584916519953458\\
86.64	0.00585700346650352\\
86.65	0.00586484797834051\\
86.66	0.00587269874427459\\
86.67	0.00588055577353761\\
86.68	0.00588841907536393\\
86.69	0.00589628865899017\\
86.7	0.00590416453365492\\
86.71	0.00591204670859844\\
86.72	0.00591993519306241\\
86.73	0.00592782999628959\\
86.74	0.00593573112752357\\
86.75	0.00594363859600842\\
86.76	0.00595155241098842\\
86.77	0.00595947258170774\\
86.78	0.00596739911741009\\
86.79	0.00597533202733844\\
86.8	0.00598327132073466\\
86.81	0.00599121700683921\\
86.82	0.00599588779591692\\
86.83	0.00599914093098286\\
86.84	0.00600239399845674\\
86.85	0.00600564699208879\\
86.86	0.00600889990560195\\
86.87	0.00601215273269177\\
86.88	0.00601540546702631\\
86.89	0.00601865810224604\\
86.9	0.00602191063196372\\
86.91	0.0060251630497643\\
86.92	0.00602841534920478\\
86.93	0.00603166752381417\\
86.94	0.00603491956709329\\
86.95	0.00603817147251475\\
86.96	0.00604142323352275\\
86.97	0.00604467484353305\\
86.98	0.00604792629593282\\
86.99	0.0060511775840805\\
87	0.00605442870130575\\
87.01	0.00605767964090931\\
87.02	0.00606093039616285\\
87.03	0.00606418096030893\\
87.04	0.0060674313265608\\
87.05	0.00607068148810239\\
87.06	0.00607393143808807\\
87.07	0.00607718116964265\\
87.08	0.0060804306758612\\
87.09	0.00608367994980895\\
87.1	0.00608692898452116\\
87.11	0.00609017777300305\\
87.12	0.00609342630822962\\
87.13	0.00609667458314557\\
87.14	0.00609992259066517\\
87.15	0.00610317032367214\\
87.16	0.00610641777501955\\
87.17	0.00610966493752969\\
87.18	0.00611291180399392\\
87.19	0.0061161583671726\\
87.2	0.00611940461979493\\
87.21	0.00612265055455885\\
87.22	0.00612589616413091\\
87.23	0.00612914144114617\\
87.24	0.00613238637820802\\
87.25	0.00613563096788813\\
87.26	0.00613887520272627\\
87.27	0.00614211907523022\\
87.28	0.00614536257787563\\
87.29	0.00614860570310588\\
87.3	0.006151848443332\\
87.31	0.00615509079093252\\
87.32	0.00615833273825332\\
87.33	0.00616157427760754\\
87.34	0.00616481540127544\\
87.35	0.00616805610150425\\
87.36	0.0061712963705081\\
87.37	0.00617453620046782\\
87.38	0.00617777558353087\\
87.39	0.00618101451181118\\
87.4	0.00618425297738902\\
87.41	0.00618749097231089\\
87.42	0.00619072848858936\\
87.43	0.00619396551820299\\
87.44	0.00619720205309612\\
87.45	0.00620043808517882\\
87.46	0.0062036736063267\\
87.47	0.0062069086083808\\
87.48	0.00621014308314747\\
87.49	0.0062133770223982\\
87.5	0.00621661041786951\\
87.51	0.00621984326126282\\
87.52	0.00622307554424432\\
87.53	0.00622630725844478\\
87.54	0.00622953839545949\\
87.55	0.00623276894684809\\
87.56	0.0062359989041344\\
87.57	0.00623922825880635\\
87.58	0.00624245700231578\\
87.59	0.00624568512607835\\
87.6	0.00624891262147336\\
87.61	0.00625213947984364\\
87.62	0.0062553656924954\\
87.63	0.00625859125069808\\
87.64	0.00626181614568424\\
87.65	0.00626504036864938\\
87.66	0.00626826391075183\\
87.67	0.00627148676311257\\
87.68	0.00627470891681513\\
87.69	0.00627793036290544\\
87.7	0.00628115109239165\\
87.71	0.00628437109624403\\
87.72	0.00628759036539479\\
87.73	0.00629080889073797\\
87.74	0.00629402666312926\\
87.75	0.00629724367338589\\
87.76	0.00630045991228645\\
87.77	0.00630367537057075\\
87.78	0.00630689003893969\\
87.79	0.00631010390805511\\
87.8	0.00631331696853962\\
87.81	0.00631652921097647\\
87.82	0.0063197406259094\\
87.83	0.00632295120384248\\
87.84	0.00632616093523997\\
87.85	0.00632936981052617\\
87.86	0.00633257782008525\\
87.87	0.00633578495426113\\
87.88	0.00633899120335729\\
87.89	0.00634219655763667\\
87.9	0.00634540100732145\\
87.91	0.00634860454259295\\
87.92	0.00635180715359147\\
87.93	0.00635500883041609\\
87.94	0.00635820956312457\\
87.95	0.00636140934173317\\
87.96	0.00636460815621651\\
87.97	0.00636780599650737\\
87.98	0.00637100285249659\\
87.99	0.00637419871403289\\
88	0.00637739357092268\\
88.01	0.00638058741292995\\
88.02	0.0063837802297761\\
88.03	0.00638697201113976\\
88.04	0.00639016274665664\\
88.05	0.00639335242591938\\
88.06	0.00639654103847739\\
88.07	0.00639972857383666\\
88.08	0.00640291502145964\\
88.09	0.00640610037076506\\
88.1	0.00640928461112773\\
88.11	0.00641246773187846\\
88.12	0.00641564972230383\\
88.13	0.00641883057164602\\
88.14	0.00642201026910271\\
88.15	0.00642518880382686\\
88.16	0.00642836616492655\\
88.17	0.00643154234146484\\
88.18	0.00643471732245958\\
88.19	0.00643789109688326\\
88.2	0.00644106365366283\\
88.21	0.00644423498167954\\
88.22	0.00644740506976877\\
88.23	0.00645057390671985\\
88.24	0.00645374148127591\\
88.25	0.00645690778213371\\
88.26	0.00646007279794346\\
88.27	0.00646323651730864\\
88.28	0.00646639892878586\\
88.29	0.00646956002088465\\
88.3	0.00647271978206732\\
88.31	0.00647587820074878\\
88.32	0.00647903526529637\\
88.33	0.00648219096402967\\
88.34	0.00648534528522033\\
88.35	0.00648849821709193\\
88.36	0.00649164974781975\\
88.37	0.00649479986553066\\
88.38	0.00649794855830287\\
88.39	0.00650109581416582\\
88.4	0.00650424162109999\\
88.41	0.00650738596703668\\
88.42	0.00651052883985788\\
88.43	0.00651367022739609\\
88.44	0.00651681011743412\\
88.45	0.00651994849770491\\
88.46	0.00652308535589139\\
88.47	0.00652622067962626\\
88.48	0.00652935445649182\\
88.49	0.00653248667401981\\
88.5	0.0065356173196912\\
88.51	0.00653874638093606\\
88.52	0.00654187384513329\\
88.53	0.00654499969961054\\
88.54	0.00654812393164395\\
88.55	0.00655124652845803\\
88.56	0.00655436747722543\\
88.57	0.00655748676506677\\
88.58	0.00656060437905047\\
88.59	0.00656372030619255\\
88.6	0.00656683453345645\\
88.61	0.00656994704775288\\
88.62	0.00657305783593957\\
88.63	0.00657616688482113\\
88.64	0.00657927418114885\\
88.65	0.00658237971162054\\
88.66	0.0065854834628803\\
88.67	0.00658858542151836\\
88.68	0.00659168557407091\\
88.69	0.00659478390701985\\
88.7	0.00659788040679269\\
88.71	0.0066009750597623\\
88.72	0.00660406785224672\\
88.73	0.00660715877050902\\
88.74	0.00661024780075708\\
88.75	0.00661333492914338\\
88.76	0.00661642014176486\\
88.77	0.00661950342466269\\
88.78	0.00662258476382209\\
88.79	0.00662566414517218\\
88.8	0.00662874155458571\\
88.81	0.00663181697787893\\
88.82	0.0066348904008114\\
88.83	0.00663796180908576\\
88.84	0.00664103118834756\\
88.85	0.0066440985241851\\
88.86	0.00664716380212917\\
88.87	0.00665022700765292\\
88.88	0.00665328812617162\\
88.89	0.00665634714304252\\
88.9	0.0066594040435646\\
88.91	0.00666245881297843\\
88.92	0.00666551143646592\\
88.93	0.00666856189915018\\
88.94	0.00667161018609532\\
88.95	0.0066746562823062\\
88.96	0.00667770017272831\\
88.97	0.00668074184224753\\
88.98	0.00668378127568996\\
88.99	0.0066868184578217\\
89	0.00668985337334868\\
89.01	0.00669288600691645\\
89.02	0.00669591634311\\
89.03	0.00669894436645356\\
89.04	0.00670197006141038\\
89.05	0.00670499341238258\\
89.06	0.00670801440371093\\
89.07	0.00671103301967463\\
89.08	0.0067140492444912\\
89.09	0.00671706306231615\\
89.1	0.00672007445724293\\
89.11	0.00672308341330262\\
89.12	0.00672608991446381\\
89.13	0.00672909394463236\\
89.14	0.00673209548765124\\
89.15	0.00673509452730028\\
89.16	0.00673809104729605\\
89.17	0.0067410850312916\\
89.18	0.0067440764628763\\
89.19	0.00674706659121478\\
89.2	0.00675005762367183\\
89.21	0.00675304955949813\\
89.22	0.00675604239794348\\
89.23	0.00675903613825679\\
89.24	0.00676203077968608\\
89.25	0.00676502632147851\\
89.26	0.00676802276288041\\
89.27	0.00677102010313725\\
89.28	0.0067740183414937\\
89.29	0.00677701747719362\\
89.3	0.00678001750948009\\
89.31	0.0067830184375954\\
89.32	0.0067860202607811\\
89.33	0.00678902297827799\\
89.34	0.00679202658932615\\
89.35	0.00679503109316494\\
89.36	0.00679803648903304\\
89.37	0.00680104277616843\\
89.38	0.00680404995380847\\
89.39	0.00680705802118982\\
89.4	0.00681006697754856\\
89.41	0.00681307682212013\\
89.42	0.00681608755413939\\
89.43	0.00681909917284061\\
89.44	0.00682211167745752\\
89.45	0.00682512506722328\\
89.46	0.00682813934137056\\
89.47	0.00683115449913149\\
89.48	0.00683417053973772\\
89.49	0.00683718746242045\\
89.5	0.00684020526641039\\
89.51	0.00684322395093785\\
89.52	0.0068462435152327\\
89.53	0.00684926395852442\\
89.54	0.00685228528004213\\
89.55	0.00685530747901456\\
89.56	0.00685833055467013\\
89.57	0.00686135450623691\\
89.58	0.00686437933294268\\
89.59	0.00686740503401495\\
89.6	0.00687043160868097\\
89.61	0.00687345905616772\\
89.62	0.00687648737570199\\
89.63	0.00687951656651036\\
89.64	0.00688254662781922\\
89.65	0.00688557755885482\\
89.66	0.00688860935884325\\
89.67	0.0068916420270105\\
89.68	0.00689467556258246\\
89.69	0.00689770996478495\\
89.7	0.00690074523284373\\
89.71	0.00690378136598452\\
89.72	0.00690681836343307\\
89.73	0.00690985622441509\\
89.74	0.00691289494815639\\
89.75	0.00691593453388278\\
89.76	0.00691897498082018\\
89.77	0.00692201628819462\\
89.78	0.00692505845523225\\
89.79	0.00692810148115938\\
89.8	0.00693114536520249\\
89.81	0.00693419010658824\\
89.82	0.00693723570454356\\
89.83	0.00694028215829559\\
89.84	0.00694332946707176\\
89.85	0.00694637763009978\\
89.86	0.00694942664660771\\
89.87	0.00695247651582393\\
89.88	0.00695552723697722\\
89.89	0.00695857880929672\\
89.9	0.00696163123201203\\
89.91	0.00696468450435319\\
89.92	0.00696773862555071\\
89.93	0.0069707935948356\\
89.94	0.00697384941143942\\
89.95	0.00697690607459425\\
89.96	0.00697996358353279\\
89.97	0.00698302193748833\\
89.98	0.0069860811356948\\
89.99	0.00698914117738681\\
90	0.00699220206179963\\
90.01	0.0069952637881693\\
90.02	0.00699832635573257\\
90.03	0.007001389763727\\
90.04	0.00700445401139091\\
90.05	0.00700751909796352\\
90.06	0.00701058502268487\\
90.07	0.00701365178479591\\
90.08	0.00701671938353854\\
90.09	0.00701978781815557\\
90.1	0.00702285708789082\\
90.11	0.00702592719198915\\
90.12	0.00702899812969643\\
90.13	0.00703206990025962\\
90.14	0.00703514250292681\\
90.15	0.00703821593694721\\
90.16	0.00704129020157122\\
90.17	0.00704436529605042\\
90.18	0.00704744121963766\\
90.19	0.00705051797158705\\
90.2	0.00705359555115399\\
90.21	0.00705667395759524\\
90.22	0.00705975319016891\\
90.23	0.00706283324813453\\
90.24	0.00706591413075306\\
90.25	0.00706899583728694\\
90.26	0.00707207836700011\\
90.27	0.00707516171915804\\
90.28	0.00707824589302781\\
90.29	0.00708133088787809\\
90.3	0.00708441670297919\\
90.31	0.00708750333760313\\
90.32	0.00709059079102363\\
90.33	0.00709367906251617\\
90.34	0.00709676815135802\\
90.35	0.0070998580568283\\
90.36	0.00710294877820799\\
90.37	0.00710604031477995\\
90.38	0.00710913266582901\\
90.39	0.00711222583064199\\
90.4	0.00711531980850769\\
90.41	0.00711841459871702\\
90.42	0.00712151020056295\\
90.43	0.0071246066133406\\
90.44	0.00712770383634728\\
90.45	0.0071308018688825\\
90.46	0.00713390071024803\\
90.47	0.00713700035974795\\
90.48	0.00714010081668868\\
90.49	0.00714320208037901\\
90.5	0.00714630415013016\\
90.51	0.00714940702525581\\
90.52	0.00715251070507215\\
90.53	0.00715561518889793\\
90.54	0.00715872047605448\\
90.55	0.00716182656586576\\
90.56	0.00716493345765844\\
90.57	0.00716804115076189\\
90.58	0.00717114964450825\\
90.59	0.00717425893823248\\
90.6	0.0071773690312724\\
90.61	0.00718047992296871\\
90.62	0.0071835916126651\\
90.63	0.00718670409970823\\
90.64	0.0071898173834478\\
90.65	0.00719293146323661\\
90.66	0.0071960463384306\\
90.67	0.00719916200838886\\
90.68	0.00720227847247376\\
90.69	0.00720539573005092\\
90.7	0.0072085137804893\\
90.71	0.00721163262316121\\
90.72	0.00721475225744244\\
90.73	0.0072178726827122\\
90.74	0.00722099389835327\\
90.75	0.00722411590375198\\
90.76	0.0072272386982983\\
90.77	0.00723036242357136\\
90.78	0.00723348728669508\\
90.79	0.00723661328655174\\
90.8	0.00723974042201773\\
90.81	0.00724286869196362\\
90.82	0.0072459980952541\\
90.83	0.00724912863074794\\
90.84	0.00725226029729803\\
90.85	0.00725539309375131\\
90.86	0.00725852701894878\\
90.87	0.00726166207172546\\
90.88	0.00726479825091039\\
90.89	0.00726793555532661\\
90.9	0.00727107398379113\\
90.91	0.00727421353511491\\
90.92	0.00727735420810285\\
90.93	0.00728049600155377\\
90.94	0.0072836389142604\\
90.95	0.00728678294500932\\
90.96	0.00728992809258101\\
90.97	0.00729307435574975\\
90.98	0.00729622173328368\\
90.99	0.00729937022394472\\
91	0.00730251982648858\\
91.01	0.00730567053966473\\
91.02	0.00730882236221638\\
91.03	0.00731197529288049\\
91.04	0.00731512933038769\\
91.05	0.00731828447346232\\
91.06	0.00732144072082237\\
91.07	0.00732459807117949\\
91.08	0.00732775652323893\\
91.09	0.00733091607569959\\
91.1	0.00733407672725392\\
91.11	0.00733723847658794\\
91.12	0.00734040132238122\\
91.13	0.00734356526330686\\
91.14	0.00734673029803145\\
91.15	0.0073498964252151\\
91.16	0.00735306364351133\\
91.17	0.00735623195156716\\
91.18	0.00735940134802299\\
91.19	0.00736257183151265\\
91.2	0.00736574340066334\\
91.21	0.00736891605409562\\
91.22	0.0073720897904234\\
91.23	0.0073752646082539\\
91.24	0.00737844050618765\\
91.25	0.00738161748281845\\
91.26	0.00738479553673337\\
91.27	0.00738797466651269\\
91.28	0.00739115487072994\\
91.29	0.00739433614795181\\
91.3	0.00739751849673819\\
91.31	0.0074007019156421\\
91.32	0.00740388640320971\\
91.33	0.00740707195798028\\
91.34	0.00741025857848618\\
91.35	0.00741344626325283\\
91.36	0.0074166350107987\\
91.37	0.00741982481963527\\
91.38	0.00742301568826703\\
91.39	0.00742620761519146\\
91.4	0.00742940059889898\\
91.41	0.00743259463787296\\
91.42	0.00743578973058967\\
91.43	0.00743898587551829\\
91.44	0.00744218307112083\\
91.45	0.0074453813158522\\
91.46	0.00744858060816011\\
91.47	0.00745178094648505\\
91.48	0.00745498232926032\\
91.49	0.00745818475491198\\
91.5	0.0074613882218588\\
91.51	0.00746459272851229\\
91.52	0.00746779827327663\\
91.53	0.00747100485454868\\
91.54	0.00747421247071793\\
91.55	0.00747742112016651\\
91.56	0.00748063080126914\\
91.57	0.00748384151239312\\
91.58	0.00748705325189829\\
91.59	0.00749026601813703\\
91.6	0.00749347980945424\\
91.61	0.00749669462418727\\
91.62	0.00749991046066596\\
91.63	0.00750312731721258\\
91.64	0.0075063451921418\\
91.65	0.0075095640837607\\
91.66	0.00751278399036871\\
91.67	0.00751600491025762\\
91.68	0.00751922684171151\\
91.69	0.00752244978300678\\
91.7	0.0075256737324121\\
91.71	0.00752889868818838\\
91.72	0.00753212464858875\\
91.73	0.00753535161185855\\
91.74	0.00753857957623529\\
91.75	0.00754180853994863\\
91.76	0.00754503850122036\\
91.77	0.00754826945826436\\
91.78	0.00755150140928662\\
91.79	0.00755473435248514\\
91.8	0.00755796828604999\\
91.81	0.00756120320816321\\
91.82	0.00756443911699885\\
91.83	0.00756767601072289\\
91.84	0.00757091388749325\\
91.85	0.00757415274545976\\
91.86	0.00757739258276413\\
91.87	0.00758063339753992\\
91.88	0.00758387518791251\\
91.89	0.00758711795199911\\
91.9	0.00759036168790869\\
91.91	0.00759360639374198\\
91.92	0.00759685206759145\\
91.93	0.00760009870754124\\
91.94	0.0076033463116672\\
91.95	0.00760659487803683\\
91.96	0.00760984440470924\\
91.97	0.00761309488973516\\
91.98	0.00761634633115686\\
91.99	0.00761959872700821\\
92	0.00762285207531455\\
92.01	0.00762610637409276\\
92.02	0.00762936162135116\\
92.03	0.00763261781508954\\
92.04	0.00763587495329908\\
92.05	0.00763913303396238\\
92.06	0.00764239205505338\\
92.07	0.00764565201453737\\
92.08	0.00764891291037095\\
92.09	0.00765217474050201\\
92.1	0.00765543750286969\\
92.11	0.00765870119540437\\
92.12	0.00766196581602763\\
92.13	0.00766523136265223\\
92.14	0.00766849783318207\\
92.15	0.00767176522551218\\
92.16	0.00767503353752869\\
92.17	0.00767830276710881\\
92.18	0.00768157291212075\\
92.19	0.00768484397042378\\
92.2	0.00768811593986813\\
92.21	0.007691388818295\\
92.22	0.00769466260353649\\
92.23	0.00769793729341566\\
92.24	0.00770121288574638\\
92.25	0.0077044893783334\\
92.26	0.00770776676897229\\
92.27	0.00771104505544937\\
92.28	0.00771432423554177\\
92.29	0.0077176043070173\\
92.3	0.00772088526763451\\
92.31	0.00772416711514258\\
92.32	0.00772744984728138\\
92.33	0.00773073346178137\\
92.34	0.00773401795636357\\
92.35	0.0077373033287396\\
92.36	0.00774058957661156\\
92.37	0.00774387669767209\\
92.38	0.00774716468960425\\
92.39	0.00775045355008156\\
92.4	0.00775374327676796\\
92.41	0.00775703386731772\\
92.42	0.00776032531937549\\
92.43	0.00776361763057624\\
92.44	0.00776691079854519\\
92.45	0.00777020482089784\\
92.46	0.00777349969523991\\
92.47	0.00777679541916732\\
92.48	0.00778009199026612\\
92.49	0.00778338940611253\\
92.5	0.00778668766427284\\
92.51	0.00778998676230343\\
92.52	0.0077932866977507\\
92.53	0.00779658746815106\\
92.54	0.00779988907103091\\
92.55	0.00780319150390658\\
92.56	0.00780649476428431\\
92.57	0.00780979884966022\\
92.58	0.00781310375752027\\
92.59	0.00781640948534026\\
92.6	0.00781971603058576\\
92.61	0.00782302339071208\\
92.62	0.00782633156316427\\
92.63	0.00782964054537703\\
92.64	0.00783295033477477\\
92.65	0.00783626092877147\\
92.66	0.00783957232477073\\
92.67	0.00784288452016569\\
92.68	0.007846197512339\\
92.69	0.00784951129866283\\
92.7	0.00785282587649879\\
92.71	0.00785614124319791\\
92.72	0.0078594573961006\\
92.73	0.00786277433253665\\
92.74	0.00786609204982515\\
92.75	0.00786941054527448\\
92.76	0.00787272981618229\\
92.77	0.00787604985983542\\
92.78	0.00787937067350992\\
92.79	0.00788269225447098\\
92.8	0.0078860145999729\\
92.81	0.00788933770725907\\
92.82	0.00789266157356193\\
92.83	0.00789598619610291\\
92.84	0.00789931157209245\\
92.85	0.0079026376987299\\
92.86	0.00790596457320359\\
92.87	0.00790929219269076\\
92.88	0.00791262055435754\\
92.89	0.00791594965535889\\
92.9	0.0079192794928386\\
92.91	0.00792261006392919\\
92.92	0.00792594136575193\\
92.93	0.00792927339541678\\
92.94	0.00793260615002235\\
92.95	0.00793593962665586\\
92.96	0.00793927382239311\\
92.97	0.00794260873429842\\
92.98	0.00794594435942461\\
92.99	0.00794928069481295\\
93	0.00795261773749315\\
93.01	0.00795595548448326\\
93.02	0.00795929393278968\\
93.03	0.00796263307940712\\
93.04	0.00796597292131849\\
93.05	0.00796931345549498\\
93.06	0.00797265467889589\\
93.07	0.00797599658846869\\
93.08	0.00797933918114892\\
93.09	0.00798268245386018\\
93.1	0.00798602640351404\\
93.11	0.00798937102701006\\
93.12	0.00799271632123572\\
93.13	0.00799606228306636\\
93.14	0.00799940890936516\\
93.15	0.0080027561969831\\
93.16	0.00800610414275888\\
93.17	0.00800945274351894\\
93.18	0.00801280199607734\\
93.19	0.00801615189723577\\
93.2	0.0080195024437835\\
93.21	0.00802285363249732\\
93.22	0.0080262054601415\\
93.23	0.00802955792346773\\
93.24	0.00803291101921512\\
93.25	0.00803626474411009\\
93.26	0.00803961909486639\\
93.27	0.008042974068185\\
93.28	0.00804632966075412\\
93.29	0.0080496858692491\\
93.3	0.0080530426903324\\
93.31	0.00805640012065355\\
93.32	0.0080597581568491\\
93.33	0.00806311679554257\\
93.34	0.00806647603334439\\
93.35	0.00806983586685186\\
93.36	0.00807319629264914\\
93.37	0.00807655730730711\\
93.38	0.00807991890738341\\
93.39	0.00808328108942237\\
93.4	0.00808664384995491\\
93.41	0.00809000718549855\\
93.42	0.00809337109255732\\
93.43	0.00809673556762173\\
93.44	0.00810010060716873\\
93.45	0.00810346620766163\\
93.46	0.00810683236555004\\
93.47	0.00811019907726988\\
93.48	0.00811356633924325\\
93.49	0.00811693414787842\\
93.5	0.0081203024995698\\
93.51	0.00812367139069782\\
93.52	0.00812704081762892\\
93.53	0.0081304107767155\\
93.54	0.00813378126429584\\
93.55	0.00813715227669408\\
93.56	0.00814052381022013\\
93.57	0.00814389586116978\\
93.58	0.00814726842582469\\
93.59	0.00815064150045237\\
93.6	0.00815401508130608\\
93.61	0.0081573891646248\\
93.62	0.00816076374663317\\
93.63	0.00816413882354141\\
93.64	0.0081675143915453\\
93.65	0.00817089044682608\\
93.66	0.00817426698555043\\
93.67	0.00817764400387037\\
93.68	0.00818102149792323\\
93.69	0.00818439946383158\\
93.7	0.00818777789770317\\
93.71	0.00819115679563085\\
93.72	0.00819453615369255\\
93.73	0.00819791596795117\\
93.74	0.00820129623445456\\
93.75	0.00820467694923541\\
93.76	0.00820805810831122\\
93.77	0.00821143970768425\\
93.78	0.00821482174334139\\
93.79	0.00821820421125416\\
93.8	0.00822158710737863\\
93.81	0.00822497042765531\\
93.82	0.00822835416800914\\
93.83	0.00823173832434939\\
93.84	0.00823512289256961\\
93.85	0.00823850786854752\\
93.86	0.00824189324814501\\
93.87	0.008245279027208\\
93.88	0.0082486652015664\\
93.89	0.00825205176703407\\
93.9	0.00825543871940869\\
93.91	0.00825882605447172\\
93.92	0.00826221376798831\\
93.93	0.00826560185570726\\
93.94	0.00826899031336091\\
93.95	0.00827237913666507\\
93.96	0.00827576832131897\\
93.97	0.00827915786300515\\
93.98	0.0082825477573894\\
93.99	0.0082859380001207\\
94	0.0082893285868311\\
94.01	0.00829271951313568\\
94.02	0.00829611077463243\\
94.03	0.00829950236690223\\
94.04	0.00830289428550869\\
94.05	0.00830628652599815\\
94.06	0.00830967908389952\\
94.07	0.00831307195472427\\
94.08	0.00831646513396628\\
94.09	0.0083198586171018\\
94.1	0.00832325239958935\\
94.11	0.00832664647686962\\
94.12	0.0083300408443654\\
94.13	0.0083334354974815\\
94.14	0.00833683043160463\\
94.15	0.00834022564210333\\
94.16	0.00834362112432788\\
94.17	0.00834701687361022\\
94.18	0.00835041288526381\\
94.19	0.0083538091545836\\
94.2	0.00835720567684589\\
94.21	0.00836060244730826\\
94.22	0.00836399946120945\\
94.23	0.00836739671376928\\
94.24	0.00837079420018855\\
94.25	0.00837419191564896\\
94.26	0.00837758985531296\\
94.27	0.00838098801432369\\
94.28	0.00838438638780487\\
94.29	0.0083877849708607\\
94.3	0.00839118375857574\\
94.31	0.00839458274601481\\
94.32	0.00839798192822291\\
94.33	0.00840138130022508\\
94.34	0.00840478085702629\\
94.35	0.00840818059361137\\
94.36	0.00841158050494486\\
94.37	0.00841498058597091\\
94.38	0.00841838083161318\\
94.39	0.00842178123677471\\
94.4	0.00842518179633781\\
94.41	0.00842858250516394\\
94.42	0.00843198335809362\\
94.43	0.00843538434994626\\
94.44	0.00843878547552008\\
94.45	0.00844218672959198\\
94.46	0.00844558810691742\\
94.47	0.00844898960223028\\
94.48	0.00845239121024275\\
94.49	0.0084557929256452\\
94.5	0.00845919474310606\\
94.51	0.00846259665727166\\
94.52	0.00846599866276615\\
94.53	0.00846940075419132\\
94.54	0.00847280292612649\\
94.55	0.0084762051731284\\
94.56	0.008479607489731\\
94.57	0.00848300987044539\\
94.58	0.00848641230975964\\
94.59	0.00848981480213866\\
94.6	0.00849321734202405\\
94.61	0.00849661992383397\\
94.62	0.00850002254196298\\
94.63	0.0085034251907819\\
94.64	0.00850682786463766\\
94.65	0.00851023055785316\\
94.66	0.00851363326472893\\
94.67	0.00851703597954653\\
94.68	0.00852043869656852\\
94.69	0.00852384141003841\\
94.7	0.00852724411418066\\
94.71	0.00853064680320056\\
94.72	0.00853404947128423\\
94.73	0.00853745211259857\\
94.74	0.00854085472129118\\
94.75	0.00854425729149035\\
94.76	0.00854765981730497\\
94.77	0.00855106229282451\\
94.78	0.00855446471211898\\
94.79	0.00855786706923885\\
94.8	0.00856126935821502\\
94.81	0.00856467157305875\\
94.82	0.00856807370776165\\
94.83	0.00857147575629561\\
94.84	0.00857487771261272\\
94.85	0.00857827957064528\\
94.86	0.0085816813243057\\
94.87	0.00858508296748647\\
94.88	0.00858848449406012\\
94.89	0.00859188589787916\\
94.9	0.00859528717277601\\
94.91	0.008598688312563\\
94.92	0.00860208931103226\\
94.93	0.00860549016195573\\
94.94	0.00860889085908505\\
94.95	0.00861229139615156\\
94.96	0.00861569176686622\\
94.97	0.00861909196491959\\
94.98	0.00862249198398172\\
94.99	0.00862589181770217\\
95	0.00862929145970991\\
95.01	0.0086326909036133\\
95.02	0.00863609014300003\\
95.03	0.00863948917143704\\
95.04	0.00864288798247051\\
95.05	0.00864628656962581\\
95.06	0.0086496849264074\\
95.07	0.00865308304629884\\
95.08	0.0086564809227627\\
95.09	0.00865987854924052\\
95.1	0.00866327591915276\\
95.11	0.00866667302589873\\
95.12	0.0086700698628566\\
95.13	0.00867346642338326\\
95.14	0.00867686270081433\\
95.15	0.00868025868846409\\
95.16	0.00868365437962544\\
95.17	0.00868704976756984\\
95.18	0.00869044484554723\\
95.19	0.00869383960678603\\
95.2	0.00869723404449306\\
95.21	0.00870062815185348\\
95.22	0.00870402192203077\\
95.23	0.00870741534816664\\
95.24	0.00871080842338101\\
95.25	0.00871420114077193\\
95.26	0.00871759349341554\\
95.27	0.00872098547436605\\
95.28	0.00872437707665561\\
95.29	0.00872776829329435\\
95.3	0.00873115911727026\\
95.31	0.00873454954154916\\
95.32	0.00873793955907465\\
95.33	0.00874132916276806\\
95.34	0.0087447183455284\\
95.35	0.00874810710023229\\
95.36	0.00875149541973392\\
95.37	0.00875488329686498\\
95.38	0.00875827072443466\\
95.39	0.00876165769522953\\
95.4	0.00876504420201353\\
95.41	0.00876843023752789\\
95.42	0.00877181579449112\\
95.43	0.0087752008655989\\
95.44	0.00877858544352408\\
95.45	0.00878196952091657\\
95.46	0.00878535309040335\\
95.47	0.00878873614458838\\
95.48	0.00879211867605256\\
95.49	0.00879550067735364\\
95.5	0.00879888214102623\\
95.51	0.00880226305958171\\
95.52	0.00880564342550816\\
95.53	0.00880902323127035\\
95.54	0.00881240246930966\\
95.55	0.00881578113204402\\
95.56	0.00881915921186788\\
95.57	0.00882253670115215\\
95.58	0.00882591359224411\\
95.59	0.00882928987746741\\
95.6	0.00883266554912201\\
95.61	0.00883604059948408\\
95.62	0.00883941502080599\\
95.63	0.00884278880531625\\
95.64	0.00884616194521943\\
95.65	0.00884953443269615\\
95.66	0.00885290625990299\\
95.67	0.00885627741897245\\
95.68	0.00885964790201289\\
95.69	0.00886301770110849\\
95.7	0.00886638680831918\\
95.71	0.00886975521568061\\
95.72	0.00887312291520406\\
95.73	0.00887648989887641\\
95.74	0.0088798561586601\\
95.75	0.00888322168649304\\
95.76	0.00888658647428859\\
95.77	0.00888995051393546\\
95.78	0.00889331379729774\\
95.79	0.00889667631621474\\
95.8	0.00890003806250103\\
95.81	0.00890339902794632\\
95.82	0.00890675920431545\\
95.83	0.00891011858334829\\
95.84	0.00891347715675975\\
95.85	0.00891683491623968\\
95.86	0.0089201918534528\\
95.87	0.0089235479600387\\
95.88	0.00892690322761175\\
95.89	0.00893025764776106\\
95.9	0.00893361121205041\\
95.91	0.00893696391201822\\
95.92	0.00894031573917746\\
95.93	0.00894366668501566\\
95.94	0.00894701674099477\\
95.95	0.00895036589855118\\
95.96	0.00895371414909562\\
95.97	0.00895706148401315\\
95.98	0.00896040789466304\\
95.99	0.0089637533723788\\
96	0.00896709790846804\\
96.01	0.00897044149421249\\
96.02	0.00897378412086791\\
96.03	0.00897712577966402\\
96.04	0.00898046646180448\\
96.05	0.00898380615846684\\
96.06	0.00898714486080244\\
96.07	0.0089904825599364\\
96.08	0.00899381924696756\\
96.09	0.00899715491296841\\
96.1	0.00900048954898505\\
96.11	0.00900382314603711\\
96.12	0.00900715569511776\\
96.13	0.00901048718719357\\
96.14	0.00901381761320453\\
96.15	0.00901714696406396\\
96.16	0.00902047523065846\\
96.17	0.00902380240384787\\
96.18	0.00902712847446519\\
96.19	0.00903045343331657\\
96.2	0.00903377727118121\\
96.21	0.00903709997881133\\
96.22	0.00904042154693213\\
96.23	0.0090437419662417\\
96.24	0.00904706122741101\\
96.25	0.00905037932108382\\
96.26	0.00905369623787665\\
96.27	0.00905701196837871\\
96.28	0.00906032650315189\\
96.29	0.00906363983273062\\
96.3	0.00906695194762191\\
96.31	0.00907026283830525\\
96.32	0.00907357249523258\\
96.33	0.00907688090882819\\
96.34	0.00908018806948873\\
96.35	0.00908349396758312\\
96.36	0.0090867985934525\\
96.37	0.0090901019374102\\
96.38	0.00909340398974166\\
96.39	0.0090967047407044\\
96.4	0.00910000418052793\\
96.41	0.00910330229941377\\
96.42	0.00910659908753532\\
96.43	0.00910989453503787\\
96.44	0.00911318863203848\\
96.45	0.00911648136862602\\
96.46	0.00911977273486103\\
96.47	0.00912306272077572\\
96.48	0.00912635131637391\\
96.49	0.00912963851163097\\
96.5	0.00913292429649376\\
96.51	0.00913620866088061\\
96.52	0.00913949159468125\\
96.53	0.00914277308775675\\
96.54	0.00914605312993947\\
96.55	0.00914933171103306\\
96.56	0.00915260882081232\\
96.57	0.00915588444902323\\
96.58	0.00915915858538284\\
96.59	0.00916243121957929\\
96.6	0.00916570234127167\\
96.61	0.00916897194009006\\
96.62	0.00917224000563539\\
96.63	0.0091755065274795\\
96.64	0.00917877149516496\\
96.65	0.00918203489820516\\
96.66	0.00918529672608412\\
96.67	0.00918855696825658\\
96.68	0.00919181561414782\\
96.69	0.00919507265315372\\
96.7	0.00919832807464064\\
96.71	0.00920158186794539\\
96.72	0.00920483402237523\\
96.73	0.00920808452720773\\
96.74	0.00921133337169079\\
96.75	0.00921458054504259\\
96.76	0.0092178260364515\\
96.77	0.00922106983507607\\
96.78	0.00922431193004498\\
96.79	0.00922755231045697\\
96.8	0.00923079096538081\\
96.81	0.00923402788385525\\
96.82	0.00923726305488898\\
96.83	0.00924049646746057\\
96.84	0.00924372811051844\\
96.85	0.00924695797298078\\
96.86	0.00925018604373554\\
96.87	0.0092534123116404\\
96.88	0.00925663676552265\\
96.89	0.00925985939417923\\
96.9	0.00926308018637662\\
96.91	0.00926629913085083\\
96.92	0.00926951621630735\\
96.93	0.00927273143142111\\
96.94	0.0092759447648364\\
96.95	0.00927915620516689\\
96.96	0.00928236574099551\\
96.97	0.00928557336087448\\
96.98	0.00928877905332521\\
96.99	0.00929198280683828\\
97	0.00929518460987343\\
97.01	0.00929838445085943\\
97.02	0.00930158231819412\\
97.03	0.00930477820024435\\
97.04	0.00930797208534589\\
97.05	0.00931116396180347\\
97.06	0.00931435381789066\\
97.07	0.00931754164184987\\
97.08	0.0093207274218923\\
97.09	0.0093239111461979\\
97.1	0.00932709280291535\\
97.11	0.00933027238016196\\
97.12	0.0093334498660237\\
97.13	0.00933662524855512\\
97.14	0.00933979851577933\\
97.15	0.00934296965568792\\
97.16	0.009346138656241\\
97.17	0.00934930550536707\\
97.18	0.00935247019096306\\
97.19	0.00935563270089424\\
97.2	0.00935879302299419\\
97.21	0.0093619511450648\\
97.22	0.00936510705487618\\
97.23	0.00936826074016668\\
97.24	0.00937141218864279\\
97.25	0.00937456138797916\\
97.26	0.00937770832581853\\
97.27	0.00938085298977171\\
97.28	0.00938399536741754\\
97.29	0.00938713544630286\\
97.3	0.00939027321394247\\
97.31	0.0093934086578191\\
97.32	0.00939654176538336\\
97.33	0.00939967252405375\\
97.34	0.00940280092121657\\
97.35	0.00940592694422593\\
97.36	0.00940905058040371\\
97.37	0.00941217181703951\\
97.38	0.00941529064139062\\
97.39	0.00941840704068204\\
97.4	0.00942152100210638\\
97.41	0.00942463251282384\\
97.42	0.00942774155996225\\
97.43	0.00943084813061695\\
97.44	0.0094339522118508\\
97.45	0.00943705379069419\\
97.46	0.00944015285414494\\
97.47	0.00944324938916831\\
97.48	0.00944634338269697\\
97.49	0.009449434821631\\
97.5	0.0094525236928378\\
97.51	0.00945560998315212\\
97.52	0.009458693679376\\
97.53	0.00946177476827877\\
97.54	0.00946485323659702\\
97.55	0.00946792907103456\\
97.56	0.00947100225826241\\
97.57	0.00947407278491878\\
97.58	0.00947714063760904\\
97.59	0.00948020580290569\\
97.6	0.00948326826734836\\
97.61	0.00948632801744377\\
97.62	0.00948938503966571\\
97.63	0.00949243932045505\\
97.64	0.00949549084621966\\
97.65	0.00949853960333446\\
97.66	0.00950158557814135\\
97.67	0.00950462875694921\\
97.68	0.00950766912603389\\
97.69	0.0095107066716382\\
97.7	0.00951374137997184\\
97.71	0.00951677323721146\\
97.72	0.0095198022295006\\
97.73	0.00952282834294968\\
97.74	0.00952585156363598\\
97.75	0.00952887187760365\\
97.76	0.00953188927086369\\
97.77	0.00953490372939391\\
97.78	0.00953791523913895\\
97.79	0.00954092378601027\\
97.8	0.00954392935588611\\
97.81	0.00954693193461148\\
97.82	0.00954993150799817\\
97.83	0.00955292806182471\\
97.84	0.00955592158183638\\
97.85	0.00955891205374521\\
97.86	0.00956189946322995\\
97.87	0.00956488379593608\\
97.88	0.00956786503747582\\
97.89	0.0095708431734281\\
97.9	0.00957381818933817\\
97.91	0.00957679007071763\\
97.92	0.00957975880304432\\
97.93	0.00958272437176239\\
97.94	0.00958568676228229\\
97.95	0.0095886459599807\\
97.96	0.0095916019502006\\
97.97	0.00959455471825123\\
97.98	0.00959750424940809\\
97.99	0.00960045052891294\\
98	0.00960339354197378\\
98.01	0.0096063332737912\\
98.02	0.00960926970957048\\
98.03	0.00961220283448236\\
98.04	0.00961513263366306\\
98.05	0.00961805909221424\\
98.06	0.00962098219520301\\
98.07	0.00962390192766177\\
98.08	0.00962681827577519\\
98.09	0.00962973122604842\\
98.1	0.00963264076497008\\
98.11	0.00963554687901242\\
98.12	0.00963844955463153\\
98.13	0.00964134877826755\\
98.14	0.00964424453634482\\
98.15	0.00964713681527215\\
98.16	0.00965002560144298\\
98.17	0.0096529108812356\\
98.18	0.00965579264101337\\
98.19	0.00965867086721747\\
98.2	0.00966154554629619\\
98.21	0.00966441666477487\\
98.22	0.009667284209581\\
98.23	0.00967014816762863\\
98.24	0.0096730085258186\\
98.25	0.00967586527103869\\
98.26	0.00967871839016394\\
98.27	0.00968156787005677\\
98.28	0.0096844136975673\\
98.29	0.00968725585953354\\
98.3	0.00969009413741489\\
98.31	0.00969292847736932\\
98.32	0.00969575886699815\\
98.33	0.00969858481230294\\
98.34	0.00970140116599026\\
98.35	0.00970420786165537\\
98.36	0.00970700483236056\\
98.37	0.00970979162671079\\
98.38	0.00971256816126521\\
98.39	0.00971533436570355\\
98.4	0.00971809016913108\\
98.41	0.00972083550007231\\
98.42	0.00972357028646462\\
98.43	0.00972629306455234\\
98.44	0.00972900332028077\\
98.45	0.00973170097090251\\
98.46	0.00973438593299963\\
98.47	0.00973705812247618\\
98.48	0.0097397174545505\\
98.49	0.00974236384374756\\
98.5	0.00974499720389108\\
98.51	0.00974761744809552\\
98.52	0.00975022448875802\\
98.53	0.00975281823755019\\
98.54	0.00975539860540969\\
98.55	0.00975796550253184\\
98.56	0.00976051883836092\\
98.57	0.00976305852158147\\
98.58	0.00976558446010939\\
98.59	0.0097680965610829\\
98.6	0.00977059473085338\\
98.61	0.00977307887497604\\
98.62	0.00977554889820042\\
98.63	0.00977800470446083\\
98.64	0.00978044619686648\\
98.65	0.00978287327769162\\
98.66	0.00978528584836517\\
98.67	0.00978768380946042\\
98.68	0.00979006706068459\\
98.69	0.00979243572973505\\
98.7	0.00979478978845448\\
98.71	0.00979712913527064\\
98.72	0.00979945366771696\\
98.73	0.00980176328319591\\
98.74	0.00980405868274031\\
98.75	0.00980634379333697\\
98.76	0.00980861855607513\\
98.77	0.00981088291163054\\
98.78	0.00981313680026141\\
98.79	0.00981538129038395\\
98.8	0.00981761665092274\\
98.81	0.00981984282880382\\
98.82	0.00982205977059208\\
98.83	0.00982426742275659\\
98.84	0.00982646573159911\\
98.85	0.00982865464305535\\
98.86	0.00983083410269208\\
98.87	0.00983300405570406\\
98.88	0.00983516444691103\\
98.89	0.00983731522075465\\
98.9	0.00983945632129542\\
98.91	0.00984158769220961\\
98.92	0.0098437092767861\\
98.93	0.00984582101792329\\
98.94	0.00984792285812586\\
98.95	0.00985001473950168\\
98.96	0.00985209660375852\\
98.97	0.00985416839220086\\
98.98	0.00985623004572663\\
98.99	0.00985828150482393\\
99	0.00986032270956777\\
99.01	0.0098623535996167\\
99.02	0.00986437411420948\\
99.03	0.00986638419216177\\
99.04	0.0098683837718627\\
99.05	0.00987037279127147\\
99.06	0.00987235118791395\\
99.07	0.00987431889887921\\
99.08	0.00987627586053859\\
99.09	0.00987822200859097\\
99.1	0.00988015727829076\\
99.11	0.00988208160444435\\
99.12	0.00988399492112197\\
99.13	0.00988589716186539\\
99.14	0.00988778825975373\\
99.15	0.00988966814739988\\
99.16	0.00989153675694688\\
99.17	0.00989339402006423\\
99.18	0.00989523986794429\\
99.19	0.00989707423129856\\
99.2	0.00989889704035396\\
99.21	0.00990070822484916\\
99.22	0.00990250771403083\\
99.23	0.00990429543664984\\
99.24	0.00990607132095756\\
99.25	0.00990783529470205\\
99.26	0.00990958728512422\\
99.27	0.00991132721895409\\
99.28	0.0099130550224069\\
99.29	0.00991477062117929\\
99.3	0.00991647394044544\\
99.31	0.00991816490485323\\
99.32	0.00991984343852032\\
99.33	0.00992150946503027\\
99.34	0.00992316290742868\\
99.35	0.00992480368821925\\
99.36	0.00992643172935984\\
99.37	0.00992804695225863\\
99.38	0.00992964927777009\\
99.39	0.00993123862619115\\
99.4	0.00993281491725719\\
99.41	0.00993437807013812\\
99.42	0.00993592800343448\\
99.43	0.00993746463517346\\
99.44	0.00993898788280498\\
99.45	0.00994049766319776\\
99.46	0.00994199389263541\\
99.47	0.00994347648681246\\
99.48	0.00994494536083052\\
99.49	0.00994640042919431\\
99.5	0.00994784160580784\\
99.51	0.00994926880397047\\
99.52	0.00995068193637311\\
99.53	0.00995208091509435\\
99.54	0.00995346565159667\\
99.55	0.00995483605672261\\
99.56	0.00995619204069105\\
99.57	0.00995753351309345\\
99.58	0.00995886038289013\\
99.59	0.00996017255840666\\
99.6	0.00996146994733017\\
99.61	0.00996275245671045\\
99.62	0.00996401999295883\\
99.63	0.00996527246184481\\
99.64	0.00996650976849268\\
99.65	0.00996773181737827\\
99.66	0.0099689385123257\\
99.67	0.00997012975650425\\
99.68	0.00997130545242531\\
99.69	0.00997246550193929\\
99.7	0.00997360980623282\\
99.71	0.009974738248558\\
99.72	0.00997585070992424\\
99.73	0.00997694707046292\\
99.74	0.00997802720942307\\
99.75	0.00997909100516722\\
99.76	0.00998013833516726\\
99.77	0.00998116907600054\\
99.78	0.00998218310334597\\
99.79	0.00998318029198038\\
99.8	0.00998416051577496\\
99.81	0.00998512364769184\\
99.82	0.00998606955978089\\
99.83	0.00998699812317667\\
99.84	0.00998790920809558\\
99.85	0.00998880268383318\\
99.86	0.0099896784187618\\
99.87	0.00999053628032826\\
99.88	0.00999137613505194\\
99.89	0.00999219784852308\\
99.9	0.00999300128540131\\
99.91	0.0099937863094145\\
99.92	0.00999455278335794\\
99.93	0.00999530056909381\\
99.94	0.009996029527551\\
99.95	0.00999673951872531\\
99.96	0.00999743040168006\\
99.97	0.00999810203454702\\
99.98	0.00999875427452794\\
99.99	0.00999938697789635\\
100	0.01\\
};
\addlegendentry{$q=1$};

\addplot [color=red,solid,forget plot]
  table[row sep=crcr]{%
0.01	0\\
0.02	0\\
0.03	0\\
0.04	0\\
0.05	0\\
0.06	0\\
0.07	0\\
0.08	0\\
0.09	0\\
0.1	0\\
0.11	0\\
0.12	0\\
0.13	0\\
0.14	0\\
0.15	0\\
0.16	0\\
0.17	0\\
0.18	0\\
0.19	0\\
0.2	0\\
0.21	0\\
0.22	0\\
0.23	0\\
0.24	0\\
0.25	0\\
0.26	0\\
0.27	0\\
0.28	0\\
0.29	0\\
0.3	0\\
0.31	0\\
0.32	0\\
0.33	0\\
0.34	0\\
0.35	0\\
0.36	0\\
0.37	0\\
0.38	0\\
0.39	0\\
0.4	0\\
0.41	0\\
0.42	0\\
0.43	0\\
0.44	0\\
0.45	0\\
0.46	0\\
0.47	0\\
0.48	0\\
0.49	0\\
0.5	0\\
0.51	0\\
0.52	0\\
0.53	0\\
0.54	0\\
0.55	0\\
0.56	0\\
0.57	0\\
0.58	0\\
0.59	0\\
0.6	0\\
0.61	0\\
0.62	0\\
0.63	0\\
0.64	0\\
0.65	0\\
0.66	0\\
0.67	0\\
0.68	0\\
0.69	0\\
0.7	0\\
0.71	0\\
0.72	0\\
0.73	0\\
0.74	0\\
0.75	0\\
0.76	0\\
0.77	0\\
0.78	0\\
0.79	0\\
0.8	0\\
0.81	0\\
0.82	0\\
0.83	0\\
0.84	0\\
0.85	0\\
0.86	0\\
0.87	0\\
0.88	0\\
0.89	0\\
0.9	0\\
0.91	0\\
0.92	0\\
0.93	0\\
0.94	0\\
0.95	0\\
0.96	0\\
0.97	0\\
0.98	0\\
0.99	0\\
1	0\\
1.01	0\\
1.02	0\\
1.03	0\\
1.04	0\\
1.05	0\\
1.06	0\\
1.07	0\\
1.08	0\\
1.09	0\\
1.1	0\\
1.11	0\\
1.12	0\\
1.13	0\\
1.14	0\\
1.15	0\\
1.16	0\\
1.17	0\\
1.18	0\\
1.19	0\\
1.2	0\\
1.21	0\\
1.22	0\\
1.23	0\\
1.24	0\\
1.25	0\\
1.26	0\\
1.27	0\\
1.28	0\\
1.29	0\\
1.3	0\\
1.31	0\\
1.32	0\\
1.33	0\\
1.34	0\\
1.35	0\\
1.36	0\\
1.37	0\\
1.38	0\\
1.39	0\\
1.4	0\\
1.41	0\\
1.42	0\\
1.43	0\\
1.44	0\\
1.45	0\\
1.46	0\\
1.47	0\\
1.48	0\\
1.49	0\\
1.5	0\\
1.51	0\\
1.52	0\\
1.53	0\\
1.54	0\\
1.55	0\\
1.56	0\\
1.57	0\\
1.58	0\\
1.59	0\\
1.6	0\\
1.61	0\\
1.62	0\\
1.63	0\\
1.64	0\\
1.65	0\\
1.66	0\\
1.67	0\\
1.68	0\\
1.69	0\\
1.7	0\\
1.71	0\\
1.72	0\\
1.73	0\\
1.74	0\\
1.75	0\\
1.76	0\\
1.77	0\\
1.78	0\\
1.79	0\\
1.8	0\\
1.81	0\\
1.82	0\\
1.83	0\\
1.84	0\\
1.85	0\\
1.86	0\\
1.87	0\\
1.88	0\\
1.89	0\\
1.9	0\\
1.91	0\\
1.92	0\\
1.93	0\\
1.94	0\\
1.95	0\\
1.96	0\\
1.97	0\\
1.98	0\\
1.99	0\\
2	0\\
2.01	0\\
2.02	0\\
2.03	0\\
2.04	0\\
2.05	0\\
2.06	0\\
2.07	0\\
2.08	0\\
2.09	0\\
2.1	0\\
2.11	0\\
2.12	0\\
2.13	0\\
2.14	0\\
2.15	0\\
2.16	0\\
2.17	0\\
2.18	0\\
2.19	0\\
2.2	0\\
2.21	0\\
2.22	0\\
2.23	0\\
2.24	0\\
2.25	0\\
2.26	0\\
2.27	0\\
2.28	0\\
2.29	0\\
2.3	0\\
2.31	0\\
2.32	0\\
2.33	0\\
2.34	0\\
2.35	0\\
2.36	0\\
2.37	0\\
2.38	0\\
2.39	0\\
2.4	0\\
2.41	0\\
2.42	0\\
2.43	0\\
2.44	0\\
2.45	0\\
2.46	0\\
2.47	0\\
2.48	0\\
2.49	0\\
2.5	0\\
2.51	0\\
2.52	0\\
2.53	0\\
2.54	0\\
2.55	0\\
2.56	0\\
2.57	0\\
2.58	0\\
2.59	0\\
2.6	0\\
2.61	0\\
2.62	0\\
2.63	0\\
2.64	0\\
2.65	0\\
2.66	0\\
2.67	0\\
2.68	0\\
2.69	0\\
2.7	0\\
2.71	0\\
2.72	0\\
2.73	0\\
2.74	0\\
2.75	0\\
2.76	0\\
2.77	0\\
2.78	0\\
2.79	0\\
2.8	0\\
2.81	0\\
2.82	0\\
2.83	0\\
2.84	0\\
2.85	0\\
2.86	0\\
2.87	0\\
2.88	0\\
2.89	0\\
2.9	0\\
2.91	0\\
2.92	0\\
2.93	0\\
2.94	0\\
2.95	0\\
2.96	0\\
2.97	0\\
2.98	0\\
2.99	0\\
3	0\\
3.01	0\\
3.02	0\\
3.03	0\\
3.04	0\\
3.05	0\\
3.06	0\\
3.07	0\\
3.08	0\\
3.09	0\\
3.1	0\\
3.11	0\\
3.12	0\\
3.13	0\\
3.14	0\\
3.15	0\\
3.16	0\\
3.17	0\\
3.18	0\\
3.19	0\\
3.2	0\\
3.21	0\\
3.22	0\\
3.23	0\\
3.24	0\\
3.25	0\\
3.26	0\\
3.27	0\\
3.28	0\\
3.29	0\\
3.3	0\\
3.31	0\\
3.32	0\\
3.33	0\\
3.34	0\\
3.35	0\\
3.36	0\\
3.37	0\\
3.38	0\\
3.39	0\\
3.4	0\\
3.41	0\\
3.42	0\\
3.43	0\\
3.44	0\\
3.45	0\\
3.46	0\\
3.47	0\\
3.48	0\\
3.49	0\\
3.5	0\\
3.51	0\\
3.52	0\\
3.53	0\\
3.54	0\\
3.55	0\\
3.56	0\\
3.57	0\\
3.58	0\\
3.59	0\\
3.6	0\\
3.61	0\\
3.62	0\\
3.63	0\\
3.64	0\\
3.65	0\\
3.66	0\\
3.67	0\\
3.68	0\\
3.69	0\\
3.7	0\\
3.71	0\\
3.72	0\\
3.73	0\\
3.74	0\\
3.75	0\\
3.76	0\\
3.77	0\\
3.78	0\\
3.79	0\\
3.8	0\\
3.81	0\\
3.82	0\\
3.83	0\\
3.84	0\\
3.85	0\\
3.86	0\\
3.87	0\\
3.88	0\\
3.89	0\\
3.9	0\\
3.91	0\\
3.92	0\\
3.93	0\\
3.94	0\\
3.95	0\\
3.96	0\\
3.97	0\\
3.98	0\\
3.99	0\\
4	0\\
4.01	0\\
4.02	0\\
4.03	0\\
4.04	0\\
4.05	0\\
4.06	0\\
4.07	0\\
4.08	0\\
4.09	0\\
4.1	0\\
4.11	0\\
4.12	0\\
4.13	0\\
4.14	0\\
4.15	0\\
4.16	0\\
4.17	0\\
4.18	0\\
4.19	0\\
4.2	0\\
4.21	0\\
4.22	0\\
4.23	0\\
4.24	0\\
4.25	0\\
4.26	0\\
4.27	0\\
4.28	0\\
4.29	0\\
4.3	0\\
4.31	0\\
4.32	0\\
4.33	0\\
4.34	0\\
4.35	0\\
4.36	0\\
4.37	0\\
4.38	0\\
4.39	0\\
4.4	0\\
4.41	0\\
4.42	0\\
4.43	0\\
4.44	0\\
4.45	0\\
4.46	0\\
4.47	0\\
4.48	0\\
4.49	0\\
4.5	0\\
4.51	0\\
4.52	0\\
4.53	0\\
4.54	0\\
4.55	0\\
4.56	0\\
4.57	0\\
4.58	0\\
4.59	0\\
4.6	0\\
4.61	0\\
4.62	0\\
4.63	0\\
4.64	0\\
4.65	0\\
4.66	0\\
4.67	0\\
4.68	0\\
4.69	0\\
4.7	0\\
4.71	0\\
4.72	0\\
4.73	0\\
4.74	0\\
4.75	0\\
4.76	0\\
4.77	0\\
4.78	0\\
4.79	0\\
4.8	0\\
4.81	0\\
4.82	0\\
4.83	0\\
4.84	0\\
4.85	0\\
4.86	0\\
4.87	0\\
4.88	0\\
4.89	0\\
4.9	0\\
4.91	0\\
4.92	0\\
4.93	0\\
4.94	0\\
4.95	0\\
4.96	0\\
4.97	0\\
4.98	0\\
4.99	0\\
5	0\\
5.01	0\\
5.02	0\\
5.03	0\\
5.04	0\\
5.05	0\\
5.06	0\\
5.07	0\\
5.08	0\\
5.09	0\\
5.1	0\\
5.11	0\\
5.12	0\\
5.13	0\\
5.14	0\\
5.15	0\\
5.16	0\\
5.17	0\\
5.18	0\\
5.19	0\\
5.2	0\\
5.21	0\\
5.22	0\\
5.23	0\\
5.24	0\\
5.25	0\\
5.26	0\\
5.27	0\\
5.28	0\\
5.29	0\\
5.3	0\\
5.31	0\\
5.32	0\\
5.33	0\\
5.34	0\\
5.35	0\\
5.36	0\\
5.37	0\\
5.38	0\\
5.39	0\\
5.4	0\\
5.41	0\\
5.42	0\\
5.43	0\\
5.44	0\\
5.45	0\\
5.46	0\\
5.47	0\\
5.48	0\\
5.49	0\\
5.5	0\\
5.51	0\\
5.52	0\\
5.53	0\\
5.54	0\\
5.55	0\\
5.56	0\\
5.57	0\\
5.58	0\\
5.59	0\\
5.6	0\\
5.61	0\\
5.62	0\\
5.63	0\\
5.64	0\\
5.65	0\\
5.66	0\\
5.67	0\\
5.68	0\\
5.69	0\\
5.7	0\\
5.71	0\\
5.72	0\\
5.73	0\\
5.74	0\\
5.75	0\\
5.76	0\\
5.77	0\\
5.78	0\\
5.79	0\\
5.8	0\\
5.81	0\\
5.82	0\\
5.83	0\\
5.84	0\\
5.85	0\\
5.86	0\\
5.87	0\\
5.88	0\\
5.89	0\\
5.9	0\\
5.91	0\\
5.92	0\\
5.93	0\\
5.94	0\\
5.95	0\\
5.96	0\\
5.97	0\\
5.98	0\\
5.99	0\\
6	0\\
6.01	0\\
6.02	0\\
6.03	0\\
6.04	0\\
6.05	0\\
6.06	0\\
6.07	0\\
6.08	0\\
6.09	0\\
6.1	0\\
6.11	0\\
6.12	0\\
6.13	0\\
6.14	0\\
6.15	0\\
6.16	0\\
6.17	0\\
6.18	0\\
6.19	0\\
6.2	0\\
6.21	0\\
6.22	0\\
6.23	0\\
6.24	0\\
6.25	0\\
6.26	0\\
6.27	0\\
6.28	0\\
6.29	0\\
6.3	0\\
6.31	0\\
6.32	0\\
6.33	0\\
6.34	0\\
6.35	0\\
6.36	0\\
6.37	0\\
6.38	0\\
6.39	0\\
6.4	0\\
6.41	0\\
6.42	0\\
6.43	0\\
6.44	0\\
6.45	0\\
6.46	0\\
6.47	0\\
6.48	0\\
6.49	0\\
6.5	0\\
6.51	0\\
6.52	0\\
6.53	0\\
6.54	0\\
6.55	0\\
6.56	0\\
6.57	0\\
6.58	0\\
6.59	0\\
6.6	0\\
6.61	0\\
6.62	0\\
6.63	0\\
6.64	0\\
6.65	0\\
6.66	0\\
6.67	0\\
6.68	0\\
6.69	0\\
6.7	0\\
6.71	0\\
6.72	0\\
6.73	0\\
6.74	0\\
6.75	0\\
6.76	0\\
6.77	0\\
6.78	0\\
6.79	0\\
6.8	0\\
6.81	0\\
6.82	0\\
6.83	0\\
6.84	0\\
6.85	0\\
6.86	0\\
6.87	0\\
6.88	0\\
6.89	0\\
6.9	0\\
6.91	0\\
6.92	0\\
6.93	0\\
6.94	0\\
6.95	0\\
6.96	0\\
6.97	0\\
6.98	0\\
6.99	0\\
7	0\\
7.01	0\\
7.02	0\\
7.03	0\\
7.04	0\\
7.05	0\\
7.06	0\\
7.07	0\\
7.08	0\\
7.09	0\\
7.1	0\\
7.11	0\\
7.12	0\\
7.13	0\\
7.14	0\\
7.15	0\\
7.16	0\\
7.17	0\\
7.18	0\\
7.19	0\\
7.2	0\\
7.21	0\\
7.22	0\\
7.23	0\\
7.24	0\\
7.25	0\\
7.26	0\\
7.27	0\\
7.28	0\\
7.29	0\\
7.3	0\\
7.31	0\\
7.32	0\\
7.33	0\\
7.34	0\\
7.35	0\\
7.36	0\\
7.37	0\\
7.38	0\\
7.39	0\\
7.4	0\\
7.41	0\\
7.42	0\\
7.43	0\\
7.44	0\\
7.45	0\\
7.46	0\\
7.47	0\\
7.48	0\\
7.49	0\\
7.5	0\\
7.51	0\\
7.52	0\\
7.53	0\\
7.54	0\\
7.55	0\\
7.56	0\\
7.57	0\\
7.58	0\\
7.59	0\\
7.6	0\\
7.61	0\\
7.62	0\\
7.63	0\\
7.64	0\\
7.65	0\\
7.66	0\\
7.67	0\\
7.68	0\\
7.69	0\\
7.7	0\\
7.71	0\\
7.72	0\\
7.73	0\\
7.74	0\\
7.75	0\\
7.76	0\\
7.77	0\\
7.78	0\\
7.79	0\\
7.8	0\\
7.81	0\\
7.82	0\\
7.83	0\\
7.84	0\\
7.85	0\\
7.86	0\\
7.87	0\\
7.88	0\\
7.89	0\\
7.9	0\\
7.91	0\\
7.92	0\\
7.93	0\\
7.94	0\\
7.95	0\\
7.96	0\\
7.97	0\\
7.98	0\\
7.99	0\\
8	0\\
8.01	0\\
8.02	0\\
8.03	0\\
8.04	0\\
8.05	0\\
8.06	0\\
8.07	0\\
8.08	0\\
8.09	0\\
8.1	0\\
8.11	0\\
8.12	0\\
8.13	0\\
8.14	0\\
8.15	0\\
8.16	0\\
8.17	0\\
8.18	0\\
8.19	0\\
8.2	0\\
8.21	0\\
8.22	0\\
8.23	0\\
8.24	0\\
8.25	0\\
8.26	0\\
8.27	0\\
8.28	0\\
8.29	0\\
8.3	0\\
8.31	0\\
8.32	0\\
8.33	0\\
8.34	0\\
8.35	0\\
8.36	0\\
8.37	0\\
8.38	0\\
8.39	0\\
8.4	0\\
8.41	0\\
8.42	0\\
8.43	0\\
8.44	0\\
8.45	0\\
8.46	0\\
8.47	0\\
8.48	0\\
8.49	0\\
8.5	0\\
8.51	0\\
8.52	0\\
8.53	0\\
8.54	0\\
8.55	0\\
8.56	0\\
8.57	0\\
8.58	0\\
8.59	0\\
8.6	0\\
8.61	0\\
8.62	0\\
8.63	0\\
8.64	0\\
8.65	0\\
8.66	0\\
8.67	0\\
8.68	0\\
8.69	0\\
8.7	0\\
8.71	0\\
8.72	0\\
8.73	0\\
8.74	0\\
8.75	0\\
8.76	0\\
8.77	0\\
8.78	0\\
8.79	0\\
8.8	0\\
8.81	0\\
8.82	0\\
8.83	0\\
8.84	0\\
8.85	0\\
8.86	0\\
8.87	0\\
8.88	0\\
8.89	0\\
8.9	0\\
8.91	0\\
8.92	0\\
8.93	0\\
8.94	0\\
8.95	0\\
8.96	0\\
8.97	0\\
8.98	0\\
8.99	0\\
9	0\\
9.01	0\\
9.02	0\\
9.03	0\\
9.04	0\\
9.05	0\\
9.06	0\\
9.07	0\\
9.08	0\\
9.09	0\\
9.1	0\\
9.11	0\\
9.12	0\\
9.13	0\\
9.14	0\\
9.15	0\\
9.16	0\\
9.17	0\\
9.18	0\\
9.19	0\\
9.2	0\\
9.21	0\\
9.22	0\\
9.23	0\\
9.24	0\\
9.25	0\\
9.26	0\\
9.27	0\\
9.28	0\\
9.29	0\\
9.3	0\\
9.31	0\\
9.32	0\\
9.33	0\\
9.34	0\\
9.35	0\\
9.36	0\\
9.37	0\\
9.38	0\\
9.39	0\\
9.4	0\\
9.41	0\\
9.42	0\\
9.43	0\\
9.44	0\\
9.45	0\\
9.46	0\\
9.47	0\\
9.48	0\\
9.49	0\\
9.5	0\\
9.51	0\\
9.52	0\\
9.53	0\\
9.54	0\\
9.55	0\\
9.56	0\\
9.57	0\\
9.58	0\\
9.59	0\\
9.6	0\\
9.61	0\\
9.62	0\\
9.63	0\\
9.64	0\\
9.65	0\\
9.66	0\\
9.67	0\\
9.68	0\\
9.69	0\\
9.7	0\\
9.71	0\\
9.72	0\\
9.73	0\\
9.74	0\\
9.75	0\\
9.76	0\\
9.77	0\\
9.78	0\\
9.79	0\\
9.8	0\\
9.81	0\\
9.82	0\\
9.83	0\\
9.84	0\\
9.85	0\\
9.86	0\\
9.87	0\\
9.88	0\\
9.89	0\\
9.9	0\\
9.91	0\\
9.92	0\\
9.93	0\\
9.94	0\\
9.95	0\\
9.96	0\\
9.97	0\\
9.98	0\\
9.99	0\\
10	0\\
10.01	0\\
10.02	0\\
10.03	0\\
10.04	0\\
10.05	0\\
10.06	0\\
10.07	0\\
10.08	0\\
10.09	0\\
10.1	0\\
10.11	0\\
10.12	0\\
10.13	0\\
10.14	0\\
10.15	0\\
10.16	0\\
10.17	0\\
10.18	0\\
10.19	0\\
10.2	0\\
10.21	0\\
10.22	0\\
10.23	0\\
10.24	0\\
10.25	0\\
10.26	0\\
10.27	0\\
10.28	0\\
10.29	0\\
10.3	0\\
10.31	0\\
10.32	0\\
10.33	0\\
10.34	0\\
10.35	0\\
10.36	0\\
10.37	0\\
10.38	0\\
10.39	0\\
10.4	0\\
10.41	0\\
10.42	0\\
10.43	0\\
10.44	0\\
10.45	0\\
10.46	0\\
10.47	0\\
10.48	0\\
10.49	0\\
10.5	0\\
10.51	0\\
10.52	0\\
10.53	0\\
10.54	0\\
10.55	0\\
10.56	0\\
10.57	0\\
10.58	0\\
10.59	0\\
10.6	0\\
10.61	0\\
10.62	0\\
10.63	0\\
10.64	0\\
10.65	0\\
10.66	0\\
10.67	0\\
10.68	0\\
10.69	0\\
10.7	0\\
10.71	0\\
10.72	0\\
10.73	0\\
10.74	0\\
10.75	0\\
10.76	0\\
10.77	0\\
10.78	0\\
10.79	0\\
10.8	0\\
10.81	0\\
10.82	0\\
10.83	0\\
10.84	0\\
10.85	0\\
10.86	0\\
10.87	0\\
10.88	0\\
10.89	0\\
10.9	0\\
10.91	0\\
10.92	0\\
10.93	0\\
10.94	0\\
10.95	0\\
10.96	0\\
10.97	0\\
10.98	0\\
10.99	0\\
11	0\\
11.01	0\\
11.02	0\\
11.03	0\\
11.04	0\\
11.05	0\\
11.06	0\\
11.07	0\\
11.08	0\\
11.09	0\\
11.1	0\\
11.11	0\\
11.12	0\\
11.13	0\\
11.14	0\\
11.15	0\\
11.16	0\\
11.17	0\\
11.18	0\\
11.19	0\\
11.2	0\\
11.21	0\\
11.22	0\\
11.23	0\\
11.24	0\\
11.25	0\\
11.26	0\\
11.27	0\\
11.28	0\\
11.29	0\\
11.3	0\\
11.31	0\\
11.32	0\\
11.33	0\\
11.34	0\\
11.35	0\\
11.36	0\\
11.37	0\\
11.38	0\\
11.39	0\\
11.4	0\\
11.41	0\\
11.42	0\\
11.43	0\\
11.44	0\\
11.45	0\\
11.46	0\\
11.47	0\\
11.48	0\\
11.49	0\\
11.5	0\\
11.51	0\\
11.52	0\\
11.53	0\\
11.54	0\\
11.55	0\\
11.56	0\\
11.57	0\\
11.58	0\\
11.59	0\\
11.6	0\\
11.61	0\\
11.62	0\\
11.63	0\\
11.64	0\\
11.65	0\\
11.66	0\\
11.67	0\\
11.68	0\\
11.69	0\\
11.7	0\\
11.71	0\\
11.72	0\\
11.73	0\\
11.74	0\\
11.75	0\\
11.76	0\\
11.77	0\\
11.78	0\\
11.79	0\\
11.8	0\\
11.81	0\\
11.82	0\\
11.83	0\\
11.84	0\\
11.85	0\\
11.86	0\\
11.87	0\\
11.88	0\\
11.89	0\\
11.9	0\\
11.91	0\\
11.92	0\\
11.93	0\\
11.94	0\\
11.95	0\\
11.96	0\\
11.97	0\\
11.98	0\\
11.99	0\\
12	0\\
12.01	0\\
12.02	0\\
12.03	0\\
12.04	0\\
12.05	0\\
12.06	0\\
12.07	0\\
12.08	0\\
12.09	0\\
12.1	0\\
12.11	0\\
12.12	0\\
12.13	0\\
12.14	0\\
12.15	0\\
12.16	0\\
12.17	0\\
12.18	0\\
12.19	0\\
12.2	0\\
12.21	0\\
12.22	0\\
12.23	0\\
12.24	0\\
12.25	0\\
12.26	0\\
12.27	0\\
12.28	0\\
12.29	0\\
12.3	0\\
12.31	0\\
12.32	0\\
12.33	0\\
12.34	0\\
12.35	0\\
12.36	0\\
12.37	0\\
12.38	0\\
12.39	0\\
12.4	0\\
12.41	0\\
12.42	0\\
12.43	0\\
12.44	0\\
12.45	0\\
12.46	0\\
12.47	0\\
12.48	0\\
12.49	0\\
12.5	0\\
12.51	0\\
12.52	0\\
12.53	0\\
12.54	0\\
12.55	0\\
12.56	0\\
12.57	0\\
12.58	0\\
12.59	0\\
12.6	0\\
12.61	0\\
12.62	0\\
12.63	0\\
12.64	0\\
12.65	0\\
12.66	0\\
12.67	0\\
12.68	0\\
12.69	0\\
12.7	0\\
12.71	0\\
12.72	0\\
12.73	0\\
12.74	0\\
12.75	0\\
12.76	0\\
12.77	0\\
12.78	0\\
12.79	0\\
12.8	0\\
12.81	0\\
12.82	0\\
12.83	0\\
12.84	0\\
12.85	0\\
12.86	0\\
12.87	0\\
12.88	0\\
12.89	0\\
12.9	0\\
12.91	0\\
12.92	0\\
12.93	0\\
12.94	0\\
12.95	0\\
12.96	0\\
12.97	0\\
12.98	0\\
12.99	0\\
13	0\\
13.01	0\\
13.02	0\\
13.03	0\\
13.04	0\\
13.05	0\\
13.06	0\\
13.07	0\\
13.08	0\\
13.09	0\\
13.1	0\\
13.11	0\\
13.12	0\\
13.13	0\\
13.14	0\\
13.15	0\\
13.16	0\\
13.17	0\\
13.18	0\\
13.19	0\\
13.2	0\\
13.21	0\\
13.22	0\\
13.23	0\\
13.24	0\\
13.25	0\\
13.26	0\\
13.27	0\\
13.28	0\\
13.29	0\\
13.3	0\\
13.31	0\\
13.32	0\\
13.33	0\\
13.34	0\\
13.35	0\\
13.36	0\\
13.37	0\\
13.38	0\\
13.39	0\\
13.4	0\\
13.41	0\\
13.42	0\\
13.43	0\\
13.44	0\\
13.45	0\\
13.46	0\\
13.47	0\\
13.48	0\\
13.49	0\\
13.5	0\\
13.51	0\\
13.52	0\\
13.53	0\\
13.54	0\\
13.55	0\\
13.56	0\\
13.57	0\\
13.58	0\\
13.59	0\\
13.6	0\\
13.61	0\\
13.62	0\\
13.63	0\\
13.64	0\\
13.65	0\\
13.66	0\\
13.67	0\\
13.68	0\\
13.69	0\\
13.7	0\\
13.71	0\\
13.72	0\\
13.73	0\\
13.74	0\\
13.75	0\\
13.76	0\\
13.77	0\\
13.78	0\\
13.79	0\\
13.8	0\\
13.81	0\\
13.82	0\\
13.83	0\\
13.84	0\\
13.85	0\\
13.86	0\\
13.87	0\\
13.88	0\\
13.89	0\\
13.9	0\\
13.91	0\\
13.92	0\\
13.93	0\\
13.94	0\\
13.95	0\\
13.96	0\\
13.97	0\\
13.98	0\\
13.99	0\\
14	0\\
14.01	0\\
14.02	0\\
14.03	0\\
14.04	0\\
14.05	0\\
14.06	0\\
14.07	0\\
14.08	0\\
14.09	0\\
14.1	0\\
14.11	0\\
14.12	0\\
14.13	0\\
14.14	0\\
14.15	0\\
14.16	0\\
14.17	0\\
14.18	0\\
14.19	0\\
14.2	0\\
14.21	0\\
14.22	0\\
14.23	0\\
14.24	0\\
14.25	0\\
14.26	0\\
14.27	0\\
14.28	0\\
14.29	0\\
14.3	0\\
14.31	0\\
14.32	0\\
14.33	0\\
14.34	0\\
14.35	0\\
14.36	0\\
14.37	0\\
14.38	0\\
14.39	0\\
14.4	0\\
14.41	0\\
14.42	0\\
14.43	0\\
14.44	0\\
14.45	0\\
14.46	0\\
14.47	0\\
14.48	0\\
14.49	0\\
14.5	0\\
14.51	0\\
14.52	0\\
14.53	0\\
14.54	0\\
14.55	0\\
14.56	0\\
14.57	0\\
14.58	0\\
14.59	0\\
14.6	0\\
14.61	0\\
14.62	0\\
14.63	0\\
14.64	0\\
14.65	0\\
14.66	0\\
14.67	0\\
14.68	0\\
14.69	0\\
14.7	0\\
14.71	0\\
14.72	0\\
14.73	0\\
14.74	0\\
14.75	0\\
14.76	0\\
14.77	0\\
14.78	0\\
14.79	0\\
14.8	0\\
14.81	0\\
14.82	0\\
14.83	0\\
14.84	0\\
14.85	0\\
14.86	0\\
14.87	0\\
14.88	0\\
14.89	0\\
14.9	0\\
14.91	0\\
14.92	0\\
14.93	0\\
14.94	0\\
14.95	0\\
14.96	0\\
14.97	0\\
14.98	0\\
14.99	0\\
15	0\\
15.01	0\\
15.02	0\\
15.03	0\\
15.04	0\\
15.05	0\\
15.06	0\\
15.07	0\\
15.08	0\\
15.09	0\\
15.1	0\\
15.11	0\\
15.12	0\\
15.13	0\\
15.14	0\\
15.15	0\\
15.16	0\\
15.17	0\\
15.18	0\\
15.19	0\\
15.2	0\\
15.21	0\\
15.22	0\\
15.23	0\\
15.24	0\\
15.25	0\\
15.26	0\\
15.27	0\\
15.28	0\\
15.29	0\\
15.3	0\\
15.31	0\\
15.32	0\\
15.33	0\\
15.34	0\\
15.35	0\\
15.36	0\\
15.37	0\\
15.38	0\\
15.39	0\\
15.4	0\\
15.41	0\\
15.42	0\\
15.43	0\\
15.44	0\\
15.45	0\\
15.46	0\\
15.47	0\\
15.48	0\\
15.49	0\\
15.5	0\\
15.51	0\\
15.52	0\\
15.53	0\\
15.54	0\\
15.55	0\\
15.56	0\\
15.57	0\\
15.58	0\\
15.59	0\\
15.6	0\\
15.61	0\\
15.62	0\\
15.63	0\\
15.64	0\\
15.65	0\\
15.66	0\\
15.67	0\\
15.68	0\\
15.69	0\\
15.7	0\\
15.71	0\\
15.72	0\\
15.73	0\\
15.74	0\\
15.75	0\\
15.76	0\\
15.77	0\\
15.78	0\\
15.79	0\\
15.8	0\\
15.81	0\\
15.82	0\\
15.83	0\\
15.84	0\\
15.85	0\\
15.86	0\\
15.87	0\\
15.88	0\\
15.89	0\\
15.9	0\\
15.91	0\\
15.92	0\\
15.93	0\\
15.94	0\\
15.95	0\\
15.96	0\\
15.97	0\\
15.98	0\\
15.99	0\\
16	0\\
16.01	0\\
16.02	0\\
16.03	0\\
16.04	0\\
16.05	0\\
16.06	0\\
16.07	0\\
16.08	0\\
16.09	0\\
16.1	0\\
16.11	0\\
16.12	0\\
16.13	0\\
16.14	0\\
16.15	0\\
16.16	0\\
16.17	0\\
16.18	0\\
16.19	0\\
16.2	0\\
16.21	0\\
16.22	0\\
16.23	0\\
16.24	0\\
16.25	0\\
16.26	0\\
16.27	0\\
16.28	0\\
16.29	0\\
16.3	0\\
16.31	0\\
16.32	0\\
16.33	0\\
16.34	0\\
16.35	0\\
16.36	0\\
16.37	0\\
16.38	0\\
16.39	0\\
16.4	0\\
16.41	0\\
16.42	0\\
16.43	0\\
16.44	0\\
16.45	0\\
16.46	0\\
16.47	0\\
16.48	0\\
16.49	0\\
16.5	0\\
16.51	0\\
16.52	0\\
16.53	0\\
16.54	0\\
16.55	0\\
16.56	0\\
16.57	0\\
16.58	0\\
16.59	0\\
16.6	0\\
16.61	0\\
16.62	0\\
16.63	0\\
16.64	0\\
16.65	0\\
16.66	0\\
16.67	0\\
16.68	0\\
16.69	0\\
16.7	0\\
16.71	0\\
16.72	0\\
16.73	0\\
16.74	0\\
16.75	0\\
16.76	0\\
16.77	0\\
16.78	0\\
16.79	0\\
16.8	0\\
16.81	0\\
16.82	0\\
16.83	0\\
16.84	0\\
16.85	0\\
16.86	0\\
16.87	0\\
16.88	0\\
16.89	0\\
16.9	0\\
16.91	0\\
16.92	0\\
16.93	0\\
16.94	0\\
16.95	0\\
16.96	0\\
16.97	0\\
16.98	0\\
16.99	0\\
17	0\\
17.01	0\\
17.02	0\\
17.03	0\\
17.04	0\\
17.05	0\\
17.06	0\\
17.07	0\\
17.08	0\\
17.09	0\\
17.1	0\\
17.11	0\\
17.12	0\\
17.13	0\\
17.14	0\\
17.15	0\\
17.16	0\\
17.17	0\\
17.18	0\\
17.19	0\\
17.2	0\\
17.21	0\\
17.22	0\\
17.23	0\\
17.24	0\\
17.25	0\\
17.26	0\\
17.27	0\\
17.28	0\\
17.29	0\\
17.3	0\\
17.31	0\\
17.32	0\\
17.33	0\\
17.34	0\\
17.35	0\\
17.36	0\\
17.37	0\\
17.38	0\\
17.39	0\\
17.4	0\\
17.41	0\\
17.42	0\\
17.43	0\\
17.44	0\\
17.45	0\\
17.46	0\\
17.47	0\\
17.48	0\\
17.49	0\\
17.5	0\\
17.51	0\\
17.52	0\\
17.53	0\\
17.54	0\\
17.55	0\\
17.56	0\\
17.57	0\\
17.58	0\\
17.59	0\\
17.6	0\\
17.61	0\\
17.62	0\\
17.63	0\\
17.64	0\\
17.65	0\\
17.66	0\\
17.67	0\\
17.68	0\\
17.69	0\\
17.7	0\\
17.71	0\\
17.72	0\\
17.73	0\\
17.74	0\\
17.75	0\\
17.76	0\\
17.77	0\\
17.78	0\\
17.79	0\\
17.8	0\\
17.81	0\\
17.82	0\\
17.83	0\\
17.84	0\\
17.85	0\\
17.86	0\\
17.87	0\\
17.88	0\\
17.89	0\\
17.9	0\\
17.91	0\\
17.92	0\\
17.93	0\\
17.94	0\\
17.95	0\\
17.96	0\\
17.97	0\\
17.98	0\\
17.99	0\\
18	0\\
18.01	0\\
18.02	0\\
18.03	0\\
18.04	0\\
18.05	0\\
18.06	0\\
18.07	0\\
18.08	0\\
18.09	0\\
18.1	0\\
18.11	0\\
18.12	0\\
18.13	0\\
18.14	0\\
18.15	0\\
18.16	0\\
18.17	0\\
18.18	0\\
18.19	0\\
18.2	0\\
18.21	0\\
18.22	0\\
18.23	0\\
18.24	0\\
18.25	0\\
18.26	0\\
18.27	0\\
18.28	0\\
18.29	0\\
18.3	0\\
18.31	0\\
18.32	0\\
18.33	0\\
18.34	0\\
18.35	0\\
18.36	0\\
18.37	0\\
18.38	0\\
18.39	0\\
18.4	0\\
18.41	0\\
18.42	0\\
18.43	0\\
18.44	0\\
18.45	0\\
18.46	0\\
18.47	0\\
18.48	0\\
18.49	0\\
18.5	0\\
18.51	0\\
18.52	0\\
18.53	0\\
18.54	0\\
18.55	0\\
18.56	0\\
18.57	0\\
18.58	0\\
18.59	0\\
18.6	0\\
18.61	0\\
18.62	0\\
18.63	0\\
18.64	0\\
18.65	0\\
18.66	0\\
18.67	0\\
18.68	0\\
18.69	0\\
18.7	0\\
18.71	0\\
18.72	0\\
18.73	0\\
18.74	0\\
18.75	0\\
18.76	0\\
18.77	0\\
18.78	0\\
18.79	0\\
18.8	0\\
18.81	0\\
18.82	0\\
18.83	0\\
18.84	0\\
18.85	0\\
18.86	0\\
18.87	0\\
18.88	0\\
18.89	0\\
18.9	0\\
18.91	0\\
18.92	0\\
18.93	0\\
18.94	0\\
18.95	0\\
18.96	0\\
18.97	0\\
18.98	0\\
18.99	0\\
19	0\\
19.01	0\\
19.02	0\\
19.03	0\\
19.04	0\\
19.05	0\\
19.06	0\\
19.07	0\\
19.08	0\\
19.09	0\\
19.1	0\\
19.11	0\\
19.12	0\\
19.13	0\\
19.14	0\\
19.15	0\\
19.16	0\\
19.17	0\\
19.18	0\\
19.19	0\\
19.2	0\\
19.21	0\\
19.22	0\\
19.23	0\\
19.24	0\\
19.25	0\\
19.26	0\\
19.27	0\\
19.28	0\\
19.29	0\\
19.3	0\\
19.31	0\\
19.32	0\\
19.33	0\\
19.34	0\\
19.35	0\\
19.36	0\\
19.37	0\\
19.38	0\\
19.39	0\\
19.4	0\\
19.41	0\\
19.42	0\\
19.43	0\\
19.44	0\\
19.45	0\\
19.46	0\\
19.47	0\\
19.48	0\\
19.49	0\\
19.5	0\\
19.51	0\\
19.52	0\\
19.53	0\\
19.54	0\\
19.55	0\\
19.56	0\\
19.57	0\\
19.58	0\\
19.59	0\\
19.6	0\\
19.61	0\\
19.62	0\\
19.63	0\\
19.64	0\\
19.65	0\\
19.66	0\\
19.67	0\\
19.68	0\\
19.69	0\\
19.7	0\\
19.71	0\\
19.72	0\\
19.73	0\\
19.74	0\\
19.75	0\\
19.76	0\\
19.77	0\\
19.78	0\\
19.79	0\\
19.8	0\\
19.81	0\\
19.82	0\\
19.83	0\\
19.84	0\\
19.85	0\\
19.86	0\\
19.87	0\\
19.88	0\\
19.89	0\\
19.9	0\\
19.91	0\\
19.92	0\\
19.93	0\\
19.94	0\\
19.95	0\\
19.96	0\\
19.97	0\\
19.98	0\\
19.99	0\\
20	0\\
20.01	0\\
20.02	0\\
20.03	0\\
20.04	0\\
20.05	0\\
20.06	0\\
20.07	0\\
20.08	0\\
20.09	0\\
20.1	0\\
20.11	0\\
20.12	0\\
20.13	0\\
20.14	0\\
20.15	0\\
20.16	0\\
20.17	0\\
20.18	0\\
20.19	0\\
20.2	0\\
20.21	0\\
20.22	0\\
20.23	0\\
20.24	0\\
20.25	0\\
20.26	0\\
20.27	0\\
20.28	0\\
20.29	0\\
20.3	0\\
20.31	0\\
20.32	0\\
20.33	0\\
20.34	0\\
20.35	0\\
20.36	0\\
20.37	0\\
20.38	0\\
20.39	0\\
20.4	0\\
20.41	0\\
20.42	0\\
20.43	0\\
20.44	0\\
20.45	0\\
20.46	0\\
20.47	0\\
20.48	0\\
20.49	0\\
20.5	0\\
20.51	0\\
20.52	0\\
20.53	0\\
20.54	0\\
20.55	0\\
20.56	0\\
20.57	0\\
20.58	0\\
20.59	0\\
20.6	0\\
20.61	0\\
20.62	0\\
20.63	0\\
20.64	0\\
20.65	0\\
20.66	0\\
20.67	0\\
20.68	0\\
20.69	0\\
20.7	0\\
20.71	0\\
20.72	0\\
20.73	0\\
20.74	0\\
20.75	0\\
20.76	0\\
20.77	0\\
20.78	0\\
20.79	0\\
20.8	0\\
20.81	0\\
20.82	0\\
20.83	0\\
20.84	0\\
20.85	0\\
20.86	0\\
20.87	0\\
20.88	0\\
20.89	0\\
20.9	0\\
20.91	0\\
20.92	0\\
20.93	0\\
20.94	0\\
20.95	0\\
20.96	0\\
20.97	0\\
20.98	0\\
20.99	0\\
21	0\\
21.01	0\\
21.02	0\\
21.03	0\\
21.04	0\\
21.05	0\\
21.06	0\\
21.07	0\\
21.08	0\\
21.09	0\\
21.1	0\\
21.11	0\\
21.12	0\\
21.13	0\\
21.14	0\\
21.15	0\\
21.16	0\\
21.17	0\\
21.18	0\\
21.19	0\\
21.2	0\\
21.21	0\\
21.22	0\\
21.23	0\\
21.24	0\\
21.25	0\\
21.26	0\\
21.27	0\\
21.28	0\\
21.29	0\\
21.3	0\\
21.31	0\\
21.32	0\\
21.33	0\\
21.34	0\\
21.35	0\\
21.36	0\\
21.37	0\\
21.38	0\\
21.39	0\\
21.4	0\\
21.41	0\\
21.42	0\\
21.43	0\\
21.44	0\\
21.45	0\\
21.46	0\\
21.47	0\\
21.48	0\\
21.49	0\\
21.5	0\\
21.51	0\\
21.52	0\\
21.53	0\\
21.54	0\\
21.55	0\\
21.56	0\\
21.57	0\\
21.58	0\\
21.59	0\\
21.6	0\\
21.61	0\\
21.62	0\\
21.63	0\\
21.64	0\\
21.65	0\\
21.66	0\\
21.67	0\\
21.68	0\\
21.69	0\\
21.7	0\\
21.71	0\\
21.72	0\\
21.73	0\\
21.74	0\\
21.75	0\\
21.76	0\\
21.77	0\\
21.78	0\\
21.79	0\\
21.8	0\\
21.81	0\\
21.82	0\\
21.83	0\\
21.84	0\\
21.85	0\\
21.86	0\\
21.87	0\\
21.88	0\\
21.89	0\\
21.9	0\\
21.91	0\\
21.92	0\\
21.93	0\\
21.94	0\\
21.95	0\\
21.96	0\\
21.97	0\\
21.98	0\\
21.99	0\\
22	0\\
22.01	0\\
22.02	0\\
22.03	0\\
22.04	0\\
22.05	0\\
22.06	0\\
22.07	0\\
22.08	0\\
22.09	0\\
22.1	0\\
22.11	0\\
22.12	0\\
22.13	0\\
22.14	0\\
22.15	0\\
22.16	0\\
22.17	0\\
22.18	0\\
22.19	0\\
22.2	0\\
22.21	0\\
22.22	0\\
22.23	0\\
22.24	0\\
22.25	0\\
22.26	0\\
22.27	0\\
22.28	0\\
22.29	0\\
22.3	0\\
22.31	0\\
22.32	0\\
22.33	0\\
22.34	0\\
22.35	0\\
22.36	0\\
22.37	0\\
22.38	0\\
22.39	0\\
22.4	0\\
22.41	0\\
22.42	0\\
22.43	0\\
22.44	0\\
22.45	0\\
22.46	0\\
22.47	0\\
22.48	0\\
22.49	0\\
22.5	0\\
22.51	0\\
22.52	0\\
22.53	0\\
22.54	0\\
22.55	0\\
22.56	0\\
22.57	0\\
22.58	0\\
22.59	0\\
22.6	0\\
22.61	0\\
22.62	0\\
22.63	0\\
22.64	0\\
22.65	0\\
22.66	0\\
22.67	0\\
22.68	0\\
22.69	0\\
22.7	0\\
22.71	0\\
22.72	0\\
22.73	0\\
22.74	0\\
22.75	0\\
22.76	0\\
22.77	0\\
22.78	0\\
22.79	0\\
22.8	0\\
22.81	0\\
22.82	0\\
22.83	0\\
22.84	0\\
22.85	0\\
22.86	0\\
22.87	0\\
22.88	0\\
22.89	0\\
22.9	0\\
22.91	0\\
22.92	0\\
22.93	0\\
22.94	0\\
22.95	0\\
22.96	0\\
22.97	0\\
22.98	0\\
22.99	0\\
23	0\\
23.01	0\\
23.02	0\\
23.03	0\\
23.04	0\\
23.05	0\\
23.06	0\\
23.07	0\\
23.08	0\\
23.09	0\\
23.1	0\\
23.11	0\\
23.12	0\\
23.13	0\\
23.14	0\\
23.15	0\\
23.16	0\\
23.17	0\\
23.18	0\\
23.19	0\\
23.2	0\\
23.21	0\\
23.22	0\\
23.23	0\\
23.24	0\\
23.25	0\\
23.26	0\\
23.27	0\\
23.28	0\\
23.29	0\\
23.3	0\\
23.31	0\\
23.32	0\\
23.33	0\\
23.34	0\\
23.35	0\\
23.36	0\\
23.37	0\\
23.38	0\\
23.39	0\\
23.4	0\\
23.41	0\\
23.42	0\\
23.43	0\\
23.44	0\\
23.45	0\\
23.46	0\\
23.47	0\\
23.48	0\\
23.49	0\\
23.5	0\\
23.51	0\\
23.52	0\\
23.53	0\\
23.54	0\\
23.55	0\\
23.56	0\\
23.57	0\\
23.58	0\\
23.59	0\\
23.6	0\\
23.61	0\\
23.62	0\\
23.63	0\\
23.64	0\\
23.65	0\\
23.66	0\\
23.67	0\\
23.68	0\\
23.69	0\\
23.7	0\\
23.71	0\\
23.72	0\\
23.73	0\\
23.74	0\\
23.75	0\\
23.76	0\\
23.77	0\\
23.78	0\\
23.79	0\\
23.8	0\\
23.81	0\\
23.82	0\\
23.83	0\\
23.84	0\\
23.85	0\\
23.86	0\\
23.87	0\\
23.88	0\\
23.89	0\\
23.9	0\\
23.91	0\\
23.92	0\\
23.93	0\\
23.94	0\\
23.95	0\\
23.96	0\\
23.97	0\\
23.98	0\\
23.99	0\\
24	0\\
24.01	0\\
24.02	0\\
24.03	0\\
24.04	0\\
24.05	0\\
24.06	0\\
24.07	0\\
24.08	0\\
24.09	0\\
24.1	0\\
24.11	0\\
24.12	0\\
24.13	0\\
24.14	0\\
24.15	0\\
24.16	0\\
24.17	0\\
24.18	0\\
24.19	0\\
24.2	0\\
24.21	0\\
24.22	0\\
24.23	0\\
24.24	0\\
24.25	0\\
24.26	0\\
24.27	0\\
24.28	0\\
24.29	0\\
24.3	0\\
24.31	0\\
24.32	0\\
24.33	0\\
24.34	0\\
24.35	0\\
24.36	0\\
24.37	0\\
24.38	0\\
24.39	0\\
24.4	0\\
24.41	0\\
24.42	0\\
24.43	0\\
24.44	0\\
24.45	0\\
24.46	0\\
24.47	0\\
24.48	0\\
24.49	0\\
24.5	0\\
24.51	0\\
24.52	0\\
24.53	0\\
24.54	0\\
24.55	0\\
24.56	0\\
24.57	0\\
24.58	0\\
24.59	0\\
24.6	0\\
24.61	0\\
24.62	0\\
24.63	0\\
24.64	0\\
24.65	0\\
24.66	0\\
24.67	0\\
24.68	0\\
24.69	0\\
24.7	0\\
24.71	0\\
24.72	0\\
24.73	0\\
24.74	0\\
24.75	0\\
24.76	0\\
24.77	0\\
24.78	0\\
24.79	0\\
24.8	0\\
24.81	0\\
24.82	0\\
24.83	0\\
24.84	0\\
24.85	0\\
24.86	0\\
24.87	0\\
24.88	0\\
24.89	0\\
24.9	0\\
24.91	0\\
24.92	0\\
24.93	0\\
24.94	0\\
24.95	0\\
24.96	0\\
24.97	0\\
24.98	0\\
24.99	0\\
25	0\\
25.01	0\\
25.02	0\\
25.03	0\\
25.04	0\\
25.05	0\\
25.06	0\\
25.07	0\\
25.08	0\\
25.09	0\\
25.1	0\\
25.11	0\\
25.12	0\\
25.13	0\\
25.14	0\\
25.15	0\\
25.16	0\\
25.17	0\\
25.18	0\\
25.19	0\\
25.2	0\\
25.21	0\\
25.22	0\\
25.23	0\\
25.24	0\\
25.25	0\\
25.26	0\\
25.27	0\\
25.28	0\\
25.29	0\\
25.3	0\\
25.31	0\\
25.32	0\\
25.33	0\\
25.34	0\\
25.35	0\\
25.36	0\\
25.37	0\\
25.38	0\\
25.39	0\\
25.4	0\\
25.41	0\\
25.42	0\\
25.43	0\\
25.44	0\\
25.45	0\\
25.46	0\\
25.47	0\\
25.48	0\\
25.49	0\\
25.5	0\\
25.51	0\\
25.52	0\\
25.53	0\\
25.54	0\\
25.55	0\\
25.56	0\\
25.57	0\\
25.58	0\\
25.59	0\\
25.6	0\\
25.61	0\\
25.62	0\\
25.63	0\\
25.64	0\\
25.65	0\\
25.66	0\\
25.67	0\\
25.68	0\\
25.69	0\\
25.7	0\\
25.71	0\\
25.72	0\\
25.73	0\\
25.74	0\\
25.75	0\\
25.76	0\\
25.77	0\\
25.78	0\\
25.79	0\\
25.8	0\\
25.81	0\\
25.82	0\\
25.83	0\\
25.84	0\\
25.85	0\\
25.86	0\\
25.87	0\\
25.88	0\\
25.89	0\\
25.9	0\\
25.91	0\\
25.92	0\\
25.93	0\\
25.94	0\\
25.95	0\\
25.96	0\\
25.97	0\\
25.98	0\\
25.99	0\\
26	0\\
26.01	0\\
26.02	0\\
26.03	0\\
26.04	0\\
26.05	0\\
26.06	0\\
26.07	0\\
26.08	0\\
26.09	0\\
26.1	0\\
26.11	0\\
26.12	0\\
26.13	0\\
26.14	0\\
26.15	0\\
26.16	0\\
26.17	0\\
26.18	0\\
26.19	0\\
26.2	0\\
26.21	0\\
26.22	0\\
26.23	0\\
26.24	0\\
26.25	0\\
26.26	0\\
26.27	0\\
26.28	0\\
26.29	0\\
26.3	0\\
26.31	0\\
26.32	0\\
26.33	0\\
26.34	0\\
26.35	0\\
26.36	0\\
26.37	0\\
26.38	0\\
26.39	0\\
26.4	0\\
26.41	0\\
26.42	0\\
26.43	0\\
26.44	0\\
26.45	0\\
26.46	0\\
26.47	0\\
26.48	0\\
26.49	0\\
26.5	0\\
26.51	0\\
26.52	0\\
26.53	0\\
26.54	0\\
26.55	0\\
26.56	0\\
26.57	0\\
26.58	0\\
26.59	0\\
26.6	0\\
26.61	0\\
26.62	0\\
26.63	0\\
26.64	0\\
26.65	0\\
26.66	0\\
26.67	0\\
26.68	0\\
26.69	0\\
26.7	0\\
26.71	0\\
26.72	0\\
26.73	0\\
26.74	0\\
26.75	0\\
26.76	0\\
26.77	0\\
26.78	0\\
26.79	0\\
26.8	0\\
26.81	0\\
26.82	0\\
26.83	0\\
26.84	0\\
26.85	0\\
26.86	0\\
26.87	0\\
26.88	0\\
26.89	0\\
26.9	0\\
26.91	0\\
26.92	0\\
26.93	0\\
26.94	0\\
26.95	0\\
26.96	0\\
26.97	0\\
26.98	0\\
26.99	0\\
27	0\\
27.01	0\\
27.02	0\\
27.03	0\\
27.04	0\\
27.05	0\\
27.06	0\\
27.07	0\\
27.08	0\\
27.09	0\\
27.1	0\\
27.11	0\\
27.12	0\\
27.13	0\\
27.14	0\\
27.15	0\\
27.16	0\\
27.17	0\\
27.18	0\\
27.19	0\\
27.2	0\\
27.21	0\\
27.22	0\\
27.23	0\\
27.24	0\\
27.25	0\\
27.26	0\\
27.27	0\\
27.28	0\\
27.29	0\\
27.3	0\\
27.31	0\\
27.32	0\\
27.33	0\\
27.34	0\\
27.35	0\\
27.36	0\\
27.37	0\\
27.38	0\\
27.39	0\\
27.4	0\\
27.41	0\\
27.42	0\\
27.43	0\\
27.44	0\\
27.45	0\\
27.46	0\\
27.47	0\\
27.48	0\\
27.49	0\\
27.5	0\\
27.51	0\\
27.52	0\\
27.53	0\\
27.54	0\\
27.55	0\\
27.56	0\\
27.57	0\\
27.58	0\\
27.59	0\\
27.6	0\\
27.61	0\\
27.62	0\\
27.63	0\\
27.64	0\\
27.65	0\\
27.66	0\\
27.67	0\\
27.68	0\\
27.69	0\\
27.7	0\\
27.71	0\\
27.72	0\\
27.73	0\\
27.74	0\\
27.75	0\\
27.76	0\\
27.77	0\\
27.78	0\\
27.79	0\\
27.8	0\\
27.81	0\\
27.82	0\\
27.83	0\\
27.84	0\\
27.85	0\\
27.86	0\\
27.87	0\\
27.88	0\\
27.89	0\\
27.9	0\\
27.91	0\\
27.92	0\\
27.93	0\\
27.94	0\\
27.95	0\\
27.96	0\\
27.97	0\\
27.98	0\\
27.99	0\\
28	0\\
28.01	0\\
28.02	0\\
28.03	0\\
28.04	0\\
28.05	0\\
28.06	0\\
28.07	0\\
28.08	0\\
28.09	0\\
28.1	0\\
28.11	0\\
28.12	0\\
28.13	0\\
28.14	0\\
28.15	0\\
28.16	0\\
28.17	0\\
28.18	0\\
28.19	0\\
28.2	0\\
28.21	0\\
28.22	0\\
28.23	0\\
28.24	0\\
28.25	0\\
28.26	0\\
28.27	0\\
28.28	0\\
28.29	0\\
28.3	0\\
28.31	0\\
28.32	0\\
28.33	0\\
28.34	0\\
28.35	0\\
28.36	0\\
28.37	0\\
28.38	0\\
28.39	0\\
28.4	0\\
28.41	0\\
28.42	0\\
28.43	0\\
28.44	0\\
28.45	0\\
28.46	0\\
28.47	0\\
28.48	0\\
28.49	0\\
28.5	0\\
28.51	0\\
28.52	0\\
28.53	0\\
28.54	0\\
28.55	0\\
28.56	0\\
28.57	0\\
28.58	0\\
28.59	0\\
28.6	0\\
28.61	0\\
28.62	0\\
28.63	0\\
28.64	0\\
28.65	0\\
28.66	0\\
28.67	0\\
28.68	0\\
28.69	0\\
28.7	0\\
28.71	0\\
28.72	0\\
28.73	0\\
28.74	0\\
28.75	0\\
28.76	0\\
28.77	0\\
28.78	0\\
28.79	0\\
28.8	0\\
28.81	0\\
28.82	0\\
28.83	0\\
28.84	0\\
28.85	0\\
28.86	0\\
28.87	0\\
28.88	0\\
28.89	0\\
28.9	0\\
28.91	0\\
28.92	0\\
28.93	0\\
28.94	0\\
28.95	0\\
28.96	0\\
28.97	0\\
28.98	0\\
28.99	0\\
29	0\\
29.01	0\\
29.02	0\\
29.03	0\\
29.04	0\\
29.05	0\\
29.06	0\\
29.07	0\\
29.08	0\\
29.09	0\\
29.1	0\\
29.11	0\\
29.12	0\\
29.13	0\\
29.14	0\\
29.15	0\\
29.16	0\\
29.17	0\\
29.18	0\\
29.19	0\\
29.2	0\\
29.21	0\\
29.22	0\\
29.23	0\\
29.24	0\\
29.25	0\\
29.26	0\\
29.27	0\\
29.28	0\\
29.29	0\\
29.3	0\\
29.31	0\\
29.32	0\\
29.33	0\\
29.34	0\\
29.35	0\\
29.36	0\\
29.37	0\\
29.38	0\\
29.39	0\\
29.4	0\\
29.41	0\\
29.42	0\\
29.43	0\\
29.44	0\\
29.45	0\\
29.46	0\\
29.47	0\\
29.48	0\\
29.49	0\\
29.5	0\\
29.51	0\\
29.52	0\\
29.53	0\\
29.54	0\\
29.55	0\\
29.56	0\\
29.57	0\\
29.58	0\\
29.59	0\\
29.6	0\\
29.61	0\\
29.62	0\\
29.63	0\\
29.64	0\\
29.65	0\\
29.66	0\\
29.67	0\\
29.68	0\\
29.69	0\\
29.7	0\\
29.71	0\\
29.72	0\\
29.73	0\\
29.74	0\\
29.75	0\\
29.76	0\\
29.77	0\\
29.78	0\\
29.79	0\\
29.8	0\\
29.81	0\\
29.82	0\\
29.83	0\\
29.84	0\\
29.85	0\\
29.86	0\\
29.87	0\\
29.88	0\\
29.89	0\\
29.9	0\\
29.91	0\\
29.92	0\\
29.93	0\\
29.94	0\\
29.95	0\\
29.96	0\\
29.97	0\\
29.98	0\\
29.99	0\\
30	0\\
30.01	0\\
30.02	0\\
30.03	0\\
30.04	0\\
30.05	0\\
30.06	0\\
30.07	0\\
30.08	0\\
30.09	0\\
30.1	0\\
30.11	0\\
30.12	0\\
30.13	0\\
30.14	0\\
30.15	0\\
30.16	0\\
30.17	0\\
30.18	0\\
30.19	0\\
30.2	0\\
30.21	0\\
30.22	0\\
30.23	0\\
30.24	0\\
30.25	0\\
30.26	0\\
30.27	0\\
30.28	0\\
30.29	0\\
30.3	0\\
30.31	0\\
30.32	0\\
30.33	0\\
30.34	0\\
30.35	0\\
30.36	0\\
30.37	0\\
30.38	0\\
30.39	0\\
30.4	0\\
30.41	0\\
30.42	0\\
30.43	0\\
30.44	0\\
30.45	0\\
30.46	0\\
30.47	0\\
30.48	0\\
30.49	0\\
30.5	0\\
30.51	0\\
30.52	0\\
30.53	0\\
30.54	0\\
30.55	0\\
30.56	0\\
30.57	0\\
30.58	0\\
30.59	0\\
30.6	0\\
30.61	0\\
30.62	0\\
30.63	0\\
30.64	0\\
30.65	0\\
30.66	0\\
30.67	0\\
30.68	0\\
30.69	0\\
30.7	0\\
30.71	0\\
30.72	0\\
30.73	0\\
30.74	0\\
30.75	0\\
30.76	0\\
30.77	0\\
30.78	0\\
30.79	0\\
30.8	0\\
30.81	0\\
30.82	0\\
30.83	0\\
30.84	0\\
30.85	0\\
30.86	0\\
30.87	0\\
30.88	0\\
30.89	0\\
30.9	0\\
30.91	0\\
30.92	0\\
30.93	0\\
30.94	0\\
30.95	0\\
30.96	0\\
30.97	0\\
30.98	0\\
30.99	0\\
31	0\\
31.01	0\\
31.02	0\\
31.03	0\\
31.04	0\\
31.05	0\\
31.06	0\\
31.07	0\\
31.08	0\\
31.09	0\\
31.1	0\\
31.11	0\\
31.12	0\\
31.13	0\\
31.14	0\\
31.15	0\\
31.16	0\\
31.17	0\\
31.18	0\\
31.19	0\\
31.2	0\\
31.21	0\\
31.22	0\\
31.23	0\\
31.24	0\\
31.25	0\\
31.26	0\\
31.27	0\\
31.28	0\\
31.29	0\\
31.3	0\\
31.31	0\\
31.32	0\\
31.33	0\\
31.34	0\\
31.35	0\\
31.36	0\\
31.37	0\\
31.38	0\\
31.39	0\\
31.4	0\\
31.41	0\\
31.42	0\\
31.43	0\\
31.44	0\\
31.45	0\\
31.46	0\\
31.47	0\\
31.48	0\\
31.49	0\\
31.5	0\\
31.51	0\\
31.52	0\\
31.53	0\\
31.54	0\\
31.55	0\\
31.56	0\\
31.57	0\\
31.58	0\\
31.59	0\\
31.6	0\\
31.61	0\\
31.62	0\\
31.63	0\\
31.64	0\\
31.65	0\\
31.66	0\\
31.67	0\\
31.68	0\\
31.69	0\\
31.7	0\\
31.71	0\\
31.72	0\\
31.73	0\\
31.74	0\\
31.75	0\\
31.76	0\\
31.77	0\\
31.78	0\\
31.79	0\\
31.8	0\\
31.81	0\\
31.82	0\\
31.83	0\\
31.84	0\\
31.85	0\\
31.86	0\\
31.87	0\\
31.88	0\\
31.89	0\\
31.9	0\\
31.91	0\\
31.92	0\\
31.93	0\\
31.94	0\\
31.95	0\\
31.96	0\\
31.97	0\\
31.98	0\\
31.99	0\\
32	0\\
32.01	0\\
32.02	0\\
32.03	0\\
32.04	0\\
32.05	0\\
32.06	0\\
32.07	0\\
32.08	0\\
32.09	0\\
32.1	0\\
32.11	0\\
32.12	0\\
32.13	0\\
32.14	0\\
32.15	0\\
32.16	0\\
32.17	0\\
32.18	0\\
32.19	0\\
32.2	0\\
32.21	0\\
32.22	0\\
32.23	0\\
32.24	0\\
32.25	0\\
32.26	0\\
32.27	0\\
32.28	0\\
32.29	0\\
32.3	0\\
32.31	0\\
32.32	0\\
32.33	0\\
32.34	0\\
32.35	0\\
32.36	0\\
32.37	0\\
32.38	0\\
32.39	0\\
32.4	0\\
32.41	0\\
32.42	0\\
32.43	0\\
32.44	0\\
32.45	0\\
32.46	0\\
32.47	0\\
32.48	0\\
32.49	0\\
32.5	0\\
32.51	0\\
32.52	0\\
32.53	0\\
32.54	0\\
32.55	0\\
32.56	0\\
32.57	0\\
32.58	0\\
32.59	0\\
32.6	0\\
32.61	0\\
32.62	0\\
32.63	0\\
32.64	0\\
32.65	0\\
32.66	0\\
32.67	0\\
32.68	0\\
32.69	0\\
32.7	0\\
32.71	0\\
32.72	0\\
32.73	0\\
32.74	0\\
32.75	0\\
32.76	0\\
32.77	0\\
32.78	0\\
32.79	0\\
32.8	0\\
32.81	0\\
32.82	0\\
32.83	0\\
32.84	0\\
32.85	0\\
32.86	0\\
32.87	0\\
32.88	0\\
32.89	0\\
32.9	0\\
32.91	0\\
32.92	0\\
32.93	0\\
32.94	0\\
32.95	0\\
32.96	0\\
32.97	0\\
32.98	0\\
32.99	0\\
33	0\\
33.01	0\\
33.02	0\\
33.03	0\\
33.04	0\\
33.05	0\\
33.06	0\\
33.07	0\\
33.08	0\\
33.09	0\\
33.1	0\\
33.11	0\\
33.12	0\\
33.13	0\\
33.14	0\\
33.15	0\\
33.16	0\\
33.17	0\\
33.18	0\\
33.19	0\\
33.2	0\\
33.21	0\\
33.22	0\\
33.23	0\\
33.24	0\\
33.25	0\\
33.26	0\\
33.27	0\\
33.28	0\\
33.29	0\\
33.3	0\\
33.31	0\\
33.32	0\\
33.33	0\\
33.34	0\\
33.35	0\\
33.36	0\\
33.37	0\\
33.38	0\\
33.39	0\\
33.4	0\\
33.41	0\\
33.42	0\\
33.43	0\\
33.44	0\\
33.45	0\\
33.46	0\\
33.47	0\\
33.48	0\\
33.49	0\\
33.5	0\\
33.51	0\\
33.52	0\\
33.53	0\\
33.54	0\\
33.55	0\\
33.56	0\\
33.57	0\\
33.58	0\\
33.59	0\\
33.6	0\\
33.61	0\\
33.62	0\\
33.63	0\\
33.64	0\\
33.65	0\\
33.66	0\\
33.67	0\\
33.68	0\\
33.69	0\\
33.7	0\\
33.71	0\\
33.72	0\\
33.73	0\\
33.74	0\\
33.75	0\\
33.76	0\\
33.77	0\\
33.78	0\\
33.79	0\\
33.8	0\\
33.81	0\\
33.82	0\\
33.83	0\\
33.84	0\\
33.85	0\\
33.86	0\\
33.87	0\\
33.88	0\\
33.89	0\\
33.9	0\\
33.91	0\\
33.92	0\\
33.93	0\\
33.94	0\\
33.95	0\\
33.96	0\\
33.97	0\\
33.98	0\\
33.99	0\\
34	0\\
34.01	0\\
34.02	0\\
34.03	0\\
34.04	0\\
34.05	0\\
34.06	0\\
34.07	0\\
34.08	0\\
34.09	0\\
34.1	0\\
34.11	0\\
34.12	0\\
34.13	0\\
34.14	0\\
34.15	0\\
34.16	0\\
34.17	0\\
34.18	0\\
34.19	0\\
34.2	0\\
34.21	0\\
34.22	0\\
34.23	0\\
34.24	0\\
34.25	0\\
34.26	0\\
34.27	0\\
34.28	0\\
34.29	0\\
34.3	0\\
34.31	0\\
34.32	0\\
34.33	0\\
34.34	0\\
34.35	0\\
34.36	0\\
34.37	0\\
34.38	0\\
34.39	0\\
34.4	0\\
34.41	0\\
34.42	0\\
34.43	0\\
34.44	0\\
34.45	0\\
34.46	0\\
34.47	0\\
34.48	0\\
34.49	0\\
34.5	0\\
34.51	0\\
34.52	0\\
34.53	0\\
34.54	0\\
34.55	0\\
34.56	0\\
34.57	0\\
34.58	0\\
34.59	0\\
34.6	0\\
34.61	0\\
34.62	0\\
34.63	0\\
34.64	0\\
34.65	0\\
34.66	0\\
34.67	0\\
34.68	0\\
34.69	0\\
34.7	0\\
34.71	0\\
34.72	0\\
34.73	0\\
34.74	0\\
34.75	0\\
34.76	0\\
34.77	0\\
34.78	0\\
34.79	0\\
34.8	0\\
34.81	0\\
34.82	0\\
34.83	0\\
34.84	0\\
34.85	0\\
34.86	0\\
34.87	0\\
34.88	0\\
34.89	0\\
34.9	0\\
34.91	0\\
34.92	0\\
34.93	0\\
34.94	0\\
34.95	0\\
34.96	0\\
34.97	0\\
34.98	0\\
34.99	0\\
35	0\\
35.01	0\\
35.02	0\\
35.03	0\\
35.04	0\\
35.05	0\\
35.06	0\\
35.07	0\\
35.08	0\\
35.09	0\\
35.1	0\\
35.11	0\\
35.12	0\\
35.13	0\\
35.14	0\\
35.15	0\\
35.16	0\\
35.17	0\\
35.18	0\\
35.19	0\\
35.2	0\\
35.21	0\\
35.22	0\\
35.23	0\\
35.24	0\\
35.25	0\\
35.26	0\\
35.27	0\\
35.28	0\\
35.29	0\\
35.3	0\\
35.31	0\\
35.32	0\\
35.33	0\\
35.34	0\\
35.35	0\\
35.36	0\\
35.37	0\\
35.38	0\\
35.39	0\\
35.4	0\\
35.41	0\\
35.42	0\\
35.43	0\\
35.44	0\\
35.45	0\\
35.46	0\\
35.47	0\\
35.48	0\\
35.49	0\\
35.5	0\\
35.51	0\\
35.52	0\\
35.53	0\\
35.54	0\\
35.55	0\\
35.56	0\\
35.57	0\\
35.58	0\\
35.59	0\\
35.6	0\\
35.61	0\\
35.62	0\\
35.63	0\\
35.64	0\\
35.65	0\\
35.66	0\\
35.67	0\\
35.68	0\\
35.69	0\\
35.7	0\\
35.71	0\\
35.72	0\\
35.73	0\\
35.74	0\\
35.75	0\\
35.76	0\\
35.77	0\\
35.78	0\\
35.79	0\\
35.8	0\\
35.81	0\\
35.82	0\\
35.83	0\\
35.84	0\\
35.85	0\\
35.86	0\\
35.87	0\\
35.88	0\\
35.89	0\\
35.9	0\\
35.91	0\\
35.92	0\\
35.93	0\\
35.94	0\\
35.95	0\\
35.96	0\\
35.97	0\\
35.98	0\\
35.99	0\\
36	0\\
36.01	0\\
36.02	0\\
36.03	0\\
36.04	0\\
36.05	0\\
36.06	0\\
36.07	0\\
36.08	0\\
36.09	0\\
36.1	0\\
36.11	0\\
36.12	0\\
36.13	0\\
36.14	0\\
36.15	0\\
36.16	0\\
36.17	0\\
36.18	0\\
36.19	0\\
36.2	0\\
36.21	0\\
36.22	0\\
36.23	0\\
36.24	0\\
36.25	0\\
36.26	0\\
36.27	0\\
36.28	0\\
36.29	0\\
36.3	0\\
36.31	0\\
36.32	0\\
36.33	0\\
36.34	0\\
36.35	0\\
36.36	0\\
36.37	0\\
36.38	0\\
36.39	0\\
36.4	0\\
36.41	0\\
36.42	0\\
36.43	0\\
36.44	0\\
36.45	0\\
36.46	0\\
36.47	0\\
36.48	0\\
36.49	0\\
36.5	0\\
36.51	0\\
36.52	0\\
36.53	0\\
36.54	0\\
36.55	0\\
36.56	0\\
36.57	0\\
36.58	0\\
36.59	0\\
36.6	0\\
36.61	0\\
36.62	0\\
36.63	0\\
36.64	0\\
36.65	0\\
36.66	0\\
36.67	0\\
36.68	0\\
36.69	0\\
36.7	0\\
36.71	0\\
36.72	0\\
36.73	0\\
36.74	0\\
36.75	0\\
36.76	0\\
36.77	0\\
36.78	0\\
36.79	0\\
36.8	0\\
36.81	0\\
36.82	0\\
36.83	0\\
36.84	0\\
36.85	0\\
36.86	0\\
36.87	0\\
36.88	0\\
36.89	0\\
36.9	0\\
36.91	0\\
36.92	0\\
36.93	0\\
36.94	0\\
36.95	0\\
36.96	0\\
36.97	0\\
36.98	0\\
36.99	0\\
37	0\\
37.01	0\\
37.02	0\\
37.03	0\\
37.04	0\\
37.05	0\\
37.06	0\\
37.07	0\\
37.08	0\\
37.09	0\\
37.1	0\\
37.11	0\\
37.12	0\\
37.13	0\\
37.14	0\\
37.15	0\\
37.16	0\\
37.17	0\\
37.18	0\\
37.19	0\\
37.2	0\\
37.21	0\\
37.22	0\\
37.23	0\\
37.24	0\\
37.25	0\\
37.26	0\\
37.27	0\\
37.28	0\\
37.29	0\\
37.3	0\\
37.31	0\\
37.32	0\\
37.33	0\\
37.34	0\\
37.35	0\\
37.36	0\\
37.37	0\\
37.38	0\\
37.39	0\\
37.4	0\\
37.41	0\\
37.42	0\\
37.43	0\\
37.44	0\\
37.45	0\\
37.46	0\\
37.47	0\\
37.48	0\\
37.49	0\\
37.5	0\\
37.51	0\\
37.52	0\\
37.53	0\\
37.54	0\\
37.55	0\\
37.56	0\\
37.57	0\\
37.58	0\\
37.59	0\\
37.6	0\\
37.61	0\\
37.62	0\\
37.63	0\\
37.64	0\\
37.65	0\\
37.66	0\\
37.67	0\\
37.68	0\\
37.69	0\\
37.7	0\\
37.71	0\\
37.72	0\\
37.73	0\\
37.74	0\\
37.75	0\\
37.76	0\\
37.77	0\\
37.78	0\\
37.79	0\\
37.8	0\\
37.81	0\\
37.82	0\\
37.83	0\\
37.84	0\\
37.85	0\\
37.86	0\\
37.87	0\\
37.88	0\\
37.89	0\\
37.9	0\\
37.91	0\\
37.92	0\\
37.93	0\\
37.94	0\\
37.95	0\\
37.96	0\\
37.97	0\\
37.98	0\\
37.99	0\\
38	0\\
38.01	0\\
38.02	0\\
38.03	0\\
38.04	0\\
38.05	0\\
38.06	0\\
38.07	0\\
38.08	0\\
38.09	0\\
38.1	0\\
38.11	0\\
38.12	0\\
38.13	0\\
38.14	0\\
38.15	0\\
38.16	0\\
38.17	0\\
38.18	0\\
38.19	0\\
38.2	0\\
38.21	0\\
38.22	0\\
38.23	0\\
38.24	0\\
38.25	0\\
38.26	0\\
38.27	0\\
38.28	0\\
38.29	0\\
38.3	0\\
38.31	0\\
38.32	0\\
38.33	0\\
38.34	0\\
38.35	0\\
38.36	0\\
38.37	0\\
38.38	0\\
38.39	0\\
38.4	0\\
38.41	0\\
38.42	0\\
38.43	0\\
38.44	0\\
38.45	0\\
38.46	0\\
38.47	0\\
38.48	0\\
38.49	0\\
38.5	0\\
38.51	0\\
38.52	0\\
38.53	0\\
38.54	0\\
38.55	0\\
38.56	0\\
38.57	0\\
38.58	0\\
38.59	0\\
38.6	0\\
38.61	0\\
38.62	0\\
38.63	0\\
38.64	0\\
38.65	0\\
38.66	0\\
38.67	0\\
38.68	0\\
38.69	0\\
38.7	0\\
38.71	0\\
38.72	0\\
38.73	0\\
38.74	0\\
38.75	0\\
38.76	0\\
38.77	0\\
38.78	0\\
38.79	0\\
38.8	0\\
38.81	0\\
38.82	0\\
38.83	0\\
38.84	0\\
38.85	0\\
38.86	0\\
38.87	0\\
38.88	0\\
38.89	0\\
38.9	0\\
38.91	0\\
38.92	0\\
38.93	0\\
38.94	0\\
38.95	0\\
38.96	0\\
38.97	0\\
38.98	0\\
38.99	0\\
39	0\\
39.01	0\\
39.02	0\\
39.03	0\\
39.04	0\\
39.05	0\\
39.06	0\\
39.07	0\\
39.08	0\\
39.09	0\\
39.1	0\\
39.11	0\\
39.12	0\\
39.13	0\\
39.14	0\\
39.15	0\\
39.16	0\\
39.17	0\\
39.18	0\\
39.19	0\\
39.2	0\\
39.21	0\\
39.22	0\\
39.23	0\\
39.24	0\\
39.25	0\\
39.26	0\\
39.27	0\\
39.28	0\\
39.29	0\\
39.3	0\\
39.31	0\\
39.32	0\\
39.33	0\\
39.34	0\\
39.35	0\\
39.36	0\\
39.37	0\\
39.38	0\\
39.39	0\\
39.4	0\\
39.41	0\\
39.42	0\\
39.43	0\\
39.44	0\\
39.45	0\\
39.46	0\\
39.47	0\\
39.48	0\\
39.49	0\\
39.5	0\\
39.51	0\\
39.52	0\\
39.53	0\\
39.54	0\\
39.55	0\\
39.56	0\\
39.57	0\\
39.58	0\\
39.59	0\\
39.6	0\\
39.61	0\\
39.62	0\\
39.63	0\\
39.64	0\\
39.65	0\\
39.66	0\\
39.67	0\\
39.68	0\\
39.69	0\\
39.7	0\\
39.71	0\\
39.72	0\\
39.73	0\\
39.74	0\\
39.75	0\\
39.76	0\\
39.77	0\\
39.78	0\\
39.79	0\\
39.8	0\\
39.81	0\\
39.82	0\\
39.83	0\\
39.84	0\\
39.85	0\\
39.86	0\\
39.87	0\\
39.88	0\\
39.89	0\\
39.9	0\\
39.91	0\\
39.92	0\\
39.93	0\\
39.94	0\\
39.95	0\\
39.96	0\\
39.97	0\\
39.98	0\\
39.99	0\\
40	0\\
40.01	0\\
};
\addplot [color=red,solid,forget plot]
  table[row sep=crcr]{%
40.01	0\\
40.02	0\\
40.03	0\\
40.04	0\\
40.05	0\\
40.06	0\\
40.07	0\\
40.08	0\\
40.09	0\\
40.1	0\\
40.11	0\\
40.12	0\\
40.13	0\\
40.14	0\\
40.15	0\\
40.16	0\\
40.17	0\\
40.18	0\\
40.19	0\\
40.2	0\\
40.21	0\\
40.22	0\\
40.23	0\\
40.24	0\\
40.25	0\\
40.26	0\\
40.27	0\\
40.28	0\\
40.29	0\\
40.3	0\\
40.31	0\\
40.32	0\\
40.33	0\\
40.34	0\\
40.35	0\\
40.36	0\\
40.37	0\\
40.38	0\\
40.39	0\\
40.4	0\\
40.41	0\\
40.42	0\\
40.43	0\\
40.44	0\\
40.45	0\\
40.46	0\\
40.47	0\\
40.48	0\\
40.49	0\\
40.5	0\\
40.51	0\\
40.52	0\\
40.53	0\\
40.54	0\\
40.55	0\\
40.56	0\\
40.57	0\\
40.58	0\\
40.59	0\\
40.6	0\\
40.61	0\\
40.62	0\\
40.63	0\\
40.64	0\\
40.65	0\\
40.66	0\\
40.67	0\\
40.68	0\\
40.69	0\\
40.7	0\\
40.71	0\\
40.72	0\\
40.73	0\\
40.74	0\\
40.75	0\\
40.76	0\\
40.77	0\\
40.78	0\\
40.79	0\\
40.8	0\\
40.81	0\\
40.82	0\\
40.83	0\\
40.84	0\\
40.85	0\\
40.86	0\\
40.87	0\\
40.88	0\\
40.89	0\\
40.9	0\\
40.91	0\\
40.92	0\\
40.93	0\\
40.94	0\\
40.95	0\\
40.96	0\\
40.97	0\\
40.98	0\\
40.99	0\\
41	0\\
41.01	0\\
41.02	0\\
41.03	0\\
41.04	0\\
41.05	0\\
41.06	0\\
41.07	0\\
41.08	0\\
41.09	0\\
41.1	0\\
41.11	0\\
41.12	0\\
41.13	0\\
41.14	0\\
41.15	0\\
41.16	0\\
41.17	0\\
41.18	0\\
41.19	0\\
41.2	0\\
41.21	0\\
41.22	0\\
41.23	0\\
41.24	0\\
41.25	0\\
41.26	0\\
41.27	0\\
41.28	0\\
41.29	0\\
41.3	0\\
41.31	0\\
41.32	0\\
41.33	0\\
41.34	0\\
41.35	0\\
41.36	0\\
41.37	0\\
41.38	0\\
41.39	0\\
41.4	0\\
41.41	0\\
41.42	0\\
41.43	0\\
41.44	0\\
41.45	0\\
41.46	0\\
41.47	0\\
41.48	0\\
41.49	0\\
41.5	0\\
41.51	0\\
41.52	0\\
41.53	0\\
41.54	0\\
41.55	0\\
41.56	0\\
41.57	0\\
41.58	0\\
41.59	0\\
41.6	0\\
41.61	0\\
41.62	0\\
41.63	0\\
41.64	0\\
41.65	0\\
41.66	0\\
41.67	0\\
41.68	0\\
41.69	0\\
41.7	0\\
41.71	0\\
41.72	0\\
41.73	0\\
41.74	0\\
41.75	0\\
41.76	0\\
41.77	0\\
41.78	0\\
41.79	0\\
41.8	0\\
41.81	0\\
41.82	0\\
41.83	0\\
41.84	0\\
41.85	0\\
41.86	0\\
41.87	0\\
41.88	0\\
41.89	0\\
41.9	0\\
41.91	0\\
41.92	0\\
41.93	0\\
41.94	0\\
41.95	0\\
41.96	0\\
41.97	0\\
41.98	0\\
41.99	0\\
42	0\\
42.01	0\\
42.02	0\\
42.03	0\\
42.04	0\\
42.05	0\\
42.06	0\\
42.07	0\\
42.08	0\\
42.09	0\\
42.1	0\\
42.11	0\\
42.12	0\\
42.13	0\\
42.14	0\\
42.15	0\\
42.16	0\\
42.17	0\\
42.18	0\\
42.19	0\\
42.2	0\\
42.21	0\\
42.22	0\\
42.23	0\\
42.24	0\\
42.25	0\\
42.26	0\\
42.27	0\\
42.28	0\\
42.29	0\\
42.3	0\\
42.31	0\\
42.32	0\\
42.33	0\\
42.34	0\\
42.35	0\\
42.36	0\\
42.37	0\\
42.38	0\\
42.39	0\\
42.4	0\\
42.41	0\\
42.42	0\\
42.43	0\\
42.44	0\\
42.45	0\\
42.46	0\\
42.47	0\\
42.48	0\\
42.49	0\\
42.5	0\\
42.51	0\\
42.52	0\\
42.53	0\\
42.54	0\\
42.55	0\\
42.56	0\\
42.57	0\\
42.58	0\\
42.59	0\\
42.6	0\\
42.61	0\\
42.62	0\\
42.63	0\\
42.64	0\\
42.65	0\\
42.66	0\\
42.67	0\\
42.68	0\\
42.69	0\\
42.7	0\\
42.71	0\\
42.72	0\\
42.73	0\\
42.74	0\\
42.75	0\\
42.76	0\\
42.77	0\\
42.78	0\\
42.79	0\\
42.8	0\\
42.81	0\\
42.82	0\\
42.83	0\\
42.84	0\\
42.85	0\\
42.86	0\\
42.87	0\\
42.88	0\\
42.89	0\\
42.9	0\\
42.91	0\\
42.92	0\\
42.93	0\\
42.94	0\\
42.95	0\\
42.96	0\\
42.97	0\\
42.98	0\\
42.99	0\\
43	0\\
43.01	0\\
43.02	0\\
43.03	0\\
43.04	0\\
43.05	0\\
43.06	0\\
43.07	0\\
43.08	0\\
43.09	0\\
43.1	0\\
43.11	0\\
43.12	0\\
43.13	0\\
43.14	0\\
43.15	0\\
43.16	0\\
43.17	0\\
43.18	0\\
43.19	0\\
43.2	0\\
43.21	0\\
43.22	0\\
43.23	0\\
43.24	0\\
43.25	0\\
43.26	0\\
43.27	0\\
43.28	0\\
43.29	0\\
43.3	0\\
43.31	0\\
43.32	0\\
43.33	0\\
43.34	0\\
43.35	0\\
43.36	0\\
43.37	0\\
43.38	0\\
43.39	0\\
43.4	0\\
43.41	0\\
43.42	0\\
43.43	0\\
43.44	0\\
43.45	0\\
43.46	0\\
43.47	0\\
43.48	0\\
43.49	0\\
43.5	0\\
43.51	0\\
43.52	0\\
43.53	0\\
43.54	0\\
43.55	0\\
43.56	0\\
43.57	0\\
43.58	0\\
43.59	0\\
43.6	0\\
43.61	0\\
43.62	0\\
43.63	0\\
43.64	0\\
43.65	0\\
43.66	0\\
43.67	0\\
43.68	0\\
43.69	0\\
43.7	0\\
43.71	0\\
43.72	0\\
43.73	0\\
43.74	0\\
43.75	0\\
43.76	0\\
43.77	0\\
43.78	0\\
43.79	0\\
43.8	0\\
43.81	0\\
43.82	0\\
43.83	0\\
43.84	0\\
43.85	0\\
43.86	0\\
43.87	0\\
43.88	0\\
43.89	0\\
43.9	0\\
43.91	0\\
43.92	0\\
43.93	0\\
43.94	0\\
43.95	0\\
43.96	0\\
43.97	0\\
43.98	0\\
43.99	0\\
44	0\\
44.01	0\\
44.02	0\\
44.03	0\\
44.04	0\\
44.05	0\\
44.06	0\\
44.07	0\\
44.08	0\\
44.09	0\\
44.1	0\\
44.11	0\\
44.12	0\\
44.13	0\\
44.14	0\\
44.15	0\\
44.16	0\\
44.17	0\\
44.18	0\\
44.19	0\\
44.2	0\\
44.21	0\\
44.22	0\\
44.23	0\\
44.24	0\\
44.25	0\\
44.26	0\\
44.27	0\\
44.28	0\\
44.29	0\\
44.3	0\\
44.31	0\\
44.32	0\\
44.33	0\\
44.34	0\\
44.35	0\\
44.36	0\\
44.37	0\\
44.38	0\\
44.39	0\\
44.4	0\\
44.41	0\\
44.42	0\\
44.43	0\\
44.44	0\\
44.45	0\\
44.46	0\\
44.47	0\\
44.48	0\\
44.49	0\\
44.5	0\\
44.51	0\\
44.52	0\\
44.53	0\\
44.54	0\\
44.55	0\\
44.56	0\\
44.57	0\\
44.58	0\\
44.59	0\\
44.6	0\\
44.61	0\\
44.62	0\\
44.63	0\\
44.64	0\\
44.65	0\\
44.66	0\\
44.67	0\\
44.68	0\\
44.69	0\\
44.7	0\\
44.71	0\\
44.72	0\\
44.73	0\\
44.74	0\\
44.75	0\\
44.76	0\\
44.77	0\\
44.78	0\\
44.79	0\\
44.8	0\\
44.81	0\\
44.82	0\\
44.83	0\\
44.84	0\\
44.85	0\\
44.86	0\\
44.87	0\\
44.88	0\\
44.89	0\\
44.9	0\\
44.91	0\\
44.92	0\\
44.93	0\\
44.94	0\\
44.95	0\\
44.96	0\\
44.97	0\\
44.98	0\\
44.99	0\\
45	0\\
45.01	0\\
45.02	0\\
45.03	0\\
45.04	0\\
45.05	0\\
45.06	0\\
45.07	0\\
45.08	0\\
45.09	0\\
45.1	0\\
45.11	0\\
45.12	0\\
45.13	0\\
45.14	0\\
45.15	0\\
45.16	0\\
45.17	0\\
45.18	0\\
45.19	0\\
45.2	0\\
45.21	0\\
45.22	0\\
45.23	0\\
45.24	0\\
45.25	0\\
45.26	0\\
45.27	0\\
45.28	0\\
45.29	0\\
45.3	0\\
45.31	0\\
45.32	0\\
45.33	0\\
45.34	0\\
45.35	0\\
45.36	0\\
45.37	0\\
45.38	0\\
45.39	0\\
45.4	0\\
45.41	0\\
45.42	0\\
45.43	0\\
45.44	0\\
45.45	0\\
45.46	0\\
45.47	0\\
45.48	0\\
45.49	0\\
45.5	0\\
45.51	0\\
45.52	0\\
45.53	0\\
45.54	0\\
45.55	0\\
45.56	0\\
45.57	0\\
45.58	0\\
45.59	0\\
45.6	0\\
45.61	0\\
45.62	0\\
45.63	0\\
45.64	0\\
45.65	0\\
45.66	0\\
45.67	0\\
45.68	0\\
45.69	0\\
45.7	0\\
45.71	0\\
45.72	0\\
45.73	0\\
45.74	0\\
45.75	0\\
45.76	0\\
45.77	0\\
45.78	0\\
45.79	0\\
45.8	0\\
45.81	0\\
45.82	0\\
45.83	0\\
45.84	0\\
45.85	0\\
45.86	0\\
45.87	0\\
45.88	0\\
45.89	0\\
45.9	0\\
45.91	0\\
45.92	0\\
45.93	0\\
45.94	0\\
45.95	0\\
45.96	0\\
45.97	0\\
45.98	0\\
45.99	0\\
46	0\\
46.01	0\\
46.02	0\\
46.03	0\\
46.04	0\\
46.05	0\\
46.06	0\\
46.07	0\\
46.08	0\\
46.09	0\\
46.1	0\\
46.11	0\\
46.12	0\\
46.13	0\\
46.14	0\\
46.15	0\\
46.16	0\\
46.17	0\\
46.18	0\\
46.19	0\\
46.2	0\\
46.21	0\\
46.22	0\\
46.23	0\\
46.24	0\\
46.25	0\\
46.26	0\\
46.27	0\\
46.28	0\\
46.29	0\\
46.3	0\\
46.31	0\\
46.32	0\\
46.33	0\\
46.34	0\\
46.35	0\\
46.36	0\\
46.37	0\\
46.38	0\\
46.39	0\\
46.4	0\\
46.41	0\\
46.42	0\\
46.43	0\\
46.44	0\\
46.45	0\\
46.46	0\\
46.47	0\\
46.48	0\\
46.49	0\\
46.5	0\\
46.51	0\\
46.52	0\\
46.53	0\\
46.54	0\\
46.55	0\\
46.56	0\\
46.57	0\\
46.58	0\\
46.59	0\\
46.6	0\\
46.61	0\\
46.62	0\\
46.63	0\\
46.64	0\\
46.65	0\\
46.66	0\\
46.67	0\\
46.68	0\\
46.69	0\\
46.7	0\\
46.71	0\\
46.72	0\\
46.73	0\\
46.74	0\\
46.75	0\\
46.76	0\\
46.77	0\\
46.78	0\\
46.79	0\\
46.8	0\\
46.81	0\\
46.82	0\\
46.83	0\\
46.84	0\\
46.85	0\\
46.86	0\\
46.87	0\\
46.88	0\\
46.89	0\\
46.9	0\\
46.91	0\\
46.92	0\\
46.93	0\\
46.94	0\\
46.95	0\\
46.96	0\\
46.97	0\\
46.98	0\\
46.99	0\\
47	0\\
47.01	0\\
47.02	0\\
47.03	0\\
47.04	0\\
47.05	0\\
47.06	0\\
47.07	0\\
47.08	0\\
47.09	0\\
47.1	0\\
47.11	0\\
47.12	0\\
47.13	0\\
47.14	0\\
47.15	0\\
47.16	0\\
47.17	0\\
47.18	0\\
47.19	0\\
47.2	0\\
47.21	0\\
47.22	0\\
47.23	0\\
47.24	0\\
47.25	0\\
47.26	0\\
47.27	0\\
47.28	0\\
47.29	0\\
47.3	0\\
47.31	0\\
47.32	0\\
47.33	0\\
47.34	0\\
47.35	0\\
47.36	0\\
47.37	0\\
47.38	0\\
47.39	0\\
47.4	0\\
47.41	0\\
47.42	0\\
47.43	0\\
47.44	0\\
47.45	0\\
47.46	0\\
47.47	0\\
47.48	0\\
47.49	0\\
47.5	0\\
47.51	0\\
47.52	0\\
47.53	0\\
47.54	0\\
47.55	0\\
47.56	0\\
47.57	0\\
47.58	0\\
47.59	0\\
47.6	0\\
47.61	0\\
47.62	0\\
47.63	0\\
47.64	0\\
47.65	0\\
47.66	0\\
47.67	0\\
47.68	0\\
47.69	0\\
47.7	0\\
47.71	0\\
47.72	0\\
47.73	0\\
47.74	0\\
47.75	0\\
47.76	0\\
47.77	0\\
47.78	0\\
47.79	0\\
47.8	0\\
47.81	0\\
47.82	0\\
47.83	0\\
47.84	0\\
47.85	0\\
47.86	0\\
47.87	0\\
47.88	0\\
47.89	0\\
47.9	0\\
47.91	0\\
47.92	0\\
47.93	0\\
47.94	0\\
47.95	0\\
47.96	0\\
47.97	0\\
47.98	0\\
47.99	0\\
48	0\\
48.01	0\\
48.02	0\\
48.03	0\\
48.04	0\\
48.05	0\\
48.06	0\\
48.07	0\\
48.08	0\\
48.09	0\\
48.1	0\\
48.11	0\\
48.12	0\\
48.13	0\\
48.14	0\\
48.15	0\\
48.16	0\\
48.17	0\\
48.18	0\\
48.19	0\\
48.2	0\\
48.21	0\\
48.22	0\\
48.23	0\\
48.24	0\\
48.25	0\\
48.26	0\\
48.27	0\\
48.28	0\\
48.29	0\\
48.3	0\\
48.31	0\\
48.32	0\\
48.33	0\\
48.34	0\\
48.35	0\\
48.36	0\\
48.37	0\\
48.38	0\\
48.39	0\\
48.4	0\\
48.41	0\\
48.42	0\\
48.43	0\\
48.44	0\\
48.45	0\\
48.46	0\\
48.47	0\\
48.48	0\\
48.49	0\\
48.5	0\\
48.51	0\\
48.52	0\\
48.53	0\\
48.54	0\\
48.55	0\\
48.56	0\\
48.57	0\\
48.58	0\\
48.59	0\\
48.6	0\\
48.61	0\\
48.62	0\\
48.63	0\\
48.64	0\\
48.65	0\\
48.66	0\\
48.67	0\\
48.68	0\\
48.69	0\\
48.7	0\\
48.71	0\\
48.72	0\\
48.73	0\\
48.74	0\\
48.75	0\\
48.76	0\\
48.77	0\\
48.78	0\\
48.79	0\\
48.8	0\\
48.81	0\\
48.82	0\\
48.83	0\\
48.84	0\\
48.85	0\\
48.86	0\\
48.87	0\\
48.88	0\\
48.89	0\\
48.9	0\\
48.91	0\\
48.92	0\\
48.93	0\\
48.94	0\\
48.95	0\\
48.96	0\\
48.97	0\\
48.98	0\\
48.99	0\\
49	0\\
49.01	0\\
49.02	0\\
49.03	0\\
49.04	0\\
49.05	0\\
49.06	0\\
49.07	0\\
49.08	0\\
49.09	0\\
49.1	0\\
49.11	0\\
49.12	0\\
49.13	0\\
49.14	0\\
49.15	0\\
49.16	0\\
49.17	0\\
49.18	0\\
49.19	0\\
49.2	0\\
49.21	0\\
49.22	0\\
49.23	0\\
49.24	0\\
49.25	0\\
49.26	0\\
49.27	0\\
49.28	0\\
49.29	0\\
49.3	0\\
49.31	0\\
49.32	0\\
49.33	0\\
49.34	0\\
49.35	0\\
49.36	0\\
49.37	0\\
49.38	0\\
49.39	0\\
49.4	0\\
49.41	0\\
49.42	0\\
49.43	0\\
49.44	0\\
49.45	0\\
49.46	0\\
49.47	0\\
49.48	0\\
49.49	0\\
49.5	0\\
49.51	0\\
49.52	0\\
49.53	0\\
49.54	0\\
49.55	0\\
49.56	0\\
49.57	0\\
49.58	0\\
49.59	0\\
49.6	0\\
49.61	0\\
49.62	0\\
49.63	0\\
49.64	0\\
49.65	0\\
49.66	0\\
49.67	0\\
49.68	0\\
49.69	0\\
49.7	0\\
49.71	0\\
49.72	0\\
49.73	0\\
49.74	0\\
49.75	0\\
49.76	0\\
49.77	0\\
49.78	0\\
49.79	0\\
49.8	0\\
49.81	0\\
49.82	0\\
49.83	0\\
49.84	0\\
49.85	0\\
49.86	0\\
49.87	0\\
49.88	0\\
49.89	0\\
49.9	0\\
49.91	0\\
49.92	0\\
49.93	0\\
49.94	0\\
49.95	0\\
49.96	0\\
49.97	0\\
49.98	0\\
49.99	0\\
50	0\\
50.01	0\\
50.02	0\\
50.03	0\\
50.04	0\\
50.05	0\\
50.06	0\\
50.07	0\\
50.08	0\\
50.09	0\\
50.1	0\\
50.11	0\\
50.12	0\\
50.13	0\\
50.14	0\\
50.15	0\\
50.16	0\\
50.17	0\\
50.18	0\\
50.19	0\\
50.2	0\\
50.21	0\\
50.22	0\\
50.23	0\\
50.24	0\\
50.25	0\\
50.26	0\\
50.27	0\\
50.28	0\\
50.29	0\\
50.3	0\\
50.31	0\\
50.32	0\\
50.33	0\\
50.34	0\\
50.35	0\\
50.36	0\\
50.37	0\\
50.38	0\\
50.39	0\\
50.4	0\\
50.41	0\\
50.42	0\\
50.43	0\\
50.44	0\\
50.45	0\\
50.46	0\\
50.47	0\\
50.48	0\\
50.49	0\\
50.5	0\\
50.51	0\\
50.52	0\\
50.53	0\\
50.54	0\\
50.55	0\\
50.56	0\\
50.57	0\\
50.58	0\\
50.59	0\\
50.6	0\\
50.61	0\\
50.62	0\\
50.63	0\\
50.64	0\\
50.65	0\\
50.66	0\\
50.67	0\\
50.68	0\\
50.69	0\\
50.7	0\\
50.71	0\\
50.72	0\\
50.73	0\\
50.74	0\\
50.75	0\\
50.76	0\\
50.77	0\\
50.78	0\\
50.79	0\\
50.8	0\\
50.81	0\\
50.82	0\\
50.83	0\\
50.84	0\\
50.85	0\\
50.86	0\\
50.87	0\\
50.88	0\\
50.89	0\\
50.9	0\\
50.91	0\\
50.92	0\\
50.93	0\\
50.94	0\\
50.95	0\\
50.96	0\\
50.97	0\\
50.98	0\\
50.99	0\\
51	0\\
51.01	0\\
51.02	0\\
51.03	0\\
51.04	0\\
51.05	0\\
51.06	0\\
51.07	0\\
51.08	0\\
51.09	0\\
51.1	0\\
51.11	0\\
51.12	0\\
51.13	0\\
51.14	0\\
51.15	0\\
51.16	0\\
51.17	0\\
51.18	0\\
51.19	0\\
51.2	0\\
51.21	0\\
51.22	0\\
51.23	0\\
51.24	0\\
51.25	0\\
51.26	0\\
51.27	0\\
51.28	0\\
51.29	0\\
51.3	0\\
51.31	0\\
51.32	0\\
51.33	0\\
51.34	0\\
51.35	0\\
51.36	0\\
51.37	0\\
51.38	0\\
51.39	0\\
51.4	0\\
51.41	0\\
51.42	0\\
51.43	0\\
51.44	0\\
51.45	0\\
51.46	0\\
51.47	0\\
51.48	0\\
51.49	0\\
51.5	0\\
51.51	0\\
51.52	0\\
51.53	0\\
51.54	0\\
51.55	0\\
51.56	0\\
51.57	0\\
51.58	0\\
51.59	0\\
51.6	0\\
51.61	0\\
51.62	0\\
51.63	0\\
51.64	0\\
51.65	0\\
51.66	0\\
51.67	0\\
51.68	0\\
51.69	0\\
51.7	0\\
51.71	0\\
51.72	0\\
51.73	0\\
51.74	0\\
51.75	0\\
51.76	0\\
51.77	0\\
51.78	0\\
51.79	0\\
51.8	0\\
51.81	0\\
51.82	0\\
51.83	0\\
51.84	0\\
51.85	0\\
51.86	0\\
51.87	0\\
51.88	0\\
51.89	0\\
51.9	0\\
51.91	0\\
51.92	0\\
51.93	0\\
51.94	0\\
51.95	0\\
51.96	0\\
51.97	0\\
51.98	0\\
51.99	0\\
52	0\\
52.01	0\\
52.02	0\\
52.03	0\\
52.04	0\\
52.05	0\\
52.06	0\\
52.07	0\\
52.08	0\\
52.09	0\\
52.1	0\\
52.11	0\\
52.12	0\\
52.13	0\\
52.14	0\\
52.15	0\\
52.16	0\\
52.17	0\\
52.18	0\\
52.19	0\\
52.2	0\\
52.21	0\\
52.22	0\\
52.23	0\\
52.24	0\\
52.25	0\\
52.26	0\\
52.27	0\\
52.28	0\\
52.29	0\\
52.3	0\\
52.31	0\\
52.32	0\\
52.33	0\\
52.34	0\\
52.35	0\\
52.36	0\\
52.37	0\\
52.38	0\\
52.39	0\\
52.4	0\\
52.41	0\\
52.42	0\\
52.43	0\\
52.44	0\\
52.45	0\\
52.46	0\\
52.47	0\\
52.48	0\\
52.49	0\\
52.5	0\\
52.51	0\\
52.52	0\\
52.53	0\\
52.54	0\\
52.55	0\\
52.56	0\\
52.57	0\\
52.58	0\\
52.59	0\\
52.6	0\\
52.61	0\\
52.62	0\\
52.63	0\\
52.64	0\\
52.65	0\\
52.66	0\\
52.67	0\\
52.68	0\\
52.69	0\\
52.7	0\\
52.71	0\\
52.72	0\\
52.73	0\\
52.74	0\\
52.75	0\\
52.76	0\\
52.77	0\\
52.78	0\\
52.79	0\\
52.8	0\\
52.81	0\\
52.82	0\\
52.83	0\\
52.84	0\\
52.85	0\\
52.86	0\\
52.87	0\\
52.88	0\\
52.89	0\\
52.9	0\\
52.91	0\\
52.92	0\\
52.93	0\\
52.94	0\\
52.95	0\\
52.96	0\\
52.97	0\\
52.98	0\\
52.99	0\\
53	0\\
53.01	0\\
53.02	0\\
53.03	0\\
53.04	0\\
53.05	0\\
53.06	0\\
53.07	0\\
53.08	0\\
53.09	0\\
53.1	0\\
53.11	0\\
53.12	0\\
53.13	0\\
53.14	0\\
53.15	0\\
53.16	0\\
53.17	0\\
53.18	0\\
53.19	0\\
53.2	0\\
53.21	0\\
53.22	0\\
53.23	0\\
53.24	0\\
53.25	0\\
53.26	0\\
53.27	0\\
53.28	0\\
53.29	0\\
53.3	0\\
53.31	0\\
53.32	0\\
53.33	0\\
53.34	0\\
53.35	0\\
53.36	0\\
53.37	0\\
53.38	0\\
53.39	0\\
53.4	0\\
53.41	0\\
53.42	0\\
53.43	0\\
53.44	0\\
53.45	0\\
53.46	0\\
53.47	0\\
53.48	0\\
53.49	0\\
53.5	0\\
53.51	0\\
53.52	0\\
53.53	0\\
53.54	0\\
53.55	0\\
53.56	0\\
53.57	0\\
53.58	0\\
53.59	0\\
53.6	0\\
53.61	0\\
53.62	0\\
53.63	0\\
53.64	0\\
53.65	0\\
53.66	0\\
53.67	0\\
53.68	0\\
53.69	0\\
53.7	0\\
53.71	0\\
53.72	0\\
53.73	0\\
53.74	0\\
53.75	0\\
53.76	0\\
53.77	0\\
53.78	0\\
53.79	0\\
53.8	0\\
53.81	0\\
53.82	0\\
53.83	0\\
53.84	0\\
53.85	0\\
53.86	0\\
53.87	0\\
53.88	0\\
53.89	0\\
53.9	0\\
53.91	0\\
53.92	0\\
53.93	0\\
53.94	0\\
53.95	0\\
53.96	0\\
53.97	0\\
53.98	0\\
53.99	0\\
54	0\\
54.01	0\\
54.02	0\\
54.03	0\\
54.04	0\\
54.05	0\\
54.06	0\\
54.07	0\\
54.08	0\\
54.09	0\\
54.1	0\\
54.11	0\\
54.12	0\\
54.13	0\\
54.14	0\\
54.15	0\\
54.16	0\\
54.17	0\\
54.18	0\\
54.19	0\\
54.2	0\\
54.21	0\\
54.22	0\\
54.23	0\\
54.24	0\\
54.25	0\\
54.26	0\\
54.27	0\\
54.28	0\\
54.29	0\\
54.3	0\\
54.31	0\\
54.32	0\\
54.33	0\\
54.34	0\\
54.35	0\\
54.36	0\\
54.37	0\\
54.38	0\\
54.39	0\\
54.4	0\\
54.41	0\\
54.42	0\\
54.43	0\\
54.44	0\\
54.45	0\\
54.46	0\\
54.47	0\\
54.48	0\\
54.49	0\\
54.5	0\\
54.51	0\\
54.52	0\\
54.53	0\\
54.54	0\\
54.55	0\\
54.56	0\\
54.57	0\\
54.58	0\\
54.59	0\\
54.6	0\\
54.61	0\\
54.62	0\\
54.63	0\\
54.64	0\\
54.65	0\\
54.66	0\\
54.67	0\\
54.68	0\\
54.69	0\\
54.7	0\\
54.71	0\\
54.72	0\\
54.73	0\\
54.74	0\\
54.75	0\\
54.76	0\\
54.77	0\\
54.78	0\\
54.79	0\\
54.8	0\\
54.81	0\\
54.82	0\\
54.83	0\\
54.84	0\\
54.85	0\\
54.86	0\\
54.87	0\\
54.88	0\\
54.89	0\\
54.9	0\\
54.91	0\\
54.92	0\\
54.93	0\\
54.94	0\\
54.95	0\\
54.96	0\\
54.97	0\\
54.98	0\\
54.99	0\\
55	0\\
55.01	0\\
55.02	0\\
55.03	0\\
55.04	0\\
55.05	0\\
55.06	0\\
55.07	0\\
55.08	0\\
55.09	0\\
55.1	0\\
55.11	0\\
55.12	0\\
55.13	0\\
55.14	0\\
55.15	0\\
55.16	0\\
55.17	0\\
55.18	0\\
55.19	0\\
55.2	0\\
55.21	0\\
55.22	0\\
55.23	0\\
55.24	0\\
55.25	0\\
55.26	0\\
55.27	0\\
55.28	0\\
55.29	0\\
55.3	0\\
55.31	0\\
55.32	0\\
55.33	0\\
55.34	0\\
55.35	0\\
55.36	0\\
55.37	0\\
55.38	0\\
55.39	0\\
55.4	0\\
55.41	0\\
55.42	0\\
55.43	0\\
55.44	0\\
55.45	0\\
55.46	0\\
55.47	0\\
55.48	0\\
55.49	0\\
55.5	0\\
55.51	0\\
55.52	0\\
55.53	0\\
55.54	0\\
55.55	0\\
55.56	0\\
55.57	0\\
55.58	0\\
55.59	0\\
55.6	0\\
55.61	0\\
55.62	0\\
55.63	0\\
55.64	0\\
55.65	0\\
55.66	0\\
55.67	0\\
55.68	0\\
55.69	0\\
55.7	0\\
55.71	0\\
55.72	0\\
55.73	0\\
55.74	0\\
55.75	0\\
55.76	0\\
55.77	0\\
55.78	0\\
55.79	0\\
55.8	0\\
55.81	0\\
55.82	0\\
55.83	0\\
55.84	0\\
55.85	0\\
55.86	0\\
55.87	0\\
55.88	0\\
55.89	0\\
55.9	0\\
55.91	0\\
55.92	0\\
55.93	0\\
55.94	0\\
55.95	0\\
55.96	0\\
55.97	0\\
55.98	0\\
55.99	0\\
56	0\\
56.01	0\\
56.02	0\\
56.03	0\\
56.04	0\\
56.05	0\\
56.06	0\\
56.07	0\\
56.08	0\\
56.09	0\\
56.1	0\\
56.11	0\\
56.12	0\\
56.13	0\\
56.14	0\\
56.15	0\\
56.16	0\\
56.17	0\\
56.18	0\\
56.19	0\\
56.2	0\\
56.21	0\\
56.22	0\\
56.23	0\\
56.24	0\\
56.25	0\\
56.26	0\\
56.27	0\\
56.28	0\\
56.29	0\\
56.3	0\\
56.31	0\\
56.32	0\\
56.33	0\\
56.34	0\\
56.35	0\\
56.36	0\\
56.37	0\\
56.38	0\\
56.39	0\\
56.4	0\\
56.41	0\\
56.42	0\\
56.43	0\\
56.44	0\\
56.45	0\\
56.46	0\\
56.47	0\\
56.48	0\\
56.49	0\\
56.5	0\\
56.51	0\\
56.52	0\\
56.53	0\\
56.54	0\\
56.55	0\\
56.56	0\\
56.57	0\\
56.58	0\\
56.59	0\\
56.6	0\\
56.61	0\\
56.62	0\\
56.63	0\\
56.64	0\\
56.65	0\\
56.66	0\\
56.67	0\\
56.68	0\\
56.69	0\\
56.7	0\\
56.71	0\\
56.72	0\\
56.73	0\\
56.74	0\\
56.75	0\\
56.76	0\\
56.77	0\\
56.78	0\\
56.79	0\\
56.8	0\\
56.81	0\\
56.82	0\\
56.83	0\\
56.84	0\\
56.85	0\\
56.86	0\\
56.87	0\\
56.88	0\\
56.89	0\\
56.9	0\\
56.91	0\\
56.92	0\\
56.93	0\\
56.94	0\\
56.95	0\\
56.96	0\\
56.97	0\\
56.98	0\\
56.99	0\\
57	0\\
57.01	0\\
57.02	0\\
57.03	0\\
57.04	0\\
57.05	0\\
57.06	0\\
57.07	0\\
57.08	0\\
57.09	0\\
57.1	0\\
57.11	0\\
57.12	0\\
57.13	0\\
57.14	0\\
57.15	0\\
57.16	0\\
57.17	0\\
57.18	0\\
57.19	0\\
57.2	0\\
57.21	0\\
57.22	0\\
57.23	0\\
57.24	0\\
57.25	0\\
57.26	0\\
57.27	0\\
57.28	0\\
57.29	0\\
57.3	0\\
57.31	0\\
57.32	0\\
57.33	0\\
57.34	0\\
57.35	0\\
57.36	0\\
57.37	0\\
57.38	0\\
57.39	0\\
57.4	0\\
57.41	0\\
57.42	0\\
57.43	0\\
57.44	0\\
57.45	0\\
57.46	0\\
57.47	0\\
57.48	0\\
57.49	0\\
57.5	0\\
57.51	0\\
57.52	0\\
57.53	0\\
57.54	0\\
57.55	0\\
57.56	0\\
57.57	0\\
57.58	0\\
57.59	0\\
57.6	0\\
57.61	0\\
57.62	0\\
57.63	0\\
57.64	0\\
57.65	0\\
57.66	0\\
57.67	0\\
57.68	0\\
57.69	0\\
57.7	0\\
57.71	0\\
57.72	0\\
57.73	0\\
57.74	0\\
57.75	0\\
57.76	0\\
57.77	0\\
57.78	0\\
57.79	0\\
57.8	0\\
57.81	0\\
57.82	0\\
57.83	0\\
57.84	0\\
57.85	0\\
57.86	0\\
57.87	0\\
57.88	0\\
57.89	0\\
57.9	0\\
57.91	0\\
57.92	0\\
57.93	0\\
57.94	0\\
57.95	0\\
57.96	0\\
57.97	0\\
57.98	0\\
57.99	0\\
58	0\\
58.01	0\\
58.02	0\\
58.03	0\\
58.04	0\\
58.05	0\\
58.06	0\\
58.07	0\\
58.08	0\\
58.09	0\\
58.1	0\\
58.11	0\\
58.12	0\\
58.13	0\\
58.14	0\\
58.15	0\\
58.16	0\\
58.17	0\\
58.18	0\\
58.19	0\\
58.2	0\\
58.21	0\\
58.22	0\\
58.23	0\\
58.24	0\\
58.25	0\\
58.26	0\\
58.27	0\\
58.28	0\\
58.29	0\\
58.3	0\\
58.31	0\\
58.32	0\\
58.33	0\\
58.34	0\\
58.35	0\\
58.36	0\\
58.37	0\\
58.38	0\\
58.39	0\\
58.4	0\\
58.41	0\\
58.42	0\\
58.43	0\\
58.44	0\\
58.45	0\\
58.46	0\\
58.47	0\\
58.48	0\\
58.49	0\\
58.5	0\\
58.51	0\\
58.52	0\\
58.53	0\\
58.54	0\\
58.55	0\\
58.56	0\\
58.57	0\\
58.58	0\\
58.59	0\\
58.6	0\\
58.61	0\\
58.62	0\\
58.63	0\\
58.64	0\\
58.65	0\\
58.66	0\\
58.67	0\\
58.68	0\\
58.69	0\\
58.7	0\\
58.71	0\\
58.72	0\\
58.73	0\\
58.74	0\\
58.75	0\\
58.76	0\\
58.77	0\\
58.78	0\\
58.79	0\\
58.8	0\\
58.81	0\\
58.82	0\\
58.83	0\\
58.84	0\\
58.85	0\\
58.86	0\\
58.87	0\\
58.88	0\\
58.89	0\\
58.9	0\\
58.91	0\\
58.92	0\\
58.93	0\\
58.94	0\\
58.95	0\\
58.96	0\\
58.97	0\\
58.98	0\\
58.99	0\\
59	0\\
59.01	0\\
59.02	0\\
59.03	0\\
59.04	0\\
59.05	0\\
59.06	0\\
59.07	0\\
59.08	0\\
59.09	0\\
59.1	0\\
59.11	0\\
59.12	0\\
59.13	0\\
59.14	0\\
59.15	0\\
59.16	0\\
59.17	0\\
59.18	0\\
59.19	0\\
59.2	0\\
59.21	0\\
59.22	0\\
59.23	0\\
59.24	0\\
59.25	0\\
59.26	0\\
59.27	0\\
59.28	0\\
59.29	0\\
59.3	0\\
59.31	0\\
59.32	0\\
59.33	0\\
59.34	0\\
59.35	0\\
59.36	0\\
59.37	0\\
59.38	0\\
59.39	0\\
59.4	0\\
59.41	0\\
59.42	0\\
59.43	0\\
59.44	0\\
59.45	0\\
59.46	0\\
59.47	0\\
59.48	0\\
59.49	0\\
59.5	0\\
59.51	0\\
59.52	0\\
59.53	0\\
59.54	0\\
59.55	0\\
59.56	0\\
59.57	0\\
59.58	0\\
59.59	0\\
59.6	0\\
59.61	0\\
59.62	0\\
59.63	0\\
59.64	0\\
59.65	0\\
59.66	0\\
59.67	0\\
59.68	0\\
59.69	0\\
59.7	0\\
59.71	0\\
59.72	0\\
59.73	0\\
59.74	0\\
59.75	0\\
59.76	0\\
59.77	0\\
59.78	0\\
59.79	0\\
59.8	0\\
59.81	0\\
59.82	0\\
59.83	0\\
59.84	0\\
59.85	0\\
59.86	0\\
59.87	0\\
59.88	0\\
59.89	0\\
59.9	0\\
59.91	0\\
59.92	0\\
59.93	0\\
59.94	0\\
59.95	0\\
59.96	0\\
59.97	0\\
59.98	0\\
59.99	0\\
60	0\\
60.01	0\\
60.02	0\\
60.03	0\\
60.04	0\\
60.05	0\\
60.06	0\\
60.07	0\\
60.08	0\\
60.09	0\\
60.1	0\\
60.11	0\\
60.12	0\\
60.13	0\\
60.14	0\\
60.15	0\\
60.16	0\\
60.17	0\\
60.18	0\\
60.19	0\\
60.2	0\\
60.21	0\\
60.22	0\\
60.23	0\\
60.24	0\\
60.25	0\\
60.26	0\\
60.27	0\\
60.28	0\\
60.29	0\\
60.3	0\\
60.31	0\\
60.32	0\\
60.33	0\\
60.34	0\\
60.35	0\\
60.36	0\\
60.37	0\\
60.38	0\\
60.39	0\\
60.4	0\\
60.41	0\\
60.42	0\\
60.43	0\\
60.44	0\\
60.45	0\\
60.46	0\\
60.47	0\\
60.48	0\\
60.49	0\\
60.5	0\\
60.51	0\\
60.52	0\\
60.53	0\\
60.54	0\\
60.55	0\\
60.56	1.57822659636181e-06\\
60.57	3.32362392112609e-06\\
60.58	5.07013025857761e-06\\
60.59	6.81774671597302e-06\\
60.6	8.56647440161326e-06\\
60.61	1.03163144248436e-05\\
60.62	1.20672678960328e-05\\
60.63	1.38193359265799e-05\\
60.64	1.55725196289008e-05\\
60.65	1.73268201164208e-05\\
60.66	1.90822385035712e-05\\
60.67	2.08387759057724e-05\\
60.68	2.25964334394368e-05\\
60.69	2.4355212221959e-05\\
60.7	2.61151133717046e-05\\
60.71	2.7876138007997e-05\\
60.72	2.96382872511242e-05\\
60.73	3.14015622223142e-05\\
60.74	3.31659640437389e-05\\
60.75	3.49314938384963e-05\\
60.76	3.66981527306076e-05\\
60.77	3.84659418450062e-05\\
60.78	4.02348623075241e-05\\
60.79	4.2004915244892e-05\\
60.8	4.37761017847219e-05\\
60.81	4.55484230554998e-05\\
60.82	4.73218801865759e-05\\
60.83	4.90964743081537e-05\\
60.84	5.08722065512834e-05\\
60.85	5.26490780478478e-05\\
60.86	5.44270899305484e-05\\
60.87	5.62062433329055e-05\\
60.88	5.79865393892341e-05\\
60.89	5.97679792346435e-05\\
60.9	6.15505640050168e-05\\
60.91	6.33342948370107e-05\\
60.92	6.51191728680348e-05\\
60.93	6.69051992362377e-05\\
60.94	6.86923750805035e-05\\
60.95	7.04807015404312e-05\\
60.96	7.22701797563341e-05\\
60.97	7.40608108692094e-05\\
60.98	7.5852596020741e-05\\
60.99	7.76455363532755e-05\\
61	7.94396330098188e-05\\
61.01	8.12348871340116e-05\\
61.02	8.30312998701226e-05\\
61.03	8.4828872363038e-05\\
61.04	8.66276057582341e-05\\
61.05	8.84275012017731e-05\\
61.06	9.02285598402865e-05\\
61.07	9.20307828209573e-05\\
61.08	9.38341712915097e-05\\
61.09	9.56387264001886e-05\\
61.1	9.74444492957451e-05\\
61.11	9.92513411274196e-05\\
61.12	0.000101059403044931\\
61.13	0.000102868636198454\\
61.14	0.000104679041738602\\
61.15	0.000106490620816416\\
61.16	0.000108303374583341\\
61.17	0.00011011730419121\\
61.18	0.000111932410792229\\
61.19	0.000113748695538953\\
61.2	0.000115566159584274\\
61.21	0.0001173848040814\\
61.22	0.00011920463018383\\
61.23	0.000121025639045345\\
61.24	0.00012284783181998\\
61.25	0.000124671209662006\\
61.26	0.000126495773725914\\
61.27	0.000128321525166387\\
61.28	0.000130148465138286\\
61.29	0.000131976594796621\\
61.3	0.000133805915296537\\
61.31	0.000135636427793288\\
61.32	0.000137468133442214\\
61.33	0.000139301033398723\\
61.34	0.000141135128818262\\
61.35	0.000142970420856297\\
61.36	0.000144806910668288\\
61.37	0.000146644599409669\\
61.38	0.000148483488235814\\
61.39	0.000150323578302022\\
61.4	0.000152164870763485\\
61.41	0.000154007366775271\\
61.42	0.000155851067492285\\
61.43	0.000157695974069252\\
61.44	0.00015954208766069\\
61.45	0.000161389409420881\\
61.46	0.000163237940503838\\
61.47	0.000165087682063289\\
61.48	0.000166938635252637\\
61.49	0.000168790801224936\\
61.5	0.000170644181132863\\
61.51	0.000172498776128686\\
61.52	0.000174354587364232\\
61.53	0.000176211615990862\\
61.54	0.000178069863159432\\
61.55	0.00017992933002027\\
61.56	0.000181790017723137\\
61.57	0.000183651927417199\\
61.58	0.000185515060250992\\
61.59	0.000187379417372385\\
61.6	0.000189244999928555\\
61.61	0.00019111180906594\\
61.62	0.000192979845930218\\
61.63	0.00019484911166626\\
61.64	0.0001967196074181\\
61.65	0.000198591334328892\\
61.66	0.000200464293540883\\
61.67	0.000202338486195369\\
61.68	0.000204213913432654\\
61.69	0.000206090576392013\\
61.7	0.00020796847621166\\
61.71	0.000209847614028695\\
61.72	0.000211727990979078\\
61.73	0.000213609608197572\\
61.74	0.000215492466817714\\
61.75	0.000217376567971769\\
61.76	0.000219261912790682\\
61.77	0.000221148502404039\\
61.78	0.000223036337940026\\
61.79	0.000224925420525372\\
61.8	0.000226815751285317\\
61.81	0.000228707331343558\\
61.82	0.000230600161822209\\
61.83	0.000232494243841742\\
61.84	0.000234389578520944\\
61.85	0.000236286166976876\\
61.86	0.000238184010324809\\
61.87	0.000240083109678189\\
61.88	0.000241983466148572\\
61.89	0.000243885080845579\\
61.9	0.000245787954876842\\
61.91	0.00024769208968017\\
61.92	0.000249597486707175\\
61.93	0.000251504147411881\\
61.94	0.000253412073250726\\
61.95	0.000255321265682566\\
61.96	0.000257231726168684\\
61.97	0.000259143456172787\\
61.98	0.00026105645716102\\
61.99	0.000262970730601962\\
62	0.000264886277966638\\
62.01	0.000266803100728522\\
62.02	0.000268721200363538\\
62.03	0.000270640578350068\\
62.04	0.000272561236168957\\
62.05	0.000274483175303514\\
62.06	0.000276406397239522\\
62.07	0.000278330903465244\\
62.08	0.000280256695471416\\
62.09	0.000282183774751265\\
62.1	0.000284112142800511\\
62.11	0.000286041801117364\\
62.12	0.000287972751202539\\
62.13	0.000289904994559251\\
62.14	0.000291838532693236\\
62.15	0.00029377336711273\\
62.16	0.0002957094993285\\
62.17	0.000297646930853834\\
62.18	0.000299585663204548\\
62.19	0.000301525697898992\\
62.2	0.000303467036458057\\
62.21	0.000305409680405176\\
62.22	0.000307353631266334\\
62.23	0.000309298890570062\\
62.24	0.000311245459847457\\
62.25	0.000313193340632175\\
62.26	0.000315142534460442\\
62.27	0.000317093042871055\\
62.28	0.000319044867405388\\
62.29	0.000320998009607398\\
62.3	0.000322952471023628\\
62.31	0.000324908253203218\\
62.32	0.000326865357697898\\
62.33	0.000328823786062003\\
62.34	0.000330783539852473\\
62.35	0.000332744620628864\\
62.36	0.000334707029953342\\
62.37	0.000336670769390694\\
62.38	0.000338635840508337\\
62.39	0.000340602244876316\\
62.4	0.000342569984067308\\
62.41	0.000344539059656635\\
62.42	0.000346509473222262\\
62.43	0.000348481226344805\\
62.44	0.00035045432060753\\
62.45	0.000352428757596366\\
62.46	0.000354404538899903\\
62.47	0.000356381666109403\\
62.48	0.0003583601408188\\
62.49	0.000360339964624701\\
62.5	0.000362321139126404\\
62.51	0.000364303665925891\\
62.52	0.000366287546627833\\
62.53	0.000368272782839602\\
62.54	0.000370259376171272\\
62.55	0.000372247328235622\\
62.56	0.000374236640648139\\
62.57	0.000376227315027029\\
62.58	0.000378219352993221\\
62.59	0.000380212756170364\\
62.6	0.000382207526184838\\
62.61	0.000384203664665756\\
62.62	0.000386201173244973\\
62.63	0.000388200053557088\\
62.64	0.000390200307239442\\
62.65	0.000392201935932136\\
62.66	0.000394204941278023\\
62.67	0.000396209324922719\\
62.68	0.000398215088514606\\
62.69	0.000400222233704839\\
62.7	0.000402230762147342\\
62.71	0.000404240675498827\\
62.72	0.000406251975418784\\
62.73	0.000408264663569493\\
62.74	0.000410278741616026\\
62.75	0.000412294211226256\\
62.76	0.000414311074070854\\
62.77	0.000416329331823298\\
62.78	0.000418348986159878\\
62.79	0.000420370038759697\\
62.8	0.000422392491304675\\
62.81	0.000424416345479559\\
62.82	0.000426441602971922\\
62.83	0.000428468265472166\\
62.84	0.000430496334673534\\
62.85	0.000432525812272103\\
62.86	0.000434556699966799\\
62.87	0.000436588999459392\\
62.88	0.000438622712454508\\
62.89	0.000440657840659626\\
62.9	0.000442694385785088\\
62.91	0.000444732349544098\\
62.92	0.000446771733652731\\
62.93	0.000448812539829927\\
62.94	0.000450854769797511\\
62.95	0.000452898425280183\\
62.96	0.000454943508005526\\
62.97	0.00045699001970401\\
62.98	0.000459037962108999\\
62.99	0.00046108733695675\\
63	0.000463138145986419\\
63.01	0.000465190390940062\\
63.02	0.000467244073562641\\
63.03	0.000469299195602027\\
63.04	0.000471355758809005\\
63.05	0.000473413764937272\\
63.06	0.000475473215743448\\
63.07	0.000477534112987073\\
63.08	0.000479596458430615\\
63.09	0.000481660253839469\\
63.1	0.000483725500981962\\
63.11	0.000485792201629358\\
63.12	0.000487860357555856\\
63.13	0.000489929970538601\\
63.14	0.000492001042357679\\
63.15	0.000494073574796122\\
63.16	0.000496147569639914\\
63.17	0.000498223028677993\\
63.18	0.000500299953702248\\
63.19	0.000502378346507532\\
63.2	0.00050445820889165\\
63.21	0.000506539542655379\\
63.22	0.000508622349602458\\
63.23	0.000510706631539591\\
63.24	0.000512792390276455\\
63.25	0.000514879627625697\\
63.26	0.000516968345402945\\
63.27	0.000519058545426797\\
63.28	0.000521150229518829\\
63.29	0.000523243399503603\\
63.3	0.000525338057208658\\
63.31	0.000527434204464518\\
63.32	0.000529531843104697\\
63.33	0.00053163097496569\\
63.34	0.000533731601886984\\
63.35	0.000535833725711057\\
63.36	0.000537937348283373\\
63.37	0.000540042471452395\\
63.38	0.000542149097069577\\
63.39	0.000544257226989368\\
63.4	0.000546366863069209\\
63.41	0.000548478007169544\\
63.42	0.000550590661153809\\
63.43	0.000552704826888437\\
63.44	0.000554820506242861\\
63.45	0.000556937701089513\\
63.46	0.000559056413303824\\
63.47	0.000561176644764221\\
63.48	0.000563298397352131\\
63.49	0.000565421672951979\\
63.5	0.000567546473451187\\
63.51	0.000569672800740179\\
63.52	0.000571800656712373\\
63.53	0.000573930043264188\\
63.54	0.000576060962295032\\
63.55	0.000578193415707313\\
63.56	0.000580327405406592\\
63.57	0.000582462933301751\\
63.58	0.000584600001304995\\
63.59	0.000586738611331863\\
63.6	0.00058887876530124\\
63.61	0.000591020465135348\\
63.62	0.000593163712759772\\
63.63	0.000595308510103455\\
63.64	0.00059745485909871\\
63.65	0.000599602761681219\\
63.66	0.000601752219790055\\
63.67	0.000603903235367673\\
63.68	0.000606055810359923\\
63.69	0.000608209946716062\\
63.7	0.000610365646388753\\
63.71	0.000612522911334082\\
63.72	0.000614681743511553\\
63.73	0.000616842144884104\\
63.74	0.000619004117418109\\
63.75	0.000621167663083393\\
63.76	0.000623332783853223\\
63.77	0.000625499481704334\\
63.78	0.000627667758616928\\
63.79	0.000629837616574678\\
63.8	0.000632009057564734\\
63.81	0.000634182083577742\\
63.82	0.000636356696607839\\
63.83	0.000638532898652663\\
63.84	0.000640710691713366\\
63.85	0.000642890077794615\\
63.86	0.0006450710589046\\
63.87	0.000647253637055048\\
63.88	0.00064943781426122\\
63.89	0.000651623592541926\\
63.9	0.000653810973919526\\
63.91	0.000655999960419946\\
63.92	0.000658190554072676\\
63.93	0.000660382756910785\\
63.94	0.000662576570970921\\
63.95	0.000664771998293328\\
63.96	0.000666969040921846\\
63.97	0.000669167700903917\\
63.98	0.000671367980290601\\
63.99	0.000673569881136576\\
64	0.000675773405500145\\
64.01	0.00067797855544325\\
64.02	0.000680185333031477\\
64.03	0.000682393740334061\\
64.04	0.000684603779423892\\
64.05	0.000686815452377532\\
64.06	0.000689028761275211\\
64.07	0.000691243708200839\\
64.08	0.000693460295242018\\
64.09	0.000695678524490043\\
64.1	0.000697898398039919\\
64.11	0.000700119917990353\\
64.12	0.000702343086443777\\
64.13	0.000704567905506347\\
64.14	0.000706794377287955\\
64.15	0.000709022503902236\\
64.16	0.000711252287466574\\
64.17	0.000713483730102112\\
64.18	0.000715716833933754\\
64.19	0.000717951601090179\\
64.2	0.000720188033703852\\
64.21	0.000722426133911019\\
64.22	0.000724665903851729\\
64.23	0.000726907345669832\\
64.24	0.000729150461512996\\
64.25	0.000731395253532701\\
64.26	0.000733641723884258\\
64.27	0.000735889874726819\\
64.28	0.000738139708223375\\
64.29	0.000740391226540772\\
64.3	0.000742644431849713\\
64.31	0.000744899326324773\\
64.32	0.000747155912144398\\
64.33	0.000749414191490924\\
64.34	0.000751674166550573\\
64.35	0.00075393583951347\\
64.36	0.000756199212573649\\
64.37	0.000758464287929059\\
64.38	0.000760731067781576\\
64.39	0.000762999554337004\\
64.4	0.000765269749805092\\
64.41	0.000767541656399532\\
64.42	0.00076981527633798\\
64.43	0.000772090611842052\\
64.44	0.000774367665137339\\
64.45	0.000776646438453411\\
64.46	0.000778926934023833\\
64.47	0.000781209154086165\\
64.48	0.000783493100881973\\
64.49	0.000785778776656837\\
64.5	0.000788066183660362\\
64.51	0.000790355324146182\\
64.52	0.00079264620037197\\
64.53	0.000794938814599449\\
64.54	0.000797233169094396\\
64.55	0.000799529266126652\\
64.56	0.000801827107970134\\
64.57	0.00080412669690284\\
64.58	0.000806428035206853\\
64.59	0.000808731125168358\\
64.6	0.00081103596907765\\
64.61	0.000813342569229127\\
64.62	0.000815650927921324\\
64.63	0.000817961047456899\\
64.64	0.000820272930142654\\
64.65	0.000822586578289541\\
64.66	0.00082490199421267\\
64.67	0.00082721918023131\\
64.68	0.000829538138668914\\
64.69	0.000831858871853112\\
64.7	0.000834181382115728\\
64.71	0.000836505671792789\\
64.72	0.000838831743224529\\
64.73	0.000841159598755394\\
64.74	0.000843489240734067\\
64.75	0.00084582067151346\\
64.76	0.000848153893450727\\
64.77	0.000850488908907279\\
64.78	0.000852825720248786\\
64.79	0.000855164329845188\\
64.8	0.000857504740070705\\
64.81	0.000859846953303843\\
64.82	0.000862190971927405\\
64.83	0.000864536798328503\\
64.84	0.000866884434898553\\
64.85	0.000869233884033302\\
64.86	0.000871585148132827\\
64.87	0.00087393822960154\\
64.88	0.000876293130848214\\
64.89	0.000878649854285967\\
64.9	0.000881008402332292\\
64.91	0.000883368777409056\\
64.92	0.000885730981942515\\
64.93	0.000888095018363309\\
64.94	0.000890460889106489\\
64.95	0.000892828596611515\\
64.96	0.000895198143322271\\
64.97	0.000897569531687063\\
64.98	0.000899942764158647\\
64.99	0.000902317843194216\\
65	0.000904694771255429\\
65.01	0.000907073550808404\\
65.02	0.000909454184323737\\
65.03	0.000911836674276509\\
65.04	0.00091422102314629\\
65.05	0.000916607233417157\\
65.06	0.000918995307577698\\
65.07	0.000921385248121021\\
65.08	0.000923777057544761\\
65.09	0.000926170738351096\\
65.1	0.000928566293046749\\
65.11	0.000930963724142999\\
65.12	0.000933363034155699\\
65.13	0.000935764225605269\\
65.14	0.000938167301016719\\
65.15	0.000940572262919652\\
65.16	0.000942979113848275\\
65.17	0.000945387856341403\\
65.18	0.000947798492942482\\
65.19	0.00095021102619958\\
65.2	0.000952625458665413\\
65.21	0.000955041792897345\\
65.22	0.000957460031457393\\
65.23	0.000959880176912252\\
65.24	0.000962302231833287\\
65.25	0.000964726198796555\\
65.26	0.00096715208038281\\
65.27	0.00096957987917751\\
65.28	0.000972009597770829\\
65.29	0.000974441238757662\\
65.3	0.000976874804737647\\
65.31	0.000979310298315161\\
65.32	0.00098174772209933\\
65.33	0.000984187078704051\\
65.34	0.000986628370747984\\
65.35	0.000989071600854579\\
65.36	0.000991516771652073\\
65.37	0.000993963885773504\\
65.38	0.000996412945856718\\
65.39	0.000998863954544384\\
65.4	0.001001316914484\\
65.41	0.00100377182832789\\
65.42	0.00100622869873326\\
65.43	0.00100868752836213\\
65.44	0.00101114831988141\\
65.45	0.00101361107596289\\
65.46	0.00101607579928325\\
65.47	0.00101854249252404\\
65.48	0.00102101115837175\\
65.49	0.00102348179951775\\
65.5	0.00102595441865837\\
65.51	0.00102842901849485\\
65.52	0.00103090560173337\\
65.53	0.00103338417108509\\
65.54	0.00103586472926612\\
65.55	0.00103834727899752\\
65.56	0.00104083182300538\\
65.57	0.00104331836402074\\
65.58	0.00104580690477965\\
65.59	0.0010482974480232\\
65.6	0.00105078999649747\\
65.61	0.00105328455295357\\
65.62	0.00105578112014768\\
65.63	0.00105827970084101\\
65.64	0.00106078029779981\\
65.65	0.00106328291379545\\
65.66	0.00106578755160433\\
65.67	0.00106829421400797\\
65.68	0.00107080290379297\\
65.69	0.00107331362375106\\
65.7	0.00107582637667907\\
65.71	0.00107834116537896\\
65.72	0.00108085799265783\\
65.73	0.00108337686132792\\
65.74	0.00108589777420665\\
65.75	0.00108842073411659\\
65.76	0.00109094574388548\\
65.77	0.00109347280634626\\
65.78	0.00109600192433705\\
65.79	0.0010985331007012\\
65.8	0.00110106633828726\\
65.81	0.001103601639949\\
65.82	0.00110613900854543\\
65.83	0.00110867844694081\\
65.84	0.00111121995800464\\
65.85	0.00111376354461171\\
65.86	0.00111630920964206\\
65.87	0.00111885695598102\\
65.88	0.00112140678651922\\
65.89	0.00112395870415258\\
65.9	0.00112651271178234\\
65.91	0.00112906881231507\\
65.92	0.00113162700866266\\
65.93	0.00113418730374234\\
65.94	0.0011367497004767\\
65.95	0.0011393142017937\\
65.96	0.00114188081062664\\
65.97	0.00114444952991423\\
65.98	0.00114702036260055\\
65.99	0.0011495933116351\\
66	0.00115216837997278\\
66.01	0.0011547455705739\\
66.02	0.0011573248864042\\
66.03	0.00115990633043488\\
66.04	0.00116248990564256\\
66.05	0.00116507561500933\\
66.06	0.00116766346152276\\
66.07	0.00117025344817586\\
66.08	0.00117284557796717\\
66.09	0.0011754398539007\\
66.1	0.00117803627898597\\
66.11	0.00118063485623801\\
66.12	0.00118323558867738\\
66.13	0.00118583847933018\\
66.14	0.00118844353122806\\
66.15	0.00119105074740818\\
66.16	0.00119366013091331\\
66.17	0.00119627168479177\\
66.18	0.00119888541209748\\
66.19	0.00120150131588991\\
66.2	0.00120411939923418\\
66.21	0.00120673966520098\\
66.22	0.00120936211686664\\
66.23	0.00121198675731311\\
66.24	0.00121461358962798\\
66.25	0.00121724261690449\\
66.26	0.00121987384224152\\
66.27	0.00122250726874365\\
66.28	0.00122514289952109\\
66.29	0.00122778073768976\\
66.3	0.00123042078637128\\
66.31	0.00123306304869295\\
66.32	0.0012357075277878\\
66.33	0.00123835422679455\\
66.34	0.00124100314885769\\
66.35	0.00124365429712743\\
66.36	0.00124630767475972\\
66.37	0.00124896328491626\\
66.38	0.00125162113076455\\
66.39	0.00125428121547783\\
66.4	0.00125694354223513\\
66.41	0.00125960811422128\\
66.42	0.00126227493462692\\
66.43	0.00126494400664847\\
66.44	0.00126761533348818\\
66.45	0.00127028891835415\\
66.46	0.0012729647644603\\
66.47	0.00127564287502637\\
66.48	0.00127832325327798\\
66.49	0.00128100590244662\\
66.5	0.00128369082576963\\
66.51	0.00128637802649022\\
66.52	0.00128906750785752\\
66.53	0.00129175927312652\\
66.54	0.00129445332555814\\
66.55	0.0012971496684192\\
66.56	0.00129984830498244\\
66.57	0.00130254923852652\\
66.58	0.00130525247233607\\
66.59	0.00130795800970161\\
66.6	0.00131066585391967\\
66.61	0.00131337600829269\\
66.62	0.00131608847612912\\
66.63	0.00131880326074336\\
66.64	0.0013215203654558\\
66.65	0.00132423979359283\\
66.66	0.00132696154848684\\
66.67	0.00132968563347621\\
66.68	0.00133241205190535\\
66.69	0.00133514080712468\\
66.7	0.00133787190249068\\
66.71	0.00134060534136582\\
66.72	0.00134334112711866\\
66.73	0.00134607926312379\\
66.74	0.00134881975276185\\
66.75	0.00135156259941956\\
66.76	0.00135430780648972\\
66.77	0.0013570553773712\\
66.78	0.00135980531546895\\
66.79	0.00136255762419402\\
66.8	0.00136531230696358\\
66.81	0.00136806936720087\\
66.82	0.00137082880833527\\
66.83	0.00137359063380228\\
66.84	0.00137635484704351\\
66.85	0.00137912145150673\\
66.86	0.00138189045064583\\
66.87	0.00138466184792084\\
66.88	0.00138743564679796\\
66.89	0.00139021185074954\\
66.9	0.00139299046325408\\
66.91	0.00139577148779628\\
66.92	0.00139855492786698\\
66.93	0.00140134078696324\\
66.94	0.00140412906858826\\
66.95	0.00140691977625147\\
66.96	0.00140971291346847\\
66.97	0.00141250848376109\\
66.98	0.00141530649065734\\
66.99	0.00141810693769146\\
67	0.0014209098284039\\
67.01	0.00142371516634133\\
67.02	0.00142652295505664\\
67.03	0.00142933319810897\\
67.04	0.00143214589906368\\
67.05	0.00143496106149238\\
67.06	0.00143777868897291\\
67.07	0.00144059878508937\\
67.08	0.00144342135343209\\
67.09	0.0014462463975977\\
67.1	0.00144907392118903\\
67.11	0.00145190392781522\\
67.12	0.00145473642109166\\
67.13	0.00145757140463999\\
67.14	0.00146040888208815\\
67.15	0.00146324885707035\\
67.16	0.00146609133322705\\
67.17	0.00146893631420503\\
67.18	0.00147178380365732\\
67.19	0.00147463380524325\\
67.2	0.00147748632262844\\
67.21	0.0014803413594848\\
67.22	0.0014831989194905\\
67.23	0.00148605900633005\\
67.24	0.00148892162369423\\
67.25	0.0014917867752801\\
67.26	0.00149465446479104\\
67.27	0.00149752469593672\\
67.28	0.0015003974724331\\
67.29	0.00150327279800246\\
67.3	0.00150615067637336\\
67.31	0.00150903111128066\\
67.32	0.00151191410646552\\
67.33	0.0015147996656754\\
67.34	0.00151768779266408\\
67.35	0.0015205784911916\\
67.36	0.00152347176502431\\
67.37	0.00152636761793488\\
67.38	0.00152926605370226\\
67.39	0.00153216707611167\\
67.4	0.00153507068895465\\
67.41	0.00153797689602902\\
67.42	0.0015408857011389\\
67.43	0.00154379710809468\\
67.44	0.00154671112071303\\
67.45	0.00154962774281692\\
67.46	0.00155254697823558\\
67.47	0.00155546883080452\\
67.48	0.00155839330436551\\
67.49	0.00156132040276662\\
67.5	0.00156425012986213\\
67.51	0.00156718248951263\\
67.52	0.00157011748558492\\
67.53	0.00157305512195207\\
67.54	0.00157599540249339\\
67.55	0.00157893833109444\\
67.56	0.00158188391164699\\
67.57	0.00158483214804905\\
67.58	0.00158778304420485\\
67.59	0.00159073660402482\\
67.6	0.00159369283142561\\
67.61	0.00159665173033006\\
67.62	0.00159961330466722\\
67.63	0.00160257755837229\\
67.64	0.00160554449538668\\
67.65	0.00160851411965795\\
67.66	0.00161148643513981\\
67.67	0.00161446144579214\\
67.68	0.00161743915558094\\
67.69	0.00162041956847836\\
67.7	0.00162340268846264\\
67.71	0.00162638851951816\\
67.72	0.00162937706563538\\
67.73	0.00163236833081084\\
67.74	0.00163536231904719\\
67.75	0.00163835903435309\\
67.76	0.00164135848074329\\
67.77	0.00164436066223857\\
67.78	0.00164736558286573\\
67.79	0.00165037324665756\\
67.8	0.00165338365765289\\
67.81	0.0016563968198965\\
67.82	0.00165941273743914\\
67.83	0.00166243141433752\\
67.84	0.00166545285465428\\
67.85	0.00166847706245798\\
67.86	0.00167150404182309\\
67.87	0.00167453379682996\\
67.88	0.0016775663315648\\
67.89	0.00168060165011968\\
67.9	0.0016836397565925\\
67.91	0.00168668065508698\\
67.92	0.00168972434971261\\
67.93	0.00169277084458467\\
67.94	0.00169582014382418\\
67.95	0.00169887225155791\\
67.96	0.00170192717191833\\
67.97	0.00170498490904358\\
67.98	0.00170804546707749\\
67.99	0.00171110885016952\\
68	0.00171417506247474\\
68.01	0.00171724410815382\\
68.02	0.00172031599137299\\
68.03	0.00172339071630405\\
68.04	0.00172646828712427\\
68.05	0.00172954870801645\\
68.06	0.00173263198316884\\
68.07	0.0017357181167751\\
68.08	0.00173880711303434\\
68.09	0.001741898976151\\
68.1	0.00174499371033489\\
68.11	0.00174809131980114\\
68.12	0.00175119180881041\\
68.13	0.0017542951816527\\
68.14	0.00175740144262318\\
68.15	0.00176051059602213\\
68.16	0.00176362264615492\\
68.17	0.001766737597332\\
68.18	0.00176985545386882\\
68.19	0.0017729762200858\\
68.2	0.0017760999003083\\
68.21	0.00177922649886662\\
68.22	0.00178235602009586\\
68.23	0.001785488468336\\
68.24	0.00178862384793176\\
68.25	0.00179176216323261\\
68.26	0.00179490341859271\\
68.27	0.00179804761837086\\
68.28	0.00180119476693048\\
68.29	0.00180434486863953\\
68.3	0.00180749792787049\\
68.31	0.0018106539490003\\
68.32	0.00181381293641031\\
68.33	0.00181697489448623\\
68.34	0.00182013982761811\\
68.35	0.00182330774020022\\
68.36	0.00182647863663106\\
68.37	0.0018296525213133\\
68.38	0.00183282939865368\\
68.39	0.001836009273063\\
68.4	0.00183919214895606\\
68.41	0.00184237803075157\\
68.42	0.00184556692287213\\
68.43	0.00184875882974414\\
68.44	0.00185195375579775\\
68.45	0.00185515170546682\\
68.46	0.00185835268318881\\
68.47	0.00186155669340476\\
68.48	0.00186476374055921\\
68.49	0.00186797382910011\\
68.5	0.00187118696347878\\
68.51	0.00187440314814986\\
68.52	0.00187762238757118\\
68.53	0.00188084468620373\\
68.54	0.00188407004851159\\
68.55	0.00188729847896183\\
68.56	0.00189052998202447\\
68.57	0.00189376456217238\\
68.58	0.00189700222388118\\
68.59	0.00190024297162921\\
68.6	0.00190348680989742\\
68.61	0.0019067337431693\\
68.62	0.00190998377593079\\
68.63	0.00191323691267016\\
68.64	0.00191649315787802\\
68.65	0.00191975251604712\\
68.66	0.00192301499167232\\
68.67	0.00192628058925053\\
68.68	0.00192954931328053\\
68.69	0.00193282116826293\\
68.7	0.0019360961587001\\
68.71	0.00193937428909602\\
68.72	0.00194265556395619\\
68.73	0.00194593998778756\\
68.74	0.00194922756509841\\
68.75	0.00195251830039823\\
68.76	0.00195581219819765\\
68.77	0.0019591092630083\\
68.78	0.0019624094993427\\
68.79	0.00196571291171418\\
68.8	0.00196901950463676\\
68.81	0.00197232928262499\\
68.82	0.00197564225019389\\
68.83	0.00197895841185881\\
68.84	0.00198227777213529\\
68.85	0.00198560033553898\\
68.86	0.00198892610658547\\
68.87	0.00199225508979019\\
68.88	0.00199558728966829\\
68.89	0.00199892271073447\\
68.9	0.00200226135750288\\
68.91	0.00200560323448699\\
68.92	0.00200894834619942\\
68.93	0.00201229669715183\\
68.94	0.00201564829312565\\
68.95	0.00201900314014025\\
68.96	0.00202236124423031\\
68.97	0.00202572261144596\\
68.98	0.00202908724785274\\
68.99	0.00203245515953172\\
69	0.00203582635257949\\
69.01	0.00203920083310826\\
69.02	0.00204257860724584\\
69.03	0.00204595968113575\\
69.04	0.00204934406093724\\
69.05	0.00205273175282534\\
69.06	0.00205612276299092\\
69.07	0.0020595170976407\\
69.08	0.00206291476299735\\
69.09	0.00206631576529951\\
69.1	0.00206972011080183\\
69.11	0.00207312780577504\\
69.12	0.00207653885650599\\
69.13	0.0020799532692977\\
69.14	0.00208337105046941\\
69.15	0.00208679220635663\\
69.16	0.00209021674331116\\
69.17	0.00209364466770122\\
69.18	0.0020970759859114\\
69.19	0.00210051070434277\\
69.2	0.00210394882941294\\
69.21	0.00210739036755606\\
69.22	0.00211083532522291\\
69.23	0.00211428370888094\\
69.24	0.00211773552501433\\
69.25	0.00212119078012401\\
69.26	0.00212464948072775\\
69.27	0.00212811163336019\\
69.28	0.0021315772445729\\
69.29	0.00213504632093443\\
69.3	0.00213851886903035\\
69.31	0.00214199489546333\\
69.32	0.00214547440685316\\
69.33	0.00214895740983683\\
69.34	0.00215244391106856\\
69.35	0.00215593391721987\\
69.36	0.00215942743497963\\
69.37	0.00216292447105411\\
69.38	0.00216642503216703\\
69.39	0.00216992912505962\\
69.4	0.00217343675649067\\
69.41	0.0021769479332366\\
69.42	0.00218046266209148\\
69.43	0.00218398094986712\\
69.44	0.00218750280339311\\
69.45	0.00219102822951687\\
69.46	0.0021945572351037\\
69.47	0.00219808982703687\\
69.48	0.00220162601221764\\
69.49	0.00220516579756531\\
69.5	0.00220870919001733\\
69.51	0.00221225619652928\\
69.52	0.00221580682407499\\
69.53	0.00221936107964656\\
69.54	0.00222291897025444\\
69.55	0.00222648050292747\\
69.56	0.00223004568471293\\
69.57	0.00223361452267663\\
69.58	0.00223718702390295\\
69.59	0.00224076319549487\\
69.6	0.00224434304457408\\
69.61	0.002247926578281\\
69.62	0.00225151380377484\\
69.63	0.0022551047282337\\
69.64	0.00225869935885456\\
69.65	0.00226229770285341\\
69.66	0.00226589976746525\\
69.67	0.0022695055599442\\
69.68	0.0022731150875635\\
69.69	0.00227672835761565\\
69.7	0.00228034537741239\\
69.71	0.00228396615428482\\
69.72	0.0022875906955834\\
69.73	0.0022912190086781\\
69.74	0.00229485110095836\\
69.75	0.00229848697983322\\
69.76	0.00230212665273137\\
69.77	0.00230577012710119\\
69.78	0.00230941741041081\\
69.79	0.00231306851014823\\
69.8	0.00231672343382129\\
69.81	0.00232038218895781\\
69.82	0.00232404478310562\\
69.83	0.00232771122383262\\
69.84	0.00233138151872685\\
69.85	0.00233505567539658\\
69.86	0.00233873370147031\\
69.87	0.0023424156045969\\
69.88	0.00234610139244559\\
69.89	0.00234979107270609\\
69.9	0.00235348465308863\\
69.91	0.00235718214132403\\
69.92	0.00236088354516377\\
69.93	0.00236458887238005\\
69.94	0.00236829752194856\\
69.95	0.00237200838171067\\
69.96	0.00237572145407267\\
69.97	0.0023794367414451\\
69.98	0.00238315424624278\\
69.99	0.0023868739708848\\
70	0.00239059591779454\\
70.01	0.00239432008939969\\
70.02	0.00239804648813223\\
70.03	0.00240177511642847\\
70.04	0.00240550597672905\\
70.05	0.00240923907147897\\
70.06	0.00241297440312756\\
70.07	0.00241671197412853\\
70.08	0.00242045178693995\\
70.09	0.00242419384402431\\
70.1	0.00242793814784846\\
70.11	0.0024316847008837\\
70.12	0.00243543350560571\\
70.13	0.00243918456449465\\
70.14	0.00244293788003507\\
70.15	0.00244669345471603\\
70.16	0.00245045129103103\\
70.17	0.00245421139147805\\
70.18	0.00245797375855956\\
70.19	0.00246173839478254\\
70.2	0.0024655053026585\\
70.21	0.00246927448470344\\
70.22	0.00247304594343793\\
70.23	0.00247681968138707\\
70.24	0.00248059570108053\\
70.25	0.00248437400505257\\
70.26	0.00248815459584201\\
70.27	0.00249193747599229\\
70.28	0.00249572264805145\\
70.29	0.00249951011457216\\
70.3	0.00250329987811171\\
70.31	0.00250709194123206\\
70.32	0.00251088630649982\\
70.33	0.00251468297648627\\
70.34	0.00251848195376738\\
70.35	0.00252228324092383\\
70.36	0.00252608684054098\\
70.37	0.00252989275520894\\
70.38	0.00253370098752255\\
70.39	0.0025375115400814\\
70.4	0.00254132441548984\\
70.41	0.00254513961635699\\
70.42	0.00254895714529677\\
70.43	0.00255277700492789\\
70.44	0.00255659919787389\\
70.45	0.00256042372676312\\
70.46	0.00256425059422878\\
70.47	0.00256807980290892\\
70.48	0.00257191135544648\\
70.49	0.00257574525448925\\
70.5	0.00257958150268992\\
70.51	0.00258342010270612\\
70.52	0.00258726105720036\\
70.53	0.00259110436884012\\
70.54	0.0025949500402978\\
70.55	0.0025987980742508\\
70.56	0.00260264847338147\\
70.57	0.00260650124037716\\
70.58	0.00261035637793022\\
70.59	0.00261421388873805\\
70.6	0.00261807377550304\\
70.61	0.00262193604093265\\
70.62	0.00262580068773943\\
70.63	0.00262966771864096\\
70.64	0.00263353713635995\\
70.65	0.0026374089436242\\
70.66	0.00264128314316664\\
70.67	0.00264515973772533\\
70.68	0.00264903873004348\\
70.69	0.00265292012286948\\
70.7	0.00265680391895689\\
70.71	0.00266069012106448\\
70.72	0.00266457873195622\\
70.73	0.00266846975440132\\
70.74	0.00267236319117421\\
70.75	0.00267625904505462\\
70.76	0.00268015731882751\\
70.77	0.00268405801528315\\
70.78	0.00268796113721714\\
70.79	0.00269186668743035\\
70.8	0.00269577466872904\\
70.81	0.00269968508392479\\
70.82	0.00270359793583457\\
70.83	0.00270751322728071\\
70.84	0.00271143096109099\\
70.85	0.00271535114009855\\
70.86	0.00271927376714201\\
70.87	0.00272319884506544\\
70.88	0.00272712637671836\\
70.89	0.00273105636495578\\
70.9	0.00273498881263822\\
70.91	0.00273892372263173\\
70.92	0.00274286109780788\\
70.93	0.0027468009410438\\
70.94	0.0027507432552222\\
70.95	0.00275468804323136\\
70.96	0.0027586353079652\\
70.97	0.00276258505232324\\
70.98	0.00276653727919371\\
70.99	0.00277049199134839\\
71	0.00277444919156405\\
71.01	0.00277840888262248\\
71.02	0.0027823710673105\\
71.03	0.00278633574841996\\
71.04	0.00279030292874776\\
71.05	0.0027942726110959\\
71.06	0.00279824479827141\\
71.07	0.00280221949308646\\
71.08	0.00280619669835828\\
71.09	0.00281017641690928\\
71.1	0.00281415865156696\\
71.11	0.00281814340516399\\
71.12	0.00282213068053819\\
71.13	0.00282612048053259\\
71.14	0.00283011280799538\\
71.15	0.00283410766577997\\
71.16	0.00283810505674499\\
71.17	0.00284210498375431\\
71.18	0.00284610744967704\\
71.19	0.00285011245738757\\
71.2	0.00285412000976556\\
71.21	0.00285813010969597\\
71.22	0.00286214276006905\\
71.23	0.00286615796378041\\
71.24	0.00287017572373096\\
71.25	0.00287419604282699\\
71.26	0.00287821892398015\\
71.27	0.00288224437010748\\
71.28	0.00288627238413141\\
71.29	0.00289030296897978\\
71.3	0.00289433612758587\\
71.31	0.0028983718628884\\
71.32	0.00290241017783157\\
71.33	0.00290645107536502\\
71.34	0.00291049455844392\\
71.35	0.00291454063002891\\
71.36	0.00291858929308618\\
71.37	0.00292264055058744\\
71.38	0.00292669440550998\\
71.39	0.00293075086083664\\
71.4	0.00293480991955585\\
71.41	0.00293887158466164\\
71.42	0.00294293585915368\\
71.43	0.00294700274603724\\
71.44	0.00295107224832327\\
71.45	0.00295514436902839\\
71.46	0.00295921911117488\\
71.47	0.00296329647779075\\
71.48	0.0029673764719097\\
71.49	0.00297145909657118\\
71.5	0.0029755443548204\\
71.51	0.00297963224970833\\
71.52	0.00298372278429171\\
71.53	0.00298781596163311\\
71.54	0.0029919117848009\\
71.55	0.0029960102568693\\
71.56	0.00300011138091838\\
71.57	0.00300421516003408\\
71.58	0.00300832159730824\\
71.59	0.00301243069583857\\
71.6	0.00301654245872876\\
71.61	0.00302065688908841\\
71.62	0.00302477399003307\\
71.63	0.00302889376468431\\
71.64	0.00303301621616965\\
71.65	0.00303714134762266\\
71.66	0.00304126916218293\\
71.67	0.00304539966299609\\
71.68	0.00304953285321386\\
71.69	0.00305366873599404\\
71.7	0.00305780731450053\\
71.71	0.00306194859190337\\
71.72	0.00306609257137874\\
71.73	0.00307023925610897\\
71.74	0.0030743886492826\\
71.75	0.00307854075409435\\
71.76	0.00308269557374517\\
71.77	0.00308685311144224\\
71.78	0.00309101337039903\\
71.79	0.00309517635383525\\
71.8	0.00309934206497693\\
71.81	0.00310351050705643\\
71.82	0.00310768168331243\\
71.83	0.00311185559698998\\
71.84	0.00311603225134051\\
71.85	0.00312021164962184\\
71.86	0.00312439379509821\\
71.87	0.00312857869104032\\
71.88	0.00313276634072531\\
71.89	0.00313695674743682\\
71.9	0.00314114991446497\\
71.91	0.00314534584510643\\
71.92	0.00314954454266438\\
71.93	0.00315374601044861\\
71.94	0.00315795025177546\\
71.95	0.0031621572699679\\
71.96	0.00316636706835552\\
71.97	0.00317057965027457\\
71.98	0.00317479501906797\\
71.99	0.00317901317808533\\
72	0.00318323413068298\\
72.01	0.003187457880224\\
72.02	0.00319168443007823\\
72.03	0.00319591378362228\\
72.04	0.00320014594423958\\
72.05	0.00320438091532039\\
72.06	0.00320861870026182\\
72.07	0.00321285930246786\\
72.08	0.0032171027253494\\
72.09	0.00322134897232424\\
72.1	0.00322559804681714\\
72.11	0.00322984995225982\\
72.12	0.00323410469209101\\
72.13	0.00323836226975643\\
72.14	0.00324262268870888\\
72.15	0.00324688595240818\\
72.16	0.00325115206432127\\
72.17	0.0032554210279222\\
72.18	0.00325969284669216\\
72.19	0.00326396752411948\\
72.2	0.00326824506369973\\
72.21	0.00327252546893563\\
72.22	0.00327680874333719\\
72.23	0.00328109489042167\\
72.24	0.0032853839137136\\
72.25	0.00328967581674485\\
72.26	0.00329397060305463\\
72.27	0.00329826827618951\\
72.28	0.00330256883970345\\
72.29	0.00330687229715784\\
72.3	0.00331117865212151\\
72.31	0.00331548790817078\\
72.32	0.00331980006888946\\
72.33	0.00332411513786888\\
72.34	0.00332843311870794\\
72.35	0.00333275401501312\\
72.36	0.0033370778303985\\
72.37	0.00334140456848581\\
72.38	0.00334573423290446\\
72.39	0.00335006682729152\\
72.4	0.00335440235529181\\
72.41	0.00335874082055789\\
72.42	0.00336308222675013\\
72.43	0.00336742657753667\\
72.44	0.00337177387659352\\
72.45	0.00337612412760453\\
72.46	0.00338047733426148\\
72.47	0.00338483350026406\\
72.48	0.00338919262931993\\
72.49	0.00339355472514472\\
72.5	0.0033979197914621\\
72.51	0.0034022878320038\\
72.52	0.00340665885050959\\
72.53	0.00341103285072739\\
72.54	0.00341540983641326\\
72.55	0.00341978981133141\\
72.56	0.0034241727792543\\
72.57	0.00342855874396259\\
72.58	0.00343294770924523\\
72.59	0.00343733967889948\\
72.6	0.00344173465673093\\
72.61	0.00344613264655354\\
72.62	0.00345053365218968\\
72.63	0.00345493767747015\\
72.64	0.00345934472623423\\
72.65	0.0034637548023297\\
72.66	0.00346816790961289\\
72.67	0.0034725840519487\\
72.68	0.00347700323321062\\
72.69	0.00348142545728081\\
72.7	0.00348585072805011\\
72.71	0.00349027904941806\\
72.72	0.00349471042529296\\
72.73	0.00349914485959189\\
72.74	0.00350358235624079\\
72.75	0.0035080229191744\\
72.76	0.00351246655233642\\
72.77	0.00351691325967944\\
72.78	0.00352136304516506\\
72.79	0.00352581591276386\\
72.8	0.0035302718664555\\
72.81	0.00353473091022869\\
72.82	0.00353919304808131\\
72.83	0.00354365828402037\\
72.84	0.00354812662206213\\
72.85	0.00355259806623204\\
72.86	0.00355707262056489\\
72.87	0.00356155028910475\\
72.88	0.0035660310759051\\
72.89	0.0035705149850288\\
72.9	0.00357500202054817\\
72.91	0.00357949218654502\\
72.92	0.0035839854871107\\
72.93	0.00358848192634614\\
72.94	0.00359298150836187\\
72.95	0.0035974842372781\\
72.96	0.00360199011722475\\
72.97	0.00360649915234148\\
72.98	0.00361101134677775\\
72.99	0.00361552670469286\\
73	0.003620045230256\\
73.01	0.00362456692764628\\
73.02	0.00362909180105279\\
73.03	0.00363361985467465\\
73.04	0.00363815109272104\\
73.05	0.00364268551941127\\
73.06	0.0036472231389748\\
73.07	0.00365176395565131\\
73.08	0.00365630797369074\\
73.09	0.00366085519735334\\
73.1	0.00366540563090972\\
73.11	0.00366995927864089\\
73.12	0.00367451614483832\\
73.13	0.00367907623380401\\
73.14	0.00368363954985049\\
73.15	0.00368820609730092\\
73.16	0.00369277588048912\\
73.17	0.00369734890375962\\
73.18	0.00370192517146773\\
73.19	0.00370650468797957\\
73.2	0.00371108745767214\\
73.21	0.00371567348493338\\
73.22	0.0037202627741622\\
73.23	0.00372485532976857\\
73.24	0.00372945115617352\\
73.25	0.00373405025780928\\
73.26	0.00373865263911926\\
73.27	0.00374325830455813\\
73.28	0.00374786725859191\\
73.29	0.003752479505698\\
73.3	0.00375709505036521\\
73.31	0.0037617138970939\\
73.32	0.00376633605039596\\
73.33	0.0037709615147949\\
73.34	0.00377559029482595\\
73.35	0.00378022239503604\\
73.36	0.00378485781998394\\
73.37	0.00378949657424028\\
73.38	0.00379413866238764\\
73.39	0.00379878408902057\\
73.4	0.00380343285874572\\
73.41	0.00380808497618184\\
73.42	0.00381274044595991\\
73.43	0.00381739927272313\\
73.44	0.00382206146112707\\
73.45	0.00382672701583968\\
73.46	0.00383139594154139\\
73.47	0.00383606824292515\\
73.48	0.00384074392469653\\
73.49	0.00384542299157378\\
73.5	0.0038501054482879\\
73.51	0.0038547912995827\\
73.52	0.00385948055021491\\
73.53	0.0038641732049542\\
73.54	0.00386886926858332\\
73.55	0.0038735687458981\\
73.56	0.0038782716417076\\
73.57	0.00388297796083414\\
73.58	0.00388768770811337\\
73.59	0.00389240088839441\\
73.6	0.00389711750653985\\
73.61	0.00390183756742589\\
73.62	0.00390656107594238\\
73.63	0.00391128803699295\\
73.64	0.00391601845549503\\
73.65	0.00392075233637999\\
73.66	0.00392548968459318\\
73.67	0.00393023050509407\\
73.68	0.00393497480285627\\
73.69	0.00393972258286766\\
73.7	0.00394447385013046\\
73.71	0.00394922860966134\\
73.72	0.00395398686649149\\
73.73	0.0039587486256667\\
73.74	0.00396351389224749\\
73.75	0.00396828267130917\\
73.76	0.00397305496794195\\
73.77	0.00397783078725101\\
73.78	0.00398261013435665\\
73.79	0.00398739301439431\\
73.8	0.00399217943251474\\
73.81	0.00399696939388405\\
73.82	0.00400176290368383\\
73.83	0.00400655996711126\\
73.84	0.00401136058937918\\
73.85	0.00401616477571623\\
73.86	0.00402097253136693\\
73.87	0.0040257838615918\\
73.88	0.00403059877166745\\
73.89	0.0040354172668867\\
73.9	0.00404023935255869\\
73.91	0.00404506503400898\\
73.92	0.00404989431657969\\
73.93	0.00405472720562955\\
73.94	0.00405956370653409\\
73.95	0.0040644038246857\\
73.96	0.00406924756549377\\
73.97	0.00407409493438481\\
73.98	0.00407894593680256\\
73.99	0.0040838005782081\\
74	0.00408865886408\\
74.01	0.00409352079991442\\
74.02	0.00409838639122525\\
74.03	0.00410325564354422\\
74.04	0.00410812856242104\\
74.05	0.00411300515342351\\
74.06	0.00411788542213769\\
74.07	0.00412276937416798\\
74.08	0.00412765701513731\\
74.09	0.00413254835068721\\
74.1	0.004137443386478\\
74.11	0.00414234212818891\\
74.12	0.00414724458151821\\
74.13	0.00415215075218336\\
74.14	0.00415706064592116\\
74.15	0.00416197426848788\\
74.16	0.00416689162565941\\
74.17	0.00417181272323141\\
74.18	0.00417673756701947\\
74.19	0.00418166616285925\\
74.2	0.00418659851660661\\
74.21	0.00419153463413783\\
74.22	0.00419647452134969\\
74.23	0.00420141818415968\\
74.24	0.00420636562850614\\
74.25	0.00421131686034844\\
74.26	0.00421627188566711\\
74.27	0.00422123071046405\\
74.28	0.00422619334076265\\
74.29	0.00423115978260801\\
74.3	0.00423613004206708\\
74.31	0.00424110412522883\\
74.32	0.00424608203820446\\
74.33	0.00425106378712754\\
74.34	0.00425604937815422\\
74.35	0.0042610388174634\\
74.36	0.0042660321112569\\
74.37	0.00427102926575967\\
74.38	0.00427603028721999\\
74.39	0.00428103518190962\\
74.4	0.00428604395612403\\
74.41	0.00429105661618259\\
74.42	0.00429607316842875\\
74.43	0.00430109361923027\\
74.44	0.00430611797497939\\
74.45	0.00431114624209308\\
74.46	0.00431617842701321\\
74.47	0.00432121453620676\\
74.48	0.00432625457616607\\
74.49	0.00433129855340905\\
74.5	0.00433634647447934\\
74.51	0.00434139834594662\\
74.52	0.00434645417440677\\
74.53	0.00435151396648212\\
74.54	0.0043565777288217\\
74.55	0.00436164546810142\\
74.56	0.00436671719102436\\
74.57	0.00437179290432099\\
74.58	0.0043768726147494\\
74.59	0.00438195632909557\\
74.6	0.00438704405417357\\
74.61	0.00439213579682589\\
74.62	0.00439723156392361\\
74.63	0.00440233136236672\\
74.64	0.00440743519908435\\
74.65	0.00441254308103504\\
74.66	0.00441765501520701\\
74.67	0.00442277100861842\\
74.68	0.00442789106831767\\
74.69	0.00443301520138366\\
74.7	0.00443814341492605\\
74.71	0.00444327571608559\\
74.72	0.00444841211203439\\
74.73	0.00445355260997618\\
74.74	0.00445869721714666\\
74.75	0.00446384594081375\\
74.76	0.00446899878827795\\
74.77	0.00447415576687257\\
74.78	0.00447931688396412\\
74.79	0.00448448214695258\\
74.8	0.00448965156327172\\
74.81	0.00449482514038944\\
74.82	0.00450000288580808\\
74.83	0.00450518480706478\\
74.84	0.00451037091173178\\
74.85	0.00451556120741678\\
74.86	0.0045207557017633\\
74.87	0.00452595440245099\\
74.88	0.004531157317196\\
74.89	0.00453636445375137\\
74.9	0.00454157581990735\\
74.91	0.00454679142349178\\
74.92	0.00455201127237047\\
74.93	0.00455723537444757\\
74.94	0.00456246373766596\\
74.95	0.00456769637000763\\
74.96	0.00457293327949407\\
74.97	0.00457817447418668\\
74.98	0.00458341996218714\\
74.99	0.00458866975163788\\
75	0.00459392385072242\\
75.01	0.00459918226766583\\
75.02	0.00460444501073517\\
75.03	0.00460971208823985\\
75.04	0.00461498350853216\\
75.05	0.00462025928000763\\
75.06	0.00462553941110551\\
75.07	0.00463082391030923\\
75.08	0.00463611278614684\\
75.09	0.0046414060471915\\
75.1	0.00464670370206192\\
75.11	0.00465200575821142\\
75.12	0.00465731222253704\\
75.13	0.00466262310195214\\
75.14	0.00466793840338646\\
75.15	0.00467325813378614\\
75.16	0.00467858230011376\\
75.17	0.00468391090934843\\
75.18	0.00468924396848578\\
75.19	0.00469458148453803\\
75.2	0.00469992346453403\\
75.21	0.00470526991551934\\
75.22	0.00471062084455619\\
75.23	0.00471597625872361\\
75.24	0.00472133616511744\\
75.25	0.00472670057085037\\
75.26	0.004732069483052\\
75.27	0.00473744290886886\\
75.28	0.00474282085546449\\
75.29	0.00474820333001946\\
75.3	0.00475359033973142\\
75.31	0.00475898189181515\\
75.32	0.0047643779935026\\
75.33	0.00476977865204295\\
75.34	0.00477518387470261\\
75.35	0.00478059366876535\\
75.36	0.00478600804153224\\
75.37	0.00479142700032179\\
75.38	0.00479685055246993\\
75.39	0.00480227870533008\\
75.4	0.00480771146627321\\
75.41	0.00481314884268786\\
75.42	0.00481859084198021\\
75.43	0.00482403747157407\\
75.44	0.00482948873891102\\
75.45	0.00483494465145036\\
75.46	0.00484040521666923\\
75.47	0.00484587044206259\\
75.48	0.00485134033514332\\
75.49	0.00485681490344223\\
75.5	0.00486229415450814\\
75.51	0.00486777809590788\\
75.52	0.00487143349896525\\
75.53	0.00487380021309105\\
75.54	0.00487616774895587\\
75.55	0.00487853610494478\\
75.56	0.00488090527943267\\
75.57	0.00488327527078415\\
75.58	0.00488564607735359\\
75.59	0.00488801769748496\\
75.6	0.00489039012951192\\
75.61	0.00489276337175763\\
75.62	0.00489513742253482\\
75.63	0.00489751228014567\\
75.64	0.00489988794288181\\
75.65	0.00490226440902422\\
75.66	0.00490464167684325\\
75.67	0.00490701974459849\\
75.68	0.0049093986105388\\
75.69	0.00491177827290222\\
75.7	0.00491415872991591\\
75.71	0.00491653997979613\\
75.72	0.00491892202074819\\
75.73	0.00492130485096638\\
75.74	0.0049236884686339\\
75.75	0.00492607287192291\\
75.76	0.00492845805899434\\
75.77	0.00493084402799794\\
75.78	0.00493323077707221\\
75.79	0.0049356183043443\\
75.8	0.00493800660793003\\
75.81	0.00494039568593379\\
75.82	0.0049427855364485\\
75.83	0.00494517615755557\\
75.84	0.00494756754732484\\
75.85	0.00494995970381451\\
75.86	0.00495235262507111\\
75.87	0.00495474630912946\\
75.88	0.00495714075401257\\
75.89	0.00495953595773162\\
75.9	0.00496193191828591\\
75.91	0.00496432863366279\\
75.92	0.00496672610183762\\
75.93	0.00496912432077368\\
75.94	0.00497152328842218\\
75.95	0.00497392300272215\\
75.96	0.00497632346160039\\
75.97	0.00497872466297145\\
75.98	0.00498112660473753\\
75.99	0.00498352928478848\\
76	0.00498593270100165\\
76.01	0.00498833685124197\\
76.02	0.00499074173336172\\
76.03	0.00499314734520067\\
76.04	0.00499555368458584\\
76.05	0.00499796074933156\\
76.06	0.00500036853723938\\
76.07	0.00500277704609799\\
76.08	0.00500518627368318\\
76.09	0.0050075962177578\\
76.1	0.00501000687607165\\
76.11	0.00501241824636148\\
76.12	0.00501483032635088\\
76.13	0.00501724311375025\\
76.14	0.00501965660625674\\
76.15	0.00502207080155417\\
76.16	0.00502448569731297\\
76.17	0.00502690129119017\\
76.18	0.00502931758082924\\
76.19	0.00503173456386013\\
76.2	0.00503415223789914\\
76.21	0.00503657060054891\\
76.22	0.0050389896493983\\
76.23	0.00504140938202235\\
76.24	0.00504382979598227\\
76.25	0.00504625088882527\\
76.26	0.0050486726580846\\
76.27	0.00505109510127942\\
76.28	0.00505351821591476\\
76.29	0.00505594199948146\\
76.3	0.0050583664494561\\
76.31	0.0050607915633009\\
76.32	0.00506321733846372\\
76.33	0.00506564377237795\\
76.34	0.00506807086246245\\
76.35	0.00507049860612148\\
76.36	0.00507292700074466\\
76.37	0.00507535604370685\\
76.38	0.00507778573236815\\
76.39	0.00508021606407378\\
76.4	0.00508264703615401\\
76.41	0.00508507864592413\\
76.42	0.00508751089068437\\
76.43	0.00508994376771979\\
76.44	0.00509237727430027\\
76.45	0.00509481140768038\\
76.46	0.00509724616509937\\
76.47	0.00509968154378105\\
76.48	0.00510211754093374\\
76.49	0.0051045541537502\\
76.5	0.00510699137940756\\
76.51	0.00510942921506722\\
76.52	0.00511186765787483\\
76.53	0.00511430670496016\\
76.54	0.00511674635343707\\
76.55	0.0051191866004034\\
76.56	0.00512162744294094\\
76.57	0.00512406887811529\\
76.58	0.00512651090297589\\
76.59	0.00512895351455581\\
76.6	0.00513139670987178\\
76.61	0.00513384048592408\\
76.62	0.00513628483969645\\
76.63	0.00513872976815602\\
76.64	0.00514117526825326\\
76.65	0.00514362133692187\\
76.66	0.00514606797107867\\
76.67	0.00514851516762364\\
76.68	0.00515096292343969\\
76.69	0.00515341123539272\\
76.7	0.00515586010033141\\
76.71	0.00515830951508726\\
76.72	0.00516075947647442\\
76.73	0.00516320998128966\\
76.74	0.00516566102631226\\
76.75	0.00516811260830394\\
76.76	0.0051705647240088\\
76.77	0.00517301737015317\\
76.78	0.00517547054344563\\
76.79	0.00517792405067916\\
76.8	0.00518037784931282\\
76.81	0.00518283193648724\\
76.82	0.00518528630933034\\
76.83	0.0051877409649573\\
76.84	0.00519019590047051\\
76.85	0.00519265111295945\\
76.86	0.00519510659950072\\
76.87	0.00519756235715793\\
76.88	0.00520001838298167\\
76.89	0.00520247467400943\\
76.9	0.00520493122726555\\
76.91	0.00520738803976118\\
76.92	0.0052098451084942\\
76.93	0.00521230243044918\\
76.94	0.0052147600025973\\
76.95	0.00521721782189631\\
76.96	0.00521967588529047\\
76.97	0.00522213418971046\\
76.98	0.00522459273207338\\
76.99	0.00522705150928262\\
77	0.00522951051822788\\
77.01	0.00523196975578501\\
77.02	0.00523442921881604\\
77.03	0.00523688890416905\\
77.04	0.00523934880867817\\
77.05	0.00524180892916347\\
77.06	0.00524426926243091\\
77.07	0.00524672980527229\\
77.08	0.00524919055446516\\
77.09	0.00525165150677279\\
77.1	0.00525411265894409\\
77.11	0.00525657400771352\\
77.12	0.00525903554980107\\
77.13	0.00526149728191217\\
77.14	0.00526395920073762\\
77.15	0.00526642130295356\\
77.16	0.00526888358522132\\
77.17	0.00527134604418746\\
77.18	0.00527380867648363\\
77.19	0.00527627147872652\\
77.2	0.00527873444751778\\
77.21	0.00528119757944401\\
77.22	0.0052836608710766\\
77.23	0.00528612431897174\\
77.24	0.00528858791967029\\
77.25	0.00529105166969776\\
77.26	0.0052935155655642\\
77.27	0.00529597960376415\\
77.28	0.00529844378077657\\
77.29	0.00530090809306474\\
77.3	0.00530337253707623\\
77.31	0.0053058371092428\\
77.32	0.00530830180598032\\
77.33	0.00531076662368871\\
77.34	0.00531323155875187\\
77.35	0.0053156966075376\\
77.36	0.00531816176639751\\
77.37	0.00532062703166697\\
77.38	0.00532309239966501\\
77.39	0.00532555786669424\\
77.4	0.00532802342904081\\
77.41	0.00533048908297431\\
77.42	0.00533295482474766\\
77.43	0.00533542065059708\\
77.44	0.00533788655674198\\
77.45	0.0053403525393849\\
77.46	0.00534281859471143\\
77.47	0.00534528471889007\\
77.48	0.00534775090807226\\
77.49	0.00535021715839219\\
77.5	0.00535268346596677\\
77.51	0.00535514982689556\\
77.52	0.00535761623726064\\
77.53	0.00536008269312656\\
77.54	0.00536254919054024\\
77.55	0.0053650157255309\\
77.56	0.00536748229410994\\
77.57	0.00536994889227091\\
77.58	0.00537241551598936\\
77.59	0.00537488216122279\\
77.6	0.00537734882391055\\
77.61	0.00537981549997377\\
77.62	0.00538228218531522\\
77.63	0.00538474887581929\\
77.64	0.00538721556735182\\
77.65	0.00538968225576009\\
77.66	0.00539214893687266\\
77.67	0.00539461560649932\\
77.68	0.00539708226043097\\
77.69	0.00539954889443953\\
77.7	0.00540201550427787\\
77.71	0.00540448208567968\\
77.72	0.0054069486343594\\
77.73	0.00540941514601209\\
77.74	0.00541188161631338\\
77.75	0.00541434804091932\\
77.76	0.00541681441546634\\
77.77	0.00541928073557107\\
77.78	0.00542174699683032\\
77.79	0.00542421319482093\\
77.8	0.00542667932509968\\
77.81	0.00542914538320318\\
77.82	0.00543161136464781\\
77.83	0.00543407726492953\\
77.84	0.00543654307952386\\
77.85	0.00543900880388572\\
77.86	0.00544147443344935\\
77.87	0.0054439399636282\\
77.88	0.00544640538981478\\
77.89	0.00544887070738064\\
77.9	0.00545133591167617\\
77.91	0.00545380099803052\\
77.92	0.00545626596175151\\
77.93	0.00545873079812549\\
77.94	0.00546119550241725\\
77.95	0.00546366006986986\\
77.96	0.00546612449570463\\
77.97	0.00546858877512091\\
77.98	0.00547105290329603\\
77.99	0.00547351687538518\\
78	0.00547598068652124\\
78.01	0.00547844433181472\\
78.02	0.00548090780635362\\
78.03	0.00548337110520328\\
78.04	0.0054858342234063\\
78.05	0.00548829715598238\\
78.06	0.00549075989792821\\
78.07	0.00549322244421761\\
78.08	0.00549568478980357\\
78.09	0.00549814692961819\\
78.1	0.00550060885857262\\
78.11	0.00550307057155694\\
78.12	0.00550553206344014\\
78.13	0.00550799332906995\\
78.14	0.00551045436327284\\
78.15	0.0055129151608539\\
78.16	0.00551537571659676\\
78.17	0.00551783602526349\\
78.18	0.00552029608159456\\
78.19	0.00552275588030871\\
78.2	0.00552521541610289\\
78.21	0.00552767468365219\\
78.22	0.00553013367760971\\
78.23	0.00553259239260651\\
78.24	0.00553505082325153\\
78.25	0.00553750896413148\\
78.26	0.00553996680981073\\
78.27	0.00554242435483132\\
78.28	0.00554488159371275\\
78.29	0.005547338520952\\
78.3	0.00554979513102334\\
78.31	0.00555225141837837\\
78.32	0.00555470737744578\\
78.33	0.0055571630026314\\
78.34	0.005559618288318\\
78.35	0.00556207322886529\\
78.36	0.00556452781860977\\
78.37	0.00556698205186466\\
78.38	0.00556943592291982\\
78.39	0.00557188942604164\\
78.4	0.00557434255547297\\
78.41	0.00557679530543299\\
78.42	0.00557924767011716\\
78.43	0.00558169964369713\\
78.44	0.00558415122032058\\
78.45	0.00558660239411122\\
78.46	0.00558905315916864\\
78.47	0.0055915035095682\\
78.48	0.00559395343936102\\
78.49	0.00559640294257377\\
78.5	0.00559885201320867\\
78.51	0.00560130064524334\\
78.52	0.00560374883263073\\
78.53	0.00560619656929903\\
78.54	0.00560864384915153\\
78.55	0.00561109066606657\\
78.56	0.00561353701389742\\
78.57	0.0056159828864722\\
78.58	0.00561842827759374\\
78.59	0.00562087318103957\\
78.6	0.00562331759056168\\
78.61	0.00562576149988657\\
78.62	0.00562820490271504\\
78.63	0.00563064779272217\\
78.64	0.00563309016355714\\
78.65	0.00563553200884321\\
78.66	0.00563797332217757\\
78.67	0.00564041409713121\\
78.68	0.00564285432724891\\
78.69	0.00564529400604905\\
78.7	0.00564773312702355\\
78.71	0.00565017168363775\\
78.72	0.00565260966933033\\
78.73	0.00565504707751318\\
78.74	0.00565748390157128\\
78.75	0.00565992013486264\\
78.76	0.00566235577071819\\
78.77	0.00566479080244163\\
78.78	0.00566722522330936\\
78.79	0.00566965902657035\\
78.8	0.00567209220544608\\
78.81	0.00567452475313036\\
78.82	0.0056769566627893\\
78.83	0.00567938792756112\\
78.84	0.00568181854055612\\
78.85	0.00568424849485651\\
78.86	0.00568667778351633\\
78.87	0.00568910639956134\\
78.88	0.00569153433598889\\
78.89	0.00569396158576784\\
78.9	0.00569638814183841\\
78.91	0.0056988139971121\\
78.92	0.00570123914447156\\
78.93	0.00570366357677048\\
78.94	0.00570608728683349\\
78.95	0.00570851026745601\\
78.96	0.00571093251140418\\
78.97	0.00571335401141473\\
78.98	0.00571577476019485\\
78.99	0.00571819475042207\\
79	0.00572061397474418\\
79.01	0.00572303242577908\\
79.02	0.00572545009611468\\
79.03	0.00572786697830876\\
79.04	0.00573028306488888\\
79.05	0.00573269834835227\\
79.06	0.00573511282116563\\
79.07	0.00573752647576513\\
79.08	0.0057399393045562\\
79.09	0.00574235129991345\\
79.1	0.00574476245418053\\
79.11	0.00574717275967002\\
79.12	0.00574958220866331\\
79.13	0.00575199079341046\\
79.14	0.00575439850613009\\
79.15	0.00575680533900927\\
79.16	0.00575921128420337\\
79.17	0.00576161633383594\\
79.18	0.00576402047999861\\
79.19	0.00576642371475091\\
79.2	0.00576882603012022\\
79.21	0.00577122741810159\\
79.22	0.0057736278706576\\
79.23	0.0057760273797183\\
79.24	0.00577842593718101\\
79.25	0.00578082353491023\\
79.26	0.0057832201647375\\
79.27	0.00578561581846126\\
79.28	0.00578801048784677\\
79.29	0.00579040416462589\\
79.3	0.00579279684049704\\
79.31	0.005795188507125\\
79.32	0.00579757915614082\\
79.33	0.00579996877914167\\
79.34	0.0058023573676907\\
79.35	0.00580474491331692\\
79.36	0.00580713140751506\\
79.37	0.00580951684174543\\
79.38	0.0058119012074338\\
79.39	0.00581428449597124\\
79.4	0.005816666698714\\
79.41	0.00581904780698337\\
79.42	0.00582142781206553\\
79.43	0.00582380670521144\\
79.44	0.00582618447763666\\
79.45	0.00582856112052126\\
79.46	0.00583093662500961\\
79.47	0.00583331098221034\\
79.48	0.00583568418319608\\
79.49	0.00583805621900342\\
79.5	0.00584042708063272\\
79.51	0.00584279675904795\\
79.52	0.00584516524517659\\
79.53	0.00584753252990945\\
79.54	0.00584989860410056\\
79.55	0.00585226345856698\\
79.56	0.00585462708408869\\
79.57	0.00585698947140844\\
79.58	0.00585935061123158\\
79.59	0.00586171049422592\\
79.6	0.00586406911102162\\
79.61	0.00586642645221096\\
79.62	0.00586878250834828\\
79.63	0.00587113726994976\\
79.64	0.00587349072749332\\
79.65	0.00587584287141844\\
79.66	0.00587819369212601\\
79.67	0.00588054317997817\\
79.68	0.00588289132529818\\
79.69	0.00588523811837024\\
79.7	0.00588758354943938\\
79.71	0.00588992760871123\\
79.72	0.00589227028635195\\
79.73	0.00589461157248799\\
79.74	0.00589695145720601\\
79.75	0.00589928993055264\\
79.76	0.00590162698253442\\
79.77	0.00590396260311754\\
79.78	0.00590629678222776\\
79.79	0.00590862950975021\\
79.8	0.00591096077552921\\
79.81	0.00591329056936817\\
79.82	0.00591561888102938\\
79.83	0.00591794570023384\\
79.84	0.00592027101666113\\
79.85	0.00592259481994924\\
79.86	0.00592491709969436\\
79.87	0.00592723784545078\\
79.88	0.00592955704673069\\
79.89	0.00593187469300399\\
79.9	0.00593419077369818\\
79.91	0.00593650527819814\\
79.92	0.00593881819584596\\
79.93	0.00594112951594082\\
79.94	0.00594343922773877\\
79.95	0.00594574732045259\\
79.96	0.00594805378325159\\
79.97	0.00595035860526145\\
79.98	0.00595266177556406\\
79.99	0.00595496328319733\\
80	0.00595726311715502\\
80.01	0.00595956126638655\\
};
\addplot [color=red,solid]
  table[row sep=crcr]{%
80.01	0.00595956126638655\\
80.02	0.00596185771979684\\
80.03	0.00596415246624615\\
80.04	0.00596644549454984\\
80.05	0.00596873679347827\\
80.06	0.00597102635175657\\
80.07	0.00597331415806446\\
80.08	0.0059756002010361\\
80.09	0.00597788446925988\\
80.1	0.00598016695127826\\
80.11	0.00598244763558754\\
80.12	0.00598472651063775\\
80.13	0.00598700356483241\\
80.14	0.00598927878652837\\
80.15	0.00599155216403559\\
80.16	0.00599382368561699\\
80.17	0.00599609333948826\\
80.18	0.00599836111381767\\
80.19	0.00600062699672582\\
80.2	0.00600289097628555\\
80.21	0.0060051530405217\\
80.22	0.00600741317741088\\
80.23	0.00600967137488136\\
80.24	0.0060119276208128\\
80.25	0.00601418190303612\\
80.26	0.00601643420933325\\
80.27	0.00601868452743696\\
80.28	0.00602093284503068\\
80.29	0.00602317914974827\\
80.3	0.00602542342917383\\
80.31	0.00602766567084153\\
80.32	0.00602990586223537\\
80.33	0.00603214399078901\\
80.34	0.00603438004388553\\
80.35	0.00603661400885729\\
80.36	0.00603884587298567\\
80.37	0.0060410756235009\\
80.38	0.0060433032475818\\
80.39	0.00604552873235568\\
80.4	0.00604775206489801\\
80.41	0.00604997323223232\\
80.42	0.00605219374020172\\
80.43	0.00605441463694088\\
80.44	0.00605663592089763\\
80.45	0.00605885759051529\\
80.46	0.00606107964423271\\
80.47	0.0060633020804842\\
80.48	0.00606552489769959\\
80.49	0.00606774809430417\\
80.5	0.00606997166871869\\
80.51	0.00607219561935941\\
80.52	0.00607441994463801\\
80.53	0.00607664464296163\\
80.54	0.00607886971273288\\
80.55	0.00608109515234977\\
80.56	0.00608332096020576\\
80.57	0.00608554713468975\\
80.58	0.00608777367418602\\
80.59	0.0060900005770743\\
80.6	0.0060922278417297\\
80.61	0.00609445546652272\\
80.62	0.00609668344981929\\
80.63	0.00609891178998067\\
80.64	0.00610114048536353\\
80.65	0.00610336953431991\\
80.66	0.00610559893519719\\
80.67	0.00610782868633813\\
80.68	0.00611005878608081\\
80.69	0.0061122892327587\\
80.7	0.00611452002470056\\
80.71	0.0061167511602305\\
80.72	0.00611898263766793\\
80.73	0.00612121445532762\\
80.74	0.0061234466115196\\
80.75	0.00612567910454923\\
80.76	0.00612791193271717\\
80.77	0.00613014509431935\\
80.78	0.00613237858764699\\
80.79	0.00613461241098659\\
80.8	0.00613684656261991\\
80.81	0.00613908104082399\\
80.82	0.00614131584387111\\
80.83	0.0061435509700288\\
80.84	0.00614578641755984\\
80.85	0.00614802218472226\\
80.86	0.00615025826976928\\
80.87	0.00615249467094939\\
80.88	0.00615473138650626\\
80.89	0.00615696841467881\\
80.9	0.00615920575370113\\
80.91	0.00616144340180252\\
80.92	0.00616368135720749\\
80.93	0.0061659196181357\\
80.94	0.00616815818280203\\
80.95	0.0061703970494165\\
80.96	0.00617263621618433\\
80.97	0.00617487568130587\\
80.98	0.00617711544297663\\
80.99	0.00617935549938731\\
81	0.00618159584872369\\
81.01	0.00618383648916674\\
81.02	0.00618607741889252\\
81.03	0.00618831863607225\\
81.04	0.00619056013887224\\
81.05	0.00619280192545394\\
81.06	0.00619504399397389\\
81.07	0.00619728634258373\\
81.08	0.00619952896943021\\
81.09	0.00620177187265514\\
81.1	0.00620401505039547\\
81.11	0.00620625850078316\\
81.12	0.00620850222194528\\
81.13	0.00621074621200397\\
81.14	0.00621299046907641\\
81.15	0.00621523499127487\\
81.16	0.00621747977670663\\
81.17	0.00621972482347403\\
81.18	0.00622197012967445\\
81.19	0.00622421569340031\\
81.2	0.00622646151273904\\
81.21	0.00622870758577313\\
81.22	0.00623095391058003\\
81.23	0.00623320048523224\\
81.24	0.00623544730779726\\
81.25	0.00623769437633759\\
81.26	0.00623994168891073\\
81.27	0.00624218924356914\\
81.28	0.00624443703836031\\
81.29	0.00624668507132668\\
81.3	0.00624893334050568\\
81.31	0.00625118184392971\\
81.32	0.00625343057962611\\
81.33	0.00625567954561721\\
81.34	0.00625792873992029\\
81.35	0.00626017816054757\\
81.36	0.0062624278055062\\
81.37	0.00626467767279832\\
81.38	0.00626692776042095\\
81.39	0.00626917806636608\\
81.4	0.00627142858862059\\
81.41	0.00627367932516633\\
81.42	0.00627593027398003\\
81.43	0.00627818143303334\\
81.44	0.00628043280029281\\
81.45	0.00628268437371992\\
81.46	0.00628493615127103\\
81.47	0.00628718813089739\\
81.48	0.00628944031054514\\
81.49	0.00629169268815532\\
81.5	0.00629394526166386\\
81.51	0.00629619802900152\\
81.52	0.00629845098809399\\
81.53	0.00630070413686179\\
81.54	0.00630295747322033\\
81.55	0.00630521099507986\\
81.56	0.00630746470034551\\
81.57	0.00630971858691725\\
81.58	0.00631197265268989\\
81.59	0.00631422689555311\\
81.6	0.00631648131339142\\
81.61	0.00631873590408416\\
81.62	0.00632099066550551\\
81.63	0.0063232455955245\\
81.64	0.00632550069200495\\
81.65	0.00632775595280556\\
81.66	0.00633001137577979\\
81.67	0.00633226695877595\\
81.68	0.00633452269963717\\
81.69	0.00633677859620137\\
81.7	0.00633903464630131\\
81.71	0.00634129084776452\\
81.72	0.00634354719841334\\
81.73	0.00634580369606493\\
81.74	0.00634806033853124\\
81.75	0.00635031712361899\\
81.76	0.00635257404912971\\
81.77	0.00635483111285973\\
81.78	0.00635708831260013\\
81.79	0.00635934564613682\\
81.8	0.00636160311125044\\
81.81	0.00636386070571645\\
81.82	0.00636611842730506\\
81.83	0.00636837627378126\\
81.84	0.00637063424290482\\
81.85	0.00637289233243028\\
81.86	0.00637515054010692\\
81.87	0.00637740886367882\\
81.88	0.00637966730088481\\
81.89	0.00638192584945847\\
81.9	0.00638418450712815\\
81.91	0.00638644327161695\\
81.92	0.00638870214064276\\
81.93	0.00639096111191818\\
81.94	0.00639322018315057\\
81.95	0.00639547935204208\\
81.96	0.00639773861628957\\
81.97	0.00639999797358467\\
81.98	0.00640225742161374\\
81.99	0.00640451695805792\\
82	0.00640677658059306\\
82.01	0.00640903628688978\\
82.02	0.00641129607461344\\
82.03	0.00641355594142414\\
82.04	0.00641581588497672\\
82.05	0.00641807590292076\\
82.06	0.00642033599290062\\
82.07	0.00642259615255533\\
82.08	0.00642485637951875\\
82.09	0.0064271166714194\\
82.1	0.0064293770258806\\
82.11	0.00643163744052039\\
82.12	0.00643389791295155\\
82.13	0.00643615844078159\\
82.14	0.0064384190216128\\
82.15	0.00644067965304218\\
82.16	0.00644294033266148\\
82.17	0.0064452010580572\\
82.18	0.00644746182681059\\
82.19	0.00644972263649764\\
82.2	0.00645198348468908\\
82.21	0.0064542443689504\\
82.22	0.00645650528684182\\
82.23	0.00645876623591834\\
82.24	0.00646102721372968\\
82.25	0.00646328821782034\\
82.26	0.00646554924572955\\
82.27	0.00646781029499131\\
82.28	0.00647007136313439\\
82.29	0.00647233244768229\\
82.3	0.0064745935461533\\
82.31	0.00647685465606046\\
82.32	0.00647911577491158\\
82.33	0.00648137690020924\\
82.34	0.00648363802945079\\
82.35	0.00648589916012836\\
82.36	0.00648816028972883\\
82.37	0.00649042141569527\\
82.38	0.00649268253546475\\
82.39	0.00649494364646921\\
82.4	0.00649720474613542\\
82.41	0.006499465831885\\
82.42	0.00650172690113441\\
82.43	0.00650398795129496\\
82.44	0.0065062489797728\\
82.45	0.00650850998396896\\
82.46	0.00651077096127931\\
82.47	0.0065130319090946\\
82.48	0.00651529282480043\\
82.49	0.00651755370577729\\
82.5	0.00651981454940051\\
82.51	0.00652207535304036\\
82.52	0.00652433611406193\\
82.53	0.00652659682982524\\
82.54	0.00652885749768518\\
82.55	0.00653111811499155\\
82.56	0.00653337867908904\\
82.57	0.00653563918731728\\
82.58	0.00653789963701076\\
82.59	0.00654016002549893\\
82.6	0.00654242035010613\\
82.61	0.00654468060815167\\
82.62	0.00654694079694975\\
82.63	0.00654920091380952\\
82.64	0.00655146095603509\\
82.65	0.0065537209209255\\
82.66	0.00655598080577478\\
82.67	0.00655824060787188\\
82.68	0.00656050032450073\\
82.69	0.00656275995294025\\
82.7	0.00656501949046433\\
82.71	0.00656727893434186\\
82.72	0.0065695382818367\\
82.73	0.00657179753020773\\
82.74	0.00657405667670883\\
82.75	0.0065763157185889\\
82.76	0.00657857465309186\\
82.77	0.00658083347745666\\
82.78	0.00658309218891729\\
82.79	0.00658535078470278\\
82.8	0.00658760926203721\\
82.81	0.00658986761813974\\
82.82	0.00659212585022456\\
82.83	0.00659438395550098\\
82.84	0.00659664193117337\\
82.85	0.00659889977444119\\
82.86	0.00660115748249902\\
82.87	0.00660341505253654\\
82.88	0.00660567248173855\\
82.89	0.00660792976728499\\
82.9	0.00661018690635091\\
82.91	0.00661244389610653\\
82.92	0.00661470073371723\\
82.93	0.00661695741634354\\
82.94	0.00661921394114118\\
82.95	0.00662147030526105\\
82.96	0.00662372650584924\\
82.97	0.00662598254004706\\
82.98	0.00662823840499103\\
82.99	0.00663049409781289\\
83	0.00663274961563963\\
83.01	0.00663500495559349\\
83.02	0.00663726011479195\\
83.03	0.00663951509034779\\
83.04	0.00664176987936906\\
83.05	0.00664402447895909\\
83.06	0.00664627888621653\\
83.07	0.00664853309823536\\
83.08	0.00665078711210485\\
83.09	0.00665304092490967\\
83.1	0.00665529453372979\\
83.11	0.00665754793564058\\
83.12	0.00665980112771278\\
83.13	0.00666205410701251\\
83.14	0.00666430687060132\\
83.15	0.00666655941553616\\
83.16	0.00666881173886942\\
83.17	0.00667106383764894\\
83.18	0.00667331570891801\\
83.19	0.0066755673497154\\
83.2	0.00667781875707537\\
83.21	0.00668006992802767\\
83.22	0.00668232085959758\\
83.23	0.00668457154880591\\
83.24	0.00668682199266901\\
83.25	0.00668907218819879\\
83.26	0.00669132213240277\\
83.27	0.006693571822284\\
83.28	0.0066958212548412\\
83.29	0.00669807042706867\\
83.3	0.00670031933595637\\
83.31	0.00670256797848992\\
83.32	0.00670481635165062\\
83.33	0.00670706445241542\\
83.34	0.00670931227775702\\
83.35	0.00671155982464382\\
83.36	0.00671380709003998\\
83.37	0.0067160540709054\\
83.38	0.00671830076419575\\
83.39	0.00672054716686254\\
83.4	0.00672279327585303\\
83.41	0.00672503908811035\\
83.42	0.00672728460057348\\
83.43	0.00672952981017725\\
83.44	0.00673177471385239\\
83.45	0.00673401930852554\\
83.46	0.00673626359111925\\
83.47	0.00673850755855202\\
83.48	0.00674075120773834\\
83.49	0.00674299453558866\\
83.5	0.00674523753900943\\
83.51	0.00674748021490316\\
83.52	0.00674972256016838\\
83.53	0.0067519645716997\\
83.54	0.00675420624638782\\
83.55	0.00675644758111955\\
83.56	0.00675868857277783\\
83.57	0.00676092921824177\\
83.58	0.00676316951438665\\
83.59	0.00676540945808395\\
83.6	0.00676764904620136\\
83.61	0.00676988827560284\\
83.62	0.00677212714314862\\
83.63	0.0067743656456952\\
83.64	0.00677660378009543\\
83.65	0.00677884154319847\\
83.66	0.00678107893184985\\
83.67	0.00678331594289154\\
83.68	0.00678555257316184\\
83.69	0.00678778881949558\\
83.7	0.00679002467872398\\
83.71	0.0067922601476748\\
83.72	0.00679449522317232\\
83.73	0.00679672990203731\\
83.74	0.00679896418108718\\
83.75	0.0068011980571359\\
83.76	0.00680343152699406\\
83.77	0.00680566458746892\\
83.78	0.00680789723536441\\
83.79	0.00681012946748117\\
83.8	0.00681236128061658\\
83.81	0.00681459267156477\\
83.82	0.00681682363711668\\
83.83	0.00681905417406006\\
83.84	0.00682128427917951\\
83.85	0.00682351394925652\\
83.86	0.00682574318106951\\
83.87	0.00682797197139379\\
83.88	0.00683020031700169\\
83.89	0.00683242821466253\\
83.9	0.00683465566114265\\
83.91	0.00683688265320546\\
83.92	0.00683910918761149\\
83.93	0.00684133526111837\\
83.94	0.00684356087048093\\
83.95	0.00684578601245114\\
83.96	0.00684801068377825\\
83.97	0.00685023488120875\\
83.98	0.00685245860148642\\
83.99	0.0068546818413524\\
84	0.00685690459754515\\
84.01	0.00685912686680056\\
84.02	0.00686134864585195\\
84.03	0.00686356993143011\\
84.04	0.00686579072026334\\
84.05	0.00686801100907745\\
84.06	0.00687023079459588\\
84.07	0.00687245007353965\\
84.08	0.00687466884262744\\
84.09	0.00687688709857564\\
84.1	0.00687910483809833\\
84.11	0.00688132205790739\\
84.12	0.00688353889481202\\
84.13	0.00688575550449723\\
84.14	0.00688797188330864\\
84.15	0.00689018802758196\\
84.16	0.00689240393364306\\
84.17	0.00689461959780786\\
84.18	0.0068968350163824\\
84.19	0.00689905018566276\\
84.2	0.00690126510193508\\
84.21	0.00690347976147553\\
84.22	0.00690569416055031\\
84.23	0.00690790829541559\\
84.24	0.00691012216231755\\
84.25	0.00691233575749234\\
84.26	0.00691454907716604\\
84.27	0.0069167621175547\\
84.28	0.00691897487486426\\
84.29	0.00692118734529059\\
84.3	0.00692339952501943\\
84.31	0.0069256114102264\\
84.32	0.00692782299707699\\
84.33	0.00693003428172653\\
84.34	0.00693224526032016\\
84.35	0.00693445592899286\\
84.36	0.00693666628386938\\
84.37	0.00693887632106428\\
84.38	0.00694108603668185\\
84.39	0.00694329542681618\\
84.4	0.00694550448755105\\
84.41	0.00694771321495999\\
84.42	0.00694992160510623\\
84.43	0.0069521296540427\\
84.44	0.006954337357812\\
84.45	0.00695654471244639\\
84.46	0.00695875171396777\\
84.47	0.00696095835838769\\
84.48	0.00696316464170733\\
84.49	0.00696537055991746\\
84.5	0.00696757610899843\\
84.51	0.00696978128492019\\
84.52	0.00697198608364223\\
84.53	0.00697419050111361\\
84.54	0.00697639453327292\\
84.55	0.00697859817604827\\
84.56	0.00698080142535727\\
84.57	0.00698300427710703\\
84.58	0.00698520672719415\\
84.59	0.00698740877150468\\
84.6	0.00698961040591415\\
84.61	0.0069918116262875\\
84.62	0.00699401242847912\\
84.63	0.00699621280833281\\
84.64	0.00699841276168178\\
84.65	0.00700061228434861\\
84.66	0.00700281137214527\\
84.67	0.00700501002087311\\
84.68	0.0070072082263228\\
84.69	0.00700940598427437\\
84.7	0.00701160329049719\\
84.71	0.00701380014074991\\
84.72	0.00701599653078053\\
84.73	0.0070181924563263\\
84.74	0.00702038791311378\\
84.75	0.00702258289685878\\
84.76	0.0070247774032664\\
84.77	0.00702697142803094\\
84.78	0.00702916496683599\\
84.79	0.00703135801535432\\
84.8	0.00703355056924793\\
84.81	0.00703574262416804\\
84.82	0.00703793417575504\\
84.83	0.00704012521963852\\
84.84	0.00704231575143723\\
84.85	0.0070445057667591\\
84.86	0.00704669526120121\\
84.87	0.00704888423034975\\
84.88	0.0070510726697801\\
84.89	0.00705326057505672\\
84.9	0.00705544794173322\\
84.91	0.00705763476535228\\
84.92	0.00705982104020222\\
84.93	0.00706200676037457\\
84.94	0.00706419191993808\\
84.95	0.00706637651293863\\
84.96	0.00706856053339917\\
84.97	0.00707074397531958\\
84.98	0.00707292683267664\\
84.99	0.00707510909942391\\
85	0.00707729076949167\\
85.01	0.00707947183678679\\
85.02	0.00708165229519269\\
85.03	0.00708383213856922\\
85.04	0.00708601136075258\\
85.05	0.00708818995555525\\
85.06	0.00709036791676585\\
85.07	0.00709254523814913\\
85.08	0.0070947219134458\\
85.09	0.00709689793637249\\
85.1	0.00709907330062166\\
85.11	0.00710124799986148\\
85.12	0.00710342202773574\\
85.13	0.0071055953778638\\
85.14	0.00710776804384045\\
85.15	0.00710994001923587\\
85.16	0.00711211129759548\\
85.17	0.00711428187243987\\
85.18	0.00711645173726472\\
85.19	0.00711862088554072\\
85.2	0.00712078931071343\\
85.21	0.00712295700620322\\
85.22	0.00712512396540515\\
85.23	0.0071272901816889\\
85.24	0.00712945564839868\\
85.25	0.00713162035885311\\
85.26	0.00713378430634512\\
85.27	0.0071359474841419\\
85.28	0.00713810988548474\\
85.29	0.00714027150358898\\
85.3	0.00714243233164389\\
85.31	0.0071445923628126\\
85.32	0.00714675159023195\\
85.33	0.00714891000701246\\
85.34	0.00715106760623816\\
85.35	0.00715322438096654\\
85.36	0.00715538032422843\\
85.37	0.00715753542902792\\
85.38	0.00715968968834223\\
85.39	0.00716184309512164\\
85.4	0.00716399564228935\\
85.41	0.00716614732274143\\
85.42	0.00716829812934667\\
85.43	0.0071704480549465\\
85.44	0.0071725970923549\\
85.45	0.00717474523435827\\
85.46	0.00717689247371535\\
85.47	0.00717903880315708\\
85.48	0.00718118421538656\\
85.49	0.00718332870307887\\
85.5	0.00718547225888103\\
85.51	0.00718761487541185\\
85.52	0.00718975654526185\\
85.53	0.00719189726099315\\
85.54	0.00719403701513933\\
85.55	0.0071961758002054\\
85.56	0.00719831360866759\\
85.57	0.00720045043297334\\
85.58	0.00720258626554113\\
85.59	0.00720472109876039\\
85.6	0.00720685492499141\\
85.61	0.00720898773656517\\
85.62	0.00721111952578332\\
85.63	0.007213250284918\\
85.64	0.00721538000621175\\
85.65	0.0072175086818774\\
85.66	0.00721963630409798\\
85.67	0.00722176286502657\\
85.68	0.0072238883567862\\
85.69	0.00722601277146977\\
85.7	0.00722813610113989\\
85.71	0.00723025833782879\\
85.72	0.00723237947353821\\
85.73	0.00723449950023927\\
85.74	0.00723661840987238\\
85.75	0.0072387361943471\\
85.76	0.00724085284554202\\
85.77	0.0072429683553047\\
85.78	0.00724508271545148\\
85.79	0.0072471959177674\\
85.8	0.00724930795400608\\
85.81	0.00725141881588961\\
85.82	0.00725352849510841\\
85.83	0.00725563698332114\\
85.84	0.00725774427215456\\
85.85	0.00725985035320341\\
85.86	0.00726195521803031\\
85.87	0.00726405885816562\\
85.88	0.00726616126510733\\
85.89	0.00726826243032092\\
85.9	0.00727036234523928\\
85.91	0.00727246100126256\\
85.92	0.00727455838975803\\
85.93	0.00727665450205999\\
85.94	0.00727874932946964\\
85.95	0.00728084286325494\\
85.96	0.00728293509465051\\
85.97	0.00728502601485748\\
85.98	0.00728711561504338\\
85.99	0.00728920388634204\\
86	0.00729129081985339\\
86.01	0.00729337640664342\\
86.02	0.00729546063774399\\
86.03	0.00729754350415274\\
86.04	0.00729962499683294\\
86.05	0.00730170510671339\\
86.06	0.00730378382468825\\
86.07	0.00730586114161695\\
86.08	0.00730793704832404\\
86.09	0.00731001153559907\\
86.1	0.00731208459419644\\
86.11	0.0073141562148353\\
86.12	0.00731622638819939\\
86.13	0.00731829510493696\\
86.14	0.00732036235566054\\
86.15	0.00732242813094691\\
86.16	0.00732449242133691\\
86.17	0.00732655521733532\\
86.18	0.00732861650941074\\
86.19	0.00733067628799542\\
86.2	0.00733273454348518\\
86.21	0.00733479126623921\\
86.22	0.00733684644657998\\
86.23	0.00733890007479311\\
86.24	0.00734095214112718\\
86.25	0.00734300263579366\\
86.26	0.00734505154896671\\
86.27	0.00734709887078311\\
86.28	0.00734914459134205\\
86.29	0.00735118870070503\\
86.3	0.00735323118889573\\
86.31	0.00735527204589984\\
86.32	0.00735731126166496\\
86.33	0.00735934882610041\\
86.34	0.00736138472907712\\
86.35	0.00736341896042749\\
86.36	0.00736545150994524\\
86.37	0.00736748236738524\\
86.38	0.00736951152246343\\
86.39	0.00737153896485663\\
86.4	0.0073735646842024\\
86.41	0.00737558867009889\\
86.42	0.00737761091210475\\
86.43	0.00737963139973889\\
86.44	0.00738165012248043\\
86.45	0.00738366706976849\\
86.46	0.00738568223100204\\
86.47	0.00738769559553983\\
86.48	0.00738970715270015\\
86.49	0.00739171689176072\\
86.5	0.00739372480195855\\
86.51	0.00739573087248978\\
86.52	0.00739773509250955\\
86.53	0.0073997374511318\\
86.54	0.00740173793742918\\
86.55	0.00740373654043286\\
86.56	0.00740573324913238\\
86.57	0.00740772805247553\\
86.58	0.00740972093936816\\
86.59	0.00741171189867405\\
86.6	0.00741370091921474\\
86.61	0.00741568798976939\\
86.62	0.00741767309907463\\
86.63	0.00741965623582438\\
86.64	0.00742163738866974\\
86.65	0.00742361654621878\\
86.66	0.00742559369703641\\
86.67	0.00742756882964423\\
86.68	0.00742954193252038\\
86.69	0.00743151299409935\\
86.7	0.00743348200277185\\
86.71	0.00743544894688464\\
86.72	0.00743741381474038\\
86.73	0.00743937659459745\\
86.74	0.00744133727466982\\
86.75	0.00744329584312687\\
86.76	0.00744525228809324\\
86.77	0.00744720659764865\\
86.78	0.00744915875982776\\
86.79	0.00745110876262001\\
86.8	0.00745305659396942\\
86.81	0.00745500224177448\\
86.82	0.00745694728972756\\
86.83	0.00745889242469601\\
86.84	0.0074608376463522\\
86.85	0.00746278295437578\\
86.86	0.00746472834845371\\
86.87	0.00746667382828035\\
86.88	0.0074686193935575\\
86.89	0.00747056504399447\\
86.9	0.00747251077930815\\
86.91	0.00747445659922308\\
86.92	0.00747640250347148\\
86.93	0.00747834849179335\\
86.94	0.00748029456393652\\
86.95	0.00748224071965672\\
86.96	0.00748418695871764\\
86.97	0.00748613328089099\\
86.98	0.00748807968595658\\
86.99	0.00749002617370239\\
87	0.00749197274392462\\
87.01	0.00749391939642775\\
87.02	0.00749586613102465\\
87.03	0.00749781294753659\\
87.04	0.00749975984579338\\
87.05	0.00750170682563335\\
87.06	0.00750365388690351\\
87.07	0.00750560102945955\\
87.08	0.00750754825316596\\
87.09	0.00750949555789604\\
87.1	0.00751144294353205\\
87.11	0.00751339040996523\\
87.12	0.00751533795709586\\
87.13	0.00751728558483338\\
87.14	0.00751923329309643\\
87.15	0.00752118108181292\\
87.16	0.00752312895092013\\
87.17	0.00752507690036475\\
87.18	0.00752702493010299\\
87.19	0.00752897304010061\\
87.2	0.00753092123033306\\
87.21	0.00753286950078548\\
87.22	0.00753481785145283\\
87.23	0.00753676628233996\\
87.24	0.00753871479346166\\
87.25	0.00754066338484275\\
87.26	0.00754261205651819\\
87.27	0.0075445608085331\\
87.28	0.00754650964094289\\
87.29	0.0075484585538133\\
87.3	0.00755040754722052\\
87.31	0.00755235662125122\\
87.32	0.00755430577600269\\
87.33	0.00755625501158287\\
87.34	0.00755820432811045\\
87.35	0.00756015372571497\\
87.36	0.00756210320453685\\
87.37	0.00756405276472755\\
87.38	0.00756600240644959\\
87.39	0.00756795212987666\\
87.4	0.0075699019351937\\
87.41	0.00757185182259698\\
87.42	0.0075738017922942\\
87.43	0.00757575184450454\\
87.44	0.00757770197945881\\
87.45	0.00757965219739947\\
87.46	0.00758160249858077\\
87.47	0.00758355288326878\\
87.48	0.00758550335174154\\
87.49	0.00758745390428912\\
87.5	0.00758940454121369\\
87.51	0.00759135526282964\\
87.52	0.00759330606946367\\
87.53	0.00759525696145485\\
87.54	0.00759720793915477\\
87.55	0.00759915900292753\\
87.56	0.00760111015314996\\
87.57	0.00760306139021162\\
87.58	0.00760501271451491\\
87.59	0.00760696412647521\\
87.6	0.0076089156265209\\
87.61	0.00761086721509354\\
87.62	0.00761281889264787\\
87.63	0.00761477065965201\\
87.64	0.00761672251658746\\
87.65	0.00761867446394927\\
87.66	0.00762062650224609\\
87.67	0.0076225786320003\\
87.68	0.00762453085374808\\
87.69	0.00762648316803953\\
87.7	0.00762843557543878\\
87.71	0.00763038807652405\\
87.72	0.0076323406718878\\
87.73	0.00763429336213677\\
87.74	0.00763624614789217\\
87.75	0.00763819902978969\\
87.76	0.00764015200847967\\
87.77	0.00764210508462716\\
87.78	0.00764405825891205\\
87.79	0.00764601153202919\\
87.8	0.00764796490468844\\
87.81	0.00764991837761483\\
87.82	0.00765187195154864\\
87.83	0.00765382562724552\\
87.84	0.00765577940547659\\
87.85	0.00765773328702853\\
87.86	0.00765968727270376\\
87.87	0.00766164136332044\\
87.88	0.00766359555971268\\
87.89	0.00766554986273059\\
87.9	0.00766750427324044\\
87.91	0.0076694587921247\\
87.92	0.00767141342028224\\
87.93	0.00767336815862837\\
87.94	0.007675323008095\\
87.95	0.00767727796963074\\
87.96	0.007679233044201\\
87.97	0.00768118823278815\\
87.98	0.00768314353639157\\
87.99	0.00768509895602783\\
88	0.00768705449273078\\
88.01	0.00768901014755165\\
88.02	0.00769096592155922\\
88.03	0.00769292181583989\\
88.04	0.00769487783149782\\
88.05	0.00769683396965505\\
88.06	0.00769879023145164\\
88.07	0.00770074661804575\\
88.08	0.00770270313061381\\
88.09	0.00770465977035062\\
88.1	0.00770661653846946\\
88.11	0.00770857343620224\\
88.12	0.00771053046479962\\
88.13	0.00771248762553114\\
88.14	0.00771444491968533\\
88.15	0.00771640234856985\\
88.16	0.00771835991351163\\
88.17	0.00772031761585696\\
88.18	0.00772227545697169\\
88.19	0.00772423343824126\\
88.2	0.00772619156107095\\
88.21	0.00772814982688591\\
88.22	0.00773010823713133\\
88.23	0.00773206679327262\\
88.24	0.00773402549679545\\
88.25	0.00773598434920597\\
88.26	0.00773794335203091\\
88.27	0.00773990250681769\\
88.28	0.00774186181513462\\
88.29	0.00774382127857099\\
88.3	0.00774578089873722\\
88.31	0.00774774067726499\\
88.32	0.00774970061580742\\
88.33	0.00775166071603915\\
88.34	0.00775362097965655\\
88.35	0.0077555814083778\\
88.36	0.00775754200394306\\
88.37	0.00775950276811462\\
88.38	0.00776146370267704\\
88.39	0.0077634248094373\\
88.4	0.00776538609022492\\
88.41	0.00776734754689214\\
88.42	0.00776930918131405\\
88.43	0.00777127099538876\\
88.44	0.00777323299103749\\
88.45	0.00777519517020482\\
88.46	0.00777715753485873\\
88.47	0.00777912008699083\\
88.48	0.00778108282861651\\
88.49	0.00778304576177502\\
88.5	0.00778500888852972\\
88.51	0.00778697221096816\\
88.52	0.00778893573120229\\
88.53	0.00779089945136859\\
88.54	0.00779286337362819\\
88.55	0.00779482750016713\\
88.56	0.00779679183319641\\
88.57	0.00779875637495222\\
88.58	0.00780072112769608\\
88.59	0.00780268609371497\\
88.6	0.00780465127532158\\
88.61	0.00780661667485435\\
88.62	0.00780858229467775\\
88.63	0.00781054813718236\\
88.64	0.0078125142047851\\
88.65	0.00781448049992934\\
88.66	0.00781644702508511\\
88.67	0.00781841378274927\\
88.68	0.00782038077544561\\
88.69	0.00782234800572514\\
88.7	0.00782431547616616\\
88.71	0.00782628318937444\\
88.72	0.00782825114798348\\
88.73	0.00783021935465458\\
88.74	0.00783218781207706\\
88.75	0.00783415652296846\\
88.76	0.00783612549007466\\
88.77	0.00783809471617011\\
88.78	0.00784006420405797\\
88.79	0.00784203395657032\\
88.8	0.00784400397656831\\
88.81	0.00784597426694237\\
88.82	0.00784794483061236\\
88.83	0.0078499156705278\\
88.84	0.00785188678966801\\
88.85	0.0078538581910423\\
88.86	0.00785582987769019\\
88.87	0.00785780185268154\\
88.88	0.00785977411911682\\
88.89	0.0078617466801272\\
88.9	0.00786371953887482\\
88.91	0.00786569269855294\\
88.92	0.00786766616238615\\
88.93	0.00786963993363053\\
88.94	0.00787161401557389\\
88.95	0.00787358841153594\\
88.96	0.00787556312486849\\
88.97	0.00787753815895564\\
88.98	0.00787951351721397\\
88.99	0.00788148920309276\\
89	0.0078834652200742\\
89.01	0.00788544157167353\\
89.02	0.00788741826143932\\
89.03	0.00788939529295362\\
89.04	0.00789137266983216\\
89.05	0.00789335039572461\\
89.06	0.00789532847431472\\
89.07	0.00789730690932056\\
89.08	0.00789928570449472\\
89.09	0.00790126486362455\\
89.1	0.0079032443905323\\
89.11	0.00790522428907539\\
89.12	0.00790720456314661\\
89.13	0.00790918521667433\\
89.14	0.00791116625362269\\
89.15	0.00791314767799186\\
89.16	0.00791512949381821\\
89.17	0.00791711170517456\\
89.18	0.0079190943161704\\
89.19	0.00792107732930387\\
89.2	0.00792306074420649\\
89.21	0.00792504456050932\\
89.22	0.00792702877784298\\
89.23	0.00792901339583763\\
89.24	0.00793099841412301\\
89.25	0.00793298383232837\\
89.26	0.00793496965008253\\
89.27	0.00793695586701381\\
89.28	0.00793894248275008\\
89.29	0.00794092949691874\\
89.3	0.00794291690914668\\
89.31	0.00794490471906033\\
89.32	0.00794689292628563\\
89.33	0.00794888153044799\\
89.34	0.00795087053117234\\
89.35	0.00795285992808312\\
89.36	0.00795484972080421\\
89.37	0.007956839908959\\
89.38	0.00795883049217036\\
89.39	0.00796082147006061\\
89.4	0.00796281284225154\\
89.41	0.00796480460836439\\
89.42	0.00796679676801986\\
89.43	0.0079687893208381\\
89.44	0.00797078226643868\\
89.45	0.00797277560444062\\
89.46	0.00797476933446234\\
89.47	0.00797676345612171\\
89.48	0.00797875796903599\\
89.49	0.00798075287282186\\
89.5	0.00798274816709538\\
89.51	0.00798474385147203\\
89.52	0.00798673992556665\\
89.53	0.00798873638899347\\
89.54	0.00799073324136609\\
89.55	0.00799273048229747\\
89.56	0.00799472811139993\\
89.57	0.00799672612828515\\
89.58	0.00799872453256412\\
89.59	0.00800072332384721\\
89.6	0.00800272250174408\\
89.61	0.00800472206586372\\
89.62	0.00800672201581444\\
89.63	0.00800872235120384\\
89.64	0.00801072307163883\\
89.65	0.0080127241767256\\
89.66	0.00801472566606961\\
89.67	0.00801672753927561\\
89.68	0.00801872979594759\\
89.69	0.00802073243568883\\
89.7	0.00802273545810181\\
89.71	0.00802473886278827\\
89.72	0.00802674264934919\\
89.73	0.00802874681738475\\
89.74	0.00803075136649435\\
89.75	0.00803275629627658\\
89.76	0.00803476160632923\\
89.77	0.00803676729624929\\
89.78	0.00803877336563289\\
89.79	0.00804077981407533\\
89.8	0.0080427866411711\\
89.81	0.00804479384651378\\
89.82	0.00804680142969614\\
89.83	0.00804880939031002\\
89.84	0.00805081772794641\\
89.85	0.00805282644219541\\
89.86	0.00805483553264618\\
89.87	0.00805684499888699\\
89.88	0.00805885484050518\\
89.89	0.00806086505708714\\
89.9	0.00806287564821834\\
89.91	0.00806488661348325\\
89.92	0.0080668979524654\\
89.93	0.00806890966474732\\
89.94	0.00807092174991057\\
89.95	0.00807293420753568\\
89.96	0.00807494703720219\\
89.97	0.00807696023848859\\
89.98	0.00807897381097235\\
89.99	0.00808098775422988\\
90	0.00808300206783652\\
90.01	0.00808501675136655\\
90.02	0.00808703180439315\\
90.03	0.00808904722648841\\
90.04	0.00809106301722331\\
90.05	0.00809307917616769\\
90.06	0.00809509570289028\\
90.07	0.00809711259695862\\
90.08	0.00809912985793912\\
90.09	0.008101147485397\\
90.1	0.0081031654788963\\
90.11	0.00810518383799984\\
90.12	0.00810720256226923\\
90.13	0.00810922165126486\\
90.14	0.00811124110454586\\
90.15	0.00811326092167012\\
90.16	0.00811528110219422\\
90.17	0.00811730164567349\\
90.18	0.00811932255166195\\
90.19	0.0081213438197123\\
90.2	0.0081233654493759\\
90.21	0.00812538744020277\\
90.22	0.00812740979174158\\
90.23	0.0081294325035396\\
90.24	0.00813145557514274\\
90.25	0.00813347900609547\\
90.26	0.00813550279594086\\
90.27	0.00813752694422053\\
90.28	0.00813955145047464\\
90.29	0.00814157631424189\\
90.3	0.00814360153505949\\
90.31	0.00814562711246315\\
90.32	0.00814765304598703\\
90.33	0.00814967933516381\\
90.34	0.00815170597952455\\
90.35	0.00815373297859879\\
90.36	0.00815576033191444\\
90.37	0.00815778803899784\\
90.38	0.00815981609937368\\
90.39	0.00816184451256501\\
90.4	0.00816387327809323\\
90.41	0.00816590239547807\\
90.42	0.00816793186423753\\
90.43	0.00816996168388792\\
90.44	0.00817199185394381\\
90.45	0.00817402237391801\\
90.46	0.00817605324332157\\
90.47	0.00817808446166374\\
90.48	0.00818011602845195\\
90.49	0.00818214794319182\\
90.5	0.00818418020538711\\
90.51	0.00818621281453969\\
90.52	0.00818824577014956\\
90.53	0.00819027907171482\\
90.54	0.00819231271873159\\
90.55	0.00819434671069409\\
90.56	0.00819638104709454\\
90.57	0.00819841572742316\\
90.58	0.00820045075116816\\
90.59	0.00820248611781571\\
90.6	0.00820452182684992\\
90.61	0.00820655787775283\\
90.62	0.00820859427000433\\
90.63	0.00821063100308224\\
90.64	0.0082126680764622\\
90.65	0.00821470548961767\\
90.66	0.00821674324201993\\
90.67	0.00821878133313804\\
90.68	0.0082208197624388\\
90.69	0.00822285852938675\\
90.7	0.00822489763344416\\
90.71	0.00822693707407095\\
90.72	0.00822897685072472\\
90.73	0.00823101696286071\\
90.74	0.00823305740993175\\
90.75	0.00823509819138827\\
90.76	0.00823713930667828\\
90.77	0.00823918075505629\\
90.78	0.00824122253549515\\
90.79	0.00824326464696576\\
90.8	0.00824530708843713\\
90.81	0.0082473498588763\\
90.82	0.00824939295724842\\
90.83	0.00825143638251671\\
90.84	0.00825348013364246\\
90.85	0.00825552420958505\\
90.86	0.00825756860930192\\
90.87	0.00825961333174863\\
90.88	0.00826165837587878\\
90.89	0.00826370374064407\\
90.9	0.00826574942499429\\
90.91	0.00826779542787729\\
90.92	0.00826984174823902\\
90.93	0.00827188838502354\\
90.94	0.00827393533717295\\
90.95	0.00827598260362745\\
90.96	0.00827803018332536\\
90.97	0.00828007807520305\\
90.98	0.00828212627819499\\
90.99	0.00828417479123376\\
91	0.00828622361325002\\
91.01	0.0082882727431725\\
91.02	0.00829032217992807\\
91.03	0.00829237192244166\\
91.04	0.00829442196963631\\
91.05	0.00829647232043315\\
91.06	0.00829852297375143\\
91.07	0.00830057392850846\\
91.08	0.00830262518361968\\
91.09	0.00830467673799865\\
91.1	0.00830672859055698\\
91.11	0.00830878074020443\\
91.12	0.00831083318584886\\
91.13	0.00831288592639621\\
91.14	0.00831493896075057\\
91.15	0.0083169922878141\\
91.16	0.0083190459064871\\
91.17	0.00832109981566797\\
91.18	0.00832315401425323\\
91.19	0.00832520850113752\\
91.2	0.00832726327521359\\
91.21	0.00832931833537231\\
91.22	0.00833137368050266\\
91.23	0.00833342930949177\\
91.24	0.00833548522122487\\
91.25	0.00833754141458534\\
91.26	0.00833959788845465\\
91.27	0.00834165464171244\\
91.28	0.00834371167323644\\
91.29	0.00834576898190255\\
91.3	0.00834782656658478\\
91.31	0.00834988442615529\\
91.32	0.00835194255948437\\
91.33	0.00835400096544046\\
91.34	0.00835605964289011\\
91.35	0.00835811859069806\\
91.36	0.00836017780772716\\
91.37	0.00836223729283843\\
91.38	0.00836429704489101\\
91.39	0.00836635706274224\\
91.4	0.00836841734524756\\
91.41	0.0083704778912606\\
91.42	0.00837253869963315\\
91.43	0.00837459976921512\\
91.44	0.00837666109885464\\
91.45	0.00837872268739795\\
91.46	0.0083807845336895\\
91.47	0.00838284663657187\\
91.48	0.00838490899488586\\
91.49	0.00838697160747039\\
91.5	0.00838903447316258\\
91.51	0.00839109759079775\\
91.52	0.00839316095920935\\
91.53	0.00839522457722906\\
91.54	0.00839728844368673\\
91.55	0.00839935255741039\\
91.56	0.00840141691722626\\
91.57	0.00840348152195876\\
91.58	0.00840554637043051\\
91.59	0.00840761146146232\\
91.6	0.00840967679387321\\
91.61	0.00841174236648039\\
91.62	0.00841380817809928\\
91.63	0.00841587422754352\\
91.64	0.00841794051362496\\
91.65	0.00842000703515365\\
91.66	0.00842207379093787\\
91.67	0.00842414077978411\\
91.68	0.0084262080004971\\
91.69	0.00842827545187979\\
91.7	0.00843034313273334\\
91.71	0.00843241104185718\\
91.72	0.00843447917804894\\
91.73	0.0084365475401045\\
91.74	0.00843861612681799\\
91.75	0.00844068493698178\\
91.76	0.00844275396938647\\
91.77	0.00844482322282095\\
91.78	0.00844689269607233\\
91.79	0.00844896238792599\\
91.8	0.00845103229716557\\
91.81	0.00845310242257297\\
91.82	0.00845517276292836\\
91.83	0.0084572433170102\\
91.84	0.00845931408359519\\
91.85	0.00846138506145832\\
91.86	0.00846345624937289\\
91.87	0.00846552764611044\\
91.88	0.00846759925044083\\
91.89	0.00846967106113219\\
91.9	0.00847174307695097\\
91.91	0.00847381529666191\\
91.92	0.00847588771902805\\
91.93	0.00847796034281073\\
91.94	0.00848003316676962\\
91.95	0.00848210618966269\\
91.96	0.00848417941024623\\
91.97	0.00848625282727485\\
91.98	0.00848832643950151\\
91.99	0.00849040024567746\\
92	0.00849247424455233\\
92.01	0.00849454843487404\\
92.02	0.00849662281538891\\
92.03	0.00849869738484155\\
92.04	0.00850077214197495\\
92.05	0.00850284708553047\\
92.06	0.00850492221424779\\
92.07	0.00850699752686498\\
92.08	0.00850907302211849\\
92.09	0.00851114869874311\\
92.1	0.00851322455547203\\
92.11	0.00851530059103681\\
92.12	0.00851737680416741\\
92.13	0.00851945319359216\\
92.14	0.00852152975803781\\
92.15	0.00852360649622948\\
92.16	0.00852568340689073\\
92.17	0.00852776048874349\\
92.18	0.00852983774050814\\
92.19	0.00853191516090345\\
92.2	0.00853399274864664\\
92.21	0.00853607050245333\\
92.22	0.0085381484210376\\
92.23	0.00854022650311195\\
92.24	0.00854230474738733\\
92.25	0.00854438315257314\\
92.26	0.00854646171737722\\
92.27	0.00854854044050589\\
92.28	0.00855061932066392\\
92.29	0.00855269835655454\\
92.3	0.00855477754687948\\
92.31	0.00855685689033894\\
92.32	0.00855893638563158\\
92.33	0.00856101603145457\\
92.34	0.00856309582650359\\
92.35	0.0085651757694728\\
92.36	0.00856725585905487\\
92.37	0.00856933609394099\\
92.38	0.00857141647282085\\
92.39	0.00857349699438267\\
92.4	0.00857557765731323\\
92.41	0.00857765846029779\\
92.42	0.00857973940202021\\
92.43	0.00858182048116283\\
92.44	0.0085839016964066\\
92.45	0.008585983046431\\
92.46	0.00858806452991408\\
92.47	0.00859014614553246\\
92.48	0.00859222789196133\\
92.49	0.00859430976787447\\
92.5	0.00859639177194425\\
92.51	0.00859847390284163\\
92.52	0.00860055615923617\\
92.53	0.00860263853979604\\
92.54	0.00860472104318802\\
92.55	0.00860680366807751\\
92.56	0.00860888641312853\\
92.57	0.00861096927700376\\
92.58	0.0086130522583645\\
92.59	0.00861513535587067\\
92.6	0.00861721856818089\\
92.61	0.00861930189395242\\
92.62	0.00862138533184117\\
92.63	0.00862346888050175\\
92.64	0.00862555253858743\\
92.65	0.00862763630475017\\
92.66	0.00862972017764062\\
92.67	0.00863180415590815\\
92.68	0.00863388823820082\\
92.69	0.0086359724231654\\
92.7	0.0086380567094474\\
92.71	0.00864014109569104\\
92.72	0.00864222558053928\\
92.73	0.00864431016263385\\
92.74	0.00864639484061519\\
92.75	0.00864847961312253\\
92.76	0.00865056447879383\\
92.77	0.00865264943626586\\
92.78	0.00865473448417417\\
92.79	0.00865681962115305\\
92.8	0.00865890484583565\\
92.81	0.00866099015685389\\
92.82	0.00866307555283848\\
92.83	0.00866516103241901\\
92.84	0.00866724659422384\\
92.85	0.00866933223688018\\
92.86	0.00867141795901411\\
92.87	0.00867350375925053\\
92.88	0.00867558963621321\\
92.89	0.00867767558852478\\
92.9	0.00867976161480676\\
92.91	0.00868184771367954\\
92.92	0.00868393388376241\\
92.93	0.00868602012367357\\
92.94	0.00868810643203009\\
92.95	0.00869019280744801\\
92.96	0.00869227924854226\\
92.97	0.00869436575392671\\
92.98	0.00869645232221418\\
92.99	0.00869853895201643\\
93	0.00870062564194421\\
93.01	0.00870271239060721\\
93.02	0.0087047991966141\\
93.03	0.00870688605857254\\
93.04	0.0087089729750892\\
93.05	0.00871105994476976\\
93.06	0.00871314696621888\\
93.07	0.00871523403804026\\
93.08	0.00871732115883666\\
93.09	0.00871940832720985\\
93.1	0.00872149554176066\\
93.11	0.00872358280108898\\
93.12	0.00872567010379378\\
93.13	0.00872775744847311\\
93.14	0.0087298448337241\\
93.15	0.00873193225814298\\
93.16	0.0087340197203251\\
93.17	0.00873610721886492\\
93.18	0.00873819475235604\\
93.19	0.0087402823193912\\
93.2	0.00874236991856227\\
93.21	0.00874445754846029\\
93.22	0.00874654520767549\\
93.23	0.00874863289479724\\
93.24	0.00875072060841415\\
93.25	0.00875280834711399\\
93.26	0.00875489610948376\\
93.27	0.00875698389410968\\
93.28	0.00875907169957719\\
93.29	0.008761159524471\\
93.3	0.00876324736737505\\
93.31	0.00876533522687256\\
93.32	0.00876742310154601\\
93.33	0.00876951098997719\\
93.34	0.00877159889074716\\
93.35	0.00877368680243631\\
93.36	0.00877577472362434\\
93.37	0.0087778626528903\\
93.38	0.00877995058881255\\
93.39	0.00878203852996884\\
93.4	0.00878412647493625\\
93.41	0.00878621442229127\\
93.42	0.00878830237060975\\
93.43	0.00879039031846697\\
93.44	0.0087924782644376\\
93.45	0.00879456620709575\\
93.46	0.00879665414501495\\
93.47	0.00879874207676819\\
93.48	0.00880083000092791\\
93.49	0.00880291791606604\\
93.5	0.00880500582075399\\
93.51	0.00880709371356264\\
93.52	0.00880918159306242\\
93.53	0.00881126945782326\\
93.54	0.00881335730641463\\
93.55	0.00881544513740553\\
93.56	0.00881753294936456\\
93.57	0.00881962074085985\\
93.58	0.00882170851045915\\
93.59	0.00882379625672977\\
93.6	0.00882588397823867\\
93.61	0.00882797167355241\\
93.62	0.0088300593412372\\
93.63	0.0088321469798589\\
93.64	0.00883423458798302\\
93.65	0.00883632216417476\\
93.66	0.00883840970699901\\
93.67	0.00884049721502037\\
93.68	0.00884258468680313\\
93.69	0.00884467212091134\\
93.7	0.00884675951590879\\
93.71	0.00884884687035903\\
93.72	0.00885093418282536\\
93.73	0.0088530214518709\\
93.74	0.00885510867605856\\
93.75	0.00885719585395106\\
93.76	0.00885928298411094\\
93.77	0.00886137006510062\\
93.78	0.00886345709548233\\
93.79	0.00886554407381821\\
93.8	0.00886763099867029\\
93.81	0.00886971786860047\\
93.82	0.00887180468217059\\
93.83	0.00887389143794243\\
93.84	0.0088759781344777\\
93.85	0.00887806477033809\\
93.86	0.00888015134408524\\
93.87	0.00888223785428083\\
93.88	0.0088843242994865\\
93.89	0.00888641067826394\\
93.9	0.00888849698917488\\
93.91	0.0088905832307811\\
93.92	0.00889266940164446\\
93.93	0.00889475550032689\\
93.94	0.00889684152539044\\
93.95	0.00889892747539728\\
93.96	0.0089010133489097\\
93.97	0.00890309914449017\\
93.98	0.0089051848607013\\
93.99	0.0089072704961059\\
94	0.00890935604926699\\
94.01	0.00891144151874779\\
94.02	0.00891352690311177\\
94.03	0.00891561220092265\\
94.04	0.00891769741074441\\
94.05	0.00891978253114133\\
94.06	0.00892186756067799\\
94.07	0.00892395249791929\\
94.08	0.00892603734143048\\
94.09	0.00892812208977715\\
94.1	0.00893020674152527\\
94.11	0.00893229129524122\\
94.12	0.00893437574949177\\
94.13	0.00893646010284411\\
94.14	0.00893854435386591\\
94.15	0.00894062850112529\\
94.16	0.00894271254319083\\
94.17	0.00894479647863166\\
94.18	0.00894688030601739\\
94.19	0.00894896402391818\\
94.2	0.00895104763090477\\
94.21	0.00895313112554844\\
94.22	0.00895521450642111\\
94.23	0.00895729777209528\\
94.24	0.0089593809211441\\
94.25	0.00896146395214137\\
94.26	0.00896354686366159\\
94.27	0.00896562965427991\\
94.28	0.00896771232257223\\
94.29	0.00896979486711518\\
94.3	0.00897187728648612\\
94.31	0.00897395957926322\\
94.32	0.00897604174402542\\
94.33	0.00897812377935248\\
94.34	0.00898020568382501\\
94.35	0.00898228745602445\\
94.36	0.00898436909453316\\
94.37	0.00898645059793436\\
94.38	0.00898853196481221\\
94.39	0.00899061319375181\\
94.4	0.00899269428333921\\
94.41	0.00899477523216148\\
94.42	0.00899685603880665\\
94.43	0.00899893670186382\\
94.44	0.00900101721992311\\
94.45	0.00900309759157573\\
94.46	0.009005177815414\\
94.47	0.00900725789003131\\
94.48	0.00900933781402223\\
94.49	0.0090114175859825\\
94.5	0.00901349720450901\\
94.51	0.00901557666819989\\
94.52	0.00901765597565448\\
94.53	0.0090197351254734\\
94.54	0.00902181411625854\\
94.55	0.00902389294661307\\
94.56	0.00902597161514153\\
94.57	0.00902805012044977\\
94.58	0.00903012846114504\\
94.59	0.00903220663583598\\
94.6	0.00903428464313266\\
94.61	0.00903636248164659\\
94.62	0.00903844014999076\\
94.63	0.00904051764677967\\
94.64	0.00904259497062931\\
94.65	0.00904467212015724\\
94.66	0.00904674909398261\\
94.67	0.00904882589072612\\
94.68	0.00905090250901014\\
94.69	0.00905297894745864\\
94.7	0.00905505520469728\\
94.71	0.00905713127935337\\
94.72	0.00905920717005597\\
94.73	0.00906128287543585\\
94.74	0.00906335839412553\\
94.75	0.00906543372475931\\
94.76	0.00906750886597329\\
94.77	0.00906958381640541\\
94.78	0.00907165857469543\\
94.79	0.00907373313948499\\
94.8	0.00907580750941763\\
94.81	0.0090778816831388\\
94.82	0.00907995565929589\\
94.83	0.00908202943653827\\
94.84	0.00908410301351727\\
94.85	0.00908617638888625\\
94.86	0.00908824956130061\\
94.87	0.00909032252941781\\
94.88	0.00909239529189738\\
94.89	0.00909446784740098\\
94.9	0.00909654019459238\\
94.91	0.00909861233213755\\
94.92	0.00910068425870458\\
94.93	0.00910275597296383\\
94.94	0.00910482747358786\\
94.95	0.0091068987592515\\
94.96	0.00910896982863185\\
94.97	0.00911104068040833\\
94.98	0.0091131113132627\\
94.99	0.00911518172587906\\
95	0.0091172519169439\\
95.01	0.00911932188514614\\
95.02	0.00912139162917712\\
95.03	0.00912346114773062\\
95.04	0.00912553043950296\\
95.05	0.00912759950319293\\
95.06	0.00912966833750187\\
95.07	0.0091317369411337\\
95.08	0.00913380531279492\\
95.09	0.00913587345119465\\
95.1	0.00913794135504466\\
95.11	0.00914000902305939\\
95.12	0.00914207645395597\\
95.13	0.00914414364645428\\
95.14	0.00914621059927692\\
95.15	0.0091482773111493\\
95.16	0.00915034378079961\\
95.17	0.0091524100069589\\
95.18	0.00915447598836106\\
95.19	0.00915654172374289\\
95.2	0.00915860721184408\\
95.21	0.00916067245140728\\
95.22	0.00916273744117813\\
95.23	0.00916480217990523\\
95.24	0.00916686666634024\\
95.25	0.00916893089923786\\
95.26	0.00917099487735588\\
95.27	0.0091730585994552\\
95.28	0.00917512206429987\\
95.29	0.00917718527065709\\
95.3	0.00917924821729728\\
95.31	0.00918131090299407\\
95.32	0.00918337332652436\\
95.33	0.00918543548666832\\
95.34	0.00918749738220945\\
95.35	0.00918955901193456\\
95.36	0.00919162037463389\\
95.37	0.00919368146910102\\
95.38	0.009195742294133\\
95.39	0.00919780284853032\\
95.4	0.00919986313109697\\
95.41	0.00920192314064047\\
95.42	0.00920398287597186\\
95.43	0.00920604233590579\\
95.44	0.0092081015192605\\
95.45	0.00921016042485788\\
95.46	0.00921221905152349\\
95.47	0.00921427739808658\\
95.48	0.00921633546338014\\
95.49	0.00921839324624094\\
95.5	0.0092204507455095\\
95.51	0.0092225079600302\\
95.52	0.00922456488865125\\
95.53	0.00922662153022477\\
95.54	0.00922867788360678\\
95.55	0.00923073394765725\\
95.56	0.00923278972124014\\
95.57	0.00923484520322339\\
95.58	0.00923690039247902\\
95.59	0.0092389552878831\\
95.6	0.00924100988831582\\
95.61	0.00924306419266149\\
95.62	0.00924511819980861\\
95.63	0.00924717190864987\\
95.64	0.00924922531808219\\
95.65	0.00925127842700677\\
95.66	0.0092533312343291\\
95.67	0.00925538373895899\\
95.68	0.00925743593981064\\
95.69	0.00925948783580262\\
95.7	0.00926153942585795\\
95.71	0.0092635907089041\\
95.72	0.00926564168387303\\
95.73	0.00926769234970125\\
95.74	0.00926974270532982\\
95.75	0.00927179274970439\\
95.76	0.00927384248177523\\
95.77	0.00927589190049731\\
95.78	0.00927794100483025\\
95.79	0.00927998979373843\\
95.8	0.00928203826619098\\
95.81	0.00928408642116183\\
95.82	0.00928613425762974\\
95.83	0.00928818177457835\\
95.84	0.00929022897099618\\
95.85	0.0092922758458767\\
95.86	0.00929432239821833\\
95.87	0.00929636862702452\\
95.88	0.00929841453130373\\
95.89	0.00930046011006951\\
95.9	0.00930250536234053\\
95.91	0.00930455028714057\\
95.92	0.00930659488349861\\
95.93	0.00930863915044885\\
95.94	0.00931068308703072\\
95.95	0.00931272669228895\\
95.96	0.00931476996527357\\
95.97	0.00931681290503999\\
95.98	0.00931885551064899\\
95.99	0.00932089778116679\\
96	0.00932293971566507\\
96.01	0.00932498131322101\\
96.02	0.00932702257291732\\
96.03	0.00932906349384229\\
96.04	0.0093311040750898\\
96.05	0.0093331443157594\\
96.06	0.00933518421495629\\
96.07	0.00933722377179142\\
96.08	0.00933926298538147\\
96.09	0.00934130185484892\\
96.1	0.00934334037932209\\
96.11	0.00934537855793513\\
96.12	0.00934741638982813\\
96.13	0.00934945387414709\\
96.14	0.00935149101004401\\
96.15	0.0093535277966769\\
96.16	0.0093555642332098\\
96.17	0.00935760031881286\\
96.18	0.00935963605266235\\
96.19	0.00936167143394072\\
96.2	0.00936370646183658\\
96.21	0.00936574113554484\\
96.22	0.00936777545426664\\
96.23	0.00936980941720946\\
96.24	0.00937184302358714\\
96.25	0.00937387627261988\\
96.26	0.00937590916353435\\
96.27	0.00937794169556369\\
96.28	0.00937997386794751\\
96.29	0.00938200567993201\\
96.3	0.00938403713076996\\
96.31	0.00938606821972075\\
96.32	0.00938809894605045\\
96.33	0.00939012930903182\\
96.34	0.00939215930794437\\
96.35	0.0093941889420744\\
96.36	0.009396218210715\\
96.37	0.00939824711316617\\
96.38	0.00940027564873478\\
96.39	0.00940230381673464\\
96.4	0.00940433161648654\\
96.41	0.00940635904731832\\
96.42	0.00940838610856484\\
96.43	0.00941041279956807\\
96.44	0.00941243911967715\\
96.45	0.00941446506824837\\
96.46	0.00941649064464523\\
96.47	0.00941851584823854\\
96.48	0.00942054067840636\\
96.49	0.00942256513453411\\
96.5	0.00942458921601461\\
96.51	0.00942661292224808\\
96.52	0.00942863625264222\\
96.53	0.00943065920661222\\
96.54	0.00943268178358082\\
96.55	0.00943470398297836\\
96.56	0.00943672580424279\\
96.57	0.00943874724681973\\
96.58	0.00944076831016253\\
96.59	0.00944278899373227\\
96.6	0.00944480929699783\\
96.61	0.00944682921943593\\
96.62	0.00944884876053116\\
96.63	0.00945086791977602\\
96.64	0.00945288669667098\\
96.65	0.00945490509072452\\
96.66	0.00945692310145313\\
96.67	0.00945894072838142\\
96.68	0.0094609579710421\\
96.69	0.00946297482897606\\
96.7	0.00946499130173242\\
96.71	0.00946700738886851\\
96.72	0.00946902308994998\\
96.73	0.00947103840455083\\
96.74	0.00947305333225341\\
96.75	0.00947506787264852\\
96.76	0.00947708202533539\\
96.77	0.0094790957899218\\
96.78	0.00948110916602406\\
96.79	0.00948312215326705\\
96.8	0.00948513475128432\\
96.81	0.00948714695971807\\
96.82	0.00948915877821925\\
96.83	0.00949117020644755\\
96.84	0.00949318124407148\\
96.85	0.00949519189076839\\
96.86	0.00949720214622452\\
96.87	0.00949921201013507\\
96.88	0.00950122148220419\\
96.89	0.00950323056214506\\
96.9	0.00950523924967994\\
96.91	0.00950724754454018\\
96.92	0.0095092554464663\\
96.93	0.009511262955208\\
96.94	0.00951327007052423\\
96.95	0.00951527679218322\\
96.96	0.00951728311996252\\
96.97	0.00951928905364905\\
96.98	0.00952129459303916\\
96.99	0.00952329973793864\\
97	0.00952530448816279\\
97.01	0.00952730884353645\\
97.02	0.00952931280389405\\
97.03	0.00953131636907965\\
97.04	0.00953331953894699\\
97.05	0.00953532231335953\\
97.06	0.0095373246921905\\
97.07	0.00953932667532292\\
97.08	0.00954132826264969\\
97.09	0.00954332945407357\\
97.1	0.0095453302495073\\
97.11	0.00954733064887357\\
97.12	0.00954933065210512\\
97.13	0.00955133025914475\\
97.14	0.0095533294699454\\
97.15	0.00955532828447014\\
97.16	0.00955732670269225\\
97.17	0.00955932472459529\\
97.18	0.00956132235017309\\
97.19	0.0095633195794298\\
97.2	0.00956531641238\\
97.21	0.00956731284904864\\
97.22	0.0095693088894712\\
97.23	0.00957130453369362\\
97.24	0.00957329978177244\\
97.25	0.00957529463377479\\
97.26	0.00957728908977844\\
97.27	0.00957928314987185\\
97.28	0.00958127681415425\\
97.29	0.0095832700827356\\
97.3	0.00958526295573674\\
97.31	0.00958725543328932\\
97.32	0.00958924751553597\\
97.33	0.00959123920263022\\
97.34	0.00959323049473664\\
97.35	0.00959522139203083\\
97.36	0.00959721189469948\\
97.37	0.00959920200294044\\
97.38	0.0096011917169627\\
97.39	0.0096031810369865\\
97.4	0.00960516996324335\\
97.41	0.00960715849597607\\
97.42	0.00960914663543882\\
97.43	0.00961113438189718\\
97.44	0.00961312173562817\\
97.45	0.00961510869692029\\
97.46	0.00961709526607359\\
97.47	0.00961908144339969\\
97.48	0.00962106722922182\\
97.49	0.00962305262387489\\
97.5	0.00962503762770551\\
97.51	0.00962702224107204\\
97.52	0.00962900646434466\\
97.53	0.00963099029790537\\
97.54	0.00963297374214805\\
97.55	0.00963495679747852\\
97.56	0.00963693946431458\\
97.57	0.00963892174308601\\
97.58	0.0096409036342347\\
97.59	0.0096428851382146\\
97.6	0.00964486625549183\\
97.61	0.00964684698654469\\
97.62	0.0096488273318637\\
97.63	0.00965080729195169\\
97.64	0.00965278686732378\\
97.65	0.00965476605850744\\
97.66	0.00965674486604252\\
97.67	0.00965872329048132\\
97.68	0.0096607013323886\\
97.69	0.00966267899234164\\
97.7	0.00966465627093029\\
97.71	0.00966663316875698\\
97.72	0.00966860968643683\\
97.73	0.0096705858245976\\
97.74	0.00967256158387982\\
97.75	0.00967453696493679\\
97.76	0.00967651196843432\\
97.77	0.00967848659505067\\
97.78	0.00968046084547652\\
97.79	0.00968243472041504\\
97.8	0.00968440822059293\\
97.81	0.00968638134680899\\
97.82	0.00968835409987373\\
97.83	0.0096903264806095\\
97.84	0.0096922984898504\\
97.85	0.00969427012844236\\
97.86	0.00969624139724315\\
97.87	0.00969821229712239\\
97.88	0.00970018282896157\\
97.89	0.00970215299365407\\
97.9	0.00970412279366507\\
97.91	0.00970609223162047\\
97.92	0.00970806131017858\\
97.93	0.00971003003203033\\
97.94	0.00971199839989956\\
97.95	0.00971396641654319\\
97.96	0.00971593408475148\\
97.97	0.00971790140734819\\
97.98	0.00971986838719079\\
97.99	0.00972183502717071\\
98	0.00972380133021363\\
98.01	0.00972576718884031\\
98.02	0.0097277324437127\\
98.03	0.00972969709868626\\
98.04	0.00973166115766945\\
98.05	0.0097336246246242\\
98.06	0.00973558750356637\\
98.07	0.00973754979903053\\
98.08	0.00973950657959474\\
98.09	0.00974145638196767\\
98.1	0.00974339914775435\\
98.11	0.00974533481796013\\
98.12	0.00974726333298333\\
98.13	0.00974918463260789\\
98.14	0.00975109865599576\\
98.15	0.0097530053416793\\
98.16	0.00975490462755354\\
98.17	0.00975679645086824\\
98.18	0.00975868074821993\\
98.19	0.00976055706691691\\
98.2	0.00976242524915769\\
98.21	0.00976428484382074\\
98.22	0.009766134034081\\
98.23	0.0097679727397121\\
98.24	0.00976980087968108\\
98.25	0.00977161837213827\\
98.26	0.00977342513440691\\
98.27	0.00977522108297272\\
98.28	0.0097770061334732\\
98.29	0.00977878020068684\\
98.3	0.00978054340417745\\
98.31	0.00978229569797715\\
98.32	0.00978403699377235\\
98.33	0.00978576768453989\\
98.34	0.00978749282210007\\
98.35	0.00978921237663149\\
98.36	0.00979092631808516\\
98.37	0.00979263500045184\\
98.38	0.00979433840901014\\
98.39	0.00979603651513999\\
98.4	0.00979772929000989\\
98.41	0.00979941670457537\\
98.42	0.00980109872957738\\
98.43	0.00980277672842595\\
98.44	0.00980445111460809\\
98.45	0.0098061218695262\\
98.46	0.00980778897446873\\
98.47	0.00980945241061039\\
98.48	0.0098111121590124\\
98.49	0.00981276820062277\\
98.5	0.00981442051627664\\
98.51	0.00981606908669666\\
98.52	0.00981771389249343\\
98.53	0.00981935491416593\\
98.54	0.00982099213210204\\
98.55	0.0098226255265788\\
98.56	0.00982425507776307\\
98.57	0.00982588076571214\\
98.58	0.00982750257037453\\
98.59	0.00982912047159071\\
98.6	0.009830734449094\\
98.61	0.0098323444825114\\
98.62	0.00983395055136465\\
98.63	0.00983555263507116\\
98.64	0.00983715071294523\\
98.65	0.0098387447641991\\
98.66	0.00984033476836617\\
98.67	0.00984192070503014\\
98.68	0.00984350255368167\\
98.69	0.00984508029344775\\
98.7	0.00984665390327529\\
98.71	0.00984822336201848\\
98.72	0.0098497886484405\\
98.73	0.00985134974121392\\
98.74	0.00985290661762233\\
98.75	0.00985445924848284\\
98.76	0.00985600760434484\\
98.77	0.00985755165548777\\
98.78	0.00985909137191898\\
98.79	0.0098606267215656\\
98.8	0.00986215767152559\\
98.81	0.00986368418857422\\
98.82	0.0098652062391609\\
98.83	0.00986672378913649\\
98.84	0.00986823680374156\\
98.85	0.00986974524761878\\
98.86	0.00987124908506735\\
98.87	0.00987274828003957\\
98.88	0.00987424279613727\\
98.89	0.00987573259660837\\
98.9	0.00987721764434327\\
98.91	0.00987869790187125\\
98.92	0.00988017333135684\\
98.93	0.00988164389459616\\
98.94	0.00988310955301321\\
98.95	0.00988457026765612\\
98.96	0.00988602599919337\\
98.97	0.00988747670790996\\
98.98	0.00988892235370357\\
98.99	0.00989036289608069\\
99	0.00989179829415259\\
99.01	0.00989322850663147\\
99.02	0.00989465349182635\\
99.03	0.00989607320763906\\
99.04	0.00989748761156016\\
99.05	0.00989889666066474\\
99.06	0.00990030031160832\\
99.07	0.00990169852062259\\
99.08	0.00990309124351133\\
99.09	0.0099044784356463\\
99.1	0.00990586005196288\\
99.11	0.00990723604695566\\
99.12	0.00990860637467526\\
99.13	0.00990997098872422\\
99.14	0.00991132984225253\\
99.15	0.00991268288795305\\
99.16	0.009914030078057\\
99.17	0.00991537136432928\\
99.18	0.00991670669806387\\
99.19	0.00991803603007907\\
99.2	0.00991935931071277\\
99.21	0.00992067648981765\\
99.22	0.0099219875167563\\
99.23	0.00992329234039635\\
99.24	0.00992459090910551\\
99.25	0.00992588317074659\\
99.26	0.00992716907267241\\
99.27	0.00992844856172076\\
99.28	0.0099297215842092\\
99.29	0.0099309880859299\\
99.3	0.00993224801214435\\
99.31	0.0099335013075781\\
99.32	0.00993474791641538\\
99.33	0.00993598778229369\\
99.34	0.00993722084829831\\
99.35	0.00993844705695684\\
99.36	0.00993966635023357\\
99.37	0.00994087866952389\\
99.38	0.00994208395564854\\
99.39	0.00994328214884795\\
99.4	0.00994447318877637\\
99.41	0.00994565701449602\\
99.42	0.00994683356447118\\
99.43	0.00994800277656223\\
99.44	0.00994916458801953\\
99.45	0.00995031893547741\\
99.46	0.00995146575494794\\
99.47	0.00995260498181472\\
99.48	0.00995373655082658\\
99.49	0.00995486039609122\\
99.5	0.0099559764510688\\
99.51	0.00995708464856543\\
99.52	0.00995818492072661\\
99.53	0.00995927719903064\\
99.54	0.00996036141428188\\
99.55	0.00996143749660399\\
99.56	0.00996250537543313\\
99.57	0.00996356497951101\\
99.58	0.00996461623687793\\
99.59	0.00996565907486572\\
99.6	0.00996669342009061\\
99.61	0.00996771918980673\\
99.62	0.00996873629602636\\
99.63	0.00996974464986599\\
99.64	0.00997074416153733\\
99.65	0.00997173474033815\\
99.66	0.00997271629464304\\
99.67	0.00997368873189412\\
99.68	0.00997465195859159\\
99.69	0.00997560588028419\\
99.7	0.00997655040155958\\
99.71	0.00997748542604711\\
99.72	0.00997841085640911\\
99.73	0.00997932659433115\\
99.74	0.00998023254051213\\
99.75	0.00998112859465432\\
99.76	0.0099820146554532\\
99.77	0.00998289062058732\\
99.78	0.00998375638670789\\
99.79	0.00998461184942835\\
99.8	0.00998545690331381\\
99.81	0.00998629144187034\\
99.82	0.00998711535753411\\
99.83	0.00998792854166048\\
99.84	0.00998873088451288\\
99.85	0.00998952227525159\\
99.86	0.00999030260192242\\
99.87	0.00999107175144514\\
99.88	0.00999182960960189\\
99.89	0.00999257606102539\\
99.9	0.00999331098918694\\
99.91	0.00999403427638442\\
99.92	0.00999474580372994\\
99.93	0.0099954454511375\\
99.94	0.00999613309731035\\
99.95	0.0099968086197283\\
99.96	0.00999747189463473\\
99.97	0.00999812279702354\\
99.98	0.00999876120062581\\
99.99	0.00999938697789635\\
100	0.01\\
};
\addlegendentry{$q=2$};

\addplot [color=mycolor1,solid,forget plot]
  table[row sep=crcr]{%
0.01	0\\
0.02	0\\
0.03	0\\
0.04	0\\
0.05	0\\
0.06	0\\
0.07	0\\
0.08	0\\
0.09	0\\
0.1	0\\
0.11	0\\
0.12	0\\
0.13	0\\
0.14	0\\
0.15	0\\
0.16	0\\
0.17	0\\
0.18	0\\
0.19	0\\
0.2	0\\
0.21	0\\
0.22	0\\
0.23	0\\
0.24	0\\
0.25	0\\
0.26	0\\
0.27	0\\
0.28	0\\
0.29	0\\
0.3	0\\
0.31	0\\
0.32	0\\
0.33	0\\
0.34	0\\
0.35	0\\
0.36	0\\
0.37	0\\
0.38	0\\
0.39	0\\
0.4	0\\
0.41	0\\
0.42	0\\
0.43	0\\
0.44	0\\
0.45	0\\
0.46	0\\
0.47	0\\
0.48	0\\
0.49	0\\
0.5	0\\
0.51	0\\
0.52	0\\
0.53	0\\
0.54	0\\
0.55	0\\
0.56	0\\
0.57	0\\
0.58	0\\
0.59	0\\
0.6	0\\
0.61	0\\
0.62	0\\
0.63	0\\
0.64	0\\
0.65	0\\
0.66	0\\
0.67	0\\
0.68	0\\
0.69	0\\
0.7	0\\
0.71	0\\
0.72	0\\
0.73	0\\
0.74	0\\
0.75	0\\
0.76	0\\
0.77	0\\
0.78	0\\
0.79	0\\
0.8	0\\
0.81	0\\
0.82	0\\
0.83	0\\
0.84	0\\
0.85	0\\
0.86	0\\
0.87	0\\
0.88	0\\
0.89	0\\
0.9	0\\
0.91	0\\
0.92	0\\
0.93	0\\
0.94	0\\
0.95	0\\
0.96	0\\
0.97	0\\
0.98	0\\
0.99	0\\
1	0\\
1.01	0\\
1.02	0\\
1.03	0\\
1.04	0\\
1.05	0\\
1.06	0\\
1.07	0\\
1.08	0\\
1.09	0\\
1.1	0\\
1.11	0\\
1.12	0\\
1.13	0\\
1.14	0\\
1.15	0\\
1.16	0\\
1.17	0\\
1.18	0\\
1.19	0\\
1.2	0\\
1.21	0\\
1.22	0\\
1.23	0\\
1.24	0\\
1.25	0\\
1.26	0\\
1.27	0\\
1.28	0\\
1.29	0\\
1.3	0\\
1.31	0\\
1.32	0\\
1.33	0\\
1.34	0\\
1.35	0\\
1.36	0\\
1.37	0\\
1.38	0\\
1.39	0\\
1.4	0\\
1.41	0\\
1.42	0\\
1.43	0\\
1.44	0\\
1.45	0\\
1.46	0\\
1.47	0\\
1.48	0\\
1.49	0\\
1.5	0\\
1.51	0\\
1.52	0\\
1.53	0\\
1.54	0\\
1.55	0\\
1.56	0\\
1.57	0\\
1.58	0\\
1.59	0\\
1.6	0\\
1.61	0\\
1.62	0\\
1.63	0\\
1.64	0\\
1.65	0\\
1.66	0\\
1.67	0\\
1.68	0\\
1.69	0\\
1.7	0\\
1.71	0\\
1.72	0\\
1.73	0\\
1.74	0\\
1.75	0\\
1.76	0\\
1.77	0\\
1.78	0\\
1.79	0\\
1.8	0\\
1.81	0\\
1.82	0\\
1.83	0\\
1.84	0\\
1.85	0\\
1.86	0\\
1.87	0\\
1.88	0\\
1.89	0\\
1.9	0\\
1.91	0\\
1.92	0\\
1.93	0\\
1.94	0\\
1.95	0\\
1.96	0\\
1.97	0\\
1.98	0\\
1.99	0\\
2	0\\
2.01	0\\
2.02	0\\
2.03	0\\
2.04	0\\
2.05	0\\
2.06	0\\
2.07	0\\
2.08	0\\
2.09	0\\
2.1	0\\
2.11	0\\
2.12	0\\
2.13	0\\
2.14	0\\
2.15	0\\
2.16	0\\
2.17	0\\
2.18	0\\
2.19	0\\
2.2	0\\
2.21	0\\
2.22	0\\
2.23	0\\
2.24	0\\
2.25	0\\
2.26	0\\
2.27	0\\
2.28	0\\
2.29	0\\
2.3	0\\
2.31	0\\
2.32	0\\
2.33	0\\
2.34	0\\
2.35	0\\
2.36	0\\
2.37	0\\
2.38	0\\
2.39	0\\
2.4	0\\
2.41	0\\
2.42	0\\
2.43	0\\
2.44	0\\
2.45	0\\
2.46	0\\
2.47	0\\
2.48	0\\
2.49	0\\
2.5	0\\
2.51	0\\
2.52	0\\
2.53	0\\
2.54	0\\
2.55	0\\
2.56	0\\
2.57	0\\
2.58	0\\
2.59	0\\
2.6	0\\
2.61	0\\
2.62	0\\
2.63	0\\
2.64	0\\
2.65	0\\
2.66	0\\
2.67	0\\
2.68	0\\
2.69	0\\
2.7	0\\
2.71	0\\
2.72	0\\
2.73	0\\
2.74	0\\
2.75	0\\
2.76	0\\
2.77	0\\
2.78	0\\
2.79	0\\
2.8	0\\
2.81	0\\
2.82	0\\
2.83	0\\
2.84	0\\
2.85	0\\
2.86	0\\
2.87	0\\
2.88	0\\
2.89	0\\
2.9	0\\
2.91	0\\
2.92	0\\
2.93	0\\
2.94	0\\
2.95	0\\
2.96	0\\
2.97	0\\
2.98	0\\
2.99	0\\
3	0\\
3.01	0\\
3.02	0\\
3.03	0\\
3.04	0\\
3.05	0\\
3.06	0\\
3.07	0\\
3.08	0\\
3.09	0\\
3.1	0\\
3.11	0\\
3.12	0\\
3.13	0\\
3.14	0\\
3.15	0\\
3.16	0\\
3.17	0\\
3.18	0\\
3.19	0\\
3.2	0\\
3.21	0\\
3.22	0\\
3.23	0\\
3.24	0\\
3.25	0\\
3.26	0\\
3.27	0\\
3.28	0\\
3.29	0\\
3.3	0\\
3.31	0\\
3.32	0\\
3.33	0\\
3.34	0\\
3.35	0\\
3.36	0\\
3.37	0\\
3.38	0\\
3.39	0\\
3.4	0\\
3.41	0\\
3.42	0\\
3.43	0\\
3.44	0\\
3.45	0\\
3.46	0\\
3.47	0\\
3.48	0\\
3.49	0\\
3.5	0\\
3.51	0\\
3.52	0\\
3.53	0\\
3.54	0\\
3.55	0\\
3.56	0\\
3.57	0\\
3.58	0\\
3.59	0\\
3.6	0\\
3.61	0\\
3.62	0\\
3.63	0\\
3.64	0\\
3.65	0\\
3.66	0\\
3.67	0\\
3.68	0\\
3.69	0\\
3.7	0\\
3.71	0\\
3.72	0\\
3.73	0\\
3.74	0\\
3.75	0\\
3.76	0\\
3.77	0\\
3.78	0\\
3.79	0\\
3.8	0\\
3.81	0\\
3.82	0\\
3.83	0\\
3.84	0\\
3.85	0\\
3.86	0\\
3.87	0\\
3.88	0\\
3.89	0\\
3.9	0\\
3.91	0\\
3.92	0\\
3.93	0\\
3.94	0\\
3.95	0\\
3.96	0\\
3.97	0\\
3.98	0\\
3.99	0\\
4	0\\
4.01	0\\
4.02	0\\
4.03	0\\
4.04	0\\
4.05	0\\
4.06	0\\
4.07	0\\
4.08	0\\
4.09	0\\
4.1	0\\
4.11	0\\
4.12	0\\
4.13	0\\
4.14	0\\
4.15	0\\
4.16	0\\
4.17	0\\
4.18	0\\
4.19	0\\
4.2	0\\
4.21	0\\
4.22	0\\
4.23	0\\
4.24	0\\
4.25	0\\
4.26	0\\
4.27	0\\
4.28	0\\
4.29	0\\
4.3	0\\
4.31	0\\
4.32	0\\
4.33	0\\
4.34	0\\
4.35	0\\
4.36	0\\
4.37	0\\
4.38	0\\
4.39	0\\
4.4	0\\
4.41	0\\
4.42	0\\
4.43	0\\
4.44	0\\
4.45	0\\
4.46	0\\
4.47	0\\
4.48	0\\
4.49	0\\
4.5	0\\
4.51	0\\
4.52	0\\
4.53	0\\
4.54	0\\
4.55	0\\
4.56	0\\
4.57	0\\
4.58	0\\
4.59	0\\
4.6	0\\
4.61	0\\
4.62	0\\
4.63	0\\
4.64	0\\
4.65	0\\
4.66	0\\
4.67	0\\
4.68	0\\
4.69	0\\
4.7	0\\
4.71	0\\
4.72	0\\
4.73	0\\
4.74	0\\
4.75	0\\
4.76	0\\
4.77	0\\
4.78	0\\
4.79	0\\
4.8	0\\
4.81	0\\
4.82	0\\
4.83	0\\
4.84	0\\
4.85	0\\
4.86	0\\
4.87	0\\
4.88	0\\
4.89	0\\
4.9	0\\
4.91	0\\
4.92	0\\
4.93	0\\
4.94	0\\
4.95	0\\
4.96	0\\
4.97	0\\
4.98	0\\
4.99	0\\
5	0\\
5.01	0\\
5.02	0\\
5.03	0\\
5.04	0\\
5.05	0\\
5.06	0\\
5.07	0\\
5.08	0\\
5.09	0\\
5.1	0\\
5.11	0\\
5.12	0\\
5.13	0\\
5.14	0\\
5.15	0\\
5.16	0\\
5.17	0\\
5.18	0\\
5.19	0\\
5.2	0\\
5.21	0\\
5.22	0\\
5.23	0\\
5.24	0\\
5.25	0\\
5.26	0\\
5.27	0\\
5.28	0\\
5.29	0\\
5.3	0\\
5.31	0\\
5.32	0\\
5.33	0\\
5.34	0\\
5.35	0\\
5.36	0\\
5.37	0\\
5.38	0\\
5.39	0\\
5.4	0\\
5.41	0\\
5.42	0\\
5.43	0\\
5.44	0\\
5.45	0\\
5.46	0\\
5.47	0\\
5.48	0\\
5.49	0\\
5.5	0\\
5.51	0\\
5.52	0\\
5.53	0\\
5.54	0\\
5.55	0\\
5.56	0\\
5.57	0\\
5.58	0\\
5.59	0\\
5.6	0\\
5.61	0\\
5.62	0\\
5.63	0\\
5.64	0\\
5.65	0\\
5.66	0\\
5.67	0\\
5.68	0\\
5.69	0\\
5.7	0\\
5.71	0\\
5.72	0\\
5.73	0\\
5.74	0\\
5.75	0\\
5.76	0\\
5.77	0\\
5.78	0\\
5.79	0\\
5.8	0\\
5.81	0\\
5.82	0\\
5.83	0\\
5.84	0\\
5.85	0\\
5.86	0\\
5.87	0\\
5.88	0\\
5.89	0\\
5.9	0\\
5.91	0\\
5.92	0\\
5.93	0\\
5.94	0\\
5.95	0\\
5.96	0\\
5.97	0\\
5.98	0\\
5.99	0\\
6	0\\
6.01	0\\
6.02	0\\
6.03	0\\
6.04	0\\
6.05	0\\
6.06	0\\
6.07	0\\
6.08	0\\
6.09	0\\
6.1	0\\
6.11	0\\
6.12	0\\
6.13	0\\
6.14	0\\
6.15	0\\
6.16	0\\
6.17	0\\
6.18	0\\
6.19	0\\
6.2	0\\
6.21	0\\
6.22	0\\
6.23	0\\
6.24	0\\
6.25	0\\
6.26	0\\
6.27	0\\
6.28	0\\
6.29	0\\
6.3	0\\
6.31	0\\
6.32	0\\
6.33	0\\
6.34	0\\
6.35	0\\
6.36	0\\
6.37	0\\
6.38	0\\
6.39	0\\
6.4	0\\
6.41	0\\
6.42	0\\
6.43	0\\
6.44	0\\
6.45	0\\
6.46	0\\
6.47	0\\
6.48	0\\
6.49	0\\
6.5	0\\
6.51	0\\
6.52	0\\
6.53	0\\
6.54	0\\
6.55	0\\
6.56	0\\
6.57	0\\
6.58	0\\
6.59	0\\
6.6	0\\
6.61	0\\
6.62	0\\
6.63	0\\
6.64	0\\
6.65	0\\
6.66	0\\
6.67	0\\
6.68	0\\
6.69	0\\
6.7	0\\
6.71	0\\
6.72	0\\
6.73	0\\
6.74	0\\
6.75	0\\
6.76	0\\
6.77	0\\
6.78	0\\
6.79	0\\
6.8	0\\
6.81	0\\
6.82	0\\
6.83	0\\
6.84	0\\
6.85	0\\
6.86	0\\
6.87	0\\
6.88	0\\
6.89	0\\
6.9	0\\
6.91	0\\
6.92	0\\
6.93	0\\
6.94	0\\
6.95	0\\
6.96	0\\
6.97	0\\
6.98	0\\
6.99	0\\
7	0\\
7.01	0\\
7.02	0\\
7.03	0\\
7.04	0\\
7.05	0\\
7.06	0\\
7.07	0\\
7.08	0\\
7.09	0\\
7.1	0\\
7.11	0\\
7.12	0\\
7.13	0\\
7.14	0\\
7.15	0\\
7.16	0\\
7.17	0\\
7.18	0\\
7.19	0\\
7.2	0\\
7.21	0\\
7.22	0\\
7.23	0\\
7.24	0\\
7.25	0\\
7.26	0\\
7.27	0\\
7.28	0\\
7.29	0\\
7.3	0\\
7.31	0\\
7.32	0\\
7.33	0\\
7.34	0\\
7.35	0\\
7.36	0\\
7.37	0\\
7.38	0\\
7.39	0\\
7.4	0\\
7.41	0\\
7.42	0\\
7.43	0\\
7.44	0\\
7.45	0\\
7.46	0\\
7.47	0\\
7.48	0\\
7.49	0\\
7.5	0\\
7.51	0\\
7.52	0\\
7.53	0\\
7.54	0\\
7.55	0\\
7.56	0\\
7.57	0\\
7.58	0\\
7.59	0\\
7.6	0\\
7.61	0\\
7.62	0\\
7.63	0\\
7.64	0\\
7.65	0\\
7.66	0\\
7.67	0\\
7.68	0\\
7.69	0\\
7.7	0\\
7.71	0\\
7.72	0\\
7.73	0\\
7.74	0\\
7.75	0\\
7.76	0\\
7.77	0\\
7.78	0\\
7.79	0\\
7.8	0\\
7.81	0\\
7.82	0\\
7.83	0\\
7.84	0\\
7.85	0\\
7.86	0\\
7.87	0\\
7.88	0\\
7.89	0\\
7.9	0\\
7.91	0\\
7.92	0\\
7.93	0\\
7.94	0\\
7.95	0\\
7.96	0\\
7.97	0\\
7.98	0\\
7.99	0\\
8	0\\
8.01	0\\
8.02	0\\
8.03	0\\
8.04	0\\
8.05	0\\
8.06	0\\
8.07	0\\
8.08	0\\
8.09	0\\
8.1	0\\
8.11	0\\
8.12	0\\
8.13	0\\
8.14	0\\
8.15	0\\
8.16	0\\
8.17	0\\
8.18	0\\
8.19	0\\
8.2	0\\
8.21	0\\
8.22	0\\
8.23	0\\
8.24	0\\
8.25	0\\
8.26	0\\
8.27	0\\
8.28	0\\
8.29	0\\
8.3	0\\
8.31	0\\
8.32	0\\
8.33	0\\
8.34	0\\
8.35	0\\
8.36	0\\
8.37	0\\
8.38	0\\
8.39	0\\
8.4	0\\
8.41	0\\
8.42	0\\
8.43	0\\
8.44	0\\
8.45	0\\
8.46	0\\
8.47	0\\
8.48	0\\
8.49	0\\
8.5	0\\
8.51	0\\
8.52	0\\
8.53	0\\
8.54	0\\
8.55	0\\
8.56	0\\
8.57	0\\
8.58	0\\
8.59	0\\
8.6	0\\
8.61	0\\
8.62	0\\
8.63	0\\
8.64	0\\
8.65	0\\
8.66	0\\
8.67	0\\
8.68	0\\
8.69	0\\
8.7	0\\
8.71	0\\
8.72	0\\
8.73	0\\
8.74	0\\
8.75	0\\
8.76	0\\
8.77	0\\
8.78	0\\
8.79	0\\
8.8	0\\
8.81	0\\
8.82	0\\
8.83	0\\
8.84	0\\
8.85	0\\
8.86	0\\
8.87	0\\
8.88	0\\
8.89	0\\
8.9	0\\
8.91	0\\
8.92	0\\
8.93	0\\
8.94	0\\
8.95	0\\
8.96	0\\
8.97	0\\
8.98	0\\
8.99	0\\
9	0\\
9.01	0\\
9.02	0\\
9.03	0\\
9.04	0\\
9.05	0\\
9.06	0\\
9.07	0\\
9.08	0\\
9.09	0\\
9.1	0\\
9.11	0\\
9.12	0\\
9.13	0\\
9.14	0\\
9.15	0\\
9.16	0\\
9.17	0\\
9.18	0\\
9.19	0\\
9.2	0\\
9.21	0\\
9.22	0\\
9.23	0\\
9.24	0\\
9.25	0\\
9.26	0\\
9.27	0\\
9.28	0\\
9.29	0\\
9.3	0\\
9.31	0\\
9.32	0\\
9.33	0\\
9.34	0\\
9.35	0\\
9.36	0\\
9.37	0\\
9.38	0\\
9.39	0\\
9.4	0\\
9.41	0\\
9.42	0\\
9.43	0\\
9.44	0\\
9.45	0\\
9.46	0\\
9.47	0\\
9.48	0\\
9.49	0\\
9.5	0\\
9.51	0\\
9.52	0\\
9.53	0\\
9.54	0\\
9.55	0\\
9.56	0\\
9.57	0\\
9.58	0\\
9.59	0\\
9.6	0\\
9.61	0\\
9.62	0\\
9.63	0\\
9.64	0\\
9.65	0\\
9.66	0\\
9.67	0\\
9.68	0\\
9.69	0\\
9.7	0\\
9.71	0\\
9.72	0\\
9.73	0\\
9.74	0\\
9.75	0\\
9.76	0\\
9.77	0\\
9.78	0\\
9.79	0\\
9.8	0\\
9.81	0\\
9.82	0\\
9.83	0\\
9.84	0\\
9.85	0\\
9.86	0\\
9.87	0\\
9.88	0\\
9.89	0\\
9.9	0\\
9.91	0\\
9.92	0\\
9.93	0\\
9.94	0\\
9.95	0\\
9.96	0\\
9.97	0\\
9.98	0\\
9.99	0\\
10	0\\
10.01	0\\
10.02	0\\
10.03	0\\
10.04	0\\
10.05	0\\
10.06	0\\
10.07	0\\
10.08	0\\
10.09	0\\
10.1	0\\
10.11	0\\
10.12	0\\
10.13	0\\
10.14	0\\
10.15	0\\
10.16	0\\
10.17	0\\
10.18	0\\
10.19	0\\
10.2	0\\
10.21	0\\
10.22	0\\
10.23	0\\
10.24	0\\
10.25	0\\
10.26	0\\
10.27	0\\
10.28	0\\
10.29	0\\
10.3	0\\
10.31	0\\
10.32	0\\
10.33	0\\
10.34	0\\
10.35	0\\
10.36	0\\
10.37	0\\
10.38	0\\
10.39	0\\
10.4	0\\
10.41	0\\
10.42	0\\
10.43	0\\
10.44	0\\
10.45	0\\
10.46	0\\
10.47	0\\
10.48	0\\
10.49	0\\
10.5	0\\
10.51	0\\
10.52	0\\
10.53	0\\
10.54	0\\
10.55	0\\
10.56	0\\
10.57	0\\
10.58	0\\
10.59	0\\
10.6	0\\
10.61	0\\
10.62	0\\
10.63	0\\
10.64	0\\
10.65	0\\
10.66	0\\
10.67	0\\
10.68	0\\
10.69	0\\
10.7	0\\
10.71	0\\
10.72	0\\
10.73	0\\
10.74	0\\
10.75	0\\
10.76	0\\
10.77	0\\
10.78	0\\
10.79	0\\
10.8	0\\
10.81	0\\
10.82	0\\
10.83	0\\
10.84	0\\
10.85	0\\
10.86	0\\
10.87	0\\
10.88	0\\
10.89	0\\
10.9	0\\
10.91	0\\
10.92	0\\
10.93	0\\
10.94	0\\
10.95	0\\
10.96	0\\
10.97	0\\
10.98	0\\
10.99	0\\
11	0\\
11.01	0\\
11.02	0\\
11.03	0\\
11.04	0\\
11.05	0\\
11.06	0\\
11.07	0\\
11.08	0\\
11.09	0\\
11.1	0\\
11.11	0\\
11.12	0\\
11.13	0\\
11.14	0\\
11.15	0\\
11.16	0\\
11.17	0\\
11.18	0\\
11.19	0\\
11.2	0\\
11.21	0\\
11.22	0\\
11.23	0\\
11.24	0\\
11.25	0\\
11.26	0\\
11.27	0\\
11.28	0\\
11.29	0\\
11.3	0\\
11.31	0\\
11.32	0\\
11.33	0\\
11.34	0\\
11.35	0\\
11.36	0\\
11.37	0\\
11.38	0\\
11.39	0\\
11.4	0\\
11.41	0\\
11.42	0\\
11.43	0\\
11.44	0\\
11.45	0\\
11.46	0\\
11.47	0\\
11.48	0\\
11.49	0\\
11.5	0\\
11.51	0\\
11.52	0\\
11.53	0\\
11.54	0\\
11.55	0\\
11.56	0\\
11.57	0\\
11.58	0\\
11.59	0\\
11.6	0\\
11.61	0\\
11.62	0\\
11.63	0\\
11.64	0\\
11.65	0\\
11.66	0\\
11.67	0\\
11.68	0\\
11.69	0\\
11.7	0\\
11.71	0\\
11.72	0\\
11.73	0\\
11.74	0\\
11.75	0\\
11.76	0\\
11.77	0\\
11.78	0\\
11.79	0\\
11.8	0\\
11.81	0\\
11.82	0\\
11.83	0\\
11.84	0\\
11.85	0\\
11.86	0\\
11.87	0\\
11.88	0\\
11.89	0\\
11.9	0\\
11.91	0\\
11.92	0\\
11.93	0\\
11.94	0\\
11.95	0\\
11.96	0\\
11.97	0\\
11.98	0\\
11.99	0\\
12	0\\
12.01	0\\
12.02	0\\
12.03	0\\
12.04	0\\
12.05	0\\
12.06	0\\
12.07	0\\
12.08	0\\
12.09	0\\
12.1	0\\
12.11	0\\
12.12	0\\
12.13	0\\
12.14	0\\
12.15	0\\
12.16	0\\
12.17	0\\
12.18	0\\
12.19	0\\
12.2	0\\
12.21	0\\
12.22	0\\
12.23	0\\
12.24	0\\
12.25	0\\
12.26	0\\
12.27	0\\
12.28	0\\
12.29	0\\
12.3	0\\
12.31	0\\
12.32	0\\
12.33	0\\
12.34	0\\
12.35	0\\
12.36	0\\
12.37	0\\
12.38	0\\
12.39	0\\
12.4	0\\
12.41	0\\
12.42	0\\
12.43	0\\
12.44	0\\
12.45	0\\
12.46	0\\
12.47	0\\
12.48	0\\
12.49	0\\
12.5	0\\
12.51	0\\
12.52	0\\
12.53	0\\
12.54	0\\
12.55	0\\
12.56	0\\
12.57	0\\
12.58	0\\
12.59	0\\
12.6	0\\
12.61	0\\
12.62	0\\
12.63	0\\
12.64	0\\
12.65	0\\
12.66	0\\
12.67	0\\
12.68	0\\
12.69	0\\
12.7	0\\
12.71	0\\
12.72	0\\
12.73	0\\
12.74	0\\
12.75	0\\
12.76	0\\
12.77	0\\
12.78	0\\
12.79	0\\
12.8	0\\
12.81	0\\
12.82	0\\
12.83	0\\
12.84	0\\
12.85	0\\
12.86	0\\
12.87	0\\
12.88	0\\
12.89	0\\
12.9	0\\
12.91	0\\
12.92	0\\
12.93	0\\
12.94	0\\
12.95	0\\
12.96	0\\
12.97	0\\
12.98	0\\
12.99	0\\
13	0\\
13.01	0\\
13.02	0\\
13.03	0\\
13.04	0\\
13.05	0\\
13.06	0\\
13.07	0\\
13.08	0\\
13.09	0\\
13.1	0\\
13.11	0\\
13.12	0\\
13.13	0\\
13.14	0\\
13.15	0\\
13.16	0\\
13.17	0\\
13.18	0\\
13.19	0\\
13.2	0\\
13.21	0\\
13.22	0\\
13.23	0\\
13.24	0\\
13.25	0\\
13.26	0\\
13.27	0\\
13.28	0\\
13.29	0\\
13.3	0\\
13.31	0\\
13.32	0\\
13.33	0\\
13.34	0\\
13.35	0\\
13.36	0\\
13.37	0\\
13.38	0\\
13.39	0\\
13.4	0\\
13.41	0\\
13.42	0\\
13.43	0\\
13.44	0\\
13.45	0\\
13.46	0\\
13.47	0\\
13.48	0\\
13.49	0\\
13.5	0\\
13.51	0\\
13.52	0\\
13.53	0\\
13.54	0\\
13.55	0\\
13.56	0\\
13.57	0\\
13.58	0\\
13.59	0\\
13.6	0\\
13.61	0\\
13.62	0\\
13.63	0\\
13.64	0\\
13.65	0\\
13.66	0\\
13.67	0\\
13.68	0\\
13.69	0\\
13.7	0\\
13.71	0\\
13.72	0\\
13.73	0\\
13.74	0\\
13.75	0\\
13.76	0\\
13.77	0\\
13.78	0\\
13.79	0\\
13.8	0\\
13.81	0\\
13.82	0\\
13.83	0\\
13.84	0\\
13.85	0\\
13.86	0\\
13.87	0\\
13.88	0\\
13.89	0\\
13.9	0\\
13.91	0\\
13.92	0\\
13.93	0\\
13.94	0\\
13.95	0\\
13.96	0\\
13.97	0\\
13.98	0\\
13.99	0\\
14	0\\
14.01	0\\
14.02	0\\
14.03	0\\
14.04	0\\
14.05	0\\
14.06	0\\
14.07	0\\
14.08	0\\
14.09	0\\
14.1	0\\
14.11	0\\
14.12	0\\
14.13	0\\
14.14	0\\
14.15	0\\
14.16	0\\
14.17	0\\
14.18	0\\
14.19	0\\
14.2	0\\
14.21	0\\
14.22	0\\
14.23	0\\
14.24	0\\
14.25	0\\
14.26	0\\
14.27	0\\
14.28	0\\
14.29	0\\
14.3	0\\
14.31	0\\
14.32	0\\
14.33	0\\
14.34	0\\
14.35	0\\
14.36	0\\
14.37	0\\
14.38	0\\
14.39	0\\
14.4	0\\
14.41	0\\
14.42	0\\
14.43	0\\
14.44	0\\
14.45	0\\
14.46	0\\
14.47	0\\
14.48	0\\
14.49	0\\
14.5	0\\
14.51	0\\
14.52	0\\
14.53	0\\
14.54	0\\
14.55	0\\
14.56	0\\
14.57	0\\
14.58	0\\
14.59	0\\
14.6	0\\
14.61	0\\
14.62	0\\
14.63	0\\
14.64	0\\
14.65	0\\
14.66	0\\
14.67	0\\
14.68	0\\
14.69	0\\
14.7	0\\
14.71	0\\
14.72	0\\
14.73	0\\
14.74	0\\
14.75	0\\
14.76	0\\
14.77	0\\
14.78	0\\
14.79	0\\
14.8	0\\
14.81	0\\
14.82	0\\
14.83	0\\
14.84	0\\
14.85	0\\
14.86	0\\
14.87	0\\
14.88	0\\
14.89	0\\
14.9	0\\
14.91	0\\
14.92	0\\
14.93	0\\
14.94	0\\
14.95	0\\
14.96	0\\
14.97	0\\
14.98	0\\
14.99	0\\
15	0\\
15.01	0\\
15.02	0\\
15.03	0\\
15.04	0\\
15.05	0\\
15.06	0\\
15.07	0\\
15.08	0\\
15.09	0\\
15.1	0\\
15.11	0\\
15.12	0\\
15.13	0\\
15.14	0\\
15.15	0\\
15.16	0\\
15.17	0\\
15.18	0\\
15.19	0\\
15.2	0\\
15.21	0\\
15.22	0\\
15.23	0\\
15.24	0\\
15.25	0\\
15.26	0\\
15.27	0\\
15.28	0\\
15.29	0\\
15.3	0\\
15.31	0\\
15.32	0\\
15.33	0\\
15.34	0\\
15.35	0\\
15.36	0\\
15.37	0\\
15.38	0\\
15.39	0\\
15.4	0\\
15.41	0\\
15.42	0\\
15.43	0\\
15.44	0\\
15.45	0\\
15.46	0\\
15.47	0\\
15.48	0\\
15.49	0\\
15.5	0\\
15.51	0\\
15.52	0\\
15.53	0\\
15.54	0\\
15.55	0\\
15.56	0\\
15.57	0\\
15.58	0\\
15.59	0\\
15.6	0\\
15.61	0\\
15.62	0\\
15.63	0\\
15.64	0\\
15.65	0\\
15.66	0\\
15.67	0\\
15.68	0\\
15.69	0\\
15.7	0\\
15.71	0\\
15.72	0\\
15.73	0\\
15.74	0\\
15.75	0\\
15.76	0\\
15.77	0\\
15.78	0\\
15.79	0\\
15.8	0\\
15.81	0\\
15.82	0\\
15.83	0\\
15.84	0\\
15.85	0\\
15.86	0\\
15.87	0\\
15.88	0\\
15.89	0\\
15.9	0\\
15.91	0\\
15.92	0\\
15.93	0\\
15.94	0\\
15.95	0\\
15.96	0\\
15.97	0\\
15.98	0\\
15.99	0\\
16	0\\
16.01	0\\
16.02	0\\
16.03	0\\
16.04	0\\
16.05	0\\
16.06	0\\
16.07	0\\
16.08	0\\
16.09	0\\
16.1	0\\
16.11	0\\
16.12	0\\
16.13	0\\
16.14	0\\
16.15	0\\
16.16	0\\
16.17	0\\
16.18	0\\
16.19	0\\
16.2	0\\
16.21	0\\
16.22	0\\
16.23	0\\
16.24	0\\
16.25	0\\
16.26	0\\
16.27	0\\
16.28	0\\
16.29	0\\
16.3	0\\
16.31	0\\
16.32	0\\
16.33	0\\
16.34	0\\
16.35	0\\
16.36	0\\
16.37	0\\
16.38	0\\
16.39	0\\
16.4	0\\
16.41	0\\
16.42	0\\
16.43	0\\
16.44	0\\
16.45	0\\
16.46	0\\
16.47	0\\
16.48	0\\
16.49	0\\
16.5	0\\
16.51	0\\
16.52	0\\
16.53	0\\
16.54	0\\
16.55	0\\
16.56	0\\
16.57	0\\
16.58	0\\
16.59	0\\
16.6	0\\
16.61	0\\
16.62	0\\
16.63	0\\
16.64	0\\
16.65	0\\
16.66	0\\
16.67	0\\
16.68	0\\
16.69	0\\
16.7	0\\
16.71	0\\
16.72	0\\
16.73	0\\
16.74	0\\
16.75	0\\
16.76	0\\
16.77	0\\
16.78	0\\
16.79	0\\
16.8	0\\
16.81	0\\
16.82	0\\
16.83	0\\
16.84	0\\
16.85	0\\
16.86	0\\
16.87	0\\
16.88	0\\
16.89	0\\
16.9	0\\
16.91	0\\
16.92	0\\
16.93	0\\
16.94	0\\
16.95	0\\
16.96	0\\
16.97	0\\
16.98	0\\
16.99	0\\
17	0\\
17.01	0\\
17.02	0\\
17.03	0\\
17.04	0\\
17.05	0\\
17.06	0\\
17.07	0\\
17.08	0\\
17.09	0\\
17.1	0\\
17.11	0\\
17.12	0\\
17.13	0\\
17.14	0\\
17.15	0\\
17.16	0\\
17.17	0\\
17.18	0\\
17.19	0\\
17.2	0\\
17.21	0\\
17.22	0\\
17.23	0\\
17.24	0\\
17.25	0\\
17.26	0\\
17.27	0\\
17.28	0\\
17.29	0\\
17.3	0\\
17.31	0\\
17.32	0\\
17.33	0\\
17.34	0\\
17.35	0\\
17.36	0\\
17.37	0\\
17.38	0\\
17.39	0\\
17.4	0\\
17.41	0\\
17.42	0\\
17.43	0\\
17.44	0\\
17.45	0\\
17.46	0\\
17.47	0\\
17.48	0\\
17.49	0\\
17.5	0\\
17.51	0\\
17.52	0\\
17.53	0\\
17.54	0\\
17.55	0\\
17.56	0\\
17.57	0\\
17.58	0\\
17.59	0\\
17.6	0\\
17.61	0\\
17.62	0\\
17.63	0\\
17.64	0\\
17.65	0\\
17.66	0\\
17.67	0\\
17.68	0\\
17.69	0\\
17.7	0\\
17.71	0\\
17.72	0\\
17.73	0\\
17.74	0\\
17.75	0\\
17.76	0\\
17.77	0\\
17.78	0\\
17.79	0\\
17.8	0\\
17.81	0\\
17.82	0\\
17.83	0\\
17.84	0\\
17.85	0\\
17.86	0\\
17.87	0\\
17.88	0\\
17.89	0\\
17.9	0\\
17.91	0\\
17.92	0\\
17.93	0\\
17.94	0\\
17.95	0\\
17.96	0\\
17.97	0\\
17.98	0\\
17.99	0\\
18	0\\
18.01	0\\
18.02	0\\
18.03	0\\
18.04	0\\
18.05	0\\
18.06	0\\
18.07	0\\
18.08	0\\
18.09	0\\
18.1	0\\
18.11	0\\
18.12	0\\
18.13	0\\
18.14	0\\
18.15	0\\
18.16	0\\
18.17	0\\
18.18	0\\
18.19	0\\
18.2	0\\
18.21	0\\
18.22	0\\
18.23	0\\
18.24	0\\
18.25	0\\
18.26	0\\
18.27	0\\
18.28	0\\
18.29	0\\
18.3	0\\
18.31	0\\
18.32	0\\
18.33	0\\
18.34	0\\
18.35	0\\
18.36	0\\
18.37	0\\
18.38	0\\
18.39	0\\
18.4	0\\
18.41	0\\
18.42	0\\
18.43	0\\
18.44	0\\
18.45	0\\
18.46	0\\
18.47	0\\
18.48	0\\
18.49	0\\
18.5	0\\
18.51	0\\
18.52	0\\
18.53	0\\
18.54	0\\
18.55	0\\
18.56	0\\
18.57	0\\
18.58	0\\
18.59	0\\
18.6	0\\
18.61	0\\
18.62	0\\
18.63	0\\
18.64	0\\
18.65	0\\
18.66	0\\
18.67	0\\
18.68	0\\
18.69	0\\
18.7	0\\
18.71	0\\
18.72	0\\
18.73	0\\
18.74	0\\
18.75	0\\
18.76	0\\
18.77	0\\
18.78	0\\
18.79	0\\
18.8	0\\
18.81	0\\
18.82	0\\
18.83	0\\
18.84	0\\
18.85	0\\
18.86	0\\
18.87	0\\
18.88	0\\
18.89	0\\
18.9	0\\
18.91	0\\
18.92	0\\
18.93	0\\
18.94	0\\
18.95	0\\
18.96	0\\
18.97	0\\
18.98	0\\
18.99	0\\
19	0\\
19.01	0\\
19.02	0\\
19.03	0\\
19.04	0\\
19.05	0\\
19.06	0\\
19.07	0\\
19.08	0\\
19.09	0\\
19.1	0\\
19.11	0\\
19.12	0\\
19.13	0\\
19.14	0\\
19.15	0\\
19.16	0\\
19.17	0\\
19.18	0\\
19.19	0\\
19.2	0\\
19.21	0\\
19.22	0\\
19.23	0\\
19.24	0\\
19.25	0\\
19.26	0\\
19.27	0\\
19.28	0\\
19.29	0\\
19.3	0\\
19.31	0\\
19.32	0\\
19.33	0\\
19.34	0\\
19.35	0\\
19.36	0\\
19.37	0\\
19.38	0\\
19.39	0\\
19.4	0\\
19.41	0\\
19.42	0\\
19.43	0\\
19.44	0\\
19.45	0\\
19.46	0\\
19.47	0\\
19.48	0\\
19.49	0\\
19.5	0\\
19.51	0\\
19.52	0\\
19.53	0\\
19.54	0\\
19.55	0\\
19.56	0\\
19.57	0\\
19.58	0\\
19.59	0\\
19.6	0\\
19.61	0\\
19.62	0\\
19.63	0\\
19.64	0\\
19.65	0\\
19.66	0\\
19.67	0\\
19.68	0\\
19.69	0\\
19.7	0\\
19.71	0\\
19.72	0\\
19.73	0\\
19.74	0\\
19.75	0\\
19.76	0\\
19.77	0\\
19.78	0\\
19.79	0\\
19.8	0\\
19.81	0\\
19.82	0\\
19.83	0\\
19.84	0\\
19.85	0\\
19.86	0\\
19.87	0\\
19.88	0\\
19.89	0\\
19.9	0\\
19.91	0\\
19.92	0\\
19.93	0\\
19.94	0\\
19.95	0\\
19.96	0\\
19.97	0\\
19.98	0\\
19.99	0\\
20	0\\
20.01	0\\
20.02	0\\
20.03	0\\
20.04	0\\
20.05	0\\
20.06	0\\
20.07	0\\
20.08	0\\
20.09	0\\
20.1	0\\
20.11	0\\
20.12	0\\
20.13	0\\
20.14	0\\
20.15	0\\
20.16	0\\
20.17	0\\
20.18	0\\
20.19	0\\
20.2	0\\
20.21	0\\
20.22	0\\
20.23	0\\
20.24	0\\
20.25	0\\
20.26	0\\
20.27	0\\
20.28	0\\
20.29	0\\
20.3	0\\
20.31	0\\
20.32	0\\
20.33	0\\
20.34	0\\
20.35	0\\
20.36	0\\
20.37	0\\
20.38	0\\
20.39	0\\
20.4	0\\
20.41	0\\
20.42	0\\
20.43	0\\
20.44	0\\
20.45	0\\
20.46	0\\
20.47	0\\
20.48	0\\
20.49	0\\
20.5	0\\
20.51	0\\
20.52	0\\
20.53	0\\
20.54	0\\
20.55	0\\
20.56	0\\
20.57	0\\
20.58	0\\
20.59	0\\
20.6	0\\
20.61	0\\
20.62	0\\
20.63	0\\
20.64	0\\
20.65	0\\
20.66	0\\
20.67	0\\
20.68	0\\
20.69	0\\
20.7	0\\
20.71	0\\
20.72	0\\
20.73	0\\
20.74	0\\
20.75	0\\
20.76	0\\
20.77	0\\
20.78	0\\
20.79	0\\
20.8	0\\
20.81	0\\
20.82	0\\
20.83	0\\
20.84	0\\
20.85	0\\
20.86	0\\
20.87	0\\
20.88	0\\
20.89	0\\
20.9	0\\
20.91	0\\
20.92	0\\
20.93	0\\
20.94	0\\
20.95	0\\
20.96	0\\
20.97	0\\
20.98	0\\
20.99	0\\
21	0\\
21.01	0\\
21.02	0\\
21.03	0\\
21.04	0\\
21.05	0\\
21.06	0\\
21.07	0\\
21.08	0\\
21.09	0\\
21.1	0\\
21.11	0\\
21.12	0\\
21.13	0\\
21.14	0\\
21.15	0\\
21.16	0\\
21.17	0\\
21.18	0\\
21.19	0\\
21.2	0\\
21.21	0\\
21.22	0\\
21.23	0\\
21.24	0\\
21.25	0\\
21.26	0\\
21.27	0\\
21.28	0\\
21.29	0\\
21.3	0\\
21.31	0\\
21.32	0\\
21.33	0\\
21.34	0\\
21.35	0\\
21.36	0\\
21.37	0\\
21.38	0\\
21.39	0\\
21.4	0\\
21.41	0\\
21.42	0\\
21.43	0\\
21.44	0\\
21.45	0\\
21.46	0\\
21.47	0\\
21.48	0\\
21.49	0\\
21.5	0\\
21.51	0\\
21.52	0\\
21.53	0\\
21.54	0\\
21.55	0\\
21.56	0\\
21.57	0\\
21.58	0\\
21.59	0\\
21.6	0\\
21.61	0\\
21.62	0\\
21.63	0\\
21.64	0\\
21.65	0\\
21.66	0\\
21.67	0\\
21.68	0\\
21.69	0\\
21.7	0\\
21.71	0\\
21.72	0\\
21.73	0\\
21.74	0\\
21.75	0\\
21.76	0\\
21.77	0\\
21.78	0\\
21.79	0\\
21.8	0\\
21.81	0\\
21.82	0\\
21.83	0\\
21.84	0\\
21.85	0\\
21.86	0\\
21.87	0\\
21.88	0\\
21.89	0\\
21.9	0\\
21.91	0\\
21.92	0\\
21.93	0\\
21.94	0\\
21.95	0\\
21.96	0\\
21.97	0\\
21.98	0\\
21.99	0\\
22	0\\
22.01	0\\
22.02	0\\
22.03	0\\
22.04	0\\
22.05	0\\
22.06	0\\
22.07	0\\
22.08	0\\
22.09	0\\
22.1	0\\
22.11	0\\
22.12	0\\
22.13	0\\
22.14	0\\
22.15	0\\
22.16	0\\
22.17	0\\
22.18	0\\
22.19	0\\
22.2	0\\
22.21	0\\
22.22	0\\
22.23	0\\
22.24	0\\
22.25	0\\
22.26	0\\
22.27	0\\
22.28	0\\
22.29	0\\
22.3	0\\
22.31	0\\
22.32	0\\
22.33	0\\
22.34	0\\
22.35	0\\
22.36	0\\
22.37	0\\
22.38	0\\
22.39	0\\
22.4	0\\
22.41	0\\
22.42	0\\
22.43	0\\
22.44	0\\
22.45	0\\
22.46	0\\
22.47	0\\
22.48	0\\
22.49	0\\
22.5	0\\
22.51	0\\
22.52	0\\
22.53	0\\
22.54	0\\
22.55	0\\
22.56	0\\
22.57	0\\
22.58	0\\
22.59	0\\
22.6	0\\
22.61	0\\
22.62	0\\
22.63	0\\
22.64	0\\
22.65	0\\
22.66	0\\
22.67	0\\
22.68	0\\
22.69	0\\
22.7	0\\
22.71	0\\
22.72	0\\
22.73	0\\
22.74	0\\
22.75	0\\
22.76	0\\
22.77	0\\
22.78	0\\
22.79	0\\
22.8	0\\
22.81	0\\
22.82	0\\
22.83	0\\
22.84	0\\
22.85	0\\
22.86	0\\
22.87	0\\
22.88	0\\
22.89	0\\
22.9	0\\
22.91	0\\
22.92	0\\
22.93	0\\
22.94	0\\
22.95	0\\
22.96	0\\
22.97	0\\
22.98	0\\
22.99	0\\
23	0\\
23.01	0\\
23.02	0\\
23.03	0\\
23.04	0\\
23.05	0\\
23.06	0\\
23.07	0\\
23.08	0\\
23.09	0\\
23.1	0\\
23.11	0\\
23.12	0\\
23.13	0\\
23.14	0\\
23.15	0\\
23.16	0\\
23.17	0\\
23.18	0\\
23.19	0\\
23.2	0\\
23.21	0\\
23.22	0\\
23.23	0\\
23.24	0\\
23.25	0\\
23.26	0\\
23.27	0\\
23.28	0\\
23.29	0\\
23.3	0\\
23.31	0\\
23.32	0\\
23.33	0\\
23.34	0\\
23.35	0\\
23.36	0\\
23.37	0\\
23.38	0\\
23.39	0\\
23.4	0\\
23.41	0\\
23.42	0\\
23.43	0\\
23.44	0\\
23.45	0\\
23.46	0\\
23.47	0\\
23.48	0\\
23.49	0\\
23.5	0\\
23.51	0\\
23.52	0\\
23.53	0\\
23.54	0\\
23.55	0\\
23.56	0\\
23.57	0\\
23.58	0\\
23.59	0\\
23.6	0\\
23.61	0\\
23.62	0\\
23.63	0\\
23.64	0\\
23.65	0\\
23.66	0\\
23.67	0\\
23.68	0\\
23.69	0\\
23.7	0\\
23.71	0\\
23.72	0\\
23.73	0\\
23.74	0\\
23.75	0\\
23.76	0\\
23.77	0\\
23.78	0\\
23.79	0\\
23.8	0\\
23.81	0\\
23.82	0\\
23.83	0\\
23.84	0\\
23.85	0\\
23.86	0\\
23.87	0\\
23.88	0\\
23.89	0\\
23.9	0\\
23.91	0\\
23.92	0\\
23.93	0\\
23.94	0\\
23.95	0\\
23.96	0\\
23.97	0\\
23.98	0\\
23.99	0\\
24	0\\
24.01	0\\
24.02	0\\
24.03	0\\
24.04	0\\
24.05	0\\
24.06	0\\
24.07	0\\
24.08	0\\
24.09	0\\
24.1	0\\
24.11	0\\
24.12	0\\
24.13	0\\
24.14	0\\
24.15	0\\
24.16	0\\
24.17	0\\
24.18	0\\
24.19	0\\
24.2	0\\
24.21	0\\
24.22	0\\
24.23	0\\
24.24	0\\
24.25	0\\
24.26	0\\
24.27	0\\
24.28	0\\
24.29	0\\
24.3	0\\
24.31	0\\
24.32	0\\
24.33	0\\
24.34	0\\
24.35	0\\
24.36	0\\
24.37	0\\
24.38	0\\
24.39	0\\
24.4	0\\
24.41	0\\
24.42	0\\
24.43	0\\
24.44	0\\
24.45	0\\
24.46	0\\
24.47	0\\
24.48	0\\
24.49	0\\
24.5	0\\
24.51	0\\
24.52	0\\
24.53	0\\
24.54	0\\
24.55	0\\
24.56	0\\
24.57	0\\
24.58	0\\
24.59	0\\
24.6	0\\
24.61	0\\
24.62	0\\
24.63	0\\
24.64	0\\
24.65	0\\
24.66	0\\
24.67	0\\
24.68	0\\
24.69	0\\
24.7	0\\
24.71	0\\
24.72	0\\
24.73	0\\
24.74	0\\
24.75	0\\
24.76	0\\
24.77	0\\
24.78	0\\
24.79	0\\
24.8	0\\
24.81	0\\
24.82	0\\
24.83	0\\
24.84	0\\
24.85	0\\
24.86	0\\
24.87	0\\
24.88	0\\
24.89	0\\
24.9	0\\
24.91	0\\
24.92	0\\
24.93	0\\
24.94	0\\
24.95	0\\
24.96	0\\
24.97	0\\
24.98	0\\
24.99	0\\
25	0\\
25.01	0\\
25.02	0\\
25.03	0\\
25.04	0\\
25.05	0\\
25.06	0\\
25.07	0\\
25.08	0\\
25.09	0\\
25.1	0\\
25.11	0\\
25.12	0\\
25.13	0\\
25.14	0\\
25.15	0\\
25.16	0\\
25.17	0\\
25.18	0\\
25.19	0\\
25.2	0\\
25.21	0\\
25.22	0\\
25.23	0\\
25.24	0\\
25.25	0\\
25.26	0\\
25.27	0\\
25.28	0\\
25.29	0\\
25.3	0\\
25.31	0\\
25.32	0\\
25.33	0\\
25.34	0\\
25.35	0\\
25.36	0\\
25.37	0\\
25.38	0\\
25.39	0\\
25.4	0\\
25.41	0\\
25.42	0\\
25.43	0\\
25.44	0\\
25.45	0\\
25.46	0\\
25.47	0\\
25.48	0\\
25.49	0\\
25.5	0\\
25.51	0\\
25.52	0\\
25.53	0\\
25.54	0\\
25.55	0\\
25.56	0\\
25.57	0\\
25.58	0\\
25.59	0\\
25.6	0\\
25.61	0\\
25.62	0\\
25.63	0\\
25.64	0\\
25.65	0\\
25.66	0\\
25.67	0\\
25.68	0\\
25.69	0\\
25.7	0\\
25.71	0\\
25.72	0\\
25.73	0\\
25.74	0\\
25.75	0\\
25.76	0\\
25.77	0\\
25.78	0\\
25.79	0\\
25.8	0\\
25.81	0\\
25.82	0\\
25.83	0\\
25.84	0\\
25.85	0\\
25.86	0\\
25.87	0\\
25.88	0\\
25.89	0\\
25.9	0\\
25.91	0\\
25.92	0\\
25.93	0\\
25.94	0\\
25.95	0\\
25.96	0\\
25.97	0\\
25.98	0\\
25.99	0\\
26	0\\
26.01	0\\
26.02	0\\
26.03	0\\
26.04	0\\
26.05	0\\
26.06	0\\
26.07	0\\
26.08	0\\
26.09	0\\
26.1	0\\
26.11	0\\
26.12	0\\
26.13	0\\
26.14	0\\
26.15	0\\
26.16	0\\
26.17	0\\
26.18	0\\
26.19	0\\
26.2	0\\
26.21	0\\
26.22	0\\
26.23	0\\
26.24	0\\
26.25	0\\
26.26	0\\
26.27	0\\
26.28	0\\
26.29	0\\
26.3	0\\
26.31	0\\
26.32	0\\
26.33	0\\
26.34	0\\
26.35	0\\
26.36	0\\
26.37	0\\
26.38	0\\
26.39	0\\
26.4	0\\
26.41	0\\
26.42	0\\
26.43	0\\
26.44	0\\
26.45	0\\
26.46	0\\
26.47	0\\
26.48	0\\
26.49	0\\
26.5	0\\
26.51	0\\
26.52	0\\
26.53	0\\
26.54	0\\
26.55	0\\
26.56	0\\
26.57	0\\
26.58	0\\
26.59	0\\
26.6	0\\
26.61	0\\
26.62	0\\
26.63	0\\
26.64	0\\
26.65	0\\
26.66	0\\
26.67	0\\
26.68	0\\
26.69	0\\
26.7	0\\
26.71	0\\
26.72	0\\
26.73	0\\
26.74	0\\
26.75	0\\
26.76	0\\
26.77	0\\
26.78	0\\
26.79	0\\
26.8	0\\
26.81	0\\
26.82	0\\
26.83	0\\
26.84	0\\
26.85	0\\
26.86	0\\
26.87	0\\
26.88	0\\
26.89	0\\
26.9	0\\
26.91	0\\
26.92	0\\
26.93	0\\
26.94	0\\
26.95	0\\
26.96	0\\
26.97	0\\
26.98	0\\
26.99	0\\
27	0\\
27.01	0\\
27.02	0\\
27.03	0\\
27.04	0\\
27.05	0\\
27.06	0\\
27.07	0\\
27.08	0\\
27.09	0\\
27.1	0\\
27.11	0\\
27.12	0\\
27.13	0\\
27.14	0\\
27.15	0\\
27.16	0\\
27.17	0\\
27.18	0\\
27.19	0\\
27.2	0\\
27.21	0\\
27.22	0\\
27.23	0\\
27.24	0\\
27.25	0\\
27.26	0\\
27.27	0\\
27.28	0\\
27.29	0\\
27.3	0\\
27.31	0\\
27.32	0\\
27.33	0\\
27.34	0\\
27.35	0\\
27.36	0\\
27.37	0\\
27.38	0\\
27.39	0\\
27.4	0\\
27.41	0\\
27.42	0\\
27.43	0\\
27.44	0\\
27.45	0\\
27.46	0\\
27.47	0\\
27.48	0\\
27.49	0\\
27.5	0\\
27.51	0\\
27.52	0\\
27.53	0\\
27.54	0\\
27.55	0\\
27.56	0\\
27.57	0\\
27.58	0\\
27.59	0\\
27.6	0\\
27.61	0\\
27.62	0\\
27.63	0\\
27.64	0\\
27.65	0\\
27.66	0\\
27.67	0\\
27.68	0\\
27.69	0\\
27.7	0\\
27.71	0\\
27.72	0\\
27.73	0\\
27.74	0\\
27.75	0\\
27.76	0\\
27.77	0\\
27.78	0\\
27.79	0\\
27.8	0\\
27.81	0\\
27.82	0\\
27.83	0\\
27.84	0\\
27.85	0\\
27.86	0\\
27.87	0\\
27.88	0\\
27.89	0\\
27.9	0\\
27.91	0\\
27.92	0\\
27.93	0\\
27.94	0\\
27.95	0\\
27.96	0\\
27.97	0\\
27.98	0\\
27.99	0\\
28	0\\
28.01	0\\
28.02	0\\
28.03	0\\
28.04	0\\
28.05	0\\
28.06	0\\
28.07	0\\
28.08	0\\
28.09	0\\
28.1	0\\
28.11	0\\
28.12	0\\
28.13	0\\
28.14	0\\
28.15	0\\
28.16	0\\
28.17	0\\
28.18	0\\
28.19	0\\
28.2	0\\
28.21	0\\
28.22	0\\
28.23	0\\
28.24	0\\
28.25	0\\
28.26	0\\
28.27	0\\
28.28	0\\
28.29	0\\
28.3	0\\
28.31	0\\
28.32	0\\
28.33	0\\
28.34	0\\
28.35	0\\
28.36	0\\
28.37	0\\
28.38	0\\
28.39	0\\
28.4	0\\
28.41	0\\
28.42	0\\
28.43	0\\
28.44	0\\
28.45	0\\
28.46	0\\
28.47	0\\
28.48	0\\
28.49	0\\
28.5	0\\
28.51	0\\
28.52	0\\
28.53	0\\
28.54	0\\
28.55	0\\
28.56	0\\
28.57	0\\
28.58	0\\
28.59	0\\
28.6	0\\
28.61	0\\
28.62	0\\
28.63	0\\
28.64	0\\
28.65	0\\
28.66	0\\
28.67	0\\
28.68	0\\
28.69	0\\
28.7	0\\
28.71	0\\
28.72	0\\
28.73	0\\
28.74	0\\
28.75	0\\
28.76	0\\
28.77	0\\
28.78	0\\
28.79	0\\
28.8	0\\
28.81	0\\
28.82	0\\
28.83	0\\
28.84	0\\
28.85	0\\
28.86	0\\
28.87	0\\
28.88	0\\
28.89	0\\
28.9	0\\
28.91	0\\
28.92	0\\
28.93	0\\
28.94	0\\
28.95	0\\
28.96	0\\
28.97	0\\
28.98	0\\
28.99	0\\
29	0\\
29.01	0\\
29.02	0\\
29.03	0\\
29.04	0\\
29.05	0\\
29.06	0\\
29.07	0\\
29.08	0\\
29.09	0\\
29.1	0\\
29.11	0\\
29.12	0\\
29.13	0\\
29.14	0\\
29.15	0\\
29.16	0\\
29.17	0\\
29.18	0\\
29.19	0\\
29.2	0\\
29.21	0\\
29.22	0\\
29.23	0\\
29.24	0\\
29.25	0\\
29.26	0\\
29.27	0\\
29.28	0\\
29.29	0\\
29.3	0\\
29.31	0\\
29.32	0\\
29.33	0\\
29.34	0\\
29.35	0\\
29.36	0\\
29.37	0\\
29.38	0\\
29.39	0\\
29.4	0\\
29.41	0\\
29.42	0\\
29.43	0\\
29.44	0\\
29.45	0\\
29.46	0\\
29.47	0\\
29.48	0\\
29.49	0\\
29.5	0\\
29.51	0\\
29.52	0\\
29.53	0\\
29.54	0\\
29.55	0\\
29.56	0\\
29.57	0\\
29.58	0\\
29.59	0\\
29.6	0\\
29.61	0\\
29.62	0\\
29.63	0\\
29.64	0\\
29.65	0\\
29.66	0\\
29.67	0\\
29.68	0\\
29.69	0\\
29.7	0\\
29.71	0\\
29.72	0\\
29.73	0\\
29.74	0\\
29.75	0\\
29.76	0\\
29.77	0\\
29.78	0\\
29.79	0\\
29.8	0\\
29.81	0\\
29.82	0\\
29.83	0\\
29.84	0\\
29.85	0\\
29.86	0\\
29.87	0\\
29.88	0\\
29.89	0\\
29.9	0\\
29.91	0\\
29.92	0\\
29.93	0\\
29.94	0\\
29.95	0\\
29.96	0\\
29.97	0\\
29.98	0\\
29.99	0\\
30	0\\
30.01	0\\
30.02	0\\
30.03	0\\
30.04	0\\
30.05	0\\
30.06	0\\
30.07	0\\
30.08	0\\
30.09	0\\
30.1	0\\
30.11	0\\
30.12	0\\
30.13	0\\
30.14	0\\
30.15	0\\
30.16	0\\
30.17	0\\
30.18	0\\
30.19	0\\
30.2	0\\
30.21	0\\
30.22	0\\
30.23	0\\
30.24	0\\
30.25	0\\
30.26	0\\
30.27	0\\
30.28	0\\
30.29	0\\
30.3	0\\
30.31	0\\
30.32	0\\
30.33	0\\
30.34	0\\
30.35	0\\
30.36	0\\
30.37	0\\
30.38	0\\
30.39	0\\
30.4	0\\
30.41	0\\
30.42	0\\
30.43	0\\
30.44	0\\
30.45	0\\
30.46	0\\
30.47	0\\
30.48	0\\
30.49	0\\
30.5	0\\
30.51	0\\
30.52	0\\
30.53	0\\
30.54	0\\
30.55	0\\
30.56	0\\
30.57	0\\
30.58	0\\
30.59	0\\
30.6	0\\
30.61	0\\
30.62	0\\
30.63	0\\
30.64	0\\
30.65	0\\
30.66	0\\
30.67	0\\
30.68	0\\
30.69	0\\
30.7	0\\
30.71	0\\
30.72	0\\
30.73	0\\
30.74	0\\
30.75	0\\
30.76	0\\
30.77	0\\
30.78	0\\
30.79	0\\
30.8	0\\
30.81	0\\
30.82	0\\
30.83	0\\
30.84	0\\
30.85	0\\
30.86	0\\
30.87	0\\
30.88	0\\
30.89	0\\
30.9	0\\
30.91	0\\
30.92	0\\
30.93	0\\
30.94	0\\
30.95	0\\
30.96	0\\
30.97	0\\
30.98	0\\
30.99	0\\
31	0\\
31.01	0\\
31.02	0\\
31.03	0\\
31.04	0\\
31.05	0\\
31.06	0\\
31.07	0\\
31.08	0\\
31.09	0\\
31.1	0\\
31.11	0\\
31.12	0\\
31.13	0\\
31.14	0\\
31.15	0\\
31.16	0\\
31.17	0\\
31.18	0\\
31.19	0\\
31.2	0\\
31.21	0\\
31.22	0\\
31.23	0\\
31.24	0\\
31.25	0\\
31.26	0\\
31.27	0\\
31.28	0\\
31.29	0\\
31.3	0\\
31.31	0\\
31.32	0\\
31.33	0\\
31.34	0\\
31.35	0\\
31.36	0\\
31.37	0\\
31.38	0\\
31.39	0\\
31.4	0\\
31.41	0\\
31.42	0\\
31.43	0\\
31.44	0\\
31.45	0\\
31.46	0\\
31.47	0\\
31.48	0\\
31.49	0\\
31.5	0\\
31.51	0\\
31.52	0\\
31.53	0\\
31.54	0\\
31.55	0\\
31.56	0\\
31.57	0\\
31.58	0\\
31.59	0\\
31.6	0\\
31.61	0\\
31.62	0\\
31.63	0\\
31.64	0\\
31.65	0\\
31.66	0\\
31.67	0\\
31.68	0\\
31.69	0\\
31.7	0\\
31.71	0\\
31.72	0\\
31.73	0\\
31.74	0\\
31.75	0\\
31.76	0\\
31.77	0\\
31.78	0\\
31.79	0\\
31.8	0\\
31.81	0\\
31.82	0\\
31.83	0\\
31.84	0\\
31.85	0\\
31.86	0\\
31.87	0\\
31.88	0\\
31.89	0\\
31.9	0\\
31.91	0\\
31.92	0\\
31.93	0\\
31.94	0\\
31.95	0\\
31.96	0\\
31.97	0\\
31.98	0\\
31.99	0\\
32	0\\
32.01	0\\
32.02	0\\
32.03	0\\
32.04	0\\
32.05	0\\
32.06	0\\
32.07	0\\
32.08	0\\
32.09	0\\
32.1	0\\
32.11	3.53033933628288e-07\\
32.12	9.03672751115248e-07\\
32.13	1.45457018996265e-06\\
32.14	2.00572640271512e-06\\
32.15	2.55714154200753e-06\\
32.16	3.10881576057187e-06\\
32.17	3.66074921123727e-06\\
32.18	4.21294204692307e-06\\
32.19	4.76539442065271e-06\\
32.2	5.31810648553982e-06\\
32.21	5.87107839478823e-06\\
32.22	6.42431030169893e-06\\
32.23	6.97780235967005e-06\\
32.24	7.53155472219685e-06\\
32.25	8.08556754287176e-06\\
32.26	8.63984097537046e-06\\
32.27	9.19437517347271e-06\\
32.28	9.74917029104155e-06\\
32.29	1.0304226482051e-05\\
32.3	1.08595439005585e-05\\
32.31	1.14151227007184e-05\\
32.32	1.19709630367754e-05\\
32.33	1.25270650630782e-05\\
32.34	1.30834289340589e-05\\
32.35	1.36400548042537e-05\\
32.36	1.41969428282818e-05\\
32.37	1.47540931608597e-05\\
32.38	1.53115059568082e-05\\
32.39	1.5869181371031e-05\\
32.4	1.64271195585222e-05\\
32.41	1.698532067438e-05\\
32.42	1.75437848737928e-05\\
32.43	1.81025123120321e-05\\
32.44	1.86615031444806e-05\\
32.45	1.92207575266043e-05\\
32.46	1.97802756139592e-05\\
32.47	2.03400575621987e-05\\
32.48	2.09001035270731e-05\\
32.49	2.14604136644161e-05\\
32.5	2.20209881301656e-05\\
32.51	2.25818270803424e-05\\
32.52	2.31429306710648e-05\\
32.53	2.37042990585482e-05\\
32.54	2.4265932399084e-05\\
32.55	2.48278308490751e-05\\
32.56	2.53899945650074e-05\\
32.57	2.59524237034639e-05\\
32.58	2.6515118421111e-05\\
32.59	2.70780788747121e-05\\
32.6	2.7641305221128e-05\\
32.61	2.82047976173025e-05\\
32.62	2.87685562202766e-05\\
32.63	2.93325811871747e-05\\
32.64	2.98968726752252e-05\\
32.65	3.04614308417467e-05\\
32.66	3.1026255844134e-05\\
32.67	3.15913478398933e-05\\
32.68	3.21567069866066e-05\\
32.69	3.27223334419535e-05\\
32.7	3.32882273637106e-05\\
32.71	3.38543889097306e-05\\
32.72	3.44208182379638e-05\\
32.73	3.49875155064572e-05\\
32.74	3.55544808733413e-05\\
32.75	3.61217144968437e-05\\
32.76	3.66892165352753e-05\\
32.77	3.72569871470371e-05\\
32.78	3.78250264906274e-05\\
32.79	3.83933347246276e-05\\
32.8	3.89619120077095e-05\\
32.81	3.95307584986418e-05\\
32.82	4.00998743562697e-05\\
32.83	4.06692597395425e-05\\
32.84	4.12389148074926e-05\\
32.85	4.18088397192359e-05\\
32.86	4.23790346339922e-05\\
32.87	4.29494997110508e-05\\
32.88	4.35202351098049e-05\\
32.89	4.40912409897382e-05\\
32.9	4.46625175104104e-05\\
32.91	4.52340648314717e-05\\
32.92	4.58058831126762e-05\\
32.93	4.63779725138475e-05\\
32.94	4.69503331949062e-05\\
32.95	4.75229653158632e-05\\
32.96	4.80958690368127e-05\\
32.97	4.86690445179391e-05\\
32.98	4.924249191951e-05\\
32.99	4.98162114018902e-05\\
33	5.03902031255207e-05\\
33.01	5.09644672509399e-05\\
33.02	5.15390039387692e-05\\
33.03	5.21138133497134e-05\\
33.04	5.26888956445676e-05\\
33.05	5.32642509842099e-05\\
33.06	5.38398795296088e-05\\
33.07	5.44157814418231e-05\\
33.08	5.49919568819945e-05\\
33.09	5.55684060113415e-05\\
33.1	5.61451289911863e-05\\
33.11	5.67221259829206e-05\\
33.12	5.72993971480334e-05\\
33.13	5.78769426480968e-05\\
33.14	5.84547626447593e-05\\
33.15	5.90328572997664e-05\\
33.16	5.96112267749402e-05\\
33.17	6.01898712321997e-05\\
33.18	6.07687908335333e-05\\
33.19	6.13479857410199e-05\\
33.2	6.19274561168282e-05\\
33.21	6.25072021232104e-05\\
33.22	6.3087223922495e-05\\
33.23	6.36675216771007e-05\\
33.24	6.42480955495295e-05\\
33.25	6.48289457023665e-05\\
33.26	6.54100722982803e-05\\
33.27	6.59914755000296e-05\\
33.28	6.65731554704424e-05\\
33.29	6.7155112372437e-05\\
33.3	6.77373463690151e-05\\
33.31	6.83198576232613e-05\\
33.32	6.89026462983439e-05\\
33.33	6.94857125575071e-05\\
33.34	7.00690565640857e-05\\
33.35	7.06526784814906e-05\\
33.36	7.12365784732161e-05\\
33.37	7.18207567028395e-05\\
33.38	7.24052133340217e-05\\
33.39	7.29899485304997e-05\\
33.4	7.35749624560939e-05\\
33.41	7.41602552747009e-05\\
33.42	7.47458271503076e-05\\
33.43	7.53316782469773e-05\\
33.44	7.59178087288565e-05\\
33.45	7.65042187601608e-05\\
33.46	7.70909085051966e-05\\
33.47	7.7677878128346e-05\\
33.48	7.82651277940749e-05\\
33.49	7.88526576669252e-05\\
33.5	7.94404679115152e-05\\
33.51	8.00285586925464e-05\\
33.52	8.06169301747969e-05\\
33.53	8.12055825231278e-05\\
33.54	8.17945159024766e-05\\
33.55	8.23837304778502e-05\\
33.56	8.29732264143457e-05\\
33.57	8.35630038771365e-05\\
33.58	8.41530630314655e-05\\
33.59	8.47434040426587e-05\\
33.6	8.53340270761255e-05\\
33.61	8.59249322973377e-05\\
33.62	8.65161198718503e-05\\
33.63	8.71075899653018e-05\\
33.64	8.76993427433997e-05\\
33.65	8.82913783719283e-05\\
33.66	8.88836970167547e-05\\
33.67	8.94762988438089e-05\\
33.68	9.00691840191106e-05\\
33.69	9.06623527087425e-05\\
33.7	9.12558050788631e-05\\
33.71	9.18495412957215e-05\\
33.72	9.24435615256222e-05\\
33.73	9.30378659349529e-05\\
33.74	9.36324546901707e-05\\
33.75	9.4227327957816e-05\\
33.76	9.48224859044916e-05\\
33.77	9.54179286968837e-05\\
33.78	9.60136565017478e-05\\
33.79	9.66096694859087e-05\\
33.8	9.72059678162676e-05\\
33.81	9.78025516598022e-05\\
33.82	9.83994211835523e-05\\
33.83	9.89965765546344e-05\\
33.84	9.9594017940241e-05\\
33.85	0.000100191745507627\\
33.86	0.000100789759424132\\
33.87	0.000101388059857148\\
33.88	0.000101986646974153\\
33.89	0.000102585520942694\\
33.9	0.000103184681930386\\
33.91	0.000103784130104909\\
33.92	0.000104383865634015\\
33.93	0.000104983888685531\\
33.94	0.000105584199427348\\
33.95	0.000106184798027431\\
33.96	0.000106785684653804\\
33.97	0.000107386859474569\\
33.98	0.00010798832265789\\
33.99	0.000108590074372007\\
34	0.000109192114785217\\
34.01	0.000109794444065885\\
34.02	0.000110397062382452\\
34.03	0.000110999969903422\\
34.04	0.000111603166797362\\
34.05	0.000112206653232909\\
34.06	0.000112810429378767\\
34.07	0.000113414495403696\\
34.08	0.000114018851476536\\
34.09	0.000114623497766184\\
34.1	0.000115228434441596\\
34.11	0.000115833661671803\\
34.12	0.0001164391796259\\
34.13	0.000117044988473036\\
34.14	0.00011765108838243\\
34.15	0.000118257479523362\\
34.16	0.000118864162065184\\
34.17	0.000119471136177295\\
34.18	0.00012007840202917\\
34.19	0.00012068595979034\\
34.2	0.000121293809630392\\
34.21	0.000121901951718989\\
34.22	0.000122510386225842\\
34.23	0.000123119113320733\\
34.24	0.000123728133173497\\
34.25	0.000124337445954033\\
34.26	0.000124947051832296\\
34.27	0.000125556950978302\\
34.28	0.000126167143562131\\
34.29	0.000126777629753917\\
34.3	0.000127388409723858\\
34.31	0.000127999483642199\\
34.32	0.000128610851679256\\
34.33	0.000129222514005392\\
34.34	0.000129834470791033\\
34.35	0.000130446722206662\\
34.36	0.000131059268422815\\
34.37	0.000131672109610084\\
34.38	0.000132285245939126\\
34.39	0.000132898677580644\\
34.4	0.000133512404705396\\
34.41	0.000134126427484199\\
34.42	0.000134740746087929\\
34.43	0.000135355360687504\\
34.44	0.000135970271453907\\
34.45	0.000136585478558167\\
34.46	0.000137200982171369\\
34.47	0.000137816782464648\\
34.48	0.000138432879609199\\
34.49	0.000139049273776261\\
34.5	0.000139665965137127\\
34.51	0.000140282953863147\\
34.52	0.00014090024012571\\
34.53	0.000141517824096264\\
34.54	0.000142135705946303\\
34.55	0.000142753885847377\\
34.56	0.000143372363971078\\
34.57	0.000143991140489054\\
34.58	0.000144610215572993\\
34.59	0.000145229589394641\\
34.6	0.000145849262125783\\
34.61	0.000146469233938254\\
34.62	0.000147089505003938\\
34.63	0.000147710075494766\\
34.64	0.000148330945582711\\
34.65	0.000148952115439797\\
34.66	0.000149573585238094\\
34.67	0.000150195355149707\\
34.68	0.000150817425346797\\
34.69	0.000151439796001566\\
34.7	0.00015206246728626\\
34.71	0.000152685439373162\\
34.72	0.000153308712434609\\
34.73	0.000153932286642969\\
34.74	0.000154556162170659\\
34.75	0.000155180339190132\\
34.76	0.000155804817873896\\
34.77	0.000156429598394486\\
34.78	0.00015705468092448\\
34.79	0.000157680065636497\\
34.8	0.000158305752703196\\
34.81	0.00015893174229728\\
34.82	0.00015955803459148\\
34.83	0.000160184629758572\\
34.84	0.000160811527971372\\
34.85	0.000161438729402733\\
34.86	0.000162066234225537\\
34.87	0.000162694042612703\\
34.88	0.000163322154737196\\
34.89	0.000163950570772009\\
34.9	0.000164579290890171\\
34.91	0.000165208315264752\\
34.92	0.000165837644068842\\
34.93	0.000166467277475574\\
34.94	0.000167097215658114\\
34.95	0.000167727458789665\\
34.96	0.00016835800704345\\
34.97	0.000168988860592731\\
34.98	0.000169620019610808\\
34.99	0.000170251484270995\\
35	0.000170883254746651\\
35.01	0.00017151533121116\\
35.02	0.000172147713837931\\
35.03	0.000172780402800408\\
35.04	0.000173413398272058\\
35.05	0.000174046700426381\\
35.06	0.000174680309436896\\
35.07	0.000175314225477161\\
35.08	0.00017594844872075\\
35.09	0.000176582979341269\\
35.1	0.00017721781751235\\
35.11	0.000177852963407638\\
35.12	0.00017848841720082\\
35.13	0.000179124179065591\\
35.14	0.000179760249175674\\
35.15	0.000180396627704818\\
35.16	0.000181033314826794\\
35.17	0.000181670310715386\\
35.18	0.000182307615544414\\
35.19	0.000182945229487703\\
35.2	0.000183583152719109\\
35.21	0.000184221385412499\\
35.22	0.000184859927741768\\
35.23	0.000185498779880826\\
35.24	0.000186137942003589\\
35.25	0.000186777414284009\\
35.26	0.000187417196896043\\
35.27	0.000188057290013664\\
35.28	0.000188697693810871\\
35.29	0.000189338408461664\\
35.3	0.000189979434140064\\
35.31	0.000190620771020104\\
35.32	0.000191262419275834\\
35.33	0.000191904379081315\\
35.34	0.000192546650610616\\
35.35	0.00019318923403782\\
35.36	0.000193832129537018\\
35.37	0.00019447533728232\\
35.38	0.000195118857447843\\
35.39	0.000195762690207706\\
35.4	0.000196406835736039\\
35.41	0.000197051294206982\\
35.42	0.000197696065794679\\
35.43	0.000198341150673284\\
35.44	0.000198986549016962\\
35.45	0.000199632260999873\\
35.46	0.000200278286796182\\
35.47	0.000200924626580064\\
35.48	0.000201571280525698\\
35.49	0.000202218248807264\\
35.5	0.000202865531598942\\
35.51	0.00020351312907492\\
35.52	0.000204161041409377\\
35.53	0.000204809268776501\\
35.54	0.000205457811350472\\
35.55	0.000206106669305477\\
35.56	0.000206755842815695\\
35.57	0.000207405332055308\\
35.58	0.000208055137198487\\
35.59	0.000208705258419407\\
35.6	0.000209355695892241\\
35.61	0.000210006449791147\\
35.62	0.000210657520290279\\
35.63	0.000211308907563795\\
35.64	0.000211960611785834\\
35.65	0.000212612633130535\\
35.66	0.000213264971772029\\
35.67	0.000213917627884433\\
35.68	0.000214570601641852\\
35.69	0.000215223893218389\\
35.7	0.000215877502788127\\
35.71	0.000216531430525142\\
35.72	0.000217185676603504\\
35.73	0.000217840241197254\\
35.74	0.000218495124480432\\
35.75	0.00021915032662706\\
35.76	0.000219805847811137\\
35.77	0.000220461688206657\\
35.78	0.00022111784798759\\
35.79	0.000221774327327888\\
35.8	0.000222431126401489\\
35.81	0.000223088245382308\\
35.82	0.000223745684444242\\
35.83	0.000224403443761172\\
35.84	0.000225061523506945\\
35.85	0.000225719923855394\\
35.86	0.000226378644980332\\
35.87	0.000227037687055544\\
35.88	0.000227697050254792\\
35.89	0.000228356734751807\\
35.9	0.000229016740720303\\
35.91	0.000229677068333961\\
35.92	0.000230337717766434\\
35.93	0.000230998689191353\\
35.94	0.000231659982782316\\
35.95	0.000232321598712885\\
35.96	0.000232983537156603\\
35.97	0.000233645798286969\\
35.98	0.000234308382277462\\
35.99	0.000234971289301514\\
36	0.00023563451953254\\
36.01	0.000236298073143903\\
36.02	0.000236961950308937\\
36.03	0.000237626151200945\\
36.04	0.000238290675993187\\
36.05	0.000238955524858885\\
36.06	0.000239620697971218\\
36.07	0.000240286195503335\\
36.08	0.000240952017628339\\
36.09	0.000241618164519293\\
36.1	0.000242284636349212\\
36.11	0.000242951433291075\\
36.12	0.000243618555517806\\
36.13	0.000244286003202296\\
36.14	0.000244953776517391\\
36.15	0.000245621875635878\\
36.16	0.000246290300730499\\
36.17	0.000246959051973958\\
36.18	0.0002476281295389\\
36.19	0.000248297533597924\\
36.2	0.00024896726432358\\
36.21	0.000249637321888353\\
36.22	0.000250307706464691\\
36.23	0.000250978418224974\\
36.24	0.000251649457341539\\
36.25	0.000252320823986657\\
36.26	0.000252992518332551\\
36.27	0.000253664540551382\\
36.28	0.000254336890815249\\
36.29	0.000255009569296201\\
36.3	0.000255682576166219\\
36.31	0.000256355911597221\\
36.32	0.00025702957576107\\
36.33	0.000257703568829551\\
36.34	0.000258377890974403\\
36.35	0.000259052542367293\\
36.36	0.000259727523179813\\
36.37	0.000260402833583501\\
36.38	0.000261078473749815\\
36.39	0.000261754443850154\\
36.4	0.000262430744055839\\
36.41	0.000263107374538121\\
36.42	0.000263784335468184\\
36.43	0.000264461627017133\\
36.44	0.000265139249356006\\
36.45	0.000265817202655748\\
36.46	0.000266495487087251\\
36.47	0.000267174102821316\\
36.48	0.00026785305002866\\
36.49	0.000268532328879939\\
36.5	0.00026921193954571\\
36.51	0.000269891882196463\\
36.52	0.000270572157002595\\
36.53	0.000271252764134429\\
36.54	0.000271933703762189\\
36.55	0.000272614976056032\\
36.56	0.000273296581186008\\
36.57	0.000273978519322093\\
36.58	0.000274660790634172\\
36.59	0.000275343395292039\\
36.6	0.000276026333465393\\
36.61	0.00027670960532384\\
36.62	0.000277393211036904\\
36.63	0.000278077150773999\\
36.64	0.000278761424704455\\
36.65	0.000279446032997496\\
36.66	0.00028013097582226\\
36.67	0.000280816253347776\\
36.68	0.000281501865742981\\
36.69	0.000282187813176703\\
36.7	0.000282874095817669\\
36.71	0.000283560713834501\\
36.72	0.000284247667395723\\
36.73	0.000284934956669751\\
36.74	0.000285622581824893\\
36.75	0.000286310543029342\\
36.76	0.000286998840451191\\
36.77	0.000287687474258419\\
36.78	0.00028837644461889\\
36.79	0.000289065751700367\\
36.8	0.000289755395670485\\
36.81	0.000290445376696764\\
36.82	0.000291135694946619\\
36.83	0.000291826350587339\\
36.84	0.000292517343786089\\
36.85	0.00029320867470993\\
36.86	0.000293900343525784\\
36.87	0.000294592350400462\\
36.88	0.000295284695500643\\
36.89	0.000295977378992882\\
36.9	0.000296670401043614\\
36.91	0.000297363761819144\\
36.92	0.000298057461485646\\
36.93	0.000298751500209159\\
36.94	0.000299445878155599\\
36.95	0.000300140595490744\\
36.96	0.000300835652380239\\
36.97	0.000301531048989598\\
36.98	0.000302226785484187\\
36.99	0.000302922862029244\\
37	0.000303619278789864\\
37.01	0.000304316035931007\\
37.02	0.000305013133617478\\
37.03	0.000305710572013944\\
37.04	0.00030640835128494\\
37.05	0.000307106471594834\\
37.06	0.000307804933107864\\
37.07	0.000308503735988105\\
37.08	0.000309202880399491\\
37.09	0.000309902366505806\\
37.1	0.000310602194470676\\
37.11	0.000311302364457568\\
37.12	0.000312002876629809\\
37.13	0.000312703731150554\\
37.14	0.000313404928182805\\
37.15	0.000314106467889402\\
37.16	0.000314808350433023\\
37.17	0.000315510575976188\\
37.18	0.000316213144681249\\
37.19	0.000316916056710401\\
37.2	0.000317619312225653\\
37.21	0.000318322911388863\\
37.22	0.000319026854361713\\
37.23	0.000319731141305714\\
37.24	0.000320435772382201\\
37.25	0.000321140747752337\\
37.26	0.000321846067577111\\
37.27	0.000322551732017326\\
37.28	0.000323257741233623\\
37.29	0.000323964095386445\\
37.3	0.000324670794636064\\
37.31	0.000325377839142563\\
37.32	0.000326085229065839\\
37.33	0.000326792964565607\\
37.34	0.000327501045801398\\
37.35	0.000328209472932543\\
37.36	0.000328918246118186\\
37.37	0.000329627365517275\\
37.38	0.000330336831288573\\
37.39	0.000331046643590641\\
37.4	0.000331756802581844\\
37.41	0.000332467308420348\\
37.42	0.00033317816126411\\
37.43	0.000333889361270899\\
37.44	0.00033460090859827\\
37.45	0.000335312803403577\\
37.46	0.000336025045843971\\
37.47	0.000336737636076383\\
37.48	0.00033745057425754\\
37.49	0.000338163860543958\\
37.5	0.000338877495091934\\
37.51	0.000339591478057559\\
37.52	0.000340305809596701\\
37.53	0.000341020489865007\\
37.54	0.000341735519017908\\
37.55	0.000342450897210612\\
37.56	0.000343166624598101\\
37.57	0.000343882701335133\\
37.58	0.000344599127576235\\
37.59	0.000345315903475717\\
37.6	0.000346033029187648\\
37.61	0.000346750504865864\\
37.62	0.000347468330663972\\
37.63	0.000348186506735335\\
37.64	0.000348905033233086\\
37.65	0.000349623910310119\\
37.66	0.000350343138119089\\
37.67	0.000351062716812393\\
37.68	0.000351782646542197\\
37.69	0.000352502927460419\\
37.7	0.000353223559718722\\
37.71	0.000353944543468529\\
37.72	0.000354665878861002\\
37.73	0.000355387566047055\\
37.74	0.000356109605177339\\
37.75	0.000356831996402254\\
37.76	0.000357554739871938\\
37.77	0.000358277835736262\\
37.78	0.000359001284144844\\
37.79	0.00035972508524703\\
37.8	0.000360449239191908\\
37.81	0.000361173746128284\\
37.82	0.000361898606204697\\
37.83	0.000362623819569426\\
37.84	0.000363349386370455\\
37.85	0.000364075306755508\\
37.86	0.000364801580872015\\
37.87	0.000365528208867143\\
37.88	0.000366255190887767\\
37.89	0.000366982527080471\\
37.9	0.00036771021759157\\
37.91	0.000368438262567071\\
37.92	0.000369166662152705\\
37.93	0.000369895416493904\\
37.94	0.000370624525735809\\
37.95	0.000371353990023263\\
37.96	0.000372083809500802\\
37.97	0.00037281398431268\\
37.98	0.000373544514602829\\
37.99	0.000374275400514892\\
38	0.000375006642192192\\
38.01	0.000375738239777754\\
38.02	0.000376470193414281\\
38.03	0.000377202503244176\\
38.04	0.000377935169409517\\
38.05	0.000378668192052074\\
38.06	0.000379401571313283\\
38.07	0.000380135307334273\\
38.08	0.000380869400255841\\
38.09	0.000381603850218459\\
38.1	0.000382338657362272\\
38.11	0.000383073821827097\\
38.12	0.000383809343752421\\
38.13	0.00038454522327739\\
38.14	0.000385281460540815\\
38.15	0.000386018055681163\\
38.16	0.000386755008836573\\
38.17	0.00038749232014483\\
38.18	0.000388229989743381\\
38.19	0.000388968017769317\\
38.2	0.000389706404359383\\
38.21	0.000390445149649969\\
38.22	0.000391184253777113\\
38.23	0.000391923716876491\\
38.24	0.000392663539083432\\
38.25	0.000393403720532891\\
38.26	0.000394144261359461\\
38.27	0.000394885161697375\\
38.28	0.000395626421680492\\
38.29	0.000396368041442295\\
38.3	0.000397110021115907\\
38.31	0.000397852360834068\\
38.32	0.000398595060729139\\
38.33	0.000399338120933103\\
38.34	0.000400081541577557\\
38.35	0.000400825322793709\\
38.36	0.000401569464712391\\
38.37	0.000402313967464034\\
38.38	0.000403058831178686\\
38.39	0.000403804055985994\\
38.4	0.000404549642015202\\
38.41	0.000405295589395165\\
38.42	0.000406041898254328\\
38.43	0.000406788568720735\\
38.44	0.000407535600922021\\
38.45	0.000408282994985409\\
38.46	0.000409030751037709\\
38.47	0.000409778869205318\\
38.48	0.00041052734961422\\
38.49	0.000411276192389974\\
38.5	0.000412025397657709\\
38.51	0.000412774965542144\\
38.52	0.000413524896167555\\
38.53	0.000414275189657795\\
38.54	0.000415025846136285\\
38.55	0.000415776865726004\\
38.56	0.0004165282485495\\
38.57	0.000417279994728875\\
38.58	0.000418032104385783\\
38.59	0.00041878457764144\\
38.6	0.000419537414616609\\
38.61	0.000420290615431605\\
38.62	0.000421044180206276\\
38.63	0.000421798109060026\\
38.64	0.000422552402111795\\
38.65	0.000423307059480051\\
38.66	0.00042406208128281\\
38.67	0.000424817467637617\\
38.68	0.000425573218661532\\
38.69	0.000426329334471162\\
38.7	0.000427085815182623\\
38.71	0.000427842660911552\\
38.72	0.000428599871773107\\
38.73	0.000429357447881959\\
38.74	0.000430115389352295\\
38.75	0.000430873696297802\\
38.76	0.000431632368831682\\
38.77	0.000432391407066635\\
38.78	0.000433150811114863\\
38.79	0.000433910581088069\\
38.8	0.000434670717097442\\
38.81	0.000435431219253671\\
38.82	0.000436192087666924\\
38.83	0.000436953322446865\\
38.84	0.000437714923702627\\
38.85	0.00043847689154284\\
38.86	0.000439239226075597\\
38.87	0.000440001927408465\\
38.88	0.00044076499564849\\
38.89	0.000441528430902179\\
38.9	0.0004422922332755\\
38.91	0.000443056402873891\\
38.92	0.000443820939802241\\
38.93	0.000444585844164902\\
38.94	0.000445351116065668\\
38.95	0.000446116755607788\\
38.96	0.000446882762893952\\
38.97	0.0004476491380263\\
38.98	0.000448415881106404\\
38.99	0.000449182992235272\\
39	0.000449950471513354\\
39.01	0.000450718319040512\\
39.02	0.000451486534916057\\
39.03	0.000452255119238708\\
39.04	0.000453024072106609\\
39.05	0.00045379339361732\\
39.06	0.000454563083867814\\
39.07	0.00045533314295447\\
39.08	0.000456103570973081\\
39.09	0.000456874368018841\\
39.1	0.000457645534186341\\
39.11	0.000458417069569568\\
39.12	0.000459188974261913\\
39.13	0.00045996124835615\\
39.14	0.000460733891944433\\
39.15	0.000461506905118307\\
39.16	0.0004622802879687\\
39.17	0.000463054040585907\\
39.18	0.000463828163059607\\
39.19	0.000464602655478832\\
39.2	0.000465377517931999\\
39.21	0.000466152750506876\\
39.22	0.00046692835329059\\
39.23	0.000467704326369625\\
39.24	0.000468480669829824\\
39.25	0.000469257383756368\\
39.26	0.000470034468233786\\
39.27	0.00047081192334595\\
39.28	0.000471589749176063\\
39.29	0.000472367945806677\\
39.3	0.000473146513319658\\
39.31	0.000473925451796205\\
39.32	0.000474704761316842\\
39.33	0.000475484442007325\\
39.34	0.000476264494117307\\
39.35	0.00047704491789672\\
39.36	0.000477825713595781\\
39.37	0.000478606881464984\\
39.38	0.000479388421755114\\
39.39	0.000480170334717235\\
39.4	0.000480952620602701\\
39.41	0.000481735279663144\\
39.42	0.00048251831215048\\
39.43	0.000483301718316918\\
39.44	0.00048408549841495\\
39.45	0.000484869652697353\\
39.46	0.000485654181417196\\
39.47	0.00048643908482783\\
39.48	0.000487224363182895\\
39.49	0.000488010016736318\\
39.5	0.000488796045742312\\
39.51	0.000489582450455391\\
39.52	0.000490369231130348\\
39.53	0.000491156388022267\\
39.54	0.000491943921386531\\
39.55	0.000492731831478806\\
39.56	0.000493520118555052\\
39.57	0.000494308782871516\\
39.58	0.000495097824684751\\
39.59	0.000495887244251585\\
39.6	0.000496677041829159\\
39.61	0.000497467217674886\\
39.62	0.000498257772046491\\
39.63	0.000499048705201989\\
39.64	0.000499840017399687\\
39.65	0.000500631708898192\\
39.66	0.000501423779956399\\
39.67	0.000502216230833505\\
39.68	0.00050300906178901\\
39.69	0.000503802273082707\\
39.7	0.000504595864974679\\
39.71	0.000505389837725322\\
39.72	0.000506184191595316\\
39.73	0.000506978926845654\\
39.74	0.00050777404373762\\
39.75	0.000508569542532797\\
39.76	0.000509365423493079\\
39.77	0.000510161686880652\\
39.78	0.000510958332958014\\
39.79	0.000511755361987949\\
39.8	0.000512552774233559\\
39.81	0.000513350569958243\\
39.82	0.000514148749425701\\
39.83	0.000514947312899944\\
39.84	0.000515746260645281\\
39.85	0.000516545592926333\\
39.86	0.000517345310008022\\
39.87	0.000518145412155578\\
39.88	0.000518945899634533\\
39.89	0.00051974677271073\\
39.9	0.000520548031650317\\
39.91	0.000521349676719754\\
39.92	0.000522151708185807\\
39.93	0.000522954126315554\\
39.94	0.000523756931376379\\
39.95	0.000524560123635977\\
39.96	0.000525363703362348\\
39.97	0.000526167670823814\\
39.98	0.000526972026288999\\
39.99	0.000527776770026842\\
40	0.000528581902306598\\
40.01	0.00052938742339783\\
};
\addplot [color=mycolor1,solid,forget plot]
  table[row sep=crcr]{%
40.01	0.00052938742339783\\
40.02	0.000530193333570413\\
40.03	0.000530999633094539\\
40.04	0.000531806322240716\\
40.05	0.000532613401279768\\
40.06	0.000533420870482824\\
40.07	0.000534228730121343\\
40.08	0.000535036980467085\\
40.09	0.000535845621792139\\
40.1	0.000536654654368904\\
40.11	0.000537464078470107\\
40.12	0.000538273894368779\\
40.13	0.000539084102338279\\
40.14	0.000539894702652276\\
40.15	0.000540705695584773\\
40.16	0.000541517081410085\\
40.17	0.000542328860402848\\
40.18	0.000543141032838014\\
40.19	0.000543953598990871\\
40.2	0.00054476655913701\\
40.21	0.000545579913552363\\
40.22	0.000546393662513175\\
40.23	0.00054720780629601\\
40.24	0.00054802234517777\\
40.25	0.000548837279435672\\
40.26	0.000549652609347265\\
40.27	0.000550468335190415\\
40.28	0.000551284457243319\\
40.29	0.000552100975784495\\
40.3	0.000552917891092797\\
40.31	0.000553735203447404\\
40.32	0.000554552913127823\\
40.33	0.000555371020413885\\
40.34	0.000556189525585754\\
40.35	0.000557008428923929\\
40.36	0.000557827730709226\\
40.37	0.000558647431222803\\
40.38	0.000559467530746142\\
40.39	0.000560288029561061\\
40.4	0.000561108927949715\\
40.41	0.000561930226194579\\
40.42	0.00056275192457847\\
40.43	0.000563574023384535\\
40.44	0.000564396522896256\\
40.45	0.000565219423397455\\
40.46	0.000566042725172279\\
40.47	0.000566866428505224\\
40.48	0.000567690533681109\\
40.49	0.000568515040985096\\
40.5	0.000569339950702685\\
40.51	0.000570165263119711\\
40.52	0.000570990978522347\\
40.53	0.000571817097197107\\
40.54	0.000572643619430845\\
40.55	0.000573470545510754\\
40.56	0.000574297875724369\\
40.57	0.000575125610359564\\
40.58	0.000575953749704546\\
40.59	0.000576782294047881\\
40.6	0.000577611243678458\\
40.61	0.000578440598885531\\
40.62	0.000579270359958674\\
40.63	0.000580100527187821\\
40.64	0.000580931100863248\\
40.65	0.000581762081275576\\
40.66	0.000582593468715767\\
40.67	0.000583425263475129\\
40.68	0.000584257465845318\\
40.69	0.000585090076118337\\
40.7	0.000585923094586542\\
40.71	0.000586756521542629\\
40.72	0.000587590357279642\\
40.73	0.000588424602090978\\
40.74	0.000589259256270361\\
40.75	0.00059009432011186\\
40.76	0.000590929793909886\\
40.77	0.00059176567795919\\
40.78	0.000592601972554854\\
40.79	0.00059343867799231\\
40.8	0.000594275794567334\\
40.81	0.000595113322576044\\
40.82	0.000595951262314896\\
40.83	0.000596789614080696\\
40.84	0.000597628378170595\\
40.85	0.000598467554882083\\
40.86	0.000599307144512992\\
40.87	0.000600147147361514\\
40.88	0.000600987563726181\\
40.89	0.000601828393905872\\
40.9	0.000602669638199813\\
40.91	0.000603511296907576\\
40.92	0.000604353370329089\\
40.93	0.000605195858764625\\
40.94	0.000606038762514811\\
40.95	0.000606882081880623\\
40.96	0.000607725817163382\\
40.97	0.000608569968664777\\
40.98	0.000609414536686829\\
40.99	0.000610259521531931\\
41	0.000611104923502817\\
41.01	0.000611950742902587\\
41.02	0.000612796980034684\\
41.03	0.000613643635202908\\
41.04	0.000614490708711425\\
41.05	0.000615338200864748\\
41.06	0.000616186111967759\\
41.07	0.000617034442325685\\
41.08	0.000617883192244116\\
41.09	0.000618732362029001\\
41.1	0.000619581951986657\\
41.11	0.000620431962423749\\
41.12	0.00062128239364731\\
41.13	0.000622133245964732\\
41.14	0.000622984519683775\\
41.15	0.000623836215112555\\
41.16	0.000624688332559548\\
41.17	0.000625540872333609\\
41.18	0.000626393834743944\\
41.19	0.00062724722010013\\
41.2	0.000628101028712115\\
41.21	0.000628955260890199\\
41.22	0.000629809916945064\\
41.23	0.000630664997187752\\
41.24	0.000631520501929675\\
41.25	0.000632376431482617\\
41.26	0.000633232786158726\\
41.27	0.000634089566270521\\
41.28	0.000634946772130898\\
41.29	0.000635804404053128\\
41.3	0.000636662462350841\\
41.31	0.000637520947338044\\
41.32	0.000638379859329123\\
41.33	0.000639239198638834\\
41.34	0.000640098965582307\\
41.35	0.000640959160475053\\
41.36	0.000641819783632952\\
41.37	0.000642680835372264\\
41.38	0.000643542316009632\\
41.39	0.000644404225862066\\
41.4	0.000645266565246966\\
41.41	0.000646129334482097\\
41.42	0.000646992533885615\\
41.43	0.000647856163776057\\
41.44	0.000648720224472335\\
41.45	0.000649584716293748\\
41.46	0.000650449639559979\\
41.47	0.00065131499459109\\
41.48	0.00065218078170752\\
41.49	0.00065304700123011\\
41.5	0.000653913653480068\\
41.51	0.000654780738779004\\
41.52	0.000655648257448904\\
41.53	0.000656516209812143\\
41.54	0.000657384596191482\\
41.55	0.00065825341691008\\
41.56	0.000659122672291478\\
41.57	0.000659992362659602\\
41.58	0.000660862488338777\\
41.59	0.000661733049653723\\
41.6	0.000662604046929534\\
41.61	0.000663475480491713\\
41.62	0.000664347350666152\\
41.63	0.000665219657779138\\
41.64	0.000666092402157349\\
41.65	0.000666965584127856\\
41.66	0.000667839204018132\\
41.67	0.000668713262156048\\
41.68	0.000669587758869862\\
41.69	0.000670462694488241\\
41.7	0.000671338069340249\\
41.71	0.000672213883755345\\
41.72	0.000673090138063388\\
41.73	0.000673966832594637\\
41.74	0.000674843967679764\\
41.75	0.000675721543649831\\
41.76	0.000676599560836301\\
41.77	0.000677478019571055\\
41.78	0.000678356920186363\\
41.79	0.000679236263014908\\
41.8	0.00068011604838978\\
41.81	0.000680996276644467\\
41.82	0.000681876948112878\\
41.83	0.000682758063129313\\
41.84	0.000683639622028491\\
41.85	0.00068452162514554\\
41.86	0.000685404072815997\\
41.87	0.000686286965375808\\
41.88	0.00068717030316133\\
41.89	0.000688054086509335\\
41.9	0.000688938315757004\\
41.91	0.000689822991241935\\
41.92	0.000690708113302144\\
41.93	0.000691593682276052\\
41.94	0.000692479698502502\\
41.95	0.000693366162320755\\
41.96	0.000694253074070478\\
41.97	0.000695140434091773\\
41.98	0.000696028242725154\\
41.99	0.000696916500311541\\
42	0.000697805207192294\\
42.01	0.000698694363709187\\
42.02	0.000699583970204413\\
42.03	0.000700474027020587\\
42.04	0.000701364534500745\\
42.05	0.000702255492988357\\
42.06	0.000703146902827308\\
42.07	0.000704038764361911\\
42.08	0.000704931077936906\\
42.09	0.00070582384389746\\
42.1	0.000706717062589166\\
42.11	0.000707610734358051\\
42.12	0.000708504859550561\\
42.13	0.000709399438513579\\
42.14	0.000710294471594416\\
42.15	0.000711189959140816\\
42.16	0.000712085901500958\\
42.17	0.000712982299023446\\
42.18	0.000713879152057326\\
42.19	0.000714776460952068\\
42.2	0.000715674226057589\\
42.21	0.000716572447724238\\
42.22	0.000717471126302799\\
42.23	0.00071837026214449\\
42.24	0.000719269855600974\\
42.25	0.000720169907024351\\
42.26	0.00072107041676716\\
42.27	0.000721971385182374\\
42.28	0.000722872812623426\\
42.29	0.000723774699444173\\
42.3	0.00072467704599892\\
42.31	0.000725579852642426\\
42.32	0.000726483119729879\\
42.33	0.000727386847616918\\
42.34	0.000728291036659633\\
42.35	0.000729195687214558\\
42.36	0.000730100799638679\\
42.37	0.000731006374289418\\
42.38	0.000731912411524663\\
42.39	0.000732818911702737\\
42.4	0.000733725875182417\\
42.41	0.000734633302322943\\
42.42	0.000735541193483999\\
42.43	0.000736449549025721\\
42.44	0.000737358369308702\\
42.45	0.000738267654693987\\
42.46	0.000739177405543086\\
42.47	0.000740087622217951\\
42.48	0.000740998305081009\\
42.49	0.000741909454495128\\
42.5	0.00074282107082365\\
42.51	0.000743733154430368\\
42.52	0.000744645705679531\\
42.53	0.000745558724935862\\
42.54	0.00074647221256454\\
42.55	0.000747386168931212\\
42.56	0.000748300594401979\\
42.57	0.000749215489343417\\
42.58	0.000750130854122566\\
42.59	0.00075104668910693\\
42.6	0.000751962994664478\\
42.61	0.000752879771163652\\
42.62	0.000753797018973365\\
42.63	0.000754714738462989\\
42.64	0.00075563293000238\\
42.65	0.00075655159396186\\
42.66	0.000757470730712223\\
42.67	0.000758390340624734\\
42.68	0.000759310424071145\\
42.69	0.000760230981423665\\
42.7	0.000761152013054989\\
42.71	0.00076207351933829\\
42.72	0.000762995500647214\\
42.73	0.000763917957355885\\
42.74	0.000764840889838914\\
42.75	0.000765764298471389\\
42.76	0.000766688183628877\\
42.77	0.000767612545687427\\
42.78	0.000768537385023568\\
42.79	0.00076946270201432\\
42.8	0.000770388497037186\\
42.81	0.000771314770470148\\
42.82	0.000772241522691682\\
42.83	0.000773168754080747\\
42.84	0.000774096465016798\\
42.85	0.000775024655879766\\
42.86	0.000775953327050084\\
42.87	0.000776882478908669\\
42.88	0.000777812111836938\\
42.89	0.000778742226216787\\
42.9	0.000779672822430619\\
42.91	0.000780603900861321\\
42.92	0.000781535461892281\\
42.93	0.000782467505907387\\
42.94	0.000783400033291019\\
42.95	0.00078433304442805\\
42.96	0.000785266539703865\\
42.97	0.000786200519504343\\
42.98	0.000787134984215856\\
42.99	0.00078806993422529\\
43	0.000789005369920028\\
43.01	0.000789941291687961\\
43.02	0.000790877699917474\\
43.03	0.00079181459499747\\
43.04	0.000792751977317346\\
43.05	0.000793689847267021\\
43.06	0.000794628205236916\\
43.07	0.000795567051617957\\
43.08	0.000796506386801579\\
43.09	0.000797446211179734\\
43.1	0.000798386525144891\\
43.11	0.000799327329090022\\
43.12	0.000800268623408609\\
43.13	0.000801210408494665\\
43.14	0.000802152684742707\\
43.15	0.000803095452547774\\
43.16	0.000804038712305423\\
43.17	0.000804982464411727\\
43.18	0.000805926709263272\\
43.19	0.000806871447257176\\
43.2	0.000807816678791082\\
43.21	0.000808762404263141\\
43.22	0.000809708624072036\\
43.23	0.000810655338616975\\
43.24	0.000811602548297687\\
43.25	0.000812550253514434\\
43.26	0.000813498454668\\
43.27	0.000814447152159697\\
43.28	0.000815396346391377\\
43.29	0.000816346037765404\\
43.3	0.000817296226684694\\
43.31	0.00081824691355268\\
43.32	0.00081919809877333\\
43.33	0.000820149782751155\\
43.34	0.000821101965891191\\
43.35	0.000822054648599017\\
43.36	0.000823007831280753\\
43.37	0.000823961514343052\\
43.38	0.000824915698193103\\
43.39	0.000825870383238642\\
43.4	0.000826825569887939\\
43.41	0.000827781258549813\\
43.42	0.000828737449633625\\
43.43	0.000829694143549282\\
43.44	0.000830651340707235\\
43.45	0.00083160904151848\\
43.46	0.000832567246394557\\
43.47	0.000833525955747566\\
43.48	0.000834485169990143\\
43.49	0.000835444889535485\\
43.5	0.000836405114797337\\
43.51	0.000837365846189993\\
43.52	0.000838327084128308\\
43.53	0.000839288829027687\\
43.54	0.000840251081304096\\
43.55	0.000841213841374047\\
43.56	0.000842177109654618\\
43.57	0.000843140886563447\\
43.58	0.000844105172518726\\
43.59	0.000845069967939212\\
43.6	0.000846035273244228\\
43.61	0.000847001088853654\\
43.62	0.000847967415187931\\
43.63	0.000848934252668077\\
43.64	0.00084990160171567\\
43.65	0.000850869462752854\\
43.66	0.000851837836202338\\
43.67	0.000852806722487408\\
43.68	0.00085377612203192\\
43.69	0.000854746035260305\\
43.7	0.00085571646259755\\
43.71	0.000856687404469238\\
43.72	0.000857658861301513\\
43.73	0.000858630833521098\\
43.74	0.00085960332155529\\
43.75	0.000860576325831974\\
43.76	0.000861549846779608\\
43.77	0.000862523884827229\\
43.78	0.00086349844040446\\
43.79	0.000864473513941505\\
43.8	0.000865449105869151\\
43.81	0.000866425216618769\\
43.82	0.000867401846622318\\
43.83	0.000868378996312341\\
43.84	0.000869356666121972\\
43.85	0.000870334856484939\\
43.86	0.000871313567835548\\
43.87	0.000872292800608708\\
43.88	0.000873272555239918\\
43.89	0.000874252832165273\\
43.9	0.000875233631821452\\
43.91	0.000876214954645745\\
43.92	0.000877196801076029\\
43.93	0.000878179171550787\\
43.94	0.0008791620665091\\
43.95	0.000880145486390642\\
43.96	0.000881129431635694\\
43.97	0.000882113902685139\\
43.98	0.000883098899980471\\
43.99	0.000884084423963781\\
44	0.00088507047507777\\
44.01	0.000886057053765751\\
44.02	0.000887044160471633\\
44.03	0.000888031795639947\\
44.04	0.000889019959715832\\
44.05	0.000890008653145043\\
44.06	0.000890997876373936\\
44.07	0.000891987629849503\\
44.08	0.000892977914019329\\
44.09	0.000893968729331632\\
44.1	0.000894960076235234\\
44.11	0.000895951955179594\\
44.12	0.000896944366614784\\
44.13	0.000897937310991491\\
44.14	0.000898930788761035\\
44.15	0.000899924800375353\\
44.16	0.00090091934628702\\
44.17	0.000901914426949223\\
44.18	0.000902910042815779\\
44.19	0.000903906194341143\\
44.2	0.000904902881980391\\
44.21	0.000905900106189239\\
44.22	0.000906897867424032\\
44.23	0.000907896166141751\\
44.24	0.000908895002800003\\
44.25	0.000909894377857046\\
44.26	0.000910894291771762\\
44.27	0.000911894745003687\\
44.28	0.000912895738012981\\
44.29	0.000913897271260455\\
44.3	0.000914899345207568\\
44.31	0.000915901960316406\\
44.32	0.00091690511704972\\
44.33	0.000917908815870894\\
44.34	0.00091891305724396\\
44.35	0.000919917841633609\\
44.36	0.000920923169505171\\
44.37	0.000921929041324633\\
44.38	0.000922935457558637\\
44.39	0.000923942418674481\\
44.4	0.000924949925140105\\
44.41	0.000925957977424119\\
44.42	0.000926966575995787\\
44.43	0.000927975721325032\\
44.44	0.000928985413882442\\
44.45	0.00092999565413926\\
44.46	0.000931006442567391\\
44.47	0.000932017779639409\\
44.48	0.000933029665828554\\
44.49	0.000934042101608731\\
44.5	0.00093505508745452\\
44.51	0.000936068623841158\\
44.52	0.000937082711244562\\
44.53	0.000938097350141318\\
44.54	0.000939112541008688\\
44.55	0.000940128284324611\\
44.56	0.000941144580567696\\
44.57	0.000942161430217234\\
44.58	0.000943178833753194\\
44.59	0.000944196791656228\\
44.6	0.000945215304407658\\
44.61	0.000946234372489509\\
44.62	0.00094725399638447\\
44.63	0.000948274176575926\\
44.64	0.000949294913547956\\
44.65	0.000950316207785309\\
44.66	0.000951338059773439\\
44.67	0.000952360469998491\\
44.68	0.00095338343894729\\
44.69	0.000954406967107364\\
44.7	0.00095543105496694\\
44.71	0.000956455703014933\\
44.72	0.000957480911740957\\
44.73	0.000958506681635329\\
44.74	0.000959533013189073\\
44.75	0.0009605599068939\\
44.76	0.000961587363242235\\
44.77	0.000962615382727204\\
44.78	0.000963643965842642\\
44.79	0.000964673113083091\\
44.8	0.0009657028249438\\
44.81	0.000966733101920733\\
44.82	0.000967763944510563\\
44.83	0.000968795353210677\\
44.84	0.000969827328519177\\
44.85	0.000970859870934877\\
44.86	0.000971892980957317\\
44.87	0.000972926659086756\\
44.88	0.000973960905824162\\
44.89	0.000974995721671236\\
44.9	0.000976031107130397\\
44.91	0.000977067062704791\\
44.92	0.000978103588898292\\
44.93	0.000979140686215504\\
44.94	0.000980178355161752\\
44.95	0.000981216596243097\\
44.96	0.000982255409966326\\
44.97	0.000983294796838972\\
44.98	0.000984334757369296\\
44.99	0.000985375292066291\\
45	0.000986416401439698\\
45.01	0.000987458085999988\\
45.02	0.000988500346258379\\
45.03	0.000989543182726829\\
45.04	0.000990586595918046\\
45.05	0.000991630586345479\\
45.06	0.000992675154523316\\
45.07	0.000993720300966512\\
45.08	0.000994766026190752\\
45.09	0.000995812330712484\\
45.1	0.000996859215048912\\
45.11	0.00099790667971799\\
45.12	0.000998954725238421\\
45.13	0.00100000335212968\\
45.14	0.00100105256091199\\
45.15	0.00100210235210633\\
45.16	0.00100315272623448\\
45.17	0.00100420368381892\\
45.18	0.00100525522538294\\
45.19	0.0010063073514506\\
45.2	0.00100736006254669\\
45.21	0.00100841335919681\\
45.22	0.00100946724192732\\
45.23	0.00101052171126534\\
45.24	0.00101157676773878\\
45.25	0.00101263241187631\\
45.26	0.00101368864420739\\
45.27	0.00101474546526228\\
45.28	0.00101580287557197\\
45.29	0.00101686087566828\\
45.3	0.00101791946608378\\
45.31	0.00101897864735186\\
45.32	0.00102003842000666\\
45.33	0.00102109878458314\\
45.34	0.00102215974161703\\
45.35	0.00102322129164487\\
45.36	0.00102428343520398\\
45.37	0.00102534617283249\\
45.38	0.0010264095050693\\
45.39	0.00102747343245414\\
45.4	0.00102853795552752\\
45.41	0.00102960307483076\\
45.42	0.00103066879090599\\
45.43	0.00103173510429612\\
45.44	0.00103280201554489\\
45.45	0.00103386952519685\\
45.46	0.00103493763379735\\
45.47	0.00103600634189256\\
45.48	0.00103707565002945\\
45.49	0.00103814555875582\\
45.5	0.00103921606862029\\
45.51	0.00104028718017226\\
45.52	0.00104135889396201\\
45.53	0.0010424312105406\\
45.54	0.00104350413045992\\
45.55	0.0010445776542727\\
45.56	0.00104565178253249\\
45.57	0.00104672651579366\\
45.58	0.00104780185461142\\
45.59	0.00104887779954182\\
45.6	0.00104995435114171\\
45.61	0.00105103150996882\\
45.62	0.00105210927658168\\
45.63	0.00105318765153969\\
45.64	0.00105426663540308\\
45.65	0.0010553462287329\\
45.66	0.00105642643209108\\
45.67	0.00105750724604037\\
45.68	0.00105858867114438\\
45.69	0.00105967070796757\\
45.7	0.00106075335707524\\
45.71	0.00106183661903356\\
45.72	0.00106292049440954\\
45.73	0.00106400498377105\\
45.74	0.00106509008768683\\
45.75	0.00106617580672648\\
45.76	0.00106726214146043\\
45.77	0.00106834909246002\\
45.78	0.00106943666029741\\
45.79	0.00107052484554566\\
45.8	0.0010716136487787\\
45.81	0.00107270307057131\\
45.82	0.00107379311149915\\
45.83	0.00107488377213877\\
45.84	0.00107597505306757\\
45.85	0.00107706695486385\\
45.86	0.00107815947810678\\
45.87	0.00107925262337642\\
45.88	0.0010803463912537\\
45.89	0.00108144078232045\\
45.9	0.00108253579715938\\
45.91	0.00108363143635409\\
45.92	0.00108472770048907\\
45.93	0.00108582459014971\\
45.94	0.0010869221059223\\
45.95	0.00108802024839399\\
45.96	0.00108911901815288\\
45.97	0.00109021841578794\\
45.98	0.00109131844188904\\
45.99	0.00109241909704697\\
46	0.00109352038185342\\
46.01	0.00109462229690098\\
46.02	0.00109572484278317\\
46.03	0.0010968280200944\\
46.04	0.00109793182943001\\
46.05	0.00109903627138624\\
46.06	0.00110014134656027\\
46.07	0.00110124705555018\\
46.08	0.00110235339895497\\
46.09	0.00110346037737459\\
46.1	0.00110456799140989\\
46.11	0.00110567624166264\\
46.12	0.00110678512873558\\
46.13	0.00110789465323233\\
46.14	0.0011090048157575\\
46.15	0.00111011561691658\\
46.16	0.00111122705731603\\
46.17	0.00111233913756325\\
46.18	0.00111345185826656\\
46.19	0.00111456522003524\\
46.2	0.00111567922347951\\
46.21	0.00111679386921055\\
46.22	0.00111790915784046\\
46.23	0.00111902508998232\\
46.24	0.00112014166625016\\
46.25	0.00112125888725894\\
46.26	0.0011223767536246\\
46.27	0.00112349526596404\\
46.28	0.00112461442489512\\
46.29	0.00112573423103666\\
46.3	0.00112685468500844\\
46.31	0.00112797578743121\\
46.32	0.00112909753892669\\
46.33	0.00113021994011758\\
46.34	0.00113134299162756\\
46.35	0.00113246669408126\\
46.36	0.0011335910481043\\
46.37	0.0011347160543233\\
46.38	0.00113584171336582\\
46.39	0.00113696802586045\\
46.4	0.00113809499243673\\
46.41	0.00113922261372521\\
46.42	0.00114035089035743\\
46.43	0.00114147982296591\\
46.44	0.00114260941218417\\
46.45	0.00114373965864674\\
46.46	0.00114487056298913\\
46.47	0.00114600212584785\\
46.48	0.00114713434786044\\
46.49	0.00114826722966541\\
46.5	0.00114940077190231\\
46.51	0.00115053497521168\\
46.52	0.00115166984023508\\
46.53	0.00115280536761508\\
46.54	0.00115394155799528\\
46.55	0.00115507841202027\\
46.56	0.00115621593033569\\
46.57	0.00115735411358818\\
46.58	0.00115849296242542\\
46.59	0.00115963247749612\\
46.6	0.00116077265945001\\
46.61	0.00116191350893784\\
46.62	0.00116305502661144\\
46.63	0.00116419721312362\\
46.64	0.00116534006912827\\
46.65	0.00116648359528029\\
46.66	0.00116762779223564\\
46.67	0.00116877266065134\\
46.68	0.00116991820118542\\
46.69	0.00117106441449699\\
46.7	0.0011722113012462\\
46.71	0.00117335886209426\\
46.72	0.00117450709770343\\
46.73	0.00117565600873704\\
46.74	0.00117680559585947\\
46.75	0.00117795585973616\\
46.76	0.00117910680103363\\
46.77	0.00118025842041946\\
46.78	0.00118141071856231\\
46.79	0.00118256369613189\\
46.8	0.00118371735379901\\
46.81	0.00118487169223555\\
46.82	0.00118602671211446\\
46.83	0.00118718241410979\\
46.84	0.00118833879889665\\
46.85	0.00118949586715127\\
46.86	0.00119065361955092\\
46.87	0.00119181205677402\\
46.88	0.00119297117950005\\
46.89	0.0011941309884096\\
46.9	0.00119529148418433\\
46.91	0.00119645266750704\\
46.92	0.00119761453906162\\
46.93	0.00119877709953305\\
46.94	0.00119994034960744\\
46.95	0.001201104289972\\
46.96	0.00120226892131505\\
46.97	0.00120343424432604\\
46.98	0.00120460025969553\\
46.99	0.0012057669681152\\
47	0.00120693437027784\\
47.01	0.0012081024668774\\
47.02	0.00120927125860892\\
47.03	0.00121044074616861\\
47.04	0.00121161093025376\\
47.05	0.00121278181156286\\
47.06	0.00121395339079548\\
47.07	0.00121512566865237\\
47.08	0.00121629864583541\\
47.09	0.00121747232304761\\
47.1	0.00121864670099316\\
47.11	0.00121982178037736\\
47.12	0.00122099756190671\\
47.13	0.00122217404628883\\
47.14	0.00122335123423252\\
47.15	0.00122452912644772\\
47.16	0.00122570772364555\\
47.17	0.00122688702653828\\
47.18	0.00122806703583938\\
47.19	0.00122924775226346\\
47.2	0.00123042917652631\\
47.21	0.00123161130934491\\
47.22	0.00123279415143741\\
47.23	0.00123397770352313\\
47.24	0.0012351619663226\\
47.25	0.00123634694055752\\
47.26	0.00123753262695078\\
47.27	0.00123871902622645\\
47.28	0.00123990613910983\\
47.29	0.00124109396632739\\
47.3	0.00124228250860681\\
47.31	0.00124347176667695\\
47.32	0.00124466174126792\\
47.33	0.001245852433111\\
47.34	0.0012470438429387\\
47.35	0.00124823597148473\\
47.36	0.00124942881948404\\
47.37	0.00125062238767276\\
47.38	0.00125181667678828\\
47.39	0.00125301168756919\\
47.4	0.00125420742075533\\
47.41	0.00125540387708774\\
47.42	0.00125660105730872\\
47.43	0.00125779896216179\\
47.44	0.00125899759239171\\
47.45	0.00126019694874448\\
47.46	0.00126139703196736\\
47.47	0.00126259784280883\\
47.48	0.00126379938201864\\
47.49	0.00126500165034778\\
47.5	0.00126620464854851\\
47.51	0.00126740837737433\\
47.52	0.00126861283758001\\
47.53	0.00126981802992157\\
47.54	0.00127102395515633\\
47.55	0.00127223061404283\\
47.56	0.00127343800734093\\
47.57	0.00127464613581173\\
47.58	0.00127585500021764\\
47.59	0.00127706460132231\\
47.6	0.00127827493989071\\
47.61	0.00127948601668907\\
47.62	0.00128069783248493\\
47.63	0.00128191038804711\\
47.64	0.00128312368414573\\
47.65	0.0012843377215522\\
47.66	0.00128555250103925\\
47.67	0.00128676802338088\\
47.68	0.00128798428935244\\
47.69	0.00128920129973055\\
47.7	0.00129041905529317\\
47.71	0.00129163755681956\\
47.72	0.00129285680509032\\
47.73	0.00129407680088733\\
47.74	0.00129529754499384\\
47.75	0.00129651903819441\\
47.76	0.00129774128127492\\
47.77	0.00129896427502259\\
47.78	0.001300188020226\\
47.79	0.00130141251767502\\
47.8	0.00130263776816092\\
47.81	0.00130386377247627\\
47.82	0.00130509053141502\\
47.83	0.00130631804577246\\
47.84	0.00130754631634522\\
47.85	0.00130877534393131\\
47.86	0.00131000512933011\\
47.87	0.00131123567334232\\
47.88	0.00131246697677005\\
47.89	0.00131369904041678\\
47.9	0.00131493186508733\\
47.91	0.00131616545158792\\
47.92	0.00131739980072616\\
47.93	0.00131863491331102\\
47.94	0.00131987079015286\\
47.95	0.00132110743206346\\
47.96	0.00132234483985594\\
47.97	0.00132358301434488\\
47.98	0.00132482195634619\\
47.99	0.00132606166667723\\
48	0.00132730214615676\\
48.01	0.00132854339560494\\
48.02	0.00132978541584333\\
48.03	0.00133102820769493\\
48.04	0.00133227177198414\\
48.05	0.0013335161095368\\
48.06	0.00133476122118015\\
48.07	0.00133600710774287\\
48.08	0.00133725377005509\\
48.09	0.00133850120894834\\
48.1	0.00133974942525562\\
48.11	0.00134099841981136\\
48.12	0.00134224819345143\\
48.13	0.00134349874701314\\
48.14	0.00134475008133527\\
48.15	0.00134600219725805\\
48.16	0.00134725509562316\\
48.17	0.00134850877727376\\
48.18	0.00134976324305445\\
48.19	0.00135101849381131\\
48.2	0.0013522745303919\\
48.21	0.00135353135364526\\
48.22	0.00135478896442187\\
48.23	0.00135604736357373\\
48.24	0.00135730655195432\\
48.25	0.00135856653041859\\
48.26	0.001359827299823\\
48.27	0.00136108886102551\\
48.28	0.00136235121488556\\
48.29	0.00136361436226411\\
48.3	0.0013648783040236\\
48.31	0.00136614304102803\\
48.32	0.00136740857414286\\
48.33	0.0013686749042351\\
48.34	0.00136994203217326\\
48.35	0.0013712099588274\\
48.36	0.00137247868506907\\
48.37	0.00137374821177138\\
48.38	0.00137501853980897\\
48.39	0.00137628967005801\\
48.4	0.00137756160339622\\
48.41	0.00137883434070286\\
48.42	0.00138010788285874\\
48.43	0.00138138223074623\\
48.44	0.00138265738524924\\
48.45	0.00138393334725324\\
48.46	0.00138521011764529\\
48.47	0.001386487697314\\
48.48	0.00138776608714954\\
48.49	0.00138904528804366\\
48.5	0.00139032530088971\\
48.51	0.0013916061265826\\
48.52	0.00139288776601883\\
48.53	0.00139417022009648\\
48.54	0.00139545348971525\\
48.55	0.00139673757577641\\
48.56	0.00139802247918284\\
48.57	0.00139930820083902\\
48.58	0.00140059474165106\\
48.59	0.00140188210252665\\
48.6	0.00140317028437511\\
48.61	0.00140445928810738\\
48.62	0.00140574911463603\\
48.63	0.00140703976487525\\
48.64	0.00140833123974086\\
48.65	0.00140962354015031\\
48.66	0.0014109166670227\\
48.67	0.00141221062127876\\
48.68	0.00141350540384089\\
48.69	0.00141480101563311\\
48.7	0.0014160974575811\\
48.71	0.00141739473061223\\
48.72	0.00141869283565549\\
48.73	0.00141999177364156\\
48.74	0.00142129154550278\\
48.75	0.00142259215217318\\
48.76	0.00142389359458844\\
48.77	0.00142519587368595\\
48.78	0.00142649899040477\\
48.79	0.00142780294568565\\
48.8	0.00142910774047104\\
48.81	0.00143041337570508\\
48.82	0.00143171985233362\\
48.83	0.00143302717130422\\
48.84	0.00143433533356613\\
48.85	0.00143564434007033\\
48.86	0.00143695419176953\\
48.87	0.00143826488961812\\
48.88	0.00143957643457227\\
48.89	0.00144088882758984\\
48.9	0.00144220206963044\\
48.91	0.00144351616165542\\
48.92	0.00144483110462786\\
48.93	0.00144614689951262\\
48.94	0.00144746354727627\\
48.95	0.00144878104888715\\
48.96	0.00145009940531538\\
48.97	0.00145141861753282\\
48.98	0.0014527386865131\\
48.99	0.00145405961323163\\
49	0.0014553813986656\\
49.01	0.00145670404379396\\
49.02	0.00145802754959749\\
49.03	0.00145935191705869\\
49.04	0.00146067714716192\\
49.05	0.0014620032408933\\
49.06	0.00146333019924077\\
49.07	0.00146465802319406\\
49.08	0.00146598671374473\\
49.09	0.00146731627188613\\
49.1	0.00146864669861346\\
49.11	0.00146997799492372\\
49.12	0.00147131016181576\\
49.13	0.00147264320029024\\
49.14	0.00147397711134967\\
49.15	0.00147531189599839\\
49.16	0.00147664755524261\\
49.17	0.00147798409009037\\
49.18	0.00147932150155158\\
49.19	0.00148065979063797\\
49.2	0.0014819989583632\\
49.21	0.00148333900574274\\
49.22	0.00148467993379396\\
49.23	0.00148602174353611\\
49.24	0.00148736443599031\\
49.25	0.00148870801217958\\
49.26	0.00149005247312881\\
49.27	0.0014913978198648\\
49.28	0.00149274405341625\\
49.29	0.00149409117481378\\
49.3	0.00149543918508988\\
49.31	0.00149678808527899\\
49.32	0.00149813787641746\\
49.33	0.00149948855954355\\
49.34	0.00150084013569747\\
49.35	0.00150219260592134\\
49.36	0.00150354597125923\\
49.37	0.00150490023275716\\
49.38	0.00150625539146309\\
49.39	0.00150761144842692\\
49.4	0.00150896840470052\\
49.41	0.00151032626133772\\
49.42	0.00151168501939431\\
49.43	0.00151304467992806\\
49.44	0.00151440524399871\\
49.45	0.00151576671266798\\
49.46	0.00151712908699957\\
49.47	0.00151849236805919\\
49.48	0.00151985655691452\\
49.49	0.00152122165463526\\
49.5	0.00152258766229309\\
49.51	0.00152395458096174\\
49.52	0.00152532241171691\\
49.53	0.00152669115563636\\
49.54	0.00152806081379984\\
49.55	0.00152943138728916\\
49.56	0.00153080287718813\\
49.57	0.00153217528458263\\
49.58	0.00153354861056056\\
49.59	0.00153492285621191\\
49.6	0.00153629802262868\\
49.61	0.00153767411090494\\
49.62	0.00153905112213684\\
49.63	0.0015404290574226\\
49.64	0.00154180791786249\\
49.65	0.00154318770455888\\
49.66	0.00154456841861623\\
49.67	0.00154595006114107\\
49.68	0.00154733263324204\\
49.69	0.00154871613602989\\
49.7	0.00155010057061744\\
49.71	0.00155148593811966\\
49.72	0.00155287223965362\\
49.73	0.00155425947633851\\
49.74	0.00155564764929564\\
49.75	0.00155703675964847\\
49.76	0.00155842680852258\\
49.77	0.0015598177970457\\
49.78	0.00156120972634771\\
49.79	0.00156260259756063\\
49.8	0.00156399641181866\\
49.81	0.00156539117025815\\
49.82	0.00156678687401763\\
49.83	0.00156818352423778\\
49.84	0.00156958112206148\\
49.85	0.00157097966863379\\
49.86	0.00157237916510198\\
49.87	0.00157377961261548\\
49.88	0.00157518101232593\\
49.89	0.00157658336538721\\
49.9	0.00157798667295538\\
49.91	0.00157939093618872\\
49.92	0.00158079615624774\\
49.93	0.00158220233429519\\
49.94	0.00158360947149604\\
49.95	0.00158501756901751\\
49.96	0.00158642662802905\\
49.97	0.00158783664970238\\
49.98	0.00158924763521148\\
49.99	0.00159065958573255\\
50	0.00159207250244413\\
50.01	0.00159348638652697\\
50.02	0.00159490123916413\\
50.03	0.00159631706154095\\
50.04	0.00159773385484505\\
50.05	0.00159915162026638\\
50.06	0.00160057035899715\\
50.07	0.00160199007223192\\
50.08	0.00160341076116752\\
50.09	0.00160483242700314\\
50.1	0.00160625507094028\\
50.11	0.00160767869418275\\
50.12	0.00160910329793675\\
50.13	0.00161052888341076\\
50.14	0.00161195545181566\\
50.15	0.00161338300436466\\
50.16	0.00161481154227334\\
50.17	0.00161624106675964\\
50.18	0.00161767157904388\\
50.19	0.00161910308034875\\
50.2	0.00162053557189935\\
50.21	0.00162196905492314\\
50.22	0.00162340353064998\\
50.23	0.00162483900031216\\
50.24	0.00162627546514436\\
50.25	0.00162771292638366\\
50.26	0.0016291513852696\\
50.27	0.00163059084304411\\
50.28	0.00163203130095157\\
50.29	0.00163347276023881\\
50.3	0.00163491522215508\\
50.31	0.00163635868795211\\
50.32	0.00163780315888407\\
50.33	0.0016392486362076\\
50.34	0.00164069512118182\\
50.35	0.0016421426150683\\
50.36	0.00164359111913113\\
50.37	0.00164504063463686\\
50.38	0.00164649116285454\\
50.39	0.00164794270505574\\
50.4	0.00164939526251453\\
50.41	0.00165084883650748\\
50.42	0.00165230342831371\\
50.43	0.00165375903921484\\
50.44	0.00165521567049504\\
50.45	0.00165667332344101\\
50.46	0.00165813199934201\\
50.47	0.00165959169948986\\
50.48	0.00166105242517891\\
50.49	0.0016625141777061\\
50.5	0.00166397695837095\\
50.51	0.00166544076847553\\
50.52	0.00166690560932452\\
50.53	0.0016683714822252\\
50.54	0.00166983838848742\\
50.55	0.00167130632942365\\
50.56	0.001672775306349\\
50.57	0.00167424532058116\\
50.58	0.00167571637344046\\
50.59	0.00167718846624987\\
50.6	0.00167866160033499\\
50.61	0.00168013577702409\\
50.62	0.00168161099764805\\
50.63	0.00168308726354046\\
50.64	0.00168456457603754\\
50.65	0.00168604293647819\\
50.66	0.001687522346204\\
50.67	0.00168900280655925\\
50.68	0.00169048431889091\\
50.69	0.00169196688454865\\
50.7	0.00169345050488485\\
50.71	0.0016949351812546\\
50.72	0.00169642091501571\\
50.73	0.00169790770752874\\
50.74	0.00169939556015696\\
50.75	0.00170088447426641\\
50.76	0.00170237445122585\\
50.77	0.00170386549240683\\
50.78	0.00170535759918364\\
50.79	0.00170685077293335\\
50.8	0.00170834501503582\\
50.81	0.00170984032687367\\
50.82	0.00171133670983234\\
50.83	0.00171283416530005\\
50.84	0.00171433269466786\\
50.85	0.0017158322993296\\
50.86	0.00171733298068196\\
50.87	0.00171883474012444\\
50.88	0.00172033757905938\\
50.89	0.00172184149889197\\
50.9	0.00172334650103025\\
50.91	0.00172485258688511\\
50.92	0.00172635975787032\\
50.93	0.00172786801540252\\
50.94	0.00172937736090123\\
50.95	0.00173088779578886\\
50.96	0.00173239932149071\\
50.97	0.001733911939435\\
50.98	0.00173542565105284\\
50.99	0.00173694045777828\\
51	0.00173845636104828\\
51.01	0.00173997336230275\\
51.02	0.00174149146298452\\
51.03	0.00174301066453941\\
51.04	0.00174453096841614\\
51.05	0.00174605237606645\\
51.06	0.00174757488894502\\
51.07	0.00174909850850952\\
51.08	0.00175062323622061\\
51.09	0.00175214907354194\\
51.1	0.00175367602194017\\
51.11	0.00175520408288498\\
51.12	0.00175673325784905\\
51.13	0.00175826354830811\\
51.14	0.00175979495574091\\
51.15	0.00176132748162924\\
51.16	0.00176286112745795\\
51.17	0.00176439589471496\\
51.18	0.00176593178489124\\
51.19	0.00176746879948084\\
51.2	0.00176900693998091\\
51.21	0.00177054620789166\\
51.22	0.00177208660471643\\
51.23	0.00177362813196165\\
51.24	0.00177517079113687\\
51.25	0.00177671458375477\\
51.26	0.00177825951133117\\
51.27	0.00177980557538501\\
51.28	0.00178135277743838\\
51.29	0.00178290111901656\\
51.3	0.00178445060164797\\
51.31	0.00178600122686419\\
51.32	0.00178755299620002\\
51.33	0.00178910591119342\\
51.34	0.00179065997338556\\
51.35	0.00179221518432083\\
51.36	0.00179377154554682\\
51.37	0.00179532905861433\\
51.38	0.00179688772507745\\
51.39	0.00179844754649344\\
51.4	0.00180000852442287\\
51.41	0.00180157066042954\\
51.42	0.00180313395608051\\
51.43	0.00180469841294614\\
51.44	0.00180626403260007\\
51.45	0.00180783081661921\\
51.46	0.00180939876658381\\
51.47	0.0018109678840774\\
51.48	0.00181253817068684\\
51.49	0.00181410962800233\\
51.5	0.0018156822576174\\
51.51	0.00181725606112891\\
51.52	0.0018188310401371\\
51.53	0.00182040719624557\\
51.54	0.00182198453106128\\
51.55	0.0018235630461946\\
51.56	0.00182514274325925\\
51.57	0.00182672362387238\\
51.58	0.00182830568965454\\
51.59	0.00182988894222972\\
51.6	0.0018314733832253\\
51.61	0.00183305901427212\\
51.62	0.00183464583700446\\
51.63	0.00183623385306007\\
51.64	0.00183782306408013\\
51.65	0.00183941347170934\\
51.66	0.00184100507759584\\
51.67	0.00184259788339129\\
51.68	0.00184419189075084\\
51.69	0.00184578710133316\\
51.7	0.00184738351680042\\
51.71	0.00184898113881835\\
51.72	0.0018505799690562\\
51.73	0.00185218000918677\\
51.74	0.00185378126088642\\
51.75	0.00185538372583508\\
51.76	0.00185698740571626\\
51.77	0.00185859230221706\\
51.78	0.00186019841702816\\
51.79	0.00186180575184385\\
51.8	0.00186341430836206\\
51.81	0.00186502408828432\\
51.82	0.00186663509331582\\
51.83	0.00186824732516536\\
51.84	0.00186986078554542\\
51.85	0.00187147547617215\\
51.86	0.00187309139876537\\
51.87	0.00187470855504857\\
51.88	0.00187632694674894\\
51.89	0.00187794657559739\\
51.9	0.00187956744332855\\
51.91	0.00188118955168073\\
51.92	0.00188281290239603\\
51.93	0.00188443749722026\\
51.94	0.001886063337903\\
51.95	0.00188769042619758\\
51.96	0.00188931876386113\\
51.97	0.00189094835265454\\
51.98	0.00189257919434252\\
51.99	0.00189421129069357\\
52	0.00189584464348001\\
52.01	0.00189747925447799\\
52.02	0.0018991151254675\\
52.03	0.00190075225823237\\
52.04	0.00190239065456029\\
52.05	0.00190403031624282\\
52.06	0.0019056712450754\\
52.07	0.00190731344285736\\
52.08	0.00190895691139193\\
52.09	0.00191060165248625\\
52.1	0.00191224766795139\\
52.11	0.00191389495960233\\
52.12	0.00191554352925801\\
52.13	0.00191719337874133\\
52.14	0.00191884450987915\\
52.15	0.00192049692450229\\
52.16	0.00192215062444557\\
52.17	0.00192380561154781\\
52.18	0.00192546188765182\\
52.19	0.00192711945460446\\
52.2	0.00192877831425658\\
52.21	0.00193043846846311\\
52.22	0.001932099919083\\
52.23	0.0019337626679793\\
52.24	0.00193542671701908\\
52.25	0.00193709206807355\\
52.26	0.00193875872301798\\
52.27	0.00194042668373177\\
52.28	0.00194209595209843\\
52.29	0.00194376653000559\\
52.3	0.00194543841934505\\
52.31	0.00194711162201273\\
52.32	0.00194878613990874\\
52.33	0.00195046197493734\\
52.34	0.00195213912900701\\
52.35	0.0019538176040304\\
52.36	0.00195549740192438\\
52.37	0.00195717852461006\\
52.38	0.00195886097401275\\
52.39	0.00196054475206204\\
52.4	0.00196222986069174\\
52.41	0.00196391630183997\\
52.42	0.00196560407744908\\
52.43	0.00196729318946578\\
52.44	0.00196898363984101\\
52.45	0.00197067543053008\\
52.46	0.00197236856349261\\
52.47	0.00197406304069254\\
52.48	0.0019757588640982\\
52.49	0.00197745603568227\\
52.5	0.00197915455742177\\
52.51	0.00198085443129817\\
52.52	0.00198255565929729\\
52.53	0.0019842582434094\\
52.54	0.00198596218562916\\
52.55	0.00198766748795568\\
52.56	0.00198937415239253\\
52.57	0.00199108218094773\\
52.58	0.00199279157563378\\
52.59	0.00199450233846767\\
52.6	0.00199621447147087\\
52.61	0.00199792797666937\\
52.62	0.00199964285609371\\
52.63	0.00200135911177891\\
52.64	0.0020030767457646\\
52.65	0.00200479576009494\\
52.66	0.00200651615681865\\
52.67	0.00200823793798907\\
52.68	0.00200996110566413\\
52.69	0.00201168566190636\\
52.7	0.00201341160878291\\
52.71	0.00201513894836561\\
52.72	0.00201686768273089\\
52.73	0.00201859781395988\\
52.74	0.00202032934413837\\
52.75	0.00202206227535685\\
52.76	0.00202379660971049\\
52.77	0.00202553234929922\\
52.78	0.00202726949622765\\
52.79	0.00202900805260518\\
52.8	0.00203074802054594\\
52.81	0.00203248940216882\\
52.82	0.00203423219959752\\
52.83	0.00203597641496051\\
52.84	0.0020377220503911\\
52.85	0.00203946910802739\\
52.86	0.00204121759001234\\
52.87	0.00204296749849375\\
52.88	0.00204471883562429\\
52.89	0.00204647160356151\\
52.9	0.00204822580446785\\
52.91	0.00204998144051063\\
52.92	0.00205173851386213\\
52.93	0.00205349702669954\\
52.94	0.00205525698120499\\
52.95	0.00205701837956559\\
52.96	0.00205878122397342\\
52.97	0.00206054551662553\\
52.98	0.00206231125972399\\
52.99	0.00206407845547588\\
53	0.00206584710609332\\
53.01	0.00206761721379348\\
53.02	0.00206938878079855\\
53.03	0.00207116180933584\\
53.04	0.00207293630163772\\
53.05	0.00207471225994168\\
53.06	0.00207648968649031\\
53.07	0.00207826858353134\\
53.08	0.00208004895331765\\
53.09	0.00208183079810728\\
53.1	0.00208361412016343\\
53.11	0.0020853989217545\\
53.12	0.0020871852051541\\
53.13	0.00208897297264105\\
53.14	0.00209076222649939\\
53.15	0.00209255296901845\\
53.16	0.00209434520249278\\
53.17	0.00209613892922223\\
53.18	0.00209793415151195\\
53.19	0.00209973087167238\\
53.2	0.00210152909201929\\
53.21	0.0021033288148738\\
53.22	0.00210513004256237\\
53.23	0.00210693277741682\\
53.24	0.00210873702177439\\
53.25	0.00211054277797769\\
53.26	0.00211235004837475\\
53.27	0.00211415883531903\\
53.28	0.00211596914116944\\
53.29	0.00211778096829035\\
53.3	0.00211959431905161\\
53.31	0.00212140919582857\\
53.32	0.00212322560100205\\
53.33	0.00212504353695846\\
53.34	0.00212686300608968\\
53.35	0.0021286840107932\\
53.36	0.00213050655347205\\
53.37	0.00213233063653486\\
53.38	0.00213415626239586\\
53.39	0.0021359834334749\\
53.4	0.00213781215219747\\
53.41	0.00213964242099472\\
53.42	0.00214147424230343\\
53.43	0.00214330761856611\\
53.44	0.00214514255223094\\
53.45	0.00214697904575184\\
53.46	0.00214881710158844\\
53.47	0.00215065672220614\\
53.48	0.0021524979100761\\
53.49	0.00215434066767525\\
53.5	0.00215618499748634\\
53.51	0.00215803090199793\\
53.52	0.00215987838370439\\
53.53	0.00216172744510598\\
53.54	0.0021635780887088\\
53.55	0.00216543031702483\\
53.56	0.00216728413257197\\
53.57	0.00216913953787403\\
53.58	0.00217099653546075\\
53.59	0.00217285512786781\\
53.6	0.00217471531763689\\
53.61	0.00217657710731563\\
53.62	0.00217844049945768\\
53.63	0.00218030549662272\\
53.64	0.00218217210137646\\
53.65	0.00218404031629066\\
53.66	0.00218591014394316\\
53.67	0.0021877815869179\\
53.68	0.0021896546478049\\
53.69	0.00219152932920034\\
53.7	0.00219340563370652\\
53.71	0.00219528356393191\\
53.72	0.00219716312249117\\
53.73	0.00219904431200513\\
53.74	0.00220092713510086\\
53.75	0.00220281159441165\\
53.76	0.00220469769257706\\
53.77	0.00220658543224288\\
53.78	0.00220847481606124\\
53.79	0.00221036584669054\\
53.8	0.00221225852679552\\
53.81	0.00221415285904724\\
53.82	0.00221604884612316\\
53.83	0.00221794649070708\\
53.84	0.00221984579548922\\
53.85	0.00222174676316623\\
53.86	0.00222364939644116\\
53.87	0.00222555369802355\\
53.88	0.00222745967062938\\
53.89	0.00222936731698115\\
53.9	0.00223127663980786\\
53.91	0.00223318764184505\\
53.92	0.00223510032583479\\
53.93	0.00223701469452572\\
53.94	0.0022389307506731\\
53.95	0.00224084849703875\\
53.96	0.00224276793639116\\
53.97	0.00224468907150543\\
53.98	0.00224661190516335\\
53.99	0.00224853644015338\\
54	0.00225046267927069\\
54.01	0.00225239062531717\\
54.02	0.00225432028110147\\
54.03	0.00225625164943898\\
54.04	0.00225818473315188\\
54.05	0.00226011953506917\\
54.06	0.00226205605802664\\
54.07	0.00226399430486696\\
54.08	0.00226593427843964\\
54.09	0.00226787598160109\\
54.1	0.0022698194172146\\
54.11	0.00227176458815039\\
54.12	0.00227371149728565\\
54.13	0.00227566014750451\\
54.14	0.0022776105416981\\
54.15	0.00227956268276454\\
54.16	0.00228151657360898\\
54.17	0.00228347221714364\\
54.18	0.00228542961628777\\
54.19	0.00228738877396774\\
54.2	0.00228934969311702\\
54.21	0.00229131237667621\\
54.22	0.00229327682759306\\
54.23	0.0022952430488225\\
54.24	0.00229721104332666\\
54.25	0.00229918081407485\\
54.26	0.00230115236404366\\
54.27	0.00230312569621692\\
54.28	0.00230510081358574\\
54.29	0.00230707771914854\\
54.3	0.00230905641591105\\
54.31	0.00231103690688635\\
54.32	0.0023130191950949\\
54.33	0.00231500328356452\\
54.34	0.00231698917533047\\
54.35	0.00231897687343542\\
54.36	0.00232096638092951\\
54.37	0.00232295770087035\\
54.38	0.00232495083632304\\
54.39	0.00232694579036023\\
54.4	0.00232894256606209\\
54.41	0.00233094116651636\\
54.42	0.00233294159481837\\
54.43	0.00233494385407105\\
54.44	0.00233694794738499\\
54.45	0.00233895387787841\\
54.46	0.00234096164867723\\
54.47	0.00234297126291506\\
54.48	0.00234498272373325\\
54.49	0.00234699603428087\\
54.5	0.00234901119771479\\
54.51	0.00235102821719967\\
54.52	0.00235304709590797\\
54.53	0.00235506783702\\
54.54	0.00235709044372397\\
54.55	0.00235911491921591\\
54.56	0.00236114126669983\\
54.57	0.00236316948938763\\
54.58	0.00236519959049918\\
54.59	0.00236723157326236\\
54.6	0.00236926544091302\\
54.61	0.00237130119669506\\
54.62	0.00237333884386043\\
54.63	0.00237537838566917\\
54.64	0.00237741982538943\\
54.65	0.00237946316629745\\
54.66	0.00238150841167767\\
54.67	0.00238355556482267\\
54.68	0.00238560462903326\\
54.69	0.00238765560761844\\
54.7	0.00238970850389552\\
54.71	0.00239176332119003\\
54.72	0.00239382006283582\\
54.73	0.00239587873217509\\
54.74	0.00239793933255835\\
54.75	0.00240000186734451\\
54.76	0.00240206633990089\\
54.77	0.0024041327536032\\
54.78	0.00240620111183566\\
54.79	0.0024082714179909\\
54.8	0.00241034367547012\\
54.81	0.002412417887683\\
54.82	0.00241449405804779\\
54.83	0.00241657218999134\\
54.84	0.00241865228694908\\
54.85	0.0024207343523651\\
54.86	0.00242281838969211\\
54.87	0.00242490440239154\\
54.88	0.00242699239393352\\
54.89	0.00242908236779692\\
54.9	0.00243117432746937\\
54.91	0.00243326827644728\\
54.92	0.00243536421823592\\
54.93	0.00243746215634936\\
54.94	0.00243956209431055\\
54.95	0.00244166403565136\\
54.96	0.00244376798391257\\
54.97	0.00244587394264391\\
54.98	0.00244798191540409\\
54.99	0.00245009190576084\\
55	0.0024522039172909\\
55.01	0.00245431795358011\\
55.02	0.00245643401822335\\
55.03	0.00245855211482467\\
55.04	0.00246067224699723\\
55.05	0.00246279441836336\\
55.06	0.00246491863255462\\
55.07	0.00246704489321178\\
55.08	0.00246917320398487\\
55.09	0.0024713035685332\\
55.1	0.00247343599052541\\
55.11	0.00247557047363946\\
55.12	0.0024777070215627\\
55.13	0.00247984563799187\\
55.14	0.00248198632663314\\
55.15	0.00248412909120215\\
55.16	0.00248627393542401\\
55.17	0.00248842086303334\\
55.18	0.00249056987777432\\
55.19	0.00249272098340069\\
55.2	0.00249487418367582\\
55.21	0.00249702948237267\\
55.22	0.00249918688327391\\
55.23	0.00250134639017187\\
55.24	0.0025035080068686\\
55.25	0.00250567173717591\\
55.26	0.00250783758491541\\
55.27	0.00251000555391848\\
55.28	0.00251217564802639\\
55.29	0.00251434787109023\\
55.3	0.00251652222697105\\
55.31	0.00251869871953977\\
55.32	0.00252087735267734\\
55.33	0.00252305813027464\\
55.34	0.00252524105623262\\
55.35	0.00252742613446227\\
55.36	0.00252961336888466\\
55.37	0.002531802763431\\
55.38	0.00253399432204262\\
55.39	0.00253618804867106\\
55.4	0.00253838394727804\\
55.41	0.00254058202183555\\
55.42	0.00254278227632586\\
55.43	0.00254498471474152\\
55.44	0.00254718934108543\\
55.45	0.00254939615937087\\
55.46	0.00255160517362153\\
55.47	0.0025538163878715\\
55.48	0.00255602980616537\\
55.49	0.00255824543255823\\
55.5	0.00256046327111568\\
55.51	0.0025626833259139\\
55.52	0.00256490560103968\\
55.53	0.00256713010059042\\
55.54	0.00256935682867419\\
55.55	0.00257158578940977\\
55.56	0.00257381698692665\\
55.57	0.0025760504253651\\
55.58	0.00257828610887618\\
55.59	0.00258052404162179\\
55.6	0.00258276422777468\\
55.61	0.00258500667151852\\
55.62	0.00258725137704788\\
55.63	0.00258949834856833\\
55.64	0.00259174759029642\\
55.65	0.00259399842110423\\
55.66	0.00259625055670001\\
55.67	0.00259850399834153\\
55.68	0.00260075874728894\\
55.69	0.00260301480480472\\
55.7	0.00260527217215373\\
55.71	0.00260753085060318\\
55.72	0.00260979084142265\\
55.73	0.00261205214588411\\
55.74	0.00261431476526192\\
55.75	0.00261657870083281\\
55.76	0.00261884395387595\\
55.77	0.00262111052567288\\
55.78	0.00262337841750759\\
55.79	0.00262564763066649\\
55.8	0.00262791816643838\\
55.81	0.00263019002611457\\
55.82	0.00263246321098876\\
55.83	0.00263473772235712\\
55.84	0.00263701356151829\\
55.85	0.00263929072977338\\
55.86	0.00264156922842598\\
55.87	0.00264384905878214\\
55.88	0.00264613022215044\\
55.89	0.00264841271984193\\
55.9	0.00265069655317019\\
55.91	0.0026529817234513\\
55.92	0.00265526823200387\\
55.93	0.00265755608014905\\
55.94	0.00265984526921052\\
55.95	0.0026621358005145\\
55.96	0.0026644276753898\\
55.97	0.00266672089516774\\
55.98	0.00266901546118227\\
55.99	0.00267131137476988\\
56	0.00267360863726965\\
56.01	0.00267590725002329\\
56.02	0.00267820721437508\\
56.03	0.00268050853167191\\
56.04	0.00268281120326332\\
56.05	0.00268511523050146\\
56.06	0.00268742061474111\\
56.07	0.00268972735733971\\
56.08	0.00269203545965735\\
56.09	0.00269434492305678\\
56.1	0.00269665574890342\\
56.11	0.00269896793856538\\
56.12	0.00270128149341344\\
56.13	0.00270359641482108\\
56.14	0.0027059127041645\\
56.15	0.0027082303628226\\
56.16	0.00271054939217699\\
56.17	0.00271286979361204\\
56.18	0.00271519156851483\\
56.19	0.00271751471827521\\
56.2	0.00271983924428578\\
56.21	0.00272216514794189\\
56.22	0.00272449243064169\\
56.23	0.00272682109378609\\
56.24	0.00272915113877881\\
56.25	0.00273148256702635\\
56.26	0.00273381537993804\\
56.27	0.00273614957892602\\
56.28	0.00273848516540525\\
56.29	0.00274082214079354\\
56.3	0.00274316050651154\\
56.31	0.00274550026398276\\
56.32	0.00274784141463356\\
56.33	0.00275018395989318\\
56.34	0.00275252790119376\\
56.35	0.00275487323997029\\
56.36	0.00275721997766071\\
56.37	0.00275956811570583\\
56.38	0.00276191765554942\\
56.39	0.00276426859863813\\
56.4	0.00276662094642158\\
56.41	0.00276897470035235\\
56.42	0.00277132986188595\\
56.43	0.00277368643248085\\
56.44	0.00277604441359854\\
56.45	0.00277840380670345\\
56.46	0.00278076461326302\\
56.47	0.00278312683474772\\
56.48	0.00278549047263101\\
56.49	0.00278785552838937\\
56.5	0.00279022200350234\\
56.51	0.00279258989945248\\
56.52	0.00279495921772543\\
56.53	0.00279732995980986\\
56.54	0.00279970212719755\\
56.55	0.00280207572138334\\
56.56	0.00280445074386517\\
56.57	0.0028068271961441\\
56.58	0.00280920507972428\\
56.59	0.002811584396113\\
56.6	0.00281396514682069\\
56.61	0.00281634733336091\\
56.62	0.00281873095725039\\
56.63	0.00282111602000901\\
56.64	0.00282350252315985\\
56.65	0.00282589046822914\\
56.66	0.00282827985674634\\
56.67	0.00283067069024412\\
56.68	0.00283306297025834\\
56.69	0.0028354566983281\\
56.7	0.00283785187599575\\
56.71	0.00284024850480688\\
56.72	0.00284264658631034\\
56.73	0.00284504612205827\\
56.74	0.00284744711360606\\
56.75	0.00284984956251241\\
56.76	0.00285225347033932\\
56.77	0.00285465883865212\\
56.78	0.00285706566901945\\
56.79	0.00285947396301328\\
56.8	0.00286188372220894\\
56.81	0.00286429494818512\\
56.82	0.00286670764252388\\
56.83	0.00286912180681064\\
56.84	0.00287153744263424\\
56.85	0.00287395455158691\\
56.86	0.00287637313526429\\
56.87	0.00287879319526545\\
56.88	0.00288121473319292\\
56.89	0.00288363775065263\\
56.9	0.00288606224925401\\
56.91	0.00288848823060996\\
56.92	0.00289091569633683\\
56.93	0.0028933446480545\\
56.94	0.00289577508738634\\
56.95	0.00289820701595924\\
56.96	0.00290064043540363\\
56.97	0.00290307534735346\\
56.98	0.00290551175344625\\
56.99	0.00290794965532309\\
57	0.00291038905462863\\
57.01	0.00291282995301112\\
57.02	0.00291527235212241\\
57.03	0.00291771625361796\\
57.04	0.00292016165915686\\
57.05	0.00292260857040182\\
57.06	0.00292505698901924\\
57.07	0.00292750691668037\\
57.08	0.00292995835510231\\
57.09	0.00293241130600555\\
57.1	0.00293486577111402\\
57.11	0.0029373217521551\\
57.12	0.00293977925085961\\
57.13	0.00294223826896185\\
57.14	0.00294469880819957\\
57.15	0.00294716087031401\\
57.16	0.0029496244570499\\
57.17	0.00295208957015547\\
57.18	0.00295455621138246\\
57.19	0.00295702438248612\\
57.2	0.00295949408522525\\
57.21	0.00296196532136218\\
57.22	0.00296443809266277\\
57.23	0.00296691240089647\\
57.24	0.00296938824783628\\
57.25	0.00297186563525879\\
57.26	0.00297434456494416\\
57.27	0.00297682503867617\\
57.28	0.0029793070582422\\
57.29	0.00298179062543324\\
57.3	0.00298427574204394\\
57.31	0.00298676240987256\\
57.32	0.002989250630721\\
57.33	0.00299174040639486\\
57.34	0.00299423173870339\\
57.35	0.00299672462945951\\
57.36	0.00299921908047984\\
57.37	0.00300171509358471\\
57.38	0.00300421267059815\\
57.39	0.00300671181334791\\
57.4	0.00300921252366548\\
57.41	0.00301171480338609\\
57.42	0.00301421865434873\\
57.43	0.00301672407839616\\
57.44	0.00301923107737487\\
57.45	0.00302173965313521\\
57.46	0.00302424980753126\\
57.47	0.00302676154242094\\
57.48	0.00302927485966598\\
57.49	0.00303178976113194\\
57.5	0.00303430624868822\\
57.51	0.00303682432420806\\
57.52	0.00303934398956859\\
57.53	0.00304186524665077\\
57.54	0.00304438809733946\\
57.55	0.00304691254352344\\
57.56	0.00304943858709536\\
57.57	0.0030519662299518\\
57.58	0.00305449547399328\\
57.59	0.00305702632112423\\
57.6	0.00305955877325304\\
57.61	0.00306209283229208\\
57.62	0.00306462850015767\\
57.63	0.00306716577877011\\
57.64	0.00306970467005373\\
57.65	0.00307224517593681\\
57.66	0.00307478729835171\\
57.67	0.00307733103923476\\
57.68	0.00307987640052636\\
57.69	0.00308242338417097\\
57.7	0.0030849719921171\\
57.71	0.00308752222631732\\
57.72	0.00309007408872832\\
57.73	0.00309262758131087\\
57.74	0.00309518270602984\\
57.75	0.00309773946485424\\
57.76	0.0031002978597572\\
57.77	0.00310285789271601\\
57.78	0.00310541956571211\\
57.79	0.00310798288073109\\
57.8	0.00311054783976274\\
57.81	0.00311311444480105\\
57.82	0.0031156826978442\\
57.83	0.00311825260089459\\
57.84	0.00312082415595885\\
57.85	0.00312339736504785\\
57.86	0.00312597223017672\\
57.87	0.00312854875336484\\
57.88	0.00313112693663588\\
57.89	0.0031337067820178\\
57.9	0.00313628829154286\\
57.91	0.00313887146724762\\
57.92	0.003141456311173\\
57.93	0.00314404282536421\\
57.94	0.00314663101187087\\
57.95	0.00314922087274692\\
57.96	0.00315181241005069\\
57.97	0.00315440562584491\\
57.98	0.0031570005221967\\
57.99	0.00315959710117758\\
58	0.00316219536486353\\
58.01	0.00316479531533496\\
58.02	0.00316739695467672\\
58.03	0.00317000028497814\\
58.04	0.00317260530833301\\
58.05	0.00317521202683963\\
58.06	0.0031778204426008\\
58.07	0.00318043055772383\\
58.08	0.00318304237432057\\
58.09	0.00318565589450741\\
58.1	0.00318827112040529\\
58.11	0.00319088805413972\\
58.12	0.00319350669784082\\
58.13	0.00319612705364326\\
58.14	0.00319874912368636\\
58.15	0.00320137291011404\\
58.16	0.00320399841507487\\
58.17	0.00320662564072206\\
58.18	0.00320925458921348\\
58.19	0.00321188526271169\\
58.2	0.00321451766338393\\
58.21	0.00321715179340215\\
58.22	0.00321978765494301\\
58.23	0.00322242525018791\\
58.24	0.00322506458132298\\
58.25	0.00322770565053912\\
58.26	0.00323034846003201\\
58.27	0.0032329930120021\\
58.28	0.00323563930865464\\
58.29	0.00323828735219971\\
58.3	0.00324093714485221\\
58.31	0.00324358868883187\\
58.32	0.00324624198636329\\
58.33	0.00324889703967593\\
58.34	0.00325155385100415\\
58.35	0.0032542124225872\\
58.36	0.00325687275666924\\
58.37	0.00325953485549934\\
58.38	0.00326219872133154\\
58.39	0.00326486435642484\\
58.4	0.00326753176304317\\
58.41	0.00327020094345549\\
58.42	0.00327287189993573\\
58.43	0.00327554463476283\\
58.44	0.00327821915022079\\
58.45	0.00328089544859862\\
58.46	0.0032835735321904\\
58.47	0.00328625340329529\\
58.48	0.00328893506421752\\
58.49	0.00329161851726643\\
58.5	0.00329430376475649\\
58.51	0.00329699080900728\\
58.52	0.00329967965234353\\
58.53	0.00330237029709514\\
58.54	0.00330506274559719\\
58.55	0.00330775700018993\\
58.56	0.00331045306321885\\
58.57	0.00331315093703463\\
58.58	0.00331585062399321\\
58.59	0.00331855212645576\\
58.6	0.00332125544678874\\
58.61	0.00332396058736388\\
58.62	0.00332666755055822\\
58.63	0.0033293763387541\\
58.64	0.00333208695433919\\
58.65	0.00333479939970652\\
58.66	0.00333751367725448\\
58.67	0.00334022978938681\\
58.68	0.00334294773851266\\
58.69	0.0033456675270466\\
58.7	0.0033483891574086\\
58.71	0.0033511126320241\\
58.72	0.00335383795332395\\
58.73	0.00335656512374452\\
58.74	0.00335929414572763\\
58.75	0.00336202502172063\\
58.76	0.00336475775417637\\
58.77	0.00336749234555324\\
58.78	0.00337022879831519\\
58.79	0.00337296711493173\\
58.8	0.00337570729787796\\
58.81	0.00337844934963457\\
58.82	0.00338119327268788\\
58.83	0.00338393906952983\\
58.84	0.00338668674265803\\
58.85	0.00338943629457573\\
58.86	0.00339218772779188\\
58.87	0.00339494104482113\\
58.88	0.00339769624818384\\
58.89	0.0034004533404061\\
58.9	0.00340321232401975\\
58.91	0.00340597320156241\\
58.92	0.00340873597557747\\
58.93	0.00341150064861413\\
58.94	0.00341426722322738\\
58.95	0.00341703570197809\\
58.96	0.00341980608743293\\
58.97	0.00342257838216449\\
58.98	0.00342535258875121\\
58.99	0.00342812870977743\\
59	0.00343090674783344\\
59.01	0.00343368670551544\\
59.02	0.00343646858542559\\
59.03	0.00343925239017203\\
59.04	0.00344203812236889\\
59.05	0.00344482578463629\\
59.06	0.00344761537960039\\
59.07	0.00345040690989339\\
59.08	0.00345320037815354\\
59.09	0.00345599578702518\\
59.1	0.00345879313915875\\
59.11	0.00346159243721077\\
59.12	0.00346439368384394\\
59.13	0.00346719688172705\\
59.14	0.00347000203353511\\
59.15	0.00347280914194929\\
59.16	0.00347561820965697\\
59.17	0.00347842923935172\\
59.18	0.0034812422337334\\
59.19	0.00348405719550809\\
59.2	0.00348687412738816\\
59.21	0.00348969303209227\\
59.22	0.00349251391234539\\
59.23	0.00349533677087882\\
59.24	0.00349816161043021\\
59.25	0.00350098843374358\\
59.26	0.00350381724356933\\
59.27	0.00350664804266427\\
59.28	0.00350948083379163\\
59.29	0.00351231561972108\\
59.3	0.00351515240322874\\
59.31	0.00351799118709724\\
59.32	0.00352083197411567\\
59.33	0.00352367476707968\\
59.34	0.0035265195687914\\
59.35	0.00352936638205957\\
59.36	0.00353221520969946\\
59.37	0.00353506605453297\\
59.38	0.00353791891938858\\
59.39	0.00354077380710142\\
59.4	0.00354363072051326\\
59.41	0.00354648966247255\\
59.42	0.00354935063583442\\
59.43	0.00355221364346071\\
59.44	0.00355507868822\\
59.45	0.00355794577298759\\
59.46	0.00356081490064557\\
59.47	0.00356368607408281\\
59.48	0.00356655929619498\\
59.49	0.00356943456988459\\
59.5	0.00357231189806097\\
59.51	0.00357519128364033\\
59.52	0.00357807272954578\\
59.53	0.00358095623870731\\
59.54	0.00358384181406184\\
59.55	0.00358672945855323\\
59.56	0.00358961917513233\\
59.57	0.00359251096675696\\
59.58	0.00359540483639192\\
59.59	0.00359830078700907\\
59.6	0.0036011988215873\\
59.61	0.00360409894311255\\
59.62	0.00360700115457788\\
59.63	0.00360990545898343\\
59.64	0.00361281185933647\\
59.65	0.00361572035865142\\
59.66	0.00361863095994986\\
59.67	0.00362154366626057\\
59.68	0.00362445848061953\\
59.69	0.00362737540606995\\
59.7	0.00363029444566228\\
59.71	0.00363321560245426\\
59.72	0.00363613887951091\\
59.73	0.00363906427990454\\
59.74	0.00364199180671483\\
59.75	0.00364492146302879\\
59.76	0.00364785325194081\\
59.77	0.00365078717655267\\
59.78	0.00365372323997357\\
59.79	0.00365666144532015\\
59.8	0.00365960179571651\\
59.81	0.00366254429429421\\
59.82	0.00366548894419233\\
59.83	0.00366843574855747\\
59.84	0.00367138471054376\\
59.85	0.00367433583331291\\
59.86	0.00367728912003421\\
59.87	0.00368024457388453\\
59.88	0.00368320219804841\\
59.89	0.00368616199571803\\
59.9	0.00368912397009321\\
59.91	0.0036920881243815\\
59.92	0.00369505446179814\\
59.93	0.00369802298556612\\
59.94	0.00370099369891619\\
59.95	0.00370396660508685\\
59.96	0.00370694170732444\\
59.97	0.00370991900888309\\
59.98	0.00371289851302481\\
59.99	0.00371588022301943\\
60	0.00371886414214471\\
60.01	0.00372185027368629\\
60.02	0.00372483862093776\\
60.03	0.00372782918720066\\
60.04	0.0037308219757845\\
60.05	0.00373381699000679\\
60.06	0.00373681423319306\\
60.07	0.0037398137086769\\
60.08	0.00374281541979993\\
60.09	0.00374581936991188\\
60.1	0.00374882556237058\\
60.11	0.00375183400054199\\
60.12	0.00375484468780024\\
60.13	0.00375785762752761\\
60.14	0.00376087282311459\\
60.15	0.00376389027795989\\
60.16	0.00376690999547046\\
60.17	0.00376993197906152\\
60.18	0.00377295623215658\\
60.19	0.00377598275818744\\
60.2	0.00377901156059425\\
60.21	0.00378204264282551\\
60.22	0.00378507600833811\\
60.23	0.0037881116605973\\
60.24	0.0037911496030768\\
60.25	0.00379418983925876\\
60.26	0.00379723237263377\\
60.27	0.00380027720670094\\
60.28	0.00380332434496788\\
60.29	0.00380637379095076\\
60.3	0.00380942554817427\\
60.31	0.00381247962017171\\
60.32	0.00381553601048496\\
60.33	0.00381859472266455\\
60.34	0.00382165576026965\\
60.35	0.0038247191268681\\
60.36	0.00382778482603643\\
60.37	0.0038308528613599\\
60.38	0.00383392323643251\\
60.39	0.003836995954857\\
60.4	0.00384007102024492\\
60.41	0.00384314843621664\\
60.42	0.00384622820640132\\
60.43	0.00384931033443701\\
60.44	0.00385239482397062\\
60.45	0.00385548167865798\\
60.46	0.00385857090216382\\
60.47	0.00386166249816183\\
60.48	0.00386475647033468\\
60.49	0.00386785282237399\\
60.5	0.00387095155798045\\
60.51	0.00387405268086375\\
60.52	0.00387715619474265\\
60.53	0.00388026210334501\\
60.54	0.00388337041040777\\
60.55	0.00388648111967701\\
60.56	0.00388801538673845\\
60.57	0.00388938420347543\\
60.58	0.0038907536328378\\
60.59	0.00389212367495261\\
60.6	0.0038934943299461\\
60.61	0.00389486559794366\\
60.62	0.00389623747906983\\
60.63	0.00389760997344829\\
60.64	0.00389898308120188\\
60.65	0.00390035680245256\\
60.66	0.00390173113732143\\
60.67	0.00390310608592875\\
60.68	0.00390448164839386\\
60.69	0.00390585782483526\\
60.7	0.00390723461537055\\
60.71	0.00390861202011646\\
60.72	0.0039099900391888\\
60.73	0.00391136867270252\\
60.74	0.00391274792077168\\
60.75	0.00391412778350939\\
60.76	0.00391550826102789\\
60.77	0.00391688935343851\\
60.78	0.00391827106085165\\
60.79	0.00391965338337681\\
60.8	0.00392103632112255\\
60.81	0.00392241987419652\\
60.82	0.0039238040427054\\
60.83	0.00392518882675501\\
60.84	0.00392657422645015\\
60.85	0.00392796024189471\\
60.86	0.00392934687319165\\
60.87	0.00393073412044295\\
60.88	0.00393212198374964\\
60.89	0.00393351046321179\\
60.9	0.00393489955892851\\
60.91	0.00393628927099793\\
60.92	0.00393767959951723\\
60.93	0.00393907054458257\\
60.94	0.00394046210628916\\
60.95	0.00394185428473122\\
60.96	0.00394324708000196\\
60.97	0.00394464049219361\\
60.98	0.0039460345213974\\
60.99	0.00394742916770354\\
61	0.00394882443120124\\
61.01	0.00395022031197872\\
61.02	0.00395161681012313\\
61.03	0.00395301392572064\\
61.04	0.00395441165885637\\
61.05	0.00395581000961442\\
61.06	0.00395720897807787\\
61.07	0.00395860856432872\\
61.08	0.00396000876844797\\
61.09	0.00396140959051551\\
61.1	0.00396281103061024\\
61.11	0.00396421308880997\\
61.12	0.00396561576519145\\
61.13	0.00396701905983036\\
61.14	0.0039684229728013\\
61.15	0.00396982750417783\\
61.16	0.00397123265403238\\
61.17	0.00397263842243633\\
61.18	0.00397404480945995\\
61.19	0.00397545181517243\\
61.2	0.00397685943964184\\
61.21	0.00397826768293517\\
61.22	0.00397967654511828\\
61.23	0.00398108602625593\\
61.24	0.00398249612641176\\
61.25	0.00398390684564828\\
61.26	0.00398531818402687\\
61.27	0.0039867301416078\\
61.28	0.00398814271845017\\
61.29	0.00398955591461198\\
61.3	0.00399096973015005\\
61.31	0.00399238416512006\\
61.32	0.00399379921957653\\
61.33	0.00399521489357284\\
61.34	0.00399663118716118\\
61.35	0.00399804810039259\\
61.36	0.00399946563331691\\
61.37	0.00400088378598284\\
61.38	0.00400230255843785\\
61.39	0.00400372195072827\\
61.4	0.0040051419628992\\
61.41	0.00400656259499456\\
61.42	0.00400798384705706\\
61.43	0.00400940571912822\\
61.44	0.00401082821124832\\
61.45	0.00401225132345643\\
61.46	0.00401367505579043\\
61.47	0.00401509940828693\\
61.48	0.00401652438098134\\
61.49	0.00401794997390784\\
61.5	0.00401937618709932\\
61.51	0.00402080302058747\\
61.52	0.00402223047440273\\
61.53	0.00402365854857427\\
61.54	0.00402508724313\\
61.55	0.00402651655809658\\
61.56	0.00402794649349937\\
61.57	0.0040293770493625\\
61.58	0.00403080822570878\\
61.59	0.00403224002255978\\
61.6	0.00403367243993574\\
61.61	0.00403510547785563\\
61.62	0.00403653913633713\\
61.63	0.00403797341539658\\
61.64	0.00403940831504906\\
61.65	0.0040408438353083\\
61.66	0.00404227997618675\\
61.67	0.00404371673769549\\
61.68	0.00404515411984433\\
61.69	0.00404659212264171\\
61.7	0.00404803074609475\\
61.71	0.00404946999020923\\
61.72	0.00405090985498956\\
61.73	0.00405235034043883\\
61.74	0.00405379144655878\\
61.75	0.00405523317334975\\
61.76	0.00405667552081077\\
61.77	0.00405811848893944\\
61.78	0.00405956207773205\\
61.79	0.00406100628718347\\
61.8	0.0040624511172872\\
61.81	0.00406389656803534\\
61.82	0.00406534263941861\\
61.83	0.00406678933142634\\
61.84	0.00406823664404644\\
61.85	0.00406968457726543\\
61.86	0.00407113313106841\\
61.87	0.00407258230543906\\
61.88	0.00407403210035965\\
61.89	0.004075482515811\\
61.9	0.00407693355177254\\
61.91	0.0040783852082054\\
61.92	0.00407983748506865\\
61.93	0.00408129038231982\\
61.94	0.00408274389991491\\
61.95	0.00408419803780843\\
61.96	0.00408565279595333\\
61.97	0.00408710817430102\\
61.98	0.00408856417280138\\
61.99	0.00409002079140271\\
62	0.00409147803005176\\
62.01	0.00409293588869374\\
62.02	0.00409439436727224\\
62.03	0.0040958534657293\\
62.04	0.00409731318400538\\
62.05	0.00409877352203932\\
62.06	0.00410023447976838\\
62.07	0.00410169605712821\\
62.08	0.00410315825405284\\
62.09	0.00410462107047469\\
62.1	0.00410608450632455\\
62.11	0.00410754856153157\\
62.12	0.00410901323602328\\
62.13	0.00411047852972553\\
62.14	0.00411194444256253\\
62.15	0.00411341097445685\\
62.16	0.00411487812532937\\
62.17	0.00411634589509929\\
62.18	0.00411781428368414\\
62.19	0.00411928329099977\\
62.2	0.00412075291696031\\
62.21	0.00412222316147821\\
62.22	0.0041236940244642\\
62.23	0.00412516550582728\\
62.24	0.00412663760547475\\
62.25	0.00412811032331217\\
62.26	0.00412958365924335\\
62.27	0.00413105761317036\\
62.28	0.00413253218499354\\
62.29	0.00413400737461142\\
62.3	0.00413548318192081\\
62.31	0.00413695960681673\\
62.32	0.0041384366491924\\
62.33	0.00413991430893929\\
62.34	0.00414139258594701\\
62.35	0.00414287148010344\\
62.36	0.00414435099129459\\
62.37	0.00414583111940467\\
62.38	0.00414731186431607\\
62.39	0.00414879322590934\\
62.4	0.0041502752040632\\
62.41	0.00415175779865448\\
62.42	0.00415324100955821\\
62.43	0.0041547248366475\\
62.44	0.00415620927979363\\
62.45	0.00415769433886598\\
62.46	0.00415918001373204\\
62.47	0.00416066630425741\\
62.48	0.00416215321030578\\
62.49	0.00416364073173895\\
62.5	0.00416512886841677\\
62.51	0.00416661762019716\\
62.52	0.00416810698693616\\
62.53	0.00416959696848779\\
62.54	0.00417108756470418\\
62.55	0.00417257877543548\\
62.56	0.00417407060052986\\
62.57	0.00417556303983352\\
62.58	0.0041770560931907\\
62.59	0.00417854976044361\\
62.6	0.0041800440414325\\
62.61	0.00418153893599559\\
62.62	0.00418303444396909\\
62.63	0.00418453056518717\\
62.64	0.00418602729948198\\
62.65	0.00418752464668365\\
62.66	0.00418902260662023\\
62.67	0.00419052117911772\\
62.68	0.00419202036400006\\
62.69	0.00419352016108912\\
62.7	0.00419502057020468\\
62.71	0.00419652159116441\\
62.72	0.00419802322378393\\
62.73	0.00419952546787671\\
62.74	0.00420102832325411\\
62.75	0.00420253178972538\\
62.76	0.00420403586709762\\
62.77	0.0042055405551758\\
62.78	0.00420704585376272\\
62.79	0.00420855176265904\\
62.8	0.00421005828166324\\
62.81	0.00421156541057164\\
62.82	0.00421307314917833\\
62.83	0.00421458149727526\\
62.84	0.00421609045465214\\
62.85	0.00421760002109647\\
62.86	0.00421911019639355\\
62.87	0.00422062098032642\\
62.88	0.00422213237267589\\
62.89	0.00422364437322054\\
62.9	0.00422515698173667\\
62.91	0.00422667019799832\\
62.92	0.00422818402177724\\
62.93	0.00422969845284293\\
62.94	0.00423121349096255\\
62.95	0.00423272913590099\\
62.96	0.00423424538742082\\
62.97	0.00423576224528226\\
62.98	0.00423727970924324\\
62.99	0.00423879777905933\\
63	0.00424031645448374\\
63.01	0.00424183573526733\\
63.02	0.00424335562115859\\
63.03	0.00424487611190363\\
63.04	0.00424639720724616\\
63.05	0.00424791890692752\\
63.06	0.00424944121068662\\
63.07	0.00425096411825995\\
63.08	0.00425248762938159\\
63.09	0.00425401174378316\\
63.1	0.00425553646119387\\
63.11	0.00425706178134042\\
63.12	0.0042585877039471\\
63.13	0.0042601142287357\\
63.14	0.00426164135542551\\
63.15	0.00426316908373332\\
63.16	0.00426469741337347\\
63.17	0.00426622634405772\\
63.18	0.00426775587549533\\
63.19	0.00426928600739304\\
63.2	0.00427081673945502\\
63.21	0.00427234807138289\\
63.22	0.00427388000287571\\
63.23	0.00427541253362995\\
63.24	0.00427694566333952\\
63.25	0.00427847939169571\\
63.26	0.00428001371838721\\
63.27	0.0042815486431001\\
63.28	0.00428308416551781\\
63.29	0.00428462028532116\\
63.3	0.0042861570021883\\
63.31	0.00428769431579474\\
63.32	0.00428923222581331\\
63.33	0.00429077073191415\\
63.34	0.00429230983376472\\
63.35	0.00429384953102979\\
63.36	0.0042953898233714\\
63.37	0.00429693071044888\\
63.38	0.00429847219191883\\
63.39	0.00430001426743507\\
63.4	0.00430155693664872\\
63.41	0.0043031001992081\\
63.42	0.00430464405475876\\
63.43	0.00430618850294345\\
63.44	0.00430773354340216\\
63.45	0.00430927917577205\\
63.46	0.00431082539968743\\
63.47	0.00431237221477982\\
63.48	0.00431391962067789\\
63.49	0.00431546761700744\\
63.5	0.00431701620339143\\
63.51	0.00431856537944992\\
63.52	0.0043201151448001\\
63.53	0.00432166549905626\\
63.54	0.00432321644182976\\
63.55	0.00432476797272908\\
63.56	0.00432632009135972\\
63.57	0.00432787279732428\\
63.58	0.00432942609022238\\
63.59	0.00433097996965068\\
63.6	0.00433253443520287\\
63.61	0.00433408948646961\\
63.62	0.00433564512303861\\
63.63	0.00433720134449454\\
63.64	0.00433875815041904\\
63.65	0.00434031554039071\\
63.66	0.00434187351398511\\
63.67	0.00434343207077474\\
63.68	0.00434499121032901\\
63.69	0.00434655093221425\\
63.7	0.0043481112359937\\
63.71	0.00434967212122747\\
63.72	0.00435123358747255\\
63.73	0.00435279563428281\\
63.74	0.00435435826120896\\
63.75	0.00435592146779853\\
63.76	0.00435748525359592\\
63.77	0.00435904961814229\\
63.78	0.00436061456097565\\
63.79	0.00436218008163076\\
63.8	0.00436374617963918\\
63.81	0.00436531285452922\\
63.82	0.00436688010582596\\
63.83	0.00436844793305117\\
63.84	0.00437001633572339\\
63.85	0.00437158531335786\\
63.86	0.00437315486546652\\
63.87	0.00437472499155796\\
63.88	0.00437629569113749\\
63.89	0.00437786696370706\\
63.9	0.00437943880876526\\
63.91	0.00438101122580731\\
63.92	0.00438258421432506\\
63.93	0.00438415777380696\\
63.94	0.00438573190373805\\
63.95	0.00438730660359996\\
63.96	0.00438888187287086\\
63.97	0.00439045771102549\\
63.98	0.00439203411753512\\
63.99	0.00439361109186757\\
64	0.00439518863348715\\
64.01	0.00439676674185464\\
64.02	0.00439834541642734\\
64.03	0.00439992465665901\\
64.04	0.00440150446199987\\
64.05	0.00440308483189656\\
64.06	0.00440466576579216\\
64.07	0.00440624726312618\\
64.08	0.00440782932333449\\
64.09	0.00440941194584939\\
64.1	0.00441099513009951\\
64.11	0.00441257887550986\\
64.12	0.00441416318150178\\
64.13	0.00441574804749295\\
64.14	0.00441733347289735\\
64.15	0.00441891945712525\\
64.16	0.00442050599958324\\
64.17	0.00442209309967413\\
64.18	0.00442368075679703\\
64.19	0.00442526897034726\\
64.2	0.00442685773971636\\
64.21	0.00442844706429213\\
64.22	0.0044300369434585\\
64.23	0.00443162737659561\\
64.24	0.00443321836307976\\
64.25	0.00443480990228342\\
64.26	0.00443640199357517\\
64.27	0.00443799463631972\\
64.28	0.00443958782987788\\
64.29	0.00444118157360655\\
64.3	0.0044427758668587\\
64.31	0.00444437070898337\\
64.32	0.00444596609932563\\
64.33	0.00444756203722659\\
64.34	0.00444915852202335\\
64.35	0.00445075555304904\\
64.36	0.00445235312963273\\
64.37	0.00445395125109949\\
64.38	0.00445554991677032\\
64.39	0.00445714912596214\\
64.4	0.00445874887798783\\
64.41	0.00446034917215612\\
64.42	0.00446195000777166\\
64.43	0.00446355138413495\\
64.44	0.00446515330054235\\
64.45	0.00446675575628605\\
64.46	0.00446835875065407\\
64.47	0.00446996228293022\\
64.48	0.00447156635239412\\
64.49	0.00447317095832112\\
64.5	0.00447477609998237\\
64.51	0.00447638177664472\\
64.52	0.00447798798757076\\
64.53	0.00447959473201878\\
64.54	0.00448120200924276\\
64.55	0.00448280981849233\\
64.56	0.0044844181590128\\
64.57	0.0044860270300451\\
64.58	0.00448763643082579\\
64.59	0.00448924636058701\\
64.6	0.00449085681855652\\
64.61	0.0044924678039576\\
64.62	0.00449407931600913\\
64.63	0.00449569135392548\\
64.64	0.00449730391691656\\
64.65	0.00449891700418777\\
64.66	0.00450053061493998\\
64.67	0.00450214474836956\\
64.68	0.00450375940366828\\
64.69	0.00450537458002333\\
64.7	0.00450699027661738\\
64.71	0.0045086064926284\\
64.72	0.0045102232272298\\
64.73	0.00451184047959031\\
64.74	0.004513458248874\\
64.75	0.00451507653424028\\
64.76	0.00451669533484383\\
64.77	0.00451831464983463\\
64.78	0.00451993447835791\\
64.79	0.00452155481955417\\
64.8	0.00452317567255912\\
64.81	0.00452479703650367\\
64.82	0.00452641891051391\\
64.83	0.00452804129371113\\
64.84	0.00452966418521176\\
64.85	0.00453128758412735\\
64.86	0.00453291148956458\\
64.87	0.00453453590062521\\
64.88	0.00453616081640609\\
64.89	0.0045377862359991\\
64.9	0.00453941215849118\\
64.91	0.00454103858296429\\
64.92	0.00454266550849535\\
64.93	0.00454429293415631\\
64.94	0.00454592085901406\\
64.95	0.00454754928213041\\
64.96	0.00454917820256209\\
64.97	0.00455080761936077\\
64.98	0.00455243753157294\\
64.99	0.00455406793824001\\
65	0.00455569883839817\\
65.01	0.00455733023107847\\
65.02	0.00455896211530676\\
65.03	0.00456059449010366\\
65.04	0.00456222735448453\\
65.05	0.00456386070745948\\
65.06	0.00456549454803337\\
65.07	0.0045671288752057\\
65.08	0.00456876368797068\\
65.09	0.00457039898531717\\
65.1	0.00457203476622868\\
65.11	0.00457367102968331\\
65.12	0.00457530777465377\\
65.13	0.00457694500010732\\
65.14	0.00457858270500579\\
65.15	0.00458022088830554\\
65.16	0.00458185954895744\\
65.17	0.00458349868590682\\
65.18	0.00458513829809352\\
65.19	0.00458677838445178\\
65.2	0.0045884189439103\\
65.21	0.00459005997539215\\
65.22	0.0045917014778148\\
65.23	0.00459334345009006\\
65.24	0.00459498589112409\\
65.25	0.00459662879981736\\
65.26	0.00459827217506463\\
65.27	0.00459991601575491\\
65.28	0.0046015603207715\\
65.29	0.00460320508899188\\
65.3	0.00460485031928776\\
65.31	0.00460649601052501\\
65.32	0.00460814216156367\\
65.33	0.00460978877125789\\
65.34	0.00461143583845599\\
65.35	0.0046130833620003\\
65.36	0.00461473134072728\\
65.37	0.0046163797734674\\
65.38	0.00461802865904513\\
65.39	0.004619677996279\\
65.4	0.00462132778398144\\
65.41	0.00462297802095889\\
65.42	0.00462462870601166\\
65.43	0.00462627983793401\\
65.44	0.00462793141551405\\
65.45	0.00462958343753375\\
65.46	0.00463123590276892\\
65.47	0.00463288880998915\\
65.48	0.00463454215795787\\
65.49	0.00463619594543219\\
65.5	0.00463785017116302\\
65.51	0.00463950483389495\\
65.52	0.00464115993236626\\
65.53	0.00464281546530888\\
65.54	0.00464447143144842\\
65.55	0.00464612782950403\\
65.56	0.00464778465818853\\
65.57	0.00464944191620822\\
65.58	0.004651099602263\\
65.59	0.00465275771504624\\
65.6	0.00465441625324484\\
65.61	0.00465607521553913\\
65.62	0.00465773460060288\\
65.63	0.00465939440710327\\
65.64	0.00466105463370087\\
65.65	0.00466271527904962\\
65.66	0.00466437634179678\\
65.67	0.00466603782058291\\
65.68	0.00466769971404189\\
65.69	0.0046693620208008\\
65.7	0.00467102473948\\
65.71	0.00467268786869304\\
65.72	0.00467435140704662\\
65.73	0.00467601535314063\\
65.74	0.00467767970556807\\
65.75	0.00467934446291501\\
65.76	0.00468100962376063\\
65.77	0.00468267518667716\\
65.78	0.0046843411502298\\
65.79	0.00468600751297679\\
65.8	0.00468767427346929\\
65.81	0.00468934143025143\\
65.82	0.00469100898186024\\
65.83	0.00469267692682561\\
65.84	0.00469434526367032\\
65.85	0.00469601399090994\\
65.86	0.00469768310705287\\
65.87	0.00469935261060025\\
65.88	0.00470102250004598\\
65.89	0.00470269277387668\\
65.9	0.00470436343057164\\
65.91	0.00470603446860281\\
65.92	0.00470770588643477\\
65.93	0.00470937768252472\\
65.94	0.00471104985532242\\
65.95	0.00471272240327015\\
65.96	0.00471439532480272\\
65.97	0.00471606861834745\\
65.98	0.00471774228232409\\
65.99	0.00471941631514482\\
66	0.0047210907152142\\
66.01	0.00472276548092922\\
66.02	0.00472444061067914\\
66.03	0.00472611610284557\\
66.04	0.00472779195580239\\
66.05	0.00472946816791572\\
66.06	0.00473114473754391\\
66.07	0.00473282166303748\\
66.08	0.00473449894273916\\
66.09	0.00473617657498376\\
66.1	0.00473785455809819\\
66.11	0.00473953289040146\\
66.12	0.0047412115702046\\
66.13	0.00474289059581064\\
66.14	0.0047445699655146\\
66.15	0.00474624967760344\\
66.16	0.00474792973035602\\
66.17	0.00474961012204311\\
66.18	0.00475129085092729\\
66.19	0.00475297191526301\\
66.2	0.00475465331329648\\
66.21	0.00475633504326566\\
66.22	0.00475801710340024\\
66.23	0.00475969949192163\\
66.24	0.00476138220704286\\
66.25	0.00476306524696861\\
66.26	0.00476474860989515\\
66.27	0.00476643229401031\\
66.28	0.00476811629749347\\
66.29	0.00476980061851548\\
66.3	0.00477148525523868\\
66.31	0.00477317020581683\\
66.32	0.00477485546839508\\
66.33	0.00477654104110997\\
66.34	0.00477822692208935\\
66.35	0.00477991310945239\\
66.36	0.00478159960130951\\
66.37	0.00478328639576236\\
66.38	0.0047849734909038\\
66.39	0.00478666088481786\\
66.4	0.00478834857557969\\
66.41	0.00479003656125551\\
66.42	0.00479172483990266\\
66.43	0.00479341340956944\\
66.44	0.00479510226829521\\
66.45	0.00479679141411024\\
66.46	0.00479848084503574\\
66.47	0.00480017055908379\\
66.48	0.00480186055425735\\
66.49	0.00480355082855019\\
66.5	0.00480524137994685\\
66.51	0.00480693220642263\\
66.52	0.00480862330594353\\
66.53	0.00481031467646622\\
66.54	0.00481200631593803\\
66.55	0.00481369822229689\\
66.56	0.00481539039347126\\
66.57	0.00481708282738018\\
66.58	0.00481877552193315\\
66.59	0.00482046847503014\\
66.6	0.00482216168456154\\
66.61	0.00482385514840813\\
66.62	0.00482554886444102\\
66.63	0.00482724283052165\\
66.64	0.00482893704450171\\
66.65	0.00483063150422313\\
66.66	0.00483232620751805\\
66.67	0.00483402115220876\\
66.68	0.00483571633610766\\
66.69	0.00483741175701725\\
66.7	0.00483910741273008\\
66.71	0.00484080330102868\\
66.72	0.00484249941968557\\
66.73	0.00484419576646319\\
66.74	0.00484589233911387\\
66.75	0.0048475891353798\\
66.76	0.00484928615299298\\
66.77	0.00485098338967518\\
66.78	0.0048526808431379\\
66.79	0.00485437851108235\\
66.8	0.00485607639119937\\
66.81	0.00485777448116943\\
66.82	0.00485947277866258\\
66.83	0.00486117128133841\\
66.84	0.00486286998684597\\
66.85	0.00486456889282381\\
66.86	0.00486626799689985\\
66.87	0.00486796729669143\\
66.88	0.00486966678980517\\
66.89	0.00487136647383703\\
66.9	0.00487306634637219\\
66.91	0.00487476640498504\\
66.92	0.00487646664723915\\
66.93	0.00487816707068721\\
66.94	0.00487986767287099\\
66.95	0.0048815684513213\\
66.96	0.00488326940355795\\
66.97	0.00488497052708971\\
66.98	0.00488667181941429\\
66.99	0.00488837327801822\\
67	0.0048900749003769\\
67.01	0.00489177668395449\\
67.02	0.00489347862620392\\
67.03	0.00489518072456681\\
67.04	0.00489688297647341\\
67.05	0.00489858537934263\\
67.06	0.00490028793058192\\
67.07	0.00490199062758725\\
67.08	0.00490369346774311\\
67.09	0.00490539644842237\\
67.1	0.00490709956698634\\
67.11	0.00490880282078466\\
67.12	0.00491050620715527\\
67.13	0.00491220972342438\\
67.14	0.0049139133669064\\
67.15	0.00491561713490393\\
67.16	0.00491732102470765\\
67.17	0.00491902503359638\\
67.18	0.00492072915883692\\
67.19	0.00492243339768408\\
67.2	0.0049241377473806\\
67.21	0.00492584220515712\\
67.22	0.00492754676823213\\
67.23	0.0049292514338119\\
67.24	0.00493095619909047\\
67.25	0.0049326610612496\\
67.26	0.00493436601745869\\
67.27	0.00493607106487474\\
67.28	0.00493777620064235\\
67.29	0.00493948142189363\\
67.3	0.00494118672574811\\
67.31	0.00494289210931281\\
67.32	0.0049445975696821\\
67.33	0.00494630310393764\\
67.34	0.00494800870914842\\
67.35	0.00494971438237062\\
67.36	0.00495142012064761\\
67.37	0.0049531259210099\\
67.38	0.00495483178047505\\
67.39	0.0049565376960477\\
67.4	0.00495824366471943\\
67.41	0.00495994968346875\\
67.42	0.00496165574926109\\
67.43	0.00496336185904869\\
67.44	0.00496506800977055\\
67.45	0.00496677419835242\\
67.46	0.00496848042170675\\
67.47	0.00497018667673258\\
67.48	0.00497189296031555\\
67.49	0.00497359926932783\\
67.5	0.00497530560062805\\
67.51	0.00497701195106126\\
67.52	0.00497871831745891\\
67.53	0.00498042469663874\\
67.54	0.00498213108540478\\
67.55	0.00498383748054725\\
67.56	0.00498554387884256\\
67.57	0.00498725027705321\\
67.58	0.00498895667192774\\
67.59	0.00499066306020074\\
67.6	0.00499236943859271\\
67.61	0.00499407580381007\\
67.62	0.00499578215254506\\
67.63	0.00499748848147573\\
67.64	0.00499919478726585\\
67.65	0.00500090106656487\\
67.66	0.00500260731600788\\
67.67	0.00500431353221552\\
67.68	0.00500601971179396\\
67.69	0.00500772585133482\\
67.7	0.00500943194741513\\
67.71	0.00501113799659728\\
67.72	0.00501284399542894\\
67.73	0.00501454994044302\\
67.74	0.00501625582815761\\
67.75	0.00501796165507593\\
67.76	0.00501966741768625\\
67.77	0.00502137311246189\\
67.78	0.00502307873586109\\
67.79	0.005024784284327\\
67.8	0.00502648975428762\\
67.81	0.0050281951421557\\
67.82	0.00502990044432876\\
67.83	0.00503160565718895\\
67.84	0.00503331077710305\\
67.85	0.00503501580042238\\
67.86	0.00503672072348275\\
67.87	0.0050384255426044\\
67.88	0.00504013025409196\\
67.89	0.00504183485423436\\
67.9	0.00504353933930479\\
67.91	0.00504524370556062\\
67.92	0.00504694794924338\\
67.93	0.00504865206657866\\
67.94	0.00505035605377607\\
67.95	0.00505205990702917\\
67.96	0.00505376362251542\\
67.97	0.00505546719639609\\
67.98	0.00505717062481626\\
67.99	0.00505887390390469\\
68	0.00506057702977379\\
68.01	0.00506227999851957\\
68.02	0.00506398280622156\\
68.03	0.00506568544894274\\
68.04	0.0050673879227295\\
68.05	0.00506909022361157\\
68.06	0.00507079234760195\\
68.07	0.00507249429069684\\
68.08	0.00507419604887561\\
68.09	0.00507589761810069\\
68.1	0.00507759899431755\\
68.11	0.0050793001734546\\
68.12	0.00508100104626924\\
68.13	0.00508270154588124\\
68.14	0.00508440166853683\\
68.15	0.00508610141046893\\
68.16	0.00508780076789721\\
68.17	0.005089499737028\\
68.18	0.00509119831405421\\
68.19	0.00509289649515535\\
68.2	0.00509459427649746\\
68.21	0.00509629165423306\\
68.22	0.00509798862450112\\
68.23	0.005099685183427\\
68.24	0.00510138132712244\\
68.25	0.00510307705168545\\
68.26	0.00510477235320034\\
68.27	0.00510646722773765\\
68.28	0.00510816167135408\\
68.29	0.00510985568009249\\
68.3	0.00511154924998179\\
68.31	0.00511324237703699\\
68.32	0.00511493505725906\\
68.33	0.00511662728663496\\
68.34	0.00511831906113752\\
68.35	0.0051200103767255\\
68.36	0.00512170122934343\\
68.37	0.00512339161492164\\
68.38	0.00512508152937619\\
68.39	0.00512677096860882\\
68.4	0.00512845992850693\\
68.41	0.0051301484049435\\
68.42	0.00513183639377707\\
68.43	0.00513352389085167\\
68.44	0.0051352108919968\\
68.45	0.00513689739302739\\
68.46	0.00513858338974369\\
68.47	0.00514026887793131\\
68.48	0.00514195385336113\\
68.49	0.00514363831178923\\
68.5	0.00514532224895691\\
68.51	0.00514700566059057\\
68.52	0.00514868854240172\\
68.53	0.00515037089008689\\
68.54	0.00515205269932763\\
68.55	0.0051537339657904\\
68.56	0.00515541468512659\\
68.57	0.00515709485297244\\
68.58	0.00515877446494897\\
68.59	0.00516045351666198\\
68.6	0.00516213200370198\\
68.61	0.00516380992164411\\
68.62	0.00516548726604815\\
68.63	0.00516716403245846\\
68.64	0.00516884021640389\\
68.65	0.00517051581339776\\
68.66	0.00517219081893783\\
68.67	0.00517386522850621\\
68.68	0.00517553903756935\\
68.69	0.00517721224157798\\
68.7	0.00517888483596703\\
68.71	0.00518055681615565\\
68.72	0.00518222817754709\\
68.73	0.00518389891552867\\
68.74	0.00518556902547177\\
68.75	0.00518723850273175\\
68.76	0.00518890734264788\\
68.77	0.00519057554054334\\
68.78	0.00519224309172513\\
68.79	0.00519390999148403\\
68.8	0.00519557623509458\\
68.81	0.00519724181781497\\
68.82	0.00519890673488707\\
68.83	0.00520057098153631\\
68.84	0.00520223455297164\\
68.85	0.00520389744438553\\
68.86	0.00520555965095388\\
68.87	0.00520722116783596\\
68.88	0.00520888199017439\\
68.89	0.00521054211309506\\
68.9	0.00521220153170713\\
68.91	0.00521386024110291\\
68.92	0.00521551823635785\\
68.93	0.00521717551253052\\
68.94	0.00521883206411153\\
68.95	0.00522048788547248\\
68.96	0.00522214297096265\\
68.97	0.00522379731490887\\
68.98	0.00522545091161548\\
68.99	0.00522710375536423\\
69	0.0052287558404142\\
69.01	0.00523040716100169\\
69.02	0.00523205771134017\\
69.03	0.00523370748562013\\
69.04	0.00523535647800908\\
69.05	0.00523700468265137\\
69.06	0.00523865209366816\\
69.07	0.00524029870515731\\
69.08	0.0052419445111933\\
69.09	0.00524358950582711\\
69.1	0.00524523368308617\\
69.11	0.00524687703697424\\
69.12	0.00524851956147133\\
69.13	0.00525016125053364\\
69.14	0.00525180209809337\\
69.15	0.00525344209805875\\
69.16	0.00525508124431384\\
69.17	0.00525671953071854\\
69.18	0.00525835695110839\\
69.19	0.00525999349929457\\
69.2	0.00526162916906375\\
69.21	0.00526326395417798\\
69.22	0.00526489784837468\\
69.23	0.00526653084536644\\
69.24	0.00526816293884098\\
69.25	0.00526979412246106\\
69.26	0.00527142438986439\\
69.27	0.00527305373466346\\
69.28	0.00527468215044553\\
69.29	0.00527630963077249\\
69.3	0.00527793616918077\\
69.31	0.00527956175918123\\
69.32	0.00528118639425908\\
69.33	0.00528281006787378\\
69.34	0.00528443277345891\\
69.35	0.00528605450442211\\
69.36	0.00528767525414495\\
69.37	0.00528929501598284\\
69.38	0.00529091378326495\\
69.39	0.00529253154929404\\
69.4	0.00529414830734647\\
69.41	0.00529576405067196\\
69.42	0.00529737877249362\\
69.43	0.00529899246600774\\
69.44	0.00530060512438375\\
69.45	0.00530221674076412\\
69.46	0.00530382730826418\\
69.47	0.00530543681997211\\
69.48	0.00530704526894878\\
69.49	0.00530865264822764\\
69.5	0.00531025895081465\\
69.51	0.00531186416968815\\
69.52	0.00531346829779874\\
69.53	0.00531507132806919\\
69.54	0.00531667325339434\\
69.55	0.00531827406664098\\
69.56	0.00531987376064774\\
69.57	0.00532147232822496\\
69.58	0.00532306976215464\\
69.59	0.00532466605519026\\
69.6	0.00532626120005671\\
69.61	0.0053278551894502\\
69.62	0.00532944801603806\\
69.63	0.00533103967245872\\
69.64	0.00533263015132159\\
69.65	0.00533421944520686\\
69.66	0.00533580754666549\\
69.67	0.00533739444821903\\
69.68	0.00533898014235956\\
69.69	0.00534056462154951\\
69.7	0.00534214787822158\\
69.71	0.00534372990477865\\
69.72	0.00534531069359363\\
69.73	0.00534689023700932\\
69.74	0.00534846852733836\\
69.75	0.00535004555686304\\
69.76	0.00535162131783525\\
69.77	0.00535319580247631\\
69.78	0.00535476900297686\\
69.79	0.00535634091149679\\
69.8	0.00535791152016502\\
69.81	0.0053594808210795\\
69.82	0.00536104880630697\\
69.83	0.00536261546788295\\
69.84	0.00536418079781152\\
69.85	0.00536574478806527\\
69.86	0.00536730743058515\\
69.87	0.00536886871728031\\
69.88	0.00537042864002807\\
69.89	0.00537198719067368\\
69.9	0.0053735443610303\\
69.91	0.00537510014287879\\
69.92	0.00537665452796764\\
69.93	0.00537820750801282\\
69.94	0.00537975968419097\\
69.95	0.00538131217004729\\
69.96	0.00538286496456235\\
69.97	0.00538441806671286\\
69.98	0.00538597147547152\\
69.99	0.00538752518980714\\
70	0.00538907920868453\\
70.01	0.00539063353106456\\
70.02	0.0053921881559041\\
70.03	0.00539374308215602\\
70.04	0.00539529830876918\\
70.05	0.00539685383468842\\
70.06	0.00539840965885453\\
70.07	0.00539996578020428\\
70.08	0.00540152219767035\\
70.09	0.00540307891018136\\
70.1	0.00540463591666182\\
70.11	0.00540619321603217\\
70.12	0.00540775080720871\\
70.13	0.00540930868910364\\
70.14	0.005410866860625\\
70.15	0.00541242532067666\\
70.16	0.00541398406815836\\
70.17	0.00541554310196564\\
70.18	0.00541710242098985\\
70.19	0.00541866202411811\\
70.2	0.00542022191023337\\
70.21	0.00542178207821431\\
70.22	0.00542334252693535\\
70.23	0.00542490325526668\\
70.24	0.00542646426207422\\
70.25	0.00542802554621956\\
70.26	0.00542958710656003\\
70.27	0.00543114894194862\\
70.28	0.005432711051234\\
70.29	0.00543427343326049\\
70.3	0.00543583608686807\\
70.31	0.00543739901089232\\
70.32	0.00543896220416446\\
70.33	0.00544052566551131\\
70.34	0.00544208939375524\\
70.35	0.00544365338771425\\
70.36	0.00544521764620186\\
70.37	0.00544678216802714\\
70.38	0.0054483469519947\\
70.39	0.00544991199690464\\
70.4	0.00545147730155261\\
70.41	0.0054530428647297\\
70.42	0.00545460868522249\\
70.43	0.00545617476181303\\
70.44	0.00545774109327878\\
70.45	0.00545930767839267\\
70.46	0.005460874515923\\
70.47	0.00546244160463352\\
70.48	0.00546400894328333\\
70.49	0.00546557653062689\\
70.5	0.00546714436541405\\
70.51	0.00546871244638998\\
70.52	0.00547028077229517\\
70.53	0.00547184934186544\\
70.54	0.00547341815383188\\
70.55	0.00547498720692088\\
70.56	0.00547655649985408\\
70.57	0.00547812603134839\\
70.58	0.00547969580011593\\
70.59	0.00548126580486406\\
70.6	0.00548283604429532\\
70.61	0.00548440651710745\\
70.62	0.00548597722199339\\
70.63	0.0054875481576412\\
70.64	0.00548911932273408\\
70.65	0.00549069071595039\\
70.66	0.00549226233596357\\
70.67	0.00549383418144216\\
70.68	0.00549540625104979\\
70.69	0.00549697854344514\\
70.7	0.00549855105728195\\
70.71	0.00550012379120898\\
70.72	0.00550169674387002\\
70.73	0.00550326991390385\\
70.74	0.00550484329994421\\
70.75	0.00550641690061984\\
70.76	0.00550799071455443\\
70.77	0.00550956474036658\\
70.78	0.00551113897666983\\
70.79	0.00551271342207259\\
70.8	0.00551428807517822\\
70.81	0.00551586293458486\\
70.82	0.00551743799888559\\
70.83	0.00551901326666825\\
70.84	0.00552058873651556\\
70.85	0.00552216440700499\\
70.86	0.00552374027670883\\
70.87	0.00552531634419413\\
70.88	0.00552689260802268\\
70.89	0.00552846906675102\\
70.9	0.00553004571893041\\
70.91	0.0055316225631068\\
70.92	0.00553319959782081\\
70.93	0.00553477682160775\\
70.94	0.00553635423299757\\
70.95	0.00553793183051485\\
70.96	0.00553950961267878\\
70.97	0.00554108757800315\\
70.98	0.00554266572499635\\
70.99	0.00554424405216136\\
71	0.00554582255799575\\
71.01	0.00554740124099169\\
71.02	0.00554898009963589\\
71.03	0.00555055913240961\\
71.04	0.00555213833778862\\
71.05	0.00555371771424322\\
71.06	0.00555529726023818\\
71.07	0.00555687697423275\\
71.08	0.00555845685468062\\
71.09	0.00556003690002994\\
71.1	0.00556161710872325\\
71.11	0.00556319747919753\\
71.12	0.0055647780098841\\
71.13	0.00556635869920866\\
71.14	0.00556793954559128\\
71.15	0.00556952054744633\\
71.16	0.00557110170318248\\
71.17	0.00557268301120274\\
71.18	0.00557426446990434\\
71.19	0.0055758460776788\\
71.2	0.00557742783291187\\
71.21	0.00557900973398353\\
71.22	0.00558059177926792\\
71.23	0.0055821739671334\\
71.24	0.00558375629594249\\
71.25	0.00558533876405184\\
71.26	0.00558692136981222\\
71.27	0.00558850411156854\\
71.28	0.00559008698765977\\
71.29	0.00559166999641895\\
71.3	0.00559325313617318\\
71.31	0.00559483640524359\\
71.32	0.00559641980194532\\
71.33	0.0055980033245875\\
71.34	0.00559958697147322\\
71.35	0.00560117074089957\\
71.36	0.00560275463115752\\
71.37	0.00560433864053198\\
71.38	0.00560592276730177\\
71.39	0.00560750700973957\\
71.4	0.00560909136611189\\
71.41	0.00561067583467915\\
71.42	0.0056122604136955\\
71.43	0.00561384510140894\\
71.44	0.00561542989606125\\
71.45	0.00561701479588794\\
71.46	0.00561859979911827\\
71.47	0.00562018490397522\\
71.48	0.00562177010867546\\
71.49	0.00562335541142935\\
71.5	0.00562494081044088\\
71.51	0.0056265263039077\\
71.52	0.00562811189002108\\
71.53	0.00562969756696585\\
71.54	0.00563128333292044\\
71.55	0.00563286918605685\\
71.56	0.00563445512454056\\
71.57	0.00563604114653062\\
71.58	0.00563762725017953\\
71.59	0.00563921343363328\\
71.6	0.00564079969503131\\
71.61	0.00564238603250647\\
71.62	0.00564397244418503\\
71.63	0.00564555892818665\\
71.64	0.00564714548262434\\
71.65	0.00564873210560446\\
71.66	0.00565031879522669\\
71.67	0.00565190554958402\\
71.68	0.00565349236676271\\
71.69	0.00565507924484226\\
71.7	0.00565666618189542\\
71.71	0.00565825317598816\\
71.72	0.00565984022517964\\
71.73	0.00566142732752217\\
71.74	0.00566301448106121\\
71.75	0.00566460168383537\\
71.76	0.00566618893387633\\
71.77	0.00566777622920887\\
71.78	0.00566936356785082\\
71.79	0.00567095094781303\\
71.8	0.00567253836709942\\
71.81	0.00567412582370682\\
71.82	0.00567571331562508\\
71.83	0.00567730084083699\\
71.84	0.00567888839731824\\
71.85	0.00568047598303742\\
71.86	0.00568206359595604\\
71.87	0.00568365123402839\\
71.88	0.00568523889520165\\
71.89	0.00568682657741578\\
71.9	0.00568841427860354\\
71.91	0.0056900019966904\\
71.92	0.00569158972959464\\
71.93	0.00569317747522719\\
71.94	0.00569476523149171\\
71.95	0.00569635299628449\\
71.96	0.00569794076749449\\
71.97	0.00569952854300328\\
71.98	0.00570111632068501\\
71.99	0.0057027040984064\\
72	0.00570429187402673\\
72.01	0.00570587964539781\\
72.02	0.0057074674103639\\
72.03	0.00570905516676178\\
72.04	0.00571064291242066\\
72.05	0.00571223064516216\\
72.06	0.00571381836280031\\
72.07	0.00571540606314153\\
72.08	0.00571699374398455\\
72.09	0.00571858140312046\\
72.1	0.00572016903833263\\
72.11	0.00572175664739671\\
72.12	0.00572334422808059\\
72.13	0.00572493177814439\\
72.14	0.00572651929534043\\
72.15	0.00572810677741319\\
72.16	0.00572969422209932\\
72.17	0.00573128162712755\\
72.18	0.00573286899021876\\
72.19	0.00573445630908585\\
72.2	0.00573604358143379\\
72.21	0.00573763080495957\\
72.22	0.00573921797735214\\
72.23	0.00574080509629247\\
72.24	0.00574239215945341\\
72.25	0.00574397916449976\\
72.26	0.0057455661090882\\
72.27	0.00574715299086726\\
72.28	0.0057487398074773\\
72.29	0.00575032655655051\\
72.3	0.00575191323571083\\
72.31	0.00575349984257397\\
72.32	0.00575508637474737\\
72.33	0.00575667282983014\\
72.34	0.0057582592054131\\
72.35	0.00575984549907868\\
72.36	0.00576143170840094\\
72.37	0.00576301783094553\\
72.38	0.00576460386426967\\
72.39	0.00576618980592208\\
72.4	0.00576777565344302\\
72.41	0.00576936140436423\\
72.42	0.00577094705620886\\
72.43	0.00577253260649151\\
72.44	0.00577411805271819\\
72.45	0.00577570339238624\\
72.46	0.00577728862298435\\
72.47	0.00577887374199252\\
72.48	0.00578045874688202\\
72.49	0.0057820436351154\\
72.5	0.0057836284041464\\
72.51	0.00578521305141996\\
72.52	0.00578679757437218\\
72.53	0.00578838197043032\\
72.54	0.00578996623701271\\
72.55	0.00579155037152878\\
72.56	0.00579313437137901\\
72.57	0.00579471823395486\\
72.58	0.00579630195663883\\
72.59	0.00579788553680434\\
72.6	0.00579946897181575\\
72.61	0.00580105225902832\\
72.62	0.00580263539578816\\
72.63	0.00580421837943225\\
72.64	0.00580580120728834\\
72.65	0.00580738387667499\\
72.66	0.00580896638490148\\
72.67	0.00581054872926782\\
72.68	0.0058121309070647\\
72.69	0.00581371291557346\\
72.7	0.00581529475206608\\
72.71	0.00581687641380511\\
72.72	0.00581845789804366\\
72.73	0.00582003920202539\\
72.74	0.00582162032298444\\
72.75	0.00582320125814542\\
72.76	0.00582478200472338\\
72.77	0.00582636255992375\\
72.78	0.00582794292094237\\
72.79	0.00582952308496539\\
72.8	0.00583110304916927\\
72.81	0.00583268281072074\\
72.82	0.00583426236677679\\
72.83	0.0058358417144846\\
72.84	0.00583742085098154\\
72.85	0.00583899977339511\\
72.86	0.00584057847884294\\
72.87	0.00584215696443271\\
72.88	0.00584373522726218\\
72.89	0.00584531326441909\\
72.9	0.00584689107298117\\
72.91	0.0058484686500161\\
72.92	0.00585004599258147\\
72.93	0.00585162309772473\\
72.94	0.0058531999624832\\
72.95	0.005854776583884\\
72.96	0.00585635295894401\\
72.97	0.00585792908466986\\
72.98	0.0058595049580579\\
72.99	0.00586108057609414\\
73	0.00586265593575422\\
73.01	0.0058642310340034\\
73.02	0.00586580586779651\\
73.03	0.00586738043407787\\
73.04	0.00586895472978135\\
73.05	0.00587052875183027\\
73.06	0.00587210249713736\\
73.07	0.00587367596260474\\
73.08	0.00587524914512392\\
73.09	0.00587682204157568\\
73.1	0.00587839464883012\\
73.11	0.00587996696374659\\
73.12	0.00588153898317361\\
73.13	0.00588311070394893\\
73.14	0.0058846821228994\\
73.15	0.005886253236841\\
73.16	0.00588782404257873\\
73.17	0.00588939453690668\\
73.18	0.00589096471660787\\
73.19	0.00589253457845432\\
73.2	0.00589410411920694\\
73.21	0.00589567333561554\\
73.22	0.00589724222441875\\
73.23	0.00589881078234401\\
73.24	0.00590037900610752\\
73.25	0.00590194689241424\\
73.26	0.00590351443795778\\
73.27	0.00590508163942041\\
73.28	0.00590664849347304\\
73.29	0.0059082149967751\\
73.3	0.00590978114597461\\
73.31	0.00591134693770805\\
73.32	0.00591291236860038\\
73.33	0.00591447743526496\\
73.34	0.00591604213430352\\
73.35	0.00591760646230617\\
73.36	0.00591917041585126\\
73.37	0.00592073399150546\\
73.38	0.0059222971858236\\
73.39	0.00592385999534873\\
73.4	0.00592542241661202\\
73.41	0.00592698444613275\\
73.42	0.00592854608041823\\
73.43	0.00593010731596381\\
73.44	0.00593166814925282\\
73.45	0.0059332285767565\\
73.46	0.00593478859493399\\
73.47	0.00593634820023229\\
73.48	0.00593790738908621\\
73.49	0.00593946615791831\\
73.5	0.00594102450313889\\
73.51	0.00594258242114593\\
73.52	0.00594413990832505\\
73.53	0.00594569696104946\\
73.54	0.00594725357567993\\
73.55	0.00594880974856477\\
73.56	0.00595036547603969\\
73.57	0.00595192075442791\\
73.58	0.00595347558003997\\
73.59	0.00595502994917379\\
73.6	0.00595658385811455\\
73.61	0.0059581373031347\\
73.62	0.00595969028049391\\
73.63	0.00596124278643899\\
73.64	0.00596279481720388\\
73.65	0.00596434636900959\\
73.66	0.00596589743806417\\
73.67	0.00596744802056262\\
73.68	0.00596899811268694\\
73.69	0.00597054771060596\\
73.7	0.00597209681047541\\
73.71	0.00597364540843777\\
73.72	0.00597519350062232\\
73.73	0.00597674108314504\\
73.74	0.00597828815210856\\
73.75	0.00597983470360213\\
73.76	0.0059813807337016\\
73.77	0.00598292623846929\\
73.78	0.00598447121395404\\
73.79	0.00598601565619111\\
73.8	0.00598755956120213\\
73.81	0.00598910292499508\\
73.82	0.0059906457435642\\
73.83	0.00599218801289\\
73.84	0.00599372972893917\\
73.85	0.00599527088766452\\
73.86	0.00599681148500497\\
73.87	0.0059983515168855\\
73.88	0.00599989097921704\\
73.89	0.00600142986789651\\
73.9	0.00600296817880671\\
73.91	0.00600450590781627\\
73.92	0.00600604305077962\\
73.93	0.00600757960353697\\
73.94	0.00600911556191417\\
73.95	0.00601065092172277\\
73.96	0.00601218567875987\\
73.97	0.00601371982880812\\
73.98	0.00601525336763569\\
73.99	0.00601678629099616\\
74	0.0060183185946285\\
74.01	0.00601985027425704\\
74.02	0.00602138132559136\\
74.03	0.00602291174432629\\
74.04	0.00602444152614185\\
74.05	0.00602597066670316\\
74.06	0.00602749916166044\\
74.07	0.00602902700664889\\
74.08	0.00603055419728871\\
74.09	0.006032080729185\\
74.1	0.00603360659792773\\
74.11	0.00603513179909164\\
74.12	0.00603665632823627\\
74.13	0.0060381801809058\\
74.14	0.00603970335262908\\
74.15	0.00604122583891954\\
74.16	0.00604274763527513\\
74.17	0.00604426873717828\\
74.18	0.00604578914009583\\
74.19	0.00604730883947899\\
74.2	0.00604882783076325\\
74.21	0.00605034610936839\\
74.22	0.00605186367069833\\
74.23	0.00605338051014115\\
74.24	0.00605489662306899\\
74.25	0.00605641200483802\\
74.26	0.00605792665078837\\
74.27	0.00605944055624405\\
74.28	0.00606095371651292\\
74.29	0.00606246612688664\\
74.3	0.00606397778264056\\
74.31	0.00606548867903373\\
74.32	0.00606699881130876\\
74.33	0.00606850817469185\\
74.34	0.00607001676439264\\
74.35	0.00607152457560424\\
74.36	0.00607303160350306\\
74.37	0.00607453784324887\\
74.38	0.00607604328998464\\
74.39	0.00607754793883654\\
74.4	0.00607905178491384\\
74.41	0.00608055482330886\\
74.42	0.00608205704909692\\
74.43	0.00608355845733626\\
74.44	0.00608505904306799\\
74.45	0.006086558801316\\
74.46	0.00608805772708695\\
74.47	0.00608955581537013\\
74.48	0.00609105306113747\\
74.49	0.00609254945934342\\
74.5	0.00609404500492493\\
74.51	0.00609553969280134\\
74.52	0.00609703351787432\\
74.53	0.00609852647502787\\
74.54	0.00610001855912816\\
74.55	0.0061015097650235\\
74.56	0.00610300008754433\\
74.57	0.00610448952150304\\
74.58	0.00610597806169398\\
74.59	0.00610746570289341\\
74.6	0.00610895243985936\\
74.61	0.0061104382673316\\
74.62	0.00611192318003157\\
74.63	0.00611340717266231\\
74.64	0.0061148902399084\\
74.65	0.00611637237643586\\
74.66	0.00611785357689211\\
74.67	0.00611933383590586\\
74.68	0.00612081314808711\\
74.69	0.00612229150802699\\
74.7	0.00612376891029775\\
74.71	0.00612524534945266\\
74.72	0.00612672082002596\\
74.73	0.00612819531653275\\
74.74	0.00612966883346895\\
74.75	0.00613114136531121\\
74.76	0.00613261290651684\\
74.77	0.00613408345152375\\
74.78	0.00613555299475032\\
74.79	0.00613702153059539\\
74.8	0.00613848905343816\\
74.81	0.00613995555763811\\
74.82	0.00614142103753489\\
74.83	0.00614288548744833\\
74.84	0.00614434890167826\\
74.85	0.0061458112745045\\
74.86	0.00614727260018677\\
74.87	0.00614873287296458\\
74.88	0.00615019208705719\\
74.89	0.00615165023666351\\
74.9	0.00615310731596201\\
74.91	0.00615456331911067\\
74.92	0.00615601824024685\\
74.93	0.00615747207348727\\
74.94	0.00615892481292787\\
74.95	0.00616037645264378\\
74.96	0.00616182698668917\\
74.97	0.00616327640909724\\
74.98	0.00616472471388009\\
74.99	0.00616617189502865\\
75	0.00616761794651257\\
75.01	0.0061690628622802\\
75.02	0.00617050663625841\\
75.03	0.00617194926235259\\
75.04	0.0061733907344465\\
75.05	0.00617483104640223\\
75.06	0.00617627019206008\\
75.07	0.00617770816523848\\
75.08	0.00617914495973389\\
75.09	0.00618058056932076\\
75.1	0.00618201498775135\\
75.11	0.00618344820875625\\
75.12	0.00618488022604447\\
75.13	0.00618631103330343\\
75.14	0.00618774062419882\\
75.15	0.00618916899237458\\
75.16	0.00619059613145279\\
75.17	0.00619202203503361\\
75.18	0.0061934466966952\\
75.19	0.00619487010999361\\
75.2	0.00619629226846278\\
75.21	0.00619771316561437\\
75.22	0.00619913279493777\\
75.23	0.00620055114989995\\
75.24	0.00620196822394542\\
75.25	0.00620338401049615\\
75.26	0.00620479850295147\\
75.27	0.00620621169468799\\
75.28	0.00620762357905958\\
75.29	0.00620903414939721\\
75.3	0.0062104433990089\\
75.31	0.00621185132117967\\
75.32	0.00621325790917141\\
75.33	0.00621466315622284\\
75.34	0.00621606705554938\\
75.35	0.00621746960034314\\
75.36	0.00621887078377275\\
75.37	0.00622027059898336\\
75.38	0.00622166903909652\\
75.39	0.00622306609721007\\
75.4	0.00622446176639811\\
75.41	0.00622585603971089\\
75.42	0.00622724891017471\\
75.43	0.00622864037079186\\
75.44	0.00623003041454055\\
75.45	0.00623141903437478\\
75.46	0.00623280622322428\\
75.47	0.00623419197399442\\
75.48	0.00623557627956615\\
75.49	0.00623695913279588\\
75.5	0.00623834052651538\\
75.51	0.00623972045353174\\
75.52	0.00624109970330925\\
75.53	0.00624247883365666\\
75.54	0.00624385784370188\\
75.55	0.00624523673257493\\
75.56	0.00624661549940799\\
75.57	0.00624799414333543\\
75.58	0.00624937266349384\\
75.59	0.00625075105902205\\
75.6	0.00625212932906117\\
75.61	0.00625350747275461\\
75.62	0.00625488548924812\\
75.63	0.00625626337768981\\
75.64	0.0062576411372302\\
75.65	0.00625901876702221\\
75.66	0.00626039626622126\\
75.67	0.00626177363398521\\
75.68	0.00626315086947447\\
75.69	0.00626452797185197\\
75.7	0.00626590494028327\\
75.71	0.0062672817739365\\
75.72	0.00626865847198246\\
75.73	0.00627003503359461\\
75.74	0.00627141145794913\\
75.75	0.00627278774422494\\
75.76	0.00627416389160375\\
75.77	0.00627553989927004\\
75.78	0.00627691576641118\\
75.79	0.00627829149221736\\
75.8	0.00627966707588173\\
75.81	0.00628104251660036\\
75.82	0.00628241781357228\\
75.83	0.00628379296599956\\
75.84	0.00628516797308729\\
75.85	0.00628654283404366\\
75.86	0.00628791754807994\\
75.87	0.0062892921144106\\
75.88	0.00629066653225325\\
75.89	0.00629204080082874\\
75.9	0.00629341491936117\\
75.91	0.00629478888707793\\
75.92	0.00629616270320976\\
75.93	0.00629753636699072\\
75.94	0.00629890987765831\\
75.95	0.00630028323445346\\
75.96	0.00630165643662055\\
75.97	0.00630302948340751\\
75.98	0.00630440237406579\\
75.99	0.00630577510785045\\
76	0.00630714768402015\\
76.01	0.00630852010183723\\
76.02	0.00630989236056775\\
76.03	0.00631126445948148\\
76.04	0.00631263639785198\\
76.05	0.00631400817495664\\
76.06	0.00631537979007668\\
76.07	0.00631675124249728\\
76.08	0.00631812253150748\\
76.09	0.00631949365640036\\
76.1	0.00632086461647299\\
76.11	0.00632223541102651\\
76.12	0.00632360603936616\\
76.13	0.00632497650080132\\
76.14	0.00632634679464556\\
76.15	0.00632771692021667\\
76.16	0.00632908687683671\\
76.17	0.00633045666383205\\
76.18	0.00633182628053341\\
76.19	0.00633319572627591\\
76.2	0.0063345650003991\\
76.21	0.00633593410224704\\
76.22	0.00633730303116826\\
76.23	0.00633867178651592\\
76.24	0.00634004036764775\\
76.25	0.00634140877392617\\
76.26	0.00634277700471828\\
76.27	0.00634414505939593\\
76.28	0.00634551293733577\\
76.29	0.00634688063791929\\
76.3	0.00634824816053285\\
76.31	0.00634961550456776\\
76.32	0.00635098266942029\\
76.33	0.00635234965449174\\
76.34	0.00635371645918848\\
76.35	0.006355083082922\\
76.36	0.00635644952510894\\
76.37	0.00635781578517117\\
76.38	0.00635918186253581\\
76.39	0.0063605477566353\\
76.4	0.00636191346690743\\
76.41	0.0063632789927954\\
76.42	0.00636464433374785\\
76.43	0.00636600948921896\\
76.44	0.00636737445866841\\
76.45	0.00636873924156155\\
76.46	0.00637010383736934\\
76.47	0.00637146824556847\\
76.48	0.00637283246564137\\
76.49	0.00637419649707629\\
76.5	0.00637556033936734\\
76.51	0.00637692399201454\\
76.52	0.00637828745452387\\
76.53	0.00637965072640735\\
76.54	0.00638101380718305\\
76.55	0.00638237669637516\\
76.56	0.00638373939351407\\
76.57	0.00638510189813637\\
76.58	0.00638646420978498\\
76.59	0.00638782632800911\\
76.6	0.0063891882523644\\
76.61	0.00639054998241292\\
76.62	0.00639191151772328\\
76.63	0.0063932728578706\\
76.64	0.00639463400243667\\
76.65	0.00639599495100991\\
76.66	0.00639735570318551\\
76.67	0.00639871625856542\\
76.68	0.00640007661675845\\
76.69	0.00640143677738032\\
76.7	0.0064027967400537\\
76.71	0.00640415650440829\\
76.72	0.00640551607008087\\
76.73	0.00640687543671535\\
76.74	0.00640823460396285\\
76.75	0.00640959357148174\\
76.76	0.00641095233893774\\
76.77	0.0064123109060039\\
76.78	0.00641366927236075\\
76.79	0.00641502762775041\\
76.8	0.00641638601127061\\
76.81	0.00641774442232639\\
76.82	0.00641910286032433\\
76.83	0.00642046132467256\\
76.84	0.00642181981478078\\
76.85	0.00642317833006031\\
76.86	0.00642453686992405\\
76.87	0.00642589543378656\\
76.88	0.00642725402106405\\
76.89	0.0064286126311744\\
76.9	0.00642997126353721\\
76.91	0.00643132991757375\\
76.92	0.00643268859270709\\
76.93	0.006434047288362\\
76.94	0.00643540600396506\\
76.95	0.00643676473894466\\
76.96	0.00643812349273099\\
76.97	0.00643948226475609\\
76.98	0.00644084105445387\\
76.99	0.00644219986126012\\
77	0.00644355868461256\\
77.01	0.0064449175239508\\
77.02	0.00644627637871644\\
77.03	0.00644763524835304\\
77.04	0.00644899413230616\\
77.05	0.00645035303002336\\
77.06	0.00645171194095429\\
77.07	0.00645307086455062\\
77.08	0.00645442980026612\\
77.09	0.00645578874755669\\
77.1	0.00645714770588034\\
77.11	0.00645850667469725\\
77.12	0.00645986565346978\\
77.13	0.0064612246416625\\
77.14	0.0064625836387422\\
77.15	0.00646394264417794\\
77.16	0.00646530165744104\\
77.17	0.00646666067800512\\
77.18	0.00646801970534614\\
77.19	0.00646937873894242\\
77.2	0.00647073777827462\\
77.21	0.00647209682282584\\
77.22	0.00647345587208158\\
77.23	0.00647481492552979\\
77.24	0.00647617398266093\\
77.25	0.00647753304296793\\
77.26	0.00647889210594625\\
77.27	0.00648025117109392\\
77.28	0.00648161023791153\\
77.29	0.0064829693059023\\
77.3	0.00648432837457207\\
77.31	0.00648568744342934\\
77.32	0.00648704651198529\\
77.33	0.00648840557975384\\
77.34	0.00648976464625162\\
77.35	0.00649112371099804\\
77.36	0.0064924827735153\\
77.37	0.00649384183332844\\
77.38	0.00649520088996534\\
77.39	0.00649655994295675\\
77.4	0.00649791899183634\\
77.41	0.00649927803614072\\
77.42	0.00650063707540944\\
77.43	0.00650199610918507\\
77.44	0.00650335513701319\\
77.45	0.00650471415844244\\
77.46	0.00650607317302453\\
77.47	0.00650743218031427\\
77.48	0.00650879117986964\\
77.49	0.00651015017125176\\
77.5	0.00651150915402498\\
77.51	0.00651286812775684\\
77.52	0.0065142270920182\\
77.53	0.00651558604638313\\
77.54	0.0065169449904291\\
77.55	0.0065183039237369\\
77.56	0.00651966284589068\\
77.57	0.00652102175647806\\
77.58	0.00652238065509006\\
77.59	0.00652373954132121\\
77.6	0.00652509841476954\\
77.61	0.00652645727503662\\
77.62	0.00652781612172761\\
77.63	0.00652917495445127\\
77.64	0.00653053377281999\\
77.65	0.00653189257644987\\
77.66	0.00653325136496069\\
77.67	0.00653461013797598\\
77.68	0.00653596889512305\\
77.69	0.00653732763603301\\
77.7	0.00653868636034084\\
77.71	0.00654004506768536\\
77.72	0.00654140375770934\\
77.73	0.00654276243005949\\
77.74	0.00654412108438648\\
77.75	0.00654547972034505\\
77.76	0.00654683833759392\\
77.77	0.00654819693579599\\
77.78	0.00654955551461821\\
77.79	0.00655091407373174\\
77.8	0.00655227261281192\\
77.81	0.00655363113153833\\
77.82	0.00655498962959484\\
77.83	0.00655634810666961\\
77.84	0.00655770656245514\\
77.85	0.00655906499664836\\
77.86	0.00656042340895058\\
77.87	0.00656178179906759\\
77.88	0.00656314016670969\\
77.89	0.0065644985115917\\
77.9	0.00656585683343304\\
77.91	0.00656721513195774\\
77.92	0.00656857340689448\\
77.93	0.00656993165797665\\
77.94	0.00657128988494237\\
77.95	0.00657264808753455\\
77.96	0.0065740062655009\\
77.97	0.00657536441859401\\
77.98	0.00657672254657136\\
77.99	0.00657808064919537\\
78	0.00657943872623346\\
78.01	0.00658079677745805\\
78.02	0.00658215480264665\\
78.03	0.00658351280158188\\
78.04	0.00658487077405151\\
78.05	0.00658622871984849\\
78.06	0.00658758663877104\\
78.07	0.00658894453062264\\
78.08	0.00659030239521212\\
78.09	0.00659166023235365\\
78.1	0.00659301804186683\\
78.11	0.00659437582357668\\
78.12	0.00659573357731377\\
78.13	0.00659709130291414\\
78.14	0.00659844900021947\\
78.15	0.006599806669077\\
78.16	0.00660116430933969\\
78.17	0.00660252192086618\\
78.18	0.00660387950352086\\
78.19	0.00660523705717393\\
78.2	0.00660659458170143\\
78.21	0.00660795207698528\\
78.22	0.00660930954291332\\
78.23	0.00661066697937938\\
78.24	0.00661202438628332\\
78.25	0.00661338176353104\\
78.26	0.00661473911103458\\
78.27	0.0066160964287121\\
78.28	0.00661745371648801\\
78.29	0.00661881097429292\\
78.3	0.00662016820206379\\
78.31	0.00662152539974389\\
78.32	0.00662288256728289\\
78.33	0.00662423970463691\\
78.34	0.00662559681176854\\
78.35	0.00662695388864692\\
78.36	0.00662831093524777\\
78.37	0.00662966795155344\\
78.38	0.00663102493755296\\
78.39	0.0066323818932421\\
78.4	0.00663373881862341\\
78.41	0.00663509571370625\\
78.42	0.00663645257850688\\
78.43	0.0066378094130485\\
78.44	0.00663916621736127\\
78.45	0.00664052299148238\\
78.46	0.00664187973545613\\
78.47	0.00664323644933392\\
78.48	0.00664459313317436\\
78.49	0.00664594978704331\\
78.5	0.00664730641101387\\
78.51	0.00664866300516655\\
78.52	0.0066500195695892\\
78.53	0.00665137610437715\\
78.54	0.00665273260963322\\
78.55	0.0066540890854678\\
78.56	0.00665544553199888\\
78.57	0.00665680194935211\\
78.58	0.00665815833766088\\
78.59	0.00665951469706633\\
78.6	0.00666087102771746\\
78.61	0.00666222732977111\\
78.62	0.0066635836033921\\
78.63	0.00666493984875325\\
78.64	0.00666629606603539\\
78.65	0.00666765225542752\\
78.66	0.00666900841712676\\
78.67	0.00667036455133848\\
78.68	0.00667172065827633\\
78.69	0.00667307673816228\\
78.7	0.00667443279122674\\
78.71	0.00667578881770855\\
78.72	0.00667714481785505\\
78.73	0.0066785007919222\\
78.74	0.00667985674017457\\
78.75	0.00668121266288542\\
78.76	0.00668256856033679\\
78.77	0.00668392443281952\\
78.78	0.00668528028063332\\
78.79	0.00668663610408686\\
78.8	0.00668799190349779\\
78.81	0.00668934767919285\\
78.82	0.00669070343150787\\
78.83	0.0066920591607879\\
78.84	0.00669341486738721\\
78.85	0.00669477055166941\\
78.86	0.00669612621400748\\
78.87	0.00669748185478383\\
78.88	0.00669883747439039\\
78.89	0.00670019307322865\\
78.9	0.00670154865170975\\
78.91	0.00670290421025451\\
78.92	0.00670425974929354\\
78.93	0.00670561526926725\\
78.94	0.00670697077062599\\
78.95	0.00670832625383004\\
78.96	0.00670968171934975\\
78.97	0.00671103716766553\\
78.98	0.00671239259926799\\
78.99	0.00671374801465797\\
79	0.00671510341434659\\
79.01	0.00671645879885539\\
79.02	0.00671781416871632\\
79.03	0.00671916952447185\\
79.04	0.00672052486667504\\
79.05	0.00672188019588961\\
79.06	0.00672323551268999\\
79.07	0.0067245908176614\\
79.08	0.00672594611139998\\
79.09	0.00672730139451274\\
79.1	0.00672865666761774\\
79.11	0.00673001193134414\\
79.12	0.00673136718633222\\
79.13	0.00673272243323353\\
79.14	0.00673407767271089\\
79.15	0.00673543290543853\\
79.16	0.00673678813210212\\
79.17	0.00673814335339888\\
79.18	0.00673949857003761\\
79.19	0.00674085378273881\\
79.2	0.00674220899223473\\
79.21	0.00674356419926946\\
79.22	0.00674491940459901\\
79.23	0.00674627460899136\\
79.24	0.00674762981322659\\
79.25	0.00674898501809689\\
79.26	0.00675034022440671\\
79.27	0.00675169543297278\\
79.28	0.00675305064462424\\
79.29	0.00675440586020266\\
79.3	0.00675576108056219\\
79.31	0.0067571163065696\\
79.32	0.00675847153910435\\
79.33	0.0067598267790587\\
79.34	0.00676118202733779\\
79.35	0.00676253728485972\\
79.36	0.0067638925525556\\
79.37	0.00676524783136969\\
79.38	0.00676660312225946\\
79.39	0.00676795842619564\\
79.4	0.00676931374416236\\
79.41	0.00677066907715721\\
79.42	0.00677202442619132\\
79.43	0.00677337979228946\\
79.44	0.00677473517649011\\
79.45	0.00677609057984557\\
79.46	0.00677744600342203\\
79.47	0.00677880144829966\\
79.48	0.00678015691557271\\
79.49	0.00678151240634958\\
79.5	0.00678286792175294\\
79.51	0.00678422346291979\\
79.52	0.00678557903100155\\
79.53	0.00678693462716419\\
79.54	0.00678829025258826\\
79.55	0.00678964590846905\\
79.56	0.00679100159601664\\
79.57	0.00679235731645597\\
79.58	0.00679371307102702\\
79.59	0.00679506886098481\\
79.6	0.00679642468759953\\
79.61	0.00679778055215668\\
79.62	0.00679913645595708\\
79.63	0.00680049240031703\\
79.64	0.00680184838656839\\
79.65	0.00680320441605866\\
79.66	0.0068045604901511\\
79.67	0.00680591661022483\\
79.68	0.0068072727776749\\
79.69	0.00680862899391241\\
79.7	0.00680998526036461\\
79.71	0.006811341578475\\
79.72	0.00681269794970343\\
79.73	0.00681405437552617\\
79.74	0.00681541085743608\\
79.75	0.00681676739694264\\
79.76	0.00681812399557211\\
79.77	0.00681948065486759\\
79.78	0.00682083737638916\\
79.79	0.00682219416171392\\
79.8	0.00682355101243622\\
79.81	0.00682490793016762\\
79.82	0.00682626491653709\\
79.83	0.00682762197319108\\
79.84	0.00682897910179365\\
79.85	0.00683033630402655\\
79.86	0.00683169358158936\\
79.87	0.00683305093619955\\
79.88	0.00683440836959265\\
79.89	0.00683576588352233\\
79.9	0.00683712347976049\\
79.91	0.00683848116009741\\
79.92	0.00683983892634185\\
79.93	0.00684119678032113\\
79.94	0.0068425547238813\\
79.95	0.0068439127588872\\
79.96	0.0068452708872226\\
79.97	0.00684662911079032\\
79.98	0.00684798743151235\\
79.99	0.00684934585132992\\
80	0.00685070437220366\\
80.01	0.00685206299611372\\
};
\addplot [color=mycolor1,solid]
  table[row sep=crcr]{%
80.01	0.00685206299611372\\
80.02	0.00685342172505986\\
80.03	0.0068547805610616\\
80.04	0.00685613950615831\\
80.05	0.00685749856240932\\
80.06	0.00685885773189409\\
80.07	0.0068602170167123\\
80.08	0.00686157641898395\\
80.09	0.00686293594084953\\
80.1	0.0068642955844701\\
80.11	0.00686565535202744\\
80.12	0.00686701524572418\\
80.13	0.00686837526778387\\
80.14	0.00686973542045118\\
80.15	0.00687109570599198\\
80.16	0.00687245612669346\\
80.17	0.00687381668486431\\
80.18	0.00687517738283478\\
80.19	0.00687653822295685\\
80.2	0.00687789920760436\\
80.21	0.00687926033917313\\
80.22	0.00688062162008107\\
80.23	0.00688198305276835\\
80.24	0.00688334463969751\\
80.25	0.00688470638335358\\
80.26	0.00688606828624426\\
80.27	0.00688743035090001\\
80.28	0.00688879257987417\\
80.29	0.00689015497574318\\
80.3	0.00689151754110659\\
80.31	0.00689288027858734\\
80.32	0.00689424319083176\\
80.33	0.00689560628050981\\
80.34	0.00689696955031516\\
80.35	0.00689833300296537\\
80.36	0.00689969664120199\\
80.37	0.00690106046779073\\
80.38	0.00690242448552159\\
80.39	0.00690378869720901\\
80.4	0.00690515310569201\\
80.41	0.00690651771383432\\
80.42	0.00690788252270627\\
80.43	0.0069092475321261\\
80.44	0.0069106127419129\\
80.45	0.00691197815188654\\
80.46	0.00691334376186768\\
80.47	0.00691470957167783\\
80.48	0.00691607558113929\\
80.49	0.00691744179007521\\
80.5	0.00691880819830953\\
80.51	0.00692017480566704\\
80.52	0.0069215416119734\\
80.53	0.00692290861705504\\
80.54	0.0069242758207393\\
80.55	0.00692564322285434\\
80.56	0.00692701082322918\\
80.57	0.0069283786216937\\
80.58	0.00692974661807866\\
80.59	0.00693111481221565\\
80.6	0.00693248320393717\\
80.61	0.0069338517930766\\
80.62	0.00693522057946817\\
80.63	0.00693658956294704\\
80.64	0.00693795874334923\\
80.65	0.00693932812051167\\
80.66	0.0069406976942722\\
80.67	0.00694206746446956\\
80.68	0.00694343743094339\\
80.69	0.00694480759353427\\
80.7	0.00694617795208369\\
80.71	0.00694754850643407\\
80.72	0.00694891925642875\\
80.73	0.00695029020191202\\
80.74	0.0069516613427291\\
80.75	0.00695303267872617\\
80.76	0.00695440420975033\\
80.77	0.00695577593564968\\
80.78	0.00695714785627323\\
80.79	0.00695851997147101\\
80.8	0.00695989228109397\\
80.81	0.00696126478499406\\
80.82	0.0069626374830242\\
80.83	0.0069640103750383\\
80.84	0.00696538346089126\\
80.85	0.00696675674043896\\
80.86	0.0069681302135383\\
80.87	0.00696950388004717\\
80.88	0.00697087773982447\\
80.89	0.0069722517927301\\
80.9	0.00697362603862501\\
80.91	0.00697500047737115\\
80.92	0.0069763751088315\\
80.93	0.00697774993287007\\
80.94	0.00697912494935192\\
80.95	0.00698050015814314\\
80.96	0.00698187555911087\\
80.97	0.00698325115212331\\
80.98	0.00698462693704971\\
80.99	0.00698600291376038\\
81	0.00698737908212669\\
81.01	0.0069887554420211\\
81.02	0.00699013199331714\\
81.03	0.00699150873588942\\
81.04	0.00699288566961363\\
81.05	0.00699426279436656\\
81.06	0.00699564011002608\\
81.07	0.00699701761647118\\
81.08	0.00699839531358196\\
81.09	0.00699977320123961\\
81.1	0.00700115127932644\\
81.11	0.00700252954772589\\
81.12	0.00700390800632253\\
81.13	0.00700528665500205\\
81.14	0.00700666549365126\\
81.15	0.00700804452215815\\
81.16	0.00700942374041182\\
81.17	0.00701080314830253\\
81.18	0.00701218274572169\\
81.19	0.0070135625325619\\
81.2	0.00701494250871688\\
81.21	0.00701632267408156\\
81.22	0.007017703028552\\
81.23	0.0070190835720255\\
81.24	0.00702046430440047\\
81.25	0.00702184522557657\\
81.26	0.00702322633545463\\
81.27	0.00702460763393668\\
81.28	0.00702598912092595\\
81.29	0.00702737079632689\\
81.3	0.00702875266004515\\
81.31	0.00703013471198762\\
81.32	0.00703151695206237\\
81.33	0.00703289938017876\\
81.34	0.00703428199624734\\
81.35	0.00703566480017989\\
81.36	0.00703704779188948\\
81.37	0.00703843097129037\\
81.38	0.00703981433829812\\
81.39	0.00704119789282952\\
81.4	0.00704258163480265\\
81.41	0.00704396556413681\\
81.42	0.00704534968075262\\
81.43	0.00704673398457195\\
81.44	0.00704811847551796\\
81.45	0.0070495031535151\\
81.46	0.00705088801848909\\
81.47	0.00705227307036698\\
81.48	0.0070536583090771\\
81.49	0.00705504373454909\\
81.5	0.00705642934671389\\
81.51	0.00705781514550378\\
81.52	0.00705920113085234\\
81.53	0.00706058730269448\\
81.54	0.00706197366096645\\
81.55	0.00706336020560581\\
81.56	0.00706474693655149\\
81.57	0.00706613385374374\\
81.58	0.00706752095712418\\
81.59	0.00706890824663575\\
81.6	0.00707029572222278\\
81.61	0.00707168338383096\\
81.62	0.00707307123140734\\
81.63	0.00707445926490034\\
81.64	0.00707584748425976\\
81.65	0.00707723588943678\\
81.66	0.00707862448038399\\
81.67	0.00708001325705532\\
81.68	0.00708140221940615\\
81.69	0.00708279136739324\\
81.7	0.00708418070097473\\
81.71	0.00708557022011022\\
81.72	0.00708695992476067\\
81.73	0.00708834981488853\\
81.74	0.00708973989045759\\
81.75	0.00709113015143312\\
81.76	0.00709252059778183\\
81.77	0.00709391122947184\\
81.78	0.00709530204647272\\
81.79	0.00709669304875549\\
81.8	0.00709808423629264\\
81.81	0.00709947560905807\\
81.82	0.00710086716702719\\
81.83	0.00710225891017684\\
81.84	0.00710365083848535\\
81.85	0.00710504295193251\\
81.86	0.00710643525049961\\
81.87	0.00710782773416939\\
81.88	0.00710922040292611\\
81.89	0.0071106132567555\\
81.9	0.00711200629564479\\
81.91	0.00711339951958271\\
81.92	0.00711479292855952\\
81.93	0.00711618652256695\\
81.94	0.00711758030159827\\
81.95	0.00711897426564826\\
81.96	0.00712036841471322\\
81.97	0.00712176274879099\\
81.98	0.00712315726788091\\
81.99	0.0071245519719839\\
82	0.00712594686110237\\
82.01	0.00712734193524032\\
82.02	0.00712873719440326\\
82.03	0.00713013263859829\\
82.04	0.00713152826783403\\
82.05	0.00713292408212068\\
82.06	0.00713432008147\\
82.07	0.00713571626589532\\
82.08	0.00713711263541155\\
82.09	0.00713850919003516\\
82.1	0.00713990592978421\\
82.11	0.00714130285467836\\
82.12	0.00714269996473883\\
82.13	0.00714409725998845\\
82.14	0.00714549474045166\\
82.15	0.00714689240615446\\
82.16	0.00714829025712451\\
82.17	0.00714968829339103\\
82.18	0.00715108651498489\\
82.19	0.00715248492193855\\
82.2	0.00715388351428609\\
82.21	0.00715528229206323\\
82.22	0.00715668125530733\\
82.23	0.00715808040405735\\
82.24	0.00715947973835391\\
82.25	0.00716087925823924\\
82.26	0.00716227896375725\\
82.27	0.00716367885495349\\
82.28	0.00716507893187512\\
82.29	0.00716647919457101\\
82.3	0.00716787964309164\\
82.31	0.00716928027748919\\
82.32	0.00717068109781748\\
82.33	0.00717208210413201\\
82.34	0.00717348329648995\\
82.35	0.00717488467495013\\
82.36	0.00717628623957308\\
82.37	0.00717768799042103\\
82.38	0.00717908992755794\\
82.39	0.00718049205104947\\
82.4	0.00718189436096303\\
82.41	0.00718329685736775\\
82.42	0.00718469954033446\\
82.43	0.00718610240993577\\
82.44	0.00718750546624599\\
82.45	0.00718890870934119\\
82.46	0.00719031213929916\\
82.47	0.00719171575619944\\
82.48	0.00719311956012333\\
82.49	0.00719452355115388\\
82.5	0.00719592772937585\\
82.51	0.00719733209487582\\
82.52	0.00719873664774207\\
82.53	0.00720014138806467\\
82.54	0.00720154631593543\\
82.55	0.00720295143144796\\
82.56	0.0072043567346976\\
82.57	0.00720576222578146\\
82.58	0.00720716790479846\\
82.59	0.00720857377184924\\
82.6	0.00720997982703625\\
82.61	0.0072113860704637\\
82.62	0.0072127925022376\\
82.63	0.00721419912246573\\
82.64	0.00721560593125764\\
82.65	0.00721701292872468\\
82.66	0.00721842011497998\\
82.67	0.00721982749013848\\
82.68	0.00722123505431688\\
82.69	0.0072226428076337\\
82.7	0.00722405075020923\\
82.71	0.00722545888216557\\
82.72	0.00722686720362662\\
82.73	0.00722827571471807\\
82.74	0.00722968441556741\\
82.75	0.00723109330630395\\
82.76	0.00723250238705878\\
82.77	0.00723391165796481\\
82.78	0.00723532111915674\\
82.79	0.0072367307707711\\
82.8	0.0072381406129462\\
82.81	0.00723955064582219\\
82.82	0.00724096086954102\\
82.83	0.00724237128424643\\
82.84	0.00724378189008401\\
82.85	0.00724519268720112\\
82.86	0.00724660367574698\\
82.87	0.00724801485587258\\
82.88	0.00724942622773076\\
82.89	0.00725083779147616\\
82.9	0.00725224954726524\\
82.91	0.00725366149525628\\
82.92	0.00725507363560937\\
82.93	0.00725648596848642\\
82.94	0.00725789849405117\\
82.95	0.00725931121246916\\
82.96	0.00726072412390776\\
82.97	0.00726213722853617\\
82.98	0.00726355052652538\\
82.99	0.00726496401804821\\
83	0.00726637770327932\\
83.01	0.00726779158239517\\
83.02	0.00726920565557403\\
83.03	0.00727061992299599\\
83.04	0.007272034384843\\
83.05	0.00727344904129876\\
83.06	0.00727486389254884\\
83.07	0.00727627893878059\\
83.08	0.0072776941801832\\
83.09	0.00727910961694767\\
83.1	0.0072805252492668\\
83.11	0.00728194107733522\\
83.12	0.00728335710134937\\
83.13	0.00728477332150748\\
83.14	0.00728618973800962\\
83.15	0.00728760635105766\\
83.16	0.00728902316085525\\
83.17	0.00729044016760789\\
83.18	0.00729185737152286\\
83.19	0.00729327477280924\\
83.2	0.00729469237167791\\
83.21	0.00729611016834157\\
83.22	0.00729752816301471\\
83.23	0.00729894635591359\\
83.24	0.00730036474725632\\
83.25	0.00730178333726273\\
83.26	0.0073032021261545\\
83.27	0.00730462111415508\\
83.28	0.00730604030148968\\
83.29	0.00730745968838534\\
83.3	0.00730887927507084\\
83.31	0.00731029906177675\\
83.32	0.00731171904873544\\
83.33	0.00731313923618102\\
83.34	0.00731455962434938\\
83.35	0.00731598021347818\\
83.36	0.00731740100380684\\
83.37	0.00731882199557655\\
83.38	0.00732024318903025\\
83.39	0.00732166458441263\\
83.4	0.00732308618197014\\
83.41	0.00732450798195099\\
83.42	0.00732592998460508\\
83.43	0.00732735219018412\\
83.44	0.0073287745989415\\
83.45	0.00733019721113239\\
83.46	0.00733162002701365\\
83.47	0.00733304304684387\\
83.48	0.00733446627088338\\
83.49	0.0073358896993942\\
83.5	0.00733731333264009\\
83.51	0.00733873717088649\\
83.52	0.00734016121440055\\
83.53	0.00734158546345112\\
83.54	0.00734300991830874\\
83.55	0.00734443457924563\\
83.56	0.00734585944653571\\
83.57	0.00734728452045456\\
83.58	0.00734870980127943\\
83.59	0.00735013528928925\\
83.6	0.00735156098476461\\
83.61	0.00735298688798773\\
83.62	0.0073544129992425\\
83.63	0.00735583931881444\\
83.64	0.00735726584699072\\
83.65	0.00735869258406015\\
83.66	0.00736011953031311\\
83.67	0.00736154668604167\\
83.68	0.00736297405153946\\
83.69	0.00736440162710174\\
83.7	0.00736582941302533\\
83.71	0.0073672574096087\\
83.72	0.00736868561715185\\
83.73	0.00737011403595638\\
83.74	0.00737154266632545\\
83.75	0.0073729715085638\\
83.76	0.0073744005629777\\
83.77	0.00737582982987497\\
83.78	0.00737725930956498\\
83.79	0.00737868900235862\\
83.8	0.00738011890856831\\
83.81	0.00738154902850797\\
83.82	0.00738297936249303\\
83.83	0.00738440991084042\\
83.84	0.00738584067386855\\
83.85	0.00738727165189733\\
83.86	0.0073887028452481\\
83.87	0.00739013425424369\\
83.88	0.00739156587920838\\
83.89	0.00739299772046789\\
83.9	0.00739442977834934\\
83.91	0.00739586205318133\\
83.92	0.00739729454529383\\
83.93	0.00739872725501822\\
83.94	0.00740016018268726\\
83.95	0.00740159332863513\\
83.96	0.00740302669319734\\
83.97	0.00740446027671078\\
83.98	0.00740589407951368\\
83.99	0.0074073281019456\\
84	0.00740876234434746\\
84.01	0.00741019680706147\\
84.02	0.00741163149043113\\
84.03	0.00741306639480126\\
84.04	0.00741450152051794\\
84.05	0.00741593686792855\\
84.06	0.00741737243738167\\
84.07	0.00741880822922717\\
84.08	0.00742024424381614\\
84.09	0.00742168048150088\\
84.1	0.00742311694263489\\
84.11	0.00742455362757288\\
84.12	0.00742599053649315\\
84.13	0.00742742766937362\\
84.14	0.00742886502619391\\
84.15	0.00743030260693534\\
84.16	0.00743174041158092\\
84.17	0.00743317844011541\\
84.18	0.00743461669252526\\
84.19	0.00743605516879868\\
84.2	0.0074374938689256\\
84.21	0.00743893279289771\\
84.22	0.00744037194070845\\
84.23	0.007441811312353\\
84.24	0.00744325090782837\\
84.25	0.00744469072713328\\
84.26	0.00744613077026829\\
84.27	0.00744757103723572\\
84.28	0.0074490115280397\\
84.29	0.0074504522426862\\
84.3	0.00745189318118296\\
84.31	0.00745333434353958\\
84.32	0.00745477572976747\\
84.33	0.00745621733987991\\
84.34	0.007457659173892\\
84.35	0.00745910123182073\\
84.36	0.00746054351368492\\
84.37	0.00746198601950529\\
84.38	0.00746342874930444\\
84.39	0.00746487170310683\\
84.4	0.00746631488093887\\
84.41	0.00746775828282883\\
84.42	0.0074692019088069\\
84.43	0.00747064575890522\\
84.44	0.00747208983315782\\
84.45	0.00747353413160071\\
84.46	0.00747497865427182\\
84.47	0.00747642340121103\\
84.48	0.00747786837246019\\
84.49	0.00747931356806312\\
84.5	0.00748075898806563\\
84.51	0.0074822046325155\\
84.52	0.0074836505014625\\
84.53	0.00748509659495843\\
84.54	0.00748654291305707\\
84.55	0.00748798945581423\\
84.56	0.00748943622328776\\
84.57	0.00749088321553753\\
84.58	0.00749233043262546\\
84.59	0.00749377787461551\\
84.6	0.00749522554157372\\
84.61	0.00749667343356819\\
84.62	0.00749812155066908\\
84.63	0.00749956989294866\\
84.64	0.00750101846048127\\
84.65	0.00750246725334337\\
84.66	0.00750391627161352\\
84.67	0.0075053655153724\\
84.68	0.0075068149847028\\
84.69	0.00750826467968966\\
84.7	0.00750971460042008\\
84.71	0.00751116474698326\\
84.72	0.00751261511947061\\
84.73	0.00751406571797568\\
84.74	0.0075155165425942\\
84.75	0.00751696759342408\\
84.76	0.00751841887056543\\
84.77	0.00751987037412057\\
84.78	0.00752132210419399\\
84.79	0.00752277406089243\\
84.8	0.00752422624432484\\
84.81	0.00752567865460243\\
84.82	0.00752713129183862\\
84.83	0.00752858415614908\\
84.84	0.00753003724765177\\
84.85	0.00753149056646688\\
84.86	0.00753294411271689\\
84.87	0.00753439788652658\\
84.88	0.007535851888023\\
84.89	0.00753730611733551\\
84.9	0.00753876057459576\\
84.91	0.00754021525993774\\
84.92	0.00754167017349906\\
84.93	0.00754312531542116\\
84.94	0.00754458068584934\\
84.95	0.00754603628493279\\
84.96	0.00754749211282461\\
84.97	0.00754894816968182\\
84.98	0.00755040445566543\\
84.99	0.00755186097094039\\
85	0.00755331771567571\\
85.01	0.00755477469004441\\
85.02	0.00755623189422357\\
85.03	0.00755768932839438\\
85.04	0.00755914699274214\\
85.05	0.00756060488745627\\
85.06	0.00756206301273038\\
85.07	0.00756352136876227\\
85.08	0.00756497995575398\\
85.09	0.00756643877391176\\
85.1	0.00756789782344616\\
85.11	0.00756935710457203\\
85.12	0.00757081661750856\\
85.13	0.00757227636247929\\
85.14	0.00757373633971213\\
85.15	0.00757519654943943\\
85.16	0.00757665699189796\\
85.17	0.00757811766732898\\
85.18	0.00757957857597823\\
85.19	0.00758103971809597\\
85.2	0.00758250109393703\\
85.21	0.00758396270376082\\
85.22	0.00758542454783135\\
85.23	0.00758688662641728\\
85.24	0.00758834893979194\\
85.25	0.00758981148823336\\
85.26	0.00759127427202428\\
85.27	0.00759273729145221\\
85.28	0.00759420054680945\\
85.29	0.0075956640383931\\
85.3	0.00759712776650513\\
85.31	0.00759859173145237\\
85.32	0.00760005593354655\\
85.33	0.00760152037310435\\
85.34	0.00760298505044741\\
85.35	0.00760444996590237\\
85.36	0.0076059151198009\\
85.37	0.00760738051247972\\
85.38	0.00760884614428064\\
85.39	0.00761031201555061\\
85.4	0.00761177812664172\\
85.41	0.00761324447791124\\
85.42	0.00761471106972164\\
85.43	0.00761617790244069\\
85.44	0.00761764497644139\\
85.45	0.00761911229210206\\
85.46	0.00762057984980639\\
85.47	0.00762204764994342\\
85.48	0.00762351569290761\\
85.49	0.00762498397909886\\
85.5	0.00762645250892255\\
85.51	0.00762792128278955\\
85.52	0.00762939030111629\\
85.53	0.00763085956432478\\
85.54	0.00763232907284261\\
85.55	0.00763379882710303\\
85.56	0.00763526882754498\\
85.57	0.00763673907461308\\
85.58	0.00763820956875771\\
85.59	0.00763968031043503\\
85.6	0.00764115130010701\\
85.61	0.00764262253824146\\
85.62	0.00764409402531209\\
85.63	0.00764556576179852\\
85.64	0.00764703774818633\\
85.65	0.00764850998496706\\
85.66	0.0076499824726383\\
85.67	0.0076514552117037\\
85.68	0.007652928202673\\
85.69	0.00765440144606207\\
85.7	0.00765587494239295\\
85.71	0.00765734869219389\\
85.72	0.00765882269599937\\
85.73	0.00766029695435017\\
85.74	0.00766177146779335\\
85.75	0.00766324623688236\\
85.76	0.00766472126217701\\
85.77	0.00766619654424356\\
85.78	0.00766767208365472\\
85.79	0.00766914788098971\\
85.8	0.0076706239368343\\
85.81	0.0076721002517808\\
85.82	0.0076735768264282\\
85.83	0.0076750536613821\\
85.84	0.0076765307572548\\
85.85	0.00767800811466536\\
85.86	0.00767948573423958\\
85.87	0.0076809636166101\\
85.88	0.0076824417624164\\
85.89	0.00768392017230485\\
85.9	0.00768539884692876\\
85.91	0.0076868777869484\\
85.92	0.00768835699303108\\
85.93	0.00768983646585113\\
85.94	0.00769131620608999\\
85.95	0.00769279621443625\\
85.96	0.00769427649158564\\
85.97	0.00769575703824114\\
85.98	0.00769723785511298\\
85.99	0.00769871894291867\\
86	0.0077002003023831\\
86.01	0.00770168193423851\\
86.02	0.00770316383922459\\
86.03	0.00770464601808849\\
86.04	0.00770612847158487\\
86.05	0.00770761120047596\\
86.06	0.00770909420553156\\
86.07	0.00771057748752915\\
86.08	0.00771206104725386\\
86.09	0.00771354488549857\\
86.1	0.00771502900306393\\
86.11	0.00771651340075839\\
86.12	0.00771799807939829\\
86.13	0.00771948303980785\\
86.14	0.00772096828281926\\
86.15	0.0077224538092727\\
86.16	0.00772393962001639\\
86.17	0.00772542571590663\\
86.18	0.00772691209780787\\
86.19	0.00772839876659271\\
86.2	0.007729885723142\\
86.21	0.00773137296834486\\
86.22	0.00773286050309871\\
86.23	0.00773434832830933\\
86.24	0.00773583644489095\\
86.25	0.00773732485376622\\
86.26	0.00773881355586631\\
86.27	0.00774030255213093\\
86.28	0.00774179184350841\\
86.29	0.00774328143095572\\
86.3	0.00774477131543853\\
86.31	0.00774626149793124\\
86.32	0.00774775197941706\\
86.33	0.00774924276088806\\
86.34	0.00775073384334517\\
86.35	0.00775222522779828\\
86.36	0.00775371691526626\\
86.37	0.00775520890677703\\
86.38	0.00775670120336763\\
86.39	0.00775819380608418\\
86.4	0.00775968671598205\\
86.41	0.00776117993412581\\
86.42	0.00776267346158937\\
86.43	0.00776416729945596\\
86.44	0.00776566144881819\\
86.45	0.00776715591077816\\
86.46	0.00776865068644744\\
86.47	0.00777014577694719\\
86.48	0.00777164118340812\\
86.49	0.00777313690697065\\
86.5	0.0077746329487849\\
86.51	0.00777612931001074\\
86.52	0.00777762599181787\\
86.53	0.00777912299538587\\
86.54	0.00778062032190423\\
86.55	0.00778211797257243\\
86.56	0.00778361594859999\\
86.57	0.00778511425120652\\
86.58	0.00778661288162177\\
86.59	0.00778811184108569\\
86.6	0.00778961113084849\\
86.61	0.00779111075217069\\
86.62	0.00779261070632319\\
86.63	0.00779411099458729\\
86.64	0.00779561161825479\\
86.65	0.00779711257862803\\
86.66	0.00779861387701992\\
86.67	0.00780011551475406\\
86.68	0.00780161749316473\\
86.69	0.00780311981359699\\
86.7	0.00780462247740672\\
86.71	0.00780612548596068\\
86.72	0.0078076288406366\\
86.73	0.00780913254282318\\
86.74	0.00781063659392019\\
86.75	0.00781214099533854\\
86.76	0.00781364574850028\\
86.77	0.00781515085483876\\
86.78	0.00781665631579855\\
86.79	0.00781816213283567\\
86.8	0.00781966830741748\\
86.81	0.00782117484102289\\
86.82	0.00782268173424422\\
86.83	0.00782418898728899\\
86.84	0.00782569660036539\\
86.85	0.00782720457368234\\
86.86	0.00782871290744944\\
86.87	0.00783022160187698\\
86.88	0.00783173065717592\\
86.89	0.00783324007355792\\
86.9	0.00783474985123528\\
86.91	0.00783625999042096\\
86.92	0.00783777049132859\\
86.93	0.00783928135417245\\
86.94	0.00784079257916745\\
86.95	0.00784230416652913\\
86.96	0.00784381611647366\\
86.97	0.00784532842921785\\
86.98	0.00784684110497909\\
86.99	0.00784835414397539\\
87	0.00784986754642535\\
87.01	0.00785138131254818\\
87.02	0.00785289544256364\\
87.03	0.0078544099366921\\
87.04	0.00785592479515445\\
87.05	0.00785744001817219\\
87.06	0.00785895560596733\\
87.07	0.00786047155876244\\
87.08	0.00786198787678063\\
87.09	0.00786350456024551\\
87.1	0.00786502160938123\\
87.11	0.00786653902441243\\
87.12	0.00786805680556427\\
87.13	0.00786957495306239\\
87.14	0.00787109346713289\\
87.15	0.00787261234800238\\
87.16	0.00787413159589791\\
87.17	0.00787565121104699\\
87.18	0.00787717119367756\\
87.19	0.00787869154401803\\
87.2	0.00788021226229718\\
87.21	0.00788173334874426\\
87.22	0.0078832548035889\\
87.23	0.00788477662706111\\
87.24	0.00788629881939131\\
87.25	0.00788782138081029\\
87.26	0.00788934431154919\\
87.27	0.00789086761183952\\
87.28	0.00789239128191311\\
87.29	0.00789391532200216\\
87.3	0.00789543973233916\\
87.31	0.00789696451315692\\
87.32	0.00789848966468853\\
87.33	0.00790001518716741\\
87.34	0.00790154108082722\\
87.35	0.00790306734590189\\
87.36	0.00790459398262561\\
87.37	0.00790612099123282\\
87.38	0.00790764837195816\\
87.39	0.00790917612503652\\
87.4	0.00791070425070296\\
87.41	0.00791223274919277\\
87.42	0.00791376162074138\\
87.43	0.00791529086558443\\
87.44	0.00791682048395768\\
87.45	0.00791835047609703\\
87.46	0.00791988084223854\\
87.47	0.00792141158261835\\
87.48	0.00792294269747272\\
87.49	0.007924474187038\\
87.5	0.00792600605155059\\
87.51	0.00792753829124698\\
87.52	0.00792907090636369\\
87.53	0.00793060389713727\\
87.54	0.00793213726380429\\
87.55	0.00793367100660133\\
87.56	0.00793520512576495\\
87.57	0.0079367396215317\\
87.58	0.00793827449413806\\
87.59	0.00793980974382048\\
87.6	0.0079413453708153\\
87.61	0.00794288137535884\\
87.62	0.00794441775768726\\
87.63	0.0079459545180366\\
87.64	0.00794749165664281\\
87.65	0.00794902917374166\\
87.66	0.00795056706956874\\
87.67	0.00795210534435948\\
87.68	0.00795364399834912\\
87.69	0.00795518303177265\\
87.7	0.00795672244486485\\
87.71	0.00795826223786023\\
87.72	0.00795980241099306\\
87.73	0.00796134296449731\\
87.74	0.00796288389860664\\
87.75	0.0079644252135544\\
87.76	0.00796596690957358\\
87.77	0.00796750898689685\\
87.78	0.00796905144575648\\
87.79	0.00797059428638434\\
87.8	0.00797213750901191\\
87.81	0.00797368111387021\\
87.82	0.00797522510118984\\
87.83	0.0079767694712009\\
87.84	0.00797831422413302\\
87.85	0.00797985936021532\\
87.86	0.00798140487967638\\
87.87	0.00798295078274425\\
87.88	0.00798449706964638\\
87.89	0.00798604374060966\\
87.9	0.00798759079586034\\
87.91	0.00798913823562408\\
87.92	0.00799068606012584\\
87.93	0.00799223426958994\\
87.94	0.00799378286423999\\
87.95	0.00799533184429889\\
87.96	0.00799688120998879\\
87.97	0.00799843096153108\\
87.98	0.00799998109914639\\
87.99	0.00800153162305452\\
88	0.00800308253347444\\
88.01	0.00800463383062428\\
88.02	0.00800618551472129\\
88.03	0.00800773758598183\\
88.04	0.00800929004462131\\
88.05	0.00801084289085424\\
88.06	0.00801239612489412\\
88.07	0.00801394974695348\\
88.08	0.00801550375724382\\
88.09	0.00801705815597559\\
88.1	0.00801861294335818\\
88.11	0.0080201681195999\\
88.12	0.00802172368490792\\
88.13	0.00802327963948827\\
88.14	0.00802483598354582\\
88.15	0.00802639271728423\\
88.16	0.00802794984090594\\
88.17	0.00802950735461215\\
88.18	0.00803106525860277\\
88.19	0.00803262355307642\\
88.2	0.00803418223823037\\
88.21	0.00803574131426055\\
88.22	0.0080373007813615\\
88.23	0.00803886063972633\\
88.24	0.00804042088954672\\
88.25	0.00804198153101288\\
88.26	0.00804354256431351\\
88.27	0.00804510398963579\\
88.28	0.00804666580716534\\
88.29	0.00804822801708618\\
88.3	0.00804979061958072\\
88.31	0.00805135361482972\\
88.32	0.00805291700301226\\
88.33	0.00805448078430571\\
88.34	0.00805604495888571\\
88.35	0.00805760952692609\\
88.36	0.00805917448859893\\
88.37	0.00806073984407444\\
88.38	0.00806230559352096\\
88.39	0.00806387173710495\\
88.4	0.00806543827499091\\
88.41	0.0080670052073414\\
88.42	0.00806857253431697\\
88.43	0.00807014025607614\\
88.44	0.00807170837277536\\
88.45	0.00807327688456898\\
88.46	0.00807484579160921\\
88.47	0.00807641509404611\\
88.48	0.00807798479202751\\
88.49	0.00807955488569901\\
88.5	0.00808112537520394\\
88.51	0.00808269626068332\\
88.52	0.00808426754227581\\
88.53	0.00808583922011769\\
88.54	0.00808741129434283\\
88.55	0.00808898376508262\\
88.56	0.00809055663246598\\
88.57	0.00809212989661928\\
88.58	0.00809370355766632\\
88.59	0.00809527761572829\\
88.6	0.00809685207092374\\
88.61	0.00809842692336852\\
88.62	0.00810000217317576\\
88.63	0.00810157782045583\\
88.64	0.00810315386531628\\
88.65	0.00810473030786181\\
88.66	0.00810630714819425\\
88.67	0.00810788438641248\\
88.68	0.00810946202261243\\
88.69	0.00811104005688701\\
88.7	0.00811261848932605\\
88.71	0.00811419732001634\\
88.72	0.00811577654904147\\
88.73	0.00811735617648189\\
88.74	0.00811893620241481\\
88.75	0.00812051662691415\\
88.76	0.00812209745005055\\
88.77	0.00812367867189127\\
88.78	0.00812526029250017\\
88.79	0.00812684231193767\\
88.8	0.00812842473026067\\
88.81	0.00813000754752255\\
88.82	0.00813159076377311\\
88.83	0.0081331743790585\\
88.84	0.00813475839342119\\
88.85	0.00813634280689994\\
88.86	0.0081379276195297\\
88.87	0.00813951283134162\\
88.88	0.00814109844236296\\
88.89	0.00814268445261708\\
88.9	0.00814427086212336\\
88.91	0.00814585767089714\\
88.92	0.00814744487894969\\
88.93	0.00814903248628819\\
88.94	0.0081506204929156\\
88.95	0.00815220889883069\\
88.96	0.00815379770402794\\
88.97	0.00815538690849749\\
88.98	0.0081569765122251\\
88.99	0.00815856651519211\\
89	0.00816015691737536\\
89.01	0.00816174771874713\\
89.02	0.00816333891927514\\
89.03	0.00816493051892242\\
89.04	0.00816652251764732\\
89.05	0.0081681149154034\\
89.06	0.00816970771213941\\
89.07	0.00817130090779925\\
89.08	0.00817289450232184\\
89.09	0.00817448849564114\\
89.1	0.00817608288768605\\
89.11	0.00817767767838037\\
89.12	0.00817927286764273\\
89.13	0.00818086845538652\\
89.14	0.00818246444151987\\
89.15	0.00818406082594554\\
89.16	0.00818565760856092\\
89.17	0.00818725478925789\\
89.18	0.00818885236792282\\
89.19	0.00819045034443835\\
89.2	0.00819204871868663\\
89.21	0.00819364749054929\\
89.22	0.00819524665990746\\
89.23	0.00819684622664177\\
89.24	0.00819844619063233\\
89.25	0.00820004655175874\\
89.26	0.00820164730990011\\
89.27	0.00820324846493501\\
89.28	0.00820485001674151\\
89.29	0.00820645196519716\\
89.3	0.00820805431017899\\
89.31	0.00820965705156352\\
89.32	0.00821126018922673\\
89.33	0.00821286372304409\\
89.34	0.00821446765289054\\
89.35	0.0082160719786405\\
89.36	0.00821767670016786\\
89.37	0.00821928181734598\\
89.38	0.00822088733004769\\
89.39	0.00822249323814528\\
89.4	0.00822409954151051\\
89.41	0.00822570624001461\\
89.42	0.00822731333352826\\
89.43	0.00822892082192164\\
89.44	0.00823052870506433\\
89.45	0.0082321369828254\\
89.46	0.00823374565507339\\
89.47	0.00823535472167628\\
89.48	0.00823696418250149\\
89.49	0.00823857403741591\\
89.5	0.00824018428628589\\
89.51	0.0082417949289772\\
89.52	0.00824340596535508\\
89.53	0.00824501739528421\\
89.54	0.00824662921862871\\
89.55	0.00824824143525216\\
89.56	0.00824985404501757\\
89.57	0.00825146704778737\\
89.58	0.00825308044342348\\
89.59	0.00825469423178721\\
89.6	0.00825630841273933\\
89.61	0.00825792298614004\\
89.62	0.00825953795184898\\
89.63	0.0082611533097252\\
89.64	0.00826276905962721\\
89.65	0.00826438520141293\\
89.66	0.00826600173493972\\
89.67	0.00826761866006436\\
89.68	0.00826923597664306\\
89.69	0.00827085368453146\\
89.7	0.00827247178358459\\
89.71	0.00827409027365696\\
89.72	0.00827570915460246\\
89.73	0.0082773284262744\\
89.74	0.00827894808852552\\
89.75	0.00828056814120797\\
89.76	0.00828218858417333\\
89.77	0.00828380941727259\\
89.78	0.00828543064035613\\
89.79	0.00828705225327378\\
89.8	0.00828867425587475\\
89.81	0.00829029664800769\\
89.82	0.00829191942952063\\
89.83	0.00829354260026103\\
89.84	0.00829516616007574\\
89.85	0.00829679010881103\\
89.86	0.00829841444631258\\
89.87	0.00830003917242545\\
89.88	0.00830166428699412\\
89.89	0.00830328978986248\\
89.9	0.00830491568087381\\
89.91	0.00830654195987079\\
89.92	0.0083081686266955\\
89.93	0.00830979568118942\\
89.94	0.00831142312319343\\
89.95	0.00831305095254782\\
89.96	0.00831467916909224\\
89.97	0.00831630777266576\\
89.98	0.00831793676310687\\
89.99	0.00831956614025339\\
90	0.0083211959039426\\
90.01	0.00832282605401112\\
90.02	0.008324456590295\\
90.03	0.00832608751262966\\
90.04	0.00832771882084991\\
90.05	0.00832935051478997\\
90.06	0.00833098259428342\\
90.07	0.00833261505916325\\
90.08	0.00833424790926184\\
90.09	0.00833588114441094\\
90.1	0.00833751476444171\\
90.11	0.00833914876918467\\
90.12	0.00834078315846975\\
90.13	0.00834241793212626\\
90.14	0.00834405308998288\\
90.15	0.0083456886318677\\
90.16	0.00834732455760819\\
90.17	0.00834896086703118\\
90.18	0.0083505975599629\\
90.19	0.00835223463622899\\
90.2	0.00835387209565443\\
90.21	0.00835550993806362\\
90.22	0.00835714816328032\\
90.23	0.00835878677112768\\
90.24	0.00836042576142824\\
90.25	0.00836206513400392\\
90.26	0.00836370488867601\\
90.27	0.00836534502526521\\
90.28	0.00836698554359158\\
90.29	0.00836862644347458\\
90.3	0.00837026772473304\\
90.31	0.00837190938718519\\
90.32	0.00837355143064862\\
90.33	0.00837519385494034\\
90.34	0.0083768366598767\\
90.35	0.00837847984527347\\
90.36	0.0083801234109458\\
90.37	0.0083817673567082\\
90.38	0.00838341168237461\\
90.39	0.0083850563877583\\
90.4	0.00838670147267199\\
90.41	0.00838834693692773\\
90.42	0.008389992780337\\
90.43	0.00839163900271064\\
90.44	0.0083932856038589\\
90.45	0.00839493258359141\\
90.46	0.00839657994171718\\
90.47	0.00839822767804464\\
90.48	0.00839987579238158\\
90.49	0.00840152428453521\\
90.5	0.00840317315431211\\
90.51	0.00840482240151828\\
90.52	0.00840647202595909\\
90.53	0.00840812202743933\\
90.54	0.00840977240576316\\
90.55	0.00841142316073418\\
90.56	0.00841307429215535\\
90.57	0.00841472579982905\\
90.58	0.00841637768355707\\
90.59	0.00841802994314058\\
90.6	0.00841968257838018\\
90.61	0.00842133558907586\\
90.62	0.00842298897502703\\
90.63	0.00842464273603251\\
90.64	0.00842629687189052\\
90.65	0.00842795138239869\\
90.66	0.00842960626735409\\
90.67	0.00843126152655317\\
90.68	0.00843291715979184\\
90.69	0.00843457316686539\\
90.7	0.00843622954756857\\
90.71	0.00843788630169551\\
90.72	0.0084395434290398\\
90.73	0.00844120092939445\\
90.74	0.00844285880255189\\
90.75	0.008444517048304\\
90.76	0.00844617566644207\\
90.77	0.00844783465675707\\
90.78	0.00844949401903992\\
90.79	0.00845115375308155\\
90.8	0.00845281385867283\\
90.81	0.00845447433560465\\
90.82	0.00845613518366784\\
90.83	0.00845779640265326\\
90.84	0.00845945799235172\\
90.85	0.00846111995255402\\
90.86	0.00846278228305096\\
90.87	0.00846444498363333\\
90.88	0.00846610805409191\\
90.89	0.00846777149421745\\
90.9	0.00846943530380073\\
90.91	0.00847109948263251\\
90.92	0.00847276403050353\\
90.93	0.00847442894720456\\
90.94	0.00847609423252634\\
90.95	0.00847775988625964\\
90.96	0.00847942590819522\\
90.97	0.00848109229812384\\
90.98	0.00848275905583627\\
90.99	0.00848442618112329\\
91	0.00848609367377571\\
91.01	0.0084877615335843\\
91.02	0.00848942976033989\\
91.03	0.00849109835383331\\
91.04	0.00849276731385541\\
91.05	0.00849443664019705\\
91.06	0.0084961063326491\\
91.07	0.00849777639100247\\
91.08	0.00849944681504809\\
91.09	0.00850111760457691\\
91.1	0.00850278875937989\\
91.11	0.00850446027924806\\
91.12	0.00850613216397242\\
91.13	0.00850780441334406\\
91.14	0.00850947702715405\\
91.15	0.00851115000519354\\
91.16	0.00851282334725368\\
91.17	0.00851449705312567\\
91.18	0.00851617112260076\\
91.19	0.00851784555547022\\
91.2	0.00851952035152537\\
91.21	0.00852119551055757\\
91.22	0.00852287103235824\\
91.23	0.00852454691671884\\
91.24	0.00852622316343087\\
91.25	0.00852789977228588\\
91.26	0.00852957674307548\\
91.27	0.00853125407559134\\
91.28	0.00853293176962516\\
91.29	0.00853460982496872\\
91.3	0.00853628824141385\\
91.31	0.00853796701875244\\
91.32	0.00853964615677645\\
91.33	0.00854132565527789\\
91.34	0.00854300551404883\\
91.35	0.00854468573288144\\
91.36	0.00854636631156791\\
91.37	0.00854804724990055\\
91.38	0.00854972854767171\\
91.39	0.00855141020467381\\
91.4	0.00855309222069938\\
91.41	0.00855477459554098\\
91.42	0.00855645732899128\\
91.43	0.00855814042084302\\
91.44	0.00855982387088903\\
91.45	0.00856150767892221\\
91.46	0.00856319184473557\\
91.47	0.00856487636812216\\
91.48	0.00856656124887518\\
91.49	0.00856824648678786\\
91.5	0.00856993208165357\\
91.51	0.00857161803326574\\
91.52	0.00857330434141793\\
91.53	0.00857499100590377\\
91.54	0.00857667802651698\\
91.55	0.00857836540305141\\
91.56	0.00858005313530101\\
91.57	0.0085817412230598\\
91.58	0.00858342966612195\\
91.59	0.0085851184642817\\
91.6	0.00858680761733342\\
91.61	0.0085884971250716\\
91.62	0.00859018698729081\\
91.63	0.00859187720378577\\
91.64	0.00859356777435129\\
91.65	0.00859525869878231\\
91.66	0.0085969499768739\\
91.67	0.00859864160842122\\
91.68	0.00860033359321958\\
91.69	0.0086020259310644\\
91.7	0.00860371862175125\\
91.71	0.00860541166507581\\
91.72	0.00860710506083388\\
91.73	0.0086087988088214\\
91.74	0.00861049290883446\\
91.75	0.00861218736066927\\
91.76	0.00861388216412218\\
91.77	0.00861557731898968\\
91.78	0.00861727282506839\\
91.79	0.00861896868215509\\
91.8	0.00862066489004669\\
91.81	0.00862236144854026\\
91.82	0.008624058357433\\
91.83	0.00862575561652227\\
91.84	0.00862745322560558\\
91.85	0.00862915118448059\\
91.86	0.00863084949294513\\
91.87	0.00863254815079715\\
91.88	0.0086342471578348\\
91.89	0.00863594651385637\\
91.9	0.0086376462186603\\
91.91	0.00863934627204521\\
91.92	0.0086410466738099\\
91.93	0.0086427474237533\\
91.94	0.00864444852167452\\
91.95	0.00864614996737287\\
91.96	0.00864785176064779\\
91.97	0.00864955390129893\\
91.98	0.00865125638912609\\
91.99	0.00865295922392927\\
92	0.00865466240550861\\
92.01	0.00865636593366449\\
92.02	0.00865806980819741\\
92.03	0.00865977402890811\\
92.04	0.00866147859559748\\
92.05	0.00866318350806661\\
92.06	0.00866488876611679\\
92.07	0.00866659436954947\\
92.08	0.00866830031816633\\
92.09	0.00867000661176924\\
92.1	0.00867171325016024\\
92.11	0.0086734202331416\\
92.12	0.00867512756051576\\
92.13	0.0086768352320854\\
92.14	0.00867854324765336\\
92.15	0.00868025160702273\\
92.16	0.00868196030999677\\
92.17	0.00868366935637898\\
92.18	0.00868537874597305\\
92.19	0.0086870884785829\\
92.2	0.00868879855401265\\
92.21	0.00869050897206663\\
92.22	0.00869221973254942\\
92.23	0.00869393083526579\\
92.24	0.00869564228002074\\
92.25	0.0086973540666195\\
92.26	0.00869906619486752\\
92.27	0.00870077866457049\\
92.28	0.0087024914755343\\
92.29	0.00870420462756509\\
92.3	0.00870591812046923\\
92.31	0.00870763195405334\\
92.32	0.00870934612812425\\
92.33	0.00871106064248903\\
92.34	0.00871277549695501\\
92.35	0.00871449069132974\\
92.36	0.00871620622542103\\
92.37	0.00871792209903691\\
92.38	0.0087196383119857\\
92.39	0.00872135486407593\\
92.4	0.00872307175511638\\
92.41	0.00872478898491611\\
92.42	0.0087265065532844\\
92.43	0.00872822446003081\\
92.44	0.00872994270496516\\
92.45	0.0087316612878975\\
92.46	0.00873338020863817\\
92.47	0.00873509946699776\\
92.48	0.00873681906278711\\
92.49	0.00873853899581736\\
92.5	0.00874025926589988\\
92.51	0.00874197987284634\\
92.52	0.00874370081646866\\
92.53	0.00874542209657904\\
92.54	0.00874714371298997\\
92.55	0.00874886566551418\\
92.56	0.00875058795396471\\
92.57	0.00875231057815486\\
92.58	0.00875403353789823\\
92.59	0.00875575683300869\\
92.6	0.00875748046330039\\
92.61	0.00875920442858777\\
92.62	0.00876092872868557\\
92.63	0.00876265336340882\\
92.64	0.00876437833257281\\
92.65	0.00876610363599317\\
92.66	0.00876782927348579\\
92.67	0.00876955524486686\\
92.68	0.00877128154995289\\
92.69	0.00877300818856067\\
92.7	0.00877473516050729\\
92.71	0.00877646246561016\\
92.72	0.00877819010368698\\
92.73	0.00877991807455578\\
92.74	0.00878164637803485\\
92.75	0.00878337501394285\\
92.76	0.0087851039820987\\
92.77	0.00878683328232167\\
92.78	0.00878856291443133\\
92.79	0.00879029287824756\\
92.8	0.00879202317359057\\
92.81	0.0087937538002809\\
92.82	0.00879548475813937\\
92.83	0.00879721604698718\\
92.84	0.0087989476666458\\
92.85	0.00880067961693707\\
92.86	0.00880241189768313\\
92.87	0.00880414450870647\\
92.88	0.0088058774498299\\
92.89	0.00880761072087655\\
92.9	0.00880934432166993\\
92.91	0.00881107825203382\\
92.92	0.00881281251179239\\
92.93	0.00881454710077015\\
92.94	0.0088162820187919\\
92.95	0.00881801726568284\\
92.96	0.00881975284126848\\
92.97	0.0088214887453747\\
92.98	0.00882322497782769\\
92.99	0.00882496153845404\\
93	0.00882669842708064\\
93.01	0.00882843564353477\\
93.02	0.00883017318764404\\
93.03	0.00883191105923643\\
93.04	0.00883364925814026\\
93.05	0.00883538778418424\\
93.06	0.00883712663719739\\
93.07	0.00883886581700915\\
93.08	0.00884060532344926\\
93.09	0.00884234515634789\\
93.1	0.00884408531553551\\
93.11	0.00884582580084302\\
93.12	0.00884756661210163\\
93.13	0.00884930774914297\\
93.14	0.00885104921179901\\
93.15	0.00885279099990211\\
93.16	0.008854533113285\\
93.17	0.00885627555178077\\
93.18	0.00885801831522291\\
93.19	0.00885976140344528\\
93.2	0.00886150481628213\\
93.21	0.00886324855356808\\
93.22	0.00886499261513813\\
93.23	0.00886673700082768\\
93.24	0.0088684817104725\\
93.25	0.00887022674390876\\
93.26	0.00887197210097301\\
93.27	0.0088737177815022\\
93.28	0.00887546378533366\\
93.29	0.00887721011230513\\
93.3	0.00887895676225473\\
93.31	0.00888070373502097\\
93.32	0.00888245103044278\\
93.33	0.00888419864835946\\
93.34	0.00888594658861074\\
93.35	0.00888769485103674\\
93.36	0.00888944343547797\\
93.37	0.00889119234177535\\
93.38	0.00889294156977021\\
93.39	0.0088946911193043\\
93.4	0.00889644099021974\\
93.41	0.00889819118235909\\
93.42	0.00889994169556532\\
93.43	0.00890169252968179\\
93.44	0.00890344368455229\\
93.45	0.00890519516002103\\
93.46	0.00890694695593261\\
93.47	0.00890869907213207\\
93.48	0.00891045150846486\\
93.49	0.00891220426477684\\
93.5	0.00891395734091431\\
93.51	0.00891571073672397\\
93.52	0.00891746445205295\\
93.53	0.00891921848674882\\
93.54	0.00892097284065955\\
93.55	0.00892272751363355\\
93.56	0.00892448250551964\\
93.57	0.0089262378161671\\
93.58	0.0089279934454256\\
93.59	0.00892974939314528\\
93.6	0.00893150565917668\\
93.61	0.00893326224337079\\
93.62	0.00893501914557903\\
93.63	0.00893677636565325\\
93.64	0.00893853390344573\\
93.65	0.00894029175880921\\
93.66	0.00894204993159684\\
93.67	0.00894380842166223\\
93.68	0.00894556722885942\\
93.69	0.00894732635304289\\
93.7	0.00894908579406756\\
93.71	0.0089508455517888\\
93.72	0.00895260562606242\\
93.73	0.00895436601674468\\
93.74	0.00895612672369227\\
93.75	0.00895788774676234\\
93.76	0.00895964908581249\\
93.77	0.00896141074070075\\
93.78	0.00896317271128562\\
93.79	0.00896493499742604\\
93.8	0.00896669759898139\\
93.81	0.00896846051581152\\
93.82	0.00897022374777673\\
93.83	0.00897198729473777\\
93.84	0.00897375115655583\\
93.85	0.00897551533309257\\
93.86	0.00897727982421012\\
93.87	0.00897904462977102\\
93.88	0.00898080974963832\\
93.89	0.0089825751836755\\
93.9	0.00898434093174649\\
93.91	0.00898610699371571\\
93.92	0.008987873369448\\
93.93	0.0089896400588087\\
93.94	0.00899140706166358\\
93.95	0.0089931743778789\\
93.96	0.00899494200732135\\
93.97	0.00899670994985811\\
93.98	0.00899847820535681\\
93.99	0.00900024677368555\\
94	0.00900201565471289\\
94.01	0.00900378484830787\\
94.02	0.00900555435433996\\
94.03	0.00900732417267914\\
94.04	0.00900909430319583\\
94.05	0.00901086474576093\\
94.06	0.0090126355002458\\
94.07	0.00901440656652225\\
94.08	0.0090161779444626\\
94.09	0.00901794963393962\\
94.1	0.00901972163482654\\
94.11	0.00902149394699706\\
94.12	0.00902326657032537\\
94.13	0.0090250395046861\\
94.14	0.00902681274995439\\
94.15	0.00902858630600581\\
94.16	0.00903036017271643\\
94.17	0.00903213434996278\\
94.18	0.00903390883762187\\
94.19	0.00903568363557117\\
94.2	0.00903745874368862\\
94.21	0.00903923416185266\\
94.22	0.00904100988994217\\
94.23	0.00904278592783652\\
94.24	0.00904456227541554\\
94.25	0.00904633893255957\\
94.26	0.00904811589914936\\
94.27	0.00904989317506619\\
94.28	0.00905167076019179\\
94.29	0.00905344865440837\\
94.3	0.00905522685759859\\
94.31	0.00905700536964562\\
94.32	0.00905878419043308\\
94.33	0.00906056331984506\\
94.34	0.00906234275776614\\
94.35	0.00906412250408137\\
94.36	0.00906590255867625\\
94.37	0.00906768292143679\\
94.38	0.00906946359224944\\
94.39	0.00907124457100114\\
94.4	0.00907302585757929\\
94.41	0.00907480745187178\\
94.42	0.00907658935376695\\
94.43	0.00907837156315364\\
94.44	0.00908015407992112\\
94.45	0.00908193690395918\\
94.46	0.00908372003515804\\
94.47	0.0090855034734084\\
94.48	0.00908728721860145\\
94.49	0.00908907127062883\\
94.5	0.00909085562938265\\
94.51	0.00909264029475549\\
94.52	0.00909442526664041\\
94.53	0.00909621054493092\\
94.54	0.009097996129521\\
94.55	0.0090997820203051\\
94.56	0.00910156821717814\\
94.57	0.00910335472003551\\
94.58	0.00910514152877305\\
94.59	0.00910692864328706\\
94.6	0.00910871606347432\\
94.61	0.00911050378923207\\
94.62	0.00911229182045799\\
94.63	0.00911408015705026\\
94.64	0.00911586879890749\\
94.65	0.00911765774592875\\
94.66	0.00911944699801358\\
94.67	0.00912123655506197\\
94.68	0.00912302641697438\\
94.69	0.0091248165836517\\
94.7	0.0091266070549953\\
94.71	0.00912839783090699\\
94.72	0.00913018891128904\\
94.73	0.00913198029604417\\
94.74	0.00913377198507554\\
94.75	0.00913556397828678\\
94.76	0.00913735627558195\\
94.77	0.00913914887686556\\
94.78	0.00914094178204258\\
94.79	0.0091427349910184\\
94.8	0.00914452850369889\\
94.81	0.00914632231999033\\
94.82	0.00914811643979945\\
94.83	0.00914991086303343\\
94.84	0.00915170558959988\\
94.85	0.00915350061940684\\
94.86	0.00915529595236281\\
94.87	0.00915709158837669\\
94.88	0.00915888752735784\\
94.89	0.00916068376921605\\
94.9	0.00916248031386152\\
94.91	0.0091642771612049\\
94.92	0.00916607431115725\\
94.93	0.00916787176363007\\
94.94	0.00916966951853527\\
94.95	0.0091714675757852\\
94.96	0.0091732659352926\\
94.97	0.00917506459697067\\
94.98	0.009176863560733\\
94.99	0.00917866282649358\\
95	0.00918046239416685\\
95.01	0.00918226226366763\\
95.02	0.00918406243491118\\
95.03	0.00918586290781314\\
95.04	0.00918766368228957\\
95.05	0.00918946475825692\\
95.06	0.00919126613563207\\
95.07	0.00919306781433226\\
95.08	0.00919486979427517\\
95.09	0.00919667207537883\\
95.1	0.00919847465756172\\
95.11	0.00920027754074264\\
95.12	0.00920208072484084\\
95.13	0.00920388420977593\\
95.14	0.00920568799546791\\
95.15	0.00920749208183716\\
95.16	0.00920929646880444\\
95.17	0.00921110115629089\\
95.18	0.00921290614421801\\
95.19	0.0092147114325077\\
95.2	0.00921651702108221\\
95.21	0.00921832290986416\\
95.22	0.00922012909877654\\
95.23	0.00922193558774271\\
95.24	0.00922374237668637\\
95.25	0.00922554946553159\\
95.26	0.00922735685420278\\
95.27	0.00922916454262474\\
95.28	0.00923097253072258\\
95.29	0.00923278081842176\\
95.3	0.0092345894056481\\
95.31	0.00923639829232776\\
95.32	0.00923820747838722\\
95.33	0.00924001696375331\\
95.34	0.00924182674835319\\
95.35	0.00924363683211434\\
95.36	0.00924544721496457\\
95.37	0.00924725789683202\\
95.38	0.00924906887764515\\
95.39	0.00925088015733271\\
95.4	0.0092526917358238\\
95.41	0.0092545036130478\\
95.42	0.00925631578893443\\
95.43	0.00925812826341366\\
95.44	0.00925994103641582\\
95.45	0.0092617541078715\\
95.46	0.00926356747771158\\
95.47	0.00926538114586726\\
95.48	0.00926719511226999\\
95.49	0.00926900937685153\\
95.5	0.0092708239395439\\
95.51	0.00927263880027942\\
95.52	0.00927445395899064\\
95.53	0.00927626941561041\\
95.54	0.00927808517007184\\
95.55	0.00927990122230831\\
95.56	0.00928171757225341\\
95.57	0.00928353421984105\\
95.58	0.00928535116500533\\
95.59	0.00928716840768062\\
95.6	0.00928898594780154\\
95.61	0.00929080378530293\\
95.62	0.00929262192011986\\
95.63	0.00929444035218764\\
95.64	0.0092962590814418\\
95.65	0.00929807810781809\\
95.66	0.00929989743125247\\
95.67	0.00930171705168111\\
95.68	0.00930353696904041\\
95.69	0.00930535718326693\\
95.7	0.00930717769429747\\
95.71	0.009308998502069\\
95.72	0.00931081960651868\\
95.73	0.00931264100758387\\
95.74	0.00931446270520208\\
95.75	0.00931628469931103\\
95.76	0.00931810698984858\\
95.77	0.00931992957675278\\
95.78	0.00932175245996181\\
95.79	0.00932357563941404\\
95.8	0.00932539911504797\\
95.81	0.00932722288680226\\
95.82	0.00932904695461568\\
95.83	0.00933087131842718\\
95.84	0.0093326959781758\\
95.85	0.00933452093380073\\
95.86	0.00933634618524128\\
95.87	0.00933817173243686\\
95.88	0.009339997575327\\
95.89	0.00934182371385134\\
95.9	0.0093436501479496\\
95.91	0.0093454768775616\\
95.92	0.00934730390262727\\
95.93	0.0093491312230866\\
95.94	0.00935095883887966\\
95.95	0.0093527867499466\\
95.96	0.00935461495622761\\
95.97	0.00935644345766298\\
95.98	0.00935827225419303\\
95.99	0.00936010134575811\\
96	0.00936193073229866\\
96.01	0.00936376041375512\\
96.02	0.00936559039006797\\
96.03	0.00936742066117771\\
96.04	0.00936925122702487\\
96.05	0.00937108208754999\\
96.06	0.00937291324269359\\
96.07	0.00937474469239623\\
96.08	0.00937657643659843\\
96.09	0.00937840847524072\\
96.1	0.0093802408082636\\
96.11	0.00938207343560755\\
96.12	0.009383906357213\\
96.13	0.00938573957302036\\
96.14	0.00938757308296999\\
96.15	0.0093894068870022\\
96.16	0.00939124098505722\\
96.17	0.00939307537707525\\
96.18	0.00939491006299639\\
96.19	0.00939674504276067\\
96.2	0.00939858031630804\\
96.21	0.00940041588357833\\
96.22	0.00940225174451131\\
96.23	0.0094040878990466\\
96.24	0.00940592434712375\\
96.25	0.00940776108868215\\
96.26	0.00940959812366107\\
96.27	0.00941143545199966\\
96.28	0.0094132730736369\\
96.29	0.00941511098851165\\
96.3	0.00941694919656258\\
96.31	0.00941878769772821\\
96.32	0.00942062649194689\\
96.33	0.00942246557915678\\
96.34	0.00942430495929584\\
96.35	0.00942614463230186\\
96.36	0.0094279845981124\\
96.37	0.00942982485666481\\
96.38	0.00943166540789624\\
96.39	0.00943350625174358\\
96.4	0.00943534738814351\\
96.41	0.00943718881703245\\
96.42	0.00943903053834657\\
96.43	0.00944087255202178\\
96.44	0.00944271485799372\\
96.45	0.00944455745619774\\
96.46	0.00944640034656892\\
96.47	0.00944824352904203\\
96.48	0.00945008700355156\\
96.49	0.00945193077003166\\
96.5	0.00945377482841617\\
96.51	0.0094556191786386\\
96.52	0.00945746382063212\\
96.53	0.00945930875432956\\
96.54	0.00946115397966338\\
96.55	0.00946299949656569\\
96.56	0.00946484530496821\\
96.57	0.00946669140480228\\
96.58	0.00946853779599887\\
96.59	0.00947038447848851\\
96.6	0.00947223145220134\\
96.61	0.00947407871706709\\
96.62	0.00947592627301503\\
96.63	0.00947777411997402\\
96.64	0.00947962225787245\\
96.65	0.00948147068663828\\
96.66	0.00948331940619895\\
96.67	0.00948516841648148\\
96.68	0.00948701771741236\\
96.69	0.00948886730891761\\
96.7	0.00949071719092272\\
96.71	0.00949256736335269\\
96.72	0.00949441782613196\\
96.73	0.00949626857918445\\
96.74	0.00949811962243354\\
96.75	0.00949997095580204\\
96.76	0.00950182257921218\\
96.77	0.00950367449258565\\
96.78	0.0095055266958435\\
96.79	0.00950737918890623\\
96.8	0.00950923197169369\\
96.81	0.00951108504412514\\
96.82	0.00951293840611919\\
96.83	0.0095147920575938\\
96.84	0.0095166459984663\\
96.85	0.00951850022865335\\
96.86	0.00952035474807092\\
96.87	0.00952220955663431\\
96.88	0.00952406465425812\\
96.89	0.00952592004085623\\
96.9	0.00952777571634182\\
96.91	0.00952963168062733\\
96.92	0.00953148793362446\\
96.93	0.00953334447524414\\
96.94	0.00953520130539657\\
96.95	0.00953705842399114\\
96.96	0.00953891583093647\\
96.97	0.00954077352614038\\
96.98	0.00954263150950987\\
96.99	0.00954448978095112\\
97	0.00954634834036948\\
97.01	0.00954820718766945\\
97.02	0.00955006632275467\\
97.03	0.00955192574552791\\
97.04	0.00955378545589105\\
97.05	0.0095556454537451\\
97.06	0.00955750573899012\\
97.07	0.00955936631152529\\
97.08	0.00956122717124883\\
97.09	0.00956308831805804\\
97.1	0.00956494975184924\\
97.11	0.00956681147251778\\
97.12	0.00956867347995805\\
97.13	0.00957053577406342\\
97.14	0.00957239835472627\\
97.15	0.00957426122183794\\
97.16	0.00957612437528876\\
97.17	0.00957798781496799\\
97.18	0.00957985154076384\\
97.19	0.00958171555256344\\
97.2	0.00958357985025284\\
97.21	0.009585444433717\\
97.22	0.00958730930283974\\
97.23	0.00958917445750377\\
97.24	0.00959103989759066\\
97.25	0.00959290562298083\\
97.26	0.00959477163355351\\
97.27	0.00959663792918678\\
97.28	0.0095985045097575\\
97.29	0.00960037137514133\\
97.3	0.00960223852521271\\
97.31	0.00960410595984484\\
97.32	0.00960597367890967\\
97.33	0.00960784168227788\\
97.34	0.00960970996981887\\
97.35	0.00961157854140076\\
97.36	0.00961344739689035\\
97.37	0.00961531653615313\\
97.38	0.00961718595905322\\
97.39	0.00961905566545344\\
97.4	0.00962092565521519\\
97.41	0.00962279592819854\\
97.42	0.00962466648426212\\
97.43	0.00962653732326318\\
97.44	0.00962840844505754\\
97.45	0.00963027984949956\\
97.46	0.00963215153644217\\
97.47	0.00963402350573681\\
97.48	0.00963589575723345\\
97.49	0.00963776829078056\\
97.5	0.00963964110622508\\
97.51	0.00964151420341242\\
97.52	0.00964338758218643\\
97.53	0.00964526124238939\\
97.54	0.00964713518386195\\
97.55	0.00964900940644317\\
97.56	0.0096508839099705\\
97.57	0.00965275869427971\\
97.58	0.00965463375920494\\
97.59	0.00965650910457864\\
97.6	0.00965838473023159\\
97.61	0.00966026063599285\\
97.62	0.00966213682168977\\
97.63	0.00966401328714796\\
97.64	0.00966589003220603\\
97.65	0.00966776705676154\\
97.66	0.00966964436070889\\
97.67	0.00967152194393926\\
97.68	0.0096733998063406\\
97.69	0.00967527794779758\\
97.7	0.00967715636819153\\
97.71	0.00967903506740043\\
97.72	0.00968091404529885\\
97.73	0.00968279330175791\\
97.74	0.00968467283664523\\
97.75	0.00968655264982492\\
97.76	0.00968843274234247\\
97.77	0.00969031311591114\\
97.78	0.00969219377226286\\
97.79	0.00969407471314843\\
97.8	0.00969595589427488\\
97.81	0.00969783706790343\\
97.82	0.00969971823671616\\
97.83	0.0097015994034335\\
97.84	0.00970348057081461\\
97.85	0.00970536174165778\\
97.86	0.0097072429188009\\
97.87	0.00970912410512182\\
97.88	0.00971100530353872\\
97.89	0.00971288651701043\\
97.9	0.0097147612775502\\
97.91	0.00971662898349836\\
97.92	0.00971848956704936\\
97.93	0.00972034295969377\\
97.94	0.00972218909220971\\
97.95	0.00972402789465418\\
97.96	0.00972585929658671\\
97.97	0.00972768322722907\\
97.98	0.00972949961505148\\
97.99	0.0097313083878552\\
98	0.00973310947272936\\
98.01	0.00973490290657201\\
98.02	0.00973668877760677\\
98.03	0.00973846701010021\\
98.04	0.00974023752749241\\
98.05	0.00974200025238653\\
98.06	0.00974375510653817\\
98.07	0.00974550006656619\\
98.08	0.00974723896368855\\
98.09	0.00974897317730456\\
98.1	0.00975070268116027\\
98.11	0.00975242744883675\\
98.12	0.0097541474537496\\
98.13	0.0097558626691484\\
98.14	0.00975757306811625\\
98.15	0.00975927862356923\\
98.16	0.00976097930825596\\
98.17	0.00976267509475716\\
98.18	0.00976436595548521\\
98.19	0.00976605225165897\\
98.2	0.00976773404885576\\
98.21	0.00976941170560958\\
98.22	0.00977108694750106\\
98.23	0.0097727597627094\\
98.24	0.00977443013941011\\
98.25	0.00977609806577754\\
98.26	0.00977776352998746\\
98.27	0.00977942652021976\\
98.28	0.00978108702466129\\
98.29	0.00978274503150872\\
98.3	0.00978440052863336\\
98.31	0.00978605350385307\\
98.32	0.00978770394500337\\
98.33	0.00978935183921957\\
98.34	0.0097909971659391\\
98.35	0.00979263990441897\\
98.36	0.00979428003373453\\
98.37	0.00979591753233545\\
98.38	0.00979755237846595\\
98.39	0.00979918455017923\\
98.4	0.00980081402533612\\
98.41	0.00980244078160381\\
98.42	0.00980406479645448\\
98.43	0.00980568604504303\\
98.44	0.0098073045016238\\
98.45	0.00980892014019995\\
98.46	0.00981053293452088\\
98.47	0.0098121428580794\\
98.48	0.00981374988410908\\
98.49	0.00981535398558175\\
98.5	0.00981695513520473\\
98.51	0.00981855330541826\\
98.52	0.00982014846839274\\
98.53	0.00982174059602603\\
98.54	0.00982332966000768\\
98.55	0.00982491563234843\\
98.56	0.00982649848478402\\
98.57	0.00982807818877242\\
98.58	0.00982965471549099\\
98.59	0.00983122803583362\\
98.6	0.00983279812040779\\
98.61	0.00983436493953169\\
98.62	0.0098359284632312\\
98.63	0.00983748866123693\\
98.64	0.00983904550298116\\
98.65	0.00984059895759477\\
98.66	0.00984214899348074\\
98.67	0.00984369557858251\\
98.68	0.00984523868052557\\
98.69	0.00984677826661434\\
98.7	0.0098483143038289\\
98.71	0.00984984675882157\\
98.72	0.00985137559791348\\
98.73	0.00985290078709106\\
98.74	0.00985442229200414\\
98.75	0.00985594007797\\
98.76	0.00985745410997008\\
98.77	0.00985896435264673\\
98.78	0.00986047077029981\\
98.79	0.00986197332688554\\
98.8	0.0098634719860138\\
98.81	0.00986496671094474\\
98.82	0.00986645746458541\\
98.83	0.00986794420948751\\
98.84	0.00986942690769653\\
98.85	0.00987090552071792\\
98.86	0.00987238000969395\\
98.87	0.00987385033540023\\
98.88	0.00987531645824225\\
98.89	0.00987677833825183\\
98.9	0.00987823593508357\\
98.91	0.00987968920801123\\
98.92	0.00988113811592412\\
98.93	0.00988258261732343\\
98.94	0.00988402267031852\\
98.95	0.00988545823262319\\
98.96	0.0098868892615519\\
98.97	0.00988831571401599\\
98.98	0.00988973754651977\\
98.99	0.00989115471515673\\
99	0.00989256717560554\\
99.01	0.00989397488312613\\
99.02	0.00989537779255572\\
99.03	0.00989677585830473\\
99.04	0.00989816903435277\\
99.05	0.00989955727424447\\
99.06	0.00990094053108541\\
99.07	0.00990231875753784\\
99.08	0.00990369190581653\\
99.09	0.00990505992768447\\
99.1	0.00990642277444855\\
99.11	0.00990778039695523\\
99.12	0.00990913274558614\\
99.13	0.00991047977025362\\
99.14	0.0099118214203963\\
99.15	0.00991315764497448\\
99.16	0.00991448839246565\\
99.17	0.00991581361085984\\
99.18	0.00991713324765496\\
99.19	0.00991844724985212\\
99.2	0.00991975556395086\\
99.21	0.00992105813594439\\
99.22	0.00992235491131472\\
99.23	0.0099236458350278\\
99.24	0.0099249308515286\\
99.25	0.00992620990473608\\
99.26	0.00992748293803825\\
99.27	0.00992874989428702\\
99.28	0.00993001071579311\\
99.29	0.00993126534432089\\
99.3	0.00993251372108315\\
99.31	0.00993375578673581\\
99.32	0.00993499148137264\\
99.33	0.00993622074451984\\
99.34	0.00993744351513067\\
99.35	0.00993865973157995\\
99.36	0.00993986933165849\\
99.37	0.0099410722525676\\
99.38	0.00994226843091338\\
99.39	0.00994345780270105\\
99.4	0.00994464030332924\\
99.41	0.00994581586758415\\
99.42	0.00994698442963372\\
99.43	0.00994814592302171\\
99.44	0.00994930028066174\\
99.45	0.00995044743483127\\
99.46	0.00995158731716552\\
99.47	0.0099527198586513\\
99.48	0.00995384498962087\\
99.49	0.00995496263974562\\
99.5	0.0099560727380298\\
99.51	0.00995717521280411\\
99.52	0.00995826999171929\\
99.53	0.00995935700173958\\
99.54	0.0099604361691362\\
99.55	0.00996150741948069\\
99.56	0.00996257067763823\\
99.57	0.00996362586776092\\
99.58	0.00996467291328088\\
99.59	0.00996571173690344\\
99.6	0.00996674226060019\\
99.61	0.00996776439644508\\
99.62	0.00996877805189459\\
99.63	0.00996978313349925\\
99.64	0.0099707795468949\\
99.65	0.00997176719679398\\
99.66	0.00997274598697662\\
99.67	0.00997371582028172\\
99.68	0.00997467659859792\\
99.69	0.00997562822285446\\
99.7	0.00997657059301199\\
99.71	0.0099775036080533\\
99.72	0.00997842716597387\\
99.73	0.00997934116377243\\
99.74	0.0099802454974414\\
99.75	0.00998114006195719\\
99.76	0.00998202475127047\\
99.77	0.00998289945829632\\
99.78	0.00998376407490432\\
99.79	0.00998461849190844\\
99.8	0.00998546259905699\\
99.81	0.00998629628502236\\
99.82	0.00998711943739068\\
99.83	0.00998793194265142\\
99.84	0.00998873368618689\\
99.85	0.00998952455226156\\
99.86	0.0099903044240114\\
99.87	0.00999107318343302\\
99.88	0.00999183071137277\\
99.89	0.00999257688751569\\
99.9	0.00999331159037439\\
99.91	0.00999403469727783\\
99.92	0.00999474608435993\\
99.93	0.00999544562654817\\
99.94	0.00999613319755202\\
99.95	0.00999680866985125\\
99.96	0.0099974719146842\\
99.97	0.00999812280203583\\
99.98	0.00999876120062581\\
99.99	0.00999938697789635\\
100	0.01\\
};
\addlegendentry{$q=3$};

\addplot [color=green,solid,forget plot]
  table[row sep=crcr]{%
0.01	0.00157538227128272\\
0.02	0.00157555783429435\\
0.03	0.00157573346993446\\
0.04	0.00157590917823627\\
0.05	0.00157608495923298\\
0.06	0.00157626081295787\\
0.07	0.0015764367394442\\
0.08	0.00157661273872527\\
0.09	0.0015767888108344\\
0.1	0.00157696495580494\\
0.11	0.00157714117367026\\
0.12	0.00157731746446374\\
0.13	0.00157749382821881\\
0.14	0.0015776702649689\\
0.15	0.00157784677474749\\
0.16	0.00157802335758805\\
0.17	0.00157820001352409\\
0.18	0.00157837674258916\\
0.19	0.0015785535448168\\
0.2	0.00157873042024061\\
0.21	0.00157890736889418\\
0.22	0.00157908439081116\\
0.23	0.00157926148602518\\
0.24	0.00157943865456993\\
0.25	0.00157961589647911\\
0.26	0.00157979321178646\\
0.27	0.00157997060052571\\
0.28	0.00158014806273064\\
0.29	0.00158032559843506\\
0.3	0.00158050320767277\\
0.31	0.00158068089047764\\
0.32	0.00158085864688352\\
0.33	0.00158103647692431\\
0.34	0.00158121438063394\\
0.35	0.00158139235804634\\
0.36	0.00158157040919549\\
0.37	0.00158174853411537\\
0.38	0.00158192673284\\
0.39	0.00158210500540343\\
0.4	0.00158228335183971\\
0.41	0.00158246177218294\\
0.42	0.00158264026646722\\
0.43	0.0015828188347267\\
0.44	0.00158299747699555\\
0.45	0.00158317619330794\\
0.46	0.0015833549836981\\
0.47	0.00158353384820025\\
0.48	0.00158371278684866\\
0.49	0.00158389179967762\\
0.5	0.00158407088672143\\
0.51	0.00158425004801442\\
0.52	0.00158442928359097\\
0.53	0.00158460859348545\\
0.54	0.00158478797773227\\
0.55	0.00158496743636588\\
0.56	0.00158514696942073\\
0.57	0.00158532657693129\\
0.58	0.00158550625893209\\
0.59	0.00158568601545765\\
0.6	0.00158586584654254\\
0.61	0.00158604575222133\\
0.62	0.00158622573252864\\
0.63	0.0015864057874991\\
0.64	0.00158658591716737\\
0.65	0.00158676612156813\\
0.66	0.0015869464007361\\
0.67	0.00158712675470601\\
0.68	0.00158730718351261\\
0.69	0.00158748768719069\\
0.7	0.00158766826577506\\
0.71	0.00158784891930056\\
0.72	0.00158802964780205\\
0.73	0.00158821045131442\\
0.74	0.00158839132987256\\
0.75	0.00158857228351143\\
0.76	0.00158875331226598\\
0.77	0.00158893441617119\\
0.78	0.0015891155952621\\
0.79	0.00158929684957372\\
0.8	0.00158947817914113\\
0.81	0.00158965958399942\\
0.82	0.00158984106418369\\
0.83	0.0015900226197291\\
0.84	0.00159020425067079\\
0.85	0.00159038595704398\\
0.86	0.00159056773888387\\
0.87	0.00159074959622571\\
0.88	0.00159093152910476\\
0.89	0.00159111353755633\\
0.9	0.00159129562161573\\
0.91	0.00159147778131831\\
0.92	0.00159166001669944\\
0.93	0.0015918423277945\\
0.94	0.00159202471463894\\
0.95	0.00159220717726821\\
0.96	0.00159238971571776\\
0.97	0.00159257233002311\\
0.98	0.00159275502021979\\
0.99	0.00159293778634333\\
1	0.00159312062842934\\
1.01	0.00159330354651341\\
1.02	0.00159348654063117\\
1.03	0.00159366961081828\\
1.04	0.00159385275711042\\
1.05	0.0015940359795433\\
1.06	0.00159421927815266\\
1.07	0.00159440265297427\\
1.08	0.00159458610404389\\
1.09	0.00159476963139737\\
1.1	0.00159495323507052\\
1.11	0.00159513691509923\\
1.12	0.00159532067151938\\
1.13	0.00159550450436689\\
1.14	0.00159568841367771\\
1.15	0.00159587239948782\\
1.16	0.0015960564618332\\
1.17	0.00159624060074988\\
1.18	0.00159642481627392\\
1.19	0.0015966091084414\\
1.2	0.00159679347728842\\
1.21	0.0015969779228511\\
1.22	0.00159716244516561\\
1.23	0.00159734704426814\\
1.24	0.00159753172019488\\
1.25	0.00159771647298209\\
1.26	0.00159790130266602\\
1.27	0.00159808620928296\\
1.28	0.00159827119286923\\
1.29	0.00159845625346118\\
1.3	0.00159864139109517\\
1.31	0.00159882660580761\\
1.32	0.00159901189763492\\
1.33	0.00159919726661353\\
1.34	0.00159938271277995\\
1.35	0.00159956823617067\\
1.36	0.00159975383682222\\
1.37	0.00159993951477116\\
1.38	0.00160012527005407\\
1.39	0.00160031110270757\\
1.4	0.00160049701276829\\
1.41	0.00160068300027291\\
1.42	0.0016008690652581\\
1.43	0.0016010552077606\\
1.44	0.00160124142781716\\
1.45	0.00160142772546454\\
1.46	0.00160161410073954\\
1.47	0.001601800553679\\
1.48	0.00160198708431977\\
1.49	0.00160217369269872\\
1.5	0.00160236037885278\\
1.51	0.00160254714281887\\
1.52	0.00160273398463397\\
1.53	0.00160292090433505\\
1.54	0.00160310790195915\\
1.55	0.00160329497754329\\
1.56	0.00160348213112457\\
1.57	0.00160366936274007\\
1.58	0.00160385667242692\\
1.59	0.00160404406022228\\
1.6	0.00160423152616333\\
1.61	0.00160441907028728\\
1.62	0.00160460669263136\\
1.63	0.00160479439323284\\
1.64	0.00160498217212901\\
1.65	0.00160517002935719\\
1.66	0.00160535796495472\\
1.67	0.00160554597895897\\
1.68	0.00160573407140735\\
1.69	0.00160592224233728\\
1.7	0.00160611049178623\\
1.71	0.00160629881979167\\
1.72	0.00160648722639112\\
1.73	0.0016066757116221\\
1.74	0.0016068642755222\\
1.75	0.001607052918129\\
1.76	0.00160724163948012\\
1.77	0.00160743043961321\\
1.78	0.00160761931856594\\
1.79	0.00160780827637603\\
1.8	0.00160799731308119\\
1.81	0.0016081864287192\\
1.82	0.00160837562332784\\
1.83	0.00160856489694491\\
1.84	0.00160875424960828\\
1.85	0.00160894368135579\\
1.86	0.00160913319222535\\
1.87	0.00160932278225488\\
1.88	0.00160951245148235\\
1.89	0.00160970219994572\\
1.9	0.001609892027683\\
1.91	0.00161008193473224\\
1.92	0.0016102719211315\\
1.93	0.00161046198691887\\
1.94	0.00161065213213245\\
1.95	0.00161084235681042\\
1.96	0.00161103266099093\\
1.97	0.0016112230447122\\
1.98	0.00161141350801244\\
1.99	0.00161160405092992\\
2	0.00161179467350289\\
2.01	0.00161198537576968\\
2.02	0.00161217615776861\\
2.03	0.00161236701953803\\
2.04	0.00161255796111631\\
2.05	0.00161274898254187\\
2.06	0.00161294008385313\\
2.07	0.00161313126508855\\
2.08	0.00161332252628661\\
2.09	0.00161351386748582\\
2.1	0.0016137052887247\\
2.11	0.00161389679004181\\
2.12	0.00161408837147573\\
2.13	0.00161428003306508\\
2.14	0.00161447177484848\\
2.15	0.0016146635968646\\
2.16	0.00161485549915212\\
2.17	0.00161504748174975\\
2.18	0.00161523954469623\\
2.19	0.00161543168803032\\
2.2	0.00161562391179081\\
2.21	0.0016158162160165\\
2.22	0.00161600860074625\\
2.23	0.0016162010660189\\
2.24	0.00161639361187336\\
2.25	0.00161658623834854\\
2.26	0.00161677894548338\\
2.27	0.00161697173331685\\
2.28	0.00161716460188794\\
2.29	0.00161735755123567\\
2.3	0.0016175505813991\\
2.31	0.00161774369241728\\
2.32	0.00161793688432932\\
2.33	0.00161813015717434\\
2.34	0.00161832351099149\\
2.35	0.00161851694581994\\
2.36	0.0016187104616989\\
2.37	0.0016189040586676\\
2.38	0.00161909773676527\\
2.39	0.00161929149603122\\
2.4	0.00161948533650473\\
2.41	0.00161967925822515\\
2.42	0.00161987326123182\\
2.43	0.00162006734556414\\
2.44	0.0016202615112615\\
2.45	0.00162045575836336\\
2.46	0.00162065008690917\\
2.47	0.00162084449693842\\
2.48	0.00162103898849061\\
2.49	0.0016212335616053\\
2.5	0.00162142821632206\\
2.51	0.00162162295268047\\
2.52	0.00162181777072015\\
2.53	0.00162201267048075\\
2.54	0.00162220765200194\\
2.55	0.00162240271532342\\
2.56	0.00162259786048492\\
2.57	0.00162279308752618\\
2.58	0.00162298839648698\\
2.59	0.00162318378740713\\
2.6	0.00162337926032646\\
2.61	0.00162357481528483\\
2.62	0.00162377045232211\\
2.63	0.00162396617147821\\
2.64	0.00162416197279308\\
2.65	0.00162435785630668\\
2.66	0.00162455382205899\\
2.67	0.00162474987009003\\
2.68	0.00162494600043984\\
2.69	0.00162514221314849\\
2.7	0.00162533850825607\\
2.71	0.0016255348858027\\
2.72	0.00162573134582853\\
2.73	0.00162592788837373\\
2.74	0.00162612451347851\\
2.75	0.00162632122118309\\
2.76	0.00162651801152771\\
2.77	0.00162671488455266\\
2.78	0.00162691184029824\\
2.79	0.00162710887880479\\
2.8	0.00162730600011266\\
2.81	0.00162750320426224\\
2.82	0.00162770049129394\\
2.83	0.00162789786124819\\
2.84	0.00162809531416547\\
2.85	0.00162829285008627\\
2.86	0.00162849046905109\\
2.87	0.00162868817110049\\
2.88	0.00162888595627504\\
2.89	0.00162908382461534\\
2.9	0.00162928177616201\\
2.91	0.0016294798109557\\
2.92	0.0016296779290371\\
2.93	0.00162987613044691\\
2.94	0.00163007441522586\\
2.95	0.00163027278341471\\
2.96	0.00163047123505425\\
2.97	0.00163066977018529\\
2.98	0.00163086838884868\\
2.99	0.00163106709108529\\
3	0.001631265876936\\
3.01	0.00163146474644173\\
3.02	0.00163166369964345\\
3.03	0.00163186273658213\\
3.04	0.00163206185729876\\
3.05	0.00163226106183438\\
3.06	0.00163246035023006\\
3.07	0.00163265972252687\\
3.08	0.00163285917876592\\
3.09	0.00163305871898836\\
3.1	0.00163325834323536\\
3.11	0.00163345805154811\\
3.12	0.00163365784396784\\
3.13	0.00163385772053579\\
3.14	0.00163405768129323\\
3.15	0.00163425772628149\\
3.16	0.00163445785554188\\
3.17	0.00163465806911577\\
3.18	0.00163485836704455\\
3.19	0.00163505874936963\\
3.2	0.00163525921613245\\
3.21	0.00163545976737449\\
3.22	0.00163566040313724\\
3.23	0.00163586112346224\\
3.24	0.00163606192839102\\
3.25	0.00163626281796518\\
3.26	0.00163646379222632\\
3.27	0.00163666485121609\\
3.28	0.00163686599497615\\
3.29	0.00163706722354819\\
3.3	0.00163726853697392\\
3.31	0.00163746993529511\\
3.32	0.00163767141855353\\
3.33	0.00163787298679097\\
3.34	0.00163807464004929\\
3.35	0.00163827637837033\\
3.36	0.00163847820179599\\
3.37	0.00163868011036818\\
3.38	0.00163888210412886\\
3.39	0.00163908418311999\\
3.4	0.00163928634738358\\
3.41	0.00163948859696165\\
3.42	0.00163969093189626\\
3.43	0.00163989335222951\\
3.44	0.0016400958580035\\
3.45	0.00164029844926039\\
3.46	0.00164050112604234\\
3.47	0.00164070388839156\\
3.48	0.00164090673635026\\
3.49	0.00164110966996072\\
3.5	0.00164131268926521\\
3.51	0.00164151579430605\\
3.52	0.00164171898512559\\
3.53	0.0016419222617662\\
3.54	0.00164212562427026\\
3.55	0.00164232907268023\\
3.56	0.00164253260703855\\
3.57	0.0016427362273877\\
3.58	0.00164293993377022\\
3.59	0.00164314372622863\\
3.6	0.00164334760480551\\
3.61	0.00164355156954346\\
3.62	0.00164375562048512\\
3.63	0.00164395975767314\\
3.64	0.00164416398115021\\
3.65	0.00164436829095906\\
3.66	0.00164457268714242\\
3.67	0.00164477716974307\\
3.68	0.00164498173880381\\
3.69	0.00164518639436748\\
3.7	0.00164539113647695\\
3.71	0.0016455959651751\\
3.72	0.00164580088050485\\
3.73	0.00164600588250916\\
3.74	0.001646210971231\\
3.75	0.00164641614671338\\
3.76	0.00164662140899935\\
3.77	0.00164682675813196\\
3.78	0.00164703219415432\\
3.79	0.00164723771710956\\
3.8	0.00164744332704081\\
3.81	0.00164764902399128\\
3.82	0.00164785480800417\\
3.83	0.00164806067912274\\
3.84	0.00164826663739025\\
3.85	0.00164847268285\\
3.86	0.00164867881554534\\
3.87	0.00164888503551962\\
3.88	0.00164909134281623\\
3.89	0.0016492977374786\\
3.9	0.00164950421955018\\
3.91	0.00164971078907444\\
3.92	0.00164991744609491\\
3.93	0.00165012419065512\\
3.94	0.00165033102279864\\
3.95	0.00165053794256906\\
3.96	0.00165074495001004\\
3.97	0.00165095204516522\\
3.98	0.00165115922807828\\
3.99	0.00165136649879297\\
4	0.00165157385735303\\
4.01	0.00165178130380222\\
4.02	0.00165198883818438\\
4.03	0.00165219646054333\\
4.04	0.00165240417092295\\
4.05	0.00165261196936713\\
4.06	0.00165281985591983\\
4.07	0.00165302783062498\\
4.08	0.00165323589352659\\
4.09	0.00165344404466868\\
4.1	0.00165365228409529\\
4.11	0.00165386061185053\\
4.12	0.00165406902797849\\
4.13	0.00165427753252332\\
4.14	0.0016544861255292\\
4.15	0.00165469480704034\\
4.16	0.00165490357710096\\
4.17	0.00165511243575533\\
4.18	0.00165532138304775\\
4.19	0.00165553041902256\\
4.2	0.0016557395437241\\
4.21	0.00165594875719676\\
4.22	0.00165615805948497\\
4.23	0.00165636745063318\\
4.24	0.00165657693068586\\
4.25	0.00165678649968753\\
4.26	0.00165699615768274\\
4.27	0.00165720590471606\\
4.28	0.00165741574083209\\
4.29	0.00165762566607546\\
4.3	0.00165783568049085\\
4.31	0.00165804578412296\\
4.32	0.00165825597701651\\
4.33	0.00165846625921626\\
4.34	0.00165867663076701\\
4.35	0.00165888709171357\\
4.36	0.00165909764210081\\
4.37	0.0016593082819736\\
4.38	0.00165951901137687\\
4.39	0.00165972983035555\\
4.4	0.00165994073895464\\
4.41	0.00166015173721914\\
4.42	0.00166036282519409\\
4.43	0.00166057400292456\\
4.44	0.00166078527045567\\
4.45	0.00166099662783254\\
4.46	0.00166120807510035\\
4.47	0.00166141961230429\\
4.48	0.0016616312394896\\
4.49	0.00166184295670153\\
4.5	0.00166205476398539\\
4.51	0.0016622666613865\\
4.52	0.00166247864895022\\
4.53	0.00166269072672194\\
4.54	0.00166290289474708\\
4.55	0.00166311515307108\\
4.56	0.00166332750173945\\
4.57	0.0016635399407977\\
4.58	0.00166375247029137\\
4.59	0.00166396509026605\\
4.6	0.00166417780076734\\
4.61	0.00166439060184091\\
4.62	0.00166460349353241\\
4.63	0.00166481647588757\\
4.64	0.00166502954895213\\
4.65	0.00166524271277185\\
4.66	0.00166545596739256\\
4.67	0.00166566931286007\\
4.68	0.00166588274922028\\
4.69	0.00166609627651907\\
4.7	0.00166630989480239\\
4.71	0.0016665236041162\\
4.72	0.0016667374045065\\
4.73	0.00166695129601933\\
4.74	0.00166716527870076\\
4.75	0.00166737935259688\\
4.76	0.00166759351775382\\
4.77	0.00166780777421775\\
4.78	0.00166802212203485\\
4.79	0.00166823656125137\\
4.8	0.00166845109191356\\
4.81	0.00166866571406772\\
4.82	0.00166888042776017\\
4.83	0.00166909523303728\\
4.84	0.00166931012994544\\
4.85	0.00166952511853106\\
4.86	0.00166974019884063\\
4.87	0.00166995537092061\\
4.88	0.00167017063481755\\
4.89	0.00167038599057799\\
4.9	0.00167060143824853\\
4.91	0.00167081697787579\\
4.92	0.00167103260950643\\
4.93	0.00167124833318715\\
4.94	0.00167146414896466\\
4.95	0.00167168005688573\\
4.96	0.00167189605699714\\
4.97	0.00167211214934571\\
4.98	0.0016723283339783\\
4.99	0.00167254461094182\\
5	0.00167276098028317\\
5.01	0.00167297744204932\\
5.02	0.00167319399628726\\
5.03	0.00167341064304401\\
5.04	0.00167362738236663\\
5.05	0.00167384421430222\\
5.06	0.00167406113889789\\
5.07	0.00167427815620082\\
5.08	0.00167449526625819\\
5.09	0.00167471246911723\\
5.1	0.00167492976482519\\
5.11	0.00167514715342939\\
5.12	0.00167536463497714\\
5.13	0.00167558220951581\\
5.14	0.00167579987709279\\
5.15	0.00167601763775552\\
5.16	0.00167623549155146\\
5.17	0.00167645343852811\\
5.18	0.001676671478733\\
5.19	0.0016768896122137\\
5.2	0.00167710783901782\\
5.21	0.00167732615919298\\
5.22	0.00167754457278686\\
5.23	0.00167776307984716\\
5.24	0.00167798168042163\\
5.25	0.00167820037455803\\
5.26	0.00167841916230417\\
5.27	0.0016786380437079\\
5.28	0.00167885701881709\\
5.29	0.00167907608767966\\
5.3	0.00167929525034354\\
5.31	0.00167951450685672\\
5.32	0.00167973385726722\\
5.33	0.00167995330162308\\
5.34	0.00168017283997239\\
5.35	0.00168039247236328\\
5.36	0.00168061219884388\\
5.37	0.0016808320194624\\
5.38	0.00168105193426706\\
5.39	0.00168127194330612\\
5.4	0.00168149204662787\\
5.41	0.00168171224428064\\
5.42	0.0016819325363128\\
5.43	0.00168215292277274\\
5.44	0.0016823734037089\\
5.45	0.00168259397916975\\
5.46	0.00168281464920379\\
5.47	0.00168303541385957\\
5.48	0.00168325627318566\\
5.49	0.00168347722723067\\
5.5	0.00168369827604326\\
5.51	0.00168391941967209\\
5.52	0.00168414065816588\\
5.53	0.0016843619915734\\
5.54	0.00168458341994343\\
5.55	0.00168480494332479\\
5.56	0.00168502656176634\\
5.57	0.00168524827531698\\
5.58	0.00168547008402564\\
5.59	0.00168569198794129\\
5.6	0.00168591398711292\\
5.61	0.00168613608158959\\
5.62	0.00168635827142035\\
5.63	0.00168658055665433\\
5.64	0.00168680293734066\\
5.65	0.00168702541352854\\
5.66	0.00168724798526718\\
5.67	0.00168747065260583\\
5.68	0.00168769341559378\\
5.69	0.00168791627428036\\
5.7	0.00168813922871493\\
5.71	0.00168836227894691\\
5.72	0.0016885854250257\\
5.73	0.0016888086670008\\
5.74	0.00168903200492171\\
5.75	0.00168925543883797\\
5.76	0.00168947896879916\\
5.77	0.0016897025948549\\
5.78	0.00168992631705486\\
5.79	0.0016901501354487\\
5.8	0.00169037405008617\\
5.81	0.00169059806101703\\
5.82	0.00169082216829108\\
5.83	0.00169104637195816\\
5.84	0.00169127067206814\\
5.85	0.00169149506867092\\
5.86	0.00169171956181647\\
5.87	0.00169194415155476\\
5.88	0.00169216883793582\\
5.89	0.0016923936210097\\
5.9	0.00169261850082651\\
5.91	0.00169284347743636\\
5.92	0.00169306855088944\\
5.93	0.00169329372123596\\
5.94	0.00169351898852615\\
5.95	0.0016937443528103\\
5.96	0.00169396981413873\\
5.97	0.00169419537256178\\
5.98	0.00169442102812987\\
5.99	0.00169464678089341\\
6	0.00169487263090289\\
6.01	0.00169509857820879\\
6.02	0.00169532462286168\\
6.03	0.00169555076491212\\
6.04	0.00169577700441075\\
6.05	0.0016960033414082\\
6.06	0.00169622977595519\\
6.07	0.00169645630810245\\
6.08	0.00169668293790073\\
6.09	0.00169690966540086\\
6.1	0.00169713649065367\\
6.11	0.00169736341371005\\
6.12	0.00169759043462092\\
6.13	0.00169781755343725\\
6.14	0.00169804477021003\\
6.15	0.0016982720849903\\
6.16	0.00169849949782913\\
6.17	0.00169872700877764\\
6.18	0.00169895461788697\\
6.19	0.00169918232520831\\
6.2	0.0016994101307929\\
6.21	0.00169963803469199\\
6.22	0.0016998660369569\\
6.23	0.00170009413763896\\
6.24	0.00170032233678955\\
6.25	0.0017005506344601\\
6.26	0.00170077903070207\\
6.27	0.00170100752556696\\
6.28	0.00170123611910629\\
6.29	0.00170146481137165\\
6.3	0.00170169360241464\\
6.31	0.00170192249228693\\
6.32	0.00170215148104019\\
6.33	0.00170238056872617\\
6.34	0.00170260975539663\\
6.35	0.00170283904110339\\
6.36	0.00170306842589828\\
6.37	0.00170329790983319\\
6.38	0.00170352749296006\\
6.39	0.00170375717533084\\
6.4	0.00170398695699754\\
6.41	0.0017042168380122\\
6.42	0.00170444681842691\\
6.43	0.00170467689829379\\
6.44	0.001704907077665\\
6.45	0.00170513735659274\\
6.46	0.00170536773512925\\
6.47	0.00170559821332682\\
6.48	0.00170582879123776\\
6.49	0.00170605946891443\\
6.5	0.00170629024640924\\
6.51	0.00170652112377462\\
6.52	0.00170675210106305\\
6.53	0.00170698317832705\\
6.54	0.00170721435561918\\
6.55	0.00170744563299204\\
6.56	0.00170767701049826\\
6.57	0.00170790848819053\\
6.58	0.00170814006612157\\
6.59	0.00170837174434414\\
6.6	0.00170860352291102\\
6.61	0.00170883540187508\\
6.62	0.00170906738128918\\
6.63	0.00170929946120624\\
6.64	0.00170953164167924\\
6.65	0.00170976392276116\\
6.66	0.00170999630450504\\
6.67	0.00171022878696398\\
6.68	0.0017104613701911\\
6.69	0.00171069405423955\\
6.7	0.00171092683916255\\
6.71	0.00171115972501333\\
6.72	0.00171139271184519\\
6.73	0.00171162579971144\\
6.74	0.00171185898866547\\
6.75	0.00171209227876067\\
6.76	0.0017123256700505\\
6.77	0.00171255916258845\\
6.78	0.00171279275642804\\
6.79	0.00171302645162285\\
6.8	0.0017132602482265\\
6.81	0.00171349414629263\\
6.82	0.00171372814587496\\
6.83	0.0017139622470272\\
6.84	0.00171419644980315\\
6.85	0.00171443075425663\\
6.86	0.00171466516044149\\
6.87	0.00171489966841164\\
6.88	0.00171513427822102\\
6.89	0.00171536898992363\\
6.9	0.00171560380357349\\
6.91	0.00171583871922466\\
6.92	0.00171607373693127\\
6.93	0.00171630885674746\\
6.94	0.00171654407872744\\
6.95	0.00171677940292544\\
6.96	0.00171701482939574\\
6.97	0.00171725035819265\\
6.98	0.00171748598937056\\
6.99	0.00171772172298384\\
7	0.00171795755908698\\
7.01	0.00171819349773444\\
7.02	0.00171842953898076\\
7.03	0.00171866568288051\\
7.04	0.00171890192948832\\
7.05	0.00171913827885884\\
7.06	0.00171937473104678\\
7.07	0.00171961128610688\\
7.08	0.00171984794409392\\
7.09	0.00172008470506275\\
7.1	0.00172032156906822\\
7.11	0.00172055853616526\\
7.12	0.00172079560640883\\
7.13	0.00172103277985392\\
7.14	0.00172127005655558\\
7.15	0.0017215074365689\\
7.16	0.00172174491994901\\
7.17	0.00172198250675108\\
7.18	0.00172222019703032\\
7.19	0.001722457990842\\
7.2	0.00172269588824141\\
7.21	0.00172293388928391\\
7.22	0.00172317199402489\\
7.23	0.00172341020251977\\
7.24	0.00172364851482403\\
7.25	0.0017238869309932\\
7.26	0.00172412545108283\\
7.27	0.00172436407514853\\
7.28	0.00172460280324596\\
7.29	0.0017248416354308\\
7.3	0.0017250805717588\\
7.31	0.00172531961228573\\
7.32	0.00172555875706742\\
7.33	0.00172579800615975\\
7.34	0.00172603735961862\\
7.35	0.00172627681749999\\
7.36	0.00172651637985986\\
7.37	0.00172675604675428\\
7.38	0.00172699581823935\\
7.39	0.00172723569437118\\
7.4	0.00172747567520597\\
7.41	0.00172771576079993\\
7.42	0.00172795595120933\\
7.43	0.00172819624649048\\
7.44	0.00172843664669975\\
7.45	0.00172867715189352\\
7.46	0.00172891776212825\\
7.47	0.00172915847746043\\
7.48	0.00172939929794658\\
7.49	0.0017296402236433\\
7.5	0.00172988125460719\\
7.51	0.00173012239089494\\
7.52	0.00173036363256326\\
7.53	0.00173060497966891\\
7.54	0.00173084643226869\\
7.55	0.00173108799041945\\
7.56	0.00173132965417809\\
7.57	0.00173157142360154\\
7.58	0.0017318132987468\\
7.59	0.00173205527967088\\
7.6	0.00173229736643087\\
7.61	0.0017325395590839\\
7.62	0.00173278185768711\\
7.63	0.00173302426229774\\
7.64	0.00173326677297303\\
7.65	0.00173350938977029\\
7.66	0.00173375211274688\\
7.67	0.00173399494196018\\
7.68	0.00173423787746763\\
7.69	0.00173448091932674\\
7.7	0.00173472406759501\\
7.71	0.00173496732233005\\
7.72	0.00173521068358947\\
7.73	0.00173545415143094\\
7.74	0.00173569772591219\\
7.75	0.00173594140709097\\
7.76	0.0017361851950251\\
7.77	0.00173642908977244\\
7.78	0.00173667309139088\\
7.79	0.00173691719993837\\
7.8	0.00173716141547291\\
7.81	0.00173740573805255\\
7.82	0.00173765016773538\\
7.83	0.00173789470457953\\
7.84	0.00173813934864318\\
7.85	0.00173838409998457\\
7.86	0.00173862895866198\\
7.87	0.00173887392473371\\
7.88	0.00173911899825815\\
7.89	0.00173936417929371\\
7.9	0.00173960946789887\\
7.91	0.00173985486413213\\
7.92	0.00174010036805204\\
7.93	0.00174034597971723\\
7.94	0.00174059169918634\\
7.95	0.00174083752651807\\
7.96	0.00174108346177118\\
7.97	0.00174132950500446\\
7.98	0.00174157565627676\\
7.99	0.00174182191564696\\
8	0.001742068283174\\
8.01	0.00174231475891689\\
8.02	0.00174256134293464\\
8.03	0.00174280803528636\\
8.04	0.00174305483603115\\
8.05	0.00174330174522822\\
8.06	0.00174354876293677\\
8.07	0.00174379588921609\\
8.08	0.00174404312412551\\
8.09	0.0017442904677244\\
8.1	0.00174453792007217\\
8.11	0.0017447854812283\\
8.12	0.0017450331512523\\
8.13	0.00174528093020374\\
8.14	0.00174552881814224\\
8.15	0.00174577681512746\\
8.16	0.00174602492121911\\
8.17	0.00174627313647695\\
8.18	0.00174652146096079\\
8.19	0.0017467698947305\\
8.2	0.00174701843784598\\
8.21	0.00174726709036718\\
8.22	0.00174751585235412\\
8.23	0.00174776472386684\\
8.24	0.00174801370496545\\
8.25	0.00174826279571012\\
8.26	0.00174851199616103\\
8.27	0.00174876130637844\\
8.28	0.00174901072642266\\
8.29	0.00174926025635403\\
8.3	0.00174950989623296\\
8.31	0.0017497596461199\\
8.32	0.00175000950607535\\
8.33	0.00175025947615985\\
8.34	0.00175050955643401\\
8.35	0.00175075974695849\\
8.36	0.00175101004779398\\
8.37	0.00175126045900123\\
8.38	0.00175151098064104\\
8.39	0.00175176161277426\\
8.4	0.00175201235546179\\
8.41	0.00175226320876459\\
8.42	0.00175251417274365\\
8.43	0.00175276524746003\\
8.44	0.00175301643297483\\
8.45	0.00175326772934921\\
8.46	0.00175351913664436\\
8.47	0.00175377065492155\\
8.48	0.00175402228424208\\
8.49	0.0017542740246673\\
8.5	0.00175452587625862\\
8.51	0.0017547778390775\\
8.52	0.00175502991318546\\
8.53	0.00175528209864404\\
8.54	0.00175553439551487\\
8.55	0.00175578680385959\\
8.56	0.00175603932373994\\
8.57	0.00175629195521767\\
8.58	0.0017565446983546\\
8.59	0.00175679755321261\\
8.6	0.00175705051985359\\
8.61	0.00175730359833955\\
8.62	0.00175755678873248\\
8.63	0.00175781009109448\\
8.64	0.00175806350548766\\
8.65	0.00175831703197421\\
8.66	0.00175857067061636\\
8.67	0.00175882442147638\\
8.68	0.00175907828461663\\
8.69	0.00175933226009947\\
8.7	0.00175958634798736\\
8.71	0.00175984054834279\\
8.72	0.00176009486122829\\
8.73	0.00176034928670647\\
8.74	0.00176060382483998\\
8.75	0.00176085847569152\\
8.76	0.00176111323932383\\
8.77	0.00176136811579974\\
8.78	0.0017616231051821\\
8.79	0.00176187820753382\\
8.8	0.00176213342291787\\
8.81	0.00176238875139727\\
8.82	0.00176264419303509\\
8.83	0.00176289974789445\\
8.84	0.00176315541603855\\
8.85	0.0017634111975306\\
8.86	0.00176366709243389\\
8.87	0.00176392310081176\\
8.88	0.00176417922272761\\
8.89	0.00176443545824488\\
8.9	0.00176469180742707\\
8.91	0.00176494827033773\\
8.92	0.00176520484704046\\
8.93	0.00176546153759895\\
8.94	0.00176571834207688\\
8.95	0.00176597526053804\\
8.96	0.00176623229304625\\
8.97	0.00176648943966538\\
8.98	0.00176674670045936\\
8.99	0.0017670040754922\\
9	0.00176726156482791\\
9.01	0.0017675191685306\\
9.02	0.00176777688666442\\
9.03	0.00176803471929357\\
9.04	0.0017682926664823\\
9.05	0.00176855072829495\\
9.06	0.00176880890479585\\
9.07	0.00176906719604945\\
9.08	0.00176932560212022\\
9.09	0.0017695841230727\\
9.1	0.00176984275897146\\
9.11	0.00177010150988116\\
9.12	0.00177036037586649\\
9.13	0.00177061935699221\\
9.14	0.00177087845332312\\
9.15	0.00177113766492409\\
9.16	0.00177139699186005\\
9.17	0.00177165643419595\\
9.18	0.00177191599199685\\
9.19	0.00177217566532782\\
9.2	0.00177243545425401\\
9.21	0.00177269535884061\\
9.22	0.00177295537915289\\
9.23	0.00177321551525615\\
9.24	0.00177347576721576\\
9.25	0.00177373613509715\\
9.26	0.00177399661896578\\
9.27	0.00177425721888721\\
9.28	0.00177451793492702\\
9.29	0.00177477876715086\\
9.3	0.00177503971562444\\
9.31	0.00177530078041352\\
9.32	0.00177556196158392\\
9.33	0.00177582325920151\\
9.34	0.00177608467333223\\
9.35	0.00177634620404207\\
9.36	0.00177660785139708\\
9.37	0.00177686961546335\\
9.38	0.00177713149630706\\
9.39	0.00177739349399442\\
9.4	0.0017776556085917\\
9.41	0.00177791784016524\\
9.42	0.00177818018878143\\
9.43	0.00177844265450672\\
9.44	0.00177870523740762\\
9.45	0.00177896793755069\\
9.46	0.00177923075500255\\
9.47	0.00177949368982987\\
9.48	0.00177975674209941\\
9.49	0.00178001991187796\\
9.5	0.00178028319923236\\
9.51	0.00178054660422954\\
9.52	0.00178081012693646\\
9.53	0.00178107376742015\\
9.54	0.0017813375257477\\
9.55	0.00178160140198626\\
9.56	0.00178186539620304\\
9.57	0.00178212950846528\\
9.58	0.00178239373884033\\
9.59	0.00178265808739556\\
9.6	0.00178292255419839\\
9.61	0.00178318713931636\\
9.62	0.001783451842817\\
9.63	0.00178371666476793\\
9.64	0.00178398160523683\\
9.65	0.00178424666429144\\
9.66	0.00178451184199955\\
9.67	0.00178477713842902\\
9.68	0.00178504255364777\\
9.69	0.00178530808772377\\
9.7	0.00178557374072504\\
9.71	0.0017858395127197\\
9.72	0.00178610540377588\\
9.73	0.00178637141396181\\
9.74	0.00178663754334576\\
9.75	0.00178690379199606\\
9.76	0.00178717015998111\\
9.77	0.00178743664736936\\
9.78	0.00178770325422934\\
9.79	0.00178796998062961\\
9.8	0.00178823682663882\\
9.81	0.00178850379232565\\
9.82	0.00178877087775887\\
9.83	0.0017890380830073\\
9.84	0.00178930540813982\\
9.85	0.00178957285322536\\
9.86	0.00178984041833294\\
9.87	0.00179010810353161\\
9.88	0.00179037590889049\\
9.89	0.00179064383447879\\
9.9	0.00179091188036573\\
9.91	0.00179118004662063\\
9.92	0.00179144833331287\\
9.93	0.00179171674051187\\
9.94	0.00179198526828713\\
9.95	0.0017922539167082\\
9.96	0.00179252268584471\\
9.97	0.00179279157576632\\
9.98	0.00179306058654279\\
9.99	0.00179332971824391\\
10	0.00179359897093957\\
10.01	0.00179386834469968\\
10.02	0.00179413783959423\\
10.03	0.00179440745569329\\
10.04	0.00179467719306696\\
10.05	0.00179494705178543\\
10.06	0.00179521703191893\\
10.07	0.00179548713353778\\
10.08	0.00179575735671235\\
10.09	0.00179602770151306\\
10.1	0.00179629816801041\\
10.11	0.00179656875627496\\
10.12	0.00179683946637732\\
10.13	0.00179711029838819\\
10.14	0.00179738125237831\\
10.15	0.00179765232841849\\
10.16	0.00179792352657961\\
10.17	0.00179819484693262\\
10.18	0.0017984662895485\\
10.19	0.00179873785449833\\
10.2	0.00179900954185325\\
10.21	0.00179928135168445\\
10.22	0.00179955328406319\\
10.23	0.00179982533906078\\
10.24	0.00180009751674862\\
10.25	0.00180036981719817\\
10.26	0.00180064224048095\\
10.27	0.00180091478666852\\
10.28	0.00180118745583255\\
10.29	0.00180146024804474\\
10.3	0.00180173316337687\\
10.31	0.00180200620190079\\
10.32	0.0018022793636884\\
10.33	0.00180255264881168\\
10.34	0.00180282605734266\\
10.35	0.00180309958935344\\
10.36	0.00180337324491621\\
10.37	0.00180364702410318\\
10.38	0.00180392092698667\\
10.39	0.00180419495363904\\
10.4	0.00180446910413271\\
10.41	0.0018047433785402\\
10.42	0.00180501777693406\\
10.43	0.00180529229938693\\
10.44	0.0018055669459715\\
10.45	0.00180584171676054\\
10.46	0.00180611661182688\\
10.47	0.0018063916312434\\
10.48	0.00180666677508309\\
10.49	0.00180694204341895\\
10.5	0.00180721743632411\\
10.51	0.00180749295387171\\
10.52	0.00180776859613499\\
10.53	0.00180804436318725\\
10.54	0.00180832025510184\\
10.55	0.00180859627195222\\
10.56	0.00180887241381187\\
10.57	0.00180914868075436\\
10.58	0.00180942507285333\\
10.59	0.00180970159018248\\
10.6	0.00180997823281559\\
10.61	0.00181025500082649\\
10.62	0.00181053189428908\\
10.63	0.00181080891327735\\
10.64	0.00181108605786534\\
10.65	0.00181136332812715\\
10.66	0.00181164072413698\\
10.67	0.00181191824596906\\
10.68	0.00181219589369772\\
10.69	0.00181247366739733\\
10.7	0.00181275156714237\\
10.71	0.00181302959300734\\
10.72	0.00181330774506684\\
10.73	0.00181358602339554\\
10.74	0.00181386442806816\\
10.75	0.00181414295915951\\
10.76	0.00181442161674446\\
10.77	0.00181470040089794\\
10.78	0.00181497931169496\\
10.79	0.0018152583492106\\
10.8	0.00181553751352002\\
10.81	0.00181581680469842\\
10.82	0.0018160962228211\\
10.83	0.00181637576796342\\
10.84	0.00181665544020081\\
10.85	0.00181693523960876\\
10.86	0.00181721516626284\\
10.87	0.0018174952202387\\
10.88	0.00181777540161204\\
10.89	0.00181805571045866\\
10.9	0.00181833614685439\\
10.91	0.00181861671087516\\
10.92	0.00181889740259697\\
10.93	0.00181917822209588\\
10.94	0.00181945916944803\\
10.95	0.00181974024472963\\
10.96	0.00182002144801695\\
10.97	0.00182030277938636\\
10.98	0.00182058423891426\\
10.99	0.00182086582667716\\
11	0.00182114754275162\\
11.01	0.00182142938721429\\
11.02	0.00182171136014186\\
11.03	0.00182199346161113\\
11.04	0.00182227569169895\\
11.05	0.00182255805048225\\
11.06	0.00182284053803803\\
11.07	0.00182312315444336\\
11.08	0.00182340589977538\\
11.09	0.00182368877411131\\
11.1	0.00182397177752846\\
11.11	0.00182425491010417\\
11.12	0.00182453817191589\\
11.13	0.00182482156304113\\
11.14	0.00182510508355747\\
11.15	0.00182538873354257\\
11.16	0.00182567251307416\\
11.17	0.00182595642223005\\
11.18	0.00182624046108811\\
11.19	0.0018265246297263\\
11.2	0.00182680892822265\\
11.21	0.00182709335665525\\
11.22	0.00182737791510228\\
11.23	0.001827662603642\\
11.24	0.00182794742235271\\
11.25	0.00182823237131283\\
11.26	0.00182851745060083\\
11.27	0.00182880266029525\\
11.28	0.00182908800047471\\
11.29	0.00182937347121792\\
11.3	0.00182965907260365\\
11.31	0.00182994480471074\\
11.32	0.00183023066761813\\
11.33	0.0018305166614048\\
11.34	0.00183080278614985\\
11.35	0.0018310890419324\\
11.36	0.0018313754288317\\
11.37	0.00183166194692704\\
11.38	0.0018319485962978\\
11.39	0.00183223537702344\\
11.4	0.00183252228918349\\
11.41	0.00183280933285755\\
11.42	0.00183309650812531\\
11.43	0.00183338381506653\\
11.44	0.00183367125376104\\
11.45	0.00183395882428877\\
11.46	0.00183424652672969\\
11.47	0.00183453436116388\\
11.48	0.00183482232767148\\
11.49	0.00183511042633272\\
11.5	0.00183539865722789\\
11.51	0.00183568702043738\\
11.52	0.00183597551604163\\
11.53	0.00183626414412118\\
11.54	0.00183655290475664\\
11.55	0.0018368417980287\\
11.56	0.00183713082401812\\
11.57	0.00183741998280576\\
11.58	0.00183770927447253\\
11.59	0.00183799869909943\\
11.6	0.00183828825676755\\
11.61	0.00183857794755805\\
11.62	0.00183886777155215\\
11.63	0.00183915772883118\\
11.64	0.00183944781947654\\
11.65	0.0018397380435697\\
11.66	0.00184002840119222\\
11.67	0.00184031889242572\\
11.68	0.00184060951735192\\
11.69	0.00184090027605261\\
11.7	0.00184119116860967\\
11.71	0.00184148219510506\\
11.72	0.00184177335562079\\
11.73	0.00184206465023899\\
11.74	0.00184235607904184\\
11.75	0.00184264764211162\\
11.76	0.0018429393395307\\
11.77	0.00184323117138149\\
11.78	0.00184352313774652\\
11.79	0.00184381523870837\\
11.8	0.00184410747434974\\
11.81	0.00184439984475337\\
11.82	0.00184469235000211\\
11.83	0.00184498499017888\\
11.84	0.00184527776536669\\
11.85	0.0018455706756486\\
11.86	0.00184586372110779\\
11.87	0.00184615690182751\\
11.88	0.00184645021789109\\
11.89	0.00184674366938194\\
11.9	0.00184703725638355\\
11.91	0.0018473309789795\\
11.92	0.00184762483725345\\
11.93	0.00184791883128915\\
11.94	0.0018482129611704\\
11.95	0.00184850722698113\\
11.96	0.00184880162880532\\
11.97	0.00184909616672705\\
11.98	0.00184939084083047\\
11.99	0.00184968565119983\\
12	0.00184998059791944\\
12.01	0.00185027568107371\\
12.02	0.00185057090074714\\
12.03	0.0018508662570243\\
12.04	0.00185116174998984\\
12.05	0.00185145737972851\\
12.06	0.00185175314632514\\
12.07	0.00185204904986465\\
12.08	0.00185234509043201\\
12.09	0.00185264126811233\\
12.1	0.00185293758299077\\
12.11	0.00185323403515256\\
12.12	0.00185353062468305\\
12.13	0.00185382735166766\\
12.14	0.0018541242161919\\
12.15	0.00185442121834136\\
12.16	0.00185471835820172\\
12.17	0.00185501563585874\\
12.18	0.00185531305139826\\
12.19	0.00185561060490622\\
12.2	0.00185590829646866\\
12.21	0.00185620612617166\\
12.22	0.00185650409410144\\
12.23	0.00185680220034425\\
12.24	0.00185710044498649\\
12.25	0.00185739882811459\\
12.26	0.00185769734981509\\
12.27	0.00185799601017463\\
12.28	0.00185829480927992\\
12.29	0.00185859374721777\\
12.3	0.00185889282407507\\
12.31	0.00185919203993878\\
12.32	0.00185949139489597\\
12.33	0.00185979088903381\\
12.34	0.00186009052243953\\
12.35	0.00186039029520046\\
12.36	0.00186069020740402\\
12.37	0.0018609902591377\\
12.38	0.00186129045048912\\
12.39	0.00186159078154595\\
12.4	0.00186189125239596\\
12.41	0.00186219186312702\\
12.42	0.00186249261382707\\
12.43	0.00186279350458416\\
12.44	0.00186309453548641\\
12.45	0.00186339570662206\\
12.46	0.00186369701807939\\
12.47	0.00186399846994681\\
12.48	0.00186430006231281\\
12.49	0.00186460179526596\\
12.5	0.00186490366889495\\
12.51	0.00186520568328851\\
12.52	0.00186550783853551\\
12.53	0.00186581013472488\\
12.54	0.00186611257194566\\
12.55	0.00186641515028697\\
12.56	0.00186671786983801\\
12.57	0.0018670207306881\\
12.58	0.00186732373292663\\
12.59	0.00186762687664309\\
12.6	0.00186793016192705\\
12.61	0.00186823358886819\\
12.62	0.00186853715755628\\
12.63	0.00186884086808116\\
12.64	0.00186914472053278\\
12.65	0.00186944871500119\\
12.66	0.00186975285157651\\
12.67	0.00187005713034898\\
12.68	0.0018703615514089\\
12.69	0.00187066611484669\\
12.7	0.00187097082075285\\
12.71	0.00187127566921799\\
12.72	0.00187158066033279\\
12.73	0.00187188579418804\\
12.74	0.00187219107087461\\
12.75	0.00187249649048347\\
12.76	0.0018728020531057\\
12.77	0.00187310775883245\\
12.78	0.00187341360775498\\
12.79	0.00187371959996464\\
12.8	0.00187402573555287\\
12.81	0.00187433201461121\\
12.82	0.00187463843723129\\
12.83	0.00187494500350485\\
12.84	0.0018752517135237\\
12.85	0.00187555856737977\\
12.86	0.00187586556516507\\
12.87	0.00187617270697171\\
12.88	0.00187647999289191\\
12.89	0.00187678742301795\\
12.9	0.00187709499744225\\
12.91	0.00187740271625729\\
12.92	0.00187771057955567\\
12.93	0.00187801858743008\\
12.94	0.0018783267399733\\
12.95	0.0018786350372782\\
12.96	0.00187894347943779\\
12.97	0.00187925206654512\\
12.98	0.00187956079869338\\
12.99	0.00187986967597582\\
13	0.00188017869848583\\
13.01	0.00188048786631688\\
13.02	0.00188079717956252\\
13.03	0.00188110663831642\\
13.04	0.00188141624267234\\
13.05	0.00188172599272415\\
13.06	0.00188203588856579\\
13.07	0.00188234593029133\\
13.08	0.00188265611799493\\
13.09	0.00188296645177084\\
13.1	0.00188327693171342\\
13.11	0.00188358755791712\\
13.12	0.0018838983304765\\
13.13	0.00188420924948621\\
13.14	0.001884520315041\\
13.15	0.00188483152723574\\
13.16	0.00188514288616537\\
13.17	0.00188545439192495\\
13.18	0.00188576604460964\\
13.19	0.00188607784431469\\
13.2	0.00188638979113546\\
13.21	0.00188670188516741\\
13.22	0.0018870141265061\\
13.23	0.00188732651524719\\
13.24	0.00188763905148644\\
13.25	0.00188795173531973\\
13.26	0.001888264566843\\
13.27	0.00188857754615235\\
13.28	0.00188889067334393\\
13.29	0.00188920394851402\\
13.3	0.00188951737175899\\
13.31	0.00188983094317533\\
13.32	0.00189014466285962\\
13.33	0.00189045853090854\\
13.34	0.00189077254741888\\
13.35	0.00189108671248753\\
13.36	0.0018914010262115\\
13.37	0.00189171548868787\\
13.38	0.00189203010001385\\
13.39	0.00189234486028676\\
13.4	0.00189265976960399\\
13.41	0.00189297482806307\\
13.42	0.00189329003576161\\
13.43	0.00189360539279736\\
13.44	0.00189392089926813\\
13.45	0.00189423655527186\\
13.46	0.00189455236090659\\
13.47	0.00189486831627047\\
13.48	0.00189518442146175\\
13.49	0.00189550067657879\\
13.5	0.00189581708172006\\
13.51	0.00189613363698411\\
13.52	0.00189645034246964\\
13.53	0.00189676719827543\\
13.54	0.00189708420450037\\
13.55	0.00189740136124344\\
13.56	0.00189771866860377\\
13.57	0.00189803612668055\\
13.58	0.00189835373557311\\
13.59	0.00189867149538088\\
13.6	0.00189898940620339\\
13.61	0.00189930746814027\\
13.62	0.0018996256812913\\
13.63	0.00189994404575631\\
13.64	0.00190026256163528\\
13.65	0.00190058122902829\\
13.66	0.00190090004803552\\
13.67	0.00190121901875726\\
13.68	0.00190153814129393\\
13.69	0.00190185741574603\\
13.7	0.00190217684221418\\
13.71	0.00190249642079911\\
13.72	0.00190281615160168\\
13.73	0.00190313603472284\\
13.74	0.00190345607026363\\
13.75	0.00190377625832524\\
13.76	0.00190409659900895\\
13.77	0.00190441709241616\\
13.78	0.00190473773864838\\
13.79	0.00190505853780721\\
13.8	0.0019053794899944\\
13.81	0.00190570059531178\\
13.82	0.00190602185386129\\
13.83	0.00190634326574502\\
13.84	0.00190666483106513\\
13.85	0.00190698654992391\\
13.86	0.00190730842242377\\
13.87	0.00190763044866721\\
13.88	0.00190795262875688\\
13.89	0.0019082749627955\\
13.9	0.00190859745088592\\
13.91	0.00190892009313113\\
13.92	0.0019092428896342\\
13.93	0.00190956584049832\\
13.94	0.0019098889458268\\
13.95	0.00191021220572307\\
13.96	0.00191053562029066\\
13.97	0.00191085918963323\\
13.98	0.00191118291385455\\
13.99	0.0019115067930585\\
14	0.00191183082734907\\
14.01	0.00191215501683038\\
14.02	0.00191247936160666\\
14.03	0.00191280386178225\\
14.04	0.00191312851746162\\
14.05	0.00191345332874934\\
14.06	0.0019137782957501\\
14.07	0.00191410341856872\\
14.08	0.00191442869731012\\
14.09	0.00191475413207935\\
14.1	0.00191507972298157\\
14.11	0.00191540547012205\\
14.12	0.0019157313736062\\
14.13	0.00191605743353952\\
14.14	0.00191638365002766\\
14.15	0.00191671002317636\\
14.16	0.00191703655309149\\
14.17	0.00191736323987903\\
14.18	0.0019176900836451\\
14.19	0.00191801708449593\\
14.2	0.00191834424253785\\
14.21	0.00191867155787732\\
14.22	0.00191899903062095\\
14.23	0.00191932666087542\\
14.24	0.00191965444874757\\
14.25	0.00191998239434433\\
14.26	0.00192031049777278\\
14.27	0.00192063875914009\\
14.28	0.00192096717855357\\
14.29	0.00192129575612066\\
14.3	0.0019216244919489\\
14.31	0.00192195338614595\\
14.32	0.00192228243881962\\
14.33	0.00192261165007782\\
14.34	0.00192294102002858\\
14.35	0.00192327054878006\\
14.36	0.00192360023644055\\
14.37	0.00192393008311844\\
14.38	0.00192426008892226\\
14.39	0.00192459025396067\\
14.4	0.00192492057834243\\
14.41	0.00192525106217645\\
14.42	0.00192558170557175\\
14.43	0.00192591250863747\\
14.44	0.00192624347148287\\
14.45	0.00192657459421736\\
14.46	0.00192690587695046\\
14.47	0.0019272373197918\\
14.48	0.00192756892285115\\
14.49	0.00192790068623841\\
14.5	0.0019282326100636\\
14.51	0.00192856469443686\\
14.52	0.00192889693946847\\
14.53	0.00192922934526881\\
14.54	0.00192956191194842\\
14.55	0.00192989463961795\\
14.56	0.00193022752838817\\
14.57	0.00193056057837\\
14.58	0.00193089378967445\\
14.59	0.00193122716241269\\
14.6	0.00193156069669601\\
14.61	0.00193189439263582\\
14.62	0.00193222825034366\\
14.63	0.00193256226993122\\
14.64	0.00193289645151028\\
14.65	0.00193323079519277\\
14.66	0.00193356530109077\\
14.67	0.00193389996931645\\
14.68	0.00193423479998213\\
14.69	0.00193456979320025\\
14.7	0.0019349049490834\\
14.71	0.00193524026774429\\
14.72	0.00193557574929575\\
14.73	0.00193591139385074\\
14.74	0.00193624720152238\\
14.75	0.00193658317242388\\
14.76	0.00193691930666862\\
14.77	0.00193725560437007\\
14.78	0.00193759206564188\\
14.79	0.0019379286905978\\
14.8	0.00193826547935171\\
14.81	0.00193860243201765\\
14.82	0.00193893954870976\\
14.83	0.00193927682954233\\
14.84	0.00193961427462979\\
14.85	0.00193995188408669\\
14.86	0.00194028965802771\\
14.87	0.00194062759656769\\
14.88	0.00194096569982158\\
14.89	0.00194130396790446\\
14.9	0.00194164240093157\\
14.91	0.00194198099901827\\
14.92	0.00194231976228004\\
14.93	0.00194265869083254\\
14.94	0.0019429977847915\\
14.95	0.00194333704427286\\
14.96	0.00194367646939263\\
14.97	0.00194401606026701\\
14.98	0.00194435581701229\\
14.99	0.00194469573974492\\
15	0.0019450358285815\\
15.01	0.00194537608363874\\
15.02	0.00194571650503351\\
15.03	0.00194605709288279\\
15.04	0.00194639784730374\\
15.05	0.00194673876841362\\
15.06	0.00194707985632985\\
15.07	0.00194742111116998\\
15.08	0.0019477625330517\\
15.09	0.00194810412209283\\
15.1	0.00194844587841136\\
15.11	0.00194878780212538\\
15.12	0.00194912989335315\\
15.13	0.00194947215221306\\
15.14	0.00194981457882363\\
15.15	0.00195015717330354\\
15.16	0.0019504999357716\\
15.17	0.00195084286634677\\
15.18	0.00195118596514813\\
15.19	0.00195152923229493\\
15.2	0.00195187266790654\\
15.21	0.00195221627210249\\
15.22	0.00195256004500244\\
15.23	0.00195290398672619\\
15.24	0.00195324809739369\\
15.25	0.00195359237712503\\
15.26	0.00195393682604045\\
15.27	0.00195428144426034\\
15.28	0.00195462623190521\\
15.29	0.00195497118909573\\
15.3	0.00195531631595272\\
15.31	0.00195566161259714\\
15.32	0.00195600707915008\\
15.33	0.00195635271573279\\
15.34	0.00195669852246667\\
15.35	0.00195704449947327\\
15.36	0.00195739064687425\\
15.37	0.00195773696479146\\
15.38	0.00195808345334688\\
15.39	0.00195843011266263\\
15.4	0.00195877694286098\\
15.41	0.00195912394406435\\
15.42	0.00195947111639532\\
15.43	0.0019598184599766\\
15.44	0.00196016597493105\\
15.45	0.0019605136613817\\
15.46	0.00196086151945169\\
15.47	0.00196120954926435\\
15.48	0.00196155775094312\\
15.49	0.00196190612461163\\
15.5	0.00196225467039363\\
15.51	0.00196260338841304\\
15.52	0.00196295227879391\\
15.53	0.00196330134166045\\
15.54	0.00196365057713703\\
15.55	0.00196399998534817\\
15.56	0.00196434956641851\\
15.57	0.0019646993204729\\
15.58	0.00196504924763629\\
15.59	0.0019653993480338\\
15.6	0.00196574962179072\\
15.61	0.00196610006903246\\
15.62	0.00196645068988461\\
15.63	0.0019668014844729\\
15.64	0.00196715245292323\\
15.65	0.00196750359536162\\
15.66	0.00196785491191428\\
15.67	0.00196820640270755\\
15.68	0.00196855806786796\\
15.69	0.00196890990752215\\
15.7	0.00196926192179694\\
15.71	0.00196961411081931\\
15.72	0.00196996647471637\\
15.73	0.00197031901361543\\
15.74	0.00197067172764392\\
15.75	0.00197102461692943\\
15.76	0.00197137768159973\\
15.77	0.00197173092178272\\
15.78	0.00197208433760648\\
15.79	0.00197243792919923\\
15.8	0.00197279169668937\\
15.81	0.00197314564020543\\
15.82	0.00197349975987612\\
15.83	0.0019738540558303\\
15.84	0.00197420852819701\\
15.85	0.00197456317710542\\
15.86	0.00197491800268488\\
15.87	0.00197527300506489\\
15.88	0.00197562818437511\\
15.89	0.00197598354074537\\
15.9	0.00197633907430565\\
15.91	0.00197669478518612\\
15.92	0.00197705067351707\\
15.93	0.00197740673942898\\
15.94	0.00197776298305248\\
15.95	0.00197811940451837\\
15.96	0.00197847600395762\\
15.97	0.00197883278150134\\
15.98	0.00197918973728083\\
15.99	0.00197954687142754\\
16	0.00197990418407308\\
16.01	0.00198026167534923\\
16.02	0.00198061934538794\\
16.03	0.00198097719432133\\
16.04	0.00198133522228167\\
16.05	0.0019816934294014\\
16.06	0.00198205181581312\\
16.07	0.00198241038164963\\
16.08	0.00198276912704385\\
16.09	0.0019831280521289\\
16.1	0.00198348715703806\\
16.11	0.00198384644190476\\
16.12	0.00198420590686262\\
16.13	0.00198456555204543\\
16.14	0.00198492537758713\\
16.15	0.00198528538362183\\
16.16	0.00198564557028384\\
16.17	0.00198600593770759\\
16.18	0.00198636648602774\\
16.19	0.00198672721537906\\
16.2	0.00198708812589653\\
16.21	0.00198744921771529\\
16.22	0.00198781049097064\\
16.23	0.00198817194579808\\
16.24	0.00198853358233324\\
16.25	0.00198889540071197\\
16.26	0.00198925740107025\\
16.27	0.00198961958354427\\
16.28	0.00198998194827035\\
16.29	0.00199034449538503\\
16.3	0.00199070722502499\\
16.31	0.00199107013732709\\
16.32	0.00199143323242839\\
16.33	0.00199179651046609\\
16.34	0.00199215997157758\\
16.35	0.00199252361590044\\
16.36	0.00199288744357239\\
16.37	0.00199325145473137\\
16.38	0.00199361564951545\\
16.39	0.00199398002806292\\
16.4	0.00199434459051222\\
16.41	0.00199470933700198\\
16.42	0.001995074267671\\
16.43	0.00199543938265825\\
16.44	0.00199580468210291\\
16.45	0.00199617016614431\\
16.46	0.00199653583492197\\
16.47	0.00199690168857559\\
16.48	0.00199726772724503\\
16.49	0.00199763395107037\\
16.5	0.00199800036019183\\
16.51	0.00199836695474984\\
16.52	0.001998733734885\\
16.53	0.00199910070073809\\
16.54	0.00199946785245007\\
16.55	0.00199983519016208\\
16.56	0.00200020271401546\\
16.57	0.00200057042415171\\
16.58	0.00200093832071253\\
16.59	0.0020013064038398\\
16.6	0.00200167467367558\\
16.61	0.00200204313036211\\
16.62	0.00200241177404184\\
16.63	0.00200278060485737\\
16.64	0.00200314962295151\\
16.65	0.00200351882846724\\
16.66	0.00200388822154773\\
16.67	0.00200425780233635\\
16.68	0.00200462757097664\\
16.69	0.00200499752761234\\
16.7	0.00200536767238738\\
16.71	0.00200573800544585\\
16.72	0.00200610852693205\\
16.73	0.00200647923699047\\
16.74	0.00200685013576579\\
16.75	0.00200722122340288\\
16.76	0.00200759250004678\\
16.77	0.00200796396584274\\
16.78	0.0020083356209362\\
16.79	0.00200870746547278\\
16.8	0.00200907949959829\\
16.81	0.00200945172345875\\
16.82	0.00200982413720036\\
16.83	0.00201019674096951\\
16.84	0.00201056953491278\\
16.85	0.00201094251917696\\
16.86	0.00201131569390901\\
16.87	0.0020116890592561\\
16.88	0.0020120626153656\\
16.89	0.00201243636238505\\
16.9	0.00201281030046221\\
16.91	0.00201318442974501\\
16.92	0.0020135587503816\\
16.93	0.00201393326252032\\
16.94	0.0020143079663097\\
16.95	0.00201468286189847\\
16.96	0.00201505794943555\\
16.97	0.00201543322907007\\
16.98	0.00201580870095136\\
16.99	0.00201618436522893\\
17	0.0020165602220525\\
17.01	0.00201693627157198\\
17.02	0.00201731251393752\\
17.03	0.00201768894929941\\
17.04	0.00201806557780817\\
17.05	0.00201844239961452\\
17.06	0.0020188194148694\\
17.07	0.00201919662372391\\
17.08	0.00201957402632939\\
17.09	0.00201995162283734\\
17.1	0.00202032941339951\\
17.11	0.00202070739816783\\
17.12	0.00202108557729443\\
17.13	0.00202146395093164\\
17.14	0.00202184251923202\\
17.15	0.00202222128234831\\
17.16	0.00202260024043346\\
17.17	0.00202297939364063\\
17.18	0.00202335874212319\\
17.19	0.0020237382860347\\
17.2	0.00202411802552895\\
17.21	0.00202449796075991\\
17.22	0.00202487809188178\\
17.23	0.00202525841904896\\
17.24	0.00202563894241605\\
17.25	0.00202601966213788\\
17.26	0.00202640057836946\\
17.27	0.00202678169126604\\
17.28	0.00202716300098305\\
17.29	0.00202754450767616\\
17.3	0.00202792621150123\\
17.31	0.00202830811261433\\
17.32	0.00202869021117175\\
17.33	0.00202907250733001\\
17.34	0.00202945500124581\\
17.35	0.00202983769307608\\
17.36	0.00203022058297794\\
17.37	0.00203060367110877\\
17.38	0.00203098695762613\\
17.39	0.0020313704426878\\
17.4	0.00203175412645178\\
17.41	0.00203213800907627\\
17.42	0.00203252209071971\\
17.43	0.00203290637154075\\
17.44	0.00203329085169824\\
17.45	0.00203367553135127\\
17.46	0.00203406041065912\\
17.47	0.00203444548978132\\
17.48	0.00203483076887759\\
17.49	0.00203521624810789\\
17.5	0.00203560192763239\\
17.51	0.00203598780761147\\
17.52	0.00203637388820575\\
17.53	0.00203676016957606\\
17.54	0.00203714665188345\\
17.55	0.00203753333528921\\
17.56	0.00203792021995481\\
17.57	0.00203830730604198\\
17.58	0.00203869459371266\\
17.59	0.00203908208312902\\
17.6	0.00203946977445345\\
17.61	0.00203985766784855\\
17.62	0.00204024576347716\\
17.63	0.00204063406150234\\
17.64	0.00204102256208739\\
17.65	0.00204141126539582\\
17.66	0.00204180017159136\\
17.67	0.00204218928083797\\
17.68	0.00204257859329986\\
17.69	0.00204296810914145\\
17.7	0.00204335782852739\\
17.71	0.00204374775162255\\
17.72	0.00204413787859205\\
17.73	0.00204452820960123\\
17.74	0.00204491874481564\\
17.75	0.00204530948440109\\
17.76	0.00204570042852361\\
17.77	0.00204609157734946\\
17.78	0.00204648293104513\\
17.79	0.00204687448977734\\
17.8	0.00204726625371306\\
17.81	0.00204765822301948\\
17.82	0.00204805039786402\\
17.83	0.00204844277841434\\
17.84	0.00204883536483833\\
17.85	0.00204922815730414\\
17.86	0.00204962115598011\\
17.87	0.00205001436103486\\
17.88	0.00205040777263722\\
17.89	0.00205080139095627\\
17.9	0.00205119521616132\\
17.91	0.00205158924842191\\
17.92	0.00205198348790786\\
17.93	0.00205237793478918\\
17.94	0.00205277258923613\\
17.95	0.00205316745141924\\
17.96	0.00205356252150924\\
17.97	0.00205395779967713\\
17.98	0.00205435328609415\\
17.99	0.00205474898093176\\
18	0.00205514488436169\\
18.01	0.00205554099655589\\
18.02	0.00205593731768657\\
18.03	0.00205633384792616\\
18.04	0.00205673058744738\\
18.05	0.00205712753642314\\
18.06	0.00205752469502664\\
18.07	0.00205792206343131\\
18.08	0.00205831964181081\\
18.09	0.00205871743033907\\
18.1	0.00205911542919027\\
18.11	0.00205951363853881\\
18.12	0.00205991205855937\\
18.13	0.00206031068942686\\
18.14	0.00206070953131644\\
18.15	0.00206110858440352\\
18.16	0.00206150784886379\\
18.17	0.00206190732487315\\
18.18	0.00206230701260777\\
18.19	0.00206270691224407\\
18.2	0.00206310702395873\\
18.21	0.00206350734792866\\
18.22	0.00206390788433106\\
18.23	0.00206430863334336\\
18.24	0.00206470959514324\\
18.25	0.00206511076990865\\
18.26	0.00206551215781779\\
18.27	0.00206591375904912\\
18.28	0.00206631557378135\\
18.29	0.00206671760219345\\
18.3	0.00206711984446466\\
18.31	0.00206752230077445\\
18.32	0.00206792497130258\\
18.33	0.00206832785622905\\
18.34	0.00206873095573412\\
18.35	0.00206913426999833\\
18.36	0.00206953779920246\\
18.37	0.00206994154352756\\
18.38	0.00207034550315495\\
18.39	0.0020707496782662\\
18.4	0.00207115406904314\\
18.41	0.00207155867566788\\
18.42	0.00207196349832278\\
18.43	0.00207236853719048\\
18.44	0.00207277379245388\\
18.45	0.00207317926429613\\
18.46	0.00207358495290066\\
18.47	0.00207399085845118\\
18.48	0.00207439698113165\\
18.49	0.00207480332112629\\
18.5	0.00207520987861962\\
18.51	0.00207561665379639\\
18.52	0.00207602364684167\\
18.53	0.00207643085794074\\
18.54	0.00207683828727921\\
18.55	0.00207724593504292\\
18.56	0.00207765380141801\\
18.57	0.00207806188659086\\
18.58	0.00207847019074816\\
18.59	0.00207887871407685\\
18.6	0.00207928745676415\\
18.61	0.00207969641899756\\
18.62	0.00208010560096485\\
18.63	0.00208051500285407\\
18.64	0.00208092462485355\\
18.65	0.00208133446715188\\
18.66	0.00208174452993796\\
18.67	0.00208215481340093\\
18.68	0.00208256531773025\\
18.69	0.00208297604311562\\
18.7	0.00208338698974705\\
18.71	0.00208379815781482\\
18.72	0.00208420954750949\\
18.73	0.00208462115902189\\
18.74	0.00208503299254317\\
18.75	0.00208544504826473\\
18.76	0.00208585732637825\\
18.77	0.00208626982707573\\
18.78	0.00208668255054942\\
18.79	0.00208709549699188\\
18.8	0.00208750866659593\\
18.81	0.00208792205955471\\
18.82	0.00208833567606161\\
18.83	0.00208874951631035\\
18.84	0.00208916358049491\\
18.85	0.00208957786880955\\
18.86	0.00208999238144886\\
18.87	0.00209040711860769\\
18.88	0.00209082208048118\\
18.89	0.00209123726726478\\
18.9	0.00209165267915421\\
18.91	0.00209206831634551\\
18.92	0.00209248417903499\\
18.93	0.00209290026741927\\
18.94	0.00209331658169525\\
18.95	0.00209373312206015\\
18.96	0.00209414988871146\\
18.97	0.00209456688184697\\
18.98	0.00209498410166479\\
18.99	0.0020954015483633\\
19	0.00209581922214121\\
19.01	0.00209623712319749\\
19.02	0.00209665525173144\\
19.03	0.00209707360794266\\
19.04	0.00209749219203103\\
19.05	0.00209791100419675\\
19.06	0.00209833004464033\\
19.07	0.00209874931356255\\
19.08	0.00209916881116454\\
19.09	0.00209958853764768\\
19.1	0.00210000849321371\\
19.11	0.00210042867806463\\
19.12	0.00210084909240279\\
19.13	0.0021012697364308\\
19.14	0.00210169061035162\\
19.15	0.00210211171436849\\
19.16	0.00210253304868498\\
19.17	0.00210295461350496\\
19.18	0.0021033764090326\\
19.19	0.00210379843547239\\
19.2	0.00210422069302914\\
19.21	0.00210464318190798\\
19.22	0.00210506590231432\\
19.23	0.0021054888544539\\
19.24	0.00210591203853279\\
19.25	0.00210633545475736\\
19.26	0.00210675910333429\\
19.27	0.00210718298447059\\
19.28	0.00210760709837359\\
19.29	0.00210803144525092\\
19.3	0.00210845602531055\\
19.31	0.00210888083876073\\
19.32	0.00210930588581009\\
19.33	0.00210973116666754\\
19.34	0.00211015668154231\\
19.35	0.00211058243064397\\
19.36	0.00211100841418241\\
19.37	0.00211143463236783\\
19.38	0.00211186108541077\\
19.39	0.00211228777352208\\
19.4	0.00211271469691295\\
19.41	0.00211314185579489\\
19.42	0.00211356925037973\\
19.43	0.00211399688087964\\
19.44	0.00211442474750711\\
19.45	0.00211485285047496\\
19.46	0.00211528118999635\\
19.47	0.00211570976628475\\
19.48	0.00211613857955398\\
19.49	0.0021165676300182\\
19.5	0.00211699691789187\\
19.51	0.00211742644338981\\
19.52	0.00211785620672716\\
19.53	0.0021182862081194\\
19.54	0.00211871644778236\\
19.55	0.00211914692593218\\
19.56	0.00211957764278536\\
19.57	0.00212000859855872\\
19.58	0.00212043979346944\\
19.59	0.00212087122773501\\
19.6	0.00212130290157329\\
19.61	0.00212173481520245\\
19.62	0.00212216696884104\\
19.63	0.00212259936270792\\
19.64	0.00212303199702231\\
19.65	0.00212346487200376\\
19.66	0.00212389798787218\\
19.67	0.00212433134484781\\
19.68	0.00212476494315125\\
19.69	0.00212519878300345\\
19.7	0.00212563286462568\\
19.71	0.00212606718823959\\
19.72	0.00212650175406716\\
19.73	0.00212693656233073\\
19.74	0.00212737161325299\\
19.75	0.00212780690705698\\
19.76	0.00212824244396608\\
19.77	0.00212867822420405\\
19.78	0.00212911424799497\\
19.79	0.0021295505155633\\
19.8	0.00212998702713386\\
19.81	0.0021304237829318\\
19.82	0.00213086078318264\\
19.83	0.00213129802811226\\
19.84	0.0021317355179469\\
19.85	0.00213217325291316\\
19.86	0.002132611233238\\
19.87	0.00213304945914872\\
19.88	0.00213348793087302\\
19.89	0.00213392664863893\\
19.9	0.00213436561267485\\
19.91	0.00213480482320957\\
19.92	0.00213524428047222\\
19.93	0.00213568398469228\\
19.94	0.00213612393609963\\
19.95	0.00213656413492451\\
19.96	0.00213700458139751\\
19.97	0.00213744527574961\\
19.98	0.00213788621821216\\
19.99	0.00213832740901685\\
20	0.00213876884839578\\
20.01	0.00213921053658141\\
20.02	0.00213965247380655\\
20.03	0.00214009466030442\\
20.04	0.00214053709630859\\
20.05	0.00214097978205302\\
20.06	0.00214142271777203\\
20.07	0.00214186590370034\\
20.08	0.00214230934007303\\
20.09	0.00214275302712557\\
20.1	0.00214319696509379\\
20.11	0.00214364115421393\\
20.12	0.00214408559472259\\
20.13	0.00214453028685676\\
20.14	0.00214497523085382\\
20.15	0.00214542042695152\\
20.16	0.002145865875388\\
20.17	0.0021463115764018\\
20.18	0.00214675753023182\\
20.19	0.00214720373711738\\
20.2	0.00214765019729816\\
20.21	0.00214809691101424\\
20.22	0.00214854387850609\\
20.23	0.00214899110001458\\
20.24	0.00214943857578095\\
20.25	0.00214988630604687\\
20.26	0.00215033429105435\\
20.27	0.00215078253104585\\
20.28	0.0021512310262642\\
20.29	0.00215167977695261\\
20.3	0.00215212878335472\\
20.31	0.00215257804571455\\
20.32	0.00215302756427651\\
20.33	0.00215347733928544\\
20.34	0.00215392737098655\\
20.35	0.00215437765962548\\
20.36	0.00215482820544824\\
20.37	0.00215527900870127\\
20.38	0.00215573006963141\\
20.39	0.0021561813884859\\
20.4	0.00215663296551239\\
20.41	0.00215708480095893\\
20.42	0.00215753689507398\\
20.43	0.00215798924810643\\
20.44	0.00215844186030555\\
20.45	0.00215889473192104\\
20.46	0.002159347863203\\
20.47	0.00215980125440195\\
20.48	0.00216025490576884\\
20.49	0.00216070881755499\\
20.5	0.00216116299001219\\
20.51	0.00216161742339261\\
20.52	0.00216207211794885\\
20.53	0.00216252707393392\\
20.54	0.00216298229160126\\
20.55	0.00216343777120474\\
20.56	0.00216389351299862\\
20.57	0.00216434951723762\\
20.58	0.00216480578417685\\
20.59	0.00216526231407186\\
20.6	0.00216571910717863\\
20.61	0.00216617616375356\\
20.62	0.00216663348405348\\
20.63	0.00216709106833564\\
20.64	0.00216754891685773\\
20.65	0.00216800702987787\\
20.66	0.00216846540765461\\
20.67	0.00216892405044691\\
20.68	0.00216938295851421\\
20.69	0.00216984213211634\\
20.7	0.00217030157151359\\
20.71	0.00217076127696667\\
20.72	0.00217122124873674\\
20.73	0.0021716814870854\\
20.74	0.00217214199227468\\
20.75	0.00217260276456705\\
20.76	0.00217306380422543\\
20.77	0.00217352511151318\\
20.78	0.00217398668669409\\
20.79	0.00217444853003241\\
20.8	0.00217491064179284\\
20.81	0.0021753730222405\\
20.82	0.00217583567164099\\
20.83	0.00217629859026033\\
20.84	0.00217676177836502\\
20.85	0.00217722523622197\\
20.86	0.00217768896409858\\
20.87	0.00217815296226269\\
20.88	0.00217861723098257\\
20.89	0.00217908177052699\\
20.9	0.00217954658116513\\
20.91	0.00218001166316667\\
20.92	0.00218047701680171\\
20.93	0.00218094264234083\\
20.94	0.00218140854005507\\
20.95	0.00218187471021591\\
20.96	0.00218234115309533\\
20.97	0.00218280786896574\\
20.98	0.00218327485810003\\
20.99	0.00218374212077154\\
21	0.0021842096572541\\
21.01	0.002184677467822\\
21.02	0.00218514555274998\\
21.03	0.00218561391231327\\
21.04	0.00218608254678757\\
21.05	0.00218655145644903\\
21.06	0.00218702064157431\\
21.07	0.00218749010244051\\
21.08	0.00218795983932523\\
21.09	0.00218842985250653\\
21.1	0.00218890014226295\\
21.11	0.00218937070887352\\
21.12	0.00218984155261773\\
21.13	0.00219031267377557\\
21.14	0.0021907840726275\\
21.15	0.00219125574945447\\
21.16	0.00219172770453792\\
21.17	0.00219219993815975\\
21.18	0.00219267245060237\\
21.19	0.00219314524214868\\
21.2	0.00219361831308204\\
21.21	0.00219409166368634\\
21.22	0.00219456529424593\\
21.23	0.00219503920504566\\
21.24	0.00219551339637089\\
21.25	0.00219598786850746\\
21.26	0.00219646262174169\\
21.27	0.00219693765636043\\
21.28	0.002197412972651\\
21.29	0.00219788857090124\\
21.3	0.00219836445139948\\
21.31	0.00219884061443455\\
21.32	0.00219931706029579\\
21.33	0.00219979378927303\\
21.34	0.00220027080165663\\
21.35	0.00220074809773743\\
21.36	0.00220122567780679\\
21.37	0.00220170354215658\\
21.38	0.00220218169107918\\
21.39	0.00220266012486748\\
21.4	0.00220313884381487\\
21.41	0.00220361784821527\\
21.42	0.00220409713836312\\
21.43	0.00220457671455335\\
21.44	0.00220505657708144\\
21.45	0.00220553672624334\\
21.46	0.00220601716233559\\
21.47	0.00220649788565518\\
21.48	0.00220697889649968\\
21.49	0.00220746019516713\\
21.5	0.00220794178195615\\
21.51	0.00220842365716584\\
21.52	0.00220890582109584\\
21.53	0.00220938827404634\\
21.54	0.00220987101631804\\
21.55	0.00221035404821217\\
21.56	0.00221083737003049\\
21.57	0.00221132098207531\\
21.58	0.00221180488464946\\
21.59	0.00221228907805631\\
21.6	0.00221277356259977\\
21.61	0.00221325833858428\\
21.62	0.00221374340631483\\
21.63	0.00221422876609696\\
21.64	0.00221471441823672\\
21.65	0.00221520036304073\\
21.66	0.00221568660081616\\
21.67	0.00221617313187069\\
21.68	0.00221665995651259\\
21.69	0.00221714707505067\\
21.7	0.00221763448779427\\
21.71	0.00221812219505329\\
21.72	0.00221861019713819\\
21.73	0.00221909849435998\\
21.74	0.00221958708703023\\
21.75	0.00222007597546105\\
21.76	0.00222056515996515\\
21.77	0.00222105464085574\\
21.78	0.00222154441844663\\
21.79	0.00222203449305219\\
21.8	0.00222252486498735\\
21.81	0.00222301553456759\\
21.82	0.00222350650210899\\
21.83	0.00222399776792816\\
21.84	0.0022244893323423\\
21.85	0.00222498119566917\\
21.86	0.00222547335822712\\
21.87	0.00222596582033506\\
21.88	0.00222645858231247\\
21.89	0.00222695164447941\\
21.9	0.00222744500715653\\
21.91	0.00222793867066504\\
21.92	0.00222843263532674\\
21.93	0.00222892690146401\\
21.94	0.00222942146939981\\
21.95	0.00222991633945769\\
21.96	0.00223041151196179\\
21.97	0.00223090698723682\\
21.98	0.00223140276560809\\
21.99	0.00223189884740151\\
22	0.00223239523294356\\
22.01	0.00223289192256133\\
22.02	0.00223338891658249\\
22.03	0.00223388621533532\\
22.04	0.00223438381914869\\
22.05	0.00223488172835207\\
22.06	0.00223537994327552\\
22.07	0.00223587846424973\\
22.08	0.00223637729160597\\
22.09	0.00223687642567612\\
22.1	0.00223737586679265\\
22.11	0.00223787561528867\\
22.12	0.00223837567149788\\
22.13	0.00223887603575459\\
22.14	0.00223937670839372\\
22.15	0.00223987768975081\\
22.16	0.00224037898016202\\
22.17	0.00224088057996412\\
22.18	0.00224138248949449\\
22.19	0.00224188470909113\\
22.2	0.00224238723909269\\
22.21	0.0022428900798384\\
22.22	0.00224339323166815\\
22.23	0.00224389669492243\\
22.24	0.00224440046994238\\
22.25	0.00224490455706974\\
22.26	0.0022454089566469\\
22.27	0.00224591366901689\\
22.28	0.00224641869452336\\
22.29	0.00224692403351058\\
22.3	0.00224742968632349\\
22.31	0.00224793565330764\\
22.32	0.00224844193480923\\
22.33	0.00224894853117511\\
22.34	0.00224945544275276\\
22.35	0.00224996266989031\\
22.36	0.00225047021293653\\
22.37	0.00225097807224085\\
22.38	0.00225148624815334\\
22.39	0.00225199474102471\\
22.4	0.00225250355120634\\
22.41	0.00225301267905026\\
22.42	0.00225352212490915\\
22.43	0.00225403188913636\\
22.44	0.00225454197208588\\
22.45	0.00225505237411236\\
22.46	0.00225556309557114\\
22.47	0.00225607413681819\\
22.48	0.00225658549821017\\
22.49	0.0022570971801044\\
22.5	0.00225760918285885\\
22.51	0.0022581215068322\\
22.52	0.00225863415238375\\
22.53	0.00225914711987352\\
22.54	0.00225966040966219\\
22.55	0.0022601740221111\\
22.56	0.00226068795758229\\
22.57	0.00226120221643847\\
22.58	0.00226171679904303\\
22.59	0.00226223170576006\\
22.6	0.00226274693695432\\
22.61	0.00226326249299125\\
22.62	0.002263778374237\\
22.63	0.00226429458105839\\
22.64	0.00226481111382295\\
22.65	0.00226532797289889\\
22.66	0.00226584515865513\\
22.67	0.00226636267146128\\
22.68	0.00226688051168763\\
22.69	0.00226739867970521\\
22.7	0.00226791717588571\\
22.71	0.00226843600060157\\
22.72	0.00226895515422589\\
22.73	0.00226947463713251\\
22.74	0.00226999444969598\\
22.75	0.00227051459229153\\
22.76	0.00227103506529513\\
22.77	0.00227155586908347\\
22.78	0.00227207700403394\\
22.79	0.00227259847052465\\
22.8	0.00227312026893445\\
22.81	0.00227364239964287\\
22.82	0.00227416486303022\\
22.83	0.00227468765947748\\
22.84	0.00227521078936641\\
22.85	0.00227573425307946\\
22.86	0.00227625805099983\\
22.87	0.00227678218351145\\
22.88	0.00227730665099898\\
22.89	0.00227783145384781\\
22.9	0.0022783565924441\\
22.91	0.00227888206717472\\
22.92	0.00227940787842729\\
22.93	0.00227993402659017\\
22.94	0.00228046051205249\\
22.95	0.0022809873352041\\
22.96	0.00228151449643561\\
22.97	0.00228204199613839\\
22.98	0.00228256983470455\\
22.99	0.00228309801252696\\
23	0.00228362652999924\\
23.01	0.0022841553875158\\
23.02	0.00228468458547178\\
23.03	0.00228521412426309\\
23.04	0.00228574400428642\\
23.05	0.0022862742259392\\
23.06	0.00228680478961966\\
23.07	0.00228733569572677\\
23.08	0.0022878669446603\\
23.09	0.00228839853682079\\
23.1	0.00228893047260953\\
23.11	0.00228946275242863\\
23.12	0.00228999537668094\\
23.13	0.00229052834577013\\
23.14	0.00229106166010064\\
23.15	0.00229159532007768\\
23.16	0.00229212932610728\\
23.17	0.00229266367859624\\
23.18	0.00229319837795215\\
23.19	0.0022937334245834\\
23.2	0.0022942688188992\\
23.21	0.00229480456130952\\
23.22	0.00229534065222516\\
23.23	0.0022958770920577\\
23.24	0.00229641388121955\\
23.25	0.00229695102012391\\
23.26	0.00229748850918478\\
23.27	0.00229802634881701\\
23.28	0.00229856453943622\\
23.29	0.00229910308145887\\
23.3	0.00229964197530223\\
23.31	0.0023001812213844\\
23.32	0.00230072082012427\\
23.33	0.0023012607719416\\
23.34	0.00230180107725694\\
23.35	0.00230234173649167\\
23.36	0.00230288275006803\\
23.37	0.00230342411840907\\
23.38	0.00230396584193867\\
23.39	0.00230450792108155\\
23.4	0.00230505035626327\\
23.41	0.00230559314791024\\
23.42	0.00230613629644971\\
23.43	0.00230667980230975\\
23.44	0.00230722366591932\\
23.45	0.0023077678877082\\
23.46	0.00230831246810703\\
23.47	0.00230885740754729\\
23.48	0.00230940270646135\\
23.49	0.00230994836528241\\
23.5	0.00231049438444453\\
23.51	0.00231104076438265\\
23.52	0.00231158750553256\\
23.53	0.00231213460833092\\
23.54	0.00231268207321526\\
23.55	0.002313229900624\\
23.56	0.0023137780909964\\
23.57	0.00231432664477262\\
23.58	0.0023148755623937\\
23.59	0.00231542484430153\\
23.6	0.00231597449093892\\
23.61	0.00231652450274955\\
23.62	0.002317074880178\\
23.63	0.0023176256236697\\
23.64	0.00231817673367102\\
23.65	0.00231872821062921\\
23.66	0.00231928005499239\\
23.67	0.00231983226720962\\
23.68	0.00232038484773084\\
23.69	0.00232093779700688\\
23.7	0.0023214911154895\\
23.71	0.00232204480363137\\
23.72	0.00232259886188604\\
23.73	0.00232315329070801\\
23.74	0.00232370809055267\\
23.75	0.00232426326187635\\
23.76	0.00232481880513628\\
23.77	0.00232537472079061\\
23.78	0.00232593100929845\\
23.79	0.00232648767111979\\
23.8	0.00232704470671559\\
23.81	0.00232760211654771\\
23.82	0.00232815990107898\\
23.83	0.00232871806077312\\
23.84	0.00232927659609485\\
23.85	0.00232983550750977\\
23.86	0.00233039479548446\\
23.87	0.00233095446048646\\
23.88	0.00233151450298422\\
23.89	0.00233207492344716\\
23.9	0.00233263572234568\\
23.91	0.00233319690015109\\
23.92	0.00233375845733569\\
23.93	0.00233432039437274\\
23.94	0.00233488271173646\\
23.95	0.00233544540990203\\
23.96	0.00233600848934561\\
23.97	0.00233657195054434\\
23.98	0.00233713579397631\\
23.99	0.0023377000201206\\
24	0.00233826462945727\\
24.01	0.00233882962246737\\
24.02	0.00233939499963293\\
24.03	0.00233996076143696\\
24.04	0.00234052690836346\\
24.05	0.00234109344089744\\
24.06	0.00234166035952488\\
24.07	0.00234222766473279\\
24.08	0.00234279535700915\\
24.09	0.00234336343684295\\
24.1	0.00234393190472421\\
24.11	0.00234450076114392\\
24.12	0.00234507000659411\\
24.13	0.00234563964156782\\
24.14	0.00234620966655909\\
24.15	0.00234678008206299\\
24.16	0.00234735088857563\\
24.17	0.00234792208659412\\
24.18	0.0023484936766166\\
24.19	0.00234906565914225\\
24.2	0.00234963803467127\\
24.21	0.00235021080370491\\
24.22	0.00235078396674546\\
24.23	0.00235135752429622\\
24.24	0.00235193147686157\\
24.25	0.00235250582494693\\
24.26	0.00235308056905875\\
24.27	0.00235365570970454\\
24.28	0.00235423124739288\\
24.29	0.00235480718263339\\
24.3	0.00235538351593677\\
24.31	0.00235596024781475\\
24.32	0.00235653737878017\\
24.33	0.00235711490934689\\
24.34	0.00235769284002989\\
24.35	0.00235827117134519\\
24.36	0.00235884990380991\\
24.37	0.00235942903794222\\
24.38	0.00236000857426141\\
24.39	0.00236058851328784\\
24.4	0.00236116885554295\\
24.41	0.00236174960154929\\
24.42	0.00236233075183048\\
24.43	0.00236291230691127\\
24.44	0.00236349426731748\\
24.45	0.00236407663357606\\
24.46	0.00236465940621505\\
24.47	0.00236524258576359\\
24.48	0.00236582617275197\\
24.49	0.00236641016771155\\
24.5	0.00236699457117485\\
24.51	0.00236757938367547\\
24.52	0.00236816460574818\\
24.53	0.00236875023792883\\
24.54	0.00236933628075444\\
24.55	0.00236992273476314\\
24.56	0.00237050960049422\\
24.57	0.00237109687848807\\
24.58	0.00237168456928626\\
24.59	0.00237227267343149\\
24.6	0.00237286119146762\\
24.61	0.00237345012393964\\
24.62	0.0023740394713937\\
24.63	0.00237462923437714\\
24.64	0.00237521941343842\\
24.65	0.00237581000912719\\
24.66	0.00237640102199426\\
24.67	0.00237699245259161\\
24.68	0.00237758430147239\\
24.69	0.00237817656919093\\
24.7	0.00237876925630275\\
24.71	0.00237936236336453\\
24.72	0.00237995589093416\\
24.73	0.0023805498395707\\
24.74	0.00238114420983443\\
24.75	0.0023817390022868\\
24.76	0.00238233421749046\\
24.77	0.00238292985600929\\
24.78	0.00238352591840835\\
24.79	0.00238412240525391\\
24.8	0.00238471931711346\\
24.81	0.00238531665455571\\
24.82	0.00238591441815057\\
24.83	0.00238651260846921\\
24.84	0.00238711122608398\\
24.85	0.00238771027156848\\
24.86	0.00238830974549756\\
24.87	0.00238890964844727\\
24.88	0.00238950998099492\\
24.89	0.00239011074371907\\
24.9	0.0023907119371995\\
24.91	0.00239131356201726\\
24.92	0.00239191561875465\\
24.93	0.00239251810799521\\
24.94	0.00239312103032377\\
24.95	0.00239372438632639\\
24.96	0.0023943281765904\\
24.97	0.00239493240170443\\
24.98	0.00239553706225835\\
24.99	0.00239614215884332\\
25	0.00239674769205179\\
25.01	0.00239735366247747\\
25.02	0.00239796007071537\\
25.03	0.00239856691736179\\
25.04	0.00239917420301432\\
25.05	0.00239978192827187\\
25.06	0.00240039009373461\\
25.07	0.00240099870000404\\
25.08	0.00240160774768298\\
25.09	0.00240221723737553\\
25.1	0.00240282716968712\\
25.11	0.00240343754522451\\
25.12	0.00240404836459578\\
25.13	0.00240465962841031\\
25.14	0.00240527133727885\\
25.15	0.00240588349181345\\
25.16	0.00240649609262751\\
25.17	0.00240710914033578\\
25.18	0.00240772263555433\\
25.19	0.00240833657890059\\
25.2	0.00240895097099336\\
25.21	0.00240956581245278\\
25.22	0.00241018110390034\\
25.23	0.00241079684595891\\
25.24	0.00241141303925271\\
25.25	0.00241202968440735\\
25.26	0.0024126467820498\\
25.27	0.0024132643328084\\
25.28	0.00241388233731291\\
25.29	0.00241450079619442\\
25.3	0.00241511971008545\\
25.31	0.00241573907961991\\
25.32	0.00241635890543309\\
25.33	0.00241697918816169\\
25.34	0.00241759992844383\\
25.35	0.002418221126919\\
25.36	0.00241884278422815\\
25.37	0.00241946490101359\\
25.38	0.00242008747791912\\
25.39	0.0024207105155899\\
25.4	0.00242133401467255\\
25.41	0.00242195797581513\\
25.42	0.00242258239966712\\
25.43	0.00242320728687944\\
25.44	0.00242383263810446\\
25.45	0.00242445845399599\\
25.46	0.00242508473520931\\
25.47	0.00242571148240114\\
25.48	0.00242633869622966\\
25.49	0.00242696637735452\\
25.5	0.00242759452643684\\
25.51	0.00242822314413921\\
25.52	0.0024288522311257\\
25.53	0.00242948178806186\\
25.54	0.0024301118156147\\
25.55	0.00243074231445276\\
25.56	0.00243137328524604\\
25.57	0.00243200472866607\\
25.58	0.00243263664538584\\
25.59	0.00243326903607988\\
25.6	0.00243390190142421\\
25.61	0.00243453524209637\\
25.62	0.00243516905877542\\
25.63	0.00243580335214195\\
25.64	0.00243643812287805\\
25.65	0.00243707337166738\\
25.66	0.00243770909919509\\
25.67	0.0024383453061479\\
25.68	0.00243898199321407\\
25.69	0.00243961916108341\\
25.7	0.00244025681044726\\
25.71	0.00244089494199855\\
25.72	0.00244153355643174\\
25.73	0.00244217265444287\\
25.74	0.00244281223672956\\
25.75	0.00244345230399098\\
25.76	0.00244409285692791\\
25.77	0.00244473389624267\\
25.78	0.0024453754226392\\
25.79	0.00244601743682303\\
25.8	0.00244665993950128\\
25.81	0.00244730293138266\\
25.82	0.00244794641317751\\
25.83	0.00244859038559775\\
25.84	0.00244923484935693\\
25.85	0.00244987980517024\\
25.86	0.00245052525375444\\
25.87	0.00245117119582797\\
25.88	0.00245181763211089\\
25.89	0.00245246456332487\\
25.9	0.00245311199019324\\
25.91	0.00245375991344099\\
25.92	0.00245440833379474\\
25.93	0.00245505725198278\\
25.94	0.00245570666873505\\
25.95	0.00245635658478315\\
25.96	0.00245700700086037\\
25.97	0.00245765791770166\\
25.98	0.00245830933604365\\
25.99	0.00245896125662465\\
26	0.00245961368018468\\
26.01	0.00246026660746542\\
26.02	0.00246092003921027\\
26.03	0.00246157397616433\\
26.04	0.0024622284190744\\
26.05	0.00246288336868901\\
26.06	0.00246353882575839\\
26.07	0.00246419479103449\\
26.08	0.002464851265271\\
26.09	0.00246550824922334\\
26.1	0.00246616574364866\\
26.11	0.00246682374930586\\
26.12	0.00246748226695558\\
26.13	0.0024681412973602\\
26.14	0.00246880084128389\\
26.15	0.00246946089949255\\
26.16	0.00247012147275386\\
26.17	0.00247078256183728\\
26.18	0.00247144416751403\\
26.19	0.00247210629055712\\
26.2	0.00247276893174134\\
26.21	0.00247343209184329\\
26.22	0.00247409577164134\\
26.23	0.00247475997191569\\
26.24	0.00247542469344832\\
26.25	0.00247608993702304\\
26.26	0.00247675570342549\\
26.27	0.00247742199344309\\
26.28	0.00247808880786513\\
26.29	0.00247875614748272\\
26.3	0.00247942401308879\\
26.31	0.00248009240547815\\
26.32	0.00248076132544743\\
26.33	0.00248143077379512\\
26.34	0.00248210075132156\\
26.35	0.00248277125882899\\
26.36	0.00248344229712148\\
26.37	0.00248411386700499\\
26.38	0.00248478596928738\\
26.39	0.00248545860477836\\
26.4	0.00248613177428955\\
26.41	0.00248680547863448\\
26.42	0.00248747971862857\\
26.43	0.00248815449508914\\
26.44	0.00248882980883542\\
26.45	0.00248950566068859\\
26.46	0.00249018205147172\\
26.47	0.00249085898200983\\
26.48	0.00249153645312987\\
26.49	0.00249221446566072\\
26.5	0.00249289302043323\\
26.51	0.00249357211828018\\
26.52	0.00249425176003632\\
26.53	0.00249493194653835\\
26.54	0.00249561267862496\\
26.55	0.0024962939571368\\
26.56	0.0024969757829165\\
26.57	0.00249765815680868\\
26.58	0.00249834107965995\\
26.59	0.00249902455231892\\
26.6	0.00249970857563619\\
26.61	0.0025003931504644\\
26.62	0.00250107827765818\\
26.63	0.00250176395807417\\
26.64	0.00250245019257107\\
26.65	0.00250313698200959\\
26.66	0.00250382432725248\\
26.67	0.00250451222916454\\
26.68	0.00250520068861261\\
26.69	0.00250588970646559\\
26.7	0.00250657928359446\\
26.71	0.00250726942087223\\
26.72	0.00250796011917401\\
26.73	0.002508651379377\\
26.74	0.00250934320236047\\
26.75	0.00251003558900576\\
26.76	0.00251072854019635\\
26.77	0.0025114220568178\\
26.78	0.00251211613975778\\
26.79	0.00251281078990609\\
26.8	0.00251350600815463\\
26.81	0.00251420179539745\\
26.82	0.00251489815253072\\
26.83	0.00251559508045276\\
26.84	0.00251629258006403\\
26.85	0.00251699065226715\\
26.86	0.00251768929796689\\
26.87	0.0025183885180702\\
26.88	0.0025190883134862\\
26.89	0.00251978868512617\\
26.9	0.00252048963390358\\
26.91	0.00252119116073412\\
26.92	0.00252189326653564\\
26.93	0.00252259595222822\\
26.94	0.00252329921873412\\
26.95	0.00252400306697786\\
26.96	0.00252470749788614\\
26.97	0.00252541251238793\\
26.98	0.0025261181114144\\
26.99	0.00252682429589898\\
27	0.00252753106677736\\
27.01	0.00252823842498746\\
27.02	0.00252894637146948\\
27.03	0.00252965490716588\\
27.04	0.00253036403302141\\
27.05	0.00253107374998308\\
27.06	0.00253178405900022\\
27.07	0.00253249496102442\\
27.08	0.00253320645700959\\
27.09	0.00253391854791197\\
27.1	0.00253463123469008\\
27.11	0.00253534451830477\\
27.12	0.00253605839971924\\
27.13	0.00253677287989901\\
27.14	0.00253748795981194\\
27.15	0.00253820364042826\\
27.16	0.00253891992272053\\
27.17	0.0025396368076637\\
27.18	0.00254035429623507\\
27.19	0.00254107238941432\\
27.2	0.00254179108818353\\
27.21	0.00254251039352717\\
27.22	0.00254323030643209\\
27.23	0.00254395082788756\\
27.24	0.00254467195888526\\
27.25	0.00254539370041929\\
27.26	0.00254611605348618\\
27.27	0.0025468390190849\\
27.28	0.00254756259821683\\
27.29	0.00254828679188585\\
27.3	0.00254901160109825\\
27.31	0.00254973702686281\\
27.32	0.00255046307019077\\
27.33	0.00255118973209584\\
27.34	0.00255191701359423\\
27.35	0.00255264491570464\\
27.36	0.00255337343944827\\
27.37	0.00255410258584882\\
27.38	0.00255483235593251\\
27.39	0.00255556275072807\\
27.4	0.00255629377126677\\
27.41	0.00255702541858244\\
27.42	0.0025577576937114\\
27.43	0.00255849059769257\\
27.44	0.00255922413156742\\
27.45	0.00255995829637996\\
27.46	0.0025606930931768\\
27.47	0.00256142852300714\\
27.48	0.00256216458692274\\
27.49	0.00256290128597798\\
27.5	0.00256363862122984\\
27.51	0.00256437659373791\\
27.52	0.00256511520456441\\
27.53	0.00256585445477418\\
27.54	0.00256659434543469\\
27.55	0.00256733487761607\\
27.56	0.0025680760523911\\
27.57	0.00256881787083521\\
27.58	0.0025695603340265\\
27.59	0.00257030344304576\\
27.6	0.00257104719897643\\
27.61	0.0025717916029047\\
27.62	0.00257253665591939\\
27.63	0.00257328235911208\\
27.64	0.00257402871357705\\
27.65	0.00257477572041129\\
27.66	0.00257552338071455\\
27.67	0.0025762716955893\\
27.68	0.00257702066614076\\
27.69	0.00257777029347692\\
27.7	0.00257852057870851\\
27.71	0.00257927152294907\\
27.72	0.00258002312731488\\
27.73	0.00258077539292504\\
27.74	0.00258152832090144\\
27.75	0.00258228191236877\\
27.76	0.00258303616845454\\
27.77	0.00258379109028908\\
27.78	0.00258454667900557\\
27.79	0.00258530293574001\\
27.8	0.00258605986163125\\
27.81	0.00258681745782099\\
27.82	0.00258757572545383\\
27.83	0.00258833466567721\\
27.84	0.00258909427964147\\
27.85	0.00258985456849983\\
27.86	0.00259061553340843\\
27.87	0.0025913771755263\\
27.88	0.0025921394960154\\
27.89	0.00259290249604059\\
27.9	0.00259366617676971\\
27.91	0.00259443053937352\\
27.92	0.00259519558502572\\
27.93	0.00259596131490301\\
27.94	0.00259672773018501\\
27.95	0.00259749483205436\\
27.96	0.00259826262169668\\
27.97	0.00259903110030059\\
27.98	0.0025998002690577\\
27.99	0.00260057012916264\\
28	0.00260134068181309\\
28.01	0.00260211192820974\\
28.02	0.00260288386955633\\
28.03	0.00260365650705965\\
28.04	0.00260442984192955\\
28.05	0.00260520387537896\\
28.06	0.00260597860862388\\
28.07	0.00260675404288342\\
28.08	0.00260753017937976\\
28.09	0.00260830701933822\\
28.1	0.00260908456398719\\
28.11	0.00260986281455824\\
28.12	0.00261064177228604\\
28.13	0.00261142143840844\\
28.14	0.0026122018141664\\
28.15	0.00261298290080409\\
28.16	0.00261376469956882\\
28.17	0.00261454721171112\\
28.18	0.00261533043848467\\
28.19	0.00261611438114639\\
28.2	0.00261689904095641\\
28.21	0.00261768441917806\\
28.22	0.00261847051707792\\
28.23	0.00261925733592581\\
28.24	0.00262004487699481\\
28.25	0.00262083314156124\\
28.26	0.00262162213090471\\
28.27	0.00262241184630813\\
28.28	0.00262320228905766\\
28.29	0.00262399346044279\\
28.3	0.00262478536175633\\
28.31	0.00262557799429439\\
28.32	0.00262637135935642\\
28.33	0.00262716545824521\\
28.34	0.00262796029226692\\
28.35	0.00262875586273106\\
28.36	0.00262955217095051\\
28.37	0.00263034921824154\\
28.38	0.00263114700592381\\
28.39	0.0026319455353204\\
28.4	0.00263274480775777\\
28.41	0.00263354482456585\\
28.42	0.00263434558707797\\
28.43	0.00263514709663092\\
28.44	0.00263594935456495\\
28.45	0.00263675236222376\\
28.46	0.00263755593112226\\
28.47	0.0026383599304226\\
28.48	0.00263916436051276\\
28.49	0.00263996922178145\\
28.5	0.00264077451461806\\
28.51	0.00264158023941267\\
28.52	0.0026423863965561\\
28.53	0.00264319298643986\\
28.54	0.00264400000945617\\
28.55	0.00264480746599796\\
28.56	0.00264561535645889\\
28.57	0.00264642368123331\\
28.58	0.00264723244071632\\
28.59	0.00264804163530372\\
28.6	0.00264885126539204\\
28.61	0.00264966133137853\\
28.62	0.00265047183366116\\
28.63	0.00265128277263865\\
28.64	0.00265209414871043\\
28.65	0.00265290596227668\\
28.66	0.0026537182137383\\
28.67	0.00265453090349694\\
28.68	0.00265534403195498\\
28.69	0.00265615759951555\\
28.7	0.00265697160658251\\
28.71	0.00265778605356049\\
28.72	0.00265860094085484\\
28.73	0.00265941626887169\\
28.74	0.00266023203801789\\
28.75	0.00266104824870106\\
28.76	0.00266186490132959\\
28.77	0.00266268199631261\\
28.78	0.00266349953406002\\
28.79	0.00266431751498248\\
28.8	0.00266513593949142\\
28.81	0.00266595480799903\\
28.82	0.00266677412091829\\
28.83	0.00266759387866291\\
28.84	0.00266841408164742\\
28.85	0.00266923473028711\\
28.86	0.00267005582499804\\
28.87	0.00267087736619707\\
28.88	0.00267169935430182\\
28.89	0.00267252178973072\\
28.9	0.00267334467290298\\
28.91	0.00267416800423859\\
28.92	0.00267499178415834\\
28.93	0.00267581601308383\\
28.94	0.00267664069143744\\
28.95	0.00267746581964234\\
28.96	0.00267829139812253\\
28.97	0.00267911742730279\\
28.98	0.00267994390760874\\
28.99	0.00268077083946677\\
29	0.0026815982233041\\
29.01	0.00268242605954877\\
29.02	0.00268325434862963\\
29.03	0.00268408309097635\\
29.04	0.00268491228701941\\
29.05	0.00268574193719014\\
29.06	0.00268657204192068\\
29.07	0.00268740260164399\\
29.08	0.00268823361679387\\
29.09	0.00268906508780496\\
29.1	0.00268989701511273\\
29.11	0.0026907293991535\\
29.12	0.0026915622403644\\
29.13	0.00269239553918345\\
29.14	0.00269322929604947\\
29.15	0.00269406351140215\\
29.16	0.00269489818568204\\
29.17	0.00269573331933053\\
29.18	0.00269656891278986\\
29.19	0.00269740496650315\\
29.2	0.00269824148091437\\
29.21	0.00269907845646835\\
29.22	0.00269991589361079\\
29.23	0.00270075379278827\\
29.24	0.00270159215444821\\
29.25	0.00270243097903894\\
29.26	0.00270327026700965\\
29.27	0.00270411001881041\\
29.28	0.00270495023489219\\
29.29	0.0027057909157068\\
29.3	0.00270663206170699\\
29.31	0.00270747367334636\\
29.32	0.00270831575107943\\
29.33	0.00270915829536161\\
29.34	0.00271000130664918\\
29.35	0.00271084478539937\\
29.36	0.00271168873207027\\
29.37	0.00271253314712089\\
29.38	0.00271337803101118\\
29.39	0.00271422338420194\\
29.4	0.00271506920715493\\
29.41	0.00271591550033283\\
29.42	0.00271676226419921\\
29.43	0.00271760949921858\\
29.44	0.00271845720585638\\
29.45	0.00271930538457897\\
29.46	0.00272015403585365\\
29.47	0.00272100316014864\\
29.48	0.0027218527579331\\
29.49	0.00272270282967715\\
29.5	0.00272355337585182\\
29.51	0.00272440439692911\\
29.52	0.00272525589338197\\
29.53	0.00272610786568428\\
29.54	0.00272696031431088\\
29.55	0.00272781323973759\\
29.56	0.00272866664244115\\
29.57	0.00272952052289931\\
29.58	0.00273037488159075\\
29.59	0.00273122971899512\\
29.6	0.00273208503559307\\
29.61	0.00273294083186618\\
29.62	0.00273379710829705\\
29.63	0.00273465386536925\\
29.64	0.00273551110356732\\
29.65	0.00273636882337678\\
29.66	0.00273722702528417\\
29.67	0.002738085709777\\
29.68	0.00273894487734377\\
29.69	0.00273980452847401\\
29.7	0.00274066466365822\\
29.71	0.00274152528338792\\
29.72	0.00274238638815563\\
29.73	0.00274324797845488\\
29.74	0.00274411005478022\\
29.75	0.00274497261762723\\
29.76	0.00274583566749249\\
29.77	0.0027466992048736\\
29.78	0.0027475632302692\\
29.79	0.00274842774417897\\
29.8	0.00274929274710359\\
29.81	0.0027501582395448\\
29.82	0.00275102422200537\\
29.83	0.00275189069498913\\
29.84	0.00275275765900093\\
29.85	0.00275362511454669\\
29.86	0.00275449306213336\\
29.87	0.00275536150226898\\
29.88	0.00275623043546261\\
29.89	0.00275709986222439\\
29.9	0.00275796978306553\\
29.91	0.0027588401984983\\
29.92	0.00275971110903604\\
29.93	0.00276058251519318\\
29.94	0.00276145441748522\\
29.95	0.00276232681642873\\
29.96	0.00276319971254137\\
29.97	0.00276407310634191\\
29.98	0.00276494699835019\\
29.99	0.00276582138908715\\
30	0.00276669627907483\\
30.01	0.00276757166883638\\
30.02	0.00276844755889603\\
30.03	0.00276932394977916\\
30.04	0.00277020084201222\\
30.05	0.0027710782361228\\
30.06	0.00277195613263961\\
30.07	0.00277283453209247\\
30.08	0.00277371343501234\\
30.09	0.00277459284193131\\
30.1	0.00277547275338259\\
30.11	0.00277635316990053\\
30.12	0.00277723409202063\\
30.13	0.00277811552027953\\
30.14	0.00277899745521502\\
30.15	0.00277987989736604\\
30.16	0.00278076284727268\\
30.17	0.00278164630547619\\
30.18	0.00278253027251899\\
30.19	0.00278341474894466\\
30.2	0.00278429973529795\\
30.21	0.00278518523212478\\
30.22	0.00278607123997225\\
30.23	0.00278695775938865\\
30.24	0.00278784479092343\\
30.25	0.00278873233512725\\
30.26	0.00278962039255196\\
30.27	0.0027905089637506\\
30.28	0.00279139804927741\\
30.29	0.00279228764968783\\
30.3	0.00279317776553851\\
30.31	0.00279406839738731\\
30.32	0.00279495954579331\\
30.33	0.0027958512113168\\
30.34	0.0027967433945193\\
30.35	0.00279763609596355\\
30.36	0.00279852931621352\\
30.37	0.00279942305583442\\
30.38	0.00280031731539269\\
30.39	0.00280121209545603\\
30.4	0.00280210739659335\\
30.41	0.00280300321937485\\
30.42	0.00280389956437195\\
30.43	0.00280479643215736\\
30.44	0.00280569382330501\\
30.45	0.00280659173839014\\
30.46	0.00280749017798924\\
30.47	0.00280838914268006\\
30.48	0.00280928863304164\\
30.49	0.00281018864965431\\
30.5	0.00281108919309968\\
30.51	0.00281199026396065\\
30.52	0.00281289186282141\\
30.53	0.00281379399026745\\
30.54	0.00281469664688556\\
30.55	0.00281559983326384\\
30.56	0.0028165035499917\\
30.57	0.00281740779765988\\
30.58	0.0028183125768604\\
30.59	0.00281921788818665\\
30.6	0.00282012373223331\\
30.61	0.00282103010959642\\
30.62	0.00282193702087333\\
30.63	0.00282284446666276\\
30.64	0.00282375244756475\\
30.65	0.00282466096418069\\
30.66	0.00282557001711334\\
30.67	0.0028264796069668\\
30.68	0.00282738973434656\\
30.69	0.00282830039985943\\
30.7	0.00282921160411364\\
30.71	0.00283012334771876\\
30.72	0.00283103563128576\\
30.73	0.00283194845542699\\
30.74	0.00283286182075617\\
30.75	0.00283377572788844\\
30.76	0.00283469017744033\\
30.77	0.00283560517002975\\
30.78	0.00283652070627605\\
30.79	0.00283743678679998\\
30.8	0.00283835341222369\\
30.81	0.00283927058317077\\
30.82	0.00284018830026622\\
30.83	0.00284110656413647\\
30.84	0.0028420253754094\\
30.85	0.00284294473471432\\
30.86	0.00284386464268199\\
30.87	0.00284478509994458\\
30.88	0.00284570610713578\\
30.89	0.00284662766489067\\
30.9	0.00284754977384583\\
30.91	0.00284847243463932\\
30.92	0.00284939564791063\\
30.93	0.00285031941430075\\
30.94	0.00285124373445216\\
30.95	0.0028521686090088\\
30.96	0.00285309403861612\\
30.97	0.00285402002392108\\
30.98	0.0028549465655721\\
30.99	0.00285587366421914\\
31	0.00285680132051366\\
31.01	0.00285772953510862\\
31.02	0.00285865830865853\\
31.03	0.00285958764181941\\
31.04	0.0028605175352488\\
31.05	0.00286144798960578\\
31.06	0.002862379005551\\
31.07	0.00286331058374661\\
31.08	0.00286424272485634\\
31.09	0.00286517542954546\\
31.1	0.00286610869848081\\
31.11	0.0028670425323308\\
31.12	0.0028679769317654\\
31.13	0.00286891189745616\\
31.14	0.00286984743007621\\
31.15	0.00287078353030026\\
31.16	0.00287172019880463\\
31.17	0.00287265743626722\\
31.18	0.00287359524336753\\
31.19	0.0028745336207867\\
31.2	0.00287547256920743\\
31.21	0.00287641208931407\\
31.22	0.0028773521817926\\
31.23	0.00287829284733061\\
31.24	0.00287923408661733\\
31.25	0.00288017590034363\\
31.26	0.00288111828920203\\
31.27	0.00288206125388667\\
31.28	0.0028830047950934\\
31.29	0.00288394891351969\\
31.3	0.00288489360986469\\
31.31	0.0028858388848292\\
31.32	0.00288678473911574\\
31.33	0.00288773117342849\\
31.34	0.00288867818847331\\
31.35	0.00288962578495775\\
31.36	0.00289057396359109\\
31.37	0.00289152272508429\\
31.38	0.00289247207015004\\
31.39	0.00289342199950272\\
31.4	0.00289437251385845\\
31.41	0.00289532361393508\\
31.42	0.00289627530045219\\
31.43	0.0028972275741311\\
31.44	0.00289818043569487\\
31.45	0.00289913388586831\\
31.46	0.00290008792537801\\
31.47	0.0029010425549523\\
31.48	0.00290199777532128\\
31.49	0.00290295358721682\\
31.5	0.00290390999137258\\
31.51	0.00290486698852403\\
31.52	0.00290582457940838\\
31.53	0.00290678276476468\\
31.54	0.00290774154533377\\
31.55	0.0029087009218583\\
31.56	0.00290966089508273\\
31.57	0.00291062146575335\\
31.58	0.00291158263461829\\
31.59	0.00291254440242749\\
31.6	0.00291350676993275\\
31.61	0.00291446973788771\\
31.62	0.00291543330704785\\
31.63	0.00291639747817055\\
31.64	0.002917362252015\\
31.65	0.0029183276293423\\
31.66	0.00291929361091542\\
31.67	0.00292026019749921\\
31.68	0.00292122738986041\\
31.69	0.00292219518876766\\
31.7	0.00292316359499151\\
31.71	0.0029241326093044\\
31.72	0.00292510223248071\\
31.73	0.00292607246529672\\
31.74	0.00292704330853066\\
31.75	0.00292801476296267\\
31.76	0.00292898682937486\\
31.77	0.00292995950855127\\
31.78	0.0029309328012779\\
31.79	0.00293190670834272\\
31.8	0.00293288123053564\\
31.81	0.00293385636864859\\
31.82	0.00293483212347544\\
31.83	0.00293580849581206\\
31.84	0.00293678548645634\\
31.85	0.00293776309620813\\
31.86	0.00293874132586933\\
31.87	0.00293972017624381\\
31.88	0.00294069964813751\\
31.89	0.00294167974235836\\
31.9	0.00294266045971634\\
31.91	0.00294364180102348\\
31.92	0.00294462376709385\\
31.93	0.00294560635874359\\
31.94	0.00294658957679086\\
31.95	0.00294757342205595\\
31.96	0.0029485578953612\\
31.97	0.00294954299753102\\
31.98	0.00295052872939192\\
31.99	0.00295151509177254\\
32	0.00295250208550357\\
32.01	0.00295348971141785\\
32.02	0.00295447797035035\\
32.03	0.00295546686313812\\
32.04	0.00295645639062039\\
32.05	0.00295744655363851\\
32.06	0.00295843735303598\\
32.07	0.00295942878965846\\
32.08	0.00296042086435377\\
32.09	0.00296141357797192\\
32.1	0.00296240693136506\\
32.11	0.00296304776470471\\
32.12	0.00296349143659751\\
32.13	0.00296393529352657\\
32.14	0.0029643793355601\\
32.15	0.0029648235627663\\
32.16	0.00296526797521342\\
32.17	0.00296571257296967\\
32.18	0.0029661573561033\\
32.19	0.00296660232468256\\
32.2	0.00296704747877571\\
32.21	0.00296749281845105\\
32.22	0.00296793834377684\\
32.23	0.00296838405482139\\
32.24	0.00296882995165302\\
32.25	0.00296927603434003\\
32.26	0.00296972230295077\\
32.27	0.00297016875755357\\
32.28	0.0029706153982168\\
32.29	0.00297106222500881\\
32.3	0.00297150923799799\\
32.31	0.00297195643725272\\
32.32	0.00297240382284143\\
32.33	0.00297285139483251\\
32.34	0.0029732991532944\\
32.35	0.00297374709829552\\
32.36	0.00297419522990433\\
32.37	0.00297464354818931\\
32.38	0.00297509205321891\\
32.39	0.00297554074506163\\
32.4	0.00297598962378598\\
32.41	0.00297643868946047\\
32.42	0.00297688794215362\\
32.43	0.00297733738193397\\
32.44	0.00297778700887009\\
32.45	0.00297823682303053\\
32.46	0.00297868682448387\\
32.47	0.00297913701329872\\
32.48	0.00297958738954366\\
32.49	0.00298003795328733\\
32.5	0.00298048870459836\\
32.51	0.00298093964354539\\
32.52	0.00298139077019709\\
32.53	0.00298184208462213\\
32.54	0.0029822935868892\\
32.55	0.00298274527706701\\
32.56	0.00298319715522428\\
32.57	0.00298364922142973\\
32.58	0.00298410147575211\\
32.59	0.0029845539182602\\
32.6	0.00298500654902275\\
32.61	0.00298545936810857\\
32.62	0.00298591237558646\\
32.63	0.00298636557152525\\
32.64	0.00298681895599377\\
32.65	0.00298727252906088\\
32.66	0.00298772629079544\\
32.67	0.00298818024126633\\
32.68	0.00298863438054246\\
32.69	0.00298908870869274\\
32.7	0.00298954322578611\\
32.71	0.00298999793189151\\
32.72	0.0029904528270779\\
32.73	0.00299090791141428\\
32.74	0.00299136318496963\\
32.75	0.00299181864781296\\
32.76	0.0029922743000133\\
32.77	0.00299273014163972\\
32.78	0.00299318617276127\\
32.79	0.00299364239344702\\
32.8	0.00299409880376609\\
32.81	0.00299455540378758\\
32.82	0.00299501219358064\\
32.83	0.00299546917321442\\
32.84	0.00299592634275807\\
32.85	0.0029963837022808\\
32.86	0.0029968412518518\\
32.87	0.00299729899154032\\
32.88	0.00299775692141557\\
32.89	0.00299821504154683\\
32.9	0.00299867335200339\\
32.91	0.00299913185285453\\
32.92	0.00299959054416957\\
32.93	0.00300004942601786\\
32.94	0.00300050849846875\\
32.95	0.00300096776159162\\
32.96	0.00300142721545587\\
32.97	0.00300188686013091\\
32.98	0.00300234669568618\\
32.99	0.00300280672219113\\
33	0.00300326693971525\\
33.01	0.00300372734832802\\
33.02	0.00300418794809897\\
33.03	0.00300464873909763\\
33.04	0.00300510972139357\\
33.05	0.00300557089505637\\
33.06	0.00300603226015563\\
33.07	0.00300649381676098\\
33.08	0.00300695556494205\\
33.09	0.00300741750476852\\
33.1	0.00300787963631009\\
33.11	0.00300834195963644\\
33.12	0.00300880447481734\\
33.13	0.00300926718192252\\
33.14	0.00300973008102177\\
33.15	0.00301019317218491\\
33.16	0.00301065645548174\\
33.17	0.00301111993098211\\
33.18	0.0030115835987559\\
33.19	0.00301204745887302\\
33.2	0.00301251151140337\\
33.21	0.0030129757564169\\
33.22	0.00301344019398359\\
33.23	0.00301390482417341\\
33.24	0.00301436964705639\\
33.25	0.00301483466270258\\
33.26	0.00301529987118204\\
33.27	0.00301576527256485\\
33.28	0.00301623086692114\\
33.29	0.00301669665432105\\
33.3	0.00301716263483476\\
33.31	0.00301762880853246\\
33.32	0.00301809517548437\\
33.33	0.00301856173576073\\
33.34	0.00301902848943183\\
33.35	0.00301949543656796\\
33.36	0.00301996257723946\\
33.37	0.00302042991151667\\
33.38	0.00302089743946999\\
33.39	0.00302136516116982\\
33.4	0.0030218330766866\\
33.41	0.00302230118609081\\
33.42	0.00302276948945294\\
33.43	0.00302323798684351\\
33.44	0.00302370667833307\\
33.45	0.00302417556399221\\
33.46	0.00302464464389154\\
33.47	0.00302511391810169\\
33.48	0.00302558338669335\\
33.49	0.00302605304973721\\
33.5	0.003026522907304\\
33.51	0.00302699295946449\\
33.52	0.00302746320628946\\
33.53	0.00302793364784974\\
33.54	0.00302840428421618\\
33.55	0.00302887511545968\\
33.56	0.00302934614165114\\
33.57	0.00302981736286152\\
33.58	0.0030302887791618\\
33.59	0.00303076039062299\\
33.6	0.00303123219731613\\
33.61	0.00303170419931232\\
33.62	0.00303217639668266\\
33.63	0.0030326487894983\\
33.64	0.00303312137783041\\
33.65	0.00303359416175022\\
33.66	0.00303406714132896\\
33.67	0.00303454031663792\\
33.68	0.00303501368774842\\
33.69	0.00303548725473181\\
33.7	0.00303596101765948\\
33.71	0.00303643497660285\\
33.72	0.00303690913163338\\
33.73	0.00303738348282257\\
33.74	0.00303785803024195\\
33.75	0.00303833277396309\\
33.76	0.00303880771405758\\
33.77	0.00303928285059708\\
33.78	0.00303975818365326\\
33.79	0.00304023371329784\\
33.8	0.00304070943960258\\
33.81	0.00304118536263926\\
33.82	0.00304166148247972\\
33.83	0.00304213779919584\\
33.84	0.00304261431285953\\
33.85	0.00304309102354273\\
33.86	0.00304356793131744\\
33.87	0.00304404503625569\\
33.88	0.00304452233842954\\
33.89	0.00304499983791112\\
33.9	0.00304547753477256\\
33.91	0.00304595542908607\\
33.92	0.00304643352092389\\
33.93	0.00304691181035829\\
33.94	0.00304739029746159\\
33.95	0.00304786898230615\\
33.96	0.00304834786496438\\
33.97	0.00304882694550874\\
33.98	0.00304930622401171\\
33.99	0.00304978570054582\\
34	0.00305026537518367\\
34.01	0.00305074524799789\\
34.02	0.00305122531906113\\
34.03	0.00305170558844611\\
34.04	0.00305218605622561\\
34.05	0.00305266672247241\\
34.06	0.00305314758725939\\
34.07	0.00305362865065943\\
34.08	0.0030541099127455\\
34.09	0.00305459137359057\\
34.1	0.0030550730332677\\
34.11	0.00305555489184998\\
34.12	0.00305603694941054\\
34.13	0.00305651920602258\\
34.14	0.00305700166175933\\
34.15	0.00305748431669409\\
34.16	0.00305796717090017\\
34.17	0.00305845022445098\\
34.18	0.00305893347741995\\
34.19	0.00305941692988057\\
34.2	0.00305990058190639\\
34.21	0.00306038443357098\\
34.22	0.00306086848494801\\
34.23	0.00306135273611115\\
34.24	0.00306183718713417\\
34.25	0.00306232183809087\\
34.26	0.00306280668905509\\
34.27	0.00306329174010076\\
34.28	0.00306377699130184\\
34.29	0.00306426244273234\\
34.3	0.00306474809446634\\
34.31	0.00306523394657798\\
34.32	0.00306571999914144\\
34.33	0.00306620625223097\\
34.34	0.00306669270592086\\
34.35	0.00306717936028547\\
34.36	0.00306766621539922\\
34.37	0.00306815327133659\\
34.38	0.00306864052817209\\
34.39	0.00306912798598034\\
34.4	0.00306961564483597\\
34.41	0.0030701035048137\\
34.42	0.00307059156598829\\
34.43	0.00307107982843458\\
34.44	0.00307156829222747\\
34.45	0.0030720569574419\\
34.46	0.00307254582415289\\
34.47	0.00307303489243553\\
34.48	0.00307352416236495\\
34.49	0.00307401363401635\\
34.5	0.00307450330746502\\
34.51	0.00307499318278627\\
34.52	0.00307548326005551\\
34.53	0.0030759735393482\\
34.54	0.00307646402073988\\
34.55	0.00307695470430613\\
34.56	0.00307744559012262\\
34.57	0.00307793667826507\\
34.58	0.00307842796880928\\
34.59	0.00307891946183113\\
34.6	0.00307941115740653\\
34.61	0.0030799030556115\\
34.62	0.0030803951565221\\
34.63	0.00308088746021448\\
34.64	0.00308137996676485\\
34.65	0.00308187267624949\\
34.66	0.00308236558874476\\
34.67	0.00308285870432708\\
34.68	0.00308335202307296\\
34.69	0.00308384554505896\\
34.7	0.00308433927036172\\
34.71	0.00308483319905798\\
34.72	0.00308532733122452\\
34.73	0.00308582166693822\\
34.74	0.00308631620627601\\
34.75	0.00308681094931493\\
34.76	0.00308730589613206\\
34.77	0.00308780104680459\\
34.78	0.00308829640140976\\
34.79	0.00308879196002491\\
34.8	0.00308928772272746\\
34.81	0.00308978368959487\\
34.82	0.00309027986070474\\
34.83	0.0030907762361347\\
34.84	0.00309127281596248\\
34.85	0.00309176960026591\\
34.86	0.00309226658912286\\
34.87	0.00309276378261134\\
34.88	0.00309326118080939\\
34.89	0.00309375878379515\\
34.9	0.00309425659164686\\
34.91	0.00309475460444283\\
34.92	0.00309525282226146\\
34.93	0.00309575124518124\\
34.94	0.00309624987328073\\
34.95	0.00309674870663861\\
34.96	0.00309724774533362\\
34.97	0.00309774698944459\\
34.98	0.00309824643905045\\
34.99	0.00309874609423023\\
35	0.00309924595506301\\
35.01	0.003099746021628\\
35.02	0.00310024629400449\\
35.03	0.00310074677227186\\
35.04	0.00310124745650959\\
35.05	0.00310174834679725\\
35.06	0.00310224944321449\\
35.07	0.00310275074584108\\
35.08	0.00310325225475686\\
35.09	0.00310375397004179\\
35.1	0.0031042558917759\\
35.11	0.00310475802003936\\
35.12	0.00310526035491238\\
35.13	0.00310576289647531\\
35.14	0.0031062656448086\\
35.15	0.00310676859999278\\
35.16	0.00310727176210849\\
35.17	0.00310777513123647\\
35.18	0.00310827870745757\\
35.19	0.00310878249085274\\
35.2	0.00310928648150301\\
35.21	0.00310979067948956\\
35.22	0.00311029508489363\\
35.23	0.00311079969779658\\
35.24	0.0031113045182799\\
35.25	0.00311180954642516\\
35.26	0.00311231478231403\\
35.27	0.00311282022602833\\
35.28	0.00311332587764994\\
35.29	0.00311383173726088\\
35.3	0.00311433780494329\\
35.31	0.00311484408077938\\
35.32	0.00311535056485152\\
35.33	0.00311585725724214\\
35.34	0.00311636415803384\\
35.35	0.00311687126730929\\
35.36	0.00311737858515131\\
35.37	0.00311788611164281\\
35.38	0.00311839384686682\\
35.39	0.00311890179090649\\
35.4	0.0031194099438451\\
35.41	0.00311991830576603\\
35.42	0.0031204268767528\\
35.43	0.00312093565688903\\
35.44	0.00312144464625848\\
35.45	0.00312195384494501\\
35.46	0.00312246325303262\\
35.47	0.00312297287060544\\
35.48	0.00312348269774771\\
35.49	0.00312399273454379\\
35.5	0.00312450298107819\\
35.51	0.0031250134374355\\
35.52	0.0031255241037005\\
35.53	0.00312603497995805\\
35.54	0.00312654606629318\\
35.55	0.00312705736279101\\
35.56	0.00312756886953681\\
35.57	0.00312808058661598\\
35.58	0.00312859251411407\\
35.59	0.00312910465211674\\
35.6	0.00312961700070977\\
35.61	0.00313012955997912\\
35.62	0.00313064233001086\\
35.63	0.0031311553108912\\
35.64	0.00313166850270648\\
35.65	0.00313218190554321\\
35.66	0.00313269551948799\\
35.67	0.0031332093446276\\
35.68	0.00313372338104894\\
35.69	0.00313423762883908\\
35.7	0.0031347520880852\\
35.71	0.00313526675887466\\
35.72	0.00313578164129492\\
35.73	0.00313629673543362\\
35.74	0.00313681204137854\\
35.75	0.00313732755921761\\
35.76	0.00313784328903891\\
35.77	0.00313835923093064\\
35.78	0.0031388753849812\\
35.79	0.00313939175127911\\
35.8	0.00313990832991306\\
35.81	0.00314042512097187\\
35.82	0.00314094212454454\\
35.83	0.00314145934072021\\
35.84	0.00314197676958818\\
35.85	0.00314249441123792\\
35.86	0.00314301226575904\\
35.87	0.00314353033324132\\
35.88	0.00314404861377469\\
35.89	0.00314456710744926\\
35.9	0.00314508581435529\\
35.91	0.00314560473458321\\
35.92	0.00314612386822362\\
35.93	0.00314664321536726\\
35.94	0.00314716277610506\\
35.95	0.00314768255052813\\
35.96	0.0031482025387277\\
35.97	0.00314872274079523\\
35.98	0.00314924315682231\\
35.99	0.00314976378690072\\
36	0.00315028463112239\\
36.01	0.00315080568957947\\
36.02	0.00315132696236425\\
36.03	0.00315184844956919\\
36.04	0.00315237015128696\\
36.05	0.00315289206761037\\
36.06	0.00315341419863246\\
36.07	0.0031539365444464\\
36.08	0.00315445910514557\\
36.09	0.00315498188082352\\
36.1	0.003155504871574\\
36.11	0.00315602807749093\\
36.12	0.00315655149866843\\
36.13	0.0031570751352008\\
36.14	0.00315759898718253\\
36.15	0.00315812305470829\\
36.16	0.00315864733787297\\
36.17	0.00315917183677163\\
36.18	0.00315969655149951\\
36.19	0.00316022148215208\\
36.2	0.00316074662882498\\
36.21	0.00316127199161405\\
36.22	0.00316179757061535\\
36.23	0.00316232336592513\\
36.24	0.0031628493776398\\
36.25	0.00316337560585605\\
36.26	0.0031639020506707\\
36.27	0.00316442871218081\\
36.28	0.00316495559048365\\
36.29	0.00316548268567669\\
36.3	0.00316600999785759\\
36.31	0.00316653752712424\\
36.32	0.00316706527357474\\
36.33	0.00316759323730741\\
36.34	0.00316812141842077\\
36.35	0.00316864981701354\\
36.36	0.00316917843318468\\
36.37	0.00316970726703338\\
36.38	0.003170236318659\\
36.39	0.00317076558816117\\
36.4	0.00317129507563971\\
36.41	0.00317182478119469\\
36.42	0.00317235470492637\\
36.43	0.00317288484693526\\
36.44	0.00317341520732208\\
36.45	0.00317394578618781\\
36.46	0.00317447658363362\\
36.47	0.00317500759976093\\
36.48	0.0031755388346714\\
36.49	0.00317607028846691\\
36.5	0.00317660196124957\\
36.51	0.00317713385312176\\
36.52	0.00317766596418605\\
36.53	0.00317819829454529\\
36.54	0.00317873084430255\\
36.55	0.00317926361356113\\
36.56	0.00317979660242462\\
36.57	0.00318032981099681\\
36.58	0.00318086323938174\\
36.59	0.00318139688768373\\
36.6	0.00318193075600732\\
36.61	0.00318246484445731\\
36.62	0.00318299915313876\\
36.63	0.00318353368215697\\
36.64	0.00318406843161751\\
36.65	0.0031846034016262\\
36.66	0.00318513859228911\\
36.67	0.00318567400371259\\
36.68	0.00318620963600322\\
36.69	0.00318674548926787\\
36.7	0.00318728156361369\\
36.71	0.00318781785914806\\
36.72	0.00318835437597864\\
36.73	0.00318889111421337\\
36.74	0.00318942807396044\\
36.75	0.00318996525532833\\
36.76	0.0031905026584258\\
36.77	0.00319104028336187\\
36.78	0.00319157813024584\\
36.79	0.0031921161991873\\
36.8	0.00319265449029611\\
36.81	0.00319319300368243\\
36.82	0.00319373173945666\\
36.83	0.00319427069772954\\
36.84	0.00319480987861207\\
36.85	0.00319534928221554\\
36.86	0.00319588890865154\\
36.87	0.00319642875803193\\
36.88	0.0031969688304689\\
36.89	0.00319750912607491\\
36.9	0.00319804964496272\\
36.91	0.0031985903872454\\
36.92	0.0031991313530363\\
36.93	0.00319967254244911\\
36.94	0.0032002139555978\\
36.95	0.00320075559259663\\
36.96	0.0032012974535602\\
36.97	0.00320183953860341\\
36.98	0.00320238184784146\\
36.99	0.00320292438138989\\
37	0.00320346713936452\\
37.01	0.00320401012188151\\
37.02	0.00320455332905734\\
37.03	0.00320509676100882\\
37.04	0.00320564041785304\\
37.05	0.00320618429970746\\
37.06	0.00320672840668985\\
37.07	0.00320727273891831\\
37.08	0.00320781729651127\\
37.09	0.00320836207958749\\
37.1	0.00320890708826606\\
37.11	0.00320945232266642\\
37.12	0.00320999778290834\\
37.13	0.00321054346911193\\
37.14	0.00321108938139763\\
37.15	0.00321163551988625\\
37.16	0.00321218188469893\\
37.17	0.00321272847595717\\
37.18	0.00321327529378279\\
37.19	0.00321382233829798\\
37.2	0.00321436960962531\\
37.21	0.00321491710788767\\
37.22	0.00321546483320831\\
37.23	0.00321601278571086\\
37.24	0.0032165609655193\\
37.25	0.00321710937275798\\
37.26	0.0032176580075516\\
37.27	0.00321820687002526\\
37.28	0.0032187559603044\\
37.29	0.00321930527851483\\
37.3	0.00321985482478277\\
37.31	0.00322040459923479\\
37.32	0.00322095460199784\\
37.33	0.00322150483319925\\
37.34	0.00322205529296675\\
37.35	0.00322260598142843\\
37.36	0.0032231568987128\\
37.37	0.00322370804494872\\
37.38	0.00322425942026548\\
37.39	0.00322481102479274\\
37.4	0.00322536285866055\\
37.41	0.0032259149219994\\
37.42	0.00322646721494013\\
37.43	0.00322701973761402\\
37.44	0.00322757249015274\\
37.45	0.00322812547268837\\
37.46	0.0032286786853534\\
37.47	0.00322923212828072\\
37.48	0.00322978580160366\\
37.49	0.00323033970545596\\
37.5	0.00323089383997176\\
37.51	0.00323144820528563\\
37.52	0.00323200280153258\\
37.53	0.00323255762884803\\
37.54	0.00323311268736785\\
37.55	0.0032336679772283\\
37.56	0.00323422349856611\\
37.57	0.00323477925151845\\
37.58	0.00323533523622289\\
37.59	0.00323589145281748\\
37.6	0.0032364479014407\\
37.61	0.00323700458223146\\
37.62	0.00323756149532914\\
37.63	0.00323811864087357\\
37.64	0.00323867601900502\\
37.65	0.00323923362986423\\
37.66	0.00323979147359238\\
37.67	0.00324034955033113\\
37.68	0.0032409078602226\\
37.69	0.00324146640340937\\
37.7	0.00324202518003448\\
37.71	0.00324258419024147\\
37.72	0.00324314343417432\\
37.73	0.00324370291197751\\
37.74	0.00324426262379599\\
37.75	0.0032448225697752\\
37.76	0.00324538275006105\\
37.77	0.00324594316479996\\
37.78	0.00324650381413882\\
37.79	0.00324706469822501\\
37.8	0.00324762581720642\\
37.81	0.00324818717123144\\
37.82	0.00324874876044894\\
37.83	0.00324931058500831\\
37.84	0.00324987264505945\\
37.85	0.00325043494075276\\
37.86	0.00325099747223915\\
37.87	0.00325156023967005\\
37.88	0.0032521232431974\\
37.89	0.00325268648297369\\
37.9	0.00325324995915188\\
37.91	0.00325381367188551\\
37.92	0.00325437762132862\\
37.93	0.00325494180763579\\
37.94	0.00325550623096212\\
37.95	0.00325607089146327\\
37.96	0.00325663578929544\\
37.97	0.00325720092461534\\
37.98	0.00325776629758028\\
37.99	0.00325833190834807\\
38	0.00325889775707711\\
38.01	0.00325946384392634\\
38.02	0.00326003016905525\\
38.03	0.0032605967326239\\
38.04	0.00326116353479292\\
38.05	0.00326173057572349\\
38.06	0.0032622978555774\\
38.07	0.00326286537451698\\
38.08	0.00326343313270513\\
38.09	0.00326400113030536\\
38.1	0.00326456936748175\\
38.11	0.00326513784439896\\
38.12	0.00326570656122225\\
38.13	0.00326627551811746\\
38.14	0.00326684471525106\\
38.15	0.00326741415279008\\
38.16	0.00326798383090216\\
38.17	0.00326855374975557\\
38.18	0.00326912390951916\\
38.19	0.00326969431036241\\
38.2	0.00327026495245542\\
38.21	0.0032708358359689\\
38.22	0.00327140696107418\\
38.23	0.00327197832794322\\
38.24	0.00327254993674862\\
38.25	0.00327312178766358\\
38.26	0.00327369388086198\\
38.27	0.00327426621651831\\
38.28	0.0032748387948077\\
38.29	0.00327541161590596\\
38.3	0.0032759846799895\\
38.31	0.00327655798723544\\
38.32	0.00327713153782151\\
38.33	0.00327770533192611\\
38.34	0.00327827936972833\\
38.35	0.00327885365140791\\
38.36	0.00327942817714526\\
38.37	0.00328000294712145\\
38.38	0.00328057796151826\\
38.39	0.00328115322051813\\
38.4	0.00328172872430419\\
38.41	0.00328230447306026\\
38.42	0.00328288046697087\\
38.43	0.00328345670622121\\
38.44	0.0032840331909972\\
38.45	0.00328460992148546\\
38.46	0.00328518689787331\\
38.47	0.00328576412034878\\
38.48	0.00328634158910062\\
38.49	0.00328691930431831\\
38.5	0.00328749726619203\\
38.51	0.00328807547491271\\
38.52	0.00328865393067201\\
38.53	0.00328923263366229\\
38.54	0.00328981158407669\\
38.55	0.00329039078210908\\
38.56	0.00329097022795406\\
38.57	0.003291549921807\\
38.58	0.00329212986386401\\
38.59	0.00329271005432199\\
38.6	0.00329329049337856\\
38.61	0.00329387118123212\\
38.62	0.00329445211808185\\
38.63	0.00329503330412771\\
38.64	0.00329561473957043\\
38.65	0.00329619642461151\\
38.66	0.00329677835945326\\
38.67	0.00329736054429876\\
38.68	0.0032979429793519\\
38.69	0.00329852566481736\\
38.7	0.00329910860090064\\
38.71	0.00329969178780802\\
38.72	0.00330027522574662\\
38.73	0.00330085891492437\\
38.74	0.00330144285555\\
38.75	0.00330202704783309\\
38.76	0.00330261149198404\\
38.77	0.00330319618821408\\
38.78	0.0033037811367353\\
38.79	0.00330436633776059\\
38.8	0.00330495179150372\\
38.81	0.0033055374981793\\
38.82	0.00330612345800281\\
38.83	0.00330670967119057\\
38.84	0.00330729613795977\\
38.85	0.00330788285852848\\
38.86	0.00330846983311564\\
38.87	0.00330905706194107\\
38.88	0.00330964454522545\\
38.89	0.00331023228319038\\
38.9	0.00331082027605835\\
38.91	0.00331140852405271\\
38.92	0.00331199702739776\\
38.93	0.00331258578631867\\
38.94	0.00331317480104154\\
38.95	0.00331376407179339\\
38.96	0.00331435359880214\\
38.97	0.00331494338229665\\
38.98	0.00331553342250671\\
38.99	0.00331612371966304\\
39	0.00331671427399731\\
39.01	0.00331730508574213\\
39.02	0.00331789615513104\\
39.03	0.00331848748239857\\
39.04	0.00331907906778017\\
39.05	0.0033196709115123\\
39.06	0.00332026301383235\\
39.07	0.00332085537497871\\
39.08	0.00332144799519073\\
39.09	0.00332204087470876\\
39.1	0.00332263401377415\\
39.11	0.00332322741262921\\
39.12	0.00332382107151727\\
39.13	0.00332441499068267\\
39.14	0.00332500917037077\\
39.15	0.00332560361082792\\
39.16	0.0033261983123015\\
39.17	0.00332679327503992\\
39.18	0.00332738849929262\\
39.19	0.00332798398531008\\
39.2	0.00332857973334382\\
39.21	0.00332917574364641\\
39.22	0.00332977201647146\\
39.23	0.00333036855207366\\
39.24	0.00333096535070874\\
39.25	0.00333156241263353\\
39.26	0.00333215973810591\\
39.27	0.00333275732738484\\
39.28	0.00333335518073039\\
39.29	0.00333395329840368\\
39.3	0.00333455168066697\\
39.31	0.00333515032778359\\
39.32	0.00333574924001802\\
39.33	0.00333634841758985\\
39.34	0.00333694786059512\\
39.35	0.00333754756912988\\
39.36	0.00333814754329021\\
39.37	0.00333874778317223\\
39.38	0.00333934828887206\\
39.39	0.00333994906048588\\
39.4	0.00334055009810984\\
39.41	0.0033411514018402\\
39.42	0.00334175297177319\\
39.43	0.00334235480800508\\
39.44	0.00334295691063218\\
39.45	0.00334355927975083\\
39.46	0.00334416191545739\\
39.47	0.00334476481784826\\
39.48	0.00334536798701988\\
39.49	0.00334597142306871\\
39.5	0.00334657512609125\\
39.51	0.00334717909618402\\
39.52	0.00334778333344361\\
39.53	0.00334838783796659\\
39.54	0.00334899260984962\\
39.55	0.00334959764918936\\
39.56	0.00335020295608253\\
39.57	0.00335080853062587\\
39.58	0.00335141437291617\\
39.59	0.00335202048305024\\
39.6	0.00335262686112495\\
39.61	0.00335323350723721\\
39.62	0.00335384042148396\\
39.63	0.00335444760396218\\
39.64	0.00335505505476889\\
39.65	0.00335566277400116\\
39.66	0.0033562707617561\\
39.67	0.00335687901813089\\
39.68	0.00335748754322269\\
39.69	0.00335809633712876\\
39.7	0.00335870539994639\\
39.71	0.00335931473177292\\
39.72	0.00335992433270572\\
39.73	0.00336053420284222\\
39.74	0.00336114434227991\\
39.75	0.00336175475111632\\
39.76	0.003362365429449\\
39.77	0.00336297637737561\\
39.78	0.0033635875949938\\
39.79	0.00336419908240131\\
39.8	0.00336481083969593\\
39.81	0.00336542286697549\\
39.82	0.00336603516433788\\
39.83	0.00336664773188104\\
39.84	0.00336726056970297\\
39.85	0.00336787367790172\\
39.86	0.00336848705657541\\
39.87	0.00336910070582221\\
39.88	0.00336971462574034\\
39.89	0.0033703288164281\\
39.9	0.00337094327798384\\
39.91	0.00337155801050595\\
39.92	0.00337217301409291\\
39.93	0.00337278828884325\\
39.94	0.00337340383485556\\
39.95	0.00337401965222851\\
39.96	0.00337463574106082\\
39.97	0.00337525210145128\\
39.98	0.00337586873349875\\
39.99	0.00337648563730213\\
40	0.00337710281296044\\
40.01	0.00337772026057272\\
};
\addplot [color=green,solid,forget plot]
  table[row sep=crcr]{%
40.01	0.00337772026057272\\
40.02	0.00337833798023811\\
40.03	0.00337895597205581\\
40.04	0.00337957423612509\\
40.05	0.0033801927725453\\
40.06	0.00338081158141584\\
40.07	0.00338143066283621\\
40.08	0.00338205001690597\\
40.09	0.00338266964372478\\
40.1	0.00338328954339234\\
40.11	0.00338390971600846\\
40.12	0.00338453016167301\\
40.13	0.00338515088048594\\
40.14	0.00338577187254729\\
40.15	0.00338639313795718\\
40.16	0.0033870146768158\\
40.17	0.00338763648922343\\
40.18	0.00338825857528045\\
40.19	0.0033888809350873\\
40.2	0.00338950356874453\\
40.21	0.00339012647635274\\
40.22	0.00339074965801267\\
40.23	0.00339137311382511\\
40.24	0.00339199684389096\\
40.25	0.00339262084831119\\
40.26	0.00339324512718688\\
40.27	0.00339386968061921\\
40.28	0.00339449450870943\\
40.29	0.0033951196115589\\
40.3	0.00339574498926908\\
40.31	0.00339637064194152\\
40.32	0.00339699656967787\\
40.33	0.00339762277257988\\
40.34	0.00339824925074939\\
40.35	0.00339887600428837\\
40.36	0.00339950303329886\\
40.37	0.00340013033788303\\
40.38	0.00340075791814314\\
40.39	0.00340138577418155\\
40.4	0.00340201390610075\\
40.41	0.00340264231400331\\
40.42	0.00340327099799195\\
40.43	0.00340389995816945\\
40.44	0.00340452919463874\\
40.45	0.00340515870750285\\
40.46	0.00340578849686492\\
40.47	0.0034064185628282\\
40.48	0.00340704890549608\\
40.49	0.00340767952497206\\
40.5	0.00340831042135973\\
40.51	0.00340894159476284\\
40.52	0.00340957304528524\\
40.53	0.0034102047730309\\
40.54	0.00341083677810394\\
40.55	0.00341146906060855\\
40.56	0.00341210162064911\\
40.57	0.00341273445833009\\
40.58	0.0034133675737561\\
40.59	0.00341400096703187\\
40.6	0.00341463463826229\\
40.61	0.00341526858755234\\
40.62	0.00341590281500717\\
40.63	0.00341653732073205\\
40.64	0.00341717210483238\\
40.65	0.00341780716741371\\
40.66	0.00341844250858173\\
40.67	0.00341907812844226\\
40.68	0.00341971402710127\\
40.69	0.00342035020466488\\
40.7	0.00342098666123933\\
40.71	0.00342162339693105\\
40.72	0.00342226041184657\\
40.73	0.00342289770610784\\
40.74	0.00342353527984559\\
40.75	0.00342417313319057\\
40.76	0.00342481126627354\\
40.77	0.00342544967922531\\
40.78	0.00342608837217668\\
40.79	0.00342672734525849\\
40.8	0.00342736659860159\\
40.81	0.00342800613233685\\
40.82	0.00342864594659516\\
40.83	0.00342928604150743\\
40.84	0.00342992641720457\\
40.85	0.00343056707381753\\
40.86	0.00343120801147729\\
40.87	0.00343184923031482\\
40.88	0.0034324907304611\\
40.89	0.00343313251204717\\
40.9	0.00343377457520405\\
40.91	0.00343441692006279\\
40.92	0.00343505954675447\\
40.93	0.00343570245541015\\
40.94	0.00343634564616093\\
40.95	0.00343698911913794\\
40.96	0.0034376328744723\\
40.97	0.00343827691229516\\
40.98	0.00343892123273768\\
40.99	0.00343956583593104\\
41	0.00344021072200642\\
41.01	0.00344085589109503\\
41.02	0.00344150134332809\\
41.03	0.00344214707883684\\
41.04	0.00344279309775252\\
41.05	0.0034434394002064\\
41.06	0.00344408598632975\\
41.07	0.00344473285625384\\
41.08	0.00344538001011\\
41.09	0.00344602744802953\\
41.1	0.00344667517014376\\
41.11	0.00344732317658403\\
41.12	0.00344797146748168\\
41.13	0.00344862004296808\\
41.14	0.0034492689031746\\
41.15	0.00344991804823262\\
41.16	0.00345056747827355\\
41.17	0.00345121719342878\\
41.18	0.00345186719382973\\
41.19	0.00345251747960784\\
41.2	0.00345316805089453\\
41.21	0.00345381890782126\\
41.22	0.00345447005051948\\
41.23	0.00345512147912065\\
41.24	0.00345577319375625\\
41.25	0.00345642519455776\\
41.26	0.00345707748165667\\
41.27	0.00345773005518448\\
41.28	0.0034583829152727\\
41.29	0.00345903606205283\\
41.3	0.00345968949565641\\
41.31	0.00346034321621497\\
41.32	0.00346099722386002\\
41.33	0.00346165151872313\\
41.34	0.00346230610093583\\
41.35	0.00346296097062969\\
41.36	0.00346361612793625\\
41.37	0.0034642715729871\\
41.38	0.00346492730591379\\
41.39	0.0034655833268479\\
41.4	0.00346623963592101\\
41.41	0.0034668962332647\\
41.42	0.00346755311901058\\
41.43	0.00346821029329022\\
41.44	0.00346886775623522\\
41.45	0.00346952550797718\\
41.46	0.00347018354864771\\
41.47	0.00347084187837839\\
41.48	0.00347150049730084\\
41.49	0.00347215940554668\\
41.5	0.0034728186032475\\
41.51	0.00347347809053493\\
41.52	0.00347413786754057\\
41.53	0.00347479793439603\\
41.54	0.00347545829123294\\
41.55	0.0034761189381829\\
41.56	0.00347677987537753\\
41.57	0.00347744110294844\\
41.58	0.00347810262102725\\
41.59	0.00347876442974556\\
41.6	0.003479426529235\\
41.61	0.00348008891962717\\
41.62	0.00348075160105367\\
41.63	0.00348141457364611\\
41.64	0.0034820778375361\\
41.65	0.00348274139285523\\
41.66	0.00348340523973509\\
41.67	0.0034840693783073\\
41.68	0.00348473380870343\\
41.69	0.00348539853105508\\
41.7	0.00348606354549381\\
41.71	0.00348672885215121\\
41.72	0.00348739445115884\\
41.73	0.00348806034264829\\
41.74	0.00348872652675111\\
41.75	0.00348939300359884\\
41.76	0.00349005977332304\\
41.77	0.00349072683605526\\
41.78	0.00349139419192702\\
41.79	0.00349206184106986\\
41.8	0.0034927297836153\\
41.81	0.00349339801969484\\
41.82	0.00349406654943998\\
41.83	0.00349473537298224\\
41.84	0.00349540449045309\\
41.85	0.003496073901984\\
41.86	0.00349674360770645\\
41.87	0.0034974136077519\\
41.88	0.00349808390225178\\
41.89	0.00349875449133754\\
41.9	0.0034994253751406\\
41.91	0.00350009655379238\\
41.92	0.00350076802742427\\
41.93	0.00350143979616768\\
41.94	0.00350211186015396\\
41.95	0.0035027842195145\\
41.96	0.00350345687438065\\
41.97	0.00350412982488373\\
41.98	0.00350480307115508\\
41.99	0.00350547661332602\\
42	0.00350615045152782\\
42.01	0.00350682458589178\\
42.02	0.00350749901654916\\
42.03	0.0035081737436312\\
42.04	0.00350884876726916\\
42.05	0.00350952408759424\\
42.06	0.00351019970473764\\
42.07	0.00351087561883055\\
42.08	0.00351155183000414\\
42.09	0.00351222833838955\\
42.1	0.00351290514411792\\
42.11	0.00351358224732035\\
42.12	0.00351425964812795\\
42.13	0.00351493734667178\\
42.14	0.0035156153430829\\
42.15	0.00351629363749233\\
42.16	0.00351697223003111\\
42.17	0.00351765112083021\\
42.18	0.00351833031002061\\
42.19	0.00351900979773326\\
42.2	0.00351968958409909\\
42.21	0.00352036966924901\\
42.22	0.00352105005331388\\
42.23	0.00352173073642457\\
42.24	0.00352241171871193\\
42.25	0.00352309300030676\\
42.26	0.00352377458133984\\
42.27	0.00352445646194195\\
42.28	0.00352513864224381\\
42.29	0.00352582112237614\\
42.3	0.00352650390246963\\
42.31	0.00352718698265493\\
42.32	0.00352787036306267\\
42.33	0.00352855404382347\\
42.34	0.00352923802506789\\
42.35	0.00352992230692649\\
42.36	0.00353060688952977\\
42.37	0.00353129177300825\\
42.38	0.00353197695749238\\
42.39	0.00353266244311258\\
42.4	0.00353334822999928\\
42.41	0.00353403431828282\\
42.42	0.00353472070809356\\
42.43	0.0035354073995618\\
42.44	0.00353609439281783\\
42.45	0.00353678168799188\\
42.46	0.00353746928521417\\
42.47	0.00353815718461488\\
42.48	0.00353884538632415\\
42.49	0.00353953389047208\\
42.5	0.00354022269718877\\
42.51	0.00354091180660424\\
42.52	0.00354160121884851\\
42.53	0.00354229093405154\\
42.54	0.00354298095234327\\
42.55	0.0035436712738536\\
42.56	0.00354436189871238\\
42.57	0.00354505282704944\\
42.58	0.00354574405899455\\
42.59	0.00354643559467747\\
42.6	0.00354712743422789\\
42.61	0.0035478195777755\\
42.62	0.0035485120254499\\
42.63	0.0035492047773807\\
42.64	0.00354989783369743\\
42.65	0.0035505911945296\\
42.66	0.00355128486000666\\
42.67	0.00355197883025805\\
42.68	0.00355267310541312\\
42.69	0.00355336768560122\\
42.7	0.00355406257095164\\
42.71	0.00355475776159362\\
42.72	0.00355545325765636\\
42.73	0.00355614905926902\\
42.74	0.00355684516656071\\
42.75	0.00355754157966048\\
42.76	0.00355823829869735\\
42.77	0.0035589353238003\\
42.78	0.00355963265509824\\
42.79	0.00356033029272005\\
42.8	0.00356102823679455\\
42.81	0.00356172648745052\\
42.82	0.00356242504481669\\
42.83	0.00356312390902172\\
42.84	0.00356382308019425\\
42.85	0.00356452255846284\\
42.86	0.00356522234395602\\
42.87	0.00356592243680226\\
42.88	0.00356662283712997\\
42.89	0.00356732354506752\\
42.9	0.0035680245607432\\
42.91	0.0035687258842853\\
42.92	0.003569427515822\\
42.93	0.00357012945548144\\
42.94	0.00357083170339171\\
42.95	0.00357153425968085\\
42.96	0.00357223712447684\\
42.97	0.00357294029790759\\
42.98	0.00357364378010096\\
42.99	0.00357434757118476\\
43	0.00357505167128672\\
43.01	0.00357575608053452\\
43.02	0.00357646079905579\\
43.03	0.00357716582697809\\
43.04	0.00357787116442893\\
43.05	0.00357857681153573\\
43.06	0.00357928276842587\\
43.07	0.00357998903522667\\
43.08	0.00358069561206538\\
43.09	0.00358140249906919\\
43.1	0.00358210969636519\\
43.11	0.00358281720408047\\
43.12	0.00358352502234201\\
43.13	0.00358423315127673\\
43.14	0.00358494159101148\\
43.15	0.00358565034167306\\
43.16	0.00358635940338818\\
43.17	0.00358706877628349\\
43.18	0.00358777846048559\\
43.19	0.00358848845612098\\
43.2	0.0035891987633161\\
43.21	0.00358990938219732\\
43.22	0.00359062031289095\\
43.23	0.00359133155552321\\
43.24	0.00359204311022025\\
43.25	0.00359275497710816\\
43.26	0.00359346715631294\\
43.27	0.00359417964796054\\
43.28	0.00359489245217678\\
43.29	0.00359560556908748\\
43.3	0.00359631899881832\\
43.31	0.00359703274149493\\
43.32	0.00359774679724287\\
43.33	0.00359846116618762\\
43.34	0.00359917584845455\\
43.35	0.00359989084416899\\
43.36	0.00360060615345617\\
43.37	0.00360132177644123\\
43.38	0.00360203771324925\\
43.39	0.00360275396400522\\
43.4	0.00360347052883405\\
43.41	0.00360418740786056\\
43.42	0.00360490460120948\\
43.43	0.00360562210900548\\
43.44	0.00360633993137312\\
43.45	0.00360705806843688\\
43.46	0.00360777652032116\\
43.47	0.00360849528715027\\
43.48	0.00360921436904842\\
43.49	0.00360993376613977\\
43.5	0.00361065347854834\\
43.51	0.00361137350639809\\
43.52	0.00361209384981288\\
43.53	0.0036128145089165\\
43.54	0.00361353548383261\\
43.55	0.0036142567746848\\
43.56	0.00361497838159657\\
43.57	0.00361570030469132\\
43.58	0.00361642254409236\\
43.59	0.00361714509992289\\
43.6	0.00361786797230603\\
43.61	0.0036185911613648\\
43.62	0.00361931466722212\\
43.63	0.00362003849000082\\
43.64	0.0036207626298236\\
43.65	0.00362148708681311\\
43.66	0.00362221186109187\\
43.67	0.0036229369527823\\
43.68	0.00362366236200672\\
43.69	0.00362438808888735\\
43.7	0.0036251141335463\\
43.71	0.0036258404961056\\
43.72	0.00362656717668714\\
43.73	0.00362729417541274\\
43.74	0.00362802149240409\\
43.75	0.00362874912778278\\
43.76	0.0036294770816703\\
43.77	0.00363020535418802\\
43.78	0.0036309339454572\\
43.79	0.00363166285559901\\
43.8	0.00363239208473449\\
43.81	0.00363312163298458\\
43.82	0.0036338515004701\\
43.83	0.00363458168731175\\
43.84	0.00363531219363016\\
43.85	0.00363604301954578\\
43.86	0.003636774165179\\
43.87	0.00363750563065008\\
43.88	0.00363823741607914\\
43.89	0.0036389695215862\\
43.9	0.00363970194729117\\
43.91	0.00364043469331383\\
43.92	0.00364116775977386\\
43.93	0.00364190114679077\\
43.94	0.003642634854484\\
43.95	0.00364336888297286\\
43.96	0.00364410323237652\\
43.97	0.00364483790281404\\
43.98	0.00364557289440435\\
43.99	0.00364630820726625\\
44	0.00364704384151842\\
44.01	0.00364777979727941\\
44.02	0.00364851607466766\\
44.03	0.00364925267380146\\
44.04	0.00364998959479897\\
44.05	0.00365072683777824\\
44.06	0.00365146440285718\\
44.07	0.00365220229015355\\
44.08	0.00365294049978501\\
44.09	0.00365367903186905\\
44.1	0.00365441788652307\\
44.11	0.0036551570638643\\
44.12	0.00365589656400985\\
44.13	0.00365663638707669\\
44.14	0.00365737653318165\\
44.15	0.00365811700244142\\
44.16	0.00365885779497256\\
44.17	0.00365959891089148\\
44.18	0.00366034035031446\\
44.19	0.00366108211335762\\
44.2	0.00366182420013696\\
44.21	0.00366256661076833\\
44.22	0.00366330934536742\\
44.23	0.00366405240404979\\
44.24	0.00366479578693085\\
44.25	0.00366553949412587\\
44.26	0.00366628352574995\\
44.27	0.00366702788191806\\
44.28	0.00366777256274503\\
44.29	0.00366851756834552\\
44.3	0.00366926289883403\\
44.31	0.00367000855432494\\
44.32	0.00367075453493245\\
44.33	0.00367150084077061\\
44.34	0.00367224747195334\\
44.35	0.00367299442859436\\
44.36	0.00367374171080727\\
44.37	0.0036744893187055\\
44.38	0.00367523725240231\\
44.39	0.0036759855120108\\
44.4	0.00367673409764394\\
44.41	0.00367748300941451\\
44.42	0.00367823224743514\\
44.43	0.00367898181181828\\
44.44	0.00367973170267622\\
44.45	0.0036804819201211\\
44.46	0.0036812324642649\\
44.47	0.00368198333521939\\
44.48	0.00368273453309622\\
44.49	0.00368348605800684\\
44.5	0.00368423791006253\\
44.51	0.00368499008937441\\
44.52	0.00368574259605343\\
44.53	0.00368649543021037\\
44.54	0.00368724859195582\\
44.55	0.0036880020814002\\
44.56	0.00368875589865376\\
44.57	0.00368951004382655\\
44.58	0.00369026451702849\\
44.59	0.00369101931836927\\
44.6	0.00369177444795842\\
44.61	0.0036925299059053\\
44.62	0.00369328569231907\\
44.63	0.00369404180730872\\
44.64	0.00369479825098304\\
44.65	0.00369555502345064\\
44.66	0.00369631212481996\\
44.67	0.00369706955519922\\
44.68	0.00369782731469649\\
44.69	0.00369858540341962\\
44.7	0.00369934382147628\\
44.71	0.00370010256897394\\
44.72	0.0037008616460199\\
44.73	0.00370162105272125\\
44.74	0.00370238078918488\\
44.75	0.00370314085551748\\
44.76	0.00370390125182558\\
44.77	0.00370466197821546\\
44.78	0.00370542303479324\\
44.79	0.00370618442166482\\
44.8	0.00370694613893589\\
44.81	0.00370770818671197\\
44.82	0.00370847056509835\\
44.83	0.00370923327420011\\
44.84	0.00370999631412216\\
44.85	0.00371075968496916\\
44.86	0.00371152338684558\\
44.87	0.00371228741985569\\
44.88	0.00371305178410354\\
44.89	0.00371381647969296\\
44.9	0.00371458150672759\\
44.91	0.00371534686531082\\
44.92	0.00371611255554587\\
44.93	0.00371687857753571\\
44.94	0.00371764493138311\\
44.95	0.0037184116171906\\
44.96	0.00371917863506053\\
44.97	0.00371994598509498\\
44.98	0.00372071366739585\\
44.99	0.00372148168206479\\
45	0.00372225002920323\\
45.01	0.0037230187089124\\
45.02	0.00372378772129326\\
45.03	0.00372455706644658\\
45.04	0.00372532674447287\\
45.05	0.00372609675547243\\
45.06	0.00372686709954533\\
45.07	0.00372763777679139\\
45.08	0.00372840878731021\\
45.09	0.00372918013120116\\
45.1	0.00372995180856334\\
45.11	0.00373072381949566\\
45.12	0.00373149616409675\\
45.13	0.00373226884246503\\
45.14	0.00373304185469867\\
45.15	0.00373381520089557\\
45.16	0.00373458888115342\\
45.17	0.00373536289556965\\
45.18	0.00373613724424145\\
45.19	0.00373691192726577\\
45.2	0.00373768694473928\\
45.21	0.00373846229675844\\
45.22	0.00373923798341941\\
45.23	0.00374001400481814\\
45.24	0.00374079036105031\\
45.25	0.00374156705221134\\
45.26	0.00374234407839641\\
45.27	0.00374312143970041\\
45.28	0.003743899136218\\
45.29	0.00374467716804357\\
45.3	0.00374545553527125\\
45.31	0.00374623423799489\\
45.32	0.00374701327630809\\
45.33	0.0037477926503042\\
45.34	0.00374857236007627\\
45.35	0.00374935240571708\\
45.36	0.00375013278731918\\
45.37	0.00375091350497481\\
45.38	0.00375169455877594\\
45.39	0.0037524759488143\\
45.4	0.00375325767518131\\
45.41	0.00375403973796811\\
45.42	0.00375482213726559\\
45.43	0.00375560487316434\\
45.44	0.00375638794575466\\
45.45	0.0037571713551266\\
45.46	0.00375795510136989\\
45.47	0.00375873918457399\\
45.48	0.00375952360482808\\
45.49	0.00376030836222103\\
45.5	0.00376109345684145\\
45.51	0.00376187888877764\\
45.52	0.00376266465811761\\
45.53	0.00376345076494906\\
45.54	0.00376423720935943\\
45.55	0.00376502399143583\\
45.56	0.00376581111126509\\
45.57	0.00376659856893373\\
45.58	0.00376738636452797\\
45.59	0.00376817449813373\\
45.6	0.00376896296983663\\
45.61	0.00376975177972197\\
45.62	0.00377054092787475\\
45.63	0.00377133041437967\\
45.64	0.00377212023932111\\
45.65	0.00377291040278312\\
45.66	0.00377370090484948\\
45.67	0.0037744917456036\\
45.68	0.00377528292512864\\
45.69	0.00377607444350737\\
45.7	0.00377686630082229\\
45.71	0.00377765849715556\\
45.72	0.00377845103258902\\
45.73	0.00377924390720419\\
45.74	0.00378003712108225\\
45.75	0.00378083067430406\\
45.76	0.00378162456695016\\
45.77	0.00378241879910076\\
45.78	0.00378321337083571\\
45.79	0.00378400828223455\\
45.8	0.00378480353337648\\
45.81	0.00378559912434036\\
45.82	0.00378639505520472\\
45.83	0.00378719132604773\\
45.84	0.00378798793694724\\
45.85	0.00378878488798074\\
45.86	0.00378958217922538\\
45.87	0.00379037981075797\\
45.88	0.00379117778265495\\
45.89	0.00379197609499244\\
45.9	0.00379277474784618\\
45.91	0.00379357374129157\\
45.92	0.00379437307540365\\
45.93	0.00379517275025711\\
45.94	0.00379597276592627\\
45.95	0.00379677312248509\\
45.96	0.00379757382000719\\
45.97	0.0037983748585658\\
45.98	0.00379917623823379\\
45.99	0.00379997795908366\\
46	0.00380078002118757\\
46.01	0.00380158242461726\\
46.02	0.00380238516944413\\
46.03	0.0038031882557392\\
46.04	0.00380399168357311\\
46.05	0.00380479545301613\\
46.06	0.00380559956413813\\
46.07	0.00380640401700862\\
46.08	0.00380720881169672\\
46.09	0.00380801394827116\\
46.1	0.00380881942680029\\
46.11	0.00380962524735206\\
46.12	0.00381043140999404\\
46.13	0.00381123791479341\\
46.14	0.00381204476181694\\
46.15	0.00381285195113103\\
46.16	0.00381365948280164\\
46.17	0.00381446735689438\\
46.18	0.00381527557347442\\
46.19	0.00381608413260654\\
46.2	0.00381689303435512\\
46.21	0.00381770227878413\\
46.22	0.00381851186595712\\
46.23	0.00381932179593724\\
46.24	0.00382013206878723\\
46.25	0.0038209426845694\\
46.26	0.00382175364334567\\
46.27	0.0038225649451775\\
46.28	0.00382337659012597\\
46.29	0.0038241885782517\\
46.3	0.00382500090961493\\
46.31	0.00382581358427543\\
46.32	0.00382662660229258\\
46.33	0.0038274399637253\\
46.34	0.00382825366863207\\
46.35	0.00382906771707098\\
46.36	0.00382988210909964\\
46.37	0.00383069684477524\\
46.38	0.00383151192415453\\
46.39	0.00383232734729381\\
46.4	0.00383314311424895\\
46.41	0.00383395922507535\\
46.42	0.00383477567982798\\
46.43	0.00383559247856136\\
46.44	0.00383640962132956\\
46.45	0.00383722710818615\\
46.46	0.00383804493918432\\
46.47	0.00383886311437674\\
46.48	0.00383968163381566\\
46.49	0.00384050049755282\\
46.5	0.00384131970563953\\
46.51	0.00384213925812664\\
46.52	0.0038429591550645\\
46.53	0.00384377939650302\\
46.54	0.00384459998249159\\
46.55	0.00384542091307918\\
46.56	0.00384624218831425\\
46.57	0.00384706380824477\\
46.58	0.00384788577291827\\
46.59	0.00384870808238175\\
46.6	0.00384953073668175\\
46.61	0.00385035373586431\\
46.62	0.00385117707997498\\
46.63	0.00385200076905882\\
46.64	0.0038528248031604\\
46.65	0.00385364918232379\\
46.66	0.00385447390659254\\
46.67	0.00385529897600972\\
46.68	0.0038561243906179\\
46.69	0.00385695015045912\\
46.7	0.00385777625557494\\
46.71	0.00385860270600637\\
46.72	0.00385942950179394\\
46.73	0.00386025664297765\\
46.74	0.00386108412959699\\
46.75	0.00386191196169092\\
46.76	0.00386274013929788\\
46.77	0.0038635686624558\\
46.78	0.00386439753120205\\
46.79	0.00386522674557351\\
46.8	0.00386605630560649\\
46.81	0.0038668862113368\\
46.82	0.00386771646279967\\
46.83	0.00386854706002985\\
46.84	0.00386937800306149\\
46.85	0.00387020929192823\\
46.86	0.00387104092666316\\
46.87	0.00387187290729881\\
46.88	0.00387270523386716\\
46.89	0.00387353790639965\\
46.9	0.00387437092492714\\
46.91	0.00387520428947997\\
46.92	0.00387603800008787\\
46.93	0.00387687205678005\\
46.94	0.00387770645958512\\
46.95	0.00387854120853116\\
46.96	0.00387937630364563\\
46.97	0.00388021174495546\\
46.98	0.00388104753248698\\
46.99	0.00388188366626594\\
47	0.00388272014631753\\
47.01	0.00388355697266633\\
47.02	0.00388439414533636\\
47.03	0.00388523166435103\\
47.04	0.00388606952973317\\
47.05	0.003886907741505\\
47.06	0.00388774629968817\\
47.07	0.00388858520430372\\
47.08	0.00388942445537207\\
47.09	0.00389026405291305\\
47.1	0.0038911039969459\\
47.11	0.00389194428748922\\
47.12	0.00389278492456102\\
47.13	0.00389362590817868\\
47.14	0.00389446723835896\\
47.15	0.00389530891511803\\
47.16	0.0038961509384714\\
47.17	0.00389699330843397\\
47.18	0.00389783602502002\\
47.19	0.00389867908824317\\
47.2	0.00389952249811644\\
47.21	0.00390036625465219\\
47.22	0.00390121035786216\\
47.23	0.00390205480775743\\
47.24	0.00390289960434843\\
47.25	0.00390374474764497\\
47.26	0.00390459023765619\\
47.27	0.00390543607439058\\
47.28	0.00390628225785596\\
47.29	0.00390712878805952\\
47.3	0.00390797566500776\\
47.31	0.00390882288870654\\
47.32	0.00390967045916103\\
47.33	0.00391051837637575\\
47.34	0.00391136664035453\\
47.35	0.00391221525110052\\
47.36	0.0039130642086162\\
47.37	0.00391391351290338\\
47.38	0.00391476316396317\\
47.39	0.00391561316179598\\
47.4	0.00391646350640156\\
47.41	0.00391731419777894\\
47.42	0.00391816523592647\\
47.43	0.00391901662084178\\
47.44	0.00391986835252182\\
47.45	0.00392072043096282\\
47.46	0.00392157285616029\\
47.47	0.00392242562810907\\
47.48	0.00392327874680325\\
47.49	0.00392413221223619\\
47.5	0.00392498602440056\\
47.51	0.0039258401832883\\
47.52	0.00392669468889061\\
47.53	0.00392754954119796\\
47.54	0.00392840474020011\\
47.55	0.00392926028588605\\
47.56	0.00393011617824404\\
47.57	0.00393097241726162\\
47.58	0.00393182900292555\\
47.59	0.00393268593522187\\
47.6	0.00393354321413585\\
47.61	0.00393440083965201\\
47.62	0.00393525881175411\\
47.63	0.00393611713042514\\
47.64	0.00393697579564734\\
47.65	0.00393783480740217\\
47.66	0.00393869416567033\\
47.67	0.00393955387043173\\
47.68	0.0039404139216655\\
47.69	0.00394127431935001\\
47.7	0.00394213506346282\\
47.71	0.00394299615398071\\
47.72	0.00394385759087969\\
47.73	0.00394471937413494\\
47.74	0.00394558150372087\\
47.75	0.00394644397961106\\
47.76	0.00394730680177831\\
47.77	0.0039481699701946\\
47.78	0.0039490334848311\\
47.79	0.00394989734565817\\
47.8	0.00395076155264534\\
47.81	0.00395162610576132\\
47.82	0.003952491004974\\
47.83	0.00395335625025044\\
47.84	0.00395422184155686\\
47.85	0.00395508777885866\\
47.86	0.00395595406212038\\
47.87	0.00395682069130573\\
47.88	0.00395768766637757\\
47.89	0.00395855498729791\\
47.9	0.0039594226540279\\
47.91	0.00396029066652784\\
47.92	0.00396115902475717\\
47.93	0.00396202772867445\\
47.94	0.00396289677823741\\
47.95	0.00396376617340286\\
47.96	0.00396463591412675\\
47.97	0.00396550600036417\\
47.98	0.00396637643206933\\
47.99	0.0039672472091955\\
48	0.00396811833169512\\
48.01	0.00396898979951969\\
48.02	0.00396986161261986\\
48.03	0.00397073377094535\\
48.04	0.00397160627444496\\
48.05	0.00397247912306661\\
48.06	0.00397335231675729\\
48.07	0.00397422585546309\\
48.08	0.00397509973912916\\
48.09	0.00397597396769973\\
48.1	0.00397684854111811\\
48.11	0.00397772345932668\\
48.12	0.00397859872226686\\
48.13	0.00397947432987917\\
48.14	0.00398035028210315\\
48.15	0.00398122657887741\\
48.16	0.00398210322013961\\
48.17	0.00398298020582644\\
48.18	0.00398385753587364\\
48.19	0.00398473521021599\\
48.2	0.00398561322878729\\
48.21	0.00398649159152039\\
48.22	0.00398737029834714\\
48.23	0.00398824934919843\\
48.24	0.00398912874400416\\
48.25	0.00399000848269323\\
48.26	0.00399088856519356\\
48.27	0.00399176899143208\\
48.28	0.00399264976133472\\
48.29	0.00399353087482637\\
48.3	0.00399441233183098\\
48.31	0.00399529413227141\\
48.32	0.00399617627606956\\
48.33	0.00399705876314629\\
48.34	0.00399794159342142\\
48.35	0.00399882476681377\\
48.36	0.00399970828324112\\
48.37	0.00400059214262018\\
48.38	0.00400147634486667\\
48.39	0.00400236088989521\\
48.4	0.00400324577761942\\
48.41	0.00400413100795181\\
48.42	0.00400501658080389\\
48.43	0.00400590249608606\\
48.44	0.00400678875370768\\
48.45	0.00400767535357702\\
48.46	0.00400856229560127\\
48.47	0.00400944957968657\\
48.48	0.00401033720573794\\
48.49	0.00401122517365932\\
48.5	0.00401211348335356\\
48.51	0.0040130021347224\\
48.52	0.0040138911276665\\
48.53	0.00401478046208539\\
48.54	0.00401567013787749\\
48.55	0.00401656015494012\\
48.56	0.00401745051316944\\
48.57	0.00401834121246053\\
48.58	0.0040192322527073\\
48.59	0.00402012363380256\\
48.6	0.00402101535563795\\
48.61	0.00402190741810397\\
48.62	0.00402279982108997\\
48.63	0.00402369256448418\\
48.64	0.00402458564817362\\
48.65	0.00402547907204416\\
48.66	0.00402637283598054\\
48.67	0.00402726693986626\\
48.68	0.0040281613835837\\
48.69	0.00402905616701403\\
48.7	0.00402995129003724\\
48.71	0.00403084675253212\\
48.72	0.00403174255437627\\
48.73	0.00403263869544609\\
48.74	0.00403353517561675\\
48.75	0.00403443199476223\\
48.76	0.00403532915275529\\
48.77	0.00403622664946747\\
48.78	0.00403712448476908\\
48.79	0.00403802265852918\\
48.8	0.00403892117061563\\
48.81	0.00403982002089501\\
48.82	0.00404071920923267\\
48.83	0.00404161873549271\\
48.84	0.00404251859953798\\
48.85	0.00404341880123005\\
48.86	0.00404431934042922\\
48.87	0.00404522021699455\\
48.88	0.00404612143078377\\
48.89	0.00404702298165337\\
48.9	0.00404792486945855\\
48.91	0.00404882709405318\\
48.92	0.00404972965528987\\
48.93	0.00405063255301991\\
48.94	0.00405153578709327\\
48.95	0.00405243935735863\\
48.96	0.00405334326366333\\
48.97	0.00405424750585339\\
48.98	0.00405515208377352\\
48.99	0.00405605699726704\\
49	0.004056962246176\\
49.01	0.00405786783034105\\
49.02	0.00405877374960149\\
49.03	0.00405968000379531\\
49.04	0.00406058659275907\\
49.05	0.00406149351632801\\
49.06	0.00406240077433598\\
49.07	0.00406330836661545\\
49.08	0.0040642162929975\\
49.09	0.00406512455331183\\
49.1	0.00406603314738673\\
49.11	0.00406694207504909\\
49.12	0.0040678513361244\\
49.13	0.00406876093043674\\
49.14	0.00406967085780876\\
49.15	0.00407058111806168\\
49.16	0.00407149171101531\\
49.17	0.00407240263648801\\
49.18	0.00407331389429669\\
49.19	0.00407422548425682\\
49.2	0.00407513740618243\\
49.21	0.00407604965988606\\
49.22	0.00407696224517881\\
49.23	0.0040778751618703\\
49.24	0.00407878840976866\\
49.25	0.00407970198868057\\
49.26	0.00408061589841117\\
49.27	0.00408153013876416\\
49.28	0.00408244470954169\\
49.29	0.00408335961054444\\
49.3	0.00408427484157156\\
49.31	0.00408519040242068\\
49.32	0.00408610629288789\\
49.33	0.00408702251276778\\
49.34	0.00408793906185338\\
49.35	0.00408885593993619\\
49.36	0.00408977314680613\\
49.37	0.00409069068225161\\
49.38	0.00409160854605943\\
49.39	0.00409252673801485\\
49.4	0.00409344525790155\\
49.41	0.00409436410550162\\
49.42	0.00409528328059557\\
49.43	0.00409620278296231\\
49.44	0.00409712261237916\\
49.45	0.00409804276862182\\
49.46	0.00409896325146439\\
49.47	0.00409988406067932\\
49.48	0.00410080519603748\\
49.49	0.00410172665730806\\
49.5	0.00410264844425865\\
49.51	0.00410357055665517\\
49.52	0.00410449299426188\\
49.53	0.00410541575684141\\
49.54	0.00410633884415471\\
49.55	0.00410726225596104\\
49.56	0.00410818599201801\\
49.57	0.00410911005208152\\
49.58	0.0041100344359058\\
49.59	0.00411095914324335\\
49.6	0.00411188417384498\\
49.61	0.00411280952745981\\
49.62	0.00411373520383521\\
49.63	0.00411466120271683\\
49.64	0.00411558752384858\\
49.65	0.00411651416697265\\
49.66	0.00411744113182946\\
49.67	0.00411836841815769\\
49.68	0.00411929602569425\\
49.69	0.00412022395417429\\
49.7	0.00412115220333119\\
49.71	0.00412208077289651\\
49.72	0.00412300966260007\\
49.73	0.00412393887216986\\
49.74	0.00412486840133208\\
49.75	0.00412579824981111\\
49.76	0.00412672841732952\\
49.77	0.00412765890360806\\
49.78	0.00412858970836563\\
49.79	0.00412952083131931\\
49.8	0.0041304522721843\\
49.81	0.00413138403067398\\
49.82	0.00413231610649985\\
49.83	0.00413324849937154\\
49.84	0.00413418120899682\\
49.85	0.00413511423508155\\
49.86	0.00413604757732972\\
49.87	0.00413698123544339\\
49.88	0.00413791520912277\\
49.89	0.00413884949806608\\
49.9	0.00413978410196969\\
49.91	0.00414071902052799\\
49.92	0.00414165425343346\\
49.93	0.00414258980037661\\
49.94	0.00414352566104603\\
49.95	0.00414446183512832\\
49.96	0.00414539832230811\\
49.97	0.0041463351222681\\
49.98	0.00414727223468895\\
49.99	0.00414820965924936\\
50	0.004149147395626\\
50.01	0.00415008544349357\\
50.02	0.00415102380252474\\
50.03	0.00415196247239014\\
50.04	0.0041529014527584\\
50.05	0.00415384074329608\\
50.06	0.00415478034366771\\
50.07	0.00415572025353575\\
50.08	0.00415666047256062\\
50.09	0.00415760100040063\\
50.1	0.00415854183671204\\
50.11	0.00415948298114902\\
50.12	0.00416042443336364\\
50.13	0.00416136619300585\\
50.14	0.0041623082597235\\
50.15	0.00416325063316231\\
50.16	0.00416419331296589\\
50.17	0.00416513629877569\\
50.18	0.00416607959023102\\
50.19	0.00416702318696903\\
50.2	0.00416796708862471\\
50.21	0.00416891129483087\\
50.22	0.00416985580521815\\
50.23	0.00417080061941501\\
50.24	0.00417174573704768\\
50.25	0.00417269115774022\\
50.26	0.00417363688111443\\
50.27	0.00417458290678991\\
50.28	0.00417552923438405\\
50.29	0.00417647586351196\\
50.3	0.0041774227937865\\
50.31	0.0041783700248183\\
50.32	0.0041793175562157\\
50.33	0.00418026538758477\\
50.34	0.00418121351852927\\
50.35	0.0041821619486507\\
50.36	0.00418311067754824\\
50.37	0.00418405970481874\\
50.38	0.00418500903005675\\
50.39	0.00418595865285448\\
50.4	0.00418690857280179\\
50.41	0.0041878587894862\\
50.42	0.00418880930249288\\
50.43	0.00418976011140461\\
50.44	0.00419071121580179\\
50.45	0.00419166261526246\\
50.46	0.00419261430936222\\
50.47	0.0041935662976743\\
50.48	0.0041945185797695\\
50.49	0.00419547115521619\\
50.5	0.00419642402358031\\
50.51	0.00419737718442537\\
50.52	0.00419833063731239\\
50.53	0.00419928438179996\\
50.54	0.00420023841744419\\
50.55	0.00420119274379869\\
50.56	0.00420214736041458\\
50.57	0.00420310226684047\\
50.58	0.0042040574626225\\
50.59	0.00420501294730424\\
50.6	0.00420596872042676\\
50.61	0.00420692478152854\\
50.62	0.00420788113014557\\
50.63	0.00420883776581123\\
50.64	0.00420979468805635\\
50.65	0.00421075189640917\\
50.66	0.00421170939039534\\
50.67	0.0042126671695379\\
50.68	0.00421362523335728\\
50.69	0.00421458358137128\\
50.7	0.00421554221309509\\
50.71	0.00421650112804122\\
50.72	0.00421746032571957\\
50.73	0.00421841980563733\\
50.74	0.00421937956729904\\
50.75	0.00422033961020653\\
50.76	0.00422129993385899\\
50.77	0.00422226053775283\\
50.78	0.00422322142138179\\
50.79	0.00422418258423686\\
50.8	0.0042251440258063\\
50.81	0.00422610574557563\\
50.82	0.00422706774302758\\
50.83	0.00422803001764215\\
50.84	0.0042289925688965\\
50.85	0.00422995539626505\\
50.86	0.0042309184992194\\
50.87	0.00423188187722833\\
50.88	0.00423284552975779\\
50.89	0.0042338094562709\\
50.9	0.00423477365622794\\
50.91	0.0042357381290863\\
50.92	0.00423670287430054\\
50.93	0.00423766789132229\\
50.94	0.00423863317960034\\
50.95	0.00423959873858053\\
50.96	0.00424056456770582\\
50.97	0.00424153066641622\\
50.98	0.00424249703414881\\
50.99	0.0042434636703377\\
51	0.00424443057441408\\
51.01	0.00424539774580612\\
51.02	0.00424636518393903\\
51.03	0.00424733288823503\\
51.04	0.00424830085811331\\
51.05	0.00424926909299006\\
51.06	0.00425023759227842\\
51.07	0.0042512063553885\\
51.08	0.00425217538172735\\
51.09	0.00425314467069896\\
51.1	0.0042541142217042\\
51.11	0.00425508403414091\\
51.12	0.00425605410740378\\
51.13	0.0042570244408844\\
51.14	0.00425799503397124\\
51.15	0.00425896588604961\\
51.16	0.00425993699650168\\
51.17	0.00426090836470645\\
51.18	0.00426187999003976\\
51.19	0.00426285187187423\\
51.2	0.0042638240095793\\
51.21	0.00426479640252118\\
51.22	0.00426576905006288\\
51.23	0.00426674195156413\\
51.24	0.00426771510638144\\
51.25	0.00426868851386804\\
51.26	0.00426966217337389\\
51.27	0.00427063608424565\\
51.28	0.00427161024582668\\
51.29	0.00427258465745702\\
51.3	0.00427355931847339\\
51.31	0.00427453422820916\\
51.32	0.00427550938599434\\
51.33	0.00427648479115559\\
51.34	0.00427746044301617\\
51.35	0.00427843634089594\\
51.36	0.00427941248411137\\
51.37	0.00428038887197551\\
51.38	0.00428136550379795\\
51.39	0.00428234237888486\\
51.4	0.00428331949653893\\
51.41	0.00428429685605938\\
51.42	0.00428527445674194\\
51.43	0.00428625229787885\\
51.44	0.00428723037875881\\
51.45	0.004288208698667\\
51.46	0.00428918725688508\\
51.47	0.00429016605269111\\
51.48	0.00429114508535961\\
51.49	0.00429212435416151\\
51.5	0.00429310385836414\\
51.51	0.0042940835972312\\
51.52	0.00429506357002278\\
51.53	0.00429604377599533\\
51.54	0.00429702421440164\\
51.55	0.00429800488449082\\
51.56	0.00429898578550832\\
51.57	0.00429996691669586\\
51.58	0.00430094827729147\\
51.59	0.00430192986652946\\
51.6	0.00430291168364036\\
51.61	0.00430389372785099\\
51.62	0.00430487599838437\\
51.63	0.00430585849445973\\
51.64	0.00430684121529253\\
51.65	0.00430782416009439\\
51.66	0.00430880732807309\\
51.67	0.0043097907184326\\
51.68	0.004310774330373\\
51.69	0.00431175816309051\\
51.7	0.00431274221577746\\
51.71	0.00431372648762226\\
51.72	0.00431471097780941\\
51.73	0.00431569568551948\\
51.74	0.00431668060992907\\
51.75	0.00431766575021084\\
51.76	0.00431865110553345\\
51.77	0.00431963667506156\\
51.78	0.00432062245795583\\
51.79	0.00432160845337289\\
51.8	0.00432259466046531\\
51.81	0.00432358107838161\\
51.82	0.00432456770626624\\
51.83	0.00432555454325956\\
51.84	0.00432654158849782\\
51.85	0.00432752884111311\\
51.86	0.00432851630023345\\
51.87	0.00432950396498264\\
51.88	0.00433049183448035\\
51.89	0.00433147990784205\\
51.9	0.004332468184179\\
51.91	0.00433345666259823\\
51.92	0.00433444534220256\\
51.93	0.00433543422209054\\
51.94	0.00433642330135644\\
51.95	0.00433741257909026\\
51.96	0.00433840205437769\\
51.97	0.00433939172630012\\
51.98	0.00434038159393455\\
51.99	0.00434137165635368\\
52	0.00434236191262583\\
52.01	0.0043433523618149\\
52.02	0.00434434300298042\\
52.03	0.00434533383517747\\
52.04	0.00434632485745672\\
52.05	0.00434731606886437\\
52.06	0.00434830746844215\\
52.07	0.00434929905522728\\
52.08	0.0043502908282525\\
52.09	0.004351282786546\\
52.1	0.00435227492913144\\
52.11	0.00435326725502792\\
52.12	0.00435425976324996\\
52.13	0.00435525245280747\\
52.14	0.00435624532270575\\
52.15	0.00435723837194549\\
52.16	0.0043582315995227\\
52.17	0.00435922500442874\\
52.18	0.00436021858565026\\
52.19	0.00436121234216925\\
52.2	0.00436220627296294\\
52.21	0.00436320037700381\\
52.22	0.00436419465325961\\
52.23	0.00436518910069329\\
52.24	0.00436618371826303\\
52.25	0.00436717850492215\\
52.26	0.00436817345961918\\
52.27	0.00436916858129777\\
52.28	0.00437016386889671\\
52.29	0.0043711593213499\\
52.3	0.00437215493758632\\
52.31	0.00437315071653002\\
52.32	0.00437414665710011\\
52.33	0.00437514275821073\\
52.34	0.00437613901877104\\
52.35	0.00437713543768518\\
52.36	0.00437813201385229\\
52.37	0.00437912874616642\\
52.38	0.00438012563351661\\
52.39	0.00438112267478676\\
52.4	0.00438211986885572\\
52.41	0.00438311721459718\\
52.42	0.0043841147108797\\
52.43	0.00438511235656666\\
52.44	0.00438611015051629\\
52.45	0.0043871080915816\\
52.46	0.00438810617861034\\
52.47	0.00438910441044509\\
52.48	0.00439010278592309\\
52.49	0.00439110130387634\\
52.5	0.00439209996313153\\
52.51	0.00439309876251\\
52.52	0.00439409770082778\\
52.53	0.00439509677689549\\
52.54	0.00439609598951839\\
52.55	0.00439709533749633\\
52.56	0.00439809481962372\\
52.57	0.00439909443468953\\
52.58	0.00440009418147724\\
52.59	0.00440109405876485\\
52.6	0.00440209406532484\\
52.61	0.00440309419992415\\
52.62	0.00440409446132417\\
52.63	0.00440509484828071\\
52.64	0.00440609535954398\\
52.65	0.00440709599385856\\
52.66	0.00440809674996338\\
52.67	0.0044090976265917\\
52.68	0.00441009862247112\\
52.69	0.00441109973632351\\
52.7	0.00441210096686501\\
52.71	0.00441310231280599\\
52.72	0.00441410377285107\\
52.73	0.00441510534569905\\
52.74	0.00441610703004291\\
52.75	0.00441710882456979\\
52.76	0.00441811072796097\\
52.77	0.00441911273889182\\
52.78	0.00442011485603181\\
52.79	0.00442111707804447\\
52.8	0.00442211940358737\\
52.81	0.0044231218313121\\
52.82	0.00442412435986424\\
52.83	0.00442512698788335\\
52.84	0.00442612971400293\\
52.85	0.00442713253685041\\
52.86	0.00442813545504712\\
52.87	0.00442913846720825\\
52.88	0.00443014157194286\\
52.89	0.00443114476785384\\
52.9	0.00443214805353789\\
52.91	0.00443315142758547\\
52.92	0.00443415488858082\\
52.93	0.0044351584351019\\
52.94	0.00443616206572038\\
52.95	0.00443716577900161\\
52.96	0.00443816957350462\\
52.97	0.00443917344778206\\
52.98	0.00444017740038018\\
52.99	0.00444118142983885\\
53	0.00444218553469146\\
53.01	0.00444318971346495\\
53.02	0.00444419396467979\\
53.03	0.00444519828684992\\
53.04	0.00444620267848274\\
53.05	0.00444720713807908\\
53.06	0.00444821166413319\\
53.07	0.0044492162551327\\
53.08	0.0044502209095586\\
53.09	0.00445122562588521\\
53.1	0.00445223040258015\\
53.11	0.00445323523810434\\
53.12	0.00445424013091193\\
53.13	0.00445524507945032\\
53.14	0.00445625008216011\\
53.15	0.00445725513747507\\
53.16	0.0044582602438221\\
53.17	0.00445926539962125\\
53.18	0.00446027060328566\\
53.19	0.00446127585322154\\
53.2	0.00446228114782813\\
53.21	0.0044632864854977\\
53.22	0.00446429186461551\\
53.23	0.00446529728355976\\
53.24	0.0044663027407016\\
53.25	0.0044673082344051\\
53.26	0.00446831376302718\\
53.27	0.00446931932491763\\
53.28	0.00447032491841905\\
53.29	0.00447133054186685\\
53.3	0.00447233619358921\\
53.31	0.00447334187190702\\
53.32	0.00447434757513393\\
53.33	0.00447535330157622\\
53.34	0.00447635904953286\\
53.35	0.00447736481729544\\
53.36	0.00447837060314815\\
53.37	0.00447937640536774\\
53.38	0.0044803822222235\\
53.39	0.00448138805197724\\
53.4	0.00448239389288324\\
53.41	0.00448339974318825\\
53.42	0.00448440560113142\\
53.43	0.00448541146494432\\
53.44	0.00448641733285089\\
53.45	0.00448742320306736\\
53.46	0.00448842907380231\\
53.47	0.00448943494325657\\
53.48	0.00449044080962323\\
53.49	0.0044914466710876\\
53.5	0.00449245252582715\\
53.51	0.00449345837201153\\
53.52	0.00449446420780251\\
53.53	0.00449547003135394\\
53.54	0.00449647584081175\\
53.55	0.0044974816343139\\
53.56	0.00449848740999034\\
53.57	0.00449949316596299\\
53.58	0.00450049890034573\\
53.59	0.00450150461124434\\
53.6	0.00450251029675646\\
53.61	0.00450351595497159\\
53.62	0.00450452158397105\\
53.63	0.00450552718182792\\
53.64	0.00450653274660705\\
53.65	0.00450753827636501\\
53.66	0.00450854376915002\\
53.67	0.004509549223002\\
53.68	0.00451055463595246\\
53.69	0.00451156000602453\\
53.7	0.00451256533123284\\
53.71	0.00451357060958361\\
53.72	0.0045145758390745\\
53.73	0.00451558101769466\\
53.74	0.00451658614342463\\
53.75	0.00451759121423638\\
53.76	0.0045185962280932\\
53.77	0.00451960118294974\\
53.78	0.0045206060767519\\
53.79	0.00452161090743686\\
53.8	0.00452261567293304\\
53.81	0.00452362037116001\\
53.82	0.00452462500002852\\
53.83	0.00452562955744042\\
53.84	0.00452663404128866\\
53.85	0.00452763844945724\\
53.86	0.00452864277982119\\
53.87	0.00452964703024647\\
53.88	0.00453065119859004\\
53.89	0.00453165528269975\\
53.9	0.00453265928041432\\
53.91	0.00453366318956332\\
53.92	0.00453466700796711\\
53.93	0.00453567073343683\\
53.94	0.00453667436377436\\
53.95	0.00453767789677227\\
53.96	0.00453868133021378\\
53.97	0.00453968466187275\\
53.98	0.00454068788951362\\
53.99	0.00454169101089137\\
54	0.00454269402375153\\
54.01	0.00454369692583007\\
54.02	0.00454469971485342\\
54.03	0.00454570238853841\\
54.04	0.00454670494459224\\
54.05	0.00454770738071244\\
54.06	0.00454870969458681\\
54.07	0.00454971188389344\\
54.08	0.0045507139463006\\
54.09	0.00455171587946676\\
54.1	0.00455271768104053\\
54.11	0.00455371934866063\\
54.12	0.00455472087995579\\
54.13	0.00455572227254485\\
54.14	0.00455672352403654\\
54.15	0.00455772463202962\\
54.16	0.00455872559411272\\
54.17	0.00455972640786433\\
54.18	0.00456072707085278\\
54.19	0.00456172758063621\\
54.2	0.00456272793476246\\
54.21	0.00456372813076913\\
54.22	0.00456472816618347\\
54.23	0.00456572803852235\\
54.24	0.00456672774529224\\
54.25	0.00456772728398917\\
54.26	0.00456872665209866\\
54.27	0.0045697258470957\\
54.28	0.00457072486644471\\
54.29	0.00457172370759949\\
54.3	0.00457272236800321\\
54.31	0.0045737208450883\\
54.32	0.00457471913627649\\
54.33	0.0045757172389787\\
54.34	0.00457671515059507\\
54.35	0.00457771286851481\\
54.36	0.00457871039011629\\
54.37	0.0045797077127669\\
54.38	0.00458070483382302\\
54.39	0.00458170175063002\\
54.4	0.00458269846052219\\
54.41	0.0045836949608227\\
54.42	0.00458469124884354\\
54.43	0.00458568732188553\\
54.44	0.0045866831772382\\
54.45	0.00458767881217979\\
54.46	0.00458867422397724\\
54.47	0.00458966940988606\\
54.48	0.00459066436715035\\
54.49	0.00459165909300276\\
54.5	0.00459265358466438\\
54.51	0.00459364783934479\\
54.52	0.00459464185424191\\
54.53	0.00459563562654206\\
54.54	0.00459662915341983\\
54.55	0.00459762243203808\\
54.56	0.00459861545954789\\
54.57	0.00459960823308849\\
54.58	0.00460060074978725\\
54.59	0.00460159300675959\\
54.6	0.00460258500110898\\
54.61	0.00460357672992686\\
54.62	0.00460456819029261\\
54.63	0.00460555937927349\\
54.64	0.00460655029392461\\
54.65	0.00460754093128886\\
54.66	0.00460853128839689\\
54.67	0.00460952136226704\\
54.68	0.00461051114990531\\
54.69	0.00461150064830528\\
54.7	0.00461248985444811\\
54.71	0.00461347876530244\\
54.72	0.00461446737782438\\
54.73	0.00461545568895745\\
54.74	0.00461644369563252\\
54.75	0.00461743139476778\\
54.76	0.00461841878326866\\
54.77	0.00461940585802784\\
54.78	0.0046203926159251\\
54.79	0.0046213790538274\\
54.8	0.00462236516858869\\
54.81	0.00462335095704999\\
54.82	0.00462433641603924\\
54.83	0.0046253215423713\\
54.84	0.00462630633284791\\
54.85	0.00462729078425756\\
54.86	0.00462827489337556\\
54.87	0.00462925865696388\\
54.88	0.00463024207177115\\
54.89	0.00463122513453262\\
54.9	0.00463220784197005\\
54.91	0.00463319019079173\\
54.92	0.00463417217769236\\
54.93	0.00463515379935306\\
54.94	0.00463613505244127\\
54.95	0.00463711593361071\\
54.96	0.00463809643950133\\
54.97	0.00463907656673926\\
54.98	0.00464005631193676\\
54.99	0.00464103567169213\\
55	0.00464201464258972\\
55.01	0.00464299322119982\\
55.02	0.00464397140407861\\
55.03	0.00464494918776814\\
55.04	0.00464592656879625\\
55.05	0.00464690354367651\\
55.06	0.00464788010890818\\
55.07	0.00464885626097614\\
55.08	0.00464983199635086\\
55.09	0.00465080731148828\\
55.1	0.00465178220282984\\
55.11	0.00465275666680238\\
55.12	0.00465373069981805\\
55.13	0.00465470429827433\\
55.14	0.00465567745855391\\
55.15	0.00465665017702463\\
55.16	0.00465762245003948\\
55.17	0.00465859427393648\\
55.18	0.00465956564503867\\
55.19	0.00466053655965402\\
55.2	0.00466150701407538\\
55.21	0.0046624770045804\\
55.22	0.00466344652743152\\
55.23	0.00466441557887588\\
55.24	0.00466538415514525\\
55.25	0.004666352252456\\
55.26	0.004667319867009\\
55.27	0.00466828699498959\\
55.28	0.00466925363256753\\
55.29	0.00467021977589691\\
55.3	0.00467118542111607\\
55.31	0.00467215056434762\\
55.32	0.00467311520169828\\
55.33	0.0046740793292589\\
55.34	0.00467504294310433\\
55.35	0.00467600603929342\\
55.36	0.00467696861386891\\
55.37	0.00467793066285738\\
55.38	0.00467889218226922\\
55.39	0.00467985316809851\\
55.4	0.00468081361632297\\
55.41	0.00468177352290396\\
55.42	0.00468273288378632\\
55.43	0.00468369169489839\\
55.44	0.00468464995215186\\
55.45	0.0046856076514418\\
55.46	0.00468656478864653\\
55.47	0.00468752135962755\\
55.48	0.00468847736022953\\
55.49	0.00468943278628019\\
55.5	0.00469038763359025\\
55.51	0.00469134189795339\\
55.52	0.00469229557514615\\
55.53	0.00469324866092785\\
55.54	0.00469420115104058\\
55.55	0.00469515304120909\\
55.56	0.00469610432714071\\
55.57	0.00469705500452534\\
55.58	0.00469800506903532\\
55.59	0.0046989545163254\\
55.6	0.00469990334203262\\
55.61	0.00470085154177633\\
55.62	0.00470179911115803\\
55.63	0.00470274604576137\\
55.64	0.00470369234115202\\
55.65	0.00470463867894219\\
55.66	0.0047055853442016\\
55.67	0.00470653233644948\\
55.68	0.00470747965520182\\
55.69	0.00470842729997135\\
55.7	0.00470937527026748\\
55.71	0.00471032356559634\\
55.72	0.00471127218546071\\
55.73	0.00471222112936008\\
55.74	0.00471317039679056\\
55.75	0.0047141199872449\\
55.76	0.00471506990021247\\
55.77	0.00471602013517924\\
55.78	0.0047169706916278\\
55.79	0.00471792156903725\\
55.8	0.00471887276688332\\
55.81	0.00471982428463821\\
55.82	0.00472077612177069\\
55.83	0.00472172827774604\\
55.84	0.00472268075202599\\
55.85	0.00472363354406879\\
55.86	0.00472458665332914\\
55.87	0.00472554007925818\\
55.88	0.00472649382130346\\
55.89	0.00472744787890898\\
55.9	0.00472840225151509\\
55.91	0.00472935693855855\\
55.92	0.00473031193947249\\
55.93	0.00473126725368634\\
55.94	0.00473222288062589\\
55.95	0.00473317881971325\\
55.96	0.00473413507036679\\
55.97	0.00473509163200119\\
55.98	0.00473604850402738\\
55.99	0.00473700568585251\\
56	0.00473796317687999\\
56.01	0.00473892097650941\\
56.02	0.00473987908413658\\
56.03	0.00474083749915345\\
56.04	0.00474179622094814\\
56.05	0.00474275524890493\\
56.06	0.00474371458240418\\
56.07	0.00474467422082239\\
56.08	0.00474563416353212\\
56.09	0.00474659440990199\\
56.1	0.00474755495929671\\
56.11	0.00474851581107696\\
56.12	0.0047494769645995\\
56.13	0.00475043841921703\\
56.14	0.00475140017427825\\
56.15	0.0047523622291278\\
56.16	0.0047533245831063\\
56.17	0.00475428723555023\\
56.18	0.00475525018579203\\
56.19	0.00475621343315998\\
56.2	0.00475717697697825\\
56.21	0.00475814081656684\\
56.22	0.00475910495124157\\
56.23	0.00476006938031409\\
56.24	0.00476103410309183\\
56.25	0.00476199911887797\\
56.26	0.00476296442697147\\
56.27	0.00476393002666698\\
56.28	0.0047648959172549\\
56.29	0.0047658620980213\\
56.3	0.00476682856824793\\
56.31	0.00476779532721217\\
56.32	0.00476876237418706\\
56.33	0.00476972970844123\\
56.34	0.00477069732923892\\
56.35	0.00477166523583993\\
56.36	0.0047726334274996\\
56.37	0.00477360190346882\\
56.38	0.00477457066299397\\
56.39	0.00477553970531695\\
56.4	0.0047765090296751\\
56.41	0.00477747863530122\\
56.42	0.00477844852142355\\
56.43	0.0047794186872657\\
56.44	0.00478038913204672\\
56.45	0.00478135985498097\\
56.46	0.00478233085527819\\
56.47	0.00478330213214343\\
56.48	0.00478427368477705\\
56.49	0.00478524551237466\\
56.5	0.00478621761412718\\
56.51	0.00478718998922072\\
56.52	0.00478816263683662\\
56.53	0.00478913555615143\\
56.54	0.00479010874633685\\
56.55	0.00479108220655973\\
56.56	0.00479205593598207\\
56.57	0.00479302993376094\\
56.58	0.00479400419904854\\
56.59	0.00479497873099208\\
56.6	0.00479595352873384\\
56.61	0.00479692859141111\\
56.62	0.00479790391815615\\
56.63	0.00479887950809624\\
56.64	0.00479985536035355\\
56.65	0.00480083147404522\\
56.66	0.00480180784828326\\
56.67	0.00480278448217458\\
56.68	0.00480376137482092\\
56.69	0.00480473852531889\\
56.7	0.00480571593275986\\
56.71	0.00480669359623003\\
56.72	0.00480767151481033\\
56.73	0.00480864968757642\\
56.74	0.00480962811359872\\
56.75	0.0048106067919423\\
56.76	0.00481158572166689\\
56.77	0.00481256490182688\\
56.78	0.00481354433147127\\
56.79	0.00481452400964365\\
56.8	0.0048155039353822\\
56.81	0.00481648410771959\\
56.82	0.00481746452568307\\
56.83	0.00481844518829433\\
56.84	0.00481942609456956\\
56.85	0.0048204072435194\\
56.86	0.00482138863414887\\
56.87	0.00482237026545741\\
56.88	0.00482335213643884\\
56.89	0.00482433424608128\\
56.9	0.00482531659336721\\
56.91	0.00482629917727336\\
56.92	0.00482728199677077\\
56.93	0.00482826505082468\\
56.94	0.00482924833839455\\
56.95	0.00483023185843404\\
56.96	0.00483121560989097\\
56.97	0.00483219959170728\\
56.98	0.00483318380281901\\
56.99	0.0048341682421563\\
57	0.00483515290864334\\
57.01	0.00483613780119834\\
57.02	0.00483712291873351\\
57.03	0.00483810826015504\\
57.04	0.00483909382436306\\
57.05	0.00484007961025161\\
57.06	0.00484106561670863\\
57.07	0.00484205183942905\\
57.08	0.00484303816927884\\
57.09	0.0048440246053821\\
57.1	0.00484501114685959\\
57.11	0.00484599779282872\\
57.12	0.00484698454240356\\
57.13	0.00484797139469477\\
57.14	0.00484895834880966\\
57.15	0.00484994540385211\\
57.16	0.00485093255892261\\
57.17	0.00485191981311825\\
57.18	0.00485290716553264\\
57.19	0.00485389461525598\\
57.2	0.00485488216137499\\
57.21	0.00485586980297293\\
57.22	0.00485685753912959\\
57.23	0.00485784536892124\\
57.24	0.00485883329142067\\
57.25	0.00485982130569712\\
57.26	0.00486080941081633\\
57.27	0.00486179760584048\\
57.28	0.00486278588982818\\
57.29	0.00486377426183449\\
57.3	0.0048647627209109\\
57.31	0.00486575126610526\\
57.32	0.00486673989646187\\
57.33	0.00486772861102136\\
57.34	0.00486871740882075\\
57.35	0.00486970628889341\\
57.36	0.00487069525026906\\
57.37	0.00487168429197372\\
57.38	0.00487267341302976\\
57.39	0.00487366261245583\\
57.4	0.00487465188926688\\
57.41	0.00487564124247412\\
57.42	0.00487663067108504\\
57.43	0.00487762017410337\\
57.44	0.00487860975052908\\
57.45	0.00487959939935836\\
57.46	0.0048805891195836\\
57.47	0.0048815789101934\\
57.48	0.00488256877017253\\
57.49	0.00488355869850195\\
57.5	0.00488454869415873\\
57.51	0.00488553875611613\\
57.52	0.0048865288833435\\
57.53	0.00488751907480634\\
57.54	0.00488850932946622\\
57.55	0.0048894996462808\\
57.56	0.00489049002420382\\
57.57	0.00489148046218506\\
57.58	0.00489247095917038\\
57.59	0.00489346151410163\\
57.6	0.0048944521259167\\
57.61	0.00489544279354948\\
57.62	0.00489643351592982\\
57.63	0.00489742429198357\\
57.64	0.00489841512063254\\
57.65	0.00489940600079446\\
57.66	0.00490039693138301\\
57.67	0.00490138791130779\\
57.68	0.00490237893947426\\
57.69	0.00490337001478382\\
57.7	0.0049043611361337\\
57.71	0.004905352302417\\
57.72	0.00490634351252267\\
57.73	0.00490733476533547\\
57.74	0.00490832605973598\\
57.75	0.00490931739460057\\
57.76	0.00491030876880141\\
57.77	0.0049113001812064\\
57.78	0.00491229163067924\\
57.79	0.00491328311607932\\
57.8	0.00491427463626177\\
57.81	0.00491526619007743\\
57.82	0.00491625777637283\\
57.83	0.00491724939399015\\
57.84	0.00491824104176727\\
57.85	0.00491923271853768\\
57.86	0.0049202244231305\\
57.87	0.00492121615437048\\
57.88	0.00492220791107796\\
57.89	0.00492319969206885\\
57.9	0.00492419149615463\\
57.91	0.00492518332214234\\
57.92	0.00492617516883453\\
57.93	0.00492716703502929\\
57.94	0.00492815891952019\\
57.95	0.0049291508210963\\
57.96	0.00493014273854215\\
57.97	0.00493113467063772\\
57.98	0.00493212661615844\\
57.99	0.00493311857387514\\
58	0.00493411054255407\\
58.01	0.00493510252095684\\
58.02	0.00493609450784047\\
58.03	0.00493708650195728\\
58.04	0.00493807850205497\\
58.05	0.00493907050687654\\
58.06	0.00494006251516031\\
58.07	0.00494105452563985\\
58.08	0.00494204653704403\\
58.09	0.00494303854809697\\
58.1	0.004944030557518\\
58.11	0.0049450225640217\\
58.12	0.00494601456631782\\
58.13	0.0049470065631113\\
58.14	0.00494799855310227\\
58.15	0.00494899053498598\\
58.16	0.00494998250745281\\
58.17	0.00495097446918826\\
58.18	0.00495196641887294\\
58.19	0.00495295835518251\\
58.2	0.00495395027678771\\
58.21	0.00495494218235431\\
58.22	0.00495593407054311\\
58.23	0.00495692594000991\\
58.24	0.00495791778940551\\
58.25	0.00495890961737565\\
58.26	0.00495990142256107\\
58.27	0.0049608932035974\\
58.28	0.0049618849591152\\
58.29	0.00496287668773992\\
58.3	0.00496386838809191\\
58.31	0.00496486005878635\\
58.32	0.00496585169843328\\
58.33	0.00496684330563756\\
58.34	0.00496783487899883\\
58.35	0.00496882641711156\\
58.36	0.00496981791856494\\
58.37	0.00497080938194294\\
58.38	0.00497180080582424\\
58.39	0.00497279218878222\\
58.4	0.00497378352938498\\
58.41	0.00497477482619527\\
58.42	0.00497576607777047\\
58.43	0.00497675728266263\\
58.44	0.0049777484394184\\
58.45	0.004978739546579\\
58.46	0.00497973060268025\\
58.47	0.00498072160625251\\
58.48	0.00498171255582067\\
58.49	0.00498270344990415\\
58.5	0.00498369428701684\\
58.51	0.00498468506566713\\
58.52	0.00498567578435784\\
58.53	0.00498666644158623\\
58.54	0.00498765703584397\\
58.55	0.00498864756561712\\
58.56	0.00498963802938614\\
58.57	0.0049906284256258\\
58.58	0.00499161875280523\\
58.59	0.00499260900938786\\
58.6	0.0049935991938314\\
58.61	0.00499458930458786\\
58.62	0.00499557934010346\\
58.63	0.00499656929881866\\
58.64	0.00499755917916814\\
58.65	0.00499854897958076\\
58.66	0.00499953869847952\\
58.67	0.00500052833428159\\
58.68	0.00500151788539826\\
58.69	0.0050025073502349\\
58.7	0.00500349672719097\\
58.71	0.00500448601465998\\
58.72	0.00500547521102949\\
58.73	0.00500646431468106\\
58.74	0.00500745332399025\\
58.75	0.00500844223732656\\
58.76	0.00500943105305347\\
58.77	0.00501041976952838\\
58.78	0.00501140838510259\\
58.79	0.00501239689812127\\
58.8	0.00501338530692345\\
58.81	0.00501437360984201\\
58.82	0.00501536180520363\\
58.83	0.00501634989132879\\
58.84	0.00501733786653173\\
58.85	0.00501832572912043\\
58.86	0.00501931347739661\\
58.87	0.00502030110965567\\
58.88	0.00502128862418669\\
58.89	0.00502227601927242\\
58.9	0.00502326329318922\\
58.91	0.00502425044420705\\
58.92	0.00502523747058947\\
58.93	0.0050262243705936\\
58.94	0.00502721114247009\\
58.95	0.00502819778446309\\
58.96	0.00502918429481027\\
58.97	0.00503017067174272\\
58.98	0.005031156913485\\
58.99	0.00503214301825509\\
59	0.00503312898426434\\
59.01	0.00503411480971749\\
59.02	0.0050351004928126\\
59.03	0.00503608603174106\\
59.04	0.00503707142468757\\
59.05	0.00503805666983009\\
59.06	0.0050390417653398\\
59.07	0.00504002670938113\\
59.08	0.0050410115001117\\
59.09	0.00504199613568229\\
59.1	0.00504298061423683\\
59.11	0.00504396493391238\\
59.12	0.00504494909283906\\
59.13	0.00504593308914012\\
59.14	0.00504691692093179\\
59.15	0.00504790058632336\\
59.16	0.00504888408341709\\
59.17	0.00504986741030822\\
59.18	0.00505085056508493\\
59.19	0.0050518335458283\\
59.2	0.00505281635061231\\
59.21	0.00505379897750381\\
59.22	0.00505478142456246\\
59.23	0.00505576368984075\\
59.24	0.00505674577138396\\
59.25	0.00505772766723011\\
59.26	0.00505870937540995\\
59.27	0.00505969089394694\\
59.28	0.00506067222085721\\
59.29	0.00506165335414955\\
59.3	0.00506263429182536\\
59.31	0.00506361503187864\\
59.32	0.00506459557229596\\
59.33	0.00506557591105641\\
59.34	0.00506655604613161\\
59.35	0.00506753597548568\\
59.36	0.00506851569707516\\
59.37	0.00506949520884905\\
59.38	0.00507047450874873\\
59.39	0.00507145359470796\\
59.4	0.00507243246465285\\
59.41	0.00507341111650181\\
59.42	0.00507438954816555\\
59.43	0.00507536775754705\\
59.44	0.0050763457425415\\
59.45	0.00507732350103629\\
59.46	0.005078301030911\\
59.47	0.00507927833003736\\
59.48	0.00508025539627918\\
59.49	0.00508123222749239\\
59.5	0.00508220882152496\\
59.51	0.00508318517621689\\
59.52	0.00508416128940017\\
59.53	0.00508513715889877\\
59.54	0.00508611278252858\\
59.55	0.00508708815809743\\
59.56	0.00508806328340499\\
59.57	0.0050890381562428\\
59.58	0.00509001277439421\\
59.59	0.00509098713563437\\
59.6	0.00509196123773017\\
59.61	0.00509293507844022\\
59.62	0.00509390865551484\\
59.63	0.00509488196669602\\
59.64	0.00509585500971738\\
59.65	0.00509682778230412\\
59.66	0.00509780028217303\\
59.67	0.00509877250703244\\
59.68	0.00509974445458218\\
59.69	0.00510071612251357\\
59.7	0.00510168750850937\\
59.71	0.00510265861024373\\
59.72	0.00510362942538222\\
59.73	0.00510459995158173\\
59.74	0.00510557018649049\\
59.75	0.00510654012774799\\
59.76	0.00510750977298498\\
59.77	0.00510847911982343\\
59.78	0.00510944816587652\\
59.79	0.00511041690874856\\
59.8	0.00511138534603496\\
59.81	0.00511235347532227\\
59.82	0.00511332129418805\\
59.83	0.00511428880020091\\
59.84	0.00511525599092043\\
59.85	0.00511622286389714\\
59.86	0.00511718941667252\\
59.87	0.00511815564677892\\
59.88	0.00511912155173953\\
59.89	0.00512008712906836\\
59.9	0.00512105237627022\\
59.91	0.00512201729084068\\
59.92	0.00512298187026598\\
59.93	0.0051239461120231\\
59.94	0.00512491001357961\\
59.95	0.00512587357239372\\
59.96	0.00512683678591421\\
59.97	0.00512779965158041\\
59.98	0.00512876216682213\\
59.99	0.00512972432905969\\
60	0.00513068613570381\\
60.01	0.00513164758415562\\
60.02	0.00513260867180659\\
60.03	0.00513356939603857\\
60.04	0.00513452975422365\\
60.05	0.00513548974372419\\
60.06	0.00513644936189278\\
60.07	0.00513740860607216\\
60.08	0.00513836747359524\\
60.09	0.00513932596178504\\
60.1	0.00514028406795463\\
60.11	0.00514124178940713\\
60.12	0.00514219912343564\\
60.13	0.00514315606732323\\
60.14	0.00514411261834288\\
60.15	0.00514506877375747\\
60.16	0.00514602453081971\\
60.17	0.0051469798867721\\
60.18	0.00514793483884695\\
60.19	0.00514888938426627\\
60.2	0.00514984352024177\\
60.21	0.0051507972439748\\
60.22	0.00515175055265635\\
60.23	0.00515270344346697\\
60.24	0.00515365591357674\\
60.25	0.00515460796014524\\
60.26	0.00515555958032151\\
60.27	0.00515651077124401\\
60.28	0.00515746153004057\\
60.29	0.00515841185382836\\
60.3	0.00515936173971386\\
60.31	0.00516031118479278\\
60.32	0.00516126018615008\\
60.33	0.00516220874085986\\
60.34	0.0051631568459854\\
60.35	0.00516410449857903\\
60.36	0.00516505169568218\\
60.37	0.00516599843432524\\
60.38	0.00516694471152761\\
60.39	0.00516789052429761\\
60.4	0.00516883586963245\\
60.41	0.00516978074451818\\
60.42	0.00517072514592966\\
60.43	0.00517166907083051\\
60.44	0.00517261251617308\\
60.45	0.00517355547889837\\
60.46	0.00517449795593604\\
60.47	0.00517543994420434\\
60.48	0.00517638144061006\\
60.49	0.00517732244204851\\
60.5	0.00517826294540343\\
60.51	0.00517920294754703\\
60.52	0.00518014244533985\\
60.53	0.00518108143563077\\
60.54	0.00518201991525699\\
60.55	0.00518295788104392\\
60.56	0.005183895951657\\
60.57	0.00518483419195687\\
60.58	0.00518577260158482\\
60.59	0.00518671118018124\\
60.6	0.00518764992738566\\
60.61	0.00518858884283672\\
60.62	0.00518952792617221\\
60.63	0.00519046717702903\\
60.64	0.00519140659504318\\
60.65	0.00519234617984981\\
60.66	0.00519328593108318\\
60.67	0.00519422584837667\\
60.68	0.00519516593136278\\
60.69	0.00519610617967312\\
60.7	0.00519704659293844\\
60.71	0.00519798717078858\\
60.72	0.00519892791285251\\
60.73	0.00519986881875831\\
60.74	0.00520080988813319\\
60.75	0.00520175112060344\\
60.76	0.00520269251579451\\
60.77	0.00520363407333092\\
60.78	0.00520457579283631\\
60.79	0.00520551767393346\\
60.8	0.00520645971624422\\
60.81	0.00520740191938959\\
60.82	0.00520834428298964\\
60.83	0.00520928680666358\\
60.84	0.0052102294900297\\
60.85	0.00521117233270542\\
60.86	0.00521211533430724\\
60.87	0.00521305849445081\\
60.88	0.00521400181275084\\
60.89	0.00521494528882117\\
60.9	0.00521588892227474\\
60.91	0.00521683271272357\\
60.92	0.0052177766597788\\
60.93	0.00521872076305069\\
60.94	0.00521966502214857\\
60.95	0.00522060943668088\\
60.96	0.00522155400625518\\
60.97	0.00522249873047809\\
60.98	0.00522344360895538\\
60.99	0.00522438864129186\\
61	0.00522533382709149\\
61.01	0.00522627916595729\\
61.02	0.0052272246574914\\
61.03	0.00522817030129503\\
61.04	0.00522911609696852\\
61.05	0.00523006204411128\\
61.06	0.0052310081423218\\
61.07	0.00523195439119771\\
61.08	0.00523290079033569\\
61.09	0.00523384733933153\\
61.1	0.00523479403778013\\
61.11	0.00523574088527543\\
61.12	0.00523668788141052\\
61.13	0.00523763502577754\\
61.14	0.00523858231796773\\
61.15	0.00523952975757144\\
61.16	0.00524047734417808\\
61.17	0.00524142507737615\\
61.18	0.00524237295675327\\
61.19	0.00524332098189611\\
61.2	0.00524426915239045\\
61.21	0.00524521746782115\\
61.22	0.00524616592777214\\
61.23	0.00524711453182647\\
61.24	0.00524806327956624\\
61.25	0.00524901217057267\\
61.26	0.00524996120442604\\
61.27	0.00525091038070571\\
61.28	0.00525185969899015\\
61.29	0.00525280915885688\\
61.3	0.00525375875988252\\
61.31	0.00525470850164279\\
61.32	0.00525565838371246\\
61.33	0.00525660840566539\\
61.34	0.00525755856707455\\
61.35	0.00525850886751195\\
61.36	0.0052594593065487\\
61.37	0.005260409883755\\
61.38	0.00526136059870011\\
61.39	0.00526231145095238\\
61.4	0.00526326244007924\\
61.41	0.0052642135656472\\
61.42	0.00526516482722184\\
61.43	0.00526611622436784\\
61.44	0.00526706775664892\\
61.45	0.00526801942362793\\
61.46	0.00526897122486675\\
61.47	0.00526992315992635\\
61.48	0.0052708752283668\\
61.49	0.00527182742974722\\
61.5	0.00527277976362581\\
61.51	0.00527373222955988\\
61.52	0.00527468482710577\\
61.53	0.00527563755581893\\
61.54	0.00527659041525385\\
61.55	0.00527754340496413\\
61.56	0.00527849652450245\\
61.57	0.00527944977342052\\
61.58	0.00528040315126917\\
61.59	0.00528135665759828\\
61.6	0.00528231029195681\\
61.61	0.00528326405389282\\
61.62	0.0052842179429534\\
61.63	0.00528517195868475\\
61.64	0.00528612610063212\\
61.65	0.00528708036833987\\
61.66	0.00528803476135137\\
61.67	0.00528898927920915\\
61.68	0.00528994392145474\\
61.69	0.00529089868762878\\
61.7	0.00529185357727096\\
61.71	0.00529280858992007\\
61.72	0.00529376372511398\\
61.73	0.0052947189823896\\
61.74	0.00529567436128294\\
61.75	0.00529662986132907\\
61.76	0.00529758548206213\\
61.77	0.00529854122301536\\
61.78	0.00529949708372104\\
61.79	0.00530045306371055\\
61.8	0.00530140916251434\\
61.81	0.0053023653796619\\
61.82	0.00530332171468185\\
61.83	0.00530427816710183\\
61.84	0.00530523473644861\\
61.85	0.00530619142224798\\
61.86	0.00530714822402484\\
61.87	0.00530810514130314\\
61.88	0.00530906217360593\\
61.89	0.00531001932045532\\
61.9	0.00531097658137249\\
61.91	0.00531193395587771\\
61.92	0.00531289144349034\\
61.93	0.00531384904372883\\
61.94	0.0053148067561107\\
61.95	0.00531576458015257\\
61.96	0.00531672251537016\\
61.97	0.00531768056127824\\
61.98	0.0053186387173907\\
61.99	0.0053195969832205\\
62	0.0053205553582797\\
62.01	0.00532151384207943\\
62.02	0.00532247243412993\\
62.03	0.00532343113394051\\
62.04	0.00532438994101959\\
62.05	0.00532534885487465\\
62.06	0.00532630787501229\\
62.07	0.00532726700093816\\
62.08	0.00532822623215705\\
62.09	0.0053291855681728\\
62.1	0.00533014500848836\\
62.11	0.00533110455260576\\
62.12	0.00533206420002613\\
62.13	0.00533302395024968\\
62.14	0.00533398380277574\\
62.15	0.00533494375710269\\
62.16	0.00533590381272805\\
62.17	0.00533686396914838\\
62.18	0.00533782422585939\\
62.19	0.00533878458235584\\
62.2	0.00533974503813162\\
62.21	0.00534070559267966\\
62.22	0.00534166624549206\\
62.23	0.00534262699605996\\
62.24	0.00534358784387361\\
62.25	0.00534454878842238\\
62.26	0.0053455098291947\\
62.27	0.00534647096567813\\
62.28	0.00534743219735932\\
62.29	0.00534839352372399\\
62.3	0.00534935494425703\\
62.31	0.00535031645844235\\
62.32	0.00535127806576301\\
62.33	0.00535223976570116\\
62.34	0.00535320155773806\\
62.35	0.00535416344135405\\
62.36	0.0053551254160286\\
62.37	0.00535608748124027\\
62.38	0.00535704963646672\\
62.39	0.00535801188118475\\
62.4	0.00535897421487023\\
62.41	0.00535993663699814\\
62.42	0.00536089914704259\\
62.43	0.00536186174447678\\
62.44	0.00536282442877303\\
62.45	0.00536378719940277\\
62.46	0.00536475005583653\\
62.47	0.00536571299754397\\
62.48	0.00536667602399386\\
62.49	0.00536763913465407\\
62.5	0.00536860232899159\\
62.51	0.00536956560647255\\
62.52	0.00537052896656215\\
62.53	0.00537149240872475\\
62.54	0.00537245593242382\\
62.55	0.00537341953712192\\
62.56	0.00537438322228077\\
62.57	0.0053753469873612\\
62.58	0.00537631083182315\\
62.59	0.0053772747551257\\
62.6	0.00537823875672704\\
62.61	0.00537920283608449\\
62.62	0.0053801669926545\\
62.63	0.00538113122589266\\
62.64	0.00538209553525366\\
62.65	0.00538305992019135\\
62.66	0.0053840243801587\\
62.67	0.00538498891460779\\
62.68	0.00538595352298989\\
62.69	0.00538691820475534\\
62.7	0.00538788295935367\\
62.71	0.00538884778623351\\
62.72	0.00538981268484265\\
62.73	0.005390777654628\\
62.74	0.00539174269503565\\
62.75	0.00539270780551078\\
62.76	0.00539367298549775\\
62.77	0.00539463823444007\\
62.78	0.00539560355178038\\
62.79	0.00539656893696046\\
62.8	0.00539753438942125\\
62.81	0.00539849990860286\\
62.82	0.00539946549394452\\
62.83	0.00540043114488464\\
62.84	0.00540139686086076\\
62.85	0.0054023626413096\\
62.86	0.00540332848566703\\
62.87	0.00540429439336807\\
62.88	0.00540526036384692\\
62.89	0.00540622639653693\\
62.9	0.00540719249087062\\
62.91	0.00540815864627967\\
62.92	0.00540912486219494\\
62.93	0.00541009113804645\\
62.94	0.00541105747326339\\
62.95	0.00541202386727413\\
62.96	0.00541299031950621\\
62.97	0.00541395682938634\\
62.98	0.00541492339634043\\
62.99	0.00541589001979355\\
63	0.00541685669916996\\
63.01	0.00541782343389308\\
63.02	0.00541879022338557\\
63.03	0.00541975706706922\\
63.04	0.00542072396436504\\
63.05	0.00542169091469323\\
63.06	0.00542265791747317\\
63.07	0.00542362497212345\\
63.08	0.00542459207806185\\
63.09	0.00542555923470533\\
63.1	0.0054265264414701\\
63.11	0.00542749369777151\\
63.12	0.00542846100302417\\
63.13	0.00542942835664186\\
63.14	0.0054303957580376\\
63.15	0.0054313632066236\\
63.16	0.00543233070181129\\
63.17	0.00543329824301131\\
63.18	0.00543426582963352\\
63.19	0.00543523346108701\\
63.2	0.00543620113678009\\
63.21	0.00543716885612028\\
63.22	0.00543813661851436\\
63.23	0.00543910442336829\\
63.24	0.0054400722700873\\
63.25	0.00544104015807585\\
63.26	0.00544200808673761\\
63.27	0.00544297605547553\\
63.28	0.00544394406369177\\
63.29	0.00544491211078774\\
63.3	0.00544588019616411\\
63.31	0.00544684831922079\\
63.32	0.00544781647935692\\
63.33	0.00544878467597094\\
63.34	0.00544975290846052\\
63.35	0.00545072117622257\\
63.36	0.00545168947865329\\
63.37	0.00545265781514815\\
63.38	0.00545362618510185\\
63.39	0.00545459458790841\\
63.4	0.00545556302296107\\
63.41	0.0054565314896524\\
63.42	0.00545749998737419\\
63.43	0.00545846851551757\\
63.44	0.00545943707347291\\
63.45	0.00546040566062989\\
63.46	0.00546137427637746\\
63.47	0.00546234292010389\\
63.48	0.00546331159119672\\
63.49	0.00546428028904281\\
63.5	0.00546524901302831\\
63.51	0.00546621776253866\\
63.52	0.00546718653695864\\
63.53	0.00546815533567231\\
63.54	0.00546912415806308\\
63.55	0.00547009300351364\\
63.56	0.00547106187140602\\
63.57	0.00547203076112158\\
63.58	0.00547299967204098\\
63.59	0.00547396860354425\\
63.6	0.0054749375550107\\
63.61	0.00547590652581904\\
63.62	0.00547687551534728\\
63.63	0.00547784452297277\\
63.64	0.00547881354807223\\
63.65	0.00547978259002173\\
63.66	0.00548075164819666\\
63.67	0.00548172072197181\\
63.68	0.0054826898107213\\
63.69	0.00548365891381864\\
63.7	0.0054846280306367\\
63.71	0.0054855971605477\\
63.72	0.00548656630292327\\
63.73	0.00548753545713439\\
63.74	0.00548850462255145\\
63.75	0.00548947379854421\\
63.76	0.00549044298448182\\
63.77	0.00549141217973283\\
63.78	0.00549238138366519\\
63.79	0.00549335059564625\\
63.8	0.00549431981504277\\
63.81	0.0054952890412209\\
63.82	0.00549625827354625\\
63.83	0.0054972275113838\\
63.84	0.00549819675409797\\
63.85	0.00549916600105262\\
63.86	0.00550013525161103\\
63.87	0.0055011045051359\\
63.88	0.0055020737609894\\
63.89	0.00550304301853312\\
63.9	0.0055040122771281\\
63.91	0.00550498153613485\\
63.92	0.00550595079491332\\
63.93	0.00550692005282292\\
63.94	0.00550788930922253\\
63.95	0.0055088585634705\\
63.96	0.00550982781492466\\
63.97	0.0055107970629423\\
63.98	0.00551176630688022\\
63.99	0.00551273554609467\\
64	0.00551370477994145\\
64.01	0.0055146740077758\\
64.02	0.00551564322895249\\
64.03	0.0055166124428258\\
64.04	0.00551758164874951\\
64.05	0.00551855084607691\\
64.06	0.00551952003416085\\
64.07	0.00552048921235365\\
64.08	0.00552145838000719\\
64.09	0.0055224275364729\\
64.1	0.00552339668110174\\
64.11	0.0055243658132442\\
64.12	0.00552533493225034\\
64.13	0.00552630403746977\\
64.14	0.00552727312825167\\
64.15	0.00552824220394478\\
64.16	0.0055292112638974\\
64.17	0.00553018030745744\\
64.18	0.00553114933397236\\
64.19	0.00553211834278921\\
64.2	0.00553308733325467\\
64.21	0.00553405630471498\\
64.22	0.005535025256516\\
64.23	0.00553599418800321\\
64.24	0.0055369630985217\\
64.25	0.00553793198741617\\
64.26	0.00553890085403095\\
64.27	0.00553986969771001\\
64.28	0.00554083851779696\\
64.29	0.00554180731363507\\
64.3	0.00554277608456722\\
64.31	0.00554374482993598\\
64.32	0.00554471354908358\\
64.33	0.00554568224135189\\
64.34	0.00554665090608249\\
64.35	0.00554761954261661\\
64.36	0.00554858815029521\\
64.37	0.0055495567284589\\
64.38	0.005550525276448\\
64.39	0.00555149379360256\\
64.4	0.0055524622792623\\
64.41	0.00555343073276669\\
64.42	0.00555439915345492\\
64.43	0.00555536754066591\\
64.44	0.00555633589373829\\
64.45	0.00555730421201049\\
64.46	0.00555827249482065\\
64.47	0.00555924074150666\\
64.48	0.00556020895140621\\
64.49	0.00556117712385674\\
64.5	0.00556214525819547\\
64.51	0.00556311335375939\\
64.52	0.00556408140988531\\
64.53	0.00556504942590982\\
64.54	0.00556601740116931\\
64.55	0.00556698533500002\\
64.56	0.00556795322673795\\
64.57	0.00556892107571898\\
64.58	0.00556988888127879\\
64.59	0.00557085664275292\\
64.6	0.00557182435947675\\
64.61	0.00557279203078552\\
64.62	0.00557375965601434\\
64.63	0.00557472723449817\\
64.64	0.00557569476557187\\
64.65	0.00557666224857017\\
64.66	0.0055776296828277\\
64.67	0.00557859706767899\\
64.68	0.00557956440245848\\
64.69	0.00558053168650054\\
64.7	0.00558149891913944\\
64.71	0.00558246609970939\\
64.72	0.00558343322754456\\
64.73	0.00558440030197904\\
64.74	0.00558536732234688\\
64.75	0.00558633428798212\\
64.76	0.00558730119821876\\
64.77	0.00558826805239077\\
64.78	0.00558923484983211\\
64.79	0.00559020158987676\\
64.8	0.00559116827185868\\
64.81	0.00559213489511186\\
64.82	0.0055931014589703\\
64.83	0.00559406796276805\\
64.84	0.00559503440583919\\
64.85	0.00559600078751783\\
64.86	0.00559696710713817\\
64.87	0.00559793336403448\\
64.88	0.00559889955754106\\
64.89	0.00559986568699233\\
64.9	0.00560083175172282\\
64.91	0.00560179775106711\\
64.92	0.00560276368435993\\
64.93	0.00560372955093612\\
64.94	0.00560469535013065\\
64.95	0.00560566108127863\\
64.96	0.00560662674371533\\
64.97	0.00560759233677614\\
64.98	0.00560855785979667\\
64.99	0.00560952331211267\\
65	0.00561048869306009\\
65.01	0.00561145400197507\\
65.02	0.00561241923819397\\
65.03	0.00561338440105335\\
65.04	0.00561434948989\\
65.05	0.00561531450404095\\
65.06	0.00561627944284347\\
65.07	0.00561724430563509\\
65.08	0.00561820909175362\\
65.09	0.00561917380053711\\
65.1	0.00562013843132393\\
65.11	0.00562110298345274\\
65.12	0.00562206745626249\\
65.13	0.00562303184909248\\
65.14	0.00562399616128229\\
65.15	0.00562496039217189\\
65.16	0.00562592454110157\\
65.17	0.00562688860741199\\
65.18	0.00562785259044418\\
65.19	0.00562881648953954\\
65.2	0.00562978030403989\\
65.21	0.00563074403328743\\
65.22	0.00563170767662478\\
65.23	0.005632671233395\\
65.24	0.00563363470294157\\
65.25	0.00563459808460844\\
65.26	0.00563556137773998\\
65.27	0.00563652458168109\\
65.28	0.0056374876957771\\
65.29	0.00563845071937387\\
65.3	0.00563941365181775\\
65.31	0.00564037649245561\\
65.32	0.00564133924063488\\
65.33	0.00564230189570348\\
65.34	0.00564326445700992\\
65.35	0.00564422692390326\\
65.36	0.00564518929573316\\
65.37	0.00564615157184984\\
65.38	0.00564711375160416\\
65.39	0.00564807583434756\\
65.4	0.00564903781943214\\
65.41	0.00564999970621061\\
65.42	0.00565096149403635\\
65.43	0.00565192318226342\\
65.44	0.00565288477024652\\
65.45	0.00565384625734108\\
65.46	0.00565480764290322\\
65.47	0.00565576892628978\\
65.48	0.00565673010685832\\
65.49	0.00565769118396715\\
65.5	0.00565865215697536\\
65.51	0.00565961302524278\\
65.52	0.00566057378813004\\
65.53	0.00566153444499854\\
65.54	0.00566249499521054\\
65.55	0.00566345543812908\\
65.56	0.00566441577311808\\
65.57	0.00566537599954227\\
65.58	0.00566633611676728\\
65.59	0.0056672961241596\\
65.6	0.00566825602108664\\
65.61	0.00566921580691668\\
65.62	0.00567017548101895\\
65.63	0.00567113504276363\\
65.64	0.00567209449152183\\
65.65	0.00567305382666564\\
65.66	0.00567401304756811\\
65.67	0.00567497215360331\\
65.68	0.00567593114414631\\
65.69	0.00567689001857321\\
65.7	0.00567784877626116\\
65.71	0.00567880741658834\\
65.72	0.00567976593893403\\
65.73	0.00568072434267859\\
65.74	0.00568168262720348\\
65.75	0.00568264079189126\\
65.76	0.00568359883612566\\
65.77	0.00568455675929153\\
65.78	0.00568551456077489\\
65.79	0.00568647223996295\\
65.8	0.00568742979624411\\
65.81	0.005688387229008\\
65.82	0.00568934453764545\\
65.83	0.00569030172154855\\
65.84	0.00569125878011065\\
65.85	0.0056922157127264\\
65.86	0.00569317251879171\\
65.87	0.00569412919770382\\
65.88	0.00569508574886129\\
65.89	0.00569604217166406\\
65.9	0.00569699846551337\\
65.91	0.0056979546298119\\
65.92	0.00569891066396368\\
65.93	0.00569986656737418\\
65.94	0.0057008223394503\\
65.95	0.00570177797960039\\
65.96	0.00570273348723425\\
65.97	0.00570368886176317\\
65.98	0.00570464410259997\\
65.99	0.00570559920915895\\
66	0.00570655418085598\\
66.01	0.00570750901710846\\
66.02	0.00570846371733538\\
66.03	0.00570941828095732\\
66.04	0.00571037270739647\\
66.05	0.00571132699607666\\
66.06	0.00571228114642335\\
66.07	0.00571323515786367\\
66.08	0.00571418902982646\\
66.09	0.00571514276174224\\
66.1	0.00571609635304327\\
66.11	0.00571704980316353\\
66.12	0.0057180031115388\\
66.13	0.00571895627760661\\
66.14	0.00571990930080631\\
66.15	0.00572086218057907\\
66.16	0.0057218149163679\\
66.17	0.00572276750761767\\
66.18	0.00572371995377514\\
66.19	0.00572467225428896\\
66.2	0.00572562440860971\\
66.21	0.00572657641618991\\
66.22	0.00572752827648404\\
66.23	0.00572847998894857\\
66.24	0.00572943155304198\\
66.25	0.00573038296822476\\
66.26	0.00573133423395947\\
66.27	0.00573228534971071\\
66.28	0.00573323631494518\\
66.29	0.00573418712913171\\
66.3	0.00573513779174124\\
66.31	0.00573608830224687\\
66.32	0.00573703866012389\\
66.33	0.00573798886484975\\
66.34	0.00573893891590416\\
66.35	0.00573988881276905\\
66.36	0.00574083855492862\\
66.37	0.00574178814186935\\
66.38	0.00574273757308005\\
66.39	0.00574368684805185\\
66.4	0.00574463596627822\\
66.41	0.00574558492725501\\
66.42	0.00574653373048051\\
66.43	0.00574748237545538\\
66.44	0.00574843086168276\\
66.45	0.00574937918866826\\
66.46	0.00575032735591997\\
66.47	0.00575127536294851\\
66.48	0.00575222320926704\\
66.49	0.00575317089439126\\
66.5	0.00575411841783952\\
66.51	0.00575506577913273\\
66.52	0.00575601297779446\\
66.53	0.00575696001335094\\
66.54	0.00575790688533109\\
66.55	0.00575885359326656\\
66.56	0.0057598001366917\\
66.57	0.00576074651514365\\
66.58	0.00576169272816236\\
66.59	0.00576263877529053\\
66.6	0.00576358465607377\\
66.61	0.00576453037006051\\
66.62	0.0057654759168021\\
66.63	0.00576642129585277\\
66.64	0.00576736650676974\\
66.65	0.00576831154911317\\
66.66	0.00576925642244622\\
66.67	0.00577020112633509\\
66.68	0.00577114566034901\\
66.69	0.00577209002406029\\
66.7	0.00577303421704436\\
66.71	0.00577397823887976\\
66.72	0.00577492208914822\\
66.73	0.00577586576743462\\
66.74	0.00577680927332707\\
66.75	0.00577775260641693\\
66.76	0.00577869576629881\\
66.77	0.00577963875257064\\
66.78	0.00578058156483367\\
66.79	0.00578152420269248\\
66.8	0.00578246666575509\\
66.81	0.00578340895363288\\
66.82	0.00578435106594069\\
66.83	0.00578529300229683\\
66.84	0.00578623476232312\\
66.85	0.00578717634564489\\
66.86	0.00578811775189104\\
66.87	0.00578905898069407\\
66.88	0.00579000003169008\\
66.89	0.00579094090451883\\
66.9	0.00579188159882375\\
66.91	0.00579282211425199\\
66.92	0.00579376245045445\\
66.93	0.00579470260708576\\
66.94	0.00579564258380441\\
66.95	0.00579658238027269\\
66.96	0.00579752199615675\\
66.97	0.00579846143112666\\
66.98	0.00579940068485639\\
66.99	0.0058003397570239\\
67	0.00580127864731111\\
67.01	0.00580221735540401\\
67.02	0.0058031558809926\\
67.03	0.00580409422377098\\
67.04	0.00580503238343741\\
67.05	0.00580597035969425\\
67.06	0.0058069081522481\\
67.07	0.00580784576080973\\
67.08	0.00580878318509422\\
67.09	0.00580972042482089\\
67.1	0.00581065747971342\\
67.11	0.00581159434949982\\
67.12	0.00581253103391251\\
67.13	0.00581346753268833\\
67.14	0.00581440384556858\\
67.15	0.00581533997229904\\
67.16	0.00581627591263005\\
67.17	0.00581721166631649\\
67.18	0.00581814723311786\\
67.19	0.00581908261279828\\
67.2	0.00582001780512655\\
67.21	0.00582095280987619\\
67.22	0.00582188762682544\\
67.23	0.00582282225575733\\
67.24	0.00582375669645972\\
67.25	0.00582469094872532\\
67.26	0.00582562501235172\\
67.27	0.00582655888714145\\
67.28	0.00582749257290199\\
67.29	0.00582842606944586\\
67.3	0.00582935937659058\\
67.31	0.00583029249415876\\
67.32	0.00583122542197815\\
67.33	0.00583215815988163\\
67.34	0.00583309070770727\\
67.35	0.00583402306529842\\
67.36	0.00583495523250365\\
67.37	0.00583588720917688\\
67.38	0.00583681899517735\\
67.39	0.00583775059036971\\
67.4	0.00583868199462405\\
67.41	0.00583961320781591\\
67.42	0.00584054422982635\\
67.43	0.00584147506054198\\
67.44	0.00584240569985502\\
67.45	0.0058433361476633\\
67.46	0.00584426640387033\\
67.47	0.00584519646838535\\
67.48	0.00584612634112336\\
67.49	0.00584705602200514\\
67.5	0.00584798551095735\\
67.51	0.00584891480791248\\
67.52	0.005849843912809\\
67.53	0.00585077282559132\\
67.54	0.00585170154620988\\
67.55	0.00585263007462116\\
67.56	0.00585355841078776\\
67.57	0.00585448655467841\\
67.58	0.00585541450626804\\
67.59	0.00585634226553781\\
67.6	0.00585726983247514\\
67.61	0.00585819720707379\\
67.62	0.00585912438933389\\
67.63	0.00586005137926197\\
67.64	0.00586097817687103\\
67.65	0.00586190478218056\\
67.66	0.0058628311952166\\
67.67	0.0058637574160118\\
67.68	0.00586468344460545\\
67.69	0.00586560928104349\\
67.7	0.00586653492537866\\
67.71	0.00586746037767044\\
67.72	0.00586838563798515\\
67.73	0.00586931070639598\\
67.74	0.00587023558298306\\
67.75	0.00587116026783348\\
67.76	0.00587208476104138\\
67.77	0.00587300906270793\\
67.78	0.00587393317294145\\
67.79	0.00587485709185743\\
67.8	0.00587578081957856\\
67.81	0.00587670435623482\\
67.82	0.0058776277019635\\
67.83	0.00587855085690925\\
67.84	0.00587947382122415\\
67.85	0.00588039659506774\\
67.86	0.00588131917860711\\
67.87	0.00588224157201689\\
67.88	0.00588316377547934\\
67.89	0.00588408578918443\\
67.9	0.00588500761332982\\
67.91	0.00588592924812097\\
67.92	0.00588685069377116\\
67.93	0.00588777195050159\\
67.94	0.00588869301854136\\
67.95	0.00588961389812758\\
67.96	0.00589053458950543\\
67.97	0.00589145509292816\\
67.98	0.00589237540865719\\
67.99	0.00589329553696216\\
68	0.00589421547812095\\
68.01	0.0058951352324198\\
68.02	0.0058960548001533\\
68.03	0.0058969741816245\\
68.04	0.0058978933771449\\
68.05	0.00589881238703457\\
68.06	0.0058997312116222\\
68.07	0.0059006498512451\\
68.08	0.00590156830624933\\
68.09	0.00590248657698971\\
68.1	0.0059034046638299\\
68.11	0.00590432256714245\\
68.12	0.00590524039254305\\
68.13	0.0059061582033246\\
68.14	0.00590707599964485\\
68.15	0.00590799378166609\\
68.16	0.00590891154955514\\
68.17	0.00590982930348342\\
68.18	0.00591074704362693\\
68.19	0.00591166477016633\\
68.2	0.00591258248328694\\
68.21	0.00591350018317875\\
68.22	0.00591441787003651\\
68.23	0.0059153355440597\\
68.24	0.00591625320545258\\
68.25	0.00591717085442425\\
68.26	0.00591808849118861\\
68.27	0.00591900611596447\\
68.28	0.00591992372897552\\
68.29	0.00592084133045039\\
68.3	0.00592175892062267\\
68.31	0.00592267649973095\\
68.32	0.00592359406801885\\
68.33	0.00592451162573503\\
68.34	0.00592542917313325\\
68.35	0.00592634671047237\\
68.36	0.00592726423801643\\
68.37	0.00592818175603461\\
68.38	0.00592909926480135\\
68.39	0.00593001676459631\\
68.4	0.00593093425570441\\
68.41	0.00593185173841591\\
68.42	0.00593276921302639\\
68.43	0.00593368667983682\\
68.44	0.00593460413915357\\
68.45	0.00593552159128843\\
68.46	0.0059364390365587\\
68.47	0.00593735647528716\\
68.48	0.00593827390780213\\
68.49	0.00593919133443752\\
68.5	0.0059401087555328\\
68.51	0.00594102617143314\\
68.52	0.00594194358248935\\
68.53	0.00594286098905794\\
68.54	0.00594377839150119\\
68.55	0.00594469579018714\\
68.56	0.00594561318548963\\
68.57	0.00594653057778836\\
68.58	0.00594744796746893\\
68.59	0.0059483653549228\\
68.6	0.00594928274054744\\
68.61	0.00595020012474627\\
68.62	0.00595111750792875\\
68.63	0.0059520348905104\\
68.64	0.0059529522729128\\
68.65	0.00595386965556371\\
68.66	0.00595478703889702\\
68.67	0.00595570442335284\\
68.68	0.0059566218093775\\
68.69	0.00595753919742365\\
68.7	0.00595845658795021\\
68.71	0.00595937398142246\\
68.72	0.00596029137831208\\
68.73	0.00596120877909716\\
68.74	0.00596212618426226\\
68.75	0.00596304359429846\\
68.76	0.00596396100970335\\
68.77	0.0059648784309811\\
68.78	0.00596579585864253\\
68.79	0.00596671329320508\\
68.8	0.00596763073519288\\
68.81	0.00596854818513683\\
68.82	0.00596946564357455\\
68.83	0.00597038311105051\\
68.84	0.00597130058811603\\
68.85	0.00597221807532929\\
68.86	0.00597313557325544\\
68.87	0.00597405308246657\\
68.88	0.00597497060354179\\
68.89	0.00597588813706726\\
68.9	0.00597680568363624\\
68.91	0.0059777232438491\\
68.92	0.00597864081831343\\
68.93	0.00597955840764398\\
68.94	0.00598047601246329\\
68.95	0.00598139363340175\\
68.96	0.0059823112710977\\
68.97	0.00598322892619747\\
68.98	0.0059841465993554\\
68.99	0.00598506429123392\\
69	0.00598598200250356\\
69.01	0.00598689973384306\\
69.02	0.00598781748593936\\
69.03	0.00598873525948768\\
69.04	0.00598965305519158\\
69.05	0.00599057087376299\\
69.06	0.00599148871592226\\
69.07	0.0059924065823982\\
69.08	0.0059933244739282\\
69.09	0.00599424239125818\\
69.1	0.00599516033514272\\
69.11	0.00599607830634508\\
69.12	0.00599699630563724\\
69.13	0.00599791433379999\\
69.14	0.00599883239162295\\
69.15	0.00599975047990464\\
69.16	0.00600066859945254\\
69.17	0.00600158675108311\\
69.18	0.00600250493562187\\
69.19	0.00600342315390346\\
69.2	0.00600434140677168\\
69.21	0.00600525969507954\\
69.22	0.00600617801968935\\
69.23	0.00600709638147272\\
69.24	0.00600801478131068\\
69.25	0.00600893322009366\\
69.26	0.00600985169872161\\
69.27	0.00601077021810404\\
69.28	0.00601168877916005\\
69.29	0.00601260738281842\\
69.3	0.00601352603001765\\
69.31	0.00601444472170604\\
69.32	0.00601536345884171\\
69.33	0.00601628224239267\\
69.34	0.00601720107333692\\
69.35	0.00601811995266244\\
69.36	0.00601903888136732\\
69.37	0.00601995786045974\\
69.38	0.00602087689095811\\
69.39	0.00602179597389108\\
69.4	0.00602271511029762\\
69.41	0.00602363430122707\\
69.42	0.0060245535477392\\
69.43	0.00602547285090429\\
69.44	0.00602639221180316\\
69.45	0.00602731163152725\\
69.46	0.00602823111117871\\
69.47	0.00602915065187039\\
69.48	0.006030070254726\\
69.49	0.00603098992088006\\
69.5	0.00603190965147807\\
69.51	0.00603282944767652\\
69.52	0.00603374931064293\\
69.53	0.00603466924155599\\
69.54	0.00603558924160554\\
69.55	0.0060365093119927\\
69.56	0.0060374294539299\\
69.57	0.00603834966864096\\
69.58	0.00603926995736114\\
69.59	0.00604019032133724\\
69.6	0.00604111076182762\\
69.61	0.0060420312801023\\
69.62	0.00604295187744303\\
69.63	0.00604387255514335\\
69.64	0.00604479331450862\\
69.65	0.00604571415685616\\
69.66	0.00604663508351527\\
69.67	0.00604755609582729\\
69.68	0.00604847719514572\\
69.69	0.00604939838283625\\
69.7	0.00605031966027682\\
69.71	0.00605124102885774\\
69.72	0.00605216248998171\\
69.73	0.00605308404506392\\
69.74	0.00605400569553211\\
69.75	0.00605492744282664\\
69.76	0.00605584928840058\\
69.77	0.00605677123371975\\
69.78	0.00605769328026284\\
69.79	0.00605861542952142\\
69.8	0.00605953768300009\\
69.81	0.00606046004221648\\
69.82	0.00606138250870136\\
69.83	0.00606230508399875\\
69.84	0.0060632277696659\\
69.85	0.00606415056727348\\
69.86	0.00606507347840554\\
69.87	0.0060659965046597\\
69.88	0.00606691964764714\\
69.89	0.00606784290899272\\
69.9	0.00606876629033502\\
69.91	0.0060696897933265\\
69.92	0.00607061341963346\\
69.93	0.0060715371709362\\
69.94	0.00607246104825247\\
69.95	0.00607338505136619\\
69.96	0.00607430918006143\\
69.97	0.00607523343412236\\
69.98	0.00607615781333327\\
69.99	0.00607708231747858\\
70	0.00607800694634279\\
70.01	0.00607893169971058\\
70.02	0.00607985657736671\\
70.03	0.0060807815790961\\
70.04	0.0060817067046838\\
70.05	0.00608263195391497\\
70.06	0.00608355732657494\\
70.07	0.00608448282244917\\
70.08	0.00608540844132325\\
70.09	0.00608633418298293\\
70.1	0.00608726004721413\\
70.11	0.00608818603380288\\
70.12	0.0060891121425354\\
70.13	0.00609003837319805\\
70.14	0.00609096472557735\\
70.15	0.00609189119946001\\
70.16	0.00609281779463289\\
70.17	0.00609374451088301\\
70.18	0.00609467134799758\\
70.19	0.00609559830576398\\
70.2	0.00609652538396976\\
70.21	0.00609745258240269\\
70.22	0.00609837990085069\\
70.23	0.00609930733910187\\
70.24	0.00610023489694455\\
70.25	0.00610116257416723\\
70.26	0.00610209037055861\\
70.27	0.0061030182859076\\
70.28	0.00610394632000332\\
70.29	0.00610487447263509\\
70.3	0.00610580274359242\\
70.31	0.00610673113266508\\
70.32	0.00610765963964302\\
70.33	0.00610858826431641\\
70.34	0.00610951700647568\\
70.35	0.00611044586591145\\
70.36	0.00611137484241459\\
70.37	0.00611230393577619\\
70.38	0.00611323314578759\\
70.39	0.00611416247224037\\
70.4	0.00611509191492635\\
70.41	0.00611602147363759\\
70.42	0.00611695114816643\\
70.43	0.00611788093830541\\
70.44	0.00611881084384739\\
70.45	0.00611974086458545\\
70.46	0.00612067100031295\\
70.47	0.00612160125082349\\
70.48	0.006122531615911\\
70.49	0.00612346209536963\\
70.5	0.00612439268899383\\
70.51	0.00612532339657834\\
70.52	0.00612625421791817\\
70.53	0.00612718515280864\\
70.54	0.00612811620104533\\
70.55	0.00612904736242414\\
70.56	0.00612997863674128\\
70.57	0.00613091002379323\\
70.58	0.00613184152337681\\
70.59	0.00613277313528913\\
70.6	0.00613370485932763\\
70.61	0.00613463669529005\\
70.62	0.00613556864297447\\
70.63	0.00613650070217929\\
70.64	0.00613743287270323\\
70.65	0.00613836515434537\\
70.66	0.0061392975469051\\
70.67	0.00614023005018216\\
70.68	0.00614116266397664\\
70.69	0.00614209538808898\\
70.7	0.00614302822231996\\
70.71	0.00614396116647072\\
70.72	0.00614489422034278\\
70.73	0.00614582738373801\\
70.74	0.00614676065645863\\
70.75	0.00614769403830727\\
70.76	0.00614862752908692\\
70.77	0.00614956112860093\\
70.78	0.00615049483665306\\
70.79	0.00615142865304746\\
70.8	0.00615236257758867\\
70.81	0.0061532966100816\\
70.82	0.00615423075033159\\
70.83	0.00615516499814438\\
70.84	0.00615609935332612\\
70.85	0.00615703381568337\\
70.86	0.00615796838502311\\
70.87	0.00615890306115273\\
70.88	0.00615983784388006\\
70.89	0.00616077273301335\\
70.9	0.0061617077283613\\
70.91	0.00616264282973303\\
70.92	0.00616357803693812\\
70.93	0.00616451334978657\\
70.94	0.00616544876808888\\
70.95	0.00616638429165596\\
70.96	0.0061673199202992\\
70.97	0.00616825565383046\\
70.98	0.00616919149206205\\
70.99	0.00617012743480677\\
71	0.0061710634818779\\
71.01	0.0061719996330892\\
71.02	0.0061729358882549\\
71.03	0.00617387224718974\\
71.04	0.00617480870970896\\
71.05	0.00617574527562829\\
71.06	0.00617668194476396\\
71.07	0.00617761871693274\\
71.08	0.00617855559195187\\
71.09	0.00617949256963915\\
71.1	0.00618042964981288\\
71.11	0.00618136683229189\\
71.12	0.00618230411689555\\
71.13	0.00618324150344377\\
71.14	0.006184178991757\\
71.15	0.00618511658165623\\
71.16	0.00618605427296302\\
71.17	0.00618699206549947\\
71.18	0.00618792995908824\\
71.19	0.00618886795355258\\
71.2	0.00618980604871629\\
71.21	0.00619074424440374\\
71.22	0.00619168254043991\\
71.23	0.00619262093665036\\
71.24	0.00619355943286121\\
71.25	0.00619449802889921\\
71.26	0.0061954367245917\\
71.27	0.00619637551976662\\
71.28	0.00619731441425254\\
71.29	0.00619825340787863\\
71.3	0.00619919250047468\\
71.31	0.00620013169187113\\
71.32	0.00620107098189902\\
71.33	0.00620201037039003\\
71.34	0.00620294985717652\\
71.35	0.00620388944209147\\
71.36	0.0062048291249685\\
71.37	0.00620576890564192\\
71.38	0.00620670878394667\\
71.39	0.00620764875971838\\
71.4	0.00620858883279336\\
71.41	0.00620952900300856\\
71.42	0.00621046927020168\\
71.43	0.00621140963421105\\
71.44	0.00621235009487572\\
71.45	0.00621329065203544\\
71.46	0.00621423130553068\\
71.47	0.00621517205520258\\
71.48	0.00621611290089306\\
71.49	0.0062170538424447\\
71.5	0.00621799487970086\\
71.51	0.00621893601250559\\
71.52	0.00621987724070371\\
71.53	0.00622081856414077\\
71.54	0.0062217599826631\\
71.55	0.00622270149611773\\
71.56	0.00622364310435253\\
71.57	0.00622458480721606\\
71.58	0.00622552660455771\\
71.59	0.00622646849622763\\
71.6	0.00622741048207674\\
71.61	0.00622835256195679\\
71.62	0.00622929473572028\\
71.63	0.00623023700322056\\
71.64	0.00623117936431176\\
71.65	0.00623212181884884\\
71.66	0.00623306436668756\\
71.67	0.00623400700768453\\
71.68	0.00623494974169719\\
71.69	0.00623589256858381\\
71.7	0.00623683548820352\\
71.71	0.00623777850041627\\
71.72	0.00623872160508291\\
71.73	0.00623966480206513\\
71.74	0.00624060809122549\\
71.75	0.00624155147242743\\
71.76	0.00624249494553528\\
71.77	0.00624343851041426\\
71.78	0.00624438216693046\\
71.79	0.0062453259149509\\
71.8	0.00624626975434349\\
71.81	0.00624721368497706\\
71.82	0.00624815770672139\\
71.83	0.00624910181944712\\
71.84	0.00625004602302589\\
71.85	0.00625099031733024\\
71.86	0.00625193470223369\\
71.87	0.00625287917761067\\
71.88	0.00625382374333661\\
71.89	0.00625476839928787\\
71.9	0.00625571314534183\\
71.91	0.00625665798137681\\
71.92	0.00625760290727214\\
71.93	0.00625854792290812\\
71.94	0.00625949302816607\\
71.95	0.00626043822292831\\
71.96	0.00626138350707817\\
71.97	0.00626232888050002\\
71.98	0.00626327434307922\\
71.99	0.00626421989470221\\
72	0.00626516553525643\\
72.01	0.0062661112646304\\
72.02	0.00626705708271367\\
72.03	0.00626800298939687\\
72.04	0.00626894898457169\\
72.05	0.00626989506813091\\
72.06	0.00627084123996836\\
72.07	0.00627178749997902\\
72.08	0.0062727338480589\\
72.09	0.00627368028410517\\
72.1	0.00627462680801608\\
72.11	0.006275573419691\\
72.12	0.00627652011903045\\
72.13	0.00627746690593607\\
72.14	0.00627841378031064\\
72.15	0.00627936074205809\\
72.16	0.00628030779108352\\
72.17	0.00628125492729317\\
72.18	0.00628220215059447\\
72.19	0.00628314946089601\\
72.2	0.0062840968581076\\
72.21	0.00628504434214022\\
72.22	0.00628599191290605\\
72.23	0.00628693957031849\\
72.24	0.00628788731429215\\
72.25	0.00628883514474288\\
72.26	0.00628978306158773\\
72.27	0.00629073106474502\\
72.28	0.00629167915413432\\
72.29	0.00629262732967643\\
72.3	0.00629357559129345\\
72.31	0.00629452393890871\\
72.32	0.00629547237244686\\
72.33	0.00629642089183381\\
72.34	0.00629736949699679\\
72.35	0.0062983181878643\\
72.36	0.0062992669643662\\
72.37	0.00630021582643361\\
72.38	0.00630116477399904\\
72.39	0.0063021138069963\\
72.4	0.00630306292536055\\
72.41	0.00630401212902831\\
72.42	0.00630496141793745\\
72.43	0.00630591079202723\\
72.44	0.00630686025123827\\
72.45	0.00630780979551259\\
72.46	0.0063087594247936\\
72.47	0.0063097091390261\\
72.48	0.00631065893815633\\
72.49	0.00631160882213191\\
72.5	0.00631255879090194\\
72.51	0.00631350884441691\\
72.52	0.00631445898262881\\
72.53	0.00631540920549103\\
72.54	0.00631635951295845\\
72.55	0.00631730990498743\\
72.56	0.0063182603815358\\
72.57	0.00631921094256287\\
72.58	0.00632016158802949\\
72.59	0.00632111231789798\\
72.6	0.00632206313213217\\
72.61	0.00632301403069745\\
72.62	0.00632396501356074\\
72.63	0.00632491608069048\\
72.64	0.00632586723205667\\
72.65	0.0063268184676309\\
72.66	0.00632776978738631\\
72.67	0.00632872119129761\\
72.68	0.00632967267934112\\
72.69	0.00633062425149476\\
72.7	0.00633157590773804\\
72.71	0.00633252764805211\\
72.72	0.00633347947241973\\
72.73	0.00633443138082532\\
72.74	0.00633538337325491\\
72.75	0.00633633544969624\\
72.76	0.00633728761013865\\
72.77	0.00633823985457322\\
72.78	0.00633919218299267\\
72.79	0.00634014459539144\\
72.8	0.00634109709176565\\
72.81	0.00634204967211316\\
72.82	0.00634300233643354\\
72.83	0.0063439550847281\\
72.84	0.0063449079169999\\
72.85	0.00634586083325372\\
72.86	0.00634681383349615\\
72.87	0.00634776691773553\\
72.88	0.00634872008598199\\
72.89	0.00634967333824744\\
72.9	0.00635062667454561\\
72.91	0.00635158009489205\\
72.92	0.00635253359930411\\
72.93	0.006353487187801\\
72.94	0.00635444086040377\\
72.95	0.00635539461713532\\
72.96	0.0063563484580204\\
72.97	0.00635730238308566\\
72.98	0.00635825639235964\\
72.99	0.00635921048587276\\
73	0.00636016466365737\\
73.01	0.00636111892574771\\
73.02	0.00636207327217997\\
73.03	0.00636302770299228\\
73.04	0.00636398221822471\\
73.05	0.0063649368179193\\
73.06	0.00636589150212005\\
73.07	0.00636684627087297\\
73.08	0.00636780112422603\\
73.09	0.00636875606222925\\
73.1	0.00636971108493461\\
73.11	0.00637066619239617\\
73.12	0.00637162138466999\\
73.13	0.00637257666181421\\
73.14	0.006373532023889\\
73.15	0.00637448747095663\\
73.16	0.00637544300308145\\
73.17	0.00637639862032988\\
73.18	0.00637735432277046\\
73.19	0.00637831011047388\\
73.2	0.00637926598351291\\
73.21	0.00638022194196249\\
73.22	0.00638117798589968\\
73.23	0.00638213411540376\\
73.24	0.00638309033055614\\
73.25	0.00638404663144043\\
73.26	0.00638500301814243\\
73.27	0.00638595949075017\\
73.28	0.0063869160493539\\
73.29	0.00638787269404609\\
73.3	0.00638882942492147\\
73.31	0.00638978624207703\\
73.32	0.00639074314561201\\
73.33	0.00639170013562795\\
73.34	0.00639265721222868\\
73.35	0.00639361437552035\\
73.36	0.00639457162561141\\
73.37	0.00639552896261266\\
73.38	0.00639648638663723\\
73.39	0.00639744389780061\\
73.4	0.00639840149622066\\
73.41	0.00639935918201763\\
73.42	0.00640031695531415\\
73.43	0.00640127481623527\\
73.44	0.00640223276490844\\
73.45	0.00640319080146358\\
73.46	0.00640414892603301\\
73.47	0.00640510713875153\\
73.48	0.00640606543975641\\
73.49	0.0064070238291874\\
73.5	0.00640798230718676\\
73.51	0.00640894087389923\\
73.52	0.00640989952947212\\
73.53	0.00641085827405521\\
73.54	0.00641181710780089\\
73.55	0.00641277603086408\\
73.56	0.00641373504340229\\
73.57	0.00641469414557561\\
73.58	0.00641565333754673\\
73.59	0.00641661261948097\\
73.6	0.00641757199154626\\
73.61	0.00641853145391318\\
73.62	0.00641949100675498\\
73.63	0.00642045065024757\\
73.64	0.00642141038456952\\
73.65	0.00642237020990215\\
73.66	0.00642333012642946\\
73.67	0.00642429013433816\\
73.68	0.00642525023381773\\
73.69	0.00642621042506038\\
73.7	0.00642717070826112\\
73.71	0.00642813108361769\\
73.72	0.00642909155133068\\
73.73	0.00643005211160345\\
73.74	0.00643101276464222\\
73.75	0.006431973510656\\
73.76	0.00643293434985671\\
73.77	0.00643389528245909\\
73.78	0.00643485630868079\\
73.79	0.00643581742874234\\
73.8	0.0064367786428672\\
73.81	0.00643773995128175\\
73.82	0.00643870135421529\\
73.83	0.0064396628519001\\
73.84	0.00644062444457143\\
73.85	0.00644158613246751\\
73.86	0.00644254791582955\\
73.87	0.00644350979490182\\
73.88	0.00644447176993157\\
73.89	0.00644543384116913\\
73.9	0.00644639600886789\\
73.91	0.00644735827328432\\
73.92	0.00644832063467794\\
73.93	0.00644928309331143\\
73.94	0.00645024564945056\\
73.95	0.00645120830336426\\
73.96	0.00645217105532459\\
73.97	0.00645313390560681\\
73.98	0.00645409685448933\\
73.99	0.0064550599022538\\
74	0.00645602304918505\\
74.01	0.00645698629557117\\
74.02	0.00645794964170349\\
74.03	0.00645891308787659\\
74.04	0.00645987663438836\\
74.05	0.00646084028153997\\
74.06	0.00646180402963589\\
74.07	0.00646276787898396\\
74.08	0.00646373182989534\\
74.09	0.00646469588268455\\
74.1	0.00646566003766949\\
74.11	0.00646662429517147\\
74.12	0.00646758865551521\\
74.13	0.00646855311902883\\
74.14	0.00646951768604394\\
74.15	0.0064704823568956\\
74.16	0.00647144713192233\\
74.17	0.00647241201146615\\
74.18	0.00647337699587262\\
74.19	0.00647434208549081\\
74.2	0.00647530728067334\\
74.21	0.0064762725817764\\
74.22	0.00647723798915978\\
74.23	0.00647820350318683\\
74.24	0.00647916912422455\\
74.25	0.00648013485264356\\
74.26	0.00648110068881815\\
74.27	0.00648206663312627\\
74.28	0.00648303268594955\\
74.29	0.00648399884767334\\
74.3	0.0064849651186867\\
74.31	0.00648593149938245\\
74.32	0.00648689799015716\\
74.33	0.00648786459141119\\
74.34	0.00648883130354867\\
74.35	0.00648979812697757\\
74.36	0.00649076506210969\\
74.37	0.00649173210936067\\
74.38	0.00649269926915002\\
74.39	0.00649366654190115\\
74.4	0.00649463392804139\\
74.41	0.00649560142800195\\
74.42	0.00649656904221803\\
74.43	0.00649753677112878\\
74.44	0.00649850461517731\\
74.45	0.00649947257481077\\
74.46	0.00650044065048032\\
74.47	0.00650140884264112\\
74.48	0.00650237715175245\\
74.49	0.00650334557827763\\
74.5	0.0065043141226841\\
74.51	0.00650528278544337\\
74.52	0.00650625156703116\\
74.53	0.00650722046792728\\
74.54	0.00650818948861576\\
74.55	0.00650915862958481\\
74.56	0.00651012789132685\\
74.57	0.00651109727433854\\
74.58	0.00651206677912081\\
74.59	0.00651303640617884\\
74.6	0.00651400615602214\\
74.61	0.00651497602916449\\
74.62	0.00651594602612404\\
74.63	0.00651691614742331\\
74.64	0.00651788639358916\\
74.65	0.00651885676515288\\
74.66	0.00651982726265015\\
74.67	0.00652079788662112\\
74.68	0.00652176863761037\\
74.69	0.006522739516167\\
74.7	0.00652371052284456\\
74.71	0.00652468165820119\\
74.72	0.00652565292279952\\
74.73	0.00652662431720676\\
74.74	0.00652759584199473\\
74.75	0.00652856749773983\\
74.76	0.0065295392850231\\
74.77	0.00653051120443024\\
74.78	0.00653148325655161\\
74.79	0.00653245544198229\\
74.8	0.00653342776132206\\
74.81	0.00653440021517543\\
74.82	0.00653537280415169\\
74.83	0.0065363455288649\\
74.84	0.00653731838993393\\
74.85	0.0065382913879825\\
74.86	0.00653926452363916\\
74.87	0.00654023779753731\\
74.88	0.0065412112103153\\
74.89	0.00654218476261634\\
74.9	0.00654315845508864\\
74.91	0.00654413228838533\\
74.92	0.00654510626316453\\
74.93	0.00654608038008941\\
74.94	0.00654705463982813\\
74.95	0.00654802904305393\\
74.96	0.00654900359044511\\
74.97	0.0065499782826851\\
74.98	0.00655095312046243\\
74.99	0.00655192810447081\\
75	0.0065529032354091\\
75.01	0.00655387851398136\\
75.02	0.00655485394089689\\
75.03	0.00655582951687023\\
75.04	0.00655680524262115\\
75.05	0.00655778111887479\\
75.06	0.00655875714636155\\
75.07	0.0065597333258172\\
75.08	0.00656070965798286\\
75.09	0.00656168614360506\\
75.1	0.00656266278343573\\
75.11	0.00656363957823226\\
75.12	0.00656461652875749\\
75.13	0.00656559363577976\\
75.14	0.00656657090007293\\
75.15	0.0065675483224164\\
75.16	0.00656852590359511\\
75.17	0.00656950364439961\\
75.18	0.00657048154562608\\
75.19	0.00657145960807632\\
75.2	0.00657243783255779\\
75.21	0.00657341621988368\\
75.22	0.00657439477087285\\
75.23	0.00657537348634992\\
75.24	0.00657635236714529\\
75.25	0.00657733141409514\\
75.26	0.00657831062804146\\
75.27	0.00657929000983213\\
75.28	0.00658026956032085\\
75.29	0.00658124928036725\\
75.3	0.00658222917083686\\
75.31	0.00658320923260118\\
75.32	0.00658418946653769\\
75.33	0.00658516987352985\\
75.34	0.00658615045446716\\
75.35	0.00658713121024521\\
75.36	0.00658811214176562\\
75.37	0.00658909324993615\\
75.38	0.0065900745356707\\
75.39	0.00659105599988933\\
75.4	0.00659203764351829\\
75.41	0.00659301946749005\\
75.42	0.00659400147274332\\
75.43	0.0065949836602231\\
75.44	0.00659596603088069\\
75.45	0.0065969485856737\\
75.46	0.00659793132556612\\
75.47	0.00659891425152833\\
75.48	0.00659989736453711\\
75.49	0.00660088066557567\\
75.5	0.00660186415563372\\
75.51	0.00660284783570745\\
75.52	0.00660383170640257\\
75.53	0.00660481576804852\\
75.54	0.00660580002097601\\
75.55	0.00660678446551707\\
75.56	0.00660776910200499\\
75.57	0.00660875393077437\\
75.58	0.00660973895216111\\
75.59	0.00661072416650236\\
75.6	0.00661170957413657\\
75.61	0.00661269517540349\\
75.62	0.0066136809706441\\
75.63	0.00661466696020068\\
75.64	0.00661565314441678\\
75.65	0.00661663952363722\\
75.66	0.00661762609820808\\
75.67	0.0066186128684767\\
75.68	0.00661959983479166\\
75.69	0.00662058699750284\\
75.7	0.00662157435696133\\
75.71	0.0066225619135195\\
75.72	0.00662354966753092\\
75.73	0.00662453761935045\\
75.74	0.00662552576933417\\
75.75	0.00662651411783937\\
75.76	0.00662750266522461\\
75.77	0.00662849141184965\\
75.78	0.00662948035807549\\
75.79	0.00663046950426434\\
75.8	0.00663145885077961\\
75.81	0.00663244839798596\\
75.82	0.00663343814624921\\
75.83	0.00663442809593642\\
75.84	0.00663541824741583\\
75.85	0.00663640860105688\\
75.86	0.0066373991572302\\
75.87	0.00663838991630759\\
75.88	0.00663938087866205\\
75.89	0.00664037204466775\\
75.9	0.00664136341470004\\
75.91	0.00664235498913541\\
75.92	0.00664334676835152\\
75.93	0.00664433875272721\\
75.94	0.00664533094264246\\
75.95	0.00664632333847837\\
75.96	0.00664731594061721\\
75.97	0.00664830874944237\\
75.98	0.0066493017653384\\
75.99	0.00665029498869092\\
76	0.00665128841988673\\
76.01	0.00665228205931369\\
76.02	0.0066532759073608\\
76.03	0.00665426996441817\\
76.04	0.00665526423087696\\
76.05	0.00665625870712947\\
76.06	0.00665725339356905\\
76.07	0.00665824829059015\\
76.08	0.00665924339858829\\
76.09	0.00666023871796003\\
76.1	0.00666123424910301\\
76.11	0.00666222999241593\\
76.12	0.0066632259482985\\
76.13	0.00666422211715152\\
76.14	0.00666521849937676\\
76.15	0.00666621509537708\\
76.16	0.0066672119055563\\
76.17	0.0066682089303193\\
76.18	0.00666920617007191\\
76.19	0.00667020362522101\\
76.2	0.00667120129617445\\
76.21	0.00667219918334104\\
76.22	0.00667319728713059\\
76.23	0.00667419560795385\\
76.24	0.00667519414622257\\
76.25	0.00667619290234941\\
76.26	0.00667719187674798\\
76.27	0.00667819106983283\\
76.28	0.00667919048201945\\
76.29	0.00668019011372421\\
76.3	0.00668118996536445\\
76.31	0.00668219003735833\\
76.32	0.00668319033012497\\
76.33	0.00668419084408434\\
76.34	0.0066851915796573\\
76.35	0.00668619253726555\\
76.36	0.00668719371733168\\
76.37	0.00668819512027909\\
76.38	0.00668919674653205\\
76.39	0.00669019859651564\\
76.4	0.00669120067065576\\
76.41	0.00669220296937913\\
76.42	0.00669320549311325\\
76.43	0.00669420824228642\\
76.44	0.00669521121732773\\
76.45	0.00669621441866702\\
76.46	0.0066972178467349\\
76.47	0.00669822150196272\\
76.48	0.00669922538478256\\
76.49	0.00670022949562726\\
76.5	0.00670123383493033\\
76.51	0.00670223840312603\\
76.52	0.00670324320064929\\
76.53	0.00670424822793572\\
76.54	0.00670525348542161\\
76.55	0.00670625897354393\\
76.56	0.00670726469274024\\
76.57	0.00670827064344882\\
76.58	0.00670927682610849\\
76.59	0.00671028324115876\\
76.6	0.00671128988903969\\
76.61	0.00671229677019194\\
76.62	0.00671330388505677\\
76.63	0.00671431123407597\\
76.64	0.00671531881769192\\
76.65	0.0067163266363475\\
76.66	0.00671733469048615\\
76.67	0.00671834298055181\\
76.68	0.0067193515069889\\
76.69	0.00672036027024236\\
76.7	0.00672136927075759\\
76.71	0.00672237850898044\\
76.72	0.00672338798535723\\
76.73	0.00672439770033469\\
76.74	0.00672540765435998\\
76.75	0.00672641784788066\\
76.76	0.00672742828134468\\
76.77	0.00672843895520036\\
76.78	0.00672944986989639\\
76.79	0.00673046102565315\\
76.8	0.00673147242264405\\
76.81	0.00673248406104336\\
76.82	0.00673349594102618\\
76.83	0.00673450806276843\\
76.84	0.00673552042644693\\
76.85	0.00673653303223929\\
76.86	0.00673754588032398\\
76.87	0.00673855897088033\\
76.88	0.0067395723040885\\
76.89	0.00674058588012951\\
76.9	0.00674159969918521\\
76.91	0.00674261376143831\\
76.92	0.00674362806707236\\
76.93	0.00674464261627178\\
76.94	0.00674565740922181\\
76.95	0.00674667244610856\\
76.96	0.00674768772711899\\
76.97	0.00674870325244089\\
76.98	0.00674971902226294\\
76.99	0.00675073503677462\\
77	0.00675175129616632\\
77.01	0.00675276780062926\\
77.02	0.00675378455035549\\
77.03	0.00675480154553794\\
77.04	0.00675581878637039\\
77.05	0.00675683627304747\\
77.06	0.00675785400576467\\
77.07	0.00675887198471832\\
77.08	0.00675989021010564\\
77.09	0.00676090868212467\\
77.1	0.00676192740097431\\
77.11	0.00676294636685435\\
77.12	0.0067639655799654\\
77.13	0.00676498504050893\\
77.14	0.00676600474868729\\
77.15	0.00676702470470366\\
77.16	0.0067680449087621\\
77.17	0.00676906536106751\\
77.18	0.00677008606182566\\
77.19	0.00677110701124317\\
77.2	0.00677212820952751\\
77.21	0.00677314965688702\\
77.22	0.00677417135353089\\
77.23	0.00677519329966917\\
77.24	0.00677621549551276\\
77.25	0.00677723794127344\\
77.26	0.0067782606371638\\
77.27	0.00677928358339734\\
77.28	0.00678030678018838\\
77.29	0.00678133022775211\\
77.3	0.00678235392630458\\
77.31	0.00678337787606268\\
77.32	0.00678440207724416\\
77.33	0.00678542653006763\\
77.34	0.00678645123475256\\
77.35	0.00678747619151926\\
77.36	0.0067885014005889\\
77.37	0.00678952686218351\\
77.38	0.00679055257652596\\
77.39	0.00679157854383997\\
77.4	0.00679260476435013\\
77.41	0.00679363123828186\\
77.42	0.00679465796586146\\
77.43	0.00679568494731604\\
77.44	0.00679671218287358\\
77.45	0.00679773967276292\\
77.46	0.00679876741721372\\
77.47	0.00679979541645651\\
77.48	0.00680082367072265\\
77.49	0.00680185218024436\\
77.5	0.00680288094525469\\
77.51	0.00680390996598756\\
77.52	0.00680493924267768\\
77.53	0.00680596877556065\\
77.54	0.0068069985648729\\
77.55	0.00680802861085169\\
77.56	0.0068090589137351\\
77.57	0.00681008947376211\\
77.58	0.00681112029117246\\
77.59	0.00681215136620679\\
77.6	0.00681318269910652\\
77.61	0.00681421429011393\\
77.62	0.00681524613947214\\
77.63	0.00681627824742508\\
77.64	0.00681731061421751\\
77.65	0.00681834324009504\\
77.66	0.00681937612530406\\
77.67	0.00682040927009184\\
77.68	0.00682144267470644\\
77.69	0.00682247633939674\\
77.7	0.00682351026441245\\
77.71	0.00682454445000409\\
77.72	0.006825578896423\\
77.73	0.00682661360392135\\
77.74	0.00682764857275209\\
77.75	0.006828683803169\\
77.76	0.00682971929542667\\
77.77	0.0068307550497805\\
77.78	0.00683179106648669\\
77.79	0.00683282734580224\\
77.8	0.00683386388798496\\
77.81	0.00683490069329344\\
77.82	0.00683593776198709\\
77.83	0.00683697509432611\\
77.84	0.00683801269057148\\
77.85	0.006839050550985\\
77.86	0.00684008867582921\\
77.87	0.00684112706536749\\
77.88	0.00684216571986396\\
77.89	0.00684320463958355\\
77.9	0.00684424382479195\\
77.91	0.00684528327575564\\
77.92	0.00684632299274187\\
77.93	0.00684736297601865\\
77.94	0.00684840322585479\\
77.95	0.00684944374251982\\
77.96	0.00685048452628407\\
77.97	0.0068515255774186\\
77.98	0.00685256689619526\\
77.99	0.00685360848288665\\
78	0.00685465033776609\\
78.01	0.00685569246110767\\
78.02	0.00685673485318623\\
78.03	0.00685777751427735\\
78.04	0.00685882044465735\\
78.05	0.00685986364460327\\
78.06	0.0068609071143929\\
78.07	0.00686195085430476\\
78.08	0.00686299486461808\\
78.09	0.00686403914561283\\
78.1	0.00686508369756969\\
78.11	0.00686612852077005\\
78.12	0.00686717361549604\\
78.13	0.00686821898203047\\
78.14	0.00686926462065685\\
78.15	0.00687031053165941\\
78.16	0.00687135671532308\\
78.17	0.00687240317193346\\
78.18	0.00687344990177687\\
78.19	0.00687449690514028\\
78.2	0.00687554418231136\\
78.21	0.00687659173357848\\
78.22	0.00687763955923064\\
78.23	0.00687868765955753\\
78.24	0.00687973603484951\\
78.25	0.00688078468539758\\
78.26	0.00688183361149342\\
78.27	0.00688288281342936\\
78.28	0.00688393229149834\\
78.29	0.00688498204599399\\
78.3	0.00688603207721056\\
78.31	0.00688708238544291\\
78.32	0.00688813297098657\\
78.33	0.00688918383413765\\
78.34	0.00689023497519291\\
78.35	0.00689128639444972\\
78.36	0.00689233809220604\\
78.37	0.00689339006876044\\
78.38	0.00689444232441211\\
78.39	0.0068954948594608\\
78.4	0.00689654767420687\\
78.41	0.00689760076895126\\
78.42	0.00689865414399547\\
78.43	0.00689970779964159\\
78.44	0.00690076173619227\\
78.45	0.00690181595395072\\
78.46	0.00690287045322071\\
78.47	0.00690392523430653\\
78.48	0.00690498029751306\\
78.49	0.00690603564314568\\
78.5	0.00690709127151032\\
78.51	0.00690814718291342\\
78.52	0.00690920337766195\\
78.53	0.0069102598560634\\
78.54	0.00691131661842573\\
78.55	0.00691237366505744\\
78.56	0.00691343099626751\\
78.57	0.00691448861236541\\
78.58	0.00691554651366108\\
78.59	0.00691660470046494\\
78.6	0.00691766317308787\\
78.61	0.00691872193184123\\
78.62	0.00691978097703681\\
78.63	0.00692084030898685\\
78.64	0.00692189992800404\\
78.65	0.00692295983440151\\
78.66	0.00692402002849279\\
78.67	0.00692508051059184\\
78.68	0.00692614128101303\\
78.69	0.00692720234007113\\
78.7	0.00692826368808131\\
78.71	0.00692932532535913\\
78.72	0.00693038725222051\\
78.73	0.00693144946898177\\
78.74	0.00693251197595958\\
78.75	0.00693357477347095\\
78.76	0.00693463786183328\\
78.77	0.00693570124136426\\
78.78	0.00693676491238195\\
78.79	0.00693782887520471\\
78.8	0.00693889313015123\\
78.81	0.0069399576775405\\
78.82	0.00694102251769181\\
78.83	0.00694208765092473\\
78.84	0.00694315307755913\\
78.85	0.00694421879791512\\
78.86	0.00694528481231311\\
78.87	0.00694635112107373\\
78.88	0.00694741772451788\\
78.89	0.00694848462296668\\
78.9	0.00694955181674147\\
78.91	0.00695061930616383\\
78.92	0.00695168709155552\\
78.93	0.00695275517323852\\
78.94	0.00695382355153497\\
78.95	0.00695489222676722\\
78.96	0.00695596119925777\\
78.97	0.00695703046932927\\
78.98	0.00695810003730453\\
78.99	0.00695916990350648\\
79	0.0069602400682582\\
79.01	0.00696131053188288\\
79.02	0.0069623812947038\\
79.03	0.00696345235704436\\
79.04	0.00696452371922802\\
79.05	0.00696559538157832\\
79.06	0.00696666734441886\\
79.07	0.00696773960807332\\
79.08	0.00696881217286538\\
79.09	0.00696988503911876\\
79.1	0.00697095820715721\\
79.11	0.00697203167730448\\
79.12	0.00697310544988431\\
79.13	0.00697417952522042\\
79.14	0.00697525390363651\\
79.15	0.00697632858545624\\
79.16	0.0069774035710032\\
79.17	0.00697847886060091\\
79.18	0.00697955445457285\\
79.19	0.00698063035324237\\
79.2	0.00698170655693274\\
79.21	0.00698278306596711\\
79.22	0.00698385988066847\\
79.23	0.00698493700135973\\
79.24	0.00698601442836358\\
79.25	0.00698709216200258\\
79.26	0.00698817020259911\\
79.27	0.00698924855047534\\
79.28	0.00699032720595323\\
79.29	0.00699140616935455\\
79.3	0.00699248544100077\\
79.31	0.00699356502121319\\
79.32	0.00699464491031278\\
79.33	0.00699572510862026\\
79.34	0.00699680561645607\\
79.35	0.00699788643414032\\
79.36	0.0069989675619928\\
79.37	0.00700004900033299\\
79.38	0.00700113074947999\\
79.39	0.00700221280975254\\
79.4	0.00700329518146902\\
79.41	0.00700437786494738\\
79.42	0.00700546086050518\\
79.43	0.00700654416845955\\
79.44	0.00700762778912718\\
79.45	0.00700871172282429\\
79.46	0.00700979596986663\\
79.47	0.00701088053056947\\
79.48	0.00701196540524755\\
79.49	0.00701305059421512\\
79.5	0.00701413609778585\\
79.51	0.00701522191627289\\
79.52	0.00701630804998879\\
79.53	0.00701739449924552\\
79.54	0.00701848126435444\\
79.55	0.0070195683456263\\
79.56	0.00702065574337118\\
79.57	0.00702174345789851\\
79.58	0.00702283148951707\\
79.59	0.00702391983853491\\
79.6	0.00702500850525936\\
79.61	0.00702609748999707\\
79.62	0.00702718679305387\\
79.63	0.00702827641473488\\
79.64	0.0070293663553444\\
79.65	0.00703045661518593\\
79.66	0.00703154719456216\\
79.67	0.00703263809377493\\
79.68	0.00703372931312518\\
79.69	0.00703482085291302\\
79.7	0.00703591271343762\\
79.71	0.00703700489499726\\
79.72	0.00703809739788923\\
79.73	0.0070391902224099\\
79.74	0.00704028336885466\\
79.75	0.00704137683751785\\
79.76	0.00704247062869283\\
79.77	0.00704356474267188\\
79.78	0.00704465917974626\\
79.79	0.0070457539402061\\
79.8	0.00704684902434044\\
79.81	0.00704794443243718\\
79.82	0.00704904016478307\\
79.83	0.0070501362216637\\
79.84	0.00705123260336345\\
79.85	0.00705232931016549\\
79.86	0.00705342634235172\\
79.87	0.00705452370020283\\
79.88	0.00705562138399819\\
79.89	0.00705671939401585\\
79.9	0.00705781773053254\\
79.91	0.00705891639382365\\
79.92	0.00706001538416316\\
79.93	0.00706111470182368\\
79.94	0.00706221434707636\\
79.95	0.00706331432019093\\
79.96	0.0070644146214356\\
79.97	0.00706551525107713\\
79.98	0.00706661620938072\\
79.99	0.00706771749661004\\
80	0.00706881911302716\\
80.01	0.00706992105889258\\
};
\addplot [color=green,solid]
  table[row sep=crcr]{%
80.01	0.00706992105889258\\
80.02	0.00707102333446515\\
80.03	0.00707212594000207\\
80.04	0.00707322887575887\\
80.05	0.00707433214198938\\
80.06	0.00707543573894569\\
80.07	0.00707653966687812\\
80.08	0.00707764392603524\\
80.09	0.00707874851666377\\
80.1	0.00707985343900862\\
80.11	0.00708095869331281\\
80.12	0.00708206427981749\\
80.13	0.00708317019876189\\
80.14	0.00708427645038324\\
80.15	0.00708538303491685\\
80.16	0.00708648995259601\\
80.17	0.00708759720365195\\
80.18	0.00708870478831386\\
80.19	0.00708981270680881\\
80.2	0.00709092095936179\\
80.21	0.00709202954619558\\
80.22	0.00709313846753084\\
80.23	0.00709424772358594\\
80.24	0.00709535731457708\\
80.25	0.00709646724071814\\
80.26	0.00709757750222071\\
80.27	0.00709868809929404\\
80.28	0.00709979903214502\\
80.29	0.00710091030097813\\
80.3	0.00710202190599541\\
80.31	0.00710313384739646\\
80.32	0.00710424612537837\\
80.33	0.0071053587401357\\
80.34	0.00710647169186045\\
80.35	0.00710758498074202\\
80.36	0.00710869860696719\\
80.37	0.00710981257072009\\
80.38	0.00711092687218212\\
80.39	0.00711204151153199\\
80.4	0.00711315648894561\\
80.41	0.0071142718045961\\
80.42	0.00711538745865561\\
80.43	0.00711650345129651\\
80.44	0.00711761978269146\\
80.45	0.00711873645301335\\
80.46	0.00711985346243536\\
80.47	0.00712097081113091\\
80.48	0.00712208849927367\\
80.49	0.00712320652703758\\
80.5	0.00712432489459684\\
80.51	0.0071254436021259\\
80.52	0.00712656264979945\\
80.53	0.00712768203779246\\
80.54	0.00712880176628013\\
80.55	0.00712992183543793\\
80.56	0.00713104224544158\\
80.57	0.00713216299646703\\
80.58	0.00713328408869052\\
80.59	0.0071344055222885\\
80.6	0.00713552729743768\\
80.61	0.00713664941431505\\
80.62	0.00713777187309781\\
80.63	0.00713889467396342\\
80.64	0.00714001781708959\\
80.65	0.00714114130265427\\
80.66	0.00714226513083565\\
80.67	0.00714338930181219\\
80.68	0.00714451381576256\\
80.69	0.00714563867286569\\
80.7	0.00714676387330076\\
80.71	0.00714788941724717\\
80.72	0.00714901530488458\\
80.73	0.00715014153639288\\
80.74	0.0071512681119522\\
80.75	0.00715239503174289\\
80.76	0.00715352229594558\\
80.77	0.0071546499047411\\
80.78	0.00715577785831054\\
80.79	0.0071569061568352\\
80.8	0.00715803480049661\\
80.81	0.00715916378947658\\
80.82	0.00716029312395711\\
80.83	0.00716142280412043\\
80.84	0.00716255283014903\\
80.85	0.0071636832022256\\
80.86	0.00716481392053309\\
80.87	0.00716594498525464\\
80.88	0.00716707639657365\\
80.89	0.00716820815467372\\
80.9	0.0071693402597387\\
80.91	0.00717047271195265\\
80.92	0.00717160551149985\\
80.93	0.00717273865856481\\
80.94	0.00717387215333226\\
80.95	0.00717500599598716\\
80.96	0.00717614018671467\\
80.97	0.00717727472570017\\
80.98	0.00717840961312929\\
80.99	0.00717954484918784\\
81	0.00718068043406186\\
81.01	0.0071818163679376\\
81.02	0.00718295265100153\\
81.03	0.00718408928344034\\
81.04	0.00718522626544092\\
81.05	0.00718636359719036\\
81.06	0.00718750127887599\\
81.07	0.00718863931068533\\
81.08	0.00718977769280611\\
81.09	0.00719091642542626\\
81.1	0.00719205550873392\\
81.11	0.00719319494291746\\
81.12	0.00719433472816541\\
81.13	0.00719547486466653\\
81.14	0.00719661535260978\\
81.15	0.00719775619218431\\
81.16	0.00719889738357949\\
81.17	0.00720003892698485\\
81.18	0.00720118082259016\\
81.19	0.00720232307058537\\
81.2	0.00720346567116061\\
81.21	0.00720460862450622\\
81.22	0.00720575193081273\\
81.23	0.00720689559027086\\
81.24	0.00720803960307152\\
81.25	0.00720918396940582\\
81.26	0.00721032868946504\\
81.27	0.00721147376344066\\
81.28	0.00721261919152434\\
81.29	0.00721376497390793\\
81.3	0.00721491111078345\\
81.31	0.00721605760234314\\
81.32	0.00721720444877936\\
81.33	0.0072183516502847\\
81.34	0.00721949920705192\\
81.35	0.00722064711927393\\
81.36	0.00722179538714386\\
81.37	0.00722294401085499\\
81.38	0.00722409299060076\\
81.39	0.0072252423265748\\
81.4	0.00722639201897092\\
81.41	0.00722754206798309\\
81.42	0.00722869247380545\\
81.43	0.0072298432366323\\
81.44	0.00723099435665812\\
81.45	0.00723214583407754\\
81.46	0.00723329766908538\\
81.47	0.00723444986187658\\
81.48	0.00723560241264629\\
81.49	0.00723675532158978\\
81.5	0.0072379085889025\\
81.51	0.00723906221478005\\
81.52	0.0072402161994182\\
81.53	0.00724137054301285\\
81.54	0.00724252524576006\\
81.55	0.00724368030785607\\
81.56	0.00724483572949722\\
81.57	0.00724599151088006\\
81.58	0.00724714765220123\\
81.59	0.00724830415365755\\
81.6	0.00724946101544599\\
81.61	0.00725061823776363\\
81.62	0.00725177582080773\\
81.63	0.00725293376477567\\
81.64	0.00725409206986496\\
81.65	0.00725525073627328\\
81.66	0.00725640976419843\\
81.67	0.00725756915383832\\
81.68	0.00725872890539105\\
81.69	0.00725988901905479\\
81.7	0.0072610494950279\\
81.71	0.00726221033350882\\
81.72	0.00726337153469616\\
81.73	0.00726453309878861\\
81.74	0.00726569502598502\\
81.75	0.00726685731648436\\
81.76	0.00726801997048571\\
81.77	0.00726918298818828\\
81.78	0.00727034636979141\\
81.79	0.00727151011549452\\
81.8	0.00727267422549718\\
81.81	0.00727383869999906\\
81.82	0.00727500353919996\\
81.83	0.00727616874329978\\
81.84	0.00727733431249851\\
81.85	0.00727850024699628\\
81.86	0.00727966654699333\\
81.87	0.00728083321268996\\
81.88	0.00728200024428662\\
81.89	0.00728316764198385\\
81.9	0.00728433540598229\\
81.91	0.00728550353648267\\
81.92	0.00728667203368582\\
81.93	0.00728784089779268\\
81.94	0.00728901012900426\\
81.95	0.00729017972752169\\
81.96	0.00729134969354617\\
81.97	0.00729252002727901\\
81.98	0.00729369072892159\\
81.99	0.00729486179867538\\
82	0.00729603323674194\\
82.01	0.0072972050433229\\
82.02	0.00729837721862\\
82.03	0.00729954976283503\\
82.04	0.00730072267616987\\
82.05	0.00730189595882648\\
82.06	0.00730306961100689\\
82.07	0.0073042436329132\\
82.08	0.00730541802474758\\
82.09	0.00730659278671229\\
82.1	0.00730776791900963\\
82.11	0.007308943421842\\
82.12	0.00731011929541182\\
82.13	0.00731129553992161\\
82.14	0.00731247215557394\\
82.15	0.00731364914257144\\
82.16	0.00731482650111678\\
82.17	0.00731600423141273\\
82.18	0.00731718233366207\\
82.19	0.00731836080806765\\
82.2	0.00731953965483239\\
82.21	0.00732071887415921\\
82.22	0.00732189846625113\\
82.23	0.00732307843131119\\
82.24	0.00732425876954247\\
82.25	0.00732543948114811\\
82.26	0.00732662056633128\\
82.27	0.00732780202529519\\
82.28	0.00732898385824308\\
82.29	0.00733016606537823\\
82.3	0.00733134864690397\\
82.31	0.00733253160302363\\
82.32	0.00733371493394061\\
82.33	0.00733489863985829\\
82.34	0.00733608272098011\\
82.35	0.00733726717750953\\
82.36	0.00733845200965003\\
82.37	0.00733963721760511\\
82.38	0.00734082280157829\\
82.39	0.0073420087617731\\
82.4	0.0073431950983931\\
82.41	0.00734438181164184\\
82.42	0.00734556890172291\\
82.43	0.0073467563688399\\
82.44	0.00734794421319639\\
82.45	0.007349132434996\\
82.46	0.00735032103444232\\
82.47	0.00735151001173897\\
82.48	0.00735269936708955\\
82.49	0.00735388910069767\\
82.5	0.00735507921276693\\
82.51	0.00735626970350094\\
82.52	0.00735746057310329\\
82.53	0.00735865182177756\\
82.54	0.00735984344972733\\
82.55	0.00736103545715615\\
82.56	0.00736222784426757\\
82.57	0.00736342061126513\\
82.58	0.00736461375835235\\
82.59	0.00736580728573269\\
82.6	0.00736700119360966\\
82.61	0.00736819548218668\\
82.62	0.00736939015166719\\
82.63	0.00737058520225457\\
82.64	0.00737178063415219\\
82.65	0.0073729764475634\\
82.66	0.00737417264269147\\
82.67	0.00737536921973969\\
82.68	0.0073765661789113\\
82.69	0.00737776352040945\\
82.7	0.00737896124443733\\
82.71	0.00738015935119803\\
82.72	0.00738135784089461\\
82.73	0.00738255671373009\\
82.74	0.00738375596990744\\
82.75	0.00738495560962959\\
82.76	0.00738615563309938\\
82.77	0.00738735604051965\\
82.78	0.00738855683209315\\
82.79	0.00738975800802258\\
82.8	0.00739095956851056\\
82.81	0.0073921615137597\\
82.82	0.0073933638439725\\
82.83	0.00739456655935141\\
82.84	0.00739576966009882\\
82.85	0.00739697314641704\\
82.86	0.00739817701850832\\
82.87	0.00739938127657482\\
82.88	0.00740058592081865\\
82.89	0.00740179095144182\\
82.9	0.00740299636864628\\
82.91	0.0074042021726339\\
82.92	0.00740540836360644\\
82.93	0.00740661494176562\\
82.94	0.00740782190731303\\
82.95	0.00740902926045021\\
82.96	0.00741023700137859\\
82.97	0.00741144513029951\\
82.98	0.00741265364741422\\
82.99	0.00741386255292388\\
83	0.00741507184702955\\
83.01	0.00741628152993219\\
83.02	0.00741749160183265\\
83.03	0.0074187020629317\\
83.04	0.00741991291342999\\
83.05	0.00742112415352807\\
83.06	0.00742233578342637\\
83.07	0.00742354780332524\\
83.08	0.00742476021342488\\
83.09	0.0074259730139254\\
83.1	0.00742718620502681\\
83.11	0.00742839978692897\\
83.12	0.00742961375983164\\
83.13	0.00743082812393446\\
83.14	0.00743204287943695\\
83.15	0.00743325802653849\\
83.16	0.00743447356543835\\
83.17	0.00743568949633567\\
83.18	0.00743690581942946\\
83.19	0.0074381225349186\\
83.2	0.00743933964300183\\
83.21	0.00744055714387778\\
83.22	0.00744177503774491\\
83.23	0.00744299332480158\\
83.24	0.00744421200524597\\
83.25	0.00744543107927615\\
83.26	0.00744665054709004\\
83.27	0.00744787040888541\\
83.28	0.00744909066485989\\
83.29	0.00745031131521097\\
83.3	0.00745153236013596\\
83.31	0.00745275379983205\\
83.32	0.00745397563449627\\
83.33	0.00745519786432549\\
83.34	0.00745642048951643\\
83.35	0.00745764351026566\\
83.36	0.00745886692676957\\
83.37	0.00746009073922442\\
83.38	0.00746131494782627\\
83.39	0.00746253955277104\\
83.4	0.0074637645542545\\
83.41	0.00746498995247222\\
83.42	0.00746621574761963\\
83.43	0.00746744193989196\\
83.44	0.00746866852948431\\
83.45	0.00746989551659157\\
83.46	0.00747112290140848\\
83.47	0.00747235068412959\\
83.48	0.00747357886494928\\
83.49	0.00747480744406175\\
83.5	0.00747603642166103\\
83.51	0.00747726579794096\\
83.52	0.00747849557309518\\
83.53	0.00747972574731719\\
83.54	0.00748095632080027\\
83.55	0.00748218729373751\\
83.56	0.00748341866632186\\
83.57	0.00748465043874601\\
83.58	0.00748588261120252\\
83.59	0.00748711518388373\\
83.6	0.00748834815698178\\
83.61	0.00748958153068865\\
83.62	0.0074908153051961\\
83.63	0.00749204948069568\\
83.64	0.00749328405737877\\
83.65	0.00749451903543655\\
83.66	0.00749575441505997\\
83.67	0.00749699019643983\\
83.68	0.00749822637976667\\
83.69	0.00749946296523087\\
83.7	0.00750069995302259\\
83.71	0.00750193734333178\\
83.72	0.00750317513634819\\
83.73	0.00750441333226138\\
83.74	0.00750565193126066\\
83.75	0.00750689093353518\\
83.76	0.00750813033927383\\
83.77	0.00750937014866532\\
83.78	0.00751061036189816\\
83.79	0.00751185097916061\\
83.8	0.00751309200064074\\
83.81	0.00751433342652641\\
83.82	0.00751557525700524\\
83.83	0.00751681749226468\\
83.84	0.0075180601324919\\
83.85	0.0075193031778739\\
83.86	0.00752054662859746\\
83.87	0.00752179048484912\\
83.88	0.0075230347468152\\
83.89	0.00752427941468183\\
83.9	0.00752552448863488\\
83.91	0.00752676996886002\\
83.92	0.00752801585554272\\
83.93	0.00752926214886817\\
83.94	0.00753050884902141\\
83.95	0.00753175595618718\\
83.96	0.00753300347055005\\
83.97	0.00753425139229436\\
83.98	0.00753549972160421\\
83.99	0.00753674845866349\\
84	0.00753799760365583\\
84.01	0.0075392471567647\\
84.02	0.00754049711817328\\
84.03	0.00754174748806456\\
84.04	0.00754299826662129\\
84.05	0.007544249454026\\
84.06	0.007545501050461\\
84.07	0.00754675305610836\\
84.08	0.00754800547114994\\
84.09	0.00754925829576735\\
84.1	0.007550511530142\\
84.11	0.00755176517445506\\
84.12	0.00755301922888766\\
84.13	0.0075542736936211\\
84.14	0.00755552856883683\\
84.15	0.00755678385471648\\
84.16	0.00755803955144182\\
84.17	0.00755929565919483\\
84.18	0.00756055217815759\\
84.19	0.00756180910851239\\
84.2	0.00756306645044166\\
84.21	0.00756432420412798\\
84.22	0.0075655823697541\\
84.23	0.00756684094750295\\
84.24	0.00756809993755756\\
84.25	0.00756935934010118\\
84.26	0.00757061915531717\\
84.27	0.00757187938338907\\
84.28	0.00757314002450055\\
84.29	0.00757440107883546\\
84.3	0.00757566254657779\\
84.31	0.00757692442791167\\
84.32	0.00757818672302141\\
84.33	0.00757944943209144\\
84.34	0.00758071255530635\\
84.35	0.00758197609285088\\
84.36	0.00758324004490992\\
84.37	0.00758450441166851\\
84.38	0.00758576919331182\\
84.39	0.00758703439002517\\
84.4	0.00758830000199404\\
84.41	0.00758956602940404\\
84.42	0.0075908324724409\\
84.43	0.00759209933129055\\
84.44	0.007593366606139\\
84.45	0.00759463429717243\\
84.46	0.00759590240457717\\
84.47	0.00759717092853964\\
84.48	0.00759843986924645\\
84.49	0.00759970922688432\\
84.5	0.00760097900164011\\
84.51	0.00760224919370081\\
84.52	0.00760351980325356\\
84.53	0.0076047908304856\\
84.54	0.00760606227558434\\
84.55	0.00760733413873728\\
84.56	0.00760860642013209\\
84.57	0.00760987911995653\\
84.58	0.00761115223839853\\
84.59	0.0076124257756461\\
84.6	0.00761369973188742\\
84.61	0.00761497410731075\\
84.62	0.00761624890210453\\
84.63	0.00761752411645727\\
84.64	0.00761879975055762\\
84.65	0.00762007580459436\\
84.66	0.00762135227875637\\
84.67	0.00762262917323268\\
84.68	0.0076239064882124\\
84.69	0.0076251842238848\\
84.7	0.00762646238043923\\
84.71	0.00762774095806515\\
84.72	0.00762901995695217\\
84.73	0.00763029937729\\
84.74	0.00763157921926843\\
84.75	0.0076328594830774\\
84.76	0.00763414016890694\\
84.77	0.00763542127694719\\
84.78	0.0076367028073884\\
84.79	0.00763798476042093\\
84.8	0.00763926713623524\\
84.81	0.0076405499350219\\
84.82	0.00764183315697156\\
84.83	0.00764311680227502\\
84.84	0.00764440087112312\\
84.85	0.00764568536370685\\
84.86	0.00764697028021728\\
84.87	0.00764825562084556\\
84.88	0.00764954138578296\\
84.89	0.00765082757522085\\
84.9	0.00765211418935067\\
84.91	0.00765340122836397\\
84.92	0.00765468869245239\\
84.93	0.00765597658180765\\
84.94	0.00765726489662155\\
84.95	0.007658553637086\\
84.96	0.00765984280339296\\
84.97	0.0076611323957345\\
84.98	0.00766242241430274\\
84.99	0.00766371285928988\\
85	0.0076650037308882\\
85.01	0.00766629502929004\\
85.02	0.0076675867546878\\
85.03	0.00766887890727396\\
85.04	0.00767017148724105\\
85.05	0.00767146449478166\\
85.06	0.00767275793008844\\
85.07	0.00767405179335408\\
85.08	0.00767534608477133\\
85.09	0.00767664080453299\\
85.1	0.00767793595283189\\
85.11	0.0076792315298609\\
85.12	0.00768052753581296\\
85.13	0.00768182397088101\\
85.14	0.00768312083525804\\
85.15	0.00768441812913708\\
85.16	0.00768571585271114\\
85.17	0.00768701400617332\\
85.18	0.00768831258971669\\
85.19	0.00768961160353436\\
85.2	0.00769091104781945\\
85.21	0.00769221092276509\\
85.22	0.00769351122856444\\
85.23	0.00769481196541063\\
85.24	0.00769611313349679\\
85.25	0.00769741473301611\\
85.26	0.0076987167641617\\
85.27	0.00770001922712671\\
85.28	0.00770132212210427\\
85.29	0.00770262544928749\\
85.3	0.00770392920886946\\
85.31	0.00770523340104325\\
85.32	0.00770653802600192\\
85.33	0.00770784308393849\\
85.34	0.00770914857504595\\
85.35	0.00771045449951726\\
85.36	0.00771176085754535\\
85.37	0.00771306764932307\\
85.38	0.00771437487504328\\
85.39	0.00771568253489876\\
85.4	0.00771699062908223\\
85.41	0.00771829915778638\\
85.42	0.00771960812120382\\
85.43	0.00772091751952711\\
85.44	0.00772222735294874\\
85.45	0.00772353762166112\\
85.46	0.00772484832585659\\
85.47	0.00772615946572742\\
85.48	0.00772747104146579\\
85.49	0.00772878305326379\\
85.5	0.00773009550131342\\
85.51	0.00773140838580661\\
85.52	0.00773272170693515\\
85.53	0.00773403546489076\\
85.54	0.00773534965986505\\
85.55	0.0077366642920495\\
85.56	0.00773797936163551\\
85.57	0.00773929486881431\\
85.58	0.00774061081377707\\
85.59	0.00774192719671477\\
85.6	0.00774324401781831\\
85.61	0.00774456127727842\\
85.62	0.00774587897528571\\
85.63	0.00774719711203063\\
85.64	0.00774851568770351\\
85.65	0.00774983470249448\\
85.66	0.00775115415659355\\
85.67	0.00775247405019056\\
85.68	0.00775379438347517\\
85.69	0.00775511515663688\\
85.7	0.00775643636986502\\
85.71	0.00775775802334872\\
85.72	0.00775908011727694\\
85.73	0.00776040265183844\\
85.74	0.00776172562722179\\
85.75	0.00776304904361535\\
85.76	0.00776437290120731\\
85.77	0.0077656972001856\\
85.78	0.00776702194073798\\
85.79	0.00776834712305195\\
85.8	0.00776967274731482\\
85.81	0.00777099881371365\\
85.82	0.00777232532243528\\
85.83	0.00777365227366629\\
85.84	0.00777497966759304\\
85.85	0.00777630750440162\\
85.86	0.00777763578427786\\
85.87	0.00777896450740736\\
85.88	0.00778029367397541\\
85.89	0.00778162328416708\\
85.9	0.00778295333816711\\
85.91	0.00778428383616\\
85.92	0.00778561477832993\\
85.93	0.00778694616486083\\
85.94	0.00778827799593627\\
85.95	0.00778961027173957\\
85.96	0.0077909429924537\\
85.97	0.00779227615826134\\
85.98	0.00779360976934484\\
85.99	0.00779494382588623\\
86	0.00779627832806718\\
86.01	0.00779761327606906\\
86.02	0.00779894867007285\\
86.03	0.00780028451025922\\
86.04	0.00780162079680847\\
86.05	0.00780295752990051\\
86.06	0.00780429470971492\\
86.07	0.00780563233643088\\
86.08	0.0078069704102272\\
86.09	0.00780830893128229\\
86.1	0.00780964789977418\\
86.11	0.00781098731588049\\
86.12	0.00781232717977842\\
86.13	0.0078136674916448\\
86.14	0.00781500825165599\\
86.15	0.00781634945998794\\
86.16	0.00781769111681618\\
86.17	0.0078190332223158\\
86.18	0.00782037577666142\\
86.19	0.00782171878002723\\
86.2	0.00782306223258695\\
86.21	0.00782440613451383\\
86.22	0.00782575048598065\\
86.23	0.00782709528715973\\
86.24	0.00782844053822285\\
86.25	0.00782978623934136\\
86.26	0.00783113239068605\\
86.27	0.00783247899242724\\
86.28	0.00783382604473472\\
86.29	0.00783517354777776\\
86.3	0.00783652150172509\\
86.31	0.00783786990674491\\
86.32	0.00783921876300489\\
86.33	0.00784056807067211\\
86.34	0.00784191782991312\\
86.35	0.00784326804089389\\
86.36	0.00784461870377982\\
86.37	0.00784596981873573\\
86.38	0.00784732138592582\\
86.39	0.00784867340551374\\
86.4	0.0078500258776625\\
86.41	0.00785137880253449\\
86.42	0.00785273218029152\\
86.43	0.00785408601109472\\
86.44	0.0078554402951046\\
86.45	0.00785679503248103\\
86.46	0.00785815022338323\\
86.47	0.00785950586796974\\
86.48	0.00786086196639843\\
86.49	0.00786221851882651\\
86.5	0.00786357552541048\\
86.51	0.00786493298630616\\
86.52	0.00786629090166866\\
86.53	0.00786764927165236\\
86.54	0.00786900809641095\\
86.55	0.00787036737609735\\
86.56	0.00787172711086379\\
86.57	0.00787308730086169\\
86.58	0.00787444794624178\\
86.59	0.00787580904715395\\
86.6	0.00787717060374738\\
86.61	0.00787853261617043\\
86.62	0.00787989508457068\\
86.63	0.0078812580090949\\
86.64	0.00788262138988905\\
86.65	0.00788398522709827\\
86.66	0.00788534952086686\\
86.67	0.00788671427133829\\
86.68	0.00788807947865518\\
86.69	0.0078894451429593\\
86.7	0.00789081126439152\\
86.71	0.00789217784309185\\
86.72	0.00789354487919942\\
86.73	0.00789491237285246\\
86.74	0.00789628032418828\\
86.75	0.00789764873334326\\
86.76	0.00789901760045288\\
86.77	0.00790038692565167\\
86.78	0.0079017567090732\\
86.79	0.00790312695085009\\
86.8	0.00790449765111401\\
86.81	0.00790586880999561\\
86.82	0.0079072404276251\\
86.83	0.00790861250413245\\
86.84	0.00790998503964736\\
86.85	0.00791135803429931\\
86.86	0.00791273148821752\\
86.87	0.00791410540153096\\
86.88	0.00791547977436833\\
86.89	0.00791685460685812\\
86.9	0.00791822989912851\\
86.91	0.00791960565130747\\
86.92	0.00792098186352269\\
86.93	0.0079223585359016\\
86.94	0.00792373566857136\\
86.95	0.00792511326165888\\
86.96	0.00792649131529079\\
86.97	0.00792786982959348\\
86.98	0.00792924880469303\\
86.99	0.00793062824071528\\
87	0.00793200813778579\\
87.01	0.00793338849602983\\
87.02	0.00793476931557242\\
87.03	0.00793615059653828\\
87.04	0.00793753233905187\\
87.05	0.00793891454323735\\
87.06	0.00794029720921861\\
87.07	0.00794168033711925\\
87.08	0.00794306392706258\\
87.09	0.00794444797917164\\
87.1	0.00794583249356914\\
87.11	0.00794721747037756\\
87.12	0.00794860290971904\\
87.13	0.00794998881171543\\
87.14	0.0079513751764883\\
87.15	0.00795276200415891\\
87.16	0.00795414929484824\\
87.17	0.00795553704867694\\
87.18	0.00795692526576537\\
87.19	0.00795831394623359\\
87.2	0.00795970309020136\\
87.21	0.00796109269778811\\
87.22	0.00796248276911298\\
87.23	0.00796387330429478\\
87.24	0.00796526430345204\\
87.25	0.00796665576670296\\
87.26	0.00796804769416539\\
87.27	0.00796944008595692\\
87.28	0.00797083294219478\\
87.29	0.00797222626299591\\
87.3	0.00797362004847689\\
87.31	0.00797501429875401\\
87.32	0.00797640901394322\\
87.33	0.00797780419416014\\
87.34	0.00797919983952008\\
87.35	0.00798059595013801\\
87.36	0.00798199252612855\\
87.37	0.007983389567606\\
87.38	0.00798478707468436\\
87.39	0.00798618504747723\\
87.4	0.00798758348609793\\
87.41	0.0079889823906594\\
87.42	0.00799038176127426\\
87.43	0.00799178159805478\\
87.44	0.0079931819011129\\
87.45	0.0079945826705602\\
87.46	0.00799598390650793\\
87.47	0.00799738560906697\\
87.48	0.00799878777834786\\
87.49	0.0080001904144608\\
87.5	0.00800159351751563\\
87.51	0.00800299708762183\\
87.52	0.00800440112488853\\
87.53	0.00800580562942451\\
87.54	0.0080072106013382\\
87.55	0.00800861604073764\\
87.56	0.00801002194773054\\
87.57	0.00801142832242422\\
87.58	0.00801283516492568\\
87.59	0.00801424247534151\\
87.6	0.00801565025377796\\
87.61	0.0080170585003409\\
87.62	0.00801846721513585\\
87.63	0.00801987639826794\\
87.64	0.00802128604984194\\
87.65	0.00802269616996226\\
87.66	0.0080241067587329\\
87.67	0.00802551781625753\\
87.68	0.00802692934263941\\
87.69	0.00802834133798144\\
87.7	0.00802975380238613\\
87.71	0.00803116673595564\\
87.72	0.0080325801387917\\
87.73	0.00803399401099571\\
87.74	0.00803540835266866\\
87.75	0.00803682316391115\\
87.76	0.00803823844482342\\
87.77	0.00803965419550531\\
87.78	0.00804107041605626\\
87.79	0.00804248710657535\\
87.8	0.00804390426716124\\
87.81	0.00804532189791223\\
87.82	0.00804673999892621\\
87.83	0.00804815857030068\\
87.84	0.00804957761213276\\
87.85	0.00805099712451916\\
87.86	0.00805241710755621\\
87.87	0.00805383756133982\\
87.88	0.00805525848596553\\
87.89	0.00805667988152847\\
87.9	0.00805810174812337\\
87.91	0.00805952408584458\\
87.92	0.00806094689478602\\
87.93	0.00806237017504122\\
87.94	0.00806379392670332\\
87.95	0.00806521814986505\\
87.96	0.00806664284461873\\
87.97	0.0080680680110563\\
87.98	0.00806949364926926\\
87.99	0.00807091975934874\\
88	0.00807234634138543\\
88.01	0.00807377339546965\\
88.02	0.00807520092169129\\
88.03	0.00807662892013983\\
88.04	0.00807805739090435\\
88.05	0.00807948633407354\\
88.06	0.00808091574973563\\
88.07	0.0080823456379785\\
88.08	0.00808377599888959\\
88.09	0.00808520683255591\\
88.1	0.00808663813906411\\
88.11	0.00808806991850038\\
88.12	0.00808950217095052\\
88.13	0.00809093489649992\\
88.14	0.00809236809523356\\
88.15	0.008093801767236\\
88.16	0.00809523591259138\\
88.17	0.00809667053138345\\
88.18	0.00809810562369552\\
88.19	0.00809954118961051\\
88.2	0.00810097722921091\\
88.21	0.00810241374257881\\
88.22	0.00810385072979588\\
88.23	0.00810528819094337\\
88.24	0.00810672612610212\\
88.25	0.00810816453535257\\
88.26	0.00810960341877471\\
88.27	0.00811104277644818\\
88.28	0.00811248260845214\\
88.29	0.00811392291486537\\
88.3	0.00811536369576623\\
88.31	0.00811680495123268\\
88.32	0.00811824668134225\\
88.33	0.00811968888617207\\
88.34	0.00812113156579884\\
88.35	0.00812257472029887\\
88.36	0.00812401834974804\\
88.37	0.00812546245422184\\
88.38	0.00812690703379533\\
88.39	0.00812835208854317\\
88.4	0.0081297976185396\\
88.41	0.00813124362385848\\
88.42	0.00813269010457322\\
88.43	0.00813413706075686\\
88.44	0.00813558449248201\\
88.45	0.00813703239982087\\
88.46	0.00813848078284526\\
88.47	0.00813992964162658\\
88.48	0.00814137897623581\\
88.49	0.00814282878674356\\
88.5	0.00814427907322002\\
88.51	0.00814572983573498\\
88.52	0.00814718107435782\\
88.53	0.00814863278915753\\
88.54	0.00815008498020272\\
88.55	0.00815153764756157\\
88.56	0.00815299079130189\\
88.57	0.00815444441149108\\
88.58	0.00815589850819616\\
88.59	0.00815735308148375\\
88.6	0.00815880813142007\\
88.61	0.00816026365807097\\
88.62	0.0081617196615019\\
88.63	0.00816317614177793\\
88.64	0.00816463309896374\\
88.65	0.00816609053312363\\
88.66	0.00816754844432152\\
88.67	0.00816900683262094\\
88.68	0.00817046569808507\\
88.69	0.00817192504077668\\
88.7	0.00817338486075818\\
88.71	0.00817484515809161\\
88.72	0.00817630593283863\\
88.73	0.00817776718506055\\
88.74	0.00817922891481829\\
88.75	0.00818069112217244\\
88.76	0.00818215380718318\\
88.77	0.00818361696991037\\
88.78	0.00818508061041348\\
88.79	0.00818654472875167\\
88.8	0.00818800932498372\\
88.81	0.00818947439916803\\
88.82	0.0081909399513627\\
88.83	0.00819240598162546\\
88.84	0.00819387249001372\\
88.85	0.00819533947658451\\
88.86	0.00819680694139456\\
88.87	0.00819827488450025\\
88.88	0.00819974330595763\\
88.89	0.00820121220582241\\
88.9	0.00820268158415\\
88.91	0.00820415144099548\\
88.92	0.00820562177641357\\
88.93	0.00820709259045874\\
88.94	0.0082085638831851\\
88.95	0.00821003565464646\\
88.96	0.00821150790489634\\
88.97	0.00821298063398793\\
88.98	0.00821445384197416\\
88.99	0.00821592752890762\\
89	0.00821740169484063\\
89.01	0.00821887633982523\\
89.02	0.00822035146391317\\
89.03	0.00822182706715589\\
89.04	0.0082233031496046\\
89.05	0.00822477971131021\\
89.06	0.00822625675232337\\
89.07	0.00822773427269446\\
89.08	0.0082292122724736\\
89.09	0.00823069075171066\\
89.1	0.00823216971045525\\
89.11	0.00823364914875676\\
89.12	0.00823512906666431\\
89.13	0.00823660946422678\\
89.14	0.00823809034149283\\
89.15	0.00823957169851089\\
89.16	0.00824105353532917\\
89.17	0.00824253585199566\\
89.18	0.00824401864855813\\
89.19	0.00824550192506416\\
89.2	0.00824698568156108\\
89.21	0.00824846991809606\\
89.22	0.00824995463471602\\
89.23	0.00825143983146769\\
89.24	0.00825292550839758\\
89.25	0.00825441166555201\\
89.26	0.00825589830297705\\
89.27	0.00825738542071859\\
89.28	0.0082588730188223\\
89.29	0.00826036109733364\\
89.3	0.00826184965629785\\
89.31	0.00826333869575996\\
89.32	0.00826482821576479\\
89.33	0.00826631821635695\\
89.34	0.00826780869758085\\
89.35	0.00826929965948065\\
89.36	0.00827079110210033\\
89.37	0.00827228302548365\\
89.38	0.00827377542967414\\
89.39	0.00827526831471515\\
89.4	0.00827676168064977\\
89.41	0.00827825552752092\\
89.42	0.00827974985537128\\
89.43	0.00828124466424333\\
89.44	0.00828273995417932\\
89.45	0.00828423572522129\\
89.46	0.00828573197741107\\
89.47	0.00828722871079028\\
89.48	0.00828872592540033\\
89.49	0.00829022362128238\\
89.5	0.00829172179847741\\
89.51	0.00829322045702617\\
89.52	0.0082947195969692\\
89.53	0.00829621921834682\\
89.54	0.00829771932119913\\
89.55	0.00829921990556603\\
89.56	0.00830072097148718\\
89.57	0.00830222251900204\\
89.58	0.00830372454814985\\
89.59	0.00830522705896964\\
89.6	0.00830673005150019\\
89.61	0.00830823352578012\\
89.62	0.00830973748184778\\
89.63	0.00831124191974133\\
89.64	0.00831274683949871\\
89.65	0.00831425224115763\\
89.66	0.0083157581247556\\
89.67	0.0083172644903299\\
89.68	0.00831877133791759\\
89.69	0.00832027866755552\\
89.7	0.00832178647928032\\
89.71	0.0083232947731284\\
89.72	0.00832480354913594\\
89.73	0.00832631280733894\\
89.74	0.00832782254777313\\
89.75	0.00832933277047405\\
89.76	0.00833084347547703\\
89.77	0.00833235466281715\\
89.78	0.0083338663325293\\
89.79	0.00833537848464813\\
89.8	0.00833689111920809\\
89.81	0.00833840423624339\\
89.82	0.00833991783578804\\
89.83	0.00834143191787581\\
89.84	0.00834294648254027\\
89.85	0.00834446152981475\\
89.86	0.00834597705973237\\
89.87	0.00834749307232604\\
89.88	0.00834900956762844\\
89.89	0.00835052654567203\\
89.9	0.00835204400648903\\
89.91	0.00835356195011148\\
89.92	0.00835508037657117\\
89.93	0.00835659928589968\\
89.94	0.00835811867812836\\
89.95	0.00835963855328834\\
89.96	0.00836115891141055\\
89.97	0.00836267975252568\\
89.98	0.00836420107666418\\
89.99	0.00836572288385633\\
90	0.00836724517413213\\
90.01	0.00836876794752141\\
90.02	0.00837029120405374\\
90.03	0.00837181494375849\\
90.04	0.0083733391666648\\
90.05	0.00837486387280159\\
90.06	0.00837638906219756\\
90.07	0.00837791473488118\\
90.08	0.00837944089088071\\
90.09	0.00838096753022417\\
90.1	0.00838249465293938\\
90.11	0.00838402225905392\\
90.12	0.00838555034859515\\
90.13	0.00838707892159022\\
90.14	0.00838860797806604\\
90.15	0.00839013751804931\\
90.16	0.00839166754156649\\
90.17	0.00839319804864386\\
90.18	0.00839472903930741\\
90.19	0.00839626051358296\\
90.2	0.00839779247149609\\
90.21	0.00839932491307215\\
90.22	0.00840085783833628\\
90.23	0.00840239124731339\\
90.24	0.00840392514002815\\
90.25	0.00840545951650504\\
90.26	0.00840699437676829\\
90.27	0.00840852972084192\\
90.28	0.00841006554874971\\
90.29	0.00841160186051523\\
90.3	0.00841313865616182\\
90.31	0.0084146759357126\\
90.32	0.00841621369919046\\
90.33	0.00841775194661807\\
90.34	0.00841929067801787\\
90.35	0.00842082989341209\\
90.36	0.00842236959282272\\
90.37	0.00842390977627152\\
90.38	0.00842545044378005\\
90.39	0.00842699159536961\\
90.4	0.00842853323106131\\
90.41	0.00843007535087602\\
90.42	0.00843161795483437\\
90.43	0.00843316104295679\\
90.44	0.00843470461526347\\
90.45	0.00843624867177437\\
90.46	0.00843779321250923\\
90.47	0.00843933823748757\\
90.48	0.00844088374672869\\
90.49	0.00844242974025163\\
90.5	0.00844397621807524\\
90.51	0.00844552318021813\\
90.52	0.00844707062669867\\
90.53	0.00844861855753504\\
90.54	0.00845016697274514\\
90.55	0.0084517158723467\\
90.56	0.00845326525635718\\
90.57	0.00845481512479383\\
90.58	0.00845636547767368\\
90.59	0.0084579163150135\\
90.6	0.00845946763682988\\
90.61	0.00846101944313915\\
90.62	0.00846257173395741\\
90.63	0.00846412450930056\\
90.64	0.00846567776918424\\
90.65	0.00846723151362388\\
90.66	0.00846878574263467\\
90.67	0.00847034045623159\\
90.68	0.00847189565442936\\
90.69	0.00847345133724251\\
90.7	0.0084750075046853\\
90.71	0.00847656415677179\\
90.72	0.0084781212935158\\
90.73	0.00847967891493092\\
90.74	0.00848123702103052\\
90.75	0.00848279561182771\\
90.76	0.0084843546873354\\
90.77	0.00848591424756627\\
90.78	0.00848747429253274\\
90.79	0.00848903482224703\\
90.8	0.0084905958367211\\
90.81	0.00849215733596669\\
90.82	0.0084937193199953\\
90.83	0.0084952817888182\\
90.84	0.00849684474244641\\
90.85	0.00849840818089074\\
90.86	0.00849997210416173\\
90.87	0.00850153651226971\\
90.88	0.00850310140522474\\
90.89	0.00850466678303666\\
90.9	0.00850623264571507\\
90.91	0.00850779899326931\\
90.92	0.0085093658257085\\
90.93	0.00851093314304149\\
90.94	0.00851250094527691\\
90.95	0.00851406923242314\\
90.96	0.00851563800448829\\
90.97	0.00851720726148025\\
90.98	0.00851877700340666\\
90.99	0.00852034723027488\\
91	0.00852191794209206\\
91.01	0.00852348913886508\\
91.02	0.00852506082060056\\
91.03	0.0085266329873049\\
91.04	0.0085282056389842\\
91.05	0.00852977877564434\\
91.06	0.00853135239729094\\
91.07	0.00853292650392935\\
91.08	0.00853450109556468\\
91.09	0.00853607617220177\\
91.1	0.00853765173384521\\
91.11	0.00853922778049931\\
91.12	0.00854080431216815\\
91.13	0.00854238132885553\\
91.14	0.00854395883056499\\
91.15	0.0085455368172998\\
91.16	0.00854711528906299\\
91.17	0.0085486942458573\\
91.18	0.00855027368768521\\
91.19	0.00855185361454893\\
91.2	0.00855343402645043\\
91.21	0.00855501492339136\\
91.22	0.00855659630537314\\
91.23	0.00855817817239691\\
91.24	0.00855976052446354\\
91.25	0.00856134336157361\\
91.26	0.00856292668372745\\
91.27	0.0085645104909251\\
91.28	0.00856609478316633\\
91.29	0.00856767956045062\\
91.3	0.00856926482277721\\
91.31	0.00857085057014501\\
91.32	0.0085724368025527\\
91.33	0.00857402351999863\\
91.34	0.00857561072248092\\
91.35	0.00857719840999736\\
91.36	0.0085787865825455\\
91.37	0.00858037524012256\\
91.38	0.00858196438272552\\
91.39	0.00858355401035102\\
91.4	0.00858514412299548\\
91.41	0.00858673472065497\\
91.42	0.00858832580332531\\
91.43	0.008589917371002\\
91.44	0.00859150942368028\\
91.45	0.00859310196135506\\
91.46	0.00859469498402098\\
91.47	0.0085962884916724\\
91.48	0.00859788248430334\\
91.49	0.00859947696190757\\
91.5	0.00860107192447852\\
91.51	0.00860266737200935\\
91.52	0.00860426330449291\\
91.53	0.00860585972192175\\
91.54	0.00860745662428811\\
91.55	0.00860905401158394\\
91.56	0.00861065188380087\\
91.57	0.00861225024093024\\
91.58	0.00861384908296308\\
91.59	0.0086154484098901\\
91.6	0.00861704822170171\\
91.61	0.00861864851838801\\
91.62	0.00862024929993878\\
91.63	0.0086218505663435\\
91.64	0.00862345231759135\\
91.65	0.00862505455367115\\
91.66	0.00862665727457144\\
91.67	0.00862826048028044\\
91.68	0.00862986417078604\\
91.69	0.00863146834607582\\
91.7	0.00863307300613703\\
91.71	0.00863467815095661\\
91.72	0.00863628378052118\\
91.73	0.00863788989481701\\
91.74	0.00863949649383009\\
91.75	0.00864110357754603\\
91.76	0.00864271114595016\\
91.77	0.00864431919902745\\
91.78	0.00864592773676255\\
91.79	0.0086475367591398\\
91.8	0.00864914626614316\\
91.81	0.00865075625775629\\
91.82	0.00865236673396253\\
91.83	0.00865397769474485\\
91.84	0.0086555891400859\\
91.85	0.00865720106996799\\
91.86	0.00865881348437309\\
91.87	0.00866042638328283\\
91.88	0.00866203976667849\\
91.89	0.00866365363454102\\
91.9	0.00866526798685101\\
91.91	0.00866688282358873\\
91.92	0.00866849814473407\\
91.93	0.0086701139502666\\
91.94	0.00867173024016552\\
91.95	0.0086733470144097\\
91.96	0.00867496427297763\\
91.97	0.00867658201584747\\
91.98	0.00867820024299702\\
91.99	0.00867981895440372\\
92	0.00868143815004465\\
92.01	0.00868305782989654\\
92.02	0.00868467799393575\\
92.03	0.0086862986421383\\
92.04	0.00868791977447982\\
92.05	0.00868954139093559\\
92.06	0.00869116349148053\\
92.07	0.00869278607608917\\
92.08	0.0086944091447357\\
92.09	0.00869603269739394\\
92.1	0.00869765673403731\\
92.11	0.00869928125463888\\
92.12	0.00870090625917136\\
92.13	0.00870253174760705\\
92.14	0.00870415771991791\\
92.15	0.0087057841760755\\
92.16	0.008707411116051\\
92.17	0.00870903853981522\\
92.18	0.00871066644733859\\
92.19	0.00871229483859115\\
92.2	0.00871392371354255\\
92.21	0.00871555307216208\\
92.22	0.0087171829144186\\
92.23	0.00871881324028064\\
92.24	0.00872044404971628\\
92.25	0.00872207534269324\\
92.26	0.00872370711917884\\
92.27	0.00872533937914002\\
92.28	0.0087269721225433\\
92.29	0.00872860534935482\\
92.3	0.00873023905954031\\
92.31	0.00873187325306512\\
92.32	0.00873350792989416\\
92.33	0.00873514308999198\\
92.34	0.00873677873332271\\
92.35	0.00873841485985005\\
92.36	0.00874005146953733\\
92.37	0.00874168856234745\\
92.38	0.0087433261382429\\
92.39	0.00874496419718576\\
92.4	0.00874660273913769\\
92.41	0.00874824176405996\\
92.42	0.0087498812719134\\
92.43	0.00875152126265842\\
92.44	0.00875316173625502\\
92.45	0.00875480269266278\\
92.46	0.00875644413184085\\
92.47	0.00875808605374797\\
92.48	0.00875972845834243\\
92.49	0.0087613713455821\\
92.5	0.00876301471542445\\
92.51	0.00876465856782648\\
92.52	0.00876630290274478\\
92.53	0.00876794772013549\\
92.54	0.00876959301995434\\
92.55	0.00877123880215659\\
92.56	0.0087728850666971\\
92.57	0.00877453181353026\\
92.58	0.00877617904261002\\
92.59	0.00877782675388989\\
92.6	0.00877947494732296\\
92.61	0.00878112362286184\\
92.62	0.0087827727804587\\
92.63	0.00878442242006527\\
92.64	0.00878607254163282\\
92.65	0.00878772314511216\\
92.66	0.00878937423045367\\
92.67	0.00879102579760725\\
92.68	0.00879267784652234\\
92.69	0.00879433037714794\\
92.7	0.00879598338943258\\
92.71	0.00879763688332431\\
92.72	0.00879929085877073\\
92.73	0.00880094531571899\\
92.74	0.00880260025411574\\
92.75	0.00880425567390717\\
92.76	0.00880591157503901\\
92.77	0.0088075679574565\\
92.78	0.00880922482110441\\
92.79	0.00881088216592705\\
92.8	0.00881253999186822\\
92.81	0.00881419829887127\\
92.82	0.00881585708687904\\
92.83	0.00881751635583391\\
92.84	0.00881917610567775\\
92.85	0.00882083633635197\\
92.86	0.00882249704779746\\
92.87	0.00882415823995466\\
92.88	0.00882581991276346\\
92.89	0.00882748206616331\\
92.9	0.00882914470009313\\
92.91	0.00883080781449135\\
92.92	0.00883247140929591\\
92.93	0.00883413548444423\\
92.94	0.00883580003987323\\
92.95	0.00883746507551934\\
92.96	0.00883913059131848\\
92.97	0.00884079658720602\\
92.98	0.00884246306311689\\
92.99	0.00884413001898544\\
93	0.00884579745474554\\
93.01	0.00884746537033053\\
93.02	0.00884913376567325\\
93.03	0.00885080264070599\\
93.04	0.00885247199536054\\
93.05	0.00885414182956815\\
93.06	0.00885581214325957\\
93.07	0.00885748293636499\\
93.08	0.00885915420881408\\
93.09	0.00886082596053599\\
93.1	0.00886249819145931\\
93.11	0.00886417090151213\\
93.12	0.00886584409062197\\
93.13	0.00886751775871583\\
93.14	0.00886919190572015\\
93.15	0.00887086653156085\\
93.16	0.00887254163616327\\
93.17	0.00887421721945224\\
93.18	0.00887589328135202\\
93.19	0.00887756982178631\\
93.2	0.00887924684067827\\
93.21	0.00888092433795051\\
93.22	0.00888260231352506\\
93.23	0.00888428076732341\\
93.24	0.00888595969926648\\
93.25	0.00888763910927462\\
93.26	0.00888931899726763\\
93.27	0.00889099936316473\\
93.28	0.00889268020688456\\
93.29	0.00889436152834522\\
93.3	0.0088960433274642\\
93.31	0.00889772560415844\\
93.32	0.00889940835834428\\
93.33	0.0089010915899375\\
93.34	0.00890277529885328\\
93.35	0.00890445948500622\\
93.36	0.00890614414831035\\
93.37	0.0089078292886791\\
93.38	0.00890951490602528\\
93.39	0.00891120100026115\\
93.4	0.00891288757129837\\
93.41	0.00891457461904797\\
93.42	0.00891626214342041\\
93.43	0.00891795014432553\\
93.44	0.00891963862167258\\
93.45	0.00892132757537021\\
93.46	0.00892301700532643\\
93.47	0.00892470691144868\\
93.48	0.00892639729364375\\
93.49	0.00892808815181785\\
93.5	0.00892977948587654\\
93.51	0.00893147129572478\\
93.52	0.00893316358126691\\
93.53	0.00893485634240664\\
93.54	0.00893654957904706\\
93.55	0.00893824329109061\\
93.56	0.00893993747843913\\
93.57	0.00894163214099382\\
93.58	0.00894332727865523\\
93.59	0.00894502289132328\\
93.6	0.00894671897889726\\
93.61	0.00894841554127581\\
93.62	0.00895011257835692\\
93.63	0.00895181009003795\\
93.64	0.0089535080762156\\
93.65	0.00895520653678591\\
93.66	0.0089569054716443\\
93.67	0.0089586048806855\\
93.68	0.00896030476380359\\
93.69	0.00896200512089201\\
93.7	0.0089637059518435\\
93.71	0.00896540725655018\\
93.72	0.00896710903490347\\
93.73	0.00896881128679413\\
93.74	0.00897051401211226\\
93.75	0.00897221721074726\\
93.76	0.00897392088258787\\
93.77	0.00897562502752216\\
93.78	0.00897732964543749\\
93.79	0.00897903473622057\\
93.8	0.00898074029975741\\
93.81	0.00898244633593331\\
93.82	0.00898415284463291\\
93.83	0.00898585982574014\\
93.84	0.00898756727913825\\
93.85	0.00898927520470976\\
93.86	0.00899098360233653\\
93.87	0.00899269247189968\\
93.88	0.00899440181327965\\
93.89	0.00899611162635616\\
93.9	0.00899782191100821\\
93.91	0.00899953266711411\\
93.92	0.00900124389455144\\
93.93	0.00900295559319706\\
93.94	0.00900466776292712\\
93.95	0.00900638040361704\\
93.96	0.0090080935151415\\
93.97	0.00900980709737449\\
93.98	0.00901152115018923\\
93.99	0.00901323567345822\\
94	0.00901495066705323\\
94.01	0.0090166661308453\\
94.02	0.00901838206470469\\
94.03	0.00902009846850097\\
94.04	0.00902181534210293\\
94.05	0.00902353268537861\\
94.06	0.00902525049819532\\
94.07	0.00902696878041959\\
94.08	0.00902868753191722\\
94.09	0.00903040675255323\\
94.1	0.00903212644219189\\
94.11	0.00903384660069669\\
94.12	0.00903556722793038\\
94.13	0.00903728832375491\\
94.14	0.00903900988803149\\
94.15	0.00904073192062054\\
94.16	0.00904245442138168\\
94.17	0.00904417739017378\\
94.18	0.00904590082685492\\
94.19	0.00904762473128239\\
94.2	0.0090493491033127\\
94.21	0.00905107394280154\\
94.22	0.00905279924960386\\
94.23	0.00905452502357376\\
94.24	0.00905625126456458\\
94.25	0.00905797797242882\\
94.26	0.00905970514701821\\
94.27	0.00906143278818367\\
94.28	0.00906316089577528\\
94.29	0.00906488946964233\\
94.3	0.0090666185096333\\
94.31	0.00906834801559584\\
94.32	0.00907007798737679\\
94.33	0.00907180842482214\\
94.34	0.00907353932777709\\
94.35	0.00907527069608597\\
94.36	0.00907700252959232\\
94.37	0.00907873482813882\\
94.38	0.00908046759156731\\
94.39	0.0090822008197188\\
94.4	0.00908393451243346\\
94.41	0.00908566866955059\\
94.42	0.00908740329090868\\
94.43	0.00908913837634532\\
94.44	0.00909087392569729\\
94.45	0.00909260993880048\\
94.46	0.00909434641548995\\
94.47	0.00909608335559985\\
94.48	0.00909782075896352\\
94.49	0.00909955862541339\\
94.5	0.00910129695478102\\
94.51	0.00910303574689713\\
94.52	0.00910477500159152\\
94.53	0.00910651471869314\\
94.54	0.00910825489803002\\
94.55	0.00910999553942936\\
94.56	0.00911173664271741\\
94.57	0.00911347820771957\\
94.58	0.00911522023426031\\
94.59	0.00911696272216324\\
94.6	0.00911870567125104\\
94.61	0.00912044908134549\\
94.62	0.00912219295226747\\
94.63	0.00912393728383695\\
94.64	0.00912568207587298\\
94.65	0.00912742732819369\\
94.66	0.00912917304061632\\
94.67	0.00913091921295714\\
94.68	0.00913266584503153\\
94.69	0.00913441293665392\\
94.7	0.00913616048763784\\
94.71	0.00913790849779584\\
94.72	0.00913965696693958\\
94.73	0.00914140589487975\\
94.74	0.00914315528142609\\
94.75	0.00914490512638742\\
94.76	0.0091466554295716\\
94.77	0.00914840619078552\\
94.78	0.00915015740983513\\
94.79	0.00915190908652542\\
94.8	0.00915366122066043\\
94.81	0.00915541381204321\\
94.82	0.00915716686047584\\
94.83	0.00915892036575946\\
94.84	0.00916067432769421\\
94.85	0.00916242874607926\\
94.86	0.0091641836207128\\
94.87	0.00916593895139203\\
94.88	0.00916769473791317\\
94.89	0.00916945098007144\\
94.9	0.00917120767766108\\
94.91	0.00917296483047533\\
94.92	0.00917472243830642\\
94.93	0.00917648050094558\\
94.94	0.00917823901818305\\
94.95	0.00917999798980803\\
94.96	0.00918175741560874\\
94.97	0.00918351729537238\\
94.98	0.00918527762888509\\
94.99	0.00918703841593204\\
95	0.00918879965629735\\
95.01	0.00919056134976411\\
95.02	0.00919232349611439\\
95.03	0.00919408609512922\\
95.04	0.00919584914658857\\
95.05	0.0091976126502714\\
95.06	0.00919937660595562\\
95.07	0.00920114101341807\\
95.08	0.00920290587243456\\
95.09	0.00920467118277984\\
95.1	0.0092064369442276\\
95.11	0.00920820315655047\\
95.12	0.00920996981952\\
95.13	0.0092117369329067\\
95.14	0.00921350449648\\
95.15	0.00921527251000824\\
95.16	0.0092170409732587\\
95.17	0.00921880988599757\\
95.18	0.00922057924798995\\
95.19	0.00922234905899986\\
95.2	0.00922411931879024\\
95.21	0.00922589002712292\\
95.22	0.00922766118375863\\
95.23	0.00922943278845701\\
95.24	0.00923120484097659\\
95.25	0.00923297734107478\\
95.26	0.0092347502885079\\
95.27	0.00923652368303115\\
95.28	0.00923829752439861\\
95.29	0.00924007181236322\\
95.3	0.00924184654667683\\
95.31	0.00924362172709014\\
95.32	0.0092453973533527\\
95.33	0.00924717342521298\\
95.34	0.00924894994241825\\
95.35	0.00925072690471467\\
95.36	0.00925250431184726\\
95.37	0.00925428216355989\\
95.38	0.00925606045959524\\
95.39	0.00925783919969489\\
95.4	0.00925961838359922\\
95.41	0.00926139801104746\\
95.42	0.00926317808177769\\
95.43	0.00926495859552679\\
95.44	0.00926673955203049\\
95.45	0.00926852095102333\\
95.46	0.00927030279223869\\
95.47	0.00927208507540874\\
95.48	0.00927386780026447\\
95.49	0.00927565096653571\\
95.5	0.00927743457395104\\
95.51	0.0092792186222379\\
95.52	0.00928100311112248\\
95.53	0.0092827880403298\\
95.54	0.00928457340958366\\
95.55	0.00928635921860665\\
95.56	0.00928814546712014\\
95.57	0.00928993215484427\\
95.58	0.009291719281498\\
95.59	0.00929350684679901\\
95.6	0.00929529485046379\\
95.61	0.00929708329220759\\
95.62	0.00929887217174441\\
95.63	0.00930066148878701\\
95.64	0.00930245124304692\\
95.65	0.00930424143423443\\
95.66	0.00930603206205855\\
95.67	0.00930782312622705\\
95.68	0.00930961462644646\\
95.69	0.00931140656242202\\
95.7	0.00931319893385773\\
95.71	0.00931499174045629\\
95.72	0.00931678498191916\\
95.73	0.0093185786579465\\
95.74	0.00932037276823722\\
95.75	0.00932216731248891\\
95.76	0.00932396229039789\\
95.77	0.00932575770165921\\
95.78	0.00932755354596659\\
95.79	0.00932934982301247\\
95.8	0.009331146532488\\
95.81	0.00933294367408301\\
95.82	0.00933474124748602\\
95.83	0.00933653925238424\\
95.84	0.00933833768846357\\
95.85	0.00934013655540859\\
95.86	0.00934193585290255\\
95.87	0.00934373558062738\\
95.88	0.00934553573826368\\
95.89	0.00934733632549071\\
95.9	0.00934913734198638\\
95.91	0.0093509387874273\\
95.92	0.0093527406614887\\
95.93	0.00935454296384446\\
95.94	0.00935634569416713\\
95.95	0.00935814885212788\\
95.96	0.00935995243739653\\
95.97	0.00936175644964156\\
95.98	0.00936356088853003\\
95.99	0.00936536575372768\\
96	0.00936717104489885\\
96.01	0.0093689767617065\\
96.02	0.00937078290381222\\
96.03	0.00937258947087621\\
96.04	0.00937439646255727\\
96.05	0.00937620387851284\\
96.06	0.00937801171839891\\
96.07	0.00937981998187011\\
96.08	0.00938162866857966\\
96.09	0.00938343777817936\\
96.1	0.00938524731031961\\
96.11	0.00938705726464938\\
96.12	0.00938886764081624\\
96.13	0.00939067843846631\\
96.14	0.00939248965724433\\
96.15	0.00939430129679355\\
96.16	0.00939611335675583\\
96.17	0.00939792583677157\\
96.18	0.00939973873647974\\
96.19	0.00940155205551786\\
96.2	0.009403365793522\\
96.21	0.00940517995012678\\
96.22	0.00940699452496536\\
96.23	0.00940880951766945\\
96.24	0.00941062492786928\\
96.25	0.00941244075519361\\
96.26	0.00941425699926975\\
96.27	0.00941607365972353\\
96.28	0.00941789073617927\\
96.29	0.00941970822825985\\
96.3	0.00942152613558663\\
96.31	0.0094233444577795\\
96.32	0.00942516319445686\\
96.33	0.00942698234523557\\
96.34	0.00942880190973106\\
96.35	0.00943062188755718\\
96.36	0.00943244227832631\\
96.37	0.00943426308164933\\
96.38	0.00943608429713557\\
96.39	0.00943790592439285\\
96.4	0.00943972796302749\\
96.41	0.00944155041264424\\
96.42	0.00944337327284635\\
96.43	0.00944519654323552\\
96.44	0.00944702022341192\\
96.45	0.00944884431297417\\
96.46	0.00945066881151935\\
96.47	0.00945249371864297\\
96.48	0.00945431903393901\\
96.49	0.00945614475699988\\
96.5	0.00945797088741644\\
96.51	0.00945979742477795\\
96.52	0.00946162436867214\\
96.53	0.00946345171868514\\
96.54	0.00946527947440154\\
96.55	0.00946710763540429\\
96.56	0.00946893620127481\\
96.57	0.0094707651715929\\
96.58	0.00947259454593678\\
96.59	0.00947442432388307\\
96.6	0.0094762545050068\\
96.61	0.00947808508888138\\
96.62	0.00947991607507863\\
96.63	0.00948174746316875\\
96.64	0.00948357925272032\\
96.65	0.00948541144330031\\
96.66	0.00948724403447406\\
96.67	0.0094890770258053\\
96.68	0.00949091041685612\\
96.69	0.00949274420718696\\
96.7	0.00949457839635666\\
96.71	0.00949641298392239\\
96.72	0.00949824796943968\\
96.73	0.00950008335246241\\
96.74	0.00950191913254283\\
96.75	0.0095037553092315\\
96.76	0.00950559188207735\\
96.77	0.00950742885062762\\
96.78	0.00950926621442791\\
96.79	0.00951110397302213\\
96.8	0.00951294212595252\\
96.81	0.00951478067275965\\
96.82	0.00951661961298239\\
96.83	0.00951845894615794\\
96.84	0.0095202986718218\\
96.85	0.0095221387895078\\
96.86	0.00952397929874804\\
96.87	0.00952582019907293\\
96.88	0.00952766149001119\\
96.89	0.00952950317108982\\
96.9	0.00953134524183412\\
96.91	0.00953318770176765\\
96.92	0.00953503055041227\\
96.93	0.00953687378728812\\
96.94	0.0095387174119136\\
96.95	0.00954056142380539\\
96.96	0.00954240582247843\\
96.97	0.00954425060744593\\
96.98	0.00954609577821935\\
96.99	0.00954794133430841\\
97	0.00954978727522108\\
97.01	0.00955163360046357\\
97.02	0.00955348030954036\\
97.03	0.00955532740195414\\
97.04	0.00955717487720585\\
97.05	0.00955902273479466\\
97.06	0.00956087097421799\\
97.07	0.00956271959497146\\
97.08	0.00956456859654891\\
97.09	0.00956641797844241\\
97.1	0.00956826774014226\\
97.11	0.00957011788113695\\
97.12	0.00957196840091317\\
97.13	0.00957381929895585\\
97.14	0.00957567057474808\\
97.15	0.00957752222777117\\
97.16	0.00957937425750463\\
97.17	0.00958122666342614\\
97.18	0.00958307944501157\\
97.19	0.00958493260173499\\
97.2	0.00958678613306863\\
97.21	0.0095886400384829\\
97.22	0.00959049431744639\\
97.23	0.00959234896942586\\
97.24	0.00959420399388621\\
97.25	0.00959605939029054\\
97.26	0.00959791515810007\\
97.27	0.00959977129677421\\
97.28	0.00960162780577049\\
97.29	0.0096034846845446\\
97.3	0.00960534193255039\\
97.31	0.00960719954923981\\
97.32	0.009609057534063\\
97.33	0.00961091588646818\\
97.34	0.00961277460590173\\
97.35	0.00961463369180817\\
97.36	0.0096164931436301\\
97.37	0.00961835296080829\\
97.38	0.00962021314278158\\
97.39	0.00962207368898695\\
97.4	0.00962393459885948\\
97.41	0.00962579587183234\\
97.42	0.0096276575073368\\
97.43	0.00962951950480218\\
97.44	0.00963138186365589\\
97.45	0.00963324458332342\\
97.46	0.00963510766322832\\
97.47	0.00963697110279221\\
97.48	0.00963883490143478\\
97.49	0.00964069905857377\\
97.5	0.00964256357362499\\
97.51	0.0096444284460103\\
97.52	0.0096462936752224\\
97.53	0.00964815926075106\\
97.54	0.00965002520208313\\
97.55	0.00965189149870251\\
97.56	0.00965375815008998\\
97.57	0.00965562515572279\\
97.58	0.00965749251507467\\
97.59	0.00965936022761576\\
97.6	0.00966122829281262\\
97.61	0.00966309671012819\\
97.62	0.00966496547902173\\
97.63	0.00966683459894884\\
97.64	0.0096687040078037\\
97.65	0.00967057345106537\\
97.66	0.00967244293111522\\
97.67	0.00967431245037442\\
97.68	0.00967618201130439\\
97.69	0.00967805161640719\\
97.7	0.00967992126822601\\
97.71	0.00968179096934558\\
97.72	0.00968366072239264\\
97.73	0.00968553053003638\\
97.74	0.00968740039498893\\
97.75	0.00968927032000576\\
97.76	0.00969113540374984\\
97.77	0.00969299287980512\\
97.78	0.00969484267370487\\
97.79	0.00969668471020148\\
97.8	0.00969851895938087\\
97.81	0.00970034559432675\\
97.82	0.00970216453691259\\
97.83	0.00970397570817427\\
97.84	0.00970577902829956\\
97.85	0.00970757441661751\\
97.86	0.00970936179158764\\
97.87	0.00971114107078894\\
97.88	0.00971291217130359\\
97.89	0.00971467500961018\\
97.9	0.00971643598000225\\
97.91	0.00971819561040886\\
97.92	0.00971995389422941\\
97.93	0.00972171082501592\\
97.94	0.00972346639638159\\
97.95	0.00972522060196552\\
97.96	0.00972697245718935\\
97.97	0.00972872025635572\\
97.98	0.00973046397766682\\
97.99	0.00973220321377041\\
98	0.00973393769619323\\
98.01	0.00973566740022747\\
98.02	0.00973739230079825\\
98.03	0.00973911237272436\\
98.04	0.00974082759071918\\
98.05	0.00974253792939164\\
98.06	0.00974424336324722\\
98.07	0.0097459458132758\\
98.08	0.00974764628114663\\
98.09	0.00974934475099875\\
98.1	0.00975104120683862\\
98.11	0.00975273563253935\\
98.12	0.00975442801183999\\
98.13	0.00975611832834473\\
98.14	0.0097578065655222\\
98.15	0.00975949270670473\\
98.16	0.00976117673508765\\
98.17	0.00976285863372854\\
98.18	0.00976453838554661\\
98.19	0.00976621597288069\\
98.2	0.00976789137782039\\
98.21	0.00976956458173835\\
98.22	0.00977123556324665\\
98.23	0.0097729043007487\\
98.24	0.00977457077243703\\
98.25	0.00977623495629106\\
98.26	0.00977789683007482\\
98.27	0.00977955637133466\\
98.28	0.00978121355739692\\
98.29	0.00978286836536555\\
98.3	0.00978452077212016\\
98.31	0.00978617075431369\\
98.32	0.00978781828836994\\
98.33	0.00978946335048191\\
98.34	0.009791105916618\\
98.35	0.00979274596251981\\
98.36	0.00979438346369993\\
98.37	0.00979601839544009\\
98.38	0.00979765073278893\\
98.39	0.00979928045055972\\
98.4	0.00980090752332797\\
98.41	0.00980253192542883\\
98.42	0.00980415363095472\\
98.43	0.00980577261375538\\
98.44	0.00980738884743632\\
98.45	0.00980900230535652\\
98.46	0.00981061296074857\\
98.47	0.00981222078723336\\
98.48	0.00981382575818507\\
98.49	0.00981542784672884\\
98.5	0.00981702702573845\\
98.51	0.00981862326783386\\
98.52	0.00982021654537887\\
98.53	0.00982180683047862\\
98.54	0.00982339409490997\\
98.55	0.00982497830958456\\
98.56	0.0098265594451396\\
98.57	0.00982813747193521\\
98.58	0.00982971236005172\\
98.59	0.00983128407928698\\
98.6	0.00983285259915361\\
98.61	0.00983441788887629\\
98.62	0.00983597991738892\\
98.63	0.00983753865333183\\
98.64	0.00983909406504894\\
98.65	0.00984064612058488\\
98.66	0.00984219478768381\\
98.67	0.00984374003378718\\
98.68	0.00984528182603079\\
98.69	0.00984682013124195\\
98.7	0.00984835491593646\\
98.71	0.00984988614631569\\
98.72	0.0098514137882636\\
98.73	0.00985293780734368\\
98.74	0.0098544581687959\\
98.75	0.00985597483753364\\
98.76	0.00985748777814056\\
98.77	0.00985899695486745\\
98.78	0.00986050233162905\\
98.79	0.00986200387200082\\
98.8	0.00986350153921575\\
98.81	0.00986499529616101\\
98.82	0.0098664851053747\\
98.83	0.00986797092904249\\
98.84	0.00986945272884565\\
98.85	0.00987093046592802\\
98.86	0.00987240410107248\\
98.87	0.00987387359469748\\
98.88	0.00987533890685353\\
98.89	0.00987679999721977\\
98.9	0.00987825682510035\\
98.91	0.00987970934942086\\
98.92	0.00988115752872478\\
98.93	0.00988260132116976\\
98.94	0.00988404068452397\\
98.95	0.00988547557616236\\
98.96	0.00988690595306295\\
98.97	0.009888331771803\\
98.98	0.00988975298855519\\
98.99	0.00989116955908376\\
99	0.00989258143874062\\
99.01	0.00989398858246139\\
99.02	0.00989539094476144\\
99.03	0.00989678847973188\\
99.04	0.00989818114103547\\
99.05	0.00989956888190258\\
99.06	0.00990095165512703\\
99.07	0.00990232941306191\\
99.08	0.00990370210761538\\
99.09	0.00990506969024643\\
99.1	0.00990643211196056\\
99.11	0.00990778932330546\\
99.12	0.00990914127436665\\
99.13	0.00991048791476302\\
99.14	0.0099118291936424\\
99.15	0.00991316505967706\\
99.16	0.00991449546105915\\
99.17	0.00991582034549609\\
99.18	0.00991713966020598\\
99.19	0.00991845335191288\\
99.2	0.00991976136684211\\
99.21	0.00992106365071546\\
99.22	0.00992236014874638\\
99.23	0.00992365080563514\\
99.24	0.00992493556556388\\
99.25	0.00992621437219167\\
99.26	0.00992748716864954\\
99.27	0.00992875389753535\\
99.28	0.00993001450090877\\
99.29	0.00993126892028608\\
99.3	0.00993251709663497\\
99.31	0.00993375897036932\\
99.32	0.00993499448134388\\
99.33	0.00993622356884892\\
99.34	0.00993744617160479\\
99.35	0.00993866222775656\\
99.36	0.0099398716748684\\
99.37	0.00994107444991809\\
99.38	0.0099422704892914\\
99.39	0.00994345972877637\\
99.4	0.00994464210355766\\
99.41	0.00994581754821071\\
99.42	0.00994698599669593\\
99.43	0.00994814738235283\\
99.44	0.00994930163789402\\
99.45	0.00995044869539926\\
99.46	0.00995158848630938\\
99.47	0.00995272094142012\\
99.48	0.00995384599087603\\
99.49	0.00995496356416414\\
99.5	0.00995607359010774\\
99.51	0.00995717599685997\\
99.52	0.00995827071189741\\
99.53	0.0099593576620136\\
99.54	0.00996043677331251\\
99.55	0.0099615079712019\\
99.56	0.00996257118038667\\
99.57	0.00996362632486211\\
99.58	0.00996467332790712\\
99.59	0.00996571211207732\\
99.6	0.00996674259919812\\
99.61	0.00996776470120001\\
99.62	0.00996877832540117\\
99.63	0.00996978337821723\\
99.64	0.00997077976515262\\
99.65	0.00997176739079183\\
99.66	0.0099727461587905\\
99.67	0.00997371597186654\\
99.68	0.00997467673179107\\
99.69	0.00997562833937934\\
99.7	0.00997657069448153\\
99.71	0.00997750369597346\\
99.72	0.00997842724174722\\
99.73	0.0099793412287017\\
99.74	0.00998024555273308\\
99.75	0.00998114010872511\\
99.76	0.00998202479053945\\
99.77	0.00998289949100579\\
99.78	0.00998376410191195\\
99.79	0.00998461851399383\\
99.8	0.00998546261692534\\
99.81	0.00998629629930815\\
99.82	0.0099871194486614\\
99.83	0.00998793195141125\\
99.84	0.00998873369288044\\
99.85	0.00998952455727761\\
99.86	0.00999030442768662\\
99.87	0.00999107318605574\\
99.88	0.0099918307131867\\
99.89	0.00999257688872371\\
99.9	0.00999331159114228\\
99.91	0.00999403469773802\\
99.92	0.00999474608461527\\
99.93	0.00999544562667568\\
99.94	0.00999613319760659\\
99.95	0.00999680866986942\\
99.96	0.00999747191468782\\
99.97	0.00999812280203583\\
99.98	0.00999876120062581\\
99.99	0.00999938697789635\\
100	0.01\\
};
\addlegendentry{$q=4$};

\end{axis}
\end{tikzpicture}%
  \caption{Continuous Time}
\end{subfigure}%
\hfill%
\begin{subfigure}{.45\linewidth}
  \centering
  \setlength\figureheight{\linewidth} 
  \setlength\figurewidth{\linewidth}
  \tikzsetnextfilename{testdp_dscr_nFPC_z8}
  % This file was created by matlab2tikz.
%
%The latest updates can be retrieved from
%  http://www.mathworks.com/matlabcentral/fileexchange/22022-matlab2tikz-matlab2tikz
%where you can also make suggestions and rate matlab2tikz.
%
\definecolor{mycolor1}{rgb}{1.00000,0.00000,1.00000}%
%
\begin{tikzpicture}[trim axis left, trim axis right]

\begin{axis}[%
width=\figurewidth,
height=\figureheight,
at={(0\figurewidth,0\figureheight)},
scale only axis,
every outer x axis line/.append style={black},
every x tick label/.append style={font=\color{black}},
xmin=0,
xmax=100,
%xlabel={Time},
every outer y axis line/.append style={black},
every y tick label/.append style={font=\color{black}},
ymin=0,
ymax=0.015,
%ylabel={Depth $\delta^+$},
axis background/.style={fill=white},
axis x line*=bottom,
axis y line*=left,
yticklabel style={
        /pgf/number format/fixed,
        /pgf/number format/precision=3
},
scaled y ticks=false,
legend style={legend cell align=left,align=left,draw=black,font=\footnotesize, at={(0.98,0.02)},anchor=south east},
every axis legend/.code={\renewcommand\addlegendentry[2][]{}}  %ignore legend locally
]
\addplot [color=green,dashed]
  table[row sep=crcr]{%
1	0.00960747766527148\\
2	0.00960459404913772\\
3	0.00960158025650961\\
4	0.0095984295068883\\
5	0.0095951345600636\\
6	0.00959168766534378\\
7	0.00958808050086004\\
8	0.00958430410026429\\
9	0.00958034876332209\\
10	0.00957620394581127\\
11	0.00957185812265307\\
12	0.00956729861615749\\
13	0.00956251137851434\\
14	0.00955748071468751\\
15	0.00955218893264844\\
16	0.00954661593090592\\
17	0.0095407388453902\\
18	0.0095345309976458\\
19	0.00952796000285703\\
20	0.00952098564345573\\
21	0.00951355270940135\\
22	0.00950556094064703\\
23	0.00946440543532272\\
24	0.00940906811304978\\
25	0.00935104996681104\\
26	0.00929018477044053\\
27	0.00922629325960029\\
28	0.00915918211537875\\
29	0.00908864275242587\\
30	0.00901444544869592\\
31	0.00893634415218274\\
32	0.0088540786026562\\
33	0.00876737337954217\\
34	0.00867594157014046\\
35	0.00857949305323717\\
36	0.0084777530018618\\
37	0.00837050592739933\\
38	0.00825987440296975\\
39	0.00814615295278687\\
40	0.00802630605978886\\
41	0.00790008363139868\\
42	0.00776735839791016\\
43	0.00762796148881317\\
44	0.00748124346793057\\
45	0.00732643878621186\\
46	0.00716259451917534\\
47	0.00698855139815732\\
48	0.0068028678544128\\
49	0.00660392300924533\\
50	0.00638983923916324\\
51	0.00615832760230671\\
52	0.00590662677333882\\
53	0.00563137494727097\\
54	0.0054833445843688\\
55	0.00531886987268333\\
56	0.00513317052593558\\
57	0.00494020700596234\\
58	0.00474015268939363\\
59	0.00453339226580794\\
60	0.00432059614673591\\
61	0.00410280954628816\\
62	0.00388155907365784\\
63	0.00365897879302804\\
64	0.00343813287075068\\
65	0.00322358765361566\\
66	0.00302187850555284\\
67	0.00284255784210886\\
68	0.00266514204955049\\
69	0.0024962160076035\\
70	0.00234029063557502\\
71	0.00219095723548395\\
72	0.00204945378603438\\
73	0.00191672431015574\\
74	0.00179269703825858\\
75	0.00167417850208544\\
76	0.00155990138313519\\
77	0.00144776499195021\\
78	0.00133806332426327\\
79	0.00123085736165786\\
80	0.00112590794744774\\
81	0.001022807448086\\
82	0.000921068875316224\\
83	0.000819624925291739\\
84	0.000721158445062816\\
85	0.000627458040754679\\
86	0.000539433042325591\\
87	0.000458553981564615\\
88	0.000386070896422318\\
89	0.000319981903864034\\
90	0.000261223634796461\\
91	0.000208334376922877\\
92	0.000159855757659064\\
93	0.00011559139109104\\
94	7.58928809352733e-05\\
95	4.17334792893222e-05\\
96	1.50512561189767e-05\\
97	0\\
98	0\\
99	0\\
100	0\\
};
\addlegendentry{$q=-4$};

\addplot [color=mycolor1,dashed]
  table[row sep=crcr]{%
1	0.00983929166573282\\
2	0.00983906428205305\\
3	0.00983882586633388\\
4	0.00983857578554717\\
5	0.00983831335893858\\
6	0.00983803785259887\\
7	0.00983774847306906\\
8	0.00983744435972472\\
9	0.00983712457558571\\
10	0.0098367880960489\\
11	0.00983643379482422\\
12	0.00983606042608525\\
13	0.00983566660167668\\
14	0.00983525076264436\\
15	0.00983481114598505\\
16	0.00983434574651927\\
17	0.00983385223240286\\
18	0.00983332773461626\\
19	0.0098327684691146\\
20	0.00983216893471232\\
21	0.00983152027173897\\
22	0.00983080876757299\\
23	0.00982768744363728\\
24	0.0098235670630728\\
25	0.00981928738838518\\
26	0.00981484115706208\\
27	0.00981022092880901\\
28	0.00980541977752457\\
29	0.00980043371283091\\
30	0.00979538939892645\\
31	0.00979023857514771\\
32	0.00978489104778528\\
33	0.0097793379880434\\
34	0.00977356953153074\\
35	0.00976757324661039\\
36	0.00976132801915651\\
37	0.0097547784149741\\
38	0.00974576309461378\\
39	0.00973379495395621\\
40	0.00972139712345876\\
41	0.00970856162990793\\
42	0.00969528093924533\\
43	0.00968152053608604\\
44	0.00966724054970232\\
45	0.0096523911529444\\
46	0.00963690992005741\\
47	0.00962071712702947\\
48	0.00960372260376675\\
49	0.00958582096415723\\
50	0.00956687303578892\\
51	0.00954673634763493\\
52	0.00952522247386327\\
53	0.00950199461908947\\
54	0.00933208601272526\\
55	0.00914906495911794\\
56	0.00895583024907716\\
57	0.00875173338744622\\
58	0.00853551301503994\\
59	0.0083056635024672\\
60	0.00806037526506734\\
61	0.00779745918206924\\
62	0.00751425052909292\\
63	0.00720750982711353\\
64	0.00687319695061563\\
65	0.00650633237721363\\
66	0.00610649935007796\\
67	0.00567132953542143\\
68	0.00521149599924193\\
69	0.0047237781675188\\
70	0.00420485594144063\\
71	0.00392783165642717\\
72	0.00365005661796192\\
73	0.00337011298152657\\
74	0.00309332054940902\\
75	0.0028289491960103\\
76	0.00258588780578947\\
77	0.00237580659121768\\
78	0.00217020600928596\\
79	0.00197107065433464\\
80	0.00178060061526567\\
81	0.00160031142738287\\
82	0.00143198780801362\\
83	0.00128088177894017\\
84	0.00114016131158019\\
85	0.00100480900001626\\
86	0.0008729626779357\\
87	0.000742732348917191\\
88	0.000615967170454213\\
89	0.000496927150175562\\
90	0.000387280994571957\\
91	0.000290236695875495\\
92	0.00020881406197963\\
93	0.000141373809608583\\
94	8.68845319872729e-05\\
95	4.45546390724813e-05\\
96	1.50512561189767e-05\\
97	0\\
98	0\\
99	0\\
100	0\\
};
\addlegendentry{$q=-3$};

\addplot [color=red,dashed]
  table[row sep=crcr]{%
1	0.0098806142486084\\
2	0.00988060225975636\\
3	0.00988058968056293\\
4	0.00988057647644751\\
5	0.00988056261020699\\
6	0.00988054804174608\\
7	0.00988053272776883\\
8	0.00988051662141958\\
9	0.00988049967185485\\
10	0.00988048182371878\\
11	0.00988046301648793\\
12	0.00988044318365599\\
13	0.00988042225174318\\
14	0.00988040013902537\\
15	0.00988037675313954\\
16	0.00988035198410419\\
17	0.00988032568714141\\
18	0.0098802976502558\\
19	0.00988026753995375\\
20	0.00988023484962275\\
21	0.00988019900984474\\
22	0.00988016003052746\\
23	0.00988011864542674\\
24	0.00988007495356235\\
25	0.00988002870760913\\
26	0.0098799795577269\\
27	0.0098799268746262\\
28	0.00987986912736902\\
29	0.00987980162433733\\
30	0.00987960134057879\\
31	0.0098793057209037\\
32	0.00987898875891997\\
33	0.00987864861229315\\
34	0.00987828292615525\\
35	0.00987788794115731\\
36	0.00987745644955523\\
37	0.00987697534178638\\
38	0.0098762849839572\\
39	0.00987536352430184\\
40	0.00987440428355443\\
41	0.00987340481784767\\
42	0.00987236224504076\\
43	0.00987127335846734\\
44	0.00987013439139508\\
45	0.0098689406252867\\
46	0.00986768605948119\\
47	0.00986636329663847\\
48	0.00986496346417182\\
49	0.00986347485171751\\
50	0.00986188376801104\\
51	0.00986017258243903\\
52	0.00985831192027749\\
53	0.00985625516009128\\
54	0.00984384693209673\\
55	0.00983095122915345\\
56	0.00981783065784758\\
57	0.0098044722867266\\
58	0.00979086099535044\\
59	0.00977697937044168\\
60	0.00976280785852619\\
61	0.00974832571042725\\
62	0.00973351469664937\\
63	0.00971834670880941\\
64	0.00970274693435033\\
65	0.00968658058077449\\
66	0.0096649363114085\\
67	0.00963313382503779\\
68	0.00960054339029398\\
69	0.0095668477180941\\
70	0.00953149571949198\\
71	0.0092346191537598\\
72	0.00891070222019736\\
73	0.00856060036876301\\
74	0.00817969621194866\\
75	0.00776210868714419\\
76	0.0073004439043749\\
77	0.00678579240107872\\
78	0.00624625223967729\\
79	0.00568168757145003\\
80	0.00508776463775421\\
81	0.00446132423589887\\
82	0.0038015365938485\\
83	0.00311634866014581\\
84	0.00268381681743506\\
85	0.00240021227450085\\
86	0.00214531516129943\\
87	0.00191176180069247\\
88	0.00167818912809278\\
89	0.00144282299053618\\
90	0.0012074225550966\\
91	0.000974345149752143\\
92	0.000746407645706487\\
93	0.000531597859427775\\
94	0.000338443796331356\\
95	0.000176929185035761\\
96	5.89679114988392e-05\\
97	0\\
98	0\\
99	0\\
100	0\\
};
\addlegendentry{$q=-2$};

\addplot [color=blue,dashed]
  table[row sep=crcr]{%
1	0.00993301187418029\\
2	0.00993301123058148\\
3	0.00993301055507122\\
4	0.00993300984572929\\
5	0.00993300910047486\\
6	0.00993300831704409\\
7	0.00993300749296184\\
8	0.00993300662550481\\
9	0.00993300571165251\\
10	0.00993300474802055\\
11	0.00993300373076705\\
12	0.00993300265545041\\
13	0.00993300151677897\\
14	0.00993300030810113\\
15	0.00993299902034793\\
16	0.00993299764017695\\
17	0.00993299614749676\\
18	0.00993299451360253\\
19	0.00993299270435077\\
20	0.00993299069697789\\
21	0.00993298850991085\\
22	0.00993298618645423\\
23	0.00993298372283397\\
24	0.00993298108128142\\
25	0.0099329781843993\\
26	0.00993297486598587\\
27	0.00993297077181648\\
28	0.00993296519982485\\
29	0.00993295703059585\\
30	0.00993293727940716\\
31	0.00993290960977507\\
32	0.00993287982097473\\
33	0.00993284759829078\\
34	0.00993281245654904\\
35	0.00993277368265537\\
36	0.00993273050356053\\
37	0.00993268292431627\\
38	0.00993263231959259\\
39	0.00993257974798528\\
40	0.0099325251006028\\
41	0.00993246825507844\\
42	0.00993240907926109\\
43	0.00993234741822372\\
44	0.00993228307369699\\
45	0.00993221578663254\\
46	0.0099321452204967\\
47	0.00993207092098236\\
48	0.00993199220833431\\
49	0.00993190816985955\\
50	0.00993181762426275\\
51	0.00993171887092656\\
52	0.00993161029110649\\
53	0.00993149279918156\\
54	0.00993136949478849\\
55	0.00993123998481406\\
56	0.00993110352070128\\
57	0.0099309592013298\\
58	0.00993080592161378\\
59	0.00993064227783493\\
60	0.00993046635252949\\
61	0.00993027511840973\\
62	0.00993006254669058\\
63	0.00992981553424646\\
64	0.00992950928850722\\
65	0.00992910357465447\\
66	0.00992823294696101\\
67	0.00992659488509032\\
68	0.009924809579626\\
69	0.00992282442090743\\
70	0.00992056415130438\\
71	0.00990028399773245\\
72	0.00987949004951769\\
73	0.00985850446534031\\
74	0.00983732510932302\\
75	0.00981594960866513\\
76	0.00979440742552927\\
77	0.00977290389215302\\
78	0.00975271679676563\\
79	0.0097324172717149\\
80	0.00971343084399797\\
81	0.00969563918808831\\
82	0.0096786318169564\\
83	0.00965088127755664\\
84	0.0093466932911918\\
85	0.00887160903576858\\
86	0.00834584872617862\\
87	0.00777892140166473\\
88	0.00719204555733556\\
89	0.00658661882228475\\
90	0.0059608239085164\\
91	0.00531283826600714\\
92	0.00464134333654843\\
93	0.00394433142034477\\
94	0.00321894058908532\\
95	0.00246108721279032\\
96	0.00166403223161984\\
97	0.000826308254754571\\
98	0\\
99	0\\
100	0\\
};
\addlegendentry{$q=-1$};

\addplot [color=black,solid]
  table[row sep=crcr]{%
1	0.000624407485134332\\
2	0.000624407519915784\\
3	0.000624407556323891\\
4	0.000624407594375936\\
5	0.000624407634189318\\
6	0.000624407676445276\\
7	0.000624407722593159\\
8	0.000624407775541716\\
9	0.000624407840924937\\
10	0.00062440792887363\\
11	0.000624408055325559\\
12	0.000624408239521597\\
13	0.000624408491433687\\
14	0.000624408789812747\\
15	0.000624409095760126\\
16	0.000624409409680003\\
17	0.000624409732843658\\
18	0.000624410069180706\\
19	0.000624410429732784\\
20	0.000624410841680389\\
21	0.000624411360596894\\
22	0.000624412070382347\\
23	0.000624413069156528\\
24	0.000624414375010669\\
25	0.000624415729590759\\
26	0.000624417088696578\\
27	0.00062441841243596\\
28	0.000624419623701465\\
29	0.000624420622615931\\
30	0.000624421521762283\\
31	0.000624422411946123\\
32	0.000624423272305066\\
33	0.000624424076334788\\
34	0.000624424781981793\\
35	0.00062442532946859\\
36	0.00062442567735886\\
37	0.000624425919786399\\
38	0.00062442618043771\\
39	0.000624426466438152\\
40	0.000624426790513287\\
41	0.000624427179565305\\
42	0.000624427691434862\\
43	0.00062442844809704\\
44	0.00062442969779383\\
45	0.000624431903854089\\
46	0.000624435788265885\\
47	0.000624442007741635\\
48	0.000624449953346516\\
49	0.000624458161669218\\
50	0.000624466383979973\\
51	0.000624474468212722\\
52	0.000624482454241307\\
53	0.000624490887043724\\
54	0.000624500168306528\\
55	0.000624511205196221\\
56	0.000624526087744902\\
57	0.000624548653900136\\
58	0.000624583366510348\\
59	0.000624623327589464\\
60	0.000624666043064573\\
61	0.000624712664097867\\
62	0.000624764543152344\\
63	0.000624821910457352\\
64	0.000624880962371869\\
65	0.000624940624648512\\
66	0.000625011307902986\\
67	0.000625118049124955\\
68	0.000625283759630578\\
69	0.000625453729813828\\
70	0.000625622432522341\\
71	0.000625805723034884\\
72	0.000626010329493192\\
73	0.000626247898949499\\
74	0.0006265436206454\\
75	0.000626935651106565\\
76	0.000627363418305458\\
77	0.00062782784435962\\
78	0.000628335442994521\\
79	0.000628772360726379\\
80	0.000631557283951244\\
81	0.000634607927527444\\
82	0.000637934956761122\\
83	0.000644443068343529\\
84	0.000641738827914312\\
85	0.000624087409766743\\
86	0.000925253858090135\\
87	0.00138013119776012\\
88	0.00189154071321051\\
89	0.00246991304304022\\
90	0.003086718548447\\
91	0.00374453088362059\\
92	0.00444193080109917\\
93	0.00518263306050222\\
94	0.00597107521850114\\
95	0.00681235368486026\\
96	0.00771339553901169\\
97	0.008683018792965\\
98	0.00969870132858304\\
99	0\\
100	0\\
};
\addlegendentry{$q=0$};

\addplot [color=blue,solid]
  table[row sep=crcr]{%
1	8.38301970140858e-06\\
2	8.38386528491081e-06\\
3	8.38474919953782e-06\\
4	8.38567603025214e-06\\
5	8.38664698922755e-06\\
6	8.38766183294111e-06\\
7	8.38872497039531e-06\\
8	8.38984481688585e-06\\
9	8.39103758840753e-06\\
10	8.39233737740437e-06\\
11	8.39381703207104e-06\\
12	8.39562374315354e-06\\
13	8.39801176035047e-06\\
14	8.40126793308153e-06\\
15	8.40528131334117e-06\\
16	8.40943197882418e-06\\
17	8.41372843142993e-06\\
18	8.41818362380834e-06\\
19	8.4228221392232e-06\\
20	8.42769923837403e-06\\
21	8.43294963500891e-06\\
22	8.43890477407434e-06\\
23	8.44634055047764e-06\\
24	8.45683716744612e-06\\
25	8.47241716854141e-06\\
26	8.48895030383199e-06\\
27	8.50600959947571e-06\\
28	8.52363200865932e-06\\
29	8.54186696005317e-06\\
30	8.56077754448614e-06\\
31	8.58042681464239e-06\\
32	8.60083196266657e-06\\
33	8.62190963800097e-06\\
34	8.64365282864932e-06\\
35	8.66610622399652e-06\\
36	8.68935615797893e-06\\
37	8.71350385225643e-06\\
38	8.73861334423382e-06\\
39	8.7647580021751e-06\\
40	8.79202424669478e-06\\
41	8.82052043243899e-06\\
42	8.85040206576697e-06\\
43	8.88194627764206e-06\\
44	8.91574836941675e-06\\
45	8.95321175932681e-06\\
46	8.99775501488281e-06\\
47	9.05734617107499e-06\\
48	9.1463626579966e-06\\
49	9.26397206883214e-06\\
50	9.38846191829702e-06\\
51	9.51773126786104e-06\\
52	9.65116243962592e-06\\
53	9.78910187031467e-06\\
54	9.93198223680158e-06\\
55	1.00805294362259e-05\\
56	1.02368395141276e-05\\
57	1.04121887200929e-05\\
58	1.06447679419695e-05\\
59	1.19882924727538e-05\\
60	1.38069458053338e-05\\
61	1.57168804265525e-05\\
62	1.77344791835758e-05\\
63	1.98883795848268e-05\\
64	2.22201228382913e-05\\
65	2.47018838786381e-05\\
66	2.73319082886247e-05\\
67	3.01694621578692e-05\\
68	3.3406610600724e-05\\
69	5.54696738036023e-05\\
70	8.02426306743528e-05\\
71	0.00010655232745994\\
72	0.000134652598962992\\
73	0.000164869792889474\\
74	0.000197594714388067\\
75	0.000233467384385008\\
76	0.000536437647948485\\
77	0.000956998611238995\\
78	0.00141178194827107\\
79	0.0019069362149086\\
80	0.00244786603977846\\
81	0.00304657142035237\\
82	0.00370195182948479\\
83	0.00439447281833135\\
84	0.00510567944503527\\
85	0.00586596287169202\\
86	0.00634543106950584\\
87	0.00670970308024902\\
88	0.00705895729635169\\
89	0.00738110433715275\\
90	0.00770609030628117\\
91	0.00803174778032167\\
92	0.00836043655066926\\
93	0.00868876266121125\\
94	0.00901176881358957\\
95	0.00932269995257548\\
96	0.00959732068383334\\
97	0.00981217566362703\\
98	0.0099605845957028\\
99	0\\
100	0\\
};
\addlegendentry{$q=1$};

\addplot [color=red,solid]
  table[row sep=crcr]{%
1	1.14758002926896e-05\\
2	1.1485015126246e-05\\
3	1.14946003885409e-05\\
4	1.15046027019851e-05\\
5	1.15150925076826e-05\\
6	1.15260904882415e-05\\
7	1.15375732305172e-05\\
8	1.15495712508768e-05\\
9	1.15621408913412e-05\\
10	1.15753770190491e-05\\
11	1.15894757738868e-05\\
12	1.16049041034829e-05\\
13	1.16228263433142e-05\\
14	1.16459995832758e-05\\
15	1.16796108326478e-05\\
16	1.17248179954996e-05\\
17	1.17715582509505e-05\\
18	1.18199028602877e-05\\
19	1.18699318496234e-05\\
20	1.19217488708434e-05\\
21	1.19755343670989e-05\\
22	1.20317336984719e-05\\
23	1.20917222287143e-05\\
24	1.21601690212965e-05\\
25	1.22535150957733e-05\\
26	1.28965866048559e-05\\
27	1.36176607243976e-05\\
28	1.43624871419559e-05\\
29	1.51323807492452e-05\\
30	1.59288977496959e-05\\
31	1.67539351428091e-05\\
32	1.7609743507198e-05\\
33	1.84984767666977e-05\\
34	1.94205448101078e-05\\
35	2.0377571158722e-05\\
36	2.13720503234054e-05\\
37	2.24073356341282e-05\\
38	2.34873335793314e-05\\
39	2.46152790267093e-05\\
40	2.57947704444773e-05\\
41	2.70298230433428e-05\\
42	2.83249283013289e-05\\
43	2.96851392068227e-05\\
44	3.11163696085861e-05\\
45	3.26262856714941e-05\\
46	3.42265936580611e-05\\
47	3.59409406857809e-05\\
48	3.78388750768602e-05\\
49	4.42477420479792e-05\\
50	5.85564078128645e-05\\
51	7.35100378273756e-05\\
52	8.9085107278434e-05\\
53	0.000105307590632848\\
54	0.000122235832780331\\
55	0.000139936031545904\\
56	0.00015848122500173\\
57	0.000177928864974253\\
58	0.000198357861165649\\
59	0.000218875527543803\\
60	0.000240239807952066\\
61	0.000263020365504108\\
62	0.000287455255416057\\
63	0.000313870836040574\\
64	0.000342793087038499\\
65	0.000559823947389461\\
66	0.000876468875381666\\
67	0.00121239452801232\\
68	0.0015698556027177\\
69	0.00193310088885079\\
70	0.00232193523002754\\
71	0.00274285791819601\\
72	0.00320228465389986\\
73	0.00370803047155976\\
74	0.00427011241454861\\
75	0.0048797154565817\\
76	0.00526184613590384\\
77	0.005562692150318\\
78	0.00586621833219415\\
79	0.00616877405175283\\
80	0.00646496155186\\
81	0.00674500915762293\\
82	0.00700933776719489\\
83	0.0072682264478649\\
84	0.00751750096070531\\
85	0.00775141758586106\\
86	0.00796379421083901\\
87	0.00816690346257312\\
88	0.00836677804294878\\
89	0.0085649295074448\\
90	0.00876160044559847\\
91	0.00895579290222694\\
92	0.00914691133703123\\
93	0.00933189662901577\\
94	0.00949682983647625\\
95	0.00963734641541312\\
96	0.00976239223053009\\
97	0.00987237869677138\\
98	0.0099605845957028\\
99	0\\
100	0\\
};
\addlegendentry{$q=2$};

\addplot [color=mycolor1,solid]
  table[row sep=crcr]{%
1	3.15734924100943e-05\\
2	3.19865904928091e-05\\
3	3.24164972977663e-05\\
4	3.28644115877258e-05\\
5	3.33318347709687e-05\\
6	3.38210776111889e-05\\
7	3.43337556653169e-05\\
8	3.48706723780252e-05\\
9	3.54335202355788e-05\\
10	3.60243831190674e-05\\
11	3.66456552234368e-05\\
12	3.73002526610017e-05\\
13	3.79923360249596e-05\\
14	3.87300616630171e-05\\
15	3.95366524188967e-05\\
16	4.16909625018281e-05\\
17	4.73360540906365e-05\\
18	5.31918706648171e-05\\
19	5.92689338736221e-05\\
20	6.5578146279307e-05\\
21	7.21305802997466e-05\\
22	7.89370078170671e-05\\
23	8.60067579608657e-05\\
24	9.33446543655605e-05\\
25	0.000100942040821112\\
26	0.000108282876571358\\
27	0.000115838622045322\\
28	0.000123676776113569\\
29	0.000131811281652549\\
30	0.000140256357601115\\
31	0.000149026368452964\\
32	0.000158136102587742\\
33	0.000167601383159383\\
34	0.000177437489465774\\
35	0.000187616302660993\\
36	0.000198142355171025\\
37	0.000209037467129171\\
38	0.000220338401675062\\
39	0.000232098041353089\\
40	0.000244353897257368\\
41	0.000257150171801934\\
42	0.000270539503325142\\
43	0.0002845851175605\\
44	0.000299363033194179\\
45	0.00031496580268087\\
46	0.000331508734095055\\
47	0.000349128165644788\\
48	0.000367942389534666\\
49	0.000383822961085242\\
50	0.000393478487088462\\
51	0.000404331818699767\\
52	0.000553940802233106\\
53	0.000762950510103654\\
54	0.000982157755706503\\
55	0.00121243359747923\\
56	0.00145473340171157\\
57	0.00171010916403584\\
58	0.00197966374140053\\
59	0.00226489522249555\\
60	0.00256833340229969\\
61	0.00289295173037276\\
62	0.00324228006131558\\
63	0.00362074196416667\\
64	0.00403387256464842\\
65	0.00429971016883075\\
66	0.00450850260116754\\
67	0.00474099695827202\\
68	0.00497833720054123\\
69	0.00521981265860384\\
70	0.00546375290856427\\
71	0.00570699553842181\\
72	0.00594501967040718\\
73	0.0061718155284326\\
74	0.00637715129575237\\
75	0.00656949614617233\\
76	0.0067545694498105\\
77	0.00693868354532434\\
78	0.00712335168451623\\
79	0.0073021161567019\\
80	0.00747271297532068\\
81	0.00763496642039432\\
82	0.0077899123949703\\
83	0.00794343262583044\\
84	0.00809580387407136\\
85	0.00824734970254782\\
86	0.0083990715623843\\
87	0.00855093187949473\\
88	0.00870315907223708\\
89	0.00885591566908325\\
90	0.00900993787050331\\
91	0.00916203308305026\\
92	0.00929991832107232\\
93	0.00942260939006479\\
94	0.0095405746029741\\
95	0.00965530092373354\\
96	0.00976697658027199\\
97	0.00987237869677138\\
98	0.0099605845957028\\
99	0\\
100	0\\
};
\addlegendentry{$q=3$};

\addplot [color=green,solid]
  table[row sep=crcr]{%
1	0.000228542668731224\\
2	0.000232779694508534\\
3	0.00023717816446103\\
4	0.000241746132465214\\
5	0.000246491289743564\\
6	0.000251420184496261\\
7	0.000256556459694019\\
8	0.00026190852277929\\
9	0.000267461596073126\\
10	0.000273223515831327\\
11	0.000279208745206394\\
12	0.000285434038114318\\
13	0.000291918294102623\\
14	0.000298680407599511\\
15	0.000305728994600443\\
16	0.000311815361374372\\
17	0.000314777690602715\\
18	0.000317823017361306\\
19	0.000320954144918255\\
20	0.000324174561448557\\
21	0.000327488760880062\\
22	0.000330902689693013\\
23	0.000334424364837541\\
24	0.000338064754865845\\
25	0.000341839258433608\\
26	0.000345771789773533\\
27	0.00034989830721003\\
28	0.000354247312078822\\
29	0.000358853124746333\\
30	0.000363759219945719\\
31	0.000369024275825289\\
32	0.000374729625351693\\
33	0.000380997866611204\\
34	0.000388033296742956\\
35	0.000485321635284121\\
36	0.000616029880551078\\
37	0.000751775060026542\\
38	0.000892878815618323\\
39	0.00103969049316254\\
40	0.00119260897081258\\
41	0.00135207793897538\\
42	0.00151859214896133\\
43	0.00169270456503266\\
44	0.00187503453406004\\
45	0.00206627431025713\\
46	0.00226721718091069\\
47	0.0024788172266664\\
48	0.00270217589477159\\
49	0.00293840034966602\\
50	0.00318908990816527\\
51	0.00345666270969956\\
52	0.00360442739682428\\
53	0.00371039619260692\\
54	0.00382486740183529\\
55	0.00394956492797292\\
56	0.00408672402413882\\
57	0.00423926087323422\\
58	0.0044109704137562\\
59	0.00459978889027843\\
60	0.00479146711718958\\
61	0.00498447850355443\\
62	0.00517655272898267\\
63	0.00536436229735506\\
64	0.00554307376960064\\
65	0.0057067246748794\\
66	0.00585530152151234\\
67	0.00598961802059368\\
68	0.0061265623487264\\
69	0.00626559643540984\\
70	0.00640605001134091\\
71	0.00654723068619584\\
72	0.00668870059954689\\
73	0.0068303531780872\\
74	0.00697264355557257\\
75	0.00711372115145026\\
76	0.0072499961052648\\
77	0.00738079132430047\\
78	0.00750570010018372\\
79	0.00762942661937307\\
80	0.00775353102018371\\
81	0.007878736918946\\
82	0.00800577213521258\\
83	0.00813474430415688\\
84	0.00826523623162049\\
85	0.00839748989621356\\
86	0.00853180785227634\\
87	0.00866903106140667\\
88	0.00880963575522944\\
89	0.00895231557028624\\
90	0.00908219203028829\\
91	0.00919991852903407\\
92	0.00931494353133929\\
93	0.00942873570002179\\
94	0.0095424200554176\\
95	0.0096556491500428\\
96	0.00976697658027199\\
97	0.00987237869677138\\
98	0.0099605845957028\\
99	0\\
100	0\\
};
\addlegendentry{$q=4$};

\end{axis}
\end{tikzpicture}%
 
  \caption{Discrete Time}
\end{subfigure}\\

\leavevmode\smash{\makebox[0pt]{\hspace{-7em}% HORIZONTAL POSITION           
  \rotatebox[origin=l]{90}{\hspace{20em}% VERTICAL POSITION
    Depth $\delta^+$}%
}}\hspace{0pt plus 1filll}\null

Time (s)

\vspace{1cm}
\begin{subfigure}{\linewidth}
  \centering
  \tikzsetnextfilename{altdeltalegend}
  \definecolor{mycolor1}{rgb}{0.00000,1.00000,0.14286}%
\definecolor{mycolor2}{rgb}{0.00000,1.00000,0.28571}%
\definecolor{mycolor3}{rgb}{0.00000,1.00000,0.42857}%
\definecolor{mycolor4}{rgb}{0.00000,1.00000,0.57143}%
\definecolor{mycolor5}{rgb}{0.00000,1.00000,0.71429}%
\definecolor{mycolor6}{rgb}{0.00000,1.00000,0.85714}%
\definecolor{mycolor7}{rgb}{0.00000,1.00000,1.00000}%
\definecolor{mycolor8}{rgb}{0.00000,0.87500,1.00000}%
\definecolor{mycolor9}{rgb}{0.00000,0.62500,1.00000}%
\definecolor{mycolor10}{rgb}{0.12500,0.00000,1.00000}%
\definecolor{mycolor11}{rgb}{0.25000,0.00000,1.00000}%
\definecolor{mycolor12}{rgb}{0.37500,0.00000,1.00000}%
\definecolor{mycolor13}{rgb}{0.50000,0.00000,1.00000}%
\definecolor{mycolor14}{rgb}{0.62500,0.00000,1.00000}%
\definecolor{mycolor15}{rgb}{0.75000,0.00000,1.00000}%
\definecolor{mycolor16}{rgb}{0.87500,0.00000,1.00000}%
\definecolor{mycolor17}{rgb}{1.00000,0.00000,1.00000}%
\definecolor{mycolor18}{rgb}{1.00000,0.00000,0.87500}%
\definecolor{mycolor19}{rgb}{1.00000,0.00000,0.62500}%
\definecolor{mycolor20}{rgb}{0.85714,0.00000,0.00000}%
\definecolor{mycolor21}{rgb}{0.71429,0.00000,0.00000}%
%[trim axis left, trim axis right]
\begin{tikzpicture}
\begin{axis}[%
    hide axis,
    scale only axis,
    height=0pt,
    width=0pt,
    point meta min=-19,
    point meta max=19,
    colormap={mymap}{[1pt] rgb(0pt)=(0,1,0); rgb(7pt)=(0,1,1); rgb(15pt)=(0,0,1); rgb(23pt)=(1,0,1); rgb(31pt)=(1,0,0); rgb(38pt)=(0,0,0)},
    colorbar horizontal,
    colorbar style={width=15cm,xtick={{-15},{-10},{-5},{0},{5},{10},{15}},ylabel={Inventory Level $Q$}, y label style={at={(axis description cs:0.5,-1)},rotate=-90,anchor=north}}
    %colorbar style={separate axis lines,every outer x axis line/.append style={black},every x tick label/.append style={font=\color{black}},every outer y axis line/.append style={black},every y tick label/.append style={font=\color{black}},yticklabels={{-19},{-17},{-15},{-13},{-11},{-9},{-7},{-5},{-3},{-1},{1},{3},{5},{7},{9},{11},{13},{15},{17},{19}},ylabel={Inventory Level $Q$}}
]%
    \addplot [draw=none] coordinates {(0,0)};
\end{axis}
\end{tikzpicture}
 
\end{subfigure}%
  \caption{Optimal buy depths $\delta^+$ for Markov state $Z=(\rho = 0, \Delta S = 0)$, implying neutral imbalance and no previous price change. We expect no change in midprice.}
  \label{fig:comp_dp_z8_test}
\end{figure}

\begin{figure}
\centering
\begin{subfigure}{.45\linewidth}
  \centering
  \setlength\figureheight{\linewidth} 
  \setlength\figurewidth{\linewidth}
  \tikzsetnextfilename{testdp_cts_z15}
  % This file was created by matlab2tikz.
%
%The latest updates can be retrieved from
%  http://www.mathworks.com/matlabcentral/fileexchange/22022-matlab2tikz-matlab2tikz
%where you can also make suggestions and rate matlab2tikz.
%
\definecolor{mycolor1}{rgb}{1.00000,0.00000,1.00000}%
%
\begin{tikzpicture}[trim axis left, trim axis right]

\begin{axis}[%
width=\figurewidth,
height=\figureheight,
at={(0\figurewidth,0\figureheight)},
scale only axis,
every outer x axis line/.append style={black},
every x tick label/.append style={font=\color{black}},
xmin=0,
xmax=100,
%xlabel={Time},
every outer y axis line/.append style={black},
every y tick label/.append style={font=\color{black}},
ymin=0,
ymax=0.015,
%ylabel={Depth $\delta^+$},
axis background/.style={fill=white},
axis x line*=bottom,
axis y line*=left,
yticklabel style={
        /pgf/number format/fixed,
        /pgf/number format/precision=3
},
scaled y ticks=false,
legend style={legend cell align=left,align=left,draw=black,font=\footnotesize, at={(0.98,0.02)},anchor=south east},
every axis legend/.code={\renewcommand\addlegendentry[2][]{}}  %ignore legend locally
]
\addplot [color=green,dashed]
  table[row sep=crcr]{%
0.01	0.01\\
1.01	0.01\\
2.01	0.01\\
3.01	0.01\\
4.01	0.01\\
5.01	0.01\\
6.01	0.01\\
7.01	0.01\\
8.01	0.01\\
9.01	0.01\\
10.01	0.01\\
11.01	0.01\\
12.01	0.01\\
13.01	0.01\\
14.01	0.01\\
15.01	0.01\\
16.01	0.01\\
17.01	0.01\\
18.01	0.01\\
19.01	0.01\\
20.01	0.01\\
21.01	0.01\\
22.01	0.01\\
23.01	0.01\\
24.01	0.01\\
25.01	0.01\\
26.01	0.01\\
27.01	0.01\\
28.01	0.01\\
29.01	0.01\\
30.01	0.01\\
31.01	0.01\\
32.01	0.01\\
33.01	0.01\\
34.01	0.01\\
35.01	0.01\\
36.01	0.01\\
37.01	0.01\\
38.01	0.01\\
39.01	0.01\\
40.01	0.01\\
41.01	0.01\\
42.01	0.01\\
43.01	0.01\\
44.01	0.01\\
45.01	0.01\\
46.01	0.01\\
47.01	0.01\\
48.01	0.01\\
49.01	0.01\\
50.01	0.01\\
51.01	0.01\\
52.01	0.01\\
53.01	0.01\\
54.01	0.01\\
55.01	0.01\\
56.01	0.01\\
57.01	0.01\\
58.01	0.01\\
59.01	0.01\\
60.01	0.01\\
61.01	0.01\\
62.01	0.01\\
63.01	0.01\\
64.01	0.01\\
65.01	0.01\\
66.01	0.01\\
67.01	0.01\\
68.01	0.01\\
69.01	0.01\\
70.01	0.01\\
71.01	0.01\\
72.01	0.01\\
73.01	0.01\\
74.01	0.01\\
75.01	0.01\\
76.01	0.01\\
77.01	0.01\\
78.01	0.01\\
79.01	0.01\\
80.01	0.01\\
81.01	0.01\\
82.01	0.01\\
83.01	0.01\\
84.01	0.01\\
85.01	0.01\\
86.01	0.01\\
87.01	0.01\\
88.01	0.01\\
89.01	0.01\\
90.01	0.01\\
91.01	0.01\\
92.01	0.01\\
93.01	0.01\\
94.01	0.01\\
95.01	0.01\\
96.01	0.01\\
97.01	0.01\\
98.01	0.01\\
99.01	0.00671154109295583\\
99.02	0.00666648979566538\\
99.03	0.00662127010377893\\
99.04	0.00657589218059403\\
99.05	0.00653030932509738\\
99.06	0.00648453023571555\\
99.07	0.00643856490098297\\
99.08	0.00639242392068952\\
99.09	0.00634611853474451\\
99.1	0.00629966065349513\\
99.11	0.00625306288962129\\
99.12	0.00620633859169888\\
99.13	0.00615950187952902\\
99.14	0.0061125419965859\\
99.15	0.00606516434576647\\
99.16	0.00601736505574436\\
99.17	0.00596914019869767\\
99.18	0.0059204857881827\\
99.19	0.00587139777688372\\
99.2	0.00582187205430372\\
99.21	0.00577190444433596\\
99.22	0.00572149075281801\\
99.23	0.00567062689318552\\
99.24	0.0056193087241921\\
99.25	0.00556753204802381\\
99.26	0.00551529260831485\\
99.27	0.00546258608805913\\
99.28	0.00540940810741223\\
99.29	0.0053557542213778\\
99.3	0.00530161991737217\\
99.31	0.00524700061266076\\
99.32	0.00519189165165917\\
99.33	0.00513628830309197\\
99.34	0.00508018575700118\\
99.35	0.0050235791215973\\
99.36	0.00496646341994358\\
99.37	0.00490883358644935\\
99.38	0.0048506844630757\\
99.39	0.00479201079544656\\
99.4	0.00473280722875311\\
99.41	0.00467306830343948\\
99.42	0.00461278915159383\\
99.43	0.00455196486264461\\
99.44	0.00449059048163354\\
99.45	0.00442866100885806\\
99.46	0.0043661713995154\\
99.47	0.00430311656334906\\
99.48	0.0042394913639357\\
99.49	0.00417529061784133\\
99.5	0.00411050909416695\\
99.51	0.00404514151408925\\
99.52	0.00397918255039652\\
99.53	0.00391262682701961\\
99.54	0.00384546891855804\\
99.55	0.00377770334980126\\
99.56	0.00370932459524504\\
99.57	0.00364032707860304\\
99.58	0.00357070517231364\\
99.59	0.0035004531970421\\
99.6	0.00342956542117795\\
99.61	0.00335803606032796\\
99.62	0.0032858592768046\\
99.63	0.00321302917911018\\
99.64	0.00313953982141679\\
99.65	0.00306538520304226\\
99.66	0.0029905592679222\\
99.67	0.00291505590407856\\
99.68	0.00283886894308471\\
99.69	0.00276199217558516\\
99.7	0.00268441933711856\\
99.71	0.00260614410548349\\
99.72	0.0025271601001977\\
99.73	0.0024474608819558\\
99.74	0.00236703995208607\\
99.75	0.00228589075200679\\
99.76	0.00220400666268282\\
99.77	0.00212138100408299\\
99.78	0.00203800703463929\\
99.79	0.00195387795070856\\
99.8	0.00186898688603761\\
99.81	0.00178332691123306\\
99.82	0.00169689103323675\\
99.83	0.00160967219480833\\
99.84	0.00152166327401621\\
99.85	0.00143285708373862\\
99.86	0.00134324637117649\\
99.87	0.00125282381737996\\
99.88	0.00116158203679085\\
99.89	0.00106951357680327\\
99.9	0.00097661091734502\\
99.91	0.000882866470482559\\
99.92	0.000788272580052774\\
99.93	0.000692821521324887\\
99.94	0.000596505500696339\\
99.95	0.000499316655426806\\
99.96	0.000401247053414959\\
99.97	0.000302288693022996\\
99.98	0.000202433502954543\\
99.99	0.00010167334219198\\
100	0\\
};
\addlegendentry{$q=-4$};

\addplot [color=mycolor1,dashed]
  table[row sep=crcr]{%
0.01	0.01\\
1.01	0.01\\
2.01	0.01\\
3.01	0.01\\
4.01	0.01\\
5.01	0.01\\
6.01	0.01\\
7.01	0.01\\
8.01	0.01\\
9.01	0.01\\
10.01	0.01\\
11.01	0.01\\
12.01	0.01\\
13.01	0.01\\
14.01	0.01\\
15.01	0.01\\
16.01	0.01\\
17.01	0.01\\
18.01	0.01\\
19.01	0.01\\
20.01	0.01\\
21.01	0.01\\
22.01	0.01\\
23.01	0.01\\
24.01	0.01\\
25.01	0.01\\
26.01	0.01\\
27.01	0.01\\
28.01	0.01\\
29.01	0.01\\
30.01	0.01\\
31.01	0.01\\
32.01	0.01\\
33.01	0.01\\
34.01	0.01\\
35.01	0.01\\
36.01	0.01\\
37.01	0.01\\
38.01	0.01\\
39.01	0.01\\
40.01	0.01\\
41.01	0.01\\
42.01	0.01\\
43.01	0.01\\
44.01	0.01\\
45.01	0.01\\
46.01	0.01\\
47.01	0.01\\
48.01	0.01\\
49.01	0.01\\
50.01	0.01\\
51.01	0.01\\
52.01	0.01\\
53.01	0.01\\
54.01	0.01\\
55.01	0.01\\
56.01	0.01\\
57.01	0.01\\
58.01	0.01\\
59.01	0.01\\
60.01	0.01\\
61.01	0.01\\
62.01	0.01\\
63.01	0.01\\
64.01	0.01\\
65.01	0.01\\
66.01	0.01\\
67.01	0.01\\
68.01	0.01\\
69.01	0.01\\
70.01	0.01\\
71.01	0.01\\
72.01	0.01\\
73.01	0.01\\
74.01	0.01\\
75.01	0.01\\
76.01	0.01\\
77.01	0.01\\
78.01	0.01\\
79.01	0.01\\
80.01	0.01\\
81.01	0.01\\
82.01	0.01\\
83.01	0.01\\
84.01	0.01\\
85.01	0.01\\
86.01	0.01\\
87.01	0.01\\
88.01	0.01\\
89.01	0.01\\
90.01	0.01\\
91.01	0.01\\
92.01	0.01\\
93.01	0.01\\
94.01	0.01\\
95.01	0.01\\
96.01	0.01\\
97.01	0.01\\
98.01	0.01\\
99.01	0.00895053971824736\\
99.02	0.00874693604588305\\
99.03	0.00854181547068092\\
99.04	0.0083351546369417\\
99.05	0.00812698671253183\\
99.06	0.00791728924939752\\
99.07	0.00770603839157192\\
99.08	0.00749320955439558\\
99.09	0.00727877739613037\\
99.1	0.007062715780018\\
99.11	0.0068449977483514\\
99.12	0.00662559549257636\\
99.13	0.00640448031894984\\
99.14	0.00619506749001899\\
99.15	0.00614343065536976\\
99.16	0.00609144762524035\\
99.17	0.00603912153677965\\
99.18	0.00598645582119131\\
99.19	0.00593345421943499\\
99.2	0.00588012078333322\\
99.21	0.0058264598893813\\
99.22	0.00577246463597955\\
99.23	0.00571809972977372\\
99.24	0.00566336885383304\\
99.25	0.0056082760216254\\
99.26	0.00555282559207382\\
99.27	0.00549702228530884\\
99.28	0.0054408711991545\\
99.29	0.00538437782638891\\
99.3	0.00532754807282254\\
99.31	0.00527038827624041\\
99.32	0.0052129052262574\\
99.33	0.0051551061851392\\
99.34	0.00509699890964516\\
99.35	0.00503859167365097\\
99.36	0.00497989329218159\\
99.37	0.00492091315045911\\
99.38	0.0048616612557091\\
99.39	0.00480214823897229\\
99.4	0.00474238538462658\\
99.41	0.00468238466146454\\
99.42	0.00462184456317968\\
99.43	0.00456075958847703\\
99.44	0.00449912481928159\\
99.45	0.00443693528457883\\
99.46	0.00437418595912791\\
99.47	0.00431087176214988\\
99.48	0.00424698765185965\\
99.49	0.00418252865380583\\
99.5	0.00411748975146728\\
99.51	0.00405186588565974\\
99.52	0.00398565195391432\\
99.53	0.00391884280982608\\
99.54	0.00385143326237089\\
99.55	0.00378341807518831\\
99.56	0.00371479196582851\\
99.57	0.00364554960496075\\
99.58	0.00357568561554109\\
99.59	0.00350519457193654\\
99.6	0.00343407099900291\\
99.61	0.00336230937111332\\
99.62	0.00328990411113425\\
99.63	0.00321684958934553\\
99.64	0.00314314012230076\\
99.65	0.00306876997162424\\
99.66	0.00299373334274013\\
99.67	0.00291802438352953\\
99.68	0.00284163718291057\\
99.69	0.00276456578468188\\
99.7	0.00268680417387409\\
99.71	0.00260834627216765\\
99.72	0.00252918593610012\\
99.73	0.00244931695514438\\
99.74	0.00236873304971111\\
99.75	0.00228742786903057\\
99.76	0.00220539498890514\\
99.77	0.00212262790932414\\
99.78	0.00203912005193141\\
99.79	0.0019548647573356\\
99.8	0.00186985528225254\\
99.81	0.00178408479646787\\
99.82	0.00169754637960787\\
99.83	0.00161023301770489\\
99.84	0.00152213759954336\\
99.85	0.001433252912771\\
99.86	0.00134357163975866\\
99.87	0.00125308635319149\\
99.88	0.0011617895113721\\
99.89	0.00106967345321565\\
99.9	0.000976730392914901\\
99.91	0.000882952414253254\\
99.92	0.000788331464541591\\
99.93	0.000692859348149801\\
99.94	0.000596527719603999\\
99.95	0.000499328076217975\\
99.96	0.000401251750225215\\
99.97	0.000302289900375147\\
99.98	0.000202433502954543\\
99.99	0.000101673342191978\\
100	0\\
};
\addlegendentry{$q=-3$};

\addplot [color=red,dashed]
  table[row sep=crcr]{%
0.01	0.01\\
1.01	0.01\\
2.01	0.01\\
3.01	0.01\\
4.01	0.01\\
5.01	0.01\\
6.01	0.01\\
7.01	0.01\\
8.01	0.01\\
9.01	0.01\\
10.01	0.01\\
11.01	0.01\\
12.01	0.01\\
13.01	0.01\\
14.01	0.01\\
15.01	0.01\\
16.01	0.01\\
17.01	0.01\\
18.01	0.01\\
19.01	0.01\\
20.01	0.01\\
21.01	0.01\\
22.01	0.01\\
23.01	0.01\\
24.01	0.01\\
25.01	0.01\\
26.01	0.01\\
27.01	0.01\\
28.01	0.01\\
29.01	0.01\\
30.01	0.01\\
31.01	0.01\\
32.01	0.01\\
33.01	0.01\\
34.01	0.01\\
35.01	0.01\\
36.01	0.01\\
37.01	0.01\\
38.01	0.01\\
39.01	0.01\\
40.01	0.01\\
41.01	0.01\\
42.01	0.01\\
43.01	0.01\\
44.01	0.01\\
45.01	0.01\\
46.01	0.01\\
47.01	0.01\\
48.01	0.01\\
49.01	0.01\\
50.01	0.01\\
51.01	0.01\\
52.01	0.01\\
53.01	0.01\\
54.01	0.01\\
55.01	0.01\\
56.01	0.01\\
57.01	0.01\\
58.01	0.01\\
59.01	0.01\\
60.01	0.01\\
61.01	0.01\\
62.01	0.01\\
63.01	0.01\\
64.01	0.01\\
65.01	0.01\\
66.01	0.01\\
67.01	0.01\\
68.01	0.01\\
69.01	0.01\\
70.01	0.01\\
71.01	0.01\\
72.01	0.01\\
73.01	0.01\\
74.01	0.01\\
75.01	0.01\\
76.01	0.01\\
77.01	0.01\\
78.01	0.01\\
79.01	0.01\\
80.01	0.01\\
81.01	0.01\\
82.01	0.01\\
83.01	0.01\\
84.01	0.01\\
85.01	0.01\\
86.01	0.01\\
87.01	0.01\\
88.01	0.01\\
89.01	0.01\\
90.01	0.01\\
91.01	0.01\\
92.01	0.01\\
93.01	0.01\\
94.01	0.01\\
95.01	0.01\\
96.01	0.01\\
97.01	0.01\\
98.01	0.01\\
99.01	0.01\\
99.02	0.01\\
99.03	0.01\\
99.04	0.01\\
99.05	0.01\\
99.06	0.01\\
99.07	0.01\\
99.08	0.01\\
99.09	0.01\\
99.1	0.01\\
99.11	0.01\\
99.12	0.01\\
99.13	0.01\\
99.14	0.00998658074540667\\
99.15	0.00981394762269664\\
99.16	0.00964021100322028\\
99.17	0.00946535581611112\\
99.18	0.0092893668729081\\
99.19	0.00911222854862415\\
99.2	0.00893392479044062\\
99.21	0.00875443910369502\\
99.22	0.00857376609587646\\
99.23	0.00839192844111959\\
99.24	0.00820890970110557\\
99.25	0.00802469297024153\\
99.26	0.00783926085981311\\
99.27	0.00765259548150028\\
99.28	0.00746467843022284\\
99.29	0.00727549076627984\\
99.3	0.00708501299674452\\
99.31	0.00689322505607424\\
99.32	0.00670010628589174\\
99.33	0.00650563541389112\\
99.34	0.00630979053181893\\
99.35	0.00611254922952013\\
99.36	0.00591388841707558\\
99.37	0.00571378427464273\\
99.38	0.00551221206393733\\
99.39	0.00530914628219126\\
99.4	0.00510456061610345\\
99.41	0.00489842792758349\\
99.42	0.00483354096659049\\
99.43	0.00476839726764135\\
99.44	0.00470269466849563\\
99.45	0.00463643043392819\\
99.46	0.0045696019104531\\
99.47	0.00450220652384934\\
99.48	0.00443422230458845\\
99.49	0.0043656225605152\\
99.5	0.0042964036119007\\
99.51	0.00422656183753986\\
99.52	0.00415609367880683\\
99.53	0.00408499564389557\\
99.54	0.00401326431225524\\
99.55	0.00394089633923041\\
99.56	0.0038678884609169\\
99.57	0.00379423749924454\\
99.58	0.00371994036729902\\
99.59	0.00364499407489556\\
99.6	0.00356939573441804\\
99.61	0.00349314256693811\\
99.62	0.00341623190862955\\
99.63	0.00333866121749428\\
99.64	0.00326042808041742\\
99.65	0.00318153022056972\\
99.66	0.00310196550517733\\
99.67	0.00302173195367941\\
99.68	0.00294082774629635\\
99.69	0.00285925123288629\\
99.7	0.00277700094237731\\
99.71	0.00269407559271374\\
99.72	0.00261047410832228\\
99.73	0.00252619563924148\\
99.74	0.00244123955819906\\
99.75	0.00235560547282003\\
99.76	0.0022692932384859\\
99.77	0.00218230297188465\\
99.78	0.002094635065294\\
99.79	0.00200629020164378\\
99.8	0.00191726937040586\\
99.81	0.001827573884364\\
99.82	0.0017372053973193\\
99.83	0.00164616592279117\\
99.84	0.00155445785377785\\
99.85	0.00146208398364519\\
99.86	0.00136904752821746\\
99.87	0.00127535214914916\\
99.88	0.00118100197866295\\
99.89	0.00108600164574474\\
99.9	0.000990356303894199\\
99.91	0.000894071660039042\\
99.92	0.000797154004978352\\
99.93	0.000699610246319197\\
99.94	0.000601447943428838\\
99.95	0.000502675344543365\\
99.96	0.000403301426184725\\
99.97	0.000303335935050114\\
99.98	0.000202789432550832\\
99.99	0.00010167334219198\\
100	0\\
};
\addlegendentry{$q=-2$};

\addplot [color=blue,dashed]
  table[row sep=crcr]{%
0.01	0.01\\
1.01	0.01\\
2.01	0.01\\
3.01	0.01\\
4.01	0.01\\
5.01	0.01\\
6.01	0.01\\
7.01	0.01\\
8.01	0.01\\
9.01	0.01\\
10.01	0.01\\
11.01	0.01\\
12.01	0.01\\
13.01	0.01\\
14.01	0.01\\
15.01	0.01\\
16.01	0.01\\
17.01	0.01\\
18.01	0.01\\
19.01	0.01\\
20.01	0.01\\
21.01	0.01\\
22.01	0.01\\
23.01	0.01\\
24.01	0.01\\
25.01	0.01\\
26.01	0.01\\
27.01	0.01\\
28.01	0.01\\
29.01	0.01\\
30.01	0.01\\
31.01	0.01\\
32.01	0.01\\
33.01	0.01\\
34.01	0.01\\
35.01	0.01\\
36.01	0.01\\
37.01	0.01\\
38.01	0.01\\
39.01	0.01\\
40.01	0.01\\
41.01	0.01\\
42.01	0.01\\
43.01	0.01\\
44.01	0.01\\
45.01	0.01\\
46.01	0.01\\
47.01	0.01\\
48.01	0.01\\
49.01	0.01\\
50.01	0.01\\
51.01	0.01\\
52.01	0.01\\
53.01	0.01\\
54.01	0.01\\
55.01	0.01\\
56.01	0.01\\
57.01	0.01\\
58.01	0.01\\
59.01	0.01\\
60.01	0.01\\
61.01	0.01\\
62.01	0.01\\
63.01	0.01\\
64.01	0.01\\
65.01	0.01\\
66.01	0.01\\
67.01	0.01\\
68.01	0.01\\
69.01	0.01\\
70.01	0.01\\
71.01	0.01\\
72.01	0.01\\
73.01	0.01\\
74.01	0.01\\
75.01	0.01\\
76.01	0.01\\
77.01	0.01\\
78.01	0.01\\
79.01	0.01\\
80.01	0.01\\
81.01	0.01\\
82.01	0.01\\
83.01	0.01\\
84.01	0.01\\
85.01	0.01\\
86.01	0.01\\
87.01	0.01\\
88.01	0.01\\
89.01	0.01\\
90.01	0.01\\
91.01	0.01\\
92.01	0.01\\
93.01	0.01\\
94.01	0.01\\
95.01	0.01\\
96.01	0.01\\
97.01	0.01\\
98.01	0.01\\
99.01	0.01\\
99.02	0.01\\
99.03	0.01\\
99.04	0.01\\
99.05	0.01\\
99.06	0.01\\
99.07	0.01\\
99.08	0.01\\
99.09	0.01\\
99.1	0.01\\
99.11	0.01\\
99.12	0.01\\
99.13	0.01\\
99.14	0.01\\
99.15	0.01\\
99.16	0.01\\
99.17	0.01\\
99.18	0.01\\
99.19	0.01\\
99.2	0.01\\
99.21	0.01\\
99.22	0.01\\
99.23	0.01\\
99.24	0.01\\
99.25	0.01\\
99.26	0.01\\
99.27	0.01\\
99.28	0.01\\
99.29	0.01\\
99.3	0.01\\
99.31	0.01\\
99.32	0.01\\
99.33	0.01\\
99.34	0.01\\
99.35	0.01\\
99.36	0.01\\
99.37	0.01\\
99.38	0.01\\
99.39	0.01\\
99.4	0.01\\
99.41	0.01\\
99.42	0.0098574925152882\\
99.43	0.00971396905841077\\
99.44	0.00956972004081591\\
99.45	0.0094247362673596\\
99.46	0.0092790082849791\\
99.47	0.00913252639168261\\
99.48	0.00898529998637358\\
99.49	0.00883734285850463\\
99.5	0.00868864559125499\\
99.51	0.00853919851239646\\
99.52	0.00838899168717304\\
99.53	0.00823801491096786\\
99.54	0.00808625770174855\\
99.55	0.00793370929228211\\
99.56	0.00778035862210953\\
99.57	0.00762619432926998\\
99.58	0.00747120474176386\\
99.59	0.00731537786874324\\
99.6	0.00715870139141777\\
99.61	0.00700116265366315\\
99.62	0.00684274865231885\\
99.63	0.00668344602716063\\
99.64	0.00652324105053277\\
99.65	0.00636211961662405\\
99.66	0.00620006723037029\\
99.67	0.00603706899596548\\
99.68	0.0058731096049623\\
99.69	0.00570817332394313\\
99.7	0.00554224398173856\\
99.71	0.00537530495617044\\
99.72	0.00520733916022993\\
99.73	0.00503832902765711\\
99.74	0.00486825649810313\\
99.75	0.00469710300170934\\
99.76	0.00452484944307186\\
99.77	0.00435147618455827\\
99.78	0.00417696302894062\\
99.79	0.00400128920130679\\
99.8	0.00382443333020954\\
99.81	0.00364637342800988\\
99.82	0.00346708687036842\\
99.83	0.00328655037483507\\
99.84	0.00310473997848419\\
99.85	0.00292163101453843\\
99.86	0.00273719808792035\\
99.87	0.00255141504966703\\
99.88	0.00236425497013749\\
99.89	0.0021756901109384\\
99.9	0.00198569189548735\\
99.91	0.0017942310464843\\
99.92	0.00160127761552209\\
99.93	0.00140680072665657\\
99.94	0.00121076854085607\\
99.95	0.00101314821860146\\
99.96	0.000813905880513196\\
99.97	0.000613006565871897\\
99.98	0.000410414188888463\\
99.99	0.000206091492568098\\
100	0\\
};
\addlegendentry{$q=-1$};

\addplot [color=black,solid]
  table[row sep=crcr]{%
0.01	0.000772572677978338\\
1.01	0.000772586960050298\\
2.01	0.000772602072581268\\
3.01	0.000772617778688537\\
4.01	0.000772634104039415\\
5.01	0.000772651075642132\\
6.01	0.000772668721937257\\
7.01	0.000772687072894957\\
8.01	0.000772706160112206\\
9.01	0.000772726016882976\\
10.01	0.000772746678134052\\
11.01	0.000772768179848622\\
12.01	0.000772790556944401\\
13.01	0.000772813838506633\\
14.01	0.000772838047871736\\
15.01	0.000772863234799767\\
16.01	0.000772889479531821\\
17.01	0.000772916839624101\\
18.01	0.000772945370516073\\
19.01	0.00077297513092211\\
20.01	0.000773006183004609\\
21.01	0.000773038592574578\\
22.01	0.000773072429470599\\
23.01	0.000773107768814724\\
24.01	0.000773144696216669\\
25.01	0.000773183330035973\\
26.01	0.000773223914706035\\
27.01	0.00077326719497765\\
28.01	0.000773315809222854\\
29.01	0.000773378937300093\\
30.01	0.000773485074332282\\
31.01	0.000773665534830962\\
32.01	0.000773863872478265\\
33.01	0.000774070091175414\\
34.01	0.000774284597755553\\
35.01	0.000774507831918101\\
36.01	0.000774740270335949\\
37.01	0.000774982431394036\\
38.01	0.000775234880561552\\
39.01	0.000775498236214334\\
40.01	0.000775773175567719\\
41.01	0.000776060442000893\\
42.01	0.000776360864373517\\
43.01	0.000776675402775383\\
44.01	0.000777005141665751\\
45.01	0.000777351244674431\\
46.01	0.000777714973115972\\
47.01	0.00077809758245926\\
48.01	0.000778499957230751\\
49.01	0.000778922178529547\\
50.01	0.000779365701884367\\
51.01	0.000779836269641\\
52.01	0.00078033854888966\\
53.01	0.000780879541511712\\
54.01	0.000781477216495834\\
55.01	0.000782187233415175\\
56.01	0.000783162452931356\\
57.01	0.00078462801604825\\
58.01	0.000786331132114963\\
59.01	0.000788114932563816\\
60.01	0.000789985425763777\\
61.01	0.000791949322413138\\
62.01	0.000794014170178546\\
63.01	0.000796188578580066\\
64.01	0.000798482690067912\\
65.01	0.000800909308936693\\
66.01	0.000803486464933658\\
67.01	0.000806239812929157\\
68.01	0.000809189496130203\\
69.01	0.00081239867048309\\
70.01	0.00081611278869758\\
71.01	0.000821166101299073\\
72.01	0.000828765825611963\\
73.01	0.000839836010926244\\
74.01	0.000855215841775262\\
75.01	0.000872160581526399\\
76.01	0.000891019008814649\\
77.01	0.000913615766736705\\
78.01	0.000947976808092561\\
79.01	0.000988163762707569\\
80.01	0.00103137174952786\\
81.01	0.00108005140637381\\
82.01	0.00113718005134342\\
83.01	0.00120552757710205\\
84.01	0.00130915617875347\\
85.01	0.00142837050157621\\
86.01	0.00155419534551001\\
87.01	0.00168938017131609\\
88.01	0.00184620959744923\\
89.01	0.0020757431789086\\
90.01	0.00233942407469962\\
91.01	0.00261418593362975\\
92.01	0.00290107740241149\\
93.01	0.00320129505312325\\
94.01	0.00351611636090081\\
95.01	0.00384634088144753\\
96.01	0.00419009984971961\\
97.01	0.00455316857454773\\
98.01	0.00498519971313656\\
99.01	0.00582217103975174\\
99.02	0.00583702608917074\\
99.03	0.00585212152213285\\
99.04	0.00586746285631636\\
99.05	0.00588305574613397\\
99.06	0.00589890598649193\\
99.07	0.00591501951666894\\
99.08	0.00593140242431951\\
99.09	0.00594806094960653\\
99.1	0.0059650014894658\\
99.11	0.00598223060201175\\
99.12	0.00599975501108825\\
99.13	0.00601758161096993\\
99.14	0.0060357174712204\\
99.15	0.00605416984171331\\
99.16	0.00607294615782251\\
99.17	0.00609205404578594\\
99.18	0.00611150132825341\\
99.19	0.00613129603002837\\
99.2	0.00615144638400911\\
99.21	0.00617196083733744\\
99.22	0.006192848057764\\
99.23	0.00621411694023918\\
99.24	0.0062357766137396\\
99.25	0.00625783644834031\\
99.26	0.00628030606254335\\
99.27	0.0063031953308743\\
99.28	0.00632651439175847\\
99.29	0.0063502736556895\\
99.3	0.00637448381370353\\
99.31	0.00639915586344442\\
99.32	0.00642430111417673\\
99.33	0.00644993118405764\\
99.34	0.00647605801001219\\
99.35	0.00650269385799353\\
99.36	0.00652985133364663\\
99.37	0.00655754339339478\\
99.38	0.00658574111450058\\
99.39	0.0066144340019461\\
99.4	0.00664363418934559\\
99.41	0.00667335415000014\\
99.42	0.00670360670988301\\
99.43	0.00673440507738807\\
99.44	0.00676574324691763\\
99.45	0.00679761637399577\\
99.46	0.0068300377389763\\
99.47	0.00686302100297037\\
99.48	0.00689658022086231\\
99.49	0.00693072985487318\\
99.5	0.00696548478869991\\
99.51	0.00700086034226031\\
99.52	0.00703687228707563\\
99.53	0.00707353686232452\\
99.54	0.00711087079160439\\
99.55	0.00714889130043846\\
99.56	0.00718761613456919\\
99.57	0.00722706357908156\\
99.58	0.00726725247840242\\
99.59	0.00730820225722517\\
99.6	0.00734993294241264\\
99.61	0.00739246518593405\\
99.62	0.00743582028889647\\
99.63	0.0074800202267347\\
99.64	0.00752508767562849\\
99.65	0.0075710460402206\\
99.66	0.00761791948271452\\
99.67	0.00766573295343635\\
99.68	0.00771451222295164\\
99.69	0.00776428391583597\\
99.7	0.00781507554620144\\
99.71	0.00786691555509222\\
99.72	0.00791983334987064\\
99.73	0.00797385934572439\\
99.74	0.00802902500943545\\
99.75	0.00808536290556256\\
99.76	0.00814290674520107\\
99.77	0.00820169143749746\\
99.78	0.00826175314411004\\
99.79	0.00832312933682349\\
99.8	0.00838585885854211\\
99.81	0.00844998198790592\\
99.82	0.00851554050779452\\
99.83	0.00858257777800699\\
99.84	0.00865113881243126\\
99.85	0.00872127036104468\\
99.86	0.00879302099711813\\
99.87	0.00886644121003055\\
99.88	0.00894158350413849\\
99.89	0.00901850250418713\\
99.9	0.00909725506779586\\
99.91	0.00917790040560339\\
99.92	0.00926050020971464\\
99.93	0.00934511879115603\\
99.94	0.0094318232271174\\
99.95	0.00952068351883868\\
99.96	0.00961177276108944\\
99.97	0.00970516732428955\\
99.98	0.00980094705043254\\
99.99	0.00989919546410015\\
100	0.01\\
};
\addlegendentry{$q=0$};

\addplot [color=blue,solid]
  table[row sep=crcr]{%
0.01	0.00597637174049999\\
1.01	0.00597648412037382\\
2.01	0.00597660970620105\\
3.01	0.00597674021398701\\
4.01	0.00597687585523506\\
5.01	0.00597701685234169\\
6.01	0.00597716343932752\\
7.01	0.00597731586262773\\
8.01	0.00597747438193883\\
9.01	0.00597763927108442\\
10.01	0.00597781081869609\\
11.01	0.00597798932767577\\
12.01	0.00597817510852149\\
13.01	0.00597836844728766\\
14.01	0.005978569518615\\
15.01	0.00597877849145988\\
16.01	0.00597899609220077\\
17.01	0.00597922288937483\\
18.01	0.00597945933480175\\
19.01	0.00597970590653835\\
20.01	0.00597996311028222\\
21.01	0.00598023148060377\\
22.01	0.005980511582169\\
23.01	0.00598080401162112\\
24.01	0.00598110940359987\\
25.01	0.00598142845767925\\
26.01	0.0059817620671399\\
27.01	0.00598211194342043\\
28.01	0.00598248358592138\\
29.01	0.00598289909207564\\
30.01	0.00598345956946713\\
31.01	0.00598475922475163\\
32.01	0.00598643671418286\\
33.01	0.0059881808016102\\
34.01	0.00598999489836592\\
35.01	0.00599188268651035\\
36.01	0.00599384815213209\\
37.01	0.00599589562406416\\
38.01	0.00599802981886118\\
39.01	0.00600025589246188\\
40.01	0.00600257949701351\\
41.01	0.00600500683691016\\
42.01	0.00600754473531797\\
43.01	0.00601020091323166\\
44.01	0.00601298440402638\\
45.01	0.00601590495653862\\
46.01	0.0060189733226658\\
47.01	0.00602220128626651\\
48.01	0.00602560013904741\\
49.01	0.0060291736273496\\
50.01	0.006032912968422\\
51.01	0.00603686096257251\\
52.01	0.00604106277693161\\
53.01	0.00604555228044429\\
54.01	0.00605038664918195\\
55.01	0.00605572878137712\\
56.01	0.0060622670694374\\
57.01	0.00607265491682967\\
58.01	0.00608727136683662\\
59.01	0.00610259539511717\\
60.01	0.00611868041468113\\
61.01	0.00613558603139828\\
62.01	0.00615337902088007\\
63.01	0.00617213453831509\\
64.01	0.00619193766548622\\
65.01	0.00621288523055736\\
66.01	0.00623508623077034\\
67.01	0.0062586559502836\\
68.01	0.00628373881167773\\
69.01	0.00631057064257134\\
70.01	0.00633970694645457\\
71.01	0.00637393934762701\\
72.01	0.00642982899472481\\
73.01	0.0064960493968363\\
74.01	0.00656504558858773\\
75.01	0.00663907700337228\\
76.01	0.00671701575837442\\
77.01	0.00679948725287152\\
78.01	0.00688078554199095\\
79.01	0.00695981358597765\\
80.01	0.00704217145429333\\
81.01	0.00713838900179958\\
82.01	0.00729531942674517\\
83.01	0.0074780597795894\\
84.01	0.00765388445422861\\
85.01	0.00782883409880153\\
86.01	0.00801153835761498\\
87.01	0.00820388145655756\\
88.01	0.00840914274726625\\
89.01	0.00859674809745502\\
90.01	0.00877379354560438\\
91.01	0.00895875007059519\\
92.01	0.00915207189682873\\
93.01	0.00935422007184203\\
94.01	0.00956552578984417\\
95.01	0.00978446542419082\\
96.01	0.00998035950606173\\
97.01	0.01\\
98.01	0.01\\
99.01	0.01\\
99.02	0.01\\
99.03	0.01\\
99.04	0.01\\
99.05	0.01\\
99.06	0.01\\
99.07	0.01\\
99.08	0.01\\
99.09	0.01\\
99.1	0.01\\
99.11	0.01\\
99.12	0.01\\
99.13	0.01\\
99.14	0.01\\
99.15	0.01\\
99.16	0.01\\
99.17	0.01\\
99.18	0.01\\
99.19	0.01\\
99.2	0.01\\
99.21	0.01\\
99.22	0.01\\
99.23	0.01\\
99.24	0.01\\
99.25	0.01\\
99.26	0.01\\
99.27	0.01\\
99.28	0.01\\
99.29	0.01\\
99.3	0.01\\
99.31	0.01\\
99.32	0.01\\
99.33	0.01\\
99.34	0.01\\
99.35	0.01\\
99.36	0.01\\
99.37	0.01\\
99.38	0.01\\
99.39	0.01\\
99.4	0.01\\
99.41	0.01\\
99.42	0.01\\
99.43	0.01\\
99.44	0.01\\
99.45	0.01\\
99.46	0.01\\
99.47	0.01\\
99.48	0.01\\
99.49	0.01\\
99.5	0.01\\
99.51	0.01\\
99.52	0.01\\
99.53	0.01\\
99.54	0.01\\
99.55	0.01\\
99.56	0.01\\
99.57	0.01\\
99.58	0.01\\
99.59	0.01\\
99.6	0.01\\
99.61	0.01\\
99.62	0.01\\
99.63	0.01\\
99.64	0.01\\
99.65	0.01\\
99.66	0.01\\
99.67	0.01\\
99.68	0.01\\
99.69	0.01\\
99.7	0.01\\
99.71	0.01\\
99.72	0.01\\
99.73	0.01\\
99.74	0.01\\
99.75	0.01\\
99.76	0.01\\
99.77	0.01\\
99.78	0.01\\
99.79	0.01\\
99.8	0.01\\
99.81	0.01\\
99.82	0.01\\
99.83	0.01\\
99.84	0.01\\
99.85	0.01\\
99.86	0.01\\
99.87	0.01\\
99.88	0.01\\
99.89	0.01\\
99.9	0.01\\
99.91	0.01\\
99.92	0.01\\
99.93	0.01\\
99.94	0.01\\
99.95	0.01\\
99.96	0.01\\
99.97	0.01\\
99.98	0.01\\
99.99	0.01\\
100	0.01\\
};
\addlegendentry{$q=1$};

\addplot [color=red,solid]
  table[row sep=crcr]{%
0.01	0.00819891155222631\\
1.01	0.00820015753419366\\
2.01	0.00820176826489647\\
3.01	0.00820344226333224\\
4.01	0.00820518224544149\\
5.01	0.00820699106507912\\
6.01	0.00820887172294877\\
7.01	0.00821082737625455\\
8.01	0.00821286134911644\\
9.01	0.00821497714373716\\
10.01	0.00821717845195611\\
11.01	0.00821946916471961\\
12.01	0.00822185336414796\\
13.01	0.00822433520686423\\
14.01	0.00822691825834294\\
15.01	0.00822960421233094\\
16.01	0.00823239983585605\\
17.01	0.00823531337608766\\
18.01	0.00823835054651003\\
19.01	0.0082415173689447\\
20.01	0.00824482019683733\\
21.01	0.00824826572672234\\
22.01	0.00825186100637006\\
23.01	0.00825561343884282\\
24.01	0.00825953078521241\\
25.01	0.00826362119018675\\
26.01	0.00826789338480721\\
27.01	0.00827235806380775\\
28.01	0.00827703713083022\\
29.01	0.00828201829708017\\
30.01	0.00828747770838528\\
31.01	0.00829288931257152\\
32.01	0.00829826043790506\\
33.01	0.00830384442809367\\
34.01	0.00830965196769699\\
35.01	0.00831569454460901\\
36.01	0.00832198453961317\\
37.01	0.00832853532865284\\
38.01	0.00833536139960729\\
39.01	0.00834247848457186\\
40.01	0.00834990370414062\\
41.01	0.00835765570326112\\
42.01	0.00836575474161264\\
43.01	0.00837422333853095\\
44.01	0.00838309015809641\\
45.01	0.00839238543768394\\
46.01	0.00840214010128022\\
47.01	0.00841238743754561\\
48.01	0.00842315580364073\\
49.01	0.00843440870608351\\
50.01	0.00844573230156728\\
51.01	0.00845709682997258\\
52.01	0.00846915109054207\\
53.01	0.00848197429192993\\
54.01	0.0084956852726762\\
55.01	0.00851067049311928\\
56.01	0.00852989465237098\\
57.01	0.00856236215112235\\
58.01	0.00859490772454737\\
59.01	0.00862892747624734\\
60.01	0.00866451082496186\\
61.01	0.00870175497656521\\
62.01	0.00874076592036578\\
63.01	0.00878165962799344\\
64.01	0.00882456346411751\\
65.01	0.00886961735713146\\
66.01	0.00891697273252878\\
67.01	0.00896679583522316\\
68.01	0.00901931261411121\\
69.01	0.0090748541535986\\
70.01	0.00913410484180677\\
71.01	0.00919869070804489\\
72.01	0.00925922041567255\\
73.01	0.00931728092044048\\
74.01	0.00939362413729078\\
75.01	0.00951723551782385\\
76.01	0.00964573885723674\\
77.01	0.00977820665852457\\
78.01	0.0099098698887369\\
79.01	0.01\\
80.01	0.01\\
81.01	0.01\\
82.01	0.01\\
83.01	0.01\\
84.01	0.01\\
85.01	0.01\\
86.01	0.01\\
87.01	0.01\\
88.01	0.01\\
89.01	0.01\\
90.01	0.01\\
91.01	0.01\\
92.01	0.01\\
93.01	0.01\\
94.01	0.01\\
95.01	0.01\\
96.01	0.01\\
97.01	0.01\\
98.01	0.01\\
99.01	0.01\\
99.02	0.01\\
99.03	0.01\\
99.04	0.01\\
99.05	0.01\\
99.06	0.01\\
99.07	0.01\\
99.08	0.01\\
99.09	0.01\\
99.1	0.01\\
99.11	0.01\\
99.12	0.01\\
99.13	0.01\\
99.14	0.01\\
99.15	0.01\\
99.16	0.01\\
99.17	0.01\\
99.18	0.01\\
99.19	0.01\\
99.2	0.01\\
99.21	0.01\\
99.22	0.01\\
99.23	0.01\\
99.24	0.01\\
99.25	0.01\\
99.26	0.01\\
99.27	0.01\\
99.28	0.01\\
99.29	0.01\\
99.3	0.01\\
99.31	0.01\\
99.32	0.01\\
99.33	0.01\\
99.34	0.01\\
99.35	0.01\\
99.36	0.01\\
99.37	0.01\\
99.38	0.01\\
99.39	0.01\\
99.4	0.01\\
99.41	0.01\\
99.42	0.01\\
99.43	0.01\\
99.44	0.01\\
99.45	0.01\\
99.46	0.01\\
99.47	0.01\\
99.48	0.01\\
99.49	0.01\\
99.5	0.01\\
99.51	0.01\\
99.52	0.01\\
99.53	0.01\\
99.54	0.01\\
99.55	0.01\\
99.56	0.01\\
99.57	0.01\\
99.58	0.01\\
99.59	0.01\\
99.6	0.01\\
99.61	0.01\\
99.62	0.01\\
99.63	0.01\\
99.64	0.01\\
99.65	0.01\\
99.66	0.01\\
99.67	0.01\\
99.68	0.01\\
99.69	0.01\\
99.7	0.01\\
99.71	0.01\\
99.72	0.01\\
99.73	0.01\\
99.74	0.01\\
99.75	0.01\\
99.76	0.01\\
99.77	0.01\\
99.78	0.01\\
99.79	0.01\\
99.8	0.01\\
99.81	0.01\\
99.82	0.01\\
99.83	0.01\\
99.84	0.01\\
99.85	0.01\\
99.86	0.01\\
99.87	0.01\\
99.88	0.01\\
99.89	0.01\\
99.9	0.01\\
99.91	0.01\\
99.92	0.01\\
99.93	0.01\\
99.94	0.01\\
99.95	0.01\\
99.96	0.01\\
99.97	0.01\\
99.98	0.01\\
99.99	0.01\\
100	0.01\\
};
\addlegendentry{$q=2$};

\addplot [color=mycolor1,solid]
  table[row sep=crcr]{%
0.01	0.00956306055999642\\
1.01	0.00956475096158899\\
2.01	0.0095662849236269\\
3.01	0.00956787879633217\\
4.01	0.0095695352344733\\
5.01	0.00957125704962719\\
6.01	0.009573047224405\\
7.01	0.00957490892853349\\
8.01	0.00957684553707455\\
9.01	0.00957886065102747\\
10.01	0.00958095812003643\\
11.01	0.00958314206306399\\
12.01	0.00958541685429289\\
13.01	0.009587786810908\\
14.01	0.00959025333986315\\
15.01	0.00959279539499291\\
16.01	0.00959540372759264\\
17.01	0.00959813146562209\\
18.01	0.00960098733127659\\
19.01	0.00960398014324473\\
20.01	0.00960711973595633\\
21.01	0.00961041713057055\\
22.01	0.00961388474206355\\
23.01	0.00961753663241435\\
24.01	0.0096213888279516\\
25.01	0.00962545975608485\\
26.01	0.00962977109592019\\
27.01	0.00963435128774966\\
28.01	0.00963926408165035\\
29.01	0.00964495644760401\\
30.01	0.00965498589107521\\
31.01	0.00966678071723932\\
32.01	0.00967896706654316\\
33.01	0.0096916266455589\\
34.01	0.00970478055239498\\
35.01	0.00971845105333241\\
36.01	0.00973266166620452\\
37.01	0.00974743725152351\\
38.01	0.00976280411205298\\
39.01	0.00977879010045143\\
40.01	0.00979542472996721\\
41.01	0.00981273927414662\\
42.01	0.00983076693023207\\
43.01	0.00984954407629783\\
44.01	0.00986911222972129\\
45.01	0.00988951305005627\\
46.01	0.00991078943308928\\
47.01	0.00993298481973196\\
48.01	0.00995612096218882\\
49.01	0.00997991879443703\\
50.01	0.00999938358127485\\
51.01	0.01\\
52.01	0.01\\
53.01	0.01\\
54.01	0.01\\
55.01	0.01\\
56.01	0.01\\
57.01	0.01\\
58.01	0.01\\
59.01	0.01\\
60.01	0.01\\
61.01	0.01\\
62.01	0.01\\
63.01	0.01\\
64.01	0.01\\
65.01	0.01\\
66.01	0.01\\
67.01	0.01\\
68.01	0.01\\
69.01	0.01\\
70.01	0.01\\
71.01	0.01\\
72.01	0.01\\
73.01	0.01\\
74.01	0.01\\
75.01	0.01\\
76.01	0.01\\
77.01	0.01\\
78.01	0.01\\
79.01	0.01\\
80.01	0.01\\
81.01	0.01\\
82.01	0.01\\
83.01	0.01\\
84.01	0.01\\
85.01	0.01\\
86.01	0.01\\
87.01	0.01\\
88.01	0.01\\
89.01	0.01\\
90.01	0.01\\
91.01	0.01\\
92.01	0.01\\
93.01	0.01\\
94.01	0.01\\
95.01	0.01\\
96.01	0.01\\
97.01	0.01\\
98.01	0.01\\
99.01	0.01\\
99.02	0.01\\
99.03	0.01\\
99.04	0.01\\
99.05	0.01\\
99.06	0.01\\
99.07	0.01\\
99.08	0.01\\
99.09	0.01\\
99.1	0.01\\
99.11	0.01\\
99.12	0.01\\
99.13	0.01\\
99.14	0.01\\
99.15	0.01\\
99.16	0.01\\
99.17	0.01\\
99.18	0.01\\
99.19	0.01\\
99.2	0.01\\
99.21	0.01\\
99.22	0.01\\
99.23	0.01\\
99.24	0.01\\
99.25	0.01\\
99.26	0.01\\
99.27	0.01\\
99.28	0.01\\
99.29	0.01\\
99.3	0.01\\
99.31	0.01\\
99.32	0.01\\
99.33	0.01\\
99.34	0.01\\
99.35	0.01\\
99.36	0.01\\
99.37	0.01\\
99.38	0.01\\
99.39	0.01\\
99.4	0.01\\
99.41	0.01\\
99.42	0.01\\
99.43	0.01\\
99.44	0.01\\
99.45	0.01\\
99.46	0.01\\
99.47	0.01\\
99.48	0.01\\
99.49	0.01\\
99.5	0.01\\
99.51	0.01\\
99.52	0.01\\
99.53	0.01\\
99.54	0.01\\
99.55	0.01\\
99.56	0.01\\
99.57	0.01\\
99.58	0.01\\
99.59	0.01\\
99.6	0.01\\
99.61	0.01\\
99.62	0.01\\
99.63	0.01\\
99.64	0.01\\
99.65	0.01\\
99.66	0.01\\
99.67	0.01\\
99.68	0.01\\
99.69	0.01\\
99.7	0.01\\
99.71	0.01\\
99.72	0.01\\
99.73	0.01\\
99.74	0.01\\
99.75	0.01\\
99.76	0.01\\
99.77	0.01\\
99.78	0.01\\
99.79	0.01\\
99.8	0.01\\
99.81	0.01\\
99.82	0.01\\
99.83	0.01\\
99.84	0.01\\
99.85	0.01\\
99.86	0.01\\
99.87	0.01\\
99.88	0.01\\
99.89	0.01\\
99.9	0.01\\
99.91	0.01\\
99.92	0.01\\
99.93	0.01\\
99.94	0.01\\
99.95	0.01\\
99.96	0.01\\
99.97	0.01\\
99.98	0.01\\
99.99	0.01\\
100	0.01\\
};
\addlegendentry{$q=3$};

\addplot [color=green,solid]
  table[row sep=crcr]{%
0.01	0.01\\
1.01	0.01\\
2.01	0.01\\
3.01	0.01\\
4.01	0.01\\
5.01	0.01\\
6.01	0.01\\
7.01	0.01\\
8.01	0.01\\
9.01	0.01\\
10.01	0.01\\
11.01	0.01\\
12.01	0.01\\
13.01	0.01\\
14.01	0.01\\
15.01	0.01\\
16.01	0.01\\
17.01	0.01\\
18.01	0.01\\
19.01	0.01\\
20.01	0.01\\
21.01	0.01\\
22.01	0.01\\
23.01	0.01\\
24.01	0.01\\
25.01	0.01\\
26.01	0.01\\
27.01	0.01\\
28.01	0.01\\
29.01	0.01\\
30.01	0.01\\
31.01	0.01\\
32.01	0.01\\
33.01	0.01\\
34.01	0.01\\
35.01	0.01\\
36.01	0.01\\
37.01	0.01\\
38.01	0.01\\
39.01	0.01\\
40.01	0.01\\
41.01	0.01\\
42.01	0.01\\
43.01	0.01\\
44.01	0.01\\
45.01	0.01\\
46.01	0.01\\
47.01	0.01\\
48.01	0.01\\
49.01	0.01\\
50.01	0.01\\
51.01	0.01\\
52.01	0.01\\
53.01	0.01\\
54.01	0.01\\
55.01	0.01\\
56.01	0.01\\
57.01	0.01\\
58.01	0.01\\
59.01	0.01\\
60.01	0.01\\
61.01	0.01\\
62.01	0.01\\
63.01	0.01\\
64.01	0.01\\
65.01	0.01\\
66.01	0.01\\
67.01	0.01\\
68.01	0.01\\
69.01	0.01\\
70.01	0.01\\
71.01	0.01\\
72.01	0.01\\
73.01	0.01\\
74.01	0.01\\
75.01	0.01\\
76.01	0.01\\
77.01	0.01\\
78.01	0.01\\
79.01	0.01\\
80.01	0.01\\
81.01	0.01\\
82.01	0.01\\
83.01	0.01\\
84.01	0.01\\
85.01	0.01\\
86.01	0.01\\
87.01	0.01\\
88.01	0.01\\
89.01	0.01\\
90.01	0.01\\
91.01	0.01\\
92.01	0.01\\
93.01	0.01\\
94.01	0.01\\
95.01	0.01\\
96.01	0.01\\
97.01	0.01\\
98.01	0.01\\
99.01	0.01\\
99.02	0.01\\
99.03	0.01\\
99.04	0.01\\
99.05	0.01\\
99.06	0.01\\
99.07	0.01\\
99.08	0.01\\
99.09	0.01\\
99.1	0.01\\
99.11	0.01\\
99.12	0.01\\
99.13	0.01\\
99.14	0.01\\
99.15	0.01\\
99.16	0.01\\
99.17	0.01\\
99.18	0.01\\
99.19	0.01\\
99.2	0.01\\
99.21	0.01\\
99.22	0.01\\
99.23	0.01\\
99.24	0.01\\
99.25	0.01\\
99.26	0.01\\
99.27	0.01\\
99.28	0.01\\
99.29	0.01\\
99.3	0.01\\
99.31	0.01\\
99.32	0.01\\
99.33	0.01\\
99.34	0.01\\
99.35	0.01\\
99.36	0.01\\
99.37	0.01\\
99.38	0.01\\
99.39	0.01\\
99.4	0.01\\
99.41	0.01\\
99.42	0.01\\
99.43	0.01\\
99.44	0.01\\
99.45	0.01\\
99.46	0.01\\
99.47	0.01\\
99.48	0.01\\
99.49	0.01\\
99.5	0.01\\
99.51	0.01\\
99.52	0.01\\
99.53	0.01\\
99.54	0.01\\
99.55	0.01\\
99.56	0.01\\
99.57	0.01\\
99.58	0.01\\
99.59	0.01\\
99.6	0.01\\
99.61	0.01\\
99.62	0.01\\
99.63	0.01\\
99.64	0.01\\
99.65	0.01\\
99.66	0.01\\
99.67	0.01\\
99.68	0.01\\
99.69	0.01\\
99.7	0.01\\
99.71	0.01\\
99.72	0.01\\
99.73	0.01\\
99.74	0.01\\
99.75	0.01\\
99.76	0.01\\
99.77	0.01\\
99.78	0.01\\
99.79	0.01\\
99.8	0.01\\
99.81	0.01\\
99.82	0.01\\
99.83	0.01\\
99.84	0.01\\
99.85	0.01\\
99.86	0.01\\
99.87	0.01\\
99.88	0.01\\
99.89	0.01\\
99.9	0.01\\
99.91	0.01\\
99.92	0.01\\
99.93	0.01\\
99.94	0.01\\
99.95	0.01\\
99.96	0.01\\
99.97	0.01\\
99.98	0.01\\
99.99	0.01\\
100	0.01\\
};
\addlegendentry{$q=4$};

\end{axis}
\end{tikzpicture}%

  \caption{Continuous Time}
\end{subfigure}%
\hfill%
\begin{subfigure}{.45\linewidth}
  \centering
  \setlength\figureheight{\linewidth} 
  \setlength\figurewidth{\linewidth}
  \tikzsetnextfilename{testdp_dscr_z15}
  % This file was created by matlab2tikz.
%
%The latest updates can be retrieved from
%  http://www.mathworks.com/matlabcentral/fileexchange/22022-matlab2tikz-matlab2tikz
%where you can also make suggestions and rate matlab2tikz.
%
\definecolor{mycolor1}{rgb}{0.00000,1.00000,0.14286}%
\definecolor{mycolor2}{rgb}{0.00000,1.00000,0.28571}%
\definecolor{mycolor3}{rgb}{0.00000,1.00000,0.42857}%
\definecolor{mycolor4}{rgb}{0.00000,1.00000,0.57143}%
\definecolor{mycolor5}{rgb}{0.00000,1.00000,0.71429}%
\definecolor{mycolor6}{rgb}{0.00000,1.00000,0.85714}%
\definecolor{mycolor7}{rgb}{0.00000,1.00000,1.00000}%
\definecolor{mycolor8}{rgb}{0.00000,0.87500,1.00000}%
\definecolor{mycolor9}{rgb}{0.00000,0.62500,1.00000}%
\definecolor{mycolor10}{rgb}{0.12500,0.00000,1.00000}%
\definecolor{mycolor11}{rgb}{0.25000,0.00000,1.00000}%
\definecolor{mycolor12}{rgb}{0.37500,0.00000,1.00000}%
\definecolor{mycolor13}{rgb}{0.50000,0.00000,1.00000}%
\definecolor{mycolor14}{rgb}{0.62500,0.00000,1.00000}%
\definecolor{mycolor15}{rgb}{0.75000,0.00000,1.00000}%
\definecolor{mycolor16}{rgb}{0.87500,0.00000,1.00000}%
\definecolor{mycolor17}{rgb}{1.00000,0.00000,1.00000}%
\definecolor{mycolor18}{rgb}{1.00000,0.00000,0.87500}%
\definecolor{mycolor19}{rgb}{1.00000,0.00000,0.62500}%
\definecolor{mycolor20}{rgb}{0.85714,0.00000,0.00000}%
\definecolor{mycolor21}{rgb}{0.71429,0.00000,0.00000}%
%
\begin{tikzpicture}

\begin{axis}[%
width=4.1in,
height=3.803in,
at={(0.809in,0.513in)},
scale only axis,
point meta min=0,
point meta max=1,
every outer x axis line/.append style={black},
every x tick label/.append style={font=\color{black}},
xmin=0,
xmax=600,
every outer y axis line/.append style={black},
every y tick label/.append style={font=\color{black}},
ymin=0,
ymax=0.007,
axis background/.style={fill=white},
axis x line*=bottom,
axis y line*=left,
colormap={mymap}{[1pt] rgb(0pt)=(0,1,0); rgb(7pt)=(0,1,1); rgb(15pt)=(0,0,1); rgb(23pt)=(1,0,1); rgb(31pt)=(1,0,0); rgb(38pt)=(0,0,0)},
colorbar,
colorbar style={separate axis lines,every outer x axis line/.append style={black},every x tick label/.append style={font=\color{black}},every outer y axis line/.append style={black},every y tick label/.append style={font=\color{black}},yticklabels={{-19},{-17},{-15},{-13},{-11},{-9},{-7},{-5},{-3},{-1},{1},{3},{5},{7},{9},{11},{13},{15},{17},{19}}}
]
\addplot [color=green,solid,forget plot]
  table[row sep=crcr]{%
1	0\\
2	0\\
3	0\\
4	0\\
5	0\\
6	0\\
7	0\\
8	0\\
9	0\\
10	0\\
11	0\\
12	0\\
13	0\\
14	0\\
15	0\\
16	0\\
17	0\\
18	0\\
19	0\\
20	0\\
21	0\\
22	0\\
23	0\\
24	0\\
25	0\\
26	0\\
27	0\\
28	0\\
29	0\\
30	0\\
31	0\\
32	0\\
33	0\\
34	0\\
35	0\\
36	0\\
37	0\\
38	0\\
39	0\\
40	0\\
41	0\\
42	0\\
43	0\\
44	0\\
45	0\\
46	0\\
47	0\\
48	0\\
49	0\\
50	0\\
51	0\\
52	0\\
53	0\\
54	0\\
55	0\\
56	0\\
57	0\\
58	0\\
59	0\\
60	0\\
61	0\\
62	0\\
63	0\\
64	0\\
65	0\\
66	0\\
67	0\\
68	0\\
69	0\\
70	0\\
71	0\\
72	0\\
73	0\\
74	0\\
75	0\\
76	0\\
77	0\\
78	0\\
79	0\\
80	0\\
81	0\\
82	0\\
83	0\\
84	0\\
85	0\\
86	0\\
87	0\\
88	0\\
89	0\\
90	0\\
91	0\\
92	0\\
93	0\\
94	0\\
95	0\\
96	0\\
97	0\\
98	0\\
99	0\\
100	0\\
101	0\\
102	0\\
103	0\\
104	0\\
105	0\\
106	0\\
107	0\\
108	0\\
109	0\\
110	0\\
111	0\\
112	0\\
113	0\\
114	0\\
115	0\\
116	0\\
117	0\\
118	0\\
119	0\\
120	0\\
121	0\\
122	0\\
123	0\\
124	0\\
125	0\\
126	0\\
127	0\\
128	0\\
129	0\\
130	0\\
131	0\\
132	0\\
133	0\\
134	0\\
135	0\\
136	0\\
137	0\\
138	0\\
139	0\\
140	0\\
141	0\\
142	0\\
143	0\\
144	0\\
145	0\\
146	0\\
147	0\\
148	0\\
149	0\\
150	0\\
151	0\\
152	0\\
153	0\\
154	0\\
155	0\\
156	0\\
157	0\\
158	0\\
159	0\\
160	0\\
161	0\\
162	0\\
163	0\\
164	0\\
165	0\\
166	0\\
167	0\\
168	0\\
169	0\\
170	0\\
171	0\\
172	0\\
173	0\\
174	0\\
175	0\\
176	0\\
177	0\\
178	0\\
179	0\\
180	0\\
181	0\\
182	0\\
183	0\\
184	0\\
185	0\\
186	0\\
187	0\\
188	0\\
189	0\\
190	0\\
191	0\\
192	0\\
193	0\\
194	0\\
195	0\\
196	0\\
197	0\\
198	0\\
199	0\\
200	0\\
201	0\\
202	0\\
203	0\\
204	0\\
205	0\\
206	0\\
207	0\\
208	0\\
209	0\\
210	0\\
211	0\\
212	0\\
213	0\\
214	0\\
215	0\\
216	0\\
217	0\\
218	0\\
219	0\\
220	0\\
221	0\\
222	0\\
223	0\\
224	0\\
225	0\\
226	0\\
227	0\\
228	0\\
229	0\\
230	0\\
231	0\\
232	0\\
233	0\\
234	0\\
235	0\\
236	0\\
237	0\\
238	0\\
239	0\\
240	0\\
241	0\\
242	0\\
243	0\\
244	0\\
245	0\\
246	0\\
247	0\\
248	0\\
249	0\\
250	0\\
251	0\\
252	0\\
253	0\\
254	0\\
255	0\\
256	0\\
257	0\\
258	0\\
259	0\\
260	0\\
261	0\\
262	0\\
263	0\\
264	0\\
265	0\\
266	0\\
267	0\\
268	0\\
269	0\\
270	0\\
271	0\\
272	0\\
273	0\\
274	0\\
275	0\\
276	0\\
277	0\\
278	0\\
279	0\\
280	0\\
281	0\\
282	0\\
283	0\\
284	0\\
285	0\\
286	0\\
287	0\\
288	0\\
289	0\\
290	0\\
291	0\\
292	0\\
293	0\\
294	0\\
295	0\\
296	0\\
297	0\\
298	0\\
299	0\\
300	0\\
301	0\\
302	0\\
303	0\\
304	0\\
305	0\\
306	0\\
307	0\\
308	0\\
309	0\\
310	0\\
311	0\\
312	0\\
313	0\\
314	0\\
315	0\\
316	0\\
317	0\\
318	0\\
319	0\\
320	0\\
321	0\\
322	0\\
323	0\\
324	0\\
325	0\\
326	0\\
327	0\\
328	0\\
329	0\\
330	0\\
331	0\\
332	0\\
333	0\\
334	0\\
335	0\\
336	0\\
337	0\\
338	0\\
339	0\\
340	0\\
341	0\\
342	0\\
343	0\\
344	0\\
345	0\\
346	0\\
347	0\\
348	0\\
349	0\\
350	0\\
351	0\\
352	0\\
353	0\\
354	0\\
355	0\\
356	0\\
357	0\\
358	0\\
359	0\\
360	0\\
361	0\\
362	0\\
363	0\\
364	0\\
365	0\\
366	0\\
367	0\\
368	0\\
369	0\\
370	0\\
371	0\\
372	0\\
373	0\\
374	0\\
375	0\\
376	0\\
377	0\\
378	0\\
379	0\\
380	0\\
381	0\\
382	0\\
383	0\\
384	0\\
385	0\\
386	0\\
387	0\\
388	0\\
389	0\\
390	0\\
391	0\\
392	0\\
393	0\\
394	0\\
395	0\\
396	0\\
397	0\\
398	0\\
399	0\\
400	0\\
401	0\\
402	0\\
403	0\\
404	0\\
405	0\\
406	0\\
407	0\\
408	0\\
409	0\\
410	0\\
411	0\\
412	0\\
413	0\\
414	0\\
415	0\\
416	0\\
417	0\\
418	0\\
419	0\\
420	0\\
421	0\\
422	0\\
423	0\\
424	0\\
425	0\\
426	0\\
427	0\\
428	0\\
429	0\\
430	0\\
431	0\\
432	0\\
433	0\\
434	0\\
435	0\\
436	0\\
437	0\\
438	0\\
439	0\\
440	0\\
441	0\\
442	0\\
443	0\\
444	0\\
445	0\\
446	0\\
447	0\\
448	0\\
449	0\\
450	0\\
451	0\\
452	0\\
453	0\\
454	0\\
455	0\\
456	0\\
457	0\\
458	0\\
459	0\\
460	0\\
461	0\\
462	0\\
463	0\\
464	0\\
465	0\\
466	0\\
467	0\\
468	0\\
469	0\\
470	0\\
471	0\\
472	0\\
473	0\\
474	0\\
475	0\\
476	0\\
477	0\\
478	0\\
479	0\\
480	0\\
481	0\\
482	0\\
483	0\\
484	0\\
485	0\\
486	0\\
487	0\\
488	0\\
489	0\\
490	0\\
491	0\\
492	0\\
493	0\\
494	0\\
495	0\\
496	0\\
497	0\\
498	0\\
499	0\\
500	0\\
501	0\\
502	0\\
503	0\\
504	0\\
505	0\\
506	0\\
507	0\\
508	0\\
509	0\\
510	0\\
511	0\\
512	0\\
513	0\\
514	0\\
515	0\\
516	0\\
517	0\\
518	0\\
519	0\\
520	0\\
521	0\\
522	0\\
523	0\\
524	0\\
525	0\\
526	0\\
527	0\\
528	0\\
529	0\\
530	0\\
531	0\\
532	0\\
533	0\\
534	0\\
535	0\\
536	0\\
537	0\\
538	0\\
539	0\\
540	0\\
541	0\\
542	0\\
543	0\\
544	0\\
545	0\\
546	0\\
547	0\\
548	0\\
549	0\\
550	0\\
551	0\\
552	0\\
553	0\\
554	0\\
555	0\\
556	0\\
557	0\\
558	0\\
559	0\\
560	0\\
561	0\\
562	0\\
563	0\\
564	0\\
565	0\\
566	0\\
567	0\\
568	0\\
569	0\\
570	0\\
571	0\\
572	0\\
573	0\\
574	0\\
575	0\\
576	0\\
577	0\\
578	0\\
579	0\\
580	0\\
581	0\\
582	0\\
583	0\\
584	0\\
585	0\\
586	0\\
587	0\\
588	0\\
589	0\\
590	0\\
591	0\\
592	0\\
593	0\\
594	0\\
595	0\\
596	0\\
597	0\\
598	0\\
599	0\\
600	0\\
};
\addplot [color=mycolor1,solid,forget plot]
  table[row sep=crcr]{%
1	0\\
2	0\\
3	0\\
4	0\\
5	0\\
6	0\\
7	0\\
8	0\\
9	0\\
10	0\\
11	0\\
12	0\\
13	0\\
14	0\\
15	0\\
16	0\\
17	0\\
18	0\\
19	0\\
20	0\\
21	0\\
22	0\\
23	0\\
24	0\\
25	0\\
26	0\\
27	0\\
28	0\\
29	0\\
30	0\\
31	0\\
32	0\\
33	0\\
34	0\\
35	0\\
36	0\\
37	0\\
38	0\\
39	0\\
40	0\\
41	0\\
42	0\\
43	0\\
44	0\\
45	0\\
46	0\\
47	0\\
48	0\\
49	0\\
50	0\\
51	0\\
52	0\\
53	0\\
54	0\\
55	0\\
56	0\\
57	0\\
58	0\\
59	0\\
60	0\\
61	0\\
62	0\\
63	0\\
64	0\\
65	0\\
66	0\\
67	0\\
68	0\\
69	0\\
70	0\\
71	0\\
72	0\\
73	0\\
74	0\\
75	0\\
76	0\\
77	0\\
78	0\\
79	0\\
80	0\\
81	0\\
82	0\\
83	0\\
84	0\\
85	0\\
86	0\\
87	0\\
88	0\\
89	0\\
90	0\\
91	0\\
92	0\\
93	0\\
94	0\\
95	0\\
96	0\\
97	0\\
98	0\\
99	0\\
100	0\\
101	0\\
102	0\\
103	0\\
104	0\\
105	0\\
106	0\\
107	0\\
108	0\\
109	0\\
110	0\\
111	0\\
112	0\\
113	0\\
114	0\\
115	0\\
116	0\\
117	0\\
118	0\\
119	0\\
120	0\\
121	0\\
122	0\\
123	0\\
124	0\\
125	0\\
126	0\\
127	0\\
128	0\\
129	0\\
130	0\\
131	0\\
132	0\\
133	0\\
134	0\\
135	0\\
136	0\\
137	0\\
138	0\\
139	0\\
140	0\\
141	0\\
142	0\\
143	0\\
144	0\\
145	0\\
146	0\\
147	0\\
148	0\\
149	0\\
150	0\\
151	0\\
152	0\\
153	0\\
154	0\\
155	0\\
156	0\\
157	0\\
158	0\\
159	0\\
160	0\\
161	0\\
162	0\\
163	0\\
164	0\\
165	0\\
166	0\\
167	0\\
168	0\\
169	0\\
170	0\\
171	0\\
172	0\\
173	0\\
174	0\\
175	0\\
176	0\\
177	0\\
178	0\\
179	0\\
180	0\\
181	0\\
182	0\\
183	0\\
184	0\\
185	0\\
186	0\\
187	0\\
188	0\\
189	0\\
190	0\\
191	0\\
192	0\\
193	0\\
194	0\\
195	0\\
196	0\\
197	0\\
198	0\\
199	0\\
200	0\\
201	0\\
202	0\\
203	0\\
204	0\\
205	0\\
206	0\\
207	0\\
208	0\\
209	0\\
210	0\\
211	0\\
212	0\\
213	0\\
214	0\\
215	0\\
216	0\\
217	0\\
218	0\\
219	0\\
220	0\\
221	0\\
222	0\\
223	0\\
224	0\\
225	0\\
226	0\\
227	0\\
228	0\\
229	0\\
230	0\\
231	0\\
232	0\\
233	0\\
234	0\\
235	0\\
236	0\\
237	0\\
238	0\\
239	0\\
240	0\\
241	0\\
242	0\\
243	0\\
244	0\\
245	0\\
246	0\\
247	0\\
248	0\\
249	0\\
250	0\\
251	0\\
252	0\\
253	0\\
254	0\\
255	0\\
256	0\\
257	0\\
258	0\\
259	0\\
260	0\\
261	0\\
262	0\\
263	0\\
264	0\\
265	0\\
266	0\\
267	0\\
268	0\\
269	0\\
270	0\\
271	0\\
272	0\\
273	0\\
274	0\\
275	0\\
276	0\\
277	0\\
278	0\\
279	0\\
280	0\\
281	0\\
282	0\\
283	0\\
284	0\\
285	0\\
286	0\\
287	0\\
288	0\\
289	0\\
290	0\\
291	0\\
292	0\\
293	0\\
294	0\\
295	0\\
296	0\\
297	0\\
298	0\\
299	0\\
300	0\\
301	0\\
302	0\\
303	0\\
304	0\\
305	0\\
306	0\\
307	0\\
308	0\\
309	0\\
310	0\\
311	0\\
312	0\\
313	0\\
314	0\\
315	0\\
316	0\\
317	0\\
318	0\\
319	0\\
320	0\\
321	0\\
322	0\\
323	0\\
324	0\\
325	0\\
326	0\\
327	0\\
328	0\\
329	0\\
330	0\\
331	0\\
332	0\\
333	0\\
334	0\\
335	0\\
336	0\\
337	0\\
338	0\\
339	0\\
340	0\\
341	0\\
342	0\\
343	0\\
344	0\\
345	0\\
346	0\\
347	0\\
348	0\\
349	0\\
350	0\\
351	0\\
352	0\\
353	0\\
354	0\\
355	0\\
356	0\\
357	0\\
358	0\\
359	0\\
360	0\\
361	0\\
362	0\\
363	0\\
364	0\\
365	0\\
366	0\\
367	0\\
368	0\\
369	0\\
370	0\\
371	0\\
372	0\\
373	0\\
374	0\\
375	0\\
376	0\\
377	0\\
378	0\\
379	0\\
380	0\\
381	0\\
382	0\\
383	0\\
384	0\\
385	0\\
386	0\\
387	0\\
388	0\\
389	0\\
390	0\\
391	0\\
392	0\\
393	0\\
394	0\\
395	0\\
396	0\\
397	0\\
398	0\\
399	0\\
400	0\\
401	0\\
402	0\\
403	0\\
404	0\\
405	0\\
406	0\\
407	0\\
408	0\\
409	0\\
410	0\\
411	0\\
412	0\\
413	0\\
414	0\\
415	0\\
416	0\\
417	0\\
418	0\\
419	0\\
420	0\\
421	0\\
422	0\\
423	0\\
424	0\\
425	0\\
426	0\\
427	0\\
428	0\\
429	0\\
430	0\\
431	0\\
432	0\\
433	0\\
434	0\\
435	0\\
436	0\\
437	0\\
438	0\\
439	0\\
440	0\\
441	0\\
442	0\\
443	0\\
444	0\\
445	0\\
446	0\\
447	0\\
448	0\\
449	0\\
450	0\\
451	0\\
452	0\\
453	0\\
454	0\\
455	0\\
456	0\\
457	0\\
458	0\\
459	0\\
460	0\\
461	0\\
462	0\\
463	0\\
464	0\\
465	0\\
466	0\\
467	0\\
468	0\\
469	0\\
470	0\\
471	0\\
472	0\\
473	0\\
474	0\\
475	0\\
476	0\\
477	0\\
478	0\\
479	0\\
480	0\\
481	0\\
482	0\\
483	0\\
484	0\\
485	0\\
486	0\\
487	0\\
488	0\\
489	0\\
490	0\\
491	0\\
492	0\\
493	0\\
494	0\\
495	0\\
496	0\\
497	0\\
498	0\\
499	0\\
500	0\\
501	0\\
502	0\\
503	0\\
504	0\\
505	0\\
506	0\\
507	0\\
508	0\\
509	0\\
510	0\\
511	0\\
512	0\\
513	0\\
514	0\\
515	0\\
516	0\\
517	0\\
518	0\\
519	0\\
520	0\\
521	0\\
522	0\\
523	0\\
524	0\\
525	0\\
526	0\\
527	0\\
528	0\\
529	0\\
530	0\\
531	0\\
532	0\\
533	0\\
534	0\\
535	0\\
536	0\\
537	0\\
538	0\\
539	0\\
540	0\\
541	0\\
542	0\\
543	0\\
544	0\\
545	0\\
546	0\\
547	0\\
548	0\\
549	0\\
550	0\\
551	0\\
552	0\\
553	0\\
554	0\\
555	0\\
556	0\\
557	0\\
558	0\\
559	0\\
560	0\\
561	0\\
562	0\\
563	0\\
564	0\\
565	0\\
566	0\\
567	0\\
568	0\\
569	0\\
570	0\\
571	0\\
572	0\\
573	0\\
574	0\\
575	0\\
576	0\\
577	0\\
578	0\\
579	0\\
580	0\\
581	0\\
582	0\\
583	0\\
584	0\\
585	0\\
586	0\\
587	0\\
588	0\\
589	0\\
590	0\\
591	0\\
592	0\\
593	0\\
594	0\\
595	0\\
596	0\\
597	0\\
598	0\\
599	0\\
600	0\\
};
\addplot [color=mycolor2,solid,forget plot]
  table[row sep=crcr]{%
1	0\\
2	0\\
3	0\\
4	0\\
5	0\\
6	0\\
7	0\\
8	0\\
9	0\\
10	0\\
11	0\\
12	0\\
13	0\\
14	0\\
15	0\\
16	0\\
17	0\\
18	0\\
19	0\\
20	0\\
21	0\\
22	0\\
23	0\\
24	0\\
25	0\\
26	0\\
27	0\\
28	0\\
29	0\\
30	0\\
31	0\\
32	0\\
33	0\\
34	0\\
35	0\\
36	0\\
37	0\\
38	0\\
39	0\\
40	0\\
41	0\\
42	0\\
43	0\\
44	0\\
45	0\\
46	0\\
47	0\\
48	0\\
49	0\\
50	0\\
51	0\\
52	0\\
53	0\\
54	0\\
55	0\\
56	0\\
57	0\\
58	0\\
59	0\\
60	0\\
61	0\\
62	0\\
63	0\\
64	0\\
65	0\\
66	0\\
67	0\\
68	0\\
69	0\\
70	0\\
71	0\\
72	0\\
73	0\\
74	0\\
75	0\\
76	0\\
77	0\\
78	0\\
79	0\\
80	0\\
81	0\\
82	0\\
83	0\\
84	0\\
85	0\\
86	0\\
87	0\\
88	0\\
89	0\\
90	0\\
91	0\\
92	0\\
93	0\\
94	0\\
95	0\\
96	0\\
97	0\\
98	0\\
99	0\\
100	0\\
101	0\\
102	0\\
103	0\\
104	0\\
105	0\\
106	0\\
107	0\\
108	0\\
109	0\\
110	0\\
111	0\\
112	0\\
113	0\\
114	0\\
115	0\\
116	0\\
117	0\\
118	0\\
119	0\\
120	0\\
121	0\\
122	0\\
123	0\\
124	0\\
125	0\\
126	0\\
127	0\\
128	0\\
129	0\\
130	0\\
131	0\\
132	0\\
133	0\\
134	0\\
135	0\\
136	0\\
137	0\\
138	0\\
139	0\\
140	0\\
141	0\\
142	0\\
143	0\\
144	0\\
145	0\\
146	0\\
147	0\\
148	0\\
149	0\\
150	0\\
151	0\\
152	0\\
153	0\\
154	0\\
155	0\\
156	0\\
157	0\\
158	0\\
159	0\\
160	0\\
161	0\\
162	0\\
163	0\\
164	0\\
165	0\\
166	0\\
167	0\\
168	0\\
169	0\\
170	0\\
171	0\\
172	0\\
173	0\\
174	0\\
175	0\\
176	0\\
177	0\\
178	0\\
179	0\\
180	0\\
181	0\\
182	0\\
183	0\\
184	0\\
185	0\\
186	0\\
187	0\\
188	0\\
189	0\\
190	0\\
191	0\\
192	0\\
193	0\\
194	0\\
195	0\\
196	0\\
197	0\\
198	0\\
199	0\\
200	0\\
201	0\\
202	0\\
203	0\\
204	0\\
205	0\\
206	0\\
207	0\\
208	0\\
209	0\\
210	0\\
211	0\\
212	0\\
213	0\\
214	0\\
215	0\\
216	0\\
217	0\\
218	0\\
219	0\\
220	0\\
221	0\\
222	0\\
223	0\\
224	0\\
225	0\\
226	0\\
227	0\\
228	0\\
229	0\\
230	0\\
231	0\\
232	0\\
233	0\\
234	0\\
235	0\\
236	0\\
237	0\\
238	0\\
239	0\\
240	0\\
241	0\\
242	0\\
243	0\\
244	0\\
245	0\\
246	0\\
247	0\\
248	0\\
249	0\\
250	0\\
251	0\\
252	0\\
253	0\\
254	0\\
255	0\\
256	0\\
257	0\\
258	0\\
259	0\\
260	0\\
261	0\\
262	0\\
263	0\\
264	0\\
265	0\\
266	0\\
267	0\\
268	0\\
269	0\\
270	0\\
271	0\\
272	0\\
273	0\\
274	0\\
275	0\\
276	0\\
277	0\\
278	0\\
279	0\\
280	0\\
281	0\\
282	0\\
283	0\\
284	0\\
285	0\\
286	0\\
287	0\\
288	0\\
289	0\\
290	0\\
291	0\\
292	0\\
293	0\\
294	0\\
295	0\\
296	0\\
297	0\\
298	0\\
299	0\\
300	0\\
301	0\\
302	0\\
303	0\\
304	0\\
305	0\\
306	0\\
307	0\\
308	0\\
309	0\\
310	0\\
311	0\\
312	0\\
313	0\\
314	0\\
315	0\\
316	0\\
317	0\\
318	0\\
319	0\\
320	0\\
321	0\\
322	0\\
323	0\\
324	0\\
325	0\\
326	0\\
327	0\\
328	0\\
329	0\\
330	0\\
331	0\\
332	0\\
333	0\\
334	0\\
335	0\\
336	0\\
337	0\\
338	0\\
339	0\\
340	0\\
341	0\\
342	0\\
343	0\\
344	0\\
345	0\\
346	0\\
347	0\\
348	0\\
349	0\\
350	0\\
351	0\\
352	0\\
353	0\\
354	0\\
355	0\\
356	0\\
357	0\\
358	0\\
359	0\\
360	0\\
361	0\\
362	0\\
363	0\\
364	0\\
365	0\\
366	0\\
367	0\\
368	0\\
369	0\\
370	0\\
371	0\\
372	0\\
373	0\\
374	0\\
375	0\\
376	0\\
377	0\\
378	0\\
379	0\\
380	0\\
381	0\\
382	0\\
383	0\\
384	0\\
385	0\\
386	0\\
387	0\\
388	0\\
389	0\\
390	0\\
391	0\\
392	0\\
393	0\\
394	0\\
395	0\\
396	0\\
397	0\\
398	0\\
399	0\\
400	0\\
401	0\\
402	0\\
403	0\\
404	0\\
405	0\\
406	0\\
407	0\\
408	0\\
409	0\\
410	0\\
411	0\\
412	0\\
413	0\\
414	0\\
415	0\\
416	0\\
417	0\\
418	0\\
419	0\\
420	0\\
421	0\\
422	0\\
423	0\\
424	0\\
425	0\\
426	0\\
427	0\\
428	0\\
429	0\\
430	0\\
431	0\\
432	0\\
433	0\\
434	0\\
435	0\\
436	0\\
437	0\\
438	0\\
439	0\\
440	0\\
441	0\\
442	0\\
443	0\\
444	0\\
445	0\\
446	0\\
447	0\\
448	0\\
449	0\\
450	0\\
451	0\\
452	0\\
453	0\\
454	0\\
455	0\\
456	0\\
457	0\\
458	0\\
459	0\\
460	0\\
461	0\\
462	0\\
463	0\\
464	0\\
465	0\\
466	0\\
467	0\\
468	0\\
469	0\\
470	0\\
471	0\\
472	0\\
473	0\\
474	0\\
475	0\\
476	0\\
477	0\\
478	0\\
479	0\\
480	0\\
481	0\\
482	0\\
483	0\\
484	0\\
485	0\\
486	0\\
487	0\\
488	0\\
489	0\\
490	0\\
491	0\\
492	0\\
493	0\\
494	0\\
495	0\\
496	0\\
497	0\\
498	0\\
499	0\\
500	0\\
501	0\\
502	0\\
503	0\\
504	0\\
505	0\\
506	0\\
507	0\\
508	0\\
509	0\\
510	0\\
511	0\\
512	0\\
513	0\\
514	0\\
515	0\\
516	0\\
517	0\\
518	0\\
519	0\\
520	0\\
521	0\\
522	0\\
523	0\\
524	0\\
525	0\\
526	0\\
527	0\\
528	0\\
529	0\\
530	0\\
531	0\\
532	0\\
533	0\\
534	0\\
535	0\\
536	0\\
537	0\\
538	0\\
539	0\\
540	0\\
541	0\\
542	0\\
543	0\\
544	0\\
545	0\\
546	0\\
547	0\\
548	0\\
549	0\\
550	0\\
551	0\\
552	0\\
553	0\\
554	0\\
555	0\\
556	0\\
557	0\\
558	0\\
559	0\\
560	0\\
561	0\\
562	0\\
563	0\\
564	0\\
565	0\\
566	0\\
567	0\\
568	0\\
569	0\\
570	0\\
571	0\\
572	0\\
573	0\\
574	0\\
575	0\\
576	0\\
577	0\\
578	0\\
579	0\\
580	0\\
581	0\\
582	0\\
583	0\\
584	0\\
585	0\\
586	0\\
587	0\\
588	0\\
589	0\\
590	0\\
591	0\\
592	0\\
593	0\\
594	0\\
595	0\\
596	0\\
597	0\\
598	0\\
599	0\\
600	0\\
};
\addplot [color=mycolor3,solid,forget plot]
  table[row sep=crcr]{%
1	0\\
2	0\\
3	0\\
4	0\\
5	0\\
6	0\\
7	0\\
8	0\\
9	0\\
10	0\\
11	0\\
12	0\\
13	0\\
14	0\\
15	0\\
16	0\\
17	0\\
18	0\\
19	0\\
20	0\\
21	0\\
22	0\\
23	0\\
24	0\\
25	0\\
26	0\\
27	0\\
28	0\\
29	0\\
30	0\\
31	0\\
32	0\\
33	0\\
34	0\\
35	0\\
36	0\\
37	0\\
38	0\\
39	0\\
40	0\\
41	0\\
42	0\\
43	0\\
44	0\\
45	0\\
46	0\\
47	0\\
48	0\\
49	0\\
50	0\\
51	0\\
52	0\\
53	0\\
54	0\\
55	0\\
56	0\\
57	0\\
58	0\\
59	0\\
60	0\\
61	0\\
62	0\\
63	0\\
64	0\\
65	0\\
66	0\\
67	0\\
68	0\\
69	0\\
70	0\\
71	0\\
72	0\\
73	0\\
74	0\\
75	0\\
76	0\\
77	0\\
78	0\\
79	0\\
80	0\\
81	0\\
82	0\\
83	0\\
84	0\\
85	0\\
86	0\\
87	0\\
88	0\\
89	0\\
90	0\\
91	0\\
92	0\\
93	0\\
94	0\\
95	0\\
96	0\\
97	0\\
98	0\\
99	0\\
100	0\\
101	0\\
102	0\\
103	0\\
104	0\\
105	0\\
106	0\\
107	0\\
108	0\\
109	0\\
110	0\\
111	0\\
112	0\\
113	0\\
114	0\\
115	0\\
116	0\\
117	0\\
118	0\\
119	0\\
120	0\\
121	0\\
122	0\\
123	0\\
124	0\\
125	0\\
126	0\\
127	0\\
128	0\\
129	0\\
130	0\\
131	0\\
132	0\\
133	0\\
134	0\\
135	0\\
136	0\\
137	0\\
138	0\\
139	0\\
140	0\\
141	0\\
142	0\\
143	0\\
144	0\\
145	0\\
146	0\\
147	0\\
148	0\\
149	0\\
150	0\\
151	0\\
152	0\\
153	0\\
154	0\\
155	0\\
156	0\\
157	0\\
158	0\\
159	0\\
160	0\\
161	0\\
162	0\\
163	0\\
164	0\\
165	0\\
166	0\\
167	0\\
168	0\\
169	0\\
170	0\\
171	0\\
172	0\\
173	0\\
174	0\\
175	0\\
176	0\\
177	0\\
178	0\\
179	0\\
180	0\\
181	0\\
182	0\\
183	0\\
184	0\\
185	0\\
186	0\\
187	0\\
188	0\\
189	0\\
190	0\\
191	0\\
192	0\\
193	0\\
194	0\\
195	0\\
196	0\\
197	0\\
198	0\\
199	0\\
200	0\\
201	0\\
202	0\\
203	0\\
204	0\\
205	0\\
206	0\\
207	0\\
208	0\\
209	0\\
210	0\\
211	0\\
212	0\\
213	0\\
214	0\\
215	0\\
216	0\\
217	0\\
218	0\\
219	0\\
220	0\\
221	0\\
222	0\\
223	0\\
224	0\\
225	0\\
226	0\\
227	0\\
228	0\\
229	0\\
230	0\\
231	0\\
232	0\\
233	0\\
234	0\\
235	0\\
236	0\\
237	0\\
238	0\\
239	0\\
240	0\\
241	0\\
242	0\\
243	0\\
244	0\\
245	0\\
246	0\\
247	0\\
248	0\\
249	0\\
250	0\\
251	0\\
252	0\\
253	0\\
254	0\\
255	0\\
256	0\\
257	0\\
258	0\\
259	0\\
260	0\\
261	0\\
262	0\\
263	0\\
264	0\\
265	0\\
266	0\\
267	0\\
268	0\\
269	0\\
270	0\\
271	0\\
272	0\\
273	0\\
274	0\\
275	0\\
276	0\\
277	0\\
278	0\\
279	0\\
280	0\\
281	0\\
282	0\\
283	0\\
284	0\\
285	0\\
286	0\\
287	0\\
288	0\\
289	0\\
290	0\\
291	0\\
292	0\\
293	0\\
294	0\\
295	0\\
296	0\\
297	0\\
298	0\\
299	0\\
300	0\\
301	0\\
302	0\\
303	0\\
304	0\\
305	0\\
306	0\\
307	0\\
308	0\\
309	0\\
310	0\\
311	0\\
312	0\\
313	0\\
314	0\\
315	0\\
316	0\\
317	0\\
318	0\\
319	0\\
320	0\\
321	0\\
322	0\\
323	0\\
324	0\\
325	0\\
326	0\\
327	0\\
328	0\\
329	0\\
330	0\\
331	0\\
332	0\\
333	0\\
334	0\\
335	0\\
336	0\\
337	0\\
338	0\\
339	0\\
340	0\\
341	0\\
342	0\\
343	0\\
344	0\\
345	0\\
346	0\\
347	0\\
348	0\\
349	0\\
350	0\\
351	0\\
352	0\\
353	0\\
354	0\\
355	0\\
356	0\\
357	0\\
358	0\\
359	0\\
360	0\\
361	0\\
362	0\\
363	0\\
364	0\\
365	0\\
366	0\\
367	0\\
368	0\\
369	0\\
370	0\\
371	0\\
372	0\\
373	0\\
374	0\\
375	0\\
376	0\\
377	0\\
378	0\\
379	0\\
380	0\\
381	0\\
382	0\\
383	0\\
384	0\\
385	0\\
386	0\\
387	0\\
388	0\\
389	0\\
390	0\\
391	0\\
392	0\\
393	0\\
394	0\\
395	0\\
396	0\\
397	0\\
398	0\\
399	0\\
400	0\\
401	0\\
402	0\\
403	0\\
404	0\\
405	0\\
406	0\\
407	0\\
408	0\\
409	0\\
410	0\\
411	0\\
412	0\\
413	0\\
414	0\\
415	0\\
416	0\\
417	0\\
418	0\\
419	0\\
420	0\\
421	0\\
422	0\\
423	0\\
424	0\\
425	0\\
426	0\\
427	0\\
428	0\\
429	0\\
430	0\\
431	0\\
432	0\\
433	0\\
434	0\\
435	0\\
436	0\\
437	0\\
438	0\\
439	0\\
440	0\\
441	0\\
442	0\\
443	0\\
444	0\\
445	0\\
446	0\\
447	0\\
448	0\\
449	0\\
450	0\\
451	0\\
452	0\\
453	0\\
454	0\\
455	0\\
456	0\\
457	0\\
458	0\\
459	0\\
460	0\\
461	0\\
462	0\\
463	0\\
464	0\\
465	0\\
466	0\\
467	0\\
468	0\\
469	0\\
470	0\\
471	0\\
472	0\\
473	0\\
474	0\\
475	0\\
476	0\\
477	0\\
478	0\\
479	0\\
480	0\\
481	0\\
482	0\\
483	0\\
484	0\\
485	0\\
486	0\\
487	0\\
488	0\\
489	0\\
490	0\\
491	0\\
492	0\\
493	0\\
494	0\\
495	0\\
496	0\\
497	0\\
498	0\\
499	0\\
500	0\\
501	0\\
502	0\\
503	0\\
504	0\\
505	0\\
506	0\\
507	0\\
508	0\\
509	0\\
510	0\\
511	0\\
512	0\\
513	0\\
514	0\\
515	0\\
516	0\\
517	0\\
518	0\\
519	0\\
520	0\\
521	0\\
522	0\\
523	0\\
524	0\\
525	0\\
526	0\\
527	0\\
528	0\\
529	0\\
530	0\\
531	0\\
532	0\\
533	0\\
534	0\\
535	0\\
536	0\\
537	0\\
538	0\\
539	0\\
540	0\\
541	0\\
542	0\\
543	0\\
544	0\\
545	0\\
546	0\\
547	0\\
548	0\\
549	0\\
550	0\\
551	0\\
552	0\\
553	0\\
554	0\\
555	0\\
556	0\\
557	0\\
558	0\\
559	0\\
560	0\\
561	0\\
562	0\\
563	0\\
564	0\\
565	0\\
566	0\\
567	0\\
568	0\\
569	0\\
570	0\\
571	0\\
572	0\\
573	0\\
574	0\\
575	0\\
576	0\\
577	0\\
578	0\\
579	0\\
580	0\\
581	0\\
582	0\\
583	0\\
584	0\\
585	0\\
586	0\\
587	0\\
588	0\\
589	0\\
590	0\\
591	0\\
592	0\\
593	0\\
594	0\\
595	0\\
596	0\\
597	0\\
598	0\\
599	0\\
600	0\\
};
\addplot [color=mycolor4,solid,forget plot]
  table[row sep=crcr]{%
1	0\\
2	0\\
3	0\\
4	0\\
5	0\\
6	0\\
7	0\\
8	0\\
9	0\\
10	0\\
11	0\\
12	0\\
13	0\\
14	0\\
15	0\\
16	0\\
17	0\\
18	0\\
19	0\\
20	0\\
21	0\\
22	0\\
23	0\\
24	0\\
25	0\\
26	0\\
27	0\\
28	0\\
29	0\\
30	0\\
31	0\\
32	0\\
33	0\\
34	0\\
35	0\\
36	0\\
37	0\\
38	0\\
39	0\\
40	0\\
41	0\\
42	0\\
43	0\\
44	0\\
45	0\\
46	0\\
47	0\\
48	0\\
49	0\\
50	0\\
51	0\\
52	0\\
53	0\\
54	0\\
55	0\\
56	0\\
57	0\\
58	0\\
59	0\\
60	0\\
61	0\\
62	0\\
63	0\\
64	0\\
65	0\\
66	0\\
67	0\\
68	0\\
69	0\\
70	0\\
71	0\\
72	0\\
73	0\\
74	0\\
75	0\\
76	0\\
77	0\\
78	0\\
79	0\\
80	0\\
81	0\\
82	0\\
83	0\\
84	0\\
85	0\\
86	0\\
87	0\\
88	0\\
89	0\\
90	0\\
91	0\\
92	0\\
93	0\\
94	0\\
95	0\\
96	0\\
97	0\\
98	0\\
99	0\\
100	0\\
101	0\\
102	0\\
103	0\\
104	0\\
105	0\\
106	0\\
107	0\\
108	0\\
109	0\\
110	0\\
111	0\\
112	0\\
113	0\\
114	0\\
115	0\\
116	0\\
117	0\\
118	0\\
119	0\\
120	0\\
121	0\\
122	0\\
123	0\\
124	0\\
125	0\\
126	0\\
127	0\\
128	0\\
129	0\\
130	0\\
131	0\\
132	0\\
133	0\\
134	0\\
135	0\\
136	0\\
137	0\\
138	0\\
139	0\\
140	0\\
141	0\\
142	0\\
143	0\\
144	0\\
145	0\\
146	0\\
147	0\\
148	0\\
149	0\\
150	0\\
151	0\\
152	0\\
153	0\\
154	0\\
155	0\\
156	0\\
157	0\\
158	0\\
159	0\\
160	0\\
161	0\\
162	0\\
163	0\\
164	0\\
165	0\\
166	0\\
167	0\\
168	0\\
169	0\\
170	0\\
171	0\\
172	0\\
173	0\\
174	0\\
175	0\\
176	0\\
177	0\\
178	0\\
179	0\\
180	0\\
181	0\\
182	0\\
183	0\\
184	0\\
185	0\\
186	0\\
187	0\\
188	0\\
189	0\\
190	0\\
191	0\\
192	0\\
193	0\\
194	0\\
195	0\\
196	0\\
197	0\\
198	0\\
199	0\\
200	0\\
201	0\\
202	0\\
203	0\\
204	0\\
205	0\\
206	0\\
207	0\\
208	0\\
209	0\\
210	0\\
211	0\\
212	0\\
213	0\\
214	0\\
215	0\\
216	0\\
217	0\\
218	0\\
219	0\\
220	0\\
221	0\\
222	0\\
223	0\\
224	0\\
225	0\\
226	0\\
227	0\\
228	0\\
229	0\\
230	0\\
231	0\\
232	0\\
233	0\\
234	0\\
235	0\\
236	0\\
237	0\\
238	0\\
239	0\\
240	0\\
241	0\\
242	0\\
243	0\\
244	0\\
245	0\\
246	0\\
247	0\\
248	0\\
249	0\\
250	0\\
251	0\\
252	0\\
253	0\\
254	0\\
255	0\\
256	0\\
257	0\\
258	0\\
259	0\\
260	0\\
261	0\\
262	0\\
263	0\\
264	0\\
265	0\\
266	0\\
267	0\\
268	0\\
269	0\\
270	0\\
271	0\\
272	0\\
273	0\\
274	0\\
275	0\\
276	0\\
277	0\\
278	0\\
279	0\\
280	0\\
281	0\\
282	0\\
283	0\\
284	0\\
285	0\\
286	0\\
287	0\\
288	0\\
289	0\\
290	0\\
291	0\\
292	0\\
293	0\\
294	0\\
295	0\\
296	0\\
297	0\\
298	0\\
299	0\\
300	0\\
301	0\\
302	0\\
303	0\\
304	0\\
305	0\\
306	0\\
307	0\\
308	0\\
309	0\\
310	0\\
311	0\\
312	0\\
313	0\\
314	0\\
315	0\\
316	0\\
317	0\\
318	0\\
319	0\\
320	0\\
321	0\\
322	0\\
323	0\\
324	0\\
325	0\\
326	0\\
327	0\\
328	0\\
329	0\\
330	0\\
331	0\\
332	0\\
333	0\\
334	0\\
335	0\\
336	0\\
337	0\\
338	0\\
339	0\\
340	0\\
341	0\\
342	0\\
343	0\\
344	0\\
345	0\\
346	0\\
347	0\\
348	0\\
349	0\\
350	0\\
351	0\\
352	0\\
353	0\\
354	0\\
355	0\\
356	0\\
357	0\\
358	0\\
359	0\\
360	0\\
361	0\\
362	0\\
363	0\\
364	0\\
365	0\\
366	0\\
367	0\\
368	0\\
369	0\\
370	0\\
371	0\\
372	0\\
373	0\\
374	0\\
375	0\\
376	0\\
377	0\\
378	0\\
379	0\\
380	0\\
381	0\\
382	0\\
383	0\\
384	0\\
385	0\\
386	0\\
387	0\\
388	0\\
389	0\\
390	0\\
391	0\\
392	0\\
393	0\\
394	0\\
395	0\\
396	0\\
397	0\\
398	0\\
399	0\\
400	0\\
401	0\\
402	0\\
403	0\\
404	0\\
405	0\\
406	0\\
407	0\\
408	0\\
409	0\\
410	0\\
411	0\\
412	0\\
413	0\\
414	0\\
415	0\\
416	0\\
417	0\\
418	0\\
419	0\\
420	0\\
421	0\\
422	0\\
423	0\\
424	0\\
425	0\\
426	0\\
427	0\\
428	0\\
429	0\\
430	0\\
431	0\\
432	0\\
433	0\\
434	0\\
435	0\\
436	0\\
437	0\\
438	0\\
439	0\\
440	0\\
441	0\\
442	0\\
443	0\\
444	0\\
445	0\\
446	0\\
447	0\\
448	0\\
449	0\\
450	0\\
451	0\\
452	0\\
453	0\\
454	0\\
455	0\\
456	0\\
457	0\\
458	0\\
459	0\\
460	0\\
461	0\\
462	0\\
463	0\\
464	0\\
465	0\\
466	0\\
467	0\\
468	0\\
469	0\\
470	0\\
471	0\\
472	0\\
473	0\\
474	0\\
475	0\\
476	0\\
477	0\\
478	0\\
479	0\\
480	0\\
481	0\\
482	0\\
483	0\\
484	0\\
485	0\\
486	0\\
487	0\\
488	0\\
489	0\\
490	0\\
491	0\\
492	0\\
493	0\\
494	0\\
495	0\\
496	0\\
497	0\\
498	0\\
499	0\\
500	0\\
501	0\\
502	0\\
503	0\\
504	0\\
505	0\\
506	0\\
507	0\\
508	0\\
509	0\\
510	0\\
511	0\\
512	0\\
513	0\\
514	0\\
515	0\\
516	0\\
517	0\\
518	0\\
519	0\\
520	0\\
521	0\\
522	0\\
523	0\\
524	0\\
525	0\\
526	0\\
527	0\\
528	0\\
529	0\\
530	0\\
531	0\\
532	0\\
533	0\\
534	0\\
535	0\\
536	0\\
537	0\\
538	0\\
539	0\\
540	0\\
541	0\\
542	0\\
543	0\\
544	0\\
545	0\\
546	0\\
547	0\\
548	0\\
549	0\\
550	0\\
551	0\\
552	0\\
553	0\\
554	0\\
555	0\\
556	0\\
557	0\\
558	0\\
559	0\\
560	0\\
561	0\\
562	0\\
563	0\\
564	0\\
565	0\\
566	0\\
567	0\\
568	0\\
569	0\\
570	0\\
571	0\\
572	0\\
573	0\\
574	0\\
575	0\\
576	0\\
577	0\\
578	0\\
579	0\\
580	0\\
581	0\\
582	0\\
583	0\\
584	0\\
585	0\\
586	0\\
587	0\\
588	0\\
589	0\\
590	0\\
591	0\\
592	0\\
593	0\\
594	0\\
595	0\\
596	0\\
597	0\\
598	0\\
599	0\\
600	0\\
};
\addplot [color=mycolor5,solid,forget plot]
  table[row sep=crcr]{%
1	0\\
2	0\\
3	0\\
4	0\\
5	0\\
6	0\\
7	0\\
8	0\\
9	0\\
10	0\\
11	0\\
12	0\\
13	0\\
14	0\\
15	0\\
16	0\\
17	0\\
18	0\\
19	0\\
20	0\\
21	0\\
22	0\\
23	0\\
24	0\\
25	0\\
26	0\\
27	0\\
28	0\\
29	0\\
30	0\\
31	0\\
32	0\\
33	0\\
34	0\\
35	0\\
36	0\\
37	0\\
38	0\\
39	0\\
40	0\\
41	0\\
42	0\\
43	0\\
44	0\\
45	0\\
46	0\\
47	0\\
48	0\\
49	0\\
50	0\\
51	0\\
52	0\\
53	0\\
54	0\\
55	0\\
56	0\\
57	0\\
58	0\\
59	0\\
60	0\\
61	0\\
62	0\\
63	0\\
64	0\\
65	0\\
66	0\\
67	0\\
68	0\\
69	0\\
70	0\\
71	0\\
72	0\\
73	0\\
74	0\\
75	0\\
76	0\\
77	0\\
78	0\\
79	0\\
80	0\\
81	0\\
82	0\\
83	0\\
84	0\\
85	0\\
86	0\\
87	0\\
88	0\\
89	0\\
90	0\\
91	0\\
92	0\\
93	0\\
94	0\\
95	0\\
96	0\\
97	0\\
98	0\\
99	0\\
100	0\\
101	0\\
102	0\\
103	0\\
104	0\\
105	0\\
106	0\\
107	0\\
108	0\\
109	0\\
110	0\\
111	0\\
112	0\\
113	0\\
114	0\\
115	0\\
116	0\\
117	0\\
118	0\\
119	0\\
120	0\\
121	0\\
122	0\\
123	0\\
124	0\\
125	0\\
126	0\\
127	0\\
128	0\\
129	0\\
130	0\\
131	0\\
132	0\\
133	0\\
134	0\\
135	0\\
136	0\\
137	0\\
138	0\\
139	0\\
140	0\\
141	0\\
142	0\\
143	0\\
144	0\\
145	0\\
146	0\\
147	0\\
148	0\\
149	0\\
150	0\\
151	0\\
152	0\\
153	0\\
154	0\\
155	0\\
156	0\\
157	0\\
158	0\\
159	0\\
160	0\\
161	0\\
162	0\\
163	0\\
164	0\\
165	0\\
166	0\\
167	0\\
168	0\\
169	0\\
170	0\\
171	0\\
172	0\\
173	0\\
174	0\\
175	0\\
176	0\\
177	0\\
178	0\\
179	0\\
180	0\\
181	0\\
182	0\\
183	0\\
184	0\\
185	0\\
186	0\\
187	0\\
188	0\\
189	0\\
190	0\\
191	0\\
192	0\\
193	0\\
194	0\\
195	0\\
196	0\\
197	0\\
198	0\\
199	0\\
200	0\\
201	0\\
202	0\\
203	0\\
204	0\\
205	0\\
206	0\\
207	0\\
208	0\\
209	0\\
210	0\\
211	0\\
212	0\\
213	0\\
214	0\\
215	0\\
216	0\\
217	0\\
218	0\\
219	0\\
220	0\\
221	0\\
222	0\\
223	0\\
224	0\\
225	0\\
226	0\\
227	0\\
228	0\\
229	0\\
230	0\\
231	0\\
232	0\\
233	0\\
234	0\\
235	0\\
236	0\\
237	0\\
238	0\\
239	0\\
240	0\\
241	0\\
242	0\\
243	0\\
244	0\\
245	0\\
246	0\\
247	0\\
248	0\\
249	0\\
250	0\\
251	0\\
252	0\\
253	0\\
254	0\\
255	0\\
256	0\\
257	0\\
258	0\\
259	0\\
260	0\\
261	0\\
262	0\\
263	0\\
264	0\\
265	0\\
266	0\\
267	0\\
268	0\\
269	0\\
270	0\\
271	0\\
272	0\\
273	0\\
274	0\\
275	0\\
276	0\\
277	0\\
278	0\\
279	0\\
280	0\\
281	0\\
282	0\\
283	0\\
284	0\\
285	0\\
286	0\\
287	0\\
288	0\\
289	0\\
290	0\\
291	0\\
292	0\\
293	0\\
294	0\\
295	0\\
296	0\\
297	0\\
298	0\\
299	0\\
300	0\\
301	0\\
302	0\\
303	0\\
304	0\\
305	0\\
306	0\\
307	0\\
308	0\\
309	0\\
310	0\\
311	0\\
312	0\\
313	0\\
314	0\\
315	0\\
316	0\\
317	0\\
318	0\\
319	0\\
320	0\\
321	0\\
322	0\\
323	0\\
324	0\\
325	0\\
326	0\\
327	0\\
328	0\\
329	0\\
330	0\\
331	0\\
332	0\\
333	0\\
334	0\\
335	0\\
336	0\\
337	0\\
338	0\\
339	0\\
340	0\\
341	0\\
342	0\\
343	0\\
344	0\\
345	0\\
346	0\\
347	0\\
348	0\\
349	0\\
350	0\\
351	0\\
352	0\\
353	0\\
354	0\\
355	0\\
356	0\\
357	0\\
358	0\\
359	0\\
360	0\\
361	0\\
362	0\\
363	0\\
364	0\\
365	0\\
366	0\\
367	0\\
368	0\\
369	0\\
370	0\\
371	0\\
372	0\\
373	0\\
374	0\\
375	0\\
376	0\\
377	0\\
378	0\\
379	0\\
380	0\\
381	0\\
382	0\\
383	0\\
384	0\\
385	0\\
386	0\\
387	0\\
388	0\\
389	0\\
390	0\\
391	0\\
392	0\\
393	0\\
394	0\\
395	0\\
396	0\\
397	0\\
398	0\\
399	0\\
400	0\\
401	0\\
402	0\\
403	0\\
404	0\\
405	0\\
406	0\\
407	0\\
408	0\\
409	0\\
410	0\\
411	0\\
412	0\\
413	0\\
414	0\\
415	0\\
416	0\\
417	0\\
418	0\\
419	0\\
420	0\\
421	0\\
422	0\\
423	0\\
424	0\\
425	0\\
426	0\\
427	0\\
428	0\\
429	0\\
430	0\\
431	0\\
432	0\\
433	0\\
434	0\\
435	0\\
436	0\\
437	0\\
438	0\\
439	0\\
440	0\\
441	0\\
442	0\\
443	0\\
444	0\\
445	0\\
446	0\\
447	0\\
448	0\\
449	0\\
450	0\\
451	0\\
452	0\\
453	0\\
454	0\\
455	0\\
456	0\\
457	0\\
458	0\\
459	0\\
460	0\\
461	0\\
462	0\\
463	0\\
464	0\\
465	0\\
466	0\\
467	0\\
468	0\\
469	0\\
470	0\\
471	0\\
472	0\\
473	0\\
474	0\\
475	0\\
476	0\\
477	0\\
478	0\\
479	0\\
480	0\\
481	0\\
482	0\\
483	0\\
484	0\\
485	0\\
486	0\\
487	0\\
488	0\\
489	0\\
490	0\\
491	0\\
492	0\\
493	0\\
494	0\\
495	0\\
496	0\\
497	0\\
498	0\\
499	0\\
500	0\\
501	0\\
502	0\\
503	0\\
504	0\\
505	0\\
506	0\\
507	0\\
508	0\\
509	0\\
510	0\\
511	0\\
512	0\\
513	0\\
514	0\\
515	0\\
516	0\\
517	0\\
518	0\\
519	0\\
520	0\\
521	0\\
522	0\\
523	0\\
524	0\\
525	0\\
526	0\\
527	0\\
528	0\\
529	0\\
530	0\\
531	0\\
532	0\\
533	0\\
534	0\\
535	0\\
536	0\\
537	0\\
538	0\\
539	0\\
540	0\\
541	0\\
542	0\\
543	0\\
544	0\\
545	0\\
546	0\\
547	0\\
548	0\\
549	0\\
550	0\\
551	0\\
552	0\\
553	0\\
554	0\\
555	0\\
556	0\\
557	0\\
558	0\\
559	0\\
560	0\\
561	0\\
562	0\\
563	0\\
564	0\\
565	0\\
566	0\\
567	0\\
568	0\\
569	0\\
570	0\\
571	0\\
572	0\\
573	0\\
574	0\\
575	0\\
576	0\\
577	0\\
578	0\\
579	0\\
580	0\\
581	0\\
582	0\\
583	0\\
584	0\\
585	0\\
586	0\\
587	0\\
588	0\\
589	0\\
590	0\\
591	0\\
592	0\\
593	0\\
594	0\\
595	0\\
596	0\\
597	0\\
598	0\\
599	0\\
600	0\\
};
\addplot [color=mycolor6,solid,forget plot]
  table[row sep=crcr]{%
1	0\\
2	0\\
3	0\\
4	0\\
5	0\\
6	0\\
7	0\\
8	0\\
9	0\\
10	0\\
11	0\\
12	0\\
13	0\\
14	0\\
15	0\\
16	0\\
17	0\\
18	0\\
19	0\\
20	0\\
21	0\\
22	0\\
23	0\\
24	0\\
25	0\\
26	0\\
27	0\\
28	0\\
29	0\\
30	0\\
31	0\\
32	0\\
33	0\\
34	0\\
35	0\\
36	0\\
37	0\\
38	0\\
39	0\\
40	0\\
41	0\\
42	0\\
43	0\\
44	0\\
45	0\\
46	0\\
47	0\\
48	0\\
49	0\\
50	0\\
51	0\\
52	0\\
53	0\\
54	0\\
55	0\\
56	0\\
57	0\\
58	0\\
59	0\\
60	0\\
61	0\\
62	0\\
63	0\\
64	0\\
65	0\\
66	0\\
67	0\\
68	0\\
69	0\\
70	0\\
71	0\\
72	0\\
73	0\\
74	0\\
75	0\\
76	0\\
77	0\\
78	0\\
79	0\\
80	0\\
81	0\\
82	0\\
83	0\\
84	0\\
85	0\\
86	0\\
87	0\\
88	0\\
89	0\\
90	0\\
91	0\\
92	0\\
93	0\\
94	0\\
95	0\\
96	0\\
97	0\\
98	0\\
99	0\\
100	0\\
101	0\\
102	0\\
103	0\\
104	0\\
105	0\\
106	0\\
107	0\\
108	0\\
109	0\\
110	0\\
111	0\\
112	0\\
113	0\\
114	0\\
115	0\\
116	0\\
117	0\\
118	0\\
119	0\\
120	0\\
121	0\\
122	0\\
123	0\\
124	0\\
125	0\\
126	0\\
127	0\\
128	0\\
129	0\\
130	0\\
131	0\\
132	0\\
133	0\\
134	0\\
135	0\\
136	0\\
137	0\\
138	0\\
139	0\\
140	0\\
141	0\\
142	0\\
143	0\\
144	0\\
145	0\\
146	0\\
147	0\\
148	0\\
149	0\\
150	0\\
151	0\\
152	0\\
153	0\\
154	0\\
155	0\\
156	0\\
157	0\\
158	0\\
159	0\\
160	0\\
161	0\\
162	0\\
163	0\\
164	0\\
165	0\\
166	0\\
167	0\\
168	0\\
169	0\\
170	0\\
171	0\\
172	0\\
173	0\\
174	0\\
175	0\\
176	0\\
177	0\\
178	0\\
179	0\\
180	0\\
181	0\\
182	0\\
183	0\\
184	0\\
185	0\\
186	0\\
187	0\\
188	0\\
189	0\\
190	0\\
191	0\\
192	0\\
193	0\\
194	0\\
195	0\\
196	0\\
197	0\\
198	0\\
199	0\\
200	0\\
201	0\\
202	0\\
203	0\\
204	0\\
205	0\\
206	0\\
207	0\\
208	0\\
209	0\\
210	0\\
211	0\\
212	0\\
213	0\\
214	0\\
215	0\\
216	0\\
217	0\\
218	0\\
219	0\\
220	0\\
221	0\\
222	0\\
223	0\\
224	0\\
225	0\\
226	0\\
227	0\\
228	0\\
229	0\\
230	0\\
231	0\\
232	0\\
233	0\\
234	0\\
235	0\\
236	0\\
237	0\\
238	0\\
239	0\\
240	0\\
241	0\\
242	0\\
243	0\\
244	0\\
245	0\\
246	0\\
247	0\\
248	0\\
249	0\\
250	0\\
251	0\\
252	0\\
253	0\\
254	0\\
255	0\\
256	0\\
257	0\\
258	0\\
259	0\\
260	0\\
261	0\\
262	0\\
263	0\\
264	0\\
265	0\\
266	0\\
267	0\\
268	0\\
269	0\\
270	0\\
271	0\\
272	0\\
273	0\\
274	0\\
275	0\\
276	0\\
277	0\\
278	0\\
279	0\\
280	0\\
281	0\\
282	0\\
283	0\\
284	0\\
285	0\\
286	0\\
287	0\\
288	0\\
289	0\\
290	0\\
291	0\\
292	0\\
293	0\\
294	0\\
295	0\\
296	0\\
297	0\\
298	0\\
299	0\\
300	0\\
301	0\\
302	0\\
303	0\\
304	0\\
305	0\\
306	0\\
307	0\\
308	0\\
309	0\\
310	0\\
311	0\\
312	0\\
313	0\\
314	0\\
315	0\\
316	0\\
317	0\\
318	0\\
319	0\\
320	0\\
321	0\\
322	0\\
323	0\\
324	0\\
325	0\\
326	0\\
327	0\\
328	0\\
329	0\\
330	0\\
331	0\\
332	0\\
333	0\\
334	0\\
335	0\\
336	0\\
337	0\\
338	0\\
339	0\\
340	0\\
341	0\\
342	0\\
343	0\\
344	0\\
345	0\\
346	0\\
347	0\\
348	0\\
349	0\\
350	0\\
351	0\\
352	0\\
353	0\\
354	0\\
355	0\\
356	0\\
357	0\\
358	0\\
359	0\\
360	0\\
361	0\\
362	0\\
363	0\\
364	0\\
365	0\\
366	0\\
367	0\\
368	0\\
369	0\\
370	0\\
371	0\\
372	0\\
373	0\\
374	0\\
375	0\\
376	0\\
377	0\\
378	0\\
379	0\\
380	0\\
381	0\\
382	0\\
383	0\\
384	0\\
385	0\\
386	0\\
387	0\\
388	0\\
389	0\\
390	0\\
391	0\\
392	0\\
393	0\\
394	0\\
395	0\\
396	0\\
397	0\\
398	0\\
399	0\\
400	0\\
401	0\\
402	0\\
403	0\\
404	0\\
405	0\\
406	0\\
407	0\\
408	0\\
409	0\\
410	0\\
411	0\\
412	0\\
413	0\\
414	0\\
415	0\\
416	0\\
417	0\\
418	0\\
419	0\\
420	0\\
421	0\\
422	0\\
423	0\\
424	0\\
425	0\\
426	0\\
427	0\\
428	0\\
429	0\\
430	0\\
431	0\\
432	0\\
433	0\\
434	0\\
435	0\\
436	0\\
437	0\\
438	0\\
439	0\\
440	0\\
441	0\\
442	0\\
443	0\\
444	0\\
445	0\\
446	0\\
447	0\\
448	0\\
449	0\\
450	0\\
451	0\\
452	0\\
453	0\\
454	0\\
455	0\\
456	0\\
457	0\\
458	0\\
459	0\\
460	0\\
461	0\\
462	0\\
463	0\\
464	0\\
465	0\\
466	0\\
467	0\\
468	0\\
469	0\\
470	0\\
471	0\\
472	0\\
473	0\\
474	0\\
475	0\\
476	0\\
477	0\\
478	0\\
479	0\\
480	0\\
481	0\\
482	0\\
483	0\\
484	0\\
485	0\\
486	0\\
487	0\\
488	0\\
489	0\\
490	0\\
491	0\\
492	0\\
493	0\\
494	0\\
495	0\\
496	0\\
497	0\\
498	0\\
499	0\\
500	0\\
501	0\\
502	0\\
503	0\\
504	0\\
505	0\\
506	0\\
507	0\\
508	0\\
509	0\\
510	0\\
511	0\\
512	0\\
513	0\\
514	0\\
515	0\\
516	0\\
517	0\\
518	0\\
519	0\\
520	0\\
521	0\\
522	0\\
523	0\\
524	0\\
525	0\\
526	0\\
527	0\\
528	0\\
529	0\\
530	0\\
531	0\\
532	0\\
533	0\\
534	0\\
535	0\\
536	0\\
537	0\\
538	0\\
539	0\\
540	0\\
541	0\\
542	0\\
543	0\\
544	0\\
545	0\\
546	0\\
547	0\\
548	0\\
549	0\\
550	0\\
551	0\\
552	0\\
553	0\\
554	0\\
555	0\\
556	0\\
557	0\\
558	0\\
559	0\\
560	0\\
561	0\\
562	0\\
563	0\\
564	0\\
565	0\\
566	0\\
567	0\\
568	0\\
569	0\\
570	0\\
571	0\\
572	0\\
573	0\\
574	0\\
575	0\\
576	0\\
577	0\\
578	0\\
579	0\\
580	0\\
581	0\\
582	0\\
583	0\\
584	0\\
585	0\\
586	0\\
587	0\\
588	0\\
589	0\\
590	0\\
591	0\\
592	0\\
593	0\\
594	0\\
595	0\\
596	0\\
597	0\\
598	0\\
599	0\\
600	0\\
};
\addplot [color=mycolor7,solid,forget plot]
  table[row sep=crcr]{%
1	0\\
2	0\\
3	0\\
4	0\\
5	0\\
6	0\\
7	0\\
8	0\\
9	0\\
10	0\\
11	0\\
12	0\\
13	0\\
14	0\\
15	0\\
16	0\\
17	0\\
18	0\\
19	0\\
20	0\\
21	0\\
22	0\\
23	0\\
24	0\\
25	0\\
26	0\\
27	0\\
28	0\\
29	0\\
30	0\\
31	0\\
32	0\\
33	0\\
34	0\\
35	0\\
36	0\\
37	0\\
38	0\\
39	0\\
40	0\\
41	0\\
42	0\\
43	0\\
44	0\\
45	0\\
46	0\\
47	0\\
48	0\\
49	0\\
50	0\\
51	0\\
52	0\\
53	0\\
54	0\\
55	0\\
56	0\\
57	0\\
58	0\\
59	0\\
60	0\\
61	0\\
62	0\\
63	0\\
64	0\\
65	0\\
66	0\\
67	0\\
68	0\\
69	0\\
70	0\\
71	0\\
72	0\\
73	0\\
74	0\\
75	0\\
76	0\\
77	0\\
78	0\\
79	0\\
80	0\\
81	0\\
82	0\\
83	0\\
84	0\\
85	0\\
86	0\\
87	0\\
88	0\\
89	0\\
90	0\\
91	0\\
92	0\\
93	0\\
94	0\\
95	0\\
96	0\\
97	0\\
98	0\\
99	0\\
100	0\\
101	0\\
102	0\\
103	0\\
104	0\\
105	0\\
106	0\\
107	0\\
108	0\\
109	0\\
110	0\\
111	0\\
112	0\\
113	0\\
114	0\\
115	0\\
116	0\\
117	0\\
118	0\\
119	0\\
120	0\\
121	0\\
122	0\\
123	0\\
124	0\\
125	0\\
126	0\\
127	0\\
128	0\\
129	0\\
130	0\\
131	0\\
132	0\\
133	0\\
134	0\\
135	0\\
136	0\\
137	0\\
138	0\\
139	0\\
140	0\\
141	0\\
142	0\\
143	0\\
144	0\\
145	0\\
146	0\\
147	0\\
148	0\\
149	0\\
150	0\\
151	0\\
152	0\\
153	0\\
154	0\\
155	0\\
156	0\\
157	0\\
158	0\\
159	0\\
160	0\\
161	0\\
162	0\\
163	0\\
164	0\\
165	0\\
166	0\\
167	0\\
168	0\\
169	0\\
170	0\\
171	0\\
172	0\\
173	0\\
174	0\\
175	0\\
176	0\\
177	0\\
178	0\\
179	0\\
180	0\\
181	0\\
182	0\\
183	0\\
184	0\\
185	0\\
186	0\\
187	0\\
188	0\\
189	0\\
190	0\\
191	0\\
192	0\\
193	0\\
194	0\\
195	0\\
196	0\\
197	0\\
198	0\\
199	0\\
200	0\\
201	0\\
202	0\\
203	0\\
204	0\\
205	0\\
206	0\\
207	0\\
208	0\\
209	0\\
210	0\\
211	0\\
212	0\\
213	0\\
214	0\\
215	0\\
216	0\\
217	0\\
218	0\\
219	0\\
220	0\\
221	0\\
222	0\\
223	0\\
224	0\\
225	0\\
226	0\\
227	0\\
228	0\\
229	0\\
230	0\\
231	0\\
232	0\\
233	0\\
234	0\\
235	0\\
236	0\\
237	0\\
238	0\\
239	0\\
240	0\\
241	0\\
242	0\\
243	0\\
244	0\\
245	0\\
246	0\\
247	0\\
248	0\\
249	0\\
250	0\\
251	0\\
252	0\\
253	0\\
254	0\\
255	0\\
256	0\\
257	0\\
258	0\\
259	0\\
260	0\\
261	0\\
262	0\\
263	0\\
264	0\\
265	0\\
266	0\\
267	0\\
268	0\\
269	0\\
270	0\\
271	0\\
272	0\\
273	0\\
274	0\\
275	0\\
276	0\\
277	0\\
278	0\\
279	0\\
280	0\\
281	0\\
282	0\\
283	0\\
284	0\\
285	0\\
286	0\\
287	0\\
288	0\\
289	0\\
290	0\\
291	0\\
292	0\\
293	0\\
294	0\\
295	0\\
296	0\\
297	0\\
298	0\\
299	0\\
300	0\\
301	0\\
302	0\\
303	0\\
304	0\\
305	0\\
306	0\\
307	0\\
308	0\\
309	0\\
310	0\\
311	0\\
312	0\\
313	0\\
314	0\\
315	0\\
316	0\\
317	0\\
318	0\\
319	0\\
320	0\\
321	0\\
322	0\\
323	0\\
324	0\\
325	0\\
326	0\\
327	0\\
328	0\\
329	0\\
330	0\\
331	0\\
332	0\\
333	0\\
334	0\\
335	0\\
336	0\\
337	0\\
338	0\\
339	0\\
340	0\\
341	0\\
342	0\\
343	0\\
344	0\\
345	0\\
346	0\\
347	0\\
348	0\\
349	0\\
350	0\\
351	0\\
352	0\\
353	0\\
354	0\\
355	0\\
356	0\\
357	0\\
358	0\\
359	0\\
360	0\\
361	0\\
362	0\\
363	0\\
364	0\\
365	0\\
366	0\\
367	0\\
368	0\\
369	0\\
370	0\\
371	0\\
372	0\\
373	0\\
374	0\\
375	0\\
376	0\\
377	0\\
378	0\\
379	0\\
380	0\\
381	0\\
382	0\\
383	0\\
384	0\\
385	0\\
386	0\\
387	0\\
388	0\\
389	0\\
390	0\\
391	0\\
392	0\\
393	0\\
394	0\\
395	0\\
396	0\\
397	0\\
398	0\\
399	0\\
400	0\\
401	0\\
402	0\\
403	0\\
404	0\\
405	0\\
406	0\\
407	0\\
408	0\\
409	0\\
410	0\\
411	0\\
412	0\\
413	0\\
414	0\\
415	0\\
416	0\\
417	0\\
418	0\\
419	0\\
420	0\\
421	0\\
422	0\\
423	0\\
424	0\\
425	0\\
426	0\\
427	0\\
428	0\\
429	0\\
430	0\\
431	0\\
432	0\\
433	0\\
434	0\\
435	0\\
436	0\\
437	0\\
438	0\\
439	0\\
440	0\\
441	0\\
442	0\\
443	0\\
444	0\\
445	0\\
446	0\\
447	0\\
448	0\\
449	0\\
450	0\\
451	0\\
452	0\\
453	0\\
454	0\\
455	0\\
456	0\\
457	0\\
458	0\\
459	0\\
460	0\\
461	0\\
462	0\\
463	0\\
464	0\\
465	0\\
466	0\\
467	0\\
468	0\\
469	0\\
470	0\\
471	0\\
472	0\\
473	0\\
474	0\\
475	0\\
476	0\\
477	0\\
478	0\\
479	0\\
480	0\\
481	0\\
482	0\\
483	0\\
484	0\\
485	0\\
486	0\\
487	0\\
488	0\\
489	0\\
490	0\\
491	0\\
492	0\\
493	0\\
494	0\\
495	0\\
496	0\\
497	0\\
498	0\\
499	0\\
500	0\\
501	0\\
502	0\\
503	0\\
504	0\\
505	0\\
506	0\\
507	0\\
508	0\\
509	0\\
510	0\\
511	0\\
512	0\\
513	0\\
514	0\\
515	0\\
516	0\\
517	0\\
518	0\\
519	0\\
520	0\\
521	0\\
522	0\\
523	0\\
524	0\\
525	0\\
526	0\\
527	0\\
528	0\\
529	0\\
530	0\\
531	0\\
532	0\\
533	0\\
534	0\\
535	0\\
536	0\\
537	0\\
538	0\\
539	0\\
540	0\\
541	0\\
542	0\\
543	0\\
544	0\\
545	0\\
546	0\\
547	0\\
548	0\\
549	0\\
550	0\\
551	0\\
552	0\\
553	0\\
554	0\\
555	0\\
556	0\\
557	0\\
558	0\\
559	0\\
560	0\\
561	0\\
562	0\\
563	0\\
564	0\\
565	0\\
566	0\\
567	0\\
568	0\\
569	0\\
570	0\\
571	0\\
572	0\\
573	0\\
574	0\\
575	0\\
576	0\\
577	0\\
578	0\\
579	0\\
580	0\\
581	0\\
582	0\\
583	0\\
584	0\\
585	0\\
586	0\\
587	0\\
588	0\\
589	0\\
590	0\\
591	0\\
592	0\\
593	0\\
594	0\\
595	0\\
596	0\\
597	0\\
598	0\\
599	0\\
600	0\\
};
\addplot [color=mycolor8,solid,forget plot]
  table[row sep=crcr]{%
1	0\\
2	0\\
3	0\\
4	0\\
5	0\\
6	0\\
7	0\\
8	0\\
9	0\\
10	0\\
11	0\\
12	0\\
13	0\\
14	0\\
15	0\\
16	0\\
17	0\\
18	0\\
19	0\\
20	0\\
21	0\\
22	0\\
23	0\\
24	0\\
25	0\\
26	0\\
27	0\\
28	0\\
29	0\\
30	0\\
31	0\\
32	0\\
33	0\\
34	0\\
35	0\\
36	0\\
37	0\\
38	0\\
39	0\\
40	0\\
41	0\\
42	0\\
43	0\\
44	0\\
45	0\\
46	0\\
47	0\\
48	0\\
49	0\\
50	0\\
51	0\\
52	0\\
53	0\\
54	0\\
55	0\\
56	0\\
57	0\\
58	0\\
59	0\\
60	0\\
61	0\\
62	0\\
63	0\\
64	0\\
65	0\\
66	0\\
67	0\\
68	0\\
69	0\\
70	0\\
71	0\\
72	0\\
73	0\\
74	0\\
75	0\\
76	0\\
77	0\\
78	0\\
79	0\\
80	0\\
81	0\\
82	0\\
83	0\\
84	0\\
85	0\\
86	0\\
87	0\\
88	0\\
89	0\\
90	0\\
91	0\\
92	0\\
93	0\\
94	0\\
95	0\\
96	0\\
97	0\\
98	0\\
99	0\\
100	0\\
101	0\\
102	0\\
103	0\\
104	0\\
105	0\\
106	0\\
107	0\\
108	0\\
109	0\\
110	0\\
111	0\\
112	0\\
113	0\\
114	0\\
115	0\\
116	0\\
117	0\\
118	0\\
119	0\\
120	0\\
121	0\\
122	0\\
123	0\\
124	0\\
125	0\\
126	0\\
127	0\\
128	0\\
129	0\\
130	0\\
131	0\\
132	0\\
133	0\\
134	0\\
135	0\\
136	0\\
137	0\\
138	0\\
139	0\\
140	0\\
141	0\\
142	0\\
143	0\\
144	0\\
145	0\\
146	0\\
147	0\\
148	0\\
149	0\\
150	0\\
151	0\\
152	0\\
153	0\\
154	0\\
155	0\\
156	0\\
157	0\\
158	0\\
159	0\\
160	0\\
161	0\\
162	0\\
163	0\\
164	0\\
165	0\\
166	0\\
167	0\\
168	0\\
169	0\\
170	0\\
171	0\\
172	0\\
173	0\\
174	0\\
175	0\\
176	0\\
177	0\\
178	0\\
179	0\\
180	0\\
181	0\\
182	0\\
183	0\\
184	0\\
185	0\\
186	0\\
187	0\\
188	0\\
189	0\\
190	0\\
191	0\\
192	0\\
193	0\\
194	0\\
195	0\\
196	0\\
197	0\\
198	0\\
199	0\\
200	0\\
201	0\\
202	0\\
203	0\\
204	0\\
205	0\\
206	0\\
207	0\\
208	0\\
209	0\\
210	0\\
211	0\\
212	0\\
213	0\\
214	0\\
215	0\\
216	0\\
217	0\\
218	0\\
219	0\\
220	0\\
221	0\\
222	0\\
223	0\\
224	0\\
225	0\\
226	0\\
227	0\\
228	0\\
229	0\\
230	0\\
231	0\\
232	0\\
233	0\\
234	0\\
235	0\\
236	0\\
237	0\\
238	0\\
239	0\\
240	0\\
241	0\\
242	0\\
243	0\\
244	0\\
245	0\\
246	0\\
247	0\\
248	0\\
249	0\\
250	0\\
251	0\\
252	0\\
253	0\\
254	0\\
255	0\\
256	0\\
257	0\\
258	0\\
259	0\\
260	0\\
261	0\\
262	0\\
263	0\\
264	0\\
265	0\\
266	0\\
267	0\\
268	0\\
269	0\\
270	0\\
271	0\\
272	0\\
273	0\\
274	0\\
275	0\\
276	0\\
277	0\\
278	0\\
279	0\\
280	0\\
281	0\\
282	0\\
283	0\\
284	0\\
285	0\\
286	0\\
287	0\\
288	0\\
289	0\\
290	0\\
291	0\\
292	0\\
293	0\\
294	0\\
295	0\\
296	0\\
297	0\\
298	0\\
299	0\\
300	0\\
301	0\\
302	0\\
303	0\\
304	0\\
305	0\\
306	0\\
307	0\\
308	0\\
309	0\\
310	0\\
311	0\\
312	0\\
313	0\\
314	0\\
315	0\\
316	0\\
317	0\\
318	0\\
319	0\\
320	0\\
321	0\\
322	0\\
323	0\\
324	0\\
325	0\\
326	0\\
327	0\\
328	0\\
329	0\\
330	0\\
331	0\\
332	0\\
333	0\\
334	0\\
335	0\\
336	0\\
337	0\\
338	0\\
339	0\\
340	0\\
341	0\\
342	0\\
343	0\\
344	0\\
345	0\\
346	0\\
347	0\\
348	0\\
349	0\\
350	0\\
351	0\\
352	0\\
353	0\\
354	0\\
355	0\\
356	0\\
357	0\\
358	0\\
359	0\\
360	0\\
361	0\\
362	0\\
363	0\\
364	0\\
365	0\\
366	0\\
367	0\\
368	0\\
369	0\\
370	0\\
371	0\\
372	0\\
373	0\\
374	0\\
375	0\\
376	0\\
377	0\\
378	0\\
379	0\\
380	0\\
381	0\\
382	0\\
383	0\\
384	0\\
385	0\\
386	0\\
387	0\\
388	0\\
389	0\\
390	0\\
391	0\\
392	0\\
393	0\\
394	0\\
395	0\\
396	0\\
397	0\\
398	0\\
399	0\\
400	0\\
401	0\\
402	0\\
403	0\\
404	0\\
405	0\\
406	0\\
407	0\\
408	0\\
409	0\\
410	0\\
411	0\\
412	0\\
413	0\\
414	0\\
415	0\\
416	0\\
417	0\\
418	0\\
419	0\\
420	0\\
421	0\\
422	0\\
423	0\\
424	0\\
425	0\\
426	0\\
427	0\\
428	0\\
429	0\\
430	0\\
431	0\\
432	0\\
433	0\\
434	0\\
435	0\\
436	0\\
437	0\\
438	0\\
439	0\\
440	0\\
441	0\\
442	0\\
443	0\\
444	0\\
445	0\\
446	0\\
447	0\\
448	0\\
449	0\\
450	0\\
451	0\\
452	0\\
453	0\\
454	0\\
455	0\\
456	0\\
457	0\\
458	0\\
459	0\\
460	0\\
461	0\\
462	0\\
463	0\\
464	0\\
465	0\\
466	0\\
467	0\\
468	0\\
469	0\\
470	0\\
471	0\\
472	0\\
473	0\\
474	0\\
475	0\\
476	0\\
477	0\\
478	0\\
479	0\\
480	0\\
481	0\\
482	0\\
483	0\\
484	0\\
485	0\\
486	0\\
487	0\\
488	0\\
489	0\\
490	0\\
491	0\\
492	0\\
493	0\\
494	0\\
495	0\\
496	0\\
497	0\\
498	0\\
499	0\\
500	0\\
501	0\\
502	0\\
503	0\\
504	0\\
505	0\\
506	0\\
507	0\\
508	0\\
509	0\\
510	0\\
511	0\\
512	0\\
513	0\\
514	0\\
515	0\\
516	0\\
517	0\\
518	0\\
519	0\\
520	0\\
521	0\\
522	0\\
523	0\\
524	0\\
525	0\\
526	0\\
527	0\\
528	0\\
529	0\\
530	0\\
531	0\\
532	0\\
533	0\\
534	0\\
535	0\\
536	0\\
537	0\\
538	0\\
539	0\\
540	0\\
541	0\\
542	0\\
543	0\\
544	0\\
545	0\\
546	0\\
547	0\\
548	0\\
549	0\\
550	0\\
551	0\\
552	0\\
553	0\\
554	0\\
555	0\\
556	0\\
557	0\\
558	0\\
559	0\\
560	0\\
561	0\\
562	0\\
563	0\\
564	0\\
565	0\\
566	0\\
567	0\\
568	0\\
569	0\\
570	0\\
571	0\\
572	0\\
573	0\\
574	0\\
575	0\\
576	0\\
577	0\\
578	0\\
579	0\\
580	0\\
581	0\\
582	0\\
583	0\\
584	0\\
585	0\\
586	0\\
587	0\\
588	0\\
589	0\\
590	0\\
591	0\\
592	0\\
593	0\\
594	0\\
595	0\\
596	0\\
597	0\\
598	0\\
599	0\\
600	0\\
};
\addplot [color=blue!25!mycolor7,solid,forget plot]
  table[row sep=crcr]{%
1	0\\
2	0\\
3	0\\
4	0\\
5	0\\
6	0\\
7	0\\
8	0\\
9	0\\
10	0\\
11	0\\
12	0\\
13	0\\
14	0\\
15	0\\
16	0\\
17	0\\
18	0\\
19	0\\
20	0\\
21	0\\
22	0\\
23	0\\
24	0\\
25	0\\
26	0\\
27	0\\
28	0\\
29	0\\
30	0\\
31	0\\
32	0\\
33	0\\
34	0\\
35	0\\
36	0\\
37	0\\
38	0\\
39	0\\
40	0\\
41	0\\
42	0\\
43	0\\
44	0\\
45	0\\
46	0\\
47	0\\
48	0\\
49	0\\
50	0\\
51	0\\
52	0\\
53	0\\
54	0\\
55	0\\
56	0\\
57	0\\
58	0\\
59	0\\
60	0\\
61	0\\
62	0\\
63	0\\
64	0\\
65	0\\
66	0\\
67	0\\
68	0\\
69	0\\
70	0\\
71	0\\
72	0\\
73	0\\
74	0\\
75	0\\
76	0\\
77	0\\
78	0\\
79	0\\
80	0\\
81	0\\
82	0\\
83	0\\
84	0\\
85	0\\
86	0\\
87	0\\
88	0\\
89	0\\
90	0\\
91	0\\
92	0\\
93	0\\
94	0\\
95	0\\
96	0\\
97	0\\
98	0\\
99	0\\
100	0\\
101	0\\
102	0\\
103	0\\
104	0\\
105	0\\
106	0\\
107	0\\
108	0\\
109	0\\
110	0\\
111	0\\
112	0\\
113	0\\
114	0\\
115	0\\
116	0\\
117	0\\
118	0\\
119	0\\
120	0\\
121	0\\
122	0\\
123	0\\
124	0\\
125	0\\
126	0\\
127	0\\
128	0\\
129	0\\
130	0\\
131	0\\
132	0\\
133	0\\
134	0\\
135	0\\
136	0\\
137	0\\
138	0\\
139	0\\
140	0\\
141	0\\
142	0\\
143	0\\
144	0\\
145	0\\
146	0\\
147	0\\
148	0\\
149	0\\
150	0\\
151	0\\
152	0\\
153	0\\
154	0\\
155	0\\
156	0\\
157	0\\
158	0\\
159	0\\
160	0\\
161	0\\
162	0\\
163	0\\
164	0\\
165	0\\
166	0\\
167	0\\
168	0\\
169	0\\
170	0\\
171	0\\
172	0\\
173	0\\
174	0\\
175	0\\
176	0\\
177	0\\
178	0\\
179	0\\
180	0\\
181	0\\
182	0\\
183	0\\
184	0\\
185	0\\
186	0\\
187	0\\
188	0\\
189	0\\
190	0\\
191	0\\
192	0\\
193	0\\
194	0\\
195	0\\
196	0\\
197	0\\
198	0\\
199	0\\
200	0\\
201	0\\
202	0\\
203	0\\
204	0\\
205	0\\
206	0\\
207	0\\
208	0\\
209	0\\
210	0\\
211	0\\
212	0\\
213	0\\
214	0\\
215	0\\
216	0\\
217	0\\
218	0\\
219	0\\
220	0\\
221	0\\
222	0\\
223	0\\
224	0\\
225	0\\
226	0\\
227	0\\
228	0\\
229	0\\
230	0\\
231	0\\
232	0\\
233	0\\
234	0\\
235	0\\
236	0\\
237	0\\
238	0\\
239	0\\
240	0\\
241	0\\
242	0\\
243	0\\
244	0\\
245	0\\
246	0\\
247	0\\
248	0\\
249	0\\
250	0\\
251	0\\
252	0\\
253	0\\
254	0\\
255	0\\
256	0\\
257	0\\
258	0\\
259	0\\
260	0\\
261	0\\
262	0\\
263	0\\
264	0\\
265	0\\
266	0\\
267	0\\
268	0\\
269	0\\
270	0\\
271	0\\
272	0\\
273	0\\
274	0\\
275	0\\
276	0\\
277	0\\
278	0\\
279	0\\
280	0\\
281	0\\
282	0\\
283	0\\
284	0\\
285	0\\
286	0\\
287	0\\
288	0\\
289	0\\
290	0\\
291	0\\
292	0\\
293	0\\
294	0\\
295	0\\
296	0\\
297	0\\
298	0\\
299	0\\
300	0\\
301	0\\
302	0\\
303	0\\
304	0\\
305	0\\
306	0\\
307	0\\
308	0\\
309	0\\
310	0\\
311	0\\
312	0\\
313	0\\
314	0\\
315	0\\
316	0\\
317	0\\
318	0\\
319	0\\
320	0\\
321	0\\
322	0\\
323	0\\
324	0\\
325	0\\
326	0\\
327	0\\
328	0\\
329	0\\
330	0\\
331	0\\
332	0\\
333	0\\
334	0\\
335	0\\
336	0\\
337	0\\
338	0\\
339	0\\
340	0\\
341	0\\
342	0\\
343	0\\
344	0\\
345	0\\
346	0\\
347	0\\
348	0\\
349	0\\
350	0\\
351	0\\
352	0\\
353	0\\
354	0\\
355	0\\
356	0\\
357	0\\
358	0\\
359	0\\
360	0\\
361	0\\
362	0\\
363	0\\
364	0\\
365	0\\
366	0\\
367	0\\
368	0\\
369	0\\
370	0\\
371	0\\
372	0\\
373	0\\
374	0\\
375	0\\
376	0\\
377	0\\
378	0\\
379	0\\
380	0\\
381	0\\
382	0\\
383	0\\
384	0\\
385	0\\
386	0\\
387	0\\
388	0\\
389	0\\
390	0\\
391	0\\
392	0\\
393	0\\
394	0\\
395	0\\
396	0\\
397	0\\
398	0\\
399	0\\
400	0\\
401	0\\
402	0\\
403	0\\
404	0\\
405	0\\
406	0\\
407	0\\
408	0\\
409	0\\
410	0\\
411	0\\
412	0\\
413	0\\
414	0\\
415	0\\
416	0\\
417	0\\
418	0\\
419	0\\
420	0\\
421	0\\
422	0\\
423	0\\
424	0\\
425	0\\
426	0\\
427	0\\
428	0\\
429	0\\
430	0\\
431	0\\
432	0\\
433	0\\
434	0\\
435	0\\
436	0\\
437	0\\
438	0\\
439	0\\
440	0\\
441	0\\
442	0\\
443	0\\
444	0\\
445	0\\
446	0\\
447	0\\
448	0\\
449	0\\
450	0\\
451	0\\
452	0\\
453	0\\
454	0\\
455	0\\
456	0\\
457	0\\
458	0\\
459	0\\
460	0\\
461	0\\
462	0\\
463	0\\
464	0\\
465	0\\
466	0\\
467	0\\
468	0\\
469	0\\
470	0\\
471	0\\
472	0\\
473	0\\
474	0\\
475	0\\
476	0\\
477	0\\
478	0\\
479	0\\
480	0\\
481	0\\
482	0\\
483	0\\
484	0\\
485	0\\
486	0\\
487	0\\
488	0\\
489	0\\
490	0\\
491	0\\
492	0\\
493	0\\
494	0\\
495	0\\
496	0\\
497	0\\
498	0\\
499	0\\
500	0\\
501	0\\
502	0\\
503	0\\
504	0\\
505	0\\
506	0\\
507	0\\
508	0\\
509	0\\
510	0\\
511	0\\
512	0\\
513	0\\
514	0\\
515	0\\
516	0\\
517	0\\
518	0\\
519	0\\
520	0\\
521	0\\
522	0\\
523	0\\
524	0\\
525	0\\
526	0\\
527	0\\
528	0\\
529	0\\
530	0\\
531	0\\
532	0\\
533	0\\
534	0\\
535	0\\
536	0\\
537	0\\
538	0\\
539	0\\
540	0\\
541	0\\
542	0\\
543	0\\
544	0\\
545	0\\
546	0\\
547	0\\
548	0\\
549	0\\
550	0\\
551	0\\
552	0\\
553	0\\
554	0\\
555	0\\
556	0\\
557	0\\
558	0\\
559	0\\
560	0\\
561	0\\
562	0\\
563	0\\
564	0\\
565	0\\
566	0\\
567	0\\
568	0\\
569	0\\
570	0\\
571	0\\
572	0\\
573	0\\
574	0\\
575	0\\
576	0\\
577	0\\
578	0\\
579	0\\
580	0\\
581	0\\
582	0\\
583	0\\
584	0\\
585	0\\
586	0\\
587	0\\
588	0\\
589	0\\
590	0\\
591	0\\
592	0\\
593	0\\
594	0\\
595	0\\
596	0\\
597	0\\
598	0\\
599	0\\
600	0\\
};
\addplot [color=mycolor9,solid,forget plot]
  table[row sep=crcr]{%
1	0\\
2	0\\
3	0\\
4	0\\
5	0\\
6	0\\
7	0\\
8	0\\
9	0\\
10	0\\
11	0\\
12	0\\
13	0\\
14	0\\
15	0\\
16	0\\
17	0\\
18	0\\
19	0\\
20	0\\
21	0\\
22	0\\
23	0\\
24	0\\
25	0\\
26	0\\
27	0\\
28	0\\
29	0\\
30	0\\
31	0\\
32	0\\
33	0\\
34	0\\
35	0\\
36	0\\
37	0\\
38	0\\
39	0\\
40	0\\
41	0\\
42	0\\
43	0\\
44	0\\
45	0\\
46	0\\
47	0\\
48	0\\
49	0\\
50	0\\
51	0\\
52	0\\
53	0\\
54	0\\
55	0\\
56	0\\
57	0\\
58	0\\
59	0\\
60	0\\
61	0\\
62	0\\
63	0\\
64	0\\
65	0\\
66	0\\
67	0\\
68	0\\
69	0\\
70	0\\
71	0\\
72	0\\
73	0\\
74	0\\
75	0\\
76	0\\
77	0\\
78	0\\
79	0\\
80	0\\
81	0\\
82	0\\
83	0\\
84	0\\
85	0\\
86	0\\
87	0\\
88	0\\
89	0\\
90	0\\
91	0\\
92	0\\
93	0\\
94	0\\
95	0\\
96	0\\
97	0\\
98	0\\
99	0\\
100	0\\
101	0\\
102	0\\
103	0\\
104	0\\
105	0\\
106	0\\
107	0\\
108	0\\
109	0\\
110	0\\
111	0\\
112	0\\
113	0\\
114	0\\
115	0\\
116	0\\
117	0\\
118	0\\
119	0\\
120	0\\
121	0\\
122	0\\
123	0\\
124	0\\
125	0\\
126	0\\
127	0\\
128	0\\
129	0\\
130	0\\
131	0\\
132	0\\
133	0\\
134	0\\
135	0\\
136	0\\
137	0\\
138	0\\
139	0\\
140	0\\
141	0\\
142	0\\
143	0\\
144	0\\
145	0\\
146	0\\
147	0\\
148	0\\
149	0\\
150	0\\
151	0\\
152	0\\
153	0\\
154	0\\
155	0\\
156	0\\
157	0\\
158	0\\
159	0\\
160	0\\
161	0\\
162	0\\
163	0\\
164	0\\
165	0\\
166	0\\
167	0\\
168	0\\
169	0\\
170	0\\
171	0\\
172	0\\
173	0\\
174	0\\
175	0\\
176	0\\
177	0\\
178	0\\
179	0\\
180	0\\
181	0\\
182	0\\
183	0\\
184	0\\
185	0\\
186	0\\
187	0\\
188	0\\
189	0\\
190	0\\
191	0\\
192	0\\
193	0\\
194	0\\
195	0\\
196	0\\
197	0\\
198	0\\
199	0\\
200	0\\
201	0\\
202	0\\
203	0\\
204	0\\
205	0\\
206	0\\
207	0\\
208	0\\
209	0\\
210	0\\
211	0\\
212	0\\
213	0\\
214	0\\
215	0\\
216	0\\
217	0\\
218	0\\
219	0\\
220	0\\
221	0\\
222	0\\
223	0\\
224	0\\
225	0\\
226	0\\
227	0\\
228	0\\
229	0\\
230	0\\
231	0\\
232	0\\
233	0\\
234	0\\
235	0\\
236	0\\
237	0\\
238	0\\
239	0\\
240	0\\
241	0\\
242	0\\
243	0\\
244	0\\
245	0\\
246	0\\
247	0\\
248	0\\
249	0\\
250	0\\
251	0\\
252	0\\
253	0\\
254	0\\
255	0\\
256	0\\
257	0\\
258	0\\
259	0\\
260	0\\
261	0\\
262	0\\
263	0\\
264	0\\
265	0\\
266	0\\
267	0\\
268	0\\
269	0\\
270	0\\
271	0\\
272	0\\
273	0\\
274	0\\
275	0\\
276	0\\
277	0\\
278	0\\
279	0\\
280	0\\
281	0\\
282	0\\
283	0\\
284	0\\
285	0\\
286	0\\
287	0\\
288	0\\
289	0\\
290	0\\
291	0\\
292	0\\
293	0\\
294	0\\
295	0\\
296	0\\
297	0\\
298	0\\
299	0\\
300	0\\
301	0\\
302	0\\
303	0\\
304	0\\
305	0\\
306	0\\
307	0\\
308	0\\
309	0\\
310	0\\
311	0\\
312	0\\
313	0\\
314	0\\
315	0\\
316	0\\
317	0\\
318	0\\
319	0\\
320	0\\
321	0\\
322	0\\
323	0\\
324	0\\
325	0\\
326	0\\
327	0\\
328	0\\
329	0\\
330	0\\
331	0\\
332	0\\
333	0\\
334	0\\
335	0\\
336	0\\
337	0\\
338	0\\
339	0\\
340	0\\
341	0\\
342	0\\
343	0\\
344	0\\
345	0\\
346	0\\
347	0\\
348	0\\
349	0\\
350	0\\
351	0\\
352	0\\
353	0\\
354	0\\
355	0\\
356	0\\
357	0\\
358	0\\
359	0\\
360	0\\
361	0\\
362	0\\
363	0\\
364	0\\
365	0\\
366	0\\
367	0\\
368	0\\
369	0\\
370	0\\
371	0\\
372	0\\
373	0\\
374	0\\
375	0\\
376	0\\
377	0\\
378	0\\
379	0\\
380	0\\
381	0\\
382	0\\
383	0\\
384	0\\
385	0\\
386	0\\
387	0\\
388	0\\
389	0\\
390	0\\
391	0\\
392	0\\
393	0\\
394	0\\
395	0\\
396	0\\
397	0\\
398	0\\
399	0\\
400	0\\
401	0\\
402	0\\
403	0\\
404	0\\
405	0\\
406	0\\
407	0\\
408	0\\
409	0\\
410	0\\
411	0\\
412	0\\
413	0\\
414	0\\
415	0\\
416	0\\
417	0\\
418	0\\
419	0\\
420	0\\
421	0\\
422	0\\
423	0\\
424	0\\
425	0\\
426	0\\
427	0\\
428	0\\
429	0\\
430	0\\
431	0\\
432	0\\
433	0\\
434	0\\
435	0\\
436	0\\
437	0\\
438	0\\
439	0\\
440	0\\
441	0\\
442	0\\
443	0\\
444	0\\
445	0\\
446	0\\
447	0\\
448	0\\
449	0\\
450	0\\
451	0\\
452	0\\
453	0\\
454	0\\
455	0\\
456	0\\
457	0\\
458	0\\
459	0\\
460	0\\
461	0\\
462	0\\
463	0\\
464	0\\
465	0\\
466	0\\
467	0\\
468	0\\
469	0\\
470	0\\
471	0\\
472	0\\
473	0\\
474	0\\
475	0\\
476	0\\
477	0\\
478	0\\
479	0\\
480	0\\
481	0\\
482	0\\
483	0\\
484	0\\
485	0\\
486	0\\
487	0\\
488	0\\
489	0\\
490	0\\
491	0\\
492	0\\
493	0\\
494	0\\
495	0\\
496	0\\
497	0\\
498	0\\
499	0\\
500	0\\
501	0\\
502	0\\
503	0\\
504	0\\
505	0\\
506	0\\
507	0\\
508	0\\
509	0\\
510	0\\
511	0\\
512	0\\
513	0\\
514	0\\
515	0\\
516	0\\
517	0\\
518	0\\
519	0\\
520	0\\
521	0\\
522	0\\
523	0\\
524	0\\
525	0\\
526	0\\
527	0\\
528	0\\
529	0\\
530	0\\
531	0\\
532	0\\
533	0\\
534	0\\
535	0\\
536	0\\
537	0\\
538	0\\
539	0\\
540	0\\
541	0\\
542	0\\
543	0\\
544	0\\
545	0\\
546	0\\
547	0\\
548	0\\
549	0\\
550	0\\
551	0\\
552	0\\
553	0\\
554	0\\
555	0\\
556	0\\
557	0\\
558	0\\
559	0\\
560	0\\
561	0\\
562	0\\
563	0\\
564	0\\
565	0\\
566	0\\
567	0\\
568	0\\
569	0\\
570	0\\
571	0\\
572	0\\
573	0\\
574	0\\
575	0\\
576	0\\
577	0\\
578	0\\
579	0\\
580	0\\
581	0\\
582	0\\
583	0\\
584	0\\
585	0\\
586	0\\
587	0\\
588	0\\
589	0\\
590	0\\
591	0\\
592	0\\
593	0\\
594	0\\
595	0\\
596	0\\
597	0\\
598	0\\
599	0\\
600	0\\
};
\addplot [color=blue!50!mycolor7,solid,forget plot]
  table[row sep=crcr]{%
1	0\\
2	0\\
3	0\\
4	0\\
5	0\\
6	0\\
7	0\\
8	0\\
9	0\\
10	0\\
11	0\\
12	0\\
13	0\\
14	0\\
15	0\\
16	0\\
17	0\\
18	0\\
19	0\\
20	0\\
21	0\\
22	0\\
23	0\\
24	0\\
25	0\\
26	0\\
27	0\\
28	0\\
29	0\\
30	0\\
31	0\\
32	0\\
33	0\\
34	0\\
35	0\\
36	0\\
37	0\\
38	0\\
39	0\\
40	0\\
41	0\\
42	0\\
43	0\\
44	0\\
45	0\\
46	0\\
47	0\\
48	0\\
49	0\\
50	0\\
51	0\\
52	0\\
53	0\\
54	0\\
55	0\\
56	0\\
57	0\\
58	0\\
59	0\\
60	0\\
61	0\\
62	0\\
63	0\\
64	0\\
65	0\\
66	0\\
67	0\\
68	0\\
69	0\\
70	0\\
71	0\\
72	0\\
73	0\\
74	0\\
75	0\\
76	0\\
77	0\\
78	0\\
79	0\\
80	0\\
81	0\\
82	0\\
83	0\\
84	0\\
85	0\\
86	0\\
87	0\\
88	0\\
89	0\\
90	0\\
91	0\\
92	0\\
93	0\\
94	0\\
95	0\\
96	0\\
97	0\\
98	0\\
99	0\\
100	0\\
101	0\\
102	0\\
103	0\\
104	0\\
105	0\\
106	0\\
107	0\\
108	0\\
109	0\\
110	0\\
111	0\\
112	0\\
113	0\\
114	0\\
115	0\\
116	0\\
117	0\\
118	0\\
119	0\\
120	0\\
121	0\\
122	0\\
123	0\\
124	0\\
125	0\\
126	0\\
127	0\\
128	0\\
129	0\\
130	0\\
131	0\\
132	0\\
133	0\\
134	0\\
135	0\\
136	0\\
137	0\\
138	0\\
139	0\\
140	0\\
141	0\\
142	0\\
143	0\\
144	0\\
145	0\\
146	0\\
147	0\\
148	0\\
149	0\\
150	0\\
151	0\\
152	0\\
153	0\\
154	0\\
155	0\\
156	0\\
157	0\\
158	0\\
159	0\\
160	0\\
161	0\\
162	0\\
163	0\\
164	0\\
165	0\\
166	0\\
167	0\\
168	0\\
169	0\\
170	0\\
171	0\\
172	0\\
173	0\\
174	0\\
175	0\\
176	0\\
177	0\\
178	0\\
179	0\\
180	0\\
181	0\\
182	0\\
183	0\\
184	0\\
185	0\\
186	0\\
187	0\\
188	0\\
189	0\\
190	0\\
191	0\\
192	0\\
193	0\\
194	0\\
195	0\\
196	0\\
197	0\\
198	0\\
199	0\\
200	0\\
201	0\\
202	0\\
203	0\\
204	0\\
205	0\\
206	0\\
207	0\\
208	0\\
209	0\\
210	0\\
211	0\\
212	0\\
213	0\\
214	0\\
215	0\\
216	0\\
217	0\\
218	0\\
219	0\\
220	0\\
221	0\\
222	0\\
223	0\\
224	0\\
225	0\\
226	0\\
227	0\\
228	0\\
229	0\\
230	0\\
231	0\\
232	0\\
233	0\\
234	0\\
235	0\\
236	0\\
237	0\\
238	0\\
239	0\\
240	0\\
241	0\\
242	0\\
243	0\\
244	0\\
245	0\\
246	0\\
247	0\\
248	0\\
249	0\\
250	0\\
251	0\\
252	0\\
253	0\\
254	0\\
255	0\\
256	0\\
257	0\\
258	0\\
259	0\\
260	0\\
261	0\\
262	0\\
263	0\\
264	0\\
265	0\\
266	0\\
267	0\\
268	0\\
269	0\\
270	0\\
271	0\\
272	0\\
273	0\\
274	0\\
275	0\\
276	0\\
277	0\\
278	0\\
279	0\\
280	0\\
281	0\\
282	0\\
283	0\\
284	0\\
285	0\\
286	0\\
287	0\\
288	0\\
289	0\\
290	0\\
291	0\\
292	0\\
293	0\\
294	0\\
295	0\\
296	0\\
297	0\\
298	0\\
299	0\\
300	0\\
301	0\\
302	0\\
303	0\\
304	0\\
305	0\\
306	0\\
307	0\\
308	0\\
309	0\\
310	0\\
311	0\\
312	0\\
313	0\\
314	0\\
315	0\\
316	0\\
317	0\\
318	0\\
319	0\\
320	0\\
321	0\\
322	0\\
323	0\\
324	0\\
325	0\\
326	0\\
327	0\\
328	0\\
329	0\\
330	0\\
331	0\\
332	0\\
333	0\\
334	0\\
335	0\\
336	0\\
337	0\\
338	0\\
339	0\\
340	0\\
341	0\\
342	0\\
343	0\\
344	0\\
345	0\\
346	0\\
347	0\\
348	0\\
349	0\\
350	0\\
351	0\\
352	0\\
353	0\\
354	0\\
355	0\\
356	0\\
357	0\\
358	0\\
359	0\\
360	0\\
361	0\\
362	0\\
363	0\\
364	0\\
365	0\\
366	0\\
367	0\\
368	0\\
369	0\\
370	0\\
371	0\\
372	0\\
373	0\\
374	0\\
375	0\\
376	0\\
377	0\\
378	0\\
379	0\\
380	0\\
381	0\\
382	0\\
383	0\\
384	0\\
385	0\\
386	0\\
387	0\\
388	0\\
389	0\\
390	0\\
391	0\\
392	0\\
393	0\\
394	0\\
395	0\\
396	0\\
397	0\\
398	0\\
399	0\\
400	0\\
401	0\\
402	0\\
403	0\\
404	0\\
405	0\\
406	0\\
407	0\\
408	0\\
409	0\\
410	0\\
411	0\\
412	0\\
413	0\\
414	0\\
415	0\\
416	0\\
417	0\\
418	0\\
419	0\\
420	0\\
421	0\\
422	0\\
423	0\\
424	0\\
425	0\\
426	0\\
427	0\\
428	0\\
429	0\\
430	0\\
431	0\\
432	0\\
433	0\\
434	0\\
435	0\\
436	0\\
437	0\\
438	0\\
439	0\\
440	0\\
441	0\\
442	0\\
443	0\\
444	0\\
445	0\\
446	0\\
447	0\\
448	0\\
449	0\\
450	0\\
451	0\\
452	0\\
453	0\\
454	0\\
455	0\\
456	0\\
457	0\\
458	0\\
459	0\\
460	0\\
461	0\\
462	0\\
463	0\\
464	0\\
465	0\\
466	0\\
467	0\\
468	0\\
469	0\\
470	0\\
471	0\\
472	0\\
473	0\\
474	0\\
475	0\\
476	0\\
477	0\\
478	0\\
479	0\\
480	0\\
481	0\\
482	0\\
483	0\\
484	0\\
485	0\\
486	0\\
487	0\\
488	0\\
489	0\\
490	0\\
491	0\\
492	0\\
493	0\\
494	0\\
495	0\\
496	0\\
497	0\\
498	0\\
499	0\\
500	0\\
501	0\\
502	0\\
503	0\\
504	0\\
505	0\\
506	0\\
507	0\\
508	0\\
509	0\\
510	0\\
511	0\\
512	0\\
513	0\\
514	0\\
515	0\\
516	0\\
517	0\\
518	0\\
519	0\\
520	0\\
521	0\\
522	0\\
523	0\\
524	0\\
525	0\\
526	0\\
527	0\\
528	0\\
529	0\\
530	0\\
531	0\\
532	0\\
533	0\\
534	0\\
535	0\\
536	0\\
537	0\\
538	0\\
539	0\\
540	0\\
541	0\\
542	0\\
543	0\\
544	0\\
545	0\\
546	0\\
547	0\\
548	0\\
549	0\\
550	0\\
551	0\\
552	0\\
553	0\\
554	0\\
555	0\\
556	0\\
557	0\\
558	0\\
559	0\\
560	0\\
561	0\\
562	0\\
563	0\\
564	0\\
565	0\\
566	0\\
567	0\\
568	0\\
569	0\\
570	0\\
571	0\\
572	0\\
573	0\\
574	0\\
575	0\\
576	0\\
577	0\\
578	0\\
579	0\\
580	0\\
581	0\\
582	0\\
583	0\\
584	0\\
585	0\\
586	0\\
587	0\\
588	0\\
589	0\\
590	0\\
591	0\\
592	0\\
593	0\\
594	0\\
595	0\\
596	0\\
597	0\\
598	0\\
599	0\\
600	0\\
};
\addplot [color=blue!40!mycolor9,solid,forget plot]
  table[row sep=crcr]{%
1	0.000572289361079503\\
2	0.000572279828444312\\
3	0.000572270135439682\\
4	0.000572260279371882\\
5	0.000572250257501964\\
6	0.000572240067045005\\
7	0.000572229705169346\\
8	0.00057221916899576\\
9	0.000572208455596701\\
10	0.000572197561995472\\
11	0.000572186485165404\\
12	0.000572175222028997\\
13	0.000572163769457075\\
14	0.000572152124267915\\
15	0.000572140283226384\\
16	0.000572128243043005\\
17	0.000572116000373054\\
18	0.000572103551815636\\
19	0.000572090893912736\\
20	0.000572078023148238\\
21	0.000572064935946996\\
22	0.000572051628673754\\
23	0.000572038097632211\\
24	0.000572024339063951\\
25	0.000572010349147427\\
26	0.000571996123996853\\
27	0.000571981659661137\\
28	0.000571966952122788\\
29	0.00057195199729678\\
30	0.000571936791029434\\
31	0.000571921329097228\\
32	0.000571905607205632\\
33	0.000571889620987887\\
34	0.000571873366003836\\
35	0.000571856837738625\\
36	0.000571840031601462\\
37	0.000571822942924336\\
38	0.000571805566960702\\
39	0.000571787898884128\\
40	0.00057176993378699\\
41	0.000571751666679058\\
42	0.000571733092486115\\
43	0.000571714206048501\\
44	0.000571695002119696\\
45	0.000571675475364842\\
46	0.000571655620359213\\
47	0.000571635431586721\\
48	0.00057161490343834\\
49	0.000571594030210513\\
50	0.000571572806103606\\
51	0.000571551225220185\\
52	0.000571529281563428\\
53	0.000571506969035371\\
54	0.000571484281435221\\
55	0.000571461212457569\\
56	0.00057143775569063\\
57	0.000571413904614426\\
58	0.000571389652598912\\
59	0.000571364992902124\\
60	0.000571339918668244\\
61	0.000571314422925635\\
62	0.000571288498584887\\
63	0.000571262138436795\\
64	0.000571235335150263\\
65	0.000571208081270274\\
66	0.00057118036921569\\
67	0.000571152191277152\\
68	0.000571123539614814\\
69	0.00057109440625612\\
70	0.000571064783093518\\
71	0.000571034661882132\\
72	0.000571004034237361\\
73	0.000570972891632495\\
74	0.000570941225396272\\
75	0.000570909026710325\\
76	0.000570876286606628\\
77	0.000570842995964968\\
78	0.000570809145510212\\
79	0.0005707747258097\\
80	0.00057073972727042\\
81	0.000570704140136268\\
82	0.000570667954485155\\
83	0.000570631160226152\\
84	0.000570593747096516\\
85	0.000570555704658669\\
86	0.000570517022297149\\
87	0.000570477689215458\\
88	0.000570437694432896\\
89	0.000570397026781319\\
90	0.000570355674901806\\
91	0.000570313627241323\\
92	0.000570270872049269\\
93	0.00057022739737396\\
94	0.000570183191059074\\
95	0.000570138240739997\\
96	0.000570092533840138\\
97	0.000570046057567128\\
98	0.000569998798908942\\
99	0.00056995074462999\\
100	0.000569901881267093\\
101	0.000569852195125393\\
102	0.00056980167227418\\
103	0.00056975029854262\\
104	0.000569698059515415\\
105	0.000569644940528375\\
106	0.000569590926663907\\
107	0.000569536002746355\\
108	0.000569480153337341\\
109	0.000569423362730919\\
110	0.000569365614948678\\
111	0.000569306893734748\\
112	0.000569247182550685\\
113	0.000569186464570213\\
114	0.000569124722673949\\
115	0.000569061939443922\\
116	0.000568998097158043\\
117	0.000568933177784415\\
118	0.000568867162975534\\
119	0.000568800034062384\\
120	0.000568731772048393\\
121	0.000568662357603254\\
122	0.000568591771056625\\
123	0.00056851999239169\\
124	0.000568447001238563\\
125	0.000568372776867582\\
126	0.000568297298182457\\
127	0.000568220543713245\\
128	0.000568142491609186\\
129	0.000568063119631404\\
130	0.000567982405145427\\
131	0.000567900325113538\\
132	0.000567816856087025\\
133	0.000567731974198204\\
134	0.000567645655152298\\
135	0.000567557874219155\\
136	0.000567468606224778\\
137	0.000567377825542686\\
138	0.000567285506085131\\
139	0.000567191621294093\\
140	0.000567096144132122\\
141	0.000566999047073011\\
142	0.00056690030209229\\
143	0.000566799880657502\\
144	0.000566697753718357\\
145	0.000566593891696704\\
146	0.000566488264476299\\
147	0.000566380841392455\\
148	0.000566271591221453\\
149	0.000566160482169865\\
150	0.0005660474818637\\
151	0.000565932557337377\\
152	0.000565815675022591\\
153	0.000565696800737025\\
154	0.000565575899672972\\
155	0.000565452936385781\\
156	0.000565327874782326\\
157	0.00056520067810925\\
158	0.000565071308941251\\
159	0.000564939729169241\\
160	0.000564805899988544\\
161	0.000564669781887021\\
162	0.000564531334633243\\
163	0.00056439051726466\\
164	0.000564247288075846\\
165	0.000564101604606802\\
166	0.000563953423631374\\
167	0.000563802701145759\\
168	0.000563649392357172\\
169	0.000563493451672658\\
170	0.000563334832688151\\
171	0.000563173488177561\\
172	0.000563009370082275\\
173	0.000562842429500666\\
174	0.000562672616677942\\
175	0.000562499880996109\\
176	0.000562324170964073\\
177	0.000562145434207974\\
178	0.000561963617461486\\
179	0.00056177866655611\\
180	0.000561590526411449\\
181	0.000561399141025247\\
182	0.000561204453462971\\
183	0.000561006405846999\\
184	0.000560804939345069\\
185	0.000560599994157678\\
186	0.000560391509504379\\
187	0.000560179423608616\\
188	0.000559963673680764\\
189	0.000559744195899196\\
190	0.00055952092538907\\
191	0.000559293796198689\\
192	0.000559062741273323\\
193	0.000558827692426567\\
194	0.000558588580309725\\
195	0.000558345334379785\\
196	0.000558097882867651\\
197	0.000557846152747879\\
198	0.000557590069711211\\
199	0.000557329558141205\\
200	0.000557064541090551\\
201	0.000556794940256949\\
202	0.000556520675958512\\
203	0.000556241667108771\\
204	0.000555957831191096\\
205	0.000555669084232686\\
206	0.000555375340778087\\
207	0.000555076513862098\\
208	0.00055477251498228\\
209	0.000554463254070845\\
210	0.000554148639466027\\
211	0.000553828577882921\\
212	0.00055350297438373\\
213	0.000553171732347426\\
214	0.000552834753438846\\
215	0.000552491937577186\\
216	0.000552143182903838\\
217	0.000551788385749663\\
218	0.000551427440601567\\
219	0.000551060240068446\\
220	0.000550686674846437\\
221	0.000550306633683534\\
222	0.000549920003343464\\
223	0.000549526668568835\\
224	0.000549126512043608\\
225	0.000548719414354775\\
226	0.000548305253953295\\
227	0.000547883907114225\\
228	0.0005474552478961\\
229	0.000547019148099482\\
230	0.000546575477224634\\
231	0.000546124102428374\\
232	0.000545664888480091\\
233	0.000545197697716812\\
234	0.000544722389997394\\
235	0.000544238822655763\\
236	0.000543746850453227\\
237	0.00054324632552977\\
238	0.000542737097354379\\
239	0.000542219012674355\\
240	0.000541691915463587\\
241	0.000541155646869651\\
242	0.000540610045160008\\
243	0.000540054945666896\\
244	0.000539490180731165\\
245	0.000538915579644934\\
246	0.000538330968592991\\
247	0.000537736170593055\\
248	0.000537131005434672\\
249	0.00053651528961688\\
250	0.000535888836284555\\
251	0.000535251455163334\\
252	0.000534602952493253\\
253	0.000533943130960853\\
254	0.00053327178962992\\
255	0.000532588723870659\\
256	0.000531893725287404\\
257	0.000531186581644703\\
258	0.000530467076791844\\
259	0.000529734990585686\\
260	0.000528990098811893\\
261	0.000528232173104341\\
262	0.00052746098086278\\
263	0.000526676285168801\\
264	0.000525877844699746\\
265	0.000525065413640908\\
266	0.000524238741595681\\
267	0.000523397573493714\\
268	0.000522541649497094\\
269	0.000521670704904323\\
270	0.000520784470052275\\
271	0.000519882670215849\\
272	0.000518965025505488\\
273	0.000518031250762205\\
274	0.000517081055450443\\
275	0.000516114143548415\\
276	0.000515130213435959\\
277	0.000514128957779876\\
278	0.00051311006341665\\
279	0.000512073211232512\\
280	0.00051101807604073\\
281	0.000509944326456171\\
282	0.000508851624766938\\
283	0.000507739626803091\\
284	0.000506607981802348\\
285	0.000505456332272721\\
286	0.000504284313852017\\
287	0.000503091555164066\\
288	0.000501877677671679\\
289	0.00050064229552625\\
290	0.000499385015413872\\
291	0.000498105436397899\\
292	0.000496803149758002\\
293	0.00049547773882538\\
294	0.000494128778814315\\
295	0.000492755836649867\\
296	0.000491358470791584\\
297	0.000489936231053213\\
298	0.000488488658418356\\
299	0.000487015284851931\\
300	0.00048551563310734\\
301	0.000483989216529351\\
302	0.000482435538852573\\
303	0.000480854093995371\\
304	0.000479244365849313\\
305	0.000477605828063928\\
306	0.000475937943826749\\
307	0.000474240165638641\\
308	0.000472511935084228\\
309	0.000470752682597514\\
310	0.000468961827222439\\
311	0.000467138776368593\\
312	0.000465282925561809\\
313	0.000463393658189717\\
314	0.000461470345242214\\
315	0.000459512345046848\\
316	0.000457519002999091\\
317	0.000455489651287555\\
318	0.000453423608614047\\
319	0.000451320179908691\\
320	0.000449178656040015\\
321	0.000446998313520135\\
322	0.00044477841420512\\
323	0.000442518204990639\\
324	0.000440216917503031\\
325	0.000437873767785933\\
326	0.000435487955982651\\
327	0.000433058666014573\\
328	0.000430585065255739\\
329	0.000428066304203932\\
330	0.000425501516148543\\
331	0.000422889816835637\\
332	0.000420230304130622\\
333	0.00041752205767883\\
334	0.000414764138564654\\
335	0.000411955588969695\\
336	0.000409095431830549\\
337	0.000406182670496853\\
338	0.000403216288390314\\
339	0.000400195248665453\\
340	0.000397118493873001\\
341	0.000393984945626641\\
342	0.000390793504274319\\
343	0.000387543048574891\\
344	0.00038423243538136\\
345	0.000380860499331881\\
346	0.000377426052549601\\
347	0.000373927884352812\\
348	0.000370364760976633\\
349	0.000366735425307715\\
350	0.000363038596633317\\
351	0.000359272970406324\\
352	0.000355437218027721\\
353	0.000351529986647915\\
354	0.000347549898988649\\
355	0.000343495553186723\\
356	0.000339365522661187\\
357	0.000335158356005238\\
358	0.000330872576904045\\
359	0.000326506684079658\\
360	0.000322059151263819\\
361	0.000317528427199376\\
362	0.000312912935670473\\
363	0.000308211075561671\\
364	0.000303421220945299\\
365	0.000298541721196293\\
366	0.000293570901132558\\
367	0.000288507061178865\\
368	0.000283348477551088\\
369	0.000278093402456643\\
370	0.000272740064306439\\
371	0.000267286667932114\\
372	0.000261731394801327\\
373	0.000256072403222602\\
374	0.000250307828529804\\
375	0.000244435783234991\\
376	0.000238454357136863\\
377	0.000232361617370447\\
378	0.000226155608382433\\
379	0.000219834351815218\\
380	0.000213395846281269\\
381	0.000206838067008912\\
382	0.000200158965339931\\
383	0.000193356468059979\\
384	0.000186428476544544\\
385	0.000179372865706474\\
386	0.00017218748273451\\
387	0.000164870145606742\\
388	0.000157418641337741\\
389	0.000149830723905435\\
390	0.000142104111922943\\
391	0.00013423648565384\\
392	0.000126225483485972\\
393	0.000118068697859579\\
394	0.00010976367063733\\
395	0.000101307887735736\\
396	9.26987715723901e-05\\
397	8.39336683081326e-05\\
398	7.50098166564155e-05\\
399	6.59242533909565e-05\\
400	5.66735099699828e-05\\
401	4.72526162782195e-05\\
402	3.76517861404188e-05\\
403	2.78452662667821e-05\\
404	1.77533983572104e-05\\
405	7.11205651264132e-06\\
406	0\\
407	0\\
408	0\\
409	0\\
410	0\\
411	0\\
412	0\\
413	0\\
414	0\\
415	0\\
416	0\\
417	0\\
418	0\\
419	0\\
420	0\\
421	0\\
422	0\\
423	0\\
424	0\\
425	0\\
426	0\\
427	0\\
428	0\\
429	0\\
430	0\\
431	0\\
432	0\\
433	0\\
434	0\\
435	0\\
436	0\\
437	0\\
438	0\\
439	0\\
440	0\\
441	0\\
442	0\\
443	0\\
444	0\\
445	0\\
446	0\\
447	0\\
448	0\\
449	0\\
450	0\\
451	0\\
452	0\\
453	0\\
454	0\\
455	0\\
456	0\\
457	0\\
458	0\\
459	0\\
460	0\\
461	0\\
462	0\\
463	0\\
464	0\\
465	0\\
466	0\\
467	0\\
468	0\\
469	0\\
470	0\\
471	0\\
472	0\\
473	0\\
474	0\\
475	0\\
476	0\\
477	0\\
478	0\\
479	0\\
480	0\\
481	0\\
482	0\\
483	0\\
484	0\\
485	0\\
486	0\\
487	0\\
488	0\\
489	0\\
490	0\\
491	0\\
492	0\\
493	0\\
494	0\\
495	0\\
496	0\\
497	0\\
498	0\\
499	0\\
500	0\\
501	0\\
502	0\\
503	0\\
504	0\\
505	0\\
506	0\\
507	0\\
508	0\\
509	0\\
510	0\\
511	0\\
512	0\\
513	0\\
514	0\\
515	0\\
516	0\\
517	0\\
518	0\\
519	0\\
520	0\\
521	0\\
522	0\\
523	0\\
524	0\\
525	0\\
526	0\\
527	0\\
528	0\\
529	0\\
530	0\\
531	0\\
532	0\\
533	0\\
534	0\\
535	0\\
536	0\\
537	0\\
538	0\\
539	0\\
540	0\\
541	0\\
542	0\\
543	0\\
544	0\\
545	0\\
546	0\\
547	0\\
548	0\\
549	0\\
550	0\\
551	0\\
552	0\\
553	0\\
554	0\\
555	0\\
556	0\\
557	0\\
558	0\\
559	0\\
560	0\\
561	0\\
562	0\\
563	0\\
564	0\\
565	0\\
566	0\\
567	0\\
568	0\\
569	0\\
570	0\\
571	0\\
572	0\\
573	0\\
574	0\\
575	0\\
576	0\\
577	0\\
578	0\\
579	0\\
580	0\\
581	0\\
582	0\\
583	0\\
584	0\\
585	0\\
586	0\\
587	0\\
588	0\\
589	0\\
590	0\\
591	0\\
592	0\\
593	0\\
594	0\\
595	0\\
596	0\\
597	0\\
598	0\\
599	0\\
600	0\\
};
\addplot [color=blue!75!mycolor7,solid,forget plot]
  table[row sep=crcr]{%
1	0.00162295980517879\\
2	0.00162295046252148\\
3	0.00162294096273216\\
4	0.00162293130316988\\
5	0.00162292148114924\\
6	0.0016229114939396\\
7	0.00162290133876437\\
8	0.00162289101280022\\
9	0.00162288051317626\\
10	0.00162286983697323\\
11	0.00162285898122273\\
12	0.00162284794290632\\
13	0.00162283671895473\\
14	0.00162282530624696\\
15	0.00162281370160937\\
16	0.00162280190181487\\
17	0.00162278990358195\\
18	0.00162277770357375\\
19	0.00162276529839714\\
20	0.00162275268460179\\
21	0.00162273985867911\\
22	0.00162272681706135\\
23	0.00162271355612052\\
24	0.0016227000721674\\
25	0.00162268636145049\\
26	0.00162267242015493\\
27	0.00162265824440144\\
28	0.00162264383024521\\
29	0.00162262917367475\\
30	0.00162261427061081\\
31	0.00162259911690516\\
32	0.00162258370833943\\
33	0.00162256804062393\\
34	0.0016225521093964\\
35	0.00162253591022075\\
36	0.00162251943858585\\
37	0.00162250268990415\\
38	0.00162248565951048\\
39	0.00162246834266063\\
40	0.00162245073452999\\
41	0.00162243283021221\\
42	0.00162241462471777\\
43	0.00162239611297254\\
44	0.00162237728981633\\
45	0.00162235815000139\\
46	0.0016223386881909\\
47	0.00162231889895745\\
48	0.00162229877678144\\
49	0.00162227831604951\\
50	0.00162225751105288\\
51	0.00162223635598575\\
52	0.00162221484494356\\
53	0.00162219297192129\\
54	0.00162217073081174\\
55	0.00162214811540372\\
56	0.00162212511938027\\
57	0.00162210173631675\\
58	0.00162207795967905\\
59	0.0016220537828216\\
60	0.00162202919898548\\
61	0.00162200420129641\\
62	0.00162197878276273\\
63	0.00162195293627333\\
64	0.00162192665459562\\
65	0.00162189993037326\\
66	0.00162187275612416\\
67	0.00162184512423811\\
68	0.00162181702697464\\
69	0.00162178845646065\\
70	0.00162175940468811\\
71	0.00162172986351164\\
72	0.00162169982464612\\
73	0.00162166927966421\\
74	0.00162163821999377\\
75	0.00162160663691539\\
76	0.00162157452155972\\
77	0.00162154186490478\\
78	0.00162150865777329\\
79	0.00162147489082988\\
80	0.00162144055457829\\
81	0.00162140563935846\\
82	0.00162137013534365\\
83	0.00162133403253743\\
84	0.00162129732077062\\
85	0.00162125998969823\\
86	0.00162122202879629\\
87	0.00162118342735862\\
88	0.00162114417449357\\
89	0.00162110425912068\\
90	0.00162106366996724\\
91	0.00162102239556486\\
92	0.00162098042424589\\
93	0.00162093774413989\\
94	0.00162089434316986\\
95	0.00162085020904855\\
96	0.00162080532927466\\
97	0.00162075969112888\\
98	0.00162071328167\\
99	0.00162066608773084\\
100	0.00162061809591408\\
101	0.00162056929258815\\
102	0.0016205196638829\\
103	0.00162046919568523\\
104	0.00162041787363467\\
105	0.00162036568311883\\
106	0.00162031260926882\\
107	0.00162025863695449\\
108	0.00162020375077967\\
109	0.00162014793507727\\
110	0.0016200911739043\\
111	0.00162003345103682\\
112	0.00161997474996473\\
113	0.00161991505388653\\
114	0.00161985434570394\\
115	0.00161979260801643\\
116	0.00161972982311564\\
117	0.00161966597297978\\
118	0.00161960103926773\\
119	0.00161953500331324\\
120	0.00161946784611892\\
121	0.00161939954835011\\
122	0.00161933009032872\\
123	0.00161925945202685\\
124	0.00161918761306037\\
125	0.00161911455268241\\
126	0.00161904024977661\\
127	0.00161896468285042\\
128	0.00161888783002816\\
129	0.00161880966904399\\
130	0.00161873017723482\\
131	0.00161864933153302\\
132	0.00161856710845903\\
133	0.00161848348411394\\
134	0.00161839843417181\\
135	0.00161831193387193\\
136	0.00161822395801102\\
137	0.00161813448093525\\
138	0.00161804347653212\\
139	0.00161795091822225\\
140	0.00161785677895108\\
141	0.00161776103118042\\
142	0.00161766364687985\\
143	0.00161756459751812\\
144	0.00161746385405431\\
145	0.00161736138692892\\
146	0.00161725716605491\\
147	0.00161715116080853\\
148	0.00161704334002016\\
149	0.00161693367196489\\
150	0.00161682212435318\\
151	0.00161670866432125\\
152	0.00161659325842153\\
153	0.00161647587261285\\
154	0.00161635647225068\\
155	0.0016162350220772\\
156	0.00161611148621123\\
157	0.00161598582813821\\
158	0.00161585801069996\\
159	0.0016157279960844\\
160	0.00161559574581517\\
161	0.00161546122074116\\
162	0.00161532438102591\\
163	0.00161518518613694\\
164	0.00161504359483497\\
165	0.00161489956516297\\
166	0.00161475305443516\\
167	0.00161460401922579\\
168	0.00161445241535785\\
169	0.00161429819789152\\
170	0.00161414132111245\\
171	0.00161398173851994\\
172	0.00161381940281468\\
173	0.00161365426588638\\
174	0.00161348627880102\\
175	0.00161331539178782\\
176	0.00161314155422576\\
177	0.00161296471462976\\
178	0.00161278482063631\\
179	0.00161260181898868\\
180	0.00161241565552155\\
181	0.001612226275145\\
182	0.00161203362182788\\
183	0.00161183763858056\\
184	0.00161163826743685\\
185	0.00161143544943531\\
186	0.00161122912459969\\
187	0.00161101923191878\\
188	0.00161080570932537\\
189	0.00161058849367476\\
190	0.00161036752072246\\
191	0.00161014272510164\\
192	0.00160991404030002\\
193	0.00160968139863671\\
194	0.00160944473123896\\
195	0.00160920396801896\\
196	0.0016089590376509\\
197	0.00160870986754805\\
198	0.00160845638383983\\
199	0.00160819851134835\\
200	0.00160793617356469\\
201	0.00160766929262451\\
202	0.00160739778928338\\
203	0.00160712158289149\\
204	0.00160684059136798\\
205	0.00160655473117476\\
206	0.00160626391728973\\
207	0.00160596806317967\\
208	0.00160566708077244\\
209	0.00160536088042871\\
210	0.00160504937091316\\
211	0.00160473245936512\\
212	0.0016044100512686\\
213	0.0016040820504218\\
214	0.001603748358906\\
215	0.00160340887705384\\
216	0.00160306350341704\\
217	0.00160271213473337\\
218	0.00160235466589315\\
219	0.00160199098990493\\
220	0.00160162099786064\\
221	0.0016012445789\\
222	0.00160086162017417\\
223	0.00160047200680884\\
224	0.00160007562186647\\
225	0.00159967234630785\\
226	0.00159926205895291\\
227	0.00159884463644074\\
228	0.00159841995318881\\
229	0.00159798788135147\\
230	0.00159754829077751\\
231	0.00159710104896702\\
232	0.00159664602102728\\
233	0.00159618306962786\\
234	0.00159571205495483\\
235	0.001595232834664\\
236	0.00159474526383333\\
237	0.00159424919491431\\
238	0.00159374447768245\\
239	0.00159323095918677\\
240	0.00159270848369826\\
241	0.00159217689265735\\
242	0.00159163602462031\\
243	0.00159108571520466\\
244	0.00159052579703339\\
245	0.00158995609967814\\
246	0.00158937644960128\\
247	0.00158878667009675\\
248	0.00158818658122983\\
249	0.00158757599977561\\
250	0.00158695473915633\\
251	0.00158632260937742\\
252	0.00158567941696233\\
253	0.00158502496488597\\
254	0.00158435905250693\\
255	0.00158368147549834\\
256	0.00158299202577729\\
257	0.00158229049143296\\
258	0.00158157665665321\\
259	0.00158085030164991\\
260	0.00158011120258253\\
261	0.0015793591314805\\
262	0.00157859385616387\\
263	0.00157781514016245\\
264	0.00157702274263341\\
265	0.00157621641827724\\
266	0.00157539591725204\\
267	0.00157456098508619\\
268	0.00157371136258933\\
269	0.00157284678576154\\
270	0.00157196698570083\\
271	0.00157107168850885\\
272	0.00157016061519472\\
273	0.0015692334815771\\
274	0.00156828999818434\\
275	0.00156732987015272\\
276	0.00156635279712276\\
277	0.00156535847313365\\
278	0.00156434658651553\\
279	0.00156331681977994\\
280	0.00156226884950806\\
281	0.00156120234623699\\
282	0.00156011697434384\\
283	0.00155901239192775\\
284	0.00155788825068978\\
285	0.00155674419581046\\
286	0.00155557986582534\\
287	0.00155439489249817\\
288	0.00155318890069183\\
289	0.00155196150823705\\
290	0.00155071232579872\\
291	0.00154944095673998\\
292	0.00154814699698386\\
293	0.00154683003487266\\
294	0.00154548965102486\\
295	0.00154412541818962\\
296	0.00154273690109895\\
297	0.00154132365631739\\
298	0.00153988523208923\\
299	0.00153842116818331\\
300	0.00153693099573536\\
301	0.00153541423708787\\
302	0.00153387040562747\\
303	0.00153229900561981\\
304	0.00153069953204209\\
305	0.00152907147041296\\
306	0.00152741429662004\\
307	0.00152572747674503\\
308	0.00152401046688623\\
309	0.00152226271297875\\
310	0.00152048365061228\\
311	0.00151867270484644\\
312	0.0015168292900237\\
313	0.00151495280958008\\
314	0.0015130426558534\\
315	0.00151109820988929\\
316	0.0015091188412449\\
317	0.00150710390779045\\
318	0.00150505275550851\\
319	0.00150296471829112\\
320	0.0015008391177349\\
321	0.00149867526293396\\
322	0.00149647245027083\\
323	0.00149422996320542\\
324	0.00149194707206201\\
325	0.0014896230338143\\
326	0.00148725709186868\\
327	0.00148484847584552\\
328	0.00148239640135883\\
329	0.00147990006979397\\
330	0.0014773586680837\\
331	0.00147477136848247\\
332	0.00147213732833884\\
333	0.00146945568986628\\
334	0.00146672557991202\\
335	0.00146394610972412\\
336	0.0014611163747165\\
337	0.00145823545423205\\
338	0.00145530241130345\\
339	0.0014523162924117\\
340	0.00144927612724208\\
341	0.00144618092843732\\
342	0.00144302969134765\\
343	0.00143982139377739\\
344	0.00143655499572767\\
345	0.00143322943913474\\
346	0.00142984364760332\\
347	0.00142639652613438\\
348	0.0014228869608465\\
349	0.00141931381868992\\
350	0.00141567594715247\\
351	0.00141197217395605\\
352	0.00140820130674252\\
353	0.00140436213274755\\
354	0.00140045341846087\\
355	0.00139647390927118\\
356	0.0013924223290937\\
357	0.00138829737997833\\
358	0.001384097741696\\
359	0.00137982207130068\\
360	0.00137546900266415\\
361	0.00137103714598066\\
362	0.00136652508723811\\
363	0.0013619313876523\\
364	0.00135725458306041\\
365	0.00135249318326998\\
366	0.0013476456713591\\
367	0.00134271050292344\\
368	0.00133768610526583\\
369	0.0013325708765238\\
370	0.0013273631847304\\
371	0.0013220613668038\\
372	0.00131666372746137\\
373	0.00131116853805386\\
374	0.0013055740353159\\
375	0.00129987842002959\\
376	0.00129407985559819\\
377	0.00128817646652833\\
378	0.00128216633681967\\
379	0.00127604750826262\\
380	0.00126981797864585\\
381	0.00126347569987737\\
382	0.00125701857602496\\
383	0.00125044446128388\\
384	0.00124375115788245\\
385	0.00123693641393882\\
386	0.00122999792128435\\
387	0.00122293331327162\\
388	0.00121574016258898\\
389	0.00120841597911064\\
390	0.00120095820781001\\
391	0.00119336422679553\\
392	0.00118563134552309\\
393	0.00117775680323375\\
394	0.00116973776762585\\
395	0.00116157133362488\\
396	0.00115325452171927\\
397	0.00114478427407384\\
398	0.00113615744329277\\
399	0.0011273707596541\\
400	0.00111842073867647\\
401	0.0011093034318014\\
402	0.00110001379547988\\
403	0.00109054426602079\\
404	0.00108088232894597\\
405	0.00107101049869601\\
406	0.00106092951432877\\
407	0.00105065977031644\\
408	0.00104019699284573\\
409	0.00102953677969847\\
410	0.00101867464720291\\
411	0.00100760617591885\\
412	0.000996327407526952\\
413	0.000984835854188185\\
414	0.000973132915070548\\
415	0.000961229198796052\\
416	0.000949154471044381\\
417	0.000936968437368094\\
418	0.000924731355869571\\
419	0.000912268345570118\\
420	0.000899549957971183\\
421	0.00088657119851197\\
422	0.000873327160728552\\
423	0.000859813072293951\\
424	0.000846024350965498\\
425	0.000831956672444576\\
426	0.000817606052518265\\
427	0.000802968946306856\\
428	0.000788042368116034\\
429	0.00077282403565568\\
430	0.000757312543762736\\
431	0.000741507575906905\\
432	0.00072541017123818\\
433	0.000709023104064201\\
434	0.000692351598338899\\
435	0.000675405257300434\\
436	0.00065820468577962\\
437	0.000640806376765546\\
438	0.000623398212885161\\
439	0.000606665885673703\\
440	0.000593192831486634\\
441	0.000581579396905864\\
442	0.000569713302659357\\
443	0.000557586012442185\\
444	0.000545188168945224\\
445	0.000532509458907316\\
446	0.000519538454379433\\
447	0.000506262425739107\\
448	0.000492667121011696\\
449	0.000478736504767938\\
450	0.00046445244806613\\
451	0.000449794358060503\\
452	0.000434738730400776\\
453	0.000419258594188202\\
454	0.000403322780750657\\
455	0.000386894828016613\\
456	0.000369930972609213\\
457	0.000352375618567614\\
458	0.000334148997715952\\
459	0.000315109979488416\\
460	0.000294938204456355\\
461	0.000272749625838252\\
462	0.000245909460098321\\
463	0.000218990310996033\\
464	0.000192182069022129\\
465	0.000164840854180496\\
466	0.000136951103771785\\
467	0.000108481453583217\\
468	7.9346960826296e-05\\
469	4.92801445037634e-05\\
470	1.73859184336513e-05\\
471	0\\
472	0\\
473	0\\
474	0\\
475	0\\
476	0\\
477	0\\
478	0\\
479	0\\
480	0\\
481	0\\
482	0\\
483	0\\
484	0\\
485	0\\
486	0\\
487	0\\
488	0\\
489	0\\
490	0\\
491	0\\
492	0\\
493	0\\
494	0\\
495	0\\
496	0\\
497	0\\
498	0\\
499	0\\
500	0\\
501	0\\
502	0\\
503	0\\
504	0\\
505	0\\
506	0\\
507	0\\
508	0\\
509	0\\
510	0\\
511	0\\
512	0\\
513	0\\
514	0\\
515	0\\
516	0\\
517	0\\
518	0\\
519	0\\
520	0\\
521	0\\
522	0\\
523	0\\
524	0\\
525	0\\
526	0\\
527	0\\
528	0\\
529	0\\
530	0\\
531	0\\
532	0\\
533	0\\
534	0\\
535	0\\
536	0\\
537	0\\
538	0\\
539	0\\
540	0\\
541	0\\
542	0\\
543	0\\
544	0\\
545	0\\
546	0\\
547	0\\
548	0\\
549	0\\
550	0\\
551	0\\
552	0\\
553	0\\
554	0\\
555	0\\
556	0\\
557	0\\
558	0\\
559	0\\
560	0\\
561	0\\
562	0\\
563	0\\
564	0\\
565	0\\
566	0\\
567	0\\
568	0\\
569	0\\
570	0\\
571	0\\
572	0\\
573	0\\
574	0\\
575	0\\
576	0\\
577	0\\
578	0\\
579	0\\
580	0\\
581	0\\
582	0\\
583	0\\
584	0\\
585	0\\
586	0\\
587	0\\
588	0\\
589	0\\
590	0\\
591	0\\
592	0\\
593	0\\
594	0\\
595	0\\
596	0\\
597	0\\
598	0\\
599	0\\
600	0\\
};
\addplot [color=blue!80!mycolor9,solid,forget plot]
  table[row sep=crcr]{%
1	0.0031488158044997\\
2	0.00314881010942653\\
3	0.00314880431857838\\
4	0.00314879843034492\\
5	0.00314879244308868\\
6	0.00314878635514459\\
7	0.00314878016481948\\
8	0.00314877387039164\\
9	0.0031487674701103\\
10	0.00314876096219517\\
11	0.0031487543448359\\
12	0.00314874761619161\\
13	0.00314874077439029\\
14	0.00314873381752838\\
15	0.00314872674367014\\
16	0.00314871955084715\\
17	0.00314871223705772\\
18	0.00314870480026634\\
19	0.00314869723840309\\
20	0.00314868954936306\\
21	0.00314868173100576\\
22	0.00314867378115445\\
23	0.00314866569759563\\
24	0.00314865747807829\\
25	0.00314864912031333\\
26	0.00314864062197292\\
27	0.00314863198068978\\
28	0.00314862319405655\\
29	0.00314861425962505\\
30	0.00314860517490563\\
31	0.00314859593736642\\
32	0.00314858654443263\\
33	0.00314857699348576\\
34	0.00314856728186291\\
35	0.00314855740685594\\
36	0.00314854736571077\\
37	0.0031485371556265\\
38	0.00314852677375466\\
39	0.00314851621719838\\
40	0.0031485054830115\\
41	0.00314849456819778\\
42	0.00314848346970997\\
43	0.00314847218444898\\
44	0.00314846070926293\\
45	0.00314844904094627\\
46	0.0031484371762388\\
47	0.00314842511182474\\
48	0.00314841284433181\\
49	0.00314840037033013\\
50	0.0031483876863313\\
51	0.00314837478878737\\
52	0.00314836167408975\\
53	0.0031483483385682\\
54	0.0031483347784897\\
55	0.00314832099005738\\
56	0.00314830696940938\\
57	0.00314829271261774\\
58	0.00314827821568716\\
59	0.0031482634745539\\
60	0.00314824848508452\\
61	0.00314823324307466\\
62	0.00314821774424781\\
63	0.00314820198425396\\
64	0.0031481859586684\\
65	0.00314816966299032\\
66	0.00314815309264149\\
67	0.00314813624296486\\
68	0.00314811910922321\\
69	0.00314810168659764\\
70	0.00314808397018622\\
71	0.0031480659550024\\
72	0.00314804763597357\\
73	0.00314802900793951\\
74	0.0031480100656508\\
75	0.00314799080376724\\
76	0.00314797121685624\\
77	0.00314795129939111\\
78	0.00314793104574942\\
79	0.00314791045021127\\
80	0.00314788950695751\\
81	0.00314786821006797\\
82	0.00314784655351965\\
83	0.00314782453118485\\
84	0.00314780213682928\\
85	0.00314777936411013\\
86	0.00314775620657416\\
87	0.00314773265765558\\
88	0.00314770871067414\\
89	0.00314768435883298\\
90	0.00314765959521654\\
91	0.00314763441278841\\
92	0.00314760880438911\\
93	0.00314758276273387\\
94	0.00314755628041037\\
95	0.00314752934987637\\
96	0.00314750196345738\\
97	0.00314747411334424\\
98	0.00314744579159069\\
99	0.00314741699011081\\
100	0.00314738770067656\\
101	0.00314735791491511\\
102	0.00314732762430623\\
103	0.00314729682017962\\
104	0.00314726549371214\\
105	0.00314723363592501\\
106	0.00314720123768105\\
107	0.00314716828968168\\
108	0.00314713478246407\\
109	0.00314710070639809\\
110	0.00314706605168329\\
111	0.00314703080834579\\
112	0.00314699496623511\\
113	0.00314695851502099\\
114	0.00314692144419008\\
115	0.00314688374304268\\
116	0.0031468454006893\\
117	0.00314680640604725\\
118	0.00314676674783712\\
119	0.0031467264145793\\
120	0.00314668539459026\\
121	0.00314664367597899\\
122	0.00314660124664317\\
123	0.00314655809426546\\
124	0.0031465142063096\\
125	0.00314646957001652\\
126	0.0031464241724004\\
127	0.00314637800024457\\
128	0.00314633104009748\\
129	0.0031462832782685\\
130	0.00314623470082376\\
131	0.00314618529358183\\
132	0.0031461350421094\\
133	0.00314608393171689\\
134	0.00314603194745396\\
135	0.00314597907410506\\
136	0.00314592529618477\\
137	0.00314587059793318\\
138	0.00314581496331124\\
139	0.00314575837599591\\
140	0.00314570081937539\\
141	0.00314564227654423\\
142	0.00314558273029834\\
143	0.00314552216313004\\
144	0.0031454605572229\\
145	0.0031453978944467\\
146	0.00314533415635217\\
147	0.0031452693241657\\
148	0.00314520337878409\\
149	0.00314513630076908\\
150	0.00314506807034191\\
151	0.0031449986673778\\
152	0.00314492807140031\\
153	0.0031448562615757\\
154	0.0031447832167071\\
155	0.00314470891522876\\
156	0.00314463333520007\\
157	0.00314455645429954\\
158	0.00314447824981876\\
159	0.00314439869865611\\
160	0.00314431777731053\\
161	0.00314423546187506\\
162	0.0031441517280303\\
163	0.0031440665510378\\
164	0.00314397990573318\\
165	0.0031438917665193\\
166	0.00314380210735908\\
167	0.00314371090176827\\
168	0.00314361812280806\\
169	0.00314352374307744\\
170	0.00314342773470537\\
171	0.0031433300693428\\
172	0.0031432307181544\\
173	0.00314312965181013\\
174	0.00314302684047649\\
175	0.00314292225380755\\
176	0.00314281586093575\\
177	0.0031427076304624\\
178	0.00314259753044784\\
179	0.00314248552840148\\
180	0.00314237159127135\\
181	0.00314225568543354\\
182	0.00314213777668128\\
183	0.00314201783021375\\
184	0.00314189581062462\\
185	0.00314177168189044\\
186	0.00314164540735863\\
187	0.00314151694973542\\
188	0.00314138627107351\\
189	0.00314125333275963\\
190	0.00314111809550184\\
191	0.00314098051931684\\
192	0.00314084056351704\\
193	0.00314069818669758\\
194	0.00314055334672318\\
195	0.00314040600071485\\
196	0.00314025610503636\\
197	0.00314010361528055\\
198	0.00313994848625536\\
199	0.00313979067196957\\
200	0.00313963012561829\\
201	0.0031394667995682\\
202	0.00313930064534247\\
203	0.00313913161360541\\
204	0.00313895965414691\\
205	0.00313878471586639\\
206	0.00313860674675669\\
207	0.00313842569388749\\
208	0.00313824150338844\\
209	0.00313805412043207\\
210	0.00313786348921623\\
211	0.00313766955294633\\
212	0.00313747225381716\\
213	0.00313727153299439\\
214	0.00313706733059575\\
215	0.0031368595856718\\
216	0.00313664823618643\\
217	0.00313643321899681\\
218	0.0031362144698332\\
219	0.00313599192327818\\
220	0.0031357655127456\\
221	0.00313553517045906\\
222	0.00313530082743007\\
223	0.00313506241343567\\
224	0.00313481985699579\\
225	0.00313457308535002\\
226	0.00313432202443405\\
227	0.0031340665988556\\
228	0.00313380673186999\\
229	0.00313354234535508\\
230	0.00313327335978592\\
231	0.00313299969420882\\
232	0.00313272126621495\\
233	0.00313243799191342\\
234	0.00313214978590395\\
235	0.00313185656124885\\
236	0.00313155822944468\\
237	0.00313125470039322\\
238	0.00313094588237195\\
239	0.00313063168200402\\
240	0.00313031200422758\\
241	0.00312998675226461\\
242	0.0031296558275891\\
243	0.00312931912989473\\
244	0.00312897655706183\\
245	0.00312862800512388\\
246	0.0031282733682332\\
247	0.00312791253862621\\
248	0.00312754540658795\\
249	0.00312717186041587\\
250	0.00312679178638316\\
251	0.00312640506870121\\
252	0.00312601158948155\\
253	0.00312561122869699\\
254	0.00312520386414209\\
255	0.00312478937139297\\
256	0.00312436762376635\\
257	0.00312393849227786\\
258	0.00312350184559967\\
259	0.00312305755001726\\
260	0.0031226054693856\\
261	0.00312214546508438\\
262	0.00312167739597262\\
263	0.00312120111834239\\
264	0.00312071648587183\\
265	0.00312022334957726\\
266	0.00311972155776458\\
267	0.00311921095597983\\
268	0.00311869138695886\\
269	0.00311816269057627\\
270	0.00311762470379345\\
271	0.00311707726060576\\
272	0.00311652019198895\\
273	0.00311595332584462\\
274	0.00311537648694485\\
275	0.00311478949687602\\
276	0.00311419217398167\\
277	0.00311358433330457\\
278	0.00311296578652785\\
279	0.0031123363419153\\
280	0.00311169580425075\\
281	0.00311104397477658\\
282	0.00311038065113134\\
283	0.00310970562728657\\
284	0.00310901869348254\\
285	0.00310831963616339\\
286	0.00310760823791114\\
287	0.00310688427737895\\
288	0.00310614752922352\\
289	0.00310539776403659\\
290	0.00310463474827559\\
291	0.00310385824419343\\
292	0.00310306800976752\\
293	0.00310226379862781\\
294	0.00310144535998416\\
295	0.00310061243855285\\
296	0.0030997647744822\\
297	0.00309890210327755\\
298	0.00309802415572534\\
299	0.00309713065781651\\
300	0.00309622133066907\\
301	0.00309529589045001\\
302	0.00309435404829648\\
303	0.0030933955102362\\
304	0.00309241997710728\\
305	0.00309142714447724\\
306	0.00309041670256155\\
307	0.00308938833614131\\
308	0.00308834172448047\\
309	0.00308727654124234\\
310	0.00308619245440554\\
311	0.00308508912617928\\
312	0.00308396621291814\\
313	0.00308282336503617\\
314	0.00308166022692044\\
315	0.00308047643684404\\
316	0.00307927162687835\\
317	0.00307804542280484\\
318	0.00307679744402615\\
319	0.00307552730347649\\
320	0.00307423460753143\\
321	0.00307291895591687\\
322	0.00307157994161721\\
323	0.00307021715078276\\
324	0.00306883016263607\\
325	0.00306741854937735\\
326	0.00306598187608868\\
327	0.00306451970063701\\
328	0.0030630315735757\\
329	0.00306151703804454\\
330	0.00305997562966797\\
331	0.00305840687645137\\
332	0.0030568102986751\\
333	0.00305518540878608\\
334	0.00305353171128658\\
335	0.0030518487026199\\
336	0.00305013587105251\\
337	0.00304839269655235\\
338	0.00304661865066269\\
339	0.00304481319637114\\
340	0.00304297578797323\\
341	0.00304110587092995\\
342	0.00303920288171855\\
343	0.00303726624767596\\
344	0.00303529538683399\\
345	0.00303328970774543\\
346	0.00303124860930018\\
347	0.00302917148053046\\
348	0.00302705770040394\\
349	0.00302490663760374\\
350	0.00302271765029403\\
351	0.00302049008587002\\
352	0.00301822328069084\\
353	0.00301591655979405\\
354	0.00301356923659012\\
355	0.00301118061253527\\
356	0.00300874997678115\\
357	0.0030062766057994\\
358	0.00300375976297957\\
359	0.00300119869819815\\
360	0.00299859264735711\\
361	0.00299594083188986\\
362	0.00299324245823253\\
363	0.00299049671725858\\
364	0.00298770278367464\\
365	0.00298485981537552\\
366	0.00298196695275613\\
367	0.00297902331797829\\
368	0.00297602801419028\\
369	0.00297298012469691\\
370	0.00296987871207802\\
371	0.00296672281725328\\
372	0.00296351145849092\\
373	0.00296024363035836\\
374	0.00295691830261222\\
375	0.00295353441902534\\
376	0.00295009089614794\\
377	0.00294658662200026\\
378	0.00294302045469302\\
379	0.00293939122097195\\
380	0.00293569771468192\\
381	0.00293193869514513\\
382	0.00292811288544692\\
383	0.0029242189706215\\
384	0.00292025559572781\\
385	0.00291622136380439\\
386	0.00291211483368896\\
387	0.00290793451768638\\
388	0.00290367887906473\\
389	0.00289934632935552\\
390	0.00289493522542896\\
391	0.00289044386630764\\
392	0.00288587048967122\\
393	0.00288121326799\\
394	0.00287647030419765\\
395	0.00287163962676288\\
396	0.00286671918390948\\
397	0.00286170683650244\\
398	0.00285660034861841\\
399	0.00285139737384472\\
400	0.00284609543383834\\
401	0.00284069188478618\\
402	0.00283518387313868\\
403	0.0028295683130977\\
404	0.00282384200316826\\
405	0.00281800200718434\\
406	0.0028120454234071\\
407	0.00280596865857969\\
408	0.0027997679172769\\
409	0.00279343919198299\\
410	0.00278697825900345\\
411	0.00278038068760407\\
412	0.00277364187577775\\
413	0.00276675713162891\\
414	0.00275972180882898\\
415	0.00275253142415608\\
416	0.00274518141276298\\
417	0.0027376656857939\\
418	0.00272997422962037\\
419	0.00272209965955078\\
420	0.00271403454251588\\
421	0.00270577084354978\\
422	0.00269729985161397\\
423	0.00268861209420816\\
424	0.00267969723882085\\
425	0.00267054397888468\\
426	0.00266113990141535\\
427	0.00265147133286866\\
428	0.00264152315879101\\
429	0.00263127861120611\\
430	0.00262071901415549\\
431	0.00260982346880575\\
432	0.00259856843392222\\
433	0.00258692707942594\\
434	0.00257486804188674\\
435	0.00256235239401671\\
436	0.00254932489016489\\
437	0.00253568608287143\\
438	0.00252119866904994\\
439	0.00250516269361007\\
440	0.00248526366011957\\
441	0.00246288514601484\\
442	0.00243997915080848\\
443	0.00241652916105559\\
444	0.0023925180371453\\
445	0.00236792800599649\\
446	0.00234274065856143\\
447	0.00231693695337167\\
448	0.00229049722759709\\
449	0.00226340121732881\\
450	0.00223562808895717\\
451	0.00220715648332688\\
452	0.00217796457304159\\
453	0.00214803012875291\\
454	0.00211733057709868\\
455	0.0020858429982616\\
456	0.00205354392497408\\
457	0.00202040860346689\\
458	0.00198640900300285\\
459	0.00195150951581333\\
460	0.0019156610278376\\
461	0.00187880925939435\\
462	0.00184101668332657\\
463	0.00180306060647431\\
464	0.00176695657113213\\
465	0.00174002127743041\\
466	0.00171255806413367\\
467	0.00168454871256319\\
468	0.00165596653527007\\
469	0.0016267693496688\\
470	0.00159690408596396\\
471	0.00156638900126274\\
472	0.001535271194343\\
473	0.00150354977120314\\
474	0.00147124695810853\\
475	0.0014384333574062\\
476	0.00140525426127058\\
477	0.00137184892090271\\
478	0.00133779609282825\\
479	0.00130297800531827\\
480	0.0012673700781407\\
481	0.00123094637685959\\
482	0.00119367951199304\\
483	0.00115554057384795\\
484	0.00111649904362321\\
485	0.00107652267505202\\
486	0.00103557736814537\\
487	0.00099362708868859\\
488	0.00095063371984951\\
489	0.000906556731923931\\
490	0.000861352959695491\\
491	0.00081497635688427\\
492	0.000767377725191132\\
493	0.000718504413599808\\
494	0.0006682999778587\\
495	0.000616703772697162\\
496	0.000563650397617503\\
497	0.000509068764401486\\
498	0.000452880105573953\\
499	0.000394992927758715\\
500	0.000335289074672282\\
501	0.00027358392453668\\
502	0.000209511731993176\\
503	0.000142196249937518\\
504	6.92959835515677e-05\\
505	0\\
506	0\\
507	0\\
508	0\\
509	0\\
510	0\\
511	0\\
512	0\\
513	0\\
514	0\\
515	0\\
516	0\\
517	0\\
518	0\\
519	0\\
520	0\\
521	0\\
522	0\\
523	0\\
524	0\\
525	0\\
526	0\\
527	0\\
528	0\\
529	0\\
530	0\\
531	0\\
532	0\\
533	0\\
534	0\\
535	0\\
536	0\\
537	0\\
538	0\\
539	0\\
540	0\\
541	0\\
542	0\\
543	0\\
544	0\\
545	0\\
546	0\\
547	0\\
548	0\\
549	0\\
550	0\\
551	0\\
552	0\\
553	0\\
554	0\\
555	0\\
556	0\\
557	0\\
558	0\\
559	0\\
560	0\\
561	0\\
562	0\\
563	0\\
564	0\\
565	0\\
566	0\\
567	0\\
568	0\\
569	0\\
570	0\\
571	0\\
572	0\\
573	0\\
574	0\\
575	0\\
576	0\\
577	0\\
578	0\\
579	0\\
580	0\\
581	0\\
582	0\\
583	0\\
584	0\\
585	0\\
586	0\\
587	0\\
588	0\\
589	0\\
590	0\\
591	0\\
592	0\\
593	0\\
594	0\\
595	0\\
596	0\\
597	0\\
598	0\\
599	0\\
600	0\\
};
\addplot [color=blue,solid,forget plot]
  table[row sep=crcr]{%
1	0.00391145423208649\\
2	0.00391145359632034\\
3	0.00391145294986301\\
4	0.00391145229253474\\
5	0.00391145162415274\\
6	0.00391145094453115\\
7	0.00391145025348098\\
8	0.00391144955081003\\
9	0.00391144883632287\\
10	0.00391144810982078\\
11	0.00391144737110169\\
12	0.00391144661996008\\
13	0.00391144585618702\\
14	0.00391144507957\\
15	0.00391144428989295\\
16	0.00391144348693615\\
17	0.00391144267047614\\
18	0.00391144184028571\\
19	0.0039114409961338\\
20	0.00391144013778546\\
21	0.00391143926500173\\
22	0.00391143837753964\\
23	0.00391143747515211\\
24	0.00391143655758784\\
25	0.00391143562459133\\
26	0.00391143467590271\\
27	0.00391143371125772\\
28	0.00391143273038765\\
29	0.00391143173301919\\
30	0.00391143071887444\\
31	0.00391142968767075\\
32	0.00391142863912071\\
33	0.00391142757293201\\
34	0.00391142648880739\\
35	0.00391142538644453\\
36	0.003911424265536\\
37	0.00391142312576913\\
38	0.00391142196682593\\
39	0.00391142078838303\\
40	0.00391141959011152\\
41	0.00391141837167694\\
42	0.0039114171327391\\
43	0.00391141587295205\\
44	0.00391141459196392\\
45	0.00391141328941686\\
46	0.00391141196494692\\
47	0.00391141061818395\\
48	0.00391140924875146\\
49	0.00391140785626657\\
50	0.00391140644033982\\
51	0.00391140500057516\\
52	0.00391140353656969\\
53	0.00391140204791371\\
54	0.00391140053419044\\
55	0.003911398994976\\
56	0.00391139742983927\\
57	0.00391139583834171\\
58	0.00391139422003729\\
59	0.00391139257447231\\
60	0.00391139090118532\\
61	0.00391138919970692\\
62	0.00391138746955967\\
63	0.00391138571025791\\
64	0.00391138392130764\\
65	0.00391138210220638\\
66	0.00391138025244296\\
67	0.00391137837149746\\
68	0.00391137645884097\\
69	0.00391137451393546\\
70	0.00391137253623364\\
71	0.00391137052517878\\
72	0.00391136848020449\\
73	0.00391136640073466\\
74	0.00391136428618315\\
75	0.00391136213595374\\
76	0.00391135994943986\\
77	0.00391135772602443\\
78	0.00391135546507968\\
79	0.00391135316596697\\
80	0.00391135082803654\\
81	0.00391134845062738\\
82	0.00391134603306697\\
83	0.00391134357467112\\
84	0.00391134107474372\\
85	0.00391133853257653\\
86	0.003911335947449\\
87	0.00391133331862798\\
88	0.00391133064536757\\
89	0.00391132792690882\\
90	0.00391132516247953\\
91	0.003911322351294\\
92	0.00391131949255279\\
93	0.00391131658544247\\
94	0.00391131362913536\\
95	0.00391131062278926\\
96	0.00391130756554721\\
97	0.00391130445653723\\
98	0.00391130129487198\\
99	0.00391129807964859\\
100	0.00391129480994825\\
101	0.00391129148483605\\
102	0.00391128810336056\\
103	0.00391128466455363\\
104	0.00391128116743004\\
105	0.00391127761098719\\
106	0.00391127399420479\\
107	0.00391127031604454\\
108	0.0039112665754498\\
109	0.00391126277134525\\
110	0.00391125890263657\\
111	0.00391125496821008\\
112	0.00391125096693238\\
113	0.00391124689765002\\
114	0.00391124275918911\\
115	0.00391123855035498\\
116	0.00391123426993176\\
117	0.00391122991668204\\
118	0.00391122548934648\\
119	0.00391122098664338\\
120	0.00391121640726832\\
121	0.00391121174989372\\
122	0.00391120701316845\\
123	0.00391120219571741\\
124	0.00391119729614108\\
125	0.00391119231301512\\
126	0.0039111872448899\\
127	0.00391118209029005\\
128	0.00391117684771408\\
129	0.00391117151563381\\
130	0.003911166092494\\
131	0.0039111605767118\\
132	0.00391115496667634\\
133	0.00391114926074819\\
134	0.00391114345725893\\
135	0.00391113755451056\\
136	0.00391113155077509\\
137	0.00391112544429398\\
138	0.0039111192332776\\
139	0.00391111291590479\\
140	0.00391110649032221\\
141	0.0039110999546439\\
142	0.00391109330695071\\
143	0.0039110865452897\\
144	0.00391107966767366\\
145	0.00391107267208049\\
146	0.00391106555645263\\
147	0.00391105831869653\\
148	0.00391105095668199\\
149	0.00391104346824163\\
150	0.00391103585117025\\
151	0.00391102810322426\\
152	0.00391102022212101\\
153	0.00391101220553822\\
154	0.00391100405111328\\
155	0.00391099575644269\\
156	0.00391098731908134\\
157	0.00391097873654187\\
158	0.003910970006294\\
159	0.00391096112576384\\
160	0.00391095209233318\\
161	0.00391094290333881\\
162	0.00391093355607176\\
163	0.00391092404777658\\
164	0.0039109143756506\\
165	0.00391090453684312\\
166	0.00391089452845464\\
167	0.00391088434753605\\
168	0.00391087399108782\\
169	0.0039108634560591\\
170	0.00391085273934689\\
171	0.00391084183779514\\
172	0.00391083074819378\\
173	0.00391081946727783\\
174	0.0039108079917264\\
175	0.00391079631816164\\
176	0.0039107844431478\\
177	0.00391077236319006\\
178	0.00391076007473353\\
179	0.00391074757416205\\
180	0.00391073485779707\\
181	0.00391072192189647\\
182	0.00391070876265335\\
183	0.00391069537619475\\
184	0.00391068175858045\\
185	0.00391066790580162\\
186	0.00391065381377956\\
187	0.00391063947836435\\
188	0.00391062489533349\\
189	0.00391061006039058\\
190	0.00391059496916392\\
191	0.0039105796172051\\
192	0.00391056399998768\\
193	0.00391054811290567\\
194	0.00391053195127217\\
195	0.0039105155103179\\
196	0.00391049878518971\\
197	0.00391048177094907\\
198	0.00391046446257056\\
199	0.00391044685494027\\
200	0.00391042894285424\\
201	0.00391041072101682\\
202	0.00391039218403905\\
203	0.00391037332643693\\
204	0.00391035414262974\\
205	0.00391033462693829\\
206	0.00391031477358314\\
207	0.0039102945766828\\
208	0.00391027403025186\\
209	0.00391025312819916\\
210	0.00391023186432583\\
211	0.00391021023232338\\
212	0.00391018822577169\\
213	0.003910165838137\\
214	0.00391014306276988\\
215	0.00391011989290308\\
216	0.00391009632164944\\
217	0.00391007234199972\\
218	0.00391004794682034\\
219	0.00391002312885119\\
220	0.00390999788070329\\
221	0.00390997219485646\\
222	0.00390994606365694\\
223	0.003909919479315\\
224	0.0039098924339024\\
225	0.00390986491934995\\
226	0.00390983692744492\\
227	0.00390980844982839\\
228	0.00390977947799267\\
229	0.00390975000327855\\
230	0.00390972001687256\\
231	0.00390968950980415\\
232	0.00390965847294288\\
233	0.00390962689699544\\
234	0.00390959477250276\\
235	0.00390956208983697\\
236	0.00390952883919834\\
237	0.00390949501061212\\
238	0.00390946059392543\\
239	0.00390942557880395\\
240	0.00390938995472869\\
241	0.00390935371099259\\
242	0.00390931683669712\\
243	0.00390927932074883\\
244	0.0039092411518558\\
245	0.00390920231852401\\
246	0.00390916280905373\\
247	0.00390912261153578\\
248	0.0039090817138477\\
249	0.00390904010364997\\
250	0.00390899776838202\\
251	0.00390895469525827\\
252	0.00390891087126407\\
253	0.00390886628315158\\
254	0.00390882091743554\\
255	0.00390877476038906\\
256	0.00390872779803919\\
257	0.00390868001616263\\
258	0.00390863140028114\\
259	0.00390858193565703\\
260	0.00390853160728851\\
261	0.00390848039990502\\
262	0.0039084282979624\\
263	0.00390837528563805\\
264	0.003908321346826\\
265	0.0039082664651319\\
266	0.00390821062386793\\
267	0.00390815380604762\\
268	0.0039080959943806\\
269	0.00390803717126733\\
270	0.00390797731879361\\
271	0.00390791641872519\\
272	0.00390785445250214\\
273	0.00390779140123325\\
274	0.0039077272456903\\
275	0.00390766196630227\\
276	0.00390759554314946\\
277	0.00390752795595753\\
278	0.0039074591840915\\
279	0.00390738920654961\\
280	0.00390731800195713\\
281	0.00390724554856015\\
282	0.00390717182421921\\
283	0.00390709680640285\\
284	0.00390702047218123\\
285	0.00390694279821945\\
286	0.00390686376077104\\
287	0.00390678333567118\\
288	0.00390670149832998\\
289	0.00390661822372562\\
290	0.00390653348639748\\
291	0.00390644726043916\\
292	0.00390635951949148\\
293	0.00390627023673536\\
294	0.00390617938488474\\
295	0.00390608693617933\\
296	0.00390599286237742\\
297	0.00390589713474855\\
298	0.00390579972406615\\
299	0.00390570060060022\\
300	0.00390559973410983\\
301	0.00390549709383573\\
302	0.00390539264849274\\
303	0.00390528636626233\\
304	0.00390517821478497\\
305	0.00390506816115258\\
306	0.00390495617190089\\
307	0.00390484221300179\\
308	0.00390472624985571\\
309	0.00390460824728389\\
310	0.00390448816952074\\
311	0.00390436598020609\\
312	0.00390424164237752\\
313	0.0039041151184626\\
314	0.00390398637027116\\
315	0.00390385535898758\\
316	0.00390372204516303\\
317	0.0039035863887077\\
318	0.00390344834888305\\
319	0.00390330788429402\\
320	0.00390316495288124\\
321	0.00390301951191325\\
322	0.0039028715179786\\
323	0.00390272092697804\\
324	0.00390256769411655\\
325	0.00390241177389542\\
326	0.0039022531201042\\
327	0.00390209168581261\\
328	0.00390192742336232\\
329	0.00390176028435864\\
330	0.00390159021966208\\
331	0.00390141717937973\\
332	0.00390124111285647\\
333	0.00390106196866593\\
334	0.00390087969460124\\
335	0.00390069423766547\\
336	0.0039005055440617\\
337	0.00390031355918274\\
338	0.00390011822760038\\
339	0.00389991949305422\\
340	0.00389971729843982\\
341	0.0038995115857963\\
342	0.00389930229629321\\
343	0.00389908937021664\\
344	0.00389887274695436\\
345	0.00389865236498013\\
346	0.00389842816183678\\
347	0.0038982000741182\\
348	0.00389796803745\\
349	0.00389773198646871\\
350	0.00389749185479949\\
351	0.003897247575032\\
352	0.00389699907869457\\
353	0.00389674629622624\\
354	0.00389648915694669\\
355	0.00389622758902382\\
356	0.00389596151943881\\
357	0.00389569087394843\\
358	0.0038954155770446\\
359	0.00389513555191069\\
360	0.00389485072037473\\
361	0.00389456100285906\\
362	0.00389426631832628\\
363	0.00389396658422146\\
364	0.00389366171641021\\
365	0.00389335162911248\\
366	0.00389303623483195\\
367	0.00389271544428074\\
368	0.00389238916629931\\
369	0.00389205730777134\\
370	0.0038917197735333\\
371	0.00389137646627871\\
372	0.00389102728645676\\
373	0.00389067213216507\\
374	0.0038903108990365\\
375	0.00388994348011961\\
376	0.0038895697657526\\
377	0.00388918964343038\\
378	0.00388880299766439\\
379	0.00388840970983479\\
380	0.00388800965803425\\
381	0.003887602716903\\
382	0.00388718875745404\\
383	0.00388676764688758\\
384	0.00388633924839353\\
385	0.00388590342094045\\
386	0.0038854600190492\\
387	0.00388500889254906\\
388	0.00388454988631361\\
389	0.00388408283997319\\
390	0.00388360758759974\\
391	0.00388312395735876\\
392	0.0038826317711215\\
393	0.00388213084402801\\
394	0.00388162098398701\\
395	0.00388110199109091\\
396	0.00388057365691044\\
397	0.00388003576361041\\
398	0.0038794880827994\\
399	0.00387893037401931\\
400	0.00387836238289632\\
401	0.00387778383944593\\
402	0.00387719445811215\\
403	0.00387659394204268\\
404	0.00387598198951732\\
405	0.00387535828191595\\
406	0.00387472246678478\\
407	0.00387407417172406\\
408	0.0038734130033376\\
409	0.00387273854653858\\
410	0.00387205036454529\\
411	0.00387134799997831\\
412	0.00387063097718826\\
413	0.00386989880449444\\
414	0.00386915097106735\\
415	0.00386838692695242\\
416	0.00386760603773274\\
417	0.00386680756080642\\
418	0.00386599080346426\\
419	0.00386515503338101\\
420	0.00386429945743454\\
421	0.00386342321410664\\
422	0.00386252536473786\\
423	0.00386160488343816\\
424	0.00386066064541408\\
425	0.00385969141341615\\
426	0.00385869582192358\\
427	0.00385767235852507\\
428	0.00385661934161984\\
429	0.00385553489274932\\
430	0.00385441689976431\\
431	0.00385326296140546\\
432	0.00385207028869819\\
433	0.00385083549785069\\
434	0.00384955412228839\\
435	0.00384821940056294\\
436	0.00384681926064951\\
437	0.0038453291565604\\
438	0.00384369704008515\\
439	0.00384182192617243\\
440	0.00383957929471537\\
441	0.00383713441622716\\
442	0.00383464069763525\\
443	0.00383209695164383\\
444	0.0038295019659953\\
445	0.00382685450630891\\
446	0.00382415331976878\\
447	0.00382139713981654\\
448	0.00381858469201794\\
449	0.00381571470126403\\
450	0.00381278590039141\\
451	0.00380979704004449\\
452	0.00380674689886397\\
453	0.00380363429115159\\
454	0.00380045806428968\\
455	0.00379721706609648\\
456	0.00379391003229393\\
457	0.00379053526911188\\
458	0.00378708980565845\\
459	0.00378356710029378\\
460	0.00377995040566156\\
461	0.00377619143105491\\
462	0.0037721334462718\\
463	0.0037672220531096\\
464	0.00375953648037518\\
465	0.00374259590734272\\
466	0.00372533544350868\\
467	0.00370774507665263\\
468	0.00368981399443503\\
469	0.00367153110135167\\
470	0.00365288634541003\\
471	0.00363386936568687\\
472	0.00361446838002628\\
473	0.00359467186430756\\
474	0.00357446889131582\\
475	0.00355384847238614\\
476	0.00353279567495486\\
477	0.00351128549018164\\
478	0.00348929938958904\\
479	0.00346681983209936\\
480	0.00344382812335562\\
481	0.00342030431059674\\
482	0.00339622706474022\\
483	0.00337157354405513\\
484	0.00334631923702561\\
485	0.00332043778241085\\
486	0.00329390076330926\\
487	0.0032666774663388\\
488	0.00323873460222879\\
489	0.0032100359918576\\
490	0.00318054220155812\\
491	0.00315021011715184\\
492	0.00311899244306825\\
493	0.00308683710781825\\
494	0.00305368654728657\\
495	0.00301947681520266\\
496	0.0029841364142171\\
497	0.0029475845893136\\
498	0.00290972839773051\\
499	0.00287045663723923\\
500	0.00282962510700878\\
501	0.00278701696808144\\
502	0.00274222972852771\\
503	0.00269434170656592\\
504	0.00264090605458067\\
505	0.00257465229279033\\
506	0.00249789749396208\\
507	0.00241955633568501\\
508	0.00233969171468241\\
509	0.00225848184594896\\
510	0.00217611043500788\\
511	0.00209209122396565\\
512	0.00200591458078461\\
513	0.00191746610396971\\
514	0.00182662038391383\\
515	0.00173323967234684\\
516	0.00163717243645703\\
517	0.00153825200793115\\
518	0.00143629628630456\\
519	0.00133111244935181\\
520	0.00122252232193912\\
521	0.00111046918952441\\
522	0.000995439653286779\\
523	0.000880092985072983\\
524	0.000775499111761147\\
525	0.000689867989463739\\
526	0.000601110061332961\\
527	0.000508813764893824\\
528	0.000412124598928745\\
529	0.000308800153975222\\
530	0.000192070774563915\\
531	5.80908334167806e-05\\
532	0\\
533	0\\
534	0\\
535	0\\
536	0\\
537	0\\
538	0\\
539	0\\
540	0\\
541	0\\
542	0\\
543	0\\
544	0\\
545	0\\
546	0\\
547	0\\
548	0\\
549	0\\
550	0\\
551	0\\
552	0\\
553	0\\
554	0\\
555	0\\
556	0\\
557	0\\
558	0\\
559	0\\
560	0\\
561	0\\
562	0\\
563	0\\
564	0\\
565	0\\
566	0\\
567	0\\
568	0\\
569	0\\
570	0\\
571	0\\
572	0\\
573	0\\
574	0\\
575	0\\
576	0\\
577	0\\
578	0\\
579	0\\
580	0\\
581	0\\
582	0\\
583	0\\
584	0\\
585	0\\
586	0\\
587	0\\
588	0\\
589	0\\
590	0\\
591	0\\
592	0\\
593	0\\
594	0\\
595	0\\
596	0\\
597	0\\
598	0\\
599	0\\
600	0\\
};
\addplot [color=mycolor10,solid,forget plot]
  table[row sep=crcr]{%
1	0.00400023926843176\\
2	0.00400023923938983\\
3	0.00400023920985951\\
4	0.0040002391798326\\
5	0.00400023914930074\\
6	0.00400023911825545\\
7	0.00400023908668808\\
8	0.00400023905458986\\
9	0.00400023902195187\\
10	0.00400023898876501\\
11	0.00400023895502006\\
12	0.00400023892070763\\
13	0.00400023888581818\\
14	0.00400023885034198\\
15	0.00400023881426918\\
16	0.00400023877758973\\
17	0.00400023874029343\\
18	0.00400023870236989\\
19	0.00400023866380856\\
20	0.00400023862459869\\
21	0.00400023858472939\\
22	0.00400023854418953\\
23	0.00400023850296783\\
24	0.00400023846105281\\
25	0.0040002384184328\\
26	0.00400023837509592\\
27	0.00400023833103009\\
28	0.00400023828622305\\
29	0.00400023824066229\\
30	0.00400023819433513\\
31	0.00400023814722864\\
32	0.0040002380993297\\
33	0.00400023805062494\\
34	0.00400023800110079\\
35	0.00400023795074342\\
36	0.0040002378995388\\
37	0.00400023784747262\\
38	0.00400023779453037\\
39	0.00400023774069726\\
40	0.00400023768595826\\
41	0.00400023763029808\\
42	0.0040002375737012\\
43	0.00400023751615178\\
44	0.00400023745763376\\
45	0.00400023739813077\\
46	0.00400023733762619\\
47	0.00400023727610309\\
48	0.00400023721354428\\
49	0.00400023714993225\\
50	0.0040002370852492\\
51	0.00400023701947703\\
52	0.00400023695259733\\
53	0.00400023688459137\\
54	0.0040002368154401\\
55	0.00400023674512415\\
56	0.00400023667362381\\
57	0.00400023660091903\\
58	0.00400023652698944\\
59	0.00400023645181429\\
60	0.00400023637537249\\
61	0.00400023629764258\\
62	0.00400023621860273\\
63	0.00400023613823075\\
64	0.00400023605650405\\
65	0.00400023597339967\\
66	0.00400023588889424\\
67	0.00400023580296397\\
68	0.00400023571558471\\
69	0.00400023562673184\\
70	0.00400023553638036\\
71	0.0040002354445048\\
72	0.00400023535107926\\
73	0.00400023525607742\\
74	0.00400023515947247\\
75	0.00400023506123715\\
76	0.00400023496134373\\
77	0.00400023485976398\\
78	0.00400023475646921\\
79	0.00400023465143022\\
80	0.00400023454461729\\
81	0.00400023443600019\\
82	0.00400023432554817\\
83	0.00400023421322995\\
84	0.00400023409901369\\
85	0.004000233982867\\
86	0.00400023386475693\\
87	0.00400023374464995\\
88	0.00400023362251195\\
89	0.00400023349830821\\
90	0.00400023337200343\\
91	0.00400023324356165\\
92	0.00400023311294632\\
93	0.00400023298012024\\
94	0.00400023284504553\\
95	0.00400023270768369\\
96	0.00400023256799551\\
97	0.0040002324259411\\
98	0.00400023228147987\\
99	0.00400023213457053\\
100	0.00400023198517103\\
101	0.00400023183323862\\
102	0.00400023167872976\\
103	0.00400023152160017\\
104	0.00400023136180476\\
105	0.00400023119929767\\
106	0.00400023103403222\\
107	0.00400023086596091\\
108	0.00400023069503539\\
109	0.00400023052120645\\
110	0.00400023034442404\\
111	0.00400023016463718\\
112	0.00400022998179403\\
113	0.0040002297958418\\
114	0.00400022960672678\\
115	0.0040002294143943\\
116	0.00400022921878872\\
117	0.00400022901985342\\
118	0.00400022881753078\\
119	0.00400022861176213\\
120	0.00400022840248779\\
121	0.004000228189647\\
122	0.00400022797317793\\
123	0.00400022775301766\\
124	0.00400022752910214\\
125	0.00400022730136618\\
126	0.00400022706974344\\
127	0.00400022683416641\\
128	0.00400022659456636\\
129	0.00400022635087337\\
130	0.00400022610301626\\
131	0.00400022585092258\\
132	0.00400022559451863\\
133	0.00400022533372936\\
134	0.00400022506847843\\
135	0.00400022479868814\\
136	0.00400022452427939\\
137	0.00400022424517172\\
138	0.00400022396128322\\
139	0.00400022367253056\\
140	0.00400022337882892\\
141	0.004000223080092\\
142	0.00400022277623198\\
143	0.00400022246715948\\
144	0.00400022215278359\\
145	0.00400022183301175\\
146	0.00400022150774984\\
147	0.00400022117690204\\
148	0.0040002208403709\\
149	0.00400022049805724\\
150	0.00400022014986016\\
151	0.00400021979567702\\
152	0.00400021943540338\\
153	0.00400021906893297\\
154	0.00400021869615773\\
155	0.00400021831696768\\
156	0.00400021793125095\\
157	0.00400021753889377\\
158	0.00400021713978037\\
159	0.004000216733793\\
160	0.00400021632081189\\
161	0.0040002159007152\\
162	0.00400021547337901\\
163	0.00400021503867728\\
164	0.00400021459648178\\
165	0.00400021414666211\\
166	0.00400021368908563\\
167	0.00400021322361742\\
168	0.00400021275012026\\
169	0.00400021226845458\\
170	0.00400021177847842\\
171	0.00400021128004738\\
172	0.00400021077301461\\
173	0.0040002102572307\\
174	0.0040002097325437\\
175	0.00400020919879907\\
176	0.00400020865583955\\
177	0.00400020810350523\\
178	0.00400020754163341\\
179	0.00400020697005859\\
180	0.00400020638861238\\
181	0.00400020579712349\\
182	0.00400020519541765\\
183	0.00400020458331754\\
184	0.00400020396064278\\
185	0.0040002033272098\\
186	0.00400020268283185\\
187	0.00400020202731888\\
188	0.00400020136047753\\
189	0.00400020068211103\\
190	0.00400019999201917\\
191	0.00400019928999819\\
192	0.00400019857584077\\
193	0.00400019784933593\\
194	0.00400019711026898\\
195	0.00400019635842145\\
196	0.004000195593571\\
197	0.00400019481549139\\
198	0.0040001940239524\\
199	0.0040001932187197\\
200	0.00400019239955489\\
201	0.0040001915662153\\
202	0.004000190718454\\
203	0.0040001898560197\\
204	0.00400018897865665\\
205	0.0040001880861046\\
206	0.00400018717809867\\
207	0.00400018625436929\\
208	0.00400018531464214\\
209	0.00400018435863801\\
210	0.00400018338607277\\
211	0.00400018239665722\\
212	0.00400018139009704\\
213	0.00400018036609271\\
214	0.00400017932433935\\
215	0.0040001782645267\\
216	0.00400017718633895\\
217	0.00400017608945471\\
218	0.00400017497354685\\
219	0.00400017383828242\\
220	0.00400017268332256\\
221	0.00400017150832235\\
222	0.00400017031293074\\
223	0.00400016909679041\\
224	0.00400016785953769\\
225	0.0040001666008024\\
226	0.00400016532020775\\
227	0.00400016401737025\\
228	0.00400016269189953\\
229	0.00400016134339824\\
230	0.00400015997146196\\
231	0.00400015857567902\\
232	0.00400015715563037\\
233	0.00400015571088948\\
234	0.00400015424102217\\
235	0.00400015274558649\\
236	0.00400015122413256\\
237	0.00400014967620246\\
238	0.00400014810133003\\
239	0.00400014649904076\\
240	0.00400014486885164\\
241	0.00400014321027096\\
242	0.00400014152279822\\
243	0.00400013980592388\\
244	0.0040001380591293\\
245	0.00400013628188648\\
246	0.00400013447365795\\
247	0.00400013263389657\\
248	0.00400013076204536\\
249	0.00400012885753732\\
250	0.00400012691979526\\
251	0.00400012494823157\\
252	0.00400012294224811\\
253	0.00400012090123594\\
254	0.00400011882457518\\
255	0.00400011671163477\\
256	0.0040001145617723\\
257	0.00400011237433378\\
258	0.00400011014865345\\
259	0.00400010788405356\\
260	0.00400010557984417\\
261	0.0040001032353229\\
262	0.0040001008497747\\
263	0.00400009842247171\\
264	0.00400009595267291\\
265	0.00400009343962396\\
266	0.00400009088255696\\
267	0.00400008828069019\\
268	0.00400008563322785\\
269	0.00400008293935985\\
270	0.00400008019826155\\
271	0.00400007740909347\\
272	0.00400007457100108\\
273	0.00400007168311449\\
274	0.00400006874454821\\
275	0.00400006575440089\\
276	0.004000062711755\\
277	0.00400005961567661\\
278	0.00400005646521507\\
279	0.00400005325940272\\
280	0.00400004999725463\\
281	0.00400004667776828\\
282	0.00400004329992329\\
283	0.00400003986268108\\
284	0.00400003636498461\\
285	0.00400003280575806\\
286	0.00400002918390648\\
287	0.00400002549831555\\
288	0.00400002174785119\\
289	0.00400001793135928\\
290	0.00400001404766534\\
291	0.00400001009557416\\
292	0.00400000607386954\\
293	0.00400000198131388\\
294	0.0039999978166479\\
295	0.00399999357859029\\
296	0.00399998926583733\\
297	0.00399998487706262\\
298	0.00399998041091665\\
299	0.00399997586602652\\
300	0.00399997124099554\\
301	0.00399996653440293\\
302	0.00399996174480339\\
303	0.00399995687072683\\
304	0.00399995191067793\\
305	0.00399994686313584\\
306	0.00399994172655379\\
307	0.00399993649935871\\
308	0.00399993117995091\\
309	0.00399992576670368\\
310	0.00399992025796293\\
311	0.00399991465204682\\
312	0.00399990894724539\\
313	0.00399990314182019\\
314	0.00399989723400391\\
315	0.00399989122199999\\
316	0.00399988510398228\\
317	0.00399987887809461\\
318	0.00399987254245047\\
319	0.00399986609513257\\
320	0.00399985953419253\\
321	0.00399985285765041\\
322	0.00399984606349439\\
323	0.00399983914968035\\
324	0.00399983211413148\\
325	0.00399982495473788\\
326	0.00399981766935617\\
327	0.00399981025580903\\
328	0.00399980271188486\\
329	0.00399979503533729\\
330	0.00399978722388479\\
331	0.00399977927521018\\
332	0.00399977118696023\\
333	0.00399976295674514\\
334	0.00399975458213807\\
335	0.00399974606067465\\
336	0.00399973738985242\\
337	0.00399972856713033\\
338	0.00399971958992814\\
339	0.0039997104556258\\
340	0.00399970116156288\\
341	0.00399969170503784\\
342	0.00399968208330732\\
343	0.00399967229358544\\
344	0.00399966233304298\\
345	0.00399965219880651\\
346	0.00399964188795746\\
347	0.00399963139753122\\
348	0.00399962072451601\\
349	0.00399960986585184\\
350	0.00399959881842922\\
351	0.00399958757908794\\
352	0.00399957614461565\\
353	0.00399956451174636\\
354	0.00399955267715886\\
355	0.00399954063747494\\
356	0.00399952838925759\\
357	0.003999515929009\\
358	0.00399950325316837\\
359	0.00399949035810966\\
360	0.00399947724013911\\
361	0.00399946389549262\\
362	0.0039994503203329\\
363	0.0039994365107465\\
364	0.00399942246274054\\
365	0.00399940817223933\\
366	0.00399939363508073\\
367	0.00399937884701221\\
368	0.00399936380368681\\
369	0.00399934850065873\\
370	0.00399933293337871\\
371	0.00399931709718917\\
372	0.00399930098731901\\
373	0.00399928459887818\\
374	0.00399926792685191\\
375	0.00399925096609462\\
376	0.00399923371132354\\
377	0.00399921615711194\\
378	0.003999198297882\\
379	0.00399918012789727\\
380	0.00399916164125467\\
381	0.00399914283187607\\
382	0.00399912369349926\\
383	0.00399910421966836\\
384	0.00399908440372357\\
385	0.00399906423879013\\
386	0.0039990437177664\\
387	0.00399902283331092\\
388	0.00399900157782828\\
389	0.00399897994345366\\
390	0.0039989579220355\\
391	0.00399893550511629\\
392	0.00399891268391063\\
393	0.0039988894492798\\
394	0.00399886579170178\\
395	0.00399884170123454\\
396	0.00399881716747023\\
397	0.00399879217947742\\
398	0.00399876672573156\\
399	0.00399874079404327\\
400	0.00399871437151541\\
401	0.00399868744458336\\
402	0.00399865999916131\\
403	0.00399863202070385\\
404	0.00399860349370784\\
405	0.00399857440128768\\
406	0.00399854472560509\\
407	0.00399851444782748\\
408	0.00399848354810668\\
409	0.00399845200558938\\
410	0.00399841979846344\\
411	0.00399838690401134\\
412	0.00399835329856065\\
413	0.00399831895709304\\
414	0.00399828385224411\\
415	0.0039982479529837\\
416	0.00399821122496071\\
417	0.00399817363471804\\
418	0.00399813514658468\\
419	0.00399809572187645\\
420	0.00399805531851538\\
421	0.00399801389059089\\
422	0.00399797138785207\\
423	0.00399792775511748\\
424	0.00399788293158426\\
425	0.00399783685000886\\
426	0.00399778943571111\\
427	0.00399774060530526\\
428	0.00399769026494495\\
429	0.00399763830758216\\
430	0.0039975846080481\\
431	0.0039975290131297\\
432	0.00399747132012266\\
433	0.00399741122957038\\
434	0.00399734824351354\\
435	0.00399728146088603\\
436	0.00399720922077238\\
437	0.00399712866805858\\
438	0.00399703587301835\\
439	0.00399692855921201\\
440	0.00399681364852388\\
441	0.00399669628499214\\
442	0.00399657640237339\\
443	0.00399645393267616\\
444	0.00399632880627346\\
445	0.00399620095205716\\
446	0.00399607029764142\\
447	0.00399593676962055\\
448	0.00399580029388026\\
449	0.00399566079594179\\
450	0.00399551820126536\\
451	0.00399537243530081\\
452	0.00399522342271854\\
453	0.00399507108434469\\
454	0.00399491532797951\\
455	0.00399475602314739\\
456	0.00399459293353915\\
457	0.00399442553680086\\
458	0.00399425254035359\\
459	0.0039940705701746\\
460	0.00399387062197271\\
461	0.00399362865816175\\
462	0.00399328224082247\\
463	0.00399268179826925\\
464	0.00399152034506261\\
465	0.00398942525004241\\
466	0.003987295346337\\
467	0.00398512966627529\\
468	0.00398292721000522\\
469	0.00398068697662909\\
470	0.00397840791133193\\
471	0.00397608888287425\\
472	0.00397372874284859\\
473	0.00397132632195255\\
474	0.00396888037428475\\
475	0.00396638943872895\\
476	0.00396385175245868\\
477	0.0039612657638208\\
478	0.00395862990298715\\
479	0.00395594249982952\\
480	0.00395320177500551\\
481	0.00395040582985437\\
482	0.0039475526346765\\
483	0.00394464001520579\\
484	0.00394166563706097\\
485	0.00393862698785699\\
486	0.00393552135645141\\
487	0.00393234580918657\\
488	0.00392909716312208\\
489	0.00392577195493935\\
490	0.00392236640455913\\
491	0.00391887637213836\\
492	0.00391529730638808\\
493	0.0039116241805659\\
494	0.00390785140870817\\
495	0.0039039727252085\\
496	0.00389998098680041\\
497	0.00389586779479689\\
498	0.0038916226814718\\
499	0.00388723122648509\\
500	0.00388267058066376\\
501	0.0038778989473318\\
502	0.00387283208197042\\
503	0.00386729674842656\\
504	0.00386096597286974\\
505	0.0038534149944631\\
506	0.00384507261769542\\
507	0.00383665866359003\\
508	0.00382817739036378\\
509	0.00381962802362775\\
510	0.00381099192896018\\
511	0.00380225392818428\\
512	0.00379341008661713\\
513	0.00378445601683264\\
514	0.00377538679341138\\
515	0.0037661967812744\\
516	0.0037568792082255\\
517	0.00374742495454606\\
518	0.00373781888331623\\
519	0.00372802825804615\\
520	0.00371796508866977\\
521	0.00370736053989885\\
522	0.00369533602505041\\
523	0.00367890636217919\\
524	0.0036476543580472\\
525	0.00359447189176134\\
526	0.00353963152376423\\
527	0.0034829720449033\\
528	0.00342425368577073\\
529	0.00336310709277027\\
530	0.00329908329211729\\
531	0.00323217575810164\\
532	0.00316298165281235\\
533	0.00309166511403029\\
534	0.00301814370668786\\
535	0.00294111313161606\\
536	0.00285499501479115\\
537	0.00274701482624045\\
538	0.00261820957907621\\
539	0.0024858002176776\\
540	0.00234954247109751\\
541	0.00220916502212144\\
542	0.0020643662871162\\
543	0.00191481016471741\\
544	0.00176011232967979\\
545	0.00159980419294568\\
546	0.00143325509308531\\
547	0.00125945611117335\\
548	0.00107633464582706\\
549	0.000878403440878852\\
550	0.000652863504763817\\
551	0.00041853810151925\\
552	0.00017128870103758\\
553	0\\
554	0\\
555	0\\
556	0\\
557	0\\
558	0\\
559	0\\
560	0\\
561	0\\
562	0\\
563	0\\
564	0\\
565	0\\
566	0\\
567	0\\
568	0\\
569	0\\
570	0\\
571	0\\
572	0\\
573	0\\
574	0\\
575	0\\
576	0\\
577	0\\
578	0\\
579	0\\
580	0\\
581	0\\
582	0\\
583	0\\
584	0\\
585	0\\
586	0\\
587	0\\
588	0\\
589	0\\
590	0\\
591	0\\
592	0\\
593	0\\
594	0\\
595	0\\
596	0\\
597	0\\
598	0\\
599	0\\
600	0\\
};
\addplot [color=mycolor11,solid,forget plot]
  table[row sep=crcr]{%
1	0.00406147387421855\\
2	0.00406147387294227\\
3	0.00406147387164454\\
4	0.00406147387032498\\
5	0.00406147386898323\\
6	0.00406147386761891\\
7	0.00406147386623165\\
8	0.00406147386482107\\
9	0.00406147386338676\\
10	0.00406147386192833\\
11	0.00406147386044538\\
12	0.00406147385893749\\
13	0.00406147385740424\\
14	0.0040614738558452\\
15	0.00406147385425995\\
16	0.00406147385264803\\
17	0.004061473851009\\
18	0.00406147384934242\\
19	0.0040614738476478\\
20	0.00406147384592468\\
21	0.00406147384417258\\
22	0.00406147384239101\\
23	0.00406147384057948\\
24	0.00406147383873747\\
25	0.00406147383686449\\
26	0.00406147383495999\\
27	0.00406147383302347\\
28	0.00406147383105436\\
29	0.00406147382905214\\
30	0.00406147382701623\\
31	0.00406147382494607\\
32	0.00406147382284109\\
33	0.00406147382070069\\
34	0.00406147381852427\\
35	0.00406147381631124\\
36	0.00406147381406098\\
37	0.00406147381177285\\
38	0.00406147380944621\\
39	0.00406147380708042\\
40	0.00406147380467482\\
41	0.00406147380222874\\
42	0.00406147379974148\\
43	0.00406147379721236\\
44	0.00406147379464067\\
45	0.0040614737920257\\
46	0.0040614737893667\\
47	0.00406147378666293\\
48	0.00406147378391364\\
49	0.00406147378111807\\
50	0.00406147377827542\\
51	0.0040614737753849\\
52	0.00406147377244571\\
53	0.00406147376945701\\
54	0.00406147376641797\\
55	0.00406147376332775\\
56	0.00406147376018546\\
57	0.00406147375699024\\
58	0.00406147375374118\\
59	0.00406147375043737\\
60	0.00406147374707789\\
61	0.00406147374366179\\
62	0.00406147374018812\\
63	0.00406147373665589\\
64	0.00406147373306411\\
65	0.00406147372941178\\
66	0.00406147372569786\\
67	0.00406147372192131\\
68	0.00406147371808107\\
69	0.00406147371417606\\
70	0.00406147371020517\\
71	0.00406147370616729\\
72	0.00406147370206127\\
73	0.00406147369788597\\
74	0.00406147369364019\\
75	0.00406147368932274\\
76	0.00406147368493241\\
77	0.00406147368046794\\
78	0.00406147367592809\\
79	0.00406147367131156\\
80	0.00406147366661704\\
81	0.00406147366184321\\
82	0.00406147365698872\\
83	0.00406147365205219\\
84	0.00406147364703221\\
85	0.00406147364192737\\
86	0.00406147363673621\\
87	0.00406147363145726\\
88	0.00406147362608902\\
89	0.00406147362062996\\
90	0.00406147361507853\\
91	0.00406147360943315\\
92	0.0040614736036922\\
93	0.00406147359785406\\
94	0.00406147359191705\\
95	0.00406147358587949\\
96	0.00406147357973964\\
97	0.00406147357349575\\
98	0.00406147356714604\\
99	0.00406147356068869\\
100	0.00406147355412185\\
101	0.00406147354744364\\
102	0.00406147354065214\\
103	0.00406147353374541\\
104	0.00406147352672145\\
105	0.00406147351957827\\
106	0.00406147351231378\\
107	0.00406147350492592\\
108	0.00406147349741255\\
109	0.0040614734897715\\
110	0.00406147348200058\\
111	0.00406147347409753\\
112	0.00406147346606009\\
113	0.00406147345788592\\
114	0.00406147344957267\\
115	0.00406147344111792\\
116	0.00406147343251923\\
117	0.0040614734237741\\
118	0.00406147341488001\\
119	0.00406147340583437\\
120	0.00406147339663454\\
121	0.00406147338727787\\
122	0.00406147337776163\\
123	0.00406147336808304\\
124	0.0040614733582393\\
125	0.00406147334822752\\
126	0.0040614733380448\\
127	0.00406147332768816\\
128	0.00406147331715458\\
129	0.00406147330644098\\
130	0.00406147329554423\\
131	0.00406147328446114\\
132	0.00406147327318846\\
133	0.0040614732617229\\
134	0.00406147325006109\\
135	0.00406147323819962\\
136	0.004061473226135\\
137	0.00406147321386371\\
138	0.00406147320138212\\
139	0.00406147318868658\\
140	0.00406147317577335\\
141	0.00406147316263864\\
142	0.00406147314927858\\
143	0.00406147313568924\\
144	0.00406147312186662\\
145	0.00406147310780666\\
146	0.0040614730935052\\
147	0.00406147307895804\\
148	0.00406147306416089\\
149	0.00406147304910937\\
150	0.00406147303379907\\
151	0.00406147301822546\\
152	0.00406147300238396\\
153	0.00406147298626988\\
154	0.00406147296987847\\
155	0.0040614729532049\\
156	0.00406147293624424\\
157	0.0040614729189915\\
158	0.00406147290144159\\
159	0.00406147288358932\\
160	0.00406147286542943\\
161	0.00406147284695656\\
162	0.00406147282816527\\
163	0.00406147280905001\\
164	0.00406147278960514\\
165	0.00406147276982494\\
166	0.00406147274970356\\
167	0.00406147272923508\\
168	0.00406147270841346\\
169	0.00406147268723257\\
170	0.00406147266568616\\
171	0.00406147264376789\\
172	0.0040614726214713\\
173	0.00406147259878982\\
174	0.00406147257571676\\
175	0.00406147255224534\\
176	0.00406147252836863\\
177	0.00406147250407961\\
178	0.00406147247937111\\
179	0.00406147245423586\\
180	0.00406147242866646\\
181	0.00406147240265536\\
182	0.0040614723761949\\
183	0.00406147234927729\\
184	0.00406147232189459\\
185	0.00406147229403871\\
186	0.00406147226570145\\
187	0.00406147223687445\\
188	0.00406147220754919\\
189	0.00406147217771702\\
190	0.00406147214736913\\
191	0.00406147211649655\\
192	0.00406147208509017\\
193	0.0040614720531407\\
194	0.00406147202063869\\
195	0.00406147198757453\\
196	0.00406147195393844\\
197	0.00406147191972046\\
198	0.00406147188491047\\
199	0.00406147184949814\\
200	0.00406147181347299\\
201	0.00406147177682435\\
202	0.00406147173954134\\
203	0.00406147170161291\\
204	0.00406147166302779\\
205	0.00406147162377455\\
206	0.00406147158384153\\
207	0.00406147154321685\\
208	0.00406147150188845\\
209	0.00406147145984404\\
210	0.00406147141707111\\
211	0.00406147137355695\\
212	0.00406147132928859\\
213	0.00406147128425285\\
214	0.00406147123843631\\
215	0.00406147119182533\\
216	0.00406147114440599\\
217	0.00406147109616416\\
218	0.00406147104708543\\
219	0.00406147099715515\\
220	0.00406147094635841\\
221	0.00406147089468003\\
222	0.00406147084210454\\
223	0.00406147078861623\\
224	0.00406147073419909\\
225	0.00406147067883681\\
226	0.00406147062251282\\
227	0.00406147056521024\\
228	0.00406147050691188\\
229	0.00406147044760024\\
230	0.00406147038725754\\
231	0.00406147032586563\\
232	0.00406147026340609\\
233	0.00406147019986012\\
234	0.00406147013520862\\
235	0.00406147006943212\\
236	0.00406147000251083\\
237	0.00406146993442459\\
238	0.00406146986515287\\
239	0.00406146979467477\\
240	0.00406146972296904\\
241	0.00406146965001403\\
242	0.0040614695757877\\
243	0.00406146950026761\\
244	0.00406146942343094\\
245	0.00406146934525444\\
246	0.00406146926571443\\
247	0.00406146918478683\\
248	0.00406146910244713\\
249	0.00406146901867034\\
250	0.00406146893343106\\
251	0.00406146884670341\\
252	0.00406146875846106\\
253	0.0040614686686772\\
254	0.00406146857732452\\
255	0.00406146848437525\\
256	0.0040614683898011\\
257	0.00406146829357326\\
258	0.00406146819566243\\
259	0.00406146809603876\\
260	0.00406146799467186\\
261	0.00406146789153082\\
262	0.00406146778658415\\
263	0.00406146767979979\\
264	0.00406146757114512\\
265	0.00406146746058691\\
266	0.00406146734809136\\
267	0.00406146723362405\\
268	0.00406146711714993\\
269	0.00406146699863334\\
270	0.00406146687803795\\
271	0.00406146675532681\\
272	0.00406146663046229\\
273	0.00406146650340609\\
274	0.00406146637411921\\
275	0.00406146624256198\\
276	0.00406146610869399\\
277	0.00406146597247412\\
278	0.00406146583386051\\
279	0.00406146569281057\\
280	0.00406146554928092\\
281	0.00406146540322742\\
282	0.00406146525460517\\
283	0.00406146510336842\\
284	0.00406146494947065\\
285	0.00406146479286449\\
286	0.00406146463350175\\
287	0.00406146447133337\\
288	0.00406146430630942\\
289	0.00406146413837912\\
290	0.00406146396749075\\
291	0.00406146379359172\\
292	0.00406146361662849\\
293	0.0040614634365466\\
294	0.00406146325329061\\
295	0.00406146306680416\\
296	0.00406146287702986\\
297	0.00406146268390935\\
298	0.00406146248738325\\
299	0.00406146228739114\\
300	0.00406146208387159\\
301	0.00406146187676207\\
302	0.00406146166599901\\
303	0.00406146145151774\\
304	0.00406146123325248\\
305	0.00406146101113634\\
306	0.00406146078510128\\
307	0.00406146055507813\\
308	0.00406146032099653\\
309	0.00406146008278495\\
310	0.00406145984037066\\
311	0.00406145959367972\\
312	0.00406145934263694\\
313	0.0040614590871659\\
314	0.00406145882718891\\
315	0.00406145856262702\\
316	0.00406145829339994\\
317	0.00406145801942611\\
318	0.00406145774062263\\
319	0.00406145745690525\\
320	0.00406145716818835\\
321	0.00406145687438495\\
322	0.00406145657540666\\
323	0.00406145627116369\\
324	0.00406145596156482\\
325	0.00406145564651736\\
326	0.00406145532592719\\
327	0.00406145499969867\\
328	0.00406145466773469\\
329	0.00406145432993661\\
330	0.00406145398620423\\
331	0.00406145363643581\\
332	0.00406145328052804\\
333	0.00406145291837597\\
334	0.00406145254987306\\
335	0.00406145217491112\\
336	0.00406145179338026\\
337	0.00406145140516892\\
338	0.00406145101016383\\
339	0.00406145060824992\\
340	0.00406145019931037\\
341	0.00406144978322657\\
342	0.00406144935987801\\
343	0.00406144892914235\\
344	0.00406144849089529\\
345	0.00406144804501061\\
346	0.00406144759136004\\
347	0.00406144712981331\\
348	0.00406144666023802\\
349	0.00406144618249962\\
350	0.00406144569646135\\
351	0.00406144520198418\\
352	0.00406144469892675\\
353	0.00406144418714527\\
354	0.00406144366649348\\
355	0.00406144313682254\\
356	0.00406144259798095\\
357	0.00406144204981445\\
358	0.00406144149216595\\
359	0.00406144092487536\\
360	0.00406144034777953\\
361	0.00406143976071209\\
362	0.00406143916350335\\
363	0.00406143855598009\\
364	0.00406143793796551\\
365	0.00406143730927897\\
366	0.00406143666973589\\
367	0.00406143601914752\\
368	0.00406143535732077\\
369	0.00406143468405802\\
370	0.00406143399915687\\
371	0.00406143330240993\\
372	0.00406143259360459\\
373	0.00406143187252274\\
374	0.00406143113894053\\
375	0.00406143039262806\\
376	0.00406142963334911\\
377	0.0040614288608608\\
378	0.00406142807491329\\
379	0.00406142727524937\\
380	0.00406142646160414\\
381	0.00406142563370459\\
382	0.00406142479126918\\
383	0.00406142393400737\\
384	0.00406142306161913\\
385	0.00406142217379441\\
386	0.00406142127021255\\
387	0.00406142035054169\\
388	0.00406141941443797\\
389	0.00406141846154484\\
390	0.0040614174914921\\
391	0.00406141650389494\\
392	0.00406141549835268\\
393	0.00406141447444738\\
394	0.00406141343174207\\
395	0.0040614123697786\\
396	0.00406141128807509\\
397	0.00406141018612313\\
398	0.00406140906338551\\
399	0.00406140791929555\\
400	0.00406140675325923\\
401	0.00406140556465838\\
402	0.00406140435284709\\
403	0.00406140311713441\\
404	0.00406140185677566\\
405	0.00406140057098452\\
406	0.00406139925893173\\
407	0.00406139791974453\\
408	0.00406139655250733\\
409	0.00406139515626258\\
410	0.00406139373000972\\
411	0.00406139227269652\\
412	0.00406139078319603\\
413	0.00406138926026895\\
414	0.00406138770253677\\
415	0.00406138610852019\\
416	0.00406138447673179\\
417	0.00406138280557569\\
418	0.00406138109331847\\
419	0.00406137933807158\\
420	0.00406137753777094\\
421	0.00406137569015324\\
422	0.00406137379272801\\
423	0.00406137184274438\\
424	0.00406136983715\\
425	0.00406136777253767\\
426	0.0040613656450696\\
427	0.00406136345035738\\
428	0.00406136118324853\\
429	0.00406135883741312\\
430	0.00406135640450675\\
431	0.00406135387247213\\
432	0.0040613512222096\\
433	0.00406134842152963\\
434	0.00406134541564776\\
435	0.00406134211645094\\
436	0.0040613384023065\\
437	0.00406133415886565\\
438	0.00406132939785058\\
439	0.00406132437530529\\
440	0.00406131924489778\\
441	0.00406131400369967\\
442	0.00406130864870767\\
443	0.00406130317684924\\
444	0.00406129758499008\\
445	0.00406129186994368\\
446	0.00406128602848235\\
447	0.0040612800573483\\
448	0.00406127395325978\\
449	0.00406126771289894\\
450	0.00406126133284752\\
451	0.00406125480938392\\
452	0.00406124813792591\\
453	0.00406124131158086\\
454	0.00406123431746821\\
455	0.00406122712752537\\
456	0.00406121967581232\\
457	0.00406121180344801\\
458	0.0040612031287363\\
459	0.00406119275532929\\
460	0.00406117867059865\\
461	0.00406115669857275\\
462	0.00406111933689257\\
463	0.0040610566265805\\
464	0.00406096569839961\\
465	0.00406087317548331\\
466	0.00406077901137583\\
467	0.00406068315815308\\
468	0.0040605855668307\\
469	0.00406048618571796\\
470	0.00406038496016992\\
471	0.00406028183402784\\
472	0.00406017674864634\\
473	0.00406006964010068\\
474	0.00405996043544122\\
475	0.00405984905393745\\
476	0.00405973542005483\\
477	0.00405961945579985\\
478	0.00405950107808962\\
479	0.00405938019828523\\
480	0.00405925672166008\\
481	0.00405913054678598\\
482	0.00405900156482686\\
483	0.0040588696587283\\
484	0.00405873470228564\\
485	0.0040585965590667\\
486	0.00405845508117609\\
487	0.00405831010784321\\
488	0.00405816146376941\\
489	0.00405800895716853\\
490	0.00405785237739415\\
491	0.00405769149195838\\
492	0.00405752604254285\\
493	0.00405735573912402\\
494	0.00405718025020847\\
495	0.0040569991845618\\
496	0.00405681205392423\\
497	0.00405661819352445\\
498	0.00405641659189319\\
499	0.00405620553753419\\
500	0.00405598193594154\\
501	0.00405574016564575\\
502	0.00405547073960516\\
503	0.00405516067814581\\
504	0.00405480160717274\\
505	0.00405441381211283\\
506	0.00405402099120583\\
507	0.0040536231225162\\
508	0.00405321987985991\\
509	0.00405281039522229\\
510	0.00405239398982295\\
511	0.00405197033797043\\
512	0.00405153907375548\\
513	0.00405109977356717\\
514	0.00405065191342985\\
515	0.00405019475676502\\
516	0.00404972704919735\\
517	0.00404924617980659\\
518	0.00404874588097109\\
519	0.00404820999895577\\
520	0.00404759604850658\\
521	0.00404679381227801\\
522	0.00404553046041244\\
523	0.00404319768801735\\
524	0.00403876878224668\\
525	0.0040321999423493\\
526	0.00402547244576288\\
527	0.00401856983045933\\
528	0.00401147129796997\\
529	0.0040041569952074\\
530	0.00399662529008306\\
531	0.00398888828751701\\
532	0.00398092784527301\\
533	0.00397268828426009\\
534	0.0039640176156458\\
535	0.00395455217310666\\
536	0.00394369284046989\\
537	0.00393062769130563\\
538	0.00391601606445189\\
539	0.00390129681388448\\
540	0.00388646202356258\\
541	0.00387150286326357\\
542	0.00385640915205039\\
543	0.00384116793333786\\
544	0.0038257600908453\\
545	0.00381015356436016\\
546	0.00379428777375626\\
547	0.00377803953381254\\
548	0.00376116179861637\\
549	0.00374325944500222\\
550	0.00372418624906977\\
551	0.00370500254122745\\
552	0.00368548067614745\\
553	0.00366542135653332\\
554	0.00364300836795439\\
555	0.00360768816192235\\
556	0.00352012484187683\\
557	0.0034200905470597\\
558	0.0033114881502614\\
559	0.00318586122956869\\
560	0.00301590666812544\\
561	0.00282709200229717\\
562	0.00263094421539516\\
563	0.00242651276754218\\
564	0.00221179816707534\\
565	0.00198061034947524\\
566	0.00171394633613848\\
567	0.00143955403725653\\
568	0.00115727971134879\\
569	0.000866069879143256\\
570	0.000563740415471681\\
571	0.000244432763086153\\
572	0\\
573	0\\
574	0\\
575	0\\
576	0\\
577	0\\
578	0\\
579	0\\
580	0\\
581	0\\
582	0\\
583	0\\
584	0\\
585	0\\
586	0\\
587	0\\
588	0\\
589	0\\
590	0\\
591	0\\
592	0\\
593	0\\
594	0\\
595	0\\
596	0\\
597	0\\
598	0\\
599	0\\
600	0\\
};
\addplot [color=mycolor12,solid,forget plot]
  table[row sep=crcr]{%
1	0.00510109179991326\\
2	0.00510109179985886\\
3	0.00510109179980354\\
4	0.00510109179974729\\
5	0.0051010917996901\\
6	0.00510109179963195\\
7	0.00510109179957282\\
8	0.00510109179951269\\
9	0.00510109179945155\\
10	0.00510109179938939\\
11	0.00510109179932617\\
12	0.0051010917992619\\
13	0.00510109179919654\\
14	0.00510109179913009\\
15	0.00510109179906252\\
16	0.00510109179899381\\
17	0.00510109179892395\\
18	0.00510109179885291\\
19	0.00510109179878067\\
20	0.00510109179870723\\
21	0.00510109179863254\\
22	0.0051010917985566\\
23	0.00510109179847938\\
24	0.00510109179840087\\
25	0.00510109179832103\\
26	0.00510109179823985\\
27	0.00510109179815731\\
28	0.00510109179807337\\
29	0.00510109179798803\\
30	0.00510109179790124\\
31	0.005101091797813\\
32	0.00510109179772328\\
33	0.00510109179763204\\
34	0.00510109179753927\\
35	0.00510109179744493\\
36	0.00510109179734901\\
37	0.00510109179725148\\
38	0.00510109179715231\\
39	0.00510109179705146\\
40	0.00510109179694892\\
41	0.00510109179684465\\
42	0.00510109179673863\\
43	0.00510109179663082\\
44	0.0051010917965212\\
45	0.00510109179640974\\
46	0.00510109179629639\\
47	0.00510109179618114\\
48	0.00510109179606395\\
49	0.00510109179594478\\
50	0.00510109179582361\\
51	0.00510109179570039\\
52	0.00510109179557511\\
53	0.00510109179544771\\
54	0.00510109179531816\\
55	0.00510109179518643\\
56	0.00510109179505248\\
57	0.00510109179491628\\
58	0.00510109179477778\\
59	0.00510109179463695\\
60	0.00510109179449374\\
61	0.00510109179434812\\
62	0.00510109179420005\\
63	0.00510109179404947\\
64	0.00510109179389636\\
65	0.00510109179374067\\
66	0.00510109179358235\\
67	0.00510109179342137\\
68	0.00510109179325766\\
69	0.0051010917930912\\
70	0.00510109179292192\\
71	0.00510109179274979\\
72	0.00510109179257475\\
73	0.00510109179239676\\
74	0.00510109179221577\\
75	0.00510109179203172\\
76	0.00510109179184456\\
77	0.00510109179165424\\
78	0.0051010917914607\\
79	0.0051010917912639\\
80	0.00510109179106377\\
81	0.00510109179086026\\
82	0.00510109179065331\\
83	0.00510109179044286\\
84	0.00510109179022885\\
85	0.00510109179001123\\
86	0.00510109178978992\\
87	0.00510109178956487\\
88	0.00510109178933601\\
89	0.00510109178910328\\
90	0.00510109178886661\\
91	0.00510109178862593\\
92	0.00510109178838118\\
93	0.00510109178813228\\
94	0.00510109178787917\\
95	0.00510109178762177\\
96	0.00510109178736\\
97	0.0051010917870938\\
98	0.00510109178682309\\
99	0.00510109178654778\\
100	0.00510109178626781\\
101	0.00510109178598308\\
102	0.00510109178569353\\
103	0.00510109178539906\\
104	0.00510109178509958\\
105	0.00510109178479503\\
106	0.0051010917844853\\
107	0.0051010917841703\\
108	0.00510109178384996\\
109	0.00510109178352417\\
110	0.00510109178319283\\
111	0.00510109178285587\\
112	0.00510109178251317\\
113	0.00510109178216463\\
114	0.00510109178181017\\
115	0.00510109178144967\\
116	0.00510109178108302\\
117	0.00510109178071014\\
118	0.00510109178033089\\
119	0.00510109177994519\\
120	0.0051010917795529\\
121	0.00510109177915392\\
122	0.00510109177874814\\
123	0.00510109177833543\\
124	0.00510109177791568\\
125	0.00510109177748875\\
126	0.00510109177705454\\
127	0.0051010917766129\\
128	0.00510109177616372\\
129	0.00510109177570685\\
130	0.00510109177524217\\
131	0.00510109177476954\\
132	0.00510109177428883\\
133	0.00510109177379988\\
134	0.00510109177330256\\
135	0.00510109177279672\\
136	0.00510109177228222\\
137	0.00510109177175889\\
138	0.0051010917712266\\
139	0.00510109177068517\\
140	0.00510109177013446\\
141	0.0051010917695743\\
142	0.00510109176900452\\
143	0.00510109176842496\\
144	0.00510109176783545\\
145	0.00510109176723581\\
146	0.00510109176662587\\
147	0.00510109176600544\\
148	0.00510109176537435\\
149	0.00510109176473241\\
150	0.00510109176407942\\
151	0.0051010917634152\\
152	0.00510109176273955\\
153	0.00510109176205227\\
154	0.00510109176135316\\
155	0.00510109176064201\\
156	0.00510109175991861\\
157	0.00510109175918275\\
158	0.00510109175843421\\
159	0.00510109175767277\\
160	0.00510109175689821\\
161	0.00510109175611029\\
162	0.00510109175530879\\
163	0.00510109175449347\\
164	0.00510109175366408\\
165	0.00510109175282039\\
166	0.00510109175196214\\
167	0.00510109175108909\\
168	0.00510109175020097\\
169	0.00510109174929752\\
170	0.00510109174837848\\
171	0.00510109174744358\\
172	0.00510109174649253\\
173	0.00510109174552507\\
174	0.0051010917445409\\
175	0.00510109174353974\\
176	0.00510109174252128\\
177	0.00510109174148524\\
178	0.00510109174043131\\
179	0.00510109173935916\\
180	0.0051010917382685\\
181	0.00510109173715899\\
182	0.00510109173603032\\
183	0.00510109173488214\\
184	0.00510109173371412\\
185	0.00510109173252591\\
186	0.00510109173131717\\
187	0.00510109173008753\\
188	0.00510109172883664\\
189	0.00510109172756412\\
190	0.0051010917262696\\
191	0.0051010917249527\\
192	0.00510109172361302\\
193	0.00510109172225017\\
194	0.00510109172086375\\
195	0.00510109171945334\\
196	0.00510109171801854\\
197	0.00510109171655891\\
198	0.00510109171507402\\
199	0.00510109171356343\\
200	0.00510109171202669\\
201	0.00510109171046336\\
202	0.00510109170887296\\
203	0.00510109170725502\\
204	0.00510109170560906\\
205	0.0051010917039346\\
206	0.00510109170223113\\
207	0.00510109170049815\\
208	0.00510109169873514\\
209	0.00510109169694158\\
210	0.00510109169511695\\
211	0.00510109169326067\\
212	0.00510109169137222\\
213	0.00510109168945103\\
214	0.00510109168749652\\
215	0.00510109168550811\\
216	0.00510109168348521\\
217	0.00510109168142722\\
218	0.00510109167933351\\
219	0.00510109167720346\\
220	0.00510109167503645\\
221	0.00510109167283181\\
222	0.00510109167058888\\
223	0.00510109166830701\\
224	0.0051010916659855\\
225	0.00510109166362365\\
226	0.00510109166122077\\
227	0.00510109165877613\\
228	0.00510109165628899\\
229	0.00510109165375861\\
230	0.00510109165118423\\
231	0.00510109164856507\\
232	0.00510109164590035\\
233	0.00510109164318926\\
234	0.005101091640431\\
235	0.00510109163762472\\
236	0.00510109163476959\\
237	0.00510109163186473\\
238	0.00510109162890929\\
239	0.00510109162590235\\
240	0.00510109162284302\\
241	0.00510109161973038\\
242	0.00510109161656347\\
243	0.00510109161334134\\
244	0.00510109161006302\\
245	0.00510109160672752\\
246	0.00510109160333382\\
247	0.00510109159988089\\
248	0.00510109159636769\\
249	0.00510109159279315\\
250	0.00510109158915619\\
251	0.0051010915854557\\
252	0.00510109158169055\\
253	0.00510109157785961\\
254	0.0051010915739617\\
255	0.00510109156999565\\
256	0.00510109156596023\\
257	0.00510109156185422\\
258	0.00510109155767638\\
259	0.00510109155342541\\
260	0.00510109154910004\\
261	0.00510109154469893\\
262	0.00510109154022075\\
263	0.00510109153566412\\
264	0.00510109153102764\\
265	0.00510109152630991\\
266	0.00510109152150948\\
267	0.00510109151662488\\
268	0.0051010915116546\\
269	0.00510109150659713\\
270	0.00510109150145092\\
271	0.0051010914962144\\
272	0.00510109149088594\\
273	0.00510109148546392\\
274	0.00510109147994667\\
275	0.00510109147433249\\
276	0.00510109146861967\\
277	0.00510109146280644\\
278	0.00510109145689103\\
279	0.0051010914508716\\
280	0.00510109144474632\\
281	0.00510109143851329\\
282	0.00510109143217059\\
283	0.00510109142571628\\
284	0.00510109141914836\\
285	0.00510109141246483\\
286	0.00510109140566361\\
287	0.00510109139874261\\
288	0.00510109139169971\\
289	0.00510109138453274\\
290	0.00510109137723948\\
291	0.0051010913698177\\
292	0.00510109136226511\\
293	0.00510109135457938\\
294	0.00510109134675816\\
295	0.00510109133879903\\
296	0.00510109133069954\\
297	0.00510109132245721\\
298	0.00510109131406951\\
299	0.00510109130553385\\
300	0.00510109129684761\\
301	0.00510109128800814\\
302	0.00510109127901271\\
303	0.00510109126985857\\
304	0.00510109126054291\\
305	0.00510109125106289\\
306	0.00510109124141559\\
307	0.00510109123159808\\
308	0.00510109122160734\\
309	0.00510109121144035\\
310	0.00510109120109398\\
311	0.0051010911905651\\
312	0.00510109117985049\\
313	0.0051010911689469\\
314	0.00510109115785102\\
315	0.00510109114655949\\
316	0.00510109113506889\\
317	0.00510109112337573\\
318	0.0051010911114765\\
319	0.0051010910993676\\
320	0.00510109108704538\\
321	0.00510109107450615\\
322	0.00510109106174613\\
323	0.00510109104876151\\
324	0.00510109103554841\\
325	0.00510109102210286\\
326	0.00510109100842088\\
327	0.00510109099449839\\
328	0.00510109098033125\\
329	0.00510109096591527\\
330	0.00510109095124619\\
331	0.00510109093631967\\
332	0.00510109092113132\\
333	0.00510109090567667\\
334	0.00510109088995119\\
335	0.00510109087395028\\
336	0.00510109085766926\\
337	0.00510109084110338\\
338	0.00510109082424782\\
339	0.00510109080709769\\
340	0.00510109078964801\\
341	0.00510109077189374\\
342	0.00510109075382975\\
343	0.00510109073545083\\
344	0.00510109071675169\\
345	0.00510109069772694\\
346	0.00510109067837114\\
347	0.00510109065867872\\
348	0.00510109063864405\\
349	0.00510109061826139\\
350	0.00510109059752491\\
351	0.00510109057642869\\
352	0.00510109055496668\\
353	0.00510109053313275\\
354	0.00510109051092066\\
355	0.00510109048832405\\
356	0.00510109046533644\\
357	0.00510109044195123\\
358	0.0051010904181617\\
359	0.005101090393961\\
360	0.00510109036934214\\
361	0.00510109034429798\\
362	0.00510109031882124\\
363	0.00510109029290449\\
364	0.00510109026654014\\
365	0.0051010902397204\\
366	0.00510109021243735\\
367	0.00510109018468284\\
368	0.00510109015644857\\
369	0.00510109012772599\\
370	0.00510109009850639\\
371	0.00510109006878078\\
372	0.00510109003853998\\
373	0.00510109000777454\\
374	0.00510108997647477\\
375	0.00510108994463069\\
376	0.00510108991223205\\
377	0.0051010898792683\\
378	0.00510108984572858\\
379	0.00510108981160168\\
380	0.00510108977687607\\
381	0.00510108974153983\\
382	0.00510108970558068\\
383	0.00510108966898591\\
384	0.0051010896317424\\
385	0.00510108959383656\\
386	0.00510108955525432\\
387	0.0051010895159811\\
388	0.00510108947600176\\
389	0.00510108943530057\\
390	0.00510108939386118\\
391	0.00510108935166651\\
392	0.00510108930869876\\
393	0.00510108926493928\\
394	0.00510108922036845\\
395	0.00510108917496566\\
396	0.00510108912870914\\
397	0.00510108908157593\\
398	0.00510108903354188\\
399	0.00510108898458167\\
400	0.00510108893466886\\
401	0.00510108888377556\\
402	0.00510108883187196\\
403	0.00510108877892617\\
404	0.00510108872490455\\
405	0.00510108866977168\\
406	0.00510108861349032\\
407	0.00510108855602139\\
408	0.00510108849732398\\
409	0.00510108843735507\\
410	0.00510108837606896\\
411	0.00510108831341628\\
412	0.00510108824934293\\
413	0.00510108818378981\\
414	0.00510108811669441\\
415	0.00510108804799222\\
416	0.00510108797761368\\
417	0.00510108790548305\\
418	0.00510108783151769\\
419	0.00510108775562703\\
420	0.00510108767771156\\
421	0.00510108759766138\\
422	0.00510108751535461\\
423	0.00510108743065502\\
424	0.0051010873434088\\
425	0.00510108725343919\\
426	0.00510108716053712\\
427	0.0051010870644438\\
428	0.00510108696481774\\
429	0.00510108686117289\\
430	0.00510108675276765\\
431	0.00510108663842159\\
432	0.00510108651625388\\
433	0.00510108638341271\\
434	0.00510108623605208\\
435	0.00510108607011444\\
436	0.00510108588358932\\
437	0.00510108567976443\\
438	0.00510108546715944\\
439	0.00510108524996865\\
440	0.00510108502806737\\
441	0.0051010848013279\\
442	0.00510108456961976\\
443	0.00510108433281003\\
444	0.00510108409076368\\
445	0.0051010838433438\\
446	0.00510108359041134\\
447	0.00510108333182357\\
448	0.00510108306742894\\
449	0.00510108279705347\\
450	0.00510108252046636\\
451	0.00510108223729676\\
452	0.00510108194683727\\
453	0.00510108164759044\\
454	0.00510108133624716\\
455	0.00510108100545546\\
456	0.00510108063915062\\
457	0.00510108020337091\\
458	0.00510107962988618\\
459	0.00510107879175149\\
460	0.00510107747983606\\
461	0.00510107541719667\\
462	0.00510107239218583\\
463	0.0051010685759856\\
464	0.00510106469226201\\
465	0.00510106073903992\\
466	0.00510105671427971\\
467	0.00510105261587299\\
468	0.00510104844159795\\
469	0.00510104418911902\\
470	0.00510103985600564\\
471	0.00510103543967317\\
472	0.00510103093728759\\
473	0.00510102634570732\\
474	0.00510102166160921\\
475	0.00510101688175904\\
476	0.00510101200277714\\
477	0.00510100702105975\\
478	0.00510100193275785\\
479	0.00510099673375294\\
480	0.00510099141962911\\
481	0.00510098598564094\\
482	0.0051009804266767\\
483	0.00510097473721582\\
484	0.00510096891127982\\
485	0.0051009629423756\\
486	0.00510095682342939\\
487	0.00510095054670774\\
488	0.0051009441037201\\
489	0.00510093748509278\\
490	0.00510093068039336\\
491	0.00510092367786143\\
492	0.00510091646395133\\
493	0.00510090902248744\\
494	0.00510090133302177\\
495	0.00510089336758987\\
496	0.00510088508440327\\
497	0.00510087641614243\\
498	0.00510086725001991\\
499	0.00510085739884259\\
500	0.00510084657245365\\
501	0.00510083438676113\\
502	0.00510082049579685\\
503	0.00510080494095675\\
504	0.0051007885116239\\
505	0.0051007718624104\\
506	0.00510075498557465\\
507	0.00510073786101444\\
508	0.00510072045661552\\
509	0.00510070274822352\\
510	0.00510068472006531\\
511	0.00510066635294463\\
512	0.00510064762101038\\
513	0.00510062848380644\\
514	0.00510060886667179\\
515	0.0051005886129507\\
516	0.00510056736990364\\
517	0.00510054432491002\\
518	0.00510051762369321\\
519	0.0051004831778553\\
520	0.00510043250933198\\
521	0.00510034977287883\\
522	0.00510021062121271\\
523	0.00509999263205191\\
524	0.00509971271547763\\
525	0.00509942494706406\\
526	0.00509912851553759\\
527	0.00509882253688619\\
528	0.00509850628966137\\
529	0.00509817947339278\\
530	0.00509784152438081\\
531	0.00509748999325015\\
532	0.00509711952279453\\
533	0.00509671942295103\\
534	0.00509627145328356\\
535	0.00509575324774495\\
536	0.00509515058158533\\
537	0.00509449543104951\\
538	0.00509383028532832\\
539	0.00509315442498853\\
540	0.00509246701986117\\
541	0.00509176705844467\\
542	0.00509105318802637\\
543	0.00509032339795339\\
544	0.00508957445850129\\
545	0.00508880094436249\\
546	0.00508799382521893\\
547	0.00508713939509688\\
548	0.0050862221275849\\
549	0.00508523810584875\\
550	0.00508420975301684\\
551	0.00508308849968024\\
552	0.00508175972602962\\
553	0.00507990484665965\\
554	0.00507669461183126\\
555	0.00507044038034362\\
556	0.00505922864793289\\
557	0.00504656909230222\\
558	0.00503244725364375\\
559	0.00501578264226783\\
560	0.00499545046704825\\
561	0.00497398626098588\\
562	0.00495230936879711\\
563	0.00493026676404129\\
564	0.00490753314712123\\
565	0.00488359464059053\\
566	0.0048582284635519\\
567	0.00483327164309529\\
568	0.00480872353417642\\
569	0.00478452488381562\\
570	0.00476060339303416\\
571	0.00473706064438208\\
572	0.0047147640350247\\
573	0.00469513866116233\\
574	0.00467754660897937\\
575	0.00465821856935375\\
576	0.00463326745163829\\
577	0.00459144151574509\\
578	0.00450001461999156\\
579	0.00425903716898513\\
580	0.00396782385977364\\
581	0.00364026773511628\\
582	0.00330161831566113\\
583	0.00294712380411371\\
584	0.00256295996406943\\
585	0.0021687264087653\\
586	0.00176496329681716\\
587	0.00134883630364181\\
588	0.000913380900315357\\
589	0.000438890169453977\\
590	0\\
591	0\\
592	0\\
593	0\\
594	0\\
595	0\\
596	0\\
597	0\\
598	0\\
599	0\\
600	0\\
};
\addplot [color=mycolor13,solid,forget plot]
  table[row sep=crcr]{%
1	0\\
2	0\\
3	0\\
4	0\\
5	0\\
6	0\\
7	0\\
8	0\\
9	0\\
10	0\\
11	0\\
12	0\\
13	0\\
14	0\\
15	0\\
16	0\\
17	0\\
18	0\\
19	0\\
20	0\\
21	0\\
22	0\\
23	0\\
24	0\\
25	0\\
26	0\\
27	0\\
28	0\\
29	0\\
30	0\\
31	0\\
32	0\\
33	0\\
34	0\\
35	0\\
36	0\\
37	0\\
38	0\\
39	0\\
40	0\\
41	0\\
42	0\\
43	0\\
44	0\\
45	0\\
46	0\\
47	0\\
48	0\\
49	0\\
50	0\\
51	0\\
52	0\\
53	0\\
54	0\\
55	0\\
56	0\\
57	0\\
58	0\\
59	0\\
60	0\\
61	0\\
62	0\\
63	0\\
64	0\\
65	0\\
66	0\\
67	0\\
68	0\\
69	0\\
70	0\\
71	0\\
72	0\\
73	0\\
74	0\\
75	0\\
76	0\\
77	0\\
78	0\\
79	0\\
80	0\\
81	0\\
82	0\\
83	0\\
84	0\\
85	0\\
86	0\\
87	0\\
88	0\\
89	0\\
90	0\\
91	0\\
92	0\\
93	0\\
94	0\\
95	0\\
96	0\\
97	0\\
98	0\\
99	0\\
100	0\\
101	0\\
102	0\\
103	0\\
104	0\\
105	0\\
106	0\\
107	0\\
108	0\\
109	0\\
110	0\\
111	0\\
112	0\\
113	0\\
114	0\\
115	0\\
116	0\\
117	0\\
118	0\\
119	0\\
120	0\\
121	0\\
122	0\\
123	0\\
124	0\\
125	0\\
126	0\\
127	0\\
128	0\\
129	0\\
130	0\\
131	0\\
132	0\\
133	0\\
134	0\\
135	0\\
136	0\\
137	0\\
138	0\\
139	0\\
140	0\\
141	0\\
142	0\\
143	0\\
144	0\\
145	0\\
146	0\\
147	0\\
148	0\\
149	0\\
150	0\\
151	0\\
152	0\\
153	0\\
154	0\\
155	0\\
156	0\\
157	0\\
158	0\\
159	0\\
160	0\\
161	0\\
162	0\\
163	0\\
164	0\\
165	0\\
166	0\\
167	0\\
168	0\\
169	0\\
170	0\\
171	0\\
172	0\\
173	0\\
174	0\\
175	0\\
176	0\\
177	0\\
178	0\\
179	0\\
180	0\\
181	0\\
182	0\\
183	0\\
184	0\\
185	0\\
186	0\\
187	0\\
188	0\\
189	0\\
190	0\\
191	0\\
192	0\\
193	0\\
194	0\\
195	0\\
196	0\\
197	0\\
198	0\\
199	0\\
200	0\\
201	0\\
202	0\\
203	0\\
204	0\\
205	0\\
206	0\\
207	0\\
208	0\\
209	0\\
210	0\\
211	0\\
212	0\\
213	0\\
214	0\\
215	0\\
216	0\\
217	0\\
218	0\\
219	0\\
220	0\\
221	0\\
222	0\\
223	0\\
224	0\\
225	0\\
226	0\\
227	0\\
228	0\\
229	0\\
230	0\\
231	0\\
232	0\\
233	0\\
234	0\\
235	0\\
236	0\\
237	0\\
238	0\\
239	0\\
240	0\\
241	0\\
242	0\\
243	0\\
244	0\\
245	0\\
246	0\\
247	0\\
248	0\\
249	0\\
250	0\\
251	0\\
252	0\\
253	0\\
254	0\\
255	0\\
256	0\\
257	0\\
258	0\\
259	0\\
260	0\\
261	0\\
262	0\\
263	0\\
264	0\\
265	0\\
266	0\\
267	0\\
268	0\\
269	0\\
270	0\\
271	0\\
272	0\\
273	0\\
274	0\\
275	0\\
276	0\\
277	0\\
278	0\\
279	0\\
280	0\\
281	0\\
282	0\\
283	0\\
284	0\\
285	0\\
286	0\\
287	0\\
288	0\\
289	0\\
290	0\\
291	0\\
292	0\\
293	0\\
294	0\\
295	0\\
296	0\\
297	0\\
298	0\\
299	0\\
300	0\\
301	0\\
302	0\\
303	0\\
304	0\\
305	0\\
306	0\\
307	0\\
308	0\\
309	0\\
310	0\\
311	0\\
312	0\\
313	0\\
314	0\\
315	0\\
316	0\\
317	0\\
318	0\\
319	0\\
320	0\\
321	0\\
322	0\\
323	0\\
324	0\\
325	0\\
326	0\\
327	0\\
328	0\\
329	0\\
330	0\\
331	0\\
332	0\\
333	0\\
334	0\\
335	0\\
336	0\\
337	0\\
338	0\\
339	0\\
340	0\\
341	0\\
342	0\\
343	0\\
344	0\\
345	0\\
346	0\\
347	0\\
348	0\\
349	0\\
350	0\\
351	0\\
352	0\\
353	0\\
354	0\\
355	0\\
356	0\\
357	0\\
358	0\\
359	0\\
360	0\\
361	0\\
362	0\\
363	0\\
364	0\\
365	0\\
366	0\\
367	0\\
368	0\\
369	0\\
370	0\\
371	0\\
372	0\\
373	0\\
374	0\\
375	0\\
376	0\\
377	0\\
378	0\\
379	0\\
380	0\\
381	0\\
382	0\\
383	0\\
384	0\\
385	0\\
386	0\\
387	0\\
388	0\\
389	0\\
390	0\\
391	0\\
392	0\\
393	0\\
394	0\\
395	0\\
396	0\\
397	0\\
398	0\\
399	0\\
400	0\\
401	0\\
402	0\\
403	0\\
404	0\\
405	0\\
406	0\\
407	0\\
408	0\\
409	0\\
410	0\\
411	0\\
412	0\\
413	0\\
414	0\\
415	0\\
416	0\\
417	0\\
418	0\\
419	0\\
420	0\\
421	0\\
422	0\\
423	0\\
424	0\\
425	0\\
426	0\\
427	0\\
428	0\\
429	0\\
430	0\\
431	0\\
432	0\\
433	0\\
434	0\\
435	0\\
436	0\\
437	0\\
438	0\\
439	0\\
440	0\\
441	0\\
442	0\\
443	0\\
444	0\\
445	0\\
446	0\\
447	0\\
448	0\\
449	0\\
450	0\\
451	0\\
452	0\\
453	0\\
454	0\\
455	0\\
456	0\\
457	0\\
458	0\\
459	0\\
460	0\\
461	0\\
462	0\\
463	0\\
464	0\\
465	0\\
466	0\\
467	0\\
468	0\\
469	0\\
470	0\\
471	0\\
472	0\\
473	0\\
474	0\\
475	0\\
476	0\\
477	0\\
478	0\\
479	0\\
480	0\\
481	0\\
482	0\\
483	0\\
484	0\\
485	0\\
486	0\\
487	0\\
488	0\\
489	0\\
490	0\\
491	0\\
492	0\\
493	0\\
494	0\\
495	0\\
496	0\\
497	0\\
498	0\\
499	0\\
500	0\\
501	0\\
502	0\\
503	0\\
504	0\\
505	0\\
506	0\\
507	0\\
508	0\\
509	0\\
510	0\\
511	0\\
512	0\\
513	0\\
514	0\\
515	0\\
516	0\\
517	0\\
518	0\\
519	0\\
520	0\\
521	0\\
522	0\\
523	0\\
524	0\\
525	0\\
526	0\\
527	0\\
528	0\\
529	0\\
530	0\\
531	0\\
532	0\\
533	0\\
534	0\\
535	0\\
536	0\\
537	0\\
538	0\\
539	0\\
540	0\\
541	0\\
542	0\\
543	0\\
544	0\\
545	0\\
546	0\\
547	0\\
548	0\\
549	0\\
550	0\\
551	0\\
552	0\\
553	0\\
554	0\\
555	0\\
556	0\\
557	0\\
558	0\\
559	0\\
560	0\\
561	0\\
562	0\\
563	0\\
564	0\\
565	0\\
566	0\\
567	0\\
568	0\\
569	0\\
570	0\\
571	0\\
572	0\\
573	0\\
574	0\\
575	0\\
576	0\\
577	0\\
578	0\\
579	0\\
580	0\\
581	0\\
582	0\\
583	0\\
584	0\\
585	0\\
586	0\\
587	0\\
588	0\\
589	0\\
590	0\\
591	0\\
592	0\\
593	0\\
594	9.98598075601821e-05\\
595	0.000594592516223747\\
596	0.00167287594548724\\
597	0.00320324344438694\\
598	0.00587935965236376\\
599	0\\
600	0\\
};
\addplot [color=mycolor14,solid,forget plot]
  table[row sep=crcr]{%
1	0\\
2	0\\
3	0\\
4	0\\
5	0\\
6	0\\
7	0\\
8	0\\
9	0\\
10	0\\
11	0\\
12	0\\
13	0\\
14	0\\
15	0\\
16	0\\
17	0\\
18	0\\
19	0\\
20	0\\
21	0\\
22	0\\
23	0\\
24	0\\
25	0\\
26	0\\
27	0\\
28	0\\
29	0\\
30	0\\
31	0\\
32	0\\
33	0\\
34	0\\
35	0\\
36	0\\
37	0\\
38	0\\
39	0\\
40	0\\
41	0\\
42	0\\
43	0\\
44	0\\
45	0\\
46	0\\
47	0\\
48	0\\
49	0\\
50	0\\
51	0\\
52	0\\
53	0\\
54	0\\
55	0\\
56	0\\
57	0\\
58	0\\
59	0\\
60	0\\
61	0\\
62	0\\
63	0\\
64	0\\
65	0\\
66	0\\
67	0\\
68	0\\
69	0\\
70	0\\
71	0\\
72	0\\
73	0\\
74	0\\
75	0\\
76	0\\
77	0\\
78	0\\
79	0\\
80	0\\
81	0\\
82	0\\
83	0\\
84	0\\
85	0\\
86	0\\
87	0\\
88	0\\
89	0\\
90	0\\
91	0\\
92	0\\
93	0\\
94	0\\
95	0\\
96	0\\
97	0\\
98	0\\
99	0\\
100	0\\
101	0\\
102	0\\
103	0\\
104	0\\
105	0\\
106	0\\
107	0\\
108	0\\
109	0\\
110	0\\
111	0\\
112	0\\
113	0\\
114	0\\
115	0\\
116	0\\
117	0\\
118	0\\
119	0\\
120	0\\
121	0\\
122	0\\
123	0\\
124	0\\
125	0\\
126	0\\
127	0\\
128	0\\
129	0\\
130	0\\
131	0\\
132	0\\
133	0\\
134	0\\
135	0\\
136	0\\
137	0\\
138	0\\
139	0\\
140	0\\
141	0\\
142	0\\
143	0\\
144	0\\
145	0\\
146	0\\
147	0\\
148	0\\
149	0\\
150	0\\
151	0\\
152	0\\
153	0\\
154	0\\
155	0\\
156	0\\
157	0\\
158	0\\
159	0\\
160	0\\
161	0\\
162	0\\
163	0\\
164	0\\
165	0\\
166	0\\
167	0\\
168	0\\
169	0\\
170	0\\
171	0\\
172	0\\
173	0\\
174	0\\
175	0\\
176	0\\
177	0\\
178	0\\
179	0\\
180	0\\
181	0\\
182	0\\
183	0\\
184	0\\
185	0\\
186	0\\
187	0\\
188	0\\
189	0\\
190	0\\
191	0\\
192	0\\
193	0\\
194	0\\
195	0\\
196	0\\
197	0\\
198	0\\
199	0\\
200	0\\
201	0\\
202	0\\
203	0\\
204	0\\
205	0\\
206	0\\
207	0\\
208	0\\
209	0\\
210	0\\
211	0\\
212	0\\
213	0\\
214	0\\
215	0\\
216	0\\
217	0\\
218	0\\
219	0\\
220	0\\
221	0\\
222	0\\
223	0\\
224	0\\
225	0\\
226	0\\
227	0\\
228	0\\
229	0\\
230	0\\
231	0\\
232	0\\
233	0\\
234	0\\
235	0\\
236	0\\
237	0\\
238	0\\
239	0\\
240	0\\
241	0\\
242	0\\
243	0\\
244	0\\
245	0\\
246	0\\
247	0\\
248	0\\
249	0\\
250	0\\
251	0\\
252	0\\
253	0\\
254	0\\
255	0\\
256	0\\
257	0\\
258	0\\
259	0\\
260	0\\
261	0\\
262	0\\
263	0\\
264	0\\
265	0\\
266	0\\
267	0\\
268	0\\
269	0\\
270	0\\
271	0\\
272	0\\
273	0\\
274	0\\
275	0\\
276	0\\
277	0\\
278	0\\
279	0\\
280	0\\
281	0\\
282	0\\
283	0\\
284	0\\
285	0\\
286	0\\
287	0\\
288	0\\
289	0\\
290	0\\
291	0\\
292	0\\
293	0\\
294	0\\
295	0\\
296	0\\
297	0\\
298	0\\
299	0\\
300	0\\
301	0\\
302	0\\
303	0\\
304	0\\
305	0\\
306	0\\
307	0\\
308	0\\
309	0\\
310	0\\
311	0\\
312	0\\
313	0\\
314	0\\
315	0\\
316	0\\
317	0\\
318	0\\
319	0\\
320	0\\
321	0\\
322	0\\
323	0\\
324	0\\
325	0\\
326	0\\
327	0\\
328	0\\
329	0\\
330	0\\
331	0\\
332	0\\
333	0\\
334	0\\
335	0\\
336	0\\
337	0\\
338	0\\
339	0\\
340	0\\
341	0\\
342	0\\
343	0\\
344	0\\
345	0\\
346	0\\
347	0\\
348	0\\
349	0\\
350	0\\
351	0\\
352	0\\
353	0\\
354	0\\
355	0\\
356	0\\
357	0\\
358	0\\
359	0\\
360	0\\
361	0\\
362	0\\
363	0\\
364	0\\
365	0\\
366	0\\
367	0\\
368	0\\
369	0\\
370	0\\
371	0\\
372	0\\
373	0\\
374	0\\
375	0\\
376	0\\
377	0\\
378	0\\
379	0\\
380	0\\
381	0\\
382	0\\
383	0\\
384	0\\
385	0\\
386	0\\
387	0\\
388	0\\
389	0\\
390	0\\
391	0\\
392	0\\
393	0\\
394	0\\
395	0\\
396	0\\
397	0\\
398	0\\
399	0\\
400	0\\
401	0\\
402	0\\
403	0\\
404	0\\
405	0\\
406	0\\
407	0\\
408	0\\
409	0\\
410	0\\
411	0\\
412	0\\
413	0\\
414	0\\
415	0\\
416	0\\
417	0\\
418	0\\
419	0\\
420	0\\
421	0\\
422	0\\
423	0\\
424	0\\
425	0\\
426	0\\
427	0\\
428	0\\
429	0\\
430	0\\
431	0\\
432	0\\
433	0\\
434	0\\
435	0\\
436	0\\
437	0\\
438	0\\
439	0\\
440	0\\
441	0\\
442	0\\
443	0\\
444	0\\
445	0\\
446	0\\
447	0\\
448	0\\
449	0\\
450	0\\
451	0\\
452	0\\
453	0\\
454	0\\
455	0\\
456	0\\
457	0\\
458	0\\
459	0\\
460	0\\
461	0\\
462	0\\
463	0\\
464	0\\
465	0\\
466	0\\
467	0\\
468	0\\
469	0\\
470	0\\
471	0\\
472	0\\
473	0\\
474	0\\
475	0\\
476	0\\
477	0\\
478	0\\
479	0\\
480	0\\
481	0\\
482	0\\
483	0\\
484	0\\
485	0\\
486	0\\
487	0\\
488	0\\
489	0\\
490	0\\
491	0\\
492	0\\
493	0\\
494	0\\
495	0\\
496	0\\
497	0\\
498	0\\
499	0\\
500	0\\
501	0\\
502	0\\
503	0\\
504	0\\
505	0\\
506	0\\
507	0\\
508	0\\
509	0\\
510	0\\
511	0\\
512	0\\
513	0\\
514	0\\
515	0\\
516	0\\
517	0\\
518	0\\
519	0\\
520	0\\
521	0\\
522	0\\
523	0\\
524	0\\
525	0\\
526	0\\
527	0\\
528	0\\
529	0\\
530	0\\
531	0\\
532	0\\
533	0\\
534	0\\
535	0\\
536	0\\
537	0\\
538	0\\
539	0\\
540	0\\
541	0\\
542	0\\
543	0\\
544	0\\
545	0\\
546	0\\
547	0\\
548	0\\
549	0\\
550	0\\
551	0\\
552	0\\
553	0\\
554	0\\
555	0\\
556	0\\
557	0\\
558	0\\
559	0\\
560	0\\
561	0\\
562	0\\
563	0\\
564	0\\
565	0\\
566	0\\
567	0\\
568	0\\
569	0\\
570	0\\
571	0\\
572	0\\
573	0\\
574	0\\
575	0\\
576	0\\
577	0\\
578	0\\
579	0\\
580	0\\
581	0\\
582	0\\
583	0\\
584	0\\
585	0\\
586	0.000167079632204449\\
587	0.000456071715075539\\
588	0.000759434260115618\\
589	0.000951398265291226\\
590	0.00114744334461536\\
591	0.0013504417486169\\
592	0.00161095250100881\\
593	0.00219885314480476\\
594	0.00282841440239993\\
595	0.00353909829286836\\
596	0.00399657994176672\\
597	0.0047545345923529\\
598	0.00632942537858856\\
599	0\\
600	0\\
};
\addplot [color=mycolor15,solid,forget plot]
  table[row sep=crcr]{%
1	0\\
2	0\\
3	0\\
4	0\\
5	0\\
6	0\\
7	0\\
8	0\\
9	0\\
10	0\\
11	0\\
12	0\\
13	0\\
14	0\\
15	0\\
16	0\\
17	0\\
18	0\\
19	0\\
20	0\\
21	0\\
22	0\\
23	0\\
24	0\\
25	0\\
26	0\\
27	0\\
28	0\\
29	0\\
30	0\\
31	0\\
32	0\\
33	0\\
34	0\\
35	0\\
36	0\\
37	0\\
38	0\\
39	0\\
40	0\\
41	0\\
42	0\\
43	0\\
44	0\\
45	0\\
46	0\\
47	0\\
48	0\\
49	0\\
50	0\\
51	0\\
52	0\\
53	0\\
54	0\\
55	0\\
56	0\\
57	0\\
58	0\\
59	0\\
60	0\\
61	0\\
62	0\\
63	0\\
64	0\\
65	0\\
66	0\\
67	0\\
68	0\\
69	0\\
70	0\\
71	0\\
72	0\\
73	0\\
74	0\\
75	0\\
76	0\\
77	0\\
78	0\\
79	0\\
80	0\\
81	0\\
82	0\\
83	0\\
84	0\\
85	0\\
86	0\\
87	0\\
88	0\\
89	0\\
90	0\\
91	0\\
92	0\\
93	0\\
94	0\\
95	0\\
96	0\\
97	0\\
98	0\\
99	0\\
100	0\\
101	0\\
102	0\\
103	0\\
104	0\\
105	0\\
106	0\\
107	0\\
108	0\\
109	0\\
110	0\\
111	0\\
112	0\\
113	0\\
114	0\\
115	0\\
116	0\\
117	0\\
118	0\\
119	0\\
120	0\\
121	0\\
122	0\\
123	0\\
124	0\\
125	0\\
126	0\\
127	0\\
128	0\\
129	0\\
130	0\\
131	0\\
132	0\\
133	0\\
134	0\\
135	0\\
136	0\\
137	0\\
138	0\\
139	0\\
140	0\\
141	0\\
142	0\\
143	0\\
144	0\\
145	0\\
146	0\\
147	0\\
148	0\\
149	0\\
150	0\\
151	0\\
152	0\\
153	0\\
154	0\\
155	0\\
156	0\\
157	0\\
158	0\\
159	0\\
160	0\\
161	0\\
162	0\\
163	0\\
164	0\\
165	0\\
166	0\\
167	0\\
168	0\\
169	0\\
170	0\\
171	0\\
172	0\\
173	0\\
174	0\\
175	0\\
176	0\\
177	0\\
178	0\\
179	0\\
180	0\\
181	0\\
182	0\\
183	0\\
184	0\\
185	0\\
186	0\\
187	0\\
188	0\\
189	0\\
190	0\\
191	0\\
192	0\\
193	0\\
194	0\\
195	0\\
196	0\\
197	0\\
198	0\\
199	0\\
200	0\\
201	0\\
202	0\\
203	0\\
204	0\\
205	0\\
206	0\\
207	0\\
208	0\\
209	0\\
210	0\\
211	0\\
212	0\\
213	0\\
214	0\\
215	0\\
216	0\\
217	0\\
218	0\\
219	0\\
220	0\\
221	0\\
222	0\\
223	0\\
224	0\\
225	0\\
226	0\\
227	0\\
228	0\\
229	0\\
230	0\\
231	0\\
232	0\\
233	0\\
234	0\\
235	0\\
236	0\\
237	0\\
238	0\\
239	0\\
240	0\\
241	0\\
242	0\\
243	0\\
244	0\\
245	0\\
246	0\\
247	0\\
248	0\\
249	0\\
250	0\\
251	0\\
252	0\\
253	0\\
254	0\\
255	0\\
256	0\\
257	0\\
258	0\\
259	0\\
260	0\\
261	0\\
262	0\\
263	0\\
264	0\\
265	0\\
266	0\\
267	0\\
268	0\\
269	0\\
270	0\\
271	0\\
272	0\\
273	0\\
274	0\\
275	0\\
276	0\\
277	0\\
278	0\\
279	0\\
280	0\\
281	0\\
282	0\\
283	0\\
284	0\\
285	0\\
286	0\\
287	0\\
288	0\\
289	0\\
290	0\\
291	0\\
292	0\\
293	0\\
294	0\\
295	0\\
296	0\\
297	0\\
298	0\\
299	0\\
300	0\\
301	0\\
302	0\\
303	0\\
304	0\\
305	0\\
306	0\\
307	0\\
308	0\\
309	0\\
310	0\\
311	0\\
312	0\\
313	0\\
314	0\\
315	0\\
316	0\\
317	0\\
318	0\\
319	0\\
320	0\\
321	0\\
322	0\\
323	0\\
324	0\\
325	0\\
326	0\\
327	0\\
328	0\\
329	0\\
330	0\\
331	0\\
332	0\\
333	0\\
334	0\\
335	0\\
336	0\\
337	0\\
338	0\\
339	0\\
340	0\\
341	0\\
342	0\\
343	0\\
344	0\\
345	0\\
346	0\\
347	0\\
348	0\\
349	0\\
350	0\\
351	0\\
352	0\\
353	0\\
354	0\\
355	0\\
356	0\\
357	0\\
358	0\\
359	0\\
360	0\\
361	0\\
362	0\\
363	0\\
364	0\\
365	0\\
366	0\\
367	0\\
368	0\\
369	0\\
370	0\\
371	0\\
372	0\\
373	0\\
374	0\\
375	0\\
376	0\\
377	0\\
378	0\\
379	0\\
380	0\\
381	0\\
382	0\\
383	0\\
384	0\\
385	0\\
386	0\\
387	0\\
388	0\\
389	0\\
390	0\\
391	0\\
392	0\\
393	0\\
394	0\\
395	0\\
396	0\\
397	0\\
398	0\\
399	0\\
400	0\\
401	0\\
402	0\\
403	0\\
404	0\\
405	0\\
406	0\\
407	0\\
408	0\\
409	0\\
410	0\\
411	0\\
412	0\\
413	0\\
414	0\\
415	0\\
416	0\\
417	0\\
418	0\\
419	0\\
420	0\\
421	0\\
422	0\\
423	0\\
424	0\\
425	0\\
426	0\\
427	0\\
428	0\\
429	0\\
430	0\\
431	0\\
432	0\\
433	0\\
434	0\\
435	0\\
436	0\\
437	0\\
438	0\\
439	0\\
440	0\\
441	0\\
442	0\\
443	0\\
444	0\\
445	0\\
446	0\\
447	0\\
448	0\\
449	0\\
450	0\\
451	0\\
452	0\\
453	0\\
454	0\\
455	0\\
456	0\\
457	0\\
458	0\\
459	0\\
460	0\\
461	0\\
462	0\\
463	0\\
464	0\\
465	0\\
466	0\\
467	0\\
468	0\\
469	0\\
470	0\\
471	0\\
472	0\\
473	0\\
474	0\\
475	0\\
476	0\\
477	0\\
478	0\\
479	0\\
480	0\\
481	0\\
482	0\\
483	0\\
484	0\\
485	0\\
486	0\\
487	0\\
488	0\\
489	0\\
490	0\\
491	0\\
492	0\\
493	0\\
494	0\\
495	0\\
496	0\\
497	0\\
498	0\\
499	0\\
500	0\\
501	0\\
502	0\\
503	0\\
504	0\\
505	0\\
506	0\\
507	0\\
508	0\\
509	0\\
510	0\\
511	0\\
512	0\\
513	0\\
514	0\\
515	0\\
516	0\\
517	0\\
518	0\\
519	0\\
520	0\\
521	0\\
522	0\\
523	0\\
524	0\\
525	0\\
526	0\\
527	0\\
528	0\\
529	0\\
530	0\\
531	0\\
532	0\\
533	0\\
534	0\\
535	0\\
536	0\\
537	0\\
538	0\\
539	0\\
540	0\\
541	0\\
542	0\\
543	0\\
544	0\\
545	0\\
546	0\\
547	0\\
548	0\\
549	0\\
550	0\\
551	0\\
552	0\\
553	0\\
554	0\\
555	0\\
556	0\\
557	0\\
558	0\\
559	0\\
560	0\\
561	0\\
562	0\\
563	0\\
564	0\\
565	0\\
566	0\\
567	0\\
568	0\\
569	0\\
570	0\\
571	0\\
572	0\\
573	0\\
574	0\\
575	0\\
576	0\\
577	0\\
578	0\\
579	0.000143936771195565\\
580	0.000398036759884528\\
581	0.00066092679016297\\
582	0.000819713415261157\\
583	0.00096387575301797\\
584	0.00110653927940749\\
585	0.00124960147875673\\
586	0.00139021956195936\\
587	0.00152910028465296\\
588	0.00165893564622842\\
589	0.00185748867593825\\
590	0.00235311191256397\\
591	0.00285266479926953\\
592	0.00331714209625941\\
593	0.00350129544269832\\
594	0.00369876633599398\\
595	0.00393138688805932\\
596	0.00424489192197758\\
597	0.0048700418989439\\
598	0.00632942537858856\\
599	0\\
600	0\\
};
\addplot [color=mycolor16,solid,forget plot]
  table[row sep=crcr]{%
1	0\\
2	0\\
3	0\\
4	0\\
5	0\\
6	0\\
7	0\\
8	0\\
9	0\\
10	0\\
11	0\\
12	0\\
13	0\\
14	0\\
15	0\\
16	0\\
17	0\\
18	0\\
19	0\\
20	0\\
21	0\\
22	0\\
23	0\\
24	0\\
25	0\\
26	0\\
27	0\\
28	0\\
29	0\\
30	0\\
31	0\\
32	0\\
33	0\\
34	0\\
35	0\\
36	0\\
37	0\\
38	0\\
39	0\\
40	0\\
41	0\\
42	0\\
43	0\\
44	0\\
45	0\\
46	0\\
47	0\\
48	0\\
49	0\\
50	0\\
51	0\\
52	0\\
53	0\\
54	0\\
55	0\\
56	0\\
57	0\\
58	0\\
59	0\\
60	0\\
61	0\\
62	0\\
63	0\\
64	0\\
65	0\\
66	0\\
67	0\\
68	0\\
69	0\\
70	0\\
71	0\\
72	0\\
73	0\\
74	0\\
75	0\\
76	0\\
77	0\\
78	0\\
79	0\\
80	0\\
81	0\\
82	0\\
83	0\\
84	0\\
85	0\\
86	0\\
87	0\\
88	0\\
89	0\\
90	0\\
91	0\\
92	0\\
93	0\\
94	0\\
95	0\\
96	0\\
97	0\\
98	0\\
99	0\\
100	0\\
101	0\\
102	0\\
103	0\\
104	0\\
105	0\\
106	0\\
107	0\\
108	0\\
109	0\\
110	0\\
111	0\\
112	0\\
113	0\\
114	0\\
115	0\\
116	0\\
117	0\\
118	0\\
119	0\\
120	0\\
121	0\\
122	0\\
123	0\\
124	0\\
125	0\\
126	0\\
127	0\\
128	0\\
129	0\\
130	0\\
131	0\\
132	0\\
133	0\\
134	0\\
135	0\\
136	0\\
137	0\\
138	0\\
139	0\\
140	0\\
141	0\\
142	0\\
143	0\\
144	0\\
145	0\\
146	0\\
147	0\\
148	0\\
149	0\\
150	0\\
151	0\\
152	0\\
153	0\\
154	0\\
155	0\\
156	0\\
157	0\\
158	0\\
159	0\\
160	0\\
161	0\\
162	0\\
163	0\\
164	0\\
165	0\\
166	0\\
167	0\\
168	0\\
169	0\\
170	0\\
171	0\\
172	0\\
173	0\\
174	0\\
175	0\\
176	0\\
177	0\\
178	0\\
179	0\\
180	0\\
181	0\\
182	0\\
183	0\\
184	0\\
185	0\\
186	0\\
187	0\\
188	0\\
189	0\\
190	0\\
191	0\\
192	0\\
193	0\\
194	0\\
195	0\\
196	0\\
197	0\\
198	0\\
199	0\\
200	0\\
201	0\\
202	0\\
203	0\\
204	0\\
205	0\\
206	0\\
207	0\\
208	0\\
209	0\\
210	0\\
211	0\\
212	0\\
213	0\\
214	0\\
215	0\\
216	0\\
217	0\\
218	0\\
219	0\\
220	0\\
221	0\\
222	0\\
223	0\\
224	0\\
225	0\\
226	0\\
227	0\\
228	0\\
229	0\\
230	0\\
231	0\\
232	0\\
233	0\\
234	0\\
235	0\\
236	0\\
237	0\\
238	0\\
239	0\\
240	0\\
241	0\\
242	0\\
243	0\\
244	0\\
245	0\\
246	0\\
247	0\\
248	0\\
249	0\\
250	0\\
251	0\\
252	0\\
253	0\\
254	0\\
255	0\\
256	0\\
257	0\\
258	0\\
259	0\\
260	0\\
261	0\\
262	0\\
263	0\\
264	0\\
265	0\\
266	0\\
267	0\\
268	0\\
269	0\\
270	0\\
271	0\\
272	0\\
273	0\\
274	0\\
275	0\\
276	0\\
277	0\\
278	0\\
279	0\\
280	0\\
281	0\\
282	0\\
283	0\\
284	0\\
285	0\\
286	0\\
287	0\\
288	0\\
289	0\\
290	0\\
291	0\\
292	0\\
293	0\\
294	0\\
295	0\\
296	0\\
297	0\\
298	0\\
299	0\\
300	0\\
301	0\\
302	0\\
303	0\\
304	0\\
305	0\\
306	0\\
307	0\\
308	0\\
309	0\\
310	0\\
311	0\\
312	0\\
313	0\\
314	0\\
315	0\\
316	0\\
317	0\\
318	0\\
319	0\\
320	0\\
321	0\\
322	0\\
323	0\\
324	0\\
325	0\\
326	0\\
327	0\\
328	0\\
329	0\\
330	0\\
331	0\\
332	0\\
333	0\\
334	0\\
335	0\\
336	0\\
337	0\\
338	0\\
339	0\\
340	0\\
341	0\\
342	0\\
343	0\\
344	0\\
345	0\\
346	0\\
347	0\\
348	0\\
349	0\\
350	0\\
351	0\\
352	0\\
353	0\\
354	0\\
355	0\\
356	0\\
357	0\\
358	0\\
359	0\\
360	0\\
361	0\\
362	0\\
363	0\\
364	0\\
365	0\\
366	0\\
367	0\\
368	0\\
369	0\\
370	0\\
371	0\\
372	0\\
373	0\\
374	0\\
375	0\\
376	0\\
377	0\\
378	0\\
379	0\\
380	0\\
381	0\\
382	0\\
383	0\\
384	0\\
385	0\\
386	0\\
387	0\\
388	0\\
389	0\\
390	0\\
391	0\\
392	0\\
393	0\\
394	0\\
395	0\\
396	0\\
397	0\\
398	0\\
399	0\\
400	0\\
401	0\\
402	0\\
403	0\\
404	0\\
405	0\\
406	0\\
407	0\\
408	0\\
409	0\\
410	0\\
411	0\\
412	0\\
413	0\\
414	0\\
415	0\\
416	0\\
417	0\\
418	0\\
419	0\\
420	0\\
421	0\\
422	0\\
423	0\\
424	0\\
425	0\\
426	0\\
427	0\\
428	0\\
429	0\\
430	0\\
431	0\\
432	0\\
433	0\\
434	0\\
435	0\\
436	0\\
437	0\\
438	0\\
439	0\\
440	0\\
441	0\\
442	0\\
443	0\\
444	0\\
445	0\\
446	0\\
447	0\\
448	0\\
449	0\\
450	0\\
451	0\\
452	0\\
453	0\\
454	0\\
455	0\\
456	0\\
457	0\\
458	0\\
459	0\\
460	0\\
461	0\\
462	0\\
463	0\\
464	0\\
465	0\\
466	0\\
467	0\\
468	0\\
469	0\\
470	0\\
471	0\\
472	0\\
473	0\\
474	0\\
475	0\\
476	0\\
477	0\\
478	0\\
479	0\\
480	0\\
481	0\\
482	0\\
483	0\\
484	0\\
485	0\\
486	0\\
487	0\\
488	0\\
489	0\\
490	0\\
491	0\\
492	0\\
493	0\\
494	0\\
495	0\\
496	0\\
497	0\\
498	0\\
499	0\\
500	0\\
501	0\\
502	0\\
503	0\\
504	0\\
505	0\\
506	0\\
507	0\\
508	0\\
509	0\\
510	0\\
511	0\\
512	0\\
513	0\\
514	0\\
515	0\\
516	0\\
517	0\\
518	0\\
519	0\\
520	0\\
521	0\\
522	0\\
523	0\\
524	0\\
525	0\\
526	0\\
527	0\\
528	0\\
529	0\\
530	0\\
531	0\\
532	0\\
533	0\\
534	0\\
535	0\\
536	0\\
537	0\\
538	0\\
539	0\\
540	0\\
541	0\\
542	0\\
543	0\\
544	0\\
545	0\\
546	0\\
547	0\\
548	0\\
549	0\\
550	0\\
551	0\\
552	0\\
553	0\\
554	0\\
555	0\\
556	0\\
557	0\\
558	0\\
559	0\\
560	0\\
561	0\\
562	0\\
563	0\\
564	0\\
565	0\\
566	0\\
567	0\\
568	0\\
569	0\\
570	0\\
571	0\\
572	0\\
573	0.000161696296701996\\
574	0.000393713864803922\\
575	0.000598899566224693\\
576	0.000720797071750311\\
577	0.000841506955469324\\
578	0.000959997374964595\\
579	0.00107594147230983\\
580	0.00118817755523261\\
581	0.00129194223306109\\
582	0.00138617795658753\\
583	0.00147953228162457\\
584	0.00157379580842985\\
585	0.00166431968199408\\
586	0.00175199502013113\\
587	0.00222288556602185\\
588	0.00270274779694502\\
589	0.00311897222604471\\
590	0.00326220003215725\\
591	0.00340447372974604\\
592	0.00353900409992704\\
593	0.00365273120413177\\
594	0.00378238158822432\\
595	0.0039567480325281\\
596	0.00425297862742962\\
597	0.0048700418989439\\
598	0.00632942537858856\\
599	0\\
600	0\\
};
\addplot [color=mycolor17,solid,forget plot]
  table[row sep=crcr]{%
1	0\\
2	0\\
3	0\\
4	0\\
5	0\\
6	0\\
7	0\\
8	0\\
9	0\\
10	0\\
11	0\\
12	0\\
13	0\\
14	0\\
15	0\\
16	0\\
17	0\\
18	0\\
19	0\\
20	0\\
21	0\\
22	0\\
23	0\\
24	0\\
25	0\\
26	0\\
27	0\\
28	0\\
29	0\\
30	0\\
31	0\\
32	0\\
33	0\\
34	0\\
35	0\\
36	0\\
37	0\\
38	0\\
39	0\\
40	0\\
41	0\\
42	0\\
43	0\\
44	0\\
45	0\\
46	0\\
47	0\\
48	0\\
49	0\\
50	0\\
51	0\\
52	0\\
53	0\\
54	0\\
55	0\\
56	0\\
57	0\\
58	0\\
59	0\\
60	0\\
61	0\\
62	0\\
63	0\\
64	0\\
65	0\\
66	0\\
67	0\\
68	0\\
69	0\\
70	0\\
71	0\\
72	0\\
73	0\\
74	0\\
75	0\\
76	0\\
77	0\\
78	0\\
79	0\\
80	0\\
81	0\\
82	0\\
83	0\\
84	0\\
85	0\\
86	0\\
87	0\\
88	0\\
89	0\\
90	0\\
91	0\\
92	0\\
93	0\\
94	0\\
95	0\\
96	0\\
97	0\\
98	0\\
99	0\\
100	0\\
101	0\\
102	0\\
103	0\\
104	0\\
105	0\\
106	0\\
107	0\\
108	0\\
109	0\\
110	0\\
111	0\\
112	0\\
113	0\\
114	0\\
115	0\\
116	0\\
117	0\\
118	0\\
119	0\\
120	0\\
121	0\\
122	0\\
123	0\\
124	0\\
125	0\\
126	0\\
127	0\\
128	0\\
129	0\\
130	0\\
131	0\\
132	0\\
133	0\\
134	0\\
135	0\\
136	0\\
137	0\\
138	0\\
139	0\\
140	0\\
141	0\\
142	0\\
143	0\\
144	0\\
145	0\\
146	0\\
147	0\\
148	0\\
149	0\\
150	0\\
151	0\\
152	0\\
153	0\\
154	0\\
155	0\\
156	0\\
157	0\\
158	0\\
159	0\\
160	0\\
161	0\\
162	0\\
163	0\\
164	0\\
165	0\\
166	0\\
167	0\\
168	0\\
169	0\\
170	0\\
171	0\\
172	0\\
173	0\\
174	0\\
175	0\\
176	0\\
177	0\\
178	0\\
179	0\\
180	0\\
181	0\\
182	0\\
183	0\\
184	0\\
185	0\\
186	0\\
187	0\\
188	0\\
189	0\\
190	0\\
191	0\\
192	0\\
193	0\\
194	0\\
195	0\\
196	0\\
197	0\\
198	0\\
199	0\\
200	0\\
201	0\\
202	0\\
203	0\\
204	0\\
205	0\\
206	0\\
207	0\\
208	0\\
209	0\\
210	0\\
211	0\\
212	0\\
213	0\\
214	0\\
215	0\\
216	0\\
217	0\\
218	0\\
219	0\\
220	0\\
221	0\\
222	0\\
223	0\\
224	0\\
225	0\\
226	0\\
227	0\\
228	0\\
229	0\\
230	0\\
231	0\\
232	0\\
233	0\\
234	0\\
235	0\\
236	0\\
237	0\\
238	0\\
239	0\\
240	0\\
241	0\\
242	0\\
243	0\\
244	0\\
245	0\\
246	0\\
247	0\\
248	0\\
249	0\\
250	0\\
251	0\\
252	0\\
253	0\\
254	0\\
255	0\\
256	0\\
257	0\\
258	0\\
259	0\\
260	0\\
261	0\\
262	0\\
263	0\\
264	0\\
265	0\\
266	0\\
267	0\\
268	0\\
269	0\\
270	0\\
271	0\\
272	0\\
273	0\\
274	0\\
275	0\\
276	0\\
277	0\\
278	0\\
279	0\\
280	0\\
281	0\\
282	0\\
283	0\\
284	0\\
285	0\\
286	0\\
287	0\\
288	0\\
289	0\\
290	0\\
291	0\\
292	0\\
293	0\\
294	0\\
295	0\\
296	0\\
297	0\\
298	0\\
299	0\\
300	0\\
301	0\\
302	0\\
303	0\\
304	0\\
305	0\\
306	0\\
307	0\\
308	0\\
309	0\\
310	0\\
311	0\\
312	0\\
313	0\\
314	0\\
315	0\\
316	0\\
317	0\\
318	0\\
319	0\\
320	0\\
321	0\\
322	0\\
323	0\\
324	0\\
325	0\\
326	0\\
327	0\\
328	0\\
329	0\\
330	0\\
331	0\\
332	0\\
333	0\\
334	0\\
335	0\\
336	0\\
337	0\\
338	0\\
339	0\\
340	0\\
341	0\\
342	0\\
343	0\\
344	0\\
345	0\\
346	0\\
347	0\\
348	0\\
349	0\\
350	0\\
351	0\\
352	0\\
353	0\\
354	0\\
355	0\\
356	0\\
357	0\\
358	0\\
359	0\\
360	0\\
361	0\\
362	0\\
363	0\\
364	0\\
365	0\\
366	0\\
367	0\\
368	0\\
369	0\\
370	0\\
371	0\\
372	0\\
373	0\\
374	0\\
375	0\\
376	0\\
377	0\\
378	0\\
379	0\\
380	0\\
381	0\\
382	0\\
383	0\\
384	0\\
385	0\\
386	0\\
387	0\\
388	0\\
389	0\\
390	0\\
391	0\\
392	0\\
393	0\\
394	0\\
395	0\\
396	0\\
397	0\\
398	0\\
399	0\\
400	0\\
401	0\\
402	0\\
403	0\\
404	0\\
405	0\\
406	0\\
407	0\\
408	0\\
409	0\\
410	0\\
411	0\\
412	0\\
413	0\\
414	0\\
415	0\\
416	0\\
417	0\\
418	0\\
419	0\\
420	0\\
421	0\\
422	0\\
423	0\\
424	0\\
425	0\\
426	0\\
427	0\\
428	0\\
429	0\\
430	0\\
431	0\\
432	0\\
433	0\\
434	0\\
435	0\\
436	0\\
437	0\\
438	0\\
439	0\\
440	0\\
441	0\\
442	0\\
443	0\\
444	0\\
445	0\\
446	0\\
447	0\\
448	0\\
449	0\\
450	0\\
451	0\\
452	0\\
453	0\\
454	0\\
455	0\\
456	0\\
457	0\\
458	0\\
459	0\\
460	0\\
461	0\\
462	0\\
463	0\\
464	0\\
465	0\\
466	0\\
467	0\\
468	0\\
469	0\\
470	0\\
471	0\\
472	0\\
473	0\\
474	0\\
475	0\\
476	0\\
477	0\\
478	0\\
479	0\\
480	0\\
481	0\\
482	0\\
483	0\\
484	0\\
485	0\\
486	0\\
487	0\\
488	0\\
489	0\\
490	0\\
491	0\\
492	0\\
493	0\\
494	0\\
495	0\\
496	0\\
497	0\\
498	0\\
499	0\\
500	0\\
501	0\\
502	0\\
503	0\\
504	0\\
505	0\\
506	0\\
507	0\\
508	0\\
509	0\\
510	0\\
511	0\\
512	0\\
513	0\\
514	0\\
515	0\\
516	0\\
517	0\\
518	0\\
519	0\\
520	0\\
521	0\\
522	0\\
523	0\\
524	0\\
525	0\\
526	0\\
527	0\\
528	0\\
529	0\\
530	0\\
531	0\\
532	0\\
533	0\\
534	0\\
535	0\\
536	0\\
537	0\\
538	0\\
539	0\\
540	0\\
541	0\\
542	0\\
543	0\\
544	0\\
545	0\\
546	0\\
547	0\\
548	0\\
549	0\\
550	0\\
551	0\\
552	0\\
553	0\\
554	0\\
555	0\\
556	0\\
557	0\\
558	0\\
559	0\\
560	0\\
561	0\\
562	0\\
563	0\\
564	0\\
565	0\\
566	0\\
567	5.88949037576846e-05\\
568	0.000278923522351577\\
569	0.000465215789347868\\
570	0.000573095561531768\\
571	0.000679064755436726\\
572	0.000781407100120683\\
573	0.000879327214528265\\
574	0.000971216243397552\\
575	0.00105576846491719\\
576	0.00113450507723386\\
577	0.00121198502541605\\
578	0.00128801880280719\\
579	0.00136213507180008\\
580	0.00143518170906471\\
581	0.00150992377660234\\
582	0.00158312346660404\\
583	0.00165386180226987\\
584	0.00192128218169381\\
585	0.00239891244873237\\
586	0.00288945262252655\\
587	0.00302337868377814\\
588	0.00315170319741699\\
589	0.00326939440428176\\
590	0.00336308691806496\\
591	0.00345791823244592\\
592	0.00355546598078343\\
593	0.00366084952809565\\
594	0.00378520541414672\\
595	0.00395742119285503\\
596	0.00425297862742962\\
597	0.0048700418989439\\
598	0.00632942537858856\\
599	0\\
600	0\\
};
\addplot [color=mycolor18,solid,forget plot]
  table[row sep=crcr]{%
1	0\\
2	0\\
3	0\\
4	0\\
5	0\\
6	0\\
7	0\\
8	0\\
9	0\\
10	0\\
11	0\\
12	0\\
13	0\\
14	0\\
15	0\\
16	0\\
17	0\\
18	0\\
19	0\\
20	0\\
21	0\\
22	0\\
23	0\\
24	0\\
25	0\\
26	0\\
27	0\\
28	0\\
29	0\\
30	0\\
31	0\\
32	0\\
33	0\\
34	0\\
35	0\\
36	0\\
37	0\\
38	0\\
39	0\\
40	0\\
41	0\\
42	0\\
43	0\\
44	0\\
45	0\\
46	0\\
47	0\\
48	0\\
49	0\\
50	0\\
51	0\\
52	0\\
53	0\\
54	0\\
55	0\\
56	0\\
57	0\\
58	0\\
59	0\\
60	0\\
61	0\\
62	0\\
63	0\\
64	0\\
65	0\\
66	0\\
67	0\\
68	0\\
69	0\\
70	0\\
71	0\\
72	0\\
73	0\\
74	0\\
75	0\\
76	0\\
77	0\\
78	0\\
79	0\\
80	0\\
81	0\\
82	0\\
83	0\\
84	0\\
85	0\\
86	0\\
87	0\\
88	0\\
89	0\\
90	0\\
91	0\\
92	0\\
93	0\\
94	0\\
95	0\\
96	0\\
97	0\\
98	0\\
99	0\\
100	0\\
101	0\\
102	0\\
103	0\\
104	0\\
105	0\\
106	0\\
107	0\\
108	0\\
109	0\\
110	0\\
111	0\\
112	0\\
113	0\\
114	0\\
115	0\\
116	0\\
117	0\\
118	0\\
119	0\\
120	0\\
121	0\\
122	0\\
123	0\\
124	0\\
125	0\\
126	0\\
127	0\\
128	0\\
129	0\\
130	0\\
131	0\\
132	0\\
133	0\\
134	0\\
135	0\\
136	0\\
137	0\\
138	0\\
139	0\\
140	0\\
141	0\\
142	0\\
143	0\\
144	0\\
145	0\\
146	0\\
147	0\\
148	0\\
149	0\\
150	0\\
151	0\\
152	0\\
153	0\\
154	0\\
155	0\\
156	0\\
157	0\\
158	0\\
159	0\\
160	0\\
161	0\\
162	0\\
163	0\\
164	0\\
165	0\\
166	0\\
167	0\\
168	0\\
169	0\\
170	0\\
171	0\\
172	0\\
173	0\\
174	0\\
175	0\\
176	0\\
177	0\\
178	0\\
179	0\\
180	0\\
181	0\\
182	0\\
183	0\\
184	0\\
185	0\\
186	0\\
187	0\\
188	0\\
189	0\\
190	0\\
191	0\\
192	0\\
193	0\\
194	0\\
195	0\\
196	0\\
197	0\\
198	0\\
199	0\\
200	0\\
201	0\\
202	0\\
203	0\\
204	0\\
205	0\\
206	0\\
207	0\\
208	0\\
209	0\\
210	0\\
211	0\\
212	0\\
213	0\\
214	0\\
215	0\\
216	0\\
217	0\\
218	0\\
219	0\\
220	0\\
221	0\\
222	0\\
223	0\\
224	0\\
225	0\\
226	0\\
227	0\\
228	0\\
229	0\\
230	0\\
231	0\\
232	0\\
233	0\\
234	0\\
235	0\\
236	0\\
237	0\\
238	0\\
239	0\\
240	0\\
241	0\\
242	0\\
243	0\\
244	0\\
245	0\\
246	0\\
247	0\\
248	0\\
249	0\\
250	0\\
251	0\\
252	0\\
253	0\\
254	0\\
255	0\\
256	0\\
257	0\\
258	0\\
259	0\\
260	0\\
261	0\\
262	0\\
263	0\\
264	0\\
265	0\\
266	0\\
267	0\\
268	0\\
269	0\\
270	0\\
271	0\\
272	0\\
273	0\\
274	0\\
275	0\\
276	0\\
277	0\\
278	0\\
279	0\\
280	0\\
281	0\\
282	0\\
283	0\\
284	0\\
285	0\\
286	0\\
287	0\\
288	0\\
289	0\\
290	0\\
291	0\\
292	0\\
293	0\\
294	0\\
295	0\\
296	0\\
297	0\\
298	0\\
299	0\\
300	0\\
301	0\\
302	0\\
303	0\\
304	0\\
305	0\\
306	0\\
307	0\\
308	0\\
309	0\\
310	0\\
311	0\\
312	0\\
313	0\\
314	0\\
315	0\\
316	0\\
317	0\\
318	0\\
319	0\\
320	0\\
321	0\\
322	0\\
323	0\\
324	0\\
325	0\\
326	0\\
327	0\\
328	0\\
329	0\\
330	0\\
331	0\\
332	0\\
333	0\\
334	0\\
335	0\\
336	0\\
337	0\\
338	0\\
339	0\\
340	0\\
341	0\\
342	0\\
343	0\\
344	0\\
345	0\\
346	0\\
347	0\\
348	0\\
349	0\\
350	0\\
351	0\\
352	0\\
353	0\\
354	0\\
355	0\\
356	0\\
357	0\\
358	0\\
359	0\\
360	0\\
361	0\\
362	0\\
363	0\\
364	0\\
365	0\\
366	0\\
367	0\\
368	0\\
369	0\\
370	0\\
371	0\\
372	0\\
373	0\\
374	0\\
375	0\\
376	0\\
377	0\\
378	0\\
379	0\\
380	0\\
381	0\\
382	0\\
383	0\\
384	0\\
385	0\\
386	0\\
387	0\\
388	0\\
389	0\\
390	0\\
391	0\\
392	0\\
393	0\\
394	0\\
395	0\\
396	0\\
397	0\\
398	0\\
399	0\\
400	0\\
401	0\\
402	0\\
403	0\\
404	0\\
405	0\\
406	0\\
407	0\\
408	0\\
409	0\\
410	0\\
411	0\\
412	0\\
413	0\\
414	0\\
415	0\\
416	0\\
417	0\\
418	0\\
419	0\\
420	0\\
421	0\\
422	0\\
423	0\\
424	0\\
425	0\\
426	0\\
427	0\\
428	0\\
429	0\\
430	0\\
431	0\\
432	0\\
433	0\\
434	0\\
435	0\\
436	0\\
437	0\\
438	0\\
439	0\\
440	0\\
441	0\\
442	0\\
443	0\\
444	0\\
445	0\\
446	0\\
447	0\\
448	0\\
449	0\\
450	0\\
451	0\\
452	0\\
453	0\\
454	0\\
455	0\\
456	0\\
457	0\\
458	0\\
459	0\\
460	0\\
461	0\\
462	0\\
463	0\\
464	0\\
465	0\\
466	0\\
467	0\\
468	0\\
469	0\\
470	0\\
471	0\\
472	0\\
473	0\\
474	0\\
475	0\\
476	0\\
477	0\\
478	0\\
479	0\\
480	0\\
481	0\\
482	0\\
483	0\\
484	0\\
485	0\\
486	0\\
487	0\\
488	0\\
489	0\\
490	0\\
491	0\\
492	0\\
493	0\\
494	0\\
495	0\\
496	0\\
497	0\\
498	0\\
499	0\\
500	0\\
501	0\\
502	0\\
503	0\\
504	0\\
505	0\\
506	0\\
507	0\\
508	0\\
509	0\\
510	0\\
511	0\\
512	0\\
513	0\\
514	0\\
515	0\\
516	0\\
517	0\\
518	0\\
519	0\\
520	0\\
521	0\\
522	0\\
523	0\\
524	0\\
525	0\\
526	0\\
527	0\\
528	0\\
529	0\\
530	0\\
531	0\\
532	0\\
533	0\\
534	0\\
535	0\\
536	0\\
537	0\\
538	0\\
539	0\\
540	0\\
541	0\\
542	0\\
543	0\\
544	0\\
545	0\\
546	0\\
547	0\\
548	0\\
549	0\\
550	0\\
551	0\\
552	0\\
553	0\\
554	0\\
555	0\\
556	0\\
557	0\\
558	0\\
559	0\\
560	0\\
561	0\\
562	9.90403746872915e-05\\
563	0.000304968769744545\\
564	0.000402367207070976\\
565	0.000497931023480415\\
566	0.000590494868905439\\
567	0.000678077044756074\\
568	0.000758685659673696\\
569	0.000830189121630715\\
570	0.000895650850952982\\
571	0.000961984822474034\\
572	0.0010278398051382\\
573	0.00109225696560847\\
574	0.00115548679208059\\
575	0.00121672171450071\\
576	0.00127705683954306\\
577	0.00133948023067548\\
578	0.00140501315387362\\
579	0.00146934333816017\\
580	0.00153055864356594\\
581	0.00158857538063284\\
582	0.00197419871052425\\
583	0.00246466618649465\\
584	0.00277686403795953\\
585	0.00289976434281973\\
586	0.00301736053854785\\
587	0.00310424045129586\\
588	0.00319120167814864\\
589	0.00327881165853131\\
590	0.00336825554262852\\
591	0.00346025420288071\\
592	0.00355649290755294\\
593	0.00366118457402185\\
594	0.00378527046594782\\
595	0.00395742119285503\\
596	0.00425297862742962\\
597	0.0048700418989439\\
598	0.00632942537858856\\
599	0\\
600	0\\
};
\addplot [color=red!25!mycolor17,solid,forget plot]
  table[row sep=crcr]{%
1	0\\
2	0\\
3	0\\
4	0\\
5	0\\
6	0\\
7	0\\
8	0\\
9	0\\
10	0\\
11	0\\
12	0\\
13	0\\
14	0\\
15	0\\
16	0\\
17	0\\
18	0\\
19	0\\
20	0\\
21	0\\
22	0\\
23	0\\
24	0\\
25	0\\
26	0\\
27	0\\
28	0\\
29	0\\
30	0\\
31	0\\
32	0\\
33	0\\
34	0\\
35	0\\
36	0\\
37	0\\
38	0\\
39	0\\
40	0\\
41	0\\
42	0\\
43	0\\
44	0\\
45	0\\
46	0\\
47	0\\
48	0\\
49	0\\
50	0\\
51	0\\
52	0\\
53	0\\
54	0\\
55	0\\
56	0\\
57	0\\
58	0\\
59	0\\
60	0\\
61	0\\
62	0\\
63	0\\
64	0\\
65	0\\
66	0\\
67	0\\
68	0\\
69	0\\
70	0\\
71	0\\
72	0\\
73	0\\
74	0\\
75	0\\
76	0\\
77	0\\
78	0\\
79	0\\
80	0\\
81	0\\
82	0\\
83	0\\
84	0\\
85	0\\
86	0\\
87	0\\
88	0\\
89	0\\
90	0\\
91	0\\
92	0\\
93	0\\
94	0\\
95	0\\
96	0\\
97	0\\
98	0\\
99	0\\
100	0\\
101	0\\
102	0\\
103	0\\
104	0\\
105	0\\
106	0\\
107	0\\
108	0\\
109	0\\
110	0\\
111	0\\
112	0\\
113	0\\
114	0\\
115	0\\
116	0\\
117	0\\
118	0\\
119	0\\
120	0\\
121	0\\
122	0\\
123	0\\
124	0\\
125	0\\
126	0\\
127	0\\
128	0\\
129	0\\
130	0\\
131	0\\
132	0\\
133	0\\
134	0\\
135	0\\
136	0\\
137	0\\
138	0\\
139	0\\
140	0\\
141	0\\
142	0\\
143	0\\
144	0\\
145	0\\
146	0\\
147	0\\
148	0\\
149	0\\
150	0\\
151	0\\
152	0\\
153	0\\
154	0\\
155	0\\
156	0\\
157	0\\
158	0\\
159	0\\
160	0\\
161	0\\
162	0\\
163	0\\
164	0\\
165	0\\
166	0\\
167	0\\
168	0\\
169	0\\
170	0\\
171	0\\
172	0\\
173	0\\
174	0\\
175	0\\
176	0\\
177	0\\
178	0\\
179	0\\
180	0\\
181	0\\
182	0\\
183	0\\
184	0\\
185	0\\
186	0\\
187	0\\
188	0\\
189	0\\
190	0\\
191	0\\
192	0\\
193	0\\
194	0\\
195	0\\
196	0\\
197	0\\
198	0\\
199	0\\
200	0\\
201	0\\
202	0\\
203	0\\
204	0\\
205	0\\
206	0\\
207	0\\
208	0\\
209	0\\
210	0\\
211	0\\
212	0\\
213	0\\
214	0\\
215	0\\
216	0\\
217	0\\
218	0\\
219	0\\
220	0\\
221	0\\
222	0\\
223	0\\
224	0\\
225	0\\
226	0\\
227	0\\
228	0\\
229	0\\
230	0\\
231	0\\
232	0\\
233	0\\
234	0\\
235	0\\
236	0\\
237	0\\
238	0\\
239	0\\
240	0\\
241	0\\
242	0\\
243	0\\
244	0\\
245	0\\
246	0\\
247	0\\
248	0\\
249	0\\
250	0\\
251	0\\
252	0\\
253	0\\
254	0\\
255	0\\
256	0\\
257	0\\
258	0\\
259	0\\
260	0\\
261	0\\
262	0\\
263	0\\
264	0\\
265	0\\
266	0\\
267	0\\
268	0\\
269	0\\
270	0\\
271	0\\
272	0\\
273	0\\
274	0\\
275	0\\
276	0\\
277	0\\
278	0\\
279	0\\
280	0\\
281	0\\
282	0\\
283	0\\
284	0\\
285	0\\
286	0\\
287	0\\
288	0\\
289	0\\
290	0\\
291	0\\
292	0\\
293	0\\
294	0\\
295	0\\
296	0\\
297	0\\
298	0\\
299	0\\
300	0\\
301	0\\
302	0\\
303	0\\
304	0\\
305	0\\
306	0\\
307	0\\
308	0\\
309	0\\
310	0\\
311	0\\
312	0\\
313	0\\
314	0\\
315	0\\
316	0\\
317	0\\
318	0\\
319	0\\
320	0\\
321	0\\
322	0\\
323	0\\
324	0\\
325	0\\
326	0\\
327	0\\
328	0\\
329	0\\
330	0\\
331	0\\
332	0\\
333	0\\
334	0\\
335	0\\
336	0\\
337	0\\
338	0\\
339	0\\
340	0\\
341	0\\
342	0\\
343	0\\
344	0\\
345	0\\
346	0\\
347	0\\
348	0\\
349	0\\
350	0\\
351	0\\
352	0\\
353	0\\
354	0\\
355	0\\
356	0\\
357	0\\
358	0\\
359	0\\
360	0\\
361	0\\
362	0\\
363	0\\
364	0\\
365	0\\
366	0\\
367	0\\
368	0\\
369	0\\
370	0\\
371	0\\
372	0\\
373	0\\
374	0\\
375	0\\
376	0\\
377	0\\
378	0\\
379	0\\
380	0\\
381	0\\
382	0\\
383	0\\
384	0\\
385	0\\
386	0\\
387	0\\
388	0\\
389	0\\
390	0\\
391	0\\
392	0\\
393	0\\
394	0\\
395	0\\
396	0\\
397	0\\
398	0\\
399	0\\
400	0\\
401	0\\
402	0\\
403	0\\
404	0\\
405	0\\
406	0\\
407	0\\
408	0\\
409	0\\
410	0\\
411	0\\
412	0\\
413	0\\
414	0\\
415	0\\
416	0\\
417	0\\
418	0\\
419	0\\
420	0\\
421	0\\
422	0\\
423	0\\
424	0\\
425	0\\
426	0\\
427	0\\
428	0\\
429	0\\
430	0\\
431	0\\
432	0\\
433	0\\
434	0\\
435	0\\
436	0\\
437	0\\
438	0\\
439	0\\
440	0\\
441	0\\
442	0\\
443	0\\
444	0\\
445	0\\
446	0\\
447	0\\
448	0\\
449	0\\
450	0\\
451	0\\
452	0\\
453	0\\
454	0\\
455	0\\
456	0\\
457	0\\
458	0\\
459	0\\
460	0\\
461	0\\
462	0\\
463	0\\
464	0\\
465	0\\
466	0\\
467	0\\
468	0\\
469	0\\
470	0\\
471	0\\
472	0\\
473	0\\
474	0\\
475	0\\
476	0\\
477	0\\
478	0\\
479	0\\
480	0\\
481	0\\
482	0\\
483	0\\
484	0\\
485	0\\
486	0\\
487	0\\
488	0\\
489	0\\
490	0\\
491	0\\
492	0\\
493	0\\
494	0\\
495	0\\
496	0\\
497	0\\
498	0\\
499	0\\
500	0\\
501	0\\
502	0\\
503	0\\
504	0\\
505	0\\
506	0\\
507	0\\
508	0\\
509	0\\
510	0\\
511	0\\
512	0\\
513	0\\
514	0\\
515	0\\
516	0\\
517	0\\
518	0\\
519	0\\
520	0\\
521	0\\
522	0\\
523	0\\
524	0\\
525	0\\
526	0\\
527	0\\
528	0\\
529	0\\
530	0\\
531	0\\
532	0\\
533	0\\
534	0\\
535	0\\
536	0\\
537	0\\
538	0\\
539	0\\
540	0\\
541	0\\
542	0\\
543	0\\
544	0\\
545	0\\
546	0\\
547	0\\
548	0\\
549	0\\
550	0\\
551	0\\
552	0\\
553	0\\
554	0\\
555	0\\
556	0\\
557	8.01734108211468e-05\\
558	0.000223371265806801\\
559	0.000311803468712962\\
560	0.000397274215465461\\
561	0.000478427144538355\\
562	0.000553370163734247\\
563	0.0006186795095291\\
564	0.000675713681809035\\
565	0.00073191039909959\\
566	0.000787276194635121\\
567	0.000843166680497421\\
568	0.000899372565017849\\
569	0.000954897785455491\\
570	0.00101018526904893\\
571	0.0010637562962866\\
572	0.00111682448780514\\
573	0.00117048000700388\\
574	0.00122509244478267\\
575	0.00128514626774489\\
576	0.00134430788764564\\
577	0.00140097157268809\\
578	0.00145403611431188\\
579	0.00150752687252756\\
580	0.00193194802301185\\
581	0.00243680986819744\\
582	0.00264404234620892\\
583	0.00276021838521539\\
584	0.00285857309192273\\
585	0.00294082185579374\\
586	0.00302342706147123\\
587	0.00310731224695134\\
588	0.00319261476955509\\
589	0.00327956317486525\\
590	0.00336860380042371\\
591	0.00346039193281902\\
592	0.00355653331527142\\
593	0.00366119131364763\\
594	0.00378527046594782\\
595	0.00395742119285503\\
596	0.00425297862742962\\
597	0.0048700418989439\\
598	0.00632942537858856\\
599	0\\
600	0\\
};
\addplot [color=mycolor19,solid,forget plot]
  table[row sep=crcr]{%
1	0\\
2	0\\
3	0\\
4	0\\
5	0\\
6	0\\
7	0\\
8	0\\
9	0\\
10	0\\
11	0\\
12	0\\
13	0\\
14	0\\
15	0\\
16	0\\
17	0\\
18	0\\
19	0\\
20	0\\
21	0\\
22	0\\
23	0\\
24	0\\
25	0\\
26	0\\
27	0\\
28	0\\
29	0\\
30	0\\
31	0\\
32	0\\
33	0\\
34	0\\
35	0\\
36	0\\
37	0\\
38	0\\
39	0\\
40	0\\
41	0\\
42	0\\
43	0\\
44	0\\
45	0\\
46	0\\
47	0\\
48	0\\
49	0\\
50	0\\
51	0\\
52	0\\
53	0\\
54	0\\
55	0\\
56	0\\
57	0\\
58	0\\
59	0\\
60	0\\
61	0\\
62	0\\
63	0\\
64	0\\
65	0\\
66	0\\
67	0\\
68	0\\
69	0\\
70	0\\
71	0\\
72	0\\
73	0\\
74	0\\
75	0\\
76	0\\
77	0\\
78	0\\
79	0\\
80	0\\
81	0\\
82	0\\
83	0\\
84	0\\
85	0\\
86	0\\
87	0\\
88	0\\
89	0\\
90	0\\
91	0\\
92	0\\
93	0\\
94	0\\
95	0\\
96	0\\
97	0\\
98	0\\
99	0\\
100	0\\
101	0\\
102	0\\
103	0\\
104	0\\
105	0\\
106	0\\
107	0\\
108	0\\
109	0\\
110	0\\
111	0\\
112	0\\
113	0\\
114	0\\
115	0\\
116	0\\
117	0\\
118	0\\
119	0\\
120	0\\
121	0\\
122	0\\
123	0\\
124	0\\
125	0\\
126	0\\
127	0\\
128	0\\
129	0\\
130	0\\
131	0\\
132	0\\
133	0\\
134	0\\
135	0\\
136	0\\
137	0\\
138	0\\
139	0\\
140	0\\
141	0\\
142	0\\
143	0\\
144	0\\
145	0\\
146	0\\
147	0\\
148	0\\
149	0\\
150	0\\
151	0\\
152	0\\
153	0\\
154	0\\
155	0\\
156	0\\
157	0\\
158	0\\
159	0\\
160	0\\
161	0\\
162	0\\
163	0\\
164	0\\
165	0\\
166	0\\
167	0\\
168	0\\
169	0\\
170	0\\
171	0\\
172	0\\
173	0\\
174	0\\
175	0\\
176	0\\
177	0\\
178	0\\
179	0\\
180	0\\
181	0\\
182	0\\
183	0\\
184	0\\
185	0\\
186	0\\
187	0\\
188	0\\
189	0\\
190	0\\
191	0\\
192	0\\
193	0\\
194	0\\
195	0\\
196	0\\
197	0\\
198	0\\
199	0\\
200	0\\
201	0\\
202	0\\
203	0\\
204	0\\
205	0\\
206	0\\
207	0\\
208	0\\
209	0\\
210	0\\
211	0\\
212	0\\
213	0\\
214	0\\
215	0\\
216	0\\
217	0\\
218	0\\
219	0\\
220	0\\
221	0\\
222	0\\
223	0\\
224	0\\
225	0\\
226	0\\
227	0\\
228	0\\
229	0\\
230	0\\
231	0\\
232	0\\
233	0\\
234	0\\
235	0\\
236	0\\
237	0\\
238	0\\
239	0\\
240	0\\
241	0\\
242	0\\
243	0\\
244	0\\
245	0\\
246	0\\
247	0\\
248	0\\
249	0\\
250	0\\
251	0\\
252	0\\
253	0\\
254	0\\
255	0\\
256	0\\
257	0\\
258	0\\
259	0\\
260	0\\
261	0\\
262	0\\
263	0\\
264	0\\
265	0\\
266	0\\
267	0\\
268	0\\
269	0\\
270	0\\
271	0\\
272	0\\
273	0\\
274	0\\
275	0\\
276	0\\
277	0\\
278	0\\
279	0\\
280	0\\
281	0\\
282	0\\
283	0\\
284	0\\
285	0\\
286	0\\
287	0\\
288	0\\
289	0\\
290	0\\
291	0\\
292	0\\
293	0\\
294	0\\
295	0\\
296	0\\
297	0\\
298	0\\
299	0\\
300	0\\
301	0\\
302	0\\
303	0\\
304	0\\
305	0\\
306	0\\
307	0\\
308	0\\
309	0\\
310	0\\
311	0\\
312	0\\
313	0\\
314	0\\
315	0\\
316	0\\
317	0\\
318	0\\
319	0\\
320	0\\
321	0\\
322	0\\
323	0\\
324	0\\
325	0\\
326	0\\
327	0\\
328	0\\
329	0\\
330	0\\
331	0\\
332	0\\
333	0\\
334	0\\
335	0\\
336	0\\
337	0\\
338	0\\
339	0\\
340	0\\
341	0\\
342	0\\
343	0\\
344	0\\
345	0\\
346	0\\
347	0\\
348	0\\
349	0\\
350	0\\
351	0\\
352	0\\
353	0\\
354	0\\
355	0\\
356	0\\
357	0\\
358	0\\
359	0\\
360	0\\
361	0\\
362	0\\
363	0\\
364	0\\
365	0\\
366	0\\
367	0\\
368	0\\
369	0\\
370	0\\
371	0\\
372	0\\
373	0\\
374	0\\
375	0\\
376	0\\
377	0\\
378	0\\
379	0\\
380	0\\
381	0\\
382	0\\
383	0\\
384	0\\
385	0\\
386	0\\
387	0\\
388	0\\
389	0\\
390	0\\
391	0\\
392	0\\
393	0\\
394	0\\
395	0\\
396	0\\
397	0\\
398	0\\
399	0\\
400	0\\
401	0\\
402	0\\
403	0\\
404	0\\
405	0\\
406	0\\
407	0\\
408	0\\
409	0\\
410	0\\
411	0\\
412	0\\
413	0\\
414	0\\
415	0\\
416	0\\
417	0\\
418	0\\
419	0\\
420	0\\
421	0\\
422	0\\
423	0\\
424	0\\
425	0\\
426	0\\
427	0\\
428	0\\
429	0\\
430	0\\
431	0\\
432	0\\
433	0\\
434	0\\
435	0\\
436	0\\
437	0\\
438	0\\
439	0\\
440	0\\
441	0\\
442	0\\
443	0\\
444	0\\
445	0\\
446	0\\
447	0\\
448	0\\
449	0\\
450	0\\
451	0\\
452	0\\
453	0\\
454	0\\
455	0\\
456	0\\
457	0\\
458	0\\
459	0\\
460	0\\
461	0\\
462	0\\
463	0\\
464	0\\
465	0\\
466	0\\
467	0\\
468	0\\
469	0\\
470	0\\
471	0\\
472	0\\
473	0\\
474	0\\
475	0\\
476	0\\
477	0\\
478	0\\
479	0\\
480	0\\
481	0\\
482	0\\
483	0\\
484	0\\
485	0\\
486	0\\
487	0\\
488	0\\
489	0\\
490	0\\
491	0\\
492	0\\
493	0\\
494	0\\
495	0\\
496	0\\
497	0\\
498	0\\
499	0\\
500	0\\
501	0\\
502	0\\
503	0\\
504	0\\
505	0\\
506	0\\
507	0\\
508	0\\
509	0\\
510	0\\
511	0\\
512	0\\
513	0\\
514	0\\
515	0\\
516	0\\
517	0\\
518	0\\
519	0\\
520	0\\
521	0\\
522	0\\
523	0\\
524	0\\
525	0\\
526	0\\
527	0\\
528	0\\
529	0\\
530	0\\
531	0\\
532	0\\
533	0\\
534	0\\
535	0\\
536	0\\
537	0\\
538	0\\
539	0\\
540	0\\
541	0\\
542	0\\
543	0\\
544	0\\
545	0\\
546	0\\
547	0\\
548	0\\
549	0\\
550	0\\
551	0\\
552	1.50670897009176e-05\\
553	0.000124382787751889\\
554	0.00020503601022344\\
555	0.000281906524719583\\
556	0.000353475266029824\\
557	0.000416928686445217\\
558	0.000471622760274698\\
559	0.000521843264876268\\
560	0.000570759546781793\\
561	0.000618739159396117\\
562	0.000666121145496063\\
563	0.000713768966029745\\
564	0.000763815747326926\\
565	0.0008137163277131\\
566	0.000863511816317939\\
567	0.000912037857953922\\
568	0.000959785906419871\\
569	0.00100821347014938\\
570	0.00105737215001796\\
571	0.00110730055108194\\
572	0.00116093038778452\\
573	0.00121697978948583\\
574	0.00127198613892506\\
575	0.00132150660081605\\
576	0.00137146061891735\\
577	0.00142187466526321\\
578	0.00180758151954677\\
579	0.00232459949440059\\
580	0.00250414369719064\\
581	0.00261607079037986\\
582	0.00270221271091265\\
583	0.00278105910457229\\
584	0.00286069531223358\\
585	0.00294160817599962\\
586	0.00302387110985324\\
587	0.00310754666812213\\
588	0.00319273161070454\\
589	0.00327961396220161\\
590	0.00336862207624017\\
591	0.00346039677087353\\
592	0.00355653402929194\\
593	0.00366119131364763\\
594	0.00378527046594783\\
595	0.00395742119285503\\
596	0.00425297862742962\\
597	0.0048700418989439\\
598	0.00632942537858856\\
599	0\\
600	0\\
};
\addplot [color=red!50!mycolor17,solid,forget plot]
  table[row sep=crcr]{%
1	0\\
2	0\\
3	0\\
4	0\\
5	0\\
6	0\\
7	0\\
8	0\\
9	0\\
10	0\\
11	0\\
12	0\\
13	0\\
14	0\\
15	0\\
16	0\\
17	0\\
18	0\\
19	0\\
20	0\\
21	0\\
22	0\\
23	0\\
24	0\\
25	0\\
26	0\\
27	0\\
28	0\\
29	0\\
30	0\\
31	0\\
32	0\\
33	0\\
34	0\\
35	0\\
36	0\\
37	0\\
38	0\\
39	0\\
40	0\\
41	0\\
42	0\\
43	0\\
44	0\\
45	0\\
46	0\\
47	0\\
48	0\\
49	0\\
50	0\\
51	0\\
52	0\\
53	0\\
54	0\\
55	0\\
56	0\\
57	0\\
58	0\\
59	0\\
60	0\\
61	0\\
62	0\\
63	0\\
64	0\\
65	0\\
66	0\\
67	0\\
68	0\\
69	0\\
70	0\\
71	0\\
72	0\\
73	0\\
74	0\\
75	0\\
76	0\\
77	0\\
78	0\\
79	0\\
80	0\\
81	0\\
82	0\\
83	0\\
84	0\\
85	0\\
86	0\\
87	0\\
88	0\\
89	0\\
90	0\\
91	0\\
92	0\\
93	0\\
94	0\\
95	0\\
96	0\\
97	0\\
98	0\\
99	0\\
100	0\\
101	0\\
102	0\\
103	0\\
104	0\\
105	0\\
106	0\\
107	0\\
108	0\\
109	0\\
110	0\\
111	0\\
112	0\\
113	0\\
114	0\\
115	0\\
116	0\\
117	0\\
118	0\\
119	0\\
120	0\\
121	0\\
122	0\\
123	0\\
124	0\\
125	0\\
126	0\\
127	0\\
128	0\\
129	0\\
130	0\\
131	0\\
132	0\\
133	0\\
134	0\\
135	0\\
136	0\\
137	0\\
138	0\\
139	0\\
140	0\\
141	0\\
142	0\\
143	0\\
144	0\\
145	0\\
146	0\\
147	0\\
148	0\\
149	0\\
150	0\\
151	0\\
152	0\\
153	0\\
154	0\\
155	0\\
156	0\\
157	0\\
158	0\\
159	0\\
160	0\\
161	0\\
162	0\\
163	0\\
164	0\\
165	0\\
166	0\\
167	0\\
168	0\\
169	0\\
170	0\\
171	0\\
172	0\\
173	0\\
174	0\\
175	0\\
176	0\\
177	0\\
178	0\\
179	0\\
180	0\\
181	0\\
182	0\\
183	0\\
184	0\\
185	0\\
186	0\\
187	0\\
188	0\\
189	0\\
190	0\\
191	0\\
192	0\\
193	0\\
194	0\\
195	0\\
196	0\\
197	0\\
198	0\\
199	0\\
200	0\\
201	0\\
202	0\\
203	0\\
204	0\\
205	0\\
206	0\\
207	0\\
208	0\\
209	0\\
210	0\\
211	0\\
212	0\\
213	0\\
214	0\\
215	0\\
216	0\\
217	0\\
218	0\\
219	0\\
220	0\\
221	0\\
222	0\\
223	0\\
224	0\\
225	0\\
226	0\\
227	0\\
228	0\\
229	0\\
230	0\\
231	0\\
232	0\\
233	0\\
234	0\\
235	0\\
236	0\\
237	0\\
238	0\\
239	0\\
240	0\\
241	0\\
242	0\\
243	0\\
244	0\\
245	0\\
246	0\\
247	0\\
248	0\\
249	0\\
250	0\\
251	0\\
252	0\\
253	0\\
254	0\\
255	0\\
256	0\\
257	0\\
258	0\\
259	0\\
260	0\\
261	0\\
262	0\\
263	0\\
264	0\\
265	0\\
266	0\\
267	0\\
268	0\\
269	0\\
270	0\\
271	0\\
272	0\\
273	0\\
274	0\\
275	0\\
276	0\\
277	0\\
278	0\\
279	0\\
280	0\\
281	0\\
282	0\\
283	0\\
284	0\\
285	0\\
286	0\\
287	0\\
288	0\\
289	0\\
290	0\\
291	0\\
292	0\\
293	0\\
294	0\\
295	0\\
296	0\\
297	0\\
298	0\\
299	0\\
300	0\\
301	0\\
302	0\\
303	0\\
304	0\\
305	0\\
306	0\\
307	0\\
308	0\\
309	0\\
310	0\\
311	0\\
312	0\\
313	0\\
314	0\\
315	0\\
316	0\\
317	0\\
318	0\\
319	0\\
320	0\\
321	0\\
322	0\\
323	0\\
324	0\\
325	0\\
326	0\\
327	0\\
328	0\\
329	0\\
330	0\\
331	0\\
332	0\\
333	0\\
334	0\\
335	0\\
336	0\\
337	0\\
338	0\\
339	0\\
340	0\\
341	0\\
342	0\\
343	0\\
344	0\\
345	0\\
346	0\\
347	0\\
348	0\\
349	0\\
350	0\\
351	0\\
352	0\\
353	0\\
354	0\\
355	0\\
356	0\\
357	0\\
358	0\\
359	0\\
360	0\\
361	0\\
362	0\\
363	0\\
364	0\\
365	0\\
366	0\\
367	0\\
368	0\\
369	0\\
370	0\\
371	0\\
372	0\\
373	0\\
374	0\\
375	0\\
376	0\\
377	0\\
378	0\\
379	0\\
380	0\\
381	0\\
382	0\\
383	0\\
384	0\\
385	0\\
386	0\\
387	0\\
388	0\\
389	0\\
390	0\\
391	0\\
392	0\\
393	0\\
394	0\\
395	0\\
396	0\\
397	0\\
398	0\\
399	0\\
400	0\\
401	0\\
402	0\\
403	0\\
404	0\\
405	0\\
406	0\\
407	0\\
408	0\\
409	0\\
410	0\\
411	0\\
412	0\\
413	0\\
414	0\\
415	0\\
416	0\\
417	0\\
418	0\\
419	0\\
420	0\\
421	0\\
422	0\\
423	0\\
424	0\\
425	0\\
426	0\\
427	0\\
428	0\\
429	0\\
430	0\\
431	0\\
432	0\\
433	0\\
434	0\\
435	0\\
436	0\\
437	0\\
438	0\\
439	0\\
440	0\\
441	0\\
442	0\\
443	0\\
444	0\\
445	0\\
446	0\\
447	0\\
448	0\\
449	0\\
450	0\\
451	0\\
452	0\\
453	0\\
454	0\\
455	0\\
456	0\\
457	0\\
458	0\\
459	0\\
460	0\\
461	0\\
462	0\\
463	0\\
464	0\\
465	0\\
466	0\\
467	0\\
468	0\\
469	0\\
470	0\\
471	0\\
472	0\\
473	0\\
474	0\\
475	0\\
476	0\\
477	0\\
478	0\\
479	0\\
480	0\\
481	0\\
482	0\\
483	0\\
484	0\\
485	0\\
486	0\\
487	0\\
488	0\\
489	0\\
490	0\\
491	0\\
492	0\\
493	0\\
494	0\\
495	0\\
496	0\\
497	0\\
498	0\\
499	0\\
500	0\\
501	0\\
502	0\\
503	0\\
504	0\\
505	0\\
506	0\\
507	0\\
508	0\\
509	0\\
510	0\\
511	0\\
512	0\\
513	0\\
514	0\\
515	0\\
516	0\\
517	0\\
518	0\\
519	0\\
520	0\\
521	0\\
522	0\\
523	0\\
524	0\\
525	0\\
526	0\\
527	0\\
528	0\\
529	0\\
530	0\\
531	0\\
532	0\\
533	0\\
534	0\\
535	0\\
536	0\\
537	0\\
538	0\\
539	0\\
540	0\\
541	0\\
542	0\\
543	0\\
544	0\\
545	0\\
546	0\\
547	0\\
548	1.64327239660599e-05\\
549	9.08311762122886e-05\\
550	0.000161022938856363\\
551	0.000224806955478363\\
552	0.000280485308906282\\
553	0.000327622000208862\\
554	0.000372012735657423\\
555	0.000415398793616001\\
556	0.000457922393037952\\
557	0.000499893593165752\\
558	0.000541753390275113\\
559	0.000584113654120537\\
560	0.000628623132028616\\
561	0.000674257201963313\\
562	0.000719918679057545\\
563	0.000765234608134834\\
564	0.000808741945395788\\
565	0.000852912506720194\\
566	0.00089779277058568\\
567	0.000943386792557535\\
568	0.000989813512646882\\
569	0.00103725286168404\\
570	0.00109022682293433\\
571	0.00114341698775839\\
572	0.00119302706840499\\
573	0.00124014217818203\\
574	0.00128770080075513\\
575	0.00133565715457315\\
576	0.00160918154831211\\
577	0.00213608599489938\\
578	0.00235964274414307\\
579	0.00246903551743822\\
580	0.00255013683265288\\
581	0.00262610814362027\\
582	0.00270308299221297\\
583	0.00278130768962817\\
584	0.00286082862410346\\
585	0.00294168211648289\\
586	0.00302390908465139\\
587	0.003107564358473\\
588	0.00319273875713431\\
589	0.00327961632359304\\
590	0.00336862264580875\\
591	0.00346039684658503\\
592	0.00355653402929193\\
593	0.00366119131364763\\
594	0.00378527046594782\\
595	0.00395742119285503\\
596	0.00425297862742962\\
597	0.0048700418989439\\
598	0.00632942537858856\\
599	0\\
600	0\\
};
\addplot [color=red!40!mycolor19,solid,forget plot]
  table[row sep=crcr]{%
1	0\\
2	0\\
3	0\\
4	0\\
5	0\\
6	0\\
7	0\\
8	0\\
9	0\\
10	0\\
11	0\\
12	0\\
13	0\\
14	0\\
15	0\\
16	0\\
17	0\\
18	0\\
19	0\\
20	0\\
21	0\\
22	0\\
23	0\\
24	0\\
25	0\\
26	0\\
27	0\\
28	0\\
29	0\\
30	0\\
31	0\\
32	0\\
33	0\\
34	0\\
35	0\\
36	0\\
37	0\\
38	0\\
39	0\\
40	0\\
41	0\\
42	0\\
43	0\\
44	0\\
45	0\\
46	0\\
47	0\\
48	0\\
49	0\\
50	0\\
51	0\\
52	0\\
53	0\\
54	0\\
55	0\\
56	0\\
57	0\\
58	0\\
59	0\\
60	0\\
61	0\\
62	0\\
63	0\\
64	0\\
65	0\\
66	0\\
67	0\\
68	0\\
69	0\\
70	0\\
71	0\\
72	0\\
73	0\\
74	0\\
75	0\\
76	0\\
77	0\\
78	0\\
79	0\\
80	0\\
81	0\\
82	0\\
83	0\\
84	0\\
85	0\\
86	0\\
87	0\\
88	0\\
89	0\\
90	0\\
91	0\\
92	0\\
93	0\\
94	0\\
95	0\\
96	0\\
97	0\\
98	0\\
99	0\\
100	0\\
101	0\\
102	0\\
103	0\\
104	0\\
105	0\\
106	0\\
107	0\\
108	0\\
109	0\\
110	0\\
111	0\\
112	0\\
113	0\\
114	0\\
115	0\\
116	0\\
117	0\\
118	0\\
119	0\\
120	0\\
121	0\\
122	0\\
123	0\\
124	0\\
125	0\\
126	0\\
127	0\\
128	0\\
129	0\\
130	0\\
131	0\\
132	0\\
133	0\\
134	0\\
135	0\\
136	0\\
137	0\\
138	0\\
139	0\\
140	0\\
141	0\\
142	0\\
143	0\\
144	0\\
145	0\\
146	0\\
147	0\\
148	0\\
149	0\\
150	0\\
151	0\\
152	0\\
153	0\\
154	0\\
155	0\\
156	0\\
157	0\\
158	0\\
159	0\\
160	0\\
161	0\\
162	0\\
163	0\\
164	0\\
165	0\\
166	0\\
167	0\\
168	0\\
169	0\\
170	0\\
171	0\\
172	0\\
173	0\\
174	0\\
175	0\\
176	0\\
177	0\\
178	0\\
179	0\\
180	0\\
181	0\\
182	0\\
183	0\\
184	0\\
185	0\\
186	0\\
187	0\\
188	0\\
189	0\\
190	0\\
191	0\\
192	0\\
193	0\\
194	0\\
195	0\\
196	0\\
197	0\\
198	0\\
199	0\\
200	0\\
201	0\\
202	0\\
203	0\\
204	0\\
205	0\\
206	0\\
207	0\\
208	0\\
209	0\\
210	0\\
211	0\\
212	0\\
213	0\\
214	0\\
215	0\\
216	0\\
217	0\\
218	0\\
219	0\\
220	0\\
221	0\\
222	0\\
223	0\\
224	0\\
225	0\\
226	0\\
227	0\\
228	0\\
229	0\\
230	0\\
231	0\\
232	0\\
233	0\\
234	0\\
235	0\\
236	0\\
237	0\\
238	0\\
239	0\\
240	0\\
241	0\\
242	0\\
243	0\\
244	0\\
245	0\\
246	0\\
247	0\\
248	0\\
249	0\\
250	0\\
251	0\\
252	0\\
253	0\\
254	0\\
255	0\\
256	0\\
257	0\\
258	0\\
259	0\\
260	0\\
261	0\\
262	0\\
263	0\\
264	0\\
265	0\\
266	0\\
267	0\\
268	0\\
269	0\\
270	0\\
271	0\\
272	0\\
273	0\\
274	0\\
275	0\\
276	0\\
277	0\\
278	0\\
279	0\\
280	0\\
281	0\\
282	0\\
283	0\\
284	0\\
285	0\\
286	0\\
287	0\\
288	0\\
289	0\\
290	0\\
291	0\\
292	0\\
293	0\\
294	0\\
295	0\\
296	0\\
297	0\\
298	0\\
299	0\\
300	0\\
301	0\\
302	0\\
303	0\\
304	0\\
305	0\\
306	0\\
307	0\\
308	0\\
309	0\\
310	0\\
311	0\\
312	0\\
313	0\\
314	0\\
315	0\\
316	0\\
317	0\\
318	0\\
319	0\\
320	0\\
321	0\\
322	0\\
323	0\\
324	0\\
325	0\\
326	0\\
327	0\\
328	0\\
329	0\\
330	0\\
331	0\\
332	0\\
333	0\\
334	0\\
335	0\\
336	0\\
337	0\\
338	0\\
339	0\\
340	0\\
341	0\\
342	0\\
343	0\\
344	0\\
345	0\\
346	0\\
347	0\\
348	0\\
349	0\\
350	0\\
351	0\\
352	0\\
353	0\\
354	0\\
355	0\\
356	0\\
357	0\\
358	0\\
359	0\\
360	0\\
361	0\\
362	0\\
363	0\\
364	0\\
365	0\\
366	0\\
367	0\\
368	0\\
369	0\\
370	0\\
371	0\\
372	0\\
373	0\\
374	0\\
375	0\\
376	0\\
377	0\\
378	0\\
379	0\\
380	0\\
381	0\\
382	0\\
383	0\\
384	0\\
385	0\\
386	0\\
387	0\\
388	0\\
389	0\\
390	0\\
391	0\\
392	0\\
393	0\\
394	0\\
395	0\\
396	0\\
397	0\\
398	0\\
399	0\\
400	0\\
401	0\\
402	0\\
403	0\\
404	0\\
405	0\\
406	0\\
407	0\\
408	0\\
409	0\\
410	0\\
411	0\\
412	0\\
413	0\\
414	0\\
415	0\\
416	0\\
417	0\\
418	0\\
419	0\\
420	0\\
421	0\\
422	0\\
423	0\\
424	0\\
425	0\\
426	0\\
427	0\\
428	0\\
429	0\\
430	0\\
431	0\\
432	0\\
433	0\\
434	0\\
435	0\\
436	0\\
437	0\\
438	0\\
439	0\\
440	0\\
441	0\\
442	0\\
443	0\\
444	0\\
445	0\\
446	0\\
447	0\\
448	0\\
449	0\\
450	0\\
451	0\\
452	0\\
453	0\\
454	0\\
455	0\\
456	0\\
457	0\\
458	0\\
459	0\\
460	0\\
461	0\\
462	0\\
463	0\\
464	0\\
465	0\\
466	0\\
467	0\\
468	0\\
469	0\\
470	0\\
471	0\\
472	0\\
473	0\\
474	0\\
475	0\\
476	0\\
477	0\\
478	0\\
479	0\\
480	0\\
481	0\\
482	0\\
483	0\\
484	0\\
485	0\\
486	0\\
487	0\\
488	0\\
489	0\\
490	0\\
491	0\\
492	0\\
493	0\\
494	0\\
495	0\\
496	0\\
497	0\\
498	0\\
499	0\\
500	0\\
501	0\\
502	0\\
503	0\\
504	0\\
505	0\\
506	0\\
507	0\\
508	0\\
509	0\\
510	0\\
511	0\\
512	0\\
513	0\\
514	0\\
515	0\\
516	0\\
517	0\\
518	0\\
519	0\\
520	0\\
521	0\\
522	0\\
523	0\\
524	0\\
525	0\\
526	0\\
527	0\\
528	0\\
529	0\\
530	0\\
531	0\\
532	0\\
533	0\\
534	0\\
535	0\\
536	0\\
537	0\\
538	0\\
539	0\\
540	0\\
541	0\\
542	0\\
543	0\\
544	0\\
545	3.82391076137234e-05\\
546	9.68023725063791e-05\\
547	0.000147720982546101\\
548	0.000190142250714629\\
549	0.00022979867141449\\
550	0.000268454303467937\\
551	0.000306284037234123\\
552	0.00034374680041995\\
553	0.000381433047254094\\
554	0.000419542603462078\\
555	0.000458172315969797\\
556	0.000497380378338157\\
557	0.00053955610927823\\
558	0.000581988250460856\\
559	0.000624481094669423\\
560	0.000665506456059995\\
561	0.000706089745012563\\
562	0.000747338613311647\\
563	0.000789267713198676\\
564	0.000831888557103819\\
565	0.000875412308035493\\
566	0.000919872431348538\\
567	0.000965414080806018\\
568	0.00101726491970261\\
569	0.00106823974206828\\
570	0.00111385344432899\\
571	0.00115893121983135\\
572	0.00120437493685983\\
573	0.00125042703199772\\
574	0.00134080429263664\\
575	0.00187574822298428\\
576	0.00221187192817254\\
577	0.00232010641060596\\
578	0.00240256038064571\\
579	0.00247601921057826\\
580	0.00255050068707274\\
581	0.00262619072506652\\
582	0.00270312408738541\\
583	0.00278133051979913\\
584	0.00286084071525918\\
585	0.00294168800565563\\
586	0.00302391165742733\\
587	0.00310756532765526\\
588	0.00319273905359195\\
589	0.00327961638930458\\
590	0.00336862265377153\\
591	0.00346039684658503\\
592	0.00355653402929194\\
593	0.00366119131364763\\
594	0.00378527046594783\\
595	0.00395742119285503\\
596	0.00425297862742962\\
597	0.0048700418989439\\
598	0.00632942537858856\\
599	0\\
600	0\\
};
\addplot [color=red!75!mycolor17,solid,forget plot]
  table[row sep=crcr]{%
1	0\\
2	0\\
3	0\\
4	0\\
5	0\\
6	0\\
7	0\\
8	0\\
9	0\\
10	0\\
11	0\\
12	0\\
13	0\\
14	0\\
15	0\\
16	0\\
17	0\\
18	0\\
19	0\\
20	0\\
21	0\\
22	0\\
23	0\\
24	0\\
25	0\\
26	0\\
27	0\\
28	0\\
29	0\\
30	0\\
31	0\\
32	0\\
33	0\\
34	0\\
35	0\\
36	0\\
37	0\\
38	0\\
39	0\\
40	0\\
41	0\\
42	0\\
43	0\\
44	0\\
45	0\\
46	0\\
47	0\\
48	0\\
49	0\\
50	0\\
51	0\\
52	0\\
53	0\\
54	0\\
55	0\\
56	0\\
57	0\\
58	0\\
59	0\\
60	0\\
61	0\\
62	0\\
63	0\\
64	0\\
65	0\\
66	0\\
67	0\\
68	0\\
69	0\\
70	0\\
71	0\\
72	0\\
73	0\\
74	0\\
75	0\\
76	0\\
77	0\\
78	0\\
79	0\\
80	0\\
81	0\\
82	0\\
83	0\\
84	0\\
85	0\\
86	0\\
87	0\\
88	0\\
89	0\\
90	0\\
91	0\\
92	0\\
93	0\\
94	0\\
95	0\\
96	0\\
97	0\\
98	0\\
99	0\\
100	0\\
101	0\\
102	0\\
103	0\\
104	0\\
105	0\\
106	0\\
107	0\\
108	0\\
109	0\\
110	0\\
111	0\\
112	0\\
113	0\\
114	0\\
115	0\\
116	0\\
117	0\\
118	0\\
119	0\\
120	0\\
121	0\\
122	0\\
123	0\\
124	0\\
125	0\\
126	0\\
127	0\\
128	0\\
129	0\\
130	0\\
131	0\\
132	0\\
133	0\\
134	0\\
135	0\\
136	0\\
137	0\\
138	0\\
139	0\\
140	0\\
141	0\\
142	0\\
143	0\\
144	0\\
145	0\\
146	0\\
147	0\\
148	0\\
149	0\\
150	0\\
151	0\\
152	0\\
153	0\\
154	0\\
155	0\\
156	0\\
157	0\\
158	0\\
159	0\\
160	0\\
161	0\\
162	0\\
163	0\\
164	0\\
165	0\\
166	0\\
167	0\\
168	0\\
169	0\\
170	0\\
171	0\\
172	0\\
173	0\\
174	0\\
175	0\\
176	0\\
177	0\\
178	0\\
179	0\\
180	0\\
181	0\\
182	0\\
183	0\\
184	0\\
185	0\\
186	0\\
187	0\\
188	0\\
189	0\\
190	0\\
191	0\\
192	0\\
193	0\\
194	0\\
195	0\\
196	0\\
197	0\\
198	0\\
199	0\\
200	0\\
201	0\\
202	0\\
203	0\\
204	0\\
205	0\\
206	0\\
207	0\\
208	0\\
209	0\\
210	0\\
211	0\\
212	0\\
213	0\\
214	0\\
215	0\\
216	0\\
217	0\\
218	0\\
219	0\\
220	0\\
221	0\\
222	0\\
223	0\\
224	0\\
225	0\\
226	0\\
227	0\\
228	0\\
229	0\\
230	0\\
231	0\\
232	0\\
233	0\\
234	0\\
235	0\\
236	0\\
237	0\\
238	0\\
239	0\\
240	0\\
241	0\\
242	0\\
243	0\\
244	0\\
245	0\\
246	0\\
247	0\\
248	0\\
249	0\\
250	0\\
251	0\\
252	0\\
253	0\\
254	0\\
255	0\\
256	0\\
257	0\\
258	0\\
259	0\\
260	0\\
261	0\\
262	0\\
263	0\\
264	0\\
265	0\\
266	0\\
267	0\\
268	0\\
269	0\\
270	0\\
271	0\\
272	0\\
273	0\\
274	0\\
275	0\\
276	0\\
277	0\\
278	0\\
279	0\\
280	0\\
281	0\\
282	0\\
283	0\\
284	0\\
285	0\\
286	0\\
287	0\\
288	0\\
289	0\\
290	0\\
291	0\\
292	0\\
293	0\\
294	0\\
295	0\\
296	0\\
297	0\\
298	0\\
299	0\\
300	0\\
301	0\\
302	0\\
303	0\\
304	0\\
305	0\\
306	0\\
307	0\\
308	0\\
309	0\\
310	0\\
311	0\\
312	0\\
313	0\\
314	0\\
315	0\\
316	0\\
317	0\\
318	0\\
319	0\\
320	0\\
321	0\\
322	0\\
323	0\\
324	0\\
325	0\\
326	0\\
327	0\\
328	0\\
329	0\\
330	0\\
331	0\\
332	0\\
333	0\\
334	0\\
335	0\\
336	0\\
337	0\\
338	0\\
339	0\\
340	0\\
341	0\\
342	0\\
343	0\\
344	0\\
345	0\\
346	0\\
347	0\\
348	0\\
349	0\\
350	0\\
351	0\\
352	0\\
353	0\\
354	0\\
355	0\\
356	0\\
357	0\\
358	0\\
359	0\\
360	0\\
361	0\\
362	0\\
363	0\\
364	0\\
365	0\\
366	0\\
367	0\\
368	0\\
369	0\\
370	0\\
371	0\\
372	0\\
373	0\\
374	0\\
375	0\\
376	0\\
377	0\\
378	0\\
379	0\\
380	0\\
381	0\\
382	0\\
383	0\\
384	0\\
385	0\\
386	0\\
387	0\\
388	0\\
389	0\\
390	0\\
391	0\\
392	0\\
393	0\\
394	0\\
395	0\\
396	0\\
397	0\\
398	0\\
399	0\\
400	0\\
401	0\\
402	0\\
403	0\\
404	0\\
405	0\\
406	0\\
407	0\\
408	0\\
409	0\\
410	0\\
411	0\\
412	0\\
413	0\\
414	0\\
415	0\\
416	0\\
417	0\\
418	0\\
419	0\\
420	0\\
421	0\\
422	0\\
423	0\\
424	0\\
425	0\\
426	0\\
427	0\\
428	0\\
429	0\\
430	0\\
431	0\\
432	0\\
433	0\\
434	0\\
435	0\\
436	0\\
437	0\\
438	0\\
439	0\\
440	0\\
441	0\\
442	0\\
443	0\\
444	0\\
445	0\\
446	0\\
447	0\\
448	0\\
449	0\\
450	0\\
451	0\\
452	0\\
453	0\\
454	0\\
455	0\\
456	0\\
457	0\\
458	0\\
459	0\\
460	0\\
461	0\\
462	0\\
463	0\\
464	0\\
465	0\\
466	0\\
467	0\\
468	0\\
469	0\\
470	0\\
471	0\\
472	0\\
473	0\\
474	0\\
475	0\\
476	0\\
477	0\\
478	0\\
479	0\\
480	0\\
481	0\\
482	0\\
483	0\\
484	0\\
485	0\\
486	0\\
487	0\\
488	0\\
489	0\\
490	0\\
491	0\\
492	0\\
493	0\\
494	0\\
495	0\\
496	0\\
497	0\\
498	0\\
499	0\\
500	0\\
501	0\\
502	0\\
503	0\\
504	0\\
505	0\\
506	0\\
507	0\\
508	0\\
509	0\\
510	0\\
511	0\\
512	0\\
513	0\\
514	0\\
515	0\\
516	0\\
517	0\\
518	0\\
519	0\\
520	0\\
521	0\\
522	0\\
523	0\\
524	0\\
525	0\\
526	0\\
527	0\\
528	0\\
529	0\\
530	0\\
531	0\\
532	0\\
533	0\\
534	0\\
535	0\\
536	0\\
537	0\\
538	0\\
539	0\\
540	0\\
541	0\\
542	2.04853442762617e-05\\
543	6.06220385510295e-05\\
544	9.64802235070158e-05\\
545	0.000131320163008711\\
546	0.000165361460811259\\
547	0.000199053503989163\\
548	0.000232946229837191\\
549	0.00026724254469209\\
550	0.000302038372857046\\
551	0.000337433066059789\\
552	0.000373524405103253\\
553	0.000410607915064151\\
554	0.000450645605495247\\
555	0.000490655083798201\\
556	0.000530561602653835\\
557	0.000568219493923374\\
558	0.000606305612849557\\
559	0.000645024893172913\\
560	0.000684358053097023\\
561	0.000724443518312681\\
562	0.000765389047441491\\
563	0.000807225433839359\\
564	0.000849982974356452\\
565	0.000893867927586915\\
566	0.000944065882789967\\
567	0.000993239584238119\\
568	0.00103619979679305\\
569	0.00107953810792023\\
570	0.00112318247816187\\
571	0.00116757410108689\\
572	0.00121282583955014\\
573	0.00154610456066244\\
574	0.00206173146214074\\
575	0.00217000160705326\\
576	0.00225943058102288\\
577	0.00233065932806681\\
578	0.00240277285646767\\
579	0.00247604740747313\\
580	0.00255051340445631\\
581	0.00262619770980058\\
582	0.00270312784773939\\
583	0.00278133241566772\\
584	0.00286084158979877\\
585	0.00294168836537039\\
586	0.00302391178438079\\
587	0.00310756536383239\\
588	0.00319273906101823\\
589	0.00327961639013216\\
590	0.00336862265377154\\
591	0.00346039684658504\\
592	0.00355653402929194\\
593	0.00366119131364762\\
594	0.00378527046594782\\
595	0.00395742119285503\\
596	0.00425297862742962\\
597	0.0048700418989439\\
598	0.00632942537858856\\
599	0\\
600	0\\
};
\addplot [color=red!80!mycolor19,solid,forget plot]
  table[row sep=crcr]{%
1	0\\
2	0\\
3	0\\
4	0\\
5	0\\
6	0\\
7	0\\
8	0\\
9	0\\
10	0\\
11	0\\
12	0\\
13	0\\
14	0\\
15	0\\
16	0\\
17	0\\
18	0\\
19	0\\
20	0\\
21	0\\
22	0\\
23	0\\
24	0\\
25	0\\
26	0\\
27	0\\
28	0\\
29	0\\
30	0\\
31	0\\
32	0\\
33	0\\
34	0\\
35	0\\
36	0\\
37	0\\
38	0\\
39	0\\
40	0\\
41	0\\
42	0\\
43	0\\
44	0\\
45	0\\
46	0\\
47	0\\
48	0\\
49	0\\
50	0\\
51	0\\
52	0\\
53	0\\
54	0\\
55	0\\
56	0\\
57	0\\
58	0\\
59	0\\
60	0\\
61	0\\
62	0\\
63	0\\
64	0\\
65	0\\
66	0\\
67	0\\
68	0\\
69	0\\
70	0\\
71	0\\
72	0\\
73	0\\
74	0\\
75	0\\
76	0\\
77	0\\
78	0\\
79	0\\
80	0\\
81	0\\
82	0\\
83	0\\
84	0\\
85	0\\
86	0\\
87	0\\
88	0\\
89	0\\
90	0\\
91	0\\
92	0\\
93	0\\
94	0\\
95	0\\
96	0\\
97	0\\
98	0\\
99	0\\
100	0\\
101	0\\
102	0\\
103	0\\
104	0\\
105	0\\
106	0\\
107	0\\
108	0\\
109	0\\
110	0\\
111	0\\
112	0\\
113	0\\
114	0\\
115	0\\
116	0\\
117	0\\
118	0\\
119	0\\
120	0\\
121	0\\
122	0\\
123	0\\
124	0\\
125	0\\
126	0\\
127	0\\
128	0\\
129	0\\
130	0\\
131	0\\
132	0\\
133	0\\
134	0\\
135	0\\
136	0\\
137	0\\
138	0\\
139	0\\
140	0\\
141	0\\
142	0\\
143	0\\
144	0\\
145	0\\
146	0\\
147	0\\
148	0\\
149	0\\
150	0\\
151	0\\
152	0\\
153	0\\
154	0\\
155	0\\
156	0\\
157	0\\
158	0\\
159	0\\
160	0\\
161	0\\
162	0\\
163	0\\
164	0\\
165	0\\
166	0\\
167	0\\
168	0\\
169	0\\
170	0\\
171	0\\
172	0\\
173	0\\
174	0\\
175	0\\
176	0\\
177	0\\
178	0\\
179	0\\
180	0\\
181	0\\
182	0\\
183	0\\
184	0\\
185	0\\
186	0\\
187	0\\
188	0\\
189	0\\
190	0\\
191	0\\
192	0\\
193	0\\
194	0\\
195	0\\
196	0\\
197	0\\
198	0\\
199	0\\
200	0\\
201	0\\
202	0\\
203	0\\
204	0\\
205	0\\
206	0\\
207	0\\
208	0\\
209	0\\
210	0\\
211	0\\
212	0\\
213	0\\
214	0\\
215	0\\
216	0\\
217	0\\
218	0\\
219	0\\
220	0\\
221	0\\
222	0\\
223	0\\
224	0\\
225	0\\
226	0\\
227	0\\
228	0\\
229	0\\
230	0\\
231	0\\
232	0\\
233	0\\
234	0\\
235	0\\
236	0\\
237	0\\
238	0\\
239	0\\
240	0\\
241	0\\
242	0\\
243	0\\
244	0\\
245	0\\
246	0\\
247	0\\
248	0\\
249	0\\
250	0\\
251	0\\
252	0\\
253	0\\
254	0\\
255	0\\
256	0\\
257	0\\
258	0\\
259	0\\
260	0\\
261	0\\
262	0\\
263	0\\
264	0\\
265	0\\
266	0\\
267	0\\
268	0\\
269	0\\
270	0\\
271	0\\
272	0\\
273	0\\
274	0\\
275	0\\
276	0\\
277	0\\
278	0\\
279	0\\
280	0\\
281	0\\
282	0\\
283	0\\
284	0\\
285	0\\
286	0\\
287	0\\
288	0\\
289	0\\
290	0\\
291	0\\
292	0\\
293	0\\
294	0\\
295	0\\
296	0\\
297	0\\
298	0\\
299	0\\
300	0\\
301	0\\
302	0\\
303	0\\
304	0\\
305	0\\
306	0\\
307	0\\
308	0\\
309	0\\
310	0\\
311	0\\
312	0\\
313	0\\
314	0\\
315	0\\
316	0\\
317	0\\
318	0\\
319	0\\
320	0\\
321	0\\
322	0\\
323	0\\
324	0\\
325	0\\
326	0\\
327	0\\
328	0\\
329	0\\
330	0\\
331	0\\
332	0\\
333	0\\
334	0\\
335	0\\
336	0\\
337	0\\
338	0\\
339	0\\
340	0\\
341	0\\
342	0\\
343	0\\
344	0\\
345	0\\
346	0\\
347	0\\
348	0\\
349	0\\
350	0\\
351	0\\
352	0\\
353	0\\
354	0\\
355	0\\
356	0\\
357	0\\
358	0\\
359	0\\
360	0\\
361	0\\
362	0\\
363	0\\
364	0\\
365	0\\
366	0\\
367	0\\
368	0\\
369	0\\
370	0\\
371	0\\
372	0\\
373	0\\
374	0\\
375	0\\
376	0\\
377	0\\
378	0\\
379	0\\
380	0\\
381	0\\
382	0\\
383	0\\
384	0\\
385	0\\
386	0\\
387	0\\
388	0\\
389	0\\
390	0\\
391	0\\
392	0\\
393	0\\
394	0\\
395	0\\
396	0\\
397	0\\
398	0\\
399	0\\
400	0\\
401	0\\
402	0\\
403	0\\
404	0\\
405	0\\
406	0\\
407	0\\
408	0\\
409	0\\
410	0\\
411	0\\
412	0\\
413	0\\
414	0\\
415	0\\
416	0\\
417	0\\
418	0\\
419	0\\
420	0\\
421	0\\
422	0\\
423	0\\
424	0\\
425	0\\
426	0\\
427	0\\
428	0\\
429	0\\
430	0\\
431	0\\
432	0\\
433	0\\
434	0\\
435	0\\
436	0\\
437	0\\
438	0\\
439	0\\
440	0\\
441	0\\
442	0\\
443	0\\
444	0\\
445	0\\
446	0\\
447	0\\
448	0\\
449	0\\
450	0\\
451	0\\
452	0\\
453	0\\
454	0\\
455	0\\
456	0\\
457	0\\
458	0\\
459	0\\
460	0\\
461	0\\
462	0\\
463	0\\
464	0\\
465	0\\
466	0\\
467	0\\
468	0\\
469	0\\
470	0\\
471	0\\
472	0\\
473	0\\
474	0\\
475	0\\
476	0\\
477	0\\
478	0\\
479	0\\
480	0\\
481	0\\
482	0\\
483	0\\
484	0\\
485	0\\
486	0\\
487	0\\
488	0\\
489	0\\
490	0\\
491	0\\
492	0\\
493	0\\
494	0\\
495	0\\
496	0\\
497	0\\
498	0\\
499	0\\
500	0\\
501	0\\
502	0\\
503	0\\
504	0\\
505	0\\
506	0\\
507	0\\
508	0\\
509	0\\
510	0\\
511	0\\
512	0\\
513	0\\
514	0\\
515	0\\
516	0\\
517	0\\
518	0\\
519	0\\
520	0\\
521	0\\
522	0\\
523	0\\
524	0\\
525	0\\
526	0\\
527	0\\
528	0\\
529	0\\
530	0\\
531	0\\
532	0\\
533	0\\
534	0\\
535	0\\
536	0\\
537	0\\
538	0\\
539	0\\
540	4.09915539253185e-06\\
541	3.50855341422652e-05\\
542	6.56241650374363e-05\\
543	9.62275076458695e-05\\
544	0.000127190880966168\\
545	0.000158606315656507\\
546	0.000190566750982985\\
547	0.000223147011209739\\
548	0.000256388207954706\\
549	0.000290340966566383\\
550	0.000325350059729218\\
551	0.00036325193318367\\
552	0.000401173491610415\\
553	0.000438840853802339\\
554	0.000474293154902854\\
555	0.000510160152892572\\
556	0.000546586920712631\\
557	0.000583562655249454\\
558	0.000621307465471656\\
559	0.00065986095225696\\
560	0.000699251876427312\\
561	0.0007395059311949\\
562	0.000780644872361251\\
563	0.000822720633982572\\
564	0.000870893647640451\\
565	0.000918570800930397\\
566	0.000959929760329168\\
567	0.00100162913418252\\
568	0.00104358144557724\\
569	0.00108631367623227\\
570	0.0011298652614946\\
571	0.00117427238843636\\
572	0.00171542356870662\\
573	0.00201939825841623\\
574	0.00212061739353802\\
575	0.0021898538250397\\
576	0.00225970681651282\\
577	0.00233067087660055\\
578	0.00240277681456536\\
579	0.00247604952405764\\
580	0.00255051455217948\\
581	0.00262619830282023\\
582	0.00270312813271299\\
583	0.00278133254043779\\
584	0.00286084163829612\\
585	0.00294168838147584\\
586	0.00302391178867891\\
587	0.00310756536465452\\
588	0.00319273906110311\\
589	0.00327961639013215\\
590	0.00336862265377152\\
591	0.00346039684658503\\
592	0.00355653402929193\\
593	0.00366119131364762\\
594	0.00378527046594782\\
595	0.00395742119285502\\
596	0.00425297862742962\\
597	0.0048700418989439\\
598	0.00632942537858856\\
599	0\\
600	0\\
};
\addplot [color=red,solid,forget plot]
  table[row sep=crcr]{%
1	0\\
2	0\\
3	0\\
4	0\\
5	0\\
6	0\\
7	0\\
8	0\\
9	0\\
10	0\\
11	0\\
12	0\\
13	0\\
14	0\\
15	0\\
16	0\\
17	0\\
18	0\\
19	0\\
20	0\\
21	0\\
22	0\\
23	0\\
24	0\\
25	0\\
26	0\\
27	0\\
28	0\\
29	0\\
30	0\\
31	0\\
32	0\\
33	0\\
34	0\\
35	0\\
36	0\\
37	0\\
38	0\\
39	0\\
40	0\\
41	0\\
42	0\\
43	0\\
44	0\\
45	0\\
46	0\\
47	0\\
48	0\\
49	0\\
50	0\\
51	0\\
52	0\\
53	0\\
54	0\\
55	0\\
56	0\\
57	0\\
58	0\\
59	0\\
60	0\\
61	0\\
62	0\\
63	0\\
64	0\\
65	0\\
66	0\\
67	0\\
68	0\\
69	0\\
70	0\\
71	0\\
72	0\\
73	0\\
74	0\\
75	0\\
76	0\\
77	0\\
78	0\\
79	0\\
80	0\\
81	0\\
82	0\\
83	0\\
84	0\\
85	0\\
86	0\\
87	0\\
88	0\\
89	0\\
90	0\\
91	0\\
92	0\\
93	0\\
94	0\\
95	0\\
96	0\\
97	0\\
98	0\\
99	0\\
100	0\\
101	0\\
102	0\\
103	0\\
104	0\\
105	0\\
106	0\\
107	0\\
108	0\\
109	0\\
110	0\\
111	0\\
112	0\\
113	0\\
114	0\\
115	0\\
116	0\\
117	0\\
118	0\\
119	0\\
120	0\\
121	0\\
122	0\\
123	0\\
124	0\\
125	0\\
126	0\\
127	0\\
128	0\\
129	0\\
130	0\\
131	0\\
132	0\\
133	0\\
134	0\\
135	0\\
136	0\\
137	0\\
138	0\\
139	0\\
140	0\\
141	0\\
142	0\\
143	0\\
144	0\\
145	0\\
146	0\\
147	0\\
148	0\\
149	0\\
150	0\\
151	0\\
152	0\\
153	0\\
154	0\\
155	0\\
156	0\\
157	0\\
158	0\\
159	0\\
160	0\\
161	0\\
162	0\\
163	0\\
164	0\\
165	0\\
166	0\\
167	0\\
168	0\\
169	0\\
170	0\\
171	0\\
172	0\\
173	0\\
174	0\\
175	0\\
176	0\\
177	0\\
178	0\\
179	0\\
180	0\\
181	0\\
182	0\\
183	0\\
184	0\\
185	0\\
186	0\\
187	0\\
188	0\\
189	0\\
190	0\\
191	0\\
192	0\\
193	0\\
194	0\\
195	0\\
196	0\\
197	0\\
198	0\\
199	0\\
200	0\\
201	0\\
202	0\\
203	0\\
204	0\\
205	0\\
206	0\\
207	0\\
208	0\\
209	0\\
210	0\\
211	0\\
212	0\\
213	0\\
214	0\\
215	0\\
216	0\\
217	0\\
218	0\\
219	0\\
220	0\\
221	0\\
222	0\\
223	0\\
224	0\\
225	0\\
226	0\\
227	0\\
228	0\\
229	0\\
230	0\\
231	0\\
232	0\\
233	0\\
234	0\\
235	0\\
236	0\\
237	0\\
238	0\\
239	0\\
240	0\\
241	0\\
242	0\\
243	0\\
244	0\\
245	0\\
246	0\\
247	0\\
248	0\\
249	0\\
250	0\\
251	0\\
252	0\\
253	0\\
254	0\\
255	0\\
256	0\\
257	0\\
258	0\\
259	0\\
260	0\\
261	0\\
262	0\\
263	0\\
264	0\\
265	0\\
266	0\\
267	0\\
268	0\\
269	0\\
270	0\\
271	0\\
272	0\\
273	0\\
274	0\\
275	0\\
276	0\\
277	0\\
278	0\\
279	0\\
280	0\\
281	0\\
282	0\\
283	0\\
284	0\\
285	0\\
286	0\\
287	0\\
288	0\\
289	0\\
290	0\\
291	0\\
292	0\\
293	0\\
294	0\\
295	0\\
296	0\\
297	0\\
298	0\\
299	0\\
300	0\\
301	0\\
302	0\\
303	0\\
304	0\\
305	0\\
306	0\\
307	0\\
308	0\\
309	0\\
310	0\\
311	0\\
312	0\\
313	0\\
314	0\\
315	0\\
316	0\\
317	0\\
318	0\\
319	0\\
320	0\\
321	0\\
322	0\\
323	0\\
324	0\\
325	0\\
326	0\\
327	0\\
328	0\\
329	0\\
330	0\\
331	0\\
332	0\\
333	0\\
334	0\\
335	0\\
336	0\\
337	0\\
338	0\\
339	0\\
340	0\\
341	0\\
342	0\\
343	0\\
344	0\\
345	0\\
346	0\\
347	0\\
348	0\\
349	0\\
350	0\\
351	0\\
352	0\\
353	0\\
354	0\\
355	0\\
356	0\\
357	0\\
358	0\\
359	0\\
360	0\\
361	0\\
362	0\\
363	0\\
364	0\\
365	0\\
366	0\\
367	0\\
368	0\\
369	0\\
370	0\\
371	0\\
372	0\\
373	0\\
374	0\\
375	0\\
376	0\\
377	0\\
378	0\\
379	0\\
380	0\\
381	0\\
382	0\\
383	0\\
384	0\\
385	0\\
386	0\\
387	0\\
388	0\\
389	0\\
390	0\\
391	0\\
392	0\\
393	0\\
394	0\\
395	0\\
396	0\\
397	0\\
398	0\\
399	0\\
400	0\\
401	0\\
402	0\\
403	0\\
404	0\\
405	0\\
406	0\\
407	0\\
408	0\\
409	0\\
410	0\\
411	0\\
412	0\\
413	0\\
414	0\\
415	0\\
416	0\\
417	0\\
418	0\\
419	0\\
420	0\\
421	0\\
422	0\\
423	0\\
424	0\\
425	0\\
426	0\\
427	0\\
428	0\\
429	0\\
430	0\\
431	0\\
432	0\\
433	0\\
434	0\\
435	0\\
436	0\\
437	0\\
438	0\\
439	0\\
440	0\\
441	0\\
442	0\\
443	0\\
444	0\\
445	0\\
446	0\\
447	0\\
448	0\\
449	0\\
450	0\\
451	0\\
452	0\\
453	0\\
454	0\\
455	0\\
456	0\\
457	0\\
458	0\\
459	0\\
460	0\\
461	0\\
462	0\\
463	0\\
464	0\\
465	0\\
466	0\\
467	0\\
468	0\\
469	0\\
470	0\\
471	0\\
472	0\\
473	0\\
474	0\\
475	0\\
476	0\\
477	0\\
478	0\\
479	0\\
480	0\\
481	0\\
482	0\\
483	0\\
484	0\\
485	0\\
486	0\\
487	0\\
488	0\\
489	0\\
490	0\\
491	0\\
492	0\\
493	0\\
494	0\\
495	0\\
496	0\\
497	0\\
498	0\\
499	0\\
500	0\\
501	0\\
502	0\\
503	0\\
504	0\\
505	0\\
506	0\\
507	0\\
508	0\\
509	0\\
510	0\\
511	0\\
512	0\\
513	0\\
514	0\\
515	0\\
516	0\\
517	0\\
518	0\\
519	0\\
520	0\\
521	0\\
522	0\\
523	0\\
524	0\\
525	0\\
526	0\\
527	0\\
528	0\\
529	0\\
530	0\\
531	0\\
532	0\\
533	0\\
534	0\\
535	0\\
536	0\\
537	0\\
538	0\\
539	0\\
540	2.72097330305856e-05\\
541	5.61210861900328e-05\\
542	8.5585431440554e-05\\
543	0.000115648715291517\\
544	0.000146340270906872\\
545	0.000177684940527103\\
546	0.000209720256944281\\
547	0.000242556230441567\\
548	0.000278557139405352\\
549	0.000314550976192727\\
550	0.00035020981117787\\
551	0.000383724276119911\\
552	0.000417754807534627\\
553	0.000452289907333273\\
554	0.000487256054489089\\
555	0.000522838353760378\\
556	0.000559179818522669\\
557	0.000596309200618129\\
558	0.000634247900431846\\
559	0.000673017586461841\\
560	0.00071263931167027\\
561	0.000753150047157175\\
562	0.000798519250497241\\
563	0.000845250121868294\\
564	0.000885803380539406\\
565	0.000926054546725999\\
566	0.00096653431220477\\
567	0.00100777034248639\\
568	0.00104980318090226\\
569	0.0010926538929629\\
570	0.00126482494133247\\
571	0.00184533543969108\\
572	0.0019701721867511\\
573	0.00205342365444128\\
574	0.00212111322836602\\
575	0.00218986312710426\\
576	0.0022597081208049\\
577	0.0023306715146185\\
578	0.00240277715940921\\
579	0.0024760497051662\\
580	0.00255051464193462\\
581	0.00262619834400932\\
582	0.00270312814987377\\
583	0.00278133254676196\\
584	0.0028608416402799\\
585	0.00294168838197388\\
586	0.00302391178876813\\
587	0.00310756536466312\\
588	0.00319273906110311\\
589	0.00327961639013215\\
590	0.00336862265377153\\
591	0.00346039684658503\\
592	0.00355653402929193\\
593	0.00366119131364763\\
594	0.00378527046594782\\
595	0.00395742119285503\\
596	0.00425297862742962\\
597	0.0048700418989439\\
598	0.00632942537858856\\
599	0\\
600	0\\
};
\addplot [color=mycolor20,solid,forget plot]
  table[row sep=crcr]{%
1	0\\
2	0\\
3	0\\
4	0\\
5	0\\
6	0\\
7	0\\
8	0\\
9	0\\
10	0\\
11	0\\
12	0\\
13	0\\
14	0\\
15	0\\
16	0\\
17	0\\
18	0\\
19	0\\
20	0\\
21	0\\
22	0\\
23	0\\
24	0\\
25	0\\
26	0\\
27	0\\
28	0\\
29	0\\
30	0\\
31	0\\
32	0\\
33	0\\
34	0\\
35	0\\
36	0\\
37	0\\
38	0\\
39	0\\
40	0\\
41	0\\
42	0\\
43	0\\
44	0\\
45	0\\
46	0\\
47	0\\
48	0\\
49	0\\
50	0\\
51	0\\
52	0\\
53	0\\
54	0\\
55	0\\
56	0\\
57	0\\
58	0\\
59	0\\
60	0\\
61	0\\
62	0\\
63	0\\
64	0\\
65	0\\
66	0\\
67	0\\
68	0\\
69	0\\
70	0\\
71	0\\
72	0\\
73	0\\
74	0\\
75	0\\
76	0\\
77	0\\
78	0\\
79	0\\
80	0\\
81	0\\
82	0\\
83	0\\
84	0\\
85	0\\
86	0\\
87	0\\
88	0\\
89	0\\
90	0\\
91	0\\
92	0\\
93	0\\
94	0\\
95	0\\
96	0\\
97	0\\
98	0\\
99	0\\
100	0\\
101	0\\
102	0\\
103	0\\
104	0\\
105	0\\
106	0\\
107	0\\
108	0\\
109	0\\
110	0\\
111	0\\
112	0\\
113	0\\
114	0\\
115	0\\
116	0\\
117	0\\
118	0\\
119	0\\
120	0\\
121	0\\
122	0\\
123	0\\
124	0\\
125	0\\
126	0\\
127	0\\
128	0\\
129	0\\
130	0\\
131	0\\
132	0\\
133	0\\
134	0\\
135	0\\
136	0\\
137	0\\
138	0\\
139	0\\
140	0\\
141	0\\
142	0\\
143	0\\
144	0\\
145	0\\
146	0\\
147	0\\
148	0\\
149	0\\
150	0\\
151	0\\
152	0\\
153	0\\
154	0\\
155	0\\
156	0\\
157	0\\
158	0\\
159	0\\
160	0\\
161	0\\
162	0\\
163	0\\
164	0\\
165	0\\
166	0\\
167	0\\
168	0\\
169	0\\
170	0\\
171	0\\
172	0\\
173	0\\
174	0\\
175	0\\
176	0\\
177	0\\
178	0\\
179	0\\
180	0\\
181	0\\
182	0\\
183	0\\
184	0\\
185	0\\
186	0\\
187	0\\
188	0\\
189	0\\
190	0\\
191	0\\
192	0\\
193	0\\
194	0\\
195	0\\
196	0\\
197	0\\
198	0\\
199	0\\
200	0\\
201	0\\
202	0\\
203	0\\
204	0\\
205	0\\
206	0\\
207	0\\
208	0\\
209	0\\
210	0\\
211	0\\
212	0\\
213	0\\
214	0\\
215	0\\
216	0\\
217	0\\
218	0\\
219	0\\
220	0\\
221	0\\
222	0\\
223	0\\
224	0\\
225	0\\
226	0\\
227	0\\
228	0\\
229	0\\
230	0\\
231	0\\
232	0\\
233	0\\
234	0\\
235	0\\
236	0\\
237	0\\
238	0\\
239	0\\
240	0\\
241	0\\
242	0\\
243	0\\
244	0\\
245	0\\
246	0\\
247	0\\
248	0\\
249	0\\
250	0\\
251	0\\
252	0\\
253	0\\
254	0\\
255	0\\
256	0\\
257	0\\
258	0\\
259	0\\
260	0\\
261	0\\
262	0\\
263	0\\
264	0\\
265	0\\
266	0\\
267	0\\
268	0\\
269	0\\
270	0\\
271	0\\
272	0\\
273	0\\
274	0\\
275	0\\
276	0\\
277	0\\
278	0\\
279	0\\
280	0\\
281	0\\
282	0\\
283	0\\
284	0\\
285	0\\
286	0\\
287	0\\
288	0\\
289	0\\
290	0\\
291	0\\
292	0\\
293	0\\
294	0\\
295	0\\
296	0\\
297	0\\
298	0\\
299	0\\
300	0\\
301	0\\
302	0\\
303	0\\
304	0\\
305	0\\
306	0\\
307	0\\
308	0\\
309	0\\
310	0\\
311	0\\
312	0\\
313	0\\
314	0\\
315	0\\
316	0\\
317	0\\
318	0\\
319	0\\
320	0\\
321	0\\
322	0\\
323	0\\
324	0\\
325	0\\
326	0\\
327	0\\
328	0\\
329	0\\
330	0\\
331	0\\
332	0\\
333	0\\
334	0\\
335	0\\
336	0\\
337	0\\
338	0\\
339	0\\
340	0\\
341	0\\
342	0\\
343	0\\
344	0\\
345	0\\
346	0\\
347	0\\
348	0\\
349	0\\
350	0\\
351	0\\
352	0\\
353	0\\
354	0\\
355	0\\
356	0\\
357	0\\
358	0\\
359	0\\
360	0\\
361	0\\
362	0\\
363	0\\
364	0\\
365	0\\
366	0\\
367	0\\
368	0\\
369	0\\
370	0\\
371	0\\
372	0\\
373	0\\
374	0\\
375	0\\
376	0\\
377	0\\
378	0\\
379	0\\
380	0\\
381	0\\
382	0\\
383	0\\
384	0\\
385	0\\
386	0\\
387	0\\
388	0\\
389	0\\
390	0\\
391	0\\
392	0\\
393	0\\
394	0\\
395	0\\
396	0\\
397	0\\
398	0\\
399	0\\
400	0\\
401	0\\
402	0\\
403	0\\
404	0\\
405	0\\
406	0\\
407	0\\
408	0\\
409	0\\
410	0\\
411	0\\
412	0\\
413	0\\
414	0\\
415	0\\
416	0\\
417	0\\
418	0\\
419	0\\
420	0\\
421	0\\
422	0\\
423	0\\
424	0\\
425	0\\
426	0\\
427	0\\
428	0\\
429	0\\
430	0\\
431	0\\
432	0\\
433	0\\
434	0\\
435	0\\
436	0\\
437	0\\
438	0\\
439	0\\
440	0\\
441	0\\
442	0\\
443	0\\
444	0\\
445	0\\
446	0\\
447	0\\
448	0\\
449	0\\
450	0\\
451	0\\
452	0\\
453	0\\
454	0\\
455	0\\
456	0\\
457	0\\
458	0\\
459	0\\
460	0\\
461	0\\
462	0\\
463	0\\
464	0\\
465	0\\
466	0\\
467	0\\
468	0\\
469	0\\
470	0\\
471	0\\
472	0\\
473	0\\
474	0\\
475	0\\
476	0\\
477	0\\
478	0\\
479	0\\
480	0\\
481	0\\
482	0\\
483	0\\
484	0\\
485	0\\
486	0\\
487	0\\
488	0\\
489	0\\
490	0\\
491	0\\
492	0\\
493	0\\
494	0\\
495	0\\
496	0\\
497	0\\
498	0\\
499	0\\
500	0\\
501	0\\
502	0\\
503	0\\
504	0\\
505	0\\
506	0\\
507	0\\
508	0\\
509	0\\
510	0\\
511	0\\
512	0\\
513	0\\
514	0\\
515	0\\
516	0\\
517	0\\
518	0\\
519	0\\
520	0\\
521	0\\
522	0\\
523	0\\
524	0\\
525	0\\
526	0\\
527	0\\
528	0\\
529	0\\
530	0\\
531	0\\
532	0\\
533	0\\
534	0\\
535	0\\
536	0\\
537	0\\
538	0\\
539	1.49492185966091e-05\\
540	4.3322083531691e-05\\
541	7.23064945701987e-05\\
542	0.000101919514130588\\
543	0.000132186799457533\\
544	0.000163148555289638\\
545	0.000196899412872687\\
546	0.000231163070872471\\
547	0.000265280313724257\\
548	0.000296939134999868\\
549	0.000329084648366537\\
550	0.000361719670533462\\
551	0.000394828145692352\\
552	0.000428606969412651\\
553	0.000463052122313129\\
554	0.000498111370312253\\
555	0.000533863357465193\\
556	0.00057039128795026\\
557	0.000607714878606064\\
558	0.000645854512820227\\
559	0.000684829623640465\\
560	0.000726776106591716\\
561	0.000772673465119796\\
562	0.000813534509467145\\
563	0.000852431652279271\\
564	0.000891516665074081\\
565	0.000931278020432426\\
566	0.000971798618077333\\
567	0.00101309731890282\\
568	0.00105519192697659\\
569	0.00134093035495452\\
570	0.00181655877875805\\
571	0.00191924729878529\\
572	0.00198681216416615\\
573	0.00205343742759807\\
574	0.00212111382330841\\
575	0.00218986332242957\\
576	0.00225970822322969\\
577	0.00233067156883697\\
578	0.00240277718689623\\
579	0.00247604971824475\\
580	0.00255051464767741\\
581	0.00262619834629172\\
582	0.0027031281506736\\
583	0.00278133254699977\\
584	0.00286084164033628\\
585	0.00294168838198338\\
586	0.00302391178876899\\
587	0.00310756536466312\\
588	0.00319273906110312\\
589	0.00327961639013216\\
590	0.00336862265377153\\
591	0.00346039684658503\\
592	0.00355653402929193\\
593	0.00366119131364763\\
594	0.00378527046594782\\
595	0.00395742119285503\\
596	0.00425297862742962\\
597	0.0048700418989439\\
598	0.00632942537858856\\
599	0\\
600	0\\
};
\addplot [color=mycolor21,solid,forget plot]
  table[row sep=crcr]{%
1	0\\
2	0\\
3	0\\
4	0\\
5	0\\
6	0\\
7	0\\
8	0\\
9	0\\
10	0\\
11	0\\
12	0\\
13	0\\
14	0\\
15	0\\
16	0\\
17	0\\
18	0\\
19	0\\
20	0\\
21	0\\
22	0\\
23	0\\
24	0\\
25	0\\
26	0\\
27	0\\
28	0\\
29	0\\
30	0\\
31	0\\
32	0\\
33	0\\
34	0\\
35	0\\
36	0\\
37	0\\
38	0\\
39	0\\
40	0\\
41	0\\
42	0\\
43	0\\
44	0\\
45	0\\
46	0\\
47	0\\
48	0\\
49	0\\
50	0\\
51	0\\
52	0\\
53	0\\
54	0\\
55	0\\
56	0\\
57	0\\
58	0\\
59	0\\
60	0\\
61	0\\
62	0\\
63	0\\
64	0\\
65	0\\
66	0\\
67	0\\
68	0\\
69	0\\
70	0\\
71	0\\
72	0\\
73	0\\
74	0\\
75	0\\
76	0\\
77	0\\
78	0\\
79	0\\
80	0\\
81	0\\
82	0\\
83	0\\
84	0\\
85	0\\
86	0\\
87	0\\
88	0\\
89	0\\
90	0\\
91	0\\
92	0\\
93	0\\
94	0\\
95	0\\
96	0\\
97	0\\
98	0\\
99	0\\
100	0\\
101	0\\
102	0\\
103	0\\
104	0\\
105	0\\
106	0\\
107	0\\
108	0\\
109	0\\
110	0\\
111	0\\
112	0\\
113	0\\
114	0\\
115	0\\
116	0\\
117	0\\
118	0\\
119	0\\
120	0\\
121	0\\
122	0\\
123	0\\
124	0\\
125	0\\
126	0\\
127	0\\
128	0\\
129	0\\
130	0\\
131	0\\
132	0\\
133	0\\
134	0\\
135	0\\
136	0\\
137	0\\
138	0\\
139	0\\
140	0\\
141	0\\
142	0\\
143	0\\
144	0\\
145	0\\
146	0\\
147	0\\
148	0\\
149	0\\
150	0\\
151	0\\
152	0\\
153	0\\
154	0\\
155	0\\
156	0\\
157	0\\
158	0\\
159	0\\
160	0\\
161	0\\
162	0\\
163	0\\
164	0\\
165	0\\
166	0\\
167	0\\
168	0\\
169	0\\
170	0\\
171	0\\
172	0\\
173	0\\
174	0\\
175	0\\
176	0\\
177	0\\
178	0\\
179	0\\
180	0\\
181	0\\
182	0\\
183	0\\
184	0\\
185	0\\
186	0\\
187	0\\
188	0\\
189	0\\
190	0\\
191	0\\
192	0\\
193	0\\
194	0\\
195	0\\
196	0\\
197	0\\
198	0\\
199	0\\
200	0\\
201	0\\
202	0\\
203	0\\
204	0\\
205	0\\
206	0\\
207	0\\
208	0\\
209	0\\
210	0\\
211	0\\
212	0\\
213	0\\
214	0\\
215	0\\
216	0\\
217	0\\
218	0\\
219	0\\
220	0\\
221	0\\
222	0\\
223	0\\
224	0\\
225	0\\
226	0\\
227	0\\
228	0\\
229	0\\
230	0\\
231	0\\
232	0\\
233	0\\
234	0\\
235	0\\
236	0\\
237	0\\
238	0\\
239	0\\
240	0\\
241	0\\
242	0\\
243	0\\
244	0\\
245	0\\
246	0\\
247	0\\
248	0\\
249	0\\
250	0\\
251	0\\
252	0\\
253	0\\
254	0\\
255	0\\
256	0\\
257	0\\
258	0\\
259	0\\
260	0\\
261	0\\
262	0\\
263	0\\
264	0\\
265	0\\
266	0\\
267	0\\
268	0\\
269	0\\
270	0\\
271	0\\
272	0\\
273	0\\
274	0\\
275	0\\
276	0\\
277	0\\
278	0\\
279	0\\
280	0\\
281	0\\
282	0\\
283	0\\
284	0\\
285	0\\
286	0\\
287	0\\
288	0\\
289	0\\
290	0\\
291	0\\
292	0\\
293	0\\
294	0\\
295	0\\
296	0\\
297	0\\
298	0\\
299	0\\
300	0\\
301	0\\
302	0\\
303	0\\
304	0\\
305	0\\
306	0\\
307	0\\
308	0\\
309	0\\
310	0\\
311	0\\
312	0\\
313	0\\
314	0\\
315	0\\
316	0\\
317	0\\
318	0\\
319	0\\
320	0\\
321	0\\
322	0\\
323	0\\
324	0\\
325	0\\
326	0\\
327	0\\
328	0\\
329	0\\
330	0\\
331	0\\
332	0\\
333	0\\
334	0\\
335	0\\
336	0\\
337	0\\
338	0\\
339	0\\
340	0\\
341	0\\
342	0\\
343	0\\
344	0\\
345	0\\
346	0\\
347	0\\
348	0\\
349	0\\
350	0\\
351	0\\
352	0\\
353	0\\
354	0\\
355	0\\
356	0\\
357	0\\
358	0\\
359	0\\
360	0\\
361	0\\
362	0\\
363	0\\
364	0\\
365	0\\
366	0\\
367	0\\
368	0\\
369	0\\
370	0\\
371	0\\
372	0\\
373	0\\
374	0\\
375	0\\
376	0\\
377	0\\
378	0\\
379	0\\
380	0\\
381	0\\
382	0\\
383	0\\
384	0\\
385	0\\
386	0\\
387	0\\
388	0\\
389	0\\
390	0\\
391	0\\
392	0\\
393	0\\
394	0\\
395	0\\
396	0\\
397	0\\
398	0\\
399	0\\
400	0\\
401	0\\
402	0\\
403	0\\
404	0\\
405	0\\
406	0\\
407	0\\
408	0\\
409	0\\
410	0\\
411	0\\
412	0\\
413	0\\
414	0\\
415	0\\
416	0\\
417	0\\
418	0\\
419	0\\
420	0\\
421	0\\
422	0\\
423	0\\
424	0\\
425	0\\
426	0\\
427	0\\
428	0\\
429	0\\
430	0\\
431	0\\
432	0\\
433	0\\
434	0\\
435	0\\
436	0\\
437	0\\
438	0\\
439	0\\
440	0\\
441	0\\
442	0\\
443	0\\
444	0\\
445	0\\
446	0\\
447	0\\
448	0\\
449	0\\
450	0\\
451	0\\
452	0\\
453	0\\
454	0\\
455	0\\
456	0\\
457	0\\
458	0\\
459	0\\
460	0\\
461	0\\
462	0\\
463	0\\
464	0\\
465	0\\
466	0\\
467	0\\
468	0\\
469	0\\
470	0\\
471	0\\
472	0\\
473	0\\
474	0\\
475	0\\
476	0\\
477	0\\
478	0\\
479	0\\
480	0\\
481	0\\
482	0\\
483	0\\
484	0\\
485	0\\
486	0\\
487	0\\
488	0\\
489	0\\
490	0\\
491	0\\
492	0\\
493	0\\
494	0\\
495	0\\
496	0\\
497	0\\
498	0\\
499	0\\
500	0\\
501	0\\
502	0\\
503	0\\
504	0\\
505	0\\
506	0\\
507	0\\
508	0\\
509	0\\
510	0\\
511	0\\
512	0\\
513	0\\
514	0\\
515	0\\
516	0\\
517	0\\
518	0\\
519	0\\
520	0\\
521	0\\
522	0\\
523	0\\
524	0\\
525	0\\
526	0\\
527	0\\
528	0\\
529	0\\
530	0\\
531	0\\
532	0\\
533	0\\
534	0\\
535	0\\
536	0\\
537	0\\
538	1.21694107886508e-06\\
539	2.92045651449222e-05\\
540	5.78049456237067e-05\\
541	8.70536593544758e-05\\
542	0.000118312744446052\\
543	0.000151016328109347\\
544	0.000183614512075941\\
545	0.000214055086743375\\
546	0.000244429244110815\\
547	0.000275261690455392\\
548	0.00030651277983146\\
549	0.000338404008295745\\
550	0.000370954886140724\\
551	0.00040418384758811\\
552	0.000438095852672766\\
553	0.000472675235959531\\
554	0.000507841527916458\\
555	0.000543753467495973\\
556	0.000580443856548278\\
557	0.000617931118089366\\
558	0.000656254294614929\\
559	0.000701069072514467\\
560	0.000743145599645497\\
561	0.000780787630662042\\
562	0.000818610761048887\\
563	0.000856957736854093\\
564	0.000896029558364049\\
565	0.000935845227354032\\
566	0.000976422049464571\\
567	0.00101777751913248\\
568	0.00138028712439642\\
569	0.00176338880029626\\
570	0.00185658807509324\\
571	0.00192121863126382\\
572	0.00198681271195394\\
573	0.00205343749576471\\
574	0.00212111385372883\\
575	0.00218986333840477\\
576	0.00225970823145245\\
577	0.00233067157285982\\
578	0.00240277718873636\\
579	0.00247604971901934\\
580	0.00255051464797174\\
581	0.00262619834639004\\
582	0.00270312815070138\\
583	0.00278133254700601\\
584	0.00286084164033727\\
585	0.00294168838198347\\
586	0.00302391178876899\\
587	0.00310756536466311\\
588	0.00319273906110311\\
589	0.00327961639013215\\
590	0.00336862265377153\\
591	0.00346039684658503\\
592	0.00355653402929193\\
593	0.00366119131364763\\
594	0.00378527046594783\\
595	0.00395742119285503\\
596	0.00425297862742962\\
597	0.0048700418989439\\
598	0.00632942537858856\\
599	0\\
600	0\\
};
\addplot [color=black!20!mycolor21,solid,forget plot]
  table[row sep=crcr]{%
1	0\\
2	0\\
3	0\\
4	0\\
5	0\\
6	0\\
7	0\\
8	0\\
9	0\\
10	0\\
11	0\\
12	0\\
13	0\\
14	0\\
15	0\\
16	0\\
17	0\\
18	0\\
19	0\\
20	0\\
21	0\\
22	0\\
23	0\\
24	0\\
25	0\\
26	0\\
27	0\\
28	0\\
29	0\\
30	0\\
31	0\\
32	0\\
33	0\\
34	0\\
35	0\\
36	0\\
37	0\\
38	0\\
39	0\\
40	0\\
41	0\\
42	0\\
43	0\\
44	0\\
45	0\\
46	0\\
47	0\\
48	0\\
49	0\\
50	0\\
51	0\\
52	0\\
53	0\\
54	0\\
55	0\\
56	0\\
57	0\\
58	0\\
59	0\\
60	0\\
61	0\\
62	0\\
63	0\\
64	0\\
65	0\\
66	0\\
67	0\\
68	0\\
69	0\\
70	0\\
71	0\\
72	0\\
73	0\\
74	0\\
75	0\\
76	0\\
77	0\\
78	0\\
79	0\\
80	0\\
81	0\\
82	0\\
83	0\\
84	0\\
85	0\\
86	0\\
87	0\\
88	0\\
89	0\\
90	0\\
91	0\\
92	0\\
93	0\\
94	0\\
95	0\\
96	0\\
97	0\\
98	0\\
99	0\\
100	0\\
101	0\\
102	0\\
103	0\\
104	0\\
105	0\\
106	0\\
107	0\\
108	0\\
109	0\\
110	0\\
111	0\\
112	0\\
113	0\\
114	0\\
115	0\\
116	0\\
117	0\\
118	0\\
119	0\\
120	0\\
121	0\\
122	0\\
123	0\\
124	0\\
125	0\\
126	0\\
127	0\\
128	0\\
129	0\\
130	0\\
131	0\\
132	0\\
133	0\\
134	0\\
135	0\\
136	0\\
137	0\\
138	0\\
139	0\\
140	0\\
141	0\\
142	0\\
143	0\\
144	0\\
145	0\\
146	0\\
147	0\\
148	0\\
149	0\\
150	0\\
151	0\\
152	0\\
153	0\\
154	0\\
155	0\\
156	0\\
157	0\\
158	0\\
159	0\\
160	0\\
161	0\\
162	0\\
163	0\\
164	0\\
165	0\\
166	0\\
167	0\\
168	0\\
169	0\\
170	0\\
171	0\\
172	0\\
173	0\\
174	0\\
175	0\\
176	0\\
177	0\\
178	0\\
179	0\\
180	0\\
181	0\\
182	0\\
183	0\\
184	0\\
185	0\\
186	0\\
187	0\\
188	0\\
189	0\\
190	0\\
191	0\\
192	0\\
193	0\\
194	0\\
195	0\\
196	0\\
197	0\\
198	0\\
199	0\\
200	0\\
201	0\\
202	0\\
203	0\\
204	0\\
205	0\\
206	0\\
207	0\\
208	0\\
209	0\\
210	0\\
211	0\\
212	0\\
213	0\\
214	0\\
215	0\\
216	0\\
217	0\\
218	0\\
219	0\\
220	0\\
221	0\\
222	0\\
223	0\\
224	0\\
225	0\\
226	0\\
227	0\\
228	0\\
229	0\\
230	0\\
231	0\\
232	0\\
233	0\\
234	0\\
235	0\\
236	0\\
237	0\\
238	0\\
239	0\\
240	0\\
241	0\\
242	0\\
243	0\\
244	0\\
245	0\\
246	0\\
247	0\\
248	0\\
249	0\\
250	0\\
251	0\\
252	0\\
253	0\\
254	0\\
255	0\\
256	0\\
257	0\\
258	0\\
259	0\\
260	0\\
261	0\\
262	0\\
263	0\\
264	0\\
265	0\\
266	0\\
267	0\\
268	0\\
269	0\\
270	0\\
271	0\\
272	0\\
273	0\\
274	0\\
275	0\\
276	0\\
277	0\\
278	0\\
279	0\\
280	0\\
281	0\\
282	0\\
283	0\\
284	0\\
285	0\\
286	0\\
287	0\\
288	0\\
289	0\\
290	0\\
291	0\\
292	0\\
293	0\\
294	0\\
295	0\\
296	0\\
297	0\\
298	0\\
299	0\\
300	0\\
301	0\\
302	0\\
303	0\\
304	0\\
305	0\\
306	0\\
307	0\\
308	0\\
309	0\\
310	0\\
311	0\\
312	0\\
313	0\\
314	0\\
315	0\\
316	0\\
317	0\\
318	0\\
319	0\\
320	0\\
321	0\\
322	0\\
323	0\\
324	0\\
325	0\\
326	0\\
327	0\\
328	0\\
329	0\\
330	0\\
331	0\\
332	0\\
333	0\\
334	0\\
335	0\\
336	0\\
337	0\\
338	0\\
339	0\\
340	0\\
341	0\\
342	0\\
343	0\\
344	0\\
345	0\\
346	0\\
347	0\\
348	0\\
349	0\\
350	0\\
351	0\\
352	0\\
353	0\\
354	0\\
355	0\\
356	0\\
357	0\\
358	0\\
359	0\\
360	0\\
361	0\\
362	0\\
363	0\\
364	0\\
365	0\\
366	0\\
367	0\\
368	0\\
369	0\\
370	0\\
371	0\\
372	0\\
373	0\\
374	0\\
375	0\\
376	0\\
377	0\\
378	0\\
379	0\\
380	0\\
381	0\\
382	0\\
383	0\\
384	0\\
385	0\\
386	0\\
387	0\\
388	0\\
389	0\\
390	0\\
391	0\\
392	0\\
393	0\\
394	0\\
395	0\\
396	0\\
397	0\\
398	0\\
399	0\\
400	0\\
401	0\\
402	0\\
403	0\\
404	0\\
405	0\\
406	0\\
407	0\\
408	0\\
409	0\\
410	0\\
411	0\\
412	0\\
413	0\\
414	0\\
415	0\\
416	0\\
417	0\\
418	0\\
419	0\\
420	0\\
421	0\\
422	0\\
423	0\\
424	0\\
425	0\\
426	0\\
427	0\\
428	0\\
429	0\\
430	0\\
431	0\\
432	0\\
433	0\\
434	0\\
435	0\\
436	0\\
437	0\\
438	0\\
439	0\\
440	0\\
441	0\\
442	0\\
443	0\\
444	0\\
445	0\\
446	0\\
447	0\\
448	0\\
449	0\\
450	0\\
451	0\\
452	0\\
453	0\\
454	0\\
455	0\\
456	0\\
457	0\\
458	0\\
459	0\\
460	0\\
461	0\\
462	0\\
463	0\\
464	0\\
465	0\\
466	0\\
467	0\\
468	0\\
469	0\\
470	0\\
471	0\\
472	0\\
473	0\\
474	0\\
475	0\\
476	0\\
477	0\\
478	0\\
479	0\\
480	0\\
481	0\\
482	0\\
483	0\\
484	0\\
485	0\\
486	0\\
487	0\\
488	0\\
489	0\\
490	0\\
491	0\\
492	0\\
493	0\\
494	0\\
495	0\\
496	0\\
497	0\\
498	0\\
499	0\\
500	0\\
501	0\\
502	0\\
503	0\\
504	0\\
505	0\\
506	0\\
507	0\\
508	0\\
509	0\\
510	0\\
511	0\\
512	0\\
513	0\\
514	0\\
515	0\\
516	0\\
517	0\\
518	0\\
519	0\\
520	0\\
521	0\\
522	0\\
523	0\\
524	0\\
525	0\\
526	0\\
527	0\\
528	0\\
529	0\\
530	0\\
531	0\\
532	0\\
533	0\\
534	0\\
535	0\\
536	0\\
537	0\\
538	1.426719923304e-05\\
539	4.29091327027637e-05\\
540	7.42159152757329e-05\\
541	0.00010540163623223\\
542	0.000135128139185704\\
543	0.000163860507202848\\
544	0.000193019670144231\\
545	0.000222567303902726\\
546	0.00025268187255965\\
547	0.000283418338955152\\
548	0.000314798578510525\\
549	0.000346837628639188\\
550	0.000379549362854509\\
551	0.00041294395250958\\
552	0.0004470207297917\\
553	0.000481745886966618\\
554	0.000517094416528536\\
555	0.000553212180424865\\
556	0.000590118819429489\\
557	0.000630458343480549\\
558	0.000674248409785912\\
559	0.000711217113060013\\
560	0.000747899309859749\\
561	0.0007848972137292\\
562	0.000822585793724534\\
563	0.00086098759293346\\
564	0.000900118540435697\\
565	0.000939995557047767\\
566	0.000980636019795443\\
567	0.00138345405867184\\
568	0.0017080408379249\\
569	0.00179304145117328\\
570	0.00185663516672567\\
571	0.00192121866981793\\
572	0.00198681272153112\\
573	0.00205343750043299\\
574	0.00212111385614151\\
575	0.00218986333960979\\
576	0.00225970823202167\\
577	0.00233067157311048\\
578	0.00240277718883768\\
579	0.00247604971905621\\
580	0.00255051464798352\\
581	0.00262619834639323\\
582	0.00270312815070207\\
583	0.00278133254700611\\
584	0.00286084164033729\\
585	0.00294168838198347\\
586	0.00302391178876899\\
587	0.00310756536466313\\
588	0.00319273906110312\\
589	0.00327961639013215\\
590	0.00336862265377153\\
591	0.00346039684658503\\
592	0.00355653402929193\\
593	0.00366119131364763\\
594	0.00378527046594782\\
595	0.00395742119285503\\
596	0.00425297862742962\\
597	0.0048700418989439\\
598	0.00632942537858856\\
599	0\\
600	0\\
};
\addplot [color=black!50!mycolor20,solid,forget plot]
  table[row sep=crcr]{%
1	0\\
2	0\\
3	0\\
4	0\\
5	0\\
6	0\\
7	0\\
8	0\\
9	0\\
10	0\\
11	0\\
12	0\\
13	0\\
14	0\\
15	0\\
16	0\\
17	0\\
18	0\\
19	0\\
20	0\\
21	0\\
22	0\\
23	0\\
24	0\\
25	0\\
26	0\\
27	0\\
28	0\\
29	0\\
30	0\\
31	0\\
32	0\\
33	0\\
34	0\\
35	0\\
36	0\\
37	0\\
38	0\\
39	0\\
40	0\\
41	0\\
42	0\\
43	0\\
44	0\\
45	0\\
46	0\\
47	0\\
48	0\\
49	0\\
50	0\\
51	0\\
52	0\\
53	0\\
54	0\\
55	0\\
56	0\\
57	0\\
58	0\\
59	0\\
60	0\\
61	0\\
62	0\\
63	0\\
64	0\\
65	0\\
66	0\\
67	0\\
68	0\\
69	0\\
70	0\\
71	0\\
72	0\\
73	0\\
74	0\\
75	0\\
76	0\\
77	0\\
78	0\\
79	0\\
80	0\\
81	0\\
82	0\\
83	0\\
84	0\\
85	0\\
86	0\\
87	0\\
88	0\\
89	0\\
90	0\\
91	0\\
92	0\\
93	0\\
94	0\\
95	0\\
96	0\\
97	0\\
98	0\\
99	0\\
100	0\\
101	0\\
102	0\\
103	0\\
104	0\\
105	0\\
106	0\\
107	0\\
108	0\\
109	0\\
110	0\\
111	0\\
112	0\\
113	0\\
114	0\\
115	0\\
116	0\\
117	0\\
118	0\\
119	0\\
120	0\\
121	0\\
122	0\\
123	0\\
124	0\\
125	0\\
126	0\\
127	0\\
128	0\\
129	0\\
130	0\\
131	0\\
132	0\\
133	0\\
134	0\\
135	0\\
136	0\\
137	0\\
138	0\\
139	0\\
140	0\\
141	0\\
142	0\\
143	0\\
144	0\\
145	0\\
146	0\\
147	0\\
148	0\\
149	0\\
150	0\\
151	0\\
152	0\\
153	0\\
154	0\\
155	0\\
156	0\\
157	0\\
158	0\\
159	0\\
160	0\\
161	0\\
162	0\\
163	0\\
164	0\\
165	0\\
166	0\\
167	0\\
168	0\\
169	0\\
170	0\\
171	0\\
172	0\\
173	0\\
174	0\\
175	0\\
176	0\\
177	0\\
178	0\\
179	0\\
180	0\\
181	0\\
182	0\\
183	0\\
184	0\\
185	0\\
186	0\\
187	0\\
188	0\\
189	0\\
190	0\\
191	0\\
192	0\\
193	0\\
194	0\\
195	0\\
196	0\\
197	0\\
198	0\\
199	0\\
200	0\\
201	0\\
202	0\\
203	0\\
204	0\\
205	0\\
206	0\\
207	0\\
208	0\\
209	0\\
210	0\\
211	0\\
212	0\\
213	0\\
214	0\\
215	0\\
216	0\\
217	0\\
218	0\\
219	0\\
220	0\\
221	0\\
222	0\\
223	0\\
224	0\\
225	0\\
226	0\\
227	0\\
228	0\\
229	0\\
230	0\\
231	0\\
232	0\\
233	0\\
234	0\\
235	0\\
236	0\\
237	0\\
238	0\\
239	0\\
240	0\\
241	0\\
242	0\\
243	0\\
244	0\\
245	0\\
246	0\\
247	0\\
248	0\\
249	0\\
250	0\\
251	0\\
252	0\\
253	0\\
254	0\\
255	0\\
256	0\\
257	0\\
258	0\\
259	0\\
260	0\\
261	0\\
262	0\\
263	0\\
264	0\\
265	0\\
266	0\\
267	0\\
268	0\\
269	0\\
270	0\\
271	0\\
272	0\\
273	0\\
274	0\\
275	0\\
276	0\\
277	0\\
278	0\\
279	0\\
280	0\\
281	0\\
282	0\\
283	0\\
284	0\\
285	0\\
286	0\\
287	0\\
288	0\\
289	0\\
290	0\\
291	0\\
292	0\\
293	0\\
294	0\\
295	0\\
296	0\\
297	0\\
298	0\\
299	0\\
300	0\\
301	0\\
302	0\\
303	0\\
304	0\\
305	0\\
306	0\\
307	0\\
308	0\\
309	0\\
310	0\\
311	0\\
312	0\\
313	0\\
314	0\\
315	0\\
316	0\\
317	0\\
318	0\\
319	0\\
320	0\\
321	0\\
322	0\\
323	0\\
324	0\\
325	0\\
326	0\\
327	0\\
328	0\\
329	0\\
330	0\\
331	0\\
332	0\\
333	0\\
334	0\\
335	0\\
336	0\\
337	0\\
338	0\\
339	0\\
340	0\\
341	0\\
342	0\\
343	0\\
344	0\\
345	0\\
346	0\\
347	0\\
348	0\\
349	0\\
350	0\\
351	0\\
352	0\\
353	0\\
354	0\\
355	0\\
356	0\\
357	0\\
358	0\\
359	0\\
360	0\\
361	0\\
362	0\\
363	0\\
364	0\\
365	0\\
366	0\\
367	0\\
368	0\\
369	0\\
370	0\\
371	0\\
372	0\\
373	0\\
374	0\\
375	0\\
376	0\\
377	0\\
378	0\\
379	0\\
380	0\\
381	0\\
382	0\\
383	0\\
384	0\\
385	0\\
386	0\\
387	0\\
388	0\\
389	0\\
390	0\\
391	0\\
392	0\\
393	0\\
394	0\\
395	0\\
396	0\\
397	0\\
398	0\\
399	0\\
400	0\\
401	0\\
402	0\\
403	0\\
404	0\\
405	0\\
406	0\\
407	0\\
408	0\\
409	0\\
410	0\\
411	0\\
412	0\\
413	0\\
414	0\\
415	0\\
416	0\\
417	0\\
418	0\\
419	0\\
420	0\\
421	0\\
422	0\\
423	0\\
424	0\\
425	0\\
426	0\\
427	0\\
428	0\\
429	0\\
430	0\\
431	0\\
432	0\\
433	0\\
434	0\\
435	0\\
436	0\\
437	0\\
438	0\\
439	0\\
440	0\\
441	0\\
442	0\\
443	0\\
444	0\\
445	0\\
446	0\\
447	0\\
448	0\\
449	0\\
450	0\\
451	0\\
452	0\\
453	0\\
454	0\\
455	0\\
456	0\\
457	0\\
458	0\\
459	0\\
460	0\\
461	0\\
462	0\\
463	0\\
464	0\\
465	0\\
466	0\\
467	0\\
468	0\\
469	0\\
470	0\\
471	0\\
472	0\\
473	0\\
474	0\\
475	0\\
476	0\\
477	0\\
478	0\\
479	0\\
480	0\\
481	0\\
482	0\\
483	0\\
484	0\\
485	0\\
486	0\\
487	0\\
488	0\\
489	0\\
490	0\\
491	0\\
492	0\\
493	0\\
494	0\\
495	0\\
496	0\\
497	0\\
498	0\\
499	0\\
500	0\\
501	0\\
502	0\\
503	0\\
504	0\\
505	0\\
506	0\\
507	0\\
508	0\\
509	0\\
510	0\\
511	0\\
512	0\\
513	0\\
514	0\\
515	0\\
516	0\\
517	0\\
518	0\\
519	0\\
520	0\\
521	0\\
522	0\\
523	0\\
524	0\\
525	0\\
526	0\\
527	0\\
528	0\\
529	0\\
530	0\\
531	0\\
532	0\\
533	0\\
534	0\\
535	0\\
536	0\\
537	6.49828894175203e-07\\
538	3.05652458911548e-05\\
539	5.99404394527822e-05\\
540	8.71312739401948e-05\\
541	0.000114716903641183\\
542	0.000142674508084298\\
543	0.000171100011748483\\
544	0.000200107410759274\\
545	0.000229717106180836\\
546	0.000259944556361032\\
547	0.000290804095303178\\
548	0.000322309893047164\\
549	0.000354475852884199\\
550	0.000387314440674418\\
551	0.000420833252736105\\
552	0.000455025048030861\\
553	0.00048983964048552\\
554	0.000525318728056389\\
555	0.000561574556760999\\
556	0.000603919415462835\\
557	0.00064369719785649\\
558	0.000679340140340759\\
559	0.000715080552741755\\
560	0.000751442297042198\\
561	0.000788488188292752\\
562	0.000826233343400132\\
563	0.000864693097341739\\
564	0.000903883877005878\\
565	0.000943822831240413\\
566	0.00135108213105103\\
567	0.00165100703747856\\
568	0.00173042226223131\\
569	0.0017930426874416\\
570	0.00185663517077488\\
571	0.00192121867121681\\
572	0.00198681272222949\\
573	0.00205343750078598\\
574	0.00212111385631247\\
575	0.00218986333968779\\
576	0.00225970823205476\\
577	0.00233067157312331\\
578	0.00240277718884216\\
579	0.00247604971905759\\
580	0.00255051464798387\\
581	0.0026261983463933\\
582	0.00270312815070208\\
583	0.0027813325470061\\
584	0.00286084164033727\\
585	0.00294168838198346\\
586	0.00302391178876899\\
587	0.00310756536466312\\
588	0.00319273906110312\\
589	0.00327961639013215\\
590	0.00336862265377153\\
591	0.00346039684658503\\
592	0.00355653402929193\\
593	0.00366119131364762\\
594	0.00378527046594782\\
595	0.00395742119285503\\
596	0.00425297862742961\\
597	0.0048700418989439\\
598	0.00632942537858856\\
599	0\\
600	0\\
};
\addplot [color=black!60!mycolor21,solid,forget plot]
  table[row sep=crcr]{%
1	0\\
2	0\\
3	0\\
4	0\\
5	0\\
6	0\\
7	0\\
8	0\\
9	0\\
10	0\\
11	0\\
12	0\\
13	0\\
14	0\\
15	0\\
16	0\\
17	0\\
18	0\\
19	0\\
20	0\\
21	0\\
22	0\\
23	0\\
24	0\\
25	0\\
26	0\\
27	0\\
28	0\\
29	0\\
30	0\\
31	0\\
32	0\\
33	0\\
34	0\\
35	0\\
36	0\\
37	0\\
38	0\\
39	0\\
40	0\\
41	0\\
42	0\\
43	0\\
44	0\\
45	0\\
46	0\\
47	0\\
48	0\\
49	0\\
50	0\\
51	0\\
52	0\\
53	0\\
54	0\\
55	0\\
56	0\\
57	0\\
58	0\\
59	0\\
60	0\\
61	0\\
62	0\\
63	0\\
64	0\\
65	0\\
66	0\\
67	0\\
68	0\\
69	0\\
70	0\\
71	0\\
72	0\\
73	0\\
74	0\\
75	0\\
76	0\\
77	0\\
78	0\\
79	0\\
80	0\\
81	0\\
82	0\\
83	0\\
84	0\\
85	0\\
86	0\\
87	0\\
88	0\\
89	0\\
90	0\\
91	0\\
92	0\\
93	0\\
94	0\\
95	0\\
96	0\\
97	0\\
98	0\\
99	0\\
100	0\\
101	0\\
102	0\\
103	0\\
104	0\\
105	0\\
106	0\\
107	0\\
108	0\\
109	0\\
110	0\\
111	0\\
112	0\\
113	0\\
114	0\\
115	0\\
116	0\\
117	0\\
118	0\\
119	0\\
120	0\\
121	0\\
122	0\\
123	0\\
124	0\\
125	0\\
126	0\\
127	0\\
128	0\\
129	0\\
130	0\\
131	0\\
132	0\\
133	0\\
134	0\\
135	0\\
136	0\\
137	0\\
138	0\\
139	0\\
140	0\\
141	0\\
142	0\\
143	0\\
144	0\\
145	0\\
146	0\\
147	0\\
148	0\\
149	0\\
150	0\\
151	0\\
152	0\\
153	0\\
154	0\\
155	0\\
156	0\\
157	0\\
158	0\\
159	0\\
160	0\\
161	0\\
162	0\\
163	0\\
164	0\\
165	0\\
166	0\\
167	0\\
168	0\\
169	0\\
170	0\\
171	0\\
172	0\\
173	0\\
174	0\\
175	0\\
176	0\\
177	0\\
178	0\\
179	0\\
180	0\\
181	0\\
182	0\\
183	0\\
184	0\\
185	0\\
186	0\\
187	0\\
188	0\\
189	0\\
190	0\\
191	0\\
192	0\\
193	0\\
194	0\\
195	0\\
196	0\\
197	0\\
198	0\\
199	0\\
200	0\\
201	0\\
202	0\\
203	0\\
204	0\\
205	0\\
206	0\\
207	0\\
208	0\\
209	0\\
210	0\\
211	0\\
212	0\\
213	0\\
214	0\\
215	0\\
216	0\\
217	0\\
218	0\\
219	0\\
220	0\\
221	0\\
222	0\\
223	0\\
224	0\\
225	0\\
226	0\\
227	0\\
228	0\\
229	0\\
230	0\\
231	0\\
232	0\\
233	0\\
234	0\\
235	0\\
236	0\\
237	0\\
238	0\\
239	0\\
240	0\\
241	0\\
242	0\\
243	0\\
244	0\\
245	0\\
246	0\\
247	0\\
248	0\\
249	0\\
250	0\\
251	0\\
252	0\\
253	0\\
254	0\\
255	0\\
256	0\\
257	0\\
258	0\\
259	0\\
260	0\\
261	0\\
262	0\\
263	0\\
264	0\\
265	0\\
266	0\\
267	0\\
268	0\\
269	0\\
270	0\\
271	0\\
272	0\\
273	0\\
274	0\\
275	0\\
276	0\\
277	0\\
278	0\\
279	0\\
280	0\\
281	0\\
282	0\\
283	0\\
284	0\\
285	0\\
286	0\\
287	0\\
288	0\\
289	0\\
290	0\\
291	0\\
292	0\\
293	0\\
294	0\\
295	0\\
296	0\\
297	0\\
298	0\\
299	0\\
300	0\\
301	0\\
302	0\\
303	0\\
304	0\\
305	0\\
306	0\\
307	0\\
308	0\\
309	0\\
310	0\\
311	0\\
312	0\\
313	0\\
314	0\\
315	0\\
316	0\\
317	0\\
318	0\\
319	0\\
320	0\\
321	0\\
322	0\\
323	0\\
324	0\\
325	0\\
326	0\\
327	0\\
328	0\\
329	0\\
330	0\\
331	0\\
332	0\\
333	0\\
334	0\\
335	0\\
336	0\\
337	0\\
338	0\\
339	0\\
340	0\\
341	0\\
342	0\\
343	0\\
344	0\\
345	0\\
346	0\\
347	0\\
348	0\\
349	0\\
350	0\\
351	0\\
352	0\\
353	0\\
354	0\\
355	0\\
356	0\\
357	0\\
358	0\\
359	0\\
360	0\\
361	0\\
362	0\\
363	0\\
364	0\\
365	0\\
366	0\\
367	0\\
368	0\\
369	0\\
370	0\\
371	0\\
372	0\\
373	0\\
374	0\\
375	0\\
376	0\\
377	0\\
378	0\\
379	0\\
380	0\\
381	0\\
382	0\\
383	0\\
384	0\\
385	0\\
386	0\\
387	0\\
388	0\\
389	0\\
390	0\\
391	0\\
392	0\\
393	0\\
394	0\\
395	0\\
396	0\\
397	0\\
398	0\\
399	0\\
400	0\\
401	0\\
402	0\\
403	0\\
404	0\\
405	0\\
406	0\\
407	0\\
408	0\\
409	0\\
410	0\\
411	0\\
412	0\\
413	0\\
414	0\\
415	0\\
416	0\\
417	0\\
418	0\\
419	0\\
420	0\\
421	0\\
422	0\\
423	0\\
424	0\\
425	0\\
426	0\\
427	0\\
428	0\\
429	0\\
430	0\\
431	0\\
432	0\\
433	0\\
434	0\\
435	0\\
436	0\\
437	0\\
438	0\\
439	0\\
440	0\\
441	0\\
442	0\\
443	0\\
444	0\\
445	0\\
446	0\\
447	0\\
448	0\\
449	0\\
450	0\\
451	0\\
452	0\\
453	0\\
454	0\\
455	0\\
456	0\\
457	0\\
458	0\\
459	0\\
460	0\\
461	0\\
462	0\\
463	0\\
464	0\\
465	0\\
466	0\\
467	0\\
468	0\\
469	0\\
470	0\\
471	0\\
472	0\\
473	0\\
474	0\\
475	0\\
476	0\\
477	0\\
478	0\\
479	0\\
480	0\\
481	0\\
482	0\\
483	0\\
484	0\\
485	0\\
486	0\\
487	0\\
488	0\\
489	0\\
490	0\\
491	0\\
492	0\\
493	0\\
494	0\\
495	0\\
496	0\\
497	0\\
498	0\\
499	0\\
500	0\\
501	0\\
502	0\\
503	0\\
504	0\\
505	0\\
506	0\\
507	0\\
508	0\\
509	0\\
510	0\\
511	0\\
512	0\\
513	0\\
514	0\\
515	0\\
516	0\\
517	0\\
518	0\\
519	0\\
520	0\\
521	0\\
522	0\\
523	0\\
524	0\\
525	0\\
526	0\\
527	0\\
528	0\\
529	0\\
530	0\\
531	0\\
532	0\\
533	0\\
534	0\\
535	0\\
536	0\\
537	1.41981037307091e-05\\
538	4.03194267452914e-05\\
539	6.68049003892867e-05\\
540	9.36455912896163e-05\\
541	0.000121031570767503\\
542	0.000148981487706946\\
543	0.000177511834873603\\
544	0.000206636133043701\\
545	0.00023636795016899\\
546	0.000266721132093178\\
547	0.000297709825439374\\
548	0.000329348337773177\\
549	0.000361650643511393\\
550	0.000394628923195159\\
551	0.000428289376922921\\
552	0.000462620483140471\\
553	0.000497556235318828\\
554	0.000534404821662524\\
555	0.00057682151016282\\
556	0.000612670727538693\\
557	0.000647448601386753\\
558	0.000682535085499909\\
559	0.000718276611240496\\
560	0.000754689045844615\\
561	0.000791786693129664\\
562	0.000829584407608802\\
563	0.000868098067449268\\
564	0.00090734463744055\\
565	0.00128398061588739\\
566	0.00159228887629074\\
567	0.00166875568695174\\
568	0.00173042230411937\\
569	0.00179304268794214\\
570	0.00185663517097841\\
571	0.00192121867131843\\
572	0.00198681272227957\\
573	0.0020534375008095\\
574	0.00212111385632287\\
575	0.00218986333969204\\
576	0.00225970823205636\\
577	0.00233067157312388\\
578	0.00240277718884235\\
579	0.00247604971905763\\
580	0.00255051464798387\\
581	0.00262619834639329\\
582	0.00270312815070207\\
583	0.00278133254700611\\
584	0.00286084164033729\\
585	0.00294168838198347\\
586	0.00302391178876898\\
587	0.00310756536466311\\
588	0.0031927390611031\\
589	0.00327961639013215\\
590	0.00336862265377153\\
591	0.00346039684658503\\
592	0.00355653402929193\\
593	0.00366119131364762\\
594	0.00378527046594782\\
595	0.00395742119285503\\
596	0.00425297862742963\\
597	0.0048700418989439\\
598	0.00632942537858856\\
599	0\\
600	0\\
};
\addplot [color=black!80!mycolor21,solid,forget plot]
  table[row sep=crcr]{%
1	0\\
2	0\\
3	0\\
4	0\\
5	0\\
6	0\\
7	0\\
8	0\\
9	0\\
10	0\\
11	0\\
12	0\\
13	0\\
14	0\\
15	0\\
16	0\\
17	0\\
18	0\\
19	0\\
20	0\\
21	0\\
22	0\\
23	0\\
24	0\\
25	0\\
26	0\\
27	0\\
28	0\\
29	0\\
30	0\\
31	0\\
32	0\\
33	0\\
34	0\\
35	0\\
36	0\\
37	0\\
38	0\\
39	0\\
40	0\\
41	0\\
42	0\\
43	0\\
44	0\\
45	0\\
46	0\\
47	0\\
48	0\\
49	0\\
50	0\\
51	0\\
52	0\\
53	0\\
54	0\\
55	0\\
56	0\\
57	0\\
58	0\\
59	0\\
60	0\\
61	0\\
62	0\\
63	0\\
64	0\\
65	0\\
66	0\\
67	0\\
68	0\\
69	0\\
70	0\\
71	0\\
72	0\\
73	0\\
74	0\\
75	0\\
76	0\\
77	0\\
78	0\\
79	0\\
80	0\\
81	0\\
82	0\\
83	0\\
84	0\\
85	0\\
86	0\\
87	0\\
88	0\\
89	0\\
90	0\\
91	0\\
92	0\\
93	0\\
94	0\\
95	0\\
96	0\\
97	0\\
98	0\\
99	0\\
100	0\\
101	0\\
102	0\\
103	0\\
104	0\\
105	0\\
106	0\\
107	0\\
108	0\\
109	0\\
110	0\\
111	0\\
112	0\\
113	0\\
114	0\\
115	0\\
116	0\\
117	0\\
118	0\\
119	0\\
120	0\\
121	0\\
122	0\\
123	0\\
124	0\\
125	0\\
126	0\\
127	0\\
128	0\\
129	0\\
130	0\\
131	0\\
132	0\\
133	0\\
134	0\\
135	0\\
136	0\\
137	0\\
138	0\\
139	0\\
140	0\\
141	0\\
142	0\\
143	0\\
144	0\\
145	0\\
146	0\\
147	0\\
148	0\\
149	0\\
150	0\\
151	0\\
152	0\\
153	0\\
154	0\\
155	0\\
156	0\\
157	0\\
158	0\\
159	0\\
160	0\\
161	0\\
162	0\\
163	0\\
164	0\\
165	0\\
166	0\\
167	0\\
168	0\\
169	0\\
170	0\\
171	0\\
172	0\\
173	0\\
174	0\\
175	0\\
176	0\\
177	0\\
178	0\\
179	0\\
180	0\\
181	0\\
182	0\\
183	0\\
184	0\\
185	0\\
186	0\\
187	0\\
188	0\\
189	0\\
190	0\\
191	0\\
192	0\\
193	0\\
194	0\\
195	0\\
196	0\\
197	0\\
198	0\\
199	0\\
200	0\\
201	0\\
202	0\\
203	0\\
204	0\\
205	0\\
206	0\\
207	0\\
208	0\\
209	0\\
210	0\\
211	0\\
212	0\\
213	0\\
214	0\\
215	0\\
216	0\\
217	0\\
218	0\\
219	0\\
220	0\\
221	0\\
222	0\\
223	0\\
224	0\\
225	0\\
226	0\\
227	0\\
228	0\\
229	0\\
230	0\\
231	0\\
232	0\\
233	0\\
234	0\\
235	0\\
236	0\\
237	0\\
238	0\\
239	0\\
240	0\\
241	0\\
242	0\\
243	0\\
244	0\\
245	0\\
246	0\\
247	0\\
248	0\\
249	0\\
250	0\\
251	0\\
252	0\\
253	0\\
254	0\\
255	0\\
256	0\\
257	0\\
258	0\\
259	0\\
260	0\\
261	0\\
262	0\\
263	0\\
264	0\\
265	0\\
266	0\\
267	0\\
268	0\\
269	0\\
270	0\\
271	0\\
272	0\\
273	0\\
274	0\\
275	0\\
276	0\\
277	0\\
278	0\\
279	0\\
280	0\\
281	0\\
282	0\\
283	0\\
284	0\\
285	0\\
286	0\\
287	0\\
288	0\\
289	0\\
290	0\\
291	0\\
292	0\\
293	0\\
294	0\\
295	0\\
296	0\\
297	0\\
298	0\\
299	0\\
300	0\\
301	0\\
302	0\\
303	0\\
304	0\\
305	0\\
306	0\\
307	0\\
308	0\\
309	0\\
310	0\\
311	0\\
312	0\\
313	0\\
314	0\\
315	0\\
316	0\\
317	0\\
318	0\\
319	0\\
320	0\\
321	0\\
322	0\\
323	0\\
324	0\\
325	0\\
326	0\\
327	0\\
328	0\\
329	0\\
330	0\\
331	0\\
332	0\\
333	0\\
334	0\\
335	0\\
336	0\\
337	0\\
338	0\\
339	0\\
340	0\\
341	0\\
342	0\\
343	0\\
344	0\\
345	0\\
346	0\\
347	0\\
348	0\\
349	0\\
350	0\\
351	0\\
352	0\\
353	0\\
354	0\\
355	0\\
356	0\\
357	0\\
358	0\\
359	0\\
360	0\\
361	0\\
362	0\\
363	0\\
364	0\\
365	0\\
366	0\\
367	0\\
368	0\\
369	0\\
370	0\\
371	0\\
372	0\\
373	0\\
374	0\\
375	0\\
376	0\\
377	0\\
378	0\\
379	0\\
380	0\\
381	0\\
382	0\\
383	0\\
384	0\\
385	0\\
386	0\\
387	0\\
388	0\\
389	0\\
390	0\\
391	0\\
392	0\\
393	0\\
394	0\\
395	0\\
396	0\\
397	0\\
398	0\\
399	0\\
400	0\\
401	0\\
402	0\\
403	0\\
404	0\\
405	0\\
406	0\\
407	0\\
408	0\\
409	0\\
410	0\\
411	0\\
412	0\\
413	0\\
414	0\\
415	0\\
416	0\\
417	0\\
418	0\\
419	0\\
420	0\\
421	0\\
422	0\\
423	0\\
424	0\\
425	0\\
426	0\\
427	0\\
428	0\\
429	0\\
430	0\\
431	0\\
432	0\\
433	0\\
434	0\\
435	0\\
436	0\\
437	0\\
438	0\\
439	0\\
440	0\\
441	0\\
442	0\\
443	0\\
444	0\\
445	0\\
446	0\\
447	0\\
448	0\\
449	0\\
450	0\\
451	0\\
452	0\\
453	0\\
454	0\\
455	0\\
456	0\\
457	0\\
458	0\\
459	0\\
460	0\\
461	0\\
462	0\\
463	0\\
464	0\\
465	0\\
466	0\\
467	0\\
468	0\\
469	0\\
470	0\\
471	0\\
472	0\\
473	0\\
474	0\\
475	0\\
476	0\\
477	0\\
478	0\\
479	0\\
480	0\\
481	0\\
482	0\\
483	0\\
484	0\\
485	0\\
486	0\\
487	0\\
488	0\\
489	0\\
490	0\\
491	0\\
492	0\\
493	0\\
494	0\\
495	0\\
496	0\\
497	0\\
498	0\\
499	0\\
500	0\\
501	0\\
502	0\\
503	0\\
504	0\\
505	0\\
506	0\\
507	0\\
508	0\\
509	0\\
510	0\\
511	0\\
512	0\\
513	0\\
514	0\\
515	0\\
516	0\\
517	0\\
518	0\\
519	0\\
520	0\\
521	0\\
522	0\\
523	0\\
524	0\\
525	0\\
526	0\\
527	0\\
528	0\\
529	0\\
530	0\\
531	0\\
532	0\\
533	0\\
534	0\\
535	0\\
536	0\\
537	2.00311818478477e-05\\
538	4.58947831388607e-05\\
539	7.22859642090788e-05\\
540	9.92227167136692e-05\\
541	0.000126717645174376\\
542	0.000154783396446169\\
543	0.000183432792668513\\
544	0.000212678988439243\\
545	0.000242535495908309\\
546	0.000273016183234799\\
547	0.000304135226905855\\
548	0.000335906938006349\\
549	0.000368345226021222\\
550	0.000401462007738327\\
551	0.00043526250918298\\
552	0.00046973121451101\\
553	0.000507806233402199\\
554	0.000547931958570056\\
555	0.000581827365374636\\
556	0.000615782626155859\\
557	0.000650269640106297\\
558	0.000685400368111508\\
559	0.000721188332977737\\
560	0.000757647340897565\\
561	0.00079479173610764\\
562	0.000832636728120108\\
563	0.000871198966466937\\
564	0.00118295200706542\\
565	0.00153193885094584\\
566	0.00160802507255323\\
567	0.00166875568906997\\
568	0.00173042230418516\\
569	0.0017930426879713\\
570	0.00185663517099282\\
571	0.00192121867132536\\
572	0.00198681272228273\\
573	0.00205343750081087\\
574	0.00212111385632341\\
575	0.00218986333969228\\
576	0.00225970823205644\\
577	0.0023306715731239\\
578	0.00240277718884234\\
579	0.00247604971905764\\
580	0.00255051464798387\\
581	0.00262619834639331\\
582	0.00270312815070208\\
583	0.00278133254700611\\
584	0.00286084164033728\\
585	0.00294168838198347\\
586	0.003023911788769\\
587	0.00310756536466313\\
588	0.00319273906110312\\
589	0.00327961639013215\\
590	0.00336862265377153\\
591	0.00346039684658503\\
592	0.00355653402929195\\
593	0.00366119131364764\\
594	0.00378527046594783\\
595	0.00395742119285504\\
596	0.00425297862742962\\
597	0.00487004189894391\\
598	0.00632942537858856\\
599	0\\
600	0\\
};
\addplot [color=black,solid,forget plot]
  table[row sep=crcr]{%
1	0\\
2	0\\
3	0\\
4	0\\
5	0\\
6	0\\
7	0\\
8	0\\
9	0\\
10	0\\
11	0\\
12	0\\
13	0\\
14	0\\
15	0\\
16	0\\
17	0\\
18	0\\
19	0\\
20	0\\
21	0\\
22	0\\
23	0\\
24	0\\
25	0\\
26	0\\
27	0\\
28	0\\
29	0\\
30	0\\
31	0\\
32	0\\
33	0\\
34	0\\
35	0\\
36	0\\
37	0\\
38	0\\
39	0\\
40	0\\
41	0\\
42	0\\
43	0\\
44	0\\
45	0\\
46	0\\
47	0\\
48	0\\
49	0\\
50	0\\
51	0\\
52	0\\
53	0\\
54	0\\
55	0\\
56	0\\
57	0\\
58	0\\
59	0\\
60	0\\
61	0\\
62	0\\
63	0\\
64	0\\
65	0\\
66	0\\
67	0\\
68	0\\
69	0\\
70	0\\
71	0\\
72	0\\
73	0\\
74	0\\
75	0\\
76	0\\
77	0\\
78	0\\
79	0\\
80	0\\
81	0\\
82	0\\
83	0\\
84	0\\
85	0\\
86	0\\
87	0\\
88	0\\
89	0\\
90	0\\
91	0\\
92	0\\
93	0\\
94	0\\
95	0\\
96	0\\
97	0\\
98	0\\
99	0\\
100	0\\
101	0\\
102	0\\
103	0\\
104	0\\
105	0\\
106	0\\
107	0\\
108	0\\
109	0\\
110	0\\
111	0\\
112	0\\
113	0\\
114	0\\
115	0\\
116	0\\
117	0\\
118	0\\
119	0\\
120	0\\
121	0\\
122	0\\
123	0\\
124	0\\
125	0\\
126	0\\
127	0\\
128	0\\
129	0\\
130	0\\
131	0\\
132	0\\
133	0\\
134	0\\
135	0\\
136	0\\
137	0\\
138	0\\
139	0\\
140	0\\
141	0\\
142	0\\
143	0\\
144	0\\
145	0\\
146	0\\
147	0\\
148	0\\
149	0\\
150	0\\
151	0\\
152	0\\
153	0\\
154	0\\
155	0\\
156	0\\
157	0\\
158	0\\
159	0\\
160	0\\
161	0\\
162	0\\
163	0\\
164	0\\
165	0\\
166	0\\
167	0\\
168	0\\
169	0\\
170	0\\
171	0\\
172	0\\
173	0\\
174	0\\
175	0\\
176	0\\
177	0\\
178	0\\
179	0\\
180	0\\
181	0\\
182	0\\
183	0\\
184	0\\
185	0\\
186	0\\
187	0\\
188	0\\
189	0\\
190	0\\
191	0\\
192	0\\
193	0\\
194	0\\
195	0\\
196	0\\
197	0\\
198	0\\
199	0\\
200	0\\
201	0\\
202	0\\
203	0\\
204	0\\
205	0\\
206	0\\
207	0\\
208	0\\
209	0\\
210	0\\
211	0\\
212	0\\
213	0\\
214	0\\
215	0\\
216	0\\
217	0\\
218	0\\
219	0\\
220	0\\
221	0\\
222	0\\
223	0\\
224	0\\
225	0\\
226	0\\
227	0\\
228	0\\
229	0\\
230	0\\
231	0\\
232	0\\
233	0\\
234	0\\
235	0\\
236	0\\
237	0\\
238	0\\
239	0\\
240	0\\
241	0\\
242	0\\
243	0\\
244	0\\
245	0\\
246	0\\
247	0\\
248	0\\
249	0\\
250	0\\
251	0\\
252	0\\
253	0\\
254	0\\
255	0\\
256	0\\
257	0\\
258	0\\
259	0\\
260	0\\
261	0\\
262	0\\
263	0\\
264	0\\
265	0\\
266	0\\
267	0\\
268	0\\
269	0\\
270	0\\
271	0\\
272	0\\
273	0\\
274	0\\
275	0\\
276	0\\
277	0\\
278	0\\
279	0\\
280	0\\
281	0\\
282	0\\
283	0\\
284	0\\
285	0\\
286	0\\
287	0\\
288	0\\
289	0\\
290	0\\
291	0\\
292	0\\
293	0\\
294	0\\
295	0\\
296	0\\
297	0\\
298	0\\
299	0\\
300	0\\
301	0\\
302	0\\
303	0\\
304	0\\
305	0\\
306	0\\
307	0\\
308	0\\
309	0\\
310	0\\
311	0\\
312	0\\
313	0\\
314	0\\
315	0\\
316	0\\
317	0\\
318	0\\
319	0\\
320	0\\
321	0\\
322	0\\
323	0\\
324	0\\
325	0\\
326	0\\
327	0\\
328	0\\
329	0\\
330	0\\
331	0\\
332	0\\
333	0\\
334	0\\
335	0\\
336	0\\
337	0\\
338	0\\
339	0\\
340	0\\
341	0\\
342	0\\
343	0\\
344	0\\
345	0\\
346	0\\
347	0\\
348	0\\
349	0\\
350	0\\
351	0\\
352	0\\
353	0\\
354	0\\
355	0\\
356	0\\
357	0\\
358	0\\
359	0\\
360	0\\
361	0\\
362	0\\
363	0\\
364	0\\
365	0\\
366	0\\
367	0\\
368	0\\
369	0\\
370	0\\
371	0\\
372	0\\
373	0\\
374	0\\
375	0\\
376	0\\
377	0\\
378	0\\
379	0\\
380	0\\
381	0\\
382	0\\
383	0\\
384	0\\
385	0\\
386	0\\
387	0\\
388	0\\
389	0\\
390	0\\
391	0\\
392	0\\
393	0\\
394	0\\
395	0\\
396	0\\
397	0\\
398	0\\
399	0\\
400	0\\
401	0\\
402	0\\
403	0\\
404	0\\
405	0\\
406	0\\
407	0\\
408	0\\
409	0\\
410	0\\
411	0\\
412	0\\
413	0\\
414	0\\
415	0\\
416	0\\
417	0\\
418	0\\
419	0\\
420	0\\
421	0\\
422	0\\
423	0\\
424	0\\
425	0\\
426	0\\
427	0\\
428	0\\
429	0\\
430	0\\
431	0\\
432	0\\
433	0\\
434	0\\
435	0\\
436	0\\
437	0\\
438	0\\
439	0\\
440	0\\
441	0\\
442	0\\
443	0\\
444	0\\
445	0\\
446	0\\
447	0\\
448	0\\
449	0\\
450	0\\
451	0\\
452	0\\
453	0\\
454	0\\
455	0\\
456	0\\
457	0\\
458	0\\
459	0\\
460	0\\
461	0\\
462	0\\
463	0\\
464	0\\
465	0\\
466	0\\
467	0\\
468	0\\
469	0\\
470	0\\
471	0\\
472	0\\
473	0\\
474	0\\
475	0\\
476	0\\
477	0\\
478	0\\
479	0\\
480	0\\
481	0\\
482	0\\
483	0\\
484	0\\
485	0\\
486	0\\
487	0\\
488	0\\
489	0\\
490	0\\
491	0\\
492	0\\
493	0\\
494	0\\
495	0\\
496	0\\
497	0\\
498	0\\
499	0\\
500	0\\
501	0\\
502	0\\
503	0\\
504	0\\
505	0\\
506	0\\
507	0\\
508	0\\
509	0\\
510	0\\
511	0\\
512	0\\
513	0\\
514	0\\
515	0\\
516	0\\
517	0\\
518	0\\
519	0\\
520	0\\
521	0\\
522	0\\
523	0\\
524	0\\
525	0\\
526	0\\
527	0\\
528	0\\
529	0\\
530	0\\
531	0\\
532	0\\
533	0\\
534	0\\
535	0\\
536	0\\
537	2.41449941958518e-05\\
538	5.01062475752778e-05\\
539	7.66056323086619e-05\\
540	0.000103655192469625\\
541	0.000131267283450877\\
542	0.000159454603022728\\
543	0.000188230216867344\\
544	0.0002176075760922\\
545	0.000247600530527473\\
546	0.000278223330210997\\
547	0.000309490592847296\\
548	0.000341417179393022\\
549	0.000374017835784068\\
550	0.000407306284469521\\
551	0.000441293207629602\\
552	0.000480387614922689\\
553	0.000517733822270981\\
554	0.000550900656801395\\
555	0.000584195800718177\\
556	0.000618106994392449\\
557	0.000652650864010514\\
558	0.000687840631956863\\
559	0.000723689894931791\\
560	0.000760212780169657\\
561	0.000797424370009976\\
562	0.000835341903111315\\
563	0.00104794067987499\\
564	0.0014700571270915\\
565	0.00154821321979401\\
566	0.00160802507270126\\
567	0.00166875568907878\\
568	0.00173042230418922\\
569	0.00179304268797327\\
570	0.00185663517099374\\
571	0.00192121867132575\\
572	0.00198681272228291\\
573	0.00205343750081092\\
574	0.00212111385632343\\
575	0.00218986333969225\\
576	0.00225970823205644\\
577	0.0023306715731239\\
578	0.00240277718884235\\
579	0.00247604971905765\\
580	0.00255051464798389\\
581	0.0026261983463933\\
582	0.00270312815070208\\
583	0.00278133254700612\\
584	0.00286084164033729\\
585	0.00294168838198346\\
586	0.00302391178876899\\
587	0.00310756536466312\\
588	0.00319273906110312\\
589	0.00327961639013216\\
590	0.00336862265377154\\
591	0.00346039684658502\\
592	0.00355653402929193\\
593	0.00366119131364762\\
594	0.00378527046594782\\
595	0.00395742119285502\\
596	0.00425297862742962\\
597	0.0048700418989439\\
598	0.00632942537858856\\
599	0\\
600	0\\
};
\end{axis}
\end{tikzpicture}% 
  \caption{Discrete Time}
\end{subfigure}\\
\vspace{1cm}
\begin{subfigure}{.45\linewidth}
  \centering
  \setlength\figureheight{\linewidth} 
  \setlength\figurewidth{\linewidth}
  \tikzsetnextfilename{testdp_cts_nFPC_z15}
  % This file was created by matlab2tikz.
%
%The latest updates can be retrieved from
%  http://www.mathworks.com/matlabcentral/fileexchange/22022-matlab2tikz-matlab2tikz
%where you can also make suggestions and rate matlab2tikz.
%
\definecolor{mycolor1}{rgb}{1.00000,0.00000,1.00000}%
%
\begin{tikzpicture}

\begin{axis}[%
width=4.564in,
height=3.803in,
at={(1.067in,0.513in)},
scale only axis,
every outer x axis line/.append style={black},
every x tick label/.append style={font=\color{black}},
xmin=0,
xmax=100,
xlabel={Time},
every outer y axis line/.append style={black},
every y tick label/.append style={font=\color{black}},
ymin=0,
ymax=0.012,
ylabel={Depth $\delta$},
axis background/.style={fill=white},
title={Z=15},
axis x line*=bottom,
axis y line*=left,
legend style={legend cell align=left,align=left,draw=black}
]
\addplot [color=green,dashed,forget plot]
  table[row sep=crcr]{%
0.01	0.01\\
0.02	0.01\\
0.03	0.01\\
0.04	0.01\\
0.05	0.01\\
0.06	0.01\\
0.07	0.01\\
0.08	0.01\\
0.09	0.01\\
0.1	0.01\\
0.11	0.01\\
0.12	0.01\\
0.13	0.01\\
0.14	0.01\\
0.15	0.01\\
0.16	0.01\\
0.17	0.01\\
0.18	0.01\\
0.19	0.01\\
0.2	0.01\\
0.21	0.01\\
0.22	0.01\\
0.23	0.01\\
0.24	0.01\\
0.25	0.01\\
0.26	0.01\\
0.27	0.01\\
0.28	0.01\\
0.29	0.01\\
0.3	0.01\\
0.31	0.01\\
0.32	0.01\\
0.33	0.01\\
0.34	0.01\\
0.35	0.01\\
0.36	0.01\\
0.37	0.01\\
0.38	0.01\\
0.39	0.01\\
0.4	0.01\\
0.41	0.01\\
0.42	0.01\\
0.43	0.01\\
0.44	0.01\\
0.45	0.01\\
0.46	0.01\\
0.47	0.01\\
0.48	0.01\\
0.49	0.01\\
0.5	0.01\\
0.51	0.01\\
0.52	0.01\\
0.53	0.01\\
0.54	0.01\\
0.55	0.01\\
0.56	0.01\\
0.57	0.01\\
0.58	0.01\\
0.59	0.01\\
0.6	0.01\\
0.61	0.01\\
0.62	0.01\\
0.63	0.01\\
0.64	0.01\\
0.65	0.01\\
0.66	0.01\\
0.67	0.01\\
0.68	0.01\\
0.69	0.01\\
0.7	0.01\\
0.71	0.01\\
0.72	0.01\\
0.73	0.01\\
0.74	0.01\\
0.75	0.01\\
0.76	0.01\\
0.77	0.01\\
0.78	0.01\\
0.79	0.01\\
0.8	0.01\\
0.81	0.01\\
0.82	0.01\\
0.83	0.01\\
0.84	0.01\\
0.85	0.01\\
0.86	0.01\\
0.87	0.01\\
0.88	0.01\\
0.89	0.01\\
0.9	0.01\\
0.91	0.01\\
0.92	0.01\\
0.93	0.01\\
0.94	0.01\\
0.95	0.01\\
0.96	0.01\\
0.97	0.01\\
0.98	0.01\\
0.99	0.01\\
1	0.01\\
1.01	0.01\\
1.02	0.01\\
1.03	0.01\\
1.04	0.01\\
1.05	0.01\\
1.06	0.01\\
1.07	0.01\\
1.08	0.01\\
1.09	0.01\\
1.1	0.01\\
1.11	0.01\\
1.12	0.01\\
1.13	0.01\\
1.14	0.01\\
1.15	0.01\\
1.16	0.01\\
1.17	0.01\\
1.18	0.01\\
1.19	0.01\\
1.2	0.01\\
1.21	0.01\\
1.22	0.01\\
1.23	0.01\\
1.24	0.01\\
1.25	0.01\\
1.26	0.01\\
1.27	0.01\\
1.28	0.01\\
1.29	0.01\\
1.3	0.01\\
1.31	0.01\\
1.32	0.01\\
1.33	0.01\\
1.34	0.01\\
1.35	0.01\\
1.36	0.01\\
1.37	0.01\\
1.38	0.01\\
1.39	0.01\\
1.4	0.01\\
1.41	0.01\\
1.42	0.01\\
1.43	0.01\\
1.44	0.01\\
1.45	0.01\\
1.46	0.01\\
1.47	0.01\\
1.48	0.01\\
1.49	0.01\\
1.5	0.01\\
1.51	0.01\\
1.52	0.01\\
1.53	0.01\\
1.54	0.01\\
1.55	0.01\\
1.56	0.01\\
1.57	0.01\\
1.58	0.01\\
1.59	0.01\\
1.6	0.01\\
1.61	0.01\\
1.62	0.01\\
1.63	0.01\\
1.64	0.01\\
1.65	0.01\\
1.66	0.01\\
1.67	0.01\\
1.68	0.01\\
1.69	0.01\\
1.7	0.01\\
1.71	0.01\\
1.72	0.01\\
1.73	0.01\\
1.74	0.01\\
1.75	0.01\\
1.76	0.01\\
1.77	0.01\\
1.78	0.01\\
1.79	0.01\\
1.8	0.01\\
1.81	0.01\\
1.82	0.01\\
1.83	0.01\\
1.84	0.01\\
1.85	0.01\\
1.86	0.01\\
1.87	0.01\\
1.88	0.01\\
1.89	0.01\\
1.9	0.01\\
1.91	0.01\\
1.92	0.01\\
1.93	0.01\\
1.94	0.01\\
1.95	0.01\\
1.96	0.01\\
1.97	0.01\\
1.98	0.01\\
1.99	0.01\\
2	0.01\\
2.01	0.01\\
2.02	0.01\\
2.03	0.01\\
2.04	0.01\\
2.05	0.01\\
2.06	0.01\\
2.07	0.01\\
2.08	0.01\\
2.09	0.01\\
2.1	0.01\\
2.11	0.01\\
2.12	0.01\\
2.13	0.01\\
2.14	0.01\\
2.15	0.01\\
2.16	0.01\\
2.17	0.01\\
2.18	0.01\\
2.19	0.01\\
2.2	0.01\\
2.21	0.01\\
2.22	0.01\\
2.23	0.01\\
2.24	0.01\\
2.25	0.01\\
2.26	0.01\\
2.27	0.01\\
2.28	0.01\\
2.29	0.01\\
2.3	0.01\\
2.31	0.01\\
2.32	0.01\\
2.33	0.01\\
2.34	0.01\\
2.35	0.01\\
2.36	0.01\\
2.37	0.01\\
2.38	0.01\\
2.39	0.01\\
2.4	0.01\\
2.41	0.01\\
2.42	0.01\\
2.43	0.01\\
2.44	0.01\\
2.45	0.01\\
2.46	0.01\\
2.47	0.01\\
2.48	0.01\\
2.49	0.01\\
2.5	0.01\\
2.51	0.01\\
2.52	0.01\\
2.53	0.01\\
2.54	0.01\\
2.55	0.01\\
2.56	0.01\\
2.57	0.01\\
2.58	0.01\\
2.59	0.01\\
2.6	0.01\\
2.61	0.01\\
2.62	0.01\\
2.63	0.01\\
2.64	0.01\\
2.65	0.01\\
2.66	0.01\\
2.67	0.01\\
2.68	0.01\\
2.69	0.01\\
2.7	0.01\\
2.71	0.01\\
2.72	0.01\\
2.73	0.01\\
2.74	0.01\\
2.75	0.01\\
2.76	0.01\\
2.77	0.01\\
2.78	0.01\\
2.79	0.01\\
2.8	0.01\\
2.81	0.01\\
2.82	0.01\\
2.83	0.01\\
2.84	0.01\\
2.85	0.01\\
2.86	0.01\\
2.87	0.01\\
2.88	0.01\\
2.89	0.01\\
2.9	0.01\\
2.91	0.01\\
2.92	0.01\\
2.93	0.01\\
2.94	0.01\\
2.95	0.01\\
2.96	0.01\\
2.97	0.01\\
2.98	0.01\\
2.99	0.01\\
3	0.01\\
3.01	0.01\\
3.02	0.01\\
3.03	0.01\\
3.04	0.01\\
3.05	0.01\\
3.06	0.01\\
3.07	0.01\\
3.08	0.01\\
3.09	0.01\\
3.1	0.01\\
3.11	0.01\\
3.12	0.01\\
3.13	0.01\\
3.14	0.01\\
3.15	0.01\\
3.16	0.01\\
3.17	0.01\\
3.18	0.01\\
3.19	0.01\\
3.2	0.01\\
3.21	0.01\\
3.22	0.01\\
3.23	0.01\\
3.24	0.01\\
3.25	0.01\\
3.26	0.01\\
3.27	0.01\\
3.28	0.01\\
3.29	0.01\\
3.3	0.01\\
3.31	0.01\\
3.32	0.01\\
3.33	0.01\\
3.34	0.01\\
3.35	0.01\\
3.36	0.01\\
3.37	0.01\\
3.38	0.01\\
3.39	0.01\\
3.4	0.01\\
3.41	0.01\\
3.42	0.01\\
3.43	0.01\\
3.44	0.01\\
3.45	0.01\\
3.46	0.01\\
3.47	0.01\\
3.48	0.01\\
3.49	0.01\\
3.5	0.01\\
3.51	0.01\\
3.52	0.01\\
3.53	0.01\\
3.54	0.01\\
3.55	0.01\\
3.56	0.01\\
3.57	0.01\\
3.58	0.01\\
3.59	0.01\\
3.6	0.01\\
3.61	0.01\\
3.62	0.01\\
3.63	0.01\\
3.64	0.01\\
3.65	0.01\\
3.66	0.01\\
3.67	0.01\\
3.68	0.01\\
3.69	0.01\\
3.7	0.01\\
3.71	0.01\\
3.72	0.01\\
3.73	0.01\\
3.74	0.01\\
3.75	0.01\\
3.76	0.01\\
3.77	0.01\\
3.78	0.01\\
3.79	0.01\\
3.8	0.01\\
3.81	0.01\\
3.82	0.01\\
3.83	0.01\\
3.84	0.01\\
3.85	0.01\\
3.86	0.01\\
3.87	0.01\\
3.88	0.01\\
3.89	0.01\\
3.9	0.01\\
3.91	0.01\\
3.92	0.01\\
3.93	0.01\\
3.94	0.01\\
3.95	0.01\\
3.96	0.01\\
3.97	0.01\\
3.98	0.01\\
3.99	0.01\\
4	0.01\\
4.01	0.01\\
4.02	0.01\\
4.03	0.01\\
4.04	0.01\\
4.05	0.01\\
4.06	0.01\\
4.07	0.01\\
4.08	0.01\\
4.09	0.01\\
4.1	0.01\\
4.11	0.01\\
4.12	0.01\\
4.13	0.01\\
4.14	0.01\\
4.15	0.01\\
4.16	0.01\\
4.17	0.01\\
4.18	0.01\\
4.19	0.01\\
4.2	0.01\\
4.21	0.01\\
4.22	0.01\\
4.23	0.01\\
4.24	0.01\\
4.25	0.01\\
4.26	0.01\\
4.27	0.01\\
4.28	0.01\\
4.29	0.01\\
4.3	0.01\\
4.31	0.01\\
4.32	0.01\\
4.33	0.01\\
4.34	0.01\\
4.35	0.01\\
4.36	0.01\\
4.37	0.01\\
4.38	0.01\\
4.39	0.01\\
4.4	0.01\\
4.41	0.01\\
4.42	0.01\\
4.43	0.01\\
4.44	0.01\\
4.45	0.01\\
4.46	0.01\\
4.47	0.01\\
4.48	0.01\\
4.49	0.01\\
4.5	0.01\\
4.51	0.01\\
4.52	0.01\\
4.53	0.01\\
4.54	0.01\\
4.55	0.01\\
4.56	0.01\\
4.57	0.01\\
4.58	0.01\\
4.59	0.01\\
4.6	0.01\\
4.61	0.01\\
4.62	0.01\\
4.63	0.01\\
4.64	0.01\\
4.65	0.01\\
4.66	0.01\\
4.67	0.01\\
4.68	0.01\\
4.69	0.01\\
4.7	0.01\\
4.71	0.01\\
4.72	0.01\\
4.73	0.01\\
4.74	0.01\\
4.75	0.01\\
4.76	0.01\\
4.77	0.01\\
4.78	0.01\\
4.79	0.01\\
4.8	0.01\\
4.81	0.01\\
4.82	0.01\\
4.83	0.01\\
4.84	0.01\\
4.85	0.01\\
4.86	0.01\\
4.87	0.01\\
4.88	0.01\\
4.89	0.01\\
4.9	0.01\\
4.91	0.01\\
4.92	0.01\\
4.93	0.01\\
4.94	0.01\\
4.95	0.01\\
4.96	0.01\\
4.97	0.01\\
4.98	0.01\\
4.99	0.01\\
5	0.01\\
5.01	0.01\\
5.02	0.01\\
5.03	0.01\\
5.04	0.01\\
5.05	0.01\\
5.06	0.01\\
5.07	0.01\\
5.08	0.01\\
5.09	0.01\\
5.1	0.01\\
5.11	0.01\\
5.12	0.01\\
5.13	0.01\\
5.14	0.01\\
5.15	0.01\\
5.16	0.01\\
5.17	0.01\\
5.18	0.01\\
5.19	0.01\\
5.2	0.01\\
5.21	0.01\\
5.22	0.01\\
5.23	0.01\\
5.24	0.01\\
5.25	0.01\\
5.26	0.01\\
5.27	0.01\\
5.28	0.01\\
5.29	0.01\\
5.3	0.01\\
5.31	0.01\\
5.32	0.01\\
5.33	0.01\\
5.34	0.01\\
5.35	0.01\\
5.36	0.01\\
5.37	0.01\\
5.38	0.01\\
5.39	0.01\\
5.4	0.01\\
5.41	0.01\\
5.42	0.01\\
5.43	0.01\\
5.44	0.01\\
5.45	0.01\\
5.46	0.01\\
5.47	0.01\\
5.48	0.01\\
5.49	0.01\\
5.5	0.01\\
5.51	0.01\\
5.52	0.01\\
5.53	0.01\\
5.54	0.01\\
5.55	0.01\\
5.56	0.01\\
5.57	0.01\\
5.58	0.01\\
5.59	0.01\\
5.6	0.01\\
5.61	0.01\\
5.62	0.01\\
5.63	0.01\\
5.64	0.01\\
5.65	0.01\\
5.66	0.01\\
5.67	0.01\\
5.68	0.01\\
5.69	0.01\\
5.7	0.01\\
5.71	0.01\\
5.72	0.01\\
5.73	0.01\\
5.74	0.01\\
5.75	0.01\\
5.76	0.01\\
5.77	0.01\\
5.78	0.01\\
5.79	0.01\\
5.8	0.01\\
5.81	0.01\\
5.82	0.01\\
5.83	0.01\\
5.84	0.01\\
5.85	0.01\\
5.86	0.01\\
5.87	0.01\\
5.88	0.01\\
5.89	0.01\\
5.9	0.01\\
5.91	0.01\\
5.92	0.01\\
5.93	0.01\\
5.94	0.01\\
5.95	0.01\\
5.96	0.01\\
5.97	0.01\\
5.98	0.01\\
5.99	0.01\\
6	0.01\\
6.01	0.01\\
6.02	0.01\\
6.03	0.01\\
6.04	0.01\\
6.05	0.01\\
6.06	0.01\\
6.07	0.01\\
6.08	0.01\\
6.09	0.01\\
6.1	0.01\\
6.11	0.01\\
6.12	0.01\\
6.13	0.01\\
6.14	0.01\\
6.15	0.01\\
6.16	0.01\\
6.17	0.01\\
6.18	0.01\\
6.19	0.01\\
6.2	0.01\\
6.21	0.01\\
6.22	0.01\\
6.23	0.01\\
6.24	0.01\\
6.25	0.01\\
6.26	0.01\\
6.27	0.01\\
6.28	0.01\\
6.29	0.01\\
6.3	0.01\\
6.31	0.01\\
6.32	0.01\\
6.33	0.01\\
6.34	0.01\\
6.35	0.01\\
6.36	0.01\\
6.37	0.01\\
6.38	0.01\\
6.39	0.01\\
6.4	0.01\\
6.41	0.01\\
6.42	0.01\\
6.43	0.01\\
6.44	0.01\\
6.45	0.01\\
6.46	0.01\\
6.47	0.01\\
6.48	0.01\\
6.49	0.01\\
6.5	0.01\\
6.51	0.01\\
6.52	0.01\\
6.53	0.01\\
6.54	0.01\\
6.55	0.01\\
6.56	0.01\\
6.57	0.01\\
6.58	0.01\\
6.59	0.01\\
6.6	0.01\\
6.61	0.01\\
6.62	0.01\\
6.63	0.01\\
6.64	0.01\\
6.65	0.01\\
6.66	0.01\\
6.67	0.01\\
6.68	0.01\\
6.69	0.01\\
6.7	0.01\\
6.71	0.01\\
6.72	0.01\\
6.73	0.01\\
6.74	0.01\\
6.75	0.01\\
6.76	0.01\\
6.77	0.01\\
6.78	0.01\\
6.79	0.01\\
6.8	0.01\\
6.81	0.01\\
6.82	0.01\\
6.83	0.01\\
6.84	0.01\\
6.85	0.01\\
6.86	0.01\\
6.87	0.01\\
6.88	0.01\\
6.89	0.01\\
6.9	0.01\\
6.91	0.01\\
6.92	0.01\\
6.93	0.01\\
6.94	0.01\\
6.95	0.01\\
6.96	0.01\\
6.97	0.01\\
6.98	0.01\\
6.99	0.01\\
7	0.01\\
7.01	0.01\\
7.02	0.01\\
7.03	0.01\\
7.04	0.01\\
7.05	0.01\\
7.06	0.01\\
7.07	0.01\\
7.08	0.01\\
7.09	0.01\\
7.1	0.01\\
7.11	0.01\\
7.12	0.01\\
7.13	0.01\\
7.14	0.01\\
7.15	0.01\\
7.16	0.01\\
7.17	0.01\\
7.18	0.01\\
7.19	0.01\\
7.2	0.01\\
7.21	0.01\\
7.22	0.01\\
7.23	0.01\\
7.24	0.01\\
7.25	0.01\\
7.26	0.01\\
7.27	0.01\\
7.28	0.01\\
7.29	0.01\\
7.3	0.01\\
7.31	0.01\\
7.32	0.01\\
7.33	0.01\\
7.34	0.01\\
7.35	0.01\\
7.36	0.01\\
7.37	0.01\\
7.38	0.01\\
7.39	0.01\\
7.4	0.01\\
7.41	0.01\\
7.42	0.01\\
7.43	0.01\\
7.44	0.01\\
7.45	0.01\\
7.46	0.01\\
7.47	0.01\\
7.48	0.01\\
7.49	0.01\\
7.5	0.01\\
7.51	0.01\\
7.52	0.01\\
7.53	0.01\\
7.54	0.01\\
7.55	0.01\\
7.56	0.01\\
7.57	0.01\\
7.58	0.01\\
7.59	0.01\\
7.6	0.01\\
7.61	0.01\\
7.62	0.01\\
7.63	0.01\\
7.64	0.01\\
7.65	0.01\\
7.66	0.01\\
7.67	0.01\\
7.68	0.01\\
7.69	0.01\\
7.7	0.01\\
7.71	0.01\\
7.72	0.01\\
7.73	0.01\\
7.74	0.01\\
7.75	0.01\\
7.76	0.01\\
7.77	0.01\\
7.78	0.01\\
7.79	0.01\\
7.8	0.01\\
7.81	0.01\\
7.82	0.01\\
7.83	0.01\\
7.84	0.01\\
7.85	0.01\\
7.86	0.01\\
7.87	0.01\\
7.88	0.01\\
7.89	0.01\\
7.9	0.01\\
7.91	0.01\\
7.92	0.01\\
7.93	0.01\\
7.94	0.01\\
7.95	0.01\\
7.96	0.01\\
7.97	0.01\\
7.98	0.01\\
7.99	0.01\\
8	0.01\\
8.01	0.01\\
8.02	0.01\\
8.03	0.01\\
8.04	0.01\\
8.05	0.01\\
8.06	0.01\\
8.07	0.01\\
8.08	0.01\\
8.09	0.01\\
8.1	0.01\\
8.11	0.01\\
8.12	0.01\\
8.13	0.01\\
8.14	0.01\\
8.15	0.01\\
8.16	0.01\\
8.17	0.01\\
8.18	0.01\\
8.19	0.01\\
8.2	0.01\\
8.21	0.01\\
8.22	0.01\\
8.23	0.01\\
8.24	0.01\\
8.25	0.01\\
8.26	0.01\\
8.27	0.01\\
8.28	0.01\\
8.29	0.01\\
8.3	0.01\\
8.31	0.01\\
8.32	0.01\\
8.33	0.01\\
8.34	0.01\\
8.35	0.01\\
8.36	0.01\\
8.37	0.01\\
8.38	0.01\\
8.39	0.01\\
8.4	0.01\\
8.41	0.01\\
8.42	0.01\\
8.43	0.01\\
8.44	0.01\\
8.45	0.01\\
8.46	0.01\\
8.47	0.01\\
8.48	0.01\\
8.49	0.01\\
8.5	0.01\\
8.51	0.01\\
8.52	0.01\\
8.53	0.01\\
8.54	0.01\\
8.55	0.01\\
8.56	0.01\\
8.57	0.01\\
8.58	0.01\\
8.59	0.01\\
8.6	0.01\\
8.61	0.01\\
8.62	0.01\\
8.63	0.01\\
8.64	0.01\\
8.65	0.01\\
8.66	0.01\\
8.67	0.01\\
8.68	0.01\\
8.69	0.01\\
8.7	0.01\\
8.71	0.01\\
8.72	0.01\\
8.73	0.01\\
8.74	0.01\\
8.75	0.01\\
8.76	0.01\\
8.77	0.01\\
8.78	0.01\\
8.79	0.01\\
8.8	0.01\\
8.81	0.01\\
8.82	0.01\\
8.83	0.01\\
8.84	0.01\\
8.85	0.01\\
8.86	0.01\\
8.87	0.01\\
8.88	0.01\\
8.89	0.01\\
8.9	0.01\\
8.91	0.01\\
8.92	0.01\\
8.93	0.01\\
8.94	0.01\\
8.95	0.01\\
8.96	0.01\\
8.97	0.01\\
8.98	0.01\\
8.99	0.01\\
9	0.01\\
9.01	0.01\\
9.02	0.01\\
9.03	0.01\\
9.04	0.01\\
9.05	0.01\\
9.06	0.01\\
9.07	0.01\\
9.08	0.01\\
9.09	0.01\\
9.1	0.01\\
9.11	0.01\\
9.12	0.01\\
9.13	0.01\\
9.14	0.01\\
9.15	0.01\\
9.16	0.01\\
9.17	0.01\\
9.18	0.01\\
9.19	0.01\\
9.2	0.01\\
9.21	0.01\\
9.22	0.01\\
9.23	0.01\\
9.24	0.01\\
9.25	0.01\\
9.26	0.01\\
9.27	0.01\\
9.28	0.01\\
9.29	0.01\\
9.3	0.01\\
9.31	0.01\\
9.32	0.01\\
9.33	0.01\\
9.34	0.01\\
9.35	0.01\\
9.36	0.01\\
9.37	0.01\\
9.38	0.01\\
9.39	0.01\\
9.4	0.01\\
9.41	0.01\\
9.42	0.01\\
9.43	0.01\\
9.44	0.01\\
9.45	0.01\\
9.46	0.01\\
9.47	0.01\\
9.48	0.01\\
9.49	0.01\\
9.5	0.01\\
9.51	0.01\\
9.52	0.01\\
9.53	0.01\\
9.54	0.01\\
9.55	0.01\\
9.56	0.01\\
9.57	0.01\\
9.58	0.01\\
9.59	0.01\\
9.6	0.01\\
9.61	0.01\\
9.62	0.01\\
9.63	0.01\\
9.64	0.01\\
9.65	0.01\\
9.66	0.01\\
9.67	0.01\\
9.68	0.01\\
9.69	0.01\\
9.7	0.01\\
9.71	0.01\\
9.72	0.01\\
9.73	0.01\\
9.74	0.01\\
9.75	0.01\\
9.76	0.01\\
9.77	0.01\\
9.78	0.01\\
9.79	0.01\\
9.8	0.01\\
9.81	0.01\\
9.82	0.01\\
9.83	0.01\\
9.84	0.01\\
9.85	0.01\\
9.86	0.01\\
9.87	0.01\\
9.88	0.01\\
9.89	0.01\\
9.9	0.01\\
9.91	0.01\\
9.92	0.01\\
9.93	0.01\\
9.94	0.01\\
9.95	0.01\\
9.96	0.01\\
9.97	0.01\\
9.98	0.01\\
9.99	0.01\\
10	0.01\\
10.01	0.01\\
10.02	0.01\\
10.03	0.01\\
10.04	0.01\\
10.05	0.01\\
10.06	0.01\\
10.07	0.01\\
10.08	0.01\\
10.09	0.01\\
10.1	0.01\\
10.11	0.01\\
10.12	0.01\\
10.13	0.01\\
10.14	0.01\\
10.15	0.01\\
10.16	0.01\\
10.17	0.01\\
10.18	0.01\\
10.19	0.01\\
10.2	0.01\\
10.21	0.01\\
10.22	0.01\\
10.23	0.01\\
10.24	0.01\\
10.25	0.01\\
10.26	0.01\\
10.27	0.01\\
10.28	0.01\\
10.29	0.01\\
10.3	0.01\\
10.31	0.01\\
10.32	0.01\\
10.33	0.01\\
10.34	0.01\\
10.35	0.01\\
10.36	0.01\\
10.37	0.01\\
10.38	0.01\\
10.39	0.01\\
10.4	0.01\\
10.41	0.01\\
10.42	0.01\\
10.43	0.01\\
10.44	0.01\\
10.45	0.01\\
10.46	0.01\\
10.47	0.01\\
10.48	0.01\\
10.49	0.01\\
10.5	0.01\\
10.51	0.01\\
10.52	0.01\\
10.53	0.01\\
10.54	0.01\\
10.55	0.01\\
10.56	0.01\\
10.57	0.01\\
10.58	0.01\\
10.59	0.01\\
10.6	0.01\\
10.61	0.01\\
10.62	0.01\\
10.63	0.01\\
10.64	0.01\\
10.65	0.01\\
10.66	0.01\\
10.67	0.01\\
10.68	0.01\\
10.69	0.01\\
10.7	0.01\\
10.71	0.01\\
10.72	0.01\\
10.73	0.01\\
10.74	0.01\\
10.75	0.01\\
10.76	0.01\\
10.77	0.01\\
10.78	0.01\\
10.79	0.01\\
10.8	0.01\\
10.81	0.01\\
10.82	0.01\\
10.83	0.01\\
10.84	0.01\\
10.85	0.01\\
10.86	0.01\\
10.87	0.01\\
10.88	0.01\\
10.89	0.01\\
10.9	0.01\\
10.91	0.01\\
10.92	0.01\\
10.93	0.01\\
10.94	0.01\\
10.95	0.01\\
10.96	0.01\\
10.97	0.01\\
10.98	0.01\\
10.99	0.01\\
11	0.01\\
11.01	0.01\\
11.02	0.01\\
11.03	0.01\\
11.04	0.01\\
11.05	0.01\\
11.06	0.01\\
11.07	0.01\\
11.08	0.01\\
11.09	0.01\\
11.1	0.01\\
11.11	0.01\\
11.12	0.01\\
11.13	0.01\\
11.14	0.01\\
11.15	0.01\\
11.16	0.01\\
11.17	0.01\\
11.18	0.01\\
11.19	0.01\\
11.2	0.01\\
11.21	0.01\\
11.22	0.01\\
11.23	0.01\\
11.24	0.01\\
11.25	0.01\\
11.26	0.01\\
11.27	0.01\\
11.28	0.01\\
11.29	0.01\\
11.3	0.01\\
11.31	0.01\\
11.32	0.01\\
11.33	0.01\\
11.34	0.01\\
11.35	0.01\\
11.36	0.01\\
11.37	0.01\\
11.38	0.01\\
11.39	0.01\\
11.4	0.01\\
11.41	0.01\\
11.42	0.01\\
11.43	0.01\\
11.44	0.01\\
11.45	0.01\\
11.46	0.01\\
11.47	0.01\\
11.48	0.01\\
11.49	0.01\\
11.5	0.01\\
11.51	0.01\\
11.52	0.01\\
11.53	0.01\\
11.54	0.01\\
11.55	0.01\\
11.56	0.01\\
11.57	0.01\\
11.58	0.01\\
11.59	0.01\\
11.6	0.01\\
11.61	0.01\\
11.62	0.01\\
11.63	0.01\\
11.64	0.01\\
11.65	0.01\\
11.66	0.01\\
11.67	0.01\\
11.68	0.01\\
11.69	0.01\\
11.7	0.01\\
11.71	0.01\\
11.72	0.01\\
11.73	0.01\\
11.74	0.01\\
11.75	0.01\\
11.76	0.01\\
11.77	0.01\\
11.78	0.01\\
11.79	0.01\\
11.8	0.01\\
11.81	0.01\\
11.82	0.01\\
11.83	0.01\\
11.84	0.01\\
11.85	0.01\\
11.86	0.01\\
11.87	0.01\\
11.88	0.01\\
11.89	0.01\\
11.9	0.01\\
11.91	0.01\\
11.92	0.01\\
11.93	0.01\\
11.94	0.01\\
11.95	0.01\\
11.96	0.01\\
11.97	0.01\\
11.98	0.01\\
11.99	0.01\\
12	0.01\\
12.01	0.01\\
12.02	0.01\\
12.03	0.01\\
12.04	0.01\\
12.05	0.01\\
12.06	0.01\\
12.07	0.01\\
12.08	0.01\\
12.09	0.01\\
12.1	0.01\\
12.11	0.01\\
12.12	0.01\\
12.13	0.01\\
12.14	0.01\\
12.15	0.01\\
12.16	0.01\\
12.17	0.01\\
12.18	0.01\\
12.19	0.01\\
12.2	0.01\\
12.21	0.01\\
12.22	0.01\\
12.23	0.01\\
12.24	0.01\\
12.25	0.01\\
12.26	0.01\\
12.27	0.01\\
12.28	0.01\\
12.29	0.01\\
12.3	0.01\\
12.31	0.01\\
12.32	0.01\\
12.33	0.01\\
12.34	0.01\\
12.35	0.01\\
12.36	0.01\\
12.37	0.01\\
12.38	0.01\\
12.39	0.01\\
12.4	0.01\\
12.41	0.01\\
12.42	0.01\\
12.43	0.01\\
12.44	0.01\\
12.45	0.01\\
12.46	0.01\\
12.47	0.01\\
12.48	0.01\\
12.49	0.01\\
12.5	0.01\\
12.51	0.01\\
12.52	0.01\\
12.53	0.01\\
12.54	0.01\\
12.55	0.01\\
12.56	0.01\\
12.57	0.01\\
12.58	0.01\\
12.59	0.01\\
12.6	0.01\\
12.61	0.01\\
12.62	0.01\\
12.63	0.01\\
12.64	0.01\\
12.65	0.01\\
12.66	0.01\\
12.67	0.01\\
12.68	0.01\\
12.69	0.01\\
12.7	0.01\\
12.71	0.01\\
12.72	0.01\\
12.73	0.01\\
12.74	0.01\\
12.75	0.01\\
12.76	0.01\\
12.77	0.01\\
12.78	0.01\\
12.79	0.01\\
12.8	0.01\\
12.81	0.01\\
12.82	0.01\\
12.83	0.01\\
12.84	0.01\\
12.85	0.01\\
12.86	0.01\\
12.87	0.01\\
12.88	0.01\\
12.89	0.01\\
12.9	0.01\\
12.91	0.01\\
12.92	0.01\\
12.93	0.01\\
12.94	0.01\\
12.95	0.01\\
12.96	0.01\\
12.97	0.01\\
12.98	0.01\\
12.99	0.01\\
13	0.01\\
13.01	0.01\\
13.02	0.01\\
13.03	0.01\\
13.04	0.01\\
13.05	0.01\\
13.06	0.01\\
13.07	0.01\\
13.08	0.01\\
13.09	0.01\\
13.1	0.01\\
13.11	0.01\\
13.12	0.01\\
13.13	0.01\\
13.14	0.01\\
13.15	0.01\\
13.16	0.01\\
13.17	0.01\\
13.18	0.01\\
13.19	0.01\\
13.2	0.01\\
13.21	0.01\\
13.22	0.01\\
13.23	0.01\\
13.24	0.01\\
13.25	0.01\\
13.26	0.01\\
13.27	0.01\\
13.28	0.01\\
13.29	0.01\\
13.3	0.01\\
13.31	0.01\\
13.32	0.01\\
13.33	0.01\\
13.34	0.01\\
13.35	0.01\\
13.36	0.01\\
13.37	0.01\\
13.38	0.01\\
13.39	0.01\\
13.4	0.01\\
13.41	0.01\\
13.42	0.01\\
13.43	0.01\\
13.44	0.01\\
13.45	0.01\\
13.46	0.01\\
13.47	0.01\\
13.48	0.01\\
13.49	0.01\\
13.5	0.01\\
13.51	0.01\\
13.52	0.01\\
13.53	0.01\\
13.54	0.01\\
13.55	0.01\\
13.56	0.01\\
13.57	0.01\\
13.58	0.01\\
13.59	0.01\\
13.6	0.01\\
13.61	0.01\\
13.62	0.01\\
13.63	0.01\\
13.64	0.01\\
13.65	0.01\\
13.66	0.01\\
13.67	0.01\\
13.68	0.01\\
13.69	0.01\\
13.7	0.01\\
13.71	0.01\\
13.72	0.01\\
13.73	0.01\\
13.74	0.01\\
13.75	0.01\\
13.76	0.01\\
13.77	0.01\\
13.78	0.01\\
13.79	0.01\\
13.8	0.01\\
13.81	0.01\\
13.82	0.01\\
13.83	0.01\\
13.84	0.01\\
13.85	0.01\\
13.86	0.01\\
13.87	0.01\\
13.88	0.01\\
13.89	0.01\\
13.9	0.01\\
13.91	0.01\\
13.92	0.01\\
13.93	0.01\\
13.94	0.01\\
13.95	0.01\\
13.96	0.01\\
13.97	0.01\\
13.98	0.01\\
13.99	0.01\\
14	0.01\\
14.01	0.01\\
14.02	0.01\\
14.03	0.01\\
14.04	0.01\\
14.05	0.01\\
14.06	0.01\\
14.07	0.01\\
14.08	0.01\\
14.09	0.01\\
14.1	0.01\\
14.11	0.01\\
14.12	0.01\\
14.13	0.01\\
14.14	0.01\\
14.15	0.01\\
14.16	0.01\\
14.17	0.01\\
14.18	0.01\\
14.19	0.01\\
14.2	0.01\\
14.21	0.01\\
14.22	0.01\\
14.23	0.01\\
14.24	0.01\\
14.25	0.01\\
14.26	0.01\\
14.27	0.01\\
14.28	0.01\\
14.29	0.01\\
14.3	0.01\\
14.31	0.01\\
14.32	0.01\\
14.33	0.01\\
14.34	0.01\\
14.35	0.01\\
14.36	0.01\\
14.37	0.01\\
14.38	0.01\\
14.39	0.01\\
14.4	0.01\\
14.41	0.01\\
14.42	0.01\\
14.43	0.01\\
14.44	0.01\\
14.45	0.01\\
14.46	0.01\\
14.47	0.01\\
14.48	0.01\\
14.49	0.01\\
14.5	0.01\\
14.51	0.01\\
14.52	0.01\\
14.53	0.01\\
14.54	0.01\\
14.55	0.01\\
14.56	0.01\\
14.57	0.01\\
14.58	0.01\\
14.59	0.01\\
14.6	0.01\\
14.61	0.01\\
14.62	0.01\\
14.63	0.01\\
14.64	0.01\\
14.65	0.01\\
14.66	0.01\\
14.67	0.01\\
14.68	0.01\\
14.69	0.01\\
14.7	0.01\\
14.71	0.01\\
14.72	0.01\\
14.73	0.01\\
14.74	0.01\\
14.75	0.01\\
14.76	0.01\\
14.77	0.01\\
14.78	0.01\\
14.79	0.01\\
14.8	0.01\\
14.81	0.01\\
14.82	0.01\\
14.83	0.01\\
14.84	0.01\\
14.85	0.01\\
14.86	0.01\\
14.87	0.01\\
14.88	0.01\\
14.89	0.01\\
14.9	0.01\\
14.91	0.01\\
14.92	0.01\\
14.93	0.01\\
14.94	0.01\\
14.95	0.01\\
14.96	0.01\\
14.97	0.01\\
14.98	0.01\\
14.99	0.01\\
15	0.01\\
15.01	0.01\\
15.02	0.01\\
15.03	0.01\\
15.04	0.01\\
15.05	0.01\\
15.06	0.01\\
15.07	0.01\\
15.08	0.01\\
15.09	0.01\\
15.1	0.01\\
15.11	0.01\\
15.12	0.01\\
15.13	0.01\\
15.14	0.01\\
15.15	0.01\\
15.16	0.01\\
15.17	0.01\\
15.18	0.01\\
15.19	0.01\\
15.2	0.01\\
15.21	0.01\\
15.22	0.01\\
15.23	0.01\\
15.24	0.01\\
15.25	0.01\\
15.26	0.01\\
15.27	0.01\\
15.28	0.01\\
15.29	0.01\\
15.3	0.01\\
15.31	0.01\\
15.32	0.01\\
15.33	0.01\\
15.34	0.01\\
15.35	0.01\\
15.36	0.01\\
15.37	0.01\\
15.38	0.01\\
15.39	0.01\\
15.4	0.01\\
15.41	0.01\\
15.42	0.01\\
15.43	0.01\\
15.44	0.01\\
15.45	0.01\\
15.46	0.01\\
15.47	0.01\\
15.48	0.01\\
15.49	0.01\\
15.5	0.01\\
15.51	0.01\\
15.52	0.01\\
15.53	0.01\\
15.54	0.01\\
15.55	0.01\\
15.56	0.01\\
15.57	0.01\\
15.58	0.01\\
15.59	0.01\\
15.6	0.01\\
15.61	0.01\\
15.62	0.01\\
15.63	0.01\\
15.64	0.01\\
15.65	0.01\\
15.66	0.01\\
15.67	0.01\\
15.68	0.01\\
15.69	0.01\\
15.7	0.01\\
15.71	0.01\\
15.72	0.01\\
15.73	0.01\\
15.74	0.01\\
15.75	0.01\\
15.76	0.01\\
15.77	0.01\\
15.78	0.01\\
15.79	0.01\\
15.8	0.01\\
15.81	0.01\\
15.82	0.01\\
15.83	0.01\\
15.84	0.01\\
15.85	0.01\\
15.86	0.01\\
15.87	0.01\\
15.88	0.01\\
15.89	0.01\\
15.9	0.01\\
15.91	0.01\\
15.92	0.01\\
15.93	0.01\\
15.94	0.01\\
15.95	0.01\\
15.96	0.01\\
15.97	0.01\\
15.98	0.01\\
15.99	0.01\\
16	0.01\\
16.01	0.01\\
16.02	0.01\\
16.03	0.01\\
16.04	0.01\\
16.05	0.01\\
16.06	0.01\\
16.07	0.01\\
16.08	0.01\\
16.09	0.01\\
16.1	0.01\\
16.11	0.01\\
16.12	0.01\\
16.13	0.01\\
16.14	0.01\\
16.15	0.01\\
16.16	0.01\\
16.17	0.01\\
16.18	0.01\\
16.19	0.01\\
16.2	0.01\\
16.21	0.01\\
16.22	0.01\\
16.23	0.01\\
16.24	0.01\\
16.25	0.01\\
16.26	0.01\\
16.27	0.01\\
16.28	0.01\\
16.29	0.01\\
16.3	0.01\\
16.31	0.01\\
16.32	0.01\\
16.33	0.01\\
16.34	0.01\\
16.35	0.01\\
16.36	0.01\\
16.37	0.01\\
16.38	0.01\\
16.39	0.01\\
16.4	0.01\\
16.41	0.01\\
16.42	0.01\\
16.43	0.01\\
16.44	0.01\\
16.45	0.01\\
16.46	0.01\\
16.47	0.01\\
16.48	0.01\\
16.49	0.01\\
16.5	0.01\\
16.51	0.01\\
16.52	0.01\\
16.53	0.01\\
16.54	0.01\\
16.55	0.01\\
16.56	0.01\\
16.57	0.01\\
16.58	0.01\\
16.59	0.01\\
16.6	0.01\\
16.61	0.01\\
16.62	0.01\\
16.63	0.01\\
16.64	0.01\\
16.65	0.01\\
16.66	0.01\\
16.67	0.01\\
16.68	0.01\\
16.69	0.01\\
16.7	0.01\\
16.71	0.01\\
16.72	0.01\\
16.73	0.01\\
16.74	0.01\\
16.75	0.01\\
16.76	0.01\\
16.77	0.01\\
16.78	0.01\\
16.79	0.01\\
16.8	0.01\\
16.81	0.01\\
16.82	0.01\\
16.83	0.01\\
16.84	0.01\\
16.85	0.01\\
16.86	0.01\\
16.87	0.01\\
16.88	0.01\\
16.89	0.01\\
16.9	0.01\\
16.91	0.01\\
16.92	0.01\\
16.93	0.01\\
16.94	0.01\\
16.95	0.01\\
16.96	0.01\\
16.97	0.01\\
16.98	0.01\\
16.99	0.01\\
17	0.01\\
17.01	0.01\\
17.02	0.01\\
17.03	0.01\\
17.04	0.01\\
17.05	0.01\\
17.06	0.01\\
17.07	0.01\\
17.08	0.01\\
17.09	0.01\\
17.1	0.01\\
17.11	0.01\\
17.12	0.01\\
17.13	0.01\\
17.14	0.01\\
17.15	0.01\\
17.16	0.01\\
17.17	0.01\\
17.18	0.01\\
17.19	0.01\\
17.2	0.01\\
17.21	0.01\\
17.22	0.01\\
17.23	0.01\\
17.24	0.01\\
17.25	0.01\\
17.26	0.01\\
17.27	0.01\\
17.28	0.01\\
17.29	0.01\\
17.3	0.01\\
17.31	0.01\\
17.32	0.01\\
17.33	0.01\\
17.34	0.01\\
17.35	0.01\\
17.36	0.01\\
17.37	0.01\\
17.38	0.01\\
17.39	0.01\\
17.4	0.01\\
17.41	0.01\\
17.42	0.01\\
17.43	0.01\\
17.44	0.01\\
17.45	0.01\\
17.46	0.01\\
17.47	0.01\\
17.48	0.01\\
17.49	0.01\\
17.5	0.01\\
17.51	0.01\\
17.52	0.01\\
17.53	0.01\\
17.54	0.01\\
17.55	0.01\\
17.56	0.01\\
17.57	0.01\\
17.58	0.01\\
17.59	0.01\\
17.6	0.01\\
17.61	0.01\\
17.62	0.01\\
17.63	0.01\\
17.64	0.01\\
17.65	0.01\\
17.66	0.01\\
17.67	0.01\\
17.68	0.01\\
17.69	0.01\\
17.7	0.01\\
17.71	0.01\\
17.72	0.01\\
17.73	0.01\\
17.74	0.01\\
17.75	0.01\\
17.76	0.01\\
17.77	0.01\\
17.78	0.01\\
17.79	0.01\\
17.8	0.01\\
17.81	0.01\\
17.82	0.01\\
17.83	0.01\\
17.84	0.01\\
17.85	0.01\\
17.86	0.01\\
17.87	0.01\\
17.88	0.01\\
17.89	0.01\\
17.9	0.01\\
17.91	0.01\\
17.92	0.01\\
17.93	0.01\\
17.94	0.01\\
17.95	0.01\\
17.96	0.01\\
17.97	0.01\\
17.98	0.01\\
17.99	0.01\\
18	0.01\\
18.01	0.01\\
18.02	0.01\\
18.03	0.01\\
18.04	0.01\\
18.05	0.01\\
18.06	0.01\\
18.07	0.01\\
18.08	0.01\\
18.09	0.01\\
18.1	0.01\\
18.11	0.01\\
18.12	0.01\\
18.13	0.01\\
18.14	0.01\\
18.15	0.01\\
18.16	0.01\\
18.17	0.01\\
18.18	0.01\\
18.19	0.01\\
18.2	0.01\\
18.21	0.01\\
18.22	0.01\\
18.23	0.01\\
18.24	0.01\\
18.25	0.01\\
18.26	0.01\\
18.27	0.01\\
18.28	0.01\\
18.29	0.01\\
18.3	0.01\\
18.31	0.01\\
18.32	0.01\\
18.33	0.01\\
18.34	0.01\\
18.35	0.01\\
18.36	0.01\\
18.37	0.01\\
18.38	0.01\\
18.39	0.01\\
18.4	0.01\\
18.41	0.01\\
18.42	0.01\\
18.43	0.01\\
18.44	0.01\\
18.45	0.01\\
18.46	0.01\\
18.47	0.01\\
18.48	0.01\\
18.49	0.01\\
18.5	0.01\\
18.51	0.01\\
18.52	0.01\\
18.53	0.01\\
18.54	0.01\\
18.55	0.01\\
18.56	0.01\\
18.57	0.01\\
18.58	0.01\\
18.59	0.01\\
18.6	0.01\\
18.61	0.01\\
18.62	0.01\\
18.63	0.01\\
18.64	0.01\\
18.65	0.01\\
18.66	0.01\\
18.67	0.01\\
18.68	0.01\\
18.69	0.01\\
18.7	0.01\\
18.71	0.01\\
18.72	0.01\\
18.73	0.01\\
18.74	0.01\\
18.75	0.01\\
18.76	0.01\\
18.77	0.01\\
18.78	0.01\\
18.79	0.01\\
18.8	0.01\\
18.81	0.01\\
18.82	0.01\\
18.83	0.01\\
18.84	0.01\\
18.85	0.01\\
18.86	0.01\\
18.87	0.01\\
18.88	0.01\\
18.89	0.01\\
18.9	0.01\\
18.91	0.01\\
18.92	0.01\\
18.93	0.01\\
18.94	0.01\\
18.95	0.01\\
18.96	0.01\\
18.97	0.01\\
18.98	0.01\\
18.99	0.01\\
19	0.01\\
19.01	0.01\\
19.02	0.01\\
19.03	0.01\\
19.04	0.01\\
19.05	0.01\\
19.06	0.01\\
19.07	0.01\\
19.08	0.01\\
19.09	0.01\\
19.1	0.01\\
19.11	0.01\\
19.12	0.01\\
19.13	0.01\\
19.14	0.01\\
19.15	0.01\\
19.16	0.01\\
19.17	0.01\\
19.18	0.01\\
19.19	0.01\\
19.2	0.01\\
19.21	0.01\\
19.22	0.01\\
19.23	0.01\\
19.24	0.01\\
19.25	0.01\\
19.26	0.01\\
19.27	0.01\\
19.28	0.01\\
19.29	0.01\\
19.3	0.01\\
19.31	0.01\\
19.32	0.01\\
19.33	0.01\\
19.34	0.01\\
19.35	0.01\\
19.36	0.01\\
19.37	0.01\\
19.38	0.01\\
19.39	0.01\\
19.4	0.01\\
19.41	0.01\\
19.42	0.01\\
19.43	0.01\\
19.44	0.01\\
19.45	0.01\\
19.46	0.01\\
19.47	0.01\\
19.48	0.01\\
19.49	0.01\\
19.5	0.01\\
19.51	0.01\\
19.52	0.01\\
19.53	0.01\\
19.54	0.01\\
19.55	0.01\\
19.56	0.01\\
19.57	0.01\\
19.58	0.01\\
19.59	0.01\\
19.6	0.01\\
19.61	0.01\\
19.62	0.01\\
19.63	0.01\\
19.64	0.01\\
19.65	0.01\\
19.66	0.01\\
19.67	0.01\\
19.68	0.01\\
19.69	0.01\\
19.7	0.01\\
19.71	0.01\\
19.72	0.01\\
19.73	0.01\\
19.74	0.01\\
19.75	0.01\\
19.76	0.01\\
19.77	0.01\\
19.78	0.01\\
19.79	0.01\\
19.8	0.01\\
19.81	0.01\\
19.82	0.01\\
19.83	0.01\\
19.84	0.01\\
19.85	0.01\\
19.86	0.01\\
19.87	0.01\\
19.88	0.01\\
19.89	0.01\\
19.9	0.01\\
19.91	0.01\\
19.92	0.01\\
19.93	0.01\\
19.94	0.01\\
19.95	0.01\\
19.96	0.01\\
19.97	0.01\\
19.98	0.01\\
19.99	0.01\\
20	0.01\\
20.01	0.01\\
20.02	0.01\\
20.03	0.01\\
20.04	0.01\\
20.05	0.01\\
20.06	0.01\\
20.07	0.01\\
20.08	0.01\\
20.09	0.01\\
20.1	0.01\\
20.11	0.01\\
20.12	0.01\\
20.13	0.01\\
20.14	0.01\\
20.15	0.01\\
20.16	0.01\\
20.17	0.01\\
20.18	0.01\\
20.19	0.01\\
20.2	0.01\\
20.21	0.01\\
20.22	0.01\\
20.23	0.01\\
20.24	0.01\\
20.25	0.01\\
20.26	0.01\\
20.27	0.01\\
20.28	0.01\\
20.29	0.01\\
20.3	0.01\\
20.31	0.01\\
20.32	0.01\\
20.33	0.01\\
20.34	0.01\\
20.35	0.01\\
20.36	0.01\\
20.37	0.01\\
20.38	0.01\\
20.39	0.01\\
20.4	0.01\\
20.41	0.01\\
20.42	0.01\\
20.43	0.01\\
20.44	0.01\\
20.45	0.01\\
20.46	0.01\\
20.47	0.01\\
20.48	0.01\\
20.49	0.01\\
20.5	0.01\\
20.51	0.01\\
20.52	0.01\\
20.53	0.01\\
20.54	0.01\\
20.55	0.01\\
20.56	0.01\\
20.57	0.01\\
20.58	0.01\\
20.59	0.01\\
20.6	0.01\\
20.61	0.01\\
20.62	0.01\\
20.63	0.01\\
20.64	0.01\\
20.65	0.01\\
20.66	0.01\\
20.67	0.01\\
20.68	0.01\\
20.69	0.01\\
20.7	0.01\\
20.71	0.01\\
20.72	0.01\\
20.73	0.01\\
20.74	0.01\\
20.75	0.01\\
20.76	0.01\\
20.77	0.01\\
20.78	0.01\\
20.79	0.01\\
20.8	0.01\\
20.81	0.01\\
20.82	0.01\\
20.83	0.01\\
20.84	0.01\\
20.85	0.01\\
20.86	0.01\\
20.87	0.01\\
20.88	0.01\\
20.89	0.01\\
20.9	0.01\\
20.91	0.01\\
20.92	0.01\\
20.93	0.01\\
20.94	0.01\\
20.95	0.01\\
20.96	0.01\\
20.97	0.01\\
20.98	0.01\\
20.99	0.01\\
21	0.01\\
21.01	0.01\\
21.02	0.01\\
21.03	0.01\\
21.04	0.01\\
21.05	0.01\\
21.06	0.01\\
21.07	0.01\\
21.08	0.01\\
21.09	0.01\\
21.1	0.01\\
21.11	0.01\\
21.12	0.01\\
21.13	0.01\\
21.14	0.01\\
21.15	0.01\\
21.16	0.01\\
21.17	0.01\\
21.18	0.01\\
21.19	0.01\\
21.2	0.01\\
21.21	0.01\\
21.22	0.01\\
21.23	0.01\\
21.24	0.01\\
21.25	0.01\\
21.26	0.01\\
21.27	0.01\\
21.28	0.01\\
21.29	0.01\\
21.3	0.01\\
21.31	0.01\\
21.32	0.01\\
21.33	0.01\\
21.34	0.01\\
21.35	0.01\\
21.36	0.01\\
21.37	0.01\\
21.38	0.01\\
21.39	0.01\\
21.4	0.01\\
21.41	0.01\\
21.42	0.01\\
21.43	0.01\\
21.44	0.01\\
21.45	0.01\\
21.46	0.01\\
21.47	0.01\\
21.48	0.01\\
21.49	0.01\\
21.5	0.01\\
21.51	0.01\\
21.52	0.01\\
21.53	0.01\\
21.54	0.01\\
21.55	0.01\\
21.56	0.01\\
21.57	0.01\\
21.58	0.01\\
21.59	0.01\\
21.6	0.01\\
21.61	0.01\\
21.62	0.01\\
21.63	0.01\\
21.64	0.01\\
21.65	0.01\\
21.66	0.01\\
21.67	0.01\\
21.68	0.01\\
21.69	0.01\\
21.7	0.01\\
21.71	0.01\\
21.72	0.01\\
21.73	0.01\\
21.74	0.01\\
21.75	0.01\\
21.76	0.01\\
21.77	0.01\\
21.78	0.01\\
21.79	0.01\\
21.8	0.01\\
21.81	0.01\\
21.82	0.01\\
21.83	0.01\\
21.84	0.01\\
21.85	0.01\\
21.86	0.01\\
21.87	0.01\\
21.88	0.01\\
21.89	0.01\\
21.9	0.01\\
21.91	0.01\\
21.92	0.01\\
21.93	0.01\\
21.94	0.01\\
21.95	0.01\\
21.96	0.01\\
21.97	0.01\\
21.98	0.01\\
21.99	0.01\\
22	0.01\\
22.01	0.01\\
22.02	0.01\\
22.03	0.01\\
22.04	0.01\\
22.05	0.01\\
22.06	0.01\\
22.07	0.01\\
22.08	0.01\\
22.09	0.01\\
22.1	0.01\\
22.11	0.01\\
22.12	0.01\\
22.13	0.01\\
22.14	0.01\\
22.15	0.01\\
22.16	0.01\\
22.17	0.01\\
22.18	0.01\\
22.19	0.01\\
22.2	0.01\\
22.21	0.01\\
22.22	0.01\\
22.23	0.01\\
22.24	0.01\\
22.25	0.01\\
22.26	0.01\\
22.27	0.01\\
22.28	0.01\\
22.29	0.01\\
22.3	0.01\\
22.31	0.01\\
22.32	0.01\\
22.33	0.01\\
22.34	0.01\\
22.35	0.01\\
22.36	0.01\\
22.37	0.01\\
22.38	0.01\\
22.39	0.01\\
22.4	0.01\\
22.41	0.01\\
22.42	0.01\\
22.43	0.01\\
22.44	0.01\\
22.45	0.01\\
22.46	0.01\\
22.47	0.01\\
22.48	0.01\\
22.49	0.01\\
22.5	0.01\\
22.51	0.01\\
22.52	0.01\\
22.53	0.01\\
22.54	0.01\\
22.55	0.01\\
22.56	0.01\\
22.57	0.01\\
22.58	0.01\\
22.59	0.01\\
22.6	0.01\\
22.61	0.01\\
22.62	0.01\\
22.63	0.01\\
22.64	0.01\\
22.65	0.01\\
22.66	0.01\\
22.67	0.01\\
22.68	0.01\\
22.69	0.01\\
22.7	0.01\\
22.71	0.01\\
22.72	0.01\\
22.73	0.01\\
22.74	0.01\\
22.75	0.01\\
22.76	0.01\\
22.77	0.01\\
22.78	0.01\\
22.79	0.01\\
22.8	0.01\\
22.81	0.01\\
22.82	0.01\\
22.83	0.01\\
22.84	0.01\\
22.85	0.01\\
22.86	0.01\\
22.87	0.01\\
22.88	0.01\\
22.89	0.01\\
22.9	0.01\\
22.91	0.01\\
22.92	0.01\\
22.93	0.01\\
22.94	0.01\\
22.95	0.01\\
22.96	0.01\\
22.97	0.01\\
22.98	0.01\\
22.99	0.01\\
23	0.01\\
23.01	0.01\\
23.02	0.01\\
23.03	0.01\\
23.04	0.01\\
23.05	0.01\\
23.06	0.01\\
23.07	0.01\\
23.08	0.01\\
23.09	0.01\\
23.1	0.01\\
23.11	0.01\\
23.12	0.01\\
23.13	0.01\\
23.14	0.01\\
23.15	0.01\\
23.16	0.01\\
23.17	0.01\\
23.18	0.01\\
23.19	0.01\\
23.2	0.01\\
23.21	0.01\\
23.22	0.01\\
23.23	0.01\\
23.24	0.01\\
23.25	0.01\\
23.26	0.01\\
23.27	0.01\\
23.28	0.01\\
23.29	0.01\\
23.3	0.01\\
23.31	0.01\\
23.32	0.01\\
23.33	0.01\\
23.34	0.01\\
23.35	0.01\\
23.36	0.01\\
23.37	0.01\\
23.38	0.01\\
23.39	0.01\\
23.4	0.01\\
23.41	0.01\\
23.42	0.01\\
23.43	0.01\\
23.44	0.01\\
23.45	0.01\\
23.46	0.01\\
23.47	0.01\\
23.48	0.01\\
23.49	0.01\\
23.5	0.01\\
23.51	0.01\\
23.52	0.01\\
23.53	0.01\\
23.54	0.01\\
23.55	0.01\\
23.56	0.01\\
23.57	0.01\\
23.58	0.01\\
23.59	0.01\\
23.6	0.01\\
23.61	0.01\\
23.62	0.01\\
23.63	0.01\\
23.64	0.01\\
23.65	0.01\\
23.66	0.01\\
23.67	0.01\\
23.68	0.01\\
23.69	0.01\\
23.7	0.01\\
23.71	0.01\\
23.72	0.01\\
23.73	0.01\\
23.74	0.01\\
23.75	0.01\\
23.76	0.01\\
23.77	0.01\\
23.78	0.01\\
23.79	0.01\\
23.8	0.01\\
23.81	0.01\\
23.82	0.01\\
23.83	0.01\\
23.84	0.01\\
23.85	0.01\\
23.86	0.01\\
23.87	0.01\\
23.88	0.01\\
23.89	0.01\\
23.9	0.01\\
23.91	0.01\\
23.92	0.01\\
23.93	0.01\\
23.94	0.01\\
23.95	0.01\\
23.96	0.01\\
23.97	0.01\\
23.98	0.01\\
23.99	0.01\\
24	0.01\\
24.01	0.01\\
24.02	0.01\\
24.03	0.01\\
24.04	0.01\\
24.05	0.01\\
24.06	0.01\\
24.07	0.01\\
24.08	0.01\\
24.09	0.01\\
24.1	0.01\\
24.11	0.01\\
24.12	0.01\\
24.13	0.01\\
24.14	0.01\\
24.15	0.01\\
24.16	0.01\\
24.17	0.01\\
24.18	0.01\\
24.19	0.01\\
24.2	0.01\\
24.21	0.01\\
24.22	0.01\\
24.23	0.01\\
24.24	0.01\\
24.25	0.01\\
24.26	0.01\\
24.27	0.01\\
24.28	0.01\\
24.29	0.01\\
24.3	0.01\\
24.31	0.01\\
24.32	0.01\\
24.33	0.01\\
24.34	0.01\\
24.35	0.01\\
24.36	0.01\\
24.37	0.01\\
24.38	0.01\\
24.39	0.01\\
24.4	0.01\\
24.41	0.01\\
24.42	0.01\\
24.43	0.01\\
24.44	0.01\\
24.45	0.01\\
24.46	0.01\\
24.47	0.01\\
24.48	0.01\\
24.49	0.01\\
24.5	0.01\\
24.51	0.01\\
24.52	0.01\\
24.53	0.01\\
24.54	0.01\\
24.55	0.01\\
24.56	0.01\\
24.57	0.01\\
24.58	0.01\\
24.59	0.01\\
24.6	0.01\\
24.61	0.01\\
24.62	0.01\\
24.63	0.01\\
24.64	0.01\\
24.65	0.01\\
24.66	0.01\\
24.67	0.01\\
24.68	0.01\\
24.69	0.01\\
24.7	0.01\\
24.71	0.01\\
24.72	0.01\\
24.73	0.01\\
24.74	0.01\\
24.75	0.01\\
24.76	0.01\\
24.77	0.01\\
24.78	0.01\\
24.79	0.01\\
24.8	0.01\\
24.81	0.01\\
24.82	0.01\\
24.83	0.01\\
24.84	0.01\\
24.85	0.01\\
24.86	0.01\\
24.87	0.01\\
24.88	0.01\\
24.89	0.01\\
24.9	0.01\\
24.91	0.01\\
24.92	0.01\\
24.93	0.01\\
24.94	0.01\\
24.95	0.01\\
24.96	0.01\\
24.97	0.01\\
24.98	0.01\\
24.99	0.01\\
25	0.01\\
25.01	0.01\\
25.02	0.01\\
25.03	0.01\\
25.04	0.01\\
25.05	0.01\\
25.06	0.01\\
25.07	0.01\\
25.08	0.01\\
25.09	0.01\\
25.1	0.01\\
25.11	0.01\\
25.12	0.01\\
25.13	0.01\\
25.14	0.01\\
25.15	0.01\\
25.16	0.01\\
25.17	0.01\\
25.18	0.01\\
25.19	0.01\\
25.2	0.01\\
25.21	0.01\\
25.22	0.01\\
25.23	0.01\\
25.24	0.01\\
25.25	0.01\\
25.26	0.01\\
25.27	0.01\\
25.28	0.01\\
25.29	0.01\\
25.3	0.01\\
25.31	0.01\\
25.32	0.01\\
25.33	0.01\\
25.34	0.01\\
25.35	0.01\\
25.36	0.01\\
25.37	0.01\\
25.38	0.01\\
25.39	0.01\\
25.4	0.01\\
25.41	0.01\\
25.42	0.01\\
25.43	0.01\\
25.44	0.01\\
25.45	0.01\\
25.46	0.01\\
25.47	0.01\\
25.48	0.01\\
25.49	0.01\\
25.5	0.01\\
25.51	0.01\\
25.52	0.01\\
25.53	0.01\\
25.54	0.01\\
25.55	0.01\\
25.56	0.01\\
25.57	0.01\\
25.58	0.01\\
25.59	0.01\\
25.6	0.01\\
25.61	0.01\\
25.62	0.01\\
25.63	0.01\\
25.64	0.01\\
25.65	0.01\\
25.66	0.01\\
25.67	0.01\\
25.68	0.01\\
25.69	0.01\\
25.7	0.01\\
25.71	0.01\\
25.72	0.01\\
25.73	0.01\\
25.74	0.01\\
25.75	0.01\\
25.76	0.01\\
25.77	0.01\\
25.78	0.01\\
25.79	0.01\\
25.8	0.01\\
25.81	0.01\\
25.82	0.01\\
25.83	0.01\\
25.84	0.01\\
25.85	0.01\\
25.86	0.01\\
25.87	0.01\\
25.88	0.01\\
25.89	0.01\\
25.9	0.01\\
25.91	0.01\\
25.92	0.01\\
25.93	0.01\\
25.94	0.01\\
25.95	0.01\\
25.96	0.01\\
25.97	0.01\\
25.98	0.01\\
25.99	0.01\\
26	0.01\\
26.01	0.01\\
26.02	0.01\\
26.03	0.01\\
26.04	0.01\\
26.05	0.01\\
26.06	0.01\\
26.07	0.01\\
26.08	0.01\\
26.09	0.01\\
26.1	0.01\\
26.11	0.01\\
26.12	0.01\\
26.13	0.01\\
26.14	0.01\\
26.15	0.01\\
26.16	0.01\\
26.17	0.01\\
26.18	0.01\\
26.19	0.01\\
26.2	0.01\\
26.21	0.01\\
26.22	0.01\\
26.23	0.01\\
26.24	0.01\\
26.25	0.01\\
26.26	0.01\\
26.27	0.01\\
26.28	0.01\\
26.29	0.01\\
26.3	0.01\\
26.31	0.01\\
26.32	0.01\\
26.33	0.01\\
26.34	0.01\\
26.35	0.01\\
26.36	0.01\\
26.37	0.01\\
26.38	0.01\\
26.39	0.01\\
26.4	0.01\\
26.41	0.01\\
26.42	0.01\\
26.43	0.01\\
26.44	0.01\\
26.45	0.01\\
26.46	0.01\\
26.47	0.01\\
26.48	0.01\\
26.49	0.01\\
26.5	0.01\\
26.51	0.01\\
26.52	0.01\\
26.53	0.01\\
26.54	0.01\\
26.55	0.01\\
26.56	0.01\\
26.57	0.01\\
26.58	0.01\\
26.59	0.01\\
26.6	0.01\\
26.61	0.01\\
26.62	0.01\\
26.63	0.01\\
26.64	0.01\\
26.65	0.01\\
26.66	0.01\\
26.67	0.01\\
26.68	0.01\\
26.69	0.01\\
26.7	0.01\\
26.71	0.01\\
26.72	0.01\\
26.73	0.01\\
26.74	0.01\\
26.75	0.01\\
26.76	0.01\\
26.77	0.01\\
26.78	0.01\\
26.79	0.01\\
26.8	0.01\\
26.81	0.01\\
26.82	0.01\\
26.83	0.01\\
26.84	0.01\\
26.85	0.01\\
26.86	0.01\\
26.87	0.01\\
26.88	0.01\\
26.89	0.01\\
26.9	0.01\\
26.91	0.01\\
26.92	0.01\\
26.93	0.01\\
26.94	0.01\\
26.95	0.01\\
26.96	0.01\\
26.97	0.01\\
26.98	0.01\\
26.99	0.01\\
27	0.01\\
27.01	0.01\\
27.02	0.01\\
27.03	0.01\\
27.04	0.01\\
27.05	0.01\\
27.06	0.01\\
27.07	0.01\\
27.08	0.01\\
27.09	0.01\\
27.1	0.01\\
27.11	0.01\\
27.12	0.01\\
27.13	0.01\\
27.14	0.01\\
27.15	0.01\\
27.16	0.01\\
27.17	0.01\\
27.18	0.01\\
27.19	0.01\\
27.2	0.01\\
27.21	0.01\\
27.22	0.01\\
27.23	0.01\\
27.24	0.01\\
27.25	0.01\\
27.26	0.01\\
27.27	0.01\\
27.28	0.01\\
27.29	0.01\\
27.3	0.01\\
27.31	0.01\\
27.32	0.01\\
27.33	0.01\\
27.34	0.01\\
27.35	0.01\\
27.36	0.01\\
27.37	0.01\\
27.38	0.01\\
27.39	0.01\\
27.4	0.01\\
27.41	0.01\\
27.42	0.01\\
27.43	0.01\\
27.44	0.01\\
27.45	0.01\\
27.46	0.01\\
27.47	0.01\\
27.48	0.01\\
27.49	0.01\\
27.5	0.01\\
27.51	0.01\\
27.52	0.01\\
27.53	0.01\\
27.54	0.01\\
27.55	0.01\\
27.56	0.01\\
27.57	0.01\\
27.58	0.01\\
27.59	0.01\\
27.6	0.01\\
27.61	0.01\\
27.62	0.01\\
27.63	0.01\\
27.64	0.01\\
27.65	0.01\\
27.66	0.01\\
27.67	0.01\\
27.68	0.01\\
27.69	0.01\\
27.7	0.01\\
27.71	0.01\\
27.72	0.01\\
27.73	0.01\\
27.74	0.01\\
27.75	0.01\\
27.76	0.01\\
27.77	0.01\\
27.78	0.01\\
27.79	0.01\\
27.8	0.01\\
27.81	0.01\\
27.82	0.01\\
27.83	0.01\\
27.84	0.01\\
27.85	0.01\\
27.86	0.01\\
27.87	0.01\\
27.88	0.01\\
27.89	0.01\\
27.9	0.01\\
27.91	0.01\\
27.92	0.01\\
27.93	0.01\\
27.94	0.01\\
27.95	0.01\\
27.96	0.01\\
27.97	0.01\\
27.98	0.01\\
27.99	0.01\\
28	0.01\\
28.01	0.01\\
28.02	0.01\\
28.03	0.01\\
28.04	0.01\\
28.05	0.01\\
28.06	0.01\\
28.07	0.01\\
28.08	0.01\\
28.09	0.01\\
28.1	0.01\\
28.11	0.01\\
28.12	0.01\\
28.13	0.01\\
28.14	0.01\\
28.15	0.01\\
28.16	0.01\\
28.17	0.01\\
28.18	0.01\\
28.19	0.01\\
28.2	0.01\\
28.21	0.01\\
28.22	0.01\\
28.23	0.01\\
28.24	0.01\\
28.25	0.01\\
28.26	0.01\\
28.27	0.01\\
28.28	0.01\\
28.29	0.01\\
28.3	0.01\\
28.31	0.01\\
28.32	0.01\\
28.33	0.01\\
28.34	0.01\\
28.35	0.01\\
28.36	0.01\\
28.37	0.01\\
28.38	0.01\\
28.39	0.01\\
28.4	0.01\\
28.41	0.01\\
28.42	0.01\\
28.43	0.01\\
28.44	0.01\\
28.45	0.01\\
28.46	0.01\\
28.47	0.01\\
28.48	0.01\\
28.49	0.01\\
28.5	0.01\\
28.51	0.01\\
28.52	0.01\\
28.53	0.01\\
28.54	0.01\\
28.55	0.01\\
28.56	0.01\\
28.57	0.01\\
28.58	0.01\\
28.59	0.01\\
28.6	0.01\\
28.61	0.01\\
28.62	0.01\\
28.63	0.01\\
28.64	0.01\\
28.65	0.01\\
28.66	0.01\\
28.67	0.01\\
28.68	0.01\\
28.69	0.01\\
28.7	0.01\\
28.71	0.01\\
28.72	0.01\\
28.73	0.01\\
28.74	0.01\\
28.75	0.01\\
28.76	0.01\\
28.77	0.01\\
28.78	0.01\\
28.79	0.01\\
28.8	0.01\\
28.81	0.01\\
28.82	0.01\\
28.83	0.01\\
28.84	0.01\\
28.85	0.01\\
28.86	0.01\\
28.87	0.01\\
28.88	0.01\\
28.89	0.01\\
28.9	0.01\\
28.91	0.01\\
28.92	0.01\\
28.93	0.01\\
28.94	0.01\\
28.95	0.01\\
28.96	0.01\\
28.97	0.01\\
28.98	0.01\\
28.99	0.01\\
29	0.01\\
29.01	0.01\\
29.02	0.01\\
29.03	0.01\\
29.04	0.01\\
29.05	0.01\\
29.06	0.01\\
29.07	0.01\\
29.08	0.01\\
29.09	0.01\\
29.1	0.01\\
29.11	0.01\\
29.12	0.01\\
29.13	0.01\\
29.14	0.01\\
29.15	0.01\\
29.16	0.01\\
29.17	0.01\\
29.18	0.01\\
29.19	0.01\\
29.2	0.01\\
29.21	0.01\\
29.22	0.01\\
29.23	0.01\\
29.24	0.01\\
29.25	0.01\\
29.26	0.01\\
29.27	0.01\\
29.28	0.01\\
29.29	0.01\\
29.3	0.01\\
29.31	0.01\\
29.32	0.01\\
29.33	0.01\\
29.34	0.01\\
29.35	0.01\\
29.36	0.01\\
29.37	0.01\\
29.38	0.01\\
29.39	0.01\\
29.4	0.01\\
29.41	0.01\\
29.42	0.01\\
29.43	0.01\\
29.44	0.01\\
29.45	0.01\\
29.46	0.01\\
29.47	0.01\\
29.48	0.01\\
29.49	0.01\\
29.5	0.01\\
29.51	0.01\\
29.52	0.01\\
29.53	0.01\\
29.54	0.01\\
29.55	0.01\\
29.56	0.01\\
29.57	0.01\\
29.58	0.01\\
29.59	0.01\\
29.6	0.01\\
29.61	0.01\\
29.62	0.01\\
29.63	0.01\\
29.64	0.01\\
29.65	0.01\\
29.66	0.01\\
29.67	0.01\\
29.68	0.01\\
29.69	0.01\\
29.7	0.01\\
29.71	0.01\\
29.72	0.01\\
29.73	0.01\\
29.74	0.01\\
29.75	0.01\\
29.76	0.01\\
29.77	0.01\\
29.78	0.01\\
29.79	0.01\\
29.8	0.01\\
29.81	0.01\\
29.82	0.01\\
29.83	0.01\\
29.84	0.01\\
29.85	0.01\\
29.86	0.01\\
29.87	0.01\\
29.88	0.01\\
29.89	0.01\\
29.9	0.01\\
29.91	0.01\\
29.92	0.01\\
29.93	0.01\\
29.94	0.01\\
29.95	0.01\\
29.96	0.01\\
29.97	0.01\\
29.98	0.01\\
29.99	0.01\\
30	0.01\\
30.01	0.01\\
30.02	0.01\\
30.03	0.01\\
30.04	0.01\\
30.05	0.01\\
30.06	0.01\\
30.07	0.01\\
30.08	0.01\\
30.09	0.01\\
30.1	0.01\\
30.11	0.01\\
30.12	0.01\\
30.13	0.01\\
30.14	0.01\\
30.15	0.01\\
30.16	0.01\\
30.17	0.01\\
30.18	0.01\\
30.19	0.01\\
30.2	0.01\\
30.21	0.01\\
30.22	0.01\\
30.23	0.01\\
30.24	0.01\\
30.25	0.01\\
30.26	0.01\\
30.27	0.01\\
30.28	0.01\\
30.29	0.01\\
30.3	0.01\\
30.31	0.01\\
30.32	0.01\\
30.33	0.01\\
30.34	0.01\\
30.35	0.01\\
30.36	0.01\\
30.37	0.01\\
30.38	0.01\\
30.39	0.01\\
30.4	0.01\\
30.41	0.01\\
30.42	0.01\\
30.43	0.01\\
30.44	0.01\\
30.45	0.01\\
30.46	0.01\\
30.47	0.01\\
30.48	0.01\\
30.49	0.01\\
30.5	0.01\\
30.51	0.01\\
30.52	0.01\\
30.53	0.01\\
30.54	0.01\\
30.55	0.01\\
30.56	0.01\\
30.57	0.01\\
30.58	0.01\\
30.59	0.01\\
30.6	0.01\\
30.61	0.01\\
30.62	0.01\\
30.63	0.01\\
30.64	0.01\\
30.65	0.01\\
30.66	0.01\\
30.67	0.01\\
30.68	0.01\\
30.69	0.01\\
30.7	0.01\\
30.71	0.01\\
30.72	0.01\\
30.73	0.01\\
30.74	0.01\\
30.75	0.01\\
30.76	0.01\\
30.77	0.01\\
30.78	0.01\\
30.79	0.01\\
30.8	0.01\\
30.81	0.01\\
30.82	0.01\\
30.83	0.01\\
30.84	0.01\\
30.85	0.01\\
30.86	0.01\\
30.87	0.01\\
30.88	0.01\\
30.89	0.01\\
30.9	0.01\\
30.91	0.01\\
30.92	0.01\\
30.93	0.01\\
30.94	0.01\\
30.95	0.01\\
30.96	0.01\\
30.97	0.01\\
30.98	0.01\\
30.99	0.01\\
31	0.01\\
31.01	0.01\\
31.02	0.01\\
31.03	0.01\\
31.04	0.01\\
31.05	0.01\\
31.06	0.01\\
31.07	0.01\\
31.08	0.01\\
31.09	0.01\\
31.1	0.01\\
31.11	0.01\\
31.12	0.01\\
31.13	0.01\\
31.14	0.01\\
31.15	0.01\\
31.16	0.01\\
31.17	0.01\\
31.18	0.01\\
31.19	0.01\\
31.2	0.01\\
31.21	0.01\\
31.22	0.01\\
31.23	0.01\\
31.24	0.01\\
31.25	0.01\\
31.26	0.01\\
31.27	0.01\\
31.28	0.01\\
31.29	0.01\\
31.3	0.01\\
31.31	0.01\\
31.32	0.01\\
31.33	0.01\\
31.34	0.01\\
31.35	0.01\\
31.36	0.01\\
31.37	0.01\\
31.38	0.01\\
31.39	0.01\\
31.4	0.01\\
31.41	0.01\\
31.42	0.01\\
31.43	0.01\\
31.44	0.01\\
31.45	0.01\\
31.46	0.01\\
31.47	0.01\\
31.48	0.01\\
31.49	0.01\\
31.5	0.01\\
31.51	0.01\\
31.52	0.01\\
31.53	0.01\\
31.54	0.01\\
31.55	0.01\\
31.56	0.01\\
31.57	0.01\\
31.58	0.01\\
31.59	0.01\\
31.6	0.01\\
31.61	0.01\\
31.62	0.01\\
31.63	0.01\\
31.64	0.01\\
31.65	0.01\\
31.66	0.01\\
31.67	0.01\\
31.68	0.01\\
31.69	0.01\\
31.7	0.01\\
31.71	0.01\\
31.72	0.01\\
31.73	0.01\\
31.74	0.01\\
31.75	0.01\\
31.76	0.01\\
31.77	0.01\\
31.78	0.01\\
31.79	0.01\\
31.8	0.01\\
31.81	0.01\\
31.82	0.01\\
31.83	0.01\\
31.84	0.01\\
31.85	0.01\\
31.86	0.01\\
31.87	0.01\\
31.88	0.01\\
31.89	0.01\\
31.9	0.01\\
31.91	0.01\\
31.92	0.01\\
31.93	0.01\\
31.94	0.01\\
31.95	0.01\\
31.96	0.01\\
31.97	0.01\\
31.98	0.01\\
31.99	0.01\\
32	0.01\\
32.01	0.01\\
32.02	0.01\\
32.03	0.01\\
32.04	0.01\\
32.05	0.01\\
32.06	0.01\\
32.07	0.01\\
32.08	0.01\\
32.09	0.01\\
32.1	0.01\\
32.11	0.01\\
32.12	0.01\\
32.13	0.01\\
32.14	0.01\\
32.15	0.01\\
32.16	0.01\\
32.17	0.01\\
32.18	0.01\\
32.19	0.01\\
32.2	0.01\\
32.21	0.01\\
32.22	0.01\\
32.23	0.01\\
32.24	0.01\\
32.25	0.01\\
32.26	0.01\\
32.27	0.01\\
32.28	0.01\\
32.29	0.01\\
32.3	0.01\\
32.31	0.01\\
32.32	0.01\\
32.33	0.01\\
32.34	0.01\\
32.35	0.01\\
32.36	0.01\\
32.37	0.01\\
32.38	0.01\\
32.39	0.01\\
32.4	0.01\\
32.41	0.01\\
32.42	0.01\\
32.43	0.01\\
32.44	0.01\\
32.45	0.01\\
32.46	0.01\\
32.47	0.01\\
32.48	0.01\\
32.49	0.01\\
32.5	0.01\\
32.51	0.01\\
32.52	0.01\\
32.53	0.01\\
32.54	0.01\\
32.55	0.01\\
32.56	0.01\\
32.57	0.01\\
32.58	0.01\\
32.59	0.01\\
32.6	0.01\\
32.61	0.01\\
32.62	0.01\\
32.63	0.01\\
32.64	0.01\\
32.65	0.01\\
32.66	0.01\\
32.67	0.01\\
32.68	0.01\\
32.69	0.01\\
32.7	0.01\\
32.71	0.01\\
32.72	0.01\\
32.73	0.01\\
32.74	0.01\\
32.75	0.01\\
32.76	0.01\\
32.77	0.01\\
32.78	0.01\\
32.79	0.01\\
32.8	0.01\\
32.81	0.01\\
32.82	0.01\\
32.83	0.01\\
32.84	0.01\\
32.85	0.01\\
32.86	0.01\\
32.87	0.01\\
32.88	0.01\\
32.89	0.01\\
32.9	0.01\\
32.91	0.01\\
32.92	0.01\\
32.93	0.01\\
32.94	0.01\\
32.95	0.01\\
32.96	0.01\\
32.97	0.01\\
32.98	0.01\\
32.99	0.01\\
33	0.01\\
33.01	0.01\\
33.02	0.01\\
33.03	0.01\\
33.04	0.01\\
33.05	0.01\\
33.06	0.01\\
33.07	0.01\\
33.08	0.01\\
33.09	0.01\\
33.1	0.01\\
33.11	0.01\\
33.12	0.01\\
33.13	0.01\\
33.14	0.01\\
33.15	0.01\\
33.16	0.01\\
33.17	0.01\\
33.18	0.01\\
33.19	0.01\\
33.2	0.01\\
33.21	0.01\\
33.22	0.01\\
33.23	0.01\\
33.24	0.01\\
33.25	0.01\\
33.26	0.01\\
33.27	0.01\\
33.28	0.01\\
33.29	0.01\\
33.3	0.01\\
33.31	0.01\\
33.32	0.01\\
33.33	0.01\\
33.34	0.01\\
33.35	0.01\\
33.36	0.01\\
33.37	0.01\\
33.38	0.01\\
33.39	0.01\\
33.4	0.01\\
33.41	0.01\\
33.42	0.01\\
33.43	0.01\\
33.44	0.01\\
33.45	0.01\\
33.46	0.01\\
33.47	0.01\\
33.48	0.01\\
33.49	0.01\\
33.5	0.01\\
33.51	0.01\\
33.52	0.01\\
33.53	0.01\\
33.54	0.01\\
33.55	0.01\\
33.56	0.01\\
33.57	0.01\\
33.58	0.01\\
33.59	0.01\\
33.6	0.01\\
33.61	0.01\\
33.62	0.01\\
33.63	0.01\\
33.64	0.01\\
33.65	0.01\\
33.66	0.01\\
33.67	0.01\\
33.68	0.01\\
33.69	0.01\\
33.7	0.01\\
33.71	0.01\\
33.72	0.01\\
33.73	0.01\\
33.74	0.01\\
33.75	0.01\\
33.76	0.01\\
33.77	0.01\\
33.78	0.01\\
33.79	0.01\\
33.8	0.01\\
33.81	0.01\\
33.82	0.01\\
33.83	0.01\\
33.84	0.01\\
33.85	0.01\\
33.86	0.01\\
33.87	0.01\\
33.88	0.01\\
33.89	0.01\\
33.9	0.01\\
33.91	0.01\\
33.92	0.01\\
33.93	0.01\\
33.94	0.01\\
33.95	0.01\\
33.96	0.01\\
33.97	0.01\\
33.98	0.01\\
33.99	0.01\\
34	0.01\\
34.01	0.01\\
34.02	0.01\\
34.03	0.01\\
34.04	0.01\\
34.05	0.01\\
34.06	0.01\\
34.07	0.01\\
34.08	0.01\\
34.09	0.01\\
34.1	0.01\\
34.11	0.01\\
34.12	0.01\\
34.13	0.01\\
34.14	0.01\\
34.15	0.01\\
34.16	0.01\\
34.17	0.01\\
34.18	0.01\\
34.19	0.01\\
34.2	0.01\\
34.21	0.01\\
34.22	0.01\\
34.23	0.01\\
34.24	0.01\\
34.25	0.01\\
34.26	0.01\\
34.27	0.01\\
34.28	0.01\\
34.29	0.01\\
34.3	0.01\\
34.31	0.01\\
34.32	0.01\\
34.33	0.01\\
34.34	0.01\\
34.35	0.01\\
34.36	0.01\\
34.37	0.01\\
34.38	0.01\\
34.39	0.01\\
34.4	0.01\\
34.41	0.01\\
34.42	0.01\\
34.43	0.01\\
34.44	0.01\\
34.45	0.01\\
34.46	0.01\\
34.47	0.01\\
34.48	0.01\\
34.49	0.01\\
34.5	0.01\\
34.51	0.01\\
34.52	0.01\\
34.53	0.01\\
34.54	0.01\\
34.55	0.01\\
34.56	0.01\\
34.57	0.01\\
34.58	0.01\\
34.59	0.01\\
34.6	0.01\\
34.61	0.01\\
34.62	0.01\\
34.63	0.01\\
34.64	0.01\\
34.65	0.01\\
34.66	0.01\\
34.67	0.01\\
34.68	0.01\\
34.69	0.01\\
34.7	0.01\\
34.71	0.01\\
34.72	0.01\\
34.73	0.01\\
34.74	0.01\\
34.75	0.01\\
34.76	0.01\\
34.77	0.01\\
34.78	0.01\\
34.79	0.01\\
34.8	0.01\\
34.81	0.01\\
34.82	0.01\\
34.83	0.01\\
34.84	0.01\\
34.85	0.01\\
34.86	0.01\\
34.87	0.01\\
34.88	0.01\\
34.89	0.01\\
34.9	0.01\\
34.91	0.01\\
34.92	0.01\\
34.93	0.01\\
34.94	0.01\\
34.95	0.01\\
34.96	0.01\\
34.97	0.01\\
34.98	0.01\\
34.99	0.01\\
35	0.01\\
35.01	0.01\\
35.02	0.01\\
35.03	0.01\\
35.04	0.01\\
35.05	0.01\\
35.06	0.01\\
35.07	0.01\\
35.08	0.01\\
35.09	0.01\\
35.1	0.01\\
35.11	0.01\\
35.12	0.01\\
35.13	0.01\\
35.14	0.01\\
35.15	0.01\\
35.16	0.01\\
35.17	0.01\\
35.18	0.01\\
35.19	0.01\\
35.2	0.01\\
35.21	0.01\\
35.22	0.01\\
35.23	0.01\\
35.24	0.01\\
35.25	0.01\\
35.26	0.01\\
35.27	0.01\\
35.28	0.01\\
35.29	0.01\\
35.3	0.01\\
35.31	0.01\\
35.32	0.01\\
35.33	0.01\\
35.34	0.01\\
35.35	0.01\\
35.36	0.01\\
35.37	0.01\\
35.38	0.01\\
35.39	0.01\\
35.4	0.01\\
35.41	0.01\\
35.42	0.01\\
35.43	0.01\\
35.44	0.01\\
35.45	0.01\\
35.46	0.01\\
35.47	0.01\\
35.48	0.01\\
35.49	0.01\\
35.5	0.01\\
35.51	0.01\\
35.52	0.01\\
35.53	0.01\\
35.54	0.01\\
35.55	0.01\\
35.56	0.01\\
35.57	0.01\\
35.58	0.01\\
35.59	0.01\\
35.6	0.01\\
35.61	0.01\\
35.62	0.01\\
35.63	0.01\\
35.64	0.01\\
35.65	0.01\\
35.66	0.01\\
35.67	0.01\\
35.68	0.01\\
35.69	0.01\\
35.7	0.01\\
35.71	0.01\\
35.72	0.01\\
35.73	0.01\\
35.74	0.01\\
35.75	0.01\\
35.76	0.01\\
35.77	0.01\\
35.78	0.01\\
35.79	0.01\\
35.8	0.01\\
35.81	0.01\\
35.82	0.01\\
35.83	0.01\\
35.84	0.01\\
35.85	0.01\\
35.86	0.01\\
35.87	0.01\\
35.88	0.01\\
35.89	0.01\\
35.9	0.01\\
35.91	0.01\\
35.92	0.01\\
35.93	0.01\\
35.94	0.01\\
35.95	0.01\\
35.96	0.01\\
35.97	0.01\\
35.98	0.01\\
35.99	0.01\\
36	0.01\\
36.01	0.01\\
36.02	0.01\\
36.03	0.01\\
36.04	0.01\\
36.05	0.01\\
36.06	0.01\\
36.07	0.01\\
36.08	0.01\\
36.09	0.01\\
36.1	0.01\\
36.11	0.01\\
36.12	0.01\\
36.13	0.01\\
36.14	0.01\\
36.15	0.01\\
36.16	0.01\\
36.17	0.01\\
36.18	0.01\\
36.19	0.01\\
36.2	0.01\\
36.21	0.01\\
36.22	0.01\\
36.23	0.01\\
36.24	0.01\\
36.25	0.01\\
36.26	0.01\\
36.27	0.01\\
36.28	0.01\\
36.29	0.01\\
36.3	0.01\\
36.31	0.01\\
36.32	0.01\\
36.33	0.01\\
36.34	0.01\\
36.35	0.01\\
36.36	0.01\\
36.37	0.01\\
36.38	0.01\\
36.39	0.01\\
36.4	0.01\\
36.41	0.01\\
36.42	0.01\\
36.43	0.01\\
36.44	0.01\\
36.45	0.01\\
36.46	0.01\\
36.47	0.01\\
36.48	0.01\\
36.49	0.01\\
36.5	0.01\\
36.51	0.01\\
36.52	0.01\\
36.53	0.01\\
36.54	0.01\\
36.55	0.01\\
36.56	0.01\\
36.57	0.01\\
36.58	0.01\\
36.59	0.01\\
36.6	0.01\\
36.61	0.01\\
36.62	0.01\\
36.63	0.01\\
36.64	0.01\\
36.65	0.01\\
36.66	0.01\\
36.67	0.01\\
36.68	0.01\\
36.69	0.01\\
36.7	0.01\\
36.71	0.01\\
36.72	0.01\\
36.73	0.01\\
36.74	0.01\\
36.75	0.01\\
36.76	0.01\\
36.77	0.01\\
36.78	0.01\\
36.79	0.01\\
36.8	0.01\\
36.81	0.01\\
36.82	0.01\\
36.83	0.01\\
36.84	0.01\\
36.85	0.01\\
36.86	0.01\\
36.87	0.01\\
36.88	0.01\\
36.89	0.01\\
36.9	0.01\\
36.91	0.01\\
36.92	0.01\\
36.93	0.01\\
36.94	0.01\\
36.95	0.01\\
36.96	0.01\\
36.97	0.01\\
36.98	0.01\\
36.99	0.01\\
37	0.01\\
37.01	0.01\\
37.02	0.01\\
37.03	0.01\\
37.04	0.01\\
37.05	0.01\\
37.06	0.01\\
37.07	0.01\\
37.08	0.01\\
37.09	0.01\\
37.1	0.01\\
37.11	0.01\\
37.12	0.01\\
37.13	0.01\\
37.14	0.01\\
37.15	0.01\\
37.16	0.01\\
37.17	0.01\\
37.18	0.01\\
37.19	0.01\\
37.2	0.01\\
37.21	0.01\\
37.22	0.01\\
37.23	0.01\\
37.24	0.01\\
37.25	0.01\\
37.26	0.01\\
37.27	0.01\\
37.28	0.01\\
37.29	0.01\\
37.3	0.01\\
37.31	0.01\\
37.32	0.01\\
37.33	0.01\\
37.34	0.01\\
37.35	0.01\\
37.36	0.01\\
37.37	0.01\\
37.38	0.01\\
37.39	0.01\\
37.4	0.01\\
37.41	0.01\\
37.42	0.01\\
37.43	0.01\\
37.44	0.01\\
37.45	0.01\\
37.46	0.01\\
37.47	0.01\\
37.48	0.01\\
37.49	0.01\\
37.5	0.01\\
37.51	0.01\\
37.52	0.01\\
37.53	0.01\\
37.54	0.01\\
37.55	0.01\\
37.56	0.01\\
37.57	0.01\\
37.58	0.01\\
37.59	0.01\\
37.6	0.01\\
37.61	0.01\\
37.62	0.01\\
37.63	0.01\\
37.64	0.01\\
37.65	0.01\\
37.66	0.01\\
37.67	0.01\\
37.68	0.01\\
37.69	0.01\\
37.7	0.01\\
37.71	0.01\\
37.72	0.01\\
37.73	0.01\\
37.74	0.01\\
37.75	0.01\\
37.76	0.01\\
37.77	0.01\\
37.78	0.01\\
37.79	0.01\\
37.8	0.01\\
37.81	0.01\\
37.82	0.01\\
37.83	0.01\\
37.84	0.01\\
37.85	0.01\\
37.86	0.01\\
37.87	0.01\\
37.88	0.01\\
37.89	0.01\\
37.9	0.01\\
37.91	0.01\\
37.92	0.01\\
37.93	0.01\\
37.94	0.01\\
37.95	0.01\\
37.96	0.01\\
37.97	0.01\\
37.98	0.01\\
37.99	0.01\\
38	0.01\\
38.01	0.01\\
38.02	0.01\\
38.03	0.01\\
38.04	0.01\\
38.05	0.01\\
38.06	0.01\\
38.07	0.01\\
38.08	0.01\\
38.09	0.01\\
38.1	0.01\\
38.11	0.01\\
38.12	0.01\\
38.13	0.01\\
38.14	0.01\\
38.15	0.01\\
38.16	0.01\\
38.17	0.01\\
38.18	0.01\\
38.19	0.01\\
38.2	0.01\\
38.21	0.01\\
38.22	0.01\\
38.23	0.01\\
38.24	0.01\\
38.25	0.01\\
38.26	0.01\\
38.27	0.01\\
38.28	0.01\\
38.29	0.01\\
38.3	0.01\\
38.31	0.01\\
38.32	0.01\\
38.33	0.01\\
38.34	0.01\\
38.35	0.01\\
38.36	0.01\\
38.37	0.01\\
38.38	0.01\\
38.39	0.01\\
38.4	0.01\\
38.41	0.01\\
38.42	0.01\\
38.43	0.01\\
38.44	0.01\\
38.45	0.01\\
38.46	0.01\\
38.47	0.01\\
38.48	0.01\\
38.49	0.01\\
38.5	0.01\\
38.51	0.01\\
38.52	0.01\\
38.53	0.01\\
38.54	0.01\\
38.55	0.01\\
38.56	0.01\\
38.57	0.01\\
38.58	0.01\\
38.59	0.01\\
38.6	0.01\\
38.61	0.01\\
38.62	0.01\\
38.63	0.01\\
38.64	0.01\\
38.65	0.01\\
38.66	0.01\\
38.67	0.01\\
38.68	0.01\\
38.69	0.01\\
38.7	0.01\\
38.71	0.01\\
38.72	0.01\\
38.73	0.01\\
38.74	0.01\\
38.75	0.01\\
38.76	0.01\\
38.77	0.01\\
38.78	0.01\\
38.79	0.01\\
38.8	0.01\\
38.81	0.01\\
38.82	0.01\\
38.83	0.01\\
38.84	0.01\\
38.85	0.01\\
38.86	0.01\\
38.87	0.01\\
38.88	0.01\\
38.89	0.01\\
38.9	0.01\\
38.91	0.01\\
38.92	0.01\\
38.93	0.01\\
38.94	0.01\\
38.95	0.01\\
38.96	0.01\\
38.97	0.01\\
38.98	0.01\\
38.99	0.01\\
39	0.01\\
39.01	0.01\\
39.02	0.01\\
39.03	0.01\\
39.04	0.01\\
39.05	0.01\\
39.06	0.01\\
39.07	0.01\\
39.08	0.01\\
39.09	0.01\\
39.1	0.01\\
39.11	0.01\\
39.12	0.01\\
39.13	0.01\\
39.14	0.01\\
39.15	0.01\\
39.16	0.01\\
39.17	0.01\\
39.18	0.01\\
39.19	0.01\\
39.2	0.01\\
39.21	0.01\\
39.22	0.01\\
39.23	0.01\\
39.24	0.01\\
39.25	0.01\\
39.26	0.01\\
39.27	0.01\\
39.28	0.01\\
39.29	0.01\\
39.3	0.01\\
39.31	0.01\\
39.32	0.01\\
39.33	0.01\\
39.34	0.01\\
39.35	0.01\\
39.36	0.01\\
39.37	0.01\\
39.38	0.01\\
39.39	0.01\\
39.4	0.01\\
39.41	0.01\\
39.42	0.01\\
39.43	0.01\\
39.44	0.01\\
39.45	0.01\\
39.46	0.01\\
39.47	0.01\\
39.48	0.01\\
39.49	0.01\\
39.5	0.01\\
39.51	0.01\\
39.52	0.01\\
39.53	0.01\\
39.54	0.01\\
39.55	0.01\\
39.56	0.01\\
39.57	0.01\\
39.58	0.01\\
39.59	0.01\\
39.6	0.01\\
39.61	0.01\\
39.62	0.01\\
39.63	0.01\\
39.64	0.01\\
39.65	0.01\\
39.66	0.01\\
39.67	0.01\\
39.68	0.01\\
39.69	0.01\\
39.7	0.01\\
39.71	0.01\\
39.72	0.01\\
39.73	0.01\\
39.74	0.01\\
39.75	0.01\\
39.76	0.01\\
39.77	0.01\\
39.78	0.01\\
39.79	0.01\\
39.8	0.01\\
39.81	0.01\\
39.82	0.01\\
39.83	0.01\\
39.84	0.01\\
39.85	0.01\\
39.86	0.01\\
39.87	0.01\\
39.88	0.01\\
39.89	0.01\\
39.9	0.01\\
39.91	0.01\\
39.92	0.01\\
39.93	0.01\\
39.94	0.01\\
39.95	0.01\\
39.96	0.01\\
39.97	0.01\\
39.98	0.01\\
39.99	0.01\\
40	0.01\\
40.01	0.01\\
};
\addplot [color=green,dashed,forget plot]
  table[row sep=crcr]{%
40.01	0.01\\
40.02	0.01\\
40.03	0.01\\
40.04	0.01\\
40.05	0.01\\
40.06	0.01\\
40.07	0.01\\
40.08	0.01\\
40.09	0.01\\
40.1	0.01\\
40.11	0.01\\
40.12	0.01\\
40.13	0.01\\
40.14	0.01\\
40.15	0.01\\
40.16	0.01\\
40.17	0.01\\
40.18	0.01\\
40.19	0.01\\
40.2	0.01\\
40.21	0.01\\
40.22	0.01\\
40.23	0.01\\
40.24	0.01\\
40.25	0.01\\
40.26	0.01\\
40.27	0.01\\
40.28	0.01\\
40.29	0.01\\
40.3	0.01\\
40.31	0.01\\
40.32	0.01\\
40.33	0.01\\
40.34	0.01\\
40.35	0.01\\
40.36	0.01\\
40.37	0.01\\
40.38	0.01\\
40.39	0.01\\
40.4	0.01\\
40.41	0.01\\
40.42	0.01\\
40.43	0.01\\
40.44	0.01\\
40.45	0.01\\
40.46	0.01\\
40.47	0.01\\
40.48	0.01\\
40.49	0.01\\
40.5	0.01\\
40.51	0.01\\
40.52	0.01\\
40.53	0.01\\
40.54	0.01\\
40.55	0.01\\
40.56	0.01\\
40.57	0.01\\
40.58	0.01\\
40.59	0.01\\
40.6	0.01\\
40.61	0.01\\
40.62	0.01\\
40.63	0.01\\
40.64	0.01\\
40.65	0.01\\
40.66	0.01\\
40.67	0.01\\
40.68	0.01\\
40.69	0.01\\
40.7	0.01\\
40.71	0.01\\
40.72	0.01\\
40.73	0.01\\
40.74	0.01\\
40.75	0.01\\
40.76	0.01\\
40.77	0.01\\
40.78	0.01\\
40.79	0.01\\
40.8	0.01\\
40.81	0.01\\
40.82	0.01\\
40.83	0.01\\
40.84	0.01\\
40.85	0.01\\
40.86	0.01\\
40.87	0.01\\
40.88	0.01\\
40.89	0.01\\
40.9	0.01\\
40.91	0.01\\
40.92	0.01\\
40.93	0.01\\
40.94	0.01\\
40.95	0.01\\
40.96	0.01\\
40.97	0.01\\
40.98	0.01\\
40.99	0.01\\
41	0.01\\
41.01	0.01\\
41.02	0.01\\
41.03	0.01\\
41.04	0.01\\
41.05	0.01\\
41.06	0.01\\
41.07	0.01\\
41.08	0.01\\
41.09	0.01\\
41.1	0.01\\
41.11	0.01\\
41.12	0.01\\
41.13	0.01\\
41.14	0.01\\
41.15	0.01\\
41.16	0.01\\
41.17	0.01\\
41.18	0.01\\
41.19	0.01\\
41.2	0.01\\
41.21	0.01\\
41.22	0.01\\
41.23	0.01\\
41.24	0.01\\
41.25	0.01\\
41.26	0.01\\
41.27	0.01\\
41.28	0.01\\
41.29	0.01\\
41.3	0.01\\
41.31	0.01\\
41.32	0.01\\
41.33	0.01\\
41.34	0.01\\
41.35	0.01\\
41.36	0.01\\
41.37	0.01\\
41.38	0.01\\
41.39	0.01\\
41.4	0.01\\
41.41	0.01\\
41.42	0.01\\
41.43	0.01\\
41.44	0.01\\
41.45	0.01\\
41.46	0.01\\
41.47	0.01\\
41.48	0.01\\
41.49	0.01\\
41.5	0.01\\
41.51	0.01\\
41.52	0.01\\
41.53	0.01\\
41.54	0.01\\
41.55	0.01\\
41.56	0.01\\
41.57	0.01\\
41.58	0.01\\
41.59	0.01\\
41.6	0.01\\
41.61	0.01\\
41.62	0.01\\
41.63	0.01\\
41.64	0.01\\
41.65	0.01\\
41.66	0.01\\
41.67	0.01\\
41.68	0.01\\
41.69	0.01\\
41.7	0.01\\
41.71	0.01\\
41.72	0.01\\
41.73	0.01\\
41.74	0.01\\
41.75	0.01\\
41.76	0.01\\
41.77	0.01\\
41.78	0.01\\
41.79	0.01\\
41.8	0.01\\
41.81	0.01\\
41.82	0.01\\
41.83	0.01\\
41.84	0.01\\
41.85	0.01\\
41.86	0.01\\
41.87	0.01\\
41.88	0.01\\
41.89	0.01\\
41.9	0.01\\
41.91	0.01\\
41.92	0.01\\
41.93	0.01\\
41.94	0.01\\
41.95	0.01\\
41.96	0.01\\
41.97	0.01\\
41.98	0.01\\
41.99	0.01\\
42	0.01\\
42.01	0.01\\
42.02	0.01\\
42.03	0.01\\
42.04	0.01\\
42.05	0.01\\
42.06	0.01\\
42.07	0.01\\
42.08	0.01\\
42.09	0.01\\
42.1	0.01\\
42.11	0.01\\
42.12	0.01\\
42.13	0.01\\
42.14	0.01\\
42.15	0.01\\
42.16	0.01\\
42.17	0.01\\
42.18	0.01\\
42.19	0.01\\
42.2	0.01\\
42.21	0.01\\
42.22	0.01\\
42.23	0.01\\
42.24	0.01\\
42.25	0.01\\
42.26	0.01\\
42.27	0.01\\
42.28	0.01\\
42.29	0.01\\
42.3	0.01\\
42.31	0.01\\
42.32	0.01\\
42.33	0.01\\
42.34	0.01\\
42.35	0.01\\
42.36	0.01\\
42.37	0.01\\
42.38	0.01\\
42.39	0.01\\
42.4	0.01\\
42.41	0.01\\
42.42	0.01\\
42.43	0.01\\
42.44	0.01\\
42.45	0.01\\
42.46	0.01\\
42.47	0.01\\
42.48	0.01\\
42.49	0.01\\
42.5	0.01\\
42.51	0.01\\
42.52	0.01\\
42.53	0.01\\
42.54	0.01\\
42.55	0.01\\
42.56	0.01\\
42.57	0.01\\
42.58	0.01\\
42.59	0.01\\
42.6	0.01\\
42.61	0.01\\
42.62	0.01\\
42.63	0.01\\
42.64	0.01\\
42.65	0.01\\
42.66	0.01\\
42.67	0.01\\
42.68	0.01\\
42.69	0.01\\
42.7	0.01\\
42.71	0.01\\
42.72	0.01\\
42.73	0.01\\
42.74	0.01\\
42.75	0.01\\
42.76	0.01\\
42.77	0.01\\
42.78	0.01\\
42.79	0.01\\
42.8	0.01\\
42.81	0.01\\
42.82	0.01\\
42.83	0.01\\
42.84	0.01\\
42.85	0.01\\
42.86	0.01\\
42.87	0.01\\
42.88	0.01\\
42.89	0.01\\
42.9	0.01\\
42.91	0.01\\
42.92	0.01\\
42.93	0.01\\
42.94	0.01\\
42.95	0.01\\
42.96	0.01\\
42.97	0.01\\
42.98	0.01\\
42.99	0.01\\
43	0.01\\
43.01	0.01\\
43.02	0.01\\
43.03	0.01\\
43.04	0.01\\
43.05	0.01\\
43.06	0.01\\
43.07	0.01\\
43.08	0.01\\
43.09	0.01\\
43.1	0.01\\
43.11	0.01\\
43.12	0.01\\
43.13	0.01\\
43.14	0.01\\
43.15	0.01\\
43.16	0.01\\
43.17	0.01\\
43.18	0.01\\
43.19	0.01\\
43.2	0.01\\
43.21	0.01\\
43.22	0.01\\
43.23	0.01\\
43.24	0.01\\
43.25	0.01\\
43.26	0.01\\
43.27	0.01\\
43.28	0.01\\
43.29	0.01\\
43.3	0.01\\
43.31	0.01\\
43.32	0.01\\
43.33	0.01\\
43.34	0.01\\
43.35	0.01\\
43.36	0.01\\
43.37	0.01\\
43.38	0.01\\
43.39	0.01\\
43.4	0.01\\
43.41	0.01\\
43.42	0.01\\
43.43	0.01\\
43.44	0.01\\
43.45	0.01\\
43.46	0.01\\
43.47	0.01\\
43.48	0.01\\
43.49	0.01\\
43.5	0.01\\
43.51	0.01\\
43.52	0.01\\
43.53	0.01\\
43.54	0.01\\
43.55	0.01\\
43.56	0.01\\
43.57	0.01\\
43.58	0.01\\
43.59	0.01\\
43.6	0.01\\
43.61	0.01\\
43.62	0.01\\
43.63	0.01\\
43.64	0.01\\
43.65	0.01\\
43.66	0.01\\
43.67	0.01\\
43.68	0.01\\
43.69	0.01\\
43.7	0.01\\
43.71	0.01\\
43.72	0.01\\
43.73	0.01\\
43.74	0.01\\
43.75	0.01\\
43.76	0.01\\
43.77	0.01\\
43.78	0.01\\
43.79	0.01\\
43.8	0.01\\
43.81	0.01\\
43.82	0.01\\
43.83	0.01\\
43.84	0.01\\
43.85	0.01\\
43.86	0.01\\
43.87	0.01\\
43.88	0.01\\
43.89	0.01\\
43.9	0.01\\
43.91	0.01\\
43.92	0.01\\
43.93	0.01\\
43.94	0.01\\
43.95	0.01\\
43.96	0.01\\
43.97	0.01\\
43.98	0.01\\
43.99	0.01\\
44	0.01\\
44.01	0.01\\
44.02	0.01\\
44.03	0.01\\
44.04	0.01\\
44.05	0.01\\
44.06	0.01\\
44.07	0.01\\
44.08	0.01\\
44.09	0.01\\
44.1	0.01\\
44.11	0.01\\
44.12	0.01\\
44.13	0.01\\
44.14	0.01\\
44.15	0.01\\
44.16	0.01\\
44.17	0.01\\
44.18	0.01\\
44.19	0.01\\
44.2	0.01\\
44.21	0.01\\
44.22	0.01\\
44.23	0.01\\
44.24	0.01\\
44.25	0.01\\
44.26	0.01\\
44.27	0.01\\
44.28	0.01\\
44.29	0.01\\
44.3	0.01\\
44.31	0.01\\
44.32	0.01\\
44.33	0.01\\
44.34	0.01\\
44.35	0.01\\
44.36	0.01\\
44.37	0.01\\
44.38	0.01\\
44.39	0.01\\
44.4	0.01\\
44.41	0.01\\
44.42	0.01\\
44.43	0.01\\
44.44	0.01\\
44.45	0.01\\
44.46	0.01\\
44.47	0.01\\
44.48	0.01\\
44.49	0.01\\
44.5	0.01\\
44.51	0.01\\
44.52	0.01\\
44.53	0.01\\
44.54	0.01\\
44.55	0.01\\
44.56	0.01\\
44.57	0.01\\
44.58	0.01\\
44.59	0.01\\
44.6	0.01\\
44.61	0.01\\
44.62	0.01\\
44.63	0.01\\
44.64	0.01\\
44.65	0.01\\
44.66	0.01\\
44.67	0.01\\
44.68	0.01\\
44.69	0.01\\
44.7	0.01\\
44.71	0.01\\
44.72	0.01\\
44.73	0.01\\
44.74	0.01\\
44.75	0.01\\
44.76	0.01\\
44.77	0.01\\
44.78	0.01\\
44.79	0.01\\
44.8	0.01\\
44.81	0.01\\
44.82	0.01\\
44.83	0.01\\
44.84	0.01\\
44.85	0.01\\
44.86	0.01\\
44.87	0.01\\
44.88	0.01\\
44.89	0.01\\
44.9	0.01\\
44.91	0.01\\
44.92	0.01\\
44.93	0.01\\
44.94	0.01\\
44.95	0.01\\
44.96	0.01\\
44.97	0.01\\
44.98	0.01\\
44.99	0.01\\
45	0.01\\
45.01	0.01\\
45.02	0.01\\
45.03	0.01\\
45.04	0.01\\
45.05	0.01\\
45.06	0.01\\
45.07	0.01\\
45.08	0.01\\
45.09	0.01\\
45.1	0.01\\
45.11	0.01\\
45.12	0.01\\
45.13	0.01\\
45.14	0.01\\
45.15	0.01\\
45.16	0.01\\
45.17	0.01\\
45.18	0.01\\
45.19	0.01\\
45.2	0.01\\
45.21	0.01\\
45.22	0.01\\
45.23	0.01\\
45.24	0.01\\
45.25	0.01\\
45.26	0.01\\
45.27	0.01\\
45.28	0.01\\
45.29	0.01\\
45.3	0.01\\
45.31	0.01\\
45.32	0.01\\
45.33	0.01\\
45.34	0.01\\
45.35	0.01\\
45.36	0.01\\
45.37	0.01\\
45.38	0.01\\
45.39	0.01\\
45.4	0.01\\
45.41	0.01\\
45.42	0.01\\
45.43	0.01\\
45.44	0.01\\
45.45	0.01\\
45.46	0.01\\
45.47	0.01\\
45.48	0.01\\
45.49	0.01\\
45.5	0.01\\
45.51	0.01\\
45.52	0.01\\
45.53	0.01\\
45.54	0.01\\
45.55	0.01\\
45.56	0.01\\
45.57	0.01\\
45.58	0.01\\
45.59	0.01\\
45.6	0.01\\
45.61	0.01\\
45.62	0.01\\
45.63	0.01\\
45.64	0.01\\
45.65	0.01\\
45.66	0.01\\
45.67	0.01\\
45.68	0.01\\
45.69	0.01\\
45.7	0.01\\
45.71	0.01\\
45.72	0.01\\
45.73	0.01\\
45.74	0.01\\
45.75	0.01\\
45.76	0.01\\
45.77	0.01\\
45.78	0.01\\
45.79	0.01\\
45.8	0.01\\
45.81	0.01\\
45.82	0.01\\
45.83	0.01\\
45.84	0.01\\
45.85	0.01\\
45.86	0.01\\
45.87	0.01\\
45.88	0.01\\
45.89	0.01\\
45.9	0.01\\
45.91	0.01\\
45.92	0.01\\
45.93	0.01\\
45.94	0.01\\
45.95	0.01\\
45.96	0.01\\
45.97	0.01\\
45.98	0.01\\
45.99	0.01\\
46	0.01\\
46.01	0.01\\
46.02	0.01\\
46.03	0.01\\
46.04	0.01\\
46.05	0.01\\
46.06	0.01\\
46.07	0.01\\
46.08	0.01\\
46.09	0.01\\
46.1	0.01\\
46.11	0.01\\
46.12	0.01\\
46.13	0.01\\
46.14	0.01\\
46.15	0.01\\
46.16	0.01\\
46.17	0.01\\
46.18	0.01\\
46.19	0.01\\
46.2	0.01\\
46.21	0.01\\
46.22	0.01\\
46.23	0.01\\
46.24	0.01\\
46.25	0.01\\
46.26	0.01\\
46.27	0.01\\
46.28	0.01\\
46.29	0.01\\
46.3	0.01\\
46.31	0.01\\
46.32	0.01\\
46.33	0.01\\
46.34	0.01\\
46.35	0.01\\
46.36	0.01\\
46.37	0.01\\
46.38	0.01\\
46.39	0.01\\
46.4	0.01\\
46.41	0.01\\
46.42	0.01\\
46.43	0.01\\
46.44	0.01\\
46.45	0.01\\
46.46	0.01\\
46.47	0.01\\
46.48	0.01\\
46.49	0.01\\
46.5	0.01\\
46.51	0.01\\
46.52	0.01\\
46.53	0.01\\
46.54	0.01\\
46.55	0.01\\
46.56	0.01\\
46.57	0.01\\
46.58	0.01\\
46.59	0.01\\
46.6	0.01\\
46.61	0.01\\
46.62	0.01\\
46.63	0.01\\
46.64	0.01\\
46.65	0.01\\
46.66	0.01\\
46.67	0.01\\
46.68	0.01\\
46.69	0.01\\
46.7	0.01\\
46.71	0.01\\
46.72	0.01\\
46.73	0.01\\
46.74	0.01\\
46.75	0.01\\
46.76	0.01\\
46.77	0.01\\
46.78	0.01\\
46.79	0.01\\
46.8	0.01\\
46.81	0.01\\
46.82	0.01\\
46.83	0.01\\
46.84	0.01\\
46.85	0.01\\
46.86	0.01\\
46.87	0.01\\
46.88	0.01\\
46.89	0.01\\
46.9	0.01\\
46.91	0.01\\
46.92	0.01\\
46.93	0.01\\
46.94	0.01\\
46.95	0.01\\
46.96	0.01\\
46.97	0.01\\
46.98	0.01\\
46.99	0.01\\
47	0.01\\
47.01	0.01\\
47.02	0.01\\
47.03	0.01\\
47.04	0.01\\
47.05	0.01\\
47.06	0.01\\
47.07	0.01\\
47.08	0.01\\
47.09	0.01\\
47.1	0.01\\
47.11	0.01\\
47.12	0.01\\
47.13	0.01\\
47.14	0.01\\
47.15	0.01\\
47.16	0.01\\
47.17	0.01\\
47.18	0.01\\
47.19	0.01\\
47.2	0.01\\
47.21	0.01\\
47.22	0.01\\
47.23	0.01\\
47.24	0.01\\
47.25	0.01\\
47.26	0.01\\
47.27	0.01\\
47.28	0.01\\
47.29	0.01\\
47.3	0.01\\
47.31	0.01\\
47.32	0.01\\
47.33	0.01\\
47.34	0.01\\
47.35	0.01\\
47.36	0.01\\
47.37	0.01\\
47.38	0.01\\
47.39	0.01\\
47.4	0.01\\
47.41	0.01\\
47.42	0.01\\
47.43	0.01\\
47.44	0.01\\
47.45	0.01\\
47.46	0.01\\
47.47	0.01\\
47.48	0.01\\
47.49	0.01\\
47.5	0.01\\
47.51	0.01\\
47.52	0.01\\
47.53	0.01\\
47.54	0.01\\
47.55	0.01\\
47.56	0.01\\
47.57	0.01\\
47.58	0.01\\
47.59	0.01\\
47.6	0.01\\
47.61	0.01\\
47.62	0.01\\
47.63	0.01\\
47.64	0.01\\
47.65	0.01\\
47.66	0.01\\
47.67	0.01\\
47.68	0.01\\
47.69	0.01\\
47.7	0.01\\
47.71	0.01\\
47.72	0.01\\
47.73	0.01\\
47.74	0.01\\
47.75	0.01\\
47.76	0.01\\
47.77	0.01\\
47.78	0.01\\
47.79	0.01\\
47.8	0.01\\
47.81	0.01\\
47.82	0.01\\
47.83	0.01\\
47.84	0.01\\
47.85	0.01\\
47.86	0.01\\
47.87	0.01\\
47.88	0.01\\
47.89	0.01\\
47.9	0.01\\
47.91	0.01\\
47.92	0.01\\
47.93	0.01\\
47.94	0.01\\
47.95	0.01\\
47.96	0.01\\
47.97	0.01\\
47.98	0.01\\
47.99	0.01\\
48	0.01\\
48.01	0.01\\
48.02	0.01\\
48.03	0.01\\
48.04	0.01\\
48.05	0.01\\
48.06	0.01\\
48.07	0.01\\
48.08	0.01\\
48.09	0.01\\
48.1	0.01\\
48.11	0.01\\
48.12	0.01\\
48.13	0.01\\
48.14	0.01\\
48.15	0.01\\
48.16	0.01\\
48.17	0.01\\
48.18	0.01\\
48.19	0.01\\
48.2	0.01\\
48.21	0.01\\
48.22	0.01\\
48.23	0.01\\
48.24	0.01\\
48.25	0.01\\
48.26	0.01\\
48.27	0.01\\
48.28	0.01\\
48.29	0.01\\
48.3	0.01\\
48.31	0.01\\
48.32	0.01\\
48.33	0.01\\
48.34	0.01\\
48.35	0.01\\
48.36	0.01\\
48.37	0.01\\
48.38	0.01\\
48.39	0.01\\
48.4	0.01\\
48.41	0.01\\
48.42	0.01\\
48.43	0.01\\
48.44	0.01\\
48.45	0.01\\
48.46	0.01\\
48.47	0.01\\
48.48	0.01\\
48.49	0.01\\
48.5	0.01\\
48.51	0.01\\
48.52	0.01\\
48.53	0.01\\
48.54	0.01\\
48.55	0.01\\
48.56	0.01\\
48.57	0.01\\
48.58	0.01\\
48.59	0.01\\
48.6	0.01\\
48.61	0.01\\
48.62	0.01\\
48.63	0.01\\
48.64	0.01\\
48.65	0.01\\
48.66	0.01\\
48.67	0.01\\
48.68	0.01\\
48.69	0.01\\
48.7	0.01\\
48.71	0.01\\
48.72	0.01\\
48.73	0.01\\
48.74	0.01\\
48.75	0.01\\
48.76	0.01\\
48.77	0.01\\
48.78	0.01\\
48.79	0.01\\
48.8	0.01\\
48.81	0.01\\
48.82	0.01\\
48.83	0.01\\
48.84	0.01\\
48.85	0.01\\
48.86	0.01\\
48.87	0.01\\
48.88	0.01\\
48.89	0.01\\
48.9	0.01\\
48.91	0.01\\
48.92	0.01\\
48.93	0.01\\
48.94	0.01\\
48.95	0.01\\
48.96	0.01\\
48.97	0.01\\
48.98	0.01\\
48.99	0.01\\
49	0.01\\
49.01	0.01\\
49.02	0.01\\
49.03	0.01\\
49.04	0.01\\
49.05	0.01\\
49.06	0.01\\
49.07	0.01\\
49.08	0.01\\
49.09	0.01\\
49.1	0.01\\
49.11	0.01\\
49.12	0.01\\
49.13	0.01\\
49.14	0.01\\
49.15	0.01\\
49.16	0.01\\
49.17	0.01\\
49.18	0.01\\
49.19	0.01\\
49.2	0.01\\
49.21	0.01\\
49.22	0.01\\
49.23	0.01\\
49.24	0.01\\
49.25	0.01\\
49.26	0.01\\
49.27	0.01\\
49.28	0.01\\
49.29	0.01\\
49.3	0.01\\
49.31	0.01\\
49.32	0.01\\
49.33	0.01\\
49.34	0.01\\
49.35	0.01\\
49.36	0.01\\
49.37	0.01\\
49.38	0.01\\
49.39	0.01\\
49.4	0.01\\
49.41	0.01\\
49.42	0.01\\
49.43	0.01\\
49.44	0.01\\
49.45	0.01\\
49.46	0.01\\
49.47	0.01\\
49.48	0.01\\
49.49	0.01\\
49.5	0.01\\
49.51	0.01\\
49.52	0.01\\
49.53	0.01\\
49.54	0.01\\
49.55	0.01\\
49.56	0.01\\
49.57	0.01\\
49.58	0.01\\
49.59	0.01\\
49.6	0.01\\
49.61	0.01\\
49.62	0.01\\
49.63	0.01\\
49.64	0.01\\
49.65	0.01\\
49.66	0.01\\
49.67	0.01\\
49.68	0.01\\
49.69	0.01\\
49.7	0.01\\
49.71	0.01\\
49.72	0.01\\
49.73	0.01\\
49.74	0.01\\
49.75	0.01\\
49.76	0.01\\
49.77	0.01\\
49.78	0.01\\
49.79	0.01\\
49.8	0.01\\
49.81	0.01\\
49.82	0.01\\
49.83	0.01\\
49.84	0.01\\
49.85	0.01\\
49.86	0.01\\
49.87	0.01\\
49.88	0.01\\
49.89	0.01\\
49.9	0.01\\
49.91	0.01\\
49.92	0.01\\
49.93	0.01\\
49.94	0.01\\
49.95	0.01\\
49.96	0.01\\
49.97	0.01\\
49.98	0.01\\
49.99	0.01\\
50	0.01\\
50.01	0.01\\
50.02	0.01\\
50.03	0.01\\
50.04	0.01\\
50.05	0.01\\
50.06	0.01\\
50.07	0.01\\
50.08	0.01\\
50.09	0.01\\
50.1	0.01\\
50.11	0.01\\
50.12	0.01\\
50.13	0.01\\
50.14	0.01\\
50.15	0.01\\
50.16	0.01\\
50.17	0.01\\
50.18	0.01\\
50.19	0.01\\
50.2	0.01\\
50.21	0.01\\
50.22	0.01\\
50.23	0.01\\
50.24	0.01\\
50.25	0.01\\
50.26	0.01\\
50.27	0.01\\
50.28	0.01\\
50.29	0.01\\
50.3	0.01\\
50.31	0.01\\
50.32	0.01\\
50.33	0.01\\
50.34	0.01\\
50.35	0.01\\
50.36	0.01\\
50.37	0.01\\
50.38	0.01\\
50.39	0.01\\
50.4	0.01\\
50.41	0.01\\
50.42	0.01\\
50.43	0.01\\
50.44	0.01\\
50.45	0.01\\
50.46	0.01\\
50.47	0.01\\
50.48	0.01\\
50.49	0.01\\
50.5	0.01\\
50.51	0.01\\
50.52	0.01\\
50.53	0.01\\
50.54	0.01\\
50.55	0.01\\
50.56	0.01\\
50.57	0.01\\
50.58	0.01\\
50.59	0.01\\
50.6	0.01\\
50.61	0.01\\
50.62	0.01\\
50.63	0.01\\
50.64	0.01\\
50.65	0.01\\
50.66	0.01\\
50.67	0.01\\
50.68	0.01\\
50.69	0.01\\
50.7	0.01\\
50.71	0.01\\
50.72	0.01\\
50.73	0.01\\
50.74	0.01\\
50.75	0.01\\
50.76	0.01\\
50.77	0.01\\
50.78	0.01\\
50.79	0.01\\
50.8	0.01\\
50.81	0.01\\
50.82	0.01\\
50.83	0.01\\
50.84	0.01\\
50.85	0.01\\
50.86	0.01\\
50.87	0.01\\
50.88	0.01\\
50.89	0.01\\
50.9	0.01\\
50.91	0.01\\
50.92	0.01\\
50.93	0.01\\
50.94	0.01\\
50.95	0.01\\
50.96	0.01\\
50.97	0.01\\
50.98	0.01\\
50.99	0.01\\
51	0.01\\
51.01	0.01\\
51.02	0.01\\
51.03	0.01\\
51.04	0.01\\
51.05	0.01\\
51.06	0.01\\
51.07	0.01\\
51.08	0.01\\
51.09	0.01\\
51.1	0.01\\
51.11	0.01\\
51.12	0.01\\
51.13	0.01\\
51.14	0.01\\
51.15	0.01\\
51.16	0.01\\
51.17	0.01\\
51.18	0.01\\
51.19	0.01\\
51.2	0.01\\
51.21	0.01\\
51.22	0.01\\
51.23	0.01\\
51.24	0.01\\
51.25	0.01\\
51.26	0.01\\
51.27	0.01\\
51.28	0.01\\
51.29	0.01\\
51.3	0.01\\
51.31	0.01\\
51.32	0.01\\
51.33	0.01\\
51.34	0.01\\
51.35	0.01\\
51.36	0.01\\
51.37	0.01\\
51.38	0.01\\
51.39	0.01\\
51.4	0.01\\
51.41	0.01\\
51.42	0.01\\
51.43	0.01\\
51.44	0.01\\
51.45	0.01\\
51.46	0.01\\
51.47	0.01\\
51.48	0.01\\
51.49	0.01\\
51.5	0.01\\
51.51	0.01\\
51.52	0.01\\
51.53	0.01\\
51.54	0.01\\
51.55	0.01\\
51.56	0.01\\
51.57	0.01\\
51.58	0.01\\
51.59	0.01\\
51.6	0.01\\
51.61	0.01\\
51.62	0.01\\
51.63	0.01\\
51.64	0.01\\
51.65	0.01\\
51.66	0.01\\
51.67	0.01\\
51.68	0.01\\
51.69	0.01\\
51.7	0.01\\
51.71	0.01\\
51.72	0.01\\
51.73	0.01\\
51.74	0.01\\
51.75	0.01\\
51.76	0.01\\
51.77	0.01\\
51.78	0.01\\
51.79	0.01\\
51.8	0.01\\
51.81	0.01\\
51.82	0.01\\
51.83	0.01\\
51.84	0.01\\
51.85	0.01\\
51.86	0.01\\
51.87	0.01\\
51.88	0.01\\
51.89	0.01\\
51.9	0.01\\
51.91	0.01\\
51.92	0.01\\
51.93	0.01\\
51.94	0.01\\
51.95	0.01\\
51.96	0.01\\
51.97	0.01\\
51.98	0.01\\
51.99	0.01\\
52	0.01\\
52.01	0.01\\
52.02	0.01\\
52.03	0.01\\
52.04	0.01\\
52.05	0.01\\
52.06	0.01\\
52.07	0.01\\
52.08	0.01\\
52.09	0.01\\
52.1	0.01\\
52.11	0.01\\
52.12	0.01\\
52.13	0.01\\
52.14	0.01\\
52.15	0.01\\
52.16	0.01\\
52.17	0.01\\
52.18	0.01\\
52.19	0.01\\
52.2	0.01\\
52.21	0.01\\
52.22	0.01\\
52.23	0.01\\
52.24	0.01\\
52.25	0.01\\
52.26	0.01\\
52.27	0.01\\
52.28	0.01\\
52.29	0.01\\
52.3	0.01\\
52.31	0.01\\
52.32	0.01\\
52.33	0.01\\
52.34	0.01\\
52.35	0.01\\
52.36	0.01\\
52.37	0.01\\
52.38	0.01\\
52.39	0.01\\
52.4	0.01\\
52.41	0.01\\
52.42	0.01\\
52.43	0.01\\
52.44	0.01\\
52.45	0.01\\
52.46	0.01\\
52.47	0.01\\
52.48	0.01\\
52.49	0.01\\
52.5	0.01\\
52.51	0.01\\
52.52	0.01\\
52.53	0.01\\
52.54	0.01\\
52.55	0.01\\
52.56	0.01\\
52.57	0.01\\
52.58	0.01\\
52.59	0.01\\
52.6	0.01\\
52.61	0.01\\
52.62	0.01\\
52.63	0.01\\
52.64	0.01\\
52.65	0.01\\
52.66	0.01\\
52.67	0.01\\
52.68	0.01\\
52.69	0.01\\
52.7	0.01\\
52.71	0.01\\
52.72	0.01\\
52.73	0.01\\
52.74	0.01\\
52.75	0.01\\
52.76	0.01\\
52.77	0.01\\
52.78	0.01\\
52.79	0.01\\
52.8	0.01\\
52.81	0.01\\
52.82	0.01\\
52.83	0.01\\
52.84	0.01\\
52.85	0.01\\
52.86	0.01\\
52.87	0.01\\
52.88	0.01\\
52.89	0.01\\
52.9	0.01\\
52.91	0.01\\
52.92	0.01\\
52.93	0.01\\
52.94	0.01\\
52.95	0.01\\
52.96	0.01\\
52.97	0.01\\
52.98	0.01\\
52.99	0.01\\
53	0.01\\
53.01	0.01\\
53.02	0.01\\
53.03	0.01\\
53.04	0.01\\
53.05	0.01\\
53.06	0.01\\
53.07	0.01\\
53.08	0.01\\
53.09	0.01\\
53.1	0.01\\
53.11	0.01\\
53.12	0.01\\
53.13	0.01\\
53.14	0.01\\
53.15	0.01\\
53.16	0.01\\
53.17	0.01\\
53.18	0.01\\
53.19	0.01\\
53.2	0.01\\
53.21	0.01\\
53.22	0.01\\
53.23	0.01\\
53.24	0.01\\
53.25	0.01\\
53.26	0.01\\
53.27	0.01\\
53.28	0.01\\
53.29	0.01\\
53.3	0.01\\
53.31	0.01\\
53.32	0.01\\
53.33	0.01\\
53.34	0.01\\
53.35	0.01\\
53.36	0.01\\
53.37	0.01\\
53.38	0.01\\
53.39	0.01\\
53.4	0.01\\
53.41	0.01\\
53.42	0.01\\
53.43	0.01\\
53.44	0.01\\
53.45	0.01\\
53.46	0.01\\
53.47	0.01\\
53.48	0.01\\
53.49	0.01\\
53.5	0.01\\
53.51	0.01\\
53.52	0.01\\
53.53	0.01\\
53.54	0.01\\
53.55	0.01\\
53.56	0.01\\
53.57	0.01\\
53.58	0.01\\
53.59	0.01\\
53.6	0.01\\
53.61	0.01\\
53.62	0.01\\
53.63	0.01\\
53.64	0.01\\
53.65	0.01\\
53.66	0.01\\
53.67	0.01\\
53.68	0.01\\
53.69	0.01\\
53.7	0.01\\
53.71	0.01\\
53.72	0.01\\
53.73	0.01\\
53.74	0.01\\
53.75	0.01\\
53.76	0.01\\
53.77	0.01\\
53.78	0.01\\
53.79	0.01\\
53.8	0.01\\
53.81	0.01\\
53.82	0.01\\
53.83	0.01\\
53.84	0.01\\
53.85	0.01\\
53.86	0.01\\
53.87	0.01\\
53.88	0.01\\
53.89	0.01\\
53.9	0.01\\
53.91	0.01\\
53.92	0.01\\
53.93	0.01\\
53.94	0.01\\
53.95	0.01\\
53.96	0.01\\
53.97	0.01\\
53.98	0.01\\
53.99	0.01\\
54	0.01\\
54.01	0.01\\
54.02	0.01\\
54.03	0.01\\
54.04	0.01\\
54.05	0.01\\
54.06	0.01\\
54.07	0.01\\
54.08	0.01\\
54.09	0.01\\
54.1	0.01\\
54.11	0.01\\
54.12	0.01\\
54.13	0.01\\
54.14	0.01\\
54.15	0.01\\
54.16	0.01\\
54.17	0.01\\
54.18	0.01\\
54.19	0.01\\
54.2	0.01\\
54.21	0.01\\
54.22	0.01\\
54.23	0.01\\
54.24	0.01\\
54.25	0.01\\
54.26	0.01\\
54.27	0.01\\
54.28	0.01\\
54.29	0.01\\
54.3	0.01\\
54.31	0.01\\
54.32	0.01\\
54.33	0.01\\
54.34	0.01\\
54.35	0.01\\
54.36	0.01\\
54.37	0.01\\
54.38	0.01\\
54.39	0.01\\
54.4	0.01\\
54.41	0.01\\
54.42	0.01\\
54.43	0.01\\
54.44	0.01\\
54.45	0.01\\
54.46	0.01\\
54.47	0.01\\
54.48	0.01\\
54.49	0.01\\
54.5	0.01\\
54.51	0.01\\
54.52	0.01\\
54.53	0.01\\
54.54	0.01\\
54.55	0.01\\
54.56	0.01\\
54.57	0.01\\
54.58	0.01\\
54.59	0.01\\
54.6	0.01\\
54.61	0.01\\
54.62	0.01\\
54.63	0.01\\
54.64	0.01\\
54.65	0.01\\
54.66	0.01\\
54.67	0.01\\
54.68	0.01\\
54.69	0.01\\
54.7	0.01\\
54.71	0.01\\
54.72	0.01\\
54.73	0.01\\
54.74	0.01\\
54.75	0.01\\
54.76	0.01\\
54.77	0.01\\
54.78	0.01\\
54.79	0.01\\
54.8	0.01\\
54.81	0.01\\
54.82	0.01\\
54.83	0.01\\
54.84	0.01\\
54.85	0.01\\
54.86	0.01\\
54.87	0.01\\
54.88	0.01\\
54.89	0.01\\
54.9	0.01\\
54.91	0.01\\
54.92	0.01\\
54.93	0.01\\
54.94	0.01\\
54.95	0.01\\
54.96	0.01\\
54.97	0.01\\
54.98	0.01\\
54.99	0.01\\
55	0.01\\
55.01	0.01\\
55.02	0.01\\
55.03	0.01\\
55.04	0.01\\
55.05	0.01\\
55.06	0.01\\
55.07	0.01\\
55.08	0.01\\
55.09	0.01\\
55.1	0.01\\
55.11	0.01\\
55.12	0.01\\
55.13	0.01\\
55.14	0.01\\
55.15	0.01\\
55.16	0.01\\
55.17	0.01\\
55.18	0.01\\
55.19	0.01\\
55.2	0.01\\
55.21	0.01\\
55.22	0.01\\
55.23	0.01\\
55.24	0.01\\
55.25	0.01\\
55.26	0.01\\
55.27	0.01\\
55.28	0.01\\
55.29	0.01\\
55.3	0.01\\
55.31	0.01\\
55.32	0.01\\
55.33	0.01\\
55.34	0.01\\
55.35	0.01\\
55.36	0.01\\
55.37	0.01\\
55.38	0.01\\
55.39	0.01\\
55.4	0.01\\
55.41	0.01\\
55.42	0.01\\
55.43	0.01\\
55.44	0.01\\
55.45	0.01\\
55.46	0.01\\
55.47	0.01\\
55.48	0.01\\
55.49	0.01\\
55.5	0.01\\
55.51	0.01\\
55.52	0.01\\
55.53	0.01\\
55.54	0.01\\
55.55	0.01\\
55.56	0.01\\
55.57	0.01\\
55.58	0.01\\
55.59	0.01\\
55.6	0.01\\
55.61	0.01\\
55.62	0.01\\
55.63	0.01\\
55.64	0.01\\
55.65	0.01\\
55.66	0.01\\
55.67	0.01\\
55.68	0.01\\
55.69	0.01\\
55.7	0.01\\
55.71	0.01\\
55.72	0.01\\
55.73	0.01\\
55.74	0.01\\
55.75	0.01\\
55.76	0.01\\
55.77	0.01\\
55.78	0.01\\
55.79	0.01\\
55.8	0.01\\
55.81	0.01\\
55.82	0.01\\
55.83	0.01\\
55.84	0.01\\
55.85	0.01\\
55.86	0.01\\
55.87	0.01\\
55.88	0.01\\
55.89	0.01\\
55.9	0.01\\
55.91	0.01\\
55.92	0.01\\
55.93	0.01\\
55.94	0.01\\
55.95	0.01\\
55.96	0.01\\
55.97	0.01\\
55.98	0.01\\
55.99	0.01\\
56	0.01\\
56.01	0.01\\
56.02	0.01\\
56.03	0.01\\
56.04	0.01\\
56.05	0.01\\
56.06	0.01\\
56.07	0.01\\
56.08	0.01\\
56.09	0.01\\
56.1	0.01\\
56.11	0.01\\
56.12	0.01\\
56.13	0.01\\
56.14	0.01\\
56.15	0.01\\
56.16	0.01\\
56.17	0.01\\
56.18	0.01\\
56.19	0.01\\
56.2	0.01\\
56.21	0.01\\
56.22	0.01\\
56.23	0.01\\
56.24	0.01\\
56.25	0.01\\
56.26	0.01\\
56.27	0.01\\
56.28	0.01\\
56.29	0.01\\
56.3	0.01\\
56.31	0.01\\
56.32	0.01\\
56.33	0.01\\
56.34	0.01\\
56.35	0.01\\
56.36	0.01\\
56.37	0.01\\
56.38	0.01\\
56.39	0.01\\
56.4	0.01\\
56.41	0.01\\
56.42	0.01\\
56.43	0.01\\
56.44	0.01\\
56.45	0.01\\
56.46	0.01\\
56.47	0.01\\
56.48	0.01\\
56.49	0.01\\
56.5	0.01\\
56.51	0.01\\
56.52	0.01\\
56.53	0.01\\
56.54	0.01\\
56.55	0.01\\
56.56	0.01\\
56.57	0.01\\
56.58	0.01\\
56.59	0.01\\
56.6	0.01\\
56.61	0.01\\
56.62	0.01\\
56.63	0.01\\
56.64	0.01\\
56.65	0.01\\
56.66	0.01\\
56.67	0.01\\
56.68	0.01\\
56.69	0.01\\
56.7	0.01\\
56.71	0.01\\
56.72	0.01\\
56.73	0.01\\
56.74	0.01\\
56.75	0.01\\
56.76	0.01\\
56.77	0.01\\
56.78	0.01\\
56.79	0.01\\
56.8	0.01\\
56.81	0.01\\
56.82	0.01\\
56.83	0.01\\
56.84	0.01\\
56.85	0.01\\
56.86	0.01\\
56.87	0.01\\
56.88	0.01\\
56.89	0.01\\
56.9	0.01\\
56.91	0.01\\
56.92	0.01\\
56.93	0.01\\
56.94	0.01\\
56.95	0.01\\
56.96	0.01\\
56.97	0.01\\
56.98	0.01\\
56.99	0.01\\
57	0.01\\
57.01	0.01\\
57.02	0.01\\
57.03	0.01\\
57.04	0.01\\
57.05	0.01\\
57.06	0.01\\
57.07	0.01\\
57.08	0.01\\
57.09	0.01\\
57.1	0.01\\
57.11	0.01\\
57.12	0.01\\
57.13	0.01\\
57.14	0.01\\
57.15	0.01\\
57.16	0.01\\
57.17	0.01\\
57.18	0.01\\
57.19	0.01\\
57.2	0.01\\
57.21	0.01\\
57.22	0.01\\
57.23	0.01\\
57.24	0.01\\
57.25	0.01\\
57.26	0.01\\
57.27	0.01\\
57.28	0.01\\
57.29	0.01\\
57.3	0.01\\
57.31	0.01\\
57.32	0.01\\
57.33	0.01\\
57.34	0.01\\
57.35	0.01\\
57.36	0.01\\
57.37	0.01\\
57.38	0.01\\
57.39	0.01\\
57.4	0.01\\
57.41	0.01\\
57.42	0.01\\
57.43	0.01\\
57.44	0.01\\
57.45	0.01\\
57.46	0.01\\
57.47	0.01\\
57.48	0.01\\
57.49	0.01\\
57.5	0.01\\
57.51	0.01\\
57.52	0.01\\
57.53	0.01\\
57.54	0.01\\
57.55	0.01\\
57.56	0.01\\
57.57	0.01\\
57.58	0.01\\
57.59	0.01\\
57.6	0.01\\
57.61	0.01\\
57.62	0.01\\
57.63	0.01\\
57.64	0.01\\
57.65	0.01\\
57.66	0.01\\
57.67	0.01\\
57.68	0.01\\
57.69	0.01\\
57.7	0.01\\
57.71	0.01\\
57.72	0.01\\
57.73	0.01\\
57.74	0.01\\
57.75	0.01\\
57.76	0.01\\
57.77	0.01\\
57.78	0.01\\
57.79	0.01\\
57.8	0.01\\
57.81	0.01\\
57.82	0.01\\
57.83	0.01\\
57.84	0.01\\
57.85	0.01\\
57.86	0.01\\
57.87	0.01\\
57.88	0.01\\
57.89	0.01\\
57.9	0.01\\
57.91	0.01\\
57.92	0.01\\
57.93	0.01\\
57.94	0.01\\
57.95	0.01\\
57.96	0.01\\
57.97	0.01\\
57.98	0.01\\
57.99	0.01\\
58	0.01\\
58.01	0.01\\
58.02	0.01\\
58.03	0.01\\
58.04	0.01\\
58.05	0.01\\
58.06	0.01\\
58.07	0.01\\
58.08	0.01\\
58.09	0.01\\
58.1	0.01\\
58.11	0.01\\
58.12	0.01\\
58.13	0.01\\
58.14	0.01\\
58.15	0.01\\
58.16	0.01\\
58.17	0.01\\
58.18	0.01\\
58.19	0.01\\
58.2	0.01\\
58.21	0.01\\
58.22	0.01\\
58.23	0.01\\
58.24	0.01\\
58.25	0.01\\
58.26	0.01\\
58.27	0.01\\
58.28	0.01\\
58.29	0.01\\
58.3	0.01\\
58.31	0.01\\
58.32	0.01\\
58.33	0.01\\
58.34	0.01\\
58.35	0.01\\
58.36	0.01\\
58.37	0.01\\
58.38	0.01\\
58.39	0.01\\
58.4	0.01\\
58.41	0.01\\
58.42	0.01\\
58.43	0.01\\
58.44	0.01\\
58.45	0.01\\
58.46	0.01\\
58.47	0.01\\
58.48	0.01\\
58.49	0.01\\
58.5	0.01\\
58.51	0.01\\
58.52	0.01\\
58.53	0.01\\
58.54	0.01\\
58.55	0.01\\
58.56	0.01\\
58.57	0.01\\
58.58	0.01\\
58.59	0.01\\
58.6	0.01\\
58.61	0.01\\
58.62	0.01\\
58.63	0.01\\
58.64	0.01\\
58.65	0.01\\
58.66	0.01\\
58.67	0.01\\
58.68	0.01\\
58.69	0.01\\
58.7	0.01\\
58.71	0.01\\
58.72	0.01\\
58.73	0.01\\
58.74	0.01\\
58.75	0.01\\
58.76	0.01\\
58.77	0.01\\
58.78	0.01\\
58.79	0.01\\
58.8	0.01\\
58.81	0.01\\
58.82	0.01\\
58.83	0.01\\
58.84	0.01\\
58.85	0.01\\
58.86	0.01\\
58.87	0.01\\
58.88	0.01\\
58.89	0.01\\
58.9	0.01\\
58.91	0.01\\
58.92	0.01\\
58.93	0.01\\
58.94	0.01\\
58.95	0.01\\
58.96	0.01\\
58.97	0.01\\
58.98	0.01\\
58.99	0.01\\
59	0.01\\
59.01	0.01\\
59.02	0.01\\
59.03	0.01\\
59.04	0.01\\
59.05	0.01\\
59.06	0.01\\
59.07	0.01\\
59.08	0.01\\
59.09	0.01\\
59.1	0.01\\
59.11	0.01\\
59.12	0.01\\
59.13	0.01\\
59.14	0.01\\
59.15	0.01\\
59.16	0.01\\
59.17	0.01\\
59.18	0.01\\
59.19	0.01\\
59.2	0.01\\
59.21	0.01\\
59.22	0.01\\
59.23	0.01\\
59.24	0.01\\
59.25	0.01\\
59.26	0.01\\
59.27	0.01\\
59.28	0.01\\
59.29	0.01\\
59.3	0.01\\
59.31	0.01\\
59.32	0.01\\
59.33	0.01\\
59.34	0.01\\
59.35	0.01\\
59.36	0.01\\
59.37	0.01\\
59.38	0.01\\
59.39	0.01\\
59.4	0.01\\
59.41	0.01\\
59.42	0.01\\
59.43	0.01\\
59.44	0.01\\
59.45	0.01\\
59.46	0.01\\
59.47	0.01\\
59.48	0.01\\
59.49	0.01\\
59.5	0.01\\
59.51	0.01\\
59.52	0.01\\
59.53	0.01\\
59.54	0.01\\
59.55	0.01\\
59.56	0.01\\
59.57	0.01\\
59.58	0.01\\
59.59	0.01\\
59.6	0.01\\
59.61	0.01\\
59.62	0.01\\
59.63	0.01\\
59.64	0.01\\
59.65	0.01\\
59.66	0.01\\
59.67	0.01\\
59.68	0.01\\
59.69	0.01\\
59.7	0.01\\
59.71	0.01\\
59.72	0.01\\
59.73	0.01\\
59.74	0.01\\
59.75	0.01\\
59.76	0.01\\
59.77	0.01\\
59.78	0.01\\
59.79	0.01\\
59.8	0.01\\
59.81	0.01\\
59.82	0.01\\
59.83	0.01\\
59.84	0.01\\
59.85	0.01\\
59.86	0.01\\
59.87	0.01\\
59.88	0.01\\
59.89	0.01\\
59.9	0.01\\
59.91	0.01\\
59.92	0.01\\
59.93	0.01\\
59.94	0.01\\
59.95	0.01\\
59.96	0.01\\
59.97	0.01\\
59.98	0.01\\
59.99	0.01\\
60	0.01\\
60.01	0.01\\
60.02	0.01\\
60.03	0.01\\
60.04	0.01\\
60.05	0.01\\
60.06	0.01\\
60.07	0.01\\
60.08	0.01\\
60.09	0.01\\
60.1	0.01\\
60.11	0.01\\
60.12	0.01\\
60.13	0.01\\
60.14	0.01\\
60.15	0.01\\
60.16	0.01\\
60.17	0.01\\
60.18	0.01\\
60.19	0.01\\
60.2	0.01\\
60.21	0.01\\
60.22	0.01\\
60.23	0.01\\
60.24	0.01\\
60.25	0.01\\
60.26	0.01\\
60.27	0.01\\
60.28	0.01\\
60.29	0.01\\
60.3	0.01\\
60.31	0.01\\
60.32	0.01\\
60.33	0.01\\
60.34	0.01\\
60.35	0.01\\
60.36	0.01\\
60.37	0.01\\
60.38	0.01\\
60.39	0.01\\
60.4	0.01\\
60.41	0.01\\
60.42	0.01\\
60.43	0.01\\
60.44	0.01\\
60.45	0.01\\
60.46	0.01\\
60.47	0.01\\
60.48	0.01\\
60.49	0.01\\
60.5	0.01\\
60.51	0.01\\
60.52	0.01\\
60.53	0.01\\
60.54	0.01\\
60.55	0.01\\
60.56	0.01\\
60.57	0.01\\
60.58	0.01\\
60.59	0.01\\
60.6	0.01\\
60.61	0.01\\
60.62	0.01\\
60.63	0.01\\
60.64	0.01\\
60.65	0.01\\
60.66	0.01\\
60.67	0.01\\
60.68	0.01\\
60.69	0.01\\
60.7	0.01\\
60.71	0.01\\
60.72	0.01\\
60.73	0.01\\
60.74	0.01\\
60.75	0.01\\
60.76	0.01\\
60.77	0.01\\
60.78	0.01\\
60.79	0.01\\
60.8	0.01\\
60.81	0.01\\
60.82	0.01\\
60.83	0.01\\
60.84	0.01\\
60.85	0.01\\
60.86	0.01\\
60.87	0.01\\
60.88	0.01\\
60.89	0.01\\
60.9	0.01\\
60.91	0.01\\
60.92	0.01\\
60.93	0.01\\
60.94	0.01\\
60.95	0.01\\
60.96	0.01\\
60.97	0.01\\
60.98	0.01\\
60.99	0.01\\
61	0.01\\
61.01	0.01\\
61.02	0.01\\
61.03	0.01\\
61.04	0.01\\
61.05	0.01\\
61.06	0.01\\
61.07	0.01\\
61.08	0.01\\
61.09	0.01\\
61.1	0.01\\
61.11	0.01\\
61.12	0.01\\
61.13	0.01\\
61.14	0.01\\
61.15	0.01\\
61.16	0.01\\
61.17	0.01\\
61.18	0.01\\
61.19	0.01\\
61.2	0.01\\
61.21	0.01\\
61.22	0.01\\
61.23	0.01\\
61.24	0.01\\
61.25	0.01\\
61.26	0.01\\
61.27	0.01\\
61.28	0.01\\
61.29	0.01\\
61.3	0.01\\
61.31	0.01\\
61.32	0.01\\
61.33	0.01\\
61.34	0.01\\
61.35	0.01\\
61.36	0.01\\
61.37	0.01\\
61.38	0.01\\
61.39	0.01\\
61.4	0.01\\
61.41	0.01\\
61.42	0.01\\
61.43	0.01\\
61.44	0.01\\
61.45	0.01\\
61.46	0.01\\
61.47	0.01\\
61.48	0.01\\
61.49	0.01\\
61.5	0.01\\
61.51	0.01\\
61.52	0.01\\
61.53	0.01\\
61.54	0.01\\
61.55	0.01\\
61.56	0.01\\
61.57	0.01\\
61.58	0.01\\
61.59	0.01\\
61.6	0.01\\
61.61	0.01\\
61.62	0.01\\
61.63	0.01\\
61.64	0.01\\
61.65	0.01\\
61.66	0.01\\
61.67	0.01\\
61.68	0.01\\
61.69	0.01\\
61.7	0.01\\
61.71	0.01\\
61.72	0.01\\
61.73	0.01\\
61.74	0.01\\
61.75	0.01\\
61.76	0.01\\
61.77	0.01\\
61.78	0.01\\
61.79	0.01\\
61.8	0.01\\
61.81	0.01\\
61.82	0.01\\
61.83	0.01\\
61.84	0.01\\
61.85	0.01\\
61.86	0.01\\
61.87	0.01\\
61.88	0.01\\
61.89	0.01\\
61.9	0.01\\
61.91	0.01\\
61.92	0.01\\
61.93	0.01\\
61.94	0.01\\
61.95	0.01\\
61.96	0.01\\
61.97	0.01\\
61.98	0.01\\
61.99	0.01\\
62	0.01\\
62.01	0.01\\
62.02	0.01\\
62.03	0.01\\
62.04	0.01\\
62.05	0.01\\
62.06	0.01\\
62.07	0.01\\
62.08	0.01\\
62.09	0.01\\
62.1	0.01\\
62.11	0.01\\
62.12	0.01\\
62.13	0.01\\
62.14	0.01\\
62.15	0.01\\
62.16	0.01\\
62.17	0.01\\
62.18	0.01\\
62.19	0.01\\
62.2	0.01\\
62.21	0.01\\
62.22	0.01\\
62.23	0.01\\
62.24	0.01\\
62.25	0.01\\
62.26	0.01\\
62.27	0.01\\
62.28	0.01\\
62.29	0.01\\
62.3	0.01\\
62.31	0.01\\
62.32	0.01\\
62.33	0.01\\
62.34	0.01\\
62.35	0.01\\
62.36	0.01\\
62.37	0.01\\
62.38	0.01\\
62.39	0.01\\
62.4	0.01\\
62.41	0.01\\
62.42	0.01\\
62.43	0.01\\
62.44	0.01\\
62.45	0.01\\
62.46	0.01\\
62.47	0.01\\
62.48	0.01\\
62.49	0.01\\
62.5	0.01\\
62.51	0.01\\
62.52	0.01\\
62.53	0.01\\
62.54	0.01\\
62.55	0.01\\
62.56	0.01\\
62.57	0.01\\
62.58	0.01\\
62.59	0.01\\
62.6	0.01\\
62.61	0.01\\
62.62	0.01\\
62.63	0.01\\
62.64	0.01\\
62.65	0.01\\
62.66	0.01\\
62.67	0.01\\
62.68	0.01\\
62.69	0.01\\
62.7	0.01\\
62.71	0.01\\
62.72	0.01\\
62.73	0.01\\
62.74	0.01\\
62.75	0.01\\
62.76	0.01\\
62.77	0.01\\
62.78	0.01\\
62.79	0.01\\
62.8	0.01\\
62.81	0.01\\
62.82	0.01\\
62.83	0.01\\
62.84	0.01\\
62.85	0.01\\
62.86	0.01\\
62.87	0.01\\
62.88	0.01\\
62.89	0.01\\
62.9	0.01\\
62.91	0.01\\
62.92	0.01\\
62.93	0.01\\
62.94	0.01\\
62.95	0.01\\
62.96	0.01\\
62.97	0.01\\
62.98	0.01\\
62.99	0.01\\
63	0.01\\
63.01	0.01\\
63.02	0.01\\
63.03	0.01\\
63.04	0.01\\
63.05	0.01\\
63.06	0.01\\
63.07	0.01\\
63.08	0.01\\
63.09	0.01\\
63.1	0.01\\
63.11	0.01\\
63.12	0.01\\
63.13	0.01\\
63.14	0.01\\
63.15	0.01\\
63.16	0.01\\
63.17	0.01\\
63.18	0.01\\
63.19	0.01\\
63.2	0.01\\
63.21	0.01\\
63.22	0.01\\
63.23	0.01\\
63.24	0.01\\
63.25	0.01\\
63.26	0.01\\
63.27	0.01\\
63.28	0.01\\
63.29	0.01\\
63.3	0.01\\
63.31	0.01\\
63.32	0.01\\
63.33	0.01\\
63.34	0.01\\
63.35	0.01\\
63.36	0.01\\
63.37	0.01\\
63.38	0.01\\
63.39	0.01\\
63.4	0.01\\
63.41	0.01\\
63.42	0.01\\
63.43	0.01\\
63.44	0.01\\
63.45	0.01\\
63.46	0.01\\
63.47	0.01\\
63.48	0.01\\
63.49	0.01\\
63.5	0.01\\
63.51	0.01\\
63.52	0.01\\
63.53	0.01\\
63.54	0.01\\
63.55	0.01\\
63.56	0.01\\
63.57	0.01\\
63.58	0.01\\
63.59	0.01\\
63.6	0.01\\
63.61	0.01\\
63.62	0.01\\
63.63	0.01\\
63.64	0.01\\
63.65	0.01\\
63.66	0.01\\
63.67	0.01\\
63.68	0.01\\
63.69	0.01\\
63.7	0.01\\
63.71	0.01\\
63.72	0.01\\
63.73	0.01\\
63.74	0.01\\
63.75	0.01\\
63.76	0.01\\
63.77	0.01\\
63.78	0.01\\
63.79	0.01\\
63.8	0.01\\
63.81	0.01\\
63.82	0.01\\
63.83	0.01\\
63.84	0.01\\
63.85	0.01\\
63.86	0.01\\
63.87	0.01\\
63.88	0.01\\
63.89	0.01\\
63.9	0.01\\
63.91	0.01\\
63.92	0.01\\
63.93	0.01\\
63.94	0.01\\
63.95	0.01\\
63.96	0.01\\
63.97	0.01\\
63.98	0.01\\
63.99	0.01\\
64	0.01\\
64.01	0.01\\
64.02	0.01\\
64.03	0.01\\
64.04	0.01\\
64.05	0.01\\
64.06	0.01\\
64.07	0.01\\
64.08	0.01\\
64.09	0.01\\
64.1	0.01\\
64.11	0.01\\
64.12	0.01\\
64.13	0.01\\
64.14	0.01\\
64.15	0.01\\
64.16	0.01\\
64.17	0.01\\
64.18	0.01\\
64.19	0.01\\
64.2	0.01\\
64.21	0.01\\
64.22	0.01\\
64.23	0.01\\
64.24	0.01\\
64.25	0.01\\
64.26	0.01\\
64.27	0.01\\
64.28	0.01\\
64.29	0.01\\
64.3	0.01\\
64.31	0.01\\
64.32	0.01\\
64.33	0.01\\
64.34	0.01\\
64.35	0.01\\
64.36	0.01\\
64.37	0.01\\
64.38	0.01\\
64.39	0.01\\
64.4	0.01\\
64.41	0.01\\
64.42	0.01\\
64.43	0.01\\
64.44	0.01\\
64.45	0.01\\
64.46	0.01\\
64.47	0.01\\
64.48	0.01\\
64.49	0.01\\
64.5	0.01\\
64.51	0.01\\
64.52	0.01\\
64.53	0.01\\
64.54	0.01\\
64.55	0.01\\
64.56	0.01\\
64.57	0.01\\
64.58	0.01\\
64.59	0.01\\
64.6	0.01\\
64.61	0.01\\
64.62	0.01\\
64.63	0.01\\
64.64	0.01\\
64.65	0.01\\
64.66	0.01\\
64.67	0.01\\
64.68	0.01\\
64.69	0.01\\
64.7	0.01\\
64.71	0.01\\
64.72	0.01\\
64.73	0.01\\
64.74	0.01\\
64.75	0.01\\
64.76	0.01\\
64.77	0.01\\
64.78	0.01\\
64.79	0.01\\
64.8	0.01\\
64.81	0.01\\
64.82	0.01\\
64.83	0.01\\
64.84	0.01\\
64.85	0.01\\
64.86	0.01\\
64.87	0.01\\
64.88	0.01\\
64.89	0.01\\
64.9	0.01\\
64.91	0.01\\
64.92	0.01\\
64.93	0.01\\
64.94	0.01\\
64.95	0.01\\
64.96	0.01\\
64.97	0.01\\
64.98	0.01\\
64.99	0.01\\
65	0.01\\
65.01	0.01\\
65.02	0.01\\
65.03	0.01\\
65.04	0.01\\
65.05	0.01\\
65.06	0.01\\
65.07	0.01\\
65.08	0.01\\
65.09	0.01\\
65.1	0.01\\
65.11	0.01\\
65.12	0.01\\
65.13	0.01\\
65.14	0.01\\
65.15	0.01\\
65.16	0.01\\
65.17	0.01\\
65.18	0.01\\
65.19	0.01\\
65.2	0.01\\
65.21	0.01\\
65.22	0.01\\
65.23	0.01\\
65.24	0.01\\
65.25	0.01\\
65.26	0.01\\
65.27	0.01\\
65.28	0.01\\
65.29	0.01\\
65.3	0.01\\
65.31	0.01\\
65.32	0.01\\
65.33	0.01\\
65.34	0.01\\
65.35	0.01\\
65.36	0.01\\
65.37	0.01\\
65.38	0.01\\
65.39	0.01\\
65.4	0.01\\
65.41	0.01\\
65.42	0.01\\
65.43	0.01\\
65.44	0.01\\
65.45	0.01\\
65.46	0.01\\
65.47	0.01\\
65.48	0.01\\
65.49	0.01\\
65.5	0.01\\
65.51	0.01\\
65.52	0.01\\
65.53	0.01\\
65.54	0.01\\
65.55	0.01\\
65.56	0.01\\
65.57	0.01\\
65.58	0.01\\
65.59	0.01\\
65.6	0.01\\
65.61	0.01\\
65.62	0.01\\
65.63	0.01\\
65.64	0.01\\
65.65	0.01\\
65.66	0.01\\
65.67	0.01\\
65.68	0.01\\
65.69	0.01\\
65.7	0.01\\
65.71	0.01\\
65.72	0.01\\
65.73	0.01\\
65.74	0.01\\
65.75	0.01\\
65.76	0.01\\
65.77	0.01\\
65.78	0.01\\
65.79	0.01\\
65.8	0.01\\
65.81	0.01\\
65.82	0.01\\
65.83	0.01\\
65.84	0.01\\
65.85	0.01\\
65.86	0.01\\
65.87	0.01\\
65.88	0.01\\
65.89	0.01\\
65.9	0.01\\
65.91	0.01\\
65.92	0.01\\
65.93	0.01\\
65.94	0.01\\
65.95	0.01\\
65.96	0.01\\
65.97	0.01\\
65.98	0.01\\
65.99	0.01\\
66	0.01\\
66.01	0.01\\
66.02	0.01\\
66.03	0.01\\
66.04	0.01\\
66.05	0.01\\
66.06	0.01\\
66.07	0.01\\
66.08	0.01\\
66.09	0.01\\
66.1	0.01\\
66.11	0.01\\
66.12	0.01\\
66.13	0.01\\
66.14	0.01\\
66.15	0.01\\
66.16	0.01\\
66.17	0.01\\
66.18	0.01\\
66.19	0.01\\
66.2	0.01\\
66.21	0.01\\
66.22	0.01\\
66.23	0.01\\
66.24	0.01\\
66.25	0.01\\
66.26	0.01\\
66.27	0.01\\
66.28	0.01\\
66.29	0.01\\
66.3	0.01\\
66.31	0.01\\
66.32	0.01\\
66.33	0.01\\
66.34	0.01\\
66.35	0.01\\
66.36	0.01\\
66.37	0.01\\
66.38	0.01\\
66.39	0.01\\
66.4	0.01\\
66.41	0.01\\
66.42	0.01\\
66.43	0.01\\
66.44	0.01\\
66.45	0.01\\
66.46	0.01\\
66.47	0.01\\
66.48	0.01\\
66.49	0.01\\
66.5	0.01\\
66.51	0.01\\
66.52	0.01\\
66.53	0.01\\
66.54	0.01\\
66.55	0.01\\
66.56	0.01\\
66.57	0.01\\
66.58	0.01\\
66.59	0.01\\
66.6	0.01\\
66.61	0.01\\
66.62	0.01\\
66.63	0.01\\
66.64	0.01\\
66.65	0.01\\
66.66	0.01\\
66.67	0.01\\
66.68	0.01\\
66.69	0.01\\
66.7	0.01\\
66.71	0.01\\
66.72	0.01\\
66.73	0.01\\
66.74	0.01\\
66.75	0.01\\
66.76	0.01\\
66.77	0.01\\
66.78	0.01\\
66.79	0.01\\
66.8	0.01\\
66.81	0.01\\
66.82	0.01\\
66.83	0.01\\
66.84	0.01\\
66.85	0.01\\
66.86	0.01\\
66.87	0.01\\
66.88	0.01\\
66.89	0.01\\
66.9	0.01\\
66.91	0.01\\
66.92	0.01\\
66.93	0.01\\
66.94	0.01\\
66.95	0.01\\
66.96	0.01\\
66.97	0.01\\
66.98	0.01\\
66.99	0.01\\
67	0.01\\
67.01	0.01\\
67.02	0.01\\
67.03	0.01\\
67.04	0.01\\
67.05	0.01\\
67.06	0.01\\
67.07	0.01\\
67.08	0.01\\
67.09	0.01\\
67.1	0.01\\
67.11	0.01\\
67.12	0.01\\
67.13	0.01\\
67.14	0.01\\
67.15	0.01\\
67.16	0.01\\
67.17	0.01\\
67.18	0.01\\
67.19	0.01\\
67.2	0.01\\
67.21	0.01\\
67.22	0.01\\
67.23	0.01\\
67.24	0.01\\
67.25	0.01\\
67.26	0.01\\
67.27	0.01\\
67.28	0.01\\
67.29	0.01\\
67.3	0.01\\
67.31	0.01\\
67.32	0.01\\
67.33	0.01\\
67.34	0.01\\
67.35	0.01\\
67.36	0.01\\
67.37	0.01\\
67.38	0.01\\
67.39	0.01\\
67.4	0.01\\
67.41	0.01\\
67.42	0.01\\
67.43	0.01\\
67.44	0.01\\
67.45	0.01\\
67.46	0.01\\
67.47	0.01\\
67.48	0.01\\
67.49	0.01\\
67.5	0.01\\
67.51	0.01\\
67.52	0.01\\
67.53	0.01\\
67.54	0.01\\
67.55	0.01\\
67.56	0.01\\
67.57	0.01\\
67.58	0.01\\
67.59	0.01\\
67.6	0.01\\
67.61	0.01\\
67.62	0.01\\
67.63	0.01\\
67.64	0.01\\
67.65	0.01\\
67.66	0.01\\
67.67	0.01\\
67.68	0.01\\
67.69	0.01\\
67.7	0.01\\
67.71	0.01\\
67.72	0.01\\
67.73	0.01\\
67.74	0.01\\
67.75	0.01\\
67.76	0.01\\
67.77	0.01\\
67.78	0.01\\
67.79	0.01\\
67.8	0.01\\
67.81	0.01\\
67.82	0.01\\
67.83	0.01\\
67.84	0.01\\
67.85	0.01\\
67.86	0.01\\
67.87	0.01\\
67.88	0.01\\
67.89	0.01\\
67.9	0.01\\
67.91	0.01\\
67.92	0.01\\
67.93	0.01\\
67.94	0.01\\
67.95	0.01\\
67.96	0.01\\
67.97	0.01\\
67.98	0.01\\
67.99	0.01\\
68	0.01\\
68.01	0.01\\
68.02	0.01\\
68.03	0.01\\
68.04	0.01\\
68.05	0.01\\
68.06	0.01\\
68.07	0.01\\
68.08	0.01\\
68.09	0.01\\
68.1	0.01\\
68.11	0.01\\
68.12	0.01\\
68.13	0.01\\
68.14	0.01\\
68.15	0.01\\
68.16	0.01\\
68.17	0.01\\
68.18	0.01\\
68.19	0.01\\
68.2	0.01\\
68.21	0.01\\
68.22	0.01\\
68.23	0.01\\
68.24	0.01\\
68.25	0.01\\
68.26	0.01\\
68.27	0.01\\
68.28	0.01\\
68.29	0.01\\
68.3	0.01\\
68.31	0.01\\
68.32	0.01\\
68.33	0.01\\
68.34	0.01\\
68.35	0.01\\
68.36	0.01\\
68.37	0.01\\
68.38	0.01\\
68.39	0.01\\
68.4	0.01\\
68.41	0.01\\
68.42	0.01\\
68.43	0.01\\
68.44	0.01\\
68.45	0.01\\
68.46	0.01\\
68.47	0.01\\
68.48	0.01\\
68.49	0.01\\
68.5	0.01\\
68.51	0.01\\
68.52	0.01\\
68.53	0.01\\
68.54	0.01\\
68.55	0.01\\
68.56	0.01\\
68.57	0.01\\
68.58	0.01\\
68.59	0.01\\
68.6	0.01\\
68.61	0.01\\
68.62	0.01\\
68.63	0.01\\
68.64	0.01\\
68.65	0.01\\
68.66	0.01\\
68.67	0.01\\
68.68	0.01\\
68.69	0.01\\
68.7	0.01\\
68.71	0.01\\
68.72	0.01\\
68.73	0.01\\
68.74	0.01\\
68.75	0.01\\
68.76	0.01\\
68.77	0.01\\
68.78	0.01\\
68.79	0.01\\
68.8	0.01\\
68.81	0.01\\
68.82	0.01\\
68.83	0.01\\
68.84	0.01\\
68.85	0.01\\
68.86	0.01\\
68.87	0.01\\
68.88	0.01\\
68.89	0.01\\
68.9	0.01\\
68.91	0.01\\
68.92	0.01\\
68.93	0.01\\
68.94	0.01\\
68.95	0.01\\
68.96	0.01\\
68.97	0.01\\
68.98	0.01\\
68.99	0.01\\
69	0.01\\
69.01	0.01\\
69.02	0.01\\
69.03	0.01\\
69.04	0.01\\
69.05	0.01\\
69.06	0.01\\
69.07	0.01\\
69.08	0.01\\
69.09	0.01\\
69.1	0.01\\
69.11	0.01\\
69.12	0.01\\
69.13	0.01\\
69.14	0.01\\
69.15	0.01\\
69.16	0.01\\
69.17	0.01\\
69.18	0.01\\
69.19	0.01\\
69.2	0.01\\
69.21	0.01\\
69.22	0.01\\
69.23	0.01\\
69.24	0.01\\
69.25	0.01\\
69.26	0.01\\
69.27	0.01\\
69.28	0.01\\
69.29	0.01\\
69.3	0.01\\
69.31	0.01\\
69.32	0.01\\
69.33	0.01\\
69.34	0.01\\
69.35	0.01\\
69.36	0.01\\
69.37	0.01\\
69.38	0.01\\
69.39	0.01\\
69.4	0.01\\
69.41	0.01\\
69.42	0.01\\
69.43	0.01\\
69.44	0.01\\
69.45	0.01\\
69.46	0.01\\
69.47	0.01\\
69.48	0.01\\
69.49	0.01\\
69.5	0.01\\
69.51	0.01\\
69.52	0.01\\
69.53	0.01\\
69.54	0.01\\
69.55	0.01\\
69.56	0.01\\
69.57	0.01\\
69.58	0.01\\
69.59	0.01\\
69.6	0.01\\
69.61	0.01\\
69.62	0.01\\
69.63	0.01\\
69.64	0.01\\
69.65	0.01\\
69.66	0.01\\
69.67	0.01\\
69.68	0.01\\
69.69	0.01\\
69.7	0.01\\
69.71	0.01\\
69.72	0.01\\
69.73	0.01\\
69.74	0.01\\
69.75	0.01\\
69.76	0.01\\
69.77	0.01\\
69.78	0.01\\
69.79	0.01\\
69.8	0.01\\
69.81	0.01\\
69.82	0.01\\
69.83	0.01\\
69.84	0.01\\
69.85	0.01\\
69.86	0.01\\
69.87	0.01\\
69.88	0.01\\
69.89	0.01\\
69.9	0.01\\
69.91	0.01\\
69.92	0.01\\
69.93	0.01\\
69.94	0.01\\
69.95	0.01\\
69.96	0.01\\
69.97	0.01\\
69.98	0.01\\
69.99	0.01\\
70	0.01\\
70.01	0.01\\
70.02	0.01\\
70.03	0.01\\
70.04	0.01\\
70.05	0.01\\
70.06	0.01\\
70.07	0.01\\
70.08	0.01\\
70.09	0.01\\
70.1	0.01\\
70.11	0.01\\
70.12	0.01\\
70.13	0.01\\
70.14	0.01\\
70.15	0.01\\
70.16	0.01\\
70.17	0.01\\
70.18	0.01\\
70.19	0.01\\
70.2	0.01\\
70.21	0.01\\
70.22	0.01\\
70.23	0.01\\
70.24	0.01\\
70.25	0.01\\
70.26	0.01\\
70.27	0.01\\
70.28	0.01\\
70.29	0.01\\
70.3	0.01\\
70.31	0.01\\
70.32	0.01\\
70.33	0.01\\
70.34	0.01\\
70.35	0.01\\
70.36	0.01\\
70.37	0.01\\
70.38	0.01\\
70.39	0.01\\
70.4	0.01\\
70.41	0.01\\
70.42	0.01\\
70.43	0.01\\
70.44	0.01\\
70.45	0.01\\
70.46	0.01\\
70.47	0.01\\
70.48	0.01\\
70.49	0.01\\
70.5	0.01\\
70.51	0.01\\
70.52	0.01\\
70.53	0.01\\
70.54	0.01\\
70.55	0.01\\
70.56	0.01\\
70.57	0.01\\
70.58	0.01\\
70.59	0.01\\
70.6	0.01\\
70.61	0.01\\
70.62	0.01\\
70.63	0.01\\
70.64	0.01\\
70.65	0.01\\
70.66	0.01\\
70.67	0.01\\
70.68	0.01\\
70.69	0.01\\
70.7	0.01\\
70.71	0.01\\
70.72	0.01\\
70.73	0.01\\
70.74	0.01\\
70.75	0.01\\
70.76	0.01\\
70.77	0.01\\
70.78	0.01\\
70.79	0.01\\
70.8	0.01\\
70.81	0.01\\
70.82	0.01\\
70.83	0.01\\
70.84	0.01\\
70.85	0.01\\
70.86	0.01\\
70.87	0.01\\
70.88	0.01\\
70.89	0.01\\
70.9	0.01\\
70.91	0.01\\
70.92	0.01\\
70.93	0.01\\
70.94	0.01\\
70.95	0.01\\
70.96	0.01\\
70.97	0.01\\
70.98	0.01\\
70.99	0.01\\
71	0.01\\
71.01	0.01\\
71.02	0.01\\
71.03	0.01\\
71.04	0.01\\
71.05	0.01\\
71.06	0.01\\
71.07	0.01\\
71.08	0.01\\
71.09	0.01\\
71.1	0.01\\
71.11	0.01\\
71.12	0.01\\
71.13	0.01\\
71.14	0.01\\
71.15	0.01\\
71.16	0.01\\
71.17	0.01\\
71.18	0.01\\
71.19	0.01\\
71.2	0.01\\
71.21	0.01\\
71.22	0.01\\
71.23	0.01\\
71.24	0.01\\
71.25	0.01\\
71.26	0.01\\
71.27	0.01\\
71.28	0.01\\
71.29	0.01\\
71.3	0.01\\
71.31	0.01\\
71.32	0.01\\
71.33	0.01\\
71.34	0.01\\
71.35	0.01\\
71.36	0.01\\
71.37	0.01\\
71.38	0.01\\
71.39	0.01\\
71.4	0.01\\
71.41	0.01\\
71.42	0.01\\
71.43	0.01\\
71.44	0.01\\
71.45	0.01\\
71.46	0.01\\
71.47	0.01\\
71.48	0.01\\
71.49	0.01\\
71.5	0.01\\
71.51	0.01\\
71.52	0.01\\
71.53	0.01\\
71.54	0.01\\
71.55	0.01\\
71.56	0.01\\
71.57	0.01\\
71.58	0.01\\
71.59	0.01\\
71.6	0.01\\
71.61	0.01\\
71.62	0.01\\
71.63	0.01\\
71.64	0.01\\
71.65	0.01\\
71.66	0.01\\
71.67	0.01\\
71.68	0.01\\
71.69	0.01\\
71.7	0.01\\
71.71	0.01\\
71.72	0.01\\
71.73	0.01\\
71.74	0.01\\
71.75	0.01\\
71.76	0.01\\
71.77	0.01\\
71.78	0.01\\
71.79	0.01\\
71.8	0.01\\
71.81	0.01\\
71.82	0.01\\
71.83	0.01\\
71.84	0.01\\
71.85	0.01\\
71.86	0.01\\
71.87	0.01\\
71.88	0.01\\
71.89	0.01\\
71.9	0.01\\
71.91	0.01\\
71.92	0.01\\
71.93	0.01\\
71.94	0.01\\
71.95	0.01\\
71.96	0.01\\
71.97	0.01\\
71.98	0.01\\
71.99	0.01\\
72	0.01\\
72.01	0.01\\
72.02	0.01\\
72.03	0.01\\
72.04	0.01\\
72.05	0.01\\
72.06	0.01\\
72.07	0.01\\
72.08	0.01\\
72.09	0.01\\
72.1	0.01\\
72.11	0.01\\
72.12	0.01\\
72.13	0.01\\
72.14	0.01\\
72.15	0.01\\
72.16	0.01\\
72.17	0.01\\
72.18	0.01\\
72.19	0.01\\
72.2	0.01\\
72.21	0.01\\
72.22	0.01\\
72.23	0.01\\
72.24	0.01\\
72.25	0.01\\
72.26	0.01\\
72.27	0.01\\
72.28	0.01\\
72.29	0.01\\
72.3	0.01\\
72.31	0.01\\
72.32	0.01\\
72.33	0.01\\
72.34	0.01\\
72.35	0.01\\
72.36	0.01\\
72.37	0.01\\
72.38	0.01\\
72.39	0.01\\
72.4	0.01\\
72.41	0.01\\
72.42	0.01\\
72.43	0.01\\
72.44	0.01\\
72.45	0.01\\
72.46	0.01\\
72.47	0.01\\
72.48	0.01\\
72.49	0.01\\
72.5	0.01\\
72.51	0.01\\
72.52	0.01\\
72.53	0.01\\
72.54	0.01\\
72.55	0.01\\
72.56	0.01\\
72.57	0.01\\
72.58	0.01\\
72.59	0.01\\
72.6	0.01\\
72.61	0.01\\
72.62	0.01\\
72.63	0.01\\
72.64	0.01\\
72.65	0.01\\
72.66	0.01\\
72.67	0.01\\
72.68	0.01\\
72.69	0.01\\
72.7	0.01\\
72.71	0.01\\
72.72	0.01\\
72.73	0.01\\
72.74	0.01\\
72.75	0.01\\
72.76	0.01\\
72.77	0.01\\
72.78	0.01\\
72.79	0.01\\
72.8	0.01\\
72.81	0.01\\
72.82	0.01\\
72.83	0.01\\
72.84	0.01\\
72.85	0.01\\
72.86	0.01\\
72.87	0.01\\
72.88	0.01\\
72.89	0.01\\
72.9	0.01\\
72.91	0.01\\
72.92	0.01\\
72.93	0.01\\
72.94	0.01\\
72.95	0.01\\
72.96	0.01\\
72.97	0.01\\
72.98	0.01\\
72.99	0.01\\
73	0.01\\
73.01	0.01\\
73.02	0.01\\
73.03	0.01\\
73.04	0.01\\
73.05	0.01\\
73.06	0.01\\
73.07	0.01\\
73.08	0.01\\
73.09	0.01\\
73.1	0.01\\
73.11	0.01\\
73.12	0.01\\
73.13	0.01\\
73.14	0.01\\
73.15	0.01\\
73.16	0.01\\
73.17	0.01\\
73.18	0.01\\
73.19	0.01\\
73.2	0.01\\
73.21	0.01\\
73.22	0.01\\
73.23	0.01\\
73.24	0.01\\
73.25	0.01\\
73.26	0.01\\
73.27	0.01\\
73.28	0.01\\
73.29	0.01\\
73.3	0.01\\
73.31	0.01\\
73.32	0.01\\
73.33	0.01\\
73.34	0.01\\
73.35	0.01\\
73.36	0.01\\
73.37	0.01\\
73.38	0.01\\
73.39	0.01\\
73.4	0.01\\
73.41	0.01\\
73.42	0.01\\
73.43	0.01\\
73.44	0.01\\
73.45	0.01\\
73.46	0.01\\
73.47	0.01\\
73.48	0.01\\
73.49	0.01\\
73.5	0.01\\
73.51	0.01\\
73.52	0.01\\
73.53	0.01\\
73.54	0.01\\
73.55	0.01\\
73.56	0.01\\
73.57	0.01\\
73.58	0.01\\
73.59	0.01\\
73.6	0.01\\
73.61	0.01\\
73.62	0.01\\
73.63	0.01\\
73.64	0.01\\
73.65	0.01\\
73.66	0.01\\
73.67	0.01\\
73.68	0.01\\
73.69	0.01\\
73.7	0.01\\
73.71	0.01\\
73.72	0.01\\
73.73	0.01\\
73.74	0.01\\
73.75	0.01\\
73.76	0.01\\
73.77	0.01\\
73.78	0.01\\
73.79	0.01\\
73.8	0.01\\
73.81	0.01\\
73.82	0.01\\
73.83	0.01\\
73.84	0.01\\
73.85	0.01\\
73.86	0.01\\
73.87	0.01\\
73.88	0.01\\
73.89	0.01\\
73.9	0.01\\
73.91	0.01\\
73.92	0.01\\
73.93	0.01\\
73.94	0.01\\
73.95	0.01\\
73.96	0.01\\
73.97	0.01\\
73.98	0.01\\
73.99	0.01\\
74	0.01\\
74.01	0.01\\
74.02	0.01\\
74.03	0.01\\
74.04	0.01\\
74.05	0.01\\
74.06	0.01\\
74.07	0.01\\
74.08	0.01\\
74.09	0.01\\
74.1	0.01\\
74.11	0.01\\
74.12	0.01\\
74.13	0.01\\
74.14	0.01\\
74.15	0.01\\
74.16	0.01\\
74.17	0.01\\
74.18	0.01\\
74.19	0.01\\
74.2	0.01\\
74.21	0.01\\
74.22	0.01\\
74.23	0.01\\
74.24	0.01\\
74.25	0.01\\
74.26	0.01\\
74.27	0.01\\
74.28	0.01\\
74.29	0.01\\
74.3	0.01\\
74.31	0.01\\
74.32	0.01\\
74.33	0.01\\
74.34	0.01\\
74.35	0.01\\
74.36	0.01\\
74.37	0.01\\
74.38	0.01\\
74.39	0.01\\
74.4	0.01\\
74.41	0.01\\
74.42	0.01\\
74.43	0.01\\
74.44	0.01\\
74.45	0.01\\
74.46	0.01\\
74.47	0.01\\
74.48	0.01\\
74.49	0.01\\
74.5	0.01\\
74.51	0.01\\
74.52	0.01\\
74.53	0.01\\
74.54	0.01\\
74.55	0.01\\
74.56	0.01\\
74.57	0.01\\
74.58	0.01\\
74.59	0.01\\
74.6	0.01\\
74.61	0.01\\
74.62	0.01\\
74.63	0.01\\
74.64	0.01\\
74.65	0.01\\
74.66	0.01\\
74.67	0.01\\
74.68	0.01\\
74.69	0.01\\
74.7	0.01\\
74.71	0.01\\
74.72	0.01\\
74.73	0.01\\
74.74	0.01\\
74.75	0.01\\
74.76	0.01\\
74.77	0.01\\
74.78	0.01\\
74.79	0.01\\
74.8	0.01\\
74.81	0.01\\
74.82	0.01\\
74.83	0.01\\
74.84	0.01\\
74.85	0.01\\
74.86	0.01\\
74.87	0.01\\
74.88	0.01\\
74.89	0.01\\
74.9	0.01\\
74.91	0.01\\
74.92	0.01\\
74.93	0.01\\
74.94	0.01\\
74.95	0.01\\
74.96	0.01\\
74.97	0.01\\
74.98	0.01\\
74.99	0.01\\
75	0.01\\
75.01	0.01\\
75.02	0.01\\
75.03	0.01\\
75.04	0.01\\
75.05	0.01\\
75.06	0.01\\
75.07	0.01\\
75.08	0.01\\
75.09	0.01\\
75.1	0.01\\
75.11	0.01\\
75.12	0.01\\
75.13	0.01\\
75.14	0.01\\
75.15	0.01\\
75.16	0.01\\
75.17	0.01\\
75.18	0.01\\
75.19	0.01\\
75.2	0.01\\
75.21	0.01\\
75.22	0.01\\
75.23	0.01\\
75.24	0.01\\
75.25	0.01\\
75.26	0.01\\
75.27	0.01\\
75.28	0.01\\
75.29	0.01\\
75.3	0.01\\
75.31	0.01\\
75.32	0.01\\
75.33	0.01\\
75.34	0.01\\
75.35	0.01\\
75.36	0.01\\
75.37	0.01\\
75.38	0.01\\
75.39	0.01\\
75.4	0.01\\
75.41	0.01\\
75.42	0.01\\
75.43	0.01\\
75.44	0.01\\
75.45	0.01\\
75.46	0.01\\
75.47	0.01\\
75.48	0.01\\
75.49	0.01\\
75.5	0.01\\
75.51	0.01\\
75.52	0.01\\
75.53	0.01\\
75.54	0.01\\
75.55	0.01\\
75.56	0.01\\
75.57	0.01\\
75.58	0.01\\
75.59	0.01\\
75.6	0.01\\
75.61	0.01\\
75.62	0.01\\
75.63	0.01\\
75.64	0.01\\
75.65	0.01\\
75.66	0.01\\
75.67	0.01\\
75.68	0.01\\
75.69	0.01\\
75.7	0.01\\
75.71	0.01\\
75.72	0.01\\
75.73	0.01\\
75.74	0.01\\
75.75	0.01\\
75.76	0.01\\
75.77	0.01\\
75.78	0.01\\
75.79	0.01\\
75.8	0.01\\
75.81	0.01\\
75.82	0.01\\
75.83	0.01\\
75.84	0.01\\
75.85	0.01\\
75.86	0.01\\
75.87	0.01\\
75.88	0.01\\
75.89	0.01\\
75.9	0.01\\
75.91	0.01\\
75.92	0.01\\
75.93	0.01\\
75.94	0.01\\
75.95	0.01\\
75.96	0.01\\
75.97	0.01\\
75.98	0.01\\
75.99	0.01\\
76	0.01\\
76.01	0.01\\
76.02	0.01\\
76.03	0.01\\
76.04	0.01\\
76.05	0.01\\
76.06	0.01\\
76.07	0.01\\
76.08	0.01\\
76.09	0.01\\
76.1	0.01\\
76.11	0.01\\
76.12	0.01\\
76.13	0.01\\
76.14	0.01\\
76.15	0.01\\
76.16	0.01\\
76.17	0.01\\
76.18	0.01\\
76.19	0.01\\
76.2	0.01\\
76.21	0.01\\
76.22	0.01\\
76.23	0.01\\
76.24	0.01\\
76.25	0.01\\
76.26	0.01\\
76.27	0.01\\
76.28	0.01\\
76.29	0.01\\
76.3	0.01\\
76.31	0.01\\
76.32	0.01\\
76.33	0.01\\
76.34	0.01\\
76.35	0.01\\
76.36	0.01\\
76.37	0.01\\
76.38	0.01\\
76.39	0.01\\
76.4	0.01\\
76.41	0.01\\
76.42	0.01\\
76.43	0.01\\
76.44	0.01\\
76.45	0.01\\
76.46	0.01\\
76.47	0.01\\
76.48	0.01\\
76.49	0.01\\
76.5	0.01\\
76.51	0.01\\
76.52	0.01\\
76.53	0.01\\
76.54	0.01\\
76.55	0.01\\
76.56	0.01\\
76.57	0.01\\
76.58	0.01\\
76.59	0.01\\
76.6	0.01\\
76.61	0.01\\
76.62	0.01\\
76.63	0.01\\
76.64	0.01\\
76.65	0.01\\
76.66	0.01\\
76.67	0.01\\
76.68	0.01\\
76.69	0.01\\
76.7	0.01\\
76.71	0.01\\
76.72	0.01\\
76.73	0.01\\
76.74	0.01\\
76.75	0.01\\
76.76	0.01\\
76.77	0.01\\
76.78	0.01\\
76.79	0.01\\
76.8	0.01\\
76.81	0.01\\
76.82	0.01\\
76.83	0.01\\
76.84	0.01\\
76.85	0.01\\
76.86	0.01\\
76.87	0.01\\
76.88	0.01\\
76.89	0.01\\
76.9	0.01\\
76.91	0.01\\
76.92	0.01\\
76.93	0.01\\
76.94	0.01\\
76.95	0.01\\
76.96	0.01\\
76.97	0.01\\
76.98	0.01\\
76.99	0.01\\
77	0.01\\
77.01	0.01\\
77.02	0.01\\
77.03	0.01\\
77.04	0.01\\
77.05	0.01\\
77.06	0.01\\
77.07	0.01\\
77.08	0.01\\
77.09	0.01\\
77.1	0.01\\
77.11	0.01\\
77.12	0.01\\
77.13	0.01\\
77.14	0.01\\
77.15	0.01\\
77.16	0.01\\
77.17	0.01\\
77.18	0.01\\
77.19	0.01\\
77.2	0.01\\
77.21	0.01\\
77.22	0.01\\
77.23	0.01\\
77.24	0.01\\
77.25	0.01\\
77.26	0.01\\
77.27	0.01\\
77.28	0.01\\
77.29	0.01\\
77.3	0.01\\
77.31	0.01\\
77.32	0.01\\
77.33	0.01\\
77.34	0.01\\
77.35	0.01\\
77.36	0.01\\
77.37	0.01\\
77.38	0.01\\
77.39	0.01\\
77.4	0.01\\
77.41	0.01\\
77.42	0.01\\
77.43	0.01\\
77.44	0.01\\
77.45	0.01\\
77.46	0.01\\
77.47	0.01\\
77.48	0.01\\
77.49	0.01\\
77.5	0.01\\
77.51	0.01\\
77.52	0.01\\
77.53	0.01\\
77.54	0.01\\
77.55	0.01\\
77.56	0.01\\
77.57	0.01\\
77.58	0.01\\
77.59	0.01\\
77.6	0.01\\
77.61	0.01\\
77.62	0.01\\
77.63	0.01\\
77.64	0.01\\
77.65	0.01\\
77.66	0.01\\
77.67	0.01\\
77.68	0.01\\
77.69	0.01\\
77.7	0.01\\
77.71	0.01\\
77.72	0.01\\
77.73	0.01\\
77.74	0.01\\
77.75	0.01\\
77.76	0.01\\
77.77	0.01\\
77.78	0.01\\
77.79	0.01\\
77.8	0.01\\
77.81	0.01\\
77.82	0.01\\
77.83	0.01\\
77.84	0.01\\
77.85	0.01\\
77.86	0.01\\
77.87	0.01\\
77.88	0.01\\
77.89	0.01\\
77.9	0.01\\
77.91	0.01\\
77.92	0.01\\
77.93	0.01\\
77.94	0.01\\
77.95	0.01\\
77.96	0.01\\
77.97	0.01\\
77.98	0.01\\
77.99	0.01\\
78	0.01\\
78.01	0.01\\
78.02	0.01\\
78.03	0.01\\
78.04	0.01\\
78.05	0.01\\
78.06	0.01\\
78.07	0.01\\
78.08	0.01\\
78.09	0.01\\
78.1	0.01\\
78.11	0.01\\
78.12	0.01\\
78.13	0.01\\
78.14	0.01\\
78.15	0.01\\
78.16	0.01\\
78.17	0.01\\
78.18	0.01\\
78.19	0.01\\
78.2	0.01\\
78.21	0.01\\
78.22	0.01\\
78.23	0.01\\
78.24	0.01\\
78.25	0.01\\
78.26	0.01\\
78.27	0.01\\
78.28	0.01\\
78.29	0.01\\
78.3	0.01\\
78.31	0.01\\
78.32	0.01\\
78.33	0.01\\
78.34	0.01\\
78.35	0.01\\
78.36	0.01\\
78.37	0.01\\
78.38	0.01\\
78.39	0.01\\
78.4	0.01\\
78.41	0.01\\
78.42	0.01\\
78.43	0.01\\
78.44	0.01\\
78.45	0.01\\
78.46	0.01\\
78.47	0.01\\
78.48	0.01\\
78.49	0.01\\
78.5	0.01\\
78.51	0.01\\
78.52	0.01\\
78.53	0.01\\
78.54	0.01\\
78.55	0.01\\
78.56	0.01\\
78.57	0.01\\
78.58	0.01\\
78.59	0.01\\
78.6	0.01\\
78.61	0.01\\
78.62	0.01\\
78.63	0.01\\
78.64	0.01\\
78.65	0.01\\
78.66	0.01\\
78.67	0.01\\
78.68	0.01\\
78.69	0.01\\
78.7	0.01\\
78.71	0.01\\
78.72	0.01\\
78.73	0.01\\
78.74	0.01\\
78.75	0.01\\
78.76	0.01\\
78.77	0.01\\
78.78	0.01\\
78.79	0.01\\
78.8	0.01\\
78.81	0.01\\
78.82	0.01\\
78.83	0.01\\
78.84	0.01\\
78.85	0.01\\
78.86	0.01\\
78.87	0.01\\
78.88	0.01\\
78.89	0.01\\
78.9	0.01\\
78.91	0.01\\
78.92	0.01\\
78.93	0.01\\
78.94	0.01\\
78.95	0.01\\
78.96	0.01\\
78.97	0.01\\
78.98	0.01\\
78.99	0.01\\
79	0.01\\
79.01	0.01\\
79.02	0.01\\
79.03	0.01\\
79.04	0.01\\
79.05	0.01\\
79.06	0.01\\
79.07	0.01\\
79.08	0.01\\
79.09	0.01\\
79.1	0.01\\
79.11	0.01\\
79.12	0.01\\
79.13	0.01\\
79.14	0.01\\
79.15	0.01\\
79.16	0.01\\
79.17	0.01\\
79.18	0.01\\
79.19	0.01\\
79.2	0.01\\
79.21	0.01\\
79.22	0.01\\
79.23	0.01\\
79.24	0.01\\
79.25	0.01\\
79.26	0.01\\
79.27	0.01\\
79.28	0.01\\
79.29	0.01\\
79.3	0.01\\
79.31	0.01\\
79.32	0.01\\
79.33	0.01\\
79.34	0.01\\
79.35	0.01\\
79.36	0.01\\
79.37	0.01\\
79.38	0.01\\
79.39	0.01\\
79.4	0.01\\
79.41	0.01\\
79.42	0.01\\
79.43	0.01\\
79.44	0.01\\
79.45	0.01\\
79.46	0.01\\
79.47	0.01\\
79.48	0.01\\
79.49	0.01\\
79.5	0.01\\
79.51	0.01\\
79.52	0.01\\
79.53	0.01\\
79.54	0.01\\
79.55	0.01\\
79.56	0.01\\
79.57	0.01\\
79.58	0.01\\
79.59	0.01\\
79.6	0.01\\
79.61	0.01\\
79.62	0.01\\
79.63	0.01\\
79.64	0.01\\
79.65	0.01\\
79.66	0.01\\
79.67	0.01\\
79.68	0.01\\
79.69	0.01\\
79.7	0.01\\
79.71	0.01\\
79.72	0.01\\
79.73	0.01\\
79.74	0.01\\
79.75	0.01\\
79.76	0.01\\
79.77	0.01\\
79.78	0.01\\
79.79	0.01\\
79.8	0.01\\
79.81	0.01\\
79.82	0.01\\
79.83	0.01\\
79.84	0.01\\
79.85	0.01\\
79.86	0.01\\
79.87	0.01\\
79.88	0.01\\
79.89	0.01\\
79.9	0.01\\
79.91	0.01\\
79.92	0.01\\
79.93	0.01\\
79.94	0.01\\
79.95	0.01\\
79.96	0.01\\
79.97	0.01\\
79.98	0.01\\
79.99	0.01\\
80	0.01\\
80.01	0.01\\
};
\addplot [color=green,dashed]
  table[row sep=crcr]{%
80.01	0.01\\
80.02	0.01\\
80.03	0.01\\
80.04	0.01\\
80.05	0.01\\
80.06	0.01\\
80.07	0.01\\
80.08	0.01\\
80.09	0.01\\
80.1	0.01\\
80.11	0.01\\
80.12	0.01\\
80.13	0.01\\
80.14	0.01\\
80.15	0.01\\
80.16	0.01\\
80.17	0.01\\
80.18	0.01\\
80.19	0.01\\
80.2	0.01\\
80.21	0.01\\
80.22	0.01\\
80.23	0.01\\
80.24	0.01\\
80.25	0.01\\
80.26	0.01\\
80.27	0.01\\
80.28	0.01\\
80.29	0.01\\
80.3	0.01\\
80.31	0.01\\
80.32	0.01\\
80.33	0.01\\
80.34	0.01\\
80.35	0.01\\
80.36	0.01\\
80.37	0.01\\
80.38	0.01\\
80.39	0.01\\
80.4	0.01\\
80.41	0.01\\
80.42	0.01\\
80.43	0.01\\
80.44	0.01\\
80.45	0.01\\
80.46	0.01\\
80.47	0.01\\
80.48	0.01\\
80.49	0.01\\
80.5	0.01\\
80.51	0.01\\
80.52	0.01\\
80.53	0.01\\
80.54	0.01\\
80.55	0.01\\
80.56	0.01\\
80.57	0.01\\
80.58	0.01\\
80.59	0.01\\
80.6	0.01\\
80.61	0.01\\
80.62	0.01\\
80.63	0.01\\
80.64	0.01\\
80.65	0.01\\
80.66	0.01\\
80.67	0.01\\
80.68	0.01\\
80.69	0.01\\
80.7	0.01\\
80.71	0.01\\
80.72	0.01\\
80.73	0.01\\
80.74	0.01\\
80.75	0.01\\
80.76	0.01\\
80.77	0.01\\
80.78	0.01\\
80.79	0.01\\
80.8	0.01\\
80.81	0.01\\
80.82	0.01\\
80.83	0.01\\
80.84	0.01\\
80.85	0.01\\
80.86	0.01\\
80.87	0.01\\
80.88	0.01\\
80.89	0.01\\
80.9	0.01\\
80.91	0.01\\
80.92	0.01\\
80.93	0.01\\
80.94	0.01\\
80.95	0.01\\
80.96	0.01\\
80.97	0.01\\
80.98	0.01\\
80.99	0.01\\
81	0.01\\
81.01	0.01\\
81.02	0.01\\
81.03	0.01\\
81.04	0.01\\
81.05	0.01\\
81.06	0.01\\
81.07	0.01\\
81.08	0.01\\
81.09	0.01\\
81.1	0.01\\
81.11	0.01\\
81.12	0.01\\
81.13	0.01\\
81.14	0.01\\
81.15	0.01\\
81.16	0.01\\
81.17	0.01\\
81.18	0.01\\
81.19	0.01\\
81.2	0.01\\
81.21	0.01\\
81.22	0.01\\
81.23	0.01\\
81.24	0.01\\
81.25	0.01\\
81.26	0.01\\
81.27	0.01\\
81.28	0.01\\
81.29	0.01\\
81.3	0.01\\
81.31	0.01\\
81.32	0.01\\
81.33	0.01\\
81.34	0.01\\
81.35	0.01\\
81.36	0.01\\
81.37	0.01\\
81.38	0.01\\
81.39	0.01\\
81.4	0.01\\
81.41	0.01\\
81.42	0.01\\
81.43	0.01\\
81.44	0.01\\
81.45	0.01\\
81.46	0.01\\
81.47	0.01\\
81.48	0.01\\
81.49	0.01\\
81.5	0.01\\
81.51	0.01\\
81.52	0.01\\
81.53	0.01\\
81.54	0.01\\
81.55	0.01\\
81.56	0.01\\
81.57	0.01\\
81.58	0.01\\
81.59	0.01\\
81.6	0.01\\
81.61	0.01\\
81.62	0.01\\
81.63	0.01\\
81.64	0.01\\
81.65	0.01\\
81.66	0.01\\
81.67	0.01\\
81.68	0.01\\
81.69	0.01\\
81.7	0.01\\
81.71	0.01\\
81.72	0.01\\
81.73	0.01\\
81.74	0.01\\
81.75	0.01\\
81.76	0.01\\
81.77	0.01\\
81.78	0.01\\
81.79	0.01\\
81.8	0.01\\
81.81	0.01\\
81.82	0.01\\
81.83	0.01\\
81.84	0.01\\
81.85	0.01\\
81.86	0.01\\
81.87	0.01\\
81.88	0.01\\
81.89	0.01\\
81.9	0.01\\
81.91	0.01\\
81.92	0.01\\
81.93	0.01\\
81.94	0.01\\
81.95	0.01\\
81.96	0.01\\
81.97	0.01\\
81.98	0.01\\
81.99	0.01\\
82	0.01\\
82.01	0.01\\
82.02	0.01\\
82.03	0.01\\
82.04	0.01\\
82.05	0.01\\
82.06	0.01\\
82.07	0.01\\
82.08	0.01\\
82.09	0.01\\
82.1	0.01\\
82.11	0.01\\
82.12	0.01\\
82.13	0.01\\
82.14	0.01\\
82.15	0.01\\
82.16	0.01\\
82.17	0.01\\
82.18	0.01\\
82.19	0.01\\
82.2	0.01\\
82.21	0.01\\
82.22	0.01\\
82.23	0.01\\
82.24	0.01\\
82.25	0.01\\
82.26	0.01\\
82.27	0.01\\
82.28	0.01\\
82.29	0.01\\
82.3	0.01\\
82.31	0.01\\
82.32	0.01\\
82.33	0.01\\
82.34	0.01\\
82.35	0.01\\
82.36	0.01\\
82.37	0.01\\
82.38	0.01\\
82.39	0.01\\
82.4	0.01\\
82.41	0.01\\
82.42	0.01\\
82.43	0.01\\
82.44	0.01\\
82.45	0.01\\
82.46	0.01\\
82.47	0.01\\
82.48	0.01\\
82.49	0.01\\
82.5	0.01\\
82.51	0.01\\
82.52	0.01\\
82.53	0.01\\
82.54	0.01\\
82.55	0.01\\
82.56	0.01\\
82.57	0.01\\
82.58	0.01\\
82.59	0.01\\
82.6	0.01\\
82.61	0.01\\
82.62	0.01\\
82.63	0.01\\
82.64	0.01\\
82.65	0.01\\
82.66	0.01\\
82.67	0.01\\
82.68	0.01\\
82.69	0.01\\
82.7	0.01\\
82.71	0.01\\
82.72	0.01\\
82.73	0.01\\
82.74	0.01\\
82.75	0.01\\
82.76	0.01\\
82.77	0.01\\
82.78	0.01\\
82.79	0.01\\
82.8	0.01\\
82.81	0.01\\
82.82	0.01\\
82.83	0.01\\
82.84	0.01\\
82.85	0.01\\
82.86	0.01\\
82.87	0.01\\
82.88	0.01\\
82.89	0.01\\
82.9	0.01\\
82.91	0.01\\
82.92	0.01\\
82.93	0.01\\
82.94	0.01\\
82.95	0.01\\
82.96	0.01\\
82.97	0.01\\
82.98	0.01\\
82.99	0.01\\
83	0.01\\
83.01	0.01\\
83.02	0.01\\
83.03	0.01\\
83.04	0.01\\
83.05	0.01\\
83.06	0.01\\
83.07	0.01\\
83.08	0.01\\
83.09	0.01\\
83.1	0.01\\
83.11	0.01\\
83.12	0.01\\
83.13	0.01\\
83.14	0.01\\
83.15	0.01\\
83.16	0.01\\
83.17	0.01\\
83.18	0.01\\
83.19	0.01\\
83.2	0.01\\
83.21	0.01\\
83.22	0.01\\
83.23	0.01\\
83.24	0.01\\
83.25	0.01\\
83.26	0.01\\
83.27	0.01\\
83.28	0.01\\
83.29	0.01\\
83.3	0.01\\
83.31	0.01\\
83.32	0.01\\
83.33	0.01\\
83.34	0.01\\
83.35	0.01\\
83.36	0.01\\
83.37	0.01\\
83.38	0.01\\
83.39	0.01\\
83.4	0.01\\
83.41	0.01\\
83.42	0.01\\
83.43	0.01\\
83.44	0.01\\
83.45	0.01\\
83.46	0.01\\
83.47	0.01\\
83.48	0.01\\
83.49	0.01\\
83.5	0.01\\
83.51	0.01\\
83.52	0.01\\
83.53	0.01\\
83.54	0.01\\
83.55	0.01\\
83.56	0.01\\
83.57	0.01\\
83.58	0.01\\
83.59	0.01\\
83.6	0.01\\
83.61	0.01\\
83.62	0.01\\
83.63	0.01\\
83.64	0.01\\
83.65	0.01\\
83.66	0.01\\
83.67	0.01\\
83.68	0.01\\
83.69	0.01\\
83.7	0.01\\
83.71	0.01\\
83.72	0.01\\
83.73	0.01\\
83.74	0.01\\
83.75	0.01\\
83.76	0.01\\
83.77	0.01\\
83.78	0.01\\
83.79	0.01\\
83.8	0.01\\
83.81	0.01\\
83.82	0.01\\
83.83	0.01\\
83.84	0.01\\
83.85	0.01\\
83.86	0.01\\
83.87	0.01\\
83.88	0.01\\
83.89	0.01\\
83.9	0.01\\
83.91	0.01\\
83.92	0.01\\
83.93	0.01\\
83.94	0.01\\
83.95	0.01\\
83.96	0.01\\
83.97	0.01\\
83.98	0.01\\
83.99	0.01\\
84	0.01\\
84.01	0.01\\
84.02	0.01\\
84.03	0.01\\
84.04	0.01\\
84.05	0.01\\
84.06	0.01\\
84.07	0.01\\
84.08	0.01\\
84.09	0.01\\
84.1	0.01\\
84.11	0.01\\
84.12	0.01\\
84.13	0.01\\
84.14	0.01\\
84.15	0.01\\
84.16	0.01\\
84.17	0.01\\
84.18	0.01\\
84.19	0.01\\
84.2	0.01\\
84.21	0.01\\
84.22	0.01\\
84.23	0.01\\
84.24	0.01\\
84.25	0.01\\
84.26	0.01\\
84.27	0.01\\
84.28	0.01\\
84.29	0.01\\
84.3	0.01\\
84.31	0.01\\
84.32	0.01\\
84.33	0.01\\
84.34	0.01\\
84.35	0.01\\
84.36	0.01\\
84.37	0.01\\
84.38	0.01\\
84.39	0.01\\
84.4	0.01\\
84.41	0.01\\
84.42	0.01\\
84.43	0.01\\
84.44	0.01\\
84.45	0.01\\
84.46	0.01\\
84.47	0.01\\
84.48	0.01\\
84.49	0.01\\
84.5	0.01\\
84.51	0.01\\
84.52	0.01\\
84.53	0.01\\
84.54	0.01\\
84.55	0.01\\
84.56	0.01\\
84.57	0.01\\
84.58	0.01\\
84.59	0.01\\
84.6	0.01\\
84.61	0.01\\
84.62	0.01\\
84.63	0.01\\
84.64	0.01\\
84.65	0.01\\
84.66	0.01\\
84.67	0.01\\
84.68	0.01\\
84.69	0.01\\
84.7	0.01\\
84.71	0.01\\
84.72	0.01\\
84.73	0.01\\
84.74	0.01\\
84.75	0.01\\
84.76	0.01\\
84.77	0.01\\
84.78	0.01\\
84.79	0.01\\
84.8	0.01\\
84.81	0.01\\
84.82	0.01\\
84.83	0.01\\
84.84	0.01\\
84.85	0.01\\
84.86	0.01\\
84.87	0.01\\
84.88	0.01\\
84.89	0.01\\
84.9	0.01\\
84.91	0.01\\
84.92	0.01\\
84.93	0.01\\
84.94	0.01\\
84.95	0.01\\
84.96	0.01\\
84.97	0.01\\
84.98	0.01\\
84.99	0.01\\
85	0.01\\
85.01	0.01\\
85.02	0.01\\
85.03	0.01\\
85.04	0.01\\
85.05	0.01\\
85.06	0.01\\
85.07	0.01\\
85.08	0.01\\
85.09	0.01\\
85.1	0.01\\
85.11	0.01\\
85.12	0.01\\
85.13	0.01\\
85.14	0.01\\
85.15	0.01\\
85.16	0.01\\
85.17	0.01\\
85.18	0.01\\
85.19	0.01\\
85.2	0.01\\
85.21	0.01\\
85.22	0.01\\
85.23	0.01\\
85.24	0.01\\
85.25	0.01\\
85.26	0.01\\
85.27	0.01\\
85.28	0.01\\
85.29	0.01\\
85.3	0.01\\
85.31	0.01\\
85.32	0.01\\
85.33	0.01\\
85.34	0.01\\
85.35	0.01\\
85.36	0.01\\
85.37	0.01\\
85.38	0.01\\
85.39	0.01\\
85.4	0.01\\
85.41	0.01\\
85.42	0.01\\
85.43	0.01\\
85.44	0.01\\
85.45	0.01\\
85.46	0.01\\
85.47	0.01\\
85.48	0.01\\
85.49	0.01\\
85.5	0.01\\
85.51	0.01\\
85.52	0.01\\
85.53	0.01\\
85.54	0.01\\
85.55	0.01\\
85.56	0.01\\
85.57	0.01\\
85.58	0.01\\
85.59	0.01\\
85.6	0.01\\
85.61	0.01\\
85.62	0.01\\
85.63	0.01\\
85.64	0.01\\
85.65	0.01\\
85.66	0.01\\
85.67	0.01\\
85.68	0.01\\
85.69	0.01\\
85.7	0.01\\
85.71	0.01\\
85.72	0.01\\
85.73	0.01\\
85.74	0.01\\
85.75	0.01\\
85.76	0.01\\
85.77	0.01\\
85.78	0.01\\
85.79	0.01\\
85.8	0.01\\
85.81	0.01\\
85.82	0.01\\
85.83	0.01\\
85.84	0.01\\
85.85	0.01\\
85.86	0.01\\
85.87	0.01\\
85.88	0.01\\
85.89	0.01\\
85.9	0.01\\
85.91	0.01\\
85.92	0.01\\
85.93	0.01\\
85.94	0.01\\
85.95	0.01\\
85.96	0.01\\
85.97	0.01\\
85.98	0.01\\
85.99	0.01\\
86	0.01\\
86.01	0.01\\
86.02	0.01\\
86.03	0.01\\
86.04	0.01\\
86.05	0.01\\
86.06	0.01\\
86.07	0.01\\
86.08	0.01\\
86.09	0.01\\
86.1	0.01\\
86.11	0.01\\
86.12	0.01\\
86.13	0.01\\
86.14	0.01\\
86.15	0.01\\
86.16	0.01\\
86.17	0.01\\
86.18	0.01\\
86.19	0.01\\
86.2	0.01\\
86.21	0.01\\
86.22	0.01\\
86.23	0.01\\
86.24	0.01\\
86.25	0.01\\
86.26	0.01\\
86.27	0.01\\
86.28	0.01\\
86.29	0.01\\
86.3	0.01\\
86.31	0.01\\
86.32	0.01\\
86.33	0.01\\
86.34	0.01\\
86.35	0.01\\
86.36	0.01\\
86.37	0.01\\
86.38	0.01\\
86.39	0.01\\
86.4	0.01\\
86.41	0.01\\
86.42	0.01\\
86.43	0.01\\
86.44	0.01\\
86.45	0.01\\
86.46	0.01\\
86.47	0.01\\
86.48	0.01\\
86.49	0.01\\
86.5	0.01\\
86.51	0.01\\
86.52	0.01\\
86.53	0.01\\
86.54	0.01\\
86.55	0.01\\
86.56	0.01\\
86.57	0.01\\
86.58	0.01\\
86.59	0.01\\
86.6	0.01\\
86.61	0.01\\
86.62	0.01\\
86.63	0.01\\
86.64	0.01\\
86.65	0.01\\
86.66	0.01\\
86.67	0.01\\
86.68	0.01\\
86.69	0.01\\
86.7	0.01\\
86.71	0.01\\
86.72	0.01\\
86.73	0.01\\
86.74	0.01\\
86.75	0.01\\
86.76	0.01\\
86.77	0.01\\
86.78	0.01\\
86.79	0.01\\
86.8	0.01\\
86.81	0.01\\
86.82	0.01\\
86.83	0.01\\
86.84	0.01\\
86.85	0.01\\
86.86	0.01\\
86.87	0.01\\
86.88	0.01\\
86.89	0.01\\
86.9	0.01\\
86.91	0.01\\
86.92	0.01\\
86.93	0.01\\
86.94	0.01\\
86.95	0.01\\
86.96	0.01\\
86.97	0.01\\
86.98	0.01\\
86.99	0.01\\
87	0.01\\
87.01	0.01\\
87.02	0.01\\
87.03	0.01\\
87.04	0.01\\
87.05	0.01\\
87.06	0.01\\
87.07	0.01\\
87.08	0.01\\
87.09	0.01\\
87.1	0.01\\
87.11	0.01\\
87.12	0.01\\
87.13	0.01\\
87.14	0.01\\
87.15	0.01\\
87.16	0.01\\
87.17	0.01\\
87.18	0.01\\
87.19	0.01\\
87.2	0.01\\
87.21	0.01\\
87.22	0.01\\
87.23	0.01\\
87.24	0.01\\
87.25	0.01\\
87.26	0.01\\
87.27	0.01\\
87.28	0.01\\
87.29	0.01\\
87.3	0.01\\
87.31	0.01\\
87.32	0.01\\
87.33	0.01\\
87.34	0.01\\
87.35	0.01\\
87.36	0.01\\
87.37	0.01\\
87.38	0.01\\
87.39	0.01\\
87.4	0.01\\
87.41	0.01\\
87.42	0.01\\
87.43	0.01\\
87.44	0.01\\
87.45	0.01\\
87.46	0.01\\
87.47	0.01\\
87.48	0.01\\
87.49	0.01\\
87.5	0.01\\
87.51	0.01\\
87.52	0.01\\
87.53	0.01\\
87.54	0.01\\
87.55	0.01\\
87.56	0.01\\
87.57	0.01\\
87.58	0.01\\
87.59	0.01\\
87.6	0.01\\
87.61	0.01\\
87.62	0.01\\
87.63	0.01\\
87.64	0.01\\
87.65	0.01\\
87.66	0.01\\
87.67	0.01\\
87.68	0.01\\
87.69	0.01\\
87.7	0.01\\
87.71	0.01\\
87.72	0.01\\
87.73	0.01\\
87.74	0.01\\
87.75	0.01\\
87.76	0.01\\
87.77	0.01\\
87.78	0.01\\
87.79	0.01\\
87.8	0.01\\
87.81	0.01\\
87.82	0.01\\
87.83	0.01\\
87.84	0.01\\
87.85	0.01\\
87.86	0.01\\
87.87	0.01\\
87.88	0.01\\
87.89	0.01\\
87.9	0.01\\
87.91	0.01\\
87.92	0.01\\
87.93	0.01\\
87.94	0.01\\
87.95	0.01\\
87.96	0.01\\
87.97	0.01\\
87.98	0.01\\
87.99	0.01\\
88	0.01\\
88.01	0.01\\
88.02	0.01\\
88.03	0.01\\
88.04	0.01\\
88.05	0.01\\
88.06	0.01\\
88.07	0.01\\
88.08	0.01\\
88.09	0.01\\
88.1	0.01\\
88.11	0.01\\
88.12	0.01\\
88.13	0.01\\
88.14	0.01\\
88.15	0.01\\
88.16	0.01\\
88.17	0.01\\
88.18	0.01\\
88.19	0.01\\
88.2	0.01\\
88.21	0.01\\
88.22	0.01\\
88.23	0.01\\
88.24	0.01\\
88.25	0.01\\
88.26	0.01\\
88.27	0.01\\
88.28	0.01\\
88.29	0.01\\
88.3	0.01\\
88.31	0.01\\
88.32	0.01\\
88.33	0.01\\
88.34	0.01\\
88.35	0.01\\
88.36	0.01\\
88.37	0.01\\
88.38	0.01\\
88.39	0.01\\
88.4	0.01\\
88.41	0.01\\
88.42	0.01\\
88.43	0.01\\
88.44	0.01\\
88.45	0.01\\
88.46	0.01\\
88.47	0.01\\
88.48	0.01\\
88.49	0.01\\
88.5	0.01\\
88.51	0.01\\
88.52	0.01\\
88.53	0.01\\
88.54	0.01\\
88.55	0.01\\
88.56	0.01\\
88.57	0.01\\
88.58	0.01\\
88.59	0.01\\
88.6	0.01\\
88.61	0.01\\
88.62	0.01\\
88.63	0.01\\
88.64	0.01\\
88.65	0.01\\
88.66	0.01\\
88.67	0.01\\
88.68	0.01\\
88.69	0.01\\
88.7	0.01\\
88.71	0.01\\
88.72	0.01\\
88.73	0.01\\
88.74	0.01\\
88.75	0.01\\
88.76	0.01\\
88.77	0.01\\
88.78	0.01\\
88.79	0.01\\
88.8	0.01\\
88.81	0.01\\
88.82	0.01\\
88.83	0.01\\
88.84	0.01\\
88.85	0.01\\
88.86	0.01\\
88.87	0.01\\
88.88	0.01\\
88.89	0.01\\
88.9	0.01\\
88.91	0.01\\
88.92	0.01\\
88.93	0.01\\
88.94	0.01\\
88.95	0.01\\
88.96	0.01\\
88.97	0.01\\
88.98	0.01\\
88.99	0.01\\
89	0.01\\
89.01	0.01\\
89.02	0.01\\
89.03	0.01\\
89.04	0.01\\
89.05	0.01\\
89.06	0.01\\
89.07	0.01\\
89.08	0.01\\
89.09	0.01\\
89.1	0.01\\
89.11	0.01\\
89.12	0.01\\
89.13	0.01\\
89.14	0.01\\
89.15	0.01\\
89.16	0.01\\
89.17	0.01\\
89.18	0.01\\
89.19	0.01\\
89.2	0.01\\
89.21	0.01\\
89.22	0.01\\
89.23	0.01\\
89.24	0.01\\
89.25	0.01\\
89.26	0.01\\
89.27	0.01\\
89.28	0.01\\
89.29	0.01\\
89.3	0.01\\
89.31	0.01\\
89.32	0.01\\
89.33	0.01\\
89.34	0.01\\
89.35	0.01\\
89.36	0.01\\
89.37	0.01\\
89.38	0.01\\
89.39	0.01\\
89.4	0.01\\
89.41	0.01\\
89.42	0.01\\
89.43	0.01\\
89.44	0.01\\
89.45	0.01\\
89.46	0.01\\
89.47	0.01\\
89.48	0.01\\
89.49	0.01\\
89.5	0.01\\
89.51	0.01\\
89.52	0.01\\
89.53	0.01\\
89.54	0.01\\
89.55	0.01\\
89.56	0.01\\
89.57	0.01\\
89.58	0.01\\
89.59	0.01\\
89.6	0.01\\
89.61	0.01\\
89.62	0.01\\
89.63	0.01\\
89.64	0.01\\
89.65	0.01\\
89.66	0.01\\
89.67	0.01\\
89.68	0.01\\
89.69	0.01\\
89.7	0.01\\
89.71	0.01\\
89.72	0.01\\
89.73	0.01\\
89.74	0.01\\
89.75	0.01\\
89.76	0.01\\
89.77	0.01\\
89.78	0.01\\
89.79	0.01\\
89.8	0.01\\
89.81	0.01\\
89.82	0.01\\
89.83	0.01\\
89.84	0.01\\
89.85	0.01\\
89.86	0.01\\
89.87	0.01\\
89.88	0.01\\
89.89	0.01\\
89.9	0.01\\
89.91	0.01\\
89.92	0.01\\
89.93	0.01\\
89.94	0.01\\
89.95	0.01\\
89.96	0.01\\
89.97	0.01\\
89.98	0.01\\
89.99	0.01\\
90	0.01\\
90.01	0.01\\
90.02	0.01\\
90.03	0.01\\
90.04	0.01\\
90.05	0.01\\
90.06	0.01\\
90.07	0.01\\
90.08	0.01\\
90.09	0.01\\
90.1	0.01\\
90.11	0.01\\
90.12	0.01\\
90.13	0.01\\
90.14	0.01\\
90.15	0.01\\
90.16	0.01\\
90.17	0.01\\
90.18	0.01\\
90.19	0.01\\
90.2	0.01\\
90.21	0.01\\
90.22	0.01\\
90.23	0.01\\
90.24	0.01\\
90.25	0.01\\
90.26	0.01\\
90.27	0.01\\
90.28	0.01\\
90.29	0.01\\
90.3	0.01\\
90.31	0.01\\
90.32	0.01\\
90.33	0.01\\
90.34	0.01\\
90.35	0.01\\
90.36	0.01\\
90.37	0.01\\
90.38	0.01\\
90.39	0.01\\
90.4	0.01\\
90.41	0.01\\
90.42	0.01\\
90.43	0.01\\
90.44	0.01\\
90.45	0.01\\
90.46	0.01\\
90.47	0.01\\
90.48	0.01\\
90.49	0.01\\
90.5	0.01\\
90.51	0.01\\
90.52	0.01\\
90.53	0.01\\
90.54	0.01\\
90.55	0.01\\
90.56	0.01\\
90.57	0.01\\
90.58	0.01\\
90.59	0.01\\
90.6	0.01\\
90.61	0.01\\
90.62	0.01\\
90.63	0.01\\
90.64	0.01\\
90.65	0.01\\
90.66	0.01\\
90.67	0.01\\
90.68	0.01\\
90.69	0.01\\
90.7	0.01\\
90.71	0.01\\
90.72	0.01\\
90.73	0.01\\
90.74	0.01\\
90.75	0.01\\
90.76	0.01\\
90.77	0.01\\
90.78	0.01\\
90.79	0.01\\
90.8	0.01\\
90.81	0.01\\
90.82	0.01\\
90.83	0.01\\
90.84	0.01\\
90.85	0.01\\
90.86	0.01\\
90.87	0.01\\
90.88	0.01\\
90.89	0.01\\
90.9	0.01\\
90.91	0.01\\
90.92	0.01\\
90.93	0.01\\
90.94	0.01\\
90.95	0.01\\
90.96	0.01\\
90.97	0.01\\
90.98	0.01\\
90.99	0.01\\
91	0.01\\
91.01	0.01\\
91.02	0.01\\
91.03	0.01\\
91.04	0.01\\
91.05	0.01\\
91.06	0.01\\
91.07	0.01\\
91.08	0.01\\
91.09	0.01\\
91.1	0.01\\
91.11	0.01\\
91.12	0.01\\
91.13	0.01\\
91.14	0.01\\
91.15	0.01\\
91.16	0.01\\
91.17	0.01\\
91.18	0.01\\
91.19	0.01\\
91.2	0.01\\
91.21	0.01\\
91.22	0.01\\
91.23	0.01\\
91.24	0.01\\
91.25	0.01\\
91.26	0.01\\
91.27	0.01\\
91.28	0.01\\
91.29	0.01\\
91.3	0.01\\
91.31	0.01\\
91.32	0.01\\
91.33	0.01\\
91.34	0.01\\
91.35	0.01\\
91.36	0.01\\
91.37	0.01\\
91.38	0.01\\
91.39	0.01\\
91.4	0.01\\
91.41	0.01\\
91.42	0.01\\
91.43	0.01\\
91.44	0.01\\
91.45	0.01\\
91.46	0.01\\
91.47	0.01\\
91.48	0.01\\
91.49	0.01\\
91.5	0.01\\
91.51	0.01\\
91.52	0.01\\
91.53	0.01\\
91.54	0.01\\
91.55	0.01\\
91.56	0.01\\
91.57	0.01\\
91.58	0.01\\
91.59	0.01\\
91.6	0.01\\
91.61	0.01\\
91.62	0.01\\
91.63	0.01\\
91.64	0.01\\
91.65	0.01\\
91.66	0.01\\
91.67	0.01\\
91.68	0.01\\
91.69	0.01\\
91.7	0.01\\
91.71	0.01\\
91.72	0.01\\
91.73	0.01\\
91.74	0.01\\
91.75	0.01\\
91.76	0.01\\
91.77	0.01\\
91.78	0.01\\
91.79	0.01\\
91.8	0.01\\
91.81	0.01\\
91.82	0.01\\
91.83	0.01\\
91.84	0.01\\
91.85	0.01\\
91.86	0.01\\
91.87	0.01\\
91.88	0.01\\
91.89	0.01\\
91.9	0.01\\
91.91	0.01\\
91.92	0.01\\
91.93	0.01\\
91.94	0.01\\
91.95	0.01\\
91.96	0.01\\
91.97	0.01\\
91.98	0.01\\
91.99	0.01\\
92	0.01\\
92.01	0.01\\
92.02	0.01\\
92.03	0.01\\
92.04	0.01\\
92.05	0.01\\
92.06	0.01\\
92.07	0.01\\
92.08	0.01\\
92.09	0.01\\
92.1	0.01\\
92.11	0.01\\
92.12	0.01\\
92.13	0.01\\
92.14	0.01\\
92.15	0.01\\
92.16	0.01\\
92.17	0.01\\
92.18	0.01\\
92.19	0.01\\
92.2	0.01\\
92.21	0.01\\
92.22	0.01\\
92.23	0.01\\
92.24	0.01\\
92.25	0.01\\
92.26	0.01\\
92.27	0.01\\
92.28	0.01\\
92.29	0.01\\
92.3	0.01\\
92.31	0.01\\
92.32	0.01\\
92.33	0.01\\
92.34	0.01\\
92.35	0.01\\
92.36	0.01\\
92.37	0.01\\
92.38	0.01\\
92.39	0.01\\
92.4	0.01\\
92.41	0.01\\
92.42	0.01\\
92.43	0.01\\
92.44	0.01\\
92.45	0.01\\
92.46	0.01\\
92.47	0.01\\
92.48	0.01\\
92.49	0.01\\
92.5	0.01\\
92.51	0.01\\
92.52	0.01\\
92.53	0.01\\
92.54	0.01\\
92.55	0.01\\
92.56	0.01\\
92.57	0.01\\
92.58	0.01\\
92.59	0.01\\
92.6	0.01\\
92.61	0.01\\
92.62	0.01\\
92.63	0.01\\
92.64	0.01\\
92.65	0.01\\
92.66	0.01\\
92.67	0.01\\
92.68	0.01\\
92.69	0.01\\
92.7	0.01\\
92.71	0.01\\
92.72	0.01\\
92.73	0.01\\
92.74	0.01\\
92.75	0.01\\
92.76	0.01\\
92.77	0.01\\
92.78	0.01\\
92.79	0.01\\
92.8	0.01\\
92.81	0.01\\
92.82	0.01\\
92.83	0.01\\
92.84	0.01\\
92.85	0.01\\
92.86	0.01\\
92.87	0.01\\
92.88	0.01\\
92.89	0.01\\
92.9	0.01\\
92.91	0.01\\
92.92	0.01\\
92.93	0.01\\
92.94	0.01\\
92.95	0.01\\
92.96	0.01\\
92.97	0.01\\
92.98	0.01\\
92.99	0.01\\
93	0.01\\
93.01	0.01\\
93.02	0.01\\
93.03	0.01\\
93.04	0.01\\
93.05	0.01\\
93.06	0.01\\
93.07	0.01\\
93.08	0.01\\
93.09	0.01\\
93.1	0.01\\
93.11	0.01\\
93.12	0.01\\
93.13	0.01\\
93.14	0.01\\
93.15	0.01\\
93.16	0.01\\
93.17	0.01\\
93.18	0.01\\
93.19	0.01\\
93.2	0.01\\
93.21	0.01\\
93.22	0.01\\
93.23	0.01\\
93.24	0.01\\
93.25	0.01\\
93.26	0.01\\
93.27	0.01\\
93.28	0.01\\
93.29	0.01\\
93.3	0.01\\
93.31	0.01\\
93.32	0.01\\
93.33	0.01\\
93.34	0.01\\
93.35	0.01\\
93.36	0.01\\
93.37	0.01\\
93.38	0.01\\
93.39	0.01\\
93.4	0.01\\
93.41	0.01\\
93.42	0.01\\
93.43	0.01\\
93.44	0.01\\
93.45	0.01\\
93.46	0.01\\
93.47	0.01\\
93.48	0.01\\
93.49	0.01\\
93.5	0.01\\
93.51	0.01\\
93.52	0.01\\
93.53	0.01\\
93.54	0.01\\
93.55	0.01\\
93.56	0.01\\
93.57	0.01\\
93.58	0.01\\
93.59	0.01\\
93.6	0.01\\
93.61	0.01\\
93.62	0.01\\
93.63	0.01\\
93.64	0.01\\
93.65	0.01\\
93.66	0.01\\
93.67	0.01\\
93.68	0.01\\
93.69	0.01\\
93.7	0.01\\
93.71	0.01\\
93.72	0.01\\
93.73	0.01\\
93.74	0.01\\
93.75	0.01\\
93.76	0.01\\
93.77	0.01\\
93.78	0.01\\
93.79	0.01\\
93.8	0.01\\
93.81	0.01\\
93.82	0.01\\
93.83	0.01\\
93.84	0.01\\
93.85	0.01\\
93.86	0.01\\
93.87	0.01\\
93.88	0.01\\
93.89	0.01\\
93.9	0.01\\
93.91	0.01\\
93.92	0.01\\
93.93	0.01\\
93.94	0.01\\
93.95	0.01\\
93.96	0.01\\
93.97	0.01\\
93.98	0.01\\
93.99	0.01\\
94	0.01\\
94.01	0.01\\
94.02	0.01\\
94.03	0.01\\
94.04	0.01\\
94.05	0.01\\
94.06	0.01\\
94.07	0.01\\
94.08	0.01\\
94.09	0.01\\
94.1	0.01\\
94.11	0.01\\
94.12	0.01\\
94.13	0.01\\
94.14	0.01\\
94.15	0.01\\
94.16	0.01\\
94.17	0.01\\
94.18	0.01\\
94.19	0.01\\
94.2	0.01\\
94.21	0.01\\
94.22	0.01\\
94.23	0.01\\
94.24	0.01\\
94.25	0.01\\
94.26	0.01\\
94.27	0.01\\
94.28	0.01\\
94.29	0.01\\
94.3	0.01\\
94.31	0.01\\
94.32	0.01\\
94.33	0.01\\
94.34	0.01\\
94.35	0.01\\
94.36	0.01\\
94.37	0.01\\
94.38	0.01\\
94.39	0.01\\
94.4	0.01\\
94.41	0.01\\
94.42	0.01\\
94.43	0.01\\
94.44	0.01\\
94.45	0.01\\
94.46	0.01\\
94.47	0.01\\
94.48	0.01\\
94.49	0.01\\
94.5	0.01\\
94.51	0.01\\
94.52	0.01\\
94.53	0.01\\
94.54	0.01\\
94.55	0.01\\
94.56	0.01\\
94.57	0.01\\
94.58	0.01\\
94.59	0.01\\
94.6	0.01\\
94.61	0.01\\
94.62	0.01\\
94.63	0.01\\
94.64	0.01\\
94.65	0.01\\
94.66	0.01\\
94.67	0.01\\
94.68	0.01\\
94.69	0.01\\
94.7	0.01\\
94.71	0.01\\
94.72	0.01\\
94.73	0.01\\
94.74	0.01\\
94.75	0.01\\
94.76	0.01\\
94.77	0.01\\
94.78	0.01\\
94.79	0.01\\
94.8	0.01\\
94.81	0.01\\
94.82	0.01\\
94.83	0.01\\
94.84	0.01\\
94.85	0.01\\
94.86	0.01\\
94.87	0.01\\
94.88	0.01\\
94.89	0.01\\
94.9	0.01\\
94.91	0.01\\
94.92	0.01\\
94.93	0.01\\
94.94	0.01\\
94.95	0.01\\
94.96	0.01\\
94.97	0.01\\
94.98	0.01\\
94.99	0.01\\
95	0.01\\
95.01	0.01\\
95.02	0.01\\
95.03	0.01\\
95.04	0.01\\
95.05	0.01\\
95.06	0.01\\
95.07	0.01\\
95.08	0.01\\
95.09	0.01\\
95.1	0.01\\
95.11	0.01\\
95.12	0.01\\
95.13	0.01\\
95.14	0.01\\
95.15	0.01\\
95.16	0.01\\
95.17	0.01\\
95.18	0.01\\
95.19	0.01\\
95.2	0.01\\
95.21	0.01\\
95.22	0.01\\
95.23	0.01\\
95.24	0.01\\
95.25	0.01\\
95.26	0.01\\
95.27	0.01\\
95.28	0.01\\
95.29	0.01\\
95.3	0.01\\
95.31	0.01\\
95.32	0.01\\
95.33	0.01\\
95.34	0.01\\
95.35	0.01\\
95.36	0.01\\
95.37	0.01\\
95.38	0.01\\
95.39	0.01\\
95.4	0.01\\
95.41	0.01\\
95.42	0.01\\
95.43	0.01\\
95.44	0.01\\
95.45	0.01\\
95.46	0.01\\
95.47	0.01\\
95.48	0.01\\
95.49	0.01\\
95.5	0.01\\
95.51	0.01\\
95.52	0.01\\
95.53	0.01\\
95.54	0.01\\
95.55	0.01\\
95.56	0.01\\
95.57	0.01\\
95.58	0.01\\
95.59	0.01\\
95.6	0.01\\
95.61	0.01\\
95.62	0.01\\
95.63	0.01\\
95.64	0.01\\
95.65	0.01\\
95.66	0.01\\
95.67	0.01\\
95.68	0.01\\
95.69	0.01\\
95.7	0.01\\
95.71	0.01\\
95.72	0.01\\
95.73	0.01\\
95.74	0.01\\
95.75	0.01\\
95.76	0.01\\
95.77	0.01\\
95.78	0.01\\
95.79	0.01\\
95.8	0.01\\
95.81	0.01\\
95.82	0.01\\
95.83	0.01\\
95.84	0.01\\
95.85	0.01\\
95.86	0.01\\
95.87	0.01\\
95.88	0.01\\
95.89	0.01\\
95.9	0.01\\
95.91	0.01\\
95.92	0.01\\
95.93	0.01\\
95.94	0.01\\
95.95	0.01\\
95.96	0.01\\
95.97	0.01\\
95.98	0.01\\
95.99	0.01\\
96	0.01\\
96.01	0.01\\
96.02	0.01\\
96.03	0.01\\
96.04	0.01\\
96.05	0.01\\
96.06	0.01\\
96.07	0.01\\
96.08	0.01\\
96.09	0.01\\
96.1	0.01\\
96.11	0.01\\
96.12	0.01\\
96.13	0.01\\
96.14	0.01\\
96.15	0.01\\
96.16	0.01\\
96.17	0.01\\
96.18	0.01\\
96.19	0.01\\
96.2	0.01\\
96.21	0.01\\
96.22	0.01\\
96.23	0.01\\
96.24	0.01\\
96.25	0.01\\
96.26	0.01\\
96.27	0.01\\
96.28	0.01\\
96.29	0.01\\
96.3	0.01\\
96.31	0.01\\
96.32	0.01\\
96.33	0.01\\
96.34	0.01\\
96.35	0.01\\
96.36	0.01\\
96.37	0.01\\
96.38	0.01\\
96.39	0.01\\
96.4	0.01\\
96.41	0.01\\
96.42	0.01\\
96.43	0.01\\
96.44	0.01\\
96.45	0.01\\
96.46	0.01\\
96.47	0.01\\
96.48	0.01\\
96.49	0.01\\
96.5	0.01\\
96.51	0.01\\
96.52	0.01\\
96.53	0.01\\
96.54	0.01\\
96.55	0.01\\
96.56	0.01\\
96.57	0.01\\
96.58	0.01\\
96.59	0.01\\
96.6	0.01\\
96.61	0.01\\
96.62	0.01\\
96.63	0.01\\
96.64	0.01\\
96.65	0.01\\
96.66	0.01\\
96.67	0.01\\
96.68	0.01\\
96.69	0.01\\
96.7	0.01\\
96.71	0.01\\
96.72	0.01\\
96.73	0.01\\
96.74	0.01\\
96.75	0.01\\
96.76	0.01\\
96.77	0.01\\
96.78	0.01\\
96.79	0.01\\
96.8	0.01\\
96.81	0.01\\
96.82	0.01\\
96.83	0.01\\
96.84	0.01\\
96.85	0.01\\
96.86	0.01\\
96.87	0.01\\
96.88	0.01\\
96.89	0.01\\
96.9	0.01\\
96.91	0.01\\
96.92	0.01\\
96.93	0.01\\
96.94	0.01\\
96.95	0.01\\
96.96	0.01\\
96.97	0.01\\
96.98	0.01\\
96.99	0.01\\
97	0.01\\
97.01	0.01\\
97.02	0.01\\
97.03	0.01\\
97.04	0.01\\
97.05	0.01\\
97.06	0.01\\
97.07	0.01\\
97.08	0.01\\
97.09	0.01\\
97.1	0.01\\
97.11	0.01\\
97.12	0.01\\
97.13	0.01\\
97.14	0.01\\
97.15	0.01\\
97.16	0.01\\
97.17	0.01\\
97.18	0.01\\
97.19	0.01\\
97.2	0.01\\
97.21	0.01\\
97.22	0.01\\
97.23	0.01\\
97.24	0.01\\
97.25	0.01\\
97.26	0.01\\
97.27	0.01\\
97.28	0.01\\
97.29	0.01\\
97.3	0.01\\
97.31	0.01\\
97.32	0.01\\
97.33	0.01\\
97.34	0.01\\
97.35	0.01\\
97.36	0.01\\
97.37	0.01\\
97.38	0.01\\
97.39	0.01\\
97.4	0.01\\
97.41	0.01\\
97.42	0.01\\
97.43	0.01\\
97.44	0.01\\
97.45	0.01\\
97.46	0.01\\
97.47	0.01\\
97.48	0.01\\
97.49	0.01\\
97.5	0.01\\
97.51	0.01\\
97.52	0.01\\
97.53	0.01\\
97.54	0.01\\
97.55	0.01\\
97.56	0.01\\
97.57	0.01\\
97.58	0.01\\
97.59	0.01\\
97.6	0.01\\
97.61	0.01\\
97.62	0.01\\
97.63	0.01\\
97.64	0.01\\
97.65	0.01\\
97.66	0.01\\
97.67	0.01\\
97.68	0.01\\
97.69	0.01\\
97.7	0.01\\
97.71	0.01\\
97.72	0.01\\
97.73	0.01\\
97.74	0.01\\
97.75	0.01\\
97.76	0.01\\
97.77	0.01\\
97.78	0.01\\
97.79	0.01\\
97.8	0.01\\
97.81	0.01\\
97.82	0.01\\
97.83	0.01\\
97.84	0.01\\
97.85	0.01\\
97.86	0.01\\
97.87	0.01\\
97.88	0.01\\
97.89	0.01\\
97.9	0.01\\
97.91	0.01\\
97.92	0.01\\
97.93	0.01\\
97.94	0.01\\
97.95	0.01\\
97.96	0.01\\
97.97	0.01\\
97.98	0.01\\
97.99	0.01\\
98	0.01\\
98.01	0.01\\
98.02	0.01\\
98.03	0.01\\
98.04	0.01\\
98.05	0.00996372752766895\\
98.06	0.00987999614471576\\
98.07	0.00979562027903187\\
98.08	0.00971059366384702\\
98.09	0.00962490996542519\\
98.1	0.00953856277553739\\
98.11	0.00945154560844692\\
98.12	0.00936385190824455\\
98.13	0.00927547504792656\\
98.14	0.00918640832845624\\
98.15	0.00909664497780845\\
98.16	0.00905441357256429\\
98.17	0.00903509579699211\\
98.18	0.00901562220828275\\
98.19	0.00899598765705672\\
98.2	0.00897619085137898\\
98.21	0.00895623048939617\\
98.22	0.00893610526244951\\
98.23	0.00891581385514894\\
98.24	0.00889535494545201\\
98.25	0.00887472720474753\\
98.26	0.0088539292979442\\
98.27	0.00883295988356457\\
98.28	0.00881181761384423\\
98.29	0.00879050113483673\\
98.3	0.0087690090865242\\
98.31	0.00874734010293407\\
98.32	0.00872549281226196\\
98.33	0.00870344911198926\\
98.34	0.00868120602500633\\
98.35	0.00865876163447077\\
98.36	0.00863611405326114\\
98.37	0.00861326137640437\\
98.38	0.00859020168090255\\
98.39	0.00856693302555777\\
98.4	0.00854345345079216\\
98.41	0.00851976097913629\\
98.42	0.00849585361986697\\
98.43	0.00847172936358991\\
98.44	0.00844738618206299\\
98.45	0.00842282202801773\\
98.46	0.00839803483497916\\
98.47	0.00837302251708372\\
98.48	0.0083477829688956\\
98.49	0.00832231406522105\\
98.5	0.0082966136609211\\
98.51	0.0082706795907222\\
98.52	0.00824450966902522\\
98.53	0.00821810168971239\\
98.54	0.0081914534259524\\
98.55	0.00816456263000364\\
98.56	0.00813742703301402\\
98.57	0.00811004434481012\\
98.58	0.00808241225369388\\
98.59	0.00805452842623716\\
98.6	0.00802639050707427\\
98.61	0.00799799611869249\\
98.62	0.00796934286122034\\
98.63	0.00794042831274018\\
98.64	0.00791125003272472\\
98.65	0.00788180555787676\\
98.66	0.00785209240191757\\
98.67	0.00782210805537347\\
98.68	0.00779184998536032\\
98.69	0.00776131563536613\\
98.7	0.00773050242503168\\
98.71	0.00769940774992909\\
98.72	0.0076680289813384\\
98.73	0.00763636346602215\\
98.74	0.00760440852598985\\
98.75	0.00757216145825925\\
98.76	0.00753961953462354\\
98.77	0.00750678000141649\\
98.78	0.00747364007927535\\
98.79	0.00744019696290129\\
98.8	0.00740644782081242\\
98.81	0.00737238979509965\\
98.82	0.00733802000118032\\
98.83	0.00730333552754955\\
98.84	0.00726833343552918\\
98.85	0.00723301075901456\\
98.86	0.00719736450421877\\
98.87	0.00716139164941458\\
98.88	0.007125089144674\\
98.89	0.00708845391160537\\
98.9	0.00705148284308801\\
98.91	0.00701417280300439\\
98.92	0.00697652062596991\\
98.93	0.00693852311706\\
98.94	0.00690017705153485\\
98.95	0.0068614791745615\\
98.96	0.00682242620093329\\
98.97	0.00678301481478685\\
98.98	0.0067432416693163\\
98.99	0.00670310338648488\\
99	0.00666259655673384\\
99.01	0.00662171773868872\\
99.02	0.00658046345886275\\
99.03	0.00653883021135762\\
99.04	0.00649681445756137\\
99.05	0.00645441262584356\\
99.06	0.00641162111124752\\
99.07	0.00636843627517979\\
99.08	0.00632485444509663\\
99.09	0.00628087191418765\\
99.1	0.00623648494105647\\
99.11	0.00619168974939843\\
99.12	0.00614648252767528\\
99.13	0.00610085942878684\\
99.14	0.00605481656973963\\
99.15	0.00600835003131242\\
99.16	0.00596145585771863\\
99.17	0.00591413005626559\\
99.18	0.0058663685970107\\
99.19	0.00581816745002337\\
99.2	0.0057695225507678\\
99.21	0.00572042979718656\\
99.22	0.00567088504935485\\
99.23	0.00562088412913144\\
99.24	0.00557042281980654\\
99.25	0.00551949686574641\\
99.26	0.00546810197203458\\
99.27	0.00541623380410996\\
99.28	0.00536388798740143\\
99.29	0.00531106010695923\\
99.3	0.00525774570708285\\
99.31	0.00520394029094553\\
99.32	0.00514963932021534\\
99.33	0.00509483821467274\\
99.34	0.00503953235182461\\
99.35	0.00498371706651476\\
99.36	0.00492738765053087\\
99.37	0.00487053935220775\\
99.38	0.00481316737602703\\
99.39	0.0047552668822131\\
99.4	0.00469683298632543\\
99.41	0.004637860758847\\
99.42	0.00457834522476916\\
99.43	0.00451828136317241\\
99.44	0.00445766410680361\\
99.45	0.00439648834164902\\
99.46	0.00433474890650368\\
99.47	0.00427244059253659\\
99.48	0.00420955814285212\\
99.49	0.00414609625204719\\
99.5	0.0040820495657645\\
99.51	0.00401741268024164\\
99.52	0.00395218014185599\\
99.53	0.00388634644666552\\
99.54	0.00381990603994536\\
99.55	0.00375285331572\\
99.56	0.00368518261629134\\
99.57	0.00361688823176235\\
99.58	0.00354796439955628\\
99.59	0.0034784053039316\\
99.6	0.00340820507549232\\
99.61	0.00333735779069392\\
99.62	0.00326585747468937\\
99.63	0.00319369810297135\\
99.64	0.00312087359552881\\
99.65	0.00304737781633648\\
99.66	0.00297320457283966\\
99.67	0.00289834761543429\\
99.68	0.00282280063694228\\
99.69	0.00274655727208197\\
99.7	0.00266961109693381\\
99.71	0.00259195562840104\\
99.72	0.00251358432366551\\
99.73	0.00243449057963848\\
99.74	0.00235466773240636\\
99.75	0.00227410905667141\\
99.76	0.0021928077651873\\
99.77	0.00211075700818945\\
99.78	0.00202794987282024\\
99.79	0.00194437938254886\\
99.8	0.0018600384965859\\
99.81	0.00177492010929258\\
99.82	0.0016890170495845\\
99.83	0.00160232208032999\\
99.84	0.00151482789774301\\
99.85	0.0014265271307703\\
99.86	0.00133741234047312\\
99.87	0.00124747601940323\\
99.88	0.00115671059097319\\
99.89	0.00106510840882085\\
99.9	0.000972661756168126\\
99.91	0.000879362845173816\\
99.92	0.00078520381628055\\
99.93	0.000690176737555767\\
99.94	0.000594273604026672\\
99.95	0.000497486337009101\\
99.96	0.000399806783430302\\
99.97	0.00030122671514549\\
99.98	0.000201737828248205\\
99.99	0.00010133174237437\\
100	0\\
};
\addlegendentry{$q=-4$};

\addplot [color=mycolor1,dashed,forget plot]
  table[row sep=crcr]{%
0.01	0.01\\
0.02	0.01\\
0.03	0.01\\
0.04	0.01\\
0.05	0.01\\
0.06	0.01\\
0.07	0.01\\
0.08	0.01\\
0.09	0.01\\
0.1	0.01\\
0.11	0.01\\
0.12	0.01\\
0.13	0.01\\
0.14	0.01\\
0.15	0.01\\
0.16	0.01\\
0.17	0.01\\
0.18	0.01\\
0.19	0.01\\
0.2	0.01\\
0.21	0.01\\
0.22	0.01\\
0.23	0.01\\
0.24	0.01\\
0.25	0.01\\
0.26	0.01\\
0.27	0.01\\
0.28	0.01\\
0.29	0.01\\
0.3	0.01\\
0.31	0.01\\
0.32	0.01\\
0.33	0.01\\
0.34	0.01\\
0.35	0.01\\
0.36	0.01\\
0.37	0.01\\
0.38	0.01\\
0.39	0.01\\
0.4	0.01\\
0.41	0.01\\
0.42	0.01\\
0.43	0.01\\
0.44	0.01\\
0.45	0.01\\
0.46	0.01\\
0.47	0.01\\
0.48	0.01\\
0.49	0.01\\
0.5	0.01\\
0.51	0.01\\
0.52	0.01\\
0.53	0.01\\
0.54	0.01\\
0.55	0.01\\
0.56	0.01\\
0.57	0.01\\
0.58	0.01\\
0.59	0.01\\
0.6	0.01\\
0.61	0.01\\
0.62	0.01\\
0.63	0.01\\
0.64	0.01\\
0.65	0.01\\
0.66	0.01\\
0.67	0.01\\
0.68	0.01\\
0.69	0.01\\
0.7	0.01\\
0.71	0.01\\
0.72	0.01\\
0.73	0.01\\
0.74	0.01\\
0.75	0.01\\
0.76	0.01\\
0.77	0.01\\
0.78	0.01\\
0.79	0.01\\
0.8	0.01\\
0.81	0.01\\
0.82	0.01\\
0.83	0.01\\
0.84	0.01\\
0.85	0.01\\
0.86	0.01\\
0.87	0.01\\
0.88	0.01\\
0.89	0.01\\
0.9	0.01\\
0.91	0.01\\
0.92	0.01\\
0.93	0.01\\
0.94	0.01\\
0.95	0.01\\
0.96	0.01\\
0.97	0.01\\
0.98	0.01\\
0.99	0.01\\
1	0.01\\
1.01	0.01\\
1.02	0.01\\
1.03	0.01\\
1.04	0.01\\
1.05	0.01\\
1.06	0.01\\
1.07	0.01\\
1.08	0.01\\
1.09	0.01\\
1.1	0.01\\
1.11	0.01\\
1.12	0.01\\
1.13	0.01\\
1.14	0.01\\
1.15	0.01\\
1.16	0.01\\
1.17	0.01\\
1.18	0.01\\
1.19	0.01\\
1.2	0.01\\
1.21	0.01\\
1.22	0.01\\
1.23	0.01\\
1.24	0.01\\
1.25	0.01\\
1.26	0.01\\
1.27	0.01\\
1.28	0.01\\
1.29	0.01\\
1.3	0.01\\
1.31	0.01\\
1.32	0.01\\
1.33	0.01\\
1.34	0.01\\
1.35	0.01\\
1.36	0.01\\
1.37	0.01\\
1.38	0.01\\
1.39	0.01\\
1.4	0.01\\
1.41	0.01\\
1.42	0.01\\
1.43	0.01\\
1.44	0.01\\
1.45	0.01\\
1.46	0.01\\
1.47	0.01\\
1.48	0.01\\
1.49	0.01\\
1.5	0.01\\
1.51	0.01\\
1.52	0.01\\
1.53	0.01\\
1.54	0.01\\
1.55	0.01\\
1.56	0.01\\
1.57	0.01\\
1.58	0.01\\
1.59	0.01\\
1.6	0.01\\
1.61	0.01\\
1.62	0.01\\
1.63	0.01\\
1.64	0.01\\
1.65	0.01\\
1.66	0.01\\
1.67	0.01\\
1.68	0.01\\
1.69	0.01\\
1.7	0.01\\
1.71	0.01\\
1.72	0.01\\
1.73	0.01\\
1.74	0.01\\
1.75	0.01\\
1.76	0.01\\
1.77	0.01\\
1.78	0.01\\
1.79	0.01\\
1.8	0.01\\
1.81	0.01\\
1.82	0.01\\
1.83	0.01\\
1.84	0.01\\
1.85	0.01\\
1.86	0.01\\
1.87	0.01\\
1.88	0.01\\
1.89	0.01\\
1.9	0.01\\
1.91	0.01\\
1.92	0.01\\
1.93	0.01\\
1.94	0.01\\
1.95	0.01\\
1.96	0.01\\
1.97	0.01\\
1.98	0.01\\
1.99	0.01\\
2	0.01\\
2.01	0.01\\
2.02	0.01\\
2.03	0.01\\
2.04	0.01\\
2.05	0.01\\
2.06	0.01\\
2.07	0.01\\
2.08	0.01\\
2.09	0.01\\
2.1	0.01\\
2.11	0.01\\
2.12	0.01\\
2.13	0.01\\
2.14	0.01\\
2.15	0.01\\
2.16	0.01\\
2.17	0.01\\
2.18	0.01\\
2.19	0.01\\
2.2	0.01\\
2.21	0.01\\
2.22	0.01\\
2.23	0.01\\
2.24	0.01\\
2.25	0.01\\
2.26	0.01\\
2.27	0.01\\
2.28	0.01\\
2.29	0.01\\
2.3	0.01\\
2.31	0.01\\
2.32	0.01\\
2.33	0.01\\
2.34	0.01\\
2.35	0.01\\
2.36	0.01\\
2.37	0.01\\
2.38	0.01\\
2.39	0.01\\
2.4	0.01\\
2.41	0.01\\
2.42	0.01\\
2.43	0.01\\
2.44	0.01\\
2.45	0.01\\
2.46	0.01\\
2.47	0.01\\
2.48	0.01\\
2.49	0.01\\
2.5	0.01\\
2.51	0.01\\
2.52	0.01\\
2.53	0.01\\
2.54	0.01\\
2.55	0.01\\
2.56	0.01\\
2.57	0.01\\
2.58	0.01\\
2.59	0.01\\
2.6	0.01\\
2.61	0.01\\
2.62	0.01\\
2.63	0.01\\
2.64	0.01\\
2.65	0.01\\
2.66	0.01\\
2.67	0.01\\
2.68	0.01\\
2.69	0.01\\
2.7	0.01\\
2.71	0.01\\
2.72	0.01\\
2.73	0.01\\
2.74	0.01\\
2.75	0.01\\
2.76	0.01\\
2.77	0.01\\
2.78	0.01\\
2.79	0.01\\
2.8	0.01\\
2.81	0.01\\
2.82	0.01\\
2.83	0.01\\
2.84	0.01\\
2.85	0.01\\
2.86	0.01\\
2.87	0.01\\
2.88	0.01\\
2.89	0.01\\
2.9	0.01\\
2.91	0.01\\
2.92	0.01\\
2.93	0.01\\
2.94	0.01\\
2.95	0.01\\
2.96	0.01\\
2.97	0.01\\
2.98	0.01\\
2.99	0.01\\
3	0.01\\
3.01	0.01\\
3.02	0.01\\
3.03	0.01\\
3.04	0.01\\
3.05	0.01\\
3.06	0.01\\
3.07	0.01\\
3.08	0.01\\
3.09	0.01\\
3.1	0.01\\
3.11	0.01\\
3.12	0.01\\
3.13	0.01\\
3.14	0.01\\
3.15	0.01\\
3.16	0.01\\
3.17	0.01\\
3.18	0.01\\
3.19	0.01\\
3.2	0.01\\
3.21	0.01\\
3.22	0.01\\
3.23	0.01\\
3.24	0.01\\
3.25	0.01\\
3.26	0.01\\
3.27	0.01\\
3.28	0.01\\
3.29	0.01\\
3.3	0.01\\
3.31	0.01\\
3.32	0.01\\
3.33	0.01\\
3.34	0.01\\
3.35	0.01\\
3.36	0.01\\
3.37	0.01\\
3.38	0.01\\
3.39	0.01\\
3.4	0.01\\
3.41	0.01\\
3.42	0.01\\
3.43	0.01\\
3.44	0.01\\
3.45	0.01\\
3.46	0.01\\
3.47	0.01\\
3.48	0.01\\
3.49	0.01\\
3.5	0.01\\
3.51	0.01\\
3.52	0.01\\
3.53	0.01\\
3.54	0.01\\
3.55	0.01\\
3.56	0.01\\
3.57	0.01\\
3.58	0.01\\
3.59	0.01\\
3.6	0.01\\
3.61	0.01\\
3.62	0.01\\
3.63	0.01\\
3.64	0.01\\
3.65	0.01\\
3.66	0.01\\
3.67	0.01\\
3.68	0.01\\
3.69	0.01\\
3.7	0.01\\
3.71	0.01\\
3.72	0.01\\
3.73	0.01\\
3.74	0.01\\
3.75	0.01\\
3.76	0.01\\
3.77	0.01\\
3.78	0.01\\
3.79	0.01\\
3.8	0.01\\
3.81	0.01\\
3.82	0.01\\
3.83	0.01\\
3.84	0.01\\
3.85	0.01\\
3.86	0.01\\
3.87	0.01\\
3.88	0.01\\
3.89	0.01\\
3.9	0.01\\
3.91	0.01\\
3.92	0.01\\
3.93	0.01\\
3.94	0.01\\
3.95	0.01\\
3.96	0.01\\
3.97	0.01\\
3.98	0.01\\
3.99	0.01\\
4	0.01\\
4.01	0.01\\
4.02	0.01\\
4.03	0.01\\
4.04	0.01\\
4.05	0.01\\
4.06	0.01\\
4.07	0.01\\
4.08	0.01\\
4.09	0.01\\
4.1	0.01\\
4.11	0.01\\
4.12	0.01\\
4.13	0.01\\
4.14	0.01\\
4.15	0.01\\
4.16	0.01\\
4.17	0.01\\
4.18	0.01\\
4.19	0.01\\
4.2	0.01\\
4.21	0.01\\
4.22	0.01\\
4.23	0.01\\
4.24	0.01\\
4.25	0.01\\
4.26	0.01\\
4.27	0.01\\
4.28	0.01\\
4.29	0.01\\
4.3	0.01\\
4.31	0.01\\
4.32	0.01\\
4.33	0.01\\
4.34	0.01\\
4.35	0.01\\
4.36	0.01\\
4.37	0.01\\
4.38	0.01\\
4.39	0.01\\
4.4	0.01\\
4.41	0.01\\
4.42	0.01\\
4.43	0.01\\
4.44	0.01\\
4.45	0.01\\
4.46	0.01\\
4.47	0.01\\
4.48	0.01\\
4.49	0.01\\
4.5	0.01\\
4.51	0.01\\
4.52	0.01\\
4.53	0.01\\
4.54	0.01\\
4.55	0.01\\
4.56	0.01\\
4.57	0.01\\
4.58	0.01\\
4.59	0.01\\
4.6	0.01\\
4.61	0.01\\
4.62	0.01\\
4.63	0.01\\
4.64	0.01\\
4.65	0.01\\
4.66	0.01\\
4.67	0.01\\
4.68	0.01\\
4.69	0.01\\
4.7	0.01\\
4.71	0.01\\
4.72	0.01\\
4.73	0.01\\
4.74	0.01\\
4.75	0.01\\
4.76	0.01\\
4.77	0.01\\
4.78	0.01\\
4.79	0.01\\
4.8	0.01\\
4.81	0.01\\
4.82	0.01\\
4.83	0.01\\
4.84	0.01\\
4.85	0.01\\
4.86	0.01\\
4.87	0.01\\
4.88	0.01\\
4.89	0.01\\
4.9	0.01\\
4.91	0.01\\
4.92	0.01\\
4.93	0.01\\
4.94	0.01\\
4.95	0.01\\
4.96	0.01\\
4.97	0.01\\
4.98	0.01\\
4.99	0.01\\
5	0.01\\
5.01	0.01\\
5.02	0.01\\
5.03	0.01\\
5.04	0.01\\
5.05	0.01\\
5.06	0.01\\
5.07	0.01\\
5.08	0.01\\
5.09	0.01\\
5.1	0.01\\
5.11	0.01\\
5.12	0.01\\
5.13	0.01\\
5.14	0.01\\
5.15	0.01\\
5.16	0.01\\
5.17	0.01\\
5.18	0.01\\
5.19	0.01\\
5.2	0.01\\
5.21	0.01\\
5.22	0.01\\
5.23	0.01\\
5.24	0.01\\
5.25	0.01\\
5.26	0.01\\
5.27	0.01\\
5.28	0.01\\
5.29	0.01\\
5.3	0.01\\
5.31	0.01\\
5.32	0.01\\
5.33	0.01\\
5.34	0.01\\
5.35	0.01\\
5.36	0.01\\
5.37	0.01\\
5.38	0.01\\
5.39	0.01\\
5.4	0.01\\
5.41	0.01\\
5.42	0.01\\
5.43	0.01\\
5.44	0.01\\
5.45	0.01\\
5.46	0.01\\
5.47	0.01\\
5.48	0.01\\
5.49	0.01\\
5.5	0.01\\
5.51	0.01\\
5.52	0.01\\
5.53	0.01\\
5.54	0.01\\
5.55	0.01\\
5.56	0.01\\
5.57	0.01\\
5.58	0.01\\
5.59	0.01\\
5.6	0.01\\
5.61	0.01\\
5.62	0.01\\
5.63	0.01\\
5.64	0.01\\
5.65	0.01\\
5.66	0.01\\
5.67	0.01\\
5.68	0.01\\
5.69	0.01\\
5.7	0.01\\
5.71	0.01\\
5.72	0.01\\
5.73	0.01\\
5.74	0.01\\
5.75	0.01\\
5.76	0.01\\
5.77	0.01\\
5.78	0.01\\
5.79	0.01\\
5.8	0.01\\
5.81	0.01\\
5.82	0.01\\
5.83	0.01\\
5.84	0.01\\
5.85	0.01\\
5.86	0.01\\
5.87	0.01\\
5.88	0.01\\
5.89	0.01\\
5.9	0.01\\
5.91	0.01\\
5.92	0.01\\
5.93	0.01\\
5.94	0.01\\
5.95	0.01\\
5.96	0.01\\
5.97	0.01\\
5.98	0.01\\
5.99	0.01\\
6	0.01\\
6.01	0.01\\
6.02	0.01\\
6.03	0.01\\
6.04	0.01\\
6.05	0.01\\
6.06	0.01\\
6.07	0.01\\
6.08	0.01\\
6.09	0.01\\
6.1	0.01\\
6.11	0.01\\
6.12	0.01\\
6.13	0.01\\
6.14	0.01\\
6.15	0.01\\
6.16	0.01\\
6.17	0.01\\
6.18	0.01\\
6.19	0.01\\
6.2	0.01\\
6.21	0.01\\
6.22	0.01\\
6.23	0.01\\
6.24	0.01\\
6.25	0.01\\
6.26	0.01\\
6.27	0.01\\
6.28	0.01\\
6.29	0.01\\
6.3	0.01\\
6.31	0.01\\
6.32	0.01\\
6.33	0.01\\
6.34	0.01\\
6.35	0.01\\
6.36	0.01\\
6.37	0.01\\
6.38	0.01\\
6.39	0.01\\
6.4	0.01\\
6.41	0.01\\
6.42	0.01\\
6.43	0.01\\
6.44	0.01\\
6.45	0.01\\
6.46	0.01\\
6.47	0.01\\
6.48	0.01\\
6.49	0.01\\
6.5	0.01\\
6.51	0.01\\
6.52	0.01\\
6.53	0.01\\
6.54	0.01\\
6.55	0.01\\
6.56	0.01\\
6.57	0.01\\
6.58	0.01\\
6.59	0.01\\
6.6	0.01\\
6.61	0.01\\
6.62	0.01\\
6.63	0.01\\
6.64	0.01\\
6.65	0.01\\
6.66	0.01\\
6.67	0.01\\
6.68	0.01\\
6.69	0.01\\
6.7	0.01\\
6.71	0.01\\
6.72	0.01\\
6.73	0.01\\
6.74	0.01\\
6.75	0.01\\
6.76	0.01\\
6.77	0.01\\
6.78	0.01\\
6.79	0.01\\
6.8	0.01\\
6.81	0.01\\
6.82	0.01\\
6.83	0.01\\
6.84	0.01\\
6.85	0.01\\
6.86	0.01\\
6.87	0.01\\
6.88	0.01\\
6.89	0.01\\
6.9	0.01\\
6.91	0.01\\
6.92	0.01\\
6.93	0.01\\
6.94	0.01\\
6.95	0.01\\
6.96	0.01\\
6.97	0.01\\
6.98	0.01\\
6.99	0.01\\
7	0.01\\
7.01	0.01\\
7.02	0.01\\
7.03	0.01\\
7.04	0.01\\
7.05	0.01\\
7.06	0.01\\
7.07	0.01\\
7.08	0.01\\
7.09	0.01\\
7.1	0.01\\
7.11	0.01\\
7.12	0.01\\
7.13	0.01\\
7.14	0.01\\
7.15	0.01\\
7.16	0.01\\
7.17	0.01\\
7.18	0.01\\
7.19	0.01\\
7.2	0.01\\
7.21	0.01\\
7.22	0.01\\
7.23	0.01\\
7.24	0.01\\
7.25	0.01\\
7.26	0.01\\
7.27	0.01\\
7.28	0.01\\
7.29	0.01\\
7.3	0.01\\
7.31	0.01\\
7.32	0.01\\
7.33	0.01\\
7.34	0.01\\
7.35	0.01\\
7.36	0.01\\
7.37	0.01\\
7.38	0.01\\
7.39	0.01\\
7.4	0.01\\
7.41	0.01\\
7.42	0.01\\
7.43	0.01\\
7.44	0.01\\
7.45	0.01\\
7.46	0.01\\
7.47	0.01\\
7.48	0.01\\
7.49	0.01\\
7.5	0.01\\
7.51	0.01\\
7.52	0.01\\
7.53	0.01\\
7.54	0.01\\
7.55	0.01\\
7.56	0.01\\
7.57	0.01\\
7.58	0.01\\
7.59	0.01\\
7.6	0.01\\
7.61	0.01\\
7.62	0.01\\
7.63	0.01\\
7.64	0.01\\
7.65	0.01\\
7.66	0.01\\
7.67	0.01\\
7.68	0.01\\
7.69	0.01\\
7.7	0.01\\
7.71	0.01\\
7.72	0.01\\
7.73	0.01\\
7.74	0.01\\
7.75	0.01\\
7.76	0.01\\
7.77	0.01\\
7.78	0.01\\
7.79	0.01\\
7.8	0.01\\
7.81	0.01\\
7.82	0.01\\
7.83	0.01\\
7.84	0.01\\
7.85	0.01\\
7.86	0.01\\
7.87	0.01\\
7.88	0.01\\
7.89	0.01\\
7.9	0.01\\
7.91	0.01\\
7.92	0.01\\
7.93	0.01\\
7.94	0.01\\
7.95	0.01\\
7.96	0.01\\
7.97	0.01\\
7.98	0.01\\
7.99	0.01\\
8	0.01\\
8.01	0.01\\
8.02	0.01\\
8.03	0.01\\
8.04	0.01\\
8.05	0.01\\
8.06	0.01\\
8.07	0.01\\
8.08	0.01\\
8.09	0.01\\
8.1	0.01\\
8.11	0.01\\
8.12	0.01\\
8.13	0.01\\
8.14	0.01\\
8.15	0.01\\
8.16	0.01\\
8.17	0.01\\
8.18	0.01\\
8.19	0.01\\
8.2	0.01\\
8.21	0.01\\
8.22	0.01\\
8.23	0.01\\
8.24	0.01\\
8.25	0.01\\
8.26	0.01\\
8.27	0.01\\
8.28	0.01\\
8.29	0.01\\
8.3	0.01\\
8.31	0.01\\
8.32	0.01\\
8.33	0.01\\
8.34	0.01\\
8.35	0.01\\
8.36	0.01\\
8.37	0.01\\
8.38	0.01\\
8.39	0.01\\
8.4	0.01\\
8.41	0.01\\
8.42	0.01\\
8.43	0.01\\
8.44	0.01\\
8.45	0.01\\
8.46	0.01\\
8.47	0.01\\
8.48	0.01\\
8.49	0.01\\
8.5	0.01\\
8.51	0.01\\
8.52	0.01\\
8.53	0.01\\
8.54	0.01\\
8.55	0.01\\
8.56	0.01\\
8.57	0.01\\
8.58	0.01\\
8.59	0.01\\
8.6	0.01\\
8.61	0.01\\
8.62	0.01\\
8.63	0.01\\
8.64	0.01\\
8.65	0.01\\
8.66	0.01\\
8.67	0.01\\
8.68	0.01\\
8.69	0.01\\
8.7	0.01\\
8.71	0.01\\
8.72	0.01\\
8.73	0.01\\
8.74	0.01\\
8.75	0.01\\
8.76	0.01\\
8.77	0.01\\
8.78	0.01\\
8.79	0.01\\
8.8	0.01\\
8.81	0.01\\
8.82	0.01\\
8.83	0.01\\
8.84	0.01\\
8.85	0.01\\
8.86	0.01\\
8.87	0.01\\
8.88	0.01\\
8.89	0.01\\
8.9	0.01\\
8.91	0.01\\
8.92	0.01\\
8.93	0.01\\
8.94	0.01\\
8.95	0.01\\
8.96	0.01\\
8.97	0.01\\
8.98	0.01\\
8.99	0.01\\
9	0.01\\
9.01	0.01\\
9.02	0.01\\
9.03	0.01\\
9.04	0.01\\
9.05	0.01\\
9.06	0.01\\
9.07	0.01\\
9.08	0.01\\
9.09	0.01\\
9.1	0.01\\
9.11	0.01\\
9.12	0.01\\
9.13	0.01\\
9.14	0.01\\
9.15	0.01\\
9.16	0.01\\
9.17	0.01\\
9.18	0.01\\
9.19	0.01\\
9.2	0.01\\
9.21	0.01\\
9.22	0.01\\
9.23	0.01\\
9.24	0.01\\
9.25	0.01\\
9.26	0.01\\
9.27	0.01\\
9.28	0.01\\
9.29	0.01\\
9.3	0.01\\
9.31	0.01\\
9.32	0.01\\
9.33	0.01\\
9.34	0.01\\
9.35	0.01\\
9.36	0.01\\
9.37	0.01\\
9.38	0.01\\
9.39	0.01\\
9.4	0.01\\
9.41	0.01\\
9.42	0.01\\
9.43	0.01\\
9.44	0.01\\
9.45	0.01\\
9.46	0.01\\
9.47	0.01\\
9.48	0.01\\
9.49	0.01\\
9.5	0.01\\
9.51	0.01\\
9.52	0.01\\
9.53	0.01\\
9.54	0.01\\
9.55	0.01\\
9.56	0.01\\
9.57	0.01\\
9.58	0.01\\
9.59	0.01\\
9.6	0.01\\
9.61	0.01\\
9.62	0.01\\
9.63	0.01\\
9.64	0.01\\
9.65	0.01\\
9.66	0.01\\
9.67	0.01\\
9.68	0.01\\
9.69	0.01\\
9.7	0.01\\
9.71	0.01\\
9.72	0.01\\
9.73	0.01\\
9.74	0.01\\
9.75	0.01\\
9.76	0.01\\
9.77	0.01\\
9.78	0.01\\
9.79	0.01\\
9.8	0.01\\
9.81	0.01\\
9.82	0.01\\
9.83	0.01\\
9.84	0.01\\
9.85	0.01\\
9.86	0.01\\
9.87	0.01\\
9.88	0.01\\
9.89	0.01\\
9.9	0.01\\
9.91	0.01\\
9.92	0.01\\
9.93	0.01\\
9.94	0.01\\
9.95	0.01\\
9.96	0.01\\
9.97	0.01\\
9.98	0.01\\
9.99	0.01\\
10	0.01\\
10.01	0.01\\
10.02	0.01\\
10.03	0.01\\
10.04	0.01\\
10.05	0.01\\
10.06	0.01\\
10.07	0.01\\
10.08	0.01\\
10.09	0.01\\
10.1	0.01\\
10.11	0.01\\
10.12	0.01\\
10.13	0.01\\
10.14	0.01\\
10.15	0.01\\
10.16	0.01\\
10.17	0.01\\
10.18	0.01\\
10.19	0.01\\
10.2	0.01\\
10.21	0.01\\
10.22	0.01\\
10.23	0.01\\
10.24	0.01\\
10.25	0.01\\
10.26	0.01\\
10.27	0.01\\
10.28	0.01\\
10.29	0.01\\
10.3	0.01\\
10.31	0.01\\
10.32	0.01\\
10.33	0.01\\
10.34	0.01\\
10.35	0.01\\
10.36	0.01\\
10.37	0.01\\
10.38	0.01\\
10.39	0.01\\
10.4	0.01\\
10.41	0.01\\
10.42	0.01\\
10.43	0.01\\
10.44	0.01\\
10.45	0.01\\
10.46	0.01\\
10.47	0.01\\
10.48	0.01\\
10.49	0.01\\
10.5	0.01\\
10.51	0.01\\
10.52	0.01\\
10.53	0.01\\
10.54	0.01\\
10.55	0.01\\
10.56	0.01\\
10.57	0.01\\
10.58	0.01\\
10.59	0.01\\
10.6	0.01\\
10.61	0.01\\
10.62	0.01\\
10.63	0.01\\
10.64	0.01\\
10.65	0.01\\
10.66	0.01\\
10.67	0.01\\
10.68	0.01\\
10.69	0.01\\
10.7	0.01\\
10.71	0.01\\
10.72	0.01\\
10.73	0.01\\
10.74	0.01\\
10.75	0.01\\
10.76	0.01\\
10.77	0.01\\
10.78	0.01\\
10.79	0.01\\
10.8	0.01\\
10.81	0.01\\
10.82	0.01\\
10.83	0.01\\
10.84	0.01\\
10.85	0.01\\
10.86	0.01\\
10.87	0.01\\
10.88	0.01\\
10.89	0.01\\
10.9	0.01\\
10.91	0.01\\
10.92	0.01\\
10.93	0.01\\
10.94	0.01\\
10.95	0.01\\
10.96	0.01\\
10.97	0.01\\
10.98	0.01\\
10.99	0.01\\
11	0.01\\
11.01	0.01\\
11.02	0.01\\
11.03	0.01\\
11.04	0.01\\
11.05	0.01\\
11.06	0.01\\
11.07	0.01\\
11.08	0.01\\
11.09	0.01\\
11.1	0.01\\
11.11	0.01\\
11.12	0.01\\
11.13	0.01\\
11.14	0.01\\
11.15	0.01\\
11.16	0.01\\
11.17	0.01\\
11.18	0.01\\
11.19	0.01\\
11.2	0.01\\
11.21	0.01\\
11.22	0.01\\
11.23	0.01\\
11.24	0.01\\
11.25	0.01\\
11.26	0.01\\
11.27	0.01\\
11.28	0.01\\
11.29	0.01\\
11.3	0.01\\
11.31	0.01\\
11.32	0.01\\
11.33	0.01\\
11.34	0.01\\
11.35	0.01\\
11.36	0.01\\
11.37	0.01\\
11.38	0.01\\
11.39	0.01\\
11.4	0.01\\
11.41	0.01\\
11.42	0.01\\
11.43	0.01\\
11.44	0.01\\
11.45	0.01\\
11.46	0.01\\
11.47	0.01\\
11.48	0.01\\
11.49	0.01\\
11.5	0.01\\
11.51	0.01\\
11.52	0.01\\
11.53	0.01\\
11.54	0.01\\
11.55	0.01\\
11.56	0.01\\
11.57	0.01\\
11.58	0.01\\
11.59	0.01\\
11.6	0.01\\
11.61	0.01\\
11.62	0.01\\
11.63	0.01\\
11.64	0.01\\
11.65	0.01\\
11.66	0.01\\
11.67	0.01\\
11.68	0.01\\
11.69	0.01\\
11.7	0.01\\
11.71	0.01\\
11.72	0.01\\
11.73	0.01\\
11.74	0.01\\
11.75	0.01\\
11.76	0.01\\
11.77	0.01\\
11.78	0.01\\
11.79	0.01\\
11.8	0.01\\
11.81	0.01\\
11.82	0.01\\
11.83	0.01\\
11.84	0.01\\
11.85	0.01\\
11.86	0.01\\
11.87	0.01\\
11.88	0.01\\
11.89	0.01\\
11.9	0.01\\
11.91	0.01\\
11.92	0.01\\
11.93	0.01\\
11.94	0.01\\
11.95	0.01\\
11.96	0.01\\
11.97	0.01\\
11.98	0.01\\
11.99	0.01\\
12	0.01\\
12.01	0.01\\
12.02	0.01\\
12.03	0.01\\
12.04	0.01\\
12.05	0.01\\
12.06	0.01\\
12.07	0.01\\
12.08	0.01\\
12.09	0.01\\
12.1	0.01\\
12.11	0.01\\
12.12	0.01\\
12.13	0.01\\
12.14	0.01\\
12.15	0.01\\
12.16	0.01\\
12.17	0.01\\
12.18	0.01\\
12.19	0.01\\
12.2	0.01\\
12.21	0.01\\
12.22	0.01\\
12.23	0.01\\
12.24	0.01\\
12.25	0.01\\
12.26	0.01\\
12.27	0.01\\
12.28	0.01\\
12.29	0.01\\
12.3	0.01\\
12.31	0.01\\
12.32	0.01\\
12.33	0.01\\
12.34	0.01\\
12.35	0.01\\
12.36	0.01\\
12.37	0.01\\
12.38	0.01\\
12.39	0.01\\
12.4	0.01\\
12.41	0.01\\
12.42	0.01\\
12.43	0.01\\
12.44	0.01\\
12.45	0.01\\
12.46	0.01\\
12.47	0.01\\
12.48	0.01\\
12.49	0.01\\
12.5	0.01\\
12.51	0.01\\
12.52	0.01\\
12.53	0.01\\
12.54	0.01\\
12.55	0.01\\
12.56	0.01\\
12.57	0.01\\
12.58	0.01\\
12.59	0.01\\
12.6	0.01\\
12.61	0.01\\
12.62	0.01\\
12.63	0.01\\
12.64	0.01\\
12.65	0.01\\
12.66	0.01\\
12.67	0.01\\
12.68	0.01\\
12.69	0.01\\
12.7	0.01\\
12.71	0.01\\
12.72	0.01\\
12.73	0.01\\
12.74	0.01\\
12.75	0.01\\
12.76	0.01\\
12.77	0.01\\
12.78	0.01\\
12.79	0.01\\
12.8	0.01\\
12.81	0.01\\
12.82	0.01\\
12.83	0.01\\
12.84	0.01\\
12.85	0.01\\
12.86	0.01\\
12.87	0.01\\
12.88	0.01\\
12.89	0.01\\
12.9	0.01\\
12.91	0.01\\
12.92	0.01\\
12.93	0.01\\
12.94	0.01\\
12.95	0.01\\
12.96	0.01\\
12.97	0.01\\
12.98	0.01\\
12.99	0.01\\
13	0.01\\
13.01	0.01\\
13.02	0.01\\
13.03	0.01\\
13.04	0.01\\
13.05	0.01\\
13.06	0.01\\
13.07	0.01\\
13.08	0.01\\
13.09	0.01\\
13.1	0.01\\
13.11	0.01\\
13.12	0.01\\
13.13	0.01\\
13.14	0.01\\
13.15	0.01\\
13.16	0.01\\
13.17	0.01\\
13.18	0.01\\
13.19	0.01\\
13.2	0.01\\
13.21	0.01\\
13.22	0.01\\
13.23	0.01\\
13.24	0.01\\
13.25	0.01\\
13.26	0.01\\
13.27	0.01\\
13.28	0.01\\
13.29	0.01\\
13.3	0.01\\
13.31	0.01\\
13.32	0.01\\
13.33	0.01\\
13.34	0.01\\
13.35	0.01\\
13.36	0.01\\
13.37	0.01\\
13.38	0.01\\
13.39	0.01\\
13.4	0.01\\
13.41	0.01\\
13.42	0.01\\
13.43	0.01\\
13.44	0.01\\
13.45	0.01\\
13.46	0.01\\
13.47	0.01\\
13.48	0.01\\
13.49	0.01\\
13.5	0.01\\
13.51	0.01\\
13.52	0.01\\
13.53	0.01\\
13.54	0.01\\
13.55	0.01\\
13.56	0.01\\
13.57	0.01\\
13.58	0.01\\
13.59	0.01\\
13.6	0.01\\
13.61	0.01\\
13.62	0.01\\
13.63	0.01\\
13.64	0.01\\
13.65	0.01\\
13.66	0.01\\
13.67	0.01\\
13.68	0.01\\
13.69	0.01\\
13.7	0.01\\
13.71	0.01\\
13.72	0.01\\
13.73	0.01\\
13.74	0.01\\
13.75	0.01\\
13.76	0.01\\
13.77	0.01\\
13.78	0.01\\
13.79	0.01\\
13.8	0.01\\
13.81	0.01\\
13.82	0.01\\
13.83	0.01\\
13.84	0.01\\
13.85	0.01\\
13.86	0.01\\
13.87	0.01\\
13.88	0.01\\
13.89	0.01\\
13.9	0.01\\
13.91	0.01\\
13.92	0.01\\
13.93	0.01\\
13.94	0.01\\
13.95	0.01\\
13.96	0.01\\
13.97	0.01\\
13.98	0.01\\
13.99	0.01\\
14	0.01\\
14.01	0.01\\
14.02	0.01\\
14.03	0.01\\
14.04	0.01\\
14.05	0.01\\
14.06	0.01\\
14.07	0.01\\
14.08	0.01\\
14.09	0.01\\
14.1	0.01\\
14.11	0.01\\
14.12	0.01\\
14.13	0.01\\
14.14	0.01\\
14.15	0.01\\
14.16	0.01\\
14.17	0.01\\
14.18	0.01\\
14.19	0.01\\
14.2	0.01\\
14.21	0.01\\
14.22	0.01\\
14.23	0.01\\
14.24	0.01\\
14.25	0.01\\
14.26	0.01\\
14.27	0.01\\
14.28	0.01\\
14.29	0.01\\
14.3	0.01\\
14.31	0.01\\
14.32	0.01\\
14.33	0.01\\
14.34	0.01\\
14.35	0.01\\
14.36	0.01\\
14.37	0.01\\
14.38	0.01\\
14.39	0.01\\
14.4	0.01\\
14.41	0.01\\
14.42	0.01\\
14.43	0.01\\
14.44	0.01\\
14.45	0.01\\
14.46	0.01\\
14.47	0.01\\
14.48	0.01\\
14.49	0.01\\
14.5	0.01\\
14.51	0.01\\
14.52	0.01\\
14.53	0.01\\
14.54	0.01\\
14.55	0.01\\
14.56	0.01\\
14.57	0.01\\
14.58	0.01\\
14.59	0.01\\
14.6	0.01\\
14.61	0.01\\
14.62	0.01\\
14.63	0.01\\
14.64	0.01\\
14.65	0.01\\
14.66	0.01\\
14.67	0.01\\
14.68	0.01\\
14.69	0.01\\
14.7	0.01\\
14.71	0.01\\
14.72	0.01\\
14.73	0.01\\
14.74	0.01\\
14.75	0.01\\
14.76	0.01\\
14.77	0.01\\
14.78	0.01\\
14.79	0.01\\
14.8	0.01\\
14.81	0.01\\
14.82	0.01\\
14.83	0.01\\
14.84	0.01\\
14.85	0.01\\
14.86	0.01\\
14.87	0.01\\
14.88	0.01\\
14.89	0.01\\
14.9	0.01\\
14.91	0.01\\
14.92	0.01\\
14.93	0.01\\
14.94	0.01\\
14.95	0.01\\
14.96	0.01\\
14.97	0.01\\
14.98	0.01\\
14.99	0.01\\
15	0.01\\
15.01	0.01\\
15.02	0.01\\
15.03	0.01\\
15.04	0.01\\
15.05	0.01\\
15.06	0.01\\
15.07	0.01\\
15.08	0.01\\
15.09	0.01\\
15.1	0.01\\
15.11	0.01\\
15.12	0.01\\
15.13	0.01\\
15.14	0.01\\
15.15	0.01\\
15.16	0.01\\
15.17	0.01\\
15.18	0.01\\
15.19	0.01\\
15.2	0.01\\
15.21	0.01\\
15.22	0.01\\
15.23	0.01\\
15.24	0.01\\
15.25	0.01\\
15.26	0.01\\
15.27	0.01\\
15.28	0.01\\
15.29	0.01\\
15.3	0.01\\
15.31	0.01\\
15.32	0.01\\
15.33	0.01\\
15.34	0.01\\
15.35	0.01\\
15.36	0.01\\
15.37	0.01\\
15.38	0.01\\
15.39	0.01\\
15.4	0.01\\
15.41	0.01\\
15.42	0.01\\
15.43	0.01\\
15.44	0.01\\
15.45	0.01\\
15.46	0.01\\
15.47	0.01\\
15.48	0.01\\
15.49	0.01\\
15.5	0.01\\
15.51	0.01\\
15.52	0.01\\
15.53	0.01\\
15.54	0.01\\
15.55	0.01\\
15.56	0.01\\
15.57	0.01\\
15.58	0.01\\
15.59	0.01\\
15.6	0.01\\
15.61	0.01\\
15.62	0.01\\
15.63	0.01\\
15.64	0.01\\
15.65	0.01\\
15.66	0.01\\
15.67	0.01\\
15.68	0.01\\
15.69	0.01\\
15.7	0.01\\
15.71	0.01\\
15.72	0.01\\
15.73	0.01\\
15.74	0.01\\
15.75	0.01\\
15.76	0.01\\
15.77	0.01\\
15.78	0.01\\
15.79	0.01\\
15.8	0.01\\
15.81	0.01\\
15.82	0.01\\
15.83	0.01\\
15.84	0.01\\
15.85	0.01\\
15.86	0.01\\
15.87	0.01\\
15.88	0.01\\
15.89	0.01\\
15.9	0.01\\
15.91	0.01\\
15.92	0.01\\
15.93	0.01\\
15.94	0.01\\
15.95	0.01\\
15.96	0.01\\
15.97	0.01\\
15.98	0.01\\
15.99	0.01\\
16	0.01\\
16.01	0.01\\
16.02	0.01\\
16.03	0.01\\
16.04	0.01\\
16.05	0.01\\
16.06	0.01\\
16.07	0.01\\
16.08	0.01\\
16.09	0.01\\
16.1	0.01\\
16.11	0.01\\
16.12	0.01\\
16.13	0.01\\
16.14	0.01\\
16.15	0.01\\
16.16	0.01\\
16.17	0.01\\
16.18	0.01\\
16.19	0.01\\
16.2	0.01\\
16.21	0.01\\
16.22	0.01\\
16.23	0.01\\
16.24	0.01\\
16.25	0.01\\
16.26	0.01\\
16.27	0.01\\
16.28	0.01\\
16.29	0.01\\
16.3	0.01\\
16.31	0.01\\
16.32	0.01\\
16.33	0.01\\
16.34	0.01\\
16.35	0.01\\
16.36	0.01\\
16.37	0.01\\
16.38	0.01\\
16.39	0.01\\
16.4	0.01\\
16.41	0.01\\
16.42	0.01\\
16.43	0.01\\
16.44	0.01\\
16.45	0.01\\
16.46	0.01\\
16.47	0.01\\
16.48	0.01\\
16.49	0.01\\
16.5	0.01\\
16.51	0.01\\
16.52	0.01\\
16.53	0.01\\
16.54	0.01\\
16.55	0.01\\
16.56	0.01\\
16.57	0.01\\
16.58	0.01\\
16.59	0.01\\
16.6	0.01\\
16.61	0.01\\
16.62	0.01\\
16.63	0.01\\
16.64	0.01\\
16.65	0.01\\
16.66	0.01\\
16.67	0.01\\
16.68	0.01\\
16.69	0.01\\
16.7	0.01\\
16.71	0.01\\
16.72	0.01\\
16.73	0.01\\
16.74	0.01\\
16.75	0.01\\
16.76	0.01\\
16.77	0.01\\
16.78	0.01\\
16.79	0.01\\
16.8	0.01\\
16.81	0.01\\
16.82	0.01\\
16.83	0.01\\
16.84	0.01\\
16.85	0.01\\
16.86	0.01\\
16.87	0.01\\
16.88	0.01\\
16.89	0.01\\
16.9	0.01\\
16.91	0.01\\
16.92	0.01\\
16.93	0.01\\
16.94	0.01\\
16.95	0.01\\
16.96	0.01\\
16.97	0.01\\
16.98	0.01\\
16.99	0.01\\
17	0.01\\
17.01	0.01\\
17.02	0.01\\
17.03	0.01\\
17.04	0.01\\
17.05	0.01\\
17.06	0.01\\
17.07	0.01\\
17.08	0.01\\
17.09	0.01\\
17.1	0.01\\
17.11	0.01\\
17.12	0.01\\
17.13	0.01\\
17.14	0.01\\
17.15	0.01\\
17.16	0.01\\
17.17	0.01\\
17.18	0.01\\
17.19	0.01\\
17.2	0.01\\
17.21	0.01\\
17.22	0.01\\
17.23	0.01\\
17.24	0.01\\
17.25	0.01\\
17.26	0.01\\
17.27	0.01\\
17.28	0.01\\
17.29	0.01\\
17.3	0.01\\
17.31	0.01\\
17.32	0.01\\
17.33	0.01\\
17.34	0.01\\
17.35	0.01\\
17.36	0.01\\
17.37	0.01\\
17.38	0.01\\
17.39	0.01\\
17.4	0.01\\
17.41	0.01\\
17.42	0.01\\
17.43	0.01\\
17.44	0.01\\
17.45	0.01\\
17.46	0.01\\
17.47	0.01\\
17.48	0.01\\
17.49	0.01\\
17.5	0.01\\
17.51	0.01\\
17.52	0.01\\
17.53	0.01\\
17.54	0.01\\
17.55	0.01\\
17.56	0.01\\
17.57	0.01\\
17.58	0.01\\
17.59	0.01\\
17.6	0.01\\
17.61	0.01\\
17.62	0.01\\
17.63	0.01\\
17.64	0.01\\
17.65	0.01\\
17.66	0.01\\
17.67	0.01\\
17.68	0.01\\
17.69	0.01\\
17.7	0.01\\
17.71	0.01\\
17.72	0.01\\
17.73	0.01\\
17.74	0.01\\
17.75	0.01\\
17.76	0.01\\
17.77	0.01\\
17.78	0.01\\
17.79	0.01\\
17.8	0.01\\
17.81	0.01\\
17.82	0.01\\
17.83	0.01\\
17.84	0.01\\
17.85	0.01\\
17.86	0.01\\
17.87	0.01\\
17.88	0.01\\
17.89	0.01\\
17.9	0.01\\
17.91	0.01\\
17.92	0.01\\
17.93	0.01\\
17.94	0.01\\
17.95	0.01\\
17.96	0.01\\
17.97	0.01\\
17.98	0.01\\
17.99	0.01\\
18	0.01\\
18.01	0.01\\
18.02	0.01\\
18.03	0.01\\
18.04	0.01\\
18.05	0.01\\
18.06	0.01\\
18.07	0.01\\
18.08	0.01\\
18.09	0.01\\
18.1	0.01\\
18.11	0.01\\
18.12	0.01\\
18.13	0.01\\
18.14	0.01\\
18.15	0.01\\
18.16	0.01\\
18.17	0.01\\
18.18	0.01\\
18.19	0.01\\
18.2	0.01\\
18.21	0.01\\
18.22	0.01\\
18.23	0.01\\
18.24	0.01\\
18.25	0.01\\
18.26	0.01\\
18.27	0.01\\
18.28	0.01\\
18.29	0.01\\
18.3	0.01\\
18.31	0.01\\
18.32	0.01\\
18.33	0.01\\
18.34	0.01\\
18.35	0.01\\
18.36	0.01\\
18.37	0.01\\
18.38	0.01\\
18.39	0.01\\
18.4	0.01\\
18.41	0.01\\
18.42	0.01\\
18.43	0.01\\
18.44	0.01\\
18.45	0.01\\
18.46	0.01\\
18.47	0.01\\
18.48	0.01\\
18.49	0.01\\
18.5	0.01\\
18.51	0.01\\
18.52	0.01\\
18.53	0.01\\
18.54	0.01\\
18.55	0.01\\
18.56	0.01\\
18.57	0.01\\
18.58	0.01\\
18.59	0.01\\
18.6	0.01\\
18.61	0.01\\
18.62	0.01\\
18.63	0.01\\
18.64	0.01\\
18.65	0.01\\
18.66	0.01\\
18.67	0.01\\
18.68	0.01\\
18.69	0.01\\
18.7	0.01\\
18.71	0.01\\
18.72	0.01\\
18.73	0.01\\
18.74	0.01\\
18.75	0.01\\
18.76	0.01\\
18.77	0.01\\
18.78	0.01\\
18.79	0.01\\
18.8	0.01\\
18.81	0.01\\
18.82	0.01\\
18.83	0.01\\
18.84	0.01\\
18.85	0.01\\
18.86	0.01\\
18.87	0.01\\
18.88	0.01\\
18.89	0.01\\
18.9	0.01\\
18.91	0.01\\
18.92	0.01\\
18.93	0.01\\
18.94	0.01\\
18.95	0.01\\
18.96	0.01\\
18.97	0.01\\
18.98	0.01\\
18.99	0.01\\
19	0.01\\
19.01	0.01\\
19.02	0.01\\
19.03	0.01\\
19.04	0.01\\
19.05	0.01\\
19.06	0.01\\
19.07	0.01\\
19.08	0.01\\
19.09	0.01\\
19.1	0.01\\
19.11	0.01\\
19.12	0.01\\
19.13	0.01\\
19.14	0.01\\
19.15	0.01\\
19.16	0.01\\
19.17	0.01\\
19.18	0.01\\
19.19	0.01\\
19.2	0.01\\
19.21	0.01\\
19.22	0.01\\
19.23	0.01\\
19.24	0.01\\
19.25	0.01\\
19.26	0.01\\
19.27	0.01\\
19.28	0.01\\
19.29	0.01\\
19.3	0.01\\
19.31	0.01\\
19.32	0.01\\
19.33	0.01\\
19.34	0.01\\
19.35	0.01\\
19.36	0.01\\
19.37	0.01\\
19.38	0.01\\
19.39	0.01\\
19.4	0.01\\
19.41	0.01\\
19.42	0.01\\
19.43	0.01\\
19.44	0.01\\
19.45	0.01\\
19.46	0.01\\
19.47	0.01\\
19.48	0.01\\
19.49	0.01\\
19.5	0.01\\
19.51	0.01\\
19.52	0.01\\
19.53	0.01\\
19.54	0.01\\
19.55	0.01\\
19.56	0.01\\
19.57	0.01\\
19.58	0.01\\
19.59	0.01\\
19.6	0.01\\
19.61	0.01\\
19.62	0.01\\
19.63	0.01\\
19.64	0.01\\
19.65	0.01\\
19.66	0.01\\
19.67	0.01\\
19.68	0.01\\
19.69	0.01\\
19.7	0.01\\
19.71	0.01\\
19.72	0.01\\
19.73	0.01\\
19.74	0.01\\
19.75	0.01\\
19.76	0.01\\
19.77	0.01\\
19.78	0.01\\
19.79	0.01\\
19.8	0.01\\
19.81	0.01\\
19.82	0.01\\
19.83	0.01\\
19.84	0.01\\
19.85	0.01\\
19.86	0.01\\
19.87	0.01\\
19.88	0.01\\
19.89	0.01\\
19.9	0.01\\
19.91	0.01\\
19.92	0.01\\
19.93	0.01\\
19.94	0.01\\
19.95	0.01\\
19.96	0.01\\
19.97	0.01\\
19.98	0.01\\
19.99	0.01\\
20	0.01\\
20.01	0.01\\
20.02	0.01\\
20.03	0.01\\
20.04	0.01\\
20.05	0.01\\
20.06	0.01\\
20.07	0.01\\
20.08	0.01\\
20.09	0.01\\
20.1	0.01\\
20.11	0.01\\
20.12	0.01\\
20.13	0.01\\
20.14	0.01\\
20.15	0.01\\
20.16	0.01\\
20.17	0.01\\
20.18	0.01\\
20.19	0.01\\
20.2	0.01\\
20.21	0.01\\
20.22	0.01\\
20.23	0.01\\
20.24	0.01\\
20.25	0.01\\
20.26	0.01\\
20.27	0.01\\
20.28	0.01\\
20.29	0.01\\
20.3	0.01\\
20.31	0.01\\
20.32	0.01\\
20.33	0.01\\
20.34	0.01\\
20.35	0.01\\
20.36	0.01\\
20.37	0.01\\
20.38	0.01\\
20.39	0.01\\
20.4	0.01\\
20.41	0.01\\
20.42	0.01\\
20.43	0.01\\
20.44	0.01\\
20.45	0.01\\
20.46	0.01\\
20.47	0.01\\
20.48	0.01\\
20.49	0.01\\
20.5	0.01\\
20.51	0.01\\
20.52	0.01\\
20.53	0.01\\
20.54	0.01\\
20.55	0.01\\
20.56	0.01\\
20.57	0.01\\
20.58	0.01\\
20.59	0.01\\
20.6	0.01\\
20.61	0.01\\
20.62	0.01\\
20.63	0.01\\
20.64	0.01\\
20.65	0.01\\
20.66	0.01\\
20.67	0.01\\
20.68	0.01\\
20.69	0.01\\
20.7	0.01\\
20.71	0.01\\
20.72	0.01\\
20.73	0.01\\
20.74	0.01\\
20.75	0.01\\
20.76	0.01\\
20.77	0.01\\
20.78	0.01\\
20.79	0.01\\
20.8	0.01\\
20.81	0.01\\
20.82	0.01\\
20.83	0.01\\
20.84	0.01\\
20.85	0.01\\
20.86	0.01\\
20.87	0.01\\
20.88	0.01\\
20.89	0.01\\
20.9	0.01\\
20.91	0.01\\
20.92	0.01\\
20.93	0.01\\
20.94	0.01\\
20.95	0.01\\
20.96	0.01\\
20.97	0.01\\
20.98	0.01\\
20.99	0.01\\
21	0.01\\
21.01	0.01\\
21.02	0.01\\
21.03	0.01\\
21.04	0.01\\
21.05	0.01\\
21.06	0.01\\
21.07	0.01\\
21.08	0.01\\
21.09	0.01\\
21.1	0.01\\
21.11	0.01\\
21.12	0.01\\
21.13	0.01\\
21.14	0.01\\
21.15	0.01\\
21.16	0.01\\
21.17	0.01\\
21.18	0.01\\
21.19	0.01\\
21.2	0.01\\
21.21	0.01\\
21.22	0.01\\
21.23	0.01\\
21.24	0.01\\
21.25	0.01\\
21.26	0.01\\
21.27	0.01\\
21.28	0.01\\
21.29	0.01\\
21.3	0.01\\
21.31	0.01\\
21.32	0.01\\
21.33	0.01\\
21.34	0.01\\
21.35	0.01\\
21.36	0.01\\
21.37	0.01\\
21.38	0.01\\
21.39	0.01\\
21.4	0.01\\
21.41	0.01\\
21.42	0.01\\
21.43	0.01\\
21.44	0.01\\
21.45	0.01\\
21.46	0.01\\
21.47	0.01\\
21.48	0.01\\
21.49	0.01\\
21.5	0.01\\
21.51	0.01\\
21.52	0.01\\
21.53	0.01\\
21.54	0.01\\
21.55	0.01\\
21.56	0.01\\
21.57	0.01\\
21.58	0.01\\
21.59	0.01\\
21.6	0.01\\
21.61	0.01\\
21.62	0.01\\
21.63	0.01\\
21.64	0.01\\
21.65	0.01\\
21.66	0.01\\
21.67	0.01\\
21.68	0.01\\
21.69	0.01\\
21.7	0.01\\
21.71	0.01\\
21.72	0.01\\
21.73	0.01\\
21.74	0.01\\
21.75	0.01\\
21.76	0.01\\
21.77	0.01\\
21.78	0.01\\
21.79	0.01\\
21.8	0.01\\
21.81	0.01\\
21.82	0.01\\
21.83	0.01\\
21.84	0.01\\
21.85	0.01\\
21.86	0.01\\
21.87	0.01\\
21.88	0.01\\
21.89	0.01\\
21.9	0.01\\
21.91	0.01\\
21.92	0.01\\
21.93	0.01\\
21.94	0.01\\
21.95	0.01\\
21.96	0.01\\
21.97	0.01\\
21.98	0.01\\
21.99	0.01\\
22	0.01\\
22.01	0.01\\
22.02	0.01\\
22.03	0.01\\
22.04	0.01\\
22.05	0.01\\
22.06	0.01\\
22.07	0.01\\
22.08	0.01\\
22.09	0.01\\
22.1	0.01\\
22.11	0.01\\
22.12	0.01\\
22.13	0.01\\
22.14	0.01\\
22.15	0.01\\
22.16	0.01\\
22.17	0.01\\
22.18	0.01\\
22.19	0.01\\
22.2	0.01\\
22.21	0.01\\
22.22	0.01\\
22.23	0.01\\
22.24	0.01\\
22.25	0.01\\
22.26	0.01\\
22.27	0.01\\
22.28	0.01\\
22.29	0.01\\
22.3	0.01\\
22.31	0.01\\
22.32	0.01\\
22.33	0.01\\
22.34	0.01\\
22.35	0.01\\
22.36	0.01\\
22.37	0.01\\
22.38	0.01\\
22.39	0.01\\
22.4	0.01\\
22.41	0.01\\
22.42	0.01\\
22.43	0.01\\
22.44	0.01\\
22.45	0.01\\
22.46	0.01\\
22.47	0.01\\
22.48	0.01\\
22.49	0.01\\
22.5	0.01\\
22.51	0.01\\
22.52	0.01\\
22.53	0.01\\
22.54	0.01\\
22.55	0.01\\
22.56	0.01\\
22.57	0.01\\
22.58	0.01\\
22.59	0.01\\
22.6	0.01\\
22.61	0.01\\
22.62	0.01\\
22.63	0.01\\
22.64	0.01\\
22.65	0.01\\
22.66	0.01\\
22.67	0.01\\
22.68	0.01\\
22.69	0.01\\
22.7	0.01\\
22.71	0.01\\
22.72	0.01\\
22.73	0.01\\
22.74	0.01\\
22.75	0.01\\
22.76	0.01\\
22.77	0.01\\
22.78	0.01\\
22.79	0.01\\
22.8	0.01\\
22.81	0.01\\
22.82	0.01\\
22.83	0.01\\
22.84	0.01\\
22.85	0.01\\
22.86	0.01\\
22.87	0.01\\
22.88	0.01\\
22.89	0.01\\
22.9	0.01\\
22.91	0.01\\
22.92	0.01\\
22.93	0.01\\
22.94	0.01\\
22.95	0.01\\
22.96	0.01\\
22.97	0.01\\
22.98	0.01\\
22.99	0.01\\
23	0.01\\
23.01	0.01\\
23.02	0.01\\
23.03	0.01\\
23.04	0.01\\
23.05	0.01\\
23.06	0.01\\
23.07	0.01\\
23.08	0.01\\
23.09	0.01\\
23.1	0.01\\
23.11	0.01\\
23.12	0.01\\
23.13	0.01\\
23.14	0.01\\
23.15	0.01\\
23.16	0.01\\
23.17	0.01\\
23.18	0.01\\
23.19	0.01\\
23.2	0.01\\
23.21	0.01\\
23.22	0.01\\
23.23	0.01\\
23.24	0.01\\
23.25	0.01\\
23.26	0.01\\
23.27	0.01\\
23.28	0.01\\
23.29	0.01\\
23.3	0.01\\
23.31	0.01\\
23.32	0.01\\
23.33	0.01\\
23.34	0.01\\
23.35	0.01\\
23.36	0.01\\
23.37	0.01\\
23.38	0.01\\
23.39	0.01\\
23.4	0.01\\
23.41	0.01\\
23.42	0.01\\
23.43	0.01\\
23.44	0.01\\
23.45	0.01\\
23.46	0.01\\
23.47	0.01\\
23.48	0.01\\
23.49	0.01\\
23.5	0.01\\
23.51	0.01\\
23.52	0.01\\
23.53	0.01\\
23.54	0.01\\
23.55	0.01\\
23.56	0.01\\
23.57	0.01\\
23.58	0.01\\
23.59	0.01\\
23.6	0.01\\
23.61	0.01\\
23.62	0.01\\
23.63	0.01\\
23.64	0.01\\
23.65	0.01\\
23.66	0.01\\
23.67	0.01\\
23.68	0.01\\
23.69	0.01\\
23.7	0.01\\
23.71	0.01\\
23.72	0.01\\
23.73	0.01\\
23.74	0.01\\
23.75	0.01\\
23.76	0.01\\
23.77	0.01\\
23.78	0.01\\
23.79	0.01\\
23.8	0.01\\
23.81	0.01\\
23.82	0.01\\
23.83	0.01\\
23.84	0.01\\
23.85	0.01\\
23.86	0.01\\
23.87	0.01\\
23.88	0.01\\
23.89	0.01\\
23.9	0.01\\
23.91	0.01\\
23.92	0.01\\
23.93	0.01\\
23.94	0.01\\
23.95	0.01\\
23.96	0.01\\
23.97	0.01\\
23.98	0.01\\
23.99	0.01\\
24	0.01\\
24.01	0.01\\
24.02	0.01\\
24.03	0.01\\
24.04	0.01\\
24.05	0.01\\
24.06	0.01\\
24.07	0.01\\
24.08	0.01\\
24.09	0.01\\
24.1	0.01\\
24.11	0.01\\
24.12	0.01\\
24.13	0.01\\
24.14	0.01\\
24.15	0.01\\
24.16	0.01\\
24.17	0.01\\
24.18	0.01\\
24.19	0.01\\
24.2	0.01\\
24.21	0.01\\
24.22	0.01\\
24.23	0.01\\
24.24	0.01\\
24.25	0.01\\
24.26	0.01\\
24.27	0.01\\
24.28	0.01\\
24.29	0.01\\
24.3	0.01\\
24.31	0.01\\
24.32	0.01\\
24.33	0.01\\
24.34	0.01\\
24.35	0.01\\
24.36	0.01\\
24.37	0.01\\
24.38	0.01\\
24.39	0.01\\
24.4	0.01\\
24.41	0.01\\
24.42	0.01\\
24.43	0.01\\
24.44	0.01\\
24.45	0.01\\
24.46	0.01\\
24.47	0.01\\
24.48	0.01\\
24.49	0.01\\
24.5	0.01\\
24.51	0.01\\
24.52	0.01\\
24.53	0.01\\
24.54	0.01\\
24.55	0.01\\
24.56	0.01\\
24.57	0.01\\
24.58	0.01\\
24.59	0.01\\
24.6	0.01\\
24.61	0.01\\
24.62	0.01\\
24.63	0.01\\
24.64	0.01\\
24.65	0.01\\
24.66	0.01\\
24.67	0.01\\
24.68	0.01\\
24.69	0.01\\
24.7	0.01\\
24.71	0.01\\
24.72	0.01\\
24.73	0.01\\
24.74	0.01\\
24.75	0.01\\
24.76	0.01\\
24.77	0.01\\
24.78	0.01\\
24.79	0.01\\
24.8	0.01\\
24.81	0.01\\
24.82	0.01\\
24.83	0.01\\
24.84	0.01\\
24.85	0.01\\
24.86	0.01\\
24.87	0.01\\
24.88	0.01\\
24.89	0.01\\
24.9	0.01\\
24.91	0.01\\
24.92	0.01\\
24.93	0.01\\
24.94	0.01\\
24.95	0.01\\
24.96	0.01\\
24.97	0.01\\
24.98	0.01\\
24.99	0.01\\
25	0.01\\
25.01	0.01\\
25.02	0.01\\
25.03	0.01\\
25.04	0.01\\
25.05	0.01\\
25.06	0.01\\
25.07	0.01\\
25.08	0.01\\
25.09	0.01\\
25.1	0.01\\
25.11	0.01\\
25.12	0.01\\
25.13	0.01\\
25.14	0.01\\
25.15	0.01\\
25.16	0.01\\
25.17	0.01\\
25.18	0.01\\
25.19	0.01\\
25.2	0.01\\
25.21	0.01\\
25.22	0.01\\
25.23	0.01\\
25.24	0.01\\
25.25	0.01\\
25.26	0.01\\
25.27	0.01\\
25.28	0.01\\
25.29	0.01\\
25.3	0.01\\
25.31	0.01\\
25.32	0.01\\
25.33	0.01\\
25.34	0.01\\
25.35	0.01\\
25.36	0.01\\
25.37	0.01\\
25.38	0.01\\
25.39	0.01\\
25.4	0.01\\
25.41	0.01\\
25.42	0.01\\
25.43	0.01\\
25.44	0.01\\
25.45	0.01\\
25.46	0.01\\
25.47	0.01\\
25.48	0.01\\
25.49	0.01\\
25.5	0.01\\
25.51	0.01\\
25.52	0.01\\
25.53	0.01\\
25.54	0.01\\
25.55	0.01\\
25.56	0.01\\
25.57	0.01\\
25.58	0.01\\
25.59	0.01\\
25.6	0.01\\
25.61	0.01\\
25.62	0.01\\
25.63	0.01\\
25.64	0.01\\
25.65	0.01\\
25.66	0.01\\
25.67	0.01\\
25.68	0.01\\
25.69	0.01\\
25.7	0.01\\
25.71	0.01\\
25.72	0.01\\
25.73	0.01\\
25.74	0.01\\
25.75	0.01\\
25.76	0.01\\
25.77	0.01\\
25.78	0.01\\
25.79	0.01\\
25.8	0.01\\
25.81	0.01\\
25.82	0.01\\
25.83	0.01\\
25.84	0.01\\
25.85	0.01\\
25.86	0.01\\
25.87	0.01\\
25.88	0.01\\
25.89	0.01\\
25.9	0.01\\
25.91	0.01\\
25.92	0.01\\
25.93	0.01\\
25.94	0.01\\
25.95	0.01\\
25.96	0.01\\
25.97	0.01\\
25.98	0.01\\
25.99	0.01\\
26	0.01\\
26.01	0.01\\
26.02	0.01\\
26.03	0.01\\
26.04	0.01\\
26.05	0.01\\
26.06	0.01\\
26.07	0.01\\
26.08	0.01\\
26.09	0.01\\
26.1	0.01\\
26.11	0.01\\
26.12	0.01\\
26.13	0.01\\
26.14	0.01\\
26.15	0.01\\
26.16	0.01\\
26.17	0.01\\
26.18	0.01\\
26.19	0.01\\
26.2	0.01\\
26.21	0.01\\
26.22	0.01\\
26.23	0.01\\
26.24	0.01\\
26.25	0.01\\
26.26	0.01\\
26.27	0.01\\
26.28	0.01\\
26.29	0.01\\
26.3	0.01\\
26.31	0.01\\
26.32	0.01\\
26.33	0.01\\
26.34	0.01\\
26.35	0.01\\
26.36	0.01\\
26.37	0.01\\
26.38	0.01\\
26.39	0.01\\
26.4	0.01\\
26.41	0.01\\
26.42	0.01\\
26.43	0.01\\
26.44	0.01\\
26.45	0.01\\
26.46	0.01\\
26.47	0.01\\
26.48	0.01\\
26.49	0.01\\
26.5	0.01\\
26.51	0.01\\
26.52	0.01\\
26.53	0.01\\
26.54	0.01\\
26.55	0.01\\
26.56	0.01\\
26.57	0.01\\
26.58	0.01\\
26.59	0.01\\
26.6	0.01\\
26.61	0.01\\
26.62	0.01\\
26.63	0.01\\
26.64	0.01\\
26.65	0.01\\
26.66	0.01\\
26.67	0.01\\
26.68	0.01\\
26.69	0.01\\
26.7	0.01\\
26.71	0.01\\
26.72	0.01\\
26.73	0.01\\
26.74	0.01\\
26.75	0.01\\
26.76	0.01\\
26.77	0.01\\
26.78	0.01\\
26.79	0.01\\
26.8	0.01\\
26.81	0.01\\
26.82	0.01\\
26.83	0.01\\
26.84	0.01\\
26.85	0.01\\
26.86	0.01\\
26.87	0.01\\
26.88	0.01\\
26.89	0.01\\
26.9	0.01\\
26.91	0.01\\
26.92	0.01\\
26.93	0.01\\
26.94	0.01\\
26.95	0.01\\
26.96	0.01\\
26.97	0.01\\
26.98	0.01\\
26.99	0.01\\
27	0.01\\
27.01	0.01\\
27.02	0.01\\
27.03	0.01\\
27.04	0.01\\
27.05	0.01\\
27.06	0.01\\
27.07	0.01\\
27.08	0.01\\
27.09	0.01\\
27.1	0.01\\
27.11	0.01\\
27.12	0.01\\
27.13	0.01\\
27.14	0.01\\
27.15	0.01\\
27.16	0.01\\
27.17	0.01\\
27.18	0.01\\
27.19	0.01\\
27.2	0.01\\
27.21	0.01\\
27.22	0.01\\
27.23	0.01\\
27.24	0.01\\
27.25	0.01\\
27.26	0.01\\
27.27	0.01\\
27.28	0.01\\
27.29	0.01\\
27.3	0.01\\
27.31	0.01\\
27.32	0.01\\
27.33	0.01\\
27.34	0.01\\
27.35	0.01\\
27.36	0.01\\
27.37	0.01\\
27.38	0.01\\
27.39	0.01\\
27.4	0.01\\
27.41	0.01\\
27.42	0.01\\
27.43	0.01\\
27.44	0.01\\
27.45	0.01\\
27.46	0.01\\
27.47	0.01\\
27.48	0.01\\
27.49	0.01\\
27.5	0.01\\
27.51	0.01\\
27.52	0.01\\
27.53	0.01\\
27.54	0.01\\
27.55	0.01\\
27.56	0.01\\
27.57	0.01\\
27.58	0.01\\
27.59	0.01\\
27.6	0.01\\
27.61	0.01\\
27.62	0.01\\
27.63	0.01\\
27.64	0.01\\
27.65	0.01\\
27.66	0.01\\
27.67	0.01\\
27.68	0.01\\
27.69	0.01\\
27.7	0.01\\
27.71	0.01\\
27.72	0.01\\
27.73	0.01\\
27.74	0.01\\
27.75	0.01\\
27.76	0.01\\
27.77	0.01\\
27.78	0.01\\
27.79	0.01\\
27.8	0.01\\
27.81	0.01\\
27.82	0.01\\
27.83	0.01\\
27.84	0.01\\
27.85	0.01\\
27.86	0.01\\
27.87	0.01\\
27.88	0.01\\
27.89	0.01\\
27.9	0.01\\
27.91	0.01\\
27.92	0.01\\
27.93	0.01\\
27.94	0.01\\
27.95	0.01\\
27.96	0.01\\
27.97	0.01\\
27.98	0.01\\
27.99	0.01\\
28	0.01\\
28.01	0.01\\
28.02	0.01\\
28.03	0.01\\
28.04	0.01\\
28.05	0.01\\
28.06	0.01\\
28.07	0.01\\
28.08	0.01\\
28.09	0.01\\
28.1	0.01\\
28.11	0.01\\
28.12	0.01\\
28.13	0.01\\
28.14	0.01\\
28.15	0.01\\
28.16	0.01\\
28.17	0.01\\
28.18	0.01\\
28.19	0.01\\
28.2	0.01\\
28.21	0.01\\
28.22	0.01\\
28.23	0.01\\
28.24	0.01\\
28.25	0.01\\
28.26	0.01\\
28.27	0.01\\
28.28	0.01\\
28.29	0.01\\
28.3	0.01\\
28.31	0.01\\
28.32	0.01\\
28.33	0.01\\
28.34	0.01\\
28.35	0.01\\
28.36	0.01\\
28.37	0.01\\
28.38	0.01\\
28.39	0.01\\
28.4	0.01\\
28.41	0.01\\
28.42	0.01\\
28.43	0.01\\
28.44	0.01\\
28.45	0.01\\
28.46	0.01\\
28.47	0.01\\
28.48	0.01\\
28.49	0.01\\
28.5	0.01\\
28.51	0.01\\
28.52	0.01\\
28.53	0.01\\
28.54	0.01\\
28.55	0.01\\
28.56	0.01\\
28.57	0.01\\
28.58	0.01\\
28.59	0.01\\
28.6	0.01\\
28.61	0.01\\
28.62	0.01\\
28.63	0.01\\
28.64	0.01\\
28.65	0.01\\
28.66	0.01\\
28.67	0.01\\
28.68	0.01\\
28.69	0.01\\
28.7	0.01\\
28.71	0.01\\
28.72	0.01\\
28.73	0.01\\
28.74	0.01\\
28.75	0.01\\
28.76	0.01\\
28.77	0.01\\
28.78	0.01\\
28.79	0.01\\
28.8	0.01\\
28.81	0.01\\
28.82	0.01\\
28.83	0.01\\
28.84	0.01\\
28.85	0.01\\
28.86	0.01\\
28.87	0.01\\
28.88	0.01\\
28.89	0.01\\
28.9	0.01\\
28.91	0.01\\
28.92	0.01\\
28.93	0.01\\
28.94	0.01\\
28.95	0.01\\
28.96	0.01\\
28.97	0.01\\
28.98	0.01\\
28.99	0.01\\
29	0.01\\
29.01	0.01\\
29.02	0.01\\
29.03	0.01\\
29.04	0.01\\
29.05	0.01\\
29.06	0.01\\
29.07	0.01\\
29.08	0.01\\
29.09	0.01\\
29.1	0.01\\
29.11	0.01\\
29.12	0.01\\
29.13	0.01\\
29.14	0.01\\
29.15	0.01\\
29.16	0.01\\
29.17	0.01\\
29.18	0.01\\
29.19	0.01\\
29.2	0.01\\
29.21	0.01\\
29.22	0.01\\
29.23	0.01\\
29.24	0.01\\
29.25	0.01\\
29.26	0.01\\
29.27	0.01\\
29.28	0.01\\
29.29	0.01\\
29.3	0.01\\
29.31	0.01\\
29.32	0.01\\
29.33	0.01\\
29.34	0.01\\
29.35	0.01\\
29.36	0.01\\
29.37	0.01\\
29.38	0.01\\
29.39	0.01\\
29.4	0.01\\
29.41	0.01\\
29.42	0.01\\
29.43	0.01\\
29.44	0.01\\
29.45	0.01\\
29.46	0.01\\
29.47	0.01\\
29.48	0.01\\
29.49	0.01\\
29.5	0.01\\
29.51	0.01\\
29.52	0.01\\
29.53	0.01\\
29.54	0.01\\
29.55	0.01\\
29.56	0.01\\
29.57	0.01\\
29.58	0.01\\
29.59	0.01\\
29.6	0.01\\
29.61	0.01\\
29.62	0.01\\
29.63	0.01\\
29.64	0.01\\
29.65	0.01\\
29.66	0.01\\
29.67	0.01\\
29.68	0.01\\
29.69	0.01\\
29.7	0.01\\
29.71	0.01\\
29.72	0.01\\
29.73	0.01\\
29.74	0.01\\
29.75	0.01\\
29.76	0.01\\
29.77	0.01\\
29.78	0.01\\
29.79	0.01\\
29.8	0.01\\
29.81	0.01\\
29.82	0.01\\
29.83	0.01\\
29.84	0.01\\
29.85	0.01\\
29.86	0.01\\
29.87	0.01\\
29.88	0.01\\
29.89	0.01\\
29.9	0.01\\
29.91	0.01\\
29.92	0.01\\
29.93	0.01\\
29.94	0.01\\
29.95	0.01\\
29.96	0.01\\
29.97	0.01\\
29.98	0.01\\
29.99	0.01\\
30	0.01\\
30.01	0.01\\
30.02	0.01\\
30.03	0.01\\
30.04	0.01\\
30.05	0.01\\
30.06	0.01\\
30.07	0.01\\
30.08	0.01\\
30.09	0.01\\
30.1	0.01\\
30.11	0.01\\
30.12	0.01\\
30.13	0.01\\
30.14	0.01\\
30.15	0.01\\
30.16	0.01\\
30.17	0.01\\
30.18	0.01\\
30.19	0.01\\
30.2	0.01\\
30.21	0.01\\
30.22	0.01\\
30.23	0.01\\
30.24	0.01\\
30.25	0.01\\
30.26	0.01\\
30.27	0.01\\
30.28	0.01\\
30.29	0.01\\
30.3	0.01\\
30.31	0.01\\
30.32	0.01\\
30.33	0.01\\
30.34	0.01\\
30.35	0.01\\
30.36	0.01\\
30.37	0.01\\
30.38	0.01\\
30.39	0.01\\
30.4	0.01\\
30.41	0.01\\
30.42	0.01\\
30.43	0.01\\
30.44	0.01\\
30.45	0.01\\
30.46	0.01\\
30.47	0.01\\
30.48	0.01\\
30.49	0.01\\
30.5	0.01\\
30.51	0.01\\
30.52	0.01\\
30.53	0.01\\
30.54	0.01\\
30.55	0.01\\
30.56	0.01\\
30.57	0.01\\
30.58	0.01\\
30.59	0.01\\
30.6	0.01\\
30.61	0.01\\
30.62	0.01\\
30.63	0.01\\
30.64	0.01\\
30.65	0.01\\
30.66	0.01\\
30.67	0.01\\
30.68	0.01\\
30.69	0.01\\
30.7	0.01\\
30.71	0.01\\
30.72	0.01\\
30.73	0.01\\
30.74	0.01\\
30.75	0.01\\
30.76	0.01\\
30.77	0.01\\
30.78	0.01\\
30.79	0.01\\
30.8	0.01\\
30.81	0.01\\
30.82	0.01\\
30.83	0.01\\
30.84	0.01\\
30.85	0.01\\
30.86	0.01\\
30.87	0.01\\
30.88	0.01\\
30.89	0.01\\
30.9	0.01\\
30.91	0.01\\
30.92	0.01\\
30.93	0.01\\
30.94	0.01\\
30.95	0.01\\
30.96	0.01\\
30.97	0.01\\
30.98	0.01\\
30.99	0.01\\
31	0.01\\
31.01	0.01\\
31.02	0.01\\
31.03	0.01\\
31.04	0.01\\
31.05	0.01\\
31.06	0.01\\
31.07	0.01\\
31.08	0.01\\
31.09	0.01\\
31.1	0.01\\
31.11	0.01\\
31.12	0.01\\
31.13	0.01\\
31.14	0.01\\
31.15	0.01\\
31.16	0.01\\
31.17	0.01\\
31.18	0.01\\
31.19	0.01\\
31.2	0.01\\
31.21	0.01\\
31.22	0.01\\
31.23	0.01\\
31.24	0.01\\
31.25	0.01\\
31.26	0.01\\
31.27	0.01\\
31.28	0.01\\
31.29	0.01\\
31.3	0.01\\
31.31	0.01\\
31.32	0.01\\
31.33	0.01\\
31.34	0.01\\
31.35	0.01\\
31.36	0.01\\
31.37	0.01\\
31.38	0.01\\
31.39	0.01\\
31.4	0.01\\
31.41	0.01\\
31.42	0.01\\
31.43	0.01\\
31.44	0.01\\
31.45	0.01\\
31.46	0.01\\
31.47	0.01\\
31.48	0.01\\
31.49	0.01\\
31.5	0.01\\
31.51	0.01\\
31.52	0.01\\
31.53	0.01\\
31.54	0.01\\
31.55	0.01\\
31.56	0.01\\
31.57	0.01\\
31.58	0.01\\
31.59	0.01\\
31.6	0.01\\
31.61	0.01\\
31.62	0.01\\
31.63	0.01\\
31.64	0.01\\
31.65	0.01\\
31.66	0.01\\
31.67	0.01\\
31.68	0.01\\
31.69	0.01\\
31.7	0.01\\
31.71	0.01\\
31.72	0.01\\
31.73	0.01\\
31.74	0.01\\
31.75	0.01\\
31.76	0.01\\
31.77	0.01\\
31.78	0.01\\
31.79	0.01\\
31.8	0.01\\
31.81	0.01\\
31.82	0.01\\
31.83	0.01\\
31.84	0.01\\
31.85	0.01\\
31.86	0.01\\
31.87	0.01\\
31.88	0.01\\
31.89	0.01\\
31.9	0.01\\
31.91	0.01\\
31.92	0.01\\
31.93	0.01\\
31.94	0.01\\
31.95	0.01\\
31.96	0.01\\
31.97	0.01\\
31.98	0.01\\
31.99	0.01\\
32	0.01\\
32.01	0.01\\
32.02	0.01\\
32.03	0.01\\
32.04	0.01\\
32.05	0.01\\
32.06	0.01\\
32.07	0.01\\
32.08	0.01\\
32.09	0.01\\
32.1	0.01\\
32.11	0.01\\
32.12	0.01\\
32.13	0.01\\
32.14	0.01\\
32.15	0.01\\
32.16	0.01\\
32.17	0.01\\
32.18	0.01\\
32.19	0.01\\
32.2	0.01\\
32.21	0.01\\
32.22	0.01\\
32.23	0.01\\
32.24	0.01\\
32.25	0.01\\
32.26	0.01\\
32.27	0.01\\
32.28	0.01\\
32.29	0.01\\
32.3	0.01\\
32.31	0.01\\
32.32	0.01\\
32.33	0.01\\
32.34	0.01\\
32.35	0.01\\
32.36	0.01\\
32.37	0.01\\
32.38	0.01\\
32.39	0.01\\
32.4	0.01\\
32.41	0.01\\
32.42	0.01\\
32.43	0.01\\
32.44	0.01\\
32.45	0.01\\
32.46	0.01\\
32.47	0.01\\
32.48	0.01\\
32.49	0.01\\
32.5	0.01\\
32.51	0.01\\
32.52	0.01\\
32.53	0.01\\
32.54	0.01\\
32.55	0.01\\
32.56	0.01\\
32.57	0.01\\
32.58	0.01\\
32.59	0.01\\
32.6	0.01\\
32.61	0.01\\
32.62	0.01\\
32.63	0.01\\
32.64	0.01\\
32.65	0.01\\
32.66	0.01\\
32.67	0.01\\
32.68	0.01\\
32.69	0.01\\
32.7	0.01\\
32.71	0.01\\
32.72	0.01\\
32.73	0.01\\
32.74	0.01\\
32.75	0.01\\
32.76	0.01\\
32.77	0.01\\
32.78	0.01\\
32.79	0.01\\
32.8	0.01\\
32.81	0.01\\
32.82	0.01\\
32.83	0.01\\
32.84	0.01\\
32.85	0.01\\
32.86	0.01\\
32.87	0.01\\
32.88	0.01\\
32.89	0.01\\
32.9	0.01\\
32.91	0.01\\
32.92	0.01\\
32.93	0.01\\
32.94	0.01\\
32.95	0.01\\
32.96	0.01\\
32.97	0.01\\
32.98	0.01\\
32.99	0.01\\
33	0.01\\
33.01	0.01\\
33.02	0.01\\
33.03	0.01\\
33.04	0.01\\
33.05	0.01\\
33.06	0.01\\
33.07	0.01\\
33.08	0.01\\
33.09	0.01\\
33.1	0.01\\
33.11	0.01\\
33.12	0.01\\
33.13	0.01\\
33.14	0.01\\
33.15	0.01\\
33.16	0.01\\
33.17	0.01\\
33.18	0.01\\
33.19	0.01\\
33.2	0.01\\
33.21	0.01\\
33.22	0.01\\
33.23	0.01\\
33.24	0.01\\
33.25	0.01\\
33.26	0.01\\
33.27	0.01\\
33.28	0.01\\
33.29	0.01\\
33.3	0.01\\
33.31	0.01\\
33.32	0.01\\
33.33	0.01\\
33.34	0.01\\
33.35	0.01\\
33.36	0.01\\
33.37	0.01\\
33.38	0.01\\
33.39	0.01\\
33.4	0.01\\
33.41	0.01\\
33.42	0.01\\
33.43	0.01\\
33.44	0.01\\
33.45	0.01\\
33.46	0.01\\
33.47	0.01\\
33.48	0.01\\
33.49	0.01\\
33.5	0.01\\
33.51	0.01\\
33.52	0.01\\
33.53	0.01\\
33.54	0.01\\
33.55	0.01\\
33.56	0.01\\
33.57	0.01\\
33.58	0.01\\
33.59	0.01\\
33.6	0.01\\
33.61	0.01\\
33.62	0.01\\
33.63	0.01\\
33.64	0.01\\
33.65	0.01\\
33.66	0.01\\
33.67	0.01\\
33.68	0.01\\
33.69	0.01\\
33.7	0.01\\
33.71	0.01\\
33.72	0.01\\
33.73	0.01\\
33.74	0.01\\
33.75	0.01\\
33.76	0.01\\
33.77	0.01\\
33.78	0.01\\
33.79	0.01\\
33.8	0.01\\
33.81	0.01\\
33.82	0.01\\
33.83	0.01\\
33.84	0.01\\
33.85	0.01\\
33.86	0.01\\
33.87	0.01\\
33.88	0.01\\
33.89	0.01\\
33.9	0.01\\
33.91	0.01\\
33.92	0.01\\
33.93	0.01\\
33.94	0.01\\
33.95	0.01\\
33.96	0.01\\
33.97	0.01\\
33.98	0.01\\
33.99	0.01\\
34	0.01\\
34.01	0.01\\
34.02	0.01\\
34.03	0.01\\
34.04	0.01\\
34.05	0.01\\
34.06	0.01\\
34.07	0.01\\
34.08	0.01\\
34.09	0.01\\
34.1	0.01\\
34.11	0.01\\
34.12	0.01\\
34.13	0.01\\
34.14	0.01\\
34.15	0.01\\
34.16	0.01\\
34.17	0.01\\
34.18	0.01\\
34.19	0.01\\
34.2	0.01\\
34.21	0.01\\
34.22	0.01\\
34.23	0.01\\
34.24	0.01\\
34.25	0.01\\
34.26	0.01\\
34.27	0.01\\
34.28	0.01\\
34.29	0.01\\
34.3	0.01\\
34.31	0.01\\
34.32	0.01\\
34.33	0.01\\
34.34	0.01\\
34.35	0.01\\
34.36	0.01\\
34.37	0.01\\
34.38	0.01\\
34.39	0.01\\
34.4	0.01\\
34.41	0.01\\
34.42	0.01\\
34.43	0.01\\
34.44	0.01\\
34.45	0.01\\
34.46	0.01\\
34.47	0.01\\
34.48	0.01\\
34.49	0.01\\
34.5	0.01\\
34.51	0.01\\
34.52	0.01\\
34.53	0.01\\
34.54	0.01\\
34.55	0.01\\
34.56	0.01\\
34.57	0.01\\
34.58	0.01\\
34.59	0.01\\
34.6	0.01\\
34.61	0.01\\
34.62	0.01\\
34.63	0.01\\
34.64	0.01\\
34.65	0.01\\
34.66	0.01\\
34.67	0.01\\
34.68	0.01\\
34.69	0.01\\
34.7	0.01\\
34.71	0.01\\
34.72	0.01\\
34.73	0.01\\
34.74	0.01\\
34.75	0.01\\
34.76	0.01\\
34.77	0.01\\
34.78	0.01\\
34.79	0.01\\
34.8	0.01\\
34.81	0.01\\
34.82	0.01\\
34.83	0.01\\
34.84	0.01\\
34.85	0.01\\
34.86	0.01\\
34.87	0.01\\
34.88	0.01\\
34.89	0.01\\
34.9	0.01\\
34.91	0.01\\
34.92	0.01\\
34.93	0.01\\
34.94	0.01\\
34.95	0.01\\
34.96	0.01\\
34.97	0.01\\
34.98	0.01\\
34.99	0.01\\
35	0.01\\
35.01	0.01\\
35.02	0.01\\
35.03	0.01\\
35.04	0.01\\
35.05	0.01\\
35.06	0.01\\
35.07	0.01\\
35.08	0.01\\
35.09	0.01\\
35.1	0.01\\
35.11	0.01\\
35.12	0.01\\
35.13	0.01\\
35.14	0.01\\
35.15	0.01\\
35.16	0.01\\
35.17	0.01\\
35.18	0.01\\
35.19	0.01\\
35.2	0.01\\
35.21	0.01\\
35.22	0.01\\
35.23	0.01\\
35.24	0.01\\
35.25	0.01\\
35.26	0.01\\
35.27	0.01\\
35.28	0.01\\
35.29	0.01\\
35.3	0.01\\
35.31	0.01\\
35.32	0.01\\
35.33	0.01\\
35.34	0.01\\
35.35	0.01\\
35.36	0.01\\
35.37	0.01\\
35.38	0.01\\
35.39	0.01\\
35.4	0.01\\
35.41	0.01\\
35.42	0.01\\
35.43	0.01\\
35.44	0.01\\
35.45	0.01\\
35.46	0.01\\
35.47	0.01\\
35.48	0.01\\
35.49	0.01\\
35.5	0.01\\
35.51	0.01\\
35.52	0.01\\
35.53	0.01\\
35.54	0.01\\
35.55	0.01\\
35.56	0.01\\
35.57	0.01\\
35.58	0.01\\
35.59	0.01\\
35.6	0.01\\
35.61	0.01\\
35.62	0.01\\
35.63	0.01\\
35.64	0.01\\
35.65	0.01\\
35.66	0.01\\
35.67	0.01\\
35.68	0.01\\
35.69	0.01\\
35.7	0.01\\
35.71	0.01\\
35.72	0.01\\
35.73	0.01\\
35.74	0.01\\
35.75	0.01\\
35.76	0.01\\
35.77	0.01\\
35.78	0.01\\
35.79	0.01\\
35.8	0.01\\
35.81	0.01\\
35.82	0.01\\
35.83	0.01\\
35.84	0.01\\
35.85	0.01\\
35.86	0.01\\
35.87	0.01\\
35.88	0.01\\
35.89	0.01\\
35.9	0.01\\
35.91	0.01\\
35.92	0.01\\
35.93	0.01\\
35.94	0.01\\
35.95	0.01\\
35.96	0.01\\
35.97	0.01\\
35.98	0.01\\
35.99	0.01\\
36	0.01\\
36.01	0.01\\
36.02	0.01\\
36.03	0.01\\
36.04	0.01\\
36.05	0.01\\
36.06	0.01\\
36.07	0.01\\
36.08	0.01\\
36.09	0.01\\
36.1	0.01\\
36.11	0.01\\
36.12	0.01\\
36.13	0.01\\
36.14	0.01\\
36.15	0.01\\
36.16	0.01\\
36.17	0.01\\
36.18	0.01\\
36.19	0.01\\
36.2	0.01\\
36.21	0.01\\
36.22	0.01\\
36.23	0.01\\
36.24	0.01\\
36.25	0.01\\
36.26	0.01\\
36.27	0.01\\
36.28	0.01\\
36.29	0.01\\
36.3	0.01\\
36.31	0.01\\
36.32	0.01\\
36.33	0.01\\
36.34	0.01\\
36.35	0.01\\
36.36	0.01\\
36.37	0.01\\
36.38	0.01\\
36.39	0.01\\
36.4	0.01\\
36.41	0.01\\
36.42	0.01\\
36.43	0.01\\
36.44	0.01\\
36.45	0.01\\
36.46	0.01\\
36.47	0.01\\
36.48	0.01\\
36.49	0.01\\
36.5	0.01\\
36.51	0.01\\
36.52	0.01\\
36.53	0.01\\
36.54	0.01\\
36.55	0.01\\
36.56	0.01\\
36.57	0.01\\
36.58	0.01\\
36.59	0.01\\
36.6	0.01\\
36.61	0.01\\
36.62	0.01\\
36.63	0.01\\
36.64	0.01\\
36.65	0.01\\
36.66	0.01\\
36.67	0.01\\
36.68	0.01\\
36.69	0.01\\
36.7	0.01\\
36.71	0.01\\
36.72	0.01\\
36.73	0.01\\
36.74	0.01\\
36.75	0.01\\
36.76	0.01\\
36.77	0.01\\
36.78	0.01\\
36.79	0.01\\
36.8	0.01\\
36.81	0.01\\
36.82	0.01\\
36.83	0.01\\
36.84	0.01\\
36.85	0.01\\
36.86	0.01\\
36.87	0.01\\
36.88	0.01\\
36.89	0.01\\
36.9	0.01\\
36.91	0.01\\
36.92	0.01\\
36.93	0.01\\
36.94	0.01\\
36.95	0.01\\
36.96	0.01\\
36.97	0.01\\
36.98	0.01\\
36.99	0.01\\
37	0.01\\
37.01	0.01\\
37.02	0.01\\
37.03	0.01\\
37.04	0.01\\
37.05	0.01\\
37.06	0.01\\
37.07	0.01\\
37.08	0.01\\
37.09	0.01\\
37.1	0.01\\
37.11	0.01\\
37.12	0.01\\
37.13	0.01\\
37.14	0.01\\
37.15	0.01\\
37.16	0.01\\
37.17	0.01\\
37.18	0.01\\
37.19	0.01\\
37.2	0.01\\
37.21	0.01\\
37.22	0.01\\
37.23	0.01\\
37.24	0.01\\
37.25	0.01\\
37.26	0.01\\
37.27	0.01\\
37.28	0.01\\
37.29	0.01\\
37.3	0.01\\
37.31	0.01\\
37.32	0.01\\
37.33	0.01\\
37.34	0.01\\
37.35	0.01\\
37.36	0.01\\
37.37	0.01\\
37.38	0.01\\
37.39	0.01\\
37.4	0.01\\
37.41	0.01\\
37.42	0.01\\
37.43	0.01\\
37.44	0.01\\
37.45	0.01\\
37.46	0.01\\
37.47	0.01\\
37.48	0.01\\
37.49	0.01\\
37.5	0.01\\
37.51	0.01\\
37.52	0.01\\
37.53	0.01\\
37.54	0.01\\
37.55	0.01\\
37.56	0.01\\
37.57	0.01\\
37.58	0.01\\
37.59	0.01\\
37.6	0.01\\
37.61	0.01\\
37.62	0.01\\
37.63	0.01\\
37.64	0.01\\
37.65	0.01\\
37.66	0.01\\
37.67	0.01\\
37.68	0.01\\
37.69	0.01\\
37.7	0.01\\
37.71	0.01\\
37.72	0.01\\
37.73	0.01\\
37.74	0.01\\
37.75	0.01\\
37.76	0.01\\
37.77	0.01\\
37.78	0.01\\
37.79	0.01\\
37.8	0.01\\
37.81	0.01\\
37.82	0.01\\
37.83	0.01\\
37.84	0.01\\
37.85	0.01\\
37.86	0.01\\
37.87	0.01\\
37.88	0.01\\
37.89	0.01\\
37.9	0.01\\
37.91	0.01\\
37.92	0.01\\
37.93	0.01\\
37.94	0.01\\
37.95	0.01\\
37.96	0.01\\
37.97	0.01\\
37.98	0.01\\
37.99	0.01\\
38	0.01\\
38.01	0.01\\
38.02	0.01\\
38.03	0.01\\
38.04	0.01\\
38.05	0.01\\
38.06	0.01\\
38.07	0.01\\
38.08	0.01\\
38.09	0.01\\
38.1	0.01\\
38.11	0.01\\
38.12	0.01\\
38.13	0.01\\
38.14	0.01\\
38.15	0.01\\
38.16	0.01\\
38.17	0.01\\
38.18	0.01\\
38.19	0.01\\
38.2	0.01\\
38.21	0.01\\
38.22	0.01\\
38.23	0.01\\
38.24	0.01\\
38.25	0.01\\
38.26	0.01\\
38.27	0.01\\
38.28	0.01\\
38.29	0.01\\
38.3	0.01\\
38.31	0.01\\
38.32	0.01\\
38.33	0.01\\
38.34	0.01\\
38.35	0.01\\
38.36	0.01\\
38.37	0.01\\
38.38	0.01\\
38.39	0.01\\
38.4	0.01\\
38.41	0.01\\
38.42	0.01\\
38.43	0.01\\
38.44	0.01\\
38.45	0.01\\
38.46	0.01\\
38.47	0.01\\
38.48	0.01\\
38.49	0.01\\
38.5	0.01\\
38.51	0.01\\
38.52	0.01\\
38.53	0.01\\
38.54	0.01\\
38.55	0.01\\
38.56	0.01\\
38.57	0.01\\
38.58	0.01\\
38.59	0.01\\
38.6	0.01\\
38.61	0.01\\
38.62	0.01\\
38.63	0.01\\
38.64	0.01\\
38.65	0.01\\
38.66	0.01\\
38.67	0.01\\
38.68	0.01\\
38.69	0.01\\
38.7	0.01\\
38.71	0.01\\
38.72	0.01\\
38.73	0.01\\
38.74	0.01\\
38.75	0.01\\
38.76	0.01\\
38.77	0.01\\
38.78	0.01\\
38.79	0.01\\
38.8	0.01\\
38.81	0.01\\
38.82	0.01\\
38.83	0.01\\
38.84	0.01\\
38.85	0.01\\
38.86	0.01\\
38.87	0.01\\
38.88	0.01\\
38.89	0.01\\
38.9	0.01\\
38.91	0.01\\
38.92	0.01\\
38.93	0.01\\
38.94	0.01\\
38.95	0.01\\
38.96	0.01\\
38.97	0.01\\
38.98	0.01\\
38.99	0.01\\
39	0.01\\
39.01	0.01\\
39.02	0.01\\
39.03	0.01\\
39.04	0.01\\
39.05	0.01\\
39.06	0.01\\
39.07	0.01\\
39.08	0.01\\
39.09	0.01\\
39.1	0.01\\
39.11	0.01\\
39.12	0.01\\
39.13	0.01\\
39.14	0.01\\
39.15	0.01\\
39.16	0.01\\
39.17	0.01\\
39.18	0.01\\
39.19	0.01\\
39.2	0.01\\
39.21	0.01\\
39.22	0.01\\
39.23	0.01\\
39.24	0.01\\
39.25	0.01\\
39.26	0.01\\
39.27	0.01\\
39.28	0.01\\
39.29	0.01\\
39.3	0.01\\
39.31	0.01\\
39.32	0.01\\
39.33	0.01\\
39.34	0.01\\
39.35	0.01\\
39.36	0.01\\
39.37	0.01\\
39.38	0.01\\
39.39	0.01\\
39.4	0.01\\
39.41	0.01\\
39.42	0.01\\
39.43	0.01\\
39.44	0.01\\
39.45	0.01\\
39.46	0.01\\
39.47	0.01\\
39.48	0.01\\
39.49	0.01\\
39.5	0.01\\
39.51	0.01\\
39.52	0.01\\
39.53	0.01\\
39.54	0.01\\
39.55	0.01\\
39.56	0.01\\
39.57	0.01\\
39.58	0.01\\
39.59	0.01\\
39.6	0.01\\
39.61	0.01\\
39.62	0.01\\
39.63	0.01\\
39.64	0.01\\
39.65	0.01\\
39.66	0.01\\
39.67	0.01\\
39.68	0.01\\
39.69	0.01\\
39.7	0.01\\
39.71	0.01\\
39.72	0.01\\
39.73	0.01\\
39.74	0.01\\
39.75	0.01\\
39.76	0.01\\
39.77	0.01\\
39.78	0.01\\
39.79	0.01\\
39.8	0.01\\
39.81	0.01\\
39.82	0.01\\
39.83	0.01\\
39.84	0.01\\
39.85	0.01\\
39.86	0.01\\
39.87	0.01\\
39.88	0.01\\
39.89	0.01\\
39.9	0.01\\
39.91	0.01\\
39.92	0.01\\
39.93	0.01\\
39.94	0.01\\
39.95	0.01\\
39.96	0.01\\
39.97	0.01\\
39.98	0.01\\
39.99	0.01\\
40	0.01\\
40.01	0.01\\
};
\addplot [color=mycolor1,dashed,forget plot]
  table[row sep=crcr]{%
40.01	0.01\\
40.02	0.01\\
40.03	0.01\\
40.04	0.01\\
40.05	0.01\\
40.06	0.01\\
40.07	0.01\\
40.08	0.01\\
40.09	0.01\\
40.1	0.01\\
40.11	0.01\\
40.12	0.01\\
40.13	0.01\\
40.14	0.01\\
40.15	0.01\\
40.16	0.01\\
40.17	0.01\\
40.18	0.01\\
40.19	0.01\\
40.2	0.01\\
40.21	0.01\\
40.22	0.01\\
40.23	0.01\\
40.24	0.01\\
40.25	0.01\\
40.26	0.01\\
40.27	0.01\\
40.28	0.01\\
40.29	0.01\\
40.3	0.01\\
40.31	0.01\\
40.32	0.01\\
40.33	0.01\\
40.34	0.01\\
40.35	0.01\\
40.36	0.01\\
40.37	0.01\\
40.38	0.01\\
40.39	0.01\\
40.4	0.01\\
40.41	0.01\\
40.42	0.01\\
40.43	0.01\\
40.44	0.01\\
40.45	0.01\\
40.46	0.01\\
40.47	0.01\\
40.48	0.01\\
40.49	0.01\\
40.5	0.01\\
40.51	0.01\\
40.52	0.01\\
40.53	0.01\\
40.54	0.01\\
40.55	0.01\\
40.56	0.01\\
40.57	0.01\\
40.58	0.01\\
40.59	0.01\\
40.6	0.01\\
40.61	0.01\\
40.62	0.01\\
40.63	0.01\\
40.64	0.01\\
40.65	0.01\\
40.66	0.01\\
40.67	0.01\\
40.68	0.01\\
40.69	0.01\\
40.7	0.01\\
40.71	0.01\\
40.72	0.01\\
40.73	0.01\\
40.74	0.01\\
40.75	0.01\\
40.76	0.01\\
40.77	0.01\\
40.78	0.01\\
40.79	0.01\\
40.8	0.01\\
40.81	0.01\\
40.82	0.01\\
40.83	0.01\\
40.84	0.01\\
40.85	0.01\\
40.86	0.01\\
40.87	0.01\\
40.88	0.01\\
40.89	0.01\\
40.9	0.01\\
40.91	0.01\\
40.92	0.01\\
40.93	0.01\\
40.94	0.01\\
40.95	0.01\\
40.96	0.01\\
40.97	0.01\\
40.98	0.01\\
40.99	0.01\\
41	0.01\\
41.01	0.01\\
41.02	0.01\\
41.03	0.01\\
41.04	0.01\\
41.05	0.01\\
41.06	0.01\\
41.07	0.01\\
41.08	0.01\\
41.09	0.01\\
41.1	0.01\\
41.11	0.01\\
41.12	0.01\\
41.13	0.01\\
41.14	0.01\\
41.15	0.01\\
41.16	0.01\\
41.17	0.01\\
41.18	0.01\\
41.19	0.01\\
41.2	0.01\\
41.21	0.01\\
41.22	0.01\\
41.23	0.01\\
41.24	0.01\\
41.25	0.01\\
41.26	0.01\\
41.27	0.01\\
41.28	0.01\\
41.29	0.01\\
41.3	0.01\\
41.31	0.01\\
41.32	0.01\\
41.33	0.01\\
41.34	0.01\\
41.35	0.01\\
41.36	0.01\\
41.37	0.01\\
41.38	0.01\\
41.39	0.01\\
41.4	0.01\\
41.41	0.01\\
41.42	0.01\\
41.43	0.01\\
41.44	0.01\\
41.45	0.01\\
41.46	0.01\\
41.47	0.01\\
41.48	0.01\\
41.49	0.01\\
41.5	0.01\\
41.51	0.01\\
41.52	0.01\\
41.53	0.01\\
41.54	0.01\\
41.55	0.01\\
41.56	0.01\\
41.57	0.01\\
41.58	0.01\\
41.59	0.01\\
41.6	0.01\\
41.61	0.01\\
41.62	0.01\\
41.63	0.01\\
41.64	0.01\\
41.65	0.01\\
41.66	0.01\\
41.67	0.01\\
41.68	0.01\\
41.69	0.01\\
41.7	0.01\\
41.71	0.01\\
41.72	0.01\\
41.73	0.01\\
41.74	0.01\\
41.75	0.01\\
41.76	0.01\\
41.77	0.01\\
41.78	0.01\\
41.79	0.01\\
41.8	0.01\\
41.81	0.01\\
41.82	0.01\\
41.83	0.01\\
41.84	0.01\\
41.85	0.01\\
41.86	0.01\\
41.87	0.01\\
41.88	0.01\\
41.89	0.01\\
41.9	0.01\\
41.91	0.01\\
41.92	0.01\\
41.93	0.01\\
41.94	0.01\\
41.95	0.01\\
41.96	0.01\\
41.97	0.01\\
41.98	0.01\\
41.99	0.01\\
42	0.01\\
42.01	0.01\\
42.02	0.01\\
42.03	0.01\\
42.04	0.01\\
42.05	0.01\\
42.06	0.01\\
42.07	0.01\\
42.08	0.01\\
42.09	0.01\\
42.1	0.01\\
42.11	0.01\\
42.12	0.01\\
42.13	0.01\\
42.14	0.01\\
42.15	0.01\\
42.16	0.01\\
42.17	0.01\\
42.18	0.01\\
42.19	0.01\\
42.2	0.01\\
42.21	0.01\\
42.22	0.01\\
42.23	0.01\\
42.24	0.01\\
42.25	0.01\\
42.26	0.01\\
42.27	0.01\\
42.28	0.01\\
42.29	0.01\\
42.3	0.01\\
42.31	0.01\\
42.32	0.01\\
42.33	0.01\\
42.34	0.01\\
42.35	0.01\\
42.36	0.01\\
42.37	0.01\\
42.38	0.01\\
42.39	0.01\\
42.4	0.01\\
42.41	0.01\\
42.42	0.01\\
42.43	0.01\\
42.44	0.01\\
42.45	0.01\\
42.46	0.01\\
42.47	0.01\\
42.48	0.01\\
42.49	0.01\\
42.5	0.01\\
42.51	0.01\\
42.52	0.01\\
42.53	0.01\\
42.54	0.01\\
42.55	0.01\\
42.56	0.01\\
42.57	0.01\\
42.58	0.01\\
42.59	0.01\\
42.6	0.01\\
42.61	0.01\\
42.62	0.01\\
42.63	0.01\\
42.64	0.01\\
42.65	0.01\\
42.66	0.01\\
42.67	0.01\\
42.68	0.01\\
42.69	0.01\\
42.7	0.01\\
42.71	0.01\\
42.72	0.01\\
42.73	0.01\\
42.74	0.01\\
42.75	0.01\\
42.76	0.01\\
42.77	0.01\\
42.78	0.01\\
42.79	0.01\\
42.8	0.01\\
42.81	0.01\\
42.82	0.01\\
42.83	0.01\\
42.84	0.01\\
42.85	0.01\\
42.86	0.01\\
42.87	0.01\\
42.88	0.01\\
42.89	0.01\\
42.9	0.01\\
42.91	0.01\\
42.92	0.01\\
42.93	0.01\\
42.94	0.01\\
42.95	0.01\\
42.96	0.01\\
42.97	0.01\\
42.98	0.01\\
42.99	0.01\\
43	0.01\\
43.01	0.01\\
43.02	0.01\\
43.03	0.01\\
43.04	0.01\\
43.05	0.01\\
43.06	0.01\\
43.07	0.01\\
43.08	0.01\\
43.09	0.01\\
43.1	0.01\\
43.11	0.01\\
43.12	0.01\\
43.13	0.01\\
43.14	0.01\\
43.15	0.01\\
43.16	0.01\\
43.17	0.01\\
43.18	0.01\\
43.19	0.01\\
43.2	0.01\\
43.21	0.01\\
43.22	0.01\\
43.23	0.01\\
43.24	0.01\\
43.25	0.01\\
43.26	0.01\\
43.27	0.01\\
43.28	0.01\\
43.29	0.01\\
43.3	0.01\\
43.31	0.01\\
43.32	0.01\\
43.33	0.01\\
43.34	0.01\\
43.35	0.01\\
43.36	0.01\\
43.37	0.01\\
43.38	0.01\\
43.39	0.01\\
43.4	0.01\\
43.41	0.01\\
43.42	0.01\\
43.43	0.01\\
43.44	0.01\\
43.45	0.01\\
43.46	0.01\\
43.47	0.01\\
43.48	0.01\\
43.49	0.01\\
43.5	0.01\\
43.51	0.01\\
43.52	0.01\\
43.53	0.01\\
43.54	0.01\\
43.55	0.01\\
43.56	0.01\\
43.57	0.01\\
43.58	0.01\\
43.59	0.01\\
43.6	0.01\\
43.61	0.01\\
43.62	0.01\\
43.63	0.01\\
43.64	0.01\\
43.65	0.01\\
43.66	0.01\\
43.67	0.01\\
43.68	0.01\\
43.69	0.01\\
43.7	0.01\\
43.71	0.01\\
43.72	0.01\\
43.73	0.01\\
43.74	0.01\\
43.75	0.01\\
43.76	0.01\\
43.77	0.01\\
43.78	0.01\\
43.79	0.01\\
43.8	0.01\\
43.81	0.01\\
43.82	0.01\\
43.83	0.01\\
43.84	0.01\\
43.85	0.01\\
43.86	0.01\\
43.87	0.01\\
43.88	0.01\\
43.89	0.01\\
43.9	0.01\\
43.91	0.01\\
43.92	0.01\\
43.93	0.01\\
43.94	0.01\\
43.95	0.01\\
43.96	0.01\\
43.97	0.01\\
43.98	0.01\\
43.99	0.01\\
44	0.01\\
44.01	0.01\\
44.02	0.01\\
44.03	0.01\\
44.04	0.01\\
44.05	0.01\\
44.06	0.01\\
44.07	0.01\\
44.08	0.01\\
44.09	0.01\\
44.1	0.01\\
44.11	0.01\\
44.12	0.01\\
44.13	0.01\\
44.14	0.01\\
44.15	0.01\\
44.16	0.01\\
44.17	0.01\\
44.18	0.01\\
44.19	0.01\\
44.2	0.01\\
44.21	0.01\\
44.22	0.01\\
44.23	0.01\\
44.24	0.01\\
44.25	0.01\\
44.26	0.01\\
44.27	0.01\\
44.28	0.01\\
44.29	0.01\\
44.3	0.01\\
44.31	0.01\\
44.32	0.01\\
44.33	0.01\\
44.34	0.01\\
44.35	0.01\\
44.36	0.01\\
44.37	0.01\\
44.38	0.01\\
44.39	0.01\\
44.4	0.01\\
44.41	0.01\\
44.42	0.01\\
44.43	0.01\\
44.44	0.01\\
44.45	0.01\\
44.46	0.01\\
44.47	0.01\\
44.48	0.01\\
44.49	0.01\\
44.5	0.01\\
44.51	0.01\\
44.52	0.01\\
44.53	0.01\\
44.54	0.01\\
44.55	0.01\\
44.56	0.01\\
44.57	0.01\\
44.58	0.01\\
44.59	0.01\\
44.6	0.01\\
44.61	0.01\\
44.62	0.01\\
44.63	0.01\\
44.64	0.01\\
44.65	0.01\\
44.66	0.01\\
44.67	0.01\\
44.68	0.01\\
44.69	0.01\\
44.7	0.01\\
44.71	0.01\\
44.72	0.01\\
44.73	0.01\\
44.74	0.01\\
44.75	0.01\\
44.76	0.01\\
44.77	0.01\\
44.78	0.01\\
44.79	0.01\\
44.8	0.01\\
44.81	0.01\\
44.82	0.01\\
44.83	0.01\\
44.84	0.01\\
44.85	0.01\\
44.86	0.01\\
44.87	0.01\\
44.88	0.01\\
44.89	0.01\\
44.9	0.01\\
44.91	0.01\\
44.92	0.01\\
44.93	0.01\\
44.94	0.01\\
44.95	0.01\\
44.96	0.01\\
44.97	0.01\\
44.98	0.01\\
44.99	0.01\\
45	0.01\\
45.01	0.01\\
45.02	0.01\\
45.03	0.01\\
45.04	0.01\\
45.05	0.01\\
45.06	0.01\\
45.07	0.01\\
45.08	0.01\\
45.09	0.01\\
45.1	0.01\\
45.11	0.01\\
45.12	0.01\\
45.13	0.01\\
45.14	0.01\\
45.15	0.01\\
45.16	0.01\\
45.17	0.01\\
45.18	0.01\\
45.19	0.01\\
45.2	0.01\\
45.21	0.01\\
45.22	0.01\\
45.23	0.01\\
45.24	0.01\\
45.25	0.01\\
45.26	0.01\\
45.27	0.01\\
45.28	0.01\\
45.29	0.01\\
45.3	0.01\\
45.31	0.01\\
45.32	0.01\\
45.33	0.01\\
45.34	0.01\\
45.35	0.01\\
45.36	0.01\\
45.37	0.01\\
45.38	0.01\\
45.39	0.01\\
45.4	0.01\\
45.41	0.01\\
45.42	0.01\\
45.43	0.01\\
45.44	0.01\\
45.45	0.01\\
45.46	0.01\\
45.47	0.01\\
45.48	0.01\\
45.49	0.01\\
45.5	0.01\\
45.51	0.01\\
45.52	0.01\\
45.53	0.01\\
45.54	0.01\\
45.55	0.01\\
45.56	0.01\\
45.57	0.01\\
45.58	0.01\\
45.59	0.01\\
45.6	0.01\\
45.61	0.01\\
45.62	0.01\\
45.63	0.01\\
45.64	0.01\\
45.65	0.01\\
45.66	0.01\\
45.67	0.01\\
45.68	0.01\\
45.69	0.01\\
45.7	0.01\\
45.71	0.01\\
45.72	0.01\\
45.73	0.01\\
45.74	0.01\\
45.75	0.01\\
45.76	0.01\\
45.77	0.01\\
45.78	0.01\\
45.79	0.01\\
45.8	0.01\\
45.81	0.01\\
45.82	0.01\\
45.83	0.01\\
45.84	0.01\\
45.85	0.01\\
45.86	0.01\\
45.87	0.01\\
45.88	0.01\\
45.89	0.01\\
45.9	0.01\\
45.91	0.01\\
45.92	0.01\\
45.93	0.01\\
45.94	0.01\\
45.95	0.01\\
45.96	0.01\\
45.97	0.01\\
45.98	0.01\\
45.99	0.01\\
46	0.01\\
46.01	0.01\\
46.02	0.01\\
46.03	0.01\\
46.04	0.01\\
46.05	0.01\\
46.06	0.01\\
46.07	0.01\\
46.08	0.01\\
46.09	0.01\\
46.1	0.01\\
46.11	0.01\\
46.12	0.01\\
46.13	0.01\\
46.14	0.01\\
46.15	0.01\\
46.16	0.01\\
46.17	0.01\\
46.18	0.01\\
46.19	0.01\\
46.2	0.01\\
46.21	0.01\\
46.22	0.01\\
46.23	0.01\\
46.24	0.01\\
46.25	0.01\\
46.26	0.01\\
46.27	0.01\\
46.28	0.01\\
46.29	0.01\\
46.3	0.01\\
46.31	0.01\\
46.32	0.01\\
46.33	0.01\\
46.34	0.01\\
46.35	0.01\\
46.36	0.01\\
46.37	0.01\\
46.38	0.01\\
46.39	0.01\\
46.4	0.01\\
46.41	0.01\\
46.42	0.01\\
46.43	0.01\\
46.44	0.01\\
46.45	0.01\\
46.46	0.01\\
46.47	0.01\\
46.48	0.01\\
46.49	0.01\\
46.5	0.01\\
46.51	0.01\\
46.52	0.01\\
46.53	0.01\\
46.54	0.01\\
46.55	0.01\\
46.56	0.01\\
46.57	0.01\\
46.58	0.01\\
46.59	0.01\\
46.6	0.01\\
46.61	0.01\\
46.62	0.01\\
46.63	0.01\\
46.64	0.01\\
46.65	0.01\\
46.66	0.01\\
46.67	0.01\\
46.68	0.01\\
46.69	0.01\\
46.7	0.01\\
46.71	0.01\\
46.72	0.01\\
46.73	0.01\\
46.74	0.01\\
46.75	0.01\\
46.76	0.01\\
46.77	0.01\\
46.78	0.01\\
46.79	0.01\\
46.8	0.01\\
46.81	0.01\\
46.82	0.01\\
46.83	0.01\\
46.84	0.01\\
46.85	0.01\\
46.86	0.01\\
46.87	0.01\\
46.88	0.01\\
46.89	0.01\\
46.9	0.01\\
46.91	0.01\\
46.92	0.01\\
46.93	0.01\\
46.94	0.01\\
46.95	0.01\\
46.96	0.01\\
46.97	0.01\\
46.98	0.01\\
46.99	0.01\\
47	0.01\\
47.01	0.01\\
47.02	0.01\\
47.03	0.01\\
47.04	0.01\\
47.05	0.01\\
47.06	0.01\\
47.07	0.01\\
47.08	0.01\\
47.09	0.01\\
47.1	0.01\\
47.11	0.01\\
47.12	0.01\\
47.13	0.01\\
47.14	0.01\\
47.15	0.01\\
47.16	0.01\\
47.17	0.01\\
47.18	0.01\\
47.19	0.01\\
47.2	0.01\\
47.21	0.01\\
47.22	0.01\\
47.23	0.01\\
47.24	0.01\\
47.25	0.01\\
47.26	0.01\\
47.27	0.01\\
47.28	0.01\\
47.29	0.01\\
47.3	0.01\\
47.31	0.01\\
47.32	0.01\\
47.33	0.01\\
47.34	0.01\\
47.35	0.01\\
47.36	0.01\\
47.37	0.01\\
47.38	0.01\\
47.39	0.01\\
47.4	0.01\\
47.41	0.01\\
47.42	0.01\\
47.43	0.01\\
47.44	0.01\\
47.45	0.01\\
47.46	0.01\\
47.47	0.01\\
47.48	0.01\\
47.49	0.01\\
47.5	0.01\\
47.51	0.01\\
47.52	0.01\\
47.53	0.01\\
47.54	0.01\\
47.55	0.01\\
47.56	0.01\\
47.57	0.01\\
47.58	0.01\\
47.59	0.01\\
47.6	0.01\\
47.61	0.01\\
47.62	0.01\\
47.63	0.01\\
47.64	0.01\\
47.65	0.01\\
47.66	0.01\\
47.67	0.01\\
47.68	0.01\\
47.69	0.01\\
47.7	0.01\\
47.71	0.01\\
47.72	0.01\\
47.73	0.01\\
47.74	0.01\\
47.75	0.01\\
47.76	0.01\\
47.77	0.01\\
47.78	0.01\\
47.79	0.01\\
47.8	0.01\\
47.81	0.01\\
47.82	0.01\\
47.83	0.01\\
47.84	0.01\\
47.85	0.01\\
47.86	0.01\\
47.87	0.01\\
47.88	0.01\\
47.89	0.01\\
47.9	0.01\\
47.91	0.01\\
47.92	0.01\\
47.93	0.01\\
47.94	0.01\\
47.95	0.01\\
47.96	0.01\\
47.97	0.01\\
47.98	0.01\\
47.99	0.01\\
48	0.01\\
48.01	0.01\\
48.02	0.01\\
48.03	0.01\\
48.04	0.01\\
48.05	0.01\\
48.06	0.01\\
48.07	0.01\\
48.08	0.01\\
48.09	0.01\\
48.1	0.01\\
48.11	0.01\\
48.12	0.01\\
48.13	0.01\\
48.14	0.01\\
48.15	0.01\\
48.16	0.01\\
48.17	0.01\\
48.18	0.01\\
48.19	0.01\\
48.2	0.01\\
48.21	0.01\\
48.22	0.01\\
48.23	0.01\\
48.24	0.01\\
48.25	0.01\\
48.26	0.01\\
48.27	0.01\\
48.28	0.01\\
48.29	0.01\\
48.3	0.01\\
48.31	0.01\\
48.32	0.01\\
48.33	0.01\\
48.34	0.01\\
48.35	0.01\\
48.36	0.01\\
48.37	0.01\\
48.38	0.01\\
48.39	0.01\\
48.4	0.01\\
48.41	0.01\\
48.42	0.01\\
48.43	0.01\\
48.44	0.01\\
48.45	0.01\\
48.46	0.01\\
48.47	0.01\\
48.48	0.01\\
48.49	0.01\\
48.5	0.01\\
48.51	0.01\\
48.52	0.01\\
48.53	0.01\\
48.54	0.01\\
48.55	0.01\\
48.56	0.01\\
48.57	0.01\\
48.58	0.01\\
48.59	0.01\\
48.6	0.01\\
48.61	0.01\\
48.62	0.01\\
48.63	0.01\\
48.64	0.01\\
48.65	0.01\\
48.66	0.01\\
48.67	0.01\\
48.68	0.01\\
48.69	0.01\\
48.7	0.01\\
48.71	0.01\\
48.72	0.01\\
48.73	0.01\\
48.74	0.01\\
48.75	0.01\\
48.76	0.01\\
48.77	0.01\\
48.78	0.01\\
48.79	0.01\\
48.8	0.01\\
48.81	0.01\\
48.82	0.01\\
48.83	0.01\\
48.84	0.01\\
48.85	0.01\\
48.86	0.01\\
48.87	0.01\\
48.88	0.01\\
48.89	0.01\\
48.9	0.01\\
48.91	0.01\\
48.92	0.01\\
48.93	0.01\\
48.94	0.01\\
48.95	0.01\\
48.96	0.01\\
48.97	0.01\\
48.98	0.01\\
48.99	0.01\\
49	0.01\\
49.01	0.01\\
49.02	0.01\\
49.03	0.01\\
49.04	0.01\\
49.05	0.01\\
49.06	0.01\\
49.07	0.01\\
49.08	0.01\\
49.09	0.01\\
49.1	0.01\\
49.11	0.01\\
49.12	0.01\\
49.13	0.01\\
49.14	0.01\\
49.15	0.01\\
49.16	0.01\\
49.17	0.01\\
49.18	0.01\\
49.19	0.01\\
49.2	0.01\\
49.21	0.01\\
49.22	0.01\\
49.23	0.01\\
49.24	0.01\\
49.25	0.01\\
49.26	0.01\\
49.27	0.01\\
49.28	0.01\\
49.29	0.01\\
49.3	0.01\\
49.31	0.01\\
49.32	0.01\\
49.33	0.01\\
49.34	0.01\\
49.35	0.01\\
49.36	0.01\\
49.37	0.01\\
49.38	0.01\\
49.39	0.01\\
49.4	0.01\\
49.41	0.01\\
49.42	0.01\\
49.43	0.01\\
49.44	0.01\\
49.45	0.01\\
49.46	0.01\\
49.47	0.01\\
49.48	0.01\\
49.49	0.01\\
49.5	0.01\\
49.51	0.01\\
49.52	0.01\\
49.53	0.01\\
49.54	0.01\\
49.55	0.01\\
49.56	0.01\\
49.57	0.01\\
49.58	0.01\\
49.59	0.01\\
49.6	0.01\\
49.61	0.01\\
49.62	0.01\\
49.63	0.01\\
49.64	0.01\\
49.65	0.01\\
49.66	0.01\\
49.67	0.01\\
49.68	0.01\\
49.69	0.01\\
49.7	0.01\\
49.71	0.01\\
49.72	0.01\\
49.73	0.01\\
49.74	0.01\\
49.75	0.01\\
49.76	0.01\\
49.77	0.01\\
49.78	0.01\\
49.79	0.01\\
49.8	0.01\\
49.81	0.01\\
49.82	0.01\\
49.83	0.01\\
49.84	0.01\\
49.85	0.01\\
49.86	0.01\\
49.87	0.01\\
49.88	0.01\\
49.89	0.01\\
49.9	0.01\\
49.91	0.01\\
49.92	0.01\\
49.93	0.01\\
49.94	0.01\\
49.95	0.01\\
49.96	0.01\\
49.97	0.01\\
49.98	0.01\\
49.99	0.01\\
50	0.01\\
50.01	0.01\\
50.02	0.01\\
50.03	0.01\\
50.04	0.01\\
50.05	0.01\\
50.06	0.01\\
50.07	0.01\\
50.08	0.01\\
50.09	0.01\\
50.1	0.01\\
50.11	0.01\\
50.12	0.01\\
50.13	0.01\\
50.14	0.01\\
50.15	0.01\\
50.16	0.01\\
50.17	0.01\\
50.18	0.01\\
50.19	0.01\\
50.2	0.01\\
50.21	0.01\\
50.22	0.01\\
50.23	0.01\\
50.24	0.01\\
50.25	0.01\\
50.26	0.01\\
50.27	0.01\\
50.28	0.01\\
50.29	0.01\\
50.3	0.01\\
50.31	0.01\\
50.32	0.01\\
50.33	0.01\\
50.34	0.01\\
50.35	0.01\\
50.36	0.01\\
50.37	0.01\\
50.38	0.01\\
50.39	0.01\\
50.4	0.01\\
50.41	0.01\\
50.42	0.01\\
50.43	0.01\\
50.44	0.01\\
50.45	0.01\\
50.46	0.01\\
50.47	0.01\\
50.48	0.01\\
50.49	0.01\\
50.5	0.01\\
50.51	0.01\\
50.52	0.01\\
50.53	0.01\\
50.54	0.01\\
50.55	0.01\\
50.56	0.01\\
50.57	0.01\\
50.58	0.01\\
50.59	0.01\\
50.6	0.01\\
50.61	0.01\\
50.62	0.01\\
50.63	0.01\\
50.64	0.01\\
50.65	0.01\\
50.66	0.01\\
50.67	0.01\\
50.68	0.01\\
50.69	0.01\\
50.7	0.01\\
50.71	0.01\\
50.72	0.01\\
50.73	0.01\\
50.74	0.01\\
50.75	0.01\\
50.76	0.01\\
50.77	0.01\\
50.78	0.01\\
50.79	0.01\\
50.8	0.01\\
50.81	0.01\\
50.82	0.01\\
50.83	0.01\\
50.84	0.01\\
50.85	0.01\\
50.86	0.01\\
50.87	0.01\\
50.88	0.01\\
50.89	0.01\\
50.9	0.01\\
50.91	0.01\\
50.92	0.01\\
50.93	0.01\\
50.94	0.01\\
50.95	0.01\\
50.96	0.01\\
50.97	0.01\\
50.98	0.01\\
50.99	0.01\\
51	0.01\\
51.01	0.01\\
51.02	0.01\\
51.03	0.01\\
51.04	0.01\\
51.05	0.01\\
51.06	0.01\\
51.07	0.01\\
51.08	0.01\\
51.09	0.01\\
51.1	0.01\\
51.11	0.01\\
51.12	0.01\\
51.13	0.01\\
51.14	0.01\\
51.15	0.01\\
51.16	0.01\\
51.17	0.01\\
51.18	0.01\\
51.19	0.01\\
51.2	0.01\\
51.21	0.01\\
51.22	0.01\\
51.23	0.01\\
51.24	0.01\\
51.25	0.01\\
51.26	0.01\\
51.27	0.01\\
51.28	0.01\\
51.29	0.01\\
51.3	0.01\\
51.31	0.01\\
51.32	0.01\\
51.33	0.01\\
51.34	0.01\\
51.35	0.01\\
51.36	0.01\\
51.37	0.01\\
51.38	0.01\\
51.39	0.01\\
51.4	0.01\\
51.41	0.01\\
51.42	0.01\\
51.43	0.01\\
51.44	0.01\\
51.45	0.01\\
51.46	0.01\\
51.47	0.01\\
51.48	0.01\\
51.49	0.01\\
51.5	0.01\\
51.51	0.01\\
51.52	0.01\\
51.53	0.01\\
51.54	0.01\\
51.55	0.01\\
51.56	0.01\\
51.57	0.01\\
51.58	0.01\\
51.59	0.01\\
51.6	0.01\\
51.61	0.01\\
51.62	0.01\\
51.63	0.01\\
51.64	0.01\\
51.65	0.01\\
51.66	0.01\\
51.67	0.01\\
51.68	0.01\\
51.69	0.01\\
51.7	0.01\\
51.71	0.01\\
51.72	0.01\\
51.73	0.01\\
51.74	0.01\\
51.75	0.01\\
51.76	0.01\\
51.77	0.01\\
51.78	0.01\\
51.79	0.01\\
51.8	0.01\\
51.81	0.01\\
51.82	0.01\\
51.83	0.01\\
51.84	0.01\\
51.85	0.01\\
51.86	0.01\\
51.87	0.01\\
51.88	0.01\\
51.89	0.01\\
51.9	0.01\\
51.91	0.01\\
51.92	0.01\\
51.93	0.01\\
51.94	0.01\\
51.95	0.01\\
51.96	0.01\\
51.97	0.01\\
51.98	0.01\\
51.99	0.01\\
52	0.01\\
52.01	0.01\\
52.02	0.01\\
52.03	0.01\\
52.04	0.01\\
52.05	0.01\\
52.06	0.01\\
52.07	0.01\\
52.08	0.01\\
52.09	0.01\\
52.1	0.01\\
52.11	0.01\\
52.12	0.01\\
52.13	0.01\\
52.14	0.01\\
52.15	0.01\\
52.16	0.01\\
52.17	0.01\\
52.18	0.01\\
52.19	0.01\\
52.2	0.01\\
52.21	0.01\\
52.22	0.01\\
52.23	0.01\\
52.24	0.01\\
52.25	0.01\\
52.26	0.01\\
52.27	0.01\\
52.28	0.01\\
52.29	0.01\\
52.3	0.01\\
52.31	0.01\\
52.32	0.01\\
52.33	0.01\\
52.34	0.01\\
52.35	0.01\\
52.36	0.01\\
52.37	0.01\\
52.38	0.01\\
52.39	0.01\\
52.4	0.01\\
52.41	0.01\\
52.42	0.01\\
52.43	0.01\\
52.44	0.01\\
52.45	0.01\\
52.46	0.01\\
52.47	0.01\\
52.48	0.01\\
52.49	0.01\\
52.5	0.01\\
52.51	0.01\\
52.52	0.01\\
52.53	0.01\\
52.54	0.01\\
52.55	0.01\\
52.56	0.01\\
52.57	0.01\\
52.58	0.01\\
52.59	0.01\\
52.6	0.01\\
52.61	0.01\\
52.62	0.01\\
52.63	0.01\\
52.64	0.01\\
52.65	0.01\\
52.66	0.01\\
52.67	0.01\\
52.68	0.01\\
52.69	0.01\\
52.7	0.01\\
52.71	0.01\\
52.72	0.01\\
52.73	0.01\\
52.74	0.01\\
52.75	0.01\\
52.76	0.01\\
52.77	0.01\\
52.78	0.01\\
52.79	0.01\\
52.8	0.01\\
52.81	0.01\\
52.82	0.01\\
52.83	0.01\\
52.84	0.01\\
52.85	0.01\\
52.86	0.01\\
52.87	0.01\\
52.88	0.01\\
52.89	0.01\\
52.9	0.01\\
52.91	0.01\\
52.92	0.01\\
52.93	0.01\\
52.94	0.01\\
52.95	0.01\\
52.96	0.01\\
52.97	0.01\\
52.98	0.01\\
52.99	0.01\\
53	0.01\\
53.01	0.01\\
53.02	0.01\\
53.03	0.01\\
53.04	0.01\\
53.05	0.01\\
53.06	0.01\\
53.07	0.01\\
53.08	0.01\\
53.09	0.01\\
53.1	0.01\\
53.11	0.01\\
53.12	0.01\\
53.13	0.01\\
53.14	0.01\\
53.15	0.01\\
53.16	0.01\\
53.17	0.01\\
53.18	0.01\\
53.19	0.01\\
53.2	0.01\\
53.21	0.01\\
53.22	0.01\\
53.23	0.01\\
53.24	0.01\\
53.25	0.01\\
53.26	0.01\\
53.27	0.01\\
53.28	0.01\\
53.29	0.01\\
53.3	0.01\\
53.31	0.01\\
53.32	0.01\\
53.33	0.01\\
53.34	0.01\\
53.35	0.01\\
53.36	0.01\\
53.37	0.01\\
53.38	0.01\\
53.39	0.01\\
53.4	0.01\\
53.41	0.01\\
53.42	0.01\\
53.43	0.01\\
53.44	0.01\\
53.45	0.01\\
53.46	0.01\\
53.47	0.01\\
53.48	0.01\\
53.49	0.01\\
53.5	0.01\\
53.51	0.01\\
53.52	0.01\\
53.53	0.01\\
53.54	0.01\\
53.55	0.01\\
53.56	0.01\\
53.57	0.01\\
53.58	0.01\\
53.59	0.01\\
53.6	0.01\\
53.61	0.01\\
53.62	0.01\\
53.63	0.01\\
53.64	0.01\\
53.65	0.01\\
53.66	0.01\\
53.67	0.01\\
53.68	0.01\\
53.69	0.01\\
53.7	0.01\\
53.71	0.01\\
53.72	0.01\\
53.73	0.01\\
53.74	0.01\\
53.75	0.01\\
53.76	0.01\\
53.77	0.01\\
53.78	0.01\\
53.79	0.01\\
53.8	0.01\\
53.81	0.01\\
53.82	0.01\\
53.83	0.01\\
53.84	0.01\\
53.85	0.01\\
53.86	0.01\\
53.87	0.01\\
53.88	0.01\\
53.89	0.01\\
53.9	0.01\\
53.91	0.01\\
53.92	0.01\\
53.93	0.01\\
53.94	0.01\\
53.95	0.01\\
53.96	0.01\\
53.97	0.01\\
53.98	0.01\\
53.99	0.01\\
54	0.01\\
54.01	0.01\\
54.02	0.01\\
54.03	0.01\\
54.04	0.01\\
54.05	0.01\\
54.06	0.01\\
54.07	0.01\\
54.08	0.01\\
54.09	0.01\\
54.1	0.01\\
54.11	0.01\\
54.12	0.01\\
54.13	0.01\\
54.14	0.01\\
54.15	0.01\\
54.16	0.01\\
54.17	0.01\\
54.18	0.01\\
54.19	0.01\\
54.2	0.01\\
54.21	0.01\\
54.22	0.01\\
54.23	0.01\\
54.24	0.01\\
54.25	0.01\\
54.26	0.01\\
54.27	0.01\\
54.28	0.01\\
54.29	0.01\\
54.3	0.01\\
54.31	0.01\\
54.32	0.01\\
54.33	0.01\\
54.34	0.01\\
54.35	0.01\\
54.36	0.01\\
54.37	0.01\\
54.38	0.01\\
54.39	0.01\\
54.4	0.01\\
54.41	0.01\\
54.42	0.01\\
54.43	0.01\\
54.44	0.01\\
54.45	0.01\\
54.46	0.01\\
54.47	0.01\\
54.48	0.01\\
54.49	0.01\\
54.5	0.01\\
54.51	0.01\\
54.52	0.01\\
54.53	0.01\\
54.54	0.01\\
54.55	0.01\\
54.56	0.01\\
54.57	0.01\\
54.58	0.01\\
54.59	0.01\\
54.6	0.01\\
54.61	0.01\\
54.62	0.01\\
54.63	0.01\\
54.64	0.01\\
54.65	0.01\\
54.66	0.01\\
54.67	0.01\\
54.68	0.01\\
54.69	0.01\\
54.7	0.01\\
54.71	0.01\\
54.72	0.01\\
54.73	0.01\\
54.74	0.01\\
54.75	0.01\\
54.76	0.01\\
54.77	0.01\\
54.78	0.01\\
54.79	0.01\\
54.8	0.01\\
54.81	0.01\\
54.82	0.01\\
54.83	0.01\\
54.84	0.01\\
54.85	0.01\\
54.86	0.01\\
54.87	0.01\\
54.88	0.01\\
54.89	0.01\\
54.9	0.01\\
54.91	0.01\\
54.92	0.01\\
54.93	0.01\\
54.94	0.01\\
54.95	0.01\\
54.96	0.01\\
54.97	0.01\\
54.98	0.01\\
54.99	0.01\\
55	0.01\\
55.01	0.01\\
55.02	0.01\\
55.03	0.01\\
55.04	0.01\\
55.05	0.01\\
55.06	0.01\\
55.07	0.01\\
55.08	0.01\\
55.09	0.01\\
55.1	0.01\\
55.11	0.01\\
55.12	0.01\\
55.13	0.01\\
55.14	0.01\\
55.15	0.01\\
55.16	0.01\\
55.17	0.01\\
55.18	0.01\\
55.19	0.01\\
55.2	0.01\\
55.21	0.01\\
55.22	0.01\\
55.23	0.01\\
55.24	0.01\\
55.25	0.01\\
55.26	0.01\\
55.27	0.01\\
55.28	0.01\\
55.29	0.01\\
55.3	0.01\\
55.31	0.01\\
55.32	0.01\\
55.33	0.01\\
55.34	0.01\\
55.35	0.01\\
55.36	0.01\\
55.37	0.01\\
55.38	0.01\\
55.39	0.01\\
55.4	0.01\\
55.41	0.01\\
55.42	0.01\\
55.43	0.01\\
55.44	0.01\\
55.45	0.01\\
55.46	0.01\\
55.47	0.01\\
55.48	0.01\\
55.49	0.01\\
55.5	0.01\\
55.51	0.01\\
55.52	0.01\\
55.53	0.01\\
55.54	0.01\\
55.55	0.01\\
55.56	0.01\\
55.57	0.01\\
55.58	0.01\\
55.59	0.01\\
55.6	0.01\\
55.61	0.01\\
55.62	0.01\\
55.63	0.01\\
55.64	0.01\\
55.65	0.01\\
55.66	0.01\\
55.67	0.01\\
55.68	0.01\\
55.69	0.01\\
55.7	0.01\\
55.71	0.01\\
55.72	0.01\\
55.73	0.01\\
55.74	0.01\\
55.75	0.01\\
55.76	0.01\\
55.77	0.01\\
55.78	0.01\\
55.79	0.01\\
55.8	0.01\\
55.81	0.01\\
55.82	0.01\\
55.83	0.01\\
55.84	0.01\\
55.85	0.01\\
55.86	0.01\\
55.87	0.01\\
55.88	0.01\\
55.89	0.01\\
55.9	0.01\\
55.91	0.01\\
55.92	0.01\\
55.93	0.01\\
55.94	0.01\\
55.95	0.01\\
55.96	0.01\\
55.97	0.01\\
55.98	0.01\\
55.99	0.01\\
56	0.01\\
56.01	0.01\\
56.02	0.01\\
56.03	0.01\\
56.04	0.01\\
56.05	0.01\\
56.06	0.01\\
56.07	0.01\\
56.08	0.01\\
56.09	0.01\\
56.1	0.01\\
56.11	0.01\\
56.12	0.01\\
56.13	0.01\\
56.14	0.01\\
56.15	0.01\\
56.16	0.01\\
56.17	0.01\\
56.18	0.01\\
56.19	0.01\\
56.2	0.01\\
56.21	0.01\\
56.22	0.01\\
56.23	0.01\\
56.24	0.01\\
56.25	0.01\\
56.26	0.01\\
56.27	0.01\\
56.28	0.01\\
56.29	0.01\\
56.3	0.01\\
56.31	0.01\\
56.32	0.01\\
56.33	0.01\\
56.34	0.01\\
56.35	0.01\\
56.36	0.01\\
56.37	0.01\\
56.38	0.01\\
56.39	0.01\\
56.4	0.01\\
56.41	0.01\\
56.42	0.01\\
56.43	0.01\\
56.44	0.01\\
56.45	0.01\\
56.46	0.01\\
56.47	0.01\\
56.48	0.01\\
56.49	0.01\\
56.5	0.01\\
56.51	0.01\\
56.52	0.01\\
56.53	0.01\\
56.54	0.01\\
56.55	0.01\\
56.56	0.01\\
56.57	0.01\\
56.58	0.01\\
56.59	0.01\\
56.6	0.01\\
56.61	0.01\\
56.62	0.01\\
56.63	0.01\\
56.64	0.01\\
56.65	0.01\\
56.66	0.01\\
56.67	0.01\\
56.68	0.01\\
56.69	0.01\\
56.7	0.01\\
56.71	0.01\\
56.72	0.01\\
56.73	0.01\\
56.74	0.01\\
56.75	0.01\\
56.76	0.01\\
56.77	0.01\\
56.78	0.01\\
56.79	0.01\\
56.8	0.01\\
56.81	0.01\\
56.82	0.01\\
56.83	0.01\\
56.84	0.01\\
56.85	0.01\\
56.86	0.01\\
56.87	0.01\\
56.88	0.01\\
56.89	0.01\\
56.9	0.01\\
56.91	0.01\\
56.92	0.01\\
56.93	0.01\\
56.94	0.01\\
56.95	0.01\\
56.96	0.01\\
56.97	0.01\\
56.98	0.01\\
56.99	0.01\\
57	0.01\\
57.01	0.01\\
57.02	0.01\\
57.03	0.01\\
57.04	0.01\\
57.05	0.01\\
57.06	0.01\\
57.07	0.01\\
57.08	0.01\\
57.09	0.01\\
57.1	0.01\\
57.11	0.01\\
57.12	0.01\\
57.13	0.01\\
57.14	0.01\\
57.15	0.01\\
57.16	0.01\\
57.17	0.01\\
57.18	0.01\\
57.19	0.01\\
57.2	0.01\\
57.21	0.01\\
57.22	0.01\\
57.23	0.01\\
57.24	0.01\\
57.25	0.01\\
57.26	0.01\\
57.27	0.01\\
57.28	0.01\\
57.29	0.01\\
57.3	0.01\\
57.31	0.01\\
57.32	0.01\\
57.33	0.01\\
57.34	0.01\\
57.35	0.01\\
57.36	0.01\\
57.37	0.01\\
57.38	0.01\\
57.39	0.01\\
57.4	0.01\\
57.41	0.01\\
57.42	0.01\\
57.43	0.01\\
57.44	0.01\\
57.45	0.01\\
57.46	0.01\\
57.47	0.01\\
57.48	0.01\\
57.49	0.01\\
57.5	0.01\\
57.51	0.01\\
57.52	0.01\\
57.53	0.01\\
57.54	0.01\\
57.55	0.01\\
57.56	0.01\\
57.57	0.01\\
57.58	0.01\\
57.59	0.01\\
57.6	0.01\\
57.61	0.01\\
57.62	0.01\\
57.63	0.01\\
57.64	0.01\\
57.65	0.01\\
57.66	0.01\\
57.67	0.01\\
57.68	0.01\\
57.69	0.01\\
57.7	0.01\\
57.71	0.01\\
57.72	0.01\\
57.73	0.01\\
57.74	0.01\\
57.75	0.01\\
57.76	0.01\\
57.77	0.01\\
57.78	0.01\\
57.79	0.01\\
57.8	0.01\\
57.81	0.01\\
57.82	0.01\\
57.83	0.01\\
57.84	0.01\\
57.85	0.01\\
57.86	0.01\\
57.87	0.01\\
57.88	0.01\\
57.89	0.01\\
57.9	0.01\\
57.91	0.01\\
57.92	0.01\\
57.93	0.01\\
57.94	0.01\\
57.95	0.01\\
57.96	0.01\\
57.97	0.01\\
57.98	0.01\\
57.99	0.01\\
58	0.01\\
58.01	0.01\\
58.02	0.01\\
58.03	0.01\\
58.04	0.01\\
58.05	0.01\\
58.06	0.01\\
58.07	0.01\\
58.08	0.01\\
58.09	0.01\\
58.1	0.01\\
58.11	0.01\\
58.12	0.01\\
58.13	0.01\\
58.14	0.01\\
58.15	0.01\\
58.16	0.01\\
58.17	0.01\\
58.18	0.01\\
58.19	0.01\\
58.2	0.01\\
58.21	0.01\\
58.22	0.01\\
58.23	0.01\\
58.24	0.01\\
58.25	0.01\\
58.26	0.01\\
58.27	0.01\\
58.28	0.01\\
58.29	0.01\\
58.3	0.01\\
58.31	0.01\\
58.32	0.01\\
58.33	0.01\\
58.34	0.01\\
58.35	0.01\\
58.36	0.01\\
58.37	0.01\\
58.38	0.01\\
58.39	0.01\\
58.4	0.01\\
58.41	0.01\\
58.42	0.01\\
58.43	0.01\\
58.44	0.01\\
58.45	0.01\\
58.46	0.01\\
58.47	0.01\\
58.48	0.01\\
58.49	0.01\\
58.5	0.01\\
58.51	0.01\\
58.52	0.01\\
58.53	0.01\\
58.54	0.01\\
58.55	0.01\\
58.56	0.01\\
58.57	0.01\\
58.58	0.01\\
58.59	0.01\\
58.6	0.01\\
58.61	0.01\\
58.62	0.01\\
58.63	0.01\\
58.64	0.01\\
58.65	0.01\\
58.66	0.01\\
58.67	0.01\\
58.68	0.01\\
58.69	0.01\\
58.7	0.01\\
58.71	0.01\\
58.72	0.01\\
58.73	0.01\\
58.74	0.01\\
58.75	0.01\\
58.76	0.01\\
58.77	0.01\\
58.78	0.01\\
58.79	0.01\\
58.8	0.01\\
58.81	0.01\\
58.82	0.01\\
58.83	0.01\\
58.84	0.01\\
58.85	0.01\\
58.86	0.01\\
58.87	0.01\\
58.88	0.01\\
58.89	0.01\\
58.9	0.01\\
58.91	0.01\\
58.92	0.01\\
58.93	0.01\\
58.94	0.01\\
58.95	0.01\\
58.96	0.01\\
58.97	0.01\\
58.98	0.01\\
58.99	0.01\\
59	0.01\\
59.01	0.01\\
59.02	0.01\\
59.03	0.01\\
59.04	0.01\\
59.05	0.01\\
59.06	0.01\\
59.07	0.01\\
59.08	0.01\\
59.09	0.01\\
59.1	0.01\\
59.11	0.01\\
59.12	0.01\\
59.13	0.01\\
59.14	0.01\\
59.15	0.01\\
59.16	0.01\\
59.17	0.01\\
59.18	0.01\\
59.19	0.01\\
59.2	0.01\\
59.21	0.01\\
59.22	0.01\\
59.23	0.01\\
59.24	0.01\\
59.25	0.01\\
59.26	0.01\\
59.27	0.01\\
59.28	0.01\\
59.29	0.01\\
59.3	0.01\\
59.31	0.01\\
59.32	0.01\\
59.33	0.01\\
59.34	0.01\\
59.35	0.01\\
59.36	0.01\\
59.37	0.01\\
59.38	0.01\\
59.39	0.01\\
59.4	0.01\\
59.41	0.01\\
59.42	0.01\\
59.43	0.01\\
59.44	0.01\\
59.45	0.01\\
59.46	0.01\\
59.47	0.01\\
59.48	0.01\\
59.49	0.01\\
59.5	0.01\\
59.51	0.01\\
59.52	0.01\\
59.53	0.01\\
59.54	0.01\\
59.55	0.01\\
59.56	0.01\\
59.57	0.01\\
59.58	0.01\\
59.59	0.01\\
59.6	0.01\\
59.61	0.01\\
59.62	0.01\\
59.63	0.01\\
59.64	0.01\\
59.65	0.01\\
59.66	0.01\\
59.67	0.01\\
59.68	0.01\\
59.69	0.01\\
59.7	0.01\\
59.71	0.01\\
59.72	0.01\\
59.73	0.01\\
59.74	0.01\\
59.75	0.01\\
59.76	0.01\\
59.77	0.01\\
59.78	0.01\\
59.79	0.01\\
59.8	0.01\\
59.81	0.01\\
59.82	0.01\\
59.83	0.01\\
59.84	0.01\\
59.85	0.01\\
59.86	0.01\\
59.87	0.01\\
59.88	0.01\\
59.89	0.01\\
59.9	0.01\\
59.91	0.01\\
59.92	0.01\\
59.93	0.01\\
59.94	0.01\\
59.95	0.01\\
59.96	0.01\\
59.97	0.01\\
59.98	0.01\\
59.99	0.01\\
60	0.01\\
60.01	0.01\\
60.02	0.01\\
60.03	0.01\\
60.04	0.01\\
60.05	0.01\\
60.06	0.01\\
60.07	0.01\\
60.08	0.01\\
60.09	0.01\\
60.1	0.01\\
60.11	0.01\\
60.12	0.01\\
60.13	0.01\\
60.14	0.01\\
60.15	0.01\\
60.16	0.01\\
60.17	0.01\\
60.18	0.01\\
60.19	0.01\\
60.2	0.01\\
60.21	0.01\\
60.22	0.01\\
60.23	0.01\\
60.24	0.01\\
60.25	0.01\\
60.26	0.01\\
60.27	0.01\\
60.28	0.01\\
60.29	0.01\\
60.3	0.01\\
60.31	0.01\\
60.32	0.01\\
60.33	0.01\\
60.34	0.01\\
60.35	0.01\\
60.36	0.01\\
60.37	0.01\\
60.38	0.01\\
60.39	0.01\\
60.4	0.01\\
60.41	0.01\\
60.42	0.01\\
60.43	0.01\\
60.44	0.01\\
60.45	0.01\\
60.46	0.01\\
60.47	0.01\\
60.48	0.01\\
60.49	0.01\\
60.5	0.01\\
60.51	0.01\\
60.52	0.01\\
60.53	0.01\\
60.54	0.01\\
60.55	0.01\\
60.56	0.01\\
60.57	0.01\\
60.58	0.01\\
60.59	0.01\\
60.6	0.01\\
60.61	0.01\\
60.62	0.01\\
60.63	0.01\\
60.64	0.01\\
60.65	0.01\\
60.66	0.01\\
60.67	0.01\\
60.68	0.01\\
60.69	0.01\\
60.7	0.01\\
60.71	0.01\\
60.72	0.01\\
60.73	0.01\\
60.74	0.01\\
60.75	0.01\\
60.76	0.01\\
60.77	0.01\\
60.78	0.01\\
60.79	0.01\\
60.8	0.01\\
60.81	0.01\\
60.82	0.01\\
60.83	0.01\\
60.84	0.01\\
60.85	0.01\\
60.86	0.01\\
60.87	0.01\\
60.88	0.01\\
60.89	0.01\\
60.9	0.01\\
60.91	0.01\\
60.92	0.01\\
60.93	0.01\\
60.94	0.01\\
60.95	0.01\\
60.96	0.01\\
60.97	0.01\\
60.98	0.01\\
60.99	0.01\\
61	0.01\\
61.01	0.01\\
61.02	0.01\\
61.03	0.01\\
61.04	0.01\\
61.05	0.01\\
61.06	0.01\\
61.07	0.01\\
61.08	0.01\\
61.09	0.01\\
61.1	0.01\\
61.11	0.01\\
61.12	0.01\\
61.13	0.01\\
61.14	0.01\\
61.15	0.01\\
61.16	0.01\\
61.17	0.01\\
61.18	0.01\\
61.19	0.01\\
61.2	0.01\\
61.21	0.01\\
61.22	0.01\\
61.23	0.01\\
61.24	0.01\\
61.25	0.01\\
61.26	0.01\\
61.27	0.01\\
61.28	0.01\\
61.29	0.01\\
61.3	0.01\\
61.31	0.01\\
61.32	0.01\\
61.33	0.01\\
61.34	0.01\\
61.35	0.01\\
61.36	0.01\\
61.37	0.01\\
61.38	0.01\\
61.39	0.01\\
61.4	0.01\\
61.41	0.01\\
61.42	0.01\\
61.43	0.01\\
61.44	0.01\\
61.45	0.01\\
61.46	0.01\\
61.47	0.01\\
61.48	0.01\\
61.49	0.01\\
61.5	0.01\\
61.51	0.01\\
61.52	0.01\\
61.53	0.01\\
61.54	0.01\\
61.55	0.01\\
61.56	0.01\\
61.57	0.01\\
61.58	0.01\\
61.59	0.01\\
61.6	0.01\\
61.61	0.01\\
61.62	0.01\\
61.63	0.01\\
61.64	0.01\\
61.65	0.01\\
61.66	0.01\\
61.67	0.01\\
61.68	0.01\\
61.69	0.01\\
61.7	0.01\\
61.71	0.01\\
61.72	0.01\\
61.73	0.01\\
61.74	0.01\\
61.75	0.01\\
61.76	0.01\\
61.77	0.01\\
61.78	0.01\\
61.79	0.01\\
61.8	0.01\\
61.81	0.01\\
61.82	0.01\\
61.83	0.01\\
61.84	0.01\\
61.85	0.01\\
61.86	0.01\\
61.87	0.01\\
61.88	0.01\\
61.89	0.01\\
61.9	0.01\\
61.91	0.01\\
61.92	0.01\\
61.93	0.01\\
61.94	0.01\\
61.95	0.01\\
61.96	0.01\\
61.97	0.01\\
61.98	0.01\\
61.99	0.01\\
62	0.01\\
62.01	0.01\\
62.02	0.01\\
62.03	0.01\\
62.04	0.01\\
62.05	0.01\\
62.06	0.01\\
62.07	0.01\\
62.08	0.01\\
62.09	0.01\\
62.1	0.01\\
62.11	0.01\\
62.12	0.01\\
62.13	0.01\\
62.14	0.01\\
62.15	0.01\\
62.16	0.01\\
62.17	0.01\\
62.18	0.01\\
62.19	0.01\\
62.2	0.01\\
62.21	0.01\\
62.22	0.01\\
62.23	0.01\\
62.24	0.01\\
62.25	0.01\\
62.26	0.01\\
62.27	0.01\\
62.28	0.01\\
62.29	0.01\\
62.3	0.01\\
62.31	0.01\\
62.32	0.01\\
62.33	0.01\\
62.34	0.01\\
62.35	0.01\\
62.36	0.01\\
62.37	0.01\\
62.38	0.01\\
62.39	0.01\\
62.4	0.01\\
62.41	0.01\\
62.42	0.01\\
62.43	0.01\\
62.44	0.01\\
62.45	0.01\\
62.46	0.01\\
62.47	0.01\\
62.48	0.01\\
62.49	0.01\\
62.5	0.01\\
62.51	0.01\\
62.52	0.01\\
62.53	0.01\\
62.54	0.01\\
62.55	0.01\\
62.56	0.01\\
62.57	0.01\\
62.58	0.01\\
62.59	0.01\\
62.6	0.01\\
62.61	0.01\\
62.62	0.01\\
62.63	0.01\\
62.64	0.01\\
62.65	0.01\\
62.66	0.01\\
62.67	0.01\\
62.68	0.01\\
62.69	0.01\\
62.7	0.01\\
62.71	0.01\\
62.72	0.01\\
62.73	0.01\\
62.74	0.01\\
62.75	0.01\\
62.76	0.01\\
62.77	0.01\\
62.78	0.01\\
62.79	0.01\\
62.8	0.01\\
62.81	0.01\\
62.82	0.01\\
62.83	0.01\\
62.84	0.01\\
62.85	0.01\\
62.86	0.01\\
62.87	0.01\\
62.88	0.01\\
62.89	0.01\\
62.9	0.01\\
62.91	0.01\\
62.92	0.01\\
62.93	0.01\\
62.94	0.01\\
62.95	0.01\\
62.96	0.01\\
62.97	0.01\\
62.98	0.01\\
62.99	0.01\\
63	0.01\\
63.01	0.01\\
63.02	0.01\\
63.03	0.01\\
63.04	0.01\\
63.05	0.01\\
63.06	0.01\\
63.07	0.01\\
63.08	0.01\\
63.09	0.01\\
63.1	0.01\\
63.11	0.01\\
63.12	0.01\\
63.13	0.01\\
63.14	0.01\\
63.15	0.01\\
63.16	0.01\\
63.17	0.01\\
63.18	0.01\\
63.19	0.01\\
63.2	0.01\\
63.21	0.01\\
63.22	0.01\\
63.23	0.01\\
63.24	0.01\\
63.25	0.01\\
63.26	0.01\\
63.27	0.01\\
63.28	0.01\\
63.29	0.01\\
63.3	0.01\\
63.31	0.01\\
63.32	0.01\\
63.33	0.01\\
63.34	0.01\\
63.35	0.01\\
63.36	0.01\\
63.37	0.01\\
63.38	0.01\\
63.39	0.01\\
63.4	0.01\\
63.41	0.01\\
63.42	0.01\\
63.43	0.01\\
63.44	0.01\\
63.45	0.01\\
63.46	0.01\\
63.47	0.01\\
63.48	0.01\\
63.49	0.01\\
63.5	0.01\\
63.51	0.01\\
63.52	0.01\\
63.53	0.01\\
63.54	0.01\\
63.55	0.01\\
63.56	0.01\\
63.57	0.01\\
63.58	0.01\\
63.59	0.01\\
63.6	0.01\\
63.61	0.01\\
63.62	0.01\\
63.63	0.01\\
63.64	0.01\\
63.65	0.01\\
63.66	0.01\\
63.67	0.01\\
63.68	0.01\\
63.69	0.01\\
63.7	0.01\\
63.71	0.01\\
63.72	0.01\\
63.73	0.01\\
63.74	0.01\\
63.75	0.01\\
63.76	0.01\\
63.77	0.01\\
63.78	0.01\\
63.79	0.01\\
63.8	0.01\\
63.81	0.01\\
63.82	0.01\\
63.83	0.01\\
63.84	0.01\\
63.85	0.01\\
63.86	0.01\\
63.87	0.01\\
63.88	0.01\\
63.89	0.01\\
63.9	0.01\\
63.91	0.01\\
63.92	0.01\\
63.93	0.01\\
63.94	0.01\\
63.95	0.01\\
63.96	0.01\\
63.97	0.01\\
63.98	0.01\\
63.99	0.01\\
64	0.01\\
64.01	0.01\\
64.02	0.01\\
64.03	0.01\\
64.04	0.01\\
64.05	0.01\\
64.06	0.01\\
64.07	0.01\\
64.08	0.01\\
64.09	0.01\\
64.1	0.01\\
64.11	0.01\\
64.12	0.01\\
64.13	0.01\\
64.14	0.01\\
64.15	0.01\\
64.16	0.01\\
64.17	0.01\\
64.18	0.01\\
64.19	0.01\\
64.2	0.01\\
64.21	0.01\\
64.22	0.01\\
64.23	0.01\\
64.24	0.01\\
64.25	0.01\\
64.26	0.01\\
64.27	0.01\\
64.28	0.01\\
64.29	0.01\\
64.3	0.01\\
64.31	0.01\\
64.32	0.01\\
64.33	0.01\\
64.34	0.01\\
64.35	0.01\\
64.36	0.01\\
64.37	0.01\\
64.38	0.01\\
64.39	0.01\\
64.4	0.01\\
64.41	0.01\\
64.42	0.01\\
64.43	0.01\\
64.44	0.01\\
64.45	0.01\\
64.46	0.01\\
64.47	0.01\\
64.48	0.01\\
64.49	0.01\\
64.5	0.01\\
64.51	0.01\\
64.52	0.01\\
64.53	0.01\\
64.54	0.01\\
64.55	0.01\\
64.56	0.01\\
64.57	0.01\\
64.58	0.01\\
64.59	0.01\\
64.6	0.01\\
64.61	0.01\\
64.62	0.01\\
64.63	0.01\\
64.64	0.01\\
64.65	0.01\\
64.66	0.01\\
64.67	0.01\\
64.68	0.01\\
64.69	0.01\\
64.7	0.01\\
64.71	0.01\\
64.72	0.01\\
64.73	0.01\\
64.74	0.01\\
64.75	0.01\\
64.76	0.01\\
64.77	0.01\\
64.78	0.01\\
64.79	0.01\\
64.8	0.01\\
64.81	0.01\\
64.82	0.01\\
64.83	0.01\\
64.84	0.01\\
64.85	0.01\\
64.86	0.01\\
64.87	0.01\\
64.88	0.01\\
64.89	0.01\\
64.9	0.01\\
64.91	0.01\\
64.92	0.01\\
64.93	0.01\\
64.94	0.01\\
64.95	0.01\\
64.96	0.01\\
64.97	0.01\\
64.98	0.01\\
64.99	0.01\\
65	0.01\\
65.01	0.01\\
65.02	0.01\\
65.03	0.01\\
65.04	0.01\\
65.05	0.01\\
65.06	0.01\\
65.07	0.01\\
65.08	0.01\\
65.09	0.01\\
65.1	0.01\\
65.11	0.01\\
65.12	0.01\\
65.13	0.01\\
65.14	0.01\\
65.15	0.01\\
65.16	0.01\\
65.17	0.01\\
65.18	0.01\\
65.19	0.01\\
65.2	0.01\\
65.21	0.01\\
65.22	0.01\\
65.23	0.01\\
65.24	0.01\\
65.25	0.01\\
65.26	0.01\\
65.27	0.01\\
65.28	0.01\\
65.29	0.01\\
65.3	0.01\\
65.31	0.01\\
65.32	0.01\\
65.33	0.01\\
65.34	0.01\\
65.35	0.01\\
65.36	0.01\\
65.37	0.01\\
65.38	0.01\\
65.39	0.01\\
65.4	0.01\\
65.41	0.01\\
65.42	0.01\\
65.43	0.01\\
65.44	0.01\\
65.45	0.01\\
65.46	0.01\\
65.47	0.01\\
65.48	0.01\\
65.49	0.01\\
65.5	0.01\\
65.51	0.01\\
65.52	0.01\\
65.53	0.01\\
65.54	0.01\\
65.55	0.01\\
65.56	0.01\\
65.57	0.01\\
65.58	0.01\\
65.59	0.01\\
65.6	0.01\\
65.61	0.01\\
65.62	0.01\\
65.63	0.01\\
65.64	0.01\\
65.65	0.01\\
65.66	0.01\\
65.67	0.01\\
65.68	0.01\\
65.69	0.01\\
65.7	0.01\\
65.71	0.01\\
65.72	0.01\\
65.73	0.01\\
65.74	0.01\\
65.75	0.01\\
65.76	0.01\\
65.77	0.01\\
65.78	0.01\\
65.79	0.01\\
65.8	0.01\\
65.81	0.01\\
65.82	0.01\\
65.83	0.01\\
65.84	0.01\\
65.85	0.01\\
65.86	0.01\\
65.87	0.01\\
65.88	0.01\\
65.89	0.01\\
65.9	0.01\\
65.91	0.01\\
65.92	0.01\\
65.93	0.01\\
65.94	0.01\\
65.95	0.01\\
65.96	0.01\\
65.97	0.01\\
65.98	0.01\\
65.99	0.01\\
66	0.01\\
66.01	0.01\\
66.02	0.01\\
66.03	0.01\\
66.04	0.01\\
66.05	0.01\\
66.06	0.01\\
66.07	0.01\\
66.08	0.01\\
66.09	0.01\\
66.1	0.01\\
66.11	0.01\\
66.12	0.01\\
66.13	0.01\\
66.14	0.01\\
66.15	0.01\\
66.16	0.01\\
66.17	0.01\\
66.18	0.01\\
66.19	0.01\\
66.2	0.01\\
66.21	0.01\\
66.22	0.01\\
66.23	0.01\\
66.24	0.01\\
66.25	0.01\\
66.26	0.01\\
66.27	0.01\\
66.28	0.01\\
66.29	0.01\\
66.3	0.01\\
66.31	0.01\\
66.32	0.01\\
66.33	0.01\\
66.34	0.01\\
66.35	0.01\\
66.36	0.01\\
66.37	0.01\\
66.38	0.01\\
66.39	0.01\\
66.4	0.01\\
66.41	0.01\\
66.42	0.01\\
66.43	0.01\\
66.44	0.01\\
66.45	0.01\\
66.46	0.01\\
66.47	0.01\\
66.48	0.01\\
66.49	0.01\\
66.5	0.01\\
66.51	0.01\\
66.52	0.01\\
66.53	0.01\\
66.54	0.01\\
66.55	0.01\\
66.56	0.01\\
66.57	0.01\\
66.58	0.01\\
66.59	0.01\\
66.6	0.01\\
66.61	0.01\\
66.62	0.01\\
66.63	0.01\\
66.64	0.01\\
66.65	0.01\\
66.66	0.01\\
66.67	0.01\\
66.68	0.01\\
66.69	0.01\\
66.7	0.01\\
66.71	0.01\\
66.72	0.01\\
66.73	0.01\\
66.74	0.01\\
66.75	0.01\\
66.76	0.01\\
66.77	0.01\\
66.78	0.01\\
66.79	0.01\\
66.8	0.01\\
66.81	0.01\\
66.82	0.01\\
66.83	0.01\\
66.84	0.01\\
66.85	0.01\\
66.86	0.01\\
66.87	0.01\\
66.88	0.01\\
66.89	0.01\\
66.9	0.01\\
66.91	0.01\\
66.92	0.01\\
66.93	0.01\\
66.94	0.01\\
66.95	0.01\\
66.96	0.01\\
66.97	0.01\\
66.98	0.01\\
66.99	0.01\\
67	0.01\\
67.01	0.01\\
67.02	0.01\\
67.03	0.01\\
67.04	0.01\\
67.05	0.01\\
67.06	0.01\\
67.07	0.01\\
67.08	0.01\\
67.09	0.01\\
67.1	0.01\\
67.11	0.01\\
67.12	0.01\\
67.13	0.01\\
67.14	0.01\\
67.15	0.01\\
67.16	0.01\\
67.17	0.01\\
67.18	0.01\\
67.19	0.01\\
67.2	0.01\\
67.21	0.01\\
67.22	0.01\\
67.23	0.01\\
67.24	0.01\\
67.25	0.01\\
67.26	0.01\\
67.27	0.01\\
67.28	0.01\\
67.29	0.01\\
67.3	0.01\\
67.31	0.01\\
67.32	0.01\\
67.33	0.01\\
67.34	0.01\\
67.35	0.01\\
67.36	0.01\\
67.37	0.01\\
67.38	0.01\\
67.39	0.01\\
67.4	0.01\\
67.41	0.01\\
67.42	0.01\\
67.43	0.01\\
67.44	0.01\\
67.45	0.01\\
67.46	0.01\\
67.47	0.01\\
67.48	0.01\\
67.49	0.01\\
67.5	0.01\\
67.51	0.01\\
67.52	0.01\\
67.53	0.01\\
67.54	0.01\\
67.55	0.01\\
67.56	0.01\\
67.57	0.01\\
67.58	0.01\\
67.59	0.01\\
67.6	0.01\\
67.61	0.01\\
67.62	0.01\\
67.63	0.01\\
67.64	0.01\\
67.65	0.01\\
67.66	0.01\\
67.67	0.01\\
67.68	0.01\\
67.69	0.01\\
67.7	0.01\\
67.71	0.01\\
67.72	0.01\\
67.73	0.01\\
67.74	0.01\\
67.75	0.01\\
67.76	0.01\\
67.77	0.01\\
67.78	0.01\\
67.79	0.01\\
67.8	0.01\\
67.81	0.01\\
67.82	0.01\\
67.83	0.01\\
67.84	0.01\\
67.85	0.01\\
67.86	0.01\\
67.87	0.01\\
67.88	0.01\\
67.89	0.01\\
67.9	0.01\\
67.91	0.01\\
67.92	0.01\\
67.93	0.01\\
67.94	0.01\\
67.95	0.01\\
67.96	0.01\\
67.97	0.01\\
67.98	0.01\\
67.99	0.01\\
68	0.01\\
68.01	0.01\\
68.02	0.01\\
68.03	0.01\\
68.04	0.01\\
68.05	0.01\\
68.06	0.01\\
68.07	0.01\\
68.08	0.01\\
68.09	0.01\\
68.1	0.01\\
68.11	0.01\\
68.12	0.01\\
68.13	0.01\\
68.14	0.01\\
68.15	0.01\\
68.16	0.01\\
68.17	0.01\\
68.18	0.01\\
68.19	0.01\\
68.2	0.01\\
68.21	0.01\\
68.22	0.01\\
68.23	0.01\\
68.24	0.01\\
68.25	0.01\\
68.26	0.01\\
68.27	0.01\\
68.28	0.01\\
68.29	0.01\\
68.3	0.01\\
68.31	0.01\\
68.32	0.01\\
68.33	0.01\\
68.34	0.01\\
68.35	0.01\\
68.36	0.01\\
68.37	0.01\\
68.38	0.01\\
68.39	0.01\\
68.4	0.01\\
68.41	0.01\\
68.42	0.01\\
68.43	0.01\\
68.44	0.01\\
68.45	0.01\\
68.46	0.01\\
68.47	0.01\\
68.48	0.01\\
68.49	0.01\\
68.5	0.01\\
68.51	0.01\\
68.52	0.01\\
68.53	0.01\\
68.54	0.01\\
68.55	0.01\\
68.56	0.01\\
68.57	0.01\\
68.58	0.01\\
68.59	0.01\\
68.6	0.01\\
68.61	0.01\\
68.62	0.01\\
68.63	0.01\\
68.64	0.01\\
68.65	0.01\\
68.66	0.01\\
68.67	0.01\\
68.68	0.01\\
68.69	0.01\\
68.7	0.01\\
68.71	0.01\\
68.72	0.01\\
68.73	0.01\\
68.74	0.01\\
68.75	0.01\\
68.76	0.01\\
68.77	0.01\\
68.78	0.01\\
68.79	0.01\\
68.8	0.01\\
68.81	0.01\\
68.82	0.01\\
68.83	0.01\\
68.84	0.01\\
68.85	0.01\\
68.86	0.01\\
68.87	0.01\\
68.88	0.01\\
68.89	0.01\\
68.9	0.01\\
68.91	0.01\\
68.92	0.01\\
68.93	0.01\\
68.94	0.01\\
68.95	0.01\\
68.96	0.01\\
68.97	0.01\\
68.98	0.01\\
68.99	0.01\\
69	0.01\\
69.01	0.01\\
69.02	0.01\\
69.03	0.01\\
69.04	0.01\\
69.05	0.01\\
69.06	0.01\\
69.07	0.01\\
69.08	0.01\\
69.09	0.01\\
69.1	0.01\\
69.11	0.01\\
69.12	0.01\\
69.13	0.01\\
69.14	0.01\\
69.15	0.01\\
69.16	0.01\\
69.17	0.01\\
69.18	0.01\\
69.19	0.01\\
69.2	0.01\\
69.21	0.01\\
69.22	0.01\\
69.23	0.01\\
69.24	0.01\\
69.25	0.01\\
69.26	0.01\\
69.27	0.01\\
69.28	0.01\\
69.29	0.01\\
69.3	0.01\\
69.31	0.01\\
69.32	0.01\\
69.33	0.01\\
69.34	0.01\\
69.35	0.01\\
69.36	0.01\\
69.37	0.01\\
69.38	0.01\\
69.39	0.01\\
69.4	0.01\\
69.41	0.01\\
69.42	0.01\\
69.43	0.01\\
69.44	0.01\\
69.45	0.01\\
69.46	0.01\\
69.47	0.01\\
69.48	0.01\\
69.49	0.01\\
69.5	0.01\\
69.51	0.01\\
69.52	0.01\\
69.53	0.01\\
69.54	0.01\\
69.55	0.01\\
69.56	0.01\\
69.57	0.01\\
69.58	0.01\\
69.59	0.01\\
69.6	0.01\\
69.61	0.01\\
69.62	0.01\\
69.63	0.01\\
69.64	0.01\\
69.65	0.01\\
69.66	0.01\\
69.67	0.01\\
69.68	0.01\\
69.69	0.01\\
69.7	0.01\\
69.71	0.01\\
69.72	0.01\\
69.73	0.01\\
69.74	0.01\\
69.75	0.01\\
69.76	0.01\\
69.77	0.01\\
69.78	0.01\\
69.79	0.01\\
69.8	0.01\\
69.81	0.01\\
69.82	0.01\\
69.83	0.01\\
69.84	0.01\\
69.85	0.01\\
69.86	0.01\\
69.87	0.01\\
69.88	0.01\\
69.89	0.01\\
69.9	0.01\\
69.91	0.01\\
69.92	0.01\\
69.93	0.01\\
69.94	0.01\\
69.95	0.01\\
69.96	0.01\\
69.97	0.01\\
69.98	0.01\\
69.99	0.01\\
70	0.01\\
70.01	0.01\\
70.02	0.01\\
70.03	0.01\\
70.04	0.01\\
70.05	0.01\\
70.06	0.01\\
70.07	0.01\\
70.08	0.01\\
70.09	0.01\\
70.1	0.01\\
70.11	0.01\\
70.12	0.01\\
70.13	0.01\\
70.14	0.01\\
70.15	0.01\\
70.16	0.01\\
70.17	0.01\\
70.18	0.01\\
70.19	0.01\\
70.2	0.01\\
70.21	0.01\\
70.22	0.01\\
70.23	0.01\\
70.24	0.01\\
70.25	0.01\\
70.26	0.01\\
70.27	0.01\\
70.28	0.01\\
70.29	0.01\\
70.3	0.01\\
70.31	0.01\\
70.32	0.01\\
70.33	0.01\\
70.34	0.01\\
70.35	0.01\\
70.36	0.01\\
70.37	0.01\\
70.38	0.01\\
70.39	0.01\\
70.4	0.01\\
70.41	0.01\\
70.42	0.01\\
70.43	0.01\\
70.44	0.01\\
70.45	0.01\\
70.46	0.01\\
70.47	0.01\\
70.48	0.01\\
70.49	0.01\\
70.5	0.01\\
70.51	0.01\\
70.52	0.01\\
70.53	0.01\\
70.54	0.01\\
70.55	0.01\\
70.56	0.01\\
70.57	0.01\\
70.58	0.01\\
70.59	0.01\\
70.6	0.01\\
70.61	0.01\\
70.62	0.01\\
70.63	0.01\\
70.64	0.01\\
70.65	0.01\\
70.66	0.01\\
70.67	0.01\\
70.68	0.01\\
70.69	0.01\\
70.7	0.01\\
70.71	0.01\\
70.72	0.01\\
70.73	0.01\\
70.74	0.01\\
70.75	0.01\\
70.76	0.01\\
70.77	0.01\\
70.78	0.01\\
70.79	0.01\\
70.8	0.01\\
70.81	0.01\\
70.82	0.01\\
70.83	0.01\\
70.84	0.01\\
70.85	0.01\\
70.86	0.01\\
70.87	0.01\\
70.88	0.01\\
70.89	0.01\\
70.9	0.01\\
70.91	0.01\\
70.92	0.01\\
70.93	0.01\\
70.94	0.01\\
70.95	0.01\\
70.96	0.01\\
70.97	0.01\\
70.98	0.01\\
70.99	0.01\\
71	0.01\\
71.01	0.01\\
71.02	0.01\\
71.03	0.01\\
71.04	0.01\\
71.05	0.01\\
71.06	0.01\\
71.07	0.01\\
71.08	0.01\\
71.09	0.01\\
71.1	0.01\\
71.11	0.01\\
71.12	0.01\\
71.13	0.01\\
71.14	0.01\\
71.15	0.01\\
71.16	0.01\\
71.17	0.01\\
71.18	0.01\\
71.19	0.01\\
71.2	0.01\\
71.21	0.01\\
71.22	0.01\\
71.23	0.01\\
71.24	0.01\\
71.25	0.01\\
71.26	0.01\\
71.27	0.01\\
71.28	0.01\\
71.29	0.01\\
71.3	0.01\\
71.31	0.01\\
71.32	0.01\\
71.33	0.01\\
71.34	0.01\\
71.35	0.01\\
71.36	0.01\\
71.37	0.01\\
71.38	0.01\\
71.39	0.01\\
71.4	0.01\\
71.41	0.01\\
71.42	0.01\\
71.43	0.01\\
71.44	0.01\\
71.45	0.01\\
71.46	0.01\\
71.47	0.01\\
71.48	0.01\\
71.49	0.01\\
71.5	0.01\\
71.51	0.01\\
71.52	0.01\\
71.53	0.01\\
71.54	0.01\\
71.55	0.01\\
71.56	0.01\\
71.57	0.01\\
71.58	0.01\\
71.59	0.01\\
71.6	0.01\\
71.61	0.01\\
71.62	0.01\\
71.63	0.01\\
71.64	0.01\\
71.65	0.01\\
71.66	0.01\\
71.67	0.01\\
71.68	0.01\\
71.69	0.01\\
71.7	0.01\\
71.71	0.01\\
71.72	0.01\\
71.73	0.01\\
71.74	0.01\\
71.75	0.01\\
71.76	0.01\\
71.77	0.01\\
71.78	0.01\\
71.79	0.01\\
71.8	0.01\\
71.81	0.01\\
71.82	0.01\\
71.83	0.01\\
71.84	0.01\\
71.85	0.01\\
71.86	0.01\\
71.87	0.01\\
71.88	0.01\\
71.89	0.01\\
71.9	0.01\\
71.91	0.01\\
71.92	0.01\\
71.93	0.01\\
71.94	0.01\\
71.95	0.01\\
71.96	0.01\\
71.97	0.01\\
71.98	0.01\\
71.99	0.01\\
72	0.01\\
72.01	0.01\\
72.02	0.01\\
72.03	0.01\\
72.04	0.01\\
72.05	0.01\\
72.06	0.01\\
72.07	0.01\\
72.08	0.01\\
72.09	0.01\\
72.1	0.01\\
72.11	0.01\\
72.12	0.01\\
72.13	0.01\\
72.14	0.01\\
72.15	0.01\\
72.16	0.01\\
72.17	0.01\\
72.18	0.01\\
72.19	0.01\\
72.2	0.01\\
72.21	0.01\\
72.22	0.01\\
72.23	0.01\\
72.24	0.01\\
72.25	0.01\\
72.26	0.01\\
72.27	0.01\\
72.28	0.01\\
72.29	0.01\\
72.3	0.01\\
72.31	0.01\\
72.32	0.01\\
72.33	0.01\\
72.34	0.01\\
72.35	0.01\\
72.36	0.01\\
72.37	0.01\\
72.38	0.01\\
72.39	0.01\\
72.4	0.01\\
72.41	0.01\\
72.42	0.01\\
72.43	0.01\\
72.44	0.01\\
72.45	0.01\\
72.46	0.01\\
72.47	0.01\\
72.48	0.01\\
72.49	0.01\\
72.5	0.01\\
72.51	0.01\\
72.52	0.01\\
72.53	0.01\\
72.54	0.01\\
72.55	0.01\\
72.56	0.01\\
72.57	0.01\\
72.58	0.01\\
72.59	0.01\\
72.6	0.01\\
72.61	0.01\\
72.62	0.01\\
72.63	0.01\\
72.64	0.01\\
72.65	0.01\\
72.66	0.01\\
72.67	0.01\\
72.68	0.01\\
72.69	0.01\\
72.7	0.01\\
72.71	0.01\\
72.72	0.01\\
72.73	0.01\\
72.74	0.01\\
72.75	0.01\\
72.76	0.01\\
72.77	0.01\\
72.78	0.01\\
72.79	0.01\\
72.8	0.01\\
72.81	0.01\\
72.82	0.01\\
72.83	0.01\\
72.84	0.01\\
72.85	0.01\\
72.86	0.01\\
72.87	0.01\\
72.88	0.01\\
72.89	0.01\\
72.9	0.01\\
72.91	0.01\\
72.92	0.01\\
72.93	0.01\\
72.94	0.01\\
72.95	0.01\\
72.96	0.01\\
72.97	0.01\\
72.98	0.01\\
72.99	0.01\\
73	0.01\\
73.01	0.01\\
73.02	0.01\\
73.03	0.01\\
73.04	0.01\\
73.05	0.01\\
73.06	0.01\\
73.07	0.01\\
73.08	0.01\\
73.09	0.01\\
73.1	0.01\\
73.11	0.01\\
73.12	0.01\\
73.13	0.01\\
73.14	0.01\\
73.15	0.01\\
73.16	0.01\\
73.17	0.01\\
73.18	0.01\\
73.19	0.01\\
73.2	0.01\\
73.21	0.01\\
73.22	0.01\\
73.23	0.01\\
73.24	0.01\\
73.25	0.01\\
73.26	0.01\\
73.27	0.01\\
73.28	0.01\\
73.29	0.01\\
73.3	0.01\\
73.31	0.01\\
73.32	0.01\\
73.33	0.01\\
73.34	0.01\\
73.35	0.01\\
73.36	0.01\\
73.37	0.01\\
73.38	0.01\\
73.39	0.01\\
73.4	0.01\\
73.41	0.01\\
73.42	0.01\\
73.43	0.01\\
73.44	0.01\\
73.45	0.01\\
73.46	0.01\\
73.47	0.01\\
73.48	0.01\\
73.49	0.01\\
73.5	0.01\\
73.51	0.01\\
73.52	0.01\\
73.53	0.01\\
73.54	0.01\\
73.55	0.01\\
73.56	0.01\\
73.57	0.01\\
73.58	0.01\\
73.59	0.01\\
73.6	0.01\\
73.61	0.01\\
73.62	0.01\\
73.63	0.01\\
73.64	0.01\\
73.65	0.01\\
73.66	0.01\\
73.67	0.01\\
73.68	0.01\\
73.69	0.01\\
73.7	0.01\\
73.71	0.01\\
73.72	0.01\\
73.73	0.01\\
73.74	0.01\\
73.75	0.01\\
73.76	0.01\\
73.77	0.01\\
73.78	0.01\\
73.79	0.01\\
73.8	0.01\\
73.81	0.01\\
73.82	0.01\\
73.83	0.01\\
73.84	0.01\\
73.85	0.01\\
73.86	0.01\\
73.87	0.01\\
73.88	0.01\\
73.89	0.01\\
73.9	0.01\\
73.91	0.01\\
73.92	0.01\\
73.93	0.01\\
73.94	0.01\\
73.95	0.01\\
73.96	0.01\\
73.97	0.01\\
73.98	0.01\\
73.99	0.01\\
74	0.01\\
74.01	0.01\\
74.02	0.01\\
74.03	0.01\\
74.04	0.01\\
74.05	0.01\\
74.06	0.01\\
74.07	0.01\\
74.08	0.01\\
74.09	0.01\\
74.1	0.01\\
74.11	0.01\\
74.12	0.01\\
74.13	0.01\\
74.14	0.01\\
74.15	0.01\\
74.16	0.01\\
74.17	0.01\\
74.18	0.01\\
74.19	0.01\\
74.2	0.01\\
74.21	0.01\\
74.22	0.01\\
74.23	0.01\\
74.24	0.01\\
74.25	0.01\\
74.26	0.01\\
74.27	0.01\\
74.28	0.01\\
74.29	0.01\\
74.3	0.01\\
74.31	0.01\\
74.32	0.01\\
74.33	0.01\\
74.34	0.01\\
74.35	0.01\\
74.36	0.01\\
74.37	0.01\\
74.38	0.01\\
74.39	0.01\\
74.4	0.01\\
74.41	0.01\\
74.42	0.01\\
74.43	0.01\\
74.44	0.01\\
74.45	0.01\\
74.46	0.01\\
74.47	0.01\\
74.48	0.01\\
74.49	0.01\\
74.5	0.01\\
74.51	0.01\\
74.52	0.01\\
74.53	0.01\\
74.54	0.01\\
74.55	0.01\\
74.56	0.01\\
74.57	0.01\\
74.58	0.01\\
74.59	0.01\\
74.6	0.01\\
74.61	0.01\\
74.62	0.01\\
74.63	0.01\\
74.64	0.01\\
74.65	0.01\\
74.66	0.01\\
74.67	0.01\\
74.68	0.01\\
74.69	0.01\\
74.7	0.01\\
74.71	0.01\\
74.72	0.01\\
74.73	0.01\\
74.74	0.01\\
74.75	0.01\\
74.76	0.01\\
74.77	0.01\\
74.78	0.01\\
74.79	0.01\\
74.8	0.01\\
74.81	0.01\\
74.82	0.01\\
74.83	0.01\\
74.84	0.01\\
74.85	0.01\\
74.86	0.01\\
74.87	0.01\\
74.88	0.01\\
74.89	0.01\\
74.9	0.01\\
74.91	0.01\\
74.92	0.01\\
74.93	0.01\\
74.94	0.01\\
74.95	0.01\\
74.96	0.01\\
74.97	0.01\\
74.98	0.01\\
74.99	0.01\\
75	0.01\\
75.01	0.01\\
75.02	0.01\\
75.03	0.01\\
75.04	0.01\\
75.05	0.01\\
75.06	0.01\\
75.07	0.01\\
75.08	0.01\\
75.09	0.01\\
75.1	0.01\\
75.11	0.01\\
75.12	0.01\\
75.13	0.01\\
75.14	0.01\\
75.15	0.01\\
75.16	0.01\\
75.17	0.01\\
75.18	0.01\\
75.19	0.01\\
75.2	0.01\\
75.21	0.01\\
75.22	0.01\\
75.23	0.01\\
75.24	0.01\\
75.25	0.01\\
75.26	0.01\\
75.27	0.01\\
75.28	0.01\\
75.29	0.01\\
75.3	0.01\\
75.31	0.01\\
75.32	0.01\\
75.33	0.01\\
75.34	0.01\\
75.35	0.01\\
75.36	0.01\\
75.37	0.01\\
75.38	0.01\\
75.39	0.01\\
75.4	0.01\\
75.41	0.01\\
75.42	0.01\\
75.43	0.01\\
75.44	0.01\\
75.45	0.01\\
75.46	0.01\\
75.47	0.01\\
75.48	0.01\\
75.49	0.01\\
75.5	0.01\\
75.51	0.01\\
75.52	0.01\\
75.53	0.01\\
75.54	0.01\\
75.55	0.01\\
75.56	0.01\\
75.57	0.01\\
75.58	0.01\\
75.59	0.01\\
75.6	0.01\\
75.61	0.01\\
75.62	0.01\\
75.63	0.01\\
75.64	0.01\\
75.65	0.01\\
75.66	0.01\\
75.67	0.01\\
75.68	0.01\\
75.69	0.01\\
75.7	0.01\\
75.71	0.01\\
75.72	0.01\\
75.73	0.01\\
75.74	0.01\\
75.75	0.01\\
75.76	0.01\\
75.77	0.01\\
75.78	0.01\\
75.79	0.01\\
75.8	0.01\\
75.81	0.01\\
75.82	0.01\\
75.83	0.01\\
75.84	0.01\\
75.85	0.01\\
75.86	0.01\\
75.87	0.01\\
75.88	0.01\\
75.89	0.01\\
75.9	0.01\\
75.91	0.01\\
75.92	0.01\\
75.93	0.01\\
75.94	0.01\\
75.95	0.01\\
75.96	0.01\\
75.97	0.01\\
75.98	0.01\\
75.99	0.01\\
76	0.01\\
76.01	0.01\\
76.02	0.01\\
76.03	0.01\\
76.04	0.01\\
76.05	0.01\\
76.06	0.01\\
76.07	0.01\\
76.08	0.01\\
76.09	0.01\\
76.1	0.01\\
76.11	0.01\\
76.12	0.01\\
76.13	0.01\\
76.14	0.01\\
76.15	0.01\\
76.16	0.01\\
76.17	0.01\\
76.18	0.01\\
76.19	0.01\\
76.2	0.01\\
76.21	0.01\\
76.22	0.01\\
76.23	0.01\\
76.24	0.01\\
76.25	0.01\\
76.26	0.01\\
76.27	0.01\\
76.28	0.01\\
76.29	0.01\\
76.3	0.01\\
76.31	0.01\\
76.32	0.01\\
76.33	0.01\\
76.34	0.01\\
76.35	0.01\\
76.36	0.01\\
76.37	0.01\\
76.38	0.01\\
76.39	0.01\\
76.4	0.01\\
76.41	0.01\\
76.42	0.01\\
76.43	0.01\\
76.44	0.01\\
76.45	0.01\\
76.46	0.01\\
76.47	0.01\\
76.48	0.01\\
76.49	0.01\\
76.5	0.01\\
76.51	0.01\\
76.52	0.01\\
76.53	0.01\\
76.54	0.01\\
76.55	0.01\\
76.56	0.01\\
76.57	0.01\\
76.58	0.01\\
76.59	0.01\\
76.6	0.01\\
76.61	0.01\\
76.62	0.01\\
76.63	0.01\\
76.64	0.01\\
76.65	0.01\\
76.66	0.01\\
76.67	0.01\\
76.68	0.01\\
76.69	0.01\\
76.7	0.01\\
76.71	0.01\\
76.72	0.01\\
76.73	0.01\\
76.74	0.01\\
76.75	0.01\\
76.76	0.01\\
76.77	0.01\\
76.78	0.01\\
76.79	0.01\\
76.8	0.01\\
76.81	0.01\\
76.82	0.01\\
76.83	0.01\\
76.84	0.01\\
76.85	0.01\\
76.86	0.01\\
76.87	0.01\\
76.88	0.01\\
76.89	0.01\\
76.9	0.01\\
76.91	0.01\\
76.92	0.01\\
76.93	0.01\\
76.94	0.01\\
76.95	0.01\\
76.96	0.01\\
76.97	0.01\\
76.98	0.01\\
76.99	0.01\\
77	0.01\\
77.01	0.01\\
77.02	0.01\\
77.03	0.01\\
77.04	0.01\\
77.05	0.01\\
77.06	0.01\\
77.07	0.01\\
77.08	0.01\\
77.09	0.01\\
77.1	0.01\\
77.11	0.01\\
77.12	0.01\\
77.13	0.01\\
77.14	0.01\\
77.15	0.01\\
77.16	0.01\\
77.17	0.01\\
77.18	0.01\\
77.19	0.01\\
77.2	0.01\\
77.21	0.01\\
77.22	0.01\\
77.23	0.01\\
77.24	0.01\\
77.25	0.01\\
77.26	0.01\\
77.27	0.01\\
77.28	0.01\\
77.29	0.01\\
77.3	0.01\\
77.31	0.01\\
77.32	0.01\\
77.33	0.01\\
77.34	0.01\\
77.35	0.01\\
77.36	0.01\\
77.37	0.01\\
77.38	0.01\\
77.39	0.01\\
77.4	0.01\\
77.41	0.01\\
77.42	0.01\\
77.43	0.01\\
77.44	0.01\\
77.45	0.01\\
77.46	0.01\\
77.47	0.01\\
77.48	0.01\\
77.49	0.01\\
77.5	0.01\\
77.51	0.01\\
77.52	0.01\\
77.53	0.01\\
77.54	0.01\\
77.55	0.01\\
77.56	0.01\\
77.57	0.01\\
77.58	0.01\\
77.59	0.01\\
77.6	0.01\\
77.61	0.01\\
77.62	0.01\\
77.63	0.01\\
77.64	0.01\\
77.65	0.01\\
77.66	0.01\\
77.67	0.01\\
77.68	0.01\\
77.69	0.01\\
77.7	0.01\\
77.71	0.01\\
77.72	0.01\\
77.73	0.01\\
77.74	0.01\\
77.75	0.01\\
77.76	0.01\\
77.77	0.01\\
77.78	0.01\\
77.79	0.01\\
77.8	0.01\\
77.81	0.01\\
77.82	0.01\\
77.83	0.01\\
77.84	0.01\\
77.85	0.01\\
77.86	0.01\\
77.87	0.01\\
77.88	0.01\\
77.89	0.01\\
77.9	0.01\\
77.91	0.01\\
77.92	0.01\\
77.93	0.01\\
77.94	0.01\\
77.95	0.01\\
77.96	0.01\\
77.97	0.01\\
77.98	0.01\\
77.99	0.01\\
78	0.01\\
78.01	0.01\\
78.02	0.01\\
78.03	0.01\\
78.04	0.01\\
78.05	0.01\\
78.06	0.01\\
78.07	0.01\\
78.08	0.01\\
78.09	0.01\\
78.1	0.01\\
78.11	0.01\\
78.12	0.01\\
78.13	0.01\\
78.14	0.01\\
78.15	0.01\\
78.16	0.01\\
78.17	0.01\\
78.18	0.01\\
78.19	0.01\\
78.2	0.01\\
78.21	0.01\\
78.22	0.01\\
78.23	0.01\\
78.24	0.01\\
78.25	0.01\\
78.26	0.01\\
78.27	0.01\\
78.28	0.01\\
78.29	0.01\\
78.3	0.01\\
78.31	0.01\\
78.32	0.01\\
78.33	0.01\\
78.34	0.01\\
78.35	0.01\\
78.36	0.01\\
78.37	0.01\\
78.38	0.01\\
78.39	0.01\\
78.4	0.01\\
78.41	0.01\\
78.42	0.01\\
78.43	0.01\\
78.44	0.01\\
78.45	0.01\\
78.46	0.01\\
78.47	0.01\\
78.48	0.01\\
78.49	0.01\\
78.5	0.01\\
78.51	0.01\\
78.52	0.01\\
78.53	0.01\\
78.54	0.01\\
78.55	0.01\\
78.56	0.01\\
78.57	0.01\\
78.58	0.01\\
78.59	0.01\\
78.6	0.01\\
78.61	0.01\\
78.62	0.01\\
78.63	0.01\\
78.64	0.01\\
78.65	0.01\\
78.66	0.01\\
78.67	0.01\\
78.68	0.01\\
78.69	0.01\\
78.7	0.01\\
78.71	0.01\\
78.72	0.01\\
78.73	0.01\\
78.74	0.01\\
78.75	0.01\\
78.76	0.01\\
78.77	0.01\\
78.78	0.01\\
78.79	0.01\\
78.8	0.01\\
78.81	0.01\\
78.82	0.01\\
78.83	0.01\\
78.84	0.01\\
78.85	0.01\\
78.86	0.01\\
78.87	0.01\\
78.88	0.01\\
78.89	0.01\\
78.9	0.01\\
78.91	0.01\\
78.92	0.01\\
78.93	0.01\\
78.94	0.01\\
78.95	0.01\\
78.96	0.01\\
78.97	0.01\\
78.98	0.01\\
78.99	0.01\\
79	0.01\\
79.01	0.01\\
79.02	0.01\\
79.03	0.01\\
79.04	0.01\\
79.05	0.01\\
79.06	0.01\\
79.07	0.01\\
79.08	0.01\\
79.09	0.01\\
79.1	0.01\\
79.11	0.01\\
79.12	0.01\\
79.13	0.01\\
79.14	0.01\\
79.15	0.01\\
79.16	0.01\\
79.17	0.01\\
79.18	0.01\\
79.19	0.01\\
79.2	0.01\\
79.21	0.01\\
79.22	0.01\\
79.23	0.01\\
79.24	0.01\\
79.25	0.01\\
79.26	0.01\\
79.27	0.01\\
79.28	0.01\\
79.29	0.01\\
79.3	0.01\\
79.31	0.01\\
79.32	0.01\\
79.33	0.01\\
79.34	0.01\\
79.35	0.01\\
79.36	0.01\\
79.37	0.01\\
79.38	0.01\\
79.39	0.01\\
79.4	0.01\\
79.41	0.01\\
79.42	0.01\\
79.43	0.01\\
79.44	0.01\\
79.45	0.01\\
79.46	0.01\\
79.47	0.01\\
79.48	0.01\\
79.49	0.01\\
79.5	0.01\\
79.51	0.01\\
79.52	0.01\\
79.53	0.01\\
79.54	0.01\\
79.55	0.01\\
79.56	0.01\\
79.57	0.01\\
79.58	0.01\\
79.59	0.01\\
79.6	0.01\\
79.61	0.01\\
79.62	0.01\\
79.63	0.01\\
79.64	0.01\\
79.65	0.01\\
79.66	0.01\\
79.67	0.01\\
79.68	0.01\\
79.69	0.01\\
79.7	0.01\\
79.71	0.01\\
79.72	0.01\\
79.73	0.01\\
79.74	0.01\\
79.75	0.01\\
79.76	0.01\\
79.77	0.01\\
79.78	0.01\\
79.79	0.01\\
79.8	0.01\\
79.81	0.01\\
79.82	0.01\\
79.83	0.01\\
79.84	0.01\\
79.85	0.01\\
79.86	0.01\\
79.87	0.01\\
79.88	0.01\\
79.89	0.01\\
79.9	0.01\\
79.91	0.01\\
79.92	0.01\\
79.93	0.01\\
79.94	0.01\\
79.95	0.01\\
79.96	0.01\\
79.97	0.01\\
79.98	0.01\\
79.99	0.01\\
80	0.01\\
80.01	0.01\\
};
\addplot [color=mycolor1,dashed]
  table[row sep=crcr]{%
80.01	0.01\\
80.02	0.01\\
80.03	0.01\\
80.04	0.01\\
80.05	0.01\\
80.06	0.01\\
80.07	0.01\\
80.08	0.01\\
80.09	0.01\\
80.1	0.01\\
80.11	0.01\\
80.12	0.01\\
80.13	0.01\\
80.14	0.01\\
80.15	0.01\\
80.16	0.01\\
80.17	0.01\\
80.18	0.01\\
80.19	0.01\\
80.2	0.01\\
80.21	0.01\\
80.22	0.01\\
80.23	0.01\\
80.24	0.01\\
80.25	0.01\\
80.26	0.01\\
80.27	0.01\\
80.28	0.01\\
80.29	0.01\\
80.3	0.01\\
80.31	0.01\\
80.32	0.01\\
80.33	0.01\\
80.34	0.01\\
80.35	0.01\\
80.36	0.01\\
80.37	0.01\\
80.38	0.01\\
80.39	0.01\\
80.4	0.01\\
80.41	0.01\\
80.42	0.01\\
80.43	0.01\\
80.44	0.01\\
80.45	0.01\\
80.46	0.01\\
80.47	0.01\\
80.48	0.01\\
80.49	0.01\\
80.5	0.01\\
80.51	0.01\\
80.52	0.01\\
80.53	0.01\\
80.54	0.01\\
80.55	0.01\\
80.56	0.01\\
80.57	0.01\\
80.58	0.01\\
80.59	0.01\\
80.6	0.01\\
80.61	0.01\\
80.62	0.01\\
80.63	0.01\\
80.64	0.01\\
80.65	0.01\\
80.66	0.01\\
80.67	0.01\\
80.68	0.01\\
80.69	0.01\\
80.7	0.01\\
80.71	0.01\\
80.72	0.01\\
80.73	0.01\\
80.74	0.01\\
80.75	0.01\\
80.76	0.01\\
80.77	0.01\\
80.78	0.01\\
80.79	0.01\\
80.8	0.01\\
80.81	0.01\\
80.82	0.01\\
80.83	0.01\\
80.84	0.01\\
80.85	0.01\\
80.86	0.01\\
80.87	0.01\\
80.88	0.01\\
80.89	0.01\\
80.9	0.01\\
80.91	0.01\\
80.92	0.01\\
80.93	0.01\\
80.94	0.01\\
80.95	0.01\\
80.96	0.01\\
80.97	0.01\\
80.98	0.01\\
80.99	0.01\\
81	0.01\\
81.01	0.01\\
81.02	0.01\\
81.03	0.01\\
81.04	0.01\\
81.05	0.01\\
81.06	0.01\\
81.07	0.01\\
81.08	0.01\\
81.09	0.01\\
81.1	0.01\\
81.11	0.01\\
81.12	0.01\\
81.13	0.01\\
81.14	0.01\\
81.15	0.01\\
81.16	0.01\\
81.17	0.01\\
81.18	0.01\\
81.19	0.01\\
81.2	0.01\\
81.21	0.01\\
81.22	0.01\\
81.23	0.01\\
81.24	0.01\\
81.25	0.01\\
81.26	0.01\\
81.27	0.01\\
81.28	0.01\\
81.29	0.01\\
81.3	0.01\\
81.31	0.01\\
81.32	0.01\\
81.33	0.01\\
81.34	0.01\\
81.35	0.01\\
81.36	0.01\\
81.37	0.01\\
81.38	0.01\\
81.39	0.01\\
81.4	0.01\\
81.41	0.01\\
81.42	0.01\\
81.43	0.01\\
81.44	0.01\\
81.45	0.01\\
81.46	0.01\\
81.47	0.01\\
81.48	0.01\\
81.49	0.01\\
81.5	0.01\\
81.51	0.01\\
81.52	0.01\\
81.53	0.01\\
81.54	0.01\\
81.55	0.01\\
81.56	0.01\\
81.57	0.01\\
81.58	0.01\\
81.59	0.01\\
81.6	0.01\\
81.61	0.01\\
81.62	0.01\\
81.63	0.01\\
81.64	0.01\\
81.65	0.01\\
81.66	0.01\\
81.67	0.01\\
81.68	0.01\\
81.69	0.01\\
81.7	0.01\\
81.71	0.01\\
81.72	0.01\\
81.73	0.01\\
81.74	0.01\\
81.75	0.01\\
81.76	0.01\\
81.77	0.01\\
81.78	0.01\\
81.79	0.01\\
81.8	0.01\\
81.81	0.01\\
81.82	0.01\\
81.83	0.01\\
81.84	0.01\\
81.85	0.01\\
81.86	0.01\\
81.87	0.01\\
81.88	0.01\\
81.89	0.01\\
81.9	0.01\\
81.91	0.01\\
81.92	0.01\\
81.93	0.01\\
81.94	0.01\\
81.95	0.01\\
81.96	0.01\\
81.97	0.01\\
81.98	0.01\\
81.99	0.01\\
82	0.01\\
82.01	0.01\\
82.02	0.01\\
82.03	0.01\\
82.04	0.01\\
82.05	0.01\\
82.06	0.01\\
82.07	0.01\\
82.08	0.01\\
82.09	0.01\\
82.1	0.01\\
82.11	0.01\\
82.12	0.01\\
82.13	0.01\\
82.14	0.01\\
82.15	0.01\\
82.16	0.01\\
82.17	0.01\\
82.18	0.01\\
82.19	0.01\\
82.2	0.01\\
82.21	0.01\\
82.22	0.01\\
82.23	0.01\\
82.24	0.01\\
82.25	0.01\\
82.26	0.01\\
82.27	0.01\\
82.28	0.01\\
82.29	0.01\\
82.3	0.01\\
82.31	0.01\\
82.32	0.01\\
82.33	0.01\\
82.34	0.01\\
82.35	0.01\\
82.36	0.01\\
82.37	0.01\\
82.38	0.01\\
82.39	0.01\\
82.4	0.01\\
82.41	0.01\\
82.42	0.01\\
82.43	0.01\\
82.44	0.01\\
82.45	0.01\\
82.46	0.01\\
82.47	0.01\\
82.48	0.01\\
82.49	0.01\\
82.5	0.01\\
82.51	0.01\\
82.52	0.01\\
82.53	0.01\\
82.54	0.01\\
82.55	0.01\\
82.56	0.01\\
82.57	0.01\\
82.58	0.01\\
82.59	0.01\\
82.6	0.01\\
82.61	0.01\\
82.62	0.01\\
82.63	0.01\\
82.64	0.01\\
82.65	0.01\\
82.66	0.01\\
82.67	0.01\\
82.68	0.01\\
82.69	0.01\\
82.7	0.01\\
82.71	0.01\\
82.72	0.01\\
82.73	0.01\\
82.74	0.01\\
82.75	0.01\\
82.76	0.01\\
82.77	0.01\\
82.78	0.01\\
82.79	0.01\\
82.8	0.01\\
82.81	0.01\\
82.82	0.01\\
82.83	0.01\\
82.84	0.01\\
82.85	0.01\\
82.86	0.01\\
82.87	0.01\\
82.88	0.01\\
82.89	0.01\\
82.9	0.01\\
82.91	0.01\\
82.92	0.01\\
82.93	0.01\\
82.94	0.01\\
82.95	0.01\\
82.96	0.01\\
82.97	0.01\\
82.98	0.01\\
82.99	0.01\\
83	0.01\\
83.01	0.01\\
83.02	0.01\\
83.03	0.01\\
83.04	0.01\\
83.05	0.01\\
83.06	0.01\\
83.07	0.01\\
83.08	0.01\\
83.09	0.01\\
83.1	0.01\\
83.11	0.01\\
83.12	0.01\\
83.13	0.01\\
83.14	0.01\\
83.15	0.01\\
83.16	0.01\\
83.17	0.01\\
83.18	0.01\\
83.19	0.01\\
83.2	0.01\\
83.21	0.01\\
83.22	0.01\\
83.23	0.01\\
83.24	0.01\\
83.25	0.01\\
83.26	0.01\\
83.27	0.01\\
83.28	0.01\\
83.29	0.01\\
83.3	0.01\\
83.31	0.01\\
83.32	0.01\\
83.33	0.01\\
83.34	0.01\\
83.35	0.01\\
83.36	0.01\\
83.37	0.01\\
83.38	0.01\\
83.39	0.01\\
83.4	0.01\\
83.41	0.01\\
83.42	0.01\\
83.43	0.01\\
83.44	0.01\\
83.45	0.01\\
83.46	0.01\\
83.47	0.01\\
83.48	0.01\\
83.49	0.01\\
83.5	0.01\\
83.51	0.01\\
83.52	0.01\\
83.53	0.01\\
83.54	0.01\\
83.55	0.01\\
83.56	0.01\\
83.57	0.01\\
83.58	0.01\\
83.59	0.01\\
83.6	0.01\\
83.61	0.01\\
83.62	0.01\\
83.63	0.01\\
83.64	0.01\\
83.65	0.01\\
83.66	0.01\\
83.67	0.01\\
83.68	0.01\\
83.69	0.01\\
83.7	0.01\\
83.71	0.01\\
83.72	0.01\\
83.73	0.01\\
83.74	0.01\\
83.75	0.01\\
83.76	0.01\\
83.77	0.01\\
83.78	0.01\\
83.79	0.01\\
83.8	0.01\\
83.81	0.01\\
83.82	0.01\\
83.83	0.01\\
83.84	0.01\\
83.85	0.01\\
83.86	0.01\\
83.87	0.01\\
83.88	0.01\\
83.89	0.01\\
83.9	0.01\\
83.91	0.01\\
83.92	0.01\\
83.93	0.01\\
83.94	0.01\\
83.95	0.01\\
83.96	0.01\\
83.97	0.01\\
83.98	0.01\\
83.99	0.01\\
84	0.01\\
84.01	0.01\\
84.02	0.01\\
84.03	0.01\\
84.04	0.01\\
84.05	0.01\\
84.06	0.01\\
84.07	0.01\\
84.08	0.01\\
84.09	0.01\\
84.1	0.01\\
84.11	0.01\\
84.12	0.01\\
84.13	0.01\\
84.14	0.01\\
84.15	0.01\\
84.16	0.01\\
84.17	0.01\\
84.18	0.01\\
84.19	0.01\\
84.2	0.01\\
84.21	0.01\\
84.22	0.01\\
84.23	0.01\\
84.24	0.01\\
84.25	0.01\\
84.26	0.01\\
84.27	0.01\\
84.28	0.01\\
84.29	0.01\\
84.3	0.01\\
84.31	0.01\\
84.32	0.01\\
84.33	0.01\\
84.34	0.01\\
84.35	0.01\\
84.36	0.01\\
84.37	0.01\\
84.38	0.01\\
84.39	0.01\\
84.4	0.01\\
84.41	0.01\\
84.42	0.01\\
84.43	0.01\\
84.44	0.01\\
84.45	0.01\\
84.46	0.01\\
84.47	0.01\\
84.48	0.01\\
84.49	0.01\\
84.5	0.01\\
84.51	0.01\\
84.52	0.01\\
84.53	0.01\\
84.54	0.01\\
84.55	0.01\\
84.56	0.01\\
84.57	0.01\\
84.58	0.01\\
84.59	0.01\\
84.6	0.01\\
84.61	0.01\\
84.62	0.01\\
84.63	0.01\\
84.64	0.01\\
84.65	0.01\\
84.66	0.01\\
84.67	0.01\\
84.68	0.01\\
84.69	0.01\\
84.7	0.01\\
84.71	0.01\\
84.72	0.01\\
84.73	0.01\\
84.74	0.01\\
84.75	0.01\\
84.76	0.01\\
84.77	0.01\\
84.78	0.01\\
84.79	0.01\\
84.8	0.01\\
84.81	0.01\\
84.82	0.01\\
84.83	0.01\\
84.84	0.01\\
84.85	0.01\\
84.86	0.01\\
84.87	0.01\\
84.88	0.01\\
84.89	0.01\\
84.9	0.01\\
84.91	0.01\\
84.92	0.01\\
84.93	0.01\\
84.94	0.01\\
84.95	0.01\\
84.96	0.01\\
84.97	0.01\\
84.98	0.01\\
84.99	0.01\\
85	0.01\\
85.01	0.01\\
85.02	0.01\\
85.03	0.01\\
85.04	0.01\\
85.05	0.01\\
85.06	0.01\\
85.07	0.01\\
85.08	0.01\\
85.09	0.01\\
85.1	0.01\\
85.11	0.01\\
85.12	0.01\\
85.13	0.01\\
85.14	0.01\\
85.15	0.01\\
85.16	0.01\\
85.17	0.01\\
85.18	0.01\\
85.19	0.01\\
85.2	0.01\\
85.21	0.01\\
85.22	0.01\\
85.23	0.01\\
85.24	0.01\\
85.25	0.01\\
85.26	0.01\\
85.27	0.01\\
85.28	0.01\\
85.29	0.01\\
85.3	0.01\\
85.31	0.01\\
85.32	0.01\\
85.33	0.01\\
85.34	0.01\\
85.35	0.01\\
85.36	0.01\\
85.37	0.01\\
85.38	0.01\\
85.39	0.01\\
85.4	0.01\\
85.41	0.01\\
85.42	0.01\\
85.43	0.01\\
85.44	0.01\\
85.45	0.01\\
85.46	0.01\\
85.47	0.01\\
85.48	0.01\\
85.49	0.01\\
85.5	0.01\\
85.51	0.01\\
85.52	0.01\\
85.53	0.01\\
85.54	0.01\\
85.55	0.01\\
85.56	0.01\\
85.57	0.01\\
85.58	0.01\\
85.59	0.01\\
85.6	0.01\\
85.61	0.01\\
85.62	0.01\\
85.63	0.01\\
85.64	0.01\\
85.65	0.01\\
85.66	0.01\\
85.67	0.01\\
85.68	0.01\\
85.69	0.01\\
85.7	0.01\\
85.71	0.01\\
85.72	0.01\\
85.73	0.01\\
85.74	0.01\\
85.75	0.01\\
85.76	0.01\\
85.77	0.01\\
85.78	0.01\\
85.79	0.01\\
85.8	0.01\\
85.81	0.01\\
85.82	0.01\\
85.83	0.01\\
85.84	0.01\\
85.85	0.01\\
85.86	0.01\\
85.87	0.01\\
85.88	0.01\\
85.89	0.01\\
85.9	0.01\\
85.91	0.01\\
85.92	0.01\\
85.93	0.01\\
85.94	0.01\\
85.95	0.01\\
85.96	0.01\\
85.97	0.01\\
85.98	0.01\\
85.99	0.01\\
86	0.01\\
86.01	0.01\\
86.02	0.01\\
86.03	0.01\\
86.04	0.01\\
86.05	0.01\\
86.06	0.01\\
86.07	0.01\\
86.08	0.01\\
86.09	0.01\\
86.1	0.01\\
86.11	0.01\\
86.12	0.01\\
86.13	0.01\\
86.14	0.01\\
86.15	0.01\\
86.16	0.01\\
86.17	0.01\\
86.18	0.01\\
86.19	0.01\\
86.2	0.01\\
86.21	0.01\\
86.22	0.01\\
86.23	0.01\\
86.24	0.01\\
86.25	0.01\\
86.26	0.01\\
86.27	0.01\\
86.28	0.01\\
86.29	0.01\\
86.3	0.01\\
86.31	0.01\\
86.32	0.01\\
86.33	0.01\\
86.34	0.01\\
86.35	0.01\\
86.36	0.01\\
86.37	0.01\\
86.38	0.01\\
86.39	0.01\\
86.4	0.01\\
86.41	0.01\\
86.42	0.01\\
86.43	0.01\\
86.44	0.01\\
86.45	0.01\\
86.46	0.01\\
86.47	0.01\\
86.48	0.01\\
86.49	0.01\\
86.5	0.01\\
86.51	0.01\\
86.52	0.01\\
86.53	0.01\\
86.54	0.01\\
86.55	0.01\\
86.56	0.01\\
86.57	0.01\\
86.58	0.01\\
86.59	0.01\\
86.6	0.01\\
86.61	0.01\\
86.62	0.01\\
86.63	0.01\\
86.64	0.01\\
86.65	0.01\\
86.66	0.01\\
86.67	0.01\\
86.68	0.01\\
86.69	0.01\\
86.7	0.01\\
86.71	0.01\\
86.72	0.01\\
86.73	0.01\\
86.74	0.01\\
86.75	0.01\\
86.76	0.01\\
86.77	0.01\\
86.78	0.01\\
86.79	0.01\\
86.8	0.01\\
86.81	0.01\\
86.82	0.01\\
86.83	0.01\\
86.84	0.01\\
86.85	0.01\\
86.86	0.01\\
86.87	0.01\\
86.88	0.01\\
86.89	0.01\\
86.9	0.01\\
86.91	0.01\\
86.92	0.01\\
86.93	0.01\\
86.94	0.01\\
86.95	0.01\\
86.96	0.01\\
86.97	0.01\\
86.98	0.01\\
86.99	0.01\\
87	0.01\\
87.01	0.01\\
87.02	0.01\\
87.03	0.01\\
87.04	0.01\\
87.05	0.01\\
87.06	0.01\\
87.07	0.01\\
87.08	0.01\\
87.09	0.01\\
87.1	0.01\\
87.11	0.01\\
87.12	0.01\\
87.13	0.01\\
87.14	0.01\\
87.15	0.01\\
87.16	0.01\\
87.17	0.01\\
87.18	0.01\\
87.19	0.01\\
87.2	0.01\\
87.21	0.01\\
87.22	0.01\\
87.23	0.01\\
87.24	0.01\\
87.25	0.01\\
87.26	0.01\\
87.27	0.01\\
87.28	0.01\\
87.29	0.01\\
87.3	0.01\\
87.31	0.01\\
87.32	0.01\\
87.33	0.01\\
87.34	0.01\\
87.35	0.01\\
87.36	0.01\\
87.37	0.01\\
87.38	0.01\\
87.39	0.01\\
87.4	0.01\\
87.41	0.01\\
87.42	0.01\\
87.43	0.01\\
87.44	0.01\\
87.45	0.01\\
87.46	0.01\\
87.47	0.01\\
87.48	0.01\\
87.49	0.01\\
87.5	0.01\\
87.51	0.01\\
87.52	0.01\\
87.53	0.01\\
87.54	0.01\\
87.55	0.01\\
87.56	0.01\\
87.57	0.01\\
87.58	0.01\\
87.59	0.01\\
87.6	0.01\\
87.61	0.01\\
87.62	0.01\\
87.63	0.01\\
87.64	0.01\\
87.65	0.01\\
87.66	0.01\\
87.67	0.01\\
87.68	0.01\\
87.69	0.01\\
87.7	0.01\\
87.71	0.01\\
87.72	0.01\\
87.73	0.01\\
87.74	0.01\\
87.75	0.01\\
87.76	0.01\\
87.77	0.01\\
87.78	0.01\\
87.79	0.01\\
87.8	0.01\\
87.81	0.01\\
87.82	0.01\\
87.83	0.01\\
87.84	0.01\\
87.85	0.01\\
87.86	0.01\\
87.87	0.01\\
87.88	0.01\\
87.89	0.01\\
87.9	0.01\\
87.91	0.01\\
87.92	0.01\\
87.93	0.01\\
87.94	0.01\\
87.95	0.01\\
87.96	0.01\\
87.97	0.01\\
87.98	0.01\\
87.99	0.01\\
88	0.01\\
88.01	0.01\\
88.02	0.01\\
88.03	0.01\\
88.04	0.01\\
88.05	0.01\\
88.06	0.01\\
88.07	0.01\\
88.08	0.01\\
88.09	0.01\\
88.1	0.01\\
88.11	0.01\\
88.12	0.01\\
88.13	0.01\\
88.14	0.01\\
88.15	0.01\\
88.16	0.01\\
88.17	0.01\\
88.18	0.01\\
88.19	0.01\\
88.2	0.01\\
88.21	0.01\\
88.22	0.01\\
88.23	0.01\\
88.24	0.01\\
88.25	0.01\\
88.26	0.01\\
88.27	0.01\\
88.28	0.01\\
88.29	0.01\\
88.3	0.01\\
88.31	0.01\\
88.32	0.01\\
88.33	0.01\\
88.34	0.01\\
88.35	0.01\\
88.36	0.01\\
88.37	0.01\\
88.38	0.01\\
88.39	0.01\\
88.4	0.01\\
88.41	0.01\\
88.42	0.01\\
88.43	0.01\\
88.44	0.01\\
88.45	0.01\\
88.46	0.01\\
88.47	0.01\\
88.48	0.01\\
88.49	0.01\\
88.5	0.01\\
88.51	0.01\\
88.52	0.01\\
88.53	0.01\\
88.54	0.01\\
88.55	0.01\\
88.56	0.01\\
88.57	0.01\\
88.58	0.01\\
88.59	0.01\\
88.6	0.01\\
88.61	0.01\\
88.62	0.01\\
88.63	0.01\\
88.64	0.01\\
88.65	0.01\\
88.66	0.01\\
88.67	0.01\\
88.68	0.01\\
88.69	0.01\\
88.7	0.01\\
88.71	0.01\\
88.72	0.01\\
88.73	0.01\\
88.74	0.01\\
88.75	0.01\\
88.76	0.01\\
88.77	0.01\\
88.78	0.01\\
88.79	0.01\\
88.8	0.01\\
88.81	0.01\\
88.82	0.01\\
88.83	0.01\\
88.84	0.01\\
88.85	0.01\\
88.86	0.01\\
88.87	0.01\\
88.88	0.01\\
88.89	0.01\\
88.9	0.01\\
88.91	0.01\\
88.92	0.01\\
88.93	0.01\\
88.94	0.01\\
88.95	0.01\\
88.96	0.01\\
88.97	0.01\\
88.98	0.01\\
88.99	0.01\\
89	0.01\\
89.01	0.01\\
89.02	0.01\\
89.03	0.01\\
89.04	0.01\\
89.05	0.01\\
89.06	0.01\\
89.07	0.01\\
89.08	0.01\\
89.09	0.01\\
89.1	0.01\\
89.11	0.01\\
89.12	0.01\\
89.13	0.01\\
89.14	0.01\\
89.15	0.01\\
89.16	0.01\\
89.17	0.01\\
89.18	0.01\\
89.19	0.01\\
89.2	0.01\\
89.21	0.01\\
89.22	0.01\\
89.23	0.01\\
89.24	0.01\\
89.25	0.01\\
89.26	0.01\\
89.27	0.01\\
89.28	0.01\\
89.29	0.01\\
89.3	0.01\\
89.31	0.01\\
89.32	0.01\\
89.33	0.01\\
89.34	0.01\\
89.35	0.01\\
89.36	0.01\\
89.37	0.01\\
89.38	0.01\\
89.39	0.01\\
89.4	0.01\\
89.41	0.01\\
89.42	0.01\\
89.43	0.01\\
89.44	0.01\\
89.45	0.01\\
89.46	0.01\\
89.47	0.01\\
89.48	0.01\\
89.49	0.01\\
89.5	0.01\\
89.51	0.01\\
89.52	0.01\\
89.53	0.01\\
89.54	0.01\\
89.55	0.01\\
89.56	0.01\\
89.57	0.01\\
89.58	0.01\\
89.59	0.01\\
89.6	0.01\\
89.61	0.01\\
89.62	0.01\\
89.63	0.01\\
89.64	0.01\\
89.65	0.01\\
89.66	0.01\\
89.67	0.01\\
89.68	0.01\\
89.69	0.01\\
89.7	0.01\\
89.71	0.01\\
89.72	0.01\\
89.73	0.01\\
89.74	0.01\\
89.75	0.01\\
89.76	0.01\\
89.77	0.01\\
89.78	0.01\\
89.79	0.01\\
89.8	0.01\\
89.81	0.01\\
89.82	0.01\\
89.83	0.01\\
89.84	0.01\\
89.85	0.01\\
89.86	0.01\\
89.87	0.01\\
89.88	0.01\\
89.89	0.01\\
89.9	0.01\\
89.91	0.01\\
89.92	0.01\\
89.93	0.01\\
89.94	0.01\\
89.95	0.01\\
89.96	0.01\\
89.97	0.01\\
89.98	0.01\\
89.99	0.01\\
90	0.01\\
90.01	0.01\\
90.02	0.01\\
90.03	0.01\\
90.04	0.01\\
90.05	0.01\\
90.06	0.01\\
90.07	0.01\\
90.08	0.01\\
90.09	0.01\\
90.1	0.01\\
90.11	0.01\\
90.12	0.01\\
90.13	0.01\\
90.14	0.01\\
90.15	0.01\\
90.16	0.01\\
90.17	0.01\\
90.18	0.01\\
90.19	0.01\\
90.2	0.01\\
90.21	0.01\\
90.22	0.01\\
90.23	0.01\\
90.24	0.01\\
90.25	0.01\\
90.26	0.01\\
90.27	0.01\\
90.28	0.01\\
90.29	0.01\\
90.3	0.01\\
90.31	0.01\\
90.32	0.01\\
90.33	0.01\\
90.34	0.01\\
90.35	0.01\\
90.36	0.01\\
90.37	0.01\\
90.38	0.01\\
90.39	0.01\\
90.4	0.01\\
90.41	0.01\\
90.42	0.01\\
90.43	0.01\\
90.44	0.01\\
90.45	0.01\\
90.46	0.01\\
90.47	0.01\\
90.48	0.01\\
90.49	0.01\\
90.5	0.01\\
90.51	0.01\\
90.52	0.01\\
90.53	0.01\\
90.54	0.01\\
90.55	0.01\\
90.56	0.01\\
90.57	0.01\\
90.58	0.01\\
90.59	0.01\\
90.6	0.01\\
90.61	0.01\\
90.62	0.01\\
90.63	0.01\\
90.64	0.01\\
90.65	0.01\\
90.66	0.01\\
90.67	0.01\\
90.68	0.01\\
90.69	0.01\\
90.7	0.01\\
90.71	0.01\\
90.72	0.01\\
90.73	0.01\\
90.74	0.01\\
90.75	0.01\\
90.76	0.01\\
90.77	0.01\\
90.78	0.01\\
90.79	0.01\\
90.8	0.01\\
90.81	0.01\\
90.82	0.01\\
90.83	0.01\\
90.84	0.01\\
90.85	0.01\\
90.86	0.01\\
90.87	0.01\\
90.88	0.01\\
90.89	0.01\\
90.9	0.01\\
90.91	0.01\\
90.92	0.01\\
90.93	0.01\\
90.94	0.01\\
90.95	0.01\\
90.96	0.01\\
90.97	0.01\\
90.98	0.01\\
90.99	0.01\\
91	0.01\\
91.01	0.01\\
91.02	0.01\\
91.03	0.01\\
91.04	0.01\\
91.05	0.01\\
91.06	0.01\\
91.07	0.01\\
91.08	0.01\\
91.09	0.01\\
91.1	0.01\\
91.11	0.01\\
91.12	0.01\\
91.13	0.01\\
91.14	0.01\\
91.15	0.01\\
91.16	0.01\\
91.17	0.01\\
91.18	0.01\\
91.19	0.01\\
91.2	0.01\\
91.21	0.01\\
91.22	0.01\\
91.23	0.01\\
91.24	0.01\\
91.25	0.01\\
91.26	0.01\\
91.27	0.01\\
91.28	0.01\\
91.29	0.01\\
91.3	0.01\\
91.31	0.01\\
91.32	0.01\\
91.33	0.01\\
91.34	0.01\\
91.35	0.01\\
91.36	0.01\\
91.37	0.01\\
91.38	0.01\\
91.39	0.01\\
91.4	0.01\\
91.41	0.01\\
91.42	0.01\\
91.43	0.01\\
91.44	0.01\\
91.45	0.01\\
91.46	0.01\\
91.47	0.01\\
91.48	0.01\\
91.49	0.01\\
91.5	0.01\\
91.51	0.01\\
91.52	0.01\\
91.53	0.01\\
91.54	0.01\\
91.55	0.01\\
91.56	0.01\\
91.57	0.01\\
91.58	0.01\\
91.59	0.01\\
91.6	0.01\\
91.61	0.01\\
91.62	0.01\\
91.63	0.01\\
91.64	0.01\\
91.65	0.01\\
91.66	0.01\\
91.67	0.01\\
91.68	0.01\\
91.69	0.01\\
91.7	0.01\\
91.71	0.01\\
91.72	0.01\\
91.73	0.01\\
91.74	0.01\\
91.75	0.01\\
91.76	0.01\\
91.77	0.01\\
91.78	0.01\\
91.79	0.01\\
91.8	0.01\\
91.81	0.01\\
91.82	0.01\\
91.83	0.01\\
91.84	0.01\\
91.85	0.01\\
91.86	0.01\\
91.87	0.01\\
91.88	0.01\\
91.89	0.01\\
91.9	0.01\\
91.91	0.01\\
91.92	0.01\\
91.93	0.01\\
91.94	0.01\\
91.95	0.01\\
91.96	0.01\\
91.97	0.01\\
91.98	0.01\\
91.99	0.01\\
92	0.01\\
92.01	0.01\\
92.02	0.01\\
92.03	0.01\\
92.04	0.01\\
92.05	0.01\\
92.06	0.01\\
92.07	0.01\\
92.08	0.01\\
92.09	0.01\\
92.1	0.01\\
92.11	0.01\\
92.12	0.01\\
92.13	0.01\\
92.14	0.01\\
92.15	0.01\\
92.16	0.01\\
92.17	0.01\\
92.18	0.01\\
92.19	0.01\\
92.2	0.01\\
92.21	0.01\\
92.22	0.01\\
92.23	0.01\\
92.24	0.01\\
92.25	0.01\\
92.26	0.01\\
92.27	0.01\\
92.28	0.01\\
92.29	0.01\\
92.3	0.01\\
92.31	0.01\\
92.32	0.01\\
92.33	0.01\\
92.34	0.01\\
92.35	0.01\\
92.36	0.01\\
92.37	0.01\\
92.38	0.01\\
92.39	0.01\\
92.4	0.01\\
92.41	0.01\\
92.42	0.01\\
92.43	0.01\\
92.44	0.01\\
92.45	0.01\\
92.46	0.01\\
92.47	0.01\\
92.48	0.01\\
92.49	0.01\\
92.5	0.01\\
92.51	0.01\\
92.52	0.01\\
92.53	0.01\\
92.54	0.01\\
92.55	0.01\\
92.56	0.01\\
92.57	0.01\\
92.58	0.01\\
92.59	0.01\\
92.6	0.01\\
92.61	0.01\\
92.62	0.01\\
92.63	0.01\\
92.64	0.01\\
92.65	0.01\\
92.66	0.01\\
92.67	0.01\\
92.68	0.01\\
92.69	0.01\\
92.7	0.01\\
92.71	0.01\\
92.72	0.01\\
92.73	0.01\\
92.74	0.01\\
92.75	0.01\\
92.76	0.01\\
92.77	0.01\\
92.78	0.01\\
92.79	0.01\\
92.8	0.01\\
92.81	0.01\\
92.82	0.01\\
92.83	0.01\\
92.84	0.01\\
92.85	0.01\\
92.86	0.01\\
92.87	0.01\\
92.88	0.01\\
92.89	0.01\\
92.9	0.01\\
92.91	0.01\\
92.92	0.01\\
92.93	0.01\\
92.94	0.01\\
92.95	0.01\\
92.96	0.01\\
92.97	0.01\\
92.98	0.01\\
92.99	0.01\\
93	0.01\\
93.01	0.01\\
93.02	0.01\\
93.03	0.01\\
93.04	0.01\\
93.05	0.01\\
93.06	0.01\\
93.07	0.01\\
93.08	0.01\\
93.09	0.01\\
93.1	0.01\\
93.11	0.01\\
93.12	0.01\\
93.13	0.01\\
93.14	0.01\\
93.15	0.01\\
93.16	0.01\\
93.17	0.01\\
93.18	0.01\\
93.19	0.01\\
93.2	0.01\\
93.21	0.01\\
93.22	0.01\\
93.23	0.01\\
93.24	0.01\\
93.25	0.01\\
93.26	0.01\\
93.27	0.01\\
93.28	0.01\\
93.29	0.01\\
93.3	0.01\\
93.31	0.01\\
93.32	0.01\\
93.33	0.01\\
93.34	0.01\\
93.35	0.01\\
93.36	0.01\\
93.37	0.01\\
93.38	0.01\\
93.39	0.01\\
93.4	0.01\\
93.41	0.01\\
93.42	0.01\\
93.43	0.01\\
93.44	0.01\\
93.45	0.01\\
93.46	0.01\\
93.47	0.01\\
93.48	0.01\\
93.49	0.01\\
93.5	0.01\\
93.51	0.01\\
93.52	0.01\\
93.53	0.01\\
93.54	0.01\\
93.55	0.01\\
93.56	0.01\\
93.57	0.01\\
93.58	0.01\\
93.59	0.01\\
93.6	0.01\\
93.61	0.01\\
93.62	0.01\\
93.63	0.01\\
93.64	0.01\\
93.65	0.01\\
93.66	0.01\\
93.67	0.01\\
93.68	0.01\\
93.69	0.01\\
93.7	0.01\\
93.71	0.01\\
93.72	0.01\\
93.73	0.01\\
93.74	0.01\\
93.75	0.01\\
93.76	0.01\\
93.77	0.01\\
93.78	0.01\\
93.79	0.01\\
93.8	0.01\\
93.81	0.01\\
93.82	0.01\\
93.83	0.01\\
93.84	0.01\\
93.85	0.01\\
93.86	0.01\\
93.87	0.01\\
93.88	0.01\\
93.89	0.01\\
93.9	0.01\\
93.91	0.01\\
93.92	0.01\\
93.93	0.01\\
93.94	0.01\\
93.95	0.01\\
93.96	0.01\\
93.97	0.01\\
93.98	0.01\\
93.99	0.01\\
94	0.01\\
94.01	0.01\\
94.02	0.01\\
94.03	0.01\\
94.04	0.01\\
94.05	0.01\\
94.06	0.01\\
94.07	0.01\\
94.08	0.01\\
94.09	0.01\\
94.1	0.01\\
94.11	0.01\\
94.12	0.01\\
94.13	0.01\\
94.14	0.01\\
94.15	0.01\\
94.16	0.01\\
94.17	0.01\\
94.18	0.01\\
94.19	0.01\\
94.2	0.01\\
94.21	0.01\\
94.22	0.01\\
94.23	0.01\\
94.24	0.01\\
94.25	0.01\\
94.26	0.01\\
94.27	0.01\\
94.28	0.01\\
94.29	0.01\\
94.3	0.01\\
94.31	0.01\\
94.32	0.01\\
94.33	0.01\\
94.34	0.01\\
94.35	0.01\\
94.36	0.01\\
94.37	0.01\\
94.38	0.01\\
94.39	0.01\\
94.4	0.01\\
94.41	0.01\\
94.42	0.01\\
94.43	0.01\\
94.44	0.01\\
94.45	0.01\\
94.46	0.01\\
94.47	0.01\\
94.48	0.01\\
94.49	0.01\\
94.5	0.01\\
94.51	0.01\\
94.52	0.01\\
94.53	0.01\\
94.54	0.01\\
94.55	0.01\\
94.56	0.01\\
94.57	0.01\\
94.58	0.01\\
94.59	0.01\\
94.6	0.01\\
94.61	0.01\\
94.62	0.01\\
94.63	0.01\\
94.64	0.01\\
94.65	0.01\\
94.66	0.01\\
94.67	0.01\\
94.68	0.01\\
94.69	0.01\\
94.7	0.01\\
94.71	0.01\\
94.72	0.01\\
94.73	0.01\\
94.74	0.01\\
94.75	0.01\\
94.76	0.01\\
94.77	0.01\\
94.78	0.01\\
94.79	0.01\\
94.8	0.01\\
94.81	0.01\\
94.82	0.01\\
94.83	0.01\\
94.84	0.01\\
94.85	0.01\\
94.86	0.01\\
94.87	0.01\\
94.88	0.01\\
94.89	0.01\\
94.9	0.01\\
94.91	0.01\\
94.92	0.01\\
94.93	0.01\\
94.94	0.01\\
94.95	0.01\\
94.96	0.01\\
94.97	0.01\\
94.98	0.01\\
94.99	0.01\\
95	0.01\\
95.01	0.01\\
95.02	0.01\\
95.03	0.01\\
95.04	0.01\\
95.05	0.01\\
95.06	0.01\\
95.07	0.01\\
95.08	0.01\\
95.09	0.01\\
95.1	0.01\\
95.11	0.01\\
95.12	0.01\\
95.13	0.01\\
95.14	0.01\\
95.15	0.01\\
95.16	0.01\\
95.17	0.01\\
95.18	0.01\\
95.19	0.01\\
95.2	0.01\\
95.21	0.01\\
95.22	0.01\\
95.23	0.01\\
95.24	0.01\\
95.25	0.01\\
95.26	0.01\\
95.27	0.01\\
95.28	0.01\\
95.29	0.01\\
95.3	0.01\\
95.31	0.01\\
95.32	0.01\\
95.33	0.01\\
95.34	0.01\\
95.35	0.01\\
95.36	0.01\\
95.37	0.01\\
95.38	0.01\\
95.39	0.01\\
95.4	0.01\\
95.41	0.01\\
95.42	0.01\\
95.43	0.01\\
95.44	0.01\\
95.45	0.01\\
95.46	0.01\\
95.47	0.01\\
95.48	0.01\\
95.49	0.01\\
95.5	0.01\\
95.51	0.01\\
95.52	0.01\\
95.53	0.01\\
95.54	0.01\\
95.55	0.01\\
95.56	0.01\\
95.57	0.01\\
95.58	0.01\\
95.59	0.01\\
95.6	0.01\\
95.61	0.01\\
95.62	0.01\\
95.63	0.01\\
95.64	0.01\\
95.65	0.01\\
95.66	0.01\\
95.67	0.01\\
95.68	0.01\\
95.69	0.01\\
95.7	0.01\\
95.71	0.01\\
95.72	0.01\\
95.73	0.01\\
95.74	0.01\\
95.75	0.01\\
95.76	0.01\\
95.77	0.01\\
95.78	0.01\\
95.79	0.01\\
95.8	0.01\\
95.81	0.01\\
95.82	0.01\\
95.83	0.01\\
95.84	0.01\\
95.85	0.01\\
95.86	0.01\\
95.87	0.01\\
95.88	0.01\\
95.89	0.01\\
95.9	0.01\\
95.91	0.01\\
95.92	0.01\\
95.93	0.01\\
95.94	0.01\\
95.95	0.01\\
95.96	0.01\\
95.97	0.01\\
95.98	0.01\\
95.99	0.01\\
96	0.01\\
96.01	0.01\\
96.02	0.01\\
96.03	0.01\\
96.04	0.01\\
96.05	0.01\\
96.06	0.01\\
96.07	0.01\\
96.08	0.01\\
96.09	0.01\\
96.1	0.01\\
96.11	0.01\\
96.12	0.01\\
96.13	0.01\\
96.14	0.01\\
96.15	0.01\\
96.16	0.01\\
96.17	0.01\\
96.18	0.01\\
96.19	0.01\\
96.2	0.01\\
96.21	0.01\\
96.22	0.01\\
96.23	0.01\\
96.24	0.01\\
96.25	0.01\\
96.26	0.01\\
96.27	0.01\\
96.28	0.01\\
96.29	0.01\\
96.3	0.01\\
96.31	0.01\\
96.32	0.01\\
96.33	0.01\\
96.34	0.01\\
96.35	0.01\\
96.36	0.01\\
96.37	0.01\\
96.38	0.01\\
96.39	0.01\\
96.4	0.01\\
96.41	0.01\\
96.42	0.01\\
96.43	0.01\\
96.44	0.01\\
96.45	0.01\\
96.46	0.01\\
96.47	0.01\\
96.48	0.01\\
96.49	0.01\\
96.5	0.01\\
96.51	0.01\\
96.52	0.01\\
96.53	0.01\\
96.54	0.01\\
96.55	0.01\\
96.56	0.01\\
96.57	0.01\\
96.58	0.01\\
96.59	0.01\\
96.6	0.01\\
96.61	0.01\\
96.62	0.01\\
96.63	0.01\\
96.64	0.01\\
96.65	0.01\\
96.66	0.01\\
96.67	0.01\\
96.68	0.01\\
96.69	0.01\\
96.7	0.01\\
96.71	0.01\\
96.72	0.01\\
96.73	0.01\\
96.74	0.01\\
96.75	0.01\\
96.76	0.01\\
96.77	0.01\\
96.78	0.01\\
96.79	0.01\\
96.8	0.01\\
96.81	0.01\\
96.82	0.01\\
96.83	0.01\\
96.84	0.01\\
96.85	0.01\\
96.86	0.01\\
96.87	0.01\\
96.88	0.01\\
96.89	0.01\\
96.9	0.01\\
96.91	0.01\\
96.92	0.01\\
96.93	0.01\\
96.94	0.01\\
96.95	0.01\\
96.96	0.01\\
96.97	0.01\\
96.98	0.01\\
96.99	0.01\\
97	0.01\\
97.01	0.01\\
97.02	0.01\\
97.03	0.01\\
97.04	0.01\\
97.05	0.01\\
97.06	0.01\\
97.07	0.01\\
97.08	0.01\\
97.09	0.01\\
97.1	0.01\\
97.11	0.01\\
97.12	0.01\\
97.13	0.01\\
97.14	0.01\\
97.15	0.01\\
97.16	0.01\\
97.17	0.01\\
97.18	0.01\\
97.19	0.01\\
97.2	0.01\\
97.21	0.01\\
97.22	0.01\\
97.23	0.01\\
97.24	0.01\\
97.25	0.01\\
97.26	0.01\\
97.27	0.01\\
97.28	0.01\\
97.29	0.01\\
97.3	0.01\\
97.31	0.01\\
97.32	0.01\\
97.33	0.01\\
97.34	0.01\\
97.35	0.01\\
97.36	0.01\\
97.37	0.01\\
97.38	0.01\\
97.39	0.01\\
97.4	0.01\\
97.41	0.01\\
97.42	0.01\\
97.43	0.01\\
97.44	0.01\\
97.45	0.01\\
97.46	0.01\\
97.47	0.01\\
97.48	0.01\\
97.49	0.01\\
97.5	0.01\\
97.51	0.01\\
97.52	0.01\\
97.53	0.01\\
97.54	0.01\\
97.55	0.01\\
97.56	0.01\\
97.57	0.01\\
97.58	0.01\\
97.59	0.01\\
97.6	0.01\\
97.61	0.01\\
97.62	0.01\\
97.63	0.01\\
97.64	0.01\\
97.65	0.01\\
97.66	0.01\\
97.67	0.01\\
97.68	0.01\\
97.69	0.01\\
97.7	0.01\\
97.71	0.01\\
97.72	0.01\\
97.73	0.01\\
97.74	0.01\\
97.75	0.01\\
97.76	0.01\\
97.77	0.01\\
97.78	0.01\\
97.79	0.01\\
97.8	0.01\\
97.81	0.01\\
97.82	0.01\\
97.83	0.01\\
97.84	0.01\\
97.85	0.01\\
97.86	0.01\\
97.87	0.01\\
97.88	0.01\\
97.89	0.01\\
97.9	0.01\\
97.91	0.01\\
97.92	0.01\\
97.93	0.01\\
97.94	0.01\\
97.95	0.01\\
97.96	0.01\\
97.97	0.01\\
97.98	0.01\\
97.99	0.01\\
98	0.01\\
98.01	0.01\\
98.02	0.01\\
98.03	0.01\\
98.04	0.01\\
98.05	0.01\\
98.06	0.01\\
98.07	0.01\\
98.08	0.01\\
98.09	0.01\\
98.1	0.01\\
98.11	0.01\\
98.12	0.01\\
98.13	0.01\\
98.14	0.01\\
98.15	0.01\\
98.16	0.00995177871030805\\
98.17	0.0098799548118399\\
98.18	0.00980759140952771\\
98.19	0.00973468725080406\\
98.2	0.00966123716359973\\
98.21	0.00958723592161363\\
98.22	0.00951267824364272\\
98.23	0.00943755879290121\\
98.24	0.0093618721763288\\
98.25	0.00928561294388785\\
98.26	0.00920877558784894\\
98.27	0.00913135454206482\\
98.28	0.0090533441812325\\
98.29	0.00897473882014288\\
98.3	0.00889553271291812\\
98.31	0.00881572005223603\\
98.32	0.00873529496854143\\
98.33	0.00870936664103572\\
98.34	0.00868679735443978\\
98.35	0.00866403754486513\\
98.36	0.00864108556610907\\
98.37	0.00861793981098157\\
98.38	0.00859459866094568\\
98.39	0.00857106048609103\\
98.4	0.00854732364644967\\
98.41	0.00852338592357966\\
98.42	0.0084992410349635\\
98.43	0.00847488721120935\\
98.44	0.00845032266989913\\
98.45	0.0084255456155479\\
98.46	0.00840055423956494\\
98.47	0.00837534672021681\\
98.48	0.00834992122259224\\
98.49	0.00832427589856912\\
98.5	0.00829840888678362\\
98.51	0.00827231831260154\\
98.52	0.00824600228809194\\
98.53	0.00821945891200329\\
98.54	0.00819268626974207\\
98.55	0.00816568243335409\\
98.56	0.00813844546192765\\
98.57	0.00811097340435198\\
98.58	0.00808326429622979\\
98.59	0.00805531615987317\\
98.6	0.00802712700430279\\
98.61	0.00799869482525061\\
98.62	0.00797001760516625\\
98.63	0.00794109170234682\\
98.64	0.00791190235367152\\
98.65	0.00788244708039298\\
98.66	0.00785272338012802\\
98.67	0.00782272872662345\\
98.68	0.00779246056951935\\
98.69	0.00776191633410978\\
98.7	0.00773109342107603\\
98.71	0.00769998920623533\\
98.72	0.00766860104029221\\
98.73	0.007636926248587\\
98.74	0.00760496213939973\\
98.75	0.007572706004797\\
98.76	0.00754015511143454\\
98.77	0.00750730670031634\\
98.78	0.00747415798654993\\
98.79	0.00744070615933222\\
98.8	0.00740694838669973\\
98.81	0.0073728818102715\\
98.82	0.00733850354500313\\
98.83	0.00730381067893851\\
98.84	0.00726880027295931\\
98.85	0.00723346936053207\\
98.86	0.007197814947453\\
98.87	0.00716183401159042\\
98.88	0.00712552350262474\\
98.89	0.00708888034178608\\
98.9	0.0070519014215895\\
98.91	0.00701458360556767\\
98.92	0.00697692372799236\\
98.93	0.00693891859360066\\
98.94	0.00690056497731961\\
98.95	0.00686185962398836\\
98.96	0.00682279924807759\\
98.97	0.00678338053340649\\
98.98	0.00674360013285694\\
98.99	0.00670345466808514\\
99	0.00666294072923045\\
99.01	0.00662205487462158\\
99.02	0.00658079363048003\\
99.03	0.00653915349062074\\
99.04	0.00649713091614992\\
99.05	0.00645472233516013\\
99.06	0.00641192414242244\\
99.07	0.00636873269907578\\
99.08	0.00632514433231329\\
99.09	0.00628115533506586\\
99.1	0.00623676196568261\\
99.11	0.00619196044760847\\
99.12	0.00614674696905861\\
99.13	0.00610111768269005\\
99.14	0.00605506870526992\\
99.15	0.00600859611734088\\
99.16	0.00596169596288317\\
99.17	0.00591436424897374\\
99.18	0.00586659694544196\\
99.19	0.00581839001403525\\
99.2	0.00576973939059186\\
99.21	0.00572064097344853\\
99.22	0.00567109062309511\\
99.23	0.00562108416182606\\
99.24	0.00557061737338879\\
99.25	0.00551968600262867\\
99.26	0.00546828575513084\\
99.27	0.00541641229685875\\
99.28	0.00536406125378932\\
99.29	0.00531122821154471\\
99.3	0.00525790871502087\\
99.31	0.00520409826801247\\
99.32	0.0051497923328345\\
99.33	0.0050949863299404\\
99.34	0.00503967563753657\\
99.35	0.00498385559119345\\
99.36	0.00492752148345294\\
99.37	0.00487066856343227\\
99.38	0.00481329203642413\\
99.39	0.00475538706349262\\
99.4	0.00469694876103752\\
99.41	0.00463797220038192\\
99.42	0.00457845240735571\\
99.43	0.00451838436187542\\
99.44	0.00445776299751998\\
99.45	0.00439658320110255\\
99.46	0.00433483981223837\\
99.47	0.00427252762290852\\
99.48	0.0042096413770196\\
99.49	0.00414617576995928\\
99.5	0.00408212544814768\\
99.51	0.00401748500858448\\
99.52	0.00395224899839186\\
99.53	0.00388641191435301\\
99.54	0.00381996820244641\\
99.55	0.00375291225737559\\
99.56	0.00368523842209455\\
99.57	0.00361694098732857\\
99.58	0.00354801419109061\\
99.59	0.00347845221819298\\
99.6	0.00340824919975444\\
99.61	0.00333739921270264\\
99.62	0.00326589628211675\\
99.63	0.00319373438395597\\
99.64	0.003120907438625\\
99.65	0.00304740931045928\\
99.66	0.00297323380720539\\
99.67	0.00289837467949644\\
99.68	0.00282282562032242\\
99.69	0.00274658026449535\\
99.7	0.0026696321881094\\
99.71	0.00259197490799565\\
99.72	0.00251360188117163\\
99.73	0.00243450650428147\\
99.74	0.0023546821130308\\
99.75	0.00227412198161954\\
99.76	0.00219281932216908\\
99.77	0.00211076728414376\\
99.78	0.00202795895376664\\
99.79	0.00194438735342945\\
99.8	0.00186004544109659\\
99.81	0.00177492610970319\\
99.82	0.00168902218654705\\
99.83	0.00160232643267447\\
99.84	0.0015148315422598\\
99.85	0.00142653014197858\\
99.86	0.00133741479037433\\
99.87	0.00124747797721875\\
99.88	0.00115671212286522\\
99.89	0.00106510957759558\\
99.9	0.000972662620960026\\
99.91	0.000879363461110016\\
99.92	0.000785204234124003\\
99.93	0.000690177003325972\\
99.94	0.000594273758596576\\
99.95	0.000497486415676754\\
99.96	0.000399806815463661\\
99.97	0.000301226723298808\\
99.98	0.000201737828248207\\
99.99	0.000101331742374368\\
100	0\\
};
\addlegendentry{$q=-3$};

\addplot [color=red,dashed,forget plot]
  table[row sep=crcr]{%
0.01	0.01\\
0.02	0.01\\
0.03	0.01\\
0.04	0.01\\
0.05	0.01\\
0.06	0.01\\
0.07	0.01\\
0.08	0.01\\
0.09	0.01\\
0.1	0.01\\
0.11	0.01\\
0.12	0.01\\
0.13	0.01\\
0.14	0.01\\
0.15	0.01\\
0.16	0.01\\
0.17	0.01\\
0.18	0.01\\
0.19	0.01\\
0.2	0.01\\
0.21	0.01\\
0.22	0.01\\
0.23	0.01\\
0.24	0.01\\
0.25	0.01\\
0.26	0.01\\
0.27	0.01\\
0.28	0.01\\
0.29	0.01\\
0.3	0.01\\
0.31	0.01\\
0.32	0.01\\
0.33	0.01\\
0.34	0.01\\
0.35	0.01\\
0.36	0.01\\
0.37	0.01\\
0.38	0.01\\
0.39	0.01\\
0.4	0.01\\
0.41	0.01\\
0.42	0.01\\
0.43	0.01\\
0.44	0.01\\
0.45	0.01\\
0.46	0.01\\
0.47	0.01\\
0.48	0.01\\
0.49	0.01\\
0.5	0.01\\
0.51	0.01\\
0.52	0.01\\
0.53	0.01\\
0.54	0.01\\
0.55	0.01\\
0.56	0.01\\
0.57	0.01\\
0.58	0.01\\
0.59	0.01\\
0.6	0.01\\
0.61	0.01\\
0.62	0.01\\
0.63	0.01\\
0.64	0.01\\
0.65	0.01\\
0.66	0.01\\
0.67	0.01\\
0.68	0.01\\
0.69	0.01\\
0.7	0.01\\
0.71	0.01\\
0.72	0.01\\
0.73	0.01\\
0.74	0.01\\
0.75	0.01\\
0.76	0.01\\
0.77	0.01\\
0.78	0.01\\
0.79	0.01\\
0.8	0.01\\
0.81	0.01\\
0.82	0.01\\
0.83	0.01\\
0.84	0.01\\
0.85	0.01\\
0.86	0.01\\
0.87	0.01\\
0.88	0.01\\
0.89	0.01\\
0.9	0.01\\
0.91	0.01\\
0.92	0.01\\
0.93	0.01\\
0.94	0.01\\
0.95	0.01\\
0.96	0.01\\
0.97	0.01\\
0.98	0.01\\
0.99	0.01\\
1	0.01\\
1.01	0.01\\
1.02	0.01\\
1.03	0.01\\
1.04	0.01\\
1.05	0.01\\
1.06	0.01\\
1.07	0.01\\
1.08	0.01\\
1.09	0.01\\
1.1	0.01\\
1.11	0.01\\
1.12	0.01\\
1.13	0.01\\
1.14	0.01\\
1.15	0.01\\
1.16	0.01\\
1.17	0.01\\
1.18	0.01\\
1.19	0.01\\
1.2	0.01\\
1.21	0.01\\
1.22	0.01\\
1.23	0.01\\
1.24	0.01\\
1.25	0.01\\
1.26	0.01\\
1.27	0.01\\
1.28	0.01\\
1.29	0.01\\
1.3	0.01\\
1.31	0.01\\
1.32	0.01\\
1.33	0.01\\
1.34	0.01\\
1.35	0.01\\
1.36	0.01\\
1.37	0.01\\
1.38	0.01\\
1.39	0.01\\
1.4	0.01\\
1.41	0.01\\
1.42	0.01\\
1.43	0.01\\
1.44	0.01\\
1.45	0.01\\
1.46	0.01\\
1.47	0.01\\
1.48	0.01\\
1.49	0.01\\
1.5	0.01\\
1.51	0.01\\
1.52	0.01\\
1.53	0.01\\
1.54	0.01\\
1.55	0.01\\
1.56	0.01\\
1.57	0.01\\
1.58	0.01\\
1.59	0.01\\
1.6	0.01\\
1.61	0.01\\
1.62	0.01\\
1.63	0.01\\
1.64	0.01\\
1.65	0.01\\
1.66	0.01\\
1.67	0.01\\
1.68	0.01\\
1.69	0.01\\
1.7	0.01\\
1.71	0.01\\
1.72	0.01\\
1.73	0.01\\
1.74	0.01\\
1.75	0.01\\
1.76	0.01\\
1.77	0.01\\
1.78	0.01\\
1.79	0.01\\
1.8	0.01\\
1.81	0.01\\
1.82	0.01\\
1.83	0.01\\
1.84	0.01\\
1.85	0.01\\
1.86	0.01\\
1.87	0.01\\
1.88	0.01\\
1.89	0.01\\
1.9	0.01\\
1.91	0.01\\
1.92	0.01\\
1.93	0.01\\
1.94	0.01\\
1.95	0.01\\
1.96	0.01\\
1.97	0.01\\
1.98	0.01\\
1.99	0.01\\
2	0.01\\
2.01	0.01\\
2.02	0.01\\
2.03	0.01\\
2.04	0.01\\
2.05	0.01\\
2.06	0.01\\
2.07	0.01\\
2.08	0.01\\
2.09	0.01\\
2.1	0.01\\
2.11	0.01\\
2.12	0.01\\
2.13	0.01\\
2.14	0.01\\
2.15	0.01\\
2.16	0.01\\
2.17	0.01\\
2.18	0.01\\
2.19	0.01\\
2.2	0.01\\
2.21	0.01\\
2.22	0.01\\
2.23	0.01\\
2.24	0.01\\
2.25	0.01\\
2.26	0.01\\
2.27	0.01\\
2.28	0.01\\
2.29	0.01\\
2.3	0.01\\
2.31	0.01\\
2.32	0.01\\
2.33	0.01\\
2.34	0.01\\
2.35	0.01\\
2.36	0.01\\
2.37	0.01\\
2.38	0.01\\
2.39	0.01\\
2.4	0.01\\
2.41	0.01\\
2.42	0.01\\
2.43	0.01\\
2.44	0.01\\
2.45	0.01\\
2.46	0.01\\
2.47	0.01\\
2.48	0.01\\
2.49	0.01\\
2.5	0.01\\
2.51	0.01\\
2.52	0.01\\
2.53	0.01\\
2.54	0.01\\
2.55	0.01\\
2.56	0.01\\
2.57	0.01\\
2.58	0.01\\
2.59	0.01\\
2.6	0.01\\
2.61	0.01\\
2.62	0.01\\
2.63	0.01\\
2.64	0.01\\
2.65	0.01\\
2.66	0.01\\
2.67	0.01\\
2.68	0.01\\
2.69	0.01\\
2.7	0.01\\
2.71	0.01\\
2.72	0.01\\
2.73	0.01\\
2.74	0.01\\
2.75	0.01\\
2.76	0.01\\
2.77	0.01\\
2.78	0.01\\
2.79	0.01\\
2.8	0.01\\
2.81	0.01\\
2.82	0.01\\
2.83	0.01\\
2.84	0.01\\
2.85	0.01\\
2.86	0.01\\
2.87	0.01\\
2.88	0.01\\
2.89	0.01\\
2.9	0.01\\
2.91	0.01\\
2.92	0.01\\
2.93	0.01\\
2.94	0.01\\
2.95	0.01\\
2.96	0.01\\
2.97	0.01\\
2.98	0.01\\
2.99	0.01\\
3	0.01\\
3.01	0.01\\
3.02	0.01\\
3.03	0.01\\
3.04	0.01\\
3.05	0.01\\
3.06	0.01\\
3.07	0.01\\
3.08	0.01\\
3.09	0.01\\
3.1	0.01\\
3.11	0.01\\
3.12	0.01\\
3.13	0.01\\
3.14	0.01\\
3.15	0.01\\
3.16	0.01\\
3.17	0.01\\
3.18	0.01\\
3.19	0.01\\
3.2	0.01\\
3.21	0.01\\
3.22	0.01\\
3.23	0.01\\
3.24	0.01\\
3.25	0.01\\
3.26	0.01\\
3.27	0.01\\
3.28	0.01\\
3.29	0.01\\
3.3	0.01\\
3.31	0.01\\
3.32	0.01\\
3.33	0.01\\
3.34	0.01\\
3.35	0.01\\
3.36	0.01\\
3.37	0.01\\
3.38	0.01\\
3.39	0.01\\
3.4	0.01\\
3.41	0.01\\
3.42	0.01\\
3.43	0.01\\
3.44	0.01\\
3.45	0.01\\
3.46	0.01\\
3.47	0.01\\
3.48	0.01\\
3.49	0.01\\
3.5	0.01\\
3.51	0.01\\
3.52	0.01\\
3.53	0.01\\
3.54	0.01\\
3.55	0.01\\
3.56	0.01\\
3.57	0.01\\
3.58	0.01\\
3.59	0.01\\
3.6	0.01\\
3.61	0.01\\
3.62	0.01\\
3.63	0.01\\
3.64	0.01\\
3.65	0.01\\
3.66	0.01\\
3.67	0.01\\
3.68	0.01\\
3.69	0.01\\
3.7	0.01\\
3.71	0.01\\
3.72	0.01\\
3.73	0.01\\
3.74	0.01\\
3.75	0.01\\
3.76	0.01\\
3.77	0.01\\
3.78	0.01\\
3.79	0.01\\
3.8	0.01\\
3.81	0.01\\
3.82	0.01\\
3.83	0.01\\
3.84	0.01\\
3.85	0.01\\
3.86	0.01\\
3.87	0.01\\
3.88	0.01\\
3.89	0.01\\
3.9	0.01\\
3.91	0.01\\
3.92	0.01\\
3.93	0.01\\
3.94	0.01\\
3.95	0.01\\
3.96	0.01\\
3.97	0.01\\
3.98	0.01\\
3.99	0.01\\
4	0.01\\
4.01	0.01\\
4.02	0.01\\
4.03	0.01\\
4.04	0.01\\
4.05	0.01\\
4.06	0.01\\
4.07	0.01\\
4.08	0.01\\
4.09	0.01\\
4.1	0.01\\
4.11	0.01\\
4.12	0.01\\
4.13	0.01\\
4.14	0.01\\
4.15	0.01\\
4.16	0.01\\
4.17	0.01\\
4.18	0.01\\
4.19	0.01\\
4.2	0.01\\
4.21	0.01\\
4.22	0.01\\
4.23	0.01\\
4.24	0.01\\
4.25	0.01\\
4.26	0.01\\
4.27	0.01\\
4.28	0.01\\
4.29	0.01\\
4.3	0.01\\
4.31	0.01\\
4.32	0.01\\
4.33	0.01\\
4.34	0.01\\
4.35	0.01\\
4.36	0.01\\
4.37	0.01\\
4.38	0.01\\
4.39	0.01\\
4.4	0.01\\
4.41	0.01\\
4.42	0.01\\
4.43	0.01\\
4.44	0.01\\
4.45	0.01\\
4.46	0.01\\
4.47	0.01\\
4.48	0.01\\
4.49	0.01\\
4.5	0.01\\
4.51	0.01\\
4.52	0.01\\
4.53	0.01\\
4.54	0.01\\
4.55	0.01\\
4.56	0.01\\
4.57	0.01\\
4.58	0.01\\
4.59	0.01\\
4.6	0.01\\
4.61	0.01\\
4.62	0.01\\
4.63	0.01\\
4.64	0.01\\
4.65	0.01\\
4.66	0.01\\
4.67	0.01\\
4.68	0.01\\
4.69	0.01\\
4.7	0.01\\
4.71	0.01\\
4.72	0.01\\
4.73	0.01\\
4.74	0.01\\
4.75	0.01\\
4.76	0.01\\
4.77	0.01\\
4.78	0.01\\
4.79	0.01\\
4.8	0.01\\
4.81	0.01\\
4.82	0.01\\
4.83	0.01\\
4.84	0.01\\
4.85	0.01\\
4.86	0.01\\
4.87	0.01\\
4.88	0.01\\
4.89	0.01\\
4.9	0.01\\
4.91	0.01\\
4.92	0.01\\
4.93	0.01\\
4.94	0.01\\
4.95	0.01\\
4.96	0.01\\
4.97	0.01\\
4.98	0.01\\
4.99	0.01\\
5	0.01\\
5.01	0.01\\
5.02	0.01\\
5.03	0.01\\
5.04	0.01\\
5.05	0.01\\
5.06	0.01\\
5.07	0.01\\
5.08	0.01\\
5.09	0.01\\
5.1	0.01\\
5.11	0.01\\
5.12	0.01\\
5.13	0.01\\
5.14	0.01\\
5.15	0.01\\
5.16	0.01\\
5.17	0.01\\
5.18	0.01\\
5.19	0.01\\
5.2	0.01\\
5.21	0.01\\
5.22	0.01\\
5.23	0.01\\
5.24	0.01\\
5.25	0.01\\
5.26	0.01\\
5.27	0.01\\
5.28	0.01\\
5.29	0.01\\
5.3	0.01\\
5.31	0.01\\
5.32	0.01\\
5.33	0.01\\
5.34	0.01\\
5.35	0.01\\
5.36	0.01\\
5.37	0.01\\
5.38	0.01\\
5.39	0.01\\
5.4	0.01\\
5.41	0.01\\
5.42	0.01\\
5.43	0.01\\
5.44	0.01\\
5.45	0.01\\
5.46	0.01\\
5.47	0.01\\
5.48	0.01\\
5.49	0.01\\
5.5	0.01\\
5.51	0.01\\
5.52	0.01\\
5.53	0.01\\
5.54	0.01\\
5.55	0.01\\
5.56	0.01\\
5.57	0.01\\
5.58	0.01\\
5.59	0.01\\
5.6	0.01\\
5.61	0.01\\
5.62	0.01\\
5.63	0.01\\
5.64	0.01\\
5.65	0.01\\
5.66	0.01\\
5.67	0.01\\
5.68	0.01\\
5.69	0.01\\
5.7	0.01\\
5.71	0.01\\
5.72	0.01\\
5.73	0.01\\
5.74	0.01\\
5.75	0.01\\
5.76	0.01\\
5.77	0.01\\
5.78	0.01\\
5.79	0.01\\
5.8	0.01\\
5.81	0.01\\
5.82	0.01\\
5.83	0.01\\
5.84	0.01\\
5.85	0.01\\
5.86	0.01\\
5.87	0.01\\
5.88	0.01\\
5.89	0.01\\
5.9	0.01\\
5.91	0.01\\
5.92	0.01\\
5.93	0.01\\
5.94	0.01\\
5.95	0.01\\
5.96	0.01\\
5.97	0.01\\
5.98	0.01\\
5.99	0.01\\
6	0.01\\
6.01	0.01\\
6.02	0.01\\
6.03	0.01\\
6.04	0.01\\
6.05	0.01\\
6.06	0.01\\
6.07	0.01\\
6.08	0.01\\
6.09	0.01\\
6.1	0.01\\
6.11	0.01\\
6.12	0.01\\
6.13	0.01\\
6.14	0.01\\
6.15	0.01\\
6.16	0.01\\
6.17	0.01\\
6.18	0.01\\
6.19	0.01\\
6.2	0.01\\
6.21	0.01\\
6.22	0.01\\
6.23	0.01\\
6.24	0.01\\
6.25	0.01\\
6.26	0.01\\
6.27	0.01\\
6.28	0.01\\
6.29	0.01\\
6.3	0.01\\
6.31	0.01\\
6.32	0.01\\
6.33	0.01\\
6.34	0.01\\
6.35	0.01\\
6.36	0.01\\
6.37	0.01\\
6.38	0.01\\
6.39	0.01\\
6.4	0.01\\
6.41	0.01\\
6.42	0.01\\
6.43	0.01\\
6.44	0.01\\
6.45	0.01\\
6.46	0.01\\
6.47	0.01\\
6.48	0.01\\
6.49	0.01\\
6.5	0.01\\
6.51	0.01\\
6.52	0.01\\
6.53	0.01\\
6.54	0.01\\
6.55	0.01\\
6.56	0.01\\
6.57	0.01\\
6.58	0.01\\
6.59	0.01\\
6.6	0.01\\
6.61	0.01\\
6.62	0.01\\
6.63	0.01\\
6.64	0.01\\
6.65	0.01\\
6.66	0.01\\
6.67	0.01\\
6.68	0.01\\
6.69	0.01\\
6.7	0.01\\
6.71	0.01\\
6.72	0.01\\
6.73	0.01\\
6.74	0.01\\
6.75	0.01\\
6.76	0.01\\
6.77	0.01\\
6.78	0.01\\
6.79	0.01\\
6.8	0.01\\
6.81	0.01\\
6.82	0.01\\
6.83	0.01\\
6.84	0.01\\
6.85	0.01\\
6.86	0.01\\
6.87	0.01\\
6.88	0.01\\
6.89	0.01\\
6.9	0.01\\
6.91	0.01\\
6.92	0.01\\
6.93	0.01\\
6.94	0.01\\
6.95	0.01\\
6.96	0.01\\
6.97	0.01\\
6.98	0.01\\
6.99	0.01\\
7	0.01\\
7.01	0.01\\
7.02	0.01\\
7.03	0.01\\
7.04	0.01\\
7.05	0.01\\
7.06	0.01\\
7.07	0.01\\
7.08	0.01\\
7.09	0.01\\
7.1	0.01\\
7.11	0.01\\
7.12	0.01\\
7.13	0.01\\
7.14	0.01\\
7.15	0.01\\
7.16	0.01\\
7.17	0.01\\
7.18	0.01\\
7.19	0.01\\
7.2	0.01\\
7.21	0.01\\
7.22	0.01\\
7.23	0.01\\
7.24	0.01\\
7.25	0.01\\
7.26	0.01\\
7.27	0.01\\
7.28	0.01\\
7.29	0.01\\
7.3	0.01\\
7.31	0.01\\
7.32	0.01\\
7.33	0.01\\
7.34	0.01\\
7.35	0.01\\
7.36	0.01\\
7.37	0.01\\
7.38	0.01\\
7.39	0.01\\
7.4	0.01\\
7.41	0.01\\
7.42	0.01\\
7.43	0.01\\
7.44	0.01\\
7.45	0.01\\
7.46	0.01\\
7.47	0.01\\
7.48	0.01\\
7.49	0.01\\
7.5	0.01\\
7.51	0.01\\
7.52	0.01\\
7.53	0.01\\
7.54	0.01\\
7.55	0.01\\
7.56	0.01\\
7.57	0.01\\
7.58	0.01\\
7.59	0.01\\
7.6	0.01\\
7.61	0.01\\
7.62	0.01\\
7.63	0.01\\
7.64	0.01\\
7.65	0.01\\
7.66	0.01\\
7.67	0.01\\
7.68	0.01\\
7.69	0.01\\
7.7	0.01\\
7.71	0.01\\
7.72	0.01\\
7.73	0.01\\
7.74	0.01\\
7.75	0.01\\
7.76	0.01\\
7.77	0.01\\
7.78	0.01\\
7.79	0.01\\
7.8	0.01\\
7.81	0.01\\
7.82	0.01\\
7.83	0.01\\
7.84	0.01\\
7.85	0.01\\
7.86	0.01\\
7.87	0.01\\
7.88	0.01\\
7.89	0.01\\
7.9	0.01\\
7.91	0.01\\
7.92	0.01\\
7.93	0.01\\
7.94	0.01\\
7.95	0.01\\
7.96	0.01\\
7.97	0.01\\
7.98	0.01\\
7.99	0.01\\
8	0.01\\
8.01	0.01\\
8.02	0.01\\
8.03	0.01\\
8.04	0.01\\
8.05	0.01\\
8.06	0.01\\
8.07	0.01\\
8.08	0.01\\
8.09	0.01\\
8.1	0.01\\
8.11	0.01\\
8.12	0.01\\
8.13	0.01\\
8.14	0.01\\
8.15	0.01\\
8.16	0.01\\
8.17	0.01\\
8.18	0.01\\
8.19	0.01\\
8.2	0.01\\
8.21	0.01\\
8.22	0.01\\
8.23	0.01\\
8.24	0.01\\
8.25	0.01\\
8.26	0.01\\
8.27	0.01\\
8.28	0.01\\
8.29	0.01\\
8.3	0.01\\
8.31	0.01\\
8.32	0.01\\
8.33	0.01\\
8.34	0.01\\
8.35	0.01\\
8.36	0.01\\
8.37	0.01\\
8.38	0.01\\
8.39	0.01\\
8.4	0.01\\
8.41	0.01\\
8.42	0.01\\
8.43	0.01\\
8.44	0.01\\
8.45	0.01\\
8.46	0.01\\
8.47	0.01\\
8.48	0.01\\
8.49	0.01\\
8.5	0.01\\
8.51	0.01\\
8.52	0.01\\
8.53	0.01\\
8.54	0.01\\
8.55	0.01\\
8.56	0.01\\
8.57	0.01\\
8.58	0.01\\
8.59	0.01\\
8.6	0.01\\
8.61	0.01\\
8.62	0.01\\
8.63	0.01\\
8.64	0.01\\
8.65	0.01\\
8.66	0.01\\
8.67	0.01\\
8.68	0.01\\
8.69	0.01\\
8.7	0.01\\
8.71	0.01\\
8.72	0.01\\
8.73	0.01\\
8.74	0.01\\
8.75	0.01\\
8.76	0.01\\
8.77	0.01\\
8.78	0.01\\
8.79	0.01\\
8.8	0.01\\
8.81	0.01\\
8.82	0.01\\
8.83	0.01\\
8.84	0.01\\
8.85	0.01\\
8.86	0.01\\
8.87	0.01\\
8.88	0.01\\
8.89	0.01\\
8.9	0.01\\
8.91	0.01\\
8.92	0.01\\
8.93	0.01\\
8.94	0.01\\
8.95	0.01\\
8.96	0.01\\
8.97	0.01\\
8.98	0.01\\
8.99	0.01\\
9	0.01\\
9.01	0.01\\
9.02	0.01\\
9.03	0.01\\
9.04	0.01\\
9.05	0.01\\
9.06	0.01\\
9.07	0.01\\
9.08	0.01\\
9.09	0.01\\
9.1	0.01\\
9.11	0.01\\
9.12	0.01\\
9.13	0.01\\
9.14	0.01\\
9.15	0.01\\
9.16	0.01\\
9.17	0.01\\
9.18	0.01\\
9.19	0.01\\
9.2	0.01\\
9.21	0.01\\
9.22	0.01\\
9.23	0.01\\
9.24	0.01\\
9.25	0.01\\
9.26	0.01\\
9.27	0.01\\
9.28	0.01\\
9.29	0.01\\
9.3	0.01\\
9.31	0.01\\
9.32	0.01\\
9.33	0.01\\
9.34	0.01\\
9.35	0.01\\
9.36	0.01\\
9.37	0.01\\
9.38	0.01\\
9.39	0.01\\
9.4	0.01\\
9.41	0.01\\
9.42	0.01\\
9.43	0.01\\
9.44	0.01\\
9.45	0.01\\
9.46	0.01\\
9.47	0.01\\
9.48	0.01\\
9.49	0.01\\
9.5	0.01\\
9.51	0.01\\
9.52	0.01\\
9.53	0.01\\
9.54	0.01\\
9.55	0.01\\
9.56	0.01\\
9.57	0.01\\
9.58	0.01\\
9.59	0.01\\
9.6	0.01\\
9.61	0.01\\
9.62	0.01\\
9.63	0.01\\
9.64	0.01\\
9.65	0.01\\
9.66	0.01\\
9.67	0.01\\
9.68	0.01\\
9.69	0.01\\
9.7	0.01\\
9.71	0.01\\
9.72	0.01\\
9.73	0.01\\
9.74	0.01\\
9.75	0.01\\
9.76	0.01\\
9.77	0.01\\
9.78	0.01\\
9.79	0.01\\
9.8	0.01\\
9.81	0.01\\
9.82	0.01\\
9.83	0.01\\
9.84	0.01\\
9.85	0.01\\
9.86	0.01\\
9.87	0.01\\
9.88	0.01\\
9.89	0.01\\
9.9	0.01\\
9.91	0.01\\
9.92	0.01\\
9.93	0.01\\
9.94	0.01\\
9.95	0.01\\
9.96	0.01\\
9.97	0.01\\
9.98	0.01\\
9.99	0.01\\
10	0.01\\
10.01	0.01\\
10.02	0.01\\
10.03	0.01\\
10.04	0.01\\
10.05	0.01\\
10.06	0.01\\
10.07	0.01\\
10.08	0.01\\
10.09	0.01\\
10.1	0.01\\
10.11	0.01\\
10.12	0.01\\
10.13	0.01\\
10.14	0.01\\
10.15	0.01\\
10.16	0.01\\
10.17	0.01\\
10.18	0.01\\
10.19	0.01\\
10.2	0.01\\
10.21	0.01\\
10.22	0.01\\
10.23	0.01\\
10.24	0.01\\
10.25	0.01\\
10.26	0.01\\
10.27	0.01\\
10.28	0.01\\
10.29	0.01\\
10.3	0.01\\
10.31	0.01\\
10.32	0.01\\
10.33	0.01\\
10.34	0.01\\
10.35	0.01\\
10.36	0.01\\
10.37	0.01\\
10.38	0.01\\
10.39	0.01\\
10.4	0.01\\
10.41	0.01\\
10.42	0.01\\
10.43	0.01\\
10.44	0.01\\
10.45	0.01\\
10.46	0.01\\
10.47	0.01\\
10.48	0.01\\
10.49	0.01\\
10.5	0.01\\
10.51	0.01\\
10.52	0.01\\
10.53	0.01\\
10.54	0.01\\
10.55	0.01\\
10.56	0.01\\
10.57	0.01\\
10.58	0.01\\
10.59	0.01\\
10.6	0.01\\
10.61	0.01\\
10.62	0.01\\
10.63	0.01\\
10.64	0.01\\
10.65	0.01\\
10.66	0.01\\
10.67	0.01\\
10.68	0.01\\
10.69	0.01\\
10.7	0.01\\
10.71	0.01\\
10.72	0.01\\
10.73	0.01\\
10.74	0.01\\
10.75	0.01\\
10.76	0.01\\
10.77	0.01\\
10.78	0.01\\
10.79	0.01\\
10.8	0.01\\
10.81	0.01\\
10.82	0.01\\
10.83	0.01\\
10.84	0.01\\
10.85	0.01\\
10.86	0.01\\
10.87	0.01\\
10.88	0.01\\
10.89	0.01\\
10.9	0.01\\
10.91	0.01\\
10.92	0.01\\
10.93	0.01\\
10.94	0.01\\
10.95	0.01\\
10.96	0.01\\
10.97	0.01\\
10.98	0.01\\
10.99	0.01\\
11	0.01\\
11.01	0.01\\
11.02	0.01\\
11.03	0.01\\
11.04	0.01\\
11.05	0.01\\
11.06	0.01\\
11.07	0.01\\
11.08	0.01\\
11.09	0.01\\
11.1	0.01\\
11.11	0.01\\
11.12	0.01\\
11.13	0.01\\
11.14	0.01\\
11.15	0.01\\
11.16	0.01\\
11.17	0.01\\
11.18	0.01\\
11.19	0.01\\
11.2	0.01\\
11.21	0.01\\
11.22	0.01\\
11.23	0.01\\
11.24	0.01\\
11.25	0.01\\
11.26	0.01\\
11.27	0.01\\
11.28	0.01\\
11.29	0.01\\
11.3	0.01\\
11.31	0.01\\
11.32	0.01\\
11.33	0.01\\
11.34	0.01\\
11.35	0.01\\
11.36	0.01\\
11.37	0.01\\
11.38	0.01\\
11.39	0.01\\
11.4	0.01\\
11.41	0.01\\
11.42	0.01\\
11.43	0.01\\
11.44	0.01\\
11.45	0.01\\
11.46	0.01\\
11.47	0.01\\
11.48	0.01\\
11.49	0.01\\
11.5	0.01\\
11.51	0.01\\
11.52	0.01\\
11.53	0.01\\
11.54	0.01\\
11.55	0.01\\
11.56	0.01\\
11.57	0.01\\
11.58	0.01\\
11.59	0.01\\
11.6	0.01\\
11.61	0.01\\
11.62	0.01\\
11.63	0.01\\
11.64	0.01\\
11.65	0.01\\
11.66	0.01\\
11.67	0.01\\
11.68	0.01\\
11.69	0.01\\
11.7	0.01\\
11.71	0.01\\
11.72	0.01\\
11.73	0.01\\
11.74	0.01\\
11.75	0.01\\
11.76	0.01\\
11.77	0.01\\
11.78	0.01\\
11.79	0.01\\
11.8	0.01\\
11.81	0.01\\
11.82	0.01\\
11.83	0.01\\
11.84	0.01\\
11.85	0.01\\
11.86	0.01\\
11.87	0.01\\
11.88	0.01\\
11.89	0.01\\
11.9	0.01\\
11.91	0.01\\
11.92	0.01\\
11.93	0.01\\
11.94	0.01\\
11.95	0.01\\
11.96	0.01\\
11.97	0.01\\
11.98	0.01\\
11.99	0.01\\
12	0.01\\
12.01	0.01\\
12.02	0.01\\
12.03	0.01\\
12.04	0.01\\
12.05	0.01\\
12.06	0.01\\
12.07	0.01\\
12.08	0.01\\
12.09	0.01\\
12.1	0.01\\
12.11	0.01\\
12.12	0.01\\
12.13	0.01\\
12.14	0.01\\
12.15	0.01\\
12.16	0.01\\
12.17	0.01\\
12.18	0.01\\
12.19	0.01\\
12.2	0.01\\
12.21	0.01\\
12.22	0.01\\
12.23	0.01\\
12.24	0.01\\
12.25	0.01\\
12.26	0.01\\
12.27	0.01\\
12.28	0.01\\
12.29	0.01\\
12.3	0.01\\
12.31	0.01\\
12.32	0.01\\
12.33	0.01\\
12.34	0.01\\
12.35	0.01\\
12.36	0.01\\
12.37	0.01\\
12.38	0.01\\
12.39	0.01\\
12.4	0.01\\
12.41	0.01\\
12.42	0.01\\
12.43	0.01\\
12.44	0.01\\
12.45	0.01\\
12.46	0.01\\
12.47	0.01\\
12.48	0.01\\
12.49	0.01\\
12.5	0.01\\
12.51	0.01\\
12.52	0.01\\
12.53	0.01\\
12.54	0.01\\
12.55	0.01\\
12.56	0.01\\
12.57	0.01\\
12.58	0.01\\
12.59	0.01\\
12.6	0.01\\
12.61	0.01\\
12.62	0.01\\
12.63	0.01\\
12.64	0.01\\
12.65	0.01\\
12.66	0.01\\
12.67	0.01\\
12.68	0.01\\
12.69	0.01\\
12.7	0.01\\
12.71	0.01\\
12.72	0.01\\
12.73	0.01\\
12.74	0.01\\
12.75	0.01\\
12.76	0.01\\
12.77	0.01\\
12.78	0.01\\
12.79	0.01\\
12.8	0.01\\
12.81	0.01\\
12.82	0.01\\
12.83	0.01\\
12.84	0.01\\
12.85	0.01\\
12.86	0.01\\
12.87	0.01\\
12.88	0.01\\
12.89	0.01\\
12.9	0.01\\
12.91	0.01\\
12.92	0.01\\
12.93	0.01\\
12.94	0.01\\
12.95	0.01\\
12.96	0.01\\
12.97	0.01\\
12.98	0.01\\
12.99	0.01\\
13	0.01\\
13.01	0.01\\
13.02	0.01\\
13.03	0.01\\
13.04	0.01\\
13.05	0.01\\
13.06	0.01\\
13.07	0.01\\
13.08	0.01\\
13.09	0.01\\
13.1	0.01\\
13.11	0.01\\
13.12	0.01\\
13.13	0.01\\
13.14	0.01\\
13.15	0.01\\
13.16	0.01\\
13.17	0.01\\
13.18	0.01\\
13.19	0.01\\
13.2	0.01\\
13.21	0.01\\
13.22	0.01\\
13.23	0.01\\
13.24	0.01\\
13.25	0.01\\
13.26	0.01\\
13.27	0.01\\
13.28	0.01\\
13.29	0.01\\
13.3	0.01\\
13.31	0.01\\
13.32	0.01\\
13.33	0.01\\
13.34	0.01\\
13.35	0.01\\
13.36	0.01\\
13.37	0.01\\
13.38	0.01\\
13.39	0.01\\
13.4	0.01\\
13.41	0.01\\
13.42	0.01\\
13.43	0.01\\
13.44	0.01\\
13.45	0.01\\
13.46	0.01\\
13.47	0.01\\
13.48	0.01\\
13.49	0.01\\
13.5	0.01\\
13.51	0.01\\
13.52	0.01\\
13.53	0.01\\
13.54	0.01\\
13.55	0.01\\
13.56	0.01\\
13.57	0.01\\
13.58	0.01\\
13.59	0.01\\
13.6	0.01\\
13.61	0.01\\
13.62	0.01\\
13.63	0.01\\
13.64	0.01\\
13.65	0.01\\
13.66	0.01\\
13.67	0.01\\
13.68	0.01\\
13.69	0.01\\
13.7	0.01\\
13.71	0.01\\
13.72	0.01\\
13.73	0.01\\
13.74	0.01\\
13.75	0.01\\
13.76	0.01\\
13.77	0.01\\
13.78	0.01\\
13.79	0.01\\
13.8	0.01\\
13.81	0.01\\
13.82	0.01\\
13.83	0.01\\
13.84	0.01\\
13.85	0.01\\
13.86	0.01\\
13.87	0.01\\
13.88	0.01\\
13.89	0.01\\
13.9	0.01\\
13.91	0.01\\
13.92	0.01\\
13.93	0.01\\
13.94	0.01\\
13.95	0.01\\
13.96	0.01\\
13.97	0.01\\
13.98	0.01\\
13.99	0.01\\
14	0.01\\
14.01	0.01\\
14.02	0.01\\
14.03	0.01\\
14.04	0.01\\
14.05	0.01\\
14.06	0.01\\
14.07	0.01\\
14.08	0.01\\
14.09	0.01\\
14.1	0.01\\
14.11	0.01\\
14.12	0.01\\
14.13	0.01\\
14.14	0.01\\
14.15	0.01\\
14.16	0.01\\
14.17	0.01\\
14.18	0.01\\
14.19	0.01\\
14.2	0.01\\
14.21	0.01\\
14.22	0.01\\
14.23	0.01\\
14.24	0.01\\
14.25	0.01\\
14.26	0.01\\
14.27	0.01\\
14.28	0.01\\
14.29	0.01\\
14.3	0.01\\
14.31	0.01\\
14.32	0.01\\
14.33	0.01\\
14.34	0.01\\
14.35	0.01\\
14.36	0.01\\
14.37	0.01\\
14.38	0.01\\
14.39	0.01\\
14.4	0.01\\
14.41	0.01\\
14.42	0.01\\
14.43	0.01\\
14.44	0.01\\
14.45	0.01\\
14.46	0.01\\
14.47	0.01\\
14.48	0.01\\
14.49	0.01\\
14.5	0.01\\
14.51	0.01\\
14.52	0.01\\
14.53	0.01\\
14.54	0.01\\
14.55	0.01\\
14.56	0.01\\
14.57	0.01\\
14.58	0.01\\
14.59	0.01\\
14.6	0.01\\
14.61	0.01\\
14.62	0.01\\
14.63	0.01\\
14.64	0.01\\
14.65	0.01\\
14.66	0.01\\
14.67	0.01\\
14.68	0.01\\
14.69	0.01\\
14.7	0.01\\
14.71	0.01\\
14.72	0.01\\
14.73	0.01\\
14.74	0.01\\
14.75	0.01\\
14.76	0.01\\
14.77	0.01\\
14.78	0.01\\
14.79	0.01\\
14.8	0.01\\
14.81	0.01\\
14.82	0.01\\
14.83	0.01\\
14.84	0.01\\
14.85	0.01\\
14.86	0.01\\
14.87	0.01\\
14.88	0.01\\
14.89	0.01\\
14.9	0.01\\
14.91	0.01\\
14.92	0.01\\
14.93	0.01\\
14.94	0.01\\
14.95	0.01\\
14.96	0.01\\
14.97	0.01\\
14.98	0.01\\
14.99	0.01\\
15	0.01\\
15.01	0.01\\
15.02	0.01\\
15.03	0.01\\
15.04	0.01\\
15.05	0.01\\
15.06	0.01\\
15.07	0.01\\
15.08	0.01\\
15.09	0.01\\
15.1	0.01\\
15.11	0.01\\
15.12	0.01\\
15.13	0.01\\
15.14	0.01\\
15.15	0.01\\
15.16	0.01\\
15.17	0.01\\
15.18	0.01\\
15.19	0.01\\
15.2	0.01\\
15.21	0.01\\
15.22	0.01\\
15.23	0.01\\
15.24	0.01\\
15.25	0.01\\
15.26	0.01\\
15.27	0.01\\
15.28	0.01\\
15.29	0.01\\
15.3	0.01\\
15.31	0.01\\
15.32	0.01\\
15.33	0.01\\
15.34	0.01\\
15.35	0.01\\
15.36	0.01\\
15.37	0.01\\
15.38	0.01\\
15.39	0.01\\
15.4	0.01\\
15.41	0.01\\
15.42	0.01\\
15.43	0.01\\
15.44	0.01\\
15.45	0.01\\
15.46	0.01\\
15.47	0.01\\
15.48	0.01\\
15.49	0.01\\
15.5	0.01\\
15.51	0.01\\
15.52	0.01\\
15.53	0.01\\
15.54	0.01\\
15.55	0.01\\
15.56	0.01\\
15.57	0.01\\
15.58	0.01\\
15.59	0.01\\
15.6	0.01\\
15.61	0.01\\
15.62	0.01\\
15.63	0.01\\
15.64	0.01\\
15.65	0.01\\
15.66	0.01\\
15.67	0.01\\
15.68	0.01\\
15.69	0.01\\
15.7	0.01\\
15.71	0.01\\
15.72	0.01\\
15.73	0.01\\
15.74	0.01\\
15.75	0.01\\
15.76	0.01\\
15.77	0.01\\
15.78	0.01\\
15.79	0.01\\
15.8	0.01\\
15.81	0.01\\
15.82	0.01\\
15.83	0.01\\
15.84	0.01\\
15.85	0.01\\
15.86	0.01\\
15.87	0.01\\
15.88	0.01\\
15.89	0.01\\
15.9	0.01\\
15.91	0.01\\
15.92	0.01\\
15.93	0.01\\
15.94	0.01\\
15.95	0.01\\
15.96	0.01\\
15.97	0.01\\
15.98	0.01\\
15.99	0.01\\
16	0.01\\
16.01	0.01\\
16.02	0.01\\
16.03	0.01\\
16.04	0.01\\
16.05	0.01\\
16.06	0.01\\
16.07	0.01\\
16.08	0.01\\
16.09	0.01\\
16.1	0.01\\
16.11	0.01\\
16.12	0.01\\
16.13	0.01\\
16.14	0.01\\
16.15	0.01\\
16.16	0.01\\
16.17	0.01\\
16.18	0.01\\
16.19	0.01\\
16.2	0.01\\
16.21	0.01\\
16.22	0.01\\
16.23	0.01\\
16.24	0.01\\
16.25	0.01\\
16.26	0.01\\
16.27	0.01\\
16.28	0.01\\
16.29	0.01\\
16.3	0.01\\
16.31	0.01\\
16.32	0.01\\
16.33	0.01\\
16.34	0.01\\
16.35	0.01\\
16.36	0.01\\
16.37	0.01\\
16.38	0.01\\
16.39	0.01\\
16.4	0.01\\
16.41	0.01\\
16.42	0.01\\
16.43	0.01\\
16.44	0.01\\
16.45	0.01\\
16.46	0.01\\
16.47	0.01\\
16.48	0.01\\
16.49	0.01\\
16.5	0.01\\
16.51	0.01\\
16.52	0.01\\
16.53	0.01\\
16.54	0.01\\
16.55	0.01\\
16.56	0.01\\
16.57	0.01\\
16.58	0.01\\
16.59	0.01\\
16.6	0.01\\
16.61	0.01\\
16.62	0.01\\
16.63	0.01\\
16.64	0.01\\
16.65	0.01\\
16.66	0.01\\
16.67	0.01\\
16.68	0.01\\
16.69	0.01\\
16.7	0.01\\
16.71	0.01\\
16.72	0.01\\
16.73	0.01\\
16.74	0.01\\
16.75	0.01\\
16.76	0.01\\
16.77	0.01\\
16.78	0.01\\
16.79	0.01\\
16.8	0.01\\
16.81	0.01\\
16.82	0.01\\
16.83	0.01\\
16.84	0.01\\
16.85	0.01\\
16.86	0.01\\
16.87	0.01\\
16.88	0.01\\
16.89	0.01\\
16.9	0.01\\
16.91	0.01\\
16.92	0.01\\
16.93	0.01\\
16.94	0.01\\
16.95	0.01\\
16.96	0.01\\
16.97	0.01\\
16.98	0.01\\
16.99	0.01\\
17	0.01\\
17.01	0.01\\
17.02	0.01\\
17.03	0.01\\
17.04	0.01\\
17.05	0.01\\
17.06	0.01\\
17.07	0.01\\
17.08	0.01\\
17.09	0.01\\
17.1	0.01\\
17.11	0.01\\
17.12	0.01\\
17.13	0.01\\
17.14	0.01\\
17.15	0.01\\
17.16	0.01\\
17.17	0.01\\
17.18	0.01\\
17.19	0.01\\
17.2	0.01\\
17.21	0.01\\
17.22	0.01\\
17.23	0.01\\
17.24	0.01\\
17.25	0.01\\
17.26	0.01\\
17.27	0.01\\
17.28	0.01\\
17.29	0.01\\
17.3	0.01\\
17.31	0.01\\
17.32	0.01\\
17.33	0.01\\
17.34	0.01\\
17.35	0.01\\
17.36	0.01\\
17.37	0.01\\
17.38	0.01\\
17.39	0.01\\
17.4	0.01\\
17.41	0.01\\
17.42	0.01\\
17.43	0.01\\
17.44	0.01\\
17.45	0.01\\
17.46	0.01\\
17.47	0.01\\
17.48	0.01\\
17.49	0.01\\
17.5	0.01\\
17.51	0.01\\
17.52	0.01\\
17.53	0.01\\
17.54	0.01\\
17.55	0.01\\
17.56	0.01\\
17.57	0.01\\
17.58	0.01\\
17.59	0.01\\
17.6	0.01\\
17.61	0.01\\
17.62	0.01\\
17.63	0.01\\
17.64	0.01\\
17.65	0.01\\
17.66	0.01\\
17.67	0.01\\
17.68	0.01\\
17.69	0.01\\
17.7	0.01\\
17.71	0.01\\
17.72	0.01\\
17.73	0.01\\
17.74	0.01\\
17.75	0.01\\
17.76	0.01\\
17.77	0.01\\
17.78	0.01\\
17.79	0.01\\
17.8	0.01\\
17.81	0.01\\
17.82	0.01\\
17.83	0.01\\
17.84	0.01\\
17.85	0.01\\
17.86	0.01\\
17.87	0.01\\
17.88	0.01\\
17.89	0.01\\
17.9	0.01\\
17.91	0.01\\
17.92	0.01\\
17.93	0.01\\
17.94	0.01\\
17.95	0.01\\
17.96	0.01\\
17.97	0.01\\
17.98	0.01\\
17.99	0.01\\
18	0.01\\
18.01	0.01\\
18.02	0.01\\
18.03	0.01\\
18.04	0.01\\
18.05	0.01\\
18.06	0.01\\
18.07	0.01\\
18.08	0.01\\
18.09	0.01\\
18.1	0.01\\
18.11	0.01\\
18.12	0.01\\
18.13	0.01\\
18.14	0.01\\
18.15	0.01\\
18.16	0.01\\
18.17	0.01\\
18.18	0.01\\
18.19	0.01\\
18.2	0.01\\
18.21	0.01\\
18.22	0.01\\
18.23	0.01\\
18.24	0.01\\
18.25	0.01\\
18.26	0.01\\
18.27	0.01\\
18.28	0.01\\
18.29	0.01\\
18.3	0.01\\
18.31	0.01\\
18.32	0.01\\
18.33	0.01\\
18.34	0.01\\
18.35	0.01\\
18.36	0.01\\
18.37	0.01\\
18.38	0.01\\
18.39	0.01\\
18.4	0.01\\
18.41	0.01\\
18.42	0.01\\
18.43	0.01\\
18.44	0.01\\
18.45	0.01\\
18.46	0.01\\
18.47	0.01\\
18.48	0.01\\
18.49	0.01\\
18.5	0.01\\
18.51	0.01\\
18.52	0.01\\
18.53	0.01\\
18.54	0.01\\
18.55	0.01\\
18.56	0.01\\
18.57	0.01\\
18.58	0.01\\
18.59	0.01\\
18.6	0.01\\
18.61	0.01\\
18.62	0.01\\
18.63	0.01\\
18.64	0.01\\
18.65	0.01\\
18.66	0.01\\
18.67	0.01\\
18.68	0.01\\
18.69	0.01\\
18.7	0.01\\
18.71	0.01\\
18.72	0.01\\
18.73	0.01\\
18.74	0.01\\
18.75	0.01\\
18.76	0.01\\
18.77	0.01\\
18.78	0.01\\
18.79	0.01\\
18.8	0.01\\
18.81	0.01\\
18.82	0.01\\
18.83	0.01\\
18.84	0.01\\
18.85	0.01\\
18.86	0.01\\
18.87	0.01\\
18.88	0.01\\
18.89	0.01\\
18.9	0.01\\
18.91	0.01\\
18.92	0.01\\
18.93	0.01\\
18.94	0.01\\
18.95	0.01\\
18.96	0.01\\
18.97	0.01\\
18.98	0.01\\
18.99	0.01\\
19	0.01\\
19.01	0.01\\
19.02	0.01\\
19.03	0.01\\
19.04	0.01\\
19.05	0.01\\
19.06	0.01\\
19.07	0.01\\
19.08	0.01\\
19.09	0.01\\
19.1	0.01\\
19.11	0.01\\
19.12	0.01\\
19.13	0.01\\
19.14	0.01\\
19.15	0.01\\
19.16	0.01\\
19.17	0.01\\
19.18	0.01\\
19.19	0.01\\
19.2	0.01\\
19.21	0.01\\
19.22	0.01\\
19.23	0.01\\
19.24	0.01\\
19.25	0.01\\
19.26	0.01\\
19.27	0.01\\
19.28	0.01\\
19.29	0.01\\
19.3	0.01\\
19.31	0.01\\
19.32	0.01\\
19.33	0.01\\
19.34	0.01\\
19.35	0.01\\
19.36	0.01\\
19.37	0.01\\
19.38	0.01\\
19.39	0.01\\
19.4	0.01\\
19.41	0.01\\
19.42	0.01\\
19.43	0.01\\
19.44	0.01\\
19.45	0.01\\
19.46	0.01\\
19.47	0.01\\
19.48	0.01\\
19.49	0.01\\
19.5	0.01\\
19.51	0.01\\
19.52	0.01\\
19.53	0.01\\
19.54	0.01\\
19.55	0.01\\
19.56	0.01\\
19.57	0.01\\
19.58	0.01\\
19.59	0.01\\
19.6	0.01\\
19.61	0.01\\
19.62	0.01\\
19.63	0.01\\
19.64	0.01\\
19.65	0.01\\
19.66	0.01\\
19.67	0.01\\
19.68	0.01\\
19.69	0.01\\
19.7	0.01\\
19.71	0.01\\
19.72	0.01\\
19.73	0.01\\
19.74	0.01\\
19.75	0.01\\
19.76	0.01\\
19.77	0.01\\
19.78	0.01\\
19.79	0.01\\
19.8	0.01\\
19.81	0.01\\
19.82	0.01\\
19.83	0.01\\
19.84	0.01\\
19.85	0.01\\
19.86	0.01\\
19.87	0.01\\
19.88	0.01\\
19.89	0.01\\
19.9	0.01\\
19.91	0.01\\
19.92	0.01\\
19.93	0.01\\
19.94	0.01\\
19.95	0.01\\
19.96	0.01\\
19.97	0.01\\
19.98	0.01\\
19.99	0.01\\
20	0.01\\
20.01	0.01\\
20.02	0.01\\
20.03	0.01\\
20.04	0.01\\
20.05	0.01\\
20.06	0.01\\
20.07	0.01\\
20.08	0.01\\
20.09	0.01\\
20.1	0.01\\
20.11	0.01\\
20.12	0.01\\
20.13	0.01\\
20.14	0.01\\
20.15	0.01\\
20.16	0.01\\
20.17	0.01\\
20.18	0.01\\
20.19	0.01\\
20.2	0.01\\
20.21	0.01\\
20.22	0.01\\
20.23	0.01\\
20.24	0.01\\
20.25	0.01\\
20.26	0.01\\
20.27	0.01\\
20.28	0.01\\
20.29	0.01\\
20.3	0.01\\
20.31	0.01\\
20.32	0.01\\
20.33	0.01\\
20.34	0.01\\
20.35	0.01\\
20.36	0.01\\
20.37	0.01\\
20.38	0.01\\
20.39	0.01\\
20.4	0.01\\
20.41	0.01\\
20.42	0.01\\
20.43	0.01\\
20.44	0.01\\
20.45	0.01\\
20.46	0.01\\
20.47	0.01\\
20.48	0.01\\
20.49	0.01\\
20.5	0.01\\
20.51	0.01\\
20.52	0.01\\
20.53	0.01\\
20.54	0.01\\
20.55	0.01\\
20.56	0.01\\
20.57	0.01\\
20.58	0.01\\
20.59	0.01\\
20.6	0.01\\
20.61	0.01\\
20.62	0.01\\
20.63	0.01\\
20.64	0.01\\
20.65	0.01\\
20.66	0.01\\
20.67	0.01\\
20.68	0.01\\
20.69	0.01\\
20.7	0.01\\
20.71	0.01\\
20.72	0.01\\
20.73	0.01\\
20.74	0.01\\
20.75	0.01\\
20.76	0.01\\
20.77	0.01\\
20.78	0.01\\
20.79	0.01\\
20.8	0.01\\
20.81	0.01\\
20.82	0.01\\
20.83	0.01\\
20.84	0.01\\
20.85	0.01\\
20.86	0.01\\
20.87	0.01\\
20.88	0.01\\
20.89	0.01\\
20.9	0.01\\
20.91	0.01\\
20.92	0.01\\
20.93	0.01\\
20.94	0.01\\
20.95	0.01\\
20.96	0.01\\
20.97	0.01\\
20.98	0.01\\
20.99	0.01\\
21	0.01\\
21.01	0.01\\
21.02	0.01\\
21.03	0.01\\
21.04	0.01\\
21.05	0.01\\
21.06	0.01\\
21.07	0.01\\
21.08	0.01\\
21.09	0.01\\
21.1	0.01\\
21.11	0.01\\
21.12	0.01\\
21.13	0.01\\
21.14	0.01\\
21.15	0.01\\
21.16	0.01\\
21.17	0.01\\
21.18	0.01\\
21.19	0.01\\
21.2	0.01\\
21.21	0.01\\
21.22	0.01\\
21.23	0.01\\
21.24	0.01\\
21.25	0.01\\
21.26	0.01\\
21.27	0.01\\
21.28	0.01\\
21.29	0.01\\
21.3	0.01\\
21.31	0.01\\
21.32	0.01\\
21.33	0.01\\
21.34	0.01\\
21.35	0.01\\
21.36	0.01\\
21.37	0.01\\
21.38	0.01\\
21.39	0.01\\
21.4	0.01\\
21.41	0.01\\
21.42	0.01\\
21.43	0.01\\
21.44	0.01\\
21.45	0.01\\
21.46	0.01\\
21.47	0.01\\
21.48	0.01\\
21.49	0.01\\
21.5	0.01\\
21.51	0.01\\
21.52	0.01\\
21.53	0.01\\
21.54	0.01\\
21.55	0.01\\
21.56	0.01\\
21.57	0.01\\
21.58	0.01\\
21.59	0.01\\
21.6	0.01\\
21.61	0.01\\
21.62	0.01\\
21.63	0.01\\
21.64	0.01\\
21.65	0.01\\
21.66	0.01\\
21.67	0.01\\
21.68	0.01\\
21.69	0.01\\
21.7	0.01\\
21.71	0.01\\
21.72	0.01\\
21.73	0.01\\
21.74	0.01\\
21.75	0.01\\
21.76	0.01\\
21.77	0.01\\
21.78	0.01\\
21.79	0.01\\
21.8	0.01\\
21.81	0.01\\
21.82	0.01\\
21.83	0.01\\
21.84	0.01\\
21.85	0.01\\
21.86	0.01\\
21.87	0.01\\
21.88	0.01\\
21.89	0.01\\
21.9	0.01\\
21.91	0.01\\
21.92	0.01\\
21.93	0.01\\
21.94	0.01\\
21.95	0.01\\
21.96	0.01\\
21.97	0.01\\
21.98	0.01\\
21.99	0.01\\
22	0.01\\
22.01	0.01\\
22.02	0.01\\
22.03	0.01\\
22.04	0.01\\
22.05	0.01\\
22.06	0.01\\
22.07	0.01\\
22.08	0.01\\
22.09	0.01\\
22.1	0.01\\
22.11	0.01\\
22.12	0.01\\
22.13	0.01\\
22.14	0.01\\
22.15	0.01\\
22.16	0.01\\
22.17	0.01\\
22.18	0.01\\
22.19	0.01\\
22.2	0.01\\
22.21	0.01\\
22.22	0.01\\
22.23	0.01\\
22.24	0.01\\
22.25	0.01\\
22.26	0.01\\
22.27	0.01\\
22.28	0.01\\
22.29	0.01\\
22.3	0.01\\
22.31	0.01\\
22.32	0.01\\
22.33	0.01\\
22.34	0.01\\
22.35	0.01\\
22.36	0.01\\
22.37	0.01\\
22.38	0.01\\
22.39	0.01\\
22.4	0.01\\
22.41	0.01\\
22.42	0.01\\
22.43	0.01\\
22.44	0.01\\
22.45	0.01\\
22.46	0.01\\
22.47	0.01\\
22.48	0.01\\
22.49	0.01\\
22.5	0.01\\
22.51	0.01\\
22.52	0.01\\
22.53	0.01\\
22.54	0.01\\
22.55	0.01\\
22.56	0.01\\
22.57	0.01\\
22.58	0.01\\
22.59	0.01\\
22.6	0.01\\
22.61	0.01\\
22.62	0.01\\
22.63	0.01\\
22.64	0.01\\
22.65	0.01\\
22.66	0.01\\
22.67	0.01\\
22.68	0.01\\
22.69	0.01\\
22.7	0.01\\
22.71	0.01\\
22.72	0.01\\
22.73	0.01\\
22.74	0.01\\
22.75	0.01\\
22.76	0.01\\
22.77	0.01\\
22.78	0.01\\
22.79	0.01\\
22.8	0.01\\
22.81	0.01\\
22.82	0.01\\
22.83	0.01\\
22.84	0.01\\
22.85	0.01\\
22.86	0.01\\
22.87	0.01\\
22.88	0.01\\
22.89	0.01\\
22.9	0.01\\
22.91	0.01\\
22.92	0.01\\
22.93	0.01\\
22.94	0.01\\
22.95	0.01\\
22.96	0.01\\
22.97	0.01\\
22.98	0.01\\
22.99	0.01\\
23	0.01\\
23.01	0.01\\
23.02	0.01\\
23.03	0.01\\
23.04	0.01\\
23.05	0.01\\
23.06	0.01\\
23.07	0.01\\
23.08	0.01\\
23.09	0.01\\
23.1	0.01\\
23.11	0.01\\
23.12	0.01\\
23.13	0.01\\
23.14	0.01\\
23.15	0.01\\
23.16	0.01\\
23.17	0.01\\
23.18	0.01\\
23.19	0.01\\
23.2	0.01\\
23.21	0.01\\
23.22	0.01\\
23.23	0.01\\
23.24	0.01\\
23.25	0.01\\
23.26	0.01\\
23.27	0.01\\
23.28	0.01\\
23.29	0.01\\
23.3	0.01\\
23.31	0.01\\
23.32	0.01\\
23.33	0.01\\
23.34	0.01\\
23.35	0.01\\
23.36	0.01\\
23.37	0.01\\
23.38	0.01\\
23.39	0.01\\
23.4	0.01\\
23.41	0.01\\
23.42	0.01\\
23.43	0.01\\
23.44	0.01\\
23.45	0.01\\
23.46	0.01\\
23.47	0.01\\
23.48	0.01\\
23.49	0.01\\
23.5	0.01\\
23.51	0.01\\
23.52	0.01\\
23.53	0.01\\
23.54	0.01\\
23.55	0.01\\
23.56	0.01\\
23.57	0.01\\
23.58	0.01\\
23.59	0.01\\
23.6	0.01\\
23.61	0.01\\
23.62	0.01\\
23.63	0.01\\
23.64	0.01\\
23.65	0.01\\
23.66	0.01\\
23.67	0.01\\
23.68	0.01\\
23.69	0.01\\
23.7	0.01\\
23.71	0.01\\
23.72	0.01\\
23.73	0.01\\
23.74	0.01\\
23.75	0.01\\
23.76	0.01\\
23.77	0.01\\
23.78	0.01\\
23.79	0.01\\
23.8	0.01\\
23.81	0.01\\
23.82	0.01\\
23.83	0.01\\
23.84	0.01\\
23.85	0.01\\
23.86	0.01\\
23.87	0.01\\
23.88	0.01\\
23.89	0.01\\
23.9	0.01\\
23.91	0.01\\
23.92	0.01\\
23.93	0.01\\
23.94	0.01\\
23.95	0.01\\
23.96	0.01\\
23.97	0.01\\
23.98	0.01\\
23.99	0.01\\
24	0.01\\
24.01	0.01\\
24.02	0.01\\
24.03	0.01\\
24.04	0.01\\
24.05	0.01\\
24.06	0.01\\
24.07	0.01\\
24.08	0.01\\
24.09	0.01\\
24.1	0.01\\
24.11	0.01\\
24.12	0.01\\
24.13	0.01\\
24.14	0.01\\
24.15	0.01\\
24.16	0.01\\
24.17	0.01\\
24.18	0.01\\
24.19	0.01\\
24.2	0.01\\
24.21	0.01\\
24.22	0.01\\
24.23	0.01\\
24.24	0.01\\
24.25	0.01\\
24.26	0.01\\
24.27	0.01\\
24.28	0.01\\
24.29	0.01\\
24.3	0.01\\
24.31	0.01\\
24.32	0.01\\
24.33	0.01\\
24.34	0.01\\
24.35	0.01\\
24.36	0.01\\
24.37	0.01\\
24.38	0.01\\
24.39	0.01\\
24.4	0.01\\
24.41	0.01\\
24.42	0.01\\
24.43	0.01\\
24.44	0.01\\
24.45	0.01\\
24.46	0.01\\
24.47	0.01\\
24.48	0.01\\
24.49	0.01\\
24.5	0.01\\
24.51	0.01\\
24.52	0.01\\
24.53	0.01\\
24.54	0.01\\
24.55	0.01\\
24.56	0.01\\
24.57	0.01\\
24.58	0.01\\
24.59	0.01\\
24.6	0.01\\
24.61	0.01\\
24.62	0.01\\
24.63	0.01\\
24.64	0.01\\
24.65	0.01\\
24.66	0.01\\
24.67	0.01\\
24.68	0.01\\
24.69	0.01\\
24.7	0.01\\
24.71	0.01\\
24.72	0.01\\
24.73	0.01\\
24.74	0.01\\
24.75	0.01\\
24.76	0.01\\
24.77	0.01\\
24.78	0.01\\
24.79	0.01\\
24.8	0.01\\
24.81	0.01\\
24.82	0.01\\
24.83	0.01\\
24.84	0.01\\
24.85	0.01\\
24.86	0.01\\
24.87	0.01\\
24.88	0.01\\
24.89	0.01\\
24.9	0.01\\
24.91	0.01\\
24.92	0.01\\
24.93	0.01\\
24.94	0.01\\
24.95	0.01\\
24.96	0.01\\
24.97	0.01\\
24.98	0.01\\
24.99	0.01\\
25	0.01\\
25.01	0.01\\
25.02	0.01\\
25.03	0.01\\
25.04	0.01\\
25.05	0.01\\
25.06	0.01\\
25.07	0.01\\
25.08	0.01\\
25.09	0.01\\
25.1	0.01\\
25.11	0.01\\
25.12	0.01\\
25.13	0.01\\
25.14	0.01\\
25.15	0.01\\
25.16	0.01\\
25.17	0.01\\
25.18	0.01\\
25.19	0.01\\
25.2	0.01\\
25.21	0.01\\
25.22	0.01\\
25.23	0.01\\
25.24	0.01\\
25.25	0.01\\
25.26	0.01\\
25.27	0.01\\
25.28	0.01\\
25.29	0.01\\
25.3	0.01\\
25.31	0.01\\
25.32	0.01\\
25.33	0.01\\
25.34	0.01\\
25.35	0.01\\
25.36	0.01\\
25.37	0.01\\
25.38	0.01\\
25.39	0.01\\
25.4	0.01\\
25.41	0.01\\
25.42	0.01\\
25.43	0.01\\
25.44	0.01\\
25.45	0.01\\
25.46	0.01\\
25.47	0.01\\
25.48	0.01\\
25.49	0.01\\
25.5	0.01\\
25.51	0.01\\
25.52	0.01\\
25.53	0.01\\
25.54	0.01\\
25.55	0.01\\
25.56	0.01\\
25.57	0.01\\
25.58	0.01\\
25.59	0.01\\
25.6	0.01\\
25.61	0.01\\
25.62	0.01\\
25.63	0.01\\
25.64	0.01\\
25.65	0.01\\
25.66	0.01\\
25.67	0.01\\
25.68	0.01\\
25.69	0.01\\
25.7	0.01\\
25.71	0.01\\
25.72	0.01\\
25.73	0.01\\
25.74	0.01\\
25.75	0.01\\
25.76	0.01\\
25.77	0.01\\
25.78	0.01\\
25.79	0.01\\
25.8	0.01\\
25.81	0.01\\
25.82	0.01\\
25.83	0.01\\
25.84	0.01\\
25.85	0.01\\
25.86	0.01\\
25.87	0.01\\
25.88	0.01\\
25.89	0.01\\
25.9	0.01\\
25.91	0.01\\
25.92	0.01\\
25.93	0.01\\
25.94	0.01\\
25.95	0.01\\
25.96	0.01\\
25.97	0.01\\
25.98	0.01\\
25.99	0.01\\
26	0.01\\
26.01	0.01\\
26.02	0.01\\
26.03	0.01\\
26.04	0.01\\
26.05	0.01\\
26.06	0.01\\
26.07	0.01\\
26.08	0.01\\
26.09	0.01\\
26.1	0.01\\
26.11	0.01\\
26.12	0.01\\
26.13	0.01\\
26.14	0.01\\
26.15	0.01\\
26.16	0.01\\
26.17	0.01\\
26.18	0.01\\
26.19	0.01\\
26.2	0.01\\
26.21	0.01\\
26.22	0.01\\
26.23	0.01\\
26.24	0.01\\
26.25	0.01\\
26.26	0.01\\
26.27	0.01\\
26.28	0.01\\
26.29	0.01\\
26.3	0.01\\
26.31	0.01\\
26.32	0.01\\
26.33	0.01\\
26.34	0.01\\
26.35	0.01\\
26.36	0.01\\
26.37	0.01\\
26.38	0.01\\
26.39	0.01\\
26.4	0.01\\
26.41	0.01\\
26.42	0.01\\
26.43	0.01\\
26.44	0.01\\
26.45	0.01\\
26.46	0.01\\
26.47	0.01\\
26.48	0.01\\
26.49	0.01\\
26.5	0.01\\
26.51	0.01\\
26.52	0.01\\
26.53	0.01\\
26.54	0.01\\
26.55	0.01\\
26.56	0.01\\
26.57	0.01\\
26.58	0.01\\
26.59	0.01\\
26.6	0.01\\
26.61	0.01\\
26.62	0.01\\
26.63	0.01\\
26.64	0.01\\
26.65	0.01\\
26.66	0.01\\
26.67	0.01\\
26.68	0.01\\
26.69	0.01\\
26.7	0.01\\
26.71	0.01\\
26.72	0.01\\
26.73	0.01\\
26.74	0.01\\
26.75	0.01\\
26.76	0.01\\
26.77	0.01\\
26.78	0.01\\
26.79	0.01\\
26.8	0.01\\
26.81	0.01\\
26.82	0.01\\
26.83	0.01\\
26.84	0.01\\
26.85	0.01\\
26.86	0.01\\
26.87	0.01\\
26.88	0.01\\
26.89	0.01\\
26.9	0.01\\
26.91	0.01\\
26.92	0.01\\
26.93	0.01\\
26.94	0.01\\
26.95	0.01\\
26.96	0.01\\
26.97	0.01\\
26.98	0.01\\
26.99	0.01\\
27	0.01\\
27.01	0.01\\
27.02	0.01\\
27.03	0.01\\
27.04	0.01\\
27.05	0.01\\
27.06	0.01\\
27.07	0.01\\
27.08	0.01\\
27.09	0.01\\
27.1	0.01\\
27.11	0.01\\
27.12	0.01\\
27.13	0.01\\
27.14	0.01\\
27.15	0.01\\
27.16	0.01\\
27.17	0.01\\
27.18	0.01\\
27.19	0.01\\
27.2	0.01\\
27.21	0.01\\
27.22	0.01\\
27.23	0.01\\
27.24	0.01\\
27.25	0.01\\
27.26	0.01\\
27.27	0.01\\
27.28	0.01\\
27.29	0.01\\
27.3	0.01\\
27.31	0.01\\
27.32	0.01\\
27.33	0.01\\
27.34	0.01\\
27.35	0.01\\
27.36	0.01\\
27.37	0.01\\
27.38	0.01\\
27.39	0.01\\
27.4	0.01\\
27.41	0.01\\
27.42	0.01\\
27.43	0.01\\
27.44	0.01\\
27.45	0.01\\
27.46	0.01\\
27.47	0.01\\
27.48	0.01\\
27.49	0.01\\
27.5	0.01\\
27.51	0.01\\
27.52	0.01\\
27.53	0.01\\
27.54	0.01\\
27.55	0.01\\
27.56	0.01\\
27.57	0.01\\
27.58	0.01\\
27.59	0.01\\
27.6	0.01\\
27.61	0.01\\
27.62	0.01\\
27.63	0.01\\
27.64	0.01\\
27.65	0.01\\
27.66	0.01\\
27.67	0.01\\
27.68	0.01\\
27.69	0.01\\
27.7	0.01\\
27.71	0.01\\
27.72	0.01\\
27.73	0.01\\
27.74	0.01\\
27.75	0.01\\
27.76	0.01\\
27.77	0.01\\
27.78	0.01\\
27.79	0.01\\
27.8	0.01\\
27.81	0.01\\
27.82	0.01\\
27.83	0.01\\
27.84	0.01\\
27.85	0.01\\
27.86	0.01\\
27.87	0.01\\
27.88	0.01\\
27.89	0.01\\
27.9	0.01\\
27.91	0.01\\
27.92	0.01\\
27.93	0.01\\
27.94	0.01\\
27.95	0.01\\
27.96	0.01\\
27.97	0.01\\
27.98	0.01\\
27.99	0.01\\
28	0.01\\
28.01	0.01\\
28.02	0.01\\
28.03	0.01\\
28.04	0.01\\
28.05	0.01\\
28.06	0.01\\
28.07	0.01\\
28.08	0.01\\
28.09	0.01\\
28.1	0.01\\
28.11	0.01\\
28.12	0.01\\
28.13	0.01\\
28.14	0.01\\
28.15	0.01\\
28.16	0.01\\
28.17	0.01\\
28.18	0.01\\
28.19	0.01\\
28.2	0.01\\
28.21	0.01\\
28.22	0.01\\
28.23	0.01\\
28.24	0.01\\
28.25	0.01\\
28.26	0.01\\
28.27	0.01\\
28.28	0.01\\
28.29	0.01\\
28.3	0.01\\
28.31	0.01\\
28.32	0.01\\
28.33	0.01\\
28.34	0.01\\
28.35	0.01\\
28.36	0.01\\
28.37	0.01\\
28.38	0.01\\
28.39	0.01\\
28.4	0.01\\
28.41	0.01\\
28.42	0.01\\
28.43	0.01\\
28.44	0.01\\
28.45	0.01\\
28.46	0.01\\
28.47	0.01\\
28.48	0.01\\
28.49	0.01\\
28.5	0.01\\
28.51	0.01\\
28.52	0.01\\
28.53	0.01\\
28.54	0.01\\
28.55	0.01\\
28.56	0.01\\
28.57	0.01\\
28.58	0.01\\
28.59	0.01\\
28.6	0.01\\
28.61	0.01\\
28.62	0.01\\
28.63	0.01\\
28.64	0.01\\
28.65	0.01\\
28.66	0.01\\
28.67	0.01\\
28.68	0.01\\
28.69	0.01\\
28.7	0.01\\
28.71	0.01\\
28.72	0.01\\
28.73	0.01\\
28.74	0.01\\
28.75	0.01\\
28.76	0.01\\
28.77	0.01\\
28.78	0.01\\
28.79	0.01\\
28.8	0.01\\
28.81	0.01\\
28.82	0.01\\
28.83	0.01\\
28.84	0.01\\
28.85	0.01\\
28.86	0.01\\
28.87	0.01\\
28.88	0.01\\
28.89	0.01\\
28.9	0.01\\
28.91	0.01\\
28.92	0.01\\
28.93	0.01\\
28.94	0.01\\
28.95	0.01\\
28.96	0.01\\
28.97	0.01\\
28.98	0.01\\
28.99	0.01\\
29	0.01\\
29.01	0.01\\
29.02	0.01\\
29.03	0.01\\
29.04	0.01\\
29.05	0.01\\
29.06	0.01\\
29.07	0.01\\
29.08	0.01\\
29.09	0.01\\
29.1	0.01\\
29.11	0.01\\
29.12	0.01\\
29.13	0.01\\
29.14	0.01\\
29.15	0.01\\
29.16	0.01\\
29.17	0.01\\
29.18	0.01\\
29.19	0.01\\
29.2	0.01\\
29.21	0.01\\
29.22	0.01\\
29.23	0.01\\
29.24	0.01\\
29.25	0.01\\
29.26	0.01\\
29.27	0.01\\
29.28	0.01\\
29.29	0.01\\
29.3	0.01\\
29.31	0.01\\
29.32	0.01\\
29.33	0.01\\
29.34	0.01\\
29.35	0.01\\
29.36	0.01\\
29.37	0.01\\
29.38	0.01\\
29.39	0.01\\
29.4	0.01\\
29.41	0.01\\
29.42	0.01\\
29.43	0.01\\
29.44	0.01\\
29.45	0.01\\
29.46	0.01\\
29.47	0.01\\
29.48	0.01\\
29.49	0.01\\
29.5	0.01\\
29.51	0.01\\
29.52	0.01\\
29.53	0.01\\
29.54	0.01\\
29.55	0.01\\
29.56	0.01\\
29.57	0.01\\
29.58	0.01\\
29.59	0.01\\
29.6	0.01\\
29.61	0.01\\
29.62	0.01\\
29.63	0.01\\
29.64	0.01\\
29.65	0.01\\
29.66	0.01\\
29.67	0.01\\
29.68	0.01\\
29.69	0.01\\
29.7	0.01\\
29.71	0.01\\
29.72	0.01\\
29.73	0.01\\
29.74	0.01\\
29.75	0.01\\
29.76	0.01\\
29.77	0.01\\
29.78	0.01\\
29.79	0.01\\
29.8	0.01\\
29.81	0.01\\
29.82	0.01\\
29.83	0.01\\
29.84	0.01\\
29.85	0.01\\
29.86	0.01\\
29.87	0.01\\
29.88	0.01\\
29.89	0.01\\
29.9	0.01\\
29.91	0.01\\
29.92	0.01\\
29.93	0.01\\
29.94	0.01\\
29.95	0.01\\
29.96	0.01\\
29.97	0.01\\
29.98	0.01\\
29.99	0.01\\
30	0.01\\
30.01	0.01\\
30.02	0.01\\
30.03	0.01\\
30.04	0.01\\
30.05	0.01\\
30.06	0.01\\
30.07	0.01\\
30.08	0.01\\
30.09	0.01\\
30.1	0.01\\
30.11	0.01\\
30.12	0.01\\
30.13	0.01\\
30.14	0.01\\
30.15	0.01\\
30.16	0.01\\
30.17	0.01\\
30.18	0.01\\
30.19	0.01\\
30.2	0.01\\
30.21	0.01\\
30.22	0.01\\
30.23	0.01\\
30.24	0.01\\
30.25	0.01\\
30.26	0.01\\
30.27	0.01\\
30.28	0.01\\
30.29	0.01\\
30.3	0.01\\
30.31	0.01\\
30.32	0.01\\
30.33	0.01\\
30.34	0.01\\
30.35	0.01\\
30.36	0.01\\
30.37	0.01\\
30.38	0.01\\
30.39	0.01\\
30.4	0.01\\
30.41	0.01\\
30.42	0.01\\
30.43	0.01\\
30.44	0.01\\
30.45	0.01\\
30.46	0.01\\
30.47	0.01\\
30.48	0.01\\
30.49	0.01\\
30.5	0.01\\
30.51	0.01\\
30.52	0.01\\
30.53	0.01\\
30.54	0.01\\
30.55	0.01\\
30.56	0.01\\
30.57	0.01\\
30.58	0.01\\
30.59	0.01\\
30.6	0.01\\
30.61	0.01\\
30.62	0.01\\
30.63	0.01\\
30.64	0.01\\
30.65	0.01\\
30.66	0.01\\
30.67	0.01\\
30.68	0.01\\
30.69	0.01\\
30.7	0.01\\
30.71	0.01\\
30.72	0.01\\
30.73	0.01\\
30.74	0.01\\
30.75	0.01\\
30.76	0.01\\
30.77	0.01\\
30.78	0.01\\
30.79	0.01\\
30.8	0.01\\
30.81	0.01\\
30.82	0.01\\
30.83	0.01\\
30.84	0.01\\
30.85	0.01\\
30.86	0.01\\
30.87	0.01\\
30.88	0.01\\
30.89	0.01\\
30.9	0.01\\
30.91	0.01\\
30.92	0.01\\
30.93	0.01\\
30.94	0.01\\
30.95	0.01\\
30.96	0.01\\
30.97	0.01\\
30.98	0.01\\
30.99	0.01\\
31	0.01\\
31.01	0.01\\
31.02	0.01\\
31.03	0.01\\
31.04	0.01\\
31.05	0.01\\
31.06	0.01\\
31.07	0.01\\
31.08	0.01\\
31.09	0.01\\
31.1	0.01\\
31.11	0.01\\
31.12	0.01\\
31.13	0.01\\
31.14	0.01\\
31.15	0.01\\
31.16	0.01\\
31.17	0.01\\
31.18	0.01\\
31.19	0.01\\
31.2	0.01\\
31.21	0.01\\
31.22	0.01\\
31.23	0.01\\
31.24	0.01\\
31.25	0.01\\
31.26	0.01\\
31.27	0.01\\
31.28	0.01\\
31.29	0.01\\
31.3	0.01\\
31.31	0.01\\
31.32	0.01\\
31.33	0.01\\
31.34	0.01\\
31.35	0.01\\
31.36	0.01\\
31.37	0.01\\
31.38	0.01\\
31.39	0.01\\
31.4	0.01\\
31.41	0.01\\
31.42	0.01\\
31.43	0.01\\
31.44	0.01\\
31.45	0.01\\
31.46	0.01\\
31.47	0.01\\
31.48	0.01\\
31.49	0.01\\
31.5	0.01\\
31.51	0.01\\
31.52	0.01\\
31.53	0.01\\
31.54	0.01\\
31.55	0.01\\
31.56	0.01\\
31.57	0.01\\
31.58	0.01\\
31.59	0.01\\
31.6	0.01\\
31.61	0.01\\
31.62	0.01\\
31.63	0.01\\
31.64	0.01\\
31.65	0.01\\
31.66	0.01\\
31.67	0.01\\
31.68	0.01\\
31.69	0.01\\
31.7	0.01\\
31.71	0.01\\
31.72	0.01\\
31.73	0.01\\
31.74	0.01\\
31.75	0.01\\
31.76	0.01\\
31.77	0.01\\
31.78	0.01\\
31.79	0.01\\
31.8	0.01\\
31.81	0.01\\
31.82	0.01\\
31.83	0.01\\
31.84	0.01\\
31.85	0.01\\
31.86	0.01\\
31.87	0.01\\
31.88	0.01\\
31.89	0.01\\
31.9	0.01\\
31.91	0.01\\
31.92	0.01\\
31.93	0.01\\
31.94	0.01\\
31.95	0.01\\
31.96	0.01\\
31.97	0.01\\
31.98	0.01\\
31.99	0.01\\
32	0.01\\
32.01	0.01\\
32.02	0.01\\
32.03	0.01\\
32.04	0.01\\
32.05	0.01\\
32.06	0.01\\
32.07	0.01\\
32.08	0.01\\
32.09	0.01\\
32.1	0.01\\
32.11	0.01\\
32.12	0.01\\
32.13	0.01\\
32.14	0.01\\
32.15	0.01\\
32.16	0.01\\
32.17	0.01\\
32.18	0.01\\
32.19	0.01\\
32.2	0.01\\
32.21	0.01\\
32.22	0.01\\
32.23	0.01\\
32.24	0.01\\
32.25	0.01\\
32.26	0.01\\
32.27	0.01\\
32.28	0.01\\
32.29	0.01\\
32.3	0.01\\
32.31	0.01\\
32.32	0.01\\
32.33	0.01\\
32.34	0.01\\
32.35	0.01\\
32.36	0.01\\
32.37	0.01\\
32.38	0.01\\
32.39	0.01\\
32.4	0.01\\
32.41	0.01\\
32.42	0.01\\
32.43	0.01\\
32.44	0.01\\
32.45	0.01\\
32.46	0.01\\
32.47	0.01\\
32.48	0.01\\
32.49	0.01\\
32.5	0.01\\
32.51	0.01\\
32.52	0.01\\
32.53	0.01\\
32.54	0.01\\
32.55	0.01\\
32.56	0.01\\
32.57	0.01\\
32.58	0.01\\
32.59	0.01\\
32.6	0.01\\
32.61	0.01\\
32.62	0.01\\
32.63	0.01\\
32.64	0.01\\
32.65	0.01\\
32.66	0.01\\
32.67	0.01\\
32.68	0.01\\
32.69	0.01\\
32.7	0.01\\
32.71	0.01\\
32.72	0.01\\
32.73	0.01\\
32.74	0.01\\
32.75	0.01\\
32.76	0.01\\
32.77	0.01\\
32.78	0.01\\
32.79	0.01\\
32.8	0.01\\
32.81	0.01\\
32.82	0.01\\
32.83	0.01\\
32.84	0.01\\
32.85	0.01\\
32.86	0.01\\
32.87	0.01\\
32.88	0.01\\
32.89	0.01\\
32.9	0.01\\
32.91	0.01\\
32.92	0.01\\
32.93	0.01\\
32.94	0.01\\
32.95	0.01\\
32.96	0.01\\
32.97	0.01\\
32.98	0.01\\
32.99	0.01\\
33	0.01\\
33.01	0.01\\
33.02	0.01\\
33.03	0.01\\
33.04	0.01\\
33.05	0.01\\
33.06	0.01\\
33.07	0.01\\
33.08	0.01\\
33.09	0.01\\
33.1	0.01\\
33.11	0.01\\
33.12	0.01\\
33.13	0.01\\
33.14	0.01\\
33.15	0.01\\
33.16	0.01\\
33.17	0.01\\
33.18	0.01\\
33.19	0.01\\
33.2	0.01\\
33.21	0.01\\
33.22	0.01\\
33.23	0.01\\
33.24	0.01\\
33.25	0.01\\
33.26	0.01\\
33.27	0.01\\
33.28	0.01\\
33.29	0.01\\
33.3	0.01\\
33.31	0.01\\
33.32	0.01\\
33.33	0.01\\
33.34	0.01\\
33.35	0.01\\
33.36	0.01\\
33.37	0.01\\
33.38	0.01\\
33.39	0.01\\
33.4	0.01\\
33.41	0.01\\
33.42	0.01\\
33.43	0.01\\
33.44	0.01\\
33.45	0.01\\
33.46	0.01\\
33.47	0.01\\
33.48	0.01\\
33.49	0.01\\
33.5	0.01\\
33.51	0.01\\
33.52	0.01\\
33.53	0.01\\
33.54	0.01\\
33.55	0.01\\
33.56	0.01\\
33.57	0.01\\
33.58	0.01\\
33.59	0.01\\
33.6	0.01\\
33.61	0.01\\
33.62	0.01\\
33.63	0.01\\
33.64	0.01\\
33.65	0.01\\
33.66	0.01\\
33.67	0.01\\
33.68	0.01\\
33.69	0.01\\
33.7	0.01\\
33.71	0.01\\
33.72	0.01\\
33.73	0.01\\
33.74	0.01\\
33.75	0.01\\
33.76	0.01\\
33.77	0.01\\
33.78	0.01\\
33.79	0.01\\
33.8	0.01\\
33.81	0.01\\
33.82	0.01\\
33.83	0.01\\
33.84	0.01\\
33.85	0.01\\
33.86	0.01\\
33.87	0.01\\
33.88	0.01\\
33.89	0.01\\
33.9	0.01\\
33.91	0.01\\
33.92	0.01\\
33.93	0.01\\
33.94	0.01\\
33.95	0.01\\
33.96	0.01\\
33.97	0.01\\
33.98	0.01\\
33.99	0.01\\
34	0.01\\
34.01	0.01\\
34.02	0.01\\
34.03	0.01\\
34.04	0.01\\
34.05	0.01\\
34.06	0.01\\
34.07	0.01\\
34.08	0.01\\
34.09	0.01\\
34.1	0.01\\
34.11	0.01\\
34.12	0.01\\
34.13	0.01\\
34.14	0.01\\
34.15	0.01\\
34.16	0.01\\
34.17	0.01\\
34.18	0.01\\
34.19	0.01\\
34.2	0.01\\
34.21	0.01\\
34.22	0.01\\
34.23	0.01\\
34.24	0.01\\
34.25	0.01\\
34.26	0.01\\
34.27	0.01\\
34.28	0.01\\
34.29	0.01\\
34.3	0.01\\
34.31	0.01\\
34.32	0.01\\
34.33	0.01\\
34.34	0.01\\
34.35	0.01\\
34.36	0.01\\
34.37	0.01\\
34.38	0.01\\
34.39	0.01\\
34.4	0.01\\
34.41	0.01\\
34.42	0.01\\
34.43	0.01\\
34.44	0.01\\
34.45	0.01\\
34.46	0.01\\
34.47	0.01\\
34.48	0.01\\
34.49	0.01\\
34.5	0.01\\
34.51	0.01\\
34.52	0.01\\
34.53	0.01\\
34.54	0.01\\
34.55	0.01\\
34.56	0.01\\
34.57	0.01\\
34.58	0.01\\
34.59	0.01\\
34.6	0.01\\
34.61	0.01\\
34.62	0.01\\
34.63	0.01\\
34.64	0.01\\
34.65	0.01\\
34.66	0.01\\
34.67	0.01\\
34.68	0.01\\
34.69	0.01\\
34.7	0.01\\
34.71	0.01\\
34.72	0.01\\
34.73	0.01\\
34.74	0.01\\
34.75	0.01\\
34.76	0.01\\
34.77	0.01\\
34.78	0.01\\
34.79	0.01\\
34.8	0.01\\
34.81	0.01\\
34.82	0.01\\
34.83	0.01\\
34.84	0.01\\
34.85	0.01\\
34.86	0.01\\
34.87	0.01\\
34.88	0.01\\
34.89	0.01\\
34.9	0.01\\
34.91	0.01\\
34.92	0.01\\
34.93	0.01\\
34.94	0.01\\
34.95	0.01\\
34.96	0.01\\
34.97	0.01\\
34.98	0.01\\
34.99	0.01\\
35	0.01\\
35.01	0.01\\
35.02	0.01\\
35.03	0.01\\
35.04	0.01\\
35.05	0.01\\
35.06	0.01\\
35.07	0.01\\
35.08	0.01\\
35.09	0.01\\
35.1	0.01\\
35.11	0.01\\
35.12	0.01\\
35.13	0.01\\
35.14	0.01\\
35.15	0.01\\
35.16	0.01\\
35.17	0.01\\
35.18	0.01\\
35.19	0.01\\
35.2	0.01\\
35.21	0.01\\
35.22	0.01\\
35.23	0.01\\
35.24	0.01\\
35.25	0.01\\
35.26	0.01\\
35.27	0.01\\
35.28	0.01\\
35.29	0.01\\
35.3	0.01\\
35.31	0.01\\
35.32	0.01\\
35.33	0.01\\
35.34	0.01\\
35.35	0.01\\
35.36	0.01\\
35.37	0.01\\
35.38	0.01\\
35.39	0.01\\
35.4	0.01\\
35.41	0.01\\
35.42	0.01\\
35.43	0.01\\
35.44	0.01\\
35.45	0.01\\
35.46	0.01\\
35.47	0.01\\
35.48	0.01\\
35.49	0.01\\
35.5	0.01\\
35.51	0.01\\
35.52	0.01\\
35.53	0.01\\
35.54	0.01\\
35.55	0.01\\
35.56	0.01\\
35.57	0.01\\
35.58	0.01\\
35.59	0.01\\
35.6	0.01\\
35.61	0.01\\
35.62	0.01\\
35.63	0.01\\
35.64	0.01\\
35.65	0.01\\
35.66	0.01\\
35.67	0.01\\
35.68	0.01\\
35.69	0.01\\
35.7	0.01\\
35.71	0.01\\
35.72	0.01\\
35.73	0.01\\
35.74	0.01\\
35.75	0.01\\
35.76	0.01\\
35.77	0.01\\
35.78	0.01\\
35.79	0.01\\
35.8	0.01\\
35.81	0.01\\
35.82	0.01\\
35.83	0.01\\
35.84	0.01\\
35.85	0.01\\
35.86	0.01\\
35.87	0.01\\
35.88	0.01\\
35.89	0.01\\
35.9	0.01\\
35.91	0.01\\
35.92	0.01\\
35.93	0.01\\
35.94	0.01\\
35.95	0.01\\
35.96	0.01\\
35.97	0.01\\
35.98	0.01\\
35.99	0.01\\
36	0.01\\
36.01	0.01\\
36.02	0.01\\
36.03	0.01\\
36.04	0.01\\
36.05	0.01\\
36.06	0.01\\
36.07	0.01\\
36.08	0.01\\
36.09	0.01\\
36.1	0.01\\
36.11	0.01\\
36.12	0.01\\
36.13	0.01\\
36.14	0.01\\
36.15	0.01\\
36.16	0.01\\
36.17	0.01\\
36.18	0.01\\
36.19	0.01\\
36.2	0.01\\
36.21	0.01\\
36.22	0.01\\
36.23	0.01\\
36.24	0.01\\
36.25	0.01\\
36.26	0.01\\
36.27	0.01\\
36.28	0.01\\
36.29	0.01\\
36.3	0.01\\
36.31	0.01\\
36.32	0.01\\
36.33	0.01\\
36.34	0.01\\
36.35	0.01\\
36.36	0.01\\
36.37	0.01\\
36.38	0.01\\
36.39	0.01\\
36.4	0.01\\
36.41	0.01\\
36.42	0.01\\
36.43	0.01\\
36.44	0.01\\
36.45	0.01\\
36.46	0.01\\
36.47	0.01\\
36.48	0.01\\
36.49	0.01\\
36.5	0.01\\
36.51	0.01\\
36.52	0.01\\
36.53	0.01\\
36.54	0.01\\
36.55	0.01\\
36.56	0.01\\
36.57	0.01\\
36.58	0.01\\
36.59	0.01\\
36.6	0.01\\
36.61	0.01\\
36.62	0.01\\
36.63	0.01\\
36.64	0.01\\
36.65	0.01\\
36.66	0.01\\
36.67	0.01\\
36.68	0.01\\
36.69	0.01\\
36.7	0.01\\
36.71	0.01\\
36.72	0.01\\
36.73	0.01\\
36.74	0.01\\
36.75	0.01\\
36.76	0.01\\
36.77	0.01\\
36.78	0.01\\
36.79	0.01\\
36.8	0.01\\
36.81	0.01\\
36.82	0.01\\
36.83	0.01\\
36.84	0.01\\
36.85	0.01\\
36.86	0.01\\
36.87	0.01\\
36.88	0.01\\
36.89	0.01\\
36.9	0.01\\
36.91	0.01\\
36.92	0.01\\
36.93	0.01\\
36.94	0.01\\
36.95	0.01\\
36.96	0.01\\
36.97	0.01\\
36.98	0.01\\
36.99	0.01\\
37	0.01\\
37.01	0.01\\
37.02	0.01\\
37.03	0.01\\
37.04	0.01\\
37.05	0.01\\
37.06	0.01\\
37.07	0.01\\
37.08	0.01\\
37.09	0.01\\
37.1	0.01\\
37.11	0.01\\
37.12	0.01\\
37.13	0.01\\
37.14	0.01\\
37.15	0.01\\
37.16	0.01\\
37.17	0.01\\
37.18	0.01\\
37.19	0.01\\
37.2	0.01\\
37.21	0.01\\
37.22	0.01\\
37.23	0.01\\
37.24	0.01\\
37.25	0.01\\
37.26	0.01\\
37.27	0.01\\
37.28	0.01\\
37.29	0.01\\
37.3	0.01\\
37.31	0.01\\
37.32	0.01\\
37.33	0.01\\
37.34	0.01\\
37.35	0.01\\
37.36	0.01\\
37.37	0.01\\
37.38	0.01\\
37.39	0.01\\
37.4	0.01\\
37.41	0.01\\
37.42	0.01\\
37.43	0.01\\
37.44	0.01\\
37.45	0.01\\
37.46	0.01\\
37.47	0.01\\
37.48	0.01\\
37.49	0.01\\
37.5	0.01\\
37.51	0.01\\
37.52	0.01\\
37.53	0.01\\
37.54	0.01\\
37.55	0.01\\
37.56	0.01\\
37.57	0.01\\
37.58	0.01\\
37.59	0.01\\
37.6	0.01\\
37.61	0.01\\
37.62	0.01\\
37.63	0.01\\
37.64	0.01\\
37.65	0.01\\
37.66	0.01\\
37.67	0.01\\
37.68	0.01\\
37.69	0.01\\
37.7	0.01\\
37.71	0.01\\
37.72	0.01\\
37.73	0.01\\
37.74	0.01\\
37.75	0.01\\
37.76	0.01\\
37.77	0.01\\
37.78	0.01\\
37.79	0.01\\
37.8	0.01\\
37.81	0.01\\
37.82	0.01\\
37.83	0.01\\
37.84	0.01\\
37.85	0.01\\
37.86	0.01\\
37.87	0.01\\
37.88	0.01\\
37.89	0.01\\
37.9	0.01\\
37.91	0.01\\
37.92	0.01\\
37.93	0.01\\
37.94	0.01\\
37.95	0.01\\
37.96	0.01\\
37.97	0.01\\
37.98	0.01\\
37.99	0.01\\
38	0.01\\
38.01	0.01\\
38.02	0.01\\
38.03	0.01\\
38.04	0.01\\
38.05	0.01\\
38.06	0.01\\
38.07	0.01\\
38.08	0.01\\
38.09	0.01\\
38.1	0.01\\
38.11	0.01\\
38.12	0.01\\
38.13	0.01\\
38.14	0.01\\
38.15	0.01\\
38.16	0.01\\
38.17	0.01\\
38.18	0.01\\
38.19	0.01\\
38.2	0.01\\
38.21	0.01\\
38.22	0.01\\
38.23	0.01\\
38.24	0.01\\
38.25	0.01\\
38.26	0.01\\
38.27	0.01\\
38.28	0.01\\
38.29	0.01\\
38.3	0.01\\
38.31	0.01\\
38.32	0.01\\
38.33	0.01\\
38.34	0.01\\
38.35	0.01\\
38.36	0.01\\
38.37	0.01\\
38.38	0.01\\
38.39	0.01\\
38.4	0.01\\
38.41	0.01\\
38.42	0.01\\
38.43	0.01\\
38.44	0.01\\
38.45	0.01\\
38.46	0.01\\
38.47	0.01\\
38.48	0.01\\
38.49	0.01\\
38.5	0.01\\
38.51	0.01\\
38.52	0.01\\
38.53	0.01\\
38.54	0.01\\
38.55	0.01\\
38.56	0.01\\
38.57	0.01\\
38.58	0.01\\
38.59	0.01\\
38.6	0.01\\
38.61	0.01\\
38.62	0.01\\
38.63	0.01\\
38.64	0.01\\
38.65	0.01\\
38.66	0.01\\
38.67	0.01\\
38.68	0.01\\
38.69	0.01\\
38.7	0.01\\
38.71	0.01\\
38.72	0.01\\
38.73	0.01\\
38.74	0.01\\
38.75	0.01\\
38.76	0.01\\
38.77	0.01\\
38.78	0.01\\
38.79	0.01\\
38.8	0.01\\
38.81	0.01\\
38.82	0.01\\
38.83	0.01\\
38.84	0.01\\
38.85	0.01\\
38.86	0.01\\
38.87	0.01\\
38.88	0.01\\
38.89	0.01\\
38.9	0.01\\
38.91	0.01\\
38.92	0.01\\
38.93	0.01\\
38.94	0.01\\
38.95	0.01\\
38.96	0.01\\
38.97	0.01\\
38.98	0.01\\
38.99	0.01\\
39	0.01\\
39.01	0.01\\
39.02	0.01\\
39.03	0.01\\
39.04	0.01\\
39.05	0.01\\
39.06	0.01\\
39.07	0.01\\
39.08	0.01\\
39.09	0.01\\
39.1	0.01\\
39.11	0.01\\
39.12	0.01\\
39.13	0.01\\
39.14	0.01\\
39.15	0.01\\
39.16	0.01\\
39.17	0.01\\
39.18	0.01\\
39.19	0.01\\
39.2	0.01\\
39.21	0.01\\
39.22	0.01\\
39.23	0.01\\
39.24	0.01\\
39.25	0.01\\
39.26	0.01\\
39.27	0.01\\
39.28	0.01\\
39.29	0.01\\
39.3	0.01\\
39.31	0.01\\
39.32	0.01\\
39.33	0.01\\
39.34	0.01\\
39.35	0.01\\
39.36	0.01\\
39.37	0.01\\
39.38	0.01\\
39.39	0.01\\
39.4	0.01\\
39.41	0.01\\
39.42	0.01\\
39.43	0.01\\
39.44	0.01\\
39.45	0.01\\
39.46	0.01\\
39.47	0.01\\
39.48	0.01\\
39.49	0.01\\
39.5	0.01\\
39.51	0.01\\
39.52	0.01\\
39.53	0.01\\
39.54	0.01\\
39.55	0.01\\
39.56	0.01\\
39.57	0.01\\
39.58	0.01\\
39.59	0.01\\
39.6	0.01\\
39.61	0.01\\
39.62	0.01\\
39.63	0.01\\
39.64	0.01\\
39.65	0.01\\
39.66	0.01\\
39.67	0.01\\
39.68	0.01\\
39.69	0.01\\
39.7	0.01\\
39.71	0.01\\
39.72	0.01\\
39.73	0.01\\
39.74	0.01\\
39.75	0.01\\
39.76	0.01\\
39.77	0.01\\
39.78	0.01\\
39.79	0.01\\
39.8	0.01\\
39.81	0.01\\
39.82	0.01\\
39.83	0.01\\
39.84	0.01\\
39.85	0.01\\
39.86	0.01\\
39.87	0.01\\
39.88	0.01\\
39.89	0.01\\
39.9	0.01\\
39.91	0.01\\
39.92	0.01\\
39.93	0.01\\
39.94	0.01\\
39.95	0.01\\
39.96	0.01\\
39.97	0.01\\
39.98	0.01\\
39.99	0.01\\
40	0.01\\
40.01	0.01\\
};
\addplot [color=red,dashed,forget plot]
  table[row sep=crcr]{%
40.01	0.01\\
40.02	0.01\\
40.03	0.01\\
40.04	0.01\\
40.05	0.01\\
40.06	0.01\\
40.07	0.01\\
40.08	0.01\\
40.09	0.01\\
40.1	0.01\\
40.11	0.01\\
40.12	0.01\\
40.13	0.01\\
40.14	0.01\\
40.15	0.01\\
40.16	0.01\\
40.17	0.01\\
40.18	0.01\\
40.19	0.01\\
40.2	0.01\\
40.21	0.01\\
40.22	0.01\\
40.23	0.01\\
40.24	0.01\\
40.25	0.01\\
40.26	0.01\\
40.27	0.01\\
40.28	0.01\\
40.29	0.01\\
40.3	0.01\\
40.31	0.01\\
40.32	0.01\\
40.33	0.01\\
40.34	0.01\\
40.35	0.01\\
40.36	0.01\\
40.37	0.01\\
40.38	0.01\\
40.39	0.01\\
40.4	0.01\\
40.41	0.01\\
40.42	0.01\\
40.43	0.01\\
40.44	0.01\\
40.45	0.01\\
40.46	0.01\\
40.47	0.01\\
40.48	0.01\\
40.49	0.01\\
40.5	0.01\\
40.51	0.01\\
40.52	0.01\\
40.53	0.01\\
40.54	0.01\\
40.55	0.01\\
40.56	0.01\\
40.57	0.01\\
40.58	0.01\\
40.59	0.01\\
40.6	0.01\\
40.61	0.01\\
40.62	0.01\\
40.63	0.01\\
40.64	0.01\\
40.65	0.01\\
40.66	0.01\\
40.67	0.01\\
40.68	0.01\\
40.69	0.01\\
40.7	0.01\\
40.71	0.01\\
40.72	0.01\\
40.73	0.01\\
40.74	0.01\\
40.75	0.01\\
40.76	0.01\\
40.77	0.01\\
40.78	0.01\\
40.79	0.01\\
40.8	0.01\\
40.81	0.01\\
40.82	0.01\\
40.83	0.01\\
40.84	0.01\\
40.85	0.01\\
40.86	0.01\\
40.87	0.01\\
40.88	0.01\\
40.89	0.01\\
40.9	0.01\\
40.91	0.01\\
40.92	0.01\\
40.93	0.01\\
40.94	0.01\\
40.95	0.01\\
40.96	0.01\\
40.97	0.01\\
40.98	0.01\\
40.99	0.01\\
41	0.01\\
41.01	0.01\\
41.02	0.01\\
41.03	0.01\\
41.04	0.01\\
41.05	0.01\\
41.06	0.01\\
41.07	0.01\\
41.08	0.01\\
41.09	0.01\\
41.1	0.01\\
41.11	0.01\\
41.12	0.01\\
41.13	0.01\\
41.14	0.01\\
41.15	0.01\\
41.16	0.01\\
41.17	0.01\\
41.18	0.01\\
41.19	0.01\\
41.2	0.01\\
41.21	0.01\\
41.22	0.01\\
41.23	0.01\\
41.24	0.01\\
41.25	0.01\\
41.26	0.01\\
41.27	0.01\\
41.28	0.01\\
41.29	0.01\\
41.3	0.01\\
41.31	0.01\\
41.32	0.01\\
41.33	0.01\\
41.34	0.01\\
41.35	0.01\\
41.36	0.01\\
41.37	0.01\\
41.38	0.01\\
41.39	0.01\\
41.4	0.01\\
41.41	0.01\\
41.42	0.01\\
41.43	0.01\\
41.44	0.01\\
41.45	0.01\\
41.46	0.01\\
41.47	0.01\\
41.48	0.01\\
41.49	0.01\\
41.5	0.01\\
41.51	0.01\\
41.52	0.01\\
41.53	0.01\\
41.54	0.01\\
41.55	0.01\\
41.56	0.01\\
41.57	0.01\\
41.58	0.01\\
41.59	0.01\\
41.6	0.01\\
41.61	0.01\\
41.62	0.01\\
41.63	0.01\\
41.64	0.01\\
41.65	0.01\\
41.66	0.01\\
41.67	0.01\\
41.68	0.01\\
41.69	0.01\\
41.7	0.01\\
41.71	0.01\\
41.72	0.01\\
41.73	0.01\\
41.74	0.01\\
41.75	0.01\\
41.76	0.01\\
41.77	0.01\\
41.78	0.01\\
41.79	0.01\\
41.8	0.01\\
41.81	0.01\\
41.82	0.01\\
41.83	0.01\\
41.84	0.01\\
41.85	0.01\\
41.86	0.01\\
41.87	0.01\\
41.88	0.01\\
41.89	0.01\\
41.9	0.01\\
41.91	0.01\\
41.92	0.01\\
41.93	0.01\\
41.94	0.01\\
41.95	0.01\\
41.96	0.01\\
41.97	0.01\\
41.98	0.01\\
41.99	0.01\\
42	0.01\\
42.01	0.01\\
42.02	0.01\\
42.03	0.01\\
42.04	0.01\\
42.05	0.01\\
42.06	0.01\\
42.07	0.01\\
42.08	0.01\\
42.09	0.01\\
42.1	0.01\\
42.11	0.01\\
42.12	0.01\\
42.13	0.01\\
42.14	0.01\\
42.15	0.01\\
42.16	0.01\\
42.17	0.01\\
42.18	0.01\\
42.19	0.01\\
42.2	0.01\\
42.21	0.01\\
42.22	0.01\\
42.23	0.01\\
42.24	0.01\\
42.25	0.01\\
42.26	0.01\\
42.27	0.01\\
42.28	0.01\\
42.29	0.01\\
42.3	0.01\\
42.31	0.01\\
42.32	0.01\\
42.33	0.01\\
42.34	0.01\\
42.35	0.01\\
42.36	0.01\\
42.37	0.01\\
42.38	0.01\\
42.39	0.01\\
42.4	0.01\\
42.41	0.01\\
42.42	0.01\\
42.43	0.01\\
42.44	0.01\\
42.45	0.01\\
42.46	0.01\\
42.47	0.01\\
42.48	0.01\\
42.49	0.01\\
42.5	0.01\\
42.51	0.01\\
42.52	0.01\\
42.53	0.01\\
42.54	0.01\\
42.55	0.01\\
42.56	0.01\\
42.57	0.01\\
42.58	0.01\\
42.59	0.01\\
42.6	0.01\\
42.61	0.01\\
42.62	0.01\\
42.63	0.01\\
42.64	0.01\\
42.65	0.01\\
42.66	0.01\\
42.67	0.01\\
42.68	0.01\\
42.69	0.01\\
42.7	0.01\\
42.71	0.01\\
42.72	0.01\\
42.73	0.01\\
42.74	0.01\\
42.75	0.01\\
42.76	0.01\\
42.77	0.01\\
42.78	0.01\\
42.79	0.01\\
42.8	0.01\\
42.81	0.01\\
42.82	0.01\\
42.83	0.01\\
42.84	0.01\\
42.85	0.01\\
42.86	0.01\\
42.87	0.01\\
42.88	0.01\\
42.89	0.01\\
42.9	0.01\\
42.91	0.01\\
42.92	0.01\\
42.93	0.01\\
42.94	0.01\\
42.95	0.01\\
42.96	0.01\\
42.97	0.01\\
42.98	0.01\\
42.99	0.01\\
43	0.01\\
43.01	0.01\\
43.02	0.01\\
43.03	0.01\\
43.04	0.01\\
43.05	0.01\\
43.06	0.01\\
43.07	0.01\\
43.08	0.01\\
43.09	0.01\\
43.1	0.01\\
43.11	0.01\\
43.12	0.01\\
43.13	0.01\\
43.14	0.01\\
43.15	0.01\\
43.16	0.01\\
43.17	0.01\\
43.18	0.01\\
43.19	0.01\\
43.2	0.01\\
43.21	0.01\\
43.22	0.01\\
43.23	0.01\\
43.24	0.01\\
43.25	0.01\\
43.26	0.01\\
43.27	0.01\\
43.28	0.01\\
43.29	0.01\\
43.3	0.01\\
43.31	0.01\\
43.32	0.01\\
43.33	0.01\\
43.34	0.01\\
43.35	0.01\\
43.36	0.01\\
43.37	0.01\\
43.38	0.01\\
43.39	0.01\\
43.4	0.01\\
43.41	0.01\\
43.42	0.01\\
43.43	0.01\\
43.44	0.01\\
43.45	0.01\\
43.46	0.01\\
43.47	0.01\\
43.48	0.01\\
43.49	0.01\\
43.5	0.01\\
43.51	0.01\\
43.52	0.01\\
43.53	0.01\\
43.54	0.01\\
43.55	0.01\\
43.56	0.01\\
43.57	0.01\\
43.58	0.01\\
43.59	0.01\\
43.6	0.01\\
43.61	0.01\\
43.62	0.01\\
43.63	0.01\\
43.64	0.01\\
43.65	0.01\\
43.66	0.01\\
43.67	0.01\\
43.68	0.01\\
43.69	0.01\\
43.7	0.01\\
43.71	0.01\\
43.72	0.01\\
43.73	0.01\\
43.74	0.01\\
43.75	0.01\\
43.76	0.01\\
43.77	0.01\\
43.78	0.01\\
43.79	0.01\\
43.8	0.01\\
43.81	0.01\\
43.82	0.01\\
43.83	0.01\\
43.84	0.01\\
43.85	0.01\\
43.86	0.01\\
43.87	0.01\\
43.88	0.01\\
43.89	0.01\\
43.9	0.01\\
43.91	0.01\\
43.92	0.01\\
43.93	0.01\\
43.94	0.01\\
43.95	0.01\\
43.96	0.01\\
43.97	0.01\\
43.98	0.01\\
43.99	0.01\\
44	0.01\\
44.01	0.01\\
44.02	0.01\\
44.03	0.01\\
44.04	0.01\\
44.05	0.01\\
44.06	0.01\\
44.07	0.01\\
44.08	0.01\\
44.09	0.01\\
44.1	0.01\\
44.11	0.01\\
44.12	0.01\\
44.13	0.01\\
44.14	0.01\\
44.15	0.01\\
44.16	0.01\\
44.17	0.01\\
44.18	0.01\\
44.19	0.01\\
44.2	0.01\\
44.21	0.01\\
44.22	0.01\\
44.23	0.01\\
44.24	0.01\\
44.25	0.01\\
44.26	0.01\\
44.27	0.01\\
44.28	0.01\\
44.29	0.01\\
44.3	0.01\\
44.31	0.01\\
44.32	0.01\\
44.33	0.01\\
44.34	0.01\\
44.35	0.01\\
44.36	0.01\\
44.37	0.01\\
44.38	0.01\\
44.39	0.01\\
44.4	0.01\\
44.41	0.01\\
44.42	0.01\\
44.43	0.01\\
44.44	0.01\\
44.45	0.01\\
44.46	0.01\\
44.47	0.01\\
44.48	0.01\\
44.49	0.01\\
44.5	0.01\\
44.51	0.01\\
44.52	0.01\\
44.53	0.01\\
44.54	0.01\\
44.55	0.01\\
44.56	0.01\\
44.57	0.01\\
44.58	0.01\\
44.59	0.01\\
44.6	0.01\\
44.61	0.01\\
44.62	0.01\\
44.63	0.01\\
44.64	0.01\\
44.65	0.01\\
44.66	0.01\\
44.67	0.01\\
44.68	0.01\\
44.69	0.01\\
44.7	0.01\\
44.71	0.01\\
44.72	0.01\\
44.73	0.01\\
44.74	0.01\\
44.75	0.01\\
44.76	0.01\\
44.77	0.01\\
44.78	0.01\\
44.79	0.01\\
44.8	0.01\\
44.81	0.01\\
44.82	0.01\\
44.83	0.01\\
44.84	0.01\\
44.85	0.01\\
44.86	0.01\\
44.87	0.01\\
44.88	0.01\\
44.89	0.01\\
44.9	0.01\\
44.91	0.01\\
44.92	0.01\\
44.93	0.01\\
44.94	0.01\\
44.95	0.01\\
44.96	0.01\\
44.97	0.01\\
44.98	0.01\\
44.99	0.01\\
45	0.01\\
45.01	0.01\\
45.02	0.01\\
45.03	0.01\\
45.04	0.01\\
45.05	0.01\\
45.06	0.01\\
45.07	0.01\\
45.08	0.01\\
45.09	0.01\\
45.1	0.01\\
45.11	0.01\\
45.12	0.01\\
45.13	0.01\\
45.14	0.01\\
45.15	0.01\\
45.16	0.01\\
45.17	0.01\\
45.18	0.01\\
45.19	0.01\\
45.2	0.01\\
45.21	0.01\\
45.22	0.01\\
45.23	0.01\\
45.24	0.01\\
45.25	0.01\\
45.26	0.01\\
45.27	0.01\\
45.28	0.01\\
45.29	0.01\\
45.3	0.01\\
45.31	0.01\\
45.32	0.01\\
45.33	0.01\\
45.34	0.01\\
45.35	0.01\\
45.36	0.01\\
45.37	0.01\\
45.38	0.01\\
45.39	0.01\\
45.4	0.01\\
45.41	0.01\\
45.42	0.01\\
45.43	0.01\\
45.44	0.01\\
45.45	0.01\\
45.46	0.01\\
45.47	0.01\\
45.48	0.01\\
45.49	0.01\\
45.5	0.01\\
45.51	0.01\\
45.52	0.01\\
45.53	0.01\\
45.54	0.01\\
45.55	0.01\\
45.56	0.01\\
45.57	0.01\\
45.58	0.01\\
45.59	0.01\\
45.6	0.01\\
45.61	0.01\\
45.62	0.01\\
45.63	0.01\\
45.64	0.01\\
45.65	0.01\\
45.66	0.01\\
45.67	0.01\\
45.68	0.01\\
45.69	0.01\\
45.7	0.01\\
45.71	0.01\\
45.72	0.01\\
45.73	0.01\\
45.74	0.01\\
45.75	0.01\\
45.76	0.01\\
45.77	0.01\\
45.78	0.01\\
45.79	0.01\\
45.8	0.01\\
45.81	0.01\\
45.82	0.01\\
45.83	0.01\\
45.84	0.01\\
45.85	0.01\\
45.86	0.01\\
45.87	0.01\\
45.88	0.01\\
45.89	0.01\\
45.9	0.01\\
45.91	0.01\\
45.92	0.01\\
45.93	0.01\\
45.94	0.01\\
45.95	0.01\\
45.96	0.01\\
45.97	0.01\\
45.98	0.01\\
45.99	0.01\\
46	0.01\\
46.01	0.01\\
46.02	0.01\\
46.03	0.01\\
46.04	0.01\\
46.05	0.01\\
46.06	0.01\\
46.07	0.01\\
46.08	0.01\\
46.09	0.01\\
46.1	0.01\\
46.11	0.01\\
46.12	0.01\\
46.13	0.01\\
46.14	0.01\\
46.15	0.01\\
46.16	0.01\\
46.17	0.01\\
46.18	0.01\\
46.19	0.01\\
46.2	0.01\\
46.21	0.01\\
46.22	0.01\\
46.23	0.01\\
46.24	0.01\\
46.25	0.01\\
46.26	0.01\\
46.27	0.01\\
46.28	0.01\\
46.29	0.01\\
46.3	0.01\\
46.31	0.01\\
46.32	0.01\\
46.33	0.01\\
46.34	0.01\\
46.35	0.01\\
46.36	0.01\\
46.37	0.01\\
46.38	0.01\\
46.39	0.01\\
46.4	0.01\\
46.41	0.01\\
46.42	0.01\\
46.43	0.01\\
46.44	0.01\\
46.45	0.01\\
46.46	0.01\\
46.47	0.01\\
46.48	0.01\\
46.49	0.01\\
46.5	0.01\\
46.51	0.01\\
46.52	0.01\\
46.53	0.01\\
46.54	0.01\\
46.55	0.01\\
46.56	0.01\\
46.57	0.01\\
46.58	0.01\\
46.59	0.01\\
46.6	0.01\\
46.61	0.01\\
46.62	0.01\\
46.63	0.01\\
46.64	0.01\\
46.65	0.01\\
46.66	0.01\\
46.67	0.01\\
46.68	0.01\\
46.69	0.01\\
46.7	0.01\\
46.71	0.01\\
46.72	0.01\\
46.73	0.01\\
46.74	0.01\\
46.75	0.01\\
46.76	0.01\\
46.77	0.01\\
46.78	0.01\\
46.79	0.01\\
46.8	0.01\\
46.81	0.01\\
46.82	0.01\\
46.83	0.01\\
46.84	0.01\\
46.85	0.01\\
46.86	0.01\\
46.87	0.01\\
46.88	0.01\\
46.89	0.01\\
46.9	0.01\\
46.91	0.01\\
46.92	0.01\\
46.93	0.01\\
46.94	0.01\\
46.95	0.01\\
46.96	0.01\\
46.97	0.01\\
46.98	0.01\\
46.99	0.01\\
47	0.01\\
47.01	0.01\\
47.02	0.01\\
47.03	0.01\\
47.04	0.01\\
47.05	0.01\\
47.06	0.01\\
47.07	0.01\\
47.08	0.01\\
47.09	0.01\\
47.1	0.01\\
47.11	0.01\\
47.12	0.01\\
47.13	0.01\\
47.14	0.01\\
47.15	0.01\\
47.16	0.01\\
47.17	0.01\\
47.18	0.01\\
47.19	0.01\\
47.2	0.01\\
47.21	0.01\\
47.22	0.01\\
47.23	0.01\\
47.24	0.01\\
47.25	0.01\\
47.26	0.01\\
47.27	0.01\\
47.28	0.01\\
47.29	0.01\\
47.3	0.01\\
47.31	0.01\\
47.32	0.01\\
47.33	0.01\\
47.34	0.01\\
47.35	0.01\\
47.36	0.01\\
47.37	0.01\\
47.38	0.01\\
47.39	0.01\\
47.4	0.01\\
47.41	0.01\\
47.42	0.01\\
47.43	0.01\\
47.44	0.01\\
47.45	0.01\\
47.46	0.01\\
47.47	0.01\\
47.48	0.01\\
47.49	0.01\\
47.5	0.01\\
47.51	0.01\\
47.52	0.01\\
47.53	0.01\\
47.54	0.01\\
47.55	0.01\\
47.56	0.01\\
47.57	0.01\\
47.58	0.01\\
47.59	0.01\\
47.6	0.01\\
47.61	0.01\\
47.62	0.01\\
47.63	0.01\\
47.64	0.01\\
47.65	0.01\\
47.66	0.01\\
47.67	0.01\\
47.68	0.01\\
47.69	0.01\\
47.7	0.01\\
47.71	0.01\\
47.72	0.01\\
47.73	0.01\\
47.74	0.01\\
47.75	0.01\\
47.76	0.01\\
47.77	0.01\\
47.78	0.01\\
47.79	0.01\\
47.8	0.01\\
47.81	0.01\\
47.82	0.01\\
47.83	0.01\\
47.84	0.01\\
47.85	0.01\\
47.86	0.01\\
47.87	0.01\\
47.88	0.01\\
47.89	0.01\\
47.9	0.01\\
47.91	0.01\\
47.92	0.01\\
47.93	0.01\\
47.94	0.01\\
47.95	0.01\\
47.96	0.01\\
47.97	0.01\\
47.98	0.01\\
47.99	0.01\\
48	0.01\\
48.01	0.01\\
48.02	0.01\\
48.03	0.01\\
48.04	0.01\\
48.05	0.01\\
48.06	0.01\\
48.07	0.01\\
48.08	0.01\\
48.09	0.01\\
48.1	0.01\\
48.11	0.01\\
48.12	0.01\\
48.13	0.01\\
48.14	0.01\\
48.15	0.01\\
48.16	0.01\\
48.17	0.01\\
48.18	0.01\\
48.19	0.01\\
48.2	0.01\\
48.21	0.01\\
48.22	0.01\\
48.23	0.01\\
48.24	0.01\\
48.25	0.01\\
48.26	0.01\\
48.27	0.01\\
48.28	0.01\\
48.29	0.01\\
48.3	0.01\\
48.31	0.01\\
48.32	0.01\\
48.33	0.01\\
48.34	0.01\\
48.35	0.01\\
48.36	0.01\\
48.37	0.01\\
48.38	0.01\\
48.39	0.01\\
48.4	0.01\\
48.41	0.01\\
48.42	0.01\\
48.43	0.01\\
48.44	0.01\\
48.45	0.01\\
48.46	0.01\\
48.47	0.01\\
48.48	0.01\\
48.49	0.01\\
48.5	0.01\\
48.51	0.01\\
48.52	0.01\\
48.53	0.01\\
48.54	0.01\\
48.55	0.01\\
48.56	0.01\\
48.57	0.01\\
48.58	0.01\\
48.59	0.01\\
48.6	0.01\\
48.61	0.01\\
48.62	0.01\\
48.63	0.01\\
48.64	0.01\\
48.65	0.01\\
48.66	0.01\\
48.67	0.01\\
48.68	0.01\\
48.69	0.01\\
48.7	0.01\\
48.71	0.01\\
48.72	0.01\\
48.73	0.01\\
48.74	0.01\\
48.75	0.01\\
48.76	0.01\\
48.77	0.01\\
48.78	0.01\\
48.79	0.01\\
48.8	0.01\\
48.81	0.01\\
48.82	0.01\\
48.83	0.01\\
48.84	0.01\\
48.85	0.01\\
48.86	0.01\\
48.87	0.01\\
48.88	0.01\\
48.89	0.01\\
48.9	0.01\\
48.91	0.01\\
48.92	0.01\\
48.93	0.01\\
48.94	0.01\\
48.95	0.01\\
48.96	0.01\\
48.97	0.01\\
48.98	0.01\\
48.99	0.01\\
49	0.01\\
49.01	0.01\\
49.02	0.01\\
49.03	0.01\\
49.04	0.01\\
49.05	0.01\\
49.06	0.01\\
49.07	0.01\\
49.08	0.01\\
49.09	0.01\\
49.1	0.01\\
49.11	0.01\\
49.12	0.01\\
49.13	0.01\\
49.14	0.01\\
49.15	0.01\\
49.16	0.01\\
49.17	0.01\\
49.18	0.01\\
49.19	0.01\\
49.2	0.01\\
49.21	0.01\\
49.22	0.01\\
49.23	0.01\\
49.24	0.01\\
49.25	0.01\\
49.26	0.01\\
49.27	0.01\\
49.28	0.01\\
49.29	0.01\\
49.3	0.01\\
49.31	0.01\\
49.32	0.01\\
49.33	0.01\\
49.34	0.01\\
49.35	0.01\\
49.36	0.01\\
49.37	0.01\\
49.38	0.01\\
49.39	0.01\\
49.4	0.01\\
49.41	0.01\\
49.42	0.01\\
49.43	0.01\\
49.44	0.01\\
49.45	0.01\\
49.46	0.01\\
49.47	0.01\\
49.48	0.01\\
49.49	0.01\\
49.5	0.01\\
49.51	0.01\\
49.52	0.01\\
49.53	0.01\\
49.54	0.01\\
49.55	0.01\\
49.56	0.01\\
49.57	0.01\\
49.58	0.01\\
49.59	0.01\\
49.6	0.01\\
49.61	0.01\\
49.62	0.01\\
49.63	0.01\\
49.64	0.01\\
49.65	0.01\\
49.66	0.01\\
49.67	0.01\\
49.68	0.01\\
49.69	0.01\\
49.7	0.01\\
49.71	0.01\\
49.72	0.01\\
49.73	0.01\\
49.74	0.01\\
49.75	0.01\\
49.76	0.01\\
49.77	0.01\\
49.78	0.01\\
49.79	0.01\\
49.8	0.01\\
49.81	0.01\\
49.82	0.01\\
49.83	0.01\\
49.84	0.01\\
49.85	0.01\\
49.86	0.01\\
49.87	0.01\\
49.88	0.01\\
49.89	0.01\\
49.9	0.01\\
49.91	0.01\\
49.92	0.01\\
49.93	0.01\\
49.94	0.01\\
49.95	0.01\\
49.96	0.01\\
49.97	0.01\\
49.98	0.01\\
49.99	0.01\\
50	0.01\\
50.01	0.01\\
50.02	0.01\\
50.03	0.01\\
50.04	0.01\\
50.05	0.01\\
50.06	0.01\\
50.07	0.01\\
50.08	0.01\\
50.09	0.01\\
50.1	0.01\\
50.11	0.01\\
50.12	0.01\\
50.13	0.01\\
50.14	0.01\\
50.15	0.01\\
50.16	0.01\\
50.17	0.01\\
50.18	0.01\\
50.19	0.01\\
50.2	0.01\\
50.21	0.01\\
50.22	0.01\\
50.23	0.01\\
50.24	0.01\\
50.25	0.01\\
50.26	0.01\\
50.27	0.01\\
50.28	0.01\\
50.29	0.01\\
50.3	0.01\\
50.31	0.01\\
50.32	0.01\\
50.33	0.01\\
50.34	0.01\\
50.35	0.01\\
50.36	0.01\\
50.37	0.01\\
50.38	0.01\\
50.39	0.01\\
50.4	0.01\\
50.41	0.01\\
50.42	0.01\\
50.43	0.01\\
50.44	0.01\\
50.45	0.01\\
50.46	0.01\\
50.47	0.01\\
50.48	0.01\\
50.49	0.01\\
50.5	0.01\\
50.51	0.01\\
50.52	0.01\\
50.53	0.01\\
50.54	0.01\\
50.55	0.01\\
50.56	0.01\\
50.57	0.01\\
50.58	0.01\\
50.59	0.01\\
50.6	0.01\\
50.61	0.01\\
50.62	0.01\\
50.63	0.01\\
50.64	0.01\\
50.65	0.01\\
50.66	0.01\\
50.67	0.01\\
50.68	0.01\\
50.69	0.01\\
50.7	0.01\\
50.71	0.01\\
50.72	0.01\\
50.73	0.01\\
50.74	0.01\\
50.75	0.01\\
50.76	0.01\\
50.77	0.01\\
50.78	0.01\\
50.79	0.01\\
50.8	0.01\\
50.81	0.01\\
50.82	0.01\\
50.83	0.01\\
50.84	0.01\\
50.85	0.01\\
50.86	0.01\\
50.87	0.01\\
50.88	0.01\\
50.89	0.01\\
50.9	0.01\\
50.91	0.01\\
50.92	0.01\\
50.93	0.01\\
50.94	0.01\\
50.95	0.01\\
50.96	0.01\\
50.97	0.01\\
50.98	0.01\\
50.99	0.01\\
51	0.01\\
51.01	0.01\\
51.02	0.01\\
51.03	0.01\\
51.04	0.01\\
51.05	0.01\\
51.06	0.01\\
51.07	0.01\\
51.08	0.01\\
51.09	0.01\\
51.1	0.01\\
51.11	0.01\\
51.12	0.01\\
51.13	0.01\\
51.14	0.01\\
51.15	0.01\\
51.16	0.01\\
51.17	0.01\\
51.18	0.01\\
51.19	0.01\\
51.2	0.01\\
51.21	0.01\\
51.22	0.01\\
51.23	0.01\\
51.24	0.01\\
51.25	0.01\\
51.26	0.01\\
51.27	0.01\\
51.28	0.01\\
51.29	0.01\\
51.3	0.01\\
51.31	0.01\\
51.32	0.01\\
51.33	0.01\\
51.34	0.01\\
51.35	0.01\\
51.36	0.01\\
51.37	0.01\\
51.38	0.01\\
51.39	0.01\\
51.4	0.01\\
51.41	0.01\\
51.42	0.01\\
51.43	0.01\\
51.44	0.01\\
51.45	0.01\\
51.46	0.01\\
51.47	0.01\\
51.48	0.01\\
51.49	0.01\\
51.5	0.01\\
51.51	0.01\\
51.52	0.01\\
51.53	0.01\\
51.54	0.01\\
51.55	0.01\\
51.56	0.01\\
51.57	0.01\\
51.58	0.01\\
51.59	0.01\\
51.6	0.01\\
51.61	0.01\\
51.62	0.01\\
51.63	0.01\\
51.64	0.01\\
51.65	0.01\\
51.66	0.01\\
51.67	0.01\\
51.68	0.01\\
51.69	0.01\\
51.7	0.01\\
51.71	0.01\\
51.72	0.01\\
51.73	0.01\\
51.74	0.01\\
51.75	0.01\\
51.76	0.01\\
51.77	0.01\\
51.78	0.01\\
51.79	0.01\\
51.8	0.01\\
51.81	0.01\\
51.82	0.01\\
51.83	0.01\\
51.84	0.01\\
51.85	0.01\\
51.86	0.01\\
51.87	0.01\\
51.88	0.01\\
51.89	0.01\\
51.9	0.01\\
51.91	0.01\\
51.92	0.01\\
51.93	0.01\\
51.94	0.01\\
51.95	0.01\\
51.96	0.01\\
51.97	0.01\\
51.98	0.01\\
51.99	0.01\\
52	0.01\\
52.01	0.01\\
52.02	0.01\\
52.03	0.01\\
52.04	0.01\\
52.05	0.01\\
52.06	0.01\\
52.07	0.01\\
52.08	0.01\\
52.09	0.01\\
52.1	0.01\\
52.11	0.01\\
52.12	0.01\\
52.13	0.01\\
52.14	0.01\\
52.15	0.01\\
52.16	0.01\\
52.17	0.01\\
52.18	0.01\\
52.19	0.01\\
52.2	0.01\\
52.21	0.01\\
52.22	0.01\\
52.23	0.01\\
52.24	0.01\\
52.25	0.01\\
52.26	0.01\\
52.27	0.01\\
52.28	0.01\\
52.29	0.01\\
52.3	0.01\\
52.31	0.01\\
52.32	0.01\\
52.33	0.01\\
52.34	0.01\\
52.35	0.01\\
52.36	0.01\\
52.37	0.01\\
52.38	0.01\\
52.39	0.01\\
52.4	0.01\\
52.41	0.01\\
52.42	0.01\\
52.43	0.01\\
52.44	0.01\\
52.45	0.01\\
52.46	0.01\\
52.47	0.01\\
52.48	0.01\\
52.49	0.01\\
52.5	0.01\\
52.51	0.01\\
52.52	0.01\\
52.53	0.01\\
52.54	0.01\\
52.55	0.01\\
52.56	0.01\\
52.57	0.01\\
52.58	0.01\\
52.59	0.01\\
52.6	0.01\\
52.61	0.01\\
52.62	0.01\\
52.63	0.01\\
52.64	0.01\\
52.65	0.01\\
52.66	0.01\\
52.67	0.01\\
52.68	0.01\\
52.69	0.01\\
52.7	0.01\\
52.71	0.01\\
52.72	0.01\\
52.73	0.01\\
52.74	0.01\\
52.75	0.01\\
52.76	0.01\\
52.77	0.01\\
52.78	0.01\\
52.79	0.01\\
52.8	0.01\\
52.81	0.01\\
52.82	0.01\\
52.83	0.01\\
52.84	0.01\\
52.85	0.01\\
52.86	0.01\\
52.87	0.01\\
52.88	0.01\\
52.89	0.01\\
52.9	0.01\\
52.91	0.01\\
52.92	0.01\\
52.93	0.01\\
52.94	0.01\\
52.95	0.01\\
52.96	0.01\\
52.97	0.01\\
52.98	0.01\\
52.99	0.01\\
53	0.01\\
53.01	0.01\\
53.02	0.01\\
53.03	0.01\\
53.04	0.01\\
53.05	0.01\\
53.06	0.01\\
53.07	0.01\\
53.08	0.01\\
53.09	0.01\\
53.1	0.01\\
53.11	0.01\\
53.12	0.01\\
53.13	0.01\\
53.14	0.01\\
53.15	0.01\\
53.16	0.01\\
53.17	0.01\\
53.18	0.01\\
53.19	0.01\\
53.2	0.01\\
53.21	0.01\\
53.22	0.01\\
53.23	0.01\\
53.24	0.01\\
53.25	0.01\\
53.26	0.01\\
53.27	0.01\\
53.28	0.01\\
53.29	0.01\\
53.3	0.01\\
53.31	0.01\\
53.32	0.01\\
53.33	0.01\\
53.34	0.01\\
53.35	0.01\\
53.36	0.01\\
53.37	0.01\\
53.38	0.01\\
53.39	0.01\\
53.4	0.01\\
53.41	0.01\\
53.42	0.01\\
53.43	0.01\\
53.44	0.01\\
53.45	0.01\\
53.46	0.01\\
53.47	0.01\\
53.48	0.01\\
53.49	0.01\\
53.5	0.01\\
53.51	0.01\\
53.52	0.01\\
53.53	0.01\\
53.54	0.01\\
53.55	0.01\\
53.56	0.01\\
53.57	0.01\\
53.58	0.01\\
53.59	0.01\\
53.6	0.01\\
53.61	0.01\\
53.62	0.01\\
53.63	0.01\\
53.64	0.01\\
53.65	0.01\\
53.66	0.01\\
53.67	0.01\\
53.68	0.01\\
53.69	0.01\\
53.7	0.01\\
53.71	0.01\\
53.72	0.01\\
53.73	0.01\\
53.74	0.01\\
53.75	0.01\\
53.76	0.01\\
53.77	0.01\\
53.78	0.01\\
53.79	0.01\\
53.8	0.01\\
53.81	0.01\\
53.82	0.01\\
53.83	0.01\\
53.84	0.01\\
53.85	0.01\\
53.86	0.01\\
53.87	0.01\\
53.88	0.01\\
53.89	0.01\\
53.9	0.01\\
53.91	0.01\\
53.92	0.01\\
53.93	0.01\\
53.94	0.01\\
53.95	0.01\\
53.96	0.01\\
53.97	0.01\\
53.98	0.01\\
53.99	0.01\\
54	0.01\\
54.01	0.01\\
54.02	0.01\\
54.03	0.01\\
54.04	0.01\\
54.05	0.01\\
54.06	0.01\\
54.07	0.01\\
54.08	0.01\\
54.09	0.01\\
54.1	0.01\\
54.11	0.01\\
54.12	0.01\\
54.13	0.01\\
54.14	0.01\\
54.15	0.01\\
54.16	0.01\\
54.17	0.01\\
54.18	0.01\\
54.19	0.01\\
54.2	0.01\\
54.21	0.01\\
54.22	0.01\\
54.23	0.01\\
54.24	0.01\\
54.25	0.01\\
54.26	0.01\\
54.27	0.01\\
54.28	0.01\\
54.29	0.01\\
54.3	0.01\\
54.31	0.01\\
54.32	0.01\\
54.33	0.01\\
54.34	0.01\\
54.35	0.01\\
54.36	0.01\\
54.37	0.01\\
54.38	0.01\\
54.39	0.01\\
54.4	0.01\\
54.41	0.01\\
54.42	0.01\\
54.43	0.01\\
54.44	0.01\\
54.45	0.01\\
54.46	0.01\\
54.47	0.01\\
54.48	0.01\\
54.49	0.01\\
54.5	0.01\\
54.51	0.01\\
54.52	0.01\\
54.53	0.01\\
54.54	0.01\\
54.55	0.01\\
54.56	0.01\\
54.57	0.01\\
54.58	0.01\\
54.59	0.01\\
54.6	0.01\\
54.61	0.01\\
54.62	0.01\\
54.63	0.01\\
54.64	0.01\\
54.65	0.01\\
54.66	0.01\\
54.67	0.01\\
54.68	0.01\\
54.69	0.01\\
54.7	0.01\\
54.71	0.01\\
54.72	0.01\\
54.73	0.01\\
54.74	0.01\\
54.75	0.01\\
54.76	0.01\\
54.77	0.01\\
54.78	0.01\\
54.79	0.01\\
54.8	0.01\\
54.81	0.01\\
54.82	0.01\\
54.83	0.01\\
54.84	0.01\\
54.85	0.01\\
54.86	0.01\\
54.87	0.01\\
54.88	0.01\\
54.89	0.01\\
54.9	0.01\\
54.91	0.01\\
54.92	0.01\\
54.93	0.01\\
54.94	0.01\\
54.95	0.01\\
54.96	0.01\\
54.97	0.01\\
54.98	0.01\\
54.99	0.01\\
55	0.01\\
55.01	0.01\\
55.02	0.01\\
55.03	0.01\\
55.04	0.01\\
55.05	0.01\\
55.06	0.01\\
55.07	0.01\\
55.08	0.01\\
55.09	0.01\\
55.1	0.01\\
55.11	0.01\\
55.12	0.01\\
55.13	0.01\\
55.14	0.01\\
55.15	0.01\\
55.16	0.01\\
55.17	0.01\\
55.18	0.01\\
55.19	0.01\\
55.2	0.01\\
55.21	0.01\\
55.22	0.01\\
55.23	0.01\\
55.24	0.01\\
55.25	0.01\\
55.26	0.01\\
55.27	0.01\\
55.28	0.01\\
55.29	0.01\\
55.3	0.01\\
55.31	0.01\\
55.32	0.01\\
55.33	0.01\\
55.34	0.01\\
55.35	0.01\\
55.36	0.01\\
55.37	0.01\\
55.38	0.01\\
55.39	0.01\\
55.4	0.01\\
55.41	0.01\\
55.42	0.01\\
55.43	0.01\\
55.44	0.01\\
55.45	0.01\\
55.46	0.01\\
55.47	0.01\\
55.48	0.01\\
55.49	0.01\\
55.5	0.01\\
55.51	0.01\\
55.52	0.01\\
55.53	0.01\\
55.54	0.01\\
55.55	0.01\\
55.56	0.01\\
55.57	0.01\\
55.58	0.01\\
55.59	0.01\\
55.6	0.01\\
55.61	0.01\\
55.62	0.01\\
55.63	0.01\\
55.64	0.01\\
55.65	0.01\\
55.66	0.01\\
55.67	0.01\\
55.68	0.01\\
55.69	0.01\\
55.7	0.01\\
55.71	0.01\\
55.72	0.01\\
55.73	0.01\\
55.74	0.01\\
55.75	0.01\\
55.76	0.01\\
55.77	0.01\\
55.78	0.01\\
55.79	0.01\\
55.8	0.01\\
55.81	0.01\\
55.82	0.01\\
55.83	0.01\\
55.84	0.01\\
55.85	0.01\\
55.86	0.01\\
55.87	0.01\\
55.88	0.01\\
55.89	0.01\\
55.9	0.01\\
55.91	0.01\\
55.92	0.01\\
55.93	0.01\\
55.94	0.01\\
55.95	0.01\\
55.96	0.01\\
55.97	0.01\\
55.98	0.01\\
55.99	0.01\\
56	0.01\\
56.01	0.01\\
56.02	0.01\\
56.03	0.01\\
56.04	0.01\\
56.05	0.01\\
56.06	0.01\\
56.07	0.01\\
56.08	0.01\\
56.09	0.01\\
56.1	0.01\\
56.11	0.01\\
56.12	0.01\\
56.13	0.01\\
56.14	0.01\\
56.15	0.01\\
56.16	0.01\\
56.17	0.01\\
56.18	0.01\\
56.19	0.01\\
56.2	0.01\\
56.21	0.01\\
56.22	0.01\\
56.23	0.01\\
56.24	0.01\\
56.25	0.01\\
56.26	0.01\\
56.27	0.01\\
56.28	0.01\\
56.29	0.01\\
56.3	0.01\\
56.31	0.01\\
56.32	0.01\\
56.33	0.01\\
56.34	0.01\\
56.35	0.01\\
56.36	0.01\\
56.37	0.01\\
56.38	0.01\\
56.39	0.01\\
56.4	0.01\\
56.41	0.01\\
56.42	0.01\\
56.43	0.01\\
56.44	0.01\\
56.45	0.01\\
56.46	0.01\\
56.47	0.01\\
56.48	0.01\\
56.49	0.01\\
56.5	0.01\\
56.51	0.01\\
56.52	0.01\\
56.53	0.01\\
56.54	0.01\\
56.55	0.01\\
56.56	0.01\\
56.57	0.01\\
56.58	0.01\\
56.59	0.01\\
56.6	0.01\\
56.61	0.01\\
56.62	0.01\\
56.63	0.01\\
56.64	0.01\\
56.65	0.01\\
56.66	0.01\\
56.67	0.01\\
56.68	0.01\\
56.69	0.01\\
56.7	0.01\\
56.71	0.01\\
56.72	0.01\\
56.73	0.01\\
56.74	0.01\\
56.75	0.01\\
56.76	0.01\\
56.77	0.01\\
56.78	0.01\\
56.79	0.01\\
56.8	0.01\\
56.81	0.01\\
56.82	0.01\\
56.83	0.01\\
56.84	0.01\\
56.85	0.01\\
56.86	0.01\\
56.87	0.01\\
56.88	0.01\\
56.89	0.01\\
56.9	0.01\\
56.91	0.01\\
56.92	0.01\\
56.93	0.01\\
56.94	0.01\\
56.95	0.01\\
56.96	0.01\\
56.97	0.01\\
56.98	0.01\\
56.99	0.01\\
57	0.01\\
57.01	0.01\\
57.02	0.01\\
57.03	0.01\\
57.04	0.01\\
57.05	0.01\\
57.06	0.01\\
57.07	0.01\\
57.08	0.01\\
57.09	0.01\\
57.1	0.01\\
57.11	0.01\\
57.12	0.01\\
57.13	0.01\\
57.14	0.01\\
57.15	0.01\\
57.16	0.01\\
57.17	0.01\\
57.18	0.01\\
57.19	0.01\\
57.2	0.01\\
57.21	0.01\\
57.22	0.01\\
57.23	0.01\\
57.24	0.01\\
57.25	0.01\\
57.26	0.01\\
57.27	0.01\\
57.28	0.01\\
57.29	0.01\\
57.3	0.01\\
57.31	0.01\\
57.32	0.01\\
57.33	0.01\\
57.34	0.01\\
57.35	0.01\\
57.36	0.01\\
57.37	0.01\\
57.38	0.01\\
57.39	0.01\\
57.4	0.01\\
57.41	0.01\\
57.42	0.01\\
57.43	0.01\\
57.44	0.01\\
57.45	0.01\\
57.46	0.01\\
57.47	0.01\\
57.48	0.01\\
57.49	0.01\\
57.5	0.01\\
57.51	0.01\\
57.52	0.01\\
57.53	0.01\\
57.54	0.01\\
57.55	0.01\\
57.56	0.01\\
57.57	0.01\\
57.58	0.01\\
57.59	0.01\\
57.6	0.01\\
57.61	0.01\\
57.62	0.01\\
57.63	0.01\\
57.64	0.01\\
57.65	0.01\\
57.66	0.01\\
57.67	0.01\\
57.68	0.01\\
57.69	0.01\\
57.7	0.01\\
57.71	0.01\\
57.72	0.01\\
57.73	0.01\\
57.74	0.01\\
57.75	0.01\\
57.76	0.01\\
57.77	0.01\\
57.78	0.01\\
57.79	0.01\\
57.8	0.01\\
57.81	0.01\\
57.82	0.01\\
57.83	0.01\\
57.84	0.01\\
57.85	0.01\\
57.86	0.01\\
57.87	0.01\\
57.88	0.01\\
57.89	0.01\\
57.9	0.01\\
57.91	0.01\\
57.92	0.01\\
57.93	0.01\\
57.94	0.01\\
57.95	0.01\\
57.96	0.01\\
57.97	0.01\\
57.98	0.01\\
57.99	0.01\\
58	0.01\\
58.01	0.01\\
58.02	0.01\\
58.03	0.01\\
58.04	0.01\\
58.05	0.01\\
58.06	0.01\\
58.07	0.01\\
58.08	0.01\\
58.09	0.01\\
58.1	0.01\\
58.11	0.01\\
58.12	0.01\\
58.13	0.01\\
58.14	0.01\\
58.15	0.01\\
58.16	0.01\\
58.17	0.01\\
58.18	0.01\\
58.19	0.01\\
58.2	0.01\\
58.21	0.01\\
58.22	0.01\\
58.23	0.01\\
58.24	0.01\\
58.25	0.01\\
58.26	0.01\\
58.27	0.01\\
58.28	0.01\\
58.29	0.01\\
58.3	0.01\\
58.31	0.01\\
58.32	0.01\\
58.33	0.01\\
58.34	0.01\\
58.35	0.01\\
58.36	0.01\\
58.37	0.01\\
58.38	0.01\\
58.39	0.01\\
58.4	0.01\\
58.41	0.01\\
58.42	0.01\\
58.43	0.01\\
58.44	0.01\\
58.45	0.01\\
58.46	0.01\\
58.47	0.01\\
58.48	0.01\\
58.49	0.01\\
58.5	0.01\\
58.51	0.01\\
58.52	0.01\\
58.53	0.01\\
58.54	0.01\\
58.55	0.01\\
58.56	0.01\\
58.57	0.01\\
58.58	0.01\\
58.59	0.01\\
58.6	0.01\\
58.61	0.01\\
58.62	0.01\\
58.63	0.01\\
58.64	0.01\\
58.65	0.01\\
58.66	0.01\\
58.67	0.01\\
58.68	0.01\\
58.69	0.01\\
58.7	0.01\\
58.71	0.01\\
58.72	0.01\\
58.73	0.01\\
58.74	0.01\\
58.75	0.01\\
58.76	0.01\\
58.77	0.01\\
58.78	0.01\\
58.79	0.01\\
58.8	0.01\\
58.81	0.01\\
58.82	0.01\\
58.83	0.01\\
58.84	0.01\\
58.85	0.01\\
58.86	0.01\\
58.87	0.01\\
58.88	0.01\\
58.89	0.01\\
58.9	0.01\\
58.91	0.01\\
58.92	0.01\\
58.93	0.01\\
58.94	0.01\\
58.95	0.01\\
58.96	0.01\\
58.97	0.01\\
58.98	0.01\\
58.99	0.01\\
59	0.01\\
59.01	0.01\\
59.02	0.01\\
59.03	0.01\\
59.04	0.01\\
59.05	0.01\\
59.06	0.01\\
59.07	0.01\\
59.08	0.01\\
59.09	0.01\\
59.1	0.01\\
59.11	0.01\\
59.12	0.01\\
59.13	0.01\\
59.14	0.01\\
59.15	0.01\\
59.16	0.01\\
59.17	0.01\\
59.18	0.01\\
59.19	0.01\\
59.2	0.01\\
59.21	0.01\\
59.22	0.01\\
59.23	0.01\\
59.24	0.01\\
59.25	0.01\\
59.26	0.01\\
59.27	0.01\\
59.28	0.01\\
59.29	0.01\\
59.3	0.01\\
59.31	0.01\\
59.32	0.01\\
59.33	0.01\\
59.34	0.01\\
59.35	0.01\\
59.36	0.01\\
59.37	0.01\\
59.38	0.01\\
59.39	0.01\\
59.4	0.01\\
59.41	0.01\\
59.42	0.01\\
59.43	0.01\\
59.44	0.01\\
59.45	0.01\\
59.46	0.01\\
59.47	0.01\\
59.48	0.01\\
59.49	0.01\\
59.5	0.01\\
59.51	0.01\\
59.52	0.01\\
59.53	0.01\\
59.54	0.01\\
59.55	0.01\\
59.56	0.01\\
59.57	0.01\\
59.58	0.01\\
59.59	0.01\\
59.6	0.01\\
59.61	0.01\\
59.62	0.01\\
59.63	0.01\\
59.64	0.01\\
59.65	0.01\\
59.66	0.01\\
59.67	0.01\\
59.68	0.01\\
59.69	0.01\\
59.7	0.01\\
59.71	0.01\\
59.72	0.01\\
59.73	0.01\\
59.74	0.01\\
59.75	0.01\\
59.76	0.01\\
59.77	0.01\\
59.78	0.01\\
59.79	0.01\\
59.8	0.01\\
59.81	0.01\\
59.82	0.01\\
59.83	0.01\\
59.84	0.01\\
59.85	0.01\\
59.86	0.01\\
59.87	0.01\\
59.88	0.01\\
59.89	0.01\\
59.9	0.01\\
59.91	0.01\\
59.92	0.01\\
59.93	0.01\\
59.94	0.01\\
59.95	0.01\\
59.96	0.01\\
59.97	0.01\\
59.98	0.01\\
59.99	0.01\\
60	0.01\\
60.01	0.01\\
60.02	0.01\\
60.03	0.01\\
60.04	0.01\\
60.05	0.01\\
60.06	0.01\\
60.07	0.01\\
60.08	0.01\\
60.09	0.01\\
60.1	0.01\\
60.11	0.01\\
60.12	0.01\\
60.13	0.01\\
60.14	0.01\\
60.15	0.01\\
60.16	0.01\\
60.17	0.01\\
60.18	0.01\\
60.19	0.01\\
60.2	0.01\\
60.21	0.01\\
60.22	0.01\\
60.23	0.01\\
60.24	0.01\\
60.25	0.01\\
60.26	0.01\\
60.27	0.01\\
60.28	0.01\\
60.29	0.01\\
60.3	0.01\\
60.31	0.01\\
60.32	0.01\\
60.33	0.01\\
60.34	0.01\\
60.35	0.01\\
60.36	0.01\\
60.37	0.01\\
60.38	0.01\\
60.39	0.01\\
60.4	0.01\\
60.41	0.01\\
60.42	0.01\\
60.43	0.01\\
60.44	0.01\\
60.45	0.01\\
60.46	0.01\\
60.47	0.01\\
60.48	0.01\\
60.49	0.01\\
60.5	0.01\\
60.51	0.01\\
60.52	0.01\\
60.53	0.01\\
60.54	0.01\\
60.55	0.01\\
60.56	0.01\\
60.57	0.01\\
60.58	0.01\\
60.59	0.01\\
60.6	0.01\\
60.61	0.01\\
60.62	0.01\\
60.63	0.01\\
60.64	0.01\\
60.65	0.01\\
60.66	0.01\\
60.67	0.01\\
60.68	0.01\\
60.69	0.01\\
60.7	0.01\\
60.71	0.01\\
60.72	0.01\\
60.73	0.01\\
60.74	0.01\\
60.75	0.01\\
60.76	0.01\\
60.77	0.01\\
60.78	0.01\\
60.79	0.01\\
60.8	0.01\\
60.81	0.01\\
60.82	0.01\\
60.83	0.01\\
60.84	0.01\\
60.85	0.01\\
60.86	0.01\\
60.87	0.01\\
60.88	0.01\\
60.89	0.01\\
60.9	0.01\\
60.91	0.01\\
60.92	0.01\\
60.93	0.01\\
60.94	0.01\\
60.95	0.01\\
60.96	0.01\\
60.97	0.01\\
60.98	0.01\\
60.99	0.01\\
61	0.01\\
61.01	0.01\\
61.02	0.01\\
61.03	0.01\\
61.04	0.01\\
61.05	0.01\\
61.06	0.01\\
61.07	0.01\\
61.08	0.01\\
61.09	0.01\\
61.1	0.01\\
61.11	0.01\\
61.12	0.01\\
61.13	0.01\\
61.14	0.01\\
61.15	0.01\\
61.16	0.01\\
61.17	0.01\\
61.18	0.01\\
61.19	0.01\\
61.2	0.01\\
61.21	0.01\\
61.22	0.01\\
61.23	0.01\\
61.24	0.01\\
61.25	0.01\\
61.26	0.01\\
61.27	0.01\\
61.28	0.01\\
61.29	0.01\\
61.3	0.01\\
61.31	0.01\\
61.32	0.01\\
61.33	0.01\\
61.34	0.01\\
61.35	0.01\\
61.36	0.01\\
61.37	0.01\\
61.38	0.01\\
61.39	0.01\\
61.4	0.01\\
61.41	0.01\\
61.42	0.01\\
61.43	0.01\\
61.44	0.01\\
61.45	0.01\\
61.46	0.01\\
61.47	0.01\\
61.48	0.01\\
61.49	0.01\\
61.5	0.01\\
61.51	0.01\\
61.52	0.01\\
61.53	0.01\\
61.54	0.01\\
61.55	0.01\\
61.56	0.01\\
61.57	0.01\\
61.58	0.01\\
61.59	0.01\\
61.6	0.01\\
61.61	0.01\\
61.62	0.01\\
61.63	0.01\\
61.64	0.01\\
61.65	0.01\\
61.66	0.01\\
61.67	0.01\\
61.68	0.01\\
61.69	0.01\\
61.7	0.01\\
61.71	0.01\\
61.72	0.01\\
61.73	0.01\\
61.74	0.01\\
61.75	0.01\\
61.76	0.01\\
61.77	0.01\\
61.78	0.01\\
61.79	0.01\\
61.8	0.01\\
61.81	0.01\\
61.82	0.01\\
61.83	0.01\\
61.84	0.01\\
61.85	0.01\\
61.86	0.01\\
61.87	0.01\\
61.88	0.01\\
61.89	0.01\\
61.9	0.01\\
61.91	0.01\\
61.92	0.01\\
61.93	0.01\\
61.94	0.01\\
61.95	0.01\\
61.96	0.01\\
61.97	0.01\\
61.98	0.01\\
61.99	0.01\\
62	0.01\\
62.01	0.01\\
62.02	0.01\\
62.03	0.01\\
62.04	0.01\\
62.05	0.01\\
62.06	0.01\\
62.07	0.01\\
62.08	0.01\\
62.09	0.01\\
62.1	0.01\\
62.11	0.01\\
62.12	0.01\\
62.13	0.01\\
62.14	0.01\\
62.15	0.01\\
62.16	0.01\\
62.17	0.01\\
62.18	0.01\\
62.19	0.01\\
62.2	0.01\\
62.21	0.01\\
62.22	0.01\\
62.23	0.01\\
62.24	0.01\\
62.25	0.01\\
62.26	0.01\\
62.27	0.01\\
62.28	0.01\\
62.29	0.01\\
62.3	0.01\\
62.31	0.01\\
62.32	0.01\\
62.33	0.01\\
62.34	0.01\\
62.35	0.01\\
62.36	0.01\\
62.37	0.01\\
62.38	0.01\\
62.39	0.01\\
62.4	0.01\\
62.41	0.01\\
62.42	0.01\\
62.43	0.01\\
62.44	0.01\\
62.45	0.01\\
62.46	0.01\\
62.47	0.01\\
62.48	0.01\\
62.49	0.01\\
62.5	0.01\\
62.51	0.01\\
62.52	0.01\\
62.53	0.01\\
62.54	0.01\\
62.55	0.01\\
62.56	0.01\\
62.57	0.01\\
62.58	0.01\\
62.59	0.01\\
62.6	0.01\\
62.61	0.01\\
62.62	0.01\\
62.63	0.01\\
62.64	0.01\\
62.65	0.01\\
62.66	0.01\\
62.67	0.01\\
62.68	0.01\\
62.69	0.01\\
62.7	0.01\\
62.71	0.01\\
62.72	0.01\\
62.73	0.01\\
62.74	0.01\\
62.75	0.01\\
62.76	0.01\\
62.77	0.01\\
62.78	0.01\\
62.79	0.01\\
62.8	0.01\\
62.81	0.01\\
62.82	0.01\\
62.83	0.01\\
62.84	0.01\\
62.85	0.01\\
62.86	0.01\\
62.87	0.01\\
62.88	0.01\\
62.89	0.01\\
62.9	0.01\\
62.91	0.01\\
62.92	0.01\\
62.93	0.01\\
62.94	0.01\\
62.95	0.01\\
62.96	0.01\\
62.97	0.01\\
62.98	0.01\\
62.99	0.01\\
63	0.01\\
63.01	0.01\\
63.02	0.01\\
63.03	0.01\\
63.04	0.01\\
63.05	0.01\\
63.06	0.01\\
63.07	0.01\\
63.08	0.01\\
63.09	0.01\\
63.1	0.01\\
63.11	0.01\\
63.12	0.01\\
63.13	0.01\\
63.14	0.01\\
63.15	0.01\\
63.16	0.01\\
63.17	0.01\\
63.18	0.01\\
63.19	0.01\\
63.2	0.01\\
63.21	0.01\\
63.22	0.01\\
63.23	0.01\\
63.24	0.01\\
63.25	0.01\\
63.26	0.01\\
63.27	0.01\\
63.28	0.01\\
63.29	0.01\\
63.3	0.01\\
63.31	0.01\\
63.32	0.01\\
63.33	0.01\\
63.34	0.01\\
63.35	0.01\\
63.36	0.01\\
63.37	0.01\\
63.38	0.01\\
63.39	0.01\\
63.4	0.01\\
63.41	0.01\\
63.42	0.01\\
63.43	0.01\\
63.44	0.01\\
63.45	0.01\\
63.46	0.01\\
63.47	0.01\\
63.48	0.01\\
63.49	0.01\\
63.5	0.01\\
63.51	0.01\\
63.52	0.01\\
63.53	0.01\\
63.54	0.01\\
63.55	0.01\\
63.56	0.01\\
63.57	0.01\\
63.58	0.01\\
63.59	0.01\\
63.6	0.01\\
63.61	0.01\\
63.62	0.01\\
63.63	0.01\\
63.64	0.01\\
63.65	0.01\\
63.66	0.01\\
63.67	0.01\\
63.68	0.01\\
63.69	0.01\\
63.7	0.01\\
63.71	0.01\\
63.72	0.01\\
63.73	0.01\\
63.74	0.01\\
63.75	0.01\\
63.76	0.01\\
63.77	0.01\\
63.78	0.01\\
63.79	0.01\\
63.8	0.01\\
63.81	0.01\\
63.82	0.01\\
63.83	0.01\\
63.84	0.01\\
63.85	0.01\\
63.86	0.01\\
63.87	0.01\\
63.88	0.01\\
63.89	0.01\\
63.9	0.01\\
63.91	0.01\\
63.92	0.01\\
63.93	0.01\\
63.94	0.01\\
63.95	0.01\\
63.96	0.01\\
63.97	0.01\\
63.98	0.01\\
63.99	0.01\\
64	0.01\\
64.01	0.01\\
64.02	0.01\\
64.03	0.01\\
64.04	0.01\\
64.05	0.01\\
64.06	0.01\\
64.07	0.01\\
64.08	0.01\\
64.09	0.01\\
64.1	0.01\\
64.11	0.01\\
64.12	0.01\\
64.13	0.01\\
64.14	0.01\\
64.15	0.01\\
64.16	0.01\\
64.17	0.01\\
64.18	0.01\\
64.19	0.01\\
64.2	0.01\\
64.21	0.01\\
64.22	0.01\\
64.23	0.01\\
64.24	0.01\\
64.25	0.01\\
64.26	0.01\\
64.27	0.01\\
64.28	0.01\\
64.29	0.01\\
64.3	0.01\\
64.31	0.01\\
64.32	0.01\\
64.33	0.01\\
64.34	0.01\\
64.35	0.01\\
64.36	0.01\\
64.37	0.01\\
64.38	0.01\\
64.39	0.01\\
64.4	0.01\\
64.41	0.01\\
64.42	0.01\\
64.43	0.01\\
64.44	0.01\\
64.45	0.01\\
64.46	0.01\\
64.47	0.01\\
64.48	0.01\\
64.49	0.01\\
64.5	0.01\\
64.51	0.01\\
64.52	0.01\\
64.53	0.01\\
64.54	0.01\\
64.55	0.01\\
64.56	0.01\\
64.57	0.01\\
64.58	0.01\\
64.59	0.01\\
64.6	0.01\\
64.61	0.01\\
64.62	0.01\\
64.63	0.01\\
64.64	0.01\\
64.65	0.01\\
64.66	0.01\\
64.67	0.01\\
64.68	0.01\\
64.69	0.01\\
64.7	0.01\\
64.71	0.01\\
64.72	0.01\\
64.73	0.01\\
64.74	0.01\\
64.75	0.01\\
64.76	0.01\\
64.77	0.01\\
64.78	0.01\\
64.79	0.01\\
64.8	0.01\\
64.81	0.01\\
64.82	0.01\\
64.83	0.01\\
64.84	0.01\\
64.85	0.01\\
64.86	0.01\\
64.87	0.01\\
64.88	0.01\\
64.89	0.01\\
64.9	0.01\\
64.91	0.01\\
64.92	0.01\\
64.93	0.01\\
64.94	0.01\\
64.95	0.01\\
64.96	0.01\\
64.97	0.01\\
64.98	0.01\\
64.99	0.01\\
65	0.01\\
65.01	0.01\\
65.02	0.01\\
65.03	0.01\\
65.04	0.01\\
65.05	0.01\\
65.06	0.01\\
65.07	0.01\\
65.08	0.01\\
65.09	0.01\\
65.1	0.01\\
65.11	0.01\\
65.12	0.01\\
65.13	0.01\\
65.14	0.01\\
65.15	0.01\\
65.16	0.01\\
65.17	0.01\\
65.18	0.01\\
65.19	0.01\\
65.2	0.01\\
65.21	0.01\\
65.22	0.01\\
65.23	0.01\\
65.24	0.01\\
65.25	0.01\\
65.26	0.01\\
65.27	0.01\\
65.28	0.01\\
65.29	0.01\\
65.3	0.01\\
65.31	0.01\\
65.32	0.01\\
65.33	0.01\\
65.34	0.01\\
65.35	0.01\\
65.36	0.01\\
65.37	0.01\\
65.38	0.01\\
65.39	0.01\\
65.4	0.01\\
65.41	0.01\\
65.42	0.01\\
65.43	0.01\\
65.44	0.01\\
65.45	0.01\\
65.46	0.01\\
65.47	0.01\\
65.48	0.01\\
65.49	0.01\\
65.5	0.01\\
65.51	0.01\\
65.52	0.01\\
65.53	0.01\\
65.54	0.01\\
65.55	0.01\\
65.56	0.01\\
65.57	0.01\\
65.58	0.01\\
65.59	0.01\\
65.6	0.01\\
65.61	0.01\\
65.62	0.01\\
65.63	0.01\\
65.64	0.01\\
65.65	0.01\\
65.66	0.01\\
65.67	0.01\\
65.68	0.01\\
65.69	0.01\\
65.7	0.01\\
65.71	0.01\\
65.72	0.01\\
65.73	0.01\\
65.74	0.01\\
65.75	0.01\\
65.76	0.01\\
65.77	0.01\\
65.78	0.01\\
65.79	0.01\\
65.8	0.01\\
65.81	0.01\\
65.82	0.01\\
65.83	0.01\\
65.84	0.01\\
65.85	0.01\\
65.86	0.01\\
65.87	0.01\\
65.88	0.01\\
65.89	0.01\\
65.9	0.01\\
65.91	0.01\\
65.92	0.01\\
65.93	0.01\\
65.94	0.01\\
65.95	0.01\\
65.96	0.01\\
65.97	0.01\\
65.98	0.01\\
65.99	0.01\\
66	0.01\\
66.01	0.01\\
66.02	0.01\\
66.03	0.01\\
66.04	0.01\\
66.05	0.01\\
66.06	0.01\\
66.07	0.01\\
66.08	0.01\\
66.09	0.01\\
66.1	0.01\\
66.11	0.01\\
66.12	0.01\\
66.13	0.01\\
66.14	0.01\\
66.15	0.01\\
66.16	0.01\\
66.17	0.01\\
66.18	0.01\\
66.19	0.01\\
66.2	0.01\\
66.21	0.01\\
66.22	0.01\\
66.23	0.01\\
66.24	0.01\\
66.25	0.01\\
66.26	0.01\\
66.27	0.01\\
66.28	0.01\\
66.29	0.01\\
66.3	0.01\\
66.31	0.01\\
66.32	0.01\\
66.33	0.01\\
66.34	0.01\\
66.35	0.01\\
66.36	0.01\\
66.37	0.01\\
66.38	0.01\\
66.39	0.01\\
66.4	0.01\\
66.41	0.01\\
66.42	0.01\\
66.43	0.01\\
66.44	0.01\\
66.45	0.01\\
66.46	0.01\\
66.47	0.01\\
66.48	0.01\\
66.49	0.01\\
66.5	0.01\\
66.51	0.01\\
66.52	0.01\\
66.53	0.01\\
66.54	0.01\\
66.55	0.01\\
66.56	0.01\\
66.57	0.01\\
66.58	0.01\\
66.59	0.01\\
66.6	0.01\\
66.61	0.01\\
66.62	0.01\\
66.63	0.01\\
66.64	0.01\\
66.65	0.01\\
66.66	0.01\\
66.67	0.01\\
66.68	0.01\\
66.69	0.01\\
66.7	0.01\\
66.71	0.01\\
66.72	0.01\\
66.73	0.01\\
66.74	0.01\\
66.75	0.01\\
66.76	0.01\\
66.77	0.01\\
66.78	0.01\\
66.79	0.01\\
66.8	0.01\\
66.81	0.01\\
66.82	0.01\\
66.83	0.01\\
66.84	0.01\\
66.85	0.01\\
66.86	0.01\\
66.87	0.01\\
66.88	0.01\\
66.89	0.01\\
66.9	0.01\\
66.91	0.01\\
66.92	0.01\\
66.93	0.01\\
66.94	0.01\\
66.95	0.01\\
66.96	0.01\\
66.97	0.01\\
66.98	0.01\\
66.99	0.01\\
67	0.01\\
67.01	0.01\\
67.02	0.01\\
67.03	0.01\\
67.04	0.01\\
67.05	0.01\\
67.06	0.01\\
67.07	0.01\\
67.08	0.01\\
67.09	0.01\\
67.1	0.01\\
67.11	0.01\\
67.12	0.01\\
67.13	0.01\\
67.14	0.01\\
67.15	0.01\\
67.16	0.01\\
67.17	0.01\\
67.18	0.01\\
67.19	0.01\\
67.2	0.01\\
67.21	0.01\\
67.22	0.01\\
67.23	0.01\\
67.24	0.01\\
67.25	0.01\\
67.26	0.01\\
67.27	0.01\\
67.28	0.01\\
67.29	0.01\\
67.3	0.01\\
67.31	0.01\\
67.32	0.01\\
67.33	0.01\\
67.34	0.01\\
67.35	0.01\\
67.36	0.01\\
67.37	0.01\\
67.38	0.01\\
67.39	0.01\\
67.4	0.01\\
67.41	0.01\\
67.42	0.01\\
67.43	0.01\\
67.44	0.01\\
67.45	0.01\\
67.46	0.01\\
67.47	0.01\\
67.48	0.01\\
67.49	0.01\\
67.5	0.01\\
67.51	0.01\\
67.52	0.01\\
67.53	0.01\\
67.54	0.01\\
67.55	0.01\\
67.56	0.01\\
67.57	0.01\\
67.58	0.01\\
67.59	0.01\\
67.6	0.01\\
67.61	0.01\\
67.62	0.01\\
67.63	0.01\\
67.64	0.01\\
67.65	0.01\\
67.66	0.01\\
67.67	0.01\\
67.68	0.01\\
67.69	0.01\\
67.7	0.01\\
67.71	0.01\\
67.72	0.01\\
67.73	0.01\\
67.74	0.01\\
67.75	0.01\\
67.76	0.01\\
67.77	0.01\\
67.78	0.01\\
67.79	0.01\\
67.8	0.01\\
67.81	0.01\\
67.82	0.01\\
67.83	0.01\\
67.84	0.01\\
67.85	0.01\\
67.86	0.01\\
67.87	0.01\\
67.88	0.01\\
67.89	0.01\\
67.9	0.01\\
67.91	0.01\\
67.92	0.01\\
67.93	0.01\\
67.94	0.01\\
67.95	0.01\\
67.96	0.01\\
67.97	0.01\\
67.98	0.01\\
67.99	0.01\\
68	0.01\\
68.01	0.01\\
68.02	0.01\\
68.03	0.01\\
68.04	0.01\\
68.05	0.01\\
68.06	0.01\\
68.07	0.01\\
68.08	0.01\\
68.09	0.01\\
68.1	0.01\\
68.11	0.01\\
68.12	0.01\\
68.13	0.01\\
68.14	0.01\\
68.15	0.01\\
68.16	0.01\\
68.17	0.01\\
68.18	0.01\\
68.19	0.01\\
68.2	0.01\\
68.21	0.01\\
68.22	0.01\\
68.23	0.01\\
68.24	0.01\\
68.25	0.01\\
68.26	0.01\\
68.27	0.01\\
68.28	0.01\\
68.29	0.01\\
68.3	0.01\\
68.31	0.01\\
68.32	0.01\\
68.33	0.01\\
68.34	0.01\\
68.35	0.01\\
68.36	0.01\\
68.37	0.01\\
68.38	0.01\\
68.39	0.01\\
68.4	0.01\\
68.41	0.01\\
68.42	0.01\\
68.43	0.01\\
68.44	0.01\\
68.45	0.01\\
68.46	0.01\\
68.47	0.01\\
68.48	0.01\\
68.49	0.01\\
68.5	0.01\\
68.51	0.01\\
68.52	0.01\\
68.53	0.01\\
68.54	0.01\\
68.55	0.01\\
68.56	0.01\\
68.57	0.01\\
68.58	0.01\\
68.59	0.01\\
68.6	0.01\\
68.61	0.01\\
68.62	0.01\\
68.63	0.01\\
68.64	0.01\\
68.65	0.01\\
68.66	0.01\\
68.67	0.01\\
68.68	0.01\\
68.69	0.01\\
68.7	0.01\\
68.71	0.01\\
68.72	0.01\\
68.73	0.01\\
68.74	0.01\\
68.75	0.01\\
68.76	0.01\\
68.77	0.01\\
68.78	0.01\\
68.79	0.01\\
68.8	0.01\\
68.81	0.01\\
68.82	0.01\\
68.83	0.01\\
68.84	0.01\\
68.85	0.01\\
68.86	0.01\\
68.87	0.01\\
68.88	0.01\\
68.89	0.01\\
68.9	0.01\\
68.91	0.01\\
68.92	0.01\\
68.93	0.01\\
68.94	0.01\\
68.95	0.01\\
68.96	0.01\\
68.97	0.01\\
68.98	0.01\\
68.99	0.01\\
69	0.01\\
69.01	0.01\\
69.02	0.01\\
69.03	0.01\\
69.04	0.01\\
69.05	0.01\\
69.06	0.01\\
69.07	0.01\\
69.08	0.01\\
69.09	0.01\\
69.1	0.01\\
69.11	0.01\\
69.12	0.01\\
69.13	0.01\\
69.14	0.01\\
69.15	0.01\\
69.16	0.01\\
69.17	0.01\\
69.18	0.01\\
69.19	0.01\\
69.2	0.01\\
69.21	0.01\\
69.22	0.01\\
69.23	0.01\\
69.24	0.01\\
69.25	0.01\\
69.26	0.01\\
69.27	0.01\\
69.28	0.01\\
69.29	0.01\\
69.3	0.01\\
69.31	0.01\\
69.32	0.01\\
69.33	0.01\\
69.34	0.01\\
69.35	0.01\\
69.36	0.01\\
69.37	0.01\\
69.38	0.01\\
69.39	0.01\\
69.4	0.01\\
69.41	0.01\\
69.42	0.01\\
69.43	0.01\\
69.44	0.01\\
69.45	0.01\\
69.46	0.01\\
69.47	0.01\\
69.48	0.01\\
69.49	0.01\\
69.5	0.01\\
69.51	0.01\\
69.52	0.01\\
69.53	0.01\\
69.54	0.01\\
69.55	0.01\\
69.56	0.01\\
69.57	0.01\\
69.58	0.01\\
69.59	0.01\\
69.6	0.01\\
69.61	0.01\\
69.62	0.01\\
69.63	0.01\\
69.64	0.01\\
69.65	0.01\\
69.66	0.01\\
69.67	0.01\\
69.68	0.01\\
69.69	0.01\\
69.7	0.01\\
69.71	0.01\\
69.72	0.01\\
69.73	0.01\\
69.74	0.01\\
69.75	0.01\\
69.76	0.01\\
69.77	0.01\\
69.78	0.01\\
69.79	0.01\\
69.8	0.01\\
69.81	0.01\\
69.82	0.01\\
69.83	0.01\\
69.84	0.01\\
69.85	0.01\\
69.86	0.01\\
69.87	0.01\\
69.88	0.01\\
69.89	0.01\\
69.9	0.01\\
69.91	0.01\\
69.92	0.01\\
69.93	0.01\\
69.94	0.01\\
69.95	0.01\\
69.96	0.01\\
69.97	0.01\\
69.98	0.01\\
69.99	0.01\\
70	0.01\\
70.01	0.01\\
70.02	0.01\\
70.03	0.01\\
70.04	0.01\\
70.05	0.01\\
70.06	0.01\\
70.07	0.01\\
70.08	0.01\\
70.09	0.01\\
70.1	0.01\\
70.11	0.01\\
70.12	0.01\\
70.13	0.01\\
70.14	0.01\\
70.15	0.01\\
70.16	0.01\\
70.17	0.01\\
70.18	0.01\\
70.19	0.01\\
70.2	0.01\\
70.21	0.01\\
70.22	0.01\\
70.23	0.01\\
70.24	0.01\\
70.25	0.01\\
70.26	0.01\\
70.27	0.01\\
70.28	0.01\\
70.29	0.01\\
70.3	0.01\\
70.31	0.01\\
70.32	0.01\\
70.33	0.01\\
70.34	0.01\\
70.35	0.01\\
70.36	0.01\\
70.37	0.01\\
70.38	0.01\\
70.39	0.01\\
70.4	0.01\\
70.41	0.01\\
70.42	0.01\\
70.43	0.01\\
70.44	0.01\\
70.45	0.01\\
70.46	0.01\\
70.47	0.01\\
70.48	0.01\\
70.49	0.01\\
70.5	0.01\\
70.51	0.01\\
70.52	0.01\\
70.53	0.01\\
70.54	0.01\\
70.55	0.01\\
70.56	0.01\\
70.57	0.01\\
70.58	0.01\\
70.59	0.01\\
70.6	0.01\\
70.61	0.01\\
70.62	0.01\\
70.63	0.01\\
70.64	0.01\\
70.65	0.01\\
70.66	0.01\\
70.67	0.01\\
70.68	0.01\\
70.69	0.01\\
70.7	0.01\\
70.71	0.01\\
70.72	0.01\\
70.73	0.01\\
70.74	0.01\\
70.75	0.01\\
70.76	0.01\\
70.77	0.01\\
70.78	0.01\\
70.79	0.01\\
70.8	0.01\\
70.81	0.01\\
70.82	0.01\\
70.83	0.01\\
70.84	0.01\\
70.85	0.01\\
70.86	0.01\\
70.87	0.01\\
70.88	0.01\\
70.89	0.01\\
70.9	0.01\\
70.91	0.01\\
70.92	0.01\\
70.93	0.01\\
70.94	0.01\\
70.95	0.01\\
70.96	0.01\\
70.97	0.01\\
70.98	0.01\\
70.99	0.01\\
71	0.01\\
71.01	0.01\\
71.02	0.01\\
71.03	0.01\\
71.04	0.01\\
71.05	0.01\\
71.06	0.01\\
71.07	0.01\\
71.08	0.01\\
71.09	0.01\\
71.1	0.01\\
71.11	0.01\\
71.12	0.01\\
71.13	0.01\\
71.14	0.01\\
71.15	0.01\\
71.16	0.01\\
71.17	0.01\\
71.18	0.01\\
71.19	0.01\\
71.2	0.01\\
71.21	0.01\\
71.22	0.01\\
71.23	0.01\\
71.24	0.01\\
71.25	0.01\\
71.26	0.01\\
71.27	0.01\\
71.28	0.01\\
71.29	0.01\\
71.3	0.01\\
71.31	0.01\\
71.32	0.01\\
71.33	0.01\\
71.34	0.01\\
71.35	0.01\\
71.36	0.01\\
71.37	0.01\\
71.38	0.01\\
71.39	0.01\\
71.4	0.01\\
71.41	0.01\\
71.42	0.01\\
71.43	0.01\\
71.44	0.01\\
71.45	0.01\\
71.46	0.01\\
71.47	0.01\\
71.48	0.01\\
71.49	0.01\\
71.5	0.01\\
71.51	0.01\\
71.52	0.01\\
71.53	0.01\\
71.54	0.01\\
71.55	0.01\\
71.56	0.01\\
71.57	0.01\\
71.58	0.01\\
71.59	0.01\\
71.6	0.01\\
71.61	0.01\\
71.62	0.01\\
71.63	0.01\\
71.64	0.01\\
71.65	0.01\\
71.66	0.01\\
71.67	0.01\\
71.68	0.01\\
71.69	0.01\\
71.7	0.01\\
71.71	0.01\\
71.72	0.01\\
71.73	0.01\\
71.74	0.01\\
71.75	0.01\\
71.76	0.01\\
71.77	0.01\\
71.78	0.01\\
71.79	0.01\\
71.8	0.01\\
71.81	0.01\\
71.82	0.01\\
71.83	0.01\\
71.84	0.01\\
71.85	0.01\\
71.86	0.01\\
71.87	0.01\\
71.88	0.01\\
71.89	0.01\\
71.9	0.01\\
71.91	0.01\\
71.92	0.01\\
71.93	0.01\\
71.94	0.01\\
71.95	0.01\\
71.96	0.01\\
71.97	0.01\\
71.98	0.01\\
71.99	0.01\\
72	0.01\\
72.01	0.01\\
72.02	0.01\\
72.03	0.01\\
72.04	0.01\\
72.05	0.01\\
72.06	0.01\\
72.07	0.01\\
72.08	0.01\\
72.09	0.01\\
72.1	0.01\\
72.11	0.01\\
72.12	0.01\\
72.13	0.01\\
72.14	0.01\\
72.15	0.01\\
72.16	0.01\\
72.17	0.01\\
72.18	0.01\\
72.19	0.01\\
72.2	0.01\\
72.21	0.01\\
72.22	0.01\\
72.23	0.01\\
72.24	0.01\\
72.25	0.01\\
72.26	0.01\\
72.27	0.01\\
72.28	0.01\\
72.29	0.01\\
72.3	0.01\\
72.31	0.01\\
72.32	0.01\\
72.33	0.01\\
72.34	0.01\\
72.35	0.01\\
72.36	0.01\\
72.37	0.01\\
72.38	0.01\\
72.39	0.01\\
72.4	0.01\\
72.41	0.01\\
72.42	0.01\\
72.43	0.01\\
72.44	0.01\\
72.45	0.01\\
72.46	0.01\\
72.47	0.01\\
72.48	0.01\\
72.49	0.01\\
72.5	0.01\\
72.51	0.01\\
72.52	0.01\\
72.53	0.01\\
72.54	0.01\\
72.55	0.01\\
72.56	0.01\\
72.57	0.01\\
72.58	0.01\\
72.59	0.01\\
72.6	0.01\\
72.61	0.01\\
72.62	0.01\\
72.63	0.01\\
72.64	0.01\\
72.65	0.01\\
72.66	0.01\\
72.67	0.01\\
72.68	0.01\\
72.69	0.01\\
72.7	0.01\\
72.71	0.01\\
72.72	0.01\\
72.73	0.01\\
72.74	0.01\\
72.75	0.01\\
72.76	0.01\\
72.77	0.01\\
72.78	0.01\\
72.79	0.01\\
72.8	0.01\\
72.81	0.01\\
72.82	0.01\\
72.83	0.01\\
72.84	0.01\\
72.85	0.01\\
72.86	0.01\\
72.87	0.01\\
72.88	0.01\\
72.89	0.01\\
72.9	0.01\\
72.91	0.01\\
72.92	0.01\\
72.93	0.01\\
72.94	0.01\\
72.95	0.01\\
72.96	0.01\\
72.97	0.01\\
72.98	0.01\\
72.99	0.01\\
73	0.01\\
73.01	0.01\\
73.02	0.01\\
73.03	0.01\\
73.04	0.01\\
73.05	0.01\\
73.06	0.01\\
73.07	0.01\\
73.08	0.01\\
73.09	0.01\\
73.1	0.01\\
73.11	0.01\\
73.12	0.01\\
73.13	0.01\\
73.14	0.01\\
73.15	0.01\\
73.16	0.01\\
73.17	0.01\\
73.18	0.01\\
73.19	0.01\\
73.2	0.01\\
73.21	0.01\\
73.22	0.01\\
73.23	0.01\\
73.24	0.01\\
73.25	0.01\\
73.26	0.01\\
73.27	0.01\\
73.28	0.01\\
73.29	0.01\\
73.3	0.01\\
73.31	0.01\\
73.32	0.01\\
73.33	0.01\\
73.34	0.01\\
73.35	0.01\\
73.36	0.01\\
73.37	0.01\\
73.38	0.01\\
73.39	0.01\\
73.4	0.01\\
73.41	0.01\\
73.42	0.01\\
73.43	0.01\\
73.44	0.01\\
73.45	0.01\\
73.46	0.01\\
73.47	0.01\\
73.48	0.01\\
73.49	0.01\\
73.5	0.01\\
73.51	0.01\\
73.52	0.01\\
73.53	0.01\\
73.54	0.01\\
73.55	0.01\\
73.56	0.01\\
73.57	0.01\\
73.58	0.01\\
73.59	0.01\\
73.6	0.01\\
73.61	0.01\\
73.62	0.01\\
73.63	0.01\\
73.64	0.01\\
73.65	0.01\\
73.66	0.01\\
73.67	0.01\\
73.68	0.01\\
73.69	0.01\\
73.7	0.01\\
73.71	0.01\\
73.72	0.01\\
73.73	0.01\\
73.74	0.01\\
73.75	0.01\\
73.76	0.01\\
73.77	0.01\\
73.78	0.01\\
73.79	0.01\\
73.8	0.01\\
73.81	0.01\\
73.82	0.01\\
73.83	0.01\\
73.84	0.01\\
73.85	0.01\\
73.86	0.01\\
73.87	0.01\\
73.88	0.01\\
73.89	0.01\\
73.9	0.01\\
73.91	0.01\\
73.92	0.01\\
73.93	0.01\\
73.94	0.01\\
73.95	0.01\\
73.96	0.01\\
73.97	0.01\\
73.98	0.01\\
73.99	0.01\\
74	0.01\\
74.01	0.01\\
74.02	0.01\\
74.03	0.01\\
74.04	0.01\\
74.05	0.01\\
74.06	0.01\\
74.07	0.01\\
74.08	0.01\\
74.09	0.01\\
74.1	0.01\\
74.11	0.01\\
74.12	0.01\\
74.13	0.01\\
74.14	0.01\\
74.15	0.01\\
74.16	0.01\\
74.17	0.01\\
74.18	0.01\\
74.19	0.01\\
74.2	0.01\\
74.21	0.01\\
74.22	0.01\\
74.23	0.01\\
74.24	0.01\\
74.25	0.01\\
74.26	0.01\\
74.27	0.01\\
74.28	0.01\\
74.29	0.01\\
74.3	0.01\\
74.31	0.01\\
74.32	0.01\\
74.33	0.01\\
74.34	0.01\\
74.35	0.01\\
74.36	0.01\\
74.37	0.01\\
74.38	0.01\\
74.39	0.01\\
74.4	0.01\\
74.41	0.01\\
74.42	0.01\\
74.43	0.01\\
74.44	0.01\\
74.45	0.01\\
74.46	0.01\\
74.47	0.01\\
74.48	0.01\\
74.49	0.01\\
74.5	0.01\\
74.51	0.01\\
74.52	0.01\\
74.53	0.01\\
74.54	0.01\\
74.55	0.01\\
74.56	0.01\\
74.57	0.01\\
74.58	0.01\\
74.59	0.01\\
74.6	0.01\\
74.61	0.01\\
74.62	0.01\\
74.63	0.01\\
74.64	0.01\\
74.65	0.01\\
74.66	0.01\\
74.67	0.01\\
74.68	0.01\\
74.69	0.01\\
74.7	0.01\\
74.71	0.01\\
74.72	0.01\\
74.73	0.01\\
74.74	0.01\\
74.75	0.01\\
74.76	0.01\\
74.77	0.01\\
74.78	0.01\\
74.79	0.01\\
74.8	0.01\\
74.81	0.01\\
74.82	0.01\\
74.83	0.01\\
74.84	0.01\\
74.85	0.01\\
74.86	0.01\\
74.87	0.01\\
74.88	0.01\\
74.89	0.01\\
74.9	0.01\\
74.91	0.01\\
74.92	0.01\\
74.93	0.01\\
74.94	0.01\\
74.95	0.01\\
74.96	0.01\\
74.97	0.01\\
74.98	0.01\\
74.99	0.01\\
75	0.01\\
75.01	0.01\\
75.02	0.01\\
75.03	0.01\\
75.04	0.01\\
75.05	0.01\\
75.06	0.01\\
75.07	0.01\\
75.08	0.01\\
75.09	0.01\\
75.1	0.01\\
75.11	0.01\\
75.12	0.01\\
75.13	0.01\\
75.14	0.01\\
75.15	0.01\\
75.16	0.01\\
75.17	0.01\\
75.18	0.01\\
75.19	0.01\\
75.2	0.01\\
75.21	0.01\\
75.22	0.01\\
75.23	0.01\\
75.24	0.01\\
75.25	0.01\\
75.26	0.01\\
75.27	0.01\\
75.28	0.01\\
75.29	0.01\\
75.3	0.01\\
75.31	0.01\\
75.32	0.01\\
75.33	0.01\\
75.34	0.01\\
75.35	0.01\\
75.36	0.01\\
75.37	0.01\\
75.38	0.01\\
75.39	0.01\\
75.4	0.01\\
75.41	0.01\\
75.42	0.01\\
75.43	0.01\\
75.44	0.01\\
75.45	0.01\\
75.46	0.01\\
75.47	0.01\\
75.48	0.01\\
75.49	0.01\\
75.5	0.01\\
75.51	0.01\\
75.52	0.01\\
75.53	0.01\\
75.54	0.01\\
75.55	0.01\\
75.56	0.01\\
75.57	0.01\\
75.58	0.01\\
75.59	0.01\\
75.6	0.01\\
75.61	0.01\\
75.62	0.01\\
75.63	0.01\\
75.64	0.01\\
75.65	0.01\\
75.66	0.01\\
75.67	0.01\\
75.68	0.01\\
75.69	0.01\\
75.7	0.01\\
75.71	0.01\\
75.72	0.01\\
75.73	0.01\\
75.74	0.01\\
75.75	0.01\\
75.76	0.01\\
75.77	0.01\\
75.78	0.01\\
75.79	0.01\\
75.8	0.01\\
75.81	0.01\\
75.82	0.01\\
75.83	0.01\\
75.84	0.01\\
75.85	0.01\\
75.86	0.01\\
75.87	0.01\\
75.88	0.01\\
75.89	0.01\\
75.9	0.01\\
75.91	0.01\\
75.92	0.01\\
75.93	0.01\\
75.94	0.01\\
75.95	0.01\\
75.96	0.01\\
75.97	0.01\\
75.98	0.01\\
75.99	0.01\\
76	0.01\\
76.01	0.01\\
76.02	0.01\\
76.03	0.01\\
76.04	0.01\\
76.05	0.01\\
76.06	0.01\\
76.07	0.01\\
76.08	0.01\\
76.09	0.01\\
76.1	0.01\\
76.11	0.01\\
76.12	0.01\\
76.13	0.01\\
76.14	0.01\\
76.15	0.01\\
76.16	0.01\\
76.17	0.01\\
76.18	0.01\\
76.19	0.01\\
76.2	0.01\\
76.21	0.01\\
76.22	0.01\\
76.23	0.01\\
76.24	0.01\\
76.25	0.01\\
76.26	0.01\\
76.27	0.01\\
76.28	0.01\\
76.29	0.01\\
76.3	0.01\\
76.31	0.01\\
76.32	0.01\\
76.33	0.01\\
76.34	0.01\\
76.35	0.01\\
76.36	0.01\\
76.37	0.01\\
76.38	0.01\\
76.39	0.01\\
76.4	0.01\\
76.41	0.01\\
76.42	0.01\\
76.43	0.01\\
76.44	0.01\\
76.45	0.01\\
76.46	0.01\\
76.47	0.01\\
76.48	0.01\\
76.49	0.01\\
76.5	0.01\\
76.51	0.01\\
76.52	0.01\\
76.53	0.01\\
76.54	0.01\\
76.55	0.01\\
76.56	0.01\\
76.57	0.01\\
76.58	0.01\\
76.59	0.01\\
76.6	0.01\\
76.61	0.01\\
76.62	0.01\\
76.63	0.01\\
76.64	0.01\\
76.65	0.01\\
76.66	0.01\\
76.67	0.01\\
76.68	0.01\\
76.69	0.01\\
76.7	0.01\\
76.71	0.01\\
76.72	0.01\\
76.73	0.01\\
76.74	0.01\\
76.75	0.01\\
76.76	0.01\\
76.77	0.01\\
76.78	0.01\\
76.79	0.01\\
76.8	0.01\\
76.81	0.01\\
76.82	0.01\\
76.83	0.01\\
76.84	0.01\\
76.85	0.01\\
76.86	0.01\\
76.87	0.01\\
76.88	0.01\\
76.89	0.01\\
76.9	0.01\\
76.91	0.01\\
76.92	0.01\\
76.93	0.01\\
76.94	0.01\\
76.95	0.01\\
76.96	0.01\\
76.97	0.01\\
76.98	0.01\\
76.99	0.01\\
77	0.01\\
77.01	0.01\\
77.02	0.01\\
77.03	0.01\\
77.04	0.01\\
77.05	0.01\\
77.06	0.01\\
77.07	0.01\\
77.08	0.01\\
77.09	0.01\\
77.1	0.01\\
77.11	0.01\\
77.12	0.01\\
77.13	0.01\\
77.14	0.01\\
77.15	0.01\\
77.16	0.01\\
77.17	0.01\\
77.18	0.01\\
77.19	0.01\\
77.2	0.01\\
77.21	0.01\\
77.22	0.01\\
77.23	0.01\\
77.24	0.01\\
77.25	0.01\\
77.26	0.01\\
77.27	0.01\\
77.28	0.01\\
77.29	0.01\\
77.3	0.01\\
77.31	0.01\\
77.32	0.01\\
77.33	0.01\\
77.34	0.01\\
77.35	0.01\\
77.36	0.01\\
77.37	0.01\\
77.38	0.01\\
77.39	0.01\\
77.4	0.01\\
77.41	0.01\\
77.42	0.01\\
77.43	0.01\\
77.44	0.01\\
77.45	0.01\\
77.46	0.01\\
77.47	0.01\\
77.48	0.01\\
77.49	0.01\\
77.5	0.01\\
77.51	0.01\\
77.52	0.01\\
77.53	0.01\\
77.54	0.01\\
77.55	0.01\\
77.56	0.01\\
77.57	0.01\\
77.58	0.01\\
77.59	0.01\\
77.6	0.01\\
77.61	0.01\\
77.62	0.01\\
77.63	0.01\\
77.64	0.01\\
77.65	0.01\\
77.66	0.01\\
77.67	0.01\\
77.68	0.01\\
77.69	0.01\\
77.7	0.01\\
77.71	0.01\\
77.72	0.01\\
77.73	0.01\\
77.74	0.01\\
77.75	0.01\\
77.76	0.01\\
77.77	0.01\\
77.78	0.01\\
77.79	0.01\\
77.8	0.01\\
77.81	0.01\\
77.82	0.01\\
77.83	0.01\\
77.84	0.01\\
77.85	0.01\\
77.86	0.01\\
77.87	0.01\\
77.88	0.01\\
77.89	0.01\\
77.9	0.01\\
77.91	0.01\\
77.92	0.01\\
77.93	0.01\\
77.94	0.01\\
77.95	0.01\\
77.96	0.01\\
77.97	0.01\\
77.98	0.01\\
77.99	0.01\\
78	0.01\\
78.01	0.01\\
78.02	0.01\\
78.03	0.01\\
78.04	0.01\\
78.05	0.01\\
78.06	0.01\\
78.07	0.01\\
78.08	0.01\\
78.09	0.01\\
78.1	0.01\\
78.11	0.01\\
78.12	0.01\\
78.13	0.01\\
78.14	0.01\\
78.15	0.01\\
78.16	0.01\\
78.17	0.01\\
78.18	0.01\\
78.19	0.01\\
78.2	0.01\\
78.21	0.01\\
78.22	0.01\\
78.23	0.01\\
78.24	0.01\\
78.25	0.01\\
78.26	0.01\\
78.27	0.01\\
78.28	0.01\\
78.29	0.01\\
78.3	0.01\\
78.31	0.01\\
78.32	0.01\\
78.33	0.01\\
78.34	0.01\\
78.35	0.01\\
78.36	0.01\\
78.37	0.01\\
78.38	0.01\\
78.39	0.01\\
78.4	0.01\\
78.41	0.01\\
78.42	0.01\\
78.43	0.01\\
78.44	0.01\\
78.45	0.01\\
78.46	0.01\\
78.47	0.01\\
78.48	0.01\\
78.49	0.01\\
78.5	0.01\\
78.51	0.01\\
78.52	0.01\\
78.53	0.01\\
78.54	0.01\\
78.55	0.01\\
78.56	0.01\\
78.57	0.01\\
78.58	0.01\\
78.59	0.01\\
78.6	0.01\\
78.61	0.01\\
78.62	0.01\\
78.63	0.01\\
78.64	0.01\\
78.65	0.01\\
78.66	0.01\\
78.67	0.01\\
78.68	0.01\\
78.69	0.01\\
78.7	0.01\\
78.71	0.01\\
78.72	0.01\\
78.73	0.01\\
78.74	0.01\\
78.75	0.01\\
78.76	0.01\\
78.77	0.01\\
78.78	0.01\\
78.79	0.01\\
78.8	0.01\\
78.81	0.01\\
78.82	0.01\\
78.83	0.01\\
78.84	0.01\\
78.85	0.01\\
78.86	0.01\\
78.87	0.01\\
78.88	0.01\\
78.89	0.01\\
78.9	0.01\\
78.91	0.01\\
78.92	0.01\\
78.93	0.01\\
78.94	0.01\\
78.95	0.01\\
78.96	0.01\\
78.97	0.01\\
78.98	0.01\\
78.99	0.01\\
79	0.01\\
79.01	0.01\\
79.02	0.01\\
79.03	0.01\\
79.04	0.01\\
79.05	0.01\\
79.06	0.01\\
79.07	0.01\\
79.08	0.01\\
79.09	0.01\\
79.1	0.01\\
79.11	0.01\\
79.12	0.01\\
79.13	0.01\\
79.14	0.01\\
79.15	0.01\\
79.16	0.01\\
79.17	0.01\\
79.18	0.01\\
79.19	0.01\\
79.2	0.01\\
79.21	0.01\\
79.22	0.01\\
79.23	0.01\\
79.24	0.01\\
79.25	0.01\\
79.26	0.01\\
79.27	0.01\\
79.28	0.01\\
79.29	0.01\\
79.3	0.01\\
79.31	0.01\\
79.32	0.01\\
79.33	0.01\\
79.34	0.01\\
79.35	0.01\\
79.36	0.01\\
79.37	0.01\\
79.38	0.01\\
79.39	0.01\\
79.4	0.01\\
79.41	0.01\\
79.42	0.01\\
79.43	0.01\\
79.44	0.01\\
79.45	0.01\\
79.46	0.01\\
79.47	0.01\\
79.48	0.01\\
79.49	0.01\\
79.5	0.01\\
79.51	0.01\\
79.52	0.01\\
79.53	0.01\\
79.54	0.01\\
79.55	0.01\\
79.56	0.01\\
79.57	0.01\\
79.58	0.01\\
79.59	0.01\\
79.6	0.01\\
79.61	0.01\\
79.62	0.01\\
79.63	0.01\\
79.64	0.01\\
79.65	0.01\\
79.66	0.01\\
79.67	0.01\\
79.68	0.01\\
79.69	0.01\\
79.7	0.01\\
79.71	0.01\\
79.72	0.01\\
79.73	0.01\\
79.74	0.01\\
79.75	0.01\\
79.76	0.01\\
79.77	0.01\\
79.78	0.01\\
79.79	0.01\\
79.8	0.01\\
79.81	0.01\\
79.82	0.01\\
79.83	0.01\\
79.84	0.01\\
79.85	0.01\\
79.86	0.01\\
79.87	0.01\\
79.88	0.01\\
79.89	0.01\\
79.9	0.01\\
79.91	0.01\\
79.92	0.01\\
79.93	0.01\\
79.94	0.01\\
79.95	0.01\\
79.96	0.01\\
79.97	0.01\\
79.98	0.01\\
79.99	0.01\\
80	0.01\\
80.01	0.01\\
};
\addplot [color=red,dashed]
  table[row sep=crcr]{%
80.01	0.01\\
80.02	0.01\\
80.03	0.01\\
80.04	0.01\\
80.05	0.01\\
80.06	0.01\\
80.07	0.01\\
80.08	0.01\\
80.09	0.01\\
80.1	0.01\\
80.11	0.01\\
80.12	0.01\\
80.13	0.01\\
80.14	0.01\\
80.15	0.01\\
80.16	0.01\\
80.17	0.01\\
80.18	0.01\\
80.19	0.01\\
80.2	0.01\\
80.21	0.01\\
80.22	0.01\\
80.23	0.01\\
80.24	0.01\\
80.25	0.01\\
80.26	0.01\\
80.27	0.01\\
80.28	0.01\\
80.29	0.01\\
80.3	0.01\\
80.31	0.01\\
80.32	0.01\\
80.33	0.01\\
80.34	0.01\\
80.35	0.01\\
80.36	0.01\\
80.37	0.01\\
80.38	0.01\\
80.39	0.01\\
80.4	0.01\\
80.41	0.01\\
80.42	0.01\\
80.43	0.01\\
80.44	0.01\\
80.45	0.01\\
80.46	0.01\\
80.47	0.01\\
80.48	0.01\\
80.49	0.01\\
80.5	0.01\\
80.51	0.01\\
80.52	0.01\\
80.53	0.01\\
80.54	0.01\\
80.55	0.01\\
80.56	0.01\\
80.57	0.01\\
80.58	0.01\\
80.59	0.01\\
80.6	0.01\\
80.61	0.01\\
80.62	0.01\\
80.63	0.01\\
80.64	0.01\\
80.65	0.01\\
80.66	0.01\\
80.67	0.01\\
80.68	0.01\\
80.69	0.01\\
80.7	0.01\\
80.71	0.01\\
80.72	0.01\\
80.73	0.01\\
80.74	0.01\\
80.75	0.01\\
80.76	0.01\\
80.77	0.01\\
80.78	0.01\\
80.79	0.01\\
80.8	0.01\\
80.81	0.01\\
80.82	0.01\\
80.83	0.01\\
80.84	0.01\\
80.85	0.01\\
80.86	0.01\\
80.87	0.01\\
80.88	0.01\\
80.89	0.01\\
80.9	0.01\\
80.91	0.01\\
80.92	0.01\\
80.93	0.01\\
80.94	0.01\\
80.95	0.01\\
80.96	0.01\\
80.97	0.01\\
80.98	0.01\\
80.99	0.01\\
81	0.01\\
81.01	0.01\\
81.02	0.01\\
81.03	0.01\\
81.04	0.01\\
81.05	0.01\\
81.06	0.01\\
81.07	0.01\\
81.08	0.01\\
81.09	0.01\\
81.1	0.01\\
81.11	0.01\\
81.12	0.01\\
81.13	0.01\\
81.14	0.01\\
81.15	0.01\\
81.16	0.01\\
81.17	0.01\\
81.18	0.01\\
81.19	0.01\\
81.2	0.01\\
81.21	0.01\\
81.22	0.01\\
81.23	0.01\\
81.24	0.01\\
81.25	0.01\\
81.26	0.01\\
81.27	0.01\\
81.28	0.01\\
81.29	0.01\\
81.3	0.01\\
81.31	0.01\\
81.32	0.01\\
81.33	0.01\\
81.34	0.01\\
81.35	0.01\\
81.36	0.01\\
81.37	0.01\\
81.38	0.01\\
81.39	0.01\\
81.4	0.01\\
81.41	0.01\\
81.42	0.01\\
81.43	0.01\\
81.44	0.01\\
81.45	0.01\\
81.46	0.01\\
81.47	0.01\\
81.48	0.01\\
81.49	0.01\\
81.5	0.01\\
81.51	0.01\\
81.52	0.01\\
81.53	0.01\\
81.54	0.01\\
81.55	0.01\\
81.56	0.01\\
81.57	0.01\\
81.58	0.01\\
81.59	0.01\\
81.6	0.01\\
81.61	0.01\\
81.62	0.01\\
81.63	0.01\\
81.64	0.01\\
81.65	0.01\\
81.66	0.01\\
81.67	0.01\\
81.68	0.01\\
81.69	0.01\\
81.7	0.01\\
81.71	0.01\\
81.72	0.01\\
81.73	0.01\\
81.74	0.01\\
81.75	0.01\\
81.76	0.01\\
81.77	0.01\\
81.78	0.01\\
81.79	0.01\\
81.8	0.01\\
81.81	0.01\\
81.82	0.01\\
81.83	0.01\\
81.84	0.01\\
81.85	0.01\\
81.86	0.01\\
81.87	0.01\\
81.88	0.01\\
81.89	0.01\\
81.9	0.01\\
81.91	0.01\\
81.92	0.01\\
81.93	0.01\\
81.94	0.01\\
81.95	0.01\\
81.96	0.01\\
81.97	0.01\\
81.98	0.01\\
81.99	0.01\\
82	0.01\\
82.01	0.01\\
82.02	0.01\\
82.03	0.01\\
82.04	0.01\\
82.05	0.01\\
82.06	0.01\\
82.07	0.01\\
82.08	0.01\\
82.09	0.01\\
82.1	0.01\\
82.11	0.01\\
82.12	0.01\\
82.13	0.01\\
82.14	0.01\\
82.15	0.01\\
82.16	0.01\\
82.17	0.01\\
82.18	0.01\\
82.19	0.01\\
82.2	0.01\\
82.21	0.01\\
82.22	0.01\\
82.23	0.01\\
82.24	0.01\\
82.25	0.01\\
82.26	0.01\\
82.27	0.01\\
82.28	0.01\\
82.29	0.01\\
82.3	0.01\\
82.31	0.01\\
82.32	0.01\\
82.33	0.01\\
82.34	0.01\\
82.35	0.01\\
82.36	0.01\\
82.37	0.01\\
82.38	0.01\\
82.39	0.01\\
82.4	0.01\\
82.41	0.01\\
82.42	0.01\\
82.43	0.01\\
82.44	0.01\\
82.45	0.01\\
82.46	0.01\\
82.47	0.01\\
82.48	0.01\\
82.49	0.01\\
82.5	0.01\\
82.51	0.01\\
82.52	0.01\\
82.53	0.01\\
82.54	0.01\\
82.55	0.01\\
82.56	0.01\\
82.57	0.01\\
82.58	0.01\\
82.59	0.01\\
82.6	0.01\\
82.61	0.01\\
82.62	0.01\\
82.63	0.01\\
82.64	0.01\\
82.65	0.01\\
82.66	0.01\\
82.67	0.01\\
82.68	0.01\\
82.69	0.01\\
82.7	0.01\\
82.71	0.01\\
82.72	0.01\\
82.73	0.01\\
82.74	0.01\\
82.75	0.01\\
82.76	0.01\\
82.77	0.01\\
82.78	0.01\\
82.79	0.01\\
82.8	0.01\\
82.81	0.01\\
82.82	0.01\\
82.83	0.01\\
82.84	0.01\\
82.85	0.01\\
82.86	0.01\\
82.87	0.01\\
82.88	0.01\\
82.89	0.01\\
82.9	0.01\\
82.91	0.01\\
82.92	0.01\\
82.93	0.01\\
82.94	0.01\\
82.95	0.01\\
82.96	0.01\\
82.97	0.01\\
82.98	0.01\\
82.99	0.01\\
83	0.01\\
83.01	0.01\\
83.02	0.01\\
83.03	0.01\\
83.04	0.01\\
83.05	0.01\\
83.06	0.01\\
83.07	0.01\\
83.08	0.01\\
83.09	0.01\\
83.1	0.01\\
83.11	0.01\\
83.12	0.01\\
83.13	0.01\\
83.14	0.01\\
83.15	0.01\\
83.16	0.01\\
83.17	0.01\\
83.18	0.01\\
83.19	0.01\\
83.2	0.01\\
83.21	0.01\\
83.22	0.01\\
83.23	0.01\\
83.24	0.01\\
83.25	0.01\\
83.26	0.01\\
83.27	0.01\\
83.28	0.01\\
83.29	0.01\\
83.3	0.01\\
83.31	0.01\\
83.32	0.01\\
83.33	0.01\\
83.34	0.01\\
83.35	0.01\\
83.36	0.01\\
83.37	0.01\\
83.38	0.01\\
83.39	0.01\\
83.4	0.01\\
83.41	0.01\\
83.42	0.01\\
83.43	0.01\\
83.44	0.01\\
83.45	0.01\\
83.46	0.01\\
83.47	0.01\\
83.48	0.01\\
83.49	0.01\\
83.5	0.01\\
83.51	0.01\\
83.52	0.01\\
83.53	0.01\\
83.54	0.01\\
83.55	0.01\\
83.56	0.01\\
83.57	0.01\\
83.58	0.01\\
83.59	0.01\\
83.6	0.01\\
83.61	0.01\\
83.62	0.01\\
83.63	0.01\\
83.64	0.01\\
83.65	0.01\\
83.66	0.01\\
83.67	0.01\\
83.68	0.01\\
83.69	0.01\\
83.7	0.01\\
83.71	0.01\\
83.72	0.01\\
83.73	0.01\\
83.74	0.01\\
83.75	0.01\\
83.76	0.01\\
83.77	0.01\\
83.78	0.01\\
83.79	0.01\\
83.8	0.01\\
83.81	0.01\\
83.82	0.01\\
83.83	0.01\\
83.84	0.01\\
83.85	0.01\\
83.86	0.01\\
83.87	0.01\\
83.88	0.01\\
83.89	0.01\\
83.9	0.01\\
83.91	0.01\\
83.92	0.01\\
83.93	0.01\\
83.94	0.01\\
83.95	0.01\\
83.96	0.01\\
83.97	0.01\\
83.98	0.01\\
83.99	0.01\\
84	0.01\\
84.01	0.01\\
84.02	0.01\\
84.03	0.01\\
84.04	0.01\\
84.05	0.01\\
84.06	0.01\\
84.07	0.01\\
84.08	0.01\\
84.09	0.01\\
84.1	0.01\\
84.11	0.01\\
84.12	0.01\\
84.13	0.01\\
84.14	0.01\\
84.15	0.01\\
84.16	0.01\\
84.17	0.01\\
84.18	0.01\\
84.19	0.01\\
84.2	0.01\\
84.21	0.01\\
84.22	0.01\\
84.23	0.01\\
84.24	0.01\\
84.25	0.01\\
84.26	0.01\\
84.27	0.01\\
84.28	0.01\\
84.29	0.01\\
84.3	0.01\\
84.31	0.01\\
84.32	0.01\\
84.33	0.01\\
84.34	0.01\\
84.35	0.01\\
84.36	0.01\\
84.37	0.01\\
84.38	0.01\\
84.39	0.01\\
84.4	0.01\\
84.41	0.01\\
84.42	0.01\\
84.43	0.01\\
84.44	0.01\\
84.45	0.01\\
84.46	0.01\\
84.47	0.01\\
84.48	0.01\\
84.49	0.01\\
84.5	0.01\\
84.51	0.01\\
84.52	0.01\\
84.53	0.01\\
84.54	0.01\\
84.55	0.01\\
84.56	0.01\\
84.57	0.01\\
84.58	0.01\\
84.59	0.01\\
84.6	0.01\\
84.61	0.01\\
84.62	0.01\\
84.63	0.01\\
84.64	0.01\\
84.65	0.01\\
84.66	0.01\\
84.67	0.01\\
84.68	0.01\\
84.69	0.01\\
84.7	0.01\\
84.71	0.01\\
84.72	0.01\\
84.73	0.01\\
84.74	0.01\\
84.75	0.01\\
84.76	0.01\\
84.77	0.01\\
84.78	0.01\\
84.79	0.01\\
84.8	0.01\\
84.81	0.01\\
84.82	0.01\\
84.83	0.01\\
84.84	0.01\\
84.85	0.01\\
84.86	0.01\\
84.87	0.01\\
84.88	0.01\\
84.89	0.01\\
84.9	0.01\\
84.91	0.01\\
84.92	0.01\\
84.93	0.01\\
84.94	0.01\\
84.95	0.01\\
84.96	0.01\\
84.97	0.01\\
84.98	0.01\\
84.99	0.01\\
85	0.01\\
85.01	0.01\\
85.02	0.01\\
85.03	0.01\\
85.04	0.01\\
85.05	0.01\\
85.06	0.01\\
85.07	0.01\\
85.08	0.01\\
85.09	0.01\\
85.1	0.01\\
85.11	0.01\\
85.12	0.01\\
85.13	0.01\\
85.14	0.01\\
85.15	0.01\\
85.16	0.01\\
85.17	0.01\\
85.18	0.01\\
85.19	0.01\\
85.2	0.01\\
85.21	0.01\\
85.22	0.01\\
85.23	0.01\\
85.24	0.01\\
85.25	0.01\\
85.26	0.01\\
85.27	0.01\\
85.28	0.01\\
85.29	0.01\\
85.3	0.01\\
85.31	0.01\\
85.32	0.01\\
85.33	0.01\\
85.34	0.01\\
85.35	0.01\\
85.36	0.01\\
85.37	0.01\\
85.38	0.01\\
85.39	0.01\\
85.4	0.01\\
85.41	0.01\\
85.42	0.01\\
85.43	0.01\\
85.44	0.01\\
85.45	0.01\\
85.46	0.01\\
85.47	0.01\\
85.48	0.01\\
85.49	0.01\\
85.5	0.01\\
85.51	0.01\\
85.52	0.01\\
85.53	0.01\\
85.54	0.01\\
85.55	0.01\\
85.56	0.01\\
85.57	0.01\\
85.58	0.01\\
85.59	0.01\\
85.6	0.01\\
85.61	0.01\\
85.62	0.01\\
85.63	0.01\\
85.64	0.01\\
85.65	0.01\\
85.66	0.01\\
85.67	0.01\\
85.68	0.01\\
85.69	0.01\\
85.7	0.01\\
85.71	0.01\\
85.72	0.01\\
85.73	0.01\\
85.74	0.01\\
85.75	0.01\\
85.76	0.01\\
85.77	0.01\\
85.78	0.01\\
85.79	0.01\\
85.8	0.01\\
85.81	0.01\\
85.82	0.01\\
85.83	0.01\\
85.84	0.01\\
85.85	0.01\\
85.86	0.01\\
85.87	0.01\\
85.88	0.01\\
85.89	0.01\\
85.9	0.01\\
85.91	0.01\\
85.92	0.01\\
85.93	0.01\\
85.94	0.01\\
85.95	0.01\\
85.96	0.01\\
85.97	0.01\\
85.98	0.01\\
85.99	0.01\\
86	0.01\\
86.01	0.01\\
86.02	0.01\\
86.03	0.01\\
86.04	0.01\\
86.05	0.01\\
86.06	0.01\\
86.07	0.01\\
86.08	0.01\\
86.09	0.01\\
86.1	0.01\\
86.11	0.01\\
86.12	0.01\\
86.13	0.01\\
86.14	0.01\\
86.15	0.01\\
86.16	0.01\\
86.17	0.01\\
86.18	0.01\\
86.19	0.01\\
86.2	0.01\\
86.21	0.01\\
86.22	0.01\\
86.23	0.01\\
86.24	0.01\\
86.25	0.01\\
86.26	0.01\\
86.27	0.01\\
86.28	0.01\\
86.29	0.01\\
86.3	0.01\\
86.31	0.01\\
86.32	0.01\\
86.33	0.01\\
86.34	0.01\\
86.35	0.01\\
86.36	0.01\\
86.37	0.01\\
86.38	0.01\\
86.39	0.01\\
86.4	0.01\\
86.41	0.01\\
86.42	0.01\\
86.43	0.01\\
86.44	0.01\\
86.45	0.01\\
86.46	0.01\\
86.47	0.01\\
86.48	0.01\\
86.49	0.01\\
86.5	0.01\\
86.51	0.01\\
86.52	0.01\\
86.53	0.01\\
86.54	0.01\\
86.55	0.01\\
86.56	0.01\\
86.57	0.01\\
86.58	0.01\\
86.59	0.01\\
86.6	0.01\\
86.61	0.01\\
86.62	0.01\\
86.63	0.01\\
86.64	0.01\\
86.65	0.01\\
86.66	0.01\\
86.67	0.01\\
86.68	0.01\\
86.69	0.01\\
86.7	0.01\\
86.71	0.01\\
86.72	0.01\\
86.73	0.01\\
86.74	0.01\\
86.75	0.01\\
86.76	0.01\\
86.77	0.01\\
86.78	0.01\\
86.79	0.01\\
86.8	0.01\\
86.81	0.01\\
86.82	0.01\\
86.83	0.01\\
86.84	0.01\\
86.85	0.01\\
86.86	0.01\\
86.87	0.01\\
86.88	0.01\\
86.89	0.01\\
86.9	0.01\\
86.91	0.01\\
86.92	0.01\\
86.93	0.01\\
86.94	0.01\\
86.95	0.01\\
86.96	0.01\\
86.97	0.01\\
86.98	0.01\\
86.99	0.01\\
87	0.01\\
87.01	0.01\\
87.02	0.01\\
87.03	0.01\\
87.04	0.01\\
87.05	0.01\\
87.06	0.01\\
87.07	0.01\\
87.08	0.01\\
87.09	0.01\\
87.1	0.01\\
87.11	0.01\\
87.12	0.01\\
87.13	0.01\\
87.14	0.01\\
87.15	0.01\\
87.16	0.01\\
87.17	0.01\\
87.18	0.01\\
87.19	0.01\\
87.2	0.01\\
87.21	0.01\\
87.22	0.01\\
87.23	0.01\\
87.24	0.01\\
87.25	0.01\\
87.26	0.01\\
87.27	0.01\\
87.28	0.01\\
87.29	0.01\\
87.3	0.01\\
87.31	0.01\\
87.32	0.01\\
87.33	0.01\\
87.34	0.01\\
87.35	0.01\\
87.36	0.01\\
87.37	0.01\\
87.38	0.01\\
87.39	0.01\\
87.4	0.01\\
87.41	0.01\\
87.42	0.01\\
87.43	0.01\\
87.44	0.01\\
87.45	0.01\\
87.46	0.01\\
87.47	0.01\\
87.48	0.01\\
87.49	0.01\\
87.5	0.01\\
87.51	0.01\\
87.52	0.01\\
87.53	0.01\\
87.54	0.01\\
87.55	0.01\\
87.56	0.01\\
87.57	0.01\\
87.58	0.01\\
87.59	0.01\\
87.6	0.01\\
87.61	0.01\\
87.62	0.01\\
87.63	0.01\\
87.64	0.01\\
87.65	0.01\\
87.66	0.01\\
87.67	0.01\\
87.68	0.01\\
87.69	0.01\\
87.7	0.01\\
87.71	0.01\\
87.72	0.01\\
87.73	0.01\\
87.74	0.01\\
87.75	0.01\\
87.76	0.01\\
87.77	0.01\\
87.78	0.01\\
87.79	0.01\\
87.8	0.01\\
87.81	0.01\\
87.82	0.01\\
87.83	0.01\\
87.84	0.01\\
87.85	0.01\\
87.86	0.01\\
87.87	0.01\\
87.88	0.01\\
87.89	0.01\\
87.9	0.01\\
87.91	0.01\\
87.92	0.01\\
87.93	0.01\\
87.94	0.01\\
87.95	0.01\\
87.96	0.01\\
87.97	0.01\\
87.98	0.01\\
87.99	0.01\\
88	0.01\\
88.01	0.01\\
88.02	0.01\\
88.03	0.01\\
88.04	0.01\\
88.05	0.01\\
88.06	0.01\\
88.07	0.01\\
88.08	0.01\\
88.09	0.01\\
88.1	0.01\\
88.11	0.01\\
88.12	0.01\\
88.13	0.01\\
88.14	0.01\\
88.15	0.01\\
88.16	0.01\\
88.17	0.01\\
88.18	0.01\\
88.19	0.01\\
88.2	0.01\\
88.21	0.01\\
88.22	0.01\\
88.23	0.01\\
88.24	0.01\\
88.25	0.01\\
88.26	0.01\\
88.27	0.01\\
88.28	0.01\\
88.29	0.01\\
88.3	0.01\\
88.31	0.01\\
88.32	0.01\\
88.33	0.01\\
88.34	0.01\\
88.35	0.01\\
88.36	0.01\\
88.37	0.01\\
88.38	0.01\\
88.39	0.01\\
88.4	0.01\\
88.41	0.01\\
88.42	0.01\\
88.43	0.01\\
88.44	0.01\\
88.45	0.01\\
88.46	0.01\\
88.47	0.01\\
88.48	0.01\\
88.49	0.01\\
88.5	0.01\\
88.51	0.01\\
88.52	0.01\\
88.53	0.01\\
88.54	0.01\\
88.55	0.01\\
88.56	0.01\\
88.57	0.01\\
88.58	0.01\\
88.59	0.01\\
88.6	0.01\\
88.61	0.01\\
88.62	0.01\\
88.63	0.01\\
88.64	0.01\\
88.65	0.01\\
88.66	0.01\\
88.67	0.01\\
88.68	0.01\\
88.69	0.01\\
88.7	0.01\\
88.71	0.01\\
88.72	0.01\\
88.73	0.01\\
88.74	0.01\\
88.75	0.01\\
88.76	0.01\\
88.77	0.01\\
88.78	0.01\\
88.79	0.01\\
88.8	0.01\\
88.81	0.01\\
88.82	0.01\\
88.83	0.01\\
88.84	0.01\\
88.85	0.01\\
88.86	0.01\\
88.87	0.01\\
88.88	0.01\\
88.89	0.01\\
88.9	0.01\\
88.91	0.01\\
88.92	0.01\\
88.93	0.01\\
88.94	0.01\\
88.95	0.01\\
88.96	0.01\\
88.97	0.01\\
88.98	0.01\\
88.99	0.01\\
89	0.01\\
89.01	0.01\\
89.02	0.01\\
89.03	0.01\\
89.04	0.01\\
89.05	0.01\\
89.06	0.01\\
89.07	0.01\\
89.08	0.01\\
89.09	0.01\\
89.1	0.01\\
89.11	0.01\\
89.12	0.01\\
89.13	0.01\\
89.14	0.01\\
89.15	0.01\\
89.16	0.01\\
89.17	0.01\\
89.18	0.01\\
89.19	0.01\\
89.2	0.01\\
89.21	0.01\\
89.22	0.01\\
89.23	0.01\\
89.24	0.01\\
89.25	0.01\\
89.26	0.01\\
89.27	0.01\\
89.28	0.01\\
89.29	0.01\\
89.3	0.01\\
89.31	0.01\\
89.32	0.01\\
89.33	0.01\\
89.34	0.01\\
89.35	0.01\\
89.36	0.01\\
89.37	0.01\\
89.38	0.01\\
89.39	0.01\\
89.4	0.01\\
89.41	0.01\\
89.42	0.01\\
89.43	0.01\\
89.44	0.01\\
89.45	0.01\\
89.46	0.01\\
89.47	0.01\\
89.48	0.01\\
89.49	0.01\\
89.5	0.01\\
89.51	0.01\\
89.52	0.01\\
89.53	0.01\\
89.54	0.01\\
89.55	0.01\\
89.56	0.01\\
89.57	0.01\\
89.58	0.01\\
89.59	0.01\\
89.6	0.01\\
89.61	0.01\\
89.62	0.01\\
89.63	0.01\\
89.64	0.01\\
89.65	0.01\\
89.66	0.01\\
89.67	0.01\\
89.68	0.01\\
89.69	0.01\\
89.7	0.01\\
89.71	0.01\\
89.72	0.01\\
89.73	0.01\\
89.74	0.01\\
89.75	0.01\\
89.76	0.01\\
89.77	0.01\\
89.78	0.01\\
89.79	0.01\\
89.8	0.01\\
89.81	0.01\\
89.82	0.01\\
89.83	0.01\\
89.84	0.01\\
89.85	0.01\\
89.86	0.01\\
89.87	0.01\\
89.88	0.01\\
89.89	0.01\\
89.9	0.01\\
89.91	0.01\\
89.92	0.01\\
89.93	0.01\\
89.94	0.01\\
89.95	0.01\\
89.96	0.01\\
89.97	0.01\\
89.98	0.01\\
89.99	0.01\\
90	0.01\\
90.01	0.01\\
90.02	0.01\\
90.03	0.01\\
90.04	0.01\\
90.05	0.01\\
90.06	0.01\\
90.07	0.01\\
90.08	0.01\\
90.09	0.01\\
90.1	0.01\\
90.11	0.01\\
90.12	0.01\\
90.13	0.01\\
90.14	0.01\\
90.15	0.01\\
90.16	0.01\\
90.17	0.01\\
90.18	0.01\\
90.19	0.01\\
90.2	0.01\\
90.21	0.01\\
90.22	0.01\\
90.23	0.01\\
90.24	0.01\\
90.25	0.01\\
90.26	0.01\\
90.27	0.01\\
90.28	0.01\\
90.29	0.01\\
90.3	0.01\\
90.31	0.01\\
90.32	0.01\\
90.33	0.01\\
90.34	0.01\\
90.35	0.01\\
90.36	0.01\\
90.37	0.01\\
90.38	0.01\\
90.39	0.01\\
90.4	0.01\\
90.41	0.01\\
90.42	0.01\\
90.43	0.01\\
90.44	0.01\\
90.45	0.01\\
90.46	0.01\\
90.47	0.01\\
90.48	0.01\\
90.49	0.01\\
90.5	0.01\\
90.51	0.01\\
90.52	0.01\\
90.53	0.01\\
90.54	0.01\\
90.55	0.01\\
90.56	0.01\\
90.57	0.01\\
90.58	0.01\\
90.59	0.01\\
90.6	0.01\\
90.61	0.01\\
90.62	0.01\\
90.63	0.01\\
90.64	0.01\\
90.65	0.01\\
90.66	0.01\\
90.67	0.01\\
90.68	0.01\\
90.69	0.01\\
90.7	0.01\\
90.71	0.01\\
90.72	0.01\\
90.73	0.01\\
90.74	0.01\\
90.75	0.01\\
90.76	0.01\\
90.77	0.01\\
90.78	0.01\\
90.79	0.01\\
90.8	0.01\\
90.81	0.01\\
90.82	0.01\\
90.83	0.01\\
90.84	0.01\\
90.85	0.01\\
90.86	0.01\\
90.87	0.01\\
90.88	0.01\\
90.89	0.01\\
90.9	0.01\\
90.91	0.01\\
90.92	0.01\\
90.93	0.01\\
90.94	0.01\\
90.95	0.01\\
90.96	0.01\\
90.97	0.01\\
90.98	0.01\\
90.99	0.01\\
91	0.01\\
91.01	0.01\\
91.02	0.01\\
91.03	0.01\\
91.04	0.01\\
91.05	0.01\\
91.06	0.01\\
91.07	0.01\\
91.08	0.01\\
91.09	0.01\\
91.1	0.01\\
91.11	0.01\\
91.12	0.01\\
91.13	0.01\\
91.14	0.01\\
91.15	0.01\\
91.16	0.01\\
91.17	0.01\\
91.18	0.01\\
91.19	0.01\\
91.2	0.01\\
91.21	0.01\\
91.22	0.01\\
91.23	0.01\\
91.24	0.01\\
91.25	0.01\\
91.26	0.01\\
91.27	0.01\\
91.28	0.01\\
91.29	0.01\\
91.3	0.01\\
91.31	0.01\\
91.32	0.01\\
91.33	0.01\\
91.34	0.01\\
91.35	0.01\\
91.36	0.01\\
91.37	0.01\\
91.38	0.01\\
91.39	0.01\\
91.4	0.01\\
91.41	0.01\\
91.42	0.01\\
91.43	0.01\\
91.44	0.01\\
91.45	0.01\\
91.46	0.01\\
91.47	0.01\\
91.48	0.01\\
91.49	0.01\\
91.5	0.01\\
91.51	0.01\\
91.52	0.01\\
91.53	0.01\\
91.54	0.01\\
91.55	0.01\\
91.56	0.01\\
91.57	0.01\\
91.58	0.01\\
91.59	0.01\\
91.6	0.01\\
91.61	0.01\\
91.62	0.01\\
91.63	0.01\\
91.64	0.01\\
91.65	0.01\\
91.66	0.01\\
91.67	0.01\\
91.68	0.01\\
91.69	0.01\\
91.7	0.01\\
91.71	0.01\\
91.72	0.01\\
91.73	0.01\\
91.74	0.01\\
91.75	0.01\\
91.76	0.01\\
91.77	0.01\\
91.78	0.01\\
91.79	0.01\\
91.8	0.01\\
91.81	0.01\\
91.82	0.01\\
91.83	0.01\\
91.84	0.01\\
91.85	0.01\\
91.86	0.01\\
91.87	0.01\\
91.88	0.01\\
91.89	0.01\\
91.9	0.01\\
91.91	0.01\\
91.92	0.01\\
91.93	0.01\\
91.94	0.01\\
91.95	0.01\\
91.96	0.01\\
91.97	0.01\\
91.98	0.01\\
91.99	0.01\\
92	0.01\\
92.01	0.01\\
92.02	0.01\\
92.03	0.01\\
92.04	0.01\\
92.05	0.01\\
92.06	0.01\\
92.07	0.01\\
92.08	0.01\\
92.09	0.01\\
92.1	0.01\\
92.11	0.01\\
92.12	0.01\\
92.13	0.01\\
92.14	0.01\\
92.15	0.01\\
92.16	0.01\\
92.17	0.01\\
92.18	0.01\\
92.19	0.01\\
92.2	0.01\\
92.21	0.01\\
92.22	0.01\\
92.23	0.01\\
92.24	0.01\\
92.25	0.01\\
92.26	0.01\\
92.27	0.01\\
92.28	0.01\\
92.29	0.01\\
92.3	0.01\\
92.31	0.01\\
92.32	0.01\\
92.33	0.01\\
92.34	0.01\\
92.35	0.01\\
92.36	0.01\\
92.37	0.01\\
92.38	0.01\\
92.39	0.01\\
92.4	0.01\\
92.41	0.01\\
92.42	0.01\\
92.43	0.01\\
92.44	0.01\\
92.45	0.01\\
92.46	0.01\\
92.47	0.01\\
92.48	0.01\\
92.49	0.01\\
92.5	0.01\\
92.51	0.01\\
92.52	0.01\\
92.53	0.01\\
92.54	0.01\\
92.55	0.01\\
92.56	0.01\\
92.57	0.01\\
92.58	0.01\\
92.59	0.01\\
92.6	0.01\\
92.61	0.01\\
92.62	0.01\\
92.63	0.01\\
92.64	0.01\\
92.65	0.01\\
92.66	0.01\\
92.67	0.01\\
92.68	0.01\\
92.69	0.01\\
92.7	0.01\\
92.71	0.01\\
92.72	0.01\\
92.73	0.01\\
92.74	0.01\\
92.75	0.01\\
92.76	0.01\\
92.77	0.01\\
92.78	0.01\\
92.79	0.01\\
92.8	0.01\\
92.81	0.01\\
92.82	0.01\\
92.83	0.01\\
92.84	0.01\\
92.85	0.01\\
92.86	0.01\\
92.87	0.01\\
92.88	0.01\\
92.89	0.01\\
92.9	0.01\\
92.91	0.01\\
92.92	0.01\\
92.93	0.01\\
92.94	0.01\\
92.95	0.01\\
92.96	0.01\\
92.97	0.01\\
92.98	0.01\\
92.99	0.01\\
93	0.01\\
93.01	0.01\\
93.02	0.01\\
93.03	0.01\\
93.04	0.01\\
93.05	0.01\\
93.06	0.01\\
93.07	0.01\\
93.08	0.01\\
93.09	0.01\\
93.1	0.01\\
93.11	0.01\\
93.12	0.01\\
93.13	0.01\\
93.14	0.01\\
93.15	0.01\\
93.16	0.01\\
93.17	0.01\\
93.18	0.01\\
93.19	0.01\\
93.2	0.01\\
93.21	0.01\\
93.22	0.01\\
93.23	0.01\\
93.24	0.01\\
93.25	0.01\\
93.26	0.01\\
93.27	0.01\\
93.28	0.01\\
93.29	0.01\\
93.3	0.01\\
93.31	0.01\\
93.32	0.01\\
93.33	0.01\\
93.34	0.01\\
93.35	0.01\\
93.36	0.01\\
93.37	0.01\\
93.38	0.01\\
93.39	0.01\\
93.4	0.01\\
93.41	0.01\\
93.42	0.01\\
93.43	0.01\\
93.44	0.01\\
93.45	0.01\\
93.46	0.01\\
93.47	0.01\\
93.48	0.01\\
93.49	0.01\\
93.5	0.01\\
93.51	0.01\\
93.52	0.01\\
93.53	0.01\\
93.54	0.01\\
93.55	0.01\\
93.56	0.01\\
93.57	0.01\\
93.58	0.01\\
93.59	0.01\\
93.6	0.01\\
93.61	0.01\\
93.62	0.01\\
93.63	0.01\\
93.64	0.01\\
93.65	0.01\\
93.66	0.01\\
93.67	0.01\\
93.68	0.01\\
93.69	0.01\\
93.7	0.01\\
93.71	0.01\\
93.72	0.01\\
93.73	0.01\\
93.74	0.01\\
93.75	0.01\\
93.76	0.01\\
93.77	0.01\\
93.78	0.01\\
93.79	0.01\\
93.8	0.01\\
93.81	0.01\\
93.82	0.01\\
93.83	0.01\\
93.84	0.01\\
93.85	0.01\\
93.86	0.01\\
93.87	0.01\\
93.88	0.01\\
93.89	0.01\\
93.9	0.01\\
93.91	0.01\\
93.92	0.01\\
93.93	0.01\\
93.94	0.01\\
93.95	0.01\\
93.96	0.01\\
93.97	0.01\\
93.98	0.01\\
93.99	0.01\\
94	0.01\\
94.01	0.01\\
94.02	0.01\\
94.03	0.01\\
94.04	0.01\\
94.05	0.01\\
94.06	0.01\\
94.07	0.01\\
94.08	0.01\\
94.09	0.01\\
94.1	0.01\\
94.11	0.01\\
94.12	0.01\\
94.13	0.01\\
94.14	0.01\\
94.15	0.01\\
94.16	0.01\\
94.17	0.01\\
94.18	0.01\\
94.19	0.01\\
94.2	0.01\\
94.21	0.01\\
94.22	0.01\\
94.23	0.01\\
94.24	0.01\\
94.25	0.01\\
94.26	0.01\\
94.27	0.01\\
94.28	0.01\\
94.29	0.01\\
94.3	0.01\\
94.31	0.01\\
94.32	0.01\\
94.33	0.01\\
94.34	0.01\\
94.35	0.01\\
94.36	0.01\\
94.37	0.01\\
94.38	0.01\\
94.39	0.01\\
94.4	0.01\\
94.41	0.01\\
94.42	0.01\\
94.43	0.01\\
94.44	0.01\\
94.45	0.01\\
94.46	0.01\\
94.47	0.01\\
94.48	0.01\\
94.49	0.01\\
94.5	0.01\\
94.51	0.01\\
94.52	0.01\\
94.53	0.01\\
94.54	0.01\\
94.55	0.01\\
94.56	0.01\\
94.57	0.01\\
94.58	0.01\\
94.59	0.01\\
94.6	0.01\\
94.61	0.01\\
94.62	0.01\\
94.63	0.01\\
94.64	0.01\\
94.65	0.01\\
94.66	0.01\\
94.67	0.01\\
94.68	0.01\\
94.69	0.01\\
94.7	0.01\\
94.71	0.01\\
94.72	0.01\\
94.73	0.01\\
94.74	0.01\\
94.75	0.01\\
94.76	0.01\\
94.77	0.01\\
94.78	0.01\\
94.79	0.01\\
94.8	0.01\\
94.81	0.01\\
94.82	0.01\\
94.83	0.01\\
94.84	0.01\\
94.85	0.01\\
94.86	0.01\\
94.87	0.01\\
94.88	0.01\\
94.89	0.01\\
94.9	0.01\\
94.91	0.01\\
94.92	0.01\\
94.93	0.01\\
94.94	0.01\\
94.95	0.01\\
94.96	0.01\\
94.97	0.01\\
94.98	0.01\\
94.99	0.01\\
95	0.01\\
95.01	0.01\\
95.02	0.01\\
95.03	0.01\\
95.04	0.01\\
95.05	0.01\\
95.06	0.01\\
95.07	0.01\\
95.08	0.01\\
95.09	0.01\\
95.1	0.01\\
95.11	0.01\\
95.12	0.01\\
95.13	0.01\\
95.14	0.01\\
95.15	0.01\\
95.16	0.01\\
95.17	0.01\\
95.18	0.01\\
95.19	0.01\\
95.2	0.01\\
95.21	0.01\\
95.22	0.01\\
95.23	0.01\\
95.24	0.01\\
95.25	0.01\\
95.26	0.01\\
95.27	0.01\\
95.28	0.01\\
95.29	0.01\\
95.3	0.01\\
95.31	0.01\\
95.32	0.01\\
95.33	0.01\\
95.34	0.01\\
95.35	0.01\\
95.36	0.01\\
95.37	0.01\\
95.38	0.01\\
95.39	0.01\\
95.4	0.01\\
95.41	0.01\\
95.42	0.01\\
95.43	0.01\\
95.44	0.01\\
95.45	0.01\\
95.46	0.01\\
95.47	0.01\\
95.48	0.01\\
95.49	0.01\\
95.5	0.01\\
95.51	0.01\\
95.52	0.01\\
95.53	0.01\\
95.54	0.01\\
95.55	0.01\\
95.56	0.01\\
95.57	0.01\\
95.58	0.01\\
95.59	0.01\\
95.6	0.01\\
95.61	0.01\\
95.62	0.01\\
95.63	0.01\\
95.64	0.01\\
95.65	0.01\\
95.66	0.01\\
95.67	0.01\\
95.68	0.01\\
95.69	0.01\\
95.7	0.01\\
95.71	0.01\\
95.72	0.01\\
95.73	0.01\\
95.74	0.01\\
95.75	0.01\\
95.76	0.01\\
95.77	0.01\\
95.78	0.01\\
95.79	0.01\\
95.8	0.01\\
95.81	0.01\\
95.82	0.01\\
95.83	0.01\\
95.84	0.01\\
95.85	0.01\\
95.86	0.01\\
95.87	0.01\\
95.88	0.01\\
95.89	0.01\\
95.9	0.01\\
95.91	0.01\\
95.92	0.01\\
95.93	0.01\\
95.94	0.01\\
95.95	0.01\\
95.96	0.01\\
95.97	0.01\\
95.98	0.01\\
95.99	0.01\\
96	0.01\\
96.01	0.01\\
96.02	0.01\\
96.03	0.01\\
96.04	0.01\\
96.05	0.01\\
96.06	0.01\\
96.07	0.01\\
96.08	0.01\\
96.09	0.01\\
96.1	0.01\\
96.11	0.01\\
96.12	0.01\\
96.13	0.01\\
96.14	0.01\\
96.15	0.01\\
96.16	0.01\\
96.17	0.01\\
96.18	0.01\\
96.19	0.01\\
96.2	0.01\\
96.21	0.01\\
96.22	0.01\\
96.23	0.01\\
96.24	0.01\\
96.25	0.01\\
96.26	0.01\\
96.27	0.01\\
96.28	0.01\\
96.29	0.01\\
96.3	0.01\\
96.31	0.01\\
96.32	0.01\\
96.33	0.01\\
96.34	0.01\\
96.35	0.01\\
96.36	0.01\\
96.37	0.01\\
96.38	0.01\\
96.39	0.01\\
96.4	0.01\\
96.41	0.01\\
96.42	0.01\\
96.43	0.01\\
96.44	0.01\\
96.45	0.01\\
96.46	0.01\\
96.47	0.01\\
96.48	0.01\\
96.49	0.01\\
96.5	0.01\\
96.51	0.01\\
96.52	0.01\\
96.53	0.01\\
96.54	0.01\\
96.55	0.01\\
96.56	0.01\\
96.57	0.01\\
96.58	0.01\\
96.59	0.01\\
96.6	0.01\\
96.61	0.01\\
96.62	0.01\\
96.63	0.01\\
96.64	0.01\\
96.65	0.01\\
96.66	0.01\\
96.67	0.01\\
96.68	0.01\\
96.69	0.01\\
96.7	0.01\\
96.71	0.01\\
96.72	0.01\\
96.73	0.01\\
96.74	0.01\\
96.75	0.01\\
96.76	0.01\\
96.77	0.01\\
96.78	0.01\\
96.79	0.01\\
96.8	0.01\\
96.81	0.01\\
96.82	0.01\\
96.83	0.01\\
96.84	0.01\\
96.85	0.01\\
96.86	0.01\\
96.87	0.01\\
96.88	0.01\\
96.89	0.01\\
96.9	0.01\\
96.91	0.01\\
96.92	0.01\\
96.93	0.01\\
96.94	0.01\\
96.95	0.01\\
96.96	0.01\\
96.97	0.01\\
96.98	0.01\\
96.99	0.01\\
97	0.01\\
97.01	0.01\\
97.02	0.01\\
97.03	0.01\\
97.04	0.01\\
97.05	0.01\\
97.06	0.01\\
97.07	0.01\\
97.08	0.01\\
97.09	0.01\\
97.1	0.01\\
97.11	0.01\\
97.12	0.01\\
97.13	0.01\\
97.14	0.01\\
97.15	0.01\\
97.16	0.01\\
97.17	0.01\\
97.18	0.01\\
97.19	0.01\\
97.2	0.01\\
97.21	0.01\\
97.22	0.01\\
97.23	0.01\\
97.24	0.01\\
97.25	0.01\\
97.26	0.01\\
97.27	0.01\\
97.28	0.01\\
97.29	0.01\\
97.3	0.01\\
97.31	0.01\\
97.32	0.01\\
97.33	0.01\\
97.34	0.01\\
97.35	0.01\\
97.36	0.01\\
97.37	0.01\\
97.38	0.01\\
97.39	0.01\\
97.4	0.01\\
97.41	0.01\\
97.42	0.01\\
97.43	0.01\\
97.44	0.01\\
97.45	0.01\\
97.46	0.01\\
97.47	0.01\\
97.48	0.01\\
97.49	0.01\\
97.5	0.01\\
97.51	0.01\\
97.52	0.01\\
97.53	0.01\\
97.54	0.01\\
97.55	0.01\\
97.56	0.01\\
97.57	0.01\\
97.58	0.01\\
97.59	0.01\\
97.6	0.01\\
97.61	0.01\\
97.62	0.01\\
97.63	0.01\\
97.64	0.01\\
97.65	0.01\\
97.66	0.01\\
97.67	0.01\\
97.68	0.01\\
97.69	0.01\\
97.7	0.01\\
97.71	0.01\\
97.72	0.01\\
97.73	0.01\\
97.74	0.01\\
97.75	0.01\\
97.76	0.01\\
97.77	0.01\\
97.78	0.01\\
97.79	0.01\\
97.8	0.01\\
97.81	0.01\\
97.82	0.01\\
97.83	0.01\\
97.84	0.01\\
97.85	0.01\\
97.86	0.01\\
97.87	0.01\\
97.88	0.01\\
97.89	0.01\\
97.9	0.01\\
97.91	0.01\\
97.92	0.01\\
97.93	0.01\\
97.94	0.01\\
97.95	0.01\\
97.96	0.01\\
97.97	0.01\\
97.98	0.01\\
97.99	0.01\\
98	0.01\\
98.01	0.01\\
98.02	0.01\\
98.03	0.01\\
98.04	0.01\\
98.05	0.01\\
98.06	0.01\\
98.07	0.01\\
98.08	0.01\\
98.09	0.01\\
98.1	0.01\\
98.11	0.01\\
98.12	0.01\\
98.13	0.01\\
98.14	0.01\\
98.15	0.01\\
98.16	0.01\\
98.17	0.01\\
98.18	0.01\\
98.19	0.01\\
98.2	0.01\\
98.21	0.01\\
98.22	0.01\\
98.23	0.01\\
98.24	0.01\\
98.25	0.01\\
98.26	0.01\\
98.27	0.01\\
98.28	0.01\\
98.29	0.01\\
98.3	0.01\\
98.31	0.01\\
98.32	0.01\\
98.33	0.00994490160815747\\
98.34	0.00988583813170677\\
98.35	0.00982635360008642\\
98.36	0.00976644407475283\\
98.37	0.00970610557633694\\
98.38	0.00964533408415264\\
98.39	0.00958412553569794\\
98.4	0.00952247582481165\\
98.41	0.00946038136888883\\
98.42	0.00939784258914091\\
98.43	0.00933485533808184\\
98.44	0.00927141542511888\\
98.45	0.00920751861602622\\
98.46	0.00914316063241053\\
98.47	0.00907833715116838\\
98.48	0.00901304380393543\\
98.49	0.00894727617652721\\
98.5	0.00888102980837118\\
98.51	0.00881430019193017\\
98.52	0.00874708277211669\\
98.53	0.00867937294569829\\
98.54	0.00861116606069348\\
98.55	0.00854245741575816\\
98.56	0.00847324225914466\\
98.57	0.00840351578530768\\
98.58	0.00833327313733897\\
98.59	0.00826250940631166\\
98.6	0.00819121963061387\\
98.61	0.00811939879527165\\
98.62	0.00804704183126073\\
98.63	0.0079790913431111\\
98.64	0.00794934915251536\\
98.65	0.00791935816584449\\
98.66	0.00788911610487732\\
98.67	0.00785862067107429\\
98.68	0.00782786954540883\\
98.69	0.00779686038819744\\
98.7	0.00776559091345907\\
98.71	0.0077340588325717\\
98.72	0.00770226183754817\\
98.73	0.00767019760088812\\
98.74	0.00763785640907247\\
98.75	0.00760522778276892\\
98.76	0.00757230906535413\\
98.77	0.00753909757596089\\
98.78	0.00750559060984833\\
98.79	0.00747178525814851\\
98.8	0.0074376747626892\\
98.81	0.00740325627133064\\
98.82	0.00736852690546005\\
98.83	0.00733348375974956\\
98.84	0.00729812390191215\\
98.85	0.00726244437245534\\
98.86	0.00722644218443286\\
98.87	0.00719011432319411\\
98.88	0.00715345774613148\\
98.89	0.0071164693824255\\
98.9	0.00707914613278774\\
98.91	0.0070414848692016\\
98.92	0.00700348243720732\\
98.93	0.00696513565347961\\
98.94	0.00692644130526317\\
98.95	0.0068873961501065\\
98.96	0.00684799691559364\\
98.97	0.00680824029907353\\
98.98	0.00676812296738722\\
98.99	0.0067276415565927\\
99	0.00668679267168765\\
99.01	0.00664557288632967\\
99.02	0.00660397874255443\\
99.03	0.00656200675049135\\
99.04	0.0065196533880771\\
99.05	0.00647691510076665\\
99.06	0.0064337883012421\\
99.07	0.00639026936911908\\
99.08	0.00634635465065084\\
99.09	0.00630204045843002\\
99.1	0.00625732307108794\\
99.11	0.00621219873299165\\
99.12	0.00616666365393848\\
99.13	0.00612071400884833\\
99.14	0.00607434593745342\\
99.15	0.0060275555439858\\
99.16	0.00598033889686234\\
99.17	0.0059326920283674\\
99.18	0.005884610934333\\
99.19	0.00583609157375363\\
99.2	0.0057871298684364\\
99.21	0.00573772170267285\\
99.22	0.00568786292290844\\
99.23	0.00563754933740963\\
99.24	0.00558677671592854\\
99.25	0.00553554078936529\\
99.26	0.00548383724942787\\
99.27	0.00543166174828981\\
99.28	0.00537900989824531\\
99.29	0.00532587727136213\\
99.3	0.00527225939913211\\
99.31	0.00521815177211941\\
99.32	0.00516354983960636\\
99.33	0.00510844900923711\\
99.34	0.00505284464665895\\
99.35	0.00499673207516135\\
99.36	0.00494010657531287\\
99.37	0.00488296338459576\\
99.38	0.00482529769703843\\
99.39	0.00476710466431692\\
99.4	0.00470837945686431\\
99.41	0.00464911720181623\\
99.42	0.00458931298265236\\
99.43	0.0045289618388362\\
99.44	0.00446805876545311\\
99.45	0.00440659871284654\\
99.46	0.00434457658625268\\
99.47	0.00428198724543351\\
99.48	0.00421882550430818\\
99.49	0.00415508613058311\\
99.5	0.00409076384538048\\
99.51	0.00402585332286552\\
99.52	0.00396034918987244\\
99.53	0.00389424602552924\\
99.54	0.00382753836088138\\
99.55	0.00376022067851448\\
99.56	0.00369228741217609\\
99.57	0.00362373294639665\\
99.58	0.00355455161610983\\
99.59	0.00348473770627216\\
99.6	0.00341428545148234\\
99.61	0.00334318903560021\\
99.62	0.00327144259136098\\
99.63	0.00319904019998451\\
99.64	0.00312597589079425\\
99.65	0.00305224364083623\\
99.66	0.00297783737449881\\
99.67	0.00290275096313283\\
99.68	0.00282697822467284\\
99.69	0.0027505129232592\\
99.7	0.00267334876886162\\
99.71	0.00259547941690411\\
99.72	0.00251689846789176\\
99.73	0.00243759947411416\\
99.74	0.00235757593859451\\
99.75	0.00227682130856416\\
99.76	0.00219532897510188\\
99.77	0.00211309227277645\\
99.78	0.00203010447929275\\
99.79	0.00194635881514191\\
99.8	0.00186184844325576\\
99.81	0.00177656646866601\\
99.82	0.00169050593816866\\
99.83	0.00160365983999406\\
99.84	0.00151602110348299\\
99.85	0.00142758259876953\\
99.86	0.00133833713647095\\
99.87	0.00124827746738539\\
99.88	0.00115739628219791\\
99.89	0.00106568621119549\\
99.9	0.000973139823991604\\
99.91	0.000879749629261267\\
99.92	0.000785508074487122\\
99.93	0.000690407545717435\\
99.94	0.000594440367336854\\
99.95	0.000497598801850791\\
99.96	0.000399875049684407\\
99.97	0.000301261248997162\\
99.98	0.000201749475514013\\
99.99	0.00010133174237437\\
100	0\\
};
\addlegendentry{$q=-2$};

\addplot [color=blue,dashed,forget plot]
  table[row sep=crcr]{%
0.01	0.01\\
0.02	0.01\\
0.03	0.01\\
0.04	0.01\\
0.05	0.01\\
0.06	0.01\\
0.07	0.01\\
0.08	0.01\\
0.09	0.01\\
0.1	0.01\\
0.11	0.01\\
0.12	0.01\\
0.13	0.01\\
0.14	0.01\\
0.15	0.01\\
0.16	0.01\\
0.17	0.01\\
0.18	0.01\\
0.19	0.01\\
0.2	0.01\\
0.21	0.01\\
0.22	0.01\\
0.23	0.01\\
0.24	0.01\\
0.25	0.01\\
0.26	0.01\\
0.27	0.01\\
0.28	0.01\\
0.29	0.01\\
0.3	0.01\\
0.31	0.01\\
0.32	0.01\\
0.33	0.01\\
0.34	0.01\\
0.35	0.01\\
0.36	0.01\\
0.37	0.01\\
0.38	0.01\\
0.39	0.01\\
0.4	0.01\\
0.41	0.01\\
0.42	0.01\\
0.43	0.01\\
0.44	0.01\\
0.45	0.01\\
0.46	0.01\\
0.47	0.01\\
0.48	0.01\\
0.49	0.01\\
0.5	0.01\\
0.51	0.01\\
0.52	0.01\\
0.53	0.01\\
0.54	0.01\\
0.55	0.01\\
0.56	0.01\\
0.57	0.01\\
0.58	0.01\\
0.59	0.01\\
0.6	0.01\\
0.61	0.01\\
0.62	0.01\\
0.63	0.01\\
0.64	0.01\\
0.65	0.01\\
0.66	0.01\\
0.67	0.01\\
0.68	0.01\\
0.69	0.01\\
0.7	0.01\\
0.71	0.01\\
0.72	0.01\\
0.73	0.01\\
0.74	0.01\\
0.75	0.01\\
0.76	0.01\\
0.77	0.01\\
0.78	0.01\\
0.79	0.01\\
0.8	0.01\\
0.81	0.01\\
0.82	0.01\\
0.83	0.01\\
0.84	0.01\\
0.85	0.01\\
0.86	0.01\\
0.87	0.01\\
0.88	0.01\\
0.89	0.01\\
0.9	0.01\\
0.91	0.01\\
0.92	0.01\\
0.93	0.01\\
0.94	0.01\\
0.95	0.01\\
0.96	0.01\\
0.97	0.01\\
0.98	0.01\\
0.99	0.01\\
1	0.01\\
1.01	0.01\\
1.02	0.01\\
1.03	0.01\\
1.04	0.01\\
1.05	0.01\\
1.06	0.01\\
1.07	0.01\\
1.08	0.01\\
1.09	0.01\\
1.1	0.01\\
1.11	0.01\\
1.12	0.01\\
1.13	0.01\\
1.14	0.01\\
1.15	0.01\\
1.16	0.01\\
1.17	0.01\\
1.18	0.01\\
1.19	0.01\\
1.2	0.01\\
1.21	0.01\\
1.22	0.01\\
1.23	0.01\\
1.24	0.01\\
1.25	0.01\\
1.26	0.01\\
1.27	0.01\\
1.28	0.01\\
1.29	0.01\\
1.3	0.01\\
1.31	0.01\\
1.32	0.01\\
1.33	0.01\\
1.34	0.01\\
1.35	0.01\\
1.36	0.01\\
1.37	0.01\\
1.38	0.01\\
1.39	0.01\\
1.4	0.01\\
1.41	0.01\\
1.42	0.01\\
1.43	0.01\\
1.44	0.01\\
1.45	0.01\\
1.46	0.01\\
1.47	0.01\\
1.48	0.01\\
1.49	0.01\\
1.5	0.01\\
1.51	0.01\\
1.52	0.01\\
1.53	0.01\\
1.54	0.01\\
1.55	0.01\\
1.56	0.01\\
1.57	0.01\\
1.58	0.01\\
1.59	0.01\\
1.6	0.01\\
1.61	0.01\\
1.62	0.01\\
1.63	0.01\\
1.64	0.01\\
1.65	0.01\\
1.66	0.01\\
1.67	0.01\\
1.68	0.01\\
1.69	0.01\\
1.7	0.01\\
1.71	0.01\\
1.72	0.01\\
1.73	0.01\\
1.74	0.01\\
1.75	0.01\\
1.76	0.01\\
1.77	0.01\\
1.78	0.01\\
1.79	0.01\\
1.8	0.01\\
1.81	0.01\\
1.82	0.01\\
1.83	0.01\\
1.84	0.01\\
1.85	0.01\\
1.86	0.01\\
1.87	0.01\\
1.88	0.01\\
1.89	0.01\\
1.9	0.01\\
1.91	0.01\\
1.92	0.01\\
1.93	0.01\\
1.94	0.01\\
1.95	0.01\\
1.96	0.01\\
1.97	0.01\\
1.98	0.01\\
1.99	0.01\\
2	0.01\\
2.01	0.01\\
2.02	0.01\\
2.03	0.01\\
2.04	0.01\\
2.05	0.01\\
2.06	0.01\\
2.07	0.01\\
2.08	0.01\\
2.09	0.01\\
2.1	0.01\\
2.11	0.01\\
2.12	0.01\\
2.13	0.01\\
2.14	0.01\\
2.15	0.01\\
2.16	0.01\\
2.17	0.01\\
2.18	0.01\\
2.19	0.01\\
2.2	0.01\\
2.21	0.01\\
2.22	0.01\\
2.23	0.01\\
2.24	0.01\\
2.25	0.01\\
2.26	0.01\\
2.27	0.01\\
2.28	0.01\\
2.29	0.01\\
2.3	0.01\\
2.31	0.01\\
2.32	0.01\\
2.33	0.01\\
2.34	0.01\\
2.35	0.01\\
2.36	0.01\\
2.37	0.01\\
2.38	0.01\\
2.39	0.01\\
2.4	0.01\\
2.41	0.01\\
2.42	0.01\\
2.43	0.01\\
2.44	0.01\\
2.45	0.01\\
2.46	0.01\\
2.47	0.01\\
2.48	0.01\\
2.49	0.01\\
2.5	0.01\\
2.51	0.01\\
2.52	0.01\\
2.53	0.01\\
2.54	0.01\\
2.55	0.01\\
2.56	0.01\\
2.57	0.01\\
2.58	0.01\\
2.59	0.01\\
2.6	0.01\\
2.61	0.01\\
2.62	0.01\\
2.63	0.01\\
2.64	0.01\\
2.65	0.01\\
2.66	0.01\\
2.67	0.01\\
2.68	0.01\\
2.69	0.01\\
2.7	0.01\\
2.71	0.01\\
2.72	0.01\\
2.73	0.01\\
2.74	0.01\\
2.75	0.01\\
2.76	0.01\\
2.77	0.01\\
2.78	0.01\\
2.79	0.01\\
2.8	0.01\\
2.81	0.01\\
2.82	0.01\\
2.83	0.01\\
2.84	0.01\\
2.85	0.01\\
2.86	0.01\\
2.87	0.01\\
2.88	0.01\\
2.89	0.01\\
2.9	0.01\\
2.91	0.01\\
2.92	0.01\\
2.93	0.01\\
2.94	0.01\\
2.95	0.01\\
2.96	0.01\\
2.97	0.01\\
2.98	0.01\\
2.99	0.01\\
3	0.01\\
3.01	0.01\\
3.02	0.01\\
3.03	0.01\\
3.04	0.01\\
3.05	0.01\\
3.06	0.01\\
3.07	0.01\\
3.08	0.01\\
3.09	0.01\\
3.1	0.01\\
3.11	0.01\\
3.12	0.01\\
3.13	0.01\\
3.14	0.01\\
3.15	0.01\\
3.16	0.01\\
3.17	0.01\\
3.18	0.01\\
3.19	0.01\\
3.2	0.01\\
3.21	0.01\\
3.22	0.01\\
3.23	0.01\\
3.24	0.01\\
3.25	0.01\\
3.26	0.01\\
3.27	0.01\\
3.28	0.01\\
3.29	0.01\\
3.3	0.01\\
3.31	0.01\\
3.32	0.01\\
3.33	0.01\\
3.34	0.01\\
3.35	0.01\\
3.36	0.01\\
3.37	0.01\\
3.38	0.01\\
3.39	0.01\\
3.4	0.01\\
3.41	0.01\\
3.42	0.01\\
3.43	0.01\\
3.44	0.01\\
3.45	0.01\\
3.46	0.01\\
3.47	0.01\\
3.48	0.01\\
3.49	0.01\\
3.5	0.01\\
3.51	0.01\\
3.52	0.01\\
3.53	0.01\\
3.54	0.01\\
3.55	0.01\\
3.56	0.01\\
3.57	0.01\\
3.58	0.01\\
3.59	0.01\\
3.6	0.01\\
3.61	0.01\\
3.62	0.01\\
3.63	0.01\\
3.64	0.01\\
3.65	0.01\\
3.66	0.01\\
3.67	0.01\\
3.68	0.01\\
3.69	0.01\\
3.7	0.01\\
3.71	0.01\\
3.72	0.01\\
3.73	0.01\\
3.74	0.01\\
3.75	0.01\\
3.76	0.01\\
3.77	0.01\\
3.78	0.01\\
3.79	0.01\\
3.8	0.01\\
3.81	0.01\\
3.82	0.01\\
3.83	0.01\\
3.84	0.01\\
3.85	0.01\\
3.86	0.01\\
3.87	0.01\\
3.88	0.01\\
3.89	0.01\\
3.9	0.01\\
3.91	0.01\\
3.92	0.01\\
3.93	0.01\\
3.94	0.01\\
3.95	0.01\\
3.96	0.01\\
3.97	0.01\\
3.98	0.01\\
3.99	0.01\\
4	0.01\\
4.01	0.01\\
4.02	0.01\\
4.03	0.01\\
4.04	0.01\\
4.05	0.01\\
4.06	0.01\\
4.07	0.01\\
4.08	0.01\\
4.09	0.01\\
4.1	0.01\\
4.11	0.01\\
4.12	0.01\\
4.13	0.01\\
4.14	0.01\\
4.15	0.01\\
4.16	0.01\\
4.17	0.01\\
4.18	0.01\\
4.19	0.01\\
4.2	0.01\\
4.21	0.01\\
4.22	0.01\\
4.23	0.01\\
4.24	0.01\\
4.25	0.01\\
4.26	0.01\\
4.27	0.01\\
4.28	0.01\\
4.29	0.01\\
4.3	0.01\\
4.31	0.01\\
4.32	0.01\\
4.33	0.01\\
4.34	0.01\\
4.35	0.01\\
4.36	0.01\\
4.37	0.01\\
4.38	0.01\\
4.39	0.01\\
4.4	0.01\\
4.41	0.01\\
4.42	0.01\\
4.43	0.01\\
4.44	0.01\\
4.45	0.01\\
4.46	0.01\\
4.47	0.01\\
4.48	0.01\\
4.49	0.01\\
4.5	0.01\\
4.51	0.01\\
4.52	0.01\\
4.53	0.01\\
4.54	0.01\\
4.55	0.01\\
4.56	0.01\\
4.57	0.01\\
4.58	0.01\\
4.59	0.01\\
4.6	0.01\\
4.61	0.01\\
4.62	0.01\\
4.63	0.01\\
4.64	0.01\\
4.65	0.01\\
4.66	0.01\\
4.67	0.01\\
4.68	0.01\\
4.69	0.01\\
4.7	0.01\\
4.71	0.01\\
4.72	0.01\\
4.73	0.01\\
4.74	0.01\\
4.75	0.01\\
4.76	0.01\\
4.77	0.01\\
4.78	0.01\\
4.79	0.01\\
4.8	0.01\\
4.81	0.01\\
4.82	0.01\\
4.83	0.01\\
4.84	0.01\\
4.85	0.01\\
4.86	0.01\\
4.87	0.01\\
4.88	0.01\\
4.89	0.01\\
4.9	0.01\\
4.91	0.01\\
4.92	0.01\\
4.93	0.01\\
4.94	0.01\\
4.95	0.01\\
4.96	0.01\\
4.97	0.01\\
4.98	0.01\\
4.99	0.01\\
5	0.01\\
5.01	0.01\\
5.02	0.01\\
5.03	0.01\\
5.04	0.01\\
5.05	0.01\\
5.06	0.01\\
5.07	0.01\\
5.08	0.01\\
5.09	0.01\\
5.1	0.01\\
5.11	0.01\\
5.12	0.01\\
5.13	0.01\\
5.14	0.01\\
5.15	0.01\\
5.16	0.01\\
5.17	0.01\\
5.18	0.01\\
5.19	0.01\\
5.2	0.01\\
5.21	0.01\\
5.22	0.01\\
5.23	0.01\\
5.24	0.01\\
5.25	0.01\\
5.26	0.01\\
5.27	0.01\\
5.28	0.01\\
5.29	0.01\\
5.3	0.01\\
5.31	0.01\\
5.32	0.01\\
5.33	0.01\\
5.34	0.01\\
5.35	0.01\\
5.36	0.01\\
5.37	0.01\\
5.38	0.01\\
5.39	0.01\\
5.4	0.01\\
5.41	0.01\\
5.42	0.01\\
5.43	0.01\\
5.44	0.01\\
5.45	0.01\\
5.46	0.01\\
5.47	0.01\\
5.48	0.01\\
5.49	0.01\\
5.5	0.01\\
5.51	0.01\\
5.52	0.01\\
5.53	0.01\\
5.54	0.01\\
5.55	0.01\\
5.56	0.01\\
5.57	0.01\\
5.58	0.01\\
5.59	0.01\\
5.6	0.01\\
5.61	0.01\\
5.62	0.01\\
5.63	0.01\\
5.64	0.01\\
5.65	0.01\\
5.66	0.01\\
5.67	0.01\\
5.68	0.01\\
5.69	0.01\\
5.7	0.01\\
5.71	0.01\\
5.72	0.01\\
5.73	0.01\\
5.74	0.01\\
5.75	0.01\\
5.76	0.01\\
5.77	0.01\\
5.78	0.01\\
5.79	0.01\\
5.8	0.01\\
5.81	0.01\\
5.82	0.01\\
5.83	0.01\\
5.84	0.01\\
5.85	0.01\\
5.86	0.01\\
5.87	0.01\\
5.88	0.01\\
5.89	0.01\\
5.9	0.01\\
5.91	0.01\\
5.92	0.01\\
5.93	0.01\\
5.94	0.01\\
5.95	0.01\\
5.96	0.01\\
5.97	0.01\\
5.98	0.01\\
5.99	0.01\\
6	0.01\\
6.01	0.01\\
6.02	0.01\\
6.03	0.01\\
6.04	0.01\\
6.05	0.01\\
6.06	0.01\\
6.07	0.01\\
6.08	0.01\\
6.09	0.01\\
6.1	0.01\\
6.11	0.01\\
6.12	0.01\\
6.13	0.01\\
6.14	0.01\\
6.15	0.01\\
6.16	0.01\\
6.17	0.01\\
6.18	0.01\\
6.19	0.01\\
6.2	0.01\\
6.21	0.01\\
6.22	0.01\\
6.23	0.01\\
6.24	0.01\\
6.25	0.01\\
6.26	0.01\\
6.27	0.01\\
6.28	0.01\\
6.29	0.01\\
6.3	0.01\\
6.31	0.01\\
6.32	0.01\\
6.33	0.01\\
6.34	0.01\\
6.35	0.01\\
6.36	0.01\\
6.37	0.01\\
6.38	0.01\\
6.39	0.01\\
6.4	0.01\\
6.41	0.01\\
6.42	0.01\\
6.43	0.01\\
6.44	0.01\\
6.45	0.01\\
6.46	0.01\\
6.47	0.01\\
6.48	0.01\\
6.49	0.01\\
6.5	0.01\\
6.51	0.01\\
6.52	0.01\\
6.53	0.01\\
6.54	0.01\\
6.55	0.01\\
6.56	0.01\\
6.57	0.01\\
6.58	0.01\\
6.59	0.01\\
6.6	0.01\\
6.61	0.01\\
6.62	0.01\\
6.63	0.01\\
6.64	0.01\\
6.65	0.01\\
6.66	0.01\\
6.67	0.01\\
6.68	0.01\\
6.69	0.01\\
6.7	0.01\\
6.71	0.01\\
6.72	0.01\\
6.73	0.01\\
6.74	0.01\\
6.75	0.01\\
6.76	0.01\\
6.77	0.01\\
6.78	0.01\\
6.79	0.01\\
6.8	0.01\\
6.81	0.01\\
6.82	0.01\\
6.83	0.01\\
6.84	0.01\\
6.85	0.01\\
6.86	0.01\\
6.87	0.01\\
6.88	0.01\\
6.89	0.01\\
6.9	0.01\\
6.91	0.01\\
6.92	0.01\\
6.93	0.01\\
6.94	0.01\\
6.95	0.01\\
6.96	0.01\\
6.97	0.01\\
6.98	0.01\\
6.99	0.01\\
7	0.01\\
7.01	0.01\\
7.02	0.01\\
7.03	0.01\\
7.04	0.01\\
7.05	0.01\\
7.06	0.01\\
7.07	0.01\\
7.08	0.01\\
7.09	0.01\\
7.1	0.01\\
7.11	0.01\\
7.12	0.01\\
7.13	0.01\\
7.14	0.01\\
7.15	0.01\\
7.16	0.01\\
7.17	0.01\\
7.18	0.01\\
7.19	0.01\\
7.2	0.01\\
7.21	0.01\\
7.22	0.01\\
7.23	0.01\\
7.24	0.01\\
7.25	0.01\\
7.26	0.01\\
7.27	0.01\\
7.28	0.01\\
7.29	0.01\\
7.3	0.01\\
7.31	0.01\\
7.32	0.01\\
7.33	0.01\\
7.34	0.01\\
7.35	0.01\\
7.36	0.01\\
7.37	0.01\\
7.38	0.01\\
7.39	0.01\\
7.4	0.01\\
7.41	0.01\\
7.42	0.01\\
7.43	0.01\\
7.44	0.01\\
7.45	0.01\\
7.46	0.01\\
7.47	0.01\\
7.48	0.01\\
7.49	0.01\\
7.5	0.01\\
7.51	0.01\\
7.52	0.01\\
7.53	0.01\\
7.54	0.01\\
7.55	0.01\\
7.56	0.01\\
7.57	0.01\\
7.58	0.01\\
7.59	0.01\\
7.6	0.01\\
7.61	0.01\\
7.62	0.01\\
7.63	0.01\\
7.64	0.01\\
7.65	0.01\\
7.66	0.01\\
7.67	0.01\\
7.68	0.01\\
7.69	0.01\\
7.7	0.01\\
7.71	0.01\\
7.72	0.01\\
7.73	0.01\\
7.74	0.01\\
7.75	0.01\\
7.76	0.01\\
7.77	0.01\\
7.78	0.01\\
7.79	0.01\\
7.8	0.01\\
7.81	0.01\\
7.82	0.01\\
7.83	0.01\\
7.84	0.01\\
7.85	0.01\\
7.86	0.01\\
7.87	0.01\\
7.88	0.01\\
7.89	0.01\\
7.9	0.01\\
7.91	0.01\\
7.92	0.01\\
7.93	0.01\\
7.94	0.01\\
7.95	0.01\\
7.96	0.01\\
7.97	0.01\\
7.98	0.01\\
7.99	0.01\\
8	0.01\\
8.01	0.01\\
8.02	0.01\\
8.03	0.01\\
8.04	0.01\\
8.05	0.01\\
8.06	0.01\\
8.07	0.01\\
8.08	0.01\\
8.09	0.01\\
8.1	0.01\\
8.11	0.01\\
8.12	0.01\\
8.13	0.01\\
8.14	0.01\\
8.15	0.01\\
8.16	0.01\\
8.17	0.01\\
8.18	0.01\\
8.19	0.01\\
8.2	0.01\\
8.21	0.01\\
8.22	0.01\\
8.23	0.01\\
8.24	0.01\\
8.25	0.01\\
8.26	0.01\\
8.27	0.01\\
8.28	0.01\\
8.29	0.01\\
8.3	0.01\\
8.31	0.01\\
8.32	0.01\\
8.33	0.01\\
8.34	0.01\\
8.35	0.01\\
8.36	0.01\\
8.37	0.01\\
8.38	0.01\\
8.39	0.01\\
8.4	0.01\\
8.41	0.01\\
8.42	0.01\\
8.43	0.01\\
8.44	0.01\\
8.45	0.01\\
8.46	0.01\\
8.47	0.01\\
8.48	0.01\\
8.49	0.01\\
8.5	0.01\\
8.51	0.01\\
8.52	0.01\\
8.53	0.01\\
8.54	0.01\\
8.55	0.01\\
8.56	0.01\\
8.57	0.01\\
8.58	0.01\\
8.59	0.01\\
8.6	0.01\\
8.61	0.01\\
8.62	0.01\\
8.63	0.01\\
8.64	0.01\\
8.65	0.01\\
8.66	0.01\\
8.67	0.01\\
8.68	0.01\\
8.69	0.01\\
8.7	0.01\\
8.71	0.01\\
8.72	0.01\\
8.73	0.01\\
8.74	0.01\\
8.75	0.01\\
8.76	0.01\\
8.77	0.01\\
8.78	0.01\\
8.79	0.01\\
8.8	0.01\\
8.81	0.01\\
8.82	0.01\\
8.83	0.01\\
8.84	0.01\\
8.85	0.01\\
8.86	0.01\\
8.87	0.01\\
8.88	0.01\\
8.89	0.01\\
8.9	0.01\\
8.91	0.01\\
8.92	0.01\\
8.93	0.01\\
8.94	0.01\\
8.95	0.01\\
8.96	0.01\\
8.97	0.01\\
8.98	0.01\\
8.99	0.01\\
9	0.01\\
9.01	0.01\\
9.02	0.01\\
9.03	0.01\\
9.04	0.01\\
9.05	0.01\\
9.06	0.01\\
9.07	0.01\\
9.08	0.01\\
9.09	0.01\\
9.1	0.01\\
9.11	0.01\\
9.12	0.01\\
9.13	0.01\\
9.14	0.01\\
9.15	0.01\\
9.16	0.01\\
9.17	0.01\\
9.18	0.01\\
9.19	0.01\\
9.2	0.01\\
9.21	0.01\\
9.22	0.01\\
9.23	0.01\\
9.24	0.01\\
9.25	0.01\\
9.26	0.01\\
9.27	0.01\\
9.28	0.01\\
9.29	0.01\\
9.3	0.01\\
9.31	0.01\\
9.32	0.01\\
9.33	0.01\\
9.34	0.01\\
9.35	0.01\\
9.36	0.01\\
9.37	0.01\\
9.38	0.01\\
9.39	0.01\\
9.4	0.01\\
9.41	0.01\\
9.42	0.01\\
9.43	0.01\\
9.44	0.01\\
9.45	0.01\\
9.46	0.01\\
9.47	0.01\\
9.48	0.01\\
9.49	0.01\\
9.5	0.01\\
9.51	0.01\\
9.52	0.01\\
9.53	0.01\\
9.54	0.01\\
9.55	0.01\\
9.56	0.01\\
9.57	0.01\\
9.58	0.01\\
9.59	0.01\\
9.6	0.01\\
9.61	0.01\\
9.62	0.01\\
9.63	0.01\\
9.64	0.01\\
9.65	0.01\\
9.66	0.01\\
9.67	0.01\\
9.68	0.01\\
9.69	0.01\\
9.7	0.01\\
9.71	0.01\\
9.72	0.01\\
9.73	0.01\\
9.74	0.01\\
9.75	0.01\\
9.76	0.01\\
9.77	0.01\\
9.78	0.01\\
9.79	0.01\\
9.8	0.01\\
9.81	0.01\\
9.82	0.01\\
9.83	0.01\\
9.84	0.01\\
9.85	0.01\\
9.86	0.01\\
9.87	0.01\\
9.88	0.01\\
9.89	0.01\\
9.9	0.01\\
9.91	0.01\\
9.92	0.01\\
9.93	0.01\\
9.94	0.01\\
9.95	0.01\\
9.96	0.01\\
9.97	0.01\\
9.98	0.01\\
9.99	0.01\\
10	0.01\\
10.01	0.01\\
10.02	0.01\\
10.03	0.01\\
10.04	0.01\\
10.05	0.01\\
10.06	0.01\\
10.07	0.01\\
10.08	0.01\\
10.09	0.01\\
10.1	0.01\\
10.11	0.01\\
10.12	0.01\\
10.13	0.01\\
10.14	0.01\\
10.15	0.01\\
10.16	0.01\\
10.17	0.01\\
10.18	0.01\\
10.19	0.01\\
10.2	0.01\\
10.21	0.01\\
10.22	0.01\\
10.23	0.01\\
10.24	0.01\\
10.25	0.01\\
10.26	0.01\\
10.27	0.01\\
10.28	0.01\\
10.29	0.01\\
10.3	0.01\\
10.31	0.01\\
10.32	0.01\\
10.33	0.01\\
10.34	0.01\\
10.35	0.01\\
10.36	0.01\\
10.37	0.01\\
10.38	0.01\\
10.39	0.01\\
10.4	0.01\\
10.41	0.01\\
10.42	0.01\\
10.43	0.01\\
10.44	0.01\\
10.45	0.01\\
10.46	0.01\\
10.47	0.01\\
10.48	0.01\\
10.49	0.01\\
10.5	0.01\\
10.51	0.01\\
10.52	0.01\\
10.53	0.01\\
10.54	0.01\\
10.55	0.01\\
10.56	0.01\\
10.57	0.01\\
10.58	0.01\\
10.59	0.01\\
10.6	0.01\\
10.61	0.01\\
10.62	0.01\\
10.63	0.01\\
10.64	0.01\\
10.65	0.01\\
10.66	0.01\\
10.67	0.01\\
10.68	0.01\\
10.69	0.01\\
10.7	0.01\\
10.71	0.01\\
10.72	0.01\\
10.73	0.01\\
10.74	0.01\\
10.75	0.01\\
10.76	0.01\\
10.77	0.01\\
10.78	0.01\\
10.79	0.01\\
10.8	0.01\\
10.81	0.01\\
10.82	0.01\\
10.83	0.01\\
10.84	0.01\\
10.85	0.01\\
10.86	0.01\\
10.87	0.01\\
10.88	0.01\\
10.89	0.01\\
10.9	0.01\\
10.91	0.01\\
10.92	0.01\\
10.93	0.01\\
10.94	0.01\\
10.95	0.01\\
10.96	0.01\\
10.97	0.01\\
10.98	0.01\\
10.99	0.01\\
11	0.01\\
11.01	0.01\\
11.02	0.01\\
11.03	0.01\\
11.04	0.01\\
11.05	0.01\\
11.06	0.01\\
11.07	0.01\\
11.08	0.01\\
11.09	0.01\\
11.1	0.01\\
11.11	0.01\\
11.12	0.01\\
11.13	0.01\\
11.14	0.01\\
11.15	0.01\\
11.16	0.01\\
11.17	0.01\\
11.18	0.01\\
11.19	0.01\\
11.2	0.01\\
11.21	0.01\\
11.22	0.01\\
11.23	0.01\\
11.24	0.01\\
11.25	0.01\\
11.26	0.01\\
11.27	0.01\\
11.28	0.01\\
11.29	0.01\\
11.3	0.01\\
11.31	0.01\\
11.32	0.01\\
11.33	0.01\\
11.34	0.01\\
11.35	0.01\\
11.36	0.01\\
11.37	0.01\\
11.38	0.01\\
11.39	0.01\\
11.4	0.01\\
11.41	0.01\\
11.42	0.01\\
11.43	0.01\\
11.44	0.01\\
11.45	0.01\\
11.46	0.01\\
11.47	0.01\\
11.48	0.01\\
11.49	0.01\\
11.5	0.01\\
11.51	0.01\\
11.52	0.01\\
11.53	0.01\\
11.54	0.01\\
11.55	0.01\\
11.56	0.01\\
11.57	0.01\\
11.58	0.01\\
11.59	0.01\\
11.6	0.01\\
11.61	0.01\\
11.62	0.01\\
11.63	0.01\\
11.64	0.01\\
11.65	0.01\\
11.66	0.01\\
11.67	0.01\\
11.68	0.01\\
11.69	0.01\\
11.7	0.01\\
11.71	0.01\\
11.72	0.01\\
11.73	0.01\\
11.74	0.01\\
11.75	0.01\\
11.76	0.01\\
11.77	0.01\\
11.78	0.01\\
11.79	0.01\\
11.8	0.01\\
11.81	0.01\\
11.82	0.01\\
11.83	0.01\\
11.84	0.01\\
11.85	0.01\\
11.86	0.01\\
11.87	0.01\\
11.88	0.01\\
11.89	0.01\\
11.9	0.01\\
11.91	0.01\\
11.92	0.01\\
11.93	0.01\\
11.94	0.01\\
11.95	0.01\\
11.96	0.01\\
11.97	0.01\\
11.98	0.01\\
11.99	0.01\\
12	0.01\\
12.01	0.01\\
12.02	0.01\\
12.03	0.01\\
12.04	0.01\\
12.05	0.01\\
12.06	0.01\\
12.07	0.01\\
12.08	0.01\\
12.09	0.01\\
12.1	0.01\\
12.11	0.01\\
12.12	0.01\\
12.13	0.01\\
12.14	0.01\\
12.15	0.01\\
12.16	0.01\\
12.17	0.01\\
12.18	0.01\\
12.19	0.01\\
12.2	0.01\\
12.21	0.01\\
12.22	0.01\\
12.23	0.01\\
12.24	0.01\\
12.25	0.01\\
12.26	0.01\\
12.27	0.01\\
12.28	0.01\\
12.29	0.01\\
12.3	0.01\\
12.31	0.01\\
12.32	0.01\\
12.33	0.01\\
12.34	0.01\\
12.35	0.01\\
12.36	0.01\\
12.37	0.01\\
12.38	0.01\\
12.39	0.01\\
12.4	0.01\\
12.41	0.01\\
12.42	0.01\\
12.43	0.01\\
12.44	0.01\\
12.45	0.01\\
12.46	0.01\\
12.47	0.01\\
12.48	0.01\\
12.49	0.01\\
12.5	0.01\\
12.51	0.01\\
12.52	0.01\\
12.53	0.01\\
12.54	0.01\\
12.55	0.01\\
12.56	0.01\\
12.57	0.01\\
12.58	0.01\\
12.59	0.01\\
12.6	0.01\\
12.61	0.01\\
12.62	0.01\\
12.63	0.01\\
12.64	0.01\\
12.65	0.01\\
12.66	0.01\\
12.67	0.01\\
12.68	0.01\\
12.69	0.01\\
12.7	0.01\\
12.71	0.01\\
12.72	0.01\\
12.73	0.01\\
12.74	0.01\\
12.75	0.01\\
12.76	0.01\\
12.77	0.01\\
12.78	0.01\\
12.79	0.01\\
12.8	0.01\\
12.81	0.01\\
12.82	0.01\\
12.83	0.01\\
12.84	0.01\\
12.85	0.01\\
12.86	0.01\\
12.87	0.01\\
12.88	0.01\\
12.89	0.01\\
12.9	0.01\\
12.91	0.01\\
12.92	0.01\\
12.93	0.01\\
12.94	0.01\\
12.95	0.01\\
12.96	0.01\\
12.97	0.01\\
12.98	0.01\\
12.99	0.01\\
13	0.01\\
13.01	0.01\\
13.02	0.01\\
13.03	0.01\\
13.04	0.01\\
13.05	0.01\\
13.06	0.01\\
13.07	0.01\\
13.08	0.01\\
13.09	0.01\\
13.1	0.01\\
13.11	0.01\\
13.12	0.01\\
13.13	0.01\\
13.14	0.01\\
13.15	0.01\\
13.16	0.01\\
13.17	0.01\\
13.18	0.01\\
13.19	0.01\\
13.2	0.01\\
13.21	0.01\\
13.22	0.01\\
13.23	0.01\\
13.24	0.01\\
13.25	0.01\\
13.26	0.01\\
13.27	0.01\\
13.28	0.01\\
13.29	0.01\\
13.3	0.01\\
13.31	0.01\\
13.32	0.01\\
13.33	0.01\\
13.34	0.01\\
13.35	0.01\\
13.36	0.01\\
13.37	0.01\\
13.38	0.01\\
13.39	0.01\\
13.4	0.01\\
13.41	0.01\\
13.42	0.01\\
13.43	0.01\\
13.44	0.01\\
13.45	0.01\\
13.46	0.01\\
13.47	0.01\\
13.48	0.01\\
13.49	0.01\\
13.5	0.01\\
13.51	0.01\\
13.52	0.01\\
13.53	0.01\\
13.54	0.01\\
13.55	0.01\\
13.56	0.01\\
13.57	0.01\\
13.58	0.01\\
13.59	0.01\\
13.6	0.01\\
13.61	0.01\\
13.62	0.01\\
13.63	0.01\\
13.64	0.01\\
13.65	0.01\\
13.66	0.01\\
13.67	0.01\\
13.68	0.01\\
13.69	0.01\\
13.7	0.01\\
13.71	0.01\\
13.72	0.01\\
13.73	0.01\\
13.74	0.01\\
13.75	0.01\\
13.76	0.01\\
13.77	0.01\\
13.78	0.01\\
13.79	0.01\\
13.8	0.01\\
13.81	0.01\\
13.82	0.01\\
13.83	0.01\\
13.84	0.01\\
13.85	0.01\\
13.86	0.01\\
13.87	0.01\\
13.88	0.01\\
13.89	0.01\\
13.9	0.01\\
13.91	0.01\\
13.92	0.01\\
13.93	0.01\\
13.94	0.01\\
13.95	0.01\\
13.96	0.01\\
13.97	0.01\\
13.98	0.01\\
13.99	0.01\\
14	0.01\\
14.01	0.01\\
14.02	0.01\\
14.03	0.01\\
14.04	0.01\\
14.05	0.01\\
14.06	0.01\\
14.07	0.01\\
14.08	0.01\\
14.09	0.01\\
14.1	0.01\\
14.11	0.01\\
14.12	0.01\\
14.13	0.01\\
14.14	0.01\\
14.15	0.01\\
14.16	0.01\\
14.17	0.01\\
14.18	0.01\\
14.19	0.01\\
14.2	0.01\\
14.21	0.01\\
14.22	0.01\\
14.23	0.01\\
14.24	0.01\\
14.25	0.01\\
14.26	0.01\\
14.27	0.01\\
14.28	0.01\\
14.29	0.01\\
14.3	0.01\\
14.31	0.01\\
14.32	0.01\\
14.33	0.01\\
14.34	0.01\\
14.35	0.01\\
14.36	0.01\\
14.37	0.01\\
14.38	0.01\\
14.39	0.01\\
14.4	0.01\\
14.41	0.01\\
14.42	0.01\\
14.43	0.01\\
14.44	0.01\\
14.45	0.01\\
14.46	0.01\\
14.47	0.01\\
14.48	0.01\\
14.49	0.01\\
14.5	0.01\\
14.51	0.01\\
14.52	0.01\\
14.53	0.01\\
14.54	0.01\\
14.55	0.01\\
14.56	0.01\\
14.57	0.01\\
14.58	0.01\\
14.59	0.01\\
14.6	0.01\\
14.61	0.01\\
14.62	0.01\\
14.63	0.01\\
14.64	0.01\\
14.65	0.01\\
14.66	0.01\\
14.67	0.01\\
14.68	0.01\\
14.69	0.01\\
14.7	0.01\\
14.71	0.01\\
14.72	0.01\\
14.73	0.01\\
14.74	0.01\\
14.75	0.01\\
14.76	0.01\\
14.77	0.01\\
14.78	0.01\\
14.79	0.01\\
14.8	0.01\\
14.81	0.01\\
14.82	0.01\\
14.83	0.01\\
14.84	0.01\\
14.85	0.01\\
14.86	0.01\\
14.87	0.01\\
14.88	0.01\\
14.89	0.01\\
14.9	0.01\\
14.91	0.01\\
14.92	0.01\\
14.93	0.01\\
14.94	0.01\\
14.95	0.01\\
14.96	0.01\\
14.97	0.01\\
14.98	0.01\\
14.99	0.01\\
15	0.01\\
15.01	0.01\\
15.02	0.01\\
15.03	0.01\\
15.04	0.01\\
15.05	0.01\\
15.06	0.01\\
15.07	0.01\\
15.08	0.01\\
15.09	0.01\\
15.1	0.01\\
15.11	0.01\\
15.12	0.01\\
15.13	0.01\\
15.14	0.01\\
15.15	0.01\\
15.16	0.01\\
15.17	0.01\\
15.18	0.01\\
15.19	0.01\\
15.2	0.01\\
15.21	0.01\\
15.22	0.01\\
15.23	0.01\\
15.24	0.01\\
15.25	0.01\\
15.26	0.01\\
15.27	0.01\\
15.28	0.01\\
15.29	0.01\\
15.3	0.01\\
15.31	0.01\\
15.32	0.01\\
15.33	0.01\\
15.34	0.01\\
15.35	0.01\\
15.36	0.01\\
15.37	0.01\\
15.38	0.01\\
15.39	0.01\\
15.4	0.01\\
15.41	0.01\\
15.42	0.01\\
15.43	0.01\\
15.44	0.01\\
15.45	0.01\\
15.46	0.01\\
15.47	0.01\\
15.48	0.01\\
15.49	0.01\\
15.5	0.01\\
15.51	0.01\\
15.52	0.01\\
15.53	0.01\\
15.54	0.01\\
15.55	0.01\\
15.56	0.01\\
15.57	0.01\\
15.58	0.01\\
15.59	0.01\\
15.6	0.01\\
15.61	0.01\\
15.62	0.01\\
15.63	0.01\\
15.64	0.01\\
15.65	0.01\\
15.66	0.01\\
15.67	0.01\\
15.68	0.01\\
15.69	0.01\\
15.7	0.01\\
15.71	0.01\\
15.72	0.01\\
15.73	0.01\\
15.74	0.01\\
15.75	0.01\\
15.76	0.01\\
15.77	0.01\\
15.78	0.01\\
15.79	0.01\\
15.8	0.01\\
15.81	0.01\\
15.82	0.01\\
15.83	0.01\\
15.84	0.01\\
15.85	0.01\\
15.86	0.01\\
15.87	0.01\\
15.88	0.01\\
15.89	0.01\\
15.9	0.01\\
15.91	0.01\\
15.92	0.01\\
15.93	0.01\\
15.94	0.01\\
15.95	0.01\\
15.96	0.01\\
15.97	0.01\\
15.98	0.01\\
15.99	0.01\\
16	0.01\\
16.01	0.01\\
16.02	0.01\\
16.03	0.01\\
16.04	0.01\\
16.05	0.01\\
16.06	0.01\\
16.07	0.01\\
16.08	0.01\\
16.09	0.01\\
16.1	0.01\\
16.11	0.01\\
16.12	0.01\\
16.13	0.01\\
16.14	0.01\\
16.15	0.01\\
16.16	0.01\\
16.17	0.01\\
16.18	0.01\\
16.19	0.01\\
16.2	0.01\\
16.21	0.01\\
16.22	0.01\\
16.23	0.01\\
16.24	0.01\\
16.25	0.01\\
16.26	0.01\\
16.27	0.01\\
16.28	0.01\\
16.29	0.01\\
16.3	0.01\\
16.31	0.01\\
16.32	0.01\\
16.33	0.01\\
16.34	0.01\\
16.35	0.01\\
16.36	0.01\\
16.37	0.01\\
16.38	0.01\\
16.39	0.01\\
16.4	0.01\\
16.41	0.01\\
16.42	0.01\\
16.43	0.01\\
16.44	0.01\\
16.45	0.01\\
16.46	0.01\\
16.47	0.01\\
16.48	0.01\\
16.49	0.01\\
16.5	0.01\\
16.51	0.01\\
16.52	0.01\\
16.53	0.01\\
16.54	0.01\\
16.55	0.01\\
16.56	0.01\\
16.57	0.01\\
16.58	0.01\\
16.59	0.01\\
16.6	0.01\\
16.61	0.01\\
16.62	0.01\\
16.63	0.01\\
16.64	0.01\\
16.65	0.01\\
16.66	0.01\\
16.67	0.01\\
16.68	0.01\\
16.69	0.01\\
16.7	0.01\\
16.71	0.01\\
16.72	0.01\\
16.73	0.01\\
16.74	0.01\\
16.75	0.01\\
16.76	0.01\\
16.77	0.01\\
16.78	0.01\\
16.79	0.01\\
16.8	0.01\\
16.81	0.01\\
16.82	0.01\\
16.83	0.01\\
16.84	0.01\\
16.85	0.01\\
16.86	0.01\\
16.87	0.01\\
16.88	0.01\\
16.89	0.01\\
16.9	0.01\\
16.91	0.01\\
16.92	0.01\\
16.93	0.01\\
16.94	0.01\\
16.95	0.01\\
16.96	0.01\\
16.97	0.01\\
16.98	0.01\\
16.99	0.01\\
17	0.01\\
17.01	0.01\\
17.02	0.01\\
17.03	0.01\\
17.04	0.01\\
17.05	0.01\\
17.06	0.01\\
17.07	0.01\\
17.08	0.01\\
17.09	0.01\\
17.1	0.01\\
17.11	0.01\\
17.12	0.01\\
17.13	0.01\\
17.14	0.01\\
17.15	0.01\\
17.16	0.01\\
17.17	0.01\\
17.18	0.01\\
17.19	0.01\\
17.2	0.01\\
17.21	0.01\\
17.22	0.01\\
17.23	0.01\\
17.24	0.01\\
17.25	0.01\\
17.26	0.01\\
17.27	0.01\\
17.28	0.01\\
17.29	0.01\\
17.3	0.01\\
17.31	0.01\\
17.32	0.01\\
17.33	0.01\\
17.34	0.01\\
17.35	0.01\\
17.36	0.01\\
17.37	0.01\\
17.38	0.01\\
17.39	0.01\\
17.4	0.01\\
17.41	0.01\\
17.42	0.01\\
17.43	0.01\\
17.44	0.01\\
17.45	0.01\\
17.46	0.01\\
17.47	0.01\\
17.48	0.01\\
17.49	0.01\\
17.5	0.01\\
17.51	0.01\\
17.52	0.01\\
17.53	0.01\\
17.54	0.01\\
17.55	0.01\\
17.56	0.01\\
17.57	0.01\\
17.58	0.01\\
17.59	0.01\\
17.6	0.01\\
17.61	0.01\\
17.62	0.01\\
17.63	0.01\\
17.64	0.01\\
17.65	0.01\\
17.66	0.01\\
17.67	0.01\\
17.68	0.01\\
17.69	0.01\\
17.7	0.01\\
17.71	0.01\\
17.72	0.01\\
17.73	0.01\\
17.74	0.01\\
17.75	0.01\\
17.76	0.01\\
17.77	0.01\\
17.78	0.01\\
17.79	0.01\\
17.8	0.01\\
17.81	0.01\\
17.82	0.01\\
17.83	0.01\\
17.84	0.01\\
17.85	0.01\\
17.86	0.01\\
17.87	0.01\\
17.88	0.01\\
17.89	0.01\\
17.9	0.01\\
17.91	0.01\\
17.92	0.01\\
17.93	0.01\\
17.94	0.01\\
17.95	0.01\\
17.96	0.01\\
17.97	0.01\\
17.98	0.01\\
17.99	0.01\\
18	0.01\\
18.01	0.01\\
18.02	0.01\\
18.03	0.01\\
18.04	0.01\\
18.05	0.01\\
18.06	0.01\\
18.07	0.01\\
18.08	0.01\\
18.09	0.01\\
18.1	0.01\\
18.11	0.01\\
18.12	0.01\\
18.13	0.01\\
18.14	0.01\\
18.15	0.01\\
18.16	0.01\\
18.17	0.01\\
18.18	0.01\\
18.19	0.01\\
18.2	0.01\\
18.21	0.01\\
18.22	0.01\\
18.23	0.01\\
18.24	0.01\\
18.25	0.01\\
18.26	0.01\\
18.27	0.01\\
18.28	0.01\\
18.29	0.01\\
18.3	0.01\\
18.31	0.01\\
18.32	0.01\\
18.33	0.01\\
18.34	0.01\\
18.35	0.01\\
18.36	0.01\\
18.37	0.01\\
18.38	0.01\\
18.39	0.01\\
18.4	0.01\\
18.41	0.01\\
18.42	0.01\\
18.43	0.01\\
18.44	0.01\\
18.45	0.01\\
18.46	0.01\\
18.47	0.01\\
18.48	0.01\\
18.49	0.01\\
18.5	0.01\\
18.51	0.01\\
18.52	0.01\\
18.53	0.01\\
18.54	0.01\\
18.55	0.01\\
18.56	0.01\\
18.57	0.01\\
18.58	0.01\\
18.59	0.01\\
18.6	0.01\\
18.61	0.01\\
18.62	0.01\\
18.63	0.01\\
18.64	0.01\\
18.65	0.01\\
18.66	0.01\\
18.67	0.01\\
18.68	0.01\\
18.69	0.01\\
18.7	0.01\\
18.71	0.01\\
18.72	0.01\\
18.73	0.01\\
18.74	0.01\\
18.75	0.01\\
18.76	0.01\\
18.77	0.01\\
18.78	0.01\\
18.79	0.01\\
18.8	0.01\\
18.81	0.01\\
18.82	0.01\\
18.83	0.01\\
18.84	0.01\\
18.85	0.01\\
18.86	0.01\\
18.87	0.01\\
18.88	0.01\\
18.89	0.01\\
18.9	0.01\\
18.91	0.01\\
18.92	0.01\\
18.93	0.01\\
18.94	0.01\\
18.95	0.01\\
18.96	0.01\\
18.97	0.01\\
18.98	0.01\\
18.99	0.01\\
19	0.01\\
19.01	0.01\\
19.02	0.01\\
19.03	0.01\\
19.04	0.01\\
19.05	0.01\\
19.06	0.01\\
19.07	0.01\\
19.08	0.01\\
19.09	0.01\\
19.1	0.01\\
19.11	0.01\\
19.12	0.01\\
19.13	0.01\\
19.14	0.01\\
19.15	0.01\\
19.16	0.01\\
19.17	0.01\\
19.18	0.01\\
19.19	0.01\\
19.2	0.01\\
19.21	0.01\\
19.22	0.01\\
19.23	0.01\\
19.24	0.01\\
19.25	0.01\\
19.26	0.01\\
19.27	0.01\\
19.28	0.01\\
19.29	0.01\\
19.3	0.01\\
19.31	0.01\\
19.32	0.01\\
19.33	0.01\\
19.34	0.01\\
19.35	0.01\\
19.36	0.01\\
19.37	0.01\\
19.38	0.01\\
19.39	0.01\\
19.4	0.01\\
19.41	0.01\\
19.42	0.01\\
19.43	0.01\\
19.44	0.01\\
19.45	0.01\\
19.46	0.01\\
19.47	0.01\\
19.48	0.01\\
19.49	0.01\\
19.5	0.01\\
19.51	0.01\\
19.52	0.01\\
19.53	0.01\\
19.54	0.01\\
19.55	0.01\\
19.56	0.01\\
19.57	0.01\\
19.58	0.01\\
19.59	0.01\\
19.6	0.01\\
19.61	0.01\\
19.62	0.01\\
19.63	0.01\\
19.64	0.01\\
19.65	0.01\\
19.66	0.01\\
19.67	0.01\\
19.68	0.01\\
19.69	0.01\\
19.7	0.01\\
19.71	0.01\\
19.72	0.01\\
19.73	0.01\\
19.74	0.01\\
19.75	0.01\\
19.76	0.01\\
19.77	0.01\\
19.78	0.01\\
19.79	0.01\\
19.8	0.01\\
19.81	0.01\\
19.82	0.01\\
19.83	0.01\\
19.84	0.01\\
19.85	0.01\\
19.86	0.01\\
19.87	0.01\\
19.88	0.01\\
19.89	0.01\\
19.9	0.01\\
19.91	0.01\\
19.92	0.01\\
19.93	0.01\\
19.94	0.01\\
19.95	0.01\\
19.96	0.01\\
19.97	0.01\\
19.98	0.01\\
19.99	0.01\\
20	0.01\\
20.01	0.01\\
20.02	0.01\\
20.03	0.01\\
20.04	0.01\\
20.05	0.01\\
20.06	0.01\\
20.07	0.01\\
20.08	0.01\\
20.09	0.01\\
20.1	0.01\\
20.11	0.01\\
20.12	0.01\\
20.13	0.01\\
20.14	0.01\\
20.15	0.01\\
20.16	0.01\\
20.17	0.01\\
20.18	0.01\\
20.19	0.01\\
20.2	0.01\\
20.21	0.01\\
20.22	0.01\\
20.23	0.01\\
20.24	0.01\\
20.25	0.01\\
20.26	0.01\\
20.27	0.01\\
20.28	0.01\\
20.29	0.01\\
20.3	0.01\\
20.31	0.01\\
20.32	0.01\\
20.33	0.01\\
20.34	0.01\\
20.35	0.01\\
20.36	0.01\\
20.37	0.01\\
20.38	0.01\\
20.39	0.01\\
20.4	0.01\\
20.41	0.01\\
20.42	0.01\\
20.43	0.01\\
20.44	0.01\\
20.45	0.01\\
20.46	0.01\\
20.47	0.01\\
20.48	0.01\\
20.49	0.01\\
20.5	0.01\\
20.51	0.01\\
20.52	0.01\\
20.53	0.01\\
20.54	0.01\\
20.55	0.01\\
20.56	0.01\\
20.57	0.01\\
20.58	0.01\\
20.59	0.01\\
20.6	0.01\\
20.61	0.01\\
20.62	0.01\\
20.63	0.01\\
20.64	0.01\\
20.65	0.01\\
20.66	0.01\\
20.67	0.01\\
20.68	0.01\\
20.69	0.01\\
20.7	0.01\\
20.71	0.01\\
20.72	0.01\\
20.73	0.01\\
20.74	0.01\\
20.75	0.01\\
20.76	0.01\\
20.77	0.01\\
20.78	0.01\\
20.79	0.01\\
20.8	0.01\\
20.81	0.01\\
20.82	0.01\\
20.83	0.01\\
20.84	0.01\\
20.85	0.01\\
20.86	0.01\\
20.87	0.01\\
20.88	0.01\\
20.89	0.01\\
20.9	0.01\\
20.91	0.01\\
20.92	0.01\\
20.93	0.01\\
20.94	0.01\\
20.95	0.01\\
20.96	0.01\\
20.97	0.01\\
20.98	0.01\\
20.99	0.01\\
21	0.01\\
21.01	0.01\\
21.02	0.01\\
21.03	0.01\\
21.04	0.01\\
21.05	0.01\\
21.06	0.01\\
21.07	0.01\\
21.08	0.01\\
21.09	0.01\\
21.1	0.01\\
21.11	0.01\\
21.12	0.01\\
21.13	0.01\\
21.14	0.01\\
21.15	0.01\\
21.16	0.01\\
21.17	0.01\\
21.18	0.01\\
21.19	0.01\\
21.2	0.01\\
21.21	0.01\\
21.22	0.01\\
21.23	0.01\\
21.24	0.01\\
21.25	0.01\\
21.26	0.01\\
21.27	0.01\\
21.28	0.01\\
21.29	0.01\\
21.3	0.01\\
21.31	0.01\\
21.32	0.01\\
21.33	0.01\\
21.34	0.01\\
21.35	0.01\\
21.36	0.01\\
21.37	0.01\\
21.38	0.01\\
21.39	0.01\\
21.4	0.01\\
21.41	0.01\\
21.42	0.01\\
21.43	0.01\\
21.44	0.01\\
21.45	0.01\\
21.46	0.01\\
21.47	0.01\\
21.48	0.01\\
21.49	0.01\\
21.5	0.01\\
21.51	0.01\\
21.52	0.01\\
21.53	0.01\\
21.54	0.01\\
21.55	0.01\\
21.56	0.01\\
21.57	0.01\\
21.58	0.01\\
21.59	0.01\\
21.6	0.01\\
21.61	0.01\\
21.62	0.01\\
21.63	0.01\\
21.64	0.01\\
21.65	0.01\\
21.66	0.01\\
21.67	0.01\\
21.68	0.01\\
21.69	0.01\\
21.7	0.01\\
21.71	0.01\\
21.72	0.01\\
21.73	0.01\\
21.74	0.01\\
21.75	0.01\\
21.76	0.01\\
21.77	0.01\\
21.78	0.01\\
21.79	0.01\\
21.8	0.01\\
21.81	0.01\\
21.82	0.01\\
21.83	0.01\\
21.84	0.01\\
21.85	0.01\\
21.86	0.01\\
21.87	0.01\\
21.88	0.01\\
21.89	0.01\\
21.9	0.01\\
21.91	0.01\\
21.92	0.01\\
21.93	0.01\\
21.94	0.01\\
21.95	0.01\\
21.96	0.01\\
21.97	0.01\\
21.98	0.01\\
21.99	0.01\\
22	0.01\\
22.01	0.01\\
22.02	0.01\\
22.03	0.01\\
22.04	0.01\\
22.05	0.01\\
22.06	0.01\\
22.07	0.01\\
22.08	0.01\\
22.09	0.01\\
22.1	0.01\\
22.11	0.01\\
22.12	0.01\\
22.13	0.01\\
22.14	0.01\\
22.15	0.01\\
22.16	0.01\\
22.17	0.01\\
22.18	0.01\\
22.19	0.01\\
22.2	0.01\\
22.21	0.01\\
22.22	0.01\\
22.23	0.01\\
22.24	0.01\\
22.25	0.01\\
22.26	0.01\\
22.27	0.01\\
22.28	0.01\\
22.29	0.01\\
22.3	0.01\\
22.31	0.01\\
22.32	0.01\\
22.33	0.01\\
22.34	0.01\\
22.35	0.01\\
22.36	0.01\\
22.37	0.01\\
22.38	0.01\\
22.39	0.01\\
22.4	0.01\\
22.41	0.01\\
22.42	0.01\\
22.43	0.01\\
22.44	0.01\\
22.45	0.01\\
22.46	0.01\\
22.47	0.01\\
22.48	0.01\\
22.49	0.01\\
22.5	0.01\\
22.51	0.01\\
22.52	0.01\\
22.53	0.01\\
22.54	0.01\\
22.55	0.01\\
22.56	0.01\\
22.57	0.01\\
22.58	0.01\\
22.59	0.01\\
22.6	0.01\\
22.61	0.01\\
22.62	0.01\\
22.63	0.01\\
22.64	0.01\\
22.65	0.01\\
22.66	0.01\\
22.67	0.01\\
22.68	0.01\\
22.69	0.01\\
22.7	0.01\\
22.71	0.01\\
22.72	0.01\\
22.73	0.01\\
22.74	0.01\\
22.75	0.01\\
22.76	0.01\\
22.77	0.01\\
22.78	0.01\\
22.79	0.01\\
22.8	0.01\\
22.81	0.01\\
22.82	0.01\\
22.83	0.01\\
22.84	0.01\\
22.85	0.01\\
22.86	0.01\\
22.87	0.01\\
22.88	0.01\\
22.89	0.01\\
22.9	0.01\\
22.91	0.01\\
22.92	0.01\\
22.93	0.01\\
22.94	0.01\\
22.95	0.01\\
22.96	0.01\\
22.97	0.01\\
22.98	0.01\\
22.99	0.01\\
23	0.01\\
23.01	0.01\\
23.02	0.01\\
23.03	0.01\\
23.04	0.01\\
23.05	0.01\\
23.06	0.01\\
23.07	0.01\\
23.08	0.01\\
23.09	0.01\\
23.1	0.01\\
23.11	0.01\\
23.12	0.01\\
23.13	0.01\\
23.14	0.01\\
23.15	0.01\\
23.16	0.01\\
23.17	0.01\\
23.18	0.01\\
23.19	0.01\\
23.2	0.01\\
23.21	0.01\\
23.22	0.01\\
23.23	0.01\\
23.24	0.01\\
23.25	0.01\\
23.26	0.01\\
23.27	0.01\\
23.28	0.01\\
23.29	0.01\\
23.3	0.01\\
23.31	0.01\\
23.32	0.01\\
23.33	0.01\\
23.34	0.01\\
23.35	0.01\\
23.36	0.01\\
23.37	0.01\\
23.38	0.01\\
23.39	0.01\\
23.4	0.01\\
23.41	0.01\\
23.42	0.01\\
23.43	0.01\\
23.44	0.01\\
23.45	0.01\\
23.46	0.01\\
23.47	0.01\\
23.48	0.01\\
23.49	0.01\\
23.5	0.01\\
23.51	0.01\\
23.52	0.01\\
23.53	0.01\\
23.54	0.01\\
23.55	0.01\\
23.56	0.01\\
23.57	0.01\\
23.58	0.01\\
23.59	0.01\\
23.6	0.01\\
23.61	0.01\\
23.62	0.01\\
23.63	0.01\\
23.64	0.01\\
23.65	0.01\\
23.66	0.01\\
23.67	0.01\\
23.68	0.01\\
23.69	0.01\\
23.7	0.01\\
23.71	0.01\\
23.72	0.01\\
23.73	0.01\\
23.74	0.01\\
23.75	0.01\\
23.76	0.01\\
23.77	0.01\\
23.78	0.01\\
23.79	0.01\\
23.8	0.01\\
23.81	0.01\\
23.82	0.01\\
23.83	0.01\\
23.84	0.01\\
23.85	0.01\\
23.86	0.01\\
23.87	0.01\\
23.88	0.01\\
23.89	0.01\\
23.9	0.01\\
23.91	0.01\\
23.92	0.01\\
23.93	0.01\\
23.94	0.01\\
23.95	0.01\\
23.96	0.01\\
23.97	0.01\\
23.98	0.01\\
23.99	0.01\\
24	0.01\\
24.01	0.01\\
24.02	0.01\\
24.03	0.01\\
24.04	0.01\\
24.05	0.01\\
24.06	0.01\\
24.07	0.01\\
24.08	0.01\\
24.09	0.01\\
24.1	0.01\\
24.11	0.01\\
24.12	0.01\\
24.13	0.01\\
24.14	0.01\\
24.15	0.01\\
24.16	0.01\\
24.17	0.01\\
24.18	0.01\\
24.19	0.01\\
24.2	0.01\\
24.21	0.01\\
24.22	0.01\\
24.23	0.01\\
24.24	0.01\\
24.25	0.01\\
24.26	0.01\\
24.27	0.01\\
24.28	0.01\\
24.29	0.01\\
24.3	0.01\\
24.31	0.01\\
24.32	0.01\\
24.33	0.01\\
24.34	0.01\\
24.35	0.01\\
24.36	0.01\\
24.37	0.01\\
24.38	0.01\\
24.39	0.01\\
24.4	0.01\\
24.41	0.01\\
24.42	0.01\\
24.43	0.01\\
24.44	0.01\\
24.45	0.01\\
24.46	0.01\\
24.47	0.01\\
24.48	0.01\\
24.49	0.01\\
24.5	0.01\\
24.51	0.01\\
24.52	0.01\\
24.53	0.01\\
24.54	0.01\\
24.55	0.01\\
24.56	0.01\\
24.57	0.01\\
24.58	0.01\\
24.59	0.01\\
24.6	0.01\\
24.61	0.01\\
24.62	0.01\\
24.63	0.01\\
24.64	0.01\\
24.65	0.01\\
24.66	0.01\\
24.67	0.01\\
24.68	0.01\\
24.69	0.01\\
24.7	0.01\\
24.71	0.01\\
24.72	0.01\\
24.73	0.01\\
24.74	0.01\\
24.75	0.01\\
24.76	0.01\\
24.77	0.01\\
24.78	0.01\\
24.79	0.01\\
24.8	0.01\\
24.81	0.01\\
24.82	0.01\\
24.83	0.01\\
24.84	0.01\\
24.85	0.01\\
24.86	0.01\\
24.87	0.01\\
24.88	0.01\\
24.89	0.01\\
24.9	0.01\\
24.91	0.01\\
24.92	0.01\\
24.93	0.01\\
24.94	0.01\\
24.95	0.01\\
24.96	0.01\\
24.97	0.01\\
24.98	0.01\\
24.99	0.01\\
25	0.01\\
25.01	0.01\\
25.02	0.01\\
25.03	0.01\\
25.04	0.01\\
25.05	0.01\\
25.06	0.01\\
25.07	0.01\\
25.08	0.01\\
25.09	0.01\\
25.1	0.01\\
25.11	0.01\\
25.12	0.01\\
25.13	0.01\\
25.14	0.01\\
25.15	0.01\\
25.16	0.01\\
25.17	0.01\\
25.18	0.01\\
25.19	0.01\\
25.2	0.01\\
25.21	0.01\\
25.22	0.01\\
25.23	0.01\\
25.24	0.01\\
25.25	0.01\\
25.26	0.01\\
25.27	0.01\\
25.28	0.01\\
25.29	0.01\\
25.3	0.01\\
25.31	0.01\\
25.32	0.01\\
25.33	0.01\\
25.34	0.01\\
25.35	0.01\\
25.36	0.01\\
25.37	0.01\\
25.38	0.01\\
25.39	0.01\\
25.4	0.01\\
25.41	0.01\\
25.42	0.01\\
25.43	0.01\\
25.44	0.01\\
25.45	0.01\\
25.46	0.01\\
25.47	0.01\\
25.48	0.01\\
25.49	0.01\\
25.5	0.01\\
25.51	0.01\\
25.52	0.01\\
25.53	0.01\\
25.54	0.01\\
25.55	0.01\\
25.56	0.01\\
25.57	0.01\\
25.58	0.01\\
25.59	0.01\\
25.6	0.01\\
25.61	0.01\\
25.62	0.01\\
25.63	0.01\\
25.64	0.01\\
25.65	0.01\\
25.66	0.01\\
25.67	0.01\\
25.68	0.01\\
25.69	0.01\\
25.7	0.01\\
25.71	0.01\\
25.72	0.01\\
25.73	0.01\\
25.74	0.01\\
25.75	0.01\\
25.76	0.01\\
25.77	0.01\\
25.78	0.01\\
25.79	0.01\\
25.8	0.01\\
25.81	0.01\\
25.82	0.01\\
25.83	0.01\\
25.84	0.01\\
25.85	0.01\\
25.86	0.01\\
25.87	0.01\\
25.88	0.01\\
25.89	0.01\\
25.9	0.01\\
25.91	0.01\\
25.92	0.01\\
25.93	0.01\\
25.94	0.01\\
25.95	0.01\\
25.96	0.01\\
25.97	0.01\\
25.98	0.01\\
25.99	0.01\\
26	0.01\\
26.01	0.01\\
26.02	0.01\\
26.03	0.01\\
26.04	0.01\\
26.05	0.01\\
26.06	0.01\\
26.07	0.01\\
26.08	0.01\\
26.09	0.01\\
26.1	0.01\\
26.11	0.01\\
26.12	0.01\\
26.13	0.01\\
26.14	0.01\\
26.15	0.01\\
26.16	0.01\\
26.17	0.01\\
26.18	0.01\\
26.19	0.01\\
26.2	0.01\\
26.21	0.01\\
26.22	0.01\\
26.23	0.01\\
26.24	0.01\\
26.25	0.01\\
26.26	0.01\\
26.27	0.01\\
26.28	0.01\\
26.29	0.01\\
26.3	0.01\\
26.31	0.01\\
26.32	0.01\\
26.33	0.01\\
26.34	0.01\\
26.35	0.01\\
26.36	0.01\\
26.37	0.01\\
26.38	0.01\\
26.39	0.01\\
26.4	0.01\\
26.41	0.01\\
26.42	0.01\\
26.43	0.01\\
26.44	0.01\\
26.45	0.01\\
26.46	0.01\\
26.47	0.01\\
26.48	0.01\\
26.49	0.01\\
26.5	0.01\\
26.51	0.01\\
26.52	0.01\\
26.53	0.01\\
26.54	0.01\\
26.55	0.01\\
26.56	0.01\\
26.57	0.01\\
26.58	0.01\\
26.59	0.01\\
26.6	0.01\\
26.61	0.01\\
26.62	0.01\\
26.63	0.01\\
26.64	0.01\\
26.65	0.01\\
26.66	0.01\\
26.67	0.01\\
26.68	0.01\\
26.69	0.01\\
26.7	0.01\\
26.71	0.01\\
26.72	0.01\\
26.73	0.01\\
26.74	0.01\\
26.75	0.01\\
26.76	0.01\\
26.77	0.01\\
26.78	0.01\\
26.79	0.01\\
26.8	0.01\\
26.81	0.01\\
26.82	0.01\\
26.83	0.01\\
26.84	0.01\\
26.85	0.01\\
26.86	0.01\\
26.87	0.01\\
26.88	0.01\\
26.89	0.01\\
26.9	0.01\\
26.91	0.01\\
26.92	0.01\\
26.93	0.01\\
26.94	0.01\\
26.95	0.01\\
26.96	0.01\\
26.97	0.01\\
26.98	0.01\\
26.99	0.01\\
27	0.01\\
27.01	0.01\\
27.02	0.01\\
27.03	0.01\\
27.04	0.01\\
27.05	0.01\\
27.06	0.01\\
27.07	0.01\\
27.08	0.01\\
27.09	0.01\\
27.1	0.01\\
27.11	0.01\\
27.12	0.01\\
27.13	0.01\\
27.14	0.01\\
27.15	0.01\\
27.16	0.01\\
27.17	0.01\\
27.18	0.01\\
27.19	0.01\\
27.2	0.01\\
27.21	0.01\\
27.22	0.01\\
27.23	0.01\\
27.24	0.01\\
27.25	0.01\\
27.26	0.01\\
27.27	0.01\\
27.28	0.01\\
27.29	0.01\\
27.3	0.01\\
27.31	0.01\\
27.32	0.01\\
27.33	0.01\\
27.34	0.01\\
27.35	0.01\\
27.36	0.01\\
27.37	0.01\\
27.38	0.01\\
27.39	0.01\\
27.4	0.01\\
27.41	0.01\\
27.42	0.01\\
27.43	0.01\\
27.44	0.01\\
27.45	0.01\\
27.46	0.01\\
27.47	0.01\\
27.48	0.01\\
27.49	0.01\\
27.5	0.01\\
27.51	0.01\\
27.52	0.01\\
27.53	0.01\\
27.54	0.01\\
27.55	0.01\\
27.56	0.01\\
27.57	0.01\\
27.58	0.01\\
27.59	0.01\\
27.6	0.01\\
27.61	0.01\\
27.62	0.01\\
27.63	0.01\\
27.64	0.01\\
27.65	0.01\\
27.66	0.01\\
27.67	0.01\\
27.68	0.01\\
27.69	0.01\\
27.7	0.01\\
27.71	0.01\\
27.72	0.01\\
27.73	0.01\\
27.74	0.01\\
27.75	0.01\\
27.76	0.01\\
27.77	0.01\\
27.78	0.01\\
27.79	0.01\\
27.8	0.01\\
27.81	0.01\\
27.82	0.01\\
27.83	0.01\\
27.84	0.01\\
27.85	0.01\\
27.86	0.01\\
27.87	0.01\\
27.88	0.01\\
27.89	0.01\\
27.9	0.01\\
27.91	0.01\\
27.92	0.01\\
27.93	0.01\\
27.94	0.01\\
27.95	0.01\\
27.96	0.01\\
27.97	0.01\\
27.98	0.01\\
27.99	0.01\\
28	0.01\\
28.01	0.01\\
28.02	0.01\\
28.03	0.01\\
28.04	0.01\\
28.05	0.01\\
28.06	0.01\\
28.07	0.01\\
28.08	0.01\\
28.09	0.01\\
28.1	0.01\\
28.11	0.01\\
28.12	0.01\\
28.13	0.01\\
28.14	0.01\\
28.15	0.01\\
28.16	0.01\\
28.17	0.01\\
28.18	0.01\\
28.19	0.01\\
28.2	0.01\\
28.21	0.01\\
28.22	0.01\\
28.23	0.01\\
28.24	0.01\\
28.25	0.01\\
28.26	0.01\\
28.27	0.01\\
28.28	0.01\\
28.29	0.01\\
28.3	0.01\\
28.31	0.01\\
28.32	0.01\\
28.33	0.01\\
28.34	0.01\\
28.35	0.01\\
28.36	0.01\\
28.37	0.01\\
28.38	0.01\\
28.39	0.01\\
28.4	0.01\\
28.41	0.01\\
28.42	0.01\\
28.43	0.01\\
28.44	0.01\\
28.45	0.01\\
28.46	0.01\\
28.47	0.01\\
28.48	0.01\\
28.49	0.01\\
28.5	0.01\\
28.51	0.01\\
28.52	0.01\\
28.53	0.01\\
28.54	0.01\\
28.55	0.01\\
28.56	0.01\\
28.57	0.01\\
28.58	0.01\\
28.59	0.01\\
28.6	0.01\\
28.61	0.01\\
28.62	0.01\\
28.63	0.01\\
28.64	0.01\\
28.65	0.01\\
28.66	0.01\\
28.67	0.01\\
28.68	0.01\\
28.69	0.01\\
28.7	0.01\\
28.71	0.01\\
28.72	0.01\\
28.73	0.01\\
28.74	0.01\\
28.75	0.01\\
28.76	0.01\\
28.77	0.01\\
28.78	0.01\\
28.79	0.01\\
28.8	0.01\\
28.81	0.01\\
28.82	0.01\\
28.83	0.01\\
28.84	0.01\\
28.85	0.01\\
28.86	0.01\\
28.87	0.01\\
28.88	0.01\\
28.89	0.01\\
28.9	0.01\\
28.91	0.01\\
28.92	0.01\\
28.93	0.01\\
28.94	0.01\\
28.95	0.01\\
28.96	0.01\\
28.97	0.01\\
28.98	0.01\\
28.99	0.01\\
29	0.01\\
29.01	0.01\\
29.02	0.01\\
29.03	0.01\\
29.04	0.01\\
29.05	0.01\\
29.06	0.01\\
29.07	0.01\\
29.08	0.01\\
29.09	0.01\\
29.1	0.01\\
29.11	0.01\\
29.12	0.01\\
29.13	0.01\\
29.14	0.01\\
29.15	0.01\\
29.16	0.01\\
29.17	0.01\\
29.18	0.01\\
29.19	0.01\\
29.2	0.01\\
29.21	0.01\\
29.22	0.01\\
29.23	0.01\\
29.24	0.01\\
29.25	0.01\\
29.26	0.01\\
29.27	0.01\\
29.28	0.01\\
29.29	0.01\\
29.3	0.01\\
29.31	0.01\\
29.32	0.01\\
29.33	0.01\\
29.34	0.01\\
29.35	0.01\\
29.36	0.01\\
29.37	0.01\\
29.38	0.01\\
29.39	0.01\\
29.4	0.01\\
29.41	0.01\\
29.42	0.01\\
29.43	0.01\\
29.44	0.01\\
29.45	0.01\\
29.46	0.01\\
29.47	0.01\\
29.48	0.01\\
29.49	0.01\\
29.5	0.01\\
29.51	0.01\\
29.52	0.01\\
29.53	0.01\\
29.54	0.01\\
29.55	0.01\\
29.56	0.01\\
29.57	0.01\\
29.58	0.01\\
29.59	0.01\\
29.6	0.01\\
29.61	0.01\\
29.62	0.01\\
29.63	0.01\\
29.64	0.01\\
29.65	0.01\\
29.66	0.01\\
29.67	0.01\\
29.68	0.01\\
29.69	0.01\\
29.7	0.01\\
29.71	0.01\\
29.72	0.01\\
29.73	0.01\\
29.74	0.01\\
29.75	0.01\\
29.76	0.01\\
29.77	0.01\\
29.78	0.01\\
29.79	0.01\\
29.8	0.01\\
29.81	0.01\\
29.82	0.01\\
29.83	0.01\\
29.84	0.01\\
29.85	0.01\\
29.86	0.01\\
29.87	0.01\\
29.88	0.01\\
29.89	0.01\\
29.9	0.01\\
29.91	0.01\\
29.92	0.01\\
29.93	0.01\\
29.94	0.01\\
29.95	0.01\\
29.96	0.01\\
29.97	0.01\\
29.98	0.01\\
29.99	0.01\\
30	0.01\\
30.01	0.01\\
30.02	0.01\\
30.03	0.01\\
30.04	0.01\\
30.05	0.01\\
30.06	0.01\\
30.07	0.01\\
30.08	0.01\\
30.09	0.01\\
30.1	0.01\\
30.11	0.01\\
30.12	0.01\\
30.13	0.01\\
30.14	0.01\\
30.15	0.01\\
30.16	0.01\\
30.17	0.01\\
30.18	0.01\\
30.19	0.01\\
30.2	0.01\\
30.21	0.01\\
30.22	0.01\\
30.23	0.01\\
30.24	0.01\\
30.25	0.01\\
30.26	0.01\\
30.27	0.01\\
30.28	0.01\\
30.29	0.01\\
30.3	0.01\\
30.31	0.01\\
30.32	0.01\\
30.33	0.01\\
30.34	0.01\\
30.35	0.01\\
30.36	0.01\\
30.37	0.01\\
30.38	0.01\\
30.39	0.01\\
30.4	0.01\\
30.41	0.01\\
30.42	0.01\\
30.43	0.01\\
30.44	0.01\\
30.45	0.01\\
30.46	0.01\\
30.47	0.01\\
30.48	0.01\\
30.49	0.01\\
30.5	0.01\\
30.51	0.01\\
30.52	0.01\\
30.53	0.01\\
30.54	0.01\\
30.55	0.01\\
30.56	0.01\\
30.57	0.01\\
30.58	0.01\\
30.59	0.01\\
30.6	0.01\\
30.61	0.01\\
30.62	0.01\\
30.63	0.01\\
30.64	0.01\\
30.65	0.01\\
30.66	0.01\\
30.67	0.01\\
30.68	0.01\\
30.69	0.01\\
30.7	0.01\\
30.71	0.01\\
30.72	0.01\\
30.73	0.01\\
30.74	0.01\\
30.75	0.01\\
30.76	0.01\\
30.77	0.01\\
30.78	0.01\\
30.79	0.01\\
30.8	0.01\\
30.81	0.01\\
30.82	0.01\\
30.83	0.01\\
30.84	0.01\\
30.85	0.01\\
30.86	0.01\\
30.87	0.01\\
30.88	0.01\\
30.89	0.01\\
30.9	0.01\\
30.91	0.01\\
30.92	0.01\\
30.93	0.01\\
30.94	0.01\\
30.95	0.01\\
30.96	0.01\\
30.97	0.01\\
30.98	0.01\\
30.99	0.01\\
31	0.01\\
31.01	0.01\\
31.02	0.01\\
31.03	0.01\\
31.04	0.01\\
31.05	0.01\\
31.06	0.01\\
31.07	0.01\\
31.08	0.01\\
31.09	0.01\\
31.1	0.01\\
31.11	0.01\\
31.12	0.01\\
31.13	0.01\\
31.14	0.01\\
31.15	0.01\\
31.16	0.01\\
31.17	0.01\\
31.18	0.01\\
31.19	0.01\\
31.2	0.01\\
31.21	0.01\\
31.22	0.01\\
31.23	0.01\\
31.24	0.01\\
31.25	0.01\\
31.26	0.01\\
31.27	0.01\\
31.28	0.01\\
31.29	0.01\\
31.3	0.01\\
31.31	0.01\\
31.32	0.01\\
31.33	0.01\\
31.34	0.01\\
31.35	0.01\\
31.36	0.01\\
31.37	0.01\\
31.38	0.01\\
31.39	0.01\\
31.4	0.01\\
31.41	0.01\\
31.42	0.01\\
31.43	0.01\\
31.44	0.01\\
31.45	0.01\\
31.46	0.01\\
31.47	0.01\\
31.48	0.01\\
31.49	0.01\\
31.5	0.01\\
31.51	0.01\\
31.52	0.01\\
31.53	0.01\\
31.54	0.01\\
31.55	0.01\\
31.56	0.01\\
31.57	0.01\\
31.58	0.01\\
31.59	0.01\\
31.6	0.01\\
31.61	0.01\\
31.62	0.01\\
31.63	0.01\\
31.64	0.01\\
31.65	0.01\\
31.66	0.01\\
31.67	0.01\\
31.68	0.01\\
31.69	0.01\\
31.7	0.01\\
31.71	0.01\\
31.72	0.01\\
31.73	0.01\\
31.74	0.01\\
31.75	0.01\\
31.76	0.01\\
31.77	0.01\\
31.78	0.01\\
31.79	0.01\\
31.8	0.01\\
31.81	0.01\\
31.82	0.01\\
31.83	0.01\\
31.84	0.01\\
31.85	0.01\\
31.86	0.01\\
31.87	0.01\\
31.88	0.01\\
31.89	0.01\\
31.9	0.01\\
31.91	0.01\\
31.92	0.01\\
31.93	0.01\\
31.94	0.01\\
31.95	0.01\\
31.96	0.01\\
31.97	0.01\\
31.98	0.01\\
31.99	0.01\\
32	0.01\\
32.01	0.01\\
32.02	0.01\\
32.03	0.01\\
32.04	0.01\\
32.05	0.01\\
32.06	0.01\\
32.07	0.01\\
32.08	0.01\\
32.09	0.01\\
32.1	0.01\\
32.11	0.01\\
32.12	0.01\\
32.13	0.01\\
32.14	0.01\\
32.15	0.01\\
32.16	0.01\\
32.17	0.01\\
32.18	0.01\\
32.19	0.01\\
32.2	0.01\\
32.21	0.01\\
32.22	0.01\\
32.23	0.01\\
32.24	0.01\\
32.25	0.01\\
32.26	0.01\\
32.27	0.01\\
32.28	0.01\\
32.29	0.01\\
32.3	0.01\\
32.31	0.01\\
32.32	0.01\\
32.33	0.01\\
32.34	0.01\\
32.35	0.01\\
32.36	0.01\\
32.37	0.01\\
32.38	0.01\\
32.39	0.01\\
32.4	0.01\\
32.41	0.01\\
32.42	0.01\\
32.43	0.01\\
32.44	0.01\\
32.45	0.01\\
32.46	0.01\\
32.47	0.01\\
32.48	0.01\\
32.49	0.01\\
32.5	0.01\\
32.51	0.01\\
32.52	0.01\\
32.53	0.01\\
32.54	0.01\\
32.55	0.01\\
32.56	0.01\\
32.57	0.01\\
32.58	0.01\\
32.59	0.01\\
32.6	0.01\\
32.61	0.01\\
32.62	0.01\\
32.63	0.01\\
32.64	0.01\\
32.65	0.01\\
32.66	0.01\\
32.67	0.01\\
32.68	0.01\\
32.69	0.01\\
32.7	0.01\\
32.71	0.01\\
32.72	0.01\\
32.73	0.01\\
32.74	0.01\\
32.75	0.01\\
32.76	0.01\\
32.77	0.01\\
32.78	0.01\\
32.79	0.01\\
32.8	0.01\\
32.81	0.01\\
32.82	0.01\\
32.83	0.01\\
32.84	0.01\\
32.85	0.01\\
32.86	0.01\\
32.87	0.01\\
32.88	0.01\\
32.89	0.01\\
32.9	0.01\\
32.91	0.01\\
32.92	0.01\\
32.93	0.01\\
32.94	0.01\\
32.95	0.01\\
32.96	0.01\\
32.97	0.01\\
32.98	0.01\\
32.99	0.01\\
33	0.01\\
33.01	0.01\\
33.02	0.01\\
33.03	0.01\\
33.04	0.01\\
33.05	0.01\\
33.06	0.01\\
33.07	0.01\\
33.08	0.01\\
33.09	0.01\\
33.1	0.01\\
33.11	0.01\\
33.12	0.01\\
33.13	0.01\\
33.14	0.01\\
33.15	0.01\\
33.16	0.01\\
33.17	0.01\\
33.18	0.01\\
33.19	0.01\\
33.2	0.01\\
33.21	0.01\\
33.22	0.01\\
33.23	0.01\\
33.24	0.01\\
33.25	0.01\\
33.26	0.01\\
33.27	0.01\\
33.28	0.01\\
33.29	0.01\\
33.3	0.01\\
33.31	0.01\\
33.32	0.01\\
33.33	0.01\\
33.34	0.01\\
33.35	0.01\\
33.36	0.01\\
33.37	0.01\\
33.38	0.01\\
33.39	0.01\\
33.4	0.01\\
33.41	0.01\\
33.42	0.01\\
33.43	0.01\\
33.44	0.01\\
33.45	0.01\\
33.46	0.01\\
33.47	0.01\\
33.48	0.01\\
33.49	0.01\\
33.5	0.01\\
33.51	0.01\\
33.52	0.01\\
33.53	0.01\\
33.54	0.01\\
33.55	0.01\\
33.56	0.01\\
33.57	0.01\\
33.58	0.01\\
33.59	0.01\\
33.6	0.01\\
33.61	0.01\\
33.62	0.01\\
33.63	0.01\\
33.64	0.01\\
33.65	0.01\\
33.66	0.01\\
33.67	0.01\\
33.68	0.01\\
33.69	0.01\\
33.7	0.01\\
33.71	0.01\\
33.72	0.01\\
33.73	0.01\\
33.74	0.01\\
33.75	0.01\\
33.76	0.01\\
33.77	0.01\\
33.78	0.01\\
33.79	0.01\\
33.8	0.01\\
33.81	0.01\\
33.82	0.01\\
33.83	0.01\\
33.84	0.01\\
33.85	0.01\\
33.86	0.01\\
33.87	0.01\\
33.88	0.01\\
33.89	0.01\\
33.9	0.01\\
33.91	0.01\\
33.92	0.01\\
33.93	0.01\\
33.94	0.01\\
33.95	0.01\\
33.96	0.01\\
33.97	0.01\\
33.98	0.01\\
33.99	0.01\\
34	0.01\\
34.01	0.01\\
34.02	0.01\\
34.03	0.01\\
34.04	0.01\\
34.05	0.01\\
34.06	0.01\\
34.07	0.01\\
34.08	0.01\\
34.09	0.01\\
34.1	0.01\\
34.11	0.01\\
34.12	0.01\\
34.13	0.01\\
34.14	0.01\\
34.15	0.01\\
34.16	0.01\\
34.17	0.01\\
34.18	0.01\\
34.19	0.01\\
34.2	0.01\\
34.21	0.01\\
34.22	0.01\\
34.23	0.01\\
34.24	0.01\\
34.25	0.01\\
34.26	0.01\\
34.27	0.01\\
34.28	0.01\\
34.29	0.01\\
34.3	0.01\\
34.31	0.01\\
34.32	0.01\\
34.33	0.01\\
34.34	0.01\\
34.35	0.01\\
34.36	0.01\\
34.37	0.01\\
34.38	0.01\\
34.39	0.01\\
34.4	0.01\\
34.41	0.01\\
34.42	0.01\\
34.43	0.01\\
34.44	0.01\\
34.45	0.01\\
34.46	0.01\\
34.47	0.01\\
34.48	0.01\\
34.49	0.01\\
34.5	0.01\\
34.51	0.01\\
34.52	0.01\\
34.53	0.01\\
34.54	0.01\\
34.55	0.01\\
34.56	0.01\\
34.57	0.01\\
34.58	0.01\\
34.59	0.01\\
34.6	0.01\\
34.61	0.01\\
34.62	0.01\\
34.63	0.01\\
34.64	0.01\\
34.65	0.01\\
34.66	0.01\\
34.67	0.01\\
34.68	0.01\\
34.69	0.01\\
34.7	0.01\\
34.71	0.01\\
34.72	0.01\\
34.73	0.01\\
34.74	0.01\\
34.75	0.01\\
34.76	0.01\\
34.77	0.01\\
34.78	0.01\\
34.79	0.01\\
34.8	0.01\\
34.81	0.01\\
34.82	0.01\\
34.83	0.01\\
34.84	0.01\\
34.85	0.01\\
34.86	0.01\\
34.87	0.01\\
34.88	0.01\\
34.89	0.01\\
34.9	0.01\\
34.91	0.01\\
34.92	0.01\\
34.93	0.01\\
34.94	0.01\\
34.95	0.01\\
34.96	0.01\\
34.97	0.01\\
34.98	0.01\\
34.99	0.01\\
35	0.01\\
35.01	0.01\\
35.02	0.01\\
35.03	0.01\\
35.04	0.01\\
35.05	0.01\\
35.06	0.01\\
35.07	0.01\\
35.08	0.01\\
35.09	0.01\\
35.1	0.01\\
35.11	0.01\\
35.12	0.01\\
35.13	0.01\\
35.14	0.01\\
35.15	0.01\\
35.16	0.01\\
35.17	0.01\\
35.18	0.01\\
35.19	0.01\\
35.2	0.01\\
35.21	0.01\\
35.22	0.01\\
35.23	0.01\\
35.24	0.01\\
35.25	0.01\\
35.26	0.01\\
35.27	0.01\\
35.28	0.01\\
35.29	0.01\\
35.3	0.01\\
35.31	0.01\\
35.32	0.01\\
35.33	0.01\\
35.34	0.01\\
35.35	0.01\\
35.36	0.01\\
35.37	0.01\\
35.38	0.01\\
35.39	0.01\\
35.4	0.01\\
35.41	0.01\\
35.42	0.01\\
35.43	0.01\\
35.44	0.01\\
35.45	0.01\\
35.46	0.01\\
35.47	0.01\\
35.48	0.01\\
35.49	0.01\\
35.5	0.01\\
35.51	0.01\\
35.52	0.01\\
35.53	0.01\\
35.54	0.01\\
35.55	0.01\\
35.56	0.01\\
35.57	0.01\\
35.58	0.01\\
35.59	0.01\\
35.6	0.01\\
35.61	0.01\\
35.62	0.01\\
35.63	0.01\\
35.64	0.01\\
35.65	0.01\\
35.66	0.01\\
35.67	0.01\\
35.68	0.01\\
35.69	0.01\\
35.7	0.01\\
35.71	0.01\\
35.72	0.01\\
35.73	0.01\\
35.74	0.01\\
35.75	0.01\\
35.76	0.01\\
35.77	0.01\\
35.78	0.01\\
35.79	0.01\\
35.8	0.01\\
35.81	0.01\\
35.82	0.01\\
35.83	0.01\\
35.84	0.01\\
35.85	0.01\\
35.86	0.01\\
35.87	0.01\\
35.88	0.01\\
35.89	0.01\\
35.9	0.01\\
35.91	0.01\\
35.92	0.01\\
35.93	0.01\\
35.94	0.01\\
35.95	0.01\\
35.96	0.01\\
35.97	0.01\\
35.98	0.01\\
35.99	0.01\\
36	0.01\\
36.01	0.01\\
36.02	0.01\\
36.03	0.01\\
36.04	0.01\\
36.05	0.01\\
36.06	0.01\\
36.07	0.01\\
36.08	0.01\\
36.09	0.01\\
36.1	0.01\\
36.11	0.01\\
36.12	0.01\\
36.13	0.01\\
36.14	0.01\\
36.15	0.01\\
36.16	0.01\\
36.17	0.01\\
36.18	0.01\\
36.19	0.01\\
36.2	0.01\\
36.21	0.01\\
36.22	0.01\\
36.23	0.01\\
36.24	0.01\\
36.25	0.01\\
36.26	0.01\\
36.27	0.01\\
36.28	0.01\\
36.29	0.01\\
36.3	0.01\\
36.31	0.01\\
36.32	0.01\\
36.33	0.01\\
36.34	0.01\\
36.35	0.01\\
36.36	0.01\\
36.37	0.01\\
36.38	0.01\\
36.39	0.01\\
36.4	0.01\\
36.41	0.01\\
36.42	0.01\\
36.43	0.01\\
36.44	0.01\\
36.45	0.01\\
36.46	0.01\\
36.47	0.01\\
36.48	0.01\\
36.49	0.01\\
36.5	0.01\\
36.51	0.01\\
36.52	0.01\\
36.53	0.01\\
36.54	0.01\\
36.55	0.01\\
36.56	0.01\\
36.57	0.01\\
36.58	0.01\\
36.59	0.01\\
36.6	0.01\\
36.61	0.01\\
36.62	0.01\\
36.63	0.01\\
36.64	0.01\\
36.65	0.01\\
36.66	0.01\\
36.67	0.01\\
36.68	0.01\\
36.69	0.01\\
36.7	0.01\\
36.71	0.01\\
36.72	0.01\\
36.73	0.01\\
36.74	0.01\\
36.75	0.01\\
36.76	0.01\\
36.77	0.01\\
36.78	0.01\\
36.79	0.01\\
36.8	0.01\\
36.81	0.01\\
36.82	0.01\\
36.83	0.01\\
36.84	0.01\\
36.85	0.01\\
36.86	0.01\\
36.87	0.01\\
36.88	0.01\\
36.89	0.01\\
36.9	0.01\\
36.91	0.01\\
36.92	0.01\\
36.93	0.01\\
36.94	0.01\\
36.95	0.01\\
36.96	0.01\\
36.97	0.01\\
36.98	0.01\\
36.99	0.01\\
37	0.01\\
37.01	0.01\\
37.02	0.01\\
37.03	0.01\\
37.04	0.01\\
37.05	0.01\\
37.06	0.01\\
37.07	0.01\\
37.08	0.01\\
37.09	0.01\\
37.1	0.01\\
37.11	0.01\\
37.12	0.01\\
37.13	0.01\\
37.14	0.01\\
37.15	0.01\\
37.16	0.01\\
37.17	0.01\\
37.18	0.01\\
37.19	0.01\\
37.2	0.01\\
37.21	0.01\\
37.22	0.01\\
37.23	0.01\\
37.24	0.01\\
37.25	0.01\\
37.26	0.01\\
37.27	0.01\\
37.28	0.01\\
37.29	0.01\\
37.3	0.01\\
37.31	0.01\\
37.32	0.01\\
37.33	0.01\\
37.34	0.01\\
37.35	0.01\\
37.36	0.01\\
37.37	0.01\\
37.38	0.01\\
37.39	0.01\\
37.4	0.01\\
37.41	0.01\\
37.42	0.01\\
37.43	0.01\\
37.44	0.01\\
37.45	0.01\\
37.46	0.01\\
37.47	0.01\\
37.48	0.01\\
37.49	0.01\\
37.5	0.01\\
37.51	0.01\\
37.52	0.01\\
37.53	0.01\\
37.54	0.01\\
37.55	0.01\\
37.56	0.01\\
37.57	0.01\\
37.58	0.01\\
37.59	0.01\\
37.6	0.01\\
37.61	0.01\\
37.62	0.01\\
37.63	0.01\\
37.64	0.01\\
37.65	0.01\\
37.66	0.01\\
37.67	0.01\\
37.68	0.01\\
37.69	0.01\\
37.7	0.01\\
37.71	0.01\\
37.72	0.01\\
37.73	0.01\\
37.74	0.01\\
37.75	0.01\\
37.76	0.01\\
37.77	0.01\\
37.78	0.01\\
37.79	0.01\\
37.8	0.01\\
37.81	0.01\\
37.82	0.01\\
37.83	0.01\\
37.84	0.01\\
37.85	0.01\\
37.86	0.01\\
37.87	0.01\\
37.88	0.01\\
37.89	0.01\\
37.9	0.01\\
37.91	0.01\\
37.92	0.01\\
37.93	0.01\\
37.94	0.01\\
37.95	0.01\\
37.96	0.01\\
37.97	0.01\\
37.98	0.01\\
37.99	0.01\\
38	0.01\\
38.01	0.01\\
38.02	0.01\\
38.03	0.01\\
38.04	0.01\\
38.05	0.01\\
38.06	0.01\\
38.07	0.01\\
38.08	0.01\\
38.09	0.01\\
38.1	0.01\\
38.11	0.01\\
38.12	0.01\\
38.13	0.01\\
38.14	0.01\\
38.15	0.01\\
38.16	0.01\\
38.17	0.01\\
38.18	0.01\\
38.19	0.01\\
38.2	0.01\\
38.21	0.01\\
38.22	0.01\\
38.23	0.01\\
38.24	0.01\\
38.25	0.01\\
38.26	0.01\\
38.27	0.01\\
38.28	0.01\\
38.29	0.01\\
38.3	0.01\\
38.31	0.01\\
38.32	0.01\\
38.33	0.01\\
38.34	0.01\\
38.35	0.01\\
38.36	0.01\\
38.37	0.01\\
38.38	0.01\\
38.39	0.01\\
38.4	0.01\\
38.41	0.01\\
38.42	0.01\\
38.43	0.01\\
38.44	0.01\\
38.45	0.01\\
38.46	0.01\\
38.47	0.01\\
38.48	0.01\\
38.49	0.01\\
38.5	0.01\\
38.51	0.01\\
38.52	0.01\\
38.53	0.01\\
38.54	0.01\\
38.55	0.01\\
38.56	0.01\\
38.57	0.01\\
38.58	0.01\\
38.59	0.01\\
38.6	0.01\\
38.61	0.01\\
38.62	0.01\\
38.63	0.01\\
38.64	0.01\\
38.65	0.01\\
38.66	0.01\\
38.67	0.01\\
38.68	0.01\\
38.69	0.01\\
38.7	0.01\\
38.71	0.01\\
38.72	0.01\\
38.73	0.01\\
38.74	0.01\\
38.75	0.01\\
38.76	0.01\\
38.77	0.01\\
38.78	0.01\\
38.79	0.01\\
38.8	0.01\\
38.81	0.01\\
38.82	0.01\\
38.83	0.01\\
38.84	0.01\\
38.85	0.01\\
38.86	0.01\\
38.87	0.01\\
38.88	0.01\\
38.89	0.01\\
38.9	0.01\\
38.91	0.01\\
38.92	0.01\\
38.93	0.01\\
38.94	0.01\\
38.95	0.01\\
38.96	0.01\\
38.97	0.01\\
38.98	0.01\\
38.99	0.01\\
39	0.01\\
39.01	0.01\\
39.02	0.01\\
39.03	0.01\\
39.04	0.01\\
39.05	0.01\\
39.06	0.01\\
39.07	0.01\\
39.08	0.01\\
39.09	0.01\\
39.1	0.01\\
39.11	0.01\\
39.12	0.01\\
39.13	0.01\\
39.14	0.01\\
39.15	0.01\\
39.16	0.01\\
39.17	0.01\\
39.18	0.01\\
39.19	0.01\\
39.2	0.01\\
39.21	0.01\\
39.22	0.01\\
39.23	0.01\\
39.24	0.01\\
39.25	0.01\\
39.26	0.01\\
39.27	0.01\\
39.28	0.01\\
39.29	0.01\\
39.3	0.01\\
39.31	0.01\\
39.32	0.01\\
39.33	0.01\\
39.34	0.01\\
39.35	0.01\\
39.36	0.01\\
39.37	0.01\\
39.38	0.01\\
39.39	0.01\\
39.4	0.01\\
39.41	0.01\\
39.42	0.01\\
39.43	0.01\\
39.44	0.01\\
39.45	0.01\\
39.46	0.01\\
39.47	0.01\\
39.48	0.01\\
39.49	0.01\\
39.5	0.01\\
39.51	0.01\\
39.52	0.01\\
39.53	0.01\\
39.54	0.01\\
39.55	0.01\\
39.56	0.01\\
39.57	0.01\\
39.58	0.01\\
39.59	0.01\\
39.6	0.01\\
39.61	0.01\\
39.62	0.01\\
39.63	0.01\\
39.64	0.01\\
39.65	0.01\\
39.66	0.01\\
39.67	0.01\\
39.68	0.01\\
39.69	0.01\\
39.7	0.01\\
39.71	0.01\\
39.72	0.01\\
39.73	0.01\\
39.74	0.01\\
39.75	0.01\\
39.76	0.01\\
39.77	0.01\\
39.78	0.01\\
39.79	0.01\\
39.8	0.01\\
39.81	0.01\\
39.82	0.01\\
39.83	0.01\\
39.84	0.01\\
39.85	0.01\\
39.86	0.01\\
39.87	0.01\\
39.88	0.01\\
39.89	0.01\\
39.9	0.01\\
39.91	0.01\\
39.92	0.01\\
39.93	0.01\\
39.94	0.01\\
39.95	0.01\\
39.96	0.01\\
39.97	0.01\\
39.98	0.01\\
39.99	0.01\\
40	0.01\\
40.01	0.01\\
};
\addplot [color=blue,dashed,forget plot]
  table[row sep=crcr]{%
40.01	0.01\\
40.02	0.01\\
40.03	0.01\\
40.04	0.01\\
40.05	0.01\\
40.06	0.01\\
40.07	0.01\\
40.08	0.01\\
40.09	0.01\\
40.1	0.01\\
40.11	0.01\\
40.12	0.01\\
40.13	0.01\\
40.14	0.01\\
40.15	0.01\\
40.16	0.01\\
40.17	0.01\\
40.18	0.01\\
40.19	0.01\\
40.2	0.01\\
40.21	0.01\\
40.22	0.01\\
40.23	0.01\\
40.24	0.01\\
40.25	0.01\\
40.26	0.01\\
40.27	0.01\\
40.28	0.01\\
40.29	0.01\\
40.3	0.01\\
40.31	0.01\\
40.32	0.01\\
40.33	0.01\\
40.34	0.01\\
40.35	0.01\\
40.36	0.01\\
40.37	0.01\\
40.38	0.01\\
40.39	0.01\\
40.4	0.01\\
40.41	0.01\\
40.42	0.01\\
40.43	0.01\\
40.44	0.01\\
40.45	0.01\\
40.46	0.01\\
40.47	0.01\\
40.48	0.01\\
40.49	0.01\\
40.5	0.01\\
40.51	0.01\\
40.52	0.01\\
40.53	0.01\\
40.54	0.01\\
40.55	0.01\\
40.56	0.01\\
40.57	0.01\\
40.58	0.01\\
40.59	0.01\\
40.6	0.01\\
40.61	0.01\\
40.62	0.01\\
40.63	0.01\\
40.64	0.01\\
40.65	0.01\\
40.66	0.01\\
40.67	0.01\\
40.68	0.01\\
40.69	0.01\\
40.7	0.01\\
40.71	0.01\\
40.72	0.01\\
40.73	0.01\\
40.74	0.01\\
40.75	0.01\\
40.76	0.01\\
40.77	0.01\\
40.78	0.01\\
40.79	0.01\\
40.8	0.01\\
40.81	0.01\\
40.82	0.01\\
40.83	0.01\\
40.84	0.01\\
40.85	0.01\\
40.86	0.01\\
40.87	0.01\\
40.88	0.01\\
40.89	0.01\\
40.9	0.01\\
40.91	0.01\\
40.92	0.01\\
40.93	0.01\\
40.94	0.01\\
40.95	0.01\\
40.96	0.01\\
40.97	0.01\\
40.98	0.01\\
40.99	0.01\\
41	0.01\\
41.01	0.01\\
41.02	0.01\\
41.03	0.01\\
41.04	0.01\\
41.05	0.01\\
41.06	0.01\\
41.07	0.01\\
41.08	0.01\\
41.09	0.01\\
41.1	0.01\\
41.11	0.01\\
41.12	0.01\\
41.13	0.01\\
41.14	0.01\\
41.15	0.01\\
41.16	0.01\\
41.17	0.01\\
41.18	0.01\\
41.19	0.01\\
41.2	0.01\\
41.21	0.01\\
41.22	0.01\\
41.23	0.01\\
41.24	0.01\\
41.25	0.01\\
41.26	0.01\\
41.27	0.01\\
41.28	0.01\\
41.29	0.01\\
41.3	0.01\\
41.31	0.01\\
41.32	0.01\\
41.33	0.01\\
41.34	0.01\\
41.35	0.01\\
41.36	0.01\\
41.37	0.01\\
41.38	0.01\\
41.39	0.01\\
41.4	0.01\\
41.41	0.01\\
41.42	0.01\\
41.43	0.01\\
41.44	0.01\\
41.45	0.01\\
41.46	0.01\\
41.47	0.01\\
41.48	0.01\\
41.49	0.01\\
41.5	0.01\\
41.51	0.01\\
41.52	0.01\\
41.53	0.01\\
41.54	0.01\\
41.55	0.01\\
41.56	0.01\\
41.57	0.01\\
41.58	0.01\\
41.59	0.01\\
41.6	0.01\\
41.61	0.01\\
41.62	0.01\\
41.63	0.01\\
41.64	0.01\\
41.65	0.01\\
41.66	0.01\\
41.67	0.01\\
41.68	0.01\\
41.69	0.01\\
41.7	0.01\\
41.71	0.01\\
41.72	0.01\\
41.73	0.01\\
41.74	0.01\\
41.75	0.01\\
41.76	0.01\\
41.77	0.01\\
41.78	0.01\\
41.79	0.01\\
41.8	0.01\\
41.81	0.01\\
41.82	0.01\\
41.83	0.01\\
41.84	0.01\\
41.85	0.01\\
41.86	0.01\\
41.87	0.01\\
41.88	0.01\\
41.89	0.01\\
41.9	0.01\\
41.91	0.01\\
41.92	0.01\\
41.93	0.01\\
41.94	0.01\\
41.95	0.01\\
41.96	0.01\\
41.97	0.01\\
41.98	0.01\\
41.99	0.01\\
42	0.01\\
42.01	0.01\\
42.02	0.01\\
42.03	0.01\\
42.04	0.01\\
42.05	0.01\\
42.06	0.01\\
42.07	0.01\\
42.08	0.01\\
42.09	0.01\\
42.1	0.01\\
42.11	0.01\\
42.12	0.01\\
42.13	0.01\\
42.14	0.01\\
42.15	0.01\\
42.16	0.01\\
42.17	0.01\\
42.18	0.01\\
42.19	0.01\\
42.2	0.01\\
42.21	0.01\\
42.22	0.01\\
42.23	0.01\\
42.24	0.01\\
42.25	0.01\\
42.26	0.01\\
42.27	0.01\\
42.28	0.01\\
42.29	0.01\\
42.3	0.01\\
42.31	0.01\\
42.32	0.01\\
42.33	0.01\\
42.34	0.01\\
42.35	0.01\\
42.36	0.01\\
42.37	0.01\\
42.38	0.01\\
42.39	0.01\\
42.4	0.01\\
42.41	0.01\\
42.42	0.01\\
42.43	0.01\\
42.44	0.01\\
42.45	0.01\\
42.46	0.01\\
42.47	0.01\\
42.48	0.01\\
42.49	0.01\\
42.5	0.01\\
42.51	0.01\\
42.52	0.01\\
42.53	0.01\\
42.54	0.01\\
42.55	0.01\\
42.56	0.01\\
42.57	0.01\\
42.58	0.01\\
42.59	0.01\\
42.6	0.01\\
42.61	0.01\\
42.62	0.01\\
42.63	0.01\\
42.64	0.01\\
42.65	0.01\\
42.66	0.01\\
42.67	0.01\\
42.68	0.01\\
42.69	0.01\\
42.7	0.01\\
42.71	0.01\\
42.72	0.01\\
42.73	0.01\\
42.74	0.01\\
42.75	0.01\\
42.76	0.01\\
42.77	0.01\\
42.78	0.01\\
42.79	0.01\\
42.8	0.01\\
42.81	0.01\\
42.82	0.01\\
42.83	0.01\\
42.84	0.01\\
42.85	0.01\\
42.86	0.01\\
42.87	0.01\\
42.88	0.01\\
42.89	0.01\\
42.9	0.01\\
42.91	0.01\\
42.92	0.01\\
42.93	0.01\\
42.94	0.01\\
42.95	0.01\\
42.96	0.01\\
42.97	0.01\\
42.98	0.01\\
42.99	0.01\\
43	0.01\\
43.01	0.01\\
43.02	0.01\\
43.03	0.01\\
43.04	0.01\\
43.05	0.01\\
43.06	0.01\\
43.07	0.01\\
43.08	0.01\\
43.09	0.01\\
43.1	0.01\\
43.11	0.01\\
43.12	0.01\\
43.13	0.01\\
43.14	0.01\\
43.15	0.01\\
43.16	0.01\\
43.17	0.01\\
43.18	0.01\\
43.19	0.01\\
43.2	0.01\\
43.21	0.01\\
43.22	0.01\\
43.23	0.01\\
43.24	0.01\\
43.25	0.01\\
43.26	0.01\\
43.27	0.01\\
43.28	0.01\\
43.29	0.01\\
43.3	0.01\\
43.31	0.01\\
43.32	0.01\\
43.33	0.01\\
43.34	0.01\\
43.35	0.01\\
43.36	0.01\\
43.37	0.01\\
43.38	0.01\\
43.39	0.01\\
43.4	0.01\\
43.41	0.01\\
43.42	0.01\\
43.43	0.01\\
43.44	0.01\\
43.45	0.01\\
43.46	0.01\\
43.47	0.01\\
43.48	0.01\\
43.49	0.01\\
43.5	0.01\\
43.51	0.01\\
43.52	0.01\\
43.53	0.01\\
43.54	0.01\\
43.55	0.01\\
43.56	0.01\\
43.57	0.01\\
43.58	0.01\\
43.59	0.01\\
43.6	0.01\\
43.61	0.01\\
43.62	0.01\\
43.63	0.01\\
43.64	0.01\\
43.65	0.01\\
43.66	0.01\\
43.67	0.01\\
43.68	0.01\\
43.69	0.01\\
43.7	0.01\\
43.71	0.01\\
43.72	0.01\\
43.73	0.01\\
43.74	0.01\\
43.75	0.01\\
43.76	0.01\\
43.77	0.01\\
43.78	0.01\\
43.79	0.01\\
43.8	0.01\\
43.81	0.01\\
43.82	0.01\\
43.83	0.01\\
43.84	0.01\\
43.85	0.01\\
43.86	0.01\\
43.87	0.01\\
43.88	0.01\\
43.89	0.01\\
43.9	0.01\\
43.91	0.01\\
43.92	0.01\\
43.93	0.01\\
43.94	0.01\\
43.95	0.01\\
43.96	0.01\\
43.97	0.01\\
43.98	0.01\\
43.99	0.01\\
44	0.01\\
44.01	0.01\\
44.02	0.01\\
44.03	0.01\\
44.04	0.01\\
44.05	0.01\\
44.06	0.01\\
44.07	0.01\\
44.08	0.01\\
44.09	0.01\\
44.1	0.01\\
44.11	0.01\\
44.12	0.01\\
44.13	0.01\\
44.14	0.01\\
44.15	0.01\\
44.16	0.01\\
44.17	0.01\\
44.18	0.01\\
44.19	0.01\\
44.2	0.01\\
44.21	0.01\\
44.22	0.01\\
44.23	0.01\\
44.24	0.01\\
44.25	0.01\\
44.26	0.01\\
44.27	0.01\\
44.28	0.01\\
44.29	0.01\\
44.3	0.01\\
44.31	0.01\\
44.32	0.01\\
44.33	0.01\\
44.34	0.01\\
44.35	0.01\\
44.36	0.01\\
44.37	0.01\\
44.38	0.01\\
44.39	0.01\\
44.4	0.01\\
44.41	0.01\\
44.42	0.01\\
44.43	0.01\\
44.44	0.01\\
44.45	0.01\\
44.46	0.01\\
44.47	0.01\\
44.48	0.01\\
44.49	0.01\\
44.5	0.01\\
44.51	0.01\\
44.52	0.01\\
44.53	0.01\\
44.54	0.01\\
44.55	0.01\\
44.56	0.01\\
44.57	0.01\\
44.58	0.01\\
44.59	0.01\\
44.6	0.01\\
44.61	0.01\\
44.62	0.01\\
44.63	0.01\\
44.64	0.01\\
44.65	0.01\\
44.66	0.01\\
44.67	0.01\\
44.68	0.01\\
44.69	0.01\\
44.7	0.01\\
44.71	0.01\\
44.72	0.01\\
44.73	0.01\\
44.74	0.01\\
44.75	0.01\\
44.76	0.01\\
44.77	0.01\\
44.78	0.01\\
44.79	0.01\\
44.8	0.01\\
44.81	0.01\\
44.82	0.01\\
44.83	0.01\\
44.84	0.01\\
44.85	0.01\\
44.86	0.01\\
44.87	0.01\\
44.88	0.01\\
44.89	0.01\\
44.9	0.01\\
44.91	0.01\\
44.92	0.01\\
44.93	0.01\\
44.94	0.01\\
44.95	0.01\\
44.96	0.01\\
44.97	0.01\\
44.98	0.01\\
44.99	0.01\\
45	0.01\\
45.01	0.01\\
45.02	0.01\\
45.03	0.01\\
45.04	0.01\\
45.05	0.01\\
45.06	0.01\\
45.07	0.01\\
45.08	0.01\\
45.09	0.01\\
45.1	0.01\\
45.11	0.01\\
45.12	0.01\\
45.13	0.01\\
45.14	0.01\\
45.15	0.01\\
45.16	0.01\\
45.17	0.01\\
45.18	0.01\\
45.19	0.01\\
45.2	0.01\\
45.21	0.01\\
45.22	0.01\\
45.23	0.01\\
45.24	0.01\\
45.25	0.01\\
45.26	0.01\\
45.27	0.01\\
45.28	0.01\\
45.29	0.01\\
45.3	0.01\\
45.31	0.01\\
45.32	0.01\\
45.33	0.01\\
45.34	0.01\\
45.35	0.01\\
45.36	0.01\\
45.37	0.01\\
45.38	0.01\\
45.39	0.01\\
45.4	0.01\\
45.41	0.01\\
45.42	0.01\\
45.43	0.01\\
45.44	0.01\\
45.45	0.01\\
45.46	0.01\\
45.47	0.01\\
45.48	0.01\\
45.49	0.01\\
45.5	0.01\\
45.51	0.01\\
45.52	0.01\\
45.53	0.01\\
45.54	0.01\\
45.55	0.01\\
45.56	0.01\\
45.57	0.01\\
45.58	0.01\\
45.59	0.01\\
45.6	0.01\\
45.61	0.01\\
45.62	0.01\\
45.63	0.01\\
45.64	0.01\\
45.65	0.01\\
45.66	0.01\\
45.67	0.01\\
45.68	0.01\\
45.69	0.01\\
45.7	0.01\\
45.71	0.01\\
45.72	0.01\\
45.73	0.01\\
45.74	0.01\\
45.75	0.01\\
45.76	0.01\\
45.77	0.01\\
45.78	0.01\\
45.79	0.01\\
45.8	0.01\\
45.81	0.01\\
45.82	0.01\\
45.83	0.01\\
45.84	0.01\\
45.85	0.01\\
45.86	0.01\\
45.87	0.01\\
45.88	0.01\\
45.89	0.01\\
45.9	0.01\\
45.91	0.01\\
45.92	0.01\\
45.93	0.01\\
45.94	0.01\\
45.95	0.01\\
45.96	0.01\\
45.97	0.01\\
45.98	0.01\\
45.99	0.01\\
46	0.01\\
46.01	0.01\\
46.02	0.01\\
46.03	0.01\\
46.04	0.01\\
46.05	0.01\\
46.06	0.01\\
46.07	0.01\\
46.08	0.01\\
46.09	0.01\\
46.1	0.01\\
46.11	0.01\\
46.12	0.01\\
46.13	0.01\\
46.14	0.01\\
46.15	0.01\\
46.16	0.01\\
46.17	0.01\\
46.18	0.01\\
46.19	0.01\\
46.2	0.01\\
46.21	0.01\\
46.22	0.01\\
46.23	0.01\\
46.24	0.01\\
46.25	0.01\\
46.26	0.01\\
46.27	0.01\\
46.28	0.01\\
46.29	0.01\\
46.3	0.01\\
46.31	0.01\\
46.32	0.01\\
46.33	0.01\\
46.34	0.01\\
46.35	0.01\\
46.36	0.01\\
46.37	0.01\\
46.38	0.01\\
46.39	0.01\\
46.4	0.01\\
46.41	0.01\\
46.42	0.01\\
46.43	0.01\\
46.44	0.01\\
46.45	0.01\\
46.46	0.01\\
46.47	0.01\\
46.48	0.01\\
46.49	0.01\\
46.5	0.01\\
46.51	0.01\\
46.52	0.01\\
46.53	0.01\\
46.54	0.01\\
46.55	0.01\\
46.56	0.01\\
46.57	0.01\\
46.58	0.01\\
46.59	0.01\\
46.6	0.01\\
46.61	0.01\\
46.62	0.01\\
46.63	0.01\\
46.64	0.01\\
46.65	0.01\\
46.66	0.01\\
46.67	0.01\\
46.68	0.01\\
46.69	0.01\\
46.7	0.01\\
46.71	0.01\\
46.72	0.01\\
46.73	0.01\\
46.74	0.01\\
46.75	0.01\\
46.76	0.01\\
46.77	0.01\\
46.78	0.01\\
46.79	0.01\\
46.8	0.01\\
46.81	0.01\\
46.82	0.01\\
46.83	0.01\\
46.84	0.01\\
46.85	0.01\\
46.86	0.01\\
46.87	0.01\\
46.88	0.01\\
46.89	0.01\\
46.9	0.01\\
46.91	0.01\\
46.92	0.01\\
46.93	0.01\\
46.94	0.01\\
46.95	0.01\\
46.96	0.01\\
46.97	0.01\\
46.98	0.01\\
46.99	0.01\\
47	0.01\\
47.01	0.01\\
47.02	0.01\\
47.03	0.01\\
47.04	0.01\\
47.05	0.01\\
47.06	0.01\\
47.07	0.01\\
47.08	0.01\\
47.09	0.01\\
47.1	0.01\\
47.11	0.01\\
47.12	0.01\\
47.13	0.01\\
47.14	0.01\\
47.15	0.01\\
47.16	0.01\\
47.17	0.01\\
47.18	0.01\\
47.19	0.01\\
47.2	0.01\\
47.21	0.01\\
47.22	0.01\\
47.23	0.01\\
47.24	0.01\\
47.25	0.01\\
47.26	0.01\\
47.27	0.01\\
47.28	0.01\\
47.29	0.01\\
47.3	0.01\\
47.31	0.01\\
47.32	0.01\\
47.33	0.01\\
47.34	0.01\\
47.35	0.01\\
47.36	0.01\\
47.37	0.01\\
47.38	0.01\\
47.39	0.01\\
47.4	0.01\\
47.41	0.01\\
47.42	0.01\\
47.43	0.01\\
47.44	0.01\\
47.45	0.01\\
47.46	0.01\\
47.47	0.01\\
47.48	0.01\\
47.49	0.01\\
47.5	0.01\\
47.51	0.01\\
47.52	0.01\\
47.53	0.01\\
47.54	0.01\\
47.55	0.01\\
47.56	0.01\\
47.57	0.01\\
47.58	0.01\\
47.59	0.01\\
47.6	0.01\\
47.61	0.01\\
47.62	0.01\\
47.63	0.01\\
47.64	0.01\\
47.65	0.01\\
47.66	0.01\\
47.67	0.01\\
47.68	0.01\\
47.69	0.01\\
47.7	0.01\\
47.71	0.01\\
47.72	0.01\\
47.73	0.01\\
47.74	0.01\\
47.75	0.01\\
47.76	0.01\\
47.77	0.01\\
47.78	0.01\\
47.79	0.01\\
47.8	0.01\\
47.81	0.01\\
47.82	0.01\\
47.83	0.01\\
47.84	0.01\\
47.85	0.01\\
47.86	0.01\\
47.87	0.01\\
47.88	0.01\\
47.89	0.01\\
47.9	0.01\\
47.91	0.01\\
47.92	0.01\\
47.93	0.01\\
47.94	0.01\\
47.95	0.01\\
47.96	0.01\\
47.97	0.01\\
47.98	0.01\\
47.99	0.01\\
48	0.01\\
48.01	0.01\\
48.02	0.01\\
48.03	0.01\\
48.04	0.01\\
48.05	0.01\\
48.06	0.01\\
48.07	0.01\\
48.08	0.01\\
48.09	0.01\\
48.1	0.01\\
48.11	0.01\\
48.12	0.01\\
48.13	0.01\\
48.14	0.01\\
48.15	0.01\\
48.16	0.01\\
48.17	0.01\\
48.18	0.01\\
48.19	0.01\\
48.2	0.01\\
48.21	0.01\\
48.22	0.01\\
48.23	0.01\\
48.24	0.01\\
48.25	0.01\\
48.26	0.01\\
48.27	0.01\\
48.28	0.01\\
48.29	0.01\\
48.3	0.01\\
48.31	0.01\\
48.32	0.01\\
48.33	0.01\\
48.34	0.01\\
48.35	0.01\\
48.36	0.01\\
48.37	0.01\\
48.38	0.01\\
48.39	0.01\\
48.4	0.01\\
48.41	0.01\\
48.42	0.01\\
48.43	0.01\\
48.44	0.01\\
48.45	0.01\\
48.46	0.01\\
48.47	0.01\\
48.48	0.01\\
48.49	0.01\\
48.5	0.01\\
48.51	0.01\\
48.52	0.01\\
48.53	0.01\\
48.54	0.01\\
48.55	0.01\\
48.56	0.01\\
48.57	0.01\\
48.58	0.01\\
48.59	0.01\\
48.6	0.01\\
48.61	0.01\\
48.62	0.01\\
48.63	0.01\\
48.64	0.01\\
48.65	0.01\\
48.66	0.01\\
48.67	0.01\\
48.68	0.01\\
48.69	0.01\\
48.7	0.01\\
48.71	0.01\\
48.72	0.01\\
48.73	0.01\\
48.74	0.01\\
48.75	0.01\\
48.76	0.01\\
48.77	0.01\\
48.78	0.01\\
48.79	0.01\\
48.8	0.01\\
48.81	0.01\\
48.82	0.01\\
48.83	0.01\\
48.84	0.01\\
48.85	0.01\\
48.86	0.01\\
48.87	0.01\\
48.88	0.01\\
48.89	0.01\\
48.9	0.01\\
48.91	0.01\\
48.92	0.01\\
48.93	0.01\\
48.94	0.01\\
48.95	0.01\\
48.96	0.01\\
48.97	0.01\\
48.98	0.01\\
48.99	0.01\\
49	0.01\\
49.01	0.01\\
49.02	0.01\\
49.03	0.01\\
49.04	0.01\\
49.05	0.01\\
49.06	0.01\\
49.07	0.01\\
49.08	0.01\\
49.09	0.01\\
49.1	0.01\\
49.11	0.01\\
49.12	0.01\\
49.13	0.01\\
49.14	0.01\\
49.15	0.01\\
49.16	0.01\\
49.17	0.01\\
49.18	0.01\\
49.19	0.01\\
49.2	0.01\\
49.21	0.01\\
49.22	0.01\\
49.23	0.01\\
49.24	0.01\\
49.25	0.01\\
49.26	0.01\\
49.27	0.01\\
49.28	0.01\\
49.29	0.01\\
49.3	0.01\\
49.31	0.01\\
49.32	0.01\\
49.33	0.01\\
49.34	0.01\\
49.35	0.01\\
49.36	0.01\\
49.37	0.01\\
49.38	0.01\\
49.39	0.01\\
49.4	0.01\\
49.41	0.01\\
49.42	0.01\\
49.43	0.01\\
49.44	0.01\\
49.45	0.01\\
49.46	0.01\\
49.47	0.01\\
49.48	0.01\\
49.49	0.01\\
49.5	0.01\\
49.51	0.01\\
49.52	0.01\\
49.53	0.01\\
49.54	0.01\\
49.55	0.01\\
49.56	0.01\\
49.57	0.01\\
49.58	0.01\\
49.59	0.01\\
49.6	0.01\\
49.61	0.01\\
49.62	0.01\\
49.63	0.01\\
49.64	0.01\\
49.65	0.01\\
49.66	0.01\\
49.67	0.01\\
49.68	0.01\\
49.69	0.01\\
49.7	0.01\\
49.71	0.01\\
49.72	0.01\\
49.73	0.01\\
49.74	0.01\\
49.75	0.01\\
49.76	0.01\\
49.77	0.01\\
49.78	0.01\\
49.79	0.01\\
49.8	0.01\\
49.81	0.01\\
49.82	0.01\\
49.83	0.01\\
49.84	0.01\\
49.85	0.01\\
49.86	0.01\\
49.87	0.01\\
49.88	0.01\\
49.89	0.01\\
49.9	0.01\\
49.91	0.01\\
49.92	0.01\\
49.93	0.01\\
49.94	0.01\\
49.95	0.01\\
49.96	0.01\\
49.97	0.01\\
49.98	0.01\\
49.99	0.01\\
50	0.01\\
50.01	0.01\\
50.02	0.01\\
50.03	0.01\\
50.04	0.01\\
50.05	0.01\\
50.06	0.01\\
50.07	0.01\\
50.08	0.01\\
50.09	0.01\\
50.1	0.01\\
50.11	0.01\\
50.12	0.01\\
50.13	0.01\\
50.14	0.01\\
50.15	0.01\\
50.16	0.01\\
50.17	0.01\\
50.18	0.01\\
50.19	0.01\\
50.2	0.01\\
50.21	0.01\\
50.22	0.01\\
50.23	0.01\\
50.24	0.01\\
50.25	0.01\\
50.26	0.01\\
50.27	0.01\\
50.28	0.01\\
50.29	0.01\\
50.3	0.01\\
50.31	0.01\\
50.32	0.01\\
50.33	0.01\\
50.34	0.01\\
50.35	0.01\\
50.36	0.01\\
50.37	0.01\\
50.38	0.01\\
50.39	0.01\\
50.4	0.01\\
50.41	0.01\\
50.42	0.01\\
50.43	0.01\\
50.44	0.01\\
50.45	0.01\\
50.46	0.01\\
50.47	0.01\\
50.48	0.01\\
50.49	0.01\\
50.5	0.01\\
50.51	0.01\\
50.52	0.01\\
50.53	0.01\\
50.54	0.01\\
50.55	0.01\\
50.56	0.01\\
50.57	0.01\\
50.58	0.01\\
50.59	0.01\\
50.6	0.01\\
50.61	0.01\\
50.62	0.01\\
50.63	0.01\\
50.64	0.01\\
50.65	0.01\\
50.66	0.01\\
50.67	0.01\\
50.68	0.01\\
50.69	0.01\\
50.7	0.01\\
50.71	0.01\\
50.72	0.01\\
50.73	0.01\\
50.74	0.01\\
50.75	0.01\\
50.76	0.01\\
50.77	0.01\\
50.78	0.01\\
50.79	0.01\\
50.8	0.01\\
50.81	0.01\\
50.82	0.01\\
50.83	0.01\\
50.84	0.01\\
50.85	0.01\\
50.86	0.01\\
50.87	0.01\\
50.88	0.01\\
50.89	0.01\\
50.9	0.01\\
50.91	0.01\\
50.92	0.01\\
50.93	0.01\\
50.94	0.01\\
50.95	0.01\\
50.96	0.01\\
50.97	0.01\\
50.98	0.01\\
50.99	0.01\\
51	0.01\\
51.01	0.01\\
51.02	0.01\\
51.03	0.01\\
51.04	0.01\\
51.05	0.01\\
51.06	0.01\\
51.07	0.01\\
51.08	0.01\\
51.09	0.01\\
51.1	0.01\\
51.11	0.01\\
51.12	0.01\\
51.13	0.01\\
51.14	0.01\\
51.15	0.01\\
51.16	0.01\\
51.17	0.01\\
51.18	0.01\\
51.19	0.01\\
51.2	0.01\\
51.21	0.01\\
51.22	0.01\\
51.23	0.01\\
51.24	0.01\\
51.25	0.01\\
51.26	0.01\\
51.27	0.01\\
51.28	0.01\\
51.29	0.01\\
51.3	0.01\\
51.31	0.01\\
51.32	0.01\\
51.33	0.01\\
51.34	0.01\\
51.35	0.01\\
51.36	0.01\\
51.37	0.01\\
51.38	0.01\\
51.39	0.01\\
51.4	0.01\\
51.41	0.01\\
51.42	0.01\\
51.43	0.01\\
51.44	0.01\\
51.45	0.01\\
51.46	0.01\\
51.47	0.01\\
51.48	0.01\\
51.49	0.01\\
51.5	0.01\\
51.51	0.01\\
51.52	0.01\\
51.53	0.01\\
51.54	0.01\\
51.55	0.01\\
51.56	0.01\\
51.57	0.01\\
51.58	0.01\\
51.59	0.01\\
51.6	0.01\\
51.61	0.01\\
51.62	0.01\\
51.63	0.01\\
51.64	0.01\\
51.65	0.01\\
51.66	0.01\\
51.67	0.01\\
51.68	0.01\\
51.69	0.01\\
51.7	0.01\\
51.71	0.01\\
51.72	0.01\\
51.73	0.01\\
51.74	0.01\\
51.75	0.01\\
51.76	0.01\\
51.77	0.01\\
51.78	0.01\\
51.79	0.01\\
51.8	0.01\\
51.81	0.01\\
51.82	0.01\\
51.83	0.01\\
51.84	0.01\\
51.85	0.01\\
51.86	0.01\\
51.87	0.01\\
51.88	0.01\\
51.89	0.01\\
51.9	0.01\\
51.91	0.01\\
51.92	0.01\\
51.93	0.01\\
51.94	0.01\\
51.95	0.01\\
51.96	0.01\\
51.97	0.01\\
51.98	0.01\\
51.99	0.01\\
52	0.01\\
52.01	0.01\\
52.02	0.01\\
52.03	0.01\\
52.04	0.01\\
52.05	0.01\\
52.06	0.01\\
52.07	0.01\\
52.08	0.01\\
52.09	0.01\\
52.1	0.01\\
52.11	0.01\\
52.12	0.01\\
52.13	0.01\\
52.14	0.01\\
52.15	0.01\\
52.16	0.01\\
52.17	0.01\\
52.18	0.01\\
52.19	0.01\\
52.2	0.01\\
52.21	0.01\\
52.22	0.01\\
52.23	0.01\\
52.24	0.01\\
52.25	0.01\\
52.26	0.01\\
52.27	0.01\\
52.28	0.01\\
52.29	0.01\\
52.3	0.01\\
52.31	0.01\\
52.32	0.01\\
52.33	0.01\\
52.34	0.01\\
52.35	0.01\\
52.36	0.01\\
52.37	0.01\\
52.38	0.01\\
52.39	0.01\\
52.4	0.01\\
52.41	0.01\\
52.42	0.01\\
52.43	0.01\\
52.44	0.01\\
52.45	0.01\\
52.46	0.01\\
52.47	0.01\\
52.48	0.01\\
52.49	0.01\\
52.5	0.01\\
52.51	0.01\\
52.52	0.01\\
52.53	0.01\\
52.54	0.01\\
52.55	0.01\\
52.56	0.01\\
52.57	0.01\\
52.58	0.01\\
52.59	0.01\\
52.6	0.01\\
52.61	0.01\\
52.62	0.01\\
52.63	0.01\\
52.64	0.01\\
52.65	0.01\\
52.66	0.01\\
52.67	0.01\\
52.68	0.01\\
52.69	0.01\\
52.7	0.01\\
52.71	0.01\\
52.72	0.01\\
52.73	0.01\\
52.74	0.01\\
52.75	0.01\\
52.76	0.01\\
52.77	0.01\\
52.78	0.01\\
52.79	0.01\\
52.8	0.01\\
52.81	0.01\\
52.82	0.01\\
52.83	0.01\\
52.84	0.01\\
52.85	0.01\\
52.86	0.01\\
52.87	0.01\\
52.88	0.01\\
52.89	0.01\\
52.9	0.01\\
52.91	0.01\\
52.92	0.01\\
52.93	0.01\\
52.94	0.01\\
52.95	0.01\\
52.96	0.01\\
52.97	0.01\\
52.98	0.01\\
52.99	0.01\\
53	0.01\\
53.01	0.01\\
53.02	0.01\\
53.03	0.01\\
53.04	0.01\\
53.05	0.01\\
53.06	0.01\\
53.07	0.01\\
53.08	0.01\\
53.09	0.01\\
53.1	0.01\\
53.11	0.01\\
53.12	0.01\\
53.13	0.01\\
53.14	0.01\\
53.15	0.01\\
53.16	0.01\\
53.17	0.01\\
53.18	0.01\\
53.19	0.01\\
53.2	0.01\\
53.21	0.01\\
53.22	0.01\\
53.23	0.01\\
53.24	0.01\\
53.25	0.01\\
53.26	0.01\\
53.27	0.01\\
53.28	0.01\\
53.29	0.01\\
53.3	0.01\\
53.31	0.01\\
53.32	0.01\\
53.33	0.01\\
53.34	0.01\\
53.35	0.01\\
53.36	0.01\\
53.37	0.01\\
53.38	0.01\\
53.39	0.01\\
53.4	0.01\\
53.41	0.01\\
53.42	0.01\\
53.43	0.01\\
53.44	0.01\\
53.45	0.01\\
53.46	0.01\\
53.47	0.01\\
53.48	0.01\\
53.49	0.01\\
53.5	0.01\\
53.51	0.01\\
53.52	0.01\\
53.53	0.01\\
53.54	0.01\\
53.55	0.01\\
53.56	0.01\\
53.57	0.01\\
53.58	0.01\\
53.59	0.01\\
53.6	0.01\\
53.61	0.01\\
53.62	0.01\\
53.63	0.01\\
53.64	0.01\\
53.65	0.01\\
53.66	0.01\\
53.67	0.01\\
53.68	0.01\\
53.69	0.01\\
53.7	0.01\\
53.71	0.01\\
53.72	0.01\\
53.73	0.01\\
53.74	0.01\\
53.75	0.01\\
53.76	0.01\\
53.77	0.01\\
53.78	0.01\\
53.79	0.01\\
53.8	0.01\\
53.81	0.01\\
53.82	0.01\\
53.83	0.01\\
53.84	0.01\\
53.85	0.01\\
53.86	0.01\\
53.87	0.01\\
53.88	0.01\\
53.89	0.01\\
53.9	0.01\\
53.91	0.01\\
53.92	0.01\\
53.93	0.01\\
53.94	0.01\\
53.95	0.01\\
53.96	0.01\\
53.97	0.01\\
53.98	0.01\\
53.99	0.01\\
54	0.01\\
54.01	0.01\\
54.02	0.01\\
54.03	0.01\\
54.04	0.01\\
54.05	0.01\\
54.06	0.01\\
54.07	0.01\\
54.08	0.01\\
54.09	0.01\\
54.1	0.01\\
54.11	0.01\\
54.12	0.01\\
54.13	0.01\\
54.14	0.01\\
54.15	0.01\\
54.16	0.01\\
54.17	0.01\\
54.18	0.01\\
54.19	0.01\\
54.2	0.01\\
54.21	0.01\\
54.22	0.01\\
54.23	0.01\\
54.24	0.01\\
54.25	0.01\\
54.26	0.01\\
54.27	0.01\\
54.28	0.01\\
54.29	0.01\\
54.3	0.01\\
54.31	0.01\\
54.32	0.01\\
54.33	0.01\\
54.34	0.01\\
54.35	0.01\\
54.36	0.01\\
54.37	0.01\\
54.38	0.01\\
54.39	0.01\\
54.4	0.01\\
54.41	0.01\\
54.42	0.01\\
54.43	0.01\\
54.44	0.01\\
54.45	0.01\\
54.46	0.01\\
54.47	0.01\\
54.48	0.01\\
54.49	0.01\\
54.5	0.01\\
54.51	0.01\\
54.52	0.01\\
54.53	0.01\\
54.54	0.01\\
54.55	0.01\\
54.56	0.01\\
54.57	0.01\\
54.58	0.01\\
54.59	0.01\\
54.6	0.01\\
54.61	0.01\\
54.62	0.01\\
54.63	0.01\\
54.64	0.01\\
54.65	0.01\\
54.66	0.01\\
54.67	0.01\\
54.68	0.01\\
54.69	0.01\\
54.7	0.01\\
54.71	0.01\\
54.72	0.01\\
54.73	0.01\\
54.74	0.01\\
54.75	0.01\\
54.76	0.01\\
54.77	0.01\\
54.78	0.01\\
54.79	0.01\\
54.8	0.01\\
54.81	0.01\\
54.82	0.01\\
54.83	0.01\\
54.84	0.01\\
54.85	0.01\\
54.86	0.01\\
54.87	0.01\\
54.88	0.01\\
54.89	0.01\\
54.9	0.01\\
54.91	0.01\\
54.92	0.01\\
54.93	0.01\\
54.94	0.01\\
54.95	0.01\\
54.96	0.01\\
54.97	0.01\\
54.98	0.01\\
54.99	0.01\\
55	0.01\\
55.01	0.01\\
55.02	0.01\\
55.03	0.01\\
55.04	0.01\\
55.05	0.01\\
55.06	0.01\\
55.07	0.01\\
55.08	0.01\\
55.09	0.01\\
55.1	0.01\\
55.11	0.01\\
55.12	0.01\\
55.13	0.01\\
55.14	0.01\\
55.15	0.01\\
55.16	0.01\\
55.17	0.01\\
55.18	0.01\\
55.19	0.01\\
55.2	0.01\\
55.21	0.01\\
55.22	0.01\\
55.23	0.01\\
55.24	0.01\\
55.25	0.01\\
55.26	0.01\\
55.27	0.01\\
55.28	0.01\\
55.29	0.01\\
55.3	0.01\\
55.31	0.01\\
55.32	0.01\\
55.33	0.01\\
55.34	0.01\\
55.35	0.01\\
55.36	0.01\\
55.37	0.01\\
55.38	0.01\\
55.39	0.01\\
55.4	0.01\\
55.41	0.01\\
55.42	0.01\\
55.43	0.01\\
55.44	0.01\\
55.45	0.01\\
55.46	0.01\\
55.47	0.01\\
55.48	0.01\\
55.49	0.01\\
55.5	0.01\\
55.51	0.01\\
55.52	0.01\\
55.53	0.01\\
55.54	0.01\\
55.55	0.01\\
55.56	0.01\\
55.57	0.01\\
55.58	0.01\\
55.59	0.01\\
55.6	0.01\\
55.61	0.01\\
55.62	0.01\\
55.63	0.01\\
55.64	0.01\\
55.65	0.01\\
55.66	0.01\\
55.67	0.01\\
55.68	0.01\\
55.69	0.01\\
55.7	0.01\\
55.71	0.01\\
55.72	0.01\\
55.73	0.01\\
55.74	0.01\\
55.75	0.01\\
55.76	0.01\\
55.77	0.01\\
55.78	0.01\\
55.79	0.01\\
55.8	0.01\\
55.81	0.01\\
55.82	0.01\\
55.83	0.01\\
55.84	0.01\\
55.85	0.01\\
55.86	0.01\\
55.87	0.01\\
55.88	0.01\\
55.89	0.01\\
55.9	0.01\\
55.91	0.01\\
55.92	0.01\\
55.93	0.01\\
55.94	0.01\\
55.95	0.01\\
55.96	0.01\\
55.97	0.01\\
55.98	0.01\\
55.99	0.01\\
56	0.01\\
56.01	0.01\\
56.02	0.01\\
56.03	0.01\\
56.04	0.01\\
56.05	0.01\\
56.06	0.01\\
56.07	0.01\\
56.08	0.01\\
56.09	0.01\\
56.1	0.01\\
56.11	0.01\\
56.12	0.01\\
56.13	0.01\\
56.14	0.01\\
56.15	0.01\\
56.16	0.01\\
56.17	0.01\\
56.18	0.01\\
56.19	0.01\\
56.2	0.01\\
56.21	0.01\\
56.22	0.01\\
56.23	0.01\\
56.24	0.01\\
56.25	0.01\\
56.26	0.01\\
56.27	0.01\\
56.28	0.01\\
56.29	0.01\\
56.3	0.01\\
56.31	0.01\\
56.32	0.01\\
56.33	0.01\\
56.34	0.01\\
56.35	0.01\\
56.36	0.01\\
56.37	0.01\\
56.38	0.01\\
56.39	0.01\\
56.4	0.01\\
56.41	0.01\\
56.42	0.01\\
56.43	0.01\\
56.44	0.01\\
56.45	0.01\\
56.46	0.01\\
56.47	0.01\\
56.48	0.01\\
56.49	0.01\\
56.5	0.01\\
56.51	0.01\\
56.52	0.01\\
56.53	0.01\\
56.54	0.01\\
56.55	0.01\\
56.56	0.01\\
56.57	0.01\\
56.58	0.01\\
56.59	0.01\\
56.6	0.01\\
56.61	0.01\\
56.62	0.01\\
56.63	0.01\\
56.64	0.01\\
56.65	0.01\\
56.66	0.01\\
56.67	0.01\\
56.68	0.01\\
56.69	0.01\\
56.7	0.01\\
56.71	0.01\\
56.72	0.01\\
56.73	0.01\\
56.74	0.01\\
56.75	0.01\\
56.76	0.01\\
56.77	0.01\\
56.78	0.01\\
56.79	0.01\\
56.8	0.01\\
56.81	0.01\\
56.82	0.01\\
56.83	0.01\\
56.84	0.01\\
56.85	0.01\\
56.86	0.01\\
56.87	0.01\\
56.88	0.01\\
56.89	0.01\\
56.9	0.01\\
56.91	0.01\\
56.92	0.01\\
56.93	0.01\\
56.94	0.01\\
56.95	0.01\\
56.96	0.01\\
56.97	0.01\\
56.98	0.01\\
56.99	0.01\\
57	0.01\\
57.01	0.01\\
57.02	0.01\\
57.03	0.01\\
57.04	0.01\\
57.05	0.01\\
57.06	0.01\\
57.07	0.01\\
57.08	0.01\\
57.09	0.01\\
57.1	0.01\\
57.11	0.01\\
57.12	0.01\\
57.13	0.01\\
57.14	0.01\\
57.15	0.01\\
57.16	0.01\\
57.17	0.01\\
57.18	0.01\\
57.19	0.01\\
57.2	0.01\\
57.21	0.01\\
57.22	0.01\\
57.23	0.01\\
57.24	0.01\\
57.25	0.01\\
57.26	0.01\\
57.27	0.01\\
57.28	0.01\\
57.29	0.01\\
57.3	0.01\\
57.31	0.01\\
57.32	0.01\\
57.33	0.01\\
57.34	0.01\\
57.35	0.01\\
57.36	0.01\\
57.37	0.01\\
57.38	0.01\\
57.39	0.01\\
57.4	0.01\\
57.41	0.01\\
57.42	0.01\\
57.43	0.01\\
57.44	0.01\\
57.45	0.01\\
57.46	0.01\\
57.47	0.01\\
57.48	0.01\\
57.49	0.01\\
57.5	0.01\\
57.51	0.01\\
57.52	0.01\\
57.53	0.01\\
57.54	0.01\\
57.55	0.01\\
57.56	0.01\\
57.57	0.01\\
57.58	0.01\\
57.59	0.01\\
57.6	0.01\\
57.61	0.01\\
57.62	0.01\\
57.63	0.01\\
57.64	0.01\\
57.65	0.01\\
57.66	0.01\\
57.67	0.01\\
57.68	0.01\\
57.69	0.01\\
57.7	0.01\\
57.71	0.01\\
57.72	0.01\\
57.73	0.01\\
57.74	0.01\\
57.75	0.01\\
57.76	0.01\\
57.77	0.01\\
57.78	0.01\\
57.79	0.01\\
57.8	0.01\\
57.81	0.01\\
57.82	0.01\\
57.83	0.01\\
57.84	0.01\\
57.85	0.01\\
57.86	0.01\\
57.87	0.01\\
57.88	0.01\\
57.89	0.01\\
57.9	0.01\\
57.91	0.01\\
57.92	0.01\\
57.93	0.01\\
57.94	0.01\\
57.95	0.01\\
57.96	0.01\\
57.97	0.01\\
57.98	0.01\\
57.99	0.01\\
58	0.01\\
58.01	0.01\\
58.02	0.01\\
58.03	0.01\\
58.04	0.01\\
58.05	0.01\\
58.06	0.01\\
58.07	0.01\\
58.08	0.01\\
58.09	0.01\\
58.1	0.01\\
58.11	0.01\\
58.12	0.01\\
58.13	0.01\\
58.14	0.01\\
58.15	0.01\\
58.16	0.01\\
58.17	0.01\\
58.18	0.01\\
58.19	0.01\\
58.2	0.01\\
58.21	0.01\\
58.22	0.01\\
58.23	0.01\\
58.24	0.01\\
58.25	0.01\\
58.26	0.01\\
58.27	0.01\\
58.28	0.01\\
58.29	0.01\\
58.3	0.01\\
58.31	0.01\\
58.32	0.01\\
58.33	0.01\\
58.34	0.01\\
58.35	0.01\\
58.36	0.01\\
58.37	0.01\\
58.38	0.01\\
58.39	0.01\\
58.4	0.01\\
58.41	0.01\\
58.42	0.01\\
58.43	0.01\\
58.44	0.01\\
58.45	0.01\\
58.46	0.01\\
58.47	0.01\\
58.48	0.01\\
58.49	0.01\\
58.5	0.01\\
58.51	0.01\\
58.52	0.01\\
58.53	0.01\\
58.54	0.01\\
58.55	0.01\\
58.56	0.01\\
58.57	0.01\\
58.58	0.01\\
58.59	0.01\\
58.6	0.01\\
58.61	0.01\\
58.62	0.01\\
58.63	0.01\\
58.64	0.01\\
58.65	0.01\\
58.66	0.01\\
58.67	0.01\\
58.68	0.01\\
58.69	0.01\\
58.7	0.01\\
58.71	0.01\\
58.72	0.01\\
58.73	0.01\\
58.74	0.01\\
58.75	0.01\\
58.76	0.01\\
58.77	0.01\\
58.78	0.01\\
58.79	0.01\\
58.8	0.01\\
58.81	0.01\\
58.82	0.01\\
58.83	0.01\\
58.84	0.01\\
58.85	0.01\\
58.86	0.01\\
58.87	0.01\\
58.88	0.01\\
58.89	0.01\\
58.9	0.01\\
58.91	0.01\\
58.92	0.01\\
58.93	0.01\\
58.94	0.01\\
58.95	0.01\\
58.96	0.01\\
58.97	0.01\\
58.98	0.01\\
58.99	0.01\\
59	0.01\\
59.01	0.01\\
59.02	0.01\\
59.03	0.01\\
59.04	0.01\\
59.05	0.01\\
59.06	0.01\\
59.07	0.01\\
59.08	0.01\\
59.09	0.01\\
59.1	0.01\\
59.11	0.01\\
59.12	0.01\\
59.13	0.01\\
59.14	0.01\\
59.15	0.01\\
59.16	0.01\\
59.17	0.01\\
59.18	0.01\\
59.19	0.01\\
59.2	0.01\\
59.21	0.01\\
59.22	0.01\\
59.23	0.01\\
59.24	0.01\\
59.25	0.01\\
59.26	0.01\\
59.27	0.01\\
59.28	0.01\\
59.29	0.01\\
59.3	0.01\\
59.31	0.01\\
59.32	0.01\\
59.33	0.01\\
59.34	0.01\\
59.35	0.01\\
59.36	0.01\\
59.37	0.01\\
59.38	0.01\\
59.39	0.01\\
59.4	0.01\\
59.41	0.01\\
59.42	0.01\\
59.43	0.01\\
59.44	0.01\\
59.45	0.01\\
59.46	0.01\\
59.47	0.01\\
59.48	0.01\\
59.49	0.01\\
59.5	0.01\\
59.51	0.01\\
59.52	0.01\\
59.53	0.01\\
59.54	0.01\\
59.55	0.01\\
59.56	0.01\\
59.57	0.01\\
59.58	0.01\\
59.59	0.01\\
59.6	0.01\\
59.61	0.01\\
59.62	0.01\\
59.63	0.01\\
59.64	0.01\\
59.65	0.01\\
59.66	0.01\\
59.67	0.01\\
59.68	0.01\\
59.69	0.01\\
59.7	0.01\\
59.71	0.01\\
59.72	0.01\\
59.73	0.01\\
59.74	0.01\\
59.75	0.01\\
59.76	0.01\\
59.77	0.01\\
59.78	0.01\\
59.79	0.01\\
59.8	0.01\\
59.81	0.01\\
59.82	0.01\\
59.83	0.01\\
59.84	0.01\\
59.85	0.01\\
59.86	0.01\\
59.87	0.01\\
59.88	0.01\\
59.89	0.01\\
59.9	0.01\\
59.91	0.01\\
59.92	0.01\\
59.93	0.01\\
59.94	0.01\\
59.95	0.01\\
59.96	0.01\\
59.97	0.01\\
59.98	0.01\\
59.99	0.01\\
60	0.01\\
60.01	0.01\\
60.02	0.01\\
60.03	0.01\\
60.04	0.01\\
60.05	0.01\\
60.06	0.01\\
60.07	0.01\\
60.08	0.01\\
60.09	0.01\\
60.1	0.01\\
60.11	0.01\\
60.12	0.01\\
60.13	0.01\\
60.14	0.01\\
60.15	0.01\\
60.16	0.01\\
60.17	0.01\\
60.18	0.01\\
60.19	0.01\\
60.2	0.01\\
60.21	0.01\\
60.22	0.01\\
60.23	0.01\\
60.24	0.01\\
60.25	0.01\\
60.26	0.01\\
60.27	0.01\\
60.28	0.01\\
60.29	0.01\\
60.3	0.01\\
60.31	0.01\\
60.32	0.01\\
60.33	0.01\\
60.34	0.01\\
60.35	0.01\\
60.36	0.01\\
60.37	0.01\\
60.38	0.01\\
60.39	0.01\\
60.4	0.01\\
60.41	0.01\\
60.42	0.01\\
60.43	0.01\\
60.44	0.01\\
60.45	0.01\\
60.46	0.01\\
60.47	0.01\\
60.48	0.01\\
60.49	0.01\\
60.5	0.01\\
60.51	0.01\\
60.52	0.01\\
60.53	0.01\\
60.54	0.01\\
60.55	0.01\\
60.56	0.01\\
60.57	0.01\\
60.58	0.01\\
60.59	0.01\\
60.6	0.01\\
60.61	0.01\\
60.62	0.01\\
60.63	0.01\\
60.64	0.01\\
60.65	0.01\\
60.66	0.01\\
60.67	0.01\\
60.68	0.01\\
60.69	0.01\\
60.7	0.01\\
60.71	0.01\\
60.72	0.01\\
60.73	0.01\\
60.74	0.01\\
60.75	0.01\\
60.76	0.01\\
60.77	0.01\\
60.78	0.01\\
60.79	0.01\\
60.8	0.01\\
60.81	0.01\\
60.82	0.01\\
60.83	0.01\\
60.84	0.01\\
60.85	0.01\\
60.86	0.01\\
60.87	0.01\\
60.88	0.01\\
60.89	0.01\\
60.9	0.01\\
60.91	0.01\\
60.92	0.01\\
60.93	0.01\\
60.94	0.01\\
60.95	0.01\\
60.96	0.01\\
60.97	0.01\\
60.98	0.01\\
60.99	0.01\\
61	0.01\\
61.01	0.01\\
61.02	0.01\\
61.03	0.01\\
61.04	0.01\\
61.05	0.01\\
61.06	0.01\\
61.07	0.01\\
61.08	0.01\\
61.09	0.01\\
61.1	0.01\\
61.11	0.01\\
61.12	0.01\\
61.13	0.01\\
61.14	0.01\\
61.15	0.01\\
61.16	0.01\\
61.17	0.01\\
61.18	0.01\\
61.19	0.01\\
61.2	0.01\\
61.21	0.01\\
61.22	0.01\\
61.23	0.01\\
61.24	0.01\\
61.25	0.01\\
61.26	0.01\\
61.27	0.01\\
61.28	0.01\\
61.29	0.01\\
61.3	0.01\\
61.31	0.01\\
61.32	0.01\\
61.33	0.01\\
61.34	0.01\\
61.35	0.01\\
61.36	0.01\\
61.37	0.01\\
61.38	0.01\\
61.39	0.01\\
61.4	0.01\\
61.41	0.01\\
61.42	0.01\\
61.43	0.01\\
61.44	0.01\\
61.45	0.01\\
61.46	0.01\\
61.47	0.01\\
61.48	0.01\\
61.49	0.01\\
61.5	0.01\\
61.51	0.01\\
61.52	0.01\\
61.53	0.01\\
61.54	0.01\\
61.55	0.01\\
61.56	0.01\\
61.57	0.01\\
61.58	0.01\\
61.59	0.01\\
61.6	0.01\\
61.61	0.01\\
61.62	0.01\\
61.63	0.01\\
61.64	0.01\\
61.65	0.01\\
61.66	0.01\\
61.67	0.01\\
61.68	0.01\\
61.69	0.01\\
61.7	0.01\\
61.71	0.01\\
61.72	0.01\\
61.73	0.01\\
61.74	0.01\\
61.75	0.01\\
61.76	0.01\\
61.77	0.01\\
61.78	0.01\\
61.79	0.01\\
61.8	0.01\\
61.81	0.01\\
61.82	0.01\\
61.83	0.01\\
61.84	0.01\\
61.85	0.01\\
61.86	0.01\\
61.87	0.01\\
61.88	0.01\\
61.89	0.01\\
61.9	0.01\\
61.91	0.01\\
61.92	0.01\\
61.93	0.01\\
61.94	0.01\\
61.95	0.01\\
61.96	0.01\\
61.97	0.01\\
61.98	0.01\\
61.99	0.01\\
62	0.01\\
62.01	0.01\\
62.02	0.01\\
62.03	0.01\\
62.04	0.01\\
62.05	0.01\\
62.06	0.01\\
62.07	0.01\\
62.08	0.01\\
62.09	0.01\\
62.1	0.01\\
62.11	0.01\\
62.12	0.01\\
62.13	0.01\\
62.14	0.01\\
62.15	0.01\\
62.16	0.01\\
62.17	0.01\\
62.18	0.01\\
62.19	0.01\\
62.2	0.01\\
62.21	0.01\\
62.22	0.01\\
62.23	0.01\\
62.24	0.01\\
62.25	0.01\\
62.26	0.01\\
62.27	0.01\\
62.28	0.01\\
62.29	0.01\\
62.3	0.01\\
62.31	0.01\\
62.32	0.01\\
62.33	0.01\\
62.34	0.01\\
62.35	0.01\\
62.36	0.01\\
62.37	0.01\\
62.38	0.01\\
62.39	0.01\\
62.4	0.01\\
62.41	0.01\\
62.42	0.01\\
62.43	0.01\\
62.44	0.01\\
62.45	0.01\\
62.46	0.01\\
62.47	0.01\\
62.48	0.01\\
62.49	0.01\\
62.5	0.01\\
62.51	0.01\\
62.52	0.01\\
62.53	0.01\\
62.54	0.01\\
62.55	0.01\\
62.56	0.01\\
62.57	0.01\\
62.58	0.01\\
62.59	0.01\\
62.6	0.01\\
62.61	0.01\\
62.62	0.01\\
62.63	0.01\\
62.64	0.01\\
62.65	0.01\\
62.66	0.01\\
62.67	0.01\\
62.68	0.01\\
62.69	0.01\\
62.7	0.01\\
62.71	0.01\\
62.72	0.01\\
62.73	0.01\\
62.74	0.01\\
62.75	0.01\\
62.76	0.01\\
62.77	0.01\\
62.78	0.01\\
62.79	0.01\\
62.8	0.01\\
62.81	0.01\\
62.82	0.01\\
62.83	0.01\\
62.84	0.01\\
62.85	0.01\\
62.86	0.01\\
62.87	0.01\\
62.88	0.01\\
62.89	0.01\\
62.9	0.01\\
62.91	0.01\\
62.92	0.01\\
62.93	0.01\\
62.94	0.01\\
62.95	0.01\\
62.96	0.01\\
62.97	0.01\\
62.98	0.01\\
62.99	0.01\\
63	0.01\\
63.01	0.01\\
63.02	0.01\\
63.03	0.01\\
63.04	0.01\\
63.05	0.01\\
63.06	0.01\\
63.07	0.01\\
63.08	0.01\\
63.09	0.01\\
63.1	0.01\\
63.11	0.01\\
63.12	0.01\\
63.13	0.01\\
63.14	0.01\\
63.15	0.01\\
63.16	0.01\\
63.17	0.01\\
63.18	0.01\\
63.19	0.01\\
63.2	0.01\\
63.21	0.01\\
63.22	0.01\\
63.23	0.01\\
63.24	0.01\\
63.25	0.01\\
63.26	0.01\\
63.27	0.01\\
63.28	0.01\\
63.29	0.01\\
63.3	0.01\\
63.31	0.01\\
63.32	0.01\\
63.33	0.01\\
63.34	0.01\\
63.35	0.01\\
63.36	0.01\\
63.37	0.01\\
63.38	0.01\\
63.39	0.01\\
63.4	0.01\\
63.41	0.01\\
63.42	0.01\\
63.43	0.01\\
63.44	0.01\\
63.45	0.01\\
63.46	0.01\\
63.47	0.01\\
63.48	0.01\\
63.49	0.01\\
63.5	0.01\\
63.51	0.01\\
63.52	0.01\\
63.53	0.01\\
63.54	0.01\\
63.55	0.01\\
63.56	0.01\\
63.57	0.01\\
63.58	0.01\\
63.59	0.01\\
63.6	0.01\\
63.61	0.01\\
63.62	0.01\\
63.63	0.01\\
63.64	0.01\\
63.65	0.01\\
63.66	0.01\\
63.67	0.01\\
63.68	0.01\\
63.69	0.01\\
63.7	0.01\\
63.71	0.01\\
63.72	0.01\\
63.73	0.01\\
63.74	0.01\\
63.75	0.01\\
63.76	0.01\\
63.77	0.01\\
63.78	0.01\\
63.79	0.01\\
63.8	0.01\\
63.81	0.01\\
63.82	0.01\\
63.83	0.01\\
63.84	0.01\\
63.85	0.01\\
63.86	0.01\\
63.87	0.01\\
63.88	0.01\\
63.89	0.01\\
63.9	0.01\\
63.91	0.01\\
63.92	0.01\\
63.93	0.01\\
63.94	0.01\\
63.95	0.01\\
63.96	0.01\\
63.97	0.01\\
63.98	0.01\\
63.99	0.01\\
64	0.01\\
64.01	0.01\\
64.02	0.01\\
64.03	0.01\\
64.04	0.01\\
64.05	0.01\\
64.06	0.01\\
64.07	0.01\\
64.08	0.01\\
64.09	0.01\\
64.1	0.01\\
64.11	0.01\\
64.12	0.01\\
64.13	0.01\\
64.14	0.01\\
64.15	0.01\\
64.16	0.01\\
64.17	0.01\\
64.18	0.01\\
64.19	0.01\\
64.2	0.01\\
64.21	0.01\\
64.22	0.01\\
64.23	0.01\\
64.24	0.01\\
64.25	0.01\\
64.26	0.01\\
64.27	0.01\\
64.28	0.01\\
64.29	0.01\\
64.3	0.01\\
64.31	0.01\\
64.32	0.01\\
64.33	0.01\\
64.34	0.01\\
64.35	0.01\\
64.36	0.01\\
64.37	0.01\\
64.38	0.01\\
64.39	0.01\\
64.4	0.01\\
64.41	0.01\\
64.42	0.01\\
64.43	0.01\\
64.44	0.01\\
64.45	0.01\\
64.46	0.01\\
64.47	0.01\\
64.48	0.01\\
64.49	0.01\\
64.5	0.01\\
64.51	0.01\\
64.52	0.01\\
64.53	0.01\\
64.54	0.01\\
64.55	0.01\\
64.56	0.01\\
64.57	0.01\\
64.58	0.01\\
64.59	0.01\\
64.6	0.01\\
64.61	0.01\\
64.62	0.01\\
64.63	0.01\\
64.64	0.01\\
64.65	0.01\\
64.66	0.01\\
64.67	0.01\\
64.68	0.01\\
64.69	0.01\\
64.7	0.01\\
64.71	0.01\\
64.72	0.01\\
64.73	0.01\\
64.74	0.01\\
64.75	0.01\\
64.76	0.01\\
64.77	0.01\\
64.78	0.01\\
64.79	0.01\\
64.8	0.01\\
64.81	0.01\\
64.82	0.01\\
64.83	0.01\\
64.84	0.01\\
64.85	0.01\\
64.86	0.01\\
64.87	0.01\\
64.88	0.01\\
64.89	0.01\\
64.9	0.01\\
64.91	0.01\\
64.92	0.01\\
64.93	0.01\\
64.94	0.01\\
64.95	0.01\\
64.96	0.01\\
64.97	0.01\\
64.98	0.01\\
64.99	0.01\\
65	0.01\\
65.01	0.01\\
65.02	0.01\\
65.03	0.01\\
65.04	0.01\\
65.05	0.01\\
65.06	0.01\\
65.07	0.01\\
65.08	0.01\\
65.09	0.01\\
65.1	0.01\\
65.11	0.01\\
65.12	0.01\\
65.13	0.01\\
65.14	0.01\\
65.15	0.01\\
65.16	0.01\\
65.17	0.01\\
65.18	0.01\\
65.19	0.01\\
65.2	0.01\\
65.21	0.01\\
65.22	0.01\\
65.23	0.01\\
65.24	0.01\\
65.25	0.01\\
65.26	0.01\\
65.27	0.01\\
65.28	0.01\\
65.29	0.01\\
65.3	0.01\\
65.31	0.01\\
65.32	0.01\\
65.33	0.01\\
65.34	0.01\\
65.35	0.01\\
65.36	0.01\\
65.37	0.01\\
65.38	0.01\\
65.39	0.01\\
65.4	0.01\\
65.41	0.01\\
65.42	0.01\\
65.43	0.01\\
65.44	0.01\\
65.45	0.01\\
65.46	0.01\\
65.47	0.01\\
65.48	0.01\\
65.49	0.01\\
65.5	0.01\\
65.51	0.01\\
65.52	0.01\\
65.53	0.01\\
65.54	0.01\\
65.55	0.01\\
65.56	0.01\\
65.57	0.01\\
65.58	0.01\\
65.59	0.01\\
65.6	0.01\\
65.61	0.01\\
65.62	0.01\\
65.63	0.01\\
65.64	0.01\\
65.65	0.01\\
65.66	0.01\\
65.67	0.01\\
65.68	0.01\\
65.69	0.01\\
65.7	0.01\\
65.71	0.01\\
65.72	0.01\\
65.73	0.01\\
65.74	0.01\\
65.75	0.01\\
65.76	0.01\\
65.77	0.01\\
65.78	0.01\\
65.79	0.01\\
65.8	0.01\\
65.81	0.01\\
65.82	0.01\\
65.83	0.01\\
65.84	0.01\\
65.85	0.01\\
65.86	0.01\\
65.87	0.01\\
65.88	0.01\\
65.89	0.01\\
65.9	0.01\\
65.91	0.01\\
65.92	0.01\\
65.93	0.01\\
65.94	0.01\\
65.95	0.01\\
65.96	0.01\\
65.97	0.01\\
65.98	0.01\\
65.99	0.01\\
66	0.01\\
66.01	0.01\\
66.02	0.01\\
66.03	0.01\\
66.04	0.01\\
66.05	0.01\\
66.06	0.01\\
66.07	0.01\\
66.08	0.01\\
66.09	0.01\\
66.1	0.01\\
66.11	0.01\\
66.12	0.01\\
66.13	0.01\\
66.14	0.01\\
66.15	0.01\\
66.16	0.01\\
66.17	0.01\\
66.18	0.01\\
66.19	0.01\\
66.2	0.01\\
66.21	0.01\\
66.22	0.01\\
66.23	0.01\\
66.24	0.01\\
66.25	0.01\\
66.26	0.01\\
66.27	0.01\\
66.28	0.01\\
66.29	0.01\\
66.3	0.01\\
66.31	0.01\\
66.32	0.01\\
66.33	0.01\\
66.34	0.01\\
66.35	0.01\\
66.36	0.01\\
66.37	0.01\\
66.38	0.01\\
66.39	0.01\\
66.4	0.01\\
66.41	0.01\\
66.42	0.01\\
66.43	0.01\\
66.44	0.01\\
66.45	0.01\\
66.46	0.01\\
66.47	0.01\\
66.48	0.01\\
66.49	0.01\\
66.5	0.01\\
66.51	0.01\\
66.52	0.01\\
66.53	0.01\\
66.54	0.01\\
66.55	0.01\\
66.56	0.01\\
66.57	0.01\\
66.58	0.01\\
66.59	0.01\\
66.6	0.01\\
66.61	0.01\\
66.62	0.01\\
66.63	0.01\\
66.64	0.01\\
66.65	0.01\\
66.66	0.01\\
66.67	0.01\\
66.68	0.01\\
66.69	0.01\\
66.7	0.01\\
66.71	0.01\\
66.72	0.01\\
66.73	0.01\\
66.74	0.01\\
66.75	0.01\\
66.76	0.01\\
66.77	0.01\\
66.78	0.01\\
66.79	0.01\\
66.8	0.01\\
66.81	0.01\\
66.82	0.01\\
66.83	0.01\\
66.84	0.01\\
66.85	0.01\\
66.86	0.01\\
66.87	0.01\\
66.88	0.01\\
66.89	0.01\\
66.9	0.01\\
66.91	0.01\\
66.92	0.01\\
66.93	0.01\\
66.94	0.01\\
66.95	0.01\\
66.96	0.01\\
66.97	0.01\\
66.98	0.01\\
66.99	0.01\\
67	0.01\\
67.01	0.01\\
67.02	0.01\\
67.03	0.01\\
67.04	0.01\\
67.05	0.01\\
67.06	0.01\\
67.07	0.01\\
67.08	0.01\\
67.09	0.01\\
67.1	0.01\\
67.11	0.01\\
67.12	0.01\\
67.13	0.01\\
67.14	0.01\\
67.15	0.01\\
67.16	0.01\\
67.17	0.01\\
67.18	0.01\\
67.19	0.01\\
67.2	0.01\\
67.21	0.01\\
67.22	0.01\\
67.23	0.01\\
67.24	0.01\\
67.25	0.01\\
67.26	0.01\\
67.27	0.01\\
67.28	0.01\\
67.29	0.01\\
67.3	0.01\\
67.31	0.01\\
67.32	0.01\\
67.33	0.01\\
67.34	0.01\\
67.35	0.01\\
67.36	0.01\\
67.37	0.01\\
67.38	0.01\\
67.39	0.01\\
67.4	0.01\\
67.41	0.01\\
67.42	0.01\\
67.43	0.01\\
67.44	0.01\\
67.45	0.01\\
67.46	0.01\\
67.47	0.01\\
67.48	0.01\\
67.49	0.01\\
67.5	0.01\\
67.51	0.01\\
67.52	0.01\\
67.53	0.01\\
67.54	0.01\\
67.55	0.01\\
67.56	0.01\\
67.57	0.01\\
67.58	0.01\\
67.59	0.01\\
67.6	0.01\\
67.61	0.01\\
67.62	0.01\\
67.63	0.01\\
67.64	0.01\\
67.65	0.01\\
67.66	0.01\\
67.67	0.01\\
67.68	0.01\\
67.69	0.01\\
67.7	0.01\\
67.71	0.01\\
67.72	0.01\\
67.73	0.01\\
67.74	0.01\\
67.75	0.01\\
67.76	0.01\\
67.77	0.01\\
67.78	0.01\\
67.79	0.01\\
67.8	0.01\\
67.81	0.01\\
67.82	0.01\\
67.83	0.01\\
67.84	0.01\\
67.85	0.01\\
67.86	0.01\\
67.87	0.01\\
67.88	0.01\\
67.89	0.01\\
67.9	0.01\\
67.91	0.01\\
67.92	0.01\\
67.93	0.01\\
67.94	0.01\\
67.95	0.01\\
67.96	0.01\\
67.97	0.01\\
67.98	0.01\\
67.99	0.01\\
68	0.01\\
68.01	0.01\\
68.02	0.01\\
68.03	0.01\\
68.04	0.01\\
68.05	0.01\\
68.06	0.01\\
68.07	0.01\\
68.08	0.01\\
68.09	0.01\\
68.1	0.01\\
68.11	0.01\\
68.12	0.01\\
68.13	0.01\\
68.14	0.01\\
68.15	0.01\\
68.16	0.01\\
68.17	0.01\\
68.18	0.01\\
68.19	0.01\\
68.2	0.01\\
68.21	0.01\\
68.22	0.01\\
68.23	0.01\\
68.24	0.01\\
68.25	0.01\\
68.26	0.01\\
68.27	0.01\\
68.28	0.01\\
68.29	0.01\\
68.3	0.01\\
68.31	0.01\\
68.32	0.01\\
68.33	0.01\\
68.34	0.01\\
68.35	0.01\\
68.36	0.01\\
68.37	0.01\\
68.38	0.01\\
68.39	0.01\\
68.4	0.01\\
68.41	0.01\\
68.42	0.01\\
68.43	0.01\\
68.44	0.01\\
68.45	0.01\\
68.46	0.01\\
68.47	0.01\\
68.48	0.01\\
68.49	0.01\\
68.5	0.01\\
68.51	0.01\\
68.52	0.01\\
68.53	0.01\\
68.54	0.01\\
68.55	0.01\\
68.56	0.01\\
68.57	0.01\\
68.58	0.01\\
68.59	0.01\\
68.6	0.01\\
68.61	0.01\\
68.62	0.01\\
68.63	0.01\\
68.64	0.01\\
68.65	0.01\\
68.66	0.01\\
68.67	0.01\\
68.68	0.01\\
68.69	0.01\\
68.7	0.01\\
68.71	0.01\\
68.72	0.01\\
68.73	0.01\\
68.74	0.01\\
68.75	0.01\\
68.76	0.01\\
68.77	0.01\\
68.78	0.01\\
68.79	0.01\\
68.8	0.01\\
68.81	0.01\\
68.82	0.01\\
68.83	0.01\\
68.84	0.01\\
68.85	0.01\\
68.86	0.01\\
68.87	0.01\\
68.88	0.01\\
68.89	0.01\\
68.9	0.01\\
68.91	0.01\\
68.92	0.01\\
68.93	0.01\\
68.94	0.01\\
68.95	0.01\\
68.96	0.01\\
68.97	0.01\\
68.98	0.01\\
68.99	0.01\\
69	0.01\\
69.01	0.01\\
69.02	0.01\\
69.03	0.01\\
69.04	0.01\\
69.05	0.01\\
69.06	0.01\\
69.07	0.01\\
69.08	0.01\\
69.09	0.01\\
69.1	0.01\\
69.11	0.01\\
69.12	0.01\\
69.13	0.01\\
69.14	0.01\\
69.15	0.01\\
69.16	0.01\\
69.17	0.01\\
69.18	0.01\\
69.19	0.01\\
69.2	0.01\\
69.21	0.01\\
69.22	0.01\\
69.23	0.01\\
69.24	0.01\\
69.25	0.01\\
69.26	0.01\\
69.27	0.01\\
69.28	0.01\\
69.29	0.01\\
69.3	0.01\\
69.31	0.01\\
69.32	0.01\\
69.33	0.01\\
69.34	0.01\\
69.35	0.01\\
69.36	0.01\\
69.37	0.01\\
69.38	0.01\\
69.39	0.01\\
69.4	0.01\\
69.41	0.01\\
69.42	0.01\\
69.43	0.01\\
69.44	0.01\\
69.45	0.01\\
69.46	0.01\\
69.47	0.01\\
69.48	0.01\\
69.49	0.01\\
69.5	0.01\\
69.51	0.01\\
69.52	0.01\\
69.53	0.01\\
69.54	0.01\\
69.55	0.01\\
69.56	0.01\\
69.57	0.01\\
69.58	0.01\\
69.59	0.01\\
69.6	0.01\\
69.61	0.01\\
69.62	0.01\\
69.63	0.01\\
69.64	0.01\\
69.65	0.01\\
69.66	0.01\\
69.67	0.01\\
69.68	0.01\\
69.69	0.01\\
69.7	0.01\\
69.71	0.01\\
69.72	0.01\\
69.73	0.01\\
69.74	0.01\\
69.75	0.01\\
69.76	0.01\\
69.77	0.01\\
69.78	0.01\\
69.79	0.01\\
69.8	0.01\\
69.81	0.01\\
69.82	0.01\\
69.83	0.01\\
69.84	0.01\\
69.85	0.01\\
69.86	0.01\\
69.87	0.01\\
69.88	0.01\\
69.89	0.01\\
69.9	0.01\\
69.91	0.01\\
69.92	0.01\\
69.93	0.01\\
69.94	0.01\\
69.95	0.01\\
69.96	0.01\\
69.97	0.01\\
69.98	0.01\\
69.99	0.01\\
70	0.01\\
70.01	0.01\\
70.02	0.01\\
70.03	0.01\\
70.04	0.01\\
70.05	0.01\\
70.06	0.01\\
70.07	0.01\\
70.08	0.01\\
70.09	0.01\\
70.1	0.01\\
70.11	0.01\\
70.12	0.01\\
70.13	0.01\\
70.14	0.01\\
70.15	0.01\\
70.16	0.01\\
70.17	0.01\\
70.18	0.01\\
70.19	0.01\\
70.2	0.01\\
70.21	0.01\\
70.22	0.01\\
70.23	0.01\\
70.24	0.01\\
70.25	0.01\\
70.26	0.01\\
70.27	0.01\\
70.28	0.01\\
70.29	0.01\\
70.3	0.01\\
70.31	0.01\\
70.32	0.01\\
70.33	0.01\\
70.34	0.01\\
70.35	0.01\\
70.36	0.01\\
70.37	0.01\\
70.38	0.01\\
70.39	0.01\\
70.4	0.01\\
70.41	0.01\\
70.42	0.01\\
70.43	0.01\\
70.44	0.01\\
70.45	0.01\\
70.46	0.01\\
70.47	0.01\\
70.48	0.01\\
70.49	0.01\\
70.5	0.01\\
70.51	0.01\\
70.52	0.01\\
70.53	0.01\\
70.54	0.01\\
70.55	0.01\\
70.56	0.01\\
70.57	0.01\\
70.58	0.01\\
70.59	0.01\\
70.6	0.01\\
70.61	0.01\\
70.62	0.01\\
70.63	0.01\\
70.64	0.01\\
70.65	0.01\\
70.66	0.01\\
70.67	0.01\\
70.68	0.01\\
70.69	0.01\\
70.7	0.01\\
70.71	0.01\\
70.72	0.01\\
70.73	0.01\\
70.74	0.01\\
70.75	0.01\\
70.76	0.01\\
70.77	0.01\\
70.78	0.01\\
70.79	0.01\\
70.8	0.01\\
70.81	0.01\\
70.82	0.01\\
70.83	0.01\\
70.84	0.01\\
70.85	0.01\\
70.86	0.01\\
70.87	0.01\\
70.88	0.01\\
70.89	0.01\\
70.9	0.01\\
70.91	0.01\\
70.92	0.01\\
70.93	0.01\\
70.94	0.01\\
70.95	0.01\\
70.96	0.01\\
70.97	0.01\\
70.98	0.01\\
70.99	0.01\\
71	0.01\\
71.01	0.01\\
71.02	0.01\\
71.03	0.01\\
71.04	0.01\\
71.05	0.01\\
71.06	0.01\\
71.07	0.01\\
71.08	0.01\\
71.09	0.01\\
71.1	0.01\\
71.11	0.01\\
71.12	0.01\\
71.13	0.01\\
71.14	0.01\\
71.15	0.01\\
71.16	0.01\\
71.17	0.01\\
71.18	0.01\\
71.19	0.01\\
71.2	0.01\\
71.21	0.01\\
71.22	0.01\\
71.23	0.01\\
71.24	0.01\\
71.25	0.01\\
71.26	0.01\\
71.27	0.01\\
71.28	0.01\\
71.29	0.01\\
71.3	0.01\\
71.31	0.01\\
71.32	0.01\\
71.33	0.01\\
71.34	0.01\\
71.35	0.01\\
71.36	0.01\\
71.37	0.01\\
71.38	0.01\\
71.39	0.01\\
71.4	0.01\\
71.41	0.01\\
71.42	0.01\\
71.43	0.01\\
71.44	0.01\\
71.45	0.01\\
71.46	0.01\\
71.47	0.01\\
71.48	0.01\\
71.49	0.01\\
71.5	0.01\\
71.51	0.01\\
71.52	0.01\\
71.53	0.01\\
71.54	0.01\\
71.55	0.01\\
71.56	0.01\\
71.57	0.01\\
71.58	0.01\\
71.59	0.01\\
71.6	0.01\\
71.61	0.01\\
71.62	0.01\\
71.63	0.01\\
71.64	0.01\\
71.65	0.01\\
71.66	0.01\\
71.67	0.01\\
71.68	0.01\\
71.69	0.01\\
71.7	0.01\\
71.71	0.01\\
71.72	0.01\\
71.73	0.01\\
71.74	0.01\\
71.75	0.01\\
71.76	0.01\\
71.77	0.01\\
71.78	0.01\\
71.79	0.01\\
71.8	0.01\\
71.81	0.01\\
71.82	0.01\\
71.83	0.01\\
71.84	0.01\\
71.85	0.01\\
71.86	0.01\\
71.87	0.01\\
71.88	0.01\\
71.89	0.01\\
71.9	0.01\\
71.91	0.01\\
71.92	0.01\\
71.93	0.01\\
71.94	0.01\\
71.95	0.01\\
71.96	0.01\\
71.97	0.01\\
71.98	0.01\\
71.99	0.01\\
72	0.01\\
72.01	0.01\\
72.02	0.01\\
72.03	0.01\\
72.04	0.01\\
72.05	0.01\\
72.06	0.01\\
72.07	0.01\\
72.08	0.01\\
72.09	0.01\\
72.1	0.01\\
72.11	0.01\\
72.12	0.01\\
72.13	0.01\\
72.14	0.01\\
72.15	0.01\\
72.16	0.01\\
72.17	0.01\\
72.18	0.01\\
72.19	0.01\\
72.2	0.01\\
72.21	0.01\\
72.22	0.01\\
72.23	0.01\\
72.24	0.01\\
72.25	0.01\\
72.26	0.01\\
72.27	0.01\\
72.28	0.01\\
72.29	0.01\\
72.3	0.01\\
72.31	0.01\\
72.32	0.01\\
72.33	0.01\\
72.34	0.01\\
72.35	0.01\\
72.36	0.01\\
72.37	0.01\\
72.38	0.01\\
72.39	0.01\\
72.4	0.01\\
72.41	0.01\\
72.42	0.01\\
72.43	0.01\\
72.44	0.01\\
72.45	0.01\\
72.46	0.01\\
72.47	0.01\\
72.48	0.01\\
72.49	0.01\\
72.5	0.01\\
72.51	0.01\\
72.52	0.01\\
72.53	0.01\\
72.54	0.01\\
72.55	0.01\\
72.56	0.01\\
72.57	0.01\\
72.58	0.01\\
72.59	0.01\\
72.6	0.01\\
72.61	0.01\\
72.62	0.01\\
72.63	0.01\\
72.64	0.01\\
72.65	0.01\\
72.66	0.01\\
72.67	0.01\\
72.68	0.01\\
72.69	0.01\\
72.7	0.01\\
72.71	0.01\\
72.72	0.01\\
72.73	0.01\\
72.74	0.01\\
72.75	0.01\\
72.76	0.01\\
72.77	0.01\\
72.78	0.01\\
72.79	0.01\\
72.8	0.01\\
72.81	0.01\\
72.82	0.01\\
72.83	0.01\\
72.84	0.01\\
72.85	0.01\\
72.86	0.01\\
72.87	0.01\\
72.88	0.01\\
72.89	0.01\\
72.9	0.01\\
72.91	0.01\\
72.92	0.01\\
72.93	0.01\\
72.94	0.01\\
72.95	0.01\\
72.96	0.01\\
72.97	0.01\\
72.98	0.01\\
72.99	0.01\\
73	0.01\\
73.01	0.01\\
73.02	0.01\\
73.03	0.01\\
73.04	0.01\\
73.05	0.01\\
73.06	0.01\\
73.07	0.01\\
73.08	0.01\\
73.09	0.01\\
73.1	0.01\\
73.11	0.01\\
73.12	0.01\\
73.13	0.01\\
73.14	0.01\\
73.15	0.01\\
73.16	0.01\\
73.17	0.01\\
73.18	0.01\\
73.19	0.01\\
73.2	0.01\\
73.21	0.01\\
73.22	0.01\\
73.23	0.01\\
73.24	0.01\\
73.25	0.01\\
73.26	0.01\\
73.27	0.01\\
73.28	0.01\\
73.29	0.01\\
73.3	0.01\\
73.31	0.01\\
73.32	0.01\\
73.33	0.01\\
73.34	0.01\\
73.35	0.01\\
73.36	0.01\\
73.37	0.01\\
73.38	0.01\\
73.39	0.01\\
73.4	0.01\\
73.41	0.01\\
73.42	0.01\\
73.43	0.01\\
73.44	0.01\\
73.45	0.01\\
73.46	0.01\\
73.47	0.01\\
73.48	0.01\\
73.49	0.01\\
73.5	0.01\\
73.51	0.01\\
73.52	0.01\\
73.53	0.01\\
73.54	0.01\\
73.55	0.01\\
73.56	0.01\\
73.57	0.01\\
73.58	0.01\\
73.59	0.01\\
73.6	0.01\\
73.61	0.01\\
73.62	0.01\\
73.63	0.01\\
73.64	0.01\\
73.65	0.01\\
73.66	0.01\\
73.67	0.01\\
73.68	0.01\\
73.69	0.01\\
73.7	0.01\\
73.71	0.01\\
73.72	0.01\\
73.73	0.01\\
73.74	0.01\\
73.75	0.01\\
73.76	0.01\\
73.77	0.01\\
73.78	0.01\\
73.79	0.01\\
73.8	0.01\\
73.81	0.01\\
73.82	0.01\\
73.83	0.01\\
73.84	0.01\\
73.85	0.01\\
73.86	0.01\\
73.87	0.01\\
73.88	0.01\\
73.89	0.01\\
73.9	0.01\\
73.91	0.01\\
73.92	0.01\\
73.93	0.01\\
73.94	0.01\\
73.95	0.01\\
73.96	0.01\\
73.97	0.01\\
73.98	0.01\\
73.99	0.01\\
74	0.01\\
74.01	0.01\\
74.02	0.01\\
74.03	0.01\\
74.04	0.01\\
74.05	0.01\\
74.06	0.01\\
74.07	0.01\\
74.08	0.01\\
74.09	0.01\\
74.1	0.01\\
74.11	0.01\\
74.12	0.01\\
74.13	0.01\\
74.14	0.01\\
74.15	0.01\\
74.16	0.01\\
74.17	0.01\\
74.18	0.01\\
74.19	0.01\\
74.2	0.01\\
74.21	0.01\\
74.22	0.01\\
74.23	0.01\\
74.24	0.01\\
74.25	0.01\\
74.26	0.01\\
74.27	0.01\\
74.28	0.01\\
74.29	0.01\\
74.3	0.01\\
74.31	0.01\\
74.32	0.01\\
74.33	0.01\\
74.34	0.01\\
74.35	0.01\\
74.36	0.01\\
74.37	0.01\\
74.38	0.01\\
74.39	0.01\\
74.4	0.01\\
74.41	0.01\\
74.42	0.01\\
74.43	0.01\\
74.44	0.01\\
74.45	0.01\\
74.46	0.01\\
74.47	0.01\\
74.48	0.01\\
74.49	0.01\\
74.5	0.01\\
74.51	0.01\\
74.52	0.01\\
74.53	0.01\\
74.54	0.01\\
74.55	0.01\\
74.56	0.01\\
74.57	0.01\\
74.58	0.01\\
74.59	0.01\\
74.6	0.01\\
74.61	0.01\\
74.62	0.01\\
74.63	0.01\\
74.64	0.01\\
74.65	0.01\\
74.66	0.01\\
74.67	0.01\\
74.68	0.01\\
74.69	0.01\\
74.7	0.01\\
74.71	0.01\\
74.72	0.01\\
74.73	0.01\\
74.74	0.01\\
74.75	0.01\\
74.76	0.01\\
74.77	0.01\\
74.78	0.01\\
74.79	0.01\\
74.8	0.01\\
74.81	0.01\\
74.82	0.01\\
74.83	0.01\\
74.84	0.01\\
74.85	0.01\\
74.86	0.01\\
74.87	0.01\\
74.88	0.01\\
74.89	0.01\\
74.9	0.01\\
74.91	0.01\\
74.92	0.01\\
74.93	0.01\\
74.94	0.01\\
74.95	0.01\\
74.96	0.01\\
74.97	0.01\\
74.98	0.01\\
74.99	0.01\\
75	0.01\\
75.01	0.01\\
75.02	0.01\\
75.03	0.01\\
75.04	0.01\\
75.05	0.01\\
75.06	0.01\\
75.07	0.01\\
75.08	0.01\\
75.09	0.01\\
75.1	0.01\\
75.11	0.01\\
75.12	0.01\\
75.13	0.01\\
75.14	0.01\\
75.15	0.01\\
75.16	0.01\\
75.17	0.01\\
75.18	0.01\\
75.19	0.01\\
75.2	0.01\\
75.21	0.01\\
75.22	0.01\\
75.23	0.01\\
75.24	0.01\\
75.25	0.01\\
75.26	0.01\\
75.27	0.01\\
75.28	0.01\\
75.29	0.01\\
75.3	0.01\\
75.31	0.01\\
75.32	0.01\\
75.33	0.01\\
75.34	0.01\\
75.35	0.01\\
75.36	0.01\\
75.37	0.01\\
75.38	0.01\\
75.39	0.01\\
75.4	0.01\\
75.41	0.01\\
75.42	0.01\\
75.43	0.01\\
75.44	0.01\\
75.45	0.01\\
75.46	0.01\\
75.47	0.01\\
75.48	0.01\\
75.49	0.01\\
75.5	0.01\\
75.51	0.01\\
75.52	0.01\\
75.53	0.01\\
75.54	0.01\\
75.55	0.01\\
75.56	0.01\\
75.57	0.01\\
75.58	0.01\\
75.59	0.01\\
75.6	0.01\\
75.61	0.01\\
75.62	0.01\\
75.63	0.01\\
75.64	0.01\\
75.65	0.01\\
75.66	0.01\\
75.67	0.01\\
75.68	0.01\\
75.69	0.01\\
75.7	0.01\\
75.71	0.01\\
75.72	0.01\\
75.73	0.01\\
75.74	0.01\\
75.75	0.01\\
75.76	0.01\\
75.77	0.01\\
75.78	0.01\\
75.79	0.01\\
75.8	0.01\\
75.81	0.01\\
75.82	0.01\\
75.83	0.01\\
75.84	0.01\\
75.85	0.01\\
75.86	0.01\\
75.87	0.01\\
75.88	0.01\\
75.89	0.01\\
75.9	0.01\\
75.91	0.01\\
75.92	0.01\\
75.93	0.01\\
75.94	0.01\\
75.95	0.01\\
75.96	0.01\\
75.97	0.01\\
75.98	0.01\\
75.99	0.01\\
76	0.01\\
76.01	0.01\\
76.02	0.01\\
76.03	0.01\\
76.04	0.01\\
76.05	0.01\\
76.06	0.01\\
76.07	0.01\\
76.08	0.01\\
76.09	0.01\\
76.1	0.01\\
76.11	0.01\\
76.12	0.01\\
76.13	0.01\\
76.14	0.01\\
76.15	0.01\\
76.16	0.01\\
76.17	0.01\\
76.18	0.01\\
76.19	0.01\\
76.2	0.01\\
76.21	0.01\\
76.22	0.01\\
76.23	0.01\\
76.24	0.01\\
76.25	0.01\\
76.26	0.01\\
76.27	0.01\\
76.28	0.01\\
76.29	0.01\\
76.3	0.01\\
76.31	0.01\\
76.32	0.01\\
76.33	0.01\\
76.34	0.01\\
76.35	0.01\\
76.36	0.01\\
76.37	0.01\\
76.38	0.01\\
76.39	0.01\\
76.4	0.01\\
76.41	0.01\\
76.42	0.01\\
76.43	0.01\\
76.44	0.01\\
76.45	0.01\\
76.46	0.01\\
76.47	0.01\\
76.48	0.01\\
76.49	0.01\\
76.5	0.01\\
76.51	0.01\\
76.52	0.01\\
76.53	0.01\\
76.54	0.01\\
76.55	0.01\\
76.56	0.01\\
76.57	0.01\\
76.58	0.01\\
76.59	0.01\\
76.6	0.01\\
76.61	0.01\\
76.62	0.01\\
76.63	0.01\\
76.64	0.01\\
76.65	0.01\\
76.66	0.01\\
76.67	0.01\\
76.68	0.01\\
76.69	0.01\\
76.7	0.01\\
76.71	0.01\\
76.72	0.01\\
76.73	0.01\\
76.74	0.01\\
76.75	0.01\\
76.76	0.01\\
76.77	0.01\\
76.78	0.01\\
76.79	0.01\\
76.8	0.01\\
76.81	0.01\\
76.82	0.01\\
76.83	0.01\\
76.84	0.01\\
76.85	0.01\\
76.86	0.01\\
76.87	0.01\\
76.88	0.01\\
76.89	0.01\\
76.9	0.01\\
76.91	0.01\\
76.92	0.01\\
76.93	0.01\\
76.94	0.01\\
76.95	0.01\\
76.96	0.01\\
76.97	0.01\\
76.98	0.01\\
76.99	0.01\\
77	0.01\\
77.01	0.01\\
77.02	0.01\\
77.03	0.01\\
77.04	0.01\\
77.05	0.01\\
77.06	0.01\\
77.07	0.01\\
77.08	0.01\\
77.09	0.01\\
77.1	0.01\\
77.11	0.01\\
77.12	0.01\\
77.13	0.01\\
77.14	0.01\\
77.15	0.01\\
77.16	0.01\\
77.17	0.01\\
77.18	0.01\\
77.19	0.01\\
77.2	0.01\\
77.21	0.01\\
77.22	0.01\\
77.23	0.01\\
77.24	0.01\\
77.25	0.01\\
77.26	0.01\\
77.27	0.01\\
77.28	0.01\\
77.29	0.01\\
77.3	0.01\\
77.31	0.01\\
77.32	0.01\\
77.33	0.01\\
77.34	0.01\\
77.35	0.01\\
77.36	0.01\\
77.37	0.01\\
77.38	0.01\\
77.39	0.01\\
77.4	0.01\\
77.41	0.01\\
77.42	0.01\\
77.43	0.01\\
77.44	0.01\\
77.45	0.01\\
77.46	0.01\\
77.47	0.01\\
77.48	0.01\\
77.49	0.01\\
77.5	0.01\\
77.51	0.01\\
77.52	0.01\\
77.53	0.01\\
77.54	0.01\\
77.55	0.01\\
77.56	0.01\\
77.57	0.01\\
77.58	0.01\\
77.59	0.01\\
77.6	0.01\\
77.61	0.01\\
77.62	0.01\\
77.63	0.01\\
77.64	0.01\\
77.65	0.01\\
77.66	0.01\\
77.67	0.01\\
77.68	0.01\\
77.69	0.01\\
77.7	0.01\\
77.71	0.01\\
77.72	0.01\\
77.73	0.01\\
77.74	0.01\\
77.75	0.01\\
77.76	0.01\\
77.77	0.01\\
77.78	0.01\\
77.79	0.01\\
77.8	0.01\\
77.81	0.01\\
77.82	0.01\\
77.83	0.01\\
77.84	0.01\\
77.85	0.01\\
77.86	0.01\\
77.87	0.01\\
77.88	0.01\\
77.89	0.01\\
77.9	0.01\\
77.91	0.01\\
77.92	0.01\\
77.93	0.01\\
77.94	0.01\\
77.95	0.01\\
77.96	0.01\\
77.97	0.01\\
77.98	0.01\\
77.99	0.01\\
78	0.01\\
78.01	0.01\\
78.02	0.01\\
78.03	0.01\\
78.04	0.01\\
78.05	0.01\\
78.06	0.01\\
78.07	0.01\\
78.08	0.01\\
78.09	0.01\\
78.1	0.01\\
78.11	0.01\\
78.12	0.01\\
78.13	0.01\\
78.14	0.01\\
78.15	0.01\\
78.16	0.01\\
78.17	0.01\\
78.18	0.01\\
78.19	0.01\\
78.2	0.01\\
78.21	0.01\\
78.22	0.01\\
78.23	0.01\\
78.24	0.01\\
78.25	0.01\\
78.26	0.01\\
78.27	0.01\\
78.28	0.01\\
78.29	0.01\\
78.3	0.01\\
78.31	0.01\\
78.32	0.01\\
78.33	0.01\\
78.34	0.01\\
78.35	0.01\\
78.36	0.01\\
78.37	0.01\\
78.38	0.01\\
78.39	0.01\\
78.4	0.01\\
78.41	0.01\\
78.42	0.01\\
78.43	0.01\\
78.44	0.01\\
78.45	0.01\\
78.46	0.01\\
78.47	0.01\\
78.48	0.01\\
78.49	0.01\\
78.5	0.01\\
78.51	0.01\\
78.52	0.01\\
78.53	0.01\\
78.54	0.01\\
78.55	0.01\\
78.56	0.01\\
78.57	0.01\\
78.58	0.01\\
78.59	0.01\\
78.6	0.01\\
78.61	0.01\\
78.62	0.01\\
78.63	0.01\\
78.64	0.01\\
78.65	0.01\\
78.66	0.01\\
78.67	0.01\\
78.68	0.01\\
78.69	0.01\\
78.7	0.01\\
78.71	0.01\\
78.72	0.01\\
78.73	0.01\\
78.74	0.01\\
78.75	0.01\\
78.76	0.01\\
78.77	0.01\\
78.78	0.01\\
78.79	0.01\\
78.8	0.01\\
78.81	0.01\\
78.82	0.01\\
78.83	0.01\\
78.84	0.01\\
78.85	0.01\\
78.86	0.01\\
78.87	0.01\\
78.88	0.01\\
78.89	0.01\\
78.9	0.01\\
78.91	0.01\\
78.92	0.01\\
78.93	0.01\\
78.94	0.01\\
78.95	0.01\\
78.96	0.01\\
78.97	0.01\\
78.98	0.01\\
78.99	0.01\\
79	0.01\\
79.01	0.01\\
79.02	0.01\\
79.03	0.01\\
79.04	0.01\\
79.05	0.01\\
79.06	0.01\\
79.07	0.01\\
79.08	0.01\\
79.09	0.01\\
79.1	0.01\\
79.11	0.01\\
79.12	0.01\\
79.13	0.01\\
79.14	0.01\\
79.15	0.01\\
79.16	0.01\\
79.17	0.01\\
79.18	0.01\\
79.19	0.01\\
79.2	0.01\\
79.21	0.01\\
79.22	0.01\\
79.23	0.01\\
79.24	0.01\\
79.25	0.01\\
79.26	0.01\\
79.27	0.01\\
79.28	0.01\\
79.29	0.01\\
79.3	0.01\\
79.31	0.01\\
79.32	0.01\\
79.33	0.01\\
79.34	0.01\\
79.35	0.01\\
79.36	0.01\\
79.37	0.01\\
79.38	0.01\\
79.39	0.01\\
79.4	0.01\\
79.41	0.01\\
79.42	0.01\\
79.43	0.01\\
79.44	0.01\\
79.45	0.01\\
79.46	0.01\\
79.47	0.01\\
79.48	0.01\\
79.49	0.01\\
79.5	0.01\\
79.51	0.01\\
79.52	0.01\\
79.53	0.01\\
79.54	0.01\\
79.55	0.01\\
79.56	0.01\\
79.57	0.01\\
79.58	0.01\\
79.59	0.01\\
79.6	0.01\\
79.61	0.01\\
79.62	0.01\\
79.63	0.01\\
79.64	0.01\\
79.65	0.01\\
79.66	0.01\\
79.67	0.01\\
79.68	0.01\\
79.69	0.01\\
79.7	0.01\\
79.71	0.01\\
79.72	0.01\\
79.73	0.01\\
79.74	0.01\\
79.75	0.01\\
79.76	0.01\\
79.77	0.01\\
79.78	0.01\\
79.79	0.01\\
79.8	0.01\\
79.81	0.01\\
79.82	0.01\\
79.83	0.01\\
79.84	0.01\\
79.85	0.01\\
79.86	0.01\\
79.87	0.01\\
79.88	0.01\\
79.89	0.01\\
79.9	0.01\\
79.91	0.01\\
79.92	0.01\\
79.93	0.01\\
79.94	0.01\\
79.95	0.01\\
79.96	0.01\\
79.97	0.01\\
79.98	0.01\\
79.99	0.01\\
80	0.01\\
80.01	0.01\\
};
\addplot [color=blue,dashed]
  table[row sep=crcr]{%
80.01	0.01\\
80.02	0.01\\
80.03	0.01\\
80.04	0.01\\
80.05	0.01\\
80.06	0.01\\
80.07	0.01\\
80.08	0.01\\
80.09	0.01\\
80.1	0.01\\
80.11	0.01\\
80.12	0.01\\
80.13	0.01\\
80.14	0.01\\
80.15	0.01\\
80.16	0.01\\
80.17	0.01\\
80.18	0.01\\
80.19	0.01\\
80.2	0.01\\
80.21	0.01\\
80.22	0.01\\
80.23	0.01\\
80.24	0.01\\
80.25	0.01\\
80.26	0.01\\
80.27	0.01\\
80.28	0.01\\
80.29	0.01\\
80.3	0.01\\
80.31	0.01\\
80.32	0.01\\
80.33	0.01\\
80.34	0.01\\
80.35	0.01\\
80.36	0.01\\
80.37	0.01\\
80.38	0.01\\
80.39	0.01\\
80.4	0.01\\
80.41	0.01\\
80.42	0.01\\
80.43	0.01\\
80.44	0.01\\
80.45	0.01\\
80.46	0.01\\
80.47	0.01\\
80.48	0.01\\
80.49	0.01\\
80.5	0.01\\
80.51	0.01\\
80.52	0.01\\
80.53	0.01\\
80.54	0.01\\
80.55	0.01\\
80.56	0.01\\
80.57	0.01\\
80.58	0.01\\
80.59	0.01\\
80.6	0.01\\
80.61	0.01\\
80.62	0.01\\
80.63	0.01\\
80.64	0.01\\
80.65	0.01\\
80.66	0.01\\
80.67	0.01\\
80.68	0.01\\
80.69	0.01\\
80.7	0.01\\
80.71	0.01\\
80.72	0.01\\
80.73	0.01\\
80.74	0.01\\
80.75	0.01\\
80.76	0.01\\
80.77	0.01\\
80.78	0.01\\
80.79	0.01\\
80.8	0.01\\
80.81	0.01\\
80.82	0.01\\
80.83	0.01\\
80.84	0.01\\
80.85	0.01\\
80.86	0.01\\
80.87	0.01\\
80.88	0.01\\
80.89	0.01\\
80.9	0.01\\
80.91	0.01\\
80.92	0.01\\
80.93	0.01\\
80.94	0.01\\
80.95	0.01\\
80.96	0.01\\
80.97	0.01\\
80.98	0.01\\
80.99	0.01\\
81	0.01\\
81.01	0.01\\
81.02	0.01\\
81.03	0.01\\
81.04	0.01\\
81.05	0.01\\
81.06	0.01\\
81.07	0.01\\
81.08	0.01\\
81.09	0.01\\
81.1	0.01\\
81.11	0.01\\
81.12	0.01\\
81.13	0.01\\
81.14	0.01\\
81.15	0.01\\
81.16	0.01\\
81.17	0.01\\
81.18	0.01\\
81.19	0.01\\
81.2	0.01\\
81.21	0.01\\
81.22	0.01\\
81.23	0.01\\
81.24	0.01\\
81.25	0.01\\
81.26	0.01\\
81.27	0.01\\
81.28	0.01\\
81.29	0.01\\
81.3	0.01\\
81.31	0.01\\
81.32	0.01\\
81.33	0.01\\
81.34	0.01\\
81.35	0.01\\
81.36	0.01\\
81.37	0.01\\
81.38	0.01\\
81.39	0.01\\
81.4	0.01\\
81.41	0.01\\
81.42	0.01\\
81.43	0.01\\
81.44	0.01\\
81.45	0.01\\
81.46	0.01\\
81.47	0.01\\
81.48	0.01\\
81.49	0.01\\
81.5	0.01\\
81.51	0.01\\
81.52	0.01\\
81.53	0.01\\
81.54	0.01\\
81.55	0.01\\
81.56	0.01\\
81.57	0.01\\
81.58	0.01\\
81.59	0.01\\
81.6	0.01\\
81.61	0.01\\
81.62	0.01\\
81.63	0.01\\
81.64	0.01\\
81.65	0.01\\
81.66	0.01\\
81.67	0.01\\
81.68	0.01\\
81.69	0.01\\
81.7	0.01\\
81.71	0.01\\
81.72	0.01\\
81.73	0.01\\
81.74	0.01\\
81.75	0.01\\
81.76	0.01\\
81.77	0.01\\
81.78	0.01\\
81.79	0.01\\
81.8	0.01\\
81.81	0.01\\
81.82	0.01\\
81.83	0.01\\
81.84	0.01\\
81.85	0.01\\
81.86	0.01\\
81.87	0.01\\
81.88	0.01\\
81.89	0.01\\
81.9	0.01\\
81.91	0.01\\
81.92	0.01\\
81.93	0.01\\
81.94	0.01\\
81.95	0.01\\
81.96	0.01\\
81.97	0.01\\
81.98	0.01\\
81.99	0.01\\
82	0.01\\
82.01	0.01\\
82.02	0.01\\
82.03	0.01\\
82.04	0.01\\
82.05	0.01\\
82.06	0.01\\
82.07	0.01\\
82.08	0.01\\
82.09	0.01\\
82.1	0.01\\
82.11	0.01\\
82.12	0.01\\
82.13	0.01\\
82.14	0.01\\
82.15	0.01\\
82.16	0.01\\
82.17	0.01\\
82.18	0.01\\
82.19	0.01\\
82.2	0.01\\
82.21	0.01\\
82.22	0.01\\
82.23	0.01\\
82.24	0.01\\
82.25	0.01\\
82.26	0.01\\
82.27	0.01\\
82.28	0.01\\
82.29	0.01\\
82.3	0.01\\
82.31	0.01\\
82.32	0.01\\
82.33	0.01\\
82.34	0.01\\
82.35	0.01\\
82.36	0.01\\
82.37	0.01\\
82.38	0.01\\
82.39	0.01\\
82.4	0.01\\
82.41	0.01\\
82.42	0.01\\
82.43	0.01\\
82.44	0.01\\
82.45	0.01\\
82.46	0.01\\
82.47	0.01\\
82.48	0.01\\
82.49	0.01\\
82.5	0.01\\
82.51	0.01\\
82.52	0.01\\
82.53	0.01\\
82.54	0.01\\
82.55	0.01\\
82.56	0.01\\
82.57	0.01\\
82.58	0.01\\
82.59	0.01\\
82.6	0.01\\
82.61	0.01\\
82.62	0.01\\
82.63	0.01\\
82.64	0.01\\
82.65	0.01\\
82.66	0.01\\
82.67	0.01\\
82.68	0.01\\
82.69	0.01\\
82.7	0.01\\
82.71	0.01\\
82.72	0.01\\
82.73	0.01\\
82.74	0.01\\
82.75	0.01\\
82.76	0.01\\
82.77	0.01\\
82.78	0.01\\
82.79	0.01\\
82.8	0.01\\
82.81	0.01\\
82.82	0.01\\
82.83	0.01\\
82.84	0.01\\
82.85	0.01\\
82.86	0.01\\
82.87	0.01\\
82.88	0.01\\
82.89	0.01\\
82.9	0.01\\
82.91	0.01\\
82.92	0.01\\
82.93	0.01\\
82.94	0.01\\
82.95	0.01\\
82.96	0.01\\
82.97	0.01\\
82.98	0.01\\
82.99	0.01\\
83	0.01\\
83.01	0.01\\
83.02	0.01\\
83.03	0.01\\
83.04	0.01\\
83.05	0.01\\
83.06	0.01\\
83.07	0.01\\
83.08	0.01\\
83.09	0.01\\
83.1	0.01\\
83.11	0.01\\
83.12	0.01\\
83.13	0.01\\
83.14	0.01\\
83.15	0.01\\
83.16	0.01\\
83.17	0.01\\
83.18	0.01\\
83.19	0.01\\
83.2	0.01\\
83.21	0.01\\
83.22	0.01\\
83.23	0.01\\
83.24	0.01\\
83.25	0.01\\
83.26	0.01\\
83.27	0.01\\
83.28	0.01\\
83.29	0.01\\
83.3	0.01\\
83.31	0.01\\
83.32	0.01\\
83.33	0.01\\
83.34	0.01\\
83.35	0.01\\
83.36	0.01\\
83.37	0.01\\
83.38	0.01\\
83.39	0.01\\
83.4	0.01\\
83.41	0.01\\
83.42	0.01\\
83.43	0.01\\
83.44	0.01\\
83.45	0.01\\
83.46	0.01\\
83.47	0.01\\
83.48	0.01\\
83.49	0.01\\
83.5	0.01\\
83.51	0.01\\
83.52	0.01\\
83.53	0.01\\
83.54	0.01\\
83.55	0.01\\
83.56	0.01\\
83.57	0.01\\
83.58	0.01\\
83.59	0.01\\
83.6	0.01\\
83.61	0.01\\
83.62	0.01\\
83.63	0.01\\
83.64	0.01\\
83.65	0.01\\
83.66	0.01\\
83.67	0.01\\
83.68	0.01\\
83.69	0.01\\
83.7	0.01\\
83.71	0.01\\
83.72	0.01\\
83.73	0.01\\
83.74	0.01\\
83.75	0.01\\
83.76	0.01\\
83.77	0.01\\
83.78	0.01\\
83.79	0.01\\
83.8	0.01\\
83.81	0.01\\
83.82	0.01\\
83.83	0.01\\
83.84	0.01\\
83.85	0.01\\
83.86	0.01\\
83.87	0.01\\
83.88	0.01\\
83.89	0.01\\
83.9	0.01\\
83.91	0.01\\
83.92	0.01\\
83.93	0.01\\
83.94	0.01\\
83.95	0.01\\
83.96	0.01\\
83.97	0.01\\
83.98	0.01\\
83.99	0.01\\
84	0.01\\
84.01	0.01\\
84.02	0.01\\
84.03	0.01\\
84.04	0.01\\
84.05	0.01\\
84.06	0.01\\
84.07	0.01\\
84.08	0.01\\
84.09	0.01\\
84.1	0.01\\
84.11	0.01\\
84.12	0.01\\
84.13	0.01\\
84.14	0.01\\
84.15	0.01\\
84.16	0.01\\
84.17	0.01\\
84.18	0.01\\
84.19	0.01\\
84.2	0.01\\
84.21	0.01\\
84.22	0.01\\
84.23	0.01\\
84.24	0.01\\
84.25	0.01\\
84.26	0.01\\
84.27	0.01\\
84.28	0.01\\
84.29	0.01\\
84.3	0.01\\
84.31	0.01\\
84.32	0.01\\
84.33	0.01\\
84.34	0.01\\
84.35	0.01\\
84.36	0.01\\
84.37	0.01\\
84.38	0.01\\
84.39	0.01\\
84.4	0.01\\
84.41	0.01\\
84.42	0.01\\
84.43	0.01\\
84.44	0.01\\
84.45	0.01\\
84.46	0.01\\
84.47	0.01\\
84.48	0.01\\
84.49	0.01\\
84.5	0.01\\
84.51	0.01\\
84.52	0.01\\
84.53	0.01\\
84.54	0.01\\
84.55	0.01\\
84.56	0.01\\
84.57	0.01\\
84.58	0.01\\
84.59	0.01\\
84.6	0.01\\
84.61	0.01\\
84.62	0.01\\
84.63	0.01\\
84.64	0.01\\
84.65	0.01\\
84.66	0.01\\
84.67	0.01\\
84.68	0.01\\
84.69	0.01\\
84.7	0.01\\
84.71	0.01\\
84.72	0.01\\
84.73	0.01\\
84.74	0.01\\
84.75	0.01\\
84.76	0.01\\
84.77	0.01\\
84.78	0.01\\
84.79	0.01\\
84.8	0.01\\
84.81	0.01\\
84.82	0.01\\
84.83	0.01\\
84.84	0.01\\
84.85	0.01\\
84.86	0.01\\
84.87	0.01\\
84.88	0.01\\
84.89	0.01\\
84.9	0.01\\
84.91	0.01\\
84.92	0.01\\
84.93	0.01\\
84.94	0.01\\
84.95	0.01\\
84.96	0.01\\
84.97	0.01\\
84.98	0.01\\
84.99	0.01\\
85	0.01\\
85.01	0.01\\
85.02	0.01\\
85.03	0.01\\
85.04	0.01\\
85.05	0.01\\
85.06	0.01\\
85.07	0.01\\
85.08	0.01\\
85.09	0.01\\
85.1	0.01\\
85.11	0.01\\
85.12	0.01\\
85.13	0.01\\
85.14	0.01\\
85.15	0.01\\
85.16	0.01\\
85.17	0.01\\
85.18	0.01\\
85.19	0.01\\
85.2	0.01\\
85.21	0.01\\
85.22	0.01\\
85.23	0.01\\
85.24	0.01\\
85.25	0.01\\
85.26	0.01\\
85.27	0.01\\
85.28	0.01\\
85.29	0.01\\
85.3	0.01\\
85.31	0.01\\
85.32	0.01\\
85.33	0.01\\
85.34	0.01\\
85.35	0.01\\
85.36	0.01\\
85.37	0.01\\
85.38	0.01\\
85.39	0.01\\
85.4	0.01\\
85.41	0.01\\
85.42	0.01\\
85.43	0.01\\
85.44	0.01\\
85.45	0.01\\
85.46	0.01\\
85.47	0.01\\
85.48	0.01\\
85.49	0.01\\
85.5	0.01\\
85.51	0.01\\
85.52	0.01\\
85.53	0.01\\
85.54	0.01\\
85.55	0.01\\
85.56	0.01\\
85.57	0.01\\
85.58	0.01\\
85.59	0.01\\
85.6	0.01\\
85.61	0.01\\
85.62	0.01\\
85.63	0.01\\
85.64	0.01\\
85.65	0.01\\
85.66	0.01\\
85.67	0.01\\
85.68	0.01\\
85.69	0.01\\
85.7	0.01\\
85.71	0.01\\
85.72	0.01\\
85.73	0.01\\
85.74	0.01\\
85.75	0.01\\
85.76	0.01\\
85.77	0.01\\
85.78	0.01\\
85.79	0.01\\
85.8	0.01\\
85.81	0.01\\
85.82	0.01\\
85.83	0.01\\
85.84	0.01\\
85.85	0.01\\
85.86	0.01\\
85.87	0.01\\
85.88	0.01\\
85.89	0.01\\
85.9	0.01\\
85.91	0.01\\
85.92	0.01\\
85.93	0.01\\
85.94	0.01\\
85.95	0.01\\
85.96	0.01\\
85.97	0.01\\
85.98	0.01\\
85.99	0.01\\
86	0.01\\
86.01	0.01\\
86.02	0.01\\
86.03	0.01\\
86.04	0.01\\
86.05	0.01\\
86.06	0.01\\
86.07	0.01\\
86.08	0.01\\
86.09	0.01\\
86.1	0.01\\
86.11	0.01\\
86.12	0.01\\
86.13	0.01\\
86.14	0.01\\
86.15	0.01\\
86.16	0.01\\
86.17	0.01\\
86.18	0.01\\
86.19	0.01\\
86.2	0.01\\
86.21	0.01\\
86.22	0.01\\
86.23	0.01\\
86.24	0.01\\
86.25	0.01\\
86.26	0.01\\
86.27	0.01\\
86.28	0.01\\
86.29	0.01\\
86.3	0.01\\
86.31	0.01\\
86.32	0.01\\
86.33	0.01\\
86.34	0.01\\
86.35	0.01\\
86.36	0.01\\
86.37	0.01\\
86.38	0.01\\
86.39	0.01\\
86.4	0.01\\
86.41	0.01\\
86.42	0.01\\
86.43	0.01\\
86.44	0.01\\
86.45	0.01\\
86.46	0.01\\
86.47	0.01\\
86.48	0.01\\
86.49	0.01\\
86.5	0.01\\
86.51	0.01\\
86.52	0.01\\
86.53	0.01\\
86.54	0.01\\
86.55	0.01\\
86.56	0.01\\
86.57	0.01\\
86.58	0.01\\
86.59	0.01\\
86.6	0.01\\
86.61	0.01\\
86.62	0.01\\
86.63	0.01\\
86.64	0.01\\
86.65	0.01\\
86.66	0.01\\
86.67	0.01\\
86.68	0.01\\
86.69	0.01\\
86.7	0.01\\
86.71	0.01\\
86.72	0.01\\
86.73	0.01\\
86.74	0.01\\
86.75	0.01\\
86.76	0.01\\
86.77	0.01\\
86.78	0.01\\
86.79	0.01\\
86.8	0.01\\
86.81	0.01\\
86.82	0.01\\
86.83	0.01\\
86.84	0.01\\
86.85	0.01\\
86.86	0.01\\
86.87	0.01\\
86.88	0.01\\
86.89	0.01\\
86.9	0.01\\
86.91	0.01\\
86.92	0.01\\
86.93	0.01\\
86.94	0.01\\
86.95	0.01\\
86.96	0.01\\
86.97	0.01\\
86.98	0.01\\
86.99	0.01\\
87	0.01\\
87.01	0.01\\
87.02	0.01\\
87.03	0.01\\
87.04	0.01\\
87.05	0.01\\
87.06	0.01\\
87.07	0.01\\
87.08	0.01\\
87.09	0.01\\
87.1	0.01\\
87.11	0.01\\
87.12	0.01\\
87.13	0.01\\
87.14	0.01\\
87.15	0.01\\
87.16	0.01\\
87.17	0.01\\
87.18	0.01\\
87.19	0.01\\
87.2	0.01\\
87.21	0.01\\
87.22	0.01\\
87.23	0.01\\
87.24	0.01\\
87.25	0.01\\
87.26	0.01\\
87.27	0.01\\
87.28	0.01\\
87.29	0.01\\
87.3	0.01\\
87.31	0.01\\
87.32	0.01\\
87.33	0.01\\
87.34	0.01\\
87.35	0.01\\
87.36	0.01\\
87.37	0.01\\
87.38	0.01\\
87.39	0.01\\
87.4	0.01\\
87.41	0.01\\
87.42	0.01\\
87.43	0.01\\
87.44	0.01\\
87.45	0.01\\
87.46	0.01\\
87.47	0.01\\
87.48	0.01\\
87.49	0.01\\
87.5	0.01\\
87.51	0.01\\
87.52	0.01\\
87.53	0.01\\
87.54	0.01\\
87.55	0.01\\
87.56	0.01\\
87.57	0.01\\
87.58	0.01\\
87.59	0.01\\
87.6	0.01\\
87.61	0.01\\
87.62	0.01\\
87.63	0.01\\
87.64	0.01\\
87.65	0.01\\
87.66	0.01\\
87.67	0.01\\
87.68	0.01\\
87.69	0.01\\
87.7	0.01\\
87.71	0.01\\
87.72	0.01\\
87.73	0.01\\
87.74	0.01\\
87.75	0.01\\
87.76	0.01\\
87.77	0.01\\
87.78	0.01\\
87.79	0.01\\
87.8	0.01\\
87.81	0.01\\
87.82	0.01\\
87.83	0.01\\
87.84	0.01\\
87.85	0.01\\
87.86	0.01\\
87.87	0.01\\
87.88	0.01\\
87.89	0.01\\
87.9	0.01\\
87.91	0.01\\
87.92	0.01\\
87.93	0.01\\
87.94	0.01\\
87.95	0.01\\
87.96	0.01\\
87.97	0.01\\
87.98	0.01\\
87.99	0.01\\
88	0.01\\
88.01	0.01\\
88.02	0.01\\
88.03	0.01\\
88.04	0.01\\
88.05	0.01\\
88.06	0.01\\
88.07	0.01\\
88.08	0.01\\
88.09	0.01\\
88.1	0.01\\
88.11	0.01\\
88.12	0.01\\
88.13	0.01\\
88.14	0.01\\
88.15	0.01\\
88.16	0.01\\
88.17	0.01\\
88.18	0.01\\
88.19	0.01\\
88.2	0.01\\
88.21	0.01\\
88.22	0.01\\
88.23	0.01\\
88.24	0.01\\
88.25	0.01\\
88.26	0.01\\
88.27	0.01\\
88.28	0.01\\
88.29	0.01\\
88.3	0.01\\
88.31	0.01\\
88.32	0.01\\
88.33	0.01\\
88.34	0.01\\
88.35	0.01\\
88.36	0.01\\
88.37	0.01\\
88.38	0.01\\
88.39	0.01\\
88.4	0.01\\
88.41	0.01\\
88.42	0.01\\
88.43	0.01\\
88.44	0.01\\
88.45	0.01\\
88.46	0.01\\
88.47	0.01\\
88.48	0.01\\
88.49	0.01\\
88.5	0.01\\
88.51	0.01\\
88.52	0.01\\
88.53	0.01\\
88.54	0.01\\
88.55	0.01\\
88.56	0.01\\
88.57	0.01\\
88.58	0.01\\
88.59	0.01\\
88.6	0.01\\
88.61	0.01\\
88.62	0.01\\
88.63	0.01\\
88.64	0.01\\
88.65	0.01\\
88.66	0.01\\
88.67	0.01\\
88.68	0.01\\
88.69	0.01\\
88.7	0.01\\
88.71	0.01\\
88.72	0.01\\
88.73	0.01\\
88.74	0.01\\
88.75	0.01\\
88.76	0.01\\
88.77	0.01\\
88.78	0.01\\
88.79	0.01\\
88.8	0.01\\
88.81	0.01\\
88.82	0.01\\
88.83	0.01\\
88.84	0.01\\
88.85	0.01\\
88.86	0.01\\
88.87	0.01\\
88.88	0.01\\
88.89	0.01\\
88.9	0.01\\
88.91	0.01\\
88.92	0.01\\
88.93	0.01\\
88.94	0.01\\
88.95	0.01\\
88.96	0.01\\
88.97	0.01\\
88.98	0.01\\
88.99	0.01\\
89	0.01\\
89.01	0.01\\
89.02	0.01\\
89.03	0.01\\
89.04	0.01\\
89.05	0.01\\
89.06	0.01\\
89.07	0.01\\
89.08	0.01\\
89.09	0.01\\
89.1	0.01\\
89.11	0.01\\
89.12	0.01\\
89.13	0.01\\
89.14	0.01\\
89.15	0.01\\
89.16	0.01\\
89.17	0.01\\
89.18	0.01\\
89.19	0.01\\
89.2	0.01\\
89.21	0.01\\
89.22	0.01\\
89.23	0.01\\
89.24	0.01\\
89.25	0.01\\
89.26	0.01\\
89.27	0.01\\
89.28	0.01\\
89.29	0.01\\
89.3	0.01\\
89.31	0.01\\
89.32	0.01\\
89.33	0.01\\
89.34	0.01\\
89.35	0.01\\
89.36	0.01\\
89.37	0.01\\
89.38	0.01\\
89.39	0.01\\
89.4	0.01\\
89.41	0.01\\
89.42	0.01\\
89.43	0.01\\
89.44	0.01\\
89.45	0.01\\
89.46	0.01\\
89.47	0.01\\
89.48	0.01\\
89.49	0.01\\
89.5	0.01\\
89.51	0.01\\
89.52	0.01\\
89.53	0.01\\
89.54	0.01\\
89.55	0.01\\
89.56	0.01\\
89.57	0.01\\
89.58	0.01\\
89.59	0.01\\
89.6	0.01\\
89.61	0.01\\
89.62	0.01\\
89.63	0.01\\
89.64	0.01\\
89.65	0.01\\
89.66	0.01\\
89.67	0.01\\
89.68	0.01\\
89.69	0.01\\
89.7	0.01\\
89.71	0.01\\
89.72	0.01\\
89.73	0.01\\
89.74	0.01\\
89.75	0.01\\
89.76	0.01\\
89.77	0.01\\
89.78	0.01\\
89.79	0.01\\
89.8	0.01\\
89.81	0.01\\
89.82	0.01\\
89.83	0.01\\
89.84	0.01\\
89.85	0.01\\
89.86	0.01\\
89.87	0.01\\
89.88	0.01\\
89.89	0.01\\
89.9	0.01\\
89.91	0.01\\
89.92	0.01\\
89.93	0.01\\
89.94	0.01\\
89.95	0.01\\
89.96	0.01\\
89.97	0.01\\
89.98	0.01\\
89.99	0.01\\
90	0.01\\
90.01	0.01\\
90.02	0.01\\
90.03	0.01\\
90.04	0.01\\
90.05	0.01\\
90.06	0.01\\
90.07	0.01\\
90.08	0.01\\
90.09	0.01\\
90.1	0.01\\
90.11	0.01\\
90.12	0.01\\
90.13	0.01\\
90.14	0.01\\
90.15	0.01\\
90.16	0.01\\
90.17	0.01\\
90.18	0.01\\
90.19	0.01\\
90.2	0.01\\
90.21	0.01\\
90.22	0.01\\
90.23	0.01\\
90.24	0.01\\
90.25	0.01\\
90.26	0.01\\
90.27	0.01\\
90.28	0.01\\
90.29	0.01\\
90.3	0.01\\
90.31	0.01\\
90.32	0.01\\
90.33	0.01\\
90.34	0.01\\
90.35	0.01\\
90.36	0.01\\
90.37	0.01\\
90.38	0.01\\
90.39	0.01\\
90.4	0.01\\
90.41	0.01\\
90.42	0.01\\
90.43	0.01\\
90.44	0.01\\
90.45	0.01\\
90.46	0.01\\
90.47	0.01\\
90.48	0.01\\
90.49	0.01\\
90.5	0.01\\
90.51	0.01\\
90.52	0.01\\
90.53	0.01\\
90.54	0.01\\
90.55	0.01\\
90.56	0.01\\
90.57	0.01\\
90.58	0.01\\
90.59	0.01\\
90.6	0.01\\
90.61	0.01\\
90.62	0.01\\
90.63	0.01\\
90.64	0.01\\
90.65	0.01\\
90.66	0.01\\
90.67	0.01\\
90.68	0.01\\
90.69	0.01\\
90.7	0.01\\
90.71	0.01\\
90.72	0.01\\
90.73	0.01\\
90.74	0.01\\
90.75	0.01\\
90.76	0.01\\
90.77	0.01\\
90.78	0.01\\
90.79	0.01\\
90.8	0.01\\
90.81	0.01\\
90.82	0.01\\
90.83	0.01\\
90.84	0.01\\
90.85	0.01\\
90.86	0.01\\
90.87	0.01\\
90.88	0.01\\
90.89	0.01\\
90.9	0.01\\
90.91	0.01\\
90.92	0.01\\
90.93	0.01\\
90.94	0.01\\
90.95	0.01\\
90.96	0.01\\
90.97	0.01\\
90.98	0.01\\
90.99	0.01\\
91	0.01\\
91.01	0.01\\
91.02	0.01\\
91.03	0.01\\
91.04	0.01\\
91.05	0.01\\
91.06	0.01\\
91.07	0.01\\
91.08	0.01\\
91.09	0.01\\
91.1	0.01\\
91.11	0.01\\
91.12	0.01\\
91.13	0.01\\
91.14	0.01\\
91.15	0.01\\
91.16	0.01\\
91.17	0.01\\
91.18	0.01\\
91.19	0.01\\
91.2	0.01\\
91.21	0.01\\
91.22	0.01\\
91.23	0.01\\
91.24	0.01\\
91.25	0.01\\
91.26	0.01\\
91.27	0.01\\
91.28	0.01\\
91.29	0.01\\
91.3	0.01\\
91.31	0.01\\
91.32	0.01\\
91.33	0.01\\
91.34	0.01\\
91.35	0.01\\
91.36	0.01\\
91.37	0.01\\
91.38	0.01\\
91.39	0.01\\
91.4	0.01\\
91.41	0.01\\
91.42	0.01\\
91.43	0.01\\
91.44	0.01\\
91.45	0.01\\
91.46	0.01\\
91.47	0.01\\
91.48	0.01\\
91.49	0.01\\
91.5	0.01\\
91.51	0.01\\
91.52	0.01\\
91.53	0.01\\
91.54	0.01\\
91.55	0.01\\
91.56	0.01\\
91.57	0.01\\
91.58	0.01\\
91.59	0.01\\
91.6	0.01\\
91.61	0.01\\
91.62	0.01\\
91.63	0.01\\
91.64	0.01\\
91.65	0.01\\
91.66	0.01\\
91.67	0.01\\
91.68	0.01\\
91.69	0.01\\
91.7	0.01\\
91.71	0.01\\
91.72	0.01\\
91.73	0.01\\
91.74	0.01\\
91.75	0.01\\
91.76	0.01\\
91.77	0.01\\
91.78	0.01\\
91.79	0.01\\
91.8	0.01\\
91.81	0.01\\
91.82	0.01\\
91.83	0.01\\
91.84	0.01\\
91.85	0.01\\
91.86	0.01\\
91.87	0.01\\
91.88	0.01\\
91.89	0.01\\
91.9	0.01\\
91.91	0.01\\
91.92	0.01\\
91.93	0.01\\
91.94	0.01\\
91.95	0.01\\
91.96	0.01\\
91.97	0.01\\
91.98	0.01\\
91.99	0.01\\
92	0.01\\
92.01	0.01\\
92.02	0.01\\
92.03	0.01\\
92.04	0.01\\
92.05	0.01\\
92.06	0.01\\
92.07	0.01\\
92.08	0.01\\
92.09	0.01\\
92.1	0.01\\
92.11	0.01\\
92.12	0.01\\
92.13	0.01\\
92.14	0.01\\
92.15	0.01\\
92.16	0.01\\
92.17	0.01\\
92.18	0.01\\
92.19	0.01\\
92.2	0.01\\
92.21	0.01\\
92.22	0.01\\
92.23	0.01\\
92.24	0.01\\
92.25	0.01\\
92.26	0.01\\
92.27	0.01\\
92.28	0.01\\
92.29	0.01\\
92.3	0.01\\
92.31	0.01\\
92.32	0.01\\
92.33	0.01\\
92.34	0.01\\
92.35	0.01\\
92.36	0.01\\
92.37	0.01\\
92.38	0.01\\
92.39	0.01\\
92.4	0.01\\
92.41	0.01\\
92.42	0.01\\
92.43	0.01\\
92.44	0.01\\
92.45	0.01\\
92.46	0.01\\
92.47	0.01\\
92.48	0.01\\
92.49	0.01\\
92.5	0.01\\
92.51	0.01\\
92.52	0.01\\
92.53	0.01\\
92.54	0.01\\
92.55	0.01\\
92.56	0.01\\
92.57	0.01\\
92.58	0.01\\
92.59	0.01\\
92.6	0.01\\
92.61	0.01\\
92.62	0.01\\
92.63	0.01\\
92.64	0.01\\
92.65	0.01\\
92.66	0.01\\
92.67	0.01\\
92.68	0.01\\
92.69	0.01\\
92.7	0.01\\
92.71	0.01\\
92.72	0.01\\
92.73	0.01\\
92.74	0.01\\
92.75	0.01\\
92.76	0.01\\
92.77	0.01\\
92.78	0.01\\
92.79	0.01\\
92.8	0.01\\
92.81	0.01\\
92.82	0.01\\
92.83	0.01\\
92.84	0.01\\
92.85	0.01\\
92.86	0.01\\
92.87	0.01\\
92.88	0.01\\
92.89	0.01\\
92.9	0.01\\
92.91	0.01\\
92.92	0.01\\
92.93	0.01\\
92.94	0.01\\
92.95	0.01\\
92.96	0.01\\
92.97	0.01\\
92.98	0.01\\
92.99	0.01\\
93	0.01\\
93.01	0.01\\
93.02	0.01\\
93.03	0.01\\
93.04	0.01\\
93.05	0.01\\
93.06	0.01\\
93.07	0.01\\
93.08	0.01\\
93.09	0.01\\
93.1	0.01\\
93.11	0.01\\
93.12	0.01\\
93.13	0.01\\
93.14	0.01\\
93.15	0.01\\
93.16	0.01\\
93.17	0.01\\
93.18	0.01\\
93.19	0.01\\
93.2	0.01\\
93.21	0.01\\
93.22	0.01\\
93.23	0.01\\
93.24	0.01\\
93.25	0.01\\
93.26	0.01\\
93.27	0.01\\
93.28	0.01\\
93.29	0.01\\
93.3	0.01\\
93.31	0.01\\
93.32	0.01\\
93.33	0.01\\
93.34	0.01\\
93.35	0.01\\
93.36	0.01\\
93.37	0.01\\
93.38	0.01\\
93.39	0.01\\
93.4	0.01\\
93.41	0.01\\
93.42	0.01\\
93.43	0.01\\
93.44	0.01\\
93.45	0.01\\
93.46	0.01\\
93.47	0.01\\
93.48	0.01\\
93.49	0.01\\
93.5	0.01\\
93.51	0.01\\
93.52	0.01\\
93.53	0.01\\
93.54	0.01\\
93.55	0.01\\
93.56	0.01\\
93.57	0.01\\
93.58	0.01\\
93.59	0.01\\
93.6	0.01\\
93.61	0.01\\
93.62	0.01\\
93.63	0.01\\
93.64	0.01\\
93.65	0.01\\
93.66	0.01\\
93.67	0.01\\
93.68	0.01\\
93.69	0.01\\
93.7	0.01\\
93.71	0.01\\
93.72	0.01\\
93.73	0.01\\
93.74	0.01\\
93.75	0.01\\
93.76	0.01\\
93.77	0.01\\
93.78	0.01\\
93.79	0.01\\
93.8	0.01\\
93.81	0.01\\
93.82	0.01\\
93.83	0.01\\
93.84	0.01\\
93.85	0.01\\
93.86	0.01\\
93.87	0.01\\
93.88	0.01\\
93.89	0.01\\
93.9	0.01\\
93.91	0.01\\
93.92	0.01\\
93.93	0.01\\
93.94	0.01\\
93.95	0.01\\
93.96	0.01\\
93.97	0.01\\
93.98	0.01\\
93.99	0.01\\
94	0.01\\
94.01	0.01\\
94.02	0.01\\
94.03	0.01\\
94.04	0.01\\
94.05	0.01\\
94.06	0.01\\
94.07	0.01\\
94.08	0.01\\
94.09	0.01\\
94.1	0.01\\
94.11	0.01\\
94.12	0.01\\
94.13	0.01\\
94.14	0.01\\
94.15	0.01\\
94.16	0.01\\
94.17	0.01\\
94.18	0.01\\
94.19	0.01\\
94.2	0.01\\
94.21	0.01\\
94.22	0.01\\
94.23	0.01\\
94.24	0.01\\
94.25	0.01\\
94.26	0.01\\
94.27	0.01\\
94.28	0.01\\
94.29	0.01\\
94.3	0.01\\
94.31	0.01\\
94.32	0.01\\
94.33	0.01\\
94.34	0.01\\
94.35	0.01\\
94.36	0.01\\
94.37	0.01\\
94.38	0.01\\
94.39	0.01\\
94.4	0.01\\
94.41	0.01\\
94.42	0.01\\
94.43	0.01\\
94.44	0.01\\
94.45	0.01\\
94.46	0.01\\
94.47	0.01\\
94.48	0.01\\
94.49	0.01\\
94.5	0.01\\
94.51	0.01\\
94.52	0.01\\
94.53	0.01\\
94.54	0.01\\
94.55	0.01\\
94.56	0.01\\
94.57	0.01\\
94.58	0.01\\
94.59	0.01\\
94.6	0.01\\
94.61	0.01\\
94.62	0.01\\
94.63	0.01\\
94.64	0.01\\
94.65	0.01\\
94.66	0.01\\
94.67	0.01\\
94.68	0.01\\
94.69	0.01\\
94.7	0.01\\
94.71	0.01\\
94.72	0.01\\
94.73	0.01\\
94.74	0.01\\
94.75	0.01\\
94.76	0.01\\
94.77	0.01\\
94.78	0.01\\
94.79	0.01\\
94.8	0.01\\
94.81	0.01\\
94.82	0.01\\
94.83	0.01\\
94.84	0.01\\
94.85	0.01\\
94.86	0.01\\
94.87	0.01\\
94.88	0.01\\
94.89	0.01\\
94.9	0.01\\
94.91	0.01\\
94.92	0.01\\
94.93	0.01\\
94.94	0.01\\
94.95	0.01\\
94.96	0.01\\
94.97	0.01\\
94.98	0.01\\
94.99	0.01\\
95	0.01\\
95.01	0.01\\
95.02	0.01\\
95.03	0.01\\
95.04	0.01\\
95.05	0.01\\
95.06	0.01\\
95.07	0.01\\
95.08	0.01\\
95.09	0.01\\
95.1	0.01\\
95.11	0.01\\
95.12	0.01\\
95.13	0.01\\
95.14	0.01\\
95.15	0.01\\
95.16	0.01\\
95.17	0.01\\
95.18	0.01\\
95.19	0.01\\
95.2	0.01\\
95.21	0.01\\
95.22	0.01\\
95.23	0.01\\
95.24	0.01\\
95.25	0.01\\
95.26	0.01\\
95.27	0.01\\
95.28	0.01\\
95.29	0.01\\
95.3	0.01\\
95.31	0.01\\
95.32	0.01\\
95.33	0.01\\
95.34	0.01\\
95.35	0.01\\
95.36	0.01\\
95.37	0.01\\
95.38	0.01\\
95.39	0.01\\
95.4	0.01\\
95.41	0.01\\
95.42	0.01\\
95.43	0.01\\
95.44	0.01\\
95.45	0.01\\
95.46	0.01\\
95.47	0.01\\
95.48	0.01\\
95.49	0.01\\
95.5	0.01\\
95.51	0.01\\
95.52	0.01\\
95.53	0.01\\
95.54	0.01\\
95.55	0.01\\
95.56	0.01\\
95.57	0.01\\
95.58	0.01\\
95.59	0.01\\
95.6	0.01\\
95.61	0.01\\
95.62	0.01\\
95.63	0.01\\
95.64	0.01\\
95.65	0.01\\
95.66	0.01\\
95.67	0.01\\
95.68	0.01\\
95.69	0.01\\
95.7	0.01\\
95.71	0.01\\
95.72	0.01\\
95.73	0.01\\
95.74	0.01\\
95.75	0.01\\
95.76	0.01\\
95.77	0.01\\
95.78	0.01\\
95.79	0.01\\
95.8	0.01\\
95.81	0.01\\
95.82	0.01\\
95.83	0.01\\
95.84	0.01\\
95.85	0.01\\
95.86	0.01\\
95.87	0.01\\
95.88	0.01\\
95.89	0.01\\
95.9	0.01\\
95.91	0.01\\
95.92	0.01\\
95.93	0.01\\
95.94	0.01\\
95.95	0.01\\
95.96	0.01\\
95.97	0.01\\
95.98	0.01\\
95.99	0.01\\
96	0.01\\
96.01	0.01\\
96.02	0.01\\
96.03	0.01\\
96.04	0.01\\
96.05	0.01\\
96.06	0.01\\
96.07	0.01\\
96.08	0.01\\
96.09	0.01\\
96.1	0.01\\
96.11	0.01\\
96.12	0.01\\
96.13	0.01\\
96.14	0.01\\
96.15	0.01\\
96.16	0.01\\
96.17	0.01\\
96.18	0.01\\
96.19	0.01\\
96.2	0.01\\
96.21	0.01\\
96.22	0.01\\
96.23	0.01\\
96.24	0.01\\
96.25	0.01\\
96.26	0.01\\
96.27	0.01\\
96.28	0.01\\
96.29	0.01\\
96.3	0.01\\
96.31	0.01\\
96.32	0.01\\
96.33	0.01\\
96.34	0.01\\
96.35	0.01\\
96.36	0.01\\
96.37	0.01\\
96.38	0.01\\
96.39	0.01\\
96.4	0.01\\
96.41	0.01\\
96.42	0.01\\
96.43	0.01\\
96.44	0.01\\
96.45	0.01\\
96.46	0.01\\
96.47	0.01\\
96.48	0.01\\
96.49	0.01\\
96.5	0.01\\
96.51	0.01\\
96.52	0.01\\
96.53	0.01\\
96.54	0.01\\
96.55	0.01\\
96.56	0.01\\
96.57	0.01\\
96.58	0.01\\
96.59	0.01\\
96.6	0.01\\
96.61	0.01\\
96.62	0.01\\
96.63	0.01\\
96.64	0.01\\
96.65	0.01\\
96.66	0.01\\
96.67	0.01\\
96.68	0.01\\
96.69	0.01\\
96.7	0.01\\
96.71	0.01\\
96.72	0.01\\
96.73	0.01\\
96.74	0.01\\
96.75	0.01\\
96.76	0.01\\
96.77	0.01\\
96.78	0.01\\
96.79	0.01\\
96.8	0.01\\
96.81	0.01\\
96.82	0.01\\
96.83	0.01\\
96.84	0.01\\
96.85	0.01\\
96.86	0.01\\
96.87	0.01\\
96.88	0.01\\
96.89	0.01\\
96.9	0.01\\
96.91	0.01\\
96.92	0.01\\
96.93	0.01\\
96.94	0.01\\
96.95	0.01\\
96.96	0.01\\
96.97	0.01\\
96.98	0.01\\
96.99	0.01\\
97	0.01\\
97.01	0.01\\
97.02	0.01\\
97.03	0.01\\
97.04	0.01\\
97.05	0.01\\
97.06	0.01\\
97.07	0.01\\
97.08	0.01\\
97.09	0.01\\
97.1	0.01\\
97.11	0.01\\
97.12	0.01\\
97.13	0.01\\
97.14	0.01\\
97.15	0.01\\
97.16	0.01\\
97.17	0.01\\
97.18	0.01\\
97.19	0.01\\
97.2	0.01\\
97.21	0.01\\
97.22	0.01\\
97.23	0.01\\
97.24	0.01\\
97.25	0.01\\
97.26	0.01\\
97.27	0.01\\
97.28	0.01\\
97.29	0.01\\
97.3	0.01\\
97.31	0.01\\
97.32	0.01\\
97.33	0.01\\
97.34	0.01\\
97.35	0.01\\
97.36	0.01\\
97.37	0.01\\
97.38	0.01\\
97.39	0.01\\
97.4	0.01\\
97.41	0.01\\
97.42	0.01\\
97.43	0.01\\
97.44	0.01\\
97.45	0.01\\
97.46	0.01\\
97.47	0.01\\
97.48	0.01\\
97.49	0.01\\
97.5	0.01\\
97.51	0.01\\
97.52	0.01\\
97.53	0.01\\
97.54	0.01\\
97.55	0.01\\
97.56	0.01\\
97.57	0.01\\
97.58	0.01\\
97.59	0.01\\
97.6	0.01\\
97.61	0.01\\
97.62	0.01\\
97.63	0.01\\
97.64	0.01\\
97.65	0.01\\
97.66	0.01\\
97.67	0.01\\
97.68	0.01\\
97.69	0.01\\
97.7	0.01\\
97.71	0.01\\
97.72	0.01\\
97.73	0.01\\
97.74	0.01\\
97.75	0.01\\
97.76	0.01\\
97.77	0.01\\
97.78	0.01\\
97.79	0.01\\
97.8	0.01\\
97.81	0.01\\
97.82	0.01\\
97.83	0.01\\
97.84	0.01\\
97.85	0.01\\
97.86	0.01\\
97.87	0.01\\
97.88	0.01\\
97.89	0.01\\
97.9	0.01\\
97.91	0.01\\
97.92	0.01\\
97.93	0.01\\
97.94	0.01\\
97.95	0.01\\
97.96	0.01\\
97.97	0.01\\
97.98	0.01\\
97.99	0.01\\
98	0.01\\
98.01	0.01\\
98.02	0.01\\
98.03	0.01\\
98.04	0.01\\
98.05	0.01\\
98.06	0.01\\
98.07	0.01\\
98.08	0.01\\
98.09	0.01\\
98.1	0.01\\
98.11	0.01\\
98.12	0.01\\
98.13	0.01\\
98.14	0.01\\
98.15	0.01\\
98.16	0.01\\
98.17	0.01\\
98.18	0.01\\
98.19	0.01\\
98.2	0.01\\
98.21	0.01\\
98.22	0.01\\
98.23	0.01\\
98.24	0.01\\
98.25	0.01\\
98.26	0.01\\
98.27	0.01\\
98.28	0.01\\
98.29	0.01\\
98.3	0.01\\
98.31	0.01\\
98.32	0.01\\
98.33	0.01\\
98.34	0.01\\
98.35	0.01\\
98.36	0.01\\
98.37	0.01\\
98.38	0.01\\
98.39	0.01\\
98.4	0.01\\
98.41	0.01\\
98.42	0.01\\
98.43	0.01\\
98.44	0.01\\
98.45	0.01\\
98.46	0.01\\
98.47	0.01\\
98.48	0.01\\
98.49	0.01\\
98.5	0.01\\
98.51	0.01\\
98.52	0.01\\
98.53	0.01\\
98.54	0.01\\
98.55	0.01\\
98.56	0.01\\
98.57	0.01\\
98.58	0.01\\
98.59	0.01\\
98.6	0.01\\
98.61	0.01\\
98.62	0.01\\
98.63	0.00999505388205829\\
98.64	0.00995136735966437\\
98.65	0.00990739270767156\\
98.66	0.00986312733885803\\
98.67	0.00981856863916014\\
98.68	0.00977371396734272\\
98.69	0.00972856065466484\\
98.7	0.00968310600435172\\
98.71	0.00963734729120488\\
98.72	0.00959128176124934\\
98.73	0.00954490663137584\\
98.74	0.0094982264449798\\
98.75	0.0094512464499936\\
98.76	0.00940396402073698\\
98.77	0.00935637650432044\\
98.78	0.00930848121972065\\
98.79	0.00926027563724096\\
98.8	0.00921176101813882\\
98.81	0.00916293466371986\\
98.82	0.00911379384725402\\
98.83	0.0090643358136209\\
98.84	0.00901455777894991\\
98.85	0.00896445693025523\\
98.86	0.00891403042506543\\
98.87	0.00886327539104762\\
98.88	0.00881218892562615\\
98.89	0.00876076809559573\\
98.9	0.00870900993672882\\
98.91	0.00865691145337731\\
98.92	0.00860446961553091\\
98.93	0.00855168136056717\\
98.94	0.00849854359313442\\
98.95	0.00844505318472779\\
98.96	0.00839120697325908\\
98.97	0.00833700176262016\\
98.98	0.00828243432223994\\
98.99	0.00822750138663475\\
99	0.00817219965495204\\
99.01	0.00811652579050724\\
99.02	0.00806047642031374\\
99.03	0.00800404813460585\\
99.04	0.00794723748635457\\
99.05	0.00789004099077617\\
99.06	0.0078324551248333\\
99.07	0.00777447632672868\\
99.08	0.00771610099539109\\
99.09	0.00765732548995361\\
99.1	0.007598146129224\\
99.11	0.00753855919114704\\
99.12	0.00747856091225866\\
99.13	0.00741814748713178\\
99.14	0.00735731506781378\\
99.15	0.00729605976325524\\
99.16	0.00723437763873002\\
99.17	0.00717226471524643\\
99.18	0.00710971696894931\\
99.19	0.00704673033051301\\
99.2	0.00698330068452509\\
99.21	0.00691942386886035\\
99.22	0.00685509567404525\\
99.23	0.00679031184261236\\
99.24	0.00672506806844475\\
99.25	0.0066593599961101\\
99.26	0.00659318322018437\\
99.27	0.00652653328456483\\
99.28	0.00645940568177213\\
99.29	0.00639179585224147\\
99.3	0.00632369918360239\\
99.31	0.00625511100994711\\
99.32	0.00618602661108714\\
99.33	0.006116441211798\\
99.34	0.00604634998105173\\
99.35	0.00597574803123699\\
99.36	0.00590463041736653\\
99.37	0.00583299213627174\\
99.38	0.00576082812578402\\
99.39	0.00568813326389982\\
99.4	0.00561490236781182\\
99.41	0.00554113019304813\\
99.42	0.00546681143259714\\
99.43	0.00539194071601783\\
99.44	0.00531651260853528\\
99.45	0.00524052161012092\\
99.46	0.00516396215455726\\
99.47	0.00508682860848681\\
99.48	0.00500911527044477\\
99.49	0.00493081636987514\\
99.5	0.00485192606612993\\
99.51	0.00477243844745095\\
99.52	0.00469234752993405\\
99.53	0.00461164725647502\\
99.54	0.00453033149569714\\
99.55	0.00444839404085956\\
99.56	0.00436582860874636\\
99.57	0.00428262883853562\\
99.58	0.00419878829064817\\
99.59	0.00411430044557536\\
99.6	0.00402915870268544\\
99.61	0.00394335637900804\\
99.62	0.00385688670799605\\
99.63	0.00376974283826455\\
99.64	0.00368191783230589\\
99.65	0.00359340466518068\\
99.66	0.00350419622318361\\
99.67	0.00341428530248387\\
99.68	0.00332366460773911\\
99.69	0.00323232675068253\\
99.7	0.00314026424868202\\
99.71	0.00304746952327101\\
99.72	0.0029539348986498\\
99.73	0.0028596526001464\\
99.74	0.00276461475264767\\
99.75	0.00266881337900764\\
99.76	0.00257224039842299\\
99.77	0.0024748876247747\\
99.78	0.00237674676493486\\
99.79	0.00227780941703762\\
99.8	0.00217806706871307\\
99.81	0.00207751109528306\\
99.82	0.00197613275791767\\
99.83	0.0018739232017511\\
99.84	0.00177087345395577\\
99.85	0.00166697442177312\\
99.86	0.00156221689049989\\
99.87	0.00145659152142833\\
99.88	0.00135008884973875\\
99.89	0.00124269928234298\\
99.9	0.00113441309567685\\
99.91	0.00102522043344024\\
99.92	0.000915111304282583\\
99.93	0.000804075579432168\\
99.94	0.000692102990267111\\
99.95	0.000579183125826001\\
99.96	0.00046530543025603\\
99.97	0.000350459200196341\\
99.98	0.000234633582094232\\
99.99	0.000117817569451738\\
100	0\\
};
\addlegendentry{$q=-1$};

\addplot [color=black,solid,forget plot]
  table[row sep=crcr]{%
0.01	0.00837856495180774\\
0.02	0.00837856495180774\\
0.03	0.00837856495180774\\
0.04	0.00837856495180774\\
0.05	0.00837856495180774\\
0.06	0.00837856495180774\\
0.07	0.00837856495180774\\
0.08	0.00837856495180774\\
0.09	0.00837856495180774\\
0.1	0.00837856495180774\\
0.11	0.00837856495180774\\
0.12	0.00837856495180774\\
0.13	0.00837856495180774\\
0.14	0.00837856495180774\\
0.15	0.00837856495180774\\
0.16	0.00837856495180774\\
0.17	0.00837856495180774\\
0.18	0.00837856495180774\\
0.19	0.00837856495180774\\
0.2	0.00837856495180774\\
0.21	0.00837856495180774\\
0.22	0.00837856495180774\\
0.23	0.00837856495180774\\
0.24	0.00837856495180774\\
0.25	0.00837856495180774\\
0.26	0.00837856495180774\\
0.27	0.00837856495180774\\
0.28	0.00837856495180774\\
0.29	0.00837856495180774\\
0.3	0.00837856495180774\\
0.31	0.00837856495180774\\
0.32	0.00837856495180774\\
0.33	0.00837856495180774\\
0.34	0.00837856495180774\\
0.35	0.00837856495180774\\
0.36	0.00837856495180774\\
0.37	0.00837856495180774\\
0.38	0.00837856495180774\\
0.39	0.00837856495180774\\
0.4	0.00837856495180774\\
0.41	0.00837856495180774\\
0.42	0.00837856495180774\\
0.43	0.00837856495180774\\
0.44	0.00837856495180774\\
0.45	0.00837856495180774\\
0.46	0.00837856495180774\\
0.47	0.00837856495180774\\
0.48	0.00837856495180774\\
0.49	0.00837856495180774\\
0.5	0.00837856495180774\\
0.51	0.00837856495180774\\
0.52	0.00837856495180774\\
0.53	0.00837856495180774\\
0.54	0.00837856495180774\\
0.55	0.00837856495180774\\
0.56	0.00837856495180774\\
0.57	0.00837856495180774\\
0.58	0.00837856495180774\\
0.59	0.00837856495180774\\
0.6	0.00837856495180774\\
0.61	0.00837856495180774\\
0.62	0.00837856495180774\\
0.63	0.00837856495180774\\
0.64	0.00837856495180774\\
0.65	0.00837856495180774\\
0.66	0.00837856495180774\\
0.67	0.00837856495180774\\
0.68	0.00837856495180774\\
0.69	0.00837856495180774\\
0.7	0.00837856495180774\\
0.71	0.00837856495180774\\
0.72	0.00837856495180774\\
0.73	0.00837856495180774\\
0.74	0.00837856495180774\\
0.75	0.00837856495180774\\
0.76	0.00837856495180774\\
0.77	0.00837856495180774\\
0.78	0.00837856495180774\\
0.79	0.00837856495180774\\
0.8	0.00837856495180774\\
0.81	0.00837856495180774\\
0.82	0.00837856495180774\\
0.83	0.00837856495180774\\
0.84	0.00837856495180774\\
0.85	0.00837856495180774\\
0.86	0.00837856495180774\\
0.87	0.00837856495180774\\
0.88	0.00837856495180774\\
0.89	0.00837856495180774\\
0.9	0.00837856495180774\\
0.91	0.00837856495180774\\
0.92	0.00837856495180774\\
0.93	0.00837856495180774\\
0.94	0.00837856495180774\\
0.95	0.00837856495180774\\
0.96	0.00837856495180774\\
0.97	0.00837856495180774\\
0.98	0.00837856495180774\\
0.99	0.00837856495180774\\
1	0.00837856495180774\\
1.01	0.00837856495180774\\
1.02	0.00837856495180774\\
1.03	0.00837856495180774\\
1.04	0.00837856495180774\\
1.05	0.00837856495180774\\
1.06	0.00837856495180774\\
1.07	0.00837856495180774\\
1.08	0.00837856495180774\\
1.09	0.00837856495180774\\
1.1	0.00837856495180774\\
1.11	0.00837856495180774\\
1.12	0.00837856495180774\\
1.13	0.00837856495180774\\
1.14	0.00837856495180774\\
1.15	0.00837856495180774\\
1.16	0.00837856495180774\\
1.17	0.00837856495180774\\
1.18	0.00837856495180774\\
1.19	0.00837856495180774\\
1.2	0.00837856495180774\\
1.21	0.00837856495180774\\
1.22	0.00837856495180774\\
1.23	0.00837856495180774\\
1.24	0.00837856495180774\\
1.25	0.00837856495180774\\
1.26	0.00837856495180774\\
1.27	0.00837856495180774\\
1.28	0.00837856495180774\\
1.29	0.00837856495180774\\
1.3	0.00837856495180774\\
1.31	0.00837856495180774\\
1.32	0.00837856495180774\\
1.33	0.00837856495180774\\
1.34	0.00837856495180774\\
1.35	0.00837856495180774\\
1.36	0.00837856495180774\\
1.37	0.00837856495180774\\
1.38	0.00837856495180774\\
1.39	0.00837856495180774\\
1.4	0.00837856495180774\\
1.41	0.00837856495180774\\
1.42	0.00837856495180774\\
1.43	0.00837856495180774\\
1.44	0.00837856495180774\\
1.45	0.00837856495180774\\
1.46	0.00837856495180774\\
1.47	0.00837856495180774\\
1.48	0.00837856495180774\\
1.49	0.00837856495180774\\
1.5	0.00837856495180774\\
1.51	0.00837856495180774\\
1.52	0.00837856495180774\\
1.53	0.00837856495180774\\
1.54	0.00837856495180774\\
1.55	0.00837856495180774\\
1.56	0.00837856495180774\\
1.57	0.00837856495180774\\
1.58	0.00837856495180774\\
1.59	0.00837856495180774\\
1.6	0.00837856495180774\\
1.61	0.00837856495180774\\
1.62	0.00837856495180774\\
1.63	0.00837856495180774\\
1.64	0.00837856495180774\\
1.65	0.00837856495180774\\
1.66	0.00837856495180774\\
1.67	0.00837856495180774\\
1.68	0.00837856495180774\\
1.69	0.00837856495180774\\
1.7	0.00837856495180774\\
1.71	0.00837856495180774\\
1.72	0.00837856495180774\\
1.73	0.00837856495180774\\
1.74	0.00837856495180774\\
1.75	0.00837856495180774\\
1.76	0.00837856495180774\\
1.77	0.00837856495180774\\
1.78	0.00837856495180774\\
1.79	0.00837856495180774\\
1.8	0.00837856495180774\\
1.81	0.00837856495180774\\
1.82	0.00837856495180774\\
1.83	0.00837856495180774\\
1.84	0.00837856495180774\\
1.85	0.00837856495180774\\
1.86	0.00837856495180774\\
1.87	0.00837856495180774\\
1.88	0.00837856495180774\\
1.89	0.00837856495180774\\
1.9	0.00837856495180774\\
1.91	0.00837856495180774\\
1.92	0.00837856495180774\\
1.93	0.00837856495180774\\
1.94	0.00837856495180774\\
1.95	0.00837856495180774\\
1.96	0.00837856495180774\\
1.97	0.00837856495180774\\
1.98	0.00837856495180774\\
1.99	0.00837856495180774\\
2	0.00837856495180774\\
2.01	0.00837856495180774\\
2.02	0.00837856495180774\\
2.03	0.00837856495180774\\
2.04	0.00837856495180774\\
2.05	0.00837856495180774\\
2.06	0.00837856495180774\\
2.07	0.00837856495180774\\
2.08	0.00837856495180774\\
2.09	0.00837856495180774\\
2.1	0.00837856495180774\\
2.11	0.00837856495180774\\
2.12	0.00837856495180774\\
2.13	0.00837856495180774\\
2.14	0.00837856495180774\\
2.15	0.00837856495180774\\
2.16	0.00837856495180774\\
2.17	0.00837856495180774\\
2.18	0.00837856495180774\\
2.19	0.00837856495180774\\
2.2	0.00837856495180774\\
2.21	0.00837856495180774\\
2.22	0.00837856495180774\\
2.23	0.00837856495180774\\
2.24	0.00837856495180774\\
2.25	0.00837856495180774\\
2.26	0.00837856495180774\\
2.27	0.00837856495180774\\
2.28	0.00837856495180774\\
2.29	0.00837856495180774\\
2.3	0.00837856495180774\\
2.31	0.00837856495180774\\
2.32	0.00837856495180774\\
2.33	0.00837856495180774\\
2.34	0.00837856495180774\\
2.35	0.00837856495180774\\
2.36	0.00837856495180774\\
2.37	0.00837856495180774\\
2.38	0.00837856495180774\\
2.39	0.00837856495180774\\
2.4	0.00837856495180774\\
2.41	0.00837856495180774\\
2.42	0.00837856495180774\\
2.43	0.00837856495180774\\
2.44	0.00837856495180774\\
2.45	0.00837856495180774\\
2.46	0.00837856495180774\\
2.47	0.00837856495180774\\
2.48	0.00837856495180774\\
2.49	0.00837856495180774\\
2.5	0.00837856495180774\\
2.51	0.00837856495180774\\
2.52	0.00837856495180774\\
2.53	0.00837856495180774\\
2.54	0.00837856495180774\\
2.55	0.00837856495180774\\
2.56	0.00837856495180774\\
2.57	0.00837856495180774\\
2.58	0.00837856495180774\\
2.59	0.00837856495180774\\
2.6	0.00837856495180774\\
2.61	0.00837856495180774\\
2.62	0.00837856495180774\\
2.63	0.00837856495180774\\
2.64	0.00837856495180774\\
2.65	0.00837856495180774\\
2.66	0.00837856495180774\\
2.67	0.00837856495180774\\
2.68	0.00837856495180774\\
2.69	0.00837856495180774\\
2.7	0.00837856495180774\\
2.71	0.00837856495180774\\
2.72	0.00837856495180774\\
2.73	0.00837856495180774\\
2.74	0.00837856495180774\\
2.75	0.00837856495180774\\
2.76	0.00837856495180774\\
2.77	0.00837856495180774\\
2.78	0.00837856495180774\\
2.79	0.00837856495180774\\
2.8	0.00837856495180774\\
2.81	0.00837856495180774\\
2.82	0.00837856495180774\\
2.83	0.00837856495180774\\
2.84	0.00837856495180774\\
2.85	0.00837856495180774\\
2.86	0.00837856495180774\\
2.87	0.00837856495180774\\
2.88	0.00837856495180774\\
2.89	0.00837856495180774\\
2.9	0.00837856495180774\\
2.91	0.00837856495180774\\
2.92	0.00837856495180774\\
2.93	0.00837856495180774\\
2.94	0.00837856495180774\\
2.95	0.00837856495180774\\
2.96	0.00837856495180774\\
2.97	0.00837856495180774\\
2.98	0.00837856495180774\\
2.99	0.00837856495180774\\
3	0.00837856495180774\\
3.01	0.00837856495180774\\
3.02	0.00837856495180774\\
3.03	0.00837856495180774\\
3.04	0.00837856495180774\\
3.05	0.00837856495180774\\
3.06	0.00837856495180774\\
3.07	0.00837856495180774\\
3.08	0.00837856495180774\\
3.09	0.00837856495180774\\
3.1	0.00837856495180774\\
3.11	0.00837856495180774\\
3.12	0.00837856495180774\\
3.13	0.00837856495180774\\
3.14	0.00837856495180774\\
3.15	0.00837856495180774\\
3.16	0.00837856495180774\\
3.17	0.00837856495180774\\
3.18	0.00837856495180774\\
3.19	0.00837856495180774\\
3.2	0.00837856495180774\\
3.21	0.00837856495180774\\
3.22	0.00837856495180774\\
3.23	0.00837856495180774\\
3.24	0.00837856495180774\\
3.25	0.00837856495180774\\
3.26	0.00837856495180774\\
3.27	0.00837856495180774\\
3.28	0.00837856495180774\\
3.29	0.00837856495180774\\
3.3	0.00837856495180774\\
3.31	0.00837856495180774\\
3.32	0.00837856495180774\\
3.33	0.00837856495180774\\
3.34	0.00837856495180774\\
3.35	0.00837856495180774\\
3.36	0.00837856495180774\\
3.37	0.00837856495180774\\
3.38	0.00837856495180774\\
3.39	0.00837856495180774\\
3.4	0.00837856495180774\\
3.41	0.00837856495180774\\
3.42	0.00837856495180774\\
3.43	0.00837856495180774\\
3.44	0.00837856495180774\\
3.45	0.00837856495180774\\
3.46	0.00837856495180774\\
3.47	0.00837856495180774\\
3.48	0.00837856495180774\\
3.49	0.00837856495180774\\
3.5	0.00837856495180774\\
3.51	0.00837856495180774\\
3.52	0.00837856495180774\\
3.53	0.00837856495180774\\
3.54	0.00837856495180774\\
3.55	0.00837856495180774\\
3.56	0.00837856495180774\\
3.57	0.00837856495180774\\
3.58	0.00837856495180774\\
3.59	0.00837856495180774\\
3.6	0.00837856495180774\\
3.61	0.00837856495180774\\
3.62	0.00837856495180774\\
3.63	0.00837856495180774\\
3.64	0.00837856495180774\\
3.65	0.00837856495180774\\
3.66	0.00837856495180774\\
3.67	0.00837856495180774\\
3.68	0.00837856495180774\\
3.69	0.00837856495180774\\
3.7	0.00837856495180774\\
3.71	0.00837856495180774\\
3.72	0.00837856495180774\\
3.73	0.00837856495180774\\
3.74	0.00837856495180774\\
3.75	0.00837856495180774\\
3.76	0.00837856495180774\\
3.77	0.00837856495180774\\
3.78	0.00837856495180774\\
3.79	0.00837856495180774\\
3.8	0.00837856495180774\\
3.81	0.00837856495180774\\
3.82	0.00837856495180774\\
3.83	0.00837856495180774\\
3.84	0.00837856495180774\\
3.85	0.00837856495180774\\
3.86	0.00837856495180774\\
3.87	0.00837856495180774\\
3.88	0.00837856495180774\\
3.89	0.00837856495180774\\
3.9	0.00837856495180774\\
3.91	0.00837856495180774\\
3.92	0.00837856495180774\\
3.93	0.00837856495180774\\
3.94	0.00837856495180774\\
3.95	0.00837856495180774\\
3.96	0.00837856495180774\\
3.97	0.00837856495180774\\
3.98	0.00837856495180774\\
3.99	0.00837856495180774\\
4	0.00837856495180774\\
4.01	0.00837856495180774\\
4.02	0.00837856495180774\\
4.03	0.00837856495180774\\
4.04	0.00837856495180774\\
4.05	0.00837856495180774\\
4.06	0.00837856495180774\\
4.07	0.00837856495180774\\
4.08	0.00837856495180774\\
4.09	0.00837856495180774\\
4.1	0.00837856495180774\\
4.11	0.00837856495180774\\
4.12	0.00837856495180774\\
4.13	0.00837856495180774\\
4.14	0.00837856495180774\\
4.15	0.00837856495180774\\
4.16	0.00837856495180774\\
4.17	0.00837856495180774\\
4.18	0.00837856495180774\\
4.19	0.00837856495180774\\
4.2	0.00837856495180774\\
4.21	0.00837856495180774\\
4.22	0.00837856495180774\\
4.23	0.00837856495180774\\
4.24	0.00837856495180774\\
4.25	0.00837856495180774\\
4.26	0.00837856495180774\\
4.27	0.00837856495180774\\
4.28	0.00837856495180774\\
4.29	0.00837856495180774\\
4.3	0.00837856495180774\\
4.31	0.00837856495180774\\
4.32	0.00837856495180774\\
4.33	0.00837856495180774\\
4.34	0.00837856495180774\\
4.35	0.00837856495180774\\
4.36	0.00837856495180774\\
4.37	0.00837856495180774\\
4.38	0.00837856495180774\\
4.39	0.00837856495180774\\
4.4	0.00837856495180774\\
4.41	0.00837856495180774\\
4.42	0.00837856495180774\\
4.43	0.00837856495180774\\
4.44	0.00837856495180774\\
4.45	0.00837856495180774\\
4.46	0.00837856495180774\\
4.47	0.00837856495180774\\
4.48	0.00837856495180774\\
4.49	0.00837856495180774\\
4.5	0.00837856495180774\\
4.51	0.00837856495180774\\
4.52	0.00837856495180774\\
4.53	0.00837856495180774\\
4.54	0.00837856495180774\\
4.55	0.00837856495180774\\
4.56	0.00837856495180774\\
4.57	0.00837856495180774\\
4.58	0.00837856495180774\\
4.59	0.00837856495180774\\
4.6	0.00837856495180774\\
4.61	0.00837856495180774\\
4.62	0.00837856495180774\\
4.63	0.00837856495180774\\
4.64	0.00837856495180774\\
4.65	0.00837856495180774\\
4.66	0.00837856495180774\\
4.67	0.00837856495180774\\
4.68	0.00837856495180774\\
4.69	0.00837856495180774\\
4.7	0.00837856495180774\\
4.71	0.00837856495180774\\
4.72	0.00837856495180774\\
4.73	0.00837856495180774\\
4.74	0.00837856495180774\\
4.75	0.00837856495180774\\
4.76	0.00837856495180774\\
4.77	0.00837856495180774\\
4.78	0.00837856495180774\\
4.79	0.00837856495180774\\
4.8	0.00837856495180774\\
4.81	0.00837856495180774\\
4.82	0.00837856495180774\\
4.83	0.00837856495180774\\
4.84	0.00837856495180774\\
4.85	0.00837856495180774\\
4.86	0.00837856495180774\\
4.87	0.00837856495180774\\
4.88	0.00837856495180774\\
4.89	0.00837856495180774\\
4.9	0.00837856495180774\\
4.91	0.00837856495180774\\
4.92	0.00837856495180774\\
4.93	0.00837856495180774\\
4.94	0.00837856495180774\\
4.95	0.00837856495180774\\
4.96	0.00837856495180774\\
4.97	0.00837856495180774\\
4.98	0.00837856495180774\\
4.99	0.00837856495180774\\
5	0.00837856495180774\\
5.01	0.00837856495180774\\
5.02	0.00837856495180774\\
5.03	0.00837856495180774\\
5.04	0.00837856495180774\\
5.05	0.00837856495180774\\
5.06	0.00837856495180774\\
5.07	0.00837856495180774\\
5.08	0.00837856495180774\\
5.09	0.00837856495180774\\
5.1	0.00837856495180774\\
5.11	0.00837856495180774\\
5.12	0.00837856495180774\\
5.13	0.00837856495180774\\
5.14	0.00837856495180774\\
5.15	0.00837856495180774\\
5.16	0.00837856495180774\\
5.17	0.00837856495180774\\
5.18	0.00837856495180774\\
5.19	0.00837856495180774\\
5.2	0.00837856495180774\\
5.21	0.00837856495180774\\
5.22	0.00837856495180774\\
5.23	0.00837856495180774\\
5.24	0.00837856495180774\\
5.25	0.00837856495180774\\
5.26	0.00837856495180774\\
5.27	0.00837856495180774\\
5.28	0.00837856495180774\\
5.29	0.00837856495180774\\
5.3	0.00837856495180774\\
5.31	0.00837856495180774\\
5.32	0.00837856495180774\\
5.33	0.00837856495180774\\
5.34	0.00837856495180774\\
5.35	0.00837856495180774\\
5.36	0.00837856495180774\\
5.37	0.00837856495180774\\
5.38	0.00837856495180774\\
5.39	0.00837856495180774\\
5.4	0.00837856495180774\\
5.41	0.00837856495180774\\
5.42	0.00837856495180774\\
5.43	0.00837856495180774\\
5.44	0.00837856495180774\\
5.45	0.00837856495180774\\
5.46	0.00837856495180774\\
5.47	0.00837856495180774\\
5.48	0.00837856495180774\\
5.49	0.00837856495180774\\
5.5	0.00837856495180774\\
5.51	0.00837856495180774\\
5.52	0.00837856495180774\\
5.53	0.00837856495180774\\
5.54	0.00837856495180774\\
5.55	0.00837856495180774\\
5.56	0.00837856495180774\\
5.57	0.00837856495180774\\
5.58	0.00837856495180774\\
5.59	0.00837856495180774\\
5.6	0.00837856495180774\\
5.61	0.00837856495180774\\
5.62	0.00837856495180774\\
5.63	0.00837856495180774\\
5.64	0.00837856495180774\\
5.65	0.00837856495180774\\
5.66	0.00837856495180774\\
5.67	0.00837856495180774\\
5.68	0.00837856495180774\\
5.69	0.00837856495180774\\
5.7	0.00837856495180774\\
5.71	0.00837856495180774\\
5.72	0.00837856495180774\\
5.73	0.00837856495180774\\
5.74	0.00837856495180774\\
5.75	0.00837856495180774\\
5.76	0.00837856495180774\\
5.77	0.00837856495180774\\
5.78	0.00837856495180774\\
5.79	0.00837856495180774\\
5.8	0.00837856495180774\\
5.81	0.00837856495180774\\
5.82	0.00837856495180774\\
5.83	0.00837856495180774\\
5.84	0.00837856495180774\\
5.85	0.00837856495180774\\
5.86	0.00837856495180774\\
5.87	0.00837856495180774\\
5.88	0.00837856495180774\\
5.89	0.00837856495180774\\
5.9	0.00837856495180774\\
5.91	0.00837856495180774\\
5.92	0.00837856495180774\\
5.93	0.00837856495180774\\
5.94	0.00837856495180774\\
5.95	0.00837856495180774\\
5.96	0.00837856495180774\\
5.97	0.00837856495180774\\
5.98	0.00837856495180774\\
5.99	0.00837856495180774\\
6	0.00837856495180774\\
6.01	0.00837856495180774\\
6.02	0.00837856495180774\\
6.03	0.00837856495180774\\
6.04	0.00837856495180774\\
6.05	0.00837856495180774\\
6.06	0.00837856495180774\\
6.07	0.00837856495180774\\
6.08	0.00837856495180774\\
6.09	0.00837856495180774\\
6.1	0.00837856495180774\\
6.11	0.00837856495180774\\
6.12	0.00837856495180774\\
6.13	0.00837856495180774\\
6.14	0.00837856495180774\\
6.15	0.00837856495180774\\
6.16	0.00837856495180774\\
6.17	0.00837856495180774\\
6.18	0.00837856495180774\\
6.19	0.00837856495180774\\
6.2	0.00837856495180774\\
6.21	0.00837856495180774\\
6.22	0.00837856495180774\\
6.23	0.00837856495180774\\
6.24	0.00837856495180774\\
6.25	0.00837856495180774\\
6.26	0.00837856495180774\\
6.27	0.00837856495180774\\
6.28	0.00837856495180774\\
6.29	0.00837856495180774\\
6.3	0.00837856495180774\\
6.31	0.00837856495180774\\
6.32	0.00837856495180774\\
6.33	0.00837856495180774\\
6.34	0.00837856495180774\\
6.35	0.00837856495180774\\
6.36	0.00837856495180774\\
6.37	0.00837856495180774\\
6.38	0.00837856495180774\\
6.39	0.00837856495180774\\
6.4	0.00837856495180774\\
6.41	0.00837856495180774\\
6.42	0.00837856495180774\\
6.43	0.00837856495180774\\
6.44	0.00837856495180774\\
6.45	0.00837856495180774\\
6.46	0.00837856495180774\\
6.47	0.00837856495180774\\
6.48	0.00837856495180774\\
6.49	0.00837856495180774\\
6.5	0.00837856495180774\\
6.51	0.00837856495180774\\
6.52	0.00837856495180774\\
6.53	0.00837856495180774\\
6.54	0.00837856495180774\\
6.55	0.00837856495180774\\
6.56	0.00837856495180774\\
6.57	0.00837856495180774\\
6.58	0.00837856495180774\\
6.59	0.00837856495180774\\
6.6	0.00837856495180774\\
6.61	0.00837856495180774\\
6.62	0.00837856495180774\\
6.63	0.00837856495180774\\
6.64	0.00837856495180774\\
6.65	0.00837856495180774\\
6.66	0.00837856495180774\\
6.67	0.00837856495180774\\
6.68	0.00837856495180774\\
6.69	0.00837856495180774\\
6.7	0.00837856495180774\\
6.71	0.00837856495180774\\
6.72	0.00837856495180774\\
6.73	0.00837856495180774\\
6.74	0.00837856495180774\\
6.75	0.00837856495180774\\
6.76	0.00837856495180774\\
6.77	0.00837856495180774\\
6.78	0.00837856495180774\\
6.79	0.00837856495180774\\
6.8	0.00837856495180774\\
6.81	0.00837856495180774\\
6.82	0.00837856495180774\\
6.83	0.00837856495180774\\
6.84	0.00837856495180774\\
6.85	0.00837856495180774\\
6.86	0.00837856495180774\\
6.87	0.00837856495180774\\
6.88	0.00837856495180774\\
6.89	0.00837856495180774\\
6.9	0.00837856495180774\\
6.91	0.00837856495180774\\
6.92	0.00837856495180774\\
6.93	0.00837856495180774\\
6.94	0.00837856495180774\\
6.95	0.00837856495180774\\
6.96	0.00837856495180774\\
6.97	0.00837856495180774\\
6.98	0.00837856495180774\\
6.99	0.00837856495180774\\
7	0.00837856495180774\\
7.01	0.00837856495180774\\
7.02	0.00837856495180774\\
7.03	0.00837856495180774\\
7.04	0.00837856495180774\\
7.05	0.00837856495180774\\
7.06	0.00837856495180774\\
7.07	0.00837856495180774\\
7.08	0.00837856495180774\\
7.09	0.00837856495180774\\
7.1	0.00837856495180774\\
7.11	0.00837856495180774\\
7.12	0.00837856495180774\\
7.13	0.00837856495180774\\
7.14	0.00837856495180774\\
7.15	0.00837856495180774\\
7.16	0.00837856495180774\\
7.17	0.00837856495180774\\
7.18	0.00837856495180774\\
7.19	0.00837856495180774\\
7.2	0.00837856495180774\\
7.21	0.00837856495180774\\
7.22	0.00837856495180774\\
7.23	0.00837856495180774\\
7.24	0.00837856495180774\\
7.25	0.00837856495180774\\
7.26	0.00837856495180774\\
7.27	0.00837856495180774\\
7.28	0.00837856495180774\\
7.29	0.00837856495180774\\
7.3	0.00837856495180774\\
7.31	0.00837856495180774\\
7.32	0.00837856495180774\\
7.33	0.00837856495180774\\
7.34	0.00837856495180774\\
7.35	0.00837856495180774\\
7.36	0.00837856495180774\\
7.37	0.00837856495180774\\
7.38	0.00837856495180774\\
7.39	0.00837856495180774\\
7.4	0.00837856495180774\\
7.41	0.00837856495180774\\
7.42	0.00837856495180774\\
7.43	0.00837856495180774\\
7.44	0.00837856495180774\\
7.45	0.00837856495180774\\
7.46	0.00837856495180774\\
7.47	0.00837856495180774\\
7.48	0.00837856495180774\\
7.49	0.00837856495180774\\
7.5	0.00837856495180774\\
7.51	0.00837856495180774\\
7.52	0.00837856495180774\\
7.53	0.00837856495180774\\
7.54	0.00837856495180774\\
7.55	0.00837856495180774\\
7.56	0.00837856495180774\\
7.57	0.00837856495180774\\
7.58	0.00837856495180774\\
7.59	0.00837856495180774\\
7.6	0.00837856495180774\\
7.61	0.00837856495180774\\
7.62	0.00837856495180774\\
7.63	0.00837856495180774\\
7.64	0.00837856495180774\\
7.65	0.00837856495180774\\
7.66	0.00837856495180774\\
7.67	0.00837856495180774\\
7.68	0.00837856495180774\\
7.69	0.00837856495180774\\
7.7	0.00837856495180774\\
7.71	0.00837856495180774\\
7.72	0.00837856495180774\\
7.73	0.00837856495180774\\
7.74	0.00837856495180774\\
7.75	0.00837856495180774\\
7.76	0.00837856495180774\\
7.77	0.00837856495180774\\
7.78	0.00837856495180774\\
7.79	0.00837856495180774\\
7.8	0.00837856495180774\\
7.81	0.00837856495180774\\
7.82	0.00837856495180774\\
7.83	0.00837856495180774\\
7.84	0.00837856495180774\\
7.85	0.00837856495180774\\
7.86	0.00837856495180774\\
7.87	0.00837856495180774\\
7.88	0.00837856495180774\\
7.89	0.00837856495180774\\
7.9	0.00837856495180774\\
7.91	0.00837856495180774\\
7.92	0.00837856495180774\\
7.93	0.00837856495180774\\
7.94	0.00837856495180774\\
7.95	0.00837856495180774\\
7.96	0.00837856495180774\\
7.97	0.00837856495180774\\
7.98	0.00837856495180774\\
7.99	0.00837856495180774\\
8	0.00837856495180774\\
8.01	0.00837856495180774\\
8.02	0.00837856495180774\\
8.03	0.00837856495180774\\
8.04	0.00837856495180774\\
8.05	0.00837856495180774\\
8.06	0.00837856495180774\\
8.07	0.00837856495180774\\
8.08	0.00837856495180774\\
8.09	0.00837856495180774\\
8.1	0.00837856495180774\\
8.11	0.00837856495180774\\
8.12	0.00837856495180774\\
8.13	0.00837856495180774\\
8.14	0.00837856495180774\\
8.15	0.00837856495180774\\
8.16	0.00837856495180774\\
8.17	0.00837856495180774\\
8.18	0.00837856495180774\\
8.19	0.00837856495180774\\
8.2	0.00837856495180774\\
8.21	0.00837856495180774\\
8.22	0.00837856495180774\\
8.23	0.00837856495180774\\
8.24	0.00837856495180774\\
8.25	0.00837856495180774\\
8.26	0.00837856495180774\\
8.27	0.00837856495180774\\
8.28	0.00837856495180774\\
8.29	0.00837856495180774\\
8.3	0.00837856495180774\\
8.31	0.00837856495180774\\
8.32	0.00837856495180774\\
8.33	0.00837856495180774\\
8.34	0.00837856495180774\\
8.35	0.00837856495180774\\
8.36	0.00837856495180774\\
8.37	0.00837856495180774\\
8.38	0.00837856495180774\\
8.39	0.00837856495180774\\
8.4	0.00837856495180774\\
8.41	0.00837856495180774\\
8.42	0.00837856495180774\\
8.43	0.00837856495180774\\
8.44	0.00837856495180774\\
8.45	0.00837856495180774\\
8.46	0.00837856495180774\\
8.47	0.00837856495180774\\
8.48	0.00837856495180774\\
8.49	0.00837856495180774\\
8.5	0.00837856495180774\\
8.51	0.00837856495180774\\
8.52	0.00837856495180774\\
8.53	0.00837856495180774\\
8.54	0.00837856495180774\\
8.55	0.00837856495180774\\
8.56	0.00837856495180774\\
8.57	0.00837856495180774\\
8.58	0.00837856495180774\\
8.59	0.00837856495180774\\
8.6	0.00837856495180774\\
8.61	0.00837856495180774\\
8.62	0.00837856495180774\\
8.63	0.00837856495180774\\
8.64	0.00837856495180774\\
8.65	0.00837856495180774\\
8.66	0.00837856495180774\\
8.67	0.00837856495180774\\
8.68	0.00837856495180774\\
8.69	0.00837856495180774\\
8.7	0.00837856495180774\\
8.71	0.00837856495180774\\
8.72	0.00837856495180774\\
8.73	0.00837856495180774\\
8.74	0.00837856495180774\\
8.75	0.00837856495180774\\
8.76	0.00837856495180774\\
8.77	0.00837856495180774\\
8.78	0.00837856495180774\\
8.79	0.00837856495180774\\
8.8	0.00837856495180774\\
8.81	0.00837856495180774\\
8.82	0.00837856495180774\\
8.83	0.00837856495180774\\
8.84	0.00837856495180774\\
8.85	0.00837856495180774\\
8.86	0.00837856495180774\\
8.87	0.00837856495180774\\
8.88	0.00837856495180774\\
8.89	0.00837856495180774\\
8.9	0.00837856495180774\\
8.91	0.00837856495180774\\
8.92	0.00837856495180774\\
8.93	0.00837856495180774\\
8.94	0.00837856495180774\\
8.95	0.00837856495180774\\
8.96	0.00837856495180774\\
8.97	0.00837856495180774\\
8.98	0.00837856495180774\\
8.99	0.00837856495180774\\
9	0.00837856495180774\\
9.01	0.00837856495180774\\
9.02	0.00837856495180774\\
9.03	0.00837856495180774\\
9.04	0.00837856495180774\\
9.05	0.00837856495180774\\
9.06	0.00837856495180774\\
9.07	0.00837856495180774\\
9.08	0.00837856495180774\\
9.09	0.00837856495180774\\
9.1	0.00837856495180774\\
9.11	0.00837856495180774\\
9.12	0.00837856495180774\\
9.13	0.00837856495180774\\
9.14	0.00837856495180774\\
9.15	0.00837856495180774\\
9.16	0.00837856495180774\\
9.17	0.00837856495180774\\
9.18	0.00837856495180774\\
9.19	0.00837856495180774\\
9.2	0.00837856495180774\\
9.21	0.00837856495180774\\
9.22	0.00837856495180774\\
9.23	0.00837856495180774\\
9.24	0.00837856495180774\\
9.25	0.00837856495180774\\
9.26	0.00837856495180774\\
9.27	0.00837856495180774\\
9.28	0.00837856495180774\\
9.29	0.00837856495180774\\
9.3	0.00837856495180774\\
9.31	0.00837856495180774\\
9.32	0.00837856495180774\\
9.33	0.00837856495180774\\
9.34	0.00837856495180774\\
9.35	0.00837856495180774\\
9.36	0.00837856495180774\\
9.37	0.00837856495180774\\
9.38	0.00837856495180774\\
9.39	0.00837856495180774\\
9.4	0.00837856495180774\\
9.41	0.00837856495180774\\
9.42	0.00837856495180774\\
9.43	0.00837856495180774\\
9.44	0.00837856495180774\\
9.45	0.00837856495180774\\
9.46	0.00837856495180774\\
9.47	0.00837856495180774\\
9.48	0.00837856495180774\\
9.49	0.00837856495180774\\
9.5	0.00837856495180774\\
9.51	0.00837856495180774\\
9.52	0.00837856495180774\\
9.53	0.00837856495180774\\
9.54	0.00837856495180774\\
9.55	0.00837856495180774\\
9.56	0.00837856495180774\\
9.57	0.00837856495180774\\
9.58	0.00837856495180774\\
9.59	0.00837856495180774\\
9.6	0.00837856495180774\\
9.61	0.00837856495180774\\
9.62	0.00837856495180774\\
9.63	0.00837856495180774\\
9.64	0.00837856495180774\\
9.65	0.00837856495180774\\
9.66	0.00837856495180774\\
9.67	0.00837856495180774\\
9.68	0.00837856495180774\\
9.69	0.00837856495180774\\
9.7	0.00837856495180774\\
9.71	0.00837856495180774\\
9.72	0.00837856495180774\\
9.73	0.00837856495180774\\
9.74	0.00837856495180774\\
9.75	0.00837856495180774\\
9.76	0.00837856495180774\\
9.77	0.00837856495180774\\
9.78	0.00837856495180774\\
9.79	0.00837856495180774\\
9.8	0.00837856495180774\\
9.81	0.00837856495180774\\
9.82	0.00837856495180774\\
9.83	0.00837856495180774\\
9.84	0.00837856495180774\\
9.85	0.00837856495180774\\
9.86	0.00837856495180774\\
9.87	0.00837856495180774\\
9.88	0.00837856495180774\\
9.89	0.00837856495180774\\
9.9	0.00837856495180774\\
9.91	0.00837856495180774\\
9.92	0.00837856495180774\\
9.93	0.00837856495180774\\
9.94	0.00837856495180774\\
9.95	0.00837856495180774\\
9.96	0.00837856495180774\\
9.97	0.00837856495180774\\
9.98	0.00837856495180774\\
9.99	0.00837856495180774\\
10	0.00837856495180774\\
10.01	0.00837856495180774\\
10.02	0.00837856495180774\\
10.03	0.00837856495180774\\
10.04	0.00837856495180774\\
10.05	0.00837856495180774\\
10.06	0.00837856495180774\\
10.07	0.00837856495180774\\
10.08	0.00837856495180774\\
10.09	0.00837856495180774\\
10.1	0.00837856495180774\\
10.11	0.00837856495180774\\
10.12	0.00837856495180774\\
10.13	0.00837856495180774\\
10.14	0.00837856495180774\\
10.15	0.00837856495180774\\
10.16	0.00837856495180774\\
10.17	0.00837856495180774\\
10.18	0.00837856495180774\\
10.19	0.00837856495180774\\
10.2	0.00837856495180774\\
10.21	0.00837856495180774\\
10.22	0.00837856495180774\\
10.23	0.00837856495180774\\
10.24	0.00837856495180774\\
10.25	0.00837856495180774\\
10.26	0.00837856495180774\\
10.27	0.00837856495180774\\
10.28	0.00837856495180774\\
10.29	0.00837856495180774\\
10.3	0.00837856495180774\\
10.31	0.00837856495180774\\
10.32	0.00837856495180774\\
10.33	0.00837856495180774\\
10.34	0.00837856495180774\\
10.35	0.00837856495180774\\
10.36	0.00837856495180774\\
10.37	0.00837856495180774\\
10.38	0.00837856495180774\\
10.39	0.00837856495180774\\
10.4	0.00837856495180774\\
10.41	0.00837856495180774\\
10.42	0.00837856495180774\\
10.43	0.00837856495180774\\
10.44	0.00837856495180774\\
10.45	0.00837856495180774\\
10.46	0.00837856495180774\\
10.47	0.00837856495180774\\
10.48	0.00837856495180774\\
10.49	0.00837856495180774\\
10.5	0.00837856495180774\\
10.51	0.00837856495180774\\
10.52	0.00837856495180774\\
10.53	0.00837856495180774\\
10.54	0.00837856495180774\\
10.55	0.00837856495180774\\
10.56	0.00837856495180774\\
10.57	0.00837856495180774\\
10.58	0.00837856495180774\\
10.59	0.00837856495180774\\
10.6	0.00837856495180774\\
10.61	0.00837856495180774\\
10.62	0.00837856495180774\\
10.63	0.00837856495180774\\
10.64	0.00837856495180774\\
10.65	0.00837856495180774\\
10.66	0.00837856495180774\\
10.67	0.00837856495180774\\
10.68	0.00837856495180774\\
10.69	0.00837856495180774\\
10.7	0.00837856495180774\\
10.71	0.00837856495180774\\
10.72	0.00837856495180774\\
10.73	0.00837856495180774\\
10.74	0.00837856495180774\\
10.75	0.00837856495180774\\
10.76	0.00837856495180774\\
10.77	0.00837856495180774\\
10.78	0.00837856495180774\\
10.79	0.00837856495180774\\
10.8	0.00837856495180774\\
10.81	0.00837856495180774\\
10.82	0.00837856495180774\\
10.83	0.00837856495180774\\
10.84	0.00837856495180774\\
10.85	0.00837856495180774\\
10.86	0.00837856495180774\\
10.87	0.00837856495180774\\
10.88	0.00837856495180774\\
10.89	0.00837856495180774\\
10.9	0.00837856495180774\\
10.91	0.00837856495180774\\
10.92	0.00837856495180774\\
10.93	0.00837856495180774\\
10.94	0.00837856495180774\\
10.95	0.00837856495180774\\
10.96	0.00837856495180774\\
10.97	0.00837856495180774\\
10.98	0.00837856495180774\\
10.99	0.00837856495180774\\
11	0.00837856495180774\\
11.01	0.00837856495180774\\
11.02	0.00837856495180774\\
11.03	0.00837856495180774\\
11.04	0.00837856495180774\\
11.05	0.00837856495180774\\
11.06	0.00837856495180774\\
11.07	0.00837856495180774\\
11.08	0.00837856495180774\\
11.09	0.00837856495180774\\
11.1	0.00837856495180774\\
11.11	0.00837856495180774\\
11.12	0.00837856495180774\\
11.13	0.00837856495180774\\
11.14	0.00837856495180774\\
11.15	0.00837856495180774\\
11.16	0.00837856495180774\\
11.17	0.00837856495180774\\
11.18	0.00837856495180774\\
11.19	0.00837856495180774\\
11.2	0.00837856495180774\\
11.21	0.00837856495180774\\
11.22	0.00837856495180774\\
11.23	0.00837856495180774\\
11.24	0.00837856495180774\\
11.25	0.00837856495180774\\
11.26	0.00837856495180774\\
11.27	0.00837856495180774\\
11.28	0.00837856495180774\\
11.29	0.00837856495180774\\
11.3	0.00837856495180774\\
11.31	0.00837856495180774\\
11.32	0.00837856495180774\\
11.33	0.00837856495180774\\
11.34	0.00837856495180774\\
11.35	0.00837856495180774\\
11.36	0.00837856495180774\\
11.37	0.00837856495180774\\
11.38	0.00837856495180774\\
11.39	0.00837856495180774\\
11.4	0.00837856495180774\\
11.41	0.00837856495180774\\
11.42	0.00837856495180774\\
11.43	0.00837856495180774\\
11.44	0.00837856495180774\\
11.45	0.00837856495180774\\
11.46	0.00837856495180774\\
11.47	0.00837856495180774\\
11.48	0.00837856495180774\\
11.49	0.00837856495180774\\
11.5	0.00837856495180774\\
11.51	0.00837856495180774\\
11.52	0.00837856495180774\\
11.53	0.00837856495180774\\
11.54	0.00837856495180774\\
11.55	0.00837856495180774\\
11.56	0.00837856495180774\\
11.57	0.00837856495180774\\
11.58	0.00837856495180774\\
11.59	0.00837856495180774\\
11.6	0.00837856495180774\\
11.61	0.00837856495180774\\
11.62	0.00837856495180774\\
11.63	0.00837856495180774\\
11.64	0.00837856495180774\\
11.65	0.00837856495180774\\
11.66	0.00837856495180774\\
11.67	0.00837856495180774\\
11.68	0.00837856495180774\\
11.69	0.00837856495180774\\
11.7	0.00837856495180774\\
11.71	0.00837856495180774\\
11.72	0.00837856495180774\\
11.73	0.00837856495180774\\
11.74	0.00837856495180774\\
11.75	0.00837856495180774\\
11.76	0.00837856495180774\\
11.77	0.00837856495180774\\
11.78	0.00837856495180774\\
11.79	0.00837856495180774\\
11.8	0.00837856495180774\\
11.81	0.00837856495180774\\
11.82	0.00837856495180774\\
11.83	0.00837856495180774\\
11.84	0.00837856495180774\\
11.85	0.00837856495180774\\
11.86	0.00837856495180774\\
11.87	0.00837856495180774\\
11.88	0.00837856495180774\\
11.89	0.00837856495180774\\
11.9	0.00837856495180774\\
11.91	0.00837856495180774\\
11.92	0.00837856495180774\\
11.93	0.00837856495180774\\
11.94	0.00837856495180774\\
11.95	0.00837856495180774\\
11.96	0.00837856495180774\\
11.97	0.00837856495180774\\
11.98	0.00837856495180774\\
11.99	0.00837856495180774\\
12	0.00837856495180774\\
12.01	0.00837856495180774\\
12.02	0.00837856495180774\\
12.03	0.00837856495180774\\
12.04	0.00837856495180774\\
12.05	0.00837856495180774\\
12.06	0.00837856495180774\\
12.07	0.00837856495180774\\
12.08	0.00837856495180774\\
12.09	0.00837856495180774\\
12.1	0.00837856495180774\\
12.11	0.00837856495180774\\
12.12	0.00837856495180774\\
12.13	0.00837856495180774\\
12.14	0.00837856495180774\\
12.15	0.00837856495180774\\
12.16	0.00837856495180774\\
12.17	0.00837856495180774\\
12.18	0.00837856495180774\\
12.19	0.00837856495180774\\
12.2	0.00837856495180774\\
12.21	0.00837856495180774\\
12.22	0.00837856495180774\\
12.23	0.00837856495180774\\
12.24	0.00837856495180774\\
12.25	0.00837856495180774\\
12.26	0.00837856495180774\\
12.27	0.00837856495180774\\
12.28	0.00837856495180774\\
12.29	0.00837856495180774\\
12.3	0.00837856495180774\\
12.31	0.00837856495180774\\
12.32	0.00837856495180774\\
12.33	0.00837856495180774\\
12.34	0.00837856495180774\\
12.35	0.00837856495180774\\
12.36	0.00837856495180774\\
12.37	0.00837856495180774\\
12.38	0.00837856495180774\\
12.39	0.00837856495180774\\
12.4	0.00837856495180774\\
12.41	0.00837856495180774\\
12.42	0.00837856495180774\\
12.43	0.00837856495180774\\
12.44	0.00837856495180774\\
12.45	0.00837856495180774\\
12.46	0.00837856495180774\\
12.47	0.00837856495180774\\
12.48	0.00837856495180774\\
12.49	0.00837856495180774\\
12.5	0.00837856495180774\\
12.51	0.00837856495180774\\
12.52	0.00837856495180774\\
12.53	0.00837856495180774\\
12.54	0.00837856495180774\\
12.55	0.00837856495180774\\
12.56	0.00837856495180774\\
12.57	0.00837856495180774\\
12.58	0.00837856495180774\\
12.59	0.00837856495180774\\
12.6	0.00837856495180774\\
12.61	0.00837856495180774\\
12.62	0.00837856495180774\\
12.63	0.00837856495180774\\
12.64	0.00837856495180774\\
12.65	0.00837856495180774\\
12.66	0.00837856495180774\\
12.67	0.00837856495180774\\
12.68	0.00837856495180774\\
12.69	0.00837856495180774\\
12.7	0.00837856495180774\\
12.71	0.00837856495180774\\
12.72	0.00837856495180774\\
12.73	0.00837856495180774\\
12.74	0.00837856495180774\\
12.75	0.00837856495180774\\
12.76	0.00837856495180774\\
12.77	0.00837856495180774\\
12.78	0.00837856495180774\\
12.79	0.00837856495180774\\
12.8	0.00837856495180774\\
12.81	0.00837856495180774\\
12.82	0.00837856495180774\\
12.83	0.00837856495180774\\
12.84	0.00837856495180774\\
12.85	0.00837856495180774\\
12.86	0.00837856495180774\\
12.87	0.00837856495180774\\
12.88	0.00837856495180774\\
12.89	0.00837856495180774\\
12.9	0.00837856495180774\\
12.91	0.00837856495180774\\
12.92	0.00837856495180774\\
12.93	0.00837856495180774\\
12.94	0.00837856495180774\\
12.95	0.00837856495180774\\
12.96	0.00837856495180774\\
12.97	0.00837856495180774\\
12.98	0.00837856495180774\\
12.99	0.00837856495180774\\
13	0.00837856495180774\\
13.01	0.00837856495180774\\
13.02	0.00837856495180774\\
13.03	0.00837856495180774\\
13.04	0.00837856495180774\\
13.05	0.00837856495180774\\
13.06	0.00837856495180774\\
13.07	0.00837856495180774\\
13.08	0.00837856495180774\\
13.09	0.00837856495180774\\
13.1	0.00837856495180774\\
13.11	0.00837856495180774\\
13.12	0.00837856495180774\\
13.13	0.00837856495180774\\
13.14	0.00837856495180774\\
13.15	0.00837856495180774\\
13.16	0.00837856495180774\\
13.17	0.00837856495180774\\
13.18	0.00837856495180774\\
13.19	0.00837856495180774\\
13.2	0.00837856495180774\\
13.21	0.00837856495180774\\
13.22	0.00837856495180774\\
13.23	0.00837856495180774\\
13.24	0.00837856495180774\\
13.25	0.00837856495180774\\
13.26	0.00837856495180774\\
13.27	0.00837856495180774\\
13.28	0.00837856495180774\\
13.29	0.00837856495180774\\
13.3	0.00837856495180774\\
13.31	0.00837856495180774\\
13.32	0.00837856495180774\\
13.33	0.00837856495180774\\
13.34	0.00837856495180774\\
13.35	0.00837856495180774\\
13.36	0.00837856495180774\\
13.37	0.00837856495180774\\
13.38	0.00837856495180774\\
13.39	0.00837856495180774\\
13.4	0.00837856495180774\\
13.41	0.00837856495180774\\
13.42	0.00837856495180774\\
13.43	0.00837856495180774\\
13.44	0.00837856495180774\\
13.45	0.00837856495180774\\
13.46	0.00837856495180774\\
13.47	0.00837856495180774\\
13.48	0.00837856495180774\\
13.49	0.00837856495180774\\
13.5	0.00837856495180774\\
13.51	0.00837856495180774\\
13.52	0.00837856495180774\\
13.53	0.00837856495180774\\
13.54	0.00837856495180774\\
13.55	0.00837856495180774\\
13.56	0.00837856495180774\\
13.57	0.00837856495180774\\
13.58	0.00837856495180774\\
13.59	0.00837856495180774\\
13.6	0.00837856495180774\\
13.61	0.00837856495180774\\
13.62	0.00837856495180774\\
13.63	0.00837856495180774\\
13.64	0.00837856495180774\\
13.65	0.00837856495180774\\
13.66	0.00837856495180774\\
13.67	0.00837856495180774\\
13.68	0.00837856495180774\\
13.69	0.00837856495180774\\
13.7	0.00837856495180774\\
13.71	0.00837856495180774\\
13.72	0.00837856495180774\\
13.73	0.00837856495180774\\
13.74	0.00837856495180774\\
13.75	0.00837856495180774\\
13.76	0.00837856495180774\\
13.77	0.00837856495180774\\
13.78	0.00837856495180774\\
13.79	0.00837856495180774\\
13.8	0.00837856495180774\\
13.81	0.00837856495180774\\
13.82	0.00837856495180774\\
13.83	0.00837856495180774\\
13.84	0.00837856495180774\\
13.85	0.00837856495180774\\
13.86	0.00837856495180774\\
13.87	0.00837856495180774\\
13.88	0.00837856495180774\\
13.89	0.00837856495180774\\
13.9	0.00837856495180774\\
13.91	0.00837856495180774\\
13.92	0.00837856495180774\\
13.93	0.00837856495180774\\
13.94	0.00837856495180774\\
13.95	0.00837856495180774\\
13.96	0.00837856495180774\\
13.97	0.00837856495180774\\
13.98	0.00837856495180774\\
13.99	0.00837856495180774\\
14	0.00837856495180774\\
14.01	0.00837856495180774\\
14.02	0.00837856495180774\\
14.03	0.00837856495180774\\
14.04	0.00837856495180774\\
14.05	0.00837856495180774\\
14.06	0.00837856495180774\\
14.07	0.00837856495180774\\
14.08	0.00837856495180774\\
14.09	0.00837856495180774\\
14.1	0.00837856495180774\\
14.11	0.00837856495180774\\
14.12	0.00837856495180774\\
14.13	0.00837856495180774\\
14.14	0.00837856495180774\\
14.15	0.00837856495180774\\
14.16	0.00837856495180774\\
14.17	0.00837856495180774\\
14.18	0.00837856495180774\\
14.19	0.00837856495180774\\
14.2	0.00837856495180774\\
14.21	0.00837856495180774\\
14.22	0.00837856495180774\\
14.23	0.00837856495180774\\
14.24	0.00837856495180774\\
14.25	0.00837856495180774\\
14.26	0.00837856495180774\\
14.27	0.00837856495180774\\
14.28	0.00837856495180774\\
14.29	0.00837856495180774\\
14.3	0.00837856495180774\\
14.31	0.00837856495180774\\
14.32	0.00837856495180774\\
14.33	0.00837856495180774\\
14.34	0.00837856495180774\\
14.35	0.00837856495180774\\
14.36	0.00837856495180774\\
14.37	0.00837856495180774\\
14.38	0.00837856495180774\\
14.39	0.00837856495180774\\
14.4	0.00837856495180774\\
14.41	0.00837856495180774\\
14.42	0.00837856495180774\\
14.43	0.00837856495180774\\
14.44	0.00837856495180774\\
14.45	0.00837856495180774\\
14.46	0.00837856495180774\\
14.47	0.00837856495180774\\
14.48	0.00837856495180774\\
14.49	0.00837856495180774\\
14.5	0.00837856495180774\\
14.51	0.00837856495180774\\
14.52	0.00837856495180774\\
14.53	0.00837856495180774\\
14.54	0.00837856495180774\\
14.55	0.00837856495180774\\
14.56	0.00837856495180774\\
14.57	0.00837856495180774\\
14.58	0.00837856495180774\\
14.59	0.00837856495180774\\
14.6	0.00837856495180774\\
14.61	0.00837856495180774\\
14.62	0.00837856495180774\\
14.63	0.00837856495180774\\
14.64	0.00837856495180774\\
14.65	0.00837856495180774\\
14.66	0.00837856495180774\\
14.67	0.00837856495180774\\
14.68	0.00837856495180774\\
14.69	0.00837856495180774\\
14.7	0.00837856495180774\\
14.71	0.00837856495180774\\
14.72	0.00837856495180774\\
14.73	0.00837856495180774\\
14.74	0.00837856495180774\\
14.75	0.00837856495180774\\
14.76	0.00837856495180774\\
14.77	0.00837856495180774\\
14.78	0.00837856495180774\\
14.79	0.00837856495180774\\
14.8	0.00837856495180774\\
14.81	0.00837856495180774\\
14.82	0.00837856495180774\\
14.83	0.00837856495180774\\
14.84	0.00837856495180774\\
14.85	0.00837856495180774\\
14.86	0.00837856495180774\\
14.87	0.00837856495180774\\
14.88	0.00837856495180774\\
14.89	0.00837856495180774\\
14.9	0.00837856495180774\\
14.91	0.00837856495180774\\
14.92	0.00837856495180774\\
14.93	0.00837856495180774\\
14.94	0.00837856495180774\\
14.95	0.00837856495180774\\
14.96	0.00837856495180774\\
14.97	0.00837856495180774\\
14.98	0.00837856495180774\\
14.99	0.00837856495180774\\
15	0.00837856495180774\\
15.01	0.00837856495180774\\
15.02	0.00837856495180774\\
15.03	0.00837856495180774\\
15.04	0.00837856495180774\\
15.05	0.00837856495180774\\
15.06	0.00837856495180774\\
15.07	0.00837856495180774\\
15.08	0.00837856495180774\\
15.09	0.00837856495180774\\
15.1	0.00837856495180774\\
15.11	0.00837856495180774\\
15.12	0.00837856495180774\\
15.13	0.00837856495180774\\
15.14	0.00837856495180774\\
15.15	0.00837856495180774\\
15.16	0.00837856495180774\\
15.17	0.00837856495180774\\
15.18	0.00837856495180774\\
15.19	0.00837856495180774\\
15.2	0.00837856495180774\\
15.21	0.00837856495180774\\
15.22	0.00837856495180774\\
15.23	0.00837856495180774\\
15.24	0.00837856495180774\\
15.25	0.00837856495180774\\
15.26	0.00837856495180774\\
15.27	0.00837856495180774\\
15.28	0.00837856495180774\\
15.29	0.00837856495180774\\
15.3	0.00837856495180774\\
15.31	0.00837856495180774\\
15.32	0.00837856495180774\\
15.33	0.00837856495180774\\
15.34	0.00837856495180774\\
15.35	0.00837856495180774\\
15.36	0.00837856495180774\\
15.37	0.00837856495180774\\
15.38	0.00837856495180774\\
15.39	0.00837856495180774\\
15.4	0.00837856495180774\\
15.41	0.00837856495180774\\
15.42	0.00837856495180774\\
15.43	0.00837856495180774\\
15.44	0.00837856495180774\\
15.45	0.00837856495180774\\
15.46	0.00837856495180774\\
15.47	0.00837856495180774\\
15.48	0.00837856495180774\\
15.49	0.00837856495180774\\
15.5	0.00837856495180774\\
15.51	0.00837856495180774\\
15.52	0.00837856495180774\\
15.53	0.00837856495180774\\
15.54	0.00837856495180774\\
15.55	0.00837856495180774\\
15.56	0.00837856495180774\\
15.57	0.00837856495180774\\
15.58	0.00837856495180774\\
15.59	0.00837856495180774\\
15.6	0.00837856495180774\\
15.61	0.00837856495180774\\
15.62	0.00837856495180774\\
15.63	0.00837856495180774\\
15.64	0.00837856495180774\\
15.65	0.00837856495180774\\
15.66	0.00837856495180774\\
15.67	0.00837856495180774\\
15.68	0.00837856495180774\\
15.69	0.00837856495180774\\
15.7	0.00837856495180774\\
15.71	0.00837856495180774\\
15.72	0.00837856495180774\\
15.73	0.00837856495180774\\
15.74	0.00837856495180774\\
15.75	0.00837856495180774\\
15.76	0.00837856495180774\\
15.77	0.00837856495180774\\
15.78	0.00837856495180774\\
15.79	0.00837856495180774\\
15.8	0.00837856495180774\\
15.81	0.00837856495180774\\
15.82	0.00837856495180774\\
15.83	0.00837856495180774\\
15.84	0.00837856495180774\\
15.85	0.00837856495180774\\
15.86	0.00837856495180774\\
15.87	0.00837856495180774\\
15.88	0.00837856495180774\\
15.89	0.00837856495180774\\
15.9	0.00837856495180774\\
15.91	0.00837856495180774\\
15.92	0.00837856495180774\\
15.93	0.00837856495180774\\
15.94	0.00837856495180774\\
15.95	0.00837856495180774\\
15.96	0.00837856495180774\\
15.97	0.00837856495180774\\
15.98	0.00837856495180774\\
15.99	0.00837856495180774\\
16	0.00837856495180774\\
16.01	0.00837856495180774\\
16.02	0.00837856495180774\\
16.03	0.00837856495180774\\
16.04	0.00837856495180774\\
16.05	0.00837856495180774\\
16.06	0.00837856495180774\\
16.07	0.00837856495180774\\
16.08	0.00837856495180774\\
16.09	0.00837856495180774\\
16.1	0.00837856495180774\\
16.11	0.00837856495180774\\
16.12	0.00837856495180774\\
16.13	0.00837856495180774\\
16.14	0.00837856495180774\\
16.15	0.00837856495180774\\
16.16	0.00837856495180774\\
16.17	0.00837856495180774\\
16.18	0.00837856495180774\\
16.19	0.00837856495180774\\
16.2	0.00837856495180774\\
16.21	0.00837856495180774\\
16.22	0.00837856495180774\\
16.23	0.00837856495180774\\
16.24	0.00837856495180774\\
16.25	0.00837856495180774\\
16.26	0.00837856495180774\\
16.27	0.00837856495180774\\
16.28	0.00837856495180774\\
16.29	0.00837856495180774\\
16.3	0.00837856495180774\\
16.31	0.00837856495180774\\
16.32	0.00837856495180774\\
16.33	0.00837856495180774\\
16.34	0.00837856495180774\\
16.35	0.00837856495180774\\
16.36	0.00837856495180774\\
16.37	0.00837856495180774\\
16.38	0.00837856495180774\\
16.39	0.00837856495180774\\
16.4	0.00837856495180774\\
16.41	0.00837856495180774\\
16.42	0.00837856495180774\\
16.43	0.00837856495180774\\
16.44	0.00837856495180774\\
16.45	0.00837856495180774\\
16.46	0.00837856495180774\\
16.47	0.00837856495180774\\
16.48	0.00837856495180774\\
16.49	0.00837856495180774\\
16.5	0.00837856495180774\\
16.51	0.00837856495180774\\
16.52	0.00837856495180774\\
16.53	0.00837856495180774\\
16.54	0.00837856495180774\\
16.55	0.00837856495180774\\
16.56	0.00837856495180774\\
16.57	0.00837856495180774\\
16.58	0.00837856495180774\\
16.59	0.00837856495180774\\
16.6	0.00837856495180774\\
16.61	0.00837856495180774\\
16.62	0.00837856495180774\\
16.63	0.00837856495180774\\
16.64	0.00837856495180774\\
16.65	0.00837856495180774\\
16.66	0.00837856495180774\\
16.67	0.00837856495180774\\
16.68	0.00837856495180774\\
16.69	0.00837856495180774\\
16.7	0.00837856495180774\\
16.71	0.00837856495180774\\
16.72	0.00837856495180774\\
16.73	0.00837856495180774\\
16.74	0.00837856495180774\\
16.75	0.00837856495180774\\
16.76	0.00837856495180774\\
16.77	0.00837856495180774\\
16.78	0.00837856495180774\\
16.79	0.00837856495180774\\
16.8	0.00837856495180774\\
16.81	0.00837856495180774\\
16.82	0.00837856495180774\\
16.83	0.00837856495180774\\
16.84	0.00837856495180774\\
16.85	0.00837856495180774\\
16.86	0.00837856495180774\\
16.87	0.00837856495180774\\
16.88	0.00837856495180774\\
16.89	0.00837856495180774\\
16.9	0.00837856495180774\\
16.91	0.00837856495180774\\
16.92	0.00837856495180774\\
16.93	0.00837856495180774\\
16.94	0.00837856495180774\\
16.95	0.00837856495180774\\
16.96	0.00837856495180774\\
16.97	0.00837856495180774\\
16.98	0.00837856495180774\\
16.99	0.00837856495180774\\
17	0.00837856495180774\\
17.01	0.00837856495180774\\
17.02	0.00837856495180774\\
17.03	0.00837856495180774\\
17.04	0.00837856495180774\\
17.05	0.00837856495180774\\
17.06	0.00837856495180774\\
17.07	0.00837856495180774\\
17.08	0.00837856495180774\\
17.09	0.00837856495180774\\
17.1	0.00837856495180774\\
17.11	0.00837856495180774\\
17.12	0.00837856495180774\\
17.13	0.00837856495180774\\
17.14	0.00837856495180774\\
17.15	0.00837856495180774\\
17.16	0.00837856495180774\\
17.17	0.00837856495180774\\
17.18	0.00837856495180774\\
17.19	0.00837856495180774\\
17.2	0.00837856495180774\\
17.21	0.00837856495180774\\
17.22	0.00837856495180774\\
17.23	0.00837856495180774\\
17.24	0.00837856495180774\\
17.25	0.00837856495180774\\
17.26	0.00837856495180774\\
17.27	0.00837856495180774\\
17.28	0.00837856495180774\\
17.29	0.00837856495180774\\
17.3	0.00837856495180774\\
17.31	0.00837856495180774\\
17.32	0.00837856495180774\\
17.33	0.00837856495180774\\
17.34	0.00837856495180774\\
17.35	0.00837856495180774\\
17.36	0.00837856495180774\\
17.37	0.00837856495180774\\
17.38	0.00837856495180774\\
17.39	0.00837856495180774\\
17.4	0.00837856495180774\\
17.41	0.00837856495180774\\
17.42	0.00837856495180774\\
17.43	0.00837856495180774\\
17.44	0.00837856495180774\\
17.45	0.00837856495180774\\
17.46	0.00837856495180774\\
17.47	0.00837856495180774\\
17.48	0.00837856495180774\\
17.49	0.00837856495180774\\
17.5	0.00837856495180774\\
17.51	0.00837856495180774\\
17.52	0.00837856495180774\\
17.53	0.00837856495180774\\
17.54	0.00837856495180774\\
17.55	0.00837856495180774\\
17.56	0.00837856495180774\\
17.57	0.00837856495180774\\
17.58	0.00837856495180774\\
17.59	0.00837856495180774\\
17.6	0.00837856495180774\\
17.61	0.00837856495180774\\
17.62	0.00837856495180774\\
17.63	0.00837856495180774\\
17.64	0.00837856495180774\\
17.65	0.00837856495180774\\
17.66	0.00837856495180774\\
17.67	0.00837856495180774\\
17.68	0.00837856495180774\\
17.69	0.00837856495180774\\
17.7	0.00837856495180774\\
17.71	0.00837856495180774\\
17.72	0.00837856495180774\\
17.73	0.00837856495180774\\
17.74	0.00837856495180774\\
17.75	0.00837856495180774\\
17.76	0.00837856495180774\\
17.77	0.00837856495180774\\
17.78	0.00837856495180774\\
17.79	0.00837856495180774\\
17.8	0.00837856495180774\\
17.81	0.00837856495180774\\
17.82	0.00837856495180774\\
17.83	0.00837856495180774\\
17.84	0.00837856495180774\\
17.85	0.00837856495180774\\
17.86	0.00837856495180774\\
17.87	0.00837856495180774\\
17.88	0.00837856495180774\\
17.89	0.00837856495180774\\
17.9	0.00837856495180774\\
17.91	0.00837856495180774\\
17.92	0.00837856495180774\\
17.93	0.00837856495180774\\
17.94	0.00837856495180774\\
17.95	0.00837856495180774\\
17.96	0.00837856495180774\\
17.97	0.00837856495180774\\
17.98	0.00837856495180774\\
17.99	0.00837856495180774\\
18	0.00837856495180774\\
18.01	0.00837856495180774\\
18.02	0.00837856495180774\\
18.03	0.00837856495180774\\
18.04	0.00837856495180774\\
18.05	0.00837856495180774\\
18.06	0.00837856495180774\\
18.07	0.00837856495180774\\
18.08	0.00837856495180774\\
18.09	0.00837856495180774\\
18.1	0.00837856495180774\\
18.11	0.00837856495180774\\
18.12	0.00837856495180774\\
18.13	0.00837856495180774\\
18.14	0.00837856495180774\\
18.15	0.00837856495180774\\
18.16	0.00837856495180774\\
18.17	0.00837856495180774\\
18.18	0.00837856495180774\\
18.19	0.00837856495180774\\
18.2	0.00837856495180774\\
18.21	0.00837856495180774\\
18.22	0.00837856495180774\\
18.23	0.00837856495180774\\
18.24	0.00837856495180774\\
18.25	0.00837856495180774\\
18.26	0.00837856495180774\\
18.27	0.00837856495180774\\
18.28	0.00837856495180774\\
18.29	0.00837856495180774\\
18.3	0.00837856495180774\\
18.31	0.00837856495180774\\
18.32	0.00837856495180774\\
18.33	0.00837856495180774\\
18.34	0.00837856495180774\\
18.35	0.00837856495180774\\
18.36	0.00837856495180774\\
18.37	0.00837856495180774\\
18.38	0.00837856495180774\\
18.39	0.00837856495180774\\
18.4	0.00837856495180774\\
18.41	0.00837856495180774\\
18.42	0.00837856495180774\\
18.43	0.00837856495180774\\
18.44	0.00837856495180774\\
18.45	0.00837856495180774\\
18.46	0.00837856495180774\\
18.47	0.00837856495180774\\
18.48	0.00837856495180774\\
18.49	0.00837856495180774\\
18.5	0.00837856495180774\\
18.51	0.00837856495180774\\
18.52	0.00837856495180774\\
18.53	0.00837856495180774\\
18.54	0.00837856495180774\\
18.55	0.00837856495180774\\
18.56	0.00837856495180774\\
18.57	0.00837856495180774\\
18.58	0.00837856495180774\\
18.59	0.00837856495180774\\
18.6	0.00837856495180774\\
18.61	0.00837856495180774\\
18.62	0.00837856495180774\\
18.63	0.00837856495180774\\
18.64	0.00837856495180774\\
18.65	0.00837856495180774\\
18.66	0.00837856495180774\\
18.67	0.00837856495180774\\
18.68	0.00837856495180774\\
18.69	0.00837856495180774\\
18.7	0.00837856495180774\\
18.71	0.00837856495180774\\
18.72	0.00837856495180774\\
18.73	0.00837856495180774\\
18.74	0.00837856495180774\\
18.75	0.00837856495180774\\
18.76	0.00837856495180774\\
18.77	0.00837856495180774\\
18.78	0.00837856495180774\\
18.79	0.00837856495180774\\
18.8	0.00837856495180774\\
18.81	0.00837856495180774\\
18.82	0.00837856495180774\\
18.83	0.00837856495180774\\
18.84	0.00837856495180774\\
18.85	0.00837856495180774\\
18.86	0.00837856495180774\\
18.87	0.00837856495180774\\
18.88	0.00837856495180774\\
18.89	0.00837856495180774\\
18.9	0.00837856495180774\\
18.91	0.00837856495180774\\
18.92	0.00837856495180774\\
18.93	0.00837856495180774\\
18.94	0.00837856495180774\\
18.95	0.00837856495180774\\
18.96	0.00837856495180774\\
18.97	0.00837856495180774\\
18.98	0.00837856495180774\\
18.99	0.00837856495180774\\
19	0.00837856495180774\\
19.01	0.00837856495180774\\
19.02	0.00837856495180774\\
19.03	0.00837856495180774\\
19.04	0.00837856495180774\\
19.05	0.00837856495180774\\
19.06	0.00837856495180774\\
19.07	0.00837856495180774\\
19.08	0.00837856495180774\\
19.09	0.00837856495180774\\
19.1	0.00837856495180774\\
19.11	0.00837856495180774\\
19.12	0.00837856495180774\\
19.13	0.00837856495180774\\
19.14	0.00837856495180774\\
19.15	0.00837856495180774\\
19.16	0.00837856495180774\\
19.17	0.00837856495180774\\
19.18	0.00837856495180774\\
19.19	0.00837856495180774\\
19.2	0.00837856495180774\\
19.21	0.00837856495180774\\
19.22	0.00837856495180774\\
19.23	0.00837856495180774\\
19.24	0.00837856495180774\\
19.25	0.00837856495180774\\
19.26	0.00837856495180774\\
19.27	0.00837856495180774\\
19.28	0.00837856495180774\\
19.29	0.00837856495180774\\
19.3	0.00837856495180774\\
19.31	0.00837856495180774\\
19.32	0.00837856495180774\\
19.33	0.00837856495180774\\
19.34	0.00837856495180774\\
19.35	0.00837856495180774\\
19.36	0.00837856495180774\\
19.37	0.00837856495180774\\
19.38	0.00837856495180774\\
19.39	0.00837856495180774\\
19.4	0.00837856495180774\\
19.41	0.00837856495180774\\
19.42	0.00837856495180774\\
19.43	0.00837856495180774\\
19.44	0.00837856495180774\\
19.45	0.00837856495180774\\
19.46	0.00837856495180774\\
19.47	0.00837856495180774\\
19.48	0.00837856495180774\\
19.49	0.00837856495180774\\
19.5	0.00837856495180774\\
19.51	0.00837856495180774\\
19.52	0.00837856495180774\\
19.53	0.00837856495180774\\
19.54	0.00837856495180774\\
19.55	0.00837856495180774\\
19.56	0.00837856495180774\\
19.57	0.00837856495180774\\
19.58	0.00837856495180774\\
19.59	0.00837856495180774\\
19.6	0.00837856495180774\\
19.61	0.00837856495180774\\
19.62	0.00837856495180774\\
19.63	0.00837856495180774\\
19.64	0.00837856495180774\\
19.65	0.00837856495180774\\
19.66	0.00837856495180774\\
19.67	0.00837856495180774\\
19.68	0.00837856495180774\\
19.69	0.00837856495180774\\
19.7	0.00837856495180774\\
19.71	0.00837856495180774\\
19.72	0.00837856495180774\\
19.73	0.00837856495180774\\
19.74	0.00837856495180774\\
19.75	0.00837856495180774\\
19.76	0.00837856495180774\\
19.77	0.00837856495180774\\
19.78	0.00837856495180774\\
19.79	0.00837856495180774\\
19.8	0.00837856495180774\\
19.81	0.00837856495180774\\
19.82	0.00837856495180774\\
19.83	0.00837856495180774\\
19.84	0.00837856495180774\\
19.85	0.00837856495180774\\
19.86	0.00837856495180774\\
19.87	0.00837856495180774\\
19.88	0.00837856495180774\\
19.89	0.00837856495180774\\
19.9	0.00837856495180774\\
19.91	0.00837856495180774\\
19.92	0.00837856495180774\\
19.93	0.00837856495180774\\
19.94	0.00837856495180774\\
19.95	0.00837856495180774\\
19.96	0.00837856495180774\\
19.97	0.00837856495180774\\
19.98	0.00837856495180774\\
19.99	0.00837856495180774\\
20	0.00837856495180774\\
20.01	0.00837856495180774\\
20.02	0.00837856495180774\\
20.03	0.00837856495180774\\
20.04	0.00837856495180774\\
20.05	0.00837856495180774\\
20.06	0.00837856495180774\\
20.07	0.00837856495180774\\
20.08	0.00837856495180774\\
20.09	0.00837856495180774\\
20.1	0.00837856495180774\\
20.11	0.00837856495180774\\
20.12	0.00837856495180774\\
20.13	0.00837856495180774\\
20.14	0.00837856495180774\\
20.15	0.00837856495180774\\
20.16	0.00837856495180774\\
20.17	0.00837856495180774\\
20.18	0.00837856495180774\\
20.19	0.00837856495180774\\
20.2	0.00837856495180774\\
20.21	0.00837856495180774\\
20.22	0.00837856495180774\\
20.23	0.00837856495180774\\
20.24	0.00837856495180774\\
20.25	0.00837856495180774\\
20.26	0.00837856495180774\\
20.27	0.00837856495180774\\
20.28	0.00837856495180774\\
20.29	0.00837856495180774\\
20.3	0.00837856495180774\\
20.31	0.00837856495180774\\
20.32	0.00837856495180774\\
20.33	0.00837856495180774\\
20.34	0.00837856495180774\\
20.35	0.00837856495180774\\
20.36	0.00837856495180774\\
20.37	0.00837856495180774\\
20.38	0.00837856495180774\\
20.39	0.00837856495180774\\
20.4	0.00837856495180774\\
20.41	0.00837856495180774\\
20.42	0.00837856495180774\\
20.43	0.00837856495180774\\
20.44	0.00837856495180774\\
20.45	0.00837856495180774\\
20.46	0.00837856495180774\\
20.47	0.00837856495180774\\
20.48	0.00837856495180774\\
20.49	0.00837856495180774\\
20.5	0.00837856495180774\\
20.51	0.00837856495180774\\
20.52	0.00837856495180774\\
20.53	0.00837856495180774\\
20.54	0.00837856495180774\\
20.55	0.00837856495180774\\
20.56	0.00837856495180774\\
20.57	0.00837856495180774\\
20.58	0.00837856495180774\\
20.59	0.00837856495180774\\
20.6	0.00837856495180774\\
20.61	0.00837856495180774\\
20.62	0.00837856495180774\\
20.63	0.00837856495180774\\
20.64	0.00837856495180774\\
20.65	0.00837856495180774\\
20.66	0.00837856495180774\\
20.67	0.00837856495180774\\
20.68	0.00837856495180774\\
20.69	0.00837856495180774\\
20.7	0.00837856495180774\\
20.71	0.00837856495180774\\
20.72	0.00837856495180774\\
20.73	0.00837856495180774\\
20.74	0.00837856495180774\\
20.75	0.00837856495180774\\
20.76	0.00837856495180774\\
20.77	0.00837856495180774\\
20.78	0.00837856495180774\\
20.79	0.00837856495180774\\
20.8	0.00837856495180774\\
20.81	0.00837856495180774\\
20.82	0.00837856495180774\\
20.83	0.00837856495180774\\
20.84	0.00837856495180774\\
20.85	0.00837856495180774\\
20.86	0.00837856495180774\\
20.87	0.00837856495180774\\
20.88	0.00837856495180774\\
20.89	0.00837856495180774\\
20.9	0.00837856495180774\\
20.91	0.00837856495180774\\
20.92	0.00837856495180774\\
20.93	0.00837856495180774\\
20.94	0.00837856495180774\\
20.95	0.00837856495180774\\
20.96	0.00837856495180774\\
20.97	0.00837856495180774\\
20.98	0.00837856495180774\\
20.99	0.00837856495180774\\
21	0.00837856495180774\\
21.01	0.00837856495180774\\
21.02	0.00837856495180774\\
21.03	0.00837856495180774\\
21.04	0.00837856495180774\\
21.05	0.00837856495180774\\
21.06	0.00837856495180774\\
21.07	0.00837856495180774\\
21.08	0.00837856495180774\\
21.09	0.00837856495180774\\
21.1	0.00837856495180774\\
21.11	0.00837856495180774\\
21.12	0.00837856495180774\\
21.13	0.00837856495180774\\
21.14	0.00837856495180774\\
21.15	0.00837856495180774\\
21.16	0.00837856495180774\\
21.17	0.00837856495180774\\
21.18	0.00837856495180774\\
21.19	0.00837856495180774\\
21.2	0.00837856495180774\\
21.21	0.00837856495180774\\
21.22	0.00837856495180774\\
21.23	0.00837856495180774\\
21.24	0.00837856495180774\\
21.25	0.00837856495180774\\
21.26	0.00837856495180774\\
21.27	0.00837856495180774\\
21.28	0.00837856495180774\\
21.29	0.00837856495180774\\
21.3	0.00837856495180774\\
21.31	0.00837856495180774\\
21.32	0.00837856495180774\\
21.33	0.00837856495180774\\
21.34	0.00837856495180774\\
21.35	0.00837856495180774\\
21.36	0.00837856495180774\\
21.37	0.00837856495180774\\
21.38	0.00837856495180774\\
21.39	0.00837856495180774\\
21.4	0.00837856495180774\\
21.41	0.00837856495180774\\
21.42	0.00837856495180774\\
21.43	0.00837856495180774\\
21.44	0.00837856495180774\\
21.45	0.00837856495180774\\
21.46	0.00837856495180774\\
21.47	0.00837856495180774\\
21.48	0.00837856495180774\\
21.49	0.00837856495180774\\
21.5	0.00837856495180774\\
21.51	0.00837856495180774\\
21.52	0.00837856495180774\\
21.53	0.00837856495180774\\
21.54	0.00837856495180774\\
21.55	0.00837856495180774\\
21.56	0.00837856495180774\\
21.57	0.00837856495180774\\
21.58	0.00837856495180774\\
21.59	0.00837856495180774\\
21.6	0.00837856495180774\\
21.61	0.00837856495180774\\
21.62	0.00837856495180774\\
21.63	0.00837856495180774\\
21.64	0.00837856495180774\\
21.65	0.00837856495180774\\
21.66	0.00837856495180774\\
21.67	0.00837856495180774\\
21.68	0.00837856495180774\\
21.69	0.00837856495180774\\
21.7	0.00837856495180774\\
21.71	0.00837856495180774\\
21.72	0.00837856495180774\\
21.73	0.00837856495180774\\
21.74	0.00837856495180774\\
21.75	0.00837856495180774\\
21.76	0.00837856495180774\\
21.77	0.00837856495180774\\
21.78	0.00837856495180774\\
21.79	0.00837856495180774\\
21.8	0.00837856495180774\\
21.81	0.00837856495180774\\
21.82	0.00837856495180774\\
21.83	0.00837856495180774\\
21.84	0.00837856495180774\\
21.85	0.00837856495180774\\
21.86	0.00837856495180774\\
21.87	0.00837856495180774\\
21.88	0.00837856495180774\\
21.89	0.00837856495180774\\
21.9	0.00837856495180774\\
21.91	0.00837856495180774\\
21.92	0.00837856495180774\\
21.93	0.00837856495180774\\
21.94	0.00837856495180774\\
21.95	0.00837856495180774\\
21.96	0.00837856495180774\\
21.97	0.00837856495180774\\
21.98	0.00837856495180774\\
21.99	0.00837856495180774\\
22	0.00837856495180774\\
22.01	0.00837856495180774\\
22.02	0.00837856495180774\\
22.03	0.00837856495180774\\
22.04	0.00837856495180774\\
22.05	0.00837856495180774\\
22.06	0.00837856495180774\\
22.07	0.00837856495180774\\
22.08	0.00837856495180774\\
22.09	0.00837856495180774\\
22.1	0.00837856495180774\\
22.11	0.00837856495180774\\
22.12	0.00837856495180774\\
22.13	0.00837856495180774\\
22.14	0.00837856495180774\\
22.15	0.00837856495180774\\
22.16	0.00837856495180774\\
22.17	0.00837856495180774\\
22.18	0.00837856495180774\\
22.19	0.00837856495180774\\
22.2	0.00837856495180774\\
22.21	0.00837856495180774\\
22.22	0.00837856495180774\\
22.23	0.00837856495180774\\
22.24	0.00837856495180774\\
22.25	0.00837856495180774\\
22.26	0.00837856495180774\\
22.27	0.00837856495180774\\
22.28	0.00837856495180774\\
22.29	0.00837856495180774\\
22.3	0.00837856495180774\\
22.31	0.00837856495180774\\
22.32	0.00837856495180774\\
22.33	0.00837856495180774\\
22.34	0.00837856495180774\\
22.35	0.00837856495180774\\
22.36	0.00837856495180774\\
22.37	0.00837856495180774\\
22.38	0.00837856495180774\\
22.39	0.00837856495180774\\
22.4	0.00837856495180774\\
22.41	0.00837856495180774\\
22.42	0.00837856495180774\\
22.43	0.00837856495180774\\
22.44	0.00837856495180774\\
22.45	0.00837856495180774\\
22.46	0.00837856495180774\\
22.47	0.00837856495180774\\
22.48	0.00837856495180774\\
22.49	0.00837856495180774\\
22.5	0.00837856495180774\\
22.51	0.00837856495180774\\
22.52	0.00837856495180774\\
22.53	0.00837856495180774\\
22.54	0.00837856495180774\\
22.55	0.00837856495180774\\
22.56	0.00837856495180774\\
22.57	0.00837856495180774\\
22.58	0.00837856495180774\\
22.59	0.00837856495180774\\
22.6	0.00837856495180774\\
22.61	0.00837856495180774\\
22.62	0.00837856495180774\\
22.63	0.00837856495180774\\
22.64	0.00837856495180774\\
22.65	0.00837856495180774\\
22.66	0.00837856495180774\\
22.67	0.00837856495180774\\
22.68	0.00837856495180774\\
22.69	0.00837856495180774\\
22.7	0.00837856495180774\\
22.71	0.00837856495180774\\
22.72	0.00837856495180774\\
22.73	0.00837856495180774\\
22.74	0.00837856495180774\\
22.75	0.00837856495180774\\
22.76	0.00837856495180774\\
22.77	0.00837856495180774\\
22.78	0.00837856495180774\\
22.79	0.00837856495180774\\
22.8	0.00837856495180774\\
22.81	0.00837856495180774\\
22.82	0.00837856495180774\\
22.83	0.00837856495180774\\
22.84	0.00837856495180774\\
22.85	0.00837856495180774\\
22.86	0.00837856495180774\\
22.87	0.00837856495180774\\
22.88	0.00837856495180774\\
22.89	0.00837856495180774\\
22.9	0.00837856495180774\\
22.91	0.00837856495180774\\
22.92	0.00837856495180774\\
22.93	0.00837856495180774\\
22.94	0.00837856495180774\\
22.95	0.00837856495180774\\
22.96	0.00837856495180774\\
22.97	0.00837856495180774\\
22.98	0.00837856495180774\\
22.99	0.00837856495180774\\
23	0.00837856495180774\\
23.01	0.00837856495180774\\
23.02	0.00837856495180774\\
23.03	0.00837856495180774\\
23.04	0.00837856495180774\\
23.05	0.00837856495180774\\
23.06	0.00837856495180774\\
23.07	0.00837856495180774\\
23.08	0.00837856495180774\\
23.09	0.00837856495180774\\
23.1	0.00837856495180774\\
23.11	0.00837856495180774\\
23.12	0.00837856495180774\\
23.13	0.00837856495180774\\
23.14	0.00837856495180774\\
23.15	0.00837856495180774\\
23.16	0.00837856495180774\\
23.17	0.00837856495180774\\
23.18	0.00837856495180774\\
23.19	0.00837856495180774\\
23.2	0.00837856495180774\\
23.21	0.00837856495180774\\
23.22	0.00837856495180774\\
23.23	0.00837856495180774\\
23.24	0.00837856495180774\\
23.25	0.00837856495180774\\
23.26	0.00837856495180774\\
23.27	0.00837856495180774\\
23.28	0.00837856495180774\\
23.29	0.00837856495180774\\
23.3	0.00837856495180774\\
23.31	0.00837856495180774\\
23.32	0.00837856495180774\\
23.33	0.00837856495180774\\
23.34	0.00837856495180774\\
23.35	0.00837856495180774\\
23.36	0.00837856495180774\\
23.37	0.00837856495180774\\
23.38	0.00837856495180774\\
23.39	0.00837856495180774\\
23.4	0.00837856495180774\\
23.41	0.00837856495180774\\
23.42	0.00837856495180774\\
23.43	0.00837856495180774\\
23.44	0.00837856495180774\\
23.45	0.00837856495180774\\
23.46	0.00837856495180774\\
23.47	0.00837856495180774\\
23.48	0.00837856495180774\\
23.49	0.00837856495180774\\
23.5	0.00837856495180774\\
23.51	0.00837856495180774\\
23.52	0.00837856495180774\\
23.53	0.00837856495180774\\
23.54	0.00837856495180774\\
23.55	0.00837856495180774\\
23.56	0.00837856495180774\\
23.57	0.00837856495180774\\
23.58	0.00837856495180774\\
23.59	0.00837856495180774\\
23.6	0.00837856495180774\\
23.61	0.00837856495180774\\
23.62	0.00837856495180774\\
23.63	0.00837856495180774\\
23.64	0.00837856495180774\\
23.65	0.00837856495180774\\
23.66	0.00837856495180774\\
23.67	0.00837856495180774\\
23.68	0.00837856495180774\\
23.69	0.00837856495180774\\
23.7	0.00837856495180774\\
23.71	0.00837856495180774\\
23.72	0.00837856495180774\\
23.73	0.00837856495180774\\
23.74	0.00837856495180774\\
23.75	0.00837856495180774\\
23.76	0.00837856495180774\\
23.77	0.00837856495180774\\
23.78	0.00837856495180774\\
23.79	0.00837856495180774\\
23.8	0.00837856495180774\\
23.81	0.00837856495180774\\
23.82	0.00837856495180774\\
23.83	0.00837856495180774\\
23.84	0.00837856495180774\\
23.85	0.00837856495180774\\
23.86	0.00837856495180774\\
23.87	0.00837856495180774\\
23.88	0.00837856495180774\\
23.89	0.00837856495180774\\
23.9	0.00837856495180774\\
23.91	0.00837856495180774\\
23.92	0.00837856495180774\\
23.93	0.00837856495180774\\
23.94	0.00837856495180774\\
23.95	0.00837856495180774\\
23.96	0.00837856495180774\\
23.97	0.00837856495180774\\
23.98	0.00837856495180774\\
23.99	0.00837856495180774\\
24	0.00837856495180774\\
24.01	0.00837856495180774\\
24.02	0.00837856495180774\\
24.03	0.00837856495180774\\
24.04	0.00837856495180774\\
24.05	0.00837856495180774\\
24.06	0.00837856495180774\\
24.07	0.00837856495180774\\
24.08	0.00837856495180774\\
24.09	0.00837856495180774\\
24.1	0.00837856495180774\\
24.11	0.00837856495180774\\
24.12	0.00837856495180774\\
24.13	0.00837856495180774\\
24.14	0.00837856495180774\\
24.15	0.00837856495180774\\
24.16	0.00837856495180774\\
24.17	0.00837856495180774\\
24.18	0.00837856495180774\\
24.19	0.00837856495180774\\
24.2	0.00837856495180774\\
24.21	0.00837856495180774\\
24.22	0.00837856495180774\\
24.23	0.00837856495180774\\
24.24	0.00837856495180774\\
24.25	0.00837856495180774\\
24.26	0.00837856495180774\\
24.27	0.00837856495180774\\
24.28	0.00837856495180774\\
24.29	0.00837856495180774\\
24.3	0.00837856495180774\\
24.31	0.00837856495180774\\
24.32	0.00837856495180774\\
24.33	0.00837856495180774\\
24.34	0.00837856495180774\\
24.35	0.00837856495180774\\
24.36	0.00837856495180774\\
24.37	0.00837856495180774\\
24.38	0.00837856495180774\\
24.39	0.00837856495180774\\
24.4	0.00837856495180774\\
24.41	0.00837856495180774\\
24.42	0.00837856495180774\\
24.43	0.00837856495180774\\
24.44	0.00837856495180774\\
24.45	0.00837856495180774\\
24.46	0.00837856495180774\\
24.47	0.00837856495180774\\
24.48	0.00837856495180774\\
24.49	0.00837856495180774\\
24.5	0.00837856495180774\\
24.51	0.00837856495180774\\
24.52	0.00837856495180774\\
24.53	0.00837856495180774\\
24.54	0.00837856495180774\\
24.55	0.00837856495180774\\
24.56	0.00837856495180774\\
24.57	0.00837856495180774\\
24.58	0.00837856495180774\\
24.59	0.00837856495180774\\
24.6	0.00837856495180774\\
24.61	0.00837856495180774\\
24.62	0.00837856495180774\\
24.63	0.00837856495180774\\
24.64	0.00837856495180774\\
24.65	0.00837856495180774\\
24.66	0.00837856495180774\\
24.67	0.00837856495180774\\
24.68	0.00837856495180774\\
24.69	0.00837856495180774\\
24.7	0.00837856495180774\\
24.71	0.00837856495180774\\
24.72	0.00837856495180774\\
24.73	0.00837856495180774\\
24.74	0.00837856495180774\\
24.75	0.00837856495180774\\
24.76	0.00837856495180774\\
24.77	0.00837856495180774\\
24.78	0.00837856495180774\\
24.79	0.00837856495180774\\
24.8	0.00837856495180774\\
24.81	0.00837856495180774\\
24.82	0.00837856495180774\\
24.83	0.00837856495180774\\
24.84	0.00837856495180774\\
24.85	0.00837856495180774\\
24.86	0.00837856495180774\\
24.87	0.00837856495180774\\
24.88	0.00837856495180774\\
24.89	0.00837856495180774\\
24.9	0.00837856495180774\\
24.91	0.00837856495180774\\
24.92	0.00837856495180774\\
24.93	0.00837856495180774\\
24.94	0.00837856495180774\\
24.95	0.00837856495180774\\
24.96	0.00837856495180774\\
24.97	0.00837856495180774\\
24.98	0.00837856495180774\\
24.99	0.00837856495180774\\
25	0.00837856495180774\\
25.01	0.00837856495180774\\
25.02	0.00837856495180774\\
25.03	0.00837856495180774\\
25.04	0.00837856495180774\\
25.05	0.00837856495180774\\
25.06	0.00837856495180774\\
25.07	0.00837856495180774\\
25.08	0.00837856495180774\\
25.09	0.00837856495180774\\
25.1	0.00837856495180774\\
25.11	0.00837856495180774\\
25.12	0.00837856495180774\\
25.13	0.00837856495180774\\
25.14	0.00837856495180774\\
25.15	0.00837856495180774\\
25.16	0.00837856495180774\\
25.17	0.00837856495180774\\
25.18	0.00837856495180774\\
25.19	0.00837856495180774\\
25.2	0.00837856495180774\\
25.21	0.00837856495180774\\
25.22	0.00837856495180774\\
25.23	0.00837856495180774\\
25.24	0.00837856495180774\\
25.25	0.00837856495180774\\
25.26	0.00837856495180774\\
25.27	0.00837856495180774\\
25.28	0.00837856495180774\\
25.29	0.00837856495180774\\
25.3	0.00837856495180774\\
25.31	0.00837856495180774\\
25.32	0.00837856495180774\\
25.33	0.00837856495180774\\
25.34	0.00837856495180774\\
25.35	0.00837856495180774\\
25.36	0.00837856495180774\\
25.37	0.00837856495180774\\
25.38	0.00837856495180774\\
25.39	0.00837856495180774\\
25.4	0.00837856495180774\\
25.41	0.00837856495180774\\
25.42	0.00837856495180774\\
25.43	0.00837856495180774\\
25.44	0.00837856495180774\\
25.45	0.00837856495180774\\
25.46	0.00837856495180774\\
25.47	0.00837856495180774\\
25.48	0.00837856495180774\\
25.49	0.00837856495180774\\
25.5	0.00837856495180774\\
25.51	0.00837856495180774\\
25.52	0.00837856495180774\\
25.53	0.00837856495180774\\
25.54	0.00837856495180774\\
25.55	0.00837856495180774\\
25.56	0.00837856495180774\\
25.57	0.00837856495180774\\
25.58	0.00837856495180774\\
25.59	0.00837856495180774\\
25.6	0.00837856495180774\\
25.61	0.00837856495180774\\
25.62	0.00837856495180774\\
25.63	0.00837856495180774\\
25.64	0.00837856495180774\\
25.65	0.00837856495180774\\
25.66	0.00837856495180774\\
25.67	0.00837856495180774\\
25.68	0.00837856495180774\\
25.69	0.00837856495180774\\
25.7	0.00837856495180774\\
25.71	0.00837856495180774\\
25.72	0.00837856495180774\\
25.73	0.00837856495180774\\
25.74	0.00837856495180774\\
25.75	0.00837856495180774\\
25.76	0.00837856495180774\\
25.77	0.00837856495180774\\
25.78	0.00837856495180774\\
25.79	0.00837856495180774\\
25.8	0.00837856495180774\\
25.81	0.00837856495180774\\
25.82	0.00837856495180774\\
25.83	0.00837856495180774\\
25.84	0.00837856495180774\\
25.85	0.00837856495180774\\
25.86	0.00837856495180774\\
25.87	0.00837856495180774\\
25.88	0.00837856495180774\\
25.89	0.00837856495180774\\
25.9	0.00837856495180774\\
25.91	0.00837856495180774\\
25.92	0.00837856495180774\\
25.93	0.00837856495180774\\
25.94	0.00837856495180774\\
25.95	0.00837856495180774\\
25.96	0.00837856495180774\\
25.97	0.00837856495180774\\
25.98	0.00837856495180774\\
25.99	0.00837856495180774\\
26	0.00837856495180774\\
26.01	0.00837856495180774\\
26.02	0.00837856495180774\\
26.03	0.00837856495180774\\
26.04	0.00837856495180774\\
26.05	0.00837856495180774\\
26.06	0.00837856495180774\\
26.07	0.00837856495180774\\
26.08	0.00837856495180774\\
26.09	0.00837856495180774\\
26.1	0.00837856495180774\\
26.11	0.00837856495180774\\
26.12	0.00837856495180774\\
26.13	0.00837856495180774\\
26.14	0.00837856495180774\\
26.15	0.00837856495180774\\
26.16	0.00837856495180774\\
26.17	0.00837856495180774\\
26.18	0.00837856495180774\\
26.19	0.00837856495180774\\
26.2	0.00837856495180774\\
26.21	0.00837856495180774\\
26.22	0.00837856495180774\\
26.23	0.00837856495180774\\
26.24	0.00837856495180774\\
26.25	0.00837856495180774\\
26.26	0.00837856495180774\\
26.27	0.00837856495180774\\
26.28	0.00837856495180774\\
26.29	0.00837856495180774\\
26.3	0.00837856495180774\\
26.31	0.00837856495180774\\
26.32	0.00837856495180774\\
26.33	0.00837856495180774\\
26.34	0.00837856495180774\\
26.35	0.00837856495180774\\
26.36	0.00837856495180774\\
26.37	0.00837856495180774\\
26.38	0.00837856495180774\\
26.39	0.00837856495180774\\
26.4	0.00837856495180774\\
26.41	0.00837856495180774\\
26.42	0.00837856495180774\\
26.43	0.00837856495180774\\
26.44	0.00837856495180774\\
26.45	0.00837856495180774\\
26.46	0.00837856495180774\\
26.47	0.00837856495180774\\
26.48	0.00837856495180774\\
26.49	0.00837856495180774\\
26.5	0.00837856495180774\\
26.51	0.00837856495180774\\
26.52	0.00837856495180774\\
26.53	0.00837856495180774\\
26.54	0.00837856495180774\\
26.55	0.00837856495180774\\
26.56	0.00837856495180774\\
26.57	0.00837856495180774\\
26.58	0.00837856495180774\\
26.59	0.00837856495180774\\
26.6	0.00837856495180774\\
26.61	0.00837856495180774\\
26.62	0.00837856495180774\\
26.63	0.00837856495180774\\
26.64	0.00837856495180774\\
26.65	0.00837856495180774\\
26.66	0.00837856495180774\\
26.67	0.00837856495180774\\
26.68	0.00837856495180774\\
26.69	0.00837856495180774\\
26.7	0.00837856495180774\\
26.71	0.00837856495180774\\
26.72	0.00837856495180774\\
26.73	0.00837856495180774\\
26.74	0.00837856495180774\\
26.75	0.00837856495180774\\
26.76	0.00837856495180774\\
26.77	0.00837856495180774\\
26.78	0.00837856495180774\\
26.79	0.00837856495180774\\
26.8	0.00837856495180774\\
26.81	0.00837856495180774\\
26.82	0.00837856495180774\\
26.83	0.00837856495180774\\
26.84	0.00837856495180774\\
26.85	0.00837856495180774\\
26.86	0.00837856495180774\\
26.87	0.00837856495180774\\
26.88	0.00837856495180774\\
26.89	0.00837856495180774\\
26.9	0.00837856495180774\\
26.91	0.00837856495180774\\
26.92	0.00837856495180774\\
26.93	0.00837856495180774\\
26.94	0.00837856495180774\\
26.95	0.00837856495180774\\
26.96	0.00837856495180774\\
26.97	0.00837856495180774\\
26.98	0.00837856495180774\\
26.99	0.00837856495180774\\
27	0.00837856495180774\\
27.01	0.00837856495180774\\
27.02	0.00837856495180774\\
27.03	0.00837856495180774\\
27.04	0.00837856495180774\\
27.05	0.00837856495180774\\
27.06	0.00837856495180774\\
27.07	0.00837856495180774\\
27.08	0.00837856495180774\\
27.09	0.00837856495180774\\
27.1	0.00837856495180774\\
27.11	0.00837856495180774\\
27.12	0.00837856495180774\\
27.13	0.00837856495180774\\
27.14	0.00837856495180774\\
27.15	0.00837856495180774\\
27.16	0.00837856495180774\\
27.17	0.00837856495180774\\
27.18	0.00837856495180774\\
27.19	0.00837856495180774\\
27.2	0.00837856495180774\\
27.21	0.00837856495180774\\
27.22	0.00837856495180774\\
27.23	0.00837856495180774\\
27.24	0.00837856495180774\\
27.25	0.00837856495180774\\
27.26	0.00837856495180774\\
27.27	0.00837856495180774\\
27.28	0.00837856495180774\\
27.29	0.00837856495180774\\
27.3	0.00837856495180774\\
27.31	0.00837856495180774\\
27.32	0.00837856495180774\\
27.33	0.00837856495180774\\
27.34	0.00837856495180774\\
27.35	0.00837856495180774\\
27.36	0.00837856495180774\\
27.37	0.00837856495180774\\
27.38	0.00837856495180774\\
27.39	0.00837856495180774\\
27.4	0.00837856495180774\\
27.41	0.00837856495180774\\
27.42	0.00837856495180774\\
27.43	0.00837856495180774\\
27.44	0.00837856495180774\\
27.45	0.00837856495180774\\
27.46	0.00837856495180774\\
27.47	0.00837856495180774\\
27.48	0.00837856495180774\\
27.49	0.00837856495180774\\
27.5	0.00837856495180774\\
27.51	0.00837856495180774\\
27.52	0.00837856495180774\\
27.53	0.00837856495180774\\
27.54	0.00837856495180774\\
27.55	0.00837856495180774\\
27.56	0.00837856495180774\\
27.57	0.00837856495180774\\
27.58	0.00837856495180774\\
27.59	0.00837856495180774\\
27.6	0.00837856495180774\\
27.61	0.00837856495180774\\
27.62	0.00837856495180774\\
27.63	0.00837856495180774\\
27.64	0.00837856495180774\\
27.65	0.00837856495180774\\
27.66	0.00837856495180774\\
27.67	0.00837856495180774\\
27.68	0.00837856495180774\\
27.69	0.00837856495180774\\
27.7	0.00837856495180774\\
27.71	0.00837856495180774\\
27.72	0.00837856495180774\\
27.73	0.00837856495180774\\
27.74	0.00837856495180774\\
27.75	0.00837856495180774\\
27.76	0.00837856495180774\\
27.77	0.00837856495180774\\
27.78	0.00837856495180774\\
27.79	0.00837856495180774\\
27.8	0.00837856495180774\\
27.81	0.00837856495180774\\
27.82	0.00837856495180774\\
27.83	0.00837856495180774\\
27.84	0.00837856495180774\\
27.85	0.00837856495180774\\
27.86	0.00837856495180774\\
27.87	0.00837856495180774\\
27.88	0.00837856495180774\\
27.89	0.00837856495180774\\
27.9	0.00837856495180774\\
27.91	0.00837856495180774\\
27.92	0.00837856495180774\\
27.93	0.00837856495180774\\
27.94	0.00837856495180774\\
27.95	0.00837856495180774\\
27.96	0.00837856495180774\\
27.97	0.00837856495180774\\
27.98	0.00837856495180774\\
27.99	0.00837856495180774\\
28	0.00837856495180774\\
28.01	0.00837856495180774\\
28.02	0.00837856495180774\\
28.03	0.00837856495180774\\
28.04	0.00837856495180774\\
28.05	0.00837856495180774\\
28.06	0.00837856495180774\\
28.07	0.00837856495180774\\
28.08	0.00837856495180774\\
28.09	0.00837856495180774\\
28.1	0.00837856495180774\\
28.11	0.00837856495180774\\
28.12	0.00837856495180774\\
28.13	0.00837856495180774\\
28.14	0.00837856495180774\\
28.15	0.00837856495180774\\
28.16	0.00837856495180774\\
28.17	0.00837856495180774\\
28.18	0.00837856495180774\\
28.19	0.00837856495180774\\
28.2	0.00837856495180774\\
28.21	0.00837856495180774\\
28.22	0.00837856495180774\\
28.23	0.00837856495180774\\
28.24	0.00837856495180774\\
28.25	0.00837856495180774\\
28.26	0.00837856495180774\\
28.27	0.00837856495180774\\
28.28	0.00837856495180774\\
28.29	0.00837856495180774\\
28.3	0.00837856495180774\\
28.31	0.00837856495180774\\
28.32	0.00837856495180774\\
28.33	0.00837856495180774\\
28.34	0.00837856495180774\\
28.35	0.00837856495180774\\
28.36	0.00837856495180774\\
28.37	0.00837856495180774\\
28.38	0.00837856495180774\\
28.39	0.00837856495180774\\
28.4	0.00837856495180774\\
28.41	0.00837856495180774\\
28.42	0.00837856495180774\\
28.43	0.00837856495180774\\
28.44	0.00837856495180774\\
28.45	0.00837856495180774\\
28.46	0.00837856495180774\\
28.47	0.00837856495180774\\
28.48	0.00837856495180774\\
28.49	0.00837856495180774\\
28.5	0.00837856495180774\\
28.51	0.00837856495180774\\
28.52	0.00837856495180774\\
28.53	0.00837856495180774\\
28.54	0.00837856495180774\\
28.55	0.00837856495180774\\
28.56	0.00837856495180774\\
28.57	0.00837856495180774\\
28.58	0.00837856495180774\\
28.59	0.00837856495180774\\
28.6	0.00837856495180774\\
28.61	0.00837856495180774\\
28.62	0.00837856495180774\\
28.63	0.00837856495180774\\
28.64	0.00837856495180774\\
28.65	0.00837856495180774\\
28.66	0.00837856495180774\\
28.67	0.00837856495180774\\
28.68	0.00837856495180774\\
28.69	0.00837856495180774\\
28.7	0.00837856495180774\\
28.71	0.00837856495180774\\
28.72	0.00837856495180774\\
28.73	0.00837856495180774\\
28.74	0.00837856495180774\\
28.75	0.00837856495180774\\
28.76	0.00837856495180774\\
28.77	0.00837856495180774\\
28.78	0.00837856495180774\\
28.79	0.00837856495180774\\
28.8	0.00837856495180774\\
28.81	0.00837856495180774\\
28.82	0.00837856495180774\\
28.83	0.00837856495180774\\
28.84	0.00837856495180774\\
28.85	0.00837856495180774\\
28.86	0.00837856495180774\\
28.87	0.00837856495180774\\
28.88	0.00837856495180774\\
28.89	0.00837856495180774\\
28.9	0.00837856495180774\\
28.91	0.00837856495180774\\
28.92	0.00837856495180774\\
28.93	0.00837856495180774\\
28.94	0.00837856495180774\\
28.95	0.00837856495180774\\
28.96	0.00837856495180774\\
28.97	0.00837856495180774\\
28.98	0.00837856495180774\\
28.99	0.00837856495180774\\
29	0.00837856495180774\\
29.01	0.00837856495180774\\
29.02	0.00837856495180774\\
29.03	0.00837856495180774\\
29.04	0.00837856495180774\\
29.05	0.00837856495180774\\
29.06	0.00837856495180774\\
29.07	0.00837856495180774\\
29.08	0.00837856495180774\\
29.09	0.00837856495180774\\
29.1	0.00837856495180774\\
29.11	0.00837856495180774\\
29.12	0.00837856495180774\\
29.13	0.00837856495180774\\
29.14	0.00837856495180774\\
29.15	0.00837856495180774\\
29.16	0.00837856495180774\\
29.17	0.00837856495180774\\
29.18	0.00837856495180774\\
29.19	0.00837856495180774\\
29.2	0.00837856495180774\\
29.21	0.00837856495180774\\
29.22	0.00837856495180774\\
29.23	0.00837856495180774\\
29.24	0.00837856495180774\\
29.25	0.00837856495180774\\
29.26	0.00837856495180774\\
29.27	0.00837856495180774\\
29.28	0.00837856495180774\\
29.29	0.00837856495180774\\
29.3	0.00837856495180774\\
29.31	0.00837856495180774\\
29.32	0.00837856495180774\\
29.33	0.00837856495180774\\
29.34	0.00837856495180774\\
29.35	0.00837856495180774\\
29.36	0.00837856495180774\\
29.37	0.00837856495180774\\
29.38	0.00837856495180774\\
29.39	0.00837856495180774\\
29.4	0.00837856495180774\\
29.41	0.00837856495180774\\
29.42	0.00837856495180774\\
29.43	0.00837856495180774\\
29.44	0.00837856495180774\\
29.45	0.00837856495180774\\
29.46	0.00837856495180774\\
29.47	0.00837856495180774\\
29.48	0.00837856495180774\\
29.49	0.00837856495180774\\
29.5	0.00837856495180774\\
29.51	0.00837856495180774\\
29.52	0.00837856495180774\\
29.53	0.00837856495180774\\
29.54	0.00837856495180774\\
29.55	0.00837856495180774\\
29.56	0.00837856495180774\\
29.57	0.00837856495180774\\
29.58	0.00837856495180774\\
29.59	0.00837856495180774\\
29.6	0.00837856495180774\\
29.61	0.00837856495180774\\
29.62	0.00837856495180774\\
29.63	0.00837856495180774\\
29.64	0.00837856495180774\\
29.65	0.00837856495180774\\
29.66	0.00837856495180774\\
29.67	0.00837856495180774\\
29.68	0.00837856495180774\\
29.69	0.00837856495180774\\
29.7	0.00837856495180774\\
29.71	0.00837856495180774\\
29.72	0.00837856495180774\\
29.73	0.00837856495180774\\
29.74	0.00837856495180774\\
29.75	0.00837856495180774\\
29.76	0.00837856495180774\\
29.77	0.00837856495180774\\
29.78	0.00837856495180774\\
29.79	0.00837856495180774\\
29.8	0.00837856495180774\\
29.81	0.00837856495180774\\
29.82	0.00837856495180774\\
29.83	0.00837856495180774\\
29.84	0.00837856495180774\\
29.85	0.00837856495180774\\
29.86	0.00837856495180774\\
29.87	0.00837856495180774\\
29.88	0.00837856495180774\\
29.89	0.00837856495180774\\
29.9	0.00837856495180774\\
29.91	0.00837856495180774\\
29.92	0.00837856495180774\\
29.93	0.00837856495180774\\
29.94	0.00837856495180774\\
29.95	0.00837856495180774\\
29.96	0.00837856495180774\\
29.97	0.00837856495180774\\
29.98	0.00837856495180774\\
29.99	0.00837856495180774\\
30	0.00837856495180774\\
30.01	0.00837856495180774\\
30.02	0.00837856495180774\\
30.03	0.00837856495180774\\
30.04	0.00837856495180774\\
30.05	0.00837856495180774\\
30.06	0.00837856495180774\\
30.07	0.00837856495180774\\
30.08	0.00837856495180774\\
30.09	0.00837856495180774\\
30.1	0.00837856495180774\\
30.11	0.00837856495180774\\
30.12	0.00837856495180774\\
30.13	0.00837856495180774\\
30.14	0.00837856495180774\\
30.15	0.00837856495180774\\
30.16	0.00837856495180774\\
30.17	0.00837856495180774\\
30.18	0.00837856495180774\\
30.19	0.00837856495180774\\
30.2	0.00837856495180774\\
30.21	0.00837856495180774\\
30.22	0.00837856495180774\\
30.23	0.00837856495180774\\
30.24	0.00837856495180774\\
30.25	0.00837856495180774\\
30.26	0.00837856495180774\\
30.27	0.00837856495180774\\
30.28	0.00837856495180774\\
30.29	0.00837856495180774\\
30.3	0.00837856495180774\\
30.31	0.00837856495180774\\
30.32	0.00837856495180774\\
30.33	0.00837856495180774\\
30.34	0.00837856495180774\\
30.35	0.00837856495180774\\
30.36	0.00837856495180774\\
30.37	0.00837856495180774\\
30.38	0.00837856495180774\\
30.39	0.00837856495180774\\
30.4	0.00837856495180774\\
30.41	0.00837856495180774\\
30.42	0.00837856495180774\\
30.43	0.00837856495180774\\
30.44	0.00837856495180774\\
30.45	0.00837856495180774\\
30.46	0.00837856495180774\\
30.47	0.00837856495180774\\
30.48	0.00837856495180774\\
30.49	0.00837856495180774\\
30.5	0.00837856495180774\\
30.51	0.00837856495180774\\
30.52	0.00837856495180774\\
30.53	0.00837856495180774\\
30.54	0.00837856495180774\\
30.55	0.00837856495180774\\
30.56	0.00837856495180774\\
30.57	0.00837856495180774\\
30.58	0.00837856495180774\\
30.59	0.00837856495180774\\
30.6	0.00837856495180774\\
30.61	0.00837856495180774\\
30.62	0.00837856495180774\\
30.63	0.00837856495180774\\
30.64	0.00837856495180774\\
30.65	0.00837856495180774\\
30.66	0.00837856495180774\\
30.67	0.00837856495180774\\
30.68	0.00837856495180774\\
30.69	0.00837856495180774\\
30.7	0.00837856495180774\\
30.71	0.00837856495180774\\
30.72	0.00837856495180774\\
30.73	0.00837856495180774\\
30.74	0.00837856495180774\\
30.75	0.00837856495180774\\
30.76	0.00837856495180774\\
30.77	0.00837856495180774\\
30.78	0.00837856495180774\\
30.79	0.00837856495180774\\
30.8	0.00837856495180774\\
30.81	0.00837856495180774\\
30.82	0.00837856495180774\\
30.83	0.00837856495180774\\
30.84	0.00837856495180774\\
30.85	0.00837856495180774\\
30.86	0.00837856495180774\\
30.87	0.00837856495180774\\
30.88	0.00837856495180774\\
30.89	0.00837856495180774\\
30.9	0.00837856495180774\\
30.91	0.00837856495180774\\
30.92	0.00837856495180774\\
30.93	0.00837856495180774\\
30.94	0.00837856495180774\\
30.95	0.00837856495180774\\
30.96	0.00837856495180774\\
30.97	0.00837856495180774\\
30.98	0.00837856495180774\\
30.99	0.00837856495180774\\
31	0.00837856495180774\\
31.01	0.00837856495180774\\
31.02	0.00837856495180774\\
31.03	0.00837856495180774\\
31.04	0.00837856495180774\\
31.05	0.00837856495180774\\
31.06	0.00837856495180774\\
31.07	0.00837856495180774\\
31.08	0.00837856495180774\\
31.09	0.00837856495180774\\
31.1	0.00837856495180774\\
31.11	0.00837856495180774\\
31.12	0.00837856495180774\\
31.13	0.00837856495180774\\
31.14	0.00837856495180774\\
31.15	0.00837856495180774\\
31.16	0.00837856495180774\\
31.17	0.00837856495180774\\
31.18	0.00837856495180774\\
31.19	0.00837856495180774\\
31.2	0.00837856495180774\\
31.21	0.00837856495180774\\
31.22	0.00837856495180774\\
31.23	0.00837856495180774\\
31.24	0.00837856495180774\\
31.25	0.00837856495180774\\
31.26	0.00837856495180774\\
31.27	0.00837856495180774\\
31.28	0.00837856495180774\\
31.29	0.00837856495180774\\
31.3	0.00837856495180774\\
31.31	0.00837856495180774\\
31.32	0.00837856495180774\\
31.33	0.00837856495180774\\
31.34	0.00837856495180774\\
31.35	0.00837856495180774\\
31.36	0.00837856495180774\\
31.37	0.00837856495180774\\
31.38	0.00837856495180774\\
31.39	0.00837856495180774\\
31.4	0.00837856495180774\\
31.41	0.00837856495180774\\
31.42	0.00837856495180774\\
31.43	0.00837856495180774\\
31.44	0.00837856495180774\\
31.45	0.00837856495180774\\
31.46	0.00837856495180774\\
31.47	0.00837856495180774\\
31.48	0.00837856495180774\\
31.49	0.00837856495180774\\
31.5	0.00837856495180774\\
31.51	0.00837856495180774\\
31.52	0.00837856495180774\\
31.53	0.00837856495180774\\
31.54	0.00837856495180774\\
31.55	0.00837856495180774\\
31.56	0.00837856495180774\\
31.57	0.00837856495180774\\
31.58	0.00837856495180774\\
31.59	0.00837856495180774\\
31.6	0.00837856495180774\\
31.61	0.00837856495180774\\
31.62	0.00837856495180774\\
31.63	0.00837856495180774\\
31.64	0.00837856495180774\\
31.65	0.00837856495180774\\
31.66	0.00837856495180774\\
31.67	0.00837856495180774\\
31.68	0.00837856495180774\\
31.69	0.00837856495180774\\
31.7	0.00837856495180774\\
31.71	0.00837856495180774\\
31.72	0.00837856495180774\\
31.73	0.00837856495180774\\
31.74	0.00837856495180774\\
31.75	0.00837856495180774\\
31.76	0.00837856495180774\\
31.77	0.00837856495180774\\
31.78	0.00837856495180774\\
31.79	0.00837856495180774\\
31.8	0.00837856495180774\\
31.81	0.00837856495180774\\
31.82	0.00837856495180774\\
31.83	0.00837856495180774\\
31.84	0.00837856495180774\\
31.85	0.00837856495180774\\
31.86	0.00837856495180774\\
31.87	0.00837856495180774\\
31.88	0.00837856495180774\\
31.89	0.00837856495180774\\
31.9	0.00837856495180774\\
31.91	0.00837856495180774\\
31.92	0.00837856495180774\\
31.93	0.00837856495180774\\
31.94	0.00837856495180774\\
31.95	0.00837856495180774\\
31.96	0.00837856495180774\\
31.97	0.00837856495180774\\
31.98	0.00837856495180774\\
31.99	0.00837856495180774\\
32	0.00837856495180774\\
32.01	0.00837856495180774\\
32.02	0.00837856495180774\\
32.03	0.00837856495180774\\
32.04	0.00837856495180774\\
32.05	0.00837856495180774\\
32.06	0.00837856495180774\\
32.07	0.00837856495180774\\
32.08	0.00837856495180774\\
32.09	0.00837856495180774\\
32.1	0.00837856495180774\\
32.11	0.00837856495180774\\
32.12	0.00837856495180774\\
32.13	0.00837856495180774\\
32.14	0.00837856495180774\\
32.15	0.00837856495180774\\
32.16	0.00837856495180774\\
32.17	0.00837856495180774\\
32.18	0.00837856495180774\\
32.19	0.00837856495180774\\
32.2	0.00837856495180774\\
32.21	0.00837856495180774\\
32.22	0.00837856495180774\\
32.23	0.00837856495180774\\
32.24	0.00837856495180774\\
32.25	0.00837856495180774\\
32.26	0.00837856495180774\\
32.27	0.00837856495180774\\
32.28	0.00837856495180774\\
32.29	0.00837856495180774\\
32.3	0.00837856495180774\\
32.31	0.00837856495180774\\
32.32	0.00837856495180774\\
32.33	0.00837856495180774\\
32.34	0.00837856495180774\\
32.35	0.00837856495180774\\
32.36	0.00837856495180774\\
32.37	0.00837856495180774\\
32.38	0.00837856495180774\\
32.39	0.00837856495180774\\
32.4	0.00837856495180774\\
32.41	0.00837856495180774\\
32.42	0.00837856495180774\\
32.43	0.00837856495180774\\
32.44	0.00837856495180774\\
32.45	0.00837856495180774\\
32.46	0.00837856495180774\\
32.47	0.00837856495180774\\
32.48	0.00837856495180774\\
32.49	0.00837856495180774\\
32.5	0.00837856495180774\\
32.51	0.00837856495180774\\
32.52	0.00837856495180774\\
32.53	0.00837856495180774\\
32.54	0.00837856495180774\\
32.55	0.00837856495180774\\
32.56	0.00837856495180774\\
32.57	0.00837856495180774\\
32.58	0.00837856495180774\\
32.59	0.00837856495180774\\
32.6	0.00837856495180774\\
32.61	0.00837856495180774\\
32.62	0.00837856495180774\\
32.63	0.00837856495180774\\
32.64	0.00837856495180774\\
32.65	0.00837856495180774\\
32.66	0.00837856495180774\\
32.67	0.00837856495180774\\
32.68	0.00837856495180774\\
32.69	0.00837856495180774\\
32.7	0.00837856495180774\\
32.71	0.00837856495180774\\
32.72	0.00837856495180774\\
32.73	0.00837856495180774\\
32.74	0.00837856495180774\\
32.75	0.00837856495180774\\
32.76	0.00837856495180774\\
32.77	0.00837856495180774\\
32.78	0.00837856495180774\\
32.79	0.00837856495180774\\
32.8	0.00837856495180774\\
32.81	0.00837856495180774\\
32.82	0.00837856495180774\\
32.83	0.00837856495180774\\
32.84	0.00837856495180774\\
32.85	0.00837856495180774\\
32.86	0.00837856495180774\\
32.87	0.00837856495180774\\
32.88	0.00837856495180774\\
32.89	0.00837856495180774\\
32.9	0.00837856495180774\\
32.91	0.00837856495180774\\
32.92	0.00837856495180774\\
32.93	0.00837856495180774\\
32.94	0.00837856495180774\\
32.95	0.00837856495180774\\
32.96	0.00837856495180774\\
32.97	0.00837856495180774\\
32.98	0.00837856495180774\\
32.99	0.00837856495180774\\
33	0.00837856495180774\\
33.01	0.00837856495180774\\
33.02	0.00837856495180774\\
33.03	0.00837856495180774\\
33.04	0.00837856495180774\\
33.05	0.00837856495180774\\
33.06	0.00837856495180774\\
33.07	0.00837856495180774\\
33.08	0.00837856495180774\\
33.09	0.00837856495180774\\
33.1	0.00837856495180774\\
33.11	0.00837856495180774\\
33.12	0.00837856495180774\\
33.13	0.00837856495180774\\
33.14	0.00837856495180774\\
33.15	0.00837856495180774\\
33.16	0.00837856495180774\\
33.17	0.00837856495180774\\
33.18	0.00837856495180774\\
33.19	0.00837856495180774\\
33.2	0.00837856495180774\\
33.21	0.00837856495180774\\
33.22	0.00837856495180774\\
33.23	0.00837856495180774\\
33.24	0.00837856495180774\\
33.25	0.00837856495180774\\
33.26	0.00837856495180774\\
33.27	0.00837856495180774\\
33.28	0.00837856495180774\\
33.29	0.00837856495180774\\
33.3	0.00837856495180774\\
33.31	0.00837856495180774\\
33.32	0.00837856495180774\\
33.33	0.00837856495180774\\
33.34	0.00837856495180774\\
33.35	0.00837856495180774\\
33.36	0.00837856495180774\\
33.37	0.00837856495180774\\
33.38	0.00837856495180774\\
33.39	0.00837856495180774\\
33.4	0.00837856495180774\\
33.41	0.00837856495180774\\
33.42	0.00837856495180774\\
33.43	0.00837856495180774\\
33.44	0.00837856495180774\\
33.45	0.00837856495180774\\
33.46	0.00837856495180774\\
33.47	0.00837856495180774\\
33.48	0.00837856495180774\\
33.49	0.00837856495180774\\
33.5	0.00837856495180774\\
33.51	0.00837856495180774\\
33.52	0.00837856495180774\\
33.53	0.00837856495180774\\
33.54	0.00837856495180774\\
33.55	0.00837856495180774\\
33.56	0.00837856495180774\\
33.57	0.00837856495180774\\
33.58	0.00837856495180774\\
33.59	0.00837856495180774\\
33.6	0.00837856495180774\\
33.61	0.00837856495180774\\
33.62	0.00837856495180774\\
33.63	0.00837856495180774\\
33.64	0.00837856495180774\\
33.65	0.00837856495180774\\
33.66	0.00837856495180774\\
33.67	0.00837856495180774\\
33.68	0.00837856495180774\\
33.69	0.00837856495180774\\
33.7	0.00837856495180774\\
33.71	0.00837856495180774\\
33.72	0.00837856495180774\\
33.73	0.00837856495180774\\
33.74	0.00837856495180774\\
33.75	0.00837856495180774\\
33.76	0.00837856495180774\\
33.77	0.00837856495180774\\
33.78	0.00837856495180774\\
33.79	0.00837856495180774\\
33.8	0.00837856495180774\\
33.81	0.00837856495180774\\
33.82	0.00837856495180774\\
33.83	0.00837856495180774\\
33.84	0.00837856495180774\\
33.85	0.00837856495180774\\
33.86	0.00837856495180774\\
33.87	0.00837856495180774\\
33.88	0.00837856495180774\\
33.89	0.00837856495180774\\
33.9	0.00837856495180774\\
33.91	0.00837856495180774\\
33.92	0.00837856495180774\\
33.93	0.00837856495180774\\
33.94	0.00837856495180774\\
33.95	0.00837856495180774\\
33.96	0.00837856495180774\\
33.97	0.00837856495180774\\
33.98	0.00837856495180774\\
33.99	0.00837856495180774\\
34	0.00837856495180774\\
34.01	0.00837856495180774\\
34.02	0.00837856495180774\\
34.03	0.00837856495180774\\
34.04	0.00837856495180774\\
34.05	0.00837856495180774\\
34.06	0.00837856495180774\\
34.07	0.00837856495180774\\
34.08	0.00837856495180774\\
34.09	0.00837856495180774\\
34.1	0.00837856495180774\\
34.11	0.00837856495180774\\
34.12	0.00837856495180774\\
34.13	0.00837856495180774\\
34.14	0.00837856495180774\\
34.15	0.00837856495180774\\
34.16	0.00837856495180774\\
34.17	0.00837856495180774\\
34.18	0.00837856495180774\\
34.19	0.00837856495180774\\
34.2	0.00837856495180774\\
34.21	0.00837856495180774\\
34.22	0.00837856495180774\\
34.23	0.00837856495180774\\
34.24	0.00837856495180774\\
34.25	0.00837856495180774\\
34.26	0.00837856495180774\\
34.27	0.00837856495180774\\
34.28	0.00837856495180774\\
34.29	0.00837856495180774\\
34.3	0.00837856495180774\\
34.31	0.00837856495180774\\
34.32	0.00837856495180774\\
34.33	0.00837856495180774\\
34.34	0.00837856495180774\\
34.35	0.00837856495180774\\
34.36	0.00837856495180774\\
34.37	0.00837856495180774\\
34.38	0.00837856495180774\\
34.39	0.00837856495180774\\
34.4	0.00837856495180774\\
34.41	0.00837856495180774\\
34.42	0.00837856495180774\\
34.43	0.00837856495180774\\
34.44	0.00837856495180774\\
34.45	0.00837856495180774\\
34.46	0.00837856495180774\\
34.47	0.00837856495180774\\
34.48	0.00837856495180774\\
34.49	0.00837856495180774\\
34.5	0.00837856495180774\\
34.51	0.00837856495180774\\
34.52	0.00837856495180774\\
34.53	0.00837856495180774\\
34.54	0.00837856495180774\\
34.55	0.00837856495180774\\
34.56	0.00837856495180774\\
34.57	0.00837856495180774\\
34.58	0.00837856495180774\\
34.59	0.00837856495180774\\
34.6	0.00837856495180774\\
34.61	0.00837856495180774\\
34.62	0.00837856495180774\\
34.63	0.00837856495180774\\
34.64	0.00837856495180774\\
34.65	0.00837856495180774\\
34.66	0.00837856495180774\\
34.67	0.00837856495180774\\
34.68	0.00837856495180774\\
34.69	0.00837856495180774\\
34.7	0.00837856495180774\\
34.71	0.00837856495180774\\
34.72	0.00837856495180774\\
34.73	0.00837856495180774\\
34.74	0.00837856495180774\\
34.75	0.00837856495180774\\
34.76	0.00837856495180774\\
34.77	0.00837856495180774\\
34.78	0.00837856495180774\\
34.79	0.00837856495180774\\
34.8	0.00837856495180774\\
34.81	0.00837856495180774\\
34.82	0.00837856495180774\\
34.83	0.00837856495180774\\
34.84	0.00837856495180774\\
34.85	0.00837856495180774\\
34.86	0.00837856495180774\\
34.87	0.00837856495180774\\
34.88	0.00837856495180774\\
34.89	0.00837856495180774\\
34.9	0.00837856495180774\\
34.91	0.00837856495180774\\
34.92	0.00837856495180774\\
34.93	0.00837856495180774\\
34.94	0.00837856495180774\\
34.95	0.00837856495180774\\
34.96	0.00837856495180774\\
34.97	0.00837856495180774\\
34.98	0.00837856495180774\\
34.99	0.00837856495180774\\
35	0.00837856495180774\\
35.01	0.00837856495180774\\
35.02	0.00837856495180774\\
35.03	0.00837856495180774\\
35.04	0.00837856495180774\\
35.05	0.00837856495180774\\
35.06	0.00837856495180774\\
35.07	0.00837856495180774\\
35.08	0.00837856495180774\\
35.09	0.00837856495180774\\
35.1	0.00837856495180774\\
35.11	0.00837856495180774\\
35.12	0.00837856495180774\\
35.13	0.00837856495180774\\
35.14	0.00837856495180774\\
35.15	0.00837856495180774\\
35.16	0.00837856495180774\\
35.17	0.00837856495180774\\
35.18	0.00837856495180774\\
35.19	0.00837856495180774\\
35.2	0.00837856495180774\\
35.21	0.00837856495180774\\
35.22	0.00837856495180774\\
35.23	0.00837856495180774\\
35.24	0.00837856495180774\\
35.25	0.00837856495180774\\
35.26	0.00837856495180774\\
35.27	0.00837856495180774\\
35.28	0.00837856495180774\\
35.29	0.00837856495180774\\
35.3	0.00837856495180774\\
35.31	0.00837856495180774\\
35.32	0.00837856495180774\\
35.33	0.00837856495180774\\
35.34	0.00837856495180774\\
35.35	0.00837856495180774\\
35.36	0.00837856495180774\\
35.37	0.00837856495180774\\
35.38	0.00837856495180774\\
35.39	0.00837856495180774\\
35.4	0.00837856495180774\\
35.41	0.00837856495180774\\
35.42	0.00837856495180774\\
35.43	0.00837856495180774\\
35.44	0.00837856495180774\\
35.45	0.00837856495180774\\
35.46	0.00837856495180774\\
35.47	0.00837856495180774\\
35.48	0.00837856495180774\\
35.49	0.00837856495180774\\
35.5	0.00837856495180774\\
35.51	0.00837856495180774\\
35.52	0.00837856495180774\\
35.53	0.00837856495180774\\
35.54	0.00837856495180774\\
35.55	0.00837856495180774\\
35.56	0.00837856495180774\\
35.57	0.00837856495180774\\
35.58	0.00837856495180774\\
35.59	0.00837856495180774\\
35.6	0.00837856495180774\\
35.61	0.00837856495180774\\
35.62	0.00837856495180774\\
35.63	0.00837856495180774\\
35.64	0.00837856495180774\\
35.65	0.00837856495180774\\
35.66	0.00837856495180774\\
35.67	0.00837856495180774\\
35.68	0.00837856495180774\\
35.69	0.00837856495180774\\
35.7	0.00837856495180774\\
35.71	0.00837856495180774\\
35.72	0.00837856495180774\\
35.73	0.00837856495180774\\
35.74	0.00837856495180774\\
35.75	0.00837856495180774\\
35.76	0.00837856495180774\\
35.77	0.00837856495180774\\
35.78	0.00837856495180774\\
35.79	0.00837856495180774\\
35.8	0.00837856495180774\\
35.81	0.00837856495180774\\
35.82	0.00837856495180774\\
35.83	0.00837856495180774\\
35.84	0.00837856495180774\\
35.85	0.00837856495180774\\
35.86	0.00837856495180774\\
35.87	0.00837856495180774\\
35.88	0.00837856495180774\\
35.89	0.00837856495180774\\
35.9	0.00837856495180774\\
35.91	0.00837856495180774\\
35.92	0.00837856495180774\\
35.93	0.00837856495180774\\
35.94	0.00837856495180774\\
35.95	0.00837856495180774\\
35.96	0.00837856495180774\\
35.97	0.00837856495180774\\
35.98	0.00837856495180774\\
35.99	0.00837856495180774\\
36	0.00837856495180774\\
36.01	0.00837856495180774\\
36.02	0.00837856495180774\\
36.03	0.00837856495180774\\
36.04	0.00837856495180774\\
36.05	0.00837856495180774\\
36.06	0.00837856495180774\\
36.07	0.00837856495180774\\
36.08	0.00837856495180774\\
36.09	0.00837856495180774\\
36.1	0.00837856495180774\\
36.11	0.00837856495180774\\
36.12	0.00837856495180774\\
36.13	0.00837856495180774\\
36.14	0.00837856495180774\\
36.15	0.00837856495180774\\
36.16	0.00837856495180774\\
36.17	0.00837856495180774\\
36.18	0.00837856495180774\\
36.19	0.00837856495180774\\
36.2	0.00837856495180774\\
36.21	0.00837856495180774\\
36.22	0.00837856495180774\\
36.23	0.00837856495180774\\
36.24	0.00837856495180774\\
36.25	0.00837856495180774\\
36.26	0.00837856495180774\\
36.27	0.00837856495180774\\
36.28	0.00837856495180774\\
36.29	0.00837856495180774\\
36.3	0.00837856495180774\\
36.31	0.00837856495180774\\
36.32	0.00837856495180774\\
36.33	0.00837856495180774\\
36.34	0.00837856495180774\\
36.35	0.00837856495180774\\
36.36	0.00837856495180774\\
36.37	0.00837856495180774\\
36.38	0.00837856495180774\\
36.39	0.00837856495180774\\
36.4	0.00837856495180774\\
36.41	0.00837856495180774\\
36.42	0.00837856495180774\\
36.43	0.00837856495180774\\
36.44	0.00837856495180774\\
36.45	0.00837856495180774\\
36.46	0.00837856495180774\\
36.47	0.00837856495180774\\
36.48	0.00837856495180774\\
36.49	0.00837856495180774\\
36.5	0.00837856495180774\\
36.51	0.00837856495180774\\
36.52	0.00837856495180774\\
36.53	0.00837856495180774\\
36.54	0.00837856495180774\\
36.55	0.00837856495180774\\
36.56	0.00837856495180774\\
36.57	0.00837856495180774\\
36.58	0.00837856495180774\\
36.59	0.00837856495180774\\
36.6	0.00837856495180774\\
36.61	0.00837856495180774\\
36.62	0.00837856495180774\\
36.63	0.00837856495180774\\
36.64	0.00837856495180774\\
36.65	0.00837856495180774\\
36.66	0.00837856495180774\\
36.67	0.00837856495180774\\
36.68	0.00837856495180774\\
36.69	0.00837856495180774\\
36.7	0.00837856495180774\\
36.71	0.00837856495180774\\
36.72	0.00837856495180774\\
36.73	0.00837856495180774\\
36.74	0.00837856495180774\\
36.75	0.00837856495180774\\
36.76	0.00837856495180774\\
36.77	0.00837856495180774\\
36.78	0.00837856495180774\\
36.79	0.00837856495180774\\
36.8	0.00837856495180774\\
36.81	0.00837856495180774\\
36.82	0.00837856495180774\\
36.83	0.00837856495180774\\
36.84	0.00837856495180774\\
36.85	0.00837856495180774\\
36.86	0.00837856495180774\\
36.87	0.00837856495180774\\
36.88	0.00837856495180774\\
36.89	0.00837856495180774\\
36.9	0.00837856495180774\\
36.91	0.00837856495180774\\
36.92	0.00837856495180774\\
36.93	0.00837856495180774\\
36.94	0.00837856495180774\\
36.95	0.00837856495180774\\
36.96	0.00837856495180774\\
36.97	0.00837856495180774\\
36.98	0.00837856495180774\\
36.99	0.00837856495180774\\
37	0.00837856495180774\\
37.01	0.00837856495180774\\
37.02	0.00837856495180774\\
37.03	0.00837856495180774\\
37.04	0.00837856495180774\\
37.05	0.00837856495180774\\
37.06	0.00837856495180774\\
37.07	0.00837856495180774\\
37.08	0.00837856495180774\\
37.09	0.00837856495180774\\
37.1	0.00837856495180774\\
37.11	0.00837856495180774\\
37.12	0.00837856495180774\\
37.13	0.00837856495180774\\
37.14	0.00837856495180774\\
37.15	0.00837856495180774\\
37.16	0.00837856495180774\\
37.17	0.00837856495180774\\
37.18	0.00837856495180774\\
37.19	0.00837856495180774\\
37.2	0.00837856495180774\\
37.21	0.00837856495180774\\
37.22	0.00837856495180774\\
37.23	0.00837856495180774\\
37.24	0.00837856495180774\\
37.25	0.00837856495180774\\
37.26	0.00837856495180774\\
37.27	0.00837856495180774\\
37.28	0.00837856495180774\\
37.29	0.00837856495180774\\
37.3	0.00837856495180774\\
37.31	0.00837856495180774\\
37.32	0.00837856495180774\\
37.33	0.00837856495180774\\
37.34	0.00837856495180774\\
37.35	0.00837856495180774\\
37.36	0.00837856495180774\\
37.37	0.00837856495180774\\
37.38	0.00837856495180774\\
37.39	0.00837856495180774\\
37.4	0.00837856495180774\\
37.41	0.00837856495180774\\
37.42	0.00837856495180774\\
37.43	0.00837856495180774\\
37.44	0.00837856495180774\\
37.45	0.00837856495180774\\
37.46	0.00837856495180774\\
37.47	0.00837856495180774\\
37.48	0.00837856495180774\\
37.49	0.00837856495180774\\
37.5	0.00837856495180774\\
37.51	0.00837856495180774\\
37.52	0.00837856495180774\\
37.53	0.00837856495180774\\
37.54	0.00837856495180774\\
37.55	0.00837856495180774\\
37.56	0.00837856495180774\\
37.57	0.00837856495180774\\
37.58	0.00837856495180774\\
37.59	0.00837856495180774\\
37.6	0.00837856495180774\\
37.61	0.00837856495180774\\
37.62	0.00837856495180774\\
37.63	0.00837856495180774\\
37.64	0.00837856495180774\\
37.65	0.00837856495180774\\
37.66	0.00837856495180774\\
37.67	0.00837856495180774\\
37.68	0.00837856495180774\\
37.69	0.00837856495180774\\
37.7	0.00837856495180774\\
37.71	0.00837856495180774\\
37.72	0.00837856495180774\\
37.73	0.00837856495180774\\
37.74	0.00837856495180774\\
37.75	0.00837856495180774\\
37.76	0.00837856495180774\\
37.77	0.00837856495180774\\
37.78	0.00837856495180774\\
37.79	0.00837856495180774\\
37.8	0.00837856495180774\\
37.81	0.00837856495180774\\
37.82	0.00837856495180774\\
37.83	0.00837856495180774\\
37.84	0.00837856495180774\\
37.85	0.00837856495180774\\
37.86	0.00837856495180774\\
37.87	0.00837856495180774\\
37.88	0.00837856495180774\\
37.89	0.00837856495180774\\
37.9	0.00837856495180774\\
37.91	0.00837856495180774\\
37.92	0.00837856495180774\\
37.93	0.00837856495180774\\
37.94	0.00837856495180774\\
37.95	0.00837856495180774\\
37.96	0.00837856495180774\\
37.97	0.00837856495180774\\
37.98	0.00837856495180774\\
37.99	0.00837856495180774\\
38	0.00837856495180774\\
38.01	0.00837856495180774\\
38.02	0.00837856495180774\\
38.03	0.00837856495180774\\
38.04	0.00837856495180774\\
38.05	0.00837856495180774\\
38.06	0.00837856495180774\\
38.07	0.00837856495180774\\
38.08	0.00837856495180774\\
38.09	0.00837856495180774\\
38.1	0.00837856495180774\\
38.11	0.00837856495180774\\
38.12	0.00837856495180774\\
38.13	0.00837856495180774\\
38.14	0.00837856495180774\\
38.15	0.00837856495180774\\
38.16	0.00837856495180774\\
38.17	0.00837856495180774\\
38.18	0.00837856495180774\\
38.19	0.00837856495180774\\
38.2	0.00837856495180774\\
38.21	0.00837856495180774\\
38.22	0.00837856495180774\\
38.23	0.00837856495180774\\
38.24	0.00837856495180774\\
38.25	0.00837856495180774\\
38.26	0.00837856495180774\\
38.27	0.00837856495180774\\
38.28	0.00837856495180774\\
38.29	0.00837856495180774\\
38.3	0.00837856495180774\\
38.31	0.00837856495180774\\
38.32	0.00837856495180774\\
38.33	0.00837856495180774\\
38.34	0.00837856495180774\\
38.35	0.00837856495180774\\
38.36	0.00837856495180774\\
38.37	0.00837856495180774\\
38.38	0.00837856495180774\\
38.39	0.00837856495180774\\
38.4	0.00837856495180774\\
38.41	0.00837856495180774\\
38.42	0.00837856495180774\\
38.43	0.00837856495180774\\
38.44	0.00837856495180774\\
38.45	0.00837856495180774\\
38.46	0.00837856495180774\\
38.47	0.00837856495180774\\
38.48	0.00837856495180774\\
38.49	0.00837856495180774\\
38.5	0.00837856495180774\\
38.51	0.00837856495180774\\
38.52	0.00837856495180774\\
38.53	0.00837856495180774\\
38.54	0.00837856495180774\\
38.55	0.00837856495180774\\
38.56	0.00837856495180774\\
38.57	0.00837856495180774\\
38.58	0.00837856495180774\\
38.59	0.00837856495180774\\
38.6	0.00837856495180774\\
38.61	0.00837856495180774\\
38.62	0.00837856495180774\\
38.63	0.00837856495180774\\
38.64	0.00837856495180774\\
38.65	0.00837856495180774\\
38.66	0.00837856495180774\\
38.67	0.00837856495180774\\
38.68	0.00837856495180774\\
38.69	0.00837856495180775\\
38.7	0.00837856495180775\\
38.71	0.00837856495180775\\
38.72	0.00837856495180775\\
38.73	0.00837856495180775\\
38.74	0.00837856495180775\\
38.75	0.00837856495180775\\
38.76	0.00837856495180775\\
38.77	0.00837856495180775\\
38.78	0.00837856495180775\\
38.79	0.00837856495180775\\
38.8	0.00837856495180775\\
38.81	0.00837856495180775\\
38.82	0.00837856495180775\\
38.83	0.00837856495180775\\
38.84	0.00837856495180775\\
38.85	0.00837856495180775\\
38.86	0.00837856495180775\\
38.87	0.00837856495180775\\
38.88	0.00837856495180775\\
38.89	0.00837856495180775\\
38.9	0.00837856495180775\\
38.91	0.00837856495180775\\
38.92	0.00837856495180775\\
38.93	0.00837856495180775\\
38.94	0.00837856495180775\\
38.95	0.00837856495180775\\
38.96	0.00837856495180775\\
38.97	0.00837856495180775\\
38.98	0.00837856495180775\\
38.99	0.00837856495180775\\
39	0.00837856495180775\\
39.01	0.00837856495180775\\
39.02	0.00837856495180775\\
39.03	0.00837856495180775\\
39.04	0.00837856495180775\\
39.05	0.00837856495180775\\
39.06	0.00837856495180775\\
39.07	0.00837856495180775\\
39.08	0.00837856495180775\\
39.09	0.00837856495180775\\
39.1	0.00837856495180775\\
39.11	0.00837856495180775\\
39.12	0.00837856495180775\\
39.13	0.00837856495180775\\
39.14	0.00837856495180775\\
39.15	0.00837856495180775\\
39.16	0.00837856495180775\\
39.17	0.00837856495180776\\
39.18	0.00837856495180776\\
39.19	0.00837856495180776\\
39.2	0.00837856495180776\\
39.21	0.00837856495180776\\
39.22	0.00837856495180776\\
39.23	0.00837856495180776\\
39.24	0.00837856495180776\\
39.25	0.00837856495180776\\
39.26	0.00837856495180776\\
39.27	0.00837856495180776\\
39.28	0.00837856495180776\\
39.29	0.00837856495180776\\
39.3	0.00837856495180776\\
39.31	0.00837856495180776\\
39.32	0.00837856495180776\\
39.33	0.00837856495180776\\
39.34	0.00837856495180776\\
39.35	0.00837856495180776\\
39.36	0.00837856495180776\\
39.37	0.00837856495180776\\
39.38	0.00837856495180776\\
39.39	0.00837856495180776\\
39.4	0.00837856495180776\\
39.41	0.00837856495180776\\
39.42	0.00837856495180776\\
39.43	0.00837856495180776\\
39.44	0.00837856495180776\\
39.45	0.00837856495180776\\
39.46	0.00837856495180776\\
39.47	0.00837856495180776\\
39.48	0.00837856495180776\\
39.49	0.00837856495180776\\
39.5	0.00837856495180776\\
39.51	0.00837856495180776\\
39.52	0.00837856495180777\\
39.53	0.00837856495180777\\
39.54	0.00837856495180777\\
39.55	0.00837856495180777\\
39.56	0.00837856495180777\\
39.57	0.00837856495180777\\
39.58	0.00837856495180777\\
39.59	0.00837856495180777\\
39.6	0.00837856495180777\\
39.61	0.00837856495180777\\
39.62	0.00837856495180777\\
39.63	0.00837856495180777\\
39.64	0.00837856495180777\\
39.65	0.00837856495180777\\
39.66	0.00837856495180777\\
39.67	0.00837856495180777\\
39.68	0.00837856495180777\\
39.69	0.00837856495180777\\
39.7	0.00837856495180777\\
39.71	0.00837856495180777\\
39.72	0.00837856495180777\\
39.73	0.00837856495180777\\
39.74	0.00837856495180777\\
39.75	0.00837856495180777\\
39.76	0.00837856495180778\\
39.77	0.00837856495180778\\
39.78	0.00837856495180778\\
39.79	0.00837856495180778\\
39.8	0.00837856495180778\\
39.81	0.00837856495180778\\
39.82	0.00837856495180778\\
39.83	0.00837856495180778\\
39.84	0.00837856495180778\\
39.85	0.00837856495180778\\
39.86	0.00837856495180778\\
39.87	0.00837856495180778\\
39.88	0.00837856495180778\\
39.89	0.00837856495180778\\
39.9	0.00837856495180778\\
39.91	0.00837856495180778\\
39.92	0.00837856495180779\\
39.93	0.00837856495180779\\
39.94	0.00837856495180779\\
39.95	0.00837856495180779\\
39.96	0.00837856495180779\\
39.97	0.00837856495180779\\
39.98	0.00837856495180779\\
39.99	0.00837856495180779\\
40	0.00837856495180779\\
40.01	0.00837856495180779\\
};
\addplot [color=black,solid,forget plot]
  table[row sep=crcr]{%
40.01	0.00837856495180779\\
40.02	0.00837856495180779\\
40.03	0.00837856495180779\\
40.04	0.00837856495180779\\
40.05	0.00837856495180779\\
40.06	0.00837856495180779\\
40.07	0.00837856495180779\\
40.08	0.0083785649518078\\
40.09	0.0083785649518078\\
40.1	0.0083785649518078\\
40.11	0.0083785649518078\\
40.12	0.0083785649518078\\
40.13	0.0083785649518078\\
40.14	0.0083785649518078\\
40.15	0.0083785649518078\\
40.16	0.0083785649518078\\
40.17	0.0083785649518078\\
40.18	0.0083785649518078\\
40.19	0.0083785649518078\\
40.2	0.0083785649518078\\
40.21	0.0083785649518078\\
40.22	0.0083785649518078\\
40.23	0.00837856495180781\\
40.24	0.00837856495180781\\
40.25	0.00837856495180781\\
40.26	0.00837856495180781\\
40.27	0.00837856495180781\\
40.28	0.00837856495180781\\
40.29	0.00837856495180781\\
40.3	0.00837856495180781\\
40.31	0.00837856495180781\\
40.32	0.00837856495180781\\
40.33	0.00837856495180781\\
40.34	0.00837856495180781\\
40.35	0.00837856495180782\\
40.36	0.00837856495180782\\
40.37	0.00837856495180782\\
40.38	0.00837856495180782\\
40.39	0.00837856495180782\\
40.4	0.00837856495180782\\
40.41	0.00837856495180782\\
40.42	0.00837856495180782\\
40.43	0.00837856495180782\\
40.44	0.00837856495180783\\
40.45	0.00837856495180783\\
40.46	0.00837856495180783\\
40.47	0.00837856495180783\\
40.48	0.00837856495180783\\
40.49	0.00837856495180783\\
40.5	0.00837856495180783\\
40.51	0.00837856495180783\\
40.52	0.00837856495180783\\
40.53	0.00837856495180784\\
40.54	0.00837856495180784\\
40.55	0.00837856495180784\\
40.56	0.00837856495180784\\
40.57	0.00837856495180784\\
40.58	0.00837856495180784\\
40.59	0.00837856495180784\\
40.6	0.00837856495180784\\
40.61	0.00837856495180784\\
40.62	0.00837856495180784\\
40.63	0.00837856495180785\\
40.64	0.00837856495180785\\
40.65	0.00837856495180785\\
40.66	0.00837856495180785\\
40.67	0.00837856495180785\\
40.68	0.00837856495180785\\
40.69	0.00837856495180785\\
40.7	0.00837856495180785\\
40.71	0.00837856495180786\\
40.72	0.00837856495180786\\
40.73	0.00837856495180786\\
40.74	0.00837856495180786\\
40.75	0.00837856495180786\\
40.76	0.00837856495180786\\
40.77	0.00837856495180786\\
40.78	0.00837856495180787\\
40.79	0.00837856495180787\\
40.8	0.00837856495180787\\
40.81	0.00837856495180787\\
40.82	0.00837856495180787\\
40.83	0.00837856495180787\\
40.84	0.00837856495180788\\
40.85	0.00837856495180788\\
40.86	0.00837856495180788\\
40.87	0.00837856495180788\\
40.88	0.00837856495180788\\
40.89	0.00837856495180788\\
40.9	0.00837856495180788\\
40.91	0.00837856495180789\\
40.92	0.00837856495180789\\
40.93	0.00837856495180789\\
40.94	0.00837856495180789\\
40.95	0.00837856495180789\\
40.96	0.00837856495180789\\
40.97	0.0083785649518079\\
40.98	0.0083785649518079\\
40.99	0.0083785649518079\\
41	0.0083785649518079\\
41.01	0.0083785649518079\\
41.02	0.0083785649518079\\
41.03	0.00837856495180791\\
41.04	0.00837856495180791\\
41.05	0.00837856495180791\\
41.06	0.00837856495180791\\
41.07	0.00837856495180792\\
41.08	0.00837856495180792\\
41.09	0.00837856495180792\\
41.1	0.00837856495180792\\
41.11	0.00837856495180792\\
41.12	0.00837856495180792\\
41.13	0.00837856495180793\\
41.14	0.00837856495180793\\
41.15	0.00837856495180793\\
41.16	0.00837856495180793\\
41.17	0.00837856495180793\\
41.18	0.00837856495180794\\
41.19	0.00837856495180794\\
41.2	0.00837856495180794\\
41.21	0.00837856495180794\\
41.22	0.00837856495180795\\
41.23	0.00837856495180795\\
41.24	0.00837856495180795\\
41.25	0.00837856495180795\\
41.26	0.00837856495180796\\
41.27	0.00837856495180796\\
41.28	0.00837856495180796\\
41.29	0.00837856495180796\\
41.3	0.00837856495180797\\
41.31	0.00837856495180797\\
41.32	0.00837856495180797\\
41.33	0.00837856495180797\\
41.34	0.00837856495180797\\
41.35	0.00837856495180798\\
41.36	0.00837856495180798\\
41.37	0.00837856495180798\\
41.38	0.00837856495180798\\
41.39	0.00837856495180799\\
41.4	0.00837856495180799\\
41.41	0.00837856495180799\\
41.42	0.008378564951808\\
41.43	0.008378564951808\\
41.44	0.008378564951808\\
41.45	0.00837856495180801\\
41.46	0.00837856495180801\\
41.47	0.00837856495180801\\
41.48	0.00837856495180801\\
41.49	0.00837856495180802\\
41.5	0.00837856495180802\\
41.51	0.00837856495180802\\
41.52	0.00837856495180803\\
41.53	0.00837856495180803\\
41.54	0.00837856495180803\\
41.55	0.00837856495180804\\
41.56	0.00837856495180804\\
41.57	0.00837856495180804\\
41.58	0.00837856495180805\\
41.59	0.00837856495180805\\
41.6	0.00837856495180805\\
41.61	0.00837856495180806\\
41.62	0.00837856495180806\\
41.63	0.00837856495180806\\
41.64	0.00837856495180807\\
41.65	0.00837856495180807\\
41.66	0.00837856495180807\\
41.67	0.00837856495180808\\
41.68	0.00837856495180808\\
41.69	0.00837856495180809\\
41.7	0.00837856495180809\\
41.71	0.00837856495180809\\
41.72	0.0083785649518081\\
41.73	0.0083785649518081\\
41.74	0.0083785649518081\\
41.75	0.00837856495180811\\
41.76	0.00837856495180811\\
41.77	0.00837856495180811\\
41.78	0.00837856495180812\\
41.79	0.00837856495180812\\
41.8	0.00837856495180813\\
41.81	0.00837856495180813\\
41.82	0.00837856495180814\\
41.83	0.00837856495180814\\
41.84	0.00837856495180814\\
41.85	0.00837856495180815\\
41.86	0.00837856495180815\\
41.87	0.00837856495180816\\
41.88	0.00837856495180816\\
41.89	0.00837856495180817\\
41.9	0.00837856495180817\\
41.91	0.00837856495180818\\
41.92	0.00837856495180818\\
41.93	0.00837856495180819\\
41.94	0.00837856495180819\\
41.95	0.0083785649518082\\
41.96	0.0083785649518082\\
41.97	0.00837856495180821\\
41.98	0.00837856495180821\\
41.99	0.00837856495180822\\
42	0.00837856495180822\\
42.01	0.00837856495180823\\
42.02	0.00837856495180823\\
42.03	0.00837856495180824\\
42.04	0.00837856495180824\\
42.05	0.00837856495180825\\
42.06	0.00837856495180825\\
42.07	0.00837856495180826\\
42.08	0.00837856495180826\\
42.09	0.00837856495180827\\
42.1	0.00837856495180828\\
42.11	0.00837856495180828\\
42.12	0.00837856495180829\\
42.13	0.00837856495180829\\
42.14	0.0083785649518083\\
42.15	0.00837856495180831\\
42.16	0.00837856495180831\\
42.17	0.00837856495180832\\
42.18	0.00837856495180832\\
42.19	0.00837856495180833\\
42.2	0.00837856495180834\\
42.21	0.00837856495180834\\
42.22	0.00837856495180835\\
42.23	0.00837856495180836\\
42.24	0.00837856495180836\\
42.25	0.00837856495180837\\
42.26	0.00837856495180837\\
42.27	0.00837856495180838\\
42.28	0.00837856495180839\\
42.29	0.0083785649518084\\
42.3	0.0083785649518084\\
42.31	0.00837856495180841\\
42.32	0.00837856495180842\\
42.33	0.00837856495180843\\
42.34	0.00837856495180843\\
42.35	0.00837856495180844\\
42.36	0.00837856495180845\\
42.37	0.00837856495180845\\
42.38	0.00837856495180846\\
42.39	0.00837856495180847\\
42.4	0.00837856495180848\\
42.41	0.00837856495180849\\
42.42	0.00837856495180849\\
42.43	0.0083785649518085\\
42.44	0.00837856495180851\\
42.45	0.00837856495180852\\
42.46	0.00837856495180853\\
42.47	0.00837856495180853\\
42.48	0.00837856495180854\\
42.49	0.00837856495180855\\
42.5	0.00837856495180856\\
42.51	0.00837856495180857\\
42.52	0.00837856495180858\\
42.53	0.00837856495180859\\
42.54	0.0083785649518086\\
42.55	0.00837856495180861\\
42.56	0.00837856495180861\\
42.57	0.00837856495180862\\
42.58	0.00837856495180863\\
42.59	0.00837856495180864\\
42.6	0.00837856495180865\\
42.61	0.00837856495180866\\
42.62	0.00837856495180867\\
42.63	0.00837856495180868\\
42.64	0.00837856495180869\\
42.65	0.0083785649518087\\
42.66	0.00837856495180871\\
42.67	0.00837856495180872\\
42.68	0.00837856495180873\\
42.69	0.00837856495180874\\
42.7	0.00837856495180875\\
42.71	0.00837856495180877\\
42.72	0.00837856495180878\\
42.73	0.00837856495180879\\
42.74	0.0083785649518088\\
42.75	0.00837856495180881\\
42.76	0.00837856495180882\\
42.77	0.00837856495180883\\
42.78	0.00837856495180884\\
42.79	0.00837856495180886\\
42.8	0.00837856495180887\\
42.81	0.00837856495180888\\
42.82	0.00837856495180889\\
42.83	0.0083785649518089\\
42.84	0.00837856495180892\\
42.85	0.00837856495180893\\
42.86	0.00837856495180894\\
42.87	0.00837856495180895\\
42.88	0.00837856495180897\\
42.89	0.00837856495180898\\
42.9	0.00837856495180899\\
42.91	0.00837856495180901\\
42.92	0.00837856495180902\\
42.93	0.00837856495180903\\
42.94	0.00837856495180905\\
42.95	0.00837856495180906\\
42.96	0.00837856495180907\\
42.97	0.00837856495180909\\
42.98	0.0083785649518091\\
42.99	0.00837856495180912\\
43	0.00837856495180913\\
43.01	0.00837856495180915\\
43.02	0.00837856495180916\\
43.03	0.00837856495180918\\
43.04	0.00837856495180919\\
43.05	0.00837856495180921\\
43.06	0.00837856495180922\\
43.07	0.00837856495180924\\
43.08	0.00837856495180925\\
43.09	0.00837856495180927\\
43.1	0.00837856495180929\\
43.11	0.0083785649518093\\
43.12	0.00837856495180932\\
43.13	0.00837856495180934\\
43.14	0.00837856495180935\\
43.15	0.00837856495180937\\
43.16	0.00837856495180939\\
43.17	0.00837856495180941\\
43.18	0.00837856495180942\\
43.19	0.00837856495180944\\
43.2	0.00837856495180946\\
43.21	0.00837856495180948\\
43.22	0.0083785649518095\\
43.23	0.00837856495180951\\
43.24	0.00837856495180953\\
43.25	0.00837856495180955\\
43.26	0.00837856495180957\\
43.27	0.00837856495180959\\
43.28	0.00837856495180961\\
43.29	0.00837856495180963\\
43.3	0.00837856495180965\\
43.31	0.00837856495180967\\
43.32	0.00837856495180969\\
43.33	0.00837856495180971\\
43.34	0.00837856495180973\\
43.35	0.00837856495180975\\
43.36	0.00837856495180977\\
43.37	0.0083785649518098\\
43.38	0.00837856495180982\\
43.39	0.00837856495180984\\
43.4	0.00837856495180986\\
43.41	0.00837856495180988\\
43.42	0.00837856495180991\\
43.43	0.00837856495180993\\
43.44	0.00837856495180995\\
43.45	0.00837856495180997\\
43.46	0.00837856495181\\
43.47	0.00837856495181002\\
43.48	0.00837856495181005\\
43.49	0.00837856495181007\\
43.5	0.0083785649518101\\
43.51	0.00837856495181012\\
43.52	0.00837856495181015\\
43.53	0.00837856495181017\\
43.54	0.0083785649518102\\
43.55	0.00837856495181022\\
43.56	0.00837856495181025\\
43.57	0.00837856495181027\\
43.58	0.0083785649518103\\
43.59	0.00837856495181033\\
43.6	0.00837856495181036\\
43.61	0.00837856495181038\\
43.62	0.00837856495181041\\
43.63	0.00837856495181044\\
43.64	0.00837856495181047\\
43.65	0.0083785649518105\\
43.66	0.00837856495181053\\
43.67	0.00837856495181056\\
43.68	0.00837856495181059\\
43.69	0.00837856495181061\\
43.7	0.00837856495181065\\
43.71	0.00837856495181068\\
43.72	0.00837856495181071\\
43.73	0.00837856495181074\\
43.74	0.00837856495181077\\
43.75	0.0083785649518108\\
43.76	0.00837856495181083\\
43.77	0.00837856495181087\\
43.78	0.0083785649518109\\
43.79	0.00837856495181093\\
43.8	0.00837856495181097\\
43.81	0.008378564951811\\
43.82	0.00837856495181103\\
43.83	0.00837856495181107\\
43.84	0.0083785649518111\\
43.85	0.00837856495181114\\
43.86	0.00837856495181118\\
43.87	0.00837856495181121\\
43.88	0.00837856495181125\\
43.89	0.00837856495181129\\
43.9	0.00837856495181132\\
43.91	0.00837856495181136\\
43.92	0.0083785649518114\\
43.93	0.00837856495181144\\
43.94	0.00837856495181147\\
43.95	0.00837856495181151\\
43.96	0.00837856495181155\\
43.97	0.00837856495181159\\
43.98	0.00837856495181163\\
43.99	0.00837856495181168\\
44	0.00837856495181172\\
44.01	0.00837856495181176\\
44.02	0.0083785649518118\\
44.03	0.00837856495181184\\
44.04	0.00837856495181189\\
44.05	0.00837856495181193\\
44.06	0.00837856495181197\\
44.07	0.00837856495181202\\
44.08	0.00837856495181206\\
44.09	0.00837856495181211\\
44.1	0.00837856495181216\\
44.11	0.0083785649518122\\
44.12	0.00837856495181225\\
44.13	0.0083785649518123\\
44.14	0.00837856495181234\\
44.15	0.00837856495181239\\
44.16	0.00837856495181244\\
44.17	0.00837856495181249\\
44.18	0.00837856495181254\\
44.19	0.00837856495181259\\
44.2	0.00837856495181264\\
44.21	0.00837856495181269\\
44.22	0.00837856495181275\\
44.23	0.0083785649518128\\
44.24	0.00837856495181285\\
44.25	0.0083785649518129\\
44.26	0.00837856495181296\\
44.27	0.00837856495181301\\
44.28	0.00837856495181307\\
44.29	0.00837856495181313\\
44.3	0.00837856495181318\\
44.31	0.00837856495181324\\
44.32	0.0083785649518133\\
44.33	0.00837856495181336\\
44.34	0.00837856495181341\\
44.35	0.00837856495181347\\
44.36	0.00837856495181353\\
44.37	0.00837856495181359\\
44.38	0.00837856495181366\\
44.39	0.00837856495181372\\
44.4	0.00837856495181378\\
44.41	0.00837856495181384\\
44.42	0.00837856495181391\\
44.43	0.00837856495181397\\
44.44	0.00837856495181404\\
44.45	0.0083785649518141\\
44.46	0.00837856495181417\\
44.47	0.00837856495181424\\
44.48	0.00837856495181431\\
44.49	0.00837856495181438\\
44.5	0.00837856495181444\\
44.51	0.00837856495181451\\
44.52	0.00837856495181459\\
44.53	0.00837856495181466\\
44.54	0.00837856495181473\\
44.55	0.0083785649518148\\
44.56	0.00837856495181488\\
44.57	0.00837856495181495\\
44.58	0.00837856495181503\\
44.59	0.0083785649518151\\
44.6	0.00837856495181518\\
44.61	0.00837856495181526\\
44.62	0.00837856495181534\\
44.63	0.00837856495181542\\
44.64	0.0083785649518155\\
44.65	0.00837856495181558\\
44.66	0.00837856495181566\\
44.67	0.00837856495181575\\
44.68	0.00837856495181583\\
44.69	0.00837856495181591\\
44.7	0.008378564951816\\
44.71	0.00837856495181609\\
44.72	0.00837856495181617\\
44.73	0.00837856495181626\\
44.74	0.00837856495181635\\
44.75	0.00837856495181644\\
44.76	0.00837856495181653\\
44.77	0.00837856495181662\\
44.78	0.00837856495181672\\
44.79	0.00837856495181681\\
44.8	0.00837856495181691\\
44.81	0.008378564951817\\
44.82	0.0083785649518171\\
44.83	0.0083785649518172\\
44.84	0.00837856495181729\\
44.85	0.0083785649518174\\
44.86	0.0083785649518175\\
44.87	0.0083785649518176\\
44.88	0.0083785649518177\\
44.89	0.00837856495181781\\
44.9	0.00837856495181791\\
44.91	0.00837856495181802\\
44.92	0.00837856495181813\\
44.93	0.00837856495181823\\
44.94	0.00837856495181834\\
44.95	0.00837856495181846\\
44.96	0.00837856495181857\\
44.97	0.00837856495181868\\
44.98	0.00837856495181879\\
44.99	0.00837856495181891\\
45	0.00837856495181903\\
45.01	0.00837856495181914\\
45.02	0.00837856495181926\\
45.03	0.00837856495181939\\
45.04	0.00837856495181951\\
45.05	0.00837856495181963\\
45.06	0.00837856495181975\\
45.07	0.00837856495181988\\
45.08	0.00837856495182001\\
45.09	0.00837856495182013\\
45.1	0.00837856495182026\\
45.11	0.00837856495182039\\
45.12	0.00837856495182053\\
45.13	0.00837856495182066\\
45.14	0.0083785649518208\\
45.15	0.00837856495182093\\
45.16	0.00837856495182107\\
45.17	0.00837856495182121\\
45.18	0.00837856495182135\\
45.19	0.00837856495182149\\
45.2	0.00837856495182164\\
45.21	0.00837856495182178\\
45.22	0.00837856495182193\\
45.23	0.00837856495182208\\
45.24	0.00837856495182223\\
45.25	0.00837856495182238\\
45.26	0.00837856495182253\\
45.27	0.00837856495182268\\
45.28	0.00837856495182284\\
45.29	0.008378564951823\\
45.3	0.00837856495182316\\
45.31	0.00837856495182332\\
45.32	0.00837856495182348\\
45.33	0.00837856495182364\\
45.34	0.00837856495182381\\
45.35	0.00837856495182398\\
45.36	0.00837856495182415\\
45.37	0.00837856495182432\\
45.38	0.00837856495182449\\
45.39	0.00837856495182467\\
45.4	0.00837856495182484\\
45.41	0.00837856495182502\\
45.42	0.0083785649518252\\
45.43	0.00837856495182538\\
45.44	0.00837856495182557\\
45.45	0.00837856495182576\\
45.46	0.00837856495182594\\
45.47	0.00837856495182613\\
45.48	0.00837856495182632\\
45.49	0.00837856495182652\\
45.5	0.00837856495182671\\
45.51	0.00837856495182691\\
45.52	0.00837856495182711\\
45.53	0.00837856495182731\\
45.54	0.00837856495182752\\
45.55	0.00837856495182772\\
45.56	0.00837856495182793\\
45.57	0.00837856495182814\\
45.58	0.00837856495182836\\
45.59	0.00837856495182857\\
45.6	0.00837856495182879\\
45.61	0.00837856495182901\\
45.62	0.00837856495182923\\
45.63	0.00837856495182945\\
45.64	0.00837856495182968\\
45.65	0.00837856495182991\\
45.66	0.00837856495183014\\
45.67	0.00837856495183037\\
45.68	0.00837856495183061\\
45.69	0.00837856495183085\\
45.7	0.00837856495183109\\
45.71	0.00837856495183133\\
45.72	0.00837856495183158\\
45.73	0.00837856495183182\\
45.74	0.00837856495183208\\
45.75	0.00837856495183233\\
45.76	0.00837856495183258\\
45.77	0.00837856495183284\\
45.78	0.00837856495183311\\
45.79	0.00837856495183337\\
45.8	0.00837856495183364\\
45.81	0.00837856495183391\\
45.82	0.00837856495183418\\
45.83	0.00837856495183445\\
45.84	0.00837856495183473\\
45.85	0.00837856495183501\\
45.86	0.0083785649518353\\
45.87	0.00837856495183558\\
45.88	0.00837856495183587\\
45.89	0.00837856495183617\\
45.9	0.00837856495183646\\
45.91	0.00837856495183676\\
45.92	0.00837856495183706\\
45.93	0.00837856495183737\\
45.94	0.00837856495183768\\
45.95	0.00837856495183799\\
45.96	0.0083785649518383\\
45.97	0.00837856495183862\\
45.98	0.00837856495183894\\
45.99	0.00837856495183927\\
46	0.0083785649518396\\
46.01	0.00837856495183993\\
46.02	0.00837856495184026\\
46.03	0.0083785649518406\\
46.04	0.00837856495184094\\
46.05	0.00837856495184129\\
46.06	0.00837856495184164\\
46.07	0.00837856495184199\\
46.08	0.00837856495184235\\
46.09	0.00837856495184271\\
46.1	0.00837856495184307\\
46.11	0.00837856495184344\\
46.12	0.00837856495184381\\
46.13	0.00837856495184418\\
46.14	0.00837856495184456\\
46.15	0.00837856495184494\\
46.16	0.00837856495184533\\
46.17	0.00837856495184572\\
46.18	0.00837856495184612\\
46.19	0.00837856495184651\\
46.2	0.00837856495184692\\
46.21	0.00837856495184732\\
46.22	0.00837856495184774\\
46.23	0.00837856495184815\\
46.24	0.00837856495184857\\
46.25	0.008378564951849\\
46.26	0.00837856495184943\\
46.27	0.00837856495184986\\
46.28	0.0083785649518503\\
46.29	0.00837856495185074\\
46.3	0.00837856495185118\\
46.31	0.00837856495185164\\
46.32	0.00837856495185209\\
46.33	0.00837856495185255\\
46.34	0.00837856495185302\\
46.35	0.00837856495185349\\
46.36	0.00837856495185396\\
46.37	0.00837856495185444\\
46.38	0.00837856495185493\\
46.39	0.00837856495185542\\
46.4	0.00837856495185591\\
46.41	0.00837856495185641\\
46.42	0.00837856495185692\\
46.43	0.00837856495185743\\
46.44	0.00837856495185794\\
46.45	0.00837856495185847\\
46.46	0.00837856495185899\\
46.47	0.00837856495185952\\
46.48	0.00837856495186006\\
46.49	0.0083785649518606\\
46.5	0.00837856495186115\\
46.51	0.00837856495186171\\
46.52	0.00837856495186227\\
46.53	0.00837856495186283\\
46.54	0.0083785649518634\\
46.55	0.00837856495186398\\
46.56	0.00837856495186456\\
46.57	0.00837856495186515\\
46.58	0.00837856495186575\\
46.59	0.00837856495186635\\
46.6	0.00837856495186696\\
46.61	0.00837856495186757\\
46.62	0.00837856495186819\\
46.63	0.00837856495186882\\
46.64	0.00837856495186945\\
46.65	0.00837856495187009\\
46.66	0.00837856495187074\\
46.67	0.00837856495187139\\
46.68	0.00837856495187205\\
46.69	0.00837856495187272\\
46.7	0.00837856495187339\\
46.71	0.00837856495187407\\
46.72	0.00837856495187476\\
46.73	0.00837856495187545\\
46.74	0.00837856495187616\\
46.75	0.00837856495187687\\
46.76	0.00837856495187758\\
46.77	0.00837856495187831\\
46.78	0.00837856495187904\\
46.79	0.00837856495187977\\
46.8	0.00837856495188052\\
46.81	0.00837856495188127\\
46.82	0.00837856495188204\\
46.83	0.00837856495188281\\
46.84	0.00837856495188358\\
46.85	0.00837856495188437\\
46.86	0.00837856495188516\\
46.87	0.00837856495188596\\
46.88	0.00837856495188677\\
46.89	0.00837856495188759\\
46.9	0.00837856495188842\\
46.91	0.00837856495188925\\
46.92	0.0083785649518901\\
46.93	0.00837856495189095\\
46.94	0.00837856495189181\\
46.95	0.00837856495189268\\
46.96	0.00837856495189356\\
46.97	0.00837856495189445\\
46.98	0.00837856495189535\\
46.99	0.00837856495189625\\
47	0.00837856495189717\\
47.01	0.00837856495189809\\
47.02	0.00837856495189903\\
47.03	0.00837856495189998\\
47.04	0.00837856495190093\\
47.05	0.00837856495190189\\
47.06	0.00837856495190287\\
47.07	0.00837856495190385\\
47.08	0.00837856495190484\\
47.09	0.00837856495190585\\
47.1	0.00837856495190686\\
47.11	0.00837856495190789\\
47.12	0.00837856495190892\\
47.13	0.00837856495190997\\
47.14	0.00837856495191103\\
47.15	0.0083785649519121\\
47.16	0.00837856495191317\\
47.17	0.00837856495191427\\
47.18	0.00837856495191537\\
47.19	0.00837856495191648\\
47.2	0.0083785649519176\\
47.21	0.00837856495191874\\
47.22	0.00837856495191988\\
47.23	0.00837856495192105\\
47.24	0.00837856495192222\\
47.25	0.0083785649519234\\
47.26	0.00837856495192459\\
47.27	0.0083785649519258\\
47.28	0.00837856495192702\\
47.29	0.00837856495192825\\
47.3	0.0083785649519295\\
47.31	0.00837856495193075\\
47.32	0.00837856495193203\\
47.33	0.00837856495193331\\
47.34	0.00837856495193461\\
47.35	0.00837856495193592\\
47.36	0.00837856495193724\\
47.37	0.00837856495193858\\
47.38	0.00837856495193993\\
47.39	0.00837856495194129\\
47.4	0.00837856495194267\\
47.41	0.00837856495194407\\
47.42	0.00837856495194547\\
47.43	0.00837856495194689\\
47.44	0.00837856495194833\\
47.45	0.00837856495194978\\
47.46	0.00837856495195125\\
47.47	0.00837856495195273\\
47.48	0.00837856495195423\\
47.49	0.00837856495195574\\
47.5	0.00837856495195726\\
47.51	0.0083785649519588\\
47.52	0.00837856495196036\\
47.53	0.00837856495196194\\
47.54	0.00837856495196353\\
47.55	0.00837856495196514\\
47.56	0.00837856495196676\\
47.57	0.0083785649519684\\
47.58	0.00837856495197006\\
47.59	0.00837856495197173\\
47.6	0.00837856495197342\\
47.61	0.00837856495197513\\
47.62	0.00837856495197685\\
47.63	0.0083785649519786\\
47.64	0.00837856495198036\\
47.65	0.00837856495198214\\
47.66	0.00837856495198394\\
47.67	0.00837856495198575\\
47.68	0.00837856495198759\\
47.69	0.00837856495198944\\
47.7	0.00837856495199131\\
47.71	0.00837856495199321\\
47.72	0.00837856495199512\\
47.73	0.00837856495199705\\
47.74	0.008378564951999\\
47.75	0.00837856495200097\\
47.76	0.00837856495200296\\
47.77	0.00837856495200497\\
47.78	0.008378564952007\\
47.79	0.00837856495200905\\
47.8	0.00837856495201113\\
47.81	0.00837856495201322\\
47.82	0.00837856495201534\\
47.83	0.00837856495201747\\
47.84	0.00837856495201963\\
47.85	0.00837856495202182\\
47.86	0.00837856495202402\\
47.87	0.00837856495202625\\
47.88	0.0083785649520285\\
47.89	0.00837856495203077\\
47.9	0.00837856495203306\\
47.91	0.00837856495203538\\
47.92	0.00837856495203772\\
47.93	0.00837856495204009\\
47.94	0.00837856495204248\\
47.95	0.0083785649520449\\
47.96	0.00837856495204734\\
47.97	0.0083785649520498\\
47.98	0.00837856495205229\\
47.99	0.00837856495205481\\
48	0.00837856495205735\\
48.01	0.00837856495205992\\
48.02	0.00837856495206251\\
48.03	0.00837856495206513\\
48.04	0.00837856495206778\\
48.05	0.00837856495207045\\
48.06	0.00837856495207315\\
48.07	0.00837856495207588\\
48.08	0.00837856495207864\\
48.09	0.00837856495208142\\
48.1	0.00837856495208423\\
48.11	0.00837856495208708\\
48.12	0.00837856495208995\\
48.13	0.00837856495209285\\
48.14	0.00837856495209578\\
48.15	0.00837856495209874\\
48.16	0.00837856495210173\\
48.17	0.00837856495210475\\
48.18	0.0083785649521078\\
48.19	0.00837856495211088\\
48.2	0.00837856495211399\\
48.21	0.00837856495211714\\
48.22	0.00837856495212032\\
48.23	0.00837856495212353\\
48.24	0.00837856495212677\\
48.25	0.00837856495213005\\
48.26	0.00837856495213336\\
48.27	0.0083785649521367\\
48.28	0.00837856495214008\\
48.29	0.00837856495214349\\
48.3	0.00837856495214694\\
48.31	0.00837856495215042\\
48.32	0.00837856495215393\\
48.33	0.00837856495215749\\
48.34	0.00837856495216108\\
48.35	0.0083785649521647\\
48.36	0.00837856495216836\\
48.37	0.00837856495217206\\
48.38	0.0083785649521758\\
48.39	0.00837856495217958\\
48.4	0.00837856495218339\\
48.41	0.00837856495218725\\
48.42	0.00837856495219114\\
48.43	0.00837856495219507\\
48.44	0.00837856495219904\\
48.45	0.00837856495220306\\
48.46	0.00837856495220711\\
48.47	0.0083785649522112\\
48.48	0.00837856495221534\\
48.49	0.00837856495221952\\
48.5	0.00837856495222374\\
48.51	0.008378564952228\\
48.52	0.00837856495223231\\
48.53	0.00837856495223666\\
48.54	0.00837856495224106\\
48.55	0.0083785649522455\\
48.56	0.00837856495224999\\
48.57	0.00837856495225452\\
48.58	0.00837856495225909\\
48.59	0.00837856495226372\\
48.6	0.00837856495226839\\
48.61	0.00837856495227311\\
48.62	0.00837856495227787\\
48.63	0.00837856495228269\\
48.64	0.00837856495228755\\
48.65	0.00837856495229247\\
48.66	0.00837856495229743\\
48.67	0.00837856495230244\\
48.68	0.00837856495230751\\
48.69	0.00837856495231262\\
48.7	0.00837856495231779\\
48.71	0.00837856495232301\\
48.72	0.00837856495232829\\
48.73	0.00837856495233361\\
48.74	0.008378564952339\\
48.75	0.00837856495234443\\
48.76	0.00837856495234992\\
48.77	0.00837856495235547\\
48.78	0.00837856495236108\\
48.79	0.00837856495236674\\
48.8	0.00837856495237245\\
48.81	0.00837856495237823\\
48.82	0.00837856495238407\\
48.83	0.00837856495238996\\
48.84	0.00837856495239591\\
48.85	0.00837856495240193\\
48.86	0.008378564952408\\
48.87	0.00837856495241414\\
48.88	0.00837856495242034\\
48.89	0.0083785649524266\\
48.9	0.00837856495243293\\
48.91	0.00837856495243932\\
48.92	0.00837856495244577\\
48.93	0.00837856495245229\\
48.94	0.00837856495245888\\
48.95	0.00837856495246553\\
48.96	0.00837856495247225\\
48.97	0.00837856495247904\\
48.98	0.0083785649524859\\
48.99	0.00837856495249283\\
49	0.00837856495249982\\
49.01	0.00837856495250689\\
49.02	0.00837856495251403\\
49.03	0.00837856495252124\\
49.04	0.00837856495252853\\
49.05	0.00837856495253589\\
49.06	0.00837856495254332\\
49.07	0.00837856495255083\\
49.08	0.00837856495255842\\
49.09	0.00837856495256608\\
49.1	0.00837856495257382\\
49.11	0.00837856495258164\\
49.12	0.00837856495258953\\
49.13	0.00837856495259751\\
49.14	0.00837856495260557\\
49.15	0.00837856495261371\\
49.16	0.00837856495262193\\
49.17	0.00837856495263024\\
49.18	0.00837856495263863\\
49.19	0.0083785649526471\\
49.2	0.00837856495265566\\
49.21	0.00837856495266431\\
49.22	0.00837856495267304\\
49.23	0.00837856495268186\\
49.24	0.00837856495269078\\
49.25	0.00837856495269978\\
49.26	0.00837856495270887\\
49.27	0.00837856495271806\\
49.28	0.00837856495272733\\
49.29	0.0083785649527367\\
49.3	0.00837856495274617\\
49.31	0.00837856495275573\\
49.32	0.00837856495276539\\
49.33	0.00837856495277515\\
49.34	0.008378564952785\\
49.35	0.00837856495279496\\
49.36	0.00837856495280501\\
49.37	0.00837856495281517\\
49.38	0.00837856495282543\\
49.39	0.00837856495283579\\
49.4	0.00837856495284626\\
49.41	0.00837856495285683\\
49.42	0.00837856495286751\\
49.43	0.0083785649528783\\
49.44	0.0083785649528892\\
49.45	0.0083785649529002\\
49.46	0.00837856495291132\\
49.47	0.00837856495292255\\
49.48	0.0083785649529339\\
49.49	0.00837856495294536\\
49.5	0.00837856495295693\\
49.51	0.00837856495296862\\
49.52	0.00837856495298043\\
49.53	0.00837856495299236\\
49.54	0.0083785649530044\\
49.55	0.00837856495301657\\
49.56	0.00837856495302887\\
49.57	0.00837856495304128\\
49.58	0.00837856495305382\\
49.59	0.00837856495306649\\
49.6	0.00837856495307929\\
49.61	0.00837856495309221\\
49.62	0.00837856495310527\\
49.63	0.00837856495311845\\
49.64	0.00837856495313177\\
49.65	0.00837856495314523\\
49.66	0.00837856495315882\\
49.67	0.00837856495317254\\
49.68	0.00837856495318641\\
49.69	0.00837856495320041\\
49.7	0.00837856495321456\\
49.71	0.00837856495322884\\
49.72	0.00837856495324328\\
49.73	0.00837856495325785\\
49.74	0.00837856495327258\\
49.75	0.00837856495328745\\
49.76	0.00837856495330247\\
49.77	0.00837856495331764\\
49.78	0.00837856495333297\\
49.79	0.00837856495334845\\
49.8	0.00837856495336408\\
49.81	0.00837856495337987\\
49.82	0.00837856495339583\\
49.83	0.00837856495341194\\
49.84	0.00837856495342821\\
49.85	0.00837856495344465\\
49.86	0.00837856495346125\\
49.87	0.00837856495347802\\
49.88	0.00837856495349496\\
49.89	0.00837856495351207\\
49.9	0.00837856495352935\\
49.91	0.0083785649535468\\
49.92	0.00837856495356443\\
49.93	0.00837856495358224\\
49.94	0.00837856495360023\\
49.95	0.00837856495361839\\
49.96	0.00837856495363674\\
49.97	0.00837856495365527\\
49.98	0.008378564953674\\
49.99	0.0083785649536929\\
50	0.008378564953712\\
50.01	0.00837856495373129\\
50.02	0.00837856495375077\\
50.03	0.00837856495377045\\
50.04	0.00837856495379032\\
50.05	0.0083785649538104\\
50.06	0.00837856495383068\\
50.07	0.00837856495385116\\
50.08	0.00837856495387184\\
50.09	0.00837856495389274\\
50.1	0.00837856495391384\\
50.11	0.00837856495393515\\
50.12	0.00837856495395668\\
50.13	0.00837856495397842\\
50.14	0.00837856495400039\\
50.15	0.00837856495402257\\
50.16	0.00837856495404497\\
50.17	0.0083785649540676\\
50.18	0.00837856495409046\\
50.19	0.00837856495411354\\
50.2	0.00837856495413686\\
50.21	0.00837856495416041\\
50.22	0.00837856495418419\\
50.23	0.00837856495420821\\
50.24	0.00837856495423248\\
50.25	0.00837856495425699\\
50.26	0.00837856495428174\\
50.27	0.00837856495430674\\
50.28	0.00837856495433199\\
50.29	0.00837856495435749\\
50.3	0.00837856495438325\\
50.31	0.00837856495440927\\
50.32	0.00837856495443554\\
50.33	0.00837856495446208\\
50.34	0.00837856495448889\\
50.35	0.00837856495451596\\
50.36	0.00837856495454331\\
50.37	0.00837856495457092\\
50.38	0.00837856495459882\\
50.39	0.00837856495462699\\
50.4	0.00837856495465544\\
50.41	0.00837856495468418\\
50.42	0.00837856495471321\\
50.43	0.00837856495474252\\
50.44	0.00837856495477213\\
50.45	0.00837856495480204\\
50.46	0.00837856495483224\\
50.47	0.00837856495486275\\
50.48	0.00837856495489355\\
50.49	0.00837856495492467\\
50.5	0.0083785649549561\\
50.51	0.00837856495498784\\
50.52	0.0083785649550199\\
50.53	0.00837856495505228\\
50.54	0.00837856495508498\\
50.55	0.00837856495511801\\
50.56	0.00837856495515137\\
50.57	0.00837856495518506\\
50.58	0.00837856495521908\\
50.59	0.00837856495525345\\
50.6	0.00837856495528816\\
50.61	0.00837856495532321\\
50.62	0.00837856495535862\\
50.63	0.00837856495539437\\
50.64	0.00837856495543049\\
50.65	0.00837856495546696\\
50.66	0.0083785649555038\\
50.67	0.008378564955541\\
50.68	0.00837856495557858\\
50.69	0.00837856495561653\\
50.7	0.00837856495565486\\
50.71	0.00837856495569357\\
50.72	0.00837856495573266\\
50.73	0.00837856495577214\\
50.74	0.00837856495581202\\
50.75	0.00837856495585229\\
50.76	0.00837856495589297\\
50.77	0.00837856495593405\\
50.78	0.00837856495597554\\
50.79	0.00837856495601744\\
50.8	0.00837856495605976\\
50.81	0.0083785649561025\\
50.82	0.00837856495614566\\
50.83	0.00837856495618926\\
50.84	0.00837856495623329\\
50.85	0.00837856495627775\\
50.86	0.00837856495632266\\
50.87	0.00837856495636801\\
50.88	0.00837856495641381\\
50.89	0.00837856495646007\\
50.9	0.00837856495650679\\
50.91	0.00837856495655398\\
50.92	0.00837856495660163\\
50.93	0.00837856495664975\\
50.94	0.00837856495669836\\
50.95	0.00837856495674744\\
50.96	0.00837856495679702\\
50.97	0.00837856495684708\\
50.98	0.00837856495689765\\
50.99	0.00837856495694871\\
51	0.00837856495700028\\
51.01	0.00837856495705237\\
51.02	0.00837856495710496\\
51.03	0.00837856495715809\\
51.04	0.00837856495721174\\
51.05	0.00837856495726592\\
51.06	0.00837856495732063\\
51.07	0.00837856495737589\\
51.08	0.0083785649574317\\
51.09	0.00837856495748806\\
51.1	0.00837856495754498\\
51.11	0.00837856495760247\\
51.12	0.00837856495766052\\
51.13	0.00837856495771915\\
51.14	0.00837856495777836\\
51.15	0.00837856495783816\\
51.16	0.00837856495789855\\
51.17	0.00837856495795953\\
51.18	0.00837856495802112\\
51.19	0.00837856495808332\\
51.2	0.00837856495814614\\
51.21	0.00837856495820957\\
51.22	0.00837856495827364\\
51.23	0.00837856495833834\\
51.24	0.00837856495840368\\
51.25	0.00837856495846966\\
51.26	0.0083785649585363\\
51.27	0.0083785649586036\\
51.28	0.00837856495867156\\
51.29	0.00837856495874019\\
51.3	0.0083785649588095\\
51.31	0.0083785649588795\\
51.32	0.00837856495895019\\
51.33	0.00837856495902158\\
51.34	0.00837856495909367\\
51.35	0.00837856495916648\\
51.36	0.00837856495924\\
51.37	0.00837856495931425\\
51.38	0.00837856495938923\\
51.39	0.00837856495946496\\
51.4	0.00837856495954143\\
51.41	0.00837856495961865\\
51.42	0.00837856495969664\\
51.43	0.0083785649597754\\
51.44	0.00837856495985493\\
51.45	0.00837856495993525\\
51.46	0.00837856496001636\\
51.47	0.00837856496009827\\
51.48	0.00837856496018099\\
51.49	0.00837856496026452\\
51.5	0.00837856496034888\\
51.51	0.00837856496043407\\
51.52	0.0083785649605201\\
51.53	0.00837856496060697\\
51.54	0.0083785649606947\\
51.55	0.0083785649607833\\
51.56	0.00837856496087277\\
51.57	0.00837856496096312\\
51.58	0.00837856496105436\\
51.59	0.00837856496114649\\
51.6	0.00837856496123954\\
51.61	0.0083785649613335\\
51.62	0.00837856496142838\\
51.63	0.0083785649615242\\
51.64	0.00837856496162096\\
51.65	0.00837856496171867\\
51.66	0.00837856496181734\\
51.67	0.00837856496191698\\
51.68	0.00837856496201761\\
51.69	0.00837856496211922\\
51.7	0.00837856496222183\\
51.71	0.00837856496232545\\
51.72	0.00837856496243008\\
51.73	0.00837856496253575\\
51.74	0.00837856496264245\\
51.75	0.0083785649627502\\
51.76	0.00837856496285901\\
51.77	0.00837856496296888\\
51.78	0.00837856496307984\\
51.79	0.00837856496319189\\
51.8	0.00837856496330503\\
51.81	0.00837856496341928\\
51.82	0.00837856496353466\\
51.83	0.00837856496365116\\
51.84	0.00837856496376881\\
51.85	0.00837856496388762\\
51.86	0.00837856496400758\\
51.87	0.00837856496412873\\
51.88	0.00837856496425106\\
51.89	0.00837856496437459\\
51.9	0.00837856496449933\\
51.91	0.00837856496462529\\
51.92	0.00837856496475249\\
51.93	0.00837856496488093\\
51.94	0.00837856496501063\\
51.95	0.0083785649651416\\
51.96	0.00837856496527385\\
51.97	0.0083785649654074\\
51.98	0.00837856496554225\\
51.99	0.00837856496567842\\
52	0.00837856496581593\\
52.01	0.00837856496595478\\
52.02	0.00837856496609498\\
52.03	0.00837856496623656\\
52.04	0.00837856496637952\\
52.05	0.00837856496652388\\
52.06	0.00837856496666966\\
52.07	0.00837856496681685\\
52.08	0.00837856496696548\\
52.09	0.00837856496711557\\
52.1	0.00837856496726712\\
52.11	0.00837856496742015\\
52.12	0.00837856496757468\\
52.13	0.00837856496773071\\
52.14	0.00837856496788827\\
52.15	0.00837856496804736\\
52.16	0.00837856496820801\\
52.17	0.00837856496837023\\
52.18	0.00837856496853402\\
52.19	0.00837856496869942\\
52.2	0.00837856496886642\\
52.21	0.00837856496903506\\
52.22	0.00837856496920534\\
52.23	0.00837856496937727\\
52.24	0.00837856496955089\\
52.25	0.00837856496972619\\
52.26	0.00837856496990321\\
52.27	0.00837856497008195\\
52.28	0.00837856497026242\\
52.29	0.00837856497044466\\
52.3	0.00837856497062867\\
52.31	0.00837856497081447\\
52.32	0.00837856497100207\\
52.33	0.00837856497119151\\
52.34	0.00837856497138278\\
52.35	0.00837856497157592\\
52.36	0.00837856497177093\\
52.37	0.00837856497196784\\
52.38	0.00837856497216667\\
52.39	0.00837856497236743\\
52.4	0.00837856497257014\\
52.41	0.00837856497277482\\
52.42	0.00837856497298148\\
52.43	0.00837856497319016\\
52.44	0.00837856497340085\\
52.45	0.0083785649736136\\
52.46	0.00837856497382841\\
52.47	0.00837856497404531\\
52.48	0.00837856497426431\\
52.49	0.00837856497448543\\
52.5	0.0083785649747087\\
52.51	0.00837856497493414\\
52.52	0.00837856497516176\\
52.53	0.00837856497539159\\
52.54	0.00837856497562365\\
52.55	0.00837856497585796\\
52.56	0.00837856497609453\\
52.57	0.0083785649763334\\
52.58	0.00837856497657459\\
52.59	0.0083785649768181\\
52.6	0.00837856497706398\\
52.61	0.00837856497731224\\
52.62	0.0083785649775629\\
52.63	0.00837856497781599\\
52.64	0.00837856497807152\\
52.65	0.00837856497832953\\
52.66	0.00837856497859003\\
52.67	0.00837856497885305\\
52.68	0.00837856497911862\\
52.69	0.00837856497938675\\
52.7	0.00837856497965747\\
52.71	0.00837856497993082\\
52.72	0.0083785649802068\\
52.73	0.00837856498048544\\
52.74	0.00837856498076678\\
52.75	0.00837856498105084\\
52.76	0.00837856498133763\\
52.77	0.0083785649816272\\
52.78	0.00837856498191956\\
52.79	0.00837856498221474\\
52.8	0.00837856498251277\\
52.81	0.00837856498281367\\
52.82	0.00837856498311748\\
52.83	0.00837856498342421\\
52.84	0.0083785649837339\\
52.85	0.00837856498404658\\
52.86	0.00837856498436227\\
52.87	0.008378564984681\\
52.88	0.0083785649850028\\
52.89	0.0083785649853277\\
52.9	0.00837856498565573\\
52.91	0.00837856498598692\\
52.92	0.00837856498632129\\
52.93	0.00837856498665888\\
52.94	0.00837856498699972\\
52.95	0.00837856498734383\\
52.96	0.00837856498769126\\
52.97	0.00837856498804203\\
52.98	0.00837856498839617\\
52.99	0.00837856498875371\\
53	0.00837856498911469\\
53.01	0.00837856498947914\\
53.02	0.00837856498984708\\
53.03	0.00837856499021856\\
53.04	0.00837856499059361\\
53.05	0.00837856499097226\\
53.06	0.00837856499135454\\
53.07	0.00837856499174049\\
53.08	0.00837856499213015\\
53.09	0.00837856499252354\\
53.1	0.0083785649929207\\
53.11	0.00837856499332167\\
53.12	0.00837856499372649\\
53.13	0.00837856499413518\\
53.14	0.00837856499454779\\
53.15	0.00837856499496436\\
53.16	0.00837856499538491\\
53.17	0.00837856499580949\\
53.18	0.00837856499623814\\
53.19	0.00837856499667088\\
53.2	0.00837856499710777\\
53.21	0.00837856499754884\\
53.22	0.00837856499799413\\
53.23	0.00837856499844368\\
53.24	0.00837856499889753\\
53.25	0.00837856499935571\\
53.26	0.00837856499981827\\
53.27	0.00837856500028526\\
53.28	0.0083785650007567\\
53.29	0.00837856500123265\\
53.3	0.00837856500171315\\
53.31	0.00837856500219823\\
53.32	0.00837856500268794\\
53.33	0.00837856500318232\\
53.34	0.00837856500368143\\
53.35	0.00837856500418529\\
53.36	0.00837856500469396\\
53.37	0.00837856500520747\\
53.38	0.00837856500572589\\
53.39	0.00837856500624924\\
53.4	0.00837856500677758\\
53.41	0.00837856500731096\\
53.42	0.00837856500784941\\
53.43	0.008378565008393\\
53.44	0.00837856500894175\\
53.45	0.00837856500949573\\
53.46	0.00837856501005498\\
53.47	0.00837856501061955\\
53.48	0.0083785650111895\\
53.49	0.00837856501176486\\
53.5	0.00837856501234569\\
53.51	0.00837856501293205\\
53.52	0.00837856501352397\\
53.53	0.00837856501412152\\
53.54	0.00837856501472474\\
53.55	0.0083785650153337\\
53.56	0.00837856501594843\\
53.57	0.008378565016569\\
53.58	0.00837856501719546\\
53.59	0.00837856501782786\\
53.6	0.00837856501846626\\
53.61	0.00837856501911071\\
53.62	0.00837856501976128\\
53.63	0.00837856502041801\\
53.64	0.00837856502108097\\
53.65	0.0083785650217502\\
53.66	0.00837856502242578\\
53.67	0.00837856502310775\\
53.68	0.00837856502379618\\
53.69	0.00837856502449112\\
53.7	0.00837856502519264\\
53.71	0.0083785650259008\\
53.72	0.00837856502661566\\
53.73	0.00837856502733727\\
53.74	0.00837856502806571\\
53.75	0.00837856502880104\\
53.76	0.00837856502954331\\
53.77	0.00837856503029259\\
53.78	0.00837856503104895\\
53.79	0.00837856503181245\\
53.8	0.00837856503258316\\
53.81	0.00837856503336114\\
53.82	0.00837856503414647\\
53.83	0.0083785650349392\\
53.84	0.00837856503573941\\
53.85	0.00837856503654716\\
53.86	0.00837856503736253\\
53.87	0.00837856503818558\\
53.88	0.00837856503901638\\
53.89	0.00837856503985501\\
53.9	0.00837856504070154\\
53.91	0.00837856504155604\\
53.92	0.00837856504241858\\
53.93	0.00837856504328923\\
53.94	0.00837856504416808\\
53.95	0.00837856504505519\\
53.96	0.00837856504595064\\
53.97	0.00837856504685451\\
53.98	0.00837856504776687\\
53.99	0.0083785650486878\\
54	0.00837856504961739\\
54.01	0.0083785650505557\\
54.02	0.00837856505150282\\
54.03	0.00837856505245883\\
54.04	0.00837856505342381\\
54.05	0.00837856505439785\\
54.06	0.00837856505538101\\
54.07	0.0083785650563734\\
54.08	0.00837856505737508\\
54.09	0.00837856505838616\\
54.1	0.0083785650594067\\
54.11	0.0083785650604368\\
54.12	0.00837856506147655\\
54.13	0.00837856506252603\\
54.14	0.00837856506358533\\
54.15	0.00837856506465454\\
54.16	0.00837856506573375\\
54.17	0.00837856506682305\\
54.18	0.00837856506792253\\
54.19	0.00837856506903229\\
54.2	0.00837856507015242\\
54.21	0.008378565071283\\
54.22	0.00837856507242415\\
54.23	0.00837856507357595\\
54.24	0.0083785650747385\\
54.25	0.00837856507591189\\
54.26	0.00837856507709623\\
54.27	0.00837856507829161\\
54.28	0.00837856507949814\\
54.29	0.00837856508071591\\
54.3	0.00837856508194502\\
54.31	0.00837856508318559\\
54.32	0.0083785650844377\\
54.33	0.00837856508570147\\
54.34	0.00837856508697701\\
54.35	0.00837856508826441\\
54.36	0.00837856508956378\\
54.37	0.00837856509087523\\
54.38	0.00837856509219888\\
54.39	0.00837856509353483\\
54.4	0.00837856509488319\\
54.41	0.00837856509624407\\
54.42	0.00837856509761759\\
54.43	0.00837856509900385\\
54.44	0.00837856510040298\\
54.45	0.00837856510181509\\
54.46	0.0083785651032403\\
54.47	0.00837856510467872\\
54.48	0.00837856510613047\\
54.49	0.00837856510759567\\
54.5	0.00837856510907445\\
54.51	0.00837856511056692\\
54.52	0.0083785651120732\\
54.53	0.00837856511359343\\
54.54	0.00837856511512772\\
54.55	0.00837856511667621\\
54.56	0.00837856511823901\\
54.57	0.00837856511981626\\
54.58	0.00837856512140808\\
54.59	0.0083785651230146\\
54.6	0.00837856512463597\\
54.61	0.0083785651262723\\
54.62	0.00837856512792374\\
54.63	0.00837856512959041\\
54.64	0.00837856513127246\\
54.65	0.00837856513297001\\
54.66	0.00837856513468322\\
54.67	0.00837856513641221\\
54.68	0.00837856513815713\\
54.69	0.00837856513991812\\
54.7	0.00837856514169533\\
54.71	0.00837856514348889\\
54.72	0.00837856514529895\\
54.73	0.00837856514712566\\
54.74	0.00837856514896917\\
54.75	0.00837856515082963\\
54.76	0.00837856515270717\\
54.77	0.00837856515460197\\
54.78	0.00837856515651417\\
54.79	0.00837856515844392\\
54.8	0.00837856516039138\\
54.81	0.00837856516235671\\
54.82	0.00837856516434006\\
54.83	0.0083785651663416\\
54.84	0.00837856516836148\\
54.85	0.00837856517039987\\
54.86	0.00837856517245693\\
54.87	0.00837856517453282\\
54.88	0.00837856517662772\\
54.89	0.00837856517874178\\
54.9	0.00837856518087518\\
54.91	0.00837856518302809\\
54.92	0.00837856518520068\\
54.93	0.00837856518739312\\
54.94	0.00837856518960558\\
54.95	0.00837856519183826\\
54.96	0.00837856519409131\\
54.97	0.00837856519636492\\
54.98	0.00837856519865928\\
54.99	0.00837856520097456\\
55	0.00837856520331094\\
55.01	0.00837856520566862\\
55.02	0.00837856520804777\\
55.03	0.00837856521044859\\
55.04	0.00837856521287127\\
55.05	0.00837856521531599\\
55.06	0.00837856521778295\\
55.07	0.00837856522027235\\
55.08	0.00837856522278438\\
55.09	0.00837856522531924\\
55.1	0.00837856522787712\\
55.11	0.00837856523045823\\
55.12	0.00837856523306278\\
55.13	0.00837856523569096\\
55.14	0.00837856523834298\\
55.15	0.00837856524101905\\
55.16	0.00837856524371938\\
55.17	0.00837856524644418\\
55.18	0.00837856524919366\\
55.19	0.00837856525196803\\
55.2	0.00837856525476752\\
55.21	0.00837856525759234\\
55.22	0.00837856526044271\\
55.23	0.00837856526331886\\
55.24	0.008378565266221\\
55.25	0.00837856526914936\\
55.26	0.00837856527210416\\
55.27	0.00837856527508565\\
55.28	0.00837856527809404\\
55.29	0.00837856528112958\\
55.3	0.00837856528419249\\
55.31	0.00837856528728301\\
55.32	0.00837856529040138\\
55.33	0.00837856529354784\\
55.34	0.00837856529672263\\
55.35	0.008378565299926\\
55.36	0.00837856530315819\\
55.37	0.00837856530641945\\
55.38	0.00837856530971004\\
55.39	0.00837856531303019\\
55.4	0.00837856531638018\\
55.41	0.00837856531976024\\
55.42	0.00837856532317065\\
55.43	0.00837856532661166\\
55.44	0.00837856533008354\\
55.45	0.00837856533358654\\
55.46	0.00837856533712095\\
55.47	0.00837856534068701\\
55.48	0.00837856534428501\\
55.49	0.00837856534791522\\
55.5	0.00837856535157791\\
55.51	0.00837856535527337\\
55.52	0.00837856535900187\\
55.53	0.00837856536276368\\
55.54	0.00837856536655911\\
55.55	0.00837856537038843\\
55.56	0.00837856537425193\\
55.57	0.00837856537814991\\
55.58	0.00837856538208265\\
55.59	0.00837856538605046\\
55.6	0.00837856539005362\\
55.61	0.00837856539409245\\
55.62	0.00837856539816723\\
55.63	0.00837856540227829\\
55.64	0.00837856540642592\\
55.65	0.00837856541061044\\
55.66	0.00837856541483216\\
55.67	0.00837856541909138\\
55.68	0.00837856542338844\\
55.69	0.00837856542772365\\
55.7	0.00837856543209733\\
55.71	0.0083785654365098\\
55.72	0.0083785654409614\\
55.73	0.00837856544545246\\
55.74	0.00837856544998329\\
55.75	0.00837856545455425\\
55.76	0.00837856545916567\\
55.77	0.00837856546381789\\
55.78	0.00837856546851125\\
55.79	0.00837856547324609\\
55.8	0.00837856547802278\\
55.81	0.00837856548284164\\
55.82	0.00837856548770305\\
55.83	0.00837856549260735\\
55.84	0.00837856549755491\\
55.85	0.00837856550254609\\
55.86	0.00837856550758125\\
55.87	0.00837856551266076\\
55.88	0.00837856551778498\\
55.89	0.0083785655229543\\
55.9	0.00837856552816909\\
55.91	0.00837856553342973\\
55.92	0.00837856553873659\\
55.93	0.00837856554409007\\
55.94	0.00837856554949055\\
55.95	0.00837856555493843\\
55.96	0.00837856556043409\\
55.97	0.00837856556597793\\
55.98	0.00837856557157035\\
55.99	0.00837856557721176\\
56	0.00837856558290256\\
56.01	0.00837856558864316\\
56.02	0.00837856559443397\\
56.03	0.0083785656002754\\
56.04	0.00837856560616788\\
56.05	0.00837856561211182\\
56.06	0.00837856561810766\\
56.07	0.00837856562415581\\
56.08	0.00837856563025671\\
56.09	0.00837856563641078\\
56.1	0.00837856564261848\\
56.11	0.00837856564888024\\
56.12	0.00837856565519651\\
56.13	0.00837856566156772\\
56.14	0.00837856566799433\\
56.15	0.0083785656744768\\
56.16	0.00837856568101558\\
56.17	0.00837856568761114\\
56.18	0.00837856569426393\\
56.19	0.00837856570097443\\
56.2	0.0083785657077431\\
56.21	0.00837856571457042\\
56.22	0.00837856572145687\\
56.23	0.00837856572840293\\
56.24	0.00837856573540908\\
56.25	0.00837856574247582\\
56.26	0.00837856574960363\\
56.27	0.00837856575679301\\
56.28	0.00837856576404447\\
56.29	0.00837856577135849\\
56.3	0.0083785657787356\\
56.31	0.0083785657861763\\
56.32	0.0083785657936811\\
56.33	0.00837856580125052\\
56.34	0.00837856580888509\\
56.35	0.00837856581658533\\
56.36	0.00837856582435176\\
56.37	0.00837856583218493\\
56.38	0.00837856584008537\\
56.39	0.00837856584805362\\
56.4	0.00837856585609022\\
56.41	0.00837856586419572\\
56.42	0.00837856587237068\\
56.43	0.00837856588061565\\
56.44	0.0083785658889312\\
56.45	0.00837856589731788\\
56.46	0.00837856590577626\\
56.47	0.00837856591430693\\
56.48	0.00837856592291045\\
56.49	0.0083785659315874\\
56.5	0.00837856594033837\\
56.51	0.00837856594916396\\
56.52	0.00837856595806474\\
56.53	0.00837856596704133\\
56.54	0.00837856597609432\\
56.55	0.00837856598522431\\
56.56	0.00837856599443193\\
56.57	0.00837856600371777\\
56.58	0.00837856601308247\\
56.59	0.00837856602252663\\
56.6	0.0083785660320509\\
56.61	0.0083785660416559\\
56.62	0.00837856605134227\\
56.63	0.00837856606111065\\
56.64	0.00837856607096168\\
56.65	0.00837856608089602\\
56.66	0.00837856609091431\\
56.67	0.00837856610101722\\
56.68	0.0083785661112054\\
56.69	0.00837856612147954\\
56.7	0.00837856613184029\\
56.71	0.00837856614228833\\
56.72	0.00837856615282435\\
56.73	0.00837856616344904\\
56.74	0.00837856617416308\\
56.75	0.00837856618496716\\
56.76	0.008378566195862\\
56.77	0.00837856620684828\\
56.78	0.00837856621792673\\
56.79	0.00837856622909806\\
56.8	0.00837856624036298\\
56.81	0.00837856625172223\\
56.82	0.00837856626317652\\
56.83	0.00837856627472659\\
56.84	0.00837856628637319\\
56.85	0.00837856629811705\\
56.86	0.00837856630995893\\
56.87	0.00837856632189957\\
56.88	0.00837856633393974\\
56.89	0.0083785663460802\\
56.9	0.00837856635832171\\
56.91	0.00837856637066506\\
56.92	0.00837856638311102\\
56.93	0.00837856639566037\\
56.94	0.00837856640831391\\
56.95	0.00837856642107242\\
56.96	0.00837856643393671\\
56.97	0.00837856644690758\\
56.98	0.00837856645998584\\
56.99	0.00837856647317231\\
57	0.0083785664864678\\
57.01	0.00837856649987315\\
57.02	0.00837856651338917\\
57.03	0.00837856652701672\\
57.04	0.00837856654075662\\
57.05	0.00837856655460974\\
57.06	0.00837856656857691\\
57.07	0.008378566582659\\
57.08	0.00837856659685688\\
57.09	0.0083785666111714\\
57.1	0.00837856662560344\\
57.11	0.00837856664015389\\
57.12	0.00837856665482362\\
57.13	0.00837856666961354\\
57.14	0.00837856668452453\\
57.15	0.0083785666995575\\
57.16	0.00837856671471335\\
57.17	0.008378566729993\\
57.18	0.00837856674539737\\
57.19	0.00837856676092738\\
57.2	0.00837856677658396\\
57.21	0.00837856679236805\\
57.22	0.00837856680828059\\
57.23	0.00837856682432253\\
57.24	0.00837856684049481\\
57.25	0.00837856685679841\\
57.26	0.00837856687323428\\
57.27	0.00837856688980339\\
57.28	0.00837856690650672\\
57.29	0.00837856692334526\\
57.3	0.00837856694031999\\
57.31	0.00837856695743191\\
57.32	0.008378566974682\\
57.33	0.0083785669920713\\
57.34	0.00837856700960079\\
57.35	0.00837856702727151\\
57.36	0.00837856704508447\\
57.37	0.00837856706304071\\
57.38	0.00837856708114126\\
57.39	0.00837856709938716\\
57.4	0.00837856711777947\\
57.41	0.00837856713631923\\
57.42	0.00837856715500751\\
57.43	0.00837856717384537\\
57.44	0.00837856719283388\\
57.45	0.00837856721197413\\
57.46	0.0083785672312672\\
57.47	0.00837856725071418\\
57.48	0.00837856727031616\\
57.49	0.00837856729007426\\
57.5	0.00837856730998957\\
57.51	0.00837856733006322\\
57.52	0.00837856735029632\\
57.53	0.00837856737069002\\
57.54	0.00837856739124543\\
57.55	0.0083785674119637\\
57.56	0.00837856743284598\\
57.57	0.00837856745389343\\
57.58	0.00837856747510719\\
57.59	0.00837856749648844\\
57.6	0.00837856751803835\\
57.61	0.0083785675397581\\
57.62	0.00837856756164887\\
57.63	0.00837856758371186\\
57.64	0.00837856760594825\\
57.65	0.00837856762835927\\
57.66	0.00837856765094611\\
57.67	0.00837856767370999\\
57.68	0.00837856769665215\\
57.69	0.00837856771977379\\
57.7	0.00837856774307617\\
57.71	0.00837856776656053\\
57.72	0.00837856779022811\\
57.73	0.00837856781408017\\
57.74	0.00837856783811797\\
57.75	0.00837856786234277\\
57.76	0.00837856788675587\\
57.77	0.00837856791135852\\
57.78	0.00837856793615203\\
57.79	0.00837856796113769\\
57.8	0.00837856798631679\\
57.81	0.00837856801169064\\
57.82	0.00837856803726056\\
57.83	0.00837856806302787\\
57.84	0.00837856808899388\\
57.85	0.00837856811515994\\
57.86	0.00837856814152739\\
57.87	0.00837856816809756\\
57.88	0.00837856819487182\\
57.89	0.00837856822185151\\
57.9	0.008378568249038\\
57.91	0.00837856827643268\\
57.92	0.0083785683040369\\
57.93	0.00837856833185206\\
57.94	0.00837856835987954\\
57.95	0.00837856838812074\\
57.96	0.00837856841657707\\
57.97	0.00837856844524994\\
57.98	0.00837856847414076\\
57.99	0.00837856850325094\\
58	0.00837856853258193\\
58.01	0.00837856856213516\\
58.02	0.00837856859191206\\
58.03	0.00837856862191408\\
58.04	0.00837856865214268\\
58.05	0.00837856868259932\\
58.06	0.00837856871328546\\
58.07	0.00837856874420258\\
58.08	0.00837856877535215\\
58.09	0.00837856880673566\\
58.1	0.00837856883835459\\
58.11	0.00837856887021046\\
58.12	0.00837856890230475\\
58.13	0.00837856893463898\\
58.14	0.00837856896721466\\
58.15	0.00837856900003332\\
58.16	0.00837856903309649\\
58.17	0.00837856906640569\\
58.18	0.00837856909996246\\
58.19	0.00837856913376835\\
58.2	0.00837856916782492\\
58.21	0.00837856920213371\\
58.22	0.0083785692366963\\
58.23	0.00837856927151425\\
58.24	0.00837856930658913\\
58.25	0.00837856934192253\\
58.26	0.00837856937751603\\
58.27	0.00837856941337123\\
58.28	0.00837856944948972\\
58.29	0.0083785694858731\\
58.3	0.00837856952252298\\
58.31	0.00837856955944098\\
58.32	0.00837856959662872\\
58.33	0.00837856963408782\\
58.34	0.00837856967181991\\
58.35	0.00837856970982663\\
58.36	0.00837856974810962\\
58.37	0.00837856978667052\\
58.38	0.008378569825511\\
58.39	0.0083785698646327\\
58.4	0.00837856990403728\\
58.41	0.00837856994372642\\
58.42	0.00837856998370179\\
58.43	0.00837857002396505\\
58.44	0.00837857006451791\\
58.45	0.00837857010536203\\
58.46	0.00837857014649912\\
58.47	0.00837857018793087\\
58.48	0.00837857022965898\\
58.49	0.00837857027168516\\
58.5	0.00837857031401111\\
58.51	0.00837857035663855\\
58.52	0.00837857039956921\\
58.53	0.00837857044280479\\
58.54	0.00837857048634704\\
58.55	0.00837857053019768\\
58.56	0.00837857057435845\\
58.57	0.00837857061883109\\
58.58	0.00837857066361734\\
58.59	0.00837857070871895\\
58.6	0.00837857075413768\\
58.61	0.00837857079987527\\
58.62	0.00837857084593349\\
58.63	0.00837857089231409\\
58.64	0.00837857093901886\\
58.65	0.00837857098604954\\
58.66	0.00837857103340793\\
58.67	0.00837857108109578\\
58.68	0.00837857112911489\\
58.69	0.00837857117746704\\
58.7	0.008378571226154\\
58.71	0.00837857127517757\\
58.72	0.00837857132453954\\
58.73	0.0083785713742417\\
58.74	0.00837857142428584\\
58.75	0.00837857147467377\\
58.76	0.00837857152540729\\
58.77	0.0083785715764882\\
58.78	0.00837857162791829\\
58.79	0.00837857167969939\\
58.8	0.00837857173183329\\
58.81	0.00837857178432182\\
58.82	0.00837857183716677\\
58.83	0.00837857189036997\\
58.84	0.00837857194393323\\
58.85	0.00837857199785836\\
58.86	0.00837857205214719\\
58.87	0.00837857210680153\\
58.88	0.0083785721618232\\
58.89	0.00837857221721401\\
58.9	0.0083785722729758\\
58.91	0.00837857232911038\\
58.92	0.00837857238561958\\
58.93	0.0083785724425052\\
58.94	0.00837857249976909\\
58.95	0.00837857255741305\\
58.96	0.00837857261543892\\
58.97	0.0083785726738485\\
58.98	0.00837857273264363\\
58.99	0.00837857279182611\\
59	0.00837857285139778\\
59.01	0.00837857291136044\\
59.02	0.00837857297171592\\
59.03	0.00837857303246602\\
59.04	0.00837857309361258\\
59.05	0.00837857315515739\\
59.06	0.00837857321710226\\
59.07	0.00837857327944901\\
59.08	0.00837857334219944\\
59.09	0.00837857340535536\\
59.1	0.00837857346891856\\
59.11	0.00837857353289085\\
59.12	0.00837857359727402\\
59.13	0.00837857366206987\\
59.14	0.00837857372728018\\
59.15	0.00837857379290674\\
59.16	0.00837857385895133\\
59.17	0.00837857392541573\\
59.18	0.00837857399230172\\
59.19	0.00837857405961106\\
59.2	0.00837857412734553\\
59.21	0.00837857419550688\\
59.22	0.00837857426409686\\
59.23	0.00837857433311724\\
59.24	0.00837857440256975\\
59.25	0.00837857447245614\\
59.26	0.00837857454277813\\
59.27	0.00837857461353747\\
59.28	0.00837857468473587\\
59.29	0.00837857475637505\\
59.3	0.00837857482845671\\
59.31	0.00837857490098257\\
59.32	0.00837857497395431\\
59.33	0.00837857504737363\\
59.34	0.0083785751212422\\
59.35	0.00837857519556169\\
59.36	0.00837857527033378\\
59.37	0.00837857534556012\\
59.38	0.00837857542124235\\
59.39	0.00837857549738212\\
59.4	0.00837857557398105\\
59.41	0.00837857565104077\\
59.42	0.00837857572856288\\
59.43	0.008378575806549\\
59.44	0.0083785758850007\\
59.45	0.00837857596391957\\
59.46	0.00837857604330717\\
59.47	0.00837857612316508\\
59.48	0.00837857620349483\\
59.49	0.00837857628429796\\
59.5	0.008378576365576\\
59.51	0.00837857644733046\\
59.52	0.00837857652956283\\
59.53	0.00837857661227461\\
59.54	0.00837857669546727\\
59.55	0.00837857677914227\\
59.56	0.00837857686330105\\
59.57	0.00837857694794506\\
59.58	0.00837857703307571\\
59.59	0.0083785771186944\\
59.6	0.00837857720480253\\
59.61	0.00837857729140147\\
59.62	0.00837857737849258\\
59.63	0.0083785774660772\\
59.64	0.00837857755415667\\
59.65	0.00837857764273228\\
59.66	0.00837857773180535\\
59.67	0.00837857782137713\\
59.68	0.00837857791144891\\
59.69	0.00837857800202191\\
59.7	0.00837857809309736\\
59.71	0.00837857818467647\\
59.72	0.00837857827676042\\
59.73	0.00837857836935039\\
59.74	0.00837857846244752\\
59.75	0.00837857855605294\\
59.76	0.00837857865016775\\
59.77	0.00837857874479306\\
59.78	0.00837857883992992\\
59.79	0.00837857893557938\\
59.8	0.00837857903174246\\
59.81	0.00837857912842017\\
59.82	0.00837857922561349\\
59.83	0.00837857932332338\\
59.84	0.00837857942155076\\
59.85	0.00837857952029656\\
59.86	0.00837857961956166\\
59.87	0.00837857971934691\\
59.88	0.00837857981965317\\
59.89	0.00837857992048124\\
59.9	0.00837858002183191\\
59.91	0.00837858012370595\\
59.92	0.00837858022610409\\
59.93	0.00837858032902704\\
59.94	0.00837858043247548\\
59.95	0.00837858053645007\\
59.96	0.00837858064095144\\
59.97	0.00837858074598019\\
59.98	0.00837858085153689\\
59.99	0.00837858095762208\\
60	0.00837858106423627\\
60.01	0.00837858117137996\\
60.02	0.00837858127905359\\
60.03	0.00837858138725758\\
60.04	0.00837858149599233\\
60.05	0.0083785816052582\\
60.06	0.00837858171505552\\
60.07	0.00837858182538459\\
60.08	0.00837858193624566\\
60.09	0.00837858204763898\\
60.1	0.00837858215956474\\
60.11	0.00837858227202311\\
60.12	0.00837858238501421\\
60.13	0.00837858249853815\\
60.14	0.00837858261259498\\
60.15	0.00837858272718473\\
60.16	0.00837858284230738\\
60.17	0.0083785829579629\\
60.18	0.0083785830741512\\
60.19	0.00837858319087216\\
60.2	0.00837858330812562\\
60.21	0.00837858342591139\\
60.22	0.00837858354422924\\
60.23	0.00837858366307888\\
60.24	0.00837858378246002\\
60.25	0.0083785839023723\\
60.26	0.00837858402281532\\
60.27	0.00837858414378866\\
60.28	0.00837858426529186\\
60.29	0.00837858438732438\\
60.3	0.00837858450988569\\
60.31	0.00837858463297518\\
60.32	0.00837858475659222\\
60.33	0.00837858488073612\\
60.34	0.00837858500540617\\
60.35	0.00837858513060159\\
60.36	0.00837858525632158\\
60.37	0.00837858538256528\\
60.38	0.00837858550933178\\
60.39	0.00837858563662016\\
60.4	0.0083785857644294\\
60.41	0.00837858589275849\\
60.42	0.00837858602160634\\
60.43	0.00837858615097181\\
60.44	0.00837858628085375\\
60.45	0.00837858641125092\\
60.46	0.00837858654216205\\
60.47	0.00837858667358584\\
60.48	0.00837858680552091\\
60.49	0.00837858693796585\\
60.5	0.0083785870709192\\
60.51	0.00837858720437946\\
60.52	0.00837858733834505\\
60.53	0.00837858747281438\\
60.54	0.00837858760778577\\
60.55	0.00837858774325754\\
60.56	0.0083785878792279\\
60.57	0.00837858801569506\\
60.58	0.00837858815265714\\
60.59	0.00837858829011225\\
60.6	0.00837858842805841\\
60.61	0.00837858856649361\\
60.62	0.00837858870541579\\
60.63	0.00837858884482281\\
60.64	0.00837858898471252\\
60.65	0.00837858912508269\\
60.66	0.00837858926593103\\
60.67	0.00837858940725523\\
60.68	0.0083785895490529\\
60.69	0.0083785896913216\\
60.7	0.00837858983405885\\
60.71	0.00837858997726211\\
60.72	0.00837859012092877\\
60.73	0.00837859026505619\\
60.74	0.00837859040964167\\
60.75	0.00837859055468246\\
60.76	0.00837859070017574\\
60.77	0.00837859084611866\\
60.78	0.00837859099250829\\
60.79	0.00837859113934166\\
60.8	0.00837859128661576\\
60.81	0.00837859143432751\\
60.82	0.00837859158247377\\
60.83	0.00837859173105136\\
60.84	0.00837859188005705\\
60.85	0.00837859202948755\\
60.86	0.0083785921793395\\
60.87	0.00837859232960953\\
60.88	0.00837859248029417\\
60.89	0.00837859263138994\\
60.9	0.00837859278289327\\
60.91	0.00837859293480056\\
60.92	0.00837859308710816\\
60.93	0.00837859323981236\\
60.94	0.0083785933929094\\
60.95	0.00837859354639547\\
60.96	0.00837859370026672\\
60.97	0.00837859385451923\\
60.98	0.00837859400914905\\
60.99	0.00837859416415217\\
61	0.00837859431952454\\
61.01	0.00837859447526206\\
61.02	0.00837859463136056\\
61.03	0.00837859478781588\\
61.04	0.00837859494462375\\
61.05	0.00837859510177989\\
61.06	0.00837859525927997\\
61.07	0.00837859541711962\\
61.08	0.00837859557529442\\
61.09	0.00837859573379991\\
61.1	0.00837859589263158\\
61.11	0.00837859605178491\\
61.12	0.00837859621125529\\
61.13	0.00837859637103813\\
61.14	0.00837859653112876\\
61.15	0.00837859669152249\\
61.16	0.00837859685221458\\
61.17	0.00837859701320029\\
61.18	0.00837859717447482\\
61.19	0.00837859733603333\\
61.2	0.00837859749787099\\
61.21	0.00837859765998289\\
61.22	0.00837859782236413\\
61.23	0.00837859798500977\\
61.24	0.00837859814791485\\
61.25	0.00837859831107439\\
61.26	0.00837859847448338\\
61.27	0.00837859863813679\\
61.28	0.00837859880202958\\
61.29	0.0083785989661567\\
61.3	0.00837859913051307\\
61.31	0.0083785992950936\\
61.32	0.00837859945989321\\
61.33	0.00837859962490678\\
61.34	0.00837859979012922\\
61.35	0.00837859995555541\\
61.36	0.00837860012118023\\
61.37	0.00837860028699858\\
61.38	0.00837860045300534\\
61.39	0.00837860061919542\\
61.4	0.00837860078556372\\
61.41	0.00837860095210515\\
61.42	0.00837860111881466\\
61.43	0.00837860128568719\\
61.44	0.0083786014527177\\
61.45	0.00837860161990119\\
61.46	0.00837860178723267\\
61.47	0.00837860195470719\\
61.48	0.00837860212231984\\
61.49	0.00837860229006571\\
61.5	0.00837860245793997\\
61.51	0.00837860262593781\\
61.52	0.00837860279405448\\
61.53	0.00837860296228525\\
61.54	0.00837860313062549\\
61.55	0.0083786032990706\\
61.56	0.00837860346761603\\
61.57	0.00837860363625733\\
61.58	0.00837860380499009\\
61.59	0.00837860397381\\
61.6	0.0083786041427128\\
61.61	0.00837860431169434\\
61.62	0.00837860448075055\\
61.63	0.00837860464987746\\
61.64	0.00837860481907117\\
61.65	0.00837860498832793\\
61.66	0.00837860515764406\\
61.67	0.00837860532701601\\
61.68	0.00837860549644035\\
61.69	0.00837860566591379\\
61.7	0.00837860583543314\\
61.71	0.00837860600499537\\
61.72	0.00837860617459758\\
61.73	0.00837860634423705\\
61.74	0.00837860651391117\\
61.75	0.00837860668361753\\
61.76	0.00837860685335388\\
61.77	0.00837860702311814\\
61.78	0.00837860719290842\\
61.79	0.00837860736272301\\
61.8	0.00837860753256043\\
61.81	0.00837860770241936\\
61.82	0.00837860787229872\\
61.83	0.00837860804219767\\
61.84	0.00837860821211556\\
61.85	0.008378608382052\\
61.86	0.00837860855200685\\
61.87	0.00837860872198022\\
61.88	0.00837860889197249\\
61.89	0.0083786090619843\\
61.9	0.00837860923201658\\
61.91	0.00837860940207029\\
61.92	0.00837860957214639\\
61.93	0.00837860974224586\\
61.94	0.00837860991236968\\
61.95	0.00837861008251885\\
61.96	0.00837861025269437\\
61.97	0.00837861042289728\\
61.98	0.00837861059312859\\
61.99	0.00837861076338935\\
62	0.00837861093368062\\
62.01	0.00837861110400346\\
62.02	0.00837861127435894\\
62.03	0.00837861144474816\\
62.04	0.00837861161517221\\
62.05	0.00837861178563219\\
62.06	0.00837861195612923\\
62.07	0.00837861212666446\\
62.08	0.00837861229723902\\
62.09	0.00837861246785406\\
62.1	0.00837861263851075\\
62.11	0.00837861280921025\\
62.12	0.00837861297995375\\
62.13	0.00837861315074246\\
62.14	0.00837861332157756\\
62.15	0.00837861349246027\\
62.16	0.00837861366339183\\
62.17	0.00837861383437347\\
62.18	0.00837861400540643\\
62.19	0.00837861417649197\\
62.2	0.00837861434763135\\
62.21	0.00837861451882585\\
62.22	0.00837861469007676\\
62.23	0.00837861486138537\\
62.24	0.00837861503275299\\
62.25	0.00837861520418093\\
62.26	0.00837861537567051\\
62.27	0.00837861554722306\\
62.28	0.00837861571883994\\
62.29	0.00837861589052249\\
62.3	0.00837861606227206\\
62.31	0.00837861623409004\\
62.32	0.00837861640597779\\
62.33	0.0083786165779367\\
62.34	0.00837861674996817\\
62.35	0.00837861692207359\\
62.36	0.00837861709425439\\
62.37	0.00837861726651198\\
62.38	0.00837861743884778\\
62.39	0.00837861761126323\\
62.4	0.00837861778375977\\
62.41	0.00837861795633885\\
62.42	0.00837861812900194\\
62.43	0.00837861830175048\\
62.44	0.00837861847458595\\
62.45	0.00837861864750983\\
62.46	0.00837861882052361\\
62.47	0.00837861899362876\\
62.48	0.0083786191668268\\
62.49	0.00837861934011922\\
62.5	0.00837861951350753\\
62.51	0.00837861968699324\\
62.52	0.00837861986057788\\
62.53	0.00837862003426296\\
62.54	0.00837862020805003\\
62.55	0.00837862038194061\\
62.56	0.00837862055593624\\
62.57	0.00837862073003846\\
62.58	0.00837862090424884\\
62.59	0.00837862107856891\\
62.6	0.00837862125300023\\
62.61	0.00837862142754436\\
62.62	0.00837862160220288\\
62.63	0.00837862177697733\\
62.64	0.0083786219518693\\
62.65	0.00837862212688036\\
62.66	0.00837862230201207\\
62.67	0.00837862247726602\\
62.68	0.00837862265264378\\
62.69	0.00837862282814694\\
62.7	0.00837862300377708\\
62.71	0.00837862317953579\\
62.72	0.00837862335542464\\
62.73	0.00837862353144522\\
62.74	0.00837862370759913\\
62.75	0.00837862388388795\\
62.76	0.00837862406031326\\
62.77	0.00837862423687665\\
62.78	0.00837862441357971\\
62.79	0.00837862459042402\\
62.8	0.00837862476741118\\
62.81	0.00837862494454276\\
62.82	0.00837862512182034\\
62.83	0.00837862529924552\\
62.84	0.00837862547681987\\
62.85	0.00837862565454496\\
62.86	0.00837862583242238\\
62.87	0.00837862601045369\\
62.88	0.00837862618864046\\
62.89	0.00837862636698427\\
62.9	0.00837862654548668\\
62.91	0.00837862672414924\\
62.92	0.00837862690297352\\
62.93	0.00837862708196106\\
62.94	0.00837862726111341\\
62.95	0.00837862744043213\\
62.96	0.00837862761991874\\
62.97	0.00837862779957479\\
62.98	0.0083786279794018\\
62.99	0.0083786281594013\\
63	0.00837862833957481\\
63.01	0.00837862851992384\\
63.02	0.00837862870044989\\
63.03	0.00837862888115447\\
63.04	0.00837862906203908\\
63.05	0.00837862924310521\\
63.06	0.00837862942435433\\
63.07	0.00837862960578792\\
63.08	0.00837862978740745\\
63.09	0.00837862996921439\\
63.1	0.00837863015121018\\
63.11	0.00837863033339628\\
63.12	0.00837863051577412\\
63.13	0.00837863069834515\\
63.14	0.00837863088111077\\
63.15	0.00837863106407242\\
63.16	0.00837863124723149\\
63.17	0.00837863143058941\\
63.18	0.00837863161414754\\
63.19	0.0083786317979073\\
63.2	0.00837863198187004\\
63.21	0.00837863216603715\\
63.22	0.00837863235040999\\
63.23	0.00837863253498991\\
63.24	0.00837863271977826\\
63.25	0.00837863290477638\\
63.26	0.00837863308998561\\
63.27	0.00837863327540727\\
63.28	0.00837863346104267\\
63.29	0.00837863364689314\\
63.3	0.00837863383295997\\
63.31	0.00837863401924446\\
63.32	0.0083786342057479\\
63.33	0.00837863439247159\\
63.34	0.00837863457941679\\
63.35	0.00837863476658478\\
63.36	0.00837863495397684\\
63.37	0.00837863514159422\\
63.38	0.00837863532943819\\
63.39	0.00837863551751\\
63.4	0.00837863570581091\\
63.41	0.00837863589434216\\
63.42	0.008378636083105\\
63.43	0.00837863627210068\\
63.44	0.00837863646133044\\
63.45	0.00837863665079552\\
63.46	0.00837863684049718\\
63.47	0.00837863703043665\\
63.48	0.00837863722061518\\
63.49	0.00837863741103403\\
63.5	0.00837863760169444\\
63.51	0.00837863779259768\\
63.52	0.008378637983745\\
63.53	0.00837863817513768\\
63.54	0.008378638366777\\
63.55	0.00837863855866425\\
63.56	0.00837863875080071\\
63.57	0.0083786389431877\\
63.58	0.00837863913582653\\
63.59	0.00837863932871852\\
63.6	0.00837863952186502\\
63.61	0.00837863971526737\\
63.62	0.00837863990892692\\
63.63	0.00837864010284504\\
63.64	0.0083786402970231\\
63.65	0.00837864049146249\\
63.66	0.00837864068616461\\
63.67	0.00837864088113085\\
63.68	0.00837864107636264\\
63.69	0.0083786412718614\\
63.7	0.00837864146762858\\
63.71	0.00837864166366561\\
63.72	0.00837864185997395\\
63.73	0.00837864205655508\\
63.74	0.00837864225341047\\
63.75	0.00837864245054162\\
63.76	0.00837864264795002\\
63.77	0.00837864284563719\\
63.78	0.00837864304360465\\
63.79	0.00837864324185393\\
63.8	0.00837864344038658\\
63.81	0.00837864363920416\\
63.82	0.00837864383830824\\
63.83	0.00837864403770038\\
63.84	0.0083786442373822\\
63.85	0.00837864443735527\\
63.86	0.00837864463762123\\
63.87	0.0083786448381817\\
63.88	0.00837864503903831\\
63.89	0.00837864524019271\\
63.9	0.00837864544164656\\
63.91	0.00837864564340154\\
63.92	0.00837864584545932\\
63.93	0.00837864604782162\\
63.94	0.00837864625049013\\
63.95	0.00837864645346658\\
63.96	0.00837864665675269\\
63.97	0.00837864686035022\\
63.98	0.00837864706426093\\
63.99	0.00837864726848658\\
64	0.00837864747302896\\
64.01	0.00837864767788987\\
64.02	0.0083786478830711\\
64.03	0.00837864808857449\\
64.04	0.00837864829440187\\
64.05	0.00837864850055509\\
64.06	0.008378648707036\\
64.07	0.00837864891384648\\
64.08	0.00837864912098842\\
64.09	0.00837864932846371\\
64.1	0.00837864953627427\\
64.11	0.00837864974442203\\
64.12	0.00837864995290893\\
64.13	0.00837865016173691\\
64.14	0.00837865037090795\\
64.15	0.00837865058042403\\
64.16	0.00837865079028714\\
64.17	0.00837865100049929\\
64.18	0.0083786512110625\\
64.19	0.00837865142197882\\
64.2	0.00837865163325028\\
64.21	0.00837865184487896\\
64.22	0.00837865205686693\\
64.23	0.00837865226921629\\
64.24	0.00837865248192914\\
64.25	0.00837865269500761\\
64.26	0.00837865290845382\\
64.27	0.00837865312226995\\
64.28	0.00837865333645814\\
64.29	0.00837865355102058\\
64.3	0.00837865376595946\\
64.31	0.008378653981277\\
64.32	0.00837865419697542\\
64.33	0.00837865441305696\\
64.34	0.00837865462952387\\
64.35	0.00837865484637843\\
64.36	0.00837865506362292\\
64.37	0.00837865528125964\\
64.38	0.00837865549929091\\
64.39	0.00837865571771906\\
64.4	0.00837865593654644\\
64.41	0.00837865615577541\\
64.42	0.00837865637540834\\
64.43	0.00837865659544764\\
64.44	0.00837865681589572\\
64.45	0.00837865703675499\\
64.46	0.00837865725802791\\
64.47	0.00837865747971693\\
64.48	0.00837865770182452\\
64.49	0.00837865792435318\\
64.5	0.00837865814730541\\
64.51	0.00837865837068374\\
64.52	0.0083786585944907\\
64.53	0.00837865881872886\\
64.54	0.00837865904340077\\
64.55	0.00837865926850904\\
64.56	0.00837865949405628\\
64.57	0.00837865972004509\\
64.58	0.00837865994647813\\
64.59	0.00837866017335804\\
64.6	0.0083786604006875\\
64.61	0.00837866062846921\\
64.62	0.00837866085670587\\
64.63	0.0083786610854002\\
64.64	0.00837866131455494\\
64.65	0.00837866154417287\\
64.66	0.00837866177425674\\
64.67	0.00837866200480936\\
64.68	0.00837866223583354\\
64.69	0.00837866246733211\\
64.7	0.0083786626993079\\
64.71	0.0083786629317638\\
64.72	0.00837866316470267\\
64.73	0.00837866339812743\\
64.74	0.00837866363204098\\
64.75	0.00837866386644626\\
64.76	0.00837866410134623\\
64.77	0.00837866433674386\\
64.78	0.00837866457264214\\
64.79	0.00837866480904407\\
64.8	0.00837866504595268\\
64.81	0.00837866528337102\\
64.82	0.00837866552130214\\
64.83	0.00837866575974914\\
64.84	0.0083786659987151\\
64.85	0.00837866623820316\\
64.86	0.00837866647821643\\
64.87	0.00837866671875808\\
64.88	0.00837866695983129\\
64.89	0.00837866720143923\\
64.9	0.00837866744358513\\
64.91	0.00837866768627222\\
64.92	0.00837866792950374\\
64.93	0.00837866817328295\\
64.94	0.00837866841761316\\
64.95	0.00837866866249765\\
64.96	0.00837866890793976\\
64.97	0.00837866915394283\\
64.98	0.00837866940051022\\
64.99	0.00837866964764531\\
65	0.0083786698953515\\
65.01	0.00837867014363222\\
65.02	0.00837867039249089\\
65.03	0.00837867064193099\\
65.04	0.00837867089195598\\
65.05	0.00837867114256936\\
65.06	0.00837867139377464\\
65.07	0.00837867164557537\\
65.08	0.0083786718979751\\
65.09	0.0083786721509774\\
65.1	0.00837867240458586\\
65.11	0.0083786726588041\\
65.12	0.00837867291363575\\
65.13	0.00837867316908446\\
65.14	0.0083786734251539\\
65.15	0.00837867368184777\\
65.16	0.00837867393916976\\
65.17	0.00837867419712362\\
65.18	0.00837867445571308\\
65.19	0.00837867471494193\\
65.2	0.00837867497481395\\
65.21	0.00837867523533294\\
65.22	0.00837867549650273\\
65.23	0.00837867575832717\\
65.24	0.00837867602081013\\
65.25	0.00837867628395549\\
65.26	0.00837867654776716\\
65.27	0.00837867681224906\\
65.28	0.00837867707740513\\
65.29	0.00837867734323935\\
65.3	0.00837867760975569\\
65.31	0.00837867787695815\\
65.32	0.00837867814485077\\
65.33	0.00837867841343758\\
65.34	0.00837867868272264\\
65.35	0.00837867895271003\\
65.36	0.00837867922340386\\
65.37	0.00837867949480824\\
65.38	0.0083786797669273\\
65.39	0.00837868003976522\\
65.4	0.00837868031332617\\
65.41	0.00837868058761433\\
65.42	0.00837868086263394\\
65.43	0.00837868113838921\\
65.44	0.00837868141488441\\
65.45	0.00837868169212381\\
65.46	0.0083786819701117\\
65.47	0.00837868224885239\\
65.48	0.00837868252835021\\
65.49	0.00837868280860951\\
65.5	0.00837868308963465\\
65.51	0.00837868337143001\\
65.52	0.00837868365400002\\
65.53	0.00837868393734907\\
65.54	0.00837868422148163\\
65.55	0.00837868450640214\\
65.56	0.00837868479211508\\
65.57	0.00837868507862494\\
65.58	0.00837868536593625\\
65.59	0.00837868565405354\\
65.6	0.00837868594298134\\
65.61	0.00837868623272423\\
65.62	0.0083786865232868\\
65.63	0.00837868681467365\\
65.64	0.00837868710688939\\
65.65	0.00837868739993866\\
65.66	0.00837868769382612\\
65.67	0.00837868798855644\\
65.68	0.0083786882841343\\
65.69	0.00837868858056442\\
65.7	0.00837868887785151\\
65.71	0.00837868917600032\\
65.72	0.00837868947501559\\
65.73	0.0083786897749021\\
65.74	0.00837869007566464\\
65.75	0.008378690377308\\
65.76	0.00837869067983702\\
65.77	0.00837869098325651\\
65.78	0.00837869128757134\\
65.79	0.00837869159278637\\
65.8	0.00837869189890647\\
65.81	0.00837869220593655\\
65.82	0.00837869251388151\\
65.83	0.00837869282274628\\
65.84	0.0083786931325358\\
65.85	0.00837869344325502\\
65.86	0.00837869375490891\\
65.87	0.00837869406750244\\
65.88	0.00837869438104062\\
65.89	0.00837869469552846\\
65.9	0.00837869501097096\\
65.91	0.00837869532737317\\
65.92	0.00837869564474013\\
65.93	0.0083786959630769\\
65.94	0.00837869628238856\\
65.95	0.00837869660268018\\
65.96	0.00837869692395686\\
65.97	0.00837869724622371\\
65.98	0.00837869756948583\\
65.99	0.00837869789374837\\
66	0.00837869821901645\\
66.01	0.00837869854529523\\
66.02	0.00837869887258986\\
66.03	0.00837869920090552\\
66.04	0.00837869953024738\\
66.05	0.00837869986062062\\
66.06	0.00837870019203045\\
66.07	0.00837870052448206\\
66.08	0.00837870085798067\\
66.09	0.00837870119253149\\
66.1	0.00837870152813976\\
66.11	0.00837870186481071\\
66.12	0.00837870220254958\\
66.13	0.00837870254136161\\
66.14	0.00837870288125206\\
66.15	0.00837870322222619\\
66.16	0.00837870356428926\\
66.17	0.00837870390744654\\
66.18	0.0083787042517033\\
66.19	0.00837870459706483\\
66.2	0.0083787049435364\\
66.21	0.0083787052911233\\
66.22	0.0083787056398308\\
66.23	0.00837870598966422\\
66.24	0.00837870634062883\\
66.25	0.00837870669272992\\
66.26	0.0083787070459728\\
66.27	0.00837870740036275\\
66.28	0.00837870775590508\\
66.29	0.00837870811260507\\
66.3	0.00837870847046802\\
66.31	0.00837870882949923\\
66.32	0.00837870918970397\\
66.33	0.00837870955108755\\
66.34	0.00837870991365525\\
66.35	0.00837871027741234\\
66.36	0.00837871064236411\\
66.37	0.00837871100851584\\
66.38	0.00837871137587278\\
66.39	0.00837871174444021\\
66.4	0.00837871211422338\\
66.41	0.00837871248522754\\
66.42	0.00837871285745794\\
66.43	0.00837871323091981\\
66.44	0.00837871360561838\\
66.45	0.00837871398155887\\
66.46	0.00837871435874649\\
66.47	0.00837871473718644\\
66.48	0.0083787151168839\\
66.49	0.00837871549784405\\
66.5	0.00837871588007206\\
66.51	0.00837871626357308\\
66.52	0.00837871664835225\\
66.53	0.00837871703441468\\
66.54	0.0083787174217655\\
66.55	0.00837871781040979\\
66.56	0.00837871820035264\\
66.57	0.0083787185915991\\
66.58	0.00837871898415422\\
66.59	0.00837871937802302\\
66.6	0.00837871977321051\\
66.61	0.00837872016972168\\
66.62	0.00837872056756149\\
66.63	0.00837872096673488\\
66.64	0.00837872136724679\\
66.65	0.0083787217691021\\
66.66	0.0083787221723057\\
66.67	0.00837872257686244\\
66.68	0.00837872298277714\\
66.69	0.0083787233900546\\
66.7	0.0083787237986996\\
66.71	0.00837872420871688\\
66.72	0.00837872462011117\\
66.73	0.00837872503288714\\
66.74	0.00837872544704945\\
66.75	0.00837872586260274\\
66.76	0.00837872627955159\\
66.77	0.00837872669790058\\
66.78	0.00837872711765421\\
66.79	0.00837872753881699\\
66.8	0.00837872796139338\\
66.81	0.00837872838538778\\
66.82	0.0083787288108046\\
66.83	0.00837872923764817\\
66.84	0.00837872966592279\\
66.85	0.00837873009563273\\
66.86	0.00837873052678221\\
66.87	0.00837873095937542\\
66.88	0.00837873139341649\\
66.89	0.0083787318289095\\
66.9	0.00837873226585852\\
66.91	0.00837873270426754\\
66.92	0.0083787331441405\\
66.93	0.00837873358548133\\
66.94	0.00837873402829386\\
66.95	0.00837873447258192\\
66.96	0.00837873491834923\\
66.97	0.00837873536559952\\
66.98	0.00837873581433642\\
66.99	0.00837873626456352\\
67	0.00837873671628435\\
67.01	0.0083787371695024\\
67.02	0.00837873762422109\\
67.03	0.00837873808044376\\
67.04	0.00837873853817373\\
67.05	0.00837873899741422\\
67.06	0.00837873945816841\\
67.07	0.00837873992043941\\
67.08	0.00837874038423027\\
67.09	0.00837874084954396\\
67.1	0.00837874131638338\\
67.11	0.00837874178475139\\
67.12	0.00837874225465076\\
67.13	0.00837874272608418\\
67.14	0.00837874319905427\\
67.15	0.00837874367356361\\
67.16	0.00837874414961465\\
67.17	0.00837874462720981\\
67.18	0.00837874510635141\\
67.19	0.00837874558704169\\
67.2	0.00837874606928282\\
67.21	0.00837874655307689\\
67.22	0.0083787470384259\\
67.23	0.00837874752533177\\
67.24	0.00837874801379633\\
67.25	0.00837874850382132\\
67.26	0.00837874899540841\\
67.27	0.00837874948855918\\
67.28	0.00837874998327509\\
67.29	0.00837875047955754\\
67.3	0.00837875097740783\\
67.31	0.00837875147682716\\
67.32	0.00837875197781663\\
67.33	0.00837875248037725\\
67.34	0.00837875298450995\\
67.35	0.00837875349021552\\
67.36	0.00837875399749469\\
67.37	0.00837875450634806\\
67.38	0.00837875501677615\\
67.39	0.00837875552877935\\
67.4	0.00837875604235796\\
67.41	0.00837875655751218\\
67.42	0.00837875707424207\\
67.43	0.00837875759254763\\
67.44	0.0083787581124287\\
67.45	0.00837875863388504\\
67.46	0.00837875915691628\\
67.47	0.00837875968152194\\
67.48	0.00837876020770143\\
67.49	0.00837876073545403\\
67.5	0.00837876126477891\\
67.51	0.00837876179567512\\
67.52	0.00837876232814158\\
67.53	0.00837876286217711\\
67.54	0.00837876339778037\\
67.55	0.00837876393494993\\
67.56	0.00837876447368422\\
67.57	0.00837876501398154\\
67.58	0.00837876555584006\\
67.59	0.00837876609925782\\
67.6	0.00837876664423275\\
67.61	0.00837876719076262\\
67.62	0.00837876773884508\\
67.63	0.00837876828847764\\
67.64	0.00837876883965768\\
67.65	0.00837876939238245\\
67.66	0.00837876994664905\\
67.67	0.00837877050245444\\
67.68	0.00837877105979546\\
67.69	0.00837877161866878\\
67.7	0.00837877217907095\\
67.71	0.00837877274099838\\
67.72	0.00837877330444731\\
67.73	0.00837877386941388\\
67.74	0.00837877443589403\\
67.75	0.0083787750038836\\
67.76	0.00837877557337827\\
67.77	0.00837877614437355\\
67.78	0.00837877671686483\\
67.79	0.00837877729084733\\
67.8	0.00837877786631615\\
67.81	0.0083787784432662\\
67.82	0.00837877902169227\\
67.83	0.00837877960158898\\
67.84	0.0083787801829508\\
67.85	0.00837878076577207\\
67.86	0.00837878135004693\\
67.87	0.00837878193576942\\
67.88	0.00837878252293339\\
67.89	0.00837878311153254\\
67.9	0.00837878370156043\\
67.91	0.00837878429301045\\
67.92	0.00837878488587585\\
67.93	0.0083787854801497\\
67.94	0.00837878607582496\\
67.95	0.00837878667289438\\
67.96	0.00837878727135059\\
67.97	0.00837878787118605\\
67.98	0.00837878847239309\\
67.99	0.00837878907496385\\
68	0.00837878967889033\\
68.01	0.0083787902841644\\
68.02	0.00837879089077773\\
68.03	0.00837879149872189\\
68.04	0.00837879210798825\\
68.05	0.00837879271856806\\
68.06	0.00837879333045242\\
68.07	0.00837879394363225\\
68.08	0.00837879455809835\\
68.09	0.00837879517384137\\
68.1	0.00837879579085181\\
68.11	0.00837879640912001\\
68.12	0.00837879702863619\\
68.13	0.00837879764939042\\
68.14	0.00837879827137261\\
68.15	0.00837879889457257\\
68.16	0.00837879951897994\\
68.17	0.00837880014458423\\
68.18	0.00837880077137483\\
68.19	0.00837880139934099\\
68.2	0.00837880202847183\\
68.21	0.00837880265875635\\
68.22	0.00837880329018342\\
68.23	0.0083788039227418\\
68.24	0.00837880455642011\\
68.25	0.00837880519120688\\
68.26	0.0083788058270905\\
68.27	0.00837880646405929\\
68.28	0.00837880710210142\\
68.29	0.00837880774120499\\
68.3	0.00837880838135798\\
68.31	0.0083788090225483\\
68.32	0.00837880966476376\\
68.33	0.00837881030799207\\
68.34	0.00837881095222087\\
68.35	0.00837881159743772\\
68.36	0.00837881224363012\\
68.37	0.00837881289078548\\
68.38	0.00837881353889117\\
68.39	0.0083788141879345\\
68.4	0.00837881483790271\\
68.41	0.00837881548878302\\
68.42	0.0083788161405626\\
68.43	0.00837881679322858\\
68.44	0.00837881744676808\\
68.45	0.00837881810116818\\
68.46	0.00837881875641597\\
68.47	0.00837881941249853\\
68.48	0.00837882006940292\\
68.49	0.00837882072711624\\
68.5	0.0083788213856256\\
68.51	0.00837882204491812\\
68.52	0.00837882270498097\\
68.53	0.00837882336580137\\
68.54	0.00837882402736658\\
68.55	0.00837882468966394\\
68.56	0.00837882535268083\\
68.57	0.00837882601640475\\
68.58	0.00837882668082328\\
68.59	0.00837882734592409\\
68.6	0.00837882801169498\\
68.61	0.00837882867812387\\
68.62	0.00837882934519884\\
68.63	0.00837883001290808\\
68.64	0.00837883068123999\\
68.65	0.0083788313501831\\
68.66	0.00837883201972616\\
68.67	0.00837883268985813\\
68.68	0.00837883336056816\\
68.69	0.00837883403184564\\
68.7	0.00837883470368023\\
68.71	0.00837883537606183\\
68.72	0.00837883604898062\\
68.73	0.00837883672242708\\
68.74	0.00837883739639201\\
68.75	0.00837883807086653\\
68.76	0.00837883874584211\\
68.77	0.00837883942131057\\
68.78	0.00837884009726414\\
68.79	0.00837884077369544\\
68.8	0.00837884145059752\\
68.81	0.00837884212796386\\
68.82	0.00837884280578841\\
68.83	0.00837884348406563\\
68.84	0.00837884416279046\\
68.85	0.00837884484195837\\
68.86	0.00837884552156541\\
68.87	0.00837884620160819\\
68.88	0.00837884688208393\\
68.89	0.00837884756299048\\
68.9	0.00837884824432634\\
68.91	0.00837884892609069\\
68.92	0.00837884960828343\\
68.93	0.00837885029090518\\
68.94	0.00837885097395669\\
68.95	0.0083788516574387\\
68.96	0.00837885234135197\\
68.97	0.00837885302569724\\
68.98	0.00837885371047526\\
68.99	0.00837885439568679\\
69	0.00837885508133259\\
69.01	0.00837885576741342\\
69.02	0.00837885645393004\\
69.03	0.00837885714088321\\
69.04	0.0083788578282737\\
69.05	0.00837885851610229\\
69.06	0.00837885920436973\\
69.07	0.00837885989307682\\
69.08	0.00837886058222431\\
69.09	0.00837886127181301\\
69.1	0.00837886196184367\\
69.11	0.0083788626523171\\
69.12	0.00837886334323408\\
69.13	0.00837886403459539\\
69.14	0.00837886472640182\\
69.15	0.00837886541865418\\
69.16	0.00837886611135326\\
69.17	0.00837886680449986\\
69.18	0.00837886749809478\\
69.19	0.00837886819213883\\
69.2	0.0083788688866328\\
69.21	0.00837886958157752\\
69.22	0.00837887027697379\\
69.23	0.00837887097282244\\
69.24	0.00837887166912426\\
69.25	0.0083788723658801\\
69.26	0.00837887306309076\\
69.27	0.00837887376075708\\
69.28	0.00837887445887988\\
69.29	0.00837887515746\\
69.3	0.00837887585649826\\
69.31	0.0083788765559955\\
69.32	0.00837887725595256\\
69.33	0.00837887795637028\\
69.34	0.00837887865724951\\
69.35	0.00837887935859109\\
69.36	0.00837888006039587\\
69.37	0.00837888076266471\\
69.38	0.00837888146539845\\
69.39	0.00837888216859796\\
69.4	0.00837888287226409\\
69.41	0.00837888357639772\\
69.42	0.00837888428099969\\
69.43	0.00837888498607089\\
69.44	0.00837888569161219\\
69.45	0.00837888639762445\\
69.46	0.00837888710410855\\
69.47	0.00837888781106538\\
69.48	0.00837888851849582\\
69.49	0.00837888922640075\\
69.5	0.00837888993478107\\
69.51	0.00837889064363765\\
69.52	0.0083788913529714\\
69.53	0.00837889206278321\\
69.54	0.00837889277307398\\
69.55	0.00837889348384462\\
69.56	0.00837889419509603\\
69.57	0.00837889490682912\\
69.58	0.0083788956190448\\
69.59	0.00837889633174398\\
69.6	0.00837889704492759\\
69.61	0.00837889775859654\\
69.62	0.00837889847275176\\
69.63	0.00837889918739416\\
69.64	0.00837889990252469\\
69.65	0.00837890061814427\\
69.66	0.00837890133425385\\
69.67	0.00837890205085435\\
69.68	0.00837890276794672\\
69.69	0.0083789034855319\\
69.7	0.00837890420361085\\
69.71	0.00837890492218451\\
69.72	0.00837890564125384\\
69.73	0.00837890636081979\\
69.74	0.00837890708088332\\
69.75	0.00837890780144541\\
69.76	0.00837890852250701\\
69.77	0.00837890924406909\\
69.78	0.00837890996613263\\
69.79	0.0083789106886986\\
69.8	0.00837891141176799\\
69.81	0.00837891213534178\\
69.82	0.00837891285942095\\
69.83	0.00837891358400649\\
69.84	0.00837891430909939\\
69.85	0.00837891503470065\\
69.86	0.00837891576081128\\
69.87	0.00837891648743227\\
69.88	0.00837891721456462\\
69.89	0.00837891794220936\\
69.9	0.00837891867036749\\
69.91	0.00837891939904003\\
69.92	0.00837892012822799\\
69.93	0.0083789208579324\\
69.94	0.0083789215881543\\
69.95	0.0083789223188947\\
69.96	0.00837892305015464\\
69.97	0.00837892378193517\\
69.98	0.00837892451423731\\
69.99	0.00837892524706212\\
70	0.00837892598041064\\
70.01	0.00837892671428393\\
70.02	0.00837892744868304\\
70.03	0.00837892818360902\\
70.04	0.00837892891906295\\
70.05	0.00837892965504589\\
70.06	0.00837893039155891\\
70.07	0.00837893112860308\\
70.08	0.00837893186617948\\
70.09	0.00837893260428919\\
70.1	0.0083789333429333\\
70.11	0.0083789340821129\\
70.12	0.00837893482182907\\
70.13	0.00837893556208292\\
70.14	0.00837893630287555\\
70.15	0.00837893704420806\\
70.16	0.00837893778608155\\
70.17	0.00837893852849715\\
70.18	0.00837893927145597\\
70.19	0.00837894001495913\\
70.2	0.00837894075900776\\
70.21	0.00837894150360297\\
70.22	0.00837894224874592\\
70.23	0.00837894299443772\\
70.24	0.00837894374067953\\
70.25	0.00837894448747248\\
70.26	0.00837894523481774\\
70.27	0.00837894598271644\\
70.28	0.00837894673116975\\
70.29	0.00837894748017884\\
70.3	0.00837894822974486\\
70.31	0.00837894897986899\\
70.32	0.0083789497305524\\
70.33	0.00837895048179627\\
70.34	0.00837895123360179\\
70.35	0.00837895198597014\\
70.36	0.00837895273890252\\
70.37	0.00837895349240012\\
70.38	0.00837895424646414\\
70.39	0.00837895500109579\\
70.4	0.00837895575629628\\
70.41	0.00837895651206683\\
70.42	0.00837895726840866\\
70.43	0.00837895802532298\\
70.44	0.00837895878281103\\
70.45	0.00837895954087404\\
70.46	0.00837896029951326\\
70.47	0.00837896105872991\\
70.48	0.00837896181852526\\
70.49	0.00837896257890055\\
70.5	0.00837896333985703\\
70.51	0.00837896410139598\\
70.52	0.00837896486351867\\
70.53	0.00837896562622635\\
70.54	0.00837896638952031\\
70.55	0.00837896715340183\\
70.56	0.0083789679178722\\
70.57	0.00837896868293271\\
70.58	0.00837896944858465\\
70.59	0.00837897021482933\\
70.6	0.00837897098166805\\
70.61	0.00837897174910214\\
70.62	0.00837897251713289\\
70.63	0.00837897328576164\\
70.64	0.00837897405498972\\
70.65	0.00837897482481845\\
70.66	0.00837897559524918\\
70.67	0.00837897636628325\\
70.68	0.008378977137922\\
70.69	0.0083789779101668\\
70.7	0.00837897868301899\\
70.71	0.00837897945647995\\
70.72	0.00837898023055105\\
70.73	0.00837898100523366\\
70.74	0.00837898178052917\\
70.75	0.00837898255643895\\
70.76	0.0083789833329644\\
70.77	0.00837898411010693\\
70.78	0.00837898488786792\\
70.79	0.0083789856662488\\
70.8	0.00837898644525098\\
70.81	0.00837898722487587\\
70.82	0.0083789880051249\\
70.83	0.00837898878599951\\
70.84	0.00837898956750113\\
70.85	0.00837899034963121\\
70.86	0.00837899113239118\\
70.87	0.00837899191578252\\
70.88	0.00837899269980667\\
70.89	0.00837899348446511\\
70.9	0.00837899426975931\\
70.91	0.00837899505569073\\
70.92	0.00837899584226088\\
70.93	0.00837899662947124\\
70.94	0.00837899741732329\\
70.95	0.00837899820581855\\
70.96	0.00837899899495853\\
70.97	0.00837899978474472\\
70.98	0.00837900057517866\\
70.99	0.00837900136626188\\
71	0.00837900215799589\\
71.01	0.00837900295038224\\
71.02	0.00837900374342247\\
71.03	0.00837900453711813\\
71.04	0.00837900533147077\\
71.05	0.00837900612648197\\
71.06	0.00837900692215327\\
71.07	0.00837900771848626\\
71.08	0.00837900851548252\\
71.09	0.00837900931314363\\
71.1	0.00837901011147118\\
71.11	0.00837901091046677\\
71.12	0.008379011710132\\
71.13	0.00837901251046849\\
71.14	0.00837901331147785\\
71.15	0.00837901411316169\\
71.16	0.00837901491552166\\
71.17	0.00837901571855937\\
71.18	0.00837901652227648\\
71.19	0.00837901732667463\\
71.2	0.00837901813175546\\
71.21	0.00837901893752065\\
71.22	0.00837901974397185\\
71.23	0.00837902055111073\\
71.24	0.00837902135893897\\
71.25	0.00837902216745826\\
71.26	0.00837902297667027\\
71.27	0.00837902378657671\\
71.28	0.00837902459717928\\
71.29	0.00837902540847968\\
71.3	0.00837902622047964\\
71.31	0.00837902703318085\\
71.32	0.00837902784658506\\
71.33	0.00837902866069399\\
71.34	0.00837902947550938\\
71.35	0.00837903029103298\\
71.36	0.00837903110726653\\
71.37	0.00837903192421178\\
71.38	0.00837903274187051\\
71.39	0.00837903356024448\\
71.4	0.00837903437933545\\
71.41	0.00837903519914522\\
71.42	0.00837903601967556\\
71.43	0.00837903684092827\\
71.44	0.00837903766290514\\
71.45	0.00837903848560798\\
71.46	0.0083790393090386\\
71.47	0.00837904013319881\\
71.48	0.00837904095809044\\
71.49	0.0083790417837153\\
71.5	0.00837904261007524\\
71.51	0.00837904343717208\\
71.52	0.00837904426500768\\
71.53	0.00837904509358389\\
71.54	0.00837904592290256\\
71.55	0.00837904675296555\\
71.56	0.00837904758377472\\
71.57	0.00837904841533196\\
71.58	0.00837904924763914\\
71.59	0.00837905008069815\\
71.6	0.00837905091451086\\
71.61	0.00837905174907919\\
71.62	0.00837905258440502\\
71.63	0.00837905342049027\\
71.64	0.00837905425733685\\
71.65	0.00837905509494666\\
71.66	0.00837905593332165\\
71.67	0.00837905677246372\\
71.68	0.00837905761237482\\
71.69	0.00837905845305689\\
71.7	0.00837905929451186\\
71.71	0.00837906013674169\\
71.72	0.00837906097974832\\
71.73	0.00837906182353372\\
71.74	0.00837906266809985\\
71.75	0.00837906351344868\\
71.76	0.00837906435958218\\
71.77	0.00837906520650233\\
71.78	0.00837906605421111\\
71.79	0.0083790669027105\\
71.8	0.00837906775200251\\
71.81	0.00837906860208913\\
71.82	0.00837906945297236\\
71.83	0.0083790703046542\\
71.84	0.00837907115713667\\
71.85	0.00837907201042178\\
71.86	0.00837907286451154\\
71.87	0.00837907371940799\\
71.88	0.00837907457511314\\
71.89	0.00837907543162904\\
71.9	0.0083790762889577\\
71.91	0.00837907714710119\\
71.92	0.00837907800606152\\
71.93	0.00837907886584076\\
71.94	0.00837907972644095\\
71.95	0.00837908058786415\\
71.96	0.00837908145011241\\
71.97	0.00837908231318779\\
71.98	0.00837908317709236\\
71.99	0.00837908404182818\\
72	0.00837908490739733\\
72.01	0.00837908577380187\\
72.02	0.00837908664104389\\
72.03	0.00837908750912546\\
72.04	0.00837908837804866\\
72.05	0.00837908924781557\\
72.06	0.00837909011842829\\
72.07	0.0083790909898889\\
72.08	0.0083790918621995\\
72.09	0.00837909273536216\\
72.1	0.008379093609379\\
72.11	0.0083790944842521\\
72.12	0.00837909535998357\\
72.13	0.0083790962365755\\
72.14	0.00837909711403\\
72.15	0.00837909799234917\\
72.16	0.00837909887153511\\
72.17	0.00837909975158993\\
72.18	0.00837910063251573\\
72.19	0.00837910151431463\\
72.2	0.00837910239698872\\
72.21	0.00837910328054012\\
72.22	0.00837910416497093\\
72.23	0.00837910505028327\\
72.24	0.00837910593647923\\
72.25	0.00837910682356094\\
72.26	0.00837910771153049\\
72.27	0.00837910860038999\\
72.28	0.00837910949014155\\
72.29	0.00837911038078728\\
72.3	0.00837911127232928\\
72.31	0.00837911216476965\\
72.32	0.00837911305811049\\
72.33	0.00837911395235391\\
72.34	0.008379114847502\\
72.35	0.00837911574355686\\
72.36	0.00837911664052059\\
72.37	0.00837911753839526\\
72.38	0.00837911843718298\\
72.39	0.00837911933688583\\
72.4	0.00837912023750589\\
72.41	0.00837912113904525\\
72.42	0.00837912204150597\\
72.43	0.00837912294489013\\
72.44	0.0083791238491998\\
72.45	0.00837912475443704\\
72.46	0.00837912566060391\\
72.47	0.00837912656770247\\
72.48	0.00837912747573476\\
72.49	0.00837912838470284\\
72.5	0.00837912929460873\\
72.51	0.00837913020545448\\
72.52	0.00837913111724211\\
72.53	0.00837913202997364\\
72.54	0.00837913294365108\\
72.55	0.00837913385827645\\
72.56	0.00837913477385173\\
72.57	0.00837913569037894\\
72.58	0.00837913660786004\\
72.59	0.00837913752629702\\
72.6	0.00837913844569184\\
72.61	0.00837913936604647\\
72.62	0.00837914028736285\\
72.63	0.00837914120964293\\
72.64	0.00837914213288863\\
72.65	0.00837914305710189\\
72.66	0.0083791439822846\\
72.67	0.00837914490843868\\
72.68	0.008379145835566\\
72.69	0.00837914676366845\\
72.7	0.0083791476927479\\
72.71	0.00837914862280619\\
72.72	0.00837914955384517\\
72.73	0.00837915048586666\\
72.74	0.00837915141887249\\
72.75	0.00837915235286445\\
72.76	0.00837915328784433\\
72.77	0.0083791542238139\\
72.78	0.00837915516077492\\
72.79	0.00837915609872913\\
72.8	0.00837915703767826\\
72.81	0.00837915797762401\\
72.82	0.00837915891856809\\
72.83	0.00837915986051216\\
72.84	0.00837916080345789\\
72.85	0.00837916174740691\\
72.86	0.00837916269236084\\
72.87	0.0083791636383213\\
72.88	0.00837916458528986\\
72.89	0.00837916553326809\\
72.9	0.00837916648225753\\
72.91	0.00837916743225969\\
72.92	0.00837916838327609\\
72.93	0.00837916933530819\\
72.94	0.00837917028835746\\
72.95	0.00837917124242532\\
72.96	0.00837917219751319\\
72.97	0.00837917315362244\\
72.98	0.00837917411075444\\
72.99	0.00837917506891053\\
73	0.008379176028092\\
73.01	0.00837917698830014\\
73.02	0.00837917794953621\\
73.03	0.00837917891180143\\
73.04	0.008379179875097\\
73.05	0.00837918083942408\\
73.06	0.00837918180478383\\
73.07	0.00837918277117734\\
73.08	0.00837918373860569\\
73.09	0.00837918470706993\\
73.1	0.00837918567657108\\
73.11	0.00837918664711011\\
73.12	0.00837918761868797\\
73.13	0.00837918859130558\\
73.14	0.00837918956496381\\
73.15	0.0083791905396635\\
73.16	0.00837919151540546\\
73.17	0.00837919249219046\\
73.18	0.00837919347001922\\
73.19	0.00837919444889244\\
73.2	0.00837919542881077\\
73.21	0.00837919640977481\\
73.22	0.00837919739178515\\
73.23	0.0083791983748423\\
73.24	0.00837919935894674\\
73.25	0.00837920034409893\\
73.26	0.00837920133029925\\
73.27	0.00837920231754806\\
73.28	0.00837920330584566\\
73.29	0.00837920429519231\\
73.3	0.00837920528558822\\
73.31	0.00837920627703354\\
73.32	0.00837920726952839\\
73.33	0.00837920826307283\\
73.34	0.00837920925766686\\
73.35	0.00837921025331044\\
73.36	0.00837921125000348\\
73.37	0.0083792122477458\\
73.38	0.00837921324653722\\
73.39	0.00837921424637746\\
73.4	0.00837921524726619\\
73.41	0.00837921624920304\\
73.42	0.00837921725218756\\
73.43	0.00837921825621925\\
73.44	0.00837921926129755\\
73.45	0.00837922026742182\\
73.46	0.00837922127459137\\
73.47	0.00837922228280545\\
73.48	0.00837922329206323\\
73.49	0.00837922430236382\\
73.5	0.00837922531370626\\
73.51	0.00837922632608951\\
73.52	0.00837922733951248\\
73.53	0.00837922835397399\\
73.54	0.00837922936947279\\
73.55	0.00837923038600757\\
73.56	0.00837923140357691\\
73.57	0.00837923242217935\\
73.58	0.00837923344181334\\
73.59	0.00837923446247722\\
73.6	0.0083792354841693\\
73.61	0.00837923650688777\\
73.62	0.00837923753063074\\
73.63	0.00837923855539625\\
73.64	0.00837923958118225\\
73.65	0.00837924060798658\\
73.66	0.00837924163580703\\
73.67	0.00837924266464125\\
73.68	0.00837924369448684\\
73.69	0.00837924472534128\\
73.7	0.00837924575720197\\
73.71	0.0083792467900662\\
73.72	0.00837924782393117\\
73.73	0.00837924885879397\\
73.74	0.00837924989465161\\
73.75	0.00837925093150097\\
73.76	0.00837925196933884\\
73.77	0.0083792530081619\\
73.78	0.00837925404796673\\
73.79	0.00837925508874978\\
73.8	0.00837925613050739\\
73.81	0.00837925717323582\\
73.82	0.00837925821693118\\
73.83	0.00837925926158947\\
73.84	0.00837926030720658\\
73.85	0.00837926135377827\\
73.86	0.00837926240130019\\
73.87	0.00837926344976785\\
73.88	0.00837926449917665\\
73.89	0.00837926554952184\\
73.9	0.00837926660079857\\
73.91	0.00837926765300182\\
73.92	0.00837926870612648\\
73.93	0.00837926976016726\\
73.94	0.00837927081511877\\
73.95	0.00837927187097547\\
73.96	0.00837927292773165\\
73.97	0.00837927398538148\\
73.98	0.008379275043919\\
73.99	0.00837927610333807\\
74	0.00837927716363241\\
74.01	0.0083792782247956\\
74.02	0.00837927928682105\\
74.03	0.00837928034970201\\
74.04	0.00837928141343159\\
74.05	0.00837928247800272\\
74.06	0.00837928354340817\\
74.07	0.00837928460964055\\
74.08	0.0083792856766923\\
74.09	0.00837928674455568\\
74.1	0.00837928781322279\\
74.11	0.00837928888268554\\
74.12	0.00837928995293567\\
74.13	0.00837929102396474\\
74.14	0.00837929209576413\\
74.15	0.00837929316832501\\
74.16	0.00837929424163839\\
74.17	0.00837929531569508\\
74.18	0.00837929639048569\\
74.19	0.00837929746600065\\
74.2	0.00837929854223015\\
74.21	0.00837929961916423\\
74.22	0.0083793006967927\\
74.23	0.00837930177510515\\
74.24	0.00837930285409099\\
74.25	0.00837930393373939\\
74.26	0.00837930501403932\\
74.27	0.00837930609497952\\
74.28	0.00837930717654851\\
74.29	0.00837930825873461\\
74.3	0.00837930934152586\\
74.31	0.00837931042491012\\
74.32	0.00837931150887498\\
74.33	0.00837931259340783\\
74.34	0.00837931367849578\\
74.35	0.00837931476412572\\
74.36	0.00837931585028429\\
74.37	0.00837931693695787\\
74.38	0.0083793180241326\\
74.39	0.00837931911179436\\
74.4	0.00837932019992877\\
74.41	0.00837932128852118\\
74.42	0.00837932237755667\\
74.43	0.00837932346702008\\
74.44	0.00837932455689594\\
74.45	0.00837932564716853\\
74.46	0.00837932673782182\\
74.47	0.00837932782883952\\
74.48	0.00837932892020505\\
74.49	0.00837933001190153\\
74.5	0.00837933110391179\\
74.51	0.00837933219621836\\
74.52	0.00837933328880347\\
74.53	0.00837933438164904\\
74.54	0.00837933547473667\\
74.55	0.00837933656804767\\
74.56	0.00837933766156301\\
74.57	0.00837933875526334\\
74.58	0.00837933984912901\\
74.59	0.00837934094314001\\
74.6	0.008379342037276\\
74.61	0.00837934313151631\\
74.62	0.00837934422583993\\
74.63	0.0083793453202255\\
74.64	0.0083793464146513\\
74.65	0.00837934750909526\\
74.66	0.00837934860353497\\
74.67	0.00837934969794762\\
74.68	0.00837935079231006\\
74.69	0.00837935188659876\\
74.7	0.00837935298078981\\
74.71	0.00837935407485892\\
74.72	0.00837935516878141\\
74.73	0.00837935626253221\\
74.74	0.00837935735608587\\
74.75	0.00837935844941652\\
74.76	0.00837935954249789\\
74.77	0.0083793606353033\\
74.78	0.00837936172780567\\
74.79	0.00837936281997748\\
74.8	0.0083793639117908\\
74.81	0.00837936500321726\\
74.82	0.00837936609422806\\
74.83	0.00837936718479397\\
74.84	0.0083793682748853\\
74.85	0.00837936936447192\\
74.86	0.00837937045352325\\
74.87	0.00837937154200824\\
74.88	0.00837937262989538\\
74.89	0.00837937371715269\\
74.9	0.0083793748037477\\
74.91	0.00837937588964749\\
74.92	0.00837937697481862\\
74.93	0.00837937805922719\\
74.94	0.00837937914283876\\
74.95	0.00837938022561843\\
74.96	0.00837938130753078\\
74.97	0.00837938238853985\\
74.98	0.00837938346860918\\
74.99	0.0083793845477018\\
75	0.00837938562578017\\
75.01	0.00837938670280624\\
75.02	0.00837938777874141\\
75.03	0.00837938885354652\\
75.04	0.00837938992718188\\
75.05	0.00837939099960721\\
75.06	0.00837939207078168\\
75.07	0.00837939314066386\\
75.08	0.00837939420921178\\
75.09	0.00837939527638286\\
75.1	0.00837939634213391\\
75.11	0.00837939740642165\\
75.12	0.00837939846920286\\
75.13	0.00837939953043446\\
75.14	0.00837940059007352\\
75.15	0.00837940164807726\\
75.16	0.0083794027044031\\
75.17	0.00837940375900869\\
75.18	0.00837940481185187\\
75.19	0.00837940586289081\\
75.2	0.0083794069120839\\
75.21	0.00837940795938991\\
75.22	0.00837940900476792\\
75.23	0.00837941004817738\\
75.24	0.00837941108957816\\
75.25	0.00837941212893056\\
75.26	0.00837941316619533\\
75.27	0.00837941420133374\\
75.28	0.00837941523430757\\
75.29	0.00837941626507917\\
75.3	0.00837941729361148\\
75.31	0.00837941831986809\\
75.32	0.00837941934381325\\
75.33	0.00837942036541193\\
75.34	0.00837942138462982\\
75.35	0.00837942240143344\\
75.36	0.00837942341579009\\
75.37	0.00837942442766799\\
75.38	0.00837942543703624\\
75.39	0.00837942644386492\\
75.4	0.00837942744812509\\
75.41	0.0083794284497889\\
75.42	0.00837942944882956\\
75.43	0.00837943044522146\\
75.44	0.00837943143894017\\
75.45	0.00837943242996252\\
75.46	0.00837943341826666\\
75.47	0.00837943440383207\\
75.48	0.00837943538663968\\
75.49	0.00837943636667188\\
75.5	0.00837943734391259\\
75.51	0.00837943831834733\\
75.52	0.00837943928996235\\
75.53	0.008379440258744\\
75.54	0.00837944122467874\\
75.55	0.00837944218775314\\
75.56	0.00837944314795387\\
75.57	0.00837944410526774\\
75.58	0.00837944505968165\\
75.59	0.00837944601118265\\
75.6	0.0083794469597579\\
75.61	0.00837944790539471\\
75.62	0.00837944884808049\\
75.63	0.00837944978780281\\
75.64	0.00837945072454939\\
75.65	0.00837945165830809\\
75.66	0.0083794525890669\\
75.67	0.00837945351681398\\
75.68	0.00837945444153765\\
75.69	0.00837945536322639\\
75.7	0.00837945628186884\\
75.71	0.00837945719745381\\
75.72	0.00837945810997028\\
75.73	0.00837945901940742\\
75.74	0.00837945992575457\\
75.75	0.00837946082900128\\
75.76	0.00837946172913725\\
75.77	0.00837946262615241\\
75.78	0.00837946352003688\\
75.79	0.00837946441078099\\
75.8	0.00837946529837526\\
75.81	0.00837946618281045\\
75.82	0.00837946706407752\\
75.83	0.00837946794216765\\
75.84	0.00837946881707228\\
75.85	0.00837946968878304\\
75.86	0.00837947055729184\\
75.87	0.0083794714225908\\
75.88	0.00837947228467231\\
75.89	0.00837947314352899\\
75.9	0.00837947399915375\\
75.91	0.00837947485153974\\
75.92	0.00837947570068039\\
75.93	0.0083794765465694\\
75.94	0.00837947738920075\\
75.95	0.00837947822856872\\
75.96	0.00837947906466786\\
75.97	0.00837947989749302\\
75.98	0.00837948072703937\\
75.99	0.00837948155330238\\
76	0.00837948237627781\\
76.01	0.00837948319596178\\
76.02	0.00837948401235071\\
76.03	0.00837948482544136\\
76.04	0.00837948563523082\\
76.05	0.00837948644171654\\
76.06	0.0083794872448963\\
76.07	0.00837948804476826\\
76.08	0.00837948884133092\\
76.09	0.00837948963458315\\
76.1	0.00837949042452422\\
76.11	0.00837949121115375\\
76.12	0.00837949199447177\\
76.13	0.00837949277447871\\
76.14	0.00837949355117536\\
76.15	0.00837949432456298\\
76.16	0.00837949509464319\\
76.17	0.00837949586141807\\
76.18	0.0083794966248901\\
76.19	0.00837949738506222\\
76.2	0.00837949814193779\\
76.21	0.00837949889552064\\
76.22	0.00837949964581504\\
76.23	0.00837950039282574\\
76.24	0.00837950113655795\\
76.25	0.00837950187701735\\
76.26	0.00837950261421013\\
76.27	0.00837950334814296\\
76.28	0.00837950407882301\\
76.29	0.00837950480625795\\
76.3	0.00837950553045598\\
76.31	0.00837950625142582\\
76.32	0.00837950696917672\\
76.33	0.00837950768371846\\
76.34	0.00837950839506138\\
76.35	0.00837950910321636\\
76.36	0.00837950980819486\\
76.37	0.0083795105100089\\
76.38	0.00837951120867107\\
76.39	0.00837951190419457\\
76.4	0.00837951259659316\\
76.41	0.00837951328588124\\
76.42	0.0083795139720738\\
76.43	0.00837951465518644\\
76.44	0.0083795153352354\\
76.45	0.00837951601223756\\
76.46	0.00837951668621044\\
76.47	0.0083795173571722\\
76.48	0.00837951802514168\\
76.49	0.00837951869013838\\
76.5	0.00837951935218246\\
76.51	0.0083795200112948\\
76.52	0.00837952066749696\\
76.53	0.00837952132081119\\
76.54	0.00837952197126047\\
76.55	0.00837952261886849\\
76.56	0.00837952326365967\\
76.57	0.00837952390565919\\
76.58	0.00837952454489294\\
76.59	0.00837952518138759\\
76.6	0.00837952581517058\\
76.61	0.0083795264462701\\
76.62	0.00837952707471513\\
76.63	0.00837952770053546\\
76.64	0.00837952832376166\\
76.65	0.00837952894442512\\
76.66	0.00837952956255803\\
76.67	0.00837953017819343\\
76.68	0.00837953079136519\\
76.69	0.008379531402108\\
76.7	0.00837953201045744\\
76.71	0.00837953261644994\\
76.72	0.00837953322012278\\
76.73	0.00837953382151414\\
76.74	0.0083795344206631\\
76.75	0.00837953501760961\\
76.76	0.00837953561239455\\
76.77	0.0083795362050597\\
76.78	0.00837953679564778\\
76.79	0.00837953738420242\\
76.8	0.00837953797076822\\
76.81	0.00837953855539071\\
76.82	0.00837953913811639\\
76.83	0.00837953971899272\\
76.84	0.00837954029806815\\
76.85	0.0083795408753921\\
76.86	0.00837954145101498\\
76.87	0.00837954202498823\\
76.88	0.00837954259736427\\
76.89	0.00837954316819656\\
76.9	0.00837954373753956\\
76.91	0.00837954430544881\\
76.92	0.00837954487198084\\
76.93	0.00837954543719328\\
76.94	0.00837954600114478\\
76.95	0.00837954656389507\\
76.96	0.00837954712550497\\
76.97	0.00837954768603637\\
76.98	0.00837954824555224\\
76.99	0.00837954880411665\\
77	0.0083795493617948\\
77.01	0.00837954991865296\\
77.02	0.00837955047475855\\
77.03	0.00837955103018011\\
77.04	0.00837955158498728\\
77.05	0.00837955213925089\\
77.06	0.00837955269304286\\
77.07	0.00837955324643631\\
77.08	0.00837955379950549\\
77.09	0.0083795543523258\\
77.1	0.00837955490497384\\
77.11	0.00837955545752736\\
77.12	0.00837955601006529\\
77.13	0.00837955656266775\\
77.14	0.00837955711541605\\
77.15	0.00837955766839268\\
77.16	0.00837955822168134\\
77.17	0.00837955877536692\\
77.18	0.00837955932953552\\
77.19	0.00837955988427443\\
77.2	0.00837956043967219\\
77.21	0.00837956099581852\\
77.22	0.00837956155280436\\
77.23	0.00837956211072187\\
77.24	0.00837956266966445\\
77.25	0.00837956322972669\\
77.26	0.00837956379100442\\
77.27	0.00837956435359469\\
77.28	0.00837956491759578\\
77.29	0.00837956548310718\\
77.3	0.00837956605022961\\
77.31	0.00837956661906503\\
77.32	0.00837956718971659\\
77.33	0.00837956776228868\\
77.34	0.00837956833688691\\
77.35	0.0083795689136181\\
77.36	0.00837956949259026\\
77.37	0.00837957007391265\\
77.38	0.00837957065769571\\
77.39	0.00837957124405108\\
77.4	0.0083795718330916\\
77.41	0.0083795724249313\\
77.42	0.00837957301968539\\
77.43	0.00837957361747024\\
77.44	0.00837957421840343\\
77.45	0.00837957482260366\\
77.46	0.00837957543019079\\
77.47	0.00837957604128584\\
77.48	0.00837957665601095\\
77.49	0.00837957727448937\\
77.5	0.00837957789684547\\
77.51	0.00837957852320473\\
77.52	0.00837957915369368\\
77.53	0.00837957978843996\\
77.54	0.00837958042757224\\
77.55	0.00837958107122023\\
77.56	0.00837958171951467\\
77.57	0.0083795823725873\\
77.58	0.00837958303057087\\
77.59	0.00837958369359906\\
77.6	0.00837958436180654\\
77.61	0.00837958503532887\\
77.62	0.00837958571430255\\
77.63	0.00837958639886493\\
77.64	0.00837958708915424\\
77.65	0.00837958778530954\\
77.66	0.00837958848747069\\
77.67	0.00837958919577834\\
77.68	0.00837958991037388\\
77.69	0.00837959063139942\\
77.7	0.00837959135899777\\
77.71	0.00837959209331239\\
77.72	0.00837959283448736\\
77.73	0.00837959358266736\\
77.74	0.00837959433799762\\
77.75	0.00837959510062387\\
77.76	0.00837959587069232\\
77.77	0.00837959664834964\\
77.78	0.00837959743374286\\
77.79	0.00837959822701937\\
77.8	0.00837959902832687\\
77.81	0.0083795998378133\\
77.82	0.00837960065562683\\
77.83	0.00837960148191578\\
77.84	0.00837960231682856\\
77.85	0.00837960316051365\\
77.86	0.00837960401311951\\
77.87	0.00837960487479456\\
77.88	0.00837960574568709\\
77.89	0.00837960662594519\\
77.9	0.00837960751571673\\
77.91	0.00837960841514925\\
77.92	0.00837960932438993\\
77.93	0.00837961024358548\\
77.94	0.00837961117288211\\
77.95	0.00837961211242542\\
77.96	0.00837961306236036\\
77.97	0.00837961402283109\\
77.98	0.00837961499398098\\
77.99	0.00837961597595246\\
78	0.00837961696888698\\
78.01	0.00837961797292488\\
78.02	0.00837961898820533\\
78.03	0.00837962001486621\\
78.04	0.00837962105304405\\
78.05	0.00837962210287388\\
78.06	0.00837962316448918\\
78.07	0.00837962423802205\\
78.08	0.00837962532360605\\
78.09	0.00837962642137627\\
78.1	0.00837962753146928\\
78.11	0.0083796286540232\\
78.12	0.00837962978917767\\
78.13	0.00837963093707393\\
78.14	0.00837963209785477\\
78.15	0.00837963327166459\\
78.16	0.0083796344586494\\
78.17	0.00837963565895684\\
78.18	0.00837963687273623\\
78.19	0.00837963810013851\\
78.2	0.00837963934131637\\
78.21	0.00837964059642415\\
78.22	0.00837964186561795\\
78.23	0.00837964314905562\\
78.24	0.00837964444689676\\
78.25	0.00837964575930276\\
78.26	0.00837964708643682\\
78.27	0.00837964842846396\\
78.28	0.00837964978555104\\
78.29	0.00837965115786681\\
78.3	0.00837965254558189\\
78.31	0.0083796539488688\\
78.32	0.00837965536790202\\
78.33	0.00837965680285795\\
78.34	0.00837965825391497\\
78.35	0.00837965972125348\\
78.36	0.00837966120505587\\
78.37	0.00837966270550658\\
78.38	0.00837966422279212\\
78.39	0.00837966575710106\\
78.4	0.00837966730862412\\
78.41	0.00837966887755411\\
78.42	0.00837967046408603\\
78.43	0.00837967206841705\\
78.44	0.00837967369074654\\
78.45	0.0083796753312761\\
78.46	0.00837967699020959\\
78.47	0.00837967866775314\\
78.48	0.00837968036411521\\
78.49	0.00837968207950655\\
78.5	0.00837968381414032\\
78.51	0.00837968556823201\\
78.52	0.00837968734199955\\
78.53	0.0083796891356633\\
78.54	0.0083796909494461\\
78.55	0.00837969278357324\\
78.56	0.00837969463827258\\
78.57	0.00837969651377449\\
78.58	0.00837969841031193\\
78.59	0.00837970032812047\\
78.6	0.00837970226743831\\
78.61	0.0083797042285063\\
78.62	0.00837970621156799\\
78.63	0.00837970821686968\\
78.64	0.00837971024466038\\
78.65	0.00837971229519192\\
78.66	0.00837971436871891\\
78.67	0.00837971646549884\\
78.68	0.00837971858579207\\
78.69	0.00837972072986185\\
78.7	0.00837972289797439\\
78.71	0.00837972509039889\\
78.72	0.00837972730740752\\
78.73	0.00837972954927554\\
78.74	0.00837973181628124\\
78.75	0.00837973410870606\\
78.76	0.00837973642683456\\
78.77	0.0083797387709545\\
78.78	0.00837974114135683\\
78.79	0.00837974353833578\\
78.8	0.00837974596218885\\
78.81	0.00837974841321687\\
78.82	0.00837975089172405\\
78.83	0.00837975339801797\\
78.84	0.00837975593240966\\
78.85	0.00837975849521364\\
78.86	0.00837976108674792\\
78.87	0.00837976370733408\\
78.88	0.0083797663572973\\
78.89	0.00837976903696637\\
78.9	0.00837977174667378\\
78.91	0.00837977448675571\\
78.92	0.00837977725755213\\
78.93	0.00837978005940678\\
78.94	0.00837978289266724\\
78.95	0.00837978575768501\\
78.96	0.00837978865481547\\
78.97	0.008379791584418\\
78.98	0.008379794546856\\
78.99	0.0083797975424969\\
79	0.00837980057171227\\
79.01	0.00837980363487782\\
79.02	0.00837980673237343\\
79.03	0.00837980986458327\\
79.04	0.00837981303189577\\
79.05	0.00837981623470371\\
79.06	0.00837981947340427\\
79.07	0.00837982274839904\\
79.08	0.00837982606009413\\
79.09	0.00837982940890014\\
79.1	0.00837983279523231\\
79.11	0.00837983621951049\\
79.12	0.00837983968215922\\
79.13	0.00837984318360778\\
79.14	0.00837984672429027\\
79.15	0.0083798503046456\\
79.16	0.00837985392511762\\
79.17	0.00837985758615511\\
79.18	0.00837986128821189\\
79.19	0.00837986503174681\\
79.2	0.00837986881722389\\
79.21	0.0083798726451123\\
79.22	0.00837987651588647\\
79.23	0.00837988043002611\\
79.24	0.0083798843880163\\
79.25	0.00837988839034754\\
79.26	0.00837989243751579\\
79.27	0.00837989653002258\\
79.28	0.00837990066837502\\
79.29	0.00837990485308588\\
79.3	0.00837990908467367\\
79.31	0.00837991336366267\\
79.32	0.00837991769058304\\
79.33	0.00837992206597085\\
79.34	0.00837992649036815\\
79.35	0.00837993096432305\\
79.36	0.00837993548838977\\
79.37	0.00837994006312874\\
79.38	0.00837994468910662\\
79.39	0.00837994936689642\\
79.4	0.00837995409707754\\
79.41	0.00837995888023585\\
79.42	0.00837996371696375\\
79.43	0.00837996860786027\\
79.44	0.00837997355353113\\
79.45	0.0083799785545888\\
79.46	0.0083799836116526\\
79.47	0.00837998872534874\\
79.48	0.00837999389631045\\
79.49	0.00837999912517803\\
79.5	0.00838000441259889\\
79.51	0.00838000975922772\\
79.52	0.00838001516572648\\
79.53	0.00838002063276453\\
79.54	0.0083800261610187\\
79.55	0.0083800317511734\\
79.56	0.00838003740392065\\
79.57	0.00838004311996019\\
79.58	0.00838004889999961\\
79.59	0.00838005474475435\\
79.6	0.00838006065494788\\
79.61	0.00838006663131171\\
79.62	0.00838007267458554\\
79.63	0.0083800787855173\\
79.64	0.0083800849648633\\
79.65	0.00838009121338827\\
79.66	0.00838009753186547\\
79.67	0.00838010392107682\\
79.68	0.00838011038181293\\
79.69	0.00838011691487325\\
79.7	0.00838012352106616\\
79.71	0.00838013020120906\\
79.72	0.00838013695612845\\
79.73	0.00838014378666008\\
79.74	0.00838015069364901\\
79.75	0.00838015767794975\\
79.76	0.00838016474042631\\
79.77	0.00838017188195238\\
79.78	0.00838017910341138\\
79.79	0.00838018640569659\\
79.8	0.00838019378971125\\
79.81	0.0083802012563687\\
79.82	0.00838020880659245\\
79.83	0.00838021644131632\\
79.84	0.00838022416148456\\
79.85	0.00838023196805194\\
79.86	0.0083802398619839\\
79.87	0.00838024784425663\\
79.88	0.00838025591585723\\
79.89	0.0083802640777838\\
79.9	0.00838027233104558\\
79.91	0.00838028067666309\\
79.92	0.00838028911566819\\
79.93	0.00838029764910429\\
79.94	0.00838030627802643\\
79.95	0.00838031500350142\\
79.96	0.00838032382660796\\
79.97	0.00838033274843681\\
79.98	0.00838034177009085\\
79.99	0.00838035089268532\\
80	0.00838036011734786\\
80.01	0.00838036944521871\\
};
\addplot [color=black,solid]
  table[row sep=crcr]{%
80.01	0.00838036944521871\\
80.02	0.00838037887745081\\
80.03	0.00838038841520997\\
80.04	0.00838039805967502\\
80.05	0.00838040781203792\\
80.06	0.00838041767350392\\
80.07	0.00838042764529172\\
80.08	0.00838043772863362\\
80.09	0.00838044792477566\\
80.1	0.00838045823497778\\
80.11	0.00838046866051395\\
80.12	0.00838047920267239\\
80.13	0.00838048986275565\\
80.14	0.00838050064208083\\
80.15	0.00838051154197969\\
80.16	0.00838052256379888\\
80.17	0.00838053370890004\\
80.18	0.00838054497866001\\
80.19	0.00838055637447098\\
80.2	0.00838056789774066\\
80.21	0.00838057954989245\\
80.22	0.00838059133236565\\
80.23	0.00838060324661558\\
80.24	0.00838061529411382\\
80.25	0.00838062747634831\\
80.26	0.00838063979482364\\
80.27	0.00838065225106114\\
80.28	0.00838066484659911\\
80.29	0.008380677582993\\
80.3	0.00838069046181561\\
80.31	0.00838070348465727\\
80.32	0.00838071665312605\\
80.33	0.00838072996884793\\
80.34	0.00838074343346705\\
80.35	0.00838075704864585\\
80.36	0.00838077081606532\\
80.37	0.00838078473742518\\
80.38	0.00838079881444411\\
80.39	0.00838081304885996\\
80.4	0.00838082744242992\\
80.41	0.0083808419969308\\
80.42	0.00838085671415921\\
80.43	0.00838087159593178\\
80.44	0.0083808866440854\\
80.45	0.00838090186047744\\
80.46	0.00838091724698595\\
80.47	0.00838093280550994\\
80.48	0.00838094853796959\\
80.49	0.00838096444630646\\
80.5	0.00838098053248378\\
80.51	0.00838099679848663\\
80.52	0.00838101324632226\\
80.53	0.00838102987802024\\
80.54	0.00838104669563281\\
80.55	0.00838106370123506\\
80.56	0.00838108089692521\\
80.57	0.00838109828482486\\
80.58	0.00838111586707927\\
80.59	0.0083811336458576\\
80.6	0.00838115162335319\\
80.61	0.00838116980178382\\
80.62	0.00838118818339198\\
80.63	0.00838120677044518\\
80.64	0.00838122556523616\\
80.65	0.00838124457008324\\
80.66	0.00838126378733057\\
80.67	0.00838128321934841\\
80.68	0.00838130286853344\\
80.69	0.00838132273730907\\
80.7	0.00838134282812567\\
80.71	0.00838136314346095\\
80.72	0.00838138368582023\\
80.73	0.00838140445773673\\
80.74	0.0083814254617719\\
80.75	0.00838144670051573\\
80.76	0.00838146817658709\\
80.77	0.00838148989263401\\
80.78	0.00838151185133402\\
80.79	0.0083815340553945\\
80.8	0.00838155650755299\\
80.81	0.00838157921057752\\
80.82	0.00838160216726699\\
80.83	0.00838162538045144\\
80.84	0.00838164885299249\\
80.85	0.00838167258778359\\
80.86	0.00838169658775046\\
80.87	0.00838172085585142\\
80.88	0.00838174539507772\\
80.89	0.00838177020845395\\
80.9	0.00838179529903841\\
80.91	0.00838182066992344\\
80.92	0.00838184632423585\\
80.93	0.00838187226513727\\
80.94	0.00838189849582457\\
80.95	0.00838192501953021\\
80.96	0.00838195183952267\\
80.97	0.00838197895910684\\
80.98	0.00838200638162441\\
80.99	0.00838203411045432\\
81	0.00838206214901312\\
81.01	0.00838209050075544\\
81.02	0.00838211916917437\\
81.03	0.00838214815780193\\
81.04	0.00838217747020946\\
81.05	0.00838220711000808\\
81.06	0.00838223708084916\\
81.07	0.0083822673864247\\
81.08	0.00838229803046784\\
81.09	0.00838232901675332\\
81.1	0.00838236034909787\\
81.11	0.00838239203136078\\
81.12	0.0083824240674443\\
81.13	0.00838245646129413\\
81.14	0.00838248921689994\\
81.15	0.00838252233829581\\
81.16	0.00838255582956076\\
81.17	0.00838258969481922\\
81.18	0.00838262393824158\\
81.19	0.00838265856404465\\
81.2	0.0083826935764922\\
81.21	0.00838272897989551\\
81.22	0.00838276477861383\\
81.23	0.00838280097705498\\
81.24	0.00838283757967586\\
81.25	0.00838287459098301\\
81.26	0.00838291201553313\\
81.27	0.00838294985793368\\
81.28	0.00838298812284341\\
81.29	0.00838302681497296\\
81.3	0.00838306593908543\\
81.31	0.00838310549999692\\
81.32	0.00838314550257721\\
81.33	0.00838318595175025\\
81.34	0.00838322685249485\\
81.35	0.00838326820984525\\
81.36	0.00838331002889173\\
81.37	0.00838335231478127\\
81.38	0.00838339507271814\\
81.39	0.00838343830796454\\
81.4	0.0083834820258413\\
81.41	0.00838352623172845\\
81.42	0.00838357093106594\\
81.43	0.00838361612935428\\
81.44	0.00838366183215523\\
81.45	0.00838370804509244\\
81.46	0.00838375477385221\\
81.47	0.00838380202418411\\
81.48	0.00838384980190175\\
81.49	0.00838389811288343\\
81.5	0.00838394696307292\\
81.51	0.00838399635848015\\
81.52	0.00838404630518194\\
81.53	0.00838409680932279\\
81.54	0.00838414787711557\\
81.55	0.00838419951484234\\
81.56	0.00838425172885507\\
81.57	0.00838430452557645\\
81.58	0.00838435791150067\\
81.59	0.00838441189319421\\
81.6	0.00838446647729663\\
81.61	0.00838452167052143\\
81.62	0.00838457747965681\\
81.63	0.00838463391156655\\
81.64	0.00838469097319083\\
81.65	0.00838474867154707\\
81.66	0.00838480701373081\\
81.67	0.00838486600691657\\
81.68	0.00838492565835873\\
81.69	0.00838498597539242\\
81.7	0.0083850469654344\\
81.71	0.008385108635984\\
81.72	0.00838517099462401\\
81.73	0.00838523404902161\\
81.74	0.00838529780692936\\
81.75	0.00838536227618605\\
81.76	0.00838542746471775\\
81.77	0.00838549338053875\\
81.78	0.00838556003175252\\
81.79	0.00838562742655274\\
81.8	0.00838569557322426\\
81.81	0.00838576448014416\\
81.82	0.00838583415578274\\
81.83	0.00838590460870458\\
81.84	0.00838597584756957\\
81.85	0.00838604788113398\\
81.86	0.00838612071825153\\
81.87	0.00838619436787448\\
81.88	0.00838626883905472\\
81.89	0.00838634414094488\\
81.9	0.00838642028279943\\
81.91	0.00838649727397586\\
81.92	0.00838657512393578\\
81.93	0.00838665384224611\\
81.94	0.00838673343858021\\
81.95	0.00838681392271911\\
81.96	0.0083868953045527\\
81.97	0.0083869775940809\\
81.98	0.00838706080141493\\
81.99	0.00838714493677854\\
82	0.00838723001050923\\
82.01	0.00838731603305954\\
82.02	0.00838740301499835\\
82.03	0.00838749096701214\\
82.04	0.00838757989990629\\
82.05	0.00838766982460644\\
82.06	0.0083877607521598\\
82.07	0.00838785269373653\\
82.08	0.00838794566063108\\
82.09	0.00838803966426355\\
82.1	0.00838813471618117\\
82.11	0.00838823082805963\\
82.12	0.00838832801170455\\
82.13	0.00838842627905292\\
82.14	0.00838852564217455\\
82.15	0.00838862611327356\\
82.16	0.00838872770468988\\
82.17	0.00838883042890074\\
82.18	0.0083889342985222\\
82.19	0.00838903932631071\\
82.2	0.00838914552516466\\
82.21	0.00838925290812598\\
82.22	0.0083893614883817\\
82.23	0.00838947127926558\\
82.24	0.00838958229425975\\
82.25	0.00838969454699637\\
82.26	0.00838980805125927\\
82.27	0.00838992282098565\\
82.28	0.00839003887026779\\
82.29	0.00839015621335479\\
82.3	0.00839027486465427\\
82.31	0.00839039483873417\\
82.32	0.00839051615032455\\
82.33	0.00839063881431933\\
82.34	0.00839076284577818\\
82.35	0.00839088825992832\\
82.36	0.00839101507216639\\
82.37	0.00839114209980636\\
82.38	0.00839126932057027\\
82.39	0.00839139673578494\\
82.4	0.00839152434679041\\
82.41	0.00839165215494015\\
82.42	0.0083917801616011\\
82.43	0.00839190836815389\\
82.44	0.00839203677599295\\
82.45	0.00839216538652664\\
82.46	0.00839229420117742\\
82.47	0.00839242322138197\\
82.48	0.00839255244859138\\
82.49	0.00839268188427122\\
82.5	0.00839281152990177\\
82.51	0.00839294138697813\\
82.52	0.00839307145701039\\
82.53	0.00839320174152376\\
82.54	0.00839333224205875\\
82.55	0.00839346296017131\\
82.56	0.00839359389743303\\
82.57	0.00839372505543122\\
82.58	0.00839385643576916\\
82.59	0.0083939880400662\\
82.6	0.00839411986995796\\
82.61	0.0083942519270965\\
82.62	0.00839438421315046\\
82.63	0.00839451672980524\\
82.64	0.0083946494787632\\
82.65	0.0083947824617438\\
82.66	0.0083949156804838\\
82.67	0.0083950491367374\\
82.68	0.00839518283227648\\
82.69	0.00839531676889073\\
82.7	0.00839545094838785\\
82.71	0.00839558537259373\\
82.72	0.00839572004335265\\
82.73	0.00839585496252746\\
82.74	0.00839599013199976\\
82.75	0.00839612555367011\\
82.76	0.00839626122945823\\
82.77	0.00839639716130314\\
82.78	0.00839653335116345\\
82.79	0.00839666980101748\\
82.8	0.00839680651286351\\
82.81	0.00839694348871996\\
82.82	0.00839708073062559\\
82.83	0.00839721824063974\\
82.84	0.00839735602084252\\
82.85	0.00839749407333501\\
82.86	0.00839763240023951\\
82.87	0.00839777100369972\\
82.88	0.00839790988588096\\
82.89	0.00839804904897044\\
82.9	0.00839818849517742\\
82.91	0.00839832822673348\\
82.92	0.00839846824589271\\
82.93	0.00839860855493198\\
82.94	0.00839874915615115\\
82.95	0.0083988900518733\\
82.96	0.00839903124444497\\
82.97	0.00839917273623643\\
82.98	0.00839931452964185\\
82.99	0.00839945662707963\\
83	0.00839959903099258\\
83.01	0.0083997417438482\\
83.02	0.00839988476813893\\
83.03	0.0084000281063824\\
83.04	0.00840017176112167\\
83.05	0.00840031573492551\\
83.06	0.00840046003038865\\
83.07	0.00840060465013207\\
83.08	0.00840074959680322\\
83.09	0.00840089487307632\\
83.1	0.00840104048165262\\
83.11	0.0084011864252607\\
83.12	0.00840133270665671\\
83.13	0.00840147932862467\\
83.14	0.00840162629397674\\
83.15	0.00840177360555354\\
83.16	0.00840192126622437\\
83.17	0.0084020692788876\\
83.18	0.00840221764647085\\
83.19	0.00840236637193138\\
83.2	0.00840251545825635\\
83.21	0.00840266490846311\\
83.22	0.00840281472559954\\
83.23	0.00840296491274432\\
83.24	0.00840311547300728\\
83.25	0.00840326640952969\\
83.26	0.00840341772548459\\
83.27	0.00840356942407708\\
83.28	0.00840372150854469\\
83.29	0.00840387398215769\\
83.3	0.0084040268482194\\
83.31	0.00840418011006653\\
83.32	0.00840433377106957\\
83.33	0.00840448783463303\\
83.34	0.00840464230419587\\
83.35	0.00840479718323182\\
83.36	0.0084049524752497\\
83.37	0.00840510818379381\\
83.38	0.0084052643124443\\
83.39	0.00840542086481745\\
83.4	0.00840557784456615\\
83.41	0.00840573525538017\\
83.42	0.00840589310098657\\
83.43	0.00840605138515008\\
83.44	0.00840621011167347\\
83.45	0.00840636928439793\\
83.46	0.00840652890720344\\
83.47	0.0084066889840092\\
83.48	0.00840684951877398\\
83.49	0.00840701051549655\\
83.5	0.00840717197821606\\
83.51	0.00840733391101244\\
83.52	0.00840749631800682\\
83.53	0.00840765920336196\\
83.54	0.00840782257128262\\
83.55	0.00840798642601602\\
83.56	0.00840815077185225\\
83.57	0.00840831561312468\\
83.58	0.00840848095421044\\
83.59	0.00840864679953081\\
83.6	0.00840881315355167\\
83.61	0.00840898002078396\\
83.62	0.00840914740578412\\
83.63	0.00840931531315456\\
83.64	0.00840948374754408\\
83.65	0.00840965271364835\\
83.66	0.0084098222162104\\
83.67	0.00840999226002107\\
83.68	0.00841016284991947\\
83.69	0.00841033399079349\\
83.7	0.00841050568758026\\
83.71	0.00841067794526666\\
83.72	0.0084108507688898\\
83.73	0.00841102416353752\\
83.74	0.0084111981343489\\
83.75	0.00841137268651476\\
83.76	0.00841154782527816\\
83.77	0.00841172355593496\\
83.78	0.0084118998838343\\
83.79	0.00841207681437914\\
83.8	0.0084122543530268\\
83.81	0.0084124325052895\\
83.82	0.00841261127673487\\
83.83	0.00841279067298656\\
83.84	0.00841297069972471\\
83.85	0.00841315136268659\\
83.86	0.00841333266766711\\
83.87	0.00841351462051939\\
83.88	0.00841369722715539\\
83.89	0.00841388049354639\\
83.9	0.00841406442572368\\
83.91	0.00841424902977908\\
83.92	0.00841443431186554\\
83.93	0.0084146202781978\\
83.94	0.00841480693505292\\
83.95	0.00841499428877094\\
83.96	0.00841518234575548\\
83.97	0.00841537111247437\\
83.98	0.00841556059546028\\
83.99	0.00841575080131136\\
84	0.00841594173669187\\
84.01	0.00841613340833281\\
84.02	0.00841632582303264\\
84.03	0.00841651898765786\\
84.04	0.00841671290914373\\
84.05	0.00841690759449491\\
84.06	0.00841710305078615\\
84.07	0.00841729928516298\\
84.08	0.00841749630484241\\
84.09	0.00841769411711358\\
84.1	0.00841789272933851\\
84.11	0.00841809214895281\\
84.12	0.00841829238346634\\
84.13	0.00841849344046402\\
84.14	0.00841869532760647\\
84.15	0.0084188980526308\\
84.16	0.00841910162335135\\
84.17	0.00841930604766043\\
84.18	0.00841951133352905\\
84.19	0.00841971748900773\\
84.2	0.00841992452222724\\
84.21	0.00842013244139941\\
84.22	0.00842034125481785\\
84.23	0.00842055097085882\\
84.24	0.00842076159798196\\
84.25	0.00842097314473117\\
84.26	0.00842118561973534\\
84.27	0.00842139903170922\\
84.28	0.00842161338945427\\
84.29	0.00842182870185942\\
84.3	0.00842204497790199\\
84.31	0.0084222622266485\\
84.32	0.00842248045725554\\
84.33	0.00842269967897064\\
84.34	0.00842291990113312\\
84.35	0.00842314113317501\\
84.36	0.00842336338462192\\
84.37	0.00842358666509391\\
84.38	0.00842381098430645\\
84.39	0.00842403635207131\\
84.4	0.00842426277829747\\
84.41	0.00842449027299205\\
84.42	0.00842471884626127\\
84.43	0.00842494850831139\\
84.44	0.00842517926944965\\
84.45	0.00842541114008525\\
84.46	0.00842564413073033\\
84.47	0.0084258782520009\\
84.48	0.00842611351461792\\
84.49	0.0084263499294082\\
84.5	0.00842658750730548\\
84.51	0.00842682625935141\\
84.52	0.0084270661966966\\
84.53	0.00842730733060161\\
84.54	0.00842754967243805\\
84.55	0.0084277932336896\\
84.56	0.00842803802595306\\
84.57	0.00842828406093946\\
84.58	0.0084285313504751\\
84.59	0.00842877990650267\\
84.6	0.00842902974108233\\
84.61	0.00842928086639283\\
84.62	0.00842953329473262\\
84.63	0.00842978703852098\\
84.64	0.00843004211029918\\
84.65	0.00843029852273158\\
84.66	0.00843055628860683\\
84.67	0.00843081542083903\\
84.68	0.00843107593246889\\
84.69	0.00843133783666494\\
84.7	0.0084316011467247\\
84.71	0.00843186587607593\\
84.72	0.00843213203827782\\
84.73	0.00843239964702221\\
84.74	0.00843266871613485\\
84.75	0.00843293925957665\\
84.76	0.00843321129144494\\
84.77	0.00843348482597473\\
84.78	0.00843375987754\\
84.79	0.008434036460655\\
84.8	0.00843431458997554\\
84.81	0.00843459428030031\\
84.82	0.00843487554657223\\
84.83	0.00843515840387976\\
84.84	0.00843544286745824\\
84.85	0.00843572895269129\\
84.86	0.00843601667511215\\
84.87	0.00843630605040508\\
84.88	0.00843659709440672\\
84.89	0.00843688982310757\\
84.9	0.00843718425265333\\
84.91	0.00843748039934636\\
84.92	0.00843777741934355\\
84.93	0.00843807519926577\\
84.94	0.00843837374598029\\
84.95	0.00843867306642709\\
84.96	0.00843897316761961\\
84.97	0.00843927405664558\\
84.98	0.00843957574066779\\
84.99	0.00843987822692487\\
85	0.00844018152273213\\
85.01	0.00844048563548238\\
85.02	0.00844079057264672\\
85.03	0.00844109634177542\\
85.04	0.00844140295049873\\
85.05	0.00844171040652775\\
85.06	0.00844201871765527\\
85.07	0.00844232789175667\\
85.08	0.00844263793679081\\
85.09	0.00844294886080084\\
85.1	0.00844326067191521\\
85.11	0.00844357337834848\\
85.12	0.00844388698840231\\
85.13	0.00844420151046633\\
85.14	0.00844451695301913\\
85.15	0.00844483332462915\\
85.16	0.00844515063395568\\
85.17	0.00844546888974982\\
85.18	0.00844578810085543\\
85.19	0.00844610827621015\\
85.2	0.00844642942484638\\
85.21	0.00844675155589229\\
85.22	0.00844707467857282\\
85.23	0.00844739880221076\\
85.24	0.00844772393622774\\
85.25	0.00844805009014529\\
85.26	0.00844837727358591\\
85.27	0.00844870549627415\\
85.28	0.00844903476803768\\
85.29	0.0084493650988084\\
85.3	0.00844969649862353\\
85.31	0.00845002897762671\\
85.32	0.00845036254606919\\
85.33	0.00845069721431093\\
85.34	0.00845103299282173\\
85.35	0.00845136989218247\\
85.36	0.00845170792308621\\
85.37	0.00845204709633941\\
85.38	0.00845238742286316\\
85.39	0.00845272891369436\\
85.4	0.00845307157998694\\
85.41	0.00845341543301315\\
85.42	0.00845376048416474\\
85.43	0.00845410674495432\\
85.44	0.00845445422701654\\
85.45	0.00845480294210945\\
85.46	0.00845515290211579\\
85.47	0.0084555041190443\\
85.48	0.00845585660503107\\
85.49	0.00845621037234086\\
85.5	0.00845656543336849\\
85.51	0.00845692180064021\\
85.52	0.00845727948681507\\
85.53	0.00845763850468636\\
85.54	0.008457998867183\\
85.55	0.00845836058737096\\
85.56	0.00845872367845477\\
85.57	0.00845908815377892\\
85.58	0.00845945402682937\\
85.59	0.00845982131123503\\
85.6	0.0084601900207693\\
85.61	0.00846056016935154\\
85.62	0.00846093177104867\\
85.63	0.00846130484007669\\
85.64	0.00846167939080228\\
85.65	0.00846205543774434\\
85.66	0.00846243299557567\\
85.67	0.00846281207912453\\
85.68	0.00846319270337631\\
85.69	0.00846357488347517\\
85.7	0.00846395863472572\\
85.71	0.00846434397259474\\
85.72	0.00846473091271283\\
85.73	0.00846511947087619\\
85.74	0.00846550966304832\\
85.75	0.00846590150536181\\
85.76	0.00846629501412013\\
85.77	0.00846669020579939\\
85.78	0.00846708709705019\\
85.79	0.00846748570469944\\
85.8	0.00846788604575221\\
85.81	0.00846828813739362\\
85.82	0.00846869199699073\\
85.83	0.00846909764209444\\
85.84	0.00846950509044143\\
85.85	0.00846991435995612\\
85.86	0.00847032546875261\\
85.87	0.00847073843513673\\
85.88	0.00847115327760799\\
85.89	0.00847157001486167\\
85.9	0.00847198866579081\\
85.91	0.00847240924948837\\
85.92	0.00847283178524923\\
85.93	0.0084732562925724\\
85.94	0.0084736827911631\\
85.95	0.00847411130093493\\
85.96	0.0084745418420121\\
85.97	0.00847497443473156\\
85.98	0.0084754090996453\\
85.99	0.00847584585752256\\
86	0.00847628472935214\\
86.01	0.00847672573634465\\
86.02	0.00847716889993487\\
86.03	0.00847761424178409\\
86.04	0.00847806178378249\\
86.05	0.00847851154805148\\
86.06	0.0084789635569462\\
86.07	0.0084794178330579\\
86.08	0.00847987439921644\\
86.09	0.00848033327849276\\
86.1	0.00848079449420145\\
86.11	0.0084812580699032\\
86.12	0.00848172402940749\\
86.13	0.00848219239677508\\
86.14	0.00848266319632071\\
86.15	0.00848313645261569\\
86.16	0.00848361219049065\\
86.17	0.00848409043503815\\
86.18	0.0084845712116155\\
86.19	0.00848505454584748\\
86.2	0.00848554046362912\\
86.21	0.00848602899112854\\
86.22	0.00848652015478975\\
86.23	0.00848701398133562\\
86.24	0.00848751049777065\\
86.25	0.00848800973138402\\
86.26	0.00848851170975249\\
86.27	0.00848901646074341\\
86.28	0.00848952401251772\\
86.29	0.00849003439353307\\
86.3	0.00849054763254681\\
86.31	0.00849106375861918\\
86.32	0.00849158280111642\\
86.33	0.00849210478971398\\
86.34	0.00849262975439969\\
86.35	0.00849315772547703\\
86.36	0.00849368873356841\\
86.37	0.00849422280961848\\
86.38	0.00849475998489744\\
86.39	0.00849530029100447\\
86.4	0.0084958437598711\\
86.41	0.00849639042376467\\
86.42	0.00849694031529182\\
86.43	0.00849749346740199\\
86.44	0.00849804991339098\\
86.45	0.00849860968690453\\
86.46	0.00849917282194195\\
86.47	0.00849973935285979\\
86.48	0.00850030931437549\\
86.49	0.00850088274157118\\
86.5	0.0085014596698974\\
86.51	0.00850204013517695\\
86.52	0.00850262417360868\\
86.53	0.00850321182177145\\
86.54	0.00850380311662799\\
86.55	0.0085043980955289\\
86.56	0.00850499679621665\\
86.57	0.00850559925682961\\
86.58	0.00850620551590615\\
86.59	0.00850681561238878\\
86.6	0.00850742958562826\\
86.61	0.00850804747538789\\
86.62	0.0085086693218477\\
86.63	0.00850929516560876\\
86.64	0.00850992504769754\\
86.65	0.00851055900957026\\
86.66	0.00851119709311731\\
86.67	0.00851183934066778\\
86.68	0.00851248579499388\\
86.69	0.00851313649931557\\
86.7	0.00851379149730513\\
86.71	0.00851445083309184\\
86.72	0.00851511455126663\\
86.73	0.00851578269688686\\
86.74	0.00851645531548108\\
86.75	0.00851713245305393\\
86.76	0.00851781415609093\\
86.77	0.00851850047156349\\
86.78	0.00851919144693388\\
86.79	0.00851988713016023\\
86.8	0.00852058756970166\\
86.81	0.00852129281452336\\
86.82	0.00852200081096595\\
86.83	0.00852271067630885\\
86.84	0.00852342242808432\\
86.85	0.00852413608402093\\
86.86	0.00852485166204575\\
86.87	0.00852556918028666\\
86.88	0.00852628865707464\\
86.89	0.00852701011094607\\
86.9	0.00852773356064515\\
86.91	0.0085284590251262\\
86.92	0.00852918652355613\\
86.93	0.00852991607531685\\
86.94	0.00853064770000772\\
86.95	0.0085313814174481\\
86.96	0.00853211724767978\\
86.97	0.00853285521096961\\
86.98	0.00853359532781204\\
86.99	0.00853433761893175\\
87	0.00853508210528626\\
87.01	0.00853582880806862\\
87.02	0.00853657774871011\\
87.03	0.00853732894888297\\
87.04	0.00853808243050315\\
87.05	0.00853883821573312\\
87.06	0.00853959632698468\\
87.07	0.00854035678692183\\
87.08	0.00854111961846369\\
87.09	0.00854188484478734\\
87.1	0.00854265248933088\\
87.11	0.00854342257579635\\
87.12	0.0085441951281528\\
87.13	0.00854497017063933\\
87.14	0.00854574772776818\\
87.15	0.00854652782432789\\
87.16	0.00854731048538644\\
87.17	0.0085480957362945\\
87.18	0.0085488836026886\\
87.19	0.00854967411049447\\
87.2	0.00855046728593035\\
87.21	0.00855126315551033\\
87.22	0.00855206174604772\\
87.23	0.00855286308465855\\
87.24	0.00855366719876498\\
87.25	0.00855447411609883\\
87.26	0.00855528386470517\\
87.27	0.00855609647294586\\
87.28	0.00855691196950321\\
87.29	0.00855773038338366\\
87.3	0.0085585517439215\\
87.31	0.00855937608078262\\
87.32	0.00856020342396832\\
87.33	0.00856103380381919\\
87.34	0.00856186725101894\\
87.35	0.0085627037965984\\
87.36	0.00856354347193948\\
87.37	0.00856438630877918\\
87.38	0.00856523233921371\\
87.39	0.00856608159570258\\
87.4	0.00856693411107277\\
87.41	0.00856778991852298\\
87.42	0.00856864905162786\\
87.43	0.00856951154434236\\
87.44	0.00857037743100606\\
87.45	0.00857124674634761\\
87.46	0.00857211952548922\\
87.47	0.00857299580395111\\
87.48	0.00857387561765617\\
87.49	0.0085747590029345\\
87.5	0.00857564599652815\\
87.51	0.00857653663559582\\
87.52	0.00857743095771764\\
87.53	0.00857832900090004\\
87.54	0.00857923080358062\\
87.55	0.00858013640463311\\
87.56	0.00858104584337238\\
87.57	0.00858195915955953\\
87.58	0.00858287639340697\\
87.59	0.00858379758558365\\
87.6	0.00858472277722028\\
87.61	0.00858565200991463\\
87.62	0.00858658532573693\\
87.63	0.00858752276723527\\
87.64	0.0085884643774411\\
87.65	0.00858941019987478\\
87.66	0.00859036027855121\\
87.67	0.00859131465798549\\
87.68	0.00859227338319871\\
87.69	0.00859323649972372\\
87.7	0.00859420405361104\\
87.71	0.00859517609143483\\
87.72	0.00859615266029883\\
87.73	0.00859713380784256\\
87.74	0.00859811958224737\\
87.75	0.00859911003224277\\
87.76	0.00860010520711264\\
87.77	0.00860110515670168\\
87.78	0.00860210993142181\\
87.79	0.00860311958225871\\
87.8	0.00860413416077843\\
87.81	0.00860515371913404\\
87.82	0.00860617831007239\\
87.83	0.00860720798694096\\
87.84	0.00860824280369474\\
87.85	0.00860928281490322\\
87.86	0.0086103280757575\\
87.87	0.00861137864207737\\
87.88	0.00861243457031861\\
87.89	0.00861349591758027\\
87.9	0.00861456274161209\\
87.91	0.00861563510082195\\
87.92	0.00861671305428352\\
87.93	0.00861779666174382\\
87.94	0.00861888598363107\\
87.95	0.00861998108106244\\
87.96	0.00862108201585205\\
87.97	0.00862218885051896\\
87.98	0.00862330164829528\\
87.99	0.0086244204731344\\
88	0.00862554538971927\\
88.01	0.0086266764634708\\
88.02	0.00862781376055641\\
88.03	0.00862895734789853\\
88.04	0.00863010729318338\\
88.05	0.00863126366486973\\
88.06	0.00863242653219779\\
88.07	0.00863359596519823\\
88.08	0.00863477203470126\\
88.09	0.00863595481234588\\
88.1	0.00863714437058916\\
88.11	0.00863834078271566\\
88.12	0.00863954412284701\\
88.13	0.0086407544659515\\
88.14	0.00864197188785389\\
88.15	0.00864319646524522\\
88.16	0.00864442827569284\\
88.17	0.00864566739765051\\
88.18	0.00864691391046856\\
88.19	0.00864816789440432\\
88.2	0.00864942943063247\\
88.21	0.00865069860125572\\
88.22	0.00865197548931541\\
88.23	0.00865326017880241\\
88.24	0.00865455275466803\\
88.25	0.00865585330283512\\
88.26	0.00865716191020926\\
88.27	0.00865847866469007\\
88.28	0.00865980365518275\\
88.29	0.00866113697160959\\
88.3	0.00866247870492178\\
88.31	0.00866382894711125\\
88.32	0.00866518779122265\\
88.33	0.00866655533136557\\
88.34	0.00866793166272676\\
88.35	0.0086693168815826\\
88.36	0.00867071108531168\\
88.37	0.00867211437240751\\
88.38	0.00867352684249139\\
88.39	0.00867494859632545\\
88.4	0.0086763797358258\\
88.41	0.0086778203640759\\
88.42	0.00867927058533997\\
88.43	0.00868073050507671\\
88.44	0.00868220022995304\\
88.45	0.0086836798678581\\
88.46	0.00868516952791733\\
88.47	0.00868666932050679\\
88.48	0.00868817935726762\\
88.49	0.00868969975112062\\
88.5	0.00869123061628108\\
88.51	0.00869277206827374\\
88.52	0.0086943242239479\\
88.53	0.00869588720149278\\
88.54	0.008697461120453\\
88.55	0.00869904610174424\\
88.56	0.00870064226766911\\
88.57	0.0087022497419332\\
88.58	0.00870386864966133\\
88.59	0.00870549911741393\\
88.6	0.00870714127320368\\
88.61	0.00870879524651233\\
88.62	0.00871046116830769\\
88.63	0.00871213917106081\\
88.64	0.00871382938876346\\
88.65	0.00871553195694567\\
88.66	0.00871724701269356\\
88.67	0.0087189746946674\\
88.68	0.0087207151431198\\
88.69	0.00872246849991421\\
88.7	0.00872423490854353\\
88.71	0.00872601451414906\\
88.72	0.00872780746353954\\
88.73	0.00872961390521054\\
88.74	0.00873143398936397\\
88.75	0.00873326786792793\\
88.76	0.00873511569457666\\
88.77	0.00873697762475084\\
88.78	0.00873885381567809\\
88.79	0.00874074442639366\\
88.8	0.00874264961776148\\
88.81	0.00874456955249532\\
88.82	0.00874650439518029\\
88.83	0.0087484543122946\\
88.84	0.00875041947223149\\
88.85	0.00875240004532152\\
88.86	0.00875439620385505\\
88.87	0.00875640812210503\\
88.88	0.00875843597635004\\
88.89	0.00876047994489757\\
88.9	0.00876254020810767\\
88.91	0.00876461694841674\\
88.92	0.00876671035036176\\
88.93	0.00876882060060467\\
88.94	0.00877094788795708\\
88.95	0.00877309240340537\\
88.96	0.00877525434013589\\
88.97	0.00877743389356066\\
88.98	0.00877963126134324\\
88.99	0.00878184664342496\\
89	0.00878408024205148\\
89.01	0.00878633226179958\\
89.02	0.00878860290960438\\
89.03	0.00879089239478681\\
89.04	0.00879320092908141\\
89.05	0.00879552872666452\\
89.06	0.00879787600418269\\
89.07	0.00880024298078156\\
89.08	0.008802629878135\\
89.09	0.00880503692047462\\
89.1	0.00880746433461962\\
89.11	0.00880991235000703\\
89.12	0.00881238119872224\\
89.13	0.00881487111553001\\
89.14	0.00881738233790573\\
89.15	0.0088199151060671\\
89.16	0.00882246966300622\\
89.17	0.00882504625452202\\
89.18	0.00882764512925312\\
89.19	0.00883026007577703\\
89.2	0.00883287994319188\\
89.21	0.00883550477209229\\
89.22	0.00883813460353192\\
89.23	0.00884076947902896\\
89.24	0.0088434094405717\\
89.25	0.00884605453062415\\
89.26	0.00884870479213179\\
89.27	0.00885136026852731\\
89.28	0.00885402100373645\\
89.29	0.00885668704218397\\
89.3	0.00885935842879958\\
89.31	0.00886203520902405\\
89.32	0.00886471742881531\\
89.33	0.0088674051346547\\
89.34	0.00887009837355324\\
89.35	0.00887279719305799\\
89.36	0.00887550164125851\\
89.37	0.00887821176679339\\
89.38	0.00888092761885681\\
89.39	0.00888364924720529\\
89.4	0.00888637670216439\\
89.41	0.0088891100346356\\
89.42	0.00889184929610328\\
89.43	0.00889459453864163\\
89.44	0.00889734581492185\\
89.45	0.00890010317821929\\
89.46	0.00890286668242075\\
89.47	0.00890563638203184\\
89.48	0.00890841233218444\\
89.49	0.00891119458864426\\
89.5	0.00891398320781847\\
89.51	0.00891677824676345\\
89.52	0.0089195797631926\\
89.53	0.0089223878154843\\
89.54	0.0089252024626899\\
89.55	0.00892802376454188\\
89.56	0.00893085178146206\\
89.57	0.0089336865745699\\
89.58	0.00893652820569097\\
89.59	0.00893937673736545\\
89.6	0.00894223223285679\\
89.61	0.00894509475616043\\
89.62	0.00894796437201267\\
89.63	0.00895084114589961\\
89.64	0.00895372514406623\\
89.65	0.00895661643352557\\
89.66	0.00895951508206804\\
89.67	0.0089624211582708\\
89.68	0.00896533473150729\\
89.69	0.0089682558719569\\
89.7	0.0089711846506147\\
89.71	0.00897412113930133\\
89.72	0.00897706541067301\\
89.73	0.00898001753823166\\
89.74	0.00898297759633518\\
89.75	0.00898594566020775\\
89.76	0.00898892180595045\\
89.77	0.00899190611055179\\
89.78	0.00899489865189854\\
89.79	0.00899789950878663\\
89.8	0.00900090876093215\\
89.81	0.00900392648898253\\
89.82	0.00900695277452791\\
89.83	0.00900998770011249\\
89.84	0.00901303134924621\\
89.85	0.00901608380641643\\
89.86	0.00901914515709983\\
89.87	0.00902221548777445\\
89.88	0.00902529488593185\\
89.89	0.00902838344008944\\
89.9	0.00903148123980295\\
89.91	0.00903458837567909\\
89.92	0.00903770493938834\\
89.93	0.00904083102367787\\
89.94	0.00904396672238467\\
89.95	0.00904711213044884\\
89.96	0.00905026734392701\\
89.97	0.00905343246000592\\
89.98	0.00905660757701627\\
89.99	0.00905979279444658\\
90	0.00906298821295736\\
90.01	0.00906619393439542\\
90.02	0.00906941006180826\\
90.03	0.00907263669945882\\
90.04	0.00907587395284027\\
90.05	0.009079121928691\\
90.06	0.0090823807350099\\
90.07	0.00908565048107169\\
90.08	0.00908893127744254\\
90.09	0.00909222323599583\\
90.1	0.00909552646992816\\
90.11	0.00909884109377552\\
90.12	0.00910216722342963\\
90.13	0.0091055049761546\\
90.14	0.00910885447060362\\
90.15	0.00911221582683609\\
90.16	0.0091155891663347\\
90.17	0.00911897461202296\\
90.18	0.00912237228828279\\
90.19	0.00912578232097243\\
90.2	0.00912920483744448\\
90.21	0.00913263996656426\\
90.22	0.00913608783872835\\
90.23	0.00913954858588335\\
90.24	0.0091430223415449\\
90.25	0.00914650924081695\\
90.26	0.00915000942041125\\
90.27	0.00915352301866706\\
90.28	0.0091570501755712\\
90.29	0.00916059103277819\\
90.3	0.00916414573363084\\
90.31	0.00916771442318095\\
90.32	0.00917129724821032\\
90.33	0.00917489435725203\\
90.34	0.00917850590061199\\
90.35	0.00918213203039074\\
90.36	0.00918577290050555\\
90.37	0.00918942866671277\\
90.38	0.00919309948663049\\
90.39	0.00919678551976147\\
90.4	0.00920048692751636\\
90.41	0.00920420387323723\\
90.42	0.00920793652222133\\
90.43	0.00921168504174528\\
90.44	0.00921544960108943\\
90.45	0.00921923037156261\\
90.46	0.00922302752652714\\
90.47	0.00922684124142424\\
90.48	0.00923067169379965\\
90.49	0.00923451906332965\\
90.5	0.00923838353184742\\
90.51	0.00924226528336965\\
90.52	0.00924616450412358\\
90.53	0.00925008138257434\\
90.54	0.00925401610945264\\
90.55	0.00925796887778279\\
90.56	0.00926193988291115\\
90.57	0.00926592932253484\\
90.58	0.00926993739673088\\
90.59	0.00927396430798573\\
90.6	0.0092780102612251\\
90.61	0.00928207546384423\\
90.62	0.00928616012573855\\
90.63	0.00929026445933466\\
90.64	0.00929438867962182\\
90.65	0.0092985330041837\\
90.66	0.00930269765323071\\
90.67	0.00930688284963252\\
90.68	0.00931108881895126\\
90.69	0.00931531578947488\\
90.7	0.00931956399225117\\
90.71	0.00932383366112202\\
90.72	0.00932812503275827\\
90.73	0.00933243834669493\\
90.74	0.00933677384536683\\
90.75	0.00934113177414482\\
90.76	0.00934551238137235\\
90.77	0.00934990537612363\\
90.78	0.00935429546733984\\
90.79	0.00935868259535401\\
90.8	0.0093630666997254\\
90.81	0.00936744771922968\\
90.82	0.00937182559184906\\
90.83	0.00937620025476222\\
90.84	0.00938057164433418\\
90.85	0.00938493969610593\\
90.86	0.00938930434478408\\
90.87	0.00939366552423028\\
90.88	0.00939802316745052\\
90.89	0.00940237720658429\\
90.9	0.00940672757289365\\
90.91	0.0094110741967521\\
90.92	0.00941541700763337\\
90.93	0.00941975593409999\\
90.94	0.00942409090379179\\
90.95	0.00942842184341424\\
90.96	0.00943274867872657\\
90.97	0.00943707133452987\\
90.98	0.00944138973465492\\
90.99	0.00944570380194992\\
91	0.00945001345826807\\
91.01	0.00945431862445496\\
91.02	0.00945861922033586\\
91.03	0.00946291516470275\\
91.04	0.00946720637530131\\
91.05	0.00947149276881765\\
91.06	0.00947577426086488\\
91.07	0.0094800507659696\\
91.08	0.00948432219755808\\
91.09	0.0094885884679424\\
91.1	0.00949284948830631\\
91.11	0.00949710516869098\\
91.12	0.00950135541798052\\
91.13	0.00950560014388737\\
91.14	0.00950983925293747\\
91.15	0.00951407265045523\\
91.16	0.00951830024054834\\
91.17	0.00952252192609241\\
91.18	0.00952673760871531\\
91.19	0.00953094718878146\\
91.2	0.00953515056537581\\
91.21	0.00953934763628766\\
91.22	0.00954353829799427\\
91.23	0.00954772244564426\\
91.24	0.00955189997304082\\
91.25	0.00955607077262468\\
91.26	0.00956023473545687\\
91.27	0.00956439175120129\\
91.28	0.00956854170810701\\
91.29	0.0095726844929904\\
91.3	0.00957681999121701\\
91.31	0.00958094808668321\\
91.32	0.00958506866179762\\
91.33	0.00958918159746226\\
91.34	0.00959328677305357\\
91.35	0.00959738406640306\\
91.36	0.00960147335377779\\
91.37	0.00960555450986058\\
91.38	0.00960962740773002\\
91.39	0.00961369191884014\\
91.4	0.00961774791299987\\
91.41	0.0096217952583523\\
91.42	0.00962583382135356\\
91.43	0.0096298634667515\\
91.44	0.00963388405756416\\
91.45	0.0096378954550578\\
91.46	0.00964189751872488\\
91.47	0.00964589010626152\\
91.48	0.0096498730735449\\
91.49	0.00965384627461022\\
91.5	0.00965780956162742\\
91.51	0.00966176278487764\\
91.52	0.00966570579272934\\
91.53	0.00966963843161414\\
91.54	0.00967356054600233\\
91.55	0.00967747197837812\\
91.56	0.00968137256921455\\
91.57	0.00968526215694807\\
91.58	0.00968914057795282\\
91.59	0.00969300766651462\\
91.6	0.00969686325480455\\
91.61	0.00970070717285226\\
91.62	0.00970453924851896\\
91.63	0.00970835930747001\\
91.64	0.00971216717314723\\
91.65	0.00971596266674078\\
91.66	0.00971974560716082\\
91.67	0.00972351581100865\\
91.68	0.00972727309254763\\
91.69	0.00973101726367368\\
91.7	0.00973474813388535\\
91.71	0.00973846551025365\\
91.72	0.00974216919739137\\
91.73	0.00974585899742213\\
91.74	0.00974953470994892\\
91.75	0.00975319613202236\\
91.76	0.00975684305810851\\
91.77	0.00976047528005627\\
91.78	0.0097640925870644\\
91.79	0.00976769476564812\\
91.8	0.00977128159960525\\
91.81	0.00977485286998203\\
91.82	0.00977840835503841\\
91.83	0.00978194783021292\\
91.84	0.00978547106808721\\
91.85	0.00978897783834998\\
91.86	0.00979246790776062\\
91.87	0.00979594104011227\\
91.88	0.0097993969961945\\
91.89	0.00980283553375552\\
91.9	0.00980625640746386\\
91.91	0.00980965936886966\\
91.92	0.00981304416636538\\
91.93	0.00981641054514615\\
91.94	0.00981975824716948\\
91.95	0.00982308701111459\\
91.96	0.00982639657234117\\
91.97	0.00982968666284767\\
91.98	0.00983295701122903\\
91.99	0.00983620734263391\\
92	0.0098394373787214\\
92.01	0.00984264683761719\\
92.02	0.00984583543386915\\
92.03	0.00984900287840246\\
92.04	0.00985214887847408\\
92.05	0.00985527313762675\\
92.06	0.00985837535564231\\
92.07	0.00986145522849459\\
92.08	0.00986451244830159\\
92.09	0.00986754670327715\\
92.1	0.00987055767768196\\
92.11	0.00987354505177405\\
92.12	0.00987650850175861\\
92.13	0.00987944769973716\\
92.14	0.00988236231365623\\
92.15	0.00988525200725525\\
92.16	0.00988811644001391\\
92.17	0.00989095526709884\\
92.18	0.0098937681393096\\
92.19	0.0098965547030241\\
92.2	0.00989931460014422\\
92.21	0.00990204746804165\\
92.22	0.00990475293950134\\
92.23	0.0099074306426642\\
92.24	0.00991008020096919\\
92.25	0.00991270123309459\\
92.26	0.00991529335289864\\
92.27	0.00991785616935945\\
92.28	0.00992038928651416\\
92.29	0.00992289230339732\\
92.3	0.00992536481397866\\
92.31	0.00992780640709991\\
92.32	0.00993021666641104\\
92.33	0.00993259517030559\\
92.34	0.00993494149185534\\
92.35	0.00993725519874404\\
92.36	0.00993953585320049\\
92.37	0.00994178301193074\\
92.38	0.00994399622604946\\
92.39	0.00994617504101053\\
92.4	0.00994831899653674\\
92.41	0.00995042762654871\\
92.42	0.0099525004590929\\
92.43	0.00995453701626878\\
92.44	0.00995653681415515\\
92.45	0.00995849936273552\\
92.46	0.00996042416582265\\
92.47	0.00996231072098219\\
92.48	0.00996415851945534\\
92.49	0.0099659670460807\\
92.5	0.00996773577921507\\
92.51	0.0099694641906534\\
92.52	0.00997115174554775\\
92.53	0.00997279790232527\\
92.54	0.00997440211260528\\
92.55	0.00997596382111525\\
92.56	0.00997748246560591\\
92.57	0.00997895747676527\\
92.58	0.00998038827813168\\
92.59	0.00998177428600582\\
92.6	0.00998311490936171\\
92.61	0.00998440954975659\\
92.62	0.00998565760123985\\
92.63	0.00998685845026079\\
92.64	0.00998801147557538\\
92.65	0.00998911604815188\\
92.66	0.00999017153107534\\
92.67	0.00999117727945111\\
92.68	0.00999213264030703\\
92.69	0.00999303695249467\\
92.7	0.0099938895465893\\
92.71	0.00999468974478876\\
92.72	0.00999543686081112\\
92.73	0.00999613019979123\\
92.74	0.00999676905817598\\
92.75	0.0099973527236184\\
92.76	0.00999788047487057\\
92.77	0.00999835158167526\\
92.78	0.00999876530465632\\
92.79	0.00999912089520785\\
92.8	0.00999941759538211\\
92.81	0.00999965463777609\\
92.82	0.00999983124541684\\
92.83	0.00999994663164551\\
92.84	0.01\\
92.85	0.01\\
92.86	0.01\\
92.87	0.01\\
92.88	0.01\\
92.89	0.01\\
92.9	0.01\\
92.91	0.01\\
92.92	0.01\\
92.93	0.01\\
92.94	0.01\\
92.95	0.01\\
92.96	0.01\\
92.97	0.01\\
92.98	0.01\\
92.99	0.01\\
93	0.01\\
93.01	0.01\\
93.02	0.01\\
93.03	0.01\\
93.04	0.01\\
93.05	0.01\\
93.06	0.01\\
93.07	0.01\\
93.08	0.01\\
93.09	0.01\\
93.1	0.01\\
93.11	0.01\\
93.12	0.01\\
93.13	0.01\\
93.14	0.01\\
93.15	0.01\\
93.16	0.01\\
93.17	0.01\\
93.18	0.01\\
93.19	0.01\\
93.2	0.01\\
93.21	0.01\\
93.22	0.01\\
93.23	0.01\\
93.24	0.01\\
93.25	0.01\\
93.26	0.01\\
93.27	0.01\\
93.28	0.01\\
93.29	0.01\\
93.3	0.01\\
93.31	0.01\\
93.32	0.01\\
93.33	0.01\\
93.34	0.01\\
93.35	0.01\\
93.36	0.01\\
93.37	0.01\\
93.38	0.01\\
93.39	0.01\\
93.4	0.01\\
93.41	0.01\\
93.42	0.01\\
93.43	0.01\\
93.44	0.01\\
93.45	0.01\\
93.46	0.01\\
93.47	0.01\\
93.48	0.01\\
93.49	0.01\\
93.5	0.01\\
93.51	0.01\\
93.52	0.01\\
93.53	0.01\\
93.54	0.01\\
93.55	0.01\\
93.56	0.01\\
93.57	0.01\\
93.58	0.01\\
93.59	0.01\\
93.6	0.01\\
93.61	0.01\\
93.62	0.01\\
93.63	0.01\\
93.64	0.01\\
93.65	0.01\\
93.66	0.01\\
93.67	0.01\\
93.68	0.01\\
93.69	0.01\\
93.7	0.01\\
93.71	0.01\\
93.72	0.01\\
93.73	0.01\\
93.74	0.01\\
93.75	0.01\\
93.76	0.01\\
93.77	0.01\\
93.78	0.01\\
93.79	0.01\\
93.8	0.01\\
93.81	0.01\\
93.82	0.01\\
93.83	0.01\\
93.84	0.01\\
93.85	0.01\\
93.86	0.01\\
93.87	0.01\\
93.88	0.01\\
93.89	0.01\\
93.9	0.01\\
93.91	0.01\\
93.92	0.01\\
93.93	0.01\\
93.94	0.01\\
93.95	0.01\\
93.96	0.01\\
93.97	0.01\\
93.98	0.01\\
93.99	0.01\\
94	0.01\\
94.01	0.01\\
94.02	0.01\\
94.03	0.01\\
94.04	0.01\\
94.05	0.01\\
94.06	0.01\\
94.07	0.01\\
94.08	0.01\\
94.09	0.01\\
94.1	0.01\\
94.11	0.01\\
94.12	0.01\\
94.13	0.01\\
94.14	0.01\\
94.15	0.01\\
94.16	0.01\\
94.17	0.01\\
94.18	0.01\\
94.19	0.01\\
94.2	0.01\\
94.21	0.01\\
94.22	0.01\\
94.23	0.01\\
94.24	0.01\\
94.25	0.01\\
94.26	0.01\\
94.27	0.01\\
94.28	0.01\\
94.29	0.01\\
94.3	0.01\\
94.31	0.01\\
94.32	0.01\\
94.33	0.01\\
94.34	0.01\\
94.35	0.01\\
94.36	0.01\\
94.37	0.01\\
94.38	0.01\\
94.39	0.01\\
94.4	0.01\\
94.41	0.01\\
94.42	0.01\\
94.43	0.01\\
94.44	0.01\\
94.45	0.01\\
94.46	0.01\\
94.47	0.01\\
94.48	0.01\\
94.49	0.01\\
94.5	0.01\\
94.51	0.01\\
94.52	0.01\\
94.53	0.01\\
94.54	0.01\\
94.55	0.01\\
94.56	0.01\\
94.57	0.01\\
94.58	0.01\\
94.59	0.01\\
94.6	0.01\\
94.61	0.01\\
94.62	0.01\\
94.63	0.01\\
94.64	0.01\\
94.65	0.01\\
94.66	0.01\\
94.67	0.01\\
94.68	0.01\\
94.69	0.01\\
94.7	0.01\\
94.71	0.01\\
94.72	0.01\\
94.73	0.01\\
94.74	0.01\\
94.75	0.01\\
94.76	0.01\\
94.77	0.01\\
94.78	0.01\\
94.79	0.01\\
94.8	0.01\\
94.81	0.01\\
94.82	0.01\\
94.83	0.01\\
94.84	0.01\\
94.85	0.01\\
94.86	0.01\\
94.87	0.01\\
94.88	0.01\\
94.89	0.01\\
94.9	0.01\\
94.91	0.01\\
94.92	0.01\\
94.93	0.01\\
94.94	0.01\\
94.95	0.01\\
94.96	0.01\\
94.97	0.01\\
94.98	0.01\\
94.99	0.01\\
95	0.01\\
95.01	0.01\\
95.02	0.01\\
95.03	0.01\\
95.04	0.01\\
95.05	0.01\\
95.06	0.01\\
95.07	0.01\\
95.08	0.01\\
95.09	0.01\\
95.1	0.01\\
95.11	0.01\\
95.12	0.01\\
95.13	0.01\\
95.14	0.01\\
95.15	0.01\\
95.16	0.01\\
95.17	0.01\\
95.18	0.01\\
95.19	0.01\\
95.2	0.01\\
95.21	0.01\\
95.22	0.01\\
95.23	0.01\\
95.24	0.01\\
95.25	0.01\\
95.26	0.01\\
95.27	0.01\\
95.28	0.01\\
95.29	0.01\\
95.3	0.01\\
95.31	0.01\\
95.32	0.01\\
95.33	0.01\\
95.34	0.01\\
95.35	0.01\\
95.36	0.01\\
95.37	0.01\\
95.38	0.01\\
95.39	0.01\\
95.4	0.01\\
95.41	0.01\\
95.42	0.01\\
95.43	0.01\\
95.44	0.01\\
95.45	0.01\\
95.46	0.01\\
95.47	0.01\\
95.48	0.01\\
95.49	0.01\\
95.5	0.01\\
95.51	0.01\\
95.52	0.01\\
95.53	0.01\\
95.54	0.01\\
95.55	0.01\\
95.56	0.01\\
95.57	0.01\\
95.58	0.01\\
95.59	0.01\\
95.6	0.01\\
95.61	0.01\\
95.62	0.01\\
95.63	0.01\\
95.64	0.01\\
95.65	0.01\\
95.66	0.01\\
95.67	0.01\\
95.68	0.01\\
95.69	0.01\\
95.7	0.01\\
95.71	0.01\\
95.72	0.01\\
95.73	0.01\\
95.74	0.01\\
95.75	0.01\\
95.76	0.01\\
95.77	0.01\\
95.78	0.01\\
95.79	0.01\\
95.8	0.01\\
95.81	0.01\\
95.82	0.01\\
95.83	0.01\\
95.84	0.01\\
95.85	0.01\\
95.86	0.01\\
95.87	0.01\\
95.88	0.01\\
95.89	0.01\\
95.9	0.01\\
95.91	0.01\\
95.92	0.01\\
95.93	0.01\\
95.94	0.01\\
95.95	0.01\\
95.96	0.01\\
95.97	0.01\\
95.98	0.01\\
95.99	0.01\\
96	0.01\\
96.01	0.01\\
96.02	0.01\\
96.03	0.01\\
96.04	0.01\\
96.05	0.01\\
96.06	0.01\\
96.07	0.01\\
96.08	0.01\\
96.09	0.01\\
96.1	0.01\\
96.11	0.01\\
96.12	0.01\\
96.13	0.01\\
96.14	0.01\\
96.15	0.01\\
96.16	0.01\\
96.17	0.01\\
96.18	0.01\\
96.19	0.01\\
96.2	0.01\\
96.21	0.01\\
96.22	0.01\\
96.23	0.01\\
96.24	0.01\\
96.25	0.01\\
96.26	0.01\\
96.27	0.01\\
96.28	0.01\\
96.29	0.01\\
96.3	0.01\\
96.31	0.01\\
96.32	0.01\\
96.33	0.01\\
96.34	0.01\\
96.35	0.01\\
96.36	0.01\\
96.37	0.01\\
96.38	0.01\\
96.39	0.01\\
96.4	0.01\\
96.41	0.01\\
96.42	0.01\\
96.43	0.01\\
96.44	0.01\\
96.45	0.01\\
96.46	0.01\\
96.47	0.01\\
96.48	0.01\\
96.49	0.01\\
96.5	0.01\\
96.51	0.01\\
96.52	0.01\\
96.53	0.01\\
96.54	0.01\\
96.55	0.01\\
96.56	0.01\\
96.57	0.01\\
96.58	0.01\\
96.59	0.01\\
96.6	0.01\\
96.61	0.01\\
96.62	0.01\\
96.63	0.01\\
96.64	0.01\\
96.65	0.01\\
96.66	0.01\\
96.67	0.01\\
96.68	0.01\\
96.69	0.01\\
96.7	0.01\\
96.71	0.01\\
96.72	0.01\\
96.73	0.01\\
96.74	0.01\\
96.75	0.01\\
96.76	0.01\\
96.77	0.01\\
96.78	0.01\\
96.79	0.01\\
96.8	0.01\\
96.81	0.01\\
96.82	0.01\\
96.83	0.01\\
96.84	0.01\\
96.85	0.01\\
96.86	0.01\\
96.87	0.01\\
96.88	0.01\\
96.89	0.01\\
96.9	0.01\\
96.91	0.01\\
96.92	0.01\\
96.93	0.01\\
96.94	0.01\\
96.95	0.01\\
96.96	0.01\\
96.97	0.01\\
96.98	0.01\\
96.99	0.01\\
97	0.01\\
97.01	0.01\\
97.02	0.01\\
97.03	0.01\\
97.04	0.01\\
97.05	0.01\\
97.06	0.01\\
97.07	0.01\\
97.08	0.01\\
97.09	0.01\\
97.1	0.01\\
97.11	0.01\\
97.12	0.01\\
97.13	0.01\\
97.14	0.01\\
97.15	0.01\\
97.16	0.01\\
97.17	0.01\\
97.18	0.01\\
97.19	0.01\\
97.2	0.01\\
97.21	0.01\\
97.22	0.01\\
97.23	0.01\\
97.24	0.01\\
97.25	0.01\\
97.26	0.01\\
97.27	0.01\\
97.28	0.01\\
97.29	0.01\\
97.3	0.01\\
97.31	0.01\\
97.32	0.01\\
97.33	0.01\\
97.34	0.01\\
97.35	0.01\\
97.36	0.01\\
97.37	0.01\\
97.38	0.01\\
97.39	0.01\\
97.4	0.01\\
97.41	0.01\\
97.42	0.01\\
97.43	0.01\\
97.44	0.01\\
97.45	0.01\\
97.46	0.01\\
97.47	0.01\\
97.48	0.01\\
97.49	0.01\\
97.5	0.01\\
97.51	0.01\\
97.52	0.01\\
97.53	0.01\\
97.54	0.01\\
97.55	0.01\\
97.56	0.01\\
97.57	0.01\\
97.58	0.01\\
97.59	0.01\\
97.6	0.01\\
97.61	0.01\\
97.62	0.01\\
97.63	0.01\\
97.64	0.01\\
97.65	0.01\\
97.66	0.01\\
97.67	0.01\\
97.68	0.01\\
97.69	0.01\\
97.7	0.01\\
97.71	0.01\\
97.72	0.01\\
97.73	0.01\\
97.74	0.01\\
97.75	0.01\\
97.76	0.01\\
97.77	0.01\\
97.78	0.01\\
97.79	0.01\\
97.8	0.01\\
97.81	0.01\\
97.82	0.01\\
97.83	0.01\\
97.84	0.01\\
97.85	0.01\\
97.86	0.01\\
97.87	0.01\\
97.88	0.01\\
97.89	0.01\\
97.9	0.01\\
97.91	0.01\\
97.92	0.01\\
97.93	0.01\\
97.94	0.01\\
97.95	0.01\\
97.96	0.01\\
97.97	0.01\\
97.98	0.01\\
97.99	0.01\\
98	0.01\\
98.01	0.01\\
98.02	0.01\\
98.03	0.01\\
98.04	0.01\\
98.05	0.01\\
98.06	0.01\\
98.07	0.01\\
98.08	0.01\\
98.09	0.01\\
98.1	0.01\\
98.11	0.01\\
98.12	0.01\\
98.13	0.01\\
98.14	0.01\\
98.15	0.01\\
98.16	0.01\\
98.17	0.01\\
98.18	0.01\\
98.19	0.01\\
98.2	0.01\\
98.21	0.01\\
98.22	0.01\\
98.23	0.01\\
98.24	0.01\\
98.25	0.01\\
98.26	0.01\\
98.27	0.01\\
98.28	0.01\\
98.29	0.01\\
98.3	0.01\\
98.31	0.01\\
98.32	0.01\\
98.33	0.01\\
98.34	0.01\\
98.35	0.01\\
98.36	0.01\\
98.37	0.01\\
98.38	0.01\\
98.39	0.01\\
98.4	0.01\\
98.41	0.01\\
98.42	0.01\\
98.43	0.01\\
98.44	0.01\\
98.45	0.01\\
98.46	0.01\\
98.47	0.01\\
98.48	0.01\\
98.49	0.01\\
98.5	0.01\\
98.51	0.01\\
98.52	0.01\\
98.53	0.01\\
98.54	0.01\\
98.55	0.01\\
98.56	0.01\\
98.57	0.01\\
98.58	0.01\\
98.59	0.01\\
98.6	0.01\\
98.61	0.01\\
98.62	0.01\\
98.63	0.01\\
98.64	0.01\\
98.65	0.01\\
98.66	0.01\\
98.67	0.01\\
98.68	0.01\\
98.69	0.01\\
98.7	0.01\\
98.71	0.01\\
98.72	0.01\\
98.73	0.01\\
98.74	0.01\\
98.75	0.01\\
98.76	0.01\\
98.77	0.01\\
98.78	0.01\\
98.79	0.01\\
98.8	0.01\\
98.81	0.01\\
98.82	0.01\\
98.83	0.01\\
98.84	0.01\\
98.85	0.01\\
98.86	0.01\\
98.87	0.01\\
98.88	0.01\\
98.89	0.01\\
98.9	0.01\\
98.91	0.01\\
98.92	0.01\\
98.93	0.01\\
98.94	0.01\\
98.95	0.01\\
98.96	0.01\\
98.97	0.01\\
98.98	0.01\\
98.99	0.01\\
99	0.01\\
99.01	0.01\\
99.02	0.01\\
99.03	0.01\\
99.04	0.01\\
99.05	0.01\\
99.06	0.01\\
99.07	0.01\\
99.08	0.01\\
99.09	0.01\\
99.1	0.01\\
99.11	0.01\\
99.12	0.01\\
99.13	0.01\\
99.14	0.01\\
99.15	0.01\\
99.16	0.01\\
99.17	0.01\\
99.18	0.01\\
99.19	0.01\\
99.2	0.01\\
99.21	0.01\\
99.22	0.01\\
99.23	0.01\\
99.24	0.01\\
99.25	0.01\\
99.26	0.01\\
99.27	0.01\\
99.28	0.01\\
99.29	0.01\\
99.3	0.01\\
99.31	0.01\\
99.32	0.01\\
99.33	0.01\\
99.34	0.01\\
99.35	0.01\\
99.36	0.01\\
99.37	0.01\\
99.38	0.01\\
99.39	0.01\\
99.4	0.01\\
99.41	0.01\\
99.42	0.01\\
99.43	0.01\\
99.44	0.01\\
99.45	0.01\\
99.46	0.01\\
99.47	0.01\\
99.48	0.01\\
99.49	0.01\\
99.5	0.01\\
99.51	0.01\\
99.52	0.01\\
99.53	0.01\\
99.54	0.01\\
99.55	0.01\\
99.56	0.01\\
99.57	0.01\\
99.58	0.01\\
99.59	0.01\\
99.6	0.01\\
99.61	0.01\\
99.62	0.01\\
99.63	0.01\\
99.64	0.01\\
99.65	0.01\\
99.66	0.01\\
99.67	0.01\\
99.68	0.01\\
99.69	0.01\\
99.7	0.01\\
99.71	0.01\\
99.72	0.01\\
99.73	0.01\\
99.74	0.01\\
99.75	0.01\\
99.76	0.01\\
99.77	0.01\\
99.78	0.01\\
99.79	0.01\\
99.8	0.01\\
99.81	0.01\\
99.82	0.01\\
99.83	0.01\\
99.84	0.01\\
99.85	0.01\\
99.86	0.01\\
99.87	0.01\\
99.88	0.01\\
99.89	0.01\\
99.9	0.01\\
99.91	0.01\\
99.92	0.01\\
99.93	0.01\\
99.94	0.01\\
99.95	0.01\\
99.96	0.01\\
99.97	0.01\\
99.98	0.01\\
99.99	0.01\\
100	0.01\\
};
\addlegendentry{$q=0$};

\addplot [color=blue,solid,forget plot]
  table[row sep=crcr]{%
0.01	0.01\\
0.02	0.01\\
0.03	0.01\\
0.04	0.01\\
0.05	0.01\\
0.06	0.01\\
0.07	0.01\\
0.08	0.01\\
0.09	0.01\\
0.1	0.01\\
0.11	0.01\\
0.12	0.01\\
0.13	0.01\\
0.14	0.01\\
0.15	0.01\\
0.16	0.01\\
0.17	0.01\\
0.18	0.01\\
0.19	0.01\\
0.2	0.01\\
0.21	0.01\\
0.22	0.01\\
0.23	0.01\\
0.24	0.01\\
0.25	0.01\\
0.26	0.01\\
0.27	0.01\\
0.28	0.01\\
0.29	0.01\\
0.3	0.01\\
0.31	0.01\\
0.32	0.01\\
0.33	0.01\\
0.34	0.01\\
0.35	0.01\\
0.36	0.01\\
0.37	0.01\\
0.38	0.01\\
0.39	0.01\\
0.4	0.01\\
0.41	0.01\\
0.42	0.01\\
0.43	0.01\\
0.44	0.01\\
0.45	0.01\\
0.46	0.01\\
0.47	0.01\\
0.48	0.01\\
0.49	0.01\\
0.5	0.01\\
0.51	0.01\\
0.52	0.01\\
0.53	0.01\\
0.54	0.01\\
0.55	0.01\\
0.56	0.01\\
0.57	0.01\\
0.58	0.01\\
0.59	0.01\\
0.6	0.01\\
0.61	0.01\\
0.62	0.01\\
0.63	0.01\\
0.64	0.01\\
0.65	0.01\\
0.66	0.01\\
0.67	0.01\\
0.68	0.01\\
0.69	0.01\\
0.7	0.01\\
0.71	0.01\\
0.72	0.01\\
0.73	0.01\\
0.74	0.01\\
0.75	0.01\\
0.76	0.01\\
0.77	0.01\\
0.78	0.01\\
0.79	0.01\\
0.8	0.01\\
0.81	0.01\\
0.82	0.01\\
0.83	0.01\\
0.84	0.01\\
0.85	0.01\\
0.86	0.01\\
0.87	0.01\\
0.88	0.01\\
0.89	0.01\\
0.9	0.01\\
0.91	0.01\\
0.92	0.01\\
0.93	0.01\\
0.94	0.01\\
0.95	0.01\\
0.96	0.01\\
0.97	0.01\\
0.98	0.01\\
0.99	0.01\\
1	0.01\\
1.01	0.01\\
1.02	0.01\\
1.03	0.01\\
1.04	0.01\\
1.05	0.01\\
1.06	0.01\\
1.07	0.01\\
1.08	0.01\\
1.09	0.01\\
1.1	0.01\\
1.11	0.01\\
1.12	0.01\\
1.13	0.01\\
1.14	0.01\\
1.15	0.01\\
1.16	0.01\\
1.17	0.01\\
1.18	0.01\\
1.19	0.01\\
1.2	0.01\\
1.21	0.01\\
1.22	0.01\\
1.23	0.01\\
1.24	0.01\\
1.25	0.01\\
1.26	0.01\\
1.27	0.01\\
1.28	0.01\\
1.29	0.01\\
1.3	0.01\\
1.31	0.01\\
1.32	0.01\\
1.33	0.01\\
1.34	0.01\\
1.35	0.01\\
1.36	0.01\\
1.37	0.01\\
1.38	0.01\\
1.39	0.01\\
1.4	0.01\\
1.41	0.01\\
1.42	0.01\\
1.43	0.01\\
1.44	0.01\\
1.45	0.01\\
1.46	0.01\\
1.47	0.01\\
1.48	0.01\\
1.49	0.01\\
1.5	0.01\\
1.51	0.01\\
1.52	0.01\\
1.53	0.01\\
1.54	0.01\\
1.55	0.01\\
1.56	0.01\\
1.57	0.01\\
1.58	0.01\\
1.59	0.01\\
1.6	0.01\\
1.61	0.01\\
1.62	0.01\\
1.63	0.01\\
1.64	0.01\\
1.65	0.01\\
1.66	0.01\\
1.67	0.01\\
1.68	0.01\\
1.69	0.01\\
1.7	0.01\\
1.71	0.01\\
1.72	0.01\\
1.73	0.01\\
1.74	0.01\\
1.75	0.01\\
1.76	0.01\\
1.77	0.01\\
1.78	0.01\\
1.79	0.01\\
1.8	0.01\\
1.81	0.01\\
1.82	0.01\\
1.83	0.01\\
1.84	0.01\\
1.85	0.01\\
1.86	0.01\\
1.87	0.01\\
1.88	0.01\\
1.89	0.01\\
1.9	0.01\\
1.91	0.01\\
1.92	0.01\\
1.93	0.01\\
1.94	0.01\\
1.95	0.01\\
1.96	0.01\\
1.97	0.01\\
1.98	0.01\\
1.99	0.01\\
2	0.01\\
2.01	0.01\\
2.02	0.01\\
2.03	0.01\\
2.04	0.01\\
2.05	0.01\\
2.06	0.01\\
2.07	0.01\\
2.08	0.01\\
2.09	0.01\\
2.1	0.01\\
2.11	0.01\\
2.12	0.01\\
2.13	0.01\\
2.14	0.01\\
2.15	0.01\\
2.16	0.01\\
2.17	0.01\\
2.18	0.01\\
2.19	0.01\\
2.2	0.01\\
2.21	0.01\\
2.22	0.01\\
2.23	0.01\\
2.24	0.01\\
2.25	0.01\\
2.26	0.01\\
2.27	0.01\\
2.28	0.01\\
2.29	0.01\\
2.3	0.01\\
2.31	0.01\\
2.32	0.01\\
2.33	0.01\\
2.34	0.01\\
2.35	0.01\\
2.36	0.01\\
2.37	0.01\\
2.38	0.01\\
2.39	0.01\\
2.4	0.01\\
2.41	0.01\\
2.42	0.01\\
2.43	0.01\\
2.44	0.01\\
2.45	0.01\\
2.46	0.01\\
2.47	0.01\\
2.48	0.01\\
2.49	0.01\\
2.5	0.01\\
2.51	0.01\\
2.52	0.01\\
2.53	0.01\\
2.54	0.01\\
2.55	0.01\\
2.56	0.01\\
2.57	0.01\\
2.58	0.01\\
2.59	0.01\\
2.6	0.01\\
2.61	0.01\\
2.62	0.01\\
2.63	0.01\\
2.64	0.01\\
2.65	0.01\\
2.66	0.01\\
2.67	0.01\\
2.68	0.01\\
2.69	0.01\\
2.7	0.01\\
2.71	0.01\\
2.72	0.01\\
2.73	0.01\\
2.74	0.01\\
2.75	0.01\\
2.76	0.01\\
2.77	0.01\\
2.78	0.01\\
2.79	0.01\\
2.8	0.01\\
2.81	0.01\\
2.82	0.01\\
2.83	0.01\\
2.84	0.01\\
2.85	0.01\\
2.86	0.01\\
2.87	0.01\\
2.88	0.01\\
2.89	0.01\\
2.9	0.01\\
2.91	0.01\\
2.92	0.01\\
2.93	0.01\\
2.94	0.01\\
2.95	0.01\\
2.96	0.01\\
2.97	0.01\\
2.98	0.01\\
2.99	0.01\\
3	0.01\\
3.01	0.01\\
3.02	0.01\\
3.03	0.01\\
3.04	0.01\\
3.05	0.01\\
3.06	0.01\\
3.07	0.01\\
3.08	0.01\\
3.09	0.01\\
3.1	0.01\\
3.11	0.01\\
3.12	0.01\\
3.13	0.01\\
3.14	0.01\\
3.15	0.01\\
3.16	0.01\\
3.17	0.01\\
3.18	0.01\\
3.19	0.01\\
3.2	0.01\\
3.21	0.01\\
3.22	0.01\\
3.23	0.01\\
3.24	0.01\\
3.25	0.01\\
3.26	0.01\\
3.27	0.01\\
3.28	0.01\\
3.29	0.01\\
3.3	0.01\\
3.31	0.01\\
3.32	0.01\\
3.33	0.01\\
3.34	0.01\\
3.35	0.01\\
3.36	0.01\\
3.37	0.01\\
3.38	0.01\\
3.39	0.01\\
3.4	0.01\\
3.41	0.01\\
3.42	0.01\\
3.43	0.01\\
3.44	0.01\\
3.45	0.01\\
3.46	0.01\\
3.47	0.01\\
3.48	0.01\\
3.49	0.01\\
3.5	0.01\\
3.51	0.01\\
3.52	0.01\\
3.53	0.01\\
3.54	0.01\\
3.55	0.01\\
3.56	0.01\\
3.57	0.01\\
3.58	0.01\\
3.59	0.01\\
3.6	0.01\\
3.61	0.01\\
3.62	0.01\\
3.63	0.01\\
3.64	0.01\\
3.65	0.01\\
3.66	0.01\\
3.67	0.01\\
3.68	0.01\\
3.69	0.01\\
3.7	0.01\\
3.71	0.01\\
3.72	0.01\\
3.73	0.01\\
3.74	0.01\\
3.75	0.01\\
3.76	0.01\\
3.77	0.01\\
3.78	0.01\\
3.79	0.01\\
3.8	0.01\\
3.81	0.01\\
3.82	0.01\\
3.83	0.01\\
3.84	0.01\\
3.85	0.01\\
3.86	0.01\\
3.87	0.01\\
3.88	0.01\\
3.89	0.01\\
3.9	0.01\\
3.91	0.01\\
3.92	0.01\\
3.93	0.01\\
3.94	0.01\\
3.95	0.01\\
3.96	0.01\\
3.97	0.01\\
3.98	0.01\\
3.99	0.01\\
4	0.01\\
4.01	0.01\\
4.02	0.01\\
4.03	0.01\\
4.04	0.01\\
4.05	0.01\\
4.06	0.01\\
4.07	0.01\\
4.08	0.01\\
4.09	0.01\\
4.1	0.01\\
4.11	0.01\\
4.12	0.01\\
4.13	0.01\\
4.14	0.01\\
4.15	0.01\\
4.16	0.01\\
4.17	0.01\\
4.18	0.01\\
4.19	0.01\\
4.2	0.01\\
4.21	0.01\\
4.22	0.01\\
4.23	0.01\\
4.24	0.01\\
4.25	0.01\\
4.26	0.01\\
4.27	0.01\\
4.28	0.01\\
4.29	0.01\\
4.3	0.01\\
4.31	0.01\\
4.32	0.01\\
4.33	0.01\\
4.34	0.01\\
4.35	0.01\\
4.36	0.01\\
4.37	0.01\\
4.38	0.01\\
4.39	0.01\\
4.4	0.01\\
4.41	0.01\\
4.42	0.01\\
4.43	0.01\\
4.44	0.01\\
4.45	0.01\\
4.46	0.01\\
4.47	0.01\\
4.48	0.01\\
4.49	0.01\\
4.5	0.01\\
4.51	0.01\\
4.52	0.01\\
4.53	0.01\\
4.54	0.01\\
4.55	0.01\\
4.56	0.01\\
4.57	0.01\\
4.58	0.01\\
4.59	0.01\\
4.6	0.01\\
4.61	0.01\\
4.62	0.01\\
4.63	0.01\\
4.64	0.01\\
4.65	0.01\\
4.66	0.01\\
4.67	0.01\\
4.68	0.01\\
4.69	0.01\\
4.7	0.01\\
4.71	0.01\\
4.72	0.01\\
4.73	0.01\\
4.74	0.01\\
4.75	0.01\\
4.76	0.01\\
4.77	0.01\\
4.78	0.01\\
4.79	0.01\\
4.8	0.01\\
4.81	0.01\\
4.82	0.01\\
4.83	0.01\\
4.84	0.01\\
4.85	0.01\\
4.86	0.01\\
4.87	0.01\\
4.88	0.01\\
4.89	0.01\\
4.9	0.01\\
4.91	0.01\\
4.92	0.01\\
4.93	0.01\\
4.94	0.01\\
4.95	0.01\\
4.96	0.01\\
4.97	0.01\\
4.98	0.01\\
4.99	0.01\\
5	0.01\\
5.01	0.01\\
5.02	0.01\\
5.03	0.01\\
5.04	0.01\\
5.05	0.01\\
5.06	0.01\\
5.07	0.01\\
5.08	0.01\\
5.09	0.01\\
5.1	0.01\\
5.11	0.01\\
5.12	0.01\\
5.13	0.01\\
5.14	0.01\\
5.15	0.01\\
5.16	0.01\\
5.17	0.01\\
5.18	0.01\\
5.19	0.01\\
5.2	0.01\\
5.21	0.01\\
5.22	0.01\\
5.23	0.01\\
5.24	0.01\\
5.25	0.01\\
5.26	0.01\\
5.27	0.01\\
5.28	0.01\\
5.29	0.01\\
5.3	0.01\\
5.31	0.01\\
5.32	0.01\\
5.33	0.01\\
5.34	0.01\\
5.35	0.01\\
5.36	0.01\\
5.37	0.01\\
5.38	0.01\\
5.39	0.01\\
5.4	0.01\\
5.41	0.01\\
5.42	0.01\\
5.43	0.01\\
5.44	0.01\\
5.45	0.01\\
5.46	0.01\\
5.47	0.01\\
5.48	0.01\\
5.49	0.01\\
5.5	0.01\\
5.51	0.01\\
5.52	0.01\\
5.53	0.01\\
5.54	0.01\\
5.55	0.01\\
5.56	0.01\\
5.57	0.01\\
5.58	0.01\\
5.59	0.01\\
5.6	0.01\\
5.61	0.01\\
5.62	0.01\\
5.63	0.01\\
5.64	0.01\\
5.65	0.01\\
5.66	0.01\\
5.67	0.01\\
5.68	0.01\\
5.69	0.01\\
5.7	0.01\\
5.71	0.01\\
5.72	0.01\\
5.73	0.01\\
5.74	0.01\\
5.75	0.01\\
5.76	0.01\\
5.77	0.01\\
5.78	0.01\\
5.79	0.01\\
5.8	0.01\\
5.81	0.01\\
5.82	0.01\\
5.83	0.01\\
5.84	0.01\\
5.85	0.01\\
5.86	0.01\\
5.87	0.01\\
5.88	0.01\\
5.89	0.01\\
5.9	0.01\\
5.91	0.01\\
5.92	0.01\\
5.93	0.01\\
5.94	0.01\\
5.95	0.01\\
5.96	0.01\\
5.97	0.01\\
5.98	0.01\\
5.99	0.01\\
6	0.01\\
6.01	0.01\\
6.02	0.01\\
6.03	0.01\\
6.04	0.01\\
6.05	0.01\\
6.06	0.01\\
6.07	0.01\\
6.08	0.01\\
6.09	0.01\\
6.1	0.01\\
6.11	0.01\\
6.12	0.01\\
6.13	0.01\\
6.14	0.01\\
6.15	0.01\\
6.16	0.01\\
6.17	0.01\\
6.18	0.01\\
6.19	0.01\\
6.2	0.01\\
6.21	0.01\\
6.22	0.01\\
6.23	0.01\\
6.24	0.01\\
6.25	0.01\\
6.26	0.01\\
6.27	0.01\\
6.28	0.01\\
6.29	0.01\\
6.3	0.01\\
6.31	0.01\\
6.32	0.01\\
6.33	0.01\\
6.34	0.01\\
6.35	0.01\\
6.36	0.01\\
6.37	0.01\\
6.38	0.01\\
6.39	0.01\\
6.4	0.01\\
6.41	0.01\\
6.42	0.01\\
6.43	0.01\\
6.44	0.01\\
6.45	0.01\\
6.46	0.01\\
6.47	0.01\\
6.48	0.01\\
6.49	0.01\\
6.5	0.01\\
6.51	0.01\\
6.52	0.01\\
6.53	0.01\\
6.54	0.01\\
6.55	0.01\\
6.56	0.01\\
6.57	0.01\\
6.58	0.01\\
6.59	0.01\\
6.6	0.01\\
6.61	0.01\\
6.62	0.01\\
6.63	0.01\\
6.64	0.01\\
6.65	0.01\\
6.66	0.01\\
6.67	0.01\\
6.68	0.01\\
6.69	0.01\\
6.7	0.01\\
6.71	0.01\\
6.72	0.01\\
6.73	0.01\\
6.74	0.01\\
6.75	0.01\\
6.76	0.01\\
6.77	0.01\\
6.78	0.01\\
6.79	0.01\\
6.8	0.01\\
6.81	0.01\\
6.82	0.01\\
6.83	0.01\\
6.84	0.01\\
6.85	0.01\\
6.86	0.01\\
6.87	0.01\\
6.88	0.01\\
6.89	0.01\\
6.9	0.01\\
6.91	0.01\\
6.92	0.01\\
6.93	0.01\\
6.94	0.01\\
6.95	0.01\\
6.96	0.01\\
6.97	0.01\\
6.98	0.01\\
6.99	0.01\\
7	0.01\\
7.01	0.01\\
7.02	0.01\\
7.03	0.01\\
7.04	0.01\\
7.05	0.01\\
7.06	0.01\\
7.07	0.01\\
7.08	0.01\\
7.09	0.01\\
7.1	0.01\\
7.11	0.01\\
7.12	0.01\\
7.13	0.01\\
7.14	0.01\\
7.15	0.01\\
7.16	0.01\\
7.17	0.01\\
7.18	0.01\\
7.19	0.01\\
7.2	0.01\\
7.21	0.01\\
7.22	0.01\\
7.23	0.01\\
7.24	0.01\\
7.25	0.01\\
7.26	0.01\\
7.27	0.01\\
7.28	0.01\\
7.29	0.01\\
7.3	0.01\\
7.31	0.01\\
7.32	0.01\\
7.33	0.01\\
7.34	0.01\\
7.35	0.01\\
7.36	0.01\\
7.37	0.01\\
7.38	0.01\\
7.39	0.01\\
7.4	0.01\\
7.41	0.01\\
7.42	0.01\\
7.43	0.01\\
7.44	0.01\\
7.45	0.01\\
7.46	0.01\\
7.47	0.01\\
7.48	0.01\\
7.49	0.01\\
7.5	0.01\\
7.51	0.01\\
7.52	0.01\\
7.53	0.01\\
7.54	0.01\\
7.55	0.01\\
7.56	0.01\\
7.57	0.01\\
7.58	0.01\\
7.59	0.01\\
7.6	0.01\\
7.61	0.01\\
7.62	0.01\\
7.63	0.01\\
7.64	0.01\\
7.65	0.01\\
7.66	0.01\\
7.67	0.01\\
7.68	0.01\\
7.69	0.01\\
7.7	0.01\\
7.71	0.01\\
7.72	0.01\\
7.73	0.01\\
7.74	0.01\\
7.75	0.01\\
7.76	0.01\\
7.77	0.01\\
7.78	0.01\\
7.79	0.01\\
7.8	0.01\\
7.81	0.01\\
7.82	0.01\\
7.83	0.01\\
7.84	0.01\\
7.85	0.01\\
7.86	0.01\\
7.87	0.01\\
7.88	0.01\\
7.89	0.01\\
7.9	0.01\\
7.91	0.01\\
7.92	0.01\\
7.93	0.01\\
7.94	0.01\\
7.95	0.01\\
7.96	0.01\\
7.97	0.01\\
7.98	0.01\\
7.99	0.01\\
8	0.01\\
8.01	0.01\\
8.02	0.01\\
8.03	0.01\\
8.04	0.01\\
8.05	0.01\\
8.06	0.01\\
8.07	0.01\\
8.08	0.01\\
8.09	0.01\\
8.1	0.01\\
8.11	0.01\\
8.12	0.01\\
8.13	0.01\\
8.14	0.01\\
8.15	0.01\\
8.16	0.01\\
8.17	0.01\\
8.18	0.01\\
8.19	0.01\\
8.2	0.01\\
8.21	0.01\\
8.22	0.01\\
8.23	0.01\\
8.24	0.01\\
8.25	0.01\\
8.26	0.01\\
8.27	0.01\\
8.28	0.01\\
8.29	0.01\\
8.3	0.01\\
8.31	0.01\\
8.32	0.01\\
8.33	0.01\\
8.34	0.01\\
8.35	0.01\\
8.36	0.01\\
8.37	0.01\\
8.38	0.01\\
8.39	0.01\\
8.4	0.01\\
8.41	0.01\\
8.42	0.01\\
8.43	0.01\\
8.44	0.01\\
8.45	0.01\\
8.46	0.01\\
8.47	0.01\\
8.48	0.01\\
8.49	0.01\\
8.5	0.01\\
8.51	0.01\\
8.52	0.01\\
8.53	0.01\\
8.54	0.01\\
8.55	0.01\\
8.56	0.01\\
8.57	0.01\\
8.58	0.01\\
8.59	0.01\\
8.6	0.01\\
8.61	0.01\\
8.62	0.01\\
8.63	0.01\\
8.64	0.01\\
8.65	0.01\\
8.66	0.01\\
8.67	0.01\\
8.68	0.01\\
8.69	0.01\\
8.7	0.01\\
8.71	0.01\\
8.72	0.01\\
8.73	0.01\\
8.74	0.01\\
8.75	0.01\\
8.76	0.01\\
8.77	0.01\\
8.78	0.01\\
8.79	0.01\\
8.8	0.01\\
8.81	0.01\\
8.82	0.01\\
8.83	0.01\\
8.84	0.01\\
8.85	0.01\\
8.86	0.01\\
8.87	0.01\\
8.88	0.01\\
8.89	0.01\\
8.9	0.01\\
8.91	0.01\\
8.92	0.01\\
8.93	0.01\\
8.94	0.01\\
8.95	0.01\\
8.96	0.01\\
8.97	0.01\\
8.98	0.01\\
8.99	0.01\\
9	0.01\\
9.01	0.01\\
9.02	0.01\\
9.03	0.01\\
9.04	0.01\\
9.05	0.01\\
9.06	0.01\\
9.07	0.01\\
9.08	0.01\\
9.09	0.01\\
9.1	0.01\\
9.11	0.01\\
9.12	0.01\\
9.13	0.01\\
9.14	0.01\\
9.15	0.01\\
9.16	0.01\\
9.17	0.01\\
9.18	0.01\\
9.19	0.01\\
9.2	0.01\\
9.21	0.01\\
9.22	0.01\\
9.23	0.01\\
9.24	0.01\\
9.25	0.01\\
9.26	0.01\\
9.27	0.01\\
9.28	0.01\\
9.29	0.01\\
9.3	0.01\\
9.31	0.01\\
9.32	0.01\\
9.33	0.01\\
9.34	0.01\\
9.35	0.01\\
9.36	0.01\\
9.37	0.01\\
9.38	0.01\\
9.39	0.01\\
9.4	0.01\\
9.41	0.01\\
9.42	0.01\\
9.43	0.01\\
9.44	0.01\\
9.45	0.01\\
9.46	0.01\\
9.47	0.01\\
9.48	0.01\\
9.49	0.01\\
9.5	0.01\\
9.51	0.01\\
9.52	0.01\\
9.53	0.01\\
9.54	0.01\\
9.55	0.01\\
9.56	0.01\\
9.57	0.01\\
9.58	0.01\\
9.59	0.01\\
9.6	0.01\\
9.61	0.01\\
9.62	0.01\\
9.63	0.01\\
9.64	0.01\\
9.65	0.01\\
9.66	0.01\\
9.67	0.01\\
9.68	0.01\\
9.69	0.01\\
9.7	0.01\\
9.71	0.01\\
9.72	0.01\\
9.73	0.01\\
9.74	0.01\\
9.75	0.01\\
9.76	0.01\\
9.77	0.01\\
9.78	0.01\\
9.79	0.01\\
9.8	0.01\\
9.81	0.01\\
9.82	0.01\\
9.83	0.01\\
9.84	0.01\\
9.85	0.01\\
9.86	0.01\\
9.87	0.01\\
9.88	0.01\\
9.89	0.01\\
9.9	0.01\\
9.91	0.01\\
9.92	0.01\\
9.93	0.01\\
9.94	0.01\\
9.95	0.01\\
9.96	0.01\\
9.97	0.01\\
9.98	0.01\\
9.99	0.01\\
10	0.01\\
10.01	0.01\\
10.02	0.01\\
10.03	0.01\\
10.04	0.01\\
10.05	0.01\\
10.06	0.01\\
10.07	0.01\\
10.08	0.01\\
10.09	0.01\\
10.1	0.01\\
10.11	0.01\\
10.12	0.01\\
10.13	0.01\\
10.14	0.01\\
10.15	0.01\\
10.16	0.01\\
10.17	0.01\\
10.18	0.01\\
10.19	0.01\\
10.2	0.01\\
10.21	0.01\\
10.22	0.01\\
10.23	0.01\\
10.24	0.01\\
10.25	0.01\\
10.26	0.01\\
10.27	0.01\\
10.28	0.01\\
10.29	0.01\\
10.3	0.01\\
10.31	0.01\\
10.32	0.01\\
10.33	0.01\\
10.34	0.01\\
10.35	0.01\\
10.36	0.01\\
10.37	0.01\\
10.38	0.01\\
10.39	0.01\\
10.4	0.01\\
10.41	0.01\\
10.42	0.01\\
10.43	0.01\\
10.44	0.01\\
10.45	0.01\\
10.46	0.01\\
10.47	0.01\\
10.48	0.01\\
10.49	0.01\\
10.5	0.01\\
10.51	0.01\\
10.52	0.01\\
10.53	0.01\\
10.54	0.01\\
10.55	0.01\\
10.56	0.01\\
10.57	0.01\\
10.58	0.01\\
10.59	0.01\\
10.6	0.01\\
10.61	0.01\\
10.62	0.01\\
10.63	0.01\\
10.64	0.01\\
10.65	0.01\\
10.66	0.01\\
10.67	0.01\\
10.68	0.01\\
10.69	0.01\\
10.7	0.01\\
10.71	0.01\\
10.72	0.01\\
10.73	0.01\\
10.74	0.01\\
10.75	0.01\\
10.76	0.01\\
10.77	0.01\\
10.78	0.01\\
10.79	0.01\\
10.8	0.01\\
10.81	0.01\\
10.82	0.01\\
10.83	0.01\\
10.84	0.01\\
10.85	0.01\\
10.86	0.01\\
10.87	0.01\\
10.88	0.01\\
10.89	0.01\\
10.9	0.01\\
10.91	0.01\\
10.92	0.01\\
10.93	0.01\\
10.94	0.01\\
10.95	0.01\\
10.96	0.01\\
10.97	0.01\\
10.98	0.01\\
10.99	0.01\\
11	0.01\\
11.01	0.01\\
11.02	0.01\\
11.03	0.01\\
11.04	0.01\\
11.05	0.01\\
11.06	0.01\\
11.07	0.01\\
11.08	0.01\\
11.09	0.01\\
11.1	0.01\\
11.11	0.01\\
11.12	0.01\\
11.13	0.01\\
11.14	0.01\\
11.15	0.01\\
11.16	0.01\\
11.17	0.01\\
11.18	0.01\\
11.19	0.01\\
11.2	0.01\\
11.21	0.01\\
11.22	0.01\\
11.23	0.01\\
11.24	0.01\\
11.25	0.01\\
11.26	0.01\\
11.27	0.01\\
11.28	0.01\\
11.29	0.01\\
11.3	0.01\\
11.31	0.01\\
11.32	0.01\\
11.33	0.01\\
11.34	0.01\\
11.35	0.01\\
11.36	0.01\\
11.37	0.01\\
11.38	0.01\\
11.39	0.01\\
11.4	0.01\\
11.41	0.01\\
11.42	0.01\\
11.43	0.01\\
11.44	0.01\\
11.45	0.01\\
11.46	0.01\\
11.47	0.01\\
11.48	0.01\\
11.49	0.01\\
11.5	0.01\\
11.51	0.01\\
11.52	0.01\\
11.53	0.01\\
11.54	0.01\\
11.55	0.01\\
11.56	0.01\\
11.57	0.01\\
11.58	0.01\\
11.59	0.01\\
11.6	0.01\\
11.61	0.01\\
11.62	0.01\\
11.63	0.01\\
11.64	0.01\\
11.65	0.01\\
11.66	0.01\\
11.67	0.01\\
11.68	0.01\\
11.69	0.01\\
11.7	0.01\\
11.71	0.01\\
11.72	0.01\\
11.73	0.01\\
11.74	0.01\\
11.75	0.01\\
11.76	0.01\\
11.77	0.01\\
11.78	0.01\\
11.79	0.01\\
11.8	0.01\\
11.81	0.01\\
11.82	0.01\\
11.83	0.01\\
11.84	0.01\\
11.85	0.01\\
11.86	0.01\\
11.87	0.01\\
11.88	0.01\\
11.89	0.01\\
11.9	0.01\\
11.91	0.01\\
11.92	0.01\\
11.93	0.01\\
11.94	0.01\\
11.95	0.01\\
11.96	0.01\\
11.97	0.01\\
11.98	0.01\\
11.99	0.01\\
12	0.01\\
12.01	0.01\\
12.02	0.01\\
12.03	0.01\\
12.04	0.01\\
12.05	0.01\\
12.06	0.01\\
12.07	0.01\\
12.08	0.01\\
12.09	0.01\\
12.1	0.01\\
12.11	0.01\\
12.12	0.01\\
12.13	0.01\\
12.14	0.01\\
12.15	0.01\\
12.16	0.01\\
12.17	0.01\\
12.18	0.01\\
12.19	0.01\\
12.2	0.01\\
12.21	0.01\\
12.22	0.01\\
12.23	0.01\\
12.24	0.01\\
12.25	0.01\\
12.26	0.01\\
12.27	0.01\\
12.28	0.01\\
12.29	0.01\\
12.3	0.01\\
12.31	0.01\\
12.32	0.01\\
12.33	0.01\\
12.34	0.01\\
12.35	0.01\\
12.36	0.01\\
12.37	0.01\\
12.38	0.01\\
12.39	0.01\\
12.4	0.01\\
12.41	0.01\\
12.42	0.01\\
12.43	0.01\\
12.44	0.01\\
12.45	0.01\\
12.46	0.01\\
12.47	0.01\\
12.48	0.01\\
12.49	0.01\\
12.5	0.01\\
12.51	0.01\\
12.52	0.01\\
12.53	0.01\\
12.54	0.01\\
12.55	0.01\\
12.56	0.01\\
12.57	0.01\\
12.58	0.01\\
12.59	0.01\\
12.6	0.01\\
12.61	0.01\\
12.62	0.01\\
12.63	0.01\\
12.64	0.01\\
12.65	0.01\\
12.66	0.01\\
12.67	0.01\\
12.68	0.01\\
12.69	0.01\\
12.7	0.01\\
12.71	0.01\\
12.72	0.01\\
12.73	0.01\\
12.74	0.01\\
12.75	0.01\\
12.76	0.01\\
12.77	0.01\\
12.78	0.01\\
12.79	0.01\\
12.8	0.01\\
12.81	0.01\\
12.82	0.01\\
12.83	0.01\\
12.84	0.01\\
12.85	0.01\\
12.86	0.01\\
12.87	0.01\\
12.88	0.01\\
12.89	0.01\\
12.9	0.01\\
12.91	0.01\\
12.92	0.01\\
12.93	0.01\\
12.94	0.01\\
12.95	0.01\\
12.96	0.01\\
12.97	0.01\\
12.98	0.01\\
12.99	0.01\\
13	0.01\\
13.01	0.01\\
13.02	0.01\\
13.03	0.01\\
13.04	0.01\\
13.05	0.01\\
13.06	0.01\\
13.07	0.01\\
13.08	0.01\\
13.09	0.01\\
13.1	0.01\\
13.11	0.01\\
13.12	0.01\\
13.13	0.01\\
13.14	0.01\\
13.15	0.01\\
13.16	0.01\\
13.17	0.01\\
13.18	0.01\\
13.19	0.01\\
13.2	0.01\\
13.21	0.01\\
13.22	0.01\\
13.23	0.01\\
13.24	0.01\\
13.25	0.01\\
13.26	0.01\\
13.27	0.01\\
13.28	0.01\\
13.29	0.01\\
13.3	0.01\\
13.31	0.01\\
13.32	0.01\\
13.33	0.01\\
13.34	0.01\\
13.35	0.01\\
13.36	0.01\\
13.37	0.01\\
13.38	0.01\\
13.39	0.01\\
13.4	0.01\\
13.41	0.01\\
13.42	0.01\\
13.43	0.01\\
13.44	0.01\\
13.45	0.01\\
13.46	0.01\\
13.47	0.01\\
13.48	0.01\\
13.49	0.01\\
13.5	0.01\\
13.51	0.01\\
13.52	0.01\\
13.53	0.01\\
13.54	0.01\\
13.55	0.01\\
13.56	0.01\\
13.57	0.01\\
13.58	0.01\\
13.59	0.01\\
13.6	0.01\\
13.61	0.01\\
13.62	0.01\\
13.63	0.01\\
13.64	0.01\\
13.65	0.01\\
13.66	0.01\\
13.67	0.01\\
13.68	0.01\\
13.69	0.01\\
13.7	0.01\\
13.71	0.01\\
13.72	0.01\\
13.73	0.01\\
13.74	0.01\\
13.75	0.01\\
13.76	0.01\\
13.77	0.01\\
13.78	0.01\\
13.79	0.01\\
13.8	0.01\\
13.81	0.01\\
13.82	0.01\\
13.83	0.01\\
13.84	0.01\\
13.85	0.01\\
13.86	0.01\\
13.87	0.01\\
13.88	0.01\\
13.89	0.01\\
13.9	0.01\\
13.91	0.01\\
13.92	0.01\\
13.93	0.01\\
13.94	0.01\\
13.95	0.01\\
13.96	0.01\\
13.97	0.01\\
13.98	0.01\\
13.99	0.01\\
14	0.01\\
14.01	0.01\\
14.02	0.01\\
14.03	0.01\\
14.04	0.01\\
14.05	0.01\\
14.06	0.01\\
14.07	0.01\\
14.08	0.01\\
14.09	0.01\\
14.1	0.01\\
14.11	0.01\\
14.12	0.01\\
14.13	0.01\\
14.14	0.01\\
14.15	0.01\\
14.16	0.01\\
14.17	0.01\\
14.18	0.01\\
14.19	0.01\\
14.2	0.01\\
14.21	0.01\\
14.22	0.01\\
14.23	0.01\\
14.24	0.01\\
14.25	0.01\\
14.26	0.01\\
14.27	0.01\\
14.28	0.01\\
14.29	0.01\\
14.3	0.01\\
14.31	0.01\\
14.32	0.01\\
14.33	0.01\\
14.34	0.01\\
14.35	0.01\\
14.36	0.01\\
14.37	0.01\\
14.38	0.01\\
14.39	0.01\\
14.4	0.01\\
14.41	0.01\\
14.42	0.01\\
14.43	0.01\\
14.44	0.01\\
14.45	0.01\\
14.46	0.01\\
14.47	0.01\\
14.48	0.01\\
14.49	0.01\\
14.5	0.01\\
14.51	0.01\\
14.52	0.01\\
14.53	0.01\\
14.54	0.01\\
14.55	0.01\\
14.56	0.01\\
14.57	0.01\\
14.58	0.01\\
14.59	0.01\\
14.6	0.01\\
14.61	0.01\\
14.62	0.01\\
14.63	0.01\\
14.64	0.01\\
14.65	0.01\\
14.66	0.01\\
14.67	0.01\\
14.68	0.01\\
14.69	0.01\\
14.7	0.01\\
14.71	0.01\\
14.72	0.01\\
14.73	0.01\\
14.74	0.01\\
14.75	0.01\\
14.76	0.01\\
14.77	0.01\\
14.78	0.01\\
14.79	0.01\\
14.8	0.01\\
14.81	0.01\\
14.82	0.01\\
14.83	0.01\\
14.84	0.01\\
14.85	0.01\\
14.86	0.01\\
14.87	0.01\\
14.88	0.01\\
14.89	0.01\\
14.9	0.01\\
14.91	0.01\\
14.92	0.01\\
14.93	0.01\\
14.94	0.01\\
14.95	0.01\\
14.96	0.01\\
14.97	0.01\\
14.98	0.01\\
14.99	0.01\\
15	0.01\\
15.01	0.01\\
15.02	0.01\\
15.03	0.01\\
15.04	0.01\\
15.05	0.01\\
15.06	0.01\\
15.07	0.01\\
15.08	0.01\\
15.09	0.01\\
15.1	0.01\\
15.11	0.01\\
15.12	0.01\\
15.13	0.01\\
15.14	0.01\\
15.15	0.01\\
15.16	0.01\\
15.17	0.01\\
15.18	0.01\\
15.19	0.01\\
15.2	0.01\\
15.21	0.01\\
15.22	0.01\\
15.23	0.01\\
15.24	0.01\\
15.25	0.01\\
15.26	0.01\\
15.27	0.01\\
15.28	0.01\\
15.29	0.01\\
15.3	0.01\\
15.31	0.01\\
15.32	0.01\\
15.33	0.01\\
15.34	0.01\\
15.35	0.01\\
15.36	0.01\\
15.37	0.01\\
15.38	0.01\\
15.39	0.01\\
15.4	0.01\\
15.41	0.01\\
15.42	0.01\\
15.43	0.01\\
15.44	0.01\\
15.45	0.01\\
15.46	0.01\\
15.47	0.01\\
15.48	0.01\\
15.49	0.01\\
15.5	0.01\\
15.51	0.01\\
15.52	0.01\\
15.53	0.01\\
15.54	0.01\\
15.55	0.01\\
15.56	0.01\\
15.57	0.01\\
15.58	0.01\\
15.59	0.01\\
15.6	0.01\\
15.61	0.01\\
15.62	0.01\\
15.63	0.01\\
15.64	0.01\\
15.65	0.01\\
15.66	0.01\\
15.67	0.01\\
15.68	0.01\\
15.69	0.01\\
15.7	0.01\\
15.71	0.01\\
15.72	0.01\\
15.73	0.01\\
15.74	0.01\\
15.75	0.01\\
15.76	0.01\\
15.77	0.01\\
15.78	0.01\\
15.79	0.01\\
15.8	0.01\\
15.81	0.01\\
15.82	0.01\\
15.83	0.01\\
15.84	0.01\\
15.85	0.01\\
15.86	0.01\\
15.87	0.01\\
15.88	0.01\\
15.89	0.01\\
15.9	0.01\\
15.91	0.01\\
15.92	0.01\\
15.93	0.01\\
15.94	0.01\\
15.95	0.01\\
15.96	0.01\\
15.97	0.01\\
15.98	0.01\\
15.99	0.01\\
16	0.01\\
16.01	0.01\\
16.02	0.01\\
16.03	0.01\\
16.04	0.01\\
16.05	0.01\\
16.06	0.01\\
16.07	0.01\\
16.08	0.01\\
16.09	0.01\\
16.1	0.01\\
16.11	0.01\\
16.12	0.01\\
16.13	0.01\\
16.14	0.01\\
16.15	0.01\\
16.16	0.01\\
16.17	0.01\\
16.18	0.01\\
16.19	0.01\\
16.2	0.01\\
16.21	0.01\\
16.22	0.01\\
16.23	0.01\\
16.24	0.01\\
16.25	0.01\\
16.26	0.01\\
16.27	0.01\\
16.28	0.01\\
16.29	0.01\\
16.3	0.01\\
16.31	0.01\\
16.32	0.01\\
16.33	0.01\\
16.34	0.01\\
16.35	0.01\\
16.36	0.01\\
16.37	0.01\\
16.38	0.01\\
16.39	0.01\\
16.4	0.01\\
16.41	0.01\\
16.42	0.01\\
16.43	0.01\\
16.44	0.01\\
16.45	0.01\\
16.46	0.01\\
16.47	0.01\\
16.48	0.01\\
16.49	0.01\\
16.5	0.01\\
16.51	0.01\\
16.52	0.01\\
16.53	0.01\\
16.54	0.01\\
16.55	0.01\\
16.56	0.01\\
16.57	0.01\\
16.58	0.01\\
16.59	0.01\\
16.6	0.01\\
16.61	0.01\\
16.62	0.01\\
16.63	0.01\\
16.64	0.01\\
16.65	0.01\\
16.66	0.01\\
16.67	0.01\\
16.68	0.01\\
16.69	0.01\\
16.7	0.01\\
16.71	0.01\\
16.72	0.01\\
16.73	0.01\\
16.74	0.01\\
16.75	0.01\\
16.76	0.01\\
16.77	0.01\\
16.78	0.01\\
16.79	0.01\\
16.8	0.01\\
16.81	0.01\\
16.82	0.01\\
16.83	0.01\\
16.84	0.01\\
16.85	0.01\\
16.86	0.01\\
16.87	0.01\\
16.88	0.01\\
16.89	0.01\\
16.9	0.01\\
16.91	0.01\\
16.92	0.01\\
16.93	0.01\\
16.94	0.01\\
16.95	0.01\\
16.96	0.01\\
16.97	0.01\\
16.98	0.01\\
16.99	0.01\\
17	0.01\\
17.01	0.01\\
17.02	0.01\\
17.03	0.01\\
17.04	0.01\\
17.05	0.01\\
17.06	0.01\\
17.07	0.01\\
17.08	0.01\\
17.09	0.01\\
17.1	0.01\\
17.11	0.01\\
17.12	0.01\\
17.13	0.01\\
17.14	0.01\\
17.15	0.01\\
17.16	0.01\\
17.17	0.01\\
17.18	0.01\\
17.19	0.01\\
17.2	0.01\\
17.21	0.01\\
17.22	0.01\\
17.23	0.01\\
17.24	0.01\\
17.25	0.01\\
17.26	0.01\\
17.27	0.01\\
17.28	0.01\\
17.29	0.01\\
17.3	0.01\\
17.31	0.01\\
17.32	0.01\\
17.33	0.01\\
17.34	0.01\\
17.35	0.01\\
17.36	0.01\\
17.37	0.01\\
17.38	0.01\\
17.39	0.01\\
17.4	0.01\\
17.41	0.01\\
17.42	0.01\\
17.43	0.01\\
17.44	0.01\\
17.45	0.01\\
17.46	0.01\\
17.47	0.01\\
17.48	0.01\\
17.49	0.01\\
17.5	0.01\\
17.51	0.01\\
17.52	0.01\\
17.53	0.01\\
17.54	0.01\\
17.55	0.01\\
17.56	0.01\\
17.57	0.01\\
17.58	0.01\\
17.59	0.01\\
17.6	0.01\\
17.61	0.01\\
17.62	0.01\\
17.63	0.01\\
17.64	0.01\\
17.65	0.01\\
17.66	0.01\\
17.67	0.01\\
17.68	0.01\\
17.69	0.01\\
17.7	0.01\\
17.71	0.01\\
17.72	0.01\\
17.73	0.01\\
17.74	0.01\\
17.75	0.01\\
17.76	0.01\\
17.77	0.01\\
17.78	0.01\\
17.79	0.01\\
17.8	0.01\\
17.81	0.01\\
17.82	0.01\\
17.83	0.01\\
17.84	0.01\\
17.85	0.01\\
17.86	0.01\\
17.87	0.01\\
17.88	0.01\\
17.89	0.01\\
17.9	0.01\\
17.91	0.01\\
17.92	0.01\\
17.93	0.01\\
17.94	0.01\\
17.95	0.01\\
17.96	0.01\\
17.97	0.01\\
17.98	0.01\\
17.99	0.01\\
18	0.01\\
18.01	0.01\\
18.02	0.01\\
18.03	0.01\\
18.04	0.01\\
18.05	0.01\\
18.06	0.01\\
18.07	0.01\\
18.08	0.01\\
18.09	0.01\\
18.1	0.01\\
18.11	0.01\\
18.12	0.01\\
18.13	0.01\\
18.14	0.01\\
18.15	0.01\\
18.16	0.01\\
18.17	0.01\\
18.18	0.01\\
18.19	0.01\\
18.2	0.01\\
18.21	0.01\\
18.22	0.01\\
18.23	0.01\\
18.24	0.01\\
18.25	0.01\\
18.26	0.01\\
18.27	0.01\\
18.28	0.01\\
18.29	0.01\\
18.3	0.01\\
18.31	0.01\\
18.32	0.01\\
18.33	0.01\\
18.34	0.01\\
18.35	0.01\\
18.36	0.01\\
18.37	0.01\\
18.38	0.01\\
18.39	0.01\\
18.4	0.01\\
18.41	0.01\\
18.42	0.01\\
18.43	0.01\\
18.44	0.01\\
18.45	0.01\\
18.46	0.01\\
18.47	0.01\\
18.48	0.01\\
18.49	0.01\\
18.5	0.01\\
18.51	0.01\\
18.52	0.01\\
18.53	0.01\\
18.54	0.01\\
18.55	0.01\\
18.56	0.01\\
18.57	0.01\\
18.58	0.01\\
18.59	0.01\\
18.6	0.01\\
18.61	0.01\\
18.62	0.01\\
18.63	0.01\\
18.64	0.01\\
18.65	0.01\\
18.66	0.01\\
18.67	0.01\\
18.68	0.01\\
18.69	0.01\\
18.7	0.01\\
18.71	0.01\\
18.72	0.01\\
18.73	0.01\\
18.74	0.01\\
18.75	0.01\\
18.76	0.01\\
18.77	0.01\\
18.78	0.01\\
18.79	0.01\\
18.8	0.01\\
18.81	0.01\\
18.82	0.01\\
18.83	0.01\\
18.84	0.01\\
18.85	0.01\\
18.86	0.01\\
18.87	0.01\\
18.88	0.01\\
18.89	0.01\\
18.9	0.01\\
18.91	0.01\\
18.92	0.01\\
18.93	0.01\\
18.94	0.01\\
18.95	0.01\\
18.96	0.01\\
18.97	0.01\\
18.98	0.01\\
18.99	0.01\\
19	0.01\\
19.01	0.01\\
19.02	0.01\\
19.03	0.01\\
19.04	0.01\\
19.05	0.01\\
19.06	0.01\\
19.07	0.01\\
19.08	0.01\\
19.09	0.01\\
19.1	0.01\\
19.11	0.01\\
19.12	0.01\\
19.13	0.01\\
19.14	0.01\\
19.15	0.01\\
19.16	0.01\\
19.17	0.01\\
19.18	0.01\\
19.19	0.01\\
19.2	0.01\\
19.21	0.01\\
19.22	0.01\\
19.23	0.01\\
19.24	0.01\\
19.25	0.01\\
19.26	0.01\\
19.27	0.01\\
19.28	0.01\\
19.29	0.01\\
19.3	0.01\\
19.31	0.01\\
19.32	0.01\\
19.33	0.01\\
19.34	0.01\\
19.35	0.01\\
19.36	0.01\\
19.37	0.01\\
19.38	0.01\\
19.39	0.01\\
19.4	0.01\\
19.41	0.01\\
19.42	0.01\\
19.43	0.01\\
19.44	0.01\\
19.45	0.01\\
19.46	0.01\\
19.47	0.01\\
19.48	0.01\\
19.49	0.01\\
19.5	0.01\\
19.51	0.01\\
19.52	0.01\\
19.53	0.01\\
19.54	0.01\\
19.55	0.01\\
19.56	0.01\\
19.57	0.01\\
19.58	0.01\\
19.59	0.01\\
19.6	0.01\\
19.61	0.01\\
19.62	0.01\\
19.63	0.01\\
19.64	0.01\\
19.65	0.01\\
19.66	0.01\\
19.67	0.01\\
19.68	0.01\\
19.69	0.01\\
19.7	0.01\\
19.71	0.01\\
19.72	0.01\\
19.73	0.01\\
19.74	0.01\\
19.75	0.01\\
19.76	0.01\\
19.77	0.01\\
19.78	0.01\\
19.79	0.01\\
19.8	0.01\\
19.81	0.01\\
19.82	0.01\\
19.83	0.01\\
19.84	0.01\\
19.85	0.01\\
19.86	0.01\\
19.87	0.01\\
19.88	0.01\\
19.89	0.01\\
19.9	0.01\\
19.91	0.01\\
19.92	0.01\\
19.93	0.01\\
19.94	0.01\\
19.95	0.01\\
19.96	0.01\\
19.97	0.01\\
19.98	0.01\\
19.99	0.01\\
20	0.01\\
20.01	0.01\\
20.02	0.01\\
20.03	0.01\\
20.04	0.01\\
20.05	0.01\\
20.06	0.01\\
20.07	0.01\\
20.08	0.01\\
20.09	0.01\\
20.1	0.01\\
20.11	0.01\\
20.12	0.01\\
20.13	0.01\\
20.14	0.01\\
20.15	0.01\\
20.16	0.01\\
20.17	0.01\\
20.18	0.01\\
20.19	0.01\\
20.2	0.01\\
20.21	0.01\\
20.22	0.01\\
20.23	0.01\\
20.24	0.01\\
20.25	0.01\\
20.26	0.01\\
20.27	0.01\\
20.28	0.01\\
20.29	0.01\\
20.3	0.01\\
20.31	0.01\\
20.32	0.01\\
20.33	0.01\\
20.34	0.01\\
20.35	0.01\\
20.36	0.01\\
20.37	0.01\\
20.38	0.01\\
20.39	0.01\\
20.4	0.01\\
20.41	0.01\\
20.42	0.01\\
20.43	0.01\\
20.44	0.01\\
20.45	0.01\\
20.46	0.01\\
20.47	0.01\\
20.48	0.01\\
20.49	0.01\\
20.5	0.01\\
20.51	0.01\\
20.52	0.01\\
20.53	0.01\\
20.54	0.01\\
20.55	0.01\\
20.56	0.01\\
20.57	0.01\\
20.58	0.01\\
20.59	0.01\\
20.6	0.01\\
20.61	0.01\\
20.62	0.01\\
20.63	0.01\\
20.64	0.01\\
20.65	0.01\\
20.66	0.01\\
20.67	0.01\\
20.68	0.01\\
20.69	0.01\\
20.7	0.01\\
20.71	0.01\\
20.72	0.01\\
20.73	0.01\\
20.74	0.01\\
20.75	0.01\\
20.76	0.01\\
20.77	0.01\\
20.78	0.01\\
20.79	0.01\\
20.8	0.01\\
20.81	0.01\\
20.82	0.01\\
20.83	0.01\\
20.84	0.01\\
20.85	0.01\\
20.86	0.01\\
20.87	0.01\\
20.88	0.01\\
20.89	0.01\\
20.9	0.01\\
20.91	0.01\\
20.92	0.01\\
20.93	0.01\\
20.94	0.01\\
20.95	0.01\\
20.96	0.01\\
20.97	0.01\\
20.98	0.01\\
20.99	0.01\\
21	0.01\\
21.01	0.01\\
21.02	0.01\\
21.03	0.01\\
21.04	0.01\\
21.05	0.01\\
21.06	0.01\\
21.07	0.01\\
21.08	0.01\\
21.09	0.01\\
21.1	0.01\\
21.11	0.01\\
21.12	0.01\\
21.13	0.01\\
21.14	0.01\\
21.15	0.01\\
21.16	0.01\\
21.17	0.01\\
21.18	0.01\\
21.19	0.01\\
21.2	0.01\\
21.21	0.01\\
21.22	0.01\\
21.23	0.01\\
21.24	0.01\\
21.25	0.01\\
21.26	0.01\\
21.27	0.01\\
21.28	0.01\\
21.29	0.01\\
21.3	0.01\\
21.31	0.01\\
21.32	0.01\\
21.33	0.01\\
21.34	0.01\\
21.35	0.01\\
21.36	0.01\\
21.37	0.01\\
21.38	0.01\\
21.39	0.01\\
21.4	0.01\\
21.41	0.01\\
21.42	0.01\\
21.43	0.01\\
21.44	0.01\\
21.45	0.01\\
21.46	0.01\\
21.47	0.01\\
21.48	0.01\\
21.49	0.01\\
21.5	0.01\\
21.51	0.01\\
21.52	0.01\\
21.53	0.01\\
21.54	0.01\\
21.55	0.01\\
21.56	0.01\\
21.57	0.01\\
21.58	0.01\\
21.59	0.01\\
21.6	0.01\\
21.61	0.01\\
21.62	0.01\\
21.63	0.01\\
21.64	0.01\\
21.65	0.01\\
21.66	0.01\\
21.67	0.01\\
21.68	0.01\\
21.69	0.01\\
21.7	0.01\\
21.71	0.01\\
21.72	0.01\\
21.73	0.01\\
21.74	0.01\\
21.75	0.01\\
21.76	0.01\\
21.77	0.01\\
21.78	0.01\\
21.79	0.01\\
21.8	0.01\\
21.81	0.01\\
21.82	0.01\\
21.83	0.01\\
21.84	0.01\\
21.85	0.01\\
21.86	0.01\\
21.87	0.01\\
21.88	0.01\\
21.89	0.01\\
21.9	0.01\\
21.91	0.01\\
21.92	0.01\\
21.93	0.01\\
21.94	0.01\\
21.95	0.01\\
21.96	0.01\\
21.97	0.01\\
21.98	0.01\\
21.99	0.01\\
22	0.01\\
22.01	0.01\\
22.02	0.01\\
22.03	0.01\\
22.04	0.01\\
22.05	0.01\\
22.06	0.01\\
22.07	0.01\\
22.08	0.01\\
22.09	0.01\\
22.1	0.01\\
22.11	0.01\\
22.12	0.01\\
22.13	0.01\\
22.14	0.01\\
22.15	0.01\\
22.16	0.01\\
22.17	0.01\\
22.18	0.01\\
22.19	0.01\\
22.2	0.01\\
22.21	0.01\\
22.22	0.01\\
22.23	0.01\\
22.24	0.01\\
22.25	0.01\\
22.26	0.01\\
22.27	0.01\\
22.28	0.01\\
22.29	0.01\\
22.3	0.01\\
22.31	0.01\\
22.32	0.01\\
22.33	0.01\\
22.34	0.01\\
22.35	0.01\\
22.36	0.01\\
22.37	0.01\\
22.38	0.01\\
22.39	0.01\\
22.4	0.01\\
22.41	0.01\\
22.42	0.01\\
22.43	0.01\\
22.44	0.01\\
22.45	0.01\\
22.46	0.01\\
22.47	0.01\\
22.48	0.01\\
22.49	0.01\\
22.5	0.01\\
22.51	0.01\\
22.52	0.01\\
22.53	0.01\\
22.54	0.01\\
22.55	0.01\\
22.56	0.01\\
22.57	0.01\\
22.58	0.01\\
22.59	0.01\\
22.6	0.01\\
22.61	0.01\\
22.62	0.01\\
22.63	0.01\\
22.64	0.01\\
22.65	0.01\\
22.66	0.01\\
22.67	0.01\\
22.68	0.01\\
22.69	0.01\\
22.7	0.01\\
22.71	0.01\\
22.72	0.01\\
22.73	0.01\\
22.74	0.01\\
22.75	0.01\\
22.76	0.01\\
22.77	0.01\\
22.78	0.01\\
22.79	0.01\\
22.8	0.01\\
22.81	0.01\\
22.82	0.01\\
22.83	0.01\\
22.84	0.01\\
22.85	0.01\\
22.86	0.01\\
22.87	0.01\\
22.88	0.01\\
22.89	0.01\\
22.9	0.01\\
22.91	0.01\\
22.92	0.01\\
22.93	0.01\\
22.94	0.01\\
22.95	0.01\\
22.96	0.01\\
22.97	0.01\\
22.98	0.01\\
22.99	0.01\\
23	0.01\\
23.01	0.01\\
23.02	0.01\\
23.03	0.01\\
23.04	0.01\\
23.05	0.01\\
23.06	0.01\\
23.07	0.01\\
23.08	0.01\\
23.09	0.01\\
23.1	0.01\\
23.11	0.01\\
23.12	0.01\\
23.13	0.01\\
23.14	0.01\\
23.15	0.01\\
23.16	0.01\\
23.17	0.01\\
23.18	0.01\\
23.19	0.01\\
23.2	0.01\\
23.21	0.01\\
23.22	0.01\\
23.23	0.01\\
23.24	0.01\\
23.25	0.01\\
23.26	0.01\\
23.27	0.01\\
23.28	0.01\\
23.29	0.01\\
23.3	0.01\\
23.31	0.01\\
23.32	0.01\\
23.33	0.01\\
23.34	0.01\\
23.35	0.01\\
23.36	0.01\\
23.37	0.01\\
23.38	0.01\\
23.39	0.01\\
23.4	0.01\\
23.41	0.01\\
23.42	0.01\\
23.43	0.01\\
23.44	0.01\\
23.45	0.01\\
23.46	0.01\\
23.47	0.01\\
23.48	0.01\\
23.49	0.01\\
23.5	0.01\\
23.51	0.01\\
23.52	0.01\\
23.53	0.01\\
23.54	0.01\\
23.55	0.01\\
23.56	0.01\\
23.57	0.01\\
23.58	0.01\\
23.59	0.01\\
23.6	0.01\\
23.61	0.01\\
23.62	0.01\\
23.63	0.01\\
23.64	0.01\\
23.65	0.01\\
23.66	0.01\\
23.67	0.01\\
23.68	0.01\\
23.69	0.01\\
23.7	0.01\\
23.71	0.01\\
23.72	0.01\\
23.73	0.01\\
23.74	0.01\\
23.75	0.01\\
23.76	0.01\\
23.77	0.01\\
23.78	0.01\\
23.79	0.01\\
23.8	0.01\\
23.81	0.01\\
23.82	0.01\\
23.83	0.01\\
23.84	0.01\\
23.85	0.01\\
23.86	0.01\\
23.87	0.01\\
23.88	0.01\\
23.89	0.01\\
23.9	0.01\\
23.91	0.01\\
23.92	0.01\\
23.93	0.01\\
23.94	0.01\\
23.95	0.01\\
23.96	0.01\\
23.97	0.01\\
23.98	0.01\\
23.99	0.01\\
24	0.01\\
24.01	0.01\\
24.02	0.01\\
24.03	0.01\\
24.04	0.01\\
24.05	0.01\\
24.06	0.01\\
24.07	0.01\\
24.08	0.01\\
24.09	0.01\\
24.1	0.01\\
24.11	0.01\\
24.12	0.01\\
24.13	0.01\\
24.14	0.01\\
24.15	0.01\\
24.16	0.01\\
24.17	0.01\\
24.18	0.01\\
24.19	0.01\\
24.2	0.01\\
24.21	0.01\\
24.22	0.01\\
24.23	0.01\\
24.24	0.01\\
24.25	0.01\\
24.26	0.01\\
24.27	0.01\\
24.28	0.01\\
24.29	0.01\\
24.3	0.01\\
24.31	0.01\\
24.32	0.01\\
24.33	0.01\\
24.34	0.01\\
24.35	0.01\\
24.36	0.01\\
24.37	0.01\\
24.38	0.01\\
24.39	0.01\\
24.4	0.01\\
24.41	0.01\\
24.42	0.01\\
24.43	0.01\\
24.44	0.01\\
24.45	0.01\\
24.46	0.01\\
24.47	0.01\\
24.48	0.01\\
24.49	0.01\\
24.5	0.01\\
24.51	0.01\\
24.52	0.01\\
24.53	0.01\\
24.54	0.01\\
24.55	0.01\\
24.56	0.01\\
24.57	0.01\\
24.58	0.01\\
24.59	0.01\\
24.6	0.01\\
24.61	0.01\\
24.62	0.01\\
24.63	0.01\\
24.64	0.01\\
24.65	0.01\\
24.66	0.01\\
24.67	0.01\\
24.68	0.01\\
24.69	0.01\\
24.7	0.01\\
24.71	0.01\\
24.72	0.01\\
24.73	0.01\\
24.74	0.01\\
24.75	0.01\\
24.76	0.01\\
24.77	0.01\\
24.78	0.01\\
24.79	0.01\\
24.8	0.01\\
24.81	0.01\\
24.82	0.01\\
24.83	0.01\\
24.84	0.01\\
24.85	0.01\\
24.86	0.01\\
24.87	0.01\\
24.88	0.01\\
24.89	0.01\\
24.9	0.01\\
24.91	0.01\\
24.92	0.01\\
24.93	0.01\\
24.94	0.01\\
24.95	0.01\\
24.96	0.01\\
24.97	0.01\\
24.98	0.01\\
24.99	0.01\\
25	0.01\\
25.01	0.01\\
25.02	0.01\\
25.03	0.01\\
25.04	0.01\\
25.05	0.01\\
25.06	0.01\\
25.07	0.01\\
25.08	0.01\\
25.09	0.01\\
25.1	0.01\\
25.11	0.01\\
25.12	0.01\\
25.13	0.01\\
25.14	0.01\\
25.15	0.01\\
25.16	0.01\\
25.17	0.01\\
25.18	0.01\\
25.19	0.01\\
25.2	0.01\\
25.21	0.01\\
25.22	0.01\\
25.23	0.01\\
25.24	0.01\\
25.25	0.01\\
25.26	0.01\\
25.27	0.01\\
25.28	0.01\\
25.29	0.01\\
25.3	0.01\\
25.31	0.01\\
25.32	0.01\\
25.33	0.01\\
25.34	0.01\\
25.35	0.01\\
25.36	0.01\\
25.37	0.01\\
25.38	0.01\\
25.39	0.01\\
25.4	0.01\\
25.41	0.01\\
25.42	0.01\\
25.43	0.01\\
25.44	0.01\\
25.45	0.01\\
25.46	0.01\\
25.47	0.01\\
25.48	0.01\\
25.49	0.01\\
25.5	0.01\\
25.51	0.01\\
25.52	0.01\\
25.53	0.01\\
25.54	0.01\\
25.55	0.01\\
25.56	0.01\\
25.57	0.01\\
25.58	0.01\\
25.59	0.01\\
25.6	0.01\\
25.61	0.01\\
25.62	0.01\\
25.63	0.01\\
25.64	0.01\\
25.65	0.01\\
25.66	0.01\\
25.67	0.01\\
25.68	0.01\\
25.69	0.01\\
25.7	0.01\\
25.71	0.01\\
25.72	0.01\\
25.73	0.01\\
25.74	0.01\\
25.75	0.01\\
25.76	0.01\\
25.77	0.01\\
25.78	0.01\\
25.79	0.01\\
25.8	0.01\\
25.81	0.01\\
25.82	0.01\\
25.83	0.01\\
25.84	0.01\\
25.85	0.01\\
25.86	0.01\\
25.87	0.01\\
25.88	0.01\\
25.89	0.01\\
25.9	0.01\\
25.91	0.01\\
25.92	0.01\\
25.93	0.01\\
25.94	0.01\\
25.95	0.01\\
25.96	0.01\\
25.97	0.01\\
25.98	0.01\\
25.99	0.01\\
26	0.01\\
26.01	0.01\\
26.02	0.01\\
26.03	0.01\\
26.04	0.01\\
26.05	0.01\\
26.06	0.01\\
26.07	0.01\\
26.08	0.01\\
26.09	0.01\\
26.1	0.01\\
26.11	0.01\\
26.12	0.01\\
26.13	0.01\\
26.14	0.01\\
26.15	0.01\\
26.16	0.01\\
26.17	0.01\\
26.18	0.01\\
26.19	0.01\\
26.2	0.01\\
26.21	0.01\\
26.22	0.01\\
26.23	0.01\\
26.24	0.01\\
26.25	0.01\\
26.26	0.01\\
26.27	0.01\\
26.28	0.01\\
26.29	0.01\\
26.3	0.01\\
26.31	0.01\\
26.32	0.01\\
26.33	0.01\\
26.34	0.01\\
26.35	0.01\\
26.36	0.01\\
26.37	0.01\\
26.38	0.01\\
26.39	0.01\\
26.4	0.01\\
26.41	0.01\\
26.42	0.01\\
26.43	0.01\\
26.44	0.01\\
26.45	0.01\\
26.46	0.01\\
26.47	0.01\\
26.48	0.01\\
26.49	0.01\\
26.5	0.01\\
26.51	0.01\\
26.52	0.01\\
26.53	0.01\\
26.54	0.01\\
26.55	0.01\\
26.56	0.01\\
26.57	0.01\\
26.58	0.01\\
26.59	0.01\\
26.6	0.01\\
26.61	0.01\\
26.62	0.01\\
26.63	0.01\\
26.64	0.01\\
26.65	0.01\\
26.66	0.01\\
26.67	0.01\\
26.68	0.01\\
26.69	0.01\\
26.7	0.01\\
26.71	0.01\\
26.72	0.01\\
26.73	0.01\\
26.74	0.01\\
26.75	0.01\\
26.76	0.01\\
26.77	0.01\\
26.78	0.01\\
26.79	0.01\\
26.8	0.01\\
26.81	0.01\\
26.82	0.01\\
26.83	0.01\\
26.84	0.01\\
26.85	0.01\\
26.86	0.01\\
26.87	0.01\\
26.88	0.01\\
26.89	0.01\\
26.9	0.01\\
26.91	0.01\\
26.92	0.01\\
26.93	0.01\\
26.94	0.01\\
26.95	0.01\\
26.96	0.01\\
26.97	0.01\\
26.98	0.01\\
26.99	0.01\\
27	0.01\\
27.01	0.01\\
27.02	0.01\\
27.03	0.01\\
27.04	0.01\\
27.05	0.01\\
27.06	0.01\\
27.07	0.01\\
27.08	0.01\\
27.09	0.01\\
27.1	0.01\\
27.11	0.01\\
27.12	0.01\\
27.13	0.01\\
27.14	0.01\\
27.15	0.01\\
27.16	0.01\\
27.17	0.01\\
27.18	0.01\\
27.19	0.01\\
27.2	0.01\\
27.21	0.01\\
27.22	0.01\\
27.23	0.01\\
27.24	0.01\\
27.25	0.01\\
27.26	0.01\\
27.27	0.01\\
27.28	0.01\\
27.29	0.01\\
27.3	0.01\\
27.31	0.01\\
27.32	0.01\\
27.33	0.01\\
27.34	0.01\\
27.35	0.01\\
27.36	0.01\\
27.37	0.01\\
27.38	0.01\\
27.39	0.01\\
27.4	0.01\\
27.41	0.01\\
27.42	0.01\\
27.43	0.01\\
27.44	0.01\\
27.45	0.01\\
27.46	0.01\\
27.47	0.01\\
27.48	0.01\\
27.49	0.01\\
27.5	0.01\\
27.51	0.01\\
27.52	0.01\\
27.53	0.01\\
27.54	0.01\\
27.55	0.01\\
27.56	0.01\\
27.57	0.01\\
27.58	0.01\\
27.59	0.01\\
27.6	0.01\\
27.61	0.01\\
27.62	0.01\\
27.63	0.01\\
27.64	0.01\\
27.65	0.01\\
27.66	0.01\\
27.67	0.01\\
27.68	0.01\\
27.69	0.01\\
27.7	0.01\\
27.71	0.01\\
27.72	0.01\\
27.73	0.01\\
27.74	0.01\\
27.75	0.01\\
27.76	0.01\\
27.77	0.01\\
27.78	0.01\\
27.79	0.01\\
27.8	0.01\\
27.81	0.01\\
27.82	0.01\\
27.83	0.01\\
27.84	0.01\\
27.85	0.01\\
27.86	0.01\\
27.87	0.01\\
27.88	0.01\\
27.89	0.01\\
27.9	0.01\\
27.91	0.01\\
27.92	0.01\\
27.93	0.01\\
27.94	0.01\\
27.95	0.01\\
27.96	0.01\\
27.97	0.01\\
27.98	0.01\\
27.99	0.01\\
28	0.01\\
28.01	0.01\\
28.02	0.01\\
28.03	0.01\\
28.04	0.01\\
28.05	0.01\\
28.06	0.01\\
28.07	0.01\\
28.08	0.01\\
28.09	0.01\\
28.1	0.01\\
28.11	0.01\\
28.12	0.01\\
28.13	0.01\\
28.14	0.01\\
28.15	0.01\\
28.16	0.01\\
28.17	0.01\\
28.18	0.01\\
28.19	0.01\\
28.2	0.01\\
28.21	0.01\\
28.22	0.01\\
28.23	0.01\\
28.24	0.01\\
28.25	0.01\\
28.26	0.01\\
28.27	0.01\\
28.28	0.01\\
28.29	0.01\\
28.3	0.01\\
28.31	0.01\\
28.32	0.01\\
28.33	0.01\\
28.34	0.01\\
28.35	0.01\\
28.36	0.01\\
28.37	0.01\\
28.38	0.01\\
28.39	0.01\\
28.4	0.01\\
28.41	0.01\\
28.42	0.01\\
28.43	0.01\\
28.44	0.01\\
28.45	0.01\\
28.46	0.01\\
28.47	0.01\\
28.48	0.01\\
28.49	0.01\\
28.5	0.01\\
28.51	0.01\\
28.52	0.01\\
28.53	0.01\\
28.54	0.01\\
28.55	0.01\\
28.56	0.01\\
28.57	0.01\\
28.58	0.01\\
28.59	0.01\\
28.6	0.01\\
28.61	0.01\\
28.62	0.01\\
28.63	0.01\\
28.64	0.01\\
28.65	0.01\\
28.66	0.01\\
28.67	0.01\\
28.68	0.01\\
28.69	0.01\\
28.7	0.01\\
28.71	0.01\\
28.72	0.01\\
28.73	0.01\\
28.74	0.01\\
28.75	0.01\\
28.76	0.01\\
28.77	0.01\\
28.78	0.01\\
28.79	0.01\\
28.8	0.01\\
28.81	0.01\\
28.82	0.01\\
28.83	0.01\\
28.84	0.01\\
28.85	0.01\\
28.86	0.01\\
28.87	0.01\\
28.88	0.01\\
28.89	0.01\\
28.9	0.01\\
28.91	0.01\\
28.92	0.01\\
28.93	0.01\\
28.94	0.01\\
28.95	0.01\\
28.96	0.01\\
28.97	0.01\\
28.98	0.01\\
28.99	0.01\\
29	0.01\\
29.01	0.01\\
29.02	0.01\\
29.03	0.01\\
29.04	0.01\\
29.05	0.01\\
29.06	0.01\\
29.07	0.01\\
29.08	0.01\\
29.09	0.01\\
29.1	0.01\\
29.11	0.01\\
29.12	0.01\\
29.13	0.01\\
29.14	0.01\\
29.15	0.01\\
29.16	0.01\\
29.17	0.01\\
29.18	0.01\\
29.19	0.01\\
29.2	0.01\\
29.21	0.01\\
29.22	0.01\\
29.23	0.01\\
29.24	0.01\\
29.25	0.01\\
29.26	0.01\\
29.27	0.01\\
29.28	0.01\\
29.29	0.01\\
29.3	0.01\\
29.31	0.01\\
29.32	0.01\\
29.33	0.01\\
29.34	0.01\\
29.35	0.01\\
29.36	0.01\\
29.37	0.01\\
29.38	0.01\\
29.39	0.01\\
29.4	0.01\\
29.41	0.01\\
29.42	0.01\\
29.43	0.01\\
29.44	0.01\\
29.45	0.01\\
29.46	0.01\\
29.47	0.01\\
29.48	0.01\\
29.49	0.01\\
29.5	0.01\\
29.51	0.01\\
29.52	0.01\\
29.53	0.01\\
29.54	0.01\\
29.55	0.01\\
29.56	0.01\\
29.57	0.01\\
29.58	0.01\\
29.59	0.01\\
29.6	0.01\\
29.61	0.01\\
29.62	0.01\\
29.63	0.01\\
29.64	0.01\\
29.65	0.01\\
29.66	0.01\\
29.67	0.01\\
29.68	0.01\\
29.69	0.01\\
29.7	0.01\\
29.71	0.01\\
29.72	0.01\\
29.73	0.01\\
29.74	0.01\\
29.75	0.01\\
29.76	0.01\\
29.77	0.01\\
29.78	0.01\\
29.79	0.01\\
29.8	0.01\\
29.81	0.01\\
29.82	0.01\\
29.83	0.01\\
29.84	0.01\\
29.85	0.01\\
29.86	0.01\\
29.87	0.01\\
29.88	0.01\\
29.89	0.01\\
29.9	0.01\\
29.91	0.01\\
29.92	0.01\\
29.93	0.01\\
29.94	0.01\\
29.95	0.01\\
29.96	0.01\\
29.97	0.01\\
29.98	0.01\\
29.99	0.01\\
30	0.01\\
30.01	0.01\\
30.02	0.01\\
30.03	0.01\\
30.04	0.01\\
30.05	0.01\\
30.06	0.01\\
30.07	0.01\\
30.08	0.01\\
30.09	0.01\\
30.1	0.01\\
30.11	0.01\\
30.12	0.01\\
30.13	0.01\\
30.14	0.01\\
30.15	0.01\\
30.16	0.01\\
30.17	0.01\\
30.18	0.01\\
30.19	0.01\\
30.2	0.01\\
30.21	0.01\\
30.22	0.01\\
30.23	0.01\\
30.24	0.01\\
30.25	0.01\\
30.26	0.01\\
30.27	0.01\\
30.28	0.01\\
30.29	0.01\\
30.3	0.01\\
30.31	0.01\\
30.32	0.01\\
30.33	0.01\\
30.34	0.01\\
30.35	0.01\\
30.36	0.01\\
30.37	0.01\\
30.38	0.01\\
30.39	0.01\\
30.4	0.01\\
30.41	0.01\\
30.42	0.01\\
30.43	0.01\\
30.44	0.01\\
30.45	0.01\\
30.46	0.01\\
30.47	0.01\\
30.48	0.01\\
30.49	0.01\\
30.5	0.01\\
30.51	0.01\\
30.52	0.01\\
30.53	0.01\\
30.54	0.01\\
30.55	0.01\\
30.56	0.01\\
30.57	0.01\\
30.58	0.01\\
30.59	0.01\\
30.6	0.01\\
30.61	0.01\\
30.62	0.01\\
30.63	0.01\\
30.64	0.01\\
30.65	0.01\\
30.66	0.01\\
30.67	0.01\\
30.68	0.01\\
30.69	0.01\\
30.7	0.01\\
30.71	0.01\\
30.72	0.01\\
30.73	0.01\\
30.74	0.01\\
30.75	0.01\\
30.76	0.01\\
30.77	0.01\\
30.78	0.01\\
30.79	0.01\\
30.8	0.01\\
30.81	0.01\\
30.82	0.01\\
30.83	0.01\\
30.84	0.01\\
30.85	0.01\\
30.86	0.01\\
30.87	0.01\\
30.88	0.01\\
30.89	0.01\\
30.9	0.01\\
30.91	0.01\\
30.92	0.01\\
30.93	0.01\\
30.94	0.01\\
30.95	0.01\\
30.96	0.01\\
30.97	0.01\\
30.98	0.01\\
30.99	0.01\\
31	0.01\\
31.01	0.01\\
31.02	0.01\\
31.03	0.01\\
31.04	0.01\\
31.05	0.01\\
31.06	0.01\\
31.07	0.01\\
31.08	0.01\\
31.09	0.01\\
31.1	0.01\\
31.11	0.01\\
31.12	0.01\\
31.13	0.01\\
31.14	0.01\\
31.15	0.01\\
31.16	0.01\\
31.17	0.01\\
31.18	0.01\\
31.19	0.01\\
31.2	0.01\\
31.21	0.01\\
31.22	0.01\\
31.23	0.01\\
31.24	0.01\\
31.25	0.01\\
31.26	0.01\\
31.27	0.01\\
31.28	0.01\\
31.29	0.01\\
31.3	0.01\\
31.31	0.01\\
31.32	0.01\\
31.33	0.01\\
31.34	0.01\\
31.35	0.01\\
31.36	0.01\\
31.37	0.01\\
31.38	0.01\\
31.39	0.01\\
31.4	0.01\\
31.41	0.01\\
31.42	0.01\\
31.43	0.01\\
31.44	0.01\\
31.45	0.01\\
31.46	0.01\\
31.47	0.01\\
31.48	0.01\\
31.49	0.01\\
31.5	0.01\\
31.51	0.01\\
31.52	0.01\\
31.53	0.01\\
31.54	0.01\\
31.55	0.01\\
31.56	0.01\\
31.57	0.01\\
31.58	0.01\\
31.59	0.01\\
31.6	0.01\\
31.61	0.01\\
31.62	0.01\\
31.63	0.01\\
31.64	0.01\\
31.65	0.01\\
31.66	0.01\\
31.67	0.01\\
31.68	0.01\\
31.69	0.01\\
31.7	0.01\\
31.71	0.01\\
31.72	0.01\\
31.73	0.01\\
31.74	0.01\\
31.75	0.01\\
31.76	0.01\\
31.77	0.01\\
31.78	0.01\\
31.79	0.01\\
31.8	0.01\\
31.81	0.01\\
31.82	0.01\\
31.83	0.01\\
31.84	0.01\\
31.85	0.01\\
31.86	0.01\\
31.87	0.01\\
31.88	0.01\\
31.89	0.01\\
31.9	0.01\\
31.91	0.01\\
31.92	0.01\\
31.93	0.01\\
31.94	0.01\\
31.95	0.01\\
31.96	0.01\\
31.97	0.01\\
31.98	0.01\\
31.99	0.01\\
32	0.01\\
32.01	0.01\\
32.02	0.01\\
32.03	0.01\\
32.04	0.01\\
32.05	0.01\\
32.06	0.01\\
32.07	0.01\\
32.08	0.01\\
32.09	0.01\\
32.1	0.01\\
32.11	0.01\\
32.12	0.01\\
32.13	0.01\\
32.14	0.01\\
32.15	0.01\\
32.16	0.01\\
32.17	0.01\\
32.18	0.01\\
32.19	0.01\\
32.2	0.01\\
32.21	0.01\\
32.22	0.01\\
32.23	0.01\\
32.24	0.01\\
32.25	0.01\\
32.26	0.01\\
32.27	0.01\\
32.28	0.01\\
32.29	0.01\\
32.3	0.01\\
32.31	0.01\\
32.32	0.01\\
32.33	0.01\\
32.34	0.01\\
32.35	0.01\\
32.36	0.01\\
32.37	0.01\\
32.38	0.01\\
32.39	0.01\\
32.4	0.01\\
32.41	0.01\\
32.42	0.01\\
32.43	0.01\\
32.44	0.01\\
32.45	0.01\\
32.46	0.01\\
32.47	0.01\\
32.48	0.01\\
32.49	0.01\\
32.5	0.01\\
32.51	0.01\\
32.52	0.01\\
32.53	0.01\\
32.54	0.01\\
32.55	0.01\\
32.56	0.01\\
32.57	0.01\\
32.58	0.01\\
32.59	0.01\\
32.6	0.01\\
32.61	0.01\\
32.62	0.01\\
32.63	0.01\\
32.64	0.01\\
32.65	0.01\\
32.66	0.01\\
32.67	0.01\\
32.68	0.01\\
32.69	0.01\\
32.7	0.01\\
32.71	0.01\\
32.72	0.01\\
32.73	0.01\\
32.74	0.01\\
32.75	0.01\\
32.76	0.01\\
32.77	0.01\\
32.78	0.01\\
32.79	0.01\\
32.8	0.01\\
32.81	0.01\\
32.82	0.01\\
32.83	0.01\\
32.84	0.01\\
32.85	0.01\\
32.86	0.01\\
32.87	0.01\\
32.88	0.01\\
32.89	0.01\\
32.9	0.01\\
32.91	0.01\\
32.92	0.01\\
32.93	0.01\\
32.94	0.01\\
32.95	0.01\\
32.96	0.01\\
32.97	0.01\\
32.98	0.01\\
32.99	0.01\\
33	0.01\\
33.01	0.01\\
33.02	0.01\\
33.03	0.01\\
33.04	0.01\\
33.05	0.01\\
33.06	0.01\\
33.07	0.01\\
33.08	0.01\\
33.09	0.01\\
33.1	0.01\\
33.11	0.01\\
33.12	0.01\\
33.13	0.01\\
33.14	0.01\\
33.15	0.01\\
33.16	0.01\\
33.17	0.01\\
33.18	0.01\\
33.19	0.01\\
33.2	0.01\\
33.21	0.01\\
33.22	0.01\\
33.23	0.01\\
33.24	0.01\\
33.25	0.01\\
33.26	0.01\\
33.27	0.01\\
33.28	0.01\\
33.29	0.01\\
33.3	0.01\\
33.31	0.01\\
33.32	0.01\\
33.33	0.01\\
33.34	0.01\\
33.35	0.01\\
33.36	0.01\\
33.37	0.01\\
33.38	0.01\\
33.39	0.01\\
33.4	0.01\\
33.41	0.01\\
33.42	0.01\\
33.43	0.01\\
33.44	0.01\\
33.45	0.01\\
33.46	0.01\\
33.47	0.01\\
33.48	0.01\\
33.49	0.01\\
33.5	0.01\\
33.51	0.01\\
33.52	0.01\\
33.53	0.01\\
33.54	0.01\\
33.55	0.01\\
33.56	0.01\\
33.57	0.01\\
33.58	0.01\\
33.59	0.01\\
33.6	0.01\\
33.61	0.01\\
33.62	0.01\\
33.63	0.01\\
33.64	0.01\\
33.65	0.01\\
33.66	0.01\\
33.67	0.01\\
33.68	0.01\\
33.69	0.01\\
33.7	0.01\\
33.71	0.01\\
33.72	0.01\\
33.73	0.01\\
33.74	0.01\\
33.75	0.01\\
33.76	0.01\\
33.77	0.01\\
33.78	0.01\\
33.79	0.01\\
33.8	0.01\\
33.81	0.01\\
33.82	0.01\\
33.83	0.01\\
33.84	0.01\\
33.85	0.01\\
33.86	0.01\\
33.87	0.01\\
33.88	0.01\\
33.89	0.01\\
33.9	0.01\\
33.91	0.01\\
33.92	0.01\\
33.93	0.01\\
33.94	0.01\\
33.95	0.01\\
33.96	0.01\\
33.97	0.01\\
33.98	0.01\\
33.99	0.01\\
34	0.01\\
34.01	0.01\\
34.02	0.01\\
34.03	0.01\\
34.04	0.01\\
34.05	0.01\\
34.06	0.01\\
34.07	0.01\\
34.08	0.01\\
34.09	0.01\\
34.1	0.01\\
34.11	0.01\\
34.12	0.01\\
34.13	0.01\\
34.14	0.01\\
34.15	0.01\\
34.16	0.01\\
34.17	0.01\\
34.18	0.01\\
34.19	0.01\\
34.2	0.01\\
34.21	0.01\\
34.22	0.01\\
34.23	0.01\\
34.24	0.01\\
34.25	0.01\\
34.26	0.01\\
34.27	0.01\\
34.28	0.01\\
34.29	0.01\\
34.3	0.01\\
34.31	0.01\\
34.32	0.01\\
34.33	0.01\\
34.34	0.01\\
34.35	0.01\\
34.36	0.01\\
34.37	0.01\\
34.38	0.01\\
34.39	0.01\\
34.4	0.01\\
34.41	0.01\\
34.42	0.01\\
34.43	0.01\\
34.44	0.01\\
34.45	0.01\\
34.46	0.01\\
34.47	0.01\\
34.48	0.01\\
34.49	0.01\\
34.5	0.01\\
34.51	0.01\\
34.52	0.01\\
34.53	0.01\\
34.54	0.01\\
34.55	0.01\\
34.56	0.01\\
34.57	0.01\\
34.58	0.01\\
34.59	0.01\\
34.6	0.01\\
34.61	0.01\\
34.62	0.01\\
34.63	0.01\\
34.64	0.01\\
34.65	0.01\\
34.66	0.01\\
34.67	0.01\\
34.68	0.01\\
34.69	0.01\\
34.7	0.01\\
34.71	0.01\\
34.72	0.01\\
34.73	0.01\\
34.74	0.01\\
34.75	0.01\\
34.76	0.01\\
34.77	0.01\\
34.78	0.01\\
34.79	0.01\\
34.8	0.01\\
34.81	0.01\\
34.82	0.01\\
34.83	0.01\\
34.84	0.01\\
34.85	0.01\\
34.86	0.01\\
34.87	0.01\\
34.88	0.01\\
34.89	0.01\\
34.9	0.01\\
34.91	0.01\\
34.92	0.01\\
34.93	0.01\\
34.94	0.01\\
34.95	0.01\\
34.96	0.01\\
34.97	0.01\\
34.98	0.01\\
34.99	0.01\\
35	0.01\\
35.01	0.01\\
35.02	0.01\\
35.03	0.01\\
35.04	0.01\\
35.05	0.01\\
35.06	0.01\\
35.07	0.01\\
35.08	0.01\\
35.09	0.01\\
35.1	0.01\\
35.11	0.01\\
35.12	0.01\\
35.13	0.01\\
35.14	0.01\\
35.15	0.01\\
35.16	0.01\\
35.17	0.01\\
35.18	0.01\\
35.19	0.01\\
35.2	0.01\\
35.21	0.01\\
35.22	0.01\\
35.23	0.01\\
35.24	0.01\\
35.25	0.01\\
35.26	0.01\\
35.27	0.01\\
35.28	0.01\\
35.29	0.01\\
35.3	0.01\\
35.31	0.01\\
35.32	0.01\\
35.33	0.01\\
35.34	0.01\\
35.35	0.01\\
35.36	0.01\\
35.37	0.01\\
35.38	0.01\\
35.39	0.01\\
35.4	0.01\\
35.41	0.01\\
35.42	0.01\\
35.43	0.01\\
35.44	0.01\\
35.45	0.01\\
35.46	0.01\\
35.47	0.01\\
35.48	0.01\\
35.49	0.01\\
35.5	0.01\\
35.51	0.01\\
35.52	0.01\\
35.53	0.01\\
35.54	0.01\\
35.55	0.01\\
35.56	0.01\\
35.57	0.01\\
35.58	0.01\\
35.59	0.01\\
35.6	0.01\\
35.61	0.01\\
35.62	0.01\\
35.63	0.01\\
35.64	0.01\\
35.65	0.01\\
35.66	0.01\\
35.67	0.01\\
35.68	0.01\\
35.69	0.01\\
35.7	0.01\\
35.71	0.01\\
35.72	0.01\\
35.73	0.01\\
35.74	0.01\\
35.75	0.01\\
35.76	0.01\\
35.77	0.01\\
35.78	0.01\\
35.79	0.01\\
35.8	0.01\\
35.81	0.01\\
35.82	0.01\\
35.83	0.01\\
35.84	0.01\\
35.85	0.01\\
35.86	0.01\\
35.87	0.01\\
35.88	0.01\\
35.89	0.01\\
35.9	0.01\\
35.91	0.01\\
35.92	0.01\\
35.93	0.01\\
35.94	0.01\\
35.95	0.01\\
35.96	0.01\\
35.97	0.01\\
35.98	0.01\\
35.99	0.01\\
36	0.01\\
36.01	0.01\\
36.02	0.01\\
36.03	0.01\\
36.04	0.01\\
36.05	0.01\\
36.06	0.01\\
36.07	0.01\\
36.08	0.01\\
36.09	0.01\\
36.1	0.01\\
36.11	0.01\\
36.12	0.01\\
36.13	0.01\\
36.14	0.01\\
36.15	0.01\\
36.16	0.01\\
36.17	0.01\\
36.18	0.01\\
36.19	0.01\\
36.2	0.01\\
36.21	0.01\\
36.22	0.01\\
36.23	0.01\\
36.24	0.01\\
36.25	0.01\\
36.26	0.01\\
36.27	0.01\\
36.28	0.01\\
36.29	0.01\\
36.3	0.01\\
36.31	0.01\\
36.32	0.01\\
36.33	0.01\\
36.34	0.01\\
36.35	0.01\\
36.36	0.01\\
36.37	0.01\\
36.38	0.01\\
36.39	0.01\\
36.4	0.01\\
36.41	0.01\\
36.42	0.01\\
36.43	0.01\\
36.44	0.01\\
36.45	0.01\\
36.46	0.01\\
36.47	0.01\\
36.48	0.01\\
36.49	0.01\\
36.5	0.01\\
36.51	0.01\\
36.52	0.01\\
36.53	0.01\\
36.54	0.01\\
36.55	0.01\\
36.56	0.01\\
36.57	0.01\\
36.58	0.01\\
36.59	0.01\\
36.6	0.01\\
36.61	0.01\\
36.62	0.01\\
36.63	0.01\\
36.64	0.01\\
36.65	0.01\\
36.66	0.01\\
36.67	0.01\\
36.68	0.01\\
36.69	0.01\\
36.7	0.01\\
36.71	0.01\\
36.72	0.01\\
36.73	0.01\\
36.74	0.01\\
36.75	0.01\\
36.76	0.01\\
36.77	0.01\\
36.78	0.01\\
36.79	0.01\\
36.8	0.01\\
36.81	0.01\\
36.82	0.01\\
36.83	0.01\\
36.84	0.01\\
36.85	0.01\\
36.86	0.01\\
36.87	0.01\\
36.88	0.01\\
36.89	0.01\\
36.9	0.01\\
36.91	0.01\\
36.92	0.01\\
36.93	0.01\\
36.94	0.01\\
36.95	0.01\\
36.96	0.01\\
36.97	0.01\\
36.98	0.01\\
36.99	0.01\\
37	0.01\\
37.01	0.01\\
37.02	0.01\\
37.03	0.01\\
37.04	0.01\\
37.05	0.01\\
37.06	0.01\\
37.07	0.01\\
37.08	0.01\\
37.09	0.01\\
37.1	0.01\\
37.11	0.01\\
37.12	0.01\\
37.13	0.01\\
37.14	0.01\\
37.15	0.01\\
37.16	0.01\\
37.17	0.01\\
37.18	0.01\\
37.19	0.01\\
37.2	0.01\\
37.21	0.01\\
37.22	0.01\\
37.23	0.01\\
37.24	0.01\\
37.25	0.01\\
37.26	0.01\\
37.27	0.01\\
37.28	0.01\\
37.29	0.01\\
37.3	0.01\\
37.31	0.01\\
37.32	0.01\\
37.33	0.01\\
37.34	0.01\\
37.35	0.01\\
37.36	0.01\\
37.37	0.01\\
37.38	0.01\\
37.39	0.01\\
37.4	0.01\\
37.41	0.01\\
37.42	0.01\\
37.43	0.01\\
37.44	0.01\\
37.45	0.01\\
37.46	0.01\\
37.47	0.01\\
37.48	0.01\\
37.49	0.01\\
37.5	0.01\\
37.51	0.01\\
37.52	0.01\\
37.53	0.01\\
37.54	0.01\\
37.55	0.01\\
37.56	0.01\\
37.57	0.01\\
37.58	0.01\\
37.59	0.01\\
37.6	0.01\\
37.61	0.01\\
37.62	0.01\\
37.63	0.01\\
37.64	0.01\\
37.65	0.01\\
37.66	0.01\\
37.67	0.01\\
37.68	0.01\\
37.69	0.01\\
37.7	0.01\\
37.71	0.01\\
37.72	0.01\\
37.73	0.01\\
37.74	0.01\\
37.75	0.01\\
37.76	0.01\\
37.77	0.01\\
37.78	0.01\\
37.79	0.01\\
37.8	0.01\\
37.81	0.01\\
37.82	0.01\\
37.83	0.01\\
37.84	0.01\\
37.85	0.01\\
37.86	0.01\\
37.87	0.01\\
37.88	0.01\\
37.89	0.01\\
37.9	0.01\\
37.91	0.01\\
37.92	0.01\\
37.93	0.01\\
37.94	0.01\\
37.95	0.01\\
37.96	0.01\\
37.97	0.01\\
37.98	0.01\\
37.99	0.01\\
38	0.01\\
38.01	0.01\\
38.02	0.01\\
38.03	0.01\\
38.04	0.01\\
38.05	0.01\\
38.06	0.01\\
38.07	0.01\\
38.08	0.01\\
38.09	0.01\\
38.1	0.01\\
38.11	0.01\\
38.12	0.01\\
38.13	0.01\\
38.14	0.01\\
38.15	0.01\\
38.16	0.01\\
38.17	0.01\\
38.18	0.01\\
38.19	0.01\\
38.2	0.01\\
38.21	0.01\\
38.22	0.01\\
38.23	0.01\\
38.24	0.01\\
38.25	0.01\\
38.26	0.01\\
38.27	0.01\\
38.28	0.01\\
38.29	0.01\\
38.3	0.01\\
38.31	0.01\\
38.32	0.01\\
38.33	0.01\\
38.34	0.01\\
38.35	0.01\\
38.36	0.01\\
38.37	0.01\\
38.38	0.01\\
38.39	0.01\\
38.4	0.01\\
38.41	0.01\\
38.42	0.01\\
38.43	0.01\\
38.44	0.01\\
38.45	0.01\\
38.46	0.01\\
38.47	0.01\\
38.48	0.01\\
38.49	0.01\\
38.5	0.01\\
38.51	0.01\\
38.52	0.01\\
38.53	0.01\\
38.54	0.01\\
38.55	0.01\\
38.56	0.01\\
38.57	0.01\\
38.58	0.01\\
38.59	0.01\\
38.6	0.01\\
38.61	0.01\\
38.62	0.01\\
38.63	0.01\\
38.64	0.01\\
38.65	0.01\\
38.66	0.01\\
38.67	0.01\\
38.68	0.01\\
38.69	0.01\\
38.7	0.01\\
38.71	0.01\\
38.72	0.01\\
38.73	0.01\\
38.74	0.01\\
38.75	0.01\\
38.76	0.01\\
38.77	0.01\\
38.78	0.01\\
38.79	0.01\\
38.8	0.01\\
38.81	0.01\\
38.82	0.01\\
38.83	0.01\\
38.84	0.01\\
38.85	0.01\\
38.86	0.01\\
38.87	0.01\\
38.88	0.01\\
38.89	0.01\\
38.9	0.01\\
38.91	0.01\\
38.92	0.01\\
38.93	0.01\\
38.94	0.01\\
38.95	0.01\\
38.96	0.01\\
38.97	0.01\\
38.98	0.01\\
38.99	0.01\\
39	0.01\\
39.01	0.01\\
39.02	0.01\\
39.03	0.01\\
39.04	0.01\\
39.05	0.01\\
39.06	0.01\\
39.07	0.01\\
39.08	0.01\\
39.09	0.01\\
39.1	0.01\\
39.11	0.01\\
39.12	0.01\\
39.13	0.01\\
39.14	0.01\\
39.15	0.01\\
39.16	0.01\\
39.17	0.01\\
39.18	0.01\\
39.19	0.01\\
39.2	0.01\\
39.21	0.01\\
39.22	0.01\\
39.23	0.01\\
39.24	0.01\\
39.25	0.01\\
39.26	0.01\\
39.27	0.01\\
39.28	0.01\\
39.29	0.01\\
39.3	0.01\\
39.31	0.01\\
39.32	0.01\\
39.33	0.01\\
39.34	0.01\\
39.35	0.01\\
39.36	0.01\\
39.37	0.01\\
39.38	0.01\\
39.39	0.01\\
39.4	0.01\\
39.41	0.01\\
39.42	0.01\\
39.43	0.01\\
39.44	0.01\\
39.45	0.01\\
39.46	0.01\\
39.47	0.01\\
39.48	0.01\\
39.49	0.01\\
39.5	0.01\\
39.51	0.01\\
39.52	0.01\\
39.53	0.01\\
39.54	0.01\\
39.55	0.01\\
39.56	0.01\\
39.57	0.01\\
39.58	0.01\\
39.59	0.01\\
39.6	0.01\\
39.61	0.01\\
39.62	0.01\\
39.63	0.01\\
39.64	0.01\\
39.65	0.01\\
39.66	0.01\\
39.67	0.01\\
39.68	0.01\\
39.69	0.01\\
39.7	0.01\\
39.71	0.01\\
39.72	0.01\\
39.73	0.01\\
39.74	0.01\\
39.75	0.01\\
39.76	0.01\\
39.77	0.01\\
39.78	0.01\\
39.79	0.01\\
39.8	0.01\\
39.81	0.01\\
39.82	0.01\\
39.83	0.01\\
39.84	0.01\\
39.85	0.01\\
39.86	0.01\\
39.87	0.01\\
39.88	0.01\\
39.89	0.01\\
39.9	0.01\\
39.91	0.01\\
39.92	0.01\\
39.93	0.01\\
39.94	0.01\\
39.95	0.01\\
39.96	0.01\\
39.97	0.01\\
39.98	0.01\\
39.99	0.01\\
40	0.01\\
40.01	0.01\\
};
\addplot [color=blue,solid,forget plot]
  table[row sep=crcr]{%
40.01	0.01\\
40.02	0.01\\
40.03	0.01\\
40.04	0.01\\
40.05	0.01\\
40.06	0.01\\
40.07	0.01\\
40.08	0.01\\
40.09	0.01\\
40.1	0.01\\
40.11	0.01\\
40.12	0.01\\
40.13	0.01\\
40.14	0.01\\
40.15	0.01\\
40.16	0.01\\
40.17	0.01\\
40.18	0.01\\
40.19	0.01\\
40.2	0.01\\
40.21	0.01\\
40.22	0.01\\
40.23	0.01\\
40.24	0.01\\
40.25	0.01\\
40.26	0.01\\
40.27	0.01\\
40.28	0.01\\
40.29	0.01\\
40.3	0.01\\
40.31	0.01\\
40.32	0.01\\
40.33	0.01\\
40.34	0.01\\
40.35	0.01\\
40.36	0.01\\
40.37	0.01\\
40.38	0.01\\
40.39	0.01\\
40.4	0.01\\
40.41	0.01\\
40.42	0.01\\
40.43	0.01\\
40.44	0.01\\
40.45	0.01\\
40.46	0.01\\
40.47	0.01\\
40.48	0.01\\
40.49	0.01\\
40.5	0.01\\
40.51	0.01\\
40.52	0.01\\
40.53	0.01\\
40.54	0.01\\
40.55	0.01\\
40.56	0.01\\
40.57	0.01\\
40.58	0.01\\
40.59	0.01\\
40.6	0.01\\
40.61	0.01\\
40.62	0.01\\
40.63	0.01\\
40.64	0.01\\
40.65	0.01\\
40.66	0.01\\
40.67	0.01\\
40.68	0.01\\
40.69	0.01\\
40.7	0.01\\
40.71	0.01\\
40.72	0.01\\
40.73	0.01\\
40.74	0.01\\
40.75	0.01\\
40.76	0.01\\
40.77	0.01\\
40.78	0.01\\
40.79	0.01\\
40.8	0.01\\
40.81	0.01\\
40.82	0.01\\
40.83	0.01\\
40.84	0.01\\
40.85	0.01\\
40.86	0.01\\
40.87	0.01\\
40.88	0.01\\
40.89	0.01\\
40.9	0.01\\
40.91	0.01\\
40.92	0.01\\
40.93	0.01\\
40.94	0.01\\
40.95	0.01\\
40.96	0.01\\
40.97	0.01\\
40.98	0.01\\
40.99	0.01\\
41	0.01\\
41.01	0.01\\
41.02	0.01\\
41.03	0.01\\
41.04	0.01\\
41.05	0.01\\
41.06	0.01\\
41.07	0.01\\
41.08	0.01\\
41.09	0.01\\
41.1	0.01\\
41.11	0.01\\
41.12	0.01\\
41.13	0.01\\
41.14	0.01\\
41.15	0.01\\
41.16	0.01\\
41.17	0.01\\
41.18	0.01\\
41.19	0.01\\
41.2	0.01\\
41.21	0.01\\
41.22	0.01\\
41.23	0.01\\
41.24	0.01\\
41.25	0.01\\
41.26	0.01\\
41.27	0.01\\
41.28	0.01\\
41.29	0.01\\
41.3	0.01\\
41.31	0.01\\
41.32	0.01\\
41.33	0.01\\
41.34	0.01\\
41.35	0.01\\
41.36	0.01\\
41.37	0.01\\
41.38	0.01\\
41.39	0.01\\
41.4	0.01\\
41.41	0.01\\
41.42	0.01\\
41.43	0.01\\
41.44	0.01\\
41.45	0.01\\
41.46	0.01\\
41.47	0.01\\
41.48	0.01\\
41.49	0.01\\
41.5	0.01\\
41.51	0.01\\
41.52	0.01\\
41.53	0.01\\
41.54	0.01\\
41.55	0.01\\
41.56	0.01\\
41.57	0.01\\
41.58	0.01\\
41.59	0.01\\
41.6	0.01\\
41.61	0.01\\
41.62	0.01\\
41.63	0.01\\
41.64	0.01\\
41.65	0.01\\
41.66	0.01\\
41.67	0.01\\
41.68	0.01\\
41.69	0.01\\
41.7	0.01\\
41.71	0.01\\
41.72	0.01\\
41.73	0.01\\
41.74	0.01\\
41.75	0.01\\
41.76	0.01\\
41.77	0.01\\
41.78	0.01\\
41.79	0.01\\
41.8	0.01\\
41.81	0.01\\
41.82	0.01\\
41.83	0.01\\
41.84	0.01\\
41.85	0.01\\
41.86	0.01\\
41.87	0.01\\
41.88	0.01\\
41.89	0.01\\
41.9	0.01\\
41.91	0.01\\
41.92	0.01\\
41.93	0.01\\
41.94	0.01\\
41.95	0.01\\
41.96	0.01\\
41.97	0.01\\
41.98	0.01\\
41.99	0.01\\
42	0.01\\
42.01	0.01\\
42.02	0.01\\
42.03	0.01\\
42.04	0.01\\
42.05	0.01\\
42.06	0.01\\
42.07	0.01\\
42.08	0.01\\
42.09	0.01\\
42.1	0.01\\
42.11	0.01\\
42.12	0.01\\
42.13	0.01\\
42.14	0.01\\
42.15	0.01\\
42.16	0.01\\
42.17	0.01\\
42.18	0.01\\
42.19	0.01\\
42.2	0.01\\
42.21	0.01\\
42.22	0.01\\
42.23	0.01\\
42.24	0.01\\
42.25	0.01\\
42.26	0.01\\
42.27	0.01\\
42.28	0.01\\
42.29	0.01\\
42.3	0.01\\
42.31	0.01\\
42.32	0.01\\
42.33	0.01\\
42.34	0.01\\
42.35	0.01\\
42.36	0.01\\
42.37	0.01\\
42.38	0.01\\
42.39	0.01\\
42.4	0.01\\
42.41	0.01\\
42.42	0.01\\
42.43	0.01\\
42.44	0.01\\
42.45	0.01\\
42.46	0.01\\
42.47	0.01\\
42.48	0.01\\
42.49	0.01\\
42.5	0.01\\
42.51	0.01\\
42.52	0.01\\
42.53	0.01\\
42.54	0.01\\
42.55	0.01\\
42.56	0.01\\
42.57	0.01\\
42.58	0.01\\
42.59	0.01\\
42.6	0.01\\
42.61	0.01\\
42.62	0.01\\
42.63	0.01\\
42.64	0.01\\
42.65	0.01\\
42.66	0.01\\
42.67	0.01\\
42.68	0.01\\
42.69	0.01\\
42.7	0.01\\
42.71	0.01\\
42.72	0.01\\
42.73	0.01\\
42.74	0.01\\
42.75	0.01\\
42.76	0.01\\
42.77	0.01\\
42.78	0.01\\
42.79	0.01\\
42.8	0.01\\
42.81	0.01\\
42.82	0.01\\
42.83	0.01\\
42.84	0.01\\
42.85	0.01\\
42.86	0.01\\
42.87	0.01\\
42.88	0.01\\
42.89	0.01\\
42.9	0.01\\
42.91	0.01\\
42.92	0.01\\
42.93	0.01\\
42.94	0.01\\
42.95	0.01\\
42.96	0.01\\
42.97	0.01\\
42.98	0.01\\
42.99	0.01\\
43	0.01\\
43.01	0.01\\
43.02	0.01\\
43.03	0.01\\
43.04	0.01\\
43.05	0.01\\
43.06	0.01\\
43.07	0.01\\
43.08	0.01\\
43.09	0.01\\
43.1	0.01\\
43.11	0.01\\
43.12	0.01\\
43.13	0.01\\
43.14	0.01\\
43.15	0.01\\
43.16	0.01\\
43.17	0.01\\
43.18	0.01\\
43.19	0.01\\
43.2	0.01\\
43.21	0.01\\
43.22	0.01\\
43.23	0.01\\
43.24	0.01\\
43.25	0.01\\
43.26	0.01\\
43.27	0.01\\
43.28	0.01\\
43.29	0.01\\
43.3	0.01\\
43.31	0.01\\
43.32	0.01\\
43.33	0.01\\
43.34	0.01\\
43.35	0.01\\
43.36	0.01\\
43.37	0.01\\
43.38	0.01\\
43.39	0.01\\
43.4	0.01\\
43.41	0.01\\
43.42	0.01\\
43.43	0.01\\
43.44	0.01\\
43.45	0.01\\
43.46	0.01\\
43.47	0.01\\
43.48	0.01\\
43.49	0.01\\
43.5	0.01\\
43.51	0.01\\
43.52	0.01\\
43.53	0.01\\
43.54	0.01\\
43.55	0.01\\
43.56	0.01\\
43.57	0.01\\
43.58	0.01\\
43.59	0.01\\
43.6	0.01\\
43.61	0.01\\
43.62	0.01\\
43.63	0.01\\
43.64	0.01\\
43.65	0.01\\
43.66	0.01\\
43.67	0.01\\
43.68	0.01\\
43.69	0.01\\
43.7	0.01\\
43.71	0.01\\
43.72	0.01\\
43.73	0.01\\
43.74	0.01\\
43.75	0.01\\
43.76	0.01\\
43.77	0.01\\
43.78	0.01\\
43.79	0.01\\
43.8	0.01\\
43.81	0.01\\
43.82	0.01\\
43.83	0.01\\
43.84	0.01\\
43.85	0.01\\
43.86	0.01\\
43.87	0.01\\
43.88	0.01\\
43.89	0.01\\
43.9	0.01\\
43.91	0.01\\
43.92	0.01\\
43.93	0.01\\
43.94	0.01\\
43.95	0.01\\
43.96	0.01\\
43.97	0.01\\
43.98	0.01\\
43.99	0.01\\
44	0.01\\
44.01	0.01\\
44.02	0.01\\
44.03	0.01\\
44.04	0.01\\
44.05	0.01\\
44.06	0.01\\
44.07	0.01\\
44.08	0.01\\
44.09	0.01\\
44.1	0.01\\
44.11	0.01\\
44.12	0.01\\
44.13	0.01\\
44.14	0.01\\
44.15	0.01\\
44.16	0.01\\
44.17	0.01\\
44.18	0.01\\
44.19	0.01\\
44.2	0.01\\
44.21	0.01\\
44.22	0.01\\
44.23	0.01\\
44.24	0.01\\
44.25	0.01\\
44.26	0.01\\
44.27	0.01\\
44.28	0.01\\
44.29	0.01\\
44.3	0.01\\
44.31	0.01\\
44.32	0.01\\
44.33	0.01\\
44.34	0.01\\
44.35	0.01\\
44.36	0.01\\
44.37	0.01\\
44.38	0.01\\
44.39	0.01\\
44.4	0.01\\
44.41	0.01\\
44.42	0.01\\
44.43	0.01\\
44.44	0.01\\
44.45	0.01\\
44.46	0.01\\
44.47	0.01\\
44.48	0.01\\
44.49	0.01\\
44.5	0.01\\
44.51	0.01\\
44.52	0.01\\
44.53	0.01\\
44.54	0.01\\
44.55	0.01\\
44.56	0.01\\
44.57	0.01\\
44.58	0.01\\
44.59	0.01\\
44.6	0.01\\
44.61	0.01\\
44.62	0.01\\
44.63	0.01\\
44.64	0.01\\
44.65	0.01\\
44.66	0.01\\
44.67	0.01\\
44.68	0.01\\
44.69	0.01\\
44.7	0.01\\
44.71	0.01\\
44.72	0.01\\
44.73	0.01\\
44.74	0.01\\
44.75	0.01\\
44.76	0.01\\
44.77	0.01\\
44.78	0.01\\
44.79	0.01\\
44.8	0.01\\
44.81	0.01\\
44.82	0.01\\
44.83	0.01\\
44.84	0.01\\
44.85	0.01\\
44.86	0.01\\
44.87	0.01\\
44.88	0.01\\
44.89	0.01\\
44.9	0.01\\
44.91	0.01\\
44.92	0.01\\
44.93	0.01\\
44.94	0.01\\
44.95	0.01\\
44.96	0.01\\
44.97	0.01\\
44.98	0.01\\
44.99	0.01\\
45	0.01\\
45.01	0.01\\
45.02	0.01\\
45.03	0.01\\
45.04	0.01\\
45.05	0.01\\
45.06	0.01\\
45.07	0.01\\
45.08	0.01\\
45.09	0.01\\
45.1	0.01\\
45.11	0.01\\
45.12	0.01\\
45.13	0.01\\
45.14	0.01\\
45.15	0.01\\
45.16	0.01\\
45.17	0.01\\
45.18	0.01\\
45.19	0.01\\
45.2	0.01\\
45.21	0.01\\
45.22	0.01\\
45.23	0.01\\
45.24	0.01\\
45.25	0.01\\
45.26	0.01\\
45.27	0.01\\
45.28	0.01\\
45.29	0.01\\
45.3	0.01\\
45.31	0.01\\
45.32	0.01\\
45.33	0.01\\
45.34	0.01\\
45.35	0.01\\
45.36	0.01\\
45.37	0.01\\
45.38	0.01\\
45.39	0.01\\
45.4	0.01\\
45.41	0.01\\
45.42	0.01\\
45.43	0.01\\
45.44	0.01\\
45.45	0.01\\
45.46	0.01\\
45.47	0.01\\
45.48	0.01\\
45.49	0.01\\
45.5	0.01\\
45.51	0.01\\
45.52	0.01\\
45.53	0.01\\
45.54	0.01\\
45.55	0.01\\
45.56	0.01\\
45.57	0.01\\
45.58	0.01\\
45.59	0.01\\
45.6	0.01\\
45.61	0.01\\
45.62	0.01\\
45.63	0.01\\
45.64	0.01\\
45.65	0.01\\
45.66	0.01\\
45.67	0.01\\
45.68	0.01\\
45.69	0.01\\
45.7	0.01\\
45.71	0.01\\
45.72	0.01\\
45.73	0.01\\
45.74	0.01\\
45.75	0.01\\
45.76	0.01\\
45.77	0.01\\
45.78	0.01\\
45.79	0.01\\
45.8	0.01\\
45.81	0.01\\
45.82	0.01\\
45.83	0.01\\
45.84	0.01\\
45.85	0.01\\
45.86	0.01\\
45.87	0.01\\
45.88	0.01\\
45.89	0.01\\
45.9	0.01\\
45.91	0.01\\
45.92	0.01\\
45.93	0.01\\
45.94	0.01\\
45.95	0.01\\
45.96	0.01\\
45.97	0.01\\
45.98	0.01\\
45.99	0.01\\
46	0.01\\
46.01	0.01\\
46.02	0.01\\
46.03	0.01\\
46.04	0.01\\
46.05	0.01\\
46.06	0.01\\
46.07	0.01\\
46.08	0.01\\
46.09	0.01\\
46.1	0.01\\
46.11	0.01\\
46.12	0.01\\
46.13	0.01\\
46.14	0.01\\
46.15	0.01\\
46.16	0.01\\
46.17	0.01\\
46.18	0.01\\
46.19	0.01\\
46.2	0.01\\
46.21	0.01\\
46.22	0.01\\
46.23	0.01\\
46.24	0.01\\
46.25	0.01\\
46.26	0.01\\
46.27	0.01\\
46.28	0.01\\
46.29	0.01\\
46.3	0.01\\
46.31	0.01\\
46.32	0.01\\
46.33	0.01\\
46.34	0.01\\
46.35	0.01\\
46.36	0.01\\
46.37	0.01\\
46.38	0.01\\
46.39	0.01\\
46.4	0.01\\
46.41	0.01\\
46.42	0.01\\
46.43	0.01\\
46.44	0.01\\
46.45	0.01\\
46.46	0.01\\
46.47	0.01\\
46.48	0.01\\
46.49	0.01\\
46.5	0.01\\
46.51	0.01\\
46.52	0.01\\
46.53	0.01\\
46.54	0.01\\
46.55	0.01\\
46.56	0.01\\
46.57	0.01\\
46.58	0.01\\
46.59	0.01\\
46.6	0.01\\
46.61	0.01\\
46.62	0.01\\
46.63	0.01\\
46.64	0.01\\
46.65	0.01\\
46.66	0.01\\
46.67	0.01\\
46.68	0.01\\
46.69	0.01\\
46.7	0.01\\
46.71	0.01\\
46.72	0.01\\
46.73	0.01\\
46.74	0.01\\
46.75	0.01\\
46.76	0.01\\
46.77	0.01\\
46.78	0.01\\
46.79	0.01\\
46.8	0.01\\
46.81	0.01\\
46.82	0.01\\
46.83	0.01\\
46.84	0.01\\
46.85	0.01\\
46.86	0.01\\
46.87	0.01\\
46.88	0.01\\
46.89	0.01\\
46.9	0.01\\
46.91	0.01\\
46.92	0.01\\
46.93	0.01\\
46.94	0.01\\
46.95	0.01\\
46.96	0.01\\
46.97	0.01\\
46.98	0.01\\
46.99	0.01\\
47	0.01\\
47.01	0.01\\
47.02	0.01\\
47.03	0.01\\
47.04	0.01\\
47.05	0.01\\
47.06	0.01\\
47.07	0.01\\
47.08	0.01\\
47.09	0.01\\
47.1	0.01\\
47.11	0.01\\
47.12	0.01\\
47.13	0.01\\
47.14	0.01\\
47.15	0.01\\
47.16	0.01\\
47.17	0.01\\
47.18	0.01\\
47.19	0.01\\
47.2	0.01\\
47.21	0.01\\
47.22	0.01\\
47.23	0.01\\
47.24	0.01\\
47.25	0.01\\
47.26	0.01\\
47.27	0.01\\
47.28	0.01\\
47.29	0.01\\
47.3	0.01\\
47.31	0.01\\
47.32	0.01\\
47.33	0.01\\
47.34	0.01\\
47.35	0.01\\
47.36	0.01\\
47.37	0.01\\
47.38	0.01\\
47.39	0.01\\
47.4	0.01\\
47.41	0.01\\
47.42	0.01\\
47.43	0.01\\
47.44	0.01\\
47.45	0.01\\
47.46	0.01\\
47.47	0.01\\
47.48	0.01\\
47.49	0.01\\
47.5	0.01\\
47.51	0.01\\
47.52	0.01\\
47.53	0.01\\
47.54	0.01\\
47.55	0.01\\
47.56	0.01\\
47.57	0.01\\
47.58	0.01\\
47.59	0.01\\
47.6	0.01\\
47.61	0.01\\
47.62	0.01\\
47.63	0.01\\
47.64	0.01\\
47.65	0.01\\
47.66	0.01\\
47.67	0.01\\
47.68	0.01\\
47.69	0.01\\
47.7	0.01\\
47.71	0.01\\
47.72	0.01\\
47.73	0.01\\
47.74	0.01\\
47.75	0.01\\
47.76	0.01\\
47.77	0.01\\
47.78	0.01\\
47.79	0.01\\
47.8	0.01\\
47.81	0.01\\
47.82	0.01\\
47.83	0.01\\
47.84	0.01\\
47.85	0.01\\
47.86	0.01\\
47.87	0.01\\
47.88	0.01\\
47.89	0.01\\
47.9	0.01\\
47.91	0.01\\
47.92	0.01\\
47.93	0.01\\
47.94	0.01\\
47.95	0.01\\
47.96	0.01\\
47.97	0.01\\
47.98	0.01\\
47.99	0.01\\
48	0.01\\
48.01	0.01\\
48.02	0.01\\
48.03	0.01\\
48.04	0.01\\
48.05	0.01\\
48.06	0.01\\
48.07	0.01\\
48.08	0.01\\
48.09	0.01\\
48.1	0.01\\
48.11	0.01\\
48.12	0.01\\
48.13	0.01\\
48.14	0.01\\
48.15	0.01\\
48.16	0.01\\
48.17	0.01\\
48.18	0.01\\
48.19	0.01\\
48.2	0.01\\
48.21	0.01\\
48.22	0.01\\
48.23	0.01\\
48.24	0.01\\
48.25	0.01\\
48.26	0.01\\
48.27	0.01\\
48.28	0.01\\
48.29	0.01\\
48.3	0.01\\
48.31	0.01\\
48.32	0.01\\
48.33	0.01\\
48.34	0.01\\
48.35	0.01\\
48.36	0.01\\
48.37	0.01\\
48.38	0.01\\
48.39	0.01\\
48.4	0.01\\
48.41	0.01\\
48.42	0.01\\
48.43	0.01\\
48.44	0.01\\
48.45	0.01\\
48.46	0.01\\
48.47	0.01\\
48.48	0.01\\
48.49	0.01\\
48.5	0.01\\
48.51	0.01\\
48.52	0.01\\
48.53	0.01\\
48.54	0.01\\
48.55	0.01\\
48.56	0.01\\
48.57	0.01\\
48.58	0.01\\
48.59	0.01\\
48.6	0.01\\
48.61	0.01\\
48.62	0.01\\
48.63	0.01\\
48.64	0.01\\
48.65	0.01\\
48.66	0.01\\
48.67	0.01\\
48.68	0.01\\
48.69	0.01\\
48.7	0.01\\
48.71	0.01\\
48.72	0.01\\
48.73	0.01\\
48.74	0.01\\
48.75	0.01\\
48.76	0.01\\
48.77	0.01\\
48.78	0.01\\
48.79	0.01\\
48.8	0.01\\
48.81	0.01\\
48.82	0.01\\
48.83	0.01\\
48.84	0.01\\
48.85	0.01\\
48.86	0.01\\
48.87	0.01\\
48.88	0.01\\
48.89	0.01\\
48.9	0.01\\
48.91	0.01\\
48.92	0.01\\
48.93	0.01\\
48.94	0.01\\
48.95	0.01\\
48.96	0.01\\
48.97	0.01\\
48.98	0.01\\
48.99	0.01\\
49	0.01\\
49.01	0.01\\
49.02	0.01\\
49.03	0.01\\
49.04	0.01\\
49.05	0.01\\
49.06	0.01\\
49.07	0.01\\
49.08	0.01\\
49.09	0.01\\
49.1	0.01\\
49.11	0.01\\
49.12	0.01\\
49.13	0.01\\
49.14	0.01\\
49.15	0.01\\
49.16	0.01\\
49.17	0.01\\
49.18	0.01\\
49.19	0.01\\
49.2	0.01\\
49.21	0.01\\
49.22	0.01\\
49.23	0.01\\
49.24	0.01\\
49.25	0.01\\
49.26	0.01\\
49.27	0.01\\
49.28	0.01\\
49.29	0.01\\
49.3	0.01\\
49.31	0.01\\
49.32	0.01\\
49.33	0.01\\
49.34	0.01\\
49.35	0.01\\
49.36	0.01\\
49.37	0.01\\
49.38	0.01\\
49.39	0.01\\
49.4	0.01\\
49.41	0.01\\
49.42	0.01\\
49.43	0.01\\
49.44	0.01\\
49.45	0.01\\
49.46	0.01\\
49.47	0.01\\
49.48	0.01\\
49.49	0.01\\
49.5	0.01\\
49.51	0.01\\
49.52	0.01\\
49.53	0.01\\
49.54	0.01\\
49.55	0.01\\
49.56	0.01\\
49.57	0.01\\
49.58	0.01\\
49.59	0.01\\
49.6	0.01\\
49.61	0.01\\
49.62	0.01\\
49.63	0.01\\
49.64	0.01\\
49.65	0.01\\
49.66	0.01\\
49.67	0.01\\
49.68	0.01\\
49.69	0.01\\
49.7	0.01\\
49.71	0.01\\
49.72	0.01\\
49.73	0.01\\
49.74	0.01\\
49.75	0.01\\
49.76	0.01\\
49.77	0.01\\
49.78	0.01\\
49.79	0.01\\
49.8	0.01\\
49.81	0.01\\
49.82	0.01\\
49.83	0.01\\
49.84	0.01\\
49.85	0.01\\
49.86	0.01\\
49.87	0.01\\
49.88	0.01\\
49.89	0.01\\
49.9	0.01\\
49.91	0.01\\
49.92	0.01\\
49.93	0.01\\
49.94	0.01\\
49.95	0.01\\
49.96	0.01\\
49.97	0.01\\
49.98	0.01\\
49.99	0.01\\
50	0.01\\
50.01	0.01\\
50.02	0.01\\
50.03	0.01\\
50.04	0.01\\
50.05	0.01\\
50.06	0.01\\
50.07	0.01\\
50.08	0.01\\
50.09	0.01\\
50.1	0.01\\
50.11	0.01\\
50.12	0.01\\
50.13	0.01\\
50.14	0.01\\
50.15	0.01\\
50.16	0.01\\
50.17	0.01\\
50.18	0.01\\
50.19	0.01\\
50.2	0.01\\
50.21	0.01\\
50.22	0.01\\
50.23	0.01\\
50.24	0.01\\
50.25	0.01\\
50.26	0.01\\
50.27	0.01\\
50.28	0.01\\
50.29	0.01\\
50.3	0.01\\
50.31	0.01\\
50.32	0.01\\
50.33	0.01\\
50.34	0.01\\
50.35	0.01\\
50.36	0.01\\
50.37	0.01\\
50.38	0.01\\
50.39	0.01\\
50.4	0.01\\
50.41	0.01\\
50.42	0.01\\
50.43	0.01\\
50.44	0.01\\
50.45	0.01\\
50.46	0.01\\
50.47	0.01\\
50.48	0.01\\
50.49	0.01\\
50.5	0.01\\
50.51	0.01\\
50.52	0.01\\
50.53	0.01\\
50.54	0.01\\
50.55	0.01\\
50.56	0.01\\
50.57	0.01\\
50.58	0.01\\
50.59	0.01\\
50.6	0.01\\
50.61	0.01\\
50.62	0.01\\
50.63	0.01\\
50.64	0.01\\
50.65	0.01\\
50.66	0.01\\
50.67	0.01\\
50.68	0.01\\
50.69	0.01\\
50.7	0.01\\
50.71	0.01\\
50.72	0.01\\
50.73	0.01\\
50.74	0.01\\
50.75	0.01\\
50.76	0.01\\
50.77	0.01\\
50.78	0.01\\
50.79	0.01\\
50.8	0.01\\
50.81	0.01\\
50.82	0.01\\
50.83	0.01\\
50.84	0.01\\
50.85	0.01\\
50.86	0.01\\
50.87	0.01\\
50.88	0.01\\
50.89	0.01\\
50.9	0.01\\
50.91	0.01\\
50.92	0.01\\
50.93	0.01\\
50.94	0.01\\
50.95	0.01\\
50.96	0.01\\
50.97	0.01\\
50.98	0.01\\
50.99	0.01\\
51	0.01\\
51.01	0.01\\
51.02	0.01\\
51.03	0.01\\
51.04	0.01\\
51.05	0.01\\
51.06	0.01\\
51.07	0.01\\
51.08	0.01\\
51.09	0.01\\
51.1	0.01\\
51.11	0.01\\
51.12	0.01\\
51.13	0.01\\
51.14	0.01\\
51.15	0.01\\
51.16	0.01\\
51.17	0.01\\
51.18	0.01\\
51.19	0.01\\
51.2	0.01\\
51.21	0.01\\
51.22	0.01\\
51.23	0.01\\
51.24	0.01\\
51.25	0.01\\
51.26	0.01\\
51.27	0.01\\
51.28	0.01\\
51.29	0.01\\
51.3	0.01\\
51.31	0.01\\
51.32	0.01\\
51.33	0.01\\
51.34	0.01\\
51.35	0.01\\
51.36	0.01\\
51.37	0.01\\
51.38	0.01\\
51.39	0.01\\
51.4	0.01\\
51.41	0.01\\
51.42	0.01\\
51.43	0.01\\
51.44	0.01\\
51.45	0.01\\
51.46	0.01\\
51.47	0.01\\
51.48	0.01\\
51.49	0.01\\
51.5	0.01\\
51.51	0.01\\
51.52	0.01\\
51.53	0.01\\
51.54	0.01\\
51.55	0.01\\
51.56	0.01\\
51.57	0.01\\
51.58	0.01\\
51.59	0.01\\
51.6	0.01\\
51.61	0.01\\
51.62	0.01\\
51.63	0.01\\
51.64	0.01\\
51.65	0.01\\
51.66	0.01\\
51.67	0.01\\
51.68	0.01\\
51.69	0.01\\
51.7	0.01\\
51.71	0.01\\
51.72	0.01\\
51.73	0.01\\
51.74	0.01\\
51.75	0.01\\
51.76	0.01\\
51.77	0.01\\
51.78	0.01\\
51.79	0.01\\
51.8	0.01\\
51.81	0.01\\
51.82	0.01\\
51.83	0.01\\
51.84	0.01\\
51.85	0.01\\
51.86	0.01\\
51.87	0.01\\
51.88	0.01\\
51.89	0.01\\
51.9	0.01\\
51.91	0.01\\
51.92	0.01\\
51.93	0.01\\
51.94	0.01\\
51.95	0.01\\
51.96	0.01\\
51.97	0.01\\
51.98	0.01\\
51.99	0.01\\
52	0.01\\
52.01	0.01\\
52.02	0.01\\
52.03	0.01\\
52.04	0.01\\
52.05	0.01\\
52.06	0.01\\
52.07	0.01\\
52.08	0.01\\
52.09	0.01\\
52.1	0.01\\
52.11	0.01\\
52.12	0.01\\
52.13	0.01\\
52.14	0.01\\
52.15	0.01\\
52.16	0.01\\
52.17	0.01\\
52.18	0.01\\
52.19	0.01\\
52.2	0.01\\
52.21	0.01\\
52.22	0.01\\
52.23	0.01\\
52.24	0.01\\
52.25	0.01\\
52.26	0.01\\
52.27	0.01\\
52.28	0.01\\
52.29	0.01\\
52.3	0.01\\
52.31	0.01\\
52.32	0.01\\
52.33	0.01\\
52.34	0.01\\
52.35	0.01\\
52.36	0.01\\
52.37	0.01\\
52.38	0.01\\
52.39	0.01\\
52.4	0.01\\
52.41	0.01\\
52.42	0.01\\
52.43	0.01\\
52.44	0.01\\
52.45	0.01\\
52.46	0.01\\
52.47	0.01\\
52.48	0.01\\
52.49	0.01\\
52.5	0.01\\
52.51	0.01\\
52.52	0.01\\
52.53	0.01\\
52.54	0.01\\
52.55	0.01\\
52.56	0.01\\
52.57	0.01\\
52.58	0.01\\
52.59	0.01\\
52.6	0.01\\
52.61	0.01\\
52.62	0.01\\
52.63	0.01\\
52.64	0.01\\
52.65	0.01\\
52.66	0.01\\
52.67	0.01\\
52.68	0.01\\
52.69	0.01\\
52.7	0.01\\
52.71	0.01\\
52.72	0.01\\
52.73	0.01\\
52.74	0.01\\
52.75	0.01\\
52.76	0.01\\
52.77	0.01\\
52.78	0.01\\
52.79	0.01\\
52.8	0.01\\
52.81	0.01\\
52.82	0.01\\
52.83	0.01\\
52.84	0.01\\
52.85	0.01\\
52.86	0.01\\
52.87	0.01\\
52.88	0.01\\
52.89	0.01\\
52.9	0.01\\
52.91	0.01\\
52.92	0.01\\
52.93	0.01\\
52.94	0.01\\
52.95	0.01\\
52.96	0.01\\
52.97	0.01\\
52.98	0.01\\
52.99	0.01\\
53	0.01\\
53.01	0.01\\
53.02	0.01\\
53.03	0.01\\
53.04	0.01\\
53.05	0.01\\
53.06	0.01\\
53.07	0.01\\
53.08	0.01\\
53.09	0.01\\
53.1	0.01\\
53.11	0.01\\
53.12	0.01\\
53.13	0.01\\
53.14	0.01\\
53.15	0.01\\
53.16	0.01\\
53.17	0.01\\
53.18	0.01\\
53.19	0.01\\
53.2	0.01\\
53.21	0.01\\
53.22	0.01\\
53.23	0.01\\
53.24	0.01\\
53.25	0.01\\
53.26	0.01\\
53.27	0.01\\
53.28	0.01\\
53.29	0.01\\
53.3	0.01\\
53.31	0.01\\
53.32	0.01\\
53.33	0.01\\
53.34	0.01\\
53.35	0.01\\
53.36	0.01\\
53.37	0.01\\
53.38	0.01\\
53.39	0.01\\
53.4	0.01\\
53.41	0.01\\
53.42	0.01\\
53.43	0.01\\
53.44	0.01\\
53.45	0.01\\
53.46	0.01\\
53.47	0.01\\
53.48	0.01\\
53.49	0.01\\
53.5	0.01\\
53.51	0.01\\
53.52	0.01\\
53.53	0.01\\
53.54	0.01\\
53.55	0.01\\
53.56	0.01\\
53.57	0.01\\
53.58	0.01\\
53.59	0.01\\
53.6	0.01\\
53.61	0.01\\
53.62	0.01\\
53.63	0.01\\
53.64	0.01\\
53.65	0.01\\
53.66	0.01\\
53.67	0.01\\
53.68	0.01\\
53.69	0.01\\
53.7	0.01\\
53.71	0.01\\
53.72	0.01\\
53.73	0.01\\
53.74	0.01\\
53.75	0.01\\
53.76	0.01\\
53.77	0.01\\
53.78	0.01\\
53.79	0.01\\
53.8	0.01\\
53.81	0.01\\
53.82	0.01\\
53.83	0.01\\
53.84	0.01\\
53.85	0.01\\
53.86	0.01\\
53.87	0.01\\
53.88	0.01\\
53.89	0.01\\
53.9	0.01\\
53.91	0.01\\
53.92	0.01\\
53.93	0.01\\
53.94	0.01\\
53.95	0.01\\
53.96	0.01\\
53.97	0.01\\
53.98	0.01\\
53.99	0.01\\
54	0.01\\
54.01	0.01\\
54.02	0.01\\
54.03	0.01\\
54.04	0.01\\
54.05	0.01\\
54.06	0.01\\
54.07	0.01\\
54.08	0.01\\
54.09	0.01\\
54.1	0.01\\
54.11	0.01\\
54.12	0.01\\
54.13	0.01\\
54.14	0.01\\
54.15	0.01\\
54.16	0.01\\
54.17	0.01\\
54.18	0.01\\
54.19	0.01\\
54.2	0.01\\
54.21	0.01\\
54.22	0.01\\
54.23	0.01\\
54.24	0.01\\
54.25	0.01\\
54.26	0.01\\
54.27	0.01\\
54.28	0.01\\
54.29	0.01\\
54.3	0.01\\
54.31	0.01\\
54.32	0.01\\
54.33	0.01\\
54.34	0.01\\
54.35	0.01\\
54.36	0.01\\
54.37	0.01\\
54.38	0.01\\
54.39	0.01\\
54.4	0.01\\
54.41	0.01\\
54.42	0.01\\
54.43	0.01\\
54.44	0.01\\
54.45	0.01\\
54.46	0.01\\
54.47	0.01\\
54.48	0.01\\
54.49	0.01\\
54.5	0.01\\
54.51	0.01\\
54.52	0.01\\
54.53	0.01\\
54.54	0.01\\
54.55	0.01\\
54.56	0.01\\
54.57	0.01\\
54.58	0.01\\
54.59	0.01\\
54.6	0.01\\
54.61	0.01\\
54.62	0.01\\
54.63	0.01\\
54.64	0.01\\
54.65	0.01\\
54.66	0.01\\
54.67	0.01\\
54.68	0.01\\
54.69	0.01\\
54.7	0.01\\
54.71	0.01\\
54.72	0.01\\
54.73	0.01\\
54.74	0.01\\
54.75	0.01\\
54.76	0.01\\
54.77	0.01\\
54.78	0.01\\
54.79	0.01\\
54.8	0.01\\
54.81	0.01\\
54.82	0.01\\
54.83	0.01\\
54.84	0.01\\
54.85	0.01\\
54.86	0.01\\
54.87	0.01\\
54.88	0.01\\
54.89	0.01\\
54.9	0.01\\
54.91	0.01\\
54.92	0.01\\
54.93	0.01\\
54.94	0.01\\
54.95	0.01\\
54.96	0.01\\
54.97	0.01\\
54.98	0.01\\
54.99	0.01\\
55	0.01\\
55.01	0.01\\
55.02	0.01\\
55.03	0.01\\
55.04	0.01\\
55.05	0.01\\
55.06	0.01\\
55.07	0.01\\
55.08	0.01\\
55.09	0.01\\
55.1	0.01\\
55.11	0.01\\
55.12	0.01\\
55.13	0.01\\
55.14	0.01\\
55.15	0.01\\
55.16	0.01\\
55.17	0.01\\
55.18	0.01\\
55.19	0.01\\
55.2	0.01\\
55.21	0.01\\
55.22	0.01\\
55.23	0.01\\
55.24	0.01\\
55.25	0.01\\
55.26	0.01\\
55.27	0.01\\
55.28	0.01\\
55.29	0.01\\
55.3	0.01\\
55.31	0.01\\
55.32	0.01\\
55.33	0.01\\
55.34	0.01\\
55.35	0.01\\
55.36	0.01\\
55.37	0.01\\
55.38	0.01\\
55.39	0.01\\
55.4	0.01\\
55.41	0.01\\
55.42	0.01\\
55.43	0.01\\
55.44	0.01\\
55.45	0.01\\
55.46	0.01\\
55.47	0.01\\
55.48	0.01\\
55.49	0.01\\
55.5	0.01\\
55.51	0.01\\
55.52	0.01\\
55.53	0.01\\
55.54	0.01\\
55.55	0.01\\
55.56	0.01\\
55.57	0.01\\
55.58	0.01\\
55.59	0.01\\
55.6	0.01\\
55.61	0.01\\
55.62	0.01\\
55.63	0.01\\
55.64	0.01\\
55.65	0.01\\
55.66	0.01\\
55.67	0.01\\
55.68	0.01\\
55.69	0.01\\
55.7	0.01\\
55.71	0.01\\
55.72	0.01\\
55.73	0.01\\
55.74	0.01\\
55.75	0.01\\
55.76	0.01\\
55.77	0.01\\
55.78	0.01\\
55.79	0.01\\
55.8	0.01\\
55.81	0.01\\
55.82	0.01\\
55.83	0.01\\
55.84	0.01\\
55.85	0.01\\
55.86	0.01\\
55.87	0.01\\
55.88	0.01\\
55.89	0.01\\
55.9	0.01\\
55.91	0.01\\
55.92	0.01\\
55.93	0.01\\
55.94	0.01\\
55.95	0.01\\
55.96	0.01\\
55.97	0.01\\
55.98	0.01\\
55.99	0.01\\
56	0.01\\
56.01	0.01\\
56.02	0.01\\
56.03	0.01\\
56.04	0.01\\
56.05	0.01\\
56.06	0.01\\
56.07	0.01\\
56.08	0.01\\
56.09	0.01\\
56.1	0.01\\
56.11	0.01\\
56.12	0.01\\
56.13	0.01\\
56.14	0.01\\
56.15	0.01\\
56.16	0.01\\
56.17	0.01\\
56.18	0.01\\
56.19	0.01\\
56.2	0.01\\
56.21	0.01\\
56.22	0.01\\
56.23	0.01\\
56.24	0.01\\
56.25	0.01\\
56.26	0.01\\
56.27	0.01\\
56.28	0.01\\
56.29	0.01\\
56.3	0.01\\
56.31	0.01\\
56.32	0.01\\
56.33	0.01\\
56.34	0.01\\
56.35	0.01\\
56.36	0.01\\
56.37	0.01\\
56.38	0.01\\
56.39	0.01\\
56.4	0.01\\
56.41	0.01\\
56.42	0.01\\
56.43	0.01\\
56.44	0.01\\
56.45	0.01\\
56.46	0.01\\
56.47	0.01\\
56.48	0.01\\
56.49	0.01\\
56.5	0.01\\
56.51	0.01\\
56.52	0.01\\
56.53	0.01\\
56.54	0.01\\
56.55	0.01\\
56.56	0.01\\
56.57	0.01\\
56.58	0.01\\
56.59	0.01\\
56.6	0.01\\
56.61	0.01\\
56.62	0.01\\
56.63	0.01\\
56.64	0.01\\
56.65	0.01\\
56.66	0.01\\
56.67	0.01\\
56.68	0.01\\
56.69	0.01\\
56.7	0.01\\
56.71	0.01\\
56.72	0.01\\
56.73	0.01\\
56.74	0.01\\
56.75	0.01\\
56.76	0.01\\
56.77	0.01\\
56.78	0.01\\
56.79	0.01\\
56.8	0.01\\
56.81	0.01\\
56.82	0.01\\
56.83	0.01\\
56.84	0.01\\
56.85	0.01\\
56.86	0.01\\
56.87	0.01\\
56.88	0.01\\
56.89	0.01\\
56.9	0.01\\
56.91	0.01\\
56.92	0.01\\
56.93	0.01\\
56.94	0.01\\
56.95	0.01\\
56.96	0.01\\
56.97	0.01\\
56.98	0.01\\
56.99	0.01\\
57	0.01\\
57.01	0.01\\
57.02	0.01\\
57.03	0.01\\
57.04	0.01\\
57.05	0.01\\
57.06	0.01\\
57.07	0.01\\
57.08	0.01\\
57.09	0.01\\
57.1	0.01\\
57.11	0.01\\
57.12	0.01\\
57.13	0.01\\
57.14	0.01\\
57.15	0.01\\
57.16	0.01\\
57.17	0.01\\
57.18	0.01\\
57.19	0.01\\
57.2	0.01\\
57.21	0.01\\
57.22	0.01\\
57.23	0.01\\
57.24	0.01\\
57.25	0.01\\
57.26	0.01\\
57.27	0.01\\
57.28	0.01\\
57.29	0.01\\
57.3	0.01\\
57.31	0.01\\
57.32	0.01\\
57.33	0.01\\
57.34	0.01\\
57.35	0.01\\
57.36	0.01\\
57.37	0.01\\
57.38	0.01\\
57.39	0.01\\
57.4	0.01\\
57.41	0.01\\
57.42	0.01\\
57.43	0.01\\
57.44	0.01\\
57.45	0.01\\
57.46	0.01\\
57.47	0.01\\
57.48	0.01\\
57.49	0.01\\
57.5	0.01\\
57.51	0.01\\
57.52	0.01\\
57.53	0.01\\
57.54	0.01\\
57.55	0.01\\
57.56	0.01\\
57.57	0.01\\
57.58	0.01\\
57.59	0.01\\
57.6	0.01\\
57.61	0.01\\
57.62	0.01\\
57.63	0.01\\
57.64	0.01\\
57.65	0.01\\
57.66	0.01\\
57.67	0.01\\
57.68	0.01\\
57.69	0.01\\
57.7	0.01\\
57.71	0.01\\
57.72	0.01\\
57.73	0.01\\
57.74	0.01\\
57.75	0.01\\
57.76	0.01\\
57.77	0.01\\
57.78	0.01\\
57.79	0.01\\
57.8	0.01\\
57.81	0.01\\
57.82	0.01\\
57.83	0.01\\
57.84	0.01\\
57.85	0.01\\
57.86	0.01\\
57.87	0.01\\
57.88	0.01\\
57.89	0.01\\
57.9	0.01\\
57.91	0.01\\
57.92	0.01\\
57.93	0.01\\
57.94	0.01\\
57.95	0.01\\
57.96	0.01\\
57.97	0.01\\
57.98	0.01\\
57.99	0.01\\
58	0.01\\
58.01	0.01\\
58.02	0.01\\
58.03	0.01\\
58.04	0.01\\
58.05	0.01\\
58.06	0.01\\
58.07	0.01\\
58.08	0.01\\
58.09	0.01\\
58.1	0.01\\
58.11	0.01\\
58.12	0.01\\
58.13	0.01\\
58.14	0.01\\
58.15	0.01\\
58.16	0.01\\
58.17	0.01\\
58.18	0.01\\
58.19	0.01\\
58.2	0.01\\
58.21	0.01\\
58.22	0.01\\
58.23	0.01\\
58.24	0.01\\
58.25	0.01\\
58.26	0.01\\
58.27	0.01\\
58.28	0.01\\
58.29	0.01\\
58.3	0.01\\
58.31	0.01\\
58.32	0.01\\
58.33	0.01\\
58.34	0.01\\
58.35	0.01\\
58.36	0.01\\
58.37	0.01\\
58.38	0.01\\
58.39	0.01\\
58.4	0.01\\
58.41	0.01\\
58.42	0.01\\
58.43	0.01\\
58.44	0.01\\
58.45	0.01\\
58.46	0.01\\
58.47	0.01\\
58.48	0.01\\
58.49	0.01\\
58.5	0.01\\
58.51	0.01\\
58.52	0.01\\
58.53	0.01\\
58.54	0.01\\
58.55	0.01\\
58.56	0.01\\
58.57	0.01\\
58.58	0.01\\
58.59	0.01\\
58.6	0.01\\
58.61	0.01\\
58.62	0.01\\
58.63	0.01\\
58.64	0.01\\
58.65	0.01\\
58.66	0.01\\
58.67	0.01\\
58.68	0.01\\
58.69	0.01\\
58.7	0.01\\
58.71	0.01\\
58.72	0.01\\
58.73	0.01\\
58.74	0.01\\
58.75	0.01\\
58.76	0.01\\
58.77	0.01\\
58.78	0.01\\
58.79	0.01\\
58.8	0.01\\
58.81	0.01\\
58.82	0.01\\
58.83	0.01\\
58.84	0.01\\
58.85	0.01\\
58.86	0.01\\
58.87	0.01\\
58.88	0.01\\
58.89	0.01\\
58.9	0.01\\
58.91	0.01\\
58.92	0.01\\
58.93	0.01\\
58.94	0.01\\
58.95	0.01\\
58.96	0.01\\
58.97	0.01\\
58.98	0.01\\
58.99	0.01\\
59	0.01\\
59.01	0.01\\
59.02	0.01\\
59.03	0.01\\
59.04	0.01\\
59.05	0.01\\
59.06	0.01\\
59.07	0.01\\
59.08	0.01\\
59.09	0.01\\
59.1	0.01\\
59.11	0.01\\
59.12	0.01\\
59.13	0.01\\
59.14	0.01\\
59.15	0.01\\
59.16	0.01\\
59.17	0.01\\
59.18	0.01\\
59.19	0.01\\
59.2	0.01\\
59.21	0.01\\
59.22	0.01\\
59.23	0.01\\
59.24	0.01\\
59.25	0.01\\
59.26	0.01\\
59.27	0.01\\
59.28	0.01\\
59.29	0.01\\
59.3	0.01\\
59.31	0.01\\
59.32	0.01\\
59.33	0.01\\
59.34	0.01\\
59.35	0.01\\
59.36	0.01\\
59.37	0.01\\
59.38	0.01\\
59.39	0.01\\
59.4	0.01\\
59.41	0.01\\
59.42	0.01\\
59.43	0.01\\
59.44	0.01\\
59.45	0.01\\
59.46	0.01\\
59.47	0.01\\
59.48	0.01\\
59.49	0.01\\
59.5	0.01\\
59.51	0.01\\
59.52	0.01\\
59.53	0.01\\
59.54	0.01\\
59.55	0.01\\
59.56	0.01\\
59.57	0.01\\
59.58	0.01\\
59.59	0.01\\
59.6	0.01\\
59.61	0.01\\
59.62	0.01\\
59.63	0.01\\
59.64	0.01\\
59.65	0.01\\
59.66	0.01\\
59.67	0.01\\
59.68	0.01\\
59.69	0.01\\
59.7	0.01\\
59.71	0.01\\
59.72	0.01\\
59.73	0.01\\
59.74	0.01\\
59.75	0.01\\
59.76	0.01\\
59.77	0.01\\
59.78	0.01\\
59.79	0.01\\
59.8	0.01\\
59.81	0.01\\
59.82	0.01\\
59.83	0.01\\
59.84	0.01\\
59.85	0.01\\
59.86	0.01\\
59.87	0.01\\
59.88	0.01\\
59.89	0.01\\
59.9	0.01\\
59.91	0.01\\
59.92	0.01\\
59.93	0.01\\
59.94	0.01\\
59.95	0.01\\
59.96	0.01\\
59.97	0.01\\
59.98	0.01\\
59.99	0.01\\
60	0.01\\
60.01	0.01\\
60.02	0.01\\
60.03	0.01\\
60.04	0.01\\
60.05	0.01\\
60.06	0.01\\
60.07	0.01\\
60.08	0.01\\
60.09	0.01\\
60.1	0.01\\
60.11	0.01\\
60.12	0.01\\
60.13	0.01\\
60.14	0.01\\
60.15	0.01\\
60.16	0.01\\
60.17	0.01\\
60.18	0.01\\
60.19	0.01\\
60.2	0.01\\
60.21	0.01\\
60.22	0.01\\
60.23	0.01\\
60.24	0.01\\
60.25	0.01\\
60.26	0.01\\
60.27	0.01\\
60.28	0.01\\
60.29	0.01\\
60.3	0.01\\
60.31	0.01\\
60.32	0.01\\
60.33	0.01\\
60.34	0.01\\
60.35	0.01\\
60.36	0.01\\
60.37	0.01\\
60.38	0.01\\
60.39	0.01\\
60.4	0.01\\
60.41	0.01\\
60.42	0.01\\
60.43	0.01\\
60.44	0.01\\
60.45	0.01\\
60.46	0.01\\
60.47	0.01\\
60.48	0.01\\
60.49	0.01\\
60.5	0.01\\
60.51	0.01\\
60.52	0.01\\
60.53	0.01\\
60.54	0.01\\
60.55	0.01\\
60.56	0.01\\
60.57	0.01\\
60.58	0.01\\
60.59	0.01\\
60.6	0.01\\
60.61	0.01\\
60.62	0.01\\
60.63	0.01\\
60.64	0.01\\
60.65	0.01\\
60.66	0.01\\
60.67	0.01\\
60.68	0.01\\
60.69	0.01\\
60.7	0.01\\
60.71	0.01\\
60.72	0.01\\
60.73	0.01\\
60.74	0.01\\
60.75	0.01\\
60.76	0.01\\
60.77	0.01\\
60.78	0.01\\
60.79	0.01\\
60.8	0.01\\
60.81	0.01\\
60.82	0.01\\
60.83	0.01\\
60.84	0.01\\
60.85	0.01\\
60.86	0.01\\
60.87	0.01\\
60.88	0.01\\
60.89	0.01\\
60.9	0.01\\
60.91	0.01\\
60.92	0.01\\
60.93	0.01\\
60.94	0.01\\
60.95	0.01\\
60.96	0.01\\
60.97	0.01\\
60.98	0.01\\
60.99	0.01\\
61	0.01\\
61.01	0.01\\
61.02	0.01\\
61.03	0.01\\
61.04	0.01\\
61.05	0.01\\
61.06	0.01\\
61.07	0.01\\
61.08	0.01\\
61.09	0.01\\
61.1	0.01\\
61.11	0.01\\
61.12	0.01\\
61.13	0.01\\
61.14	0.01\\
61.15	0.01\\
61.16	0.01\\
61.17	0.01\\
61.18	0.01\\
61.19	0.01\\
61.2	0.01\\
61.21	0.01\\
61.22	0.01\\
61.23	0.01\\
61.24	0.01\\
61.25	0.01\\
61.26	0.01\\
61.27	0.01\\
61.28	0.01\\
61.29	0.01\\
61.3	0.01\\
61.31	0.01\\
61.32	0.01\\
61.33	0.01\\
61.34	0.01\\
61.35	0.01\\
61.36	0.01\\
61.37	0.01\\
61.38	0.01\\
61.39	0.01\\
61.4	0.01\\
61.41	0.01\\
61.42	0.01\\
61.43	0.01\\
61.44	0.01\\
61.45	0.01\\
61.46	0.01\\
61.47	0.01\\
61.48	0.01\\
61.49	0.01\\
61.5	0.01\\
61.51	0.01\\
61.52	0.01\\
61.53	0.01\\
61.54	0.01\\
61.55	0.01\\
61.56	0.01\\
61.57	0.01\\
61.58	0.01\\
61.59	0.01\\
61.6	0.01\\
61.61	0.01\\
61.62	0.01\\
61.63	0.01\\
61.64	0.01\\
61.65	0.01\\
61.66	0.01\\
61.67	0.01\\
61.68	0.01\\
61.69	0.01\\
61.7	0.01\\
61.71	0.01\\
61.72	0.01\\
61.73	0.01\\
61.74	0.01\\
61.75	0.01\\
61.76	0.01\\
61.77	0.01\\
61.78	0.01\\
61.79	0.01\\
61.8	0.01\\
61.81	0.01\\
61.82	0.01\\
61.83	0.01\\
61.84	0.01\\
61.85	0.01\\
61.86	0.01\\
61.87	0.01\\
61.88	0.01\\
61.89	0.01\\
61.9	0.01\\
61.91	0.01\\
61.92	0.01\\
61.93	0.01\\
61.94	0.01\\
61.95	0.01\\
61.96	0.01\\
61.97	0.01\\
61.98	0.01\\
61.99	0.01\\
62	0.01\\
62.01	0.01\\
62.02	0.01\\
62.03	0.01\\
62.04	0.01\\
62.05	0.01\\
62.06	0.01\\
62.07	0.01\\
62.08	0.01\\
62.09	0.01\\
62.1	0.01\\
62.11	0.01\\
62.12	0.01\\
62.13	0.01\\
62.14	0.01\\
62.15	0.01\\
62.16	0.01\\
62.17	0.01\\
62.18	0.01\\
62.19	0.01\\
62.2	0.01\\
62.21	0.01\\
62.22	0.01\\
62.23	0.01\\
62.24	0.01\\
62.25	0.01\\
62.26	0.01\\
62.27	0.01\\
62.28	0.01\\
62.29	0.01\\
62.3	0.01\\
62.31	0.01\\
62.32	0.01\\
62.33	0.01\\
62.34	0.01\\
62.35	0.01\\
62.36	0.01\\
62.37	0.01\\
62.38	0.01\\
62.39	0.01\\
62.4	0.01\\
62.41	0.01\\
62.42	0.01\\
62.43	0.01\\
62.44	0.01\\
62.45	0.01\\
62.46	0.01\\
62.47	0.01\\
62.48	0.01\\
62.49	0.01\\
62.5	0.01\\
62.51	0.01\\
62.52	0.01\\
62.53	0.01\\
62.54	0.01\\
62.55	0.01\\
62.56	0.01\\
62.57	0.01\\
62.58	0.01\\
62.59	0.01\\
62.6	0.01\\
62.61	0.01\\
62.62	0.01\\
62.63	0.01\\
62.64	0.01\\
62.65	0.01\\
62.66	0.01\\
62.67	0.01\\
62.68	0.01\\
62.69	0.01\\
62.7	0.01\\
62.71	0.01\\
62.72	0.01\\
62.73	0.01\\
62.74	0.01\\
62.75	0.01\\
62.76	0.01\\
62.77	0.01\\
62.78	0.01\\
62.79	0.01\\
62.8	0.01\\
62.81	0.01\\
62.82	0.01\\
62.83	0.01\\
62.84	0.01\\
62.85	0.01\\
62.86	0.01\\
62.87	0.01\\
62.88	0.01\\
62.89	0.01\\
62.9	0.01\\
62.91	0.01\\
62.92	0.01\\
62.93	0.01\\
62.94	0.01\\
62.95	0.01\\
62.96	0.01\\
62.97	0.01\\
62.98	0.01\\
62.99	0.01\\
63	0.01\\
63.01	0.01\\
63.02	0.01\\
63.03	0.01\\
63.04	0.01\\
63.05	0.01\\
63.06	0.01\\
63.07	0.01\\
63.08	0.01\\
63.09	0.01\\
63.1	0.01\\
63.11	0.01\\
63.12	0.01\\
63.13	0.01\\
63.14	0.01\\
63.15	0.01\\
63.16	0.01\\
63.17	0.01\\
63.18	0.01\\
63.19	0.01\\
63.2	0.01\\
63.21	0.01\\
63.22	0.01\\
63.23	0.01\\
63.24	0.01\\
63.25	0.01\\
63.26	0.01\\
63.27	0.01\\
63.28	0.01\\
63.29	0.01\\
63.3	0.01\\
63.31	0.01\\
63.32	0.01\\
63.33	0.01\\
63.34	0.01\\
63.35	0.01\\
63.36	0.01\\
63.37	0.01\\
63.38	0.01\\
63.39	0.01\\
63.4	0.01\\
63.41	0.01\\
63.42	0.01\\
63.43	0.01\\
63.44	0.01\\
63.45	0.01\\
63.46	0.01\\
63.47	0.01\\
63.48	0.01\\
63.49	0.01\\
63.5	0.01\\
63.51	0.01\\
63.52	0.01\\
63.53	0.01\\
63.54	0.01\\
63.55	0.01\\
63.56	0.01\\
63.57	0.01\\
63.58	0.01\\
63.59	0.01\\
63.6	0.01\\
63.61	0.01\\
63.62	0.01\\
63.63	0.01\\
63.64	0.01\\
63.65	0.01\\
63.66	0.01\\
63.67	0.01\\
63.68	0.01\\
63.69	0.01\\
63.7	0.01\\
63.71	0.01\\
63.72	0.01\\
63.73	0.01\\
63.74	0.01\\
63.75	0.01\\
63.76	0.01\\
63.77	0.01\\
63.78	0.01\\
63.79	0.01\\
63.8	0.01\\
63.81	0.01\\
63.82	0.01\\
63.83	0.01\\
63.84	0.01\\
63.85	0.01\\
63.86	0.01\\
63.87	0.01\\
63.88	0.01\\
63.89	0.01\\
63.9	0.01\\
63.91	0.01\\
63.92	0.01\\
63.93	0.01\\
63.94	0.01\\
63.95	0.01\\
63.96	0.01\\
63.97	0.01\\
63.98	0.01\\
63.99	0.01\\
64	0.01\\
64.01	0.01\\
64.02	0.01\\
64.03	0.01\\
64.04	0.01\\
64.05	0.01\\
64.06	0.01\\
64.07	0.01\\
64.08	0.01\\
64.09	0.01\\
64.1	0.01\\
64.11	0.01\\
64.12	0.01\\
64.13	0.01\\
64.14	0.01\\
64.15	0.01\\
64.16	0.01\\
64.17	0.01\\
64.18	0.01\\
64.19	0.01\\
64.2	0.01\\
64.21	0.01\\
64.22	0.01\\
64.23	0.01\\
64.24	0.01\\
64.25	0.01\\
64.26	0.01\\
64.27	0.01\\
64.28	0.01\\
64.29	0.01\\
64.3	0.01\\
64.31	0.01\\
64.32	0.01\\
64.33	0.01\\
64.34	0.01\\
64.35	0.01\\
64.36	0.01\\
64.37	0.01\\
64.38	0.01\\
64.39	0.01\\
64.4	0.01\\
64.41	0.01\\
64.42	0.01\\
64.43	0.01\\
64.44	0.01\\
64.45	0.01\\
64.46	0.01\\
64.47	0.01\\
64.48	0.01\\
64.49	0.01\\
64.5	0.01\\
64.51	0.01\\
64.52	0.01\\
64.53	0.01\\
64.54	0.01\\
64.55	0.01\\
64.56	0.01\\
64.57	0.01\\
64.58	0.01\\
64.59	0.01\\
64.6	0.01\\
64.61	0.01\\
64.62	0.01\\
64.63	0.01\\
64.64	0.01\\
64.65	0.01\\
64.66	0.01\\
64.67	0.01\\
64.68	0.01\\
64.69	0.01\\
64.7	0.01\\
64.71	0.01\\
64.72	0.01\\
64.73	0.01\\
64.74	0.01\\
64.75	0.01\\
64.76	0.01\\
64.77	0.01\\
64.78	0.01\\
64.79	0.01\\
64.8	0.01\\
64.81	0.01\\
64.82	0.01\\
64.83	0.01\\
64.84	0.01\\
64.85	0.01\\
64.86	0.01\\
64.87	0.01\\
64.88	0.01\\
64.89	0.01\\
64.9	0.01\\
64.91	0.01\\
64.92	0.01\\
64.93	0.01\\
64.94	0.01\\
64.95	0.01\\
64.96	0.01\\
64.97	0.01\\
64.98	0.01\\
64.99	0.01\\
65	0.01\\
65.01	0.01\\
65.02	0.01\\
65.03	0.01\\
65.04	0.01\\
65.05	0.01\\
65.06	0.01\\
65.07	0.01\\
65.08	0.01\\
65.09	0.01\\
65.1	0.01\\
65.11	0.01\\
65.12	0.01\\
65.13	0.01\\
65.14	0.01\\
65.15	0.01\\
65.16	0.01\\
65.17	0.01\\
65.18	0.01\\
65.19	0.01\\
65.2	0.01\\
65.21	0.01\\
65.22	0.01\\
65.23	0.01\\
65.24	0.01\\
65.25	0.01\\
65.26	0.01\\
65.27	0.01\\
65.28	0.01\\
65.29	0.01\\
65.3	0.01\\
65.31	0.01\\
65.32	0.01\\
65.33	0.01\\
65.34	0.01\\
65.35	0.01\\
65.36	0.01\\
65.37	0.01\\
65.38	0.01\\
65.39	0.01\\
65.4	0.01\\
65.41	0.01\\
65.42	0.01\\
65.43	0.01\\
65.44	0.01\\
65.45	0.01\\
65.46	0.01\\
65.47	0.01\\
65.48	0.01\\
65.49	0.01\\
65.5	0.01\\
65.51	0.01\\
65.52	0.01\\
65.53	0.01\\
65.54	0.01\\
65.55	0.01\\
65.56	0.01\\
65.57	0.01\\
65.58	0.01\\
65.59	0.01\\
65.6	0.01\\
65.61	0.01\\
65.62	0.01\\
65.63	0.01\\
65.64	0.01\\
65.65	0.01\\
65.66	0.01\\
65.67	0.01\\
65.68	0.01\\
65.69	0.01\\
65.7	0.01\\
65.71	0.01\\
65.72	0.01\\
65.73	0.01\\
65.74	0.01\\
65.75	0.01\\
65.76	0.01\\
65.77	0.01\\
65.78	0.01\\
65.79	0.01\\
65.8	0.01\\
65.81	0.01\\
65.82	0.01\\
65.83	0.01\\
65.84	0.01\\
65.85	0.01\\
65.86	0.01\\
65.87	0.01\\
65.88	0.01\\
65.89	0.01\\
65.9	0.01\\
65.91	0.01\\
65.92	0.01\\
65.93	0.01\\
65.94	0.01\\
65.95	0.01\\
65.96	0.01\\
65.97	0.01\\
65.98	0.01\\
65.99	0.01\\
66	0.01\\
66.01	0.01\\
66.02	0.01\\
66.03	0.01\\
66.04	0.01\\
66.05	0.01\\
66.06	0.01\\
66.07	0.01\\
66.08	0.01\\
66.09	0.01\\
66.1	0.01\\
66.11	0.01\\
66.12	0.01\\
66.13	0.01\\
66.14	0.01\\
66.15	0.01\\
66.16	0.01\\
66.17	0.01\\
66.18	0.01\\
66.19	0.01\\
66.2	0.01\\
66.21	0.01\\
66.22	0.01\\
66.23	0.01\\
66.24	0.01\\
66.25	0.01\\
66.26	0.01\\
66.27	0.01\\
66.28	0.01\\
66.29	0.01\\
66.3	0.01\\
66.31	0.01\\
66.32	0.01\\
66.33	0.01\\
66.34	0.01\\
66.35	0.01\\
66.36	0.01\\
66.37	0.01\\
66.38	0.01\\
66.39	0.01\\
66.4	0.01\\
66.41	0.01\\
66.42	0.01\\
66.43	0.01\\
66.44	0.01\\
66.45	0.01\\
66.46	0.01\\
66.47	0.01\\
66.48	0.01\\
66.49	0.01\\
66.5	0.01\\
66.51	0.01\\
66.52	0.01\\
66.53	0.01\\
66.54	0.01\\
66.55	0.01\\
66.56	0.01\\
66.57	0.01\\
66.58	0.01\\
66.59	0.01\\
66.6	0.01\\
66.61	0.01\\
66.62	0.01\\
66.63	0.01\\
66.64	0.01\\
66.65	0.01\\
66.66	0.01\\
66.67	0.01\\
66.68	0.01\\
66.69	0.01\\
66.7	0.01\\
66.71	0.01\\
66.72	0.01\\
66.73	0.01\\
66.74	0.01\\
66.75	0.01\\
66.76	0.01\\
66.77	0.01\\
66.78	0.01\\
66.79	0.01\\
66.8	0.01\\
66.81	0.01\\
66.82	0.01\\
66.83	0.01\\
66.84	0.01\\
66.85	0.01\\
66.86	0.01\\
66.87	0.01\\
66.88	0.01\\
66.89	0.01\\
66.9	0.01\\
66.91	0.01\\
66.92	0.01\\
66.93	0.01\\
66.94	0.01\\
66.95	0.01\\
66.96	0.01\\
66.97	0.01\\
66.98	0.01\\
66.99	0.01\\
67	0.01\\
67.01	0.01\\
67.02	0.01\\
67.03	0.01\\
67.04	0.01\\
67.05	0.01\\
67.06	0.01\\
67.07	0.01\\
67.08	0.01\\
67.09	0.01\\
67.1	0.01\\
67.11	0.01\\
67.12	0.01\\
67.13	0.01\\
67.14	0.01\\
67.15	0.01\\
67.16	0.01\\
67.17	0.01\\
67.18	0.01\\
67.19	0.01\\
67.2	0.01\\
67.21	0.01\\
67.22	0.01\\
67.23	0.01\\
67.24	0.01\\
67.25	0.01\\
67.26	0.01\\
67.27	0.01\\
67.28	0.01\\
67.29	0.01\\
67.3	0.01\\
67.31	0.01\\
67.32	0.01\\
67.33	0.01\\
67.34	0.01\\
67.35	0.01\\
67.36	0.01\\
67.37	0.01\\
67.38	0.01\\
67.39	0.01\\
67.4	0.01\\
67.41	0.01\\
67.42	0.01\\
67.43	0.01\\
67.44	0.01\\
67.45	0.01\\
67.46	0.01\\
67.47	0.01\\
67.48	0.01\\
67.49	0.01\\
67.5	0.01\\
67.51	0.01\\
67.52	0.01\\
67.53	0.01\\
67.54	0.01\\
67.55	0.01\\
67.56	0.01\\
67.57	0.01\\
67.58	0.01\\
67.59	0.01\\
67.6	0.01\\
67.61	0.01\\
67.62	0.01\\
67.63	0.01\\
67.64	0.01\\
67.65	0.01\\
67.66	0.01\\
67.67	0.01\\
67.68	0.01\\
67.69	0.01\\
67.7	0.01\\
67.71	0.01\\
67.72	0.01\\
67.73	0.01\\
67.74	0.01\\
67.75	0.01\\
67.76	0.01\\
67.77	0.01\\
67.78	0.01\\
67.79	0.01\\
67.8	0.01\\
67.81	0.01\\
67.82	0.01\\
67.83	0.01\\
67.84	0.01\\
67.85	0.01\\
67.86	0.01\\
67.87	0.01\\
67.88	0.01\\
67.89	0.01\\
67.9	0.01\\
67.91	0.01\\
67.92	0.01\\
67.93	0.01\\
67.94	0.01\\
67.95	0.01\\
67.96	0.01\\
67.97	0.01\\
67.98	0.01\\
67.99	0.01\\
68	0.01\\
68.01	0.01\\
68.02	0.01\\
68.03	0.01\\
68.04	0.01\\
68.05	0.01\\
68.06	0.01\\
68.07	0.01\\
68.08	0.01\\
68.09	0.01\\
68.1	0.01\\
68.11	0.01\\
68.12	0.01\\
68.13	0.01\\
68.14	0.01\\
68.15	0.01\\
68.16	0.01\\
68.17	0.01\\
68.18	0.01\\
68.19	0.01\\
68.2	0.01\\
68.21	0.01\\
68.22	0.01\\
68.23	0.01\\
68.24	0.01\\
68.25	0.01\\
68.26	0.01\\
68.27	0.01\\
68.28	0.01\\
68.29	0.01\\
68.3	0.01\\
68.31	0.01\\
68.32	0.01\\
68.33	0.01\\
68.34	0.01\\
68.35	0.01\\
68.36	0.01\\
68.37	0.01\\
68.38	0.01\\
68.39	0.01\\
68.4	0.01\\
68.41	0.01\\
68.42	0.01\\
68.43	0.01\\
68.44	0.01\\
68.45	0.01\\
68.46	0.01\\
68.47	0.01\\
68.48	0.01\\
68.49	0.01\\
68.5	0.01\\
68.51	0.01\\
68.52	0.01\\
68.53	0.01\\
68.54	0.01\\
68.55	0.01\\
68.56	0.01\\
68.57	0.01\\
68.58	0.01\\
68.59	0.01\\
68.6	0.01\\
68.61	0.01\\
68.62	0.01\\
68.63	0.01\\
68.64	0.01\\
68.65	0.01\\
68.66	0.01\\
68.67	0.01\\
68.68	0.01\\
68.69	0.01\\
68.7	0.01\\
68.71	0.01\\
68.72	0.01\\
68.73	0.01\\
68.74	0.01\\
68.75	0.01\\
68.76	0.01\\
68.77	0.01\\
68.78	0.01\\
68.79	0.01\\
68.8	0.01\\
68.81	0.01\\
68.82	0.01\\
68.83	0.01\\
68.84	0.01\\
68.85	0.01\\
68.86	0.01\\
68.87	0.01\\
68.88	0.01\\
68.89	0.01\\
68.9	0.01\\
68.91	0.01\\
68.92	0.01\\
68.93	0.01\\
68.94	0.01\\
68.95	0.01\\
68.96	0.01\\
68.97	0.01\\
68.98	0.01\\
68.99	0.01\\
69	0.01\\
69.01	0.01\\
69.02	0.01\\
69.03	0.01\\
69.04	0.01\\
69.05	0.01\\
69.06	0.01\\
69.07	0.01\\
69.08	0.01\\
69.09	0.01\\
69.1	0.01\\
69.11	0.01\\
69.12	0.01\\
69.13	0.01\\
69.14	0.01\\
69.15	0.01\\
69.16	0.01\\
69.17	0.01\\
69.18	0.01\\
69.19	0.01\\
69.2	0.01\\
69.21	0.01\\
69.22	0.01\\
69.23	0.01\\
69.24	0.01\\
69.25	0.01\\
69.26	0.01\\
69.27	0.01\\
69.28	0.01\\
69.29	0.01\\
69.3	0.01\\
69.31	0.01\\
69.32	0.01\\
69.33	0.01\\
69.34	0.01\\
69.35	0.01\\
69.36	0.01\\
69.37	0.01\\
69.38	0.01\\
69.39	0.01\\
69.4	0.01\\
69.41	0.01\\
69.42	0.01\\
69.43	0.01\\
69.44	0.01\\
69.45	0.01\\
69.46	0.01\\
69.47	0.01\\
69.48	0.01\\
69.49	0.01\\
69.5	0.01\\
69.51	0.01\\
69.52	0.01\\
69.53	0.01\\
69.54	0.01\\
69.55	0.01\\
69.56	0.01\\
69.57	0.01\\
69.58	0.01\\
69.59	0.01\\
69.6	0.01\\
69.61	0.01\\
69.62	0.01\\
69.63	0.01\\
69.64	0.01\\
69.65	0.01\\
69.66	0.01\\
69.67	0.01\\
69.68	0.01\\
69.69	0.01\\
69.7	0.01\\
69.71	0.01\\
69.72	0.01\\
69.73	0.01\\
69.74	0.01\\
69.75	0.01\\
69.76	0.01\\
69.77	0.01\\
69.78	0.01\\
69.79	0.01\\
69.8	0.01\\
69.81	0.01\\
69.82	0.01\\
69.83	0.01\\
69.84	0.01\\
69.85	0.01\\
69.86	0.01\\
69.87	0.01\\
69.88	0.01\\
69.89	0.01\\
69.9	0.01\\
69.91	0.01\\
69.92	0.01\\
69.93	0.01\\
69.94	0.01\\
69.95	0.01\\
69.96	0.01\\
69.97	0.01\\
69.98	0.01\\
69.99	0.01\\
70	0.01\\
70.01	0.01\\
70.02	0.01\\
70.03	0.01\\
70.04	0.01\\
70.05	0.01\\
70.06	0.01\\
70.07	0.01\\
70.08	0.01\\
70.09	0.01\\
70.1	0.01\\
70.11	0.01\\
70.12	0.01\\
70.13	0.01\\
70.14	0.01\\
70.15	0.01\\
70.16	0.01\\
70.17	0.01\\
70.18	0.01\\
70.19	0.01\\
70.2	0.01\\
70.21	0.01\\
70.22	0.01\\
70.23	0.01\\
70.24	0.01\\
70.25	0.01\\
70.26	0.01\\
70.27	0.01\\
70.28	0.01\\
70.29	0.01\\
70.3	0.01\\
70.31	0.01\\
70.32	0.01\\
70.33	0.01\\
70.34	0.01\\
70.35	0.01\\
70.36	0.01\\
70.37	0.01\\
70.38	0.01\\
70.39	0.01\\
70.4	0.01\\
70.41	0.01\\
70.42	0.01\\
70.43	0.01\\
70.44	0.01\\
70.45	0.01\\
70.46	0.01\\
70.47	0.01\\
70.48	0.01\\
70.49	0.01\\
70.5	0.01\\
70.51	0.01\\
70.52	0.01\\
70.53	0.01\\
70.54	0.01\\
70.55	0.01\\
70.56	0.01\\
70.57	0.01\\
70.58	0.01\\
70.59	0.01\\
70.6	0.01\\
70.61	0.01\\
70.62	0.01\\
70.63	0.01\\
70.64	0.01\\
70.65	0.01\\
70.66	0.01\\
70.67	0.01\\
70.68	0.01\\
70.69	0.01\\
70.7	0.01\\
70.71	0.01\\
70.72	0.01\\
70.73	0.01\\
70.74	0.01\\
70.75	0.01\\
70.76	0.01\\
70.77	0.01\\
70.78	0.01\\
70.79	0.01\\
70.8	0.01\\
70.81	0.01\\
70.82	0.01\\
70.83	0.01\\
70.84	0.01\\
70.85	0.01\\
70.86	0.01\\
70.87	0.01\\
70.88	0.01\\
70.89	0.01\\
70.9	0.01\\
70.91	0.01\\
70.92	0.01\\
70.93	0.01\\
70.94	0.01\\
70.95	0.01\\
70.96	0.01\\
70.97	0.01\\
70.98	0.01\\
70.99	0.01\\
71	0.01\\
71.01	0.01\\
71.02	0.01\\
71.03	0.01\\
71.04	0.01\\
71.05	0.01\\
71.06	0.01\\
71.07	0.01\\
71.08	0.01\\
71.09	0.01\\
71.1	0.01\\
71.11	0.01\\
71.12	0.01\\
71.13	0.01\\
71.14	0.01\\
71.15	0.01\\
71.16	0.01\\
71.17	0.01\\
71.18	0.01\\
71.19	0.01\\
71.2	0.01\\
71.21	0.01\\
71.22	0.01\\
71.23	0.01\\
71.24	0.01\\
71.25	0.01\\
71.26	0.01\\
71.27	0.01\\
71.28	0.01\\
71.29	0.01\\
71.3	0.01\\
71.31	0.01\\
71.32	0.01\\
71.33	0.01\\
71.34	0.01\\
71.35	0.01\\
71.36	0.01\\
71.37	0.01\\
71.38	0.01\\
71.39	0.01\\
71.4	0.01\\
71.41	0.01\\
71.42	0.01\\
71.43	0.01\\
71.44	0.01\\
71.45	0.01\\
71.46	0.01\\
71.47	0.01\\
71.48	0.01\\
71.49	0.01\\
71.5	0.01\\
71.51	0.01\\
71.52	0.01\\
71.53	0.01\\
71.54	0.01\\
71.55	0.01\\
71.56	0.01\\
71.57	0.01\\
71.58	0.01\\
71.59	0.01\\
71.6	0.01\\
71.61	0.01\\
71.62	0.01\\
71.63	0.01\\
71.64	0.01\\
71.65	0.01\\
71.66	0.01\\
71.67	0.01\\
71.68	0.01\\
71.69	0.01\\
71.7	0.01\\
71.71	0.01\\
71.72	0.01\\
71.73	0.01\\
71.74	0.01\\
71.75	0.01\\
71.76	0.01\\
71.77	0.01\\
71.78	0.01\\
71.79	0.01\\
71.8	0.01\\
71.81	0.01\\
71.82	0.01\\
71.83	0.01\\
71.84	0.01\\
71.85	0.01\\
71.86	0.01\\
71.87	0.01\\
71.88	0.01\\
71.89	0.01\\
71.9	0.01\\
71.91	0.01\\
71.92	0.01\\
71.93	0.01\\
71.94	0.01\\
71.95	0.01\\
71.96	0.01\\
71.97	0.01\\
71.98	0.01\\
71.99	0.01\\
72	0.01\\
72.01	0.01\\
72.02	0.01\\
72.03	0.01\\
72.04	0.01\\
72.05	0.01\\
72.06	0.01\\
72.07	0.01\\
72.08	0.01\\
72.09	0.01\\
72.1	0.01\\
72.11	0.01\\
72.12	0.01\\
72.13	0.01\\
72.14	0.01\\
72.15	0.01\\
72.16	0.01\\
72.17	0.01\\
72.18	0.01\\
72.19	0.01\\
72.2	0.01\\
72.21	0.01\\
72.22	0.01\\
72.23	0.01\\
72.24	0.01\\
72.25	0.01\\
72.26	0.01\\
72.27	0.01\\
72.28	0.01\\
72.29	0.01\\
72.3	0.01\\
72.31	0.01\\
72.32	0.01\\
72.33	0.01\\
72.34	0.01\\
72.35	0.01\\
72.36	0.01\\
72.37	0.01\\
72.38	0.01\\
72.39	0.01\\
72.4	0.01\\
72.41	0.01\\
72.42	0.01\\
72.43	0.01\\
72.44	0.01\\
72.45	0.01\\
72.46	0.01\\
72.47	0.01\\
72.48	0.01\\
72.49	0.01\\
72.5	0.01\\
72.51	0.01\\
72.52	0.01\\
72.53	0.01\\
72.54	0.01\\
72.55	0.01\\
72.56	0.01\\
72.57	0.01\\
72.58	0.01\\
72.59	0.01\\
72.6	0.01\\
72.61	0.01\\
72.62	0.01\\
72.63	0.01\\
72.64	0.01\\
72.65	0.01\\
72.66	0.01\\
72.67	0.01\\
72.68	0.01\\
72.69	0.01\\
72.7	0.01\\
72.71	0.01\\
72.72	0.01\\
72.73	0.01\\
72.74	0.01\\
72.75	0.01\\
72.76	0.01\\
72.77	0.01\\
72.78	0.01\\
72.79	0.01\\
72.8	0.01\\
72.81	0.01\\
72.82	0.01\\
72.83	0.01\\
72.84	0.01\\
72.85	0.01\\
72.86	0.01\\
72.87	0.01\\
72.88	0.01\\
72.89	0.01\\
72.9	0.01\\
72.91	0.01\\
72.92	0.01\\
72.93	0.01\\
72.94	0.01\\
72.95	0.01\\
72.96	0.01\\
72.97	0.01\\
72.98	0.01\\
72.99	0.01\\
73	0.01\\
73.01	0.01\\
73.02	0.01\\
73.03	0.01\\
73.04	0.01\\
73.05	0.01\\
73.06	0.01\\
73.07	0.01\\
73.08	0.01\\
73.09	0.01\\
73.1	0.01\\
73.11	0.01\\
73.12	0.01\\
73.13	0.01\\
73.14	0.01\\
73.15	0.01\\
73.16	0.01\\
73.17	0.01\\
73.18	0.01\\
73.19	0.01\\
73.2	0.01\\
73.21	0.01\\
73.22	0.01\\
73.23	0.01\\
73.24	0.01\\
73.25	0.01\\
73.26	0.01\\
73.27	0.01\\
73.28	0.01\\
73.29	0.01\\
73.3	0.01\\
73.31	0.01\\
73.32	0.01\\
73.33	0.01\\
73.34	0.01\\
73.35	0.01\\
73.36	0.01\\
73.37	0.01\\
73.38	0.01\\
73.39	0.01\\
73.4	0.01\\
73.41	0.01\\
73.42	0.01\\
73.43	0.01\\
73.44	0.01\\
73.45	0.01\\
73.46	0.01\\
73.47	0.01\\
73.48	0.01\\
73.49	0.01\\
73.5	0.01\\
73.51	0.01\\
73.52	0.01\\
73.53	0.01\\
73.54	0.01\\
73.55	0.01\\
73.56	0.01\\
73.57	0.01\\
73.58	0.01\\
73.59	0.01\\
73.6	0.01\\
73.61	0.01\\
73.62	0.01\\
73.63	0.01\\
73.64	0.01\\
73.65	0.01\\
73.66	0.01\\
73.67	0.01\\
73.68	0.01\\
73.69	0.01\\
73.7	0.01\\
73.71	0.01\\
73.72	0.01\\
73.73	0.01\\
73.74	0.01\\
73.75	0.01\\
73.76	0.01\\
73.77	0.01\\
73.78	0.01\\
73.79	0.01\\
73.8	0.01\\
73.81	0.01\\
73.82	0.01\\
73.83	0.01\\
73.84	0.01\\
73.85	0.01\\
73.86	0.01\\
73.87	0.01\\
73.88	0.01\\
73.89	0.01\\
73.9	0.01\\
73.91	0.01\\
73.92	0.01\\
73.93	0.01\\
73.94	0.01\\
73.95	0.01\\
73.96	0.01\\
73.97	0.01\\
73.98	0.01\\
73.99	0.01\\
74	0.01\\
74.01	0.01\\
74.02	0.01\\
74.03	0.01\\
74.04	0.01\\
74.05	0.01\\
74.06	0.01\\
74.07	0.01\\
74.08	0.01\\
74.09	0.01\\
74.1	0.01\\
74.11	0.01\\
74.12	0.01\\
74.13	0.01\\
74.14	0.01\\
74.15	0.01\\
74.16	0.01\\
74.17	0.01\\
74.18	0.01\\
74.19	0.01\\
74.2	0.01\\
74.21	0.01\\
74.22	0.01\\
74.23	0.01\\
74.24	0.01\\
74.25	0.01\\
74.26	0.01\\
74.27	0.01\\
74.28	0.01\\
74.29	0.01\\
74.3	0.01\\
74.31	0.01\\
74.32	0.01\\
74.33	0.01\\
74.34	0.01\\
74.35	0.01\\
74.36	0.01\\
74.37	0.01\\
74.38	0.01\\
74.39	0.01\\
74.4	0.01\\
74.41	0.01\\
74.42	0.01\\
74.43	0.01\\
74.44	0.01\\
74.45	0.01\\
74.46	0.01\\
74.47	0.01\\
74.48	0.01\\
74.49	0.01\\
74.5	0.01\\
74.51	0.01\\
74.52	0.01\\
74.53	0.01\\
74.54	0.01\\
74.55	0.01\\
74.56	0.01\\
74.57	0.01\\
74.58	0.01\\
74.59	0.01\\
74.6	0.01\\
74.61	0.01\\
74.62	0.01\\
74.63	0.01\\
74.64	0.01\\
74.65	0.01\\
74.66	0.01\\
74.67	0.01\\
74.68	0.01\\
74.69	0.01\\
74.7	0.01\\
74.71	0.01\\
74.72	0.01\\
74.73	0.01\\
74.74	0.01\\
74.75	0.01\\
74.76	0.01\\
74.77	0.01\\
74.78	0.01\\
74.79	0.01\\
74.8	0.01\\
74.81	0.01\\
74.82	0.01\\
74.83	0.01\\
74.84	0.01\\
74.85	0.01\\
74.86	0.01\\
74.87	0.01\\
74.88	0.01\\
74.89	0.01\\
74.9	0.01\\
74.91	0.01\\
74.92	0.01\\
74.93	0.01\\
74.94	0.01\\
74.95	0.01\\
74.96	0.01\\
74.97	0.01\\
74.98	0.01\\
74.99	0.01\\
75	0.01\\
75.01	0.01\\
75.02	0.01\\
75.03	0.01\\
75.04	0.01\\
75.05	0.01\\
75.06	0.01\\
75.07	0.01\\
75.08	0.01\\
75.09	0.01\\
75.1	0.01\\
75.11	0.01\\
75.12	0.01\\
75.13	0.01\\
75.14	0.01\\
75.15	0.01\\
75.16	0.01\\
75.17	0.01\\
75.18	0.01\\
75.19	0.01\\
75.2	0.01\\
75.21	0.01\\
75.22	0.01\\
75.23	0.01\\
75.24	0.01\\
75.25	0.01\\
75.26	0.01\\
75.27	0.01\\
75.28	0.01\\
75.29	0.01\\
75.3	0.01\\
75.31	0.01\\
75.32	0.01\\
75.33	0.01\\
75.34	0.01\\
75.35	0.01\\
75.36	0.01\\
75.37	0.01\\
75.38	0.01\\
75.39	0.01\\
75.4	0.01\\
75.41	0.01\\
75.42	0.01\\
75.43	0.01\\
75.44	0.01\\
75.45	0.01\\
75.46	0.01\\
75.47	0.01\\
75.48	0.01\\
75.49	0.01\\
75.5	0.01\\
75.51	0.01\\
75.52	0.01\\
75.53	0.01\\
75.54	0.01\\
75.55	0.01\\
75.56	0.01\\
75.57	0.01\\
75.58	0.01\\
75.59	0.01\\
75.6	0.01\\
75.61	0.01\\
75.62	0.01\\
75.63	0.01\\
75.64	0.01\\
75.65	0.01\\
75.66	0.01\\
75.67	0.01\\
75.68	0.01\\
75.69	0.01\\
75.7	0.01\\
75.71	0.01\\
75.72	0.01\\
75.73	0.01\\
75.74	0.01\\
75.75	0.01\\
75.76	0.01\\
75.77	0.01\\
75.78	0.01\\
75.79	0.01\\
75.8	0.01\\
75.81	0.01\\
75.82	0.01\\
75.83	0.01\\
75.84	0.01\\
75.85	0.01\\
75.86	0.01\\
75.87	0.01\\
75.88	0.01\\
75.89	0.01\\
75.9	0.01\\
75.91	0.01\\
75.92	0.01\\
75.93	0.01\\
75.94	0.01\\
75.95	0.01\\
75.96	0.01\\
75.97	0.01\\
75.98	0.01\\
75.99	0.01\\
76	0.01\\
76.01	0.01\\
76.02	0.01\\
76.03	0.01\\
76.04	0.01\\
76.05	0.01\\
76.06	0.01\\
76.07	0.01\\
76.08	0.01\\
76.09	0.01\\
76.1	0.01\\
76.11	0.01\\
76.12	0.01\\
76.13	0.01\\
76.14	0.01\\
76.15	0.01\\
76.16	0.01\\
76.17	0.01\\
76.18	0.01\\
76.19	0.01\\
76.2	0.01\\
76.21	0.01\\
76.22	0.01\\
76.23	0.01\\
76.24	0.01\\
76.25	0.01\\
76.26	0.01\\
76.27	0.01\\
76.28	0.01\\
76.29	0.01\\
76.3	0.01\\
76.31	0.01\\
76.32	0.01\\
76.33	0.01\\
76.34	0.01\\
76.35	0.01\\
76.36	0.01\\
76.37	0.01\\
76.38	0.01\\
76.39	0.01\\
76.4	0.01\\
76.41	0.01\\
76.42	0.01\\
76.43	0.01\\
76.44	0.01\\
76.45	0.01\\
76.46	0.01\\
76.47	0.01\\
76.48	0.01\\
76.49	0.01\\
76.5	0.01\\
76.51	0.01\\
76.52	0.01\\
76.53	0.01\\
76.54	0.01\\
76.55	0.01\\
76.56	0.01\\
76.57	0.01\\
76.58	0.01\\
76.59	0.01\\
76.6	0.01\\
76.61	0.01\\
76.62	0.01\\
76.63	0.01\\
76.64	0.01\\
76.65	0.01\\
76.66	0.01\\
76.67	0.01\\
76.68	0.01\\
76.69	0.01\\
76.7	0.01\\
76.71	0.01\\
76.72	0.01\\
76.73	0.01\\
76.74	0.01\\
76.75	0.01\\
76.76	0.01\\
76.77	0.01\\
76.78	0.01\\
76.79	0.01\\
76.8	0.01\\
76.81	0.01\\
76.82	0.01\\
76.83	0.01\\
76.84	0.01\\
76.85	0.01\\
76.86	0.01\\
76.87	0.01\\
76.88	0.01\\
76.89	0.01\\
76.9	0.01\\
76.91	0.01\\
76.92	0.01\\
76.93	0.01\\
76.94	0.01\\
76.95	0.01\\
76.96	0.01\\
76.97	0.01\\
76.98	0.01\\
76.99	0.01\\
77	0.01\\
77.01	0.01\\
77.02	0.01\\
77.03	0.01\\
77.04	0.01\\
77.05	0.01\\
77.06	0.01\\
77.07	0.01\\
77.08	0.01\\
77.09	0.01\\
77.1	0.01\\
77.11	0.01\\
77.12	0.01\\
77.13	0.01\\
77.14	0.01\\
77.15	0.01\\
77.16	0.01\\
77.17	0.01\\
77.18	0.01\\
77.19	0.01\\
77.2	0.01\\
77.21	0.01\\
77.22	0.01\\
77.23	0.01\\
77.24	0.01\\
77.25	0.01\\
77.26	0.01\\
77.27	0.01\\
77.28	0.01\\
77.29	0.01\\
77.3	0.01\\
77.31	0.01\\
77.32	0.01\\
77.33	0.01\\
77.34	0.01\\
77.35	0.01\\
77.36	0.01\\
77.37	0.01\\
77.38	0.01\\
77.39	0.01\\
77.4	0.01\\
77.41	0.01\\
77.42	0.01\\
77.43	0.01\\
77.44	0.01\\
77.45	0.01\\
77.46	0.01\\
77.47	0.01\\
77.48	0.01\\
77.49	0.01\\
77.5	0.01\\
77.51	0.01\\
77.52	0.01\\
77.53	0.01\\
77.54	0.01\\
77.55	0.01\\
77.56	0.01\\
77.57	0.01\\
77.58	0.01\\
77.59	0.01\\
77.6	0.01\\
77.61	0.01\\
77.62	0.01\\
77.63	0.01\\
77.64	0.01\\
77.65	0.01\\
77.66	0.01\\
77.67	0.01\\
77.68	0.01\\
77.69	0.01\\
77.7	0.01\\
77.71	0.01\\
77.72	0.01\\
77.73	0.01\\
77.74	0.01\\
77.75	0.01\\
77.76	0.01\\
77.77	0.01\\
77.78	0.01\\
77.79	0.01\\
77.8	0.01\\
77.81	0.01\\
77.82	0.01\\
77.83	0.01\\
77.84	0.01\\
77.85	0.01\\
77.86	0.01\\
77.87	0.01\\
77.88	0.01\\
77.89	0.01\\
77.9	0.01\\
77.91	0.01\\
77.92	0.01\\
77.93	0.01\\
77.94	0.01\\
77.95	0.01\\
77.96	0.01\\
77.97	0.01\\
77.98	0.01\\
77.99	0.01\\
78	0.01\\
78.01	0.01\\
78.02	0.01\\
78.03	0.01\\
78.04	0.01\\
78.05	0.01\\
78.06	0.01\\
78.07	0.01\\
78.08	0.01\\
78.09	0.01\\
78.1	0.01\\
78.11	0.01\\
78.12	0.01\\
78.13	0.01\\
78.14	0.01\\
78.15	0.01\\
78.16	0.01\\
78.17	0.01\\
78.18	0.01\\
78.19	0.01\\
78.2	0.01\\
78.21	0.01\\
78.22	0.01\\
78.23	0.01\\
78.24	0.01\\
78.25	0.01\\
78.26	0.01\\
78.27	0.01\\
78.28	0.01\\
78.29	0.01\\
78.3	0.01\\
78.31	0.01\\
78.32	0.01\\
78.33	0.01\\
78.34	0.01\\
78.35	0.01\\
78.36	0.01\\
78.37	0.01\\
78.38	0.01\\
78.39	0.01\\
78.4	0.01\\
78.41	0.01\\
78.42	0.01\\
78.43	0.01\\
78.44	0.01\\
78.45	0.01\\
78.46	0.01\\
78.47	0.01\\
78.48	0.01\\
78.49	0.01\\
78.5	0.01\\
78.51	0.01\\
78.52	0.01\\
78.53	0.01\\
78.54	0.01\\
78.55	0.01\\
78.56	0.01\\
78.57	0.01\\
78.58	0.01\\
78.59	0.01\\
78.6	0.01\\
78.61	0.01\\
78.62	0.01\\
78.63	0.01\\
78.64	0.01\\
78.65	0.01\\
78.66	0.01\\
78.67	0.01\\
78.68	0.01\\
78.69	0.01\\
78.7	0.01\\
78.71	0.01\\
78.72	0.01\\
78.73	0.01\\
78.74	0.01\\
78.75	0.01\\
78.76	0.01\\
78.77	0.01\\
78.78	0.01\\
78.79	0.01\\
78.8	0.01\\
78.81	0.01\\
78.82	0.01\\
78.83	0.01\\
78.84	0.01\\
78.85	0.01\\
78.86	0.01\\
78.87	0.01\\
78.88	0.01\\
78.89	0.01\\
78.9	0.01\\
78.91	0.01\\
78.92	0.01\\
78.93	0.01\\
78.94	0.01\\
78.95	0.01\\
78.96	0.01\\
78.97	0.01\\
78.98	0.01\\
78.99	0.01\\
79	0.01\\
79.01	0.01\\
79.02	0.01\\
79.03	0.01\\
79.04	0.01\\
79.05	0.01\\
79.06	0.01\\
79.07	0.01\\
79.08	0.01\\
79.09	0.01\\
79.1	0.01\\
79.11	0.01\\
79.12	0.01\\
79.13	0.01\\
79.14	0.01\\
79.15	0.01\\
79.16	0.01\\
79.17	0.01\\
79.18	0.01\\
79.19	0.01\\
79.2	0.01\\
79.21	0.01\\
79.22	0.01\\
79.23	0.01\\
79.24	0.01\\
79.25	0.01\\
79.26	0.01\\
79.27	0.01\\
79.28	0.01\\
79.29	0.01\\
79.3	0.01\\
79.31	0.01\\
79.32	0.01\\
79.33	0.01\\
79.34	0.01\\
79.35	0.01\\
79.36	0.01\\
79.37	0.01\\
79.38	0.01\\
79.39	0.01\\
79.4	0.01\\
79.41	0.01\\
79.42	0.01\\
79.43	0.01\\
79.44	0.01\\
79.45	0.01\\
79.46	0.01\\
79.47	0.01\\
79.48	0.01\\
79.49	0.01\\
79.5	0.01\\
79.51	0.01\\
79.52	0.01\\
79.53	0.01\\
79.54	0.01\\
79.55	0.01\\
79.56	0.01\\
79.57	0.01\\
79.58	0.01\\
79.59	0.01\\
79.6	0.01\\
79.61	0.01\\
79.62	0.01\\
79.63	0.01\\
79.64	0.01\\
79.65	0.01\\
79.66	0.01\\
79.67	0.01\\
79.68	0.01\\
79.69	0.01\\
79.7	0.01\\
79.71	0.01\\
79.72	0.01\\
79.73	0.01\\
79.74	0.01\\
79.75	0.01\\
79.76	0.01\\
79.77	0.01\\
79.78	0.01\\
79.79	0.01\\
79.8	0.01\\
79.81	0.01\\
79.82	0.01\\
79.83	0.01\\
79.84	0.01\\
79.85	0.01\\
79.86	0.01\\
79.87	0.01\\
79.88	0.01\\
79.89	0.01\\
79.9	0.01\\
79.91	0.01\\
79.92	0.01\\
79.93	0.01\\
79.94	0.01\\
79.95	0.01\\
79.96	0.01\\
79.97	0.01\\
79.98	0.01\\
79.99	0.01\\
80	0.01\\
80.01	0.01\\
};
\addplot [color=blue,solid]
  table[row sep=crcr]{%
80.01	0.01\\
80.02	0.01\\
80.03	0.01\\
80.04	0.01\\
80.05	0.01\\
80.06	0.01\\
80.07	0.01\\
80.08	0.01\\
80.09	0.01\\
80.1	0.01\\
80.11	0.01\\
80.12	0.01\\
80.13	0.01\\
80.14	0.01\\
80.15	0.01\\
80.16	0.01\\
80.17	0.01\\
80.18	0.01\\
80.19	0.01\\
80.2	0.01\\
80.21	0.01\\
80.22	0.01\\
80.23	0.01\\
80.24	0.01\\
80.25	0.01\\
80.26	0.01\\
80.27	0.01\\
80.28	0.01\\
80.29	0.01\\
80.3	0.01\\
80.31	0.01\\
80.32	0.01\\
80.33	0.01\\
80.34	0.01\\
80.35	0.01\\
80.36	0.01\\
80.37	0.01\\
80.38	0.01\\
80.39	0.01\\
80.4	0.01\\
80.41	0.01\\
80.42	0.01\\
80.43	0.01\\
80.44	0.01\\
80.45	0.01\\
80.46	0.01\\
80.47	0.01\\
80.48	0.01\\
80.49	0.01\\
80.5	0.01\\
80.51	0.01\\
80.52	0.01\\
80.53	0.01\\
80.54	0.01\\
80.55	0.01\\
80.56	0.01\\
80.57	0.01\\
80.58	0.01\\
80.59	0.01\\
80.6	0.01\\
80.61	0.01\\
80.62	0.01\\
80.63	0.01\\
80.64	0.01\\
80.65	0.01\\
80.66	0.01\\
80.67	0.01\\
80.68	0.01\\
80.69	0.01\\
80.7	0.01\\
80.71	0.01\\
80.72	0.01\\
80.73	0.01\\
80.74	0.01\\
80.75	0.01\\
80.76	0.01\\
80.77	0.01\\
80.78	0.01\\
80.79	0.01\\
80.8	0.01\\
80.81	0.01\\
80.82	0.01\\
80.83	0.01\\
80.84	0.01\\
80.85	0.01\\
80.86	0.01\\
80.87	0.01\\
80.88	0.01\\
80.89	0.01\\
80.9	0.01\\
80.91	0.01\\
80.92	0.01\\
80.93	0.01\\
80.94	0.01\\
80.95	0.01\\
80.96	0.01\\
80.97	0.01\\
80.98	0.01\\
80.99	0.01\\
81	0.01\\
81.01	0.01\\
81.02	0.01\\
81.03	0.01\\
81.04	0.01\\
81.05	0.01\\
81.06	0.01\\
81.07	0.01\\
81.08	0.01\\
81.09	0.01\\
81.1	0.01\\
81.11	0.01\\
81.12	0.01\\
81.13	0.01\\
81.14	0.01\\
81.15	0.01\\
81.16	0.01\\
81.17	0.01\\
81.18	0.01\\
81.19	0.01\\
81.2	0.01\\
81.21	0.01\\
81.22	0.01\\
81.23	0.01\\
81.24	0.01\\
81.25	0.01\\
81.26	0.01\\
81.27	0.01\\
81.28	0.01\\
81.29	0.01\\
81.3	0.01\\
81.31	0.01\\
81.32	0.01\\
81.33	0.01\\
81.34	0.01\\
81.35	0.01\\
81.36	0.01\\
81.37	0.01\\
81.38	0.01\\
81.39	0.01\\
81.4	0.01\\
81.41	0.01\\
81.42	0.01\\
81.43	0.01\\
81.44	0.01\\
81.45	0.01\\
81.46	0.01\\
81.47	0.01\\
81.48	0.01\\
81.49	0.01\\
81.5	0.01\\
81.51	0.01\\
81.52	0.01\\
81.53	0.01\\
81.54	0.01\\
81.55	0.01\\
81.56	0.01\\
81.57	0.01\\
81.58	0.01\\
81.59	0.01\\
81.6	0.01\\
81.61	0.01\\
81.62	0.01\\
81.63	0.01\\
81.64	0.01\\
81.65	0.01\\
81.66	0.01\\
81.67	0.01\\
81.68	0.01\\
81.69	0.01\\
81.7	0.01\\
81.71	0.01\\
81.72	0.01\\
81.73	0.01\\
81.74	0.01\\
81.75	0.01\\
81.76	0.01\\
81.77	0.01\\
81.78	0.01\\
81.79	0.01\\
81.8	0.01\\
81.81	0.01\\
81.82	0.01\\
81.83	0.01\\
81.84	0.01\\
81.85	0.01\\
81.86	0.01\\
81.87	0.01\\
81.88	0.01\\
81.89	0.01\\
81.9	0.01\\
81.91	0.01\\
81.92	0.01\\
81.93	0.01\\
81.94	0.01\\
81.95	0.01\\
81.96	0.01\\
81.97	0.01\\
81.98	0.01\\
81.99	0.01\\
82	0.01\\
82.01	0.01\\
82.02	0.01\\
82.03	0.01\\
82.04	0.01\\
82.05	0.01\\
82.06	0.01\\
82.07	0.01\\
82.08	0.01\\
82.09	0.01\\
82.1	0.01\\
82.11	0.01\\
82.12	0.01\\
82.13	0.01\\
82.14	0.01\\
82.15	0.01\\
82.16	0.01\\
82.17	0.01\\
82.18	0.01\\
82.19	0.01\\
82.2	0.01\\
82.21	0.01\\
82.22	0.01\\
82.23	0.01\\
82.24	0.01\\
82.25	0.01\\
82.26	0.01\\
82.27	0.01\\
82.28	0.01\\
82.29	0.01\\
82.3	0.01\\
82.31	0.01\\
82.32	0.01\\
82.33	0.01\\
82.34	0.01\\
82.35	0.01\\
82.36	0.01\\
82.37	0.01\\
82.38	0.01\\
82.39	0.01\\
82.4	0.01\\
82.41	0.01\\
82.42	0.01\\
82.43	0.01\\
82.44	0.01\\
82.45	0.01\\
82.46	0.01\\
82.47	0.01\\
82.48	0.01\\
82.49	0.01\\
82.5	0.01\\
82.51	0.01\\
82.52	0.01\\
82.53	0.01\\
82.54	0.01\\
82.55	0.01\\
82.56	0.01\\
82.57	0.01\\
82.58	0.01\\
82.59	0.01\\
82.6	0.01\\
82.61	0.01\\
82.62	0.01\\
82.63	0.01\\
82.64	0.01\\
82.65	0.01\\
82.66	0.01\\
82.67	0.01\\
82.68	0.01\\
82.69	0.01\\
82.7	0.01\\
82.71	0.01\\
82.72	0.01\\
82.73	0.01\\
82.74	0.01\\
82.75	0.01\\
82.76	0.01\\
82.77	0.01\\
82.78	0.01\\
82.79	0.01\\
82.8	0.01\\
82.81	0.01\\
82.82	0.01\\
82.83	0.01\\
82.84	0.01\\
82.85	0.01\\
82.86	0.01\\
82.87	0.01\\
82.88	0.01\\
82.89	0.01\\
82.9	0.01\\
82.91	0.01\\
82.92	0.01\\
82.93	0.01\\
82.94	0.01\\
82.95	0.01\\
82.96	0.01\\
82.97	0.01\\
82.98	0.01\\
82.99	0.01\\
83	0.01\\
83.01	0.01\\
83.02	0.01\\
83.03	0.01\\
83.04	0.01\\
83.05	0.01\\
83.06	0.01\\
83.07	0.01\\
83.08	0.01\\
83.09	0.01\\
83.1	0.01\\
83.11	0.01\\
83.12	0.01\\
83.13	0.01\\
83.14	0.01\\
83.15	0.01\\
83.16	0.01\\
83.17	0.01\\
83.18	0.01\\
83.19	0.01\\
83.2	0.01\\
83.21	0.01\\
83.22	0.01\\
83.23	0.01\\
83.24	0.01\\
83.25	0.01\\
83.26	0.01\\
83.27	0.01\\
83.28	0.01\\
83.29	0.01\\
83.3	0.01\\
83.31	0.01\\
83.32	0.01\\
83.33	0.01\\
83.34	0.01\\
83.35	0.01\\
83.36	0.01\\
83.37	0.01\\
83.38	0.01\\
83.39	0.01\\
83.4	0.01\\
83.41	0.01\\
83.42	0.01\\
83.43	0.01\\
83.44	0.01\\
83.45	0.01\\
83.46	0.01\\
83.47	0.01\\
83.48	0.01\\
83.49	0.01\\
83.5	0.01\\
83.51	0.01\\
83.52	0.01\\
83.53	0.01\\
83.54	0.01\\
83.55	0.01\\
83.56	0.01\\
83.57	0.01\\
83.58	0.01\\
83.59	0.01\\
83.6	0.01\\
83.61	0.01\\
83.62	0.01\\
83.63	0.01\\
83.64	0.01\\
83.65	0.01\\
83.66	0.01\\
83.67	0.01\\
83.68	0.01\\
83.69	0.01\\
83.7	0.01\\
83.71	0.01\\
83.72	0.01\\
83.73	0.01\\
83.74	0.01\\
83.75	0.01\\
83.76	0.01\\
83.77	0.01\\
83.78	0.01\\
83.79	0.01\\
83.8	0.01\\
83.81	0.01\\
83.82	0.01\\
83.83	0.01\\
83.84	0.01\\
83.85	0.01\\
83.86	0.01\\
83.87	0.01\\
83.88	0.01\\
83.89	0.01\\
83.9	0.01\\
83.91	0.01\\
83.92	0.01\\
83.93	0.01\\
83.94	0.01\\
83.95	0.01\\
83.96	0.01\\
83.97	0.01\\
83.98	0.01\\
83.99	0.01\\
84	0.01\\
84.01	0.01\\
84.02	0.01\\
84.03	0.01\\
84.04	0.01\\
84.05	0.01\\
84.06	0.01\\
84.07	0.01\\
84.08	0.01\\
84.09	0.01\\
84.1	0.01\\
84.11	0.01\\
84.12	0.01\\
84.13	0.01\\
84.14	0.01\\
84.15	0.01\\
84.16	0.01\\
84.17	0.01\\
84.18	0.01\\
84.19	0.01\\
84.2	0.01\\
84.21	0.01\\
84.22	0.01\\
84.23	0.01\\
84.24	0.01\\
84.25	0.01\\
84.26	0.01\\
84.27	0.01\\
84.28	0.01\\
84.29	0.01\\
84.3	0.01\\
84.31	0.01\\
84.32	0.01\\
84.33	0.01\\
84.34	0.01\\
84.35	0.01\\
84.36	0.01\\
84.37	0.01\\
84.38	0.01\\
84.39	0.01\\
84.4	0.01\\
84.41	0.01\\
84.42	0.01\\
84.43	0.01\\
84.44	0.01\\
84.45	0.01\\
84.46	0.01\\
84.47	0.01\\
84.48	0.01\\
84.49	0.01\\
84.5	0.01\\
84.51	0.01\\
84.52	0.01\\
84.53	0.01\\
84.54	0.01\\
84.55	0.01\\
84.56	0.01\\
84.57	0.01\\
84.58	0.01\\
84.59	0.01\\
84.6	0.01\\
84.61	0.01\\
84.62	0.01\\
84.63	0.01\\
84.64	0.01\\
84.65	0.01\\
84.66	0.01\\
84.67	0.01\\
84.68	0.01\\
84.69	0.01\\
84.7	0.01\\
84.71	0.01\\
84.72	0.01\\
84.73	0.01\\
84.74	0.01\\
84.75	0.01\\
84.76	0.01\\
84.77	0.01\\
84.78	0.01\\
84.79	0.01\\
84.8	0.01\\
84.81	0.01\\
84.82	0.01\\
84.83	0.01\\
84.84	0.01\\
84.85	0.01\\
84.86	0.01\\
84.87	0.01\\
84.88	0.01\\
84.89	0.01\\
84.9	0.01\\
84.91	0.01\\
84.92	0.01\\
84.93	0.01\\
84.94	0.01\\
84.95	0.01\\
84.96	0.01\\
84.97	0.01\\
84.98	0.01\\
84.99	0.01\\
85	0.01\\
85.01	0.01\\
85.02	0.01\\
85.03	0.01\\
85.04	0.01\\
85.05	0.01\\
85.06	0.01\\
85.07	0.01\\
85.08	0.01\\
85.09	0.01\\
85.1	0.01\\
85.11	0.01\\
85.12	0.01\\
85.13	0.01\\
85.14	0.01\\
85.15	0.01\\
85.16	0.01\\
85.17	0.01\\
85.18	0.01\\
85.19	0.01\\
85.2	0.01\\
85.21	0.01\\
85.22	0.01\\
85.23	0.01\\
85.24	0.01\\
85.25	0.01\\
85.26	0.01\\
85.27	0.01\\
85.28	0.01\\
85.29	0.01\\
85.3	0.01\\
85.31	0.01\\
85.32	0.01\\
85.33	0.01\\
85.34	0.01\\
85.35	0.01\\
85.36	0.01\\
85.37	0.01\\
85.38	0.01\\
85.39	0.01\\
85.4	0.01\\
85.41	0.01\\
85.42	0.01\\
85.43	0.01\\
85.44	0.01\\
85.45	0.01\\
85.46	0.01\\
85.47	0.01\\
85.48	0.01\\
85.49	0.01\\
85.5	0.01\\
85.51	0.01\\
85.52	0.01\\
85.53	0.01\\
85.54	0.01\\
85.55	0.01\\
85.56	0.01\\
85.57	0.01\\
85.58	0.01\\
85.59	0.01\\
85.6	0.01\\
85.61	0.01\\
85.62	0.01\\
85.63	0.01\\
85.64	0.01\\
85.65	0.01\\
85.66	0.01\\
85.67	0.01\\
85.68	0.01\\
85.69	0.01\\
85.7	0.01\\
85.71	0.01\\
85.72	0.01\\
85.73	0.01\\
85.74	0.01\\
85.75	0.01\\
85.76	0.01\\
85.77	0.01\\
85.78	0.01\\
85.79	0.01\\
85.8	0.01\\
85.81	0.01\\
85.82	0.01\\
85.83	0.01\\
85.84	0.01\\
85.85	0.01\\
85.86	0.01\\
85.87	0.01\\
85.88	0.01\\
85.89	0.01\\
85.9	0.01\\
85.91	0.01\\
85.92	0.01\\
85.93	0.01\\
85.94	0.01\\
85.95	0.01\\
85.96	0.01\\
85.97	0.01\\
85.98	0.01\\
85.99	0.01\\
86	0.01\\
86.01	0.01\\
86.02	0.01\\
86.03	0.01\\
86.04	0.01\\
86.05	0.01\\
86.06	0.01\\
86.07	0.01\\
86.08	0.01\\
86.09	0.01\\
86.1	0.01\\
86.11	0.01\\
86.12	0.01\\
86.13	0.01\\
86.14	0.01\\
86.15	0.01\\
86.16	0.01\\
86.17	0.01\\
86.18	0.01\\
86.19	0.01\\
86.2	0.01\\
86.21	0.01\\
86.22	0.01\\
86.23	0.01\\
86.24	0.01\\
86.25	0.01\\
86.26	0.01\\
86.27	0.01\\
86.28	0.01\\
86.29	0.01\\
86.3	0.01\\
86.31	0.01\\
86.32	0.01\\
86.33	0.01\\
86.34	0.01\\
86.35	0.01\\
86.36	0.01\\
86.37	0.01\\
86.38	0.01\\
86.39	0.01\\
86.4	0.01\\
86.41	0.01\\
86.42	0.01\\
86.43	0.01\\
86.44	0.01\\
86.45	0.01\\
86.46	0.01\\
86.47	0.01\\
86.48	0.01\\
86.49	0.01\\
86.5	0.01\\
86.51	0.01\\
86.52	0.01\\
86.53	0.01\\
86.54	0.01\\
86.55	0.01\\
86.56	0.01\\
86.57	0.01\\
86.58	0.01\\
86.59	0.01\\
86.6	0.01\\
86.61	0.01\\
86.62	0.01\\
86.63	0.01\\
86.64	0.01\\
86.65	0.01\\
86.66	0.01\\
86.67	0.01\\
86.68	0.01\\
86.69	0.01\\
86.7	0.01\\
86.71	0.01\\
86.72	0.01\\
86.73	0.01\\
86.74	0.01\\
86.75	0.01\\
86.76	0.01\\
86.77	0.01\\
86.78	0.01\\
86.79	0.01\\
86.8	0.01\\
86.81	0.01\\
86.82	0.01\\
86.83	0.01\\
86.84	0.01\\
86.85	0.01\\
86.86	0.01\\
86.87	0.01\\
86.88	0.01\\
86.89	0.01\\
86.9	0.01\\
86.91	0.01\\
86.92	0.01\\
86.93	0.01\\
86.94	0.01\\
86.95	0.01\\
86.96	0.01\\
86.97	0.01\\
86.98	0.01\\
86.99	0.01\\
87	0.01\\
87.01	0.01\\
87.02	0.01\\
87.03	0.01\\
87.04	0.01\\
87.05	0.01\\
87.06	0.01\\
87.07	0.01\\
87.08	0.01\\
87.09	0.01\\
87.1	0.01\\
87.11	0.01\\
87.12	0.01\\
87.13	0.01\\
87.14	0.01\\
87.15	0.01\\
87.16	0.01\\
87.17	0.01\\
87.18	0.01\\
87.19	0.01\\
87.2	0.01\\
87.21	0.01\\
87.22	0.01\\
87.23	0.01\\
87.24	0.01\\
87.25	0.01\\
87.26	0.01\\
87.27	0.01\\
87.28	0.01\\
87.29	0.01\\
87.3	0.01\\
87.31	0.01\\
87.32	0.01\\
87.33	0.01\\
87.34	0.01\\
87.35	0.01\\
87.36	0.01\\
87.37	0.01\\
87.38	0.01\\
87.39	0.01\\
87.4	0.01\\
87.41	0.01\\
87.42	0.01\\
87.43	0.01\\
87.44	0.01\\
87.45	0.01\\
87.46	0.01\\
87.47	0.01\\
87.48	0.01\\
87.49	0.01\\
87.5	0.01\\
87.51	0.01\\
87.52	0.01\\
87.53	0.01\\
87.54	0.01\\
87.55	0.01\\
87.56	0.01\\
87.57	0.01\\
87.58	0.01\\
87.59	0.01\\
87.6	0.01\\
87.61	0.01\\
87.62	0.01\\
87.63	0.01\\
87.64	0.01\\
87.65	0.01\\
87.66	0.01\\
87.67	0.01\\
87.68	0.01\\
87.69	0.01\\
87.7	0.01\\
87.71	0.01\\
87.72	0.01\\
87.73	0.01\\
87.74	0.01\\
87.75	0.01\\
87.76	0.01\\
87.77	0.01\\
87.78	0.01\\
87.79	0.01\\
87.8	0.01\\
87.81	0.01\\
87.82	0.01\\
87.83	0.01\\
87.84	0.01\\
87.85	0.01\\
87.86	0.01\\
87.87	0.01\\
87.88	0.01\\
87.89	0.01\\
87.9	0.01\\
87.91	0.01\\
87.92	0.01\\
87.93	0.01\\
87.94	0.01\\
87.95	0.01\\
87.96	0.01\\
87.97	0.01\\
87.98	0.01\\
87.99	0.01\\
88	0.01\\
88.01	0.01\\
88.02	0.01\\
88.03	0.01\\
88.04	0.01\\
88.05	0.01\\
88.06	0.01\\
88.07	0.01\\
88.08	0.01\\
88.09	0.01\\
88.1	0.01\\
88.11	0.01\\
88.12	0.01\\
88.13	0.01\\
88.14	0.01\\
88.15	0.01\\
88.16	0.01\\
88.17	0.01\\
88.18	0.01\\
88.19	0.01\\
88.2	0.01\\
88.21	0.01\\
88.22	0.01\\
88.23	0.01\\
88.24	0.01\\
88.25	0.01\\
88.26	0.01\\
88.27	0.01\\
88.28	0.01\\
88.29	0.01\\
88.3	0.01\\
88.31	0.01\\
88.32	0.01\\
88.33	0.01\\
88.34	0.01\\
88.35	0.01\\
88.36	0.01\\
88.37	0.01\\
88.38	0.01\\
88.39	0.01\\
88.4	0.01\\
88.41	0.01\\
88.42	0.01\\
88.43	0.01\\
88.44	0.01\\
88.45	0.01\\
88.46	0.01\\
88.47	0.01\\
88.48	0.01\\
88.49	0.01\\
88.5	0.01\\
88.51	0.01\\
88.52	0.01\\
88.53	0.01\\
88.54	0.01\\
88.55	0.01\\
88.56	0.01\\
88.57	0.01\\
88.58	0.01\\
88.59	0.01\\
88.6	0.01\\
88.61	0.01\\
88.62	0.01\\
88.63	0.01\\
88.64	0.01\\
88.65	0.01\\
88.66	0.01\\
88.67	0.01\\
88.68	0.01\\
88.69	0.01\\
88.7	0.01\\
88.71	0.01\\
88.72	0.01\\
88.73	0.01\\
88.74	0.01\\
88.75	0.01\\
88.76	0.01\\
88.77	0.01\\
88.78	0.01\\
88.79	0.01\\
88.8	0.01\\
88.81	0.01\\
88.82	0.01\\
88.83	0.01\\
88.84	0.01\\
88.85	0.01\\
88.86	0.01\\
88.87	0.01\\
88.88	0.01\\
88.89	0.01\\
88.9	0.01\\
88.91	0.01\\
88.92	0.01\\
88.93	0.01\\
88.94	0.01\\
88.95	0.01\\
88.96	0.01\\
88.97	0.01\\
88.98	0.01\\
88.99	0.01\\
89	0.01\\
89.01	0.01\\
89.02	0.01\\
89.03	0.01\\
89.04	0.01\\
89.05	0.01\\
89.06	0.01\\
89.07	0.01\\
89.08	0.01\\
89.09	0.01\\
89.1	0.01\\
89.11	0.01\\
89.12	0.01\\
89.13	0.01\\
89.14	0.01\\
89.15	0.01\\
89.16	0.01\\
89.17	0.01\\
89.18	0.01\\
89.19	0.01\\
89.2	0.01\\
89.21	0.01\\
89.22	0.01\\
89.23	0.01\\
89.24	0.01\\
89.25	0.01\\
89.26	0.01\\
89.27	0.01\\
89.28	0.01\\
89.29	0.01\\
89.3	0.01\\
89.31	0.01\\
89.32	0.01\\
89.33	0.01\\
89.34	0.01\\
89.35	0.01\\
89.36	0.01\\
89.37	0.01\\
89.38	0.01\\
89.39	0.01\\
89.4	0.01\\
89.41	0.01\\
89.42	0.01\\
89.43	0.01\\
89.44	0.01\\
89.45	0.01\\
89.46	0.01\\
89.47	0.01\\
89.48	0.01\\
89.49	0.01\\
89.5	0.01\\
89.51	0.01\\
89.52	0.01\\
89.53	0.01\\
89.54	0.01\\
89.55	0.01\\
89.56	0.01\\
89.57	0.01\\
89.58	0.01\\
89.59	0.01\\
89.6	0.01\\
89.61	0.01\\
89.62	0.01\\
89.63	0.01\\
89.64	0.01\\
89.65	0.01\\
89.66	0.01\\
89.67	0.01\\
89.68	0.01\\
89.69	0.01\\
89.7	0.01\\
89.71	0.01\\
89.72	0.01\\
89.73	0.01\\
89.74	0.01\\
89.75	0.01\\
89.76	0.01\\
89.77	0.01\\
89.78	0.01\\
89.79	0.01\\
89.8	0.01\\
89.81	0.01\\
89.82	0.01\\
89.83	0.01\\
89.84	0.01\\
89.85	0.01\\
89.86	0.01\\
89.87	0.01\\
89.88	0.01\\
89.89	0.01\\
89.9	0.01\\
89.91	0.01\\
89.92	0.01\\
89.93	0.01\\
89.94	0.01\\
89.95	0.01\\
89.96	0.01\\
89.97	0.01\\
89.98	0.01\\
89.99	0.01\\
90	0.01\\
90.01	0.01\\
90.02	0.01\\
90.03	0.01\\
90.04	0.01\\
90.05	0.01\\
90.06	0.01\\
90.07	0.01\\
90.08	0.01\\
90.09	0.01\\
90.1	0.01\\
90.11	0.01\\
90.12	0.01\\
90.13	0.01\\
90.14	0.01\\
90.15	0.01\\
90.16	0.01\\
90.17	0.01\\
90.18	0.01\\
90.19	0.01\\
90.2	0.01\\
90.21	0.01\\
90.22	0.01\\
90.23	0.01\\
90.24	0.01\\
90.25	0.01\\
90.26	0.01\\
90.27	0.01\\
90.28	0.01\\
90.29	0.01\\
90.3	0.01\\
90.31	0.01\\
90.32	0.01\\
90.33	0.01\\
90.34	0.01\\
90.35	0.01\\
90.36	0.01\\
90.37	0.01\\
90.38	0.01\\
90.39	0.01\\
90.4	0.01\\
90.41	0.01\\
90.42	0.01\\
90.43	0.01\\
90.44	0.01\\
90.45	0.01\\
90.46	0.01\\
90.47	0.01\\
90.48	0.01\\
90.49	0.01\\
90.5	0.01\\
90.51	0.01\\
90.52	0.01\\
90.53	0.01\\
90.54	0.01\\
90.55	0.01\\
90.56	0.01\\
90.57	0.01\\
90.58	0.01\\
90.59	0.01\\
90.6	0.01\\
90.61	0.01\\
90.62	0.01\\
90.63	0.01\\
90.64	0.01\\
90.65	0.01\\
90.66	0.01\\
90.67	0.01\\
90.68	0.01\\
90.69	0.01\\
90.7	0.01\\
90.71	0.01\\
90.72	0.01\\
90.73	0.01\\
90.74	0.01\\
90.75	0.01\\
90.76	0.01\\
90.77	0.01\\
90.78	0.01\\
90.79	0.01\\
90.8	0.01\\
90.81	0.01\\
90.82	0.01\\
90.83	0.01\\
90.84	0.01\\
90.85	0.01\\
90.86	0.01\\
90.87	0.01\\
90.88	0.01\\
90.89	0.01\\
90.9	0.01\\
90.91	0.01\\
90.92	0.01\\
90.93	0.01\\
90.94	0.01\\
90.95	0.01\\
90.96	0.01\\
90.97	0.01\\
90.98	0.01\\
90.99	0.01\\
91	0.01\\
91.01	0.01\\
91.02	0.01\\
91.03	0.01\\
91.04	0.01\\
91.05	0.01\\
91.06	0.01\\
91.07	0.01\\
91.08	0.01\\
91.09	0.01\\
91.1	0.01\\
91.11	0.01\\
91.12	0.01\\
91.13	0.01\\
91.14	0.01\\
91.15	0.01\\
91.16	0.01\\
91.17	0.01\\
91.18	0.01\\
91.19	0.01\\
91.2	0.01\\
91.21	0.01\\
91.22	0.01\\
91.23	0.01\\
91.24	0.01\\
91.25	0.01\\
91.26	0.01\\
91.27	0.01\\
91.28	0.01\\
91.29	0.01\\
91.3	0.01\\
91.31	0.01\\
91.32	0.01\\
91.33	0.01\\
91.34	0.01\\
91.35	0.01\\
91.36	0.01\\
91.37	0.01\\
91.38	0.01\\
91.39	0.01\\
91.4	0.01\\
91.41	0.01\\
91.42	0.01\\
91.43	0.01\\
91.44	0.01\\
91.45	0.01\\
91.46	0.01\\
91.47	0.01\\
91.48	0.01\\
91.49	0.01\\
91.5	0.01\\
91.51	0.01\\
91.52	0.01\\
91.53	0.01\\
91.54	0.01\\
91.55	0.01\\
91.56	0.01\\
91.57	0.01\\
91.58	0.01\\
91.59	0.01\\
91.6	0.01\\
91.61	0.01\\
91.62	0.01\\
91.63	0.01\\
91.64	0.01\\
91.65	0.01\\
91.66	0.01\\
91.67	0.01\\
91.68	0.01\\
91.69	0.01\\
91.7	0.01\\
91.71	0.01\\
91.72	0.01\\
91.73	0.01\\
91.74	0.01\\
91.75	0.01\\
91.76	0.01\\
91.77	0.01\\
91.78	0.01\\
91.79	0.01\\
91.8	0.01\\
91.81	0.01\\
91.82	0.01\\
91.83	0.01\\
91.84	0.01\\
91.85	0.01\\
91.86	0.01\\
91.87	0.01\\
91.88	0.01\\
91.89	0.01\\
91.9	0.01\\
91.91	0.01\\
91.92	0.01\\
91.93	0.01\\
91.94	0.01\\
91.95	0.01\\
91.96	0.01\\
91.97	0.01\\
91.98	0.01\\
91.99	0.01\\
92	0.01\\
92.01	0.01\\
92.02	0.01\\
92.03	0.01\\
92.04	0.01\\
92.05	0.01\\
92.06	0.01\\
92.07	0.01\\
92.08	0.01\\
92.09	0.01\\
92.1	0.01\\
92.11	0.01\\
92.12	0.01\\
92.13	0.01\\
92.14	0.01\\
92.15	0.01\\
92.16	0.01\\
92.17	0.01\\
92.18	0.01\\
92.19	0.01\\
92.2	0.01\\
92.21	0.01\\
92.22	0.01\\
92.23	0.01\\
92.24	0.01\\
92.25	0.01\\
92.26	0.01\\
92.27	0.01\\
92.28	0.01\\
92.29	0.01\\
92.3	0.01\\
92.31	0.01\\
92.32	0.01\\
92.33	0.01\\
92.34	0.01\\
92.35	0.01\\
92.36	0.01\\
92.37	0.01\\
92.38	0.01\\
92.39	0.01\\
92.4	0.01\\
92.41	0.01\\
92.42	0.01\\
92.43	0.01\\
92.44	0.01\\
92.45	0.01\\
92.46	0.01\\
92.47	0.01\\
92.48	0.01\\
92.49	0.01\\
92.5	0.01\\
92.51	0.01\\
92.52	0.01\\
92.53	0.01\\
92.54	0.01\\
92.55	0.01\\
92.56	0.01\\
92.57	0.01\\
92.58	0.01\\
92.59	0.01\\
92.6	0.01\\
92.61	0.01\\
92.62	0.01\\
92.63	0.01\\
92.64	0.01\\
92.65	0.01\\
92.66	0.01\\
92.67	0.01\\
92.68	0.01\\
92.69	0.01\\
92.7	0.01\\
92.71	0.01\\
92.72	0.01\\
92.73	0.01\\
92.74	0.01\\
92.75	0.01\\
92.76	0.01\\
92.77	0.01\\
92.78	0.01\\
92.79	0.01\\
92.8	0.01\\
92.81	0.01\\
92.82	0.01\\
92.83	0.01\\
92.84	0.01\\
92.85	0.01\\
92.86	0.01\\
92.87	0.01\\
92.88	0.01\\
92.89	0.01\\
92.9	0.01\\
92.91	0.01\\
92.92	0.01\\
92.93	0.01\\
92.94	0.01\\
92.95	0.01\\
92.96	0.01\\
92.97	0.01\\
92.98	0.01\\
92.99	0.01\\
93	0.01\\
93.01	0.01\\
93.02	0.01\\
93.03	0.01\\
93.04	0.01\\
93.05	0.01\\
93.06	0.01\\
93.07	0.01\\
93.08	0.01\\
93.09	0.01\\
93.1	0.01\\
93.11	0.01\\
93.12	0.01\\
93.13	0.01\\
93.14	0.01\\
93.15	0.01\\
93.16	0.01\\
93.17	0.01\\
93.18	0.01\\
93.19	0.01\\
93.2	0.01\\
93.21	0.01\\
93.22	0.01\\
93.23	0.01\\
93.24	0.01\\
93.25	0.01\\
93.26	0.01\\
93.27	0.01\\
93.28	0.01\\
93.29	0.01\\
93.3	0.01\\
93.31	0.01\\
93.32	0.01\\
93.33	0.01\\
93.34	0.01\\
93.35	0.01\\
93.36	0.01\\
93.37	0.01\\
93.38	0.01\\
93.39	0.01\\
93.4	0.01\\
93.41	0.01\\
93.42	0.01\\
93.43	0.01\\
93.44	0.01\\
93.45	0.01\\
93.46	0.01\\
93.47	0.01\\
93.48	0.01\\
93.49	0.01\\
93.5	0.01\\
93.51	0.01\\
93.52	0.01\\
93.53	0.01\\
93.54	0.01\\
93.55	0.01\\
93.56	0.01\\
93.57	0.01\\
93.58	0.01\\
93.59	0.01\\
93.6	0.01\\
93.61	0.01\\
93.62	0.01\\
93.63	0.01\\
93.64	0.01\\
93.65	0.01\\
93.66	0.01\\
93.67	0.01\\
93.68	0.01\\
93.69	0.01\\
93.7	0.01\\
93.71	0.01\\
93.72	0.01\\
93.73	0.01\\
93.74	0.01\\
93.75	0.01\\
93.76	0.01\\
93.77	0.01\\
93.78	0.01\\
93.79	0.01\\
93.8	0.01\\
93.81	0.01\\
93.82	0.01\\
93.83	0.01\\
93.84	0.01\\
93.85	0.01\\
93.86	0.01\\
93.87	0.01\\
93.88	0.01\\
93.89	0.01\\
93.9	0.01\\
93.91	0.01\\
93.92	0.01\\
93.93	0.01\\
93.94	0.01\\
93.95	0.01\\
93.96	0.01\\
93.97	0.01\\
93.98	0.01\\
93.99	0.01\\
94	0.01\\
94.01	0.01\\
94.02	0.01\\
94.03	0.01\\
94.04	0.01\\
94.05	0.01\\
94.06	0.01\\
94.07	0.01\\
94.08	0.01\\
94.09	0.01\\
94.1	0.01\\
94.11	0.01\\
94.12	0.01\\
94.13	0.01\\
94.14	0.01\\
94.15	0.01\\
94.16	0.01\\
94.17	0.01\\
94.18	0.01\\
94.19	0.01\\
94.2	0.01\\
94.21	0.01\\
94.22	0.01\\
94.23	0.01\\
94.24	0.01\\
94.25	0.01\\
94.26	0.01\\
94.27	0.01\\
94.28	0.01\\
94.29	0.01\\
94.3	0.01\\
94.31	0.01\\
94.32	0.01\\
94.33	0.01\\
94.34	0.01\\
94.35	0.01\\
94.36	0.01\\
94.37	0.01\\
94.38	0.01\\
94.39	0.01\\
94.4	0.01\\
94.41	0.01\\
94.42	0.01\\
94.43	0.01\\
94.44	0.01\\
94.45	0.01\\
94.46	0.01\\
94.47	0.01\\
94.48	0.01\\
94.49	0.01\\
94.5	0.01\\
94.51	0.01\\
94.52	0.01\\
94.53	0.01\\
94.54	0.01\\
94.55	0.01\\
94.56	0.01\\
94.57	0.01\\
94.58	0.01\\
94.59	0.01\\
94.6	0.01\\
94.61	0.01\\
94.62	0.01\\
94.63	0.01\\
94.64	0.01\\
94.65	0.01\\
94.66	0.01\\
94.67	0.01\\
94.68	0.01\\
94.69	0.01\\
94.7	0.01\\
94.71	0.01\\
94.72	0.01\\
94.73	0.01\\
94.74	0.01\\
94.75	0.01\\
94.76	0.01\\
94.77	0.01\\
94.78	0.01\\
94.79	0.01\\
94.8	0.01\\
94.81	0.01\\
94.82	0.01\\
94.83	0.01\\
94.84	0.01\\
94.85	0.01\\
94.86	0.01\\
94.87	0.01\\
94.88	0.01\\
94.89	0.01\\
94.9	0.01\\
94.91	0.01\\
94.92	0.01\\
94.93	0.01\\
94.94	0.01\\
94.95	0.01\\
94.96	0.01\\
94.97	0.01\\
94.98	0.01\\
94.99	0.01\\
95	0.01\\
95.01	0.01\\
95.02	0.01\\
95.03	0.01\\
95.04	0.01\\
95.05	0.01\\
95.06	0.01\\
95.07	0.01\\
95.08	0.01\\
95.09	0.01\\
95.1	0.01\\
95.11	0.01\\
95.12	0.01\\
95.13	0.01\\
95.14	0.01\\
95.15	0.01\\
95.16	0.01\\
95.17	0.01\\
95.18	0.01\\
95.19	0.01\\
95.2	0.01\\
95.21	0.01\\
95.22	0.01\\
95.23	0.01\\
95.24	0.01\\
95.25	0.01\\
95.26	0.01\\
95.27	0.01\\
95.28	0.01\\
95.29	0.01\\
95.3	0.01\\
95.31	0.01\\
95.32	0.01\\
95.33	0.01\\
95.34	0.01\\
95.35	0.01\\
95.36	0.01\\
95.37	0.01\\
95.38	0.01\\
95.39	0.01\\
95.4	0.01\\
95.41	0.01\\
95.42	0.01\\
95.43	0.01\\
95.44	0.01\\
95.45	0.01\\
95.46	0.01\\
95.47	0.01\\
95.48	0.01\\
95.49	0.01\\
95.5	0.01\\
95.51	0.01\\
95.52	0.01\\
95.53	0.01\\
95.54	0.01\\
95.55	0.01\\
95.56	0.01\\
95.57	0.01\\
95.58	0.01\\
95.59	0.01\\
95.6	0.01\\
95.61	0.01\\
95.62	0.01\\
95.63	0.01\\
95.64	0.01\\
95.65	0.01\\
95.66	0.01\\
95.67	0.01\\
95.68	0.01\\
95.69	0.01\\
95.7	0.01\\
95.71	0.01\\
95.72	0.01\\
95.73	0.01\\
95.74	0.01\\
95.75	0.01\\
95.76	0.01\\
95.77	0.01\\
95.78	0.01\\
95.79	0.01\\
95.8	0.01\\
95.81	0.01\\
95.82	0.01\\
95.83	0.01\\
95.84	0.01\\
95.85	0.01\\
95.86	0.01\\
95.87	0.01\\
95.88	0.01\\
95.89	0.01\\
95.9	0.01\\
95.91	0.01\\
95.92	0.01\\
95.93	0.01\\
95.94	0.01\\
95.95	0.01\\
95.96	0.01\\
95.97	0.01\\
95.98	0.01\\
95.99	0.01\\
96	0.01\\
96.01	0.01\\
96.02	0.01\\
96.03	0.01\\
96.04	0.01\\
96.05	0.01\\
96.06	0.01\\
96.07	0.01\\
96.08	0.01\\
96.09	0.01\\
96.1	0.01\\
96.11	0.01\\
96.12	0.01\\
96.13	0.01\\
96.14	0.01\\
96.15	0.01\\
96.16	0.01\\
96.17	0.01\\
96.18	0.01\\
96.19	0.01\\
96.2	0.01\\
96.21	0.01\\
96.22	0.01\\
96.23	0.01\\
96.24	0.01\\
96.25	0.01\\
96.26	0.01\\
96.27	0.01\\
96.28	0.01\\
96.29	0.01\\
96.3	0.01\\
96.31	0.01\\
96.32	0.01\\
96.33	0.01\\
96.34	0.01\\
96.35	0.01\\
96.36	0.01\\
96.37	0.01\\
96.38	0.01\\
96.39	0.01\\
96.4	0.01\\
96.41	0.01\\
96.42	0.01\\
96.43	0.01\\
96.44	0.01\\
96.45	0.01\\
96.46	0.01\\
96.47	0.01\\
96.48	0.01\\
96.49	0.01\\
96.5	0.01\\
96.51	0.01\\
96.52	0.01\\
96.53	0.01\\
96.54	0.01\\
96.55	0.01\\
96.56	0.01\\
96.57	0.01\\
96.58	0.01\\
96.59	0.01\\
96.6	0.01\\
96.61	0.01\\
96.62	0.01\\
96.63	0.01\\
96.64	0.01\\
96.65	0.01\\
96.66	0.01\\
96.67	0.01\\
96.68	0.01\\
96.69	0.01\\
96.7	0.01\\
96.71	0.01\\
96.72	0.01\\
96.73	0.01\\
96.74	0.01\\
96.75	0.01\\
96.76	0.01\\
96.77	0.01\\
96.78	0.01\\
96.79	0.01\\
96.8	0.01\\
96.81	0.01\\
96.82	0.01\\
96.83	0.01\\
96.84	0.01\\
96.85	0.01\\
96.86	0.01\\
96.87	0.01\\
96.88	0.01\\
96.89	0.01\\
96.9	0.01\\
96.91	0.01\\
96.92	0.01\\
96.93	0.01\\
96.94	0.01\\
96.95	0.01\\
96.96	0.01\\
96.97	0.01\\
96.98	0.01\\
96.99	0.01\\
97	0.01\\
97.01	0.01\\
97.02	0.01\\
97.03	0.01\\
97.04	0.01\\
97.05	0.01\\
97.06	0.01\\
97.07	0.01\\
97.08	0.01\\
97.09	0.01\\
97.1	0.01\\
97.11	0.01\\
97.12	0.01\\
97.13	0.01\\
97.14	0.01\\
97.15	0.01\\
97.16	0.01\\
97.17	0.01\\
97.18	0.01\\
97.19	0.01\\
97.2	0.01\\
97.21	0.01\\
97.22	0.01\\
97.23	0.01\\
97.24	0.01\\
97.25	0.01\\
97.26	0.01\\
97.27	0.01\\
97.28	0.01\\
97.29	0.01\\
97.3	0.01\\
97.31	0.01\\
97.32	0.01\\
97.33	0.01\\
97.34	0.01\\
97.35	0.01\\
97.36	0.01\\
97.37	0.01\\
97.38	0.01\\
97.39	0.01\\
97.4	0.01\\
97.41	0.01\\
97.42	0.01\\
97.43	0.01\\
97.44	0.01\\
97.45	0.01\\
97.46	0.01\\
97.47	0.01\\
97.48	0.01\\
97.49	0.01\\
97.5	0.01\\
97.51	0.01\\
97.52	0.01\\
97.53	0.01\\
97.54	0.01\\
97.55	0.01\\
97.56	0.01\\
97.57	0.01\\
97.58	0.01\\
97.59	0.01\\
97.6	0.01\\
97.61	0.01\\
97.62	0.01\\
97.63	0.01\\
97.64	0.01\\
97.65	0.01\\
97.66	0.01\\
97.67	0.01\\
97.68	0.01\\
97.69	0.01\\
97.7	0.01\\
97.71	0.01\\
97.72	0.01\\
97.73	0.01\\
97.74	0.01\\
97.75	0.01\\
97.76	0.01\\
97.77	0.01\\
97.78	0.01\\
97.79	0.01\\
97.8	0.01\\
97.81	0.01\\
97.82	0.01\\
97.83	0.01\\
97.84	0.01\\
97.85	0.01\\
97.86	0.01\\
97.87	0.01\\
97.88	0.01\\
97.89	0.01\\
97.9	0.01\\
97.91	0.01\\
97.92	0.01\\
97.93	0.01\\
97.94	0.01\\
97.95	0.01\\
97.96	0.01\\
97.97	0.01\\
97.98	0.01\\
97.99	0.01\\
98	0.01\\
98.01	0.01\\
98.02	0.01\\
98.03	0.01\\
98.04	0.01\\
98.05	0.01\\
98.06	0.01\\
98.07	0.01\\
98.08	0.01\\
98.09	0.01\\
98.1	0.01\\
98.11	0.01\\
98.12	0.01\\
98.13	0.01\\
98.14	0.01\\
98.15	0.01\\
98.16	0.01\\
98.17	0.01\\
98.18	0.01\\
98.19	0.01\\
98.2	0.01\\
98.21	0.01\\
98.22	0.01\\
98.23	0.01\\
98.24	0.01\\
98.25	0.01\\
98.26	0.01\\
98.27	0.01\\
98.28	0.01\\
98.29	0.01\\
98.3	0.01\\
98.31	0.01\\
98.32	0.01\\
98.33	0.01\\
98.34	0.01\\
98.35	0.01\\
98.36	0.01\\
98.37	0.01\\
98.38	0.01\\
98.39	0.01\\
98.4	0.01\\
98.41	0.01\\
98.42	0.01\\
98.43	0.01\\
98.44	0.01\\
98.45	0.01\\
98.46	0.01\\
98.47	0.01\\
98.48	0.01\\
98.49	0.01\\
98.5	0.01\\
98.51	0.01\\
98.52	0.01\\
98.53	0.01\\
98.54	0.01\\
98.55	0.01\\
98.56	0.01\\
98.57	0.01\\
98.58	0.01\\
98.59	0.01\\
98.6	0.01\\
98.61	0.01\\
98.62	0.01\\
98.63	0.01\\
98.64	0.01\\
98.65	0.01\\
98.66	0.01\\
98.67	0.01\\
98.68	0.01\\
98.69	0.01\\
98.7	0.01\\
98.71	0.01\\
98.72	0.01\\
98.73	0.01\\
98.74	0.01\\
98.75	0.01\\
98.76	0.01\\
98.77	0.01\\
98.78	0.01\\
98.79	0.01\\
98.8	0.01\\
98.81	0.01\\
98.82	0.01\\
98.83	0.01\\
98.84	0.01\\
98.85	0.01\\
98.86	0.01\\
98.87	0.01\\
98.88	0.01\\
98.89	0.01\\
98.9	0.01\\
98.91	0.01\\
98.92	0.01\\
98.93	0.01\\
98.94	0.01\\
98.95	0.01\\
98.96	0.01\\
98.97	0.01\\
98.98	0.01\\
98.99	0.01\\
99	0.01\\
99.01	0.01\\
99.02	0.01\\
99.03	0.01\\
99.04	0.01\\
99.05	0.01\\
99.06	0.01\\
99.07	0.01\\
99.08	0.01\\
99.09	0.01\\
99.1	0.01\\
99.11	0.01\\
99.12	0.01\\
99.13	0.01\\
99.14	0.01\\
99.15	0.01\\
99.16	0.01\\
99.17	0.01\\
99.18	0.01\\
99.19	0.01\\
99.2	0.01\\
99.21	0.01\\
99.22	0.01\\
99.23	0.01\\
99.24	0.01\\
99.25	0.01\\
99.26	0.01\\
99.27	0.01\\
99.28	0.01\\
99.29	0.01\\
99.3	0.01\\
99.31	0.01\\
99.32	0.01\\
99.33	0.01\\
99.34	0.01\\
99.35	0.01\\
99.36	0.01\\
99.37	0.01\\
99.38	0.01\\
99.39	0.01\\
99.4	0.01\\
99.41	0.01\\
99.42	0.01\\
99.43	0.01\\
99.44	0.01\\
99.45	0.01\\
99.46	0.01\\
99.47	0.01\\
99.48	0.01\\
99.49	0.01\\
99.5	0.01\\
99.51	0.01\\
99.52	0.01\\
99.53	0.01\\
99.54	0.01\\
99.55	0.01\\
99.56	0.01\\
99.57	0.01\\
99.58	0.01\\
99.59	0.01\\
99.6	0.01\\
99.61	0.01\\
99.62	0.01\\
99.63	0.01\\
99.64	0.01\\
99.65	0.01\\
99.66	0.01\\
99.67	0.01\\
99.68	0.01\\
99.69	0.01\\
99.7	0.01\\
99.71	0.01\\
99.72	0.01\\
99.73	0.01\\
99.74	0.01\\
99.75	0.01\\
99.76	0.01\\
99.77	0.01\\
99.78	0.01\\
99.79	0.01\\
99.8	0.01\\
99.81	0.01\\
99.82	0.01\\
99.83	0.01\\
99.84	0.01\\
99.85	0.01\\
99.86	0.01\\
99.87	0.01\\
99.88	0.01\\
99.89	0.01\\
99.9	0.01\\
99.91	0.01\\
99.92	0.01\\
99.93	0.01\\
99.94	0.01\\
99.95	0.01\\
99.96	0.01\\
99.97	0.01\\
99.98	0.01\\
99.99	0.01\\
100	0.01\\
};
\addlegendentry{$q=1$};

\addplot [color=red,solid,forget plot]
  table[row sep=crcr]{%
0.01	0.01\\
0.02	0.01\\
0.03	0.01\\
0.04	0.01\\
0.05	0.01\\
0.06	0.01\\
0.07	0.01\\
0.08	0.01\\
0.09	0.01\\
0.1	0.01\\
0.11	0.01\\
0.12	0.01\\
0.13	0.01\\
0.14	0.01\\
0.15	0.01\\
0.16	0.01\\
0.17	0.01\\
0.18	0.01\\
0.19	0.01\\
0.2	0.01\\
0.21	0.01\\
0.22	0.01\\
0.23	0.01\\
0.24	0.01\\
0.25	0.01\\
0.26	0.01\\
0.27	0.01\\
0.28	0.01\\
0.29	0.01\\
0.3	0.01\\
0.31	0.01\\
0.32	0.01\\
0.33	0.01\\
0.34	0.01\\
0.35	0.01\\
0.36	0.01\\
0.37	0.01\\
0.38	0.01\\
0.39	0.01\\
0.4	0.01\\
0.41	0.01\\
0.42	0.01\\
0.43	0.01\\
0.44	0.01\\
0.45	0.01\\
0.46	0.01\\
0.47	0.01\\
0.48	0.01\\
0.49	0.01\\
0.5	0.01\\
0.51	0.01\\
0.52	0.01\\
0.53	0.01\\
0.54	0.01\\
0.55	0.01\\
0.56	0.01\\
0.57	0.01\\
0.58	0.01\\
0.59	0.01\\
0.6	0.01\\
0.61	0.01\\
0.62	0.01\\
0.63	0.01\\
0.64	0.01\\
0.65	0.01\\
0.66	0.01\\
0.67	0.01\\
0.68	0.01\\
0.69	0.01\\
0.7	0.01\\
0.71	0.01\\
0.72	0.01\\
0.73	0.01\\
0.74	0.01\\
0.75	0.01\\
0.76	0.01\\
0.77	0.01\\
0.78	0.01\\
0.79	0.01\\
0.8	0.01\\
0.81	0.01\\
0.82	0.01\\
0.83	0.01\\
0.84	0.01\\
0.85	0.01\\
0.86	0.01\\
0.87	0.01\\
0.88	0.01\\
0.89	0.01\\
0.9	0.01\\
0.91	0.01\\
0.92	0.01\\
0.93	0.01\\
0.94	0.01\\
0.95	0.01\\
0.96	0.01\\
0.97	0.01\\
0.98	0.01\\
0.99	0.01\\
1	0.01\\
1.01	0.01\\
1.02	0.01\\
1.03	0.01\\
1.04	0.01\\
1.05	0.01\\
1.06	0.01\\
1.07	0.01\\
1.08	0.01\\
1.09	0.01\\
1.1	0.01\\
1.11	0.01\\
1.12	0.01\\
1.13	0.01\\
1.14	0.01\\
1.15	0.01\\
1.16	0.01\\
1.17	0.01\\
1.18	0.01\\
1.19	0.01\\
1.2	0.01\\
1.21	0.01\\
1.22	0.01\\
1.23	0.01\\
1.24	0.01\\
1.25	0.01\\
1.26	0.01\\
1.27	0.01\\
1.28	0.01\\
1.29	0.01\\
1.3	0.01\\
1.31	0.01\\
1.32	0.01\\
1.33	0.01\\
1.34	0.01\\
1.35	0.01\\
1.36	0.01\\
1.37	0.01\\
1.38	0.01\\
1.39	0.01\\
1.4	0.01\\
1.41	0.01\\
1.42	0.01\\
1.43	0.01\\
1.44	0.01\\
1.45	0.01\\
1.46	0.01\\
1.47	0.01\\
1.48	0.01\\
1.49	0.01\\
1.5	0.01\\
1.51	0.01\\
1.52	0.01\\
1.53	0.01\\
1.54	0.01\\
1.55	0.01\\
1.56	0.01\\
1.57	0.01\\
1.58	0.01\\
1.59	0.01\\
1.6	0.01\\
1.61	0.01\\
1.62	0.01\\
1.63	0.01\\
1.64	0.01\\
1.65	0.01\\
1.66	0.01\\
1.67	0.01\\
1.68	0.01\\
1.69	0.01\\
1.7	0.01\\
1.71	0.01\\
1.72	0.01\\
1.73	0.01\\
1.74	0.01\\
1.75	0.01\\
1.76	0.01\\
1.77	0.01\\
1.78	0.01\\
1.79	0.01\\
1.8	0.01\\
1.81	0.01\\
1.82	0.01\\
1.83	0.01\\
1.84	0.01\\
1.85	0.01\\
1.86	0.01\\
1.87	0.01\\
1.88	0.01\\
1.89	0.01\\
1.9	0.01\\
1.91	0.01\\
1.92	0.01\\
1.93	0.01\\
1.94	0.01\\
1.95	0.01\\
1.96	0.01\\
1.97	0.01\\
1.98	0.01\\
1.99	0.01\\
2	0.01\\
2.01	0.01\\
2.02	0.01\\
2.03	0.01\\
2.04	0.01\\
2.05	0.01\\
2.06	0.01\\
2.07	0.01\\
2.08	0.01\\
2.09	0.01\\
2.1	0.01\\
2.11	0.01\\
2.12	0.01\\
2.13	0.01\\
2.14	0.01\\
2.15	0.01\\
2.16	0.01\\
2.17	0.01\\
2.18	0.01\\
2.19	0.01\\
2.2	0.01\\
2.21	0.01\\
2.22	0.01\\
2.23	0.01\\
2.24	0.01\\
2.25	0.01\\
2.26	0.01\\
2.27	0.01\\
2.28	0.01\\
2.29	0.01\\
2.3	0.01\\
2.31	0.01\\
2.32	0.01\\
2.33	0.01\\
2.34	0.01\\
2.35	0.01\\
2.36	0.01\\
2.37	0.01\\
2.38	0.01\\
2.39	0.01\\
2.4	0.01\\
2.41	0.01\\
2.42	0.01\\
2.43	0.01\\
2.44	0.01\\
2.45	0.01\\
2.46	0.01\\
2.47	0.01\\
2.48	0.01\\
2.49	0.01\\
2.5	0.01\\
2.51	0.01\\
2.52	0.01\\
2.53	0.01\\
2.54	0.01\\
2.55	0.01\\
2.56	0.01\\
2.57	0.01\\
2.58	0.01\\
2.59	0.01\\
2.6	0.01\\
2.61	0.01\\
2.62	0.01\\
2.63	0.01\\
2.64	0.01\\
2.65	0.01\\
2.66	0.01\\
2.67	0.01\\
2.68	0.01\\
2.69	0.01\\
2.7	0.01\\
2.71	0.01\\
2.72	0.01\\
2.73	0.01\\
2.74	0.01\\
2.75	0.01\\
2.76	0.01\\
2.77	0.01\\
2.78	0.01\\
2.79	0.01\\
2.8	0.01\\
2.81	0.01\\
2.82	0.01\\
2.83	0.01\\
2.84	0.01\\
2.85	0.01\\
2.86	0.01\\
2.87	0.01\\
2.88	0.01\\
2.89	0.01\\
2.9	0.01\\
2.91	0.01\\
2.92	0.01\\
2.93	0.01\\
2.94	0.01\\
2.95	0.01\\
2.96	0.01\\
2.97	0.01\\
2.98	0.01\\
2.99	0.01\\
3	0.01\\
3.01	0.01\\
3.02	0.01\\
3.03	0.01\\
3.04	0.01\\
3.05	0.01\\
3.06	0.01\\
3.07	0.01\\
3.08	0.01\\
3.09	0.01\\
3.1	0.01\\
3.11	0.01\\
3.12	0.01\\
3.13	0.01\\
3.14	0.01\\
3.15	0.01\\
3.16	0.01\\
3.17	0.01\\
3.18	0.01\\
3.19	0.01\\
3.2	0.01\\
3.21	0.01\\
3.22	0.01\\
3.23	0.01\\
3.24	0.01\\
3.25	0.01\\
3.26	0.01\\
3.27	0.01\\
3.28	0.01\\
3.29	0.01\\
3.3	0.01\\
3.31	0.01\\
3.32	0.01\\
3.33	0.01\\
3.34	0.01\\
3.35	0.01\\
3.36	0.01\\
3.37	0.01\\
3.38	0.01\\
3.39	0.01\\
3.4	0.01\\
3.41	0.01\\
3.42	0.01\\
3.43	0.01\\
3.44	0.01\\
3.45	0.01\\
3.46	0.01\\
3.47	0.01\\
3.48	0.01\\
3.49	0.01\\
3.5	0.01\\
3.51	0.01\\
3.52	0.01\\
3.53	0.01\\
3.54	0.01\\
3.55	0.01\\
3.56	0.01\\
3.57	0.01\\
3.58	0.01\\
3.59	0.01\\
3.6	0.01\\
3.61	0.01\\
3.62	0.01\\
3.63	0.01\\
3.64	0.01\\
3.65	0.01\\
3.66	0.01\\
3.67	0.01\\
3.68	0.01\\
3.69	0.01\\
3.7	0.01\\
3.71	0.01\\
3.72	0.01\\
3.73	0.01\\
3.74	0.01\\
3.75	0.01\\
3.76	0.01\\
3.77	0.01\\
3.78	0.01\\
3.79	0.01\\
3.8	0.01\\
3.81	0.01\\
3.82	0.01\\
3.83	0.01\\
3.84	0.01\\
3.85	0.01\\
3.86	0.01\\
3.87	0.01\\
3.88	0.01\\
3.89	0.01\\
3.9	0.01\\
3.91	0.01\\
3.92	0.01\\
3.93	0.01\\
3.94	0.01\\
3.95	0.01\\
3.96	0.01\\
3.97	0.01\\
3.98	0.01\\
3.99	0.01\\
4	0.01\\
4.01	0.01\\
4.02	0.01\\
4.03	0.01\\
4.04	0.01\\
4.05	0.01\\
4.06	0.01\\
4.07	0.01\\
4.08	0.01\\
4.09	0.01\\
4.1	0.01\\
4.11	0.01\\
4.12	0.01\\
4.13	0.01\\
4.14	0.01\\
4.15	0.01\\
4.16	0.01\\
4.17	0.01\\
4.18	0.01\\
4.19	0.01\\
4.2	0.01\\
4.21	0.01\\
4.22	0.01\\
4.23	0.01\\
4.24	0.01\\
4.25	0.01\\
4.26	0.01\\
4.27	0.01\\
4.28	0.01\\
4.29	0.01\\
4.3	0.01\\
4.31	0.01\\
4.32	0.01\\
4.33	0.01\\
4.34	0.01\\
4.35	0.01\\
4.36	0.01\\
4.37	0.01\\
4.38	0.01\\
4.39	0.01\\
4.4	0.01\\
4.41	0.01\\
4.42	0.01\\
4.43	0.01\\
4.44	0.01\\
4.45	0.01\\
4.46	0.01\\
4.47	0.01\\
4.48	0.01\\
4.49	0.01\\
4.5	0.01\\
4.51	0.01\\
4.52	0.01\\
4.53	0.01\\
4.54	0.01\\
4.55	0.01\\
4.56	0.01\\
4.57	0.01\\
4.58	0.01\\
4.59	0.01\\
4.6	0.01\\
4.61	0.01\\
4.62	0.01\\
4.63	0.01\\
4.64	0.01\\
4.65	0.01\\
4.66	0.01\\
4.67	0.01\\
4.68	0.01\\
4.69	0.01\\
4.7	0.01\\
4.71	0.01\\
4.72	0.01\\
4.73	0.01\\
4.74	0.01\\
4.75	0.01\\
4.76	0.01\\
4.77	0.01\\
4.78	0.01\\
4.79	0.01\\
4.8	0.01\\
4.81	0.01\\
4.82	0.01\\
4.83	0.01\\
4.84	0.01\\
4.85	0.01\\
4.86	0.01\\
4.87	0.01\\
4.88	0.01\\
4.89	0.01\\
4.9	0.01\\
4.91	0.01\\
4.92	0.01\\
4.93	0.01\\
4.94	0.01\\
4.95	0.01\\
4.96	0.01\\
4.97	0.01\\
4.98	0.01\\
4.99	0.01\\
5	0.01\\
5.01	0.01\\
5.02	0.01\\
5.03	0.01\\
5.04	0.01\\
5.05	0.01\\
5.06	0.01\\
5.07	0.01\\
5.08	0.01\\
5.09	0.01\\
5.1	0.01\\
5.11	0.01\\
5.12	0.01\\
5.13	0.01\\
5.14	0.01\\
5.15	0.01\\
5.16	0.01\\
5.17	0.01\\
5.18	0.01\\
5.19	0.01\\
5.2	0.01\\
5.21	0.01\\
5.22	0.01\\
5.23	0.01\\
5.24	0.01\\
5.25	0.01\\
5.26	0.01\\
5.27	0.01\\
5.28	0.01\\
5.29	0.01\\
5.3	0.01\\
5.31	0.01\\
5.32	0.01\\
5.33	0.01\\
5.34	0.01\\
5.35	0.01\\
5.36	0.01\\
5.37	0.01\\
5.38	0.01\\
5.39	0.01\\
5.4	0.01\\
5.41	0.01\\
5.42	0.01\\
5.43	0.01\\
5.44	0.01\\
5.45	0.01\\
5.46	0.01\\
5.47	0.01\\
5.48	0.01\\
5.49	0.01\\
5.5	0.01\\
5.51	0.01\\
5.52	0.01\\
5.53	0.01\\
5.54	0.01\\
5.55	0.01\\
5.56	0.01\\
5.57	0.01\\
5.58	0.01\\
5.59	0.01\\
5.6	0.01\\
5.61	0.01\\
5.62	0.01\\
5.63	0.01\\
5.64	0.01\\
5.65	0.01\\
5.66	0.01\\
5.67	0.01\\
5.68	0.01\\
5.69	0.01\\
5.7	0.01\\
5.71	0.01\\
5.72	0.01\\
5.73	0.01\\
5.74	0.01\\
5.75	0.01\\
5.76	0.01\\
5.77	0.01\\
5.78	0.01\\
5.79	0.01\\
5.8	0.01\\
5.81	0.01\\
5.82	0.01\\
5.83	0.01\\
5.84	0.01\\
5.85	0.01\\
5.86	0.01\\
5.87	0.01\\
5.88	0.01\\
5.89	0.01\\
5.9	0.01\\
5.91	0.01\\
5.92	0.01\\
5.93	0.01\\
5.94	0.01\\
5.95	0.01\\
5.96	0.01\\
5.97	0.01\\
5.98	0.01\\
5.99	0.01\\
6	0.01\\
6.01	0.01\\
6.02	0.01\\
6.03	0.01\\
6.04	0.01\\
6.05	0.01\\
6.06	0.01\\
6.07	0.01\\
6.08	0.01\\
6.09	0.01\\
6.1	0.01\\
6.11	0.01\\
6.12	0.01\\
6.13	0.01\\
6.14	0.01\\
6.15	0.01\\
6.16	0.01\\
6.17	0.01\\
6.18	0.01\\
6.19	0.01\\
6.2	0.01\\
6.21	0.01\\
6.22	0.01\\
6.23	0.01\\
6.24	0.01\\
6.25	0.01\\
6.26	0.01\\
6.27	0.01\\
6.28	0.01\\
6.29	0.01\\
6.3	0.01\\
6.31	0.01\\
6.32	0.01\\
6.33	0.01\\
6.34	0.01\\
6.35	0.01\\
6.36	0.01\\
6.37	0.01\\
6.38	0.01\\
6.39	0.01\\
6.4	0.01\\
6.41	0.01\\
6.42	0.01\\
6.43	0.01\\
6.44	0.01\\
6.45	0.01\\
6.46	0.01\\
6.47	0.01\\
6.48	0.01\\
6.49	0.01\\
6.5	0.01\\
6.51	0.01\\
6.52	0.01\\
6.53	0.01\\
6.54	0.01\\
6.55	0.01\\
6.56	0.01\\
6.57	0.01\\
6.58	0.01\\
6.59	0.01\\
6.6	0.01\\
6.61	0.01\\
6.62	0.01\\
6.63	0.01\\
6.64	0.01\\
6.65	0.01\\
6.66	0.01\\
6.67	0.01\\
6.68	0.01\\
6.69	0.01\\
6.7	0.01\\
6.71	0.01\\
6.72	0.01\\
6.73	0.01\\
6.74	0.01\\
6.75	0.01\\
6.76	0.01\\
6.77	0.01\\
6.78	0.01\\
6.79	0.01\\
6.8	0.01\\
6.81	0.01\\
6.82	0.01\\
6.83	0.01\\
6.84	0.01\\
6.85	0.01\\
6.86	0.01\\
6.87	0.01\\
6.88	0.01\\
6.89	0.01\\
6.9	0.01\\
6.91	0.01\\
6.92	0.01\\
6.93	0.01\\
6.94	0.01\\
6.95	0.01\\
6.96	0.01\\
6.97	0.01\\
6.98	0.01\\
6.99	0.01\\
7	0.01\\
7.01	0.01\\
7.02	0.01\\
7.03	0.01\\
7.04	0.01\\
7.05	0.01\\
7.06	0.01\\
7.07	0.01\\
7.08	0.01\\
7.09	0.01\\
7.1	0.01\\
7.11	0.01\\
7.12	0.01\\
7.13	0.01\\
7.14	0.01\\
7.15	0.01\\
7.16	0.01\\
7.17	0.01\\
7.18	0.01\\
7.19	0.01\\
7.2	0.01\\
7.21	0.01\\
7.22	0.01\\
7.23	0.01\\
7.24	0.01\\
7.25	0.01\\
7.26	0.01\\
7.27	0.01\\
7.28	0.01\\
7.29	0.01\\
7.3	0.01\\
7.31	0.01\\
7.32	0.01\\
7.33	0.01\\
7.34	0.01\\
7.35	0.01\\
7.36	0.01\\
7.37	0.01\\
7.38	0.01\\
7.39	0.01\\
7.4	0.01\\
7.41	0.01\\
7.42	0.01\\
7.43	0.01\\
7.44	0.01\\
7.45	0.01\\
7.46	0.01\\
7.47	0.01\\
7.48	0.01\\
7.49	0.01\\
7.5	0.01\\
7.51	0.01\\
7.52	0.01\\
7.53	0.01\\
7.54	0.01\\
7.55	0.01\\
7.56	0.01\\
7.57	0.01\\
7.58	0.01\\
7.59	0.01\\
7.6	0.01\\
7.61	0.01\\
7.62	0.01\\
7.63	0.01\\
7.64	0.01\\
7.65	0.01\\
7.66	0.01\\
7.67	0.01\\
7.68	0.01\\
7.69	0.01\\
7.7	0.01\\
7.71	0.01\\
7.72	0.01\\
7.73	0.01\\
7.74	0.01\\
7.75	0.01\\
7.76	0.01\\
7.77	0.01\\
7.78	0.01\\
7.79	0.01\\
7.8	0.01\\
7.81	0.01\\
7.82	0.01\\
7.83	0.01\\
7.84	0.01\\
7.85	0.01\\
7.86	0.01\\
7.87	0.01\\
7.88	0.01\\
7.89	0.01\\
7.9	0.01\\
7.91	0.01\\
7.92	0.01\\
7.93	0.01\\
7.94	0.01\\
7.95	0.01\\
7.96	0.01\\
7.97	0.01\\
7.98	0.01\\
7.99	0.01\\
8	0.01\\
8.01	0.01\\
8.02	0.01\\
8.03	0.01\\
8.04	0.01\\
8.05	0.01\\
8.06	0.01\\
8.07	0.01\\
8.08	0.01\\
8.09	0.01\\
8.1	0.01\\
8.11	0.01\\
8.12	0.01\\
8.13	0.01\\
8.14	0.01\\
8.15	0.01\\
8.16	0.01\\
8.17	0.01\\
8.18	0.01\\
8.19	0.01\\
8.2	0.01\\
8.21	0.01\\
8.22	0.01\\
8.23	0.01\\
8.24	0.01\\
8.25	0.01\\
8.26	0.01\\
8.27	0.01\\
8.28	0.01\\
8.29	0.01\\
8.3	0.01\\
8.31	0.01\\
8.32	0.01\\
8.33	0.01\\
8.34	0.01\\
8.35	0.01\\
8.36	0.01\\
8.37	0.01\\
8.38	0.01\\
8.39	0.01\\
8.4	0.01\\
8.41	0.01\\
8.42	0.01\\
8.43	0.01\\
8.44	0.01\\
8.45	0.01\\
8.46	0.01\\
8.47	0.01\\
8.48	0.01\\
8.49	0.01\\
8.5	0.01\\
8.51	0.01\\
8.52	0.01\\
8.53	0.01\\
8.54	0.01\\
8.55	0.01\\
8.56	0.01\\
8.57	0.01\\
8.58	0.01\\
8.59	0.01\\
8.6	0.01\\
8.61	0.01\\
8.62	0.01\\
8.63	0.01\\
8.64	0.01\\
8.65	0.01\\
8.66	0.01\\
8.67	0.01\\
8.68	0.01\\
8.69	0.01\\
8.7	0.01\\
8.71	0.01\\
8.72	0.01\\
8.73	0.01\\
8.74	0.01\\
8.75	0.01\\
8.76	0.01\\
8.77	0.01\\
8.78	0.01\\
8.79	0.01\\
8.8	0.01\\
8.81	0.01\\
8.82	0.01\\
8.83	0.01\\
8.84	0.01\\
8.85	0.01\\
8.86	0.01\\
8.87	0.01\\
8.88	0.01\\
8.89	0.01\\
8.9	0.01\\
8.91	0.01\\
8.92	0.01\\
8.93	0.01\\
8.94	0.01\\
8.95	0.01\\
8.96	0.01\\
8.97	0.01\\
8.98	0.01\\
8.99	0.01\\
9	0.01\\
9.01	0.01\\
9.02	0.01\\
9.03	0.01\\
9.04	0.01\\
9.05	0.01\\
9.06	0.01\\
9.07	0.01\\
9.08	0.01\\
9.09	0.01\\
9.1	0.01\\
9.11	0.01\\
9.12	0.01\\
9.13	0.01\\
9.14	0.01\\
9.15	0.01\\
9.16	0.01\\
9.17	0.01\\
9.18	0.01\\
9.19	0.01\\
9.2	0.01\\
9.21	0.01\\
9.22	0.01\\
9.23	0.01\\
9.24	0.01\\
9.25	0.01\\
9.26	0.01\\
9.27	0.01\\
9.28	0.01\\
9.29	0.01\\
9.3	0.01\\
9.31	0.01\\
9.32	0.01\\
9.33	0.01\\
9.34	0.01\\
9.35	0.01\\
9.36	0.01\\
9.37	0.01\\
9.38	0.01\\
9.39	0.01\\
9.4	0.01\\
9.41	0.01\\
9.42	0.01\\
9.43	0.01\\
9.44	0.01\\
9.45	0.01\\
9.46	0.01\\
9.47	0.01\\
9.48	0.01\\
9.49	0.01\\
9.5	0.01\\
9.51	0.01\\
9.52	0.01\\
9.53	0.01\\
9.54	0.01\\
9.55	0.01\\
9.56	0.01\\
9.57	0.01\\
9.58	0.01\\
9.59	0.01\\
9.6	0.01\\
9.61	0.01\\
9.62	0.01\\
9.63	0.01\\
9.64	0.01\\
9.65	0.01\\
9.66	0.01\\
9.67	0.01\\
9.68	0.01\\
9.69	0.01\\
9.7	0.01\\
9.71	0.01\\
9.72	0.01\\
9.73	0.01\\
9.74	0.01\\
9.75	0.01\\
9.76	0.01\\
9.77	0.01\\
9.78	0.01\\
9.79	0.01\\
9.8	0.01\\
9.81	0.01\\
9.82	0.01\\
9.83	0.01\\
9.84	0.01\\
9.85	0.01\\
9.86	0.01\\
9.87	0.01\\
9.88	0.01\\
9.89	0.01\\
9.9	0.01\\
9.91	0.01\\
9.92	0.01\\
9.93	0.01\\
9.94	0.01\\
9.95	0.01\\
9.96	0.01\\
9.97	0.01\\
9.98	0.01\\
9.99	0.01\\
10	0.01\\
10.01	0.01\\
10.02	0.01\\
10.03	0.01\\
10.04	0.01\\
10.05	0.01\\
10.06	0.01\\
10.07	0.01\\
10.08	0.01\\
10.09	0.01\\
10.1	0.01\\
10.11	0.01\\
10.12	0.01\\
10.13	0.01\\
10.14	0.01\\
10.15	0.01\\
10.16	0.01\\
10.17	0.01\\
10.18	0.01\\
10.19	0.01\\
10.2	0.01\\
10.21	0.01\\
10.22	0.01\\
10.23	0.01\\
10.24	0.01\\
10.25	0.01\\
10.26	0.01\\
10.27	0.01\\
10.28	0.01\\
10.29	0.01\\
10.3	0.01\\
10.31	0.01\\
10.32	0.01\\
10.33	0.01\\
10.34	0.01\\
10.35	0.01\\
10.36	0.01\\
10.37	0.01\\
10.38	0.01\\
10.39	0.01\\
10.4	0.01\\
10.41	0.01\\
10.42	0.01\\
10.43	0.01\\
10.44	0.01\\
10.45	0.01\\
10.46	0.01\\
10.47	0.01\\
10.48	0.01\\
10.49	0.01\\
10.5	0.01\\
10.51	0.01\\
10.52	0.01\\
10.53	0.01\\
10.54	0.01\\
10.55	0.01\\
10.56	0.01\\
10.57	0.01\\
10.58	0.01\\
10.59	0.01\\
10.6	0.01\\
10.61	0.01\\
10.62	0.01\\
10.63	0.01\\
10.64	0.01\\
10.65	0.01\\
10.66	0.01\\
10.67	0.01\\
10.68	0.01\\
10.69	0.01\\
10.7	0.01\\
10.71	0.01\\
10.72	0.01\\
10.73	0.01\\
10.74	0.01\\
10.75	0.01\\
10.76	0.01\\
10.77	0.01\\
10.78	0.01\\
10.79	0.01\\
10.8	0.01\\
10.81	0.01\\
10.82	0.01\\
10.83	0.01\\
10.84	0.01\\
10.85	0.01\\
10.86	0.01\\
10.87	0.01\\
10.88	0.01\\
10.89	0.01\\
10.9	0.01\\
10.91	0.01\\
10.92	0.01\\
10.93	0.01\\
10.94	0.01\\
10.95	0.01\\
10.96	0.01\\
10.97	0.01\\
10.98	0.01\\
10.99	0.01\\
11	0.01\\
11.01	0.01\\
11.02	0.01\\
11.03	0.01\\
11.04	0.01\\
11.05	0.01\\
11.06	0.01\\
11.07	0.01\\
11.08	0.01\\
11.09	0.01\\
11.1	0.01\\
11.11	0.01\\
11.12	0.01\\
11.13	0.01\\
11.14	0.01\\
11.15	0.01\\
11.16	0.01\\
11.17	0.01\\
11.18	0.01\\
11.19	0.01\\
11.2	0.01\\
11.21	0.01\\
11.22	0.01\\
11.23	0.01\\
11.24	0.01\\
11.25	0.01\\
11.26	0.01\\
11.27	0.01\\
11.28	0.01\\
11.29	0.01\\
11.3	0.01\\
11.31	0.01\\
11.32	0.01\\
11.33	0.01\\
11.34	0.01\\
11.35	0.01\\
11.36	0.01\\
11.37	0.01\\
11.38	0.01\\
11.39	0.01\\
11.4	0.01\\
11.41	0.01\\
11.42	0.01\\
11.43	0.01\\
11.44	0.01\\
11.45	0.01\\
11.46	0.01\\
11.47	0.01\\
11.48	0.01\\
11.49	0.01\\
11.5	0.01\\
11.51	0.01\\
11.52	0.01\\
11.53	0.01\\
11.54	0.01\\
11.55	0.01\\
11.56	0.01\\
11.57	0.01\\
11.58	0.01\\
11.59	0.01\\
11.6	0.01\\
11.61	0.01\\
11.62	0.01\\
11.63	0.01\\
11.64	0.01\\
11.65	0.01\\
11.66	0.01\\
11.67	0.01\\
11.68	0.01\\
11.69	0.01\\
11.7	0.01\\
11.71	0.01\\
11.72	0.01\\
11.73	0.01\\
11.74	0.01\\
11.75	0.01\\
11.76	0.01\\
11.77	0.01\\
11.78	0.01\\
11.79	0.01\\
11.8	0.01\\
11.81	0.01\\
11.82	0.01\\
11.83	0.01\\
11.84	0.01\\
11.85	0.01\\
11.86	0.01\\
11.87	0.01\\
11.88	0.01\\
11.89	0.01\\
11.9	0.01\\
11.91	0.01\\
11.92	0.01\\
11.93	0.01\\
11.94	0.01\\
11.95	0.01\\
11.96	0.01\\
11.97	0.01\\
11.98	0.01\\
11.99	0.01\\
12	0.01\\
12.01	0.01\\
12.02	0.01\\
12.03	0.01\\
12.04	0.01\\
12.05	0.01\\
12.06	0.01\\
12.07	0.01\\
12.08	0.01\\
12.09	0.01\\
12.1	0.01\\
12.11	0.01\\
12.12	0.01\\
12.13	0.01\\
12.14	0.01\\
12.15	0.01\\
12.16	0.01\\
12.17	0.01\\
12.18	0.01\\
12.19	0.01\\
12.2	0.01\\
12.21	0.01\\
12.22	0.01\\
12.23	0.01\\
12.24	0.01\\
12.25	0.01\\
12.26	0.01\\
12.27	0.01\\
12.28	0.01\\
12.29	0.01\\
12.3	0.01\\
12.31	0.01\\
12.32	0.01\\
12.33	0.01\\
12.34	0.01\\
12.35	0.01\\
12.36	0.01\\
12.37	0.01\\
12.38	0.01\\
12.39	0.01\\
12.4	0.01\\
12.41	0.01\\
12.42	0.01\\
12.43	0.01\\
12.44	0.01\\
12.45	0.01\\
12.46	0.01\\
12.47	0.01\\
12.48	0.01\\
12.49	0.01\\
12.5	0.01\\
12.51	0.01\\
12.52	0.01\\
12.53	0.01\\
12.54	0.01\\
12.55	0.01\\
12.56	0.01\\
12.57	0.01\\
12.58	0.01\\
12.59	0.01\\
12.6	0.01\\
12.61	0.01\\
12.62	0.01\\
12.63	0.01\\
12.64	0.01\\
12.65	0.01\\
12.66	0.01\\
12.67	0.01\\
12.68	0.01\\
12.69	0.01\\
12.7	0.01\\
12.71	0.01\\
12.72	0.01\\
12.73	0.01\\
12.74	0.01\\
12.75	0.01\\
12.76	0.01\\
12.77	0.01\\
12.78	0.01\\
12.79	0.01\\
12.8	0.01\\
12.81	0.01\\
12.82	0.01\\
12.83	0.01\\
12.84	0.01\\
12.85	0.01\\
12.86	0.01\\
12.87	0.01\\
12.88	0.01\\
12.89	0.01\\
12.9	0.01\\
12.91	0.01\\
12.92	0.01\\
12.93	0.01\\
12.94	0.01\\
12.95	0.01\\
12.96	0.01\\
12.97	0.01\\
12.98	0.01\\
12.99	0.01\\
13	0.01\\
13.01	0.01\\
13.02	0.01\\
13.03	0.01\\
13.04	0.01\\
13.05	0.01\\
13.06	0.01\\
13.07	0.01\\
13.08	0.01\\
13.09	0.01\\
13.1	0.01\\
13.11	0.01\\
13.12	0.01\\
13.13	0.01\\
13.14	0.01\\
13.15	0.01\\
13.16	0.01\\
13.17	0.01\\
13.18	0.01\\
13.19	0.01\\
13.2	0.01\\
13.21	0.01\\
13.22	0.01\\
13.23	0.01\\
13.24	0.01\\
13.25	0.01\\
13.26	0.01\\
13.27	0.01\\
13.28	0.01\\
13.29	0.01\\
13.3	0.01\\
13.31	0.01\\
13.32	0.01\\
13.33	0.01\\
13.34	0.01\\
13.35	0.01\\
13.36	0.01\\
13.37	0.01\\
13.38	0.01\\
13.39	0.01\\
13.4	0.01\\
13.41	0.01\\
13.42	0.01\\
13.43	0.01\\
13.44	0.01\\
13.45	0.01\\
13.46	0.01\\
13.47	0.01\\
13.48	0.01\\
13.49	0.01\\
13.5	0.01\\
13.51	0.01\\
13.52	0.01\\
13.53	0.01\\
13.54	0.01\\
13.55	0.01\\
13.56	0.01\\
13.57	0.01\\
13.58	0.01\\
13.59	0.01\\
13.6	0.01\\
13.61	0.01\\
13.62	0.01\\
13.63	0.01\\
13.64	0.01\\
13.65	0.01\\
13.66	0.01\\
13.67	0.01\\
13.68	0.01\\
13.69	0.01\\
13.7	0.01\\
13.71	0.01\\
13.72	0.01\\
13.73	0.01\\
13.74	0.01\\
13.75	0.01\\
13.76	0.01\\
13.77	0.01\\
13.78	0.01\\
13.79	0.01\\
13.8	0.01\\
13.81	0.01\\
13.82	0.01\\
13.83	0.01\\
13.84	0.01\\
13.85	0.01\\
13.86	0.01\\
13.87	0.01\\
13.88	0.01\\
13.89	0.01\\
13.9	0.01\\
13.91	0.01\\
13.92	0.01\\
13.93	0.01\\
13.94	0.01\\
13.95	0.01\\
13.96	0.01\\
13.97	0.01\\
13.98	0.01\\
13.99	0.01\\
14	0.01\\
14.01	0.01\\
14.02	0.01\\
14.03	0.01\\
14.04	0.01\\
14.05	0.01\\
14.06	0.01\\
14.07	0.01\\
14.08	0.01\\
14.09	0.01\\
14.1	0.01\\
14.11	0.01\\
14.12	0.01\\
14.13	0.01\\
14.14	0.01\\
14.15	0.01\\
14.16	0.01\\
14.17	0.01\\
14.18	0.01\\
14.19	0.01\\
14.2	0.01\\
14.21	0.01\\
14.22	0.01\\
14.23	0.01\\
14.24	0.01\\
14.25	0.01\\
14.26	0.01\\
14.27	0.01\\
14.28	0.01\\
14.29	0.01\\
14.3	0.01\\
14.31	0.01\\
14.32	0.01\\
14.33	0.01\\
14.34	0.01\\
14.35	0.01\\
14.36	0.01\\
14.37	0.01\\
14.38	0.01\\
14.39	0.01\\
14.4	0.01\\
14.41	0.01\\
14.42	0.01\\
14.43	0.01\\
14.44	0.01\\
14.45	0.01\\
14.46	0.01\\
14.47	0.01\\
14.48	0.01\\
14.49	0.01\\
14.5	0.01\\
14.51	0.01\\
14.52	0.01\\
14.53	0.01\\
14.54	0.01\\
14.55	0.01\\
14.56	0.01\\
14.57	0.01\\
14.58	0.01\\
14.59	0.01\\
14.6	0.01\\
14.61	0.01\\
14.62	0.01\\
14.63	0.01\\
14.64	0.01\\
14.65	0.01\\
14.66	0.01\\
14.67	0.01\\
14.68	0.01\\
14.69	0.01\\
14.7	0.01\\
14.71	0.01\\
14.72	0.01\\
14.73	0.01\\
14.74	0.01\\
14.75	0.01\\
14.76	0.01\\
14.77	0.01\\
14.78	0.01\\
14.79	0.01\\
14.8	0.01\\
14.81	0.01\\
14.82	0.01\\
14.83	0.01\\
14.84	0.01\\
14.85	0.01\\
14.86	0.01\\
14.87	0.01\\
14.88	0.01\\
14.89	0.01\\
14.9	0.01\\
14.91	0.01\\
14.92	0.01\\
14.93	0.01\\
14.94	0.01\\
14.95	0.01\\
14.96	0.01\\
14.97	0.01\\
14.98	0.01\\
14.99	0.01\\
15	0.01\\
15.01	0.01\\
15.02	0.01\\
15.03	0.01\\
15.04	0.01\\
15.05	0.01\\
15.06	0.01\\
15.07	0.01\\
15.08	0.01\\
15.09	0.01\\
15.1	0.01\\
15.11	0.01\\
15.12	0.01\\
15.13	0.01\\
15.14	0.01\\
15.15	0.01\\
15.16	0.01\\
15.17	0.01\\
15.18	0.01\\
15.19	0.01\\
15.2	0.01\\
15.21	0.01\\
15.22	0.01\\
15.23	0.01\\
15.24	0.01\\
15.25	0.01\\
15.26	0.01\\
15.27	0.01\\
15.28	0.01\\
15.29	0.01\\
15.3	0.01\\
15.31	0.01\\
15.32	0.01\\
15.33	0.01\\
15.34	0.01\\
15.35	0.01\\
15.36	0.01\\
15.37	0.01\\
15.38	0.01\\
15.39	0.01\\
15.4	0.01\\
15.41	0.01\\
15.42	0.01\\
15.43	0.01\\
15.44	0.01\\
15.45	0.01\\
15.46	0.01\\
15.47	0.01\\
15.48	0.01\\
15.49	0.01\\
15.5	0.01\\
15.51	0.01\\
15.52	0.01\\
15.53	0.01\\
15.54	0.01\\
15.55	0.01\\
15.56	0.01\\
15.57	0.01\\
15.58	0.01\\
15.59	0.01\\
15.6	0.01\\
15.61	0.01\\
15.62	0.01\\
15.63	0.01\\
15.64	0.01\\
15.65	0.01\\
15.66	0.01\\
15.67	0.01\\
15.68	0.01\\
15.69	0.01\\
15.7	0.01\\
15.71	0.01\\
15.72	0.01\\
15.73	0.01\\
15.74	0.01\\
15.75	0.01\\
15.76	0.01\\
15.77	0.01\\
15.78	0.01\\
15.79	0.01\\
15.8	0.01\\
15.81	0.01\\
15.82	0.01\\
15.83	0.01\\
15.84	0.01\\
15.85	0.01\\
15.86	0.01\\
15.87	0.01\\
15.88	0.01\\
15.89	0.01\\
15.9	0.01\\
15.91	0.01\\
15.92	0.01\\
15.93	0.01\\
15.94	0.01\\
15.95	0.01\\
15.96	0.01\\
15.97	0.01\\
15.98	0.01\\
15.99	0.01\\
16	0.01\\
16.01	0.01\\
16.02	0.01\\
16.03	0.01\\
16.04	0.01\\
16.05	0.01\\
16.06	0.01\\
16.07	0.01\\
16.08	0.01\\
16.09	0.01\\
16.1	0.01\\
16.11	0.01\\
16.12	0.01\\
16.13	0.01\\
16.14	0.01\\
16.15	0.01\\
16.16	0.01\\
16.17	0.01\\
16.18	0.01\\
16.19	0.01\\
16.2	0.01\\
16.21	0.01\\
16.22	0.01\\
16.23	0.01\\
16.24	0.01\\
16.25	0.01\\
16.26	0.01\\
16.27	0.01\\
16.28	0.01\\
16.29	0.01\\
16.3	0.01\\
16.31	0.01\\
16.32	0.01\\
16.33	0.01\\
16.34	0.01\\
16.35	0.01\\
16.36	0.01\\
16.37	0.01\\
16.38	0.01\\
16.39	0.01\\
16.4	0.01\\
16.41	0.01\\
16.42	0.01\\
16.43	0.01\\
16.44	0.01\\
16.45	0.01\\
16.46	0.01\\
16.47	0.01\\
16.48	0.01\\
16.49	0.01\\
16.5	0.01\\
16.51	0.01\\
16.52	0.01\\
16.53	0.01\\
16.54	0.01\\
16.55	0.01\\
16.56	0.01\\
16.57	0.01\\
16.58	0.01\\
16.59	0.01\\
16.6	0.01\\
16.61	0.01\\
16.62	0.01\\
16.63	0.01\\
16.64	0.01\\
16.65	0.01\\
16.66	0.01\\
16.67	0.01\\
16.68	0.01\\
16.69	0.01\\
16.7	0.01\\
16.71	0.01\\
16.72	0.01\\
16.73	0.01\\
16.74	0.01\\
16.75	0.01\\
16.76	0.01\\
16.77	0.01\\
16.78	0.01\\
16.79	0.01\\
16.8	0.01\\
16.81	0.01\\
16.82	0.01\\
16.83	0.01\\
16.84	0.01\\
16.85	0.01\\
16.86	0.01\\
16.87	0.01\\
16.88	0.01\\
16.89	0.01\\
16.9	0.01\\
16.91	0.01\\
16.92	0.01\\
16.93	0.01\\
16.94	0.01\\
16.95	0.01\\
16.96	0.01\\
16.97	0.01\\
16.98	0.01\\
16.99	0.01\\
17	0.01\\
17.01	0.01\\
17.02	0.01\\
17.03	0.01\\
17.04	0.01\\
17.05	0.01\\
17.06	0.01\\
17.07	0.01\\
17.08	0.01\\
17.09	0.01\\
17.1	0.01\\
17.11	0.01\\
17.12	0.01\\
17.13	0.01\\
17.14	0.01\\
17.15	0.01\\
17.16	0.01\\
17.17	0.01\\
17.18	0.01\\
17.19	0.01\\
17.2	0.01\\
17.21	0.01\\
17.22	0.01\\
17.23	0.01\\
17.24	0.01\\
17.25	0.01\\
17.26	0.01\\
17.27	0.01\\
17.28	0.01\\
17.29	0.01\\
17.3	0.01\\
17.31	0.01\\
17.32	0.01\\
17.33	0.01\\
17.34	0.01\\
17.35	0.01\\
17.36	0.01\\
17.37	0.01\\
17.38	0.01\\
17.39	0.01\\
17.4	0.01\\
17.41	0.01\\
17.42	0.01\\
17.43	0.01\\
17.44	0.01\\
17.45	0.01\\
17.46	0.01\\
17.47	0.01\\
17.48	0.01\\
17.49	0.01\\
17.5	0.01\\
17.51	0.01\\
17.52	0.01\\
17.53	0.01\\
17.54	0.01\\
17.55	0.01\\
17.56	0.01\\
17.57	0.01\\
17.58	0.01\\
17.59	0.01\\
17.6	0.01\\
17.61	0.01\\
17.62	0.01\\
17.63	0.01\\
17.64	0.01\\
17.65	0.01\\
17.66	0.01\\
17.67	0.01\\
17.68	0.01\\
17.69	0.01\\
17.7	0.01\\
17.71	0.01\\
17.72	0.01\\
17.73	0.01\\
17.74	0.01\\
17.75	0.01\\
17.76	0.01\\
17.77	0.01\\
17.78	0.01\\
17.79	0.01\\
17.8	0.01\\
17.81	0.01\\
17.82	0.01\\
17.83	0.01\\
17.84	0.01\\
17.85	0.01\\
17.86	0.01\\
17.87	0.01\\
17.88	0.01\\
17.89	0.01\\
17.9	0.01\\
17.91	0.01\\
17.92	0.01\\
17.93	0.01\\
17.94	0.01\\
17.95	0.01\\
17.96	0.01\\
17.97	0.01\\
17.98	0.01\\
17.99	0.01\\
18	0.01\\
18.01	0.01\\
18.02	0.01\\
18.03	0.01\\
18.04	0.01\\
18.05	0.01\\
18.06	0.01\\
18.07	0.01\\
18.08	0.01\\
18.09	0.01\\
18.1	0.01\\
18.11	0.01\\
18.12	0.01\\
18.13	0.01\\
18.14	0.01\\
18.15	0.01\\
18.16	0.01\\
18.17	0.01\\
18.18	0.01\\
18.19	0.01\\
18.2	0.01\\
18.21	0.01\\
18.22	0.01\\
18.23	0.01\\
18.24	0.01\\
18.25	0.01\\
18.26	0.01\\
18.27	0.01\\
18.28	0.01\\
18.29	0.01\\
18.3	0.01\\
18.31	0.01\\
18.32	0.01\\
18.33	0.01\\
18.34	0.01\\
18.35	0.01\\
18.36	0.01\\
18.37	0.01\\
18.38	0.01\\
18.39	0.01\\
18.4	0.01\\
18.41	0.01\\
18.42	0.01\\
18.43	0.01\\
18.44	0.01\\
18.45	0.01\\
18.46	0.01\\
18.47	0.01\\
18.48	0.01\\
18.49	0.01\\
18.5	0.01\\
18.51	0.01\\
18.52	0.01\\
18.53	0.01\\
18.54	0.01\\
18.55	0.01\\
18.56	0.01\\
18.57	0.01\\
18.58	0.01\\
18.59	0.01\\
18.6	0.01\\
18.61	0.01\\
18.62	0.01\\
18.63	0.01\\
18.64	0.01\\
18.65	0.01\\
18.66	0.01\\
18.67	0.01\\
18.68	0.01\\
18.69	0.01\\
18.7	0.01\\
18.71	0.01\\
18.72	0.01\\
18.73	0.01\\
18.74	0.01\\
18.75	0.01\\
18.76	0.01\\
18.77	0.01\\
18.78	0.01\\
18.79	0.01\\
18.8	0.01\\
18.81	0.01\\
18.82	0.01\\
18.83	0.01\\
18.84	0.01\\
18.85	0.01\\
18.86	0.01\\
18.87	0.01\\
18.88	0.01\\
18.89	0.01\\
18.9	0.01\\
18.91	0.01\\
18.92	0.01\\
18.93	0.01\\
18.94	0.01\\
18.95	0.01\\
18.96	0.01\\
18.97	0.01\\
18.98	0.01\\
18.99	0.01\\
19	0.01\\
19.01	0.01\\
19.02	0.01\\
19.03	0.01\\
19.04	0.01\\
19.05	0.01\\
19.06	0.01\\
19.07	0.01\\
19.08	0.01\\
19.09	0.01\\
19.1	0.01\\
19.11	0.01\\
19.12	0.01\\
19.13	0.01\\
19.14	0.01\\
19.15	0.01\\
19.16	0.01\\
19.17	0.01\\
19.18	0.01\\
19.19	0.01\\
19.2	0.01\\
19.21	0.01\\
19.22	0.01\\
19.23	0.01\\
19.24	0.01\\
19.25	0.01\\
19.26	0.01\\
19.27	0.01\\
19.28	0.01\\
19.29	0.01\\
19.3	0.01\\
19.31	0.01\\
19.32	0.01\\
19.33	0.01\\
19.34	0.01\\
19.35	0.01\\
19.36	0.01\\
19.37	0.01\\
19.38	0.01\\
19.39	0.01\\
19.4	0.01\\
19.41	0.01\\
19.42	0.01\\
19.43	0.01\\
19.44	0.01\\
19.45	0.01\\
19.46	0.01\\
19.47	0.01\\
19.48	0.01\\
19.49	0.01\\
19.5	0.01\\
19.51	0.01\\
19.52	0.01\\
19.53	0.01\\
19.54	0.01\\
19.55	0.01\\
19.56	0.01\\
19.57	0.01\\
19.58	0.01\\
19.59	0.01\\
19.6	0.01\\
19.61	0.01\\
19.62	0.01\\
19.63	0.01\\
19.64	0.01\\
19.65	0.01\\
19.66	0.01\\
19.67	0.01\\
19.68	0.01\\
19.69	0.01\\
19.7	0.01\\
19.71	0.01\\
19.72	0.01\\
19.73	0.01\\
19.74	0.01\\
19.75	0.01\\
19.76	0.01\\
19.77	0.01\\
19.78	0.01\\
19.79	0.01\\
19.8	0.01\\
19.81	0.01\\
19.82	0.01\\
19.83	0.01\\
19.84	0.01\\
19.85	0.01\\
19.86	0.01\\
19.87	0.01\\
19.88	0.01\\
19.89	0.01\\
19.9	0.01\\
19.91	0.01\\
19.92	0.01\\
19.93	0.01\\
19.94	0.01\\
19.95	0.01\\
19.96	0.01\\
19.97	0.01\\
19.98	0.01\\
19.99	0.01\\
20	0.01\\
20.01	0.01\\
20.02	0.01\\
20.03	0.01\\
20.04	0.01\\
20.05	0.01\\
20.06	0.01\\
20.07	0.01\\
20.08	0.01\\
20.09	0.01\\
20.1	0.01\\
20.11	0.01\\
20.12	0.01\\
20.13	0.01\\
20.14	0.01\\
20.15	0.01\\
20.16	0.01\\
20.17	0.01\\
20.18	0.01\\
20.19	0.01\\
20.2	0.01\\
20.21	0.01\\
20.22	0.01\\
20.23	0.01\\
20.24	0.01\\
20.25	0.01\\
20.26	0.01\\
20.27	0.01\\
20.28	0.01\\
20.29	0.01\\
20.3	0.01\\
20.31	0.01\\
20.32	0.01\\
20.33	0.01\\
20.34	0.01\\
20.35	0.01\\
20.36	0.01\\
20.37	0.01\\
20.38	0.01\\
20.39	0.01\\
20.4	0.01\\
20.41	0.01\\
20.42	0.01\\
20.43	0.01\\
20.44	0.01\\
20.45	0.01\\
20.46	0.01\\
20.47	0.01\\
20.48	0.01\\
20.49	0.01\\
20.5	0.01\\
20.51	0.01\\
20.52	0.01\\
20.53	0.01\\
20.54	0.01\\
20.55	0.01\\
20.56	0.01\\
20.57	0.01\\
20.58	0.01\\
20.59	0.01\\
20.6	0.01\\
20.61	0.01\\
20.62	0.01\\
20.63	0.01\\
20.64	0.01\\
20.65	0.01\\
20.66	0.01\\
20.67	0.01\\
20.68	0.01\\
20.69	0.01\\
20.7	0.01\\
20.71	0.01\\
20.72	0.01\\
20.73	0.01\\
20.74	0.01\\
20.75	0.01\\
20.76	0.01\\
20.77	0.01\\
20.78	0.01\\
20.79	0.01\\
20.8	0.01\\
20.81	0.01\\
20.82	0.01\\
20.83	0.01\\
20.84	0.01\\
20.85	0.01\\
20.86	0.01\\
20.87	0.01\\
20.88	0.01\\
20.89	0.01\\
20.9	0.01\\
20.91	0.01\\
20.92	0.01\\
20.93	0.01\\
20.94	0.01\\
20.95	0.01\\
20.96	0.01\\
20.97	0.01\\
20.98	0.01\\
20.99	0.01\\
21	0.01\\
21.01	0.01\\
21.02	0.01\\
21.03	0.01\\
21.04	0.01\\
21.05	0.01\\
21.06	0.01\\
21.07	0.01\\
21.08	0.01\\
21.09	0.01\\
21.1	0.01\\
21.11	0.01\\
21.12	0.01\\
21.13	0.01\\
21.14	0.01\\
21.15	0.01\\
21.16	0.01\\
21.17	0.01\\
21.18	0.01\\
21.19	0.01\\
21.2	0.01\\
21.21	0.01\\
21.22	0.01\\
21.23	0.01\\
21.24	0.01\\
21.25	0.01\\
21.26	0.01\\
21.27	0.01\\
21.28	0.01\\
21.29	0.01\\
21.3	0.01\\
21.31	0.01\\
21.32	0.01\\
21.33	0.01\\
21.34	0.01\\
21.35	0.01\\
21.36	0.01\\
21.37	0.01\\
21.38	0.01\\
21.39	0.01\\
21.4	0.01\\
21.41	0.01\\
21.42	0.01\\
21.43	0.01\\
21.44	0.01\\
21.45	0.01\\
21.46	0.01\\
21.47	0.01\\
21.48	0.01\\
21.49	0.01\\
21.5	0.01\\
21.51	0.01\\
21.52	0.01\\
21.53	0.01\\
21.54	0.01\\
21.55	0.01\\
21.56	0.01\\
21.57	0.01\\
21.58	0.01\\
21.59	0.01\\
21.6	0.01\\
21.61	0.01\\
21.62	0.01\\
21.63	0.01\\
21.64	0.01\\
21.65	0.01\\
21.66	0.01\\
21.67	0.01\\
21.68	0.01\\
21.69	0.01\\
21.7	0.01\\
21.71	0.01\\
21.72	0.01\\
21.73	0.01\\
21.74	0.01\\
21.75	0.01\\
21.76	0.01\\
21.77	0.01\\
21.78	0.01\\
21.79	0.01\\
21.8	0.01\\
21.81	0.01\\
21.82	0.01\\
21.83	0.01\\
21.84	0.01\\
21.85	0.01\\
21.86	0.01\\
21.87	0.01\\
21.88	0.01\\
21.89	0.01\\
21.9	0.01\\
21.91	0.01\\
21.92	0.01\\
21.93	0.01\\
21.94	0.01\\
21.95	0.01\\
21.96	0.01\\
21.97	0.01\\
21.98	0.01\\
21.99	0.01\\
22	0.01\\
22.01	0.01\\
22.02	0.01\\
22.03	0.01\\
22.04	0.01\\
22.05	0.01\\
22.06	0.01\\
22.07	0.01\\
22.08	0.01\\
22.09	0.01\\
22.1	0.01\\
22.11	0.01\\
22.12	0.01\\
22.13	0.01\\
22.14	0.01\\
22.15	0.01\\
22.16	0.01\\
22.17	0.01\\
22.18	0.01\\
22.19	0.01\\
22.2	0.01\\
22.21	0.01\\
22.22	0.01\\
22.23	0.01\\
22.24	0.01\\
22.25	0.01\\
22.26	0.01\\
22.27	0.01\\
22.28	0.01\\
22.29	0.01\\
22.3	0.01\\
22.31	0.01\\
22.32	0.01\\
22.33	0.01\\
22.34	0.01\\
22.35	0.01\\
22.36	0.01\\
22.37	0.01\\
22.38	0.01\\
22.39	0.01\\
22.4	0.01\\
22.41	0.01\\
22.42	0.01\\
22.43	0.01\\
22.44	0.01\\
22.45	0.01\\
22.46	0.01\\
22.47	0.01\\
22.48	0.01\\
22.49	0.01\\
22.5	0.01\\
22.51	0.01\\
22.52	0.01\\
22.53	0.01\\
22.54	0.01\\
22.55	0.01\\
22.56	0.01\\
22.57	0.01\\
22.58	0.01\\
22.59	0.01\\
22.6	0.01\\
22.61	0.01\\
22.62	0.01\\
22.63	0.01\\
22.64	0.01\\
22.65	0.01\\
22.66	0.01\\
22.67	0.01\\
22.68	0.01\\
22.69	0.01\\
22.7	0.01\\
22.71	0.01\\
22.72	0.01\\
22.73	0.01\\
22.74	0.01\\
22.75	0.01\\
22.76	0.01\\
22.77	0.01\\
22.78	0.01\\
22.79	0.01\\
22.8	0.01\\
22.81	0.01\\
22.82	0.01\\
22.83	0.01\\
22.84	0.01\\
22.85	0.01\\
22.86	0.01\\
22.87	0.01\\
22.88	0.01\\
22.89	0.01\\
22.9	0.01\\
22.91	0.01\\
22.92	0.01\\
22.93	0.01\\
22.94	0.01\\
22.95	0.01\\
22.96	0.01\\
22.97	0.01\\
22.98	0.01\\
22.99	0.01\\
23	0.01\\
23.01	0.01\\
23.02	0.01\\
23.03	0.01\\
23.04	0.01\\
23.05	0.01\\
23.06	0.01\\
23.07	0.01\\
23.08	0.01\\
23.09	0.01\\
23.1	0.01\\
23.11	0.01\\
23.12	0.01\\
23.13	0.01\\
23.14	0.01\\
23.15	0.01\\
23.16	0.01\\
23.17	0.01\\
23.18	0.01\\
23.19	0.01\\
23.2	0.01\\
23.21	0.01\\
23.22	0.01\\
23.23	0.01\\
23.24	0.01\\
23.25	0.01\\
23.26	0.01\\
23.27	0.01\\
23.28	0.01\\
23.29	0.01\\
23.3	0.01\\
23.31	0.01\\
23.32	0.01\\
23.33	0.01\\
23.34	0.01\\
23.35	0.01\\
23.36	0.01\\
23.37	0.01\\
23.38	0.01\\
23.39	0.01\\
23.4	0.01\\
23.41	0.01\\
23.42	0.01\\
23.43	0.01\\
23.44	0.01\\
23.45	0.01\\
23.46	0.01\\
23.47	0.01\\
23.48	0.01\\
23.49	0.01\\
23.5	0.01\\
23.51	0.01\\
23.52	0.01\\
23.53	0.01\\
23.54	0.01\\
23.55	0.01\\
23.56	0.01\\
23.57	0.01\\
23.58	0.01\\
23.59	0.01\\
23.6	0.01\\
23.61	0.01\\
23.62	0.01\\
23.63	0.01\\
23.64	0.01\\
23.65	0.01\\
23.66	0.01\\
23.67	0.01\\
23.68	0.01\\
23.69	0.01\\
23.7	0.01\\
23.71	0.01\\
23.72	0.01\\
23.73	0.01\\
23.74	0.01\\
23.75	0.01\\
23.76	0.01\\
23.77	0.01\\
23.78	0.01\\
23.79	0.01\\
23.8	0.01\\
23.81	0.01\\
23.82	0.01\\
23.83	0.01\\
23.84	0.01\\
23.85	0.01\\
23.86	0.01\\
23.87	0.01\\
23.88	0.01\\
23.89	0.01\\
23.9	0.01\\
23.91	0.01\\
23.92	0.01\\
23.93	0.01\\
23.94	0.01\\
23.95	0.01\\
23.96	0.01\\
23.97	0.01\\
23.98	0.01\\
23.99	0.01\\
24	0.01\\
24.01	0.01\\
24.02	0.01\\
24.03	0.01\\
24.04	0.01\\
24.05	0.01\\
24.06	0.01\\
24.07	0.01\\
24.08	0.01\\
24.09	0.01\\
24.1	0.01\\
24.11	0.01\\
24.12	0.01\\
24.13	0.01\\
24.14	0.01\\
24.15	0.01\\
24.16	0.01\\
24.17	0.01\\
24.18	0.01\\
24.19	0.01\\
24.2	0.01\\
24.21	0.01\\
24.22	0.01\\
24.23	0.01\\
24.24	0.01\\
24.25	0.01\\
24.26	0.01\\
24.27	0.01\\
24.28	0.01\\
24.29	0.01\\
24.3	0.01\\
24.31	0.01\\
24.32	0.01\\
24.33	0.01\\
24.34	0.01\\
24.35	0.01\\
24.36	0.01\\
24.37	0.01\\
24.38	0.01\\
24.39	0.01\\
24.4	0.01\\
24.41	0.01\\
24.42	0.01\\
24.43	0.01\\
24.44	0.01\\
24.45	0.01\\
24.46	0.01\\
24.47	0.01\\
24.48	0.01\\
24.49	0.01\\
24.5	0.01\\
24.51	0.01\\
24.52	0.01\\
24.53	0.01\\
24.54	0.01\\
24.55	0.01\\
24.56	0.01\\
24.57	0.01\\
24.58	0.01\\
24.59	0.01\\
24.6	0.01\\
24.61	0.01\\
24.62	0.01\\
24.63	0.01\\
24.64	0.01\\
24.65	0.01\\
24.66	0.01\\
24.67	0.01\\
24.68	0.01\\
24.69	0.01\\
24.7	0.01\\
24.71	0.01\\
24.72	0.01\\
24.73	0.01\\
24.74	0.01\\
24.75	0.01\\
24.76	0.01\\
24.77	0.01\\
24.78	0.01\\
24.79	0.01\\
24.8	0.01\\
24.81	0.01\\
24.82	0.01\\
24.83	0.01\\
24.84	0.01\\
24.85	0.01\\
24.86	0.01\\
24.87	0.01\\
24.88	0.01\\
24.89	0.01\\
24.9	0.01\\
24.91	0.01\\
24.92	0.01\\
24.93	0.01\\
24.94	0.01\\
24.95	0.01\\
24.96	0.01\\
24.97	0.01\\
24.98	0.01\\
24.99	0.01\\
25	0.01\\
25.01	0.01\\
25.02	0.01\\
25.03	0.01\\
25.04	0.01\\
25.05	0.01\\
25.06	0.01\\
25.07	0.01\\
25.08	0.01\\
25.09	0.01\\
25.1	0.01\\
25.11	0.01\\
25.12	0.01\\
25.13	0.01\\
25.14	0.01\\
25.15	0.01\\
25.16	0.01\\
25.17	0.01\\
25.18	0.01\\
25.19	0.01\\
25.2	0.01\\
25.21	0.01\\
25.22	0.01\\
25.23	0.01\\
25.24	0.01\\
25.25	0.01\\
25.26	0.01\\
25.27	0.01\\
25.28	0.01\\
25.29	0.01\\
25.3	0.01\\
25.31	0.01\\
25.32	0.01\\
25.33	0.01\\
25.34	0.01\\
25.35	0.01\\
25.36	0.01\\
25.37	0.01\\
25.38	0.01\\
25.39	0.01\\
25.4	0.01\\
25.41	0.01\\
25.42	0.01\\
25.43	0.01\\
25.44	0.01\\
25.45	0.01\\
25.46	0.01\\
25.47	0.01\\
25.48	0.01\\
25.49	0.01\\
25.5	0.01\\
25.51	0.01\\
25.52	0.01\\
25.53	0.01\\
25.54	0.01\\
25.55	0.01\\
25.56	0.01\\
25.57	0.01\\
25.58	0.01\\
25.59	0.01\\
25.6	0.01\\
25.61	0.01\\
25.62	0.01\\
25.63	0.01\\
25.64	0.01\\
25.65	0.01\\
25.66	0.01\\
25.67	0.01\\
25.68	0.01\\
25.69	0.01\\
25.7	0.01\\
25.71	0.01\\
25.72	0.01\\
25.73	0.01\\
25.74	0.01\\
25.75	0.01\\
25.76	0.01\\
25.77	0.01\\
25.78	0.01\\
25.79	0.01\\
25.8	0.01\\
25.81	0.01\\
25.82	0.01\\
25.83	0.01\\
25.84	0.01\\
25.85	0.01\\
25.86	0.01\\
25.87	0.01\\
25.88	0.01\\
25.89	0.01\\
25.9	0.01\\
25.91	0.01\\
25.92	0.01\\
25.93	0.01\\
25.94	0.01\\
25.95	0.01\\
25.96	0.01\\
25.97	0.01\\
25.98	0.01\\
25.99	0.01\\
26	0.01\\
26.01	0.01\\
26.02	0.01\\
26.03	0.01\\
26.04	0.01\\
26.05	0.01\\
26.06	0.01\\
26.07	0.01\\
26.08	0.01\\
26.09	0.01\\
26.1	0.01\\
26.11	0.01\\
26.12	0.01\\
26.13	0.01\\
26.14	0.01\\
26.15	0.01\\
26.16	0.01\\
26.17	0.01\\
26.18	0.01\\
26.19	0.01\\
26.2	0.01\\
26.21	0.01\\
26.22	0.01\\
26.23	0.01\\
26.24	0.01\\
26.25	0.01\\
26.26	0.01\\
26.27	0.01\\
26.28	0.01\\
26.29	0.01\\
26.3	0.01\\
26.31	0.01\\
26.32	0.01\\
26.33	0.01\\
26.34	0.01\\
26.35	0.01\\
26.36	0.01\\
26.37	0.01\\
26.38	0.01\\
26.39	0.01\\
26.4	0.01\\
26.41	0.01\\
26.42	0.01\\
26.43	0.01\\
26.44	0.01\\
26.45	0.01\\
26.46	0.01\\
26.47	0.01\\
26.48	0.01\\
26.49	0.01\\
26.5	0.01\\
26.51	0.01\\
26.52	0.01\\
26.53	0.01\\
26.54	0.01\\
26.55	0.01\\
26.56	0.01\\
26.57	0.01\\
26.58	0.01\\
26.59	0.01\\
26.6	0.01\\
26.61	0.01\\
26.62	0.01\\
26.63	0.01\\
26.64	0.01\\
26.65	0.01\\
26.66	0.01\\
26.67	0.01\\
26.68	0.01\\
26.69	0.01\\
26.7	0.01\\
26.71	0.01\\
26.72	0.01\\
26.73	0.01\\
26.74	0.01\\
26.75	0.01\\
26.76	0.01\\
26.77	0.01\\
26.78	0.01\\
26.79	0.01\\
26.8	0.01\\
26.81	0.01\\
26.82	0.01\\
26.83	0.01\\
26.84	0.01\\
26.85	0.01\\
26.86	0.01\\
26.87	0.01\\
26.88	0.01\\
26.89	0.01\\
26.9	0.01\\
26.91	0.01\\
26.92	0.01\\
26.93	0.01\\
26.94	0.01\\
26.95	0.01\\
26.96	0.01\\
26.97	0.01\\
26.98	0.01\\
26.99	0.01\\
27	0.01\\
27.01	0.01\\
27.02	0.01\\
27.03	0.01\\
27.04	0.01\\
27.05	0.01\\
27.06	0.01\\
27.07	0.01\\
27.08	0.01\\
27.09	0.01\\
27.1	0.01\\
27.11	0.01\\
27.12	0.01\\
27.13	0.01\\
27.14	0.01\\
27.15	0.01\\
27.16	0.01\\
27.17	0.01\\
27.18	0.01\\
27.19	0.01\\
27.2	0.01\\
27.21	0.01\\
27.22	0.01\\
27.23	0.01\\
27.24	0.01\\
27.25	0.01\\
27.26	0.01\\
27.27	0.01\\
27.28	0.01\\
27.29	0.01\\
27.3	0.01\\
27.31	0.01\\
27.32	0.01\\
27.33	0.01\\
27.34	0.01\\
27.35	0.01\\
27.36	0.01\\
27.37	0.01\\
27.38	0.01\\
27.39	0.01\\
27.4	0.01\\
27.41	0.01\\
27.42	0.01\\
27.43	0.01\\
27.44	0.01\\
27.45	0.01\\
27.46	0.01\\
27.47	0.01\\
27.48	0.01\\
27.49	0.01\\
27.5	0.01\\
27.51	0.01\\
27.52	0.01\\
27.53	0.01\\
27.54	0.01\\
27.55	0.01\\
27.56	0.01\\
27.57	0.01\\
27.58	0.01\\
27.59	0.01\\
27.6	0.01\\
27.61	0.01\\
27.62	0.01\\
27.63	0.01\\
27.64	0.01\\
27.65	0.01\\
27.66	0.01\\
27.67	0.01\\
27.68	0.01\\
27.69	0.01\\
27.7	0.01\\
27.71	0.01\\
27.72	0.01\\
27.73	0.01\\
27.74	0.01\\
27.75	0.01\\
27.76	0.01\\
27.77	0.01\\
27.78	0.01\\
27.79	0.01\\
27.8	0.01\\
27.81	0.01\\
27.82	0.01\\
27.83	0.01\\
27.84	0.01\\
27.85	0.01\\
27.86	0.01\\
27.87	0.01\\
27.88	0.01\\
27.89	0.01\\
27.9	0.01\\
27.91	0.01\\
27.92	0.01\\
27.93	0.01\\
27.94	0.01\\
27.95	0.01\\
27.96	0.01\\
27.97	0.01\\
27.98	0.01\\
27.99	0.01\\
28	0.01\\
28.01	0.01\\
28.02	0.01\\
28.03	0.01\\
28.04	0.01\\
28.05	0.01\\
28.06	0.01\\
28.07	0.01\\
28.08	0.01\\
28.09	0.01\\
28.1	0.01\\
28.11	0.01\\
28.12	0.01\\
28.13	0.01\\
28.14	0.01\\
28.15	0.01\\
28.16	0.01\\
28.17	0.01\\
28.18	0.01\\
28.19	0.01\\
28.2	0.01\\
28.21	0.01\\
28.22	0.01\\
28.23	0.01\\
28.24	0.01\\
28.25	0.01\\
28.26	0.01\\
28.27	0.01\\
28.28	0.01\\
28.29	0.01\\
28.3	0.01\\
28.31	0.01\\
28.32	0.01\\
28.33	0.01\\
28.34	0.01\\
28.35	0.01\\
28.36	0.01\\
28.37	0.01\\
28.38	0.01\\
28.39	0.01\\
28.4	0.01\\
28.41	0.01\\
28.42	0.01\\
28.43	0.01\\
28.44	0.01\\
28.45	0.01\\
28.46	0.01\\
28.47	0.01\\
28.48	0.01\\
28.49	0.01\\
28.5	0.01\\
28.51	0.01\\
28.52	0.01\\
28.53	0.01\\
28.54	0.01\\
28.55	0.01\\
28.56	0.01\\
28.57	0.01\\
28.58	0.01\\
28.59	0.01\\
28.6	0.01\\
28.61	0.01\\
28.62	0.01\\
28.63	0.01\\
28.64	0.01\\
28.65	0.01\\
28.66	0.01\\
28.67	0.01\\
28.68	0.01\\
28.69	0.01\\
28.7	0.01\\
28.71	0.01\\
28.72	0.01\\
28.73	0.01\\
28.74	0.01\\
28.75	0.01\\
28.76	0.01\\
28.77	0.01\\
28.78	0.01\\
28.79	0.01\\
28.8	0.01\\
28.81	0.01\\
28.82	0.01\\
28.83	0.01\\
28.84	0.01\\
28.85	0.01\\
28.86	0.01\\
28.87	0.01\\
28.88	0.01\\
28.89	0.01\\
28.9	0.01\\
28.91	0.01\\
28.92	0.01\\
28.93	0.01\\
28.94	0.01\\
28.95	0.01\\
28.96	0.01\\
28.97	0.01\\
28.98	0.01\\
28.99	0.01\\
29	0.01\\
29.01	0.01\\
29.02	0.01\\
29.03	0.01\\
29.04	0.01\\
29.05	0.01\\
29.06	0.01\\
29.07	0.01\\
29.08	0.01\\
29.09	0.01\\
29.1	0.01\\
29.11	0.01\\
29.12	0.01\\
29.13	0.01\\
29.14	0.01\\
29.15	0.01\\
29.16	0.01\\
29.17	0.01\\
29.18	0.01\\
29.19	0.01\\
29.2	0.01\\
29.21	0.01\\
29.22	0.01\\
29.23	0.01\\
29.24	0.01\\
29.25	0.01\\
29.26	0.01\\
29.27	0.01\\
29.28	0.01\\
29.29	0.01\\
29.3	0.01\\
29.31	0.01\\
29.32	0.01\\
29.33	0.01\\
29.34	0.01\\
29.35	0.01\\
29.36	0.01\\
29.37	0.01\\
29.38	0.01\\
29.39	0.01\\
29.4	0.01\\
29.41	0.01\\
29.42	0.01\\
29.43	0.01\\
29.44	0.01\\
29.45	0.01\\
29.46	0.01\\
29.47	0.01\\
29.48	0.01\\
29.49	0.01\\
29.5	0.01\\
29.51	0.01\\
29.52	0.01\\
29.53	0.01\\
29.54	0.01\\
29.55	0.01\\
29.56	0.01\\
29.57	0.01\\
29.58	0.01\\
29.59	0.01\\
29.6	0.01\\
29.61	0.01\\
29.62	0.01\\
29.63	0.01\\
29.64	0.01\\
29.65	0.01\\
29.66	0.01\\
29.67	0.01\\
29.68	0.01\\
29.69	0.01\\
29.7	0.01\\
29.71	0.01\\
29.72	0.01\\
29.73	0.01\\
29.74	0.01\\
29.75	0.01\\
29.76	0.01\\
29.77	0.01\\
29.78	0.01\\
29.79	0.01\\
29.8	0.01\\
29.81	0.01\\
29.82	0.01\\
29.83	0.01\\
29.84	0.01\\
29.85	0.01\\
29.86	0.01\\
29.87	0.01\\
29.88	0.01\\
29.89	0.01\\
29.9	0.01\\
29.91	0.01\\
29.92	0.01\\
29.93	0.01\\
29.94	0.01\\
29.95	0.01\\
29.96	0.01\\
29.97	0.01\\
29.98	0.01\\
29.99	0.01\\
30	0.01\\
30.01	0.01\\
30.02	0.01\\
30.03	0.01\\
30.04	0.01\\
30.05	0.01\\
30.06	0.01\\
30.07	0.01\\
30.08	0.01\\
30.09	0.01\\
30.1	0.01\\
30.11	0.01\\
30.12	0.01\\
30.13	0.01\\
30.14	0.01\\
30.15	0.01\\
30.16	0.01\\
30.17	0.01\\
30.18	0.01\\
30.19	0.01\\
30.2	0.01\\
30.21	0.01\\
30.22	0.01\\
30.23	0.01\\
30.24	0.01\\
30.25	0.01\\
30.26	0.01\\
30.27	0.01\\
30.28	0.01\\
30.29	0.01\\
30.3	0.01\\
30.31	0.01\\
30.32	0.01\\
30.33	0.01\\
30.34	0.01\\
30.35	0.01\\
30.36	0.01\\
30.37	0.01\\
30.38	0.01\\
30.39	0.01\\
30.4	0.01\\
30.41	0.01\\
30.42	0.01\\
30.43	0.01\\
30.44	0.01\\
30.45	0.01\\
30.46	0.01\\
30.47	0.01\\
30.48	0.01\\
30.49	0.01\\
30.5	0.01\\
30.51	0.01\\
30.52	0.01\\
30.53	0.01\\
30.54	0.01\\
30.55	0.01\\
30.56	0.01\\
30.57	0.01\\
30.58	0.01\\
30.59	0.01\\
30.6	0.01\\
30.61	0.01\\
30.62	0.01\\
30.63	0.01\\
30.64	0.01\\
30.65	0.01\\
30.66	0.01\\
30.67	0.01\\
30.68	0.01\\
30.69	0.01\\
30.7	0.01\\
30.71	0.01\\
30.72	0.01\\
30.73	0.01\\
30.74	0.01\\
30.75	0.01\\
30.76	0.01\\
30.77	0.01\\
30.78	0.01\\
30.79	0.01\\
30.8	0.01\\
30.81	0.01\\
30.82	0.01\\
30.83	0.01\\
30.84	0.01\\
30.85	0.01\\
30.86	0.01\\
30.87	0.01\\
30.88	0.01\\
30.89	0.01\\
30.9	0.01\\
30.91	0.01\\
30.92	0.01\\
30.93	0.01\\
30.94	0.01\\
30.95	0.01\\
30.96	0.01\\
30.97	0.01\\
30.98	0.01\\
30.99	0.01\\
31	0.01\\
31.01	0.01\\
31.02	0.01\\
31.03	0.01\\
31.04	0.01\\
31.05	0.01\\
31.06	0.01\\
31.07	0.01\\
31.08	0.01\\
31.09	0.01\\
31.1	0.01\\
31.11	0.01\\
31.12	0.01\\
31.13	0.01\\
31.14	0.01\\
31.15	0.01\\
31.16	0.01\\
31.17	0.01\\
31.18	0.01\\
31.19	0.01\\
31.2	0.01\\
31.21	0.01\\
31.22	0.01\\
31.23	0.01\\
31.24	0.01\\
31.25	0.01\\
31.26	0.01\\
31.27	0.01\\
31.28	0.01\\
31.29	0.01\\
31.3	0.01\\
31.31	0.01\\
31.32	0.01\\
31.33	0.01\\
31.34	0.01\\
31.35	0.01\\
31.36	0.01\\
31.37	0.01\\
31.38	0.01\\
31.39	0.01\\
31.4	0.01\\
31.41	0.01\\
31.42	0.01\\
31.43	0.01\\
31.44	0.01\\
31.45	0.01\\
31.46	0.01\\
31.47	0.01\\
31.48	0.01\\
31.49	0.01\\
31.5	0.01\\
31.51	0.01\\
31.52	0.01\\
31.53	0.01\\
31.54	0.01\\
31.55	0.01\\
31.56	0.01\\
31.57	0.01\\
31.58	0.01\\
31.59	0.01\\
31.6	0.01\\
31.61	0.01\\
31.62	0.01\\
31.63	0.01\\
31.64	0.01\\
31.65	0.01\\
31.66	0.01\\
31.67	0.01\\
31.68	0.01\\
31.69	0.01\\
31.7	0.01\\
31.71	0.01\\
31.72	0.01\\
31.73	0.01\\
31.74	0.01\\
31.75	0.01\\
31.76	0.01\\
31.77	0.01\\
31.78	0.01\\
31.79	0.01\\
31.8	0.01\\
31.81	0.01\\
31.82	0.01\\
31.83	0.01\\
31.84	0.01\\
31.85	0.01\\
31.86	0.01\\
31.87	0.01\\
31.88	0.01\\
31.89	0.01\\
31.9	0.01\\
31.91	0.01\\
31.92	0.01\\
31.93	0.01\\
31.94	0.01\\
31.95	0.01\\
31.96	0.01\\
31.97	0.01\\
31.98	0.01\\
31.99	0.01\\
32	0.01\\
32.01	0.01\\
32.02	0.01\\
32.03	0.01\\
32.04	0.01\\
32.05	0.01\\
32.06	0.01\\
32.07	0.01\\
32.08	0.01\\
32.09	0.01\\
32.1	0.01\\
32.11	0.01\\
32.12	0.01\\
32.13	0.01\\
32.14	0.01\\
32.15	0.01\\
32.16	0.01\\
32.17	0.01\\
32.18	0.01\\
32.19	0.01\\
32.2	0.01\\
32.21	0.01\\
32.22	0.01\\
32.23	0.01\\
32.24	0.01\\
32.25	0.01\\
32.26	0.01\\
32.27	0.01\\
32.28	0.01\\
32.29	0.01\\
32.3	0.01\\
32.31	0.01\\
32.32	0.01\\
32.33	0.01\\
32.34	0.01\\
32.35	0.01\\
32.36	0.01\\
32.37	0.01\\
32.38	0.01\\
32.39	0.01\\
32.4	0.01\\
32.41	0.01\\
32.42	0.01\\
32.43	0.01\\
32.44	0.01\\
32.45	0.01\\
32.46	0.01\\
32.47	0.01\\
32.48	0.01\\
32.49	0.01\\
32.5	0.01\\
32.51	0.01\\
32.52	0.01\\
32.53	0.01\\
32.54	0.01\\
32.55	0.01\\
32.56	0.01\\
32.57	0.01\\
32.58	0.01\\
32.59	0.01\\
32.6	0.01\\
32.61	0.01\\
32.62	0.01\\
32.63	0.01\\
32.64	0.01\\
32.65	0.01\\
32.66	0.01\\
32.67	0.01\\
32.68	0.01\\
32.69	0.01\\
32.7	0.01\\
32.71	0.01\\
32.72	0.01\\
32.73	0.01\\
32.74	0.01\\
32.75	0.01\\
32.76	0.01\\
32.77	0.01\\
32.78	0.01\\
32.79	0.01\\
32.8	0.01\\
32.81	0.01\\
32.82	0.01\\
32.83	0.01\\
32.84	0.01\\
32.85	0.01\\
32.86	0.01\\
32.87	0.01\\
32.88	0.01\\
32.89	0.01\\
32.9	0.01\\
32.91	0.01\\
32.92	0.01\\
32.93	0.01\\
32.94	0.01\\
32.95	0.01\\
32.96	0.01\\
32.97	0.01\\
32.98	0.01\\
32.99	0.01\\
33	0.01\\
33.01	0.01\\
33.02	0.01\\
33.03	0.01\\
33.04	0.01\\
33.05	0.01\\
33.06	0.01\\
33.07	0.01\\
33.08	0.01\\
33.09	0.01\\
33.1	0.01\\
33.11	0.01\\
33.12	0.01\\
33.13	0.01\\
33.14	0.01\\
33.15	0.01\\
33.16	0.01\\
33.17	0.01\\
33.18	0.01\\
33.19	0.01\\
33.2	0.01\\
33.21	0.01\\
33.22	0.01\\
33.23	0.01\\
33.24	0.01\\
33.25	0.01\\
33.26	0.01\\
33.27	0.01\\
33.28	0.01\\
33.29	0.01\\
33.3	0.01\\
33.31	0.01\\
33.32	0.01\\
33.33	0.01\\
33.34	0.01\\
33.35	0.01\\
33.36	0.01\\
33.37	0.01\\
33.38	0.01\\
33.39	0.01\\
33.4	0.01\\
33.41	0.01\\
33.42	0.01\\
33.43	0.01\\
33.44	0.01\\
33.45	0.01\\
33.46	0.01\\
33.47	0.01\\
33.48	0.01\\
33.49	0.01\\
33.5	0.01\\
33.51	0.01\\
33.52	0.01\\
33.53	0.01\\
33.54	0.01\\
33.55	0.01\\
33.56	0.01\\
33.57	0.01\\
33.58	0.01\\
33.59	0.01\\
33.6	0.01\\
33.61	0.01\\
33.62	0.01\\
33.63	0.01\\
33.64	0.01\\
33.65	0.01\\
33.66	0.01\\
33.67	0.01\\
33.68	0.01\\
33.69	0.01\\
33.7	0.01\\
33.71	0.01\\
33.72	0.01\\
33.73	0.01\\
33.74	0.01\\
33.75	0.01\\
33.76	0.01\\
33.77	0.01\\
33.78	0.01\\
33.79	0.01\\
33.8	0.01\\
33.81	0.01\\
33.82	0.01\\
33.83	0.01\\
33.84	0.01\\
33.85	0.01\\
33.86	0.01\\
33.87	0.01\\
33.88	0.01\\
33.89	0.01\\
33.9	0.01\\
33.91	0.01\\
33.92	0.01\\
33.93	0.01\\
33.94	0.01\\
33.95	0.01\\
33.96	0.01\\
33.97	0.01\\
33.98	0.01\\
33.99	0.01\\
34	0.01\\
34.01	0.01\\
34.02	0.01\\
34.03	0.01\\
34.04	0.01\\
34.05	0.01\\
34.06	0.01\\
34.07	0.01\\
34.08	0.01\\
34.09	0.01\\
34.1	0.01\\
34.11	0.01\\
34.12	0.01\\
34.13	0.01\\
34.14	0.01\\
34.15	0.01\\
34.16	0.01\\
34.17	0.01\\
34.18	0.01\\
34.19	0.01\\
34.2	0.01\\
34.21	0.01\\
34.22	0.01\\
34.23	0.01\\
34.24	0.01\\
34.25	0.01\\
34.26	0.01\\
34.27	0.01\\
34.28	0.01\\
34.29	0.01\\
34.3	0.01\\
34.31	0.01\\
34.32	0.01\\
34.33	0.01\\
34.34	0.01\\
34.35	0.01\\
34.36	0.01\\
34.37	0.01\\
34.38	0.01\\
34.39	0.01\\
34.4	0.01\\
34.41	0.01\\
34.42	0.01\\
34.43	0.01\\
34.44	0.01\\
34.45	0.01\\
34.46	0.01\\
34.47	0.01\\
34.48	0.01\\
34.49	0.01\\
34.5	0.01\\
34.51	0.01\\
34.52	0.01\\
34.53	0.01\\
34.54	0.01\\
34.55	0.01\\
34.56	0.01\\
34.57	0.01\\
34.58	0.01\\
34.59	0.01\\
34.6	0.01\\
34.61	0.01\\
34.62	0.01\\
34.63	0.01\\
34.64	0.01\\
34.65	0.01\\
34.66	0.01\\
34.67	0.01\\
34.68	0.01\\
34.69	0.01\\
34.7	0.01\\
34.71	0.01\\
34.72	0.01\\
34.73	0.01\\
34.74	0.01\\
34.75	0.01\\
34.76	0.01\\
34.77	0.01\\
34.78	0.01\\
34.79	0.01\\
34.8	0.01\\
34.81	0.01\\
34.82	0.01\\
34.83	0.01\\
34.84	0.01\\
34.85	0.01\\
34.86	0.01\\
34.87	0.01\\
34.88	0.01\\
34.89	0.01\\
34.9	0.01\\
34.91	0.01\\
34.92	0.01\\
34.93	0.01\\
34.94	0.01\\
34.95	0.01\\
34.96	0.01\\
34.97	0.01\\
34.98	0.01\\
34.99	0.01\\
35	0.01\\
35.01	0.01\\
35.02	0.01\\
35.03	0.01\\
35.04	0.01\\
35.05	0.01\\
35.06	0.01\\
35.07	0.01\\
35.08	0.01\\
35.09	0.01\\
35.1	0.01\\
35.11	0.01\\
35.12	0.01\\
35.13	0.01\\
35.14	0.01\\
35.15	0.01\\
35.16	0.01\\
35.17	0.01\\
35.18	0.01\\
35.19	0.01\\
35.2	0.01\\
35.21	0.01\\
35.22	0.01\\
35.23	0.01\\
35.24	0.01\\
35.25	0.01\\
35.26	0.01\\
35.27	0.01\\
35.28	0.01\\
35.29	0.01\\
35.3	0.01\\
35.31	0.01\\
35.32	0.01\\
35.33	0.01\\
35.34	0.01\\
35.35	0.01\\
35.36	0.01\\
35.37	0.01\\
35.38	0.01\\
35.39	0.01\\
35.4	0.01\\
35.41	0.01\\
35.42	0.01\\
35.43	0.01\\
35.44	0.01\\
35.45	0.01\\
35.46	0.01\\
35.47	0.01\\
35.48	0.01\\
35.49	0.01\\
35.5	0.01\\
35.51	0.01\\
35.52	0.01\\
35.53	0.01\\
35.54	0.01\\
35.55	0.01\\
35.56	0.01\\
35.57	0.01\\
35.58	0.01\\
35.59	0.01\\
35.6	0.01\\
35.61	0.01\\
35.62	0.01\\
35.63	0.01\\
35.64	0.01\\
35.65	0.01\\
35.66	0.01\\
35.67	0.01\\
35.68	0.01\\
35.69	0.01\\
35.7	0.01\\
35.71	0.01\\
35.72	0.01\\
35.73	0.01\\
35.74	0.01\\
35.75	0.01\\
35.76	0.01\\
35.77	0.01\\
35.78	0.01\\
35.79	0.01\\
35.8	0.01\\
35.81	0.01\\
35.82	0.01\\
35.83	0.01\\
35.84	0.01\\
35.85	0.01\\
35.86	0.01\\
35.87	0.01\\
35.88	0.01\\
35.89	0.01\\
35.9	0.01\\
35.91	0.01\\
35.92	0.01\\
35.93	0.01\\
35.94	0.01\\
35.95	0.01\\
35.96	0.01\\
35.97	0.01\\
35.98	0.01\\
35.99	0.01\\
36	0.01\\
36.01	0.01\\
36.02	0.01\\
36.03	0.01\\
36.04	0.01\\
36.05	0.01\\
36.06	0.01\\
36.07	0.01\\
36.08	0.01\\
36.09	0.01\\
36.1	0.01\\
36.11	0.01\\
36.12	0.01\\
36.13	0.01\\
36.14	0.01\\
36.15	0.01\\
36.16	0.01\\
36.17	0.01\\
36.18	0.01\\
36.19	0.01\\
36.2	0.01\\
36.21	0.01\\
36.22	0.01\\
36.23	0.01\\
36.24	0.01\\
36.25	0.01\\
36.26	0.01\\
36.27	0.01\\
36.28	0.01\\
36.29	0.01\\
36.3	0.01\\
36.31	0.01\\
36.32	0.01\\
36.33	0.01\\
36.34	0.01\\
36.35	0.01\\
36.36	0.01\\
36.37	0.01\\
36.38	0.01\\
36.39	0.01\\
36.4	0.01\\
36.41	0.01\\
36.42	0.01\\
36.43	0.01\\
36.44	0.01\\
36.45	0.01\\
36.46	0.01\\
36.47	0.01\\
36.48	0.01\\
36.49	0.01\\
36.5	0.01\\
36.51	0.01\\
36.52	0.01\\
36.53	0.01\\
36.54	0.01\\
36.55	0.01\\
36.56	0.01\\
36.57	0.01\\
36.58	0.01\\
36.59	0.01\\
36.6	0.01\\
36.61	0.01\\
36.62	0.01\\
36.63	0.01\\
36.64	0.01\\
36.65	0.01\\
36.66	0.01\\
36.67	0.01\\
36.68	0.01\\
36.69	0.01\\
36.7	0.01\\
36.71	0.01\\
36.72	0.01\\
36.73	0.01\\
36.74	0.01\\
36.75	0.01\\
36.76	0.01\\
36.77	0.01\\
36.78	0.01\\
36.79	0.01\\
36.8	0.01\\
36.81	0.01\\
36.82	0.01\\
36.83	0.01\\
36.84	0.01\\
36.85	0.01\\
36.86	0.01\\
36.87	0.01\\
36.88	0.01\\
36.89	0.01\\
36.9	0.01\\
36.91	0.01\\
36.92	0.01\\
36.93	0.01\\
36.94	0.01\\
36.95	0.01\\
36.96	0.01\\
36.97	0.01\\
36.98	0.01\\
36.99	0.01\\
37	0.01\\
37.01	0.01\\
37.02	0.01\\
37.03	0.01\\
37.04	0.01\\
37.05	0.01\\
37.06	0.01\\
37.07	0.01\\
37.08	0.01\\
37.09	0.01\\
37.1	0.01\\
37.11	0.01\\
37.12	0.01\\
37.13	0.01\\
37.14	0.01\\
37.15	0.01\\
37.16	0.01\\
37.17	0.01\\
37.18	0.01\\
37.19	0.01\\
37.2	0.01\\
37.21	0.01\\
37.22	0.01\\
37.23	0.01\\
37.24	0.01\\
37.25	0.01\\
37.26	0.01\\
37.27	0.01\\
37.28	0.01\\
37.29	0.01\\
37.3	0.01\\
37.31	0.01\\
37.32	0.01\\
37.33	0.01\\
37.34	0.01\\
37.35	0.01\\
37.36	0.01\\
37.37	0.01\\
37.38	0.01\\
37.39	0.01\\
37.4	0.01\\
37.41	0.01\\
37.42	0.01\\
37.43	0.01\\
37.44	0.01\\
37.45	0.01\\
37.46	0.01\\
37.47	0.01\\
37.48	0.01\\
37.49	0.01\\
37.5	0.01\\
37.51	0.01\\
37.52	0.01\\
37.53	0.01\\
37.54	0.01\\
37.55	0.01\\
37.56	0.01\\
37.57	0.01\\
37.58	0.01\\
37.59	0.01\\
37.6	0.01\\
37.61	0.01\\
37.62	0.01\\
37.63	0.01\\
37.64	0.01\\
37.65	0.01\\
37.66	0.01\\
37.67	0.01\\
37.68	0.01\\
37.69	0.01\\
37.7	0.01\\
37.71	0.01\\
37.72	0.01\\
37.73	0.01\\
37.74	0.01\\
37.75	0.01\\
37.76	0.01\\
37.77	0.01\\
37.78	0.01\\
37.79	0.01\\
37.8	0.01\\
37.81	0.01\\
37.82	0.01\\
37.83	0.01\\
37.84	0.01\\
37.85	0.01\\
37.86	0.01\\
37.87	0.01\\
37.88	0.01\\
37.89	0.01\\
37.9	0.01\\
37.91	0.01\\
37.92	0.01\\
37.93	0.01\\
37.94	0.01\\
37.95	0.01\\
37.96	0.01\\
37.97	0.01\\
37.98	0.01\\
37.99	0.01\\
38	0.01\\
38.01	0.01\\
38.02	0.01\\
38.03	0.01\\
38.04	0.01\\
38.05	0.01\\
38.06	0.01\\
38.07	0.01\\
38.08	0.01\\
38.09	0.01\\
38.1	0.01\\
38.11	0.01\\
38.12	0.01\\
38.13	0.01\\
38.14	0.01\\
38.15	0.01\\
38.16	0.01\\
38.17	0.01\\
38.18	0.01\\
38.19	0.01\\
38.2	0.01\\
38.21	0.01\\
38.22	0.01\\
38.23	0.01\\
38.24	0.01\\
38.25	0.01\\
38.26	0.01\\
38.27	0.01\\
38.28	0.01\\
38.29	0.01\\
38.3	0.01\\
38.31	0.01\\
38.32	0.01\\
38.33	0.01\\
38.34	0.01\\
38.35	0.01\\
38.36	0.01\\
38.37	0.01\\
38.38	0.01\\
38.39	0.01\\
38.4	0.01\\
38.41	0.01\\
38.42	0.01\\
38.43	0.01\\
38.44	0.01\\
38.45	0.01\\
38.46	0.01\\
38.47	0.01\\
38.48	0.01\\
38.49	0.01\\
38.5	0.01\\
38.51	0.01\\
38.52	0.01\\
38.53	0.01\\
38.54	0.01\\
38.55	0.01\\
38.56	0.01\\
38.57	0.01\\
38.58	0.01\\
38.59	0.01\\
38.6	0.01\\
38.61	0.01\\
38.62	0.01\\
38.63	0.01\\
38.64	0.01\\
38.65	0.01\\
38.66	0.01\\
38.67	0.01\\
38.68	0.01\\
38.69	0.01\\
38.7	0.01\\
38.71	0.01\\
38.72	0.01\\
38.73	0.01\\
38.74	0.01\\
38.75	0.01\\
38.76	0.01\\
38.77	0.01\\
38.78	0.01\\
38.79	0.01\\
38.8	0.01\\
38.81	0.01\\
38.82	0.01\\
38.83	0.01\\
38.84	0.01\\
38.85	0.01\\
38.86	0.01\\
38.87	0.01\\
38.88	0.01\\
38.89	0.01\\
38.9	0.01\\
38.91	0.01\\
38.92	0.01\\
38.93	0.01\\
38.94	0.01\\
38.95	0.01\\
38.96	0.01\\
38.97	0.01\\
38.98	0.01\\
38.99	0.01\\
39	0.01\\
39.01	0.01\\
39.02	0.01\\
39.03	0.01\\
39.04	0.01\\
39.05	0.01\\
39.06	0.01\\
39.07	0.01\\
39.08	0.01\\
39.09	0.01\\
39.1	0.01\\
39.11	0.01\\
39.12	0.01\\
39.13	0.01\\
39.14	0.01\\
39.15	0.01\\
39.16	0.01\\
39.17	0.01\\
39.18	0.01\\
39.19	0.01\\
39.2	0.01\\
39.21	0.01\\
39.22	0.01\\
39.23	0.01\\
39.24	0.01\\
39.25	0.01\\
39.26	0.01\\
39.27	0.01\\
39.28	0.01\\
39.29	0.01\\
39.3	0.01\\
39.31	0.01\\
39.32	0.01\\
39.33	0.01\\
39.34	0.01\\
39.35	0.01\\
39.36	0.01\\
39.37	0.01\\
39.38	0.01\\
39.39	0.01\\
39.4	0.01\\
39.41	0.01\\
39.42	0.01\\
39.43	0.01\\
39.44	0.01\\
39.45	0.01\\
39.46	0.01\\
39.47	0.01\\
39.48	0.01\\
39.49	0.01\\
39.5	0.01\\
39.51	0.01\\
39.52	0.01\\
39.53	0.01\\
39.54	0.01\\
39.55	0.01\\
39.56	0.01\\
39.57	0.01\\
39.58	0.01\\
39.59	0.01\\
39.6	0.01\\
39.61	0.01\\
39.62	0.01\\
39.63	0.01\\
39.64	0.01\\
39.65	0.01\\
39.66	0.01\\
39.67	0.01\\
39.68	0.01\\
39.69	0.01\\
39.7	0.01\\
39.71	0.01\\
39.72	0.01\\
39.73	0.01\\
39.74	0.01\\
39.75	0.01\\
39.76	0.01\\
39.77	0.01\\
39.78	0.01\\
39.79	0.01\\
39.8	0.01\\
39.81	0.01\\
39.82	0.01\\
39.83	0.01\\
39.84	0.01\\
39.85	0.01\\
39.86	0.01\\
39.87	0.01\\
39.88	0.01\\
39.89	0.01\\
39.9	0.01\\
39.91	0.01\\
39.92	0.01\\
39.93	0.01\\
39.94	0.01\\
39.95	0.01\\
39.96	0.01\\
39.97	0.01\\
39.98	0.01\\
39.99	0.01\\
40	0.01\\
40.01	0.01\\
};
\addplot [color=red,solid,forget plot]
  table[row sep=crcr]{%
40.01	0.01\\
40.02	0.01\\
40.03	0.01\\
40.04	0.01\\
40.05	0.01\\
40.06	0.01\\
40.07	0.01\\
40.08	0.01\\
40.09	0.01\\
40.1	0.01\\
40.11	0.01\\
40.12	0.01\\
40.13	0.01\\
40.14	0.01\\
40.15	0.01\\
40.16	0.01\\
40.17	0.01\\
40.18	0.01\\
40.19	0.01\\
40.2	0.01\\
40.21	0.01\\
40.22	0.01\\
40.23	0.01\\
40.24	0.01\\
40.25	0.01\\
40.26	0.01\\
40.27	0.01\\
40.28	0.01\\
40.29	0.01\\
40.3	0.01\\
40.31	0.01\\
40.32	0.01\\
40.33	0.01\\
40.34	0.01\\
40.35	0.01\\
40.36	0.01\\
40.37	0.01\\
40.38	0.01\\
40.39	0.01\\
40.4	0.01\\
40.41	0.01\\
40.42	0.01\\
40.43	0.01\\
40.44	0.01\\
40.45	0.01\\
40.46	0.01\\
40.47	0.01\\
40.48	0.01\\
40.49	0.01\\
40.5	0.01\\
40.51	0.01\\
40.52	0.01\\
40.53	0.01\\
40.54	0.01\\
40.55	0.01\\
40.56	0.01\\
40.57	0.01\\
40.58	0.01\\
40.59	0.01\\
40.6	0.01\\
40.61	0.01\\
40.62	0.01\\
40.63	0.01\\
40.64	0.01\\
40.65	0.01\\
40.66	0.01\\
40.67	0.01\\
40.68	0.01\\
40.69	0.01\\
40.7	0.01\\
40.71	0.01\\
40.72	0.01\\
40.73	0.01\\
40.74	0.01\\
40.75	0.01\\
40.76	0.01\\
40.77	0.01\\
40.78	0.01\\
40.79	0.01\\
40.8	0.01\\
40.81	0.01\\
40.82	0.01\\
40.83	0.01\\
40.84	0.01\\
40.85	0.01\\
40.86	0.01\\
40.87	0.01\\
40.88	0.01\\
40.89	0.01\\
40.9	0.01\\
40.91	0.01\\
40.92	0.01\\
40.93	0.01\\
40.94	0.01\\
40.95	0.01\\
40.96	0.01\\
40.97	0.01\\
40.98	0.01\\
40.99	0.01\\
41	0.01\\
41.01	0.01\\
41.02	0.01\\
41.03	0.01\\
41.04	0.01\\
41.05	0.01\\
41.06	0.01\\
41.07	0.01\\
41.08	0.01\\
41.09	0.01\\
41.1	0.01\\
41.11	0.01\\
41.12	0.01\\
41.13	0.01\\
41.14	0.01\\
41.15	0.01\\
41.16	0.01\\
41.17	0.01\\
41.18	0.01\\
41.19	0.01\\
41.2	0.01\\
41.21	0.01\\
41.22	0.01\\
41.23	0.01\\
41.24	0.01\\
41.25	0.01\\
41.26	0.01\\
41.27	0.01\\
41.28	0.01\\
41.29	0.01\\
41.3	0.01\\
41.31	0.01\\
41.32	0.01\\
41.33	0.01\\
41.34	0.01\\
41.35	0.01\\
41.36	0.01\\
41.37	0.01\\
41.38	0.01\\
41.39	0.01\\
41.4	0.01\\
41.41	0.01\\
41.42	0.01\\
41.43	0.01\\
41.44	0.01\\
41.45	0.01\\
41.46	0.01\\
41.47	0.01\\
41.48	0.01\\
41.49	0.01\\
41.5	0.01\\
41.51	0.01\\
41.52	0.01\\
41.53	0.01\\
41.54	0.01\\
41.55	0.01\\
41.56	0.01\\
41.57	0.01\\
41.58	0.01\\
41.59	0.01\\
41.6	0.01\\
41.61	0.01\\
41.62	0.01\\
41.63	0.01\\
41.64	0.01\\
41.65	0.01\\
41.66	0.01\\
41.67	0.01\\
41.68	0.01\\
41.69	0.01\\
41.7	0.01\\
41.71	0.01\\
41.72	0.01\\
41.73	0.01\\
41.74	0.01\\
41.75	0.01\\
41.76	0.01\\
41.77	0.01\\
41.78	0.01\\
41.79	0.01\\
41.8	0.01\\
41.81	0.01\\
41.82	0.01\\
41.83	0.01\\
41.84	0.01\\
41.85	0.01\\
41.86	0.01\\
41.87	0.01\\
41.88	0.01\\
41.89	0.01\\
41.9	0.01\\
41.91	0.01\\
41.92	0.01\\
41.93	0.01\\
41.94	0.01\\
41.95	0.01\\
41.96	0.01\\
41.97	0.01\\
41.98	0.01\\
41.99	0.01\\
42	0.01\\
42.01	0.01\\
42.02	0.01\\
42.03	0.01\\
42.04	0.01\\
42.05	0.01\\
42.06	0.01\\
42.07	0.01\\
42.08	0.01\\
42.09	0.01\\
42.1	0.01\\
42.11	0.01\\
42.12	0.01\\
42.13	0.01\\
42.14	0.01\\
42.15	0.01\\
42.16	0.01\\
42.17	0.01\\
42.18	0.01\\
42.19	0.01\\
42.2	0.01\\
42.21	0.01\\
42.22	0.01\\
42.23	0.01\\
42.24	0.01\\
42.25	0.01\\
42.26	0.01\\
42.27	0.01\\
42.28	0.01\\
42.29	0.01\\
42.3	0.01\\
42.31	0.01\\
42.32	0.01\\
42.33	0.01\\
42.34	0.01\\
42.35	0.01\\
42.36	0.01\\
42.37	0.01\\
42.38	0.01\\
42.39	0.01\\
42.4	0.01\\
42.41	0.01\\
42.42	0.01\\
42.43	0.01\\
42.44	0.01\\
42.45	0.01\\
42.46	0.01\\
42.47	0.01\\
42.48	0.01\\
42.49	0.01\\
42.5	0.01\\
42.51	0.01\\
42.52	0.01\\
42.53	0.01\\
42.54	0.01\\
42.55	0.01\\
42.56	0.01\\
42.57	0.01\\
42.58	0.01\\
42.59	0.01\\
42.6	0.01\\
42.61	0.01\\
42.62	0.01\\
42.63	0.01\\
42.64	0.01\\
42.65	0.01\\
42.66	0.01\\
42.67	0.01\\
42.68	0.01\\
42.69	0.01\\
42.7	0.01\\
42.71	0.01\\
42.72	0.01\\
42.73	0.01\\
42.74	0.01\\
42.75	0.01\\
42.76	0.01\\
42.77	0.01\\
42.78	0.01\\
42.79	0.01\\
42.8	0.01\\
42.81	0.01\\
42.82	0.01\\
42.83	0.01\\
42.84	0.01\\
42.85	0.01\\
42.86	0.01\\
42.87	0.01\\
42.88	0.01\\
42.89	0.01\\
42.9	0.01\\
42.91	0.01\\
42.92	0.01\\
42.93	0.01\\
42.94	0.01\\
42.95	0.01\\
42.96	0.01\\
42.97	0.01\\
42.98	0.01\\
42.99	0.01\\
43	0.01\\
43.01	0.01\\
43.02	0.01\\
43.03	0.01\\
43.04	0.01\\
43.05	0.01\\
43.06	0.01\\
43.07	0.01\\
43.08	0.01\\
43.09	0.01\\
43.1	0.01\\
43.11	0.01\\
43.12	0.01\\
43.13	0.01\\
43.14	0.01\\
43.15	0.01\\
43.16	0.01\\
43.17	0.01\\
43.18	0.01\\
43.19	0.01\\
43.2	0.01\\
43.21	0.01\\
43.22	0.01\\
43.23	0.01\\
43.24	0.01\\
43.25	0.01\\
43.26	0.01\\
43.27	0.01\\
43.28	0.01\\
43.29	0.01\\
43.3	0.01\\
43.31	0.01\\
43.32	0.01\\
43.33	0.01\\
43.34	0.01\\
43.35	0.01\\
43.36	0.01\\
43.37	0.01\\
43.38	0.01\\
43.39	0.01\\
43.4	0.01\\
43.41	0.01\\
43.42	0.01\\
43.43	0.01\\
43.44	0.01\\
43.45	0.01\\
43.46	0.01\\
43.47	0.01\\
43.48	0.01\\
43.49	0.01\\
43.5	0.01\\
43.51	0.01\\
43.52	0.01\\
43.53	0.01\\
43.54	0.01\\
43.55	0.01\\
43.56	0.01\\
43.57	0.01\\
43.58	0.01\\
43.59	0.01\\
43.6	0.01\\
43.61	0.01\\
43.62	0.01\\
43.63	0.01\\
43.64	0.01\\
43.65	0.01\\
43.66	0.01\\
43.67	0.01\\
43.68	0.01\\
43.69	0.01\\
43.7	0.01\\
43.71	0.01\\
43.72	0.01\\
43.73	0.01\\
43.74	0.01\\
43.75	0.01\\
43.76	0.01\\
43.77	0.01\\
43.78	0.01\\
43.79	0.01\\
43.8	0.01\\
43.81	0.01\\
43.82	0.01\\
43.83	0.01\\
43.84	0.01\\
43.85	0.01\\
43.86	0.01\\
43.87	0.01\\
43.88	0.01\\
43.89	0.01\\
43.9	0.01\\
43.91	0.01\\
43.92	0.01\\
43.93	0.01\\
43.94	0.01\\
43.95	0.01\\
43.96	0.01\\
43.97	0.01\\
43.98	0.01\\
43.99	0.01\\
44	0.01\\
44.01	0.01\\
44.02	0.01\\
44.03	0.01\\
44.04	0.01\\
44.05	0.01\\
44.06	0.01\\
44.07	0.01\\
44.08	0.01\\
44.09	0.01\\
44.1	0.01\\
44.11	0.01\\
44.12	0.01\\
44.13	0.01\\
44.14	0.01\\
44.15	0.01\\
44.16	0.01\\
44.17	0.01\\
44.18	0.01\\
44.19	0.01\\
44.2	0.01\\
44.21	0.01\\
44.22	0.01\\
44.23	0.01\\
44.24	0.01\\
44.25	0.01\\
44.26	0.01\\
44.27	0.01\\
44.28	0.01\\
44.29	0.01\\
44.3	0.01\\
44.31	0.01\\
44.32	0.01\\
44.33	0.01\\
44.34	0.01\\
44.35	0.01\\
44.36	0.01\\
44.37	0.01\\
44.38	0.01\\
44.39	0.01\\
44.4	0.01\\
44.41	0.01\\
44.42	0.01\\
44.43	0.01\\
44.44	0.01\\
44.45	0.01\\
44.46	0.01\\
44.47	0.01\\
44.48	0.01\\
44.49	0.01\\
44.5	0.01\\
44.51	0.01\\
44.52	0.01\\
44.53	0.01\\
44.54	0.01\\
44.55	0.01\\
44.56	0.01\\
44.57	0.01\\
44.58	0.01\\
44.59	0.01\\
44.6	0.01\\
44.61	0.01\\
44.62	0.01\\
44.63	0.01\\
44.64	0.01\\
44.65	0.01\\
44.66	0.01\\
44.67	0.01\\
44.68	0.01\\
44.69	0.01\\
44.7	0.01\\
44.71	0.01\\
44.72	0.01\\
44.73	0.01\\
44.74	0.01\\
44.75	0.01\\
44.76	0.01\\
44.77	0.01\\
44.78	0.01\\
44.79	0.01\\
44.8	0.01\\
44.81	0.01\\
44.82	0.01\\
44.83	0.01\\
44.84	0.01\\
44.85	0.01\\
44.86	0.01\\
44.87	0.01\\
44.88	0.01\\
44.89	0.01\\
44.9	0.01\\
44.91	0.01\\
44.92	0.01\\
44.93	0.01\\
44.94	0.01\\
44.95	0.01\\
44.96	0.01\\
44.97	0.01\\
44.98	0.01\\
44.99	0.01\\
45	0.01\\
45.01	0.01\\
45.02	0.01\\
45.03	0.01\\
45.04	0.01\\
45.05	0.01\\
45.06	0.01\\
45.07	0.01\\
45.08	0.01\\
45.09	0.01\\
45.1	0.01\\
45.11	0.01\\
45.12	0.01\\
45.13	0.01\\
45.14	0.01\\
45.15	0.01\\
45.16	0.01\\
45.17	0.01\\
45.18	0.01\\
45.19	0.01\\
45.2	0.01\\
45.21	0.01\\
45.22	0.01\\
45.23	0.01\\
45.24	0.01\\
45.25	0.01\\
45.26	0.01\\
45.27	0.01\\
45.28	0.01\\
45.29	0.01\\
45.3	0.01\\
45.31	0.01\\
45.32	0.01\\
45.33	0.01\\
45.34	0.01\\
45.35	0.01\\
45.36	0.01\\
45.37	0.01\\
45.38	0.01\\
45.39	0.01\\
45.4	0.01\\
45.41	0.01\\
45.42	0.01\\
45.43	0.01\\
45.44	0.01\\
45.45	0.01\\
45.46	0.01\\
45.47	0.01\\
45.48	0.01\\
45.49	0.01\\
45.5	0.01\\
45.51	0.01\\
45.52	0.01\\
45.53	0.01\\
45.54	0.01\\
45.55	0.01\\
45.56	0.01\\
45.57	0.01\\
45.58	0.01\\
45.59	0.01\\
45.6	0.01\\
45.61	0.01\\
45.62	0.01\\
45.63	0.01\\
45.64	0.01\\
45.65	0.01\\
45.66	0.01\\
45.67	0.01\\
45.68	0.01\\
45.69	0.01\\
45.7	0.01\\
45.71	0.01\\
45.72	0.01\\
45.73	0.01\\
45.74	0.01\\
45.75	0.01\\
45.76	0.01\\
45.77	0.01\\
45.78	0.01\\
45.79	0.01\\
45.8	0.01\\
45.81	0.01\\
45.82	0.01\\
45.83	0.01\\
45.84	0.01\\
45.85	0.01\\
45.86	0.01\\
45.87	0.01\\
45.88	0.01\\
45.89	0.01\\
45.9	0.01\\
45.91	0.01\\
45.92	0.01\\
45.93	0.01\\
45.94	0.01\\
45.95	0.01\\
45.96	0.01\\
45.97	0.01\\
45.98	0.01\\
45.99	0.01\\
46	0.01\\
46.01	0.01\\
46.02	0.01\\
46.03	0.01\\
46.04	0.01\\
46.05	0.01\\
46.06	0.01\\
46.07	0.01\\
46.08	0.01\\
46.09	0.01\\
46.1	0.01\\
46.11	0.01\\
46.12	0.01\\
46.13	0.01\\
46.14	0.01\\
46.15	0.01\\
46.16	0.01\\
46.17	0.01\\
46.18	0.01\\
46.19	0.01\\
46.2	0.01\\
46.21	0.01\\
46.22	0.01\\
46.23	0.01\\
46.24	0.01\\
46.25	0.01\\
46.26	0.01\\
46.27	0.01\\
46.28	0.01\\
46.29	0.01\\
46.3	0.01\\
46.31	0.01\\
46.32	0.01\\
46.33	0.01\\
46.34	0.01\\
46.35	0.01\\
46.36	0.01\\
46.37	0.01\\
46.38	0.01\\
46.39	0.01\\
46.4	0.01\\
46.41	0.01\\
46.42	0.01\\
46.43	0.01\\
46.44	0.01\\
46.45	0.01\\
46.46	0.01\\
46.47	0.01\\
46.48	0.01\\
46.49	0.01\\
46.5	0.01\\
46.51	0.01\\
46.52	0.01\\
46.53	0.01\\
46.54	0.01\\
46.55	0.01\\
46.56	0.01\\
46.57	0.01\\
46.58	0.01\\
46.59	0.01\\
46.6	0.01\\
46.61	0.01\\
46.62	0.01\\
46.63	0.01\\
46.64	0.01\\
46.65	0.01\\
46.66	0.01\\
46.67	0.01\\
46.68	0.01\\
46.69	0.01\\
46.7	0.01\\
46.71	0.01\\
46.72	0.01\\
46.73	0.01\\
46.74	0.01\\
46.75	0.01\\
46.76	0.01\\
46.77	0.01\\
46.78	0.01\\
46.79	0.01\\
46.8	0.01\\
46.81	0.01\\
46.82	0.01\\
46.83	0.01\\
46.84	0.01\\
46.85	0.01\\
46.86	0.01\\
46.87	0.01\\
46.88	0.01\\
46.89	0.01\\
46.9	0.01\\
46.91	0.01\\
46.92	0.01\\
46.93	0.01\\
46.94	0.01\\
46.95	0.01\\
46.96	0.01\\
46.97	0.01\\
46.98	0.01\\
46.99	0.01\\
47	0.01\\
47.01	0.01\\
47.02	0.01\\
47.03	0.01\\
47.04	0.01\\
47.05	0.01\\
47.06	0.01\\
47.07	0.01\\
47.08	0.01\\
47.09	0.01\\
47.1	0.01\\
47.11	0.01\\
47.12	0.01\\
47.13	0.01\\
47.14	0.01\\
47.15	0.01\\
47.16	0.01\\
47.17	0.01\\
47.18	0.01\\
47.19	0.01\\
47.2	0.01\\
47.21	0.01\\
47.22	0.01\\
47.23	0.01\\
47.24	0.01\\
47.25	0.01\\
47.26	0.01\\
47.27	0.01\\
47.28	0.01\\
47.29	0.01\\
47.3	0.01\\
47.31	0.01\\
47.32	0.01\\
47.33	0.01\\
47.34	0.01\\
47.35	0.01\\
47.36	0.01\\
47.37	0.01\\
47.38	0.01\\
47.39	0.01\\
47.4	0.01\\
47.41	0.01\\
47.42	0.01\\
47.43	0.01\\
47.44	0.01\\
47.45	0.01\\
47.46	0.01\\
47.47	0.01\\
47.48	0.01\\
47.49	0.01\\
47.5	0.01\\
47.51	0.01\\
47.52	0.01\\
47.53	0.01\\
47.54	0.01\\
47.55	0.01\\
47.56	0.01\\
47.57	0.01\\
47.58	0.01\\
47.59	0.01\\
47.6	0.01\\
47.61	0.01\\
47.62	0.01\\
47.63	0.01\\
47.64	0.01\\
47.65	0.01\\
47.66	0.01\\
47.67	0.01\\
47.68	0.01\\
47.69	0.01\\
47.7	0.01\\
47.71	0.01\\
47.72	0.01\\
47.73	0.01\\
47.74	0.01\\
47.75	0.01\\
47.76	0.01\\
47.77	0.01\\
47.78	0.01\\
47.79	0.01\\
47.8	0.01\\
47.81	0.01\\
47.82	0.01\\
47.83	0.01\\
47.84	0.01\\
47.85	0.01\\
47.86	0.01\\
47.87	0.01\\
47.88	0.01\\
47.89	0.01\\
47.9	0.01\\
47.91	0.01\\
47.92	0.01\\
47.93	0.01\\
47.94	0.01\\
47.95	0.01\\
47.96	0.01\\
47.97	0.01\\
47.98	0.01\\
47.99	0.01\\
48	0.01\\
48.01	0.01\\
48.02	0.01\\
48.03	0.01\\
48.04	0.01\\
48.05	0.01\\
48.06	0.01\\
48.07	0.01\\
48.08	0.01\\
48.09	0.01\\
48.1	0.01\\
48.11	0.01\\
48.12	0.01\\
48.13	0.01\\
48.14	0.01\\
48.15	0.01\\
48.16	0.01\\
48.17	0.01\\
48.18	0.01\\
48.19	0.01\\
48.2	0.01\\
48.21	0.01\\
48.22	0.01\\
48.23	0.01\\
48.24	0.01\\
48.25	0.01\\
48.26	0.01\\
48.27	0.01\\
48.28	0.01\\
48.29	0.01\\
48.3	0.01\\
48.31	0.01\\
48.32	0.01\\
48.33	0.01\\
48.34	0.01\\
48.35	0.01\\
48.36	0.01\\
48.37	0.01\\
48.38	0.01\\
48.39	0.01\\
48.4	0.01\\
48.41	0.01\\
48.42	0.01\\
48.43	0.01\\
48.44	0.01\\
48.45	0.01\\
48.46	0.01\\
48.47	0.01\\
48.48	0.01\\
48.49	0.01\\
48.5	0.01\\
48.51	0.01\\
48.52	0.01\\
48.53	0.01\\
48.54	0.01\\
48.55	0.01\\
48.56	0.01\\
48.57	0.01\\
48.58	0.01\\
48.59	0.01\\
48.6	0.01\\
48.61	0.01\\
48.62	0.01\\
48.63	0.01\\
48.64	0.01\\
48.65	0.01\\
48.66	0.01\\
48.67	0.01\\
48.68	0.01\\
48.69	0.01\\
48.7	0.01\\
48.71	0.01\\
48.72	0.01\\
48.73	0.01\\
48.74	0.01\\
48.75	0.01\\
48.76	0.01\\
48.77	0.01\\
48.78	0.01\\
48.79	0.01\\
48.8	0.01\\
48.81	0.01\\
48.82	0.01\\
48.83	0.01\\
48.84	0.01\\
48.85	0.01\\
48.86	0.01\\
48.87	0.01\\
48.88	0.01\\
48.89	0.01\\
48.9	0.01\\
48.91	0.01\\
48.92	0.01\\
48.93	0.01\\
48.94	0.01\\
48.95	0.01\\
48.96	0.01\\
48.97	0.01\\
48.98	0.01\\
48.99	0.01\\
49	0.01\\
49.01	0.01\\
49.02	0.01\\
49.03	0.01\\
49.04	0.01\\
49.05	0.01\\
49.06	0.01\\
49.07	0.01\\
49.08	0.01\\
49.09	0.01\\
49.1	0.01\\
49.11	0.01\\
49.12	0.01\\
49.13	0.01\\
49.14	0.01\\
49.15	0.01\\
49.16	0.01\\
49.17	0.01\\
49.18	0.01\\
49.19	0.01\\
49.2	0.01\\
49.21	0.01\\
49.22	0.01\\
49.23	0.01\\
49.24	0.01\\
49.25	0.01\\
49.26	0.01\\
49.27	0.01\\
49.28	0.01\\
49.29	0.01\\
49.3	0.01\\
49.31	0.01\\
49.32	0.01\\
49.33	0.01\\
49.34	0.01\\
49.35	0.01\\
49.36	0.01\\
49.37	0.01\\
49.38	0.01\\
49.39	0.01\\
49.4	0.01\\
49.41	0.01\\
49.42	0.01\\
49.43	0.01\\
49.44	0.01\\
49.45	0.01\\
49.46	0.01\\
49.47	0.01\\
49.48	0.01\\
49.49	0.01\\
49.5	0.01\\
49.51	0.01\\
49.52	0.01\\
49.53	0.01\\
49.54	0.01\\
49.55	0.01\\
49.56	0.01\\
49.57	0.01\\
49.58	0.01\\
49.59	0.01\\
49.6	0.01\\
49.61	0.01\\
49.62	0.01\\
49.63	0.01\\
49.64	0.01\\
49.65	0.01\\
49.66	0.01\\
49.67	0.01\\
49.68	0.01\\
49.69	0.01\\
49.7	0.01\\
49.71	0.01\\
49.72	0.01\\
49.73	0.01\\
49.74	0.01\\
49.75	0.01\\
49.76	0.01\\
49.77	0.01\\
49.78	0.01\\
49.79	0.01\\
49.8	0.01\\
49.81	0.01\\
49.82	0.01\\
49.83	0.01\\
49.84	0.01\\
49.85	0.01\\
49.86	0.01\\
49.87	0.01\\
49.88	0.01\\
49.89	0.01\\
49.9	0.01\\
49.91	0.01\\
49.92	0.01\\
49.93	0.01\\
49.94	0.01\\
49.95	0.01\\
49.96	0.01\\
49.97	0.01\\
49.98	0.01\\
49.99	0.01\\
50	0.01\\
50.01	0.01\\
50.02	0.01\\
50.03	0.01\\
50.04	0.01\\
50.05	0.01\\
50.06	0.01\\
50.07	0.01\\
50.08	0.01\\
50.09	0.01\\
50.1	0.01\\
50.11	0.01\\
50.12	0.01\\
50.13	0.01\\
50.14	0.01\\
50.15	0.01\\
50.16	0.01\\
50.17	0.01\\
50.18	0.01\\
50.19	0.01\\
50.2	0.01\\
50.21	0.01\\
50.22	0.01\\
50.23	0.01\\
50.24	0.01\\
50.25	0.01\\
50.26	0.01\\
50.27	0.01\\
50.28	0.01\\
50.29	0.01\\
50.3	0.01\\
50.31	0.01\\
50.32	0.01\\
50.33	0.01\\
50.34	0.01\\
50.35	0.01\\
50.36	0.01\\
50.37	0.01\\
50.38	0.01\\
50.39	0.01\\
50.4	0.01\\
50.41	0.01\\
50.42	0.01\\
50.43	0.01\\
50.44	0.01\\
50.45	0.01\\
50.46	0.01\\
50.47	0.01\\
50.48	0.01\\
50.49	0.01\\
50.5	0.01\\
50.51	0.01\\
50.52	0.01\\
50.53	0.01\\
50.54	0.01\\
50.55	0.01\\
50.56	0.01\\
50.57	0.01\\
50.58	0.01\\
50.59	0.01\\
50.6	0.01\\
50.61	0.01\\
50.62	0.01\\
50.63	0.01\\
50.64	0.01\\
50.65	0.01\\
50.66	0.01\\
50.67	0.01\\
50.68	0.01\\
50.69	0.01\\
50.7	0.01\\
50.71	0.01\\
50.72	0.01\\
50.73	0.01\\
50.74	0.01\\
50.75	0.01\\
50.76	0.01\\
50.77	0.01\\
50.78	0.01\\
50.79	0.01\\
50.8	0.01\\
50.81	0.01\\
50.82	0.01\\
50.83	0.01\\
50.84	0.01\\
50.85	0.01\\
50.86	0.01\\
50.87	0.01\\
50.88	0.01\\
50.89	0.01\\
50.9	0.01\\
50.91	0.01\\
50.92	0.01\\
50.93	0.01\\
50.94	0.01\\
50.95	0.01\\
50.96	0.01\\
50.97	0.01\\
50.98	0.01\\
50.99	0.01\\
51	0.01\\
51.01	0.01\\
51.02	0.01\\
51.03	0.01\\
51.04	0.01\\
51.05	0.01\\
51.06	0.01\\
51.07	0.01\\
51.08	0.01\\
51.09	0.01\\
51.1	0.01\\
51.11	0.01\\
51.12	0.01\\
51.13	0.01\\
51.14	0.01\\
51.15	0.01\\
51.16	0.01\\
51.17	0.01\\
51.18	0.01\\
51.19	0.01\\
51.2	0.01\\
51.21	0.01\\
51.22	0.01\\
51.23	0.01\\
51.24	0.01\\
51.25	0.01\\
51.26	0.01\\
51.27	0.01\\
51.28	0.01\\
51.29	0.01\\
51.3	0.01\\
51.31	0.01\\
51.32	0.01\\
51.33	0.01\\
51.34	0.01\\
51.35	0.01\\
51.36	0.01\\
51.37	0.01\\
51.38	0.01\\
51.39	0.01\\
51.4	0.01\\
51.41	0.01\\
51.42	0.01\\
51.43	0.01\\
51.44	0.01\\
51.45	0.01\\
51.46	0.01\\
51.47	0.01\\
51.48	0.01\\
51.49	0.01\\
51.5	0.01\\
51.51	0.01\\
51.52	0.01\\
51.53	0.01\\
51.54	0.01\\
51.55	0.01\\
51.56	0.01\\
51.57	0.01\\
51.58	0.01\\
51.59	0.01\\
51.6	0.01\\
51.61	0.01\\
51.62	0.01\\
51.63	0.01\\
51.64	0.01\\
51.65	0.01\\
51.66	0.01\\
51.67	0.01\\
51.68	0.01\\
51.69	0.01\\
51.7	0.01\\
51.71	0.01\\
51.72	0.01\\
51.73	0.01\\
51.74	0.01\\
51.75	0.01\\
51.76	0.01\\
51.77	0.01\\
51.78	0.01\\
51.79	0.01\\
51.8	0.01\\
51.81	0.01\\
51.82	0.01\\
51.83	0.01\\
51.84	0.01\\
51.85	0.01\\
51.86	0.01\\
51.87	0.01\\
51.88	0.01\\
51.89	0.01\\
51.9	0.01\\
51.91	0.01\\
51.92	0.01\\
51.93	0.01\\
51.94	0.01\\
51.95	0.01\\
51.96	0.01\\
51.97	0.01\\
51.98	0.01\\
51.99	0.01\\
52	0.01\\
52.01	0.01\\
52.02	0.01\\
52.03	0.01\\
52.04	0.01\\
52.05	0.01\\
52.06	0.01\\
52.07	0.01\\
52.08	0.01\\
52.09	0.01\\
52.1	0.01\\
52.11	0.01\\
52.12	0.01\\
52.13	0.01\\
52.14	0.01\\
52.15	0.01\\
52.16	0.01\\
52.17	0.01\\
52.18	0.01\\
52.19	0.01\\
52.2	0.01\\
52.21	0.01\\
52.22	0.01\\
52.23	0.01\\
52.24	0.01\\
52.25	0.01\\
52.26	0.01\\
52.27	0.01\\
52.28	0.01\\
52.29	0.01\\
52.3	0.01\\
52.31	0.01\\
52.32	0.01\\
52.33	0.01\\
52.34	0.01\\
52.35	0.01\\
52.36	0.01\\
52.37	0.01\\
52.38	0.01\\
52.39	0.01\\
52.4	0.01\\
52.41	0.01\\
52.42	0.01\\
52.43	0.01\\
52.44	0.01\\
52.45	0.01\\
52.46	0.01\\
52.47	0.01\\
52.48	0.01\\
52.49	0.01\\
52.5	0.01\\
52.51	0.01\\
52.52	0.01\\
52.53	0.01\\
52.54	0.01\\
52.55	0.01\\
52.56	0.01\\
52.57	0.01\\
52.58	0.01\\
52.59	0.01\\
52.6	0.01\\
52.61	0.01\\
52.62	0.01\\
52.63	0.01\\
52.64	0.01\\
52.65	0.01\\
52.66	0.01\\
52.67	0.01\\
52.68	0.01\\
52.69	0.01\\
52.7	0.01\\
52.71	0.01\\
52.72	0.01\\
52.73	0.01\\
52.74	0.01\\
52.75	0.01\\
52.76	0.01\\
52.77	0.01\\
52.78	0.01\\
52.79	0.01\\
52.8	0.01\\
52.81	0.01\\
52.82	0.01\\
52.83	0.01\\
52.84	0.01\\
52.85	0.01\\
52.86	0.01\\
52.87	0.01\\
52.88	0.01\\
52.89	0.01\\
52.9	0.01\\
52.91	0.01\\
52.92	0.01\\
52.93	0.01\\
52.94	0.01\\
52.95	0.01\\
52.96	0.01\\
52.97	0.01\\
52.98	0.01\\
52.99	0.01\\
53	0.01\\
53.01	0.01\\
53.02	0.01\\
53.03	0.01\\
53.04	0.01\\
53.05	0.01\\
53.06	0.01\\
53.07	0.01\\
53.08	0.01\\
53.09	0.01\\
53.1	0.01\\
53.11	0.01\\
53.12	0.01\\
53.13	0.01\\
53.14	0.01\\
53.15	0.01\\
53.16	0.01\\
53.17	0.01\\
53.18	0.01\\
53.19	0.01\\
53.2	0.01\\
53.21	0.01\\
53.22	0.01\\
53.23	0.01\\
53.24	0.01\\
53.25	0.01\\
53.26	0.01\\
53.27	0.01\\
53.28	0.01\\
53.29	0.01\\
53.3	0.01\\
53.31	0.01\\
53.32	0.01\\
53.33	0.01\\
53.34	0.01\\
53.35	0.01\\
53.36	0.01\\
53.37	0.01\\
53.38	0.01\\
53.39	0.01\\
53.4	0.01\\
53.41	0.01\\
53.42	0.01\\
53.43	0.01\\
53.44	0.01\\
53.45	0.01\\
53.46	0.01\\
53.47	0.01\\
53.48	0.01\\
53.49	0.01\\
53.5	0.01\\
53.51	0.01\\
53.52	0.01\\
53.53	0.01\\
53.54	0.01\\
53.55	0.01\\
53.56	0.01\\
53.57	0.01\\
53.58	0.01\\
53.59	0.01\\
53.6	0.01\\
53.61	0.01\\
53.62	0.01\\
53.63	0.01\\
53.64	0.01\\
53.65	0.01\\
53.66	0.01\\
53.67	0.01\\
53.68	0.01\\
53.69	0.01\\
53.7	0.01\\
53.71	0.01\\
53.72	0.01\\
53.73	0.01\\
53.74	0.01\\
53.75	0.01\\
53.76	0.01\\
53.77	0.01\\
53.78	0.01\\
53.79	0.01\\
53.8	0.01\\
53.81	0.01\\
53.82	0.01\\
53.83	0.01\\
53.84	0.01\\
53.85	0.01\\
53.86	0.01\\
53.87	0.01\\
53.88	0.01\\
53.89	0.01\\
53.9	0.01\\
53.91	0.01\\
53.92	0.01\\
53.93	0.01\\
53.94	0.01\\
53.95	0.01\\
53.96	0.01\\
53.97	0.01\\
53.98	0.01\\
53.99	0.01\\
54	0.01\\
54.01	0.01\\
54.02	0.01\\
54.03	0.01\\
54.04	0.01\\
54.05	0.01\\
54.06	0.01\\
54.07	0.01\\
54.08	0.01\\
54.09	0.01\\
54.1	0.01\\
54.11	0.01\\
54.12	0.01\\
54.13	0.01\\
54.14	0.01\\
54.15	0.01\\
54.16	0.01\\
54.17	0.01\\
54.18	0.01\\
54.19	0.01\\
54.2	0.01\\
54.21	0.01\\
54.22	0.01\\
54.23	0.01\\
54.24	0.01\\
54.25	0.01\\
54.26	0.01\\
54.27	0.01\\
54.28	0.01\\
54.29	0.01\\
54.3	0.01\\
54.31	0.01\\
54.32	0.01\\
54.33	0.01\\
54.34	0.01\\
54.35	0.01\\
54.36	0.01\\
54.37	0.01\\
54.38	0.01\\
54.39	0.01\\
54.4	0.01\\
54.41	0.01\\
54.42	0.01\\
54.43	0.01\\
54.44	0.01\\
54.45	0.01\\
54.46	0.01\\
54.47	0.01\\
54.48	0.01\\
54.49	0.01\\
54.5	0.01\\
54.51	0.01\\
54.52	0.01\\
54.53	0.01\\
54.54	0.01\\
54.55	0.01\\
54.56	0.01\\
54.57	0.01\\
54.58	0.01\\
54.59	0.01\\
54.6	0.01\\
54.61	0.01\\
54.62	0.01\\
54.63	0.01\\
54.64	0.01\\
54.65	0.01\\
54.66	0.01\\
54.67	0.01\\
54.68	0.01\\
54.69	0.01\\
54.7	0.01\\
54.71	0.01\\
54.72	0.01\\
54.73	0.01\\
54.74	0.01\\
54.75	0.01\\
54.76	0.01\\
54.77	0.01\\
54.78	0.01\\
54.79	0.01\\
54.8	0.01\\
54.81	0.01\\
54.82	0.01\\
54.83	0.01\\
54.84	0.01\\
54.85	0.01\\
54.86	0.01\\
54.87	0.01\\
54.88	0.01\\
54.89	0.01\\
54.9	0.01\\
54.91	0.01\\
54.92	0.01\\
54.93	0.01\\
54.94	0.01\\
54.95	0.01\\
54.96	0.01\\
54.97	0.01\\
54.98	0.01\\
54.99	0.01\\
55	0.01\\
55.01	0.01\\
55.02	0.01\\
55.03	0.01\\
55.04	0.01\\
55.05	0.01\\
55.06	0.01\\
55.07	0.01\\
55.08	0.01\\
55.09	0.01\\
55.1	0.01\\
55.11	0.01\\
55.12	0.01\\
55.13	0.01\\
55.14	0.01\\
55.15	0.01\\
55.16	0.01\\
55.17	0.01\\
55.18	0.01\\
55.19	0.01\\
55.2	0.01\\
55.21	0.01\\
55.22	0.01\\
55.23	0.01\\
55.24	0.01\\
55.25	0.01\\
55.26	0.01\\
55.27	0.01\\
55.28	0.01\\
55.29	0.01\\
55.3	0.01\\
55.31	0.01\\
55.32	0.01\\
55.33	0.01\\
55.34	0.01\\
55.35	0.01\\
55.36	0.01\\
55.37	0.01\\
55.38	0.01\\
55.39	0.01\\
55.4	0.01\\
55.41	0.01\\
55.42	0.01\\
55.43	0.01\\
55.44	0.01\\
55.45	0.01\\
55.46	0.01\\
55.47	0.01\\
55.48	0.01\\
55.49	0.01\\
55.5	0.01\\
55.51	0.01\\
55.52	0.01\\
55.53	0.01\\
55.54	0.01\\
55.55	0.01\\
55.56	0.01\\
55.57	0.01\\
55.58	0.01\\
55.59	0.01\\
55.6	0.01\\
55.61	0.01\\
55.62	0.01\\
55.63	0.01\\
55.64	0.01\\
55.65	0.01\\
55.66	0.01\\
55.67	0.01\\
55.68	0.01\\
55.69	0.01\\
55.7	0.01\\
55.71	0.01\\
55.72	0.01\\
55.73	0.01\\
55.74	0.01\\
55.75	0.01\\
55.76	0.01\\
55.77	0.01\\
55.78	0.01\\
55.79	0.01\\
55.8	0.01\\
55.81	0.01\\
55.82	0.01\\
55.83	0.01\\
55.84	0.01\\
55.85	0.01\\
55.86	0.01\\
55.87	0.01\\
55.88	0.01\\
55.89	0.01\\
55.9	0.01\\
55.91	0.01\\
55.92	0.01\\
55.93	0.01\\
55.94	0.01\\
55.95	0.01\\
55.96	0.01\\
55.97	0.01\\
55.98	0.01\\
55.99	0.01\\
56	0.01\\
56.01	0.01\\
56.02	0.01\\
56.03	0.01\\
56.04	0.01\\
56.05	0.01\\
56.06	0.01\\
56.07	0.01\\
56.08	0.01\\
56.09	0.01\\
56.1	0.01\\
56.11	0.01\\
56.12	0.01\\
56.13	0.01\\
56.14	0.01\\
56.15	0.01\\
56.16	0.01\\
56.17	0.01\\
56.18	0.01\\
56.19	0.01\\
56.2	0.01\\
56.21	0.01\\
56.22	0.01\\
56.23	0.01\\
56.24	0.01\\
56.25	0.01\\
56.26	0.01\\
56.27	0.01\\
56.28	0.01\\
56.29	0.01\\
56.3	0.01\\
56.31	0.01\\
56.32	0.01\\
56.33	0.01\\
56.34	0.01\\
56.35	0.01\\
56.36	0.01\\
56.37	0.01\\
56.38	0.01\\
56.39	0.01\\
56.4	0.01\\
56.41	0.01\\
56.42	0.01\\
56.43	0.01\\
56.44	0.01\\
56.45	0.01\\
56.46	0.01\\
56.47	0.01\\
56.48	0.01\\
56.49	0.01\\
56.5	0.01\\
56.51	0.01\\
56.52	0.01\\
56.53	0.01\\
56.54	0.01\\
56.55	0.01\\
56.56	0.01\\
56.57	0.01\\
56.58	0.01\\
56.59	0.01\\
56.6	0.01\\
56.61	0.01\\
56.62	0.01\\
56.63	0.01\\
56.64	0.01\\
56.65	0.01\\
56.66	0.01\\
56.67	0.01\\
56.68	0.01\\
56.69	0.01\\
56.7	0.01\\
56.71	0.01\\
56.72	0.01\\
56.73	0.01\\
56.74	0.01\\
56.75	0.01\\
56.76	0.01\\
56.77	0.01\\
56.78	0.01\\
56.79	0.01\\
56.8	0.01\\
56.81	0.01\\
56.82	0.01\\
56.83	0.01\\
56.84	0.01\\
56.85	0.01\\
56.86	0.01\\
56.87	0.01\\
56.88	0.01\\
56.89	0.01\\
56.9	0.01\\
56.91	0.01\\
56.92	0.01\\
56.93	0.01\\
56.94	0.01\\
56.95	0.01\\
56.96	0.01\\
56.97	0.01\\
56.98	0.01\\
56.99	0.01\\
57	0.01\\
57.01	0.01\\
57.02	0.01\\
57.03	0.01\\
57.04	0.01\\
57.05	0.01\\
57.06	0.01\\
57.07	0.01\\
57.08	0.01\\
57.09	0.01\\
57.1	0.01\\
57.11	0.01\\
57.12	0.01\\
57.13	0.01\\
57.14	0.01\\
57.15	0.01\\
57.16	0.01\\
57.17	0.01\\
57.18	0.01\\
57.19	0.01\\
57.2	0.01\\
57.21	0.01\\
57.22	0.01\\
57.23	0.01\\
57.24	0.01\\
57.25	0.01\\
57.26	0.01\\
57.27	0.01\\
57.28	0.01\\
57.29	0.01\\
57.3	0.01\\
57.31	0.01\\
57.32	0.01\\
57.33	0.01\\
57.34	0.01\\
57.35	0.01\\
57.36	0.01\\
57.37	0.01\\
57.38	0.01\\
57.39	0.01\\
57.4	0.01\\
57.41	0.01\\
57.42	0.01\\
57.43	0.01\\
57.44	0.01\\
57.45	0.01\\
57.46	0.01\\
57.47	0.01\\
57.48	0.01\\
57.49	0.01\\
57.5	0.01\\
57.51	0.01\\
57.52	0.01\\
57.53	0.01\\
57.54	0.01\\
57.55	0.01\\
57.56	0.01\\
57.57	0.01\\
57.58	0.01\\
57.59	0.01\\
57.6	0.01\\
57.61	0.01\\
57.62	0.01\\
57.63	0.01\\
57.64	0.01\\
57.65	0.01\\
57.66	0.01\\
57.67	0.01\\
57.68	0.01\\
57.69	0.01\\
57.7	0.01\\
57.71	0.01\\
57.72	0.01\\
57.73	0.01\\
57.74	0.01\\
57.75	0.01\\
57.76	0.01\\
57.77	0.01\\
57.78	0.01\\
57.79	0.01\\
57.8	0.01\\
57.81	0.01\\
57.82	0.01\\
57.83	0.01\\
57.84	0.01\\
57.85	0.01\\
57.86	0.01\\
57.87	0.01\\
57.88	0.01\\
57.89	0.01\\
57.9	0.01\\
57.91	0.01\\
57.92	0.01\\
57.93	0.01\\
57.94	0.01\\
57.95	0.01\\
57.96	0.01\\
57.97	0.01\\
57.98	0.01\\
57.99	0.01\\
58	0.01\\
58.01	0.01\\
58.02	0.01\\
58.03	0.01\\
58.04	0.01\\
58.05	0.01\\
58.06	0.01\\
58.07	0.01\\
58.08	0.01\\
58.09	0.01\\
58.1	0.01\\
58.11	0.01\\
58.12	0.01\\
58.13	0.01\\
58.14	0.01\\
58.15	0.01\\
58.16	0.01\\
58.17	0.01\\
58.18	0.01\\
58.19	0.01\\
58.2	0.01\\
58.21	0.01\\
58.22	0.01\\
58.23	0.01\\
58.24	0.01\\
58.25	0.01\\
58.26	0.01\\
58.27	0.01\\
58.28	0.01\\
58.29	0.01\\
58.3	0.01\\
58.31	0.01\\
58.32	0.01\\
58.33	0.01\\
58.34	0.01\\
58.35	0.01\\
58.36	0.01\\
58.37	0.01\\
58.38	0.01\\
58.39	0.01\\
58.4	0.01\\
58.41	0.01\\
58.42	0.01\\
58.43	0.01\\
58.44	0.01\\
58.45	0.01\\
58.46	0.01\\
58.47	0.01\\
58.48	0.01\\
58.49	0.01\\
58.5	0.01\\
58.51	0.01\\
58.52	0.01\\
58.53	0.01\\
58.54	0.01\\
58.55	0.01\\
58.56	0.01\\
58.57	0.01\\
58.58	0.01\\
58.59	0.01\\
58.6	0.01\\
58.61	0.01\\
58.62	0.01\\
58.63	0.01\\
58.64	0.01\\
58.65	0.01\\
58.66	0.01\\
58.67	0.01\\
58.68	0.01\\
58.69	0.01\\
58.7	0.01\\
58.71	0.01\\
58.72	0.01\\
58.73	0.01\\
58.74	0.01\\
58.75	0.01\\
58.76	0.01\\
58.77	0.01\\
58.78	0.01\\
58.79	0.01\\
58.8	0.01\\
58.81	0.01\\
58.82	0.01\\
58.83	0.01\\
58.84	0.01\\
58.85	0.01\\
58.86	0.01\\
58.87	0.01\\
58.88	0.01\\
58.89	0.01\\
58.9	0.01\\
58.91	0.01\\
58.92	0.01\\
58.93	0.01\\
58.94	0.01\\
58.95	0.01\\
58.96	0.01\\
58.97	0.01\\
58.98	0.01\\
58.99	0.01\\
59	0.01\\
59.01	0.01\\
59.02	0.01\\
59.03	0.01\\
59.04	0.01\\
59.05	0.01\\
59.06	0.01\\
59.07	0.01\\
59.08	0.01\\
59.09	0.01\\
59.1	0.01\\
59.11	0.01\\
59.12	0.01\\
59.13	0.01\\
59.14	0.01\\
59.15	0.01\\
59.16	0.01\\
59.17	0.01\\
59.18	0.01\\
59.19	0.01\\
59.2	0.01\\
59.21	0.01\\
59.22	0.01\\
59.23	0.01\\
59.24	0.01\\
59.25	0.01\\
59.26	0.01\\
59.27	0.01\\
59.28	0.01\\
59.29	0.01\\
59.3	0.01\\
59.31	0.01\\
59.32	0.01\\
59.33	0.01\\
59.34	0.01\\
59.35	0.01\\
59.36	0.01\\
59.37	0.01\\
59.38	0.01\\
59.39	0.01\\
59.4	0.01\\
59.41	0.01\\
59.42	0.01\\
59.43	0.01\\
59.44	0.01\\
59.45	0.01\\
59.46	0.01\\
59.47	0.01\\
59.48	0.01\\
59.49	0.01\\
59.5	0.01\\
59.51	0.01\\
59.52	0.01\\
59.53	0.01\\
59.54	0.01\\
59.55	0.01\\
59.56	0.01\\
59.57	0.01\\
59.58	0.01\\
59.59	0.01\\
59.6	0.01\\
59.61	0.01\\
59.62	0.01\\
59.63	0.01\\
59.64	0.01\\
59.65	0.01\\
59.66	0.01\\
59.67	0.01\\
59.68	0.01\\
59.69	0.01\\
59.7	0.01\\
59.71	0.01\\
59.72	0.01\\
59.73	0.01\\
59.74	0.01\\
59.75	0.01\\
59.76	0.01\\
59.77	0.01\\
59.78	0.01\\
59.79	0.01\\
59.8	0.01\\
59.81	0.01\\
59.82	0.01\\
59.83	0.01\\
59.84	0.01\\
59.85	0.01\\
59.86	0.01\\
59.87	0.01\\
59.88	0.01\\
59.89	0.01\\
59.9	0.01\\
59.91	0.01\\
59.92	0.01\\
59.93	0.01\\
59.94	0.01\\
59.95	0.01\\
59.96	0.01\\
59.97	0.01\\
59.98	0.01\\
59.99	0.01\\
60	0.01\\
60.01	0.01\\
60.02	0.01\\
60.03	0.01\\
60.04	0.01\\
60.05	0.01\\
60.06	0.01\\
60.07	0.01\\
60.08	0.01\\
60.09	0.01\\
60.1	0.01\\
60.11	0.01\\
60.12	0.01\\
60.13	0.01\\
60.14	0.01\\
60.15	0.01\\
60.16	0.01\\
60.17	0.01\\
60.18	0.01\\
60.19	0.01\\
60.2	0.01\\
60.21	0.01\\
60.22	0.01\\
60.23	0.01\\
60.24	0.01\\
60.25	0.01\\
60.26	0.01\\
60.27	0.01\\
60.28	0.01\\
60.29	0.01\\
60.3	0.01\\
60.31	0.01\\
60.32	0.01\\
60.33	0.01\\
60.34	0.01\\
60.35	0.01\\
60.36	0.01\\
60.37	0.01\\
60.38	0.01\\
60.39	0.01\\
60.4	0.01\\
60.41	0.01\\
60.42	0.01\\
60.43	0.01\\
60.44	0.01\\
60.45	0.01\\
60.46	0.01\\
60.47	0.01\\
60.48	0.01\\
60.49	0.01\\
60.5	0.01\\
60.51	0.01\\
60.52	0.01\\
60.53	0.01\\
60.54	0.01\\
60.55	0.01\\
60.56	0.01\\
60.57	0.01\\
60.58	0.01\\
60.59	0.01\\
60.6	0.01\\
60.61	0.01\\
60.62	0.01\\
60.63	0.01\\
60.64	0.01\\
60.65	0.01\\
60.66	0.01\\
60.67	0.01\\
60.68	0.01\\
60.69	0.01\\
60.7	0.01\\
60.71	0.01\\
60.72	0.01\\
60.73	0.01\\
60.74	0.01\\
60.75	0.01\\
60.76	0.01\\
60.77	0.01\\
60.78	0.01\\
60.79	0.01\\
60.8	0.01\\
60.81	0.01\\
60.82	0.01\\
60.83	0.01\\
60.84	0.01\\
60.85	0.01\\
60.86	0.01\\
60.87	0.01\\
60.88	0.01\\
60.89	0.01\\
60.9	0.01\\
60.91	0.01\\
60.92	0.01\\
60.93	0.01\\
60.94	0.01\\
60.95	0.01\\
60.96	0.01\\
60.97	0.01\\
60.98	0.01\\
60.99	0.01\\
61	0.01\\
61.01	0.01\\
61.02	0.01\\
61.03	0.01\\
61.04	0.01\\
61.05	0.01\\
61.06	0.01\\
61.07	0.01\\
61.08	0.01\\
61.09	0.01\\
61.1	0.01\\
61.11	0.01\\
61.12	0.01\\
61.13	0.01\\
61.14	0.01\\
61.15	0.01\\
61.16	0.01\\
61.17	0.01\\
61.18	0.01\\
61.19	0.01\\
61.2	0.01\\
61.21	0.01\\
61.22	0.01\\
61.23	0.01\\
61.24	0.01\\
61.25	0.01\\
61.26	0.01\\
61.27	0.01\\
61.28	0.01\\
61.29	0.01\\
61.3	0.01\\
61.31	0.01\\
61.32	0.01\\
61.33	0.01\\
61.34	0.01\\
61.35	0.01\\
61.36	0.01\\
61.37	0.01\\
61.38	0.01\\
61.39	0.01\\
61.4	0.01\\
61.41	0.01\\
61.42	0.01\\
61.43	0.01\\
61.44	0.01\\
61.45	0.01\\
61.46	0.01\\
61.47	0.01\\
61.48	0.01\\
61.49	0.01\\
61.5	0.01\\
61.51	0.01\\
61.52	0.01\\
61.53	0.01\\
61.54	0.01\\
61.55	0.01\\
61.56	0.01\\
61.57	0.01\\
61.58	0.01\\
61.59	0.01\\
61.6	0.01\\
61.61	0.01\\
61.62	0.01\\
61.63	0.01\\
61.64	0.01\\
61.65	0.01\\
61.66	0.01\\
61.67	0.01\\
61.68	0.01\\
61.69	0.01\\
61.7	0.01\\
61.71	0.01\\
61.72	0.01\\
61.73	0.01\\
61.74	0.01\\
61.75	0.01\\
61.76	0.01\\
61.77	0.01\\
61.78	0.01\\
61.79	0.01\\
61.8	0.01\\
61.81	0.01\\
61.82	0.01\\
61.83	0.01\\
61.84	0.01\\
61.85	0.01\\
61.86	0.01\\
61.87	0.01\\
61.88	0.01\\
61.89	0.01\\
61.9	0.01\\
61.91	0.01\\
61.92	0.01\\
61.93	0.01\\
61.94	0.01\\
61.95	0.01\\
61.96	0.01\\
61.97	0.01\\
61.98	0.01\\
61.99	0.01\\
62	0.01\\
62.01	0.01\\
62.02	0.01\\
62.03	0.01\\
62.04	0.01\\
62.05	0.01\\
62.06	0.01\\
62.07	0.01\\
62.08	0.01\\
62.09	0.01\\
62.1	0.01\\
62.11	0.01\\
62.12	0.01\\
62.13	0.01\\
62.14	0.01\\
62.15	0.01\\
62.16	0.01\\
62.17	0.01\\
62.18	0.01\\
62.19	0.01\\
62.2	0.01\\
62.21	0.01\\
62.22	0.01\\
62.23	0.01\\
62.24	0.01\\
62.25	0.01\\
62.26	0.01\\
62.27	0.01\\
62.28	0.01\\
62.29	0.01\\
62.3	0.01\\
62.31	0.01\\
62.32	0.01\\
62.33	0.01\\
62.34	0.01\\
62.35	0.01\\
62.36	0.01\\
62.37	0.01\\
62.38	0.01\\
62.39	0.01\\
62.4	0.01\\
62.41	0.01\\
62.42	0.01\\
62.43	0.01\\
62.44	0.01\\
62.45	0.01\\
62.46	0.01\\
62.47	0.01\\
62.48	0.01\\
62.49	0.01\\
62.5	0.01\\
62.51	0.01\\
62.52	0.01\\
62.53	0.01\\
62.54	0.01\\
62.55	0.01\\
62.56	0.01\\
62.57	0.01\\
62.58	0.01\\
62.59	0.01\\
62.6	0.01\\
62.61	0.01\\
62.62	0.01\\
62.63	0.01\\
62.64	0.01\\
62.65	0.01\\
62.66	0.01\\
62.67	0.01\\
62.68	0.01\\
62.69	0.01\\
62.7	0.01\\
62.71	0.01\\
62.72	0.01\\
62.73	0.01\\
62.74	0.01\\
62.75	0.01\\
62.76	0.01\\
62.77	0.01\\
62.78	0.01\\
62.79	0.01\\
62.8	0.01\\
62.81	0.01\\
62.82	0.01\\
62.83	0.01\\
62.84	0.01\\
62.85	0.01\\
62.86	0.01\\
62.87	0.01\\
62.88	0.01\\
62.89	0.01\\
62.9	0.01\\
62.91	0.01\\
62.92	0.01\\
62.93	0.01\\
62.94	0.01\\
62.95	0.01\\
62.96	0.01\\
62.97	0.01\\
62.98	0.01\\
62.99	0.01\\
63	0.01\\
63.01	0.01\\
63.02	0.01\\
63.03	0.01\\
63.04	0.01\\
63.05	0.01\\
63.06	0.01\\
63.07	0.01\\
63.08	0.01\\
63.09	0.01\\
63.1	0.01\\
63.11	0.01\\
63.12	0.01\\
63.13	0.01\\
63.14	0.01\\
63.15	0.01\\
63.16	0.01\\
63.17	0.01\\
63.18	0.01\\
63.19	0.01\\
63.2	0.01\\
63.21	0.01\\
63.22	0.01\\
63.23	0.01\\
63.24	0.01\\
63.25	0.01\\
63.26	0.01\\
63.27	0.01\\
63.28	0.01\\
63.29	0.01\\
63.3	0.01\\
63.31	0.01\\
63.32	0.01\\
63.33	0.01\\
63.34	0.01\\
63.35	0.01\\
63.36	0.01\\
63.37	0.01\\
63.38	0.01\\
63.39	0.01\\
63.4	0.01\\
63.41	0.01\\
63.42	0.01\\
63.43	0.01\\
63.44	0.01\\
63.45	0.01\\
63.46	0.01\\
63.47	0.01\\
63.48	0.01\\
63.49	0.01\\
63.5	0.01\\
63.51	0.01\\
63.52	0.01\\
63.53	0.01\\
63.54	0.01\\
63.55	0.01\\
63.56	0.01\\
63.57	0.01\\
63.58	0.01\\
63.59	0.01\\
63.6	0.01\\
63.61	0.01\\
63.62	0.01\\
63.63	0.01\\
63.64	0.01\\
63.65	0.01\\
63.66	0.01\\
63.67	0.01\\
63.68	0.01\\
63.69	0.01\\
63.7	0.01\\
63.71	0.01\\
63.72	0.01\\
63.73	0.01\\
63.74	0.01\\
63.75	0.01\\
63.76	0.01\\
63.77	0.01\\
63.78	0.01\\
63.79	0.01\\
63.8	0.01\\
63.81	0.01\\
63.82	0.01\\
63.83	0.01\\
63.84	0.01\\
63.85	0.01\\
63.86	0.01\\
63.87	0.01\\
63.88	0.01\\
63.89	0.01\\
63.9	0.01\\
63.91	0.01\\
63.92	0.01\\
63.93	0.01\\
63.94	0.01\\
63.95	0.01\\
63.96	0.01\\
63.97	0.01\\
63.98	0.01\\
63.99	0.01\\
64	0.01\\
64.01	0.01\\
64.02	0.01\\
64.03	0.01\\
64.04	0.01\\
64.05	0.01\\
64.06	0.01\\
64.07	0.01\\
64.08	0.01\\
64.09	0.01\\
64.1	0.01\\
64.11	0.01\\
64.12	0.01\\
64.13	0.01\\
64.14	0.01\\
64.15	0.01\\
64.16	0.01\\
64.17	0.01\\
64.18	0.01\\
64.19	0.01\\
64.2	0.01\\
64.21	0.01\\
64.22	0.01\\
64.23	0.01\\
64.24	0.01\\
64.25	0.01\\
64.26	0.01\\
64.27	0.01\\
64.28	0.01\\
64.29	0.01\\
64.3	0.01\\
64.31	0.01\\
64.32	0.01\\
64.33	0.01\\
64.34	0.01\\
64.35	0.01\\
64.36	0.01\\
64.37	0.01\\
64.38	0.01\\
64.39	0.01\\
64.4	0.01\\
64.41	0.01\\
64.42	0.01\\
64.43	0.01\\
64.44	0.01\\
64.45	0.01\\
64.46	0.01\\
64.47	0.01\\
64.48	0.01\\
64.49	0.01\\
64.5	0.01\\
64.51	0.01\\
64.52	0.01\\
64.53	0.01\\
64.54	0.01\\
64.55	0.01\\
64.56	0.01\\
64.57	0.01\\
64.58	0.01\\
64.59	0.01\\
64.6	0.01\\
64.61	0.01\\
64.62	0.01\\
64.63	0.01\\
64.64	0.01\\
64.65	0.01\\
64.66	0.01\\
64.67	0.01\\
64.68	0.01\\
64.69	0.01\\
64.7	0.01\\
64.71	0.01\\
64.72	0.01\\
64.73	0.01\\
64.74	0.01\\
64.75	0.01\\
64.76	0.01\\
64.77	0.01\\
64.78	0.01\\
64.79	0.01\\
64.8	0.01\\
64.81	0.01\\
64.82	0.01\\
64.83	0.01\\
64.84	0.01\\
64.85	0.01\\
64.86	0.01\\
64.87	0.01\\
64.88	0.01\\
64.89	0.01\\
64.9	0.01\\
64.91	0.01\\
64.92	0.01\\
64.93	0.01\\
64.94	0.01\\
64.95	0.01\\
64.96	0.01\\
64.97	0.01\\
64.98	0.01\\
64.99	0.01\\
65	0.01\\
65.01	0.01\\
65.02	0.01\\
65.03	0.01\\
65.04	0.01\\
65.05	0.01\\
65.06	0.01\\
65.07	0.01\\
65.08	0.01\\
65.09	0.01\\
65.1	0.01\\
65.11	0.01\\
65.12	0.01\\
65.13	0.01\\
65.14	0.01\\
65.15	0.01\\
65.16	0.01\\
65.17	0.01\\
65.18	0.01\\
65.19	0.01\\
65.2	0.01\\
65.21	0.01\\
65.22	0.01\\
65.23	0.01\\
65.24	0.01\\
65.25	0.01\\
65.26	0.01\\
65.27	0.01\\
65.28	0.01\\
65.29	0.01\\
65.3	0.01\\
65.31	0.01\\
65.32	0.01\\
65.33	0.01\\
65.34	0.01\\
65.35	0.01\\
65.36	0.01\\
65.37	0.01\\
65.38	0.01\\
65.39	0.01\\
65.4	0.01\\
65.41	0.01\\
65.42	0.01\\
65.43	0.01\\
65.44	0.01\\
65.45	0.01\\
65.46	0.01\\
65.47	0.01\\
65.48	0.01\\
65.49	0.01\\
65.5	0.01\\
65.51	0.01\\
65.52	0.01\\
65.53	0.01\\
65.54	0.01\\
65.55	0.01\\
65.56	0.01\\
65.57	0.01\\
65.58	0.01\\
65.59	0.01\\
65.6	0.01\\
65.61	0.01\\
65.62	0.01\\
65.63	0.01\\
65.64	0.01\\
65.65	0.01\\
65.66	0.01\\
65.67	0.01\\
65.68	0.01\\
65.69	0.01\\
65.7	0.01\\
65.71	0.01\\
65.72	0.01\\
65.73	0.01\\
65.74	0.01\\
65.75	0.01\\
65.76	0.01\\
65.77	0.01\\
65.78	0.01\\
65.79	0.01\\
65.8	0.01\\
65.81	0.01\\
65.82	0.01\\
65.83	0.01\\
65.84	0.01\\
65.85	0.01\\
65.86	0.01\\
65.87	0.01\\
65.88	0.01\\
65.89	0.01\\
65.9	0.01\\
65.91	0.01\\
65.92	0.01\\
65.93	0.01\\
65.94	0.01\\
65.95	0.01\\
65.96	0.01\\
65.97	0.01\\
65.98	0.01\\
65.99	0.01\\
66	0.01\\
66.01	0.01\\
66.02	0.01\\
66.03	0.01\\
66.04	0.01\\
66.05	0.01\\
66.06	0.01\\
66.07	0.01\\
66.08	0.01\\
66.09	0.01\\
66.1	0.01\\
66.11	0.01\\
66.12	0.01\\
66.13	0.01\\
66.14	0.01\\
66.15	0.01\\
66.16	0.01\\
66.17	0.01\\
66.18	0.01\\
66.19	0.01\\
66.2	0.01\\
66.21	0.01\\
66.22	0.01\\
66.23	0.01\\
66.24	0.01\\
66.25	0.01\\
66.26	0.01\\
66.27	0.01\\
66.28	0.01\\
66.29	0.01\\
66.3	0.01\\
66.31	0.01\\
66.32	0.01\\
66.33	0.01\\
66.34	0.01\\
66.35	0.01\\
66.36	0.01\\
66.37	0.01\\
66.38	0.01\\
66.39	0.01\\
66.4	0.01\\
66.41	0.01\\
66.42	0.01\\
66.43	0.01\\
66.44	0.01\\
66.45	0.01\\
66.46	0.01\\
66.47	0.01\\
66.48	0.01\\
66.49	0.01\\
66.5	0.01\\
66.51	0.01\\
66.52	0.01\\
66.53	0.01\\
66.54	0.01\\
66.55	0.01\\
66.56	0.01\\
66.57	0.01\\
66.58	0.01\\
66.59	0.01\\
66.6	0.01\\
66.61	0.01\\
66.62	0.01\\
66.63	0.01\\
66.64	0.01\\
66.65	0.01\\
66.66	0.01\\
66.67	0.01\\
66.68	0.01\\
66.69	0.01\\
66.7	0.01\\
66.71	0.01\\
66.72	0.01\\
66.73	0.01\\
66.74	0.01\\
66.75	0.01\\
66.76	0.01\\
66.77	0.01\\
66.78	0.01\\
66.79	0.01\\
66.8	0.01\\
66.81	0.01\\
66.82	0.01\\
66.83	0.01\\
66.84	0.01\\
66.85	0.01\\
66.86	0.01\\
66.87	0.01\\
66.88	0.01\\
66.89	0.01\\
66.9	0.01\\
66.91	0.01\\
66.92	0.01\\
66.93	0.01\\
66.94	0.01\\
66.95	0.01\\
66.96	0.01\\
66.97	0.01\\
66.98	0.01\\
66.99	0.01\\
67	0.01\\
67.01	0.01\\
67.02	0.01\\
67.03	0.01\\
67.04	0.01\\
67.05	0.01\\
67.06	0.01\\
67.07	0.01\\
67.08	0.01\\
67.09	0.01\\
67.1	0.01\\
67.11	0.01\\
67.12	0.01\\
67.13	0.01\\
67.14	0.01\\
67.15	0.01\\
67.16	0.01\\
67.17	0.01\\
67.18	0.01\\
67.19	0.01\\
67.2	0.01\\
67.21	0.01\\
67.22	0.01\\
67.23	0.01\\
67.24	0.01\\
67.25	0.01\\
67.26	0.01\\
67.27	0.01\\
67.28	0.01\\
67.29	0.01\\
67.3	0.01\\
67.31	0.01\\
67.32	0.01\\
67.33	0.01\\
67.34	0.01\\
67.35	0.01\\
67.36	0.01\\
67.37	0.01\\
67.38	0.01\\
67.39	0.01\\
67.4	0.01\\
67.41	0.01\\
67.42	0.01\\
67.43	0.01\\
67.44	0.01\\
67.45	0.01\\
67.46	0.01\\
67.47	0.01\\
67.48	0.01\\
67.49	0.01\\
67.5	0.01\\
67.51	0.01\\
67.52	0.01\\
67.53	0.01\\
67.54	0.01\\
67.55	0.01\\
67.56	0.01\\
67.57	0.01\\
67.58	0.01\\
67.59	0.01\\
67.6	0.01\\
67.61	0.01\\
67.62	0.01\\
67.63	0.01\\
67.64	0.01\\
67.65	0.01\\
67.66	0.01\\
67.67	0.01\\
67.68	0.01\\
67.69	0.01\\
67.7	0.01\\
67.71	0.01\\
67.72	0.01\\
67.73	0.01\\
67.74	0.01\\
67.75	0.01\\
67.76	0.01\\
67.77	0.01\\
67.78	0.01\\
67.79	0.01\\
67.8	0.01\\
67.81	0.01\\
67.82	0.01\\
67.83	0.01\\
67.84	0.01\\
67.85	0.01\\
67.86	0.01\\
67.87	0.01\\
67.88	0.01\\
67.89	0.01\\
67.9	0.01\\
67.91	0.01\\
67.92	0.01\\
67.93	0.01\\
67.94	0.01\\
67.95	0.01\\
67.96	0.01\\
67.97	0.01\\
67.98	0.01\\
67.99	0.01\\
68	0.01\\
68.01	0.01\\
68.02	0.01\\
68.03	0.01\\
68.04	0.01\\
68.05	0.01\\
68.06	0.01\\
68.07	0.01\\
68.08	0.01\\
68.09	0.01\\
68.1	0.01\\
68.11	0.01\\
68.12	0.01\\
68.13	0.01\\
68.14	0.01\\
68.15	0.01\\
68.16	0.01\\
68.17	0.01\\
68.18	0.01\\
68.19	0.01\\
68.2	0.01\\
68.21	0.01\\
68.22	0.01\\
68.23	0.01\\
68.24	0.01\\
68.25	0.01\\
68.26	0.01\\
68.27	0.01\\
68.28	0.01\\
68.29	0.01\\
68.3	0.01\\
68.31	0.01\\
68.32	0.01\\
68.33	0.01\\
68.34	0.01\\
68.35	0.01\\
68.36	0.01\\
68.37	0.01\\
68.38	0.01\\
68.39	0.01\\
68.4	0.01\\
68.41	0.01\\
68.42	0.01\\
68.43	0.01\\
68.44	0.01\\
68.45	0.01\\
68.46	0.01\\
68.47	0.01\\
68.48	0.01\\
68.49	0.01\\
68.5	0.01\\
68.51	0.01\\
68.52	0.01\\
68.53	0.01\\
68.54	0.01\\
68.55	0.01\\
68.56	0.01\\
68.57	0.01\\
68.58	0.01\\
68.59	0.01\\
68.6	0.01\\
68.61	0.01\\
68.62	0.01\\
68.63	0.01\\
68.64	0.01\\
68.65	0.01\\
68.66	0.01\\
68.67	0.01\\
68.68	0.01\\
68.69	0.01\\
68.7	0.01\\
68.71	0.01\\
68.72	0.01\\
68.73	0.01\\
68.74	0.01\\
68.75	0.01\\
68.76	0.01\\
68.77	0.01\\
68.78	0.01\\
68.79	0.01\\
68.8	0.01\\
68.81	0.01\\
68.82	0.01\\
68.83	0.01\\
68.84	0.01\\
68.85	0.01\\
68.86	0.01\\
68.87	0.01\\
68.88	0.01\\
68.89	0.01\\
68.9	0.01\\
68.91	0.01\\
68.92	0.01\\
68.93	0.01\\
68.94	0.01\\
68.95	0.01\\
68.96	0.01\\
68.97	0.01\\
68.98	0.01\\
68.99	0.01\\
69	0.01\\
69.01	0.01\\
69.02	0.01\\
69.03	0.01\\
69.04	0.01\\
69.05	0.01\\
69.06	0.01\\
69.07	0.01\\
69.08	0.01\\
69.09	0.01\\
69.1	0.01\\
69.11	0.01\\
69.12	0.01\\
69.13	0.01\\
69.14	0.01\\
69.15	0.01\\
69.16	0.01\\
69.17	0.01\\
69.18	0.01\\
69.19	0.01\\
69.2	0.01\\
69.21	0.01\\
69.22	0.01\\
69.23	0.01\\
69.24	0.01\\
69.25	0.01\\
69.26	0.01\\
69.27	0.01\\
69.28	0.01\\
69.29	0.01\\
69.3	0.01\\
69.31	0.01\\
69.32	0.01\\
69.33	0.01\\
69.34	0.01\\
69.35	0.01\\
69.36	0.01\\
69.37	0.01\\
69.38	0.01\\
69.39	0.01\\
69.4	0.01\\
69.41	0.01\\
69.42	0.01\\
69.43	0.01\\
69.44	0.01\\
69.45	0.01\\
69.46	0.01\\
69.47	0.01\\
69.48	0.01\\
69.49	0.01\\
69.5	0.01\\
69.51	0.01\\
69.52	0.01\\
69.53	0.01\\
69.54	0.01\\
69.55	0.01\\
69.56	0.01\\
69.57	0.01\\
69.58	0.01\\
69.59	0.01\\
69.6	0.01\\
69.61	0.01\\
69.62	0.01\\
69.63	0.01\\
69.64	0.01\\
69.65	0.01\\
69.66	0.01\\
69.67	0.01\\
69.68	0.01\\
69.69	0.01\\
69.7	0.01\\
69.71	0.01\\
69.72	0.01\\
69.73	0.01\\
69.74	0.01\\
69.75	0.01\\
69.76	0.01\\
69.77	0.01\\
69.78	0.01\\
69.79	0.01\\
69.8	0.01\\
69.81	0.01\\
69.82	0.01\\
69.83	0.01\\
69.84	0.01\\
69.85	0.01\\
69.86	0.01\\
69.87	0.01\\
69.88	0.01\\
69.89	0.01\\
69.9	0.01\\
69.91	0.01\\
69.92	0.01\\
69.93	0.01\\
69.94	0.01\\
69.95	0.01\\
69.96	0.01\\
69.97	0.01\\
69.98	0.01\\
69.99	0.01\\
70	0.01\\
70.01	0.01\\
70.02	0.01\\
70.03	0.01\\
70.04	0.01\\
70.05	0.01\\
70.06	0.01\\
70.07	0.01\\
70.08	0.01\\
70.09	0.01\\
70.1	0.01\\
70.11	0.01\\
70.12	0.01\\
70.13	0.01\\
70.14	0.01\\
70.15	0.01\\
70.16	0.01\\
70.17	0.01\\
70.18	0.01\\
70.19	0.01\\
70.2	0.01\\
70.21	0.01\\
70.22	0.01\\
70.23	0.01\\
70.24	0.01\\
70.25	0.01\\
70.26	0.01\\
70.27	0.01\\
70.28	0.01\\
70.29	0.01\\
70.3	0.01\\
70.31	0.01\\
70.32	0.01\\
70.33	0.01\\
70.34	0.01\\
70.35	0.01\\
70.36	0.01\\
70.37	0.01\\
70.38	0.01\\
70.39	0.01\\
70.4	0.01\\
70.41	0.01\\
70.42	0.01\\
70.43	0.01\\
70.44	0.01\\
70.45	0.01\\
70.46	0.01\\
70.47	0.01\\
70.48	0.01\\
70.49	0.01\\
70.5	0.01\\
70.51	0.01\\
70.52	0.01\\
70.53	0.01\\
70.54	0.01\\
70.55	0.01\\
70.56	0.01\\
70.57	0.01\\
70.58	0.01\\
70.59	0.01\\
70.6	0.01\\
70.61	0.01\\
70.62	0.01\\
70.63	0.01\\
70.64	0.01\\
70.65	0.01\\
70.66	0.01\\
70.67	0.01\\
70.68	0.01\\
70.69	0.01\\
70.7	0.01\\
70.71	0.01\\
70.72	0.01\\
70.73	0.01\\
70.74	0.01\\
70.75	0.01\\
70.76	0.01\\
70.77	0.01\\
70.78	0.01\\
70.79	0.01\\
70.8	0.01\\
70.81	0.01\\
70.82	0.01\\
70.83	0.01\\
70.84	0.01\\
70.85	0.01\\
70.86	0.01\\
70.87	0.01\\
70.88	0.01\\
70.89	0.01\\
70.9	0.01\\
70.91	0.01\\
70.92	0.01\\
70.93	0.01\\
70.94	0.01\\
70.95	0.01\\
70.96	0.01\\
70.97	0.01\\
70.98	0.01\\
70.99	0.01\\
71	0.01\\
71.01	0.01\\
71.02	0.01\\
71.03	0.01\\
71.04	0.01\\
71.05	0.01\\
71.06	0.01\\
71.07	0.01\\
71.08	0.01\\
71.09	0.01\\
71.1	0.01\\
71.11	0.01\\
71.12	0.01\\
71.13	0.01\\
71.14	0.01\\
71.15	0.01\\
71.16	0.01\\
71.17	0.01\\
71.18	0.01\\
71.19	0.01\\
71.2	0.01\\
71.21	0.01\\
71.22	0.01\\
71.23	0.01\\
71.24	0.01\\
71.25	0.01\\
71.26	0.01\\
71.27	0.01\\
71.28	0.01\\
71.29	0.01\\
71.3	0.01\\
71.31	0.01\\
71.32	0.01\\
71.33	0.01\\
71.34	0.01\\
71.35	0.01\\
71.36	0.01\\
71.37	0.01\\
71.38	0.01\\
71.39	0.01\\
71.4	0.01\\
71.41	0.01\\
71.42	0.01\\
71.43	0.01\\
71.44	0.01\\
71.45	0.01\\
71.46	0.01\\
71.47	0.01\\
71.48	0.01\\
71.49	0.01\\
71.5	0.01\\
71.51	0.01\\
71.52	0.01\\
71.53	0.01\\
71.54	0.01\\
71.55	0.01\\
71.56	0.01\\
71.57	0.01\\
71.58	0.01\\
71.59	0.01\\
71.6	0.01\\
71.61	0.01\\
71.62	0.01\\
71.63	0.01\\
71.64	0.01\\
71.65	0.01\\
71.66	0.01\\
71.67	0.01\\
71.68	0.01\\
71.69	0.01\\
71.7	0.01\\
71.71	0.01\\
71.72	0.01\\
71.73	0.01\\
71.74	0.01\\
71.75	0.01\\
71.76	0.01\\
71.77	0.01\\
71.78	0.01\\
71.79	0.01\\
71.8	0.01\\
71.81	0.01\\
71.82	0.01\\
71.83	0.01\\
71.84	0.01\\
71.85	0.01\\
71.86	0.01\\
71.87	0.01\\
71.88	0.01\\
71.89	0.01\\
71.9	0.01\\
71.91	0.01\\
71.92	0.01\\
71.93	0.01\\
71.94	0.01\\
71.95	0.01\\
71.96	0.01\\
71.97	0.01\\
71.98	0.01\\
71.99	0.01\\
72	0.01\\
72.01	0.01\\
72.02	0.01\\
72.03	0.01\\
72.04	0.01\\
72.05	0.01\\
72.06	0.01\\
72.07	0.01\\
72.08	0.01\\
72.09	0.01\\
72.1	0.01\\
72.11	0.01\\
72.12	0.01\\
72.13	0.01\\
72.14	0.01\\
72.15	0.01\\
72.16	0.01\\
72.17	0.01\\
72.18	0.01\\
72.19	0.01\\
72.2	0.01\\
72.21	0.01\\
72.22	0.01\\
72.23	0.01\\
72.24	0.01\\
72.25	0.01\\
72.26	0.01\\
72.27	0.01\\
72.28	0.01\\
72.29	0.01\\
72.3	0.01\\
72.31	0.01\\
72.32	0.01\\
72.33	0.01\\
72.34	0.01\\
72.35	0.01\\
72.36	0.01\\
72.37	0.01\\
72.38	0.01\\
72.39	0.01\\
72.4	0.01\\
72.41	0.01\\
72.42	0.01\\
72.43	0.01\\
72.44	0.01\\
72.45	0.01\\
72.46	0.01\\
72.47	0.01\\
72.48	0.01\\
72.49	0.01\\
72.5	0.01\\
72.51	0.01\\
72.52	0.01\\
72.53	0.01\\
72.54	0.01\\
72.55	0.01\\
72.56	0.01\\
72.57	0.01\\
72.58	0.01\\
72.59	0.01\\
72.6	0.01\\
72.61	0.01\\
72.62	0.01\\
72.63	0.01\\
72.64	0.01\\
72.65	0.01\\
72.66	0.01\\
72.67	0.01\\
72.68	0.01\\
72.69	0.01\\
72.7	0.01\\
72.71	0.01\\
72.72	0.01\\
72.73	0.01\\
72.74	0.01\\
72.75	0.01\\
72.76	0.01\\
72.77	0.01\\
72.78	0.01\\
72.79	0.01\\
72.8	0.01\\
72.81	0.01\\
72.82	0.01\\
72.83	0.01\\
72.84	0.01\\
72.85	0.01\\
72.86	0.01\\
72.87	0.01\\
72.88	0.01\\
72.89	0.01\\
72.9	0.01\\
72.91	0.01\\
72.92	0.01\\
72.93	0.01\\
72.94	0.01\\
72.95	0.01\\
72.96	0.01\\
72.97	0.01\\
72.98	0.01\\
72.99	0.01\\
73	0.01\\
73.01	0.01\\
73.02	0.01\\
73.03	0.01\\
73.04	0.01\\
73.05	0.01\\
73.06	0.01\\
73.07	0.01\\
73.08	0.01\\
73.09	0.01\\
73.1	0.01\\
73.11	0.01\\
73.12	0.01\\
73.13	0.01\\
73.14	0.01\\
73.15	0.01\\
73.16	0.01\\
73.17	0.01\\
73.18	0.01\\
73.19	0.01\\
73.2	0.01\\
73.21	0.01\\
73.22	0.01\\
73.23	0.01\\
73.24	0.01\\
73.25	0.01\\
73.26	0.01\\
73.27	0.01\\
73.28	0.01\\
73.29	0.01\\
73.3	0.01\\
73.31	0.01\\
73.32	0.01\\
73.33	0.01\\
73.34	0.01\\
73.35	0.01\\
73.36	0.01\\
73.37	0.01\\
73.38	0.01\\
73.39	0.01\\
73.4	0.01\\
73.41	0.01\\
73.42	0.01\\
73.43	0.01\\
73.44	0.01\\
73.45	0.01\\
73.46	0.01\\
73.47	0.01\\
73.48	0.01\\
73.49	0.01\\
73.5	0.01\\
73.51	0.01\\
73.52	0.01\\
73.53	0.01\\
73.54	0.01\\
73.55	0.01\\
73.56	0.01\\
73.57	0.01\\
73.58	0.01\\
73.59	0.01\\
73.6	0.01\\
73.61	0.01\\
73.62	0.01\\
73.63	0.01\\
73.64	0.01\\
73.65	0.01\\
73.66	0.01\\
73.67	0.01\\
73.68	0.01\\
73.69	0.01\\
73.7	0.01\\
73.71	0.01\\
73.72	0.01\\
73.73	0.01\\
73.74	0.01\\
73.75	0.01\\
73.76	0.01\\
73.77	0.01\\
73.78	0.01\\
73.79	0.01\\
73.8	0.01\\
73.81	0.01\\
73.82	0.01\\
73.83	0.01\\
73.84	0.01\\
73.85	0.01\\
73.86	0.01\\
73.87	0.01\\
73.88	0.01\\
73.89	0.01\\
73.9	0.01\\
73.91	0.01\\
73.92	0.01\\
73.93	0.01\\
73.94	0.01\\
73.95	0.01\\
73.96	0.01\\
73.97	0.01\\
73.98	0.01\\
73.99	0.01\\
74	0.01\\
74.01	0.01\\
74.02	0.01\\
74.03	0.01\\
74.04	0.01\\
74.05	0.01\\
74.06	0.01\\
74.07	0.01\\
74.08	0.01\\
74.09	0.01\\
74.1	0.01\\
74.11	0.01\\
74.12	0.01\\
74.13	0.01\\
74.14	0.01\\
74.15	0.01\\
74.16	0.01\\
74.17	0.01\\
74.18	0.01\\
74.19	0.01\\
74.2	0.01\\
74.21	0.01\\
74.22	0.01\\
74.23	0.01\\
74.24	0.01\\
74.25	0.01\\
74.26	0.01\\
74.27	0.01\\
74.28	0.01\\
74.29	0.01\\
74.3	0.01\\
74.31	0.01\\
74.32	0.01\\
74.33	0.01\\
74.34	0.01\\
74.35	0.01\\
74.36	0.01\\
74.37	0.01\\
74.38	0.01\\
74.39	0.01\\
74.4	0.01\\
74.41	0.01\\
74.42	0.01\\
74.43	0.01\\
74.44	0.01\\
74.45	0.01\\
74.46	0.01\\
74.47	0.01\\
74.48	0.01\\
74.49	0.01\\
74.5	0.01\\
74.51	0.01\\
74.52	0.01\\
74.53	0.01\\
74.54	0.01\\
74.55	0.01\\
74.56	0.01\\
74.57	0.01\\
74.58	0.01\\
74.59	0.01\\
74.6	0.01\\
74.61	0.01\\
74.62	0.01\\
74.63	0.01\\
74.64	0.01\\
74.65	0.01\\
74.66	0.01\\
74.67	0.01\\
74.68	0.01\\
74.69	0.01\\
74.7	0.01\\
74.71	0.01\\
74.72	0.01\\
74.73	0.01\\
74.74	0.01\\
74.75	0.01\\
74.76	0.01\\
74.77	0.01\\
74.78	0.01\\
74.79	0.01\\
74.8	0.01\\
74.81	0.01\\
74.82	0.01\\
74.83	0.01\\
74.84	0.01\\
74.85	0.01\\
74.86	0.01\\
74.87	0.01\\
74.88	0.01\\
74.89	0.01\\
74.9	0.01\\
74.91	0.01\\
74.92	0.01\\
74.93	0.01\\
74.94	0.01\\
74.95	0.01\\
74.96	0.01\\
74.97	0.01\\
74.98	0.01\\
74.99	0.01\\
75	0.01\\
75.01	0.01\\
75.02	0.01\\
75.03	0.01\\
75.04	0.01\\
75.05	0.01\\
75.06	0.01\\
75.07	0.01\\
75.08	0.01\\
75.09	0.01\\
75.1	0.01\\
75.11	0.01\\
75.12	0.01\\
75.13	0.01\\
75.14	0.01\\
75.15	0.01\\
75.16	0.01\\
75.17	0.01\\
75.18	0.01\\
75.19	0.01\\
75.2	0.01\\
75.21	0.01\\
75.22	0.01\\
75.23	0.01\\
75.24	0.01\\
75.25	0.01\\
75.26	0.01\\
75.27	0.01\\
75.28	0.01\\
75.29	0.01\\
75.3	0.01\\
75.31	0.01\\
75.32	0.01\\
75.33	0.01\\
75.34	0.01\\
75.35	0.01\\
75.36	0.01\\
75.37	0.01\\
75.38	0.01\\
75.39	0.01\\
75.4	0.01\\
75.41	0.01\\
75.42	0.01\\
75.43	0.01\\
75.44	0.01\\
75.45	0.01\\
75.46	0.01\\
75.47	0.01\\
75.48	0.01\\
75.49	0.01\\
75.5	0.01\\
75.51	0.01\\
75.52	0.01\\
75.53	0.01\\
75.54	0.01\\
75.55	0.01\\
75.56	0.01\\
75.57	0.01\\
75.58	0.01\\
75.59	0.01\\
75.6	0.01\\
75.61	0.01\\
75.62	0.01\\
75.63	0.01\\
75.64	0.01\\
75.65	0.01\\
75.66	0.01\\
75.67	0.01\\
75.68	0.01\\
75.69	0.01\\
75.7	0.01\\
75.71	0.01\\
75.72	0.01\\
75.73	0.01\\
75.74	0.01\\
75.75	0.01\\
75.76	0.01\\
75.77	0.01\\
75.78	0.01\\
75.79	0.01\\
75.8	0.01\\
75.81	0.01\\
75.82	0.01\\
75.83	0.01\\
75.84	0.01\\
75.85	0.01\\
75.86	0.01\\
75.87	0.01\\
75.88	0.01\\
75.89	0.01\\
75.9	0.01\\
75.91	0.01\\
75.92	0.01\\
75.93	0.01\\
75.94	0.01\\
75.95	0.01\\
75.96	0.01\\
75.97	0.01\\
75.98	0.01\\
75.99	0.01\\
76	0.01\\
76.01	0.01\\
76.02	0.01\\
76.03	0.01\\
76.04	0.01\\
76.05	0.01\\
76.06	0.01\\
76.07	0.01\\
76.08	0.01\\
76.09	0.01\\
76.1	0.01\\
76.11	0.01\\
76.12	0.01\\
76.13	0.01\\
76.14	0.01\\
76.15	0.01\\
76.16	0.01\\
76.17	0.01\\
76.18	0.01\\
76.19	0.01\\
76.2	0.01\\
76.21	0.01\\
76.22	0.01\\
76.23	0.01\\
76.24	0.01\\
76.25	0.01\\
76.26	0.01\\
76.27	0.01\\
76.28	0.01\\
76.29	0.01\\
76.3	0.01\\
76.31	0.01\\
76.32	0.01\\
76.33	0.01\\
76.34	0.01\\
76.35	0.01\\
76.36	0.01\\
76.37	0.01\\
76.38	0.01\\
76.39	0.01\\
76.4	0.01\\
76.41	0.01\\
76.42	0.01\\
76.43	0.01\\
76.44	0.01\\
76.45	0.01\\
76.46	0.01\\
76.47	0.01\\
76.48	0.01\\
76.49	0.01\\
76.5	0.01\\
76.51	0.01\\
76.52	0.01\\
76.53	0.01\\
76.54	0.01\\
76.55	0.01\\
76.56	0.01\\
76.57	0.01\\
76.58	0.01\\
76.59	0.01\\
76.6	0.01\\
76.61	0.01\\
76.62	0.01\\
76.63	0.01\\
76.64	0.01\\
76.65	0.01\\
76.66	0.01\\
76.67	0.01\\
76.68	0.01\\
76.69	0.01\\
76.7	0.01\\
76.71	0.01\\
76.72	0.01\\
76.73	0.01\\
76.74	0.01\\
76.75	0.01\\
76.76	0.01\\
76.77	0.01\\
76.78	0.01\\
76.79	0.01\\
76.8	0.01\\
76.81	0.01\\
76.82	0.01\\
76.83	0.01\\
76.84	0.01\\
76.85	0.01\\
76.86	0.01\\
76.87	0.01\\
76.88	0.01\\
76.89	0.01\\
76.9	0.01\\
76.91	0.01\\
76.92	0.01\\
76.93	0.01\\
76.94	0.01\\
76.95	0.01\\
76.96	0.01\\
76.97	0.01\\
76.98	0.01\\
76.99	0.01\\
77	0.01\\
77.01	0.01\\
77.02	0.01\\
77.03	0.01\\
77.04	0.01\\
77.05	0.01\\
77.06	0.01\\
77.07	0.01\\
77.08	0.01\\
77.09	0.01\\
77.1	0.01\\
77.11	0.01\\
77.12	0.01\\
77.13	0.01\\
77.14	0.01\\
77.15	0.01\\
77.16	0.01\\
77.17	0.01\\
77.18	0.01\\
77.19	0.01\\
77.2	0.01\\
77.21	0.01\\
77.22	0.01\\
77.23	0.01\\
77.24	0.01\\
77.25	0.01\\
77.26	0.01\\
77.27	0.01\\
77.28	0.01\\
77.29	0.01\\
77.3	0.01\\
77.31	0.01\\
77.32	0.01\\
77.33	0.01\\
77.34	0.01\\
77.35	0.01\\
77.36	0.01\\
77.37	0.01\\
77.38	0.01\\
77.39	0.01\\
77.4	0.01\\
77.41	0.01\\
77.42	0.01\\
77.43	0.01\\
77.44	0.01\\
77.45	0.01\\
77.46	0.01\\
77.47	0.01\\
77.48	0.01\\
77.49	0.01\\
77.5	0.01\\
77.51	0.01\\
77.52	0.01\\
77.53	0.01\\
77.54	0.01\\
77.55	0.01\\
77.56	0.01\\
77.57	0.01\\
77.58	0.01\\
77.59	0.01\\
77.6	0.01\\
77.61	0.01\\
77.62	0.01\\
77.63	0.01\\
77.64	0.01\\
77.65	0.01\\
77.66	0.01\\
77.67	0.01\\
77.68	0.01\\
77.69	0.01\\
77.7	0.01\\
77.71	0.01\\
77.72	0.01\\
77.73	0.01\\
77.74	0.01\\
77.75	0.01\\
77.76	0.01\\
77.77	0.01\\
77.78	0.01\\
77.79	0.01\\
77.8	0.01\\
77.81	0.01\\
77.82	0.01\\
77.83	0.01\\
77.84	0.01\\
77.85	0.01\\
77.86	0.01\\
77.87	0.01\\
77.88	0.01\\
77.89	0.01\\
77.9	0.01\\
77.91	0.01\\
77.92	0.01\\
77.93	0.01\\
77.94	0.01\\
77.95	0.01\\
77.96	0.01\\
77.97	0.01\\
77.98	0.01\\
77.99	0.01\\
78	0.01\\
78.01	0.01\\
78.02	0.01\\
78.03	0.01\\
78.04	0.01\\
78.05	0.01\\
78.06	0.01\\
78.07	0.01\\
78.08	0.01\\
78.09	0.01\\
78.1	0.01\\
78.11	0.01\\
78.12	0.01\\
78.13	0.01\\
78.14	0.01\\
78.15	0.01\\
78.16	0.01\\
78.17	0.01\\
78.18	0.01\\
78.19	0.01\\
78.2	0.01\\
78.21	0.01\\
78.22	0.01\\
78.23	0.01\\
78.24	0.01\\
78.25	0.01\\
78.26	0.01\\
78.27	0.01\\
78.28	0.01\\
78.29	0.01\\
78.3	0.01\\
78.31	0.01\\
78.32	0.01\\
78.33	0.01\\
78.34	0.01\\
78.35	0.01\\
78.36	0.01\\
78.37	0.01\\
78.38	0.01\\
78.39	0.01\\
78.4	0.01\\
78.41	0.01\\
78.42	0.01\\
78.43	0.01\\
78.44	0.01\\
78.45	0.01\\
78.46	0.01\\
78.47	0.01\\
78.48	0.01\\
78.49	0.01\\
78.5	0.01\\
78.51	0.01\\
78.52	0.01\\
78.53	0.01\\
78.54	0.01\\
78.55	0.01\\
78.56	0.01\\
78.57	0.01\\
78.58	0.01\\
78.59	0.01\\
78.6	0.01\\
78.61	0.01\\
78.62	0.01\\
78.63	0.01\\
78.64	0.01\\
78.65	0.01\\
78.66	0.01\\
78.67	0.01\\
78.68	0.01\\
78.69	0.01\\
78.7	0.01\\
78.71	0.01\\
78.72	0.01\\
78.73	0.01\\
78.74	0.01\\
78.75	0.01\\
78.76	0.01\\
78.77	0.01\\
78.78	0.01\\
78.79	0.01\\
78.8	0.01\\
78.81	0.01\\
78.82	0.01\\
78.83	0.01\\
78.84	0.01\\
78.85	0.01\\
78.86	0.01\\
78.87	0.01\\
78.88	0.01\\
78.89	0.01\\
78.9	0.01\\
78.91	0.01\\
78.92	0.01\\
78.93	0.01\\
78.94	0.01\\
78.95	0.01\\
78.96	0.01\\
78.97	0.01\\
78.98	0.01\\
78.99	0.01\\
79	0.01\\
79.01	0.01\\
79.02	0.01\\
79.03	0.01\\
79.04	0.01\\
79.05	0.01\\
79.06	0.01\\
79.07	0.01\\
79.08	0.01\\
79.09	0.01\\
79.1	0.01\\
79.11	0.01\\
79.12	0.01\\
79.13	0.01\\
79.14	0.01\\
79.15	0.01\\
79.16	0.01\\
79.17	0.01\\
79.18	0.01\\
79.19	0.01\\
79.2	0.01\\
79.21	0.01\\
79.22	0.01\\
79.23	0.01\\
79.24	0.01\\
79.25	0.01\\
79.26	0.01\\
79.27	0.01\\
79.28	0.01\\
79.29	0.01\\
79.3	0.01\\
79.31	0.01\\
79.32	0.01\\
79.33	0.01\\
79.34	0.01\\
79.35	0.01\\
79.36	0.01\\
79.37	0.01\\
79.38	0.01\\
79.39	0.01\\
79.4	0.01\\
79.41	0.01\\
79.42	0.01\\
79.43	0.01\\
79.44	0.01\\
79.45	0.01\\
79.46	0.01\\
79.47	0.01\\
79.48	0.01\\
79.49	0.01\\
79.5	0.01\\
79.51	0.01\\
79.52	0.01\\
79.53	0.01\\
79.54	0.01\\
79.55	0.01\\
79.56	0.01\\
79.57	0.01\\
79.58	0.01\\
79.59	0.01\\
79.6	0.01\\
79.61	0.01\\
79.62	0.01\\
79.63	0.01\\
79.64	0.01\\
79.65	0.01\\
79.66	0.01\\
79.67	0.01\\
79.68	0.01\\
79.69	0.01\\
79.7	0.01\\
79.71	0.01\\
79.72	0.01\\
79.73	0.01\\
79.74	0.01\\
79.75	0.01\\
79.76	0.01\\
79.77	0.01\\
79.78	0.01\\
79.79	0.01\\
79.8	0.01\\
79.81	0.01\\
79.82	0.01\\
79.83	0.01\\
79.84	0.01\\
79.85	0.01\\
79.86	0.01\\
79.87	0.01\\
79.88	0.01\\
79.89	0.01\\
79.9	0.01\\
79.91	0.01\\
79.92	0.01\\
79.93	0.01\\
79.94	0.01\\
79.95	0.01\\
79.96	0.01\\
79.97	0.01\\
79.98	0.01\\
79.99	0.01\\
80	0.01\\
80.01	0.01\\
};
\addplot [color=red,solid]
  table[row sep=crcr]{%
80.01	0.01\\
80.02	0.01\\
80.03	0.01\\
80.04	0.01\\
80.05	0.01\\
80.06	0.01\\
80.07	0.01\\
80.08	0.01\\
80.09	0.01\\
80.1	0.01\\
80.11	0.01\\
80.12	0.01\\
80.13	0.01\\
80.14	0.01\\
80.15	0.01\\
80.16	0.01\\
80.17	0.01\\
80.18	0.01\\
80.19	0.01\\
80.2	0.01\\
80.21	0.01\\
80.22	0.01\\
80.23	0.01\\
80.24	0.01\\
80.25	0.01\\
80.26	0.01\\
80.27	0.01\\
80.28	0.01\\
80.29	0.01\\
80.3	0.01\\
80.31	0.01\\
80.32	0.01\\
80.33	0.01\\
80.34	0.01\\
80.35	0.01\\
80.36	0.01\\
80.37	0.01\\
80.38	0.01\\
80.39	0.01\\
80.4	0.01\\
80.41	0.01\\
80.42	0.01\\
80.43	0.01\\
80.44	0.01\\
80.45	0.01\\
80.46	0.01\\
80.47	0.01\\
80.48	0.01\\
80.49	0.01\\
80.5	0.01\\
80.51	0.01\\
80.52	0.01\\
80.53	0.01\\
80.54	0.01\\
80.55	0.01\\
80.56	0.01\\
80.57	0.01\\
80.58	0.01\\
80.59	0.01\\
80.6	0.01\\
80.61	0.01\\
80.62	0.01\\
80.63	0.01\\
80.64	0.01\\
80.65	0.01\\
80.66	0.01\\
80.67	0.01\\
80.68	0.01\\
80.69	0.01\\
80.7	0.01\\
80.71	0.01\\
80.72	0.01\\
80.73	0.01\\
80.74	0.01\\
80.75	0.01\\
80.76	0.01\\
80.77	0.01\\
80.78	0.01\\
80.79	0.01\\
80.8	0.01\\
80.81	0.01\\
80.82	0.01\\
80.83	0.01\\
80.84	0.01\\
80.85	0.01\\
80.86	0.01\\
80.87	0.01\\
80.88	0.01\\
80.89	0.01\\
80.9	0.01\\
80.91	0.01\\
80.92	0.01\\
80.93	0.01\\
80.94	0.01\\
80.95	0.01\\
80.96	0.01\\
80.97	0.01\\
80.98	0.01\\
80.99	0.01\\
81	0.01\\
81.01	0.01\\
81.02	0.01\\
81.03	0.01\\
81.04	0.01\\
81.05	0.01\\
81.06	0.01\\
81.07	0.01\\
81.08	0.01\\
81.09	0.01\\
81.1	0.01\\
81.11	0.01\\
81.12	0.01\\
81.13	0.01\\
81.14	0.01\\
81.15	0.01\\
81.16	0.01\\
81.17	0.01\\
81.18	0.01\\
81.19	0.01\\
81.2	0.01\\
81.21	0.01\\
81.22	0.01\\
81.23	0.01\\
81.24	0.01\\
81.25	0.01\\
81.26	0.01\\
81.27	0.01\\
81.28	0.01\\
81.29	0.01\\
81.3	0.01\\
81.31	0.01\\
81.32	0.01\\
81.33	0.01\\
81.34	0.01\\
81.35	0.01\\
81.36	0.01\\
81.37	0.01\\
81.38	0.01\\
81.39	0.01\\
81.4	0.01\\
81.41	0.01\\
81.42	0.01\\
81.43	0.01\\
81.44	0.01\\
81.45	0.01\\
81.46	0.01\\
81.47	0.01\\
81.48	0.01\\
81.49	0.01\\
81.5	0.01\\
81.51	0.01\\
81.52	0.01\\
81.53	0.01\\
81.54	0.01\\
81.55	0.01\\
81.56	0.01\\
81.57	0.01\\
81.58	0.01\\
81.59	0.01\\
81.6	0.01\\
81.61	0.01\\
81.62	0.01\\
81.63	0.01\\
81.64	0.01\\
81.65	0.01\\
81.66	0.01\\
81.67	0.01\\
81.68	0.01\\
81.69	0.01\\
81.7	0.01\\
81.71	0.01\\
81.72	0.01\\
81.73	0.01\\
81.74	0.01\\
81.75	0.01\\
81.76	0.01\\
81.77	0.01\\
81.78	0.01\\
81.79	0.01\\
81.8	0.01\\
81.81	0.01\\
81.82	0.01\\
81.83	0.01\\
81.84	0.01\\
81.85	0.01\\
81.86	0.01\\
81.87	0.01\\
81.88	0.01\\
81.89	0.01\\
81.9	0.01\\
81.91	0.01\\
81.92	0.01\\
81.93	0.01\\
81.94	0.01\\
81.95	0.01\\
81.96	0.01\\
81.97	0.01\\
81.98	0.01\\
81.99	0.01\\
82	0.01\\
82.01	0.01\\
82.02	0.01\\
82.03	0.01\\
82.04	0.01\\
82.05	0.01\\
82.06	0.01\\
82.07	0.01\\
82.08	0.01\\
82.09	0.01\\
82.1	0.01\\
82.11	0.01\\
82.12	0.01\\
82.13	0.01\\
82.14	0.01\\
82.15	0.01\\
82.16	0.01\\
82.17	0.01\\
82.18	0.01\\
82.19	0.01\\
82.2	0.01\\
82.21	0.01\\
82.22	0.01\\
82.23	0.01\\
82.24	0.01\\
82.25	0.01\\
82.26	0.01\\
82.27	0.01\\
82.28	0.01\\
82.29	0.01\\
82.3	0.01\\
82.31	0.01\\
82.32	0.01\\
82.33	0.01\\
82.34	0.01\\
82.35	0.01\\
82.36	0.01\\
82.37	0.01\\
82.38	0.01\\
82.39	0.01\\
82.4	0.01\\
82.41	0.01\\
82.42	0.01\\
82.43	0.01\\
82.44	0.01\\
82.45	0.01\\
82.46	0.01\\
82.47	0.01\\
82.48	0.01\\
82.49	0.01\\
82.5	0.01\\
82.51	0.01\\
82.52	0.01\\
82.53	0.01\\
82.54	0.01\\
82.55	0.01\\
82.56	0.01\\
82.57	0.01\\
82.58	0.01\\
82.59	0.01\\
82.6	0.01\\
82.61	0.01\\
82.62	0.01\\
82.63	0.01\\
82.64	0.01\\
82.65	0.01\\
82.66	0.01\\
82.67	0.01\\
82.68	0.01\\
82.69	0.01\\
82.7	0.01\\
82.71	0.01\\
82.72	0.01\\
82.73	0.01\\
82.74	0.01\\
82.75	0.01\\
82.76	0.01\\
82.77	0.01\\
82.78	0.01\\
82.79	0.01\\
82.8	0.01\\
82.81	0.01\\
82.82	0.01\\
82.83	0.01\\
82.84	0.01\\
82.85	0.01\\
82.86	0.01\\
82.87	0.01\\
82.88	0.01\\
82.89	0.01\\
82.9	0.01\\
82.91	0.01\\
82.92	0.01\\
82.93	0.01\\
82.94	0.01\\
82.95	0.01\\
82.96	0.01\\
82.97	0.01\\
82.98	0.01\\
82.99	0.01\\
83	0.01\\
83.01	0.01\\
83.02	0.01\\
83.03	0.01\\
83.04	0.01\\
83.05	0.01\\
83.06	0.01\\
83.07	0.01\\
83.08	0.01\\
83.09	0.01\\
83.1	0.01\\
83.11	0.01\\
83.12	0.01\\
83.13	0.01\\
83.14	0.01\\
83.15	0.01\\
83.16	0.01\\
83.17	0.01\\
83.18	0.01\\
83.19	0.01\\
83.2	0.01\\
83.21	0.01\\
83.22	0.01\\
83.23	0.01\\
83.24	0.01\\
83.25	0.01\\
83.26	0.01\\
83.27	0.01\\
83.28	0.01\\
83.29	0.01\\
83.3	0.01\\
83.31	0.01\\
83.32	0.01\\
83.33	0.01\\
83.34	0.01\\
83.35	0.01\\
83.36	0.01\\
83.37	0.01\\
83.38	0.01\\
83.39	0.01\\
83.4	0.01\\
83.41	0.01\\
83.42	0.01\\
83.43	0.01\\
83.44	0.01\\
83.45	0.01\\
83.46	0.01\\
83.47	0.01\\
83.48	0.01\\
83.49	0.01\\
83.5	0.01\\
83.51	0.01\\
83.52	0.01\\
83.53	0.01\\
83.54	0.01\\
83.55	0.01\\
83.56	0.01\\
83.57	0.01\\
83.58	0.01\\
83.59	0.01\\
83.6	0.01\\
83.61	0.01\\
83.62	0.01\\
83.63	0.01\\
83.64	0.01\\
83.65	0.01\\
83.66	0.01\\
83.67	0.01\\
83.68	0.01\\
83.69	0.01\\
83.7	0.01\\
83.71	0.01\\
83.72	0.01\\
83.73	0.01\\
83.74	0.01\\
83.75	0.01\\
83.76	0.01\\
83.77	0.01\\
83.78	0.01\\
83.79	0.01\\
83.8	0.01\\
83.81	0.01\\
83.82	0.01\\
83.83	0.01\\
83.84	0.01\\
83.85	0.01\\
83.86	0.01\\
83.87	0.01\\
83.88	0.01\\
83.89	0.01\\
83.9	0.01\\
83.91	0.01\\
83.92	0.01\\
83.93	0.01\\
83.94	0.01\\
83.95	0.01\\
83.96	0.01\\
83.97	0.01\\
83.98	0.01\\
83.99	0.01\\
84	0.01\\
84.01	0.01\\
84.02	0.01\\
84.03	0.01\\
84.04	0.01\\
84.05	0.01\\
84.06	0.01\\
84.07	0.01\\
84.08	0.01\\
84.09	0.01\\
84.1	0.01\\
84.11	0.01\\
84.12	0.01\\
84.13	0.01\\
84.14	0.01\\
84.15	0.01\\
84.16	0.01\\
84.17	0.01\\
84.18	0.01\\
84.19	0.01\\
84.2	0.01\\
84.21	0.01\\
84.22	0.01\\
84.23	0.01\\
84.24	0.01\\
84.25	0.01\\
84.26	0.01\\
84.27	0.01\\
84.28	0.01\\
84.29	0.01\\
84.3	0.01\\
84.31	0.01\\
84.32	0.01\\
84.33	0.01\\
84.34	0.01\\
84.35	0.01\\
84.36	0.01\\
84.37	0.01\\
84.38	0.01\\
84.39	0.01\\
84.4	0.01\\
84.41	0.01\\
84.42	0.01\\
84.43	0.01\\
84.44	0.01\\
84.45	0.01\\
84.46	0.01\\
84.47	0.01\\
84.48	0.01\\
84.49	0.01\\
84.5	0.01\\
84.51	0.01\\
84.52	0.01\\
84.53	0.01\\
84.54	0.01\\
84.55	0.01\\
84.56	0.01\\
84.57	0.01\\
84.58	0.01\\
84.59	0.01\\
84.6	0.01\\
84.61	0.01\\
84.62	0.01\\
84.63	0.01\\
84.64	0.01\\
84.65	0.01\\
84.66	0.01\\
84.67	0.01\\
84.68	0.01\\
84.69	0.01\\
84.7	0.01\\
84.71	0.01\\
84.72	0.01\\
84.73	0.01\\
84.74	0.01\\
84.75	0.01\\
84.76	0.01\\
84.77	0.01\\
84.78	0.01\\
84.79	0.01\\
84.8	0.01\\
84.81	0.01\\
84.82	0.01\\
84.83	0.01\\
84.84	0.01\\
84.85	0.01\\
84.86	0.01\\
84.87	0.01\\
84.88	0.01\\
84.89	0.01\\
84.9	0.01\\
84.91	0.01\\
84.92	0.01\\
84.93	0.01\\
84.94	0.01\\
84.95	0.01\\
84.96	0.01\\
84.97	0.01\\
84.98	0.01\\
84.99	0.01\\
85	0.01\\
85.01	0.01\\
85.02	0.01\\
85.03	0.01\\
85.04	0.01\\
85.05	0.01\\
85.06	0.01\\
85.07	0.01\\
85.08	0.01\\
85.09	0.01\\
85.1	0.01\\
85.11	0.01\\
85.12	0.01\\
85.13	0.01\\
85.14	0.01\\
85.15	0.01\\
85.16	0.01\\
85.17	0.01\\
85.18	0.01\\
85.19	0.01\\
85.2	0.01\\
85.21	0.01\\
85.22	0.01\\
85.23	0.01\\
85.24	0.01\\
85.25	0.01\\
85.26	0.01\\
85.27	0.01\\
85.28	0.01\\
85.29	0.01\\
85.3	0.01\\
85.31	0.01\\
85.32	0.01\\
85.33	0.01\\
85.34	0.01\\
85.35	0.01\\
85.36	0.01\\
85.37	0.01\\
85.38	0.01\\
85.39	0.01\\
85.4	0.01\\
85.41	0.01\\
85.42	0.01\\
85.43	0.01\\
85.44	0.01\\
85.45	0.01\\
85.46	0.01\\
85.47	0.01\\
85.48	0.01\\
85.49	0.01\\
85.5	0.01\\
85.51	0.01\\
85.52	0.01\\
85.53	0.01\\
85.54	0.01\\
85.55	0.01\\
85.56	0.01\\
85.57	0.01\\
85.58	0.01\\
85.59	0.01\\
85.6	0.01\\
85.61	0.01\\
85.62	0.01\\
85.63	0.01\\
85.64	0.01\\
85.65	0.01\\
85.66	0.01\\
85.67	0.01\\
85.68	0.01\\
85.69	0.01\\
85.7	0.01\\
85.71	0.01\\
85.72	0.01\\
85.73	0.01\\
85.74	0.01\\
85.75	0.01\\
85.76	0.01\\
85.77	0.01\\
85.78	0.01\\
85.79	0.01\\
85.8	0.01\\
85.81	0.01\\
85.82	0.01\\
85.83	0.01\\
85.84	0.01\\
85.85	0.01\\
85.86	0.01\\
85.87	0.01\\
85.88	0.01\\
85.89	0.01\\
85.9	0.01\\
85.91	0.01\\
85.92	0.01\\
85.93	0.01\\
85.94	0.01\\
85.95	0.01\\
85.96	0.01\\
85.97	0.01\\
85.98	0.01\\
85.99	0.01\\
86	0.01\\
86.01	0.01\\
86.02	0.01\\
86.03	0.01\\
86.04	0.01\\
86.05	0.01\\
86.06	0.01\\
86.07	0.01\\
86.08	0.01\\
86.09	0.01\\
86.1	0.01\\
86.11	0.01\\
86.12	0.01\\
86.13	0.01\\
86.14	0.01\\
86.15	0.01\\
86.16	0.01\\
86.17	0.01\\
86.18	0.01\\
86.19	0.01\\
86.2	0.01\\
86.21	0.01\\
86.22	0.01\\
86.23	0.01\\
86.24	0.01\\
86.25	0.01\\
86.26	0.01\\
86.27	0.01\\
86.28	0.01\\
86.29	0.01\\
86.3	0.01\\
86.31	0.01\\
86.32	0.01\\
86.33	0.01\\
86.34	0.01\\
86.35	0.01\\
86.36	0.01\\
86.37	0.01\\
86.38	0.01\\
86.39	0.01\\
86.4	0.01\\
86.41	0.01\\
86.42	0.01\\
86.43	0.01\\
86.44	0.01\\
86.45	0.01\\
86.46	0.01\\
86.47	0.01\\
86.48	0.01\\
86.49	0.01\\
86.5	0.01\\
86.51	0.01\\
86.52	0.01\\
86.53	0.01\\
86.54	0.01\\
86.55	0.01\\
86.56	0.01\\
86.57	0.01\\
86.58	0.01\\
86.59	0.01\\
86.6	0.01\\
86.61	0.01\\
86.62	0.01\\
86.63	0.01\\
86.64	0.01\\
86.65	0.01\\
86.66	0.01\\
86.67	0.01\\
86.68	0.01\\
86.69	0.01\\
86.7	0.01\\
86.71	0.01\\
86.72	0.01\\
86.73	0.01\\
86.74	0.01\\
86.75	0.01\\
86.76	0.01\\
86.77	0.01\\
86.78	0.01\\
86.79	0.01\\
86.8	0.01\\
86.81	0.01\\
86.82	0.01\\
86.83	0.01\\
86.84	0.01\\
86.85	0.01\\
86.86	0.01\\
86.87	0.01\\
86.88	0.01\\
86.89	0.01\\
86.9	0.01\\
86.91	0.01\\
86.92	0.01\\
86.93	0.01\\
86.94	0.01\\
86.95	0.01\\
86.96	0.01\\
86.97	0.01\\
86.98	0.01\\
86.99	0.01\\
87	0.01\\
87.01	0.01\\
87.02	0.01\\
87.03	0.01\\
87.04	0.01\\
87.05	0.01\\
87.06	0.01\\
87.07	0.01\\
87.08	0.01\\
87.09	0.01\\
87.1	0.01\\
87.11	0.01\\
87.12	0.01\\
87.13	0.01\\
87.14	0.01\\
87.15	0.01\\
87.16	0.01\\
87.17	0.01\\
87.18	0.01\\
87.19	0.01\\
87.2	0.01\\
87.21	0.01\\
87.22	0.01\\
87.23	0.01\\
87.24	0.01\\
87.25	0.01\\
87.26	0.01\\
87.27	0.01\\
87.28	0.01\\
87.29	0.01\\
87.3	0.01\\
87.31	0.01\\
87.32	0.01\\
87.33	0.01\\
87.34	0.01\\
87.35	0.01\\
87.36	0.01\\
87.37	0.01\\
87.38	0.01\\
87.39	0.01\\
87.4	0.01\\
87.41	0.01\\
87.42	0.01\\
87.43	0.01\\
87.44	0.01\\
87.45	0.01\\
87.46	0.01\\
87.47	0.01\\
87.48	0.01\\
87.49	0.01\\
87.5	0.01\\
87.51	0.01\\
87.52	0.01\\
87.53	0.01\\
87.54	0.01\\
87.55	0.01\\
87.56	0.01\\
87.57	0.01\\
87.58	0.01\\
87.59	0.01\\
87.6	0.01\\
87.61	0.01\\
87.62	0.01\\
87.63	0.01\\
87.64	0.01\\
87.65	0.01\\
87.66	0.01\\
87.67	0.01\\
87.68	0.01\\
87.69	0.01\\
87.7	0.01\\
87.71	0.01\\
87.72	0.01\\
87.73	0.01\\
87.74	0.01\\
87.75	0.01\\
87.76	0.01\\
87.77	0.01\\
87.78	0.01\\
87.79	0.01\\
87.8	0.01\\
87.81	0.01\\
87.82	0.01\\
87.83	0.01\\
87.84	0.01\\
87.85	0.01\\
87.86	0.01\\
87.87	0.01\\
87.88	0.01\\
87.89	0.01\\
87.9	0.01\\
87.91	0.01\\
87.92	0.01\\
87.93	0.01\\
87.94	0.01\\
87.95	0.01\\
87.96	0.01\\
87.97	0.01\\
87.98	0.01\\
87.99	0.01\\
88	0.01\\
88.01	0.01\\
88.02	0.01\\
88.03	0.01\\
88.04	0.01\\
88.05	0.01\\
88.06	0.01\\
88.07	0.01\\
88.08	0.01\\
88.09	0.01\\
88.1	0.01\\
88.11	0.01\\
88.12	0.01\\
88.13	0.01\\
88.14	0.01\\
88.15	0.01\\
88.16	0.01\\
88.17	0.01\\
88.18	0.01\\
88.19	0.01\\
88.2	0.01\\
88.21	0.01\\
88.22	0.01\\
88.23	0.01\\
88.24	0.01\\
88.25	0.01\\
88.26	0.01\\
88.27	0.01\\
88.28	0.01\\
88.29	0.01\\
88.3	0.01\\
88.31	0.01\\
88.32	0.01\\
88.33	0.01\\
88.34	0.01\\
88.35	0.01\\
88.36	0.01\\
88.37	0.01\\
88.38	0.01\\
88.39	0.01\\
88.4	0.01\\
88.41	0.01\\
88.42	0.01\\
88.43	0.01\\
88.44	0.01\\
88.45	0.01\\
88.46	0.01\\
88.47	0.01\\
88.48	0.01\\
88.49	0.01\\
88.5	0.01\\
88.51	0.01\\
88.52	0.01\\
88.53	0.01\\
88.54	0.01\\
88.55	0.01\\
88.56	0.01\\
88.57	0.01\\
88.58	0.01\\
88.59	0.01\\
88.6	0.01\\
88.61	0.01\\
88.62	0.01\\
88.63	0.01\\
88.64	0.01\\
88.65	0.01\\
88.66	0.01\\
88.67	0.01\\
88.68	0.01\\
88.69	0.01\\
88.7	0.01\\
88.71	0.01\\
88.72	0.01\\
88.73	0.01\\
88.74	0.01\\
88.75	0.01\\
88.76	0.01\\
88.77	0.01\\
88.78	0.01\\
88.79	0.01\\
88.8	0.01\\
88.81	0.01\\
88.82	0.01\\
88.83	0.01\\
88.84	0.01\\
88.85	0.01\\
88.86	0.01\\
88.87	0.01\\
88.88	0.01\\
88.89	0.01\\
88.9	0.01\\
88.91	0.01\\
88.92	0.01\\
88.93	0.01\\
88.94	0.01\\
88.95	0.01\\
88.96	0.01\\
88.97	0.01\\
88.98	0.01\\
88.99	0.01\\
89	0.01\\
89.01	0.01\\
89.02	0.01\\
89.03	0.01\\
89.04	0.01\\
89.05	0.01\\
89.06	0.01\\
89.07	0.01\\
89.08	0.01\\
89.09	0.01\\
89.1	0.01\\
89.11	0.01\\
89.12	0.01\\
89.13	0.01\\
89.14	0.01\\
89.15	0.01\\
89.16	0.01\\
89.17	0.01\\
89.18	0.01\\
89.19	0.01\\
89.2	0.01\\
89.21	0.01\\
89.22	0.01\\
89.23	0.01\\
89.24	0.01\\
89.25	0.01\\
89.26	0.01\\
89.27	0.01\\
89.28	0.01\\
89.29	0.01\\
89.3	0.01\\
89.31	0.01\\
89.32	0.01\\
89.33	0.01\\
89.34	0.01\\
89.35	0.01\\
89.36	0.01\\
89.37	0.01\\
89.38	0.01\\
89.39	0.01\\
89.4	0.01\\
89.41	0.01\\
89.42	0.01\\
89.43	0.01\\
89.44	0.01\\
89.45	0.01\\
89.46	0.01\\
89.47	0.01\\
89.48	0.01\\
89.49	0.01\\
89.5	0.01\\
89.51	0.01\\
89.52	0.01\\
89.53	0.01\\
89.54	0.01\\
89.55	0.01\\
89.56	0.01\\
89.57	0.01\\
89.58	0.01\\
89.59	0.01\\
89.6	0.01\\
89.61	0.01\\
89.62	0.01\\
89.63	0.01\\
89.64	0.01\\
89.65	0.01\\
89.66	0.01\\
89.67	0.01\\
89.68	0.01\\
89.69	0.01\\
89.7	0.01\\
89.71	0.01\\
89.72	0.01\\
89.73	0.01\\
89.74	0.01\\
89.75	0.01\\
89.76	0.01\\
89.77	0.01\\
89.78	0.01\\
89.79	0.01\\
89.8	0.01\\
89.81	0.01\\
89.82	0.01\\
89.83	0.01\\
89.84	0.01\\
89.85	0.01\\
89.86	0.01\\
89.87	0.01\\
89.88	0.01\\
89.89	0.01\\
89.9	0.01\\
89.91	0.01\\
89.92	0.01\\
89.93	0.01\\
89.94	0.01\\
89.95	0.01\\
89.96	0.01\\
89.97	0.01\\
89.98	0.01\\
89.99	0.01\\
90	0.01\\
90.01	0.01\\
90.02	0.01\\
90.03	0.01\\
90.04	0.01\\
90.05	0.01\\
90.06	0.01\\
90.07	0.01\\
90.08	0.01\\
90.09	0.01\\
90.1	0.01\\
90.11	0.01\\
90.12	0.01\\
90.13	0.01\\
90.14	0.01\\
90.15	0.01\\
90.16	0.01\\
90.17	0.01\\
90.18	0.01\\
90.19	0.01\\
90.2	0.01\\
90.21	0.01\\
90.22	0.01\\
90.23	0.01\\
90.24	0.01\\
90.25	0.01\\
90.26	0.01\\
90.27	0.01\\
90.28	0.01\\
90.29	0.01\\
90.3	0.01\\
90.31	0.01\\
90.32	0.01\\
90.33	0.01\\
90.34	0.01\\
90.35	0.01\\
90.36	0.01\\
90.37	0.01\\
90.38	0.01\\
90.39	0.01\\
90.4	0.01\\
90.41	0.01\\
90.42	0.01\\
90.43	0.01\\
90.44	0.01\\
90.45	0.01\\
90.46	0.01\\
90.47	0.01\\
90.48	0.01\\
90.49	0.01\\
90.5	0.01\\
90.51	0.01\\
90.52	0.01\\
90.53	0.01\\
90.54	0.01\\
90.55	0.01\\
90.56	0.01\\
90.57	0.01\\
90.58	0.01\\
90.59	0.01\\
90.6	0.01\\
90.61	0.01\\
90.62	0.01\\
90.63	0.01\\
90.64	0.01\\
90.65	0.01\\
90.66	0.01\\
90.67	0.01\\
90.68	0.01\\
90.69	0.01\\
90.7	0.01\\
90.71	0.01\\
90.72	0.01\\
90.73	0.01\\
90.74	0.01\\
90.75	0.01\\
90.76	0.01\\
90.77	0.01\\
90.78	0.01\\
90.79	0.01\\
90.8	0.01\\
90.81	0.01\\
90.82	0.01\\
90.83	0.01\\
90.84	0.01\\
90.85	0.01\\
90.86	0.01\\
90.87	0.01\\
90.88	0.01\\
90.89	0.01\\
90.9	0.01\\
90.91	0.01\\
90.92	0.01\\
90.93	0.01\\
90.94	0.01\\
90.95	0.01\\
90.96	0.01\\
90.97	0.01\\
90.98	0.01\\
90.99	0.01\\
91	0.01\\
91.01	0.01\\
91.02	0.01\\
91.03	0.01\\
91.04	0.01\\
91.05	0.01\\
91.06	0.01\\
91.07	0.01\\
91.08	0.01\\
91.09	0.01\\
91.1	0.01\\
91.11	0.01\\
91.12	0.01\\
91.13	0.01\\
91.14	0.01\\
91.15	0.01\\
91.16	0.01\\
91.17	0.01\\
91.18	0.01\\
91.19	0.01\\
91.2	0.01\\
91.21	0.01\\
91.22	0.01\\
91.23	0.01\\
91.24	0.01\\
91.25	0.01\\
91.26	0.01\\
91.27	0.01\\
91.28	0.01\\
91.29	0.01\\
91.3	0.01\\
91.31	0.01\\
91.32	0.01\\
91.33	0.01\\
91.34	0.01\\
91.35	0.01\\
91.36	0.01\\
91.37	0.01\\
91.38	0.01\\
91.39	0.01\\
91.4	0.01\\
91.41	0.01\\
91.42	0.01\\
91.43	0.01\\
91.44	0.01\\
91.45	0.01\\
91.46	0.01\\
91.47	0.01\\
91.48	0.01\\
91.49	0.01\\
91.5	0.01\\
91.51	0.01\\
91.52	0.01\\
91.53	0.01\\
91.54	0.01\\
91.55	0.01\\
91.56	0.01\\
91.57	0.01\\
91.58	0.01\\
91.59	0.01\\
91.6	0.01\\
91.61	0.01\\
91.62	0.01\\
91.63	0.01\\
91.64	0.01\\
91.65	0.01\\
91.66	0.01\\
91.67	0.01\\
91.68	0.01\\
91.69	0.01\\
91.7	0.01\\
91.71	0.01\\
91.72	0.01\\
91.73	0.01\\
91.74	0.01\\
91.75	0.01\\
91.76	0.01\\
91.77	0.01\\
91.78	0.01\\
91.79	0.01\\
91.8	0.01\\
91.81	0.01\\
91.82	0.01\\
91.83	0.01\\
91.84	0.01\\
91.85	0.01\\
91.86	0.01\\
91.87	0.01\\
91.88	0.01\\
91.89	0.01\\
91.9	0.01\\
91.91	0.01\\
91.92	0.01\\
91.93	0.01\\
91.94	0.01\\
91.95	0.01\\
91.96	0.01\\
91.97	0.01\\
91.98	0.01\\
91.99	0.01\\
92	0.01\\
92.01	0.01\\
92.02	0.01\\
92.03	0.01\\
92.04	0.01\\
92.05	0.01\\
92.06	0.01\\
92.07	0.01\\
92.08	0.01\\
92.09	0.01\\
92.1	0.01\\
92.11	0.01\\
92.12	0.01\\
92.13	0.01\\
92.14	0.01\\
92.15	0.01\\
92.16	0.01\\
92.17	0.01\\
92.18	0.01\\
92.19	0.01\\
92.2	0.01\\
92.21	0.01\\
92.22	0.01\\
92.23	0.01\\
92.24	0.01\\
92.25	0.01\\
92.26	0.01\\
92.27	0.01\\
92.28	0.01\\
92.29	0.01\\
92.3	0.01\\
92.31	0.01\\
92.32	0.01\\
92.33	0.01\\
92.34	0.01\\
92.35	0.01\\
92.36	0.01\\
92.37	0.01\\
92.38	0.01\\
92.39	0.01\\
92.4	0.01\\
92.41	0.01\\
92.42	0.01\\
92.43	0.01\\
92.44	0.01\\
92.45	0.01\\
92.46	0.01\\
92.47	0.01\\
92.48	0.01\\
92.49	0.01\\
92.5	0.01\\
92.51	0.01\\
92.52	0.01\\
92.53	0.01\\
92.54	0.01\\
92.55	0.01\\
92.56	0.01\\
92.57	0.01\\
92.58	0.01\\
92.59	0.01\\
92.6	0.01\\
92.61	0.01\\
92.62	0.01\\
92.63	0.01\\
92.64	0.01\\
92.65	0.01\\
92.66	0.01\\
92.67	0.01\\
92.68	0.01\\
92.69	0.01\\
92.7	0.01\\
92.71	0.01\\
92.72	0.01\\
92.73	0.01\\
92.74	0.01\\
92.75	0.01\\
92.76	0.01\\
92.77	0.01\\
92.78	0.01\\
92.79	0.01\\
92.8	0.01\\
92.81	0.01\\
92.82	0.01\\
92.83	0.01\\
92.84	0.01\\
92.85	0.01\\
92.86	0.01\\
92.87	0.01\\
92.88	0.01\\
92.89	0.01\\
92.9	0.01\\
92.91	0.01\\
92.92	0.01\\
92.93	0.01\\
92.94	0.01\\
92.95	0.01\\
92.96	0.01\\
92.97	0.01\\
92.98	0.01\\
92.99	0.01\\
93	0.01\\
93.01	0.01\\
93.02	0.01\\
93.03	0.01\\
93.04	0.01\\
93.05	0.01\\
93.06	0.01\\
93.07	0.01\\
93.08	0.01\\
93.09	0.01\\
93.1	0.01\\
93.11	0.01\\
93.12	0.01\\
93.13	0.01\\
93.14	0.01\\
93.15	0.01\\
93.16	0.01\\
93.17	0.01\\
93.18	0.01\\
93.19	0.01\\
93.2	0.01\\
93.21	0.01\\
93.22	0.01\\
93.23	0.01\\
93.24	0.01\\
93.25	0.01\\
93.26	0.01\\
93.27	0.01\\
93.28	0.01\\
93.29	0.01\\
93.3	0.01\\
93.31	0.01\\
93.32	0.01\\
93.33	0.01\\
93.34	0.01\\
93.35	0.01\\
93.36	0.01\\
93.37	0.01\\
93.38	0.01\\
93.39	0.01\\
93.4	0.01\\
93.41	0.01\\
93.42	0.01\\
93.43	0.01\\
93.44	0.01\\
93.45	0.01\\
93.46	0.01\\
93.47	0.01\\
93.48	0.01\\
93.49	0.01\\
93.5	0.01\\
93.51	0.01\\
93.52	0.01\\
93.53	0.01\\
93.54	0.01\\
93.55	0.01\\
93.56	0.01\\
93.57	0.01\\
93.58	0.01\\
93.59	0.01\\
93.6	0.01\\
93.61	0.01\\
93.62	0.01\\
93.63	0.01\\
93.64	0.01\\
93.65	0.01\\
93.66	0.01\\
93.67	0.01\\
93.68	0.01\\
93.69	0.01\\
93.7	0.01\\
93.71	0.01\\
93.72	0.01\\
93.73	0.01\\
93.74	0.01\\
93.75	0.01\\
93.76	0.01\\
93.77	0.01\\
93.78	0.01\\
93.79	0.01\\
93.8	0.01\\
93.81	0.01\\
93.82	0.01\\
93.83	0.01\\
93.84	0.01\\
93.85	0.01\\
93.86	0.01\\
93.87	0.01\\
93.88	0.01\\
93.89	0.01\\
93.9	0.01\\
93.91	0.01\\
93.92	0.01\\
93.93	0.01\\
93.94	0.01\\
93.95	0.01\\
93.96	0.01\\
93.97	0.01\\
93.98	0.01\\
93.99	0.01\\
94	0.01\\
94.01	0.01\\
94.02	0.01\\
94.03	0.01\\
94.04	0.01\\
94.05	0.01\\
94.06	0.01\\
94.07	0.01\\
94.08	0.01\\
94.09	0.01\\
94.1	0.01\\
94.11	0.01\\
94.12	0.01\\
94.13	0.01\\
94.14	0.01\\
94.15	0.01\\
94.16	0.01\\
94.17	0.01\\
94.18	0.01\\
94.19	0.01\\
94.2	0.01\\
94.21	0.01\\
94.22	0.01\\
94.23	0.01\\
94.24	0.01\\
94.25	0.01\\
94.26	0.01\\
94.27	0.01\\
94.28	0.01\\
94.29	0.01\\
94.3	0.01\\
94.31	0.01\\
94.32	0.01\\
94.33	0.01\\
94.34	0.01\\
94.35	0.01\\
94.36	0.01\\
94.37	0.01\\
94.38	0.01\\
94.39	0.01\\
94.4	0.01\\
94.41	0.01\\
94.42	0.01\\
94.43	0.01\\
94.44	0.01\\
94.45	0.01\\
94.46	0.01\\
94.47	0.01\\
94.48	0.01\\
94.49	0.01\\
94.5	0.01\\
94.51	0.01\\
94.52	0.01\\
94.53	0.01\\
94.54	0.01\\
94.55	0.01\\
94.56	0.01\\
94.57	0.01\\
94.58	0.01\\
94.59	0.01\\
94.6	0.01\\
94.61	0.01\\
94.62	0.01\\
94.63	0.01\\
94.64	0.01\\
94.65	0.01\\
94.66	0.01\\
94.67	0.01\\
94.68	0.01\\
94.69	0.01\\
94.7	0.01\\
94.71	0.01\\
94.72	0.01\\
94.73	0.01\\
94.74	0.01\\
94.75	0.01\\
94.76	0.01\\
94.77	0.01\\
94.78	0.01\\
94.79	0.01\\
94.8	0.01\\
94.81	0.01\\
94.82	0.01\\
94.83	0.01\\
94.84	0.01\\
94.85	0.01\\
94.86	0.01\\
94.87	0.01\\
94.88	0.01\\
94.89	0.01\\
94.9	0.01\\
94.91	0.01\\
94.92	0.01\\
94.93	0.01\\
94.94	0.01\\
94.95	0.01\\
94.96	0.01\\
94.97	0.01\\
94.98	0.01\\
94.99	0.01\\
95	0.01\\
95.01	0.01\\
95.02	0.01\\
95.03	0.01\\
95.04	0.01\\
95.05	0.01\\
95.06	0.01\\
95.07	0.01\\
95.08	0.01\\
95.09	0.01\\
95.1	0.01\\
95.11	0.01\\
95.12	0.01\\
95.13	0.01\\
95.14	0.01\\
95.15	0.01\\
95.16	0.01\\
95.17	0.01\\
95.18	0.01\\
95.19	0.01\\
95.2	0.01\\
95.21	0.01\\
95.22	0.01\\
95.23	0.01\\
95.24	0.01\\
95.25	0.01\\
95.26	0.01\\
95.27	0.01\\
95.28	0.01\\
95.29	0.01\\
95.3	0.01\\
95.31	0.01\\
95.32	0.01\\
95.33	0.01\\
95.34	0.01\\
95.35	0.01\\
95.36	0.01\\
95.37	0.01\\
95.38	0.01\\
95.39	0.01\\
95.4	0.01\\
95.41	0.01\\
95.42	0.01\\
95.43	0.01\\
95.44	0.01\\
95.45	0.01\\
95.46	0.01\\
95.47	0.01\\
95.48	0.01\\
95.49	0.01\\
95.5	0.01\\
95.51	0.01\\
95.52	0.01\\
95.53	0.01\\
95.54	0.01\\
95.55	0.01\\
95.56	0.01\\
95.57	0.01\\
95.58	0.01\\
95.59	0.01\\
95.6	0.01\\
95.61	0.01\\
95.62	0.01\\
95.63	0.01\\
95.64	0.01\\
95.65	0.01\\
95.66	0.01\\
95.67	0.01\\
95.68	0.01\\
95.69	0.01\\
95.7	0.01\\
95.71	0.01\\
95.72	0.01\\
95.73	0.01\\
95.74	0.01\\
95.75	0.01\\
95.76	0.01\\
95.77	0.01\\
95.78	0.01\\
95.79	0.01\\
95.8	0.01\\
95.81	0.01\\
95.82	0.01\\
95.83	0.01\\
95.84	0.01\\
95.85	0.01\\
95.86	0.01\\
95.87	0.01\\
95.88	0.01\\
95.89	0.01\\
95.9	0.01\\
95.91	0.01\\
95.92	0.01\\
95.93	0.01\\
95.94	0.01\\
95.95	0.01\\
95.96	0.01\\
95.97	0.01\\
95.98	0.01\\
95.99	0.01\\
96	0.01\\
96.01	0.01\\
96.02	0.01\\
96.03	0.01\\
96.04	0.01\\
96.05	0.01\\
96.06	0.01\\
96.07	0.01\\
96.08	0.01\\
96.09	0.01\\
96.1	0.01\\
96.11	0.01\\
96.12	0.01\\
96.13	0.01\\
96.14	0.01\\
96.15	0.01\\
96.16	0.01\\
96.17	0.01\\
96.18	0.01\\
96.19	0.01\\
96.2	0.01\\
96.21	0.01\\
96.22	0.01\\
96.23	0.01\\
96.24	0.01\\
96.25	0.01\\
96.26	0.01\\
96.27	0.01\\
96.28	0.01\\
96.29	0.01\\
96.3	0.01\\
96.31	0.01\\
96.32	0.01\\
96.33	0.01\\
96.34	0.01\\
96.35	0.01\\
96.36	0.01\\
96.37	0.01\\
96.38	0.01\\
96.39	0.01\\
96.4	0.01\\
96.41	0.01\\
96.42	0.01\\
96.43	0.01\\
96.44	0.01\\
96.45	0.01\\
96.46	0.01\\
96.47	0.01\\
96.48	0.01\\
96.49	0.01\\
96.5	0.01\\
96.51	0.01\\
96.52	0.01\\
96.53	0.01\\
96.54	0.01\\
96.55	0.01\\
96.56	0.01\\
96.57	0.01\\
96.58	0.01\\
96.59	0.01\\
96.6	0.01\\
96.61	0.01\\
96.62	0.01\\
96.63	0.01\\
96.64	0.01\\
96.65	0.01\\
96.66	0.01\\
96.67	0.01\\
96.68	0.01\\
96.69	0.01\\
96.7	0.01\\
96.71	0.01\\
96.72	0.01\\
96.73	0.01\\
96.74	0.01\\
96.75	0.01\\
96.76	0.01\\
96.77	0.01\\
96.78	0.01\\
96.79	0.01\\
96.8	0.01\\
96.81	0.01\\
96.82	0.01\\
96.83	0.01\\
96.84	0.01\\
96.85	0.01\\
96.86	0.01\\
96.87	0.01\\
96.88	0.01\\
96.89	0.01\\
96.9	0.01\\
96.91	0.01\\
96.92	0.01\\
96.93	0.01\\
96.94	0.01\\
96.95	0.01\\
96.96	0.01\\
96.97	0.01\\
96.98	0.01\\
96.99	0.01\\
97	0.01\\
97.01	0.01\\
97.02	0.01\\
97.03	0.01\\
97.04	0.01\\
97.05	0.01\\
97.06	0.01\\
97.07	0.01\\
97.08	0.01\\
97.09	0.01\\
97.1	0.01\\
97.11	0.01\\
97.12	0.01\\
97.13	0.01\\
97.14	0.01\\
97.15	0.01\\
97.16	0.01\\
97.17	0.01\\
97.18	0.01\\
97.19	0.01\\
97.2	0.01\\
97.21	0.01\\
97.22	0.01\\
97.23	0.01\\
97.24	0.01\\
97.25	0.01\\
97.26	0.01\\
97.27	0.01\\
97.28	0.01\\
97.29	0.01\\
97.3	0.01\\
97.31	0.01\\
97.32	0.01\\
97.33	0.01\\
97.34	0.01\\
97.35	0.01\\
97.36	0.01\\
97.37	0.01\\
97.38	0.01\\
97.39	0.01\\
97.4	0.01\\
97.41	0.01\\
97.42	0.01\\
97.43	0.01\\
97.44	0.01\\
97.45	0.01\\
97.46	0.01\\
97.47	0.01\\
97.48	0.01\\
97.49	0.01\\
97.5	0.01\\
97.51	0.01\\
97.52	0.01\\
97.53	0.01\\
97.54	0.01\\
97.55	0.01\\
97.56	0.01\\
97.57	0.01\\
97.58	0.01\\
97.59	0.01\\
97.6	0.01\\
97.61	0.01\\
97.62	0.01\\
97.63	0.01\\
97.64	0.01\\
97.65	0.01\\
97.66	0.01\\
97.67	0.01\\
97.68	0.01\\
97.69	0.01\\
97.7	0.01\\
97.71	0.01\\
97.72	0.01\\
97.73	0.01\\
97.74	0.01\\
97.75	0.01\\
97.76	0.01\\
97.77	0.01\\
97.78	0.01\\
97.79	0.01\\
97.8	0.01\\
97.81	0.01\\
97.82	0.01\\
97.83	0.01\\
97.84	0.01\\
97.85	0.01\\
97.86	0.01\\
97.87	0.01\\
97.88	0.01\\
97.89	0.01\\
97.9	0.01\\
97.91	0.01\\
97.92	0.01\\
97.93	0.01\\
97.94	0.01\\
97.95	0.01\\
97.96	0.01\\
97.97	0.01\\
97.98	0.01\\
97.99	0.01\\
98	0.01\\
98.01	0.01\\
98.02	0.01\\
98.03	0.01\\
98.04	0.01\\
98.05	0.01\\
98.06	0.01\\
98.07	0.01\\
98.08	0.01\\
98.09	0.01\\
98.1	0.01\\
98.11	0.01\\
98.12	0.01\\
98.13	0.01\\
98.14	0.01\\
98.15	0.01\\
98.16	0.01\\
98.17	0.01\\
98.18	0.01\\
98.19	0.01\\
98.2	0.01\\
98.21	0.01\\
98.22	0.01\\
98.23	0.01\\
98.24	0.01\\
98.25	0.01\\
98.26	0.01\\
98.27	0.01\\
98.28	0.01\\
98.29	0.01\\
98.3	0.01\\
98.31	0.01\\
98.32	0.01\\
98.33	0.01\\
98.34	0.01\\
98.35	0.01\\
98.36	0.01\\
98.37	0.01\\
98.38	0.01\\
98.39	0.01\\
98.4	0.01\\
98.41	0.01\\
98.42	0.01\\
98.43	0.01\\
98.44	0.01\\
98.45	0.01\\
98.46	0.01\\
98.47	0.01\\
98.48	0.01\\
98.49	0.01\\
98.5	0.01\\
98.51	0.01\\
98.52	0.01\\
98.53	0.01\\
98.54	0.01\\
98.55	0.01\\
98.56	0.01\\
98.57	0.01\\
98.58	0.01\\
98.59	0.01\\
98.6	0.01\\
98.61	0.01\\
98.62	0.01\\
98.63	0.01\\
98.64	0.01\\
98.65	0.01\\
98.66	0.01\\
98.67	0.01\\
98.68	0.01\\
98.69	0.01\\
98.7	0.01\\
98.71	0.01\\
98.72	0.01\\
98.73	0.01\\
98.74	0.01\\
98.75	0.01\\
98.76	0.01\\
98.77	0.01\\
98.78	0.01\\
98.79	0.01\\
98.8	0.01\\
98.81	0.01\\
98.82	0.01\\
98.83	0.01\\
98.84	0.01\\
98.85	0.01\\
98.86	0.01\\
98.87	0.01\\
98.88	0.01\\
98.89	0.01\\
98.9	0.01\\
98.91	0.01\\
98.92	0.01\\
98.93	0.01\\
98.94	0.01\\
98.95	0.01\\
98.96	0.01\\
98.97	0.01\\
98.98	0.01\\
98.99	0.01\\
99	0.01\\
99.01	0.01\\
99.02	0.01\\
99.03	0.01\\
99.04	0.01\\
99.05	0.01\\
99.06	0.01\\
99.07	0.01\\
99.08	0.01\\
99.09	0.01\\
99.1	0.01\\
99.11	0.01\\
99.12	0.01\\
99.13	0.01\\
99.14	0.01\\
99.15	0.01\\
99.16	0.01\\
99.17	0.01\\
99.18	0.01\\
99.19	0.01\\
99.2	0.01\\
99.21	0.01\\
99.22	0.01\\
99.23	0.01\\
99.24	0.01\\
99.25	0.01\\
99.26	0.01\\
99.27	0.01\\
99.28	0.01\\
99.29	0.01\\
99.3	0.01\\
99.31	0.01\\
99.32	0.01\\
99.33	0.01\\
99.34	0.01\\
99.35	0.01\\
99.36	0.01\\
99.37	0.01\\
99.38	0.01\\
99.39	0.01\\
99.4	0.01\\
99.41	0.01\\
99.42	0.01\\
99.43	0.01\\
99.44	0.01\\
99.45	0.01\\
99.46	0.01\\
99.47	0.01\\
99.48	0.01\\
99.49	0.01\\
99.5	0.01\\
99.51	0.01\\
99.52	0.01\\
99.53	0.01\\
99.54	0.01\\
99.55	0.01\\
99.56	0.01\\
99.57	0.01\\
99.58	0.01\\
99.59	0.01\\
99.6	0.01\\
99.61	0.01\\
99.62	0.01\\
99.63	0.01\\
99.64	0.01\\
99.65	0.01\\
99.66	0.01\\
99.67	0.01\\
99.68	0.01\\
99.69	0.01\\
99.7	0.01\\
99.71	0.01\\
99.72	0.01\\
99.73	0.01\\
99.74	0.01\\
99.75	0.01\\
99.76	0.01\\
99.77	0.01\\
99.78	0.01\\
99.79	0.01\\
99.8	0.01\\
99.81	0.01\\
99.82	0.01\\
99.83	0.01\\
99.84	0.01\\
99.85	0.01\\
99.86	0.01\\
99.87	0.01\\
99.88	0.01\\
99.89	0.01\\
99.9	0.01\\
99.91	0.01\\
99.92	0.01\\
99.93	0.01\\
99.94	0.01\\
99.95	0.01\\
99.96	0.01\\
99.97	0.01\\
99.98	0.01\\
99.99	0.01\\
100	0.01\\
};
\addlegendentry{$q=2$};

\addplot [color=mycolor1,solid,forget plot]
  table[row sep=crcr]{%
0.01	0.01\\
0.02	0.01\\
0.03	0.01\\
0.04	0.01\\
0.05	0.01\\
0.06	0.01\\
0.07	0.01\\
0.08	0.01\\
0.09	0.01\\
0.1	0.01\\
0.11	0.01\\
0.12	0.01\\
0.13	0.01\\
0.14	0.01\\
0.15	0.01\\
0.16	0.01\\
0.17	0.01\\
0.18	0.01\\
0.19	0.01\\
0.2	0.01\\
0.21	0.01\\
0.22	0.01\\
0.23	0.01\\
0.24	0.01\\
0.25	0.01\\
0.26	0.01\\
0.27	0.01\\
0.28	0.01\\
0.29	0.01\\
0.3	0.01\\
0.31	0.01\\
0.32	0.01\\
0.33	0.01\\
0.34	0.01\\
0.35	0.01\\
0.36	0.01\\
0.37	0.01\\
0.38	0.01\\
0.39	0.01\\
0.4	0.01\\
0.41	0.01\\
0.42	0.01\\
0.43	0.01\\
0.44	0.01\\
0.45	0.01\\
0.46	0.01\\
0.47	0.01\\
0.48	0.01\\
0.49	0.01\\
0.5	0.01\\
0.51	0.01\\
0.52	0.01\\
0.53	0.01\\
0.54	0.01\\
0.55	0.01\\
0.56	0.01\\
0.57	0.01\\
0.58	0.01\\
0.59	0.01\\
0.6	0.01\\
0.61	0.01\\
0.62	0.01\\
0.63	0.01\\
0.64	0.01\\
0.65	0.01\\
0.66	0.01\\
0.67	0.01\\
0.68	0.01\\
0.69	0.01\\
0.7	0.01\\
0.71	0.01\\
0.72	0.01\\
0.73	0.01\\
0.74	0.01\\
0.75	0.01\\
0.76	0.01\\
0.77	0.01\\
0.78	0.01\\
0.79	0.01\\
0.8	0.01\\
0.81	0.01\\
0.82	0.01\\
0.83	0.01\\
0.84	0.01\\
0.85	0.01\\
0.86	0.01\\
0.87	0.01\\
0.88	0.01\\
0.89	0.01\\
0.9	0.01\\
0.91	0.01\\
0.92	0.01\\
0.93	0.01\\
0.94	0.01\\
0.95	0.01\\
0.96	0.01\\
0.97	0.01\\
0.98	0.01\\
0.99	0.01\\
1	0.01\\
1.01	0.01\\
1.02	0.01\\
1.03	0.01\\
1.04	0.01\\
1.05	0.01\\
1.06	0.01\\
1.07	0.01\\
1.08	0.01\\
1.09	0.01\\
1.1	0.01\\
1.11	0.01\\
1.12	0.01\\
1.13	0.01\\
1.14	0.01\\
1.15	0.01\\
1.16	0.01\\
1.17	0.01\\
1.18	0.01\\
1.19	0.01\\
1.2	0.01\\
1.21	0.01\\
1.22	0.01\\
1.23	0.01\\
1.24	0.01\\
1.25	0.01\\
1.26	0.01\\
1.27	0.01\\
1.28	0.01\\
1.29	0.01\\
1.3	0.01\\
1.31	0.01\\
1.32	0.01\\
1.33	0.01\\
1.34	0.01\\
1.35	0.01\\
1.36	0.01\\
1.37	0.01\\
1.38	0.01\\
1.39	0.01\\
1.4	0.01\\
1.41	0.01\\
1.42	0.01\\
1.43	0.01\\
1.44	0.01\\
1.45	0.01\\
1.46	0.01\\
1.47	0.01\\
1.48	0.01\\
1.49	0.01\\
1.5	0.01\\
1.51	0.01\\
1.52	0.01\\
1.53	0.01\\
1.54	0.01\\
1.55	0.01\\
1.56	0.01\\
1.57	0.01\\
1.58	0.01\\
1.59	0.01\\
1.6	0.01\\
1.61	0.01\\
1.62	0.01\\
1.63	0.01\\
1.64	0.01\\
1.65	0.01\\
1.66	0.01\\
1.67	0.01\\
1.68	0.01\\
1.69	0.01\\
1.7	0.01\\
1.71	0.01\\
1.72	0.01\\
1.73	0.01\\
1.74	0.01\\
1.75	0.01\\
1.76	0.01\\
1.77	0.01\\
1.78	0.01\\
1.79	0.01\\
1.8	0.01\\
1.81	0.01\\
1.82	0.01\\
1.83	0.01\\
1.84	0.01\\
1.85	0.01\\
1.86	0.01\\
1.87	0.01\\
1.88	0.01\\
1.89	0.01\\
1.9	0.01\\
1.91	0.01\\
1.92	0.01\\
1.93	0.01\\
1.94	0.01\\
1.95	0.01\\
1.96	0.01\\
1.97	0.01\\
1.98	0.01\\
1.99	0.01\\
2	0.01\\
2.01	0.01\\
2.02	0.01\\
2.03	0.01\\
2.04	0.01\\
2.05	0.01\\
2.06	0.01\\
2.07	0.01\\
2.08	0.01\\
2.09	0.01\\
2.1	0.01\\
2.11	0.01\\
2.12	0.01\\
2.13	0.01\\
2.14	0.01\\
2.15	0.01\\
2.16	0.01\\
2.17	0.01\\
2.18	0.01\\
2.19	0.01\\
2.2	0.01\\
2.21	0.01\\
2.22	0.01\\
2.23	0.01\\
2.24	0.01\\
2.25	0.01\\
2.26	0.01\\
2.27	0.01\\
2.28	0.01\\
2.29	0.01\\
2.3	0.01\\
2.31	0.01\\
2.32	0.01\\
2.33	0.01\\
2.34	0.01\\
2.35	0.01\\
2.36	0.01\\
2.37	0.01\\
2.38	0.01\\
2.39	0.01\\
2.4	0.01\\
2.41	0.01\\
2.42	0.01\\
2.43	0.01\\
2.44	0.01\\
2.45	0.01\\
2.46	0.01\\
2.47	0.01\\
2.48	0.01\\
2.49	0.01\\
2.5	0.01\\
2.51	0.01\\
2.52	0.01\\
2.53	0.01\\
2.54	0.01\\
2.55	0.01\\
2.56	0.01\\
2.57	0.01\\
2.58	0.01\\
2.59	0.01\\
2.6	0.01\\
2.61	0.01\\
2.62	0.01\\
2.63	0.01\\
2.64	0.01\\
2.65	0.01\\
2.66	0.01\\
2.67	0.01\\
2.68	0.01\\
2.69	0.01\\
2.7	0.01\\
2.71	0.01\\
2.72	0.01\\
2.73	0.01\\
2.74	0.01\\
2.75	0.01\\
2.76	0.01\\
2.77	0.01\\
2.78	0.01\\
2.79	0.01\\
2.8	0.01\\
2.81	0.01\\
2.82	0.01\\
2.83	0.01\\
2.84	0.01\\
2.85	0.01\\
2.86	0.01\\
2.87	0.01\\
2.88	0.01\\
2.89	0.01\\
2.9	0.01\\
2.91	0.01\\
2.92	0.01\\
2.93	0.01\\
2.94	0.01\\
2.95	0.01\\
2.96	0.01\\
2.97	0.01\\
2.98	0.01\\
2.99	0.01\\
3	0.01\\
3.01	0.01\\
3.02	0.01\\
3.03	0.01\\
3.04	0.01\\
3.05	0.01\\
3.06	0.01\\
3.07	0.01\\
3.08	0.01\\
3.09	0.01\\
3.1	0.01\\
3.11	0.01\\
3.12	0.01\\
3.13	0.01\\
3.14	0.01\\
3.15	0.01\\
3.16	0.01\\
3.17	0.01\\
3.18	0.01\\
3.19	0.01\\
3.2	0.01\\
3.21	0.01\\
3.22	0.01\\
3.23	0.01\\
3.24	0.01\\
3.25	0.01\\
3.26	0.01\\
3.27	0.01\\
3.28	0.01\\
3.29	0.01\\
3.3	0.01\\
3.31	0.01\\
3.32	0.01\\
3.33	0.01\\
3.34	0.01\\
3.35	0.01\\
3.36	0.01\\
3.37	0.01\\
3.38	0.01\\
3.39	0.01\\
3.4	0.01\\
3.41	0.01\\
3.42	0.01\\
3.43	0.01\\
3.44	0.01\\
3.45	0.01\\
3.46	0.01\\
3.47	0.01\\
3.48	0.01\\
3.49	0.01\\
3.5	0.01\\
3.51	0.01\\
3.52	0.01\\
3.53	0.01\\
3.54	0.01\\
3.55	0.01\\
3.56	0.01\\
3.57	0.01\\
3.58	0.01\\
3.59	0.01\\
3.6	0.01\\
3.61	0.01\\
3.62	0.01\\
3.63	0.01\\
3.64	0.01\\
3.65	0.01\\
3.66	0.01\\
3.67	0.01\\
3.68	0.01\\
3.69	0.01\\
3.7	0.01\\
3.71	0.01\\
3.72	0.01\\
3.73	0.01\\
3.74	0.01\\
3.75	0.01\\
3.76	0.01\\
3.77	0.01\\
3.78	0.01\\
3.79	0.01\\
3.8	0.01\\
3.81	0.01\\
3.82	0.01\\
3.83	0.01\\
3.84	0.01\\
3.85	0.01\\
3.86	0.01\\
3.87	0.01\\
3.88	0.01\\
3.89	0.01\\
3.9	0.01\\
3.91	0.01\\
3.92	0.01\\
3.93	0.01\\
3.94	0.01\\
3.95	0.01\\
3.96	0.01\\
3.97	0.01\\
3.98	0.01\\
3.99	0.01\\
4	0.01\\
4.01	0.01\\
4.02	0.01\\
4.03	0.01\\
4.04	0.01\\
4.05	0.01\\
4.06	0.01\\
4.07	0.01\\
4.08	0.01\\
4.09	0.01\\
4.1	0.01\\
4.11	0.01\\
4.12	0.01\\
4.13	0.01\\
4.14	0.01\\
4.15	0.01\\
4.16	0.01\\
4.17	0.01\\
4.18	0.01\\
4.19	0.01\\
4.2	0.01\\
4.21	0.01\\
4.22	0.01\\
4.23	0.01\\
4.24	0.01\\
4.25	0.01\\
4.26	0.01\\
4.27	0.01\\
4.28	0.01\\
4.29	0.01\\
4.3	0.01\\
4.31	0.01\\
4.32	0.01\\
4.33	0.01\\
4.34	0.01\\
4.35	0.01\\
4.36	0.01\\
4.37	0.01\\
4.38	0.01\\
4.39	0.01\\
4.4	0.01\\
4.41	0.01\\
4.42	0.01\\
4.43	0.01\\
4.44	0.01\\
4.45	0.01\\
4.46	0.01\\
4.47	0.01\\
4.48	0.01\\
4.49	0.01\\
4.5	0.01\\
4.51	0.01\\
4.52	0.01\\
4.53	0.01\\
4.54	0.01\\
4.55	0.01\\
4.56	0.01\\
4.57	0.01\\
4.58	0.01\\
4.59	0.01\\
4.6	0.01\\
4.61	0.01\\
4.62	0.01\\
4.63	0.01\\
4.64	0.01\\
4.65	0.01\\
4.66	0.01\\
4.67	0.01\\
4.68	0.01\\
4.69	0.01\\
4.7	0.01\\
4.71	0.01\\
4.72	0.01\\
4.73	0.01\\
4.74	0.01\\
4.75	0.01\\
4.76	0.01\\
4.77	0.01\\
4.78	0.01\\
4.79	0.01\\
4.8	0.01\\
4.81	0.01\\
4.82	0.01\\
4.83	0.01\\
4.84	0.01\\
4.85	0.01\\
4.86	0.01\\
4.87	0.01\\
4.88	0.01\\
4.89	0.01\\
4.9	0.01\\
4.91	0.01\\
4.92	0.01\\
4.93	0.01\\
4.94	0.01\\
4.95	0.01\\
4.96	0.01\\
4.97	0.01\\
4.98	0.01\\
4.99	0.01\\
5	0.01\\
5.01	0.01\\
5.02	0.01\\
5.03	0.01\\
5.04	0.01\\
5.05	0.01\\
5.06	0.01\\
5.07	0.01\\
5.08	0.01\\
5.09	0.01\\
5.1	0.01\\
5.11	0.01\\
5.12	0.01\\
5.13	0.01\\
5.14	0.01\\
5.15	0.01\\
5.16	0.01\\
5.17	0.01\\
5.18	0.01\\
5.19	0.01\\
5.2	0.01\\
5.21	0.01\\
5.22	0.01\\
5.23	0.01\\
5.24	0.01\\
5.25	0.01\\
5.26	0.01\\
5.27	0.01\\
5.28	0.01\\
5.29	0.01\\
5.3	0.01\\
5.31	0.01\\
5.32	0.01\\
5.33	0.01\\
5.34	0.01\\
5.35	0.01\\
5.36	0.01\\
5.37	0.01\\
5.38	0.01\\
5.39	0.01\\
5.4	0.01\\
5.41	0.01\\
5.42	0.01\\
5.43	0.01\\
5.44	0.01\\
5.45	0.01\\
5.46	0.01\\
5.47	0.01\\
5.48	0.01\\
5.49	0.01\\
5.5	0.01\\
5.51	0.01\\
5.52	0.01\\
5.53	0.01\\
5.54	0.01\\
5.55	0.01\\
5.56	0.01\\
5.57	0.01\\
5.58	0.01\\
5.59	0.01\\
5.6	0.01\\
5.61	0.01\\
5.62	0.01\\
5.63	0.01\\
5.64	0.01\\
5.65	0.01\\
5.66	0.01\\
5.67	0.01\\
5.68	0.01\\
5.69	0.01\\
5.7	0.01\\
5.71	0.01\\
5.72	0.01\\
5.73	0.01\\
5.74	0.01\\
5.75	0.01\\
5.76	0.01\\
5.77	0.01\\
5.78	0.01\\
5.79	0.01\\
5.8	0.01\\
5.81	0.01\\
5.82	0.01\\
5.83	0.01\\
5.84	0.01\\
5.85	0.01\\
5.86	0.01\\
5.87	0.01\\
5.88	0.01\\
5.89	0.01\\
5.9	0.01\\
5.91	0.01\\
5.92	0.01\\
5.93	0.01\\
5.94	0.01\\
5.95	0.01\\
5.96	0.01\\
5.97	0.01\\
5.98	0.01\\
5.99	0.01\\
6	0.01\\
6.01	0.01\\
6.02	0.01\\
6.03	0.01\\
6.04	0.01\\
6.05	0.01\\
6.06	0.01\\
6.07	0.01\\
6.08	0.01\\
6.09	0.01\\
6.1	0.01\\
6.11	0.01\\
6.12	0.01\\
6.13	0.01\\
6.14	0.01\\
6.15	0.01\\
6.16	0.01\\
6.17	0.01\\
6.18	0.01\\
6.19	0.01\\
6.2	0.01\\
6.21	0.01\\
6.22	0.01\\
6.23	0.01\\
6.24	0.01\\
6.25	0.01\\
6.26	0.01\\
6.27	0.01\\
6.28	0.01\\
6.29	0.01\\
6.3	0.01\\
6.31	0.01\\
6.32	0.01\\
6.33	0.01\\
6.34	0.01\\
6.35	0.01\\
6.36	0.01\\
6.37	0.01\\
6.38	0.01\\
6.39	0.01\\
6.4	0.01\\
6.41	0.01\\
6.42	0.01\\
6.43	0.01\\
6.44	0.01\\
6.45	0.01\\
6.46	0.01\\
6.47	0.01\\
6.48	0.01\\
6.49	0.01\\
6.5	0.01\\
6.51	0.01\\
6.52	0.01\\
6.53	0.01\\
6.54	0.01\\
6.55	0.01\\
6.56	0.01\\
6.57	0.01\\
6.58	0.01\\
6.59	0.01\\
6.6	0.01\\
6.61	0.01\\
6.62	0.01\\
6.63	0.01\\
6.64	0.01\\
6.65	0.01\\
6.66	0.01\\
6.67	0.01\\
6.68	0.01\\
6.69	0.01\\
6.7	0.01\\
6.71	0.01\\
6.72	0.01\\
6.73	0.01\\
6.74	0.01\\
6.75	0.01\\
6.76	0.01\\
6.77	0.01\\
6.78	0.01\\
6.79	0.01\\
6.8	0.01\\
6.81	0.01\\
6.82	0.01\\
6.83	0.01\\
6.84	0.01\\
6.85	0.01\\
6.86	0.01\\
6.87	0.01\\
6.88	0.01\\
6.89	0.01\\
6.9	0.01\\
6.91	0.01\\
6.92	0.01\\
6.93	0.01\\
6.94	0.01\\
6.95	0.01\\
6.96	0.01\\
6.97	0.01\\
6.98	0.01\\
6.99	0.01\\
7	0.01\\
7.01	0.01\\
7.02	0.01\\
7.03	0.01\\
7.04	0.01\\
7.05	0.01\\
7.06	0.01\\
7.07	0.01\\
7.08	0.01\\
7.09	0.01\\
7.1	0.01\\
7.11	0.01\\
7.12	0.01\\
7.13	0.01\\
7.14	0.01\\
7.15	0.01\\
7.16	0.01\\
7.17	0.01\\
7.18	0.01\\
7.19	0.01\\
7.2	0.01\\
7.21	0.01\\
7.22	0.01\\
7.23	0.01\\
7.24	0.01\\
7.25	0.01\\
7.26	0.01\\
7.27	0.01\\
7.28	0.01\\
7.29	0.01\\
7.3	0.01\\
7.31	0.01\\
7.32	0.01\\
7.33	0.01\\
7.34	0.01\\
7.35	0.01\\
7.36	0.01\\
7.37	0.01\\
7.38	0.01\\
7.39	0.01\\
7.4	0.01\\
7.41	0.01\\
7.42	0.01\\
7.43	0.01\\
7.44	0.01\\
7.45	0.01\\
7.46	0.01\\
7.47	0.01\\
7.48	0.01\\
7.49	0.01\\
7.5	0.01\\
7.51	0.01\\
7.52	0.01\\
7.53	0.01\\
7.54	0.01\\
7.55	0.01\\
7.56	0.01\\
7.57	0.01\\
7.58	0.01\\
7.59	0.01\\
7.6	0.01\\
7.61	0.01\\
7.62	0.01\\
7.63	0.01\\
7.64	0.01\\
7.65	0.01\\
7.66	0.01\\
7.67	0.01\\
7.68	0.01\\
7.69	0.01\\
7.7	0.01\\
7.71	0.01\\
7.72	0.01\\
7.73	0.01\\
7.74	0.01\\
7.75	0.01\\
7.76	0.01\\
7.77	0.01\\
7.78	0.01\\
7.79	0.01\\
7.8	0.01\\
7.81	0.01\\
7.82	0.01\\
7.83	0.01\\
7.84	0.01\\
7.85	0.01\\
7.86	0.01\\
7.87	0.01\\
7.88	0.01\\
7.89	0.01\\
7.9	0.01\\
7.91	0.01\\
7.92	0.01\\
7.93	0.01\\
7.94	0.01\\
7.95	0.01\\
7.96	0.01\\
7.97	0.01\\
7.98	0.01\\
7.99	0.01\\
8	0.01\\
8.01	0.01\\
8.02	0.01\\
8.03	0.01\\
8.04	0.01\\
8.05	0.01\\
8.06	0.01\\
8.07	0.01\\
8.08	0.01\\
8.09	0.01\\
8.1	0.01\\
8.11	0.01\\
8.12	0.01\\
8.13	0.01\\
8.14	0.01\\
8.15	0.01\\
8.16	0.01\\
8.17	0.01\\
8.18	0.01\\
8.19	0.01\\
8.2	0.01\\
8.21	0.01\\
8.22	0.01\\
8.23	0.01\\
8.24	0.01\\
8.25	0.01\\
8.26	0.01\\
8.27	0.01\\
8.28	0.01\\
8.29	0.01\\
8.3	0.01\\
8.31	0.01\\
8.32	0.01\\
8.33	0.01\\
8.34	0.01\\
8.35	0.01\\
8.36	0.01\\
8.37	0.01\\
8.38	0.01\\
8.39	0.01\\
8.4	0.01\\
8.41	0.01\\
8.42	0.01\\
8.43	0.01\\
8.44	0.01\\
8.45	0.01\\
8.46	0.01\\
8.47	0.01\\
8.48	0.01\\
8.49	0.01\\
8.5	0.01\\
8.51	0.01\\
8.52	0.01\\
8.53	0.01\\
8.54	0.01\\
8.55	0.01\\
8.56	0.01\\
8.57	0.01\\
8.58	0.01\\
8.59	0.01\\
8.6	0.01\\
8.61	0.01\\
8.62	0.01\\
8.63	0.01\\
8.64	0.01\\
8.65	0.01\\
8.66	0.01\\
8.67	0.01\\
8.68	0.01\\
8.69	0.01\\
8.7	0.01\\
8.71	0.01\\
8.72	0.01\\
8.73	0.01\\
8.74	0.01\\
8.75	0.01\\
8.76	0.01\\
8.77	0.01\\
8.78	0.01\\
8.79	0.01\\
8.8	0.01\\
8.81	0.01\\
8.82	0.01\\
8.83	0.01\\
8.84	0.01\\
8.85	0.01\\
8.86	0.01\\
8.87	0.01\\
8.88	0.01\\
8.89	0.01\\
8.9	0.01\\
8.91	0.01\\
8.92	0.01\\
8.93	0.01\\
8.94	0.01\\
8.95	0.01\\
8.96	0.01\\
8.97	0.01\\
8.98	0.01\\
8.99	0.01\\
9	0.01\\
9.01	0.01\\
9.02	0.01\\
9.03	0.01\\
9.04	0.01\\
9.05	0.01\\
9.06	0.01\\
9.07	0.01\\
9.08	0.01\\
9.09	0.01\\
9.1	0.01\\
9.11	0.01\\
9.12	0.01\\
9.13	0.01\\
9.14	0.01\\
9.15	0.01\\
9.16	0.01\\
9.17	0.01\\
9.18	0.01\\
9.19	0.01\\
9.2	0.01\\
9.21	0.01\\
9.22	0.01\\
9.23	0.01\\
9.24	0.01\\
9.25	0.01\\
9.26	0.01\\
9.27	0.01\\
9.28	0.01\\
9.29	0.01\\
9.3	0.01\\
9.31	0.01\\
9.32	0.01\\
9.33	0.01\\
9.34	0.01\\
9.35	0.01\\
9.36	0.01\\
9.37	0.01\\
9.38	0.01\\
9.39	0.01\\
9.4	0.01\\
9.41	0.01\\
9.42	0.01\\
9.43	0.01\\
9.44	0.01\\
9.45	0.01\\
9.46	0.01\\
9.47	0.01\\
9.48	0.01\\
9.49	0.01\\
9.5	0.01\\
9.51	0.01\\
9.52	0.01\\
9.53	0.01\\
9.54	0.01\\
9.55	0.01\\
9.56	0.01\\
9.57	0.01\\
9.58	0.01\\
9.59	0.01\\
9.6	0.01\\
9.61	0.01\\
9.62	0.01\\
9.63	0.01\\
9.64	0.01\\
9.65	0.01\\
9.66	0.01\\
9.67	0.01\\
9.68	0.01\\
9.69	0.01\\
9.7	0.01\\
9.71	0.01\\
9.72	0.01\\
9.73	0.01\\
9.74	0.01\\
9.75	0.01\\
9.76	0.01\\
9.77	0.01\\
9.78	0.01\\
9.79	0.01\\
9.8	0.01\\
9.81	0.01\\
9.82	0.01\\
9.83	0.01\\
9.84	0.01\\
9.85	0.01\\
9.86	0.01\\
9.87	0.01\\
9.88	0.01\\
9.89	0.01\\
9.9	0.01\\
9.91	0.01\\
9.92	0.01\\
9.93	0.01\\
9.94	0.01\\
9.95	0.01\\
9.96	0.01\\
9.97	0.01\\
9.98	0.01\\
9.99	0.01\\
10	0.01\\
10.01	0.01\\
10.02	0.01\\
10.03	0.01\\
10.04	0.01\\
10.05	0.01\\
10.06	0.01\\
10.07	0.01\\
10.08	0.01\\
10.09	0.01\\
10.1	0.01\\
10.11	0.01\\
10.12	0.01\\
10.13	0.01\\
10.14	0.01\\
10.15	0.01\\
10.16	0.01\\
10.17	0.01\\
10.18	0.01\\
10.19	0.01\\
10.2	0.01\\
10.21	0.01\\
10.22	0.01\\
10.23	0.01\\
10.24	0.01\\
10.25	0.01\\
10.26	0.01\\
10.27	0.01\\
10.28	0.01\\
10.29	0.01\\
10.3	0.01\\
10.31	0.01\\
10.32	0.01\\
10.33	0.01\\
10.34	0.01\\
10.35	0.01\\
10.36	0.01\\
10.37	0.01\\
10.38	0.01\\
10.39	0.01\\
10.4	0.01\\
10.41	0.01\\
10.42	0.01\\
10.43	0.01\\
10.44	0.01\\
10.45	0.01\\
10.46	0.01\\
10.47	0.01\\
10.48	0.01\\
10.49	0.01\\
10.5	0.01\\
10.51	0.01\\
10.52	0.01\\
10.53	0.01\\
10.54	0.01\\
10.55	0.01\\
10.56	0.01\\
10.57	0.01\\
10.58	0.01\\
10.59	0.01\\
10.6	0.01\\
10.61	0.01\\
10.62	0.01\\
10.63	0.01\\
10.64	0.01\\
10.65	0.01\\
10.66	0.01\\
10.67	0.01\\
10.68	0.01\\
10.69	0.01\\
10.7	0.01\\
10.71	0.01\\
10.72	0.01\\
10.73	0.01\\
10.74	0.01\\
10.75	0.01\\
10.76	0.01\\
10.77	0.01\\
10.78	0.01\\
10.79	0.01\\
10.8	0.01\\
10.81	0.01\\
10.82	0.01\\
10.83	0.01\\
10.84	0.01\\
10.85	0.01\\
10.86	0.01\\
10.87	0.01\\
10.88	0.01\\
10.89	0.01\\
10.9	0.01\\
10.91	0.01\\
10.92	0.01\\
10.93	0.01\\
10.94	0.01\\
10.95	0.01\\
10.96	0.01\\
10.97	0.01\\
10.98	0.01\\
10.99	0.01\\
11	0.01\\
11.01	0.01\\
11.02	0.01\\
11.03	0.01\\
11.04	0.01\\
11.05	0.01\\
11.06	0.01\\
11.07	0.01\\
11.08	0.01\\
11.09	0.01\\
11.1	0.01\\
11.11	0.01\\
11.12	0.01\\
11.13	0.01\\
11.14	0.01\\
11.15	0.01\\
11.16	0.01\\
11.17	0.01\\
11.18	0.01\\
11.19	0.01\\
11.2	0.01\\
11.21	0.01\\
11.22	0.01\\
11.23	0.01\\
11.24	0.01\\
11.25	0.01\\
11.26	0.01\\
11.27	0.01\\
11.28	0.01\\
11.29	0.01\\
11.3	0.01\\
11.31	0.01\\
11.32	0.01\\
11.33	0.01\\
11.34	0.01\\
11.35	0.01\\
11.36	0.01\\
11.37	0.01\\
11.38	0.01\\
11.39	0.01\\
11.4	0.01\\
11.41	0.01\\
11.42	0.01\\
11.43	0.01\\
11.44	0.01\\
11.45	0.01\\
11.46	0.01\\
11.47	0.01\\
11.48	0.01\\
11.49	0.01\\
11.5	0.01\\
11.51	0.01\\
11.52	0.01\\
11.53	0.01\\
11.54	0.01\\
11.55	0.01\\
11.56	0.01\\
11.57	0.01\\
11.58	0.01\\
11.59	0.01\\
11.6	0.01\\
11.61	0.01\\
11.62	0.01\\
11.63	0.01\\
11.64	0.01\\
11.65	0.01\\
11.66	0.01\\
11.67	0.01\\
11.68	0.01\\
11.69	0.01\\
11.7	0.01\\
11.71	0.01\\
11.72	0.01\\
11.73	0.01\\
11.74	0.01\\
11.75	0.01\\
11.76	0.01\\
11.77	0.01\\
11.78	0.01\\
11.79	0.01\\
11.8	0.01\\
11.81	0.01\\
11.82	0.01\\
11.83	0.01\\
11.84	0.01\\
11.85	0.01\\
11.86	0.01\\
11.87	0.01\\
11.88	0.01\\
11.89	0.01\\
11.9	0.01\\
11.91	0.01\\
11.92	0.01\\
11.93	0.01\\
11.94	0.01\\
11.95	0.01\\
11.96	0.01\\
11.97	0.01\\
11.98	0.01\\
11.99	0.01\\
12	0.01\\
12.01	0.01\\
12.02	0.01\\
12.03	0.01\\
12.04	0.01\\
12.05	0.01\\
12.06	0.01\\
12.07	0.01\\
12.08	0.01\\
12.09	0.01\\
12.1	0.01\\
12.11	0.01\\
12.12	0.01\\
12.13	0.01\\
12.14	0.01\\
12.15	0.01\\
12.16	0.01\\
12.17	0.01\\
12.18	0.01\\
12.19	0.01\\
12.2	0.01\\
12.21	0.01\\
12.22	0.01\\
12.23	0.01\\
12.24	0.01\\
12.25	0.01\\
12.26	0.01\\
12.27	0.01\\
12.28	0.01\\
12.29	0.01\\
12.3	0.01\\
12.31	0.01\\
12.32	0.01\\
12.33	0.01\\
12.34	0.01\\
12.35	0.01\\
12.36	0.01\\
12.37	0.01\\
12.38	0.01\\
12.39	0.01\\
12.4	0.01\\
12.41	0.01\\
12.42	0.01\\
12.43	0.01\\
12.44	0.01\\
12.45	0.01\\
12.46	0.01\\
12.47	0.01\\
12.48	0.01\\
12.49	0.01\\
12.5	0.01\\
12.51	0.01\\
12.52	0.01\\
12.53	0.01\\
12.54	0.01\\
12.55	0.01\\
12.56	0.01\\
12.57	0.01\\
12.58	0.01\\
12.59	0.01\\
12.6	0.01\\
12.61	0.01\\
12.62	0.01\\
12.63	0.01\\
12.64	0.01\\
12.65	0.01\\
12.66	0.01\\
12.67	0.01\\
12.68	0.01\\
12.69	0.01\\
12.7	0.01\\
12.71	0.01\\
12.72	0.01\\
12.73	0.01\\
12.74	0.01\\
12.75	0.01\\
12.76	0.01\\
12.77	0.01\\
12.78	0.01\\
12.79	0.01\\
12.8	0.01\\
12.81	0.01\\
12.82	0.01\\
12.83	0.01\\
12.84	0.01\\
12.85	0.01\\
12.86	0.01\\
12.87	0.01\\
12.88	0.01\\
12.89	0.01\\
12.9	0.01\\
12.91	0.01\\
12.92	0.01\\
12.93	0.01\\
12.94	0.01\\
12.95	0.01\\
12.96	0.01\\
12.97	0.01\\
12.98	0.01\\
12.99	0.01\\
13	0.01\\
13.01	0.01\\
13.02	0.01\\
13.03	0.01\\
13.04	0.01\\
13.05	0.01\\
13.06	0.01\\
13.07	0.01\\
13.08	0.01\\
13.09	0.01\\
13.1	0.01\\
13.11	0.01\\
13.12	0.01\\
13.13	0.01\\
13.14	0.01\\
13.15	0.01\\
13.16	0.01\\
13.17	0.01\\
13.18	0.01\\
13.19	0.01\\
13.2	0.01\\
13.21	0.01\\
13.22	0.01\\
13.23	0.01\\
13.24	0.01\\
13.25	0.01\\
13.26	0.01\\
13.27	0.01\\
13.28	0.01\\
13.29	0.01\\
13.3	0.01\\
13.31	0.01\\
13.32	0.01\\
13.33	0.01\\
13.34	0.01\\
13.35	0.01\\
13.36	0.01\\
13.37	0.01\\
13.38	0.01\\
13.39	0.01\\
13.4	0.01\\
13.41	0.01\\
13.42	0.01\\
13.43	0.01\\
13.44	0.01\\
13.45	0.01\\
13.46	0.01\\
13.47	0.01\\
13.48	0.01\\
13.49	0.01\\
13.5	0.01\\
13.51	0.01\\
13.52	0.01\\
13.53	0.01\\
13.54	0.01\\
13.55	0.01\\
13.56	0.01\\
13.57	0.01\\
13.58	0.01\\
13.59	0.01\\
13.6	0.01\\
13.61	0.01\\
13.62	0.01\\
13.63	0.01\\
13.64	0.01\\
13.65	0.01\\
13.66	0.01\\
13.67	0.01\\
13.68	0.01\\
13.69	0.01\\
13.7	0.01\\
13.71	0.01\\
13.72	0.01\\
13.73	0.01\\
13.74	0.01\\
13.75	0.01\\
13.76	0.01\\
13.77	0.01\\
13.78	0.01\\
13.79	0.01\\
13.8	0.01\\
13.81	0.01\\
13.82	0.01\\
13.83	0.01\\
13.84	0.01\\
13.85	0.01\\
13.86	0.01\\
13.87	0.01\\
13.88	0.01\\
13.89	0.01\\
13.9	0.01\\
13.91	0.01\\
13.92	0.01\\
13.93	0.01\\
13.94	0.01\\
13.95	0.01\\
13.96	0.01\\
13.97	0.01\\
13.98	0.01\\
13.99	0.01\\
14	0.01\\
14.01	0.01\\
14.02	0.01\\
14.03	0.01\\
14.04	0.01\\
14.05	0.01\\
14.06	0.01\\
14.07	0.01\\
14.08	0.01\\
14.09	0.01\\
14.1	0.01\\
14.11	0.01\\
14.12	0.01\\
14.13	0.01\\
14.14	0.01\\
14.15	0.01\\
14.16	0.01\\
14.17	0.01\\
14.18	0.01\\
14.19	0.01\\
14.2	0.01\\
14.21	0.01\\
14.22	0.01\\
14.23	0.01\\
14.24	0.01\\
14.25	0.01\\
14.26	0.01\\
14.27	0.01\\
14.28	0.01\\
14.29	0.01\\
14.3	0.01\\
14.31	0.01\\
14.32	0.01\\
14.33	0.01\\
14.34	0.01\\
14.35	0.01\\
14.36	0.01\\
14.37	0.01\\
14.38	0.01\\
14.39	0.01\\
14.4	0.01\\
14.41	0.01\\
14.42	0.01\\
14.43	0.01\\
14.44	0.01\\
14.45	0.01\\
14.46	0.01\\
14.47	0.01\\
14.48	0.01\\
14.49	0.01\\
14.5	0.01\\
14.51	0.01\\
14.52	0.01\\
14.53	0.01\\
14.54	0.01\\
14.55	0.01\\
14.56	0.01\\
14.57	0.01\\
14.58	0.01\\
14.59	0.01\\
14.6	0.01\\
14.61	0.01\\
14.62	0.01\\
14.63	0.01\\
14.64	0.01\\
14.65	0.01\\
14.66	0.01\\
14.67	0.01\\
14.68	0.01\\
14.69	0.01\\
14.7	0.01\\
14.71	0.01\\
14.72	0.01\\
14.73	0.01\\
14.74	0.01\\
14.75	0.01\\
14.76	0.01\\
14.77	0.01\\
14.78	0.01\\
14.79	0.01\\
14.8	0.01\\
14.81	0.01\\
14.82	0.01\\
14.83	0.01\\
14.84	0.01\\
14.85	0.01\\
14.86	0.01\\
14.87	0.01\\
14.88	0.01\\
14.89	0.01\\
14.9	0.01\\
14.91	0.01\\
14.92	0.01\\
14.93	0.01\\
14.94	0.01\\
14.95	0.01\\
14.96	0.01\\
14.97	0.01\\
14.98	0.01\\
14.99	0.01\\
15	0.01\\
15.01	0.01\\
15.02	0.01\\
15.03	0.01\\
15.04	0.01\\
15.05	0.01\\
15.06	0.01\\
15.07	0.01\\
15.08	0.01\\
15.09	0.01\\
15.1	0.01\\
15.11	0.01\\
15.12	0.01\\
15.13	0.01\\
15.14	0.01\\
15.15	0.01\\
15.16	0.01\\
15.17	0.01\\
15.18	0.01\\
15.19	0.01\\
15.2	0.01\\
15.21	0.01\\
15.22	0.01\\
15.23	0.01\\
15.24	0.01\\
15.25	0.01\\
15.26	0.01\\
15.27	0.01\\
15.28	0.01\\
15.29	0.01\\
15.3	0.01\\
15.31	0.01\\
15.32	0.01\\
15.33	0.01\\
15.34	0.01\\
15.35	0.01\\
15.36	0.01\\
15.37	0.01\\
15.38	0.01\\
15.39	0.01\\
15.4	0.01\\
15.41	0.01\\
15.42	0.01\\
15.43	0.01\\
15.44	0.01\\
15.45	0.01\\
15.46	0.01\\
15.47	0.01\\
15.48	0.01\\
15.49	0.01\\
15.5	0.01\\
15.51	0.01\\
15.52	0.01\\
15.53	0.01\\
15.54	0.01\\
15.55	0.01\\
15.56	0.01\\
15.57	0.01\\
15.58	0.01\\
15.59	0.01\\
15.6	0.01\\
15.61	0.01\\
15.62	0.01\\
15.63	0.01\\
15.64	0.01\\
15.65	0.01\\
15.66	0.01\\
15.67	0.01\\
15.68	0.01\\
15.69	0.01\\
15.7	0.01\\
15.71	0.01\\
15.72	0.01\\
15.73	0.01\\
15.74	0.01\\
15.75	0.01\\
15.76	0.01\\
15.77	0.01\\
15.78	0.01\\
15.79	0.01\\
15.8	0.01\\
15.81	0.01\\
15.82	0.01\\
15.83	0.01\\
15.84	0.01\\
15.85	0.01\\
15.86	0.01\\
15.87	0.01\\
15.88	0.01\\
15.89	0.01\\
15.9	0.01\\
15.91	0.01\\
15.92	0.01\\
15.93	0.01\\
15.94	0.01\\
15.95	0.01\\
15.96	0.01\\
15.97	0.01\\
15.98	0.01\\
15.99	0.01\\
16	0.01\\
16.01	0.01\\
16.02	0.01\\
16.03	0.01\\
16.04	0.01\\
16.05	0.01\\
16.06	0.01\\
16.07	0.01\\
16.08	0.01\\
16.09	0.01\\
16.1	0.01\\
16.11	0.01\\
16.12	0.01\\
16.13	0.01\\
16.14	0.01\\
16.15	0.01\\
16.16	0.01\\
16.17	0.01\\
16.18	0.01\\
16.19	0.01\\
16.2	0.01\\
16.21	0.01\\
16.22	0.01\\
16.23	0.01\\
16.24	0.01\\
16.25	0.01\\
16.26	0.01\\
16.27	0.01\\
16.28	0.01\\
16.29	0.01\\
16.3	0.01\\
16.31	0.01\\
16.32	0.01\\
16.33	0.01\\
16.34	0.01\\
16.35	0.01\\
16.36	0.01\\
16.37	0.01\\
16.38	0.01\\
16.39	0.01\\
16.4	0.01\\
16.41	0.01\\
16.42	0.01\\
16.43	0.01\\
16.44	0.01\\
16.45	0.01\\
16.46	0.01\\
16.47	0.01\\
16.48	0.01\\
16.49	0.01\\
16.5	0.01\\
16.51	0.01\\
16.52	0.01\\
16.53	0.01\\
16.54	0.01\\
16.55	0.01\\
16.56	0.01\\
16.57	0.01\\
16.58	0.01\\
16.59	0.01\\
16.6	0.01\\
16.61	0.01\\
16.62	0.01\\
16.63	0.01\\
16.64	0.01\\
16.65	0.01\\
16.66	0.01\\
16.67	0.01\\
16.68	0.01\\
16.69	0.01\\
16.7	0.01\\
16.71	0.01\\
16.72	0.01\\
16.73	0.01\\
16.74	0.01\\
16.75	0.01\\
16.76	0.01\\
16.77	0.01\\
16.78	0.01\\
16.79	0.01\\
16.8	0.01\\
16.81	0.01\\
16.82	0.01\\
16.83	0.01\\
16.84	0.01\\
16.85	0.01\\
16.86	0.01\\
16.87	0.01\\
16.88	0.01\\
16.89	0.01\\
16.9	0.01\\
16.91	0.01\\
16.92	0.01\\
16.93	0.01\\
16.94	0.01\\
16.95	0.01\\
16.96	0.01\\
16.97	0.01\\
16.98	0.01\\
16.99	0.01\\
17	0.01\\
17.01	0.01\\
17.02	0.01\\
17.03	0.01\\
17.04	0.01\\
17.05	0.01\\
17.06	0.01\\
17.07	0.01\\
17.08	0.01\\
17.09	0.01\\
17.1	0.01\\
17.11	0.01\\
17.12	0.01\\
17.13	0.01\\
17.14	0.01\\
17.15	0.01\\
17.16	0.01\\
17.17	0.01\\
17.18	0.01\\
17.19	0.01\\
17.2	0.01\\
17.21	0.01\\
17.22	0.01\\
17.23	0.01\\
17.24	0.01\\
17.25	0.01\\
17.26	0.01\\
17.27	0.01\\
17.28	0.01\\
17.29	0.01\\
17.3	0.01\\
17.31	0.01\\
17.32	0.01\\
17.33	0.01\\
17.34	0.01\\
17.35	0.01\\
17.36	0.01\\
17.37	0.01\\
17.38	0.01\\
17.39	0.01\\
17.4	0.01\\
17.41	0.01\\
17.42	0.01\\
17.43	0.01\\
17.44	0.01\\
17.45	0.01\\
17.46	0.01\\
17.47	0.01\\
17.48	0.01\\
17.49	0.01\\
17.5	0.01\\
17.51	0.01\\
17.52	0.01\\
17.53	0.01\\
17.54	0.01\\
17.55	0.01\\
17.56	0.01\\
17.57	0.01\\
17.58	0.01\\
17.59	0.01\\
17.6	0.01\\
17.61	0.01\\
17.62	0.01\\
17.63	0.01\\
17.64	0.01\\
17.65	0.01\\
17.66	0.01\\
17.67	0.01\\
17.68	0.01\\
17.69	0.01\\
17.7	0.01\\
17.71	0.01\\
17.72	0.01\\
17.73	0.01\\
17.74	0.01\\
17.75	0.01\\
17.76	0.01\\
17.77	0.01\\
17.78	0.01\\
17.79	0.01\\
17.8	0.01\\
17.81	0.01\\
17.82	0.01\\
17.83	0.01\\
17.84	0.01\\
17.85	0.01\\
17.86	0.01\\
17.87	0.01\\
17.88	0.01\\
17.89	0.01\\
17.9	0.01\\
17.91	0.01\\
17.92	0.01\\
17.93	0.01\\
17.94	0.01\\
17.95	0.01\\
17.96	0.01\\
17.97	0.01\\
17.98	0.01\\
17.99	0.01\\
18	0.01\\
18.01	0.01\\
18.02	0.01\\
18.03	0.01\\
18.04	0.01\\
18.05	0.01\\
18.06	0.01\\
18.07	0.01\\
18.08	0.01\\
18.09	0.01\\
18.1	0.01\\
18.11	0.01\\
18.12	0.01\\
18.13	0.01\\
18.14	0.01\\
18.15	0.01\\
18.16	0.01\\
18.17	0.01\\
18.18	0.01\\
18.19	0.01\\
18.2	0.01\\
18.21	0.01\\
18.22	0.01\\
18.23	0.01\\
18.24	0.01\\
18.25	0.01\\
18.26	0.01\\
18.27	0.01\\
18.28	0.01\\
18.29	0.01\\
18.3	0.01\\
18.31	0.01\\
18.32	0.01\\
18.33	0.01\\
18.34	0.01\\
18.35	0.01\\
18.36	0.01\\
18.37	0.01\\
18.38	0.01\\
18.39	0.01\\
18.4	0.01\\
18.41	0.01\\
18.42	0.01\\
18.43	0.01\\
18.44	0.01\\
18.45	0.01\\
18.46	0.01\\
18.47	0.01\\
18.48	0.01\\
18.49	0.01\\
18.5	0.01\\
18.51	0.01\\
18.52	0.01\\
18.53	0.01\\
18.54	0.01\\
18.55	0.01\\
18.56	0.01\\
18.57	0.01\\
18.58	0.01\\
18.59	0.01\\
18.6	0.01\\
18.61	0.01\\
18.62	0.01\\
18.63	0.01\\
18.64	0.01\\
18.65	0.01\\
18.66	0.01\\
18.67	0.01\\
18.68	0.01\\
18.69	0.01\\
18.7	0.01\\
18.71	0.01\\
18.72	0.01\\
18.73	0.01\\
18.74	0.01\\
18.75	0.01\\
18.76	0.01\\
18.77	0.01\\
18.78	0.01\\
18.79	0.01\\
18.8	0.01\\
18.81	0.01\\
18.82	0.01\\
18.83	0.01\\
18.84	0.01\\
18.85	0.01\\
18.86	0.01\\
18.87	0.01\\
18.88	0.01\\
18.89	0.01\\
18.9	0.01\\
18.91	0.01\\
18.92	0.01\\
18.93	0.01\\
18.94	0.01\\
18.95	0.01\\
18.96	0.01\\
18.97	0.01\\
18.98	0.01\\
18.99	0.01\\
19	0.01\\
19.01	0.01\\
19.02	0.01\\
19.03	0.01\\
19.04	0.01\\
19.05	0.01\\
19.06	0.01\\
19.07	0.01\\
19.08	0.01\\
19.09	0.01\\
19.1	0.01\\
19.11	0.01\\
19.12	0.01\\
19.13	0.01\\
19.14	0.01\\
19.15	0.01\\
19.16	0.01\\
19.17	0.01\\
19.18	0.01\\
19.19	0.01\\
19.2	0.01\\
19.21	0.01\\
19.22	0.01\\
19.23	0.01\\
19.24	0.01\\
19.25	0.01\\
19.26	0.01\\
19.27	0.01\\
19.28	0.01\\
19.29	0.01\\
19.3	0.01\\
19.31	0.01\\
19.32	0.01\\
19.33	0.01\\
19.34	0.01\\
19.35	0.01\\
19.36	0.01\\
19.37	0.01\\
19.38	0.01\\
19.39	0.01\\
19.4	0.01\\
19.41	0.01\\
19.42	0.01\\
19.43	0.01\\
19.44	0.01\\
19.45	0.01\\
19.46	0.01\\
19.47	0.01\\
19.48	0.01\\
19.49	0.01\\
19.5	0.01\\
19.51	0.01\\
19.52	0.01\\
19.53	0.01\\
19.54	0.01\\
19.55	0.01\\
19.56	0.01\\
19.57	0.01\\
19.58	0.01\\
19.59	0.01\\
19.6	0.01\\
19.61	0.01\\
19.62	0.01\\
19.63	0.01\\
19.64	0.01\\
19.65	0.01\\
19.66	0.01\\
19.67	0.01\\
19.68	0.01\\
19.69	0.01\\
19.7	0.01\\
19.71	0.01\\
19.72	0.01\\
19.73	0.01\\
19.74	0.01\\
19.75	0.01\\
19.76	0.01\\
19.77	0.01\\
19.78	0.01\\
19.79	0.01\\
19.8	0.01\\
19.81	0.01\\
19.82	0.01\\
19.83	0.01\\
19.84	0.01\\
19.85	0.01\\
19.86	0.01\\
19.87	0.01\\
19.88	0.01\\
19.89	0.01\\
19.9	0.01\\
19.91	0.01\\
19.92	0.01\\
19.93	0.01\\
19.94	0.01\\
19.95	0.01\\
19.96	0.01\\
19.97	0.01\\
19.98	0.01\\
19.99	0.01\\
20	0.01\\
20.01	0.01\\
20.02	0.01\\
20.03	0.01\\
20.04	0.01\\
20.05	0.01\\
20.06	0.01\\
20.07	0.01\\
20.08	0.01\\
20.09	0.01\\
20.1	0.01\\
20.11	0.01\\
20.12	0.01\\
20.13	0.01\\
20.14	0.01\\
20.15	0.01\\
20.16	0.01\\
20.17	0.01\\
20.18	0.01\\
20.19	0.01\\
20.2	0.01\\
20.21	0.01\\
20.22	0.01\\
20.23	0.01\\
20.24	0.01\\
20.25	0.01\\
20.26	0.01\\
20.27	0.01\\
20.28	0.01\\
20.29	0.01\\
20.3	0.01\\
20.31	0.01\\
20.32	0.01\\
20.33	0.01\\
20.34	0.01\\
20.35	0.01\\
20.36	0.01\\
20.37	0.01\\
20.38	0.01\\
20.39	0.01\\
20.4	0.01\\
20.41	0.01\\
20.42	0.01\\
20.43	0.01\\
20.44	0.01\\
20.45	0.01\\
20.46	0.01\\
20.47	0.01\\
20.48	0.01\\
20.49	0.01\\
20.5	0.01\\
20.51	0.01\\
20.52	0.01\\
20.53	0.01\\
20.54	0.01\\
20.55	0.01\\
20.56	0.01\\
20.57	0.01\\
20.58	0.01\\
20.59	0.01\\
20.6	0.01\\
20.61	0.01\\
20.62	0.01\\
20.63	0.01\\
20.64	0.01\\
20.65	0.01\\
20.66	0.01\\
20.67	0.01\\
20.68	0.01\\
20.69	0.01\\
20.7	0.01\\
20.71	0.01\\
20.72	0.01\\
20.73	0.01\\
20.74	0.01\\
20.75	0.01\\
20.76	0.01\\
20.77	0.01\\
20.78	0.01\\
20.79	0.01\\
20.8	0.01\\
20.81	0.01\\
20.82	0.01\\
20.83	0.01\\
20.84	0.01\\
20.85	0.01\\
20.86	0.01\\
20.87	0.01\\
20.88	0.01\\
20.89	0.01\\
20.9	0.01\\
20.91	0.01\\
20.92	0.01\\
20.93	0.01\\
20.94	0.01\\
20.95	0.01\\
20.96	0.01\\
20.97	0.01\\
20.98	0.01\\
20.99	0.01\\
21	0.01\\
21.01	0.01\\
21.02	0.01\\
21.03	0.01\\
21.04	0.01\\
21.05	0.01\\
21.06	0.01\\
21.07	0.01\\
21.08	0.01\\
21.09	0.01\\
21.1	0.01\\
21.11	0.01\\
21.12	0.01\\
21.13	0.01\\
21.14	0.01\\
21.15	0.01\\
21.16	0.01\\
21.17	0.01\\
21.18	0.01\\
21.19	0.01\\
21.2	0.01\\
21.21	0.01\\
21.22	0.01\\
21.23	0.01\\
21.24	0.01\\
21.25	0.01\\
21.26	0.01\\
21.27	0.01\\
21.28	0.01\\
21.29	0.01\\
21.3	0.01\\
21.31	0.01\\
21.32	0.01\\
21.33	0.01\\
21.34	0.01\\
21.35	0.01\\
21.36	0.01\\
21.37	0.01\\
21.38	0.01\\
21.39	0.01\\
21.4	0.01\\
21.41	0.01\\
21.42	0.01\\
21.43	0.01\\
21.44	0.01\\
21.45	0.01\\
21.46	0.01\\
21.47	0.01\\
21.48	0.01\\
21.49	0.01\\
21.5	0.01\\
21.51	0.01\\
21.52	0.01\\
21.53	0.01\\
21.54	0.01\\
21.55	0.01\\
21.56	0.01\\
21.57	0.01\\
21.58	0.01\\
21.59	0.01\\
21.6	0.01\\
21.61	0.01\\
21.62	0.01\\
21.63	0.01\\
21.64	0.01\\
21.65	0.01\\
21.66	0.01\\
21.67	0.01\\
21.68	0.01\\
21.69	0.01\\
21.7	0.01\\
21.71	0.01\\
21.72	0.01\\
21.73	0.01\\
21.74	0.01\\
21.75	0.01\\
21.76	0.01\\
21.77	0.01\\
21.78	0.01\\
21.79	0.01\\
21.8	0.01\\
21.81	0.01\\
21.82	0.01\\
21.83	0.01\\
21.84	0.01\\
21.85	0.01\\
21.86	0.01\\
21.87	0.01\\
21.88	0.01\\
21.89	0.01\\
21.9	0.01\\
21.91	0.01\\
21.92	0.01\\
21.93	0.01\\
21.94	0.01\\
21.95	0.01\\
21.96	0.01\\
21.97	0.01\\
21.98	0.01\\
21.99	0.01\\
22	0.01\\
22.01	0.01\\
22.02	0.01\\
22.03	0.01\\
22.04	0.01\\
22.05	0.01\\
22.06	0.01\\
22.07	0.01\\
22.08	0.01\\
22.09	0.01\\
22.1	0.01\\
22.11	0.01\\
22.12	0.01\\
22.13	0.01\\
22.14	0.01\\
22.15	0.01\\
22.16	0.01\\
22.17	0.01\\
22.18	0.01\\
22.19	0.01\\
22.2	0.01\\
22.21	0.01\\
22.22	0.01\\
22.23	0.01\\
22.24	0.01\\
22.25	0.01\\
22.26	0.01\\
22.27	0.01\\
22.28	0.01\\
22.29	0.01\\
22.3	0.01\\
22.31	0.01\\
22.32	0.01\\
22.33	0.01\\
22.34	0.01\\
22.35	0.01\\
22.36	0.01\\
22.37	0.01\\
22.38	0.01\\
22.39	0.01\\
22.4	0.01\\
22.41	0.01\\
22.42	0.01\\
22.43	0.01\\
22.44	0.01\\
22.45	0.01\\
22.46	0.01\\
22.47	0.01\\
22.48	0.01\\
22.49	0.01\\
22.5	0.01\\
22.51	0.01\\
22.52	0.01\\
22.53	0.01\\
22.54	0.01\\
22.55	0.01\\
22.56	0.01\\
22.57	0.01\\
22.58	0.01\\
22.59	0.01\\
22.6	0.01\\
22.61	0.01\\
22.62	0.01\\
22.63	0.01\\
22.64	0.01\\
22.65	0.01\\
22.66	0.01\\
22.67	0.01\\
22.68	0.01\\
22.69	0.01\\
22.7	0.01\\
22.71	0.01\\
22.72	0.01\\
22.73	0.01\\
22.74	0.01\\
22.75	0.01\\
22.76	0.01\\
22.77	0.01\\
22.78	0.01\\
22.79	0.01\\
22.8	0.01\\
22.81	0.01\\
22.82	0.01\\
22.83	0.01\\
22.84	0.01\\
22.85	0.01\\
22.86	0.01\\
22.87	0.01\\
22.88	0.01\\
22.89	0.01\\
22.9	0.01\\
22.91	0.01\\
22.92	0.01\\
22.93	0.01\\
22.94	0.01\\
22.95	0.01\\
22.96	0.01\\
22.97	0.01\\
22.98	0.01\\
22.99	0.01\\
23	0.01\\
23.01	0.01\\
23.02	0.01\\
23.03	0.01\\
23.04	0.01\\
23.05	0.01\\
23.06	0.01\\
23.07	0.01\\
23.08	0.01\\
23.09	0.01\\
23.1	0.01\\
23.11	0.01\\
23.12	0.01\\
23.13	0.01\\
23.14	0.01\\
23.15	0.01\\
23.16	0.01\\
23.17	0.01\\
23.18	0.01\\
23.19	0.01\\
23.2	0.01\\
23.21	0.01\\
23.22	0.01\\
23.23	0.01\\
23.24	0.01\\
23.25	0.01\\
23.26	0.01\\
23.27	0.01\\
23.28	0.01\\
23.29	0.01\\
23.3	0.01\\
23.31	0.01\\
23.32	0.01\\
23.33	0.01\\
23.34	0.01\\
23.35	0.01\\
23.36	0.01\\
23.37	0.01\\
23.38	0.01\\
23.39	0.01\\
23.4	0.01\\
23.41	0.01\\
23.42	0.01\\
23.43	0.01\\
23.44	0.01\\
23.45	0.01\\
23.46	0.01\\
23.47	0.01\\
23.48	0.01\\
23.49	0.01\\
23.5	0.01\\
23.51	0.01\\
23.52	0.01\\
23.53	0.01\\
23.54	0.01\\
23.55	0.01\\
23.56	0.01\\
23.57	0.01\\
23.58	0.01\\
23.59	0.01\\
23.6	0.01\\
23.61	0.01\\
23.62	0.01\\
23.63	0.01\\
23.64	0.01\\
23.65	0.01\\
23.66	0.01\\
23.67	0.01\\
23.68	0.01\\
23.69	0.01\\
23.7	0.01\\
23.71	0.01\\
23.72	0.01\\
23.73	0.01\\
23.74	0.01\\
23.75	0.01\\
23.76	0.01\\
23.77	0.01\\
23.78	0.01\\
23.79	0.01\\
23.8	0.01\\
23.81	0.01\\
23.82	0.01\\
23.83	0.01\\
23.84	0.01\\
23.85	0.01\\
23.86	0.01\\
23.87	0.01\\
23.88	0.01\\
23.89	0.01\\
23.9	0.01\\
23.91	0.01\\
23.92	0.01\\
23.93	0.01\\
23.94	0.01\\
23.95	0.01\\
23.96	0.01\\
23.97	0.01\\
23.98	0.01\\
23.99	0.01\\
24	0.01\\
24.01	0.01\\
24.02	0.01\\
24.03	0.01\\
24.04	0.01\\
24.05	0.01\\
24.06	0.01\\
24.07	0.01\\
24.08	0.01\\
24.09	0.01\\
24.1	0.01\\
24.11	0.01\\
24.12	0.01\\
24.13	0.01\\
24.14	0.01\\
24.15	0.01\\
24.16	0.01\\
24.17	0.01\\
24.18	0.01\\
24.19	0.01\\
24.2	0.01\\
24.21	0.01\\
24.22	0.01\\
24.23	0.01\\
24.24	0.01\\
24.25	0.01\\
24.26	0.01\\
24.27	0.01\\
24.28	0.01\\
24.29	0.01\\
24.3	0.01\\
24.31	0.01\\
24.32	0.01\\
24.33	0.01\\
24.34	0.01\\
24.35	0.01\\
24.36	0.01\\
24.37	0.01\\
24.38	0.01\\
24.39	0.01\\
24.4	0.01\\
24.41	0.01\\
24.42	0.01\\
24.43	0.01\\
24.44	0.01\\
24.45	0.01\\
24.46	0.01\\
24.47	0.01\\
24.48	0.01\\
24.49	0.01\\
24.5	0.01\\
24.51	0.01\\
24.52	0.01\\
24.53	0.01\\
24.54	0.01\\
24.55	0.01\\
24.56	0.01\\
24.57	0.01\\
24.58	0.01\\
24.59	0.01\\
24.6	0.01\\
24.61	0.01\\
24.62	0.01\\
24.63	0.01\\
24.64	0.01\\
24.65	0.01\\
24.66	0.01\\
24.67	0.01\\
24.68	0.01\\
24.69	0.01\\
24.7	0.01\\
24.71	0.01\\
24.72	0.01\\
24.73	0.01\\
24.74	0.01\\
24.75	0.01\\
24.76	0.01\\
24.77	0.01\\
24.78	0.01\\
24.79	0.01\\
24.8	0.01\\
24.81	0.01\\
24.82	0.01\\
24.83	0.01\\
24.84	0.01\\
24.85	0.01\\
24.86	0.01\\
24.87	0.01\\
24.88	0.01\\
24.89	0.01\\
24.9	0.01\\
24.91	0.01\\
24.92	0.01\\
24.93	0.01\\
24.94	0.01\\
24.95	0.01\\
24.96	0.01\\
24.97	0.01\\
24.98	0.01\\
24.99	0.01\\
25	0.01\\
25.01	0.01\\
25.02	0.01\\
25.03	0.01\\
25.04	0.01\\
25.05	0.01\\
25.06	0.01\\
25.07	0.01\\
25.08	0.01\\
25.09	0.01\\
25.1	0.01\\
25.11	0.01\\
25.12	0.01\\
25.13	0.01\\
25.14	0.01\\
25.15	0.01\\
25.16	0.01\\
25.17	0.01\\
25.18	0.01\\
25.19	0.01\\
25.2	0.01\\
25.21	0.01\\
25.22	0.01\\
25.23	0.01\\
25.24	0.01\\
25.25	0.01\\
25.26	0.01\\
25.27	0.01\\
25.28	0.01\\
25.29	0.01\\
25.3	0.01\\
25.31	0.01\\
25.32	0.01\\
25.33	0.01\\
25.34	0.01\\
25.35	0.01\\
25.36	0.01\\
25.37	0.01\\
25.38	0.01\\
25.39	0.01\\
25.4	0.01\\
25.41	0.01\\
25.42	0.01\\
25.43	0.01\\
25.44	0.01\\
25.45	0.01\\
25.46	0.01\\
25.47	0.01\\
25.48	0.01\\
25.49	0.01\\
25.5	0.01\\
25.51	0.01\\
25.52	0.01\\
25.53	0.01\\
25.54	0.01\\
25.55	0.01\\
25.56	0.01\\
25.57	0.01\\
25.58	0.01\\
25.59	0.01\\
25.6	0.01\\
25.61	0.01\\
25.62	0.01\\
25.63	0.01\\
25.64	0.01\\
25.65	0.01\\
25.66	0.01\\
25.67	0.01\\
25.68	0.01\\
25.69	0.01\\
25.7	0.01\\
25.71	0.01\\
25.72	0.01\\
25.73	0.01\\
25.74	0.01\\
25.75	0.01\\
25.76	0.01\\
25.77	0.01\\
25.78	0.01\\
25.79	0.01\\
25.8	0.01\\
25.81	0.01\\
25.82	0.01\\
25.83	0.01\\
25.84	0.01\\
25.85	0.01\\
25.86	0.01\\
25.87	0.01\\
25.88	0.01\\
25.89	0.01\\
25.9	0.01\\
25.91	0.01\\
25.92	0.01\\
25.93	0.01\\
25.94	0.01\\
25.95	0.01\\
25.96	0.01\\
25.97	0.01\\
25.98	0.01\\
25.99	0.01\\
26	0.01\\
26.01	0.01\\
26.02	0.01\\
26.03	0.01\\
26.04	0.01\\
26.05	0.01\\
26.06	0.01\\
26.07	0.01\\
26.08	0.01\\
26.09	0.01\\
26.1	0.01\\
26.11	0.01\\
26.12	0.01\\
26.13	0.01\\
26.14	0.01\\
26.15	0.01\\
26.16	0.01\\
26.17	0.01\\
26.18	0.01\\
26.19	0.01\\
26.2	0.01\\
26.21	0.01\\
26.22	0.01\\
26.23	0.01\\
26.24	0.01\\
26.25	0.01\\
26.26	0.01\\
26.27	0.01\\
26.28	0.01\\
26.29	0.01\\
26.3	0.01\\
26.31	0.01\\
26.32	0.01\\
26.33	0.01\\
26.34	0.01\\
26.35	0.01\\
26.36	0.01\\
26.37	0.01\\
26.38	0.01\\
26.39	0.01\\
26.4	0.01\\
26.41	0.01\\
26.42	0.01\\
26.43	0.01\\
26.44	0.01\\
26.45	0.01\\
26.46	0.01\\
26.47	0.01\\
26.48	0.01\\
26.49	0.01\\
26.5	0.01\\
26.51	0.01\\
26.52	0.01\\
26.53	0.01\\
26.54	0.01\\
26.55	0.01\\
26.56	0.01\\
26.57	0.01\\
26.58	0.01\\
26.59	0.01\\
26.6	0.01\\
26.61	0.01\\
26.62	0.01\\
26.63	0.01\\
26.64	0.01\\
26.65	0.01\\
26.66	0.01\\
26.67	0.01\\
26.68	0.01\\
26.69	0.01\\
26.7	0.01\\
26.71	0.01\\
26.72	0.01\\
26.73	0.01\\
26.74	0.01\\
26.75	0.01\\
26.76	0.01\\
26.77	0.01\\
26.78	0.01\\
26.79	0.01\\
26.8	0.01\\
26.81	0.01\\
26.82	0.01\\
26.83	0.01\\
26.84	0.01\\
26.85	0.01\\
26.86	0.01\\
26.87	0.01\\
26.88	0.01\\
26.89	0.01\\
26.9	0.01\\
26.91	0.01\\
26.92	0.01\\
26.93	0.01\\
26.94	0.01\\
26.95	0.01\\
26.96	0.01\\
26.97	0.01\\
26.98	0.01\\
26.99	0.01\\
27	0.01\\
27.01	0.01\\
27.02	0.01\\
27.03	0.01\\
27.04	0.01\\
27.05	0.01\\
27.06	0.01\\
27.07	0.01\\
27.08	0.01\\
27.09	0.01\\
27.1	0.01\\
27.11	0.01\\
27.12	0.01\\
27.13	0.01\\
27.14	0.01\\
27.15	0.01\\
27.16	0.01\\
27.17	0.01\\
27.18	0.01\\
27.19	0.01\\
27.2	0.01\\
27.21	0.01\\
27.22	0.01\\
27.23	0.01\\
27.24	0.01\\
27.25	0.01\\
27.26	0.01\\
27.27	0.01\\
27.28	0.01\\
27.29	0.01\\
27.3	0.01\\
27.31	0.01\\
27.32	0.01\\
27.33	0.01\\
27.34	0.01\\
27.35	0.01\\
27.36	0.01\\
27.37	0.01\\
27.38	0.01\\
27.39	0.01\\
27.4	0.01\\
27.41	0.01\\
27.42	0.01\\
27.43	0.01\\
27.44	0.01\\
27.45	0.01\\
27.46	0.01\\
27.47	0.01\\
27.48	0.01\\
27.49	0.01\\
27.5	0.01\\
27.51	0.01\\
27.52	0.01\\
27.53	0.01\\
27.54	0.01\\
27.55	0.01\\
27.56	0.01\\
27.57	0.01\\
27.58	0.01\\
27.59	0.01\\
27.6	0.01\\
27.61	0.01\\
27.62	0.01\\
27.63	0.01\\
27.64	0.01\\
27.65	0.01\\
27.66	0.01\\
27.67	0.01\\
27.68	0.01\\
27.69	0.01\\
27.7	0.01\\
27.71	0.01\\
27.72	0.01\\
27.73	0.01\\
27.74	0.01\\
27.75	0.01\\
27.76	0.01\\
27.77	0.01\\
27.78	0.01\\
27.79	0.01\\
27.8	0.01\\
27.81	0.01\\
27.82	0.01\\
27.83	0.01\\
27.84	0.01\\
27.85	0.01\\
27.86	0.01\\
27.87	0.01\\
27.88	0.01\\
27.89	0.01\\
27.9	0.01\\
27.91	0.01\\
27.92	0.01\\
27.93	0.01\\
27.94	0.01\\
27.95	0.01\\
27.96	0.01\\
27.97	0.01\\
27.98	0.01\\
27.99	0.01\\
28	0.01\\
28.01	0.01\\
28.02	0.01\\
28.03	0.01\\
28.04	0.01\\
28.05	0.01\\
28.06	0.01\\
28.07	0.01\\
28.08	0.01\\
28.09	0.01\\
28.1	0.01\\
28.11	0.01\\
28.12	0.01\\
28.13	0.01\\
28.14	0.01\\
28.15	0.01\\
28.16	0.01\\
28.17	0.01\\
28.18	0.01\\
28.19	0.01\\
28.2	0.01\\
28.21	0.01\\
28.22	0.01\\
28.23	0.01\\
28.24	0.01\\
28.25	0.01\\
28.26	0.01\\
28.27	0.01\\
28.28	0.01\\
28.29	0.01\\
28.3	0.01\\
28.31	0.01\\
28.32	0.01\\
28.33	0.01\\
28.34	0.01\\
28.35	0.01\\
28.36	0.01\\
28.37	0.01\\
28.38	0.01\\
28.39	0.01\\
28.4	0.01\\
28.41	0.01\\
28.42	0.01\\
28.43	0.01\\
28.44	0.01\\
28.45	0.01\\
28.46	0.01\\
28.47	0.01\\
28.48	0.01\\
28.49	0.01\\
28.5	0.01\\
28.51	0.01\\
28.52	0.01\\
28.53	0.01\\
28.54	0.01\\
28.55	0.01\\
28.56	0.01\\
28.57	0.01\\
28.58	0.01\\
28.59	0.01\\
28.6	0.01\\
28.61	0.01\\
28.62	0.01\\
28.63	0.01\\
28.64	0.01\\
28.65	0.01\\
28.66	0.01\\
28.67	0.01\\
28.68	0.01\\
28.69	0.01\\
28.7	0.01\\
28.71	0.01\\
28.72	0.01\\
28.73	0.01\\
28.74	0.01\\
28.75	0.01\\
28.76	0.01\\
28.77	0.01\\
28.78	0.01\\
28.79	0.01\\
28.8	0.01\\
28.81	0.01\\
28.82	0.01\\
28.83	0.01\\
28.84	0.01\\
28.85	0.01\\
28.86	0.01\\
28.87	0.01\\
28.88	0.01\\
28.89	0.01\\
28.9	0.01\\
28.91	0.01\\
28.92	0.01\\
28.93	0.01\\
28.94	0.01\\
28.95	0.01\\
28.96	0.01\\
28.97	0.01\\
28.98	0.01\\
28.99	0.01\\
29	0.01\\
29.01	0.01\\
29.02	0.01\\
29.03	0.01\\
29.04	0.01\\
29.05	0.01\\
29.06	0.01\\
29.07	0.01\\
29.08	0.01\\
29.09	0.01\\
29.1	0.01\\
29.11	0.01\\
29.12	0.01\\
29.13	0.01\\
29.14	0.01\\
29.15	0.01\\
29.16	0.01\\
29.17	0.01\\
29.18	0.01\\
29.19	0.01\\
29.2	0.01\\
29.21	0.01\\
29.22	0.01\\
29.23	0.01\\
29.24	0.01\\
29.25	0.01\\
29.26	0.01\\
29.27	0.01\\
29.28	0.01\\
29.29	0.01\\
29.3	0.01\\
29.31	0.01\\
29.32	0.01\\
29.33	0.01\\
29.34	0.01\\
29.35	0.01\\
29.36	0.01\\
29.37	0.01\\
29.38	0.01\\
29.39	0.01\\
29.4	0.01\\
29.41	0.01\\
29.42	0.01\\
29.43	0.01\\
29.44	0.01\\
29.45	0.01\\
29.46	0.01\\
29.47	0.01\\
29.48	0.01\\
29.49	0.01\\
29.5	0.01\\
29.51	0.01\\
29.52	0.01\\
29.53	0.01\\
29.54	0.01\\
29.55	0.01\\
29.56	0.01\\
29.57	0.01\\
29.58	0.01\\
29.59	0.01\\
29.6	0.01\\
29.61	0.01\\
29.62	0.01\\
29.63	0.01\\
29.64	0.01\\
29.65	0.01\\
29.66	0.01\\
29.67	0.01\\
29.68	0.01\\
29.69	0.01\\
29.7	0.01\\
29.71	0.01\\
29.72	0.01\\
29.73	0.01\\
29.74	0.01\\
29.75	0.01\\
29.76	0.01\\
29.77	0.01\\
29.78	0.01\\
29.79	0.01\\
29.8	0.01\\
29.81	0.01\\
29.82	0.01\\
29.83	0.01\\
29.84	0.01\\
29.85	0.01\\
29.86	0.01\\
29.87	0.01\\
29.88	0.01\\
29.89	0.01\\
29.9	0.01\\
29.91	0.01\\
29.92	0.01\\
29.93	0.01\\
29.94	0.01\\
29.95	0.01\\
29.96	0.01\\
29.97	0.01\\
29.98	0.01\\
29.99	0.01\\
30	0.01\\
30.01	0.01\\
30.02	0.01\\
30.03	0.01\\
30.04	0.01\\
30.05	0.01\\
30.06	0.01\\
30.07	0.01\\
30.08	0.01\\
30.09	0.01\\
30.1	0.01\\
30.11	0.01\\
30.12	0.01\\
30.13	0.01\\
30.14	0.01\\
30.15	0.01\\
30.16	0.01\\
30.17	0.01\\
30.18	0.01\\
30.19	0.01\\
30.2	0.01\\
30.21	0.01\\
30.22	0.01\\
30.23	0.01\\
30.24	0.01\\
30.25	0.01\\
30.26	0.01\\
30.27	0.01\\
30.28	0.01\\
30.29	0.01\\
30.3	0.01\\
30.31	0.01\\
30.32	0.01\\
30.33	0.01\\
30.34	0.01\\
30.35	0.01\\
30.36	0.01\\
30.37	0.01\\
30.38	0.01\\
30.39	0.01\\
30.4	0.01\\
30.41	0.01\\
30.42	0.01\\
30.43	0.01\\
30.44	0.01\\
30.45	0.01\\
30.46	0.01\\
30.47	0.01\\
30.48	0.01\\
30.49	0.01\\
30.5	0.01\\
30.51	0.01\\
30.52	0.01\\
30.53	0.01\\
30.54	0.01\\
30.55	0.01\\
30.56	0.01\\
30.57	0.01\\
30.58	0.01\\
30.59	0.01\\
30.6	0.01\\
30.61	0.01\\
30.62	0.01\\
30.63	0.01\\
30.64	0.01\\
30.65	0.01\\
30.66	0.01\\
30.67	0.01\\
30.68	0.01\\
30.69	0.01\\
30.7	0.01\\
30.71	0.01\\
30.72	0.01\\
30.73	0.01\\
30.74	0.01\\
30.75	0.01\\
30.76	0.01\\
30.77	0.01\\
30.78	0.01\\
30.79	0.01\\
30.8	0.01\\
30.81	0.01\\
30.82	0.01\\
30.83	0.01\\
30.84	0.01\\
30.85	0.01\\
30.86	0.01\\
30.87	0.01\\
30.88	0.01\\
30.89	0.01\\
30.9	0.01\\
30.91	0.01\\
30.92	0.01\\
30.93	0.01\\
30.94	0.01\\
30.95	0.01\\
30.96	0.01\\
30.97	0.01\\
30.98	0.01\\
30.99	0.01\\
31	0.01\\
31.01	0.01\\
31.02	0.01\\
31.03	0.01\\
31.04	0.01\\
31.05	0.01\\
31.06	0.01\\
31.07	0.01\\
31.08	0.01\\
31.09	0.01\\
31.1	0.01\\
31.11	0.01\\
31.12	0.01\\
31.13	0.01\\
31.14	0.01\\
31.15	0.01\\
31.16	0.01\\
31.17	0.01\\
31.18	0.01\\
31.19	0.01\\
31.2	0.01\\
31.21	0.01\\
31.22	0.01\\
31.23	0.01\\
31.24	0.01\\
31.25	0.01\\
31.26	0.01\\
31.27	0.01\\
31.28	0.01\\
31.29	0.01\\
31.3	0.01\\
31.31	0.01\\
31.32	0.01\\
31.33	0.01\\
31.34	0.01\\
31.35	0.01\\
31.36	0.01\\
31.37	0.01\\
31.38	0.01\\
31.39	0.01\\
31.4	0.01\\
31.41	0.01\\
31.42	0.01\\
31.43	0.01\\
31.44	0.01\\
31.45	0.01\\
31.46	0.01\\
31.47	0.01\\
31.48	0.01\\
31.49	0.01\\
31.5	0.01\\
31.51	0.01\\
31.52	0.01\\
31.53	0.01\\
31.54	0.01\\
31.55	0.01\\
31.56	0.01\\
31.57	0.01\\
31.58	0.01\\
31.59	0.01\\
31.6	0.01\\
31.61	0.01\\
31.62	0.01\\
31.63	0.01\\
31.64	0.01\\
31.65	0.01\\
31.66	0.01\\
31.67	0.01\\
31.68	0.01\\
31.69	0.01\\
31.7	0.01\\
31.71	0.01\\
31.72	0.01\\
31.73	0.01\\
31.74	0.01\\
31.75	0.01\\
31.76	0.01\\
31.77	0.01\\
31.78	0.01\\
31.79	0.01\\
31.8	0.01\\
31.81	0.01\\
31.82	0.01\\
31.83	0.01\\
31.84	0.01\\
31.85	0.01\\
31.86	0.01\\
31.87	0.01\\
31.88	0.01\\
31.89	0.01\\
31.9	0.01\\
31.91	0.01\\
31.92	0.01\\
31.93	0.01\\
31.94	0.01\\
31.95	0.01\\
31.96	0.01\\
31.97	0.01\\
31.98	0.01\\
31.99	0.01\\
32	0.01\\
32.01	0.01\\
32.02	0.01\\
32.03	0.01\\
32.04	0.01\\
32.05	0.01\\
32.06	0.01\\
32.07	0.01\\
32.08	0.01\\
32.09	0.01\\
32.1	0.01\\
32.11	0.01\\
32.12	0.01\\
32.13	0.01\\
32.14	0.01\\
32.15	0.01\\
32.16	0.01\\
32.17	0.01\\
32.18	0.01\\
32.19	0.01\\
32.2	0.01\\
32.21	0.01\\
32.22	0.01\\
32.23	0.01\\
32.24	0.01\\
32.25	0.01\\
32.26	0.01\\
32.27	0.01\\
32.28	0.01\\
32.29	0.01\\
32.3	0.01\\
32.31	0.01\\
32.32	0.01\\
32.33	0.01\\
32.34	0.01\\
32.35	0.01\\
32.36	0.01\\
32.37	0.01\\
32.38	0.01\\
32.39	0.01\\
32.4	0.01\\
32.41	0.01\\
32.42	0.01\\
32.43	0.01\\
32.44	0.01\\
32.45	0.01\\
32.46	0.01\\
32.47	0.01\\
32.48	0.01\\
32.49	0.01\\
32.5	0.01\\
32.51	0.01\\
32.52	0.01\\
32.53	0.01\\
32.54	0.01\\
32.55	0.01\\
32.56	0.01\\
32.57	0.01\\
32.58	0.01\\
32.59	0.01\\
32.6	0.01\\
32.61	0.01\\
32.62	0.01\\
32.63	0.01\\
32.64	0.01\\
32.65	0.01\\
32.66	0.01\\
32.67	0.01\\
32.68	0.01\\
32.69	0.01\\
32.7	0.01\\
32.71	0.01\\
32.72	0.01\\
32.73	0.01\\
32.74	0.01\\
32.75	0.01\\
32.76	0.01\\
32.77	0.01\\
32.78	0.01\\
32.79	0.01\\
32.8	0.01\\
32.81	0.01\\
32.82	0.01\\
32.83	0.01\\
32.84	0.01\\
32.85	0.01\\
32.86	0.01\\
32.87	0.01\\
32.88	0.01\\
32.89	0.01\\
32.9	0.01\\
32.91	0.01\\
32.92	0.01\\
32.93	0.01\\
32.94	0.01\\
32.95	0.01\\
32.96	0.01\\
32.97	0.01\\
32.98	0.01\\
32.99	0.01\\
33	0.01\\
33.01	0.01\\
33.02	0.01\\
33.03	0.01\\
33.04	0.01\\
33.05	0.01\\
33.06	0.01\\
33.07	0.01\\
33.08	0.01\\
33.09	0.01\\
33.1	0.01\\
33.11	0.01\\
33.12	0.01\\
33.13	0.01\\
33.14	0.01\\
33.15	0.01\\
33.16	0.01\\
33.17	0.01\\
33.18	0.01\\
33.19	0.01\\
33.2	0.01\\
33.21	0.01\\
33.22	0.01\\
33.23	0.01\\
33.24	0.01\\
33.25	0.01\\
33.26	0.01\\
33.27	0.01\\
33.28	0.01\\
33.29	0.01\\
33.3	0.01\\
33.31	0.01\\
33.32	0.01\\
33.33	0.01\\
33.34	0.01\\
33.35	0.01\\
33.36	0.01\\
33.37	0.01\\
33.38	0.01\\
33.39	0.01\\
33.4	0.01\\
33.41	0.01\\
33.42	0.01\\
33.43	0.01\\
33.44	0.01\\
33.45	0.01\\
33.46	0.01\\
33.47	0.01\\
33.48	0.01\\
33.49	0.01\\
33.5	0.01\\
33.51	0.01\\
33.52	0.01\\
33.53	0.01\\
33.54	0.01\\
33.55	0.01\\
33.56	0.01\\
33.57	0.01\\
33.58	0.01\\
33.59	0.01\\
33.6	0.01\\
33.61	0.01\\
33.62	0.01\\
33.63	0.01\\
33.64	0.01\\
33.65	0.01\\
33.66	0.01\\
33.67	0.01\\
33.68	0.01\\
33.69	0.01\\
33.7	0.01\\
33.71	0.01\\
33.72	0.01\\
33.73	0.01\\
33.74	0.01\\
33.75	0.01\\
33.76	0.01\\
33.77	0.01\\
33.78	0.01\\
33.79	0.01\\
33.8	0.01\\
33.81	0.01\\
33.82	0.01\\
33.83	0.01\\
33.84	0.01\\
33.85	0.01\\
33.86	0.01\\
33.87	0.01\\
33.88	0.01\\
33.89	0.01\\
33.9	0.01\\
33.91	0.01\\
33.92	0.01\\
33.93	0.01\\
33.94	0.01\\
33.95	0.01\\
33.96	0.01\\
33.97	0.01\\
33.98	0.01\\
33.99	0.01\\
34	0.01\\
34.01	0.01\\
34.02	0.01\\
34.03	0.01\\
34.04	0.01\\
34.05	0.01\\
34.06	0.01\\
34.07	0.01\\
34.08	0.01\\
34.09	0.01\\
34.1	0.01\\
34.11	0.01\\
34.12	0.01\\
34.13	0.01\\
34.14	0.01\\
34.15	0.01\\
34.16	0.01\\
34.17	0.01\\
34.18	0.01\\
34.19	0.01\\
34.2	0.01\\
34.21	0.01\\
34.22	0.01\\
34.23	0.01\\
34.24	0.01\\
34.25	0.01\\
34.26	0.01\\
34.27	0.01\\
34.28	0.01\\
34.29	0.01\\
34.3	0.01\\
34.31	0.01\\
34.32	0.01\\
34.33	0.01\\
34.34	0.01\\
34.35	0.01\\
34.36	0.01\\
34.37	0.01\\
34.38	0.01\\
34.39	0.01\\
34.4	0.01\\
34.41	0.01\\
34.42	0.01\\
34.43	0.01\\
34.44	0.01\\
34.45	0.01\\
34.46	0.01\\
34.47	0.01\\
34.48	0.01\\
34.49	0.01\\
34.5	0.01\\
34.51	0.01\\
34.52	0.01\\
34.53	0.01\\
34.54	0.01\\
34.55	0.01\\
34.56	0.01\\
34.57	0.01\\
34.58	0.01\\
34.59	0.01\\
34.6	0.01\\
34.61	0.01\\
34.62	0.01\\
34.63	0.01\\
34.64	0.01\\
34.65	0.01\\
34.66	0.01\\
34.67	0.01\\
34.68	0.01\\
34.69	0.01\\
34.7	0.01\\
34.71	0.01\\
34.72	0.01\\
34.73	0.01\\
34.74	0.01\\
34.75	0.01\\
34.76	0.01\\
34.77	0.01\\
34.78	0.01\\
34.79	0.01\\
34.8	0.01\\
34.81	0.01\\
34.82	0.01\\
34.83	0.01\\
34.84	0.01\\
34.85	0.01\\
34.86	0.01\\
34.87	0.01\\
34.88	0.01\\
34.89	0.01\\
34.9	0.01\\
34.91	0.01\\
34.92	0.01\\
34.93	0.01\\
34.94	0.01\\
34.95	0.01\\
34.96	0.01\\
34.97	0.01\\
34.98	0.01\\
34.99	0.01\\
35	0.01\\
35.01	0.01\\
35.02	0.01\\
35.03	0.01\\
35.04	0.01\\
35.05	0.01\\
35.06	0.01\\
35.07	0.01\\
35.08	0.01\\
35.09	0.01\\
35.1	0.01\\
35.11	0.01\\
35.12	0.01\\
35.13	0.01\\
35.14	0.01\\
35.15	0.01\\
35.16	0.01\\
35.17	0.01\\
35.18	0.01\\
35.19	0.01\\
35.2	0.01\\
35.21	0.01\\
35.22	0.01\\
35.23	0.01\\
35.24	0.01\\
35.25	0.01\\
35.26	0.01\\
35.27	0.01\\
35.28	0.01\\
35.29	0.01\\
35.3	0.01\\
35.31	0.01\\
35.32	0.01\\
35.33	0.01\\
35.34	0.01\\
35.35	0.01\\
35.36	0.01\\
35.37	0.01\\
35.38	0.01\\
35.39	0.01\\
35.4	0.01\\
35.41	0.01\\
35.42	0.01\\
35.43	0.01\\
35.44	0.01\\
35.45	0.01\\
35.46	0.01\\
35.47	0.01\\
35.48	0.01\\
35.49	0.01\\
35.5	0.01\\
35.51	0.01\\
35.52	0.01\\
35.53	0.01\\
35.54	0.01\\
35.55	0.01\\
35.56	0.01\\
35.57	0.01\\
35.58	0.01\\
35.59	0.01\\
35.6	0.01\\
35.61	0.01\\
35.62	0.01\\
35.63	0.01\\
35.64	0.01\\
35.65	0.01\\
35.66	0.01\\
35.67	0.01\\
35.68	0.01\\
35.69	0.01\\
35.7	0.01\\
35.71	0.01\\
35.72	0.01\\
35.73	0.01\\
35.74	0.01\\
35.75	0.01\\
35.76	0.01\\
35.77	0.01\\
35.78	0.01\\
35.79	0.01\\
35.8	0.01\\
35.81	0.01\\
35.82	0.01\\
35.83	0.01\\
35.84	0.01\\
35.85	0.01\\
35.86	0.01\\
35.87	0.01\\
35.88	0.01\\
35.89	0.01\\
35.9	0.01\\
35.91	0.01\\
35.92	0.01\\
35.93	0.01\\
35.94	0.01\\
35.95	0.01\\
35.96	0.01\\
35.97	0.01\\
35.98	0.01\\
35.99	0.01\\
36	0.01\\
36.01	0.01\\
36.02	0.01\\
36.03	0.01\\
36.04	0.01\\
36.05	0.01\\
36.06	0.01\\
36.07	0.01\\
36.08	0.01\\
36.09	0.01\\
36.1	0.01\\
36.11	0.01\\
36.12	0.01\\
36.13	0.01\\
36.14	0.01\\
36.15	0.01\\
36.16	0.01\\
36.17	0.01\\
36.18	0.01\\
36.19	0.01\\
36.2	0.01\\
36.21	0.01\\
36.22	0.01\\
36.23	0.01\\
36.24	0.01\\
36.25	0.01\\
36.26	0.01\\
36.27	0.01\\
36.28	0.01\\
36.29	0.01\\
36.3	0.01\\
36.31	0.01\\
36.32	0.01\\
36.33	0.01\\
36.34	0.01\\
36.35	0.01\\
36.36	0.01\\
36.37	0.01\\
36.38	0.01\\
36.39	0.01\\
36.4	0.01\\
36.41	0.01\\
36.42	0.01\\
36.43	0.01\\
36.44	0.01\\
36.45	0.01\\
36.46	0.01\\
36.47	0.01\\
36.48	0.01\\
36.49	0.01\\
36.5	0.01\\
36.51	0.01\\
36.52	0.01\\
36.53	0.01\\
36.54	0.01\\
36.55	0.01\\
36.56	0.01\\
36.57	0.01\\
36.58	0.01\\
36.59	0.01\\
36.6	0.01\\
36.61	0.01\\
36.62	0.01\\
36.63	0.01\\
36.64	0.01\\
36.65	0.01\\
36.66	0.01\\
36.67	0.01\\
36.68	0.01\\
36.69	0.01\\
36.7	0.01\\
36.71	0.01\\
36.72	0.01\\
36.73	0.01\\
36.74	0.01\\
36.75	0.01\\
36.76	0.01\\
36.77	0.01\\
36.78	0.01\\
36.79	0.01\\
36.8	0.01\\
36.81	0.01\\
36.82	0.01\\
36.83	0.01\\
36.84	0.01\\
36.85	0.01\\
36.86	0.01\\
36.87	0.01\\
36.88	0.01\\
36.89	0.01\\
36.9	0.01\\
36.91	0.01\\
36.92	0.01\\
36.93	0.01\\
36.94	0.01\\
36.95	0.01\\
36.96	0.01\\
36.97	0.01\\
36.98	0.01\\
36.99	0.01\\
37	0.01\\
37.01	0.01\\
37.02	0.01\\
37.03	0.01\\
37.04	0.01\\
37.05	0.01\\
37.06	0.01\\
37.07	0.01\\
37.08	0.01\\
37.09	0.01\\
37.1	0.01\\
37.11	0.01\\
37.12	0.01\\
37.13	0.01\\
37.14	0.01\\
37.15	0.01\\
37.16	0.01\\
37.17	0.01\\
37.18	0.01\\
37.19	0.01\\
37.2	0.01\\
37.21	0.01\\
37.22	0.01\\
37.23	0.01\\
37.24	0.01\\
37.25	0.01\\
37.26	0.01\\
37.27	0.01\\
37.28	0.01\\
37.29	0.01\\
37.3	0.01\\
37.31	0.01\\
37.32	0.01\\
37.33	0.01\\
37.34	0.01\\
37.35	0.01\\
37.36	0.01\\
37.37	0.01\\
37.38	0.01\\
37.39	0.01\\
37.4	0.01\\
37.41	0.01\\
37.42	0.01\\
37.43	0.01\\
37.44	0.01\\
37.45	0.01\\
37.46	0.01\\
37.47	0.01\\
37.48	0.01\\
37.49	0.01\\
37.5	0.01\\
37.51	0.01\\
37.52	0.01\\
37.53	0.01\\
37.54	0.01\\
37.55	0.01\\
37.56	0.01\\
37.57	0.01\\
37.58	0.01\\
37.59	0.01\\
37.6	0.01\\
37.61	0.01\\
37.62	0.01\\
37.63	0.01\\
37.64	0.01\\
37.65	0.01\\
37.66	0.01\\
37.67	0.01\\
37.68	0.01\\
37.69	0.01\\
37.7	0.01\\
37.71	0.01\\
37.72	0.01\\
37.73	0.01\\
37.74	0.01\\
37.75	0.01\\
37.76	0.01\\
37.77	0.01\\
37.78	0.01\\
37.79	0.01\\
37.8	0.01\\
37.81	0.01\\
37.82	0.01\\
37.83	0.01\\
37.84	0.01\\
37.85	0.01\\
37.86	0.01\\
37.87	0.01\\
37.88	0.01\\
37.89	0.01\\
37.9	0.01\\
37.91	0.01\\
37.92	0.01\\
37.93	0.01\\
37.94	0.01\\
37.95	0.01\\
37.96	0.01\\
37.97	0.01\\
37.98	0.01\\
37.99	0.01\\
38	0.01\\
38.01	0.01\\
38.02	0.01\\
38.03	0.01\\
38.04	0.01\\
38.05	0.01\\
38.06	0.01\\
38.07	0.01\\
38.08	0.01\\
38.09	0.01\\
38.1	0.01\\
38.11	0.01\\
38.12	0.01\\
38.13	0.01\\
38.14	0.01\\
38.15	0.01\\
38.16	0.01\\
38.17	0.01\\
38.18	0.01\\
38.19	0.01\\
38.2	0.01\\
38.21	0.01\\
38.22	0.01\\
38.23	0.01\\
38.24	0.01\\
38.25	0.01\\
38.26	0.01\\
38.27	0.01\\
38.28	0.01\\
38.29	0.01\\
38.3	0.01\\
38.31	0.01\\
38.32	0.01\\
38.33	0.01\\
38.34	0.01\\
38.35	0.01\\
38.36	0.01\\
38.37	0.01\\
38.38	0.01\\
38.39	0.01\\
38.4	0.01\\
38.41	0.01\\
38.42	0.01\\
38.43	0.01\\
38.44	0.01\\
38.45	0.01\\
38.46	0.01\\
38.47	0.01\\
38.48	0.01\\
38.49	0.01\\
38.5	0.01\\
38.51	0.01\\
38.52	0.01\\
38.53	0.01\\
38.54	0.01\\
38.55	0.01\\
38.56	0.01\\
38.57	0.01\\
38.58	0.01\\
38.59	0.01\\
38.6	0.01\\
38.61	0.01\\
38.62	0.01\\
38.63	0.01\\
38.64	0.01\\
38.65	0.01\\
38.66	0.01\\
38.67	0.01\\
38.68	0.01\\
38.69	0.01\\
38.7	0.01\\
38.71	0.01\\
38.72	0.01\\
38.73	0.01\\
38.74	0.01\\
38.75	0.01\\
38.76	0.01\\
38.77	0.01\\
38.78	0.01\\
38.79	0.01\\
38.8	0.01\\
38.81	0.01\\
38.82	0.01\\
38.83	0.01\\
38.84	0.01\\
38.85	0.01\\
38.86	0.01\\
38.87	0.01\\
38.88	0.01\\
38.89	0.01\\
38.9	0.01\\
38.91	0.01\\
38.92	0.01\\
38.93	0.01\\
38.94	0.01\\
38.95	0.01\\
38.96	0.01\\
38.97	0.01\\
38.98	0.01\\
38.99	0.01\\
39	0.01\\
39.01	0.01\\
39.02	0.01\\
39.03	0.01\\
39.04	0.01\\
39.05	0.01\\
39.06	0.01\\
39.07	0.01\\
39.08	0.01\\
39.09	0.01\\
39.1	0.01\\
39.11	0.01\\
39.12	0.01\\
39.13	0.01\\
39.14	0.01\\
39.15	0.01\\
39.16	0.01\\
39.17	0.01\\
39.18	0.01\\
39.19	0.01\\
39.2	0.01\\
39.21	0.01\\
39.22	0.01\\
39.23	0.01\\
39.24	0.01\\
39.25	0.01\\
39.26	0.01\\
39.27	0.01\\
39.28	0.01\\
39.29	0.01\\
39.3	0.01\\
39.31	0.01\\
39.32	0.01\\
39.33	0.01\\
39.34	0.01\\
39.35	0.01\\
39.36	0.01\\
39.37	0.01\\
39.38	0.01\\
39.39	0.01\\
39.4	0.01\\
39.41	0.01\\
39.42	0.01\\
39.43	0.01\\
39.44	0.01\\
39.45	0.01\\
39.46	0.01\\
39.47	0.01\\
39.48	0.01\\
39.49	0.01\\
39.5	0.01\\
39.51	0.01\\
39.52	0.01\\
39.53	0.01\\
39.54	0.01\\
39.55	0.01\\
39.56	0.01\\
39.57	0.01\\
39.58	0.01\\
39.59	0.01\\
39.6	0.01\\
39.61	0.01\\
39.62	0.01\\
39.63	0.01\\
39.64	0.01\\
39.65	0.01\\
39.66	0.01\\
39.67	0.01\\
39.68	0.01\\
39.69	0.01\\
39.7	0.01\\
39.71	0.01\\
39.72	0.01\\
39.73	0.01\\
39.74	0.01\\
39.75	0.01\\
39.76	0.01\\
39.77	0.01\\
39.78	0.01\\
39.79	0.01\\
39.8	0.01\\
39.81	0.01\\
39.82	0.01\\
39.83	0.01\\
39.84	0.01\\
39.85	0.01\\
39.86	0.01\\
39.87	0.01\\
39.88	0.01\\
39.89	0.01\\
39.9	0.01\\
39.91	0.01\\
39.92	0.01\\
39.93	0.01\\
39.94	0.01\\
39.95	0.01\\
39.96	0.01\\
39.97	0.01\\
39.98	0.01\\
39.99	0.01\\
40	0.01\\
40.01	0.01\\
};
\addplot [color=mycolor1,solid,forget plot]
  table[row sep=crcr]{%
40.01	0.01\\
40.02	0.01\\
40.03	0.01\\
40.04	0.01\\
40.05	0.01\\
40.06	0.01\\
40.07	0.01\\
40.08	0.01\\
40.09	0.01\\
40.1	0.01\\
40.11	0.01\\
40.12	0.01\\
40.13	0.01\\
40.14	0.01\\
40.15	0.01\\
40.16	0.01\\
40.17	0.01\\
40.18	0.01\\
40.19	0.01\\
40.2	0.01\\
40.21	0.01\\
40.22	0.01\\
40.23	0.01\\
40.24	0.01\\
40.25	0.01\\
40.26	0.01\\
40.27	0.01\\
40.28	0.01\\
40.29	0.01\\
40.3	0.01\\
40.31	0.01\\
40.32	0.01\\
40.33	0.01\\
40.34	0.01\\
40.35	0.01\\
40.36	0.01\\
40.37	0.01\\
40.38	0.01\\
40.39	0.01\\
40.4	0.01\\
40.41	0.01\\
40.42	0.01\\
40.43	0.01\\
40.44	0.01\\
40.45	0.01\\
40.46	0.01\\
40.47	0.01\\
40.48	0.01\\
40.49	0.01\\
40.5	0.01\\
40.51	0.01\\
40.52	0.01\\
40.53	0.01\\
40.54	0.01\\
40.55	0.01\\
40.56	0.01\\
40.57	0.01\\
40.58	0.01\\
40.59	0.01\\
40.6	0.01\\
40.61	0.01\\
40.62	0.01\\
40.63	0.01\\
40.64	0.01\\
40.65	0.01\\
40.66	0.01\\
40.67	0.01\\
40.68	0.01\\
40.69	0.01\\
40.7	0.01\\
40.71	0.01\\
40.72	0.01\\
40.73	0.01\\
40.74	0.01\\
40.75	0.01\\
40.76	0.01\\
40.77	0.01\\
40.78	0.01\\
40.79	0.01\\
40.8	0.01\\
40.81	0.01\\
40.82	0.01\\
40.83	0.01\\
40.84	0.01\\
40.85	0.01\\
40.86	0.01\\
40.87	0.01\\
40.88	0.01\\
40.89	0.01\\
40.9	0.01\\
40.91	0.01\\
40.92	0.01\\
40.93	0.01\\
40.94	0.01\\
40.95	0.01\\
40.96	0.01\\
40.97	0.01\\
40.98	0.01\\
40.99	0.01\\
41	0.01\\
41.01	0.01\\
41.02	0.01\\
41.03	0.01\\
41.04	0.01\\
41.05	0.01\\
41.06	0.01\\
41.07	0.01\\
41.08	0.01\\
41.09	0.01\\
41.1	0.01\\
41.11	0.01\\
41.12	0.01\\
41.13	0.01\\
41.14	0.01\\
41.15	0.01\\
41.16	0.01\\
41.17	0.01\\
41.18	0.01\\
41.19	0.01\\
41.2	0.01\\
41.21	0.01\\
41.22	0.01\\
41.23	0.01\\
41.24	0.01\\
41.25	0.01\\
41.26	0.01\\
41.27	0.01\\
41.28	0.01\\
41.29	0.01\\
41.3	0.01\\
41.31	0.01\\
41.32	0.01\\
41.33	0.01\\
41.34	0.01\\
41.35	0.01\\
41.36	0.01\\
41.37	0.01\\
41.38	0.01\\
41.39	0.01\\
41.4	0.01\\
41.41	0.01\\
41.42	0.01\\
41.43	0.01\\
41.44	0.01\\
41.45	0.01\\
41.46	0.01\\
41.47	0.01\\
41.48	0.01\\
41.49	0.01\\
41.5	0.01\\
41.51	0.01\\
41.52	0.01\\
41.53	0.01\\
41.54	0.01\\
41.55	0.01\\
41.56	0.01\\
41.57	0.01\\
41.58	0.01\\
41.59	0.01\\
41.6	0.01\\
41.61	0.01\\
41.62	0.01\\
41.63	0.01\\
41.64	0.01\\
41.65	0.01\\
41.66	0.01\\
41.67	0.01\\
41.68	0.01\\
41.69	0.01\\
41.7	0.01\\
41.71	0.01\\
41.72	0.01\\
41.73	0.01\\
41.74	0.01\\
41.75	0.01\\
41.76	0.01\\
41.77	0.01\\
41.78	0.01\\
41.79	0.01\\
41.8	0.01\\
41.81	0.01\\
41.82	0.01\\
41.83	0.01\\
41.84	0.01\\
41.85	0.01\\
41.86	0.01\\
41.87	0.01\\
41.88	0.01\\
41.89	0.01\\
41.9	0.01\\
41.91	0.01\\
41.92	0.01\\
41.93	0.01\\
41.94	0.01\\
41.95	0.01\\
41.96	0.01\\
41.97	0.01\\
41.98	0.01\\
41.99	0.01\\
42	0.01\\
42.01	0.01\\
42.02	0.01\\
42.03	0.01\\
42.04	0.01\\
42.05	0.01\\
42.06	0.01\\
42.07	0.01\\
42.08	0.01\\
42.09	0.01\\
42.1	0.01\\
42.11	0.01\\
42.12	0.01\\
42.13	0.01\\
42.14	0.01\\
42.15	0.01\\
42.16	0.01\\
42.17	0.01\\
42.18	0.01\\
42.19	0.01\\
42.2	0.01\\
42.21	0.01\\
42.22	0.01\\
42.23	0.01\\
42.24	0.01\\
42.25	0.01\\
42.26	0.01\\
42.27	0.01\\
42.28	0.01\\
42.29	0.01\\
42.3	0.01\\
42.31	0.01\\
42.32	0.01\\
42.33	0.01\\
42.34	0.01\\
42.35	0.01\\
42.36	0.01\\
42.37	0.01\\
42.38	0.01\\
42.39	0.01\\
42.4	0.01\\
42.41	0.01\\
42.42	0.01\\
42.43	0.01\\
42.44	0.01\\
42.45	0.01\\
42.46	0.01\\
42.47	0.01\\
42.48	0.01\\
42.49	0.01\\
42.5	0.01\\
42.51	0.01\\
42.52	0.01\\
42.53	0.01\\
42.54	0.01\\
42.55	0.01\\
42.56	0.01\\
42.57	0.01\\
42.58	0.01\\
42.59	0.01\\
42.6	0.01\\
42.61	0.01\\
42.62	0.01\\
42.63	0.01\\
42.64	0.01\\
42.65	0.01\\
42.66	0.01\\
42.67	0.01\\
42.68	0.01\\
42.69	0.01\\
42.7	0.01\\
42.71	0.01\\
42.72	0.01\\
42.73	0.01\\
42.74	0.01\\
42.75	0.01\\
42.76	0.01\\
42.77	0.01\\
42.78	0.01\\
42.79	0.01\\
42.8	0.01\\
42.81	0.01\\
42.82	0.01\\
42.83	0.01\\
42.84	0.01\\
42.85	0.01\\
42.86	0.01\\
42.87	0.01\\
42.88	0.01\\
42.89	0.01\\
42.9	0.01\\
42.91	0.01\\
42.92	0.01\\
42.93	0.01\\
42.94	0.01\\
42.95	0.01\\
42.96	0.01\\
42.97	0.01\\
42.98	0.01\\
42.99	0.01\\
43	0.01\\
43.01	0.01\\
43.02	0.01\\
43.03	0.01\\
43.04	0.01\\
43.05	0.01\\
43.06	0.01\\
43.07	0.01\\
43.08	0.01\\
43.09	0.01\\
43.1	0.01\\
43.11	0.01\\
43.12	0.01\\
43.13	0.01\\
43.14	0.01\\
43.15	0.01\\
43.16	0.01\\
43.17	0.01\\
43.18	0.01\\
43.19	0.01\\
43.2	0.01\\
43.21	0.01\\
43.22	0.01\\
43.23	0.01\\
43.24	0.01\\
43.25	0.01\\
43.26	0.01\\
43.27	0.01\\
43.28	0.01\\
43.29	0.01\\
43.3	0.01\\
43.31	0.01\\
43.32	0.01\\
43.33	0.01\\
43.34	0.01\\
43.35	0.01\\
43.36	0.01\\
43.37	0.01\\
43.38	0.01\\
43.39	0.01\\
43.4	0.01\\
43.41	0.01\\
43.42	0.01\\
43.43	0.01\\
43.44	0.01\\
43.45	0.01\\
43.46	0.01\\
43.47	0.01\\
43.48	0.01\\
43.49	0.01\\
43.5	0.01\\
43.51	0.01\\
43.52	0.01\\
43.53	0.01\\
43.54	0.01\\
43.55	0.01\\
43.56	0.01\\
43.57	0.01\\
43.58	0.01\\
43.59	0.01\\
43.6	0.01\\
43.61	0.01\\
43.62	0.01\\
43.63	0.01\\
43.64	0.01\\
43.65	0.01\\
43.66	0.01\\
43.67	0.01\\
43.68	0.01\\
43.69	0.01\\
43.7	0.01\\
43.71	0.01\\
43.72	0.01\\
43.73	0.01\\
43.74	0.01\\
43.75	0.01\\
43.76	0.01\\
43.77	0.01\\
43.78	0.01\\
43.79	0.01\\
43.8	0.01\\
43.81	0.01\\
43.82	0.01\\
43.83	0.01\\
43.84	0.01\\
43.85	0.01\\
43.86	0.01\\
43.87	0.01\\
43.88	0.01\\
43.89	0.01\\
43.9	0.01\\
43.91	0.01\\
43.92	0.01\\
43.93	0.01\\
43.94	0.01\\
43.95	0.01\\
43.96	0.01\\
43.97	0.01\\
43.98	0.01\\
43.99	0.01\\
44	0.01\\
44.01	0.01\\
44.02	0.01\\
44.03	0.01\\
44.04	0.01\\
44.05	0.01\\
44.06	0.01\\
44.07	0.01\\
44.08	0.01\\
44.09	0.01\\
44.1	0.01\\
44.11	0.01\\
44.12	0.01\\
44.13	0.01\\
44.14	0.01\\
44.15	0.01\\
44.16	0.01\\
44.17	0.01\\
44.18	0.01\\
44.19	0.01\\
44.2	0.01\\
44.21	0.01\\
44.22	0.01\\
44.23	0.01\\
44.24	0.01\\
44.25	0.01\\
44.26	0.01\\
44.27	0.01\\
44.28	0.01\\
44.29	0.01\\
44.3	0.01\\
44.31	0.01\\
44.32	0.01\\
44.33	0.01\\
44.34	0.01\\
44.35	0.01\\
44.36	0.01\\
44.37	0.01\\
44.38	0.01\\
44.39	0.01\\
44.4	0.01\\
44.41	0.01\\
44.42	0.01\\
44.43	0.01\\
44.44	0.01\\
44.45	0.01\\
44.46	0.01\\
44.47	0.01\\
44.48	0.01\\
44.49	0.01\\
44.5	0.01\\
44.51	0.01\\
44.52	0.01\\
44.53	0.01\\
44.54	0.01\\
44.55	0.01\\
44.56	0.01\\
44.57	0.01\\
44.58	0.01\\
44.59	0.01\\
44.6	0.01\\
44.61	0.01\\
44.62	0.01\\
44.63	0.01\\
44.64	0.01\\
44.65	0.01\\
44.66	0.01\\
44.67	0.01\\
44.68	0.01\\
44.69	0.01\\
44.7	0.01\\
44.71	0.01\\
44.72	0.01\\
44.73	0.01\\
44.74	0.01\\
44.75	0.01\\
44.76	0.01\\
44.77	0.01\\
44.78	0.01\\
44.79	0.01\\
44.8	0.01\\
44.81	0.01\\
44.82	0.01\\
44.83	0.01\\
44.84	0.01\\
44.85	0.01\\
44.86	0.01\\
44.87	0.01\\
44.88	0.01\\
44.89	0.01\\
44.9	0.01\\
44.91	0.01\\
44.92	0.01\\
44.93	0.01\\
44.94	0.01\\
44.95	0.01\\
44.96	0.01\\
44.97	0.01\\
44.98	0.01\\
44.99	0.01\\
45	0.01\\
45.01	0.01\\
45.02	0.01\\
45.03	0.01\\
45.04	0.01\\
45.05	0.01\\
45.06	0.01\\
45.07	0.01\\
45.08	0.01\\
45.09	0.01\\
45.1	0.01\\
45.11	0.01\\
45.12	0.01\\
45.13	0.01\\
45.14	0.01\\
45.15	0.01\\
45.16	0.01\\
45.17	0.01\\
45.18	0.01\\
45.19	0.01\\
45.2	0.01\\
45.21	0.01\\
45.22	0.01\\
45.23	0.01\\
45.24	0.01\\
45.25	0.01\\
45.26	0.01\\
45.27	0.01\\
45.28	0.01\\
45.29	0.01\\
45.3	0.01\\
45.31	0.01\\
45.32	0.01\\
45.33	0.01\\
45.34	0.01\\
45.35	0.01\\
45.36	0.01\\
45.37	0.01\\
45.38	0.01\\
45.39	0.01\\
45.4	0.01\\
45.41	0.01\\
45.42	0.01\\
45.43	0.01\\
45.44	0.01\\
45.45	0.01\\
45.46	0.01\\
45.47	0.01\\
45.48	0.01\\
45.49	0.01\\
45.5	0.01\\
45.51	0.01\\
45.52	0.01\\
45.53	0.01\\
45.54	0.01\\
45.55	0.01\\
45.56	0.01\\
45.57	0.01\\
45.58	0.01\\
45.59	0.01\\
45.6	0.01\\
45.61	0.01\\
45.62	0.01\\
45.63	0.01\\
45.64	0.01\\
45.65	0.01\\
45.66	0.01\\
45.67	0.01\\
45.68	0.01\\
45.69	0.01\\
45.7	0.01\\
45.71	0.01\\
45.72	0.01\\
45.73	0.01\\
45.74	0.01\\
45.75	0.01\\
45.76	0.01\\
45.77	0.01\\
45.78	0.01\\
45.79	0.01\\
45.8	0.01\\
45.81	0.01\\
45.82	0.01\\
45.83	0.01\\
45.84	0.01\\
45.85	0.01\\
45.86	0.01\\
45.87	0.01\\
45.88	0.01\\
45.89	0.01\\
45.9	0.01\\
45.91	0.01\\
45.92	0.01\\
45.93	0.01\\
45.94	0.01\\
45.95	0.01\\
45.96	0.01\\
45.97	0.01\\
45.98	0.01\\
45.99	0.01\\
46	0.01\\
46.01	0.01\\
46.02	0.01\\
46.03	0.01\\
46.04	0.01\\
46.05	0.01\\
46.06	0.01\\
46.07	0.01\\
46.08	0.01\\
46.09	0.01\\
46.1	0.01\\
46.11	0.01\\
46.12	0.01\\
46.13	0.01\\
46.14	0.01\\
46.15	0.01\\
46.16	0.01\\
46.17	0.01\\
46.18	0.01\\
46.19	0.01\\
46.2	0.01\\
46.21	0.01\\
46.22	0.01\\
46.23	0.01\\
46.24	0.01\\
46.25	0.01\\
46.26	0.01\\
46.27	0.01\\
46.28	0.01\\
46.29	0.01\\
46.3	0.01\\
46.31	0.01\\
46.32	0.01\\
46.33	0.01\\
46.34	0.01\\
46.35	0.01\\
46.36	0.01\\
46.37	0.01\\
46.38	0.01\\
46.39	0.01\\
46.4	0.01\\
46.41	0.01\\
46.42	0.01\\
46.43	0.01\\
46.44	0.01\\
46.45	0.01\\
46.46	0.01\\
46.47	0.01\\
46.48	0.01\\
46.49	0.01\\
46.5	0.01\\
46.51	0.01\\
46.52	0.01\\
46.53	0.01\\
46.54	0.01\\
46.55	0.01\\
46.56	0.01\\
46.57	0.01\\
46.58	0.01\\
46.59	0.01\\
46.6	0.01\\
46.61	0.01\\
46.62	0.01\\
46.63	0.01\\
46.64	0.01\\
46.65	0.01\\
46.66	0.01\\
46.67	0.01\\
46.68	0.01\\
46.69	0.01\\
46.7	0.01\\
46.71	0.01\\
46.72	0.01\\
46.73	0.01\\
46.74	0.01\\
46.75	0.01\\
46.76	0.01\\
46.77	0.01\\
46.78	0.01\\
46.79	0.01\\
46.8	0.01\\
46.81	0.01\\
46.82	0.01\\
46.83	0.01\\
46.84	0.01\\
46.85	0.01\\
46.86	0.01\\
46.87	0.01\\
46.88	0.01\\
46.89	0.01\\
46.9	0.01\\
46.91	0.01\\
46.92	0.01\\
46.93	0.01\\
46.94	0.01\\
46.95	0.01\\
46.96	0.01\\
46.97	0.01\\
46.98	0.01\\
46.99	0.01\\
47	0.01\\
47.01	0.01\\
47.02	0.01\\
47.03	0.01\\
47.04	0.01\\
47.05	0.01\\
47.06	0.01\\
47.07	0.01\\
47.08	0.01\\
47.09	0.01\\
47.1	0.01\\
47.11	0.01\\
47.12	0.01\\
47.13	0.01\\
47.14	0.01\\
47.15	0.01\\
47.16	0.01\\
47.17	0.01\\
47.18	0.01\\
47.19	0.01\\
47.2	0.01\\
47.21	0.01\\
47.22	0.01\\
47.23	0.01\\
47.24	0.01\\
47.25	0.01\\
47.26	0.01\\
47.27	0.01\\
47.28	0.01\\
47.29	0.01\\
47.3	0.01\\
47.31	0.01\\
47.32	0.01\\
47.33	0.01\\
47.34	0.01\\
47.35	0.01\\
47.36	0.01\\
47.37	0.01\\
47.38	0.01\\
47.39	0.01\\
47.4	0.01\\
47.41	0.01\\
47.42	0.01\\
47.43	0.01\\
47.44	0.01\\
47.45	0.01\\
47.46	0.01\\
47.47	0.01\\
47.48	0.01\\
47.49	0.01\\
47.5	0.01\\
47.51	0.01\\
47.52	0.01\\
47.53	0.01\\
47.54	0.01\\
47.55	0.01\\
47.56	0.01\\
47.57	0.01\\
47.58	0.01\\
47.59	0.01\\
47.6	0.01\\
47.61	0.01\\
47.62	0.01\\
47.63	0.01\\
47.64	0.01\\
47.65	0.01\\
47.66	0.01\\
47.67	0.01\\
47.68	0.01\\
47.69	0.01\\
47.7	0.01\\
47.71	0.01\\
47.72	0.01\\
47.73	0.01\\
47.74	0.01\\
47.75	0.01\\
47.76	0.01\\
47.77	0.01\\
47.78	0.01\\
47.79	0.01\\
47.8	0.01\\
47.81	0.01\\
47.82	0.01\\
47.83	0.01\\
47.84	0.01\\
47.85	0.01\\
47.86	0.01\\
47.87	0.01\\
47.88	0.01\\
47.89	0.01\\
47.9	0.01\\
47.91	0.01\\
47.92	0.01\\
47.93	0.01\\
47.94	0.01\\
47.95	0.01\\
47.96	0.01\\
47.97	0.01\\
47.98	0.01\\
47.99	0.01\\
48	0.01\\
48.01	0.01\\
48.02	0.01\\
48.03	0.01\\
48.04	0.01\\
48.05	0.01\\
48.06	0.01\\
48.07	0.01\\
48.08	0.01\\
48.09	0.01\\
48.1	0.01\\
48.11	0.01\\
48.12	0.01\\
48.13	0.01\\
48.14	0.01\\
48.15	0.01\\
48.16	0.01\\
48.17	0.01\\
48.18	0.01\\
48.19	0.01\\
48.2	0.01\\
48.21	0.01\\
48.22	0.01\\
48.23	0.01\\
48.24	0.01\\
48.25	0.01\\
48.26	0.01\\
48.27	0.01\\
48.28	0.01\\
48.29	0.01\\
48.3	0.01\\
48.31	0.01\\
48.32	0.01\\
48.33	0.01\\
48.34	0.01\\
48.35	0.01\\
48.36	0.01\\
48.37	0.01\\
48.38	0.01\\
48.39	0.01\\
48.4	0.01\\
48.41	0.01\\
48.42	0.01\\
48.43	0.01\\
48.44	0.01\\
48.45	0.01\\
48.46	0.01\\
48.47	0.01\\
48.48	0.01\\
48.49	0.01\\
48.5	0.01\\
48.51	0.01\\
48.52	0.01\\
48.53	0.01\\
48.54	0.01\\
48.55	0.01\\
48.56	0.01\\
48.57	0.01\\
48.58	0.01\\
48.59	0.01\\
48.6	0.01\\
48.61	0.01\\
48.62	0.01\\
48.63	0.01\\
48.64	0.01\\
48.65	0.01\\
48.66	0.01\\
48.67	0.01\\
48.68	0.01\\
48.69	0.01\\
48.7	0.01\\
48.71	0.01\\
48.72	0.01\\
48.73	0.01\\
48.74	0.01\\
48.75	0.01\\
48.76	0.01\\
48.77	0.01\\
48.78	0.01\\
48.79	0.01\\
48.8	0.01\\
48.81	0.01\\
48.82	0.01\\
48.83	0.01\\
48.84	0.01\\
48.85	0.01\\
48.86	0.01\\
48.87	0.01\\
48.88	0.01\\
48.89	0.01\\
48.9	0.01\\
48.91	0.01\\
48.92	0.01\\
48.93	0.01\\
48.94	0.01\\
48.95	0.01\\
48.96	0.01\\
48.97	0.01\\
48.98	0.01\\
48.99	0.01\\
49	0.01\\
49.01	0.01\\
49.02	0.01\\
49.03	0.01\\
49.04	0.01\\
49.05	0.01\\
49.06	0.01\\
49.07	0.01\\
49.08	0.01\\
49.09	0.01\\
49.1	0.01\\
49.11	0.01\\
49.12	0.01\\
49.13	0.01\\
49.14	0.01\\
49.15	0.01\\
49.16	0.01\\
49.17	0.01\\
49.18	0.01\\
49.19	0.01\\
49.2	0.01\\
49.21	0.01\\
49.22	0.01\\
49.23	0.01\\
49.24	0.01\\
49.25	0.01\\
49.26	0.01\\
49.27	0.01\\
49.28	0.01\\
49.29	0.01\\
49.3	0.01\\
49.31	0.01\\
49.32	0.01\\
49.33	0.01\\
49.34	0.01\\
49.35	0.01\\
49.36	0.01\\
49.37	0.01\\
49.38	0.01\\
49.39	0.01\\
49.4	0.01\\
49.41	0.01\\
49.42	0.01\\
49.43	0.01\\
49.44	0.01\\
49.45	0.01\\
49.46	0.01\\
49.47	0.01\\
49.48	0.01\\
49.49	0.01\\
49.5	0.01\\
49.51	0.01\\
49.52	0.01\\
49.53	0.01\\
49.54	0.01\\
49.55	0.01\\
49.56	0.01\\
49.57	0.01\\
49.58	0.01\\
49.59	0.01\\
49.6	0.01\\
49.61	0.01\\
49.62	0.01\\
49.63	0.01\\
49.64	0.01\\
49.65	0.01\\
49.66	0.01\\
49.67	0.01\\
49.68	0.01\\
49.69	0.01\\
49.7	0.01\\
49.71	0.01\\
49.72	0.01\\
49.73	0.01\\
49.74	0.01\\
49.75	0.01\\
49.76	0.01\\
49.77	0.01\\
49.78	0.01\\
49.79	0.01\\
49.8	0.01\\
49.81	0.01\\
49.82	0.01\\
49.83	0.01\\
49.84	0.01\\
49.85	0.01\\
49.86	0.01\\
49.87	0.01\\
49.88	0.01\\
49.89	0.01\\
49.9	0.01\\
49.91	0.01\\
49.92	0.01\\
49.93	0.01\\
49.94	0.01\\
49.95	0.01\\
49.96	0.01\\
49.97	0.01\\
49.98	0.01\\
49.99	0.01\\
50	0.01\\
50.01	0.01\\
50.02	0.01\\
50.03	0.01\\
50.04	0.01\\
50.05	0.01\\
50.06	0.01\\
50.07	0.01\\
50.08	0.01\\
50.09	0.01\\
50.1	0.01\\
50.11	0.01\\
50.12	0.01\\
50.13	0.01\\
50.14	0.01\\
50.15	0.01\\
50.16	0.01\\
50.17	0.01\\
50.18	0.01\\
50.19	0.01\\
50.2	0.01\\
50.21	0.01\\
50.22	0.01\\
50.23	0.01\\
50.24	0.01\\
50.25	0.01\\
50.26	0.01\\
50.27	0.01\\
50.28	0.01\\
50.29	0.01\\
50.3	0.01\\
50.31	0.01\\
50.32	0.01\\
50.33	0.01\\
50.34	0.01\\
50.35	0.01\\
50.36	0.01\\
50.37	0.01\\
50.38	0.01\\
50.39	0.01\\
50.4	0.01\\
50.41	0.01\\
50.42	0.01\\
50.43	0.01\\
50.44	0.01\\
50.45	0.01\\
50.46	0.01\\
50.47	0.01\\
50.48	0.01\\
50.49	0.01\\
50.5	0.01\\
50.51	0.01\\
50.52	0.01\\
50.53	0.01\\
50.54	0.01\\
50.55	0.01\\
50.56	0.01\\
50.57	0.01\\
50.58	0.01\\
50.59	0.01\\
50.6	0.01\\
50.61	0.01\\
50.62	0.01\\
50.63	0.01\\
50.64	0.01\\
50.65	0.01\\
50.66	0.01\\
50.67	0.01\\
50.68	0.01\\
50.69	0.01\\
50.7	0.01\\
50.71	0.01\\
50.72	0.01\\
50.73	0.01\\
50.74	0.01\\
50.75	0.01\\
50.76	0.01\\
50.77	0.01\\
50.78	0.01\\
50.79	0.01\\
50.8	0.01\\
50.81	0.01\\
50.82	0.01\\
50.83	0.01\\
50.84	0.01\\
50.85	0.01\\
50.86	0.01\\
50.87	0.01\\
50.88	0.01\\
50.89	0.01\\
50.9	0.01\\
50.91	0.01\\
50.92	0.01\\
50.93	0.01\\
50.94	0.01\\
50.95	0.01\\
50.96	0.01\\
50.97	0.01\\
50.98	0.01\\
50.99	0.01\\
51	0.01\\
51.01	0.01\\
51.02	0.01\\
51.03	0.01\\
51.04	0.01\\
51.05	0.01\\
51.06	0.01\\
51.07	0.01\\
51.08	0.01\\
51.09	0.01\\
51.1	0.01\\
51.11	0.01\\
51.12	0.01\\
51.13	0.01\\
51.14	0.01\\
51.15	0.01\\
51.16	0.01\\
51.17	0.01\\
51.18	0.01\\
51.19	0.01\\
51.2	0.01\\
51.21	0.01\\
51.22	0.01\\
51.23	0.01\\
51.24	0.01\\
51.25	0.01\\
51.26	0.01\\
51.27	0.01\\
51.28	0.01\\
51.29	0.01\\
51.3	0.01\\
51.31	0.01\\
51.32	0.01\\
51.33	0.01\\
51.34	0.01\\
51.35	0.01\\
51.36	0.01\\
51.37	0.01\\
51.38	0.01\\
51.39	0.01\\
51.4	0.01\\
51.41	0.01\\
51.42	0.01\\
51.43	0.01\\
51.44	0.01\\
51.45	0.01\\
51.46	0.01\\
51.47	0.01\\
51.48	0.01\\
51.49	0.01\\
51.5	0.01\\
51.51	0.01\\
51.52	0.01\\
51.53	0.01\\
51.54	0.01\\
51.55	0.01\\
51.56	0.01\\
51.57	0.01\\
51.58	0.01\\
51.59	0.01\\
51.6	0.01\\
51.61	0.01\\
51.62	0.01\\
51.63	0.01\\
51.64	0.01\\
51.65	0.01\\
51.66	0.01\\
51.67	0.01\\
51.68	0.01\\
51.69	0.01\\
51.7	0.01\\
51.71	0.01\\
51.72	0.01\\
51.73	0.01\\
51.74	0.01\\
51.75	0.01\\
51.76	0.01\\
51.77	0.01\\
51.78	0.01\\
51.79	0.01\\
51.8	0.01\\
51.81	0.01\\
51.82	0.01\\
51.83	0.01\\
51.84	0.01\\
51.85	0.01\\
51.86	0.01\\
51.87	0.01\\
51.88	0.01\\
51.89	0.01\\
51.9	0.01\\
51.91	0.01\\
51.92	0.01\\
51.93	0.01\\
51.94	0.01\\
51.95	0.01\\
51.96	0.01\\
51.97	0.01\\
51.98	0.01\\
51.99	0.01\\
52	0.01\\
52.01	0.01\\
52.02	0.01\\
52.03	0.01\\
52.04	0.01\\
52.05	0.01\\
52.06	0.01\\
52.07	0.01\\
52.08	0.01\\
52.09	0.01\\
52.1	0.01\\
52.11	0.01\\
52.12	0.01\\
52.13	0.01\\
52.14	0.01\\
52.15	0.01\\
52.16	0.01\\
52.17	0.01\\
52.18	0.01\\
52.19	0.01\\
52.2	0.01\\
52.21	0.01\\
52.22	0.01\\
52.23	0.01\\
52.24	0.01\\
52.25	0.01\\
52.26	0.01\\
52.27	0.01\\
52.28	0.01\\
52.29	0.01\\
52.3	0.01\\
52.31	0.01\\
52.32	0.01\\
52.33	0.01\\
52.34	0.01\\
52.35	0.01\\
52.36	0.01\\
52.37	0.01\\
52.38	0.01\\
52.39	0.01\\
52.4	0.01\\
52.41	0.01\\
52.42	0.01\\
52.43	0.01\\
52.44	0.01\\
52.45	0.01\\
52.46	0.01\\
52.47	0.01\\
52.48	0.01\\
52.49	0.01\\
52.5	0.01\\
52.51	0.01\\
52.52	0.01\\
52.53	0.01\\
52.54	0.01\\
52.55	0.01\\
52.56	0.01\\
52.57	0.01\\
52.58	0.01\\
52.59	0.01\\
52.6	0.01\\
52.61	0.01\\
52.62	0.01\\
52.63	0.01\\
52.64	0.01\\
52.65	0.01\\
52.66	0.01\\
52.67	0.01\\
52.68	0.01\\
52.69	0.01\\
52.7	0.01\\
52.71	0.01\\
52.72	0.01\\
52.73	0.01\\
52.74	0.01\\
52.75	0.01\\
52.76	0.01\\
52.77	0.01\\
52.78	0.01\\
52.79	0.01\\
52.8	0.01\\
52.81	0.01\\
52.82	0.01\\
52.83	0.01\\
52.84	0.01\\
52.85	0.01\\
52.86	0.01\\
52.87	0.01\\
52.88	0.01\\
52.89	0.01\\
52.9	0.01\\
52.91	0.01\\
52.92	0.01\\
52.93	0.01\\
52.94	0.01\\
52.95	0.01\\
52.96	0.01\\
52.97	0.01\\
52.98	0.01\\
52.99	0.01\\
53	0.01\\
53.01	0.01\\
53.02	0.01\\
53.03	0.01\\
53.04	0.01\\
53.05	0.01\\
53.06	0.01\\
53.07	0.01\\
53.08	0.01\\
53.09	0.01\\
53.1	0.01\\
53.11	0.01\\
53.12	0.01\\
53.13	0.01\\
53.14	0.01\\
53.15	0.01\\
53.16	0.01\\
53.17	0.01\\
53.18	0.01\\
53.19	0.01\\
53.2	0.01\\
53.21	0.01\\
53.22	0.01\\
53.23	0.01\\
53.24	0.01\\
53.25	0.01\\
53.26	0.01\\
53.27	0.01\\
53.28	0.01\\
53.29	0.01\\
53.3	0.01\\
53.31	0.01\\
53.32	0.01\\
53.33	0.01\\
53.34	0.01\\
53.35	0.01\\
53.36	0.01\\
53.37	0.01\\
53.38	0.01\\
53.39	0.01\\
53.4	0.01\\
53.41	0.01\\
53.42	0.01\\
53.43	0.01\\
53.44	0.01\\
53.45	0.01\\
53.46	0.01\\
53.47	0.01\\
53.48	0.01\\
53.49	0.01\\
53.5	0.01\\
53.51	0.01\\
53.52	0.01\\
53.53	0.01\\
53.54	0.01\\
53.55	0.01\\
53.56	0.01\\
53.57	0.01\\
53.58	0.01\\
53.59	0.01\\
53.6	0.01\\
53.61	0.01\\
53.62	0.01\\
53.63	0.01\\
53.64	0.01\\
53.65	0.01\\
53.66	0.01\\
53.67	0.01\\
53.68	0.01\\
53.69	0.01\\
53.7	0.01\\
53.71	0.01\\
53.72	0.01\\
53.73	0.01\\
53.74	0.01\\
53.75	0.01\\
53.76	0.01\\
53.77	0.01\\
53.78	0.01\\
53.79	0.01\\
53.8	0.01\\
53.81	0.01\\
53.82	0.01\\
53.83	0.01\\
53.84	0.01\\
53.85	0.01\\
53.86	0.01\\
53.87	0.01\\
53.88	0.01\\
53.89	0.01\\
53.9	0.01\\
53.91	0.01\\
53.92	0.01\\
53.93	0.01\\
53.94	0.01\\
53.95	0.01\\
53.96	0.01\\
53.97	0.01\\
53.98	0.01\\
53.99	0.01\\
54	0.01\\
54.01	0.01\\
54.02	0.01\\
54.03	0.01\\
54.04	0.01\\
54.05	0.01\\
54.06	0.01\\
54.07	0.01\\
54.08	0.01\\
54.09	0.01\\
54.1	0.01\\
54.11	0.01\\
54.12	0.01\\
54.13	0.01\\
54.14	0.01\\
54.15	0.01\\
54.16	0.01\\
54.17	0.01\\
54.18	0.01\\
54.19	0.01\\
54.2	0.01\\
54.21	0.01\\
54.22	0.01\\
54.23	0.01\\
54.24	0.01\\
54.25	0.01\\
54.26	0.01\\
54.27	0.01\\
54.28	0.01\\
54.29	0.01\\
54.3	0.01\\
54.31	0.01\\
54.32	0.01\\
54.33	0.01\\
54.34	0.01\\
54.35	0.01\\
54.36	0.01\\
54.37	0.01\\
54.38	0.01\\
54.39	0.01\\
54.4	0.01\\
54.41	0.01\\
54.42	0.01\\
54.43	0.01\\
54.44	0.01\\
54.45	0.01\\
54.46	0.01\\
54.47	0.01\\
54.48	0.01\\
54.49	0.01\\
54.5	0.01\\
54.51	0.01\\
54.52	0.01\\
54.53	0.01\\
54.54	0.01\\
54.55	0.01\\
54.56	0.01\\
54.57	0.01\\
54.58	0.01\\
54.59	0.01\\
54.6	0.01\\
54.61	0.01\\
54.62	0.01\\
54.63	0.01\\
54.64	0.01\\
54.65	0.01\\
54.66	0.01\\
54.67	0.01\\
54.68	0.01\\
54.69	0.01\\
54.7	0.01\\
54.71	0.01\\
54.72	0.01\\
54.73	0.01\\
54.74	0.01\\
54.75	0.01\\
54.76	0.01\\
54.77	0.01\\
54.78	0.01\\
54.79	0.01\\
54.8	0.01\\
54.81	0.01\\
54.82	0.01\\
54.83	0.01\\
54.84	0.01\\
54.85	0.01\\
54.86	0.01\\
54.87	0.01\\
54.88	0.01\\
54.89	0.01\\
54.9	0.01\\
54.91	0.01\\
54.92	0.01\\
54.93	0.01\\
54.94	0.01\\
54.95	0.01\\
54.96	0.01\\
54.97	0.01\\
54.98	0.01\\
54.99	0.01\\
55	0.01\\
55.01	0.01\\
55.02	0.01\\
55.03	0.01\\
55.04	0.01\\
55.05	0.01\\
55.06	0.01\\
55.07	0.01\\
55.08	0.01\\
55.09	0.01\\
55.1	0.01\\
55.11	0.01\\
55.12	0.01\\
55.13	0.01\\
55.14	0.01\\
55.15	0.01\\
55.16	0.01\\
55.17	0.01\\
55.18	0.01\\
55.19	0.01\\
55.2	0.01\\
55.21	0.01\\
55.22	0.01\\
55.23	0.01\\
55.24	0.01\\
55.25	0.01\\
55.26	0.01\\
55.27	0.01\\
55.28	0.01\\
55.29	0.01\\
55.3	0.01\\
55.31	0.01\\
55.32	0.01\\
55.33	0.01\\
55.34	0.01\\
55.35	0.01\\
55.36	0.01\\
55.37	0.01\\
55.38	0.01\\
55.39	0.01\\
55.4	0.01\\
55.41	0.01\\
55.42	0.01\\
55.43	0.01\\
55.44	0.01\\
55.45	0.01\\
55.46	0.01\\
55.47	0.01\\
55.48	0.01\\
55.49	0.01\\
55.5	0.01\\
55.51	0.01\\
55.52	0.01\\
55.53	0.01\\
55.54	0.01\\
55.55	0.01\\
55.56	0.01\\
55.57	0.01\\
55.58	0.01\\
55.59	0.01\\
55.6	0.01\\
55.61	0.01\\
55.62	0.01\\
55.63	0.01\\
55.64	0.01\\
55.65	0.01\\
55.66	0.01\\
55.67	0.01\\
55.68	0.01\\
55.69	0.01\\
55.7	0.01\\
55.71	0.01\\
55.72	0.01\\
55.73	0.01\\
55.74	0.01\\
55.75	0.01\\
55.76	0.01\\
55.77	0.01\\
55.78	0.01\\
55.79	0.01\\
55.8	0.01\\
55.81	0.01\\
55.82	0.01\\
55.83	0.01\\
55.84	0.01\\
55.85	0.01\\
55.86	0.01\\
55.87	0.01\\
55.88	0.01\\
55.89	0.01\\
55.9	0.01\\
55.91	0.01\\
55.92	0.01\\
55.93	0.01\\
55.94	0.01\\
55.95	0.01\\
55.96	0.01\\
55.97	0.01\\
55.98	0.01\\
55.99	0.01\\
56	0.01\\
56.01	0.01\\
56.02	0.01\\
56.03	0.01\\
56.04	0.01\\
56.05	0.01\\
56.06	0.01\\
56.07	0.01\\
56.08	0.01\\
56.09	0.01\\
56.1	0.01\\
56.11	0.01\\
56.12	0.01\\
56.13	0.01\\
56.14	0.01\\
56.15	0.01\\
56.16	0.01\\
56.17	0.01\\
56.18	0.01\\
56.19	0.01\\
56.2	0.01\\
56.21	0.01\\
56.22	0.01\\
56.23	0.01\\
56.24	0.01\\
56.25	0.01\\
56.26	0.01\\
56.27	0.01\\
56.28	0.01\\
56.29	0.01\\
56.3	0.01\\
56.31	0.01\\
56.32	0.01\\
56.33	0.01\\
56.34	0.01\\
56.35	0.01\\
56.36	0.01\\
56.37	0.01\\
56.38	0.01\\
56.39	0.01\\
56.4	0.01\\
56.41	0.01\\
56.42	0.01\\
56.43	0.01\\
56.44	0.01\\
56.45	0.01\\
56.46	0.01\\
56.47	0.01\\
56.48	0.01\\
56.49	0.01\\
56.5	0.01\\
56.51	0.01\\
56.52	0.01\\
56.53	0.01\\
56.54	0.01\\
56.55	0.01\\
56.56	0.01\\
56.57	0.01\\
56.58	0.01\\
56.59	0.01\\
56.6	0.01\\
56.61	0.01\\
56.62	0.01\\
56.63	0.01\\
56.64	0.01\\
56.65	0.01\\
56.66	0.01\\
56.67	0.01\\
56.68	0.01\\
56.69	0.01\\
56.7	0.01\\
56.71	0.01\\
56.72	0.01\\
56.73	0.01\\
56.74	0.01\\
56.75	0.01\\
56.76	0.01\\
56.77	0.01\\
56.78	0.01\\
56.79	0.01\\
56.8	0.01\\
56.81	0.01\\
56.82	0.01\\
56.83	0.01\\
56.84	0.01\\
56.85	0.01\\
56.86	0.01\\
56.87	0.01\\
56.88	0.01\\
56.89	0.01\\
56.9	0.01\\
56.91	0.01\\
56.92	0.01\\
56.93	0.01\\
56.94	0.01\\
56.95	0.01\\
56.96	0.01\\
56.97	0.01\\
56.98	0.01\\
56.99	0.01\\
57	0.01\\
57.01	0.01\\
57.02	0.01\\
57.03	0.01\\
57.04	0.01\\
57.05	0.01\\
57.06	0.01\\
57.07	0.01\\
57.08	0.01\\
57.09	0.01\\
57.1	0.01\\
57.11	0.01\\
57.12	0.01\\
57.13	0.01\\
57.14	0.01\\
57.15	0.01\\
57.16	0.01\\
57.17	0.01\\
57.18	0.01\\
57.19	0.01\\
57.2	0.01\\
57.21	0.01\\
57.22	0.01\\
57.23	0.01\\
57.24	0.01\\
57.25	0.01\\
57.26	0.01\\
57.27	0.01\\
57.28	0.01\\
57.29	0.01\\
57.3	0.01\\
57.31	0.01\\
57.32	0.01\\
57.33	0.01\\
57.34	0.01\\
57.35	0.01\\
57.36	0.01\\
57.37	0.01\\
57.38	0.01\\
57.39	0.01\\
57.4	0.01\\
57.41	0.01\\
57.42	0.01\\
57.43	0.01\\
57.44	0.01\\
57.45	0.01\\
57.46	0.01\\
57.47	0.01\\
57.48	0.01\\
57.49	0.01\\
57.5	0.01\\
57.51	0.01\\
57.52	0.01\\
57.53	0.01\\
57.54	0.01\\
57.55	0.01\\
57.56	0.01\\
57.57	0.01\\
57.58	0.01\\
57.59	0.01\\
57.6	0.01\\
57.61	0.01\\
57.62	0.01\\
57.63	0.01\\
57.64	0.01\\
57.65	0.01\\
57.66	0.01\\
57.67	0.01\\
57.68	0.01\\
57.69	0.01\\
57.7	0.01\\
57.71	0.01\\
57.72	0.01\\
57.73	0.01\\
57.74	0.01\\
57.75	0.01\\
57.76	0.01\\
57.77	0.01\\
57.78	0.01\\
57.79	0.01\\
57.8	0.01\\
57.81	0.01\\
57.82	0.01\\
57.83	0.01\\
57.84	0.01\\
57.85	0.01\\
57.86	0.01\\
57.87	0.01\\
57.88	0.01\\
57.89	0.01\\
57.9	0.01\\
57.91	0.01\\
57.92	0.01\\
57.93	0.01\\
57.94	0.01\\
57.95	0.01\\
57.96	0.01\\
57.97	0.01\\
57.98	0.01\\
57.99	0.01\\
58	0.01\\
58.01	0.01\\
58.02	0.01\\
58.03	0.01\\
58.04	0.01\\
58.05	0.01\\
58.06	0.01\\
58.07	0.01\\
58.08	0.01\\
58.09	0.01\\
58.1	0.01\\
58.11	0.01\\
58.12	0.01\\
58.13	0.01\\
58.14	0.01\\
58.15	0.01\\
58.16	0.01\\
58.17	0.01\\
58.18	0.01\\
58.19	0.01\\
58.2	0.01\\
58.21	0.01\\
58.22	0.01\\
58.23	0.01\\
58.24	0.01\\
58.25	0.01\\
58.26	0.01\\
58.27	0.01\\
58.28	0.01\\
58.29	0.01\\
58.3	0.01\\
58.31	0.01\\
58.32	0.01\\
58.33	0.01\\
58.34	0.01\\
58.35	0.01\\
58.36	0.01\\
58.37	0.01\\
58.38	0.01\\
58.39	0.01\\
58.4	0.01\\
58.41	0.01\\
58.42	0.01\\
58.43	0.01\\
58.44	0.01\\
58.45	0.01\\
58.46	0.01\\
58.47	0.01\\
58.48	0.01\\
58.49	0.01\\
58.5	0.01\\
58.51	0.01\\
58.52	0.01\\
58.53	0.01\\
58.54	0.01\\
58.55	0.01\\
58.56	0.01\\
58.57	0.01\\
58.58	0.01\\
58.59	0.01\\
58.6	0.01\\
58.61	0.01\\
58.62	0.01\\
58.63	0.01\\
58.64	0.01\\
58.65	0.01\\
58.66	0.01\\
58.67	0.01\\
58.68	0.01\\
58.69	0.01\\
58.7	0.01\\
58.71	0.01\\
58.72	0.01\\
58.73	0.01\\
58.74	0.01\\
58.75	0.01\\
58.76	0.01\\
58.77	0.01\\
58.78	0.01\\
58.79	0.01\\
58.8	0.01\\
58.81	0.01\\
58.82	0.01\\
58.83	0.01\\
58.84	0.01\\
58.85	0.01\\
58.86	0.01\\
58.87	0.01\\
58.88	0.01\\
58.89	0.01\\
58.9	0.01\\
58.91	0.01\\
58.92	0.01\\
58.93	0.01\\
58.94	0.01\\
58.95	0.01\\
58.96	0.01\\
58.97	0.01\\
58.98	0.01\\
58.99	0.01\\
59	0.01\\
59.01	0.01\\
59.02	0.01\\
59.03	0.01\\
59.04	0.01\\
59.05	0.01\\
59.06	0.01\\
59.07	0.01\\
59.08	0.01\\
59.09	0.01\\
59.1	0.01\\
59.11	0.01\\
59.12	0.01\\
59.13	0.01\\
59.14	0.01\\
59.15	0.01\\
59.16	0.01\\
59.17	0.01\\
59.18	0.01\\
59.19	0.01\\
59.2	0.01\\
59.21	0.01\\
59.22	0.01\\
59.23	0.01\\
59.24	0.01\\
59.25	0.01\\
59.26	0.01\\
59.27	0.01\\
59.28	0.01\\
59.29	0.01\\
59.3	0.01\\
59.31	0.01\\
59.32	0.01\\
59.33	0.01\\
59.34	0.01\\
59.35	0.01\\
59.36	0.01\\
59.37	0.01\\
59.38	0.01\\
59.39	0.01\\
59.4	0.01\\
59.41	0.01\\
59.42	0.01\\
59.43	0.01\\
59.44	0.01\\
59.45	0.01\\
59.46	0.01\\
59.47	0.01\\
59.48	0.01\\
59.49	0.01\\
59.5	0.01\\
59.51	0.01\\
59.52	0.01\\
59.53	0.01\\
59.54	0.01\\
59.55	0.01\\
59.56	0.01\\
59.57	0.01\\
59.58	0.01\\
59.59	0.01\\
59.6	0.01\\
59.61	0.01\\
59.62	0.01\\
59.63	0.01\\
59.64	0.01\\
59.65	0.01\\
59.66	0.01\\
59.67	0.01\\
59.68	0.01\\
59.69	0.01\\
59.7	0.01\\
59.71	0.01\\
59.72	0.01\\
59.73	0.01\\
59.74	0.01\\
59.75	0.01\\
59.76	0.01\\
59.77	0.01\\
59.78	0.01\\
59.79	0.01\\
59.8	0.01\\
59.81	0.01\\
59.82	0.01\\
59.83	0.01\\
59.84	0.01\\
59.85	0.01\\
59.86	0.01\\
59.87	0.01\\
59.88	0.01\\
59.89	0.01\\
59.9	0.01\\
59.91	0.01\\
59.92	0.01\\
59.93	0.01\\
59.94	0.01\\
59.95	0.01\\
59.96	0.01\\
59.97	0.01\\
59.98	0.01\\
59.99	0.01\\
60	0.01\\
60.01	0.01\\
60.02	0.01\\
60.03	0.01\\
60.04	0.01\\
60.05	0.01\\
60.06	0.01\\
60.07	0.01\\
60.08	0.01\\
60.09	0.01\\
60.1	0.01\\
60.11	0.01\\
60.12	0.01\\
60.13	0.01\\
60.14	0.01\\
60.15	0.01\\
60.16	0.01\\
60.17	0.01\\
60.18	0.01\\
60.19	0.01\\
60.2	0.01\\
60.21	0.01\\
60.22	0.01\\
60.23	0.01\\
60.24	0.01\\
60.25	0.01\\
60.26	0.01\\
60.27	0.01\\
60.28	0.01\\
60.29	0.01\\
60.3	0.01\\
60.31	0.01\\
60.32	0.01\\
60.33	0.01\\
60.34	0.01\\
60.35	0.01\\
60.36	0.01\\
60.37	0.01\\
60.38	0.01\\
60.39	0.01\\
60.4	0.01\\
60.41	0.01\\
60.42	0.01\\
60.43	0.01\\
60.44	0.01\\
60.45	0.01\\
60.46	0.01\\
60.47	0.01\\
60.48	0.01\\
60.49	0.01\\
60.5	0.01\\
60.51	0.01\\
60.52	0.01\\
60.53	0.01\\
60.54	0.01\\
60.55	0.01\\
60.56	0.01\\
60.57	0.01\\
60.58	0.01\\
60.59	0.01\\
60.6	0.01\\
60.61	0.01\\
60.62	0.01\\
60.63	0.01\\
60.64	0.01\\
60.65	0.01\\
60.66	0.01\\
60.67	0.01\\
60.68	0.01\\
60.69	0.01\\
60.7	0.01\\
60.71	0.01\\
60.72	0.01\\
60.73	0.01\\
60.74	0.01\\
60.75	0.01\\
60.76	0.01\\
60.77	0.01\\
60.78	0.01\\
60.79	0.01\\
60.8	0.01\\
60.81	0.01\\
60.82	0.01\\
60.83	0.01\\
60.84	0.01\\
60.85	0.01\\
60.86	0.01\\
60.87	0.01\\
60.88	0.01\\
60.89	0.01\\
60.9	0.01\\
60.91	0.01\\
60.92	0.01\\
60.93	0.01\\
60.94	0.01\\
60.95	0.01\\
60.96	0.01\\
60.97	0.01\\
60.98	0.01\\
60.99	0.01\\
61	0.01\\
61.01	0.01\\
61.02	0.01\\
61.03	0.01\\
61.04	0.01\\
61.05	0.01\\
61.06	0.01\\
61.07	0.01\\
61.08	0.01\\
61.09	0.01\\
61.1	0.01\\
61.11	0.01\\
61.12	0.01\\
61.13	0.01\\
61.14	0.01\\
61.15	0.01\\
61.16	0.01\\
61.17	0.01\\
61.18	0.01\\
61.19	0.01\\
61.2	0.01\\
61.21	0.01\\
61.22	0.01\\
61.23	0.01\\
61.24	0.01\\
61.25	0.01\\
61.26	0.01\\
61.27	0.01\\
61.28	0.01\\
61.29	0.01\\
61.3	0.01\\
61.31	0.01\\
61.32	0.01\\
61.33	0.01\\
61.34	0.01\\
61.35	0.01\\
61.36	0.01\\
61.37	0.01\\
61.38	0.01\\
61.39	0.01\\
61.4	0.01\\
61.41	0.01\\
61.42	0.01\\
61.43	0.01\\
61.44	0.01\\
61.45	0.01\\
61.46	0.01\\
61.47	0.01\\
61.48	0.01\\
61.49	0.01\\
61.5	0.01\\
61.51	0.01\\
61.52	0.01\\
61.53	0.01\\
61.54	0.01\\
61.55	0.01\\
61.56	0.01\\
61.57	0.01\\
61.58	0.01\\
61.59	0.01\\
61.6	0.01\\
61.61	0.01\\
61.62	0.01\\
61.63	0.01\\
61.64	0.01\\
61.65	0.01\\
61.66	0.01\\
61.67	0.01\\
61.68	0.01\\
61.69	0.01\\
61.7	0.01\\
61.71	0.01\\
61.72	0.01\\
61.73	0.01\\
61.74	0.01\\
61.75	0.01\\
61.76	0.01\\
61.77	0.01\\
61.78	0.01\\
61.79	0.01\\
61.8	0.01\\
61.81	0.01\\
61.82	0.01\\
61.83	0.01\\
61.84	0.01\\
61.85	0.01\\
61.86	0.01\\
61.87	0.01\\
61.88	0.01\\
61.89	0.01\\
61.9	0.01\\
61.91	0.01\\
61.92	0.01\\
61.93	0.01\\
61.94	0.01\\
61.95	0.01\\
61.96	0.01\\
61.97	0.01\\
61.98	0.01\\
61.99	0.01\\
62	0.01\\
62.01	0.01\\
62.02	0.01\\
62.03	0.01\\
62.04	0.01\\
62.05	0.01\\
62.06	0.01\\
62.07	0.01\\
62.08	0.01\\
62.09	0.01\\
62.1	0.01\\
62.11	0.01\\
62.12	0.01\\
62.13	0.01\\
62.14	0.01\\
62.15	0.01\\
62.16	0.01\\
62.17	0.01\\
62.18	0.01\\
62.19	0.01\\
62.2	0.01\\
62.21	0.01\\
62.22	0.01\\
62.23	0.01\\
62.24	0.01\\
62.25	0.01\\
62.26	0.01\\
62.27	0.01\\
62.28	0.01\\
62.29	0.01\\
62.3	0.01\\
62.31	0.01\\
62.32	0.01\\
62.33	0.01\\
62.34	0.01\\
62.35	0.01\\
62.36	0.01\\
62.37	0.01\\
62.38	0.01\\
62.39	0.01\\
62.4	0.01\\
62.41	0.01\\
62.42	0.01\\
62.43	0.01\\
62.44	0.01\\
62.45	0.01\\
62.46	0.01\\
62.47	0.01\\
62.48	0.01\\
62.49	0.01\\
62.5	0.01\\
62.51	0.01\\
62.52	0.01\\
62.53	0.01\\
62.54	0.01\\
62.55	0.01\\
62.56	0.01\\
62.57	0.01\\
62.58	0.01\\
62.59	0.01\\
62.6	0.01\\
62.61	0.01\\
62.62	0.01\\
62.63	0.01\\
62.64	0.01\\
62.65	0.01\\
62.66	0.01\\
62.67	0.01\\
62.68	0.01\\
62.69	0.01\\
62.7	0.01\\
62.71	0.01\\
62.72	0.01\\
62.73	0.01\\
62.74	0.01\\
62.75	0.01\\
62.76	0.01\\
62.77	0.01\\
62.78	0.01\\
62.79	0.01\\
62.8	0.01\\
62.81	0.01\\
62.82	0.01\\
62.83	0.01\\
62.84	0.01\\
62.85	0.01\\
62.86	0.01\\
62.87	0.01\\
62.88	0.01\\
62.89	0.01\\
62.9	0.01\\
62.91	0.01\\
62.92	0.01\\
62.93	0.01\\
62.94	0.01\\
62.95	0.01\\
62.96	0.01\\
62.97	0.01\\
62.98	0.01\\
62.99	0.01\\
63	0.01\\
63.01	0.01\\
63.02	0.01\\
63.03	0.01\\
63.04	0.01\\
63.05	0.01\\
63.06	0.01\\
63.07	0.01\\
63.08	0.01\\
63.09	0.01\\
63.1	0.01\\
63.11	0.01\\
63.12	0.01\\
63.13	0.01\\
63.14	0.01\\
63.15	0.01\\
63.16	0.01\\
63.17	0.01\\
63.18	0.01\\
63.19	0.01\\
63.2	0.01\\
63.21	0.01\\
63.22	0.01\\
63.23	0.01\\
63.24	0.01\\
63.25	0.01\\
63.26	0.01\\
63.27	0.01\\
63.28	0.01\\
63.29	0.01\\
63.3	0.01\\
63.31	0.01\\
63.32	0.01\\
63.33	0.01\\
63.34	0.01\\
63.35	0.01\\
63.36	0.01\\
63.37	0.01\\
63.38	0.01\\
63.39	0.01\\
63.4	0.01\\
63.41	0.01\\
63.42	0.01\\
63.43	0.01\\
63.44	0.01\\
63.45	0.01\\
63.46	0.01\\
63.47	0.01\\
63.48	0.01\\
63.49	0.01\\
63.5	0.01\\
63.51	0.01\\
63.52	0.01\\
63.53	0.01\\
63.54	0.01\\
63.55	0.01\\
63.56	0.01\\
63.57	0.01\\
63.58	0.01\\
63.59	0.01\\
63.6	0.01\\
63.61	0.01\\
63.62	0.01\\
63.63	0.01\\
63.64	0.01\\
63.65	0.01\\
63.66	0.01\\
63.67	0.01\\
63.68	0.01\\
63.69	0.01\\
63.7	0.01\\
63.71	0.01\\
63.72	0.01\\
63.73	0.01\\
63.74	0.01\\
63.75	0.01\\
63.76	0.01\\
63.77	0.01\\
63.78	0.01\\
63.79	0.01\\
63.8	0.01\\
63.81	0.01\\
63.82	0.01\\
63.83	0.01\\
63.84	0.01\\
63.85	0.01\\
63.86	0.01\\
63.87	0.01\\
63.88	0.01\\
63.89	0.01\\
63.9	0.01\\
63.91	0.01\\
63.92	0.01\\
63.93	0.01\\
63.94	0.01\\
63.95	0.01\\
63.96	0.01\\
63.97	0.01\\
63.98	0.01\\
63.99	0.01\\
64	0.01\\
64.01	0.01\\
64.02	0.01\\
64.03	0.01\\
64.04	0.01\\
64.05	0.01\\
64.06	0.01\\
64.07	0.01\\
64.08	0.01\\
64.09	0.01\\
64.1	0.01\\
64.11	0.01\\
64.12	0.01\\
64.13	0.01\\
64.14	0.01\\
64.15	0.01\\
64.16	0.01\\
64.17	0.01\\
64.18	0.01\\
64.19	0.01\\
64.2	0.01\\
64.21	0.01\\
64.22	0.01\\
64.23	0.01\\
64.24	0.01\\
64.25	0.01\\
64.26	0.01\\
64.27	0.01\\
64.28	0.01\\
64.29	0.01\\
64.3	0.01\\
64.31	0.01\\
64.32	0.01\\
64.33	0.01\\
64.34	0.01\\
64.35	0.01\\
64.36	0.01\\
64.37	0.01\\
64.38	0.01\\
64.39	0.01\\
64.4	0.01\\
64.41	0.01\\
64.42	0.01\\
64.43	0.01\\
64.44	0.01\\
64.45	0.01\\
64.46	0.01\\
64.47	0.01\\
64.48	0.01\\
64.49	0.01\\
64.5	0.01\\
64.51	0.01\\
64.52	0.01\\
64.53	0.01\\
64.54	0.01\\
64.55	0.01\\
64.56	0.01\\
64.57	0.01\\
64.58	0.01\\
64.59	0.01\\
64.6	0.01\\
64.61	0.01\\
64.62	0.01\\
64.63	0.01\\
64.64	0.01\\
64.65	0.01\\
64.66	0.01\\
64.67	0.01\\
64.68	0.01\\
64.69	0.01\\
64.7	0.01\\
64.71	0.01\\
64.72	0.01\\
64.73	0.01\\
64.74	0.01\\
64.75	0.01\\
64.76	0.01\\
64.77	0.01\\
64.78	0.01\\
64.79	0.01\\
64.8	0.01\\
64.81	0.01\\
64.82	0.01\\
64.83	0.01\\
64.84	0.01\\
64.85	0.01\\
64.86	0.01\\
64.87	0.01\\
64.88	0.01\\
64.89	0.01\\
64.9	0.01\\
64.91	0.01\\
64.92	0.01\\
64.93	0.01\\
64.94	0.01\\
64.95	0.01\\
64.96	0.01\\
64.97	0.01\\
64.98	0.01\\
64.99	0.01\\
65	0.01\\
65.01	0.01\\
65.02	0.01\\
65.03	0.01\\
65.04	0.01\\
65.05	0.01\\
65.06	0.01\\
65.07	0.01\\
65.08	0.01\\
65.09	0.01\\
65.1	0.01\\
65.11	0.01\\
65.12	0.01\\
65.13	0.01\\
65.14	0.01\\
65.15	0.01\\
65.16	0.01\\
65.17	0.01\\
65.18	0.01\\
65.19	0.01\\
65.2	0.01\\
65.21	0.01\\
65.22	0.01\\
65.23	0.01\\
65.24	0.01\\
65.25	0.01\\
65.26	0.01\\
65.27	0.01\\
65.28	0.01\\
65.29	0.01\\
65.3	0.01\\
65.31	0.01\\
65.32	0.01\\
65.33	0.01\\
65.34	0.01\\
65.35	0.01\\
65.36	0.01\\
65.37	0.01\\
65.38	0.01\\
65.39	0.01\\
65.4	0.01\\
65.41	0.01\\
65.42	0.01\\
65.43	0.01\\
65.44	0.01\\
65.45	0.01\\
65.46	0.01\\
65.47	0.01\\
65.48	0.01\\
65.49	0.01\\
65.5	0.01\\
65.51	0.01\\
65.52	0.01\\
65.53	0.01\\
65.54	0.01\\
65.55	0.01\\
65.56	0.01\\
65.57	0.01\\
65.58	0.01\\
65.59	0.01\\
65.6	0.01\\
65.61	0.01\\
65.62	0.01\\
65.63	0.01\\
65.64	0.01\\
65.65	0.01\\
65.66	0.01\\
65.67	0.01\\
65.68	0.01\\
65.69	0.01\\
65.7	0.01\\
65.71	0.01\\
65.72	0.01\\
65.73	0.01\\
65.74	0.01\\
65.75	0.01\\
65.76	0.01\\
65.77	0.01\\
65.78	0.01\\
65.79	0.01\\
65.8	0.01\\
65.81	0.01\\
65.82	0.01\\
65.83	0.01\\
65.84	0.01\\
65.85	0.01\\
65.86	0.01\\
65.87	0.01\\
65.88	0.01\\
65.89	0.01\\
65.9	0.01\\
65.91	0.01\\
65.92	0.01\\
65.93	0.01\\
65.94	0.01\\
65.95	0.01\\
65.96	0.01\\
65.97	0.01\\
65.98	0.01\\
65.99	0.01\\
66	0.01\\
66.01	0.01\\
66.02	0.01\\
66.03	0.01\\
66.04	0.01\\
66.05	0.01\\
66.06	0.01\\
66.07	0.01\\
66.08	0.01\\
66.09	0.01\\
66.1	0.01\\
66.11	0.01\\
66.12	0.01\\
66.13	0.01\\
66.14	0.01\\
66.15	0.01\\
66.16	0.01\\
66.17	0.01\\
66.18	0.01\\
66.19	0.01\\
66.2	0.01\\
66.21	0.01\\
66.22	0.01\\
66.23	0.01\\
66.24	0.01\\
66.25	0.01\\
66.26	0.01\\
66.27	0.01\\
66.28	0.01\\
66.29	0.01\\
66.3	0.01\\
66.31	0.01\\
66.32	0.01\\
66.33	0.01\\
66.34	0.01\\
66.35	0.01\\
66.36	0.01\\
66.37	0.01\\
66.38	0.01\\
66.39	0.01\\
66.4	0.01\\
66.41	0.01\\
66.42	0.01\\
66.43	0.01\\
66.44	0.01\\
66.45	0.01\\
66.46	0.01\\
66.47	0.01\\
66.48	0.01\\
66.49	0.01\\
66.5	0.01\\
66.51	0.01\\
66.52	0.01\\
66.53	0.01\\
66.54	0.01\\
66.55	0.01\\
66.56	0.01\\
66.57	0.01\\
66.58	0.01\\
66.59	0.01\\
66.6	0.01\\
66.61	0.01\\
66.62	0.01\\
66.63	0.01\\
66.64	0.01\\
66.65	0.01\\
66.66	0.01\\
66.67	0.01\\
66.68	0.01\\
66.69	0.01\\
66.7	0.01\\
66.71	0.01\\
66.72	0.01\\
66.73	0.01\\
66.74	0.01\\
66.75	0.01\\
66.76	0.01\\
66.77	0.01\\
66.78	0.01\\
66.79	0.01\\
66.8	0.01\\
66.81	0.01\\
66.82	0.01\\
66.83	0.01\\
66.84	0.01\\
66.85	0.01\\
66.86	0.01\\
66.87	0.01\\
66.88	0.01\\
66.89	0.01\\
66.9	0.01\\
66.91	0.01\\
66.92	0.01\\
66.93	0.01\\
66.94	0.01\\
66.95	0.01\\
66.96	0.01\\
66.97	0.01\\
66.98	0.01\\
66.99	0.01\\
67	0.01\\
67.01	0.01\\
67.02	0.01\\
67.03	0.01\\
67.04	0.01\\
67.05	0.01\\
67.06	0.01\\
67.07	0.01\\
67.08	0.01\\
67.09	0.01\\
67.1	0.01\\
67.11	0.01\\
67.12	0.01\\
67.13	0.01\\
67.14	0.01\\
67.15	0.01\\
67.16	0.01\\
67.17	0.01\\
67.18	0.01\\
67.19	0.01\\
67.2	0.01\\
67.21	0.01\\
67.22	0.01\\
67.23	0.01\\
67.24	0.01\\
67.25	0.01\\
67.26	0.01\\
67.27	0.01\\
67.28	0.01\\
67.29	0.01\\
67.3	0.01\\
67.31	0.01\\
67.32	0.01\\
67.33	0.01\\
67.34	0.01\\
67.35	0.01\\
67.36	0.01\\
67.37	0.01\\
67.38	0.01\\
67.39	0.01\\
67.4	0.01\\
67.41	0.01\\
67.42	0.01\\
67.43	0.01\\
67.44	0.01\\
67.45	0.01\\
67.46	0.01\\
67.47	0.01\\
67.48	0.01\\
67.49	0.01\\
67.5	0.01\\
67.51	0.01\\
67.52	0.01\\
67.53	0.01\\
67.54	0.01\\
67.55	0.01\\
67.56	0.01\\
67.57	0.01\\
67.58	0.01\\
67.59	0.01\\
67.6	0.01\\
67.61	0.01\\
67.62	0.01\\
67.63	0.01\\
67.64	0.01\\
67.65	0.01\\
67.66	0.01\\
67.67	0.01\\
67.68	0.01\\
67.69	0.01\\
67.7	0.01\\
67.71	0.01\\
67.72	0.01\\
67.73	0.01\\
67.74	0.01\\
67.75	0.01\\
67.76	0.01\\
67.77	0.01\\
67.78	0.01\\
67.79	0.01\\
67.8	0.01\\
67.81	0.01\\
67.82	0.01\\
67.83	0.01\\
67.84	0.01\\
67.85	0.01\\
67.86	0.01\\
67.87	0.01\\
67.88	0.01\\
67.89	0.01\\
67.9	0.01\\
67.91	0.01\\
67.92	0.01\\
67.93	0.01\\
67.94	0.01\\
67.95	0.01\\
67.96	0.01\\
67.97	0.01\\
67.98	0.01\\
67.99	0.01\\
68	0.01\\
68.01	0.01\\
68.02	0.01\\
68.03	0.01\\
68.04	0.01\\
68.05	0.01\\
68.06	0.01\\
68.07	0.01\\
68.08	0.01\\
68.09	0.01\\
68.1	0.01\\
68.11	0.01\\
68.12	0.01\\
68.13	0.01\\
68.14	0.01\\
68.15	0.01\\
68.16	0.01\\
68.17	0.01\\
68.18	0.01\\
68.19	0.01\\
68.2	0.01\\
68.21	0.01\\
68.22	0.01\\
68.23	0.01\\
68.24	0.01\\
68.25	0.01\\
68.26	0.01\\
68.27	0.01\\
68.28	0.01\\
68.29	0.01\\
68.3	0.01\\
68.31	0.01\\
68.32	0.01\\
68.33	0.01\\
68.34	0.01\\
68.35	0.01\\
68.36	0.01\\
68.37	0.01\\
68.38	0.01\\
68.39	0.01\\
68.4	0.01\\
68.41	0.01\\
68.42	0.01\\
68.43	0.01\\
68.44	0.01\\
68.45	0.01\\
68.46	0.01\\
68.47	0.01\\
68.48	0.01\\
68.49	0.01\\
68.5	0.01\\
68.51	0.01\\
68.52	0.01\\
68.53	0.01\\
68.54	0.01\\
68.55	0.01\\
68.56	0.01\\
68.57	0.01\\
68.58	0.01\\
68.59	0.01\\
68.6	0.01\\
68.61	0.01\\
68.62	0.01\\
68.63	0.01\\
68.64	0.01\\
68.65	0.01\\
68.66	0.01\\
68.67	0.01\\
68.68	0.01\\
68.69	0.01\\
68.7	0.01\\
68.71	0.01\\
68.72	0.01\\
68.73	0.01\\
68.74	0.01\\
68.75	0.01\\
68.76	0.01\\
68.77	0.01\\
68.78	0.01\\
68.79	0.01\\
68.8	0.01\\
68.81	0.01\\
68.82	0.01\\
68.83	0.01\\
68.84	0.01\\
68.85	0.01\\
68.86	0.01\\
68.87	0.01\\
68.88	0.01\\
68.89	0.01\\
68.9	0.01\\
68.91	0.01\\
68.92	0.01\\
68.93	0.01\\
68.94	0.01\\
68.95	0.01\\
68.96	0.01\\
68.97	0.01\\
68.98	0.01\\
68.99	0.01\\
69	0.01\\
69.01	0.01\\
69.02	0.01\\
69.03	0.01\\
69.04	0.01\\
69.05	0.01\\
69.06	0.01\\
69.07	0.01\\
69.08	0.01\\
69.09	0.01\\
69.1	0.01\\
69.11	0.01\\
69.12	0.01\\
69.13	0.01\\
69.14	0.01\\
69.15	0.01\\
69.16	0.01\\
69.17	0.01\\
69.18	0.01\\
69.19	0.01\\
69.2	0.01\\
69.21	0.01\\
69.22	0.01\\
69.23	0.01\\
69.24	0.01\\
69.25	0.01\\
69.26	0.01\\
69.27	0.01\\
69.28	0.01\\
69.29	0.01\\
69.3	0.01\\
69.31	0.01\\
69.32	0.01\\
69.33	0.01\\
69.34	0.01\\
69.35	0.01\\
69.36	0.01\\
69.37	0.01\\
69.38	0.01\\
69.39	0.01\\
69.4	0.01\\
69.41	0.01\\
69.42	0.01\\
69.43	0.01\\
69.44	0.01\\
69.45	0.01\\
69.46	0.01\\
69.47	0.01\\
69.48	0.01\\
69.49	0.01\\
69.5	0.01\\
69.51	0.01\\
69.52	0.01\\
69.53	0.01\\
69.54	0.01\\
69.55	0.01\\
69.56	0.01\\
69.57	0.01\\
69.58	0.01\\
69.59	0.01\\
69.6	0.01\\
69.61	0.01\\
69.62	0.01\\
69.63	0.01\\
69.64	0.01\\
69.65	0.01\\
69.66	0.01\\
69.67	0.01\\
69.68	0.01\\
69.69	0.01\\
69.7	0.01\\
69.71	0.01\\
69.72	0.01\\
69.73	0.01\\
69.74	0.01\\
69.75	0.01\\
69.76	0.01\\
69.77	0.01\\
69.78	0.01\\
69.79	0.01\\
69.8	0.01\\
69.81	0.01\\
69.82	0.01\\
69.83	0.01\\
69.84	0.01\\
69.85	0.01\\
69.86	0.01\\
69.87	0.01\\
69.88	0.01\\
69.89	0.01\\
69.9	0.01\\
69.91	0.01\\
69.92	0.01\\
69.93	0.01\\
69.94	0.01\\
69.95	0.01\\
69.96	0.01\\
69.97	0.01\\
69.98	0.01\\
69.99	0.01\\
70	0.01\\
70.01	0.01\\
70.02	0.01\\
70.03	0.01\\
70.04	0.01\\
70.05	0.01\\
70.06	0.01\\
70.07	0.01\\
70.08	0.01\\
70.09	0.01\\
70.1	0.01\\
70.11	0.01\\
70.12	0.01\\
70.13	0.01\\
70.14	0.01\\
70.15	0.01\\
70.16	0.01\\
70.17	0.01\\
70.18	0.01\\
70.19	0.01\\
70.2	0.01\\
70.21	0.01\\
70.22	0.01\\
70.23	0.01\\
70.24	0.01\\
70.25	0.01\\
70.26	0.01\\
70.27	0.01\\
70.28	0.01\\
70.29	0.01\\
70.3	0.01\\
70.31	0.01\\
70.32	0.01\\
70.33	0.01\\
70.34	0.01\\
70.35	0.01\\
70.36	0.01\\
70.37	0.01\\
70.38	0.01\\
70.39	0.01\\
70.4	0.01\\
70.41	0.01\\
70.42	0.01\\
70.43	0.01\\
70.44	0.01\\
70.45	0.01\\
70.46	0.01\\
70.47	0.01\\
70.48	0.01\\
70.49	0.01\\
70.5	0.01\\
70.51	0.01\\
70.52	0.01\\
70.53	0.01\\
70.54	0.01\\
70.55	0.01\\
70.56	0.01\\
70.57	0.01\\
70.58	0.01\\
70.59	0.01\\
70.6	0.01\\
70.61	0.01\\
70.62	0.01\\
70.63	0.01\\
70.64	0.01\\
70.65	0.01\\
70.66	0.01\\
70.67	0.01\\
70.68	0.01\\
70.69	0.01\\
70.7	0.01\\
70.71	0.01\\
70.72	0.01\\
70.73	0.01\\
70.74	0.01\\
70.75	0.01\\
70.76	0.01\\
70.77	0.01\\
70.78	0.01\\
70.79	0.01\\
70.8	0.01\\
70.81	0.01\\
70.82	0.01\\
70.83	0.01\\
70.84	0.01\\
70.85	0.01\\
70.86	0.01\\
70.87	0.01\\
70.88	0.01\\
70.89	0.01\\
70.9	0.01\\
70.91	0.01\\
70.92	0.01\\
70.93	0.01\\
70.94	0.01\\
70.95	0.01\\
70.96	0.01\\
70.97	0.01\\
70.98	0.01\\
70.99	0.01\\
71	0.01\\
71.01	0.01\\
71.02	0.01\\
71.03	0.01\\
71.04	0.01\\
71.05	0.01\\
71.06	0.01\\
71.07	0.01\\
71.08	0.01\\
71.09	0.01\\
71.1	0.01\\
71.11	0.01\\
71.12	0.01\\
71.13	0.01\\
71.14	0.01\\
71.15	0.01\\
71.16	0.01\\
71.17	0.01\\
71.18	0.01\\
71.19	0.01\\
71.2	0.01\\
71.21	0.01\\
71.22	0.01\\
71.23	0.01\\
71.24	0.01\\
71.25	0.01\\
71.26	0.01\\
71.27	0.01\\
71.28	0.01\\
71.29	0.01\\
71.3	0.01\\
71.31	0.01\\
71.32	0.01\\
71.33	0.01\\
71.34	0.01\\
71.35	0.01\\
71.36	0.01\\
71.37	0.01\\
71.38	0.01\\
71.39	0.01\\
71.4	0.01\\
71.41	0.01\\
71.42	0.01\\
71.43	0.01\\
71.44	0.01\\
71.45	0.01\\
71.46	0.01\\
71.47	0.01\\
71.48	0.01\\
71.49	0.01\\
71.5	0.01\\
71.51	0.01\\
71.52	0.01\\
71.53	0.01\\
71.54	0.01\\
71.55	0.01\\
71.56	0.01\\
71.57	0.01\\
71.58	0.01\\
71.59	0.01\\
71.6	0.01\\
71.61	0.01\\
71.62	0.01\\
71.63	0.01\\
71.64	0.01\\
71.65	0.01\\
71.66	0.01\\
71.67	0.01\\
71.68	0.01\\
71.69	0.01\\
71.7	0.01\\
71.71	0.01\\
71.72	0.01\\
71.73	0.01\\
71.74	0.01\\
71.75	0.01\\
71.76	0.01\\
71.77	0.01\\
71.78	0.01\\
71.79	0.01\\
71.8	0.01\\
71.81	0.01\\
71.82	0.01\\
71.83	0.01\\
71.84	0.01\\
71.85	0.01\\
71.86	0.01\\
71.87	0.01\\
71.88	0.01\\
71.89	0.01\\
71.9	0.01\\
71.91	0.01\\
71.92	0.01\\
71.93	0.01\\
71.94	0.01\\
71.95	0.01\\
71.96	0.01\\
71.97	0.01\\
71.98	0.01\\
71.99	0.01\\
72	0.01\\
72.01	0.01\\
72.02	0.01\\
72.03	0.01\\
72.04	0.01\\
72.05	0.01\\
72.06	0.01\\
72.07	0.01\\
72.08	0.01\\
72.09	0.01\\
72.1	0.01\\
72.11	0.01\\
72.12	0.01\\
72.13	0.01\\
72.14	0.01\\
72.15	0.01\\
72.16	0.01\\
72.17	0.01\\
72.18	0.01\\
72.19	0.01\\
72.2	0.01\\
72.21	0.01\\
72.22	0.01\\
72.23	0.01\\
72.24	0.01\\
72.25	0.01\\
72.26	0.01\\
72.27	0.01\\
72.28	0.01\\
72.29	0.01\\
72.3	0.01\\
72.31	0.01\\
72.32	0.01\\
72.33	0.01\\
72.34	0.01\\
72.35	0.01\\
72.36	0.01\\
72.37	0.01\\
72.38	0.01\\
72.39	0.01\\
72.4	0.01\\
72.41	0.01\\
72.42	0.01\\
72.43	0.01\\
72.44	0.01\\
72.45	0.01\\
72.46	0.01\\
72.47	0.01\\
72.48	0.01\\
72.49	0.01\\
72.5	0.01\\
72.51	0.01\\
72.52	0.01\\
72.53	0.01\\
72.54	0.01\\
72.55	0.01\\
72.56	0.01\\
72.57	0.01\\
72.58	0.01\\
72.59	0.01\\
72.6	0.01\\
72.61	0.01\\
72.62	0.01\\
72.63	0.01\\
72.64	0.01\\
72.65	0.01\\
72.66	0.01\\
72.67	0.01\\
72.68	0.01\\
72.69	0.01\\
72.7	0.01\\
72.71	0.01\\
72.72	0.01\\
72.73	0.01\\
72.74	0.01\\
72.75	0.01\\
72.76	0.01\\
72.77	0.01\\
72.78	0.01\\
72.79	0.01\\
72.8	0.01\\
72.81	0.01\\
72.82	0.01\\
72.83	0.01\\
72.84	0.01\\
72.85	0.01\\
72.86	0.01\\
72.87	0.01\\
72.88	0.01\\
72.89	0.01\\
72.9	0.01\\
72.91	0.01\\
72.92	0.01\\
72.93	0.01\\
72.94	0.01\\
72.95	0.01\\
72.96	0.01\\
72.97	0.01\\
72.98	0.01\\
72.99	0.01\\
73	0.01\\
73.01	0.01\\
73.02	0.01\\
73.03	0.01\\
73.04	0.01\\
73.05	0.01\\
73.06	0.01\\
73.07	0.01\\
73.08	0.01\\
73.09	0.01\\
73.1	0.01\\
73.11	0.01\\
73.12	0.01\\
73.13	0.01\\
73.14	0.01\\
73.15	0.01\\
73.16	0.01\\
73.17	0.01\\
73.18	0.01\\
73.19	0.01\\
73.2	0.01\\
73.21	0.01\\
73.22	0.01\\
73.23	0.01\\
73.24	0.01\\
73.25	0.01\\
73.26	0.01\\
73.27	0.01\\
73.28	0.01\\
73.29	0.01\\
73.3	0.01\\
73.31	0.01\\
73.32	0.01\\
73.33	0.01\\
73.34	0.01\\
73.35	0.01\\
73.36	0.01\\
73.37	0.01\\
73.38	0.01\\
73.39	0.01\\
73.4	0.01\\
73.41	0.01\\
73.42	0.01\\
73.43	0.01\\
73.44	0.01\\
73.45	0.01\\
73.46	0.01\\
73.47	0.01\\
73.48	0.01\\
73.49	0.01\\
73.5	0.01\\
73.51	0.01\\
73.52	0.01\\
73.53	0.01\\
73.54	0.01\\
73.55	0.01\\
73.56	0.01\\
73.57	0.01\\
73.58	0.01\\
73.59	0.01\\
73.6	0.01\\
73.61	0.01\\
73.62	0.01\\
73.63	0.01\\
73.64	0.01\\
73.65	0.01\\
73.66	0.01\\
73.67	0.01\\
73.68	0.01\\
73.69	0.01\\
73.7	0.01\\
73.71	0.01\\
73.72	0.01\\
73.73	0.01\\
73.74	0.01\\
73.75	0.01\\
73.76	0.01\\
73.77	0.01\\
73.78	0.01\\
73.79	0.01\\
73.8	0.01\\
73.81	0.01\\
73.82	0.01\\
73.83	0.01\\
73.84	0.01\\
73.85	0.01\\
73.86	0.01\\
73.87	0.01\\
73.88	0.01\\
73.89	0.01\\
73.9	0.01\\
73.91	0.01\\
73.92	0.01\\
73.93	0.01\\
73.94	0.01\\
73.95	0.01\\
73.96	0.01\\
73.97	0.01\\
73.98	0.01\\
73.99	0.01\\
74	0.01\\
74.01	0.01\\
74.02	0.01\\
74.03	0.01\\
74.04	0.01\\
74.05	0.01\\
74.06	0.01\\
74.07	0.01\\
74.08	0.01\\
74.09	0.01\\
74.1	0.01\\
74.11	0.01\\
74.12	0.01\\
74.13	0.01\\
74.14	0.01\\
74.15	0.01\\
74.16	0.01\\
74.17	0.01\\
74.18	0.01\\
74.19	0.01\\
74.2	0.01\\
74.21	0.01\\
74.22	0.01\\
74.23	0.01\\
74.24	0.01\\
74.25	0.01\\
74.26	0.01\\
74.27	0.01\\
74.28	0.01\\
74.29	0.01\\
74.3	0.01\\
74.31	0.01\\
74.32	0.01\\
74.33	0.01\\
74.34	0.01\\
74.35	0.01\\
74.36	0.01\\
74.37	0.01\\
74.38	0.01\\
74.39	0.01\\
74.4	0.01\\
74.41	0.01\\
74.42	0.01\\
74.43	0.01\\
74.44	0.01\\
74.45	0.01\\
74.46	0.01\\
74.47	0.01\\
74.48	0.01\\
74.49	0.01\\
74.5	0.01\\
74.51	0.01\\
74.52	0.01\\
74.53	0.01\\
74.54	0.01\\
74.55	0.01\\
74.56	0.01\\
74.57	0.01\\
74.58	0.01\\
74.59	0.01\\
74.6	0.01\\
74.61	0.01\\
74.62	0.01\\
74.63	0.01\\
74.64	0.01\\
74.65	0.01\\
74.66	0.01\\
74.67	0.01\\
74.68	0.01\\
74.69	0.01\\
74.7	0.01\\
74.71	0.01\\
74.72	0.01\\
74.73	0.01\\
74.74	0.01\\
74.75	0.01\\
74.76	0.01\\
74.77	0.01\\
74.78	0.01\\
74.79	0.01\\
74.8	0.01\\
74.81	0.01\\
74.82	0.01\\
74.83	0.01\\
74.84	0.01\\
74.85	0.01\\
74.86	0.01\\
74.87	0.01\\
74.88	0.01\\
74.89	0.01\\
74.9	0.01\\
74.91	0.01\\
74.92	0.01\\
74.93	0.01\\
74.94	0.01\\
74.95	0.01\\
74.96	0.01\\
74.97	0.01\\
74.98	0.01\\
74.99	0.01\\
75	0.01\\
75.01	0.01\\
75.02	0.01\\
75.03	0.01\\
75.04	0.01\\
75.05	0.01\\
75.06	0.01\\
75.07	0.01\\
75.08	0.01\\
75.09	0.01\\
75.1	0.01\\
75.11	0.01\\
75.12	0.01\\
75.13	0.01\\
75.14	0.01\\
75.15	0.01\\
75.16	0.01\\
75.17	0.01\\
75.18	0.01\\
75.19	0.01\\
75.2	0.01\\
75.21	0.01\\
75.22	0.01\\
75.23	0.01\\
75.24	0.01\\
75.25	0.01\\
75.26	0.01\\
75.27	0.01\\
75.28	0.01\\
75.29	0.01\\
75.3	0.01\\
75.31	0.01\\
75.32	0.01\\
75.33	0.01\\
75.34	0.01\\
75.35	0.01\\
75.36	0.01\\
75.37	0.01\\
75.38	0.01\\
75.39	0.01\\
75.4	0.01\\
75.41	0.01\\
75.42	0.01\\
75.43	0.01\\
75.44	0.01\\
75.45	0.01\\
75.46	0.01\\
75.47	0.01\\
75.48	0.01\\
75.49	0.01\\
75.5	0.01\\
75.51	0.01\\
75.52	0.01\\
75.53	0.01\\
75.54	0.01\\
75.55	0.01\\
75.56	0.01\\
75.57	0.01\\
75.58	0.01\\
75.59	0.01\\
75.6	0.01\\
75.61	0.01\\
75.62	0.01\\
75.63	0.01\\
75.64	0.01\\
75.65	0.01\\
75.66	0.01\\
75.67	0.01\\
75.68	0.01\\
75.69	0.01\\
75.7	0.01\\
75.71	0.01\\
75.72	0.01\\
75.73	0.01\\
75.74	0.01\\
75.75	0.01\\
75.76	0.01\\
75.77	0.01\\
75.78	0.01\\
75.79	0.01\\
75.8	0.01\\
75.81	0.01\\
75.82	0.01\\
75.83	0.01\\
75.84	0.01\\
75.85	0.01\\
75.86	0.01\\
75.87	0.01\\
75.88	0.01\\
75.89	0.01\\
75.9	0.01\\
75.91	0.01\\
75.92	0.01\\
75.93	0.01\\
75.94	0.01\\
75.95	0.01\\
75.96	0.01\\
75.97	0.01\\
75.98	0.01\\
75.99	0.01\\
76	0.01\\
76.01	0.01\\
76.02	0.01\\
76.03	0.01\\
76.04	0.01\\
76.05	0.01\\
76.06	0.01\\
76.07	0.01\\
76.08	0.01\\
76.09	0.01\\
76.1	0.01\\
76.11	0.01\\
76.12	0.01\\
76.13	0.01\\
76.14	0.01\\
76.15	0.01\\
76.16	0.01\\
76.17	0.01\\
76.18	0.01\\
76.19	0.01\\
76.2	0.01\\
76.21	0.01\\
76.22	0.01\\
76.23	0.01\\
76.24	0.01\\
76.25	0.01\\
76.26	0.01\\
76.27	0.01\\
76.28	0.01\\
76.29	0.01\\
76.3	0.01\\
76.31	0.01\\
76.32	0.01\\
76.33	0.01\\
76.34	0.01\\
76.35	0.01\\
76.36	0.01\\
76.37	0.01\\
76.38	0.01\\
76.39	0.01\\
76.4	0.01\\
76.41	0.01\\
76.42	0.01\\
76.43	0.01\\
76.44	0.01\\
76.45	0.01\\
76.46	0.01\\
76.47	0.01\\
76.48	0.01\\
76.49	0.01\\
76.5	0.01\\
76.51	0.01\\
76.52	0.01\\
76.53	0.01\\
76.54	0.01\\
76.55	0.01\\
76.56	0.01\\
76.57	0.01\\
76.58	0.01\\
76.59	0.01\\
76.6	0.01\\
76.61	0.01\\
76.62	0.01\\
76.63	0.01\\
76.64	0.01\\
76.65	0.01\\
76.66	0.01\\
76.67	0.01\\
76.68	0.01\\
76.69	0.01\\
76.7	0.01\\
76.71	0.01\\
76.72	0.01\\
76.73	0.01\\
76.74	0.01\\
76.75	0.01\\
76.76	0.01\\
76.77	0.01\\
76.78	0.01\\
76.79	0.01\\
76.8	0.01\\
76.81	0.01\\
76.82	0.01\\
76.83	0.01\\
76.84	0.01\\
76.85	0.01\\
76.86	0.01\\
76.87	0.01\\
76.88	0.01\\
76.89	0.01\\
76.9	0.01\\
76.91	0.01\\
76.92	0.01\\
76.93	0.01\\
76.94	0.01\\
76.95	0.01\\
76.96	0.01\\
76.97	0.01\\
76.98	0.01\\
76.99	0.01\\
77	0.01\\
77.01	0.01\\
77.02	0.01\\
77.03	0.01\\
77.04	0.01\\
77.05	0.01\\
77.06	0.01\\
77.07	0.01\\
77.08	0.01\\
77.09	0.01\\
77.1	0.01\\
77.11	0.01\\
77.12	0.01\\
77.13	0.01\\
77.14	0.01\\
77.15	0.01\\
77.16	0.01\\
77.17	0.01\\
77.18	0.01\\
77.19	0.01\\
77.2	0.01\\
77.21	0.01\\
77.22	0.01\\
77.23	0.01\\
77.24	0.01\\
77.25	0.01\\
77.26	0.01\\
77.27	0.01\\
77.28	0.01\\
77.29	0.01\\
77.3	0.01\\
77.31	0.01\\
77.32	0.01\\
77.33	0.01\\
77.34	0.01\\
77.35	0.01\\
77.36	0.01\\
77.37	0.01\\
77.38	0.01\\
77.39	0.01\\
77.4	0.01\\
77.41	0.01\\
77.42	0.01\\
77.43	0.01\\
77.44	0.01\\
77.45	0.01\\
77.46	0.01\\
77.47	0.01\\
77.48	0.01\\
77.49	0.01\\
77.5	0.01\\
77.51	0.01\\
77.52	0.01\\
77.53	0.01\\
77.54	0.01\\
77.55	0.01\\
77.56	0.01\\
77.57	0.01\\
77.58	0.01\\
77.59	0.01\\
77.6	0.01\\
77.61	0.01\\
77.62	0.01\\
77.63	0.01\\
77.64	0.01\\
77.65	0.01\\
77.66	0.01\\
77.67	0.01\\
77.68	0.01\\
77.69	0.01\\
77.7	0.01\\
77.71	0.01\\
77.72	0.01\\
77.73	0.01\\
77.74	0.01\\
77.75	0.01\\
77.76	0.01\\
77.77	0.01\\
77.78	0.01\\
77.79	0.01\\
77.8	0.01\\
77.81	0.01\\
77.82	0.01\\
77.83	0.01\\
77.84	0.01\\
77.85	0.01\\
77.86	0.01\\
77.87	0.01\\
77.88	0.01\\
77.89	0.01\\
77.9	0.01\\
77.91	0.01\\
77.92	0.01\\
77.93	0.01\\
77.94	0.01\\
77.95	0.01\\
77.96	0.01\\
77.97	0.01\\
77.98	0.01\\
77.99	0.01\\
78	0.01\\
78.01	0.01\\
78.02	0.01\\
78.03	0.01\\
78.04	0.01\\
78.05	0.01\\
78.06	0.01\\
78.07	0.01\\
78.08	0.01\\
78.09	0.01\\
78.1	0.01\\
78.11	0.01\\
78.12	0.01\\
78.13	0.01\\
78.14	0.01\\
78.15	0.01\\
78.16	0.01\\
78.17	0.01\\
78.18	0.01\\
78.19	0.01\\
78.2	0.01\\
78.21	0.01\\
78.22	0.01\\
78.23	0.01\\
78.24	0.01\\
78.25	0.01\\
78.26	0.01\\
78.27	0.01\\
78.28	0.01\\
78.29	0.01\\
78.3	0.01\\
78.31	0.01\\
78.32	0.01\\
78.33	0.01\\
78.34	0.01\\
78.35	0.01\\
78.36	0.01\\
78.37	0.01\\
78.38	0.01\\
78.39	0.01\\
78.4	0.01\\
78.41	0.01\\
78.42	0.01\\
78.43	0.01\\
78.44	0.01\\
78.45	0.01\\
78.46	0.01\\
78.47	0.01\\
78.48	0.01\\
78.49	0.01\\
78.5	0.01\\
78.51	0.01\\
78.52	0.01\\
78.53	0.01\\
78.54	0.01\\
78.55	0.01\\
78.56	0.01\\
78.57	0.01\\
78.58	0.01\\
78.59	0.01\\
78.6	0.01\\
78.61	0.01\\
78.62	0.01\\
78.63	0.01\\
78.64	0.01\\
78.65	0.01\\
78.66	0.01\\
78.67	0.01\\
78.68	0.01\\
78.69	0.01\\
78.7	0.01\\
78.71	0.01\\
78.72	0.01\\
78.73	0.01\\
78.74	0.01\\
78.75	0.01\\
78.76	0.01\\
78.77	0.01\\
78.78	0.01\\
78.79	0.01\\
78.8	0.01\\
78.81	0.01\\
78.82	0.01\\
78.83	0.01\\
78.84	0.01\\
78.85	0.01\\
78.86	0.01\\
78.87	0.01\\
78.88	0.01\\
78.89	0.01\\
78.9	0.01\\
78.91	0.01\\
78.92	0.01\\
78.93	0.01\\
78.94	0.01\\
78.95	0.01\\
78.96	0.01\\
78.97	0.01\\
78.98	0.01\\
78.99	0.01\\
79	0.01\\
79.01	0.01\\
79.02	0.01\\
79.03	0.01\\
79.04	0.01\\
79.05	0.01\\
79.06	0.01\\
79.07	0.01\\
79.08	0.01\\
79.09	0.01\\
79.1	0.01\\
79.11	0.01\\
79.12	0.01\\
79.13	0.01\\
79.14	0.01\\
79.15	0.01\\
79.16	0.01\\
79.17	0.01\\
79.18	0.01\\
79.19	0.01\\
79.2	0.01\\
79.21	0.01\\
79.22	0.01\\
79.23	0.01\\
79.24	0.01\\
79.25	0.01\\
79.26	0.01\\
79.27	0.01\\
79.28	0.01\\
79.29	0.01\\
79.3	0.01\\
79.31	0.01\\
79.32	0.01\\
79.33	0.01\\
79.34	0.01\\
79.35	0.01\\
79.36	0.01\\
79.37	0.01\\
79.38	0.01\\
79.39	0.01\\
79.4	0.01\\
79.41	0.01\\
79.42	0.01\\
79.43	0.01\\
79.44	0.01\\
79.45	0.01\\
79.46	0.01\\
79.47	0.01\\
79.48	0.01\\
79.49	0.01\\
79.5	0.01\\
79.51	0.01\\
79.52	0.01\\
79.53	0.01\\
79.54	0.01\\
79.55	0.01\\
79.56	0.01\\
79.57	0.01\\
79.58	0.01\\
79.59	0.01\\
79.6	0.01\\
79.61	0.01\\
79.62	0.01\\
79.63	0.01\\
79.64	0.01\\
79.65	0.01\\
79.66	0.01\\
79.67	0.01\\
79.68	0.01\\
79.69	0.01\\
79.7	0.01\\
79.71	0.01\\
79.72	0.01\\
79.73	0.01\\
79.74	0.01\\
79.75	0.01\\
79.76	0.01\\
79.77	0.01\\
79.78	0.01\\
79.79	0.01\\
79.8	0.01\\
79.81	0.01\\
79.82	0.01\\
79.83	0.01\\
79.84	0.01\\
79.85	0.01\\
79.86	0.01\\
79.87	0.01\\
79.88	0.01\\
79.89	0.01\\
79.9	0.01\\
79.91	0.01\\
79.92	0.01\\
79.93	0.01\\
79.94	0.01\\
79.95	0.01\\
79.96	0.01\\
79.97	0.01\\
79.98	0.01\\
79.99	0.01\\
80	0.01\\
80.01	0.01\\
};
\addplot [color=mycolor1,solid]
  table[row sep=crcr]{%
80.01	0.01\\
80.02	0.01\\
80.03	0.01\\
80.04	0.01\\
80.05	0.01\\
80.06	0.01\\
80.07	0.01\\
80.08	0.01\\
80.09	0.01\\
80.1	0.01\\
80.11	0.01\\
80.12	0.01\\
80.13	0.01\\
80.14	0.01\\
80.15	0.01\\
80.16	0.01\\
80.17	0.01\\
80.18	0.01\\
80.19	0.01\\
80.2	0.01\\
80.21	0.01\\
80.22	0.01\\
80.23	0.01\\
80.24	0.01\\
80.25	0.01\\
80.26	0.01\\
80.27	0.01\\
80.28	0.01\\
80.29	0.01\\
80.3	0.01\\
80.31	0.01\\
80.32	0.01\\
80.33	0.01\\
80.34	0.01\\
80.35	0.01\\
80.36	0.01\\
80.37	0.01\\
80.38	0.01\\
80.39	0.01\\
80.4	0.01\\
80.41	0.01\\
80.42	0.01\\
80.43	0.01\\
80.44	0.01\\
80.45	0.01\\
80.46	0.01\\
80.47	0.01\\
80.48	0.01\\
80.49	0.01\\
80.5	0.01\\
80.51	0.01\\
80.52	0.01\\
80.53	0.01\\
80.54	0.01\\
80.55	0.01\\
80.56	0.01\\
80.57	0.01\\
80.58	0.01\\
80.59	0.01\\
80.6	0.01\\
80.61	0.01\\
80.62	0.01\\
80.63	0.01\\
80.64	0.01\\
80.65	0.01\\
80.66	0.01\\
80.67	0.01\\
80.68	0.01\\
80.69	0.01\\
80.7	0.01\\
80.71	0.01\\
80.72	0.01\\
80.73	0.01\\
80.74	0.01\\
80.75	0.01\\
80.76	0.01\\
80.77	0.01\\
80.78	0.01\\
80.79	0.01\\
80.8	0.01\\
80.81	0.01\\
80.82	0.01\\
80.83	0.01\\
80.84	0.01\\
80.85	0.01\\
80.86	0.01\\
80.87	0.01\\
80.88	0.01\\
80.89	0.01\\
80.9	0.01\\
80.91	0.01\\
80.92	0.01\\
80.93	0.01\\
80.94	0.01\\
80.95	0.01\\
80.96	0.01\\
80.97	0.01\\
80.98	0.01\\
80.99	0.01\\
81	0.01\\
81.01	0.01\\
81.02	0.01\\
81.03	0.01\\
81.04	0.01\\
81.05	0.01\\
81.06	0.01\\
81.07	0.01\\
81.08	0.01\\
81.09	0.01\\
81.1	0.01\\
81.11	0.01\\
81.12	0.01\\
81.13	0.01\\
81.14	0.01\\
81.15	0.01\\
81.16	0.01\\
81.17	0.01\\
81.18	0.01\\
81.19	0.01\\
81.2	0.01\\
81.21	0.01\\
81.22	0.01\\
81.23	0.01\\
81.24	0.01\\
81.25	0.01\\
81.26	0.01\\
81.27	0.01\\
81.28	0.01\\
81.29	0.01\\
81.3	0.01\\
81.31	0.01\\
81.32	0.01\\
81.33	0.01\\
81.34	0.01\\
81.35	0.01\\
81.36	0.01\\
81.37	0.01\\
81.38	0.01\\
81.39	0.01\\
81.4	0.01\\
81.41	0.01\\
81.42	0.01\\
81.43	0.01\\
81.44	0.01\\
81.45	0.01\\
81.46	0.01\\
81.47	0.01\\
81.48	0.01\\
81.49	0.01\\
81.5	0.01\\
81.51	0.01\\
81.52	0.01\\
81.53	0.01\\
81.54	0.01\\
81.55	0.01\\
81.56	0.01\\
81.57	0.01\\
81.58	0.01\\
81.59	0.01\\
81.6	0.01\\
81.61	0.01\\
81.62	0.01\\
81.63	0.01\\
81.64	0.01\\
81.65	0.01\\
81.66	0.01\\
81.67	0.01\\
81.68	0.01\\
81.69	0.01\\
81.7	0.01\\
81.71	0.01\\
81.72	0.01\\
81.73	0.01\\
81.74	0.01\\
81.75	0.01\\
81.76	0.01\\
81.77	0.01\\
81.78	0.01\\
81.79	0.01\\
81.8	0.01\\
81.81	0.01\\
81.82	0.01\\
81.83	0.01\\
81.84	0.01\\
81.85	0.01\\
81.86	0.01\\
81.87	0.01\\
81.88	0.01\\
81.89	0.01\\
81.9	0.01\\
81.91	0.01\\
81.92	0.01\\
81.93	0.01\\
81.94	0.01\\
81.95	0.01\\
81.96	0.01\\
81.97	0.01\\
81.98	0.01\\
81.99	0.01\\
82	0.01\\
82.01	0.01\\
82.02	0.01\\
82.03	0.01\\
82.04	0.01\\
82.05	0.01\\
82.06	0.01\\
82.07	0.01\\
82.08	0.01\\
82.09	0.01\\
82.1	0.01\\
82.11	0.01\\
82.12	0.01\\
82.13	0.01\\
82.14	0.01\\
82.15	0.01\\
82.16	0.01\\
82.17	0.01\\
82.18	0.01\\
82.19	0.01\\
82.2	0.01\\
82.21	0.01\\
82.22	0.01\\
82.23	0.01\\
82.24	0.01\\
82.25	0.01\\
82.26	0.01\\
82.27	0.01\\
82.28	0.01\\
82.29	0.01\\
82.3	0.01\\
82.31	0.01\\
82.32	0.01\\
82.33	0.01\\
82.34	0.01\\
82.35	0.01\\
82.36	0.01\\
82.37	0.01\\
82.38	0.01\\
82.39	0.01\\
82.4	0.01\\
82.41	0.01\\
82.42	0.01\\
82.43	0.01\\
82.44	0.01\\
82.45	0.01\\
82.46	0.01\\
82.47	0.01\\
82.48	0.01\\
82.49	0.01\\
82.5	0.01\\
82.51	0.01\\
82.52	0.01\\
82.53	0.01\\
82.54	0.01\\
82.55	0.01\\
82.56	0.01\\
82.57	0.01\\
82.58	0.01\\
82.59	0.01\\
82.6	0.01\\
82.61	0.01\\
82.62	0.01\\
82.63	0.01\\
82.64	0.01\\
82.65	0.01\\
82.66	0.01\\
82.67	0.01\\
82.68	0.01\\
82.69	0.01\\
82.7	0.01\\
82.71	0.01\\
82.72	0.01\\
82.73	0.01\\
82.74	0.01\\
82.75	0.01\\
82.76	0.01\\
82.77	0.01\\
82.78	0.01\\
82.79	0.01\\
82.8	0.01\\
82.81	0.01\\
82.82	0.01\\
82.83	0.01\\
82.84	0.01\\
82.85	0.01\\
82.86	0.01\\
82.87	0.01\\
82.88	0.01\\
82.89	0.01\\
82.9	0.01\\
82.91	0.01\\
82.92	0.01\\
82.93	0.01\\
82.94	0.01\\
82.95	0.01\\
82.96	0.01\\
82.97	0.01\\
82.98	0.01\\
82.99	0.01\\
83	0.01\\
83.01	0.01\\
83.02	0.01\\
83.03	0.01\\
83.04	0.01\\
83.05	0.01\\
83.06	0.01\\
83.07	0.01\\
83.08	0.01\\
83.09	0.01\\
83.1	0.01\\
83.11	0.01\\
83.12	0.01\\
83.13	0.01\\
83.14	0.01\\
83.15	0.01\\
83.16	0.01\\
83.17	0.01\\
83.18	0.01\\
83.19	0.01\\
83.2	0.01\\
83.21	0.01\\
83.22	0.01\\
83.23	0.01\\
83.24	0.01\\
83.25	0.01\\
83.26	0.01\\
83.27	0.01\\
83.28	0.01\\
83.29	0.01\\
83.3	0.01\\
83.31	0.01\\
83.32	0.01\\
83.33	0.01\\
83.34	0.01\\
83.35	0.01\\
83.36	0.01\\
83.37	0.01\\
83.38	0.01\\
83.39	0.01\\
83.4	0.01\\
83.41	0.01\\
83.42	0.01\\
83.43	0.01\\
83.44	0.01\\
83.45	0.01\\
83.46	0.01\\
83.47	0.01\\
83.48	0.01\\
83.49	0.01\\
83.5	0.01\\
83.51	0.01\\
83.52	0.01\\
83.53	0.01\\
83.54	0.01\\
83.55	0.01\\
83.56	0.01\\
83.57	0.01\\
83.58	0.01\\
83.59	0.01\\
83.6	0.01\\
83.61	0.01\\
83.62	0.01\\
83.63	0.01\\
83.64	0.01\\
83.65	0.01\\
83.66	0.01\\
83.67	0.01\\
83.68	0.01\\
83.69	0.01\\
83.7	0.01\\
83.71	0.01\\
83.72	0.01\\
83.73	0.01\\
83.74	0.01\\
83.75	0.01\\
83.76	0.01\\
83.77	0.01\\
83.78	0.01\\
83.79	0.01\\
83.8	0.01\\
83.81	0.01\\
83.82	0.01\\
83.83	0.01\\
83.84	0.01\\
83.85	0.01\\
83.86	0.01\\
83.87	0.01\\
83.88	0.01\\
83.89	0.01\\
83.9	0.01\\
83.91	0.01\\
83.92	0.01\\
83.93	0.01\\
83.94	0.01\\
83.95	0.01\\
83.96	0.01\\
83.97	0.01\\
83.98	0.01\\
83.99	0.01\\
84	0.01\\
84.01	0.01\\
84.02	0.01\\
84.03	0.01\\
84.04	0.01\\
84.05	0.01\\
84.06	0.01\\
84.07	0.01\\
84.08	0.01\\
84.09	0.01\\
84.1	0.01\\
84.11	0.01\\
84.12	0.01\\
84.13	0.01\\
84.14	0.01\\
84.15	0.01\\
84.16	0.01\\
84.17	0.01\\
84.18	0.01\\
84.19	0.01\\
84.2	0.01\\
84.21	0.01\\
84.22	0.01\\
84.23	0.01\\
84.24	0.01\\
84.25	0.01\\
84.26	0.01\\
84.27	0.01\\
84.28	0.01\\
84.29	0.01\\
84.3	0.01\\
84.31	0.01\\
84.32	0.01\\
84.33	0.01\\
84.34	0.01\\
84.35	0.01\\
84.36	0.01\\
84.37	0.01\\
84.38	0.01\\
84.39	0.01\\
84.4	0.01\\
84.41	0.01\\
84.42	0.01\\
84.43	0.01\\
84.44	0.01\\
84.45	0.01\\
84.46	0.01\\
84.47	0.01\\
84.48	0.01\\
84.49	0.01\\
84.5	0.01\\
84.51	0.01\\
84.52	0.01\\
84.53	0.01\\
84.54	0.01\\
84.55	0.01\\
84.56	0.01\\
84.57	0.01\\
84.58	0.01\\
84.59	0.01\\
84.6	0.01\\
84.61	0.01\\
84.62	0.01\\
84.63	0.01\\
84.64	0.01\\
84.65	0.01\\
84.66	0.01\\
84.67	0.01\\
84.68	0.01\\
84.69	0.01\\
84.7	0.01\\
84.71	0.01\\
84.72	0.01\\
84.73	0.01\\
84.74	0.01\\
84.75	0.01\\
84.76	0.01\\
84.77	0.01\\
84.78	0.01\\
84.79	0.01\\
84.8	0.01\\
84.81	0.01\\
84.82	0.01\\
84.83	0.01\\
84.84	0.01\\
84.85	0.01\\
84.86	0.01\\
84.87	0.01\\
84.88	0.01\\
84.89	0.01\\
84.9	0.01\\
84.91	0.01\\
84.92	0.01\\
84.93	0.01\\
84.94	0.01\\
84.95	0.01\\
84.96	0.01\\
84.97	0.01\\
84.98	0.01\\
84.99	0.01\\
85	0.01\\
85.01	0.01\\
85.02	0.01\\
85.03	0.01\\
85.04	0.01\\
85.05	0.01\\
85.06	0.01\\
85.07	0.01\\
85.08	0.01\\
85.09	0.01\\
85.1	0.01\\
85.11	0.01\\
85.12	0.01\\
85.13	0.01\\
85.14	0.01\\
85.15	0.01\\
85.16	0.01\\
85.17	0.01\\
85.18	0.01\\
85.19	0.01\\
85.2	0.01\\
85.21	0.01\\
85.22	0.01\\
85.23	0.01\\
85.24	0.01\\
85.25	0.01\\
85.26	0.01\\
85.27	0.01\\
85.28	0.01\\
85.29	0.01\\
85.3	0.01\\
85.31	0.01\\
85.32	0.01\\
85.33	0.01\\
85.34	0.01\\
85.35	0.01\\
85.36	0.01\\
85.37	0.01\\
85.38	0.01\\
85.39	0.01\\
85.4	0.01\\
85.41	0.01\\
85.42	0.01\\
85.43	0.01\\
85.44	0.01\\
85.45	0.01\\
85.46	0.01\\
85.47	0.01\\
85.48	0.01\\
85.49	0.01\\
85.5	0.01\\
85.51	0.01\\
85.52	0.01\\
85.53	0.01\\
85.54	0.01\\
85.55	0.01\\
85.56	0.01\\
85.57	0.01\\
85.58	0.01\\
85.59	0.01\\
85.6	0.01\\
85.61	0.01\\
85.62	0.01\\
85.63	0.01\\
85.64	0.01\\
85.65	0.01\\
85.66	0.01\\
85.67	0.01\\
85.68	0.01\\
85.69	0.01\\
85.7	0.01\\
85.71	0.01\\
85.72	0.01\\
85.73	0.01\\
85.74	0.01\\
85.75	0.01\\
85.76	0.01\\
85.77	0.01\\
85.78	0.01\\
85.79	0.01\\
85.8	0.01\\
85.81	0.01\\
85.82	0.01\\
85.83	0.01\\
85.84	0.01\\
85.85	0.01\\
85.86	0.01\\
85.87	0.01\\
85.88	0.01\\
85.89	0.01\\
85.9	0.01\\
85.91	0.01\\
85.92	0.01\\
85.93	0.01\\
85.94	0.01\\
85.95	0.01\\
85.96	0.01\\
85.97	0.01\\
85.98	0.01\\
85.99	0.01\\
86	0.01\\
86.01	0.01\\
86.02	0.01\\
86.03	0.01\\
86.04	0.01\\
86.05	0.01\\
86.06	0.01\\
86.07	0.01\\
86.08	0.01\\
86.09	0.01\\
86.1	0.01\\
86.11	0.01\\
86.12	0.01\\
86.13	0.01\\
86.14	0.01\\
86.15	0.01\\
86.16	0.01\\
86.17	0.01\\
86.18	0.01\\
86.19	0.01\\
86.2	0.01\\
86.21	0.01\\
86.22	0.01\\
86.23	0.01\\
86.24	0.01\\
86.25	0.01\\
86.26	0.01\\
86.27	0.01\\
86.28	0.01\\
86.29	0.01\\
86.3	0.01\\
86.31	0.01\\
86.32	0.01\\
86.33	0.01\\
86.34	0.01\\
86.35	0.01\\
86.36	0.01\\
86.37	0.01\\
86.38	0.01\\
86.39	0.01\\
86.4	0.01\\
86.41	0.01\\
86.42	0.01\\
86.43	0.01\\
86.44	0.01\\
86.45	0.01\\
86.46	0.01\\
86.47	0.01\\
86.48	0.01\\
86.49	0.01\\
86.5	0.01\\
86.51	0.01\\
86.52	0.01\\
86.53	0.01\\
86.54	0.01\\
86.55	0.01\\
86.56	0.01\\
86.57	0.01\\
86.58	0.01\\
86.59	0.01\\
86.6	0.01\\
86.61	0.01\\
86.62	0.01\\
86.63	0.01\\
86.64	0.01\\
86.65	0.01\\
86.66	0.01\\
86.67	0.01\\
86.68	0.01\\
86.69	0.01\\
86.7	0.01\\
86.71	0.01\\
86.72	0.01\\
86.73	0.01\\
86.74	0.01\\
86.75	0.01\\
86.76	0.01\\
86.77	0.01\\
86.78	0.01\\
86.79	0.01\\
86.8	0.01\\
86.81	0.01\\
86.82	0.01\\
86.83	0.01\\
86.84	0.01\\
86.85	0.01\\
86.86	0.01\\
86.87	0.01\\
86.88	0.01\\
86.89	0.01\\
86.9	0.01\\
86.91	0.01\\
86.92	0.01\\
86.93	0.01\\
86.94	0.01\\
86.95	0.01\\
86.96	0.01\\
86.97	0.01\\
86.98	0.01\\
86.99	0.01\\
87	0.01\\
87.01	0.01\\
87.02	0.01\\
87.03	0.01\\
87.04	0.01\\
87.05	0.01\\
87.06	0.01\\
87.07	0.01\\
87.08	0.01\\
87.09	0.01\\
87.1	0.01\\
87.11	0.01\\
87.12	0.01\\
87.13	0.01\\
87.14	0.01\\
87.15	0.01\\
87.16	0.01\\
87.17	0.01\\
87.18	0.01\\
87.19	0.01\\
87.2	0.01\\
87.21	0.01\\
87.22	0.01\\
87.23	0.01\\
87.24	0.01\\
87.25	0.01\\
87.26	0.01\\
87.27	0.01\\
87.28	0.01\\
87.29	0.01\\
87.3	0.01\\
87.31	0.01\\
87.32	0.01\\
87.33	0.01\\
87.34	0.01\\
87.35	0.01\\
87.36	0.01\\
87.37	0.01\\
87.38	0.01\\
87.39	0.01\\
87.4	0.01\\
87.41	0.01\\
87.42	0.01\\
87.43	0.01\\
87.44	0.01\\
87.45	0.01\\
87.46	0.01\\
87.47	0.01\\
87.48	0.01\\
87.49	0.01\\
87.5	0.01\\
87.51	0.01\\
87.52	0.01\\
87.53	0.01\\
87.54	0.01\\
87.55	0.01\\
87.56	0.01\\
87.57	0.01\\
87.58	0.01\\
87.59	0.01\\
87.6	0.01\\
87.61	0.01\\
87.62	0.01\\
87.63	0.01\\
87.64	0.01\\
87.65	0.01\\
87.66	0.01\\
87.67	0.01\\
87.68	0.01\\
87.69	0.01\\
87.7	0.01\\
87.71	0.01\\
87.72	0.01\\
87.73	0.01\\
87.74	0.01\\
87.75	0.01\\
87.76	0.01\\
87.77	0.01\\
87.78	0.01\\
87.79	0.01\\
87.8	0.01\\
87.81	0.01\\
87.82	0.01\\
87.83	0.01\\
87.84	0.01\\
87.85	0.01\\
87.86	0.01\\
87.87	0.01\\
87.88	0.01\\
87.89	0.01\\
87.9	0.01\\
87.91	0.01\\
87.92	0.01\\
87.93	0.01\\
87.94	0.01\\
87.95	0.01\\
87.96	0.01\\
87.97	0.01\\
87.98	0.01\\
87.99	0.01\\
88	0.01\\
88.01	0.01\\
88.02	0.01\\
88.03	0.01\\
88.04	0.01\\
88.05	0.01\\
88.06	0.01\\
88.07	0.01\\
88.08	0.01\\
88.09	0.01\\
88.1	0.01\\
88.11	0.01\\
88.12	0.01\\
88.13	0.01\\
88.14	0.01\\
88.15	0.01\\
88.16	0.01\\
88.17	0.01\\
88.18	0.01\\
88.19	0.01\\
88.2	0.01\\
88.21	0.01\\
88.22	0.01\\
88.23	0.01\\
88.24	0.01\\
88.25	0.01\\
88.26	0.01\\
88.27	0.01\\
88.28	0.01\\
88.29	0.01\\
88.3	0.01\\
88.31	0.01\\
88.32	0.01\\
88.33	0.01\\
88.34	0.01\\
88.35	0.01\\
88.36	0.01\\
88.37	0.01\\
88.38	0.01\\
88.39	0.01\\
88.4	0.01\\
88.41	0.01\\
88.42	0.01\\
88.43	0.01\\
88.44	0.01\\
88.45	0.01\\
88.46	0.01\\
88.47	0.01\\
88.48	0.01\\
88.49	0.01\\
88.5	0.01\\
88.51	0.01\\
88.52	0.01\\
88.53	0.01\\
88.54	0.01\\
88.55	0.01\\
88.56	0.01\\
88.57	0.01\\
88.58	0.01\\
88.59	0.01\\
88.6	0.01\\
88.61	0.01\\
88.62	0.01\\
88.63	0.01\\
88.64	0.01\\
88.65	0.01\\
88.66	0.01\\
88.67	0.01\\
88.68	0.01\\
88.69	0.01\\
88.7	0.01\\
88.71	0.01\\
88.72	0.01\\
88.73	0.01\\
88.74	0.01\\
88.75	0.01\\
88.76	0.01\\
88.77	0.01\\
88.78	0.01\\
88.79	0.01\\
88.8	0.01\\
88.81	0.01\\
88.82	0.01\\
88.83	0.01\\
88.84	0.01\\
88.85	0.01\\
88.86	0.01\\
88.87	0.01\\
88.88	0.01\\
88.89	0.01\\
88.9	0.01\\
88.91	0.01\\
88.92	0.01\\
88.93	0.01\\
88.94	0.01\\
88.95	0.01\\
88.96	0.01\\
88.97	0.01\\
88.98	0.01\\
88.99	0.01\\
89	0.01\\
89.01	0.01\\
89.02	0.01\\
89.03	0.01\\
89.04	0.01\\
89.05	0.01\\
89.06	0.01\\
89.07	0.01\\
89.08	0.01\\
89.09	0.01\\
89.1	0.01\\
89.11	0.01\\
89.12	0.01\\
89.13	0.01\\
89.14	0.01\\
89.15	0.01\\
89.16	0.01\\
89.17	0.01\\
89.18	0.01\\
89.19	0.01\\
89.2	0.01\\
89.21	0.01\\
89.22	0.01\\
89.23	0.01\\
89.24	0.01\\
89.25	0.01\\
89.26	0.01\\
89.27	0.01\\
89.28	0.01\\
89.29	0.01\\
89.3	0.01\\
89.31	0.01\\
89.32	0.01\\
89.33	0.01\\
89.34	0.01\\
89.35	0.01\\
89.36	0.01\\
89.37	0.01\\
89.38	0.01\\
89.39	0.01\\
89.4	0.01\\
89.41	0.01\\
89.42	0.01\\
89.43	0.01\\
89.44	0.01\\
89.45	0.01\\
89.46	0.01\\
89.47	0.01\\
89.48	0.01\\
89.49	0.01\\
89.5	0.01\\
89.51	0.01\\
89.52	0.01\\
89.53	0.01\\
89.54	0.01\\
89.55	0.01\\
89.56	0.01\\
89.57	0.01\\
89.58	0.01\\
89.59	0.01\\
89.6	0.01\\
89.61	0.01\\
89.62	0.01\\
89.63	0.01\\
89.64	0.01\\
89.65	0.01\\
89.66	0.01\\
89.67	0.01\\
89.68	0.01\\
89.69	0.01\\
89.7	0.01\\
89.71	0.01\\
89.72	0.01\\
89.73	0.01\\
89.74	0.01\\
89.75	0.01\\
89.76	0.01\\
89.77	0.01\\
89.78	0.01\\
89.79	0.01\\
89.8	0.01\\
89.81	0.01\\
89.82	0.01\\
89.83	0.01\\
89.84	0.01\\
89.85	0.01\\
89.86	0.01\\
89.87	0.01\\
89.88	0.01\\
89.89	0.01\\
89.9	0.01\\
89.91	0.01\\
89.92	0.01\\
89.93	0.01\\
89.94	0.01\\
89.95	0.01\\
89.96	0.01\\
89.97	0.01\\
89.98	0.01\\
89.99	0.01\\
90	0.01\\
90.01	0.01\\
90.02	0.01\\
90.03	0.01\\
90.04	0.01\\
90.05	0.01\\
90.06	0.01\\
90.07	0.01\\
90.08	0.01\\
90.09	0.01\\
90.1	0.01\\
90.11	0.01\\
90.12	0.01\\
90.13	0.01\\
90.14	0.01\\
90.15	0.01\\
90.16	0.01\\
90.17	0.01\\
90.18	0.01\\
90.19	0.01\\
90.2	0.01\\
90.21	0.01\\
90.22	0.01\\
90.23	0.01\\
90.24	0.01\\
90.25	0.01\\
90.26	0.01\\
90.27	0.01\\
90.28	0.01\\
90.29	0.01\\
90.3	0.01\\
90.31	0.01\\
90.32	0.01\\
90.33	0.01\\
90.34	0.01\\
90.35	0.01\\
90.36	0.01\\
90.37	0.01\\
90.38	0.01\\
90.39	0.01\\
90.4	0.01\\
90.41	0.01\\
90.42	0.01\\
90.43	0.01\\
90.44	0.01\\
90.45	0.01\\
90.46	0.01\\
90.47	0.01\\
90.48	0.01\\
90.49	0.01\\
90.5	0.01\\
90.51	0.01\\
90.52	0.01\\
90.53	0.01\\
90.54	0.01\\
90.55	0.01\\
90.56	0.01\\
90.57	0.01\\
90.58	0.01\\
90.59	0.01\\
90.6	0.01\\
90.61	0.01\\
90.62	0.01\\
90.63	0.01\\
90.64	0.01\\
90.65	0.01\\
90.66	0.01\\
90.67	0.01\\
90.68	0.01\\
90.69	0.01\\
90.7	0.01\\
90.71	0.01\\
90.72	0.01\\
90.73	0.01\\
90.74	0.01\\
90.75	0.01\\
90.76	0.01\\
90.77	0.01\\
90.78	0.01\\
90.79	0.01\\
90.8	0.01\\
90.81	0.01\\
90.82	0.01\\
90.83	0.01\\
90.84	0.01\\
90.85	0.01\\
90.86	0.01\\
90.87	0.01\\
90.88	0.01\\
90.89	0.01\\
90.9	0.01\\
90.91	0.01\\
90.92	0.01\\
90.93	0.01\\
90.94	0.01\\
90.95	0.01\\
90.96	0.01\\
90.97	0.01\\
90.98	0.01\\
90.99	0.01\\
91	0.01\\
91.01	0.01\\
91.02	0.01\\
91.03	0.01\\
91.04	0.01\\
91.05	0.01\\
91.06	0.01\\
91.07	0.01\\
91.08	0.01\\
91.09	0.01\\
91.1	0.01\\
91.11	0.01\\
91.12	0.01\\
91.13	0.01\\
91.14	0.01\\
91.15	0.01\\
91.16	0.01\\
91.17	0.01\\
91.18	0.01\\
91.19	0.01\\
91.2	0.01\\
91.21	0.01\\
91.22	0.01\\
91.23	0.01\\
91.24	0.01\\
91.25	0.01\\
91.26	0.01\\
91.27	0.01\\
91.28	0.01\\
91.29	0.01\\
91.3	0.01\\
91.31	0.01\\
91.32	0.01\\
91.33	0.01\\
91.34	0.01\\
91.35	0.01\\
91.36	0.01\\
91.37	0.01\\
91.38	0.01\\
91.39	0.01\\
91.4	0.01\\
91.41	0.01\\
91.42	0.01\\
91.43	0.01\\
91.44	0.01\\
91.45	0.01\\
91.46	0.01\\
91.47	0.01\\
91.48	0.01\\
91.49	0.01\\
91.5	0.01\\
91.51	0.01\\
91.52	0.01\\
91.53	0.01\\
91.54	0.01\\
91.55	0.01\\
91.56	0.01\\
91.57	0.01\\
91.58	0.01\\
91.59	0.01\\
91.6	0.01\\
91.61	0.01\\
91.62	0.01\\
91.63	0.01\\
91.64	0.01\\
91.65	0.01\\
91.66	0.01\\
91.67	0.01\\
91.68	0.01\\
91.69	0.01\\
91.7	0.01\\
91.71	0.01\\
91.72	0.01\\
91.73	0.01\\
91.74	0.01\\
91.75	0.01\\
91.76	0.01\\
91.77	0.01\\
91.78	0.01\\
91.79	0.01\\
91.8	0.01\\
91.81	0.01\\
91.82	0.01\\
91.83	0.01\\
91.84	0.01\\
91.85	0.01\\
91.86	0.01\\
91.87	0.01\\
91.88	0.01\\
91.89	0.01\\
91.9	0.01\\
91.91	0.01\\
91.92	0.01\\
91.93	0.01\\
91.94	0.01\\
91.95	0.01\\
91.96	0.01\\
91.97	0.01\\
91.98	0.01\\
91.99	0.01\\
92	0.01\\
92.01	0.01\\
92.02	0.01\\
92.03	0.01\\
92.04	0.01\\
92.05	0.01\\
92.06	0.01\\
92.07	0.01\\
92.08	0.01\\
92.09	0.01\\
92.1	0.01\\
92.11	0.01\\
92.12	0.01\\
92.13	0.01\\
92.14	0.01\\
92.15	0.01\\
92.16	0.01\\
92.17	0.01\\
92.18	0.01\\
92.19	0.01\\
92.2	0.01\\
92.21	0.01\\
92.22	0.01\\
92.23	0.01\\
92.24	0.01\\
92.25	0.01\\
92.26	0.01\\
92.27	0.01\\
92.28	0.01\\
92.29	0.01\\
92.3	0.01\\
92.31	0.01\\
92.32	0.01\\
92.33	0.01\\
92.34	0.01\\
92.35	0.01\\
92.36	0.01\\
92.37	0.01\\
92.38	0.01\\
92.39	0.01\\
92.4	0.01\\
92.41	0.01\\
92.42	0.01\\
92.43	0.01\\
92.44	0.01\\
92.45	0.01\\
92.46	0.01\\
92.47	0.01\\
92.48	0.01\\
92.49	0.01\\
92.5	0.01\\
92.51	0.01\\
92.52	0.01\\
92.53	0.01\\
92.54	0.01\\
92.55	0.01\\
92.56	0.01\\
92.57	0.01\\
92.58	0.01\\
92.59	0.01\\
92.6	0.01\\
92.61	0.01\\
92.62	0.01\\
92.63	0.01\\
92.64	0.01\\
92.65	0.01\\
92.66	0.01\\
92.67	0.01\\
92.68	0.01\\
92.69	0.01\\
92.7	0.01\\
92.71	0.01\\
92.72	0.01\\
92.73	0.01\\
92.74	0.01\\
92.75	0.01\\
92.76	0.01\\
92.77	0.01\\
92.78	0.01\\
92.79	0.01\\
92.8	0.01\\
92.81	0.01\\
92.82	0.01\\
92.83	0.01\\
92.84	0.01\\
92.85	0.01\\
92.86	0.01\\
92.87	0.01\\
92.88	0.01\\
92.89	0.01\\
92.9	0.01\\
92.91	0.01\\
92.92	0.01\\
92.93	0.01\\
92.94	0.01\\
92.95	0.01\\
92.96	0.01\\
92.97	0.01\\
92.98	0.01\\
92.99	0.01\\
93	0.01\\
93.01	0.01\\
93.02	0.01\\
93.03	0.01\\
93.04	0.01\\
93.05	0.01\\
93.06	0.01\\
93.07	0.01\\
93.08	0.01\\
93.09	0.01\\
93.1	0.01\\
93.11	0.01\\
93.12	0.01\\
93.13	0.01\\
93.14	0.01\\
93.15	0.01\\
93.16	0.01\\
93.17	0.01\\
93.18	0.01\\
93.19	0.01\\
93.2	0.01\\
93.21	0.01\\
93.22	0.01\\
93.23	0.01\\
93.24	0.01\\
93.25	0.01\\
93.26	0.01\\
93.27	0.01\\
93.28	0.01\\
93.29	0.01\\
93.3	0.01\\
93.31	0.01\\
93.32	0.01\\
93.33	0.01\\
93.34	0.01\\
93.35	0.01\\
93.36	0.01\\
93.37	0.01\\
93.38	0.01\\
93.39	0.01\\
93.4	0.01\\
93.41	0.01\\
93.42	0.01\\
93.43	0.01\\
93.44	0.01\\
93.45	0.01\\
93.46	0.01\\
93.47	0.01\\
93.48	0.01\\
93.49	0.01\\
93.5	0.01\\
93.51	0.01\\
93.52	0.01\\
93.53	0.01\\
93.54	0.01\\
93.55	0.01\\
93.56	0.01\\
93.57	0.01\\
93.58	0.01\\
93.59	0.01\\
93.6	0.01\\
93.61	0.01\\
93.62	0.01\\
93.63	0.01\\
93.64	0.01\\
93.65	0.01\\
93.66	0.01\\
93.67	0.01\\
93.68	0.01\\
93.69	0.01\\
93.7	0.01\\
93.71	0.01\\
93.72	0.01\\
93.73	0.01\\
93.74	0.01\\
93.75	0.01\\
93.76	0.01\\
93.77	0.01\\
93.78	0.01\\
93.79	0.01\\
93.8	0.01\\
93.81	0.01\\
93.82	0.01\\
93.83	0.01\\
93.84	0.01\\
93.85	0.01\\
93.86	0.01\\
93.87	0.01\\
93.88	0.01\\
93.89	0.01\\
93.9	0.01\\
93.91	0.01\\
93.92	0.01\\
93.93	0.01\\
93.94	0.01\\
93.95	0.01\\
93.96	0.01\\
93.97	0.01\\
93.98	0.01\\
93.99	0.01\\
94	0.01\\
94.01	0.01\\
94.02	0.01\\
94.03	0.01\\
94.04	0.01\\
94.05	0.01\\
94.06	0.01\\
94.07	0.01\\
94.08	0.01\\
94.09	0.01\\
94.1	0.01\\
94.11	0.01\\
94.12	0.01\\
94.13	0.01\\
94.14	0.01\\
94.15	0.01\\
94.16	0.01\\
94.17	0.01\\
94.18	0.01\\
94.19	0.01\\
94.2	0.01\\
94.21	0.01\\
94.22	0.01\\
94.23	0.01\\
94.24	0.01\\
94.25	0.01\\
94.26	0.01\\
94.27	0.01\\
94.28	0.01\\
94.29	0.01\\
94.3	0.01\\
94.31	0.01\\
94.32	0.01\\
94.33	0.01\\
94.34	0.01\\
94.35	0.01\\
94.36	0.01\\
94.37	0.01\\
94.38	0.01\\
94.39	0.01\\
94.4	0.01\\
94.41	0.01\\
94.42	0.01\\
94.43	0.01\\
94.44	0.01\\
94.45	0.01\\
94.46	0.01\\
94.47	0.01\\
94.48	0.01\\
94.49	0.01\\
94.5	0.01\\
94.51	0.01\\
94.52	0.01\\
94.53	0.01\\
94.54	0.01\\
94.55	0.01\\
94.56	0.01\\
94.57	0.01\\
94.58	0.01\\
94.59	0.01\\
94.6	0.01\\
94.61	0.01\\
94.62	0.01\\
94.63	0.01\\
94.64	0.01\\
94.65	0.01\\
94.66	0.01\\
94.67	0.01\\
94.68	0.01\\
94.69	0.01\\
94.7	0.01\\
94.71	0.01\\
94.72	0.01\\
94.73	0.01\\
94.74	0.01\\
94.75	0.01\\
94.76	0.01\\
94.77	0.01\\
94.78	0.01\\
94.79	0.01\\
94.8	0.01\\
94.81	0.01\\
94.82	0.01\\
94.83	0.01\\
94.84	0.01\\
94.85	0.01\\
94.86	0.01\\
94.87	0.01\\
94.88	0.01\\
94.89	0.01\\
94.9	0.01\\
94.91	0.01\\
94.92	0.01\\
94.93	0.01\\
94.94	0.01\\
94.95	0.01\\
94.96	0.01\\
94.97	0.01\\
94.98	0.01\\
94.99	0.01\\
95	0.01\\
95.01	0.01\\
95.02	0.01\\
95.03	0.01\\
95.04	0.01\\
95.05	0.01\\
95.06	0.01\\
95.07	0.01\\
95.08	0.01\\
95.09	0.01\\
95.1	0.01\\
95.11	0.01\\
95.12	0.01\\
95.13	0.01\\
95.14	0.01\\
95.15	0.01\\
95.16	0.01\\
95.17	0.01\\
95.18	0.01\\
95.19	0.01\\
95.2	0.01\\
95.21	0.01\\
95.22	0.01\\
95.23	0.01\\
95.24	0.01\\
95.25	0.01\\
95.26	0.01\\
95.27	0.01\\
95.28	0.01\\
95.29	0.01\\
95.3	0.01\\
95.31	0.01\\
95.32	0.01\\
95.33	0.01\\
95.34	0.01\\
95.35	0.01\\
95.36	0.01\\
95.37	0.01\\
95.38	0.01\\
95.39	0.01\\
95.4	0.01\\
95.41	0.01\\
95.42	0.01\\
95.43	0.01\\
95.44	0.01\\
95.45	0.01\\
95.46	0.01\\
95.47	0.01\\
95.48	0.01\\
95.49	0.01\\
95.5	0.01\\
95.51	0.01\\
95.52	0.01\\
95.53	0.01\\
95.54	0.01\\
95.55	0.01\\
95.56	0.01\\
95.57	0.01\\
95.58	0.01\\
95.59	0.01\\
95.6	0.01\\
95.61	0.01\\
95.62	0.01\\
95.63	0.01\\
95.64	0.01\\
95.65	0.01\\
95.66	0.01\\
95.67	0.01\\
95.68	0.01\\
95.69	0.01\\
95.7	0.01\\
95.71	0.01\\
95.72	0.01\\
95.73	0.01\\
95.74	0.01\\
95.75	0.01\\
95.76	0.01\\
95.77	0.01\\
95.78	0.01\\
95.79	0.01\\
95.8	0.01\\
95.81	0.01\\
95.82	0.01\\
95.83	0.01\\
95.84	0.01\\
95.85	0.01\\
95.86	0.01\\
95.87	0.01\\
95.88	0.01\\
95.89	0.01\\
95.9	0.01\\
95.91	0.01\\
95.92	0.01\\
95.93	0.01\\
95.94	0.01\\
95.95	0.01\\
95.96	0.01\\
95.97	0.01\\
95.98	0.01\\
95.99	0.01\\
96	0.01\\
96.01	0.01\\
96.02	0.01\\
96.03	0.01\\
96.04	0.01\\
96.05	0.01\\
96.06	0.01\\
96.07	0.01\\
96.08	0.01\\
96.09	0.01\\
96.1	0.01\\
96.11	0.01\\
96.12	0.01\\
96.13	0.01\\
96.14	0.01\\
96.15	0.01\\
96.16	0.01\\
96.17	0.01\\
96.18	0.01\\
96.19	0.01\\
96.2	0.01\\
96.21	0.01\\
96.22	0.01\\
96.23	0.01\\
96.24	0.01\\
96.25	0.01\\
96.26	0.01\\
96.27	0.01\\
96.28	0.01\\
96.29	0.01\\
96.3	0.01\\
96.31	0.01\\
96.32	0.01\\
96.33	0.01\\
96.34	0.01\\
96.35	0.01\\
96.36	0.01\\
96.37	0.01\\
96.38	0.01\\
96.39	0.01\\
96.4	0.01\\
96.41	0.01\\
96.42	0.01\\
96.43	0.01\\
96.44	0.01\\
96.45	0.01\\
96.46	0.01\\
96.47	0.01\\
96.48	0.01\\
96.49	0.01\\
96.5	0.01\\
96.51	0.01\\
96.52	0.01\\
96.53	0.01\\
96.54	0.01\\
96.55	0.01\\
96.56	0.01\\
96.57	0.01\\
96.58	0.01\\
96.59	0.01\\
96.6	0.01\\
96.61	0.01\\
96.62	0.01\\
96.63	0.01\\
96.64	0.01\\
96.65	0.01\\
96.66	0.01\\
96.67	0.01\\
96.68	0.01\\
96.69	0.01\\
96.7	0.01\\
96.71	0.01\\
96.72	0.01\\
96.73	0.01\\
96.74	0.01\\
96.75	0.01\\
96.76	0.01\\
96.77	0.01\\
96.78	0.01\\
96.79	0.01\\
96.8	0.01\\
96.81	0.01\\
96.82	0.01\\
96.83	0.01\\
96.84	0.01\\
96.85	0.01\\
96.86	0.01\\
96.87	0.01\\
96.88	0.01\\
96.89	0.01\\
96.9	0.01\\
96.91	0.01\\
96.92	0.01\\
96.93	0.01\\
96.94	0.01\\
96.95	0.01\\
96.96	0.01\\
96.97	0.01\\
96.98	0.01\\
96.99	0.01\\
97	0.01\\
97.01	0.01\\
97.02	0.01\\
97.03	0.01\\
97.04	0.01\\
97.05	0.01\\
97.06	0.01\\
97.07	0.01\\
97.08	0.01\\
97.09	0.01\\
97.1	0.01\\
97.11	0.01\\
97.12	0.01\\
97.13	0.01\\
97.14	0.01\\
97.15	0.01\\
97.16	0.01\\
97.17	0.01\\
97.18	0.01\\
97.19	0.01\\
97.2	0.01\\
97.21	0.01\\
97.22	0.01\\
97.23	0.01\\
97.24	0.01\\
97.25	0.01\\
97.26	0.01\\
97.27	0.01\\
97.28	0.01\\
97.29	0.01\\
97.3	0.01\\
97.31	0.01\\
97.32	0.01\\
97.33	0.01\\
97.34	0.01\\
97.35	0.01\\
97.36	0.01\\
97.37	0.01\\
97.38	0.01\\
97.39	0.01\\
97.4	0.01\\
97.41	0.01\\
97.42	0.01\\
97.43	0.01\\
97.44	0.01\\
97.45	0.01\\
97.46	0.01\\
97.47	0.01\\
97.48	0.01\\
97.49	0.01\\
97.5	0.01\\
97.51	0.01\\
97.52	0.01\\
97.53	0.01\\
97.54	0.01\\
97.55	0.01\\
97.56	0.01\\
97.57	0.01\\
97.58	0.01\\
97.59	0.01\\
97.6	0.01\\
97.61	0.01\\
97.62	0.01\\
97.63	0.01\\
97.64	0.01\\
97.65	0.01\\
97.66	0.01\\
97.67	0.01\\
97.68	0.01\\
97.69	0.01\\
97.7	0.01\\
97.71	0.01\\
97.72	0.01\\
97.73	0.01\\
97.74	0.01\\
97.75	0.01\\
97.76	0.01\\
97.77	0.01\\
97.78	0.01\\
97.79	0.01\\
97.8	0.01\\
97.81	0.01\\
97.82	0.01\\
97.83	0.01\\
97.84	0.01\\
97.85	0.01\\
97.86	0.01\\
97.87	0.01\\
97.88	0.01\\
97.89	0.01\\
97.9	0.01\\
97.91	0.01\\
97.92	0.01\\
97.93	0.01\\
97.94	0.01\\
97.95	0.01\\
97.96	0.01\\
97.97	0.01\\
97.98	0.01\\
97.99	0.01\\
98	0.01\\
98.01	0.01\\
98.02	0.01\\
98.03	0.01\\
98.04	0.01\\
98.05	0.01\\
98.06	0.01\\
98.07	0.01\\
98.08	0.01\\
98.09	0.01\\
98.1	0.01\\
98.11	0.01\\
98.12	0.01\\
98.13	0.01\\
98.14	0.01\\
98.15	0.01\\
98.16	0.01\\
98.17	0.01\\
98.18	0.01\\
98.19	0.01\\
98.2	0.01\\
98.21	0.01\\
98.22	0.01\\
98.23	0.01\\
98.24	0.01\\
98.25	0.01\\
98.26	0.01\\
98.27	0.01\\
98.28	0.01\\
98.29	0.01\\
98.3	0.01\\
98.31	0.01\\
98.32	0.01\\
98.33	0.01\\
98.34	0.01\\
98.35	0.01\\
98.36	0.01\\
98.37	0.01\\
98.38	0.01\\
98.39	0.01\\
98.4	0.01\\
98.41	0.01\\
98.42	0.01\\
98.43	0.01\\
98.44	0.01\\
98.45	0.01\\
98.46	0.01\\
98.47	0.01\\
98.48	0.01\\
98.49	0.01\\
98.5	0.01\\
98.51	0.01\\
98.52	0.01\\
98.53	0.01\\
98.54	0.01\\
98.55	0.01\\
98.56	0.01\\
98.57	0.01\\
98.58	0.01\\
98.59	0.01\\
98.6	0.01\\
98.61	0.01\\
98.62	0.01\\
98.63	0.01\\
98.64	0.01\\
98.65	0.01\\
98.66	0.01\\
98.67	0.01\\
98.68	0.01\\
98.69	0.01\\
98.7	0.01\\
98.71	0.01\\
98.72	0.01\\
98.73	0.01\\
98.74	0.01\\
98.75	0.01\\
98.76	0.01\\
98.77	0.01\\
98.78	0.01\\
98.79	0.01\\
98.8	0.01\\
98.81	0.01\\
98.82	0.01\\
98.83	0.01\\
98.84	0.01\\
98.85	0.01\\
98.86	0.01\\
98.87	0.01\\
98.88	0.01\\
98.89	0.01\\
98.9	0.01\\
98.91	0.01\\
98.92	0.01\\
98.93	0.01\\
98.94	0.01\\
98.95	0.01\\
98.96	0.01\\
98.97	0.01\\
98.98	0.01\\
98.99	0.01\\
99	0.01\\
99.01	0.01\\
99.02	0.01\\
99.03	0.01\\
99.04	0.01\\
99.05	0.01\\
99.06	0.01\\
99.07	0.01\\
99.08	0.01\\
99.09	0.01\\
99.1	0.01\\
99.11	0.01\\
99.12	0.01\\
99.13	0.01\\
99.14	0.01\\
99.15	0.01\\
99.16	0.01\\
99.17	0.01\\
99.18	0.01\\
99.19	0.01\\
99.2	0.01\\
99.21	0.01\\
99.22	0.01\\
99.23	0.01\\
99.24	0.01\\
99.25	0.01\\
99.26	0.01\\
99.27	0.01\\
99.28	0.01\\
99.29	0.01\\
99.3	0.01\\
99.31	0.01\\
99.32	0.01\\
99.33	0.01\\
99.34	0.01\\
99.35	0.01\\
99.36	0.01\\
99.37	0.01\\
99.38	0.01\\
99.39	0.01\\
99.4	0.01\\
99.41	0.01\\
99.42	0.01\\
99.43	0.01\\
99.44	0.01\\
99.45	0.01\\
99.46	0.01\\
99.47	0.01\\
99.48	0.01\\
99.49	0.01\\
99.5	0.01\\
99.51	0.01\\
99.52	0.01\\
99.53	0.01\\
99.54	0.01\\
99.55	0.01\\
99.56	0.01\\
99.57	0.01\\
99.58	0.01\\
99.59	0.01\\
99.6	0.01\\
99.61	0.01\\
99.62	0.01\\
99.63	0.01\\
99.64	0.01\\
99.65	0.01\\
99.66	0.01\\
99.67	0.01\\
99.68	0.01\\
99.69	0.01\\
99.7	0.01\\
99.71	0.01\\
99.72	0.01\\
99.73	0.01\\
99.74	0.01\\
99.75	0.01\\
99.76	0.01\\
99.77	0.01\\
99.78	0.01\\
99.79	0.01\\
99.8	0.01\\
99.81	0.01\\
99.82	0.01\\
99.83	0.01\\
99.84	0.01\\
99.85	0.01\\
99.86	0.01\\
99.87	0.01\\
99.88	0.01\\
99.89	0.01\\
99.9	0.01\\
99.91	0.01\\
99.92	0.01\\
99.93	0.01\\
99.94	0.01\\
99.95	0.01\\
99.96	0.01\\
99.97	0.01\\
99.98	0.01\\
99.99	0.01\\
100	0.01\\
};
\addlegendentry{$q=3$};

\addplot [color=green,solid,forget plot]
  table[row sep=crcr]{%
0.01	0.01\\
0.02	0.01\\
0.03	0.01\\
0.04	0.01\\
0.05	0.01\\
0.06	0.01\\
0.07	0.01\\
0.08	0.01\\
0.09	0.01\\
0.1	0.01\\
0.11	0.01\\
0.12	0.01\\
0.13	0.01\\
0.14	0.01\\
0.15	0.01\\
0.16	0.01\\
0.17	0.01\\
0.18	0.01\\
0.19	0.01\\
0.2	0.01\\
0.21	0.01\\
0.22	0.01\\
0.23	0.01\\
0.24	0.01\\
0.25	0.01\\
0.26	0.01\\
0.27	0.01\\
0.28	0.01\\
0.29	0.01\\
0.3	0.01\\
0.31	0.01\\
0.32	0.01\\
0.33	0.01\\
0.34	0.01\\
0.35	0.01\\
0.36	0.01\\
0.37	0.01\\
0.38	0.01\\
0.39	0.01\\
0.4	0.01\\
0.41	0.01\\
0.42	0.01\\
0.43	0.01\\
0.44	0.01\\
0.45	0.01\\
0.46	0.01\\
0.47	0.01\\
0.48	0.01\\
0.49	0.01\\
0.5	0.01\\
0.51	0.01\\
0.52	0.01\\
0.53	0.01\\
0.54	0.01\\
0.55	0.01\\
0.56	0.01\\
0.57	0.01\\
0.58	0.01\\
0.59	0.01\\
0.6	0.01\\
0.61	0.01\\
0.62	0.01\\
0.63	0.01\\
0.64	0.01\\
0.65	0.01\\
0.66	0.01\\
0.67	0.01\\
0.68	0.01\\
0.69	0.01\\
0.7	0.01\\
0.71	0.01\\
0.72	0.01\\
0.73	0.01\\
0.74	0.01\\
0.75	0.01\\
0.76	0.01\\
0.77	0.01\\
0.78	0.01\\
0.79	0.01\\
0.8	0.01\\
0.81	0.01\\
0.82	0.01\\
0.83	0.01\\
0.84	0.01\\
0.85	0.01\\
0.86	0.01\\
0.87	0.01\\
0.88	0.01\\
0.89	0.01\\
0.9	0.01\\
0.91	0.01\\
0.92	0.01\\
0.93	0.01\\
0.94	0.01\\
0.95	0.01\\
0.96	0.01\\
0.97	0.01\\
0.98	0.01\\
0.99	0.01\\
1	0.01\\
1.01	0.01\\
1.02	0.01\\
1.03	0.01\\
1.04	0.01\\
1.05	0.01\\
1.06	0.01\\
1.07	0.01\\
1.08	0.01\\
1.09	0.01\\
1.1	0.01\\
1.11	0.01\\
1.12	0.01\\
1.13	0.01\\
1.14	0.01\\
1.15	0.01\\
1.16	0.01\\
1.17	0.01\\
1.18	0.01\\
1.19	0.01\\
1.2	0.01\\
1.21	0.01\\
1.22	0.01\\
1.23	0.01\\
1.24	0.01\\
1.25	0.01\\
1.26	0.01\\
1.27	0.01\\
1.28	0.01\\
1.29	0.01\\
1.3	0.01\\
1.31	0.01\\
1.32	0.01\\
1.33	0.01\\
1.34	0.01\\
1.35	0.01\\
1.36	0.01\\
1.37	0.01\\
1.38	0.01\\
1.39	0.01\\
1.4	0.01\\
1.41	0.01\\
1.42	0.01\\
1.43	0.01\\
1.44	0.01\\
1.45	0.01\\
1.46	0.01\\
1.47	0.01\\
1.48	0.01\\
1.49	0.01\\
1.5	0.01\\
1.51	0.01\\
1.52	0.01\\
1.53	0.01\\
1.54	0.01\\
1.55	0.01\\
1.56	0.01\\
1.57	0.01\\
1.58	0.01\\
1.59	0.01\\
1.6	0.01\\
1.61	0.01\\
1.62	0.01\\
1.63	0.01\\
1.64	0.01\\
1.65	0.01\\
1.66	0.01\\
1.67	0.01\\
1.68	0.01\\
1.69	0.01\\
1.7	0.01\\
1.71	0.01\\
1.72	0.01\\
1.73	0.01\\
1.74	0.01\\
1.75	0.01\\
1.76	0.01\\
1.77	0.01\\
1.78	0.01\\
1.79	0.01\\
1.8	0.01\\
1.81	0.01\\
1.82	0.01\\
1.83	0.01\\
1.84	0.01\\
1.85	0.01\\
1.86	0.01\\
1.87	0.01\\
1.88	0.01\\
1.89	0.01\\
1.9	0.01\\
1.91	0.01\\
1.92	0.01\\
1.93	0.01\\
1.94	0.01\\
1.95	0.01\\
1.96	0.01\\
1.97	0.01\\
1.98	0.01\\
1.99	0.01\\
2	0.01\\
2.01	0.01\\
2.02	0.01\\
2.03	0.01\\
2.04	0.01\\
2.05	0.01\\
2.06	0.01\\
2.07	0.01\\
2.08	0.01\\
2.09	0.01\\
2.1	0.01\\
2.11	0.01\\
2.12	0.01\\
2.13	0.01\\
2.14	0.01\\
2.15	0.01\\
2.16	0.01\\
2.17	0.01\\
2.18	0.01\\
2.19	0.01\\
2.2	0.01\\
2.21	0.01\\
2.22	0.01\\
2.23	0.01\\
2.24	0.01\\
2.25	0.01\\
2.26	0.01\\
2.27	0.01\\
2.28	0.01\\
2.29	0.01\\
2.3	0.01\\
2.31	0.01\\
2.32	0.01\\
2.33	0.01\\
2.34	0.01\\
2.35	0.01\\
2.36	0.01\\
2.37	0.01\\
2.38	0.01\\
2.39	0.01\\
2.4	0.01\\
2.41	0.01\\
2.42	0.01\\
2.43	0.01\\
2.44	0.01\\
2.45	0.01\\
2.46	0.01\\
2.47	0.01\\
2.48	0.01\\
2.49	0.01\\
2.5	0.01\\
2.51	0.01\\
2.52	0.01\\
2.53	0.01\\
2.54	0.01\\
2.55	0.01\\
2.56	0.01\\
2.57	0.01\\
2.58	0.01\\
2.59	0.01\\
2.6	0.01\\
2.61	0.01\\
2.62	0.01\\
2.63	0.01\\
2.64	0.01\\
2.65	0.01\\
2.66	0.01\\
2.67	0.01\\
2.68	0.01\\
2.69	0.01\\
2.7	0.01\\
2.71	0.01\\
2.72	0.01\\
2.73	0.01\\
2.74	0.01\\
2.75	0.01\\
2.76	0.01\\
2.77	0.01\\
2.78	0.01\\
2.79	0.01\\
2.8	0.01\\
2.81	0.01\\
2.82	0.01\\
2.83	0.01\\
2.84	0.01\\
2.85	0.01\\
2.86	0.01\\
2.87	0.01\\
2.88	0.01\\
2.89	0.01\\
2.9	0.01\\
2.91	0.01\\
2.92	0.01\\
2.93	0.01\\
2.94	0.01\\
2.95	0.01\\
2.96	0.01\\
2.97	0.01\\
2.98	0.01\\
2.99	0.01\\
3	0.01\\
3.01	0.01\\
3.02	0.01\\
3.03	0.01\\
3.04	0.01\\
3.05	0.01\\
3.06	0.01\\
3.07	0.01\\
3.08	0.01\\
3.09	0.01\\
3.1	0.01\\
3.11	0.01\\
3.12	0.01\\
3.13	0.01\\
3.14	0.01\\
3.15	0.01\\
3.16	0.01\\
3.17	0.01\\
3.18	0.01\\
3.19	0.01\\
3.2	0.01\\
3.21	0.01\\
3.22	0.01\\
3.23	0.01\\
3.24	0.01\\
3.25	0.01\\
3.26	0.01\\
3.27	0.01\\
3.28	0.01\\
3.29	0.01\\
3.3	0.01\\
3.31	0.01\\
3.32	0.01\\
3.33	0.01\\
3.34	0.01\\
3.35	0.01\\
3.36	0.01\\
3.37	0.01\\
3.38	0.01\\
3.39	0.01\\
3.4	0.01\\
3.41	0.01\\
3.42	0.01\\
3.43	0.01\\
3.44	0.01\\
3.45	0.01\\
3.46	0.01\\
3.47	0.01\\
3.48	0.01\\
3.49	0.01\\
3.5	0.01\\
3.51	0.01\\
3.52	0.01\\
3.53	0.01\\
3.54	0.01\\
3.55	0.01\\
3.56	0.01\\
3.57	0.01\\
3.58	0.01\\
3.59	0.01\\
3.6	0.01\\
3.61	0.01\\
3.62	0.01\\
3.63	0.01\\
3.64	0.01\\
3.65	0.01\\
3.66	0.01\\
3.67	0.01\\
3.68	0.01\\
3.69	0.01\\
3.7	0.01\\
3.71	0.01\\
3.72	0.01\\
3.73	0.01\\
3.74	0.01\\
3.75	0.01\\
3.76	0.01\\
3.77	0.01\\
3.78	0.01\\
3.79	0.01\\
3.8	0.01\\
3.81	0.01\\
3.82	0.01\\
3.83	0.01\\
3.84	0.01\\
3.85	0.01\\
3.86	0.01\\
3.87	0.01\\
3.88	0.01\\
3.89	0.01\\
3.9	0.01\\
3.91	0.01\\
3.92	0.01\\
3.93	0.01\\
3.94	0.01\\
3.95	0.01\\
3.96	0.01\\
3.97	0.01\\
3.98	0.01\\
3.99	0.01\\
4	0.01\\
4.01	0.01\\
4.02	0.01\\
4.03	0.01\\
4.04	0.01\\
4.05	0.01\\
4.06	0.01\\
4.07	0.01\\
4.08	0.01\\
4.09	0.01\\
4.1	0.01\\
4.11	0.01\\
4.12	0.01\\
4.13	0.01\\
4.14	0.01\\
4.15	0.01\\
4.16	0.01\\
4.17	0.01\\
4.18	0.01\\
4.19	0.01\\
4.2	0.01\\
4.21	0.01\\
4.22	0.01\\
4.23	0.01\\
4.24	0.01\\
4.25	0.01\\
4.26	0.01\\
4.27	0.01\\
4.28	0.01\\
4.29	0.01\\
4.3	0.01\\
4.31	0.01\\
4.32	0.01\\
4.33	0.01\\
4.34	0.01\\
4.35	0.01\\
4.36	0.01\\
4.37	0.01\\
4.38	0.01\\
4.39	0.01\\
4.4	0.01\\
4.41	0.01\\
4.42	0.01\\
4.43	0.01\\
4.44	0.01\\
4.45	0.01\\
4.46	0.01\\
4.47	0.01\\
4.48	0.01\\
4.49	0.01\\
4.5	0.01\\
4.51	0.01\\
4.52	0.01\\
4.53	0.01\\
4.54	0.01\\
4.55	0.01\\
4.56	0.01\\
4.57	0.01\\
4.58	0.01\\
4.59	0.01\\
4.6	0.01\\
4.61	0.01\\
4.62	0.01\\
4.63	0.01\\
4.64	0.01\\
4.65	0.01\\
4.66	0.01\\
4.67	0.01\\
4.68	0.01\\
4.69	0.01\\
4.7	0.01\\
4.71	0.01\\
4.72	0.01\\
4.73	0.01\\
4.74	0.01\\
4.75	0.01\\
4.76	0.01\\
4.77	0.01\\
4.78	0.01\\
4.79	0.01\\
4.8	0.01\\
4.81	0.01\\
4.82	0.01\\
4.83	0.01\\
4.84	0.01\\
4.85	0.01\\
4.86	0.01\\
4.87	0.01\\
4.88	0.01\\
4.89	0.01\\
4.9	0.01\\
4.91	0.01\\
4.92	0.01\\
4.93	0.01\\
4.94	0.01\\
4.95	0.01\\
4.96	0.01\\
4.97	0.01\\
4.98	0.01\\
4.99	0.01\\
5	0.01\\
5.01	0.01\\
5.02	0.01\\
5.03	0.01\\
5.04	0.01\\
5.05	0.01\\
5.06	0.01\\
5.07	0.01\\
5.08	0.01\\
5.09	0.01\\
5.1	0.01\\
5.11	0.01\\
5.12	0.01\\
5.13	0.01\\
5.14	0.01\\
5.15	0.01\\
5.16	0.01\\
5.17	0.01\\
5.18	0.01\\
5.19	0.01\\
5.2	0.01\\
5.21	0.01\\
5.22	0.01\\
5.23	0.01\\
5.24	0.01\\
5.25	0.01\\
5.26	0.01\\
5.27	0.01\\
5.28	0.01\\
5.29	0.01\\
5.3	0.01\\
5.31	0.01\\
5.32	0.01\\
5.33	0.01\\
5.34	0.01\\
5.35	0.01\\
5.36	0.01\\
5.37	0.01\\
5.38	0.01\\
5.39	0.01\\
5.4	0.01\\
5.41	0.01\\
5.42	0.01\\
5.43	0.01\\
5.44	0.01\\
5.45	0.01\\
5.46	0.01\\
5.47	0.01\\
5.48	0.01\\
5.49	0.01\\
5.5	0.01\\
5.51	0.01\\
5.52	0.01\\
5.53	0.01\\
5.54	0.01\\
5.55	0.01\\
5.56	0.01\\
5.57	0.01\\
5.58	0.01\\
5.59	0.01\\
5.6	0.01\\
5.61	0.01\\
5.62	0.01\\
5.63	0.01\\
5.64	0.01\\
5.65	0.01\\
5.66	0.01\\
5.67	0.01\\
5.68	0.01\\
5.69	0.01\\
5.7	0.01\\
5.71	0.01\\
5.72	0.01\\
5.73	0.01\\
5.74	0.01\\
5.75	0.01\\
5.76	0.01\\
5.77	0.01\\
5.78	0.01\\
5.79	0.01\\
5.8	0.01\\
5.81	0.01\\
5.82	0.01\\
5.83	0.01\\
5.84	0.01\\
5.85	0.01\\
5.86	0.01\\
5.87	0.01\\
5.88	0.01\\
5.89	0.01\\
5.9	0.01\\
5.91	0.01\\
5.92	0.01\\
5.93	0.01\\
5.94	0.01\\
5.95	0.01\\
5.96	0.01\\
5.97	0.01\\
5.98	0.01\\
5.99	0.01\\
6	0.01\\
6.01	0.01\\
6.02	0.01\\
6.03	0.01\\
6.04	0.01\\
6.05	0.01\\
6.06	0.01\\
6.07	0.01\\
6.08	0.01\\
6.09	0.01\\
6.1	0.01\\
6.11	0.01\\
6.12	0.01\\
6.13	0.01\\
6.14	0.01\\
6.15	0.01\\
6.16	0.01\\
6.17	0.01\\
6.18	0.01\\
6.19	0.01\\
6.2	0.01\\
6.21	0.01\\
6.22	0.01\\
6.23	0.01\\
6.24	0.01\\
6.25	0.01\\
6.26	0.01\\
6.27	0.01\\
6.28	0.01\\
6.29	0.01\\
6.3	0.01\\
6.31	0.01\\
6.32	0.01\\
6.33	0.01\\
6.34	0.01\\
6.35	0.01\\
6.36	0.01\\
6.37	0.01\\
6.38	0.01\\
6.39	0.01\\
6.4	0.01\\
6.41	0.01\\
6.42	0.01\\
6.43	0.01\\
6.44	0.01\\
6.45	0.01\\
6.46	0.01\\
6.47	0.01\\
6.48	0.01\\
6.49	0.01\\
6.5	0.01\\
6.51	0.01\\
6.52	0.01\\
6.53	0.01\\
6.54	0.01\\
6.55	0.01\\
6.56	0.01\\
6.57	0.01\\
6.58	0.01\\
6.59	0.01\\
6.6	0.01\\
6.61	0.01\\
6.62	0.01\\
6.63	0.01\\
6.64	0.01\\
6.65	0.01\\
6.66	0.01\\
6.67	0.01\\
6.68	0.01\\
6.69	0.01\\
6.7	0.01\\
6.71	0.01\\
6.72	0.01\\
6.73	0.01\\
6.74	0.01\\
6.75	0.01\\
6.76	0.01\\
6.77	0.01\\
6.78	0.01\\
6.79	0.01\\
6.8	0.01\\
6.81	0.01\\
6.82	0.01\\
6.83	0.01\\
6.84	0.01\\
6.85	0.01\\
6.86	0.01\\
6.87	0.01\\
6.88	0.01\\
6.89	0.01\\
6.9	0.01\\
6.91	0.01\\
6.92	0.01\\
6.93	0.01\\
6.94	0.01\\
6.95	0.01\\
6.96	0.01\\
6.97	0.01\\
6.98	0.01\\
6.99	0.01\\
7	0.01\\
7.01	0.01\\
7.02	0.01\\
7.03	0.01\\
7.04	0.01\\
7.05	0.01\\
7.06	0.01\\
7.07	0.01\\
7.08	0.01\\
7.09	0.01\\
7.1	0.01\\
7.11	0.01\\
7.12	0.01\\
7.13	0.01\\
7.14	0.01\\
7.15	0.01\\
7.16	0.01\\
7.17	0.01\\
7.18	0.01\\
7.19	0.01\\
7.2	0.01\\
7.21	0.01\\
7.22	0.01\\
7.23	0.01\\
7.24	0.01\\
7.25	0.01\\
7.26	0.01\\
7.27	0.01\\
7.28	0.01\\
7.29	0.01\\
7.3	0.01\\
7.31	0.01\\
7.32	0.01\\
7.33	0.01\\
7.34	0.01\\
7.35	0.01\\
7.36	0.01\\
7.37	0.01\\
7.38	0.01\\
7.39	0.01\\
7.4	0.01\\
7.41	0.01\\
7.42	0.01\\
7.43	0.01\\
7.44	0.01\\
7.45	0.01\\
7.46	0.01\\
7.47	0.01\\
7.48	0.01\\
7.49	0.01\\
7.5	0.01\\
7.51	0.01\\
7.52	0.01\\
7.53	0.01\\
7.54	0.01\\
7.55	0.01\\
7.56	0.01\\
7.57	0.01\\
7.58	0.01\\
7.59	0.01\\
7.6	0.01\\
7.61	0.01\\
7.62	0.01\\
7.63	0.01\\
7.64	0.01\\
7.65	0.01\\
7.66	0.01\\
7.67	0.01\\
7.68	0.01\\
7.69	0.01\\
7.7	0.01\\
7.71	0.01\\
7.72	0.01\\
7.73	0.01\\
7.74	0.01\\
7.75	0.01\\
7.76	0.01\\
7.77	0.01\\
7.78	0.01\\
7.79	0.01\\
7.8	0.01\\
7.81	0.01\\
7.82	0.01\\
7.83	0.01\\
7.84	0.01\\
7.85	0.01\\
7.86	0.01\\
7.87	0.01\\
7.88	0.01\\
7.89	0.01\\
7.9	0.01\\
7.91	0.01\\
7.92	0.01\\
7.93	0.01\\
7.94	0.01\\
7.95	0.01\\
7.96	0.01\\
7.97	0.01\\
7.98	0.01\\
7.99	0.01\\
8	0.01\\
8.01	0.01\\
8.02	0.01\\
8.03	0.01\\
8.04	0.01\\
8.05	0.01\\
8.06	0.01\\
8.07	0.01\\
8.08	0.01\\
8.09	0.01\\
8.1	0.01\\
8.11	0.01\\
8.12	0.01\\
8.13	0.01\\
8.14	0.01\\
8.15	0.01\\
8.16	0.01\\
8.17	0.01\\
8.18	0.01\\
8.19	0.01\\
8.2	0.01\\
8.21	0.01\\
8.22	0.01\\
8.23	0.01\\
8.24	0.01\\
8.25	0.01\\
8.26	0.01\\
8.27	0.01\\
8.28	0.01\\
8.29	0.01\\
8.3	0.01\\
8.31	0.01\\
8.32	0.01\\
8.33	0.01\\
8.34	0.01\\
8.35	0.01\\
8.36	0.01\\
8.37	0.01\\
8.38	0.01\\
8.39	0.01\\
8.4	0.01\\
8.41	0.01\\
8.42	0.01\\
8.43	0.01\\
8.44	0.01\\
8.45	0.01\\
8.46	0.01\\
8.47	0.01\\
8.48	0.01\\
8.49	0.01\\
8.5	0.01\\
8.51	0.01\\
8.52	0.01\\
8.53	0.01\\
8.54	0.01\\
8.55	0.01\\
8.56	0.01\\
8.57	0.01\\
8.58	0.01\\
8.59	0.01\\
8.6	0.01\\
8.61	0.01\\
8.62	0.01\\
8.63	0.01\\
8.64	0.01\\
8.65	0.01\\
8.66	0.01\\
8.67	0.01\\
8.68	0.01\\
8.69	0.01\\
8.7	0.01\\
8.71	0.01\\
8.72	0.01\\
8.73	0.01\\
8.74	0.01\\
8.75	0.01\\
8.76	0.01\\
8.77	0.01\\
8.78	0.01\\
8.79	0.01\\
8.8	0.01\\
8.81	0.01\\
8.82	0.01\\
8.83	0.01\\
8.84	0.01\\
8.85	0.01\\
8.86	0.01\\
8.87	0.01\\
8.88	0.01\\
8.89	0.01\\
8.9	0.01\\
8.91	0.01\\
8.92	0.01\\
8.93	0.01\\
8.94	0.01\\
8.95	0.01\\
8.96	0.01\\
8.97	0.01\\
8.98	0.01\\
8.99	0.01\\
9	0.01\\
9.01	0.01\\
9.02	0.01\\
9.03	0.01\\
9.04	0.01\\
9.05	0.01\\
9.06	0.01\\
9.07	0.01\\
9.08	0.01\\
9.09	0.01\\
9.1	0.01\\
9.11	0.01\\
9.12	0.01\\
9.13	0.01\\
9.14	0.01\\
9.15	0.01\\
9.16	0.01\\
9.17	0.01\\
9.18	0.01\\
9.19	0.01\\
9.2	0.01\\
9.21	0.01\\
9.22	0.01\\
9.23	0.01\\
9.24	0.01\\
9.25	0.01\\
9.26	0.01\\
9.27	0.01\\
9.28	0.01\\
9.29	0.01\\
9.3	0.01\\
9.31	0.01\\
9.32	0.01\\
9.33	0.01\\
9.34	0.01\\
9.35	0.01\\
9.36	0.01\\
9.37	0.01\\
9.38	0.01\\
9.39	0.01\\
9.4	0.01\\
9.41	0.01\\
9.42	0.01\\
9.43	0.01\\
9.44	0.01\\
9.45	0.01\\
9.46	0.01\\
9.47	0.01\\
9.48	0.01\\
9.49	0.01\\
9.5	0.01\\
9.51	0.01\\
9.52	0.01\\
9.53	0.01\\
9.54	0.01\\
9.55	0.01\\
9.56	0.01\\
9.57	0.01\\
9.58	0.01\\
9.59	0.01\\
9.6	0.01\\
9.61	0.01\\
9.62	0.01\\
9.63	0.01\\
9.64	0.01\\
9.65	0.01\\
9.66	0.01\\
9.67	0.01\\
9.68	0.01\\
9.69	0.01\\
9.7	0.01\\
9.71	0.01\\
9.72	0.01\\
9.73	0.01\\
9.74	0.01\\
9.75	0.01\\
9.76	0.01\\
9.77	0.01\\
9.78	0.01\\
9.79	0.01\\
9.8	0.01\\
9.81	0.01\\
9.82	0.01\\
9.83	0.01\\
9.84	0.01\\
9.85	0.01\\
9.86	0.01\\
9.87	0.01\\
9.88	0.01\\
9.89	0.01\\
9.9	0.01\\
9.91	0.01\\
9.92	0.01\\
9.93	0.01\\
9.94	0.01\\
9.95	0.01\\
9.96	0.01\\
9.97	0.01\\
9.98	0.01\\
9.99	0.01\\
10	0.01\\
10.01	0.01\\
10.02	0.01\\
10.03	0.01\\
10.04	0.01\\
10.05	0.01\\
10.06	0.01\\
10.07	0.01\\
10.08	0.01\\
10.09	0.01\\
10.1	0.01\\
10.11	0.01\\
10.12	0.01\\
10.13	0.01\\
10.14	0.01\\
10.15	0.01\\
10.16	0.01\\
10.17	0.01\\
10.18	0.01\\
10.19	0.01\\
10.2	0.01\\
10.21	0.01\\
10.22	0.01\\
10.23	0.01\\
10.24	0.01\\
10.25	0.01\\
10.26	0.01\\
10.27	0.01\\
10.28	0.01\\
10.29	0.01\\
10.3	0.01\\
10.31	0.01\\
10.32	0.01\\
10.33	0.01\\
10.34	0.01\\
10.35	0.01\\
10.36	0.01\\
10.37	0.01\\
10.38	0.01\\
10.39	0.01\\
10.4	0.01\\
10.41	0.01\\
10.42	0.01\\
10.43	0.01\\
10.44	0.01\\
10.45	0.01\\
10.46	0.01\\
10.47	0.01\\
10.48	0.01\\
10.49	0.01\\
10.5	0.01\\
10.51	0.01\\
10.52	0.01\\
10.53	0.01\\
10.54	0.01\\
10.55	0.01\\
10.56	0.01\\
10.57	0.01\\
10.58	0.01\\
10.59	0.01\\
10.6	0.01\\
10.61	0.01\\
10.62	0.01\\
10.63	0.01\\
10.64	0.01\\
10.65	0.01\\
10.66	0.01\\
10.67	0.01\\
10.68	0.01\\
10.69	0.01\\
10.7	0.01\\
10.71	0.01\\
10.72	0.01\\
10.73	0.01\\
10.74	0.01\\
10.75	0.01\\
10.76	0.01\\
10.77	0.01\\
10.78	0.01\\
10.79	0.01\\
10.8	0.01\\
10.81	0.01\\
10.82	0.01\\
10.83	0.01\\
10.84	0.01\\
10.85	0.01\\
10.86	0.01\\
10.87	0.01\\
10.88	0.01\\
10.89	0.01\\
10.9	0.01\\
10.91	0.01\\
10.92	0.01\\
10.93	0.01\\
10.94	0.01\\
10.95	0.01\\
10.96	0.01\\
10.97	0.01\\
10.98	0.01\\
10.99	0.01\\
11	0.01\\
11.01	0.01\\
11.02	0.01\\
11.03	0.01\\
11.04	0.01\\
11.05	0.01\\
11.06	0.01\\
11.07	0.01\\
11.08	0.01\\
11.09	0.01\\
11.1	0.01\\
11.11	0.01\\
11.12	0.01\\
11.13	0.01\\
11.14	0.01\\
11.15	0.01\\
11.16	0.01\\
11.17	0.01\\
11.18	0.01\\
11.19	0.01\\
11.2	0.01\\
11.21	0.01\\
11.22	0.01\\
11.23	0.01\\
11.24	0.01\\
11.25	0.01\\
11.26	0.01\\
11.27	0.01\\
11.28	0.01\\
11.29	0.01\\
11.3	0.01\\
11.31	0.01\\
11.32	0.01\\
11.33	0.01\\
11.34	0.01\\
11.35	0.01\\
11.36	0.01\\
11.37	0.01\\
11.38	0.01\\
11.39	0.01\\
11.4	0.01\\
11.41	0.01\\
11.42	0.01\\
11.43	0.01\\
11.44	0.01\\
11.45	0.01\\
11.46	0.01\\
11.47	0.01\\
11.48	0.01\\
11.49	0.01\\
11.5	0.01\\
11.51	0.01\\
11.52	0.01\\
11.53	0.01\\
11.54	0.01\\
11.55	0.01\\
11.56	0.01\\
11.57	0.01\\
11.58	0.01\\
11.59	0.01\\
11.6	0.01\\
11.61	0.01\\
11.62	0.01\\
11.63	0.01\\
11.64	0.01\\
11.65	0.01\\
11.66	0.01\\
11.67	0.01\\
11.68	0.01\\
11.69	0.01\\
11.7	0.01\\
11.71	0.01\\
11.72	0.01\\
11.73	0.01\\
11.74	0.01\\
11.75	0.01\\
11.76	0.01\\
11.77	0.01\\
11.78	0.01\\
11.79	0.01\\
11.8	0.01\\
11.81	0.01\\
11.82	0.01\\
11.83	0.01\\
11.84	0.01\\
11.85	0.01\\
11.86	0.01\\
11.87	0.01\\
11.88	0.01\\
11.89	0.01\\
11.9	0.01\\
11.91	0.01\\
11.92	0.01\\
11.93	0.01\\
11.94	0.01\\
11.95	0.01\\
11.96	0.01\\
11.97	0.01\\
11.98	0.01\\
11.99	0.01\\
12	0.01\\
12.01	0.01\\
12.02	0.01\\
12.03	0.01\\
12.04	0.01\\
12.05	0.01\\
12.06	0.01\\
12.07	0.01\\
12.08	0.01\\
12.09	0.01\\
12.1	0.01\\
12.11	0.01\\
12.12	0.01\\
12.13	0.01\\
12.14	0.01\\
12.15	0.01\\
12.16	0.01\\
12.17	0.01\\
12.18	0.01\\
12.19	0.01\\
12.2	0.01\\
12.21	0.01\\
12.22	0.01\\
12.23	0.01\\
12.24	0.01\\
12.25	0.01\\
12.26	0.01\\
12.27	0.01\\
12.28	0.01\\
12.29	0.01\\
12.3	0.01\\
12.31	0.01\\
12.32	0.01\\
12.33	0.01\\
12.34	0.01\\
12.35	0.01\\
12.36	0.01\\
12.37	0.01\\
12.38	0.01\\
12.39	0.01\\
12.4	0.01\\
12.41	0.01\\
12.42	0.01\\
12.43	0.01\\
12.44	0.01\\
12.45	0.01\\
12.46	0.01\\
12.47	0.01\\
12.48	0.01\\
12.49	0.01\\
12.5	0.01\\
12.51	0.01\\
12.52	0.01\\
12.53	0.01\\
12.54	0.01\\
12.55	0.01\\
12.56	0.01\\
12.57	0.01\\
12.58	0.01\\
12.59	0.01\\
12.6	0.01\\
12.61	0.01\\
12.62	0.01\\
12.63	0.01\\
12.64	0.01\\
12.65	0.01\\
12.66	0.01\\
12.67	0.01\\
12.68	0.01\\
12.69	0.01\\
12.7	0.01\\
12.71	0.01\\
12.72	0.01\\
12.73	0.01\\
12.74	0.01\\
12.75	0.01\\
12.76	0.01\\
12.77	0.01\\
12.78	0.01\\
12.79	0.01\\
12.8	0.01\\
12.81	0.01\\
12.82	0.01\\
12.83	0.01\\
12.84	0.01\\
12.85	0.01\\
12.86	0.01\\
12.87	0.01\\
12.88	0.01\\
12.89	0.01\\
12.9	0.01\\
12.91	0.01\\
12.92	0.01\\
12.93	0.01\\
12.94	0.01\\
12.95	0.01\\
12.96	0.01\\
12.97	0.01\\
12.98	0.01\\
12.99	0.01\\
13	0.01\\
13.01	0.01\\
13.02	0.01\\
13.03	0.01\\
13.04	0.01\\
13.05	0.01\\
13.06	0.01\\
13.07	0.01\\
13.08	0.01\\
13.09	0.01\\
13.1	0.01\\
13.11	0.01\\
13.12	0.01\\
13.13	0.01\\
13.14	0.01\\
13.15	0.01\\
13.16	0.01\\
13.17	0.01\\
13.18	0.01\\
13.19	0.01\\
13.2	0.01\\
13.21	0.01\\
13.22	0.01\\
13.23	0.01\\
13.24	0.01\\
13.25	0.01\\
13.26	0.01\\
13.27	0.01\\
13.28	0.01\\
13.29	0.01\\
13.3	0.01\\
13.31	0.01\\
13.32	0.01\\
13.33	0.01\\
13.34	0.01\\
13.35	0.01\\
13.36	0.01\\
13.37	0.01\\
13.38	0.01\\
13.39	0.01\\
13.4	0.01\\
13.41	0.01\\
13.42	0.01\\
13.43	0.01\\
13.44	0.01\\
13.45	0.01\\
13.46	0.01\\
13.47	0.01\\
13.48	0.01\\
13.49	0.01\\
13.5	0.01\\
13.51	0.01\\
13.52	0.01\\
13.53	0.01\\
13.54	0.01\\
13.55	0.01\\
13.56	0.01\\
13.57	0.01\\
13.58	0.01\\
13.59	0.01\\
13.6	0.01\\
13.61	0.01\\
13.62	0.01\\
13.63	0.01\\
13.64	0.01\\
13.65	0.01\\
13.66	0.01\\
13.67	0.01\\
13.68	0.01\\
13.69	0.01\\
13.7	0.01\\
13.71	0.01\\
13.72	0.01\\
13.73	0.01\\
13.74	0.01\\
13.75	0.01\\
13.76	0.01\\
13.77	0.01\\
13.78	0.01\\
13.79	0.01\\
13.8	0.01\\
13.81	0.01\\
13.82	0.01\\
13.83	0.01\\
13.84	0.01\\
13.85	0.01\\
13.86	0.01\\
13.87	0.01\\
13.88	0.01\\
13.89	0.01\\
13.9	0.01\\
13.91	0.01\\
13.92	0.01\\
13.93	0.01\\
13.94	0.01\\
13.95	0.01\\
13.96	0.01\\
13.97	0.01\\
13.98	0.01\\
13.99	0.01\\
14	0.01\\
14.01	0.01\\
14.02	0.01\\
14.03	0.01\\
14.04	0.01\\
14.05	0.01\\
14.06	0.01\\
14.07	0.01\\
14.08	0.01\\
14.09	0.01\\
14.1	0.01\\
14.11	0.01\\
14.12	0.01\\
14.13	0.01\\
14.14	0.01\\
14.15	0.01\\
14.16	0.01\\
14.17	0.01\\
14.18	0.01\\
14.19	0.01\\
14.2	0.01\\
14.21	0.01\\
14.22	0.01\\
14.23	0.01\\
14.24	0.01\\
14.25	0.01\\
14.26	0.01\\
14.27	0.01\\
14.28	0.01\\
14.29	0.01\\
14.3	0.01\\
14.31	0.01\\
14.32	0.01\\
14.33	0.01\\
14.34	0.01\\
14.35	0.01\\
14.36	0.01\\
14.37	0.01\\
14.38	0.01\\
14.39	0.01\\
14.4	0.01\\
14.41	0.01\\
14.42	0.01\\
14.43	0.01\\
14.44	0.01\\
14.45	0.01\\
14.46	0.01\\
14.47	0.01\\
14.48	0.01\\
14.49	0.01\\
14.5	0.01\\
14.51	0.01\\
14.52	0.01\\
14.53	0.01\\
14.54	0.01\\
14.55	0.01\\
14.56	0.01\\
14.57	0.01\\
14.58	0.01\\
14.59	0.01\\
14.6	0.01\\
14.61	0.01\\
14.62	0.01\\
14.63	0.01\\
14.64	0.01\\
14.65	0.01\\
14.66	0.01\\
14.67	0.01\\
14.68	0.01\\
14.69	0.01\\
14.7	0.01\\
14.71	0.01\\
14.72	0.01\\
14.73	0.01\\
14.74	0.01\\
14.75	0.01\\
14.76	0.01\\
14.77	0.01\\
14.78	0.01\\
14.79	0.01\\
14.8	0.01\\
14.81	0.01\\
14.82	0.01\\
14.83	0.01\\
14.84	0.01\\
14.85	0.01\\
14.86	0.01\\
14.87	0.01\\
14.88	0.01\\
14.89	0.01\\
14.9	0.01\\
14.91	0.01\\
14.92	0.01\\
14.93	0.01\\
14.94	0.01\\
14.95	0.01\\
14.96	0.01\\
14.97	0.01\\
14.98	0.01\\
14.99	0.01\\
15	0.01\\
15.01	0.01\\
15.02	0.01\\
15.03	0.01\\
15.04	0.01\\
15.05	0.01\\
15.06	0.01\\
15.07	0.01\\
15.08	0.01\\
15.09	0.01\\
15.1	0.01\\
15.11	0.01\\
15.12	0.01\\
15.13	0.01\\
15.14	0.01\\
15.15	0.01\\
15.16	0.01\\
15.17	0.01\\
15.18	0.01\\
15.19	0.01\\
15.2	0.01\\
15.21	0.01\\
15.22	0.01\\
15.23	0.01\\
15.24	0.01\\
15.25	0.01\\
15.26	0.01\\
15.27	0.01\\
15.28	0.01\\
15.29	0.01\\
15.3	0.01\\
15.31	0.01\\
15.32	0.01\\
15.33	0.01\\
15.34	0.01\\
15.35	0.01\\
15.36	0.01\\
15.37	0.01\\
15.38	0.01\\
15.39	0.01\\
15.4	0.01\\
15.41	0.01\\
15.42	0.01\\
15.43	0.01\\
15.44	0.01\\
15.45	0.01\\
15.46	0.01\\
15.47	0.01\\
15.48	0.01\\
15.49	0.01\\
15.5	0.01\\
15.51	0.01\\
15.52	0.01\\
15.53	0.01\\
15.54	0.01\\
15.55	0.01\\
15.56	0.01\\
15.57	0.01\\
15.58	0.01\\
15.59	0.01\\
15.6	0.01\\
15.61	0.01\\
15.62	0.01\\
15.63	0.01\\
15.64	0.01\\
15.65	0.01\\
15.66	0.01\\
15.67	0.01\\
15.68	0.01\\
15.69	0.01\\
15.7	0.01\\
15.71	0.01\\
15.72	0.01\\
15.73	0.01\\
15.74	0.01\\
15.75	0.01\\
15.76	0.01\\
15.77	0.01\\
15.78	0.01\\
15.79	0.01\\
15.8	0.01\\
15.81	0.01\\
15.82	0.01\\
15.83	0.01\\
15.84	0.01\\
15.85	0.01\\
15.86	0.01\\
15.87	0.01\\
15.88	0.01\\
15.89	0.01\\
15.9	0.01\\
15.91	0.01\\
15.92	0.01\\
15.93	0.01\\
15.94	0.01\\
15.95	0.01\\
15.96	0.01\\
15.97	0.01\\
15.98	0.01\\
15.99	0.01\\
16	0.01\\
16.01	0.01\\
16.02	0.01\\
16.03	0.01\\
16.04	0.01\\
16.05	0.01\\
16.06	0.01\\
16.07	0.01\\
16.08	0.01\\
16.09	0.01\\
16.1	0.01\\
16.11	0.01\\
16.12	0.01\\
16.13	0.01\\
16.14	0.01\\
16.15	0.01\\
16.16	0.01\\
16.17	0.01\\
16.18	0.01\\
16.19	0.01\\
16.2	0.01\\
16.21	0.01\\
16.22	0.01\\
16.23	0.01\\
16.24	0.01\\
16.25	0.01\\
16.26	0.01\\
16.27	0.01\\
16.28	0.01\\
16.29	0.01\\
16.3	0.01\\
16.31	0.01\\
16.32	0.01\\
16.33	0.01\\
16.34	0.01\\
16.35	0.01\\
16.36	0.01\\
16.37	0.01\\
16.38	0.01\\
16.39	0.01\\
16.4	0.01\\
16.41	0.01\\
16.42	0.01\\
16.43	0.01\\
16.44	0.01\\
16.45	0.01\\
16.46	0.01\\
16.47	0.01\\
16.48	0.01\\
16.49	0.01\\
16.5	0.01\\
16.51	0.01\\
16.52	0.01\\
16.53	0.01\\
16.54	0.01\\
16.55	0.01\\
16.56	0.01\\
16.57	0.01\\
16.58	0.01\\
16.59	0.01\\
16.6	0.01\\
16.61	0.01\\
16.62	0.01\\
16.63	0.01\\
16.64	0.01\\
16.65	0.01\\
16.66	0.01\\
16.67	0.01\\
16.68	0.01\\
16.69	0.01\\
16.7	0.01\\
16.71	0.01\\
16.72	0.01\\
16.73	0.01\\
16.74	0.01\\
16.75	0.01\\
16.76	0.01\\
16.77	0.01\\
16.78	0.01\\
16.79	0.01\\
16.8	0.01\\
16.81	0.01\\
16.82	0.01\\
16.83	0.01\\
16.84	0.01\\
16.85	0.01\\
16.86	0.01\\
16.87	0.01\\
16.88	0.01\\
16.89	0.01\\
16.9	0.01\\
16.91	0.01\\
16.92	0.01\\
16.93	0.01\\
16.94	0.01\\
16.95	0.01\\
16.96	0.01\\
16.97	0.01\\
16.98	0.01\\
16.99	0.01\\
17	0.01\\
17.01	0.01\\
17.02	0.01\\
17.03	0.01\\
17.04	0.01\\
17.05	0.01\\
17.06	0.01\\
17.07	0.01\\
17.08	0.01\\
17.09	0.01\\
17.1	0.01\\
17.11	0.01\\
17.12	0.01\\
17.13	0.01\\
17.14	0.01\\
17.15	0.01\\
17.16	0.01\\
17.17	0.01\\
17.18	0.01\\
17.19	0.01\\
17.2	0.01\\
17.21	0.01\\
17.22	0.01\\
17.23	0.01\\
17.24	0.01\\
17.25	0.01\\
17.26	0.01\\
17.27	0.01\\
17.28	0.01\\
17.29	0.01\\
17.3	0.01\\
17.31	0.01\\
17.32	0.01\\
17.33	0.01\\
17.34	0.01\\
17.35	0.01\\
17.36	0.01\\
17.37	0.01\\
17.38	0.01\\
17.39	0.01\\
17.4	0.01\\
17.41	0.01\\
17.42	0.01\\
17.43	0.01\\
17.44	0.01\\
17.45	0.01\\
17.46	0.01\\
17.47	0.01\\
17.48	0.01\\
17.49	0.01\\
17.5	0.01\\
17.51	0.01\\
17.52	0.01\\
17.53	0.01\\
17.54	0.01\\
17.55	0.01\\
17.56	0.01\\
17.57	0.01\\
17.58	0.01\\
17.59	0.01\\
17.6	0.01\\
17.61	0.01\\
17.62	0.01\\
17.63	0.01\\
17.64	0.01\\
17.65	0.01\\
17.66	0.01\\
17.67	0.01\\
17.68	0.01\\
17.69	0.01\\
17.7	0.01\\
17.71	0.01\\
17.72	0.01\\
17.73	0.01\\
17.74	0.01\\
17.75	0.01\\
17.76	0.01\\
17.77	0.01\\
17.78	0.01\\
17.79	0.01\\
17.8	0.01\\
17.81	0.01\\
17.82	0.01\\
17.83	0.01\\
17.84	0.01\\
17.85	0.01\\
17.86	0.01\\
17.87	0.01\\
17.88	0.01\\
17.89	0.01\\
17.9	0.01\\
17.91	0.01\\
17.92	0.01\\
17.93	0.01\\
17.94	0.01\\
17.95	0.01\\
17.96	0.01\\
17.97	0.01\\
17.98	0.01\\
17.99	0.01\\
18	0.01\\
18.01	0.01\\
18.02	0.01\\
18.03	0.01\\
18.04	0.01\\
18.05	0.01\\
18.06	0.01\\
18.07	0.01\\
18.08	0.01\\
18.09	0.01\\
18.1	0.01\\
18.11	0.01\\
18.12	0.01\\
18.13	0.01\\
18.14	0.01\\
18.15	0.01\\
18.16	0.01\\
18.17	0.01\\
18.18	0.01\\
18.19	0.01\\
18.2	0.01\\
18.21	0.01\\
18.22	0.01\\
18.23	0.01\\
18.24	0.01\\
18.25	0.01\\
18.26	0.01\\
18.27	0.01\\
18.28	0.01\\
18.29	0.01\\
18.3	0.01\\
18.31	0.01\\
18.32	0.01\\
18.33	0.01\\
18.34	0.01\\
18.35	0.01\\
18.36	0.01\\
18.37	0.01\\
18.38	0.01\\
18.39	0.01\\
18.4	0.01\\
18.41	0.01\\
18.42	0.01\\
18.43	0.01\\
18.44	0.01\\
18.45	0.01\\
18.46	0.01\\
18.47	0.01\\
18.48	0.01\\
18.49	0.01\\
18.5	0.01\\
18.51	0.01\\
18.52	0.01\\
18.53	0.01\\
18.54	0.01\\
18.55	0.01\\
18.56	0.01\\
18.57	0.01\\
18.58	0.01\\
18.59	0.01\\
18.6	0.01\\
18.61	0.01\\
18.62	0.01\\
18.63	0.01\\
18.64	0.01\\
18.65	0.01\\
18.66	0.01\\
18.67	0.01\\
18.68	0.01\\
18.69	0.01\\
18.7	0.01\\
18.71	0.01\\
18.72	0.01\\
18.73	0.01\\
18.74	0.01\\
18.75	0.01\\
18.76	0.01\\
18.77	0.01\\
18.78	0.01\\
18.79	0.01\\
18.8	0.01\\
18.81	0.01\\
18.82	0.01\\
18.83	0.01\\
18.84	0.01\\
18.85	0.01\\
18.86	0.01\\
18.87	0.01\\
18.88	0.01\\
18.89	0.01\\
18.9	0.01\\
18.91	0.01\\
18.92	0.01\\
18.93	0.01\\
18.94	0.01\\
18.95	0.01\\
18.96	0.01\\
18.97	0.01\\
18.98	0.01\\
18.99	0.01\\
19	0.01\\
19.01	0.01\\
19.02	0.01\\
19.03	0.01\\
19.04	0.01\\
19.05	0.01\\
19.06	0.01\\
19.07	0.01\\
19.08	0.01\\
19.09	0.01\\
19.1	0.01\\
19.11	0.01\\
19.12	0.01\\
19.13	0.01\\
19.14	0.01\\
19.15	0.01\\
19.16	0.01\\
19.17	0.01\\
19.18	0.01\\
19.19	0.01\\
19.2	0.01\\
19.21	0.01\\
19.22	0.01\\
19.23	0.01\\
19.24	0.01\\
19.25	0.01\\
19.26	0.01\\
19.27	0.01\\
19.28	0.01\\
19.29	0.01\\
19.3	0.01\\
19.31	0.01\\
19.32	0.01\\
19.33	0.01\\
19.34	0.01\\
19.35	0.01\\
19.36	0.01\\
19.37	0.01\\
19.38	0.01\\
19.39	0.01\\
19.4	0.01\\
19.41	0.01\\
19.42	0.01\\
19.43	0.01\\
19.44	0.01\\
19.45	0.01\\
19.46	0.01\\
19.47	0.01\\
19.48	0.01\\
19.49	0.01\\
19.5	0.01\\
19.51	0.01\\
19.52	0.01\\
19.53	0.01\\
19.54	0.01\\
19.55	0.01\\
19.56	0.01\\
19.57	0.01\\
19.58	0.01\\
19.59	0.01\\
19.6	0.01\\
19.61	0.01\\
19.62	0.01\\
19.63	0.01\\
19.64	0.01\\
19.65	0.01\\
19.66	0.01\\
19.67	0.01\\
19.68	0.01\\
19.69	0.01\\
19.7	0.01\\
19.71	0.01\\
19.72	0.01\\
19.73	0.01\\
19.74	0.01\\
19.75	0.01\\
19.76	0.01\\
19.77	0.01\\
19.78	0.01\\
19.79	0.01\\
19.8	0.01\\
19.81	0.01\\
19.82	0.01\\
19.83	0.01\\
19.84	0.01\\
19.85	0.01\\
19.86	0.01\\
19.87	0.01\\
19.88	0.01\\
19.89	0.01\\
19.9	0.01\\
19.91	0.01\\
19.92	0.01\\
19.93	0.01\\
19.94	0.01\\
19.95	0.01\\
19.96	0.01\\
19.97	0.01\\
19.98	0.01\\
19.99	0.01\\
20	0.01\\
20.01	0.01\\
20.02	0.01\\
20.03	0.01\\
20.04	0.01\\
20.05	0.01\\
20.06	0.01\\
20.07	0.01\\
20.08	0.01\\
20.09	0.01\\
20.1	0.01\\
20.11	0.01\\
20.12	0.01\\
20.13	0.01\\
20.14	0.01\\
20.15	0.01\\
20.16	0.01\\
20.17	0.01\\
20.18	0.01\\
20.19	0.01\\
20.2	0.01\\
20.21	0.01\\
20.22	0.01\\
20.23	0.01\\
20.24	0.01\\
20.25	0.01\\
20.26	0.01\\
20.27	0.01\\
20.28	0.01\\
20.29	0.01\\
20.3	0.01\\
20.31	0.01\\
20.32	0.01\\
20.33	0.01\\
20.34	0.01\\
20.35	0.01\\
20.36	0.01\\
20.37	0.01\\
20.38	0.01\\
20.39	0.01\\
20.4	0.01\\
20.41	0.01\\
20.42	0.01\\
20.43	0.01\\
20.44	0.01\\
20.45	0.01\\
20.46	0.01\\
20.47	0.01\\
20.48	0.01\\
20.49	0.01\\
20.5	0.01\\
20.51	0.01\\
20.52	0.01\\
20.53	0.01\\
20.54	0.01\\
20.55	0.01\\
20.56	0.01\\
20.57	0.01\\
20.58	0.01\\
20.59	0.01\\
20.6	0.01\\
20.61	0.01\\
20.62	0.01\\
20.63	0.01\\
20.64	0.01\\
20.65	0.01\\
20.66	0.01\\
20.67	0.01\\
20.68	0.01\\
20.69	0.01\\
20.7	0.01\\
20.71	0.01\\
20.72	0.01\\
20.73	0.01\\
20.74	0.01\\
20.75	0.01\\
20.76	0.01\\
20.77	0.01\\
20.78	0.01\\
20.79	0.01\\
20.8	0.01\\
20.81	0.01\\
20.82	0.01\\
20.83	0.01\\
20.84	0.01\\
20.85	0.01\\
20.86	0.01\\
20.87	0.01\\
20.88	0.01\\
20.89	0.01\\
20.9	0.01\\
20.91	0.01\\
20.92	0.01\\
20.93	0.01\\
20.94	0.01\\
20.95	0.01\\
20.96	0.01\\
20.97	0.01\\
20.98	0.01\\
20.99	0.01\\
21	0.01\\
21.01	0.01\\
21.02	0.01\\
21.03	0.01\\
21.04	0.01\\
21.05	0.01\\
21.06	0.01\\
21.07	0.01\\
21.08	0.01\\
21.09	0.01\\
21.1	0.01\\
21.11	0.01\\
21.12	0.01\\
21.13	0.01\\
21.14	0.01\\
21.15	0.01\\
21.16	0.01\\
21.17	0.01\\
21.18	0.01\\
21.19	0.01\\
21.2	0.01\\
21.21	0.01\\
21.22	0.01\\
21.23	0.01\\
21.24	0.01\\
21.25	0.01\\
21.26	0.01\\
21.27	0.01\\
21.28	0.01\\
21.29	0.01\\
21.3	0.01\\
21.31	0.01\\
21.32	0.01\\
21.33	0.01\\
21.34	0.01\\
21.35	0.01\\
21.36	0.01\\
21.37	0.01\\
21.38	0.01\\
21.39	0.01\\
21.4	0.01\\
21.41	0.01\\
21.42	0.01\\
21.43	0.01\\
21.44	0.01\\
21.45	0.01\\
21.46	0.01\\
21.47	0.01\\
21.48	0.01\\
21.49	0.01\\
21.5	0.01\\
21.51	0.01\\
21.52	0.01\\
21.53	0.01\\
21.54	0.01\\
21.55	0.01\\
21.56	0.01\\
21.57	0.01\\
21.58	0.01\\
21.59	0.01\\
21.6	0.01\\
21.61	0.01\\
21.62	0.01\\
21.63	0.01\\
21.64	0.01\\
21.65	0.01\\
21.66	0.01\\
21.67	0.01\\
21.68	0.01\\
21.69	0.01\\
21.7	0.01\\
21.71	0.01\\
21.72	0.01\\
21.73	0.01\\
21.74	0.01\\
21.75	0.01\\
21.76	0.01\\
21.77	0.01\\
21.78	0.01\\
21.79	0.01\\
21.8	0.01\\
21.81	0.01\\
21.82	0.01\\
21.83	0.01\\
21.84	0.01\\
21.85	0.01\\
21.86	0.01\\
21.87	0.01\\
21.88	0.01\\
21.89	0.01\\
21.9	0.01\\
21.91	0.01\\
21.92	0.01\\
21.93	0.01\\
21.94	0.01\\
21.95	0.01\\
21.96	0.01\\
21.97	0.01\\
21.98	0.01\\
21.99	0.01\\
22	0.01\\
22.01	0.01\\
22.02	0.01\\
22.03	0.01\\
22.04	0.01\\
22.05	0.01\\
22.06	0.01\\
22.07	0.01\\
22.08	0.01\\
22.09	0.01\\
22.1	0.01\\
22.11	0.01\\
22.12	0.01\\
22.13	0.01\\
22.14	0.01\\
22.15	0.01\\
22.16	0.01\\
22.17	0.01\\
22.18	0.01\\
22.19	0.01\\
22.2	0.01\\
22.21	0.01\\
22.22	0.01\\
22.23	0.01\\
22.24	0.01\\
22.25	0.01\\
22.26	0.01\\
22.27	0.01\\
22.28	0.01\\
22.29	0.01\\
22.3	0.01\\
22.31	0.01\\
22.32	0.01\\
22.33	0.01\\
22.34	0.01\\
22.35	0.01\\
22.36	0.01\\
22.37	0.01\\
22.38	0.01\\
22.39	0.01\\
22.4	0.01\\
22.41	0.01\\
22.42	0.01\\
22.43	0.01\\
22.44	0.01\\
22.45	0.01\\
22.46	0.01\\
22.47	0.01\\
22.48	0.01\\
22.49	0.01\\
22.5	0.01\\
22.51	0.01\\
22.52	0.01\\
22.53	0.01\\
22.54	0.01\\
22.55	0.01\\
22.56	0.01\\
22.57	0.01\\
22.58	0.01\\
22.59	0.01\\
22.6	0.01\\
22.61	0.01\\
22.62	0.01\\
22.63	0.01\\
22.64	0.01\\
22.65	0.01\\
22.66	0.01\\
22.67	0.01\\
22.68	0.01\\
22.69	0.01\\
22.7	0.01\\
22.71	0.01\\
22.72	0.01\\
22.73	0.01\\
22.74	0.01\\
22.75	0.01\\
22.76	0.01\\
22.77	0.01\\
22.78	0.01\\
22.79	0.01\\
22.8	0.01\\
22.81	0.01\\
22.82	0.01\\
22.83	0.01\\
22.84	0.01\\
22.85	0.01\\
22.86	0.01\\
22.87	0.01\\
22.88	0.01\\
22.89	0.01\\
22.9	0.01\\
22.91	0.01\\
22.92	0.01\\
22.93	0.01\\
22.94	0.01\\
22.95	0.01\\
22.96	0.01\\
22.97	0.01\\
22.98	0.01\\
22.99	0.01\\
23	0.01\\
23.01	0.01\\
23.02	0.01\\
23.03	0.01\\
23.04	0.01\\
23.05	0.01\\
23.06	0.01\\
23.07	0.01\\
23.08	0.01\\
23.09	0.01\\
23.1	0.01\\
23.11	0.01\\
23.12	0.01\\
23.13	0.01\\
23.14	0.01\\
23.15	0.01\\
23.16	0.01\\
23.17	0.01\\
23.18	0.01\\
23.19	0.01\\
23.2	0.01\\
23.21	0.01\\
23.22	0.01\\
23.23	0.01\\
23.24	0.01\\
23.25	0.01\\
23.26	0.01\\
23.27	0.01\\
23.28	0.01\\
23.29	0.01\\
23.3	0.01\\
23.31	0.01\\
23.32	0.01\\
23.33	0.01\\
23.34	0.01\\
23.35	0.01\\
23.36	0.01\\
23.37	0.01\\
23.38	0.01\\
23.39	0.01\\
23.4	0.01\\
23.41	0.01\\
23.42	0.01\\
23.43	0.01\\
23.44	0.01\\
23.45	0.01\\
23.46	0.01\\
23.47	0.01\\
23.48	0.01\\
23.49	0.01\\
23.5	0.01\\
23.51	0.01\\
23.52	0.01\\
23.53	0.01\\
23.54	0.01\\
23.55	0.01\\
23.56	0.01\\
23.57	0.01\\
23.58	0.01\\
23.59	0.01\\
23.6	0.01\\
23.61	0.01\\
23.62	0.01\\
23.63	0.01\\
23.64	0.01\\
23.65	0.01\\
23.66	0.01\\
23.67	0.01\\
23.68	0.01\\
23.69	0.01\\
23.7	0.01\\
23.71	0.01\\
23.72	0.01\\
23.73	0.01\\
23.74	0.01\\
23.75	0.01\\
23.76	0.01\\
23.77	0.01\\
23.78	0.01\\
23.79	0.01\\
23.8	0.01\\
23.81	0.01\\
23.82	0.01\\
23.83	0.01\\
23.84	0.01\\
23.85	0.01\\
23.86	0.01\\
23.87	0.01\\
23.88	0.01\\
23.89	0.01\\
23.9	0.01\\
23.91	0.01\\
23.92	0.01\\
23.93	0.01\\
23.94	0.01\\
23.95	0.01\\
23.96	0.01\\
23.97	0.01\\
23.98	0.01\\
23.99	0.01\\
24	0.01\\
24.01	0.01\\
24.02	0.01\\
24.03	0.01\\
24.04	0.01\\
24.05	0.01\\
24.06	0.01\\
24.07	0.01\\
24.08	0.01\\
24.09	0.01\\
24.1	0.01\\
24.11	0.01\\
24.12	0.01\\
24.13	0.01\\
24.14	0.01\\
24.15	0.01\\
24.16	0.01\\
24.17	0.01\\
24.18	0.01\\
24.19	0.01\\
24.2	0.01\\
24.21	0.01\\
24.22	0.01\\
24.23	0.01\\
24.24	0.01\\
24.25	0.01\\
24.26	0.01\\
24.27	0.01\\
24.28	0.01\\
24.29	0.01\\
24.3	0.01\\
24.31	0.01\\
24.32	0.01\\
24.33	0.01\\
24.34	0.01\\
24.35	0.01\\
24.36	0.01\\
24.37	0.01\\
24.38	0.01\\
24.39	0.01\\
24.4	0.01\\
24.41	0.01\\
24.42	0.01\\
24.43	0.01\\
24.44	0.01\\
24.45	0.01\\
24.46	0.01\\
24.47	0.01\\
24.48	0.01\\
24.49	0.01\\
24.5	0.01\\
24.51	0.01\\
24.52	0.01\\
24.53	0.01\\
24.54	0.01\\
24.55	0.01\\
24.56	0.01\\
24.57	0.01\\
24.58	0.01\\
24.59	0.01\\
24.6	0.01\\
24.61	0.01\\
24.62	0.01\\
24.63	0.01\\
24.64	0.01\\
24.65	0.01\\
24.66	0.01\\
24.67	0.01\\
24.68	0.01\\
24.69	0.01\\
24.7	0.01\\
24.71	0.01\\
24.72	0.01\\
24.73	0.01\\
24.74	0.01\\
24.75	0.01\\
24.76	0.01\\
24.77	0.01\\
24.78	0.01\\
24.79	0.01\\
24.8	0.01\\
24.81	0.01\\
24.82	0.01\\
24.83	0.01\\
24.84	0.01\\
24.85	0.01\\
24.86	0.01\\
24.87	0.01\\
24.88	0.01\\
24.89	0.01\\
24.9	0.01\\
24.91	0.01\\
24.92	0.01\\
24.93	0.01\\
24.94	0.01\\
24.95	0.01\\
24.96	0.01\\
24.97	0.01\\
24.98	0.01\\
24.99	0.01\\
25	0.01\\
25.01	0.01\\
25.02	0.01\\
25.03	0.01\\
25.04	0.01\\
25.05	0.01\\
25.06	0.01\\
25.07	0.01\\
25.08	0.01\\
25.09	0.01\\
25.1	0.01\\
25.11	0.01\\
25.12	0.01\\
25.13	0.01\\
25.14	0.01\\
25.15	0.01\\
25.16	0.01\\
25.17	0.01\\
25.18	0.01\\
25.19	0.01\\
25.2	0.01\\
25.21	0.01\\
25.22	0.01\\
25.23	0.01\\
25.24	0.01\\
25.25	0.01\\
25.26	0.01\\
25.27	0.01\\
25.28	0.01\\
25.29	0.01\\
25.3	0.01\\
25.31	0.01\\
25.32	0.01\\
25.33	0.01\\
25.34	0.01\\
25.35	0.01\\
25.36	0.01\\
25.37	0.01\\
25.38	0.01\\
25.39	0.01\\
25.4	0.01\\
25.41	0.01\\
25.42	0.01\\
25.43	0.01\\
25.44	0.01\\
25.45	0.01\\
25.46	0.01\\
25.47	0.01\\
25.48	0.01\\
25.49	0.01\\
25.5	0.01\\
25.51	0.01\\
25.52	0.01\\
25.53	0.01\\
25.54	0.01\\
25.55	0.01\\
25.56	0.01\\
25.57	0.01\\
25.58	0.01\\
25.59	0.01\\
25.6	0.01\\
25.61	0.01\\
25.62	0.01\\
25.63	0.01\\
25.64	0.01\\
25.65	0.01\\
25.66	0.01\\
25.67	0.01\\
25.68	0.01\\
25.69	0.01\\
25.7	0.01\\
25.71	0.01\\
25.72	0.01\\
25.73	0.01\\
25.74	0.01\\
25.75	0.01\\
25.76	0.01\\
25.77	0.01\\
25.78	0.01\\
25.79	0.01\\
25.8	0.01\\
25.81	0.01\\
25.82	0.01\\
25.83	0.01\\
25.84	0.01\\
25.85	0.01\\
25.86	0.01\\
25.87	0.01\\
25.88	0.01\\
25.89	0.01\\
25.9	0.01\\
25.91	0.01\\
25.92	0.01\\
25.93	0.01\\
25.94	0.01\\
25.95	0.01\\
25.96	0.01\\
25.97	0.01\\
25.98	0.01\\
25.99	0.01\\
26	0.01\\
26.01	0.01\\
26.02	0.01\\
26.03	0.01\\
26.04	0.01\\
26.05	0.01\\
26.06	0.01\\
26.07	0.01\\
26.08	0.01\\
26.09	0.01\\
26.1	0.01\\
26.11	0.01\\
26.12	0.01\\
26.13	0.01\\
26.14	0.01\\
26.15	0.01\\
26.16	0.01\\
26.17	0.01\\
26.18	0.01\\
26.19	0.01\\
26.2	0.01\\
26.21	0.01\\
26.22	0.01\\
26.23	0.01\\
26.24	0.01\\
26.25	0.01\\
26.26	0.01\\
26.27	0.01\\
26.28	0.01\\
26.29	0.01\\
26.3	0.01\\
26.31	0.01\\
26.32	0.01\\
26.33	0.01\\
26.34	0.01\\
26.35	0.01\\
26.36	0.01\\
26.37	0.01\\
26.38	0.01\\
26.39	0.01\\
26.4	0.01\\
26.41	0.01\\
26.42	0.01\\
26.43	0.01\\
26.44	0.01\\
26.45	0.01\\
26.46	0.01\\
26.47	0.01\\
26.48	0.01\\
26.49	0.01\\
26.5	0.01\\
26.51	0.01\\
26.52	0.01\\
26.53	0.01\\
26.54	0.01\\
26.55	0.01\\
26.56	0.01\\
26.57	0.01\\
26.58	0.01\\
26.59	0.01\\
26.6	0.01\\
26.61	0.01\\
26.62	0.01\\
26.63	0.01\\
26.64	0.01\\
26.65	0.01\\
26.66	0.01\\
26.67	0.01\\
26.68	0.01\\
26.69	0.01\\
26.7	0.01\\
26.71	0.01\\
26.72	0.01\\
26.73	0.01\\
26.74	0.01\\
26.75	0.01\\
26.76	0.01\\
26.77	0.01\\
26.78	0.01\\
26.79	0.01\\
26.8	0.01\\
26.81	0.01\\
26.82	0.01\\
26.83	0.01\\
26.84	0.01\\
26.85	0.01\\
26.86	0.01\\
26.87	0.01\\
26.88	0.01\\
26.89	0.01\\
26.9	0.01\\
26.91	0.01\\
26.92	0.01\\
26.93	0.01\\
26.94	0.01\\
26.95	0.01\\
26.96	0.01\\
26.97	0.01\\
26.98	0.01\\
26.99	0.01\\
27	0.01\\
27.01	0.01\\
27.02	0.01\\
27.03	0.01\\
27.04	0.01\\
27.05	0.01\\
27.06	0.01\\
27.07	0.01\\
27.08	0.01\\
27.09	0.01\\
27.1	0.01\\
27.11	0.01\\
27.12	0.01\\
27.13	0.01\\
27.14	0.01\\
27.15	0.01\\
27.16	0.01\\
27.17	0.01\\
27.18	0.01\\
27.19	0.01\\
27.2	0.01\\
27.21	0.01\\
27.22	0.01\\
27.23	0.01\\
27.24	0.01\\
27.25	0.01\\
27.26	0.01\\
27.27	0.01\\
27.28	0.01\\
27.29	0.01\\
27.3	0.01\\
27.31	0.01\\
27.32	0.01\\
27.33	0.01\\
27.34	0.01\\
27.35	0.01\\
27.36	0.01\\
27.37	0.01\\
27.38	0.01\\
27.39	0.01\\
27.4	0.01\\
27.41	0.01\\
27.42	0.01\\
27.43	0.01\\
27.44	0.01\\
27.45	0.01\\
27.46	0.01\\
27.47	0.01\\
27.48	0.01\\
27.49	0.01\\
27.5	0.01\\
27.51	0.01\\
27.52	0.01\\
27.53	0.01\\
27.54	0.01\\
27.55	0.01\\
27.56	0.01\\
27.57	0.01\\
27.58	0.01\\
27.59	0.01\\
27.6	0.01\\
27.61	0.01\\
27.62	0.01\\
27.63	0.01\\
27.64	0.01\\
27.65	0.01\\
27.66	0.01\\
27.67	0.01\\
27.68	0.01\\
27.69	0.01\\
27.7	0.01\\
27.71	0.01\\
27.72	0.01\\
27.73	0.01\\
27.74	0.01\\
27.75	0.01\\
27.76	0.01\\
27.77	0.01\\
27.78	0.01\\
27.79	0.01\\
27.8	0.01\\
27.81	0.01\\
27.82	0.01\\
27.83	0.01\\
27.84	0.01\\
27.85	0.01\\
27.86	0.01\\
27.87	0.01\\
27.88	0.01\\
27.89	0.01\\
27.9	0.01\\
27.91	0.01\\
27.92	0.01\\
27.93	0.01\\
27.94	0.01\\
27.95	0.01\\
27.96	0.01\\
27.97	0.01\\
27.98	0.01\\
27.99	0.01\\
28	0.01\\
28.01	0.01\\
28.02	0.01\\
28.03	0.01\\
28.04	0.01\\
28.05	0.01\\
28.06	0.01\\
28.07	0.01\\
28.08	0.01\\
28.09	0.01\\
28.1	0.01\\
28.11	0.01\\
28.12	0.01\\
28.13	0.01\\
28.14	0.01\\
28.15	0.01\\
28.16	0.01\\
28.17	0.01\\
28.18	0.01\\
28.19	0.01\\
28.2	0.01\\
28.21	0.01\\
28.22	0.01\\
28.23	0.01\\
28.24	0.01\\
28.25	0.01\\
28.26	0.01\\
28.27	0.01\\
28.28	0.01\\
28.29	0.01\\
28.3	0.01\\
28.31	0.01\\
28.32	0.01\\
28.33	0.01\\
28.34	0.01\\
28.35	0.01\\
28.36	0.01\\
28.37	0.01\\
28.38	0.01\\
28.39	0.01\\
28.4	0.01\\
28.41	0.01\\
28.42	0.01\\
28.43	0.01\\
28.44	0.01\\
28.45	0.01\\
28.46	0.01\\
28.47	0.01\\
28.48	0.01\\
28.49	0.01\\
28.5	0.01\\
28.51	0.01\\
28.52	0.01\\
28.53	0.01\\
28.54	0.01\\
28.55	0.01\\
28.56	0.01\\
28.57	0.01\\
28.58	0.01\\
28.59	0.01\\
28.6	0.01\\
28.61	0.01\\
28.62	0.01\\
28.63	0.01\\
28.64	0.01\\
28.65	0.01\\
28.66	0.01\\
28.67	0.01\\
28.68	0.01\\
28.69	0.01\\
28.7	0.01\\
28.71	0.01\\
28.72	0.01\\
28.73	0.01\\
28.74	0.01\\
28.75	0.01\\
28.76	0.01\\
28.77	0.01\\
28.78	0.01\\
28.79	0.01\\
28.8	0.01\\
28.81	0.01\\
28.82	0.01\\
28.83	0.01\\
28.84	0.01\\
28.85	0.01\\
28.86	0.01\\
28.87	0.01\\
28.88	0.01\\
28.89	0.01\\
28.9	0.01\\
28.91	0.01\\
28.92	0.01\\
28.93	0.01\\
28.94	0.01\\
28.95	0.01\\
28.96	0.01\\
28.97	0.01\\
28.98	0.01\\
28.99	0.01\\
29	0.01\\
29.01	0.01\\
29.02	0.01\\
29.03	0.01\\
29.04	0.01\\
29.05	0.01\\
29.06	0.01\\
29.07	0.01\\
29.08	0.01\\
29.09	0.01\\
29.1	0.01\\
29.11	0.01\\
29.12	0.01\\
29.13	0.01\\
29.14	0.01\\
29.15	0.01\\
29.16	0.01\\
29.17	0.01\\
29.18	0.01\\
29.19	0.01\\
29.2	0.01\\
29.21	0.01\\
29.22	0.01\\
29.23	0.01\\
29.24	0.01\\
29.25	0.01\\
29.26	0.01\\
29.27	0.01\\
29.28	0.01\\
29.29	0.01\\
29.3	0.01\\
29.31	0.01\\
29.32	0.01\\
29.33	0.01\\
29.34	0.01\\
29.35	0.01\\
29.36	0.01\\
29.37	0.01\\
29.38	0.01\\
29.39	0.01\\
29.4	0.01\\
29.41	0.01\\
29.42	0.01\\
29.43	0.01\\
29.44	0.01\\
29.45	0.01\\
29.46	0.01\\
29.47	0.01\\
29.48	0.01\\
29.49	0.01\\
29.5	0.01\\
29.51	0.01\\
29.52	0.01\\
29.53	0.01\\
29.54	0.01\\
29.55	0.01\\
29.56	0.01\\
29.57	0.01\\
29.58	0.01\\
29.59	0.01\\
29.6	0.01\\
29.61	0.01\\
29.62	0.01\\
29.63	0.01\\
29.64	0.01\\
29.65	0.01\\
29.66	0.01\\
29.67	0.01\\
29.68	0.01\\
29.69	0.01\\
29.7	0.01\\
29.71	0.01\\
29.72	0.01\\
29.73	0.01\\
29.74	0.01\\
29.75	0.01\\
29.76	0.01\\
29.77	0.01\\
29.78	0.01\\
29.79	0.01\\
29.8	0.01\\
29.81	0.01\\
29.82	0.01\\
29.83	0.01\\
29.84	0.01\\
29.85	0.01\\
29.86	0.01\\
29.87	0.01\\
29.88	0.01\\
29.89	0.01\\
29.9	0.01\\
29.91	0.01\\
29.92	0.01\\
29.93	0.01\\
29.94	0.01\\
29.95	0.01\\
29.96	0.01\\
29.97	0.01\\
29.98	0.01\\
29.99	0.01\\
30	0.01\\
30.01	0.01\\
30.02	0.01\\
30.03	0.01\\
30.04	0.01\\
30.05	0.01\\
30.06	0.01\\
30.07	0.01\\
30.08	0.01\\
30.09	0.01\\
30.1	0.01\\
30.11	0.01\\
30.12	0.01\\
30.13	0.01\\
30.14	0.01\\
30.15	0.01\\
30.16	0.01\\
30.17	0.01\\
30.18	0.01\\
30.19	0.01\\
30.2	0.01\\
30.21	0.01\\
30.22	0.01\\
30.23	0.01\\
30.24	0.01\\
30.25	0.01\\
30.26	0.01\\
30.27	0.01\\
30.28	0.01\\
30.29	0.01\\
30.3	0.01\\
30.31	0.01\\
30.32	0.01\\
30.33	0.01\\
30.34	0.01\\
30.35	0.01\\
30.36	0.01\\
30.37	0.01\\
30.38	0.01\\
30.39	0.01\\
30.4	0.01\\
30.41	0.01\\
30.42	0.01\\
30.43	0.01\\
30.44	0.01\\
30.45	0.01\\
30.46	0.01\\
30.47	0.01\\
30.48	0.01\\
30.49	0.01\\
30.5	0.01\\
30.51	0.01\\
30.52	0.01\\
30.53	0.01\\
30.54	0.01\\
30.55	0.01\\
30.56	0.01\\
30.57	0.01\\
30.58	0.01\\
30.59	0.01\\
30.6	0.01\\
30.61	0.01\\
30.62	0.01\\
30.63	0.01\\
30.64	0.01\\
30.65	0.01\\
30.66	0.01\\
30.67	0.01\\
30.68	0.01\\
30.69	0.01\\
30.7	0.01\\
30.71	0.01\\
30.72	0.01\\
30.73	0.01\\
30.74	0.01\\
30.75	0.01\\
30.76	0.01\\
30.77	0.01\\
30.78	0.01\\
30.79	0.01\\
30.8	0.01\\
30.81	0.01\\
30.82	0.01\\
30.83	0.01\\
30.84	0.01\\
30.85	0.01\\
30.86	0.01\\
30.87	0.01\\
30.88	0.01\\
30.89	0.01\\
30.9	0.01\\
30.91	0.01\\
30.92	0.01\\
30.93	0.01\\
30.94	0.01\\
30.95	0.01\\
30.96	0.01\\
30.97	0.01\\
30.98	0.01\\
30.99	0.01\\
31	0.01\\
31.01	0.01\\
31.02	0.01\\
31.03	0.01\\
31.04	0.01\\
31.05	0.01\\
31.06	0.01\\
31.07	0.01\\
31.08	0.01\\
31.09	0.01\\
31.1	0.01\\
31.11	0.01\\
31.12	0.01\\
31.13	0.01\\
31.14	0.01\\
31.15	0.01\\
31.16	0.01\\
31.17	0.01\\
31.18	0.01\\
31.19	0.01\\
31.2	0.01\\
31.21	0.01\\
31.22	0.01\\
31.23	0.01\\
31.24	0.01\\
31.25	0.01\\
31.26	0.01\\
31.27	0.01\\
31.28	0.01\\
31.29	0.01\\
31.3	0.01\\
31.31	0.01\\
31.32	0.01\\
31.33	0.01\\
31.34	0.01\\
31.35	0.01\\
31.36	0.01\\
31.37	0.01\\
31.38	0.01\\
31.39	0.01\\
31.4	0.01\\
31.41	0.01\\
31.42	0.01\\
31.43	0.01\\
31.44	0.01\\
31.45	0.01\\
31.46	0.01\\
31.47	0.01\\
31.48	0.01\\
31.49	0.01\\
31.5	0.01\\
31.51	0.01\\
31.52	0.01\\
31.53	0.01\\
31.54	0.01\\
31.55	0.01\\
31.56	0.01\\
31.57	0.01\\
31.58	0.01\\
31.59	0.01\\
31.6	0.01\\
31.61	0.01\\
31.62	0.01\\
31.63	0.01\\
31.64	0.01\\
31.65	0.01\\
31.66	0.01\\
31.67	0.01\\
31.68	0.01\\
31.69	0.01\\
31.7	0.01\\
31.71	0.01\\
31.72	0.01\\
31.73	0.01\\
31.74	0.01\\
31.75	0.01\\
31.76	0.01\\
31.77	0.01\\
31.78	0.01\\
31.79	0.01\\
31.8	0.01\\
31.81	0.01\\
31.82	0.01\\
31.83	0.01\\
31.84	0.01\\
31.85	0.01\\
31.86	0.01\\
31.87	0.01\\
31.88	0.01\\
31.89	0.01\\
31.9	0.01\\
31.91	0.01\\
31.92	0.01\\
31.93	0.01\\
31.94	0.01\\
31.95	0.01\\
31.96	0.01\\
31.97	0.01\\
31.98	0.01\\
31.99	0.01\\
32	0.01\\
32.01	0.01\\
32.02	0.01\\
32.03	0.01\\
32.04	0.01\\
32.05	0.01\\
32.06	0.01\\
32.07	0.01\\
32.08	0.01\\
32.09	0.01\\
32.1	0.01\\
32.11	0.01\\
32.12	0.01\\
32.13	0.01\\
32.14	0.01\\
32.15	0.01\\
32.16	0.01\\
32.17	0.01\\
32.18	0.01\\
32.19	0.01\\
32.2	0.01\\
32.21	0.01\\
32.22	0.01\\
32.23	0.01\\
32.24	0.01\\
32.25	0.01\\
32.26	0.01\\
32.27	0.01\\
32.28	0.01\\
32.29	0.01\\
32.3	0.01\\
32.31	0.01\\
32.32	0.01\\
32.33	0.01\\
32.34	0.01\\
32.35	0.01\\
32.36	0.01\\
32.37	0.01\\
32.38	0.01\\
32.39	0.01\\
32.4	0.01\\
32.41	0.01\\
32.42	0.01\\
32.43	0.01\\
32.44	0.01\\
32.45	0.01\\
32.46	0.01\\
32.47	0.01\\
32.48	0.01\\
32.49	0.01\\
32.5	0.01\\
32.51	0.01\\
32.52	0.01\\
32.53	0.01\\
32.54	0.01\\
32.55	0.01\\
32.56	0.01\\
32.57	0.01\\
32.58	0.01\\
32.59	0.01\\
32.6	0.01\\
32.61	0.01\\
32.62	0.01\\
32.63	0.01\\
32.64	0.01\\
32.65	0.01\\
32.66	0.01\\
32.67	0.01\\
32.68	0.01\\
32.69	0.01\\
32.7	0.01\\
32.71	0.01\\
32.72	0.01\\
32.73	0.01\\
32.74	0.01\\
32.75	0.01\\
32.76	0.01\\
32.77	0.01\\
32.78	0.01\\
32.79	0.01\\
32.8	0.01\\
32.81	0.01\\
32.82	0.01\\
32.83	0.01\\
32.84	0.01\\
32.85	0.01\\
32.86	0.01\\
32.87	0.01\\
32.88	0.01\\
32.89	0.01\\
32.9	0.01\\
32.91	0.01\\
32.92	0.01\\
32.93	0.01\\
32.94	0.01\\
32.95	0.01\\
32.96	0.01\\
32.97	0.01\\
32.98	0.01\\
32.99	0.01\\
33	0.01\\
33.01	0.01\\
33.02	0.01\\
33.03	0.01\\
33.04	0.01\\
33.05	0.01\\
33.06	0.01\\
33.07	0.01\\
33.08	0.01\\
33.09	0.01\\
33.1	0.01\\
33.11	0.01\\
33.12	0.01\\
33.13	0.01\\
33.14	0.01\\
33.15	0.01\\
33.16	0.01\\
33.17	0.01\\
33.18	0.01\\
33.19	0.01\\
33.2	0.01\\
33.21	0.01\\
33.22	0.01\\
33.23	0.01\\
33.24	0.01\\
33.25	0.01\\
33.26	0.01\\
33.27	0.01\\
33.28	0.01\\
33.29	0.01\\
33.3	0.01\\
33.31	0.01\\
33.32	0.01\\
33.33	0.01\\
33.34	0.01\\
33.35	0.01\\
33.36	0.01\\
33.37	0.01\\
33.38	0.01\\
33.39	0.01\\
33.4	0.01\\
33.41	0.01\\
33.42	0.01\\
33.43	0.01\\
33.44	0.01\\
33.45	0.01\\
33.46	0.01\\
33.47	0.01\\
33.48	0.01\\
33.49	0.01\\
33.5	0.01\\
33.51	0.01\\
33.52	0.01\\
33.53	0.01\\
33.54	0.01\\
33.55	0.01\\
33.56	0.01\\
33.57	0.01\\
33.58	0.01\\
33.59	0.01\\
33.6	0.01\\
33.61	0.01\\
33.62	0.01\\
33.63	0.01\\
33.64	0.01\\
33.65	0.01\\
33.66	0.01\\
33.67	0.01\\
33.68	0.01\\
33.69	0.01\\
33.7	0.01\\
33.71	0.01\\
33.72	0.01\\
33.73	0.01\\
33.74	0.01\\
33.75	0.01\\
33.76	0.01\\
33.77	0.01\\
33.78	0.01\\
33.79	0.01\\
33.8	0.01\\
33.81	0.01\\
33.82	0.01\\
33.83	0.01\\
33.84	0.01\\
33.85	0.01\\
33.86	0.01\\
33.87	0.01\\
33.88	0.01\\
33.89	0.01\\
33.9	0.01\\
33.91	0.01\\
33.92	0.01\\
33.93	0.01\\
33.94	0.01\\
33.95	0.01\\
33.96	0.01\\
33.97	0.01\\
33.98	0.01\\
33.99	0.01\\
34	0.01\\
34.01	0.01\\
34.02	0.01\\
34.03	0.01\\
34.04	0.01\\
34.05	0.01\\
34.06	0.01\\
34.07	0.01\\
34.08	0.01\\
34.09	0.01\\
34.1	0.01\\
34.11	0.01\\
34.12	0.01\\
34.13	0.01\\
34.14	0.01\\
34.15	0.01\\
34.16	0.01\\
34.17	0.01\\
34.18	0.01\\
34.19	0.01\\
34.2	0.01\\
34.21	0.01\\
34.22	0.01\\
34.23	0.01\\
34.24	0.01\\
34.25	0.01\\
34.26	0.01\\
34.27	0.01\\
34.28	0.01\\
34.29	0.01\\
34.3	0.01\\
34.31	0.01\\
34.32	0.01\\
34.33	0.01\\
34.34	0.01\\
34.35	0.01\\
34.36	0.01\\
34.37	0.01\\
34.38	0.01\\
34.39	0.01\\
34.4	0.01\\
34.41	0.01\\
34.42	0.01\\
34.43	0.01\\
34.44	0.01\\
34.45	0.01\\
34.46	0.01\\
34.47	0.01\\
34.48	0.01\\
34.49	0.01\\
34.5	0.01\\
34.51	0.01\\
34.52	0.01\\
34.53	0.01\\
34.54	0.01\\
34.55	0.01\\
34.56	0.01\\
34.57	0.01\\
34.58	0.01\\
34.59	0.01\\
34.6	0.01\\
34.61	0.01\\
34.62	0.01\\
34.63	0.01\\
34.64	0.01\\
34.65	0.01\\
34.66	0.01\\
34.67	0.01\\
34.68	0.01\\
34.69	0.01\\
34.7	0.01\\
34.71	0.01\\
34.72	0.01\\
34.73	0.01\\
34.74	0.01\\
34.75	0.01\\
34.76	0.01\\
34.77	0.01\\
34.78	0.01\\
34.79	0.01\\
34.8	0.01\\
34.81	0.01\\
34.82	0.01\\
34.83	0.01\\
34.84	0.01\\
34.85	0.01\\
34.86	0.01\\
34.87	0.01\\
34.88	0.01\\
34.89	0.01\\
34.9	0.01\\
34.91	0.01\\
34.92	0.01\\
34.93	0.01\\
34.94	0.01\\
34.95	0.01\\
34.96	0.01\\
34.97	0.01\\
34.98	0.01\\
34.99	0.01\\
35	0.01\\
35.01	0.01\\
35.02	0.01\\
35.03	0.01\\
35.04	0.01\\
35.05	0.01\\
35.06	0.01\\
35.07	0.01\\
35.08	0.01\\
35.09	0.01\\
35.1	0.01\\
35.11	0.01\\
35.12	0.01\\
35.13	0.01\\
35.14	0.01\\
35.15	0.01\\
35.16	0.01\\
35.17	0.01\\
35.18	0.01\\
35.19	0.01\\
35.2	0.01\\
35.21	0.01\\
35.22	0.01\\
35.23	0.01\\
35.24	0.01\\
35.25	0.01\\
35.26	0.01\\
35.27	0.01\\
35.28	0.01\\
35.29	0.01\\
35.3	0.01\\
35.31	0.01\\
35.32	0.01\\
35.33	0.01\\
35.34	0.01\\
35.35	0.01\\
35.36	0.01\\
35.37	0.01\\
35.38	0.01\\
35.39	0.01\\
35.4	0.01\\
35.41	0.01\\
35.42	0.01\\
35.43	0.01\\
35.44	0.01\\
35.45	0.01\\
35.46	0.01\\
35.47	0.01\\
35.48	0.01\\
35.49	0.01\\
35.5	0.01\\
35.51	0.01\\
35.52	0.01\\
35.53	0.01\\
35.54	0.01\\
35.55	0.01\\
35.56	0.01\\
35.57	0.01\\
35.58	0.01\\
35.59	0.01\\
35.6	0.01\\
35.61	0.01\\
35.62	0.01\\
35.63	0.01\\
35.64	0.01\\
35.65	0.01\\
35.66	0.01\\
35.67	0.01\\
35.68	0.01\\
35.69	0.01\\
35.7	0.01\\
35.71	0.01\\
35.72	0.01\\
35.73	0.01\\
35.74	0.01\\
35.75	0.01\\
35.76	0.01\\
35.77	0.01\\
35.78	0.01\\
35.79	0.01\\
35.8	0.01\\
35.81	0.01\\
35.82	0.01\\
35.83	0.01\\
35.84	0.01\\
35.85	0.01\\
35.86	0.01\\
35.87	0.01\\
35.88	0.01\\
35.89	0.01\\
35.9	0.01\\
35.91	0.01\\
35.92	0.01\\
35.93	0.01\\
35.94	0.01\\
35.95	0.01\\
35.96	0.01\\
35.97	0.01\\
35.98	0.01\\
35.99	0.01\\
36	0.01\\
36.01	0.01\\
36.02	0.01\\
36.03	0.01\\
36.04	0.01\\
36.05	0.01\\
36.06	0.01\\
36.07	0.01\\
36.08	0.01\\
36.09	0.01\\
36.1	0.01\\
36.11	0.01\\
36.12	0.01\\
36.13	0.01\\
36.14	0.01\\
36.15	0.01\\
36.16	0.01\\
36.17	0.01\\
36.18	0.01\\
36.19	0.01\\
36.2	0.01\\
36.21	0.01\\
36.22	0.01\\
36.23	0.01\\
36.24	0.01\\
36.25	0.01\\
36.26	0.01\\
36.27	0.01\\
36.28	0.01\\
36.29	0.01\\
36.3	0.01\\
36.31	0.01\\
36.32	0.01\\
36.33	0.01\\
36.34	0.01\\
36.35	0.01\\
36.36	0.01\\
36.37	0.01\\
36.38	0.01\\
36.39	0.01\\
36.4	0.01\\
36.41	0.01\\
36.42	0.01\\
36.43	0.01\\
36.44	0.01\\
36.45	0.01\\
36.46	0.01\\
36.47	0.01\\
36.48	0.01\\
36.49	0.01\\
36.5	0.01\\
36.51	0.01\\
36.52	0.01\\
36.53	0.01\\
36.54	0.01\\
36.55	0.01\\
36.56	0.01\\
36.57	0.01\\
36.58	0.01\\
36.59	0.01\\
36.6	0.01\\
36.61	0.01\\
36.62	0.01\\
36.63	0.01\\
36.64	0.01\\
36.65	0.01\\
36.66	0.01\\
36.67	0.01\\
36.68	0.01\\
36.69	0.01\\
36.7	0.01\\
36.71	0.01\\
36.72	0.01\\
36.73	0.01\\
36.74	0.01\\
36.75	0.01\\
36.76	0.01\\
36.77	0.01\\
36.78	0.01\\
36.79	0.01\\
36.8	0.01\\
36.81	0.01\\
36.82	0.01\\
36.83	0.01\\
36.84	0.01\\
36.85	0.01\\
36.86	0.01\\
36.87	0.01\\
36.88	0.01\\
36.89	0.01\\
36.9	0.01\\
36.91	0.01\\
36.92	0.01\\
36.93	0.01\\
36.94	0.01\\
36.95	0.01\\
36.96	0.01\\
36.97	0.01\\
36.98	0.01\\
36.99	0.01\\
37	0.01\\
37.01	0.01\\
37.02	0.01\\
37.03	0.01\\
37.04	0.01\\
37.05	0.01\\
37.06	0.01\\
37.07	0.01\\
37.08	0.01\\
37.09	0.01\\
37.1	0.01\\
37.11	0.01\\
37.12	0.01\\
37.13	0.01\\
37.14	0.01\\
37.15	0.01\\
37.16	0.01\\
37.17	0.01\\
37.18	0.01\\
37.19	0.01\\
37.2	0.01\\
37.21	0.01\\
37.22	0.01\\
37.23	0.01\\
37.24	0.01\\
37.25	0.01\\
37.26	0.01\\
37.27	0.01\\
37.28	0.01\\
37.29	0.01\\
37.3	0.01\\
37.31	0.01\\
37.32	0.01\\
37.33	0.01\\
37.34	0.01\\
37.35	0.01\\
37.36	0.01\\
37.37	0.01\\
37.38	0.01\\
37.39	0.01\\
37.4	0.01\\
37.41	0.01\\
37.42	0.01\\
37.43	0.01\\
37.44	0.01\\
37.45	0.01\\
37.46	0.01\\
37.47	0.01\\
37.48	0.01\\
37.49	0.01\\
37.5	0.01\\
37.51	0.01\\
37.52	0.01\\
37.53	0.01\\
37.54	0.01\\
37.55	0.01\\
37.56	0.01\\
37.57	0.01\\
37.58	0.01\\
37.59	0.01\\
37.6	0.01\\
37.61	0.01\\
37.62	0.01\\
37.63	0.01\\
37.64	0.01\\
37.65	0.01\\
37.66	0.01\\
37.67	0.01\\
37.68	0.01\\
37.69	0.01\\
37.7	0.01\\
37.71	0.01\\
37.72	0.01\\
37.73	0.01\\
37.74	0.01\\
37.75	0.01\\
37.76	0.01\\
37.77	0.01\\
37.78	0.01\\
37.79	0.01\\
37.8	0.01\\
37.81	0.01\\
37.82	0.01\\
37.83	0.01\\
37.84	0.01\\
37.85	0.01\\
37.86	0.01\\
37.87	0.01\\
37.88	0.01\\
37.89	0.01\\
37.9	0.01\\
37.91	0.01\\
37.92	0.01\\
37.93	0.01\\
37.94	0.01\\
37.95	0.01\\
37.96	0.01\\
37.97	0.01\\
37.98	0.01\\
37.99	0.01\\
38	0.01\\
38.01	0.01\\
38.02	0.01\\
38.03	0.01\\
38.04	0.01\\
38.05	0.01\\
38.06	0.01\\
38.07	0.01\\
38.08	0.01\\
38.09	0.01\\
38.1	0.01\\
38.11	0.01\\
38.12	0.01\\
38.13	0.01\\
38.14	0.01\\
38.15	0.01\\
38.16	0.01\\
38.17	0.01\\
38.18	0.01\\
38.19	0.01\\
38.2	0.01\\
38.21	0.01\\
38.22	0.01\\
38.23	0.01\\
38.24	0.01\\
38.25	0.01\\
38.26	0.01\\
38.27	0.01\\
38.28	0.01\\
38.29	0.01\\
38.3	0.01\\
38.31	0.01\\
38.32	0.01\\
38.33	0.01\\
38.34	0.01\\
38.35	0.01\\
38.36	0.01\\
38.37	0.01\\
38.38	0.01\\
38.39	0.01\\
38.4	0.01\\
38.41	0.01\\
38.42	0.01\\
38.43	0.01\\
38.44	0.01\\
38.45	0.01\\
38.46	0.01\\
38.47	0.01\\
38.48	0.01\\
38.49	0.01\\
38.5	0.01\\
38.51	0.01\\
38.52	0.01\\
38.53	0.01\\
38.54	0.01\\
38.55	0.01\\
38.56	0.01\\
38.57	0.01\\
38.58	0.01\\
38.59	0.01\\
38.6	0.01\\
38.61	0.01\\
38.62	0.01\\
38.63	0.01\\
38.64	0.01\\
38.65	0.01\\
38.66	0.01\\
38.67	0.01\\
38.68	0.01\\
38.69	0.01\\
38.7	0.01\\
38.71	0.01\\
38.72	0.01\\
38.73	0.01\\
38.74	0.01\\
38.75	0.01\\
38.76	0.01\\
38.77	0.01\\
38.78	0.01\\
38.79	0.01\\
38.8	0.01\\
38.81	0.01\\
38.82	0.01\\
38.83	0.01\\
38.84	0.01\\
38.85	0.01\\
38.86	0.01\\
38.87	0.01\\
38.88	0.01\\
38.89	0.01\\
38.9	0.01\\
38.91	0.01\\
38.92	0.01\\
38.93	0.01\\
38.94	0.01\\
38.95	0.01\\
38.96	0.01\\
38.97	0.01\\
38.98	0.01\\
38.99	0.01\\
39	0.01\\
39.01	0.01\\
39.02	0.01\\
39.03	0.01\\
39.04	0.01\\
39.05	0.01\\
39.06	0.01\\
39.07	0.01\\
39.08	0.01\\
39.09	0.01\\
39.1	0.01\\
39.11	0.01\\
39.12	0.01\\
39.13	0.01\\
39.14	0.01\\
39.15	0.01\\
39.16	0.01\\
39.17	0.01\\
39.18	0.01\\
39.19	0.01\\
39.2	0.01\\
39.21	0.01\\
39.22	0.01\\
39.23	0.01\\
39.24	0.01\\
39.25	0.01\\
39.26	0.01\\
39.27	0.01\\
39.28	0.01\\
39.29	0.01\\
39.3	0.01\\
39.31	0.01\\
39.32	0.01\\
39.33	0.01\\
39.34	0.01\\
39.35	0.01\\
39.36	0.01\\
39.37	0.01\\
39.38	0.01\\
39.39	0.01\\
39.4	0.01\\
39.41	0.01\\
39.42	0.01\\
39.43	0.01\\
39.44	0.01\\
39.45	0.01\\
39.46	0.01\\
39.47	0.01\\
39.48	0.01\\
39.49	0.01\\
39.5	0.01\\
39.51	0.01\\
39.52	0.01\\
39.53	0.01\\
39.54	0.01\\
39.55	0.01\\
39.56	0.01\\
39.57	0.01\\
39.58	0.01\\
39.59	0.01\\
39.6	0.01\\
39.61	0.01\\
39.62	0.01\\
39.63	0.01\\
39.64	0.01\\
39.65	0.01\\
39.66	0.01\\
39.67	0.01\\
39.68	0.01\\
39.69	0.01\\
39.7	0.01\\
39.71	0.01\\
39.72	0.01\\
39.73	0.01\\
39.74	0.01\\
39.75	0.01\\
39.76	0.01\\
39.77	0.01\\
39.78	0.01\\
39.79	0.01\\
39.8	0.01\\
39.81	0.01\\
39.82	0.01\\
39.83	0.01\\
39.84	0.01\\
39.85	0.01\\
39.86	0.01\\
39.87	0.01\\
39.88	0.01\\
39.89	0.01\\
39.9	0.01\\
39.91	0.01\\
39.92	0.01\\
39.93	0.01\\
39.94	0.01\\
39.95	0.01\\
39.96	0.01\\
39.97	0.01\\
39.98	0.01\\
39.99	0.01\\
40	0.01\\
40.01	0.01\\
};
\addplot [color=green,solid,forget plot]
  table[row sep=crcr]{%
40.01	0.01\\
40.02	0.01\\
40.03	0.01\\
40.04	0.01\\
40.05	0.01\\
40.06	0.01\\
40.07	0.01\\
40.08	0.01\\
40.09	0.01\\
40.1	0.01\\
40.11	0.01\\
40.12	0.01\\
40.13	0.01\\
40.14	0.01\\
40.15	0.01\\
40.16	0.01\\
40.17	0.01\\
40.18	0.01\\
40.19	0.01\\
40.2	0.01\\
40.21	0.01\\
40.22	0.01\\
40.23	0.01\\
40.24	0.01\\
40.25	0.01\\
40.26	0.01\\
40.27	0.01\\
40.28	0.01\\
40.29	0.01\\
40.3	0.01\\
40.31	0.01\\
40.32	0.01\\
40.33	0.01\\
40.34	0.01\\
40.35	0.01\\
40.36	0.01\\
40.37	0.01\\
40.38	0.01\\
40.39	0.01\\
40.4	0.01\\
40.41	0.01\\
40.42	0.01\\
40.43	0.01\\
40.44	0.01\\
40.45	0.01\\
40.46	0.01\\
40.47	0.01\\
40.48	0.01\\
40.49	0.01\\
40.5	0.01\\
40.51	0.01\\
40.52	0.01\\
40.53	0.01\\
40.54	0.01\\
40.55	0.01\\
40.56	0.01\\
40.57	0.01\\
40.58	0.01\\
40.59	0.01\\
40.6	0.01\\
40.61	0.01\\
40.62	0.01\\
40.63	0.01\\
40.64	0.01\\
40.65	0.01\\
40.66	0.01\\
40.67	0.01\\
40.68	0.01\\
40.69	0.01\\
40.7	0.01\\
40.71	0.01\\
40.72	0.01\\
40.73	0.01\\
40.74	0.01\\
40.75	0.01\\
40.76	0.01\\
40.77	0.01\\
40.78	0.01\\
40.79	0.01\\
40.8	0.01\\
40.81	0.01\\
40.82	0.01\\
40.83	0.01\\
40.84	0.01\\
40.85	0.01\\
40.86	0.01\\
40.87	0.01\\
40.88	0.01\\
40.89	0.01\\
40.9	0.01\\
40.91	0.01\\
40.92	0.01\\
40.93	0.01\\
40.94	0.01\\
40.95	0.01\\
40.96	0.01\\
40.97	0.01\\
40.98	0.01\\
40.99	0.01\\
41	0.01\\
41.01	0.01\\
41.02	0.01\\
41.03	0.01\\
41.04	0.01\\
41.05	0.01\\
41.06	0.01\\
41.07	0.01\\
41.08	0.01\\
41.09	0.01\\
41.1	0.01\\
41.11	0.01\\
41.12	0.01\\
41.13	0.01\\
41.14	0.01\\
41.15	0.01\\
41.16	0.01\\
41.17	0.01\\
41.18	0.01\\
41.19	0.01\\
41.2	0.01\\
41.21	0.01\\
41.22	0.01\\
41.23	0.01\\
41.24	0.01\\
41.25	0.01\\
41.26	0.01\\
41.27	0.01\\
41.28	0.01\\
41.29	0.01\\
41.3	0.01\\
41.31	0.01\\
41.32	0.01\\
41.33	0.01\\
41.34	0.01\\
41.35	0.01\\
41.36	0.01\\
41.37	0.01\\
41.38	0.01\\
41.39	0.01\\
41.4	0.01\\
41.41	0.01\\
41.42	0.01\\
41.43	0.01\\
41.44	0.01\\
41.45	0.01\\
41.46	0.01\\
41.47	0.01\\
41.48	0.01\\
41.49	0.01\\
41.5	0.01\\
41.51	0.01\\
41.52	0.01\\
41.53	0.01\\
41.54	0.01\\
41.55	0.01\\
41.56	0.01\\
41.57	0.01\\
41.58	0.01\\
41.59	0.01\\
41.6	0.01\\
41.61	0.01\\
41.62	0.01\\
41.63	0.01\\
41.64	0.01\\
41.65	0.01\\
41.66	0.01\\
41.67	0.01\\
41.68	0.01\\
41.69	0.01\\
41.7	0.01\\
41.71	0.01\\
41.72	0.01\\
41.73	0.01\\
41.74	0.01\\
41.75	0.01\\
41.76	0.01\\
41.77	0.01\\
41.78	0.01\\
41.79	0.01\\
41.8	0.01\\
41.81	0.01\\
41.82	0.01\\
41.83	0.01\\
41.84	0.01\\
41.85	0.01\\
41.86	0.01\\
41.87	0.01\\
41.88	0.01\\
41.89	0.01\\
41.9	0.01\\
41.91	0.01\\
41.92	0.01\\
41.93	0.01\\
41.94	0.01\\
41.95	0.01\\
41.96	0.01\\
41.97	0.01\\
41.98	0.01\\
41.99	0.01\\
42	0.01\\
42.01	0.01\\
42.02	0.01\\
42.03	0.01\\
42.04	0.01\\
42.05	0.01\\
42.06	0.01\\
42.07	0.01\\
42.08	0.01\\
42.09	0.01\\
42.1	0.01\\
42.11	0.01\\
42.12	0.01\\
42.13	0.01\\
42.14	0.01\\
42.15	0.01\\
42.16	0.01\\
42.17	0.01\\
42.18	0.01\\
42.19	0.01\\
42.2	0.01\\
42.21	0.01\\
42.22	0.01\\
42.23	0.01\\
42.24	0.01\\
42.25	0.01\\
42.26	0.01\\
42.27	0.01\\
42.28	0.01\\
42.29	0.01\\
42.3	0.01\\
42.31	0.01\\
42.32	0.01\\
42.33	0.01\\
42.34	0.01\\
42.35	0.01\\
42.36	0.01\\
42.37	0.01\\
42.38	0.01\\
42.39	0.01\\
42.4	0.01\\
42.41	0.01\\
42.42	0.01\\
42.43	0.01\\
42.44	0.01\\
42.45	0.01\\
42.46	0.01\\
42.47	0.01\\
42.48	0.01\\
42.49	0.01\\
42.5	0.01\\
42.51	0.01\\
42.52	0.01\\
42.53	0.01\\
42.54	0.01\\
42.55	0.01\\
42.56	0.01\\
42.57	0.01\\
42.58	0.01\\
42.59	0.01\\
42.6	0.01\\
42.61	0.01\\
42.62	0.01\\
42.63	0.01\\
42.64	0.01\\
42.65	0.01\\
42.66	0.01\\
42.67	0.01\\
42.68	0.01\\
42.69	0.01\\
42.7	0.01\\
42.71	0.01\\
42.72	0.01\\
42.73	0.01\\
42.74	0.01\\
42.75	0.01\\
42.76	0.01\\
42.77	0.01\\
42.78	0.01\\
42.79	0.01\\
42.8	0.01\\
42.81	0.01\\
42.82	0.01\\
42.83	0.01\\
42.84	0.01\\
42.85	0.01\\
42.86	0.01\\
42.87	0.01\\
42.88	0.01\\
42.89	0.01\\
42.9	0.01\\
42.91	0.01\\
42.92	0.01\\
42.93	0.01\\
42.94	0.01\\
42.95	0.01\\
42.96	0.01\\
42.97	0.01\\
42.98	0.01\\
42.99	0.01\\
43	0.01\\
43.01	0.01\\
43.02	0.01\\
43.03	0.01\\
43.04	0.01\\
43.05	0.01\\
43.06	0.01\\
43.07	0.01\\
43.08	0.01\\
43.09	0.01\\
43.1	0.01\\
43.11	0.01\\
43.12	0.01\\
43.13	0.01\\
43.14	0.01\\
43.15	0.01\\
43.16	0.01\\
43.17	0.01\\
43.18	0.01\\
43.19	0.01\\
43.2	0.01\\
43.21	0.01\\
43.22	0.01\\
43.23	0.01\\
43.24	0.01\\
43.25	0.01\\
43.26	0.01\\
43.27	0.01\\
43.28	0.01\\
43.29	0.01\\
43.3	0.01\\
43.31	0.01\\
43.32	0.01\\
43.33	0.01\\
43.34	0.01\\
43.35	0.01\\
43.36	0.01\\
43.37	0.01\\
43.38	0.01\\
43.39	0.01\\
43.4	0.01\\
43.41	0.01\\
43.42	0.01\\
43.43	0.01\\
43.44	0.01\\
43.45	0.01\\
43.46	0.01\\
43.47	0.01\\
43.48	0.01\\
43.49	0.01\\
43.5	0.01\\
43.51	0.01\\
43.52	0.01\\
43.53	0.01\\
43.54	0.01\\
43.55	0.01\\
43.56	0.01\\
43.57	0.01\\
43.58	0.01\\
43.59	0.01\\
43.6	0.01\\
43.61	0.01\\
43.62	0.01\\
43.63	0.01\\
43.64	0.01\\
43.65	0.01\\
43.66	0.01\\
43.67	0.01\\
43.68	0.01\\
43.69	0.01\\
43.7	0.01\\
43.71	0.01\\
43.72	0.01\\
43.73	0.01\\
43.74	0.01\\
43.75	0.01\\
43.76	0.01\\
43.77	0.01\\
43.78	0.01\\
43.79	0.01\\
43.8	0.01\\
43.81	0.01\\
43.82	0.01\\
43.83	0.01\\
43.84	0.01\\
43.85	0.01\\
43.86	0.01\\
43.87	0.01\\
43.88	0.01\\
43.89	0.01\\
43.9	0.01\\
43.91	0.01\\
43.92	0.01\\
43.93	0.01\\
43.94	0.01\\
43.95	0.01\\
43.96	0.01\\
43.97	0.01\\
43.98	0.01\\
43.99	0.01\\
44	0.01\\
44.01	0.01\\
44.02	0.01\\
44.03	0.01\\
44.04	0.01\\
44.05	0.01\\
44.06	0.01\\
44.07	0.01\\
44.08	0.01\\
44.09	0.01\\
44.1	0.01\\
44.11	0.01\\
44.12	0.01\\
44.13	0.01\\
44.14	0.01\\
44.15	0.01\\
44.16	0.01\\
44.17	0.01\\
44.18	0.01\\
44.19	0.01\\
44.2	0.01\\
44.21	0.01\\
44.22	0.01\\
44.23	0.01\\
44.24	0.01\\
44.25	0.01\\
44.26	0.01\\
44.27	0.01\\
44.28	0.01\\
44.29	0.01\\
44.3	0.01\\
44.31	0.01\\
44.32	0.01\\
44.33	0.01\\
44.34	0.01\\
44.35	0.01\\
44.36	0.01\\
44.37	0.01\\
44.38	0.01\\
44.39	0.01\\
44.4	0.01\\
44.41	0.01\\
44.42	0.01\\
44.43	0.01\\
44.44	0.01\\
44.45	0.01\\
44.46	0.01\\
44.47	0.01\\
44.48	0.01\\
44.49	0.01\\
44.5	0.01\\
44.51	0.01\\
44.52	0.01\\
44.53	0.01\\
44.54	0.01\\
44.55	0.01\\
44.56	0.01\\
44.57	0.01\\
44.58	0.01\\
44.59	0.01\\
44.6	0.01\\
44.61	0.01\\
44.62	0.01\\
44.63	0.01\\
44.64	0.01\\
44.65	0.01\\
44.66	0.01\\
44.67	0.01\\
44.68	0.01\\
44.69	0.01\\
44.7	0.01\\
44.71	0.01\\
44.72	0.01\\
44.73	0.01\\
44.74	0.01\\
44.75	0.01\\
44.76	0.01\\
44.77	0.01\\
44.78	0.01\\
44.79	0.01\\
44.8	0.01\\
44.81	0.01\\
44.82	0.01\\
44.83	0.01\\
44.84	0.01\\
44.85	0.01\\
44.86	0.01\\
44.87	0.01\\
44.88	0.01\\
44.89	0.01\\
44.9	0.01\\
44.91	0.01\\
44.92	0.01\\
44.93	0.01\\
44.94	0.01\\
44.95	0.01\\
44.96	0.01\\
44.97	0.01\\
44.98	0.01\\
44.99	0.01\\
45	0.01\\
45.01	0.01\\
45.02	0.01\\
45.03	0.01\\
45.04	0.01\\
45.05	0.01\\
45.06	0.01\\
45.07	0.01\\
45.08	0.01\\
45.09	0.01\\
45.1	0.01\\
45.11	0.01\\
45.12	0.01\\
45.13	0.01\\
45.14	0.01\\
45.15	0.01\\
45.16	0.01\\
45.17	0.01\\
45.18	0.01\\
45.19	0.01\\
45.2	0.01\\
45.21	0.01\\
45.22	0.01\\
45.23	0.01\\
45.24	0.01\\
45.25	0.01\\
45.26	0.01\\
45.27	0.01\\
45.28	0.01\\
45.29	0.01\\
45.3	0.01\\
45.31	0.01\\
45.32	0.01\\
45.33	0.01\\
45.34	0.01\\
45.35	0.01\\
45.36	0.01\\
45.37	0.01\\
45.38	0.01\\
45.39	0.01\\
45.4	0.01\\
45.41	0.01\\
45.42	0.01\\
45.43	0.01\\
45.44	0.01\\
45.45	0.01\\
45.46	0.01\\
45.47	0.01\\
45.48	0.01\\
45.49	0.01\\
45.5	0.01\\
45.51	0.01\\
45.52	0.01\\
45.53	0.01\\
45.54	0.01\\
45.55	0.01\\
45.56	0.01\\
45.57	0.01\\
45.58	0.01\\
45.59	0.01\\
45.6	0.01\\
45.61	0.01\\
45.62	0.01\\
45.63	0.01\\
45.64	0.01\\
45.65	0.01\\
45.66	0.01\\
45.67	0.01\\
45.68	0.01\\
45.69	0.01\\
45.7	0.01\\
45.71	0.01\\
45.72	0.01\\
45.73	0.01\\
45.74	0.01\\
45.75	0.01\\
45.76	0.01\\
45.77	0.01\\
45.78	0.01\\
45.79	0.01\\
45.8	0.01\\
45.81	0.01\\
45.82	0.01\\
45.83	0.01\\
45.84	0.01\\
45.85	0.01\\
45.86	0.01\\
45.87	0.01\\
45.88	0.01\\
45.89	0.01\\
45.9	0.01\\
45.91	0.01\\
45.92	0.01\\
45.93	0.01\\
45.94	0.01\\
45.95	0.01\\
45.96	0.01\\
45.97	0.01\\
45.98	0.01\\
45.99	0.01\\
46	0.01\\
46.01	0.01\\
46.02	0.01\\
46.03	0.01\\
46.04	0.01\\
46.05	0.01\\
46.06	0.01\\
46.07	0.01\\
46.08	0.01\\
46.09	0.01\\
46.1	0.01\\
46.11	0.01\\
46.12	0.01\\
46.13	0.01\\
46.14	0.01\\
46.15	0.01\\
46.16	0.01\\
46.17	0.01\\
46.18	0.01\\
46.19	0.01\\
46.2	0.01\\
46.21	0.01\\
46.22	0.01\\
46.23	0.01\\
46.24	0.01\\
46.25	0.01\\
46.26	0.01\\
46.27	0.01\\
46.28	0.01\\
46.29	0.01\\
46.3	0.01\\
46.31	0.01\\
46.32	0.01\\
46.33	0.01\\
46.34	0.01\\
46.35	0.01\\
46.36	0.01\\
46.37	0.01\\
46.38	0.01\\
46.39	0.01\\
46.4	0.01\\
46.41	0.01\\
46.42	0.01\\
46.43	0.01\\
46.44	0.01\\
46.45	0.01\\
46.46	0.01\\
46.47	0.01\\
46.48	0.01\\
46.49	0.01\\
46.5	0.01\\
46.51	0.01\\
46.52	0.01\\
46.53	0.01\\
46.54	0.01\\
46.55	0.01\\
46.56	0.01\\
46.57	0.01\\
46.58	0.01\\
46.59	0.01\\
46.6	0.01\\
46.61	0.01\\
46.62	0.01\\
46.63	0.01\\
46.64	0.01\\
46.65	0.01\\
46.66	0.01\\
46.67	0.01\\
46.68	0.01\\
46.69	0.01\\
46.7	0.01\\
46.71	0.01\\
46.72	0.01\\
46.73	0.01\\
46.74	0.01\\
46.75	0.01\\
46.76	0.01\\
46.77	0.01\\
46.78	0.01\\
46.79	0.01\\
46.8	0.01\\
46.81	0.01\\
46.82	0.01\\
46.83	0.01\\
46.84	0.01\\
46.85	0.01\\
46.86	0.01\\
46.87	0.01\\
46.88	0.01\\
46.89	0.01\\
46.9	0.01\\
46.91	0.01\\
46.92	0.01\\
46.93	0.01\\
46.94	0.01\\
46.95	0.01\\
46.96	0.01\\
46.97	0.01\\
46.98	0.01\\
46.99	0.01\\
47	0.01\\
47.01	0.01\\
47.02	0.01\\
47.03	0.01\\
47.04	0.01\\
47.05	0.01\\
47.06	0.01\\
47.07	0.01\\
47.08	0.01\\
47.09	0.01\\
47.1	0.01\\
47.11	0.01\\
47.12	0.01\\
47.13	0.01\\
47.14	0.01\\
47.15	0.01\\
47.16	0.01\\
47.17	0.01\\
47.18	0.01\\
47.19	0.01\\
47.2	0.01\\
47.21	0.01\\
47.22	0.01\\
47.23	0.01\\
47.24	0.01\\
47.25	0.01\\
47.26	0.01\\
47.27	0.01\\
47.28	0.01\\
47.29	0.01\\
47.3	0.01\\
47.31	0.01\\
47.32	0.01\\
47.33	0.01\\
47.34	0.01\\
47.35	0.01\\
47.36	0.01\\
47.37	0.01\\
47.38	0.01\\
47.39	0.01\\
47.4	0.01\\
47.41	0.01\\
47.42	0.01\\
47.43	0.01\\
47.44	0.01\\
47.45	0.01\\
47.46	0.01\\
47.47	0.01\\
47.48	0.01\\
47.49	0.01\\
47.5	0.01\\
47.51	0.01\\
47.52	0.01\\
47.53	0.01\\
47.54	0.01\\
47.55	0.01\\
47.56	0.01\\
47.57	0.01\\
47.58	0.01\\
47.59	0.01\\
47.6	0.01\\
47.61	0.01\\
47.62	0.01\\
47.63	0.01\\
47.64	0.01\\
47.65	0.01\\
47.66	0.01\\
47.67	0.01\\
47.68	0.01\\
47.69	0.01\\
47.7	0.01\\
47.71	0.01\\
47.72	0.01\\
47.73	0.01\\
47.74	0.01\\
47.75	0.01\\
47.76	0.01\\
47.77	0.01\\
47.78	0.01\\
47.79	0.01\\
47.8	0.01\\
47.81	0.01\\
47.82	0.01\\
47.83	0.01\\
47.84	0.01\\
47.85	0.01\\
47.86	0.01\\
47.87	0.01\\
47.88	0.01\\
47.89	0.01\\
47.9	0.01\\
47.91	0.01\\
47.92	0.01\\
47.93	0.01\\
47.94	0.01\\
47.95	0.01\\
47.96	0.01\\
47.97	0.01\\
47.98	0.01\\
47.99	0.01\\
48	0.01\\
48.01	0.01\\
48.02	0.01\\
48.03	0.01\\
48.04	0.01\\
48.05	0.01\\
48.06	0.01\\
48.07	0.01\\
48.08	0.01\\
48.09	0.01\\
48.1	0.01\\
48.11	0.01\\
48.12	0.01\\
48.13	0.01\\
48.14	0.01\\
48.15	0.01\\
48.16	0.01\\
48.17	0.01\\
48.18	0.01\\
48.19	0.01\\
48.2	0.01\\
48.21	0.01\\
48.22	0.01\\
48.23	0.01\\
48.24	0.01\\
48.25	0.01\\
48.26	0.01\\
48.27	0.01\\
48.28	0.01\\
48.29	0.01\\
48.3	0.01\\
48.31	0.01\\
48.32	0.01\\
48.33	0.01\\
48.34	0.01\\
48.35	0.01\\
48.36	0.01\\
48.37	0.01\\
48.38	0.01\\
48.39	0.01\\
48.4	0.01\\
48.41	0.01\\
48.42	0.01\\
48.43	0.01\\
48.44	0.01\\
48.45	0.01\\
48.46	0.01\\
48.47	0.01\\
48.48	0.01\\
48.49	0.01\\
48.5	0.01\\
48.51	0.01\\
48.52	0.01\\
48.53	0.01\\
48.54	0.01\\
48.55	0.01\\
48.56	0.01\\
48.57	0.01\\
48.58	0.01\\
48.59	0.01\\
48.6	0.01\\
48.61	0.01\\
48.62	0.01\\
48.63	0.01\\
48.64	0.01\\
48.65	0.01\\
48.66	0.01\\
48.67	0.01\\
48.68	0.01\\
48.69	0.01\\
48.7	0.01\\
48.71	0.01\\
48.72	0.01\\
48.73	0.01\\
48.74	0.01\\
48.75	0.01\\
48.76	0.01\\
48.77	0.01\\
48.78	0.01\\
48.79	0.01\\
48.8	0.01\\
48.81	0.01\\
48.82	0.01\\
48.83	0.01\\
48.84	0.01\\
48.85	0.01\\
48.86	0.01\\
48.87	0.01\\
48.88	0.01\\
48.89	0.01\\
48.9	0.01\\
48.91	0.01\\
48.92	0.01\\
48.93	0.01\\
48.94	0.01\\
48.95	0.01\\
48.96	0.01\\
48.97	0.01\\
48.98	0.01\\
48.99	0.01\\
49	0.01\\
49.01	0.01\\
49.02	0.01\\
49.03	0.01\\
49.04	0.01\\
49.05	0.01\\
49.06	0.01\\
49.07	0.01\\
49.08	0.01\\
49.09	0.01\\
49.1	0.01\\
49.11	0.01\\
49.12	0.01\\
49.13	0.01\\
49.14	0.01\\
49.15	0.01\\
49.16	0.01\\
49.17	0.01\\
49.18	0.01\\
49.19	0.01\\
49.2	0.01\\
49.21	0.01\\
49.22	0.01\\
49.23	0.01\\
49.24	0.01\\
49.25	0.01\\
49.26	0.01\\
49.27	0.01\\
49.28	0.01\\
49.29	0.01\\
49.3	0.01\\
49.31	0.01\\
49.32	0.01\\
49.33	0.01\\
49.34	0.01\\
49.35	0.01\\
49.36	0.01\\
49.37	0.01\\
49.38	0.01\\
49.39	0.01\\
49.4	0.01\\
49.41	0.01\\
49.42	0.01\\
49.43	0.01\\
49.44	0.01\\
49.45	0.01\\
49.46	0.01\\
49.47	0.01\\
49.48	0.01\\
49.49	0.01\\
49.5	0.01\\
49.51	0.01\\
49.52	0.01\\
49.53	0.01\\
49.54	0.01\\
49.55	0.01\\
49.56	0.01\\
49.57	0.01\\
49.58	0.01\\
49.59	0.01\\
49.6	0.01\\
49.61	0.01\\
49.62	0.01\\
49.63	0.01\\
49.64	0.01\\
49.65	0.01\\
49.66	0.01\\
49.67	0.01\\
49.68	0.01\\
49.69	0.01\\
49.7	0.01\\
49.71	0.01\\
49.72	0.01\\
49.73	0.01\\
49.74	0.01\\
49.75	0.01\\
49.76	0.01\\
49.77	0.01\\
49.78	0.01\\
49.79	0.01\\
49.8	0.01\\
49.81	0.01\\
49.82	0.01\\
49.83	0.01\\
49.84	0.01\\
49.85	0.01\\
49.86	0.01\\
49.87	0.01\\
49.88	0.01\\
49.89	0.01\\
49.9	0.01\\
49.91	0.01\\
49.92	0.01\\
49.93	0.01\\
49.94	0.01\\
49.95	0.01\\
49.96	0.01\\
49.97	0.01\\
49.98	0.01\\
49.99	0.01\\
50	0.01\\
50.01	0.01\\
50.02	0.01\\
50.03	0.01\\
50.04	0.01\\
50.05	0.01\\
50.06	0.01\\
50.07	0.01\\
50.08	0.01\\
50.09	0.01\\
50.1	0.01\\
50.11	0.01\\
50.12	0.01\\
50.13	0.01\\
50.14	0.01\\
50.15	0.01\\
50.16	0.01\\
50.17	0.01\\
50.18	0.01\\
50.19	0.01\\
50.2	0.01\\
50.21	0.01\\
50.22	0.01\\
50.23	0.01\\
50.24	0.01\\
50.25	0.01\\
50.26	0.01\\
50.27	0.01\\
50.28	0.01\\
50.29	0.01\\
50.3	0.01\\
50.31	0.01\\
50.32	0.01\\
50.33	0.01\\
50.34	0.01\\
50.35	0.01\\
50.36	0.01\\
50.37	0.01\\
50.38	0.01\\
50.39	0.01\\
50.4	0.01\\
50.41	0.01\\
50.42	0.01\\
50.43	0.01\\
50.44	0.01\\
50.45	0.01\\
50.46	0.01\\
50.47	0.01\\
50.48	0.01\\
50.49	0.01\\
50.5	0.01\\
50.51	0.01\\
50.52	0.01\\
50.53	0.01\\
50.54	0.01\\
50.55	0.01\\
50.56	0.01\\
50.57	0.01\\
50.58	0.01\\
50.59	0.01\\
50.6	0.01\\
50.61	0.01\\
50.62	0.01\\
50.63	0.01\\
50.64	0.01\\
50.65	0.01\\
50.66	0.01\\
50.67	0.01\\
50.68	0.01\\
50.69	0.01\\
50.7	0.01\\
50.71	0.01\\
50.72	0.01\\
50.73	0.01\\
50.74	0.01\\
50.75	0.01\\
50.76	0.01\\
50.77	0.01\\
50.78	0.01\\
50.79	0.01\\
50.8	0.01\\
50.81	0.01\\
50.82	0.01\\
50.83	0.01\\
50.84	0.01\\
50.85	0.01\\
50.86	0.01\\
50.87	0.01\\
50.88	0.01\\
50.89	0.01\\
50.9	0.01\\
50.91	0.01\\
50.92	0.01\\
50.93	0.01\\
50.94	0.01\\
50.95	0.01\\
50.96	0.01\\
50.97	0.01\\
50.98	0.01\\
50.99	0.01\\
51	0.01\\
51.01	0.01\\
51.02	0.01\\
51.03	0.01\\
51.04	0.01\\
51.05	0.01\\
51.06	0.01\\
51.07	0.01\\
51.08	0.01\\
51.09	0.01\\
51.1	0.01\\
51.11	0.01\\
51.12	0.01\\
51.13	0.01\\
51.14	0.01\\
51.15	0.01\\
51.16	0.01\\
51.17	0.01\\
51.18	0.01\\
51.19	0.01\\
51.2	0.01\\
51.21	0.01\\
51.22	0.01\\
51.23	0.01\\
51.24	0.01\\
51.25	0.01\\
51.26	0.01\\
51.27	0.01\\
51.28	0.01\\
51.29	0.01\\
51.3	0.01\\
51.31	0.01\\
51.32	0.01\\
51.33	0.01\\
51.34	0.01\\
51.35	0.01\\
51.36	0.01\\
51.37	0.01\\
51.38	0.01\\
51.39	0.01\\
51.4	0.01\\
51.41	0.01\\
51.42	0.01\\
51.43	0.01\\
51.44	0.01\\
51.45	0.01\\
51.46	0.01\\
51.47	0.01\\
51.48	0.01\\
51.49	0.01\\
51.5	0.01\\
51.51	0.01\\
51.52	0.01\\
51.53	0.01\\
51.54	0.01\\
51.55	0.01\\
51.56	0.01\\
51.57	0.01\\
51.58	0.01\\
51.59	0.01\\
51.6	0.01\\
51.61	0.01\\
51.62	0.01\\
51.63	0.01\\
51.64	0.01\\
51.65	0.01\\
51.66	0.01\\
51.67	0.01\\
51.68	0.01\\
51.69	0.01\\
51.7	0.01\\
51.71	0.01\\
51.72	0.01\\
51.73	0.01\\
51.74	0.01\\
51.75	0.01\\
51.76	0.01\\
51.77	0.01\\
51.78	0.01\\
51.79	0.01\\
51.8	0.01\\
51.81	0.01\\
51.82	0.01\\
51.83	0.01\\
51.84	0.01\\
51.85	0.01\\
51.86	0.01\\
51.87	0.01\\
51.88	0.01\\
51.89	0.01\\
51.9	0.01\\
51.91	0.01\\
51.92	0.01\\
51.93	0.01\\
51.94	0.01\\
51.95	0.01\\
51.96	0.01\\
51.97	0.01\\
51.98	0.01\\
51.99	0.01\\
52	0.01\\
52.01	0.01\\
52.02	0.01\\
52.03	0.01\\
52.04	0.01\\
52.05	0.01\\
52.06	0.01\\
52.07	0.01\\
52.08	0.01\\
52.09	0.01\\
52.1	0.01\\
52.11	0.01\\
52.12	0.01\\
52.13	0.01\\
52.14	0.01\\
52.15	0.01\\
52.16	0.01\\
52.17	0.01\\
52.18	0.01\\
52.19	0.01\\
52.2	0.01\\
52.21	0.01\\
52.22	0.01\\
52.23	0.01\\
52.24	0.01\\
52.25	0.01\\
52.26	0.01\\
52.27	0.01\\
52.28	0.01\\
52.29	0.01\\
52.3	0.01\\
52.31	0.01\\
52.32	0.01\\
52.33	0.01\\
52.34	0.01\\
52.35	0.01\\
52.36	0.01\\
52.37	0.01\\
52.38	0.01\\
52.39	0.01\\
52.4	0.01\\
52.41	0.01\\
52.42	0.01\\
52.43	0.01\\
52.44	0.01\\
52.45	0.01\\
52.46	0.01\\
52.47	0.01\\
52.48	0.01\\
52.49	0.01\\
52.5	0.01\\
52.51	0.01\\
52.52	0.01\\
52.53	0.01\\
52.54	0.01\\
52.55	0.01\\
52.56	0.01\\
52.57	0.01\\
52.58	0.01\\
52.59	0.01\\
52.6	0.01\\
52.61	0.01\\
52.62	0.01\\
52.63	0.01\\
52.64	0.01\\
52.65	0.01\\
52.66	0.01\\
52.67	0.01\\
52.68	0.01\\
52.69	0.01\\
52.7	0.01\\
52.71	0.01\\
52.72	0.01\\
52.73	0.01\\
52.74	0.01\\
52.75	0.01\\
52.76	0.01\\
52.77	0.01\\
52.78	0.01\\
52.79	0.01\\
52.8	0.01\\
52.81	0.01\\
52.82	0.01\\
52.83	0.01\\
52.84	0.01\\
52.85	0.01\\
52.86	0.01\\
52.87	0.01\\
52.88	0.01\\
52.89	0.01\\
52.9	0.01\\
52.91	0.01\\
52.92	0.01\\
52.93	0.01\\
52.94	0.01\\
52.95	0.01\\
52.96	0.01\\
52.97	0.01\\
52.98	0.01\\
52.99	0.01\\
53	0.01\\
53.01	0.01\\
53.02	0.01\\
53.03	0.01\\
53.04	0.01\\
53.05	0.01\\
53.06	0.01\\
53.07	0.01\\
53.08	0.01\\
53.09	0.01\\
53.1	0.01\\
53.11	0.01\\
53.12	0.01\\
53.13	0.01\\
53.14	0.01\\
53.15	0.01\\
53.16	0.01\\
53.17	0.01\\
53.18	0.01\\
53.19	0.01\\
53.2	0.01\\
53.21	0.01\\
53.22	0.01\\
53.23	0.01\\
53.24	0.01\\
53.25	0.01\\
53.26	0.01\\
53.27	0.01\\
53.28	0.01\\
53.29	0.01\\
53.3	0.01\\
53.31	0.01\\
53.32	0.01\\
53.33	0.01\\
53.34	0.01\\
53.35	0.01\\
53.36	0.01\\
53.37	0.01\\
53.38	0.01\\
53.39	0.01\\
53.4	0.01\\
53.41	0.01\\
53.42	0.01\\
53.43	0.01\\
53.44	0.01\\
53.45	0.01\\
53.46	0.01\\
53.47	0.01\\
53.48	0.01\\
53.49	0.01\\
53.5	0.01\\
53.51	0.01\\
53.52	0.01\\
53.53	0.01\\
53.54	0.01\\
53.55	0.01\\
53.56	0.01\\
53.57	0.01\\
53.58	0.01\\
53.59	0.01\\
53.6	0.01\\
53.61	0.01\\
53.62	0.01\\
53.63	0.01\\
53.64	0.01\\
53.65	0.01\\
53.66	0.01\\
53.67	0.01\\
53.68	0.01\\
53.69	0.01\\
53.7	0.01\\
53.71	0.01\\
53.72	0.01\\
53.73	0.01\\
53.74	0.01\\
53.75	0.01\\
53.76	0.01\\
53.77	0.01\\
53.78	0.01\\
53.79	0.01\\
53.8	0.01\\
53.81	0.01\\
53.82	0.01\\
53.83	0.01\\
53.84	0.01\\
53.85	0.01\\
53.86	0.01\\
53.87	0.01\\
53.88	0.01\\
53.89	0.01\\
53.9	0.01\\
53.91	0.01\\
53.92	0.01\\
53.93	0.01\\
53.94	0.01\\
53.95	0.01\\
53.96	0.01\\
53.97	0.01\\
53.98	0.01\\
53.99	0.01\\
54	0.01\\
54.01	0.01\\
54.02	0.01\\
54.03	0.01\\
54.04	0.01\\
54.05	0.01\\
54.06	0.01\\
54.07	0.01\\
54.08	0.01\\
54.09	0.01\\
54.1	0.01\\
54.11	0.01\\
54.12	0.01\\
54.13	0.01\\
54.14	0.01\\
54.15	0.01\\
54.16	0.01\\
54.17	0.01\\
54.18	0.01\\
54.19	0.01\\
54.2	0.01\\
54.21	0.01\\
54.22	0.01\\
54.23	0.01\\
54.24	0.01\\
54.25	0.01\\
54.26	0.01\\
54.27	0.01\\
54.28	0.01\\
54.29	0.01\\
54.3	0.01\\
54.31	0.01\\
54.32	0.01\\
54.33	0.01\\
54.34	0.01\\
54.35	0.01\\
54.36	0.01\\
54.37	0.01\\
54.38	0.01\\
54.39	0.01\\
54.4	0.01\\
54.41	0.01\\
54.42	0.01\\
54.43	0.01\\
54.44	0.01\\
54.45	0.01\\
54.46	0.01\\
54.47	0.01\\
54.48	0.01\\
54.49	0.01\\
54.5	0.01\\
54.51	0.01\\
54.52	0.01\\
54.53	0.01\\
54.54	0.01\\
54.55	0.01\\
54.56	0.01\\
54.57	0.01\\
54.58	0.01\\
54.59	0.01\\
54.6	0.01\\
54.61	0.01\\
54.62	0.01\\
54.63	0.01\\
54.64	0.01\\
54.65	0.01\\
54.66	0.01\\
54.67	0.01\\
54.68	0.01\\
54.69	0.01\\
54.7	0.01\\
54.71	0.01\\
54.72	0.01\\
54.73	0.01\\
54.74	0.01\\
54.75	0.01\\
54.76	0.01\\
54.77	0.01\\
54.78	0.01\\
54.79	0.01\\
54.8	0.01\\
54.81	0.01\\
54.82	0.01\\
54.83	0.01\\
54.84	0.01\\
54.85	0.01\\
54.86	0.01\\
54.87	0.01\\
54.88	0.01\\
54.89	0.01\\
54.9	0.01\\
54.91	0.01\\
54.92	0.01\\
54.93	0.01\\
54.94	0.01\\
54.95	0.01\\
54.96	0.01\\
54.97	0.01\\
54.98	0.01\\
54.99	0.01\\
55	0.01\\
55.01	0.01\\
55.02	0.01\\
55.03	0.01\\
55.04	0.01\\
55.05	0.01\\
55.06	0.01\\
55.07	0.01\\
55.08	0.01\\
55.09	0.01\\
55.1	0.01\\
55.11	0.01\\
55.12	0.01\\
55.13	0.01\\
55.14	0.01\\
55.15	0.01\\
55.16	0.01\\
55.17	0.01\\
55.18	0.01\\
55.19	0.01\\
55.2	0.01\\
55.21	0.01\\
55.22	0.01\\
55.23	0.01\\
55.24	0.01\\
55.25	0.01\\
55.26	0.01\\
55.27	0.01\\
55.28	0.01\\
55.29	0.01\\
55.3	0.01\\
55.31	0.01\\
55.32	0.01\\
55.33	0.01\\
55.34	0.01\\
55.35	0.01\\
55.36	0.01\\
55.37	0.01\\
55.38	0.01\\
55.39	0.01\\
55.4	0.01\\
55.41	0.01\\
55.42	0.01\\
55.43	0.01\\
55.44	0.01\\
55.45	0.01\\
55.46	0.01\\
55.47	0.01\\
55.48	0.01\\
55.49	0.01\\
55.5	0.01\\
55.51	0.01\\
55.52	0.01\\
55.53	0.01\\
55.54	0.01\\
55.55	0.01\\
55.56	0.01\\
55.57	0.01\\
55.58	0.01\\
55.59	0.01\\
55.6	0.01\\
55.61	0.01\\
55.62	0.01\\
55.63	0.01\\
55.64	0.01\\
55.65	0.01\\
55.66	0.01\\
55.67	0.01\\
55.68	0.01\\
55.69	0.01\\
55.7	0.01\\
55.71	0.01\\
55.72	0.01\\
55.73	0.01\\
55.74	0.01\\
55.75	0.01\\
55.76	0.01\\
55.77	0.01\\
55.78	0.01\\
55.79	0.01\\
55.8	0.01\\
55.81	0.01\\
55.82	0.01\\
55.83	0.01\\
55.84	0.01\\
55.85	0.01\\
55.86	0.01\\
55.87	0.01\\
55.88	0.01\\
55.89	0.01\\
55.9	0.01\\
55.91	0.01\\
55.92	0.01\\
55.93	0.01\\
55.94	0.01\\
55.95	0.01\\
55.96	0.01\\
55.97	0.01\\
55.98	0.01\\
55.99	0.01\\
56	0.01\\
56.01	0.01\\
56.02	0.01\\
56.03	0.01\\
56.04	0.01\\
56.05	0.01\\
56.06	0.01\\
56.07	0.01\\
56.08	0.01\\
56.09	0.01\\
56.1	0.01\\
56.11	0.01\\
56.12	0.01\\
56.13	0.01\\
56.14	0.01\\
56.15	0.01\\
56.16	0.01\\
56.17	0.01\\
56.18	0.01\\
56.19	0.01\\
56.2	0.01\\
56.21	0.01\\
56.22	0.01\\
56.23	0.01\\
56.24	0.01\\
56.25	0.01\\
56.26	0.01\\
56.27	0.01\\
56.28	0.01\\
56.29	0.01\\
56.3	0.01\\
56.31	0.01\\
56.32	0.01\\
56.33	0.01\\
56.34	0.01\\
56.35	0.01\\
56.36	0.01\\
56.37	0.01\\
56.38	0.01\\
56.39	0.01\\
56.4	0.01\\
56.41	0.01\\
56.42	0.01\\
56.43	0.01\\
56.44	0.01\\
56.45	0.01\\
56.46	0.01\\
56.47	0.01\\
56.48	0.01\\
56.49	0.01\\
56.5	0.01\\
56.51	0.01\\
56.52	0.01\\
56.53	0.01\\
56.54	0.01\\
56.55	0.01\\
56.56	0.01\\
56.57	0.01\\
56.58	0.01\\
56.59	0.01\\
56.6	0.01\\
56.61	0.01\\
56.62	0.01\\
56.63	0.01\\
56.64	0.01\\
56.65	0.01\\
56.66	0.01\\
56.67	0.01\\
56.68	0.01\\
56.69	0.01\\
56.7	0.01\\
56.71	0.01\\
56.72	0.01\\
56.73	0.01\\
56.74	0.01\\
56.75	0.01\\
56.76	0.01\\
56.77	0.01\\
56.78	0.01\\
56.79	0.01\\
56.8	0.01\\
56.81	0.01\\
56.82	0.01\\
56.83	0.01\\
56.84	0.01\\
56.85	0.01\\
56.86	0.01\\
56.87	0.01\\
56.88	0.01\\
56.89	0.01\\
56.9	0.01\\
56.91	0.01\\
56.92	0.01\\
56.93	0.01\\
56.94	0.01\\
56.95	0.01\\
56.96	0.01\\
56.97	0.01\\
56.98	0.01\\
56.99	0.01\\
57	0.01\\
57.01	0.01\\
57.02	0.01\\
57.03	0.01\\
57.04	0.01\\
57.05	0.01\\
57.06	0.01\\
57.07	0.01\\
57.08	0.01\\
57.09	0.01\\
57.1	0.01\\
57.11	0.01\\
57.12	0.01\\
57.13	0.01\\
57.14	0.01\\
57.15	0.01\\
57.16	0.01\\
57.17	0.01\\
57.18	0.01\\
57.19	0.01\\
57.2	0.01\\
57.21	0.01\\
57.22	0.01\\
57.23	0.01\\
57.24	0.01\\
57.25	0.01\\
57.26	0.01\\
57.27	0.01\\
57.28	0.01\\
57.29	0.01\\
57.3	0.01\\
57.31	0.01\\
57.32	0.01\\
57.33	0.01\\
57.34	0.01\\
57.35	0.01\\
57.36	0.01\\
57.37	0.01\\
57.38	0.01\\
57.39	0.01\\
57.4	0.01\\
57.41	0.01\\
57.42	0.01\\
57.43	0.01\\
57.44	0.01\\
57.45	0.01\\
57.46	0.01\\
57.47	0.01\\
57.48	0.01\\
57.49	0.01\\
57.5	0.01\\
57.51	0.01\\
57.52	0.01\\
57.53	0.01\\
57.54	0.01\\
57.55	0.01\\
57.56	0.01\\
57.57	0.01\\
57.58	0.01\\
57.59	0.01\\
57.6	0.01\\
57.61	0.01\\
57.62	0.01\\
57.63	0.01\\
57.64	0.01\\
57.65	0.01\\
57.66	0.01\\
57.67	0.01\\
57.68	0.01\\
57.69	0.01\\
57.7	0.01\\
57.71	0.01\\
57.72	0.01\\
57.73	0.01\\
57.74	0.01\\
57.75	0.01\\
57.76	0.01\\
57.77	0.01\\
57.78	0.01\\
57.79	0.01\\
57.8	0.01\\
57.81	0.01\\
57.82	0.01\\
57.83	0.01\\
57.84	0.01\\
57.85	0.01\\
57.86	0.01\\
57.87	0.01\\
57.88	0.01\\
57.89	0.01\\
57.9	0.01\\
57.91	0.01\\
57.92	0.01\\
57.93	0.01\\
57.94	0.01\\
57.95	0.01\\
57.96	0.01\\
57.97	0.01\\
57.98	0.01\\
57.99	0.01\\
58	0.01\\
58.01	0.01\\
58.02	0.01\\
58.03	0.01\\
58.04	0.01\\
58.05	0.01\\
58.06	0.01\\
58.07	0.01\\
58.08	0.01\\
58.09	0.01\\
58.1	0.01\\
58.11	0.01\\
58.12	0.01\\
58.13	0.01\\
58.14	0.01\\
58.15	0.01\\
58.16	0.01\\
58.17	0.01\\
58.18	0.01\\
58.19	0.01\\
58.2	0.01\\
58.21	0.01\\
58.22	0.01\\
58.23	0.01\\
58.24	0.01\\
58.25	0.01\\
58.26	0.01\\
58.27	0.01\\
58.28	0.01\\
58.29	0.01\\
58.3	0.01\\
58.31	0.01\\
58.32	0.01\\
58.33	0.01\\
58.34	0.01\\
58.35	0.01\\
58.36	0.01\\
58.37	0.01\\
58.38	0.01\\
58.39	0.01\\
58.4	0.01\\
58.41	0.01\\
58.42	0.01\\
58.43	0.01\\
58.44	0.01\\
58.45	0.01\\
58.46	0.01\\
58.47	0.01\\
58.48	0.01\\
58.49	0.01\\
58.5	0.01\\
58.51	0.01\\
58.52	0.01\\
58.53	0.01\\
58.54	0.01\\
58.55	0.01\\
58.56	0.01\\
58.57	0.01\\
58.58	0.01\\
58.59	0.01\\
58.6	0.01\\
58.61	0.01\\
58.62	0.01\\
58.63	0.01\\
58.64	0.01\\
58.65	0.01\\
58.66	0.01\\
58.67	0.01\\
58.68	0.01\\
58.69	0.01\\
58.7	0.01\\
58.71	0.01\\
58.72	0.01\\
58.73	0.01\\
58.74	0.01\\
58.75	0.01\\
58.76	0.01\\
58.77	0.01\\
58.78	0.01\\
58.79	0.01\\
58.8	0.01\\
58.81	0.01\\
58.82	0.01\\
58.83	0.01\\
58.84	0.01\\
58.85	0.01\\
58.86	0.01\\
58.87	0.01\\
58.88	0.01\\
58.89	0.01\\
58.9	0.01\\
58.91	0.01\\
58.92	0.01\\
58.93	0.01\\
58.94	0.01\\
58.95	0.01\\
58.96	0.01\\
58.97	0.01\\
58.98	0.01\\
58.99	0.01\\
59	0.01\\
59.01	0.01\\
59.02	0.01\\
59.03	0.01\\
59.04	0.01\\
59.05	0.01\\
59.06	0.01\\
59.07	0.01\\
59.08	0.01\\
59.09	0.01\\
59.1	0.01\\
59.11	0.01\\
59.12	0.01\\
59.13	0.01\\
59.14	0.01\\
59.15	0.01\\
59.16	0.01\\
59.17	0.01\\
59.18	0.01\\
59.19	0.01\\
59.2	0.01\\
59.21	0.01\\
59.22	0.01\\
59.23	0.01\\
59.24	0.01\\
59.25	0.01\\
59.26	0.01\\
59.27	0.01\\
59.28	0.01\\
59.29	0.01\\
59.3	0.01\\
59.31	0.01\\
59.32	0.01\\
59.33	0.01\\
59.34	0.01\\
59.35	0.01\\
59.36	0.01\\
59.37	0.01\\
59.38	0.01\\
59.39	0.01\\
59.4	0.01\\
59.41	0.01\\
59.42	0.01\\
59.43	0.01\\
59.44	0.01\\
59.45	0.01\\
59.46	0.01\\
59.47	0.01\\
59.48	0.01\\
59.49	0.01\\
59.5	0.01\\
59.51	0.01\\
59.52	0.01\\
59.53	0.01\\
59.54	0.01\\
59.55	0.01\\
59.56	0.01\\
59.57	0.01\\
59.58	0.01\\
59.59	0.01\\
59.6	0.01\\
59.61	0.01\\
59.62	0.01\\
59.63	0.01\\
59.64	0.01\\
59.65	0.01\\
59.66	0.01\\
59.67	0.01\\
59.68	0.01\\
59.69	0.01\\
59.7	0.01\\
59.71	0.01\\
59.72	0.01\\
59.73	0.01\\
59.74	0.01\\
59.75	0.01\\
59.76	0.01\\
59.77	0.01\\
59.78	0.01\\
59.79	0.01\\
59.8	0.01\\
59.81	0.01\\
59.82	0.01\\
59.83	0.01\\
59.84	0.01\\
59.85	0.01\\
59.86	0.01\\
59.87	0.01\\
59.88	0.01\\
59.89	0.01\\
59.9	0.01\\
59.91	0.01\\
59.92	0.01\\
59.93	0.01\\
59.94	0.01\\
59.95	0.01\\
59.96	0.01\\
59.97	0.01\\
59.98	0.01\\
59.99	0.01\\
60	0.01\\
60.01	0.01\\
60.02	0.01\\
60.03	0.01\\
60.04	0.01\\
60.05	0.01\\
60.06	0.01\\
60.07	0.01\\
60.08	0.01\\
60.09	0.01\\
60.1	0.01\\
60.11	0.01\\
60.12	0.01\\
60.13	0.01\\
60.14	0.01\\
60.15	0.01\\
60.16	0.01\\
60.17	0.01\\
60.18	0.01\\
60.19	0.01\\
60.2	0.01\\
60.21	0.01\\
60.22	0.01\\
60.23	0.01\\
60.24	0.01\\
60.25	0.01\\
60.26	0.01\\
60.27	0.01\\
60.28	0.01\\
60.29	0.01\\
60.3	0.01\\
60.31	0.01\\
60.32	0.01\\
60.33	0.01\\
60.34	0.01\\
60.35	0.01\\
60.36	0.01\\
60.37	0.01\\
60.38	0.01\\
60.39	0.01\\
60.4	0.01\\
60.41	0.01\\
60.42	0.01\\
60.43	0.01\\
60.44	0.01\\
60.45	0.01\\
60.46	0.01\\
60.47	0.01\\
60.48	0.01\\
60.49	0.01\\
60.5	0.01\\
60.51	0.01\\
60.52	0.01\\
60.53	0.01\\
60.54	0.01\\
60.55	0.01\\
60.56	0.01\\
60.57	0.01\\
60.58	0.01\\
60.59	0.01\\
60.6	0.01\\
60.61	0.01\\
60.62	0.01\\
60.63	0.01\\
60.64	0.01\\
60.65	0.01\\
60.66	0.01\\
60.67	0.01\\
60.68	0.01\\
60.69	0.01\\
60.7	0.01\\
60.71	0.01\\
60.72	0.01\\
60.73	0.01\\
60.74	0.01\\
60.75	0.01\\
60.76	0.01\\
60.77	0.01\\
60.78	0.01\\
60.79	0.01\\
60.8	0.01\\
60.81	0.01\\
60.82	0.01\\
60.83	0.01\\
60.84	0.01\\
60.85	0.01\\
60.86	0.01\\
60.87	0.01\\
60.88	0.01\\
60.89	0.01\\
60.9	0.01\\
60.91	0.01\\
60.92	0.01\\
60.93	0.01\\
60.94	0.01\\
60.95	0.01\\
60.96	0.01\\
60.97	0.01\\
60.98	0.01\\
60.99	0.01\\
61	0.01\\
61.01	0.01\\
61.02	0.01\\
61.03	0.01\\
61.04	0.01\\
61.05	0.01\\
61.06	0.01\\
61.07	0.01\\
61.08	0.01\\
61.09	0.01\\
61.1	0.01\\
61.11	0.01\\
61.12	0.01\\
61.13	0.01\\
61.14	0.01\\
61.15	0.01\\
61.16	0.01\\
61.17	0.01\\
61.18	0.01\\
61.19	0.01\\
61.2	0.01\\
61.21	0.01\\
61.22	0.01\\
61.23	0.01\\
61.24	0.01\\
61.25	0.01\\
61.26	0.01\\
61.27	0.01\\
61.28	0.01\\
61.29	0.01\\
61.3	0.01\\
61.31	0.01\\
61.32	0.01\\
61.33	0.01\\
61.34	0.01\\
61.35	0.01\\
61.36	0.01\\
61.37	0.01\\
61.38	0.01\\
61.39	0.01\\
61.4	0.01\\
61.41	0.01\\
61.42	0.01\\
61.43	0.01\\
61.44	0.01\\
61.45	0.01\\
61.46	0.01\\
61.47	0.01\\
61.48	0.01\\
61.49	0.01\\
61.5	0.01\\
61.51	0.01\\
61.52	0.01\\
61.53	0.01\\
61.54	0.01\\
61.55	0.01\\
61.56	0.01\\
61.57	0.01\\
61.58	0.01\\
61.59	0.01\\
61.6	0.01\\
61.61	0.01\\
61.62	0.01\\
61.63	0.01\\
61.64	0.01\\
61.65	0.01\\
61.66	0.01\\
61.67	0.01\\
61.68	0.01\\
61.69	0.01\\
61.7	0.01\\
61.71	0.01\\
61.72	0.01\\
61.73	0.01\\
61.74	0.01\\
61.75	0.01\\
61.76	0.01\\
61.77	0.01\\
61.78	0.01\\
61.79	0.01\\
61.8	0.01\\
61.81	0.01\\
61.82	0.01\\
61.83	0.01\\
61.84	0.01\\
61.85	0.01\\
61.86	0.01\\
61.87	0.01\\
61.88	0.01\\
61.89	0.01\\
61.9	0.01\\
61.91	0.01\\
61.92	0.01\\
61.93	0.01\\
61.94	0.01\\
61.95	0.01\\
61.96	0.01\\
61.97	0.01\\
61.98	0.01\\
61.99	0.01\\
62	0.01\\
62.01	0.01\\
62.02	0.01\\
62.03	0.01\\
62.04	0.01\\
62.05	0.01\\
62.06	0.01\\
62.07	0.01\\
62.08	0.01\\
62.09	0.01\\
62.1	0.01\\
62.11	0.01\\
62.12	0.01\\
62.13	0.01\\
62.14	0.01\\
62.15	0.01\\
62.16	0.01\\
62.17	0.01\\
62.18	0.01\\
62.19	0.01\\
62.2	0.01\\
62.21	0.01\\
62.22	0.01\\
62.23	0.01\\
62.24	0.01\\
62.25	0.01\\
62.26	0.01\\
62.27	0.01\\
62.28	0.01\\
62.29	0.01\\
62.3	0.01\\
62.31	0.01\\
62.32	0.01\\
62.33	0.01\\
62.34	0.01\\
62.35	0.01\\
62.36	0.01\\
62.37	0.01\\
62.38	0.01\\
62.39	0.01\\
62.4	0.01\\
62.41	0.01\\
62.42	0.01\\
62.43	0.01\\
62.44	0.01\\
62.45	0.01\\
62.46	0.01\\
62.47	0.01\\
62.48	0.01\\
62.49	0.01\\
62.5	0.01\\
62.51	0.01\\
62.52	0.01\\
62.53	0.01\\
62.54	0.01\\
62.55	0.01\\
62.56	0.01\\
62.57	0.01\\
62.58	0.01\\
62.59	0.01\\
62.6	0.01\\
62.61	0.01\\
62.62	0.01\\
62.63	0.01\\
62.64	0.01\\
62.65	0.01\\
62.66	0.01\\
62.67	0.01\\
62.68	0.01\\
62.69	0.01\\
62.7	0.01\\
62.71	0.01\\
62.72	0.01\\
62.73	0.01\\
62.74	0.01\\
62.75	0.01\\
62.76	0.01\\
62.77	0.01\\
62.78	0.01\\
62.79	0.01\\
62.8	0.01\\
62.81	0.01\\
62.82	0.01\\
62.83	0.01\\
62.84	0.01\\
62.85	0.01\\
62.86	0.01\\
62.87	0.01\\
62.88	0.01\\
62.89	0.01\\
62.9	0.01\\
62.91	0.01\\
62.92	0.01\\
62.93	0.01\\
62.94	0.01\\
62.95	0.01\\
62.96	0.01\\
62.97	0.01\\
62.98	0.01\\
62.99	0.01\\
63	0.01\\
63.01	0.01\\
63.02	0.01\\
63.03	0.01\\
63.04	0.01\\
63.05	0.01\\
63.06	0.01\\
63.07	0.01\\
63.08	0.01\\
63.09	0.01\\
63.1	0.01\\
63.11	0.01\\
63.12	0.01\\
63.13	0.01\\
63.14	0.01\\
63.15	0.01\\
63.16	0.01\\
63.17	0.01\\
63.18	0.01\\
63.19	0.01\\
63.2	0.01\\
63.21	0.01\\
63.22	0.01\\
63.23	0.01\\
63.24	0.01\\
63.25	0.01\\
63.26	0.01\\
63.27	0.01\\
63.28	0.01\\
63.29	0.01\\
63.3	0.01\\
63.31	0.01\\
63.32	0.01\\
63.33	0.01\\
63.34	0.01\\
63.35	0.01\\
63.36	0.01\\
63.37	0.01\\
63.38	0.01\\
63.39	0.01\\
63.4	0.01\\
63.41	0.01\\
63.42	0.01\\
63.43	0.01\\
63.44	0.01\\
63.45	0.01\\
63.46	0.01\\
63.47	0.01\\
63.48	0.01\\
63.49	0.01\\
63.5	0.01\\
63.51	0.01\\
63.52	0.01\\
63.53	0.01\\
63.54	0.01\\
63.55	0.01\\
63.56	0.01\\
63.57	0.01\\
63.58	0.01\\
63.59	0.01\\
63.6	0.01\\
63.61	0.01\\
63.62	0.01\\
63.63	0.01\\
63.64	0.01\\
63.65	0.01\\
63.66	0.01\\
63.67	0.01\\
63.68	0.01\\
63.69	0.01\\
63.7	0.01\\
63.71	0.01\\
63.72	0.01\\
63.73	0.01\\
63.74	0.01\\
63.75	0.01\\
63.76	0.01\\
63.77	0.01\\
63.78	0.01\\
63.79	0.01\\
63.8	0.01\\
63.81	0.01\\
63.82	0.01\\
63.83	0.01\\
63.84	0.01\\
63.85	0.01\\
63.86	0.01\\
63.87	0.01\\
63.88	0.01\\
63.89	0.01\\
63.9	0.01\\
63.91	0.01\\
63.92	0.01\\
63.93	0.01\\
63.94	0.01\\
63.95	0.01\\
63.96	0.01\\
63.97	0.01\\
63.98	0.01\\
63.99	0.01\\
64	0.01\\
64.01	0.01\\
64.02	0.01\\
64.03	0.01\\
64.04	0.01\\
64.05	0.01\\
64.06	0.01\\
64.07	0.01\\
64.08	0.01\\
64.09	0.01\\
64.1	0.01\\
64.11	0.01\\
64.12	0.01\\
64.13	0.01\\
64.14	0.01\\
64.15	0.01\\
64.16	0.01\\
64.17	0.01\\
64.18	0.01\\
64.19	0.01\\
64.2	0.01\\
64.21	0.01\\
64.22	0.01\\
64.23	0.01\\
64.24	0.01\\
64.25	0.01\\
64.26	0.01\\
64.27	0.01\\
64.28	0.01\\
64.29	0.01\\
64.3	0.01\\
64.31	0.01\\
64.32	0.01\\
64.33	0.01\\
64.34	0.01\\
64.35	0.01\\
64.36	0.01\\
64.37	0.01\\
64.38	0.01\\
64.39	0.01\\
64.4	0.01\\
64.41	0.01\\
64.42	0.01\\
64.43	0.01\\
64.44	0.01\\
64.45	0.01\\
64.46	0.01\\
64.47	0.01\\
64.48	0.01\\
64.49	0.01\\
64.5	0.01\\
64.51	0.01\\
64.52	0.01\\
64.53	0.01\\
64.54	0.01\\
64.55	0.01\\
64.56	0.01\\
64.57	0.01\\
64.58	0.01\\
64.59	0.01\\
64.6	0.01\\
64.61	0.01\\
64.62	0.01\\
64.63	0.01\\
64.64	0.01\\
64.65	0.01\\
64.66	0.01\\
64.67	0.01\\
64.68	0.01\\
64.69	0.01\\
64.7	0.01\\
64.71	0.01\\
64.72	0.01\\
64.73	0.01\\
64.74	0.01\\
64.75	0.01\\
64.76	0.01\\
64.77	0.01\\
64.78	0.01\\
64.79	0.01\\
64.8	0.01\\
64.81	0.01\\
64.82	0.01\\
64.83	0.01\\
64.84	0.01\\
64.85	0.01\\
64.86	0.01\\
64.87	0.01\\
64.88	0.01\\
64.89	0.01\\
64.9	0.01\\
64.91	0.01\\
64.92	0.01\\
64.93	0.01\\
64.94	0.01\\
64.95	0.01\\
64.96	0.01\\
64.97	0.01\\
64.98	0.01\\
64.99	0.01\\
65	0.01\\
65.01	0.01\\
65.02	0.01\\
65.03	0.01\\
65.04	0.01\\
65.05	0.01\\
65.06	0.01\\
65.07	0.01\\
65.08	0.01\\
65.09	0.01\\
65.1	0.01\\
65.11	0.01\\
65.12	0.01\\
65.13	0.01\\
65.14	0.01\\
65.15	0.01\\
65.16	0.01\\
65.17	0.01\\
65.18	0.01\\
65.19	0.01\\
65.2	0.01\\
65.21	0.01\\
65.22	0.01\\
65.23	0.01\\
65.24	0.01\\
65.25	0.01\\
65.26	0.01\\
65.27	0.01\\
65.28	0.01\\
65.29	0.01\\
65.3	0.01\\
65.31	0.01\\
65.32	0.01\\
65.33	0.01\\
65.34	0.01\\
65.35	0.01\\
65.36	0.01\\
65.37	0.01\\
65.38	0.01\\
65.39	0.01\\
65.4	0.01\\
65.41	0.01\\
65.42	0.01\\
65.43	0.01\\
65.44	0.01\\
65.45	0.01\\
65.46	0.01\\
65.47	0.01\\
65.48	0.01\\
65.49	0.01\\
65.5	0.01\\
65.51	0.01\\
65.52	0.01\\
65.53	0.01\\
65.54	0.01\\
65.55	0.01\\
65.56	0.01\\
65.57	0.01\\
65.58	0.01\\
65.59	0.01\\
65.6	0.01\\
65.61	0.01\\
65.62	0.01\\
65.63	0.01\\
65.64	0.01\\
65.65	0.01\\
65.66	0.01\\
65.67	0.01\\
65.68	0.01\\
65.69	0.01\\
65.7	0.01\\
65.71	0.01\\
65.72	0.01\\
65.73	0.01\\
65.74	0.01\\
65.75	0.01\\
65.76	0.01\\
65.77	0.01\\
65.78	0.01\\
65.79	0.01\\
65.8	0.01\\
65.81	0.01\\
65.82	0.01\\
65.83	0.01\\
65.84	0.01\\
65.85	0.01\\
65.86	0.01\\
65.87	0.01\\
65.88	0.01\\
65.89	0.01\\
65.9	0.01\\
65.91	0.01\\
65.92	0.01\\
65.93	0.01\\
65.94	0.01\\
65.95	0.01\\
65.96	0.01\\
65.97	0.01\\
65.98	0.01\\
65.99	0.01\\
66	0.01\\
66.01	0.01\\
66.02	0.01\\
66.03	0.01\\
66.04	0.01\\
66.05	0.01\\
66.06	0.01\\
66.07	0.01\\
66.08	0.01\\
66.09	0.01\\
66.1	0.01\\
66.11	0.01\\
66.12	0.01\\
66.13	0.01\\
66.14	0.01\\
66.15	0.01\\
66.16	0.01\\
66.17	0.01\\
66.18	0.01\\
66.19	0.01\\
66.2	0.01\\
66.21	0.01\\
66.22	0.01\\
66.23	0.01\\
66.24	0.01\\
66.25	0.01\\
66.26	0.01\\
66.27	0.01\\
66.28	0.01\\
66.29	0.01\\
66.3	0.01\\
66.31	0.01\\
66.32	0.01\\
66.33	0.01\\
66.34	0.01\\
66.35	0.01\\
66.36	0.01\\
66.37	0.01\\
66.38	0.01\\
66.39	0.01\\
66.4	0.01\\
66.41	0.01\\
66.42	0.01\\
66.43	0.01\\
66.44	0.01\\
66.45	0.01\\
66.46	0.01\\
66.47	0.01\\
66.48	0.01\\
66.49	0.01\\
66.5	0.01\\
66.51	0.01\\
66.52	0.01\\
66.53	0.01\\
66.54	0.01\\
66.55	0.01\\
66.56	0.01\\
66.57	0.01\\
66.58	0.01\\
66.59	0.01\\
66.6	0.01\\
66.61	0.01\\
66.62	0.01\\
66.63	0.01\\
66.64	0.01\\
66.65	0.01\\
66.66	0.01\\
66.67	0.01\\
66.68	0.01\\
66.69	0.01\\
66.7	0.01\\
66.71	0.01\\
66.72	0.01\\
66.73	0.01\\
66.74	0.01\\
66.75	0.01\\
66.76	0.01\\
66.77	0.01\\
66.78	0.01\\
66.79	0.01\\
66.8	0.01\\
66.81	0.01\\
66.82	0.01\\
66.83	0.01\\
66.84	0.01\\
66.85	0.01\\
66.86	0.01\\
66.87	0.01\\
66.88	0.01\\
66.89	0.01\\
66.9	0.01\\
66.91	0.01\\
66.92	0.01\\
66.93	0.01\\
66.94	0.01\\
66.95	0.01\\
66.96	0.01\\
66.97	0.01\\
66.98	0.01\\
66.99	0.01\\
67	0.01\\
67.01	0.01\\
67.02	0.01\\
67.03	0.01\\
67.04	0.01\\
67.05	0.01\\
67.06	0.01\\
67.07	0.01\\
67.08	0.01\\
67.09	0.01\\
67.1	0.01\\
67.11	0.01\\
67.12	0.01\\
67.13	0.01\\
67.14	0.01\\
67.15	0.01\\
67.16	0.01\\
67.17	0.01\\
67.18	0.01\\
67.19	0.01\\
67.2	0.01\\
67.21	0.01\\
67.22	0.01\\
67.23	0.01\\
67.24	0.01\\
67.25	0.01\\
67.26	0.01\\
67.27	0.01\\
67.28	0.01\\
67.29	0.01\\
67.3	0.01\\
67.31	0.01\\
67.32	0.01\\
67.33	0.01\\
67.34	0.01\\
67.35	0.01\\
67.36	0.01\\
67.37	0.01\\
67.38	0.01\\
67.39	0.01\\
67.4	0.01\\
67.41	0.01\\
67.42	0.01\\
67.43	0.01\\
67.44	0.01\\
67.45	0.01\\
67.46	0.01\\
67.47	0.01\\
67.48	0.01\\
67.49	0.01\\
67.5	0.01\\
67.51	0.01\\
67.52	0.01\\
67.53	0.01\\
67.54	0.01\\
67.55	0.01\\
67.56	0.01\\
67.57	0.01\\
67.58	0.01\\
67.59	0.01\\
67.6	0.01\\
67.61	0.01\\
67.62	0.01\\
67.63	0.01\\
67.64	0.01\\
67.65	0.01\\
67.66	0.01\\
67.67	0.01\\
67.68	0.01\\
67.69	0.01\\
67.7	0.01\\
67.71	0.01\\
67.72	0.01\\
67.73	0.01\\
67.74	0.01\\
67.75	0.01\\
67.76	0.01\\
67.77	0.01\\
67.78	0.01\\
67.79	0.01\\
67.8	0.01\\
67.81	0.01\\
67.82	0.01\\
67.83	0.01\\
67.84	0.01\\
67.85	0.01\\
67.86	0.01\\
67.87	0.01\\
67.88	0.01\\
67.89	0.01\\
67.9	0.01\\
67.91	0.01\\
67.92	0.01\\
67.93	0.01\\
67.94	0.01\\
67.95	0.01\\
67.96	0.01\\
67.97	0.01\\
67.98	0.01\\
67.99	0.01\\
68	0.01\\
68.01	0.01\\
68.02	0.01\\
68.03	0.01\\
68.04	0.01\\
68.05	0.01\\
68.06	0.01\\
68.07	0.01\\
68.08	0.01\\
68.09	0.01\\
68.1	0.01\\
68.11	0.01\\
68.12	0.01\\
68.13	0.01\\
68.14	0.01\\
68.15	0.01\\
68.16	0.01\\
68.17	0.01\\
68.18	0.01\\
68.19	0.01\\
68.2	0.01\\
68.21	0.01\\
68.22	0.01\\
68.23	0.01\\
68.24	0.01\\
68.25	0.01\\
68.26	0.01\\
68.27	0.01\\
68.28	0.01\\
68.29	0.01\\
68.3	0.01\\
68.31	0.01\\
68.32	0.01\\
68.33	0.01\\
68.34	0.01\\
68.35	0.01\\
68.36	0.01\\
68.37	0.01\\
68.38	0.01\\
68.39	0.01\\
68.4	0.01\\
68.41	0.01\\
68.42	0.01\\
68.43	0.01\\
68.44	0.01\\
68.45	0.01\\
68.46	0.01\\
68.47	0.01\\
68.48	0.01\\
68.49	0.01\\
68.5	0.01\\
68.51	0.01\\
68.52	0.01\\
68.53	0.01\\
68.54	0.01\\
68.55	0.01\\
68.56	0.01\\
68.57	0.01\\
68.58	0.01\\
68.59	0.01\\
68.6	0.01\\
68.61	0.01\\
68.62	0.01\\
68.63	0.01\\
68.64	0.01\\
68.65	0.01\\
68.66	0.01\\
68.67	0.01\\
68.68	0.01\\
68.69	0.01\\
68.7	0.01\\
68.71	0.01\\
68.72	0.01\\
68.73	0.01\\
68.74	0.01\\
68.75	0.01\\
68.76	0.01\\
68.77	0.01\\
68.78	0.01\\
68.79	0.01\\
68.8	0.01\\
68.81	0.01\\
68.82	0.01\\
68.83	0.01\\
68.84	0.01\\
68.85	0.01\\
68.86	0.01\\
68.87	0.01\\
68.88	0.01\\
68.89	0.01\\
68.9	0.01\\
68.91	0.01\\
68.92	0.01\\
68.93	0.01\\
68.94	0.01\\
68.95	0.01\\
68.96	0.01\\
68.97	0.01\\
68.98	0.01\\
68.99	0.01\\
69	0.01\\
69.01	0.01\\
69.02	0.01\\
69.03	0.01\\
69.04	0.01\\
69.05	0.01\\
69.06	0.01\\
69.07	0.01\\
69.08	0.01\\
69.09	0.01\\
69.1	0.01\\
69.11	0.01\\
69.12	0.01\\
69.13	0.01\\
69.14	0.01\\
69.15	0.01\\
69.16	0.01\\
69.17	0.01\\
69.18	0.01\\
69.19	0.01\\
69.2	0.01\\
69.21	0.01\\
69.22	0.01\\
69.23	0.01\\
69.24	0.01\\
69.25	0.01\\
69.26	0.01\\
69.27	0.01\\
69.28	0.01\\
69.29	0.01\\
69.3	0.01\\
69.31	0.01\\
69.32	0.01\\
69.33	0.01\\
69.34	0.01\\
69.35	0.01\\
69.36	0.01\\
69.37	0.01\\
69.38	0.01\\
69.39	0.01\\
69.4	0.01\\
69.41	0.01\\
69.42	0.01\\
69.43	0.01\\
69.44	0.01\\
69.45	0.01\\
69.46	0.01\\
69.47	0.01\\
69.48	0.01\\
69.49	0.01\\
69.5	0.01\\
69.51	0.01\\
69.52	0.01\\
69.53	0.01\\
69.54	0.01\\
69.55	0.01\\
69.56	0.01\\
69.57	0.01\\
69.58	0.01\\
69.59	0.01\\
69.6	0.01\\
69.61	0.01\\
69.62	0.01\\
69.63	0.01\\
69.64	0.01\\
69.65	0.01\\
69.66	0.01\\
69.67	0.01\\
69.68	0.01\\
69.69	0.01\\
69.7	0.01\\
69.71	0.01\\
69.72	0.01\\
69.73	0.01\\
69.74	0.01\\
69.75	0.01\\
69.76	0.01\\
69.77	0.01\\
69.78	0.01\\
69.79	0.01\\
69.8	0.01\\
69.81	0.01\\
69.82	0.01\\
69.83	0.01\\
69.84	0.01\\
69.85	0.01\\
69.86	0.01\\
69.87	0.01\\
69.88	0.01\\
69.89	0.01\\
69.9	0.01\\
69.91	0.01\\
69.92	0.01\\
69.93	0.01\\
69.94	0.01\\
69.95	0.01\\
69.96	0.01\\
69.97	0.01\\
69.98	0.01\\
69.99	0.01\\
70	0.01\\
70.01	0.01\\
70.02	0.01\\
70.03	0.01\\
70.04	0.01\\
70.05	0.01\\
70.06	0.01\\
70.07	0.01\\
70.08	0.01\\
70.09	0.01\\
70.1	0.01\\
70.11	0.01\\
70.12	0.01\\
70.13	0.01\\
70.14	0.01\\
70.15	0.01\\
70.16	0.01\\
70.17	0.01\\
70.18	0.01\\
70.19	0.01\\
70.2	0.01\\
70.21	0.01\\
70.22	0.01\\
70.23	0.01\\
70.24	0.01\\
70.25	0.01\\
70.26	0.01\\
70.27	0.01\\
70.28	0.01\\
70.29	0.01\\
70.3	0.01\\
70.31	0.01\\
70.32	0.01\\
70.33	0.01\\
70.34	0.01\\
70.35	0.01\\
70.36	0.01\\
70.37	0.01\\
70.38	0.01\\
70.39	0.01\\
70.4	0.01\\
70.41	0.01\\
70.42	0.01\\
70.43	0.01\\
70.44	0.01\\
70.45	0.01\\
70.46	0.01\\
70.47	0.01\\
70.48	0.01\\
70.49	0.01\\
70.5	0.01\\
70.51	0.01\\
70.52	0.01\\
70.53	0.01\\
70.54	0.01\\
70.55	0.01\\
70.56	0.01\\
70.57	0.01\\
70.58	0.01\\
70.59	0.01\\
70.6	0.01\\
70.61	0.01\\
70.62	0.01\\
70.63	0.01\\
70.64	0.01\\
70.65	0.01\\
70.66	0.01\\
70.67	0.01\\
70.68	0.01\\
70.69	0.01\\
70.7	0.01\\
70.71	0.01\\
70.72	0.01\\
70.73	0.01\\
70.74	0.01\\
70.75	0.01\\
70.76	0.01\\
70.77	0.01\\
70.78	0.01\\
70.79	0.01\\
70.8	0.01\\
70.81	0.01\\
70.82	0.01\\
70.83	0.01\\
70.84	0.01\\
70.85	0.01\\
70.86	0.01\\
70.87	0.01\\
70.88	0.01\\
70.89	0.01\\
70.9	0.01\\
70.91	0.01\\
70.92	0.01\\
70.93	0.01\\
70.94	0.01\\
70.95	0.01\\
70.96	0.01\\
70.97	0.01\\
70.98	0.01\\
70.99	0.01\\
71	0.01\\
71.01	0.01\\
71.02	0.01\\
71.03	0.01\\
71.04	0.01\\
71.05	0.01\\
71.06	0.01\\
71.07	0.01\\
71.08	0.01\\
71.09	0.01\\
71.1	0.01\\
71.11	0.01\\
71.12	0.01\\
71.13	0.01\\
71.14	0.01\\
71.15	0.01\\
71.16	0.01\\
71.17	0.01\\
71.18	0.01\\
71.19	0.01\\
71.2	0.01\\
71.21	0.01\\
71.22	0.01\\
71.23	0.01\\
71.24	0.01\\
71.25	0.01\\
71.26	0.01\\
71.27	0.01\\
71.28	0.01\\
71.29	0.01\\
71.3	0.01\\
71.31	0.01\\
71.32	0.01\\
71.33	0.01\\
71.34	0.01\\
71.35	0.01\\
71.36	0.01\\
71.37	0.01\\
71.38	0.01\\
71.39	0.01\\
71.4	0.01\\
71.41	0.01\\
71.42	0.01\\
71.43	0.01\\
71.44	0.01\\
71.45	0.01\\
71.46	0.01\\
71.47	0.01\\
71.48	0.01\\
71.49	0.01\\
71.5	0.01\\
71.51	0.01\\
71.52	0.01\\
71.53	0.01\\
71.54	0.01\\
71.55	0.01\\
71.56	0.01\\
71.57	0.01\\
71.58	0.01\\
71.59	0.01\\
71.6	0.01\\
71.61	0.01\\
71.62	0.01\\
71.63	0.01\\
71.64	0.01\\
71.65	0.01\\
71.66	0.01\\
71.67	0.01\\
71.68	0.01\\
71.69	0.01\\
71.7	0.01\\
71.71	0.01\\
71.72	0.01\\
71.73	0.01\\
71.74	0.01\\
71.75	0.01\\
71.76	0.01\\
71.77	0.01\\
71.78	0.01\\
71.79	0.01\\
71.8	0.01\\
71.81	0.01\\
71.82	0.01\\
71.83	0.01\\
71.84	0.01\\
71.85	0.01\\
71.86	0.01\\
71.87	0.01\\
71.88	0.01\\
71.89	0.01\\
71.9	0.01\\
71.91	0.01\\
71.92	0.01\\
71.93	0.01\\
71.94	0.01\\
71.95	0.01\\
71.96	0.01\\
71.97	0.01\\
71.98	0.01\\
71.99	0.01\\
72	0.01\\
72.01	0.01\\
72.02	0.01\\
72.03	0.01\\
72.04	0.01\\
72.05	0.01\\
72.06	0.01\\
72.07	0.01\\
72.08	0.01\\
72.09	0.01\\
72.1	0.01\\
72.11	0.01\\
72.12	0.01\\
72.13	0.01\\
72.14	0.01\\
72.15	0.01\\
72.16	0.01\\
72.17	0.01\\
72.18	0.01\\
72.19	0.01\\
72.2	0.01\\
72.21	0.01\\
72.22	0.01\\
72.23	0.01\\
72.24	0.01\\
72.25	0.01\\
72.26	0.01\\
72.27	0.01\\
72.28	0.01\\
72.29	0.01\\
72.3	0.01\\
72.31	0.01\\
72.32	0.01\\
72.33	0.01\\
72.34	0.01\\
72.35	0.01\\
72.36	0.01\\
72.37	0.01\\
72.38	0.01\\
72.39	0.01\\
72.4	0.01\\
72.41	0.01\\
72.42	0.01\\
72.43	0.01\\
72.44	0.01\\
72.45	0.01\\
72.46	0.01\\
72.47	0.01\\
72.48	0.01\\
72.49	0.01\\
72.5	0.01\\
72.51	0.01\\
72.52	0.01\\
72.53	0.01\\
72.54	0.01\\
72.55	0.01\\
72.56	0.01\\
72.57	0.01\\
72.58	0.01\\
72.59	0.01\\
72.6	0.01\\
72.61	0.01\\
72.62	0.01\\
72.63	0.01\\
72.64	0.01\\
72.65	0.01\\
72.66	0.01\\
72.67	0.01\\
72.68	0.01\\
72.69	0.01\\
72.7	0.01\\
72.71	0.01\\
72.72	0.01\\
72.73	0.01\\
72.74	0.01\\
72.75	0.01\\
72.76	0.01\\
72.77	0.01\\
72.78	0.01\\
72.79	0.01\\
72.8	0.01\\
72.81	0.01\\
72.82	0.01\\
72.83	0.01\\
72.84	0.01\\
72.85	0.01\\
72.86	0.01\\
72.87	0.01\\
72.88	0.01\\
72.89	0.01\\
72.9	0.01\\
72.91	0.01\\
72.92	0.01\\
72.93	0.01\\
72.94	0.01\\
72.95	0.01\\
72.96	0.01\\
72.97	0.01\\
72.98	0.01\\
72.99	0.01\\
73	0.01\\
73.01	0.01\\
73.02	0.01\\
73.03	0.01\\
73.04	0.01\\
73.05	0.01\\
73.06	0.01\\
73.07	0.01\\
73.08	0.01\\
73.09	0.01\\
73.1	0.01\\
73.11	0.01\\
73.12	0.01\\
73.13	0.01\\
73.14	0.01\\
73.15	0.01\\
73.16	0.01\\
73.17	0.01\\
73.18	0.01\\
73.19	0.01\\
73.2	0.01\\
73.21	0.01\\
73.22	0.01\\
73.23	0.01\\
73.24	0.01\\
73.25	0.01\\
73.26	0.01\\
73.27	0.01\\
73.28	0.01\\
73.29	0.01\\
73.3	0.01\\
73.31	0.01\\
73.32	0.01\\
73.33	0.01\\
73.34	0.01\\
73.35	0.01\\
73.36	0.01\\
73.37	0.01\\
73.38	0.01\\
73.39	0.01\\
73.4	0.01\\
73.41	0.01\\
73.42	0.01\\
73.43	0.01\\
73.44	0.01\\
73.45	0.01\\
73.46	0.01\\
73.47	0.01\\
73.48	0.01\\
73.49	0.01\\
73.5	0.01\\
73.51	0.01\\
73.52	0.01\\
73.53	0.01\\
73.54	0.01\\
73.55	0.01\\
73.56	0.01\\
73.57	0.01\\
73.58	0.01\\
73.59	0.01\\
73.6	0.01\\
73.61	0.01\\
73.62	0.01\\
73.63	0.01\\
73.64	0.01\\
73.65	0.01\\
73.66	0.01\\
73.67	0.01\\
73.68	0.01\\
73.69	0.01\\
73.7	0.01\\
73.71	0.01\\
73.72	0.01\\
73.73	0.01\\
73.74	0.01\\
73.75	0.01\\
73.76	0.01\\
73.77	0.01\\
73.78	0.01\\
73.79	0.01\\
73.8	0.01\\
73.81	0.01\\
73.82	0.01\\
73.83	0.01\\
73.84	0.01\\
73.85	0.01\\
73.86	0.01\\
73.87	0.01\\
73.88	0.01\\
73.89	0.01\\
73.9	0.01\\
73.91	0.01\\
73.92	0.01\\
73.93	0.01\\
73.94	0.01\\
73.95	0.01\\
73.96	0.01\\
73.97	0.01\\
73.98	0.01\\
73.99	0.01\\
74	0.01\\
74.01	0.01\\
74.02	0.01\\
74.03	0.01\\
74.04	0.01\\
74.05	0.01\\
74.06	0.01\\
74.07	0.01\\
74.08	0.01\\
74.09	0.01\\
74.1	0.01\\
74.11	0.01\\
74.12	0.01\\
74.13	0.01\\
74.14	0.01\\
74.15	0.01\\
74.16	0.01\\
74.17	0.01\\
74.18	0.01\\
74.19	0.01\\
74.2	0.01\\
74.21	0.01\\
74.22	0.01\\
74.23	0.01\\
74.24	0.01\\
74.25	0.01\\
74.26	0.01\\
74.27	0.01\\
74.28	0.01\\
74.29	0.01\\
74.3	0.01\\
74.31	0.01\\
74.32	0.01\\
74.33	0.01\\
74.34	0.01\\
74.35	0.01\\
74.36	0.01\\
74.37	0.01\\
74.38	0.01\\
74.39	0.01\\
74.4	0.01\\
74.41	0.01\\
74.42	0.01\\
74.43	0.01\\
74.44	0.01\\
74.45	0.01\\
74.46	0.01\\
74.47	0.01\\
74.48	0.01\\
74.49	0.01\\
74.5	0.01\\
74.51	0.01\\
74.52	0.01\\
74.53	0.01\\
74.54	0.01\\
74.55	0.01\\
74.56	0.01\\
74.57	0.01\\
74.58	0.01\\
74.59	0.01\\
74.6	0.01\\
74.61	0.01\\
74.62	0.01\\
74.63	0.01\\
74.64	0.01\\
74.65	0.01\\
74.66	0.01\\
74.67	0.01\\
74.68	0.01\\
74.69	0.01\\
74.7	0.01\\
74.71	0.01\\
74.72	0.01\\
74.73	0.01\\
74.74	0.01\\
74.75	0.01\\
74.76	0.01\\
74.77	0.01\\
74.78	0.01\\
74.79	0.01\\
74.8	0.01\\
74.81	0.01\\
74.82	0.01\\
74.83	0.01\\
74.84	0.01\\
74.85	0.01\\
74.86	0.01\\
74.87	0.01\\
74.88	0.01\\
74.89	0.01\\
74.9	0.01\\
74.91	0.01\\
74.92	0.01\\
74.93	0.01\\
74.94	0.01\\
74.95	0.01\\
74.96	0.01\\
74.97	0.01\\
74.98	0.01\\
74.99	0.01\\
75	0.01\\
75.01	0.01\\
75.02	0.01\\
75.03	0.01\\
75.04	0.01\\
75.05	0.01\\
75.06	0.01\\
75.07	0.01\\
75.08	0.01\\
75.09	0.01\\
75.1	0.01\\
75.11	0.01\\
75.12	0.01\\
75.13	0.01\\
75.14	0.01\\
75.15	0.01\\
75.16	0.01\\
75.17	0.01\\
75.18	0.01\\
75.19	0.01\\
75.2	0.01\\
75.21	0.01\\
75.22	0.01\\
75.23	0.01\\
75.24	0.01\\
75.25	0.01\\
75.26	0.01\\
75.27	0.01\\
75.28	0.01\\
75.29	0.01\\
75.3	0.01\\
75.31	0.01\\
75.32	0.01\\
75.33	0.01\\
75.34	0.01\\
75.35	0.01\\
75.36	0.01\\
75.37	0.01\\
75.38	0.01\\
75.39	0.01\\
75.4	0.01\\
75.41	0.01\\
75.42	0.01\\
75.43	0.01\\
75.44	0.01\\
75.45	0.01\\
75.46	0.01\\
75.47	0.01\\
75.48	0.01\\
75.49	0.01\\
75.5	0.01\\
75.51	0.01\\
75.52	0.01\\
75.53	0.01\\
75.54	0.01\\
75.55	0.01\\
75.56	0.01\\
75.57	0.01\\
75.58	0.01\\
75.59	0.01\\
75.6	0.01\\
75.61	0.01\\
75.62	0.01\\
75.63	0.01\\
75.64	0.01\\
75.65	0.01\\
75.66	0.01\\
75.67	0.01\\
75.68	0.01\\
75.69	0.01\\
75.7	0.01\\
75.71	0.01\\
75.72	0.01\\
75.73	0.01\\
75.74	0.01\\
75.75	0.01\\
75.76	0.01\\
75.77	0.01\\
75.78	0.01\\
75.79	0.01\\
75.8	0.01\\
75.81	0.01\\
75.82	0.01\\
75.83	0.01\\
75.84	0.01\\
75.85	0.01\\
75.86	0.01\\
75.87	0.01\\
75.88	0.01\\
75.89	0.01\\
75.9	0.01\\
75.91	0.01\\
75.92	0.01\\
75.93	0.01\\
75.94	0.01\\
75.95	0.01\\
75.96	0.01\\
75.97	0.01\\
75.98	0.01\\
75.99	0.01\\
76	0.01\\
76.01	0.01\\
76.02	0.01\\
76.03	0.01\\
76.04	0.01\\
76.05	0.01\\
76.06	0.01\\
76.07	0.01\\
76.08	0.01\\
76.09	0.01\\
76.1	0.01\\
76.11	0.01\\
76.12	0.01\\
76.13	0.01\\
76.14	0.01\\
76.15	0.01\\
76.16	0.01\\
76.17	0.01\\
76.18	0.01\\
76.19	0.01\\
76.2	0.01\\
76.21	0.01\\
76.22	0.01\\
76.23	0.01\\
76.24	0.01\\
76.25	0.01\\
76.26	0.01\\
76.27	0.01\\
76.28	0.01\\
76.29	0.01\\
76.3	0.01\\
76.31	0.01\\
76.32	0.01\\
76.33	0.01\\
76.34	0.01\\
76.35	0.01\\
76.36	0.01\\
76.37	0.01\\
76.38	0.01\\
76.39	0.01\\
76.4	0.01\\
76.41	0.01\\
76.42	0.01\\
76.43	0.01\\
76.44	0.01\\
76.45	0.01\\
76.46	0.01\\
76.47	0.01\\
76.48	0.01\\
76.49	0.01\\
76.5	0.01\\
76.51	0.01\\
76.52	0.01\\
76.53	0.01\\
76.54	0.01\\
76.55	0.01\\
76.56	0.01\\
76.57	0.01\\
76.58	0.01\\
76.59	0.01\\
76.6	0.01\\
76.61	0.01\\
76.62	0.01\\
76.63	0.01\\
76.64	0.01\\
76.65	0.01\\
76.66	0.01\\
76.67	0.01\\
76.68	0.01\\
76.69	0.01\\
76.7	0.01\\
76.71	0.01\\
76.72	0.01\\
76.73	0.01\\
76.74	0.01\\
76.75	0.01\\
76.76	0.01\\
76.77	0.01\\
76.78	0.01\\
76.79	0.01\\
76.8	0.01\\
76.81	0.01\\
76.82	0.01\\
76.83	0.01\\
76.84	0.01\\
76.85	0.01\\
76.86	0.01\\
76.87	0.01\\
76.88	0.01\\
76.89	0.01\\
76.9	0.01\\
76.91	0.01\\
76.92	0.01\\
76.93	0.01\\
76.94	0.01\\
76.95	0.01\\
76.96	0.01\\
76.97	0.01\\
76.98	0.01\\
76.99	0.01\\
77	0.01\\
77.01	0.01\\
77.02	0.01\\
77.03	0.01\\
77.04	0.01\\
77.05	0.01\\
77.06	0.01\\
77.07	0.01\\
77.08	0.01\\
77.09	0.01\\
77.1	0.01\\
77.11	0.01\\
77.12	0.01\\
77.13	0.01\\
77.14	0.01\\
77.15	0.01\\
77.16	0.01\\
77.17	0.01\\
77.18	0.01\\
77.19	0.01\\
77.2	0.01\\
77.21	0.01\\
77.22	0.01\\
77.23	0.01\\
77.24	0.01\\
77.25	0.01\\
77.26	0.01\\
77.27	0.01\\
77.28	0.01\\
77.29	0.01\\
77.3	0.01\\
77.31	0.01\\
77.32	0.01\\
77.33	0.01\\
77.34	0.01\\
77.35	0.01\\
77.36	0.01\\
77.37	0.01\\
77.38	0.01\\
77.39	0.01\\
77.4	0.01\\
77.41	0.01\\
77.42	0.01\\
77.43	0.01\\
77.44	0.01\\
77.45	0.01\\
77.46	0.01\\
77.47	0.01\\
77.48	0.01\\
77.49	0.01\\
77.5	0.01\\
77.51	0.01\\
77.52	0.01\\
77.53	0.01\\
77.54	0.01\\
77.55	0.01\\
77.56	0.01\\
77.57	0.01\\
77.58	0.01\\
77.59	0.01\\
77.6	0.01\\
77.61	0.01\\
77.62	0.01\\
77.63	0.01\\
77.64	0.01\\
77.65	0.01\\
77.66	0.01\\
77.67	0.01\\
77.68	0.01\\
77.69	0.01\\
77.7	0.01\\
77.71	0.01\\
77.72	0.01\\
77.73	0.01\\
77.74	0.01\\
77.75	0.01\\
77.76	0.01\\
77.77	0.01\\
77.78	0.01\\
77.79	0.01\\
77.8	0.01\\
77.81	0.01\\
77.82	0.01\\
77.83	0.01\\
77.84	0.01\\
77.85	0.01\\
77.86	0.01\\
77.87	0.01\\
77.88	0.01\\
77.89	0.01\\
77.9	0.01\\
77.91	0.01\\
77.92	0.01\\
77.93	0.01\\
77.94	0.01\\
77.95	0.01\\
77.96	0.01\\
77.97	0.01\\
77.98	0.01\\
77.99	0.01\\
78	0.01\\
78.01	0.01\\
78.02	0.01\\
78.03	0.01\\
78.04	0.01\\
78.05	0.01\\
78.06	0.01\\
78.07	0.01\\
78.08	0.01\\
78.09	0.01\\
78.1	0.01\\
78.11	0.01\\
78.12	0.01\\
78.13	0.01\\
78.14	0.01\\
78.15	0.01\\
78.16	0.01\\
78.17	0.01\\
78.18	0.01\\
78.19	0.01\\
78.2	0.01\\
78.21	0.01\\
78.22	0.01\\
78.23	0.01\\
78.24	0.01\\
78.25	0.01\\
78.26	0.01\\
78.27	0.01\\
78.28	0.01\\
78.29	0.01\\
78.3	0.01\\
78.31	0.01\\
78.32	0.01\\
78.33	0.01\\
78.34	0.01\\
78.35	0.01\\
78.36	0.01\\
78.37	0.01\\
78.38	0.01\\
78.39	0.01\\
78.4	0.01\\
78.41	0.01\\
78.42	0.01\\
78.43	0.01\\
78.44	0.01\\
78.45	0.01\\
78.46	0.01\\
78.47	0.01\\
78.48	0.01\\
78.49	0.01\\
78.5	0.01\\
78.51	0.01\\
78.52	0.01\\
78.53	0.01\\
78.54	0.01\\
78.55	0.01\\
78.56	0.01\\
78.57	0.01\\
78.58	0.01\\
78.59	0.01\\
78.6	0.01\\
78.61	0.01\\
78.62	0.01\\
78.63	0.01\\
78.64	0.01\\
78.65	0.01\\
78.66	0.01\\
78.67	0.01\\
78.68	0.01\\
78.69	0.01\\
78.7	0.01\\
78.71	0.01\\
78.72	0.01\\
78.73	0.01\\
78.74	0.01\\
78.75	0.01\\
78.76	0.01\\
78.77	0.01\\
78.78	0.01\\
78.79	0.01\\
78.8	0.01\\
78.81	0.01\\
78.82	0.01\\
78.83	0.01\\
78.84	0.01\\
78.85	0.01\\
78.86	0.01\\
78.87	0.01\\
78.88	0.01\\
78.89	0.01\\
78.9	0.01\\
78.91	0.01\\
78.92	0.01\\
78.93	0.01\\
78.94	0.01\\
78.95	0.01\\
78.96	0.01\\
78.97	0.01\\
78.98	0.01\\
78.99	0.01\\
79	0.01\\
79.01	0.01\\
79.02	0.01\\
79.03	0.01\\
79.04	0.01\\
79.05	0.01\\
79.06	0.01\\
79.07	0.01\\
79.08	0.01\\
79.09	0.01\\
79.1	0.01\\
79.11	0.01\\
79.12	0.01\\
79.13	0.01\\
79.14	0.01\\
79.15	0.01\\
79.16	0.01\\
79.17	0.01\\
79.18	0.01\\
79.19	0.01\\
79.2	0.01\\
79.21	0.01\\
79.22	0.01\\
79.23	0.01\\
79.24	0.01\\
79.25	0.01\\
79.26	0.01\\
79.27	0.01\\
79.28	0.01\\
79.29	0.01\\
79.3	0.01\\
79.31	0.01\\
79.32	0.01\\
79.33	0.01\\
79.34	0.01\\
79.35	0.01\\
79.36	0.01\\
79.37	0.01\\
79.38	0.01\\
79.39	0.01\\
79.4	0.01\\
79.41	0.01\\
79.42	0.01\\
79.43	0.01\\
79.44	0.01\\
79.45	0.01\\
79.46	0.01\\
79.47	0.01\\
79.48	0.01\\
79.49	0.01\\
79.5	0.01\\
79.51	0.01\\
79.52	0.01\\
79.53	0.01\\
79.54	0.01\\
79.55	0.01\\
79.56	0.01\\
79.57	0.01\\
79.58	0.01\\
79.59	0.01\\
79.6	0.01\\
79.61	0.01\\
79.62	0.01\\
79.63	0.01\\
79.64	0.01\\
79.65	0.01\\
79.66	0.01\\
79.67	0.01\\
79.68	0.01\\
79.69	0.01\\
79.7	0.01\\
79.71	0.01\\
79.72	0.01\\
79.73	0.01\\
79.74	0.01\\
79.75	0.01\\
79.76	0.01\\
79.77	0.01\\
79.78	0.01\\
79.79	0.01\\
79.8	0.01\\
79.81	0.01\\
79.82	0.01\\
79.83	0.01\\
79.84	0.01\\
79.85	0.01\\
79.86	0.01\\
79.87	0.01\\
79.88	0.01\\
79.89	0.01\\
79.9	0.01\\
79.91	0.01\\
79.92	0.01\\
79.93	0.01\\
79.94	0.01\\
79.95	0.01\\
79.96	0.01\\
79.97	0.01\\
79.98	0.01\\
79.99	0.01\\
80	0.01\\
80.01	0.01\\
};
\addplot [color=green,solid]
  table[row sep=crcr]{%
80.01	0.01\\
80.02	0.01\\
80.03	0.01\\
80.04	0.01\\
80.05	0.01\\
80.06	0.01\\
80.07	0.01\\
80.08	0.01\\
80.09	0.01\\
80.1	0.01\\
80.11	0.01\\
80.12	0.01\\
80.13	0.01\\
80.14	0.01\\
80.15	0.01\\
80.16	0.01\\
80.17	0.01\\
80.18	0.01\\
80.19	0.01\\
80.2	0.01\\
80.21	0.01\\
80.22	0.01\\
80.23	0.01\\
80.24	0.01\\
80.25	0.01\\
80.26	0.01\\
80.27	0.01\\
80.28	0.01\\
80.29	0.01\\
80.3	0.01\\
80.31	0.01\\
80.32	0.01\\
80.33	0.01\\
80.34	0.01\\
80.35	0.01\\
80.36	0.01\\
80.37	0.01\\
80.38	0.01\\
80.39	0.01\\
80.4	0.01\\
80.41	0.01\\
80.42	0.01\\
80.43	0.01\\
80.44	0.01\\
80.45	0.01\\
80.46	0.01\\
80.47	0.01\\
80.48	0.01\\
80.49	0.01\\
80.5	0.01\\
80.51	0.01\\
80.52	0.01\\
80.53	0.01\\
80.54	0.01\\
80.55	0.01\\
80.56	0.01\\
80.57	0.01\\
80.58	0.01\\
80.59	0.01\\
80.6	0.01\\
80.61	0.01\\
80.62	0.01\\
80.63	0.01\\
80.64	0.01\\
80.65	0.01\\
80.66	0.01\\
80.67	0.01\\
80.68	0.01\\
80.69	0.01\\
80.7	0.01\\
80.71	0.01\\
80.72	0.01\\
80.73	0.01\\
80.74	0.01\\
80.75	0.01\\
80.76	0.01\\
80.77	0.01\\
80.78	0.01\\
80.79	0.01\\
80.8	0.01\\
80.81	0.01\\
80.82	0.01\\
80.83	0.01\\
80.84	0.01\\
80.85	0.01\\
80.86	0.01\\
80.87	0.01\\
80.88	0.01\\
80.89	0.01\\
80.9	0.01\\
80.91	0.01\\
80.92	0.01\\
80.93	0.01\\
80.94	0.01\\
80.95	0.01\\
80.96	0.01\\
80.97	0.01\\
80.98	0.01\\
80.99	0.01\\
81	0.01\\
81.01	0.01\\
81.02	0.01\\
81.03	0.01\\
81.04	0.01\\
81.05	0.01\\
81.06	0.01\\
81.07	0.01\\
81.08	0.01\\
81.09	0.01\\
81.1	0.01\\
81.11	0.01\\
81.12	0.01\\
81.13	0.01\\
81.14	0.01\\
81.15	0.01\\
81.16	0.01\\
81.17	0.01\\
81.18	0.01\\
81.19	0.01\\
81.2	0.01\\
81.21	0.01\\
81.22	0.01\\
81.23	0.01\\
81.24	0.01\\
81.25	0.01\\
81.26	0.01\\
81.27	0.01\\
81.28	0.01\\
81.29	0.01\\
81.3	0.01\\
81.31	0.01\\
81.32	0.01\\
81.33	0.01\\
81.34	0.01\\
81.35	0.01\\
81.36	0.01\\
81.37	0.01\\
81.38	0.01\\
81.39	0.01\\
81.4	0.01\\
81.41	0.01\\
81.42	0.01\\
81.43	0.01\\
81.44	0.01\\
81.45	0.01\\
81.46	0.01\\
81.47	0.01\\
81.48	0.01\\
81.49	0.01\\
81.5	0.01\\
81.51	0.01\\
81.52	0.01\\
81.53	0.01\\
81.54	0.01\\
81.55	0.01\\
81.56	0.01\\
81.57	0.01\\
81.58	0.01\\
81.59	0.01\\
81.6	0.01\\
81.61	0.01\\
81.62	0.01\\
81.63	0.01\\
81.64	0.01\\
81.65	0.01\\
81.66	0.01\\
81.67	0.01\\
81.68	0.01\\
81.69	0.01\\
81.7	0.01\\
81.71	0.01\\
81.72	0.01\\
81.73	0.01\\
81.74	0.01\\
81.75	0.01\\
81.76	0.01\\
81.77	0.01\\
81.78	0.01\\
81.79	0.01\\
81.8	0.01\\
81.81	0.01\\
81.82	0.01\\
81.83	0.01\\
81.84	0.01\\
81.85	0.01\\
81.86	0.01\\
81.87	0.01\\
81.88	0.01\\
81.89	0.01\\
81.9	0.01\\
81.91	0.01\\
81.92	0.01\\
81.93	0.01\\
81.94	0.01\\
81.95	0.01\\
81.96	0.01\\
81.97	0.01\\
81.98	0.01\\
81.99	0.01\\
82	0.01\\
82.01	0.01\\
82.02	0.01\\
82.03	0.01\\
82.04	0.01\\
82.05	0.01\\
82.06	0.01\\
82.07	0.01\\
82.08	0.01\\
82.09	0.01\\
82.1	0.01\\
82.11	0.01\\
82.12	0.01\\
82.13	0.01\\
82.14	0.01\\
82.15	0.01\\
82.16	0.01\\
82.17	0.01\\
82.18	0.01\\
82.19	0.01\\
82.2	0.01\\
82.21	0.01\\
82.22	0.01\\
82.23	0.01\\
82.24	0.01\\
82.25	0.01\\
82.26	0.01\\
82.27	0.01\\
82.28	0.01\\
82.29	0.01\\
82.3	0.01\\
82.31	0.01\\
82.32	0.01\\
82.33	0.01\\
82.34	0.01\\
82.35	0.01\\
82.36	0.01\\
82.37	0.01\\
82.38	0.01\\
82.39	0.01\\
82.4	0.01\\
82.41	0.01\\
82.42	0.01\\
82.43	0.01\\
82.44	0.01\\
82.45	0.01\\
82.46	0.01\\
82.47	0.01\\
82.48	0.01\\
82.49	0.01\\
82.5	0.01\\
82.51	0.01\\
82.52	0.01\\
82.53	0.01\\
82.54	0.01\\
82.55	0.01\\
82.56	0.01\\
82.57	0.01\\
82.58	0.01\\
82.59	0.01\\
82.6	0.01\\
82.61	0.01\\
82.62	0.01\\
82.63	0.01\\
82.64	0.01\\
82.65	0.01\\
82.66	0.01\\
82.67	0.01\\
82.68	0.01\\
82.69	0.01\\
82.7	0.01\\
82.71	0.01\\
82.72	0.01\\
82.73	0.01\\
82.74	0.01\\
82.75	0.01\\
82.76	0.01\\
82.77	0.01\\
82.78	0.01\\
82.79	0.01\\
82.8	0.01\\
82.81	0.01\\
82.82	0.01\\
82.83	0.01\\
82.84	0.01\\
82.85	0.01\\
82.86	0.01\\
82.87	0.01\\
82.88	0.01\\
82.89	0.01\\
82.9	0.01\\
82.91	0.01\\
82.92	0.01\\
82.93	0.01\\
82.94	0.01\\
82.95	0.01\\
82.96	0.01\\
82.97	0.01\\
82.98	0.01\\
82.99	0.01\\
83	0.01\\
83.01	0.01\\
83.02	0.01\\
83.03	0.01\\
83.04	0.01\\
83.05	0.01\\
83.06	0.01\\
83.07	0.01\\
83.08	0.01\\
83.09	0.01\\
83.1	0.01\\
83.11	0.01\\
83.12	0.01\\
83.13	0.01\\
83.14	0.01\\
83.15	0.01\\
83.16	0.01\\
83.17	0.01\\
83.18	0.01\\
83.19	0.01\\
83.2	0.01\\
83.21	0.01\\
83.22	0.01\\
83.23	0.01\\
83.24	0.01\\
83.25	0.01\\
83.26	0.01\\
83.27	0.01\\
83.28	0.01\\
83.29	0.01\\
83.3	0.01\\
83.31	0.01\\
83.32	0.01\\
83.33	0.01\\
83.34	0.01\\
83.35	0.01\\
83.36	0.01\\
83.37	0.01\\
83.38	0.01\\
83.39	0.01\\
83.4	0.01\\
83.41	0.01\\
83.42	0.01\\
83.43	0.01\\
83.44	0.01\\
83.45	0.01\\
83.46	0.01\\
83.47	0.01\\
83.48	0.01\\
83.49	0.01\\
83.5	0.01\\
83.51	0.01\\
83.52	0.01\\
83.53	0.01\\
83.54	0.01\\
83.55	0.01\\
83.56	0.01\\
83.57	0.01\\
83.58	0.01\\
83.59	0.01\\
83.6	0.01\\
83.61	0.01\\
83.62	0.01\\
83.63	0.01\\
83.64	0.01\\
83.65	0.01\\
83.66	0.01\\
83.67	0.01\\
83.68	0.01\\
83.69	0.01\\
83.7	0.01\\
83.71	0.01\\
83.72	0.01\\
83.73	0.01\\
83.74	0.01\\
83.75	0.01\\
83.76	0.01\\
83.77	0.01\\
83.78	0.01\\
83.79	0.01\\
83.8	0.01\\
83.81	0.01\\
83.82	0.01\\
83.83	0.01\\
83.84	0.01\\
83.85	0.01\\
83.86	0.01\\
83.87	0.01\\
83.88	0.01\\
83.89	0.01\\
83.9	0.01\\
83.91	0.01\\
83.92	0.01\\
83.93	0.01\\
83.94	0.01\\
83.95	0.01\\
83.96	0.01\\
83.97	0.01\\
83.98	0.01\\
83.99	0.01\\
84	0.01\\
84.01	0.01\\
84.02	0.01\\
84.03	0.01\\
84.04	0.01\\
84.05	0.01\\
84.06	0.01\\
84.07	0.01\\
84.08	0.01\\
84.09	0.01\\
84.1	0.01\\
84.11	0.01\\
84.12	0.01\\
84.13	0.01\\
84.14	0.01\\
84.15	0.01\\
84.16	0.01\\
84.17	0.01\\
84.18	0.01\\
84.19	0.01\\
84.2	0.01\\
84.21	0.01\\
84.22	0.01\\
84.23	0.01\\
84.24	0.01\\
84.25	0.01\\
84.26	0.01\\
84.27	0.01\\
84.28	0.01\\
84.29	0.01\\
84.3	0.01\\
84.31	0.01\\
84.32	0.01\\
84.33	0.01\\
84.34	0.01\\
84.35	0.01\\
84.36	0.01\\
84.37	0.01\\
84.38	0.01\\
84.39	0.01\\
84.4	0.01\\
84.41	0.01\\
84.42	0.01\\
84.43	0.01\\
84.44	0.01\\
84.45	0.01\\
84.46	0.01\\
84.47	0.01\\
84.48	0.01\\
84.49	0.01\\
84.5	0.01\\
84.51	0.01\\
84.52	0.01\\
84.53	0.01\\
84.54	0.01\\
84.55	0.01\\
84.56	0.01\\
84.57	0.01\\
84.58	0.01\\
84.59	0.01\\
84.6	0.01\\
84.61	0.01\\
84.62	0.01\\
84.63	0.01\\
84.64	0.01\\
84.65	0.01\\
84.66	0.01\\
84.67	0.01\\
84.68	0.01\\
84.69	0.01\\
84.7	0.01\\
84.71	0.01\\
84.72	0.01\\
84.73	0.01\\
84.74	0.01\\
84.75	0.01\\
84.76	0.01\\
84.77	0.01\\
84.78	0.01\\
84.79	0.01\\
84.8	0.01\\
84.81	0.01\\
84.82	0.01\\
84.83	0.01\\
84.84	0.01\\
84.85	0.01\\
84.86	0.01\\
84.87	0.01\\
84.88	0.01\\
84.89	0.01\\
84.9	0.01\\
84.91	0.01\\
84.92	0.01\\
84.93	0.01\\
84.94	0.01\\
84.95	0.01\\
84.96	0.01\\
84.97	0.01\\
84.98	0.01\\
84.99	0.01\\
85	0.01\\
85.01	0.01\\
85.02	0.01\\
85.03	0.01\\
85.04	0.01\\
85.05	0.01\\
85.06	0.01\\
85.07	0.01\\
85.08	0.01\\
85.09	0.01\\
85.1	0.01\\
85.11	0.01\\
85.12	0.01\\
85.13	0.01\\
85.14	0.01\\
85.15	0.01\\
85.16	0.01\\
85.17	0.01\\
85.18	0.01\\
85.19	0.01\\
85.2	0.01\\
85.21	0.01\\
85.22	0.01\\
85.23	0.01\\
85.24	0.01\\
85.25	0.01\\
85.26	0.01\\
85.27	0.01\\
85.28	0.01\\
85.29	0.01\\
85.3	0.01\\
85.31	0.01\\
85.32	0.01\\
85.33	0.01\\
85.34	0.01\\
85.35	0.01\\
85.36	0.01\\
85.37	0.01\\
85.38	0.01\\
85.39	0.01\\
85.4	0.01\\
85.41	0.01\\
85.42	0.01\\
85.43	0.01\\
85.44	0.01\\
85.45	0.01\\
85.46	0.01\\
85.47	0.01\\
85.48	0.01\\
85.49	0.01\\
85.5	0.01\\
85.51	0.01\\
85.52	0.01\\
85.53	0.01\\
85.54	0.01\\
85.55	0.01\\
85.56	0.01\\
85.57	0.01\\
85.58	0.01\\
85.59	0.01\\
85.6	0.01\\
85.61	0.01\\
85.62	0.01\\
85.63	0.01\\
85.64	0.01\\
85.65	0.01\\
85.66	0.01\\
85.67	0.01\\
85.68	0.01\\
85.69	0.01\\
85.7	0.01\\
85.71	0.01\\
85.72	0.01\\
85.73	0.01\\
85.74	0.01\\
85.75	0.01\\
85.76	0.01\\
85.77	0.01\\
85.78	0.01\\
85.79	0.01\\
85.8	0.01\\
85.81	0.01\\
85.82	0.01\\
85.83	0.01\\
85.84	0.01\\
85.85	0.01\\
85.86	0.01\\
85.87	0.01\\
85.88	0.01\\
85.89	0.01\\
85.9	0.01\\
85.91	0.01\\
85.92	0.01\\
85.93	0.01\\
85.94	0.01\\
85.95	0.01\\
85.96	0.01\\
85.97	0.01\\
85.98	0.01\\
85.99	0.01\\
86	0.01\\
86.01	0.01\\
86.02	0.01\\
86.03	0.01\\
86.04	0.01\\
86.05	0.01\\
86.06	0.01\\
86.07	0.01\\
86.08	0.01\\
86.09	0.01\\
86.1	0.01\\
86.11	0.01\\
86.12	0.01\\
86.13	0.01\\
86.14	0.01\\
86.15	0.01\\
86.16	0.01\\
86.17	0.01\\
86.18	0.01\\
86.19	0.01\\
86.2	0.01\\
86.21	0.01\\
86.22	0.01\\
86.23	0.01\\
86.24	0.01\\
86.25	0.01\\
86.26	0.01\\
86.27	0.01\\
86.28	0.01\\
86.29	0.01\\
86.3	0.01\\
86.31	0.01\\
86.32	0.01\\
86.33	0.01\\
86.34	0.01\\
86.35	0.01\\
86.36	0.01\\
86.37	0.01\\
86.38	0.01\\
86.39	0.01\\
86.4	0.01\\
86.41	0.01\\
86.42	0.01\\
86.43	0.01\\
86.44	0.01\\
86.45	0.01\\
86.46	0.01\\
86.47	0.01\\
86.48	0.01\\
86.49	0.01\\
86.5	0.01\\
86.51	0.01\\
86.52	0.01\\
86.53	0.01\\
86.54	0.01\\
86.55	0.01\\
86.56	0.01\\
86.57	0.01\\
86.58	0.01\\
86.59	0.01\\
86.6	0.01\\
86.61	0.01\\
86.62	0.01\\
86.63	0.01\\
86.64	0.01\\
86.65	0.01\\
86.66	0.01\\
86.67	0.01\\
86.68	0.01\\
86.69	0.01\\
86.7	0.01\\
86.71	0.01\\
86.72	0.01\\
86.73	0.01\\
86.74	0.01\\
86.75	0.01\\
86.76	0.01\\
86.77	0.01\\
86.78	0.01\\
86.79	0.01\\
86.8	0.01\\
86.81	0.01\\
86.82	0.01\\
86.83	0.01\\
86.84	0.01\\
86.85	0.01\\
86.86	0.01\\
86.87	0.01\\
86.88	0.01\\
86.89	0.01\\
86.9	0.01\\
86.91	0.01\\
86.92	0.01\\
86.93	0.01\\
86.94	0.01\\
86.95	0.01\\
86.96	0.01\\
86.97	0.01\\
86.98	0.01\\
86.99	0.01\\
87	0.01\\
87.01	0.01\\
87.02	0.01\\
87.03	0.01\\
87.04	0.01\\
87.05	0.01\\
87.06	0.01\\
87.07	0.01\\
87.08	0.01\\
87.09	0.01\\
87.1	0.01\\
87.11	0.01\\
87.12	0.01\\
87.13	0.01\\
87.14	0.01\\
87.15	0.01\\
87.16	0.01\\
87.17	0.01\\
87.18	0.01\\
87.19	0.01\\
87.2	0.01\\
87.21	0.01\\
87.22	0.01\\
87.23	0.01\\
87.24	0.01\\
87.25	0.01\\
87.26	0.01\\
87.27	0.01\\
87.28	0.01\\
87.29	0.01\\
87.3	0.01\\
87.31	0.01\\
87.32	0.01\\
87.33	0.01\\
87.34	0.01\\
87.35	0.01\\
87.36	0.01\\
87.37	0.01\\
87.38	0.01\\
87.39	0.01\\
87.4	0.01\\
87.41	0.01\\
87.42	0.01\\
87.43	0.01\\
87.44	0.01\\
87.45	0.01\\
87.46	0.01\\
87.47	0.01\\
87.48	0.01\\
87.49	0.01\\
87.5	0.01\\
87.51	0.01\\
87.52	0.01\\
87.53	0.01\\
87.54	0.01\\
87.55	0.01\\
87.56	0.01\\
87.57	0.01\\
87.58	0.01\\
87.59	0.01\\
87.6	0.01\\
87.61	0.01\\
87.62	0.01\\
87.63	0.01\\
87.64	0.01\\
87.65	0.01\\
87.66	0.01\\
87.67	0.01\\
87.68	0.01\\
87.69	0.01\\
87.7	0.01\\
87.71	0.01\\
87.72	0.01\\
87.73	0.01\\
87.74	0.01\\
87.75	0.01\\
87.76	0.01\\
87.77	0.01\\
87.78	0.01\\
87.79	0.01\\
87.8	0.01\\
87.81	0.01\\
87.82	0.01\\
87.83	0.01\\
87.84	0.01\\
87.85	0.01\\
87.86	0.01\\
87.87	0.01\\
87.88	0.01\\
87.89	0.01\\
87.9	0.01\\
87.91	0.01\\
87.92	0.01\\
87.93	0.01\\
87.94	0.01\\
87.95	0.01\\
87.96	0.01\\
87.97	0.01\\
87.98	0.01\\
87.99	0.01\\
88	0.01\\
88.01	0.01\\
88.02	0.01\\
88.03	0.01\\
88.04	0.01\\
88.05	0.01\\
88.06	0.01\\
88.07	0.01\\
88.08	0.01\\
88.09	0.01\\
88.1	0.01\\
88.11	0.01\\
88.12	0.01\\
88.13	0.01\\
88.14	0.01\\
88.15	0.01\\
88.16	0.01\\
88.17	0.01\\
88.18	0.01\\
88.19	0.01\\
88.2	0.01\\
88.21	0.01\\
88.22	0.01\\
88.23	0.01\\
88.24	0.01\\
88.25	0.01\\
88.26	0.01\\
88.27	0.01\\
88.28	0.01\\
88.29	0.01\\
88.3	0.01\\
88.31	0.01\\
88.32	0.01\\
88.33	0.01\\
88.34	0.01\\
88.35	0.01\\
88.36	0.01\\
88.37	0.01\\
88.38	0.01\\
88.39	0.01\\
88.4	0.01\\
88.41	0.01\\
88.42	0.01\\
88.43	0.01\\
88.44	0.01\\
88.45	0.01\\
88.46	0.01\\
88.47	0.01\\
88.48	0.01\\
88.49	0.01\\
88.5	0.01\\
88.51	0.01\\
88.52	0.01\\
88.53	0.01\\
88.54	0.01\\
88.55	0.01\\
88.56	0.01\\
88.57	0.01\\
88.58	0.01\\
88.59	0.01\\
88.6	0.01\\
88.61	0.01\\
88.62	0.01\\
88.63	0.01\\
88.64	0.01\\
88.65	0.01\\
88.66	0.01\\
88.67	0.01\\
88.68	0.01\\
88.69	0.01\\
88.7	0.01\\
88.71	0.01\\
88.72	0.01\\
88.73	0.01\\
88.74	0.01\\
88.75	0.01\\
88.76	0.01\\
88.77	0.01\\
88.78	0.01\\
88.79	0.01\\
88.8	0.01\\
88.81	0.01\\
88.82	0.01\\
88.83	0.01\\
88.84	0.01\\
88.85	0.01\\
88.86	0.01\\
88.87	0.01\\
88.88	0.01\\
88.89	0.01\\
88.9	0.01\\
88.91	0.01\\
88.92	0.01\\
88.93	0.01\\
88.94	0.01\\
88.95	0.01\\
88.96	0.01\\
88.97	0.01\\
88.98	0.01\\
88.99	0.01\\
89	0.01\\
89.01	0.01\\
89.02	0.01\\
89.03	0.01\\
89.04	0.01\\
89.05	0.01\\
89.06	0.01\\
89.07	0.01\\
89.08	0.01\\
89.09	0.01\\
89.1	0.01\\
89.11	0.01\\
89.12	0.01\\
89.13	0.01\\
89.14	0.01\\
89.15	0.01\\
89.16	0.01\\
89.17	0.01\\
89.18	0.01\\
89.19	0.01\\
89.2	0.01\\
89.21	0.01\\
89.22	0.01\\
89.23	0.01\\
89.24	0.01\\
89.25	0.01\\
89.26	0.01\\
89.27	0.01\\
89.28	0.01\\
89.29	0.01\\
89.3	0.01\\
89.31	0.01\\
89.32	0.01\\
89.33	0.01\\
89.34	0.01\\
89.35	0.01\\
89.36	0.01\\
89.37	0.01\\
89.38	0.01\\
89.39	0.01\\
89.4	0.01\\
89.41	0.01\\
89.42	0.01\\
89.43	0.01\\
89.44	0.01\\
89.45	0.01\\
89.46	0.01\\
89.47	0.01\\
89.48	0.01\\
89.49	0.01\\
89.5	0.01\\
89.51	0.01\\
89.52	0.01\\
89.53	0.01\\
89.54	0.01\\
89.55	0.01\\
89.56	0.01\\
89.57	0.01\\
89.58	0.01\\
89.59	0.01\\
89.6	0.01\\
89.61	0.01\\
89.62	0.01\\
89.63	0.01\\
89.64	0.01\\
89.65	0.01\\
89.66	0.01\\
89.67	0.01\\
89.68	0.01\\
89.69	0.01\\
89.7	0.01\\
89.71	0.01\\
89.72	0.01\\
89.73	0.01\\
89.74	0.01\\
89.75	0.01\\
89.76	0.01\\
89.77	0.01\\
89.78	0.01\\
89.79	0.01\\
89.8	0.01\\
89.81	0.01\\
89.82	0.01\\
89.83	0.01\\
89.84	0.01\\
89.85	0.01\\
89.86	0.01\\
89.87	0.01\\
89.88	0.01\\
89.89	0.01\\
89.9	0.01\\
89.91	0.01\\
89.92	0.01\\
89.93	0.01\\
89.94	0.01\\
89.95	0.01\\
89.96	0.01\\
89.97	0.01\\
89.98	0.01\\
89.99	0.01\\
90	0.01\\
90.01	0.01\\
90.02	0.01\\
90.03	0.01\\
90.04	0.01\\
90.05	0.01\\
90.06	0.01\\
90.07	0.01\\
90.08	0.01\\
90.09	0.01\\
90.1	0.01\\
90.11	0.01\\
90.12	0.01\\
90.13	0.01\\
90.14	0.01\\
90.15	0.01\\
90.16	0.01\\
90.17	0.01\\
90.18	0.01\\
90.19	0.01\\
90.2	0.01\\
90.21	0.01\\
90.22	0.01\\
90.23	0.01\\
90.24	0.01\\
90.25	0.01\\
90.26	0.01\\
90.27	0.01\\
90.28	0.01\\
90.29	0.01\\
90.3	0.01\\
90.31	0.01\\
90.32	0.01\\
90.33	0.01\\
90.34	0.01\\
90.35	0.01\\
90.36	0.01\\
90.37	0.01\\
90.38	0.01\\
90.39	0.01\\
90.4	0.01\\
90.41	0.01\\
90.42	0.01\\
90.43	0.01\\
90.44	0.01\\
90.45	0.01\\
90.46	0.01\\
90.47	0.01\\
90.48	0.01\\
90.49	0.01\\
90.5	0.01\\
90.51	0.01\\
90.52	0.01\\
90.53	0.01\\
90.54	0.01\\
90.55	0.01\\
90.56	0.01\\
90.57	0.01\\
90.58	0.01\\
90.59	0.01\\
90.6	0.01\\
90.61	0.01\\
90.62	0.01\\
90.63	0.01\\
90.64	0.01\\
90.65	0.01\\
90.66	0.01\\
90.67	0.01\\
90.68	0.01\\
90.69	0.01\\
90.7	0.01\\
90.71	0.01\\
90.72	0.01\\
90.73	0.01\\
90.74	0.01\\
90.75	0.01\\
90.76	0.01\\
90.77	0.01\\
90.78	0.01\\
90.79	0.01\\
90.8	0.01\\
90.81	0.01\\
90.82	0.01\\
90.83	0.01\\
90.84	0.01\\
90.85	0.01\\
90.86	0.01\\
90.87	0.01\\
90.88	0.01\\
90.89	0.01\\
90.9	0.01\\
90.91	0.01\\
90.92	0.01\\
90.93	0.01\\
90.94	0.01\\
90.95	0.01\\
90.96	0.01\\
90.97	0.01\\
90.98	0.01\\
90.99	0.01\\
91	0.01\\
91.01	0.01\\
91.02	0.01\\
91.03	0.01\\
91.04	0.01\\
91.05	0.01\\
91.06	0.01\\
91.07	0.01\\
91.08	0.01\\
91.09	0.01\\
91.1	0.01\\
91.11	0.01\\
91.12	0.01\\
91.13	0.01\\
91.14	0.01\\
91.15	0.01\\
91.16	0.01\\
91.17	0.01\\
91.18	0.01\\
91.19	0.01\\
91.2	0.01\\
91.21	0.01\\
91.22	0.01\\
91.23	0.01\\
91.24	0.01\\
91.25	0.01\\
91.26	0.01\\
91.27	0.01\\
91.28	0.01\\
91.29	0.01\\
91.3	0.01\\
91.31	0.01\\
91.32	0.01\\
91.33	0.01\\
91.34	0.01\\
91.35	0.01\\
91.36	0.01\\
91.37	0.01\\
91.38	0.01\\
91.39	0.01\\
91.4	0.01\\
91.41	0.01\\
91.42	0.01\\
91.43	0.01\\
91.44	0.01\\
91.45	0.01\\
91.46	0.01\\
91.47	0.01\\
91.48	0.01\\
91.49	0.01\\
91.5	0.01\\
91.51	0.01\\
91.52	0.01\\
91.53	0.01\\
91.54	0.01\\
91.55	0.01\\
91.56	0.01\\
91.57	0.01\\
91.58	0.01\\
91.59	0.01\\
91.6	0.01\\
91.61	0.01\\
91.62	0.01\\
91.63	0.01\\
91.64	0.01\\
91.65	0.01\\
91.66	0.01\\
91.67	0.01\\
91.68	0.01\\
91.69	0.01\\
91.7	0.01\\
91.71	0.01\\
91.72	0.01\\
91.73	0.01\\
91.74	0.01\\
91.75	0.01\\
91.76	0.01\\
91.77	0.01\\
91.78	0.01\\
91.79	0.01\\
91.8	0.01\\
91.81	0.01\\
91.82	0.01\\
91.83	0.01\\
91.84	0.01\\
91.85	0.01\\
91.86	0.01\\
91.87	0.01\\
91.88	0.01\\
91.89	0.01\\
91.9	0.01\\
91.91	0.01\\
91.92	0.01\\
91.93	0.01\\
91.94	0.01\\
91.95	0.01\\
91.96	0.01\\
91.97	0.01\\
91.98	0.01\\
91.99	0.01\\
92	0.01\\
92.01	0.01\\
92.02	0.01\\
92.03	0.01\\
92.04	0.01\\
92.05	0.01\\
92.06	0.01\\
92.07	0.01\\
92.08	0.01\\
92.09	0.01\\
92.1	0.01\\
92.11	0.01\\
92.12	0.01\\
92.13	0.01\\
92.14	0.01\\
92.15	0.01\\
92.16	0.01\\
92.17	0.01\\
92.18	0.01\\
92.19	0.01\\
92.2	0.01\\
92.21	0.01\\
92.22	0.01\\
92.23	0.01\\
92.24	0.01\\
92.25	0.01\\
92.26	0.01\\
92.27	0.01\\
92.28	0.01\\
92.29	0.01\\
92.3	0.01\\
92.31	0.01\\
92.32	0.01\\
92.33	0.01\\
92.34	0.01\\
92.35	0.01\\
92.36	0.01\\
92.37	0.01\\
92.38	0.01\\
92.39	0.01\\
92.4	0.01\\
92.41	0.01\\
92.42	0.01\\
92.43	0.01\\
92.44	0.01\\
92.45	0.01\\
92.46	0.01\\
92.47	0.01\\
92.48	0.01\\
92.49	0.01\\
92.5	0.01\\
92.51	0.01\\
92.52	0.01\\
92.53	0.01\\
92.54	0.01\\
92.55	0.01\\
92.56	0.01\\
92.57	0.01\\
92.58	0.01\\
92.59	0.01\\
92.6	0.01\\
92.61	0.01\\
92.62	0.01\\
92.63	0.01\\
92.64	0.01\\
92.65	0.01\\
92.66	0.01\\
92.67	0.01\\
92.68	0.01\\
92.69	0.01\\
92.7	0.01\\
92.71	0.01\\
92.72	0.01\\
92.73	0.01\\
92.74	0.01\\
92.75	0.01\\
92.76	0.01\\
92.77	0.01\\
92.78	0.01\\
92.79	0.01\\
92.8	0.01\\
92.81	0.01\\
92.82	0.01\\
92.83	0.01\\
92.84	0.01\\
92.85	0.01\\
92.86	0.01\\
92.87	0.01\\
92.88	0.01\\
92.89	0.01\\
92.9	0.01\\
92.91	0.01\\
92.92	0.01\\
92.93	0.01\\
92.94	0.01\\
92.95	0.01\\
92.96	0.01\\
92.97	0.01\\
92.98	0.01\\
92.99	0.01\\
93	0.01\\
93.01	0.01\\
93.02	0.01\\
93.03	0.01\\
93.04	0.01\\
93.05	0.01\\
93.06	0.01\\
93.07	0.01\\
93.08	0.01\\
93.09	0.01\\
93.1	0.01\\
93.11	0.01\\
93.12	0.01\\
93.13	0.01\\
93.14	0.01\\
93.15	0.01\\
93.16	0.01\\
93.17	0.01\\
93.18	0.01\\
93.19	0.01\\
93.2	0.01\\
93.21	0.01\\
93.22	0.01\\
93.23	0.01\\
93.24	0.01\\
93.25	0.01\\
93.26	0.01\\
93.27	0.01\\
93.28	0.01\\
93.29	0.01\\
93.3	0.01\\
93.31	0.01\\
93.32	0.01\\
93.33	0.01\\
93.34	0.01\\
93.35	0.01\\
93.36	0.01\\
93.37	0.01\\
93.38	0.01\\
93.39	0.01\\
93.4	0.01\\
93.41	0.01\\
93.42	0.01\\
93.43	0.01\\
93.44	0.01\\
93.45	0.01\\
93.46	0.01\\
93.47	0.01\\
93.48	0.01\\
93.49	0.01\\
93.5	0.01\\
93.51	0.01\\
93.52	0.01\\
93.53	0.01\\
93.54	0.01\\
93.55	0.01\\
93.56	0.01\\
93.57	0.01\\
93.58	0.01\\
93.59	0.01\\
93.6	0.01\\
93.61	0.01\\
93.62	0.01\\
93.63	0.01\\
93.64	0.01\\
93.65	0.01\\
93.66	0.01\\
93.67	0.01\\
93.68	0.01\\
93.69	0.01\\
93.7	0.01\\
93.71	0.01\\
93.72	0.01\\
93.73	0.01\\
93.74	0.01\\
93.75	0.01\\
93.76	0.01\\
93.77	0.01\\
93.78	0.01\\
93.79	0.01\\
93.8	0.01\\
93.81	0.01\\
93.82	0.01\\
93.83	0.01\\
93.84	0.01\\
93.85	0.01\\
93.86	0.01\\
93.87	0.01\\
93.88	0.01\\
93.89	0.01\\
93.9	0.01\\
93.91	0.01\\
93.92	0.01\\
93.93	0.01\\
93.94	0.01\\
93.95	0.01\\
93.96	0.01\\
93.97	0.01\\
93.98	0.01\\
93.99	0.01\\
94	0.01\\
94.01	0.01\\
94.02	0.01\\
94.03	0.01\\
94.04	0.01\\
94.05	0.01\\
94.06	0.01\\
94.07	0.01\\
94.08	0.01\\
94.09	0.01\\
94.1	0.01\\
94.11	0.01\\
94.12	0.01\\
94.13	0.01\\
94.14	0.01\\
94.15	0.01\\
94.16	0.01\\
94.17	0.01\\
94.18	0.01\\
94.19	0.01\\
94.2	0.01\\
94.21	0.01\\
94.22	0.01\\
94.23	0.01\\
94.24	0.01\\
94.25	0.01\\
94.26	0.01\\
94.27	0.01\\
94.28	0.01\\
94.29	0.01\\
94.3	0.01\\
94.31	0.01\\
94.32	0.01\\
94.33	0.01\\
94.34	0.01\\
94.35	0.01\\
94.36	0.01\\
94.37	0.01\\
94.38	0.01\\
94.39	0.01\\
94.4	0.01\\
94.41	0.01\\
94.42	0.01\\
94.43	0.01\\
94.44	0.01\\
94.45	0.01\\
94.46	0.01\\
94.47	0.01\\
94.48	0.01\\
94.49	0.01\\
94.5	0.01\\
94.51	0.01\\
94.52	0.01\\
94.53	0.01\\
94.54	0.01\\
94.55	0.01\\
94.56	0.01\\
94.57	0.01\\
94.58	0.01\\
94.59	0.01\\
94.6	0.01\\
94.61	0.01\\
94.62	0.01\\
94.63	0.01\\
94.64	0.01\\
94.65	0.01\\
94.66	0.01\\
94.67	0.01\\
94.68	0.01\\
94.69	0.01\\
94.7	0.01\\
94.71	0.01\\
94.72	0.01\\
94.73	0.01\\
94.74	0.01\\
94.75	0.01\\
94.76	0.01\\
94.77	0.01\\
94.78	0.01\\
94.79	0.01\\
94.8	0.01\\
94.81	0.01\\
94.82	0.01\\
94.83	0.01\\
94.84	0.01\\
94.85	0.01\\
94.86	0.01\\
94.87	0.01\\
94.88	0.01\\
94.89	0.01\\
94.9	0.01\\
94.91	0.01\\
94.92	0.01\\
94.93	0.01\\
94.94	0.01\\
94.95	0.01\\
94.96	0.01\\
94.97	0.01\\
94.98	0.01\\
94.99	0.01\\
95	0.01\\
95.01	0.01\\
95.02	0.01\\
95.03	0.01\\
95.04	0.01\\
95.05	0.01\\
95.06	0.01\\
95.07	0.01\\
95.08	0.01\\
95.09	0.01\\
95.1	0.01\\
95.11	0.01\\
95.12	0.01\\
95.13	0.01\\
95.14	0.01\\
95.15	0.01\\
95.16	0.01\\
95.17	0.01\\
95.18	0.01\\
95.19	0.01\\
95.2	0.01\\
95.21	0.01\\
95.22	0.01\\
95.23	0.01\\
95.24	0.01\\
95.25	0.01\\
95.26	0.01\\
95.27	0.01\\
95.28	0.01\\
95.29	0.01\\
95.3	0.01\\
95.31	0.01\\
95.32	0.01\\
95.33	0.01\\
95.34	0.01\\
95.35	0.01\\
95.36	0.01\\
95.37	0.01\\
95.38	0.01\\
95.39	0.01\\
95.4	0.01\\
95.41	0.01\\
95.42	0.01\\
95.43	0.01\\
95.44	0.01\\
95.45	0.01\\
95.46	0.01\\
95.47	0.01\\
95.48	0.01\\
95.49	0.01\\
95.5	0.01\\
95.51	0.01\\
95.52	0.01\\
95.53	0.01\\
95.54	0.01\\
95.55	0.01\\
95.56	0.01\\
95.57	0.01\\
95.58	0.01\\
95.59	0.01\\
95.6	0.01\\
95.61	0.01\\
95.62	0.01\\
95.63	0.01\\
95.64	0.01\\
95.65	0.01\\
95.66	0.01\\
95.67	0.01\\
95.68	0.01\\
95.69	0.01\\
95.7	0.01\\
95.71	0.01\\
95.72	0.01\\
95.73	0.01\\
95.74	0.01\\
95.75	0.01\\
95.76	0.01\\
95.77	0.01\\
95.78	0.01\\
95.79	0.01\\
95.8	0.01\\
95.81	0.01\\
95.82	0.01\\
95.83	0.01\\
95.84	0.01\\
95.85	0.01\\
95.86	0.01\\
95.87	0.01\\
95.88	0.01\\
95.89	0.01\\
95.9	0.01\\
95.91	0.01\\
95.92	0.01\\
95.93	0.01\\
95.94	0.01\\
95.95	0.01\\
95.96	0.01\\
95.97	0.01\\
95.98	0.01\\
95.99	0.01\\
96	0.01\\
96.01	0.01\\
96.02	0.01\\
96.03	0.01\\
96.04	0.01\\
96.05	0.01\\
96.06	0.01\\
96.07	0.01\\
96.08	0.01\\
96.09	0.01\\
96.1	0.01\\
96.11	0.01\\
96.12	0.01\\
96.13	0.01\\
96.14	0.01\\
96.15	0.01\\
96.16	0.01\\
96.17	0.01\\
96.18	0.01\\
96.19	0.01\\
96.2	0.01\\
96.21	0.01\\
96.22	0.01\\
96.23	0.01\\
96.24	0.01\\
96.25	0.01\\
96.26	0.01\\
96.27	0.01\\
96.28	0.01\\
96.29	0.01\\
96.3	0.01\\
96.31	0.01\\
96.32	0.01\\
96.33	0.01\\
96.34	0.01\\
96.35	0.01\\
96.36	0.01\\
96.37	0.01\\
96.38	0.01\\
96.39	0.01\\
96.4	0.01\\
96.41	0.01\\
96.42	0.01\\
96.43	0.01\\
96.44	0.01\\
96.45	0.01\\
96.46	0.01\\
96.47	0.01\\
96.48	0.01\\
96.49	0.01\\
96.5	0.01\\
96.51	0.01\\
96.52	0.01\\
96.53	0.01\\
96.54	0.01\\
96.55	0.01\\
96.56	0.01\\
96.57	0.01\\
96.58	0.01\\
96.59	0.01\\
96.6	0.01\\
96.61	0.01\\
96.62	0.01\\
96.63	0.01\\
96.64	0.01\\
96.65	0.01\\
96.66	0.01\\
96.67	0.01\\
96.68	0.01\\
96.69	0.01\\
96.7	0.01\\
96.71	0.01\\
96.72	0.01\\
96.73	0.01\\
96.74	0.01\\
96.75	0.01\\
96.76	0.01\\
96.77	0.01\\
96.78	0.01\\
96.79	0.01\\
96.8	0.01\\
96.81	0.01\\
96.82	0.01\\
96.83	0.01\\
96.84	0.01\\
96.85	0.01\\
96.86	0.01\\
96.87	0.01\\
96.88	0.01\\
96.89	0.01\\
96.9	0.01\\
96.91	0.01\\
96.92	0.01\\
96.93	0.01\\
96.94	0.01\\
96.95	0.01\\
96.96	0.01\\
96.97	0.01\\
96.98	0.01\\
96.99	0.01\\
97	0.01\\
97.01	0.01\\
97.02	0.01\\
97.03	0.01\\
97.04	0.01\\
97.05	0.01\\
97.06	0.01\\
97.07	0.01\\
97.08	0.01\\
97.09	0.01\\
97.1	0.01\\
97.11	0.01\\
97.12	0.01\\
97.13	0.01\\
97.14	0.01\\
97.15	0.01\\
97.16	0.01\\
97.17	0.01\\
97.18	0.01\\
97.19	0.01\\
97.2	0.01\\
97.21	0.01\\
97.22	0.01\\
97.23	0.01\\
97.24	0.01\\
97.25	0.01\\
97.26	0.01\\
97.27	0.01\\
97.28	0.01\\
97.29	0.01\\
97.3	0.01\\
97.31	0.01\\
97.32	0.01\\
97.33	0.01\\
97.34	0.01\\
97.35	0.01\\
97.36	0.01\\
97.37	0.01\\
97.38	0.01\\
97.39	0.01\\
97.4	0.01\\
97.41	0.01\\
97.42	0.01\\
97.43	0.01\\
97.44	0.01\\
97.45	0.01\\
97.46	0.01\\
97.47	0.01\\
97.48	0.01\\
97.49	0.01\\
97.5	0.01\\
97.51	0.01\\
97.52	0.01\\
97.53	0.01\\
97.54	0.01\\
97.55	0.01\\
97.56	0.01\\
97.57	0.01\\
97.58	0.01\\
97.59	0.01\\
97.6	0.01\\
97.61	0.01\\
97.62	0.01\\
97.63	0.01\\
97.64	0.01\\
97.65	0.01\\
97.66	0.01\\
97.67	0.01\\
97.68	0.01\\
97.69	0.01\\
97.7	0.01\\
97.71	0.01\\
97.72	0.01\\
97.73	0.01\\
97.74	0.01\\
97.75	0.01\\
97.76	0.01\\
97.77	0.01\\
97.78	0.01\\
97.79	0.01\\
97.8	0.01\\
97.81	0.01\\
97.82	0.01\\
97.83	0.01\\
97.84	0.01\\
97.85	0.01\\
97.86	0.01\\
97.87	0.01\\
97.88	0.01\\
97.89	0.01\\
97.9	0.01\\
97.91	0.01\\
97.92	0.01\\
97.93	0.01\\
97.94	0.01\\
97.95	0.01\\
97.96	0.01\\
97.97	0.01\\
97.98	0.01\\
97.99	0.01\\
98	0.01\\
98.01	0.01\\
98.02	0.01\\
98.03	0.01\\
98.04	0.01\\
98.05	0.01\\
98.06	0.01\\
98.07	0.01\\
98.08	0.01\\
98.09	0.01\\
98.1	0.01\\
98.11	0.01\\
98.12	0.01\\
98.13	0.01\\
98.14	0.01\\
98.15	0.01\\
98.16	0.01\\
98.17	0.01\\
98.18	0.01\\
98.19	0.01\\
98.2	0.01\\
98.21	0.01\\
98.22	0.01\\
98.23	0.01\\
98.24	0.01\\
98.25	0.01\\
98.26	0.01\\
98.27	0.01\\
98.28	0.01\\
98.29	0.01\\
98.3	0.01\\
98.31	0.01\\
98.32	0.01\\
98.33	0.01\\
98.34	0.01\\
98.35	0.01\\
98.36	0.01\\
98.37	0.01\\
98.38	0.01\\
98.39	0.01\\
98.4	0.01\\
98.41	0.01\\
98.42	0.01\\
98.43	0.01\\
98.44	0.01\\
98.45	0.01\\
98.46	0.01\\
98.47	0.01\\
98.48	0.01\\
98.49	0.01\\
98.5	0.01\\
98.51	0.01\\
98.52	0.01\\
98.53	0.01\\
98.54	0.01\\
98.55	0.01\\
98.56	0.01\\
98.57	0.01\\
98.58	0.01\\
98.59	0.01\\
98.6	0.01\\
98.61	0.01\\
98.62	0.01\\
98.63	0.01\\
98.64	0.01\\
98.65	0.01\\
98.66	0.01\\
98.67	0.01\\
98.68	0.01\\
98.69	0.01\\
98.7	0.01\\
98.71	0.01\\
98.72	0.01\\
98.73	0.01\\
98.74	0.01\\
98.75	0.01\\
98.76	0.01\\
98.77	0.01\\
98.78	0.01\\
98.79	0.01\\
98.8	0.01\\
98.81	0.01\\
98.82	0.01\\
98.83	0.01\\
98.84	0.01\\
98.85	0.01\\
98.86	0.01\\
98.87	0.01\\
98.88	0.01\\
98.89	0.01\\
98.9	0.01\\
98.91	0.01\\
98.92	0.01\\
98.93	0.01\\
98.94	0.01\\
98.95	0.01\\
98.96	0.01\\
98.97	0.01\\
98.98	0.01\\
98.99	0.01\\
99	0.01\\
99.01	0.01\\
99.02	0.01\\
99.03	0.01\\
99.04	0.01\\
99.05	0.01\\
99.06	0.01\\
99.07	0.01\\
99.08	0.01\\
99.09	0.01\\
99.1	0.01\\
99.11	0.01\\
99.12	0.01\\
99.13	0.01\\
99.14	0.01\\
99.15	0.01\\
99.16	0.01\\
99.17	0.01\\
99.18	0.01\\
99.19	0.01\\
99.2	0.01\\
99.21	0.01\\
99.22	0.01\\
99.23	0.01\\
99.24	0.01\\
99.25	0.01\\
99.26	0.01\\
99.27	0.01\\
99.28	0.01\\
99.29	0.01\\
99.3	0.01\\
99.31	0.01\\
99.32	0.01\\
99.33	0.01\\
99.34	0.01\\
99.35	0.01\\
99.36	0.01\\
99.37	0.01\\
99.38	0.01\\
99.39	0.01\\
99.4	0.01\\
99.41	0.01\\
99.42	0.01\\
99.43	0.01\\
99.44	0.01\\
99.45	0.01\\
99.46	0.01\\
99.47	0.01\\
99.48	0.01\\
99.49	0.01\\
99.5	0.01\\
99.51	0.01\\
99.52	0.01\\
99.53	0.01\\
99.54	0.01\\
99.55	0.01\\
99.56	0.01\\
99.57	0.01\\
99.58	0.01\\
99.59	0.01\\
99.6	0.01\\
99.61	0.01\\
99.62	0.01\\
99.63	0.01\\
99.64	0.01\\
99.65	0.01\\
99.66	0.01\\
99.67	0.01\\
99.68	0.01\\
99.69	0.01\\
99.7	0.01\\
99.71	0.01\\
99.72	0.01\\
99.73	0.01\\
99.74	0.01\\
99.75	0.01\\
99.76	0.01\\
99.77	0.01\\
99.78	0.01\\
99.79	0.01\\
99.8	0.01\\
99.81	0.01\\
99.82	0.01\\
99.83	0.01\\
99.84	0.01\\
99.85	0.01\\
99.86	0.01\\
99.87	0.01\\
99.88	0.01\\
99.89	0.01\\
99.9	0.01\\
99.91	0.01\\
99.92	0.01\\
99.93	0.01\\
99.94	0.01\\
99.95	0.01\\
99.96	0.01\\
99.97	0.01\\
99.98	0.01\\
99.99	0.01\\
100	0.01\\
};
\addlegendentry{$q=4$};

\end{axis}
\end{tikzpicture}%
  \caption{Continuous Time}
\end{subfigure}%
\hfill%
\begin{subfigure}{.45\linewidth}
  \centering
  \setlength\figureheight{\linewidth} 
  \setlength\figurewidth{\linewidth}
  \tikzsetnextfilename{testdp_dscr_nFPC_z15}
  % This file was created by matlab2tikz.
%
%The latest updates can be retrieved from
%  http://www.mathworks.com/matlabcentral/fileexchange/22022-matlab2tikz-matlab2tikz
%where you can also make suggestions and rate matlab2tikz.
%
\definecolor{mycolor1}{rgb}{0.00000,1.00000,0.14286}%
\definecolor{mycolor2}{rgb}{0.00000,1.00000,0.28571}%
\definecolor{mycolor3}{rgb}{0.00000,1.00000,0.42857}%
\definecolor{mycolor4}{rgb}{0.00000,1.00000,0.57143}%
\definecolor{mycolor5}{rgb}{0.00000,1.00000,0.71429}%
\definecolor{mycolor6}{rgb}{0.00000,1.00000,0.85714}%
\definecolor{mycolor7}{rgb}{0.00000,1.00000,1.00000}%
\definecolor{mycolor8}{rgb}{0.00000,0.87500,1.00000}%
\definecolor{mycolor9}{rgb}{0.00000,0.62500,1.00000}%
\definecolor{mycolor10}{rgb}{0.12500,0.00000,1.00000}%
\definecolor{mycolor11}{rgb}{0.25000,0.00000,1.00000}%
\definecolor{mycolor12}{rgb}{0.37500,0.00000,1.00000}%
\definecolor{mycolor13}{rgb}{0.50000,0.00000,1.00000}%
\definecolor{mycolor14}{rgb}{0.62500,0.00000,1.00000}%
\definecolor{mycolor15}{rgb}{0.75000,0.00000,1.00000}%
\definecolor{mycolor16}{rgb}{0.87500,0.00000,1.00000}%
\definecolor{mycolor17}{rgb}{1.00000,0.00000,1.00000}%
\definecolor{mycolor18}{rgb}{1.00000,0.00000,0.87500}%
\definecolor{mycolor19}{rgb}{1.00000,0.00000,0.62500}%
\definecolor{mycolor20}{rgb}{0.85714,0.00000,0.00000}%
\definecolor{mycolor21}{rgb}{0.71429,0.00000,0.00000}%
%
\begin{tikzpicture}[trim axis left, trim axis right]

\begin{axis}[%
width=\figurewidth,
height=\figureheight,
at={(0\figurewidth,0\figureheight)},
scale only axis,
every outer x axis line/.append style={black},
every x tick label/.append style={font=\color{black}},
xmin=0,
xmax=600,
every outer y axis line/.append style={black},
every y tick label/.append style={font=\color{black}},
ymin=0,
ymax=0.014,
axis background/.style={fill=white},
axis x line*=bottom,
axis y line*=left,
yticklabel style={
        /pgf/number format/fixed,
        /pgf/number format/precision=3
},
scaled y ticks=false
]
\addplot [color=green,solid,forget plot]
  table[row sep=crcr]{%
1	0\\
2	0\\
3	0\\
4	0\\
5	0\\
6	0\\
7	0\\
8	0\\
9	0\\
10	0\\
11	0\\
12	0\\
13	0\\
14	0\\
15	0\\
16	0\\
17	0\\
18	0\\
19	0\\
20	0\\
21	0\\
22	0\\
23	0\\
24	0\\
25	0\\
26	0\\
27	0\\
28	0\\
29	0\\
30	0\\
31	0\\
32	0\\
33	0\\
34	0\\
35	0\\
36	0\\
37	0\\
38	0\\
39	0\\
40	0\\
41	0\\
42	0\\
43	0\\
44	0\\
45	0\\
46	0\\
47	0\\
48	0\\
49	0\\
50	0\\
51	0\\
52	0\\
53	0\\
54	0\\
55	0\\
56	0\\
57	0\\
58	0\\
59	0\\
60	0\\
61	0\\
62	0\\
63	0\\
64	0\\
65	0\\
66	0\\
67	0\\
68	0\\
69	0\\
70	0\\
71	0\\
72	0\\
73	0\\
74	0\\
75	0\\
76	0\\
77	0\\
78	0\\
79	0\\
80	0\\
81	0\\
82	0\\
83	0\\
84	0\\
85	0\\
86	0\\
87	0\\
88	0\\
89	0\\
90	0\\
91	0\\
92	0\\
93	0\\
94	0\\
95	0\\
96	0\\
97	0\\
98	0\\
99	0\\
100	0\\
101	0\\
102	0\\
103	0\\
104	0\\
105	0\\
106	0\\
107	0\\
108	0\\
109	0\\
110	0\\
111	0\\
112	0\\
113	0\\
114	0\\
115	0\\
116	0\\
117	0\\
118	0\\
119	0\\
120	0\\
121	0\\
122	0\\
123	0\\
124	0\\
125	0\\
126	0\\
127	0\\
128	0\\
129	0\\
130	0\\
131	0\\
132	0\\
133	0\\
134	0\\
135	0\\
136	0\\
137	0\\
138	0\\
139	0\\
140	0\\
141	0\\
142	0\\
143	0\\
144	0\\
145	0\\
146	0\\
147	0\\
148	0\\
149	0\\
150	0\\
151	0\\
152	0\\
153	0\\
154	0\\
155	0\\
156	0\\
157	0\\
158	0\\
159	0\\
160	0\\
161	0\\
162	0\\
163	0\\
164	0\\
165	0\\
166	0\\
167	0\\
168	0\\
169	0\\
170	0\\
171	0\\
172	0\\
173	0\\
174	0\\
175	0\\
176	0\\
177	0\\
178	0\\
179	0\\
180	0\\
181	0\\
182	0\\
183	0\\
184	0\\
185	0\\
186	0\\
187	0\\
188	0\\
189	0\\
190	0\\
191	0\\
192	0\\
193	0\\
194	0\\
195	0\\
196	0\\
197	0\\
198	0\\
199	0\\
200	0\\
201	0\\
202	0\\
203	0\\
204	0\\
205	0\\
206	0\\
207	0\\
208	0\\
209	0\\
210	0\\
211	0\\
212	0\\
213	0\\
214	0\\
215	0\\
216	0\\
217	0\\
218	0\\
219	0\\
220	0\\
221	0\\
222	0\\
223	0\\
224	0\\
225	0\\
226	0\\
227	0\\
228	0\\
229	0\\
230	0\\
231	0\\
232	0\\
233	0\\
234	0\\
235	0\\
236	0\\
237	0\\
238	0\\
239	0\\
240	0\\
241	0\\
242	0\\
243	0\\
244	0\\
245	0\\
246	0\\
247	0\\
248	0\\
249	0\\
250	0\\
251	0\\
252	0\\
253	0\\
254	0\\
255	0\\
256	0\\
257	0\\
258	0\\
259	0\\
260	0\\
261	0\\
262	0\\
263	0\\
264	0\\
265	0\\
266	0\\
267	0\\
268	0\\
269	0\\
270	0\\
271	0\\
272	0\\
273	0\\
274	0\\
275	0\\
276	0\\
277	0\\
278	0\\
279	0\\
280	0\\
281	0\\
282	0\\
283	0\\
284	0\\
285	0\\
286	0\\
287	0\\
288	0\\
289	0\\
290	0\\
291	0\\
292	0\\
293	0\\
294	0\\
295	0\\
296	0\\
297	0\\
298	0\\
299	0\\
300	0\\
301	0\\
302	0\\
303	0\\
304	0\\
305	0\\
306	0\\
307	0\\
308	0\\
309	0\\
310	0\\
311	0\\
312	0\\
313	0\\
314	0\\
315	0\\
316	0\\
317	0\\
318	0\\
319	0\\
320	0\\
321	0\\
322	0\\
323	0\\
324	0\\
325	0\\
326	0\\
327	0\\
328	0\\
329	0\\
330	0\\
331	0\\
332	0\\
333	0\\
334	0\\
335	0\\
336	0\\
337	0\\
338	0\\
339	0\\
340	0\\
341	0\\
342	0\\
343	0\\
344	0\\
345	0\\
346	0\\
347	0\\
348	0\\
349	0\\
350	0\\
351	0\\
352	0\\
353	0\\
354	0\\
355	0\\
356	0\\
357	0\\
358	0\\
359	0\\
360	0\\
361	0\\
362	0\\
363	0\\
364	0\\
365	0\\
366	0\\
367	0\\
368	0\\
369	0\\
370	0\\
371	0\\
372	0\\
373	0\\
374	0\\
375	0\\
376	0\\
377	0\\
378	0\\
379	0\\
380	0\\
381	0\\
382	0\\
383	0\\
384	0\\
385	0\\
386	0\\
387	0\\
388	0\\
389	0\\
390	0\\
391	0\\
392	0\\
393	0\\
394	0\\
395	0\\
396	0\\
397	0\\
398	0\\
399	0\\
400	0\\
401	0\\
402	0\\
403	0\\
404	0\\
405	0\\
406	0\\
407	0\\
408	0\\
409	0\\
410	0\\
411	0\\
412	0\\
413	0\\
414	0\\
415	0\\
416	0\\
417	0\\
418	0\\
419	0\\
420	0\\
421	0\\
422	0\\
423	0\\
424	0\\
425	0\\
426	0\\
427	0\\
428	0\\
429	0\\
430	0\\
431	0\\
432	0\\
433	0\\
434	0\\
435	0\\
436	0\\
437	0\\
438	0\\
439	0\\
440	0\\
441	0\\
442	0\\
443	0\\
444	0\\
445	0\\
446	0\\
447	0\\
448	0\\
449	0\\
450	0\\
451	0\\
452	0\\
453	0\\
454	0\\
455	0\\
456	0\\
457	0\\
458	0\\
459	0\\
460	0\\
461	0\\
462	0\\
463	0\\
464	0\\
465	0\\
466	0\\
467	0\\
468	0\\
469	0\\
470	0\\
471	0\\
472	0\\
473	0\\
474	0\\
475	0\\
476	0\\
477	0\\
478	0\\
479	0\\
480	0\\
481	0\\
482	0\\
483	0\\
484	0\\
485	0\\
486	0\\
487	0\\
488	0\\
489	0\\
490	0\\
491	0\\
492	0\\
493	0\\
494	0\\
495	0\\
496	0\\
497	0\\
498	0\\
499	0\\
500	0\\
501	0\\
502	0\\
503	0\\
504	0\\
505	0\\
506	0\\
507	0\\
508	0\\
509	0\\
510	0\\
511	0\\
512	0\\
513	0\\
514	0\\
515	0\\
516	0\\
517	0\\
518	0\\
519	0\\
520	0\\
521	0\\
522	0\\
523	0\\
524	0\\
525	0\\
526	0\\
527	0\\
528	0\\
529	0\\
530	0\\
531	0\\
532	0\\
533	0\\
534	0\\
535	0\\
536	0\\
537	0\\
538	0\\
539	0\\
540	0\\
541	0\\
542	0\\
543	0\\
544	0\\
545	0\\
546	0\\
547	0\\
548	0\\
549	0\\
550	0\\
551	0\\
552	0\\
553	0\\
554	0\\
555	0\\
556	0\\
557	0\\
558	0\\
559	0\\
560	0\\
561	0\\
562	0\\
563	0\\
564	0\\
565	0\\
566	0\\
567	0\\
568	0\\
569	0\\
570	0\\
571	0\\
572	0\\
573	0\\
574	0\\
575	0\\
576	0\\
577	0\\
578	0\\
579	0\\
580	0\\
581	0\\
582	0\\
583	0\\
584	0\\
585	0\\
586	0\\
587	0\\
588	0\\
589	0\\
590	0\\
591	0\\
592	0\\
593	0\\
594	0\\
595	0\\
596	0\\
597	0\\
598	0\\
599	0\\
600	0\\
};
\addplot [color=mycolor1,solid,forget plot]
  table[row sep=crcr]{%
1	0\\
2	0\\
3	0\\
4	0\\
5	0\\
6	0\\
7	0\\
8	0\\
9	0\\
10	0\\
11	0\\
12	0\\
13	0\\
14	0\\
15	0\\
16	0\\
17	0\\
18	0\\
19	0\\
20	0\\
21	0\\
22	0\\
23	0\\
24	0\\
25	0\\
26	0\\
27	0\\
28	0\\
29	0\\
30	0\\
31	0\\
32	0\\
33	0\\
34	0\\
35	0\\
36	0\\
37	0\\
38	0\\
39	0\\
40	0\\
41	0\\
42	0\\
43	0\\
44	0\\
45	0\\
46	0\\
47	0\\
48	0\\
49	0\\
50	0\\
51	0\\
52	0\\
53	0\\
54	0\\
55	0\\
56	0\\
57	0\\
58	0\\
59	0\\
60	0\\
61	0\\
62	0\\
63	0\\
64	0\\
65	0\\
66	0\\
67	0\\
68	0\\
69	0\\
70	0\\
71	0\\
72	0\\
73	0\\
74	0\\
75	0\\
76	0\\
77	0\\
78	0\\
79	0\\
80	0\\
81	0\\
82	0\\
83	0\\
84	0\\
85	0\\
86	0\\
87	0\\
88	0\\
89	0\\
90	0\\
91	0\\
92	0\\
93	0\\
94	0\\
95	0\\
96	0\\
97	0\\
98	0\\
99	0\\
100	0\\
101	0\\
102	0\\
103	0\\
104	0\\
105	0\\
106	0\\
107	0\\
108	0\\
109	0\\
110	0\\
111	0\\
112	0\\
113	0\\
114	0\\
115	0\\
116	0\\
117	0\\
118	0\\
119	0\\
120	0\\
121	0\\
122	0\\
123	0\\
124	0\\
125	0\\
126	0\\
127	0\\
128	0\\
129	0\\
130	0\\
131	0\\
132	0\\
133	0\\
134	0\\
135	0\\
136	0\\
137	0\\
138	0\\
139	0\\
140	0\\
141	0\\
142	0\\
143	0\\
144	0\\
145	0\\
146	0\\
147	0\\
148	0\\
149	0\\
150	0\\
151	0\\
152	0\\
153	0\\
154	0\\
155	0\\
156	0\\
157	0\\
158	0\\
159	0\\
160	0\\
161	0\\
162	0\\
163	0\\
164	0\\
165	0\\
166	0\\
167	0\\
168	0\\
169	0\\
170	0\\
171	0\\
172	0\\
173	0\\
174	0\\
175	0\\
176	0\\
177	0\\
178	0\\
179	0\\
180	0\\
181	0\\
182	0\\
183	0\\
184	0\\
185	0\\
186	0\\
187	0\\
188	0\\
189	0\\
190	0\\
191	0\\
192	0\\
193	0\\
194	0\\
195	0\\
196	0\\
197	0\\
198	0\\
199	0\\
200	0\\
201	0\\
202	0\\
203	0\\
204	0\\
205	0\\
206	0\\
207	0\\
208	0\\
209	0\\
210	0\\
211	0\\
212	0\\
213	0\\
214	0\\
215	0\\
216	0\\
217	0\\
218	0\\
219	0\\
220	0\\
221	0\\
222	0\\
223	0\\
224	0\\
225	0\\
226	0\\
227	0\\
228	0\\
229	0\\
230	0\\
231	0\\
232	0\\
233	0\\
234	0\\
235	0\\
236	0\\
237	0\\
238	0\\
239	0\\
240	0\\
241	0\\
242	0\\
243	0\\
244	0\\
245	0\\
246	0\\
247	0\\
248	0\\
249	0\\
250	0\\
251	0\\
252	0\\
253	0\\
254	0\\
255	0\\
256	0\\
257	0\\
258	0\\
259	0\\
260	0\\
261	0\\
262	0\\
263	0\\
264	0\\
265	0\\
266	0\\
267	0\\
268	0\\
269	0\\
270	0\\
271	0\\
272	0\\
273	0\\
274	0\\
275	0\\
276	0\\
277	0\\
278	0\\
279	0\\
280	0\\
281	0\\
282	0\\
283	0\\
284	0\\
285	0\\
286	0\\
287	0\\
288	0\\
289	0\\
290	0\\
291	0\\
292	0\\
293	0\\
294	0\\
295	0\\
296	0\\
297	0\\
298	0\\
299	0\\
300	0\\
301	0\\
302	0\\
303	0\\
304	0\\
305	0\\
306	0\\
307	0\\
308	0\\
309	0\\
310	0\\
311	0\\
312	0\\
313	0\\
314	0\\
315	0\\
316	0\\
317	0\\
318	0\\
319	0\\
320	0\\
321	0\\
322	0\\
323	0\\
324	0\\
325	0\\
326	0\\
327	0\\
328	0\\
329	0\\
330	0\\
331	0\\
332	0\\
333	0\\
334	0\\
335	0\\
336	0\\
337	0\\
338	0\\
339	0\\
340	0\\
341	0\\
342	0\\
343	0\\
344	0\\
345	0\\
346	0\\
347	0\\
348	0\\
349	0\\
350	0\\
351	0\\
352	0\\
353	0\\
354	0\\
355	0\\
356	0\\
357	0\\
358	0\\
359	0\\
360	0\\
361	0\\
362	0\\
363	0\\
364	0\\
365	0\\
366	0\\
367	0\\
368	0\\
369	0\\
370	0\\
371	0\\
372	0\\
373	0\\
374	0\\
375	0\\
376	0\\
377	0\\
378	0\\
379	0\\
380	0\\
381	0\\
382	0\\
383	0\\
384	0\\
385	0\\
386	0\\
387	0\\
388	0\\
389	0\\
390	0\\
391	0\\
392	0\\
393	0\\
394	0\\
395	0\\
396	0\\
397	0\\
398	0\\
399	0\\
400	0\\
401	0\\
402	0\\
403	0\\
404	0\\
405	0\\
406	0\\
407	0\\
408	0\\
409	0\\
410	0\\
411	0\\
412	0\\
413	0\\
414	0\\
415	0\\
416	0\\
417	0\\
418	0\\
419	0\\
420	0\\
421	0\\
422	0\\
423	0\\
424	0\\
425	0\\
426	0\\
427	0\\
428	0\\
429	0\\
430	0\\
431	0\\
432	0\\
433	0\\
434	0\\
435	0\\
436	0\\
437	0\\
438	0\\
439	0\\
440	0\\
441	0\\
442	0\\
443	0\\
444	0\\
445	0\\
446	0\\
447	0\\
448	0\\
449	0\\
450	0\\
451	0\\
452	0\\
453	0\\
454	0\\
455	0\\
456	0\\
457	0\\
458	0\\
459	0\\
460	0\\
461	0\\
462	0\\
463	0\\
464	0\\
465	0\\
466	0\\
467	0\\
468	0\\
469	0\\
470	0\\
471	0\\
472	0\\
473	0\\
474	0\\
475	0\\
476	0\\
477	0\\
478	0\\
479	0\\
480	0\\
481	0\\
482	0\\
483	0\\
484	0\\
485	0\\
486	0\\
487	0\\
488	0\\
489	0\\
490	0\\
491	0\\
492	0\\
493	0\\
494	0\\
495	0\\
496	0\\
497	0\\
498	0\\
499	0\\
500	0\\
501	0\\
502	0\\
503	0\\
504	0\\
505	0\\
506	0\\
507	0\\
508	0\\
509	0\\
510	0\\
511	0\\
512	0\\
513	0\\
514	0\\
515	0\\
516	0\\
517	0\\
518	0\\
519	0\\
520	0\\
521	0\\
522	0\\
523	0\\
524	0\\
525	0\\
526	0\\
527	0\\
528	0\\
529	0\\
530	0\\
531	0\\
532	0\\
533	0\\
534	0\\
535	0\\
536	0\\
537	0\\
538	0\\
539	0\\
540	0\\
541	0\\
542	0\\
543	0\\
544	0\\
545	0\\
546	0\\
547	0\\
548	0\\
549	0\\
550	0\\
551	0\\
552	0\\
553	0\\
554	0\\
555	0\\
556	0\\
557	0\\
558	0\\
559	0\\
560	0\\
561	0\\
562	0\\
563	0\\
564	0\\
565	0\\
566	0\\
567	0\\
568	0\\
569	0\\
570	0\\
571	0\\
572	0\\
573	0\\
574	0\\
575	0\\
576	0\\
577	0\\
578	0\\
579	0\\
580	0\\
581	0\\
582	0\\
583	0\\
584	0\\
585	0\\
586	0\\
587	0\\
588	0\\
589	0\\
590	0\\
591	0\\
592	0\\
593	0\\
594	0\\
595	0\\
596	0\\
597	0\\
598	0\\
599	0\\
600	0\\
};
\addplot [color=mycolor2,solid,forget plot]
  table[row sep=crcr]{%
1	0\\
2	0\\
3	0\\
4	0\\
5	0\\
6	0\\
7	0\\
8	0\\
9	0\\
10	0\\
11	0\\
12	0\\
13	0\\
14	0\\
15	0\\
16	0\\
17	0\\
18	0\\
19	0\\
20	0\\
21	0\\
22	0\\
23	0\\
24	0\\
25	0\\
26	0\\
27	0\\
28	0\\
29	0\\
30	0\\
31	0\\
32	0\\
33	0\\
34	0\\
35	0\\
36	0\\
37	0\\
38	0\\
39	0\\
40	0\\
41	0\\
42	0\\
43	0\\
44	0\\
45	0\\
46	0\\
47	0\\
48	0\\
49	0\\
50	0\\
51	0\\
52	0\\
53	0\\
54	0\\
55	0\\
56	0\\
57	0\\
58	0\\
59	0\\
60	0\\
61	0\\
62	0\\
63	0\\
64	0\\
65	0\\
66	0\\
67	0\\
68	0\\
69	0\\
70	0\\
71	0\\
72	0\\
73	0\\
74	0\\
75	0\\
76	0\\
77	0\\
78	0\\
79	0\\
80	0\\
81	0\\
82	0\\
83	0\\
84	0\\
85	0\\
86	0\\
87	0\\
88	0\\
89	0\\
90	0\\
91	0\\
92	0\\
93	0\\
94	0\\
95	0\\
96	0\\
97	0\\
98	0\\
99	0\\
100	0\\
101	0\\
102	0\\
103	0\\
104	0\\
105	0\\
106	0\\
107	0\\
108	0\\
109	0\\
110	0\\
111	0\\
112	0\\
113	0\\
114	0\\
115	0\\
116	0\\
117	0\\
118	0\\
119	0\\
120	0\\
121	0\\
122	0\\
123	0\\
124	0\\
125	0\\
126	0\\
127	0\\
128	0\\
129	0\\
130	0\\
131	0\\
132	0\\
133	0\\
134	0\\
135	0\\
136	0\\
137	0\\
138	0\\
139	0\\
140	0\\
141	0\\
142	0\\
143	0\\
144	0\\
145	0\\
146	0\\
147	0\\
148	0\\
149	0\\
150	0\\
151	0\\
152	0\\
153	0\\
154	0\\
155	0\\
156	0\\
157	0\\
158	0\\
159	0\\
160	0\\
161	0\\
162	0\\
163	0\\
164	0\\
165	0\\
166	0\\
167	0\\
168	0\\
169	0\\
170	0\\
171	0\\
172	0\\
173	0\\
174	0\\
175	0\\
176	0\\
177	0\\
178	0\\
179	0\\
180	0\\
181	0\\
182	0\\
183	0\\
184	0\\
185	0\\
186	0\\
187	0\\
188	0\\
189	0\\
190	0\\
191	0\\
192	0\\
193	0\\
194	0\\
195	0\\
196	0\\
197	0\\
198	0\\
199	0\\
200	0\\
201	0\\
202	0\\
203	0\\
204	0\\
205	0\\
206	0\\
207	0\\
208	0\\
209	0\\
210	0\\
211	0\\
212	0\\
213	0\\
214	0\\
215	0\\
216	0\\
217	0\\
218	0\\
219	0\\
220	0\\
221	0\\
222	0\\
223	0\\
224	0\\
225	0\\
226	0\\
227	0\\
228	0\\
229	0\\
230	0\\
231	0\\
232	0\\
233	0\\
234	0\\
235	0\\
236	0\\
237	0\\
238	0\\
239	0\\
240	0\\
241	0\\
242	0\\
243	0\\
244	0\\
245	0\\
246	0\\
247	0\\
248	0\\
249	0\\
250	0\\
251	0\\
252	0\\
253	0\\
254	0\\
255	0\\
256	0\\
257	0\\
258	0\\
259	0\\
260	0\\
261	0\\
262	0\\
263	0\\
264	0\\
265	0\\
266	0\\
267	0\\
268	0\\
269	0\\
270	0\\
271	0\\
272	0\\
273	0\\
274	0\\
275	0\\
276	0\\
277	0\\
278	0\\
279	0\\
280	0\\
281	0\\
282	0\\
283	0\\
284	0\\
285	0\\
286	0\\
287	0\\
288	0\\
289	0\\
290	0\\
291	0\\
292	0\\
293	0\\
294	0\\
295	0\\
296	0\\
297	0\\
298	0\\
299	0\\
300	0\\
301	0\\
302	0\\
303	0\\
304	0\\
305	0\\
306	0\\
307	0\\
308	0\\
309	0\\
310	0\\
311	0\\
312	0\\
313	0\\
314	0\\
315	0\\
316	0\\
317	0\\
318	0\\
319	0\\
320	0\\
321	0\\
322	0\\
323	0\\
324	0\\
325	0\\
326	0\\
327	0\\
328	0\\
329	0\\
330	0\\
331	0\\
332	0\\
333	0\\
334	0\\
335	0\\
336	0\\
337	0\\
338	0\\
339	0\\
340	0\\
341	0\\
342	0\\
343	0\\
344	0\\
345	0\\
346	0\\
347	0\\
348	0\\
349	0\\
350	0\\
351	0\\
352	0\\
353	0\\
354	0\\
355	0\\
356	0\\
357	0\\
358	0\\
359	0\\
360	0\\
361	0\\
362	0\\
363	0\\
364	0\\
365	0\\
366	0\\
367	0\\
368	0\\
369	0\\
370	0\\
371	0\\
372	0\\
373	0\\
374	0\\
375	0\\
376	0\\
377	0\\
378	0\\
379	0\\
380	0\\
381	0\\
382	0\\
383	0\\
384	0\\
385	0\\
386	0\\
387	0\\
388	0\\
389	0\\
390	0\\
391	0\\
392	0\\
393	0\\
394	0\\
395	0\\
396	0\\
397	0\\
398	0\\
399	0\\
400	0\\
401	0\\
402	0\\
403	0\\
404	0\\
405	0\\
406	0\\
407	0\\
408	0\\
409	0\\
410	0\\
411	0\\
412	0\\
413	0\\
414	0\\
415	0\\
416	0\\
417	0\\
418	0\\
419	0\\
420	0\\
421	0\\
422	0\\
423	0\\
424	0\\
425	0\\
426	0\\
427	0\\
428	0\\
429	0\\
430	0\\
431	0\\
432	0\\
433	0\\
434	0\\
435	0\\
436	0\\
437	0\\
438	0\\
439	0\\
440	0\\
441	0\\
442	0\\
443	0\\
444	0\\
445	0\\
446	0\\
447	0\\
448	0\\
449	0\\
450	0\\
451	0\\
452	0\\
453	0\\
454	0\\
455	0\\
456	0\\
457	0\\
458	0\\
459	0\\
460	0\\
461	0\\
462	0\\
463	0\\
464	0\\
465	0\\
466	0\\
467	0\\
468	0\\
469	0\\
470	0\\
471	0\\
472	0\\
473	0\\
474	0\\
475	0\\
476	0\\
477	0\\
478	0\\
479	0\\
480	0\\
481	0\\
482	0\\
483	0\\
484	0\\
485	0\\
486	0\\
487	0\\
488	0\\
489	0\\
490	0\\
491	0\\
492	0\\
493	0\\
494	0\\
495	0\\
496	0\\
497	0\\
498	0\\
499	0\\
500	0\\
501	0\\
502	0\\
503	0\\
504	0\\
505	0\\
506	0\\
507	0\\
508	0\\
509	0\\
510	0\\
511	0\\
512	0\\
513	0\\
514	0\\
515	0\\
516	0\\
517	0\\
518	0\\
519	0\\
520	0\\
521	0\\
522	0\\
523	0\\
524	0\\
525	0\\
526	0\\
527	0\\
528	0\\
529	0\\
530	0\\
531	0\\
532	0\\
533	0\\
534	0\\
535	0\\
536	0\\
537	0\\
538	0\\
539	0\\
540	0\\
541	0\\
542	0\\
543	0\\
544	0\\
545	0\\
546	0\\
547	0\\
548	0\\
549	0\\
550	0\\
551	0\\
552	0\\
553	0\\
554	0\\
555	0\\
556	0\\
557	0\\
558	0\\
559	0\\
560	0\\
561	0\\
562	0\\
563	0\\
564	0\\
565	0\\
566	0\\
567	0\\
568	0\\
569	0\\
570	0\\
571	0\\
572	0\\
573	0\\
574	0\\
575	0\\
576	0\\
577	0\\
578	0\\
579	0\\
580	0\\
581	0\\
582	0\\
583	0\\
584	0\\
585	0\\
586	0\\
587	0\\
588	0\\
589	0\\
590	0\\
591	0\\
592	0\\
593	0\\
594	0\\
595	0\\
596	0\\
597	0\\
598	0\\
599	0\\
600	0\\
};
\addplot [color=mycolor3,solid,forget plot]
  table[row sep=crcr]{%
1	0\\
2	0\\
3	0\\
4	0\\
5	0\\
6	0\\
7	0\\
8	0\\
9	0\\
10	0\\
11	0\\
12	0\\
13	0\\
14	0\\
15	0\\
16	0\\
17	0\\
18	0\\
19	0\\
20	0\\
21	0\\
22	0\\
23	0\\
24	0\\
25	0\\
26	0\\
27	0\\
28	0\\
29	0\\
30	0\\
31	0\\
32	0\\
33	0\\
34	0\\
35	0\\
36	0\\
37	0\\
38	0\\
39	0\\
40	0\\
41	0\\
42	0\\
43	0\\
44	0\\
45	0\\
46	0\\
47	0\\
48	0\\
49	0\\
50	0\\
51	0\\
52	0\\
53	0\\
54	0\\
55	0\\
56	0\\
57	0\\
58	0\\
59	0\\
60	0\\
61	0\\
62	0\\
63	0\\
64	0\\
65	0\\
66	0\\
67	0\\
68	0\\
69	0\\
70	0\\
71	0\\
72	0\\
73	0\\
74	0\\
75	0\\
76	0\\
77	0\\
78	0\\
79	0\\
80	0\\
81	0\\
82	0\\
83	0\\
84	0\\
85	0\\
86	0\\
87	0\\
88	0\\
89	0\\
90	0\\
91	0\\
92	0\\
93	0\\
94	0\\
95	0\\
96	0\\
97	0\\
98	0\\
99	0\\
100	0\\
101	0\\
102	0\\
103	0\\
104	0\\
105	0\\
106	0\\
107	0\\
108	0\\
109	0\\
110	0\\
111	0\\
112	0\\
113	0\\
114	0\\
115	0\\
116	0\\
117	0\\
118	0\\
119	0\\
120	0\\
121	0\\
122	0\\
123	0\\
124	0\\
125	0\\
126	0\\
127	0\\
128	0\\
129	0\\
130	0\\
131	0\\
132	0\\
133	0\\
134	0\\
135	0\\
136	0\\
137	0\\
138	0\\
139	0\\
140	0\\
141	0\\
142	0\\
143	0\\
144	0\\
145	0\\
146	0\\
147	0\\
148	0\\
149	0\\
150	0\\
151	0\\
152	0\\
153	0\\
154	0\\
155	0\\
156	0\\
157	0\\
158	0\\
159	0\\
160	0\\
161	0\\
162	0\\
163	0\\
164	0\\
165	0\\
166	0\\
167	0\\
168	0\\
169	0\\
170	0\\
171	0\\
172	0\\
173	0\\
174	0\\
175	0\\
176	0\\
177	0\\
178	0\\
179	0\\
180	0\\
181	0\\
182	0\\
183	0\\
184	0\\
185	0\\
186	0\\
187	0\\
188	0\\
189	0\\
190	0\\
191	0\\
192	0\\
193	0\\
194	0\\
195	0\\
196	0\\
197	0\\
198	0\\
199	0\\
200	0\\
201	0\\
202	0\\
203	0\\
204	0\\
205	0\\
206	0\\
207	0\\
208	0\\
209	0\\
210	0\\
211	0\\
212	0\\
213	0\\
214	0\\
215	0\\
216	0\\
217	0\\
218	0\\
219	0\\
220	0\\
221	0\\
222	0\\
223	0\\
224	0\\
225	0\\
226	0\\
227	0\\
228	0\\
229	0\\
230	0\\
231	0\\
232	0\\
233	0\\
234	0\\
235	0\\
236	0\\
237	0\\
238	0\\
239	0\\
240	0\\
241	0\\
242	0\\
243	0\\
244	0\\
245	0\\
246	0\\
247	0\\
248	0\\
249	0\\
250	0\\
251	0\\
252	0\\
253	0\\
254	0\\
255	0\\
256	0\\
257	0\\
258	0\\
259	0\\
260	0\\
261	0\\
262	0\\
263	0\\
264	0\\
265	0\\
266	0\\
267	0\\
268	0\\
269	0\\
270	0\\
271	0\\
272	0\\
273	0\\
274	0\\
275	0\\
276	0\\
277	0\\
278	0\\
279	0\\
280	0\\
281	0\\
282	0\\
283	0\\
284	0\\
285	0\\
286	0\\
287	0\\
288	0\\
289	0\\
290	0\\
291	0\\
292	0\\
293	0\\
294	0\\
295	0\\
296	0\\
297	0\\
298	0\\
299	0\\
300	0\\
301	0\\
302	0\\
303	0\\
304	0\\
305	0\\
306	0\\
307	0\\
308	0\\
309	0\\
310	0\\
311	0\\
312	0\\
313	0\\
314	0\\
315	0\\
316	0\\
317	0\\
318	0\\
319	0\\
320	0\\
321	0\\
322	0\\
323	0\\
324	0\\
325	0\\
326	0\\
327	0\\
328	0\\
329	0\\
330	0\\
331	0\\
332	0\\
333	0\\
334	0\\
335	0\\
336	0\\
337	0\\
338	0\\
339	0\\
340	0\\
341	0\\
342	0\\
343	0\\
344	0\\
345	0\\
346	0\\
347	0\\
348	0\\
349	0\\
350	0\\
351	0\\
352	0\\
353	0\\
354	0\\
355	0\\
356	0\\
357	0\\
358	0\\
359	0\\
360	0\\
361	0\\
362	0\\
363	0\\
364	0\\
365	0\\
366	0\\
367	0\\
368	0\\
369	0\\
370	0\\
371	0\\
372	0\\
373	0\\
374	0\\
375	0\\
376	0\\
377	0\\
378	0\\
379	0\\
380	0\\
381	0\\
382	0\\
383	0\\
384	0\\
385	0\\
386	0\\
387	0\\
388	0\\
389	0\\
390	0\\
391	0\\
392	0\\
393	0\\
394	0\\
395	0\\
396	0\\
397	0\\
398	0\\
399	0\\
400	0\\
401	0\\
402	0\\
403	0\\
404	0\\
405	0\\
406	0\\
407	0\\
408	0\\
409	0\\
410	0\\
411	0\\
412	0\\
413	0\\
414	0\\
415	0\\
416	0\\
417	0\\
418	0\\
419	0\\
420	0\\
421	0\\
422	0\\
423	0\\
424	0\\
425	0\\
426	0\\
427	0\\
428	0\\
429	0\\
430	0\\
431	0\\
432	0\\
433	0\\
434	0\\
435	0\\
436	0\\
437	0\\
438	0\\
439	0\\
440	0\\
441	0\\
442	0\\
443	0\\
444	0\\
445	0\\
446	0\\
447	0\\
448	0\\
449	0\\
450	0\\
451	0\\
452	0\\
453	0\\
454	0\\
455	0\\
456	0\\
457	0\\
458	0\\
459	0\\
460	0\\
461	0\\
462	0\\
463	0\\
464	0\\
465	0\\
466	0\\
467	0\\
468	0\\
469	0\\
470	0\\
471	0\\
472	0\\
473	0\\
474	0\\
475	0\\
476	0\\
477	0\\
478	0\\
479	0\\
480	0\\
481	0\\
482	0\\
483	0\\
484	0\\
485	0\\
486	0\\
487	0\\
488	0\\
489	0\\
490	0\\
491	0\\
492	0\\
493	0\\
494	0\\
495	0\\
496	0\\
497	0\\
498	0\\
499	0\\
500	0\\
501	0\\
502	0\\
503	0\\
504	0\\
505	0\\
506	0\\
507	0\\
508	0\\
509	0\\
510	0\\
511	0\\
512	0\\
513	0\\
514	0\\
515	0\\
516	0\\
517	0\\
518	0\\
519	0\\
520	0\\
521	0\\
522	0\\
523	0\\
524	0\\
525	0\\
526	0\\
527	0\\
528	0\\
529	0\\
530	0\\
531	0\\
532	0\\
533	0\\
534	0\\
535	0\\
536	0\\
537	0\\
538	0\\
539	0\\
540	0\\
541	0\\
542	0\\
543	0\\
544	0\\
545	0\\
546	0\\
547	0\\
548	0\\
549	0\\
550	0\\
551	0\\
552	0\\
553	0\\
554	0\\
555	0\\
556	0\\
557	0\\
558	0\\
559	0\\
560	0\\
561	0\\
562	0\\
563	0\\
564	0\\
565	0\\
566	0\\
567	0\\
568	0\\
569	0\\
570	0\\
571	0\\
572	0\\
573	0\\
574	0\\
575	0\\
576	0\\
577	0\\
578	0\\
579	0\\
580	0\\
581	0\\
582	0\\
583	0\\
584	0\\
585	0\\
586	0\\
587	0\\
588	0\\
589	0\\
590	0\\
591	0\\
592	0\\
593	0\\
594	0\\
595	0\\
596	0\\
597	0\\
598	0\\
599	0\\
600	0\\
};
\addplot [color=mycolor4,solid,forget plot]
  table[row sep=crcr]{%
1	0\\
2	0\\
3	0\\
4	0\\
5	0\\
6	0\\
7	0\\
8	0\\
9	0\\
10	0\\
11	0\\
12	0\\
13	0\\
14	0\\
15	0\\
16	0\\
17	0\\
18	0\\
19	0\\
20	0\\
21	0\\
22	0\\
23	0\\
24	0\\
25	0\\
26	0\\
27	0\\
28	0\\
29	0\\
30	0\\
31	0\\
32	0\\
33	0\\
34	0\\
35	0\\
36	0\\
37	0\\
38	0\\
39	0\\
40	0\\
41	0\\
42	0\\
43	0\\
44	0\\
45	0\\
46	0\\
47	0\\
48	0\\
49	0\\
50	0\\
51	0\\
52	0\\
53	0\\
54	0\\
55	0\\
56	0\\
57	0\\
58	0\\
59	0\\
60	0\\
61	0\\
62	0\\
63	0\\
64	0\\
65	0\\
66	0\\
67	0\\
68	0\\
69	0\\
70	0\\
71	0\\
72	0\\
73	0\\
74	0\\
75	0\\
76	0\\
77	0\\
78	0\\
79	0\\
80	0\\
81	0\\
82	0\\
83	0\\
84	0\\
85	0\\
86	0\\
87	0\\
88	0\\
89	0\\
90	0\\
91	0\\
92	0\\
93	0\\
94	0\\
95	0\\
96	0\\
97	0\\
98	0\\
99	0\\
100	0\\
101	0\\
102	0\\
103	0\\
104	0\\
105	0\\
106	0\\
107	0\\
108	0\\
109	0\\
110	0\\
111	0\\
112	0\\
113	0\\
114	0\\
115	0\\
116	0\\
117	0\\
118	0\\
119	0\\
120	0\\
121	0\\
122	0\\
123	0\\
124	0\\
125	0\\
126	0\\
127	0\\
128	0\\
129	0\\
130	0\\
131	0\\
132	0\\
133	0\\
134	0\\
135	0\\
136	0\\
137	0\\
138	0\\
139	0\\
140	0\\
141	0\\
142	0\\
143	0\\
144	0\\
145	0\\
146	0\\
147	0\\
148	0\\
149	0\\
150	0\\
151	0\\
152	0\\
153	0\\
154	0\\
155	0\\
156	0\\
157	0\\
158	0\\
159	0\\
160	0\\
161	0\\
162	0\\
163	0\\
164	0\\
165	0\\
166	0\\
167	0\\
168	0\\
169	0\\
170	0\\
171	0\\
172	0\\
173	0\\
174	0\\
175	0\\
176	0\\
177	0\\
178	0\\
179	0\\
180	0\\
181	0\\
182	0\\
183	0\\
184	0\\
185	0\\
186	0\\
187	0\\
188	0\\
189	0\\
190	0\\
191	0\\
192	0\\
193	0\\
194	0\\
195	0\\
196	0\\
197	0\\
198	0\\
199	0\\
200	0\\
201	0\\
202	0\\
203	0\\
204	0\\
205	0\\
206	0\\
207	0\\
208	0\\
209	0\\
210	0\\
211	0\\
212	0\\
213	0\\
214	0\\
215	0\\
216	0\\
217	0\\
218	0\\
219	0\\
220	0\\
221	0\\
222	0\\
223	0\\
224	0\\
225	0\\
226	0\\
227	0\\
228	0\\
229	0\\
230	0\\
231	0\\
232	0\\
233	0\\
234	0\\
235	0\\
236	0\\
237	0\\
238	0\\
239	0\\
240	0\\
241	0\\
242	0\\
243	0\\
244	0\\
245	0\\
246	0\\
247	0\\
248	0\\
249	0\\
250	0\\
251	0\\
252	0\\
253	0\\
254	0\\
255	0\\
256	0\\
257	0\\
258	0\\
259	0\\
260	0\\
261	0\\
262	0\\
263	0\\
264	0\\
265	0\\
266	0\\
267	0\\
268	0\\
269	0\\
270	0\\
271	0\\
272	0\\
273	0\\
274	0\\
275	0\\
276	0\\
277	0\\
278	0\\
279	0\\
280	0\\
281	0\\
282	0\\
283	0\\
284	0\\
285	0\\
286	0\\
287	0\\
288	0\\
289	0\\
290	0\\
291	0\\
292	0\\
293	0\\
294	0\\
295	0\\
296	0\\
297	0\\
298	0\\
299	0\\
300	0\\
301	0\\
302	0\\
303	0\\
304	0\\
305	0\\
306	0\\
307	0\\
308	0\\
309	0\\
310	0\\
311	0\\
312	0\\
313	0\\
314	0\\
315	0\\
316	0\\
317	0\\
318	0\\
319	0\\
320	0\\
321	0\\
322	0\\
323	0\\
324	0\\
325	0\\
326	0\\
327	0\\
328	0\\
329	0\\
330	0\\
331	0\\
332	0\\
333	0\\
334	0\\
335	0\\
336	0\\
337	0\\
338	0\\
339	0\\
340	0\\
341	0\\
342	0\\
343	0\\
344	0\\
345	0\\
346	0\\
347	0\\
348	0\\
349	0\\
350	0\\
351	0\\
352	0\\
353	0\\
354	0\\
355	0\\
356	0\\
357	0\\
358	0\\
359	0\\
360	0\\
361	0\\
362	0\\
363	0\\
364	0\\
365	0\\
366	0\\
367	0\\
368	0\\
369	0\\
370	0\\
371	0\\
372	0\\
373	0\\
374	0\\
375	0\\
376	0\\
377	0\\
378	0\\
379	0\\
380	0\\
381	0\\
382	0\\
383	0\\
384	0\\
385	0\\
386	0\\
387	0\\
388	0\\
389	0\\
390	0\\
391	0\\
392	0\\
393	0\\
394	0\\
395	0\\
396	0\\
397	0\\
398	0\\
399	0\\
400	0\\
401	0\\
402	0\\
403	0\\
404	0\\
405	0\\
406	0\\
407	0\\
408	0\\
409	0\\
410	0\\
411	0\\
412	0\\
413	0\\
414	0\\
415	0\\
416	0\\
417	0\\
418	0\\
419	0\\
420	0\\
421	0\\
422	0\\
423	0\\
424	0\\
425	0\\
426	0\\
427	0\\
428	0\\
429	0\\
430	0\\
431	0\\
432	0\\
433	0\\
434	0\\
435	0\\
436	0\\
437	0\\
438	0\\
439	0\\
440	0\\
441	0\\
442	0\\
443	0\\
444	0\\
445	0\\
446	0\\
447	0\\
448	0\\
449	0\\
450	0\\
451	0\\
452	0\\
453	0\\
454	0\\
455	0\\
456	0\\
457	0\\
458	0\\
459	0\\
460	0\\
461	0\\
462	0\\
463	0\\
464	0\\
465	0\\
466	0\\
467	0\\
468	0\\
469	0\\
470	0\\
471	0\\
472	0\\
473	0\\
474	0\\
475	0\\
476	0\\
477	0\\
478	0\\
479	0\\
480	0\\
481	0\\
482	0\\
483	0\\
484	0\\
485	0\\
486	0\\
487	0\\
488	0\\
489	0\\
490	0\\
491	0\\
492	0\\
493	0\\
494	0\\
495	0\\
496	0\\
497	0\\
498	0\\
499	0\\
500	0\\
501	0\\
502	0\\
503	0\\
504	0\\
505	0\\
506	0\\
507	0\\
508	0\\
509	0\\
510	0\\
511	0\\
512	0\\
513	0\\
514	0\\
515	0\\
516	0\\
517	0\\
518	0\\
519	0\\
520	0\\
521	0\\
522	0\\
523	0\\
524	0\\
525	0\\
526	0\\
527	0\\
528	0\\
529	0\\
530	0\\
531	0\\
532	0\\
533	0\\
534	0\\
535	0\\
536	0\\
537	0\\
538	0\\
539	0\\
540	0\\
541	0\\
542	0\\
543	0\\
544	0\\
545	0\\
546	0\\
547	0\\
548	0\\
549	0\\
550	0\\
551	0\\
552	0\\
553	0\\
554	0\\
555	0\\
556	0\\
557	0\\
558	0\\
559	0\\
560	0\\
561	0\\
562	0\\
563	0\\
564	0\\
565	0\\
566	0\\
567	0\\
568	0\\
569	0\\
570	0\\
571	0\\
572	0\\
573	0\\
574	0\\
575	0\\
576	0\\
577	0\\
578	0\\
579	0\\
580	0\\
581	0\\
582	0\\
583	0\\
584	0\\
585	0\\
586	0\\
587	0\\
588	0\\
589	0\\
590	0\\
591	0\\
592	0\\
593	0\\
594	0\\
595	0\\
596	0\\
597	0\\
598	0\\
599	0\\
600	0\\
};
\addplot [color=mycolor5,solid,forget plot]
  table[row sep=crcr]{%
1	0\\
2	0\\
3	0\\
4	0\\
5	0\\
6	0\\
7	0\\
8	0\\
9	0\\
10	0\\
11	0\\
12	0\\
13	0\\
14	0\\
15	0\\
16	0\\
17	0\\
18	0\\
19	0\\
20	0\\
21	0\\
22	0\\
23	0\\
24	0\\
25	0\\
26	0\\
27	0\\
28	0\\
29	0\\
30	0\\
31	0\\
32	0\\
33	0\\
34	0\\
35	0\\
36	0\\
37	0\\
38	0\\
39	0\\
40	0\\
41	0\\
42	0\\
43	0\\
44	0\\
45	0\\
46	0\\
47	0\\
48	0\\
49	0\\
50	0\\
51	0\\
52	0\\
53	0\\
54	0\\
55	0\\
56	0\\
57	0\\
58	0\\
59	0\\
60	0\\
61	0\\
62	0\\
63	0\\
64	0\\
65	0\\
66	0\\
67	0\\
68	0\\
69	0\\
70	0\\
71	0\\
72	0\\
73	0\\
74	0\\
75	0\\
76	0\\
77	0\\
78	0\\
79	0\\
80	0\\
81	0\\
82	0\\
83	0\\
84	0\\
85	0\\
86	0\\
87	0\\
88	0\\
89	0\\
90	0\\
91	0\\
92	0\\
93	0\\
94	0\\
95	0\\
96	0\\
97	0\\
98	0\\
99	0\\
100	0\\
101	0\\
102	0\\
103	0\\
104	0\\
105	0\\
106	0\\
107	0\\
108	0\\
109	0\\
110	0\\
111	0\\
112	0\\
113	0\\
114	0\\
115	0\\
116	0\\
117	0\\
118	0\\
119	0\\
120	0\\
121	0\\
122	0\\
123	0\\
124	0\\
125	0\\
126	0\\
127	0\\
128	0\\
129	0\\
130	0\\
131	0\\
132	0\\
133	0\\
134	0\\
135	0\\
136	0\\
137	0\\
138	0\\
139	0\\
140	0\\
141	0\\
142	0\\
143	0\\
144	0\\
145	0\\
146	0\\
147	0\\
148	0\\
149	0\\
150	0\\
151	0\\
152	0\\
153	0\\
154	0\\
155	0\\
156	0\\
157	0\\
158	0\\
159	0\\
160	0\\
161	0\\
162	0\\
163	0\\
164	0\\
165	0\\
166	0\\
167	0\\
168	0\\
169	0\\
170	0\\
171	0\\
172	0\\
173	0\\
174	0\\
175	0\\
176	0\\
177	0\\
178	0\\
179	0\\
180	0\\
181	0\\
182	0\\
183	0\\
184	0\\
185	0\\
186	0\\
187	0\\
188	0\\
189	0\\
190	0\\
191	0\\
192	0\\
193	0\\
194	0\\
195	0\\
196	0\\
197	0\\
198	0\\
199	0\\
200	0\\
201	0\\
202	0\\
203	0\\
204	0\\
205	0\\
206	0\\
207	0\\
208	0\\
209	0\\
210	0\\
211	0\\
212	0\\
213	0\\
214	0\\
215	0\\
216	0\\
217	0\\
218	0\\
219	0\\
220	0\\
221	0\\
222	0\\
223	0\\
224	0\\
225	0\\
226	0\\
227	0\\
228	0\\
229	0\\
230	0\\
231	0\\
232	0\\
233	0\\
234	0\\
235	0\\
236	0\\
237	0\\
238	0\\
239	0\\
240	0\\
241	0\\
242	0\\
243	0\\
244	0\\
245	0\\
246	0\\
247	0\\
248	0\\
249	0\\
250	0\\
251	0\\
252	0\\
253	0\\
254	0\\
255	0\\
256	0\\
257	0\\
258	0\\
259	0\\
260	0\\
261	0\\
262	0\\
263	0\\
264	0\\
265	0\\
266	0\\
267	0\\
268	0\\
269	0\\
270	0\\
271	0\\
272	0\\
273	0\\
274	0\\
275	0\\
276	0\\
277	0\\
278	0\\
279	0\\
280	0\\
281	0\\
282	0\\
283	0\\
284	0\\
285	0\\
286	0\\
287	0\\
288	0\\
289	0\\
290	0\\
291	0\\
292	0\\
293	0\\
294	0\\
295	0\\
296	0\\
297	0\\
298	0\\
299	0\\
300	0\\
301	0\\
302	0\\
303	0\\
304	0\\
305	0\\
306	0\\
307	0\\
308	0\\
309	0\\
310	0\\
311	0\\
312	0\\
313	0\\
314	0\\
315	0\\
316	0\\
317	0\\
318	0\\
319	0\\
320	0\\
321	0\\
322	0\\
323	0\\
324	0\\
325	0\\
326	0\\
327	0\\
328	0\\
329	0\\
330	0\\
331	0\\
332	0\\
333	0\\
334	0\\
335	0\\
336	0\\
337	0\\
338	0\\
339	0\\
340	0\\
341	0\\
342	0\\
343	0\\
344	0\\
345	0\\
346	0\\
347	0\\
348	0\\
349	0\\
350	0\\
351	0\\
352	0\\
353	0\\
354	0\\
355	0\\
356	0\\
357	0\\
358	0\\
359	0\\
360	0\\
361	0\\
362	0\\
363	0\\
364	0\\
365	0\\
366	0\\
367	0\\
368	0\\
369	0\\
370	0\\
371	0\\
372	0\\
373	0\\
374	0\\
375	0\\
376	0\\
377	0\\
378	0\\
379	0\\
380	0\\
381	0\\
382	0\\
383	0\\
384	0\\
385	0\\
386	0\\
387	0\\
388	0\\
389	0\\
390	0\\
391	0\\
392	0\\
393	0\\
394	0\\
395	0\\
396	0\\
397	0\\
398	0\\
399	0\\
400	0\\
401	0\\
402	0\\
403	0\\
404	0\\
405	0\\
406	0\\
407	0\\
408	0\\
409	0\\
410	0\\
411	0\\
412	0\\
413	0\\
414	0\\
415	0\\
416	0\\
417	0\\
418	0\\
419	0\\
420	0\\
421	0\\
422	0\\
423	0\\
424	0\\
425	0\\
426	0\\
427	0\\
428	0\\
429	0\\
430	0\\
431	0\\
432	0\\
433	0\\
434	0\\
435	0\\
436	0\\
437	0\\
438	0\\
439	0\\
440	0\\
441	0\\
442	0\\
443	0\\
444	0\\
445	0\\
446	0\\
447	0\\
448	0\\
449	0\\
450	0\\
451	0\\
452	0\\
453	0\\
454	0\\
455	0\\
456	0\\
457	0\\
458	0\\
459	0\\
460	0\\
461	0\\
462	0\\
463	0\\
464	0\\
465	0\\
466	0\\
467	0\\
468	0\\
469	0\\
470	0\\
471	0\\
472	0\\
473	0\\
474	0\\
475	0\\
476	0\\
477	0\\
478	0\\
479	0\\
480	0\\
481	0\\
482	0\\
483	0\\
484	0\\
485	0\\
486	0\\
487	0\\
488	0\\
489	0\\
490	0\\
491	0\\
492	0\\
493	0\\
494	0\\
495	0\\
496	0\\
497	0\\
498	0\\
499	0\\
500	0\\
501	0\\
502	0\\
503	0\\
504	0\\
505	0\\
506	0\\
507	0\\
508	0\\
509	0\\
510	0\\
511	0\\
512	0\\
513	0\\
514	0\\
515	0\\
516	0\\
517	0\\
518	0\\
519	0\\
520	0\\
521	0\\
522	0\\
523	0\\
524	0\\
525	0\\
526	0\\
527	0\\
528	0\\
529	0\\
530	0\\
531	0\\
532	0\\
533	0\\
534	0\\
535	0\\
536	0\\
537	0\\
538	0\\
539	0\\
540	0\\
541	0\\
542	0\\
543	0\\
544	0\\
545	0\\
546	0\\
547	0\\
548	0\\
549	0\\
550	0\\
551	0\\
552	0\\
553	0\\
554	0\\
555	0\\
556	0\\
557	0\\
558	0\\
559	0\\
560	0\\
561	0\\
562	0\\
563	0\\
564	0\\
565	0\\
566	0\\
567	0\\
568	0\\
569	0\\
570	0\\
571	0\\
572	0\\
573	0\\
574	0\\
575	0\\
576	0\\
577	0\\
578	0\\
579	0\\
580	0\\
581	0\\
582	0\\
583	0\\
584	0\\
585	0\\
586	0\\
587	0\\
588	0\\
589	0\\
590	0\\
591	0\\
592	0\\
593	0\\
594	0\\
595	0\\
596	0\\
597	0\\
598	0\\
599	0\\
600	0\\
};
\addplot [color=mycolor6,solid,forget plot]
  table[row sep=crcr]{%
1	0\\
2	0\\
3	0\\
4	0\\
5	0\\
6	0\\
7	0\\
8	0\\
9	0\\
10	0\\
11	0\\
12	0\\
13	0\\
14	0\\
15	0\\
16	0\\
17	0\\
18	0\\
19	0\\
20	0\\
21	0\\
22	0\\
23	0\\
24	0\\
25	0\\
26	0\\
27	0\\
28	0\\
29	0\\
30	0\\
31	0\\
32	0\\
33	0\\
34	0\\
35	0\\
36	0\\
37	0\\
38	0\\
39	0\\
40	0\\
41	0\\
42	0\\
43	0\\
44	0\\
45	0\\
46	0\\
47	0\\
48	0\\
49	0\\
50	0\\
51	0\\
52	0\\
53	0\\
54	0\\
55	0\\
56	0\\
57	0\\
58	0\\
59	0\\
60	0\\
61	0\\
62	0\\
63	0\\
64	0\\
65	0\\
66	0\\
67	0\\
68	0\\
69	0\\
70	0\\
71	0\\
72	0\\
73	0\\
74	0\\
75	0\\
76	0\\
77	0\\
78	0\\
79	0\\
80	0\\
81	0\\
82	0\\
83	0\\
84	0\\
85	0\\
86	0\\
87	0\\
88	0\\
89	0\\
90	0\\
91	0\\
92	0\\
93	0\\
94	0\\
95	0\\
96	0\\
97	0\\
98	0\\
99	0\\
100	0\\
101	0\\
102	0\\
103	0\\
104	0\\
105	0\\
106	0\\
107	0\\
108	0\\
109	0\\
110	0\\
111	0\\
112	0\\
113	0\\
114	0\\
115	0\\
116	0\\
117	0\\
118	0\\
119	0\\
120	0\\
121	0\\
122	0\\
123	0\\
124	0\\
125	0\\
126	0\\
127	0\\
128	0\\
129	0\\
130	0\\
131	0\\
132	0\\
133	0\\
134	0\\
135	0\\
136	0\\
137	0\\
138	0\\
139	0\\
140	0\\
141	0\\
142	0\\
143	0\\
144	0\\
145	0\\
146	0\\
147	0\\
148	0\\
149	0\\
150	0\\
151	0\\
152	0\\
153	0\\
154	0\\
155	0\\
156	0\\
157	0\\
158	0\\
159	0\\
160	0\\
161	0\\
162	0\\
163	0\\
164	0\\
165	0\\
166	0\\
167	0\\
168	0\\
169	0\\
170	0\\
171	0\\
172	0\\
173	0\\
174	0\\
175	0\\
176	0\\
177	0\\
178	0\\
179	0\\
180	0\\
181	0\\
182	0\\
183	0\\
184	0\\
185	0\\
186	0\\
187	0\\
188	0\\
189	0\\
190	0\\
191	0\\
192	0\\
193	0\\
194	0\\
195	0\\
196	0\\
197	0\\
198	0\\
199	0\\
200	0\\
201	0\\
202	0\\
203	0\\
204	0\\
205	0\\
206	0\\
207	0\\
208	0\\
209	0\\
210	0\\
211	0\\
212	0\\
213	0\\
214	0\\
215	0\\
216	0\\
217	0\\
218	0\\
219	0\\
220	0\\
221	0\\
222	0\\
223	0\\
224	0\\
225	0\\
226	0\\
227	0\\
228	0\\
229	0\\
230	0\\
231	0\\
232	0\\
233	0\\
234	0\\
235	0\\
236	0\\
237	0\\
238	0\\
239	0\\
240	0\\
241	0\\
242	0\\
243	0\\
244	0\\
245	0\\
246	0\\
247	0\\
248	0\\
249	0\\
250	0\\
251	0\\
252	0\\
253	0\\
254	0\\
255	0\\
256	0\\
257	0\\
258	0\\
259	0\\
260	0\\
261	0\\
262	0\\
263	0\\
264	0\\
265	0\\
266	0\\
267	0\\
268	0\\
269	0\\
270	0\\
271	0\\
272	0\\
273	0\\
274	0\\
275	0\\
276	0\\
277	0\\
278	0\\
279	0\\
280	0\\
281	0\\
282	0\\
283	0\\
284	0\\
285	0\\
286	0\\
287	0\\
288	0\\
289	0\\
290	0\\
291	0\\
292	0\\
293	0\\
294	0\\
295	0\\
296	0\\
297	0\\
298	0\\
299	0\\
300	0\\
301	0\\
302	0\\
303	0\\
304	0\\
305	0\\
306	0\\
307	0\\
308	0\\
309	0\\
310	0\\
311	0\\
312	0\\
313	0\\
314	0\\
315	0\\
316	0\\
317	0\\
318	0\\
319	0\\
320	0\\
321	0\\
322	0\\
323	0\\
324	0\\
325	0\\
326	0\\
327	0\\
328	0\\
329	0\\
330	0\\
331	0\\
332	0\\
333	0\\
334	0\\
335	0\\
336	0\\
337	0\\
338	0\\
339	0\\
340	0\\
341	0\\
342	0\\
343	0\\
344	0\\
345	0\\
346	0\\
347	0\\
348	0\\
349	0\\
350	0\\
351	0\\
352	0\\
353	0\\
354	0\\
355	0\\
356	0\\
357	0\\
358	0\\
359	0\\
360	0\\
361	0\\
362	0\\
363	0\\
364	0\\
365	0\\
366	0\\
367	0\\
368	0\\
369	0\\
370	0\\
371	0\\
372	0\\
373	0\\
374	0\\
375	0\\
376	0\\
377	0\\
378	0\\
379	0\\
380	0\\
381	0\\
382	0\\
383	0\\
384	0\\
385	0\\
386	0\\
387	0\\
388	0\\
389	0\\
390	0\\
391	0\\
392	0\\
393	0\\
394	0\\
395	0\\
396	0\\
397	0\\
398	0\\
399	0\\
400	0\\
401	0\\
402	0\\
403	0\\
404	0\\
405	0\\
406	0\\
407	0\\
408	0\\
409	0\\
410	0\\
411	0\\
412	0\\
413	0\\
414	0\\
415	0\\
416	0\\
417	0\\
418	0\\
419	0\\
420	0\\
421	0\\
422	0\\
423	0\\
424	0\\
425	0\\
426	0\\
427	0\\
428	0\\
429	0\\
430	0\\
431	0\\
432	0\\
433	0\\
434	0\\
435	0\\
436	0\\
437	0\\
438	0\\
439	0\\
440	0\\
441	0\\
442	0\\
443	0\\
444	0\\
445	0\\
446	0\\
447	0\\
448	0\\
449	0\\
450	0\\
451	0\\
452	0\\
453	0\\
454	0\\
455	0\\
456	0\\
457	0\\
458	0\\
459	0\\
460	0\\
461	0\\
462	0\\
463	0\\
464	0\\
465	0\\
466	0\\
467	0\\
468	0\\
469	0\\
470	0\\
471	0\\
472	0\\
473	0\\
474	0\\
475	0\\
476	0\\
477	0\\
478	0\\
479	0\\
480	0\\
481	0\\
482	0\\
483	0\\
484	0\\
485	0\\
486	0\\
487	0\\
488	0\\
489	0\\
490	0\\
491	0\\
492	0\\
493	0\\
494	0\\
495	0\\
496	0\\
497	0\\
498	0\\
499	0\\
500	0\\
501	0\\
502	0\\
503	0\\
504	0\\
505	0\\
506	0\\
507	0\\
508	0\\
509	0\\
510	0\\
511	0\\
512	0\\
513	0\\
514	0\\
515	0\\
516	0\\
517	0\\
518	0\\
519	0\\
520	0\\
521	0\\
522	0\\
523	0\\
524	0\\
525	0\\
526	0\\
527	0\\
528	0\\
529	0\\
530	0\\
531	0\\
532	0\\
533	0\\
534	0\\
535	0\\
536	0\\
537	0\\
538	0\\
539	0\\
540	0\\
541	0\\
542	0\\
543	0\\
544	0\\
545	0\\
546	0\\
547	0\\
548	0\\
549	0\\
550	0\\
551	0\\
552	0\\
553	0\\
554	0\\
555	0\\
556	0\\
557	0\\
558	0\\
559	0\\
560	0\\
561	0\\
562	0\\
563	0\\
564	0\\
565	0\\
566	0\\
567	0\\
568	0\\
569	0\\
570	0\\
571	0\\
572	0\\
573	0\\
574	0\\
575	0\\
576	0\\
577	0\\
578	0\\
579	0\\
580	0\\
581	0\\
582	0\\
583	0\\
584	0\\
585	0\\
586	0\\
587	0\\
588	0\\
589	0\\
590	0\\
591	0\\
592	0\\
593	0\\
594	0\\
595	0\\
596	0\\
597	0\\
598	0\\
599	0\\
600	0\\
};
\addplot [color=mycolor7,solid,forget plot]
  table[row sep=crcr]{%
1	0\\
2	0\\
3	0\\
4	0\\
5	0\\
6	0\\
7	0\\
8	0\\
9	0\\
10	0\\
11	0\\
12	0\\
13	0\\
14	0\\
15	0\\
16	0\\
17	0\\
18	0\\
19	0\\
20	0\\
21	0\\
22	0\\
23	0\\
24	0\\
25	0\\
26	0\\
27	0\\
28	0\\
29	0\\
30	0\\
31	0\\
32	0\\
33	0\\
34	0\\
35	0\\
36	0\\
37	0\\
38	0\\
39	0\\
40	0\\
41	0\\
42	0\\
43	0\\
44	0\\
45	0\\
46	0\\
47	0\\
48	0\\
49	0\\
50	0\\
51	0\\
52	0\\
53	0\\
54	0\\
55	0\\
56	0\\
57	0\\
58	0\\
59	0\\
60	0\\
61	0\\
62	0\\
63	0\\
64	0\\
65	0\\
66	0\\
67	0\\
68	0\\
69	0\\
70	0\\
71	0\\
72	0\\
73	0\\
74	0\\
75	0\\
76	0\\
77	0\\
78	0\\
79	0\\
80	0\\
81	0\\
82	0\\
83	0\\
84	0\\
85	0\\
86	0\\
87	0\\
88	0\\
89	0\\
90	0\\
91	0\\
92	0\\
93	0\\
94	0\\
95	0\\
96	0\\
97	0\\
98	0\\
99	0\\
100	0\\
101	0\\
102	0\\
103	0\\
104	0\\
105	0\\
106	0\\
107	0\\
108	0\\
109	0\\
110	0\\
111	0\\
112	0\\
113	0\\
114	0\\
115	0\\
116	0\\
117	0\\
118	0\\
119	0\\
120	0\\
121	0\\
122	0\\
123	0\\
124	0\\
125	0\\
126	0\\
127	0\\
128	0\\
129	0\\
130	0\\
131	0\\
132	0\\
133	0\\
134	0\\
135	0\\
136	0\\
137	0\\
138	0\\
139	0\\
140	0\\
141	0\\
142	0\\
143	0\\
144	0\\
145	0\\
146	0\\
147	0\\
148	0\\
149	0\\
150	0\\
151	0\\
152	0\\
153	0\\
154	0\\
155	0\\
156	0\\
157	0\\
158	0\\
159	0\\
160	0\\
161	0\\
162	0\\
163	0\\
164	0\\
165	0\\
166	0\\
167	0\\
168	0\\
169	0\\
170	0\\
171	0\\
172	0\\
173	0\\
174	0\\
175	0\\
176	0\\
177	0\\
178	0\\
179	0\\
180	0\\
181	0\\
182	0\\
183	0\\
184	0\\
185	0\\
186	0\\
187	0\\
188	0\\
189	0\\
190	0\\
191	0\\
192	0\\
193	0\\
194	0\\
195	0\\
196	0\\
197	0\\
198	0\\
199	0\\
200	0\\
201	0\\
202	0\\
203	0\\
204	0\\
205	0\\
206	0\\
207	0\\
208	0\\
209	0\\
210	0\\
211	0\\
212	0\\
213	0\\
214	0\\
215	0\\
216	0\\
217	0\\
218	0\\
219	0\\
220	0\\
221	0\\
222	0\\
223	0\\
224	0\\
225	0\\
226	0\\
227	0\\
228	0\\
229	0\\
230	0\\
231	0\\
232	0\\
233	0\\
234	0\\
235	0\\
236	0\\
237	0\\
238	0\\
239	0\\
240	0\\
241	0\\
242	0\\
243	0\\
244	0\\
245	0\\
246	0\\
247	0\\
248	0\\
249	0\\
250	0\\
251	0\\
252	0\\
253	0\\
254	0\\
255	0\\
256	0\\
257	0\\
258	0\\
259	0\\
260	0\\
261	0\\
262	0\\
263	0\\
264	0\\
265	0\\
266	0\\
267	0\\
268	0\\
269	0\\
270	0\\
271	0\\
272	0\\
273	0\\
274	0\\
275	0\\
276	0\\
277	0\\
278	0\\
279	0\\
280	0\\
281	0\\
282	0\\
283	0\\
284	0\\
285	0\\
286	0\\
287	0\\
288	0\\
289	0\\
290	0\\
291	0\\
292	0\\
293	0\\
294	0\\
295	0\\
296	0\\
297	0\\
298	0\\
299	0\\
300	0\\
301	0\\
302	0\\
303	0\\
304	0\\
305	0\\
306	0\\
307	0\\
308	0\\
309	0\\
310	0\\
311	0\\
312	0\\
313	0\\
314	0\\
315	0\\
316	0\\
317	0\\
318	0\\
319	0\\
320	0\\
321	0\\
322	0\\
323	0\\
324	0\\
325	0\\
326	0\\
327	0\\
328	0\\
329	0\\
330	0\\
331	0\\
332	0\\
333	0\\
334	0\\
335	0\\
336	0\\
337	0\\
338	0\\
339	0\\
340	0\\
341	0\\
342	0\\
343	0\\
344	0\\
345	0\\
346	0\\
347	0\\
348	0\\
349	0\\
350	0\\
351	0\\
352	0\\
353	0\\
354	0\\
355	0\\
356	0\\
357	0\\
358	0\\
359	0\\
360	0\\
361	0\\
362	0\\
363	0\\
364	0\\
365	0\\
366	0\\
367	0\\
368	0\\
369	0\\
370	0\\
371	0\\
372	0\\
373	0\\
374	0\\
375	0\\
376	0\\
377	0\\
378	0\\
379	0\\
380	0\\
381	0\\
382	0\\
383	0\\
384	0\\
385	0\\
386	0\\
387	0\\
388	0\\
389	0\\
390	0\\
391	0\\
392	0\\
393	0\\
394	0\\
395	0\\
396	0\\
397	0\\
398	0\\
399	0\\
400	0\\
401	0\\
402	0\\
403	0\\
404	0\\
405	0\\
406	0\\
407	0\\
408	0\\
409	0\\
410	0\\
411	0\\
412	0\\
413	0\\
414	0\\
415	0\\
416	0\\
417	0\\
418	0\\
419	0\\
420	0\\
421	0\\
422	0\\
423	0\\
424	0\\
425	0\\
426	0\\
427	0\\
428	0\\
429	0\\
430	0\\
431	0\\
432	0\\
433	0\\
434	0\\
435	0\\
436	0\\
437	0\\
438	0\\
439	0\\
440	0\\
441	0\\
442	0\\
443	0\\
444	0\\
445	0\\
446	0\\
447	0\\
448	0\\
449	0\\
450	0\\
451	0\\
452	0\\
453	0\\
454	0\\
455	0\\
456	0\\
457	0\\
458	0\\
459	0\\
460	0\\
461	0\\
462	0\\
463	0\\
464	0\\
465	0\\
466	0\\
467	0\\
468	0\\
469	0\\
470	0\\
471	0\\
472	0\\
473	0\\
474	0\\
475	0\\
476	0\\
477	0\\
478	0\\
479	0\\
480	0\\
481	0\\
482	0\\
483	0\\
484	0\\
485	0\\
486	0\\
487	0\\
488	0\\
489	0\\
490	0\\
491	0\\
492	0\\
493	0\\
494	0\\
495	0\\
496	0\\
497	0\\
498	0\\
499	0\\
500	0\\
501	0\\
502	0\\
503	0\\
504	0\\
505	0\\
506	0\\
507	0\\
508	0\\
509	0\\
510	0\\
511	0\\
512	0\\
513	0\\
514	0\\
515	0\\
516	0\\
517	0\\
518	0\\
519	0\\
520	0\\
521	0\\
522	0\\
523	0\\
524	0\\
525	0\\
526	0\\
527	0\\
528	0\\
529	0\\
530	0\\
531	0\\
532	0\\
533	0\\
534	0\\
535	0\\
536	0\\
537	0\\
538	0\\
539	0\\
540	0\\
541	0\\
542	0\\
543	0\\
544	0\\
545	0\\
546	0\\
547	0\\
548	0\\
549	0\\
550	0\\
551	0\\
552	0\\
553	0\\
554	0\\
555	0\\
556	0\\
557	0\\
558	0\\
559	0\\
560	0\\
561	0\\
562	0\\
563	0\\
564	0\\
565	0\\
566	0\\
567	0\\
568	0\\
569	0\\
570	0\\
571	0\\
572	0\\
573	0\\
574	0\\
575	0\\
576	0\\
577	0\\
578	0\\
579	0\\
580	0\\
581	0\\
582	0\\
583	0\\
584	0\\
585	0\\
586	0\\
587	0\\
588	0\\
589	0\\
590	0\\
591	0\\
592	0\\
593	0\\
594	0\\
595	0\\
596	0\\
597	0\\
598	0\\
599	0\\
600	0\\
};
\addplot [color=mycolor8,solid,forget plot]
  table[row sep=crcr]{%
1	0\\
2	0\\
3	0\\
4	0\\
5	0\\
6	0\\
7	0\\
8	0\\
9	0\\
10	0\\
11	0\\
12	0\\
13	0\\
14	0\\
15	0\\
16	0\\
17	0\\
18	0\\
19	0\\
20	0\\
21	0\\
22	0\\
23	0\\
24	0\\
25	0\\
26	0\\
27	0\\
28	0\\
29	0\\
30	0\\
31	0\\
32	0\\
33	0\\
34	0\\
35	0\\
36	0\\
37	0\\
38	0\\
39	0\\
40	0\\
41	0\\
42	0\\
43	0\\
44	0\\
45	0\\
46	0\\
47	0\\
48	0\\
49	0\\
50	0\\
51	0\\
52	0\\
53	0\\
54	0\\
55	0\\
56	0\\
57	0\\
58	0\\
59	0\\
60	0\\
61	0\\
62	0\\
63	0\\
64	0\\
65	0\\
66	0\\
67	0\\
68	0\\
69	0\\
70	0\\
71	0\\
72	0\\
73	0\\
74	0\\
75	0\\
76	0\\
77	0\\
78	0\\
79	0\\
80	0\\
81	0\\
82	0\\
83	0\\
84	0\\
85	0\\
86	0\\
87	0\\
88	0\\
89	0\\
90	0\\
91	0\\
92	0\\
93	0\\
94	0\\
95	0\\
96	0\\
97	0\\
98	0\\
99	0\\
100	0\\
101	0\\
102	0\\
103	0\\
104	0\\
105	0\\
106	0\\
107	0\\
108	0\\
109	0\\
110	0\\
111	0\\
112	0\\
113	0\\
114	0\\
115	0\\
116	0\\
117	0\\
118	0\\
119	0\\
120	0\\
121	0\\
122	0\\
123	0\\
124	0\\
125	0\\
126	0\\
127	0\\
128	0\\
129	0\\
130	0\\
131	0\\
132	0\\
133	0\\
134	0\\
135	0\\
136	0\\
137	0\\
138	0\\
139	0\\
140	0\\
141	0\\
142	0\\
143	0\\
144	0\\
145	0\\
146	0\\
147	0\\
148	0\\
149	0\\
150	0\\
151	0\\
152	0\\
153	0\\
154	0\\
155	0\\
156	0\\
157	0\\
158	0\\
159	0\\
160	0\\
161	0\\
162	0\\
163	0\\
164	0\\
165	0\\
166	0\\
167	0\\
168	0\\
169	0\\
170	0\\
171	0\\
172	0\\
173	0\\
174	0\\
175	0\\
176	0\\
177	0\\
178	0\\
179	0\\
180	0\\
181	0\\
182	0\\
183	0\\
184	0\\
185	0\\
186	0\\
187	0\\
188	0\\
189	0\\
190	0\\
191	0\\
192	0\\
193	0\\
194	0\\
195	0\\
196	0\\
197	0\\
198	0\\
199	0\\
200	0\\
201	0\\
202	0\\
203	0\\
204	0\\
205	0\\
206	0\\
207	0\\
208	0\\
209	0\\
210	0\\
211	0\\
212	0\\
213	0\\
214	0\\
215	0\\
216	0\\
217	0\\
218	0\\
219	0\\
220	0\\
221	0\\
222	0\\
223	0\\
224	0\\
225	0\\
226	0\\
227	0\\
228	0\\
229	0\\
230	0\\
231	0\\
232	0\\
233	0\\
234	0\\
235	0\\
236	0\\
237	0\\
238	0\\
239	0\\
240	0\\
241	0\\
242	0\\
243	0\\
244	0\\
245	0\\
246	0\\
247	0\\
248	0\\
249	0\\
250	0\\
251	0\\
252	0\\
253	0\\
254	0\\
255	0\\
256	0\\
257	0\\
258	0\\
259	0\\
260	0\\
261	0\\
262	0\\
263	0\\
264	0\\
265	0\\
266	0\\
267	0\\
268	0\\
269	0\\
270	0\\
271	0\\
272	0\\
273	0\\
274	0\\
275	0\\
276	0\\
277	0\\
278	0\\
279	0\\
280	0\\
281	0\\
282	0\\
283	0\\
284	0\\
285	0\\
286	0\\
287	0\\
288	0\\
289	0\\
290	0\\
291	0\\
292	0\\
293	0\\
294	0\\
295	0\\
296	0\\
297	0\\
298	0\\
299	0\\
300	0\\
301	0\\
302	0\\
303	0\\
304	0\\
305	0\\
306	0\\
307	0\\
308	0\\
309	0\\
310	0\\
311	0\\
312	0\\
313	0\\
314	0\\
315	0\\
316	0\\
317	0\\
318	0\\
319	0\\
320	0\\
321	0\\
322	0\\
323	0\\
324	0\\
325	0\\
326	0\\
327	0\\
328	0\\
329	0\\
330	0\\
331	0\\
332	0\\
333	0\\
334	0\\
335	0\\
336	0\\
337	0\\
338	0\\
339	0\\
340	0\\
341	0\\
342	0\\
343	0\\
344	0\\
345	0\\
346	0\\
347	0\\
348	0\\
349	0\\
350	0\\
351	0\\
352	0\\
353	0\\
354	0\\
355	0\\
356	0\\
357	0\\
358	0\\
359	0\\
360	0\\
361	0\\
362	0\\
363	0\\
364	0\\
365	0\\
366	0\\
367	0\\
368	0\\
369	0\\
370	0\\
371	0\\
372	0\\
373	0\\
374	0\\
375	0\\
376	0\\
377	0\\
378	0\\
379	0\\
380	0\\
381	0\\
382	0\\
383	0\\
384	0\\
385	0\\
386	0\\
387	0\\
388	0\\
389	0\\
390	0\\
391	0\\
392	0\\
393	0\\
394	0\\
395	0\\
396	0\\
397	0\\
398	0\\
399	0\\
400	0\\
401	0\\
402	0\\
403	0\\
404	0\\
405	0\\
406	0\\
407	0\\
408	0\\
409	0\\
410	0\\
411	0\\
412	0\\
413	0\\
414	0\\
415	0\\
416	0\\
417	0\\
418	0\\
419	0\\
420	0\\
421	0\\
422	0\\
423	0\\
424	0\\
425	0\\
426	0\\
427	0\\
428	0\\
429	0\\
430	0\\
431	0\\
432	0\\
433	0\\
434	0\\
435	0\\
436	0\\
437	0\\
438	0\\
439	0\\
440	0\\
441	0\\
442	0\\
443	0\\
444	0\\
445	0\\
446	0\\
447	0\\
448	0\\
449	0\\
450	0\\
451	0\\
452	0\\
453	0\\
454	0\\
455	0\\
456	0\\
457	0\\
458	0\\
459	0\\
460	0\\
461	0\\
462	0\\
463	0\\
464	0\\
465	0\\
466	0\\
467	0\\
468	0\\
469	0\\
470	0\\
471	0\\
472	0\\
473	0\\
474	0\\
475	0\\
476	0\\
477	0\\
478	0\\
479	0\\
480	0\\
481	0\\
482	0\\
483	0\\
484	0\\
485	0\\
486	0\\
487	0\\
488	0\\
489	0\\
490	0\\
491	0\\
492	0\\
493	0\\
494	0\\
495	0\\
496	0\\
497	0\\
498	0\\
499	0\\
500	0\\
501	0\\
502	0\\
503	0\\
504	0\\
505	0\\
506	0\\
507	0\\
508	0\\
509	0\\
510	0\\
511	0\\
512	0\\
513	0\\
514	0\\
515	0\\
516	0\\
517	0\\
518	0\\
519	0\\
520	0\\
521	0\\
522	0\\
523	0\\
524	0\\
525	0\\
526	0\\
527	0\\
528	0\\
529	0\\
530	0\\
531	0\\
532	0\\
533	0\\
534	0\\
535	0\\
536	0\\
537	0\\
538	0\\
539	0\\
540	0\\
541	0\\
542	0\\
543	0\\
544	0\\
545	0\\
546	0\\
547	0\\
548	0\\
549	0\\
550	0\\
551	0\\
552	0\\
553	0\\
554	0\\
555	0\\
556	0\\
557	0\\
558	0\\
559	0\\
560	0\\
561	0\\
562	0\\
563	0\\
564	0\\
565	0\\
566	0\\
567	0\\
568	0\\
569	0\\
570	0\\
571	0\\
572	0\\
573	0\\
574	0\\
575	0\\
576	0\\
577	0\\
578	0\\
579	0\\
580	0\\
581	0\\
582	0\\
583	0\\
584	0\\
585	0\\
586	0\\
587	0\\
588	0\\
589	0\\
590	0\\
591	0\\
592	0\\
593	0\\
594	0\\
595	0\\
596	0\\
597	0\\
598	0\\
599	0\\
600	0\\
};
\addplot [color=blue!25!mycolor7,solid,forget plot]
  table[row sep=crcr]{%
1	0\\
2	0\\
3	0\\
4	0\\
5	0\\
6	0\\
7	0\\
8	0\\
9	0\\
10	0\\
11	0\\
12	0\\
13	0\\
14	0\\
15	0\\
16	0\\
17	0\\
18	0\\
19	0\\
20	0\\
21	0\\
22	0\\
23	0\\
24	0\\
25	0\\
26	0\\
27	0\\
28	0\\
29	0\\
30	0\\
31	0\\
32	0\\
33	0\\
34	0\\
35	0\\
36	0\\
37	0\\
38	0\\
39	0\\
40	0\\
41	0\\
42	0\\
43	0\\
44	0\\
45	0\\
46	0\\
47	0\\
48	0\\
49	0\\
50	0\\
51	0\\
52	0\\
53	0\\
54	0\\
55	0\\
56	0\\
57	0\\
58	0\\
59	0\\
60	0\\
61	0\\
62	0\\
63	0\\
64	0\\
65	0\\
66	0\\
67	0\\
68	0\\
69	0\\
70	0\\
71	0\\
72	0\\
73	0\\
74	0\\
75	0\\
76	0\\
77	0\\
78	0\\
79	0\\
80	0\\
81	0\\
82	0\\
83	0\\
84	0\\
85	0\\
86	0\\
87	0\\
88	0\\
89	0\\
90	0\\
91	0\\
92	0\\
93	0\\
94	0\\
95	0\\
96	0\\
97	0\\
98	0\\
99	0\\
100	0\\
101	0\\
102	0\\
103	0\\
104	0\\
105	0\\
106	0\\
107	0\\
108	0\\
109	0\\
110	0\\
111	0\\
112	0\\
113	0\\
114	0\\
115	0\\
116	0\\
117	0\\
118	0\\
119	0\\
120	0\\
121	0\\
122	0\\
123	0\\
124	0\\
125	0\\
126	0\\
127	0\\
128	0\\
129	0\\
130	0\\
131	0\\
132	0\\
133	0\\
134	0\\
135	0\\
136	0\\
137	0\\
138	0\\
139	0\\
140	0\\
141	0\\
142	0\\
143	0\\
144	0\\
145	0\\
146	0\\
147	0\\
148	0\\
149	0\\
150	0\\
151	0\\
152	0\\
153	0\\
154	0\\
155	0\\
156	0\\
157	0\\
158	0\\
159	0\\
160	0\\
161	0\\
162	0\\
163	0\\
164	0\\
165	0\\
166	0\\
167	0\\
168	0\\
169	0\\
170	0\\
171	0\\
172	0\\
173	0\\
174	0\\
175	0\\
176	0\\
177	0\\
178	0\\
179	0\\
180	0\\
181	0\\
182	0\\
183	0\\
184	0\\
185	0\\
186	0\\
187	0\\
188	0\\
189	0\\
190	0\\
191	0\\
192	0\\
193	0\\
194	0\\
195	0\\
196	0\\
197	0\\
198	0\\
199	0\\
200	0\\
201	0\\
202	0\\
203	0\\
204	0\\
205	0\\
206	0\\
207	0\\
208	0\\
209	0\\
210	0\\
211	0\\
212	0\\
213	0\\
214	0\\
215	0\\
216	0\\
217	0\\
218	0\\
219	0\\
220	0\\
221	0\\
222	0\\
223	0\\
224	0\\
225	0\\
226	0\\
227	0\\
228	0\\
229	0\\
230	0\\
231	0\\
232	0\\
233	0\\
234	0\\
235	0\\
236	0\\
237	0\\
238	0\\
239	0\\
240	0\\
241	0\\
242	0\\
243	0\\
244	0\\
245	0\\
246	0\\
247	0\\
248	0\\
249	0\\
250	0\\
251	0\\
252	0\\
253	0\\
254	0\\
255	0\\
256	0\\
257	0\\
258	0\\
259	0\\
260	0\\
261	0\\
262	0\\
263	0\\
264	0\\
265	0\\
266	0\\
267	0\\
268	0\\
269	0\\
270	0\\
271	0\\
272	0\\
273	0\\
274	0\\
275	0\\
276	0\\
277	0\\
278	0\\
279	0\\
280	0\\
281	0\\
282	0\\
283	0\\
284	0\\
285	0\\
286	0\\
287	0\\
288	0\\
289	0\\
290	0\\
291	0\\
292	0\\
293	0\\
294	0\\
295	0\\
296	0\\
297	0\\
298	0\\
299	0\\
300	0\\
301	0\\
302	0\\
303	0\\
304	0\\
305	0\\
306	0\\
307	0\\
308	0\\
309	0\\
310	0\\
311	0\\
312	0\\
313	0\\
314	0\\
315	0\\
316	0\\
317	0\\
318	0\\
319	0\\
320	0\\
321	0\\
322	0\\
323	0\\
324	0\\
325	0\\
326	0\\
327	0\\
328	0\\
329	0\\
330	0\\
331	0\\
332	0\\
333	0\\
334	0\\
335	0\\
336	0\\
337	0\\
338	0\\
339	0\\
340	0\\
341	0\\
342	0\\
343	0\\
344	0\\
345	0\\
346	0\\
347	0\\
348	0\\
349	0\\
350	0\\
351	0\\
352	0\\
353	0\\
354	0\\
355	0\\
356	0\\
357	0\\
358	0\\
359	0\\
360	0\\
361	0\\
362	0\\
363	0\\
364	0\\
365	0\\
366	0\\
367	0\\
368	0\\
369	0\\
370	0\\
371	0\\
372	0\\
373	0\\
374	0\\
375	0\\
376	0\\
377	0\\
378	0\\
379	0\\
380	0\\
381	0\\
382	0\\
383	0\\
384	0\\
385	0\\
386	0\\
387	0\\
388	0\\
389	0\\
390	0\\
391	0\\
392	0\\
393	0\\
394	0\\
395	0\\
396	0\\
397	0\\
398	0\\
399	0\\
400	0\\
401	0\\
402	0\\
403	0\\
404	0\\
405	0\\
406	0\\
407	0\\
408	0\\
409	0\\
410	0\\
411	0\\
412	0\\
413	0\\
414	0\\
415	0\\
416	0\\
417	0\\
418	0\\
419	0\\
420	0\\
421	0\\
422	0\\
423	0\\
424	0\\
425	0\\
426	0\\
427	0\\
428	0\\
429	0\\
430	0\\
431	0\\
432	0\\
433	0\\
434	0\\
435	0\\
436	0\\
437	0\\
438	0\\
439	0\\
440	0\\
441	0\\
442	0\\
443	0\\
444	0\\
445	0\\
446	0\\
447	0\\
448	0\\
449	0\\
450	0\\
451	0\\
452	0\\
453	0\\
454	0\\
455	0\\
456	0\\
457	0\\
458	0\\
459	0\\
460	0\\
461	0\\
462	0\\
463	0\\
464	0\\
465	0\\
466	0\\
467	0\\
468	0\\
469	0\\
470	0\\
471	0\\
472	0\\
473	0\\
474	0\\
475	0\\
476	0\\
477	0\\
478	0\\
479	0\\
480	0\\
481	0\\
482	0\\
483	0\\
484	0\\
485	0\\
486	0\\
487	0\\
488	0\\
489	0\\
490	0\\
491	0\\
492	0\\
493	0\\
494	0\\
495	0\\
496	0\\
497	0\\
498	0\\
499	0\\
500	0\\
501	0\\
502	0\\
503	0\\
504	0\\
505	0\\
506	0\\
507	0\\
508	0\\
509	0\\
510	0\\
511	0\\
512	0\\
513	0\\
514	0\\
515	0\\
516	0\\
517	0\\
518	0\\
519	0\\
520	0\\
521	0\\
522	0\\
523	0\\
524	0\\
525	0\\
526	0\\
527	0\\
528	0\\
529	0\\
530	0\\
531	0\\
532	0\\
533	0\\
534	0\\
535	0\\
536	0\\
537	0\\
538	0\\
539	0\\
540	0\\
541	0\\
542	0\\
543	0\\
544	0\\
545	0\\
546	0\\
547	0\\
548	0\\
549	0\\
550	0\\
551	0\\
552	0\\
553	0\\
554	0\\
555	0\\
556	0\\
557	0\\
558	0\\
559	0\\
560	0\\
561	0\\
562	0\\
563	0\\
564	0\\
565	0\\
566	0\\
567	0\\
568	0\\
569	0\\
570	0\\
571	0\\
572	0\\
573	0\\
574	0\\
575	0\\
576	0\\
577	0\\
578	0\\
579	0\\
580	0\\
581	0\\
582	0\\
583	0\\
584	0\\
585	0\\
586	0\\
587	0\\
588	0\\
589	0\\
590	0\\
591	0\\
592	0\\
593	0\\
594	0\\
595	0\\
596	0\\
597	0\\
598	0\\
599	0\\
600	0\\
};
\addplot [color=mycolor9,solid,forget plot]
  table[row sep=crcr]{%
1	0\\
2	0\\
3	0\\
4	0\\
5	0\\
6	0\\
7	0\\
8	0\\
9	0\\
10	0\\
11	0\\
12	0\\
13	0\\
14	0\\
15	0\\
16	0\\
17	0\\
18	0\\
19	0\\
20	0\\
21	0\\
22	0\\
23	0\\
24	0\\
25	0\\
26	0\\
27	0\\
28	0\\
29	0\\
30	0\\
31	0\\
32	0\\
33	0\\
34	0\\
35	0\\
36	0\\
37	0\\
38	0\\
39	0\\
40	0\\
41	0\\
42	0\\
43	0\\
44	0\\
45	0\\
46	0\\
47	0\\
48	0\\
49	0\\
50	0\\
51	0\\
52	0\\
53	0\\
54	0\\
55	0\\
56	0\\
57	0\\
58	0\\
59	0\\
60	0\\
61	0\\
62	0\\
63	0\\
64	0\\
65	0\\
66	0\\
67	0\\
68	0\\
69	0\\
70	0\\
71	0\\
72	0\\
73	0\\
74	0\\
75	0\\
76	0\\
77	0\\
78	0\\
79	0\\
80	0\\
81	0\\
82	0\\
83	0\\
84	0\\
85	0\\
86	0\\
87	0\\
88	0\\
89	0\\
90	0\\
91	0\\
92	0\\
93	0\\
94	0\\
95	0\\
96	0\\
97	0\\
98	0\\
99	0\\
100	0\\
101	0\\
102	0\\
103	0\\
104	0\\
105	0\\
106	0\\
107	0\\
108	0\\
109	0\\
110	0\\
111	0\\
112	0\\
113	0\\
114	0\\
115	0\\
116	0\\
117	0\\
118	0\\
119	0\\
120	0\\
121	0\\
122	0\\
123	0\\
124	0\\
125	0\\
126	0\\
127	0\\
128	0\\
129	0\\
130	0\\
131	0\\
132	0\\
133	0\\
134	0\\
135	0\\
136	0\\
137	0\\
138	0\\
139	0\\
140	0\\
141	0\\
142	0\\
143	0\\
144	0\\
145	0\\
146	0\\
147	0\\
148	0\\
149	0\\
150	0\\
151	0\\
152	0\\
153	0\\
154	0\\
155	0\\
156	0\\
157	0\\
158	0\\
159	0\\
160	0\\
161	0\\
162	0\\
163	0\\
164	0\\
165	0\\
166	0\\
167	0\\
168	0\\
169	0\\
170	0\\
171	0\\
172	0\\
173	0\\
174	0\\
175	0\\
176	0\\
177	0\\
178	0\\
179	0\\
180	0\\
181	0\\
182	0\\
183	0\\
184	0\\
185	0\\
186	0\\
187	0\\
188	0\\
189	0\\
190	0\\
191	0\\
192	0\\
193	0\\
194	0\\
195	0\\
196	0\\
197	0\\
198	0\\
199	0\\
200	0\\
201	0\\
202	0\\
203	0\\
204	0\\
205	0\\
206	0\\
207	0\\
208	0\\
209	0\\
210	0\\
211	0\\
212	0\\
213	0\\
214	0\\
215	0\\
216	0\\
217	0\\
218	0\\
219	0\\
220	0\\
221	0\\
222	0\\
223	0\\
224	0\\
225	0\\
226	0\\
227	0\\
228	0\\
229	0\\
230	0\\
231	0\\
232	0\\
233	0\\
234	0\\
235	0\\
236	0\\
237	0\\
238	0\\
239	0\\
240	0\\
241	0\\
242	0\\
243	0\\
244	0\\
245	0\\
246	0\\
247	0\\
248	0\\
249	0\\
250	0\\
251	0\\
252	0\\
253	0\\
254	0\\
255	0\\
256	0\\
257	0\\
258	0\\
259	0\\
260	0\\
261	0\\
262	0\\
263	0\\
264	0\\
265	0\\
266	0\\
267	0\\
268	0\\
269	0\\
270	0\\
271	0\\
272	0\\
273	0\\
274	0\\
275	0\\
276	0\\
277	0\\
278	0\\
279	0\\
280	0\\
281	0\\
282	0\\
283	0\\
284	0\\
285	0\\
286	0\\
287	0\\
288	0\\
289	0\\
290	0\\
291	0\\
292	0\\
293	0\\
294	0\\
295	0\\
296	0\\
297	0\\
298	0\\
299	0\\
300	0\\
301	0\\
302	0\\
303	0\\
304	0\\
305	0\\
306	0\\
307	0\\
308	0\\
309	0\\
310	0\\
311	0\\
312	0\\
313	0\\
314	0\\
315	0\\
316	0\\
317	0\\
318	0\\
319	0\\
320	0\\
321	0\\
322	0\\
323	0\\
324	0\\
325	0\\
326	0\\
327	0\\
328	0\\
329	0\\
330	0\\
331	0\\
332	0\\
333	0\\
334	0\\
335	0\\
336	0\\
337	0\\
338	0\\
339	0\\
340	0\\
341	0\\
342	0\\
343	0\\
344	0\\
345	0\\
346	0\\
347	0\\
348	0\\
349	0\\
350	0\\
351	0\\
352	0\\
353	0\\
354	0\\
355	0\\
356	0\\
357	0\\
358	0\\
359	0\\
360	0\\
361	0\\
362	0\\
363	0\\
364	0\\
365	0\\
366	0\\
367	0\\
368	0\\
369	0\\
370	0\\
371	0\\
372	0\\
373	0\\
374	0\\
375	0\\
376	0\\
377	0\\
378	0\\
379	0\\
380	0\\
381	0\\
382	0\\
383	0\\
384	0\\
385	0\\
386	0\\
387	0\\
388	0\\
389	0\\
390	0\\
391	0\\
392	0\\
393	0\\
394	0\\
395	0\\
396	0\\
397	0\\
398	0\\
399	0\\
400	0\\
401	0\\
402	0\\
403	0\\
404	0\\
405	0\\
406	0\\
407	0\\
408	0\\
409	0\\
410	0\\
411	0\\
412	0\\
413	0\\
414	0\\
415	0\\
416	0\\
417	0\\
418	0\\
419	0\\
420	0\\
421	0\\
422	0\\
423	0\\
424	0\\
425	0\\
426	0\\
427	0\\
428	0\\
429	0\\
430	0\\
431	0\\
432	0\\
433	0\\
434	0\\
435	0\\
436	0\\
437	0\\
438	0\\
439	0\\
440	0\\
441	0\\
442	0\\
443	0\\
444	0\\
445	0\\
446	0\\
447	0\\
448	0\\
449	0\\
450	0\\
451	0\\
452	0\\
453	0\\
454	0\\
455	0\\
456	0\\
457	0\\
458	0\\
459	0\\
460	0\\
461	0\\
462	0\\
463	0\\
464	0\\
465	0\\
466	0\\
467	0\\
468	0\\
469	0\\
470	0\\
471	0\\
472	0\\
473	0\\
474	0\\
475	0\\
476	0\\
477	0\\
478	0\\
479	0\\
480	0\\
481	0\\
482	0\\
483	0\\
484	0\\
485	0\\
486	0\\
487	0\\
488	0\\
489	0\\
490	0\\
491	0\\
492	0\\
493	0\\
494	0\\
495	0\\
496	0\\
497	0\\
498	0\\
499	0\\
500	0\\
501	0\\
502	0\\
503	0\\
504	0\\
505	0\\
506	0\\
507	0\\
508	0\\
509	0\\
510	0\\
511	0\\
512	0\\
513	0\\
514	0\\
515	0\\
516	0\\
517	0\\
518	0\\
519	0\\
520	0\\
521	0\\
522	0\\
523	0\\
524	0\\
525	0\\
526	0\\
527	0\\
528	0\\
529	0\\
530	0\\
531	0\\
532	0\\
533	0\\
534	0\\
535	0\\
536	0\\
537	0\\
538	0\\
539	0\\
540	0\\
541	0\\
542	0\\
543	0\\
544	0\\
545	0\\
546	0\\
547	0\\
548	0\\
549	0\\
550	0\\
551	0\\
552	0\\
553	0\\
554	0\\
555	0\\
556	0\\
557	0\\
558	0\\
559	0\\
560	0\\
561	0\\
562	0\\
563	0\\
564	0\\
565	0\\
566	0\\
567	0\\
568	0\\
569	0\\
570	0\\
571	0\\
572	0\\
573	0\\
574	0\\
575	0\\
576	0\\
577	0\\
578	0\\
579	0\\
580	0\\
581	0\\
582	0\\
583	0\\
584	0\\
585	0\\
586	0\\
587	0\\
588	0\\
589	0\\
590	0\\
591	0\\
592	0\\
593	0\\
594	0\\
595	0\\
596	0\\
597	0\\
598	0\\
599	0\\
600	0\\
};
\addplot [color=blue!50!mycolor7,solid,forget plot]
  table[row sep=crcr]{%
1	0\\
2	0\\
3	0\\
4	0\\
5	0\\
6	0\\
7	0\\
8	0\\
9	0\\
10	0\\
11	0\\
12	0\\
13	0\\
14	0\\
15	0\\
16	0\\
17	0\\
18	0\\
19	0\\
20	0\\
21	0\\
22	0\\
23	0\\
24	0\\
25	0\\
26	0\\
27	0\\
28	0\\
29	0\\
30	0\\
31	0\\
32	0\\
33	0\\
34	0\\
35	0\\
36	0\\
37	0\\
38	0\\
39	0\\
40	0\\
41	0\\
42	0\\
43	0\\
44	0\\
45	0\\
46	0\\
47	0\\
48	0\\
49	0\\
50	0\\
51	0\\
52	0\\
53	0\\
54	0\\
55	0\\
56	0\\
57	0\\
58	0\\
59	0\\
60	0\\
61	0\\
62	0\\
63	0\\
64	0\\
65	0\\
66	0\\
67	0\\
68	0\\
69	0\\
70	0\\
71	0\\
72	0\\
73	0\\
74	0\\
75	0\\
76	0\\
77	0\\
78	0\\
79	0\\
80	0\\
81	0\\
82	0\\
83	0\\
84	0\\
85	0\\
86	0\\
87	0\\
88	0\\
89	0\\
90	0\\
91	0\\
92	0\\
93	0\\
94	0\\
95	0\\
96	0\\
97	0\\
98	0\\
99	0\\
100	0\\
101	0\\
102	0\\
103	0\\
104	0\\
105	0\\
106	0\\
107	0\\
108	0\\
109	0\\
110	0\\
111	0\\
112	0\\
113	0\\
114	0\\
115	0\\
116	0\\
117	0\\
118	0\\
119	0\\
120	0\\
121	0\\
122	0\\
123	0\\
124	0\\
125	0\\
126	0\\
127	0\\
128	0\\
129	0\\
130	0\\
131	0\\
132	0\\
133	0\\
134	0\\
135	0\\
136	0\\
137	0\\
138	0\\
139	0\\
140	0\\
141	0\\
142	0\\
143	0\\
144	0\\
145	0\\
146	0\\
147	0\\
148	0\\
149	0\\
150	0\\
151	0\\
152	0\\
153	0\\
154	0\\
155	0\\
156	0\\
157	0\\
158	0\\
159	0\\
160	0\\
161	0\\
162	0\\
163	0\\
164	0\\
165	0\\
166	0\\
167	0\\
168	0\\
169	0\\
170	0\\
171	0\\
172	0\\
173	0\\
174	0\\
175	0\\
176	0\\
177	0\\
178	0\\
179	0\\
180	0\\
181	0\\
182	0\\
183	0\\
184	0\\
185	0\\
186	0\\
187	0\\
188	0\\
189	0\\
190	0\\
191	0\\
192	0\\
193	0\\
194	0\\
195	0\\
196	0\\
197	0\\
198	0\\
199	0\\
200	0\\
201	0\\
202	0\\
203	0\\
204	0\\
205	0\\
206	0\\
207	0\\
208	0\\
209	0\\
210	0\\
211	0\\
212	0\\
213	0\\
214	0\\
215	0\\
216	0\\
217	0\\
218	0\\
219	0\\
220	0\\
221	0\\
222	0\\
223	0\\
224	0\\
225	0\\
226	0\\
227	0\\
228	0\\
229	0\\
230	0\\
231	0\\
232	0\\
233	0\\
234	0\\
235	0\\
236	0\\
237	0\\
238	0\\
239	0\\
240	0\\
241	0\\
242	0\\
243	0\\
244	0\\
245	0\\
246	0\\
247	0\\
248	0\\
249	0\\
250	0\\
251	0\\
252	0\\
253	0\\
254	0\\
255	0\\
256	0\\
257	0\\
258	0\\
259	0\\
260	0\\
261	0\\
262	0\\
263	0\\
264	0\\
265	0\\
266	0\\
267	0\\
268	0\\
269	0\\
270	0\\
271	0\\
272	0\\
273	0\\
274	0\\
275	0\\
276	0\\
277	0\\
278	0\\
279	0\\
280	0\\
281	0\\
282	0\\
283	0\\
284	0\\
285	0\\
286	0\\
287	0\\
288	0\\
289	0\\
290	0\\
291	0\\
292	0\\
293	0\\
294	0\\
295	0\\
296	0\\
297	0\\
298	0\\
299	0\\
300	0\\
301	0\\
302	0\\
303	0\\
304	0\\
305	0\\
306	0\\
307	0\\
308	0\\
309	0\\
310	0\\
311	0\\
312	0\\
313	0\\
314	0\\
315	0\\
316	0\\
317	0\\
318	0\\
319	0\\
320	0\\
321	0\\
322	0\\
323	0\\
324	0\\
325	0\\
326	0\\
327	0\\
328	0\\
329	0\\
330	0\\
331	0\\
332	0\\
333	0\\
334	0\\
335	0\\
336	0\\
337	0\\
338	0\\
339	0\\
340	0\\
341	0\\
342	0\\
343	0\\
344	0\\
345	0\\
346	0\\
347	0\\
348	0\\
349	0\\
350	0\\
351	0\\
352	0\\
353	0\\
354	0\\
355	0\\
356	0\\
357	0\\
358	0\\
359	0\\
360	0\\
361	0\\
362	0\\
363	0\\
364	0\\
365	0\\
366	0\\
367	0\\
368	0\\
369	0\\
370	0\\
371	0\\
372	0\\
373	0\\
374	0\\
375	0\\
376	0\\
377	0\\
378	0\\
379	0\\
380	0\\
381	0\\
382	0\\
383	0\\
384	0\\
385	0\\
386	0\\
387	0\\
388	0\\
389	0\\
390	0\\
391	0\\
392	0\\
393	0\\
394	0\\
395	0\\
396	0\\
397	0\\
398	0\\
399	0\\
400	0\\
401	0\\
402	0\\
403	0\\
404	0\\
405	0\\
406	0\\
407	0\\
408	0\\
409	0\\
410	0\\
411	0\\
412	0\\
413	0\\
414	0\\
415	0\\
416	0\\
417	0\\
418	0\\
419	0\\
420	0\\
421	0\\
422	0\\
423	0\\
424	0\\
425	0\\
426	0\\
427	0\\
428	0\\
429	0\\
430	0\\
431	0\\
432	0\\
433	0\\
434	0\\
435	0\\
436	0\\
437	0\\
438	0\\
439	0\\
440	0\\
441	0\\
442	0\\
443	0\\
444	0\\
445	0\\
446	0\\
447	0\\
448	0\\
449	0\\
450	0\\
451	0\\
452	0\\
453	0\\
454	0\\
455	0\\
456	0\\
457	0\\
458	0\\
459	0\\
460	0\\
461	0\\
462	0\\
463	0\\
464	0\\
465	0\\
466	0\\
467	0\\
468	0\\
469	0\\
470	0\\
471	0\\
472	0\\
473	0\\
474	0\\
475	0\\
476	0\\
477	0\\
478	0\\
479	0\\
480	0\\
481	0\\
482	0\\
483	0\\
484	0\\
485	0\\
486	0\\
487	0\\
488	0\\
489	0\\
490	0\\
491	0\\
492	0\\
493	0\\
494	0\\
495	0\\
496	0\\
497	0\\
498	0\\
499	0\\
500	0\\
501	0\\
502	0\\
503	0\\
504	0\\
505	0\\
506	0\\
507	0\\
508	0\\
509	0\\
510	0\\
511	0\\
512	0\\
513	0\\
514	0\\
515	0\\
516	0\\
517	0\\
518	0\\
519	0\\
520	0\\
521	0\\
522	0\\
523	0\\
524	0\\
525	0\\
526	0\\
527	0\\
528	0\\
529	0\\
530	0\\
531	0\\
532	0\\
533	0\\
534	0\\
535	0\\
536	0\\
537	0\\
538	0\\
539	0\\
540	0\\
541	0\\
542	0\\
543	0\\
544	0\\
545	0\\
546	0\\
547	0\\
548	0\\
549	0\\
550	0\\
551	0\\
552	0\\
553	0\\
554	0\\
555	0\\
556	0\\
557	0\\
558	0\\
559	0\\
560	0\\
561	0\\
562	0\\
563	0\\
564	0\\
565	0\\
566	0\\
567	0\\
568	0\\
569	0\\
570	0\\
571	0\\
572	0\\
573	0\\
574	0\\
575	0\\
576	0\\
577	0\\
578	0\\
579	0\\
580	0\\
581	0\\
582	0\\
583	0\\
584	0\\
585	0\\
586	0\\
587	0\\
588	0\\
589	0\\
590	0\\
591	0\\
592	0\\
593	0\\
594	0\\
595	0\\
596	0\\
597	0\\
598	0\\
599	0\\
600	0\\
};
\addplot [color=blue!40!mycolor9,solid,forget plot]
  table[row sep=crcr]{%
1	0\\
2	0\\
3	0\\
4	0\\
5	0\\
6	0\\
7	0\\
8	0\\
9	0\\
10	0\\
11	0\\
12	0\\
13	0\\
14	0\\
15	0\\
16	0\\
17	0\\
18	0\\
19	0\\
20	0\\
21	0\\
22	0\\
23	0\\
24	0\\
25	0\\
26	0\\
27	0\\
28	0\\
29	0\\
30	0\\
31	0\\
32	0\\
33	0\\
34	0\\
35	0\\
36	0\\
37	0\\
38	0\\
39	0\\
40	0\\
41	0\\
42	0\\
43	0\\
44	0\\
45	0\\
46	0\\
47	0\\
48	0\\
49	0\\
50	0\\
51	0\\
52	0\\
53	0\\
54	0\\
55	0\\
56	0\\
57	0\\
58	0\\
59	0\\
60	0\\
61	0\\
62	0\\
63	0\\
64	0\\
65	0\\
66	0\\
67	0\\
68	0\\
69	0\\
70	0\\
71	0\\
72	0\\
73	0\\
74	0\\
75	0\\
76	0\\
77	0\\
78	0\\
79	0\\
80	0\\
81	0\\
82	0\\
83	0\\
84	0\\
85	0\\
86	0\\
87	0\\
88	0\\
89	0\\
90	0\\
91	0\\
92	0\\
93	0\\
94	0\\
95	0\\
96	0\\
97	0\\
98	0\\
99	0\\
100	0\\
101	0\\
102	0\\
103	0\\
104	0\\
105	0\\
106	0\\
107	0\\
108	0\\
109	0\\
110	0\\
111	0\\
112	0\\
113	0\\
114	0\\
115	0\\
116	0\\
117	0\\
118	0\\
119	0\\
120	0\\
121	0\\
122	0\\
123	0\\
124	0\\
125	0\\
126	0\\
127	0\\
128	0\\
129	0\\
130	0\\
131	0\\
132	0\\
133	0\\
134	0\\
135	0\\
136	0\\
137	0\\
138	0\\
139	0\\
140	0\\
141	0\\
142	0\\
143	0\\
144	0\\
145	0\\
146	0\\
147	0\\
148	0\\
149	0\\
150	0\\
151	0\\
152	0\\
153	0\\
154	0\\
155	0\\
156	0\\
157	0\\
158	0\\
159	0\\
160	0\\
161	0\\
162	0\\
163	0\\
164	0\\
165	0\\
166	0\\
167	0\\
168	0\\
169	0\\
170	0\\
171	0\\
172	0\\
173	0\\
174	0\\
175	0\\
176	0\\
177	0\\
178	0\\
179	0\\
180	0\\
181	0\\
182	0\\
183	0\\
184	0\\
185	0\\
186	0\\
187	0\\
188	0\\
189	0\\
190	0\\
191	0\\
192	0\\
193	0\\
194	0\\
195	0\\
196	0\\
197	0\\
198	0\\
199	0\\
200	0\\
201	0\\
202	0\\
203	0\\
204	0\\
205	0\\
206	0\\
207	0\\
208	0\\
209	0\\
210	0\\
211	0\\
212	0\\
213	0\\
214	0\\
215	0\\
216	0\\
217	0\\
218	0\\
219	0\\
220	0\\
221	0\\
222	0\\
223	0\\
224	0\\
225	0\\
226	0\\
227	0\\
228	0\\
229	0\\
230	0\\
231	0\\
232	0\\
233	0\\
234	0\\
235	0\\
236	0\\
237	0\\
238	0\\
239	0\\
240	0\\
241	0\\
242	0\\
243	0\\
244	0\\
245	0\\
246	0\\
247	0\\
248	0\\
249	0\\
250	0\\
251	0\\
252	0\\
253	0\\
254	0\\
255	0\\
256	0\\
257	0\\
258	0\\
259	0\\
260	0\\
261	0\\
262	0\\
263	0\\
264	0\\
265	0\\
266	0\\
267	0\\
268	0\\
269	0\\
270	0\\
271	0\\
272	0\\
273	0\\
274	0\\
275	0\\
276	0\\
277	0\\
278	0\\
279	0\\
280	0\\
281	0\\
282	0\\
283	0\\
284	0\\
285	0\\
286	0\\
287	0\\
288	0\\
289	0\\
290	0\\
291	0\\
292	0\\
293	0\\
294	0\\
295	0\\
296	0\\
297	0\\
298	0\\
299	0\\
300	0\\
301	0\\
302	0\\
303	0\\
304	0\\
305	0\\
306	0\\
307	0\\
308	0\\
309	0\\
310	0\\
311	0\\
312	0\\
313	0\\
314	0\\
315	0\\
316	0\\
317	0\\
318	0\\
319	0\\
320	0\\
321	0\\
322	0\\
323	0\\
324	0\\
325	0\\
326	0\\
327	0\\
328	0\\
329	0\\
330	0\\
331	0\\
332	0\\
333	0\\
334	0\\
335	0\\
336	0\\
337	0\\
338	0\\
339	0\\
340	0\\
341	0\\
342	0\\
343	0\\
344	0\\
345	0\\
346	0\\
347	0\\
348	0\\
349	0\\
350	0\\
351	0\\
352	0\\
353	0\\
354	0\\
355	0\\
356	0\\
357	0\\
358	0\\
359	0\\
360	0\\
361	0\\
362	0\\
363	0\\
364	0\\
365	0\\
366	0\\
367	0\\
368	0\\
369	0\\
370	0\\
371	0\\
372	0\\
373	0\\
374	0\\
375	0\\
376	0\\
377	0\\
378	0\\
379	0\\
380	0\\
381	0\\
382	0\\
383	0\\
384	0\\
385	0\\
386	0\\
387	0\\
388	0\\
389	0\\
390	0\\
391	0\\
392	0\\
393	0\\
394	0\\
395	0\\
396	0\\
397	0\\
398	0\\
399	0\\
400	0\\
401	0\\
402	0\\
403	0\\
404	0\\
405	0\\
406	0\\
407	0\\
408	0\\
409	0\\
410	0\\
411	0\\
412	0\\
413	0\\
414	0\\
415	0\\
416	0\\
417	0\\
418	0\\
419	0\\
420	0\\
421	0\\
422	0\\
423	0\\
424	0\\
425	0\\
426	0\\
427	0\\
428	0\\
429	0\\
430	0\\
431	0\\
432	0\\
433	0\\
434	0\\
435	0\\
436	0\\
437	0\\
438	0\\
439	0\\
440	0\\
441	0\\
442	0\\
443	0\\
444	0\\
445	0\\
446	0\\
447	0\\
448	0\\
449	0\\
450	0\\
451	0\\
452	0\\
453	0\\
454	0\\
455	0\\
456	0\\
457	0\\
458	0\\
459	0\\
460	0\\
461	0\\
462	0\\
463	0\\
464	0\\
465	0\\
466	0\\
467	0\\
468	0\\
469	0\\
470	0\\
471	0\\
472	0\\
473	0\\
474	0\\
475	0\\
476	0\\
477	0\\
478	0\\
479	0\\
480	0\\
481	0\\
482	0\\
483	0\\
484	0\\
485	0\\
486	0\\
487	0\\
488	0\\
489	0\\
490	0\\
491	0\\
492	0\\
493	0\\
494	0\\
495	0\\
496	0\\
497	0\\
498	0\\
499	0\\
500	0\\
501	0\\
502	0\\
503	0\\
504	0\\
505	0\\
506	0\\
507	0\\
508	0\\
509	0\\
510	0\\
511	0\\
512	0\\
513	0\\
514	0\\
515	0\\
516	0\\
517	0\\
518	0\\
519	0\\
520	0\\
521	0\\
522	0\\
523	0\\
524	0\\
525	0\\
526	0\\
527	0\\
528	0\\
529	0\\
530	0\\
531	0\\
532	0\\
533	0\\
534	0\\
535	0\\
536	0\\
537	0\\
538	0\\
539	0\\
540	0\\
541	0\\
542	0\\
543	0\\
544	0\\
545	0\\
546	0\\
547	0\\
548	0\\
549	0\\
550	0\\
551	0\\
552	0\\
553	0\\
554	0\\
555	0\\
556	0\\
557	0\\
558	0\\
559	0\\
560	0\\
561	0\\
562	0\\
563	0\\
564	0\\
565	0\\
566	0\\
567	0\\
568	0\\
569	0\\
570	0\\
571	0\\
572	0\\
573	0\\
574	0\\
575	0\\
576	0\\
577	0\\
578	0\\
579	0\\
580	0\\
581	0\\
582	0\\
583	0\\
584	0\\
585	0\\
586	0\\
587	0\\
588	0\\
589	0\\
590	0\\
591	0\\
592	0\\
593	0\\
594	0\\
595	0\\
596	0\\
597	0\\
598	0\\
599	0\\
600	0\\
};
\addplot [color=blue!75!mycolor7,solid,forget plot]
  table[row sep=crcr]{%
1	0\\
2	0\\
3	0\\
4	0\\
5	0\\
6	0\\
7	0\\
8	0\\
9	0\\
10	0\\
11	0\\
12	0\\
13	0\\
14	0\\
15	0\\
16	0\\
17	0\\
18	0\\
19	0\\
20	0\\
21	0\\
22	0\\
23	0\\
24	0\\
25	0\\
26	0\\
27	0\\
28	0\\
29	0\\
30	0\\
31	0\\
32	0\\
33	0\\
34	0\\
35	0\\
36	0\\
37	0\\
38	0\\
39	0\\
40	0\\
41	0\\
42	0\\
43	0\\
44	0\\
45	0\\
46	0\\
47	0\\
48	0\\
49	0\\
50	0\\
51	0\\
52	0\\
53	0\\
54	0\\
55	0\\
56	0\\
57	0\\
58	0\\
59	0\\
60	0\\
61	0\\
62	0\\
63	0\\
64	0\\
65	0\\
66	0\\
67	0\\
68	0\\
69	0\\
70	0\\
71	0\\
72	0\\
73	0\\
74	0\\
75	0\\
76	0\\
77	0\\
78	0\\
79	0\\
80	0\\
81	0\\
82	0\\
83	0\\
84	0\\
85	0\\
86	0\\
87	0\\
88	0\\
89	0\\
90	0\\
91	0\\
92	0\\
93	0\\
94	0\\
95	0\\
96	0\\
97	0\\
98	0\\
99	0\\
100	0\\
101	0\\
102	0\\
103	0\\
104	0\\
105	0\\
106	0\\
107	0\\
108	0\\
109	0\\
110	0\\
111	0\\
112	0\\
113	0\\
114	0\\
115	0\\
116	0\\
117	0\\
118	0\\
119	0\\
120	0\\
121	0\\
122	0\\
123	0\\
124	0\\
125	0\\
126	0\\
127	0\\
128	0\\
129	0\\
130	0\\
131	0\\
132	0\\
133	0\\
134	0\\
135	0\\
136	0\\
137	0\\
138	0\\
139	0\\
140	0\\
141	0\\
142	0\\
143	0\\
144	0\\
145	0\\
146	0\\
147	0\\
148	0\\
149	0\\
150	0\\
151	0\\
152	0\\
153	0\\
154	0\\
155	0\\
156	0\\
157	0\\
158	0\\
159	0\\
160	0\\
161	0\\
162	0\\
163	0\\
164	0\\
165	0\\
166	0\\
167	0\\
168	0\\
169	0\\
170	0\\
171	0\\
172	0\\
173	0\\
174	0\\
175	0\\
176	0\\
177	0\\
178	0\\
179	0\\
180	0\\
181	0\\
182	0\\
183	0\\
184	0\\
185	0\\
186	0\\
187	0\\
188	0\\
189	0\\
190	0\\
191	0\\
192	0\\
193	0\\
194	0\\
195	0\\
196	0\\
197	0\\
198	0\\
199	0\\
200	0\\
201	0\\
202	0\\
203	0\\
204	0\\
205	0\\
206	0\\
207	0\\
208	0\\
209	0\\
210	0\\
211	0\\
212	0\\
213	0\\
214	0\\
215	0\\
216	0\\
217	0\\
218	0\\
219	0\\
220	0\\
221	0\\
222	0\\
223	0\\
224	0\\
225	0\\
226	0\\
227	0\\
228	0\\
229	0\\
230	0\\
231	0\\
232	0\\
233	0\\
234	0\\
235	0\\
236	0\\
237	0\\
238	0\\
239	0\\
240	0\\
241	0\\
242	0\\
243	0\\
244	0\\
245	0\\
246	0\\
247	0\\
248	0\\
249	0\\
250	0\\
251	0\\
252	0\\
253	0\\
254	0\\
255	0\\
256	0\\
257	0\\
258	0\\
259	0\\
260	0\\
261	0\\
262	0\\
263	0\\
264	0\\
265	0\\
266	0\\
267	0\\
268	0\\
269	0\\
270	0\\
271	0\\
272	0\\
273	0\\
274	0\\
275	0\\
276	0\\
277	0\\
278	0\\
279	0\\
280	0\\
281	0\\
282	0\\
283	0\\
284	0\\
285	0\\
286	0\\
287	0\\
288	0\\
289	0\\
290	0\\
291	0\\
292	0\\
293	0\\
294	0\\
295	0\\
296	0\\
297	0\\
298	0\\
299	0\\
300	0\\
301	0\\
302	0\\
303	0\\
304	0\\
305	0\\
306	0\\
307	0\\
308	0\\
309	0\\
310	0\\
311	0\\
312	0\\
313	0\\
314	0\\
315	0\\
316	0\\
317	0\\
318	0\\
319	0\\
320	0\\
321	0\\
322	0\\
323	0\\
324	0\\
325	0\\
326	0\\
327	0\\
328	0\\
329	0\\
330	0\\
331	0\\
332	0\\
333	0\\
334	0\\
335	0\\
336	0\\
337	0\\
338	0\\
339	0\\
340	0\\
341	0\\
342	0\\
343	0\\
344	0\\
345	0\\
346	0\\
347	0\\
348	0\\
349	0\\
350	0\\
351	0\\
352	0\\
353	0\\
354	0\\
355	0\\
356	0\\
357	0\\
358	0\\
359	0\\
360	0\\
361	0\\
362	0\\
363	0\\
364	0\\
365	0\\
366	0\\
367	0\\
368	0\\
369	0\\
370	0\\
371	0\\
372	0\\
373	0\\
374	0\\
375	0\\
376	0\\
377	0\\
378	0\\
379	0\\
380	0\\
381	0\\
382	0\\
383	0\\
384	0\\
385	0\\
386	0\\
387	0\\
388	0\\
389	0\\
390	0\\
391	0\\
392	0\\
393	0\\
394	0\\
395	0\\
396	0\\
397	0\\
398	0\\
399	0\\
400	0\\
401	0\\
402	0\\
403	0\\
404	0\\
405	0\\
406	0\\
407	0\\
408	0\\
409	0\\
410	0\\
411	0\\
412	0\\
413	0\\
414	0\\
415	0\\
416	0\\
417	0\\
418	0\\
419	0\\
420	0\\
421	0\\
422	0\\
423	0\\
424	0\\
425	0\\
426	0\\
427	0\\
428	0\\
429	0\\
430	0\\
431	0\\
432	0\\
433	0\\
434	0\\
435	0\\
436	0\\
437	0\\
438	0\\
439	0\\
440	0\\
441	0\\
442	0\\
443	0\\
444	0\\
445	0\\
446	0\\
447	0\\
448	0\\
449	0\\
450	0\\
451	0\\
452	0\\
453	0\\
454	0\\
455	0\\
456	0\\
457	0\\
458	0\\
459	0\\
460	0\\
461	0\\
462	0\\
463	0\\
464	0\\
465	0\\
466	0\\
467	0\\
468	0\\
469	0\\
470	0\\
471	0\\
472	0\\
473	0\\
474	0\\
475	0\\
476	0\\
477	0\\
478	0\\
479	0\\
480	0\\
481	0\\
482	0\\
483	0\\
484	0\\
485	0\\
486	0\\
487	0\\
488	0\\
489	0\\
490	0\\
491	0\\
492	0\\
493	0\\
494	0\\
495	0\\
496	0\\
497	0\\
498	0\\
499	0\\
500	0\\
501	0\\
502	0\\
503	0\\
504	0\\
505	0\\
506	0\\
507	0\\
508	0\\
509	0\\
510	0\\
511	0\\
512	0\\
513	0\\
514	0\\
515	0\\
516	0\\
517	0\\
518	0\\
519	0\\
520	0\\
521	0\\
522	0\\
523	0\\
524	0\\
525	0\\
526	0\\
527	0\\
528	0\\
529	0\\
530	0\\
531	0\\
532	0\\
533	0\\
534	0\\
535	0\\
536	0\\
537	0\\
538	0\\
539	0\\
540	0\\
541	0\\
542	0\\
543	0\\
544	0\\
545	0\\
546	0\\
547	0\\
548	0\\
549	0\\
550	0\\
551	0\\
552	0\\
553	0\\
554	0\\
555	0\\
556	0\\
557	0\\
558	0\\
559	0\\
560	0\\
561	0\\
562	0\\
563	0\\
564	0\\
565	0\\
566	0\\
567	0\\
568	0\\
569	0\\
570	0\\
571	0\\
572	0\\
573	0\\
574	0\\
575	0\\
576	0\\
577	0\\
578	0\\
579	0\\
580	0\\
581	0\\
582	0\\
583	0\\
584	0\\
585	0\\
586	0\\
587	0\\
588	0\\
589	0\\
590	0\\
591	0\\
592	0\\
593	0\\
594	0\\
595	0\\
596	0\\
597	0\\
598	0\\
599	0\\
600	0\\
};
\addplot [color=blue!80!mycolor9,solid,forget plot]
  table[row sep=crcr]{%
1	0.000420463240150569\\
2	0.00042044912030799\\
3	0.000420434759003375\\
4	0.000420420152089871\\
5	0.00042040529534938\\
6	0.000420390184491366\\
7	0.00042037481515161\\
8	0.00042035918289092\\
9	0.000420343283193896\\
10	0.000420327111467586\\
11	0.000420310663040172\\
12	0.000420293933159629\\
13	0.000420276916992343\\
14	0.000420259609621713\\
15	0.000420242006046732\\
16	0.00042022410118055\\
17	0.000420205889849007\\
18	0.000420187366789113\\
19	0.000420168526647563\\
20	0.000420149363979176\\
21	0.000420129873245317\\
22	0.000420110048812314\\
23	0.000420089884949808\\
24	0.000420069375829147\\
25	0.00042004851552164\\
26	0.00042002729799689\\
27	0.000420005717121061\\
28	0.000419983766655072\\
29	0.000419961440252826\\
30	0.000419938731459381\\
31	0.000419915633709058\\
32	0.0004198921403236\\
33	0.000419868244510194\\
34	0.000419843939359556\\
35	0.000419819217843911\\
36	0.000419794072814985\\
37	0.000419768497001938\\
38	0.000419742483009261\\
39	0.000419716023314677\\
40	0.000419689110266941\\
41	0.000419661736083651\\
42	0.000419633892849017\\
43	0.000419605572511561\\
44	0.00041957676688182\\
45	0.000419547467629971\\
46	0.000419517666283459\\
47	0.00041948735422453\\
48	0.000419456522687782\\
49	0.000419425162757618\\
50	0.000419393265365691\\
51	0.000419360821288323\\
52	0.000419327821143804\\
53	0.000419294255389739\\
54	0.000419260114320285\\
55	0.000419225388063368\\
56	0.000419190066577843\\
57	0.00041915413965062\\
58	0.000419117596893731\\
59	0.00041908042774133\\
60	0.000419042621446685\\
61	0.000419004167079077\\
62	0.000418965053520673\\
63	0.000418925269463327\\
64	0.000418884803405359\\
65	0.000418843643648231\\
66	0.000418801778293208\\
67	0.000418759195237954\\
68	0.000418715882173041\\
69	0.000418671826578452\\
70	0.000418627015719973\\
71	0.000418581436645575\\
72	0.000418535076181664\\
73	0.000418487920929352\\
74	0.000418439957260604\\
75	0.000418391171314359\\
76	0.000418341548992521\\
77	0.000418291075956001\\
78	0.000418239737620543\\
79	0.000418187519152627\\
80	0.000418134405465184\\
81	0.000418080381213302\\
82	0.000418025430789865\\
83	0.000417969538321074\\
84	0.000417912687661938\\
85	0.000417854862391662\\
86	0.000417796045808958\\
87	0.000417736220927311\\
88	0.000417675370470104\\
89	0.000417613476865731\\
90	0.00041755052224259\\
91	0.000417486488423966\\
92	0.000417421356922902\\
93	0.000417355108936909\\
94	0.000417287725342635\\
95	0.000417219186690436\\
96	0.000417149473198828\\
97	0.000417078564748892\\
98	0.000417006440878541\\
99	0.000416933080776717\\
100	0.000416858463277496\\
101	0.00041678256685407\\
102	0.000416705369612641\\
103	0.000416626849286235\\
104	0.000416546983228358\\
105	0.000416465748406587\\
106	0.000416383121396063\\
107	0.000416299078372831\\
108	0.000416213595107108\\
109	0.000416126646956417\\
110	0.000416038208858606\\
111	0.000415948255324737\\
112	0.000415856760431929\\
113	0.00041576369781595\\
114	0.000415669040663792\\
115	0.000415572761706077\\
116	0.000415474833209363\\
117	0.000415375226968243\\
118	0.000415273914297402\\
119	0.000415170866023496\\
120	0.000415066052476872\\
121	0.000414959443483184\\
122	0.000414851008354854\\
123	0.000414740715882372\\
124	0.000414628534325447\\
125	0.000414514431404022\\
126	0.000414398374289128\\
127	0.00041428032959356\\
128	0.000414160263362431\\
129	0.000414038141063515\\
130	0.000413913927577448\\
131	0.000413787587187775\\
132	0.000413659083570779\\
133	0.000413528379785166\\
134	0.000413395438261569\\
135	0.000413260220791854\\
136	0.000413122688518262\\
137	0.000412982801922331\\
138	0.000412840520813672\\
139	0.000412695804318507\\
140	0.000412548610868033\\
141	0.000412398898186595\\
142	0.000412246623279623\\
143	0.000412091742421269\\
144	0.000411934211141827\\
145	0.000411773984214908\\
146	0.00041161101564502\\
147	0.000411445258655372\\
148	0.000411276665673553\\
149	0.000411105188317988\\
150	0.000410930777384189\\
151	0.000410753382830757\\
152	0.000410572953765137\\
153	0.000410389438429166\\
154	0.00041020278418429\\
155	0.000410012937496649\\
156	0.00040981984392177\\
157	0.000409623448089111\\
158	0.000409423693686277\\
159	0.000409220523442977\\
160	0.000409013879114696\\
161	0.000408803701466113\\
162	0.000408589930254217\\
163	0.00040837250421111\\
164	0.000408151361026537\\
165	0.00040792643733012\\
166	0.00040769766867327\\
167	0.000407464989510785\\
168	0.000407228333182126\\
169	0.000406987631892374\\
170	0.000406742816692882\\
171	0.000406493817461535\\
172	0.0004062405628827\\
173	0.000405982980426843\\
174	0.000405720996329757\\
175	0.000405454535571414\\
176	0.000405183521854547\\
177	0.000404907877582701\\
178	0.000404627523838044\\
179	0.000404342380358698\\
180	0.000404052365515714\\
181	0.000403757396289626\\
182	0.000403457388246604\\
183	0.000403152255514178\\
184	0.000402841910756524\\
185	0.000402526265149358\\
186	0.000402205228354325\\
187	0.000401878708492977\\
188	0.000401546612120275\\
189	0.000401208844197626\\
190	0.00040086530806543\\
191	0.000400515905415176\\
192	0.000400160536260948\\
193	0.000399799098910555\\
194	0.000399431489936029\\
195	0.000399057604143654\\
196	0.000398677334543464\\
197	0.000398290572318125\\
198	0.000397897206791365\\
199	0.000397497125395721\\
200	0.000397090213639784\\
201	0.000396676355074823\\
202	0.000396255431260803\\
203	0.00039582732173178\\
204	0.000395391903960692\\
205	0.00039494905332346\\
206	0.000394498643062507\\
207	0.000394040544249511\\
208	0.000393574625747573\\
209	0.000393100754172621\\
210	0.000392618793854148\\
211	0.000392128606795174\\
212	0.000391630052631514\\
213	0.00039112298859028\\
214	0.000390607269447579\\
215	0.000390082747485471\\
216	0.000389549272448078\\
217	0.000389006691496911\\
218	0.000388454849165321\\
219	0.000387893587312119\\
220	0.000387322745074294\\
221	0.000386742158818885\\
222	0.000386151662093866\\
223	0.000385551085578178\\
224	0.000384940257030721\\
225	0.000384319001238465\\
226	0.000383687139963456\\
227	0.000383044491888938\\
228	0.00038239087256427\\
229	0.000381726094348952\\
230	0.000381049966355412\\
231	0.000380362294390771\\
232	0.00037966288089746\\
233	0.000378951524892641\\
234	0.000378228021906455\\
235	0.000377492163919073\\
236	0.000376743739296484\\
237	0.000375982532725003\\
238	0.000375208325144516\\
239	0.000374420893680352\\
240	0.000373620011573821\\
241	0.000372805448111359\\
242	0.000371976968552242\\
243	0.00037113433405487\\
244	0.000370277301601518\\
245	0.000369405623921625\\
246	0.00036851904941349\\
247	0.000367617322064392\\
248	0.000366700181369074\\
249	0.000365767362246593\\
250	0.000364818594955471\\
251	0.000363853605007071\\
252	0.000362872113077254\\
253	0.000361873834916212\\
254	0.000360858481256441\\
255	0.000359825757718859\\
256	0.000358775364716957\\
257	0.000357706997359041\\
258	0.000356620345348432\\
259	0.000355515092881627\\
260	0.000354390918544418\\
261	0.000353247495205839\\
262	0.000352084489909975\\
263	0.000350901563765589\\
264	0.000349698371833468\\
265	0.000348474563011527\\
266	0.000347229779917538\\
267	0.000345963658769561\\
268	0.000344675829263897\\
269	0.000343365914450595\\
270	0.000342033530606426\\
271	0.000340678287105166\\
272	0.000339299786285226\\
273	0.000337897623314623\\
274	0.000336471386054643\\
275	0.000335020654925319\\
276	0.000333545002775104\\
277	0.000332043994732079\\
278	0.000330517187994447\\
279	0.000328964131722927\\
280	0.000327384366888339\\
281	0.000325777426113161\\
282	0.000324142833509996\\
283	0.00032248010451688\\
284	0.000320788745729423\\
285	0.000319068254729721\\
286	0.000317318119912109\\
287	0.000315537820305653\\
288	0.000313726825393466\\
289	0.00031188459492877\\
290	0.000310010578747858\\
291	0.000308104216579848\\
292	0.00030616493785344\\
293	0.000304192161500579\\
294	0.000302185295757328\\
295	0.000300143737961838\\
296	0.000298066874349785\\
297	0.000295954079847361\\
298	0.000293804717861986\\
299	0.000291618140071164\\
300	0.00028939368620967\\
301	0.00028713068385552\\
302	0.000284828448215171\\
303	0.00028248628190848\\
304	0.000280103474754065\\
305	0.00027767930355581\\
306	0.000275213031891497\\
307	0.000272703909904661\\
308	0.000270151174100987\\
309	0.000267554047150566\\
310	0.000264911737695903\\
311	0.000262223440161519\\
312	0.000259488334553529\\
313	0.000256705586246752\\
314	0.00025387434585603\\
315	0.00025099374935309\\
316	0.000248062917857218\\
317	0.000245080957487294\\
318	0.00024204695931954\\
319	0.000238959999380692\\
320	0.00023581913868224\\
321	0.000232623423301923\\
322	0.00022937188451976\\
323	0.000226063539016568\\
324	0.000222697389144192\\
325	0.000219272423277816\\
326	0.000215787616262068\\
327	0.000212241929964235\\
328	0.000208634313949584\\
329	0.000204963706295905\\
330	0.000201229034566454\\
331	0.000197429216963232\\
332	0.000193563163685222\\
333	0.000189629778519623\\
334	0.000185627960697774\\
335	0.000181556607051661\\
336	0.000177414614511688\\
337	0.000173200882991794\\
338	0.000168914318714217\\
339	0.000164553838033065\\
340	0.000160118371823967\\
341	0.000155606870515935\\
342	0.000151018309852072\\
343	0.000146351697477284\\
344	0.000141606080464596\\
345	0.000136780553906929\\
346	0.000131874270718485\\
347	0.000126886452809903\\
348	0.000121816403823762\\
349	0.000116663523643309\\
350	0.000111427324916498\\
351	0.000106107451871358\\
352	0.000100703701736782\\
353	9.52160491260074e-05\\
354	8.96446737914473e-05\\
355	8.39899922321503e-05\\
356	7.82526937707919e-05\\
357	7.24337819585433e-05\\
358	6.65346219905049e-05\\
359	6.05569928351092e-05\\
360	5.45031451081293e-05\\
361	4.83758688722205e-05\\
362	4.21785594234741e-05\\
363	3.59152575781306e-05\\
364	2.95905654047414e-05\\
365	2.3209056409204e-05\\
366	1.67726731719349e-05\\
367	1.026963175326e-05\\
368	3.62063219975855e-06\\
369	0\\
370	0\\
371	0\\
372	0\\
373	0\\
374	0\\
375	0\\
376	0\\
377	0\\
378	0\\
379	0\\
380	0\\
381	0\\
382	0\\
383	0\\
384	0\\
385	0\\
386	0\\
387	0\\
388	0\\
389	0\\
390	0\\
391	0\\
392	0\\
393	0\\
394	0\\
395	0\\
396	0\\
397	0\\
398	0\\
399	0\\
400	0\\
401	0\\
402	0\\
403	0\\
404	0\\
405	0\\
406	0\\
407	0\\
408	0\\
409	0\\
410	0\\
411	0\\
412	0\\
413	0\\
414	0\\
415	0\\
416	0\\
417	0\\
418	0\\
419	0\\
420	0\\
421	0\\
422	0\\
423	0\\
424	0\\
425	0\\
426	0\\
427	0\\
428	0\\
429	0\\
430	0\\
431	0\\
432	0\\
433	0\\
434	0\\
435	0\\
436	0\\
437	0\\
438	0\\
439	0\\
440	0\\
441	0\\
442	0\\
443	0\\
444	0\\
445	0\\
446	0\\
447	0\\
448	0\\
449	0\\
450	0\\
451	0\\
452	0\\
453	0\\
454	0\\
455	0\\
456	0\\
457	0\\
458	0\\
459	0\\
460	0\\
461	0\\
462	0\\
463	0\\
464	0\\
465	0\\
466	0\\
467	0\\
468	0\\
469	0\\
470	0\\
471	0\\
472	0\\
473	0\\
474	0\\
475	0\\
476	0\\
477	0\\
478	0\\
479	0\\
480	0\\
481	0\\
482	0\\
483	0\\
484	0\\
485	0\\
486	0\\
487	0\\
488	0\\
489	0\\
490	0\\
491	0\\
492	0\\
493	0\\
494	0\\
495	0\\
496	0\\
497	0\\
498	0\\
499	0\\
500	0\\
501	0\\
502	0\\
503	0\\
504	0\\
505	0\\
506	0\\
507	0\\
508	0\\
509	0\\
510	0\\
511	0\\
512	0\\
513	0\\
514	0\\
515	0\\
516	0\\
517	0\\
518	0\\
519	0\\
520	0\\
521	0\\
522	0\\
523	0\\
524	0\\
525	0\\
526	0\\
527	0\\
528	0\\
529	0\\
530	0\\
531	0\\
532	0\\
533	0\\
534	0\\
535	0\\
536	0\\
537	0\\
538	0\\
539	0\\
540	0\\
541	0\\
542	0\\
543	0\\
544	0\\
545	0\\
546	0\\
547	0\\
548	0\\
549	0\\
550	0\\
551	0\\
552	0\\
553	0\\
554	0\\
555	0\\
556	0\\
557	0\\
558	0\\
559	0\\
560	0\\
561	0\\
562	0\\
563	0\\
564	0\\
565	0\\
566	0\\
567	0\\
568	0\\
569	0\\
570	0\\
571	0\\
572	0\\
573	0\\
574	0\\
575	0\\
576	0\\
577	0\\
578	0\\
579	0\\
580	0\\
581	0\\
582	0\\
583	0\\
584	0\\
585	0\\
586	0\\
587	0\\
588	0\\
589	0\\
590	0\\
591	0\\
592	0\\
593	0\\
594	0\\
595	0\\
596	0\\
597	0\\
598	0\\
599	0\\
600	0\\
};
\addplot [color=blue,solid,forget plot]
  table[row sep=crcr]{%
1	0.00225547679368458\\
2	0.00225546436267677\\
3	0.0022554517189349\\
4	0.00225543885880797\\
5	0.00225542577858241\\
6	0.00225541247448096\\
7	0.00225539894266163\\
8	0.00225538517921661\\
9	0.00225537118017108\\
10	0.00225535694148212\\
11	0.00225534245903754\\
12	0.00225532772865468\\
13	0.00225531274607924\\
14	0.00225529750698401\\
15	0.00225528200696767\\
16	0.00225526624155352\\
17	0.00225525020618817\\
18	0.00225523389624026\\
19	0.00225521730699912\\
20	0.00225520043367341\\
21	0.00225518327138976\\
22	0.00225516581519137\\
23	0.00225514806003657\\
24	0.0022551300007974\\
25	0.00225511163225812\\
26	0.00225509294911375\\
27	0.00225507394596848\\
28	0.00225505461733422\\
29	0.00225503495762892\\
30	0.00225501496117508\\
31	0.00225499462219804\\
32	0.00225497393482437\\
33	0.00225495289308018\\
34	0.00225493149088941\\
35	0.00225490972207209\\
36	0.00225488758034257\\
37	0.00225486505930775\\
38	0.00225484215246521\\
39	0.00225481885320136\\
40	0.00225479515478961\\
41	0.00225477105038835\\
42	0.00225474653303909\\
43	0.00225472159566441\\
44	0.00225469623106597\\
45	0.00225467043192244\\
46	0.00225464419078744\\
47	0.00225461750008738\\
48	0.00225459035211933\\
49	0.00225456273904881\\
50	0.00225453465290756\\
51	0.00225450608559128\\
52	0.0022544770288573\\
53	0.00225444747432227\\
54	0.00225441741345974\\
55	0.00225438683759773\\
56	0.00225435573791633\\
57	0.00225432410544509\\
58	0.00225429193106057\\
59	0.00225425920548367\\
60	0.00225422591927705\\
61	0.0022541920628424\\
62	0.00225415762641777\\
63	0.00225412260007477\\
64	0.00225408697371574\\
65	0.00225405073707093\\
66	0.00225401387969554\\
67	0.0022539763909668\\
68	0.00225393826008092\\
69	0.0022538994760501\\
70	0.00225386002769933\\
71	0.0022538199036633\\
72	0.00225377909238315\\
73	0.00225373758210322\\
74	0.0022536953608677\\
75	0.00225365241651727\\
76	0.00225360873668569\\
77	0.00225356430879626\\
78	0.0022535191200583\\
79	0.00225347315746352\\
80	0.0022534264077824\\
81	0.00225337885756039\\
82	0.00225333049311418\\
83	0.00225328130052779\\
84	0.0022532312656487\\
85	0.00225318037408382\\
86	0.00225312861119549\\
87	0.00225307596209731\\
88	0.00225302241164999\\
89	0.00225296794445707\\
90	0.0022529125448606\\
91	0.00225285619693674\\
92	0.0022527988844913\\
93	0.00225274059105517\\
94	0.0022526812998797\\
95	0.00225262099393201\\
96	0.00225255965589022\\
97	0.00225249726813854\\
98	0.00225243381276239\\
99	0.00225236927154336\\
100	0.00225230362595409\\
101	0.00225223685715307\\
102	0.00225216894597939\\
103	0.00225209987294737\\
104	0.00225202961824105\\
105	0.00225195816170873\\
106	0.00225188548285725\\
107	0.00225181156084631\\
108	0.00225173637448258\\
109	0.00225165990221383\\
110	0.00225158212212285\\
111	0.00225150301192136\\
112	0.00225142254894375\\
113	0.00225134071014072\\
114	0.00225125747207289\\
115	0.00225117281090421\\
116	0.00225108670239529\\
117	0.00225099912189661\\
118	0.0022509100443417\\
119	0.00225081944424\\
120	0.00225072729566989\\
121	0.00225063357227131\\
122	0.00225053824723845\\
123	0.00225044129331226\\
124	0.0022503426827728\\
125	0.0022502423874315\\
126	0.0022501403786233\\
127	0.00225003662719858\\
128	0.00224993110351509\\
129	0.00224982377742956\\
130	0.00224971461828938\\
131	0.00224960359492395\\
132	0.00224949067563599\\
133	0.00224937582819269\\
134	0.00224925901981669\\
135	0.00224914021717691\\
136	0.00224901938637924\\
137	0.00224889649295707\\
138	0.00224877150186167\\
139	0.00224864437745237\\
140	0.00224851508348662\\
141	0.00224838358310987\\
142	0.00224824983884521\\
143	0.00224811381258292\\
144	0.00224797546556979\\
145	0.00224783475839845\\
146	0.00224769165099652\\
147	0.00224754610261519\\
148	0.0022473980718179\\
149	0.00224724751646874\\
150	0.00224709439372072\\
151	0.00224693866000382\\
152	0.00224678027101282\\
153	0.00224661918169497\\
154	0.0022464553462374\\
155	0.00224628871805434\\
156	0.00224611924977416\\
157	0.00224594689322612\\
158	0.00224577159942695\\
159	0.00224559331856717\\
160	0.00224541199999721\\
161	0.00224522759221328\\
162	0.00224504004284297\\
163	0.00224484929863065\\
164	0.00224465530542263\\
165	0.002244458008152\\
166	0.00224425735082327\\
167	0.00224405327649673\\
168	0.00224384572727259\\
169	0.00224363464427471\\
170	0.00224341996763425\\
171	0.00224320163647287\\
172	0.00224297958888574\\
173	0.0022427537619242\\
174	0.00224252409157822\\
175	0.0022422905127584\\
176	0.00224205295927782\\
177	0.00224181136383346\\
178	0.0022415656579874\\
179	0.00224131577214762\\
180	0.00224106163554846\\
181	0.00224080317623084\\
182	0.00224054032102202\\
183	0.00224027299551509\\
184	0.00224000112404807\\
185	0.00223972462968265\\
186	0.00223944343418256\\
187	0.00223915745799157\\
188	0.00223886662021111\\
189	0.00223857083857746\\
190	0.0022382700294386\\
191	0.00223796410773064\\
192	0.00223765298695376\\
193	0.00223733657914782\\
194	0.00223701479486753\\
195	0.00223668754315712\\
196	0.00223635473152461\\
197	0.00223601626591562\\
198	0.0022356720506867\\
199	0.00223532198857818\\
200	0.00223496598068657\\
201	0.00223460392643642\\
202	0.00223423572355172\\
203	0.00223386126802676\\
204	0.00223348045409645\\
205	0.00223309317420614\\
206	0.00223269931898089\\
207	0.00223229877719413\\
208	0.00223189143573586\\
209	0.00223147717958013\\
210	0.00223105589175204\\
211	0.00223062745329407\\
212	0.00223019174323182\\
213	0.00222974863853913\\
214	0.00222929801410253\\
215	0.002228839742685\\
216	0.00222837369488917\\
217	0.00222789973911968\\
218	0.00222741774154498\\
219	0.00222692756605833\\
220	0.00222642907423807\\
221	0.00222592212530717\\
222	0.00222540657609198\\
223	0.00222488228098028\\
224	0.00222434909187839\\
225	0.00222380685816758\\
226	0.0022232554266596\\
227	0.00222269464155133\\
228	0.0022221243443786\\
229	0.00222154437396911\\
230	0.00222095456639435\\
231	0.00222035475492071\\
232	0.00221974476995955\\
233	0.00221912443901628\\
234	0.00221849358663846\\
235	0.00221785203436292\\
236	0.00221719960066168\\
237	0.00221653610088695\\
238	0.00221586134721493\\
239	0.00221517514858848\\
240	0.00221447731065868\\
241	0.00221376763572513\\
242	0.0022130459226751\\
243	0.00221231196692134\\
244	0.00221156556033872\\
245	0.00221080649119944\\
246	0.00221003454410697\\
247	0.00220924949992857\\
248	0.00220845113572638\\
249	0.00220763922468709\\
250	0.00220681353605007\\
251	0.00220597383503397\\
252	0.00220511988276175\\
253	0.00220425143618409\\
254	0.00220336824800114\\
255	0.00220247006658248\\
256	0.00220155663588542\\
257	0.00220062769537139\\
258	0.00219968297992052\\
259	0.00219872221974431\\
260	0.0021977451402962\\
261	0.00219675146218023\\
262	0.00219574090105754\\
263	0.00219471316755064\\
264	0.00219366796714554\\
265	0.00219260500009143\\
266	0.00219152396129811\\
267	0.00219042454023081\\
268	0.00218930642080246\\
269	0.00218816928126336\\
270	0.00218701279408796\\
271	0.00218583662585887\\
272	0.00218464043714798\\
273	0.00218342388239507\\
274	0.00218218660978429\\
275	0.00218092826111787\\
276	0.00217964847168101\\
277	0.00217834687009569\\
278	0.00217702307819071\\
279	0.00217567671085904\\
280	0.00217430737591045\\
281	0.00217291467392013\\
282	0.00217149819807274\\
283	0.00217005753400202\\
284	0.00216859225962569\\
285	0.00216710194497527\\
286	0.00216558615202082\\
287	0.00216404443449025\\
288	0.00216247633768293\\
289	0.00216088139827743\\
290	0.00215925914413306\\
291	0.00215760909408485\\
292	0.00215593075773177\\
293	0.00215422363521778\\
294	0.00215248721700542\\
295	0.00215072098364137\\
296	0.00214892440551389\\
297	0.00214709694260139\\
298	0.00214523804421187\\
299	0.00214334714871261\\
300	0.00214142368324968\\
301	0.00213946706345662\\
302	0.00213747669315173\\
303	0.00213545196402326\\
304	0.00213339225530193\\
305	0.00213129693341986\\
306	0.00212916535165536\\
307	0.00212699684976246\\
308	0.00212479075358427\\
309	0.00212254637464819\\
310	0.00212026300974022\\
311	0.0021179399404549\\
312	0.00211557643272357\\
313	0.0021131717363424\\
314	0.00211072508450726\\
315	0.00210823569325357\\
316	0.00210570276087912\\
317	0.00210312546735227\\
318	0.0021005029736881\\
319	0.00209783442128978\\
320	0.00209511893125242\\
321	0.00209235560362633\\
322	0.00208954351663622\\
323	0.00208668172585272\\
324	0.00208376926331188\\
325	0.00208080513657825\\
326	0.00207778832774644\\
327	0.00207471779237544\\
328	0.00207159245834967\\
329	0.0020684112246598\\
330	0.00206517296009569\\
331	0.00206187650184302\\
332	0.00205852065397409\\
333	0.00205510418582252\\
334	0.00205162583022995\\
335	0.00204808428165185\\
336	0.00204447819410802\\
337	0.00204080617896145\\
338	0.00203706680250769\\
339	0.00203325858335442\\
340	0.00202937998956875\\
341	0.00202542943556714\\
342	0.00202140527871949\\
343	0.00201730581563618\\
344	0.00201312927810233\\
345	0.00200887382861977\\
346	0.00200453755551226\\
347	0.00200011846754379\\
348	0.00199561448799404\\
349	0.00199102344812764\\
350	0.00198634307998626\\
351	0.00198157100842312\\
352	0.00197670474228982\\
353	0.00197174166467503\\
354	0.0019666790220888\\
355	0.00196151391248782\\
356	0.0019562432720273\\
357	0.00195086386027465\\
358	0.00194537224329756\\
359	0.00193976477445752\\
360	0.00193403757236674\\
361	0.00192818649089298\\
362	0.00192220706749014\\
363	0.00191609439936248\\
364	0.00190984276577908\\
365	0.001903444344481\\
366	0.00189688471607222\\
367	0.00189012753770206\\
368	0.00188307204594991\\
369	0.00187532607576948\\
370	0.00186741687321954\\
371	0.00185937371942756\\
372	0.00185119429012358\\
373	0.00184287622349694\\
374	0.00183441711805204\\
375	0.0018258145280678\\
376	0.00181706595713902\\
377	0.00180816886457825\\
378	0.0017991207509713\\
379	0.00178991932478652\\
380	0.00178056209059273\\
381	0.00177104636415697\\
382	0.0017613694237474\\
383	0.00175152851070232\\
384	0.00174152083010462\\
385	0.00173134355156662\\
386	0.00172099381012973\\
387	0.00171046870728252\\
388	0.00169976531209971\\
389	0.00168888066250364\\
390	0.00167781176664776\\
391	0.00166655560442005\\
392	0.00165510912906219\\
393	0.00164346926889762\\
394	0.00163163292916047\\
395	0.00161959699391801\\
396	0.00160735832808923\\
397	0.00159491377959592\\
398	0.00158226018178169\\
399	0.00156939435649901\\
400	0.00155631311892567\\
401	0.00154301328669679\\
402	0.00152949169902099\\
403	0.00151574525607928\\
404	0.00150177098980964\\
405	0.0014875661528874\\
406	0.00147312828649604\\
407	0.00145845627135828\\
408	0.00144355769227254\\
409	0.00142846262402851\\
410	0.00141326501158894\\
411	0.00139822731853432\\
412	0.00138388740253358\\
413	0.00137011704235746\\
414	0.00135607976325594\\
415	0.00134177068139713\\
416	0.0013271838316562\\
417	0.00131231249661213\\
418	0.0012971497300087\\
419	0.00128168834618063\\
420	0.00126592090865607\\
421	0.00124983971672581\\
422	0.00123343678767319\\
423	0.00121670383904977\\
424	0.00119963228876237\\
425	0.00118221324775915\\
426	0.00116443749003025\\
427	0.00114629543543994\\
428	0.0011277771315331\\
429	0.00110887223425105\\
430	0.00108956998748904\\
431	0.00106985920142357\\
432	0.00104972822953443\\
433	0.00102916494424355\\
434	0.00100815671108913\\
435	0.000986690361352274\\
436	0.000964752163052489\\
437	0.000942327790230635\\
438	0.000919402290441712\\
439	0.000895960050373461\\
440	0.000871984759420209\\
441	0.000847459370560392\\
442	0.000822366055769701\\
443	0.000796686145148563\\
444	0.000770400012820854\\
445	0.0007434868148554\\
446	0.000715924066459259\\
447	0.000687689175228605\\
448	0.000658765806217179\\
449	0.000629129746906497\\
450	0.000598755261626512\\
451	0.00056761530781351\\
452	0.00053568151467907\\
453	0.000502924087514885\\
454	0.000469311590360925\\
455	0.000434810187659242\\
456	0.000399380515904592\\
457	0.000362964325713837\\
458	0.000325427915679923\\
459	0.000286121859631025\\
460	0.00024577460918375\\
461	0.000204529627511481\\
462	0.000162346680408577\\
463	0.000119118848819356\\
464	7.45041232746955e-05\\
465	2.6781960106478e-05\\
466	0\\
467	0\\
468	0\\
469	0\\
470	0\\
471	0\\
472	0\\
473	0\\
474	0\\
475	0\\
476	0\\
477	0\\
478	0\\
479	0\\
480	0\\
481	0\\
482	0\\
483	0\\
484	0\\
485	0\\
486	0\\
487	0\\
488	0\\
489	0\\
490	0\\
491	0\\
492	0\\
493	0\\
494	0\\
495	0\\
496	0\\
497	0\\
498	0\\
499	0\\
500	0\\
501	0\\
502	0\\
503	0\\
504	0\\
505	0\\
506	0\\
507	0\\
508	0\\
509	0\\
510	0\\
511	0\\
512	0\\
513	0\\
514	0\\
515	0\\
516	0\\
517	0\\
518	0\\
519	0\\
520	0\\
521	0\\
522	0\\
523	0\\
524	0\\
525	0\\
526	0\\
527	0\\
528	0\\
529	0\\
530	0\\
531	0\\
532	0\\
533	0\\
534	0\\
535	0\\
536	0\\
537	0\\
538	0\\
539	0\\
540	0\\
541	0\\
542	0\\
543	0\\
544	0\\
545	0\\
546	0\\
547	0\\
548	0\\
549	0\\
550	0\\
551	0\\
552	0\\
553	0\\
554	0\\
555	0\\
556	0\\
557	0\\
558	0\\
559	0\\
560	0\\
561	0\\
562	0\\
563	0\\
564	0\\
565	0\\
566	0\\
567	0\\
568	0\\
569	0\\
570	0\\
571	0\\
572	0\\
573	0\\
574	0\\
575	0\\
576	0\\
577	0\\
578	0\\
579	0\\
580	0\\
581	0\\
582	0\\
583	0\\
584	0\\
585	0\\
586	0\\
587	0\\
588	0\\
589	0\\
590	0\\
591	0\\
592	0\\
593	0\\
594	0\\
595	0\\
596	0\\
597	0\\
598	0\\
599	0\\
600	0\\
};
\addplot [color=mycolor10,solid,forget plot]
  table[row sep=crcr]{%
1	0.00367136005952458\\
2	0.0036713542087014\\
3	0.00367134825773797\\
4	0.00367134220491635\\
5	0.00367133604848915\\
6	0.00367132978667907\\
7	0.00367132341767836\\
8	0.00367131693964833\\
9	0.00367131035071882\\
10	0.00367130364898763\\
11	0.00367129683252001\\
12	0.0036712898993481\\
13	0.00367128284747037\\
14	0.003671275674851\\
15	0.00367126837941938\\
16	0.00367126095906944\\
17	0.00367125341165909\\
18	0.00367124573500959\\
19	0.00367123792690494\\
20	0.00367122998509123\\
21	0.00367122190727602\\
22	0.00367121369112766\\
23	0.00367120533427462\\
24	0.00367119683430483\\
25	0.003671188188765\\
26	0.0036711793951599\\
27	0.00367117045095163\\
28	0.00367116135355894\\
29	0.00367115210035647\\
30	0.00367114268867399\\
31	0.00367113311579566\\
32	0.00367112337895925\\
33	0.00367111347535534\\
34	0.00367110340212652\\
35	0.00367109315636661\\
36	0.00367108273511978\\
37	0.00367107213537974\\
38	0.00367106135408891\\
39	0.00367105038813747\\
40	0.00367103923436256\\
41	0.00367102788954732\\
42	0.00367101635042003\\
43	0.00367100461365313\\
44	0.0036709926758623\\
45	0.00367098053360549\\
46	0.00367096818338196\\
47	0.00367095562163127\\
48	0.00367094284473225\\
49	0.00367092984900204\\
50	0.00367091663069497\\
51	0.00367090318600154\\
52	0.00367088951104733\\
53	0.00367087560189192\\
54	0.00367086145452774\\
55	0.00367084706487896\\
56	0.00367083242880035\\
57	0.00367081754207607\\
58	0.0036708024004185\\
59	0.00367078699946705\\
60	0.00367077133478688\\
61	0.00367075540186769\\
62	0.00367073919612242\\
63	0.00367072271288598\\
64	0.00367070594741391\\
65	0.00367068889488107\\
66	0.00367067155038026\\
67	0.00367065390892083\\
68	0.0036706359654273\\
69	0.00367061771473792\\
70	0.00367059915160324\\
71	0.00367058027068457\\
72	0.00367056106655255\\
73	0.0036705415336856\\
74	0.00367052166646835\\
75	0.0036705014591901\\
76	0.00367048090604318\\
77	0.00367046000112136\\
78	0.00367043873841816\\
79	0.00367041711182519\\
80	0.00367039511513042\\
81	0.00367037274201647\\
82	0.0036703499860588\\
83	0.00367032684072394\\
84	0.00367030329936769\\
85	0.00367027935523319\\
86	0.0036702550014491\\
87	0.00367023023102764\\
88	0.00367020503686266\\
89	0.00367017941172767\\
90	0.00367015334827377\\
91	0.00367012683902765\\
92	0.0036700998763895\\
93	0.00367007245263087\\
94	0.00367004455989254\\
95	0.00367001619018233\\
96	0.00366998733537283\\
97	0.00366995798719921\\
98	0.00366992813725689\\
99	0.00366989777699916\\
100	0.00366986689773488\\
101	0.003669835490626\\
102	0.00366980354668515\\
103	0.00366977105677311\\
104	0.00366973801159628\\
105	0.0036697044017041\\
106	0.00366967021748645\\
107	0.00366963544917095\\
108	0.00366960008682027\\
109	0.00366956412032934\\
110	0.00366952753942259\\
111	0.00366949033365109\\
112	0.00366945249238962\\
113	0.00366941400483377\\
114	0.00366937485999693\\
115	0.00366933504670721\\
116	0.00366929455360442\\
117	0.00366925336913683\\
118	0.00366921148155806\\
119	0.00366916887892377\\
120	0.00366912554908839\\
121	0.0036690814797017\\
122	0.00366903665820549\\
123	0.00366899107183002\\
124	0.00366894470759055\\
125	0.00366889755228367\\
126	0.00366884959248373\\
127	0.00366880081453906\\
128	0.00366875120456826\\
129	0.0036687007484563\\
130	0.00366864943185067\\
131	0.0036685972401574\\
132	0.00366854415853704\\
133	0.00366849017190053\\
134	0.00366843526490509\\
135	0.00366837942194993\\
136	0.00366832262717202\\
137	0.00366826486444162\\
138	0.00366820611735795\\
139	0.00366814636924458\\
140	0.00366808560314488\\
141	0.00366802380181734\\
142	0.00366796094773078\\
143	0.00366789702305959\\
144	0.0036678320096788\\
145	0.00366776588915917\\
146	0.00366769864276207\\
147	0.00366763025143433\\
148	0.00366756069580298\\
149	0.00366748995616998\\
150	0.00366741801250679\\
151	0.00366734484444885\\
152	0.00366727043129002\\
153	0.0036671947519769\\
154	0.00366711778510305\\
155	0.0036670395089031\\
156	0.00366695990124682\\
157	0.00366687893963302\\
158	0.0036667966011834\\
159	0.00366671286263627\\
160	0.0036666277003402\\
161	0.00366654109024748\\
162	0.00366645300790763\\
163	0.0036663634284606\\
164	0.00366627232663002\\
165	0.00366617967671628\\
166	0.00366608545258947\\
167	0.00366598962768222\\
168	0.00366589217498246\\
169	0.00366579306702598\\
170	0.00366569227588894\\
171	0.00366558977318025\\
172	0.0036654855300337\\
173	0.0036653795171002\\
174	0.00366527170453961\\
175	0.00366516206201264\\
176	0.00366505055867255\\
177	0.00366493716315665\\
178	0.00366482184357777\\
179	0.00366470456751549\\
180	0.00366458530200726\\
181	0.00366446401353939\\
182	0.00366434066803786\\
183	0.00366421523085895\\
184	0.0036640876667798\\
185	0.00366395793998869\\
186	0.00366382601407527\\
187	0.00366369185202057\\
188	0.00366355541618677\\
189	0.00366341666830697\\
190	0.00366327556947463\\
191	0.00366313208013285\\
192	0.00366298616006357\\
193	0.00366283776837647\\
194	0.00366268686349774\\
195	0.00366253340315862\\
196	0.00366237734438379\\
197	0.0036622186434795\\
198	0.00366205725602156\\
199	0.00366189313684307\\
200	0.00366172624002197\\
201	0.00366155651886834\\
202	0.00366138392591151\\
203	0.00366120841288693\\
204	0.00366102993072281\\
205	0.00366084842952655\\
206	0.00366066385857087\\
207	0.00366047616627979\\
208	0.00366028530021424\\
209	0.00366009120705756\\
210	0.00365989383260061\\
211	0.00365969312172667\\
212	0.00365948901839613\\
213	0.00365928146563077\\
214	0.0036590704054979\\
215	0.0036588557790941\\
216	0.00365863752652877\\
217	0.00365841558690728\\
218	0.00365818989831388\\
219	0.0036579603977943\\
220	0.00365772702133799\\
221	0.00365748970386007\\
222	0.00365724837918293\\
223	0.00365700298001751\\
224	0.00365675343794421\\
225	0.00365649968339345\\
226	0.0036562416456259\\
227	0.00365597925271227\\
228	0.00365571243151281\\
229	0.00365544110765633\\
230	0.00365516520551892\\
231	0.00365488464820216\\
232	0.00365459935751102\\
233	0.00365430925393122\\
234	0.0036540142566063\\
235	0.00365371428331407\\
236	0.00365340925044277\\
237	0.00365309907296666\\
238	0.00365278366442114\\
239	0.00365246293687742\\
240	0.00365213680091666\\
241	0.00365180516560358\\
242	0.00365146793845959\\
243	0.00365112502543529\\
244	0.00365077633088253\\
245	0.00365042175752576\\
246	0.00365006120643292\\
247	0.00364969457698562\\
248	0.00364932176684879\\
249	0.0036489426719396\\
250	0.00364855718639577\\
251	0.00364816520254322\\
252	0.00364776661086298\\
253	0.0036473612999574\\
254	0.0036469491565156\\
255	0.00364653006527817\\
256	0.0036461039090011\\
257	0.00364567056841883\\
258	0.0036452299222065\\
259	0.00364478184694136\\
260	0.00364432621706316\\
261	0.00364386290483372\\
262	0.00364339178029548\\
263	0.00364291271122899\\
264	0.00364242556310947\\
265	0.00364193019906216\\
266	0.00364142647981658\\
267	0.00364091426365965\\
268	0.00364039340638746\\
269	0.00363986376125587\\
270	0.00363932517892966\\
271	0.00363877750743043\\
272	0.00363822059208316\\
273	0.00363765427546155\\
274	0.00363707839733184\\
275	0.0036364927945936\\
276	0.00363589730121682\\
277	0.00363529174818207\\
278	0.00363467596341668\\
279	0.00363404977172866\\
280	0.00363341299473854\\
281	0.00363276545080885\\
282	0.00363210695497132\\
283	0.00363143731885151\\
284	0.0036307563505908\\
285	0.00363006385476552\\
286	0.00362935963230314\\
287	0.00362864348039527\\
288	0.00362791519240729\\
289	0.0036271745577844\\
290	0.00362642136195385\\
291	0.00362565538622313\\
292	0.00362487640767375\\
293	0.00362408419905044\\
294	0.00362327852864536\\
295	0.00362245916017698\\
296	0.0036216258526632\\
297	0.00362077836028839\\
298	0.00361991643226378\\
299	0.00361903981268069\\
300	0.00361814824035613\\
301	0.00361724144867005\\
302	0.00361631916539358\\
303	0.00361538111250761\\
304	0.00361442700601071\\
305	0.00361345655571573\\
306	0.00361246946503382\\
307	0.00361146543074481\\
308	0.00361044414275246\\
309	0.00360940528382247\\
310	0.0036083485293015\\
311	0.00360727354681671\\
312	0.00360617999595942\\
313	0.00360506752795359\\
314	0.00360393578528118\\
315	0.00360278440128144\\
316	0.00360161299972272\\
317	0.00360042119433957\\
318	0.00359920858833189\\
319	0.00359797477382177\\
320	0.0035967193312638\\
321	0.00359544182880375\\
322	0.00359414182157997\\
323	0.00359281885096147\\
324	0.00359147244371551\\
325	0.00359010211109698\\
326	0.00358870734785085\\
327	0.00358728763111789\\
328	0.00358584241923271\\
329	0.00358437115040184\\
330	0.00358287324124808\\
331	0.00358134808520573\\
332	0.00357979505074929\\
333	0.00357821347943625\\
334	0.00357660268374211\\
335	0.00357496194466306\\
336	0.00357329050905873\\
337	0.00357158758670406\\
338	0.00356985234701524\\
339	0.00356808391541051\\
340	0.00356628136926162\\
341	0.00356444373338597\\
342	0.0035625699750233\\
343	0.00356065899823356\\
344	0.00355870963764444\\
345	0.0035567206514678\\
346	0.00355469071369377\\
347	0.00355261840535946\\
348	0.00355050220477556\\
349	0.00354834047657885\\
350	0.0035461314594607\\
351	0.00354387325240232\\
352	0.00354156379922452\\
353	0.00353920087123593\\
354	0.00353678204773568\\
355	0.00353430469408663\\
356	0.00353176593699083\\
357	0.00352916263644676\\
358	0.00352649135382048\\
359	0.00352374831505654\\
360	0.00352092936611005\\
361	0.00351802991340023\\
362	0.00351504482719273\\
363	0.00351196823786181\\
364	0.00350879299930346\\
365	0.00350550909462353\\
366	0.00350209870266089\\
367	0.00349852084308253\\
368	0.00349465963001152\\
369	0.00349010686104009\\
370	0.00348544033389064\\
371	0.00348070343857037\\
372	0.00347589524788633\\
373	0.0034710148278648\\
374	0.00346606123691451\\
375	0.00346103352543611\\
376	0.00345593074002277\\
377	0.00345075194222006\\
378	0.00344549623478454\\
379	0.0034401626745596\\
380	0.00343475028031016\\
381	0.0034292580655807\\
382	0.00342368503853128\\
383	0.00341803020166984\\
384	0.00341229255146185\\
385	0.00340647107779616\\
386	0.00340056476328271\\
387	0.00339457258235557\\
388	0.00338849350015083\\
389	0.00338232647112589\\
390	0.00337607043738275\\
391	0.00336972432665376\\
392	0.0033632870499045\\
393	0.00335675749850444\\
394	0.00335013454091347\\
395	0.00334341701883367\\
396	0.00333660374278532\\
397	0.00332969348709849\\
398	0.00332268498439105\\
399	0.00331557691978132\\
400	0.00330836792542846\\
401	0.00330105657652901\\
402	0.00329364139035451\\
403	0.00328612082939462\\
404	0.00327849330934434\\
405	0.00327075724168134\\
406	0.00326291136982867\\
407	0.00325495636404665\\
408	0.00324689763874729\\
409	0.00323875231388637\\
410	0.00323056097037767\\
411	0.00322238312421398\\
412	0.00321417988858275\\
413	0.0032058389817077\\
414	0.00319735795331092\\
415	0.00318873408097052\\
416	0.00317996444183571\\
417	0.00317104601964941\\
418	0.0031619757002138\\
419	0.00315275026644325\\
420	0.00314336639261583\\
421	0.00313382063738896\\
422	0.00312410943678926\\
423	0.00311422910202385\\
424	0.00310417581523831\\
425	0.00309394561880077\\
426	0.00308353440793545\\
427	0.00307293792291095\\
428	0.00306215174075572\\
429	0.00305117126647352\\
430	0.00303999172372835\\
431	0.00302860814496947\\
432	0.0030170153609647\\
433	0.00300520798970998\\
434	0.00299318042468135\\
435	0.00298092682239276\\
436	0.00296844108921502\\
437	0.00295571686738979\\
438	0.00294274752011353\\
439	0.00292952611540876\\
440	0.00291604540808881\\
441	0.00290229781811762\\
442	0.00288827540156554\\
443	0.00287396980789372\\
444	0.00285937222539627\\
445	0.00284447339254048\\
446	0.00282926403681843\\
447	0.0028137358241746\\
448	0.00279787831489232\\
449	0.00278168049455409\\
450	0.00276513081731052\\
451	0.00274821718821128\\
452	0.00273092691441245\\
453	0.00271324658327365\\
454	0.00269516167107363\\
455	0.00267665513762852\\
456	0.0026577021184723\\
457	0.00263824930352147\\
458	0.00261813291909802\\
459	0.00259690828776238\\
460	0.00257525010215384\\
461	0.00255329429443533\\
462	0.00253101105647798\\
463	0.00250831014118649\\
464	0.00248489435582048\\
465	0.00246008673831945\\
466	0.00243435385994019\\
467	0.00240879072737593\\
468	0.00237947399244218\\
469	0.0023482221547078\\
470	0.00231642921410384\\
471	0.0022841442400962\\
472	0.00225135420759003\\
473	0.00221804954900144\\
474	0.00218423389328816\\
475	0.00214995437136689\\
476	0.00211539324347675\\
477	0.00208111881827806\\
478	0.00204850191368291\\
479	0.00201860096918738\\
480	0.0019879012972538\\
481	0.00195636036166827\\
482	0.00192393154007199\\
483	0.00189056366105782\\
484	0.00185619990513972\\
485	0.00182077534695337\\
486	0.00178420965799812\\
487	0.00174637801333911\\
488	0.00170696088632293\\
489	0.00166344164530458\\
490	0.00161757795728889\\
491	0.00157061342368385\\
492	0.00152251019137918\\
493	0.00147322862401854\\
494	0.00142272720797365\\
495	0.00137096245850899\\
496	0.00131788886726747\\
497	0.00126345902010643\\
498	0.00120762420237788\\
499	0.00115033542055204\\
500	0.00109153966695643\\
501	0.00103116247127905\\
502	0.000969113373693511\\
503	0.000905317299843979\\
504	0.000839604316548774\\
505	0.000771463659200474\\
506	0.000701477745376201\\
507	0.000629526791820348\\
508	0.000554952970295424\\
509	0.000476998662898304\\
510	0.000397384018534237\\
511	0.000316032786533761\\
512	0.00023278160760913\\
513	0.000147012106123333\\
514	5.53656290951778e-05\\
515	0\\
516	0\\
517	0\\
518	0\\
519	0\\
520	0\\
521	0\\
522	0\\
523	0\\
524	0\\
525	0\\
526	0\\
527	0\\
528	0\\
529	0\\
530	0\\
531	0\\
532	0\\
533	0\\
534	0\\
535	0\\
536	0\\
537	0\\
538	0\\
539	0\\
540	0\\
541	0\\
542	0\\
543	0\\
544	0\\
545	0\\
546	0\\
547	0\\
548	0\\
549	0\\
550	0\\
551	0\\
552	0\\
553	0\\
554	0\\
555	0\\
556	0\\
557	0\\
558	0\\
559	0\\
560	0\\
561	0\\
562	0\\
563	0\\
564	0\\
565	0\\
566	0\\
567	0\\
568	0\\
569	0\\
570	0\\
571	0\\
572	0\\
573	0\\
574	0\\
575	0\\
576	0\\
577	0\\
578	0\\
579	0\\
580	0\\
581	0\\
582	0\\
583	0\\
584	0\\
585	0\\
586	0\\
587	0\\
588	0\\
589	0\\
590	0\\
591	0\\
592	0\\
593	0\\
594	0\\
595	0\\
596	0\\
597	0\\
598	0\\
599	0\\
600	0\\
};
\addplot [color=mycolor11,solid,forget plot]
  table[row sep=crcr]{%
1	0.00516053505583464\\
2	0.00516053396220163\\
3	0.00516053284984937\\
4	0.00516053171845674\\
5	0.0051605305676971\\
6	0.00516052939723823\\
7	0.00516052820674222\\
8	0.00516052699586537\\
9	0.00516052576425811\\
10	0.00516052451156487\\
11	0.005160523237424\\
12	0.00516052194146766\\
13	0.00516052062332171\\
14	0.00516051928260563\\
15	0.00516051791893236\\
16	0.00516051653190824\\
17	0.00516051512113286\\
18	0.00516051368619896\\
19	0.00516051222669234\\
20	0.00516051074219169\\
21	0.00516050923226851\\
22	0.00516050769648695\\
23	0.00516050613440375\\
24	0.00516050454556803\\
25	0.00516050292952122\\
26	0.00516050128579693\\
27	0.00516049961392076\\
28	0.00516049791341025\\
29	0.00516049618377466\\
30	0.00516049442451487\\
31	0.00516049263512326\\
32	0.00516049081508352\\
33	0.0051604889638705\\
34	0.00516048708095013\\
35	0.00516048516577919\\
36	0.00516048321780519\\
37	0.00516048123646621\\
38	0.00516047922119073\\
39	0.0051604771713975\\
40	0.00516047508649533\\
41	0.00516047296588294\\
42	0.0051604708089488\\
43	0.00516046861507095\\
44	0.00516046638361682\\
45	0.00516046411394304\\
46	0.0051604618053953\\
47	0.00516045945730809\\
48	0.00516045706900458\\
49	0.00516045463979641\\
50	0.00516045216898348\\
51	0.00516044965585375\\
52	0.00516044709968305\\
53	0.00516044449973488\\
54	0.0051604418552602\\
55	0.0051604391654972\\
56	0.0051604364296711\\
57	0.00516043364699393\\
58	0.00516043081666431\\
59	0.00516042793786721\\
60	0.00516042500977374\\
61	0.00516042203154088\\
62	0.00516041900231132\\
63	0.0051604159212131\\
64	0.00516041278735947\\
65	0.0051604095998486\\
66	0.00516040635776331\\
67	0.00516040306017083\\
68	0.00516039970612255\\
69	0.00516039629465373\\
70	0.00516039282478322\\
71	0.00516038929551324\\
72	0.00516038570582903\\
73	0.00516038205469859\\
74	0.00516037834107243\\
75	0.00516037456388322\\
76	0.00516037072204551\\
77	0.00516036681445543\\
78	0.00516036283999039\\
79	0.00516035879750876\\
80	0.00516035468584951\\
81	0.00516035050383195\\
82	0.00516034625025537\\
83	0.00516034192389869\\
84	0.00516033752352013\\
85	0.00516033304785689\\
86	0.00516032849562474\\
87	0.00516032386551772\\
88	0.00516031915620774\\
89	0.00516031436634421\\
90	0.00516030949455368\\
91	0.00516030453943946\\
92	0.0051602994995812\\
93	0.00516029437353453\\
94	0.00516028915983063\\
95	0.00516028385697584\\
96	0.00516027846345123\\
97	0.00516027297771219\\
98	0.00516026739818799\\
99	0.00516026172328136\\
100	0.00516025595136801\\
101	0.00516025008079623\\
102	0.00516024410988637\\
103	0.00516023803693045\\
104	0.00516023186019159\\
105	0.00516022557790363\\
106	0.00516021918827055\\
107	0.00516021268946602\\
108	0.00516020607963291\\
109	0.0051601993568827\\
110	0.00516019251929506\\
111	0.00516018556491722\\
112	0.00516017849176349\\
113	0.00516017129781467\\
114	0.00516016398101754\\
115	0.00516015653928424\\
116	0.00516014897049172\\
117	0.00516014127248113\\
118	0.00516013344305727\\
119	0.00516012547998791\\
120	0.00516011738100323\\
121	0.00516010914379517\\
122	0.00516010076601679\\
123	0.00516009224528163\\
124	0.00516008357916303\\
125	0.00516007476519348\\
126	0.00516006580086392\\
127	0.00516005668362308\\
128	0.00516004741087672\\
129	0.00516003797998697\\
130	0.00516002838827154\\
131	0.00516001863300304\\
132	0.0051600087114082\\
133	0.0051599986206671\\
134	0.00515998835791238\\
135	0.00515997792022849\\
136	0.00515996730465084\\
137	0.00515995650816502\\
138	0.00515994552770596\\
139	0.00515993436015705\\
140	0.00515992300234932\\
141	0.00515991145106056\\
142	0.00515989970301442\\
143	0.00515988775487952\\
144	0.00515987560326854\\
145	0.00515986324473728\\
146	0.00515985067578372\\
147	0.00515983789284705\\
148	0.00515982489230669\\
149	0.00515981167048129\\
150	0.00515979822362773\\
151	0.00515978454794009\\
152	0.00515977063954862\\
153	0.00515975649451863\\
154	0.00515974210884946\\
155	0.00515972747847336\\
156	0.00515971259925437\\
157	0.00515969746698721\\
158	0.00515968207739608\\
159	0.00515966642613356\\
160	0.00515965050877932\\
161	0.00515963432083899\\
162	0.0051596178577429\\
163	0.00515960111484483\\
164	0.00515958408742071\\
165	0.00515956677066737\\
166	0.00515954915970118\\
167	0.00515953124955675\\
168	0.00515951303518555\\
169	0.0051594945114545\\
170	0.00515947567314463\\
171	0.00515945651494957\\
172	0.00515943703147415\\
173	0.00515941721723292\\
174	0.00515939706664861\\
175	0.00515937657405062\\
176	0.00515935573367348\\
177	0.00515933453965525\\
178	0.00515931298603592\\
179	0.00515929106675577\\
180	0.00515926877565369\\
181	0.00515924610646555\\
182	0.00515922305282241\\
183	0.00515919960824879\\
184	0.00515917576616094\\
185	0.00515915151986495\\
186	0.00515912686255499\\
187	0.00515910178731138\\
188	0.00515907628709871\\
189	0.00515905035476392\\
190	0.0051590239830343\\
191	0.0051589971645155\\
192	0.00515896989168953\\
193	0.0051589421569126\\
194	0.00515891395241311\\
195	0.00515888527028945\\
196	0.00515885610250783\\
197	0.00515882644090004\\
198	0.00515879627716127\\
199	0.00515876560284769\\
200	0.00515873440937424\\
201	0.00515870268801215\\
202	0.00515867042988658\\
203	0.00515863762597415\\
204	0.00515860426710041\\
205	0.00515857034393732\\
206	0.00515853584700062\\
207	0.00515850076664722\\
208	0.00515846509307253\\
209	0.00515842881630766\\
210	0.00515839192621668\\
211	0.0051583544124938\\
212	0.00515831626466043\\
213	0.00515827747206232\\
214	0.00515823802386648\\
215	0.0051581979090582\\
216	0.00515815711643796\\
217	0.00515811563461819\\
218	0.00515807345202016\\
219	0.00515803055687064\\
220	0.00515798693719857\\
221	0.0051579425808317\\
222	0.00515789747539312\\
223	0.00515785160829771\\
224	0.00515780496674857\\
225	0.00515775753773339\\
226	0.00515770930802069\\
227	0.00515766026415605\\
228	0.00515761039245821\\
229	0.00515755967901517\\
230	0.00515750810968016\\
231	0.00515745567006754\\
232	0.00515740234554864\\
233	0.0051573481212475\\
234	0.00515729298203654\\
235	0.00515723691253213\\
236	0.0051571798970901\\
237	0.00515712191980113\\
238	0.00515706296448607\\
239	0.00515700301469116\\
240	0.00515694205368316\\
241	0.00515688006444434\\
242	0.00515681702966746\\
243	0.00515675293175053\\
244	0.00515668775279156\\
245	0.00515662147458317\\
246	0.00515655407860704\\
247	0.00515648554602832\\
248	0.00515641585768987\\
249	0.00515634499410639\\
250	0.00515627293545845\\
251	0.00515619966158632\\
252	0.00515612515198376\\
253	0.00515604938579157\\
254	0.00515597234179111\\
255	0.00515589399839756\\
256	0.0051558143336531\\
257	0.00515573332521991\\
258	0.00515565095037301\\
259	0.00515556718599293\\
260	0.00515548200855819\\
261	0.00515539539413762\\
262	0.00515530731838246\\
263	0.00515521775651835\\
264	0.00515512668333697\\
265	0.00515503407318759\\
266	0.00515493989996835\\
267	0.0051548441371173\\
268	0.00515474675760322\\
269	0.0051546477339162\\
270	0.0051545470380579\\
271	0.00515444464153171\\
272	0.00515434051533257\\
273	0.00515423462993652\\
274	0.00515412695528961\\
275	0.00515401746079621\\
276	0.00515390611530788\\
277	0.00515379288711148\\
278	0.00515367774391692\\
279	0.00515356065284438\\
280	0.00515344158041124\\
281	0.00515332049251849\\
282	0.00515319735443667\\
283	0.00515307213079131\\
284	0.00515294478554787\\
285	0.00515281528199611\\
286	0.00515268358273379\\
287	0.00515254964964986\\
288	0.00515241344390688\\
289	0.00515227492592275\\
290	0.00515213405535171\\
291	0.00515199079106441\\
292	0.00515184509112727\\
293	0.00515169691278076\\
294	0.00515154621241674\\
295	0.00515139294555468\\
296	0.00515123706681681\\
297	0.0051510785299019\\
298	0.00515091728755775\\
299	0.00515075329155227\\
300	0.0051505864926429\\
301	0.0051504168405444\\
302	0.00515024428389477\\
303	0.00515006877021921\\
304	0.00514989024589186\\
305	0.00514970865609519\\
306	0.00514952394477684\\
307	0.00514933605460346\\
308	0.00514914492691124\\
309	0.00514895050165295\\
310	0.00514875271734128\\
311	0.00514855151098949\\
312	0.00514834681804893\\
313	0.00514813857233824\\
314	0.0051479267059676\\
315	0.00514771114925784\\
316	0.00514749183065296\\
317	0.00514726867662538\\
318	0.00514704161157321\\
319	0.00514681055770862\\
320	0.00514657543493639\\
321	0.00514633616072161\\
322	0.00514609264994523\\
323	0.00514584481474628\\
324	0.0051455925643491\\
325	0.00514533580487412\\
326	0.0051450744391301\\
327	0.00514480836638589\\
328	0.00514453748211933\\
329	0.0051442616777406\\
330	0.00514398084028718\\
331	0.005143694852087\\
332	0.00514340359038611\\
333	0.00514310692693669\\
334	0.00514280472754071\\
335	0.00514249685154394\\
336	0.00514218315127439\\
337	0.00514186347141843\\
338	0.00514153764832721\\
339	0.00514120550924475\\
340	0.00514086687144821\\
341	0.00514052154128956\\
342	0.00514016931312651\\
343	0.00513980996812899\\
344	0.00513944327294557\\
345	0.00513906897821246\\
346	0.00513868681688523\\
347	0.00513829650237064\\
348	0.00513789772643345\\
349	0.00513749015684897\\
350	0.0051370734347688\\
351	0.00513664717176195\\
352	0.00513621094648797\\
353	0.00513576430094946\\
354	0.00513530673625371\\
355	0.00513483770777226\\
356	0.00513435661949786\\
357	0.00513386281723007\\
358	0.00513335557972848\\
359	0.00513283410557918\\
360	0.00513229749024203\\
361	0.00513174467925832\\
362	0.00513117436263682\\
363	0.00513058472537829\\
364	0.00512997285665973\\
365	0.00512933339553511\\
366	0.0051286556461704\\
367	0.00512791809116632\\
368	0.00512707556718617\\
369	0.00512621365677636\\
370	0.00512533867006539\\
371	0.00512445043152696\\
372	0.00512354876402859\\
373	0.00512263348865814\\
374	0.00512170442482359\\
375	0.00512076139150791\\
376	0.00511980421108008\\
377	0.00511883271336735\\
378	0.00511784671898175\\
379	0.00511684604105501\\
380	0.00511583049149508\\
381	0.00511479988093934\\
382	0.00511375401868709\\
383	0.00511269271260762\\
384	0.00511161576901979\\
385	0.00511052299253852\\
386	0.00510941418588275\\
387	0.00510828914963941\\
388	0.00510714768197668\\
389	0.00510598957829947\\
390	0.00510481463083948\\
391	0.00510362262817123\\
392	0.00510241335464608\\
393	0.00510118658973624\\
394	0.00509994210728471\\
395	0.0050986796746647\\
396	0.00509739905187089\\
397	0.00509609999060445\\
398	0.0050947822334919\\
399	0.0050934455137228\\
400	0.00509208955564237\\
401	0.00509071407725901\\
402	0.00508931879649426\\
403	0.00508790344576597\\
404	0.00508646781104311\\
405	0.00508501185636\\
406	0.00508353610539178\\
407	0.00508204219128767\\
408	0.00508053380441878\\
409	0.00507901748525117\\
410	0.00507749997480398\\
411	0.00507597297051831\\
412	0.00507441994867542\\
413	0.00507284042967778\\
414	0.00507123388203909\\
415	0.00506959973481857\\
416	0.00506793739819613\\
417	0.00506624626253051\\
418	0.0050645256973056\\
419	0.00506277504987666\\
420	0.00506099364396145\\
421	0.00505918077819892\\
422	0.00505733572583228\\
423	0.00505545773387667\\
424	0.00505354602078519\\
425	0.00505159977490034\\
426	0.00504961815280787\\
427	0.00504760027758672\\
428	0.00504554523694768\\
429	0.00504345208125428\\
430	0.00504131982141767\\
431	0.00503914742665731\\
432	0.00503693382211835\\
433	0.00503467788633475\\
434	0.00503237844852357\\
435	0.00503003428568796\\
436	0.00502764411948796\\
437	0.00502520661279766\\
438	0.00502272036578203\\
439	0.00502018391115925\\
440	0.00501759570803928\\
441	0.00501495413345738\\
442	0.00501225747122418\\
443	0.00500950390204384\\
444	0.00500669151594826\\
445	0.00500381841065281\\
446	0.00500088292228883\\
447	0.0049978829043709\\
448	0.00499481608057586\\
449	0.00499168004997979\\
450	0.00498847227099568\\
451	0.00498519002024984\\
452	0.00498183029229815\\
453	0.00497838954526938\\
454	0.00497486305463673\\
455	0.00497124330908564\\
456	0.00496751623004757\\
457	0.00496365317341908\\
458	0.00495959803985892\\
459	0.00495545847701009\\
460	0.00495125238269577\\
461	0.0049469672450374\\
462	0.00494257469975514\\
463	0.00493801023687081\\
464	0.00493315076640688\\
465	0.00492798717113069\\
466	0.00492242062516231\\
467	0.00491552816158838\\
468	0.00490322237550712\\
469	0.00488875730748457\\
470	0.0048740630976783\\
471	0.0048591286169599\\
472	0.00484394361654972\\
473	0.00482850186063932\\
474	0.00481280854466351\\
475	0.00479689621963633\\
476	0.00478085290830506\\
477	0.00476484288526039\\
478	0.00474893442827621\\
479	0.0047325699405543\\
480	0.00471571948336329\\
481	0.00469834975067911\\
482	0.00468042346135682\\
483	0.00466189768238898\\
484	0.00464271982189616\\
485	0.00462281735643028\\
486	0.00460207060107256\\
487	0.00458023940772211\\
488	0.00455676418504533\\
489	0.00452948477304095\\
490	0.00450046664968492\\
491	0.00447100333840286\\
492	0.00444108528019643\\
493	0.0044107025642607\\
494	0.00437984492848087\\
495	0.00434850177656219\\
496	0.00431666220770141\\
497	0.00428431500858952\\
498	0.00425144835853428\\
499	0.00421804861420509\\
500	0.00418409693491697\\
501	0.00414957075012113\\
502	0.00411444927293261\\
503	0.00407869978160293\\
504	0.00404226856299079\\
505	0.00400519790882657\\
506	0.00396741722630471\\
507	0.0039287588917257\\
508	0.00388876341634874\\
509	0.00384719310549665\\
510	0.00380535624004568\\
511	0.00376318093942821\\
512	0.0037204691751195\\
513	0.00367670479506395\\
514	0.00363073524723451\\
515	0.00358294946602486\\
516	0.00353505653663763\\
517	0.00348709399632349\\
518	0.00343914842590566\\
519	0.00339142939890124\\
520	0.00334444598783759\\
521	0.00329937957587658\\
522	0.00325854851736096\\
523	0.00322419649826356\\
524	0.00318922244626424\\
525	0.00315326103247786\\
526	0.00311514836296493\\
527	0.00307123341315276\\
528	0.0030098509516261\\
529	0.00294020748052116\\
530	0.00286758526591559\\
531	0.00279104864096957\\
532	0.00270989433435107\\
533	0.00262004142303426\\
534	0.00252218784358832\\
535	0.00242178491691014\\
536	0.00231871896930725\\
537	0.00221285935169346\\
538	0.00210403327127977\\
539	0.00199194507799155\\
540	0.00187591619531658\\
541	0.00175411070440876\\
542	0.00162813783804569\\
543	0.00149958178120082\\
544	0.00136835829867665\\
545	0.00123439189756769\\
546	0.00109760179491459\\
547	0.000957899619460617\\
548	0.000815177798917114\\
549	0.000669268858551629\\
550	0.000519806197145862\\
551	0.000365747564450615\\
552	0.000203770911622493\\
553	2.3893647376284e-05\\
554	0\\
555	0\\
556	0\\
557	0\\
558	0\\
559	0\\
560	0\\
561	0\\
562	0\\
563	0\\
564	0\\
565	0\\
566	0\\
567	0\\
568	0\\
569	0\\
570	0\\
571	0\\
572	0\\
573	0\\
574	0\\
575	0\\
576	0\\
577	0\\
578	0\\
579	0\\
580	0\\
581	0\\
582	0\\
583	0\\
584	0\\
585	0\\
586	0\\
587	0\\
588	0\\
589	0\\
590	0\\
591	0\\
592	0\\
593	0\\
594	0\\
595	0\\
596	0\\
597	0\\
598	0\\
599	0\\
600	0\\
};
\addplot [color=mycolor12,solid,forget plot]
  table[row sep=crcr]{%
1	0.00638811260735049\\
2	0.00638811227827753\\
3	0.00638811194357173\\
4	0.00638811160313645\\
5	0.00638811125687342\\
6	0.00638811090468267\\
7	0.00638811054646253\\
8	0.00638811018210957\\
9	0.00638810981151861\\
10	0.00638810943458267\\
11	0.00638810905119292\\
12	0.00638810866123869\\
13	0.00638810826460739\\
14	0.00638810786118453\\
15	0.00638810745085365\\
16	0.0063881070334963\\
17	0.006388106608992\\
18	0.0063881061772182\\
19	0.00638810573805029\\
20	0.00638810529136148\\
21	0.00638810483702286\\
22	0.00638810437490328\\
23	0.00638810390486936\\
24	0.00638810342678545\\
25	0.00638810294051357\\
26	0.00638810244591339\\
27	0.00638810194284216\\
28	0.00638810143115473\\
29	0.00638810091070344\\
30	0.00638810038133811\\
31	0.00638809984290601\\
32	0.00638809929525179\\
33	0.00638809873821745\\
34	0.00638809817164229\\
35	0.00638809759536288\\
36	0.00638809700921297\\
37	0.00638809641302351\\
38	0.00638809580662254\\
39	0.00638809518983516\\
40	0.00638809456248351\\
41	0.00638809392438669\\
42	0.00638809327536069\\
43	0.00638809261521839\\
44	0.00638809194376946\\
45	0.00638809126082035\\
46	0.00638809056617417\\
47	0.00638808985963071\\
48	0.00638808914098634\\
49	0.00638808841003393\\
50	0.00638808766656286\\
51	0.00638808691035891\\
52	0.0063880861412042\\
53	0.00638808535887714\\
54	0.00638808456315237\\
55	0.00638808375380071\\
56	0.00638808293058904\\
57	0.00638808209328031\\
58	0.00638808124163341\\
59	0.00638808037540313\\
60	0.0063880794943401\\
61	0.00638807859819071\\
62	0.00638807768669701\\
63	0.00638807675959669\\
64	0.00638807581662296\\
65	0.00638807485750451\\
66	0.00638807388196541\\
67	0.00638807288972504\\
68	0.00638807188049802\\
69	0.0063880708539941\\
70	0.00638806980991812\\
71	0.0063880687479699\\
72	0.00638806766784416\\
73	0.00638806656923044\\
74	0.00638806545181301\\
75	0.00638806431527078\\
76	0.0063880631592772\\
77	0.00638806198350021\\
78	0.00638806078760207\\
79	0.00638805957123936\\
80	0.00638805833406281\\
81	0.00638805707571723\\
82	0.00638805579584141\\
83	0.00638805449406802\\
84	0.0063880531700235\\
85	0.00638805182332797\\
86	0.00638805045359511\\
87	0.00638804906043206\\
88	0.00638804764343931\\
89	0.00638804620221056\\
90	0.00638804473633268\\
91	0.0063880432453855\\
92	0.00638804172894177\\
93	0.00638804018656701\\
94	0.00638803861781938\\
95	0.00638803702224957\\
96	0.00638803539940067\\
97	0.00638803374880806\\
98	0.00638803206999925\\
99	0.00638803036249378\\
100	0.00638802862580305\\
101	0.00638802685943024\\
102	0.00638802506287009\\
103	0.00638802323560886\\
104	0.00638802137712409\\
105	0.00638801948688452\\
106	0.00638801756434994\\
107	0.00638801560897098\\
108	0.00638801362018904\\
109	0.00638801159743608\\
110	0.00638800954013446\\
111	0.00638800744769684\\
112	0.00638800531952593\\
113	0.00638800315501439\\
114	0.00638800095354464\\
115	0.00638799871448869\\
116	0.00638799643720797\\
117	0.00638799412105314\\
118	0.00638799176536393\\
119	0.00638798936946893\\
120	0.00638798693268543\\
121	0.00638798445431922\\
122	0.00638798193366442\\
123	0.00638797937000323\\
124	0.00638797676260578\\
125	0.00638797411072992\\
126	0.00638797141362101\\
127	0.0063879686705117\\
128	0.00638796588062174\\
129	0.00638796304315773\\
130	0.00638796015731296\\
131	0.00638795722226712\\
132	0.00638795423718612\\
133	0.00638795120122183\\
134	0.00638794811351189\\
135	0.00638794497317941\\
136	0.00638794177933278\\
137	0.0063879385310654\\
138	0.00638793522745544\\
139	0.00638793186756558\\
140	0.00638792845044275\\
141	0.00638792497511786\\
142	0.00638792144060558\\
143	0.00638791784590399\\
144	0.00638791418999439\\
145	0.00638791047184096\\
146	0.0063879066903905\\
147	0.00638790284457213\\
148	0.006387898933297\\
149	0.006387894955458\\
150	0.00638789090992943\\
151	0.00638788679556675\\
152	0.00638788261120617\\
153	0.00638787835566442\\
154	0.0063878740277384\\
155	0.00638786962620481\\
156	0.00638786514981987\\
157	0.00638786059731894\\
158	0.0063878559674162\\
159	0.00638785125880428\\
160	0.0063878464701539\\
161	0.00638784160011354\\
162	0.00638783664730901\\
163	0.00638783161034311\\
164	0.00638782648779528\\
165	0.00638782127822112\\
166	0.00638781598015209\\
167	0.00638781059209505\\
168	0.00638780511253186\\
169	0.00638779953991899\\
170	0.00638779387268706\\
171	0.00638778810924043\\
172	0.00638778224795679\\
173	0.00638777628718663\\
174	0.00638777022525291\\
175	0.00638776406045047\\
176	0.00638775779104569\\
177	0.00638775141527589\\
178	0.00638774493134897\\
179	0.00638773833744281\\
180	0.00638773163170484\\
181	0.00638772481225151\\
182	0.00638771787716775\\
183	0.00638771082450651\\
184	0.00638770365228814\\
185	0.00638769635849991\\
186	0.00638768894109543\\
187	0.00638768139799409\\
188	0.00638767372708052\\
189	0.00638766592620393\\
190	0.00638765799317761\\
191	0.00638764992577827\\
192	0.00638764172174544\\
193	0.00638763337878085\\
194	0.00638762489454782\\
195	0.00638761626667057\\
196	0.00638760749273358\\
197	0.00638759857028096\\
198	0.00638758949681571\\
199	0.00638758026979909\\
200	0.00638757088664987\\
201	0.00638756134474363\\
202	0.00638755164141204\\
203	0.00638754177394213\\
204	0.0063875317395755\\
205	0.00638752153550759\\
206	0.00638751115888687\\
207	0.00638750060681408\\
208	0.0063874898763414\\
209	0.00638747896447161\\
210	0.0063874678681573\\
211	0.00638745658429997\\
212	0.00638744510974919\\
213	0.00638743344130171\\
214	0.00638742157570056\\
215	0.00638740950963414\\
216	0.00638739723973526\\
217	0.00638738476258024\\
218	0.0063873720746879\\
219	0.00638735917251859\\
220	0.00638734605247319\\
221	0.00638733271089208\\
222	0.00638731914405413\\
223	0.00638730534817558\\
224	0.00638729131940902\\
225	0.00638727705384226\\
226	0.00638726254749722\\
227	0.00638724779632878\\
228	0.00638723279622361\\
229	0.00638721754299901\\
230	0.00638720203240168\\
231	0.00638718626010648\\
232	0.0063871702217152\\
233	0.00638715391275527\\
234	0.00638713732867844\\
235	0.00638712046485946\\
236	0.00638710331659474\\
237	0.00638708587910093\\
238	0.00638706814751352\\
239	0.00638705011688543\\
240	0.00638703178218551\\
241	0.00638701313829702\\
242	0.00638699418001616\\
243	0.00638697490205046\\
244	0.00638695529901721\\
245	0.0063869353654418\\
246	0.00638691509575612\\
247	0.00638689448429679\\
248	0.00638687352530348\\
249	0.00638685221291712\\
250	0.00638683054117809\\
251	0.00638680850402438\\
252	0.00638678609528968\\
253	0.00638676330870148\\
254	0.00638674013787907\\
255	0.00638671657633152\\
256	0.00638669261745562\\
257	0.00638666825453377\\
258	0.00638664348073181\\
259	0.00638661828909679\\
260	0.0063865926725547\\
261	0.00638656662390817\\
262	0.00638654013583404\\
263	0.00638651320088097\\
264	0.00638648581146687\\
265	0.00638645795987635\\
266	0.0063864296382581\\
267	0.00638640083862214\\
268	0.00638637155283704\\
269	0.00638634177262707\\
270	0.00638631148956926\\
271	0.00638628069509043\\
272	0.00638624938046401\\
273	0.00638621753680686\\
274	0.00638618515507574\\
275	0.00638615222606409\\
276	0.00638611874039846\\
277	0.00638608468853491\\
278	0.00638605006075521\\
279	0.00638601484716301\\
280	0.00638597903767974\\
281	0.00638594262204055\\
282	0.00638590558978992\\
283	0.00638586793027723\\
284	0.00638582963265214\\
285	0.00638579068585975\\
286	0.00638575107863561\\
287	0.00638571079950053\\
288	0.00638566983675517\\
289	0.00638562817847436\\
290	0.00638558581250126\\
291	0.00638554272644116\\
292	0.00638549890765511\\
293	0.00638545434325314\\
294	0.00638540902008723\\
295	0.00638536292474393\\
296	0.00638531604353653\\
297	0.00638526836249695\\
298	0.00638521986736709\\
299	0.00638517054358977\\
300	0.00638512037629914\\
301	0.00638506935031051\\
302	0.00638501745010968\\
303	0.00638496465984148\\
304	0.00638491096329772\\
305	0.00638485634390431\\
306	0.00638480078470742\\
307	0.00638474426835884\\
308	0.00638468677710006\\
309	0.00638462829274554\\
310	0.00638456879666494\\
311	0.00638450826976433\\
312	0.00638444669246502\\
313	0.00638438404468094\\
314	0.00638432030579448\\
315	0.00638425545463028\\
316	0.00638418946942683\\
317	0.00638412232780574\\
318	0.00638405400673823\\
319	0.00638398448250878\\
320	0.00638391373067538\\
321	0.00638384172602621\\
322	0.00638376844253228\\
323	0.00638369385329557\\
324	0.00638361793049211\\
325	0.00638354064530964\\
326	0.00638346196787895\\
327	0.00638338186719835\\
328	0.00638330031105052\\
329	0.00638321726591067\\
330	0.00638313269684517\\
331	0.00638304656739947\\
332	0.00638295883947404\\
333	0.00638286947318688\\
334	0.0063827784267211\\
335	0.0063826856561557\\
336	0.00638259111527748\\
337	0.00638249475537197\\
338	0.00638239652499067\\
339	0.00638229636969169\\
340	0.00638219423175074\\
341	0.00638209004983853\\
342	0.00638198375866058\\
343	0.00638187528855478\\
344	0.00638176456504129\\
345	0.00638165150831882\\
346	0.00638153603270058\\
347	0.00638141804598205\\
348	0.00638129744873196\\
349	0.00638117413349642\\
350	0.00638104798390462\\
351	0.00638091887366286\\
352	0.00638078666541956\\
353	0.00638065120947726\\
354	0.00638051234231107\\
355	0.00638036988481699\\
356	0.00638022364013296\\
357	0.0063800733906508\\
358	0.00637991889324608\\
359	0.00637975987032702\\
360	0.0063795959907522\\
361	0.00637942682626216\\
362	0.00637925175035374\\
363	0.00637906970942778\\
364	0.00637887874041027\\
365	0.00637867510162692\\
366	0.00637845230977798\\
367	0.00637820317999826\\
368	0.0063779486827292\\
369	0.00637769033540312\\
370	0.00637742808668752\\
371	0.00637716188478615\\
372	0.00637689167741668\\
373	0.00637661741192001\\
374	0.00637633903573714\\
375	0.00637605649738284\\
376	0.00637576974673577\\
377	0.00637547873112774\\
378	0.00637518339603238\\
379	0.00637488368657652\\
380	0.00637457954752541\\
381	0.00637427092326166\\
382	0.00637395775775675\\
383	0.00637363999453385\\
384	0.00637331757662061\\
385	0.00637299044649023\\
386	0.0063726585459893\\
387	0.00637232181625021\\
388	0.00637198019758615\\
389	0.00637163362936644\\
390	0.00637128204986959\\
391	0.00637092539611196\\
392	0.00637056360364985\\
393	0.00637019660635467\\
394	0.00636982433616361\\
395	0.00636944672281565\\
396	0.00636906369359698\\
397	0.00636867517314809\\
398	0.00636828108343642\\
399	0.00636788134409195\\
400	0.00636747587349228\\
401	0.00636706459147168\\
402	0.00636664742611355\\
403	0.00636622433229439\\
404	0.00636579534338305\\
405	0.00636536069391904\\
406	0.00636492099227602\\
407	0.00636447744632584\\
408	0.00636403184281951\\
409	0.00636358533742221\\
410	0.0063631352141959\\
411	0.00636267744808165\\
412	0.0063622118967298\\
413	0.00636173840340145\\
414	0.00636125680084869\\
415	0.00636076691629268\\
416	0.00636026857114308\\
417	0.00635976158068075\\
418	0.00635924575368311\\
419	0.00635872089199831\\
420	0.00635818679017809\\
421	0.00635764323535161\\
422	0.00635709000691684\\
423	0.00635652687590073\\
424	0.00635595360450518\\
425	0.00635536994562446\\
426	0.00635477564233225\\
427	0.00635417042733614\\
428	0.0063535540223977\\
429	0.00635292613771579\\
430	0.00635228647127053\\
431	0.00635163470812541\\
432	0.0063509705196839\\
433	0.0063502935628958\\
434	0.00634960347940539\\
435	0.00634889989462626\\
436	0.00634818241671315\\
437	0.00634745063537106\\
438	0.00634670412039079\\
439	0.00634594241973513\\
440	0.00634516505701116\\
441	0.00634437152861232\\
442	0.00634356130271134\\
443	0.00634273382722985\\
444	0.00634188855859463\\
445	0.0063410249902627\\
446	0.00634014249251388\\
447	0.00633924039810001\\
448	0.00633831800238986\\
449	0.00633737455658749\\
450	0.0063364092506127\\
451	0.00633542117093243\\
452	0.00633440919541388\\
453	0.00633337173729664\\
454	0.00633230615981661\\
455	0.00633120761774576\\
456	0.00633006763021697\\
457	0.00632887711133775\\
458	0.00632766282066212\\
459	0.00632642830815137\\
460	0.00632516878913098\\
461	0.00632387383674367\\
462	0.00632252232735194\\
463	0.00632108164693185\\
464	0.00631953543083207\\
465	0.00631779877024216\\
466	0.00631550642290275\\
467	0.00631181128253835\\
468	0.00630758251061971\\
469	0.0063032886687178\\
470	0.00629892670499869\\
471	0.00629449405557355\\
472	0.00628998981684965\\
473	0.00628541731536539\\
474	0.00628078894352176\\
475	0.00627613232540686\\
476	0.00627148627539327\\
477	0.00626684319508133\\
478	0.00626204858967383\\
479	0.00625711260959464\\
480	0.00625202665348447\\
481	0.00624677980986374\\
482	0.00624135926103719\\
483	0.00623574865960632\\
484	0.00622992393734656\\
485	0.00622384233034361\\
486	0.00621741401359264\\
487	0.00621043315997757\\
488	0.00620244420629681\\
489	0.00619400965776899\\
490	0.0061854524589268\\
491	0.00617676989141488\\
492	0.00616795912551411\\
493	0.0061590172263189\\
494	0.00614994117299156\\
495	0.00614072789386952\\
496	0.00613137429602339\\
497	0.00612187716581177\\
498	0.00611223259079458\\
499	0.00610243450128191\\
500	0.00609247530938651\\
501	0.00608234825131482\\
502	0.0060720425385402\\
503	0.00606153892911889\\
504	0.00605084367335183\\
505	0.00603994004356179\\
506	0.00602877154022212\\
507	0.00601721376170444\\
508	0.00600523421996853\\
509	0.00599317503690893\\
510	0.00598101128349881\\
511	0.00596867831489254\\
512	0.00595601896616613\\
513	0.00594273840786949\\
514	0.00592896458149047\\
515	0.00591514726359842\\
516	0.00590129512228386\\
517	0.00588743147298232\\
518	0.00587361595475236\\
519	0.00585999549864396\\
520	0.00584690599328163\\
521	0.00583500592475455\\
522	0.00582490792822357\\
523	0.0058146042098332\\
524	0.00580398987612161\\
525	0.00579273376978727\\
526	0.00577980378322902\\
527	0.00576202624003748\\
528	0.00572377841157942\\
529	0.00567699475830189\\
530	0.00562869022943325\\
531	0.0055786912893133\\
532	0.00552598034901514\\
533	0.00546614625759935\\
534	0.00539998269028576\\
535	0.00533380742098718\\
536	0.00526762319228078\\
537	0.00520142728789096\\
538	0.00513519630576685\\
539	0.00506883791671606\\
540	0.00500203947876236\\
541	0.00493320527137919\\
542	0.00486390184218159\\
543	0.00479565705149332\\
544	0.00472849590663815\\
545	0.00466244415217564\\
546	0.00459752828388643\\
547	0.0045337742286058\\
548	0.00447120240486943\\
549	0.00440981133989045\\
550	0.00434952421512032\\
551	0.00429001456373431\\
552	0.00423008609656117\\
553	0.0041662083985624\\
554	0.00410400778309653\\
555	0.00404418208447621\\
556	0.00398683441474163\\
557	0.00393197442332481\\
558	0.00387962678014643\\
559	0.00382986338433296\\
560	0.00378275613597533\\
561	0.00373847095787726\\
562	0.00369740504797728\\
563	0.00366052991955191\\
564	0.00363023942217948\\
565	0.00360500619487998\\
566	0.00352973838064425\\
567	0.00344272230220717\\
568	0.00334614191579287\\
569	0.00324691932347805\\
570	0.00314497980923681\\
571	0.0030401664970947\\
572	0.00293180965279257\\
573	0.00281238234143975\\
574	0.00264080565626875\\
575	0.00246652723666886\\
576	0.00228946683670179\\
577	0.00210943841006402\\
578	0.00192677268954282\\
579	0.00174132574584865\\
580	0.00155260207180459\\
581	0.00136033065535838\\
582	0.00116605697170137\\
583	0.000969668956010576\\
584	0.000770929493206139\\
585	0.000569036622337176\\
586	0.000361344392078551\\
587	0.000140168163487911\\
588	0\\
589	0\\
590	0\\
591	0\\
592	0\\
593	0\\
594	0\\
595	0\\
596	0\\
597	0\\
598	0\\
599	0\\
600	0\\
};
\addplot [color=mycolor13,solid,forget plot]
  table[row sep=crcr]{%
1	0.00142341042040566\\
2	0.00142341042040566\\
3	0.00142341042040566\\
4	0.00142341042040566\\
5	0.00142341042040566\\
6	0.00142341042040566\\
7	0.00142341042040566\\
8	0.00142341042040566\\
9	0.00142341042040566\\
10	0.00142341042040566\\
11	0.00142341042040566\\
12	0.00142341042040566\\
13	0.00142341042040566\\
14	0.00142341042040566\\
15	0.00142341042040566\\
16	0.00142341042040566\\
17	0.00142341042040566\\
18	0.00142341042040566\\
19	0.00142341042040566\\
20	0.00142341042040566\\
21	0.00142341042040566\\
22	0.00142341042040566\\
23	0.00142341042040566\\
24	0.00142341042040566\\
25	0.00142341042040566\\
26	0.00142341042040566\\
27	0.00142341042040566\\
28	0.00142341042040566\\
29	0.00142341042040566\\
30	0.00142341042040566\\
31	0.00142341042040566\\
32	0.00142341042040566\\
33	0.00142341042040566\\
34	0.00142341042040566\\
35	0.00142341042040566\\
36	0.00142341042040566\\
37	0.00142341042040566\\
38	0.00142341042040566\\
39	0.00142341042040566\\
40	0.00142341042040566\\
41	0.00142341042040566\\
42	0.00142341042040566\\
43	0.00142341042040566\\
44	0.00142341042040566\\
45	0.00142341042040566\\
46	0.00142341042040566\\
47	0.00142341042040566\\
48	0.00142341042040566\\
49	0.00142341042040566\\
50	0.00142341042040566\\
51	0.00142341042040566\\
52	0.00142341042040566\\
53	0.00142341042040566\\
54	0.00142341042040566\\
55	0.00142341042040566\\
56	0.00142341042040566\\
57	0.00142341042040566\\
58	0.00142341042040566\\
59	0.00142341042040566\\
60	0.00142341042040566\\
61	0.00142341042040566\\
62	0.00142341042040566\\
63	0.00142341042040566\\
64	0.00142341042040566\\
65	0.00142341042040566\\
66	0.00142341042040566\\
67	0.00142341042040566\\
68	0.00142341042040566\\
69	0.00142341042040566\\
70	0.00142341042040566\\
71	0.00142341042040566\\
72	0.00142341042040566\\
73	0.00142341042040566\\
74	0.00142341042040566\\
75	0.00142341042040566\\
76	0.00142341042040566\\
77	0.00142341042040566\\
78	0.00142341042040566\\
79	0.00142341042040566\\
80	0.00142341042040566\\
81	0.00142341042040566\\
82	0.00142341042040566\\
83	0.00142341042040566\\
84	0.00142341042040566\\
85	0.00142341042040566\\
86	0.00142341042040566\\
87	0.00142341042040566\\
88	0.00142341042040566\\
89	0.00142341042040566\\
90	0.00142341042040566\\
91	0.00142341042040566\\
92	0.00142341042040566\\
93	0.00142341042040566\\
94	0.00142341042040566\\
95	0.00142341042040566\\
96	0.00142341042040566\\
97	0.00142341042040566\\
98	0.00142341042040566\\
99	0.00142341042040566\\
100	0.00142341042040566\\
101	0.00142341042040566\\
102	0.00142341042040566\\
103	0.00142341042040566\\
104	0.00142341042040566\\
105	0.00142341042040566\\
106	0.00142341042040566\\
107	0.00142341042040566\\
108	0.00142341042040566\\
109	0.00142341042040566\\
110	0.00142341042040566\\
111	0.00142341042040566\\
112	0.00142341042040566\\
113	0.00142341042040566\\
114	0.00142341042040566\\
115	0.00142341042040566\\
116	0.00142341042040566\\
117	0.00142341042040566\\
118	0.00142341042040566\\
119	0.00142341042040566\\
120	0.00142341042040566\\
121	0.00142341042040566\\
122	0.00142341042040566\\
123	0.00142341042040566\\
124	0.00142341042040566\\
125	0.00142341042040566\\
126	0.00142341042040566\\
127	0.00142341042040566\\
128	0.00142341042040566\\
129	0.00142341042040566\\
130	0.00142341042040566\\
131	0.00142341042040566\\
132	0.00142341042040566\\
133	0.00142341042040566\\
134	0.00142341042040566\\
135	0.00142341042040566\\
136	0.00142341042040566\\
137	0.00142341042040566\\
138	0.00142341042040566\\
139	0.00142341042040566\\
140	0.00142341042040566\\
141	0.00142341042040566\\
142	0.00142341042040566\\
143	0.00142341042040566\\
144	0.00142341042040566\\
145	0.00142341042040566\\
146	0.00142341042040566\\
147	0.00142341042040566\\
148	0.00142341042040566\\
149	0.00142341042040566\\
150	0.00142341042040566\\
151	0.00142341042040566\\
152	0.00142341042040566\\
153	0.00142341042040566\\
154	0.00142341042040566\\
155	0.00142341042040566\\
156	0.00142341042040566\\
157	0.00142341042040566\\
158	0.00142341042040566\\
159	0.00142341042040566\\
160	0.00142341042040566\\
161	0.00142341042040566\\
162	0.00142341042040566\\
163	0.00142341042040566\\
164	0.00142341042040566\\
165	0.00142341042040566\\
166	0.00142341042040566\\
167	0.00142341042040566\\
168	0.00142341042040566\\
169	0.00142341042040566\\
170	0.00142341042040566\\
171	0.00142341042040566\\
172	0.00142341042040566\\
173	0.00142341042040566\\
174	0.00142341042040566\\
175	0.00142341042040566\\
176	0.00142341042040566\\
177	0.00142341042040566\\
178	0.00142341042040566\\
179	0.00142341042040566\\
180	0.00142341042040566\\
181	0.00142341042040566\\
182	0.00142341042040566\\
183	0.00142341042040566\\
184	0.00142341042040566\\
185	0.00142341042040566\\
186	0.00142341042040566\\
187	0.00142341042040566\\
188	0.00142341042040566\\
189	0.00142341042040566\\
190	0.00142341042040566\\
191	0.00142341042040566\\
192	0.00142341042040566\\
193	0.00142341042040566\\
194	0.00142341042040566\\
195	0.00142341042040566\\
196	0.00142341042040566\\
197	0.00142341042040566\\
198	0.00142341042040566\\
199	0.00142341042040566\\
200	0.00142341042040566\\
201	0.00142341042040566\\
202	0.00142341042040566\\
203	0.00142341042040566\\
204	0.00142341042040566\\
205	0.00142341042040566\\
206	0.00142341042040566\\
207	0.00142341042040566\\
208	0.00142341042040566\\
209	0.00142341042040566\\
210	0.00142341042040566\\
211	0.00142341042040566\\
212	0.00142341042040566\\
213	0.00142341042040566\\
214	0.00142341042040566\\
215	0.00142341042040566\\
216	0.00142341042040566\\
217	0.00142341042040566\\
218	0.00142341042040566\\
219	0.00142341042040566\\
220	0.00142341042040566\\
221	0.00142341042040566\\
222	0.00142341042040566\\
223	0.00142341042040566\\
224	0.00142341042040566\\
225	0.00142341042040566\\
226	0.00142341042040566\\
227	0.00142341042040566\\
228	0.00142341042040566\\
229	0.00142341042040566\\
230	0.00142341042040566\\
231	0.00142341042040566\\
232	0.00142341042040566\\
233	0.00142341042040566\\
234	0.00142341042040566\\
235	0.00142341042040566\\
236	0.00142341042040566\\
237	0.00142341042040566\\
238	0.00142341042040566\\
239	0.00142341042040566\\
240	0.00142341042040566\\
241	0.00142341042040566\\
242	0.00142341042040566\\
243	0.00142341042040566\\
244	0.00142341042040566\\
245	0.00142341042040566\\
246	0.00142341042040566\\
247	0.00142341042040566\\
248	0.00142341042040566\\
249	0.00142341042040566\\
250	0.00142341042040566\\
251	0.00142341042040566\\
252	0.00142341042040566\\
253	0.00142341042040566\\
254	0.00142341042040566\\
255	0.00142341042040566\\
256	0.00142341042040566\\
257	0.00142341042040566\\
258	0.00142341042040566\\
259	0.00142341042040566\\
260	0.00142341042040566\\
261	0.00142341042040566\\
262	0.00142341042040566\\
263	0.00142341042040566\\
264	0.00142341042040566\\
265	0.00142341042040566\\
266	0.00142341042040566\\
267	0.00142341042040566\\
268	0.00142341042040566\\
269	0.00142341042040566\\
270	0.00142341042040566\\
271	0.00142341042040566\\
272	0.00142341042040566\\
273	0.00142341042040566\\
274	0.00142341042040566\\
275	0.00142341042040566\\
276	0.00142341042040566\\
277	0.00142341042040566\\
278	0.00142341042040566\\
279	0.00142341042040566\\
280	0.00142341042040566\\
281	0.00142341042040566\\
282	0.00142341042040566\\
283	0.00142341042040566\\
284	0.00142341042040566\\
285	0.00142341042040566\\
286	0.00142341042040566\\
287	0.00142341042040566\\
288	0.00142341042040566\\
289	0.00142341042040566\\
290	0.00142341042040566\\
291	0.00142341042040566\\
292	0.00142341042040566\\
293	0.00142341042040566\\
294	0.00142341042040566\\
295	0.00142341042040566\\
296	0.00142341042040566\\
297	0.00142341042040566\\
298	0.00142341042040566\\
299	0.00142341042040566\\
300	0.00142341042040566\\
301	0.00142341042040566\\
302	0.00142341042040566\\
303	0.00142341042040566\\
304	0.00142341042040566\\
305	0.00142341042040566\\
306	0.00142341042040566\\
307	0.00142341042040566\\
308	0.00142341042040566\\
309	0.00142341042040566\\
310	0.00142341042040566\\
311	0.00142341042040566\\
312	0.00142341042040566\\
313	0.00142341042040566\\
314	0.00142341042040566\\
315	0.00142341042040566\\
316	0.00142341042040566\\
317	0.00142341042040566\\
318	0.00142341042040566\\
319	0.00142341042040566\\
320	0.00142341042040566\\
321	0.00142341042040566\\
322	0.00142341042040566\\
323	0.00142341042040566\\
324	0.00142341042040566\\
325	0.00142341042040566\\
326	0.00142341042040566\\
327	0.00142341042040566\\
328	0.00142341042040566\\
329	0.00142341042040566\\
330	0.00142341042040566\\
331	0.00142341042040566\\
332	0.00142341042040566\\
333	0.00142341042040566\\
334	0.00142341042040566\\
335	0.00142341042040566\\
336	0.00142341042040566\\
337	0.00142341042040566\\
338	0.00142341042040566\\
339	0.00142341042040566\\
340	0.00142341042040566\\
341	0.00142341042040566\\
342	0.00142341042040566\\
343	0.00142341042040566\\
344	0.00142341042040566\\
345	0.00142341042040566\\
346	0.00142341042040566\\
347	0.00142341042040566\\
348	0.00142341042040566\\
349	0.00142341042040566\\
350	0.00142341042040566\\
351	0.00142341042040566\\
352	0.00142341042040566\\
353	0.00142341042040566\\
354	0.00142341042040566\\
355	0.00142341042040566\\
356	0.00142341042040566\\
357	0.00142341042040566\\
358	0.00142341042040566\\
359	0.00142341042040566\\
360	0.00142341042040566\\
361	0.00142341042040566\\
362	0.00142341042040566\\
363	0.00142341042040566\\
364	0.00142341042040566\\
365	0.00142341042040566\\
366	0.00142341042040566\\
367	0.00142341042040566\\
368	0.00142341042040566\\
369	0.00142341042040566\\
370	0.00142341042040566\\
371	0.00142341042040566\\
372	0.00142341042040566\\
373	0.00142341042040566\\
374	0.00142341042040566\\
375	0.00142341042040566\\
376	0.00142341042040566\\
377	0.00142341042040566\\
378	0.00142341042040566\\
379	0.00142341042040566\\
380	0.00142341042040566\\
381	0.00142341042040566\\
382	0.00142341042040566\\
383	0.00142341042040566\\
384	0.00142341042040566\\
385	0.00142341042040566\\
386	0.00142341042040566\\
387	0.00142341042040566\\
388	0.00142341042040566\\
389	0.00142341042040566\\
390	0.00142341042040566\\
391	0.00142341042040566\\
392	0.00142341042040566\\
393	0.00142341042040566\\
394	0.00142341042040566\\
395	0.00142341042040566\\
396	0.00142341042040566\\
397	0.00142341042040566\\
398	0.00142341042040566\\
399	0.00142341042040566\\
400	0.00142341042040566\\
401	0.00142341042040566\\
402	0.00142341042040566\\
403	0.00142341042040566\\
404	0.00142341042040566\\
405	0.00142341042040566\\
406	0.00142341042040566\\
407	0.00142341042040566\\
408	0.00142341042040566\\
409	0.00142341042040566\\
410	0.00142341042040566\\
411	0.00142341042040566\\
412	0.00142341042040566\\
413	0.00142341042040566\\
414	0.00142341042040566\\
415	0.00142341042040566\\
416	0.00142341042040566\\
417	0.00142341042040566\\
418	0.00142341042040566\\
419	0.00142341042040566\\
420	0.00142341042040566\\
421	0.00142341042040566\\
422	0.00142341042040566\\
423	0.00142341042040566\\
424	0.00142341042040566\\
425	0.00142341042040566\\
426	0.00142341042040566\\
427	0.00142341042040566\\
428	0.00142341042040566\\
429	0.00142341042040566\\
430	0.00142341042040566\\
431	0.00142341042040566\\
432	0.00142341042040566\\
433	0.00142341042040566\\
434	0.00142341042040566\\
435	0.00142341042040566\\
436	0.00142341042040566\\
437	0.00142341042040566\\
438	0.00142341042040566\\
439	0.00142341042040566\\
440	0.00142341042040566\\
441	0.00142341042040566\\
442	0.00142341042040566\\
443	0.00142341042040566\\
444	0.00142341042040566\\
445	0.00142341042040566\\
446	0.00142341042040566\\
447	0.00142341042040566\\
448	0.00142341042040566\\
449	0.00142341042040566\\
450	0.00142341042040566\\
451	0.00142341042040566\\
452	0.00142341042040566\\
453	0.00142341042040566\\
454	0.00142341042040566\\
455	0.00142341042040566\\
456	0.00142341042040566\\
457	0.00142341042040566\\
458	0.00142341042040566\\
459	0.00142341042040566\\
460	0.00142341042040566\\
461	0.00142341042040566\\
462	0.00142341042040566\\
463	0.00142341042040566\\
464	0.00142341042040566\\
465	0.00142341042040566\\
466	0.00142341042040566\\
467	0.00142341042040566\\
468	0.00142341042040566\\
469	0.00142341042040566\\
470	0.00142341042040566\\
471	0.00142341042040566\\
472	0.00142341042040566\\
473	0.00142341042040566\\
474	0.00142341042040566\\
475	0.00142341042040566\\
476	0.00142341042040566\\
477	0.00142341042040566\\
478	0.00142341042040566\\
479	0.00142341042040566\\
480	0.00142341042040566\\
481	0.00142341042040566\\
482	0.00142341042040566\\
483	0.00142341042040566\\
484	0.00142341042040566\\
485	0.00142341042040566\\
486	0.00142341042040566\\
487	0.00142341042040566\\
488	0.00142341042040566\\
489	0.00142341042040566\\
490	0.00142341042040566\\
491	0.00142341042040566\\
492	0.00142341042040566\\
493	0.00142341042040566\\
494	0.00142341042040566\\
495	0.00142341042040566\\
496	0.00142341042040566\\
497	0.00142341042040566\\
498	0.00142341042040566\\
499	0.00142341042040566\\
500	0.00142341042040566\\
501	0.00142341042040566\\
502	0.00142341042040566\\
503	0.00142341042040566\\
504	0.00142341042040566\\
505	0.00142341042040566\\
506	0.00142341042040566\\
507	0.00142341042040566\\
508	0.00142341042040566\\
509	0.00142341042040566\\
510	0.00142341042040566\\
511	0.00142341042040566\\
512	0.00142341042040566\\
513	0.00142341042040566\\
514	0.00142341042040566\\
515	0.00142341042040566\\
516	0.00142341042040566\\
517	0.00142341042040566\\
518	0.00142341042040566\\
519	0.00142341042040566\\
520	0.00142341042040566\\
521	0.00142341042040566\\
522	0.00142341042040566\\
523	0.00142341042040566\\
524	0.00142341042040566\\
525	0.00142341042040566\\
526	0.00142341042040566\\
527	0.00142341042040566\\
528	0.00142341042040566\\
529	0.00142341042040566\\
530	0.00142341042040566\\
531	0.00142341042040566\\
532	0.00142341042040566\\
533	0.00142341042040566\\
534	0.00142341042040566\\
535	0.00142341042040566\\
536	0.00142341042040566\\
537	0.00142341042040566\\
538	0.00142341042040566\\
539	0.00142341042040566\\
540	0.00142341042040566\\
541	0.00142341042040566\\
542	0.00142341042040566\\
543	0.00142341042040566\\
544	0.00142341042040566\\
545	0.00142341042040566\\
546	0.00142341042040566\\
547	0.00142341042040566\\
548	0.00142341042040566\\
549	0.00142341042040566\\
550	0.00142341042040566\\
551	0.00142341042040566\\
552	0.00142341042040566\\
553	0.00142341042040566\\
554	0.00142341042040566\\
555	0.00142341042040566\\
556	0.00142341042040566\\
557	0.00142341042040566\\
558	0.00142341042040566\\
559	0.00142341042040566\\
560	0.00142341042040566\\
561	0.00142341042040566\\
562	0.00142341042040566\\
563	0.00142341042040566\\
564	0.00142341042040566\\
565	0.00141808680651778\\
566	0.00132573838914067\\
567	0.00121575937955735\\
568	0.00110234926539761\\
569	0.000991487125274636\\
570	0.000881523074320698\\
571	0.000772562041893994\\
572	0.000667166343829018\\
573	0.000574706660558508\\
574	0.000427188374771164\\
575	0.000287780647258003\\
576	0.000156402139129492\\
577	3.29294685952495e-05\\
578	0\\
579	0\\
580	0\\
581	0\\
582	0\\
583	0\\
584	0\\
585	0\\
586	0\\
587	0\\
588	0\\
589	0\\
590	0\\
591	0\\
592	5.56832960656924e-05\\
593	0.000164162311403634\\
594	0.00028999157753712\\
595	0.000432641173258269\\
596	0.000588326666482334\\
597	0.000749148955320047\\
598	0.00347642786857269\\
599	0\\
600	0\\
};
\addplot [color=mycolor14,solid,forget plot]
  table[row sep=crcr]{%
1	0\\
2	0\\
3	0\\
4	0\\
5	0\\
6	0\\
7	0\\
8	0\\
9	0\\
10	0\\
11	0\\
12	0\\
13	0\\
14	0\\
15	0\\
16	0\\
17	0\\
18	0\\
19	0\\
20	0\\
21	0\\
22	0\\
23	0\\
24	0\\
25	0\\
26	0\\
27	0\\
28	0\\
29	0\\
30	0\\
31	0\\
32	0\\
33	0\\
34	0\\
35	0\\
36	0\\
37	0\\
38	0\\
39	0\\
40	0\\
41	0\\
42	0\\
43	0\\
44	0\\
45	0\\
46	0\\
47	0\\
48	0\\
49	0\\
50	0\\
51	0\\
52	0\\
53	0\\
54	0\\
55	0\\
56	0\\
57	0\\
58	0\\
59	0\\
60	0\\
61	0\\
62	0\\
63	0\\
64	0\\
65	0\\
66	0\\
67	0\\
68	0\\
69	0\\
70	0\\
71	0\\
72	0\\
73	0\\
74	0\\
75	0\\
76	0\\
77	0\\
78	0\\
79	0\\
80	0\\
81	0\\
82	0\\
83	0\\
84	0\\
85	0\\
86	0\\
87	0\\
88	0\\
89	0\\
90	0\\
91	0\\
92	0\\
93	0\\
94	0\\
95	0\\
96	0\\
97	0\\
98	0\\
99	0\\
100	0\\
101	0\\
102	0\\
103	0\\
104	0\\
105	0\\
106	0\\
107	0\\
108	0\\
109	0\\
110	0\\
111	0\\
112	0\\
113	0\\
114	0\\
115	0\\
116	0\\
117	0\\
118	0\\
119	0\\
120	0\\
121	0\\
122	0\\
123	0\\
124	0\\
125	0\\
126	0\\
127	0\\
128	0\\
129	0\\
130	0\\
131	0\\
132	0\\
133	0\\
134	0\\
135	0\\
136	0\\
137	0\\
138	0\\
139	0\\
140	0\\
141	0\\
142	0\\
143	0\\
144	0\\
145	0\\
146	0\\
147	0\\
148	0\\
149	0\\
150	0\\
151	0\\
152	0\\
153	0\\
154	0\\
155	0\\
156	0\\
157	0\\
158	0\\
159	0\\
160	0\\
161	0\\
162	0\\
163	0\\
164	0\\
165	0\\
166	0\\
167	0\\
168	0\\
169	0\\
170	0\\
171	0\\
172	0\\
173	0\\
174	0\\
175	0\\
176	0\\
177	0\\
178	0\\
179	0\\
180	0\\
181	0\\
182	0\\
183	0\\
184	0\\
185	0\\
186	0\\
187	0\\
188	0\\
189	0\\
190	0\\
191	0\\
192	0\\
193	0\\
194	0\\
195	0\\
196	0\\
197	0\\
198	0\\
199	0\\
200	0\\
201	0\\
202	0\\
203	0\\
204	0\\
205	0\\
206	0\\
207	0\\
208	0\\
209	0\\
210	0\\
211	0\\
212	0\\
213	0\\
214	0\\
215	0\\
216	0\\
217	0\\
218	0\\
219	0\\
220	0\\
221	0\\
222	0\\
223	0\\
224	0\\
225	0\\
226	0\\
227	0\\
228	0\\
229	0\\
230	0\\
231	0\\
232	0\\
233	0\\
234	0\\
235	0\\
236	0\\
237	0\\
238	0\\
239	0\\
240	0\\
241	0\\
242	0\\
243	0\\
244	0\\
245	0\\
246	0\\
247	0\\
248	0\\
249	0\\
250	0\\
251	0\\
252	0\\
253	0\\
254	0\\
255	0\\
256	0\\
257	0\\
258	0\\
259	0\\
260	0\\
261	0\\
262	0\\
263	0\\
264	0\\
265	0\\
266	0\\
267	0\\
268	0\\
269	0\\
270	0\\
271	0\\
272	0\\
273	0\\
274	0\\
275	0\\
276	0\\
277	0\\
278	0\\
279	0\\
280	0\\
281	0\\
282	0\\
283	0\\
284	0\\
285	0\\
286	0\\
287	0\\
288	0\\
289	0\\
290	0\\
291	0\\
292	0\\
293	0\\
294	0\\
295	0\\
296	0\\
297	0\\
298	0\\
299	0\\
300	0\\
301	0\\
302	0\\
303	0\\
304	0\\
305	0\\
306	0\\
307	0\\
308	0\\
309	0\\
310	0\\
311	0\\
312	0\\
313	0\\
314	0\\
315	0\\
316	0\\
317	0\\
318	0\\
319	0\\
320	0\\
321	0\\
322	0\\
323	0\\
324	0\\
325	0\\
326	0\\
327	0\\
328	0\\
329	0\\
330	0\\
331	0\\
332	0\\
333	0\\
334	0\\
335	0\\
336	0\\
337	0\\
338	0\\
339	0\\
340	0\\
341	0\\
342	0\\
343	0\\
344	0\\
345	0\\
346	0\\
347	0\\
348	0\\
349	0\\
350	0\\
351	0\\
352	0\\
353	0\\
354	0\\
355	0\\
356	0\\
357	0\\
358	0\\
359	0\\
360	0\\
361	0\\
362	0\\
363	0\\
364	0\\
365	0\\
366	0\\
367	0\\
368	0\\
369	0\\
370	0\\
371	0\\
372	0\\
373	0\\
374	0\\
375	0\\
376	0\\
377	0\\
378	0\\
379	0\\
380	0\\
381	0\\
382	0\\
383	0\\
384	0\\
385	0\\
386	0\\
387	0\\
388	0\\
389	0\\
390	0\\
391	0\\
392	0\\
393	0\\
394	0\\
395	0\\
396	0\\
397	0\\
398	0\\
399	0\\
400	0\\
401	0\\
402	0\\
403	0\\
404	0\\
405	0\\
406	0\\
407	0\\
408	0\\
409	0\\
410	0\\
411	0\\
412	0\\
413	0\\
414	0\\
415	0\\
416	0\\
417	0\\
418	0\\
419	0\\
420	0\\
421	0\\
422	0\\
423	0\\
424	0\\
425	0\\
426	0\\
427	0\\
428	0\\
429	0\\
430	0\\
431	0\\
432	0\\
433	0\\
434	0\\
435	0\\
436	0\\
437	0\\
438	0\\
439	0\\
440	0\\
441	0\\
442	0\\
443	0\\
444	0\\
445	0\\
446	0\\
447	0\\
448	0\\
449	0\\
450	0\\
451	0\\
452	0\\
453	0\\
454	0\\
455	0\\
456	0\\
457	0\\
458	0\\
459	0\\
460	0\\
461	0\\
462	0\\
463	0\\
464	0\\
465	0\\
466	0\\
467	0\\
468	0\\
469	0\\
470	0\\
471	0\\
472	0\\
473	0\\
474	0\\
475	0\\
476	0\\
477	0\\
478	0\\
479	0\\
480	0\\
481	0\\
482	0\\
483	0\\
484	0\\
485	0\\
486	0\\
487	0\\
488	0\\
489	0\\
490	0\\
491	0\\
492	0\\
493	0\\
494	0\\
495	0\\
496	0\\
497	0\\
498	0\\
499	0\\
500	0\\
501	0\\
502	0\\
503	0\\
504	0\\
505	0\\
506	0\\
507	0\\
508	0\\
509	0\\
510	0\\
511	0\\
512	0\\
513	0\\
514	0\\
515	0\\
516	0\\
517	0\\
518	0\\
519	0\\
520	0\\
521	0\\
522	0\\
523	0\\
524	0\\
525	0\\
526	0\\
527	0\\
528	0\\
529	0\\
530	0\\
531	0\\
532	0\\
533	0\\
534	0\\
535	0\\
536	0\\
537	0\\
538	0\\
539	0\\
540	0\\
541	0\\
542	0\\
543	0\\
544	0\\
545	0\\
546	0\\
547	0\\
548	0\\
549	0\\
550	0\\
551	0\\
552	0\\
553	0\\
554	0\\
555	0\\
556	0\\
557	0\\
558	0\\
559	0\\
560	0\\
561	0\\
562	0\\
563	0\\
564	0\\
565	0\\
566	0\\
567	0\\
568	0\\
569	0\\
570	0\\
571	0\\
572	0\\
573	0\\
574	0\\
575	0\\
576	0\\
577	0\\
578	0\\
579	0\\
580	0\\
581	0\\
582	0\\
583	0\\
584	0\\
585	0.000134515837093855\\
586	0.000281838017575936\\
587	0.000435050286072654\\
588	0.000594542785312444\\
589	0.0007604561023499\\
590	0.000933270892818283\\
591	0.00111362236991578\\
592	0.00130217091110698\\
593	0.00149974464812233\\
594	0.00170715331224199\\
595	0.00192583625726673\\
596	0.00215816651678185\\
597	0.00340423165914315\\
598	0.00644286460810295\\
599	0\\
600	0\\
};
\addplot [color=mycolor15,solid,forget plot]
  table[row sep=crcr]{%
1	0\\
2	0\\
3	0\\
4	0\\
5	0\\
6	0\\
7	0\\
8	0\\
9	0\\
10	0\\
11	0\\
12	0\\
13	0\\
14	0\\
15	0\\
16	0\\
17	0\\
18	0\\
19	0\\
20	0\\
21	0\\
22	0\\
23	0\\
24	0\\
25	0\\
26	0\\
27	0\\
28	0\\
29	0\\
30	0\\
31	0\\
32	0\\
33	0\\
34	0\\
35	0\\
36	0\\
37	0\\
38	0\\
39	0\\
40	0\\
41	0\\
42	0\\
43	0\\
44	0\\
45	0\\
46	0\\
47	0\\
48	0\\
49	0\\
50	0\\
51	0\\
52	0\\
53	0\\
54	0\\
55	0\\
56	0\\
57	0\\
58	0\\
59	0\\
60	0\\
61	0\\
62	0\\
63	0\\
64	0\\
65	0\\
66	0\\
67	0\\
68	0\\
69	0\\
70	0\\
71	0\\
72	0\\
73	0\\
74	0\\
75	0\\
76	0\\
77	0\\
78	0\\
79	0\\
80	0\\
81	0\\
82	0\\
83	0\\
84	0\\
85	0\\
86	0\\
87	0\\
88	0\\
89	0\\
90	0\\
91	0\\
92	0\\
93	0\\
94	0\\
95	0\\
96	0\\
97	0\\
98	0\\
99	0\\
100	0\\
101	0\\
102	0\\
103	0\\
104	0\\
105	0\\
106	0\\
107	0\\
108	0\\
109	0\\
110	0\\
111	0\\
112	0\\
113	0\\
114	0\\
115	0\\
116	0\\
117	0\\
118	0\\
119	0\\
120	0\\
121	0\\
122	0\\
123	0\\
124	0\\
125	0\\
126	0\\
127	0\\
128	0\\
129	0\\
130	0\\
131	0\\
132	0\\
133	0\\
134	0\\
135	0\\
136	0\\
137	0\\
138	0\\
139	0\\
140	0\\
141	0\\
142	0\\
143	0\\
144	0\\
145	0\\
146	0\\
147	0\\
148	0\\
149	0\\
150	0\\
151	0\\
152	0\\
153	0\\
154	0\\
155	0\\
156	0\\
157	0\\
158	0\\
159	0\\
160	0\\
161	0\\
162	0\\
163	0\\
164	0\\
165	0\\
166	0\\
167	0\\
168	0\\
169	0\\
170	0\\
171	0\\
172	0\\
173	0\\
174	0\\
175	0\\
176	0\\
177	0\\
178	0\\
179	0\\
180	0\\
181	0\\
182	0\\
183	0\\
184	0\\
185	0\\
186	0\\
187	0\\
188	0\\
189	0\\
190	0\\
191	0\\
192	0\\
193	0\\
194	0\\
195	0\\
196	0\\
197	0\\
198	0\\
199	0\\
200	0\\
201	0\\
202	0\\
203	0\\
204	0\\
205	0\\
206	0\\
207	0\\
208	0\\
209	0\\
210	0\\
211	0\\
212	0\\
213	0\\
214	0\\
215	0\\
216	0\\
217	0\\
218	0\\
219	0\\
220	0\\
221	0\\
222	0\\
223	0\\
224	0\\
225	0\\
226	0\\
227	0\\
228	0\\
229	0\\
230	0\\
231	0\\
232	0\\
233	0\\
234	0\\
235	0\\
236	0\\
237	0\\
238	0\\
239	0\\
240	0\\
241	0\\
242	0\\
243	0\\
244	0\\
245	0\\
246	0\\
247	0\\
248	0\\
249	0\\
250	0\\
251	0\\
252	0\\
253	0\\
254	0\\
255	0\\
256	0\\
257	0\\
258	0\\
259	0\\
260	0\\
261	0\\
262	0\\
263	0\\
264	0\\
265	0\\
266	0\\
267	0\\
268	0\\
269	0\\
270	0\\
271	0\\
272	0\\
273	0\\
274	0\\
275	0\\
276	0\\
277	0\\
278	0\\
279	0\\
280	0\\
281	0\\
282	0\\
283	0\\
284	0\\
285	0\\
286	0\\
287	0\\
288	0\\
289	0\\
290	0\\
291	0\\
292	0\\
293	0\\
294	0\\
295	0\\
296	0\\
297	0\\
298	0\\
299	0\\
300	0\\
301	0\\
302	0\\
303	0\\
304	0\\
305	0\\
306	0\\
307	0\\
308	0\\
309	0\\
310	0\\
311	0\\
312	0\\
313	0\\
314	0\\
315	0\\
316	0\\
317	0\\
318	0\\
319	0\\
320	0\\
321	0\\
322	0\\
323	0\\
324	0\\
325	0\\
326	0\\
327	0\\
328	0\\
329	0\\
330	0\\
331	0\\
332	0\\
333	0\\
334	0\\
335	0\\
336	0\\
337	0\\
338	0\\
339	0\\
340	0\\
341	0\\
342	0\\
343	0\\
344	0\\
345	0\\
346	0\\
347	0\\
348	0\\
349	0\\
350	0\\
351	0\\
352	0\\
353	0\\
354	0\\
355	0\\
356	0\\
357	0\\
358	0\\
359	0\\
360	0\\
361	0\\
362	0\\
363	0\\
364	0\\
365	0\\
366	0\\
367	0\\
368	0\\
369	0\\
370	0\\
371	0\\
372	0\\
373	0\\
374	0\\
375	0\\
376	0\\
377	0\\
378	0\\
379	0\\
380	0\\
381	0\\
382	0\\
383	0\\
384	0\\
385	0\\
386	0\\
387	0\\
388	0\\
389	0\\
390	0\\
391	0\\
392	0\\
393	0\\
394	0\\
395	0\\
396	0\\
397	0\\
398	0\\
399	0\\
400	0\\
401	0\\
402	0\\
403	0\\
404	0\\
405	0\\
406	0\\
407	0\\
408	0\\
409	0\\
410	0\\
411	0\\
412	0\\
413	0\\
414	0\\
415	0\\
416	0\\
417	0\\
418	0\\
419	0\\
420	0\\
421	0\\
422	0\\
423	0\\
424	0\\
425	0\\
426	0\\
427	0\\
428	0\\
429	0\\
430	0\\
431	0\\
432	0\\
433	0\\
434	0\\
435	0\\
436	0\\
437	0\\
438	0\\
439	0\\
440	0\\
441	0\\
442	0\\
443	0\\
444	0\\
445	0\\
446	0\\
447	0\\
448	0\\
449	0\\
450	0\\
451	0\\
452	0\\
453	0\\
454	0\\
455	0\\
456	0\\
457	0\\
458	0\\
459	0\\
460	0\\
461	0\\
462	0\\
463	0\\
464	0\\
465	0\\
466	0\\
467	0\\
468	0\\
469	0\\
470	0\\
471	0\\
472	0\\
473	0\\
474	0\\
475	0\\
476	0\\
477	0\\
478	0\\
479	0\\
480	0\\
481	0\\
482	0\\
483	0\\
484	0\\
485	0\\
486	0\\
487	0\\
488	0\\
489	0\\
490	0\\
491	0\\
492	0\\
493	0\\
494	0\\
495	0\\
496	0\\
497	0\\
498	0\\
499	0\\
500	0\\
501	0\\
502	0\\
503	0\\
504	0\\
505	0\\
506	0\\
507	0\\
508	0\\
509	0\\
510	0\\
511	0\\
512	0\\
513	0\\
514	0\\
515	0\\
516	0\\
517	0\\
518	0\\
519	0\\
520	0\\
521	0\\
522	0\\
523	0\\
524	0\\
525	0\\
526	0\\
527	0\\
528	0\\
529	0\\
530	0\\
531	0\\
532	0\\
533	0\\
534	0\\
535	0\\
536	0\\
537	0\\
538	0\\
539	0\\
540	0\\
541	0\\
542	0\\
543	0\\
544	0\\
545	0\\
546	0\\
547	0\\
548	0\\
549	0\\
550	0\\
551	0\\
552	0\\
553	0\\
554	0\\
555	0\\
556	0\\
557	0\\
558	0\\
559	0\\
560	0\\
561	0\\
562	0\\
563	0\\
564	0\\
565	0\\
566	0\\
567	0\\
568	0\\
569	0\\
570	0\\
571	0\\
572	0\\
573	2.84550322669103e-05\\
574	0.000136105490885286\\
575	0.000246360904768022\\
576	0.00035920980371322\\
577	0.000474576836408107\\
578	0.000592251873280648\\
579	0.000712460866895543\\
580	0.000835098336304564\\
581	0.000931987463067577\\
582	0.00102435477630977\\
583	0.00111851701928916\\
584	0.00121497734033712\\
585	0.00131373271689993\\
586	0.00141477870147017\\
587	0.00151807891510166\\
588	0.00162355054108618\\
589	0.00173106951191172\\
590	0.00184045717402978\\
591	0.00195146461619112\\
592	0.00206375309347527\\
593	0.00217686668353044\\
594	0.00229019015194498\\
595	0.00264340599857734\\
596	0.00335667832726079\\
597	0.00469756192662931\\
598	0.00644286460810295\\
599	0\\
600	0\\
};
\addplot [color=mycolor16,solid,forget plot]
  table[row sep=crcr]{%
1	0\\
2	0\\
3	0\\
4	0\\
5	0\\
6	0\\
7	0\\
8	0\\
9	0\\
10	0\\
11	0\\
12	0\\
13	0\\
14	0\\
15	0\\
16	0\\
17	0\\
18	0\\
19	0\\
20	0\\
21	0\\
22	0\\
23	0\\
24	0\\
25	0\\
26	0\\
27	0\\
28	0\\
29	0\\
30	0\\
31	0\\
32	0\\
33	0\\
34	0\\
35	0\\
36	0\\
37	0\\
38	0\\
39	0\\
40	0\\
41	0\\
42	0\\
43	0\\
44	0\\
45	0\\
46	0\\
47	0\\
48	0\\
49	0\\
50	0\\
51	0\\
52	0\\
53	0\\
54	0\\
55	0\\
56	0\\
57	0\\
58	0\\
59	0\\
60	0\\
61	0\\
62	0\\
63	0\\
64	0\\
65	0\\
66	0\\
67	0\\
68	0\\
69	0\\
70	0\\
71	0\\
72	0\\
73	0\\
74	0\\
75	0\\
76	0\\
77	0\\
78	0\\
79	0\\
80	0\\
81	0\\
82	0\\
83	0\\
84	0\\
85	0\\
86	0\\
87	0\\
88	0\\
89	0\\
90	0\\
91	0\\
92	0\\
93	0\\
94	0\\
95	0\\
96	0\\
97	0\\
98	0\\
99	0\\
100	0\\
101	0\\
102	0\\
103	0\\
104	0\\
105	0\\
106	0\\
107	0\\
108	0\\
109	0\\
110	0\\
111	0\\
112	0\\
113	0\\
114	0\\
115	0\\
116	0\\
117	0\\
118	0\\
119	0\\
120	0\\
121	0\\
122	0\\
123	0\\
124	0\\
125	0\\
126	0\\
127	0\\
128	0\\
129	0\\
130	0\\
131	0\\
132	0\\
133	0\\
134	0\\
135	0\\
136	0\\
137	0\\
138	0\\
139	0\\
140	0\\
141	0\\
142	0\\
143	0\\
144	0\\
145	0\\
146	0\\
147	0\\
148	0\\
149	0\\
150	0\\
151	0\\
152	0\\
153	0\\
154	0\\
155	0\\
156	0\\
157	0\\
158	0\\
159	0\\
160	0\\
161	0\\
162	0\\
163	0\\
164	0\\
165	0\\
166	0\\
167	0\\
168	0\\
169	0\\
170	0\\
171	0\\
172	0\\
173	0\\
174	0\\
175	0\\
176	0\\
177	0\\
178	0\\
179	0\\
180	0\\
181	0\\
182	0\\
183	0\\
184	0\\
185	0\\
186	0\\
187	0\\
188	0\\
189	0\\
190	0\\
191	0\\
192	0\\
193	0\\
194	0\\
195	0\\
196	0\\
197	0\\
198	0\\
199	0\\
200	0\\
201	0\\
202	0\\
203	0\\
204	0\\
205	0\\
206	0\\
207	0\\
208	0\\
209	0\\
210	0\\
211	0\\
212	0\\
213	0\\
214	0\\
215	0\\
216	0\\
217	0\\
218	0\\
219	0\\
220	0\\
221	0\\
222	0\\
223	0\\
224	0\\
225	0\\
226	0\\
227	0\\
228	0\\
229	0\\
230	0\\
231	0\\
232	0\\
233	0\\
234	0\\
235	0\\
236	0\\
237	0\\
238	0\\
239	0\\
240	0\\
241	0\\
242	0\\
243	0\\
244	0\\
245	0\\
246	0\\
247	0\\
248	0\\
249	0\\
250	0\\
251	0\\
252	0\\
253	0\\
254	0\\
255	0\\
256	0\\
257	0\\
258	0\\
259	0\\
260	0\\
261	0\\
262	0\\
263	0\\
264	0\\
265	0\\
266	0\\
267	0\\
268	0\\
269	0\\
270	0\\
271	0\\
272	0\\
273	0\\
274	0\\
275	0\\
276	0\\
277	0\\
278	0\\
279	0\\
280	0\\
281	0\\
282	0\\
283	0\\
284	0\\
285	0\\
286	0\\
287	0\\
288	0\\
289	0\\
290	0\\
291	0\\
292	0\\
293	0\\
294	0\\
295	0\\
296	0\\
297	0\\
298	0\\
299	0\\
300	0\\
301	0\\
302	0\\
303	0\\
304	0\\
305	0\\
306	0\\
307	0\\
308	0\\
309	0\\
310	0\\
311	0\\
312	0\\
313	0\\
314	0\\
315	0\\
316	0\\
317	0\\
318	0\\
319	0\\
320	0\\
321	0\\
322	0\\
323	0\\
324	0\\
325	0\\
326	0\\
327	0\\
328	0\\
329	0\\
330	0\\
331	0\\
332	0\\
333	0\\
334	0\\
335	0\\
336	0\\
337	0\\
338	0\\
339	0\\
340	0\\
341	0\\
342	0\\
343	0\\
344	0\\
345	0\\
346	0\\
347	0\\
348	0\\
349	0\\
350	0\\
351	0\\
352	0\\
353	0\\
354	0\\
355	0\\
356	0\\
357	0\\
358	0\\
359	0\\
360	0\\
361	0\\
362	0\\
363	0\\
364	0\\
365	0\\
366	0\\
367	0\\
368	0\\
369	0\\
370	0\\
371	0\\
372	0\\
373	0\\
374	0\\
375	0\\
376	0\\
377	0\\
378	0\\
379	0\\
380	0\\
381	0\\
382	0\\
383	0\\
384	0\\
385	0\\
386	0\\
387	0\\
388	0\\
389	0\\
390	0\\
391	0\\
392	0\\
393	0\\
394	0\\
395	0\\
396	0\\
397	0\\
398	0\\
399	0\\
400	0\\
401	0\\
402	0\\
403	0\\
404	0\\
405	0\\
406	0\\
407	0\\
408	0\\
409	0\\
410	0\\
411	0\\
412	0\\
413	0\\
414	0\\
415	0\\
416	0\\
417	0\\
418	0\\
419	0\\
420	0\\
421	0\\
422	0\\
423	0\\
424	0\\
425	0\\
426	0\\
427	0\\
428	0\\
429	0\\
430	0\\
431	0\\
432	0\\
433	0\\
434	0\\
435	0\\
436	0\\
437	0\\
438	0\\
439	0\\
440	0\\
441	0\\
442	0\\
443	0\\
444	0\\
445	0\\
446	0\\
447	0\\
448	0\\
449	0\\
450	0\\
451	0\\
452	0\\
453	0\\
454	0\\
455	0\\
456	0\\
457	0\\
458	0\\
459	0\\
460	0\\
461	0\\
462	0\\
463	0\\
464	0\\
465	0\\
466	0\\
467	0\\
468	0\\
469	0\\
470	0\\
471	0\\
472	0\\
473	0\\
474	0\\
475	0\\
476	0\\
477	0\\
478	0\\
479	0\\
480	0\\
481	0\\
482	0\\
483	0\\
484	0\\
485	0\\
486	0\\
487	0\\
488	0\\
489	0\\
490	0\\
491	0\\
492	0\\
493	0\\
494	0\\
495	0\\
496	0\\
497	0\\
498	0\\
499	0\\
500	0\\
501	0\\
502	0\\
503	0\\
504	0\\
505	0\\
506	0\\
507	0\\
508	0\\
509	0\\
510	0\\
511	0\\
512	0\\
513	0\\
514	0\\
515	0\\
516	0\\
517	0\\
518	0\\
519	0\\
520	0\\
521	0\\
522	0\\
523	0\\
524	0\\
525	0\\
526	0\\
527	0\\
528	0\\
529	0\\
530	0\\
531	0\\
532	0\\
533	0\\
534	0\\
535	0\\
536	0\\
537	0\\
538	0\\
539	0\\
540	0\\
541	0\\
542	0\\
543	0\\
544	0\\
545	0\\
546	0\\
547	0\\
548	0\\
549	0\\
550	0\\
551	0\\
552	0\\
553	0\\
554	0\\
555	0\\
556	0\\
557	0\\
558	0\\
559	0\\
560	0\\
561	0\\
562	0\\
563	0\\
564	6.11241591688388e-05\\
565	0.000153833609333673\\
566	0.000247621371611974\\
567	0.000342275882838338\\
568	0.000426454386023444\\
569	0.000490146502349677\\
570	0.000554714051789001\\
571	0.000620092626673308\\
572	0.000686241977351121\\
573	0.000753338581932168\\
574	0.000821311602791528\\
575	0.000890073230408475\\
576	0.000959515814844145\\
577	0.0010295081722191\\
578	0.0010998907600857\\
579	0.00117046937600367\\
580	0.00124101035188772\\
581	0.00131279084750798\\
582	0.00138613243267675\\
583	0.00146105292021269\\
584	0.00153756685550637\\
585	0.00161568839009441\\
586	0.00169543074241313\\
587	0.00177681010814605\\
588	0.00185987394737956\\
589	0.00194472985818468\\
590	0.00203176867243585\\
591	0.00218539535278842\\
592	0.00246148836585481\\
593	0.00274934460856801\\
594	0.00309920221510837\\
595	0.0035236770965596\\
596	0.00425769487060698\\
597	0.00503983077166121\\
598	0.00644286460810295\\
599	0\\
600	0\\
};
\addplot [color=mycolor17,solid,forget plot]
  table[row sep=crcr]{%
1	0\\
2	0\\
3	0\\
4	0\\
5	0\\
6	0\\
7	0\\
8	0\\
9	0\\
10	0\\
11	0\\
12	0\\
13	0\\
14	0\\
15	0\\
16	0\\
17	0\\
18	0\\
19	0\\
20	0\\
21	0\\
22	0\\
23	0\\
24	0\\
25	0\\
26	0\\
27	0\\
28	0\\
29	0\\
30	0\\
31	0\\
32	0\\
33	0\\
34	0\\
35	0\\
36	0\\
37	0\\
38	0\\
39	0\\
40	0\\
41	0\\
42	0\\
43	0\\
44	0\\
45	0\\
46	0\\
47	0\\
48	0\\
49	0\\
50	0\\
51	0\\
52	0\\
53	0\\
54	0\\
55	0\\
56	0\\
57	0\\
58	0\\
59	0\\
60	0\\
61	0\\
62	0\\
63	0\\
64	0\\
65	0\\
66	0\\
67	0\\
68	0\\
69	0\\
70	0\\
71	0\\
72	0\\
73	0\\
74	0\\
75	0\\
76	0\\
77	0\\
78	0\\
79	0\\
80	0\\
81	0\\
82	0\\
83	0\\
84	0\\
85	0\\
86	0\\
87	0\\
88	0\\
89	0\\
90	0\\
91	0\\
92	0\\
93	0\\
94	0\\
95	0\\
96	0\\
97	0\\
98	0\\
99	0\\
100	0\\
101	0\\
102	0\\
103	0\\
104	0\\
105	0\\
106	0\\
107	0\\
108	0\\
109	0\\
110	0\\
111	0\\
112	0\\
113	0\\
114	0\\
115	0\\
116	0\\
117	0\\
118	0\\
119	0\\
120	0\\
121	0\\
122	0\\
123	0\\
124	0\\
125	0\\
126	0\\
127	0\\
128	0\\
129	0\\
130	0\\
131	0\\
132	0\\
133	0\\
134	0\\
135	0\\
136	0\\
137	0\\
138	0\\
139	0\\
140	0\\
141	0\\
142	0\\
143	0\\
144	0\\
145	0\\
146	0\\
147	0\\
148	0\\
149	0\\
150	0\\
151	0\\
152	0\\
153	0\\
154	0\\
155	0\\
156	0\\
157	0\\
158	0\\
159	0\\
160	0\\
161	0\\
162	0\\
163	0\\
164	0\\
165	0\\
166	0\\
167	0\\
168	0\\
169	0\\
170	0\\
171	0\\
172	0\\
173	0\\
174	0\\
175	0\\
176	0\\
177	0\\
178	0\\
179	0\\
180	0\\
181	0\\
182	0\\
183	0\\
184	0\\
185	0\\
186	0\\
187	0\\
188	0\\
189	0\\
190	0\\
191	0\\
192	0\\
193	0\\
194	0\\
195	0\\
196	0\\
197	0\\
198	0\\
199	0\\
200	0\\
201	0\\
202	0\\
203	0\\
204	0\\
205	0\\
206	0\\
207	0\\
208	0\\
209	0\\
210	0\\
211	0\\
212	0\\
213	0\\
214	0\\
215	0\\
216	0\\
217	0\\
218	0\\
219	0\\
220	0\\
221	0\\
222	0\\
223	0\\
224	0\\
225	0\\
226	0\\
227	0\\
228	0\\
229	0\\
230	0\\
231	0\\
232	0\\
233	0\\
234	0\\
235	0\\
236	0\\
237	0\\
238	0\\
239	0\\
240	0\\
241	0\\
242	0\\
243	0\\
244	0\\
245	0\\
246	0\\
247	0\\
248	0\\
249	0\\
250	0\\
251	0\\
252	0\\
253	0\\
254	0\\
255	0\\
256	0\\
257	0\\
258	0\\
259	0\\
260	0\\
261	0\\
262	0\\
263	0\\
264	0\\
265	0\\
266	0\\
267	0\\
268	0\\
269	0\\
270	0\\
271	0\\
272	0\\
273	0\\
274	0\\
275	0\\
276	0\\
277	0\\
278	0\\
279	0\\
280	0\\
281	0\\
282	0\\
283	0\\
284	0\\
285	0\\
286	0\\
287	0\\
288	0\\
289	0\\
290	0\\
291	0\\
292	0\\
293	0\\
294	0\\
295	0\\
296	0\\
297	0\\
298	0\\
299	0\\
300	0\\
301	0\\
302	0\\
303	0\\
304	0\\
305	0\\
306	0\\
307	0\\
308	0\\
309	0\\
310	0\\
311	0\\
312	0\\
313	0\\
314	0\\
315	0\\
316	0\\
317	0\\
318	0\\
319	0\\
320	0\\
321	0\\
322	0\\
323	0\\
324	0\\
325	0\\
326	0\\
327	0\\
328	0\\
329	0\\
330	0\\
331	0\\
332	0\\
333	0\\
334	0\\
335	0\\
336	0\\
337	0\\
338	0\\
339	0\\
340	0\\
341	0\\
342	0\\
343	0\\
344	0\\
345	0\\
346	0\\
347	0\\
348	0\\
349	0\\
350	0\\
351	0\\
352	0\\
353	0\\
354	0\\
355	0\\
356	0\\
357	0\\
358	0\\
359	0\\
360	0\\
361	0\\
362	0\\
363	0\\
364	0\\
365	0\\
366	0\\
367	0\\
368	0\\
369	0\\
370	0\\
371	0\\
372	0\\
373	0\\
374	0\\
375	0\\
376	0\\
377	0\\
378	0\\
379	0\\
380	0\\
381	0\\
382	0\\
383	0\\
384	0\\
385	0\\
386	0\\
387	0\\
388	0\\
389	0\\
390	0\\
391	0\\
392	0\\
393	0\\
394	0\\
395	0\\
396	0\\
397	0\\
398	0\\
399	0\\
400	0\\
401	0\\
402	0\\
403	0\\
404	0\\
405	0\\
406	0\\
407	0\\
408	0\\
409	0\\
410	0\\
411	0\\
412	0\\
413	0\\
414	0\\
415	0\\
416	0\\
417	0\\
418	0\\
419	0\\
420	0\\
421	0\\
422	0\\
423	0\\
424	0\\
425	0\\
426	0\\
427	0\\
428	0\\
429	0\\
430	0\\
431	0\\
432	0\\
433	0\\
434	0\\
435	0\\
436	0\\
437	0\\
438	0\\
439	0\\
440	0\\
441	0\\
442	0\\
443	0\\
444	0\\
445	0\\
446	0\\
447	0\\
448	0\\
449	0\\
450	0\\
451	0\\
452	0\\
453	0\\
454	0\\
455	0\\
456	0\\
457	0\\
458	0\\
459	0\\
460	0\\
461	0\\
462	0\\
463	0\\
464	0\\
465	0\\
466	0\\
467	0\\
468	0\\
469	0\\
470	0\\
471	0\\
472	0\\
473	0\\
474	0\\
475	0\\
476	0\\
477	0\\
478	0\\
479	0\\
480	0\\
481	0\\
482	0\\
483	0\\
484	0\\
485	0\\
486	0\\
487	0\\
488	0\\
489	0\\
490	0\\
491	0\\
492	0\\
493	0\\
494	0\\
495	0\\
496	0\\
497	0\\
498	0\\
499	0\\
500	0\\
501	0\\
502	0\\
503	0\\
504	0\\
505	0\\
506	0\\
507	0\\
508	0\\
509	0\\
510	0\\
511	0\\
512	0\\
513	0\\
514	0\\
515	0\\
516	0\\
517	0\\
518	0\\
519	0\\
520	0\\
521	0\\
522	0\\
523	0\\
524	0\\
525	0\\
526	0\\
527	0\\
528	0\\
529	0\\
530	0\\
531	0\\
532	0\\
533	0\\
534	0\\
535	0\\
536	0\\
537	0\\
538	0\\
539	0\\
540	0\\
541	0\\
542	0\\
543	0\\
544	0\\
545	0\\
546	0\\
547	0\\
548	0\\
549	0\\
550	0\\
551	0\\
552	0\\
553	0\\
554	0\\
555	2.26142481073285e-05\\
556	9.40356632703977e-05\\
557	0.000142834796507731\\
558	0.000192136179755935\\
559	0.000241890470607289\\
560	0.000292049373855764\\
561	0.000342545162924321\\
562	0.00039329687625306\\
563	0.000444211080034401\\
564	0.000495180373312776\\
565	0.000546081753293092\\
566	0.000596774292680829\\
567	0.000647098131380762\\
568	0.00069742347279142\\
569	0.000748696607560934\\
570	0.000800939380062976\\
571	0.000854177294682413\\
572	0.000908439254781951\\
573	0.000963750984062382\\
574	0.00102014366327751\\
575	0.00107765573610446\\
576	0.00113633523369278\\
577	0.00119624324802379\\
578	0.00125745746076917\\
579	0.00132007724518431\\
580	0.00138423112120714\\
581	0.0014499990846374\\
582	0.00151746183138254\\
583	0.001586633971824\\
584	0.00165775679116301\\
585	0.00173067211608291\\
586	0.00180616539338125\\
587	0.0018831539132818\\
588	0.00209862842761386\\
589	0.00237026081777734\\
590	0.00265207182564143\\
591	0.00293299955993958\\
592	0.0031940225228887\\
593	0.00348473277442626\\
594	0.00383471000680442\\
595	0.00412245907862188\\
596	0.00444436306303141\\
597	0.00503983077166121\\
598	0.00644286460810295\\
599	0\\
600	0\\
};
\addplot [color=mycolor18,solid,forget plot]
  table[row sep=crcr]{%
1	0\\
2	0\\
3	0\\
4	0\\
5	0\\
6	0\\
7	0\\
8	0\\
9	0\\
10	0\\
11	0\\
12	0\\
13	0\\
14	0\\
15	0\\
16	0\\
17	0\\
18	0\\
19	0\\
20	0\\
21	0\\
22	0\\
23	0\\
24	0\\
25	0\\
26	0\\
27	0\\
28	0\\
29	0\\
30	0\\
31	0\\
32	0\\
33	0\\
34	0\\
35	0\\
36	0\\
37	0\\
38	0\\
39	0\\
40	0\\
41	0\\
42	0\\
43	0\\
44	0\\
45	0\\
46	0\\
47	0\\
48	0\\
49	0\\
50	0\\
51	0\\
52	0\\
53	0\\
54	0\\
55	0\\
56	0\\
57	0\\
58	0\\
59	0\\
60	0\\
61	0\\
62	0\\
63	0\\
64	0\\
65	0\\
66	0\\
67	0\\
68	0\\
69	0\\
70	0\\
71	0\\
72	0\\
73	0\\
74	0\\
75	0\\
76	0\\
77	0\\
78	0\\
79	0\\
80	0\\
81	0\\
82	0\\
83	0\\
84	0\\
85	0\\
86	0\\
87	0\\
88	0\\
89	0\\
90	0\\
91	0\\
92	0\\
93	0\\
94	0\\
95	0\\
96	0\\
97	0\\
98	0\\
99	0\\
100	0\\
101	0\\
102	0\\
103	0\\
104	0\\
105	0\\
106	0\\
107	0\\
108	0\\
109	0\\
110	0\\
111	0\\
112	0\\
113	0\\
114	0\\
115	0\\
116	0\\
117	0\\
118	0\\
119	0\\
120	0\\
121	0\\
122	0\\
123	0\\
124	0\\
125	0\\
126	0\\
127	0\\
128	0\\
129	0\\
130	0\\
131	0\\
132	0\\
133	0\\
134	0\\
135	0\\
136	0\\
137	0\\
138	0\\
139	0\\
140	0\\
141	0\\
142	0\\
143	0\\
144	0\\
145	0\\
146	0\\
147	0\\
148	0\\
149	0\\
150	0\\
151	0\\
152	0\\
153	0\\
154	0\\
155	0\\
156	0\\
157	0\\
158	0\\
159	0\\
160	0\\
161	0\\
162	0\\
163	0\\
164	0\\
165	0\\
166	0\\
167	0\\
168	0\\
169	0\\
170	0\\
171	0\\
172	0\\
173	0\\
174	0\\
175	0\\
176	0\\
177	0\\
178	0\\
179	0\\
180	0\\
181	0\\
182	0\\
183	0\\
184	0\\
185	0\\
186	0\\
187	0\\
188	0\\
189	0\\
190	0\\
191	0\\
192	0\\
193	0\\
194	0\\
195	0\\
196	0\\
197	0\\
198	0\\
199	0\\
200	0\\
201	0\\
202	0\\
203	0\\
204	0\\
205	0\\
206	0\\
207	0\\
208	0\\
209	0\\
210	0\\
211	0\\
212	0\\
213	0\\
214	0\\
215	0\\
216	0\\
217	0\\
218	0\\
219	0\\
220	0\\
221	0\\
222	0\\
223	0\\
224	0\\
225	0\\
226	0\\
227	0\\
228	0\\
229	0\\
230	0\\
231	0\\
232	0\\
233	0\\
234	0\\
235	0\\
236	0\\
237	0\\
238	0\\
239	0\\
240	0\\
241	0\\
242	0\\
243	0\\
244	0\\
245	0\\
246	0\\
247	0\\
248	0\\
249	0\\
250	0\\
251	0\\
252	0\\
253	0\\
254	0\\
255	0\\
256	0\\
257	0\\
258	0\\
259	0\\
260	0\\
261	0\\
262	0\\
263	0\\
264	0\\
265	0\\
266	0\\
267	0\\
268	0\\
269	0\\
270	0\\
271	0\\
272	0\\
273	0\\
274	0\\
275	0\\
276	0\\
277	0\\
278	0\\
279	0\\
280	0\\
281	0\\
282	0\\
283	0\\
284	0\\
285	0\\
286	0\\
287	0\\
288	0\\
289	0\\
290	0\\
291	0\\
292	0\\
293	0\\
294	0\\
295	0\\
296	0\\
297	0\\
298	0\\
299	0\\
300	0\\
301	0\\
302	0\\
303	0\\
304	0\\
305	0\\
306	0\\
307	0\\
308	0\\
309	0\\
310	0\\
311	0\\
312	0\\
313	0\\
314	0\\
315	0\\
316	0\\
317	0\\
318	0\\
319	0\\
320	0\\
321	0\\
322	0\\
323	0\\
324	0\\
325	0\\
326	0\\
327	0\\
328	0\\
329	0\\
330	0\\
331	0\\
332	0\\
333	0\\
334	0\\
335	0\\
336	0\\
337	0\\
338	0\\
339	0\\
340	0\\
341	0\\
342	0\\
343	0\\
344	0\\
345	0\\
346	0\\
347	0\\
348	0\\
349	0\\
350	0\\
351	0\\
352	0\\
353	0\\
354	0\\
355	0\\
356	0\\
357	0\\
358	0\\
359	0\\
360	0\\
361	0\\
362	0\\
363	0\\
364	0\\
365	0\\
366	0\\
367	0\\
368	0\\
369	0\\
370	0\\
371	0\\
372	0\\
373	0\\
374	0\\
375	0\\
376	0\\
377	0\\
378	0\\
379	0\\
380	0\\
381	0\\
382	0\\
383	0\\
384	0\\
385	0\\
386	0\\
387	0\\
388	0\\
389	0\\
390	0\\
391	0\\
392	0\\
393	0\\
394	0\\
395	0\\
396	0\\
397	0\\
398	0\\
399	0\\
400	0\\
401	0\\
402	0\\
403	0\\
404	0\\
405	0\\
406	0\\
407	0\\
408	0\\
409	0\\
410	0\\
411	0\\
412	0\\
413	0\\
414	0\\
415	0\\
416	0\\
417	0\\
418	0\\
419	0\\
420	0\\
421	0\\
422	0\\
423	0\\
424	0\\
425	0\\
426	0\\
427	0\\
428	0\\
429	0\\
430	0\\
431	0\\
432	0\\
433	0\\
434	0\\
435	0\\
436	0\\
437	0\\
438	0\\
439	0\\
440	0\\
441	0\\
442	0\\
443	0\\
444	0\\
445	0\\
446	0\\
447	0\\
448	0\\
449	0\\
450	0\\
451	0\\
452	0\\
453	0\\
454	0\\
455	0\\
456	0\\
457	0\\
458	0\\
459	0\\
460	0\\
461	0\\
462	0\\
463	0\\
464	0\\
465	0\\
466	0\\
467	0\\
468	0\\
469	0\\
470	0\\
471	0\\
472	0\\
473	0\\
474	0\\
475	0\\
476	0\\
477	0\\
478	0\\
479	0\\
480	0\\
481	0\\
482	0\\
483	0\\
484	0\\
485	0\\
486	0\\
487	0\\
488	0\\
489	0\\
490	0\\
491	0\\
492	0\\
493	0\\
494	0\\
495	0\\
496	0\\
497	0\\
498	0\\
499	0\\
500	0\\
501	0\\
502	0\\
503	0\\
504	0\\
505	0\\
506	0\\
507	0\\
508	0\\
509	0\\
510	0\\
511	0\\
512	0\\
513	0\\
514	0\\
515	0\\
516	0\\
517	0\\
518	0\\
519	0\\
520	0\\
521	0\\
522	0\\
523	0\\
524	0\\
525	0\\
526	0\\
527	0\\
528	0\\
529	0\\
530	0\\
531	0\\
532	0\\
533	0\\
534	0\\
535	0\\
536	0\\
537	0\\
538	0\\
539	0\\
540	0\\
541	0\\
542	0\\
543	0\\
544	0\\
545	0\\
546	0\\
547	0\\
548	0\\
549	2.32520210821688e-05\\
550	6.29822389686342e-05\\
551	0.000102594284985584\\
552	0.000142000070962094\\
553	0.000181109221367284\\
554	0.000219804609413708\\
555	0.000257961564445722\\
556	0.000295780758493718\\
557	0.000334217678983119\\
558	0.000373283831162645\\
559	0.000412993211787373\\
560	0.000453363060135725\\
561	0.000494414867686697\\
562	0.000536175585096235\\
563	0.000578679087463031\\
564	0.000621967988206539\\
565	0.000666095878568849\\
566	0.000711130094990775\\
567	0.000757155082147057\\
568	0.000804247376185817\\
569	0.000852443417848277\\
570	0.00090178212884265\\
571	0.000952305207691893\\
572	0.00100405760152108\\
573	0.00105708760588472\\
574	0.00111144706147627\\
575	0.0011671914887648\\
576	0.00122442232407532\\
577	0.00128313096863521\\
578	0.0013433810070614\\
579	0.00140538131268943\\
580	0.00146902614413746\\
581	0.00153430958574611\\
582	0.00160209512119075\\
583	0.00167132716120464\\
584	0.0017419265568923\\
585	0.0019582562819987\\
586	0.00223293796086997\\
587	0.00252513250926167\\
588	0.00278502806553789\\
589	0.00303946052792341\\
590	0.0033058313870396\\
591	0.00353307097397887\\
592	0.00367241590802784\\
593	0.00382251134333667\\
594	0.00397190834964223\\
595	0.00415280575412962\\
596	0.00444436306303141\\
597	0.00503983077166121\\
598	0.00644286460810295\\
599	0\\
600	0\\
};
\addplot [color=red!25!mycolor17,solid,forget plot]
  table[row sep=crcr]{%
1	0\\
2	0\\
3	0\\
4	0\\
5	0\\
6	0\\
7	0\\
8	0\\
9	0\\
10	0\\
11	0\\
12	0\\
13	0\\
14	0\\
15	0\\
16	0\\
17	0\\
18	0\\
19	0\\
20	0\\
21	0\\
22	0\\
23	0\\
24	0\\
25	0\\
26	0\\
27	0\\
28	0\\
29	0\\
30	0\\
31	0\\
32	0\\
33	0\\
34	0\\
35	0\\
36	0\\
37	0\\
38	0\\
39	0\\
40	0\\
41	0\\
42	0\\
43	0\\
44	0\\
45	0\\
46	0\\
47	0\\
48	0\\
49	0\\
50	0\\
51	0\\
52	0\\
53	0\\
54	0\\
55	0\\
56	0\\
57	0\\
58	0\\
59	0\\
60	0\\
61	0\\
62	0\\
63	0\\
64	0\\
65	0\\
66	0\\
67	0\\
68	0\\
69	0\\
70	0\\
71	0\\
72	0\\
73	0\\
74	0\\
75	0\\
76	0\\
77	0\\
78	0\\
79	0\\
80	0\\
81	0\\
82	0\\
83	0\\
84	0\\
85	0\\
86	0\\
87	0\\
88	0\\
89	0\\
90	0\\
91	0\\
92	0\\
93	0\\
94	0\\
95	0\\
96	0\\
97	0\\
98	0\\
99	0\\
100	0\\
101	0\\
102	0\\
103	0\\
104	0\\
105	0\\
106	0\\
107	0\\
108	0\\
109	0\\
110	0\\
111	0\\
112	0\\
113	0\\
114	0\\
115	0\\
116	0\\
117	0\\
118	0\\
119	0\\
120	0\\
121	0\\
122	0\\
123	0\\
124	0\\
125	0\\
126	0\\
127	0\\
128	0\\
129	0\\
130	0\\
131	0\\
132	0\\
133	0\\
134	0\\
135	0\\
136	0\\
137	0\\
138	0\\
139	0\\
140	0\\
141	0\\
142	0\\
143	0\\
144	0\\
145	0\\
146	0\\
147	0\\
148	0\\
149	0\\
150	0\\
151	0\\
152	0\\
153	0\\
154	0\\
155	0\\
156	0\\
157	0\\
158	0\\
159	0\\
160	0\\
161	0\\
162	0\\
163	0\\
164	0\\
165	0\\
166	0\\
167	0\\
168	0\\
169	0\\
170	0\\
171	0\\
172	0\\
173	0\\
174	0\\
175	0\\
176	0\\
177	0\\
178	0\\
179	0\\
180	0\\
181	0\\
182	0\\
183	0\\
184	0\\
185	0\\
186	0\\
187	0\\
188	0\\
189	0\\
190	0\\
191	0\\
192	0\\
193	0\\
194	0\\
195	0\\
196	0\\
197	0\\
198	0\\
199	0\\
200	0\\
201	0\\
202	0\\
203	0\\
204	0\\
205	0\\
206	0\\
207	0\\
208	0\\
209	0\\
210	0\\
211	0\\
212	0\\
213	0\\
214	0\\
215	0\\
216	0\\
217	0\\
218	0\\
219	0\\
220	0\\
221	0\\
222	0\\
223	0\\
224	0\\
225	0\\
226	0\\
227	0\\
228	0\\
229	0\\
230	0\\
231	0\\
232	0\\
233	0\\
234	0\\
235	0\\
236	0\\
237	0\\
238	0\\
239	0\\
240	0\\
241	0\\
242	0\\
243	0\\
244	0\\
245	0\\
246	0\\
247	0\\
248	0\\
249	0\\
250	0\\
251	0\\
252	0\\
253	0\\
254	0\\
255	0\\
256	0\\
257	0\\
258	0\\
259	0\\
260	0\\
261	0\\
262	0\\
263	0\\
264	0\\
265	0\\
266	0\\
267	0\\
268	0\\
269	0\\
270	0\\
271	0\\
272	0\\
273	0\\
274	0\\
275	0\\
276	0\\
277	0\\
278	0\\
279	0\\
280	0\\
281	0\\
282	0\\
283	0\\
284	0\\
285	0\\
286	0\\
287	0\\
288	0\\
289	0\\
290	0\\
291	0\\
292	0\\
293	0\\
294	0\\
295	0\\
296	0\\
297	0\\
298	0\\
299	0\\
300	0\\
301	0\\
302	0\\
303	0\\
304	0\\
305	0\\
306	0\\
307	0\\
308	0\\
309	0\\
310	0\\
311	0\\
312	0\\
313	0\\
314	0\\
315	0\\
316	0\\
317	0\\
318	0\\
319	0\\
320	0\\
321	0\\
322	0\\
323	0\\
324	0\\
325	0\\
326	0\\
327	0\\
328	0\\
329	0\\
330	0\\
331	0\\
332	0\\
333	0\\
334	0\\
335	0\\
336	0\\
337	0\\
338	0\\
339	0\\
340	0\\
341	0\\
342	0\\
343	0\\
344	0\\
345	0\\
346	0\\
347	0\\
348	0\\
349	0\\
350	0\\
351	0\\
352	0\\
353	0\\
354	0\\
355	0\\
356	0\\
357	0\\
358	0\\
359	0\\
360	0\\
361	0\\
362	0\\
363	0\\
364	0\\
365	0\\
366	0\\
367	0\\
368	0\\
369	0\\
370	0\\
371	0\\
372	0\\
373	0\\
374	0\\
375	0\\
376	0\\
377	0\\
378	0\\
379	0\\
380	0\\
381	0\\
382	0\\
383	0\\
384	0\\
385	0\\
386	0\\
387	0\\
388	0\\
389	0\\
390	0\\
391	0\\
392	0\\
393	0\\
394	0\\
395	0\\
396	0\\
397	0\\
398	0\\
399	0\\
400	0\\
401	0\\
402	0\\
403	0\\
404	0\\
405	0\\
406	0\\
407	0\\
408	0\\
409	0\\
410	0\\
411	0\\
412	0\\
413	0\\
414	0\\
415	0\\
416	0\\
417	0\\
418	0\\
419	0\\
420	0\\
421	0\\
422	0\\
423	0\\
424	0\\
425	0\\
426	0\\
427	0\\
428	0\\
429	0\\
430	0\\
431	0\\
432	0\\
433	0\\
434	0\\
435	0\\
436	0\\
437	0\\
438	0\\
439	0\\
440	0\\
441	0\\
442	0\\
443	0\\
444	0\\
445	0\\
446	0\\
447	0\\
448	0\\
449	0\\
450	0\\
451	0\\
452	0\\
453	0\\
454	0\\
455	0\\
456	0\\
457	0\\
458	0\\
459	0\\
460	0\\
461	0\\
462	0\\
463	0\\
464	0\\
465	0\\
466	0\\
467	0\\
468	0\\
469	0\\
470	0\\
471	0\\
472	0\\
473	0\\
474	0\\
475	0\\
476	0\\
477	0\\
478	0\\
479	0\\
480	0\\
481	0\\
482	0\\
483	0\\
484	0\\
485	0\\
486	0\\
487	0\\
488	0\\
489	0\\
490	0\\
491	0\\
492	0\\
493	0\\
494	0\\
495	0\\
496	0\\
497	0\\
498	0\\
499	0\\
500	0\\
501	0\\
502	0\\
503	0\\
504	0\\
505	0\\
506	0\\
507	0\\
508	0\\
509	0\\
510	0\\
511	0\\
512	0\\
513	0\\
514	0\\
515	0\\
516	0\\
517	0\\
518	0\\
519	0\\
520	0\\
521	0\\
522	0\\
523	0\\
524	0\\
525	0\\
526	0\\
527	0\\
528	0\\
529	0\\
530	0\\
531	0\\
532	0\\
533	0\\
534	0\\
535	0\\
536	0\\
537	0\\
538	0\\
539	0\\
540	0\\
541	0\\
542	0\\
543	0\\
544	0\\
545	1.60914883874299e-05\\
546	4.61802426780253e-05\\
547	7.66629135509208e-05\\
548	0.000107555208654854\\
549	0.000138876541337362\\
550	0.000170650593952384\\
551	0.000202914517604868\\
552	0.000235708465118579\\
553	0.000269081896836238\\
554	0.000303087169673518\\
555	0.000337789238210994\\
556	0.000373251698431509\\
557	0.000409499257060565\\
558	0.000446558227820018\\
559	0.000484456617435164\\
560	0.000523224188690424\\
561	0.000562892484763894\\
562	0.000603494791845105\\
563	0.000645066023735202\\
564	0.000687642494400291\\
565	0.000731261531316776\\
566	0.000775960935897401\\
567	0.000821778458148387\\
568	0.000868751612779525\\
569	0.000916919738931042\\
570	0.000966334419799042\\
571	0.00101705468567766\\
572	0.00106906417782553\\
573	0.00112241341855564\\
574	0.00117715592173479\\
575	0.00123354169018176\\
576	0.00129128803526078\\
577	0.00135048220327041\\
578	0.00141156374788578\\
579	0.00147451749231703\\
580	0.00153861733116412\\
581	0.00160404882434682\\
582	0.0017589725145217\\
583	0.00203755571568777\\
584	0.00233345386073883\\
585	0.00259266855288805\\
586	0.00284513355551299\\
587	0.00310713312521459\\
588	0.0032721022788783\\
589	0.00339546551785458\\
590	0.00351237211383029\\
591	0.00361868673458143\\
592	0.0037291949721508\\
593	0.00384481075529708\\
594	0.00397672044113206\\
595	0.00415280575412961\\
596	0.00444436306303141\\
597	0.00503983077166121\\
598	0.00644286460810295\\
599	0\\
600	0\\
};
\addplot [color=mycolor19,solid,forget plot]
  table[row sep=crcr]{%
1	0\\
2	0\\
3	0\\
4	0\\
5	0\\
6	0\\
7	0\\
8	0\\
9	0\\
10	0\\
11	0\\
12	0\\
13	0\\
14	0\\
15	0\\
16	0\\
17	0\\
18	0\\
19	0\\
20	0\\
21	0\\
22	0\\
23	0\\
24	0\\
25	0\\
26	0\\
27	0\\
28	0\\
29	0\\
30	0\\
31	0\\
32	0\\
33	0\\
34	0\\
35	0\\
36	0\\
37	0\\
38	0\\
39	0\\
40	0\\
41	0\\
42	0\\
43	0\\
44	0\\
45	0\\
46	0\\
47	0\\
48	0\\
49	0\\
50	0\\
51	0\\
52	0\\
53	0\\
54	0\\
55	0\\
56	0\\
57	0\\
58	0\\
59	0\\
60	0\\
61	0\\
62	0\\
63	0\\
64	0\\
65	0\\
66	0\\
67	0\\
68	0\\
69	0\\
70	0\\
71	0\\
72	0\\
73	0\\
74	0\\
75	0\\
76	0\\
77	0\\
78	0\\
79	0\\
80	0\\
81	0\\
82	0\\
83	0\\
84	0\\
85	0\\
86	0\\
87	0\\
88	0\\
89	0\\
90	0\\
91	0\\
92	0\\
93	0\\
94	0\\
95	0\\
96	0\\
97	0\\
98	0\\
99	0\\
100	0\\
101	0\\
102	0\\
103	0\\
104	0\\
105	0\\
106	0\\
107	0\\
108	0\\
109	0\\
110	0\\
111	0\\
112	0\\
113	0\\
114	0\\
115	0\\
116	0\\
117	0\\
118	0\\
119	0\\
120	0\\
121	0\\
122	0\\
123	0\\
124	0\\
125	0\\
126	0\\
127	0\\
128	0\\
129	0\\
130	0\\
131	0\\
132	0\\
133	0\\
134	0\\
135	0\\
136	0\\
137	0\\
138	0\\
139	0\\
140	0\\
141	0\\
142	0\\
143	0\\
144	0\\
145	0\\
146	0\\
147	0\\
148	0\\
149	0\\
150	0\\
151	0\\
152	0\\
153	0\\
154	0\\
155	0\\
156	0\\
157	0\\
158	0\\
159	0\\
160	0\\
161	0\\
162	0\\
163	0\\
164	0\\
165	0\\
166	0\\
167	0\\
168	0\\
169	0\\
170	0\\
171	0\\
172	0\\
173	0\\
174	0\\
175	0\\
176	0\\
177	0\\
178	0\\
179	0\\
180	0\\
181	0\\
182	0\\
183	0\\
184	0\\
185	0\\
186	0\\
187	0\\
188	0\\
189	0\\
190	0\\
191	0\\
192	0\\
193	0\\
194	0\\
195	0\\
196	0\\
197	0\\
198	0\\
199	0\\
200	0\\
201	0\\
202	0\\
203	0\\
204	0\\
205	0\\
206	0\\
207	0\\
208	0\\
209	0\\
210	0\\
211	0\\
212	0\\
213	0\\
214	0\\
215	0\\
216	0\\
217	0\\
218	0\\
219	0\\
220	0\\
221	0\\
222	0\\
223	0\\
224	0\\
225	0\\
226	0\\
227	0\\
228	0\\
229	0\\
230	0\\
231	0\\
232	0\\
233	0\\
234	0\\
235	0\\
236	0\\
237	0\\
238	0\\
239	0\\
240	0\\
241	0\\
242	0\\
243	0\\
244	0\\
245	0\\
246	0\\
247	0\\
248	0\\
249	0\\
250	0\\
251	0\\
252	0\\
253	0\\
254	0\\
255	0\\
256	0\\
257	0\\
258	0\\
259	0\\
260	0\\
261	0\\
262	0\\
263	0\\
264	0\\
265	0\\
266	0\\
267	0\\
268	0\\
269	0\\
270	0\\
271	0\\
272	0\\
273	0\\
274	0\\
275	0\\
276	0\\
277	0\\
278	0\\
279	0\\
280	0\\
281	0\\
282	0\\
283	0\\
284	0\\
285	0\\
286	0\\
287	0\\
288	0\\
289	0\\
290	0\\
291	0\\
292	0\\
293	0\\
294	0\\
295	0\\
296	0\\
297	0\\
298	0\\
299	0\\
300	0\\
301	0\\
302	0\\
303	0\\
304	0\\
305	0\\
306	0\\
307	0\\
308	0\\
309	0\\
310	0\\
311	0\\
312	0\\
313	0\\
314	0\\
315	0\\
316	0\\
317	0\\
318	0\\
319	0\\
320	0\\
321	0\\
322	0\\
323	0\\
324	0\\
325	0\\
326	0\\
327	0\\
328	0\\
329	0\\
330	0\\
331	0\\
332	0\\
333	0\\
334	0\\
335	0\\
336	0\\
337	0\\
338	0\\
339	0\\
340	0\\
341	0\\
342	0\\
343	0\\
344	0\\
345	0\\
346	0\\
347	0\\
348	0\\
349	0\\
350	0\\
351	0\\
352	0\\
353	0\\
354	0\\
355	0\\
356	0\\
357	0\\
358	0\\
359	0\\
360	0\\
361	0\\
362	0\\
363	0\\
364	0\\
365	0\\
366	0\\
367	0\\
368	0\\
369	0\\
370	0\\
371	0\\
372	0\\
373	0\\
374	0\\
375	0\\
376	0\\
377	0\\
378	0\\
379	0\\
380	0\\
381	0\\
382	0\\
383	0\\
384	0\\
385	0\\
386	0\\
387	0\\
388	0\\
389	0\\
390	0\\
391	0\\
392	0\\
393	0\\
394	0\\
395	0\\
396	0\\
397	0\\
398	0\\
399	0\\
400	0\\
401	0\\
402	0\\
403	0\\
404	0\\
405	0\\
406	0\\
407	0\\
408	0\\
409	0\\
410	0\\
411	0\\
412	0\\
413	0\\
414	0\\
415	0\\
416	0\\
417	0\\
418	0\\
419	0\\
420	0\\
421	0\\
422	0\\
423	0\\
424	0\\
425	0\\
426	0\\
427	0\\
428	0\\
429	0\\
430	0\\
431	0\\
432	0\\
433	0\\
434	0\\
435	0\\
436	0\\
437	0\\
438	0\\
439	0\\
440	0\\
441	0\\
442	0\\
443	0\\
444	0\\
445	0\\
446	0\\
447	0\\
448	0\\
449	0\\
450	0\\
451	0\\
452	0\\
453	0\\
454	0\\
455	0\\
456	0\\
457	0\\
458	0\\
459	0\\
460	0\\
461	0\\
462	0\\
463	0\\
464	0\\
465	0\\
466	0\\
467	0\\
468	0\\
469	0\\
470	0\\
471	0\\
472	0\\
473	0\\
474	0\\
475	0\\
476	0\\
477	0\\
478	0\\
479	0\\
480	0\\
481	0\\
482	0\\
483	0\\
484	0\\
485	0\\
486	0\\
487	0\\
488	0\\
489	0\\
490	0\\
491	0\\
492	0\\
493	0\\
494	0\\
495	0\\
496	0\\
497	0\\
498	0\\
499	0\\
500	0\\
501	0\\
502	0\\
503	0\\
504	0\\
505	0\\
506	0\\
507	0\\
508	0\\
509	0\\
510	0\\
511	0\\
512	0\\
513	0\\
514	0\\
515	0\\
516	0\\
517	0\\
518	0\\
519	0\\
520	0\\
521	0\\
522	0\\
523	0\\
524	0\\
525	0\\
526	0\\
527	0\\
528	0\\
529	0\\
530	0\\
531	0\\
532	0\\
533	0\\
534	0\\
535	0\\
536	0\\
537	0\\
538	0\\
539	0\\
540	0\\
541	0\\
542	0\\
543	1.6281523779886e-05\\
544	4.35246637006791e-05\\
545	7.13087122634737e-05\\
546	9.96552821353406e-05\\
547	0.000128587210985315\\
548	0.000158128564717491\\
549	0.000188304258646555\\
550	0.000219151430388676\\
551	0.000250701151250487\\
552	0.000282981360877781\\
553	0.000316017445589515\\
554	0.000349834654432245\\
555	0.000384457427831735\\
556	0.000419909302190068\\
557	0.000456214671566793\\
558	0.000493398798486647\\
559	0.000531487829578367\\
560	0.000570508810018441\\
561	0.000610489697648098\\
562	0.000651459746879398\\
563	0.000693449329616425\\
564	0.000736489807219065\\
565	0.000780619589166497\\
566	0.000825901473907119\\
567	0.000872296831358665\\
568	0.000919843835008248\\
569	0.000968584807559033\\
570	0.00101855905585974\\
571	0.00106996928087748\\
572	0.0011226278711555\\
573	0.00117653489629999\\
574	0.00123176683108536\\
575	0.00128877167129672\\
576	0.00134736204101222\\
577	0.00140700004607742\\
578	0.00146757979920448\\
579	0.00152899875679746\\
580	0.00178174393908874\\
581	0.00207611800840707\\
582	0.00235517143267944\\
583	0.0026041334094422\\
584	0.00286156611296512\\
585	0.00301924574469048\\
586	0.00313397161576881\\
587	0.00323536006699526\\
588	0.00333221291604738\\
589	0.00342954842160734\\
590	0.0035273364337839\\
591	0.00362803120808679\\
592	0.00373275376052986\\
593	0.00384555904940006\\
594	0.00397672044113206\\
595	0.00415280575412961\\
596	0.00444436306303141\\
597	0.00503983077166121\\
598	0.00644286460810295\\
599	0\\
600	0\\
};
\addplot [color=red!50!mycolor17,solid,forget plot]
  table[row sep=crcr]{%
1	0\\
2	0\\
3	0\\
4	0\\
5	0\\
6	0\\
7	0\\
8	0\\
9	0\\
10	0\\
11	0\\
12	0\\
13	0\\
14	0\\
15	0\\
16	0\\
17	0\\
18	0\\
19	0\\
20	0\\
21	0\\
22	0\\
23	0\\
24	0\\
25	0\\
26	0\\
27	0\\
28	0\\
29	0\\
30	0\\
31	0\\
32	0\\
33	0\\
34	0\\
35	0\\
36	0\\
37	0\\
38	0\\
39	0\\
40	0\\
41	0\\
42	0\\
43	0\\
44	0\\
45	0\\
46	0\\
47	0\\
48	0\\
49	0\\
50	0\\
51	0\\
52	0\\
53	0\\
54	0\\
55	0\\
56	0\\
57	0\\
58	0\\
59	0\\
60	0\\
61	0\\
62	0\\
63	0\\
64	0\\
65	0\\
66	0\\
67	0\\
68	0\\
69	0\\
70	0\\
71	0\\
72	0\\
73	0\\
74	0\\
75	0\\
76	0\\
77	0\\
78	0\\
79	0\\
80	0\\
81	0\\
82	0\\
83	0\\
84	0\\
85	0\\
86	0\\
87	0\\
88	0\\
89	0\\
90	0\\
91	0\\
92	0\\
93	0\\
94	0\\
95	0\\
96	0\\
97	0\\
98	0\\
99	0\\
100	0\\
101	0\\
102	0\\
103	0\\
104	0\\
105	0\\
106	0\\
107	0\\
108	0\\
109	0\\
110	0\\
111	0\\
112	0\\
113	0\\
114	0\\
115	0\\
116	0\\
117	0\\
118	0\\
119	0\\
120	0\\
121	0\\
122	0\\
123	0\\
124	0\\
125	0\\
126	0\\
127	0\\
128	0\\
129	0\\
130	0\\
131	0\\
132	0\\
133	0\\
134	0\\
135	0\\
136	0\\
137	0\\
138	0\\
139	0\\
140	0\\
141	0\\
142	0\\
143	0\\
144	0\\
145	0\\
146	0\\
147	0\\
148	0\\
149	0\\
150	0\\
151	0\\
152	0\\
153	0\\
154	0\\
155	0\\
156	0\\
157	0\\
158	0\\
159	0\\
160	0\\
161	0\\
162	0\\
163	0\\
164	0\\
165	0\\
166	0\\
167	0\\
168	0\\
169	0\\
170	0\\
171	0\\
172	0\\
173	0\\
174	0\\
175	0\\
176	0\\
177	0\\
178	0\\
179	0\\
180	0\\
181	0\\
182	0\\
183	0\\
184	0\\
185	0\\
186	0\\
187	0\\
188	0\\
189	0\\
190	0\\
191	0\\
192	0\\
193	0\\
194	0\\
195	0\\
196	0\\
197	0\\
198	0\\
199	0\\
200	0\\
201	0\\
202	0\\
203	0\\
204	0\\
205	0\\
206	0\\
207	0\\
208	0\\
209	0\\
210	0\\
211	0\\
212	0\\
213	0\\
214	0\\
215	0\\
216	0\\
217	0\\
218	0\\
219	0\\
220	0\\
221	0\\
222	0\\
223	0\\
224	0\\
225	0\\
226	0\\
227	0\\
228	0\\
229	0\\
230	0\\
231	0\\
232	0\\
233	0\\
234	0\\
235	0\\
236	0\\
237	0\\
238	0\\
239	0\\
240	0\\
241	0\\
242	0\\
243	0\\
244	0\\
245	0\\
246	0\\
247	0\\
248	0\\
249	0\\
250	0\\
251	0\\
252	0\\
253	0\\
254	0\\
255	0\\
256	0\\
257	0\\
258	0\\
259	0\\
260	0\\
261	0\\
262	0\\
263	0\\
264	0\\
265	0\\
266	0\\
267	0\\
268	0\\
269	0\\
270	0\\
271	0\\
272	0\\
273	0\\
274	0\\
275	0\\
276	0\\
277	0\\
278	0\\
279	0\\
280	0\\
281	0\\
282	0\\
283	0\\
284	0\\
285	0\\
286	0\\
287	0\\
288	0\\
289	0\\
290	0\\
291	0\\
292	0\\
293	0\\
294	0\\
295	0\\
296	0\\
297	0\\
298	0\\
299	0\\
300	0\\
301	0\\
302	0\\
303	0\\
304	0\\
305	0\\
306	0\\
307	0\\
308	0\\
309	0\\
310	0\\
311	0\\
312	0\\
313	0\\
314	0\\
315	0\\
316	0\\
317	0\\
318	0\\
319	0\\
320	0\\
321	0\\
322	0\\
323	0\\
324	0\\
325	0\\
326	0\\
327	0\\
328	0\\
329	0\\
330	0\\
331	0\\
332	0\\
333	0\\
334	0\\
335	0\\
336	0\\
337	0\\
338	0\\
339	0\\
340	0\\
341	0\\
342	0\\
343	0\\
344	0\\
345	0\\
346	0\\
347	0\\
348	0\\
349	0\\
350	0\\
351	0\\
352	0\\
353	0\\
354	0\\
355	0\\
356	0\\
357	0\\
358	0\\
359	0\\
360	0\\
361	0\\
362	0\\
363	0\\
364	0\\
365	0\\
366	0\\
367	0\\
368	0\\
369	0\\
370	0\\
371	0\\
372	0\\
373	0\\
374	0\\
375	0\\
376	0\\
377	0\\
378	0\\
379	0\\
380	0\\
381	0\\
382	0\\
383	0\\
384	0\\
385	0\\
386	0\\
387	0\\
388	0\\
389	0\\
390	0\\
391	0\\
392	0\\
393	0\\
394	0\\
395	0\\
396	0\\
397	0\\
398	0\\
399	0\\
400	0\\
401	0\\
402	0\\
403	0\\
404	0\\
405	0\\
406	0\\
407	0\\
408	0\\
409	0\\
410	0\\
411	0\\
412	0\\
413	0\\
414	0\\
415	0\\
416	0\\
417	0\\
418	0\\
419	0\\
420	0\\
421	0\\
422	0\\
423	0\\
424	0\\
425	0\\
426	0\\
427	0\\
428	0\\
429	0\\
430	0\\
431	0\\
432	0\\
433	0\\
434	0\\
435	0\\
436	0\\
437	0\\
438	0\\
439	0\\
440	0\\
441	0\\
442	0\\
443	0\\
444	0\\
445	0\\
446	0\\
447	0\\
448	0\\
449	0\\
450	0\\
451	0\\
452	0\\
453	0\\
454	0\\
455	0\\
456	0\\
457	0\\
458	0\\
459	0\\
460	0\\
461	0\\
462	0\\
463	0\\
464	0\\
465	0\\
466	0\\
467	0\\
468	0\\
469	0\\
470	0\\
471	0\\
472	0\\
473	0\\
474	0\\
475	0\\
476	0\\
477	0\\
478	0\\
479	0\\
480	0\\
481	0\\
482	0\\
483	0\\
484	0\\
485	0\\
486	0\\
487	0\\
488	0\\
489	0\\
490	0\\
491	0\\
492	0\\
493	0\\
494	0\\
495	0\\
496	0\\
497	0\\
498	0\\
499	0\\
500	0\\
501	0\\
502	0\\
503	0\\
504	0\\
505	0\\
506	0\\
507	0\\
508	0\\
509	0\\
510	0\\
511	0\\
512	0\\
513	0\\
514	0\\
515	0\\
516	0\\
517	0\\
518	0\\
519	0\\
520	0\\
521	0\\
522	0\\
523	0\\
524	0\\
525	0\\
526	0\\
527	0\\
528	0\\
529	0\\
530	0\\
531	0\\
532	0\\
533	0\\
534	0\\
535	0\\
536	0\\
537	0\\
538	0\\
539	0\\
540	0\\
541	0\\
542	2.31819356943729e-05\\
543	4.97808753607325e-05\\
544	7.69628614176778e-05\\
545	0.000104747319125633\\
546	0.000133154249522804\\
547	0.000162204210497156\\
548	0.000191918039953681\\
549	0.000222329556741014\\
550	0.000253464773718292\\
551	0.000285344204403849\\
552	0.00031798791545124\\
553	0.000351416424199199\\
554	0.000385650737194353\\
555	0.00042071243627783\\
556	0.000456623783107471\\
557	0.000493407751197362\\
558	0.000531088593263046\\
559	0.00056969136713803\\
560	0.000609241967672466\\
561	0.000649799792050662\\
562	0.000691347682418427\\
563	0.000733888225033008\\
564	0.000777453378418064\\
565	0.00082207615087447\\
566	0.000867771153143799\\
567	0.000914688584196929\\
568	0.000962856697167533\\
569	0.00101209789593015\\
570	0.00106246397516016\\
571	0.00111390186467904\\
572	0.00116698002710393\\
573	0.00122180007812752\\
574	0.00127748242167784\\
575	0.00133371885625811\\
576	0.00139081764574271\\
577	0.00146406439474163\\
578	0.00175490587469454\\
579	0.00206663004967047\\
580	0.00232104643036537\\
581	0.00257412158610118\\
582	0.00277111122536755\\
583	0.00287961411679738\\
584	0.00297466536060553\\
585	0.00306490423400819\\
586	0.00315466992901367\\
587	0.00324537223361411\\
588	0.0033377976514749\\
589	0.00343211111647921\\
590	0.00352885108616147\\
591	0.00362859107064115\\
592	0.00373286830485375\\
593	0.00384555904940007\\
594	0.00397672044113206\\
595	0.00415280575412962\\
596	0.00444436306303141\\
597	0.00503983077166121\\
598	0.00644286460810295\\
599	0\\
600	0\\
};
\addplot [color=red!40!mycolor19,solid,forget plot]
  table[row sep=crcr]{%
1	0\\
2	0\\
3	0\\
4	0\\
5	0\\
6	0\\
7	0\\
8	0\\
9	0\\
10	0\\
11	0\\
12	0\\
13	0\\
14	0\\
15	0\\
16	0\\
17	0\\
18	0\\
19	0\\
20	0\\
21	0\\
22	0\\
23	0\\
24	0\\
25	0\\
26	0\\
27	0\\
28	0\\
29	0\\
30	0\\
31	0\\
32	0\\
33	0\\
34	0\\
35	0\\
36	0\\
37	0\\
38	0\\
39	0\\
40	0\\
41	0\\
42	0\\
43	0\\
44	0\\
45	0\\
46	0\\
47	0\\
48	0\\
49	0\\
50	0\\
51	0\\
52	0\\
53	0\\
54	0\\
55	0\\
56	0\\
57	0\\
58	0\\
59	0\\
60	0\\
61	0\\
62	0\\
63	0\\
64	0\\
65	0\\
66	0\\
67	0\\
68	0\\
69	0\\
70	0\\
71	0\\
72	0\\
73	0\\
74	0\\
75	0\\
76	0\\
77	0\\
78	0\\
79	0\\
80	0\\
81	0\\
82	0\\
83	0\\
84	0\\
85	0\\
86	0\\
87	0\\
88	0\\
89	0\\
90	0\\
91	0\\
92	0\\
93	0\\
94	0\\
95	0\\
96	0\\
97	0\\
98	0\\
99	0\\
100	0\\
101	0\\
102	0\\
103	0\\
104	0\\
105	0\\
106	0\\
107	0\\
108	0\\
109	0\\
110	0\\
111	0\\
112	0\\
113	0\\
114	0\\
115	0\\
116	0\\
117	0\\
118	0\\
119	0\\
120	0\\
121	0\\
122	0\\
123	0\\
124	0\\
125	0\\
126	0\\
127	0\\
128	0\\
129	0\\
130	0\\
131	0\\
132	0\\
133	0\\
134	0\\
135	0\\
136	0\\
137	0\\
138	0\\
139	0\\
140	0\\
141	0\\
142	0\\
143	0\\
144	0\\
145	0\\
146	0\\
147	0\\
148	0\\
149	0\\
150	0\\
151	0\\
152	0\\
153	0\\
154	0\\
155	0\\
156	0\\
157	0\\
158	0\\
159	0\\
160	0\\
161	0\\
162	0\\
163	0\\
164	0\\
165	0\\
166	0\\
167	0\\
168	0\\
169	0\\
170	0\\
171	0\\
172	0\\
173	0\\
174	0\\
175	0\\
176	0\\
177	0\\
178	0\\
179	0\\
180	0\\
181	0\\
182	0\\
183	0\\
184	0\\
185	0\\
186	0\\
187	0\\
188	0\\
189	0\\
190	0\\
191	0\\
192	0\\
193	0\\
194	0\\
195	0\\
196	0\\
197	0\\
198	0\\
199	0\\
200	0\\
201	0\\
202	0\\
203	0\\
204	0\\
205	0\\
206	0\\
207	0\\
208	0\\
209	0\\
210	0\\
211	0\\
212	0\\
213	0\\
214	0\\
215	0\\
216	0\\
217	0\\
218	0\\
219	0\\
220	0\\
221	0\\
222	0\\
223	0\\
224	0\\
225	0\\
226	0\\
227	0\\
228	0\\
229	0\\
230	0\\
231	0\\
232	0\\
233	0\\
234	0\\
235	0\\
236	0\\
237	0\\
238	0\\
239	0\\
240	0\\
241	0\\
242	0\\
243	0\\
244	0\\
245	0\\
246	0\\
247	0\\
248	0\\
249	0\\
250	0\\
251	0\\
252	0\\
253	0\\
254	0\\
255	0\\
256	0\\
257	0\\
258	0\\
259	0\\
260	0\\
261	0\\
262	0\\
263	0\\
264	0\\
265	0\\
266	0\\
267	0\\
268	0\\
269	0\\
270	0\\
271	0\\
272	0\\
273	0\\
274	0\\
275	0\\
276	0\\
277	0\\
278	0\\
279	0\\
280	0\\
281	0\\
282	0\\
283	0\\
284	0\\
285	0\\
286	0\\
287	0\\
288	0\\
289	0\\
290	0\\
291	0\\
292	0\\
293	0\\
294	0\\
295	0\\
296	0\\
297	0\\
298	0\\
299	0\\
300	0\\
301	0\\
302	0\\
303	0\\
304	0\\
305	0\\
306	0\\
307	0\\
308	0\\
309	0\\
310	0\\
311	0\\
312	0\\
313	0\\
314	0\\
315	0\\
316	0\\
317	0\\
318	0\\
319	0\\
320	0\\
321	0\\
322	0\\
323	0\\
324	0\\
325	0\\
326	0\\
327	0\\
328	0\\
329	0\\
330	0\\
331	0\\
332	0\\
333	0\\
334	0\\
335	0\\
336	0\\
337	0\\
338	0\\
339	0\\
340	0\\
341	0\\
342	0\\
343	0\\
344	0\\
345	0\\
346	0\\
347	0\\
348	0\\
349	0\\
350	0\\
351	0\\
352	0\\
353	0\\
354	0\\
355	0\\
356	0\\
357	0\\
358	0\\
359	0\\
360	0\\
361	0\\
362	0\\
363	0\\
364	0\\
365	0\\
366	0\\
367	0\\
368	0\\
369	0\\
370	0\\
371	0\\
372	0\\
373	0\\
374	0\\
375	0\\
376	0\\
377	0\\
378	0\\
379	0\\
380	0\\
381	0\\
382	0\\
383	0\\
384	0\\
385	0\\
386	0\\
387	0\\
388	0\\
389	0\\
390	0\\
391	0\\
392	0\\
393	0\\
394	0\\
395	0\\
396	0\\
397	0\\
398	0\\
399	0\\
400	0\\
401	0\\
402	0\\
403	0\\
404	0\\
405	0\\
406	0\\
407	0\\
408	0\\
409	0\\
410	0\\
411	0\\
412	0\\
413	0\\
414	0\\
415	0\\
416	0\\
417	0\\
418	0\\
419	0\\
420	0\\
421	0\\
422	0\\
423	0\\
424	0\\
425	0\\
426	0\\
427	0\\
428	0\\
429	0\\
430	0\\
431	0\\
432	0\\
433	0\\
434	0\\
435	0\\
436	0\\
437	0\\
438	0\\
439	0\\
440	0\\
441	0\\
442	0\\
443	0\\
444	0\\
445	0\\
446	0\\
447	0\\
448	0\\
449	0\\
450	0\\
451	0\\
452	0\\
453	0\\
454	0\\
455	0\\
456	0\\
457	0\\
458	0\\
459	0\\
460	0\\
461	0\\
462	0\\
463	0\\
464	0\\
465	0\\
466	0\\
467	0\\
468	0\\
469	0\\
470	0\\
471	0\\
472	0\\
473	0\\
474	0\\
475	0\\
476	0\\
477	0\\
478	0\\
479	0\\
480	0\\
481	0\\
482	0\\
483	0\\
484	0\\
485	0\\
486	0\\
487	0\\
488	0\\
489	0\\
490	0\\
491	0\\
492	0\\
493	0\\
494	0\\
495	0\\
496	0\\
497	0\\
498	0\\
499	0\\
500	0\\
501	0\\
502	0\\
503	0\\
504	0\\
505	0\\
506	0\\
507	0\\
508	0\\
509	0\\
510	0\\
511	0\\
512	0\\
513	0\\
514	0\\
515	0\\
516	0\\
517	0\\
518	0\\
519	0\\
520	0\\
521	0\\
522	0\\
523	0\\
524	0\\
525	0\\
526	0\\
527	0\\
528	0\\
529	0\\
530	0\\
531	0\\
532	0\\
533	0\\
534	0\\
535	0\\
536	0\\
537	0\\
538	0\\
539	0\\
540	0\\
541	2.25145755298818e-05\\
542	4.88271536578352e-05\\
543	7.57244088362082e-05\\
544	0.000103224216248849\\
545	0.000131344948204077\\
546	0.000160105482906932\\
547	0.000189525285251365\\
548	0.000219636184945385\\
549	0.00025046323928381\\
550	0.000282024413180058\\
551	0.000314338152631727\\
552	0.000347423417862681\\
553	0.000381299724883977\\
554	0.000415987720097952\\
555	0.000451508665813917\\
556	0.000487884480583933\\
557	0.000525184470262948\\
558	0.000563348240308322\\
559	0.000602398540839097\\
560	0.00064236432155272\\
561	0.00068325304476226\\
562	0.000725110760189988\\
563	0.000767980392286332\\
564	0.000812039238838757\\
565	0.000857143872510442\\
566	0.000903200365809114\\
567	0.000950252013351359\\
568	0.00099832220385339\\
569	0.0010475654496909\\
570	0.00109902998654938\\
571	0.00115110526262422\\
572	0.00120382222347614\\
573	0.00125713529321454\\
574	0.0013114218600033\\
575	0.00138416572299911\\
576	0.00168362914241537\\
577	0.00200121253368579\\
578	0.00225018105728763\\
579	0.00250573441948143\\
580	0.00263216434825482\\
581	0.00272775687042408\\
582	0.00281259168714533\\
583	0.00289684044792754\\
584	0.00298182678347091\\
585	0.00306822904379038\\
586	0.00315632033977421\\
587	0.00324628153905063\\
588	0.00333822811831753\\
589	0.00343235311897413\\
590	0.00352893796606579\\
591	0.00362860835901882\\
592	0.00373286830485375\\
593	0.00384555904940006\\
594	0.00397672044113206\\
595	0.00415280575412962\\
596	0.00444436306303141\\
597	0.00503983077166121\\
598	0.00644286460810295\\
599	0\\
600	0\\
};
\addplot [color=red!75!mycolor17,solid,forget plot]
  table[row sep=crcr]{%
1	0\\
2	0\\
3	0\\
4	0\\
5	0\\
6	0\\
7	0\\
8	0\\
9	0\\
10	0\\
11	0\\
12	0\\
13	0\\
14	0\\
15	0\\
16	0\\
17	0\\
18	0\\
19	0\\
20	0\\
21	0\\
22	0\\
23	0\\
24	0\\
25	0\\
26	0\\
27	0\\
28	0\\
29	0\\
30	0\\
31	0\\
32	0\\
33	0\\
34	0\\
35	0\\
36	0\\
37	0\\
38	0\\
39	0\\
40	0\\
41	0\\
42	0\\
43	0\\
44	0\\
45	0\\
46	0\\
47	0\\
48	0\\
49	0\\
50	0\\
51	0\\
52	0\\
53	0\\
54	0\\
55	0\\
56	0\\
57	0\\
58	0\\
59	0\\
60	0\\
61	0\\
62	0\\
63	0\\
64	0\\
65	0\\
66	0\\
67	0\\
68	0\\
69	0\\
70	0\\
71	0\\
72	0\\
73	0\\
74	0\\
75	0\\
76	0\\
77	0\\
78	0\\
79	0\\
80	0\\
81	0\\
82	0\\
83	0\\
84	0\\
85	0\\
86	0\\
87	0\\
88	0\\
89	0\\
90	0\\
91	0\\
92	0\\
93	0\\
94	0\\
95	0\\
96	0\\
97	0\\
98	0\\
99	0\\
100	0\\
101	0\\
102	0\\
103	0\\
104	0\\
105	0\\
106	0\\
107	0\\
108	0\\
109	0\\
110	0\\
111	0\\
112	0\\
113	0\\
114	0\\
115	0\\
116	0\\
117	0\\
118	0\\
119	0\\
120	0\\
121	0\\
122	0\\
123	0\\
124	0\\
125	0\\
126	0\\
127	0\\
128	0\\
129	0\\
130	0\\
131	0\\
132	0\\
133	0\\
134	0\\
135	0\\
136	0\\
137	0\\
138	0\\
139	0\\
140	0\\
141	0\\
142	0\\
143	0\\
144	0\\
145	0\\
146	0\\
147	0\\
148	0\\
149	0\\
150	0\\
151	0\\
152	0\\
153	0\\
154	0\\
155	0\\
156	0\\
157	0\\
158	0\\
159	0\\
160	0\\
161	0\\
162	0\\
163	0\\
164	0\\
165	0\\
166	0\\
167	0\\
168	0\\
169	0\\
170	0\\
171	0\\
172	0\\
173	0\\
174	0\\
175	0\\
176	0\\
177	0\\
178	0\\
179	0\\
180	0\\
181	0\\
182	0\\
183	0\\
184	0\\
185	0\\
186	0\\
187	0\\
188	0\\
189	0\\
190	0\\
191	0\\
192	0\\
193	0\\
194	0\\
195	0\\
196	0\\
197	0\\
198	0\\
199	0\\
200	0\\
201	0\\
202	0\\
203	0\\
204	0\\
205	0\\
206	0\\
207	0\\
208	0\\
209	0\\
210	0\\
211	0\\
212	0\\
213	0\\
214	0\\
215	0\\
216	0\\
217	0\\
218	0\\
219	0\\
220	0\\
221	0\\
222	0\\
223	0\\
224	0\\
225	0\\
226	0\\
227	0\\
228	0\\
229	0\\
230	0\\
231	0\\
232	0\\
233	0\\
234	0\\
235	0\\
236	0\\
237	0\\
238	0\\
239	0\\
240	0\\
241	0\\
242	0\\
243	0\\
244	0\\
245	0\\
246	0\\
247	0\\
248	0\\
249	0\\
250	0\\
251	0\\
252	0\\
253	0\\
254	0\\
255	0\\
256	0\\
257	0\\
258	0\\
259	0\\
260	0\\
261	0\\
262	0\\
263	0\\
264	0\\
265	0\\
266	0\\
267	0\\
268	0\\
269	0\\
270	0\\
271	0\\
272	0\\
273	0\\
274	0\\
275	0\\
276	0\\
277	0\\
278	0\\
279	0\\
280	0\\
281	0\\
282	0\\
283	0\\
284	0\\
285	0\\
286	0\\
287	0\\
288	0\\
289	0\\
290	0\\
291	0\\
292	0\\
293	0\\
294	0\\
295	0\\
296	0\\
297	0\\
298	0\\
299	0\\
300	0\\
301	0\\
302	0\\
303	0\\
304	0\\
305	0\\
306	0\\
307	0\\
308	0\\
309	0\\
310	0\\
311	0\\
312	0\\
313	0\\
314	0\\
315	0\\
316	0\\
317	0\\
318	0\\
319	0\\
320	0\\
321	0\\
322	0\\
323	0\\
324	0\\
325	0\\
326	0\\
327	0\\
328	0\\
329	0\\
330	0\\
331	0\\
332	0\\
333	0\\
334	0\\
335	0\\
336	0\\
337	0\\
338	0\\
339	0\\
340	0\\
341	0\\
342	0\\
343	0\\
344	0\\
345	0\\
346	0\\
347	0\\
348	0\\
349	0\\
350	0\\
351	0\\
352	0\\
353	0\\
354	0\\
355	0\\
356	0\\
357	0\\
358	0\\
359	0\\
360	0\\
361	0\\
362	0\\
363	0\\
364	0\\
365	0\\
366	0\\
367	0\\
368	0\\
369	0\\
370	0\\
371	0\\
372	0\\
373	0\\
374	0\\
375	0\\
376	0\\
377	0\\
378	0\\
379	0\\
380	0\\
381	0\\
382	0\\
383	0\\
384	0\\
385	0\\
386	0\\
387	0\\
388	0\\
389	0\\
390	0\\
391	0\\
392	0\\
393	0\\
394	0\\
395	0\\
396	0\\
397	0\\
398	0\\
399	0\\
400	0\\
401	0\\
402	0\\
403	0\\
404	0\\
405	0\\
406	0\\
407	0\\
408	0\\
409	0\\
410	0\\
411	0\\
412	0\\
413	0\\
414	0\\
415	0\\
416	0\\
417	0\\
418	0\\
419	0\\
420	0\\
421	0\\
422	0\\
423	0\\
424	0\\
425	0\\
426	0\\
427	0\\
428	0\\
429	0\\
430	0\\
431	0\\
432	0\\
433	0\\
434	0\\
435	0\\
436	0\\
437	0\\
438	0\\
439	0\\
440	0\\
441	0\\
442	0\\
443	0\\
444	0\\
445	0\\
446	0\\
447	0\\
448	0\\
449	0\\
450	0\\
451	0\\
452	0\\
453	0\\
454	0\\
455	0\\
456	0\\
457	0\\
458	0\\
459	0\\
460	0\\
461	0\\
462	0\\
463	0\\
464	0\\
465	0\\
466	0\\
467	0\\
468	0\\
469	0\\
470	0\\
471	0\\
472	0\\
473	0\\
474	0\\
475	0\\
476	0\\
477	0\\
478	0\\
479	0\\
480	0\\
481	0\\
482	0\\
483	0\\
484	0\\
485	0\\
486	0\\
487	0\\
488	0\\
489	0\\
490	0\\
491	0\\
492	0\\
493	0\\
494	0\\
495	0\\
496	0\\
497	0\\
498	0\\
499	0\\
500	0\\
501	0\\
502	0\\
503	0\\
504	0\\
505	0\\
506	0\\
507	0\\
508	0\\
509	0\\
510	0\\
511	0\\
512	0\\
513	0\\
514	0\\
515	0\\
516	0\\
517	0\\
518	0\\
519	0\\
520	0\\
521	0\\
522	0\\
523	0\\
524	0\\
525	0\\
526	0\\
527	0\\
528	0\\
529	0\\
530	0\\
531	0\\
532	0\\
533	0\\
534	0\\
535	0\\
536	0\\
537	0\\
538	0\\
539	0\\
540	1.80572611580464e-05\\
541	4.4118006200525e-05\\
542	7.07542032362198e-05\\
543	9.79828118109392e-05\\
544	0.000125821262619649\\
545	0.000154287471883814\\
546	0.000183399616108164\\
547	0.000213189224881146\\
548	0.000243680143284574\\
549	0.00027488924859386\\
550	0.000306834372255217\\
551	0.000339533902369266\\
552	0.00037300762948773\\
553	0.000407321246047958\\
554	0.000442407509995026\\
555	0.000478288527401575\\
556	0.000514990430784664\\
557	0.000552506247906185\\
558	0.000590897395376276\\
559	0.000630191158809039\\
560	0.000670413411890405\\
561	0.000711735510261396\\
562	0.00075401250105808\\
563	0.000797191808578931\\
564	0.000841224772035708\\
565	0.000886213838323151\\
566	0.000932258739848622\\
567	0.000979764676566103\\
568	0.00102876658617279\\
569	0.00107851644739995\\
570	0.00112845627858277\\
571	0.00117925389621404\\
572	0.00123071645413494\\
573	0.00128292972718838\\
574	0.00156765047686712\\
575	0.00189256888265377\\
576	0.00214409180946553\\
577	0.00239107643868339\\
578	0.00249057912042086\\
579	0.00257377822812136\\
580	0.00265393568636939\\
581	0.00273394370368479\\
582	0.00281520413723953\\
583	0.00289795905152283\\
584	0.00298235807749845\\
585	0.00306849821799598\\
586	0.00315646723333688\\
587	0.003246352526171\\
588	0.00333826624313612\\
589	0.00343236642273867\\
590	0.00352894054086423\\
591	0.00362860835901881\\
592	0.00373286830485375\\
593	0.00384555904940006\\
594	0.00397672044113205\\
595	0.00415280575412962\\
596	0.00444436306303141\\
597	0.00503983077166121\\
598	0.00644286460810295\\
599	0\\
600	0\\
};
\addplot [color=red!80!mycolor19,solid,forget plot]
  table[row sep=crcr]{%
1	0\\
2	0\\
3	0\\
4	0\\
5	0\\
6	0\\
7	0\\
8	0\\
9	0\\
10	0\\
11	0\\
12	0\\
13	0\\
14	0\\
15	0\\
16	0\\
17	0\\
18	0\\
19	0\\
20	0\\
21	0\\
22	0\\
23	0\\
24	0\\
25	0\\
26	0\\
27	0\\
28	0\\
29	0\\
30	0\\
31	0\\
32	0\\
33	0\\
34	0\\
35	0\\
36	0\\
37	0\\
38	0\\
39	0\\
40	0\\
41	0\\
42	0\\
43	0\\
44	0\\
45	0\\
46	0\\
47	0\\
48	0\\
49	0\\
50	0\\
51	0\\
52	0\\
53	0\\
54	0\\
55	0\\
56	0\\
57	0\\
58	0\\
59	0\\
60	0\\
61	0\\
62	0\\
63	0\\
64	0\\
65	0\\
66	0\\
67	0\\
68	0\\
69	0\\
70	0\\
71	0\\
72	0\\
73	0\\
74	0\\
75	0\\
76	0\\
77	0\\
78	0\\
79	0\\
80	0\\
81	0\\
82	0\\
83	0\\
84	0\\
85	0\\
86	0\\
87	0\\
88	0\\
89	0\\
90	0\\
91	0\\
92	0\\
93	0\\
94	0\\
95	0\\
96	0\\
97	0\\
98	0\\
99	0\\
100	0\\
101	0\\
102	0\\
103	0\\
104	0\\
105	0\\
106	0\\
107	0\\
108	0\\
109	0\\
110	0\\
111	0\\
112	0\\
113	0\\
114	0\\
115	0\\
116	0\\
117	0\\
118	0\\
119	0\\
120	0\\
121	0\\
122	0\\
123	0\\
124	0\\
125	0\\
126	0\\
127	0\\
128	0\\
129	0\\
130	0\\
131	0\\
132	0\\
133	0\\
134	0\\
135	0\\
136	0\\
137	0\\
138	0\\
139	0\\
140	0\\
141	0\\
142	0\\
143	0\\
144	0\\
145	0\\
146	0\\
147	0\\
148	0\\
149	0\\
150	0\\
151	0\\
152	0\\
153	0\\
154	0\\
155	0\\
156	0\\
157	0\\
158	0\\
159	0\\
160	0\\
161	0\\
162	0\\
163	0\\
164	0\\
165	0\\
166	0\\
167	0\\
168	0\\
169	0\\
170	0\\
171	0\\
172	0\\
173	0\\
174	0\\
175	0\\
176	0\\
177	0\\
178	0\\
179	0\\
180	0\\
181	0\\
182	0\\
183	0\\
184	0\\
185	0\\
186	0\\
187	0\\
188	0\\
189	0\\
190	0\\
191	0\\
192	0\\
193	0\\
194	0\\
195	0\\
196	0\\
197	0\\
198	0\\
199	0\\
200	0\\
201	0\\
202	0\\
203	0\\
204	0\\
205	0\\
206	0\\
207	0\\
208	0\\
209	0\\
210	0\\
211	0\\
212	0\\
213	0\\
214	0\\
215	0\\
216	0\\
217	0\\
218	0\\
219	0\\
220	0\\
221	0\\
222	0\\
223	0\\
224	0\\
225	0\\
226	0\\
227	0\\
228	0\\
229	0\\
230	0\\
231	0\\
232	0\\
233	0\\
234	0\\
235	0\\
236	0\\
237	0\\
238	0\\
239	0\\
240	0\\
241	0\\
242	0\\
243	0\\
244	0\\
245	0\\
246	0\\
247	0\\
248	0\\
249	0\\
250	0\\
251	0\\
252	0\\
253	0\\
254	0\\
255	0\\
256	0\\
257	0\\
258	0\\
259	0\\
260	0\\
261	0\\
262	0\\
263	0\\
264	0\\
265	0\\
266	0\\
267	0\\
268	0\\
269	0\\
270	0\\
271	0\\
272	0\\
273	0\\
274	0\\
275	0\\
276	0\\
277	0\\
278	0\\
279	0\\
280	0\\
281	0\\
282	0\\
283	0\\
284	0\\
285	0\\
286	0\\
287	0\\
288	0\\
289	0\\
290	0\\
291	0\\
292	0\\
293	0\\
294	0\\
295	0\\
296	0\\
297	0\\
298	0\\
299	0\\
300	0\\
301	0\\
302	0\\
303	0\\
304	0\\
305	0\\
306	0\\
307	0\\
308	0\\
309	0\\
310	0\\
311	0\\
312	0\\
313	0\\
314	0\\
315	0\\
316	0\\
317	0\\
318	0\\
319	0\\
320	0\\
321	0\\
322	0\\
323	0\\
324	0\\
325	0\\
326	0\\
327	0\\
328	0\\
329	0\\
330	0\\
331	0\\
332	0\\
333	0\\
334	0\\
335	0\\
336	0\\
337	0\\
338	0\\
339	0\\
340	0\\
341	0\\
342	0\\
343	0\\
344	0\\
345	0\\
346	0\\
347	0\\
348	0\\
349	0\\
350	0\\
351	0\\
352	0\\
353	0\\
354	0\\
355	0\\
356	0\\
357	0\\
358	0\\
359	0\\
360	0\\
361	0\\
362	0\\
363	0\\
364	0\\
365	0\\
366	0\\
367	0\\
368	0\\
369	0\\
370	0\\
371	0\\
372	0\\
373	0\\
374	0\\
375	0\\
376	0\\
377	0\\
378	0\\
379	0\\
380	0\\
381	0\\
382	0\\
383	0\\
384	0\\
385	0\\
386	0\\
387	0\\
388	0\\
389	0\\
390	0\\
391	0\\
392	0\\
393	0\\
394	0\\
395	0\\
396	0\\
397	0\\
398	0\\
399	0\\
400	0\\
401	0\\
402	0\\
403	0\\
404	0\\
405	0\\
406	0\\
407	0\\
408	0\\
409	0\\
410	0\\
411	0\\
412	0\\
413	0\\
414	0\\
415	0\\
416	0\\
417	0\\
418	0\\
419	0\\
420	0\\
421	0\\
422	0\\
423	0\\
424	0\\
425	0\\
426	0\\
427	0\\
428	0\\
429	0\\
430	0\\
431	0\\
432	0\\
433	0\\
434	0\\
435	0\\
436	0\\
437	0\\
438	0\\
439	0\\
440	0\\
441	0\\
442	0\\
443	0\\
444	0\\
445	0\\
446	0\\
447	0\\
448	0\\
449	0\\
450	0\\
451	0\\
452	0\\
453	0\\
454	0\\
455	0\\
456	0\\
457	0\\
458	0\\
459	0\\
460	0\\
461	0\\
462	0\\
463	0\\
464	0\\
465	0\\
466	0\\
467	0\\
468	0\\
469	0\\
470	0\\
471	0\\
472	0\\
473	0\\
474	0\\
475	0\\
476	0\\
477	0\\
478	0\\
479	0\\
480	0\\
481	0\\
482	0\\
483	0\\
484	0\\
485	0\\
486	0\\
487	0\\
488	0\\
489	0\\
490	0\\
491	0\\
492	0\\
493	0\\
494	0\\
495	0\\
496	0\\
497	0\\
498	0\\
499	0\\
500	0\\
501	0\\
502	0\\
503	0\\
504	0\\
505	0\\
506	0\\
507	0\\
508	0\\
509	0\\
510	0\\
511	0\\
512	0\\
513	0\\
514	0\\
515	0\\
516	0\\
517	0\\
518	0\\
519	0\\
520	0\\
521	0\\
522	0\\
523	0\\
524	0\\
525	0\\
526	0\\
527	0\\
528	0\\
529	0\\
530	0\\
531	0\\
532	0\\
533	0\\
534	0\\
535	0\\
536	0\\
537	0\\
538	0\\
539	1.09971123564833e-05\\
540	3.67777687958609e-05\\
541	6.31213625684237e-05\\
542	9.00441133591266e-05\\
543	0.00011756267562623\\
544	0.000145694153413752\\
545	0.000174455739705744\\
546	0.000203878155683555\\
547	0.00023398579118046\\
548	0.000264795179737243\\
549	0.000296366118124326\\
550	0.000328644365716624\\
551	0.000361637574244406\\
552	0.000395368372265394\\
553	0.000429826296342454\\
554	0.000465072355643675\\
555	0.00050112828782099\\
556	0.000538014176942687\\
557	0.00057573329173453\\
558	0.000614484468182088\\
559	0.000654187719273962\\
560	0.000694706083181759\\
561	0.00073598111257611\\
562	0.00077813831988304\\
563	0.000821236596718493\\
564	0.000865236156171182\\
565	0.000910969534659842\\
566	0.000957920024337713\\
567	0.00100529101218676\\
568	0.0010530258921727\\
569	0.00110148835858687\\
570	0.00115017727882363\\
571	0.00120010163547213\\
572	0.00140776483562387\\
573	0.00174519063704481\\
574	0.00200611363934863\\
575	0.00225233627854361\\
576	0.00234714808882544\\
577	0.00242410810580428\\
578	0.00250007651960148\\
579	0.00257684337440823\\
580	0.00265485410432229\\
581	0.00273433854836808\\
582	0.00281537804625545\\
583	0.0028980434054312\\
584	0.00298240150095583\\
585	0.00306852173201496\\
586	0.00315647873661191\\
587	0.00324635845014596\\
588	0.00333826825403972\\
589	0.00343236680134408\\
590	0.00352894054086423\\
591	0.00362860835901881\\
592	0.00373286830485375\\
593	0.00384555904940006\\
594	0.00397672044113206\\
595	0.00415280575412962\\
596	0.00444436306303141\\
597	0.00503983077166121\\
598	0.00644286460810295\\
599	0\\
600	0\\
};
\addplot [color=red,solid,forget plot]
  table[row sep=crcr]{%
1	0\\
2	0\\
3	0\\
4	0\\
5	0\\
6	0\\
7	0\\
8	0\\
9	0\\
10	0\\
11	0\\
12	0\\
13	0\\
14	0\\
15	0\\
16	0\\
17	0\\
18	0\\
19	0\\
20	0\\
21	0\\
22	0\\
23	0\\
24	0\\
25	0\\
26	0\\
27	0\\
28	0\\
29	0\\
30	0\\
31	0\\
32	0\\
33	0\\
34	0\\
35	0\\
36	0\\
37	0\\
38	0\\
39	0\\
40	0\\
41	0\\
42	0\\
43	0\\
44	0\\
45	0\\
46	0\\
47	0\\
48	0\\
49	0\\
50	0\\
51	0\\
52	0\\
53	0\\
54	0\\
55	0\\
56	0\\
57	0\\
58	0\\
59	0\\
60	0\\
61	0\\
62	0\\
63	0\\
64	0\\
65	0\\
66	0\\
67	0\\
68	0\\
69	0\\
70	0\\
71	0\\
72	0\\
73	0\\
74	0\\
75	0\\
76	0\\
77	0\\
78	0\\
79	0\\
80	0\\
81	0\\
82	0\\
83	0\\
84	0\\
85	0\\
86	0\\
87	0\\
88	0\\
89	0\\
90	0\\
91	0\\
92	0\\
93	0\\
94	0\\
95	0\\
96	0\\
97	0\\
98	0\\
99	0\\
100	0\\
101	0\\
102	0\\
103	0\\
104	0\\
105	0\\
106	0\\
107	0\\
108	0\\
109	0\\
110	0\\
111	0\\
112	0\\
113	0\\
114	0\\
115	0\\
116	0\\
117	0\\
118	0\\
119	0\\
120	0\\
121	0\\
122	0\\
123	0\\
124	0\\
125	0\\
126	0\\
127	0\\
128	0\\
129	0\\
130	0\\
131	0\\
132	0\\
133	0\\
134	0\\
135	0\\
136	0\\
137	0\\
138	0\\
139	0\\
140	0\\
141	0\\
142	0\\
143	0\\
144	0\\
145	0\\
146	0\\
147	0\\
148	0\\
149	0\\
150	0\\
151	0\\
152	0\\
153	0\\
154	0\\
155	0\\
156	0\\
157	0\\
158	0\\
159	0\\
160	0\\
161	0\\
162	0\\
163	0\\
164	0\\
165	0\\
166	0\\
167	0\\
168	0\\
169	0\\
170	0\\
171	0\\
172	0\\
173	0\\
174	0\\
175	0\\
176	0\\
177	0\\
178	0\\
179	0\\
180	0\\
181	0\\
182	0\\
183	0\\
184	0\\
185	0\\
186	0\\
187	0\\
188	0\\
189	0\\
190	0\\
191	0\\
192	0\\
193	0\\
194	0\\
195	0\\
196	0\\
197	0\\
198	0\\
199	0\\
200	0\\
201	0\\
202	0\\
203	0\\
204	0\\
205	0\\
206	0\\
207	0\\
208	0\\
209	0\\
210	0\\
211	0\\
212	0\\
213	0\\
214	0\\
215	0\\
216	0\\
217	0\\
218	0\\
219	0\\
220	0\\
221	0\\
222	0\\
223	0\\
224	0\\
225	0\\
226	0\\
227	0\\
228	0\\
229	0\\
230	0\\
231	0\\
232	0\\
233	0\\
234	0\\
235	0\\
236	0\\
237	0\\
238	0\\
239	0\\
240	0\\
241	0\\
242	0\\
243	0\\
244	0\\
245	0\\
246	0\\
247	0\\
248	0\\
249	0\\
250	0\\
251	0\\
252	0\\
253	0\\
254	0\\
255	0\\
256	0\\
257	0\\
258	0\\
259	0\\
260	0\\
261	0\\
262	0\\
263	0\\
264	0\\
265	0\\
266	0\\
267	0\\
268	0\\
269	0\\
270	0\\
271	0\\
272	0\\
273	0\\
274	0\\
275	0\\
276	0\\
277	0\\
278	0\\
279	0\\
280	0\\
281	0\\
282	0\\
283	0\\
284	0\\
285	0\\
286	0\\
287	0\\
288	0\\
289	0\\
290	0\\
291	0\\
292	0\\
293	0\\
294	0\\
295	0\\
296	0\\
297	0\\
298	0\\
299	0\\
300	0\\
301	0\\
302	0\\
303	0\\
304	0\\
305	0\\
306	0\\
307	0\\
308	0\\
309	0\\
310	0\\
311	0\\
312	0\\
313	0\\
314	0\\
315	0\\
316	0\\
317	0\\
318	0\\
319	0\\
320	0\\
321	0\\
322	0\\
323	0\\
324	0\\
325	0\\
326	0\\
327	0\\
328	0\\
329	0\\
330	0\\
331	0\\
332	0\\
333	0\\
334	0\\
335	0\\
336	0\\
337	0\\
338	0\\
339	0\\
340	0\\
341	0\\
342	0\\
343	0\\
344	0\\
345	0\\
346	0\\
347	0\\
348	0\\
349	0\\
350	0\\
351	0\\
352	0\\
353	0\\
354	0\\
355	0\\
356	0\\
357	0\\
358	0\\
359	0\\
360	0\\
361	0\\
362	0\\
363	0\\
364	0\\
365	0\\
366	0\\
367	0\\
368	0\\
369	0\\
370	0\\
371	0\\
372	0\\
373	0\\
374	0\\
375	0\\
376	0\\
377	0\\
378	0\\
379	0\\
380	0\\
381	0\\
382	0\\
383	0\\
384	0\\
385	0\\
386	0\\
387	0\\
388	0\\
389	0\\
390	0\\
391	0\\
392	0\\
393	0\\
394	0\\
395	0\\
396	0\\
397	0\\
398	0\\
399	0\\
400	0\\
401	0\\
402	0\\
403	0\\
404	0\\
405	0\\
406	0\\
407	0\\
408	0\\
409	0\\
410	0\\
411	0\\
412	0\\
413	0\\
414	0\\
415	0\\
416	0\\
417	0\\
418	0\\
419	0\\
420	0\\
421	0\\
422	0\\
423	0\\
424	0\\
425	0\\
426	0\\
427	0\\
428	0\\
429	0\\
430	0\\
431	0\\
432	0\\
433	0\\
434	0\\
435	0\\
436	0\\
437	0\\
438	0\\
439	0\\
440	0\\
441	0\\
442	0\\
443	0\\
444	0\\
445	0\\
446	0\\
447	0\\
448	0\\
449	0\\
450	0\\
451	0\\
452	0\\
453	0\\
454	0\\
455	0\\
456	0\\
457	0\\
458	0\\
459	0\\
460	0\\
461	0\\
462	0\\
463	0\\
464	0\\
465	0\\
466	0\\
467	0\\
468	0\\
469	0\\
470	0\\
471	0\\
472	0\\
473	0\\
474	0\\
475	0\\
476	0\\
477	0\\
478	0\\
479	0\\
480	0\\
481	0\\
482	0\\
483	0\\
484	0\\
485	0\\
486	0\\
487	0\\
488	0\\
489	0\\
490	0\\
491	0\\
492	0\\
493	0\\
494	0\\
495	0\\
496	0\\
497	0\\
498	0\\
499	0\\
500	0\\
501	0\\
502	0\\
503	0\\
504	0\\
505	0\\
506	0\\
507	0\\
508	0\\
509	0\\
510	0\\
511	0\\
512	0\\
513	0\\
514	0\\
515	0\\
516	0\\
517	0\\
518	0\\
519	0\\
520	0\\
521	0\\
522	0\\
523	0\\
524	0\\
525	0\\
526	0\\
527	0\\
528	0\\
529	0\\
530	0\\
531	0\\
532	0\\
533	0\\
534	0\\
535	0\\
536	0\\
537	0\\
538	1.87556343843139e-06\\
539	2.73381310324152e-05\\
540	5.33495288270524e-05\\
541	7.99252542932924e-05\\
542	0.000107081251424983\\
543	0.000134834453286783\\
544	0.000163201833838333\\
545	0.000192240295758037\\
546	0.000221947411190644\\
547	0.000252298954717094\\
548	0.000283315360389555\\
549	0.000314986488951994\\
550	0.000347362970500003\\
551	0.000380469167030157\\
552	0.000414321565112513\\
553	0.000448919563346855\\
554	0.00048433294388368\\
555	0.000520634170094015\\
556	0.000557948080283569\\
557	0.000596000691589419\\
558	0.000634749598647108\\
559	0.000674269636362017\\
560	0.000714637530630376\\
561	0.000755805415251623\\
562	0.000797923053763769\\
563	0.000841878133127252\\
564	0.000886913723009165\\
565	0.00093214151302975\\
566	0.000977757767295516\\
567	0.00102380313004\\
568	0.00107036709861644\\
569	0.00111805503001417\\
570	0.00120939113958143\\
571	0.00154663998556902\\
572	0.00183959615121693\\
573	0.00209756278158636\\
574	0.00220493769069792\\
575	0.00227885628634348\\
576	0.00235163153275207\\
577	0.00242542435592874\\
578	0.00250050669328025\\
579	0.00257697908889624\\
580	0.00265491355281054\\
581	0.00273436544093214\\
582	0.00281539134096267\\
583	0.00289805033361497\\
584	0.00298240522740585\\
585	0.00306852356540592\\
586	0.00315647964482259\\
587	0.0032463587502771\\
588	0.0033382683090296\\
589	0.00343236680134408\\
590	0.00352894054086422\\
591	0.00362860835901881\\
592	0.00373286830485375\\
593	0.00384555904940006\\
594	0.00397672044113206\\
595	0.00415280575412962\\
596	0.00444436306303141\\
597	0.00503983077166121\\
598	0.00644286460810295\\
599	0\\
600	0\\
};
\addplot [color=mycolor20,solid,forget plot]
  table[row sep=crcr]{%
1	0\\
2	0\\
3	0\\
4	0\\
5	0\\
6	0\\
7	0\\
8	0\\
9	0\\
10	0\\
11	0\\
12	0\\
13	0\\
14	0\\
15	0\\
16	0\\
17	0\\
18	0\\
19	0\\
20	0\\
21	0\\
22	0\\
23	0\\
24	0\\
25	0\\
26	0\\
27	0\\
28	0\\
29	0\\
30	0\\
31	0\\
32	0\\
33	0\\
34	0\\
35	0\\
36	0\\
37	0\\
38	0\\
39	0\\
40	0\\
41	0\\
42	0\\
43	0\\
44	0\\
45	0\\
46	0\\
47	0\\
48	0\\
49	0\\
50	0\\
51	0\\
52	0\\
53	0\\
54	0\\
55	0\\
56	0\\
57	0\\
58	0\\
59	0\\
60	0\\
61	0\\
62	0\\
63	0\\
64	0\\
65	0\\
66	0\\
67	0\\
68	0\\
69	0\\
70	0\\
71	0\\
72	0\\
73	0\\
74	0\\
75	0\\
76	0\\
77	0\\
78	0\\
79	0\\
80	0\\
81	0\\
82	0\\
83	0\\
84	0\\
85	0\\
86	0\\
87	0\\
88	0\\
89	0\\
90	0\\
91	0\\
92	0\\
93	0\\
94	0\\
95	0\\
96	0\\
97	0\\
98	0\\
99	0\\
100	0\\
101	0\\
102	0\\
103	0\\
104	0\\
105	0\\
106	0\\
107	0\\
108	0\\
109	0\\
110	0\\
111	0\\
112	0\\
113	0\\
114	0\\
115	0\\
116	0\\
117	0\\
118	0\\
119	0\\
120	0\\
121	0\\
122	0\\
123	0\\
124	0\\
125	0\\
126	0\\
127	0\\
128	0\\
129	0\\
130	0\\
131	0\\
132	0\\
133	0\\
134	0\\
135	0\\
136	0\\
137	0\\
138	0\\
139	0\\
140	0\\
141	0\\
142	0\\
143	0\\
144	0\\
145	0\\
146	0\\
147	0\\
148	0\\
149	0\\
150	0\\
151	0\\
152	0\\
153	0\\
154	0\\
155	0\\
156	0\\
157	0\\
158	0\\
159	0\\
160	0\\
161	0\\
162	0\\
163	0\\
164	0\\
165	0\\
166	0\\
167	0\\
168	0\\
169	0\\
170	0\\
171	0\\
172	0\\
173	0\\
174	0\\
175	0\\
176	0\\
177	0\\
178	0\\
179	0\\
180	0\\
181	0\\
182	0\\
183	0\\
184	0\\
185	0\\
186	0\\
187	0\\
188	0\\
189	0\\
190	0\\
191	0\\
192	0\\
193	0\\
194	0\\
195	0\\
196	0\\
197	0\\
198	0\\
199	0\\
200	0\\
201	0\\
202	0\\
203	0\\
204	0\\
205	0\\
206	0\\
207	0\\
208	0\\
209	0\\
210	0\\
211	0\\
212	0\\
213	0\\
214	0\\
215	0\\
216	0\\
217	0\\
218	0\\
219	0\\
220	0\\
221	0\\
222	0\\
223	0\\
224	0\\
225	0\\
226	0\\
227	0\\
228	0\\
229	0\\
230	0\\
231	0\\
232	0\\
233	0\\
234	0\\
235	0\\
236	0\\
237	0\\
238	0\\
239	0\\
240	0\\
241	0\\
242	0\\
243	0\\
244	0\\
245	0\\
246	0\\
247	0\\
248	0\\
249	0\\
250	0\\
251	0\\
252	0\\
253	0\\
254	0\\
255	0\\
256	0\\
257	0\\
258	0\\
259	0\\
260	0\\
261	0\\
262	0\\
263	0\\
264	0\\
265	0\\
266	0\\
267	0\\
268	0\\
269	0\\
270	0\\
271	0\\
272	0\\
273	0\\
274	0\\
275	0\\
276	0\\
277	0\\
278	0\\
279	0\\
280	0\\
281	0\\
282	0\\
283	0\\
284	0\\
285	0\\
286	0\\
287	0\\
288	0\\
289	0\\
290	0\\
291	0\\
292	0\\
293	0\\
294	0\\
295	0\\
296	0\\
297	0\\
298	0\\
299	0\\
300	0\\
301	0\\
302	0\\
303	0\\
304	0\\
305	0\\
306	0\\
307	0\\
308	0\\
309	0\\
310	0\\
311	0\\
312	0\\
313	0\\
314	0\\
315	0\\
316	0\\
317	0\\
318	0\\
319	0\\
320	0\\
321	0\\
322	0\\
323	0\\
324	0\\
325	0\\
326	0\\
327	0\\
328	0\\
329	0\\
330	0\\
331	0\\
332	0\\
333	0\\
334	0\\
335	0\\
336	0\\
337	0\\
338	0\\
339	0\\
340	0\\
341	0\\
342	0\\
343	0\\
344	0\\
345	0\\
346	0\\
347	0\\
348	0\\
349	0\\
350	0\\
351	0\\
352	0\\
353	0\\
354	0\\
355	0\\
356	0\\
357	0\\
358	0\\
359	0\\
360	0\\
361	0\\
362	0\\
363	0\\
364	0\\
365	0\\
366	0\\
367	0\\
368	0\\
369	0\\
370	0\\
371	0\\
372	0\\
373	0\\
374	0\\
375	0\\
376	0\\
377	0\\
378	0\\
379	0\\
380	0\\
381	0\\
382	0\\
383	0\\
384	0\\
385	0\\
386	0\\
387	0\\
388	0\\
389	0\\
390	0\\
391	0\\
392	0\\
393	0\\
394	0\\
395	0\\
396	0\\
397	0\\
398	0\\
399	0\\
400	0\\
401	0\\
402	0\\
403	0\\
404	0\\
405	0\\
406	0\\
407	0\\
408	0\\
409	0\\
410	0\\
411	0\\
412	0\\
413	0\\
414	0\\
415	0\\
416	0\\
417	0\\
418	0\\
419	0\\
420	0\\
421	0\\
422	0\\
423	0\\
424	0\\
425	0\\
426	0\\
427	0\\
428	0\\
429	0\\
430	0\\
431	0\\
432	0\\
433	0\\
434	0\\
435	0\\
436	0\\
437	0\\
438	0\\
439	0\\
440	0\\
441	0\\
442	0\\
443	0\\
444	0\\
445	0\\
446	0\\
447	0\\
448	0\\
449	0\\
450	0\\
451	0\\
452	0\\
453	0\\
454	0\\
455	0\\
456	0\\
457	0\\
458	0\\
459	0\\
460	0\\
461	0\\
462	0\\
463	0\\
464	0\\
465	0\\
466	0\\
467	0\\
468	0\\
469	0\\
470	0\\
471	0\\
472	0\\
473	0\\
474	0\\
475	0\\
476	0\\
477	0\\
478	0\\
479	0\\
480	0\\
481	0\\
482	0\\
483	0\\
484	0\\
485	0\\
486	0\\
487	0\\
488	0\\
489	0\\
490	0\\
491	0\\
492	0\\
493	0\\
494	0\\
495	0\\
496	0\\
497	0\\
498	0\\
499	0\\
500	0\\
501	0\\
502	0\\
503	0\\
504	0\\
505	0\\
506	0\\
507	0\\
508	0\\
509	0\\
510	0\\
511	0\\
512	0\\
513	0\\
514	0\\
515	0\\
516	0\\
517	0\\
518	0\\
519	0\\
520	0\\
521	0\\
522	0\\
523	0\\
524	0\\
525	0\\
526	0\\
527	0\\
528	0\\
529	0\\
530	0\\
531	0\\
532	0\\
533	0\\
534	0\\
535	0\\
536	0\\
537	0\\
538	1.6243643999888e-05\\
539	4.18918417134826e-05\\
540	6.80899130433921e-05\\
541	9.48564491413589e-05\\
542	0.000122248418487188\\
543	0.000150198053260629\\
544	0.000178739275558913\\
545	0.000207878949442831\\
546	0.000237642445840086\\
547	0.000268066066433872\\
548	0.000299164768053091\\
549	0.000330937123302737\\
550	0.000363440569240543\\
551	0.000396699448537722\\
552	0.000430733142917223\\
553	0.000465742743696337\\
554	0.000501529298393566\\
555	0.000537988092118647\\
556	0.000575079712924593\\
557	0.000612943047238721\\
558	0.000651541350800113\\
559	0.000690963850311138\\
560	0.000731349768291028\\
561	0.000773468762908814\\
562	0.000816756500267696\\
563	0.000860107264034101\\
564	0.000903803885929959\\
565	0.000947716400417519\\
566	0.000992275947024247\\
567	0.00103793439844082\\
568	0.00108481900650592\\
569	0.00130432755340724\\
570	0.00164623612034606\\
571	0.00190410168763289\\
572	0.00206499561851727\\
573	0.00213820104703062\\
574	0.00220835825523849\\
575	0.00227946097953584\\
576	0.00235181287951247\\
577	0.00242548455611651\\
578	0.0025005266540219\\
579	0.00257698800197174\\
580	0.00265491768689218\\
581	0.0027343675196084\\
582	0.00281539243421504\\
583	0.00289805091793385\\
584	0.00298240551503779\\
585	0.00306852370283625\\
586	0.00315647968906845\\
587	0.00324635875816935\\
588	0.00333826830902959\\
589	0.00343236680134408\\
590	0.00352894054086422\\
591	0.00362860835901881\\
592	0.00373286830485375\\
593	0.00384555904940006\\
594	0.00397672044113206\\
595	0.00415280575412961\\
596	0.00444436306303141\\
597	0.00503983077166121\\
598	0.00644286460810295\\
599	0\\
600	0\\
};
\addplot [color=mycolor21,solid,forget plot]
  table[row sep=crcr]{%
1	0\\
2	0\\
3	0\\
4	0\\
5	0\\
6	0\\
7	0\\
8	0\\
9	0\\
10	0\\
11	0\\
12	0\\
13	0\\
14	0\\
15	0\\
16	0\\
17	0\\
18	0\\
19	0\\
20	0\\
21	0\\
22	0\\
23	0\\
24	0\\
25	0\\
26	0\\
27	0\\
28	0\\
29	0\\
30	0\\
31	0\\
32	0\\
33	0\\
34	0\\
35	0\\
36	0\\
37	0\\
38	0\\
39	0\\
40	0\\
41	0\\
42	0\\
43	0\\
44	0\\
45	0\\
46	0\\
47	0\\
48	0\\
49	0\\
50	0\\
51	0\\
52	0\\
53	0\\
54	0\\
55	0\\
56	0\\
57	0\\
58	0\\
59	0\\
60	0\\
61	0\\
62	0\\
63	0\\
64	0\\
65	0\\
66	0\\
67	0\\
68	0\\
69	0\\
70	0\\
71	0\\
72	0\\
73	0\\
74	0\\
75	0\\
76	0\\
77	0\\
78	0\\
79	0\\
80	0\\
81	0\\
82	0\\
83	0\\
84	0\\
85	0\\
86	0\\
87	0\\
88	0\\
89	0\\
90	0\\
91	0\\
92	0\\
93	0\\
94	0\\
95	0\\
96	0\\
97	0\\
98	0\\
99	0\\
100	0\\
101	0\\
102	0\\
103	0\\
104	0\\
105	0\\
106	0\\
107	0\\
108	0\\
109	0\\
110	0\\
111	0\\
112	0\\
113	0\\
114	0\\
115	0\\
116	0\\
117	0\\
118	0\\
119	0\\
120	0\\
121	0\\
122	0\\
123	0\\
124	0\\
125	0\\
126	0\\
127	0\\
128	0\\
129	0\\
130	0\\
131	0\\
132	0\\
133	0\\
134	0\\
135	0\\
136	0\\
137	0\\
138	0\\
139	0\\
140	0\\
141	0\\
142	0\\
143	0\\
144	0\\
145	0\\
146	0\\
147	0\\
148	0\\
149	0\\
150	0\\
151	0\\
152	0\\
153	0\\
154	0\\
155	0\\
156	0\\
157	0\\
158	0\\
159	0\\
160	0\\
161	0\\
162	0\\
163	0\\
164	0\\
165	0\\
166	0\\
167	0\\
168	0\\
169	0\\
170	0\\
171	0\\
172	0\\
173	0\\
174	0\\
175	0\\
176	0\\
177	0\\
178	0\\
179	0\\
180	0\\
181	0\\
182	0\\
183	0\\
184	0\\
185	0\\
186	0\\
187	0\\
188	0\\
189	0\\
190	0\\
191	0\\
192	0\\
193	0\\
194	0\\
195	0\\
196	0\\
197	0\\
198	0\\
199	0\\
200	0\\
201	0\\
202	0\\
203	0\\
204	0\\
205	0\\
206	0\\
207	0\\
208	0\\
209	0\\
210	0\\
211	0\\
212	0\\
213	0\\
214	0\\
215	0\\
216	0\\
217	0\\
218	0\\
219	0\\
220	0\\
221	0\\
222	0\\
223	0\\
224	0\\
225	0\\
226	0\\
227	0\\
228	0\\
229	0\\
230	0\\
231	0\\
232	0\\
233	0\\
234	0\\
235	0\\
236	0\\
237	0\\
238	0\\
239	0\\
240	0\\
241	0\\
242	0\\
243	0\\
244	0\\
245	0\\
246	0\\
247	0\\
248	0\\
249	0\\
250	0\\
251	0\\
252	0\\
253	0\\
254	0\\
255	0\\
256	0\\
257	0\\
258	0\\
259	0\\
260	0\\
261	0\\
262	0\\
263	0\\
264	0\\
265	0\\
266	0\\
267	0\\
268	0\\
269	0\\
270	0\\
271	0\\
272	0\\
273	0\\
274	0\\
275	0\\
276	0\\
277	0\\
278	0\\
279	0\\
280	0\\
281	0\\
282	0\\
283	0\\
284	0\\
285	0\\
286	0\\
287	0\\
288	0\\
289	0\\
290	0\\
291	0\\
292	0\\
293	0\\
294	0\\
295	0\\
296	0\\
297	0\\
298	0\\
299	0\\
300	0\\
301	0\\
302	0\\
303	0\\
304	0\\
305	0\\
306	0\\
307	0\\
308	0\\
309	0\\
310	0\\
311	0\\
312	0\\
313	0\\
314	0\\
315	0\\
316	0\\
317	0\\
318	0\\
319	0\\
320	0\\
321	0\\
322	0\\
323	0\\
324	0\\
325	0\\
326	0\\
327	0\\
328	0\\
329	0\\
330	0\\
331	0\\
332	0\\
333	0\\
334	0\\
335	0\\
336	0\\
337	0\\
338	0\\
339	0\\
340	0\\
341	0\\
342	0\\
343	0\\
344	0\\
345	0\\
346	0\\
347	0\\
348	0\\
349	0\\
350	0\\
351	0\\
352	0\\
353	0\\
354	0\\
355	0\\
356	0\\
357	0\\
358	0\\
359	0\\
360	0\\
361	0\\
362	0\\
363	0\\
364	0\\
365	0\\
366	0\\
367	0\\
368	0\\
369	0\\
370	0\\
371	0\\
372	0\\
373	0\\
374	0\\
375	0\\
376	0\\
377	0\\
378	0\\
379	0\\
380	0\\
381	0\\
382	0\\
383	0\\
384	0\\
385	0\\
386	0\\
387	0\\
388	0\\
389	0\\
390	0\\
391	0\\
392	0\\
393	0\\
394	0\\
395	0\\
396	0\\
397	0\\
398	0\\
399	0\\
400	0\\
401	0\\
402	0\\
403	0\\
404	0\\
405	0\\
406	0\\
407	0\\
408	0\\
409	0\\
410	0\\
411	0\\
412	0\\
413	0\\
414	0\\
415	0\\
416	0\\
417	0\\
418	0\\
419	0\\
420	0\\
421	0\\
422	0\\
423	0\\
424	0\\
425	0\\
426	0\\
427	0\\
428	0\\
429	0\\
430	0\\
431	0\\
432	0\\
433	0\\
434	0\\
435	0\\
436	0\\
437	0\\
438	0\\
439	0\\
440	0\\
441	0\\
442	0\\
443	0\\
444	0\\
445	0\\
446	0\\
447	0\\
448	0\\
449	0\\
450	0\\
451	0\\
452	0\\
453	0\\
454	0\\
455	0\\
456	0\\
457	0\\
458	0\\
459	0\\
460	0\\
461	0\\
462	0\\
463	0\\
464	0\\
465	0\\
466	0\\
467	0\\
468	0\\
469	0\\
470	0\\
471	0\\
472	0\\
473	0\\
474	0\\
475	0\\
476	0\\
477	0\\
478	0\\
479	0\\
480	0\\
481	0\\
482	0\\
483	0\\
484	0\\
485	0\\
486	0\\
487	0\\
488	0\\
489	0\\
490	0\\
491	0\\
492	0\\
493	0\\
494	0\\
495	0\\
496	0\\
497	0\\
498	0\\
499	0\\
500	0\\
501	0\\
502	0\\
503	0\\
504	0\\
505	0\\
506	0\\
507	0\\
508	0\\
509	0\\
510	0\\
511	0\\
512	0\\
513	0\\
514	0\\
515	0\\
516	0\\
517	0\\
518	0\\
519	0\\
520	0\\
521	0\\
522	0\\
523	0\\
524	0\\
525	0\\
526	0\\
527	0\\
528	0\\
529	0\\
530	0\\
531	0\\
532	0\\
533	0\\
534	0\\
535	0\\
536	0\\
537	3.82550301366919e-06\\
538	2.9106131297474e-05\\
539	5.49338526053371e-05\\
540	8.12709214509474e-05\\
541	0.000108136282402065\\
542	0.000135519390228108\\
543	0.000163492059064573\\
544	0.000192077027657505\\
545	0.000221277232734906\\
546	0.000251125248797619\\
547	0.00028165907332523\\
548	0.000312894077851403\\
549	0.000344852457097526\\
550	0.000377575892313951\\
551	0.000411272352801193\\
552	0.000445600411375425\\
553	0.000480493797716272\\
554	0.000516041013830302\\
555	0.000552284502233579\\
556	0.000589200581950451\\
557	0.000626998320133113\\
558	0.000665716783444826\\
559	0.000705887875314932\\
560	0.000747561296246264\\
561	0.000789266191669505\\
562	0.000831174369496882\\
563	0.000873178095902131\\
564	0.000915807659414534\\
565	0.000959488231720514\\
566	0.00100432096429517\\
567	0.0010503705610188\\
568	0.00138555014159372\\
569	0.00168867149361451\\
570	0.00192455116100545\\
571	0.00200205571737336\\
572	0.00207001668793317\\
573	0.00213863941240985\\
574	0.00220843933480784\\
575	0.00227948582459503\\
576	0.00235182128121044\\
577	0.00242548747819655\\
578	0.00250052798438547\\
579	0.00257698863349794\\
580	0.0026549180091809\\
581	0.0027343676902359\\
582	0.0028153925248352\\
583	0.00289805096238546\\
584	0.00298240553557085\\
585	0.00306852370928107\\
586	0.00315647969018814\\
587	0.00324635875816936\\
588	0.0033382683090296\\
589	0.00343236680134408\\
590	0.00352894054086422\\
591	0.00362860835901882\\
592	0.00373286830485375\\
593	0.00384555904940006\\
594	0.00397672044113206\\
595	0.00415280575412961\\
596	0.00444436306303141\\
597	0.00503983077166121\\
598	0.00644286460810295\\
599	0\\
600	0\\
};
\addplot [color=black!20!mycolor21,solid,forget plot]
  table[row sep=crcr]{%
1	0\\
2	0\\
3	0\\
4	0\\
5	0\\
6	0\\
7	0\\
8	0\\
9	0\\
10	0\\
11	0\\
12	0\\
13	0\\
14	0\\
15	0\\
16	0\\
17	0\\
18	0\\
19	0\\
20	0\\
21	0\\
22	0\\
23	0\\
24	0\\
25	0\\
26	0\\
27	0\\
28	0\\
29	0\\
30	0\\
31	0\\
32	0\\
33	0\\
34	0\\
35	0\\
36	0\\
37	0\\
38	0\\
39	0\\
40	0\\
41	0\\
42	0\\
43	0\\
44	0\\
45	0\\
46	0\\
47	0\\
48	0\\
49	0\\
50	0\\
51	0\\
52	0\\
53	0\\
54	0\\
55	0\\
56	0\\
57	0\\
58	0\\
59	0\\
60	0\\
61	0\\
62	0\\
63	0\\
64	0\\
65	0\\
66	0\\
67	0\\
68	0\\
69	0\\
70	0\\
71	0\\
72	0\\
73	0\\
74	0\\
75	0\\
76	0\\
77	0\\
78	0\\
79	0\\
80	0\\
81	0\\
82	0\\
83	0\\
84	0\\
85	0\\
86	0\\
87	0\\
88	0\\
89	0\\
90	0\\
91	0\\
92	0\\
93	0\\
94	0\\
95	0\\
96	0\\
97	0\\
98	0\\
99	0\\
100	0\\
101	0\\
102	0\\
103	0\\
104	0\\
105	0\\
106	0\\
107	0\\
108	0\\
109	0\\
110	0\\
111	0\\
112	0\\
113	0\\
114	0\\
115	0\\
116	0\\
117	0\\
118	0\\
119	0\\
120	0\\
121	0\\
122	0\\
123	0\\
124	0\\
125	0\\
126	0\\
127	0\\
128	0\\
129	0\\
130	0\\
131	0\\
132	0\\
133	0\\
134	0\\
135	0\\
136	0\\
137	0\\
138	0\\
139	0\\
140	0\\
141	0\\
142	0\\
143	0\\
144	0\\
145	0\\
146	0\\
147	0\\
148	0\\
149	0\\
150	0\\
151	0\\
152	0\\
153	0\\
154	0\\
155	0\\
156	0\\
157	0\\
158	0\\
159	0\\
160	0\\
161	0\\
162	0\\
163	0\\
164	0\\
165	0\\
166	0\\
167	0\\
168	0\\
169	0\\
170	0\\
171	0\\
172	0\\
173	0\\
174	0\\
175	0\\
176	0\\
177	0\\
178	0\\
179	0\\
180	0\\
181	0\\
182	0\\
183	0\\
184	0\\
185	0\\
186	0\\
187	0\\
188	0\\
189	0\\
190	0\\
191	0\\
192	0\\
193	0\\
194	0\\
195	0\\
196	0\\
197	0\\
198	0\\
199	0\\
200	0\\
201	0\\
202	0\\
203	0\\
204	0\\
205	0\\
206	0\\
207	0\\
208	0\\
209	0\\
210	0\\
211	0\\
212	0\\
213	0\\
214	0\\
215	0\\
216	0\\
217	0\\
218	0\\
219	0\\
220	0\\
221	0\\
222	0\\
223	0\\
224	0\\
225	0\\
226	0\\
227	0\\
228	0\\
229	0\\
230	0\\
231	0\\
232	0\\
233	0\\
234	0\\
235	0\\
236	0\\
237	0\\
238	0\\
239	0\\
240	0\\
241	0\\
242	0\\
243	0\\
244	0\\
245	0\\
246	0\\
247	0\\
248	0\\
249	0\\
250	0\\
251	0\\
252	0\\
253	0\\
254	0\\
255	0\\
256	0\\
257	0\\
258	0\\
259	0\\
260	0\\
261	0\\
262	0\\
263	0\\
264	0\\
265	0\\
266	0\\
267	0\\
268	0\\
269	0\\
270	0\\
271	0\\
272	0\\
273	0\\
274	0\\
275	0\\
276	0\\
277	0\\
278	0\\
279	0\\
280	0\\
281	0\\
282	0\\
283	0\\
284	0\\
285	0\\
286	0\\
287	0\\
288	0\\
289	0\\
290	0\\
291	0\\
292	0\\
293	0\\
294	0\\
295	0\\
296	0\\
297	0\\
298	0\\
299	0\\
300	0\\
301	0\\
302	0\\
303	0\\
304	0\\
305	0\\
306	0\\
307	0\\
308	0\\
309	0\\
310	0\\
311	0\\
312	0\\
313	0\\
314	0\\
315	0\\
316	0\\
317	0\\
318	0\\
319	0\\
320	0\\
321	0\\
322	0\\
323	0\\
324	0\\
325	0\\
326	0\\
327	0\\
328	0\\
329	0\\
330	0\\
331	0\\
332	0\\
333	0\\
334	0\\
335	0\\
336	0\\
337	0\\
338	0\\
339	0\\
340	0\\
341	0\\
342	0\\
343	0\\
344	0\\
345	0\\
346	0\\
347	0\\
348	0\\
349	0\\
350	0\\
351	0\\
352	0\\
353	0\\
354	0\\
355	0\\
356	0\\
357	0\\
358	0\\
359	0\\
360	0\\
361	0\\
362	0\\
363	0\\
364	0\\
365	0\\
366	0\\
367	0\\
368	0\\
369	0\\
370	0\\
371	0\\
372	0\\
373	0\\
374	0\\
375	0\\
376	0\\
377	0\\
378	0\\
379	0\\
380	0\\
381	0\\
382	0\\
383	0\\
384	0\\
385	0\\
386	0\\
387	0\\
388	0\\
389	0\\
390	0\\
391	0\\
392	0\\
393	0\\
394	0\\
395	0\\
396	0\\
397	0\\
398	0\\
399	0\\
400	0\\
401	0\\
402	0\\
403	0\\
404	0\\
405	0\\
406	0\\
407	0\\
408	0\\
409	0\\
410	0\\
411	0\\
412	0\\
413	0\\
414	0\\
415	0\\
416	0\\
417	0\\
418	0\\
419	0\\
420	0\\
421	0\\
422	0\\
423	0\\
424	0\\
425	0\\
426	0\\
427	0\\
428	0\\
429	0\\
430	0\\
431	0\\
432	0\\
433	0\\
434	0\\
435	0\\
436	0\\
437	0\\
438	0\\
439	0\\
440	0\\
441	0\\
442	0\\
443	0\\
444	0\\
445	0\\
446	0\\
447	0\\
448	0\\
449	0\\
450	0\\
451	0\\
452	0\\
453	0\\
454	0\\
455	0\\
456	0\\
457	0\\
458	0\\
459	0\\
460	0\\
461	0\\
462	0\\
463	0\\
464	0\\
465	0\\
466	0\\
467	0\\
468	0\\
469	0\\
470	0\\
471	0\\
472	0\\
473	0\\
474	0\\
475	0\\
476	0\\
477	0\\
478	0\\
479	0\\
480	0\\
481	0\\
482	0\\
483	0\\
484	0\\
485	0\\
486	0\\
487	0\\
488	0\\
489	0\\
490	0\\
491	0\\
492	0\\
493	0\\
494	0\\
495	0\\
496	0\\
497	0\\
498	0\\
499	0\\
500	0\\
501	0\\
502	0\\
503	0\\
504	0\\
505	0\\
506	0\\
507	0\\
508	0\\
509	0\\
510	0\\
511	0\\
512	0\\
513	0\\
514	0\\
515	0\\
516	0\\
517	0\\
518	0\\
519	0\\
520	0\\
521	0\\
522	0\\
523	0\\
524	0\\
525	0\\
526	0\\
527	0\\
528	0\\
529	0\\
530	0\\
531	0\\
532	0\\
533	0\\
534	0\\
535	0\\
536	0\\
537	1.5257515615078e-05\\
538	4.05712495559513e-05\\
539	6.63714757039172e-05\\
540	9.27016742245395e-05\\
541	0.000119575710927847\\
542	0.000147000013995511\\
543	0.00017504869257464\\
544	0.000203735987885211\\
545	0.000233079425442273\\
546	0.000263096693183489\\
547	0.000293803947281356\\
548	0.000325302371137699\\
549	0.000357642385253218\\
550	0.000390582931632927\\
551	0.00042399701337393\\
552	0.000458068777525668\\
553	0.000492727463226109\\
554	0.000528135608302436\\
555	0.000564380253208309\\
556	0.000601502109540624\\
557	0.000639665961938304\\
558	0.000679881186959353\\
559	0.000720197887011426\\
560	0.000760489350549704\\
561	0.00080084928102869\\
562	0.000841683425308702\\
563	0.000883510817814942\\
564	0.000926424297840563\\
565	0.000970477398949334\\
566	0.00106843923578318\\
567	0.00144259009169947\\
568	0.00171200842408046\\
569	0.0018698844738892\\
570	0.00193635519907031\\
571	0.00200267110891418\\
572	0.00207007254415696\\
573	0.00213865022157796\\
574	0.00220844272080478\\
575	0.00227948699417363\\
576	0.0023518217070068\\
577	0.0024254876758363\\
578	0.00250052808022166\\
579	0.0025769886830341\\
580	0.00265491803552198\\
581	0.00273436770413291\\
582	0.00281539253160681\\
583	0.0028980509654155\\
584	0.00298240553649866\\
585	0.00306852370943816\\
586	0.00315647969018814\\
587	0.00324635875816936\\
588	0.0033382683090296\\
589	0.00343236680134408\\
590	0.00352894054086423\\
591	0.00362860835901881\\
592	0.00373286830485375\\
593	0.00384555904940006\\
594	0.00397672044113205\\
595	0.00415280575412962\\
596	0.00444436306303141\\
597	0.00503983077166121\\
598	0.00644286460810295\\
599	0\\
600	0\\
};
\addplot [color=black!50!mycolor20,solid,forget plot]
  table[row sep=crcr]{%
1	0\\
2	0\\
3	0\\
4	0\\
5	0\\
6	0\\
7	0\\
8	0\\
9	0\\
10	0\\
11	0\\
12	0\\
13	0\\
14	0\\
15	0\\
16	0\\
17	0\\
18	0\\
19	0\\
20	0\\
21	0\\
22	0\\
23	0\\
24	0\\
25	0\\
26	0\\
27	0\\
28	0\\
29	0\\
30	0\\
31	0\\
32	0\\
33	0\\
34	0\\
35	0\\
36	0\\
37	0\\
38	0\\
39	0\\
40	0\\
41	0\\
42	0\\
43	0\\
44	0\\
45	0\\
46	0\\
47	0\\
48	0\\
49	0\\
50	0\\
51	0\\
52	0\\
53	0\\
54	0\\
55	0\\
56	0\\
57	0\\
58	0\\
59	0\\
60	0\\
61	0\\
62	0\\
63	0\\
64	0\\
65	0\\
66	0\\
67	0\\
68	0\\
69	0\\
70	0\\
71	0\\
72	0\\
73	0\\
74	0\\
75	0\\
76	0\\
77	0\\
78	0\\
79	0\\
80	0\\
81	0\\
82	0\\
83	0\\
84	0\\
85	0\\
86	0\\
87	0\\
88	0\\
89	0\\
90	0\\
91	0\\
92	0\\
93	0\\
94	0\\
95	0\\
96	0\\
97	0\\
98	0\\
99	0\\
100	0\\
101	0\\
102	0\\
103	0\\
104	0\\
105	0\\
106	0\\
107	0\\
108	0\\
109	0\\
110	0\\
111	0\\
112	0\\
113	0\\
114	0\\
115	0\\
116	0\\
117	0\\
118	0\\
119	0\\
120	0\\
121	0\\
122	0\\
123	0\\
124	0\\
125	0\\
126	0\\
127	0\\
128	0\\
129	0\\
130	0\\
131	0\\
132	0\\
133	0\\
134	0\\
135	0\\
136	0\\
137	0\\
138	0\\
139	0\\
140	0\\
141	0\\
142	0\\
143	0\\
144	0\\
145	0\\
146	0\\
147	0\\
148	0\\
149	0\\
150	0\\
151	0\\
152	0\\
153	0\\
154	0\\
155	0\\
156	0\\
157	0\\
158	0\\
159	0\\
160	0\\
161	0\\
162	0\\
163	0\\
164	0\\
165	0\\
166	0\\
167	0\\
168	0\\
169	0\\
170	0\\
171	0\\
172	0\\
173	0\\
174	0\\
175	0\\
176	0\\
177	0\\
178	0\\
179	0\\
180	0\\
181	0\\
182	0\\
183	0\\
184	0\\
185	0\\
186	0\\
187	0\\
188	0\\
189	0\\
190	0\\
191	0\\
192	0\\
193	0\\
194	0\\
195	0\\
196	0\\
197	0\\
198	0\\
199	0\\
200	0\\
201	0\\
202	0\\
203	0\\
204	0\\
205	0\\
206	0\\
207	0\\
208	0\\
209	0\\
210	0\\
211	0\\
212	0\\
213	0\\
214	0\\
215	0\\
216	0\\
217	0\\
218	0\\
219	0\\
220	0\\
221	0\\
222	0\\
223	0\\
224	0\\
225	0\\
226	0\\
227	0\\
228	0\\
229	0\\
230	0\\
231	0\\
232	0\\
233	0\\
234	0\\
235	0\\
236	0\\
237	0\\
238	0\\
239	0\\
240	0\\
241	0\\
242	0\\
243	0\\
244	0\\
245	0\\
246	0\\
247	0\\
248	0\\
249	0\\
250	0\\
251	0\\
252	0\\
253	0\\
254	0\\
255	0\\
256	0\\
257	0\\
258	0\\
259	0\\
260	0\\
261	0\\
262	0\\
263	0\\
264	0\\
265	0\\
266	0\\
267	0\\
268	0\\
269	0\\
270	0\\
271	0\\
272	0\\
273	0\\
274	0\\
275	0\\
276	0\\
277	0\\
278	0\\
279	0\\
280	0\\
281	0\\
282	0\\
283	0\\
284	0\\
285	0\\
286	0\\
287	0\\
288	0\\
289	0\\
290	0\\
291	0\\
292	0\\
293	0\\
294	0\\
295	0\\
296	0\\
297	0\\
298	0\\
299	0\\
300	0\\
301	0\\
302	0\\
303	0\\
304	0\\
305	0\\
306	0\\
307	0\\
308	0\\
309	0\\
310	0\\
311	0\\
312	0\\
313	0\\
314	0\\
315	0\\
316	0\\
317	0\\
318	0\\
319	0\\
320	0\\
321	0\\
322	0\\
323	0\\
324	0\\
325	0\\
326	0\\
327	0\\
328	0\\
329	0\\
330	0\\
331	0\\
332	0\\
333	0\\
334	0\\
335	0\\
336	0\\
337	0\\
338	0\\
339	0\\
340	0\\
341	0\\
342	0\\
343	0\\
344	0\\
345	0\\
346	0\\
347	0\\
348	0\\
349	0\\
350	0\\
351	0\\
352	0\\
353	0\\
354	0\\
355	0\\
356	0\\
357	0\\
358	0\\
359	0\\
360	0\\
361	0\\
362	0\\
363	0\\
364	0\\
365	0\\
366	0\\
367	0\\
368	0\\
369	0\\
370	0\\
371	0\\
372	0\\
373	0\\
374	0\\
375	0\\
376	0\\
377	0\\
378	0\\
379	0\\
380	0\\
381	0\\
382	0\\
383	0\\
384	0\\
385	0\\
386	0\\
387	0\\
388	0\\
389	0\\
390	0\\
391	0\\
392	0\\
393	0\\
394	0\\
395	0\\
396	0\\
397	0\\
398	0\\
399	0\\
400	0\\
401	0\\
402	0\\
403	0\\
404	0\\
405	0\\
406	0\\
407	0\\
408	0\\
409	0\\
410	0\\
411	0\\
412	0\\
413	0\\
414	0\\
415	0\\
416	0\\
417	0\\
418	0\\
419	0\\
420	0\\
421	0\\
422	0\\
423	0\\
424	0\\
425	0\\
426	0\\
427	0\\
428	0\\
429	0\\
430	0\\
431	0\\
432	0\\
433	0\\
434	0\\
435	0\\
436	0\\
437	0\\
438	0\\
439	0\\
440	0\\
441	0\\
442	0\\
443	0\\
444	0\\
445	0\\
446	0\\
447	0\\
448	0\\
449	0\\
450	0\\
451	0\\
452	0\\
453	0\\
454	0\\
455	0\\
456	0\\
457	0\\
458	0\\
459	0\\
460	0\\
461	0\\
462	0\\
463	0\\
464	0\\
465	0\\
466	0\\
467	0\\
468	0\\
469	0\\
470	0\\
471	0\\
472	0\\
473	0\\
474	0\\
475	0\\
476	0\\
477	0\\
478	0\\
479	0\\
480	0\\
481	0\\
482	0\\
483	0\\
484	0\\
485	0\\
486	0\\
487	0\\
488	0\\
489	0\\
490	0\\
491	0\\
492	0\\
493	0\\
494	0\\
495	0\\
496	0\\
497	0\\
498	0\\
499	0\\
500	0\\
501	0\\
502	0\\
503	0\\
504	0\\
505	0\\
506	0\\
507	0\\
508	0\\
509	0\\
510	0\\
511	0\\
512	0\\
513	0\\
514	0\\
515	0\\
516	0\\
517	0\\
518	0\\
519	0\\
520	0\\
521	0\\
522	0\\
523	0\\
524	0\\
525	0\\
526	0\\
527	0\\
528	0\\
529	0\\
530	0\\
531	0\\
532	0\\
533	0\\
534	0\\
535	0\\
536	3.78359368673223e-07\\
537	2.51996931176172e-05\\
538	5.05127313110625e-05\\
539	7.6341234946441e-05\\
540	0.000102729583218711\\
541	0.000129693335033606\\
542	0.000157267192673928\\
543	0.000185470106681217\\
544	0.000214316398288709\\
545	0.000243821524482409\\
546	0.000274119909113173\\
547	0.000305191906140328\\
548	0.00033677287195165\\
549	0.000368831234562262\\
550	0.000401484922657279\\
551	0.000434681869006069\\
552	0.00046864575352553\\
553	0.000503409758869739\\
554	0.000539018952998663\\
555	0.000575501755643469\\
556	0.000613762678061962\\
557	0.000652925523529455\\
558	0.000691742840127115\\
559	0.000730684352716771\\
560	0.000769821722848036\\
561	0.000809880072409532\\
562	0.000850959223737115\\
563	0.000893107882836268\\
564	0.00093637240331507\\
565	0.00109010374729446\\
566	0.00145489706217203\\
567	0.0017205761280254\\
568	0.00180709087821166\\
569	0.00187129168921907\\
570	0.0019364300875453\\
571	0.00200267818666582\\
572	0.00207007397726512\\
573	0.00213865068083015\\
574	0.00220844288323674\\
575	0.00227948705593812\\
576	0.00235182173622467\\
577	0.00242548769027979\\
578	0.00250052808776776\\
579	0.00257698868705665\\
580	0.00265491803762909\\
581	0.00273436770515037\\
582	0.00281539253204859\\
583	0.00289805096554754\\
584	0.00298240553652045\\
585	0.00306852370943815\\
586	0.00315647969018813\\
587	0.00324635875816935\\
588	0.0033382683090296\\
589	0.00343236680134408\\
590	0.00352894054086423\\
591	0.00362860835901881\\
592	0.00373286830485375\\
593	0.00384555904940006\\
594	0.00397672044113205\\
595	0.00415280575412962\\
596	0.00444436306303141\\
597	0.00503983077166121\\
598	0.00644286460810295\\
599	0\\
600	0\\
};
\addplot [color=black!60!mycolor21,solid,forget plot]
  table[row sep=crcr]{%
1	0\\
2	0\\
3	0\\
4	0\\
5	0\\
6	0\\
7	0\\
8	0\\
9	0\\
10	0\\
11	0\\
12	0\\
13	0\\
14	0\\
15	0\\
16	0\\
17	0\\
18	0\\
19	0\\
20	0\\
21	0\\
22	0\\
23	0\\
24	0\\
25	0\\
26	0\\
27	0\\
28	0\\
29	0\\
30	0\\
31	0\\
32	0\\
33	0\\
34	0\\
35	0\\
36	0\\
37	0\\
38	0\\
39	0\\
40	0\\
41	0\\
42	0\\
43	0\\
44	0\\
45	0\\
46	0\\
47	0\\
48	0\\
49	0\\
50	0\\
51	0\\
52	0\\
53	0\\
54	0\\
55	0\\
56	0\\
57	0\\
58	0\\
59	0\\
60	0\\
61	0\\
62	0\\
63	0\\
64	0\\
65	0\\
66	0\\
67	0\\
68	0\\
69	0\\
70	0\\
71	0\\
72	0\\
73	0\\
74	0\\
75	0\\
76	0\\
77	0\\
78	0\\
79	0\\
80	0\\
81	0\\
82	0\\
83	0\\
84	0\\
85	0\\
86	0\\
87	0\\
88	0\\
89	0\\
90	0\\
91	0\\
92	0\\
93	0\\
94	0\\
95	0\\
96	0\\
97	0\\
98	0\\
99	0\\
100	0\\
101	0\\
102	0\\
103	0\\
104	0\\
105	0\\
106	0\\
107	0\\
108	0\\
109	0\\
110	0\\
111	0\\
112	0\\
113	0\\
114	0\\
115	0\\
116	0\\
117	0\\
118	0\\
119	0\\
120	0\\
121	0\\
122	0\\
123	0\\
124	0\\
125	0\\
126	0\\
127	0\\
128	0\\
129	0\\
130	0\\
131	0\\
132	0\\
133	0\\
134	0\\
135	0\\
136	0\\
137	0\\
138	0\\
139	0\\
140	0\\
141	0\\
142	0\\
143	0\\
144	0\\
145	0\\
146	0\\
147	0\\
148	0\\
149	0\\
150	0\\
151	0\\
152	0\\
153	0\\
154	0\\
155	0\\
156	0\\
157	0\\
158	0\\
159	0\\
160	0\\
161	0\\
162	0\\
163	0\\
164	0\\
165	0\\
166	0\\
167	0\\
168	0\\
169	0\\
170	0\\
171	0\\
172	0\\
173	0\\
174	0\\
175	0\\
176	0\\
177	0\\
178	0\\
179	0\\
180	0\\
181	0\\
182	0\\
183	0\\
184	0\\
185	0\\
186	0\\
187	0\\
188	0\\
189	0\\
190	0\\
191	0\\
192	0\\
193	0\\
194	0\\
195	0\\
196	0\\
197	0\\
198	0\\
199	0\\
200	0\\
201	0\\
202	0\\
203	0\\
204	0\\
205	0\\
206	0\\
207	0\\
208	0\\
209	0\\
210	0\\
211	0\\
212	0\\
213	0\\
214	0\\
215	0\\
216	0\\
217	0\\
218	0\\
219	0\\
220	0\\
221	0\\
222	0\\
223	0\\
224	0\\
225	0\\
226	0\\
227	0\\
228	0\\
229	0\\
230	0\\
231	0\\
232	0\\
233	0\\
234	0\\
235	0\\
236	0\\
237	0\\
238	0\\
239	0\\
240	0\\
241	0\\
242	0\\
243	0\\
244	0\\
245	0\\
246	0\\
247	0\\
248	0\\
249	0\\
250	0\\
251	0\\
252	0\\
253	0\\
254	0\\
255	0\\
256	0\\
257	0\\
258	0\\
259	0\\
260	0\\
261	0\\
262	0\\
263	0\\
264	0\\
265	0\\
266	0\\
267	0\\
268	0\\
269	0\\
270	0\\
271	0\\
272	0\\
273	0\\
274	0\\
275	0\\
276	0\\
277	0\\
278	0\\
279	0\\
280	0\\
281	0\\
282	0\\
283	0\\
284	0\\
285	0\\
286	0\\
287	0\\
288	0\\
289	0\\
290	0\\
291	0\\
292	0\\
293	0\\
294	0\\
295	0\\
296	0\\
297	0\\
298	0\\
299	0\\
300	0\\
301	0\\
302	0\\
303	0\\
304	0\\
305	0\\
306	0\\
307	0\\
308	0\\
309	0\\
310	0\\
311	0\\
312	0\\
313	0\\
314	0\\
315	0\\
316	0\\
317	0\\
318	0\\
319	0\\
320	0\\
321	0\\
322	0\\
323	0\\
324	0\\
325	0\\
326	0\\
327	0\\
328	0\\
329	0\\
330	0\\
331	0\\
332	0\\
333	0\\
334	0\\
335	0\\
336	0\\
337	0\\
338	0\\
339	0\\
340	0\\
341	0\\
342	0\\
343	0\\
344	0\\
345	0\\
346	0\\
347	0\\
348	0\\
349	0\\
350	0\\
351	0\\
352	0\\
353	0\\
354	0\\
355	0\\
356	0\\
357	0\\
358	0\\
359	0\\
360	0\\
361	0\\
362	0\\
363	0\\
364	0\\
365	0\\
366	0\\
367	0\\
368	0\\
369	0\\
370	0\\
371	0\\
372	0\\
373	0\\
374	0\\
375	0\\
376	0\\
377	0\\
378	0\\
379	0\\
380	0\\
381	0\\
382	0\\
383	0\\
384	0\\
385	0\\
386	0\\
387	0\\
388	0\\
389	0\\
390	0\\
391	0\\
392	0\\
393	0\\
394	0\\
395	0\\
396	0\\
397	0\\
398	0\\
399	0\\
400	0\\
401	0\\
402	0\\
403	0\\
404	0\\
405	0\\
406	0\\
407	0\\
408	0\\
409	0\\
410	0\\
411	0\\
412	0\\
413	0\\
414	0\\
415	0\\
416	0\\
417	0\\
418	0\\
419	0\\
420	0\\
421	0\\
422	0\\
423	0\\
424	0\\
425	0\\
426	0\\
427	0\\
428	0\\
429	0\\
430	0\\
431	0\\
432	0\\
433	0\\
434	0\\
435	0\\
436	0\\
437	0\\
438	0\\
439	0\\
440	0\\
441	0\\
442	0\\
443	0\\
444	0\\
445	0\\
446	0\\
447	0\\
448	0\\
449	0\\
450	0\\
451	0\\
452	0\\
453	0\\
454	0\\
455	0\\
456	0\\
457	0\\
458	0\\
459	0\\
460	0\\
461	0\\
462	0\\
463	0\\
464	0\\
465	0\\
466	0\\
467	0\\
468	0\\
469	0\\
470	0\\
471	0\\
472	0\\
473	0\\
474	0\\
475	0\\
476	0\\
477	0\\
478	0\\
479	0\\
480	0\\
481	0\\
482	0\\
483	0\\
484	0\\
485	0\\
486	0\\
487	0\\
488	0\\
489	0\\
490	0\\
491	0\\
492	0\\
493	0\\
494	0\\
495	0\\
496	0\\
497	0\\
498	0\\
499	0\\
500	0\\
501	0\\
502	0\\
503	0\\
504	0\\
505	0\\
506	0\\
507	0\\
508	0\\
509	0\\
510	0\\
511	0\\
512	0\\
513	0\\
514	0\\
515	0\\
516	0\\
517	0\\
518	0\\
519	0\\
520	0\\
521	0\\
522	0\\
523	0\\
524	0\\
525	0\\
526	0\\
527	0\\
528	0\\
529	0\\
530	0\\
531	0\\
532	0\\
533	0\\
534	0\\
535	0\\
536	9.02145954336361e-06\\
537	3.38870820114369e-05\\
538	5.92843281107733e-05\\
539	8.52316144653158e-05\\
540	0.000111744299307854\\
541	0.000138841481860631\\
542	0.000166552064353744\\
543	0.00019489091642814\\
544	0.0002240061624105\\
545	0.000253856225130063\\
546	0.000284149213149703\\
547	0.000314894675820076\\
548	0.000346177518049505\\
549	0.000378018067102527\\
550	0.000410577800625992\\
551	0.000443915441746857\\
552	0.00047805631403732\\
553	0.000513026094184202\\
554	0.00054907441425197\\
555	0.000587044911282405\\
556	0.000624783568171093\\
557	0.000662504517028127\\
558	0.000700022606374216\\
559	0.000738375853450751\\
560	0.000777685135972991\\
561	0.000817995895139835\\
562	0.000859350044913817\\
563	0.000901791512692584\\
564	0.00109093692626027\\
565	0.00144213576563695\\
566	0.00167960457611589\\
567	0.00174425603612933\\
568	0.00180725733542332\\
569	0.00187130073891484\\
570	0.00193643097966661\\
571	0.00200267837569042\\
572	0.00207007403928857\\
573	0.00213865070333988\\
574	0.00220844289215589\\
575	0.00227948706023545\\
576	0.00235182173838593\\
577	0.00242548769141891\\
578	0.00250052808837546\\
579	0.00257698868737254\\
580	0.00265491803777995\\
581	0.00273436770521399\\
582	0.00281539253206718\\
583	0.00289805096555054\\
584	0.00298240553652045\\
585	0.00306852370943814\\
586	0.00315647969018814\\
587	0.00324635875816936\\
588	0.0033382683090296\\
589	0.00343236680134407\\
590	0.00352894054086422\\
591	0.00362860835901882\\
592	0.00373286830485375\\
593	0.00384555904940006\\
594	0.00397672044113206\\
595	0.00415280575412962\\
596	0.00444436306303141\\
597	0.00503983077166121\\
598	0.00644286460810295\\
599	0\\
600	0\\
};
\addplot [color=black!80!mycolor21,solid,forget plot]
  table[row sep=crcr]{%
1	0\\
2	0\\
3	0\\
4	0\\
5	0\\
6	0\\
7	0\\
8	0\\
9	0\\
10	0\\
11	0\\
12	0\\
13	0\\
14	0\\
15	0\\
16	0\\
17	0\\
18	0\\
19	0\\
20	0\\
21	0\\
22	0\\
23	0\\
24	0\\
25	0\\
26	0\\
27	0\\
28	0\\
29	0\\
30	0\\
31	0\\
32	0\\
33	0\\
34	0\\
35	0\\
36	0\\
37	0\\
38	0\\
39	0\\
40	0\\
41	0\\
42	0\\
43	0\\
44	0\\
45	0\\
46	0\\
47	0\\
48	0\\
49	0\\
50	0\\
51	0\\
52	0\\
53	0\\
54	0\\
55	0\\
56	0\\
57	0\\
58	0\\
59	0\\
60	0\\
61	0\\
62	0\\
63	0\\
64	0\\
65	0\\
66	0\\
67	0\\
68	0\\
69	0\\
70	0\\
71	0\\
72	0\\
73	0\\
74	0\\
75	0\\
76	0\\
77	0\\
78	0\\
79	0\\
80	0\\
81	0\\
82	0\\
83	0\\
84	0\\
85	0\\
86	0\\
87	0\\
88	0\\
89	0\\
90	0\\
91	0\\
92	0\\
93	0\\
94	0\\
95	0\\
96	0\\
97	0\\
98	0\\
99	0\\
100	0\\
101	0\\
102	0\\
103	0\\
104	0\\
105	0\\
106	0\\
107	0\\
108	0\\
109	0\\
110	0\\
111	0\\
112	0\\
113	0\\
114	0\\
115	0\\
116	0\\
117	0\\
118	0\\
119	0\\
120	0\\
121	0\\
122	0\\
123	0\\
124	0\\
125	0\\
126	0\\
127	0\\
128	0\\
129	0\\
130	0\\
131	0\\
132	0\\
133	0\\
134	0\\
135	0\\
136	0\\
137	0\\
138	0\\
139	0\\
140	0\\
141	0\\
142	0\\
143	0\\
144	0\\
145	0\\
146	0\\
147	0\\
148	0\\
149	0\\
150	0\\
151	0\\
152	0\\
153	0\\
154	0\\
155	0\\
156	0\\
157	0\\
158	0\\
159	0\\
160	0\\
161	0\\
162	0\\
163	0\\
164	0\\
165	0\\
166	0\\
167	0\\
168	0\\
169	0\\
170	0\\
171	0\\
172	0\\
173	0\\
174	0\\
175	0\\
176	0\\
177	0\\
178	0\\
179	0\\
180	0\\
181	0\\
182	0\\
183	0\\
184	0\\
185	0\\
186	0\\
187	0\\
188	0\\
189	0\\
190	0\\
191	0\\
192	0\\
193	0\\
194	0\\
195	0\\
196	0\\
197	0\\
198	0\\
199	0\\
200	0\\
201	0\\
202	0\\
203	0\\
204	0\\
205	0\\
206	0\\
207	0\\
208	0\\
209	0\\
210	0\\
211	0\\
212	0\\
213	0\\
214	0\\
215	0\\
216	0\\
217	0\\
218	0\\
219	0\\
220	0\\
221	0\\
222	0\\
223	0\\
224	0\\
225	0\\
226	0\\
227	0\\
228	0\\
229	0\\
230	0\\
231	0\\
232	0\\
233	0\\
234	0\\
235	0\\
236	0\\
237	0\\
238	0\\
239	0\\
240	0\\
241	0\\
242	0\\
243	0\\
244	0\\
245	0\\
246	0\\
247	0\\
248	0\\
249	0\\
250	0\\
251	0\\
252	0\\
253	0\\
254	0\\
255	0\\
256	0\\
257	0\\
258	0\\
259	0\\
260	0\\
261	0\\
262	0\\
263	0\\
264	0\\
265	0\\
266	0\\
267	0\\
268	0\\
269	0\\
270	0\\
271	0\\
272	0\\
273	0\\
274	0\\
275	0\\
276	0\\
277	0\\
278	0\\
279	0\\
280	0\\
281	0\\
282	0\\
283	0\\
284	0\\
285	0\\
286	0\\
287	0\\
288	0\\
289	0\\
290	0\\
291	0\\
292	0\\
293	0\\
294	0\\
295	0\\
296	0\\
297	0\\
298	0\\
299	0\\
300	0\\
301	0\\
302	0\\
303	0\\
304	0\\
305	0\\
306	0\\
307	0\\
308	0\\
309	0\\
310	0\\
311	0\\
312	0\\
313	0\\
314	0\\
315	0\\
316	0\\
317	0\\
318	0\\
319	0\\
320	0\\
321	0\\
322	0\\
323	0\\
324	0\\
325	0\\
326	0\\
327	0\\
328	0\\
329	0\\
330	0\\
331	0\\
332	0\\
333	0\\
334	0\\
335	0\\
336	0\\
337	0\\
338	0\\
339	0\\
340	0\\
341	0\\
342	0\\
343	0\\
344	0\\
345	0\\
346	0\\
347	0\\
348	0\\
349	0\\
350	0\\
351	0\\
352	0\\
353	0\\
354	0\\
355	0\\
356	0\\
357	0\\
358	0\\
359	0\\
360	0\\
361	0\\
362	0\\
363	0\\
364	0\\
365	0\\
366	0\\
367	0\\
368	0\\
369	0\\
370	0\\
371	0\\
372	0\\
373	0\\
374	0\\
375	0\\
376	0\\
377	0\\
378	0\\
379	0\\
380	0\\
381	0\\
382	0\\
383	0\\
384	0\\
385	0\\
386	0\\
387	0\\
388	0\\
389	0\\
390	0\\
391	0\\
392	0\\
393	0\\
394	0\\
395	0\\
396	0\\
397	0\\
398	0\\
399	0\\
400	0\\
401	0\\
402	0\\
403	0\\
404	0\\
405	0\\
406	0\\
407	0\\
408	0\\
409	0\\
410	0\\
411	0\\
412	0\\
413	0\\
414	0\\
415	0\\
416	0\\
417	0\\
418	0\\
419	0\\
420	0\\
421	0\\
422	0\\
423	0\\
424	0\\
425	0\\
426	0\\
427	0\\
428	0\\
429	0\\
430	0\\
431	0\\
432	0\\
433	0\\
434	0\\
435	0\\
436	0\\
437	0\\
438	0\\
439	0\\
440	0\\
441	0\\
442	0\\
443	0\\
444	0\\
445	0\\
446	0\\
447	0\\
448	0\\
449	0\\
450	0\\
451	0\\
452	0\\
453	0\\
454	0\\
455	0\\
456	0\\
457	0\\
458	0\\
459	0\\
460	0\\
461	0\\
462	0\\
463	0\\
464	0\\
465	0\\
466	0\\
467	0\\
468	0\\
469	0\\
470	0\\
471	0\\
472	0\\
473	0\\
474	0\\
475	0\\
476	0\\
477	0\\
478	0\\
479	0\\
480	0\\
481	0\\
482	0\\
483	0\\
484	0\\
485	0\\
486	0\\
487	0\\
488	0\\
489	0\\
490	0\\
491	0\\
492	0\\
493	0\\
494	0\\
495	0\\
496	0\\
497	0\\
498	0\\
499	0\\
500	0\\
501	0\\
502	0\\
503	0\\
504	0\\
505	0\\
506	0\\
507	0\\
508	0\\
509	0\\
510	0\\
511	0\\
512	0\\
513	0\\
514	0\\
515	0\\
516	0\\
517	0\\
518	0\\
519	0\\
520	0\\
521	0\\
522	0\\
523	0\\
524	0\\
525	0\\
526	0\\
527	0\\
528	0\\
529	0\\
530	0\\
531	0\\
532	0\\
533	0\\
534	0\\
535	0\\
536	1.67358631738589e-05\\
537	4.17113787144746e-05\\
538	6.72225789001668e-05\\
539	9.32840951840519e-05\\
540	0.000119910708273509\\
541	0.000147126549934032\\
542	0.000175083363931519\\
543	0.000203765392134202\\
544	0.000232847505447229\\
545	0.000262347629977608\\
546	0.000292337815156544\\
547	0.000322869907415254\\
548	0.000354099010332315\\
549	0.000386067917324203\\
550	0.000418801906923749\\
551	0.000452323301093407\\
552	0.000486654809378515\\
553	0.000522613247243088\\
554	0.000559557726971212\\
555	0.000596060905993073\\
556	0.000632410263998094\\
557	0.000669141593433294\\
558	0.000706767938764675\\
559	0.000745333893433888\\
560	0.000784877380144605\\
561	0.000825438279498713\\
562	0.000867058997353549\\
563	0.00106897995898566\\
564	0.00141439733785219\\
565	0.00162107369184165\\
566	0.00168232368728045\\
567	0.00174427557656458\\
568	0.00180725842153865\\
569	0.00187130085080805\\
570	0.00193643100447991\\
571	0.00200267838403535\\
572	0.00207007404240156\\
573	0.00213865070462208\\
574	0.00220844289278456\\
575	0.00227948706055647\\
576	0.00235182173855629\\
577	0.00242548769150975\\
578	0.00250052808842228\\
579	0.00257698868739462\\
580	0.00265491803778901\\
581	0.00273436770521657\\
582	0.00281539253206758\\
583	0.00289805096555055\\
584	0.00298240553652045\\
585	0.00306852370943816\\
586	0.00315647969018813\\
587	0.00324635875816936\\
588	0.0033382683090296\\
589	0.00343236680134409\\
590	0.00352894054086423\\
591	0.00362860835901882\\
592	0.00373286830485376\\
593	0.00384555904940007\\
594	0.00397672044113206\\
595	0.00415280575412962\\
596	0.00444436306303141\\
597	0.00503983077166122\\
598	0.00644286460810295\\
599	0\\
600	0\\
};
\addplot [color=black,solid,forget plot]
  table[row sep=crcr]{%
1	0\\
2	0\\
3	0\\
4	0\\
5	0\\
6	0\\
7	0\\
8	0\\
9	0\\
10	0\\
11	0\\
12	0\\
13	0\\
14	0\\
15	0\\
16	0\\
17	0\\
18	0\\
19	0\\
20	0\\
21	0\\
22	0\\
23	0\\
24	0\\
25	0\\
26	0\\
27	0\\
28	0\\
29	0\\
30	0\\
31	0\\
32	0\\
33	0\\
34	0\\
35	0\\
36	0\\
37	0\\
38	0\\
39	0\\
40	0\\
41	0\\
42	0\\
43	0\\
44	0\\
45	0\\
46	0\\
47	0\\
48	0\\
49	0\\
50	0\\
51	0\\
52	0\\
53	0\\
54	0\\
55	0\\
56	0\\
57	0\\
58	0\\
59	0\\
60	0\\
61	0\\
62	0\\
63	0\\
64	0\\
65	0\\
66	0\\
67	0\\
68	0\\
69	0\\
70	0\\
71	0\\
72	0\\
73	0\\
74	0\\
75	0\\
76	0\\
77	0\\
78	0\\
79	0\\
80	0\\
81	0\\
82	0\\
83	0\\
84	0\\
85	0\\
86	0\\
87	0\\
88	0\\
89	0\\
90	0\\
91	0\\
92	0\\
93	0\\
94	0\\
95	0\\
96	0\\
97	0\\
98	0\\
99	0\\
100	0\\
101	0\\
102	0\\
103	0\\
104	0\\
105	0\\
106	0\\
107	0\\
108	0\\
109	0\\
110	0\\
111	0\\
112	0\\
113	0\\
114	0\\
115	0\\
116	0\\
117	0\\
118	0\\
119	0\\
120	0\\
121	0\\
122	0\\
123	0\\
124	0\\
125	0\\
126	0\\
127	0\\
128	0\\
129	0\\
130	0\\
131	0\\
132	0\\
133	0\\
134	0\\
135	0\\
136	0\\
137	0\\
138	0\\
139	0\\
140	0\\
141	0\\
142	0\\
143	0\\
144	0\\
145	0\\
146	0\\
147	0\\
148	0\\
149	0\\
150	0\\
151	0\\
152	0\\
153	0\\
154	0\\
155	0\\
156	0\\
157	0\\
158	0\\
159	0\\
160	0\\
161	0\\
162	0\\
163	0\\
164	0\\
165	0\\
166	0\\
167	0\\
168	0\\
169	0\\
170	0\\
171	0\\
172	0\\
173	0\\
174	0\\
175	0\\
176	0\\
177	0\\
178	0\\
179	0\\
180	0\\
181	0\\
182	0\\
183	0\\
184	0\\
185	0\\
186	0\\
187	0\\
188	0\\
189	0\\
190	0\\
191	0\\
192	0\\
193	0\\
194	0\\
195	0\\
196	0\\
197	0\\
198	0\\
199	0\\
200	0\\
201	0\\
202	0\\
203	0\\
204	0\\
205	0\\
206	0\\
207	0\\
208	0\\
209	0\\
210	0\\
211	0\\
212	0\\
213	0\\
214	0\\
215	0\\
216	0\\
217	0\\
218	0\\
219	0\\
220	0\\
221	0\\
222	0\\
223	0\\
224	0\\
225	0\\
226	0\\
227	0\\
228	0\\
229	0\\
230	0\\
231	0\\
232	0\\
233	0\\
234	0\\
235	0\\
236	0\\
237	0\\
238	0\\
239	0\\
240	0\\
241	0\\
242	0\\
243	0\\
244	0\\
245	0\\
246	0\\
247	0\\
248	0\\
249	0\\
250	0\\
251	0\\
252	0\\
253	0\\
254	0\\
255	0\\
256	0\\
257	0\\
258	0\\
259	0\\
260	0\\
261	0\\
262	0\\
263	0\\
264	0\\
265	0\\
266	0\\
267	0\\
268	0\\
269	0\\
270	0\\
271	0\\
272	0\\
273	0\\
274	0\\
275	0\\
276	0\\
277	0\\
278	0\\
279	0\\
280	0\\
281	0\\
282	0\\
283	0\\
284	0\\
285	0\\
286	0\\
287	0\\
288	0\\
289	0\\
290	0\\
291	0\\
292	0\\
293	0\\
294	0\\
295	0\\
296	0\\
297	0\\
298	0\\
299	0\\
300	0\\
301	0\\
302	0\\
303	0\\
304	0\\
305	0\\
306	0\\
307	0\\
308	0\\
309	0\\
310	0\\
311	0\\
312	0\\
313	0\\
314	0\\
315	0\\
316	0\\
317	0\\
318	0\\
319	0\\
320	0\\
321	0\\
322	0\\
323	0\\
324	0\\
325	0\\
326	0\\
327	0\\
328	0\\
329	0\\
330	0\\
331	0\\
332	0\\
333	0\\
334	0\\
335	0\\
336	0\\
337	0\\
338	0\\
339	0\\
340	0\\
341	0\\
342	0\\
343	0\\
344	0\\
345	0\\
346	0\\
347	0\\
348	0\\
349	0\\
350	0\\
351	0\\
352	0\\
353	0\\
354	0\\
355	0\\
356	0\\
357	0\\
358	0\\
359	0\\
360	0\\
361	0\\
362	0\\
363	0\\
364	0\\
365	0\\
366	0\\
367	0\\
368	0\\
369	0\\
370	0\\
371	0\\
372	0\\
373	0\\
374	0\\
375	0\\
376	0\\
377	0\\
378	0\\
379	0\\
380	0\\
381	0\\
382	0\\
383	0\\
384	0\\
385	0\\
386	0\\
387	0\\
388	0\\
389	0\\
390	0\\
391	0\\
392	0\\
393	0\\
394	0\\
395	0\\
396	0\\
397	0\\
398	0\\
399	0\\
400	0\\
401	0\\
402	0\\
403	0\\
404	0\\
405	0\\
406	0\\
407	0\\
408	0\\
409	0\\
410	0\\
411	0\\
412	0\\
413	0\\
414	0\\
415	0\\
416	0\\
417	0\\
418	0\\
419	0\\
420	0\\
421	0\\
422	0\\
423	0\\
424	0\\
425	0\\
426	0\\
427	0\\
428	0\\
429	0\\
430	0\\
431	0\\
432	0\\
433	0\\
434	0\\
435	0\\
436	0\\
437	0\\
438	0\\
439	0\\
440	0\\
441	0\\
442	0\\
443	0\\
444	0\\
445	0\\
446	0\\
447	0\\
448	0\\
449	0\\
450	0\\
451	0\\
452	0\\
453	0\\
454	0\\
455	0\\
456	0\\
457	0\\
458	0\\
459	0\\
460	0\\
461	0\\
462	0\\
463	0\\
464	0\\
465	0\\
466	0\\
467	0\\
468	0\\
469	0\\
470	0\\
471	0\\
472	0\\
473	0\\
474	0\\
475	0\\
476	0\\
477	0\\
478	0\\
479	0\\
480	0\\
481	0\\
482	0\\
483	0\\
484	0\\
485	0\\
486	0\\
487	0\\
488	0\\
489	0\\
490	0\\
491	0\\
492	0\\
493	0\\
494	0\\
495	0\\
496	0\\
497	0\\
498	0\\
499	0\\
500	0\\
501	0\\
502	0\\
503	0\\
504	0\\
505	0\\
506	0\\
507	0\\
508	0\\
509	0\\
510	0\\
511	0\\
512	0\\
513	0\\
514	0\\
515	0\\
516	0\\
517	0\\
518	0\\
519	0\\
520	0\\
521	0\\
522	0\\
523	0\\
524	0\\
525	0\\
526	0\\
527	0\\
528	0\\
529	0\\
530	0\\
531	0\\
532	0\\
533	0\\
534	0\\
535	0\\
536	2.3273899000436e-05\\
537	4.83574677493769e-05\\
538	7.39764419503986e-05\\
539	0.000100145056725748\\
540	0.000126927776089947\\
541	0.000154551208357287\\
542	0.000182537594263675\\
543	0.000210877875154721\\
544	0.000239654581585972\\
545	0.000268942646031741\\
546	0.000298901101391198\\
547	0.000329565590919966\\
548	0.000360958489926797\\
549	0.000393100761277305\\
550	0.0004260137847175\\
551	0.000459720061076944\\
552	0.000495493917640628\\
553	0.000531330574309899\\
554	0.000566818867744161\\
555	0.000602062179114877\\
556	0.000638100502800796\\
557	0.000675022523672534\\
558	0.000712862997966947\\
559	0.00075165794188291\\
560	0.000791445930933432\\
561	0.000832268907521252\\
562	0.00101617204181345\\
563	0.00137303696588922\\
564	0.0015613898810912\\
565	0.00162138355504735\\
566	0.00168232596368277\\
567	0.0017442757060301\\
568	0.00180725843549943\\
569	0.00187130085404748\\
570	0.00193643100559733\\
571	0.00200267838446417\\
572	0.00207007404258462\\
573	0.0021386507047133\\
574	0.00220844289283176\\
575	0.00227948706058165\\
576	0.00235182173856971\\
577	0.0024254876915166\\
578	0.00250052808842547\\
579	0.00257698868739587\\
580	0.00265491803778937\\
581	0.00273436770521664\\
582	0.00281539253206758\\
583	0.00289805096555053\\
584	0.00298240553652045\\
585	0.00306852370943815\\
586	0.00315647969018813\\
587	0.00324635875816936\\
588	0.00333826830902961\\
589	0.00343236680134409\\
590	0.00352894054086423\\
591	0.00362860835901881\\
592	0.00373286830485375\\
593	0.00384555904940006\\
594	0.00397672044113206\\
595	0.00415280575412962\\
596	0.00444436306303141\\
597	0.00503983077166122\\
598	0.00644286460810295\\
599	0\\
600	0\\
};
\end{axis}
\end{tikzpicture}%
 
  \caption{Discrete Time}
\end{subfigure}\\

\leavevmode\smash{\makebox[0pt]{\hspace{-7em}% HORIZONTAL POSITION           
  \rotatebox[origin=l]{90}{\hspace{20em}% VERTICAL POSITION
    Depth $\delta^+$}%
}}\hspace{0pt plus 1filll}\null

Time (s)

\vspace{1cm}
\begin{subfigure}{\linewidth}
  \centering
  \tikzsetnextfilename{altdeltalegend}
  \definecolor{mycolor1}{rgb}{0.00000,1.00000,0.14286}%
\definecolor{mycolor2}{rgb}{0.00000,1.00000,0.28571}%
\definecolor{mycolor3}{rgb}{0.00000,1.00000,0.42857}%
\definecolor{mycolor4}{rgb}{0.00000,1.00000,0.57143}%
\definecolor{mycolor5}{rgb}{0.00000,1.00000,0.71429}%
\definecolor{mycolor6}{rgb}{0.00000,1.00000,0.85714}%
\definecolor{mycolor7}{rgb}{0.00000,1.00000,1.00000}%
\definecolor{mycolor8}{rgb}{0.00000,0.87500,1.00000}%
\definecolor{mycolor9}{rgb}{0.00000,0.62500,1.00000}%
\definecolor{mycolor10}{rgb}{0.12500,0.00000,1.00000}%
\definecolor{mycolor11}{rgb}{0.25000,0.00000,1.00000}%
\definecolor{mycolor12}{rgb}{0.37500,0.00000,1.00000}%
\definecolor{mycolor13}{rgb}{0.50000,0.00000,1.00000}%
\definecolor{mycolor14}{rgb}{0.62500,0.00000,1.00000}%
\definecolor{mycolor15}{rgb}{0.75000,0.00000,1.00000}%
\definecolor{mycolor16}{rgb}{0.87500,0.00000,1.00000}%
\definecolor{mycolor17}{rgb}{1.00000,0.00000,1.00000}%
\definecolor{mycolor18}{rgb}{1.00000,0.00000,0.87500}%
\definecolor{mycolor19}{rgb}{1.00000,0.00000,0.62500}%
\definecolor{mycolor20}{rgb}{0.85714,0.00000,0.00000}%
\definecolor{mycolor21}{rgb}{0.71429,0.00000,0.00000}%
%[trim axis left, trim axis right]
\begin{tikzpicture}
\begin{axis}[%
    hide axis,
    scale only axis,
    height=0pt,
    width=0pt,
    point meta min=-19,
    point meta max=19,
    colormap={mymap}{[1pt] rgb(0pt)=(0,1,0); rgb(7pt)=(0,1,1); rgb(15pt)=(0,0,1); rgb(23pt)=(1,0,1); rgb(31pt)=(1,0,0); rgb(38pt)=(0,0,0)},
    colorbar horizontal,
    colorbar style={width=15cm,xtick={{-15},{-10},{-5},{0},{5},{10},{15}},ylabel={Inventory Level $Q$}, y label style={at={(axis description cs:0.5,-1)},rotate=-90,anchor=north}}
    %colorbar style={separate axis lines,every outer x axis line/.append style={black},every x tick label/.append style={font=\color{black}},every outer y axis line/.append style={black},every y tick label/.append style={font=\color{black}},yticklabels={{-19},{-17},{-15},{-13},{-11},{-9},{-7},{-5},{-3},{-1},{1},{3},{5},{7},{9},{11},{13},{15},{17},{19}},ylabel={Inventory Level $Q$}}
]%
    \addplot [draw=none] coordinates {(0,0)};
\end{axis}
\end{tikzpicture}
 
\end{subfigure}%
  \caption{Optimal buy depths $\delta^{+}$ for Markov state $Z=(\rho = +1, \Delta S = +1)$, implying heavy imbalance in favor of buy pressure, and having previously seen an upward price change. We expect the midprice to rise.}
  \label{fig:comp_dp_z15_test}
\end{figure}
The first notable conclusion we can make is the symmetry that has emerged between $\delta^+$ and $\delta^-$ in `opposite' Markov states. This is evident when comparing $\delta^+$ in $Z=(-1,-1)$ (\autoref{fig:comp_dp_z1}) with $\delta^-$ in $Z=(+1,+1)$ (\autoref{fig:comp_dm_z15}), $\delta^+$ in $Z=(0,0)$ (\autoref{fig:comp_dp_z8}) with $\delta^-$ in $Z=(0,0)$ (\autoref{fig:comp_dm_z8}), and $\delta^+$ in $Z=(+1,+1)$ (\autoref{fig:comp_dp_z15}) with $\delta^-$ in $Z=(-1,-1)$ (\autoref{fig:comp_dm_z1}). Thus, we focus the discussion here on the behavior of $\delta^+$.

In this calibration we have taken $\kappa=100$ and $\xi = 0.005$. From \eqref{eq:ctsBuycondition}, we thus know that a necessary condition for a buy market order to be executed is ${\delta^+}^* =  \frac{1}{\kappa} - 2 \xi = 0$.

{\bf Markov State $Z=(-1,-1)$ (\autoref{fig:comp_dp_z1})} \\
{\bf Cts versus Cts w nFPC:} For $q \geq 0$, both strategies post aggressive bid depths that suggest an inclination to buy. The nFPC strategy is less aggressive for $q<0$, posting closer to zero depth only for larger short positions.  \\
{\bf Dscr versus Dscr w nFPC:} The behaviour of these two calibrations is very similar. Both models show a discontinuity in dynamics at $q=0$, where at $q=-1$ it is posting at maximal depth, at $q=0$ it jumps to approximately \$0.006, and at $q=1$ again jumps lower. Otherwise, the nFPC strategy posts slightly more aggressively only when $q=1$ or 2. \\
{\bf Cts vs Dscr:} These models produce behaviours that are worlds apart. Whereas the Cts model seems to be saying that it wants to take this opportunity to go long, perhaps stocking up inventory while prices are low, the Dscr strategy suggests it's pulling out and avoiding purchasing. 

{\bf Markov State $Z=(0,0)$ (\autoref{fig:comp_dp_z8})}\\
Here we see near identical model behaviour. Similarities or correlations in backtesting performance can likely be attributed to this behaviour, as the majority of the day is spent in a Markov state for which $\Delta S = 0$, as seen here. (Recall that $\rho$, by contrast, is computed via evenly spaced percentiles symmetric around zero, so that time spent in each imbalance state is evenly distributed.)

{\bf Markov State $Z=(+1,+1)$ (\autoref{fig:comp_dp_z15})}\\
{\bf Cts versus Cts w nFPC:} We see a nearly symmetric behaviour compared with the opposite Markov state. Here for $q \leq 0$, both strategies post maximal bid depths, suggesting a disinclination toward buying. The nFPC strategy is more aggressive for $q>0$ and small inventory positions.\\
{\bf Dscr versus Dscr w nFPC:} Again these calibrations yield very similar behaviours, and as in the Cts case, it is near opposite to what was seen in the opposite Markov state.\\
{\bf Cts vs Dscr:} As before, these two models display near opposite behaviours between each other. The Dscr model is posting aggressive depths near zero in an attempt to purchase, while the Cts model is posting deeper into the book to avoid purchasing.

Finally, we note that we see stability in the posting depths beyond a time horizon of 600 seconds. This is consistent with the findings in \autoref{tbl:pvalues}, where we saw that the transition probability matrix $\mat{P}(t)$ converged for $\texttt{ORCL}$ within 404 timesteps of 1s each. \fxnote{If gonna use the colorbar, change to INTC and 771 timesteps, but note error threshold $10^{-10}$}

\begin{figure}
\centering
\begin{subfigure}{.45\linewidth}
  \centering
  \setlength\figureheight{\linewidth} 
  \setlength\figurewidth{\linewidth}
  \tikzsetnextfilename{dm_cts_z1}
  % This file was created by matlab2tikz.
%
%The latest updates can be retrieved from
%  http://www.mathworks.com/matlabcentral/fileexchange/22022-matlab2tikz-matlab2tikz
%where you can also make suggestions and rate matlab2tikz.
%
\definecolor{mycolor1}{rgb}{0.00000,1.00000,0.14286}%
\definecolor{mycolor2}{rgb}{0.00000,1.00000,0.28571}%
\definecolor{mycolor3}{rgb}{0.00000,1.00000,0.42857}%
\definecolor{mycolor4}{rgb}{0.00000,1.00000,0.57143}%
\definecolor{mycolor5}{rgb}{0.00000,1.00000,0.71429}%
\definecolor{mycolor6}{rgb}{0.00000,1.00000,0.85714}%
\definecolor{mycolor7}{rgb}{0.00000,1.00000,1.00000}%
\definecolor{mycolor8}{rgb}{0.00000,0.87500,1.00000}%
\definecolor{mycolor9}{rgb}{0.00000,0.62500,1.00000}%
\definecolor{mycolor10}{rgb}{0.12500,0.00000,1.00000}%
\definecolor{mycolor11}{rgb}{0.25000,0.00000,1.00000}%
\definecolor{mycolor12}{rgb}{0.37500,0.00000,1.00000}%
\definecolor{mycolor13}{rgb}{0.50000,0.00000,1.00000}%
\definecolor{mycolor14}{rgb}{0.62500,0.00000,1.00000}%
\definecolor{mycolor15}{rgb}{0.75000,0.00000,1.00000}%
\definecolor{mycolor16}{rgb}{0.87500,0.00000,1.00000}%
\definecolor{mycolor17}{rgb}{1.00000,0.00000,1.00000}%
\definecolor{mycolor18}{rgb}{1.00000,0.00000,0.87500}%
\definecolor{mycolor19}{rgb}{1.00000,0.00000,0.62500}%
\definecolor{mycolor20}{rgb}{0.85714,0.00000,0.00000}%
\definecolor{mycolor21}{rgb}{0.71429,0.00000,0.00000}%
%
\begin{tikzpicture}

\begin{axis}[%
width=4.1in,
height=3.803in,
at={(0.809in,0.513in)},
scale only axis,
point meta min=0,
point meta max=1,
every outer x axis line/.append style={black},
every x tick label/.append style={font=\color{black}},
xmin=0,
xmax=600,
every outer y axis line/.append style={black},
every y tick label/.append style={font=\color{black}},
ymin=0,
ymax=0.012,
axis background/.style={fill=white},
axis x line*=bottom,
axis y line*=left,
colormap={mymap}{[1pt] rgb(0pt)=(0,1,0); rgb(7pt)=(0,1,1); rgb(15pt)=(0,0,1); rgb(23pt)=(1,0,1); rgb(31pt)=(1,0,0); rgb(38pt)=(0,0,0)},
colorbar,
colorbar style={separate axis lines,every outer x axis line/.append style={black},every x tick label/.append style={font=\color{black}},every outer y axis line/.append style={black},every y tick label/.append style={font=\color{black}},yticklabels={{-19},{-17},{-15},{-13},{-11},{-9},{-7},{-5},{-3},{-1},{1},{3},{5},{7},{9},{11},{13},{15},{17},{19}}}
]
\addplot [color=green,solid,forget plot]
  table[row sep=crcr]{%
0.01	0.01\\
1.01	0.01\\
2.01	0.01\\
3.01	0.01\\
4.01	0.01\\
5.01	0.01\\
6.01	0.01\\
7.01	0.01\\
8.01	0.01\\
9.01	0.01\\
10.01	0.01\\
11.01	0.01\\
12.01	0.01\\
13.01	0.01\\
14.01	0.01\\
15.01	0.01\\
16.01	0.01\\
17.01	0.01\\
18.01	0.01\\
19.01	0.01\\
20.01	0.01\\
21.01	0.01\\
22.01	0.01\\
23.01	0.01\\
24.01	0.01\\
25.01	0.01\\
26.01	0.01\\
27.01	0.01\\
28.01	0.01\\
29.01	0.01\\
30.01	0.01\\
31.01	0.01\\
32.01	0.01\\
33.01	0.01\\
34.01	0.01\\
35.01	0.01\\
36.01	0.01\\
37.01	0.01\\
38.01	0.01\\
39.01	0.01\\
40.01	0.01\\
41.01	0.01\\
42.01	0.01\\
43.01	0.01\\
44.01	0.01\\
45.01	0.01\\
46.01	0.01\\
47.01	0.01\\
48.01	0.01\\
49.01	0.01\\
50.01	0.01\\
51.01	0.01\\
52.01	0.01\\
53.01	0.01\\
54.01	0.01\\
55.01	0.01\\
56.01	0.01\\
57.01	0.01\\
58.01	0.01\\
59.01	0.01\\
60.01	0.01\\
61.01	0.01\\
62.01	0.01\\
63.01	0.01\\
64.01	0.01\\
65.01	0.01\\
66.01	0.01\\
67.01	0.01\\
68.01	0.01\\
69.01	0.01\\
70.01	0.01\\
71.01	0.01\\
72.01	0.01\\
73.01	0.01\\
74.01	0.01\\
75.01	0.01\\
76.01	0.01\\
77.01	0.01\\
78.01	0.01\\
79.01	0.01\\
80.01	0.01\\
81.01	0.01\\
82.01	0.01\\
83.01	0.01\\
84.01	0.01\\
85.01	0.01\\
86.01	0.01\\
87.01	0.01\\
88.01	0.01\\
89.01	0.01\\
90.01	0.01\\
91.01	0.01\\
92.01	0.01\\
93.01	0.01\\
94.01	0.01\\
95.01	0.01\\
96.01	0.01\\
97.01	0.01\\
98.01	0.01\\
99.01	0.01\\
100.01	0.01\\
101.01	0.01\\
102.01	0.01\\
103.01	0.01\\
104.01	0.01\\
105.01	0.01\\
106.01	0.01\\
107.01	0.01\\
108.01	0.01\\
109.01	0.01\\
110.01	0.01\\
111.01	0.01\\
112.01	0.01\\
113.01	0.01\\
114.01	0.01\\
115.01	0.01\\
116.01	0.01\\
117.01	0.01\\
118.01	0.01\\
119.01	0.01\\
120.01	0.01\\
121.01	0.01\\
122.01	0.01\\
123.01	0.01\\
124.01	0.01\\
125.01	0.01\\
126.01	0.01\\
127.01	0.01\\
128.01	0.01\\
129.01	0.01\\
130.01	0.01\\
131.01	0.01\\
132.01	0.01\\
133.01	0.01\\
134.01	0.01\\
135.01	0.01\\
136.01	0.01\\
137.01	0.01\\
138.01	0.01\\
139.01	0.01\\
140.01	0.01\\
141.01	0.01\\
142.01	0.01\\
143.01	0.01\\
144.01	0.01\\
145.01	0.01\\
146.01	0.01\\
147.01	0.01\\
148.01	0.01\\
149.01	0.01\\
150.01	0.01\\
151.01	0.01\\
152.01	0.01\\
153.01	0.01\\
154.01	0.01\\
155.01	0.01\\
156.01	0.01\\
157.01	0.01\\
158.01	0.01\\
159.01	0.01\\
160.01	0.01\\
161.01	0.01\\
162.01	0.01\\
163.01	0.01\\
164.01	0.01\\
165.01	0.01\\
166.01	0.01\\
167.01	0.01\\
168.01	0.01\\
169.01	0.01\\
170.01	0.01\\
171.01	0.01\\
172.01	0.01\\
173.01	0.01\\
174.01	0.01\\
175.01	0.01\\
176.01	0.01\\
177.01	0.01\\
178.01	0.01\\
179.01	0.01\\
180.01	0.01\\
181.01	0.01\\
182.01	0.01\\
183.01	0.01\\
184.01	0.01\\
185.01	0.01\\
186.01	0.01\\
187.01	0.01\\
188.01	0.01\\
189.01	0.01\\
190.01	0.01\\
191.01	0.01\\
192.01	0.01\\
193.01	0.01\\
194.01	0.01\\
195.01	0.01\\
196.01	0.01\\
197.01	0.01\\
198.01	0.01\\
199.01	0.01\\
200.01	0.01\\
201.01	0.01\\
202.01	0.01\\
203.01	0.01\\
204.01	0.01\\
205.01	0.01\\
206.01	0.01\\
207.01	0.01\\
208.01	0.01\\
209.01	0.01\\
210.01	0.01\\
211.01	0.01\\
212.01	0.01\\
213.01	0.01\\
214.01	0.01\\
215.01	0.01\\
216.01	0.01\\
217.01	0.01\\
218.01	0.01\\
219.01	0.01\\
220.01	0.01\\
221.01	0.01\\
222.01	0.01\\
223.01	0.01\\
224.01	0.01\\
225.01	0.01\\
226.01	0.01\\
227.01	0.01\\
228.01	0.01\\
229.01	0.01\\
230.01	0.01\\
231.01	0.01\\
232.01	0.01\\
233.01	0.01\\
234.01	0.01\\
235.01	0.01\\
236.01	0.01\\
237.01	0.01\\
238.01	0.01\\
239.01	0.01\\
240.01	0.01\\
241.01	0.01\\
242.01	0.01\\
243.01	0.01\\
244.01	0.01\\
245.01	0.01\\
246.01	0.01\\
247.01	0.01\\
248.01	0.01\\
249.01	0.01\\
250.01	0.01\\
251.01	0.01\\
252.01	0.01\\
253.01	0.01\\
254.01	0.01\\
255.01	0.01\\
256.01	0.01\\
257.01	0.01\\
258.01	0.01\\
259.01	0.01\\
260.01	0.01\\
261.01	0.01\\
262.01	0.01\\
263.01	0.01\\
264.01	0.01\\
265.01	0.01\\
266.01	0.01\\
267.01	0.01\\
268.01	0.01\\
269.01	0.01\\
270.01	0.01\\
271.01	0.01\\
272.01	0.01\\
273.01	0.01\\
274.01	0.01\\
275.01	0.01\\
276.01	0.01\\
277.01	0.01\\
278.01	0.01\\
279.01	0.01\\
280.01	0.01\\
281.01	0.01\\
282.01	0.01\\
283.01	0.01\\
284.01	0.01\\
285.01	0.01\\
286.01	0.01\\
287.01	0.01\\
288.01	0.01\\
289.01	0.01\\
290.01	0.01\\
291.01	0.01\\
292.01	0.01\\
293.01	0.01\\
294.01	0.01\\
295.01	0.01\\
296.01	0.01\\
297.01	0.01\\
298.01	0.01\\
299.01	0.01\\
300.01	0.01\\
301.01	0.01\\
302.01	0.01\\
303.01	0.01\\
304.01	0.01\\
305.01	0.01\\
306.01	0.01\\
307.01	0.01\\
308.01	0.01\\
309.01	0.01\\
310.01	0.01\\
311.01	0.01\\
312.01	0.01\\
313.01	0.01\\
314.01	0.01\\
315.01	0.01\\
316.01	0.01\\
317.01	0.01\\
318.01	0.01\\
319.01	0.01\\
320.01	0.01\\
321.01	0.01\\
322.01	0.01\\
323.01	0.01\\
324.01	0.01\\
325.01	0.01\\
326.01	0.01\\
327.01	0.01\\
328.01	0.01\\
329.01	0.01\\
330.01	0.01\\
331.01	0.01\\
332.01	0.01\\
333.01	0.01\\
334.01	0.01\\
335.01	0.01\\
336.01	0.01\\
337.01	0.01\\
338.01	0.01\\
339.01	0.01\\
340.01	0.01\\
341.01	0.01\\
342.01	0.01\\
343.01	0.01\\
344.01	0.01\\
345.01	0.01\\
346.01	0.01\\
347.01	0.01\\
348.01	0.01\\
349.01	0.01\\
350.01	0.01\\
351.01	0.01\\
352.01	0.01\\
353.01	0.01\\
354.01	0.01\\
355.01	0.01\\
356.01	0.01\\
357.01	0.01\\
358.01	0.01\\
359.01	0.01\\
360.01	0.01\\
361.01	0.01\\
362.01	0.01\\
363.01	0.01\\
364.01	0.01\\
365.01	0.01\\
366.01	0.01\\
367.01	0.01\\
368.01	0.01\\
369.01	0.01\\
370.01	0.01\\
371.01	0.01\\
372.01	0.01\\
373.01	0.01\\
374.01	0.01\\
375.01	0.01\\
376.01	0.01\\
377.01	0.01\\
378.01	0.01\\
379.01	0.01\\
380.01	0.01\\
381.01	0.01\\
382.01	0.01\\
383.01	0.01\\
384.01	0.01\\
385.01	0.01\\
386.01	0.01\\
387.01	0.01\\
388.01	0.01\\
389.01	0.01\\
390.01	0.01\\
391.01	0.01\\
392.01	0.01\\
393.01	0.01\\
394.01	0.01\\
395.01	0.01\\
396.01	0.01\\
397.01	0.01\\
398.01	0.01\\
399.01	0.01\\
400.01	0.01\\
401.01	0.01\\
402.01	0.01\\
403.01	0.01\\
404.01	0.01\\
405.01	0.01\\
406.01	0.01\\
407.01	0.01\\
408.01	0.01\\
409.01	0.01\\
410.01	0.01\\
411.01	0.01\\
412.01	0.01\\
413.01	0.01\\
414.01	0.01\\
415.01	0.01\\
416.01	0.01\\
417.01	0.01\\
418.01	0.01\\
419.01	0.01\\
420.01	0.01\\
421.01	0.01\\
422.01	0.01\\
423.01	0.01\\
424.01	0.01\\
425.01	0.01\\
426.01	0.01\\
427.01	0.01\\
428.01	0.01\\
429.01	0.01\\
430.01	0.01\\
431.01	0.01\\
432.01	0.01\\
433.01	0.01\\
434.01	0.01\\
435.01	0.01\\
436.01	0.01\\
437.01	0.01\\
438.01	0.01\\
439.01	0.01\\
440.01	0.01\\
441.01	0.01\\
442.01	0.01\\
443.01	0.01\\
444.01	0.01\\
445.01	0.01\\
446.01	0.01\\
447.01	0.01\\
448.01	0.01\\
449.01	0.01\\
450.01	0.01\\
451.01	0.01\\
452.01	0.01\\
453.01	0.01\\
454.01	0.01\\
455.01	0.01\\
456.01	0.01\\
457.01	0.01\\
458.01	0.01\\
459.01	0.01\\
460.01	0.01\\
461.01	0.01\\
462.01	0.01\\
463.01	0.01\\
464.01	0.01\\
465.01	0.01\\
466.01	0.01\\
467.01	0.01\\
468.01	0.01\\
469.01	0.01\\
470.01	0.01\\
471.01	0.01\\
472.01	0.01\\
473.01	0.01\\
474.01	0.01\\
475.01	0.01\\
476.01	0.01\\
477.01	0.01\\
478.01	0.01\\
479.01	0.01\\
480.01	0.01\\
481.01	0.01\\
482.01	0.01\\
483.01	0.01\\
484.01	0.01\\
485.01	0.01\\
486.01	0.01\\
487.01	0.01\\
488.01	0.01\\
489.01	0.01\\
490.01	0.01\\
491.01	0.01\\
492.01	0.01\\
493.01	0.01\\
494.01	0.01\\
495.01	0.01\\
496.01	0.01\\
497.01	0.01\\
498.01	0.01\\
499.01	0.01\\
500.01	0.01\\
501.01	0.01\\
502.01	0.01\\
503.01	0.01\\
504.01	0.01\\
505.01	0.01\\
506.01	0.01\\
507.01	0.01\\
508.01	0.01\\
509.01	0.01\\
510.01	0.01\\
511.01	0.01\\
512.01	0.01\\
513.01	0.01\\
514.01	0.01\\
515.01	0.01\\
516.01	0.01\\
517.01	0.01\\
518.01	0.01\\
519.01	0.01\\
520.01	0.01\\
521.01	0.01\\
522.01	0.01\\
523.01	0.01\\
524.01	0.01\\
525.01	0.01\\
526.01	0.01\\
527.01	0.01\\
528.01	0.01\\
529.01	0.01\\
530.01	0.01\\
531.01	0.01\\
532.01	0.01\\
533.01	0.01\\
534.01	0.01\\
535.01	0.01\\
536.01	0.01\\
537.01	0.01\\
538.01	0.01\\
539.01	0.01\\
540.01	0.01\\
541.01	0.01\\
542.01	0.01\\
543.01	0.01\\
544.01	0.01\\
545.01	0.01\\
546.01	0.01\\
547.01	0.01\\
548.01	0.01\\
549.01	0.01\\
550.01	0.01\\
551.01	0.01\\
552.01	0.01\\
553.01	0.01\\
554.01	0.01\\
555.01	0.01\\
556.01	0.01\\
557.01	0.01\\
558.01	0.01\\
559.01	0.01\\
560.01	0.01\\
561.01	0.01\\
562.01	0.01\\
563.01	0.01\\
564.01	0.01\\
565.01	0.01\\
566.01	0.01\\
567.01	0.01\\
568.01	0.01\\
569.01	0.01\\
570.01	0.01\\
571.01	0.01\\
572.01	0.01\\
573.01	0.01\\
574.01	0.01\\
575.01	0.01\\
576.01	0.01\\
577.01	0.01\\
578.01	0.01\\
579.01	0.01\\
580.01	0.01\\
581.01	0.01\\
582.01	0.01\\
583.01	0.01\\
584.01	0.01\\
585.01	0.01\\
586.01	0.01\\
587.01	0.01\\
588.01	0.01\\
589.01	0.01\\
590.01	0.01\\
591.01	0.01\\
592.01	0.01\\
593.01	0.01\\
594.01	0.01\\
595.01	0.01\\
596.01	0.01\\
597.01	0.01\\
598.01	0.01\\
599.01	0.01\\
599.02	0.01\\
599.03	0.01\\
599.04	0.01\\
599.05	0.01\\
599.06	0.01\\
599.07	0.01\\
599.08	0.01\\
599.09	0.01\\
599.1	0.01\\
599.11	0.01\\
599.12	0.01\\
599.13	0.01\\
599.14	0.01\\
599.15	0.01\\
599.16	0.01\\
599.17	0.01\\
599.18	0.01\\
599.19	0.01\\
599.2	0.01\\
599.21	0.01\\
599.22	0.01\\
599.23	0.01\\
599.24	0.01\\
599.25	0.01\\
599.26	0.01\\
599.27	0.01\\
599.28	0.01\\
599.29	0.01\\
599.3	0.01\\
599.31	0.01\\
599.32	0.01\\
599.33	0.01\\
599.34	0.01\\
599.35	0.01\\
599.36	0.01\\
599.37	0.01\\
599.38	0.01\\
599.39	0.01\\
599.4	0.01\\
599.41	0.01\\
599.42	0.01\\
599.43	0.01\\
599.44	0.01\\
599.45	0.01\\
599.46	0.01\\
599.47	0.01\\
599.48	0.01\\
599.49	0.01\\
599.5	0.01\\
599.51	0.01\\
599.52	0.01\\
599.53	0.01\\
599.54	0.01\\
599.55	0.01\\
599.56	0.01\\
599.57	0.01\\
599.58	0.01\\
599.59	0.01\\
599.6	0.01\\
599.61	0.01\\
599.62	0.01\\
599.63	0.01\\
599.64	0.01\\
599.65	0.01\\
599.66	0.01\\
599.67	0.01\\
599.68	0.01\\
599.69	0.01\\
599.7	0.01\\
599.71	0.01\\
599.72	0.01\\
599.73	0.01\\
599.74	0.01\\
599.75	0.01\\
599.76	0.01\\
599.77	0.01\\
599.78	0.01\\
599.79	0.01\\
599.8	0.01\\
599.81	0.01\\
599.82	0.01\\
599.83	0.01\\
599.84	0.01\\
599.85	0.01\\
599.86	0.01\\
599.87	0.01\\
599.88	0.01\\
599.89	0.01\\
599.9	0.01\\
599.91	0.01\\
599.92	0.01\\
599.93	0.01\\
599.94	0.01\\
599.95	0.01\\
599.96	0.01\\
599.97	0.01\\
599.98	0.01\\
599.99	0.01\\
600	0.01\\
};
\addplot [color=mycolor1,solid,forget plot]
  table[row sep=crcr]{%
0.01	0.01\\
1.01	0.01\\
2.01	0.01\\
3.01	0.01\\
4.01	0.01\\
5.01	0.01\\
6.01	0.01\\
7.01	0.01\\
8.01	0.01\\
9.01	0.01\\
10.01	0.01\\
11.01	0.01\\
12.01	0.01\\
13.01	0.01\\
14.01	0.01\\
15.01	0.01\\
16.01	0.01\\
17.01	0.01\\
18.01	0.01\\
19.01	0.01\\
20.01	0.01\\
21.01	0.01\\
22.01	0.01\\
23.01	0.01\\
24.01	0.01\\
25.01	0.01\\
26.01	0.01\\
27.01	0.01\\
28.01	0.01\\
29.01	0.01\\
30.01	0.01\\
31.01	0.01\\
32.01	0.01\\
33.01	0.01\\
34.01	0.01\\
35.01	0.01\\
36.01	0.01\\
37.01	0.01\\
38.01	0.01\\
39.01	0.01\\
40.01	0.01\\
41.01	0.01\\
42.01	0.01\\
43.01	0.01\\
44.01	0.01\\
45.01	0.01\\
46.01	0.01\\
47.01	0.01\\
48.01	0.01\\
49.01	0.01\\
50.01	0.01\\
51.01	0.01\\
52.01	0.01\\
53.01	0.01\\
54.01	0.01\\
55.01	0.01\\
56.01	0.01\\
57.01	0.01\\
58.01	0.01\\
59.01	0.01\\
60.01	0.01\\
61.01	0.01\\
62.01	0.01\\
63.01	0.01\\
64.01	0.01\\
65.01	0.01\\
66.01	0.01\\
67.01	0.01\\
68.01	0.01\\
69.01	0.01\\
70.01	0.01\\
71.01	0.01\\
72.01	0.01\\
73.01	0.01\\
74.01	0.01\\
75.01	0.01\\
76.01	0.01\\
77.01	0.01\\
78.01	0.01\\
79.01	0.01\\
80.01	0.01\\
81.01	0.01\\
82.01	0.01\\
83.01	0.01\\
84.01	0.01\\
85.01	0.01\\
86.01	0.01\\
87.01	0.01\\
88.01	0.01\\
89.01	0.01\\
90.01	0.01\\
91.01	0.01\\
92.01	0.01\\
93.01	0.01\\
94.01	0.01\\
95.01	0.01\\
96.01	0.01\\
97.01	0.01\\
98.01	0.01\\
99.01	0.01\\
100.01	0.01\\
101.01	0.01\\
102.01	0.01\\
103.01	0.01\\
104.01	0.01\\
105.01	0.01\\
106.01	0.01\\
107.01	0.01\\
108.01	0.01\\
109.01	0.01\\
110.01	0.01\\
111.01	0.01\\
112.01	0.01\\
113.01	0.01\\
114.01	0.01\\
115.01	0.01\\
116.01	0.01\\
117.01	0.01\\
118.01	0.01\\
119.01	0.01\\
120.01	0.01\\
121.01	0.01\\
122.01	0.01\\
123.01	0.01\\
124.01	0.01\\
125.01	0.01\\
126.01	0.01\\
127.01	0.01\\
128.01	0.01\\
129.01	0.01\\
130.01	0.01\\
131.01	0.01\\
132.01	0.01\\
133.01	0.01\\
134.01	0.01\\
135.01	0.01\\
136.01	0.01\\
137.01	0.01\\
138.01	0.01\\
139.01	0.01\\
140.01	0.01\\
141.01	0.01\\
142.01	0.01\\
143.01	0.01\\
144.01	0.01\\
145.01	0.01\\
146.01	0.01\\
147.01	0.01\\
148.01	0.01\\
149.01	0.01\\
150.01	0.01\\
151.01	0.01\\
152.01	0.01\\
153.01	0.01\\
154.01	0.01\\
155.01	0.01\\
156.01	0.01\\
157.01	0.01\\
158.01	0.01\\
159.01	0.01\\
160.01	0.01\\
161.01	0.01\\
162.01	0.01\\
163.01	0.01\\
164.01	0.01\\
165.01	0.01\\
166.01	0.01\\
167.01	0.01\\
168.01	0.01\\
169.01	0.01\\
170.01	0.01\\
171.01	0.01\\
172.01	0.01\\
173.01	0.01\\
174.01	0.01\\
175.01	0.01\\
176.01	0.01\\
177.01	0.01\\
178.01	0.01\\
179.01	0.01\\
180.01	0.01\\
181.01	0.01\\
182.01	0.01\\
183.01	0.01\\
184.01	0.01\\
185.01	0.01\\
186.01	0.01\\
187.01	0.01\\
188.01	0.01\\
189.01	0.01\\
190.01	0.01\\
191.01	0.01\\
192.01	0.01\\
193.01	0.01\\
194.01	0.01\\
195.01	0.01\\
196.01	0.01\\
197.01	0.01\\
198.01	0.01\\
199.01	0.01\\
200.01	0.01\\
201.01	0.01\\
202.01	0.01\\
203.01	0.01\\
204.01	0.01\\
205.01	0.01\\
206.01	0.01\\
207.01	0.01\\
208.01	0.01\\
209.01	0.01\\
210.01	0.01\\
211.01	0.01\\
212.01	0.01\\
213.01	0.01\\
214.01	0.01\\
215.01	0.01\\
216.01	0.01\\
217.01	0.01\\
218.01	0.01\\
219.01	0.01\\
220.01	0.01\\
221.01	0.01\\
222.01	0.01\\
223.01	0.01\\
224.01	0.01\\
225.01	0.01\\
226.01	0.01\\
227.01	0.01\\
228.01	0.01\\
229.01	0.01\\
230.01	0.01\\
231.01	0.01\\
232.01	0.01\\
233.01	0.01\\
234.01	0.01\\
235.01	0.01\\
236.01	0.01\\
237.01	0.01\\
238.01	0.01\\
239.01	0.01\\
240.01	0.01\\
241.01	0.01\\
242.01	0.01\\
243.01	0.01\\
244.01	0.01\\
245.01	0.01\\
246.01	0.01\\
247.01	0.01\\
248.01	0.01\\
249.01	0.01\\
250.01	0.01\\
251.01	0.01\\
252.01	0.01\\
253.01	0.01\\
254.01	0.01\\
255.01	0.01\\
256.01	0.01\\
257.01	0.01\\
258.01	0.01\\
259.01	0.01\\
260.01	0.01\\
261.01	0.01\\
262.01	0.01\\
263.01	0.01\\
264.01	0.01\\
265.01	0.01\\
266.01	0.01\\
267.01	0.01\\
268.01	0.01\\
269.01	0.01\\
270.01	0.01\\
271.01	0.01\\
272.01	0.01\\
273.01	0.01\\
274.01	0.01\\
275.01	0.01\\
276.01	0.01\\
277.01	0.01\\
278.01	0.01\\
279.01	0.01\\
280.01	0.01\\
281.01	0.01\\
282.01	0.01\\
283.01	0.01\\
284.01	0.01\\
285.01	0.01\\
286.01	0.01\\
287.01	0.01\\
288.01	0.01\\
289.01	0.01\\
290.01	0.01\\
291.01	0.01\\
292.01	0.01\\
293.01	0.01\\
294.01	0.01\\
295.01	0.01\\
296.01	0.01\\
297.01	0.01\\
298.01	0.01\\
299.01	0.01\\
300.01	0.01\\
301.01	0.01\\
302.01	0.01\\
303.01	0.01\\
304.01	0.01\\
305.01	0.01\\
306.01	0.01\\
307.01	0.01\\
308.01	0.01\\
309.01	0.01\\
310.01	0.01\\
311.01	0.01\\
312.01	0.01\\
313.01	0.01\\
314.01	0.01\\
315.01	0.01\\
316.01	0.01\\
317.01	0.01\\
318.01	0.01\\
319.01	0.01\\
320.01	0.01\\
321.01	0.01\\
322.01	0.01\\
323.01	0.01\\
324.01	0.01\\
325.01	0.01\\
326.01	0.01\\
327.01	0.01\\
328.01	0.01\\
329.01	0.01\\
330.01	0.01\\
331.01	0.01\\
332.01	0.01\\
333.01	0.01\\
334.01	0.01\\
335.01	0.01\\
336.01	0.01\\
337.01	0.01\\
338.01	0.01\\
339.01	0.01\\
340.01	0.01\\
341.01	0.01\\
342.01	0.01\\
343.01	0.01\\
344.01	0.01\\
345.01	0.01\\
346.01	0.01\\
347.01	0.01\\
348.01	0.01\\
349.01	0.01\\
350.01	0.01\\
351.01	0.01\\
352.01	0.01\\
353.01	0.01\\
354.01	0.01\\
355.01	0.01\\
356.01	0.01\\
357.01	0.01\\
358.01	0.01\\
359.01	0.01\\
360.01	0.01\\
361.01	0.01\\
362.01	0.01\\
363.01	0.01\\
364.01	0.01\\
365.01	0.01\\
366.01	0.01\\
367.01	0.01\\
368.01	0.01\\
369.01	0.01\\
370.01	0.01\\
371.01	0.01\\
372.01	0.01\\
373.01	0.01\\
374.01	0.01\\
375.01	0.01\\
376.01	0.01\\
377.01	0.01\\
378.01	0.01\\
379.01	0.01\\
380.01	0.01\\
381.01	0.01\\
382.01	0.01\\
383.01	0.01\\
384.01	0.01\\
385.01	0.01\\
386.01	0.01\\
387.01	0.01\\
388.01	0.01\\
389.01	0.01\\
390.01	0.01\\
391.01	0.01\\
392.01	0.01\\
393.01	0.01\\
394.01	0.01\\
395.01	0.01\\
396.01	0.01\\
397.01	0.01\\
398.01	0.01\\
399.01	0.01\\
400.01	0.01\\
401.01	0.01\\
402.01	0.01\\
403.01	0.01\\
404.01	0.01\\
405.01	0.01\\
406.01	0.01\\
407.01	0.01\\
408.01	0.01\\
409.01	0.01\\
410.01	0.01\\
411.01	0.01\\
412.01	0.01\\
413.01	0.01\\
414.01	0.01\\
415.01	0.01\\
416.01	0.01\\
417.01	0.01\\
418.01	0.01\\
419.01	0.01\\
420.01	0.01\\
421.01	0.01\\
422.01	0.01\\
423.01	0.01\\
424.01	0.01\\
425.01	0.01\\
426.01	0.01\\
427.01	0.01\\
428.01	0.01\\
429.01	0.01\\
430.01	0.01\\
431.01	0.01\\
432.01	0.01\\
433.01	0.01\\
434.01	0.01\\
435.01	0.01\\
436.01	0.01\\
437.01	0.01\\
438.01	0.01\\
439.01	0.01\\
440.01	0.01\\
441.01	0.01\\
442.01	0.01\\
443.01	0.01\\
444.01	0.01\\
445.01	0.01\\
446.01	0.01\\
447.01	0.01\\
448.01	0.01\\
449.01	0.01\\
450.01	0.01\\
451.01	0.01\\
452.01	0.01\\
453.01	0.01\\
454.01	0.01\\
455.01	0.01\\
456.01	0.01\\
457.01	0.01\\
458.01	0.01\\
459.01	0.01\\
460.01	0.01\\
461.01	0.01\\
462.01	0.01\\
463.01	0.01\\
464.01	0.01\\
465.01	0.01\\
466.01	0.01\\
467.01	0.01\\
468.01	0.01\\
469.01	0.01\\
470.01	0.01\\
471.01	0.01\\
472.01	0.01\\
473.01	0.01\\
474.01	0.01\\
475.01	0.01\\
476.01	0.01\\
477.01	0.01\\
478.01	0.01\\
479.01	0.01\\
480.01	0.01\\
481.01	0.01\\
482.01	0.01\\
483.01	0.01\\
484.01	0.01\\
485.01	0.01\\
486.01	0.01\\
487.01	0.01\\
488.01	0.01\\
489.01	0.01\\
490.01	0.01\\
491.01	0.01\\
492.01	0.01\\
493.01	0.01\\
494.01	0.01\\
495.01	0.01\\
496.01	0.01\\
497.01	0.01\\
498.01	0.01\\
499.01	0.01\\
500.01	0.01\\
501.01	0.01\\
502.01	0.01\\
503.01	0.01\\
504.01	0.01\\
505.01	0.01\\
506.01	0.01\\
507.01	0.01\\
508.01	0.01\\
509.01	0.01\\
510.01	0.01\\
511.01	0.01\\
512.01	0.01\\
513.01	0.01\\
514.01	0.01\\
515.01	0.01\\
516.01	0.01\\
517.01	0.01\\
518.01	0.01\\
519.01	0.01\\
520.01	0.01\\
521.01	0.01\\
522.01	0.01\\
523.01	0.01\\
524.01	0.01\\
525.01	0.01\\
526.01	0.01\\
527.01	0.01\\
528.01	0.01\\
529.01	0.01\\
530.01	0.01\\
531.01	0.01\\
532.01	0.01\\
533.01	0.01\\
534.01	0.01\\
535.01	0.01\\
536.01	0.01\\
537.01	0.01\\
538.01	0.01\\
539.01	0.01\\
540.01	0.01\\
541.01	0.01\\
542.01	0.01\\
543.01	0.01\\
544.01	0.01\\
545.01	0.01\\
546.01	0.01\\
547.01	0.01\\
548.01	0.01\\
549.01	0.01\\
550.01	0.01\\
551.01	0.01\\
552.01	0.01\\
553.01	0.01\\
554.01	0.01\\
555.01	0.01\\
556.01	0.01\\
557.01	0.01\\
558.01	0.01\\
559.01	0.01\\
560.01	0.01\\
561.01	0.01\\
562.01	0.01\\
563.01	0.01\\
564.01	0.01\\
565.01	0.01\\
566.01	0.01\\
567.01	0.01\\
568.01	0.01\\
569.01	0.01\\
570.01	0.01\\
571.01	0.01\\
572.01	0.01\\
573.01	0.01\\
574.01	0.01\\
575.01	0.01\\
576.01	0.01\\
577.01	0.01\\
578.01	0.01\\
579.01	0.01\\
580.01	0.01\\
581.01	0.01\\
582.01	0.01\\
583.01	0.01\\
584.01	0.01\\
585.01	0.01\\
586.01	0.01\\
587.01	0.01\\
588.01	0.01\\
589.01	0.01\\
590.01	0.01\\
591.01	0.01\\
592.01	0.01\\
593.01	0.01\\
594.01	0.01\\
595.01	0.01\\
596.01	0.01\\
597.01	0.01\\
598.01	0.01\\
599.01	0.01\\
599.02	0.01\\
599.03	0.01\\
599.04	0.01\\
599.05	0.01\\
599.06	0.01\\
599.07	0.01\\
599.08	0.01\\
599.09	0.01\\
599.1	0.01\\
599.11	0.01\\
599.12	0.01\\
599.13	0.01\\
599.14	0.01\\
599.15	0.01\\
599.16	0.01\\
599.17	0.01\\
599.18	0.01\\
599.19	0.01\\
599.2	0.01\\
599.21	0.01\\
599.22	0.01\\
599.23	0.01\\
599.24	0.01\\
599.25	0.01\\
599.26	0.01\\
599.27	0.01\\
599.28	0.01\\
599.29	0.01\\
599.3	0.01\\
599.31	0.01\\
599.32	0.01\\
599.33	0.01\\
599.34	0.01\\
599.35	0.01\\
599.36	0.01\\
599.37	0.01\\
599.38	0.01\\
599.39	0.01\\
599.4	0.01\\
599.41	0.01\\
599.42	0.01\\
599.43	0.01\\
599.44	0.01\\
599.45	0.01\\
599.46	0.01\\
599.47	0.01\\
599.48	0.01\\
599.49	0.01\\
599.5	0.01\\
599.51	0.01\\
599.52	0.01\\
599.53	0.01\\
599.54	0.01\\
599.55	0.01\\
599.56	0.01\\
599.57	0.01\\
599.58	0.01\\
599.59	0.01\\
599.6	0.01\\
599.61	0.01\\
599.62	0.01\\
599.63	0.01\\
599.64	0.01\\
599.65	0.01\\
599.66	0.01\\
599.67	0.01\\
599.68	0.01\\
599.69	0.01\\
599.7	0.01\\
599.71	0.01\\
599.72	0.01\\
599.73	0.01\\
599.74	0.01\\
599.75	0.01\\
599.76	0.01\\
599.77	0.01\\
599.78	0.01\\
599.79	0.01\\
599.8	0.01\\
599.81	0.01\\
599.82	0.01\\
599.83	0.01\\
599.84	0.01\\
599.85	0.01\\
599.86	0.01\\
599.87	0.01\\
599.88	0.01\\
599.89	0.01\\
599.9	0.01\\
599.91	0.01\\
599.92	0.01\\
599.93	0.01\\
599.94	0.01\\
599.95	0.01\\
599.96	0.01\\
599.97	0.01\\
599.98	0.01\\
599.99	0.01\\
600	0.01\\
};
\addplot [color=mycolor2,solid,forget plot]
  table[row sep=crcr]{%
0.01	0.01\\
1.01	0.01\\
2.01	0.01\\
3.01	0.01\\
4.01	0.01\\
5.01	0.01\\
6.01	0.01\\
7.01	0.01\\
8.01	0.01\\
9.01	0.01\\
10.01	0.01\\
11.01	0.01\\
12.01	0.01\\
13.01	0.01\\
14.01	0.01\\
15.01	0.01\\
16.01	0.01\\
17.01	0.01\\
18.01	0.01\\
19.01	0.01\\
20.01	0.01\\
21.01	0.01\\
22.01	0.01\\
23.01	0.01\\
24.01	0.01\\
25.01	0.01\\
26.01	0.01\\
27.01	0.01\\
28.01	0.01\\
29.01	0.01\\
30.01	0.01\\
31.01	0.01\\
32.01	0.01\\
33.01	0.01\\
34.01	0.01\\
35.01	0.01\\
36.01	0.01\\
37.01	0.01\\
38.01	0.01\\
39.01	0.01\\
40.01	0.01\\
41.01	0.01\\
42.01	0.01\\
43.01	0.01\\
44.01	0.01\\
45.01	0.01\\
46.01	0.01\\
47.01	0.01\\
48.01	0.01\\
49.01	0.01\\
50.01	0.01\\
51.01	0.01\\
52.01	0.01\\
53.01	0.01\\
54.01	0.01\\
55.01	0.01\\
56.01	0.01\\
57.01	0.01\\
58.01	0.01\\
59.01	0.01\\
60.01	0.01\\
61.01	0.01\\
62.01	0.01\\
63.01	0.01\\
64.01	0.01\\
65.01	0.01\\
66.01	0.01\\
67.01	0.01\\
68.01	0.01\\
69.01	0.01\\
70.01	0.01\\
71.01	0.01\\
72.01	0.01\\
73.01	0.01\\
74.01	0.01\\
75.01	0.01\\
76.01	0.01\\
77.01	0.01\\
78.01	0.01\\
79.01	0.01\\
80.01	0.01\\
81.01	0.01\\
82.01	0.01\\
83.01	0.01\\
84.01	0.01\\
85.01	0.01\\
86.01	0.01\\
87.01	0.01\\
88.01	0.01\\
89.01	0.01\\
90.01	0.01\\
91.01	0.01\\
92.01	0.01\\
93.01	0.01\\
94.01	0.01\\
95.01	0.01\\
96.01	0.01\\
97.01	0.01\\
98.01	0.01\\
99.01	0.01\\
100.01	0.01\\
101.01	0.01\\
102.01	0.01\\
103.01	0.01\\
104.01	0.01\\
105.01	0.01\\
106.01	0.01\\
107.01	0.01\\
108.01	0.01\\
109.01	0.01\\
110.01	0.01\\
111.01	0.01\\
112.01	0.01\\
113.01	0.01\\
114.01	0.01\\
115.01	0.01\\
116.01	0.01\\
117.01	0.01\\
118.01	0.01\\
119.01	0.01\\
120.01	0.01\\
121.01	0.01\\
122.01	0.01\\
123.01	0.01\\
124.01	0.01\\
125.01	0.01\\
126.01	0.01\\
127.01	0.01\\
128.01	0.01\\
129.01	0.01\\
130.01	0.01\\
131.01	0.01\\
132.01	0.01\\
133.01	0.01\\
134.01	0.01\\
135.01	0.01\\
136.01	0.01\\
137.01	0.01\\
138.01	0.01\\
139.01	0.01\\
140.01	0.01\\
141.01	0.01\\
142.01	0.01\\
143.01	0.01\\
144.01	0.01\\
145.01	0.01\\
146.01	0.01\\
147.01	0.01\\
148.01	0.01\\
149.01	0.01\\
150.01	0.01\\
151.01	0.01\\
152.01	0.01\\
153.01	0.01\\
154.01	0.01\\
155.01	0.01\\
156.01	0.01\\
157.01	0.01\\
158.01	0.01\\
159.01	0.01\\
160.01	0.01\\
161.01	0.01\\
162.01	0.01\\
163.01	0.01\\
164.01	0.01\\
165.01	0.01\\
166.01	0.01\\
167.01	0.01\\
168.01	0.01\\
169.01	0.01\\
170.01	0.01\\
171.01	0.01\\
172.01	0.01\\
173.01	0.01\\
174.01	0.01\\
175.01	0.01\\
176.01	0.01\\
177.01	0.01\\
178.01	0.01\\
179.01	0.01\\
180.01	0.01\\
181.01	0.01\\
182.01	0.01\\
183.01	0.01\\
184.01	0.01\\
185.01	0.01\\
186.01	0.01\\
187.01	0.01\\
188.01	0.01\\
189.01	0.01\\
190.01	0.01\\
191.01	0.01\\
192.01	0.01\\
193.01	0.01\\
194.01	0.01\\
195.01	0.01\\
196.01	0.01\\
197.01	0.01\\
198.01	0.01\\
199.01	0.01\\
200.01	0.01\\
201.01	0.01\\
202.01	0.01\\
203.01	0.01\\
204.01	0.01\\
205.01	0.01\\
206.01	0.01\\
207.01	0.01\\
208.01	0.01\\
209.01	0.01\\
210.01	0.01\\
211.01	0.01\\
212.01	0.01\\
213.01	0.01\\
214.01	0.01\\
215.01	0.01\\
216.01	0.01\\
217.01	0.01\\
218.01	0.01\\
219.01	0.01\\
220.01	0.01\\
221.01	0.01\\
222.01	0.01\\
223.01	0.01\\
224.01	0.01\\
225.01	0.01\\
226.01	0.01\\
227.01	0.01\\
228.01	0.01\\
229.01	0.01\\
230.01	0.01\\
231.01	0.01\\
232.01	0.01\\
233.01	0.01\\
234.01	0.01\\
235.01	0.01\\
236.01	0.01\\
237.01	0.01\\
238.01	0.01\\
239.01	0.01\\
240.01	0.01\\
241.01	0.01\\
242.01	0.01\\
243.01	0.01\\
244.01	0.01\\
245.01	0.01\\
246.01	0.01\\
247.01	0.01\\
248.01	0.01\\
249.01	0.01\\
250.01	0.01\\
251.01	0.01\\
252.01	0.01\\
253.01	0.01\\
254.01	0.01\\
255.01	0.01\\
256.01	0.01\\
257.01	0.01\\
258.01	0.01\\
259.01	0.01\\
260.01	0.01\\
261.01	0.01\\
262.01	0.01\\
263.01	0.01\\
264.01	0.01\\
265.01	0.01\\
266.01	0.01\\
267.01	0.01\\
268.01	0.01\\
269.01	0.01\\
270.01	0.01\\
271.01	0.01\\
272.01	0.01\\
273.01	0.01\\
274.01	0.01\\
275.01	0.01\\
276.01	0.01\\
277.01	0.01\\
278.01	0.01\\
279.01	0.01\\
280.01	0.01\\
281.01	0.01\\
282.01	0.01\\
283.01	0.01\\
284.01	0.01\\
285.01	0.01\\
286.01	0.01\\
287.01	0.01\\
288.01	0.01\\
289.01	0.01\\
290.01	0.01\\
291.01	0.01\\
292.01	0.01\\
293.01	0.01\\
294.01	0.01\\
295.01	0.01\\
296.01	0.01\\
297.01	0.01\\
298.01	0.01\\
299.01	0.01\\
300.01	0.01\\
301.01	0.01\\
302.01	0.01\\
303.01	0.01\\
304.01	0.01\\
305.01	0.01\\
306.01	0.01\\
307.01	0.01\\
308.01	0.01\\
309.01	0.01\\
310.01	0.01\\
311.01	0.01\\
312.01	0.01\\
313.01	0.01\\
314.01	0.01\\
315.01	0.01\\
316.01	0.01\\
317.01	0.01\\
318.01	0.01\\
319.01	0.01\\
320.01	0.01\\
321.01	0.01\\
322.01	0.01\\
323.01	0.01\\
324.01	0.01\\
325.01	0.01\\
326.01	0.01\\
327.01	0.01\\
328.01	0.01\\
329.01	0.01\\
330.01	0.01\\
331.01	0.01\\
332.01	0.01\\
333.01	0.01\\
334.01	0.01\\
335.01	0.01\\
336.01	0.01\\
337.01	0.01\\
338.01	0.01\\
339.01	0.01\\
340.01	0.01\\
341.01	0.01\\
342.01	0.01\\
343.01	0.01\\
344.01	0.01\\
345.01	0.01\\
346.01	0.01\\
347.01	0.01\\
348.01	0.01\\
349.01	0.01\\
350.01	0.01\\
351.01	0.01\\
352.01	0.01\\
353.01	0.01\\
354.01	0.01\\
355.01	0.01\\
356.01	0.01\\
357.01	0.01\\
358.01	0.01\\
359.01	0.01\\
360.01	0.01\\
361.01	0.01\\
362.01	0.01\\
363.01	0.01\\
364.01	0.01\\
365.01	0.01\\
366.01	0.01\\
367.01	0.01\\
368.01	0.01\\
369.01	0.01\\
370.01	0.01\\
371.01	0.01\\
372.01	0.01\\
373.01	0.01\\
374.01	0.01\\
375.01	0.01\\
376.01	0.01\\
377.01	0.01\\
378.01	0.01\\
379.01	0.01\\
380.01	0.01\\
381.01	0.01\\
382.01	0.01\\
383.01	0.01\\
384.01	0.01\\
385.01	0.01\\
386.01	0.01\\
387.01	0.01\\
388.01	0.01\\
389.01	0.01\\
390.01	0.01\\
391.01	0.01\\
392.01	0.01\\
393.01	0.01\\
394.01	0.01\\
395.01	0.01\\
396.01	0.01\\
397.01	0.01\\
398.01	0.01\\
399.01	0.01\\
400.01	0.01\\
401.01	0.01\\
402.01	0.01\\
403.01	0.01\\
404.01	0.01\\
405.01	0.01\\
406.01	0.01\\
407.01	0.01\\
408.01	0.01\\
409.01	0.01\\
410.01	0.01\\
411.01	0.01\\
412.01	0.01\\
413.01	0.01\\
414.01	0.01\\
415.01	0.01\\
416.01	0.01\\
417.01	0.01\\
418.01	0.01\\
419.01	0.01\\
420.01	0.01\\
421.01	0.01\\
422.01	0.01\\
423.01	0.01\\
424.01	0.01\\
425.01	0.01\\
426.01	0.01\\
427.01	0.01\\
428.01	0.01\\
429.01	0.01\\
430.01	0.01\\
431.01	0.01\\
432.01	0.01\\
433.01	0.01\\
434.01	0.01\\
435.01	0.01\\
436.01	0.01\\
437.01	0.01\\
438.01	0.01\\
439.01	0.01\\
440.01	0.01\\
441.01	0.01\\
442.01	0.01\\
443.01	0.01\\
444.01	0.01\\
445.01	0.01\\
446.01	0.01\\
447.01	0.01\\
448.01	0.01\\
449.01	0.01\\
450.01	0.01\\
451.01	0.01\\
452.01	0.01\\
453.01	0.01\\
454.01	0.01\\
455.01	0.01\\
456.01	0.01\\
457.01	0.01\\
458.01	0.01\\
459.01	0.01\\
460.01	0.01\\
461.01	0.01\\
462.01	0.01\\
463.01	0.01\\
464.01	0.01\\
465.01	0.01\\
466.01	0.01\\
467.01	0.01\\
468.01	0.01\\
469.01	0.01\\
470.01	0.01\\
471.01	0.01\\
472.01	0.01\\
473.01	0.01\\
474.01	0.01\\
475.01	0.01\\
476.01	0.01\\
477.01	0.01\\
478.01	0.01\\
479.01	0.01\\
480.01	0.01\\
481.01	0.01\\
482.01	0.01\\
483.01	0.01\\
484.01	0.01\\
485.01	0.01\\
486.01	0.01\\
487.01	0.01\\
488.01	0.01\\
489.01	0.01\\
490.01	0.01\\
491.01	0.01\\
492.01	0.01\\
493.01	0.01\\
494.01	0.01\\
495.01	0.01\\
496.01	0.01\\
497.01	0.01\\
498.01	0.01\\
499.01	0.01\\
500.01	0.01\\
501.01	0.01\\
502.01	0.01\\
503.01	0.01\\
504.01	0.01\\
505.01	0.01\\
506.01	0.01\\
507.01	0.01\\
508.01	0.01\\
509.01	0.01\\
510.01	0.01\\
511.01	0.01\\
512.01	0.01\\
513.01	0.01\\
514.01	0.01\\
515.01	0.01\\
516.01	0.01\\
517.01	0.01\\
518.01	0.01\\
519.01	0.01\\
520.01	0.01\\
521.01	0.01\\
522.01	0.01\\
523.01	0.01\\
524.01	0.01\\
525.01	0.01\\
526.01	0.01\\
527.01	0.01\\
528.01	0.01\\
529.01	0.01\\
530.01	0.01\\
531.01	0.01\\
532.01	0.01\\
533.01	0.01\\
534.01	0.01\\
535.01	0.01\\
536.01	0.01\\
537.01	0.01\\
538.01	0.01\\
539.01	0.01\\
540.01	0.01\\
541.01	0.01\\
542.01	0.01\\
543.01	0.01\\
544.01	0.01\\
545.01	0.01\\
546.01	0.01\\
547.01	0.01\\
548.01	0.01\\
549.01	0.01\\
550.01	0.01\\
551.01	0.01\\
552.01	0.01\\
553.01	0.01\\
554.01	0.01\\
555.01	0.01\\
556.01	0.01\\
557.01	0.01\\
558.01	0.01\\
559.01	0.01\\
560.01	0.01\\
561.01	0.01\\
562.01	0.01\\
563.01	0.01\\
564.01	0.01\\
565.01	0.01\\
566.01	0.01\\
567.01	0.01\\
568.01	0.01\\
569.01	0.01\\
570.01	0.01\\
571.01	0.01\\
572.01	0.01\\
573.01	0.01\\
574.01	0.01\\
575.01	0.01\\
576.01	0.01\\
577.01	0.01\\
578.01	0.01\\
579.01	0.01\\
580.01	0.01\\
581.01	0.01\\
582.01	0.01\\
583.01	0.01\\
584.01	0.01\\
585.01	0.01\\
586.01	0.01\\
587.01	0.01\\
588.01	0.01\\
589.01	0.01\\
590.01	0.01\\
591.01	0.01\\
592.01	0.01\\
593.01	0.01\\
594.01	0.01\\
595.01	0.01\\
596.01	0.01\\
597.01	0.01\\
598.01	0.01\\
599.01	0.01\\
599.02	0.01\\
599.03	0.01\\
599.04	0.01\\
599.05	0.01\\
599.06	0.01\\
599.07	0.01\\
599.08	0.01\\
599.09	0.01\\
599.1	0.01\\
599.11	0.01\\
599.12	0.01\\
599.13	0.01\\
599.14	0.01\\
599.15	0.01\\
599.16	0.01\\
599.17	0.01\\
599.18	0.01\\
599.19	0.01\\
599.2	0.01\\
599.21	0.01\\
599.22	0.01\\
599.23	0.01\\
599.24	0.01\\
599.25	0.01\\
599.26	0.01\\
599.27	0.01\\
599.28	0.01\\
599.29	0.01\\
599.3	0.01\\
599.31	0.01\\
599.32	0.01\\
599.33	0.01\\
599.34	0.01\\
599.35	0.01\\
599.36	0.01\\
599.37	0.01\\
599.38	0.01\\
599.39	0.01\\
599.4	0.01\\
599.41	0.01\\
599.42	0.01\\
599.43	0.01\\
599.44	0.01\\
599.45	0.01\\
599.46	0.01\\
599.47	0.01\\
599.48	0.01\\
599.49	0.01\\
599.5	0.01\\
599.51	0.01\\
599.52	0.01\\
599.53	0.01\\
599.54	0.01\\
599.55	0.01\\
599.56	0.01\\
599.57	0.01\\
599.58	0.01\\
599.59	0.01\\
599.6	0.01\\
599.61	0.01\\
599.62	0.01\\
599.63	0.01\\
599.64	0.01\\
599.65	0.01\\
599.66	0.01\\
599.67	0.01\\
599.68	0.01\\
599.69	0.01\\
599.7	0.01\\
599.71	0.01\\
599.72	0.01\\
599.73	0.01\\
599.74	0.01\\
599.75	0.01\\
599.76	0.01\\
599.77	0.01\\
599.78	0.01\\
599.79	0.01\\
599.8	0.01\\
599.81	0.01\\
599.82	0.01\\
599.83	0.01\\
599.84	0.01\\
599.85	0.01\\
599.86	0.01\\
599.87	0.01\\
599.88	0.01\\
599.89	0.01\\
599.9	0.01\\
599.91	0.01\\
599.92	0.01\\
599.93	0.01\\
599.94	0.01\\
599.95	0.01\\
599.96	0.01\\
599.97	0.01\\
599.98	0.01\\
599.99	0.01\\
600	0.01\\
};
\addplot [color=mycolor3,solid,forget plot]
  table[row sep=crcr]{%
0.01	0.01\\
1.01	0.01\\
2.01	0.01\\
3.01	0.01\\
4.01	0.01\\
5.01	0.01\\
6.01	0.01\\
7.01	0.01\\
8.01	0.01\\
9.01	0.01\\
10.01	0.01\\
11.01	0.01\\
12.01	0.01\\
13.01	0.01\\
14.01	0.01\\
15.01	0.01\\
16.01	0.01\\
17.01	0.01\\
18.01	0.01\\
19.01	0.01\\
20.01	0.01\\
21.01	0.01\\
22.01	0.01\\
23.01	0.01\\
24.01	0.01\\
25.01	0.01\\
26.01	0.01\\
27.01	0.01\\
28.01	0.01\\
29.01	0.01\\
30.01	0.01\\
31.01	0.01\\
32.01	0.01\\
33.01	0.01\\
34.01	0.01\\
35.01	0.01\\
36.01	0.01\\
37.01	0.01\\
38.01	0.01\\
39.01	0.01\\
40.01	0.01\\
41.01	0.01\\
42.01	0.01\\
43.01	0.01\\
44.01	0.01\\
45.01	0.01\\
46.01	0.01\\
47.01	0.01\\
48.01	0.01\\
49.01	0.01\\
50.01	0.01\\
51.01	0.01\\
52.01	0.01\\
53.01	0.01\\
54.01	0.01\\
55.01	0.01\\
56.01	0.01\\
57.01	0.01\\
58.01	0.01\\
59.01	0.01\\
60.01	0.01\\
61.01	0.01\\
62.01	0.01\\
63.01	0.01\\
64.01	0.01\\
65.01	0.01\\
66.01	0.01\\
67.01	0.01\\
68.01	0.01\\
69.01	0.01\\
70.01	0.01\\
71.01	0.01\\
72.01	0.01\\
73.01	0.01\\
74.01	0.01\\
75.01	0.01\\
76.01	0.01\\
77.01	0.01\\
78.01	0.01\\
79.01	0.01\\
80.01	0.01\\
81.01	0.01\\
82.01	0.01\\
83.01	0.01\\
84.01	0.01\\
85.01	0.01\\
86.01	0.01\\
87.01	0.01\\
88.01	0.01\\
89.01	0.01\\
90.01	0.01\\
91.01	0.01\\
92.01	0.01\\
93.01	0.01\\
94.01	0.01\\
95.01	0.01\\
96.01	0.01\\
97.01	0.01\\
98.01	0.01\\
99.01	0.01\\
100.01	0.01\\
101.01	0.01\\
102.01	0.01\\
103.01	0.01\\
104.01	0.01\\
105.01	0.01\\
106.01	0.01\\
107.01	0.01\\
108.01	0.01\\
109.01	0.01\\
110.01	0.01\\
111.01	0.01\\
112.01	0.01\\
113.01	0.01\\
114.01	0.01\\
115.01	0.01\\
116.01	0.01\\
117.01	0.01\\
118.01	0.01\\
119.01	0.01\\
120.01	0.01\\
121.01	0.01\\
122.01	0.01\\
123.01	0.01\\
124.01	0.01\\
125.01	0.01\\
126.01	0.01\\
127.01	0.01\\
128.01	0.01\\
129.01	0.01\\
130.01	0.01\\
131.01	0.01\\
132.01	0.01\\
133.01	0.01\\
134.01	0.01\\
135.01	0.01\\
136.01	0.01\\
137.01	0.01\\
138.01	0.01\\
139.01	0.01\\
140.01	0.01\\
141.01	0.01\\
142.01	0.01\\
143.01	0.01\\
144.01	0.01\\
145.01	0.01\\
146.01	0.01\\
147.01	0.01\\
148.01	0.01\\
149.01	0.01\\
150.01	0.01\\
151.01	0.01\\
152.01	0.01\\
153.01	0.01\\
154.01	0.01\\
155.01	0.01\\
156.01	0.01\\
157.01	0.01\\
158.01	0.01\\
159.01	0.01\\
160.01	0.01\\
161.01	0.01\\
162.01	0.01\\
163.01	0.01\\
164.01	0.01\\
165.01	0.01\\
166.01	0.01\\
167.01	0.01\\
168.01	0.01\\
169.01	0.01\\
170.01	0.01\\
171.01	0.01\\
172.01	0.01\\
173.01	0.01\\
174.01	0.01\\
175.01	0.01\\
176.01	0.01\\
177.01	0.01\\
178.01	0.01\\
179.01	0.01\\
180.01	0.01\\
181.01	0.01\\
182.01	0.01\\
183.01	0.01\\
184.01	0.01\\
185.01	0.01\\
186.01	0.01\\
187.01	0.01\\
188.01	0.01\\
189.01	0.01\\
190.01	0.01\\
191.01	0.01\\
192.01	0.01\\
193.01	0.01\\
194.01	0.01\\
195.01	0.01\\
196.01	0.01\\
197.01	0.01\\
198.01	0.01\\
199.01	0.01\\
200.01	0.01\\
201.01	0.01\\
202.01	0.01\\
203.01	0.01\\
204.01	0.01\\
205.01	0.01\\
206.01	0.01\\
207.01	0.01\\
208.01	0.01\\
209.01	0.01\\
210.01	0.01\\
211.01	0.01\\
212.01	0.01\\
213.01	0.01\\
214.01	0.01\\
215.01	0.01\\
216.01	0.01\\
217.01	0.01\\
218.01	0.01\\
219.01	0.01\\
220.01	0.01\\
221.01	0.01\\
222.01	0.01\\
223.01	0.01\\
224.01	0.01\\
225.01	0.01\\
226.01	0.01\\
227.01	0.01\\
228.01	0.01\\
229.01	0.01\\
230.01	0.01\\
231.01	0.01\\
232.01	0.01\\
233.01	0.01\\
234.01	0.01\\
235.01	0.01\\
236.01	0.01\\
237.01	0.01\\
238.01	0.01\\
239.01	0.01\\
240.01	0.01\\
241.01	0.01\\
242.01	0.01\\
243.01	0.01\\
244.01	0.01\\
245.01	0.01\\
246.01	0.01\\
247.01	0.01\\
248.01	0.01\\
249.01	0.01\\
250.01	0.01\\
251.01	0.01\\
252.01	0.01\\
253.01	0.01\\
254.01	0.01\\
255.01	0.01\\
256.01	0.01\\
257.01	0.01\\
258.01	0.01\\
259.01	0.01\\
260.01	0.01\\
261.01	0.01\\
262.01	0.01\\
263.01	0.01\\
264.01	0.01\\
265.01	0.01\\
266.01	0.01\\
267.01	0.01\\
268.01	0.01\\
269.01	0.01\\
270.01	0.01\\
271.01	0.01\\
272.01	0.01\\
273.01	0.01\\
274.01	0.01\\
275.01	0.01\\
276.01	0.01\\
277.01	0.01\\
278.01	0.01\\
279.01	0.01\\
280.01	0.01\\
281.01	0.01\\
282.01	0.01\\
283.01	0.01\\
284.01	0.01\\
285.01	0.01\\
286.01	0.01\\
287.01	0.01\\
288.01	0.01\\
289.01	0.01\\
290.01	0.01\\
291.01	0.01\\
292.01	0.01\\
293.01	0.01\\
294.01	0.01\\
295.01	0.01\\
296.01	0.01\\
297.01	0.01\\
298.01	0.01\\
299.01	0.01\\
300.01	0.01\\
301.01	0.01\\
302.01	0.01\\
303.01	0.01\\
304.01	0.01\\
305.01	0.01\\
306.01	0.01\\
307.01	0.01\\
308.01	0.01\\
309.01	0.01\\
310.01	0.01\\
311.01	0.01\\
312.01	0.01\\
313.01	0.01\\
314.01	0.01\\
315.01	0.01\\
316.01	0.01\\
317.01	0.01\\
318.01	0.01\\
319.01	0.01\\
320.01	0.01\\
321.01	0.01\\
322.01	0.01\\
323.01	0.01\\
324.01	0.01\\
325.01	0.01\\
326.01	0.01\\
327.01	0.01\\
328.01	0.01\\
329.01	0.01\\
330.01	0.01\\
331.01	0.01\\
332.01	0.01\\
333.01	0.01\\
334.01	0.01\\
335.01	0.01\\
336.01	0.01\\
337.01	0.01\\
338.01	0.01\\
339.01	0.01\\
340.01	0.01\\
341.01	0.01\\
342.01	0.01\\
343.01	0.01\\
344.01	0.01\\
345.01	0.01\\
346.01	0.01\\
347.01	0.01\\
348.01	0.01\\
349.01	0.01\\
350.01	0.01\\
351.01	0.01\\
352.01	0.01\\
353.01	0.01\\
354.01	0.01\\
355.01	0.01\\
356.01	0.01\\
357.01	0.01\\
358.01	0.01\\
359.01	0.01\\
360.01	0.01\\
361.01	0.01\\
362.01	0.01\\
363.01	0.01\\
364.01	0.01\\
365.01	0.01\\
366.01	0.01\\
367.01	0.01\\
368.01	0.01\\
369.01	0.01\\
370.01	0.01\\
371.01	0.01\\
372.01	0.01\\
373.01	0.01\\
374.01	0.01\\
375.01	0.01\\
376.01	0.01\\
377.01	0.01\\
378.01	0.01\\
379.01	0.01\\
380.01	0.01\\
381.01	0.01\\
382.01	0.01\\
383.01	0.01\\
384.01	0.01\\
385.01	0.01\\
386.01	0.01\\
387.01	0.01\\
388.01	0.01\\
389.01	0.01\\
390.01	0.01\\
391.01	0.01\\
392.01	0.01\\
393.01	0.01\\
394.01	0.01\\
395.01	0.01\\
396.01	0.01\\
397.01	0.01\\
398.01	0.01\\
399.01	0.01\\
400.01	0.01\\
401.01	0.01\\
402.01	0.01\\
403.01	0.01\\
404.01	0.01\\
405.01	0.01\\
406.01	0.01\\
407.01	0.01\\
408.01	0.01\\
409.01	0.01\\
410.01	0.01\\
411.01	0.01\\
412.01	0.01\\
413.01	0.01\\
414.01	0.01\\
415.01	0.01\\
416.01	0.01\\
417.01	0.01\\
418.01	0.01\\
419.01	0.01\\
420.01	0.01\\
421.01	0.01\\
422.01	0.01\\
423.01	0.01\\
424.01	0.01\\
425.01	0.01\\
426.01	0.01\\
427.01	0.01\\
428.01	0.01\\
429.01	0.01\\
430.01	0.01\\
431.01	0.01\\
432.01	0.01\\
433.01	0.01\\
434.01	0.01\\
435.01	0.01\\
436.01	0.01\\
437.01	0.01\\
438.01	0.01\\
439.01	0.01\\
440.01	0.01\\
441.01	0.01\\
442.01	0.01\\
443.01	0.01\\
444.01	0.01\\
445.01	0.01\\
446.01	0.01\\
447.01	0.01\\
448.01	0.01\\
449.01	0.01\\
450.01	0.01\\
451.01	0.01\\
452.01	0.01\\
453.01	0.01\\
454.01	0.01\\
455.01	0.01\\
456.01	0.01\\
457.01	0.01\\
458.01	0.01\\
459.01	0.01\\
460.01	0.01\\
461.01	0.01\\
462.01	0.01\\
463.01	0.01\\
464.01	0.01\\
465.01	0.01\\
466.01	0.01\\
467.01	0.01\\
468.01	0.01\\
469.01	0.01\\
470.01	0.01\\
471.01	0.01\\
472.01	0.01\\
473.01	0.01\\
474.01	0.01\\
475.01	0.01\\
476.01	0.01\\
477.01	0.01\\
478.01	0.01\\
479.01	0.01\\
480.01	0.01\\
481.01	0.01\\
482.01	0.01\\
483.01	0.01\\
484.01	0.01\\
485.01	0.01\\
486.01	0.01\\
487.01	0.01\\
488.01	0.01\\
489.01	0.01\\
490.01	0.01\\
491.01	0.01\\
492.01	0.01\\
493.01	0.01\\
494.01	0.01\\
495.01	0.01\\
496.01	0.01\\
497.01	0.01\\
498.01	0.01\\
499.01	0.01\\
500.01	0.01\\
501.01	0.01\\
502.01	0.01\\
503.01	0.01\\
504.01	0.01\\
505.01	0.01\\
506.01	0.01\\
507.01	0.01\\
508.01	0.01\\
509.01	0.01\\
510.01	0.01\\
511.01	0.01\\
512.01	0.01\\
513.01	0.01\\
514.01	0.01\\
515.01	0.01\\
516.01	0.01\\
517.01	0.01\\
518.01	0.01\\
519.01	0.01\\
520.01	0.01\\
521.01	0.01\\
522.01	0.01\\
523.01	0.01\\
524.01	0.01\\
525.01	0.01\\
526.01	0.01\\
527.01	0.01\\
528.01	0.01\\
529.01	0.01\\
530.01	0.01\\
531.01	0.01\\
532.01	0.01\\
533.01	0.01\\
534.01	0.01\\
535.01	0.01\\
536.01	0.01\\
537.01	0.01\\
538.01	0.01\\
539.01	0.01\\
540.01	0.01\\
541.01	0.01\\
542.01	0.01\\
543.01	0.01\\
544.01	0.01\\
545.01	0.01\\
546.01	0.01\\
547.01	0.01\\
548.01	0.01\\
549.01	0.01\\
550.01	0.01\\
551.01	0.01\\
552.01	0.01\\
553.01	0.01\\
554.01	0.01\\
555.01	0.01\\
556.01	0.01\\
557.01	0.01\\
558.01	0.01\\
559.01	0.01\\
560.01	0.01\\
561.01	0.01\\
562.01	0.01\\
563.01	0.01\\
564.01	0.01\\
565.01	0.01\\
566.01	0.01\\
567.01	0.01\\
568.01	0.01\\
569.01	0.01\\
570.01	0.01\\
571.01	0.01\\
572.01	0.01\\
573.01	0.01\\
574.01	0.01\\
575.01	0.01\\
576.01	0.01\\
577.01	0.01\\
578.01	0.01\\
579.01	0.01\\
580.01	0.01\\
581.01	0.01\\
582.01	0.01\\
583.01	0.01\\
584.01	0.01\\
585.01	0.01\\
586.01	0.01\\
587.01	0.01\\
588.01	0.01\\
589.01	0.01\\
590.01	0.01\\
591.01	0.01\\
592.01	0.01\\
593.01	0.01\\
594.01	0.01\\
595.01	0.01\\
596.01	0.01\\
597.01	0.01\\
598.01	0.01\\
599.01	0.01\\
599.02	0.01\\
599.03	0.01\\
599.04	0.01\\
599.05	0.01\\
599.06	0.01\\
599.07	0.01\\
599.08	0.01\\
599.09	0.01\\
599.1	0.01\\
599.11	0.01\\
599.12	0.01\\
599.13	0.01\\
599.14	0.01\\
599.15	0.01\\
599.16	0.01\\
599.17	0.01\\
599.18	0.01\\
599.19	0.01\\
599.2	0.01\\
599.21	0.01\\
599.22	0.01\\
599.23	0.01\\
599.24	0.01\\
599.25	0.01\\
599.26	0.01\\
599.27	0.01\\
599.28	0.01\\
599.29	0.01\\
599.3	0.01\\
599.31	0.01\\
599.32	0.01\\
599.33	0.01\\
599.34	0.01\\
599.35	0.01\\
599.36	0.01\\
599.37	0.01\\
599.38	0.01\\
599.39	0.01\\
599.4	0.01\\
599.41	0.01\\
599.42	0.01\\
599.43	0.01\\
599.44	0.01\\
599.45	0.01\\
599.46	0.01\\
599.47	0.01\\
599.48	0.01\\
599.49	0.01\\
599.5	0.01\\
599.51	0.01\\
599.52	0.01\\
599.53	0.01\\
599.54	0.01\\
599.55	0.01\\
599.56	0.01\\
599.57	0.01\\
599.58	0.01\\
599.59	0.01\\
599.6	0.01\\
599.61	0.01\\
599.62	0.01\\
599.63	0.01\\
599.64	0.01\\
599.65	0.01\\
599.66	0.01\\
599.67	0.01\\
599.68	0.01\\
599.69	0.01\\
599.7	0.01\\
599.71	0.01\\
599.72	0.01\\
599.73	0.01\\
599.74	0.01\\
599.75	0.01\\
599.76	0.01\\
599.77	0.01\\
599.78	0.01\\
599.79	0.01\\
599.8	0.01\\
599.81	0.01\\
599.82	0.01\\
599.83	0.01\\
599.84	0.01\\
599.85	0.01\\
599.86	0.01\\
599.87	0.01\\
599.88	0.01\\
599.89	0.01\\
599.9	0.01\\
599.91	0.01\\
599.92	0.01\\
599.93	0.01\\
599.94	0.01\\
599.95	0.01\\
599.96	0.01\\
599.97	0.01\\
599.98	0.01\\
599.99	0.01\\
600	0.01\\
};
\addplot [color=mycolor4,solid,forget plot]
  table[row sep=crcr]{%
0.01	0.01\\
1.01	0.01\\
2.01	0.01\\
3.01	0.01\\
4.01	0.01\\
5.01	0.01\\
6.01	0.01\\
7.01	0.01\\
8.01	0.01\\
9.01	0.01\\
10.01	0.01\\
11.01	0.01\\
12.01	0.01\\
13.01	0.01\\
14.01	0.01\\
15.01	0.01\\
16.01	0.01\\
17.01	0.01\\
18.01	0.01\\
19.01	0.01\\
20.01	0.01\\
21.01	0.01\\
22.01	0.01\\
23.01	0.01\\
24.01	0.01\\
25.01	0.01\\
26.01	0.01\\
27.01	0.01\\
28.01	0.01\\
29.01	0.01\\
30.01	0.01\\
31.01	0.01\\
32.01	0.01\\
33.01	0.01\\
34.01	0.01\\
35.01	0.01\\
36.01	0.01\\
37.01	0.01\\
38.01	0.01\\
39.01	0.01\\
40.01	0.01\\
41.01	0.01\\
42.01	0.01\\
43.01	0.01\\
44.01	0.01\\
45.01	0.01\\
46.01	0.01\\
47.01	0.01\\
48.01	0.01\\
49.01	0.01\\
50.01	0.01\\
51.01	0.01\\
52.01	0.01\\
53.01	0.01\\
54.01	0.01\\
55.01	0.01\\
56.01	0.01\\
57.01	0.01\\
58.01	0.01\\
59.01	0.01\\
60.01	0.01\\
61.01	0.01\\
62.01	0.01\\
63.01	0.01\\
64.01	0.01\\
65.01	0.01\\
66.01	0.01\\
67.01	0.01\\
68.01	0.01\\
69.01	0.01\\
70.01	0.01\\
71.01	0.01\\
72.01	0.01\\
73.01	0.01\\
74.01	0.01\\
75.01	0.01\\
76.01	0.01\\
77.01	0.01\\
78.01	0.01\\
79.01	0.01\\
80.01	0.01\\
81.01	0.01\\
82.01	0.01\\
83.01	0.01\\
84.01	0.01\\
85.01	0.01\\
86.01	0.01\\
87.01	0.01\\
88.01	0.01\\
89.01	0.01\\
90.01	0.01\\
91.01	0.01\\
92.01	0.01\\
93.01	0.01\\
94.01	0.01\\
95.01	0.01\\
96.01	0.01\\
97.01	0.01\\
98.01	0.01\\
99.01	0.01\\
100.01	0.01\\
101.01	0.01\\
102.01	0.01\\
103.01	0.01\\
104.01	0.01\\
105.01	0.01\\
106.01	0.01\\
107.01	0.01\\
108.01	0.01\\
109.01	0.01\\
110.01	0.01\\
111.01	0.01\\
112.01	0.01\\
113.01	0.01\\
114.01	0.01\\
115.01	0.01\\
116.01	0.01\\
117.01	0.01\\
118.01	0.01\\
119.01	0.01\\
120.01	0.01\\
121.01	0.01\\
122.01	0.01\\
123.01	0.01\\
124.01	0.01\\
125.01	0.01\\
126.01	0.01\\
127.01	0.01\\
128.01	0.01\\
129.01	0.01\\
130.01	0.01\\
131.01	0.01\\
132.01	0.01\\
133.01	0.01\\
134.01	0.01\\
135.01	0.01\\
136.01	0.01\\
137.01	0.01\\
138.01	0.01\\
139.01	0.01\\
140.01	0.01\\
141.01	0.01\\
142.01	0.01\\
143.01	0.01\\
144.01	0.01\\
145.01	0.01\\
146.01	0.01\\
147.01	0.01\\
148.01	0.01\\
149.01	0.01\\
150.01	0.01\\
151.01	0.01\\
152.01	0.01\\
153.01	0.01\\
154.01	0.01\\
155.01	0.01\\
156.01	0.01\\
157.01	0.01\\
158.01	0.01\\
159.01	0.01\\
160.01	0.01\\
161.01	0.01\\
162.01	0.01\\
163.01	0.01\\
164.01	0.01\\
165.01	0.01\\
166.01	0.01\\
167.01	0.01\\
168.01	0.01\\
169.01	0.01\\
170.01	0.01\\
171.01	0.01\\
172.01	0.01\\
173.01	0.01\\
174.01	0.01\\
175.01	0.01\\
176.01	0.01\\
177.01	0.01\\
178.01	0.01\\
179.01	0.01\\
180.01	0.01\\
181.01	0.01\\
182.01	0.01\\
183.01	0.01\\
184.01	0.01\\
185.01	0.01\\
186.01	0.01\\
187.01	0.01\\
188.01	0.01\\
189.01	0.01\\
190.01	0.01\\
191.01	0.01\\
192.01	0.01\\
193.01	0.01\\
194.01	0.01\\
195.01	0.01\\
196.01	0.01\\
197.01	0.01\\
198.01	0.01\\
199.01	0.01\\
200.01	0.01\\
201.01	0.01\\
202.01	0.01\\
203.01	0.01\\
204.01	0.01\\
205.01	0.01\\
206.01	0.01\\
207.01	0.01\\
208.01	0.01\\
209.01	0.01\\
210.01	0.01\\
211.01	0.01\\
212.01	0.01\\
213.01	0.01\\
214.01	0.01\\
215.01	0.01\\
216.01	0.01\\
217.01	0.01\\
218.01	0.01\\
219.01	0.01\\
220.01	0.01\\
221.01	0.01\\
222.01	0.01\\
223.01	0.01\\
224.01	0.01\\
225.01	0.01\\
226.01	0.01\\
227.01	0.01\\
228.01	0.01\\
229.01	0.01\\
230.01	0.01\\
231.01	0.01\\
232.01	0.01\\
233.01	0.01\\
234.01	0.01\\
235.01	0.01\\
236.01	0.01\\
237.01	0.01\\
238.01	0.01\\
239.01	0.01\\
240.01	0.01\\
241.01	0.01\\
242.01	0.01\\
243.01	0.01\\
244.01	0.01\\
245.01	0.01\\
246.01	0.01\\
247.01	0.01\\
248.01	0.01\\
249.01	0.01\\
250.01	0.01\\
251.01	0.01\\
252.01	0.01\\
253.01	0.01\\
254.01	0.01\\
255.01	0.01\\
256.01	0.01\\
257.01	0.01\\
258.01	0.01\\
259.01	0.01\\
260.01	0.01\\
261.01	0.01\\
262.01	0.01\\
263.01	0.01\\
264.01	0.01\\
265.01	0.01\\
266.01	0.01\\
267.01	0.01\\
268.01	0.01\\
269.01	0.01\\
270.01	0.01\\
271.01	0.01\\
272.01	0.01\\
273.01	0.01\\
274.01	0.01\\
275.01	0.01\\
276.01	0.01\\
277.01	0.01\\
278.01	0.01\\
279.01	0.01\\
280.01	0.01\\
281.01	0.01\\
282.01	0.01\\
283.01	0.01\\
284.01	0.01\\
285.01	0.01\\
286.01	0.01\\
287.01	0.01\\
288.01	0.01\\
289.01	0.01\\
290.01	0.01\\
291.01	0.01\\
292.01	0.01\\
293.01	0.01\\
294.01	0.01\\
295.01	0.01\\
296.01	0.01\\
297.01	0.01\\
298.01	0.01\\
299.01	0.01\\
300.01	0.01\\
301.01	0.01\\
302.01	0.01\\
303.01	0.01\\
304.01	0.01\\
305.01	0.01\\
306.01	0.01\\
307.01	0.01\\
308.01	0.01\\
309.01	0.01\\
310.01	0.01\\
311.01	0.01\\
312.01	0.01\\
313.01	0.01\\
314.01	0.01\\
315.01	0.01\\
316.01	0.01\\
317.01	0.01\\
318.01	0.01\\
319.01	0.01\\
320.01	0.01\\
321.01	0.01\\
322.01	0.01\\
323.01	0.01\\
324.01	0.01\\
325.01	0.01\\
326.01	0.01\\
327.01	0.01\\
328.01	0.01\\
329.01	0.01\\
330.01	0.01\\
331.01	0.01\\
332.01	0.01\\
333.01	0.01\\
334.01	0.01\\
335.01	0.01\\
336.01	0.01\\
337.01	0.01\\
338.01	0.01\\
339.01	0.01\\
340.01	0.01\\
341.01	0.01\\
342.01	0.01\\
343.01	0.01\\
344.01	0.01\\
345.01	0.01\\
346.01	0.01\\
347.01	0.01\\
348.01	0.01\\
349.01	0.01\\
350.01	0.01\\
351.01	0.01\\
352.01	0.01\\
353.01	0.01\\
354.01	0.01\\
355.01	0.01\\
356.01	0.01\\
357.01	0.01\\
358.01	0.01\\
359.01	0.01\\
360.01	0.01\\
361.01	0.01\\
362.01	0.01\\
363.01	0.01\\
364.01	0.01\\
365.01	0.01\\
366.01	0.01\\
367.01	0.01\\
368.01	0.01\\
369.01	0.01\\
370.01	0.01\\
371.01	0.01\\
372.01	0.01\\
373.01	0.01\\
374.01	0.01\\
375.01	0.01\\
376.01	0.01\\
377.01	0.01\\
378.01	0.01\\
379.01	0.01\\
380.01	0.01\\
381.01	0.01\\
382.01	0.01\\
383.01	0.01\\
384.01	0.01\\
385.01	0.01\\
386.01	0.01\\
387.01	0.01\\
388.01	0.01\\
389.01	0.01\\
390.01	0.01\\
391.01	0.01\\
392.01	0.01\\
393.01	0.01\\
394.01	0.01\\
395.01	0.01\\
396.01	0.01\\
397.01	0.01\\
398.01	0.01\\
399.01	0.01\\
400.01	0.01\\
401.01	0.01\\
402.01	0.01\\
403.01	0.01\\
404.01	0.01\\
405.01	0.01\\
406.01	0.01\\
407.01	0.01\\
408.01	0.01\\
409.01	0.01\\
410.01	0.01\\
411.01	0.01\\
412.01	0.01\\
413.01	0.01\\
414.01	0.01\\
415.01	0.01\\
416.01	0.01\\
417.01	0.01\\
418.01	0.01\\
419.01	0.01\\
420.01	0.01\\
421.01	0.01\\
422.01	0.01\\
423.01	0.01\\
424.01	0.01\\
425.01	0.01\\
426.01	0.01\\
427.01	0.01\\
428.01	0.01\\
429.01	0.01\\
430.01	0.01\\
431.01	0.01\\
432.01	0.01\\
433.01	0.01\\
434.01	0.01\\
435.01	0.01\\
436.01	0.01\\
437.01	0.01\\
438.01	0.01\\
439.01	0.01\\
440.01	0.01\\
441.01	0.01\\
442.01	0.01\\
443.01	0.01\\
444.01	0.01\\
445.01	0.01\\
446.01	0.01\\
447.01	0.01\\
448.01	0.01\\
449.01	0.01\\
450.01	0.01\\
451.01	0.01\\
452.01	0.01\\
453.01	0.01\\
454.01	0.01\\
455.01	0.01\\
456.01	0.01\\
457.01	0.01\\
458.01	0.01\\
459.01	0.01\\
460.01	0.01\\
461.01	0.01\\
462.01	0.01\\
463.01	0.01\\
464.01	0.01\\
465.01	0.01\\
466.01	0.01\\
467.01	0.01\\
468.01	0.01\\
469.01	0.01\\
470.01	0.01\\
471.01	0.01\\
472.01	0.01\\
473.01	0.01\\
474.01	0.01\\
475.01	0.01\\
476.01	0.01\\
477.01	0.01\\
478.01	0.01\\
479.01	0.01\\
480.01	0.01\\
481.01	0.01\\
482.01	0.01\\
483.01	0.01\\
484.01	0.01\\
485.01	0.01\\
486.01	0.01\\
487.01	0.01\\
488.01	0.01\\
489.01	0.01\\
490.01	0.01\\
491.01	0.01\\
492.01	0.01\\
493.01	0.01\\
494.01	0.01\\
495.01	0.01\\
496.01	0.01\\
497.01	0.01\\
498.01	0.01\\
499.01	0.01\\
500.01	0.01\\
501.01	0.01\\
502.01	0.01\\
503.01	0.01\\
504.01	0.01\\
505.01	0.01\\
506.01	0.01\\
507.01	0.01\\
508.01	0.01\\
509.01	0.01\\
510.01	0.01\\
511.01	0.01\\
512.01	0.01\\
513.01	0.01\\
514.01	0.01\\
515.01	0.01\\
516.01	0.01\\
517.01	0.01\\
518.01	0.01\\
519.01	0.01\\
520.01	0.01\\
521.01	0.01\\
522.01	0.01\\
523.01	0.01\\
524.01	0.01\\
525.01	0.01\\
526.01	0.01\\
527.01	0.01\\
528.01	0.01\\
529.01	0.01\\
530.01	0.01\\
531.01	0.01\\
532.01	0.01\\
533.01	0.01\\
534.01	0.01\\
535.01	0.01\\
536.01	0.01\\
537.01	0.01\\
538.01	0.01\\
539.01	0.01\\
540.01	0.01\\
541.01	0.01\\
542.01	0.01\\
543.01	0.01\\
544.01	0.01\\
545.01	0.01\\
546.01	0.01\\
547.01	0.01\\
548.01	0.01\\
549.01	0.01\\
550.01	0.01\\
551.01	0.01\\
552.01	0.01\\
553.01	0.01\\
554.01	0.01\\
555.01	0.01\\
556.01	0.01\\
557.01	0.01\\
558.01	0.01\\
559.01	0.01\\
560.01	0.01\\
561.01	0.01\\
562.01	0.01\\
563.01	0.01\\
564.01	0.01\\
565.01	0.01\\
566.01	0.01\\
567.01	0.01\\
568.01	0.01\\
569.01	0.01\\
570.01	0.01\\
571.01	0.01\\
572.01	0.01\\
573.01	0.01\\
574.01	0.01\\
575.01	0.01\\
576.01	0.01\\
577.01	0.01\\
578.01	0.01\\
579.01	0.01\\
580.01	0.01\\
581.01	0.01\\
582.01	0.01\\
583.01	0.01\\
584.01	0.01\\
585.01	0.01\\
586.01	0.01\\
587.01	0.01\\
588.01	0.01\\
589.01	0.01\\
590.01	0.01\\
591.01	0.01\\
592.01	0.01\\
593.01	0.01\\
594.01	0.01\\
595.01	0.01\\
596.01	0.01\\
597.01	0.01\\
598.01	0.01\\
599.01	0.01\\
599.02	0.01\\
599.03	0.01\\
599.04	0.01\\
599.05	0.01\\
599.06	0.01\\
599.07	0.01\\
599.08	0.01\\
599.09	0.01\\
599.1	0.01\\
599.11	0.01\\
599.12	0.01\\
599.13	0.01\\
599.14	0.01\\
599.15	0.01\\
599.16	0.01\\
599.17	0.01\\
599.18	0.01\\
599.19	0.01\\
599.2	0.01\\
599.21	0.01\\
599.22	0.01\\
599.23	0.01\\
599.24	0.01\\
599.25	0.01\\
599.26	0.01\\
599.27	0.01\\
599.28	0.01\\
599.29	0.01\\
599.3	0.01\\
599.31	0.01\\
599.32	0.01\\
599.33	0.01\\
599.34	0.01\\
599.35	0.01\\
599.36	0.01\\
599.37	0.01\\
599.38	0.01\\
599.39	0.01\\
599.4	0.01\\
599.41	0.01\\
599.42	0.01\\
599.43	0.01\\
599.44	0.01\\
599.45	0.01\\
599.46	0.01\\
599.47	0.01\\
599.48	0.01\\
599.49	0.01\\
599.5	0.01\\
599.51	0.01\\
599.52	0.01\\
599.53	0.01\\
599.54	0.01\\
599.55	0.01\\
599.56	0.01\\
599.57	0.01\\
599.58	0.01\\
599.59	0.01\\
599.6	0.01\\
599.61	0.01\\
599.62	0.01\\
599.63	0.01\\
599.64	0.01\\
599.65	0.01\\
599.66	0.01\\
599.67	0.01\\
599.68	0.01\\
599.69	0.01\\
599.7	0.01\\
599.71	0.01\\
599.72	0.01\\
599.73	0.01\\
599.74	0.01\\
599.75	0.01\\
599.76	0.01\\
599.77	0.01\\
599.78	0.01\\
599.79	0.01\\
599.8	0.01\\
599.81	0.01\\
599.82	0.01\\
599.83	0.01\\
599.84	0.01\\
599.85	0.01\\
599.86	0.01\\
599.87	0.01\\
599.88	0.01\\
599.89	0.01\\
599.9	0.01\\
599.91	0.01\\
599.92	0.01\\
599.93	0.01\\
599.94	0.01\\
599.95	0.01\\
599.96	0.01\\
599.97	0.01\\
599.98	0.01\\
599.99	0.01\\
600	0.01\\
};
\addplot [color=mycolor5,solid,forget plot]
  table[row sep=crcr]{%
0.01	0.01\\
1.01	0.01\\
2.01	0.01\\
3.01	0.01\\
4.01	0.01\\
5.01	0.01\\
6.01	0.01\\
7.01	0.01\\
8.01	0.01\\
9.01	0.01\\
10.01	0.01\\
11.01	0.01\\
12.01	0.01\\
13.01	0.01\\
14.01	0.01\\
15.01	0.01\\
16.01	0.01\\
17.01	0.01\\
18.01	0.01\\
19.01	0.01\\
20.01	0.01\\
21.01	0.01\\
22.01	0.01\\
23.01	0.01\\
24.01	0.01\\
25.01	0.01\\
26.01	0.01\\
27.01	0.01\\
28.01	0.01\\
29.01	0.01\\
30.01	0.01\\
31.01	0.01\\
32.01	0.01\\
33.01	0.01\\
34.01	0.01\\
35.01	0.01\\
36.01	0.01\\
37.01	0.01\\
38.01	0.01\\
39.01	0.01\\
40.01	0.01\\
41.01	0.01\\
42.01	0.01\\
43.01	0.01\\
44.01	0.01\\
45.01	0.01\\
46.01	0.01\\
47.01	0.01\\
48.01	0.01\\
49.01	0.01\\
50.01	0.01\\
51.01	0.01\\
52.01	0.01\\
53.01	0.01\\
54.01	0.01\\
55.01	0.01\\
56.01	0.01\\
57.01	0.01\\
58.01	0.01\\
59.01	0.01\\
60.01	0.01\\
61.01	0.01\\
62.01	0.01\\
63.01	0.01\\
64.01	0.01\\
65.01	0.01\\
66.01	0.01\\
67.01	0.01\\
68.01	0.01\\
69.01	0.01\\
70.01	0.01\\
71.01	0.01\\
72.01	0.01\\
73.01	0.01\\
74.01	0.01\\
75.01	0.01\\
76.01	0.01\\
77.01	0.01\\
78.01	0.01\\
79.01	0.01\\
80.01	0.01\\
81.01	0.01\\
82.01	0.01\\
83.01	0.01\\
84.01	0.01\\
85.01	0.01\\
86.01	0.01\\
87.01	0.01\\
88.01	0.01\\
89.01	0.01\\
90.01	0.01\\
91.01	0.01\\
92.01	0.01\\
93.01	0.01\\
94.01	0.01\\
95.01	0.01\\
96.01	0.01\\
97.01	0.01\\
98.01	0.01\\
99.01	0.01\\
100.01	0.01\\
101.01	0.01\\
102.01	0.01\\
103.01	0.01\\
104.01	0.01\\
105.01	0.01\\
106.01	0.01\\
107.01	0.01\\
108.01	0.01\\
109.01	0.01\\
110.01	0.01\\
111.01	0.01\\
112.01	0.01\\
113.01	0.01\\
114.01	0.01\\
115.01	0.01\\
116.01	0.01\\
117.01	0.01\\
118.01	0.01\\
119.01	0.01\\
120.01	0.01\\
121.01	0.01\\
122.01	0.01\\
123.01	0.01\\
124.01	0.01\\
125.01	0.01\\
126.01	0.01\\
127.01	0.01\\
128.01	0.01\\
129.01	0.01\\
130.01	0.01\\
131.01	0.01\\
132.01	0.01\\
133.01	0.01\\
134.01	0.01\\
135.01	0.01\\
136.01	0.01\\
137.01	0.01\\
138.01	0.01\\
139.01	0.01\\
140.01	0.01\\
141.01	0.01\\
142.01	0.01\\
143.01	0.01\\
144.01	0.01\\
145.01	0.01\\
146.01	0.01\\
147.01	0.01\\
148.01	0.01\\
149.01	0.01\\
150.01	0.01\\
151.01	0.01\\
152.01	0.01\\
153.01	0.01\\
154.01	0.01\\
155.01	0.01\\
156.01	0.01\\
157.01	0.01\\
158.01	0.01\\
159.01	0.01\\
160.01	0.01\\
161.01	0.01\\
162.01	0.01\\
163.01	0.01\\
164.01	0.01\\
165.01	0.01\\
166.01	0.01\\
167.01	0.01\\
168.01	0.01\\
169.01	0.01\\
170.01	0.01\\
171.01	0.01\\
172.01	0.01\\
173.01	0.01\\
174.01	0.01\\
175.01	0.01\\
176.01	0.01\\
177.01	0.01\\
178.01	0.01\\
179.01	0.01\\
180.01	0.01\\
181.01	0.01\\
182.01	0.01\\
183.01	0.01\\
184.01	0.01\\
185.01	0.01\\
186.01	0.01\\
187.01	0.01\\
188.01	0.01\\
189.01	0.01\\
190.01	0.01\\
191.01	0.01\\
192.01	0.01\\
193.01	0.01\\
194.01	0.01\\
195.01	0.01\\
196.01	0.01\\
197.01	0.01\\
198.01	0.01\\
199.01	0.01\\
200.01	0.01\\
201.01	0.01\\
202.01	0.01\\
203.01	0.01\\
204.01	0.01\\
205.01	0.01\\
206.01	0.01\\
207.01	0.01\\
208.01	0.01\\
209.01	0.01\\
210.01	0.01\\
211.01	0.01\\
212.01	0.01\\
213.01	0.01\\
214.01	0.01\\
215.01	0.01\\
216.01	0.01\\
217.01	0.01\\
218.01	0.01\\
219.01	0.01\\
220.01	0.01\\
221.01	0.01\\
222.01	0.01\\
223.01	0.01\\
224.01	0.01\\
225.01	0.01\\
226.01	0.01\\
227.01	0.01\\
228.01	0.01\\
229.01	0.01\\
230.01	0.01\\
231.01	0.01\\
232.01	0.01\\
233.01	0.01\\
234.01	0.01\\
235.01	0.01\\
236.01	0.01\\
237.01	0.01\\
238.01	0.01\\
239.01	0.01\\
240.01	0.01\\
241.01	0.01\\
242.01	0.01\\
243.01	0.01\\
244.01	0.01\\
245.01	0.01\\
246.01	0.01\\
247.01	0.01\\
248.01	0.01\\
249.01	0.01\\
250.01	0.01\\
251.01	0.01\\
252.01	0.01\\
253.01	0.01\\
254.01	0.01\\
255.01	0.01\\
256.01	0.01\\
257.01	0.01\\
258.01	0.01\\
259.01	0.01\\
260.01	0.01\\
261.01	0.01\\
262.01	0.01\\
263.01	0.01\\
264.01	0.01\\
265.01	0.01\\
266.01	0.01\\
267.01	0.01\\
268.01	0.01\\
269.01	0.01\\
270.01	0.01\\
271.01	0.01\\
272.01	0.01\\
273.01	0.01\\
274.01	0.01\\
275.01	0.01\\
276.01	0.01\\
277.01	0.01\\
278.01	0.01\\
279.01	0.01\\
280.01	0.01\\
281.01	0.01\\
282.01	0.01\\
283.01	0.01\\
284.01	0.01\\
285.01	0.01\\
286.01	0.01\\
287.01	0.01\\
288.01	0.01\\
289.01	0.01\\
290.01	0.01\\
291.01	0.01\\
292.01	0.01\\
293.01	0.01\\
294.01	0.01\\
295.01	0.01\\
296.01	0.01\\
297.01	0.01\\
298.01	0.01\\
299.01	0.01\\
300.01	0.01\\
301.01	0.01\\
302.01	0.01\\
303.01	0.01\\
304.01	0.01\\
305.01	0.01\\
306.01	0.01\\
307.01	0.01\\
308.01	0.01\\
309.01	0.01\\
310.01	0.01\\
311.01	0.01\\
312.01	0.01\\
313.01	0.01\\
314.01	0.01\\
315.01	0.01\\
316.01	0.01\\
317.01	0.01\\
318.01	0.01\\
319.01	0.01\\
320.01	0.01\\
321.01	0.01\\
322.01	0.01\\
323.01	0.01\\
324.01	0.01\\
325.01	0.01\\
326.01	0.01\\
327.01	0.01\\
328.01	0.01\\
329.01	0.01\\
330.01	0.01\\
331.01	0.01\\
332.01	0.01\\
333.01	0.01\\
334.01	0.01\\
335.01	0.01\\
336.01	0.01\\
337.01	0.01\\
338.01	0.01\\
339.01	0.01\\
340.01	0.01\\
341.01	0.01\\
342.01	0.01\\
343.01	0.01\\
344.01	0.01\\
345.01	0.01\\
346.01	0.01\\
347.01	0.01\\
348.01	0.01\\
349.01	0.01\\
350.01	0.01\\
351.01	0.01\\
352.01	0.01\\
353.01	0.01\\
354.01	0.01\\
355.01	0.01\\
356.01	0.01\\
357.01	0.01\\
358.01	0.01\\
359.01	0.01\\
360.01	0.01\\
361.01	0.01\\
362.01	0.01\\
363.01	0.01\\
364.01	0.01\\
365.01	0.01\\
366.01	0.01\\
367.01	0.01\\
368.01	0.01\\
369.01	0.01\\
370.01	0.01\\
371.01	0.01\\
372.01	0.01\\
373.01	0.01\\
374.01	0.01\\
375.01	0.01\\
376.01	0.01\\
377.01	0.01\\
378.01	0.01\\
379.01	0.01\\
380.01	0.01\\
381.01	0.01\\
382.01	0.01\\
383.01	0.01\\
384.01	0.01\\
385.01	0.01\\
386.01	0.01\\
387.01	0.01\\
388.01	0.01\\
389.01	0.01\\
390.01	0.01\\
391.01	0.01\\
392.01	0.01\\
393.01	0.01\\
394.01	0.01\\
395.01	0.01\\
396.01	0.01\\
397.01	0.01\\
398.01	0.01\\
399.01	0.01\\
400.01	0.01\\
401.01	0.01\\
402.01	0.01\\
403.01	0.01\\
404.01	0.01\\
405.01	0.01\\
406.01	0.01\\
407.01	0.01\\
408.01	0.01\\
409.01	0.01\\
410.01	0.01\\
411.01	0.01\\
412.01	0.01\\
413.01	0.01\\
414.01	0.01\\
415.01	0.01\\
416.01	0.01\\
417.01	0.01\\
418.01	0.01\\
419.01	0.01\\
420.01	0.01\\
421.01	0.01\\
422.01	0.01\\
423.01	0.01\\
424.01	0.01\\
425.01	0.01\\
426.01	0.01\\
427.01	0.01\\
428.01	0.01\\
429.01	0.01\\
430.01	0.01\\
431.01	0.01\\
432.01	0.01\\
433.01	0.01\\
434.01	0.01\\
435.01	0.01\\
436.01	0.01\\
437.01	0.01\\
438.01	0.01\\
439.01	0.01\\
440.01	0.01\\
441.01	0.01\\
442.01	0.01\\
443.01	0.01\\
444.01	0.01\\
445.01	0.01\\
446.01	0.01\\
447.01	0.01\\
448.01	0.01\\
449.01	0.01\\
450.01	0.01\\
451.01	0.01\\
452.01	0.01\\
453.01	0.01\\
454.01	0.01\\
455.01	0.01\\
456.01	0.01\\
457.01	0.01\\
458.01	0.01\\
459.01	0.01\\
460.01	0.01\\
461.01	0.01\\
462.01	0.01\\
463.01	0.01\\
464.01	0.01\\
465.01	0.01\\
466.01	0.01\\
467.01	0.01\\
468.01	0.01\\
469.01	0.01\\
470.01	0.01\\
471.01	0.01\\
472.01	0.01\\
473.01	0.01\\
474.01	0.01\\
475.01	0.01\\
476.01	0.01\\
477.01	0.01\\
478.01	0.01\\
479.01	0.01\\
480.01	0.01\\
481.01	0.01\\
482.01	0.01\\
483.01	0.01\\
484.01	0.01\\
485.01	0.01\\
486.01	0.01\\
487.01	0.01\\
488.01	0.01\\
489.01	0.01\\
490.01	0.01\\
491.01	0.01\\
492.01	0.01\\
493.01	0.01\\
494.01	0.01\\
495.01	0.01\\
496.01	0.01\\
497.01	0.01\\
498.01	0.01\\
499.01	0.01\\
500.01	0.01\\
501.01	0.01\\
502.01	0.01\\
503.01	0.01\\
504.01	0.01\\
505.01	0.01\\
506.01	0.01\\
507.01	0.01\\
508.01	0.01\\
509.01	0.01\\
510.01	0.01\\
511.01	0.01\\
512.01	0.01\\
513.01	0.01\\
514.01	0.01\\
515.01	0.01\\
516.01	0.01\\
517.01	0.01\\
518.01	0.01\\
519.01	0.01\\
520.01	0.01\\
521.01	0.01\\
522.01	0.01\\
523.01	0.01\\
524.01	0.01\\
525.01	0.01\\
526.01	0.01\\
527.01	0.01\\
528.01	0.01\\
529.01	0.01\\
530.01	0.01\\
531.01	0.01\\
532.01	0.01\\
533.01	0.01\\
534.01	0.01\\
535.01	0.01\\
536.01	0.01\\
537.01	0.01\\
538.01	0.01\\
539.01	0.01\\
540.01	0.01\\
541.01	0.01\\
542.01	0.01\\
543.01	0.01\\
544.01	0.01\\
545.01	0.01\\
546.01	0.01\\
547.01	0.01\\
548.01	0.01\\
549.01	0.01\\
550.01	0.01\\
551.01	0.01\\
552.01	0.01\\
553.01	0.01\\
554.01	0.01\\
555.01	0.01\\
556.01	0.01\\
557.01	0.01\\
558.01	0.01\\
559.01	0.01\\
560.01	0.01\\
561.01	0.01\\
562.01	0.01\\
563.01	0.01\\
564.01	0.01\\
565.01	0.01\\
566.01	0.01\\
567.01	0.01\\
568.01	0.01\\
569.01	0.01\\
570.01	0.01\\
571.01	0.01\\
572.01	0.01\\
573.01	0.01\\
574.01	0.01\\
575.01	0.01\\
576.01	0.01\\
577.01	0.01\\
578.01	0.01\\
579.01	0.01\\
580.01	0.01\\
581.01	0.01\\
582.01	0.01\\
583.01	0.01\\
584.01	0.01\\
585.01	0.01\\
586.01	0.01\\
587.01	0.01\\
588.01	0.01\\
589.01	0.01\\
590.01	0.01\\
591.01	0.01\\
592.01	0.01\\
593.01	0.01\\
594.01	0.01\\
595.01	0.01\\
596.01	0.01\\
597.01	0.01\\
598.01	0.01\\
599.01	0.01\\
599.02	0.01\\
599.03	0.01\\
599.04	0.01\\
599.05	0.01\\
599.06	0.01\\
599.07	0.01\\
599.08	0.01\\
599.09	0.01\\
599.1	0.01\\
599.11	0.01\\
599.12	0.01\\
599.13	0.01\\
599.14	0.01\\
599.15	0.01\\
599.16	0.01\\
599.17	0.01\\
599.18	0.01\\
599.19	0.01\\
599.2	0.01\\
599.21	0.01\\
599.22	0.01\\
599.23	0.01\\
599.24	0.01\\
599.25	0.01\\
599.26	0.01\\
599.27	0.01\\
599.28	0.01\\
599.29	0.01\\
599.3	0.01\\
599.31	0.01\\
599.32	0.01\\
599.33	0.01\\
599.34	0.01\\
599.35	0.01\\
599.36	0.01\\
599.37	0.01\\
599.38	0.01\\
599.39	0.01\\
599.4	0.01\\
599.41	0.01\\
599.42	0.01\\
599.43	0.01\\
599.44	0.01\\
599.45	0.01\\
599.46	0.01\\
599.47	0.01\\
599.48	0.01\\
599.49	0.01\\
599.5	0.01\\
599.51	0.01\\
599.52	0.01\\
599.53	0.01\\
599.54	0.01\\
599.55	0.01\\
599.56	0.01\\
599.57	0.01\\
599.58	0.01\\
599.59	0.01\\
599.6	0.01\\
599.61	0.01\\
599.62	0.01\\
599.63	0.01\\
599.64	0.01\\
599.65	0.01\\
599.66	0.01\\
599.67	0.01\\
599.68	0.01\\
599.69	0.01\\
599.7	0.01\\
599.71	0.01\\
599.72	0.01\\
599.73	0.01\\
599.74	0.01\\
599.75	0.01\\
599.76	0.01\\
599.77	0.01\\
599.78	0.01\\
599.79	0.01\\
599.8	0.01\\
599.81	0.01\\
599.82	0.01\\
599.83	0.01\\
599.84	0.01\\
599.85	0.01\\
599.86	0.01\\
599.87	0.01\\
599.88	0.01\\
599.89	0.01\\
599.9	0.01\\
599.91	0.01\\
599.92	0.01\\
599.93	0.01\\
599.94	0.01\\
599.95	0.01\\
599.96	0.01\\
599.97	0.01\\
599.98	0.01\\
599.99	0.01\\
600	0.01\\
};
\addplot [color=mycolor6,solid,forget plot]
  table[row sep=crcr]{%
0.01	0.01\\
1.01	0.01\\
2.01	0.01\\
3.01	0.01\\
4.01	0.01\\
5.01	0.01\\
6.01	0.01\\
7.01	0.01\\
8.01	0.01\\
9.01	0.01\\
10.01	0.01\\
11.01	0.01\\
12.01	0.01\\
13.01	0.01\\
14.01	0.01\\
15.01	0.01\\
16.01	0.01\\
17.01	0.01\\
18.01	0.01\\
19.01	0.01\\
20.01	0.01\\
21.01	0.01\\
22.01	0.01\\
23.01	0.01\\
24.01	0.01\\
25.01	0.01\\
26.01	0.01\\
27.01	0.01\\
28.01	0.01\\
29.01	0.01\\
30.01	0.01\\
31.01	0.01\\
32.01	0.01\\
33.01	0.01\\
34.01	0.01\\
35.01	0.01\\
36.01	0.01\\
37.01	0.01\\
38.01	0.01\\
39.01	0.01\\
40.01	0.01\\
41.01	0.01\\
42.01	0.01\\
43.01	0.01\\
44.01	0.01\\
45.01	0.01\\
46.01	0.01\\
47.01	0.01\\
48.01	0.01\\
49.01	0.01\\
50.01	0.01\\
51.01	0.01\\
52.01	0.01\\
53.01	0.01\\
54.01	0.01\\
55.01	0.01\\
56.01	0.01\\
57.01	0.01\\
58.01	0.01\\
59.01	0.01\\
60.01	0.01\\
61.01	0.01\\
62.01	0.01\\
63.01	0.01\\
64.01	0.01\\
65.01	0.01\\
66.01	0.01\\
67.01	0.01\\
68.01	0.01\\
69.01	0.01\\
70.01	0.01\\
71.01	0.01\\
72.01	0.01\\
73.01	0.01\\
74.01	0.01\\
75.01	0.01\\
76.01	0.01\\
77.01	0.01\\
78.01	0.01\\
79.01	0.01\\
80.01	0.01\\
81.01	0.01\\
82.01	0.01\\
83.01	0.01\\
84.01	0.01\\
85.01	0.01\\
86.01	0.01\\
87.01	0.01\\
88.01	0.01\\
89.01	0.01\\
90.01	0.01\\
91.01	0.01\\
92.01	0.01\\
93.01	0.01\\
94.01	0.01\\
95.01	0.01\\
96.01	0.01\\
97.01	0.01\\
98.01	0.01\\
99.01	0.01\\
100.01	0.01\\
101.01	0.01\\
102.01	0.01\\
103.01	0.01\\
104.01	0.01\\
105.01	0.01\\
106.01	0.01\\
107.01	0.01\\
108.01	0.01\\
109.01	0.01\\
110.01	0.01\\
111.01	0.01\\
112.01	0.01\\
113.01	0.01\\
114.01	0.01\\
115.01	0.01\\
116.01	0.01\\
117.01	0.01\\
118.01	0.01\\
119.01	0.01\\
120.01	0.01\\
121.01	0.01\\
122.01	0.01\\
123.01	0.01\\
124.01	0.01\\
125.01	0.01\\
126.01	0.01\\
127.01	0.01\\
128.01	0.01\\
129.01	0.01\\
130.01	0.01\\
131.01	0.01\\
132.01	0.01\\
133.01	0.01\\
134.01	0.01\\
135.01	0.01\\
136.01	0.01\\
137.01	0.01\\
138.01	0.01\\
139.01	0.01\\
140.01	0.01\\
141.01	0.01\\
142.01	0.01\\
143.01	0.01\\
144.01	0.01\\
145.01	0.01\\
146.01	0.01\\
147.01	0.01\\
148.01	0.01\\
149.01	0.01\\
150.01	0.01\\
151.01	0.01\\
152.01	0.01\\
153.01	0.01\\
154.01	0.01\\
155.01	0.01\\
156.01	0.01\\
157.01	0.01\\
158.01	0.01\\
159.01	0.01\\
160.01	0.01\\
161.01	0.01\\
162.01	0.01\\
163.01	0.01\\
164.01	0.01\\
165.01	0.01\\
166.01	0.01\\
167.01	0.01\\
168.01	0.01\\
169.01	0.01\\
170.01	0.01\\
171.01	0.01\\
172.01	0.01\\
173.01	0.01\\
174.01	0.01\\
175.01	0.01\\
176.01	0.01\\
177.01	0.01\\
178.01	0.01\\
179.01	0.01\\
180.01	0.01\\
181.01	0.01\\
182.01	0.01\\
183.01	0.01\\
184.01	0.01\\
185.01	0.01\\
186.01	0.01\\
187.01	0.01\\
188.01	0.01\\
189.01	0.01\\
190.01	0.01\\
191.01	0.01\\
192.01	0.01\\
193.01	0.01\\
194.01	0.01\\
195.01	0.01\\
196.01	0.01\\
197.01	0.01\\
198.01	0.01\\
199.01	0.01\\
200.01	0.01\\
201.01	0.01\\
202.01	0.01\\
203.01	0.01\\
204.01	0.01\\
205.01	0.01\\
206.01	0.01\\
207.01	0.01\\
208.01	0.01\\
209.01	0.01\\
210.01	0.01\\
211.01	0.01\\
212.01	0.01\\
213.01	0.01\\
214.01	0.01\\
215.01	0.01\\
216.01	0.01\\
217.01	0.01\\
218.01	0.01\\
219.01	0.01\\
220.01	0.01\\
221.01	0.01\\
222.01	0.01\\
223.01	0.01\\
224.01	0.01\\
225.01	0.01\\
226.01	0.01\\
227.01	0.01\\
228.01	0.01\\
229.01	0.01\\
230.01	0.01\\
231.01	0.01\\
232.01	0.01\\
233.01	0.01\\
234.01	0.01\\
235.01	0.01\\
236.01	0.01\\
237.01	0.01\\
238.01	0.01\\
239.01	0.01\\
240.01	0.01\\
241.01	0.01\\
242.01	0.01\\
243.01	0.01\\
244.01	0.01\\
245.01	0.01\\
246.01	0.01\\
247.01	0.01\\
248.01	0.01\\
249.01	0.01\\
250.01	0.01\\
251.01	0.01\\
252.01	0.01\\
253.01	0.01\\
254.01	0.01\\
255.01	0.01\\
256.01	0.01\\
257.01	0.01\\
258.01	0.01\\
259.01	0.01\\
260.01	0.01\\
261.01	0.01\\
262.01	0.01\\
263.01	0.01\\
264.01	0.01\\
265.01	0.01\\
266.01	0.01\\
267.01	0.01\\
268.01	0.01\\
269.01	0.01\\
270.01	0.01\\
271.01	0.01\\
272.01	0.01\\
273.01	0.01\\
274.01	0.01\\
275.01	0.01\\
276.01	0.01\\
277.01	0.01\\
278.01	0.01\\
279.01	0.01\\
280.01	0.01\\
281.01	0.01\\
282.01	0.01\\
283.01	0.01\\
284.01	0.01\\
285.01	0.01\\
286.01	0.01\\
287.01	0.01\\
288.01	0.01\\
289.01	0.01\\
290.01	0.01\\
291.01	0.01\\
292.01	0.01\\
293.01	0.01\\
294.01	0.01\\
295.01	0.01\\
296.01	0.01\\
297.01	0.01\\
298.01	0.01\\
299.01	0.01\\
300.01	0.01\\
301.01	0.01\\
302.01	0.01\\
303.01	0.01\\
304.01	0.01\\
305.01	0.01\\
306.01	0.01\\
307.01	0.01\\
308.01	0.01\\
309.01	0.01\\
310.01	0.01\\
311.01	0.01\\
312.01	0.01\\
313.01	0.01\\
314.01	0.01\\
315.01	0.01\\
316.01	0.01\\
317.01	0.01\\
318.01	0.01\\
319.01	0.01\\
320.01	0.01\\
321.01	0.01\\
322.01	0.01\\
323.01	0.01\\
324.01	0.01\\
325.01	0.01\\
326.01	0.01\\
327.01	0.01\\
328.01	0.01\\
329.01	0.01\\
330.01	0.01\\
331.01	0.01\\
332.01	0.01\\
333.01	0.01\\
334.01	0.01\\
335.01	0.01\\
336.01	0.01\\
337.01	0.01\\
338.01	0.01\\
339.01	0.01\\
340.01	0.01\\
341.01	0.01\\
342.01	0.01\\
343.01	0.01\\
344.01	0.01\\
345.01	0.01\\
346.01	0.01\\
347.01	0.01\\
348.01	0.01\\
349.01	0.01\\
350.01	0.01\\
351.01	0.01\\
352.01	0.01\\
353.01	0.01\\
354.01	0.01\\
355.01	0.01\\
356.01	0.01\\
357.01	0.01\\
358.01	0.01\\
359.01	0.01\\
360.01	0.01\\
361.01	0.01\\
362.01	0.01\\
363.01	0.01\\
364.01	0.01\\
365.01	0.01\\
366.01	0.01\\
367.01	0.01\\
368.01	0.01\\
369.01	0.01\\
370.01	0.01\\
371.01	0.01\\
372.01	0.01\\
373.01	0.01\\
374.01	0.01\\
375.01	0.01\\
376.01	0.01\\
377.01	0.01\\
378.01	0.01\\
379.01	0.01\\
380.01	0.01\\
381.01	0.01\\
382.01	0.01\\
383.01	0.01\\
384.01	0.01\\
385.01	0.01\\
386.01	0.01\\
387.01	0.01\\
388.01	0.01\\
389.01	0.01\\
390.01	0.01\\
391.01	0.01\\
392.01	0.01\\
393.01	0.01\\
394.01	0.01\\
395.01	0.01\\
396.01	0.01\\
397.01	0.01\\
398.01	0.01\\
399.01	0.01\\
400.01	0.01\\
401.01	0.01\\
402.01	0.01\\
403.01	0.01\\
404.01	0.01\\
405.01	0.01\\
406.01	0.01\\
407.01	0.01\\
408.01	0.01\\
409.01	0.01\\
410.01	0.01\\
411.01	0.01\\
412.01	0.01\\
413.01	0.01\\
414.01	0.01\\
415.01	0.01\\
416.01	0.01\\
417.01	0.01\\
418.01	0.01\\
419.01	0.01\\
420.01	0.01\\
421.01	0.01\\
422.01	0.01\\
423.01	0.01\\
424.01	0.01\\
425.01	0.01\\
426.01	0.01\\
427.01	0.01\\
428.01	0.01\\
429.01	0.01\\
430.01	0.01\\
431.01	0.01\\
432.01	0.01\\
433.01	0.01\\
434.01	0.01\\
435.01	0.01\\
436.01	0.01\\
437.01	0.01\\
438.01	0.01\\
439.01	0.01\\
440.01	0.01\\
441.01	0.01\\
442.01	0.01\\
443.01	0.01\\
444.01	0.01\\
445.01	0.01\\
446.01	0.01\\
447.01	0.01\\
448.01	0.01\\
449.01	0.01\\
450.01	0.01\\
451.01	0.01\\
452.01	0.01\\
453.01	0.01\\
454.01	0.01\\
455.01	0.01\\
456.01	0.01\\
457.01	0.01\\
458.01	0.01\\
459.01	0.01\\
460.01	0.01\\
461.01	0.01\\
462.01	0.01\\
463.01	0.01\\
464.01	0.01\\
465.01	0.01\\
466.01	0.01\\
467.01	0.01\\
468.01	0.01\\
469.01	0.01\\
470.01	0.01\\
471.01	0.01\\
472.01	0.01\\
473.01	0.01\\
474.01	0.01\\
475.01	0.01\\
476.01	0.01\\
477.01	0.01\\
478.01	0.01\\
479.01	0.01\\
480.01	0.01\\
481.01	0.01\\
482.01	0.01\\
483.01	0.01\\
484.01	0.01\\
485.01	0.01\\
486.01	0.01\\
487.01	0.01\\
488.01	0.01\\
489.01	0.01\\
490.01	0.01\\
491.01	0.01\\
492.01	0.01\\
493.01	0.01\\
494.01	0.01\\
495.01	0.01\\
496.01	0.01\\
497.01	0.01\\
498.01	0.01\\
499.01	0.01\\
500.01	0.01\\
501.01	0.01\\
502.01	0.01\\
503.01	0.01\\
504.01	0.01\\
505.01	0.01\\
506.01	0.01\\
507.01	0.01\\
508.01	0.01\\
509.01	0.01\\
510.01	0.01\\
511.01	0.01\\
512.01	0.01\\
513.01	0.01\\
514.01	0.01\\
515.01	0.01\\
516.01	0.01\\
517.01	0.01\\
518.01	0.01\\
519.01	0.01\\
520.01	0.01\\
521.01	0.01\\
522.01	0.01\\
523.01	0.01\\
524.01	0.01\\
525.01	0.01\\
526.01	0.01\\
527.01	0.01\\
528.01	0.01\\
529.01	0.01\\
530.01	0.01\\
531.01	0.01\\
532.01	0.01\\
533.01	0.01\\
534.01	0.01\\
535.01	0.01\\
536.01	0.01\\
537.01	0.01\\
538.01	0.01\\
539.01	0.01\\
540.01	0.01\\
541.01	0.01\\
542.01	0.01\\
543.01	0.01\\
544.01	0.01\\
545.01	0.01\\
546.01	0.01\\
547.01	0.01\\
548.01	0.01\\
549.01	0.01\\
550.01	0.01\\
551.01	0.01\\
552.01	0.01\\
553.01	0.01\\
554.01	0.01\\
555.01	0.01\\
556.01	0.01\\
557.01	0.01\\
558.01	0.01\\
559.01	0.01\\
560.01	0.01\\
561.01	0.01\\
562.01	0.01\\
563.01	0.01\\
564.01	0.01\\
565.01	0.01\\
566.01	0.01\\
567.01	0.01\\
568.01	0.01\\
569.01	0.01\\
570.01	0.01\\
571.01	0.01\\
572.01	0.01\\
573.01	0.01\\
574.01	0.01\\
575.01	0.01\\
576.01	0.01\\
577.01	0.01\\
578.01	0.01\\
579.01	0.01\\
580.01	0.01\\
581.01	0.01\\
582.01	0.01\\
583.01	0.01\\
584.01	0.01\\
585.01	0.01\\
586.01	0.01\\
587.01	0.01\\
588.01	0.01\\
589.01	0.01\\
590.01	0.01\\
591.01	0.01\\
592.01	0.01\\
593.01	0.01\\
594.01	0.01\\
595.01	0.01\\
596.01	0.01\\
597.01	0.01\\
598.01	0.01\\
599.01	0.01\\
599.02	0.01\\
599.03	0.01\\
599.04	0.01\\
599.05	0.01\\
599.06	0.01\\
599.07	0.01\\
599.08	0.01\\
599.09	0.01\\
599.1	0.01\\
599.11	0.01\\
599.12	0.01\\
599.13	0.01\\
599.14	0.01\\
599.15	0.01\\
599.16	0.01\\
599.17	0.01\\
599.18	0.01\\
599.19	0.01\\
599.2	0.01\\
599.21	0.01\\
599.22	0.01\\
599.23	0.01\\
599.24	0.01\\
599.25	0.01\\
599.26	0.01\\
599.27	0.01\\
599.28	0.01\\
599.29	0.01\\
599.3	0.01\\
599.31	0.01\\
599.32	0.01\\
599.33	0.01\\
599.34	0.01\\
599.35	0.01\\
599.36	0.01\\
599.37	0.01\\
599.38	0.01\\
599.39	0.01\\
599.4	0.01\\
599.41	0.01\\
599.42	0.01\\
599.43	0.01\\
599.44	0.01\\
599.45	0.01\\
599.46	0.01\\
599.47	0.01\\
599.48	0.01\\
599.49	0.01\\
599.5	0.01\\
599.51	0.01\\
599.52	0.01\\
599.53	0.01\\
599.54	0.01\\
599.55	0.01\\
599.56	0.01\\
599.57	0.01\\
599.58	0.01\\
599.59	0.01\\
599.6	0.01\\
599.61	0.01\\
599.62	0.01\\
599.63	0.01\\
599.64	0.01\\
599.65	0.01\\
599.66	0.01\\
599.67	0.01\\
599.68	0.01\\
599.69	0.01\\
599.7	0.01\\
599.71	0.01\\
599.72	0.01\\
599.73	0.01\\
599.74	0.01\\
599.75	0.01\\
599.76	0.01\\
599.77	0.01\\
599.78	0.01\\
599.79	0.01\\
599.8	0.01\\
599.81	0.01\\
599.82	0.01\\
599.83	0.01\\
599.84	0.01\\
599.85	0.01\\
599.86	0.01\\
599.87	0.01\\
599.88	0.01\\
599.89	0.01\\
599.9	0.01\\
599.91	0.01\\
599.92	0.01\\
599.93	0.01\\
599.94	0.01\\
599.95	0.01\\
599.96	0.01\\
599.97	0.01\\
599.98	0.01\\
599.99	0.01\\
600	0.01\\
};
\addplot [color=mycolor7,solid,forget plot]
  table[row sep=crcr]{%
0.01	0.01\\
1.01	0.01\\
2.01	0.01\\
3.01	0.01\\
4.01	0.01\\
5.01	0.01\\
6.01	0.01\\
7.01	0.01\\
8.01	0.01\\
9.01	0.01\\
10.01	0.01\\
11.01	0.01\\
12.01	0.01\\
13.01	0.01\\
14.01	0.01\\
15.01	0.01\\
16.01	0.01\\
17.01	0.01\\
18.01	0.01\\
19.01	0.01\\
20.01	0.01\\
21.01	0.01\\
22.01	0.01\\
23.01	0.01\\
24.01	0.01\\
25.01	0.01\\
26.01	0.01\\
27.01	0.01\\
28.01	0.01\\
29.01	0.01\\
30.01	0.01\\
31.01	0.01\\
32.01	0.01\\
33.01	0.01\\
34.01	0.01\\
35.01	0.01\\
36.01	0.01\\
37.01	0.01\\
38.01	0.01\\
39.01	0.01\\
40.01	0.01\\
41.01	0.01\\
42.01	0.01\\
43.01	0.01\\
44.01	0.01\\
45.01	0.01\\
46.01	0.01\\
47.01	0.01\\
48.01	0.01\\
49.01	0.01\\
50.01	0.01\\
51.01	0.01\\
52.01	0.01\\
53.01	0.01\\
54.01	0.01\\
55.01	0.01\\
56.01	0.01\\
57.01	0.01\\
58.01	0.01\\
59.01	0.01\\
60.01	0.01\\
61.01	0.01\\
62.01	0.01\\
63.01	0.01\\
64.01	0.01\\
65.01	0.01\\
66.01	0.01\\
67.01	0.01\\
68.01	0.01\\
69.01	0.01\\
70.01	0.01\\
71.01	0.01\\
72.01	0.01\\
73.01	0.01\\
74.01	0.01\\
75.01	0.01\\
76.01	0.01\\
77.01	0.01\\
78.01	0.01\\
79.01	0.01\\
80.01	0.01\\
81.01	0.01\\
82.01	0.01\\
83.01	0.01\\
84.01	0.01\\
85.01	0.01\\
86.01	0.01\\
87.01	0.01\\
88.01	0.01\\
89.01	0.01\\
90.01	0.01\\
91.01	0.01\\
92.01	0.01\\
93.01	0.01\\
94.01	0.01\\
95.01	0.01\\
96.01	0.01\\
97.01	0.01\\
98.01	0.01\\
99.01	0.01\\
100.01	0.01\\
101.01	0.01\\
102.01	0.01\\
103.01	0.01\\
104.01	0.01\\
105.01	0.01\\
106.01	0.01\\
107.01	0.01\\
108.01	0.01\\
109.01	0.01\\
110.01	0.01\\
111.01	0.01\\
112.01	0.01\\
113.01	0.01\\
114.01	0.01\\
115.01	0.01\\
116.01	0.01\\
117.01	0.01\\
118.01	0.01\\
119.01	0.01\\
120.01	0.01\\
121.01	0.01\\
122.01	0.01\\
123.01	0.01\\
124.01	0.01\\
125.01	0.01\\
126.01	0.01\\
127.01	0.01\\
128.01	0.01\\
129.01	0.01\\
130.01	0.01\\
131.01	0.01\\
132.01	0.01\\
133.01	0.01\\
134.01	0.01\\
135.01	0.01\\
136.01	0.01\\
137.01	0.01\\
138.01	0.01\\
139.01	0.01\\
140.01	0.01\\
141.01	0.01\\
142.01	0.01\\
143.01	0.01\\
144.01	0.01\\
145.01	0.01\\
146.01	0.01\\
147.01	0.01\\
148.01	0.01\\
149.01	0.01\\
150.01	0.01\\
151.01	0.01\\
152.01	0.01\\
153.01	0.01\\
154.01	0.01\\
155.01	0.01\\
156.01	0.01\\
157.01	0.01\\
158.01	0.01\\
159.01	0.01\\
160.01	0.01\\
161.01	0.01\\
162.01	0.01\\
163.01	0.01\\
164.01	0.01\\
165.01	0.01\\
166.01	0.01\\
167.01	0.01\\
168.01	0.01\\
169.01	0.01\\
170.01	0.01\\
171.01	0.01\\
172.01	0.01\\
173.01	0.01\\
174.01	0.01\\
175.01	0.01\\
176.01	0.01\\
177.01	0.01\\
178.01	0.01\\
179.01	0.01\\
180.01	0.01\\
181.01	0.01\\
182.01	0.01\\
183.01	0.01\\
184.01	0.01\\
185.01	0.01\\
186.01	0.01\\
187.01	0.01\\
188.01	0.01\\
189.01	0.01\\
190.01	0.01\\
191.01	0.01\\
192.01	0.01\\
193.01	0.01\\
194.01	0.01\\
195.01	0.01\\
196.01	0.01\\
197.01	0.01\\
198.01	0.01\\
199.01	0.01\\
200.01	0.01\\
201.01	0.01\\
202.01	0.01\\
203.01	0.01\\
204.01	0.01\\
205.01	0.01\\
206.01	0.01\\
207.01	0.01\\
208.01	0.01\\
209.01	0.01\\
210.01	0.01\\
211.01	0.01\\
212.01	0.01\\
213.01	0.01\\
214.01	0.01\\
215.01	0.01\\
216.01	0.01\\
217.01	0.01\\
218.01	0.01\\
219.01	0.01\\
220.01	0.01\\
221.01	0.01\\
222.01	0.01\\
223.01	0.01\\
224.01	0.01\\
225.01	0.01\\
226.01	0.01\\
227.01	0.01\\
228.01	0.01\\
229.01	0.01\\
230.01	0.01\\
231.01	0.01\\
232.01	0.01\\
233.01	0.01\\
234.01	0.01\\
235.01	0.01\\
236.01	0.01\\
237.01	0.01\\
238.01	0.01\\
239.01	0.01\\
240.01	0.01\\
241.01	0.01\\
242.01	0.01\\
243.01	0.01\\
244.01	0.01\\
245.01	0.01\\
246.01	0.01\\
247.01	0.01\\
248.01	0.01\\
249.01	0.01\\
250.01	0.01\\
251.01	0.01\\
252.01	0.01\\
253.01	0.01\\
254.01	0.01\\
255.01	0.01\\
256.01	0.01\\
257.01	0.01\\
258.01	0.01\\
259.01	0.01\\
260.01	0.01\\
261.01	0.01\\
262.01	0.01\\
263.01	0.01\\
264.01	0.01\\
265.01	0.01\\
266.01	0.01\\
267.01	0.01\\
268.01	0.01\\
269.01	0.01\\
270.01	0.01\\
271.01	0.01\\
272.01	0.01\\
273.01	0.01\\
274.01	0.01\\
275.01	0.01\\
276.01	0.01\\
277.01	0.01\\
278.01	0.01\\
279.01	0.01\\
280.01	0.01\\
281.01	0.01\\
282.01	0.01\\
283.01	0.01\\
284.01	0.01\\
285.01	0.01\\
286.01	0.01\\
287.01	0.01\\
288.01	0.01\\
289.01	0.01\\
290.01	0.01\\
291.01	0.01\\
292.01	0.01\\
293.01	0.01\\
294.01	0.01\\
295.01	0.01\\
296.01	0.01\\
297.01	0.01\\
298.01	0.01\\
299.01	0.01\\
300.01	0.01\\
301.01	0.01\\
302.01	0.01\\
303.01	0.01\\
304.01	0.01\\
305.01	0.01\\
306.01	0.01\\
307.01	0.01\\
308.01	0.01\\
309.01	0.01\\
310.01	0.01\\
311.01	0.01\\
312.01	0.01\\
313.01	0.01\\
314.01	0.01\\
315.01	0.01\\
316.01	0.01\\
317.01	0.01\\
318.01	0.01\\
319.01	0.01\\
320.01	0.01\\
321.01	0.01\\
322.01	0.01\\
323.01	0.01\\
324.01	0.01\\
325.01	0.01\\
326.01	0.01\\
327.01	0.01\\
328.01	0.01\\
329.01	0.01\\
330.01	0.01\\
331.01	0.01\\
332.01	0.01\\
333.01	0.01\\
334.01	0.01\\
335.01	0.01\\
336.01	0.01\\
337.01	0.01\\
338.01	0.01\\
339.01	0.01\\
340.01	0.01\\
341.01	0.01\\
342.01	0.01\\
343.01	0.01\\
344.01	0.01\\
345.01	0.01\\
346.01	0.01\\
347.01	0.01\\
348.01	0.01\\
349.01	0.01\\
350.01	0.01\\
351.01	0.01\\
352.01	0.01\\
353.01	0.01\\
354.01	0.01\\
355.01	0.01\\
356.01	0.01\\
357.01	0.01\\
358.01	0.01\\
359.01	0.01\\
360.01	0.01\\
361.01	0.01\\
362.01	0.01\\
363.01	0.01\\
364.01	0.01\\
365.01	0.01\\
366.01	0.01\\
367.01	0.01\\
368.01	0.01\\
369.01	0.01\\
370.01	0.01\\
371.01	0.01\\
372.01	0.01\\
373.01	0.01\\
374.01	0.01\\
375.01	0.01\\
376.01	0.01\\
377.01	0.01\\
378.01	0.01\\
379.01	0.01\\
380.01	0.01\\
381.01	0.01\\
382.01	0.01\\
383.01	0.01\\
384.01	0.01\\
385.01	0.01\\
386.01	0.01\\
387.01	0.01\\
388.01	0.01\\
389.01	0.01\\
390.01	0.01\\
391.01	0.01\\
392.01	0.01\\
393.01	0.01\\
394.01	0.01\\
395.01	0.01\\
396.01	0.01\\
397.01	0.01\\
398.01	0.01\\
399.01	0.01\\
400.01	0.01\\
401.01	0.01\\
402.01	0.01\\
403.01	0.01\\
404.01	0.01\\
405.01	0.01\\
406.01	0.01\\
407.01	0.01\\
408.01	0.01\\
409.01	0.01\\
410.01	0.01\\
411.01	0.01\\
412.01	0.01\\
413.01	0.01\\
414.01	0.01\\
415.01	0.01\\
416.01	0.01\\
417.01	0.01\\
418.01	0.01\\
419.01	0.01\\
420.01	0.01\\
421.01	0.01\\
422.01	0.01\\
423.01	0.01\\
424.01	0.01\\
425.01	0.01\\
426.01	0.01\\
427.01	0.01\\
428.01	0.01\\
429.01	0.01\\
430.01	0.01\\
431.01	0.01\\
432.01	0.01\\
433.01	0.01\\
434.01	0.01\\
435.01	0.01\\
436.01	0.01\\
437.01	0.01\\
438.01	0.01\\
439.01	0.01\\
440.01	0.01\\
441.01	0.01\\
442.01	0.01\\
443.01	0.01\\
444.01	0.01\\
445.01	0.01\\
446.01	0.01\\
447.01	0.01\\
448.01	0.01\\
449.01	0.01\\
450.01	0.01\\
451.01	0.01\\
452.01	0.01\\
453.01	0.01\\
454.01	0.01\\
455.01	0.01\\
456.01	0.01\\
457.01	0.01\\
458.01	0.01\\
459.01	0.01\\
460.01	0.01\\
461.01	0.01\\
462.01	0.01\\
463.01	0.01\\
464.01	0.01\\
465.01	0.01\\
466.01	0.01\\
467.01	0.01\\
468.01	0.01\\
469.01	0.01\\
470.01	0.01\\
471.01	0.01\\
472.01	0.01\\
473.01	0.01\\
474.01	0.01\\
475.01	0.01\\
476.01	0.01\\
477.01	0.01\\
478.01	0.01\\
479.01	0.01\\
480.01	0.01\\
481.01	0.01\\
482.01	0.01\\
483.01	0.01\\
484.01	0.01\\
485.01	0.01\\
486.01	0.01\\
487.01	0.01\\
488.01	0.01\\
489.01	0.01\\
490.01	0.01\\
491.01	0.01\\
492.01	0.01\\
493.01	0.01\\
494.01	0.01\\
495.01	0.01\\
496.01	0.01\\
497.01	0.01\\
498.01	0.01\\
499.01	0.01\\
500.01	0.01\\
501.01	0.01\\
502.01	0.01\\
503.01	0.01\\
504.01	0.01\\
505.01	0.01\\
506.01	0.01\\
507.01	0.01\\
508.01	0.01\\
509.01	0.01\\
510.01	0.01\\
511.01	0.01\\
512.01	0.01\\
513.01	0.01\\
514.01	0.01\\
515.01	0.01\\
516.01	0.01\\
517.01	0.01\\
518.01	0.01\\
519.01	0.01\\
520.01	0.01\\
521.01	0.01\\
522.01	0.01\\
523.01	0.01\\
524.01	0.01\\
525.01	0.01\\
526.01	0.01\\
527.01	0.01\\
528.01	0.01\\
529.01	0.01\\
530.01	0.01\\
531.01	0.01\\
532.01	0.01\\
533.01	0.01\\
534.01	0.01\\
535.01	0.01\\
536.01	0.01\\
537.01	0.01\\
538.01	0.01\\
539.01	0.01\\
540.01	0.01\\
541.01	0.01\\
542.01	0.01\\
543.01	0.01\\
544.01	0.01\\
545.01	0.01\\
546.01	0.01\\
547.01	0.01\\
548.01	0.01\\
549.01	0.01\\
550.01	0.01\\
551.01	0.01\\
552.01	0.01\\
553.01	0.01\\
554.01	0.01\\
555.01	0.01\\
556.01	0.01\\
557.01	0.01\\
558.01	0.01\\
559.01	0.01\\
560.01	0.01\\
561.01	0.01\\
562.01	0.01\\
563.01	0.01\\
564.01	0.01\\
565.01	0.01\\
566.01	0.01\\
567.01	0.01\\
568.01	0.01\\
569.01	0.01\\
570.01	0.01\\
571.01	0.01\\
572.01	0.01\\
573.01	0.01\\
574.01	0.01\\
575.01	0.01\\
576.01	0.01\\
577.01	0.01\\
578.01	0.01\\
579.01	0.01\\
580.01	0.01\\
581.01	0.01\\
582.01	0.01\\
583.01	0.01\\
584.01	0.01\\
585.01	0.01\\
586.01	0.01\\
587.01	0.01\\
588.01	0.01\\
589.01	0.01\\
590.01	0.01\\
591.01	0.01\\
592.01	0.01\\
593.01	0.01\\
594.01	0.01\\
595.01	0.01\\
596.01	0.01\\
597.01	0.01\\
598.01	0.01\\
599.01	0.01\\
599.02	0.01\\
599.03	0.01\\
599.04	0.01\\
599.05	0.01\\
599.06	0.01\\
599.07	0.01\\
599.08	0.01\\
599.09	0.01\\
599.1	0.01\\
599.11	0.01\\
599.12	0.01\\
599.13	0.01\\
599.14	0.01\\
599.15	0.01\\
599.16	0.01\\
599.17	0.01\\
599.18	0.01\\
599.19	0.01\\
599.2	0.01\\
599.21	0.01\\
599.22	0.01\\
599.23	0.01\\
599.24	0.01\\
599.25	0.01\\
599.26	0.01\\
599.27	0.01\\
599.28	0.01\\
599.29	0.01\\
599.3	0.01\\
599.31	0.01\\
599.32	0.01\\
599.33	0.01\\
599.34	0.01\\
599.35	0.01\\
599.36	0.01\\
599.37	0.01\\
599.38	0.01\\
599.39	0.01\\
599.4	0.01\\
599.41	0.01\\
599.42	0.01\\
599.43	0.01\\
599.44	0.01\\
599.45	0.01\\
599.46	0.01\\
599.47	0.01\\
599.48	0.01\\
599.49	0.01\\
599.5	0.01\\
599.51	0.01\\
599.52	0.01\\
599.53	0.01\\
599.54	0.01\\
599.55	0.01\\
599.56	0.01\\
599.57	0.01\\
599.58	0.01\\
599.59	0.01\\
599.6	0.01\\
599.61	0.01\\
599.62	0.01\\
599.63	0.01\\
599.64	0.01\\
599.65	0.01\\
599.66	0.01\\
599.67	0.01\\
599.68	0.01\\
599.69	0.01\\
599.7	0.01\\
599.71	0.01\\
599.72	0.01\\
599.73	0.01\\
599.74	0.01\\
599.75	0.01\\
599.76	0.01\\
599.77	0.01\\
599.78	0.01\\
599.79	0.01\\
599.8	0.01\\
599.81	0.01\\
599.82	0.01\\
599.83	0.01\\
599.84	0.01\\
599.85	0.01\\
599.86	0.01\\
599.87	0.01\\
599.88	0.01\\
599.89	0.01\\
599.9	0.01\\
599.91	0.01\\
599.92	0.01\\
599.93	0.01\\
599.94	0.01\\
599.95	0.01\\
599.96	0.01\\
599.97	0.01\\
599.98	0.01\\
599.99	0.01\\
600	0.01\\
};
\addplot [color=mycolor8,solid,forget plot]
  table[row sep=crcr]{%
0.01	0.01\\
1.01	0.01\\
2.01	0.01\\
3.01	0.01\\
4.01	0.01\\
5.01	0.01\\
6.01	0.01\\
7.01	0.01\\
8.01	0.01\\
9.01	0.01\\
10.01	0.01\\
11.01	0.01\\
12.01	0.01\\
13.01	0.01\\
14.01	0.01\\
15.01	0.01\\
16.01	0.01\\
17.01	0.01\\
18.01	0.01\\
19.01	0.01\\
20.01	0.01\\
21.01	0.01\\
22.01	0.01\\
23.01	0.01\\
24.01	0.01\\
25.01	0.01\\
26.01	0.01\\
27.01	0.01\\
28.01	0.01\\
29.01	0.01\\
30.01	0.01\\
31.01	0.01\\
32.01	0.01\\
33.01	0.01\\
34.01	0.01\\
35.01	0.01\\
36.01	0.01\\
37.01	0.01\\
38.01	0.01\\
39.01	0.01\\
40.01	0.01\\
41.01	0.01\\
42.01	0.01\\
43.01	0.01\\
44.01	0.01\\
45.01	0.01\\
46.01	0.01\\
47.01	0.01\\
48.01	0.01\\
49.01	0.01\\
50.01	0.01\\
51.01	0.01\\
52.01	0.01\\
53.01	0.01\\
54.01	0.01\\
55.01	0.01\\
56.01	0.01\\
57.01	0.01\\
58.01	0.01\\
59.01	0.01\\
60.01	0.01\\
61.01	0.01\\
62.01	0.01\\
63.01	0.01\\
64.01	0.01\\
65.01	0.01\\
66.01	0.01\\
67.01	0.01\\
68.01	0.01\\
69.01	0.01\\
70.01	0.01\\
71.01	0.01\\
72.01	0.01\\
73.01	0.01\\
74.01	0.01\\
75.01	0.01\\
76.01	0.01\\
77.01	0.01\\
78.01	0.01\\
79.01	0.01\\
80.01	0.01\\
81.01	0.01\\
82.01	0.01\\
83.01	0.01\\
84.01	0.01\\
85.01	0.01\\
86.01	0.01\\
87.01	0.01\\
88.01	0.01\\
89.01	0.01\\
90.01	0.01\\
91.01	0.01\\
92.01	0.01\\
93.01	0.01\\
94.01	0.01\\
95.01	0.01\\
96.01	0.01\\
97.01	0.01\\
98.01	0.01\\
99.01	0.01\\
100.01	0.01\\
101.01	0.01\\
102.01	0.01\\
103.01	0.01\\
104.01	0.01\\
105.01	0.01\\
106.01	0.01\\
107.01	0.01\\
108.01	0.01\\
109.01	0.01\\
110.01	0.01\\
111.01	0.01\\
112.01	0.01\\
113.01	0.01\\
114.01	0.01\\
115.01	0.01\\
116.01	0.01\\
117.01	0.01\\
118.01	0.01\\
119.01	0.01\\
120.01	0.01\\
121.01	0.01\\
122.01	0.01\\
123.01	0.01\\
124.01	0.01\\
125.01	0.01\\
126.01	0.01\\
127.01	0.01\\
128.01	0.01\\
129.01	0.01\\
130.01	0.01\\
131.01	0.01\\
132.01	0.01\\
133.01	0.01\\
134.01	0.01\\
135.01	0.01\\
136.01	0.01\\
137.01	0.01\\
138.01	0.01\\
139.01	0.01\\
140.01	0.01\\
141.01	0.01\\
142.01	0.01\\
143.01	0.01\\
144.01	0.01\\
145.01	0.01\\
146.01	0.01\\
147.01	0.01\\
148.01	0.01\\
149.01	0.01\\
150.01	0.01\\
151.01	0.01\\
152.01	0.01\\
153.01	0.01\\
154.01	0.01\\
155.01	0.01\\
156.01	0.01\\
157.01	0.01\\
158.01	0.01\\
159.01	0.01\\
160.01	0.01\\
161.01	0.01\\
162.01	0.01\\
163.01	0.01\\
164.01	0.01\\
165.01	0.01\\
166.01	0.01\\
167.01	0.01\\
168.01	0.01\\
169.01	0.01\\
170.01	0.01\\
171.01	0.01\\
172.01	0.01\\
173.01	0.01\\
174.01	0.01\\
175.01	0.01\\
176.01	0.01\\
177.01	0.01\\
178.01	0.01\\
179.01	0.01\\
180.01	0.01\\
181.01	0.01\\
182.01	0.01\\
183.01	0.01\\
184.01	0.01\\
185.01	0.01\\
186.01	0.01\\
187.01	0.01\\
188.01	0.01\\
189.01	0.01\\
190.01	0.01\\
191.01	0.01\\
192.01	0.01\\
193.01	0.01\\
194.01	0.01\\
195.01	0.01\\
196.01	0.01\\
197.01	0.01\\
198.01	0.01\\
199.01	0.01\\
200.01	0.01\\
201.01	0.01\\
202.01	0.01\\
203.01	0.01\\
204.01	0.01\\
205.01	0.01\\
206.01	0.01\\
207.01	0.01\\
208.01	0.01\\
209.01	0.01\\
210.01	0.01\\
211.01	0.01\\
212.01	0.01\\
213.01	0.01\\
214.01	0.01\\
215.01	0.01\\
216.01	0.01\\
217.01	0.01\\
218.01	0.01\\
219.01	0.01\\
220.01	0.01\\
221.01	0.01\\
222.01	0.01\\
223.01	0.01\\
224.01	0.01\\
225.01	0.01\\
226.01	0.01\\
227.01	0.01\\
228.01	0.01\\
229.01	0.01\\
230.01	0.01\\
231.01	0.01\\
232.01	0.01\\
233.01	0.01\\
234.01	0.01\\
235.01	0.01\\
236.01	0.01\\
237.01	0.01\\
238.01	0.01\\
239.01	0.01\\
240.01	0.01\\
241.01	0.01\\
242.01	0.01\\
243.01	0.01\\
244.01	0.01\\
245.01	0.01\\
246.01	0.01\\
247.01	0.01\\
248.01	0.01\\
249.01	0.01\\
250.01	0.01\\
251.01	0.01\\
252.01	0.01\\
253.01	0.01\\
254.01	0.01\\
255.01	0.01\\
256.01	0.01\\
257.01	0.01\\
258.01	0.01\\
259.01	0.01\\
260.01	0.01\\
261.01	0.01\\
262.01	0.01\\
263.01	0.01\\
264.01	0.01\\
265.01	0.01\\
266.01	0.01\\
267.01	0.01\\
268.01	0.01\\
269.01	0.01\\
270.01	0.01\\
271.01	0.01\\
272.01	0.01\\
273.01	0.01\\
274.01	0.01\\
275.01	0.01\\
276.01	0.01\\
277.01	0.01\\
278.01	0.01\\
279.01	0.01\\
280.01	0.01\\
281.01	0.01\\
282.01	0.01\\
283.01	0.01\\
284.01	0.01\\
285.01	0.01\\
286.01	0.01\\
287.01	0.01\\
288.01	0.01\\
289.01	0.01\\
290.01	0.01\\
291.01	0.01\\
292.01	0.01\\
293.01	0.01\\
294.01	0.01\\
295.01	0.01\\
296.01	0.01\\
297.01	0.01\\
298.01	0.01\\
299.01	0.01\\
300.01	0.01\\
301.01	0.01\\
302.01	0.01\\
303.01	0.01\\
304.01	0.01\\
305.01	0.01\\
306.01	0.01\\
307.01	0.01\\
308.01	0.01\\
309.01	0.01\\
310.01	0.01\\
311.01	0.01\\
312.01	0.01\\
313.01	0.01\\
314.01	0.01\\
315.01	0.01\\
316.01	0.01\\
317.01	0.01\\
318.01	0.01\\
319.01	0.01\\
320.01	0.01\\
321.01	0.01\\
322.01	0.01\\
323.01	0.01\\
324.01	0.01\\
325.01	0.01\\
326.01	0.01\\
327.01	0.01\\
328.01	0.01\\
329.01	0.01\\
330.01	0.01\\
331.01	0.01\\
332.01	0.01\\
333.01	0.01\\
334.01	0.01\\
335.01	0.01\\
336.01	0.01\\
337.01	0.01\\
338.01	0.01\\
339.01	0.01\\
340.01	0.01\\
341.01	0.01\\
342.01	0.01\\
343.01	0.01\\
344.01	0.01\\
345.01	0.01\\
346.01	0.01\\
347.01	0.01\\
348.01	0.01\\
349.01	0.01\\
350.01	0.01\\
351.01	0.01\\
352.01	0.01\\
353.01	0.01\\
354.01	0.01\\
355.01	0.01\\
356.01	0.01\\
357.01	0.01\\
358.01	0.01\\
359.01	0.01\\
360.01	0.01\\
361.01	0.01\\
362.01	0.01\\
363.01	0.01\\
364.01	0.01\\
365.01	0.01\\
366.01	0.01\\
367.01	0.01\\
368.01	0.01\\
369.01	0.01\\
370.01	0.01\\
371.01	0.01\\
372.01	0.01\\
373.01	0.01\\
374.01	0.01\\
375.01	0.01\\
376.01	0.01\\
377.01	0.01\\
378.01	0.01\\
379.01	0.01\\
380.01	0.01\\
381.01	0.01\\
382.01	0.01\\
383.01	0.01\\
384.01	0.01\\
385.01	0.01\\
386.01	0.01\\
387.01	0.01\\
388.01	0.01\\
389.01	0.01\\
390.01	0.01\\
391.01	0.01\\
392.01	0.01\\
393.01	0.01\\
394.01	0.01\\
395.01	0.01\\
396.01	0.01\\
397.01	0.01\\
398.01	0.01\\
399.01	0.01\\
400.01	0.01\\
401.01	0.01\\
402.01	0.01\\
403.01	0.01\\
404.01	0.01\\
405.01	0.01\\
406.01	0.01\\
407.01	0.01\\
408.01	0.01\\
409.01	0.01\\
410.01	0.01\\
411.01	0.01\\
412.01	0.01\\
413.01	0.01\\
414.01	0.01\\
415.01	0.01\\
416.01	0.01\\
417.01	0.01\\
418.01	0.01\\
419.01	0.01\\
420.01	0.01\\
421.01	0.01\\
422.01	0.01\\
423.01	0.01\\
424.01	0.01\\
425.01	0.01\\
426.01	0.01\\
427.01	0.01\\
428.01	0.01\\
429.01	0.01\\
430.01	0.01\\
431.01	0.01\\
432.01	0.01\\
433.01	0.01\\
434.01	0.01\\
435.01	0.01\\
436.01	0.01\\
437.01	0.01\\
438.01	0.01\\
439.01	0.01\\
440.01	0.01\\
441.01	0.01\\
442.01	0.01\\
443.01	0.01\\
444.01	0.01\\
445.01	0.01\\
446.01	0.01\\
447.01	0.01\\
448.01	0.01\\
449.01	0.01\\
450.01	0.01\\
451.01	0.01\\
452.01	0.01\\
453.01	0.01\\
454.01	0.01\\
455.01	0.01\\
456.01	0.01\\
457.01	0.01\\
458.01	0.01\\
459.01	0.01\\
460.01	0.01\\
461.01	0.01\\
462.01	0.01\\
463.01	0.01\\
464.01	0.01\\
465.01	0.01\\
466.01	0.01\\
467.01	0.01\\
468.01	0.01\\
469.01	0.01\\
470.01	0.01\\
471.01	0.01\\
472.01	0.01\\
473.01	0.01\\
474.01	0.01\\
475.01	0.01\\
476.01	0.01\\
477.01	0.01\\
478.01	0.01\\
479.01	0.01\\
480.01	0.01\\
481.01	0.01\\
482.01	0.01\\
483.01	0.01\\
484.01	0.01\\
485.01	0.01\\
486.01	0.01\\
487.01	0.01\\
488.01	0.01\\
489.01	0.01\\
490.01	0.01\\
491.01	0.01\\
492.01	0.01\\
493.01	0.01\\
494.01	0.01\\
495.01	0.01\\
496.01	0.01\\
497.01	0.01\\
498.01	0.01\\
499.01	0.01\\
500.01	0.01\\
501.01	0.01\\
502.01	0.01\\
503.01	0.01\\
504.01	0.01\\
505.01	0.01\\
506.01	0.01\\
507.01	0.01\\
508.01	0.01\\
509.01	0.01\\
510.01	0.01\\
511.01	0.01\\
512.01	0.01\\
513.01	0.01\\
514.01	0.01\\
515.01	0.01\\
516.01	0.01\\
517.01	0.01\\
518.01	0.01\\
519.01	0.01\\
520.01	0.01\\
521.01	0.01\\
522.01	0.01\\
523.01	0.01\\
524.01	0.01\\
525.01	0.01\\
526.01	0.01\\
527.01	0.01\\
528.01	0.01\\
529.01	0.01\\
530.01	0.01\\
531.01	0.01\\
532.01	0.01\\
533.01	0.01\\
534.01	0.01\\
535.01	0.01\\
536.01	0.01\\
537.01	0.01\\
538.01	0.01\\
539.01	0.01\\
540.01	0.01\\
541.01	0.01\\
542.01	0.01\\
543.01	0.01\\
544.01	0.01\\
545.01	0.01\\
546.01	0.01\\
547.01	0.01\\
548.01	0.01\\
549.01	0.01\\
550.01	0.01\\
551.01	0.01\\
552.01	0.01\\
553.01	0.01\\
554.01	0.01\\
555.01	0.01\\
556.01	0.01\\
557.01	0.01\\
558.01	0.01\\
559.01	0.01\\
560.01	0.01\\
561.01	0.01\\
562.01	0.01\\
563.01	0.01\\
564.01	0.01\\
565.01	0.01\\
566.01	0.01\\
567.01	0.01\\
568.01	0.01\\
569.01	0.01\\
570.01	0.01\\
571.01	0.01\\
572.01	0.01\\
573.01	0.01\\
574.01	0.01\\
575.01	0.01\\
576.01	0.01\\
577.01	0.01\\
578.01	0.01\\
579.01	0.01\\
580.01	0.01\\
581.01	0.01\\
582.01	0.01\\
583.01	0.01\\
584.01	0.01\\
585.01	0.01\\
586.01	0.01\\
587.01	0.01\\
588.01	0.01\\
589.01	0.01\\
590.01	0.01\\
591.01	0.01\\
592.01	0.01\\
593.01	0.01\\
594.01	0.01\\
595.01	0.01\\
596.01	0.01\\
597.01	0.01\\
598.01	0.01\\
599.01	0.01\\
599.02	0.01\\
599.03	0.01\\
599.04	0.01\\
599.05	0.01\\
599.06	0.01\\
599.07	0.01\\
599.08	0.01\\
599.09	0.01\\
599.1	0.01\\
599.11	0.01\\
599.12	0.01\\
599.13	0.01\\
599.14	0.01\\
599.15	0.01\\
599.16	0.01\\
599.17	0.01\\
599.18	0.01\\
599.19	0.01\\
599.2	0.01\\
599.21	0.01\\
599.22	0.01\\
599.23	0.01\\
599.24	0.01\\
599.25	0.01\\
599.26	0.01\\
599.27	0.01\\
599.28	0.01\\
599.29	0.01\\
599.3	0.01\\
599.31	0.01\\
599.32	0.01\\
599.33	0.01\\
599.34	0.01\\
599.35	0.01\\
599.36	0.01\\
599.37	0.01\\
599.38	0.01\\
599.39	0.01\\
599.4	0.01\\
599.41	0.01\\
599.42	0.01\\
599.43	0.01\\
599.44	0.01\\
599.45	0.01\\
599.46	0.01\\
599.47	0.01\\
599.48	0.01\\
599.49	0.01\\
599.5	0.01\\
599.51	0.01\\
599.52	0.01\\
599.53	0.01\\
599.54	0.01\\
599.55	0.01\\
599.56	0.01\\
599.57	0.01\\
599.58	0.01\\
599.59	0.01\\
599.6	0.01\\
599.61	0.01\\
599.62	0.01\\
599.63	0.01\\
599.64	0.01\\
599.65	0.01\\
599.66	0.01\\
599.67	0.01\\
599.68	0.01\\
599.69	0.01\\
599.7	0.01\\
599.71	0.01\\
599.72	0.01\\
599.73	0.01\\
599.74	0.01\\
599.75	0.01\\
599.76	0.01\\
599.77	0.01\\
599.78	0.01\\
599.79	0.01\\
599.8	0.01\\
599.81	0.01\\
599.82	0.01\\
599.83	0.01\\
599.84	0.01\\
599.85	0.01\\
599.86	0.01\\
599.87	0.01\\
599.88	0.01\\
599.89	0.01\\
599.9	0.01\\
599.91	0.01\\
599.92	0.01\\
599.93	0.01\\
599.94	0.01\\
599.95	0.01\\
599.96	0.01\\
599.97	0.01\\
599.98	0.01\\
599.99	0.01\\
600	0.01\\
};
\addplot [color=blue!25!mycolor7,solid,forget plot]
  table[row sep=crcr]{%
0.01	0.01\\
1.01	0.01\\
2.01	0.01\\
3.01	0.01\\
4.01	0.01\\
5.01	0.01\\
6.01	0.01\\
7.01	0.01\\
8.01	0.01\\
9.01	0.01\\
10.01	0.01\\
11.01	0.01\\
12.01	0.01\\
13.01	0.01\\
14.01	0.01\\
15.01	0.01\\
16.01	0.01\\
17.01	0.01\\
18.01	0.01\\
19.01	0.01\\
20.01	0.01\\
21.01	0.01\\
22.01	0.01\\
23.01	0.01\\
24.01	0.01\\
25.01	0.01\\
26.01	0.01\\
27.01	0.01\\
28.01	0.01\\
29.01	0.01\\
30.01	0.01\\
31.01	0.01\\
32.01	0.01\\
33.01	0.01\\
34.01	0.01\\
35.01	0.01\\
36.01	0.01\\
37.01	0.01\\
38.01	0.01\\
39.01	0.01\\
40.01	0.01\\
41.01	0.01\\
42.01	0.01\\
43.01	0.01\\
44.01	0.01\\
45.01	0.01\\
46.01	0.01\\
47.01	0.01\\
48.01	0.01\\
49.01	0.01\\
50.01	0.01\\
51.01	0.01\\
52.01	0.01\\
53.01	0.01\\
54.01	0.01\\
55.01	0.01\\
56.01	0.01\\
57.01	0.01\\
58.01	0.01\\
59.01	0.01\\
60.01	0.01\\
61.01	0.01\\
62.01	0.01\\
63.01	0.01\\
64.01	0.01\\
65.01	0.01\\
66.01	0.01\\
67.01	0.01\\
68.01	0.01\\
69.01	0.01\\
70.01	0.01\\
71.01	0.01\\
72.01	0.01\\
73.01	0.01\\
74.01	0.01\\
75.01	0.01\\
76.01	0.01\\
77.01	0.01\\
78.01	0.01\\
79.01	0.01\\
80.01	0.01\\
81.01	0.01\\
82.01	0.01\\
83.01	0.01\\
84.01	0.01\\
85.01	0.01\\
86.01	0.01\\
87.01	0.01\\
88.01	0.01\\
89.01	0.01\\
90.01	0.01\\
91.01	0.01\\
92.01	0.01\\
93.01	0.01\\
94.01	0.01\\
95.01	0.01\\
96.01	0.01\\
97.01	0.01\\
98.01	0.01\\
99.01	0.01\\
100.01	0.01\\
101.01	0.01\\
102.01	0.01\\
103.01	0.01\\
104.01	0.01\\
105.01	0.01\\
106.01	0.01\\
107.01	0.01\\
108.01	0.01\\
109.01	0.01\\
110.01	0.01\\
111.01	0.01\\
112.01	0.01\\
113.01	0.01\\
114.01	0.01\\
115.01	0.01\\
116.01	0.01\\
117.01	0.01\\
118.01	0.01\\
119.01	0.01\\
120.01	0.01\\
121.01	0.01\\
122.01	0.01\\
123.01	0.01\\
124.01	0.01\\
125.01	0.01\\
126.01	0.01\\
127.01	0.01\\
128.01	0.01\\
129.01	0.01\\
130.01	0.01\\
131.01	0.01\\
132.01	0.01\\
133.01	0.01\\
134.01	0.01\\
135.01	0.01\\
136.01	0.01\\
137.01	0.01\\
138.01	0.01\\
139.01	0.01\\
140.01	0.01\\
141.01	0.01\\
142.01	0.01\\
143.01	0.01\\
144.01	0.01\\
145.01	0.01\\
146.01	0.01\\
147.01	0.01\\
148.01	0.01\\
149.01	0.01\\
150.01	0.01\\
151.01	0.01\\
152.01	0.01\\
153.01	0.01\\
154.01	0.01\\
155.01	0.01\\
156.01	0.01\\
157.01	0.01\\
158.01	0.01\\
159.01	0.01\\
160.01	0.01\\
161.01	0.01\\
162.01	0.01\\
163.01	0.01\\
164.01	0.01\\
165.01	0.01\\
166.01	0.01\\
167.01	0.01\\
168.01	0.01\\
169.01	0.01\\
170.01	0.01\\
171.01	0.01\\
172.01	0.01\\
173.01	0.01\\
174.01	0.01\\
175.01	0.01\\
176.01	0.01\\
177.01	0.01\\
178.01	0.01\\
179.01	0.01\\
180.01	0.01\\
181.01	0.01\\
182.01	0.01\\
183.01	0.01\\
184.01	0.01\\
185.01	0.01\\
186.01	0.01\\
187.01	0.01\\
188.01	0.01\\
189.01	0.01\\
190.01	0.01\\
191.01	0.01\\
192.01	0.01\\
193.01	0.01\\
194.01	0.01\\
195.01	0.01\\
196.01	0.01\\
197.01	0.01\\
198.01	0.01\\
199.01	0.01\\
200.01	0.01\\
201.01	0.01\\
202.01	0.01\\
203.01	0.01\\
204.01	0.01\\
205.01	0.01\\
206.01	0.01\\
207.01	0.01\\
208.01	0.01\\
209.01	0.01\\
210.01	0.01\\
211.01	0.01\\
212.01	0.01\\
213.01	0.01\\
214.01	0.01\\
215.01	0.01\\
216.01	0.01\\
217.01	0.01\\
218.01	0.01\\
219.01	0.01\\
220.01	0.01\\
221.01	0.01\\
222.01	0.01\\
223.01	0.01\\
224.01	0.01\\
225.01	0.01\\
226.01	0.01\\
227.01	0.01\\
228.01	0.01\\
229.01	0.01\\
230.01	0.01\\
231.01	0.01\\
232.01	0.01\\
233.01	0.01\\
234.01	0.01\\
235.01	0.01\\
236.01	0.01\\
237.01	0.01\\
238.01	0.01\\
239.01	0.01\\
240.01	0.01\\
241.01	0.01\\
242.01	0.01\\
243.01	0.01\\
244.01	0.01\\
245.01	0.01\\
246.01	0.01\\
247.01	0.01\\
248.01	0.01\\
249.01	0.01\\
250.01	0.01\\
251.01	0.01\\
252.01	0.01\\
253.01	0.01\\
254.01	0.01\\
255.01	0.01\\
256.01	0.01\\
257.01	0.01\\
258.01	0.01\\
259.01	0.01\\
260.01	0.01\\
261.01	0.01\\
262.01	0.01\\
263.01	0.01\\
264.01	0.01\\
265.01	0.01\\
266.01	0.01\\
267.01	0.01\\
268.01	0.01\\
269.01	0.01\\
270.01	0.01\\
271.01	0.01\\
272.01	0.01\\
273.01	0.01\\
274.01	0.01\\
275.01	0.01\\
276.01	0.01\\
277.01	0.01\\
278.01	0.01\\
279.01	0.01\\
280.01	0.01\\
281.01	0.01\\
282.01	0.01\\
283.01	0.01\\
284.01	0.01\\
285.01	0.01\\
286.01	0.01\\
287.01	0.01\\
288.01	0.01\\
289.01	0.01\\
290.01	0.01\\
291.01	0.01\\
292.01	0.01\\
293.01	0.01\\
294.01	0.01\\
295.01	0.01\\
296.01	0.01\\
297.01	0.01\\
298.01	0.01\\
299.01	0.01\\
300.01	0.01\\
301.01	0.01\\
302.01	0.01\\
303.01	0.01\\
304.01	0.01\\
305.01	0.01\\
306.01	0.01\\
307.01	0.01\\
308.01	0.01\\
309.01	0.01\\
310.01	0.01\\
311.01	0.01\\
312.01	0.01\\
313.01	0.01\\
314.01	0.01\\
315.01	0.01\\
316.01	0.01\\
317.01	0.01\\
318.01	0.01\\
319.01	0.01\\
320.01	0.01\\
321.01	0.01\\
322.01	0.01\\
323.01	0.01\\
324.01	0.01\\
325.01	0.01\\
326.01	0.01\\
327.01	0.01\\
328.01	0.01\\
329.01	0.01\\
330.01	0.01\\
331.01	0.01\\
332.01	0.01\\
333.01	0.01\\
334.01	0.01\\
335.01	0.01\\
336.01	0.01\\
337.01	0.01\\
338.01	0.01\\
339.01	0.01\\
340.01	0.01\\
341.01	0.01\\
342.01	0.01\\
343.01	0.01\\
344.01	0.01\\
345.01	0.01\\
346.01	0.01\\
347.01	0.01\\
348.01	0.01\\
349.01	0.01\\
350.01	0.01\\
351.01	0.01\\
352.01	0.01\\
353.01	0.01\\
354.01	0.01\\
355.01	0.01\\
356.01	0.01\\
357.01	0.01\\
358.01	0.01\\
359.01	0.01\\
360.01	0.01\\
361.01	0.01\\
362.01	0.01\\
363.01	0.01\\
364.01	0.01\\
365.01	0.01\\
366.01	0.01\\
367.01	0.01\\
368.01	0.01\\
369.01	0.01\\
370.01	0.01\\
371.01	0.01\\
372.01	0.01\\
373.01	0.01\\
374.01	0.01\\
375.01	0.01\\
376.01	0.01\\
377.01	0.01\\
378.01	0.01\\
379.01	0.01\\
380.01	0.01\\
381.01	0.01\\
382.01	0.01\\
383.01	0.01\\
384.01	0.01\\
385.01	0.01\\
386.01	0.01\\
387.01	0.01\\
388.01	0.01\\
389.01	0.01\\
390.01	0.01\\
391.01	0.01\\
392.01	0.01\\
393.01	0.01\\
394.01	0.01\\
395.01	0.01\\
396.01	0.01\\
397.01	0.01\\
398.01	0.01\\
399.01	0.01\\
400.01	0.01\\
401.01	0.01\\
402.01	0.01\\
403.01	0.01\\
404.01	0.01\\
405.01	0.01\\
406.01	0.01\\
407.01	0.01\\
408.01	0.01\\
409.01	0.01\\
410.01	0.01\\
411.01	0.01\\
412.01	0.01\\
413.01	0.01\\
414.01	0.01\\
415.01	0.01\\
416.01	0.01\\
417.01	0.01\\
418.01	0.01\\
419.01	0.01\\
420.01	0.01\\
421.01	0.01\\
422.01	0.01\\
423.01	0.01\\
424.01	0.01\\
425.01	0.01\\
426.01	0.01\\
427.01	0.01\\
428.01	0.01\\
429.01	0.01\\
430.01	0.01\\
431.01	0.01\\
432.01	0.01\\
433.01	0.01\\
434.01	0.01\\
435.01	0.01\\
436.01	0.01\\
437.01	0.01\\
438.01	0.01\\
439.01	0.01\\
440.01	0.01\\
441.01	0.01\\
442.01	0.01\\
443.01	0.01\\
444.01	0.01\\
445.01	0.01\\
446.01	0.01\\
447.01	0.01\\
448.01	0.01\\
449.01	0.01\\
450.01	0.01\\
451.01	0.01\\
452.01	0.01\\
453.01	0.01\\
454.01	0.01\\
455.01	0.01\\
456.01	0.01\\
457.01	0.01\\
458.01	0.01\\
459.01	0.01\\
460.01	0.01\\
461.01	0.01\\
462.01	0.01\\
463.01	0.01\\
464.01	0.01\\
465.01	0.01\\
466.01	0.01\\
467.01	0.01\\
468.01	0.01\\
469.01	0.01\\
470.01	0.01\\
471.01	0.01\\
472.01	0.01\\
473.01	0.01\\
474.01	0.01\\
475.01	0.01\\
476.01	0.01\\
477.01	0.01\\
478.01	0.01\\
479.01	0.01\\
480.01	0.01\\
481.01	0.01\\
482.01	0.01\\
483.01	0.01\\
484.01	0.01\\
485.01	0.01\\
486.01	0.01\\
487.01	0.01\\
488.01	0.01\\
489.01	0.01\\
490.01	0.01\\
491.01	0.01\\
492.01	0.01\\
493.01	0.01\\
494.01	0.01\\
495.01	0.01\\
496.01	0.01\\
497.01	0.01\\
498.01	0.01\\
499.01	0.01\\
500.01	0.01\\
501.01	0.01\\
502.01	0.01\\
503.01	0.01\\
504.01	0.01\\
505.01	0.01\\
506.01	0.01\\
507.01	0.01\\
508.01	0.01\\
509.01	0.01\\
510.01	0.01\\
511.01	0.01\\
512.01	0.01\\
513.01	0.01\\
514.01	0.01\\
515.01	0.01\\
516.01	0.01\\
517.01	0.01\\
518.01	0.01\\
519.01	0.01\\
520.01	0.01\\
521.01	0.01\\
522.01	0.01\\
523.01	0.01\\
524.01	0.01\\
525.01	0.01\\
526.01	0.01\\
527.01	0.01\\
528.01	0.01\\
529.01	0.01\\
530.01	0.01\\
531.01	0.01\\
532.01	0.01\\
533.01	0.01\\
534.01	0.01\\
535.01	0.01\\
536.01	0.01\\
537.01	0.01\\
538.01	0.01\\
539.01	0.01\\
540.01	0.01\\
541.01	0.01\\
542.01	0.01\\
543.01	0.01\\
544.01	0.01\\
545.01	0.01\\
546.01	0.01\\
547.01	0.01\\
548.01	0.01\\
549.01	0.01\\
550.01	0.01\\
551.01	0.01\\
552.01	0.01\\
553.01	0.01\\
554.01	0.01\\
555.01	0.01\\
556.01	0.01\\
557.01	0.01\\
558.01	0.01\\
559.01	0.01\\
560.01	0.01\\
561.01	0.01\\
562.01	0.01\\
563.01	0.01\\
564.01	0.01\\
565.01	0.01\\
566.01	0.01\\
567.01	0.01\\
568.01	0.01\\
569.01	0.01\\
570.01	0.01\\
571.01	0.01\\
572.01	0.01\\
573.01	0.01\\
574.01	0.01\\
575.01	0.01\\
576.01	0.01\\
577.01	0.01\\
578.01	0.01\\
579.01	0.01\\
580.01	0.01\\
581.01	0.01\\
582.01	0.01\\
583.01	0.01\\
584.01	0.01\\
585.01	0.01\\
586.01	0.01\\
587.01	0.01\\
588.01	0.01\\
589.01	0.01\\
590.01	0.01\\
591.01	0.01\\
592.01	0.01\\
593.01	0.01\\
594.01	0.01\\
595.01	0.01\\
596.01	0.01\\
597.01	0.01\\
598.01	0.01\\
599.01	0.01\\
599.02	0.01\\
599.03	0.01\\
599.04	0.01\\
599.05	0.01\\
599.06	0.01\\
599.07	0.01\\
599.08	0.01\\
599.09	0.01\\
599.1	0.01\\
599.11	0.01\\
599.12	0.01\\
599.13	0.01\\
599.14	0.01\\
599.15	0.01\\
599.16	0.01\\
599.17	0.01\\
599.18	0.01\\
599.19	0.01\\
599.2	0.01\\
599.21	0.01\\
599.22	0.01\\
599.23	0.01\\
599.24	0.01\\
599.25	0.01\\
599.26	0.01\\
599.27	0.01\\
599.28	0.01\\
599.29	0.01\\
599.3	0.01\\
599.31	0.01\\
599.32	0.01\\
599.33	0.01\\
599.34	0.01\\
599.35	0.01\\
599.36	0.01\\
599.37	0.01\\
599.38	0.01\\
599.39	0.01\\
599.4	0.01\\
599.41	0.01\\
599.42	0.01\\
599.43	0.01\\
599.44	0.01\\
599.45	0.01\\
599.46	0.01\\
599.47	0.01\\
599.48	0.01\\
599.49	0.01\\
599.5	0.01\\
599.51	0.01\\
599.52	0.01\\
599.53	0.01\\
599.54	0.01\\
599.55	0.01\\
599.56	0.01\\
599.57	0.01\\
599.58	0.01\\
599.59	0.01\\
599.6	0.01\\
599.61	0.01\\
599.62	0.01\\
599.63	0.01\\
599.64	0.01\\
599.65	0.01\\
599.66	0.01\\
599.67	0.01\\
599.68	0.01\\
599.69	0.01\\
599.7	0.01\\
599.71	0.01\\
599.72	0.01\\
599.73	0.01\\
599.74	0.01\\
599.75	0.01\\
599.76	0.01\\
599.77	0.01\\
599.78	0.01\\
599.79	0.01\\
599.8	0.01\\
599.81	0.01\\
599.82	0.01\\
599.83	0.01\\
599.84	0.01\\
599.85	0.01\\
599.86	0.01\\
599.87	0.01\\
599.88	0.01\\
599.89	0.01\\
599.9	0.01\\
599.91	0.01\\
599.92	0.01\\
599.93	0.01\\
599.94	0.01\\
599.95	0.01\\
599.96	0.01\\
599.97	0.01\\
599.98	0.01\\
599.99	0.01\\
600	0.01\\
};
\addplot [color=mycolor9,solid,forget plot]
  table[row sep=crcr]{%
0.01	0.01\\
1.01	0.01\\
2.01	0.01\\
3.01	0.01\\
4.01	0.01\\
5.01	0.01\\
6.01	0.01\\
7.01	0.01\\
8.01	0.01\\
9.01	0.01\\
10.01	0.01\\
11.01	0.01\\
12.01	0.01\\
13.01	0.01\\
14.01	0.01\\
15.01	0.01\\
16.01	0.01\\
17.01	0.01\\
18.01	0.01\\
19.01	0.01\\
20.01	0.01\\
21.01	0.01\\
22.01	0.01\\
23.01	0.01\\
24.01	0.01\\
25.01	0.01\\
26.01	0.01\\
27.01	0.01\\
28.01	0.01\\
29.01	0.01\\
30.01	0.01\\
31.01	0.01\\
32.01	0.01\\
33.01	0.01\\
34.01	0.01\\
35.01	0.01\\
36.01	0.01\\
37.01	0.01\\
38.01	0.01\\
39.01	0.01\\
40.01	0.01\\
41.01	0.01\\
42.01	0.01\\
43.01	0.01\\
44.01	0.01\\
45.01	0.01\\
46.01	0.01\\
47.01	0.01\\
48.01	0.01\\
49.01	0.01\\
50.01	0.01\\
51.01	0.01\\
52.01	0.01\\
53.01	0.01\\
54.01	0.01\\
55.01	0.01\\
56.01	0.01\\
57.01	0.01\\
58.01	0.01\\
59.01	0.01\\
60.01	0.01\\
61.01	0.01\\
62.01	0.01\\
63.01	0.01\\
64.01	0.01\\
65.01	0.01\\
66.01	0.01\\
67.01	0.01\\
68.01	0.01\\
69.01	0.01\\
70.01	0.01\\
71.01	0.01\\
72.01	0.01\\
73.01	0.01\\
74.01	0.01\\
75.01	0.01\\
76.01	0.01\\
77.01	0.01\\
78.01	0.01\\
79.01	0.01\\
80.01	0.01\\
81.01	0.01\\
82.01	0.01\\
83.01	0.01\\
84.01	0.01\\
85.01	0.01\\
86.01	0.01\\
87.01	0.01\\
88.01	0.01\\
89.01	0.01\\
90.01	0.01\\
91.01	0.01\\
92.01	0.01\\
93.01	0.01\\
94.01	0.01\\
95.01	0.01\\
96.01	0.01\\
97.01	0.01\\
98.01	0.01\\
99.01	0.01\\
100.01	0.01\\
101.01	0.01\\
102.01	0.01\\
103.01	0.01\\
104.01	0.01\\
105.01	0.01\\
106.01	0.01\\
107.01	0.01\\
108.01	0.01\\
109.01	0.01\\
110.01	0.01\\
111.01	0.01\\
112.01	0.01\\
113.01	0.01\\
114.01	0.01\\
115.01	0.01\\
116.01	0.01\\
117.01	0.01\\
118.01	0.01\\
119.01	0.01\\
120.01	0.01\\
121.01	0.01\\
122.01	0.01\\
123.01	0.01\\
124.01	0.01\\
125.01	0.01\\
126.01	0.01\\
127.01	0.01\\
128.01	0.01\\
129.01	0.01\\
130.01	0.01\\
131.01	0.01\\
132.01	0.01\\
133.01	0.01\\
134.01	0.01\\
135.01	0.01\\
136.01	0.01\\
137.01	0.01\\
138.01	0.01\\
139.01	0.01\\
140.01	0.01\\
141.01	0.01\\
142.01	0.01\\
143.01	0.01\\
144.01	0.01\\
145.01	0.01\\
146.01	0.01\\
147.01	0.01\\
148.01	0.01\\
149.01	0.01\\
150.01	0.01\\
151.01	0.01\\
152.01	0.01\\
153.01	0.01\\
154.01	0.01\\
155.01	0.01\\
156.01	0.01\\
157.01	0.01\\
158.01	0.01\\
159.01	0.01\\
160.01	0.01\\
161.01	0.01\\
162.01	0.01\\
163.01	0.01\\
164.01	0.01\\
165.01	0.01\\
166.01	0.01\\
167.01	0.01\\
168.01	0.01\\
169.01	0.01\\
170.01	0.01\\
171.01	0.01\\
172.01	0.01\\
173.01	0.01\\
174.01	0.01\\
175.01	0.01\\
176.01	0.01\\
177.01	0.01\\
178.01	0.01\\
179.01	0.01\\
180.01	0.01\\
181.01	0.01\\
182.01	0.01\\
183.01	0.01\\
184.01	0.01\\
185.01	0.01\\
186.01	0.01\\
187.01	0.01\\
188.01	0.01\\
189.01	0.01\\
190.01	0.01\\
191.01	0.01\\
192.01	0.01\\
193.01	0.01\\
194.01	0.01\\
195.01	0.01\\
196.01	0.01\\
197.01	0.01\\
198.01	0.01\\
199.01	0.01\\
200.01	0.01\\
201.01	0.01\\
202.01	0.01\\
203.01	0.01\\
204.01	0.01\\
205.01	0.01\\
206.01	0.01\\
207.01	0.01\\
208.01	0.01\\
209.01	0.01\\
210.01	0.01\\
211.01	0.01\\
212.01	0.01\\
213.01	0.01\\
214.01	0.01\\
215.01	0.01\\
216.01	0.01\\
217.01	0.01\\
218.01	0.01\\
219.01	0.01\\
220.01	0.01\\
221.01	0.01\\
222.01	0.01\\
223.01	0.01\\
224.01	0.01\\
225.01	0.01\\
226.01	0.01\\
227.01	0.01\\
228.01	0.01\\
229.01	0.01\\
230.01	0.01\\
231.01	0.01\\
232.01	0.01\\
233.01	0.01\\
234.01	0.01\\
235.01	0.01\\
236.01	0.01\\
237.01	0.01\\
238.01	0.01\\
239.01	0.01\\
240.01	0.01\\
241.01	0.01\\
242.01	0.01\\
243.01	0.01\\
244.01	0.01\\
245.01	0.01\\
246.01	0.01\\
247.01	0.01\\
248.01	0.01\\
249.01	0.01\\
250.01	0.01\\
251.01	0.01\\
252.01	0.01\\
253.01	0.01\\
254.01	0.01\\
255.01	0.01\\
256.01	0.01\\
257.01	0.01\\
258.01	0.01\\
259.01	0.01\\
260.01	0.01\\
261.01	0.01\\
262.01	0.01\\
263.01	0.01\\
264.01	0.01\\
265.01	0.01\\
266.01	0.01\\
267.01	0.01\\
268.01	0.01\\
269.01	0.01\\
270.01	0.01\\
271.01	0.01\\
272.01	0.01\\
273.01	0.01\\
274.01	0.01\\
275.01	0.01\\
276.01	0.01\\
277.01	0.01\\
278.01	0.01\\
279.01	0.01\\
280.01	0.01\\
281.01	0.01\\
282.01	0.01\\
283.01	0.01\\
284.01	0.01\\
285.01	0.01\\
286.01	0.01\\
287.01	0.01\\
288.01	0.01\\
289.01	0.01\\
290.01	0.01\\
291.01	0.01\\
292.01	0.01\\
293.01	0.01\\
294.01	0.01\\
295.01	0.01\\
296.01	0.01\\
297.01	0.01\\
298.01	0.01\\
299.01	0.01\\
300.01	0.01\\
301.01	0.01\\
302.01	0.01\\
303.01	0.01\\
304.01	0.01\\
305.01	0.01\\
306.01	0.01\\
307.01	0.01\\
308.01	0.01\\
309.01	0.01\\
310.01	0.01\\
311.01	0.01\\
312.01	0.01\\
313.01	0.01\\
314.01	0.01\\
315.01	0.01\\
316.01	0.01\\
317.01	0.01\\
318.01	0.01\\
319.01	0.01\\
320.01	0.01\\
321.01	0.01\\
322.01	0.01\\
323.01	0.01\\
324.01	0.01\\
325.01	0.01\\
326.01	0.01\\
327.01	0.01\\
328.01	0.01\\
329.01	0.01\\
330.01	0.01\\
331.01	0.01\\
332.01	0.01\\
333.01	0.01\\
334.01	0.01\\
335.01	0.01\\
336.01	0.01\\
337.01	0.01\\
338.01	0.01\\
339.01	0.01\\
340.01	0.01\\
341.01	0.01\\
342.01	0.01\\
343.01	0.01\\
344.01	0.01\\
345.01	0.01\\
346.01	0.01\\
347.01	0.01\\
348.01	0.01\\
349.01	0.01\\
350.01	0.01\\
351.01	0.01\\
352.01	0.01\\
353.01	0.01\\
354.01	0.01\\
355.01	0.01\\
356.01	0.01\\
357.01	0.01\\
358.01	0.01\\
359.01	0.01\\
360.01	0.01\\
361.01	0.01\\
362.01	0.01\\
363.01	0.01\\
364.01	0.01\\
365.01	0.01\\
366.01	0.01\\
367.01	0.01\\
368.01	0.01\\
369.01	0.01\\
370.01	0.01\\
371.01	0.01\\
372.01	0.01\\
373.01	0.01\\
374.01	0.01\\
375.01	0.01\\
376.01	0.01\\
377.01	0.01\\
378.01	0.01\\
379.01	0.01\\
380.01	0.01\\
381.01	0.01\\
382.01	0.01\\
383.01	0.01\\
384.01	0.01\\
385.01	0.01\\
386.01	0.01\\
387.01	0.01\\
388.01	0.01\\
389.01	0.01\\
390.01	0.01\\
391.01	0.01\\
392.01	0.01\\
393.01	0.01\\
394.01	0.01\\
395.01	0.01\\
396.01	0.01\\
397.01	0.01\\
398.01	0.01\\
399.01	0.01\\
400.01	0.01\\
401.01	0.01\\
402.01	0.01\\
403.01	0.01\\
404.01	0.01\\
405.01	0.01\\
406.01	0.01\\
407.01	0.01\\
408.01	0.01\\
409.01	0.01\\
410.01	0.01\\
411.01	0.01\\
412.01	0.01\\
413.01	0.01\\
414.01	0.01\\
415.01	0.01\\
416.01	0.01\\
417.01	0.01\\
418.01	0.01\\
419.01	0.01\\
420.01	0.01\\
421.01	0.01\\
422.01	0.01\\
423.01	0.01\\
424.01	0.01\\
425.01	0.01\\
426.01	0.01\\
427.01	0.01\\
428.01	0.01\\
429.01	0.01\\
430.01	0.01\\
431.01	0.01\\
432.01	0.01\\
433.01	0.01\\
434.01	0.01\\
435.01	0.01\\
436.01	0.01\\
437.01	0.01\\
438.01	0.01\\
439.01	0.01\\
440.01	0.01\\
441.01	0.01\\
442.01	0.01\\
443.01	0.01\\
444.01	0.01\\
445.01	0.01\\
446.01	0.01\\
447.01	0.01\\
448.01	0.01\\
449.01	0.01\\
450.01	0.01\\
451.01	0.01\\
452.01	0.01\\
453.01	0.01\\
454.01	0.01\\
455.01	0.01\\
456.01	0.01\\
457.01	0.01\\
458.01	0.01\\
459.01	0.01\\
460.01	0.01\\
461.01	0.01\\
462.01	0.01\\
463.01	0.01\\
464.01	0.01\\
465.01	0.01\\
466.01	0.01\\
467.01	0.01\\
468.01	0.01\\
469.01	0.01\\
470.01	0.01\\
471.01	0.01\\
472.01	0.01\\
473.01	0.01\\
474.01	0.01\\
475.01	0.01\\
476.01	0.01\\
477.01	0.01\\
478.01	0.01\\
479.01	0.01\\
480.01	0.01\\
481.01	0.01\\
482.01	0.01\\
483.01	0.01\\
484.01	0.01\\
485.01	0.01\\
486.01	0.01\\
487.01	0.01\\
488.01	0.01\\
489.01	0.01\\
490.01	0.01\\
491.01	0.01\\
492.01	0.01\\
493.01	0.01\\
494.01	0.01\\
495.01	0.01\\
496.01	0.01\\
497.01	0.01\\
498.01	0.01\\
499.01	0.01\\
500.01	0.01\\
501.01	0.01\\
502.01	0.01\\
503.01	0.01\\
504.01	0.01\\
505.01	0.01\\
506.01	0.01\\
507.01	0.01\\
508.01	0.01\\
509.01	0.01\\
510.01	0.01\\
511.01	0.01\\
512.01	0.01\\
513.01	0.01\\
514.01	0.01\\
515.01	0.01\\
516.01	0.01\\
517.01	0.01\\
518.01	0.01\\
519.01	0.01\\
520.01	0.01\\
521.01	0.01\\
522.01	0.01\\
523.01	0.01\\
524.01	0.01\\
525.01	0.01\\
526.01	0.01\\
527.01	0.01\\
528.01	0.01\\
529.01	0.01\\
530.01	0.01\\
531.01	0.01\\
532.01	0.01\\
533.01	0.01\\
534.01	0.01\\
535.01	0.01\\
536.01	0.01\\
537.01	0.01\\
538.01	0.01\\
539.01	0.01\\
540.01	0.01\\
541.01	0.01\\
542.01	0.01\\
543.01	0.01\\
544.01	0.01\\
545.01	0.01\\
546.01	0.01\\
547.01	0.01\\
548.01	0.01\\
549.01	0.01\\
550.01	0.01\\
551.01	0.01\\
552.01	0.01\\
553.01	0.01\\
554.01	0.01\\
555.01	0.01\\
556.01	0.01\\
557.01	0.01\\
558.01	0.01\\
559.01	0.01\\
560.01	0.01\\
561.01	0.01\\
562.01	0.01\\
563.01	0.01\\
564.01	0.01\\
565.01	0.01\\
566.01	0.01\\
567.01	0.01\\
568.01	0.01\\
569.01	0.01\\
570.01	0.01\\
571.01	0.01\\
572.01	0.01\\
573.01	0.01\\
574.01	0.01\\
575.01	0.01\\
576.01	0.01\\
577.01	0.01\\
578.01	0.01\\
579.01	0.01\\
580.01	0.01\\
581.01	0.01\\
582.01	0.01\\
583.01	0.01\\
584.01	0.01\\
585.01	0.01\\
586.01	0.01\\
587.01	0.01\\
588.01	0.01\\
589.01	0.01\\
590.01	0.01\\
591.01	0.01\\
592.01	0.01\\
593.01	0.01\\
594.01	0.01\\
595.01	0.01\\
596.01	0.01\\
597.01	0.01\\
598.01	0.01\\
599.01	0.01\\
599.02	0.01\\
599.03	0.01\\
599.04	0.01\\
599.05	0.01\\
599.06	0.01\\
599.07	0.01\\
599.08	0.01\\
599.09	0.01\\
599.1	0.01\\
599.11	0.01\\
599.12	0.01\\
599.13	0.01\\
599.14	0.01\\
599.15	0.01\\
599.16	0.01\\
599.17	0.01\\
599.18	0.01\\
599.19	0.01\\
599.2	0.01\\
599.21	0.01\\
599.22	0.01\\
599.23	0.01\\
599.24	0.01\\
599.25	0.01\\
599.26	0.01\\
599.27	0.01\\
599.28	0.01\\
599.29	0.01\\
599.3	0.01\\
599.31	0.01\\
599.32	0.01\\
599.33	0.01\\
599.34	0.01\\
599.35	0.01\\
599.36	0.01\\
599.37	0.01\\
599.38	0.01\\
599.39	0.01\\
599.4	0.01\\
599.41	0.01\\
599.42	0.01\\
599.43	0.01\\
599.44	0.01\\
599.45	0.01\\
599.46	0.01\\
599.47	0.01\\
599.48	0.01\\
599.49	0.01\\
599.5	0.01\\
599.51	0.01\\
599.52	0.01\\
599.53	0.01\\
599.54	0.01\\
599.55	0.01\\
599.56	0.01\\
599.57	0.01\\
599.58	0.01\\
599.59	0.01\\
599.6	0.01\\
599.61	0.01\\
599.62	0.01\\
599.63	0.01\\
599.64	0.01\\
599.65	0.01\\
599.66	0.01\\
599.67	0.01\\
599.68	0.01\\
599.69	0.01\\
599.7	0.01\\
599.71	0.01\\
599.72	0.01\\
599.73	0.01\\
599.74	0.01\\
599.75	0.01\\
599.76	0.01\\
599.77	0.01\\
599.78	0.01\\
599.79	0.01\\
599.8	0.01\\
599.81	0.01\\
599.82	0.01\\
599.83	0.01\\
599.84	0.01\\
599.85	0.01\\
599.86	0.01\\
599.87	0.01\\
599.88	0.01\\
599.89	0.01\\
599.9	0.01\\
599.91	0.01\\
599.92	0.01\\
599.93	0.01\\
599.94	0.01\\
599.95	0.01\\
599.96	0.01\\
599.97	0.01\\
599.98	0.01\\
599.99	0.01\\
600	0.01\\
};
\addplot [color=blue!50!mycolor7,solid,forget plot]
  table[row sep=crcr]{%
0.01	0.01\\
1.01	0.01\\
2.01	0.01\\
3.01	0.01\\
4.01	0.01\\
5.01	0.01\\
6.01	0.01\\
7.01	0.01\\
8.01	0.01\\
9.01	0.01\\
10.01	0.01\\
11.01	0.01\\
12.01	0.01\\
13.01	0.01\\
14.01	0.01\\
15.01	0.01\\
16.01	0.01\\
17.01	0.01\\
18.01	0.01\\
19.01	0.01\\
20.01	0.01\\
21.01	0.01\\
22.01	0.01\\
23.01	0.01\\
24.01	0.01\\
25.01	0.01\\
26.01	0.01\\
27.01	0.01\\
28.01	0.01\\
29.01	0.01\\
30.01	0.01\\
31.01	0.01\\
32.01	0.01\\
33.01	0.01\\
34.01	0.01\\
35.01	0.01\\
36.01	0.01\\
37.01	0.01\\
38.01	0.01\\
39.01	0.01\\
40.01	0.01\\
41.01	0.01\\
42.01	0.01\\
43.01	0.01\\
44.01	0.01\\
45.01	0.01\\
46.01	0.01\\
47.01	0.01\\
48.01	0.01\\
49.01	0.01\\
50.01	0.01\\
51.01	0.01\\
52.01	0.01\\
53.01	0.01\\
54.01	0.01\\
55.01	0.01\\
56.01	0.01\\
57.01	0.01\\
58.01	0.01\\
59.01	0.01\\
60.01	0.01\\
61.01	0.01\\
62.01	0.01\\
63.01	0.01\\
64.01	0.01\\
65.01	0.01\\
66.01	0.01\\
67.01	0.01\\
68.01	0.01\\
69.01	0.01\\
70.01	0.01\\
71.01	0.01\\
72.01	0.01\\
73.01	0.01\\
74.01	0.01\\
75.01	0.01\\
76.01	0.01\\
77.01	0.01\\
78.01	0.01\\
79.01	0.01\\
80.01	0.01\\
81.01	0.01\\
82.01	0.01\\
83.01	0.01\\
84.01	0.01\\
85.01	0.01\\
86.01	0.01\\
87.01	0.01\\
88.01	0.01\\
89.01	0.01\\
90.01	0.01\\
91.01	0.01\\
92.01	0.01\\
93.01	0.01\\
94.01	0.01\\
95.01	0.01\\
96.01	0.01\\
97.01	0.01\\
98.01	0.01\\
99.01	0.01\\
100.01	0.01\\
101.01	0.01\\
102.01	0.01\\
103.01	0.01\\
104.01	0.01\\
105.01	0.01\\
106.01	0.01\\
107.01	0.01\\
108.01	0.01\\
109.01	0.01\\
110.01	0.01\\
111.01	0.01\\
112.01	0.01\\
113.01	0.01\\
114.01	0.01\\
115.01	0.01\\
116.01	0.01\\
117.01	0.01\\
118.01	0.01\\
119.01	0.01\\
120.01	0.01\\
121.01	0.01\\
122.01	0.01\\
123.01	0.01\\
124.01	0.01\\
125.01	0.01\\
126.01	0.01\\
127.01	0.01\\
128.01	0.01\\
129.01	0.01\\
130.01	0.01\\
131.01	0.01\\
132.01	0.01\\
133.01	0.01\\
134.01	0.01\\
135.01	0.01\\
136.01	0.01\\
137.01	0.01\\
138.01	0.01\\
139.01	0.01\\
140.01	0.01\\
141.01	0.01\\
142.01	0.01\\
143.01	0.01\\
144.01	0.01\\
145.01	0.01\\
146.01	0.01\\
147.01	0.01\\
148.01	0.01\\
149.01	0.01\\
150.01	0.01\\
151.01	0.01\\
152.01	0.01\\
153.01	0.01\\
154.01	0.01\\
155.01	0.01\\
156.01	0.01\\
157.01	0.01\\
158.01	0.01\\
159.01	0.01\\
160.01	0.01\\
161.01	0.01\\
162.01	0.01\\
163.01	0.01\\
164.01	0.01\\
165.01	0.01\\
166.01	0.01\\
167.01	0.01\\
168.01	0.01\\
169.01	0.01\\
170.01	0.01\\
171.01	0.01\\
172.01	0.01\\
173.01	0.01\\
174.01	0.01\\
175.01	0.01\\
176.01	0.01\\
177.01	0.01\\
178.01	0.01\\
179.01	0.01\\
180.01	0.01\\
181.01	0.01\\
182.01	0.01\\
183.01	0.01\\
184.01	0.01\\
185.01	0.01\\
186.01	0.01\\
187.01	0.01\\
188.01	0.01\\
189.01	0.01\\
190.01	0.01\\
191.01	0.01\\
192.01	0.01\\
193.01	0.01\\
194.01	0.01\\
195.01	0.01\\
196.01	0.01\\
197.01	0.01\\
198.01	0.01\\
199.01	0.01\\
200.01	0.01\\
201.01	0.01\\
202.01	0.01\\
203.01	0.01\\
204.01	0.01\\
205.01	0.01\\
206.01	0.01\\
207.01	0.01\\
208.01	0.01\\
209.01	0.01\\
210.01	0.01\\
211.01	0.01\\
212.01	0.01\\
213.01	0.01\\
214.01	0.01\\
215.01	0.01\\
216.01	0.01\\
217.01	0.01\\
218.01	0.01\\
219.01	0.01\\
220.01	0.01\\
221.01	0.01\\
222.01	0.01\\
223.01	0.01\\
224.01	0.01\\
225.01	0.01\\
226.01	0.01\\
227.01	0.01\\
228.01	0.01\\
229.01	0.01\\
230.01	0.01\\
231.01	0.01\\
232.01	0.01\\
233.01	0.01\\
234.01	0.01\\
235.01	0.01\\
236.01	0.01\\
237.01	0.01\\
238.01	0.01\\
239.01	0.01\\
240.01	0.01\\
241.01	0.01\\
242.01	0.01\\
243.01	0.01\\
244.01	0.01\\
245.01	0.01\\
246.01	0.01\\
247.01	0.01\\
248.01	0.01\\
249.01	0.01\\
250.01	0.01\\
251.01	0.01\\
252.01	0.01\\
253.01	0.01\\
254.01	0.01\\
255.01	0.01\\
256.01	0.01\\
257.01	0.01\\
258.01	0.01\\
259.01	0.01\\
260.01	0.01\\
261.01	0.01\\
262.01	0.01\\
263.01	0.01\\
264.01	0.01\\
265.01	0.01\\
266.01	0.01\\
267.01	0.01\\
268.01	0.01\\
269.01	0.01\\
270.01	0.01\\
271.01	0.01\\
272.01	0.01\\
273.01	0.01\\
274.01	0.01\\
275.01	0.01\\
276.01	0.01\\
277.01	0.01\\
278.01	0.01\\
279.01	0.01\\
280.01	0.01\\
281.01	0.01\\
282.01	0.01\\
283.01	0.01\\
284.01	0.01\\
285.01	0.01\\
286.01	0.01\\
287.01	0.01\\
288.01	0.01\\
289.01	0.01\\
290.01	0.01\\
291.01	0.01\\
292.01	0.01\\
293.01	0.01\\
294.01	0.01\\
295.01	0.01\\
296.01	0.01\\
297.01	0.01\\
298.01	0.01\\
299.01	0.01\\
300.01	0.01\\
301.01	0.01\\
302.01	0.01\\
303.01	0.01\\
304.01	0.01\\
305.01	0.01\\
306.01	0.01\\
307.01	0.01\\
308.01	0.01\\
309.01	0.01\\
310.01	0.01\\
311.01	0.01\\
312.01	0.01\\
313.01	0.01\\
314.01	0.01\\
315.01	0.01\\
316.01	0.01\\
317.01	0.01\\
318.01	0.01\\
319.01	0.01\\
320.01	0.01\\
321.01	0.01\\
322.01	0.01\\
323.01	0.01\\
324.01	0.01\\
325.01	0.01\\
326.01	0.01\\
327.01	0.01\\
328.01	0.01\\
329.01	0.01\\
330.01	0.01\\
331.01	0.01\\
332.01	0.01\\
333.01	0.01\\
334.01	0.01\\
335.01	0.01\\
336.01	0.01\\
337.01	0.01\\
338.01	0.01\\
339.01	0.01\\
340.01	0.01\\
341.01	0.01\\
342.01	0.01\\
343.01	0.01\\
344.01	0.01\\
345.01	0.01\\
346.01	0.01\\
347.01	0.01\\
348.01	0.01\\
349.01	0.01\\
350.01	0.01\\
351.01	0.01\\
352.01	0.01\\
353.01	0.01\\
354.01	0.01\\
355.01	0.01\\
356.01	0.01\\
357.01	0.01\\
358.01	0.01\\
359.01	0.01\\
360.01	0.01\\
361.01	0.01\\
362.01	0.01\\
363.01	0.01\\
364.01	0.01\\
365.01	0.01\\
366.01	0.01\\
367.01	0.01\\
368.01	0.01\\
369.01	0.01\\
370.01	0.01\\
371.01	0.01\\
372.01	0.01\\
373.01	0.01\\
374.01	0.01\\
375.01	0.01\\
376.01	0.01\\
377.01	0.01\\
378.01	0.01\\
379.01	0.01\\
380.01	0.01\\
381.01	0.01\\
382.01	0.01\\
383.01	0.01\\
384.01	0.01\\
385.01	0.01\\
386.01	0.01\\
387.01	0.01\\
388.01	0.01\\
389.01	0.01\\
390.01	0.01\\
391.01	0.01\\
392.01	0.01\\
393.01	0.01\\
394.01	0.01\\
395.01	0.01\\
396.01	0.01\\
397.01	0.01\\
398.01	0.01\\
399.01	0.01\\
400.01	0.01\\
401.01	0.01\\
402.01	0.01\\
403.01	0.01\\
404.01	0.01\\
405.01	0.01\\
406.01	0.01\\
407.01	0.01\\
408.01	0.01\\
409.01	0.01\\
410.01	0.01\\
411.01	0.01\\
412.01	0.01\\
413.01	0.01\\
414.01	0.01\\
415.01	0.01\\
416.01	0.01\\
417.01	0.01\\
418.01	0.01\\
419.01	0.01\\
420.01	0.01\\
421.01	0.01\\
422.01	0.01\\
423.01	0.01\\
424.01	0.01\\
425.01	0.01\\
426.01	0.01\\
427.01	0.01\\
428.01	0.01\\
429.01	0.01\\
430.01	0.01\\
431.01	0.01\\
432.01	0.01\\
433.01	0.01\\
434.01	0.01\\
435.01	0.01\\
436.01	0.01\\
437.01	0.01\\
438.01	0.01\\
439.01	0.01\\
440.01	0.01\\
441.01	0.01\\
442.01	0.01\\
443.01	0.01\\
444.01	0.01\\
445.01	0.01\\
446.01	0.01\\
447.01	0.01\\
448.01	0.01\\
449.01	0.01\\
450.01	0.01\\
451.01	0.01\\
452.01	0.01\\
453.01	0.01\\
454.01	0.01\\
455.01	0.01\\
456.01	0.01\\
457.01	0.01\\
458.01	0.01\\
459.01	0.01\\
460.01	0.01\\
461.01	0.01\\
462.01	0.01\\
463.01	0.01\\
464.01	0.01\\
465.01	0.01\\
466.01	0.01\\
467.01	0.01\\
468.01	0.01\\
469.01	0.01\\
470.01	0.01\\
471.01	0.01\\
472.01	0.01\\
473.01	0.01\\
474.01	0.01\\
475.01	0.01\\
476.01	0.01\\
477.01	0.01\\
478.01	0.01\\
479.01	0.01\\
480.01	0.01\\
481.01	0.01\\
482.01	0.01\\
483.01	0.01\\
484.01	0.01\\
485.01	0.01\\
486.01	0.01\\
487.01	0.01\\
488.01	0.01\\
489.01	0.01\\
490.01	0.01\\
491.01	0.01\\
492.01	0.01\\
493.01	0.01\\
494.01	0.01\\
495.01	0.01\\
496.01	0.01\\
497.01	0.01\\
498.01	0.01\\
499.01	0.01\\
500.01	0.01\\
501.01	0.01\\
502.01	0.01\\
503.01	0.01\\
504.01	0.01\\
505.01	0.01\\
506.01	0.01\\
507.01	0.01\\
508.01	0.01\\
509.01	0.01\\
510.01	0.01\\
511.01	0.01\\
512.01	0.01\\
513.01	0.01\\
514.01	0.01\\
515.01	0.01\\
516.01	0.01\\
517.01	0.01\\
518.01	0.01\\
519.01	0.01\\
520.01	0.01\\
521.01	0.01\\
522.01	0.01\\
523.01	0.01\\
524.01	0.01\\
525.01	0.01\\
526.01	0.01\\
527.01	0.01\\
528.01	0.01\\
529.01	0.01\\
530.01	0.01\\
531.01	0.01\\
532.01	0.01\\
533.01	0.01\\
534.01	0.01\\
535.01	0.01\\
536.01	0.01\\
537.01	0.01\\
538.01	0.01\\
539.01	0.01\\
540.01	0.01\\
541.01	0.01\\
542.01	0.01\\
543.01	0.01\\
544.01	0.01\\
545.01	0.01\\
546.01	0.01\\
547.01	0.01\\
548.01	0.01\\
549.01	0.01\\
550.01	0.01\\
551.01	0.01\\
552.01	0.01\\
553.01	0.01\\
554.01	0.01\\
555.01	0.01\\
556.01	0.01\\
557.01	0.01\\
558.01	0.01\\
559.01	0.01\\
560.01	0.01\\
561.01	0.01\\
562.01	0.01\\
563.01	0.01\\
564.01	0.01\\
565.01	0.01\\
566.01	0.01\\
567.01	0.01\\
568.01	0.01\\
569.01	0.01\\
570.01	0.01\\
571.01	0.01\\
572.01	0.01\\
573.01	0.01\\
574.01	0.01\\
575.01	0.01\\
576.01	0.01\\
577.01	0.01\\
578.01	0.01\\
579.01	0.01\\
580.01	0.01\\
581.01	0.01\\
582.01	0.01\\
583.01	0.01\\
584.01	0.01\\
585.01	0.01\\
586.01	0.01\\
587.01	0.01\\
588.01	0.01\\
589.01	0.01\\
590.01	0.01\\
591.01	0.01\\
592.01	0.01\\
593.01	0.01\\
594.01	0.01\\
595.01	0.01\\
596.01	0.01\\
597.01	0.01\\
598.01	0.01\\
599.01	0.01\\
599.02	0.01\\
599.03	0.01\\
599.04	0.01\\
599.05	0.01\\
599.06	0.01\\
599.07	0.01\\
599.08	0.01\\
599.09	0.01\\
599.1	0.01\\
599.11	0.01\\
599.12	0.01\\
599.13	0.01\\
599.14	0.01\\
599.15	0.01\\
599.16	0.01\\
599.17	0.01\\
599.18	0.01\\
599.19	0.01\\
599.2	0.01\\
599.21	0.01\\
599.22	0.01\\
599.23	0.01\\
599.24	0.01\\
599.25	0.01\\
599.26	0.01\\
599.27	0.01\\
599.28	0.01\\
599.29	0.01\\
599.3	0.01\\
599.31	0.01\\
599.32	0.01\\
599.33	0.01\\
599.34	0.01\\
599.35	0.01\\
599.36	0.01\\
599.37	0.01\\
599.38	0.01\\
599.39	0.01\\
599.4	0.01\\
599.41	0.01\\
599.42	0.01\\
599.43	0.01\\
599.44	0.01\\
599.45	0.01\\
599.46	0.01\\
599.47	0.01\\
599.48	0.01\\
599.49	0.01\\
599.5	0.01\\
599.51	0.01\\
599.52	0.01\\
599.53	0.01\\
599.54	0.01\\
599.55	0.01\\
599.56	0.01\\
599.57	0.01\\
599.58	0.01\\
599.59	0.01\\
599.6	0.01\\
599.61	0.01\\
599.62	0.01\\
599.63	0.01\\
599.64	0.01\\
599.65	0.01\\
599.66	0.01\\
599.67	0.01\\
599.68	0.01\\
599.69	0.01\\
599.7	0.01\\
599.71	0.01\\
599.72	0.01\\
599.73	0.01\\
599.74	0.01\\
599.75	0.01\\
599.76	0.01\\
599.77	0.01\\
599.78	0.01\\
599.79	0.01\\
599.8	0.01\\
599.81	0.01\\
599.82	0.01\\
599.83	0.01\\
599.84	0.01\\
599.85	0.01\\
599.86	0.01\\
599.87	0.01\\
599.88	0.01\\
599.89	0.01\\
599.9	0.01\\
599.91	0.01\\
599.92	0.01\\
599.93	0.01\\
599.94	0.01\\
599.95	0.01\\
599.96	0.01\\
599.97	0.01\\
599.98	0.01\\
599.99	0.01\\
600	0.01\\
};
\addplot [color=blue!40!mycolor9,solid,forget plot]
  table[row sep=crcr]{%
0.01	0.01\\
1.01	0.01\\
2.01	0.01\\
3.01	0.01\\
4.01	0.01\\
5.01	0.01\\
6.01	0.01\\
7.01	0.01\\
8.01	0.01\\
9.01	0.01\\
10.01	0.01\\
11.01	0.01\\
12.01	0.01\\
13.01	0.01\\
14.01	0.01\\
15.01	0.01\\
16.01	0.01\\
17.01	0.01\\
18.01	0.01\\
19.01	0.01\\
20.01	0.01\\
21.01	0.01\\
22.01	0.01\\
23.01	0.01\\
24.01	0.01\\
25.01	0.01\\
26.01	0.01\\
27.01	0.01\\
28.01	0.01\\
29.01	0.01\\
30.01	0.01\\
31.01	0.01\\
32.01	0.01\\
33.01	0.01\\
34.01	0.01\\
35.01	0.01\\
36.01	0.01\\
37.01	0.01\\
38.01	0.01\\
39.01	0.01\\
40.01	0.01\\
41.01	0.01\\
42.01	0.01\\
43.01	0.01\\
44.01	0.01\\
45.01	0.01\\
46.01	0.01\\
47.01	0.01\\
48.01	0.01\\
49.01	0.01\\
50.01	0.01\\
51.01	0.01\\
52.01	0.01\\
53.01	0.01\\
54.01	0.01\\
55.01	0.01\\
56.01	0.01\\
57.01	0.01\\
58.01	0.01\\
59.01	0.01\\
60.01	0.01\\
61.01	0.01\\
62.01	0.01\\
63.01	0.01\\
64.01	0.01\\
65.01	0.01\\
66.01	0.01\\
67.01	0.01\\
68.01	0.01\\
69.01	0.01\\
70.01	0.01\\
71.01	0.01\\
72.01	0.01\\
73.01	0.01\\
74.01	0.01\\
75.01	0.01\\
76.01	0.01\\
77.01	0.01\\
78.01	0.01\\
79.01	0.01\\
80.01	0.01\\
81.01	0.01\\
82.01	0.01\\
83.01	0.01\\
84.01	0.01\\
85.01	0.01\\
86.01	0.01\\
87.01	0.01\\
88.01	0.01\\
89.01	0.01\\
90.01	0.01\\
91.01	0.01\\
92.01	0.01\\
93.01	0.01\\
94.01	0.01\\
95.01	0.01\\
96.01	0.01\\
97.01	0.01\\
98.01	0.01\\
99.01	0.01\\
100.01	0.01\\
101.01	0.01\\
102.01	0.01\\
103.01	0.01\\
104.01	0.01\\
105.01	0.01\\
106.01	0.01\\
107.01	0.01\\
108.01	0.01\\
109.01	0.01\\
110.01	0.01\\
111.01	0.01\\
112.01	0.01\\
113.01	0.01\\
114.01	0.01\\
115.01	0.01\\
116.01	0.01\\
117.01	0.01\\
118.01	0.01\\
119.01	0.01\\
120.01	0.01\\
121.01	0.01\\
122.01	0.01\\
123.01	0.01\\
124.01	0.01\\
125.01	0.01\\
126.01	0.01\\
127.01	0.01\\
128.01	0.01\\
129.01	0.01\\
130.01	0.01\\
131.01	0.01\\
132.01	0.01\\
133.01	0.01\\
134.01	0.01\\
135.01	0.01\\
136.01	0.01\\
137.01	0.01\\
138.01	0.01\\
139.01	0.01\\
140.01	0.01\\
141.01	0.01\\
142.01	0.01\\
143.01	0.01\\
144.01	0.01\\
145.01	0.01\\
146.01	0.01\\
147.01	0.01\\
148.01	0.01\\
149.01	0.01\\
150.01	0.01\\
151.01	0.01\\
152.01	0.01\\
153.01	0.01\\
154.01	0.01\\
155.01	0.01\\
156.01	0.01\\
157.01	0.01\\
158.01	0.01\\
159.01	0.01\\
160.01	0.01\\
161.01	0.01\\
162.01	0.01\\
163.01	0.01\\
164.01	0.01\\
165.01	0.01\\
166.01	0.01\\
167.01	0.01\\
168.01	0.01\\
169.01	0.01\\
170.01	0.01\\
171.01	0.01\\
172.01	0.01\\
173.01	0.01\\
174.01	0.01\\
175.01	0.01\\
176.01	0.01\\
177.01	0.01\\
178.01	0.01\\
179.01	0.01\\
180.01	0.01\\
181.01	0.01\\
182.01	0.01\\
183.01	0.01\\
184.01	0.01\\
185.01	0.01\\
186.01	0.01\\
187.01	0.01\\
188.01	0.01\\
189.01	0.01\\
190.01	0.01\\
191.01	0.01\\
192.01	0.01\\
193.01	0.01\\
194.01	0.01\\
195.01	0.01\\
196.01	0.01\\
197.01	0.01\\
198.01	0.01\\
199.01	0.01\\
200.01	0.01\\
201.01	0.01\\
202.01	0.01\\
203.01	0.01\\
204.01	0.01\\
205.01	0.01\\
206.01	0.01\\
207.01	0.01\\
208.01	0.01\\
209.01	0.01\\
210.01	0.01\\
211.01	0.01\\
212.01	0.01\\
213.01	0.01\\
214.01	0.01\\
215.01	0.01\\
216.01	0.01\\
217.01	0.01\\
218.01	0.01\\
219.01	0.01\\
220.01	0.01\\
221.01	0.01\\
222.01	0.01\\
223.01	0.01\\
224.01	0.01\\
225.01	0.01\\
226.01	0.01\\
227.01	0.01\\
228.01	0.01\\
229.01	0.01\\
230.01	0.01\\
231.01	0.01\\
232.01	0.01\\
233.01	0.01\\
234.01	0.01\\
235.01	0.01\\
236.01	0.01\\
237.01	0.01\\
238.01	0.01\\
239.01	0.01\\
240.01	0.01\\
241.01	0.01\\
242.01	0.01\\
243.01	0.01\\
244.01	0.01\\
245.01	0.01\\
246.01	0.01\\
247.01	0.01\\
248.01	0.01\\
249.01	0.01\\
250.01	0.01\\
251.01	0.01\\
252.01	0.01\\
253.01	0.01\\
254.01	0.01\\
255.01	0.01\\
256.01	0.01\\
257.01	0.01\\
258.01	0.01\\
259.01	0.01\\
260.01	0.01\\
261.01	0.01\\
262.01	0.01\\
263.01	0.01\\
264.01	0.01\\
265.01	0.01\\
266.01	0.01\\
267.01	0.01\\
268.01	0.01\\
269.01	0.01\\
270.01	0.01\\
271.01	0.01\\
272.01	0.01\\
273.01	0.01\\
274.01	0.01\\
275.01	0.01\\
276.01	0.01\\
277.01	0.01\\
278.01	0.01\\
279.01	0.01\\
280.01	0.01\\
281.01	0.01\\
282.01	0.01\\
283.01	0.01\\
284.01	0.01\\
285.01	0.01\\
286.01	0.01\\
287.01	0.01\\
288.01	0.01\\
289.01	0.01\\
290.01	0.01\\
291.01	0.01\\
292.01	0.01\\
293.01	0.01\\
294.01	0.01\\
295.01	0.01\\
296.01	0.01\\
297.01	0.01\\
298.01	0.01\\
299.01	0.01\\
300.01	0.01\\
301.01	0.01\\
302.01	0.01\\
303.01	0.01\\
304.01	0.01\\
305.01	0.01\\
306.01	0.01\\
307.01	0.01\\
308.01	0.01\\
309.01	0.01\\
310.01	0.01\\
311.01	0.01\\
312.01	0.01\\
313.01	0.01\\
314.01	0.01\\
315.01	0.01\\
316.01	0.01\\
317.01	0.01\\
318.01	0.01\\
319.01	0.01\\
320.01	0.01\\
321.01	0.01\\
322.01	0.01\\
323.01	0.01\\
324.01	0.01\\
325.01	0.01\\
326.01	0.01\\
327.01	0.01\\
328.01	0.01\\
329.01	0.01\\
330.01	0.01\\
331.01	0.01\\
332.01	0.01\\
333.01	0.01\\
334.01	0.01\\
335.01	0.01\\
336.01	0.01\\
337.01	0.01\\
338.01	0.01\\
339.01	0.01\\
340.01	0.01\\
341.01	0.01\\
342.01	0.01\\
343.01	0.01\\
344.01	0.01\\
345.01	0.01\\
346.01	0.01\\
347.01	0.01\\
348.01	0.01\\
349.01	0.01\\
350.01	0.01\\
351.01	0.01\\
352.01	0.01\\
353.01	0.01\\
354.01	0.01\\
355.01	0.01\\
356.01	0.01\\
357.01	0.01\\
358.01	0.01\\
359.01	0.01\\
360.01	0.01\\
361.01	0.01\\
362.01	0.01\\
363.01	0.01\\
364.01	0.01\\
365.01	0.01\\
366.01	0.01\\
367.01	0.01\\
368.01	0.01\\
369.01	0.01\\
370.01	0.01\\
371.01	0.01\\
372.01	0.01\\
373.01	0.01\\
374.01	0.01\\
375.01	0.01\\
376.01	0.01\\
377.01	0.01\\
378.01	0.01\\
379.01	0.01\\
380.01	0.01\\
381.01	0.01\\
382.01	0.01\\
383.01	0.01\\
384.01	0.01\\
385.01	0.01\\
386.01	0.01\\
387.01	0.01\\
388.01	0.01\\
389.01	0.01\\
390.01	0.01\\
391.01	0.01\\
392.01	0.01\\
393.01	0.01\\
394.01	0.01\\
395.01	0.01\\
396.01	0.01\\
397.01	0.01\\
398.01	0.01\\
399.01	0.01\\
400.01	0.01\\
401.01	0.01\\
402.01	0.01\\
403.01	0.01\\
404.01	0.01\\
405.01	0.01\\
406.01	0.01\\
407.01	0.01\\
408.01	0.01\\
409.01	0.01\\
410.01	0.01\\
411.01	0.01\\
412.01	0.01\\
413.01	0.01\\
414.01	0.01\\
415.01	0.01\\
416.01	0.01\\
417.01	0.01\\
418.01	0.01\\
419.01	0.01\\
420.01	0.01\\
421.01	0.01\\
422.01	0.01\\
423.01	0.01\\
424.01	0.01\\
425.01	0.01\\
426.01	0.01\\
427.01	0.01\\
428.01	0.01\\
429.01	0.01\\
430.01	0.01\\
431.01	0.01\\
432.01	0.01\\
433.01	0.01\\
434.01	0.01\\
435.01	0.01\\
436.01	0.01\\
437.01	0.01\\
438.01	0.01\\
439.01	0.01\\
440.01	0.01\\
441.01	0.01\\
442.01	0.01\\
443.01	0.01\\
444.01	0.01\\
445.01	0.01\\
446.01	0.01\\
447.01	0.01\\
448.01	0.01\\
449.01	0.01\\
450.01	0.01\\
451.01	0.01\\
452.01	0.01\\
453.01	0.01\\
454.01	0.01\\
455.01	0.01\\
456.01	0.01\\
457.01	0.01\\
458.01	0.01\\
459.01	0.01\\
460.01	0.01\\
461.01	0.01\\
462.01	0.01\\
463.01	0.01\\
464.01	0.01\\
465.01	0.01\\
466.01	0.01\\
467.01	0.01\\
468.01	0.01\\
469.01	0.01\\
470.01	0.01\\
471.01	0.01\\
472.01	0.01\\
473.01	0.01\\
474.01	0.01\\
475.01	0.01\\
476.01	0.01\\
477.01	0.01\\
478.01	0.01\\
479.01	0.01\\
480.01	0.01\\
481.01	0.01\\
482.01	0.01\\
483.01	0.01\\
484.01	0.01\\
485.01	0.01\\
486.01	0.01\\
487.01	0.01\\
488.01	0.01\\
489.01	0.01\\
490.01	0.01\\
491.01	0.01\\
492.01	0.01\\
493.01	0.01\\
494.01	0.01\\
495.01	0.01\\
496.01	0.01\\
497.01	0.01\\
498.01	0.01\\
499.01	0.01\\
500.01	0.01\\
501.01	0.01\\
502.01	0.01\\
503.01	0.01\\
504.01	0.01\\
505.01	0.01\\
506.01	0.01\\
507.01	0.01\\
508.01	0.01\\
509.01	0.01\\
510.01	0.01\\
511.01	0.01\\
512.01	0.01\\
513.01	0.01\\
514.01	0.01\\
515.01	0.01\\
516.01	0.01\\
517.01	0.01\\
518.01	0.01\\
519.01	0.01\\
520.01	0.01\\
521.01	0.01\\
522.01	0.01\\
523.01	0.01\\
524.01	0.01\\
525.01	0.01\\
526.01	0.01\\
527.01	0.01\\
528.01	0.01\\
529.01	0.01\\
530.01	0.01\\
531.01	0.01\\
532.01	0.01\\
533.01	0.01\\
534.01	0.01\\
535.01	0.01\\
536.01	0.01\\
537.01	0.01\\
538.01	0.01\\
539.01	0.01\\
540.01	0.01\\
541.01	0.01\\
542.01	0.01\\
543.01	0.01\\
544.01	0.01\\
545.01	0.01\\
546.01	0.01\\
547.01	0.01\\
548.01	0.01\\
549.01	0.01\\
550.01	0.01\\
551.01	0.01\\
552.01	0.01\\
553.01	0.01\\
554.01	0.01\\
555.01	0.01\\
556.01	0.01\\
557.01	0.01\\
558.01	0.01\\
559.01	0.01\\
560.01	0.01\\
561.01	0.01\\
562.01	0.01\\
563.01	0.01\\
564.01	0.01\\
565.01	0.01\\
566.01	0.01\\
567.01	0.01\\
568.01	0.01\\
569.01	0.01\\
570.01	0.01\\
571.01	0.01\\
572.01	0.01\\
573.01	0.01\\
574.01	0.01\\
575.01	0.01\\
576.01	0.01\\
577.01	0.01\\
578.01	0.01\\
579.01	0.01\\
580.01	0.01\\
581.01	0.01\\
582.01	0.01\\
583.01	0.01\\
584.01	0.01\\
585.01	0.01\\
586.01	0.01\\
587.01	0.01\\
588.01	0.01\\
589.01	0.01\\
590.01	0.01\\
591.01	0.01\\
592.01	0.01\\
593.01	0.01\\
594.01	0.01\\
595.01	0.01\\
596.01	0.01\\
597.01	0.01\\
598.01	0.01\\
599.01	0.01\\
599.02	0.01\\
599.03	0.01\\
599.04	0.01\\
599.05	0.01\\
599.06	0.01\\
599.07	0.01\\
599.08	0.01\\
599.09	0.01\\
599.1	0.01\\
599.11	0.01\\
599.12	0.01\\
599.13	0.01\\
599.14	0.01\\
599.15	0.01\\
599.16	0.01\\
599.17	0.01\\
599.18	0.01\\
599.19	0.01\\
599.2	0.01\\
599.21	0.01\\
599.22	0.01\\
599.23	0.01\\
599.24	0.01\\
599.25	0.01\\
599.26	0.01\\
599.27	0.01\\
599.28	0.01\\
599.29	0.01\\
599.3	0.01\\
599.31	0.01\\
599.32	0.01\\
599.33	0.01\\
599.34	0.01\\
599.35	0.01\\
599.36	0.01\\
599.37	0.01\\
599.38	0.01\\
599.39	0.01\\
599.4	0.01\\
599.41	0.01\\
599.42	0.01\\
599.43	0.01\\
599.44	0.01\\
599.45	0.01\\
599.46	0.01\\
599.47	0.01\\
599.48	0.01\\
599.49	0.01\\
599.5	0.01\\
599.51	0.01\\
599.52	0.01\\
599.53	0.01\\
599.54	0.01\\
599.55	0.01\\
599.56	0.01\\
599.57	0.01\\
599.58	0.01\\
599.59	0.01\\
599.6	0.01\\
599.61	0.01\\
599.62	0.01\\
599.63	0.01\\
599.64	0.01\\
599.65	0.01\\
599.66	0.01\\
599.67	0.01\\
599.68	0.01\\
599.69	0.01\\
599.7	0.01\\
599.71	0.01\\
599.72	0.01\\
599.73	0.01\\
599.74	0.01\\
599.75	0.01\\
599.76	0.01\\
599.77	0.01\\
599.78	0.01\\
599.79	0.01\\
599.8	0.01\\
599.81	0.01\\
599.82	0.01\\
599.83	0.01\\
599.84	0.01\\
599.85	0.01\\
599.86	0.01\\
599.87	0.01\\
599.88	0.01\\
599.89	0.01\\
599.9	0.01\\
599.91	0.01\\
599.92	0.01\\
599.93	0.01\\
599.94	0.01\\
599.95	0.01\\
599.96	0.01\\
599.97	0.01\\
599.98	0.01\\
599.99	0.01\\
600	0.01\\
};
\addplot [color=blue!75!mycolor7,solid,forget plot]
  table[row sep=crcr]{%
0.01	0.01\\
1.01	0.01\\
2.01	0.01\\
3.01	0.01\\
4.01	0.01\\
5.01	0.01\\
6.01	0.01\\
7.01	0.01\\
8.01	0.01\\
9.01	0.01\\
10.01	0.01\\
11.01	0.01\\
12.01	0.01\\
13.01	0.01\\
14.01	0.01\\
15.01	0.01\\
16.01	0.01\\
17.01	0.01\\
18.01	0.01\\
19.01	0.01\\
20.01	0.01\\
21.01	0.01\\
22.01	0.01\\
23.01	0.01\\
24.01	0.01\\
25.01	0.01\\
26.01	0.01\\
27.01	0.01\\
28.01	0.01\\
29.01	0.01\\
30.01	0.01\\
31.01	0.01\\
32.01	0.01\\
33.01	0.01\\
34.01	0.01\\
35.01	0.01\\
36.01	0.01\\
37.01	0.01\\
38.01	0.01\\
39.01	0.01\\
40.01	0.01\\
41.01	0.01\\
42.01	0.01\\
43.01	0.01\\
44.01	0.01\\
45.01	0.01\\
46.01	0.01\\
47.01	0.01\\
48.01	0.01\\
49.01	0.01\\
50.01	0.01\\
51.01	0.01\\
52.01	0.01\\
53.01	0.01\\
54.01	0.01\\
55.01	0.01\\
56.01	0.01\\
57.01	0.01\\
58.01	0.01\\
59.01	0.01\\
60.01	0.01\\
61.01	0.01\\
62.01	0.01\\
63.01	0.01\\
64.01	0.01\\
65.01	0.01\\
66.01	0.01\\
67.01	0.01\\
68.01	0.01\\
69.01	0.01\\
70.01	0.01\\
71.01	0.01\\
72.01	0.01\\
73.01	0.01\\
74.01	0.01\\
75.01	0.01\\
76.01	0.01\\
77.01	0.01\\
78.01	0.01\\
79.01	0.01\\
80.01	0.01\\
81.01	0.01\\
82.01	0.01\\
83.01	0.01\\
84.01	0.01\\
85.01	0.01\\
86.01	0.01\\
87.01	0.01\\
88.01	0.01\\
89.01	0.01\\
90.01	0.01\\
91.01	0.01\\
92.01	0.01\\
93.01	0.01\\
94.01	0.01\\
95.01	0.01\\
96.01	0.01\\
97.01	0.01\\
98.01	0.01\\
99.01	0.01\\
100.01	0.01\\
101.01	0.01\\
102.01	0.01\\
103.01	0.01\\
104.01	0.01\\
105.01	0.01\\
106.01	0.01\\
107.01	0.01\\
108.01	0.01\\
109.01	0.01\\
110.01	0.01\\
111.01	0.01\\
112.01	0.01\\
113.01	0.01\\
114.01	0.01\\
115.01	0.01\\
116.01	0.01\\
117.01	0.01\\
118.01	0.01\\
119.01	0.01\\
120.01	0.01\\
121.01	0.01\\
122.01	0.01\\
123.01	0.01\\
124.01	0.01\\
125.01	0.01\\
126.01	0.01\\
127.01	0.01\\
128.01	0.01\\
129.01	0.01\\
130.01	0.01\\
131.01	0.01\\
132.01	0.01\\
133.01	0.01\\
134.01	0.01\\
135.01	0.01\\
136.01	0.01\\
137.01	0.01\\
138.01	0.01\\
139.01	0.01\\
140.01	0.01\\
141.01	0.01\\
142.01	0.01\\
143.01	0.01\\
144.01	0.01\\
145.01	0.01\\
146.01	0.01\\
147.01	0.01\\
148.01	0.01\\
149.01	0.01\\
150.01	0.01\\
151.01	0.01\\
152.01	0.01\\
153.01	0.01\\
154.01	0.01\\
155.01	0.01\\
156.01	0.01\\
157.01	0.01\\
158.01	0.01\\
159.01	0.01\\
160.01	0.01\\
161.01	0.01\\
162.01	0.01\\
163.01	0.01\\
164.01	0.01\\
165.01	0.01\\
166.01	0.01\\
167.01	0.01\\
168.01	0.01\\
169.01	0.01\\
170.01	0.01\\
171.01	0.01\\
172.01	0.01\\
173.01	0.01\\
174.01	0.01\\
175.01	0.01\\
176.01	0.01\\
177.01	0.01\\
178.01	0.01\\
179.01	0.01\\
180.01	0.01\\
181.01	0.01\\
182.01	0.01\\
183.01	0.01\\
184.01	0.01\\
185.01	0.01\\
186.01	0.01\\
187.01	0.01\\
188.01	0.01\\
189.01	0.01\\
190.01	0.01\\
191.01	0.01\\
192.01	0.01\\
193.01	0.01\\
194.01	0.01\\
195.01	0.01\\
196.01	0.01\\
197.01	0.01\\
198.01	0.01\\
199.01	0.01\\
200.01	0.01\\
201.01	0.01\\
202.01	0.01\\
203.01	0.01\\
204.01	0.01\\
205.01	0.01\\
206.01	0.01\\
207.01	0.01\\
208.01	0.01\\
209.01	0.01\\
210.01	0.01\\
211.01	0.01\\
212.01	0.01\\
213.01	0.01\\
214.01	0.01\\
215.01	0.01\\
216.01	0.01\\
217.01	0.01\\
218.01	0.01\\
219.01	0.01\\
220.01	0.01\\
221.01	0.01\\
222.01	0.01\\
223.01	0.01\\
224.01	0.01\\
225.01	0.01\\
226.01	0.01\\
227.01	0.01\\
228.01	0.01\\
229.01	0.01\\
230.01	0.01\\
231.01	0.01\\
232.01	0.01\\
233.01	0.01\\
234.01	0.01\\
235.01	0.01\\
236.01	0.01\\
237.01	0.01\\
238.01	0.01\\
239.01	0.01\\
240.01	0.01\\
241.01	0.01\\
242.01	0.01\\
243.01	0.01\\
244.01	0.01\\
245.01	0.01\\
246.01	0.01\\
247.01	0.01\\
248.01	0.01\\
249.01	0.01\\
250.01	0.01\\
251.01	0.01\\
252.01	0.01\\
253.01	0.01\\
254.01	0.01\\
255.01	0.01\\
256.01	0.01\\
257.01	0.01\\
258.01	0.01\\
259.01	0.01\\
260.01	0.01\\
261.01	0.01\\
262.01	0.01\\
263.01	0.01\\
264.01	0.01\\
265.01	0.01\\
266.01	0.01\\
267.01	0.01\\
268.01	0.01\\
269.01	0.01\\
270.01	0.01\\
271.01	0.01\\
272.01	0.01\\
273.01	0.01\\
274.01	0.01\\
275.01	0.01\\
276.01	0.01\\
277.01	0.01\\
278.01	0.01\\
279.01	0.01\\
280.01	0.01\\
281.01	0.01\\
282.01	0.01\\
283.01	0.01\\
284.01	0.01\\
285.01	0.01\\
286.01	0.01\\
287.01	0.01\\
288.01	0.01\\
289.01	0.01\\
290.01	0.01\\
291.01	0.01\\
292.01	0.01\\
293.01	0.01\\
294.01	0.01\\
295.01	0.01\\
296.01	0.01\\
297.01	0.01\\
298.01	0.01\\
299.01	0.01\\
300.01	0.01\\
301.01	0.01\\
302.01	0.01\\
303.01	0.01\\
304.01	0.01\\
305.01	0.01\\
306.01	0.01\\
307.01	0.01\\
308.01	0.01\\
309.01	0.01\\
310.01	0.01\\
311.01	0.01\\
312.01	0.01\\
313.01	0.01\\
314.01	0.01\\
315.01	0.01\\
316.01	0.01\\
317.01	0.01\\
318.01	0.01\\
319.01	0.01\\
320.01	0.01\\
321.01	0.01\\
322.01	0.01\\
323.01	0.01\\
324.01	0.01\\
325.01	0.01\\
326.01	0.01\\
327.01	0.01\\
328.01	0.01\\
329.01	0.01\\
330.01	0.01\\
331.01	0.01\\
332.01	0.01\\
333.01	0.01\\
334.01	0.01\\
335.01	0.01\\
336.01	0.01\\
337.01	0.01\\
338.01	0.01\\
339.01	0.01\\
340.01	0.01\\
341.01	0.01\\
342.01	0.01\\
343.01	0.01\\
344.01	0.01\\
345.01	0.01\\
346.01	0.01\\
347.01	0.01\\
348.01	0.01\\
349.01	0.01\\
350.01	0.01\\
351.01	0.01\\
352.01	0.01\\
353.01	0.01\\
354.01	0.01\\
355.01	0.01\\
356.01	0.01\\
357.01	0.01\\
358.01	0.01\\
359.01	0.01\\
360.01	0.01\\
361.01	0.01\\
362.01	0.01\\
363.01	0.01\\
364.01	0.01\\
365.01	0.01\\
366.01	0.01\\
367.01	0.01\\
368.01	0.01\\
369.01	0.01\\
370.01	0.01\\
371.01	0.01\\
372.01	0.01\\
373.01	0.01\\
374.01	0.01\\
375.01	0.01\\
376.01	0.01\\
377.01	0.01\\
378.01	0.01\\
379.01	0.01\\
380.01	0.01\\
381.01	0.01\\
382.01	0.01\\
383.01	0.01\\
384.01	0.01\\
385.01	0.01\\
386.01	0.01\\
387.01	0.01\\
388.01	0.01\\
389.01	0.01\\
390.01	0.01\\
391.01	0.01\\
392.01	0.01\\
393.01	0.01\\
394.01	0.01\\
395.01	0.01\\
396.01	0.01\\
397.01	0.01\\
398.01	0.01\\
399.01	0.01\\
400.01	0.01\\
401.01	0.01\\
402.01	0.01\\
403.01	0.01\\
404.01	0.01\\
405.01	0.01\\
406.01	0.01\\
407.01	0.01\\
408.01	0.01\\
409.01	0.01\\
410.01	0.01\\
411.01	0.01\\
412.01	0.01\\
413.01	0.01\\
414.01	0.01\\
415.01	0.01\\
416.01	0.01\\
417.01	0.01\\
418.01	0.01\\
419.01	0.01\\
420.01	0.01\\
421.01	0.01\\
422.01	0.01\\
423.01	0.01\\
424.01	0.01\\
425.01	0.01\\
426.01	0.01\\
427.01	0.01\\
428.01	0.01\\
429.01	0.01\\
430.01	0.01\\
431.01	0.01\\
432.01	0.01\\
433.01	0.01\\
434.01	0.01\\
435.01	0.01\\
436.01	0.01\\
437.01	0.01\\
438.01	0.01\\
439.01	0.01\\
440.01	0.01\\
441.01	0.01\\
442.01	0.01\\
443.01	0.01\\
444.01	0.01\\
445.01	0.01\\
446.01	0.01\\
447.01	0.01\\
448.01	0.01\\
449.01	0.01\\
450.01	0.01\\
451.01	0.01\\
452.01	0.01\\
453.01	0.01\\
454.01	0.01\\
455.01	0.01\\
456.01	0.01\\
457.01	0.01\\
458.01	0.01\\
459.01	0.01\\
460.01	0.01\\
461.01	0.01\\
462.01	0.01\\
463.01	0.01\\
464.01	0.01\\
465.01	0.01\\
466.01	0.01\\
467.01	0.01\\
468.01	0.01\\
469.01	0.01\\
470.01	0.01\\
471.01	0.01\\
472.01	0.01\\
473.01	0.01\\
474.01	0.01\\
475.01	0.01\\
476.01	0.01\\
477.01	0.01\\
478.01	0.01\\
479.01	0.01\\
480.01	0.01\\
481.01	0.01\\
482.01	0.01\\
483.01	0.01\\
484.01	0.01\\
485.01	0.01\\
486.01	0.01\\
487.01	0.01\\
488.01	0.01\\
489.01	0.01\\
490.01	0.01\\
491.01	0.01\\
492.01	0.01\\
493.01	0.01\\
494.01	0.01\\
495.01	0.01\\
496.01	0.01\\
497.01	0.01\\
498.01	0.01\\
499.01	0.01\\
500.01	0.01\\
501.01	0.01\\
502.01	0.01\\
503.01	0.01\\
504.01	0.01\\
505.01	0.01\\
506.01	0.01\\
507.01	0.01\\
508.01	0.01\\
509.01	0.01\\
510.01	0.01\\
511.01	0.01\\
512.01	0.01\\
513.01	0.01\\
514.01	0.01\\
515.01	0.01\\
516.01	0.01\\
517.01	0.01\\
518.01	0.01\\
519.01	0.01\\
520.01	0.01\\
521.01	0.01\\
522.01	0.01\\
523.01	0.01\\
524.01	0.01\\
525.01	0.01\\
526.01	0.01\\
527.01	0.01\\
528.01	0.01\\
529.01	0.01\\
530.01	0.01\\
531.01	0.01\\
532.01	0.01\\
533.01	0.01\\
534.01	0.01\\
535.01	0.01\\
536.01	0.01\\
537.01	0.01\\
538.01	0.01\\
539.01	0.01\\
540.01	0.01\\
541.01	0.01\\
542.01	0.01\\
543.01	0.01\\
544.01	0.01\\
545.01	0.01\\
546.01	0.01\\
547.01	0.01\\
548.01	0.01\\
549.01	0.01\\
550.01	0.01\\
551.01	0.01\\
552.01	0.01\\
553.01	0.01\\
554.01	0.01\\
555.01	0.01\\
556.01	0.01\\
557.01	0.01\\
558.01	0.01\\
559.01	0.01\\
560.01	0.01\\
561.01	0.01\\
562.01	0.01\\
563.01	0.01\\
564.01	0.01\\
565.01	0.01\\
566.01	0.01\\
567.01	0.01\\
568.01	0.01\\
569.01	0.01\\
570.01	0.01\\
571.01	0.01\\
572.01	0.01\\
573.01	0.01\\
574.01	0.01\\
575.01	0.01\\
576.01	0.01\\
577.01	0.01\\
578.01	0.01\\
579.01	0.01\\
580.01	0.01\\
581.01	0.01\\
582.01	0.01\\
583.01	0.01\\
584.01	0.01\\
585.01	0.01\\
586.01	0.01\\
587.01	0.01\\
588.01	0.01\\
589.01	0.01\\
590.01	0.01\\
591.01	0.01\\
592.01	0.01\\
593.01	0.01\\
594.01	0.01\\
595.01	0.01\\
596.01	0.01\\
597.01	0.01\\
598.01	0.01\\
599.01	0.01\\
599.02	0.01\\
599.03	0.01\\
599.04	0.01\\
599.05	0.01\\
599.06	0.01\\
599.07	0.01\\
599.08	0.01\\
599.09	0.01\\
599.1	0.01\\
599.11	0.01\\
599.12	0.01\\
599.13	0.01\\
599.14	0.01\\
599.15	0.01\\
599.16	0.01\\
599.17	0.01\\
599.18	0.01\\
599.19	0.01\\
599.2	0.01\\
599.21	0.01\\
599.22	0.01\\
599.23	0.01\\
599.24	0.01\\
599.25	0.01\\
599.26	0.01\\
599.27	0.01\\
599.28	0.01\\
599.29	0.01\\
599.3	0.01\\
599.31	0.01\\
599.32	0.01\\
599.33	0.01\\
599.34	0.01\\
599.35	0.01\\
599.36	0.01\\
599.37	0.01\\
599.38	0.01\\
599.39	0.01\\
599.4	0.01\\
599.41	0.01\\
599.42	0.01\\
599.43	0.01\\
599.44	0.01\\
599.45	0.01\\
599.46	0.01\\
599.47	0.01\\
599.48	0.01\\
599.49	0.01\\
599.5	0.01\\
599.51	0.01\\
599.52	0.01\\
599.53	0.01\\
599.54	0.01\\
599.55	0.01\\
599.56	0.01\\
599.57	0.01\\
599.58	0.01\\
599.59	0.01\\
599.6	0.01\\
599.61	0.01\\
599.62	0.01\\
599.63	0.01\\
599.64	0.01\\
599.65	0.01\\
599.66	0.01\\
599.67	0.01\\
599.68	0.01\\
599.69	0.01\\
599.7	0.01\\
599.71	0.01\\
599.72	0.01\\
599.73	0.01\\
599.74	0.01\\
599.75	0.01\\
599.76	0.01\\
599.77	0.01\\
599.78	0.01\\
599.79	0.01\\
599.8	0.01\\
599.81	0.01\\
599.82	0.01\\
599.83	0.01\\
599.84	0.01\\
599.85	0.01\\
599.86	0.01\\
599.87	0.01\\
599.88	0.01\\
599.89	0.01\\
599.9	0.01\\
599.91	0.01\\
599.92	0.01\\
599.93	0.01\\
599.94	0.01\\
599.95	0.01\\
599.96	0.01\\
599.97	0.01\\
599.98	0.01\\
599.99	0.01\\
600	0.01\\
};
\addplot [color=blue!80!mycolor9,solid,forget plot]
  table[row sep=crcr]{%
0.01	0.01\\
1.01	0.01\\
2.01	0.01\\
3.01	0.01\\
4.01	0.01\\
5.01	0.01\\
6.01	0.01\\
7.01	0.01\\
8.01	0.01\\
9.01	0.01\\
10.01	0.01\\
11.01	0.01\\
12.01	0.01\\
13.01	0.01\\
14.01	0.01\\
15.01	0.01\\
16.01	0.01\\
17.01	0.01\\
18.01	0.01\\
19.01	0.01\\
20.01	0.01\\
21.01	0.01\\
22.01	0.01\\
23.01	0.01\\
24.01	0.01\\
25.01	0.01\\
26.01	0.01\\
27.01	0.01\\
28.01	0.01\\
29.01	0.01\\
30.01	0.01\\
31.01	0.01\\
32.01	0.01\\
33.01	0.01\\
34.01	0.01\\
35.01	0.01\\
36.01	0.01\\
37.01	0.01\\
38.01	0.01\\
39.01	0.01\\
40.01	0.01\\
41.01	0.01\\
42.01	0.01\\
43.01	0.01\\
44.01	0.01\\
45.01	0.01\\
46.01	0.01\\
47.01	0.01\\
48.01	0.01\\
49.01	0.01\\
50.01	0.01\\
51.01	0.01\\
52.01	0.01\\
53.01	0.01\\
54.01	0.01\\
55.01	0.01\\
56.01	0.01\\
57.01	0.01\\
58.01	0.01\\
59.01	0.01\\
60.01	0.01\\
61.01	0.01\\
62.01	0.01\\
63.01	0.01\\
64.01	0.01\\
65.01	0.01\\
66.01	0.01\\
67.01	0.01\\
68.01	0.01\\
69.01	0.01\\
70.01	0.01\\
71.01	0.01\\
72.01	0.01\\
73.01	0.01\\
74.01	0.01\\
75.01	0.01\\
76.01	0.01\\
77.01	0.01\\
78.01	0.01\\
79.01	0.01\\
80.01	0.01\\
81.01	0.01\\
82.01	0.01\\
83.01	0.01\\
84.01	0.01\\
85.01	0.01\\
86.01	0.01\\
87.01	0.01\\
88.01	0.01\\
89.01	0.01\\
90.01	0.01\\
91.01	0.01\\
92.01	0.01\\
93.01	0.01\\
94.01	0.01\\
95.01	0.01\\
96.01	0.01\\
97.01	0.01\\
98.01	0.01\\
99.01	0.01\\
100.01	0.01\\
101.01	0.01\\
102.01	0.01\\
103.01	0.01\\
104.01	0.01\\
105.01	0.01\\
106.01	0.01\\
107.01	0.01\\
108.01	0.01\\
109.01	0.01\\
110.01	0.01\\
111.01	0.01\\
112.01	0.01\\
113.01	0.01\\
114.01	0.01\\
115.01	0.01\\
116.01	0.01\\
117.01	0.01\\
118.01	0.01\\
119.01	0.01\\
120.01	0.01\\
121.01	0.01\\
122.01	0.01\\
123.01	0.01\\
124.01	0.01\\
125.01	0.01\\
126.01	0.01\\
127.01	0.01\\
128.01	0.01\\
129.01	0.01\\
130.01	0.01\\
131.01	0.01\\
132.01	0.01\\
133.01	0.01\\
134.01	0.01\\
135.01	0.01\\
136.01	0.01\\
137.01	0.01\\
138.01	0.01\\
139.01	0.01\\
140.01	0.01\\
141.01	0.01\\
142.01	0.01\\
143.01	0.01\\
144.01	0.01\\
145.01	0.01\\
146.01	0.01\\
147.01	0.01\\
148.01	0.01\\
149.01	0.01\\
150.01	0.01\\
151.01	0.01\\
152.01	0.01\\
153.01	0.01\\
154.01	0.01\\
155.01	0.01\\
156.01	0.01\\
157.01	0.01\\
158.01	0.01\\
159.01	0.01\\
160.01	0.01\\
161.01	0.01\\
162.01	0.01\\
163.01	0.01\\
164.01	0.01\\
165.01	0.01\\
166.01	0.01\\
167.01	0.01\\
168.01	0.01\\
169.01	0.01\\
170.01	0.01\\
171.01	0.01\\
172.01	0.01\\
173.01	0.01\\
174.01	0.01\\
175.01	0.01\\
176.01	0.01\\
177.01	0.01\\
178.01	0.01\\
179.01	0.01\\
180.01	0.01\\
181.01	0.01\\
182.01	0.01\\
183.01	0.01\\
184.01	0.01\\
185.01	0.01\\
186.01	0.01\\
187.01	0.01\\
188.01	0.01\\
189.01	0.01\\
190.01	0.01\\
191.01	0.01\\
192.01	0.01\\
193.01	0.01\\
194.01	0.01\\
195.01	0.01\\
196.01	0.01\\
197.01	0.01\\
198.01	0.01\\
199.01	0.01\\
200.01	0.01\\
201.01	0.01\\
202.01	0.01\\
203.01	0.01\\
204.01	0.01\\
205.01	0.01\\
206.01	0.01\\
207.01	0.01\\
208.01	0.01\\
209.01	0.01\\
210.01	0.01\\
211.01	0.01\\
212.01	0.01\\
213.01	0.01\\
214.01	0.01\\
215.01	0.01\\
216.01	0.01\\
217.01	0.01\\
218.01	0.01\\
219.01	0.01\\
220.01	0.01\\
221.01	0.01\\
222.01	0.01\\
223.01	0.01\\
224.01	0.01\\
225.01	0.01\\
226.01	0.01\\
227.01	0.01\\
228.01	0.01\\
229.01	0.01\\
230.01	0.01\\
231.01	0.01\\
232.01	0.01\\
233.01	0.01\\
234.01	0.01\\
235.01	0.01\\
236.01	0.01\\
237.01	0.01\\
238.01	0.01\\
239.01	0.01\\
240.01	0.01\\
241.01	0.01\\
242.01	0.01\\
243.01	0.01\\
244.01	0.01\\
245.01	0.01\\
246.01	0.01\\
247.01	0.01\\
248.01	0.01\\
249.01	0.01\\
250.01	0.01\\
251.01	0.01\\
252.01	0.01\\
253.01	0.01\\
254.01	0.01\\
255.01	0.01\\
256.01	0.01\\
257.01	0.01\\
258.01	0.01\\
259.01	0.01\\
260.01	0.01\\
261.01	0.01\\
262.01	0.01\\
263.01	0.01\\
264.01	0.01\\
265.01	0.01\\
266.01	0.01\\
267.01	0.01\\
268.01	0.01\\
269.01	0.01\\
270.01	0.01\\
271.01	0.01\\
272.01	0.01\\
273.01	0.01\\
274.01	0.01\\
275.01	0.01\\
276.01	0.01\\
277.01	0.01\\
278.01	0.01\\
279.01	0.01\\
280.01	0.01\\
281.01	0.01\\
282.01	0.01\\
283.01	0.01\\
284.01	0.01\\
285.01	0.01\\
286.01	0.01\\
287.01	0.01\\
288.01	0.01\\
289.01	0.01\\
290.01	0.01\\
291.01	0.01\\
292.01	0.01\\
293.01	0.01\\
294.01	0.01\\
295.01	0.01\\
296.01	0.01\\
297.01	0.01\\
298.01	0.01\\
299.01	0.01\\
300.01	0.01\\
301.01	0.01\\
302.01	0.01\\
303.01	0.01\\
304.01	0.01\\
305.01	0.01\\
306.01	0.01\\
307.01	0.01\\
308.01	0.01\\
309.01	0.01\\
310.01	0.01\\
311.01	0.01\\
312.01	0.01\\
313.01	0.01\\
314.01	0.01\\
315.01	0.01\\
316.01	0.01\\
317.01	0.01\\
318.01	0.01\\
319.01	0.01\\
320.01	0.01\\
321.01	0.01\\
322.01	0.01\\
323.01	0.01\\
324.01	0.01\\
325.01	0.01\\
326.01	0.01\\
327.01	0.01\\
328.01	0.01\\
329.01	0.01\\
330.01	0.01\\
331.01	0.01\\
332.01	0.01\\
333.01	0.01\\
334.01	0.01\\
335.01	0.01\\
336.01	0.01\\
337.01	0.01\\
338.01	0.01\\
339.01	0.01\\
340.01	0.01\\
341.01	0.01\\
342.01	0.01\\
343.01	0.01\\
344.01	0.01\\
345.01	0.01\\
346.01	0.01\\
347.01	0.01\\
348.01	0.01\\
349.01	0.01\\
350.01	0.01\\
351.01	0.01\\
352.01	0.01\\
353.01	0.01\\
354.01	0.01\\
355.01	0.01\\
356.01	0.01\\
357.01	0.01\\
358.01	0.01\\
359.01	0.01\\
360.01	0.01\\
361.01	0.01\\
362.01	0.01\\
363.01	0.01\\
364.01	0.01\\
365.01	0.01\\
366.01	0.01\\
367.01	0.01\\
368.01	0.01\\
369.01	0.01\\
370.01	0.01\\
371.01	0.01\\
372.01	0.01\\
373.01	0.01\\
374.01	0.01\\
375.01	0.01\\
376.01	0.01\\
377.01	0.01\\
378.01	0.01\\
379.01	0.01\\
380.01	0.01\\
381.01	0.01\\
382.01	0.01\\
383.01	0.01\\
384.01	0.01\\
385.01	0.01\\
386.01	0.01\\
387.01	0.01\\
388.01	0.01\\
389.01	0.01\\
390.01	0.01\\
391.01	0.01\\
392.01	0.01\\
393.01	0.01\\
394.01	0.01\\
395.01	0.01\\
396.01	0.01\\
397.01	0.01\\
398.01	0.01\\
399.01	0.01\\
400.01	0.01\\
401.01	0.01\\
402.01	0.01\\
403.01	0.01\\
404.01	0.01\\
405.01	0.01\\
406.01	0.01\\
407.01	0.01\\
408.01	0.01\\
409.01	0.01\\
410.01	0.01\\
411.01	0.01\\
412.01	0.01\\
413.01	0.01\\
414.01	0.01\\
415.01	0.01\\
416.01	0.01\\
417.01	0.01\\
418.01	0.01\\
419.01	0.01\\
420.01	0.01\\
421.01	0.01\\
422.01	0.01\\
423.01	0.01\\
424.01	0.01\\
425.01	0.01\\
426.01	0.01\\
427.01	0.01\\
428.01	0.01\\
429.01	0.01\\
430.01	0.01\\
431.01	0.01\\
432.01	0.01\\
433.01	0.01\\
434.01	0.01\\
435.01	0.01\\
436.01	0.01\\
437.01	0.01\\
438.01	0.01\\
439.01	0.01\\
440.01	0.01\\
441.01	0.01\\
442.01	0.01\\
443.01	0.01\\
444.01	0.01\\
445.01	0.01\\
446.01	0.01\\
447.01	0.01\\
448.01	0.01\\
449.01	0.01\\
450.01	0.01\\
451.01	0.01\\
452.01	0.01\\
453.01	0.01\\
454.01	0.01\\
455.01	0.01\\
456.01	0.01\\
457.01	0.01\\
458.01	0.01\\
459.01	0.01\\
460.01	0.01\\
461.01	0.01\\
462.01	0.01\\
463.01	0.01\\
464.01	0.01\\
465.01	0.01\\
466.01	0.01\\
467.01	0.01\\
468.01	0.01\\
469.01	0.01\\
470.01	0.01\\
471.01	0.01\\
472.01	0.01\\
473.01	0.01\\
474.01	0.01\\
475.01	0.01\\
476.01	0.01\\
477.01	0.01\\
478.01	0.01\\
479.01	0.01\\
480.01	0.01\\
481.01	0.01\\
482.01	0.01\\
483.01	0.01\\
484.01	0.01\\
485.01	0.01\\
486.01	0.01\\
487.01	0.01\\
488.01	0.01\\
489.01	0.01\\
490.01	0.01\\
491.01	0.01\\
492.01	0.01\\
493.01	0.01\\
494.01	0.01\\
495.01	0.01\\
496.01	0.01\\
497.01	0.01\\
498.01	0.01\\
499.01	0.01\\
500.01	0.01\\
501.01	0.01\\
502.01	0.01\\
503.01	0.01\\
504.01	0.01\\
505.01	0.01\\
506.01	0.01\\
507.01	0.01\\
508.01	0.01\\
509.01	0.01\\
510.01	0.01\\
511.01	0.01\\
512.01	0.01\\
513.01	0.01\\
514.01	0.01\\
515.01	0.01\\
516.01	0.01\\
517.01	0.01\\
518.01	0.01\\
519.01	0.01\\
520.01	0.01\\
521.01	0.01\\
522.01	0.01\\
523.01	0.01\\
524.01	0.01\\
525.01	0.01\\
526.01	0.01\\
527.01	0.01\\
528.01	0.01\\
529.01	0.01\\
530.01	0.01\\
531.01	0.01\\
532.01	0.01\\
533.01	0.01\\
534.01	0.01\\
535.01	0.01\\
536.01	0.01\\
537.01	0.01\\
538.01	0.01\\
539.01	0.01\\
540.01	0.01\\
541.01	0.01\\
542.01	0.01\\
543.01	0.01\\
544.01	0.01\\
545.01	0.01\\
546.01	0.01\\
547.01	0.01\\
548.01	0.01\\
549.01	0.01\\
550.01	0.01\\
551.01	0.01\\
552.01	0.01\\
553.01	0.01\\
554.01	0.01\\
555.01	0.01\\
556.01	0.01\\
557.01	0.01\\
558.01	0.01\\
559.01	0.01\\
560.01	0.01\\
561.01	0.01\\
562.01	0.01\\
563.01	0.01\\
564.01	0.01\\
565.01	0.01\\
566.01	0.01\\
567.01	0.01\\
568.01	0.01\\
569.01	0.01\\
570.01	0.01\\
571.01	0.01\\
572.01	0.01\\
573.01	0.01\\
574.01	0.01\\
575.01	0.01\\
576.01	0.01\\
577.01	0.01\\
578.01	0.01\\
579.01	0.01\\
580.01	0.01\\
581.01	0.01\\
582.01	0.01\\
583.01	0.01\\
584.01	0.01\\
585.01	0.01\\
586.01	0.01\\
587.01	0.01\\
588.01	0.01\\
589.01	0.01\\
590.01	0.01\\
591.01	0.01\\
592.01	0.01\\
593.01	0.01\\
594.01	0.01\\
595.01	0.01\\
596.01	0.01\\
597.01	0.01\\
598.01	0.01\\
599.01	0.01\\
599.02	0.01\\
599.03	0.01\\
599.04	0.01\\
599.05	0.01\\
599.06	0.01\\
599.07	0.01\\
599.08	0.01\\
599.09	0.01\\
599.1	0.01\\
599.11	0.01\\
599.12	0.01\\
599.13	0.01\\
599.14	0.01\\
599.15	0.01\\
599.16	0.01\\
599.17	0.01\\
599.18	0.01\\
599.19	0.01\\
599.2	0.01\\
599.21	0.01\\
599.22	0.01\\
599.23	0.01\\
599.24	0.01\\
599.25	0.01\\
599.26	0.01\\
599.27	0.01\\
599.28	0.01\\
599.29	0.01\\
599.3	0.01\\
599.31	0.01\\
599.32	0.01\\
599.33	0.01\\
599.34	0.01\\
599.35	0.01\\
599.36	0.01\\
599.37	0.01\\
599.38	0.01\\
599.39	0.01\\
599.4	0.01\\
599.41	0.01\\
599.42	0.01\\
599.43	0.01\\
599.44	0.01\\
599.45	0.01\\
599.46	0.01\\
599.47	0.01\\
599.48	0.01\\
599.49	0.01\\
599.5	0.01\\
599.51	0.01\\
599.52	0.01\\
599.53	0.01\\
599.54	0.01\\
599.55	0.01\\
599.56	0.01\\
599.57	0.01\\
599.58	0.01\\
599.59	0.01\\
599.6	0.01\\
599.61	0.01\\
599.62	0.01\\
599.63	0.01\\
599.64	0.01\\
599.65	0.01\\
599.66	0.01\\
599.67	0.01\\
599.68	0.01\\
599.69	0.01\\
599.7	0.01\\
599.71	0.01\\
599.72	0.01\\
599.73	0.01\\
599.74	0.01\\
599.75	0.01\\
599.76	0.01\\
599.77	0.01\\
599.78	0.01\\
599.79	0.01\\
599.8	0.01\\
599.81	0.01\\
599.82	0.01\\
599.83	0.01\\
599.84	0.01\\
599.85	0.01\\
599.86	0.01\\
599.87	0.01\\
599.88	0.01\\
599.89	0.01\\
599.9	0.01\\
599.91	0.01\\
599.92	0.01\\
599.93	0.01\\
599.94	0.01\\
599.95	0.01\\
599.96	0.01\\
599.97	0.01\\
599.98	0.01\\
599.99	0.01\\
600	0.01\\
};
\addplot [color=blue,solid,forget plot]
  table[row sep=crcr]{%
0.01	0.01\\
1.01	0.01\\
2.01	0.01\\
3.01	0.01\\
4.01	0.01\\
5.01	0.01\\
6.01	0.01\\
7.01	0.01\\
8.01	0.01\\
9.01	0.01\\
10.01	0.01\\
11.01	0.01\\
12.01	0.01\\
13.01	0.01\\
14.01	0.01\\
15.01	0.01\\
16.01	0.01\\
17.01	0.01\\
18.01	0.01\\
19.01	0.01\\
20.01	0.01\\
21.01	0.01\\
22.01	0.01\\
23.01	0.01\\
24.01	0.01\\
25.01	0.01\\
26.01	0.01\\
27.01	0.01\\
28.01	0.01\\
29.01	0.01\\
30.01	0.01\\
31.01	0.01\\
32.01	0.01\\
33.01	0.01\\
34.01	0.01\\
35.01	0.01\\
36.01	0.01\\
37.01	0.01\\
38.01	0.01\\
39.01	0.01\\
40.01	0.01\\
41.01	0.01\\
42.01	0.01\\
43.01	0.01\\
44.01	0.01\\
45.01	0.01\\
46.01	0.01\\
47.01	0.01\\
48.01	0.01\\
49.01	0.01\\
50.01	0.01\\
51.01	0.01\\
52.01	0.01\\
53.01	0.01\\
54.01	0.01\\
55.01	0.01\\
56.01	0.01\\
57.01	0.01\\
58.01	0.01\\
59.01	0.01\\
60.01	0.01\\
61.01	0.01\\
62.01	0.01\\
63.01	0.01\\
64.01	0.01\\
65.01	0.01\\
66.01	0.01\\
67.01	0.01\\
68.01	0.01\\
69.01	0.01\\
70.01	0.01\\
71.01	0.01\\
72.01	0.01\\
73.01	0.01\\
74.01	0.01\\
75.01	0.01\\
76.01	0.01\\
77.01	0.01\\
78.01	0.01\\
79.01	0.01\\
80.01	0.01\\
81.01	0.01\\
82.01	0.01\\
83.01	0.01\\
84.01	0.01\\
85.01	0.01\\
86.01	0.01\\
87.01	0.01\\
88.01	0.01\\
89.01	0.01\\
90.01	0.01\\
91.01	0.01\\
92.01	0.01\\
93.01	0.01\\
94.01	0.01\\
95.01	0.01\\
96.01	0.01\\
97.01	0.01\\
98.01	0.01\\
99.01	0.01\\
100.01	0.01\\
101.01	0.01\\
102.01	0.01\\
103.01	0.01\\
104.01	0.01\\
105.01	0.01\\
106.01	0.01\\
107.01	0.01\\
108.01	0.01\\
109.01	0.01\\
110.01	0.01\\
111.01	0.01\\
112.01	0.01\\
113.01	0.01\\
114.01	0.01\\
115.01	0.01\\
116.01	0.01\\
117.01	0.01\\
118.01	0.01\\
119.01	0.01\\
120.01	0.01\\
121.01	0.01\\
122.01	0.01\\
123.01	0.01\\
124.01	0.01\\
125.01	0.01\\
126.01	0.01\\
127.01	0.01\\
128.01	0.01\\
129.01	0.01\\
130.01	0.01\\
131.01	0.01\\
132.01	0.01\\
133.01	0.01\\
134.01	0.01\\
135.01	0.01\\
136.01	0.01\\
137.01	0.01\\
138.01	0.01\\
139.01	0.01\\
140.01	0.01\\
141.01	0.01\\
142.01	0.01\\
143.01	0.01\\
144.01	0.01\\
145.01	0.01\\
146.01	0.01\\
147.01	0.01\\
148.01	0.01\\
149.01	0.01\\
150.01	0.01\\
151.01	0.01\\
152.01	0.01\\
153.01	0.01\\
154.01	0.01\\
155.01	0.01\\
156.01	0.01\\
157.01	0.01\\
158.01	0.01\\
159.01	0.01\\
160.01	0.01\\
161.01	0.01\\
162.01	0.01\\
163.01	0.01\\
164.01	0.01\\
165.01	0.01\\
166.01	0.01\\
167.01	0.01\\
168.01	0.01\\
169.01	0.01\\
170.01	0.01\\
171.01	0.01\\
172.01	0.01\\
173.01	0.01\\
174.01	0.01\\
175.01	0.01\\
176.01	0.01\\
177.01	0.01\\
178.01	0.01\\
179.01	0.01\\
180.01	0.01\\
181.01	0.01\\
182.01	0.01\\
183.01	0.01\\
184.01	0.01\\
185.01	0.01\\
186.01	0.01\\
187.01	0.01\\
188.01	0.01\\
189.01	0.01\\
190.01	0.01\\
191.01	0.01\\
192.01	0.01\\
193.01	0.01\\
194.01	0.01\\
195.01	0.01\\
196.01	0.01\\
197.01	0.01\\
198.01	0.01\\
199.01	0.01\\
200.01	0.01\\
201.01	0.01\\
202.01	0.01\\
203.01	0.01\\
204.01	0.01\\
205.01	0.01\\
206.01	0.01\\
207.01	0.01\\
208.01	0.01\\
209.01	0.01\\
210.01	0.01\\
211.01	0.01\\
212.01	0.01\\
213.01	0.01\\
214.01	0.01\\
215.01	0.01\\
216.01	0.01\\
217.01	0.01\\
218.01	0.01\\
219.01	0.01\\
220.01	0.01\\
221.01	0.01\\
222.01	0.01\\
223.01	0.01\\
224.01	0.01\\
225.01	0.01\\
226.01	0.01\\
227.01	0.01\\
228.01	0.01\\
229.01	0.01\\
230.01	0.01\\
231.01	0.01\\
232.01	0.01\\
233.01	0.01\\
234.01	0.01\\
235.01	0.01\\
236.01	0.01\\
237.01	0.01\\
238.01	0.01\\
239.01	0.01\\
240.01	0.01\\
241.01	0.01\\
242.01	0.01\\
243.01	0.01\\
244.01	0.01\\
245.01	0.01\\
246.01	0.01\\
247.01	0.01\\
248.01	0.01\\
249.01	0.01\\
250.01	0.01\\
251.01	0.01\\
252.01	0.01\\
253.01	0.01\\
254.01	0.01\\
255.01	0.01\\
256.01	0.01\\
257.01	0.01\\
258.01	0.01\\
259.01	0.01\\
260.01	0.01\\
261.01	0.01\\
262.01	0.01\\
263.01	0.01\\
264.01	0.01\\
265.01	0.01\\
266.01	0.01\\
267.01	0.01\\
268.01	0.01\\
269.01	0.01\\
270.01	0.01\\
271.01	0.01\\
272.01	0.01\\
273.01	0.01\\
274.01	0.01\\
275.01	0.01\\
276.01	0.01\\
277.01	0.01\\
278.01	0.01\\
279.01	0.01\\
280.01	0.01\\
281.01	0.01\\
282.01	0.01\\
283.01	0.01\\
284.01	0.01\\
285.01	0.01\\
286.01	0.01\\
287.01	0.01\\
288.01	0.01\\
289.01	0.01\\
290.01	0.01\\
291.01	0.01\\
292.01	0.01\\
293.01	0.01\\
294.01	0.01\\
295.01	0.01\\
296.01	0.01\\
297.01	0.01\\
298.01	0.01\\
299.01	0.01\\
300.01	0.01\\
301.01	0.01\\
302.01	0.01\\
303.01	0.01\\
304.01	0.01\\
305.01	0.01\\
306.01	0.01\\
307.01	0.01\\
308.01	0.01\\
309.01	0.01\\
310.01	0.01\\
311.01	0.01\\
312.01	0.01\\
313.01	0.01\\
314.01	0.01\\
315.01	0.01\\
316.01	0.01\\
317.01	0.01\\
318.01	0.01\\
319.01	0.01\\
320.01	0.01\\
321.01	0.01\\
322.01	0.01\\
323.01	0.01\\
324.01	0.01\\
325.01	0.01\\
326.01	0.01\\
327.01	0.01\\
328.01	0.01\\
329.01	0.01\\
330.01	0.01\\
331.01	0.01\\
332.01	0.01\\
333.01	0.01\\
334.01	0.01\\
335.01	0.01\\
336.01	0.01\\
337.01	0.01\\
338.01	0.01\\
339.01	0.01\\
340.01	0.01\\
341.01	0.01\\
342.01	0.01\\
343.01	0.01\\
344.01	0.01\\
345.01	0.01\\
346.01	0.01\\
347.01	0.01\\
348.01	0.01\\
349.01	0.01\\
350.01	0.01\\
351.01	0.01\\
352.01	0.01\\
353.01	0.01\\
354.01	0.01\\
355.01	0.01\\
356.01	0.01\\
357.01	0.01\\
358.01	0.01\\
359.01	0.01\\
360.01	0.01\\
361.01	0.01\\
362.01	0.01\\
363.01	0.01\\
364.01	0.01\\
365.01	0.01\\
366.01	0.01\\
367.01	0.01\\
368.01	0.01\\
369.01	0.01\\
370.01	0.01\\
371.01	0.01\\
372.01	0.01\\
373.01	0.01\\
374.01	0.01\\
375.01	0.01\\
376.01	0.01\\
377.01	0.01\\
378.01	0.01\\
379.01	0.01\\
380.01	0.01\\
381.01	0.01\\
382.01	0.01\\
383.01	0.01\\
384.01	0.01\\
385.01	0.01\\
386.01	0.01\\
387.01	0.01\\
388.01	0.01\\
389.01	0.01\\
390.01	0.01\\
391.01	0.01\\
392.01	0.01\\
393.01	0.01\\
394.01	0.01\\
395.01	0.01\\
396.01	0.01\\
397.01	0.01\\
398.01	0.01\\
399.01	0.01\\
400.01	0.01\\
401.01	0.01\\
402.01	0.01\\
403.01	0.01\\
404.01	0.01\\
405.01	0.01\\
406.01	0.01\\
407.01	0.01\\
408.01	0.01\\
409.01	0.01\\
410.01	0.01\\
411.01	0.01\\
412.01	0.01\\
413.01	0.01\\
414.01	0.01\\
415.01	0.01\\
416.01	0.01\\
417.01	0.01\\
418.01	0.01\\
419.01	0.01\\
420.01	0.01\\
421.01	0.01\\
422.01	0.01\\
423.01	0.01\\
424.01	0.01\\
425.01	0.01\\
426.01	0.01\\
427.01	0.01\\
428.01	0.01\\
429.01	0.01\\
430.01	0.01\\
431.01	0.01\\
432.01	0.01\\
433.01	0.01\\
434.01	0.01\\
435.01	0.01\\
436.01	0.01\\
437.01	0.01\\
438.01	0.01\\
439.01	0.01\\
440.01	0.01\\
441.01	0.01\\
442.01	0.01\\
443.01	0.01\\
444.01	0.01\\
445.01	0.01\\
446.01	0.01\\
447.01	0.01\\
448.01	0.01\\
449.01	0.01\\
450.01	0.01\\
451.01	0.01\\
452.01	0.01\\
453.01	0.01\\
454.01	0.01\\
455.01	0.01\\
456.01	0.01\\
457.01	0.01\\
458.01	0.01\\
459.01	0.01\\
460.01	0.01\\
461.01	0.01\\
462.01	0.01\\
463.01	0.01\\
464.01	0.01\\
465.01	0.01\\
466.01	0.01\\
467.01	0.01\\
468.01	0.01\\
469.01	0.01\\
470.01	0.01\\
471.01	0.01\\
472.01	0.01\\
473.01	0.01\\
474.01	0.01\\
475.01	0.01\\
476.01	0.01\\
477.01	0.01\\
478.01	0.01\\
479.01	0.01\\
480.01	0.01\\
481.01	0.01\\
482.01	0.01\\
483.01	0.01\\
484.01	0.01\\
485.01	0.01\\
486.01	0.01\\
487.01	0.01\\
488.01	0.01\\
489.01	0.01\\
490.01	0.01\\
491.01	0.01\\
492.01	0.01\\
493.01	0.01\\
494.01	0.01\\
495.01	0.01\\
496.01	0.01\\
497.01	0.01\\
498.01	0.01\\
499.01	0.01\\
500.01	0.01\\
501.01	0.01\\
502.01	0.01\\
503.01	0.01\\
504.01	0.01\\
505.01	0.01\\
506.01	0.01\\
507.01	0.01\\
508.01	0.01\\
509.01	0.01\\
510.01	0.01\\
511.01	0.01\\
512.01	0.01\\
513.01	0.01\\
514.01	0.01\\
515.01	0.01\\
516.01	0.01\\
517.01	0.01\\
518.01	0.01\\
519.01	0.01\\
520.01	0.01\\
521.01	0.01\\
522.01	0.01\\
523.01	0.01\\
524.01	0.01\\
525.01	0.01\\
526.01	0.01\\
527.01	0.01\\
528.01	0.01\\
529.01	0.01\\
530.01	0.01\\
531.01	0.01\\
532.01	0.01\\
533.01	0.01\\
534.01	0.01\\
535.01	0.01\\
536.01	0.01\\
537.01	0.01\\
538.01	0.01\\
539.01	0.01\\
540.01	0.01\\
541.01	0.01\\
542.01	0.01\\
543.01	0.01\\
544.01	0.01\\
545.01	0.01\\
546.01	0.01\\
547.01	0.01\\
548.01	0.01\\
549.01	0.01\\
550.01	0.01\\
551.01	0.01\\
552.01	0.01\\
553.01	0.01\\
554.01	0.01\\
555.01	0.01\\
556.01	0.01\\
557.01	0.01\\
558.01	0.01\\
559.01	0.01\\
560.01	0.01\\
561.01	0.01\\
562.01	0.01\\
563.01	0.01\\
564.01	0.01\\
565.01	0.01\\
566.01	0.01\\
567.01	0.01\\
568.01	0.01\\
569.01	0.01\\
570.01	0.01\\
571.01	0.01\\
572.01	0.01\\
573.01	0.01\\
574.01	0.01\\
575.01	0.01\\
576.01	0.01\\
577.01	0.01\\
578.01	0.01\\
579.01	0.01\\
580.01	0.01\\
581.01	0.01\\
582.01	0.01\\
583.01	0.01\\
584.01	0.01\\
585.01	0.01\\
586.01	0.01\\
587.01	0.01\\
588.01	0.01\\
589.01	0.01\\
590.01	0.01\\
591.01	0.01\\
592.01	0.01\\
593.01	0.01\\
594.01	0.01\\
595.01	0.01\\
596.01	0.01\\
597.01	0.01\\
598.01	0.01\\
599.01	0.01\\
599.02	0.01\\
599.03	0.01\\
599.04	0.01\\
599.05	0.01\\
599.06	0.01\\
599.07	0.01\\
599.08	0.01\\
599.09	0.01\\
599.1	0.01\\
599.11	0.01\\
599.12	0.01\\
599.13	0.01\\
599.14	0.01\\
599.15	0.01\\
599.16	0.01\\
599.17	0.01\\
599.18	0.01\\
599.19	0.01\\
599.2	0.01\\
599.21	0.01\\
599.22	0.01\\
599.23	0.01\\
599.24	0.01\\
599.25	0.01\\
599.26	0.01\\
599.27	0.01\\
599.28	0.01\\
599.29	0.01\\
599.3	0.01\\
599.31	0.01\\
599.32	0.01\\
599.33	0.01\\
599.34	0.01\\
599.35	0.01\\
599.36	0.01\\
599.37	0.01\\
599.38	0.01\\
599.39	0.01\\
599.4	0.01\\
599.41	0.01\\
599.42	0.01\\
599.43	0.01\\
599.44	0.01\\
599.45	0.01\\
599.46	0.01\\
599.47	0.01\\
599.48	0.01\\
599.49	0.01\\
599.5	0.01\\
599.51	0.01\\
599.52	0.01\\
599.53	0.01\\
599.54	0.01\\
599.55	0.01\\
599.56	0.01\\
599.57	0.01\\
599.58	0.01\\
599.59	0.01\\
599.6	0.01\\
599.61	0.01\\
599.62	0.01\\
599.63	0.01\\
599.64	0.01\\
599.65	0.01\\
599.66	0.01\\
599.67	0.01\\
599.68	0.01\\
599.69	0.01\\
599.7	0.01\\
599.71	0.01\\
599.72	0.01\\
599.73	0.01\\
599.74	0.01\\
599.75	0.01\\
599.76	0.01\\
599.77	0.01\\
599.78	0.01\\
599.79	0.01\\
599.8	0.01\\
599.81	0.01\\
599.82	0.01\\
599.83	0.01\\
599.84	0.01\\
599.85	0.01\\
599.86	0.01\\
599.87	0.01\\
599.88	0.01\\
599.89	0.01\\
599.9	0.01\\
599.91	0.01\\
599.92	0.01\\
599.93	0.01\\
599.94	0.01\\
599.95	0.01\\
599.96	0.01\\
599.97	0.01\\
599.98	0.01\\
599.99	0.01\\
600	0.01\\
};
\addplot [color=mycolor10,solid,forget plot]
  table[row sep=crcr]{%
0.01	0.01\\
1.01	0.01\\
2.01	0.01\\
3.01	0.01\\
4.01	0.01\\
5.01	0.01\\
6.01	0.01\\
7.01	0.01\\
8.01	0.01\\
9.01	0.01\\
10.01	0.01\\
11.01	0.01\\
12.01	0.01\\
13.01	0.01\\
14.01	0.01\\
15.01	0.01\\
16.01	0.01\\
17.01	0.01\\
18.01	0.01\\
19.01	0.01\\
20.01	0.01\\
21.01	0.01\\
22.01	0.01\\
23.01	0.01\\
24.01	0.01\\
25.01	0.01\\
26.01	0.01\\
27.01	0.01\\
28.01	0.01\\
29.01	0.01\\
30.01	0.01\\
31.01	0.01\\
32.01	0.01\\
33.01	0.01\\
34.01	0.01\\
35.01	0.01\\
36.01	0.01\\
37.01	0.01\\
38.01	0.01\\
39.01	0.01\\
40.01	0.01\\
41.01	0.01\\
42.01	0.01\\
43.01	0.01\\
44.01	0.01\\
45.01	0.01\\
46.01	0.01\\
47.01	0.01\\
48.01	0.01\\
49.01	0.01\\
50.01	0.01\\
51.01	0.01\\
52.01	0.01\\
53.01	0.01\\
54.01	0.01\\
55.01	0.01\\
56.01	0.01\\
57.01	0.01\\
58.01	0.01\\
59.01	0.01\\
60.01	0.01\\
61.01	0.01\\
62.01	0.01\\
63.01	0.01\\
64.01	0.01\\
65.01	0.01\\
66.01	0.01\\
67.01	0.01\\
68.01	0.01\\
69.01	0.01\\
70.01	0.01\\
71.01	0.01\\
72.01	0.01\\
73.01	0.01\\
74.01	0.01\\
75.01	0.01\\
76.01	0.01\\
77.01	0.01\\
78.01	0.01\\
79.01	0.01\\
80.01	0.01\\
81.01	0.01\\
82.01	0.01\\
83.01	0.01\\
84.01	0.01\\
85.01	0.01\\
86.01	0.01\\
87.01	0.01\\
88.01	0.01\\
89.01	0.01\\
90.01	0.01\\
91.01	0.01\\
92.01	0.01\\
93.01	0.01\\
94.01	0.01\\
95.01	0.01\\
96.01	0.01\\
97.01	0.01\\
98.01	0.01\\
99.01	0.01\\
100.01	0.01\\
101.01	0.01\\
102.01	0.01\\
103.01	0.01\\
104.01	0.01\\
105.01	0.01\\
106.01	0.01\\
107.01	0.01\\
108.01	0.01\\
109.01	0.01\\
110.01	0.01\\
111.01	0.01\\
112.01	0.01\\
113.01	0.01\\
114.01	0.01\\
115.01	0.01\\
116.01	0.01\\
117.01	0.01\\
118.01	0.01\\
119.01	0.01\\
120.01	0.01\\
121.01	0.01\\
122.01	0.01\\
123.01	0.01\\
124.01	0.01\\
125.01	0.01\\
126.01	0.01\\
127.01	0.01\\
128.01	0.01\\
129.01	0.01\\
130.01	0.01\\
131.01	0.01\\
132.01	0.01\\
133.01	0.01\\
134.01	0.01\\
135.01	0.01\\
136.01	0.01\\
137.01	0.01\\
138.01	0.01\\
139.01	0.01\\
140.01	0.01\\
141.01	0.01\\
142.01	0.01\\
143.01	0.01\\
144.01	0.01\\
145.01	0.01\\
146.01	0.01\\
147.01	0.01\\
148.01	0.01\\
149.01	0.01\\
150.01	0.01\\
151.01	0.01\\
152.01	0.01\\
153.01	0.01\\
154.01	0.01\\
155.01	0.01\\
156.01	0.01\\
157.01	0.01\\
158.01	0.01\\
159.01	0.01\\
160.01	0.01\\
161.01	0.01\\
162.01	0.01\\
163.01	0.01\\
164.01	0.01\\
165.01	0.01\\
166.01	0.01\\
167.01	0.01\\
168.01	0.01\\
169.01	0.01\\
170.01	0.01\\
171.01	0.01\\
172.01	0.01\\
173.01	0.01\\
174.01	0.01\\
175.01	0.01\\
176.01	0.01\\
177.01	0.01\\
178.01	0.01\\
179.01	0.01\\
180.01	0.01\\
181.01	0.01\\
182.01	0.01\\
183.01	0.01\\
184.01	0.01\\
185.01	0.01\\
186.01	0.01\\
187.01	0.01\\
188.01	0.01\\
189.01	0.01\\
190.01	0.01\\
191.01	0.01\\
192.01	0.01\\
193.01	0.01\\
194.01	0.01\\
195.01	0.01\\
196.01	0.01\\
197.01	0.01\\
198.01	0.01\\
199.01	0.01\\
200.01	0.01\\
201.01	0.01\\
202.01	0.01\\
203.01	0.01\\
204.01	0.01\\
205.01	0.01\\
206.01	0.01\\
207.01	0.01\\
208.01	0.01\\
209.01	0.01\\
210.01	0.01\\
211.01	0.01\\
212.01	0.01\\
213.01	0.01\\
214.01	0.01\\
215.01	0.01\\
216.01	0.01\\
217.01	0.01\\
218.01	0.01\\
219.01	0.01\\
220.01	0.01\\
221.01	0.01\\
222.01	0.01\\
223.01	0.01\\
224.01	0.01\\
225.01	0.01\\
226.01	0.01\\
227.01	0.01\\
228.01	0.01\\
229.01	0.01\\
230.01	0.01\\
231.01	0.01\\
232.01	0.01\\
233.01	0.01\\
234.01	0.01\\
235.01	0.01\\
236.01	0.01\\
237.01	0.01\\
238.01	0.01\\
239.01	0.01\\
240.01	0.01\\
241.01	0.01\\
242.01	0.01\\
243.01	0.01\\
244.01	0.01\\
245.01	0.01\\
246.01	0.01\\
247.01	0.01\\
248.01	0.01\\
249.01	0.01\\
250.01	0.01\\
251.01	0.01\\
252.01	0.01\\
253.01	0.01\\
254.01	0.01\\
255.01	0.01\\
256.01	0.01\\
257.01	0.01\\
258.01	0.01\\
259.01	0.01\\
260.01	0.01\\
261.01	0.01\\
262.01	0.01\\
263.01	0.01\\
264.01	0.01\\
265.01	0.01\\
266.01	0.01\\
267.01	0.01\\
268.01	0.01\\
269.01	0.01\\
270.01	0.01\\
271.01	0.01\\
272.01	0.01\\
273.01	0.01\\
274.01	0.01\\
275.01	0.01\\
276.01	0.01\\
277.01	0.01\\
278.01	0.01\\
279.01	0.01\\
280.01	0.01\\
281.01	0.01\\
282.01	0.01\\
283.01	0.01\\
284.01	0.01\\
285.01	0.01\\
286.01	0.01\\
287.01	0.01\\
288.01	0.01\\
289.01	0.01\\
290.01	0.01\\
291.01	0.01\\
292.01	0.01\\
293.01	0.01\\
294.01	0.01\\
295.01	0.01\\
296.01	0.01\\
297.01	0.01\\
298.01	0.01\\
299.01	0.01\\
300.01	0.01\\
301.01	0.01\\
302.01	0.01\\
303.01	0.01\\
304.01	0.01\\
305.01	0.01\\
306.01	0.01\\
307.01	0.01\\
308.01	0.01\\
309.01	0.01\\
310.01	0.01\\
311.01	0.01\\
312.01	0.01\\
313.01	0.01\\
314.01	0.01\\
315.01	0.01\\
316.01	0.01\\
317.01	0.01\\
318.01	0.01\\
319.01	0.01\\
320.01	0.01\\
321.01	0.01\\
322.01	0.01\\
323.01	0.01\\
324.01	0.01\\
325.01	0.01\\
326.01	0.01\\
327.01	0.01\\
328.01	0.01\\
329.01	0.01\\
330.01	0.01\\
331.01	0.01\\
332.01	0.01\\
333.01	0.01\\
334.01	0.01\\
335.01	0.01\\
336.01	0.01\\
337.01	0.01\\
338.01	0.01\\
339.01	0.01\\
340.01	0.01\\
341.01	0.01\\
342.01	0.01\\
343.01	0.01\\
344.01	0.01\\
345.01	0.01\\
346.01	0.01\\
347.01	0.01\\
348.01	0.01\\
349.01	0.01\\
350.01	0.01\\
351.01	0.01\\
352.01	0.01\\
353.01	0.01\\
354.01	0.01\\
355.01	0.01\\
356.01	0.01\\
357.01	0.01\\
358.01	0.01\\
359.01	0.01\\
360.01	0.01\\
361.01	0.01\\
362.01	0.01\\
363.01	0.01\\
364.01	0.01\\
365.01	0.01\\
366.01	0.01\\
367.01	0.01\\
368.01	0.01\\
369.01	0.01\\
370.01	0.01\\
371.01	0.01\\
372.01	0.01\\
373.01	0.01\\
374.01	0.01\\
375.01	0.01\\
376.01	0.01\\
377.01	0.01\\
378.01	0.01\\
379.01	0.01\\
380.01	0.01\\
381.01	0.01\\
382.01	0.01\\
383.01	0.01\\
384.01	0.01\\
385.01	0.01\\
386.01	0.01\\
387.01	0.01\\
388.01	0.01\\
389.01	0.01\\
390.01	0.01\\
391.01	0.01\\
392.01	0.01\\
393.01	0.01\\
394.01	0.01\\
395.01	0.01\\
396.01	0.01\\
397.01	0.01\\
398.01	0.01\\
399.01	0.01\\
400.01	0.01\\
401.01	0.01\\
402.01	0.01\\
403.01	0.01\\
404.01	0.01\\
405.01	0.01\\
406.01	0.01\\
407.01	0.01\\
408.01	0.01\\
409.01	0.01\\
410.01	0.01\\
411.01	0.01\\
412.01	0.01\\
413.01	0.01\\
414.01	0.01\\
415.01	0.01\\
416.01	0.01\\
417.01	0.01\\
418.01	0.01\\
419.01	0.01\\
420.01	0.01\\
421.01	0.01\\
422.01	0.01\\
423.01	0.01\\
424.01	0.01\\
425.01	0.01\\
426.01	0.01\\
427.01	0.01\\
428.01	0.01\\
429.01	0.01\\
430.01	0.01\\
431.01	0.01\\
432.01	0.01\\
433.01	0.01\\
434.01	0.01\\
435.01	0.01\\
436.01	0.01\\
437.01	0.01\\
438.01	0.01\\
439.01	0.01\\
440.01	0.01\\
441.01	0.01\\
442.01	0.01\\
443.01	0.01\\
444.01	0.01\\
445.01	0.01\\
446.01	0.01\\
447.01	0.01\\
448.01	0.01\\
449.01	0.01\\
450.01	0.01\\
451.01	0.01\\
452.01	0.01\\
453.01	0.01\\
454.01	0.01\\
455.01	0.01\\
456.01	0.01\\
457.01	0.01\\
458.01	0.01\\
459.01	0.01\\
460.01	0.01\\
461.01	0.01\\
462.01	0.01\\
463.01	0.01\\
464.01	0.01\\
465.01	0.01\\
466.01	0.01\\
467.01	0.01\\
468.01	0.01\\
469.01	0.01\\
470.01	0.01\\
471.01	0.01\\
472.01	0.01\\
473.01	0.01\\
474.01	0.01\\
475.01	0.01\\
476.01	0.01\\
477.01	0.01\\
478.01	0.01\\
479.01	0.01\\
480.01	0.01\\
481.01	0.01\\
482.01	0.01\\
483.01	0.01\\
484.01	0.01\\
485.01	0.01\\
486.01	0.01\\
487.01	0.01\\
488.01	0.01\\
489.01	0.01\\
490.01	0.01\\
491.01	0.01\\
492.01	0.01\\
493.01	0.01\\
494.01	0.01\\
495.01	0.01\\
496.01	0.01\\
497.01	0.01\\
498.01	0.01\\
499.01	0.01\\
500.01	0.01\\
501.01	0.01\\
502.01	0.01\\
503.01	0.01\\
504.01	0.01\\
505.01	0.01\\
506.01	0.01\\
507.01	0.01\\
508.01	0.01\\
509.01	0.01\\
510.01	0.01\\
511.01	0.01\\
512.01	0.01\\
513.01	0.01\\
514.01	0.01\\
515.01	0.01\\
516.01	0.01\\
517.01	0.01\\
518.01	0.01\\
519.01	0.01\\
520.01	0.01\\
521.01	0.01\\
522.01	0.01\\
523.01	0.01\\
524.01	0.01\\
525.01	0.01\\
526.01	0.01\\
527.01	0.01\\
528.01	0.01\\
529.01	0.01\\
530.01	0.01\\
531.01	0.01\\
532.01	0.01\\
533.01	0.01\\
534.01	0.01\\
535.01	0.01\\
536.01	0.01\\
537.01	0.01\\
538.01	0.01\\
539.01	0.01\\
540.01	0.01\\
541.01	0.01\\
542.01	0.01\\
543.01	0.01\\
544.01	0.01\\
545.01	0.01\\
546.01	0.01\\
547.01	0.01\\
548.01	0.01\\
549.01	0.01\\
550.01	0.01\\
551.01	0.01\\
552.01	0.01\\
553.01	0.01\\
554.01	0.01\\
555.01	0.01\\
556.01	0.01\\
557.01	0.01\\
558.01	0.01\\
559.01	0.01\\
560.01	0.01\\
561.01	0.01\\
562.01	0.01\\
563.01	0.01\\
564.01	0.01\\
565.01	0.01\\
566.01	0.01\\
567.01	0.01\\
568.01	0.01\\
569.01	0.01\\
570.01	0.01\\
571.01	0.01\\
572.01	0.01\\
573.01	0.01\\
574.01	0.01\\
575.01	0.01\\
576.01	0.01\\
577.01	0.01\\
578.01	0.01\\
579.01	0.01\\
580.01	0.01\\
581.01	0.01\\
582.01	0.01\\
583.01	0.01\\
584.01	0.01\\
585.01	0.01\\
586.01	0.01\\
587.01	0.01\\
588.01	0.01\\
589.01	0.01\\
590.01	0.01\\
591.01	0.01\\
592.01	0.01\\
593.01	0.01\\
594.01	0.01\\
595.01	0.01\\
596.01	0.01\\
597.01	0.01\\
598.01	0.01\\
599.01	0.01\\
599.02	0.01\\
599.03	0.01\\
599.04	0.01\\
599.05	0.01\\
599.06	0.01\\
599.07	0.01\\
599.08	0.01\\
599.09	0.01\\
599.1	0.01\\
599.11	0.01\\
599.12	0.01\\
599.13	0.01\\
599.14	0.01\\
599.15	0.01\\
599.16	0.01\\
599.17	0.01\\
599.18	0.01\\
599.19	0.01\\
599.2	0.01\\
599.21	0.01\\
599.22	0.01\\
599.23	0.01\\
599.24	0.01\\
599.25	0.01\\
599.26	0.01\\
599.27	0.01\\
599.28	0.01\\
599.29	0.01\\
599.3	0.01\\
599.31	0.01\\
599.32	0.01\\
599.33	0.01\\
599.34	0.01\\
599.35	0.01\\
599.36	0.01\\
599.37	0.01\\
599.38	0.01\\
599.39	0.01\\
599.4	0.01\\
599.41	0.01\\
599.42	0.01\\
599.43	0.01\\
599.44	0.01\\
599.45	0.01\\
599.46	0.01\\
599.47	0.01\\
599.48	0.01\\
599.49	0.01\\
599.5	0.01\\
599.51	0.01\\
599.52	0.01\\
599.53	0.01\\
599.54	0.01\\
599.55	0.01\\
599.56	0.01\\
599.57	0.01\\
599.58	0.01\\
599.59	0.01\\
599.6	0.01\\
599.61	0.01\\
599.62	0.01\\
599.63	0.01\\
599.64	0.01\\
599.65	0.01\\
599.66	0.01\\
599.67	0.01\\
599.68	0.01\\
599.69	0.01\\
599.7	0.01\\
599.71	0.01\\
599.72	0.01\\
599.73	0.01\\
599.74	0.01\\
599.75	0.01\\
599.76	0.01\\
599.77	0.01\\
599.78	0.01\\
599.79	0.01\\
599.8	0.01\\
599.81	0.01\\
599.82	0.01\\
599.83	0.01\\
599.84	0.01\\
599.85	0.01\\
599.86	0.01\\
599.87	0.01\\
599.88	0.01\\
599.89	0.01\\
599.9	0.01\\
599.91	0.01\\
599.92	0.01\\
599.93	0.01\\
599.94	0.01\\
599.95	0.01\\
599.96	0.01\\
599.97	0.01\\
599.98	0.01\\
599.99	0.01\\
600	0.01\\
};
\addplot [color=mycolor11,solid,forget plot]
  table[row sep=crcr]{%
0.01	0.00996366700982437\\
1.01	0.00996366700982437\\
2.01	0.00996366700982437\\
3.01	0.00996366700982437\\
4.01	0.00996366700982437\\
5.01	0.00996366700982437\\
6.01	0.00996366700982437\\
7.01	0.00996366700982437\\
8.01	0.00996366700982437\\
9.01	0.00996366700982437\\
10.01	0.00996366700982437\\
11.01	0.00996366700982437\\
12.01	0.00996366700982437\\
13.01	0.00996366700982437\\
14.01	0.00996366700982437\\
15.01	0.00996366700982437\\
16.01	0.00996366700982437\\
17.01	0.00996366700982437\\
18.01	0.00996366700982437\\
19.01	0.00996366700982437\\
20.01	0.00996366700982437\\
21.01	0.00996366700982437\\
22.01	0.00996366700982437\\
23.01	0.00996366700982437\\
24.01	0.00996366700982437\\
25.01	0.00996366700982437\\
26.01	0.00996366700982437\\
27.01	0.00996366700982437\\
28.01	0.00996366700982437\\
29.01	0.00996366700982437\\
30.01	0.00996366700982437\\
31.01	0.00996366700982437\\
32.01	0.00996366700982437\\
33.01	0.00996366700982437\\
34.01	0.00996366700982437\\
35.01	0.00996366700982437\\
36.01	0.00996366700982437\\
37.01	0.00996366700982437\\
38.01	0.00996366700982437\\
39.01	0.00996366700982437\\
40.01	0.00996366700982437\\
41.01	0.00996366700982437\\
42.01	0.00996366700982437\\
43.01	0.00996366700982437\\
44.01	0.00996366700982437\\
45.01	0.00996366700982437\\
46.01	0.00996366700982437\\
47.01	0.00996366700982437\\
48.01	0.00996366700982437\\
49.01	0.00996366700982437\\
50.01	0.00996366700982437\\
51.01	0.00996366700982437\\
52.01	0.00996366700982437\\
53.01	0.00996366700982437\\
54.01	0.00996366700982437\\
55.01	0.00996366700982437\\
56.01	0.00996366700982437\\
57.01	0.00996366700982437\\
58.01	0.00996366700982437\\
59.01	0.00996366700982437\\
60.01	0.00996366700982437\\
61.01	0.00996366700982437\\
62.01	0.00996366700982437\\
63.01	0.00996366700982437\\
64.01	0.00996366700982437\\
65.01	0.00996366700982437\\
66.01	0.00996366700982437\\
67.01	0.00996366700982437\\
68.01	0.00996366700982437\\
69.01	0.00996366700982437\\
70.01	0.00996366700982437\\
71.01	0.00996366700982437\\
72.01	0.00996366700982437\\
73.01	0.00996366700982437\\
74.01	0.00996366700982437\\
75.01	0.00996366700982437\\
76.01	0.00996366700982437\\
77.01	0.00996366700982437\\
78.01	0.00996366700982437\\
79.01	0.00996366700982437\\
80.01	0.00996366700982437\\
81.01	0.00996366700982437\\
82.01	0.00996366700982437\\
83.01	0.00996366700982437\\
84.01	0.00996366700982437\\
85.01	0.00996366700982437\\
86.01	0.00996366700982437\\
87.01	0.00996366700982437\\
88.01	0.00996366700982437\\
89.01	0.00996366700982437\\
90.01	0.00996366700982437\\
91.01	0.00996366700982437\\
92.01	0.00996366700982437\\
93.01	0.00996366700982437\\
94.01	0.00996366700982437\\
95.01	0.00996366700982437\\
96.01	0.00996366700982437\\
97.01	0.00996366700982437\\
98.01	0.00996366700982437\\
99.01	0.00996366700982437\\
100.01	0.00996366700982437\\
101.01	0.00996366700982437\\
102.01	0.00996366700982437\\
103.01	0.00996366700982437\\
104.01	0.00996366700982437\\
105.01	0.00996366700982437\\
106.01	0.00996366700982437\\
107.01	0.00996366700982437\\
108.01	0.00996366700982437\\
109.01	0.00996366700982437\\
110.01	0.00996366700982437\\
111.01	0.00996366700982437\\
112.01	0.00996366700982437\\
113.01	0.00996366700982437\\
114.01	0.00996366700982437\\
115.01	0.00996366700982437\\
116.01	0.00996366700982437\\
117.01	0.00996366700982437\\
118.01	0.00996366700982437\\
119.01	0.00996366700982437\\
120.01	0.00996366700982437\\
121.01	0.00996366700982437\\
122.01	0.00996366700982437\\
123.01	0.00996366700982437\\
124.01	0.00996366700982437\\
125.01	0.00996366700982437\\
126.01	0.00996366700982437\\
127.01	0.00996366700982437\\
128.01	0.00996366700982437\\
129.01	0.00996366700982437\\
130.01	0.00996366700982437\\
131.01	0.00996366700982437\\
132.01	0.00996366700982437\\
133.01	0.00996366700982437\\
134.01	0.00996366700982437\\
135.01	0.00996366700982437\\
136.01	0.00996366700982437\\
137.01	0.00996366700982437\\
138.01	0.00996366700982437\\
139.01	0.00996366700982437\\
140.01	0.00996366700982437\\
141.01	0.00996366700982437\\
142.01	0.00996366700982437\\
143.01	0.00996366700982437\\
144.01	0.00996366700982437\\
145.01	0.00996366700982437\\
146.01	0.00996366700982437\\
147.01	0.00996366700982437\\
148.01	0.00996366700982437\\
149.01	0.00996366700982437\\
150.01	0.00996366700982437\\
151.01	0.00996366700982437\\
152.01	0.00996366700982437\\
153.01	0.00996366700982437\\
154.01	0.00996366700982437\\
155.01	0.00996366700982437\\
156.01	0.00996366700982437\\
157.01	0.00996366700982437\\
158.01	0.00996366700982437\\
159.01	0.00996366700982437\\
160.01	0.00996366700982437\\
161.01	0.00996366700982437\\
162.01	0.00996366700982437\\
163.01	0.00996366700982437\\
164.01	0.00996366700982437\\
165.01	0.00996366700982437\\
166.01	0.00996366700982437\\
167.01	0.00996366700982437\\
168.01	0.00996366700982437\\
169.01	0.00996366700982437\\
170.01	0.00996366700982437\\
171.01	0.00996366700982437\\
172.01	0.00996366700982437\\
173.01	0.00996366700982437\\
174.01	0.00996366700982437\\
175.01	0.00996366700982437\\
176.01	0.00996366700982437\\
177.01	0.00996366700982437\\
178.01	0.00996366700982437\\
179.01	0.00996366700982437\\
180.01	0.00996366700982437\\
181.01	0.00996366700982437\\
182.01	0.00996366700982437\\
183.01	0.00996366700982437\\
184.01	0.00996366700982437\\
185.01	0.00996366700982437\\
186.01	0.00996366700982437\\
187.01	0.00996366700982437\\
188.01	0.00996366700982437\\
189.01	0.00996366700982437\\
190.01	0.00996366700982437\\
191.01	0.00996366700982437\\
192.01	0.00996366700982437\\
193.01	0.00996366700982437\\
194.01	0.00996366700982437\\
195.01	0.00996366700982437\\
196.01	0.00996366700982437\\
197.01	0.00996366700982437\\
198.01	0.00996366700982437\\
199.01	0.00996366700982437\\
200.01	0.00996366700982437\\
201.01	0.00996366700982437\\
202.01	0.00996366700982437\\
203.01	0.00996366700982437\\
204.01	0.00996366700982437\\
205.01	0.00996366700982437\\
206.01	0.00996366700982437\\
207.01	0.00996366700982437\\
208.01	0.00996366700982437\\
209.01	0.00996366700982437\\
210.01	0.00996366700982437\\
211.01	0.00996366700982437\\
212.01	0.00996366700982437\\
213.01	0.00996366700982437\\
214.01	0.00996366700982437\\
215.01	0.00996366700982437\\
216.01	0.00996366700982437\\
217.01	0.00996366700982437\\
218.01	0.00996366700982437\\
219.01	0.00996366700982437\\
220.01	0.00996366700982437\\
221.01	0.00996366700982437\\
222.01	0.00996366700982437\\
223.01	0.00996366700982437\\
224.01	0.00996366700982437\\
225.01	0.00996366700982437\\
226.01	0.00996366700982437\\
227.01	0.00996366700982437\\
228.01	0.00996366700982437\\
229.01	0.00996366700982437\\
230.01	0.00996366700982437\\
231.01	0.00996366700982437\\
232.01	0.00996366700982437\\
233.01	0.00996366700982437\\
234.01	0.00996366700982437\\
235.01	0.00996366700982437\\
236.01	0.00996366700982437\\
237.01	0.00996366700982437\\
238.01	0.00996366700982437\\
239.01	0.00996366700982437\\
240.01	0.00996366700982437\\
241.01	0.00996366700982437\\
242.01	0.00996366700982437\\
243.01	0.00996366700982437\\
244.01	0.00996366700982437\\
245.01	0.00996366700982437\\
246.01	0.00996366700982437\\
247.01	0.00996366700982437\\
248.01	0.00996366700982437\\
249.01	0.00996366700982437\\
250.01	0.00996366700982437\\
251.01	0.00996366700982437\\
252.01	0.00996366700982437\\
253.01	0.00996366700982437\\
254.01	0.00996366700982437\\
255.01	0.00996366700982437\\
256.01	0.00996366700982437\\
257.01	0.00996366700982437\\
258.01	0.00996366700982437\\
259.01	0.00996366700982437\\
260.01	0.00996366700982437\\
261.01	0.00996366700982437\\
262.01	0.00996366700982437\\
263.01	0.00996366700982437\\
264.01	0.00996366700982437\\
265.01	0.00996366700982437\\
266.01	0.00996366700982437\\
267.01	0.00996366700982437\\
268.01	0.00996366700982437\\
269.01	0.00996366700982437\\
270.01	0.00996366700982437\\
271.01	0.00996366700982437\\
272.01	0.00996366700982437\\
273.01	0.00996366700982437\\
274.01	0.00996366700982437\\
275.01	0.00996366700982437\\
276.01	0.00996366700982437\\
277.01	0.00996366700982437\\
278.01	0.00996366700982437\\
279.01	0.00996366700982437\\
280.01	0.00996366700982437\\
281.01	0.00996366700982437\\
282.01	0.00996366700982437\\
283.01	0.00996366700982437\\
284.01	0.00996366700982437\\
285.01	0.00996366700982437\\
286.01	0.00996366700982437\\
287.01	0.00996366700982437\\
288.01	0.00996366700982437\\
289.01	0.00996366700982437\\
290.01	0.00996366700982437\\
291.01	0.00996366700982437\\
292.01	0.00996366700982437\\
293.01	0.00996366700982437\\
294.01	0.00996366700982437\\
295.01	0.00996366700982437\\
296.01	0.00996366700982437\\
297.01	0.00996366700982437\\
298.01	0.00996366700982437\\
299.01	0.00996366700982437\\
300.01	0.00996366700982437\\
301.01	0.00996366700982437\\
302.01	0.00996366700982437\\
303.01	0.00996366700982437\\
304.01	0.00996366700982437\\
305.01	0.00996366700982437\\
306.01	0.00996366700982437\\
307.01	0.00996366700982437\\
308.01	0.00996366700982437\\
309.01	0.00996366700982437\\
310.01	0.00996366700982437\\
311.01	0.00996366700982437\\
312.01	0.00996366700982437\\
313.01	0.00996366700982437\\
314.01	0.00996366700982437\\
315.01	0.00996366700982437\\
316.01	0.00996366700982437\\
317.01	0.00996366700982437\\
318.01	0.00996366700982437\\
319.01	0.00996366700982437\\
320.01	0.00996366700982437\\
321.01	0.00996366700982437\\
322.01	0.00996366700982437\\
323.01	0.00996366700982437\\
324.01	0.00996366700982437\\
325.01	0.00996366700982437\\
326.01	0.00996366700982437\\
327.01	0.00996366700982437\\
328.01	0.00996366700982437\\
329.01	0.00996366700982437\\
330.01	0.00996366700982437\\
331.01	0.00996366700982437\\
332.01	0.00996366700982437\\
333.01	0.00996366700982437\\
334.01	0.00996366700982437\\
335.01	0.00996366700982437\\
336.01	0.00996366700982437\\
337.01	0.00996366700982437\\
338.01	0.00996366700982437\\
339.01	0.00996366700982437\\
340.01	0.00996366700982437\\
341.01	0.00996366700982437\\
342.01	0.00996366700982437\\
343.01	0.00996366700982437\\
344.01	0.00996366700982437\\
345.01	0.00996366700982437\\
346.01	0.00996366700982437\\
347.01	0.00996366700982437\\
348.01	0.00996366700982437\\
349.01	0.00996366700982437\\
350.01	0.00996366700982437\\
351.01	0.00996366700982437\\
352.01	0.00996366700982437\\
353.01	0.00996366700982437\\
354.01	0.00996366700982437\\
355.01	0.00996366700982437\\
356.01	0.00996366700982437\\
357.01	0.00996366700982437\\
358.01	0.00996366700982437\\
359.01	0.00996366700982437\\
360.01	0.00996366700982437\\
361.01	0.00996366700982437\\
362.01	0.00996366700982437\\
363.01	0.00996366700982437\\
364.01	0.00996366700982437\\
365.01	0.00996366700982437\\
366.01	0.00996366700982437\\
367.01	0.00996366700982437\\
368.01	0.00996366700982437\\
369.01	0.00996366700982437\\
370.01	0.00996366700982437\\
371.01	0.00996366700982437\\
372.01	0.00996366700982437\\
373.01	0.00996366700982437\\
374.01	0.00996366700982437\\
375.01	0.00996366700982437\\
376.01	0.00996366700982437\\
377.01	0.00996366700982437\\
378.01	0.00996366700982437\\
379.01	0.00996366700982437\\
380.01	0.00996366700982437\\
381.01	0.00996366700982437\\
382.01	0.00996366700982437\\
383.01	0.00996366700982437\\
384.01	0.00996366700982437\\
385.01	0.00996366700982437\\
386.01	0.00996366700982437\\
387.01	0.00996366700982437\\
388.01	0.00996366700982437\\
389.01	0.00996366700982437\\
390.01	0.00996366700982437\\
391.01	0.00996366700982437\\
392.01	0.00996366700982437\\
393.01	0.00996366700982437\\
394.01	0.00996366700982437\\
395.01	0.00996366700982437\\
396.01	0.00996366700982437\\
397.01	0.00996366700982437\\
398.01	0.00996366700982437\\
399.01	0.00996366700982437\\
400.01	0.00996366700982437\\
401.01	0.00996366700982437\\
402.01	0.00996366700982437\\
403.01	0.00996366700982437\\
404.01	0.00996366700982437\\
405.01	0.00996366700982437\\
406.01	0.00996366700982437\\
407.01	0.00996366700982437\\
408.01	0.00996366700982437\\
409.01	0.00996366700982437\\
410.01	0.00996366700982437\\
411.01	0.00996366700982437\\
412.01	0.00996366700982437\\
413.01	0.00996366700982437\\
414.01	0.00996366700982437\\
415.01	0.00996366700982437\\
416.01	0.00996366700982437\\
417.01	0.00996366700982437\\
418.01	0.00996366700982437\\
419.01	0.00996366700982437\\
420.01	0.00996366700982437\\
421.01	0.00996366700982437\\
422.01	0.00996366700982437\\
423.01	0.00996366700982437\\
424.01	0.00996366700982437\\
425.01	0.00996366700982437\\
426.01	0.00996366700982437\\
427.01	0.00996366700982437\\
428.01	0.00996366700982437\\
429.01	0.00996366700982437\\
430.01	0.00996366700982437\\
431.01	0.00996366700982437\\
432.01	0.00996366700982437\\
433.01	0.00996366700982437\\
434.01	0.00996366700982437\\
435.01	0.00996366700982437\\
436.01	0.00996366700982437\\
437.01	0.00996366700982437\\
438.01	0.00996366700982437\\
439.01	0.00996366700982437\\
440.01	0.00996366700982437\\
441.01	0.00996366700982437\\
442.01	0.00996366700982437\\
443.01	0.00996366700982437\\
444.01	0.00996366700982437\\
445.01	0.00996366700982437\\
446.01	0.00996366700982439\\
447.01	0.00996366700982443\\
448.01	0.00996366700982454\\
449.01	0.00996366700982482\\
450.01	0.00996366700982559\\
451.01	0.00996366700982768\\
452.01	0.00996366700983333\\
453.01	0.00996366700984862\\
454.01	0.00996366700988997\\
455.01	0.00996366701000152\\
456.01	0.00996366701030177\\
457.01	0.00996366701110693\\
458.01	0.00996366701325515\\
459.01	0.00996366701894585\\
460.01	0.00996366703387018\\
461.01	0.00996366707246326\\
462.01	0.00996366717029316\\
463.01	0.00996366741128801\\
464.01	0.00996366798049386\\
465.01	0.00996366924142346\\
466.01	0.00996367176219535\\
467.01	0.00996367599497879\\
468.01	0.00996368134731494\\
469.01	0.00996368689941463\\
470.01	0.0099636925764223\\
471.01	0.00996369836529948\\
472.01	0.00996370423020533\\
473.01	0.00996371008258417\\
474.01	0.00996371572135344\\
475.01	0.00996372073812157\\
476.01	0.00996372445402037\\
477.01	0.00996372619616819\\
478.01	0.00996372639769227\\
479.01	0.00996372639769227\\
480.01	0.00996372639769227\\
481.01	0.00996372639769227\\
482.01	0.00996372639769227\\
483.01	0.00996372639769227\\
484.01	0.00996372639769227\\
485.01	0.00996372639769227\\
486.01	0.00996372639769227\\
487.01	0.00996372639769227\\
488.01	0.00996372639769227\\
489.01	0.00996372639769227\\
490.01	0.00996372639769227\\
491.01	0.00996372639769227\\
492.01	0.00996372639769227\\
493.01	0.00996372639769227\\
494.01	0.00996372639769227\\
495.01	0.00996372639769227\\
496.01	0.00996372639769227\\
497.01	0.00996372639769227\\
498.01	0.00996372639769227\\
499.01	0.00996372639769227\\
500.01	0.00996372639769227\\
501.01	0.00996372639769227\\
502.01	0.00996372639769227\\
503.01	0.00996372639769227\\
504.01	0.00996372639769227\\
505.01	0.00996372639769227\\
506.01	0.00996372639769227\\
507.01	0.00996372639769227\\
508.01	0.00996372639769227\\
509.01	0.00996372639769229\\
510.01	0.00996372639769231\\
511.01	0.00996372639769239\\
512.01	0.00996372639769261\\
513.01	0.00996372639769323\\
514.01	0.00996372639769497\\
515.01	0.00996372639769988\\
516.01	0.00996372639771377\\
517.01	0.00996372639775324\\
518.01	0.00996372639786592\\
519.01	0.00996372639818926\\
520.01	0.00996372639912247\\
521.01	0.0099637264018331\\
522.01	0.00996372640976174\\
523.01	0.00996372643312934\\
524.01	0.00996372650256124\\
525.01	0.00996372671065422\\
526.01	0.00996372734003741\\
527.01	0.00996372926186353\\
528.01	0.00996373518845381\\
529.01	0.00996375365113089\\
530.01	0.00996381175799089\\
531.01	0.00996399649552124\\
532.01	0.00996458955470606\\
533.01	0.00996651040092495\\
534.01	0.00997132192982238\\
535.01	0.0099766650102472\\
536.01	0.00998210853833325\\
537.01	0.00998766992706658\\
538.01	0.00999322440478397\\
539.01	0.00999809400572441\\
540.01	0.01\\
541.01	0.01\\
542.01	0.01\\
543.01	0.01\\
544.01	0.01\\
545.01	0.01\\
546.01	0.01\\
547.01	0.01\\
548.01	0.01\\
549.01	0.01\\
550.01	0.01\\
551.01	0.01\\
552.01	0.01\\
553.01	0.01\\
554.01	0.01\\
555.01	0.01\\
556.01	0.01\\
557.01	0.01\\
558.01	0.01\\
559.01	0.01\\
560.01	0.01\\
561.01	0.01\\
562.01	0.01\\
563.01	0.01\\
564.01	0.01\\
565.01	0.01\\
566.01	0.01\\
567.01	0.01\\
568.01	0.01\\
569.01	0.01\\
570.01	0.01\\
571.01	0.01\\
572.01	0.01\\
573.01	0.01\\
574.01	0.01\\
575.01	0.01\\
576.01	0.01\\
577.01	0.01\\
578.01	0.01\\
579.01	0.01\\
580.01	0.01\\
581.01	0.01\\
582.01	0.01\\
583.01	0.01\\
584.01	0.01\\
585.01	0.01\\
586.01	0.01\\
587.01	0.01\\
588.01	0.01\\
589.01	0.01\\
590.01	0.01\\
591.01	0.01\\
592.01	0.01\\
593.01	0.01\\
594.01	0.01\\
595.01	0.01\\
596.01	0.01\\
597.01	0.01\\
598.01	0.01\\
599.01	0.01\\
599.02	0.01\\
599.03	0.01\\
599.04	0.01\\
599.05	0.01\\
599.06	0.01\\
599.07	0.01\\
599.08	0.01\\
599.09	0.01\\
599.1	0.01\\
599.11	0.01\\
599.12	0.01\\
599.13	0.01\\
599.14	0.01\\
599.15	0.01\\
599.16	0.01\\
599.17	0.01\\
599.18	0.01\\
599.19	0.01\\
599.2	0.01\\
599.21	0.01\\
599.22	0.01\\
599.23	0.01\\
599.24	0.01\\
599.25	0.01\\
599.26	0.01\\
599.27	0.01\\
599.28	0.01\\
599.29	0.01\\
599.3	0.01\\
599.31	0.01\\
599.32	0.01\\
599.33	0.01\\
599.34	0.01\\
599.35	0.01\\
599.36	0.01\\
599.37	0.01\\
599.38	0.01\\
599.39	0.01\\
599.4	0.01\\
599.41	0.01\\
599.42	0.01\\
599.43	0.01\\
599.44	0.01\\
599.45	0.01\\
599.46	0.01\\
599.47	0.01\\
599.48	0.01\\
599.49	0.01\\
599.5	0.01\\
599.51	0.01\\
599.52	0.01\\
599.53	0.01\\
599.54	0.01\\
599.55	0.01\\
599.56	0.01\\
599.57	0.01\\
599.58	0.01\\
599.59	0.01\\
599.6	0.01\\
599.61	0.01\\
599.62	0.01\\
599.63	0.01\\
599.64	0.01\\
599.65	0.01\\
599.66	0.01\\
599.67	0.01\\
599.68	0.01\\
599.69	0.01\\
599.7	0.01\\
599.71	0.01\\
599.72	0.01\\
599.73	0.01\\
599.74	0.01\\
599.75	0.01\\
599.76	0.01\\
599.77	0.01\\
599.78	0.01\\
599.79	0.01\\
599.8	0.01\\
599.81	0.01\\
599.82	0.01\\
599.83	0.01\\
599.84	0.01\\
599.85	0.01\\
599.86	0.01\\
599.87	0.01\\
599.88	0.01\\
599.89	0.01\\
599.9	0.01\\
599.91	0.01\\
599.92	0.01\\
599.93	0.01\\
599.94	0.01\\
599.95	0.01\\
599.96	0.01\\
599.97	0.01\\
599.98	0.01\\
599.99	0.01\\
600	0.01\\
};
\addplot [color=mycolor12,solid,forget plot]
  table[row sep=crcr]{%
0.01	0.00967651673248613\\
1.01	0.00967651673248613\\
2.01	0.00967651673248613\\
3.01	0.00967651673248613\\
4.01	0.00967651673248613\\
5.01	0.00967651673248613\\
6.01	0.00967651673248613\\
7.01	0.00967651673248613\\
8.01	0.00967651673248613\\
9.01	0.00967651673248613\\
10.01	0.00967651673248613\\
11.01	0.00967651673248613\\
12.01	0.00967651673248613\\
13.01	0.00967651673248613\\
14.01	0.00967651673248613\\
15.01	0.00967651673248613\\
16.01	0.00967651673248613\\
17.01	0.00967651673248613\\
18.01	0.00967651673248613\\
19.01	0.00967651673248613\\
20.01	0.00967651673248613\\
21.01	0.00967651673248613\\
22.01	0.00967651673248613\\
23.01	0.00967651673248613\\
24.01	0.00967651673248613\\
25.01	0.00967651673248613\\
26.01	0.00967651673248613\\
27.01	0.00967651673248613\\
28.01	0.00967651673248613\\
29.01	0.00967651673248613\\
30.01	0.00967651673248613\\
31.01	0.00967651673248613\\
32.01	0.00967651673248613\\
33.01	0.00967651673248613\\
34.01	0.00967651673248613\\
35.01	0.00967651673248613\\
36.01	0.00967651673248613\\
37.01	0.00967651673248613\\
38.01	0.00967651673248613\\
39.01	0.00967651673248613\\
40.01	0.00967651673248613\\
41.01	0.00967651673248613\\
42.01	0.00967651673248613\\
43.01	0.00967651673248613\\
44.01	0.00967651673248613\\
45.01	0.00967651673248613\\
46.01	0.00967651673248613\\
47.01	0.00967651673248613\\
48.01	0.00967651673248613\\
49.01	0.00967651673248613\\
50.01	0.00967651673248613\\
51.01	0.00967651673248613\\
52.01	0.00967651673248613\\
53.01	0.00967651673248613\\
54.01	0.00967651673248613\\
55.01	0.00967651673248613\\
56.01	0.00967651673248613\\
57.01	0.00967651673248613\\
58.01	0.00967651673248613\\
59.01	0.00967651673248613\\
60.01	0.00967651673248613\\
61.01	0.00967651673248613\\
62.01	0.00967651673248613\\
63.01	0.00967651673248613\\
64.01	0.00967651673248613\\
65.01	0.00967651673248613\\
66.01	0.00967651673248613\\
67.01	0.00967651673248613\\
68.01	0.00967651673248613\\
69.01	0.00967651673248613\\
70.01	0.00967651673248613\\
71.01	0.00967651673248613\\
72.01	0.00967651673248613\\
73.01	0.00967651673248613\\
74.01	0.00967651673248613\\
75.01	0.00967651673248613\\
76.01	0.00967651673248613\\
77.01	0.00967651673248613\\
78.01	0.00967651673248613\\
79.01	0.00967651673248613\\
80.01	0.00967651673248613\\
81.01	0.00967651673248613\\
82.01	0.00967651673248613\\
83.01	0.00967651673248613\\
84.01	0.00967651673248613\\
85.01	0.00967651673248613\\
86.01	0.00967651673248613\\
87.01	0.00967651673248613\\
88.01	0.00967651673248613\\
89.01	0.00967651673248613\\
90.01	0.00967651673248613\\
91.01	0.00967651673248613\\
92.01	0.00967651673248613\\
93.01	0.00967651673248613\\
94.01	0.00967651673248613\\
95.01	0.00967651673248613\\
96.01	0.00967651673248613\\
97.01	0.00967651673248613\\
98.01	0.00967651673248613\\
99.01	0.00967651673248613\\
100.01	0.00967651673248613\\
101.01	0.00967651673248613\\
102.01	0.00967651673248613\\
103.01	0.00967651673248613\\
104.01	0.00967651673248613\\
105.01	0.00967651673248613\\
106.01	0.00967651673248613\\
107.01	0.00967651673248613\\
108.01	0.00967651673248613\\
109.01	0.00967651673248613\\
110.01	0.00967651673248613\\
111.01	0.00967651673248613\\
112.01	0.00967651673248613\\
113.01	0.00967651673248613\\
114.01	0.00967651673248613\\
115.01	0.00967651673248613\\
116.01	0.00967651673248613\\
117.01	0.00967651673248613\\
118.01	0.00967651673248613\\
119.01	0.00967651673248613\\
120.01	0.00967651673248613\\
121.01	0.00967651673248613\\
122.01	0.00967651673248613\\
123.01	0.00967651673248613\\
124.01	0.00967651673248613\\
125.01	0.00967651673248613\\
126.01	0.00967651673248613\\
127.01	0.00967651673248613\\
128.01	0.00967651673248613\\
129.01	0.00967651673248613\\
130.01	0.00967651673248613\\
131.01	0.00967651673248613\\
132.01	0.00967651673248613\\
133.01	0.00967651673248613\\
134.01	0.00967651673248613\\
135.01	0.00967651673248613\\
136.01	0.00967651673248613\\
137.01	0.00967651673248613\\
138.01	0.00967651673248613\\
139.01	0.00967651673248613\\
140.01	0.00967651673248613\\
141.01	0.00967651673248613\\
142.01	0.00967651673248613\\
143.01	0.00967651673248613\\
144.01	0.00967651673248613\\
145.01	0.00967651673248613\\
146.01	0.00967651673248613\\
147.01	0.00967651673248613\\
148.01	0.00967651673248613\\
149.01	0.00967651673248613\\
150.01	0.00967651673248613\\
151.01	0.00967651673248613\\
152.01	0.00967651673248613\\
153.01	0.00967651673248613\\
154.01	0.00967651673248613\\
155.01	0.00967651673248613\\
156.01	0.00967651673248613\\
157.01	0.00967651673248613\\
158.01	0.00967651673248613\\
159.01	0.00967651673248613\\
160.01	0.00967651673248613\\
161.01	0.00967651673248613\\
162.01	0.00967651673248613\\
163.01	0.00967651673248613\\
164.01	0.00967651673248613\\
165.01	0.00967651673248613\\
166.01	0.00967651673248613\\
167.01	0.00967651673248613\\
168.01	0.00967651673248613\\
169.01	0.00967651673248613\\
170.01	0.00967651673248613\\
171.01	0.00967651673248613\\
172.01	0.00967651673248613\\
173.01	0.00967651673248613\\
174.01	0.00967651673248613\\
175.01	0.00967651673248613\\
176.01	0.00967651673248613\\
177.01	0.00967651673248613\\
178.01	0.00967651673248613\\
179.01	0.00967651673248613\\
180.01	0.00967651673248613\\
181.01	0.00967651673248613\\
182.01	0.00967651673248613\\
183.01	0.00967651673248613\\
184.01	0.00967651673248613\\
185.01	0.00967651673248613\\
186.01	0.00967651673248613\\
187.01	0.00967651673248613\\
188.01	0.00967651673248613\\
189.01	0.00967651673248613\\
190.01	0.00967651673248613\\
191.01	0.00967651673248613\\
192.01	0.00967651673248613\\
193.01	0.00967651673248613\\
194.01	0.00967651673248613\\
195.01	0.00967651673248613\\
196.01	0.00967651673248613\\
197.01	0.00967651673248613\\
198.01	0.00967651673248613\\
199.01	0.00967651673248613\\
200.01	0.00967651673248613\\
201.01	0.00967651673248613\\
202.01	0.00967651673248613\\
203.01	0.00967651673248613\\
204.01	0.00967651673248613\\
205.01	0.00967651673248613\\
206.01	0.00967651673248613\\
207.01	0.00967651673248613\\
208.01	0.00967651673248613\\
209.01	0.00967651673248613\\
210.01	0.00967651673248613\\
211.01	0.00967651673248613\\
212.01	0.00967651673248613\\
213.01	0.00967651673248613\\
214.01	0.00967651673248613\\
215.01	0.00967651673248613\\
216.01	0.00967651673248613\\
217.01	0.00967651673248613\\
218.01	0.00967651673248613\\
219.01	0.00967651673248613\\
220.01	0.00967651673248613\\
221.01	0.00967651673248613\\
222.01	0.00967651673248613\\
223.01	0.00967651673248613\\
224.01	0.00967651673248613\\
225.01	0.00967651673248613\\
226.01	0.00967651673248613\\
227.01	0.00967651673248613\\
228.01	0.00967651673248613\\
229.01	0.00967651673248613\\
230.01	0.00967651673248613\\
231.01	0.00967651673248613\\
232.01	0.00967651673248613\\
233.01	0.00967651673248613\\
234.01	0.00967651673248613\\
235.01	0.00967651673248613\\
236.01	0.00967651673248613\\
237.01	0.00967651673248613\\
238.01	0.00967651673248613\\
239.01	0.00967651673248613\\
240.01	0.00967651673248613\\
241.01	0.00967651673248613\\
242.01	0.00967651673248613\\
243.01	0.00967651673248613\\
244.01	0.00967651673248613\\
245.01	0.00967651673248613\\
246.01	0.00967651673248613\\
247.01	0.00967651673248613\\
248.01	0.00967651673248613\\
249.01	0.00967651673248613\\
250.01	0.00967651673248613\\
251.01	0.00967651673248613\\
252.01	0.00967651673248613\\
253.01	0.00967651673248613\\
254.01	0.00967651673248613\\
255.01	0.00967651673248613\\
256.01	0.00967651673248613\\
257.01	0.00967651673248613\\
258.01	0.00967651673248613\\
259.01	0.00967651673248613\\
260.01	0.00967651673248613\\
261.01	0.00967651673248613\\
262.01	0.00967651673248613\\
263.01	0.00967651673248613\\
264.01	0.00967651673248613\\
265.01	0.00967651673248613\\
266.01	0.00967651673248613\\
267.01	0.00967651673248613\\
268.01	0.00967651673248613\\
269.01	0.00967651673248613\\
270.01	0.00967651673248613\\
271.01	0.00967651673248613\\
272.01	0.00967651673248613\\
273.01	0.00967651673248613\\
274.01	0.00967651673248613\\
275.01	0.00967651673248613\\
276.01	0.00967651673248613\\
277.01	0.00967651673248613\\
278.01	0.00967651673248613\\
279.01	0.00967651673248613\\
280.01	0.00967651673248613\\
281.01	0.00967651673248613\\
282.01	0.00967651673248613\\
283.01	0.00967651673248613\\
284.01	0.00967651673248613\\
285.01	0.00967651673248613\\
286.01	0.00967651673248613\\
287.01	0.00967651673248613\\
288.01	0.00967651673248613\\
289.01	0.00967651673248613\\
290.01	0.00967651673248613\\
291.01	0.00967651673248613\\
292.01	0.00967651673248613\\
293.01	0.00967651673248613\\
294.01	0.00967651673248613\\
295.01	0.00967651673248613\\
296.01	0.00967651673248613\\
297.01	0.00967651673248613\\
298.01	0.00967651673248613\\
299.01	0.00967651673248613\\
300.01	0.00967651673248613\\
301.01	0.00967651673248613\\
302.01	0.00967651673248613\\
303.01	0.00967651673248613\\
304.01	0.00967651673248613\\
305.01	0.00967651673248613\\
306.01	0.00967651673248613\\
307.01	0.00967651673248613\\
308.01	0.00967651673248613\\
309.01	0.00967651673248613\\
310.01	0.00967651673248613\\
311.01	0.00967651673248613\\
312.01	0.00967651673248613\\
313.01	0.00967651673248613\\
314.01	0.00967651673248613\\
315.01	0.00967651673248613\\
316.01	0.00967651673248613\\
317.01	0.00967651673248613\\
318.01	0.00967651673248613\\
319.01	0.00967651673248613\\
320.01	0.00967651673248613\\
321.01	0.00967651673248613\\
322.01	0.00967651673248613\\
323.01	0.00967651673248613\\
324.01	0.00967651673248613\\
325.01	0.00967651673248613\\
326.01	0.00967651673248613\\
327.01	0.00967651673248613\\
328.01	0.00967651673248613\\
329.01	0.00967651673248613\\
330.01	0.00967651673248613\\
331.01	0.00967651673248613\\
332.01	0.00967651673248613\\
333.01	0.00967651673248613\\
334.01	0.00967651673248613\\
335.01	0.00967651673248613\\
336.01	0.00967651673248613\\
337.01	0.00967651673248613\\
338.01	0.00967651673248613\\
339.01	0.00967651673248613\\
340.01	0.00967651673248613\\
341.01	0.00967651673248613\\
342.01	0.00967651673248613\\
343.01	0.00967651673248613\\
344.01	0.00967651673248613\\
345.01	0.00967651673248613\\
346.01	0.00967651673248613\\
347.01	0.00967651673248613\\
348.01	0.00967651673248613\\
349.01	0.00967651673248613\\
350.01	0.00967651673248613\\
351.01	0.00967651673248613\\
352.01	0.00967651673248613\\
353.01	0.00967651673248613\\
354.01	0.00967651673248613\\
355.01	0.00967651673248613\\
356.01	0.00967651673248613\\
357.01	0.00967651673248613\\
358.01	0.00967651673248613\\
359.01	0.00967651673248613\\
360.01	0.00967651673248613\\
361.01	0.00967651673248613\\
362.01	0.00967651673248613\\
363.01	0.00967651673248613\\
364.01	0.00967651673248613\\
365.01	0.00967651673248613\\
366.01	0.00967651673248613\\
367.01	0.00967651673248613\\
368.01	0.00967651673248613\\
369.01	0.00967651673248613\\
370.01	0.00967651673248613\\
371.01	0.00967651673248613\\
372.01	0.00967651673248613\\
373.01	0.00967651673248613\\
374.01	0.00967651673248613\\
375.01	0.00967651673248613\\
376.01	0.00967651673248613\\
377.01	0.00967651673248613\\
378.01	0.00967651673248613\\
379.01	0.00967651673248613\\
380.01	0.00967651673248613\\
381.01	0.00967651673248613\\
382.01	0.00967651673248613\\
383.01	0.00967651673248613\\
384.01	0.00967651673248613\\
385.01	0.00967651673248613\\
386.01	0.00967651673248613\\
387.01	0.00967651673248613\\
388.01	0.00967651673248613\\
389.01	0.00967651673248613\\
390.01	0.00967651673248613\\
391.01	0.00967651673248613\\
392.01	0.00967651673248613\\
393.01	0.00967651673248613\\
394.01	0.00967651673248613\\
395.01	0.00967651673248613\\
396.01	0.00967651673248613\\
397.01	0.00967651673248613\\
398.01	0.00967651673248613\\
399.01	0.00967651673248613\\
400.01	0.00967651673248613\\
401.01	0.00967651673248613\\
402.01	0.00967651673248613\\
403.01	0.00967651673248613\\
404.01	0.00967651673248613\\
405.01	0.00967651673248613\\
406.01	0.00967651673248613\\
407.01	0.00967651673248613\\
408.01	0.00967651673248613\\
409.01	0.00967651673248613\\
410.01	0.00967651673248613\\
411.01	0.00967651673248613\\
412.01	0.00967651673248613\\
413.01	0.00967651673248613\\
414.01	0.00967651673248613\\
415.01	0.00967651673248613\\
416.01	0.00967651673248613\\
417.01	0.00967651673248613\\
418.01	0.00967651673248613\\
419.01	0.00967651673248613\\
420.01	0.00967651673248613\\
421.01	0.00967651673248613\\
422.01	0.00967651673248613\\
423.01	0.00967651673248613\\
424.01	0.00967651673248613\\
425.01	0.00967651673248613\\
426.01	0.00967651673248613\\
427.01	0.00967651673248613\\
428.01	0.00967651673248613\\
429.01	0.00967651673248613\\
430.01	0.00967651673248613\\
431.01	0.00967651673248613\\
432.01	0.00967651673248613\\
433.01	0.00967651673248613\\
434.01	0.00967651673248613\\
435.01	0.00967651673248613\\
436.01	0.00967651673248613\\
437.01	0.00967651673248613\\
438.01	0.00967651673248613\\
439.01	0.00967651673248613\\
440.01	0.00967651673248613\\
441.01	0.00967651673248613\\
442.01	0.00967651673248613\\
443.01	0.00967651673248613\\
444.01	0.00967651673248613\\
445.01	0.00967651673248613\\
446.01	0.00967651673248615\\
447.01	0.00967651673248618\\
448.01	0.00967651673248628\\
449.01	0.00967651673248654\\
450.01	0.00967651673248722\\
451.01	0.00967651673248902\\
452.01	0.00967651673249374\\
453.01	0.00967651673250604\\
454.01	0.00967651673253799\\
455.01	0.00967651673262034\\
456.01	0.00967651673283089\\
457.01	0.00967651673336386\\
458.01	0.00967651673469632\\
459.01	0.00967651673797565\\
460.01	0.00967651674588576\\
461.01	0.00967651676447267\\
462.01	0.00967651680665792\\
463.01	0.00967651689801418\\
464.01	0.00967651708342587\\
465.01	0.00967651742666246\\
466.01	0.00967651798306489\\
467.01	0.00967651873185113\\
468.01	0.00967651955833319\\
469.01	0.00967652040453275\\
470.01	0.00967652126610503\\
471.01	0.00967652213679882\\
472.01	0.00967652300270552\\
473.01	0.009676523835489\\
474.01	0.00967652458375329\\
475.01	0.0096765251704144\\
476.01	0.00967652551834645\\
477.01	0.00967652562863829\\
478.01	0.00967652563495645\\
479.01	0.00967652563495645\\
480.01	0.00967652563495645\\
481.01	0.00967652563495645\\
482.01	0.00967652563495645\\
483.01	0.00967652563495645\\
484.01	0.00967652563495645\\
485.01	0.00967652563495645\\
486.01	0.00967652563495645\\
487.01	0.00967652563495645\\
488.01	0.00967652563495645\\
489.01	0.00967652563495645\\
490.01	0.00967652563495645\\
491.01	0.00967652563495645\\
492.01	0.00967652563495645\\
493.01	0.00967652563495645\\
494.01	0.00967652563495645\\
495.01	0.00967652563495645\\
496.01	0.00967652563495645\\
497.01	0.00967652563495645\\
498.01	0.00967652563495645\\
499.01	0.00967652563495645\\
500.01	0.00967652563495645\\
501.01	0.00967652563495645\\
502.01	0.00967652563495645\\
503.01	0.00967652563495645\\
504.01	0.00967652563495645\\
505.01	0.00967652563495645\\
506.01	0.00967652563495645\\
507.01	0.00967652563495645\\
508.01	0.00967652563495645\\
509.01	0.00967652563495646\\
510.01	0.00967652563495649\\
511.01	0.00967652563495657\\
512.01	0.0096765256349568\\
513.01	0.00967652563495742\\
514.01	0.00967652563495913\\
515.01	0.00967652563496388\\
516.01	0.009676525634977\\
517.01	0.00967652563501329\\
518.01	0.00967652563511373\\
519.01	0.00967652563539185\\
520.01	0.00967652563616228\\
521.01	0.00967652563829747\\
522.01	0.00967652564421621\\
523.01	0.00967652566062035\\
524.01	0.0096765257060475\\
525.01	0.00967652583160485\\
526.01	0.00967652617738298\\
527.01	0.00967652712375175\\
528.01	0.00967652968780631\\
529.01	0.00967653652331241\\
530.01	0.0096765542808498\\
531.01	0.00967659849291123\\
532.01	0.00967670065037677\\
533.01	0.0096769034162493\\
534.01	0.00967717950589578\\
535.01	0.00967745543686686\\
536.01	0.00967772559437291\\
537.01	0.0096779879879917\\
538.01	0.00967821827891777\\
539.01	0.00967836412195298\\
540.01	0.00967839043255852\\
541.01	0.0096783904325598\\
542.01	0.00967839043256345\\
543.01	0.00967839043257377\\
544.01	0.00967839043260301\\
545.01	0.00967839043268583\\
546.01	0.00967839043292068\\
547.01	0.00967839043358738\\
548.01	0.00967839043548311\\
549.01	0.00967839044088543\\
550.01	0.00967839045632535\\
551.01	0.00967839050061744\\
552.01	0.00967839062827181\\
553.01	0.00967839099829992\\
554.01	0.00967839207828928\\
555.01	0.00967839525586712\\
556.01	0.00967840469093425\\
557.01	0.00967843299007349\\
558.01	0.00967851878515725\\
559.01	0.00967878176054673\\
560.01	0.00967959648303295\\
561.01	0.00968214548147199\\
562.01	0.00969018723424609\\
563.01	0.00970947877481986\\
564.01	0.00974118436316241\\
565.01	0.00977680310942774\\
566.01	0.00981999488625989\\
567.01	0.00988259717032558\\
568.01	0.00994846893530071\\
569.01	0.00999560720848116\\
570.01	0.01\\
571.01	0.01\\
572.01	0.01\\
573.01	0.01\\
574.01	0.01\\
575.01	0.01\\
576.01	0.01\\
577.01	0.01\\
578.01	0.01\\
579.01	0.01\\
580.01	0.01\\
581.01	0.01\\
582.01	0.01\\
583.01	0.01\\
584.01	0.01\\
585.01	0.01\\
586.01	0.01\\
587.01	0.01\\
588.01	0.01\\
589.01	0.01\\
590.01	0.01\\
591.01	0.01\\
592.01	0.01\\
593.01	0.01\\
594.01	0.01\\
595.01	0.01\\
596.01	0.01\\
597.01	0.01\\
598.01	0.01\\
599.01	0.01\\
599.02	0.01\\
599.03	0.01\\
599.04	0.01\\
599.05	0.01\\
599.06	0.01\\
599.07	0.01\\
599.08	0.01\\
599.09	0.01\\
599.1	0.01\\
599.11	0.01\\
599.12	0.01\\
599.13	0.01\\
599.14	0.01\\
599.15	0.01\\
599.16	0.01\\
599.17	0.01\\
599.18	0.01\\
599.19	0.01\\
599.2	0.01\\
599.21	0.01\\
599.22	0.01\\
599.23	0.01\\
599.24	0.01\\
599.25	0.01\\
599.26	0.01\\
599.27	0.01\\
599.28	0.01\\
599.29	0.01\\
599.3	0.01\\
599.31	0.01\\
599.32	0.01\\
599.33	0.01\\
599.34	0.01\\
599.35	0.01\\
599.36	0.01\\
599.37	0.01\\
599.38	0.01\\
599.39	0.01\\
599.4	0.01\\
599.41	0.01\\
599.42	0.01\\
599.43	0.01\\
599.44	0.01\\
599.45	0.01\\
599.46	0.01\\
599.47	0.01\\
599.48	0.01\\
599.49	0.01\\
599.5	0.01\\
599.51	0.01\\
599.52	0.01\\
599.53	0.01\\
599.54	0.01\\
599.55	0.01\\
599.56	0.01\\
599.57	0.01\\
599.58	0.01\\
599.59	0.01\\
599.6	0.01\\
599.61	0.01\\
599.62	0.01\\
599.63	0.01\\
599.64	0.01\\
599.65	0.01\\
599.66	0.01\\
599.67	0.01\\
599.68	0.01\\
599.69	0.01\\
599.7	0.01\\
599.71	0.01\\
599.72	0.01\\
599.73	0.01\\
599.74	0.01\\
599.75	0.01\\
599.76	0.01\\
599.77	0.01\\
599.78	0.01\\
599.79	0.01\\
599.8	0.01\\
599.81	0.01\\
599.82	0.01\\
599.83	0.01\\
599.84	0.01\\
599.85	0.01\\
599.86	0.01\\
599.87	0.01\\
599.88	0.01\\
599.89	0.01\\
599.9	0.01\\
599.91	0.01\\
599.92	0.01\\
599.93	0.01\\
599.94	0.01\\
599.95	0.01\\
599.96	0.01\\
599.97	0.01\\
599.98	0.01\\
599.99	0.01\\
600	0.01\\
};
\addplot [color=mycolor13,solid,forget plot]
  table[row sep=crcr]{%
0.01	0.00769167322411612\\
1.01	0.00769167322411612\\
2.01	0.00769167322411612\\
3.01	0.00769167322411612\\
4.01	0.00769167322411612\\
5.01	0.00769167322411612\\
6.01	0.00769167322411612\\
7.01	0.00769167322411612\\
8.01	0.00769167322411612\\
9.01	0.00769167322411612\\
10.01	0.00769167322411612\\
11.01	0.00769167322411612\\
12.01	0.00769167322411612\\
13.01	0.00769167322411612\\
14.01	0.00769167322411612\\
15.01	0.00769167322411612\\
16.01	0.00769167322411612\\
17.01	0.00769167322411612\\
18.01	0.00769167322411612\\
19.01	0.00769167322411612\\
20.01	0.00769167322411612\\
21.01	0.00769167322411612\\
22.01	0.00769167322411612\\
23.01	0.00769167322411612\\
24.01	0.00769167322411612\\
25.01	0.00769167322411612\\
26.01	0.00769167322411612\\
27.01	0.00769167322411612\\
28.01	0.00769167322411612\\
29.01	0.00769167322411612\\
30.01	0.00769167322411612\\
31.01	0.00769167322411612\\
32.01	0.00769167322411612\\
33.01	0.00769167322411612\\
34.01	0.00769167322411612\\
35.01	0.00769167322411612\\
36.01	0.00769167322411612\\
37.01	0.00769167322411612\\
38.01	0.00769167322411612\\
39.01	0.00769167322411612\\
40.01	0.00769167322411612\\
41.01	0.00769167322411612\\
42.01	0.00769167322411612\\
43.01	0.00769167322411612\\
44.01	0.00769167322411612\\
45.01	0.00769167322411612\\
46.01	0.00769167322411612\\
47.01	0.00769167322411612\\
48.01	0.00769167322411612\\
49.01	0.00769167322411612\\
50.01	0.00769167322411612\\
51.01	0.00769167322411612\\
52.01	0.00769167322411612\\
53.01	0.00769167322411612\\
54.01	0.00769167322411612\\
55.01	0.00769167322411612\\
56.01	0.00769167322411612\\
57.01	0.00769167322411612\\
58.01	0.00769167322411612\\
59.01	0.00769167322411612\\
60.01	0.00769167322411612\\
61.01	0.00769167322411612\\
62.01	0.00769167322411612\\
63.01	0.00769167322411612\\
64.01	0.00769167322411612\\
65.01	0.00769167322411612\\
66.01	0.00769167322411612\\
67.01	0.00769167322411612\\
68.01	0.00769167322411612\\
69.01	0.00769167322411612\\
70.01	0.00769167322411612\\
71.01	0.00769167322411612\\
72.01	0.00769167322411612\\
73.01	0.00769167322411612\\
74.01	0.00769167322411612\\
75.01	0.00769167322411612\\
76.01	0.00769167322411612\\
77.01	0.00769167322411612\\
78.01	0.00769167322411612\\
79.01	0.00769167322411612\\
80.01	0.00769167322411612\\
81.01	0.00769167322411612\\
82.01	0.00769167322411612\\
83.01	0.00769167322411612\\
84.01	0.00769167322411612\\
85.01	0.00769167322411612\\
86.01	0.00769167322411612\\
87.01	0.00769167322411612\\
88.01	0.00769167322411612\\
89.01	0.00769167322411612\\
90.01	0.00769167322411612\\
91.01	0.00769167322411612\\
92.01	0.00769167322411612\\
93.01	0.00769167322411612\\
94.01	0.00769167322411612\\
95.01	0.00769167322411612\\
96.01	0.00769167322411612\\
97.01	0.00769167322411612\\
98.01	0.00769167322411612\\
99.01	0.00769167322411612\\
100.01	0.00769167322411612\\
101.01	0.00769167322411612\\
102.01	0.00769167322411612\\
103.01	0.00769167322411612\\
104.01	0.00769167322411612\\
105.01	0.00769167322411612\\
106.01	0.00769167322411612\\
107.01	0.00769167322411612\\
108.01	0.00769167322411612\\
109.01	0.00769167322411612\\
110.01	0.00769167322411612\\
111.01	0.00769167322411612\\
112.01	0.00769167322411612\\
113.01	0.00769167322411612\\
114.01	0.00769167322411612\\
115.01	0.00769167322411612\\
116.01	0.00769167322411612\\
117.01	0.00769167322411612\\
118.01	0.00769167322411612\\
119.01	0.00769167322411612\\
120.01	0.00769167322411612\\
121.01	0.00769167322411612\\
122.01	0.00769167322411612\\
123.01	0.00769167322411612\\
124.01	0.00769167322411612\\
125.01	0.00769167322411612\\
126.01	0.00769167322411612\\
127.01	0.00769167322411612\\
128.01	0.00769167322411612\\
129.01	0.00769167322411612\\
130.01	0.00769167322411612\\
131.01	0.00769167322411612\\
132.01	0.00769167322411612\\
133.01	0.00769167322411612\\
134.01	0.00769167322411612\\
135.01	0.00769167322411612\\
136.01	0.00769167322411612\\
137.01	0.00769167322411612\\
138.01	0.00769167322411612\\
139.01	0.00769167322411612\\
140.01	0.00769167322411612\\
141.01	0.00769167322411612\\
142.01	0.00769167322411612\\
143.01	0.00769167322411612\\
144.01	0.00769167322411612\\
145.01	0.00769167322411612\\
146.01	0.00769167322411612\\
147.01	0.00769167322411612\\
148.01	0.00769167322411612\\
149.01	0.00769167322411612\\
150.01	0.00769167322411612\\
151.01	0.00769167322411612\\
152.01	0.00769167322411612\\
153.01	0.00769167322411612\\
154.01	0.00769167322411612\\
155.01	0.00769167322411612\\
156.01	0.00769167322411612\\
157.01	0.00769167322411612\\
158.01	0.00769167322411612\\
159.01	0.00769167322411612\\
160.01	0.00769167322411612\\
161.01	0.00769167322411612\\
162.01	0.00769167322411612\\
163.01	0.00769167322411612\\
164.01	0.00769167322411612\\
165.01	0.00769167322411612\\
166.01	0.00769167322411612\\
167.01	0.00769167322411612\\
168.01	0.00769167322411612\\
169.01	0.00769167322411612\\
170.01	0.00769167322411612\\
171.01	0.00769167322411612\\
172.01	0.00769167322411612\\
173.01	0.00769167322411612\\
174.01	0.00769167322411612\\
175.01	0.00769167322411612\\
176.01	0.00769167322411612\\
177.01	0.00769167322411612\\
178.01	0.00769167322411612\\
179.01	0.00769167322411612\\
180.01	0.00769167322411612\\
181.01	0.00769167322411612\\
182.01	0.00769167322411612\\
183.01	0.00769167322411612\\
184.01	0.00769167322411612\\
185.01	0.00769167322411612\\
186.01	0.00769167322411612\\
187.01	0.00769167322411612\\
188.01	0.00769167322411612\\
189.01	0.00769167322411612\\
190.01	0.00769167322411612\\
191.01	0.00769167322411612\\
192.01	0.00769167322411612\\
193.01	0.00769167322411612\\
194.01	0.00769167322411612\\
195.01	0.00769167322411612\\
196.01	0.00769167322411612\\
197.01	0.00769167322411612\\
198.01	0.00769167322411612\\
199.01	0.00769167322411612\\
200.01	0.00769167322411612\\
201.01	0.00769167322411612\\
202.01	0.00769167322411612\\
203.01	0.00769167322411612\\
204.01	0.00769167322411612\\
205.01	0.00769167322411612\\
206.01	0.00769167322411612\\
207.01	0.00769167322411612\\
208.01	0.00769167322411612\\
209.01	0.00769167322411612\\
210.01	0.00769167322411612\\
211.01	0.00769167322411612\\
212.01	0.00769167322411612\\
213.01	0.00769167322411612\\
214.01	0.00769167322411612\\
215.01	0.00769167322411612\\
216.01	0.00769167322411612\\
217.01	0.00769167322411612\\
218.01	0.00769167322411612\\
219.01	0.00769167322411612\\
220.01	0.00769167322411612\\
221.01	0.00769167322411612\\
222.01	0.00769167322411612\\
223.01	0.00769167322411612\\
224.01	0.00769167322411612\\
225.01	0.00769167322411612\\
226.01	0.00769167322411612\\
227.01	0.00769167322411612\\
228.01	0.00769167322411612\\
229.01	0.00769167322411612\\
230.01	0.00769167322411612\\
231.01	0.00769167322411612\\
232.01	0.00769167322411612\\
233.01	0.00769167322411612\\
234.01	0.00769167322411612\\
235.01	0.00769167322411612\\
236.01	0.00769167322411612\\
237.01	0.00769167322411612\\
238.01	0.00769167322411612\\
239.01	0.00769167322411612\\
240.01	0.00769167322411612\\
241.01	0.00769167322411612\\
242.01	0.00769167322411612\\
243.01	0.00769167322411612\\
244.01	0.00769167322411612\\
245.01	0.00769167322411612\\
246.01	0.00769167322411612\\
247.01	0.00769167322411612\\
248.01	0.00769167322411612\\
249.01	0.00769167322411612\\
250.01	0.00769167322411612\\
251.01	0.00769167322411612\\
252.01	0.00769167322411612\\
253.01	0.00769167322411612\\
254.01	0.00769167322411612\\
255.01	0.00769167322411612\\
256.01	0.00769167322411612\\
257.01	0.00769167322411612\\
258.01	0.00769167322411612\\
259.01	0.00769167322411612\\
260.01	0.00769167322411612\\
261.01	0.00769167322411612\\
262.01	0.00769167322411612\\
263.01	0.00769167322411612\\
264.01	0.00769167322411612\\
265.01	0.00769167322411612\\
266.01	0.00769167322411612\\
267.01	0.00769167322411612\\
268.01	0.00769167322411612\\
269.01	0.00769167322411612\\
270.01	0.00769167322411612\\
271.01	0.00769167322411612\\
272.01	0.00769167322411612\\
273.01	0.00769167322411612\\
274.01	0.00769167322411612\\
275.01	0.00769167322411612\\
276.01	0.00769167322411612\\
277.01	0.00769167322411612\\
278.01	0.00769167322411612\\
279.01	0.00769167322411612\\
280.01	0.00769167322411612\\
281.01	0.00769167322411612\\
282.01	0.00769167322411612\\
283.01	0.00769167322411612\\
284.01	0.00769167322411612\\
285.01	0.00769167322411612\\
286.01	0.00769167322411612\\
287.01	0.00769167322411612\\
288.01	0.00769167322411612\\
289.01	0.00769167322411612\\
290.01	0.00769167322411612\\
291.01	0.00769167322411612\\
292.01	0.00769167322411612\\
293.01	0.00769167322411612\\
294.01	0.00769167322411612\\
295.01	0.00769167322411612\\
296.01	0.00769167322411612\\
297.01	0.00769167322411612\\
298.01	0.00769167322411612\\
299.01	0.00769167322411612\\
300.01	0.00769167322411612\\
301.01	0.00769167322411612\\
302.01	0.00769167322411612\\
303.01	0.00769167322411612\\
304.01	0.00769167322411612\\
305.01	0.00769167322411612\\
306.01	0.00769167322411612\\
307.01	0.00769167322411612\\
308.01	0.00769167322411612\\
309.01	0.00769167322411612\\
310.01	0.00769167322411612\\
311.01	0.00769167322411612\\
312.01	0.00769167322411612\\
313.01	0.00769167322411612\\
314.01	0.00769167322411612\\
315.01	0.00769167322411612\\
316.01	0.00769167322411612\\
317.01	0.00769167322411612\\
318.01	0.00769167322411612\\
319.01	0.00769167322411612\\
320.01	0.00769167322411612\\
321.01	0.00769167322411612\\
322.01	0.00769167322411612\\
323.01	0.00769167322411612\\
324.01	0.00769167322411612\\
325.01	0.00769167322411612\\
326.01	0.00769167322411612\\
327.01	0.00769167322411612\\
328.01	0.00769167322411612\\
329.01	0.00769167322411612\\
330.01	0.00769167322411612\\
331.01	0.00769167322411612\\
332.01	0.00769167322411612\\
333.01	0.00769167322411612\\
334.01	0.00769167322411612\\
335.01	0.00769167322411612\\
336.01	0.00769167322411612\\
337.01	0.00769167322411612\\
338.01	0.00769167322411612\\
339.01	0.00769167322411612\\
340.01	0.00769167322411612\\
341.01	0.00769167322411612\\
342.01	0.00769167322411612\\
343.01	0.00769167322411612\\
344.01	0.00769167322411612\\
345.01	0.00769167322411612\\
346.01	0.00769167322411612\\
347.01	0.00769167322411612\\
348.01	0.00769167322411612\\
349.01	0.00769167322411612\\
350.01	0.00769167322411612\\
351.01	0.00769167322411612\\
352.01	0.00769167322411612\\
353.01	0.00769167322411612\\
354.01	0.00769167322411612\\
355.01	0.00769167322411612\\
356.01	0.00769167322411612\\
357.01	0.00769167322411612\\
358.01	0.00769167322411612\\
359.01	0.00769167322411612\\
360.01	0.00769167322411612\\
361.01	0.00769167322411612\\
362.01	0.00769167322411612\\
363.01	0.00769167322411612\\
364.01	0.00769167322411612\\
365.01	0.00769167322411612\\
366.01	0.00769167322411612\\
367.01	0.00769167322411612\\
368.01	0.00769167322411612\\
369.01	0.00769167322411612\\
370.01	0.00769167322411612\\
371.01	0.00769167322411612\\
372.01	0.00769167322411612\\
373.01	0.00769167322411612\\
374.01	0.00769167322411612\\
375.01	0.00769167322411612\\
376.01	0.00769167322411612\\
377.01	0.00769167322411612\\
378.01	0.00769167322411612\\
379.01	0.00769167322411612\\
380.01	0.00769167322411612\\
381.01	0.00769167322411612\\
382.01	0.00769167322411612\\
383.01	0.00769167322411612\\
384.01	0.00769167322411612\\
385.01	0.00769167322411612\\
386.01	0.00769167322411612\\
387.01	0.00769167322411612\\
388.01	0.00769167322411612\\
389.01	0.00769167322411612\\
390.01	0.00769167322411612\\
391.01	0.00769167322411612\\
392.01	0.00769167322411612\\
393.01	0.00769167322411612\\
394.01	0.00769167322411612\\
395.01	0.00769167322411612\\
396.01	0.00769167322411612\\
397.01	0.00769167322411612\\
398.01	0.00769167322411612\\
399.01	0.00769167322411612\\
400.01	0.00769167322411612\\
401.01	0.00769167322411612\\
402.01	0.00769167322411612\\
403.01	0.00769167322411612\\
404.01	0.00769167322411612\\
405.01	0.00769167322411612\\
406.01	0.00769167322411612\\
407.01	0.00769167322411612\\
408.01	0.00769167322411612\\
409.01	0.00769167322411612\\
410.01	0.00769167322411612\\
411.01	0.00769167322411612\\
412.01	0.00769167322411612\\
413.01	0.00769167322411612\\
414.01	0.00769167322411612\\
415.01	0.00769167322411612\\
416.01	0.00769167322411612\\
417.01	0.00769167322411612\\
418.01	0.00769167322411612\\
419.01	0.00769167322411612\\
420.01	0.00769167322411612\\
421.01	0.00769167322411612\\
422.01	0.00769167322411612\\
423.01	0.00769167322411612\\
424.01	0.00769167322411612\\
425.01	0.00769167322411612\\
426.01	0.00769167322411612\\
427.01	0.00769167322411612\\
428.01	0.00769167322411612\\
429.01	0.00769167322411612\\
430.01	0.00769167322411612\\
431.01	0.00769167322411612\\
432.01	0.00769167322411612\\
433.01	0.00769167322411612\\
434.01	0.00769167322411612\\
435.01	0.00769167322411612\\
436.01	0.00769167322411612\\
437.01	0.00769167322411612\\
438.01	0.00769167322411612\\
439.01	0.00769167322411612\\
440.01	0.00769167322411612\\
441.01	0.00769167322411612\\
442.01	0.00769167322411612\\
443.01	0.00769167322411612\\
444.01	0.00769167322411612\\
445.01	0.00769167322411612\\
446.01	0.00769167322411612\\
447.01	0.00769167322411615\\
448.01	0.00769167322411622\\
449.01	0.00769167322411642\\
450.01	0.00769167322411692\\
451.01	0.00769167322411822\\
452.01	0.00769167322412152\\
453.01	0.00769167322412983\\
454.01	0.00769167322415054\\
455.01	0.00769167322420148\\
456.01	0.00769167322432479\\
457.01	0.00769167322461789\\
458.01	0.00769167322529933\\
459.01	0.00769167322684209\\
460.01	0.00769167323022314\\
461.01	0.00769167323734048\\
462.01	0.00769167325158358\\
463.01	0.00769167327830329\\
464.01	0.00769167332440718\\
465.01	0.00769167339575999\\
466.01	0.00769167349200403\\
467.01	0.00769167360355303\\
468.01	0.00769167371987498\\
469.01	0.00769167383808841\\
470.01	0.00769167395712654\\
471.01	0.00769167407490468\\
472.01	0.00769167418759356\\
473.01	0.00769167428894982\\
474.01	0.00769167437050099\\
475.01	0.00769167442417814\\
476.01	0.00769167444842201\\
477.01	0.00769167445342524\\
478.01	0.00769167445356292\\
479.01	0.00769167445356292\\
480.01	0.00769167445356292\\
481.01	0.00769167445356292\\
482.01	0.00769167445356292\\
483.01	0.00769167445356292\\
484.01	0.00769167445356292\\
485.01	0.00769167445356292\\
486.01	0.00769167445356292\\
487.01	0.00769167445356292\\
488.01	0.00769167445356292\\
489.01	0.00769167445356292\\
490.01	0.00769167445356292\\
491.01	0.00769167445356292\\
492.01	0.00769167445356292\\
493.01	0.00769167445356292\\
494.01	0.00769167445356292\\
495.01	0.00769167445356292\\
496.01	0.00769167445356292\\
497.01	0.00769167445356292\\
498.01	0.00769167445356292\\
499.01	0.00769167445356292\\
500.01	0.00769167445356292\\
501.01	0.00769167445356292\\
502.01	0.00769167445356292\\
503.01	0.00769167445356292\\
504.01	0.00769167445356292\\
505.01	0.00769167445356292\\
506.01	0.00769167445356292\\
507.01	0.00769167445356292\\
508.01	0.00769167445356292\\
509.01	0.00769167445356292\\
510.01	0.00769167445356294\\
511.01	0.00769167445356301\\
512.01	0.00769167445356319\\
513.01	0.00769167445356368\\
514.01	0.00769167445356503\\
515.01	0.0076916744535687\\
516.01	0.00769167445357861\\
517.01	0.00769167445360535\\
518.01	0.00769167445367712\\
519.01	0.00769167445386874\\
520.01	0.00769167445437708\\
521.01	0.00769167445571562\\
522.01	0.00769167445920878\\
523.01	0.0076916744682268\\
524.01	0.00769167449120357\\
525.01	0.00769167454880401\\
526.01	0.007691674690309\\
527.01	0.00769167502912161\\
528.01	0.00769167581380602\\
529.01	0.00769167755250824\\
530.01	0.00769168117807564\\
531.01	0.00769168810749828\\
532.01	0.00769169971533626\\
533.01	0.00769171549529513\\
534.01	0.00769173184638621\\
535.01	0.00769174699260179\\
536.01	0.00769176095622008\\
537.01	0.00769177318106421\\
538.01	0.00769178187514481\\
539.01	0.00769178552620409\\
540.01	0.00769178582668765\\
541.01	0.007691785826689\\
542.01	0.0076917858266928\\
543.01	0.00769178582670351\\
544.01	0.00769178582673366\\
545.01	0.00769178582681834\\
546.01	0.00769178582705583\\
547.01	0.00769178582772066\\
548.01	0.00769178582957759\\
549.01	0.00769178583475054\\
550.01	0.00769178584911612\\
551.01	0.00769178588886009\\
552.01	0.0076917859983129\\
553.01	0.00769178629804083\\
554.01	0.00769178711304887\\
555.01	0.0076917893094506\\
556.01	0.00769179516075398\\
557.01	0.00769181051385862\\
558.01	0.00769184997877368\\
559.01	0.00769194853645048\\
560.01	0.00769218437773804\\
561.01	0.00769271133628891\\
562.01	0.00769374871156672\\
563.01	0.00769536460442572\\
564.01	0.00769738536052675\\
565.01	0.00769972097997195\\
566.01	0.00770260954384591\\
567.01	0.00770585595853309\\
568.01	0.00770839498437718\\
569.01	0.00770944882846351\\
570.01	0.00770947501369504\\
571.01	0.00770947576129193\\
572.01	0.00770947799854698\\
573.01	0.00770948468764728\\
574.01	0.00770950466754875\\
575.01	0.00770956428293946\\
576.01	0.00770974195751159\\
577.01	0.00771027083207493\\
578.01	0.00771184298989215\\
579.01	0.00771650970156975\\
580.01	0.00773034097924347\\
581.01	0.00777121832804716\\
582.01	0.00786035135883771\\
583.01	0.00798273718433828\\
584.01	0.00814060582745707\\
585.01	0.00835882264825733\\
586.01	0.00859157438433649\\
587.01	0.00883552978830324\\
588.01	0.00910011903697823\\
589.01	0.00941202160651596\\
590.01	0.00977515234756295\\
591.01	0.00999624016460066\\
592.01	0.01\\
593.01	0.01\\
594.01	0.01\\
595.01	0.01\\
596.01	0.01\\
597.01	0.01\\
598.01	0.01\\
599.01	0.01\\
599.02	0.01\\
599.03	0.01\\
599.04	0.01\\
599.05	0.01\\
599.06	0.01\\
599.07	0.01\\
599.08	0.01\\
599.09	0.01\\
599.1	0.01\\
599.11	0.01\\
599.12	0.01\\
599.13	0.01\\
599.14	0.01\\
599.15	0.01\\
599.16	0.01\\
599.17	0.01\\
599.18	0.01\\
599.19	0.01\\
599.2	0.01\\
599.21	0.01\\
599.22	0.01\\
599.23	0.01\\
599.24	0.01\\
599.25	0.01\\
599.26	0.01\\
599.27	0.01\\
599.28	0.01\\
599.29	0.01\\
599.3	0.01\\
599.31	0.01\\
599.32	0.01\\
599.33	0.01\\
599.34	0.01\\
599.35	0.01\\
599.36	0.01\\
599.37	0.01\\
599.38	0.01\\
599.39	0.01\\
599.4	0.01\\
599.41	0.01\\
599.42	0.01\\
599.43	0.01\\
599.44	0.01\\
599.45	0.01\\
599.46	0.01\\
599.47	0.01\\
599.48	0.01\\
599.49	0.01\\
599.5	0.01\\
599.51	0.01\\
599.52	0.01\\
599.53	0.01\\
599.54	0.01\\
599.55	0.01\\
599.56	0.01\\
599.57	0.01\\
599.58	0.01\\
599.59	0.01\\
599.6	0.01\\
599.61	0.01\\
599.62	0.01\\
599.63	0.01\\
599.64	0.01\\
599.65	0.01\\
599.66	0.01\\
599.67	0.01\\
599.68	0.01\\
599.69	0.01\\
599.7	0.01\\
599.71	0.01\\
599.72	0.01\\
599.73	0.01\\
599.74	0.01\\
599.75	0.01\\
599.76	0.01\\
599.77	0.01\\
599.78	0.01\\
599.79	0.01\\
599.8	0.01\\
599.81	0.01\\
599.82	0.01\\
599.83	0.01\\
599.84	0.01\\
599.85	0.01\\
599.86	0.01\\
599.87	0.01\\
599.88	0.01\\
599.89	0.01\\
599.9	0.01\\
599.91	0.01\\
599.92	0.01\\
599.93	0.01\\
599.94	0.01\\
599.95	0.01\\
599.96	0.01\\
599.97	0.01\\
599.98	0.01\\
599.99	0.01\\
600	0.01\\
};
\addplot [color=mycolor14,solid,forget plot]
  table[row sep=crcr]{%
0.01	0.01\\
1.01	0.01\\
2.01	0.01\\
3.01	0.01\\
4.01	0.01\\
5.01	0.01\\
6.01	0.01\\
7.01	0.01\\
8.01	0.01\\
9.01	0.01\\
10.01	0.01\\
11.01	0.01\\
12.01	0.01\\
13.01	0.01\\
14.01	0.01\\
15.01	0.01\\
16.01	0.01\\
17.01	0.01\\
18.01	0.01\\
19.01	0.01\\
20.01	0.01\\
21.01	0.01\\
22.01	0.01\\
23.01	0.01\\
24.01	0.01\\
25.01	0.01\\
26.01	0.01\\
27.01	0.01\\
28.01	0.01\\
29.01	0.01\\
30.01	0.01\\
31.01	0.01\\
32.01	0.01\\
33.01	0.01\\
34.01	0.01\\
35.01	0.01\\
36.01	0.01\\
37.01	0.01\\
38.01	0.01\\
39.01	0.01\\
40.01	0.01\\
41.01	0.01\\
42.01	0.01\\
43.01	0.01\\
44.01	0.01\\
45.01	0.01\\
46.01	0.01\\
47.01	0.01\\
48.01	0.01\\
49.01	0.01\\
50.01	0.01\\
51.01	0.01\\
52.01	0.01\\
53.01	0.01\\
54.01	0.01\\
55.01	0.01\\
56.01	0.01\\
57.01	0.01\\
58.01	0.01\\
59.01	0.01\\
60.01	0.01\\
61.01	0.01\\
62.01	0.01\\
63.01	0.01\\
64.01	0.01\\
65.01	0.01\\
66.01	0.01\\
67.01	0.01\\
68.01	0.01\\
69.01	0.01\\
70.01	0.01\\
71.01	0.01\\
72.01	0.01\\
73.01	0.01\\
74.01	0.01\\
75.01	0.01\\
76.01	0.01\\
77.01	0.01\\
78.01	0.01\\
79.01	0.01\\
80.01	0.01\\
81.01	0.01\\
82.01	0.01\\
83.01	0.01\\
84.01	0.01\\
85.01	0.01\\
86.01	0.01\\
87.01	0.01\\
88.01	0.01\\
89.01	0.01\\
90.01	0.01\\
91.01	0.01\\
92.01	0.01\\
93.01	0.01\\
94.01	0.01\\
95.01	0.01\\
96.01	0.01\\
97.01	0.01\\
98.01	0.01\\
99.01	0.01\\
100.01	0.01\\
101.01	0.01\\
102.01	0.01\\
103.01	0.01\\
104.01	0.01\\
105.01	0.01\\
106.01	0.01\\
107.01	0.01\\
108.01	0.01\\
109.01	0.01\\
110.01	0.01\\
111.01	0.01\\
112.01	0.01\\
113.01	0.01\\
114.01	0.01\\
115.01	0.01\\
116.01	0.01\\
117.01	0.01\\
118.01	0.01\\
119.01	0.01\\
120.01	0.01\\
121.01	0.01\\
122.01	0.01\\
123.01	0.01\\
124.01	0.01\\
125.01	0.01\\
126.01	0.01\\
127.01	0.01\\
128.01	0.01\\
129.01	0.01\\
130.01	0.01\\
131.01	0.01\\
132.01	0.01\\
133.01	0.01\\
134.01	0.01\\
135.01	0.01\\
136.01	0.01\\
137.01	0.01\\
138.01	0.01\\
139.01	0.01\\
140.01	0.01\\
141.01	0.01\\
142.01	0.01\\
143.01	0.01\\
144.01	0.01\\
145.01	0.01\\
146.01	0.01\\
147.01	0.01\\
148.01	0.01\\
149.01	0.01\\
150.01	0.01\\
151.01	0.01\\
152.01	0.01\\
153.01	0.01\\
154.01	0.01\\
155.01	0.01\\
156.01	0.01\\
157.01	0.01\\
158.01	0.01\\
159.01	0.01\\
160.01	0.01\\
161.01	0.01\\
162.01	0.01\\
163.01	0.01\\
164.01	0.01\\
165.01	0.01\\
166.01	0.01\\
167.01	0.01\\
168.01	0.01\\
169.01	0.01\\
170.01	0.01\\
171.01	0.01\\
172.01	0.01\\
173.01	0.01\\
174.01	0.01\\
175.01	0.01\\
176.01	0.01\\
177.01	0.01\\
178.01	0.01\\
179.01	0.01\\
180.01	0.01\\
181.01	0.01\\
182.01	0.01\\
183.01	0.01\\
184.01	0.01\\
185.01	0.01\\
186.01	0.01\\
187.01	0.01\\
188.01	0.01\\
189.01	0.01\\
190.01	0.01\\
191.01	0.01\\
192.01	0.01\\
193.01	0.01\\
194.01	0.01\\
195.01	0.01\\
196.01	0.01\\
197.01	0.01\\
198.01	0.01\\
199.01	0.01\\
200.01	0.01\\
201.01	0.01\\
202.01	0.01\\
203.01	0.01\\
204.01	0.01\\
205.01	0.01\\
206.01	0.01\\
207.01	0.01\\
208.01	0.01\\
209.01	0.01\\
210.01	0.01\\
211.01	0.01\\
212.01	0.01\\
213.01	0.01\\
214.01	0.01\\
215.01	0.01\\
216.01	0.01\\
217.01	0.01\\
218.01	0.01\\
219.01	0.01\\
220.01	0.01\\
221.01	0.01\\
222.01	0.01\\
223.01	0.01\\
224.01	0.01\\
225.01	0.01\\
226.01	0.01\\
227.01	0.01\\
228.01	0.01\\
229.01	0.01\\
230.01	0.01\\
231.01	0.01\\
232.01	0.01\\
233.01	0.01\\
234.01	0.01\\
235.01	0.01\\
236.01	0.01\\
237.01	0.01\\
238.01	0.01\\
239.01	0.01\\
240.01	0.01\\
241.01	0.01\\
242.01	0.01\\
243.01	0.01\\
244.01	0.01\\
245.01	0.01\\
246.01	0.01\\
247.01	0.01\\
248.01	0.01\\
249.01	0.01\\
250.01	0.01\\
251.01	0.01\\
252.01	0.01\\
253.01	0.01\\
254.01	0.01\\
255.01	0.01\\
256.01	0.01\\
257.01	0.01\\
258.01	0.01\\
259.01	0.01\\
260.01	0.01\\
261.01	0.01\\
262.01	0.01\\
263.01	0.01\\
264.01	0.01\\
265.01	0.01\\
266.01	0.01\\
267.01	0.01\\
268.01	0.01\\
269.01	0.01\\
270.01	0.01\\
271.01	0.01\\
272.01	0.01\\
273.01	0.01\\
274.01	0.01\\
275.01	0.01\\
276.01	0.01\\
277.01	0.01\\
278.01	0.01\\
279.01	0.01\\
280.01	0.01\\
281.01	0.01\\
282.01	0.01\\
283.01	0.01\\
284.01	0.01\\
285.01	0.01\\
286.01	0.01\\
287.01	0.01\\
288.01	0.01\\
289.01	0.01\\
290.01	0.01\\
291.01	0.01\\
292.01	0.01\\
293.01	0.01\\
294.01	0.01\\
295.01	0.01\\
296.01	0.01\\
297.01	0.01\\
298.01	0.01\\
299.01	0.01\\
300.01	0.01\\
301.01	0.01\\
302.01	0.01\\
303.01	0.01\\
304.01	0.01\\
305.01	0.01\\
306.01	0.01\\
307.01	0.01\\
308.01	0.01\\
309.01	0.01\\
310.01	0.01\\
311.01	0.01\\
312.01	0.01\\
313.01	0.01\\
314.01	0.01\\
315.01	0.01\\
316.01	0.01\\
317.01	0.01\\
318.01	0.01\\
319.01	0.01\\
320.01	0.01\\
321.01	0.01\\
322.01	0.01\\
323.01	0.01\\
324.01	0.01\\
325.01	0.01\\
326.01	0.01\\
327.01	0.01\\
328.01	0.01\\
329.01	0.01\\
330.01	0.01\\
331.01	0.01\\
332.01	0.01\\
333.01	0.01\\
334.01	0.01\\
335.01	0.01\\
336.01	0.01\\
337.01	0.01\\
338.01	0.01\\
339.01	0.01\\
340.01	0.01\\
341.01	0.01\\
342.01	0.01\\
343.01	0.01\\
344.01	0.01\\
345.01	0.01\\
346.01	0.01\\
347.01	0.01\\
348.01	0.01\\
349.01	0.01\\
350.01	0.01\\
351.01	0.01\\
352.01	0.01\\
353.01	0.01\\
354.01	0.01\\
355.01	0.01\\
356.01	0.01\\
357.01	0.01\\
358.01	0.01\\
359.01	0.01\\
360.01	0.01\\
361.01	0.01\\
362.01	0.01\\
363.01	0.01\\
364.01	0.01\\
365.01	0.01\\
366.01	0.01\\
367.01	0.01\\
368.01	0.01\\
369.01	0.01\\
370.01	0.01\\
371.01	0.01\\
372.01	0.01\\
373.01	0.01\\
374.01	0.01\\
375.01	0.01\\
376.01	0.01\\
377.01	0.01\\
378.01	0.01\\
379.01	0.01\\
380.01	0.01\\
381.01	0.01\\
382.01	0.01\\
383.01	0.01\\
384.01	0.01\\
385.01	0.01\\
386.01	0.01\\
387.01	0.01\\
388.01	0.01\\
389.01	0.01\\
390.01	0.01\\
391.01	0.01\\
392.01	0.01\\
393.01	0.01\\
394.01	0.01\\
395.01	0.01\\
396.01	0.01\\
397.01	0.01\\
398.01	0.01\\
399.01	0.01\\
400.01	0.01\\
401.01	0.01\\
402.01	0.01\\
403.01	0.01\\
404.01	0.01\\
405.01	0.01\\
406.01	0.01\\
407.01	0.01\\
408.01	0.01\\
409.01	0.01\\
410.01	0.01\\
411.01	0.01\\
412.01	0.01\\
413.01	0.01\\
414.01	0.01\\
415.01	0.01\\
416.01	0.01\\
417.01	0.01\\
418.01	0.01\\
419.01	0.01\\
420.01	0.01\\
421.01	0.01\\
422.01	0.01\\
423.01	0.01\\
424.01	0.01\\
425.01	0.01\\
426.01	0.01\\
427.01	0.01\\
428.01	0.01\\
429.01	0.01\\
430.01	0.01\\
431.01	0.01\\
432.01	0.01\\
433.01	0.01\\
434.01	0.01\\
435.01	0.01\\
436.01	0.01\\
437.01	0.01\\
438.01	0.01\\
439.01	0.01\\
440.01	0.01\\
441.01	0.01\\
442.01	0.01\\
443.01	0.01\\
444.01	0.01\\
445.01	0.01\\
446.01	0.01\\
447.01	0.01\\
448.01	0.01\\
449.01	0.01\\
450.01	0.01\\
451.01	0.01\\
452.01	0.01\\
453.01	0.01\\
454.01	0.01\\
455.01	0.01\\
456.01	0.01\\
457.01	0.01\\
458.01	0.01\\
459.01	0.01\\
460.01	0.01\\
461.01	0.01\\
462.01	0.01\\
463.01	0.01\\
464.01	0.01\\
465.01	0.01\\
466.01	0.01\\
467.01	0.01\\
468.01	0.01\\
469.01	0.01\\
470.01	0.01\\
471.01	0.01\\
472.01	0.01\\
473.01	0.01\\
474.01	0.01\\
475.01	0.01\\
476.01	0.01\\
477.01	0.01\\
478.01	0.01\\
479.01	0.01\\
480.01	0.01\\
481.01	0.01\\
482.01	0.01\\
483.01	0.01\\
484.01	0.01\\
485.01	0.01\\
486.01	0.01\\
487.01	0.01\\
488.01	0.01\\
489.01	0.01\\
490.01	0.01\\
491.01	0.01\\
492.01	0.01\\
493.01	0.01\\
494.01	0.01\\
495.01	0.01\\
496.01	0.01\\
497.01	0.01\\
498.01	0.01\\
499.01	0.01\\
500.01	0.01\\
501.01	0.01\\
502.01	0.01\\
503.01	0.01\\
504.01	0.01\\
505.01	0.01\\
506.01	0.01\\
507.01	0.01\\
508.01	0.01\\
509.01	0.01\\
510.01	0.01\\
511.01	0.01\\
512.01	0.01\\
513.01	0.01\\
514.01	0.01\\
515.01	0.01\\
516.01	0.01\\
517.01	0.01\\
518.01	0.01\\
519.01	0.01\\
520.01	0.01\\
521.01	0.01\\
522.01	0.01\\
523.01	0.01\\
524.01	0.01\\
525.01	0.01\\
526.01	0.01\\
527.01	0.01\\
528.01	0.01\\
529.01	0.01\\
530.01	0.01\\
531.01	0.01\\
532.01	0.01\\
533.01	0.01\\
534.01	0.01\\
535.01	0.01\\
536.01	0.01\\
537.01	0.01\\
538.01	0.01\\
539.01	0.01\\
540.01	0.01\\
541.01	0.01\\
542.01	0.01\\
543.01	0.01\\
544.01	0.01\\
545.01	0.01\\
546.01	0.01\\
547.01	0.01\\
548.01	0.01\\
549.01	0.01\\
550.01	0.01\\
551.01	0.01\\
552.01	0.01\\
553.01	0.01\\
554.01	0.01\\
555.01	0.01\\
556.01	0.01\\
557.01	0.01\\
558.01	0.01\\
559.01	0.01\\
560.01	0.01\\
561.01	0.01\\
562.01	0.01\\
563.01	0.01\\
564.01	0.01\\
565.01	0.01\\
566.01	0.01\\
567.01	0.01\\
568.01	0.01\\
569.01	0.01\\
570.01	0.01\\
571.01	0.01\\
572.01	0.01\\
573.01	0.01\\
574.01	0.01\\
575.01	0.01\\
576.01	0.01\\
577.01	0.01\\
578.01	0.01\\
579.01	0.01\\
580.01	0.01\\
581.01	0.01\\
582.01	0.01\\
583.01	0.01\\
584.01	0.01\\
585.01	0.01\\
586.01	0.01\\
587.01	0.01\\
588.01	0.01\\
589.01	0.01\\
590.01	0.01\\
591.01	0.01\\
592.01	0.01\\
593.01	0.01\\
594.01	0.01\\
595.01	0.01\\
596.01	0.01\\
597.01	0.01\\
598.01	0.01\\
599.01	0.00776171810396721\\
599.02	0.00771170284599059\\
599.03	0.00766129155011235\\
599.04	0.00761048017065781\\
599.05	0.00755926461338118\\
599.06	0.00750764073475421\\
599.07	0.00745560434124214\\
599.08	0.00740315118856652\\
599.09	0.00735027698095472\\
599.1	0.0072969773703757\\
599.11	0.00724324795576186\\
599.12	0.00718908428221659\\
599.13	0.00713448184020714\\
599.14	0.00707943606474259\\
599.15	0.00702394233453649\\
599.16	0.00696799597115382\\
599.17	0.00691159223814197\\
599.18	0.00685472634014519\\
599.19	0.00679739342200237\\
599.2	0.00673958856782754\\
599.21	0.00668130680007274\\
599.22	0.00662254307857288\\
599.23	0.00656329229957207\\
599.24	0.00650354929473104\\
599.25	0.00644330883011511\\
599.26	0.0063825656051623\\
599.27	0.00632131425163102\\
599.28	0.00625954933252691\\
599.29	0.00619726534100818\\
599.3	0.00613445669926893\\
599.31	0.00607111775739999\\
599.32	0.00600724279222646\\
599.33	0.00594282600612167\\
599.34	0.00587786152579658\\
599.35	0.00581234340106422\\
599.36	0.0057462656035784\\
599.37	0.00567962202554607\\
599.38	0.00561240647841241\\
599.39	0.00554461269151822\\
599.4	0.00547623431072853\\
599.41	0.0054072648970319\\
599.42	0.00533769792510932\\
599.43	0.0052675267818721\\
599.44	0.00519674476496768\\
599.45	0.0051253450812526\\
599.46	0.00505332084523055\\
599.47	0.004980665077417\\
599.48	0.00490737070270704\\
599.49	0.00483343054870632\\
599.5	0.00475883734402394\\
599.51	0.00468358371652611\\
599.52	0.00460766219154944\\
599.53	0.00453106519007261\\
599.54	0.00445378502684509\\
599.55	0.0043758139084716\\
599.56	0.00429714393145104\\
599.57	0.00421776708016814\\
599.58	0.00413767522483669\\
599.59	0.00405686011939259\\
599.6	0.003975313399335\\
599.61	0.00389302657951419\\
599.62	0.00380999105186401\\
599.63	0.00372619808307736\\
599.64	0.00364163881222268\\
599.65	0.00355630424829939\\
599.66	0.00347018526773029\\
599.67	0.00338327261178877\\
599.68	0.00329555688395846\\
599.69	0.00320702854722304\\
599.7	0.00311767792128371\\
599.71	0.00302749517970179\\
599.72	0.0029364703469637\\
599.73	0.00284459329546554\\
599.74	0.00275185374241435\\
599.75	0.00265824124664297\\
599.76	0.00256374520533526\\
599.77	0.0024683548506584\\
599.78	0.00237205924629865\\
599.79	0.00227484728389699\\
599.8	0.00217670767938078\\
599.81	0.00207762896918731\\
599.82	0.00197759950637519\\
599.83	0.00187660745661892\\
599.84	0.00177464079408229\\
599.85	0.00167168729716548\\
599.86	0.00156773454412088\\
599.87	0.00146276990853229\\
599.88	0.00135678055465179\\
599.89	0.00124975343258846\\
599.9	0.0011416752733427\\
599.91	0.0010325325836796\\
599.92	0.000922311640834542\\
599.93	0.000810998487043897\\
599.94	0.000698578923893012\\
599.95	0.00058503850647375\\
599.96	0.000470362537343004\\
599.97	0.000354536060273428\\
599.98	0.000237543853786985\\
599.99	0.000119370424461526\\
600	0\\
};
\addplot [color=mycolor15,solid,forget plot]
  table[row sep=crcr]{%
0.01	0.01\\
1.01	0.01\\
2.01	0.01\\
3.01	0.01\\
4.01	0.01\\
5.01	0.01\\
6.01	0.01\\
7.01	0.01\\
8.01	0.01\\
9.01	0.01\\
10.01	0.01\\
11.01	0.01\\
12.01	0.01\\
13.01	0.01\\
14.01	0.01\\
15.01	0.01\\
16.01	0.01\\
17.01	0.01\\
18.01	0.01\\
19.01	0.01\\
20.01	0.01\\
21.01	0.01\\
22.01	0.01\\
23.01	0.01\\
24.01	0.01\\
25.01	0.01\\
26.01	0.01\\
27.01	0.01\\
28.01	0.01\\
29.01	0.01\\
30.01	0.01\\
31.01	0.01\\
32.01	0.01\\
33.01	0.01\\
34.01	0.01\\
35.01	0.01\\
36.01	0.01\\
37.01	0.01\\
38.01	0.01\\
39.01	0.01\\
40.01	0.01\\
41.01	0.01\\
42.01	0.01\\
43.01	0.01\\
44.01	0.01\\
45.01	0.01\\
46.01	0.01\\
47.01	0.01\\
48.01	0.01\\
49.01	0.01\\
50.01	0.01\\
51.01	0.01\\
52.01	0.01\\
53.01	0.01\\
54.01	0.01\\
55.01	0.01\\
56.01	0.01\\
57.01	0.01\\
58.01	0.01\\
59.01	0.01\\
60.01	0.01\\
61.01	0.01\\
62.01	0.01\\
63.01	0.01\\
64.01	0.01\\
65.01	0.01\\
66.01	0.01\\
67.01	0.01\\
68.01	0.01\\
69.01	0.01\\
70.01	0.01\\
71.01	0.01\\
72.01	0.01\\
73.01	0.01\\
74.01	0.01\\
75.01	0.01\\
76.01	0.01\\
77.01	0.01\\
78.01	0.01\\
79.01	0.01\\
80.01	0.01\\
81.01	0.01\\
82.01	0.01\\
83.01	0.01\\
84.01	0.01\\
85.01	0.01\\
86.01	0.01\\
87.01	0.01\\
88.01	0.01\\
89.01	0.01\\
90.01	0.01\\
91.01	0.01\\
92.01	0.01\\
93.01	0.01\\
94.01	0.01\\
95.01	0.01\\
96.01	0.01\\
97.01	0.01\\
98.01	0.01\\
99.01	0.01\\
100.01	0.01\\
101.01	0.01\\
102.01	0.01\\
103.01	0.01\\
104.01	0.01\\
105.01	0.01\\
106.01	0.01\\
107.01	0.01\\
108.01	0.01\\
109.01	0.01\\
110.01	0.01\\
111.01	0.01\\
112.01	0.01\\
113.01	0.01\\
114.01	0.01\\
115.01	0.01\\
116.01	0.01\\
117.01	0.01\\
118.01	0.01\\
119.01	0.01\\
120.01	0.01\\
121.01	0.01\\
122.01	0.01\\
123.01	0.01\\
124.01	0.01\\
125.01	0.01\\
126.01	0.01\\
127.01	0.01\\
128.01	0.01\\
129.01	0.01\\
130.01	0.01\\
131.01	0.01\\
132.01	0.01\\
133.01	0.01\\
134.01	0.01\\
135.01	0.01\\
136.01	0.01\\
137.01	0.01\\
138.01	0.01\\
139.01	0.01\\
140.01	0.01\\
141.01	0.01\\
142.01	0.01\\
143.01	0.01\\
144.01	0.01\\
145.01	0.01\\
146.01	0.01\\
147.01	0.01\\
148.01	0.01\\
149.01	0.01\\
150.01	0.01\\
151.01	0.01\\
152.01	0.01\\
153.01	0.01\\
154.01	0.01\\
155.01	0.01\\
156.01	0.01\\
157.01	0.01\\
158.01	0.01\\
159.01	0.01\\
160.01	0.01\\
161.01	0.01\\
162.01	0.01\\
163.01	0.01\\
164.01	0.01\\
165.01	0.01\\
166.01	0.01\\
167.01	0.01\\
168.01	0.01\\
169.01	0.01\\
170.01	0.01\\
171.01	0.01\\
172.01	0.01\\
173.01	0.01\\
174.01	0.01\\
175.01	0.01\\
176.01	0.01\\
177.01	0.01\\
178.01	0.01\\
179.01	0.01\\
180.01	0.01\\
181.01	0.01\\
182.01	0.01\\
183.01	0.01\\
184.01	0.01\\
185.01	0.01\\
186.01	0.01\\
187.01	0.01\\
188.01	0.01\\
189.01	0.01\\
190.01	0.01\\
191.01	0.01\\
192.01	0.01\\
193.01	0.01\\
194.01	0.01\\
195.01	0.01\\
196.01	0.01\\
197.01	0.01\\
198.01	0.01\\
199.01	0.01\\
200.01	0.01\\
201.01	0.01\\
202.01	0.01\\
203.01	0.01\\
204.01	0.01\\
205.01	0.01\\
206.01	0.01\\
207.01	0.01\\
208.01	0.01\\
209.01	0.01\\
210.01	0.01\\
211.01	0.01\\
212.01	0.01\\
213.01	0.01\\
214.01	0.01\\
215.01	0.01\\
216.01	0.01\\
217.01	0.01\\
218.01	0.01\\
219.01	0.01\\
220.01	0.01\\
221.01	0.01\\
222.01	0.01\\
223.01	0.01\\
224.01	0.01\\
225.01	0.01\\
226.01	0.01\\
227.01	0.01\\
228.01	0.01\\
229.01	0.01\\
230.01	0.01\\
231.01	0.01\\
232.01	0.01\\
233.01	0.01\\
234.01	0.01\\
235.01	0.01\\
236.01	0.01\\
237.01	0.01\\
238.01	0.01\\
239.01	0.01\\
240.01	0.01\\
241.01	0.01\\
242.01	0.01\\
243.01	0.01\\
244.01	0.01\\
245.01	0.01\\
246.01	0.01\\
247.01	0.01\\
248.01	0.01\\
249.01	0.01\\
250.01	0.01\\
251.01	0.01\\
252.01	0.01\\
253.01	0.01\\
254.01	0.01\\
255.01	0.01\\
256.01	0.01\\
257.01	0.01\\
258.01	0.01\\
259.01	0.01\\
260.01	0.01\\
261.01	0.01\\
262.01	0.01\\
263.01	0.01\\
264.01	0.01\\
265.01	0.01\\
266.01	0.01\\
267.01	0.01\\
268.01	0.01\\
269.01	0.01\\
270.01	0.01\\
271.01	0.01\\
272.01	0.01\\
273.01	0.01\\
274.01	0.01\\
275.01	0.01\\
276.01	0.01\\
277.01	0.01\\
278.01	0.01\\
279.01	0.01\\
280.01	0.01\\
281.01	0.01\\
282.01	0.01\\
283.01	0.01\\
284.01	0.01\\
285.01	0.01\\
286.01	0.01\\
287.01	0.01\\
288.01	0.01\\
289.01	0.01\\
290.01	0.01\\
291.01	0.01\\
292.01	0.01\\
293.01	0.01\\
294.01	0.01\\
295.01	0.01\\
296.01	0.01\\
297.01	0.01\\
298.01	0.01\\
299.01	0.01\\
300.01	0.01\\
301.01	0.01\\
302.01	0.01\\
303.01	0.01\\
304.01	0.01\\
305.01	0.01\\
306.01	0.01\\
307.01	0.01\\
308.01	0.01\\
309.01	0.01\\
310.01	0.01\\
311.01	0.01\\
312.01	0.01\\
313.01	0.01\\
314.01	0.01\\
315.01	0.01\\
316.01	0.01\\
317.01	0.01\\
318.01	0.01\\
319.01	0.01\\
320.01	0.01\\
321.01	0.01\\
322.01	0.01\\
323.01	0.01\\
324.01	0.01\\
325.01	0.01\\
326.01	0.01\\
327.01	0.01\\
328.01	0.01\\
329.01	0.01\\
330.01	0.01\\
331.01	0.01\\
332.01	0.01\\
333.01	0.01\\
334.01	0.01\\
335.01	0.01\\
336.01	0.01\\
337.01	0.01\\
338.01	0.01\\
339.01	0.01\\
340.01	0.01\\
341.01	0.01\\
342.01	0.01\\
343.01	0.01\\
344.01	0.01\\
345.01	0.01\\
346.01	0.01\\
347.01	0.01\\
348.01	0.01\\
349.01	0.01\\
350.01	0.01\\
351.01	0.01\\
352.01	0.01\\
353.01	0.01\\
354.01	0.01\\
355.01	0.01\\
356.01	0.01\\
357.01	0.01\\
358.01	0.01\\
359.01	0.01\\
360.01	0.01\\
361.01	0.01\\
362.01	0.01\\
363.01	0.01\\
364.01	0.01\\
365.01	0.01\\
366.01	0.01\\
367.01	0.01\\
368.01	0.01\\
369.01	0.01\\
370.01	0.01\\
371.01	0.01\\
372.01	0.01\\
373.01	0.01\\
374.01	0.01\\
375.01	0.01\\
376.01	0.01\\
377.01	0.01\\
378.01	0.01\\
379.01	0.01\\
380.01	0.01\\
381.01	0.01\\
382.01	0.01\\
383.01	0.01\\
384.01	0.01\\
385.01	0.01\\
386.01	0.01\\
387.01	0.01\\
388.01	0.01\\
389.01	0.01\\
390.01	0.01\\
391.01	0.01\\
392.01	0.01\\
393.01	0.01\\
394.01	0.01\\
395.01	0.01\\
396.01	0.01\\
397.01	0.01\\
398.01	0.01\\
399.01	0.01\\
400.01	0.01\\
401.01	0.01\\
402.01	0.01\\
403.01	0.01\\
404.01	0.01\\
405.01	0.01\\
406.01	0.01\\
407.01	0.01\\
408.01	0.01\\
409.01	0.01\\
410.01	0.01\\
411.01	0.01\\
412.01	0.01\\
413.01	0.01\\
414.01	0.01\\
415.01	0.01\\
416.01	0.01\\
417.01	0.01\\
418.01	0.01\\
419.01	0.01\\
420.01	0.01\\
421.01	0.01\\
422.01	0.01\\
423.01	0.01\\
424.01	0.01\\
425.01	0.01\\
426.01	0.01\\
427.01	0.01\\
428.01	0.01\\
429.01	0.01\\
430.01	0.01\\
431.01	0.01\\
432.01	0.01\\
433.01	0.01\\
434.01	0.01\\
435.01	0.01\\
436.01	0.01\\
437.01	0.01\\
438.01	0.01\\
439.01	0.01\\
440.01	0.01\\
441.01	0.01\\
442.01	0.01\\
443.01	0.01\\
444.01	0.01\\
445.01	0.01\\
446.01	0.01\\
447.01	0.01\\
448.01	0.01\\
449.01	0.01\\
450.01	0.01\\
451.01	0.01\\
452.01	0.01\\
453.01	0.01\\
454.01	0.01\\
455.01	0.01\\
456.01	0.01\\
457.01	0.01\\
458.01	0.01\\
459.01	0.01\\
460.01	0.01\\
461.01	0.01\\
462.01	0.01\\
463.01	0.01\\
464.01	0.01\\
465.01	0.01\\
466.01	0.01\\
467.01	0.01\\
468.01	0.01\\
469.01	0.01\\
470.01	0.01\\
471.01	0.01\\
472.01	0.01\\
473.01	0.01\\
474.01	0.01\\
475.01	0.01\\
476.01	0.01\\
477.01	0.01\\
478.01	0.01\\
479.01	0.01\\
480.01	0.01\\
481.01	0.01\\
482.01	0.01\\
483.01	0.01\\
484.01	0.01\\
485.01	0.01\\
486.01	0.01\\
487.01	0.01\\
488.01	0.01\\
489.01	0.01\\
490.01	0.01\\
491.01	0.01\\
492.01	0.01\\
493.01	0.01\\
494.01	0.01\\
495.01	0.01\\
496.01	0.01\\
497.01	0.01\\
498.01	0.01\\
499.01	0.01\\
500.01	0.01\\
501.01	0.01\\
502.01	0.01\\
503.01	0.01\\
504.01	0.01\\
505.01	0.01\\
506.01	0.01\\
507.01	0.01\\
508.01	0.01\\
509.01	0.01\\
510.01	0.01\\
511.01	0.01\\
512.01	0.01\\
513.01	0.01\\
514.01	0.01\\
515.01	0.01\\
516.01	0.01\\
517.01	0.01\\
518.01	0.01\\
519.01	0.01\\
520.01	0.01\\
521.01	0.01\\
522.01	0.01\\
523.01	0.01\\
524.01	0.01\\
525.01	0.01\\
526.01	0.01\\
527.01	0.01\\
528.01	0.01\\
529.01	0.01\\
530.01	0.01\\
531.01	0.01\\
532.01	0.01\\
533.01	0.01\\
534.01	0.01\\
535.01	0.01\\
536.01	0.01\\
537.01	0.01\\
538.01	0.01\\
539.01	0.01\\
540.01	0.01\\
541.01	0.01\\
542.01	0.01\\
543.01	0.01\\
544.01	0.01\\
545.01	0.01\\
546.01	0.01\\
547.01	0.01\\
548.01	0.01\\
549.01	0.01\\
550.01	0.01\\
551.01	0.01\\
552.01	0.01\\
553.01	0.01\\
554.01	0.01\\
555.01	0.01\\
556.01	0.01\\
557.01	0.01\\
558.01	0.01\\
559.01	0.01\\
560.01	0.01\\
561.01	0.01\\
562.01	0.01\\
563.01	0.01\\
564.01	0.01\\
565.01	0.01\\
566.01	0.01\\
567.01	0.01\\
568.01	0.01\\
569.01	0.01\\
570.01	0.01\\
571.01	0.01\\
572.01	0.01\\
573.01	0.01\\
574.01	0.01\\
575.01	0.01\\
576.01	0.01\\
577.01	0.01\\
578.01	0.01\\
579.01	0.01\\
580.01	0.01\\
581.01	0.01\\
582.01	0.01\\
583.01	0.01\\
584.01	0.01\\
585.01	0.01\\
586.01	0.01\\
587.01	0.01\\
588.01	0.01\\
589.01	0.01\\
590.01	0.01\\
591.01	0.01\\
592.01	0.01\\
593.01	0.01\\
594.01	0.01\\
595.01	0.01\\
596.01	0.01\\
597.01	0.01\\
598.01	0.00996305361369385\\
599.01	0.00620555099105032\\
599.02	0.00616729625629733\\
599.03	0.00612868222866261\\
599.04	0.00608970546580035\\
599.05	0.0060503624936144\\
599.06	0.006010649806003\\
599.07	0.00597056386460287\\
599.08	0.00593010109853261\\
599.09	0.00588925790413564\\
599.1	0.00584803064472266\\
599.11	0.00580641565031359\\
599.12	0.00576440921737937\\
599.13	0.00572200760858338\\
599.14	0.00567920705252272\\
599.15	0.00563600374346951\\
599.16	0.00559239384111208\\
599.17	0.00554837347029643\\
599.18	0.00550393872076778\\
599.19	0.00545908564691261\\
599.2	0.00541381026750106\\
599.21	0.00536810856542995\\
599.22	0.00532197648746648\\
599.23	0.00527540994399289\\
599.24	0.00522840480875195\\
599.25	0.00518095691859373\\
599.26	0.00513306207321158\\
599.27	0.00508471603487618\\
599.28	0.00503591452818626\\
599.29	0.00498665323982119\\
599.3	0.00493692781829558\\
599.31	0.00488673387371613\\
599.32	0.00483606697754091\\
599.33	0.00478492266234134\\
599.34	0.00473329642156706\\
599.35	0.00468118370931399\\
599.36	0.00462857994009582\\
599.37	0.00457548048861914\\
599.38	0.00452188068956266\\
599.39	0.00446777583736067\\
599.4	0.00441316118599112\\
599.41	0.00435803194876869\\
599.42	0.00430238329814314\\
599.43	0.0042462103655034\\
599.44	0.00418950824098764\\
599.45	0.00413227197329998\\
599.46	0.00407449657017897\\
599.47	0.00401617702613909\\
599.48	0.00395730829433481\\
599.49	0.00389788528641694\\
599.5	0.00383790287239801\\
599.51	0.00377735588052731\\
599.52	0.00371623909717613\\
599.53	0.00365454726673385\\
599.54	0.00359227509151553\\
599.55	0.00352941723168163\\
599.56	0.00346596830517072\\
599.57	0.00340192288764581\\
599.58	0.00333727551245523\\
599.59	0.00327202067060878\\
599.6	0.00320615281077012\\
599.61	0.0031396663392664\\
599.62	0.00307255562011594\\
599.63	0.00300481497507525\\
599.64	0.00293643868370625\\
599.65	0.002867420983465\\
599.66	0.00279775606981322\\
599.67	0.00272743809635361\\
599.68	0.00265646117499074\\
599.69	0.00258481937611856\\
599.7	0.00251250672883627\\
599.71	0.00243951722119419\\
599.72	0.00236584480047103\\
599.73	0.00229148337348472\\
599.74	0.00221642680693839\\
599.75	0.00214066892780361\\
599.76	0.00206420352374302\\
599.77	0.00198702434357445\\
599.78	0.00190912509777899\\
599.79	0.00183049945905542\\
599.8	0.00175114106292368\\
599.81	0.00167104350838001\\
599.82	0.00159020035860692\\
599.83	0.00150860514174086\\
599.84	0.00142625135170101\\
599.85	0.0013431324490826\\
599.86	0.00125924186211849\\
599.87	0.00117457298771287\\
599.88	0.00108911919255117\\
599.89	0.00100287381429064\\
599.9	0.000915830162836176\\
599.91	0.00082798152170631\\
599.92	0.000739321149494604\\
599.93	0.000649842281431971\\
599.94	0.000559538131055751\\
599.95	0.000468401891991852\\
599.96	0.000376426739856522\\
599.97	0.000283605834284767\\
599.98	0.000189932321092956\\
599.99	9.53993345834923e-05\\
600	0\\
};
\addplot [color=mycolor16,solid,forget plot]
  table[row sep=crcr]{%
0.01	0.01\\
1.01	0.01\\
2.01	0.01\\
3.01	0.01\\
4.01	0.01\\
5.01	0.01\\
6.01	0.01\\
7.01	0.01\\
8.01	0.01\\
9.01	0.01\\
10.01	0.01\\
11.01	0.01\\
12.01	0.01\\
13.01	0.01\\
14.01	0.01\\
15.01	0.01\\
16.01	0.01\\
17.01	0.01\\
18.01	0.01\\
19.01	0.01\\
20.01	0.01\\
21.01	0.01\\
22.01	0.01\\
23.01	0.01\\
24.01	0.01\\
25.01	0.01\\
26.01	0.01\\
27.01	0.01\\
28.01	0.01\\
29.01	0.01\\
30.01	0.01\\
31.01	0.01\\
32.01	0.01\\
33.01	0.01\\
34.01	0.01\\
35.01	0.01\\
36.01	0.01\\
37.01	0.01\\
38.01	0.01\\
39.01	0.01\\
40.01	0.01\\
41.01	0.01\\
42.01	0.01\\
43.01	0.01\\
44.01	0.01\\
45.01	0.01\\
46.01	0.01\\
47.01	0.01\\
48.01	0.01\\
49.01	0.01\\
50.01	0.01\\
51.01	0.01\\
52.01	0.01\\
53.01	0.01\\
54.01	0.01\\
55.01	0.01\\
56.01	0.01\\
57.01	0.01\\
58.01	0.01\\
59.01	0.01\\
60.01	0.01\\
61.01	0.01\\
62.01	0.01\\
63.01	0.01\\
64.01	0.01\\
65.01	0.01\\
66.01	0.01\\
67.01	0.01\\
68.01	0.01\\
69.01	0.01\\
70.01	0.01\\
71.01	0.01\\
72.01	0.01\\
73.01	0.01\\
74.01	0.01\\
75.01	0.01\\
76.01	0.01\\
77.01	0.01\\
78.01	0.01\\
79.01	0.01\\
80.01	0.01\\
81.01	0.01\\
82.01	0.01\\
83.01	0.01\\
84.01	0.01\\
85.01	0.01\\
86.01	0.01\\
87.01	0.01\\
88.01	0.01\\
89.01	0.01\\
90.01	0.01\\
91.01	0.01\\
92.01	0.01\\
93.01	0.01\\
94.01	0.01\\
95.01	0.01\\
96.01	0.01\\
97.01	0.01\\
98.01	0.01\\
99.01	0.01\\
100.01	0.01\\
101.01	0.01\\
102.01	0.01\\
103.01	0.01\\
104.01	0.01\\
105.01	0.01\\
106.01	0.01\\
107.01	0.01\\
108.01	0.01\\
109.01	0.01\\
110.01	0.01\\
111.01	0.01\\
112.01	0.01\\
113.01	0.01\\
114.01	0.01\\
115.01	0.01\\
116.01	0.01\\
117.01	0.01\\
118.01	0.01\\
119.01	0.01\\
120.01	0.01\\
121.01	0.01\\
122.01	0.01\\
123.01	0.01\\
124.01	0.01\\
125.01	0.01\\
126.01	0.01\\
127.01	0.01\\
128.01	0.01\\
129.01	0.01\\
130.01	0.01\\
131.01	0.01\\
132.01	0.01\\
133.01	0.01\\
134.01	0.01\\
135.01	0.01\\
136.01	0.01\\
137.01	0.01\\
138.01	0.01\\
139.01	0.01\\
140.01	0.01\\
141.01	0.01\\
142.01	0.01\\
143.01	0.01\\
144.01	0.01\\
145.01	0.01\\
146.01	0.01\\
147.01	0.01\\
148.01	0.01\\
149.01	0.01\\
150.01	0.01\\
151.01	0.01\\
152.01	0.01\\
153.01	0.01\\
154.01	0.01\\
155.01	0.01\\
156.01	0.01\\
157.01	0.01\\
158.01	0.01\\
159.01	0.01\\
160.01	0.01\\
161.01	0.01\\
162.01	0.01\\
163.01	0.01\\
164.01	0.01\\
165.01	0.01\\
166.01	0.01\\
167.01	0.01\\
168.01	0.01\\
169.01	0.01\\
170.01	0.01\\
171.01	0.01\\
172.01	0.01\\
173.01	0.01\\
174.01	0.01\\
175.01	0.01\\
176.01	0.01\\
177.01	0.01\\
178.01	0.01\\
179.01	0.01\\
180.01	0.01\\
181.01	0.01\\
182.01	0.01\\
183.01	0.01\\
184.01	0.01\\
185.01	0.01\\
186.01	0.01\\
187.01	0.01\\
188.01	0.01\\
189.01	0.01\\
190.01	0.01\\
191.01	0.01\\
192.01	0.01\\
193.01	0.01\\
194.01	0.01\\
195.01	0.01\\
196.01	0.01\\
197.01	0.01\\
198.01	0.01\\
199.01	0.01\\
200.01	0.01\\
201.01	0.01\\
202.01	0.01\\
203.01	0.01\\
204.01	0.01\\
205.01	0.01\\
206.01	0.01\\
207.01	0.01\\
208.01	0.01\\
209.01	0.01\\
210.01	0.01\\
211.01	0.01\\
212.01	0.01\\
213.01	0.01\\
214.01	0.01\\
215.01	0.01\\
216.01	0.01\\
217.01	0.01\\
218.01	0.01\\
219.01	0.01\\
220.01	0.01\\
221.01	0.01\\
222.01	0.01\\
223.01	0.01\\
224.01	0.01\\
225.01	0.01\\
226.01	0.01\\
227.01	0.01\\
228.01	0.01\\
229.01	0.01\\
230.01	0.01\\
231.01	0.01\\
232.01	0.01\\
233.01	0.01\\
234.01	0.01\\
235.01	0.01\\
236.01	0.01\\
237.01	0.01\\
238.01	0.01\\
239.01	0.01\\
240.01	0.01\\
241.01	0.01\\
242.01	0.01\\
243.01	0.01\\
244.01	0.01\\
245.01	0.01\\
246.01	0.01\\
247.01	0.01\\
248.01	0.01\\
249.01	0.01\\
250.01	0.01\\
251.01	0.01\\
252.01	0.01\\
253.01	0.01\\
254.01	0.01\\
255.01	0.01\\
256.01	0.01\\
257.01	0.01\\
258.01	0.01\\
259.01	0.01\\
260.01	0.01\\
261.01	0.01\\
262.01	0.01\\
263.01	0.01\\
264.01	0.01\\
265.01	0.01\\
266.01	0.01\\
267.01	0.01\\
268.01	0.01\\
269.01	0.01\\
270.01	0.01\\
271.01	0.01\\
272.01	0.01\\
273.01	0.01\\
274.01	0.01\\
275.01	0.01\\
276.01	0.01\\
277.01	0.01\\
278.01	0.01\\
279.01	0.01\\
280.01	0.01\\
281.01	0.01\\
282.01	0.01\\
283.01	0.01\\
284.01	0.01\\
285.01	0.01\\
286.01	0.01\\
287.01	0.01\\
288.01	0.01\\
289.01	0.01\\
290.01	0.01\\
291.01	0.01\\
292.01	0.01\\
293.01	0.01\\
294.01	0.01\\
295.01	0.01\\
296.01	0.01\\
297.01	0.01\\
298.01	0.01\\
299.01	0.01\\
300.01	0.01\\
301.01	0.01\\
302.01	0.01\\
303.01	0.01\\
304.01	0.01\\
305.01	0.01\\
306.01	0.01\\
307.01	0.01\\
308.01	0.01\\
309.01	0.01\\
310.01	0.01\\
311.01	0.01\\
312.01	0.01\\
313.01	0.01\\
314.01	0.01\\
315.01	0.01\\
316.01	0.01\\
317.01	0.01\\
318.01	0.01\\
319.01	0.01\\
320.01	0.01\\
321.01	0.01\\
322.01	0.01\\
323.01	0.01\\
324.01	0.01\\
325.01	0.01\\
326.01	0.01\\
327.01	0.01\\
328.01	0.01\\
329.01	0.01\\
330.01	0.01\\
331.01	0.01\\
332.01	0.01\\
333.01	0.01\\
334.01	0.01\\
335.01	0.01\\
336.01	0.01\\
337.01	0.01\\
338.01	0.01\\
339.01	0.01\\
340.01	0.01\\
341.01	0.01\\
342.01	0.01\\
343.01	0.01\\
344.01	0.01\\
345.01	0.01\\
346.01	0.01\\
347.01	0.01\\
348.01	0.01\\
349.01	0.01\\
350.01	0.01\\
351.01	0.01\\
352.01	0.01\\
353.01	0.01\\
354.01	0.01\\
355.01	0.01\\
356.01	0.01\\
357.01	0.01\\
358.01	0.01\\
359.01	0.01\\
360.01	0.01\\
361.01	0.01\\
362.01	0.01\\
363.01	0.01\\
364.01	0.01\\
365.01	0.01\\
366.01	0.01\\
367.01	0.01\\
368.01	0.01\\
369.01	0.01\\
370.01	0.01\\
371.01	0.01\\
372.01	0.01\\
373.01	0.01\\
374.01	0.01\\
375.01	0.01\\
376.01	0.01\\
377.01	0.01\\
378.01	0.01\\
379.01	0.01\\
380.01	0.01\\
381.01	0.01\\
382.01	0.01\\
383.01	0.01\\
384.01	0.01\\
385.01	0.01\\
386.01	0.01\\
387.01	0.01\\
388.01	0.01\\
389.01	0.01\\
390.01	0.01\\
391.01	0.01\\
392.01	0.01\\
393.01	0.01\\
394.01	0.01\\
395.01	0.01\\
396.01	0.01\\
397.01	0.01\\
398.01	0.01\\
399.01	0.01\\
400.01	0.01\\
401.01	0.01\\
402.01	0.01\\
403.01	0.01\\
404.01	0.01\\
405.01	0.01\\
406.01	0.01\\
407.01	0.01\\
408.01	0.01\\
409.01	0.01\\
410.01	0.01\\
411.01	0.01\\
412.01	0.01\\
413.01	0.01\\
414.01	0.01\\
415.01	0.01\\
416.01	0.01\\
417.01	0.01\\
418.01	0.01\\
419.01	0.01\\
420.01	0.01\\
421.01	0.01\\
422.01	0.01\\
423.01	0.01\\
424.01	0.01\\
425.01	0.01\\
426.01	0.01\\
427.01	0.01\\
428.01	0.01\\
429.01	0.01\\
430.01	0.01\\
431.01	0.01\\
432.01	0.01\\
433.01	0.01\\
434.01	0.01\\
435.01	0.01\\
436.01	0.01\\
437.01	0.01\\
438.01	0.01\\
439.01	0.01\\
440.01	0.01\\
441.01	0.01\\
442.01	0.01\\
443.01	0.01\\
444.01	0.01\\
445.01	0.01\\
446.01	0.01\\
447.01	0.01\\
448.01	0.01\\
449.01	0.01\\
450.01	0.01\\
451.01	0.01\\
452.01	0.01\\
453.01	0.01\\
454.01	0.01\\
455.01	0.01\\
456.01	0.01\\
457.01	0.01\\
458.01	0.01\\
459.01	0.01\\
460.01	0.01\\
461.01	0.01\\
462.01	0.01\\
463.01	0.01\\
464.01	0.01\\
465.01	0.01\\
466.01	0.01\\
467.01	0.01\\
468.01	0.01\\
469.01	0.01\\
470.01	0.01\\
471.01	0.01\\
472.01	0.01\\
473.01	0.01\\
474.01	0.01\\
475.01	0.01\\
476.01	0.01\\
477.01	0.01\\
478.01	0.01\\
479.01	0.01\\
480.01	0.01\\
481.01	0.01\\
482.01	0.01\\
483.01	0.01\\
484.01	0.01\\
485.01	0.01\\
486.01	0.01\\
487.01	0.01\\
488.01	0.01\\
489.01	0.01\\
490.01	0.01\\
491.01	0.01\\
492.01	0.01\\
493.01	0.01\\
494.01	0.01\\
495.01	0.01\\
496.01	0.01\\
497.01	0.01\\
498.01	0.01\\
499.01	0.01\\
500.01	0.01\\
501.01	0.01\\
502.01	0.01\\
503.01	0.01\\
504.01	0.01\\
505.01	0.01\\
506.01	0.01\\
507.01	0.01\\
508.01	0.01\\
509.01	0.01\\
510.01	0.01\\
511.01	0.01\\
512.01	0.01\\
513.01	0.01\\
514.01	0.01\\
515.01	0.01\\
516.01	0.01\\
517.01	0.01\\
518.01	0.01\\
519.01	0.01\\
520.01	0.01\\
521.01	0.01\\
522.01	0.01\\
523.01	0.01\\
524.01	0.01\\
525.01	0.01\\
526.01	0.01\\
527.01	0.01\\
528.01	0.01\\
529.01	0.01\\
530.01	0.01\\
531.01	0.01\\
532.01	0.01\\
533.01	0.01\\
534.01	0.01\\
535.01	0.01\\
536.01	0.01\\
537.01	0.01\\
538.01	0.01\\
539.01	0.01\\
540.01	0.01\\
541.01	0.01\\
542.01	0.01\\
543.01	0.01\\
544.01	0.01\\
545.01	0.01\\
546.01	0.01\\
547.01	0.01\\
548.01	0.01\\
549.01	0.01\\
550.01	0.01\\
551.01	0.01\\
552.01	0.01\\
553.01	0.01\\
554.01	0.01\\
555.01	0.01\\
556.01	0.01\\
557.01	0.01\\
558.01	0.01\\
559.01	0.01\\
560.01	0.01\\
561.01	0.01\\
562.01	0.01\\
563.01	0.01\\
564.01	0.01\\
565.01	0.01\\
566.01	0.01\\
567.01	0.01\\
568.01	0.01\\
569.01	0.01\\
570.01	0.01\\
571.01	0.01\\
572.01	0.01\\
573.01	0.01\\
574.01	0.01\\
575.01	0.01\\
576.01	0.01\\
577.01	0.01\\
578.01	0.01\\
579.01	0.01\\
580.01	0.01\\
581.01	0.01\\
582.01	0.01\\
583.01	0.01\\
584.01	0.01\\
585.01	0.01\\
586.01	0.01\\
587.01	0.01\\
588.01	0.01\\
589.01	0.01\\
590.01	0.01\\
591.01	0.01\\
592.01	0.01\\
593.01	0.01\\
594.01	0.01\\
595.01	0.01\\
596.01	0.01\\
597.01	0.01\\
598.01	0.00858631439240976\\
599.01	0.00614539388216774\\
599.02	0.00610782331917581\\
599.03	0.00606989521491409\\
599.04	0.00603160611881512\\
599.05	0.00599295254708737\\
599.06	0.00595393098239515\\
599.07	0.00591453787353531\\
599.08	0.00587476963511083\\
599.09	0.00583462264720117\\
599.1	0.00579409325502938\\
599.11	0.00575317776862597\\
599.12	0.0057118724624894\\
599.13	0.00567017357524326\\
599.14	0.0056280773092901\\
599.15	0.00558557983046171\\
599.16	0.00554267726766606\\
599.17	0.00549936571253071\\
599.18	0.00545564121904254\\
599.19	0.00541149980318408\\
599.2	0.00536693744256611\\
599.21	0.00532195007605655\\
599.22	0.0052765336034057\\
599.23	0.00523068388486769\\
599.24	0.00518439674081807\\
599.25	0.00513766795136762\\
599.26	0.00509049326390392\\
599.27	0.00504286839503934\\
599.28	0.0049947890203083\\
599.29	0.0049462507737713\\
599.3	0.00489724924761472\\
599.31	0.00484777999174682\\
599.32	0.00479783851338942\\
599.33	0.00474742027666554\\
599.34	0.00469652070218275\\
599.35	0.00464513516661223\\
599.36	0.00459325900226353\\
599.37	0.00454088749665488\\
599.38	0.00448801589207906\\
599.39	0.00443463938516468\\
599.4	0.00438075312643297\\
599.41	0.00432635221984972\\
599.42	0.00427143172237266\\
599.43	0.00421598664349386\\
599.44	0.00416001194477734\\
599.45	0.00410350253939166\\
599.46	0.00404645329163675\\
599.47	0.00398885901643419\\
599.48	0.00393071447884265\\
599.49	0.00387201439356773\\
599.5	0.00381275342446643\\
599.51	0.00375292618404589\\
599.52	0.00369252723295655\\
599.53	0.00363155107947932\\
599.54	0.00356999217900692\\
599.55	0.00350784493351912\\
599.56	0.00344510369105174\\
599.57	0.00338176274515938\\
599.58	0.00331781633437164\\
599.59	0.00325325864164278\\
599.6	0.00318808379379451\\
599.61	0.003122285860952\\
599.62	0.00305585885597271\\
599.63	0.00298879673386794\\
599.64	0.00292109339121705\\
599.65	0.00285274266557392\\
599.66	0.00278373833486563\\
599.67	0.0027140741167831\\
599.68	0.00264374366816346\\
599.69	0.00257274058436389\\
599.7	0.00250105839862672\\
599.71	0.00242869058143554\\
599.72	0.00235563053986202\\
599.73	0.00228187161690313\\
599.74	0.00220740709080853\\
599.75	0.00213223017439773\\
599.76	0.0020563340143667\\
599.77	0.00197971169058367\\
599.78	0.00190235621537348\\
599.79	0.00182426053279044\\
599.8	0.00174541751787897\\
599.81	0.00166581997592167\\
599.82	0.00158546064167443\\
599.83	0.00150433217858791\\
599.84	0.00142242717801498\\
599.85	0.00133973815840344\\
599.86	0.00125625756447358\\
599.87	0.00117197776637964\\
599.88	0.00108689105885485\\
599.89	0.00100098966033894\\
599.9	0.000914265712087704\\
599.91	0.000826711277263432\\
599.92	0.000738318340005614\\
599.93	0.00064907880448079\\
599.94	0.000558984493910604\\
599.95	0.000468027149577001\\
599.96	0.000376198429803352\\
599.97	0.000283489908910344\\
599.98	0.000189893076145302\\
599.99	9.53993345834923e-05\\
600	0\\
};
\addplot [color=mycolor17,solid,forget plot]
  table[row sep=crcr]{%
0.01	0.01\\
1.01	0.01\\
2.01	0.01\\
3.01	0.01\\
4.01	0.01\\
5.01	0.01\\
6.01	0.01\\
7.01	0.01\\
8.01	0.01\\
9.01	0.01\\
10.01	0.01\\
11.01	0.01\\
12.01	0.01\\
13.01	0.01\\
14.01	0.01\\
15.01	0.01\\
16.01	0.01\\
17.01	0.01\\
18.01	0.01\\
19.01	0.01\\
20.01	0.01\\
21.01	0.01\\
22.01	0.01\\
23.01	0.01\\
24.01	0.01\\
25.01	0.01\\
26.01	0.01\\
27.01	0.01\\
28.01	0.01\\
29.01	0.01\\
30.01	0.01\\
31.01	0.01\\
32.01	0.01\\
33.01	0.01\\
34.01	0.01\\
35.01	0.01\\
36.01	0.01\\
37.01	0.01\\
38.01	0.01\\
39.01	0.01\\
40.01	0.01\\
41.01	0.01\\
42.01	0.01\\
43.01	0.01\\
44.01	0.01\\
45.01	0.01\\
46.01	0.01\\
47.01	0.01\\
48.01	0.01\\
49.01	0.01\\
50.01	0.01\\
51.01	0.01\\
52.01	0.01\\
53.01	0.01\\
54.01	0.01\\
55.01	0.01\\
56.01	0.01\\
57.01	0.01\\
58.01	0.01\\
59.01	0.01\\
60.01	0.01\\
61.01	0.01\\
62.01	0.01\\
63.01	0.01\\
64.01	0.01\\
65.01	0.01\\
66.01	0.01\\
67.01	0.01\\
68.01	0.01\\
69.01	0.01\\
70.01	0.01\\
71.01	0.01\\
72.01	0.01\\
73.01	0.01\\
74.01	0.01\\
75.01	0.01\\
76.01	0.01\\
77.01	0.01\\
78.01	0.01\\
79.01	0.01\\
80.01	0.01\\
81.01	0.01\\
82.01	0.01\\
83.01	0.01\\
84.01	0.01\\
85.01	0.01\\
86.01	0.01\\
87.01	0.01\\
88.01	0.01\\
89.01	0.01\\
90.01	0.01\\
91.01	0.01\\
92.01	0.01\\
93.01	0.01\\
94.01	0.01\\
95.01	0.01\\
96.01	0.01\\
97.01	0.01\\
98.01	0.01\\
99.01	0.01\\
100.01	0.01\\
101.01	0.01\\
102.01	0.01\\
103.01	0.01\\
104.01	0.01\\
105.01	0.01\\
106.01	0.01\\
107.01	0.01\\
108.01	0.01\\
109.01	0.01\\
110.01	0.01\\
111.01	0.01\\
112.01	0.01\\
113.01	0.01\\
114.01	0.01\\
115.01	0.01\\
116.01	0.01\\
117.01	0.01\\
118.01	0.01\\
119.01	0.01\\
120.01	0.01\\
121.01	0.01\\
122.01	0.01\\
123.01	0.01\\
124.01	0.01\\
125.01	0.01\\
126.01	0.01\\
127.01	0.01\\
128.01	0.01\\
129.01	0.01\\
130.01	0.01\\
131.01	0.01\\
132.01	0.01\\
133.01	0.01\\
134.01	0.01\\
135.01	0.01\\
136.01	0.01\\
137.01	0.01\\
138.01	0.01\\
139.01	0.01\\
140.01	0.01\\
141.01	0.01\\
142.01	0.01\\
143.01	0.01\\
144.01	0.01\\
145.01	0.01\\
146.01	0.01\\
147.01	0.01\\
148.01	0.01\\
149.01	0.01\\
150.01	0.01\\
151.01	0.01\\
152.01	0.01\\
153.01	0.01\\
154.01	0.01\\
155.01	0.01\\
156.01	0.01\\
157.01	0.01\\
158.01	0.01\\
159.01	0.01\\
160.01	0.01\\
161.01	0.01\\
162.01	0.01\\
163.01	0.01\\
164.01	0.01\\
165.01	0.01\\
166.01	0.01\\
167.01	0.01\\
168.01	0.01\\
169.01	0.01\\
170.01	0.01\\
171.01	0.01\\
172.01	0.01\\
173.01	0.01\\
174.01	0.01\\
175.01	0.01\\
176.01	0.01\\
177.01	0.01\\
178.01	0.01\\
179.01	0.01\\
180.01	0.01\\
181.01	0.01\\
182.01	0.01\\
183.01	0.01\\
184.01	0.01\\
185.01	0.01\\
186.01	0.01\\
187.01	0.01\\
188.01	0.01\\
189.01	0.01\\
190.01	0.01\\
191.01	0.01\\
192.01	0.01\\
193.01	0.01\\
194.01	0.01\\
195.01	0.01\\
196.01	0.01\\
197.01	0.01\\
198.01	0.01\\
199.01	0.01\\
200.01	0.01\\
201.01	0.01\\
202.01	0.01\\
203.01	0.01\\
204.01	0.01\\
205.01	0.01\\
206.01	0.01\\
207.01	0.01\\
208.01	0.01\\
209.01	0.01\\
210.01	0.01\\
211.01	0.01\\
212.01	0.01\\
213.01	0.01\\
214.01	0.01\\
215.01	0.01\\
216.01	0.01\\
217.01	0.01\\
218.01	0.01\\
219.01	0.01\\
220.01	0.01\\
221.01	0.01\\
222.01	0.01\\
223.01	0.01\\
224.01	0.01\\
225.01	0.01\\
226.01	0.01\\
227.01	0.01\\
228.01	0.01\\
229.01	0.01\\
230.01	0.01\\
231.01	0.01\\
232.01	0.01\\
233.01	0.01\\
234.01	0.01\\
235.01	0.01\\
236.01	0.01\\
237.01	0.01\\
238.01	0.01\\
239.01	0.01\\
240.01	0.01\\
241.01	0.01\\
242.01	0.01\\
243.01	0.01\\
244.01	0.01\\
245.01	0.01\\
246.01	0.01\\
247.01	0.01\\
248.01	0.01\\
249.01	0.01\\
250.01	0.01\\
251.01	0.01\\
252.01	0.01\\
253.01	0.01\\
254.01	0.01\\
255.01	0.01\\
256.01	0.01\\
257.01	0.01\\
258.01	0.01\\
259.01	0.01\\
260.01	0.01\\
261.01	0.01\\
262.01	0.01\\
263.01	0.01\\
264.01	0.01\\
265.01	0.01\\
266.01	0.01\\
267.01	0.01\\
268.01	0.01\\
269.01	0.01\\
270.01	0.01\\
271.01	0.01\\
272.01	0.01\\
273.01	0.01\\
274.01	0.01\\
275.01	0.01\\
276.01	0.01\\
277.01	0.01\\
278.01	0.01\\
279.01	0.01\\
280.01	0.01\\
281.01	0.01\\
282.01	0.01\\
283.01	0.01\\
284.01	0.01\\
285.01	0.01\\
286.01	0.01\\
287.01	0.01\\
288.01	0.01\\
289.01	0.01\\
290.01	0.01\\
291.01	0.01\\
292.01	0.01\\
293.01	0.01\\
294.01	0.01\\
295.01	0.01\\
296.01	0.01\\
297.01	0.01\\
298.01	0.01\\
299.01	0.01\\
300.01	0.01\\
301.01	0.01\\
302.01	0.01\\
303.01	0.01\\
304.01	0.01\\
305.01	0.01\\
306.01	0.01\\
307.01	0.01\\
308.01	0.01\\
309.01	0.01\\
310.01	0.01\\
311.01	0.01\\
312.01	0.01\\
313.01	0.01\\
314.01	0.01\\
315.01	0.01\\
316.01	0.01\\
317.01	0.01\\
318.01	0.01\\
319.01	0.01\\
320.01	0.01\\
321.01	0.01\\
322.01	0.01\\
323.01	0.01\\
324.01	0.01\\
325.01	0.01\\
326.01	0.01\\
327.01	0.01\\
328.01	0.01\\
329.01	0.01\\
330.01	0.01\\
331.01	0.01\\
332.01	0.01\\
333.01	0.01\\
334.01	0.01\\
335.01	0.01\\
336.01	0.01\\
337.01	0.01\\
338.01	0.01\\
339.01	0.01\\
340.01	0.01\\
341.01	0.01\\
342.01	0.01\\
343.01	0.01\\
344.01	0.01\\
345.01	0.01\\
346.01	0.01\\
347.01	0.01\\
348.01	0.01\\
349.01	0.01\\
350.01	0.01\\
351.01	0.01\\
352.01	0.01\\
353.01	0.01\\
354.01	0.01\\
355.01	0.01\\
356.01	0.01\\
357.01	0.01\\
358.01	0.01\\
359.01	0.01\\
360.01	0.01\\
361.01	0.01\\
362.01	0.01\\
363.01	0.01\\
364.01	0.01\\
365.01	0.01\\
366.01	0.01\\
367.01	0.01\\
368.01	0.01\\
369.01	0.01\\
370.01	0.01\\
371.01	0.01\\
372.01	0.01\\
373.01	0.01\\
374.01	0.01\\
375.01	0.01\\
376.01	0.01\\
377.01	0.01\\
378.01	0.01\\
379.01	0.01\\
380.01	0.01\\
381.01	0.01\\
382.01	0.01\\
383.01	0.01\\
384.01	0.01\\
385.01	0.01\\
386.01	0.01\\
387.01	0.01\\
388.01	0.01\\
389.01	0.01\\
390.01	0.01\\
391.01	0.01\\
392.01	0.01\\
393.01	0.01\\
394.01	0.01\\
395.01	0.01\\
396.01	0.01\\
397.01	0.01\\
398.01	0.01\\
399.01	0.01\\
400.01	0.01\\
401.01	0.01\\
402.01	0.01\\
403.01	0.01\\
404.01	0.01\\
405.01	0.01\\
406.01	0.01\\
407.01	0.01\\
408.01	0.01\\
409.01	0.01\\
410.01	0.01\\
411.01	0.01\\
412.01	0.01\\
413.01	0.01\\
414.01	0.01\\
415.01	0.01\\
416.01	0.01\\
417.01	0.01\\
418.01	0.01\\
419.01	0.01\\
420.01	0.01\\
421.01	0.01\\
422.01	0.01\\
423.01	0.01\\
424.01	0.01\\
425.01	0.01\\
426.01	0.01\\
427.01	0.01\\
428.01	0.01\\
429.01	0.01\\
430.01	0.01\\
431.01	0.01\\
432.01	0.01\\
433.01	0.01\\
434.01	0.01\\
435.01	0.01\\
436.01	0.01\\
437.01	0.01\\
438.01	0.01\\
439.01	0.01\\
440.01	0.01\\
441.01	0.01\\
442.01	0.01\\
443.01	0.01\\
444.01	0.01\\
445.01	0.01\\
446.01	0.01\\
447.01	0.01\\
448.01	0.01\\
449.01	0.01\\
450.01	0.01\\
451.01	0.01\\
452.01	0.01\\
453.01	0.01\\
454.01	0.01\\
455.01	0.01\\
456.01	0.01\\
457.01	0.01\\
458.01	0.01\\
459.01	0.01\\
460.01	0.01\\
461.01	0.01\\
462.01	0.01\\
463.01	0.01\\
464.01	0.01\\
465.01	0.01\\
466.01	0.01\\
467.01	0.01\\
468.01	0.01\\
469.01	0.01\\
470.01	0.01\\
471.01	0.01\\
472.01	0.01\\
473.01	0.01\\
474.01	0.01\\
475.01	0.01\\
476.01	0.01\\
477.01	0.01\\
478.01	0.01\\
479.01	0.01\\
480.01	0.01\\
481.01	0.01\\
482.01	0.01\\
483.01	0.01\\
484.01	0.01\\
485.01	0.01\\
486.01	0.01\\
487.01	0.01\\
488.01	0.01\\
489.01	0.01\\
490.01	0.01\\
491.01	0.01\\
492.01	0.01\\
493.01	0.01\\
494.01	0.01\\
495.01	0.01\\
496.01	0.01\\
497.01	0.01\\
498.01	0.01\\
499.01	0.01\\
500.01	0.01\\
501.01	0.01\\
502.01	0.01\\
503.01	0.01\\
504.01	0.01\\
505.01	0.01\\
506.01	0.01\\
507.01	0.01\\
508.01	0.01\\
509.01	0.01\\
510.01	0.01\\
511.01	0.01\\
512.01	0.01\\
513.01	0.01\\
514.01	0.01\\
515.01	0.01\\
516.01	0.01\\
517.01	0.01\\
518.01	0.01\\
519.01	0.01\\
520.01	0.01\\
521.01	0.01\\
522.01	0.01\\
523.01	0.01\\
524.01	0.01\\
525.01	0.01\\
526.01	0.01\\
527.01	0.01\\
528.01	0.01\\
529.01	0.01\\
530.01	0.01\\
531.01	0.01\\
532.01	0.01\\
533.01	0.01\\
534.01	0.01\\
535.01	0.01\\
536.01	0.01\\
537.01	0.01\\
538.01	0.01\\
539.01	0.01\\
540.01	0.01\\
541.01	0.01\\
542.01	0.01\\
543.01	0.01\\
544.01	0.01\\
545.01	0.01\\
546.01	0.01\\
547.01	0.01\\
548.01	0.01\\
549.01	0.01\\
550.01	0.01\\
551.01	0.01\\
552.01	0.01\\
553.01	0.01\\
554.01	0.01\\
555.01	0.01\\
556.01	0.01\\
557.01	0.01\\
558.01	0.01\\
559.01	0.01\\
560.01	0.01\\
561.01	0.01\\
562.01	0.01\\
563.01	0.01\\
564.01	0.01\\
565.01	0.01\\
566.01	0.01\\
567.01	0.01\\
568.01	0.01\\
569.01	0.01\\
570.01	0.01\\
571.01	0.01\\
572.01	0.01\\
573.01	0.01\\
574.01	0.01\\
575.01	0.01\\
576.01	0.01\\
577.01	0.01\\
578.01	0.01\\
579.01	0.01\\
580.01	0.01\\
581.01	0.01\\
582.01	0.01\\
583.01	0.01\\
584.01	0.01\\
585.01	0.01\\
586.01	0.01\\
587.01	0.01\\
588.01	0.01\\
589.01	0.01\\
590.01	0.01\\
591.01	0.01\\
592.01	0.01\\
593.01	0.01\\
594.01	0.01\\
595.01	0.01\\
596.01	0.01\\
597.01	0.01\\
598.01	0.0085625962751722\\
599.01	0.00614343963478779\\
599.02	0.00610590275444205\\
599.03	0.00606800823173612\\
599.04	0.00602975261272226\\
599.05	0.00599113241011609\\
599.06	0.00595214410297507\\
599.07	0.0059127841363737\\
599.08	0.00587304892107572\\
599.09	0.00583293483320304\\
599.1	0.00579243821390159\\
599.11	0.00575155536900382\\
599.12	0.00571028256868804\\
599.13	0.00566861604713446\\
599.14	0.00562655200217782\\
599.15	0.00558408659495681\\
599.16	0.00554121594956\\
599.17	0.00549793615266839\\
599.18	0.0054542432531945\\
599.19	0.00541013326191803\\
599.2	0.00536560215111789\\
599.21	0.0053206458542008\\
599.22	0.00527526026532628\\
599.23	0.00522944123902788\\
599.24	0.00518318458983103\\
599.25	0.0051364860918669\\
599.26	0.00508934148916821\\
599.27	0.00504174649188929\\
599.28	0.00499369676895965\\
599.29	0.00494518794768798\\
599.3	0.00489621561336226\\
599.31	0.00484677530884621\\
599.32	0.0047968625341717\\
599.33	0.00474647274612728\\
599.34	0.00469560135784286\\
599.35	0.00464424373837022\\
599.36	0.00459239521225965\\
599.37	0.00454005105913244\\
599.38	0.00448720651324929\\
599.39	0.00443385676307457\\
599.4	0.00437999695083635\\
599.41	0.00432562217208232\\
599.42	0.00427072747523129\\
599.43	0.00421530786112051\\
599.44	0.00415935828254859\\
599.45	0.00410287364381403\\
599.46	0.00404584880024931\\
599.47	0.00398827855775059\\
599.48	0.00393015767230284\\
599.49	0.00387148084950048\\
599.5	0.00381224274406338\\
599.51	0.00375243795934817\\
599.52	0.00369206104685504\\
599.53	0.00363110650572963\\
599.54	0.00356956878226021\\
599.55	0.00350744226937014\\
599.56	0.00344472130610529\\
599.57	0.00338140017711678\\
599.58	0.00331747311213853\\
599.59	0.00325293428546005\\
599.6	0.00318777781539403\\
599.61	0.00312199776373892\\
599.62	0.00305558813523637\\
599.63	0.00298854287702349\\
599.64	0.00292085587807993\\
599.65	0.00285252096866961\\
599.66	0.00278353191977726\\
599.67	0.00271388244253948\\
599.68	0.00264356618767053\\
599.69	0.00257257674488254\\
599.7	0.00250090764230037\\
599.71	0.00242855234587079\\
599.72	0.00235550425876621\\
599.73	0.00228175672078272\\
599.74	0.00220730300773244\\
599.75	0.00213213633083023\\
599.76	0.00205624983607459\\
599.77	0.00197963660362281\\
599.78	0.00190228964716019\\
599.79	0.00182420191326353\\
599.8	0.00174536628075859\\
599.81	0.00166577556007161\\
599.82	0.00158542249257488\\
599.83	0.00150429974992627\\
599.84	0.00142239993340269\\
599.85	0.00133971557322744\\
599.86	0.0012562391278915\\
599.87	0.00117196298346859\\
599.88	0.00108687945292415\\
599.89	0.00100098077541806\\
599.9	0.000914259115601139\\
599.91	0.000826706562905496\\
599.92	0.000738315130828553\\
599.93	0.000649076756210875\\
599.94	0.000558983298507749\\
599.95	0.000468026539054494\\
599.96	0.00037619818032557\\
599.97	0.000283489845187453\\
599.98	0.0001898930761453\\
599.99	9.53993345834923e-05\\
600	0\\
};
\addplot [color=mycolor18,solid,forget plot]
  table[row sep=crcr]{%
0.01	0.01\\
1.01	0.01\\
2.01	0.01\\
3.01	0.01\\
4.01	0.01\\
5.01	0.01\\
6.01	0.01\\
7.01	0.01\\
8.01	0.01\\
9.01	0.01\\
10.01	0.01\\
11.01	0.01\\
12.01	0.01\\
13.01	0.01\\
14.01	0.01\\
15.01	0.01\\
16.01	0.01\\
17.01	0.01\\
18.01	0.01\\
19.01	0.01\\
20.01	0.01\\
21.01	0.01\\
22.01	0.01\\
23.01	0.01\\
24.01	0.01\\
25.01	0.01\\
26.01	0.01\\
27.01	0.01\\
28.01	0.01\\
29.01	0.01\\
30.01	0.01\\
31.01	0.01\\
32.01	0.01\\
33.01	0.01\\
34.01	0.01\\
35.01	0.01\\
36.01	0.01\\
37.01	0.01\\
38.01	0.01\\
39.01	0.01\\
40.01	0.01\\
41.01	0.01\\
42.01	0.01\\
43.01	0.01\\
44.01	0.01\\
45.01	0.01\\
46.01	0.01\\
47.01	0.01\\
48.01	0.01\\
49.01	0.01\\
50.01	0.01\\
51.01	0.01\\
52.01	0.01\\
53.01	0.01\\
54.01	0.01\\
55.01	0.01\\
56.01	0.01\\
57.01	0.01\\
58.01	0.01\\
59.01	0.01\\
60.01	0.01\\
61.01	0.01\\
62.01	0.01\\
63.01	0.01\\
64.01	0.01\\
65.01	0.01\\
66.01	0.01\\
67.01	0.01\\
68.01	0.01\\
69.01	0.01\\
70.01	0.01\\
71.01	0.01\\
72.01	0.01\\
73.01	0.01\\
74.01	0.01\\
75.01	0.01\\
76.01	0.01\\
77.01	0.01\\
78.01	0.01\\
79.01	0.01\\
80.01	0.01\\
81.01	0.01\\
82.01	0.01\\
83.01	0.01\\
84.01	0.01\\
85.01	0.01\\
86.01	0.01\\
87.01	0.01\\
88.01	0.01\\
89.01	0.01\\
90.01	0.01\\
91.01	0.01\\
92.01	0.01\\
93.01	0.01\\
94.01	0.01\\
95.01	0.01\\
96.01	0.01\\
97.01	0.01\\
98.01	0.01\\
99.01	0.01\\
100.01	0.01\\
101.01	0.01\\
102.01	0.01\\
103.01	0.01\\
104.01	0.01\\
105.01	0.01\\
106.01	0.01\\
107.01	0.01\\
108.01	0.01\\
109.01	0.01\\
110.01	0.01\\
111.01	0.01\\
112.01	0.01\\
113.01	0.01\\
114.01	0.01\\
115.01	0.01\\
116.01	0.01\\
117.01	0.01\\
118.01	0.01\\
119.01	0.01\\
120.01	0.01\\
121.01	0.01\\
122.01	0.01\\
123.01	0.01\\
124.01	0.01\\
125.01	0.01\\
126.01	0.01\\
127.01	0.01\\
128.01	0.01\\
129.01	0.01\\
130.01	0.01\\
131.01	0.01\\
132.01	0.01\\
133.01	0.01\\
134.01	0.01\\
135.01	0.01\\
136.01	0.01\\
137.01	0.01\\
138.01	0.01\\
139.01	0.01\\
140.01	0.01\\
141.01	0.01\\
142.01	0.01\\
143.01	0.01\\
144.01	0.01\\
145.01	0.01\\
146.01	0.01\\
147.01	0.01\\
148.01	0.01\\
149.01	0.01\\
150.01	0.01\\
151.01	0.01\\
152.01	0.01\\
153.01	0.01\\
154.01	0.01\\
155.01	0.01\\
156.01	0.01\\
157.01	0.01\\
158.01	0.01\\
159.01	0.01\\
160.01	0.01\\
161.01	0.01\\
162.01	0.01\\
163.01	0.01\\
164.01	0.01\\
165.01	0.01\\
166.01	0.01\\
167.01	0.01\\
168.01	0.01\\
169.01	0.01\\
170.01	0.01\\
171.01	0.01\\
172.01	0.01\\
173.01	0.01\\
174.01	0.01\\
175.01	0.01\\
176.01	0.01\\
177.01	0.01\\
178.01	0.01\\
179.01	0.01\\
180.01	0.01\\
181.01	0.01\\
182.01	0.01\\
183.01	0.01\\
184.01	0.01\\
185.01	0.01\\
186.01	0.01\\
187.01	0.01\\
188.01	0.01\\
189.01	0.01\\
190.01	0.01\\
191.01	0.01\\
192.01	0.01\\
193.01	0.01\\
194.01	0.01\\
195.01	0.01\\
196.01	0.01\\
197.01	0.01\\
198.01	0.01\\
199.01	0.01\\
200.01	0.01\\
201.01	0.01\\
202.01	0.01\\
203.01	0.01\\
204.01	0.01\\
205.01	0.01\\
206.01	0.01\\
207.01	0.01\\
208.01	0.01\\
209.01	0.01\\
210.01	0.01\\
211.01	0.01\\
212.01	0.01\\
213.01	0.01\\
214.01	0.01\\
215.01	0.01\\
216.01	0.01\\
217.01	0.01\\
218.01	0.01\\
219.01	0.01\\
220.01	0.01\\
221.01	0.01\\
222.01	0.01\\
223.01	0.01\\
224.01	0.01\\
225.01	0.01\\
226.01	0.01\\
227.01	0.01\\
228.01	0.01\\
229.01	0.01\\
230.01	0.01\\
231.01	0.01\\
232.01	0.01\\
233.01	0.01\\
234.01	0.01\\
235.01	0.01\\
236.01	0.01\\
237.01	0.01\\
238.01	0.01\\
239.01	0.01\\
240.01	0.01\\
241.01	0.01\\
242.01	0.01\\
243.01	0.01\\
244.01	0.01\\
245.01	0.01\\
246.01	0.01\\
247.01	0.01\\
248.01	0.01\\
249.01	0.01\\
250.01	0.01\\
251.01	0.01\\
252.01	0.01\\
253.01	0.01\\
254.01	0.01\\
255.01	0.01\\
256.01	0.01\\
257.01	0.01\\
258.01	0.01\\
259.01	0.01\\
260.01	0.01\\
261.01	0.01\\
262.01	0.01\\
263.01	0.01\\
264.01	0.01\\
265.01	0.01\\
266.01	0.01\\
267.01	0.01\\
268.01	0.01\\
269.01	0.01\\
270.01	0.01\\
271.01	0.01\\
272.01	0.01\\
273.01	0.01\\
274.01	0.01\\
275.01	0.01\\
276.01	0.01\\
277.01	0.01\\
278.01	0.01\\
279.01	0.01\\
280.01	0.01\\
281.01	0.01\\
282.01	0.01\\
283.01	0.01\\
284.01	0.01\\
285.01	0.01\\
286.01	0.01\\
287.01	0.01\\
288.01	0.01\\
289.01	0.01\\
290.01	0.01\\
291.01	0.01\\
292.01	0.01\\
293.01	0.01\\
294.01	0.01\\
295.01	0.01\\
296.01	0.01\\
297.01	0.01\\
298.01	0.01\\
299.01	0.01\\
300.01	0.01\\
301.01	0.01\\
302.01	0.01\\
303.01	0.01\\
304.01	0.01\\
305.01	0.01\\
306.01	0.01\\
307.01	0.01\\
308.01	0.01\\
309.01	0.01\\
310.01	0.01\\
311.01	0.01\\
312.01	0.01\\
313.01	0.01\\
314.01	0.01\\
315.01	0.01\\
316.01	0.01\\
317.01	0.01\\
318.01	0.01\\
319.01	0.01\\
320.01	0.01\\
321.01	0.01\\
322.01	0.01\\
323.01	0.01\\
324.01	0.01\\
325.01	0.01\\
326.01	0.01\\
327.01	0.01\\
328.01	0.01\\
329.01	0.01\\
330.01	0.01\\
331.01	0.01\\
332.01	0.01\\
333.01	0.01\\
334.01	0.01\\
335.01	0.01\\
336.01	0.01\\
337.01	0.01\\
338.01	0.01\\
339.01	0.01\\
340.01	0.01\\
341.01	0.01\\
342.01	0.01\\
343.01	0.01\\
344.01	0.01\\
345.01	0.01\\
346.01	0.01\\
347.01	0.01\\
348.01	0.01\\
349.01	0.01\\
350.01	0.01\\
351.01	0.01\\
352.01	0.01\\
353.01	0.01\\
354.01	0.01\\
355.01	0.01\\
356.01	0.01\\
357.01	0.01\\
358.01	0.01\\
359.01	0.01\\
360.01	0.01\\
361.01	0.01\\
362.01	0.01\\
363.01	0.01\\
364.01	0.01\\
365.01	0.01\\
366.01	0.01\\
367.01	0.01\\
368.01	0.01\\
369.01	0.01\\
370.01	0.01\\
371.01	0.01\\
372.01	0.01\\
373.01	0.01\\
374.01	0.01\\
375.01	0.01\\
376.01	0.01\\
377.01	0.01\\
378.01	0.01\\
379.01	0.01\\
380.01	0.01\\
381.01	0.01\\
382.01	0.01\\
383.01	0.01\\
384.01	0.01\\
385.01	0.01\\
386.01	0.01\\
387.01	0.01\\
388.01	0.01\\
389.01	0.01\\
390.01	0.01\\
391.01	0.01\\
392.01	0.01\\
393.01	0.01\\
394.01	0.01\\
395.01	0.01\\
396.01	0.01\\
397.01	0.01\\
398.01	0.01\\
399.01	0.01\\
400.01	0.01\\
401.01	0.01\\
402.01	0.01\\
403.01	0.01\\
404.01	0.01\\
405.01	0.01\\
406.01	0.01\\
407.01	0.01\\
408.01	0.01\\
409.01	0.01\\
410.01	0.01\\
411.01	0.01\\
412.01	0.01\\
413.01	0.01\\
414.01	0.01\\
415.01	0.01\\
416.01	0.01\\
417.01	0.01\\
418.01	0.01\\
419.01	0.01\\
420.01	0.01\\
421.01	0.01\\
422.01	0.01\\
423.01	0.01\\
424.01	0.01\\
425.01	0.01\\
426.01	0.01\\
427.01	0.01\\
428.01	0.01\\
429.01	0.01\\
430.01	0.01\\
431.01	0.01\\
432.01	0.01\\
433.01	0.01\\
434.01	0.01\\
435.01	0.01\\
436.01	0.01\\
437.01	0.01\\
438.01	0.01\\
439.01	0.01\\
440.01	0.01\\
441.01	0.01\\
442.01	0.01\\
443.01	0.01\\
444.01	0.01\\
445.01	0.01\\
446.01	0.01\\
447.01	0.01\\
448.01	0.01\\
449.01	0.01\\
450.01	0.01\\
451.01	0.01\\
452.01	0.01\\
453.01	0.01\\
454.01	0.01\\
455.01	0.01\\
456.01	0.01\\
457.01	0.01\\
458.01	0.01\\
459.01	0.01\\
460.01	0.01\\
461.01	0.01\\
462.01	0.01\\
463.01	0.01\\
464.01	0.01\\
465.01	0.01\\
466.01	0.01\\
467.01	0.01\\
468.01	0.01\\
469.01	0.01\\
470.01	0.01\\
471.01	0.01\\
472.01	0.01\\
473.01	0.01\\
474.01	0.01\\
475.01	0.01\\
476.01	0.01\\
477.01	0.01\\
478.01	0.01\\
479.01	0.01\\
480.01	0.01\\
481.01	0.01\\
482.01	0.01\\
483.01	0.01\\
484.01	0.01\\
485.01	0.01\\
486.01	0.01\\
487.01	0.01\\
488.01	0.01\\
489.01	0.01\\
490.01	0.01\\
491.01	0.01\\
492.01	0.01\\
493.01	0.01\\
494.01	0.01\\
495.01	0.01\\
496.01	0.01\\
497.01	0.01\\
498.01	0.01\\
499.01	0.01\\
500.01	0.01\\
501.01	0.01\\
502.01	0.01\\
503.01	0.01\\
504.01	0.01\\
505.01	0.01\\
506.01	0.01\\
507.01	0.01\\
508.01	0.01\\
509.01	0.01\\
510.01	0.01\\
511.01	0.01\\
512.01	0.01\\
513.01	0.01\\
514.01	0.01\\
515.01	0.01\\
516.01	0.01\\
517.01	0.01\\
518.01	0.01\\
519.01	0.01\\
520.01	0.01\\
521.01	0.01\\
522.01	0.01\\
523.01	0.01\\
524.01	0.01\\
525.01	0.01\\
526.01	0.01\\
527.01	0.01\\
528.01	0.01\\
529.01	0.01\\
530.01	0.01\\
531.01	0.01\\
532.01	0.01\\
533.01	0.01\\
534.01	0.01\\
535.01	0.01\\
536.01	0.01\\
537.01	0.01\\
538.01	0.01\\
539.01	0.01\\
540.01	0.01\\
541.01	0.01\\
542.01	0.01\\
543.01	0.01\\
544.01	0.01\\
545.01	0.01\\
546.01	0.01\\
547.01	0.01\\
548.01	0.01\\
549.01	0.01\\
550.01	0.01\\
551.01	0.01\\
552.01	0.01\\
553.01	0.01\\
554.01	0.01\\
555.01	0.01\\
556.01	0.01\\
557.01	0.01\\
558.01	0.01\\
559.01	0.01\\
560.01	0.01\\
561.01	0.01\\
562.01	0.01\\
563.01	0.01\\
564.01	0.01\\
565.01	0.01\\
566.01	0.01\\
567.01	0.01\\
568.01	0.01\\
569.01	0.01\\
570.01	0.01\\
571.01	0.01\\
572.01	0.01\\
573.01	0.01\\
574.01	0.01\\
575.01	0.01\\
576.01	0.01\\
577.01	0.01\\
578.01	0.01\\
579.01	0.01\\
580.01	0.01\\
581.01	0.01\\
582.01	0.01\\
583.01	0.01\\
584.01	0.01\\
585.01	0.01\\
586.01	0.01\\
587.01	0.01\\
588.01	0.01\\
589.01	0.01\\
590.01	0.01\\
591.01	0.01\\
592.01	0.01\\
593.01	0.01\\
594.01	0.01\\
595.01	0.01\\
596.01	0.01\\
597.01	0.01\\
598.01	0.00856177250016541\\
599.01	0.00614338855316429\\
599.02	0.00610585290136099\\
599.03	0.00606795959470588\\
599.04	0.00602970517912885\\
599.05	0.00599108616722146\\
599.06	0.00595209903791553\\
599.07	0.00591274023615851\\
599.08	0.00587300617258568\\
599.09	0.00583289322318931\\
599.1	0.00579239772898455\\
599.11	0.00575151599567213\\
599.12	0.00571024429329779\\
599.13	0.00566857885590845\\
599.14	0.00562651588120507\\
599.15	0.00558405153019215\\
599.16	0.00554118192682382\\
599.17	0.00549790315764664\\
599.18	0.00545421127143879\\
599.19	0.00541010227884592\\
599.2	0.00536557215201344\\
599.21	0.00532061682421524\\
599.22	0.00527523218947892\\
599.23	0.00522941410220728\\
599.24	0.00518315837679629\\
599.25	0.00513646078724932\\
599.26	0.00508931707748411\\
599.27	0.00504172295752704\\
599.28	0.00499367409618226\\
599.29	0.00494516612063596\\
599.3	0.00489619461605689\\
599.31	0.00484675512519297\\
599.32	0.00479684314796412\\
599.33	0.00474645414105115\\
599.34	0.00469558351748072\\
599.35	0.00464422664620627\\
599.36	0.004592378851685\\
599.37	0.00454003541345078\\
599.38	0.00448719156568293\\
599.39	0.00443384249677086\\
599.4	0.00437998334887462\\
599.41	0.00432560921748115\\
599.42	0.00427071515095633\\
599.43	0.00421529615009266\\
599.44	0.00415934716765269\\
599.45	0.00410286310790805\\
599.46	0.00404583882617403\\
599.47	0.00398826912833974\\
599.48	0.00393014877039371\\
599.49	0.003871472457945\\
599.5	0.0038122348457397\\
599.51	0.00375243053717278\\
599.52	0.00369205408379527\\
599.53	0.00363109998481676\\
599.54	0.00356956268660302\\
599.55	0.00350743658216895\\
599.56	0.00344471601066653\\
599.57	0.00338139525686797\\
599.58	0.00331746855064385\\
599.59	0.00325293006643631\\
599.6	0.00318777392272706\\
599.61	0.0031219941815005\\
599.62	0.0030555848477015\\
599.63	0.00298853986868809\\
599.64	0.0029208531336789\\
599.65	0.00285251847319523\\
599.66	0.00278352965849788\\
599.67	0.00271388040101852\\
599.68	0.00264356435178558\\
599.69	0.00257257510084473\\
599.7	0.00250090617667373\\
599.71	0.00242855104559168\\
599.72	0.00235550311116267\\
599.73	0.00228175571359365\\
599.74	0.00220730212912655\\
599.75	0.00213213556942455\\
599.76	0.00205624918095254\\
599.77	0.0019796360443516\\
599.78	0.00190228917380741\\
599.79	0.00182420151641279\\
599.8	0.00174536595152391\\
599.81	0.00166577529011051\\
599.82	0.0015854222740998\\
599.83	0.00150429957571408\\
599.84	0.00142239979680206\\
599.85	0.00133971546816369\\
599.86	0.00125623904886856\\
599.87	0.00117196292556781\\
599.88	0.00108687941179934\\
599.89	0.00100098074728643\\
599.9	0.000914259097229629\\
599.91	0.000826706551591819\\
599.92	0.000738315124376427\\
599.93	0.000649076752898733\\
599.94	0.000558983297050165\\
599.95	0.000468026538555518\\
599.96	0.000376198180223067\\
599.97	0.000283489845187451\\
599.98	0.000189893076145302\\
599.99	9.53993345834906e-05\\
600	0\\
};
\addplot [color=red!25!mycolor17,solid,forget plot]
  table[row sep=crcr]{%
0.01	0.01\\
1.01	0.01\\
2.01	0.01\\
3.01	0.01\\
4.01	0.01\\
5.01	0.01\\
6.01	0.01\\
7.01	0.01\\
8.01	0.01\\
9.01	0.01\\
10.01	0.01\\
11.01	0.01\\
12.01	0.01\\
13.01	0.01\\
14.01	0.01\\
15.01	0.01\\
16.01	0.01\\
17.01	0.01\\
18.01	0.01\\
19.01	0.01\\
20.01	0.01\\
21.01	0.01\\
22.01	0.01\\
23.01	0.01\\
24.01	0.01\\
25.01	0.01\\
26.01	0.01\\
27.01	0.01\\
28.01	0.01\\
29.01	0.01\\
30.01	0.01\\
31.01	0.01\\
32.01	0.01\\
33.01	0.01\\
34.01	0.01\\
35.01	0.01\\
36.01	0.01\\
37.01	0.01\\
38.01	0.01\\
39.01	0.01\\
40.01	0.01\\
41.01	0.01\\
42.01	0.01\\
43.01	0.01\\
44.01	0.01\\
45.01	0.01\\
46.01	0.01\\
47.01	0.01\\
48.01	0.01\\
49.01	0.01\\
50.01	0.01\\
51.01	0.01\\
52.01	0.01\\
53.01	0.01\\
54.01	0.01\\
55.01	0.01\\
56.01	0.01\\
57.01	0.01\\
58.01	0.01\\
59.01	0.01\\
60.01	0.01\\
61.01	0.01\\
62.01	0.01\\
63.01	0.01\\
64.01	0.01\\
65.01	0.01\\
66.01	0.01\\
67.01	0.01\\
68.01	0.01\\
69.01	0.01\\
70.01	0.01\\
71.01	0.01\\
72.01	0.01\\
73.01	0.01\\
74.01	0.01\\
75.01	0.01\\
76.01	0.01\\
77.01	0.01\\
78.01	0.01\\
79.01	0.01\\
80.01	0.01\\
81.01	0.01\\
82.01	0.01\\
83.01	0.01\\
84.01	0.01\\
85.01	0.01\\
86.01	0.01\\
87.01	0.01\\
88.01	0.01\\
89.01	0.01\\
90.01	0.01\\
91.01	0.01\\
92.01	0.01\\
93.01	0.01\\
94.01	0.01\\
95.01	0.01\\
96.01	0.01\\
97.01	0.01\\
98.01	0.01\\
99.01	0.01\\
100.01	0.01\\
101.01	0.01\\
102.01	0.01\\
103.01	0.01\\
104.01	0.01\\
105.01	0.01\\
106.01	0.01\\
107.01	0.01\\
108.01	0.01\\
109.01	0.01\\
110.01	0.01\\
111.01	0.01\\
112.01	0.01\\
113.01	0.01\\
114.01	0.01\\
115.01	0.01\\
116.01	0.01\\
117.01	0.01\\
118.01	0.01\\
119.01	0.01\\
120.01	0.01\\
121.01	0.01\\
122.01	0.01\\
123.01	0.01\\
124.01	0.01\\
125.01	0.01\\
126.01	0.01\\
127.01	0.01\\
128.01	0.01\\
129.01	0.01\\
130.01	0.01\\
131.01	0.01\\
132.01	0.01\\
133.01	0.01\\
134.01	0.01\\
135.01	0.01\\
136.01	0.01\\
137.01	0.01\\
138.01	0.01\\
139.01	0.01\\
140.01	0.01\\
141.01	0.01\\
142.01	0.01\\
143.01	0.01\\
144.01	0.01\\
145.01	0.01\\
146.01	0.01\\
147.01	0.01\\
148.01	0.01\\
149.01	0.01\\
150.01	0.01\\
151.01	0.01\\
152.01	0.01\\
153.01	0.01\\
154.01	0.01\\
155.01	0.01\\
156.01	0.01\\
157.01	0.01\\
158.01	0.01\\
159.01	0.01\\
160.01	0.01\\
161.01	0.01\\
162.01	0.01\\
163.01	0.01\\
164.01	0.01\\
165.01	0.01\\
166.01	0.01\\
167.01	0.01\\
168.01	0.01\\
169.01	0.01\\
170.01	0.01\\
171.01	0.01\\
172.01	0.01\\
173.01	0.01\\
174.01	0.01\\
175.01	0.01\\
176.01	0.01\\
177.01	0.01\\
178.01	0.01\\
179.01	0.01\\
180.01	0.01\\
181.01	0.01\\
182.01	0.01\\
183.01	0.01\\
184.01	0.01\\
185.01	0.01\\
186.01	0.01\\
187.01	0.01\\
188.01	0.01\\
189.01	0.01\\
190.01	0.01\\
191.01	0.01\\
192.01	0.01\\
193.01	0.01\\
194.01	0.01\\
195.01	0.01\\
196.01	0.01\\
197.01	0.01\\
198.01	0.01\\
199.01	0.01\\
200.01	0.01\\
201.01	0.01\\
202.01	0.01\\
203.01	0.01\\
204.01	0.01\\
205.01	0.01\\
206.01	0.01\\
207.01	0.01\\
208.01	0.01\\
209.01	0.01\\
210.01	0.01\\
211.01	0.01\\
212.01	0.01\\
213.01	0.01\\
214.01	0.01\\
215.01	0.01\\
216.01	0.01\\
217.01	0.01\\
218.01	0.01\\
219.01	0.01\\
220.01	0.01\\
221.01	0.01\\
222.01	0.01\\
223.01	0.01\\
224.01	0.01\\
225.01	0.01\\
226.01	0.01\\
227.01	0.01\\
228.01	0.01\\
229.01	0.01\\
230.01	0.01\\
231.01	0.01\\
232.01	0.01\\
233.01	0.01\\
234.01	0.01\\
235.01	0.01\\
236.01	0.01\\
237.01	0.01\\
238.01	0.01\\
239.01	0.01\\
240.01	0.01\\
241.01	0.01\\
242.01	0.01\\
243.01	0.01\\
244.01	0.01\\
245.01	0.01\\
246.01	0.01\\
247.01	0.01\\
248.01	0.01\\
249.01	0.01\\
250.01	0.01\\
251.01	0.01\\
252.01	0.01\\
253.01	0.01\\
254.01	0.01\\
255.01	0.01\\
256.01	0.01\\
257.01	0.01\\
258.01	0.01\\
259.01	0.01\\
260.01	0.01\\
261.01	0.01\\
262.01	0.01\\
263.01	0.01\\
264.01	0.01\\
265.01	0.01\\
266.01	0.01\\
267.01	0.01\\
268.01	0.01\\
269.01	0.01\\
270.01	0.01\\
271.01	0.01\\
272.01	0.01\\
273.01	0.01\\
274.01	0.01\\
275.01	0.01\\
276.01	0.01\\
277.01	0.01\\
278.01	0.01\\
279.01	0.01\\
280.01	0.01\\
281.01	0.01\\
282.01	0.01\\
283.01	0.01\\
284.01	0.01\\
285.01	0.01\\
286.01	0.01\\
287.01	0.01\\
288.01	0.01\\
289.01	0.01\\
290.01	0.01\\
291.01	0.01\\
292.01	0.01\\
293.01	0.01\\
294.01	0.01\\
295.01	0.01\\
296.01	0.01\\
297.01	0.01\\
298.01	0.01\\
299.01	0.01\\
300.01	0.01\\
301.01	0.01\\
302.01	0.01\\
303.01	0.01\\
304.01	0.01\\
305.01	0.01\\
306.01	0.01\\
307.01	0.01\\
308.01	0.01\\
309.01	0.01\\
310.01	0.01\\
311.01	0.01\\
312.01	0.01\\
313.01	0.01\\
314.01	0.01\\
315.01	0.01\\
316.01	0.01\\
317.01	0.01\\
318.01	0.01\\
319.01	0.01\\
320.01	0.01\\
321.01	0.01\\
322.01	0.01\\
323.01	0.01\\
324.01	0.01\\
325.01	0.01\\
326.01	0.01\\
327.01	0.01\\
328.01	0.01\\
329.01	0.01\\
330.01	0.01\\
331.01	0.01\\
332.01	0.01\\
333.01	0.01\\
334.01	0.01\\
335.01	0.01\\
336.01	0.01\\
337.01	0.01\\
338.01	0.01\\
339.01	0.01\\
340.01	0.01\\
341.01	0.01\\
342.01	0.01\\
343.01	0.01\\
344.01	0.01\\
345.01	0.01\\
346.01	0.01\\
347.01	0.01\\
348.01	0.01\\
349.01	0.01\\
350.01	0.01\\
351.01	0.01\\
352.01	0.01\\
353.01	0.01\\
354.01	0.01\\
355.01	0.01\\
356.01	0.01\\
357.01	0.01\\
358.01	0.01\\
359.01	0.01\\
360.01	0.01\\
361.01	0.01\\
362.01	0.01\\
363.01	0.01\\
364.01	0.01\\
365.01	0.01\\
366.01	0.01\\
367.01	0.01\\
368.01	0.01\\
369.01	0.01\\
370.01	0.01\\
371.01	0.01\\
372.01	0.01\\
373.01	0.01\\
374.01	0.01\\
375.01	0.01\\
376.01	0.01\\
377.01	0.01\\
378.01	0.01\\
379.01	0.01\\
380.01	0.01\\
381.01	0.01\\
382.01	0.01\\
383.01	0.01\\
384.01	0.01\\
385.01	0.01\\
386.01	0.01\\
387.01	0.01\\
388.01	0.01\\
389.01	0.01\\
390.01	0.01\\
391.01	0.01\\
392.01	0.01\\
393.01	0.01\\
394.01	0.01\\
395.01	0.01\\
396.01	0.01\\
397.01	0.01\\
398.01	0.01\\
399.01	0.01\\
400.01	0.01\\
401.01	0.01\\
402.01	0.01\\
403.01	0.01\\
404.01	0.01\\
405.01	0.01\\
406.01	0.01\\
407.01	0.01\\
408.01	0.01\\
409.01	0.01\\
410.01	0.01\\
411.01	0.01\\
412.01	0.01\\
413.01	0.01\\
414.01	0.01\\
415.01	0.01\\
416.01	0.01\\
417.01	0.01\\
418.01	0.01\\
419.01	0.01\\
420.01	0.01\\
421.01	0.01\\
422.01	0.01\\
423.01	0.01\\
424.01	0.01\\
425.01	0.01\\
426.01	0.01\\
427.01	0.01\\
428.01	0.01\\
429.01	0.01\\
430.01	0.01\\
431.01	0.01\\
432.01	0.01\\
433.01	0.01\\
434.01	0.01\\
435.01	0.01\\
436.01	0.01\\
437.01	0.01\\
438.01	0.01\\
439.01	0.01\\
440.01	0.01\\
441.01	0.01\\
442.01	0.01\\
443.01	0.01\\
444.01	0.01\\
445.01	0.01\\
446.01	0.01\\
447.01	0.01\\
448.01	0.01\\
449.01	0.01\\
450.01	0.01\\
451.01	0.01\\
452.01	0.01\\
453.01	0.01\\
454.01	0.01\\
455.01	0.01\\
456.01	0.01\\
457.01	0.01\\
458.01	0.01\\
459.01	0.01\\
460.01	0.01\\
461.01	0.01\\
462.01	0.01\\
463.01	0.01\\
464.01	0.01\\
465.01	0.01\\
466.01	0.01\\
467.01	0.01\\
468.01	0.01\\
469.01	0.01\\
470.01	0.01\\
471.01	0.01\\
472.01	0.01\\
473.01	0.01\\
474.01	0.01\\
475.01	0.01\\
476.01	0.01\\
477.01	0.01\\
478.01	0.01\\
479.01	0.01\\
480.01	0.01\\
481.01	0.01\\
482.01	0.01\\
483.01	0.01\\
484.01	0.01\\
485.01	0.01\\
486.01	0.01\\
487.01	0.01\\
488.01	0.01\\
489.01	0.01\\
490.01	0.01\\
491.01	0.01\\
492.01	0.01\\
493.01	0.01\\
494.01	0.01\\
495.01	0.01\\
496.01	0.01\\
497.01	0.01\\
498.01	0.01\\
499.01	0.01\\
500.01	0.01\\
501.01	0.01\\
502.01	0.01\\
503.01	0.01\\
504.01	0.01\\
505.01	0.01\\
506.01	0.01\\
507.01	0.01\\
508.01	0.01\\
509.01	0.01\\
510.01	0.01\\
511.01	0.01\\
512.01	0.01\\
513.01	0.01\\
514.01	0.01\\
515.01	0.01\\
516.01	0.01\\
517.01	0.01\\
518.01	0.01\\
519.01	0.01\\
520.01	0.01\\
521.01	0.01\\
522.01	0.01\\
523.01	0.01\\
524.01	0.01\\
525.01	0.01\\
526.01	0.01\\
527.01	0.01\\
528.01	0.01\\
529.01	0.01\\
530.01	0.01\\
531.01	0.01\\
532.01	0.01\\
533.01	0.01\\
534.01	0.01\\
535.01	0.01\\
536.01	0.01\\
537.01	0.01\\
538.01	0.01\\
539.01	0.01\\
540.01	0.01\\
541.01	0.01\\
542.01	0.01\\
543.01	0.01\\
544.01	0.01\\
545.01	0.01\\
546.01	0.01\\
547.01	0.01\\
548.01	0.01\\
549.01	0.01\\
550.01	0.01\\
551.01	0.01\\
552.01	0.01\\
553.01	0.01\\
554.01	0.01\\
555.01	0.01\\
556.01	0.01\\
557.01	0.01\\
558.01	0.01\\
559.01	0.01\\
560.01	0.01\\
561.01	0.01\\
562.01	0.01\\
563.01	0.01\\
564.01	0.01\\
565.01	0.01\\
566.01	0.01\\
567.01	0.01\\
568.01	0.01\\
569.01	0.01\\
570.01	0.01\\
571.01	0.01\\
572.01	0.01\\
573.01	0.01\\
574.01	0.01\\
575.01	0.01\\
576.01	0.01\\
577.01	0.01\\
578.01	0.01\\
579.01	0.01\\
580.01	0.01\\
581.01	0.01\\
582.01	0.01\\
583.01	0.01\\
584.01	0.01\\
585.01	0.01\\
586.01	0.01\\
587.01	0.01\\
588.01	0.01\\
589.01	0.01\\
590.01	0.01\\
591.01	0.01\\
592.01	0.01\\
593.01	0.01\\
594.01	0.01\\
595.01	0.01\\
596.01	0.01\\
597.01	0.01\\
598.01	0.00856167982142255\\
599.01	0.00614338744854443\\
599.02	0.00610585183149378\\
599.03	0.00606795855896235\\
599.04	0.00602970417688114\\
599.05	0.00599108519784297\\
599.06	0.00595209810078091\\
599.07	0.00591273933064373\\
599.08	0.00587300529806812\\
599.09	0.00583289237904784\\
599.1	0.00579239691459957\\
599.11	0.00575151521042569\\
599.12	0.00571024353657369\\
599.13	0.00566857812709233\\
599.14	0.00562651517968449\\
599.15	0.00558405085535673\\
599.16	0.00554118127806537\\
599.17	0.00549790253435924\\
599.18	0.00545421067301896\\
599.19	0.00541010170469274\\
599.2	0.0053655716015287\\
599.21	0.0053206162968036\\
599.22	0.00527523168454804\\
599.23	0.00522941361916799\\
599.24	0.00518315791506279\\
599.25	0.00513646034623931\\
599.26	0.00508931665661926\\
599.27	0.00504172255623291\\
599.28	0.00499367371388847\\
599.29	0.0049451657567764\\
599.3	0.0048961942700699\\
599.31	0.00484675479652159\\
599.32	0.00479684283605632\\
599.33	0.00474645384536002\\
599.34	0.00469558323746471\\
599.35	0.00464422638132944\\
599.36	0.00459237860141723\\
599.37	0.00454003517726804\\
599.38	0.00448719134306749\\
599.39	0.00443384228721157\\
599.4	0.00437998315186716\\
599.41	0.00432560903252826\\
599.42	0.00427071497756806\\
599.43	0.00421529598778667\\
599.44	0.00415934701595446\\
599.45	0.00410286296635117\\
599.46	0.00404583869430045\\
599.47	0.00398826900569998\\
599.48	0.00393014865654718\\
599.49	0.00387147235246022\\
599.5	0.00381223474819453\\
599.51	0.0037524304471547\\
599.52	0.0036920540009016\\
599.53	0.00363109990865488\\
599.54	0.00356956261679062\\
599.55	0.00350743651833422\\
599.56	0.0034447159524484\\
599.57	0.0033813952039163\\
599.58	0.00331746850261961\\
599.59	0.00325293002301176\\
599.6	0.00318777388358591\\
599.61	0.00312199414633807\\
599.62	0.00305558481622484\\
599.63	0.00298853984061611\\
599.64	0.00292085310874244\\
599.65	0.00285251845113719\\
599.66	0.00278352963907322\\
599.67	0.00271388038399433\\
599.68	0.00264356433694107\\
599.69	0.00257257508797122\\
599.7	0.00250090616557461\\
599.71	0.00242855103608235\\
599.72	0.00235550310307042\\
599.73	0.00228175570675753\\
599.74	0.00220730212339721\\
599.75	0.00213213556466405\\
599.76	0.00205624917703414\\
599.77	0.00197963604115943\\
599.78	0.00190228917123624\\
599.79	0.00182420151436763\\
599.8	0.00174536594991967\\
599.81	0.00166577528887161\\
599.82	0.00158542227315966\\
599.83	0.00150429957501471\\
599.84	0.00142239979629348\\
599.85	0.00133971546780342\\
599.86	0.00125623904862105\\
599.87	0.00117196292540381\\
599.88	0.0010868794116953\\
599.89	0.00100098074722385\\
599.9	0.00091425909719443\\
599.91	0.000826706551573665\\
599.92	0.000738315124368102\\
599.93	0.000649076752895515\\
599.94	0.000558983297049216\\
599.95	0.000468026538555355\\
599.96	0.000376198180223067\\
599.97	0.000283489845187453\\
599.98	0.0001898930761453\\
599.99	9.53993345834906e-05\\
600	0\\
};
\addplot [color=mycolor19,solid,forget plot]
  table[row sep=crcr]{%
0.01	0.01\\
1.01	0.01\\
2.01	0.01\\
3.01	0.01\\
4.01	0.01\\
5.01	0.01\\
6.01	0.01\\
7.01	0.01\\
8.01	0.01\\
9.01	0.01\\
10.01	0.01\\
11.01	0.01\\
12.01	0.01\\
13.01	0.01\\
14.01	0.01\\
15.01	0.01\\
16.01	0.01\\
17.01	0.01\\
18.01	0.01\\
19.01	0.01\\
20.01	0.01\\
21.01	0.01\\
22.01	0.01\\
23.01	0.01\\
24.01	0.01\\
25.01	0.01\\
26.01	0.01\\
27.01	0.01\\
28.01	0.01\\
29.01	0.01\\
30.01	0.01\\
31.01	0.01\\
32.01	0.01\\
33.01	0.01\\
34.01	0.01\\
35.01	0.01\\
36.01	0.01\\
37.01	0.01\\
38.01	0.01\\
39.01	0.01\\
40.01	0.01\\
41.01	0.01\\
42.01	0.01\\
43.01	0.01\\
44.01	0.01\\
45.01	0.01\\
46.01	0.01\\
47.01	0.01\\
48.01	0.01\\
49.01	0.01\\
50.01	0.01\\
51.01	0.01\\
52.01	0.01\\
53.01	0.01\\
54.01	0.01\\
55.01	0.01\\
56.01	0.01\\
57.01	0.01\\
58.01	0.01\\
59.01	0.01\\
60.01	0.01\\
61.01	0.01\\
62.01	0.01\\
63.01	0.01\\
64.01	0.01\\
65.01	0.01\\
66.01	0.01\\
67.01	0.01\\
68.01	0.01\\
69.01	0.01\\
70.01	0.01\\
71.01	0.01\\
72.01	0.01\\
73.01	0.01\\
74.01	0.01\\
75.01	0.01\\
76.01	0.01\\
77.01	0.01\\
78.01	0.01\\
79.01	0.01\\
80.01	0.01\\
81.01	0.01\\
82.01	0.01\\
83.01	0.01\\
84.01	0.01\\
85.01	0.01\\
86.01	0.01\\
87.01	0.01\\
88.01	0.01\\
89.01	0.01\\
90.01	0.01\\
91.01	0.01\\
92.01	0.01\\
93.01	0.01\\
94.01	0.01\\
95.01	0.01\\
96.01	0.01\\
97.01	0.01\\
98.01	0.01\\
99.01	0.01\\
100.01	0.01\\
101.01	0.01\\
102.01	0.01\\
103.01	0.01\\
104.01	0.01\\
105.01	0.01\\
106.01	0.01\\
107.01	0.01\\
108.01	0.01\\
109.01	0.01\\
110.01	0.01\\
111.01	0.01\\
112.01	0.01\\
113.01	0.01\\
114.01	0.01\\
115.01	0.01\\
116.01	0.01\\
117.01	0.01\\
118.01	0.01\\
119.01	0.01\\
120.01	0.01\\
121.01	0.01\\
122.01	0.01\\
123.01	0.01\\
124.01	0.01\\
125.01	0.01\\
126.01	0.01\\
127.01	0.01\\
128.01	0.01\\
129.01	0.01\\
130.01	0.01\\
131.01	0.01\\
132.01	0.01\\
133.01	0.01\\
134.01	0.01\\
135.01	0.01\\
136.01	0.01\\
137.01	0.01\\
138.01	0.01\\
139.01	0.01\\
140.01	0.01\\
141.01	0.01\\
142.01	0.01\\
143.01	0.01\\
144.01	0.01\\
145.01	0.01\\
146.01	0.01\\
147.01	0.01\\
148.01	0.01\\
149.01	0.01\\
150.01	0.01\\
151.01	0.01\\
152.01	0.01\\
153.01	0.01\\
154.01	0.01\\
155.01	0.01\\
156.01	0.01\\
157.01	0.01\\
158.01	0.01\\
159.01	0.01\\
160.01	0.01\\
161.01	0.01\\
162.01	0.01\\
163.01	0.01\\
164.01	0.01\\
165.01	0.01\\
166.01	0.01\\
167.01	0.01\\
168.01	0.01\\
169.01	0.01\\
170.01	0.01\\
171.01	0.01\\
172.01	0.01\\
173.01	0.01\\
174.01	0.01\\
175.01	0.01\\
176.01	0.01\\
177.01	0.01\\
178.01	0.01\\
179.01	0.01\\
180.01	0.01\\
181.01	0.01\\
182.01	0.01\\
183.01	0.01\\
184.01	0.01\\
185.01	0.01\\
186.01	0.01\\
187.01	0.01\\
188.01	0.01\\
189.01	0.01\\
190.01	0.01\\
191.01	0.01\\
192.01	0.01\\
193.01	0.01\\
194.01	0.01\\
195.01	0.01\\
196.01	0.01\\
197.01	0.01\\
198.01	0.01\\
199.01	0.01\\
200.01	0.01\\
201.01	0.01\\
202.01	0.01\\
203.01	0.01\\
204.01	0.01\\
205.01	0.01\\
206.01	0.01\\
207.01	0.01\\
208.01	0.01\\
209.01	0.01\\
210.01	0.01\\
211.01	0.01\\
212.01	0.01\\
213.01	0.01\\
214.01	0.01\\
215.01	0.01\\
216.01	0.01\\
217.01	0.01\\
218.01	0.01\\
219.01	0.01\\
220.01	0.01\\
221.01	0.01\\
222.01	0.01\\
223.01	0.01\\
224.01	0.01\\
225.01	0.01\\
226.01	0.01\\
227.01	0.01\\
228.01	0.01\\
229.01	0.01\\
230.01	0.01\\
231.01	0.01\\
232.01	0.01\\
233.01	0.01\\
234.01	0.01\\
235.01	0.01\\
236.01	0.01\\
237.01	0.01\\
238.01	0.01\\
239.01	0.01\\
240.01	0.01\\
241.01	0.01\\
242.01	0.01\\
243.01	0.01\\
244.01	0.01\\
245.01	0.01\\
246.01	0.01\\
247.01	0.01\\
248.01	0.01\\
249.01	0.01\\
250.01	0.01\\
251.01	0.01\\
252.01	0.01\\
253.01	0.01\\
254.01	0.01\\
255.01	0.01\\
256.01	0.01\\
257.01	0.01\\
258.01	0.01\\
259.01	0.01\\
260.01	0.01\\
261.01	0.01\\
262.01	0.01\\
263.01	0.01\\
264.01	0.01\\
265.01	0.01\\
266.01	0.01\\
267.01	0.01\\
268.01	0.01\\
269.01	0.01\\
270.01	0.01\\
271.01	0.01\\
272.01	0.01\\
273.01	0.01\\
274.01	0.01\\
275.01	0.01\\
276.01	0.01\\
277.01	0.01\\
278.01	0.01\\
279.01	0.01\\
280.01	0.01\\
281.01	0.01\\
282.01	0.01\\
283.01	0.01\\
284.01	0.01\\
285.01	0.01\\
286.01	0.01\\
287.01	0.01\\
288.01	0.01\\
289.01	0.01\\
290.01	0.01\\
291.01	0.01\\
292.01	0.01\\
293.01	0.01\\
294.01	0.01\\
295.01	0.01\\
296.01	0.01\\
297.01	0.01\\
298.01	0.01\\
299.01	0.01\\
300.01	0.01\\
301.01	0.01\\
302.01	0.01\\
303.01	0.01\\
304.01	0.01\\
305.01	0.01\\
306.01	0.01\\
307.01	0.01\\
308.01	0.01\\
309.01	0.01\\
310.01	0.01\\
311.01	0.01\\
312.01	0.01\\
313.01	0.01\\
314.01	0.01\\
315.01	0.01\\
316.01	0.01\\
317.01	0.01\\
318.01	0.01\\
319.01	0.01\\
320.01	0.01\\
321.01	0.01\\
322.01	0.01\\
323.01	0.01\\
324.01	0.01\\
325.01	0.01\\
326.01	0.01\\
327.01	0.01\\
328.01	0.01\\
329.01	0.01\\
330.01	0.01\\
331.01	0.01\\
332.01	0.01\\
333.01	0.01\\
334.01	0.01\\
335.01	0.01\\
336.01	0.01\\
337.01	0.01\\
338.01	0.01\\
339.01	0.01\\
340.01	0.01\\
341.01	0.01\\
342.01	0.01\\
343.01	0.01\\
344.01	0.01\\
345.01	0.01\\
346.01	0.01\\
347.01	0.01\\
348.01	0.01\\
349.01	0.01\\
350.01	0.01\\
351.01	0.01\\
352.01	0.01\\
353.01	0.01\\
354.01	0.01\\
355.01	0.01\\
356.01	0.01\\
357.01	0.01\\
358.01	0.01\\
359.01	0.01\\
360.01	0.01\\
361.01	0.01\\
362.01	0.01\\
363.01	0.01\\
364.01	0.01\\
365.01	0.01\\
366.01	0.01\\
367.01	0.01\\
368.01	0.01\\
369.01	0.01\\
370.01	0.01\\
371.01	0.01\\
372.01	0.01\\
373.01	0.01\\
374.01	0.01\\
375.01	0.01\\
376.01	0.01\\
377.01	0.01\\
378.01	0.01\\
379.01	0.01\\
380.01	0.01\\
381.01	0.01\\
382.01	0.01\\
383.01	0.01\\
384.01	0.01\\
385.01	0.01\\
386.01	0.01\\
387.01	0.01\\
388.01	0.01\\
389.01	0.01\\
390.01	0.01\\
391.01	0.01\\
392.01	0.01\\
393.01	0.01\\
394.01	0.01\\
395.01	0.01\\
396.01	0.01\\
397.01	0.01\\
398.01	0.01\\
399.01	0.01\\
400.01	0.01\\
401.01	0.01\\
402.01	0.01\\
403.01	0.01\\
404.01	0.01\\
405.01	0.01\\
406.01	0.01\\
407.01	0.01\\
408.01	0.01\\
409.01	0.01\\
410.01	0.01\\
411.01	0.01\\
412.01	0.01\\
413.01	0.01\\
414.01	0.01\\
415.01	0.01\\
416.01	0.01\\
417.01	0.01\\
418.01	0.01\\
419.01	0.01\\
420.01	0.01\\
421.01	0.01\\
422.01	0.01\\
423.01	0.01\\
424.01	0.01\\
425.01	0.01\\
426.01	0.01\\
427.01	0.01\\
428.01	0.01\\
429.01	0.01\\
430.01	0.01\\
431.01	0.01\\
432.01	0.01\\
433.01	0.01\\
434.01	0.01\\
435.01	0.01\\
436.01	0.01\\
437.01	0.01\\
438.01	0.01\\
439.01	0.01\\
440.01	0.01\\
441.01	0.01\\
442.01	0.01\\
443.01	0.01\\
444.01	0.01\\
445.01	0.01\\
446.01	0.01\\
447.01	0.01\\
448.01	0.01\\
449.01	0.01\\
450.01	0.01\\
451.01	0.01\\
452.01	0.01\\
453.01	0.01\\
454.01	0.01\\
455.01	0.01\\
456.01	0.01\\
457.01	0.01\\
458.01	0.01\\
459.01	0.01\\
460.01	0.01\\
461.01	0.01\\
462.01	0.01\\
463.01	0.01\\
464.01	0.01\\
465.01	0.01\\
466.01	0.01\\
467.01	0.01\\
468.01	0.01\\
469.01	0.01\\
470.01	0.01\\
471.01	0.01\\
472.01	0.01\\
473.01	0.01\\
474.01	0.01\\
475.01	0.01\\
476.01	0.01\\
477.01	0.01\\
478.01	0.01\\
479.01	0.01\\
480.01	0.01\\
481.01	0.01\\
482.01	0.01\\
483.01	0.01\\
484.01	0.01\\
485.01	0.01\\
486.01	0.01\\
487.01	0.01\\
488.01	0.01\\
489.01	0.01\\
490.01	0.01\\
491.01	0.01\\
492.01	0.01\\
493.01	0.01\\
494.01	0.01\\
495.01	0.01\\
496.01	0.01\\
497.01	0.01\\
498.01	0.01\\
499.01	0.01\\
500.01	0.01\\
501.01	0.01\\
502.01	0.01\\
503.01	0.01\\
504.01	0.01\\
505.01	0.01\\
506.01	0.01\\
507.01	0.01\\
508.01	0.01\\
509.01	0.01\\
510.01	0.01\\
511.01	0.01\\
512.01	0.01\\
513.01	0.01\\
514.01	0.01\\
515.01	0.01\\
516.01	0.01\\
517.01	0.01\\
518.01	0.01\\
519.01	0.01\\
520.01	0.01\\
521.01	0.01\\
522.01	0.01\\
523.01	0.01\\
524.01	0.01\\
525.01	0.01\\
526.01	0.01\\
527.01	0.01\\
528.01	0.01\\
529.01	0.01\\
530.01	0.01\\
531.01	0.01\\
532.01	0.01\\
533.01	0.01\\
534.01	0.01\\
535.01	0.01\\
536.01	0.01\\
537.01	0.01\\
538.01	0.01\\
539.01	0.01\\
540.01	0.01\\
541.01	0.01\\
542.01	0.01\\
543.01	0.01\\
544.01	0.01\\
545.01	0.01\\
546.01	0.01\\
547.01	0.01\\
548.01	0.01\\
549.01	0.01\\
550.01	0.01\\
551.01	0.01\\
552.01	0.01\\
553.01	0.01\\
554.01	0.01\\
555.01	0.01\\
556.01	0.01\\
557.01	0.01\\
558.01	0.01\\
559.01	0.01\\
560.01	0.01\\
561.01	0.01\\
562.01	0.01\\
563.01	0.01\\
564.01	0.01\\
565.01	0.01\\
566.01	0.01\\
567.01	0.01\\
568.01	0.01\\
569.01	0.01\\
570.01	0.01\\
571.01	0.01\\
572.01	0.01\\
573.01	0.01\\
574.01	0.01\\
575.01	0.01\\
576.01	0.01\\
577.01	0.01\\
578.01	0.01\\
579.01	0.01\\
580.01	0.01\\
581.01	0.01\\
582.01	0.01\\
583.01	0.01\\
584.01	0.01\\
585.01	0.01\\
586.01	0.01\\
587.01	0.01\\
588.01	0.01\\
589.01	0.01\\
590.01	0.01\\
591.01	0.01\\
592.01	0.01\\
593.01	0.01\\
594.01	0.01\\
595.01	0.01\\
596.01	0.01\\
597.01	0.01\\
598.01	0.008561679326351\\
599.01	0.00614338742821567\\
599.02	0.0061058518119638\\
599.03	0.00606795854021002\\
599.04	0.00602970415888558\\
599.05	0.00599108518058355\\
599.06	0.00595209808423727\\
599.07	0.00591273931479576\\
599.08	0.00587300528289598\\
599.09	0.00583289236453193\\
599.1	0.00579239690072057\\
599.11	0.00575151519716454\\
599.12	0.00571024352391159\\
599.13	0.00566857811501075\\
599.14	0.00562651516816516\\
599.15	0.00558405084438166\\
599.16	0.00554118126761685\\
599.17	0.00549790252441982\\
599.18	0.00545421066357148\\
599.19	0.00541010169572032\\
599.2	0.00536557159301473\\
599.21	0.00532061628873176\\
599.22	0.0052752316769023\\
599.23	0.00522941361193262\\
599.24	0.00518315790822233\\
599.25	0.00513646033977861\\
599.26	0.00508931665052346\\
599.27	0.00504172255048744\\
599.28	0.00499367370847907\\
599.29	0.00494516575168911\\
599.3	0.00489619426529107\\
599.31	0.00484675479203787\\
599.32	0.00479684283185467\\
599.33	0.00474645384142771\\
599.34	0.00469558323378931\\
599.35	0.00464422637789884\\
599.36	0.00459237859821964\\
599.37	0.00454003517429197\\
599.38	0.00448719134030178\\
599.39	0.00443384228464538\\
599.4	0.00437998314948995\\
599.41	0.00432560903032982\\
599.42	0.00427071497553852\\
599.43	0.00421529598591643\\
599.44	0.00415934701423429\\
599.45	0.00410286296477212\\
599.46	0.0040458386928539\\
599.47	0.00398826900437764\\
599.48	0.00393014865534107\\
599.49	0.00387147235136265\\
599.5	0.00381223474719816\\
599.51	0.00375243044625247\\
599.52	0.00369205400008678\\
599.53	0.00363109990792102\\
599.54	0.0035695626161316\\
599.55	0.00350743651774419\\
599.56	0.00344471595192183\\
599.57	0.00338139520344793\\
599.58	0.00331746850220447\\
599.59	0.00325293002264517\\
599.6	0.00318777388326346\\
599.61	0.00312199414605561\\
599.62	0.00305558481597849\\
599.63	0.00298853984040225\\
599.64	0.0029208531085577\\
599.65	0.00285251845097843\\
599.66	0.00278352963893755\\
599.67	0.00271388038387908\\
599.68	0.00264356433684378\\
599.69	0.00257257508788965\\
599.7	0.00250090616550671\\
599.71	0.00242855103602627\\
599.72	0.0023555031030245\\
599.73	0.00228175570672028\\
599.74	0.00220730212336728\\
599.75	0.00213213556464027\\
599.76	0.00205624917701546\\
599.77	0.00197963604114495\\
599.78	0.00190228917122518\\
599.79	0.00182420151435931\\
599.8	0.00174536594991354\\
599.81	0.00166577528886716\\
599.82	0.00158542227315652\\
599.83	0.00150429957501254\\
599.84	0.00142239979629203\\
599.85	0.00133971546780248\\
599.86	0.00125623904862046\\
599.87	0.00117196292540346\\
599.88	0.0010868794116951\\
599.89	0.00100098074722375\\
599.9	0.000914259097194382\\
599.91	0.000826706551573646\\
599.92	0.000738315124368095\\
599.93	0.00064907675289551\\
599.94	0.000558983297049214\\
599.95	0.000468026538555355\\
599.96	0.000376198180223065\\
599.97	0.000283489845187451\\
599.98	0.000189893076145302\\
599.99	9.53993345834906e-05\\
600	0\\
};
\addplot [color=red!50!mycolor17,solid,forget plot]
  table[row sep=crcr]{%
0.01	0.00999999999999999\\
1.01	0.00999999999999999\\
2.01	0.00999999999999999\\
3.01	0.00999999999999999\\
4.01	0.00999999999999999\\
5.01	0.00999999999999999\\
6.01	0.00999999999999999\\
7.01	0.00999999999999999\\
8.01	0.00999999999999999\\
9.01	0.00999999999999999\\
10.01	0.00999999999999999\\
11.01	0.00999999999999999\\
12.01	0.00999999999999999\\
13.01	0.00999999999999999\\
14.01	0.00999999999999999\\
15.01	0.00999999999999999\\
16.01	0.00999999999999999\\
17.01	0.00999999999999999\\
18.01	0.00999999999999999\\
19.01	0.00999999999999999\\
20.01	0.00999999999999999\\
21.01	0.00999999999999999\\
22.01	0.00999999999999999\\
23.01	0.00999999999999999\\
24.01	0.00999999999999999\\
25.01	0.00999999999999999\\
26.01	0.00999999999999999\\
27.01	0.00999999999999999\\
28.01	0.00999999999999999\\
29.01	0.00999999999999999\\
30.01	0.00999999999999999\\
31.01	0.00999999999999999\\
32.01	0.00999999999999999\\
33.01	0.00999999999999999\\
34.01	0.00999999999999999\\
35.01	0.00999999999999999\\
36.01	0.00999999999999999\\
37.01	0.00999999999999999\\
38.01	0.00999999999999999\\
39.01	0.00999999999999999\\
40.01	0.00999999999999999\\
41.01	0.00999999999999999\\
42.01	0.00999999999999999\\
43.01	0.00999999999999999\\
44.01	0.00999999999999999\\
45.01	0.00999999999999999\\
46.01	0.00999999999999999\\
47.01	0.00999999999999999\\
48.01	0.00999999999999999\\
49.01	0.00999999999999999\\
50.01	0.00999999999999999\\
51.01	0.00999999999999999\\
52.01	0.00999999999999999\\
53.01	0.00999999999999999\\
54.01	0.00999999999999999\\
55.01	0.00999999999999999\\
56.01	0.00999999999999999\\
57.01	0.00999999999999999\\
58.01	0.00999999999999999\\
59.01	0.00999999999999999\\
60.01	0.00999999999999999\\
61.01	0.00999999999999999\\
62.01	0.00999999999999999\\
63.01	0.00999999999999999\\
64.01	0.00999999999999999\\
65.01	0.00999999999999999\\
66.01	0.00999999999999999\\
67.01	0.00999999999999999\\
68.01	0.00999999999999999\\
69.01	0.00999999999999999\\
70.01	0.00999999999999999\\
71.01	0.00999999999999999\\
72.01	0.00999999999999999\\
73.01	0.00999999999999999\\
74.01	0.00999999999999999\\
75.01	0.00999999999999999\\
76.01	0.00999999999999999\\
77.01	0.00999999999999999\\
78.01	0.00999999999999999\\
79.01	0.00999999999999999\\
80.01	0.00999999999999999\\
81.01	0.00999999999999999\\
82.01	0.00999999999999999\\
83.01	0.00999999999999999\\
84.01	0.00999999999999999\\
85.01	0.00999999999999999\\
86.01	0.00999999999999999\\
87.01	0.00999999999999999\\
88.01	0.00999999999999999\\
89.01	0.00999999999999999\\
90.01	0.00999999999999999\\
91.01	0.00999999999999999\\
92.01	0.00999999999999999\\
93.01	0.00999999999999999\\
94.01	0.00999999999999999\\
95.01	0.00999999999999999\\
96.01	0.00999999999999999\\
97.01	0.00999999999999999\\
98.01	0.00999999999999999\\
99.01	0.00999999999999999\\
100.01	0.00999999999999999\\
101.01	0.00999999999999999\\
102.01	0.00999999999999999\\
103.01	0.00999999999999999\\
104.01	0.00999999999999999\\
105.01	0.00999999999999999\\
106.01	0.00999999999999999\\
107.01	0.00999999999999999\\
108.01	0.00999999999999999\\
109.01	0.00999999999999999\\
110.01	0.00999999999999999\\
111.01	0.00999999999999999\\
112.01	0.00999999999999999\\
113.01	0.00999999999999999\\
114.01	0.00999999999999999\\
115.01	0.00999999999999999\\
116.01	0.00999999999999999\\
117.01	0.00999999999999999\\
118.01	0.00999999999999999\\
119.01	0.00999999999999999\\
120.01	0.00999999999999999\\
121.01	0.00999999999999999\\
122.01	0.00999999999999999\\
123.01	0.00999999999999999\\
124.01	0.00999999999999999\\
125.01	0.00999999999999999\\
126.01	0.00999999999999999\\
127.01	0.00999999999999999\\
128.01	0.00999999999999999\\
129.01	0.00999999999999999\\
130.01	0.00999999999999999\\
131.01	0.00999999999999999\\
132.01	0.00999999999999999\\
133.01	0.00999999999999999\\
134.01	0.00999999999999999\\
135.01	0.00999999999999999\\
136.01	0.00999999999999999\\
137.01	0.00999999999999999\\
138.01	0.00999999999999999\\
139.01	0.00999999999999999\\
140.01	0.00999999999999999\\
141.01	0.00999999999999999\\
142.01	0.00999999999999999\\
143.01	0.00999999999999999\\
144.01	0.00999999999999999\\
145.01	0.00999999999999999\\
146.01	0.00999999999999999\\
147.01	0.00999999999999999\\
148.01	0.00999999999999999\\
149.01	0.00999999999999999\\
150.01	0.00999999999999999\\
151.01	0.00999999999999999\\
152.01	0.00999999999999999\\
153.01	0.00999999999999999\\
154.01	0.00999999999999999\\
155.01	0.00999999999999999\\
156.01	0.00999999999999999\\
157.01	0.00999999999999999\\
158.01	0.00999999999999999\\
159.01	0.00999999999999999\\
160.01	0.00999999999999999\\
161.01	0.00999999999999999\\
162.01	0.00999999999999999\\
163.01	0.00999999999999999\\
164.01	0.00999999999999999\\
165.01	0.00999999999999999\\
166.01	0.00999999999999999\\
167.01	0.00999999999999999\\
168.01	0.00999999999999999\\
169.01	0.00999999999999999\\
170.01	0.00999999999999999\\
171.01	0.00999999999999999\\
172.01	0.00999999999999999\\
173.01	0.00999999999999999\\
174.01	0.00999999999999999\\
175.01	0.00999999999999999\\
176.01	0.00999999999999999\\
177.01	0.00999999999999999\\
178.01	0.00999999999999999\\
179.01	0.00999999999999999\\
180.01	0.00999999999999999\\
181.01	0.00999999999999999\\
182.01	0.00999999999999999\\
183.01	0.00999999999999999\\
184.01	0.00999999999999999\\
185.01	0.00999999999999999\\
186.01	0.00999999999999999\\
187.01	0.00999999999999999\\
188.01	0.00999999999999999\\
189.01	0.00999999999999999\\
190.01	0.00999999999999999\\
191.01	0.00999999999999999\\
192.01	0.00999999999999999\\
193.01	0.00999999999999999\\
194.01	0.00999999999999999\\
195.01	0.00999999999999999\\
196.01	0.00999999999999999\\
197.01	0.00999999999999999\\
198.01	0.00999999999999999\\
199.01	0.00999999999999999\\
200.01	0.00999999999999999\\
201.01	0.00999999999999999\\
202.01	0.00999999999999999\\
203.01	0.00999999999999999\\
204.01	0.00999999999999999\\
205.01	0.00999999999999999\\
206.01	0.00999999999999999\\
207.01	0.00999999999999999\\
208.01	0.00999999999999999\\
209.01	0.00999999999999999\\
210.01	0.00999999999999999\\
211.01	0.00999999999999999\\
212.01	0.00999999999999999\\
213.01	0.00999999999999999\\
214.01	0.00999999999999999\\
215.01	0.00999999999999999\\
216.01	0.00999999999999999\\
217.01	0.00999999999999999\\
218.01	0.00999999999999999\\
219.01	0.00999999999999999\\
220.01	0.00999999999999999\\
221.01	0.00999999999999999\\
222.01	0.00999999999999999\\
223.01	0.00999999999999999\\
224.01	0.00999999999999999\\
225.01	0.00999999999999999\\
226.01	0.00999999999999999\\
227.01	0.00999999999999999\\
228.01	0.00999999999999999\\
229.01	0.00999999999999999\\
230.01	0.00999999999999999\\
231.01	0.00999999999999999\\
232.01	0.00999999999999999\\
233.01	0.00999999999999999\\
234.01	0.00999999999999999\\
235.01	0.00999999999999999\\
236.01	0.00999999999999999\\
237.01	0.00999999999999999\\
238.01	0.00999999999999999\\
239.01	0.00999999999999999\\
240.01	0.00999999999999999\\
241.01	0.00999999999999999\\
242.01	0.00999999999999999\\
243.01	0.00999999999999999\\
244.01	0.00999999999999999\\
245.01	0.00999999999999999\\
246.01	0.00999999999999999\\
247.01	0.00999999999999999\\
248.01	0.00999999999999999\\
249.01	0.00999999999999999\\
250.01	0.00999999999999999\\
251.01	0.00999999999999999\\
252.01	0.00999999999999999\\
253.01	0.00999999999999999\\
254.01	0.00999999999999999\\
255.01	0.00999999999999999\\
256.01	0.00999999999999999\\
257.01	0.00999999999999999\\
258.01	0.00999999999999999\\
259.01	0.00999999999999999\\
260.01	0.00999999999999999\\
261.01	0.00999999999999999\\
262.01	0.00999999999999999\\
263.01	0.00999999999999999\\
264.01	0.00999999999999999\\
265.01	0.00999999999999999\\
266.01	0.00999999999999999\\
267.01	0.00999999999999999\\
268.01	0.00999999999999999\\
269.01	0.00999999999999999\\
270.01	0.00999999999999999\\
271.01	0.00999999999999999\\
272.01	0.00999999999999999\\
273.01	0.00999999999999999\\
274.01	0.00999999999999999\\
275.01	0.00999999999999999\\
276.01	0.00999999999999999\\
277.01	0.00999999999999999\\
278.01	0.00999999999999999\\
279.01	0.00999999999999999\\
280.01	0.00999999999999999\\
281.01	0.00999999999999999\\
282.01	0.00999999999999999\\
283.01	0.00999999999999999\\
284.01	0.00999999999999999\\
285.01	0.00999999999999999\\
286.01	0.00999999999999999\\
287.01	0.00999999999999999\\
288.01	0.00999999999999999\\
289.01	0.00999999999999999\\
290.01	0.00999999999999999\\
291.01	0.00999999999999999\\
292.01	0.00999999999999999\\
293.01	0.00999999999999999\\
294.01	0.00999999999999999\\
295.01	0.00999999999999999\\
296.01	0.00999999999999999\\
297.01	0.00999999999999999\\
298.01	0.00999999999999999\\
299.01	0.00999999999999999\\
300.01	0.00999999999999999\\
301.01	0.00999999999999999\\
302.01	0.00999999999999999\\
303.01	0.00999999999999999\\
304.01	0.00999999999999999\\
305.01	0.00999999999999999\\
306.01	0.00999999999999999\\
307.01	0.00999999999999999\\
308.01	0.00999999999999999\\
309.01	0.00999999999999999\\
310.01	0.00999999999999999\\
311.01	0.00999999999999999\\
312.01	0.00999999999999999\\
313.01	0.00999999999999999\\
314.01	0.00999999999999999\\
315.01	0.00999999999999999\\
316.01	0.00999999999999999\\
317.01	0.00999999999999999\\
318.01	0.00999999999999999\\
319.01	0.00999999999999999\\
320.01	0.00999999999999999\\
321.01	0.00999999999999999\\
322.01	0.00999999999999999\\
323.01	0.00999999999999999\\
324.01	0.00999999999999999\\
325.01	0.00999999999999999\\
326.01	0.00999999999999999\\
327.01	0.00999999999999999\\
328.01	0.00999999999999999\\
329.01	0.00999999999999999\\
330.01	0.00999999999999999\\
331.01	0.00999999999999999\\
332.01	0.00999999999999999\\
333.01	0.00999999999999999\\
334.01	0.00999999999999999\\
335.01	0.00999999999999999\\
336.01	0.00999999999999999\\
337.01	0.00999999999999999\\
338.01	0.00999999999999999\\
339.01	0.00999999999999999\\
340.01	0.00999999999999999\\
341.01	0.00999999999999999\\
342.01	0.00999999999999999\\
343.01	0.00999999999999999\\
344.01	0.00999999999999999\\
345.01	0.00999999999999999\\
346.01	0.00999999999999999\\
347.01	0.00999999999999999\\
348.01	0.00999999999999999\\
349.01	0.00999999999999999\\
350.01	0.00999999999999999\\
351.01	0.00999999999999999\\
352.01	0.00999999999999999\\
353.01	0.00999999999999999\\
354.01	0.00999999999999999\\
355.01	0.00999999999999999\\
356.01	0.00999999999999999\\
357.01	0.00999999999999999\\
358.01	0.00999999999999999\\
359.01	0.00999999999999999\\
360.01	0.00999999999999999\\
361.01	0.00999999999999999\\
362.01	0.00999999999999999\\
363.01	0.00999999999999999\\
364.01	0.00999999999999999\\
365.01	0.00999999999999999\\
366.01	0.00999999999999999\\
367.01	0.00999999999999999\\
368.01	0.00999999999999999\\
369.01	0.00999999999999999\\
370.01	0.00999999999999999\\
371.01	0.00999999999999999\\
372.01	0.00999999999999999\\
373.01	0.00999999999999999\\
374.01	0.00999999999999999\\
375.01	0.00999999999999999\\
376.01	0.00999999999999999\\
377.01	0.00999999999999999\\
378.01	0.00999999999999999\\
379.01	0.00999999999999999\\
380.01	0.00999999999999999\\
381.01	0.00999999999999999\\
382.01	0.00999999999999999\\
383.01	0.00999999999999999\\
384.01	0.00999999999999999\\
385.01	0.00999999999999999\\
386.01	0.00999999999999999\\
387.01	0.00999999999999999\\
388.01	0.00999999999999999\\
389.01	0.00999999999999999\\
390.01	0.00999999999999999\\
391.01	0.00999999999999999\\
392.01	0.00999999999999999\\
393.01	0.00999999999999999\\
394.01	0.00999999999999999\\
395.01	0.00999999999999999\\
396.01	0.00999999999999999\\
397.01	0.00999999999999999\\
398.01	0.00999999999999999\\
399.01	0.00999999999999999\\
400.01	0.00999999999999999\\
401.01	0.00999999999999999\\
402.01	0.00999999999999999\\
403.01	0.00999999999999999\\
404.01	0.00999999999999999\\
405.01	0.00999999999999999\\
406.01	0.00999999999999999\\
407.01	0.00999999999999999\\
408.01	0.00999999999999999\\
409.01	0.00999999999999999\\
410.01	0.00999999999999999\\
411.01	0.00999999999999999\\
412.01	0.00999999999999999\\
413.01	0.00999999999999999\\
414.01	0.00999999999999999\\
415.01	0.00999999999999999\\
416.01	0.00999999999999999\\
417.01	0.00999999999999999\\
418.01	0.00999999999999999\\
419.01	0.00999999999999999\\
420.01	0.00999999999999999\\
421.01	0.00999999999999999\\
422.01	0.00999999999999999\\
423.01	0.00999999999999999\\
424.01	0.00999999999999999\\
425.01	0.00999999999999999\\
426.01	0.00999999999999999\\
427.01	0.00999999999999999\\
428.01	0.00999999999999999\\
429.01	0.00999999999999999\\
430.01	0.00999999999999999\\
431.01	0.00999999999999999\\
432.01	0.00999999999999999\\
433.01	0.00999999999999999\\
434.01	0.00999999999999999\\
435.01	0.00999999999999999\\
436.01	0.00999999999999999\\
437.01	0.00999999999999999\\
438.01	0.00999999999999999\\
439.01	0.00999999999999999\\
440.01	0.00999999999999999\\
441.01	0.00999999999999999\\
442.01	0.00999999999999999\\
443.01	0.00999999999999999\\
444.01	0.00999999999999999\\
445.01	0.00999999999999999\\
446.01	0.00999999999999999\\
447.01	0.00999999999999999\\
448.01	0.00999999999999999\\
449.01	0.00999999999999999\\
450.01	0.00999999999999999\\
451.01	0.00999999999999999\\
452.01	0.00999999999999999\\
453.01	0.00999999999999999\\
454.01	0.00999999999999999\\
455.01	0.00999999999999999\\
456.01	0.00999999999999999\\
457.01	0.00999999999999999\\
458.01	0.00999999999999999\\
459.01	0.00999999999999999\\
460.01	0.00999999999999999\\
461.01	0.00999999999999999\\
462.01	0.00999999999999999\\
463.01	0.00999999999999999\\
464.01	0.00999999999999999\\
465.01	0.00999999999999999\\
466.01	0.00999999999999999\\
467.01	0.00999999999999999\\
468.01	0.00999999999999999\\
469.01	0.00999999999999999\\
470.01	0.00999999999999999\\
471.01	0.00999999999999999\\
472.01	0.00999999999999999\\
473.01	0.00999999999999999\\
474.01	0.00999999999999999\\
475.01	0.00999999999999999\\
476.01	0.00999999999999999\\
477.01	0.00999999999999999\\
478.01	0.00999999999999999\\
479.01	0.00999999999999999\\
480.01	0.00999999999999999\\
481.01	0.00999999999999999\\
482.01	0.00999999999999999\\
483.01	0.00999999999999999\\
484.01	0.00999999999999999\\
485.01	0.00999999999999999\\
486.01	0.00999999999999999\\
487.01	0.00999999999999999\\
488.01	0.00999999999999999\\
489.01	0.00999999999999999\\
490.01	0.00999999999999999\\
491.01	0.00999999999999999\\
492.01	0.00999999999999999\\
493.01	0.00999999999999999\\
494.01	0.00999999999999999\\
495.01	0.00999999999999999\\
496.01	0.00999999999999999\\
497.01	0.00999999999999999\\
498.01	0.00999999999999999\\
499.01	0.00999999999999999\\
500.01	0.00999999999999999\\
501.01	0.00999999999999999\\
502.01	0.00999999999999999\\
503.01	0.00999999999999999\\
504.01	0.00999999999999999\\
505.01	0.00999999999999999\\
506.01	0.00999999999999999\\
507.01	0.00999999999999999\\
508.01	0.00999999999999999\\
509.01	0.00999999999999999\\
510.01	0.00999999999999999\\
511.01	0.00999999999999999\\
512.01	0.00999999999999999\\
513.01	0.00999999999999999\\
514.01	0.00999999999999999\\
515.01	0.00999999999999999\\
516.01	0.00999999999999999\\
517.01	0.00999999999999999\\
518.01	0.00999999999999999\\
519.01	0.00999999999999999\\
520.01	0.00999999999999999\\
521.01	0.00999999999999999\\
522.01	0.00999999999999999\\
523.01	0.00999999999999999\\
524.01	0.00999999999999999\\
525.01	0.00999999999999999\\
526.01	0.00999999999999999\\
527.01	0.00999999999999999\\
528.01	0.00999999999999999\\
529.01	0.00999999999999999\\
530.01	0.00999999999999999\\
531.01	0.00999999999999999\\
532.01	0.00999999999999999\\
533.01	0.00999999999999999\\
534.01	0.00999999999999999\\
535.01	0.00999999999999999\\
536.01	0.00999999999999999\\
537.01	0.00999999999999999\\
538.01	0.00999999999999999\\
539.01	0.00999999999999999\\
540.01	0.00999999999999999\\
541.01	0.00999999999999999\\
542.01	0.00999999999999999\\
543.01	0.00999999999999999\\
544.01	0.00999999999999999\\
545.01	0.00999999999999999\\
546.01	0.00999999999999999\\
547.01	0.00999999999999999\\
548.01	0.00999999999999999\\
549.01	0.00999999999999999\\
550.01	0.00999999999999999\\
551.01	0.00999999999999999\\
552.01	0.00999999999999999\\
553.01	0.00999999999999999\\
554.01	0.00999999999999999\\
555.01	0.00999999999999999\\
556.01	0.00999999999999999\\
557.01	0.00999999999999999\\
558.01	0.00999999999999999\\
559.01	0.00999999999999999\\
560.01	0.00999999999999999\\
561.01	0.00999999999999999\\
562.01	0.00999999999999999\\
563.01	0.00999999999999999\\
564.01	0.00999999999999999\\
565.01	0.00999999999999999\\
566.01	0.00999999999999999\\
567.01	0.00999999999999999\\
568.01	0.00999999999999999\\
569.01	0.00999999999999999\\
570.01	0.00999999999999999\\
571.01	0.00999999999999999\\
572.01	0.00999999999999999\\
573.01	0.00999999999999999\\
574.01	0.00999999999999999\\
575.01	0.00999999999999999\\
576.01	0.00999999999999999\\
577.01	0.00999999999999999\\
578.01	0.00999999999999999\\
579.01	0.00999999999999999\\
580.01	0.00999999999999999\\
581.01	0.00999999999999999\\
582.01	0.00999999999999999\\
583.01	0.00999999999999999\\
584.01	0.00999999999999999\\
585.01	0.00999999999999999\\
586.01	0.00999999999999999\\
587.01	0.00999999999999999\\
588.01	0.00999999999999999\\
589.01	0.00999999999999999\\
590.01	0.00999999999999999\\
591.01	0.00999999999999999\\
592.01	0.00999999999999999\\
593.01	0.00999999999999999\\
594.01	0.00999999999999999\\
595.01	0.00999999999999999\\
596.01	0.00999999999999999\\
597.01	0.00999999999999999\\
598.01	0.00856167931836022\\
599.01	0.00614338742788857\\
599.02	0.00610585181165224\\
599.03	0.00606795853991343\\
599.04	0.00602970415860345\\
599.05	0.00599108518031535\\
599.06	0.00595209808398249\\
599.07	0.00591273931455391\\
599.08	0.00587300528266657\\
599.09	0.00583289236431447\\
599.1	0.00579239690051461\\
599.11	0.00575151519696963\\
599.12	0.00571024352372728\\
599.13	0.0056685781148366\\
599.14	0.00562651516800077\\
599.15	0.00558405084422662\\
599.16	0.00554118126747075\\
599.17	0.00549790252428228\\
599.18	0.00545421066344212\\
599.19	0.00541010169559877\\
599.2	0.00536557159290064\\
599.21	0.00532061628862479\\
599.22	0.0052752316768021\\
599.23	0.00522941361183888\\
599.24	0.00518315790813472\\
599.25	0.00513646033969682\\
599.26	0.0050893166504472\\
599.27	0.00504172255041643\\
599.28	0.00499367370841304\\
599.29	0.00494516575162778\\
599.3	0.00489619426523419\\
599.31	0.0048467547919852\\
599.32	0.00479684283180595\\
599.33	0.00474645384138272\\
599.34	0.00469558323374783\\
599.35	0.00464422637786066\\
599.36	0.00459237859818455\\
599.37	0.00454003517425979\\
599.38	0.00448719134027231\\
599.39	0.00443384228461844\\
599.4	0.00437998314946538\\
599.41	0.00432560903030746\\
599.42	0.0042707149755182\\
599.43	0.00421529598589801\\
599.44	0.00415934701421762\\
599.45	0.00410286296475708\\
599.46	0.00404583869284036\\
599.47	0.00398826900436549\\
599.48	0.00393014865533018\\
599.49	0.00387147235135293\\
599.5	0.0038122347471895\\
599.51	0.00375243044624479\\
599.52	0.00369205400007998\\
599.53	0.00363109990791502\\
599.54	0.00356956261612632\\
599.55	0.00350743651773957\\
599.56	0.0034447159519178\\
599.57	0.00338139520344443\\
599.58	0.00331746850220145\\
599.59	0.00325293002264255\\
599.6	0.00318777388326121\\
599.61	0.00312199414605369\\
599.62	0.00305558481597687\\
599.63	0.00298853984040088\\
599.64	0.00292085310855656\\
599.65	0.00285251845097748\\
599.66	0.00278352963893677\\
599.67	0.00271388038387843\\
599.68	0.00264356433684325\\
599.69	0.00257257508788922\\
599.7	0.00250090616550638\\
599.71	0.00242855103602601\\
599.72	0.0023555031030243\\
599.73	0.00228175570672012\\
599.74	0.00220730212336716\\
599.75	0.00213213556464018\\
599.76	0.0020562491770154\\
599.77	0.00197963604114491\\
599.78	0.00190228917122515\\
599.79	0.0018242015143593\\
599.8	0.00174536594991352\\
599.81	0.00166577528886715\\
599.82	0.00158542227315651\\
599.83	0.00150429957501254\\
599.84	0.00142239979629203\\
599.85	0.00133971546780248\\
599.86	0.00125623904862046\\
599.87	0.00117196292540346\\
599.88	0.0010868794116951\\
599.89	0.00100098074722375\\
599.9	0.000914259097194382\\
599.91	0.000826706551573646\\
599.92	0.000738315124368095\\
599.93	0.000649076752895513\\
599.94	0.000558983297049216\\
599.95	0.000468026538555355\\
599.96	0.000376198180223067\\
599.97	0.000283489845187453\\
599.98	0.0001898930761453\\
599.99	9.53993345834923e-05\\
600	0\\
};
\addplot [color=red!40!mycolor19,solid,forget plot]
  table[row sep=crcr]{%
0.01	0.00999999999999999\\
1.01	0.00999999999999999\\
2.01	0.00999999999999999\\
3.01	0.00999999999999999\\
4.01	0.00999999999999999\\
5.01	0.00999999999999999\\
6.01	0.00999999999999999\\
7.01	0.00999999999999999\\
8.01	0.00999999999999999\\
9.01	0.00999999999999999\\
10.01	0.00999999999999999\\
11.01	0.00999999999999999\\
12.01	0.00999999999999999\\
13.01	0.00999999999999999\\
14.01	0.00999999999999999\\
15.01	0.00999999999999999\\
16.01	0.00999999999999999\\
17.01	0.00999999999999999\\
18.01	0.00999999999999999\\
19.01	0.00999999999999999\\
20.01	0.00999999999999999\\
21.01	0.00999999999999999\\
22.01	0.00999999999999999\\
23.01	0.00999999999999999\\
24.01	0.00999999999999999\\
25.01	0.00999999999999999\\
26.01	0.00999999999999999\\
27.01	0.00999999999999999\\
28.01	0.00999999999999999\\
29.01	0.00999999999999999\\
30.01	0.00999999999999999\\
31.01	0.00999999999999999\\
32.01	0.00999999999999999\\
33.01	0.00999999999999999\\
34.01	0.00999999999999999\\
35.01	0.00999999999999999\\
36.01	0.00999999999999999\\
37.01	0.00999999999999999\\
38.01	0.00999999999999999\\
39.01	0.00999999999999999\\
40.01	0.00999999999999999\\
41.01	0.00999999999999999\\
42.01	0.00999999999999999\\
43.01	0.00999999999999999\\
44.01	0.00999999999999999\\
45.01	0.00999999999999999\\
46.01	0.00999999999999999\\
47.01	0.00999999999999999\\
48.01	0.00999999999999999\\
49.01	0.00999999999999999\\
50.01	0.00999999999999999\\
51.01	0.00999999999999999\\
52.01	0.00999999999999999\\
53.01	0.00999999999999999\\
54.01	0.00999999999999999\\
55.01	0.00999999999999999\\
56.01	0.00999999999999999\\
57.01	0.00999999999999999\\
58.01	0.00999999999999999\\
59.01	0.00999999999999999\\
60.01	0.00999999999999999\\
61.01	0.00999999999999999\\
62.01	0.00999999999999999\\
63.01	0.00999999999999999\\
64.01	0.00999999999999999\\
65.01	0.00999999999999999\\
66.01	0.00999999999999999\\
67.01	0.00999999999999999\\
68.01	0.00999999999999999\\
69.01	0.00999999999999999\\
70.01	0.00999999999999999\\
71.01	0.00999999999999999\\
72.01	0.00999999999999999\\
73.01	0.00999999999999999\\
74.01	0.00999999999999999\\
75.01	0.00999999999999999\\
76.01	0.00999999999999999\\
77.01	0.00999999999999999\\
78.01	0.00999999999999999\\
79.01	0.00999999999999999\\
80.01	0.00999999999999999\\
81.01	0.00999999999999999\\
82.01	0.00999999999999999\\
83.01	0.00999999999999999\\
84.01	0.00999999999999999\\
85.01	0.00999999999999999\\
86.01	0.00999999999999999\\
87.01	0.00999999999999999\\
88.01	0.00999999999999999\\
89.01	0.00999999999999999\\
90.01	0.00999999999999999\\
91.01	0.00999999999999999\\
92.01	0.00999999999999999\\
93.01	0.00999999999999999\\
94.01	0.00999999999999999\\
95.01	0.00999999999999999\\
96.01	0.00999999999999999\\
97.01	0.00999999999999999\\
98.01	0.00999999999999999\\
99.01	0.00999999999999999\\
100.01	0.00999999999999999\\
101.01	0.00999999999999999\\
102.01	0.00999999999999999\\
103.01	0.00999999999999999\\
104.01	0.00999999999999999\\
105.01	0.00999999999999999\\
106.01	0.00999999999999999\\
107.01	0.00999999999999999\\
108.01	0.00999999999999999\\
109.01	0.00999999999999999\\
110.01	0.00999999999999999\\
111.01	0.00999999999999999\\
112.01	0.00999999999999999\\
113.01	0.00999999999999999\\
114.01	0.00999999999999999\\
115.01	0.00999999999999999\\
116.01	0.00999999999999999\\
117.01	0.00999999999999999\\
118.01	0.00999999999999999\\
119.01	0.00999999999999999\\
120.01	0.00999999999999999\\
121.01	0.00999999999999999\\
122.01	0.00999999999999999\\
123.01	0.00999999999999999\\
124.01	0.00999999999999999\\
125.01	0.00999999999999999\\
126.01	0.00999999999999999\\
127.01	0.00999999999999999\\
128.01	0.00999999999999999\\
129.01	0.00999999999999999\\
130.01	0.00999999999999999\\
131.01	0.00999999999999999\\
132.01	0.00999999999999999\\
133.01	0.00999999999999999\\
134.01	0.00999999999999999\\
135.01	0.00999999999999999\\
136.01	0.00999999999999999\\
137.01	0.00999999999999999\\
138.01	0.00999999999999999\\
139.01	0.00999999999999999\\
140.01	0.00999999999999999\\
141.01	0.00999999999999999\\
142.01	0.00999999999999999\\
143.01	0.00999999999999999\\
144.01	0.00999999999999999\\
145.01	0.00999999999999999\\
146.01	0.00999999999999999\\
147.01	0.00999999999999999\\
148.01	0.00999999999999999\\
149.01	0.00999999999999999\\
150.01	0.00999999999999999\\
151.01	0.00999999999999999\\
152.01	0.00999999999999999\\
153.01	0.00999999999999999\\
154.01	0.00999999999999999\\
155.01	0.00999999999999999\\
156.01	0.00999999999999999\\
157.01	0.00999999999999999\\
158.01	0.00999999999999999\\
159.01	0.00999999999999999\\
160.01	0.00999999999999999\\
161.01	0.00999999999999999\\
162.01	0.00999999999999999\\
163.01	0.00999999999999999\\
164.01	0.00999999999999999\\
165.01	0.00999999999999999\\
166.01	0.00999999999999999\\
167.01	0.00999999999999999\\
168.01	0.00999999999999999\\
169.01	0.00999999999999999\\
170.01	0.00999999999999999\\
171.01	0.00999999999999999\\
172.01	0.00999999999999999\\
173.01	0.00999999999999999\\
174.01	0.00999999999999999\\
175.01	0.00999999999999999\\
176.01	0.00999999999999999\\
177.01	0.00999999999999999\\
178.01	0.00999999999999999\\
179.01	0.00999999999999999\\
180.01	0.00999999999999999\\
181.01	0.00999999999999999\\
182.01	0.00999999999999999\\
183.01	0.00999999999999999\\
184.01	0.00999999999999999\\
185.01	0.00999999999999999\\
186.01	0.00999999999999999\\
187.01	0.00999999999999999\\
188.01	0.00999999999999999\\
189.01	0.00999999999999999\\
190.01	0.00999999999999999\\
191.01	0.00999999999999999\\
192.01	0.00999999999999999\\
193.01	0.00999999999999999\\
194.01	0.00999999999999999\\
195.01	0.00999999999999999\\
196.01	0.00999999999999999\\
197.01	0.00999999999999999\\
198.01	0.00999999999999999\\
199.01	0.00999999999999999\\
200.01	0.00999999999999999\\
201.01	0.00999999999999999\\
202.01	0.00999999999999999\\
203.01	0.00999999999999999\\
204.01	0.00999999999999999\\
205.01	0.00999999999999999\\
206.01	0.00999999999999999\\
207.01	0.00999999999999999\\
208.01	0.00999999999999999\\
209.01	0.00999999999999999\\
210.01	0.00999999999999999\\
211.01	0.00999999999999999\\
212.01	0.00999999999999999\\
213.01	0.00999999999999999\\
214.01	0.00999999999999999\\
215.01	0.00999999999999999\\
216.01	0.00999999999999999\\
217.01	0.00999999999999999\\
218.01	0.00999999999999999\\
219.01	0.00999999999999999\\
220.01	0.00999999999999999\\
221.01	0.00999999999999999\\
222.01	0.00999999999999999\\
223.01	0.00999999999999999\\
224.01	0.00999999999999999\\
225.01	0.00999999999999999\\
226.01	0.00999999999999999\\
227.01	0.00999999999999999\\
228.01	0.00999999999999999\\
229.01	0.00999999999999999\\
230.01	0.00999999999999999\\
231.01	0.00999999999999999\\
232.01	0.00999999999999999\\
233.01	0.00999999999999999\\
234.01	0.00999999999999999\\
235.01	0.00999999999999999\\
236.01	0.00999999999999999\\
237.01	0.00999999999999999\\
238.01	0.00999999999999999\\
239.01	0.00999999999999999\\
240.01	0.00999999999999999\\
241.01	0.00999999999999999\\
242.01	0.00999999999999999\\
243.01	0.00999999999999999\\
244.01	0.00999999999999999\\
245.01	0.00999999999999999\\
246.01	0.00999999999999999\\
247.01	0.00999999999999999\\
248.01	0.00999999999999999\\
249.01	0.00999999999999999\\
250.01	0.00999999999999999\\
251.01	0.00999999999999999\\
252.01	0.00999999999999999\\
253.01	0.00999999999999999\\
254.01	0.00999999999999999\\
255.01	0.00999999999999999\\
256.01	0.00999999999999999\\
257.01	0.00999999999999999\\
258.01	0.00999999999999999\\
259.01	0.00999999999999999\\
260.01	0.00999999999999999\\
261.01	0.00999999999999999\\
262.01	0.00999999999999999\\
263.01	0.00999999999999999\\
264.01	0.00999999999999999\\
265.01	0.00999999999999999\\
266.01	0.00999999999999999\\
267.01	0.00999999999999999\\
268.01	0.00999999999999999\\
269.01	0.00999999999999999\\
270.01	0.00999999999999999\\
271.01	0.00999999999999999\\
272.01	0.00999999999999999\\
273.01	0.00999999999999999\\
274.01	0.00999999999999999\\
275.01	0.00999999999999999\\
276.01	0.00999999999999999\\
277.01	0.00999999999999999\\
278.01	0.00999999999999999\\
279.01	0.00999999999999999\\
280.01	0.00999999999999999\\
281.01	0.00999999999999999\\
282.01	0.00999999999999999\\
283.01	0.00999999999999999\\
284.01	0.00999999999999999\\
285.01	0.00999999999999999\\
286.01	0.00999999999999999\\
287.01	0.00999999999999999\\
288.01	0.00999999999999999\\
289.01	0.00999999999999999\\
290.01	0.00999999999999999\\
291.01	0.00999999999999999\\
292.01	0.00999999999999999\\
293.01	0.00999999999999999\\
294.01	0.00999999999999999\\
295.01	0.00999999999999999\\
296.01	0.00999999999999999\\
297.01	0.00999999999999999\\
298.01	0.00999999999999999\\
299.01	0.00999999999999999\\
300.01	0.00999999999999999\\
301.01	0.00999999999999999\\
302.01	0.00999999999999999\\
303.01	0.00999999999999999\\
304.01	0.00999999999999999\\
305.01	0.00999999999999999\\
306.01	0.00999999999999999\\
307.01	0.00999999999999999\\
308.01	0.00999999999999999\\
309.01	0.00999999999999999\\
310.01	0.00999999999999999\\
311.01	0.00999999999999999\\
312.01	0.00999999999999999\\
313.01	0.00999999999999999\\
314.01	0.00999999999999999\\
315.01	0.00999999999999999\\
316.01	0.00999999999999999\\
317.01	0.00999999999999999\\
318.01	0.00999999999999999\\
319.01	0.00999999999999999\\
320.01	0.00999999999999999\\
321.01	0.00999999999999999\\
322.01	0.00999999999999999\\
323.01	0.00999999999999999\\
324.01	0.00999999999999999\\
325.01	0.00999999999999999\\
326.01	0.00999999999999999\\
327.01	0.00999999999999999\\
328.01	0.00999999999999999\\
329.01	0.00999999999999999\\
330.01	0.00999999999999999\\
331.01	0.00999999999999999\\
332.01	0.00999999999999999\\
333.01	0.00999999999999999\\
334.01	0.00999999999999999\\
335.01	0.00999999999999999\\
336.01	0.00999999999999999\\
337.01	0.00999999999999999\\
338.01	0.00999999999999999\\
339.01	0.00999999999999999\\
340.01	0.00999999999999999\\
341.01	0.00999999999999999\\
342.01	0.00999999999999999\\
343.01	0.00999999999999999\\
344.01	0.00999999999999999\\
345.01	0.00999999999999999\\
346.01	0.00999999999999999\\
347.01	0.00999999999999999\\
348.01	0.00999999999999999\\
349.01	0.00999999999999999\\
350.01	0.00999999999999999\\
351.01	0.00999999999999999\\
352.01	0.00999999999999999\\
353.01	0.00999999999999999\\
354.01	0.00999999999999999\\
355.01	0.00999999999999999\\
356.01	0.00999999999999999\\
357.01	0.00999999999999999\\
358.01	0.00999999999999999\\
359.01	0.00999999999999999\\
360.01	0.00999999999999999\\
361.01	0.00999999999999999\\
362.01	0.00999999999999999\\
363.01	0.00999999999999999\\
364.01	0.00999999999999999\\
365.01	0.00999999999999999\\
366.01	0.00999999999999999\\
367.01	0.00999999999999999\\
368.01	0.00999999999999999\\
369.01	0.00999999999999999\\
370.01	0.00999999999999999\\
371.01	0.00999999999999999\\
372.01	0.00999999999999999\\
373.01	0.00999999999999999\\
374.01	0.00999999999999999\\
375.01	0.00999999999999999\\
376.01	0.00999999999999999\\
377.01	0.00999999999999999\\
378.01	0.00999999999999999\\
379.01	0.00999999999999999\\
380.01	0.00999999999999999\\
381.01	0.00999999999999999\\
382.01	0.00999999999999999\\
383.01	0.00999999999999999\\
384.01	0.00999999999999999\\
385.01	0.00999999999999999\\
386.01	0.00999999999999999\\
387.01	0.00999999999999999\\
388.01	0.00999999999999999\\
389.01	0.00999999999999999\\
390.01	0.00999999999999999\\
391.01	0.00999999999999999\\
392.01	0.00999999999999999\\
393.01	0.00999999999999999\\
394.01	0.00999999999999999\\
395.01	0.00999999999999999\\
396.01	0.00999999999999999\\
397.01	0.00999999999999999\\
398.01	0.00999999999999999\\
399.01	0.00999999999999999\\
400.01	0.00999999999999999\\
401.01	0.00999999999999999\\
402.01	0.00999999999999999\\
403.01	0.00999999999999999\\
404.01	0.00999999999999999\\
405.01	0.00999999999999999\\
406.01	0.00999999999999999\\
407.01	0.00999999999999999\\
408.01	0.00999999999999999\\
409.01	0.00999999999999999\\
410.01	0.00999999999999999\\
411.01	0.00999999999999999\\
412.01	0.00999999999999999\\
413.01	0.00999999999999999\\
414.01	0.00999999999999999\\
415.01	0.00999999999999999\\
416.01	0.00999999999999999\\
417.01	0.00999999999999999\\
418.01	0.00999999999999999\\
419.01	0.00999999999999999\\
420.01	0.00999999999999999\\
421.01	0.00999999999999999\\
422.01	0.00999999999999999\\
423.01	0.00999999999999999\\
424.01	0.00999999999999999\\
425.01	0.00999999999999999\\
426.01	0.00999999999999999\\
427.01	0.00999999999999999\\
428.01	0.00999999999999999\\
429.01	0.00999999999999999\\
430.01	0.00999999999999999\\
431.01	0.00999999999999999\\
432.01	0.00999999999999999\\
433.01	0.00999999999999999\\
434.01	0.00999999999999999\\
435.01	0.00999999999999999\\
436.01	0.00999999999999999\\
437.01	0.00999999999999999\\
438.01	0.00999999999999999\\
439.01	0.00999999999999999\\
440.01	0.00999999999999999\\
441.01	0.00999999999999999\\
442.01	0.00999999999999999\\
443.01	0.00999999999999999\\
444.01	0.00999999999999999\\
445.01	0.00999999999999999\\
446.01	0.00999999999999999\\
447.01	0.00999999999999999\\
448.01	0.00999999999999999\\
449.01	0.00999999999999999\\
450.01	0.00999999999999999\\
451.01	0.00999999999999999\\
452.01	0.00999999999999999\\
453.01	0.00999999999999999\\
454.01	0.00999999999999999\\
455.01	0.00999999999999999\\
456.01	0.00999999999999999\\
457.01	0.00999999999999999\\
458.01	0.00999999999999999\\
459.01	0.00999999999999999\\
460.01	0.00999999999999999\\
461.01	0.00999999999999999\\
462.01	0.00999999999999999\\
463.01	0.00999999999999999\\
464.01	0.00999999999999999\\
465.01	0.00999999999999999\\
466.01	0.00999999999999999\\
467.01	0.00999999999999999\\
468.01	0.00999999999999999\\
469.01	0.00999999999999999\\
470.01	0.00999999999999999\\
471.01	0.00999999999999999\\
472.01	0.00999999999999999\\
473.01	0.00999999999999999\\
474.01	0.00999999999999999\\
475.01	0.00999999999999999\\
476.01	0.00999999999999999\\
477.01	0.00999999999999999\\
478.01	0.00999999999999999\\
479.01	0.00999999999999999\\
480.01	0.00999999999999999\\
481.01	0.00999999999999999\\
482.01	0.00999999999999999\\
483.01	0.00999999999999999\\
484.01	0.00999999999999999\\
485.01	0.00999999999999999\\
486.01	0.00999999999999999\\
487.01	0.00999999999999999\\
488.01	0.00999999999999999\\
489.01	0.00999999999999999\\
490.01	0.00999999999999999\\
491.01	0.00999999999999999\\
492.01	0.00999999999999999\\
493.01	0.00999999999999999\\
494.01	0.00999999999999999\\
495.01	0.00999999999999999\\
496.01	0.00999999999999999\\
497.01	0.00999999999999999\\
498.01	0.00999999999999999\\
499.01	0.00999999999999999\\
500.01	0.00999999999999999\\
501.01	0.00999999999999999\\
502.01	0.00999999999999999\\
503.01	0.00999999999999999\\
504.01	0.00999999999999999\\
505.01	0.00999999999999999\\
506.01	0.00999999999999999\\
507.01	0.00999999999999999\\
508.01	0.00999999999999999\\
509.01	0.00999999999999999\\
510.01	0.00999999999999999\\
511.01	0.00999999999999999\\
512.01	0.00999999999999999\\
513.01	0.00999999999999999\\
514.01	0.00999999999999999\\
515.01	0.00999999999999999\\
516.01	0.00999999999999999\\
517.01	0.00999999999999999\\
518.01	0.00999999999999999\\
519.01	0.00999999999999999\\
520.01	0.00999999999999999\\
521.01	0.00999999999999999\\
522.01	0.00999999999999999\\
523.01	0.00999999999999999\\
524.01	0.00999999999999999\\
525.01	0.00999999999999999\\
526.01	0.00999999999999999\\
527.01	0.00999999999999999\\
528.01	0.00999999999999999\\
529.01	0.00999999999999999\\
530.01	0.00999999999999999\\
531.01	0.00999999999999999\\
532.01	0.00999999999999999\\
533.01	0.00999999999999999\\
534.01	0.00999999999999999\\
535.01	0.00999999999999999\\
536.01	0.00999999999999999\\
537.01	0.00999999999999999\\
538.01	0.00999999999999999\\
539.01	0.00999999999999999\\
540.01	0.00999999999999999\\
541.01	0.00999999999999999\\
542.01	0.00999999999999999\\
543.01	0.00999999999999999\\
544.01	0.00999999999999999\\
545.01	0.00999999999999999\\
546.01	0.00999999999999999\\
547.01	0.00999999999999999\\
548.01	0.00999999999999999\\
549.01	0.00999999999999999\\
550.01	0.00999999999999999\\
551.01	0.00999999999999999\\
552.01	0.00999999999999999\\
553.01	0.00999999999999999\\
554.01	0.00999999999999999\\
555.01	0.00999999999999999\\
556.01	0.00999999999999999\\
557.01	0.00999999999999999\\
558.01	0.00999999999999999\\
559.01	0.00999999999999999\\
560.01	0.00999999999999999\\
561.01	0.00999999999999999\\
562.01	0.00999999999999999\\
563.01	0.00999999999999999\\
564.01	0.00999999999999999\\
565.01	0.00999999999999999\\
566.01	0.00999999999999999\\
567.01	0.00999999999999999\\
568.01	0.00999999999999999\\
569.01	0.00999999999999999\\
570.01	0.00999999999999999\\
571.01	0.00999999999999999\\
572.01	0.00999999999999999\\
573.01	0.00999999999999999\\
574.01	0.00999999999999999\\
575.01	0.00999999999999999\\
576.01	0.00999999999999999\\
577.01	0.00999999999999999\\
578.01	0.00999999999999999\\
579.01	0.00999999999999999\\
580.01	0.00999999999999999\\
581.01	0.00999999999999999\\
582.01	0.00999999999999999\\
583.01	0.00999999999999999\\
584.01	0.00999999999999999\\
585.01	0.00999999999999999\\
586.01	0.00999999999999999\\
587.01	0.00999999999999999\\
588.01	0.00999999999999999\\
589.01	0.00999999999999999\\
590.01	0.00999999999999999\\
591.01	0.00999999999999999\\
592.01	0.00999999999999999\\
593.01	0.00999999999999999\\
594.01	0.00999999999999999\\
595.01	0.00999999999999999\\
596.01	0.00999999999999999\\
597.01	0.00953004398302777\\
598.01	0.00856167931815354\\
599.01	0.00614338742788383\\
599.02	0.00610585181164776\\
599.03	0.00606795853990921\\
599.04	0.00602970415859947\\
599.05	0.0059910851803116\\
599.06	0.00595209808397897\\
599.07	0.00591273931455058\\
599.08	0.00587300528266344\\
599.09	0.00583289236431154\\
599.1	0.00579239690051186\\
599.11	0.00575151519696705\\
599.12	0.00571024352372486\\
599.13	0.00566857811483434\\
599.14	0.00562651516799866\\
599.15	0.00558405084422464\\
599.16	0.00554118126746891\\
599.17	0.00549790252428056\\
599.18	0.00545421066344051\\
599.19	0.00541010169559728\\
599.2	0.00536557159289926\\
599.21	0.00532061628862351\\
599.22	0.00527523167680091\\
599.23	0.00522941361183777\\
599.24	0.0051831579081337\\
599.25	0.00513646033969589\\
599.26	0.00508931665044635\\
599.27	0.00504172255041565\\
599.28	0.00499367370841231\\
599.29	0.00494516575162712\\
599.3	0.00489619426523358\\
599.31	0.00484675479198464\\
599.32	0.00479684283180545\\
599.33	0.00474645384138226\\
599.34	0.00469558323374742\\
599.35	0.00464422637786028\\
599.36	0.00459237859818421\\
599.37	0.00454003517425947\\
599.38	0.00448719134027202\\
599.39	0.00443384228461819\\
599.4	0.00437998314946515\\
599.41	0.00432560903030725\\
599.42	0.00427071497551801\\
599.43	0.00421529598589785\\
599.44	0.00415934701421748\\
599.45	0.00410286296475696\\
599.46	0.00404583869284026\\
599.47	0.0039882690043654\\
599.48	0.00393014865533011\\
599.49	0.00387147235135287\\
599.5	0.00381223474718945\\
599.51	0.00375243044624474\\
599.52	0.00369205400007994\\
599.53	0.00363109990791499\\
599.54	0.00356956261612629\\
599.55	0.00350743651773955\\
599.56	0.00344471595191778\\
599.57	0.00338139520344441\\
599.58	0.00331746850220144\\
599.59	0.00325293002264255\\
599.6	0.00318777388326122\\
599.61	0.0031219941460537\\
599.62	0.00305558481597687\\
599.63	0.00298853984040088\\
599.64	0.00292085310855656\\
599.65	0.00285251845097748\\
599.66	0.00278352963893676\\
599.67	0.00271388038387843\\
599.68	0.00264356433684325\\
599.69	0.00257257508788922\\
599.7	0.00250090616550637\\
599.71	0.002428551036026\\
599.72	0.00235550310302429\\
599.73	0.00228175570672011\\
599.74	0.00220730212336716\\
599.75	0.00213213556464018\\
599.76	0.0020562491770154\\
599.77	0.00197963604114491\\
599.78	0.00190228917122515\\
599.79	0.00182420151435929\\
599.8	0.00174536594991352\\
599.81	0.00166577528886715\\
599.82	0.00158542227315651\\
599.83	0.00150429957501254\\
599.84	0.00142239979629203\\
599.85	0.00133971546780248\\
599.86	0.00125623904862047\\
599.87	0.00117196292540346\\
599.88	0.0010868794116951\\
599.89	0.00100098074722375\\
599.9	0.000914259097194383\\
599.91	0.000826706551573648\\
599.92	0.000738315124368097\\
599.93	0.000649076752895513\\
599.94	0.000558983297049216\\
599.95	0.000468026538555355\\
599.96	0.000376198180223067\\
599.97	0.000283489845187451\\
599.98	0.000189893076145302\\
599.99	9.53993345834923e-05\\
600	0\\
};
\addplot [color=red!75!mycolor17,solid,forget plot]
  table[row sep=crcr]{%
0.01	0.00999999999999999\\
1.01	0.00999999999999999\\
2.01	0.00999999999999999\\
3.01	0.00999999999999999\\
4.01	0.00999999999999999\\
5.01	0.00999999999999999\\
6.01	0.00999999999999999\\
7.01	0.00999999999999999\\
8.01	0.00999999999999999\\
9.01	0.00999999999999999\\
10.01	0.00999999999999999\\
11.01	0.00999999999999999\\
12.01	0.00999999999999999\\
13.01	0.00999999999999999\\
14.01	0.00999999999999999\\
15.01	0.00999999999999999\\
16.01	0.00999999999999999\\
17.01	0.00999999999999999\\
18.01	0.00999999999999999\\
19.01	0.00999999999999999\\
20.01	0.00999999999999999\\
21.01	0.00999999999999999\\
22.01	0.00999999999999999\\
23.01	0.00999999999999999\\
24.01	0.00999999999999999\\
25.01	0.00999999999999999\\
26.01	0.00999999999999999\\
27.01	0.00999999999999999\\
28.01	0.00999999999999999\\
29.01	0.00999999999999999\\
30.01	0.00999999999999999\\
31.01	0.00999999999999999\\
32.01	0.00999999999999999\\
33.01	0.00999999999999999\\
34.01	0.00999999999999999\\
35.01	0.00999999999999999\\
36.01	0.00999999999999999\\
37.01	0.00999999999999999\\
38.01	0.00999999999999999\\
39.01	0.00999999999999999\\
40.01	0.00999999999999999\\
41.01	0.00999999999999999\\
42.01	0.00999999999999999\\
43.01	0.00999999999999999\\
44.01	0.00999999999999999\\
45.01	0.00999999999999999\\
46.01	0.00999999999999999\\
47.01	0.00999999999999999\\
48.01	0.00999999999999999\\
49.01	0.00999999999999999\\
50.01	0.00999999999999999\\
51.01	0.00999999999999999\\
52.01	0.00999999999999999\\
53.01	0.00999999999999999\\
54.01	0.00999999999999999\\
55.01	0.00999999999999999\\
56.01	0.00999999999999999\\
57.01	0.00999999999999999\\
58.01	0.00999999999999999\\
59.01	0.00999999999999999\\
60.01	0.00999999999999999\\
61.01	0.00999999999999999\\
62.01	0.00999999999999999\\
63.01	0.00999999999999999\\
64.01	0.00999999999999999\\
65.01	0.00999999999999999\\
66.01	0.00999999999999999\\
67.01	0.00999999999999999\\
68.01	0.00999999999999999\\
69.01	0.00999999999999999\\
70.01	0.00999999999999999\\
71.01	0.00999999999999999\\
72.01	0.00999999999999999\\
73.01	0.00999999999999999\\
74.01	0.00999999999999999\\
75.01	0.00999999999999999\\
76.01	0.00999999999999999\\
77.01	0.00999999999999999\\
78.01	0.00999999999999999\\
79.01	0.00999999999999999\\
80.01	0.00999999999999999\\
81.01	0.00999999999999999\\
82.01	0.00999999999999999\\
83.01	0.00999999999999999\\
84.01	0.00999999999999999\\
85.01	0.00999999999999999\\
86.01	0.00999999999999999\\
87.01	0.00999999999999999\\
88.01	0.00999999999999999\\
89.01	0.00999999999999999\\
90.01	0.00999999999999999\\
91.01	0.00999999999999999\\
92.01	0.00999999999999999\\
93.01	0.00999999999999999\\
94.01	0.00999999999999999\\
95.01	0.00999999999999999\\
96.01	0.00999999999999999\\
97.01	0.00999999999999999\\
98.01	0.00999999999999999\\
99.01	0.00999999999999999\\
100.01	0.00999999999999999\\
101.01	0.00999999999999999\\
102.01	0.00999999999999999\\
103.01	0.00999999999999999\\
104.01	0.00999999999999999\\
105.01	0.00999999999999999\\
106.01	0.00999999999999999\\
107.01	0.00999999999999999\\
108.01	0.00999999999999999\\
109.01	0.00999999999999999\\
110.01	0.00999999999999999\\
111.01	0.00999999999999999\\
112.01	0.00999999999999999\\
113.01	0.00999999999999999\\
114.01	0.00999999999999999\\
115.01	0.00999999999999999\\
116.01	0.00999999999999999\\
117.01	0.00999999999999999\\
118.01	0.00999999999999999\\
119.01	0.00999999999999999\\
120.01	0.00999999999999999\\
121.01	0.00999999999999999\\
122.01	0.00999999999999999\\
123.01	0.00999999999999999\\
124.01	0.00999999999999999\\
125.01	0.00999999999999999\\
126.01	0.00999999999999999\\
127.01	0.00999999999999999\\
128.01	0.00999999999999999\\
129.01	0.00999999999999999\\
130.01	0.00999999999999999\\
131.01	0.00999999999999999\\
132.01	0.00999999999999999\\
133.01	0.00999999999999999\\
134.01	0.00999999999999999\\
135.01	0.00999999999999999\\
136.01	0.00999999999999999\\
137.01	0.00999999999999999\\
138.01	0.00999999999999999\\
139.01	0.00999999999999999\\
140.01	0.00999999999999999\\
141.01	0.00999999999999999\\
142.01	0.00999999999999999\\
143.01	0.00999999999999999\\
144.01	0.00999999999999999\\
145.01	0.00999999999999999\\
146.01	0.00999999999999999\\
147.01	0.00999999999999999\\
148.01	0.00999999999999999\\
149.01	0.00999999999999999\\
150.01	0.00999999999999999\\
151.01	0.00999999999999999\\
152.01	0.00999999999999999\\
153.01	0.00999999999999999\\
154.01	0.00999999999999999\\
155.01	0.00999999999999999\\
156.01	0.00999999999999999\\
157.01	0.00999999999999999\\
158.01	0.00999999999999999\\
159.01	0.00999999999999999\\
160.01	0.00999999999999999\\
161.01	0.00999999999999999\\
162.01	0.00999999999999999\\
163.01	0.00999999999999999\\
164.01	0.00999999999999999\\
165.01	0.00999999999999999\\
166.01	0.00999999999999999\\
167.01	0.00999999999999999\\
168.01	0.00999999999999999\\
169.01	0.00999999999999999\\
170.01	0.00999999999999999\\
171.01	0.00999999999999999\\
172.01	0.00999999999999999\\
173.01	0.00999999999999999\\
174.01	0.00999999999999999\\
175.01	0.00999999999999999\\
176.01	0.00999999999999999\\
177.01	0.00999999999999999\\
178.01	0.00999999999999999\\
179.01	0.00999999999999999\\
180.01	0.00999999999999999\\
181.01	0.00999999999999999\\
182.01	0.00999999999999999\\
183.01	0.00999999999999999\\
184.01	0.00999999999999999\\
185.01	0.00999999999999999\\
186.01	0.00999999999999999\\
187.01	0.00999999999999999\\
188.01	0.00999999999999999\\
189.01	0.00999999999999999\\
190.01	0.00999999999999999\\
191.01	0.00999999999999999\\
192.01	0.00999999999999999\\
193.01	0.00999999999999999\\
194.01	0.00999999999999999\\
195.01	0.00999999999999999\\
196.01	0.00999999999999999\\
197.01	0.00999999999999999\\
198.01	0.00999999999999999\\
199.01	0.00999999999999999\\
200.01	0.00999999999999999\\
201.01	0.00999999999999999\\
202.01	0.00999999999999999\\
203.01	0.00999999999999999\\
204.01	0.00999999999999999\\
205.01	0.00999999999999999\\
206.01	0.00999999999999999\\
207.01	0.00999999999999999\\
208.01	0.00999999999999999\\
209.01	0.00999999999999999\\
210.01	0.00999999999999999\\
211.01	0.00999999999999999\\
212.01	0.00999999999999999\\
213.01	0.00999999999999999\\
214.01	0.00999999999999999\\
215.01	0.00999999999999999\\
216.01	0.00999999999999999\\
217.01	0.00999999999999999\\
218.01	0.00999999999999999\\
219.01	0.00999999999999999\\
220.01	0.00999999999999999\\
221.01	0.00999999999999999\\
222.01	0.00999999999999999\\
223.01	0.00999999999999999\\
224.01	0.00999999999999999\\
225.01	0.00999999999999999\\
226.01	0.00999999999999999\\
227.01	0.00999999999999999\\
228.01	0.00999999999999999\\
229.01	0.00999999999999999\\
230.01	0.00999999999999999\\
231.01	0.00999999999999999\\
232.01	0.00999999999999999\\
233.01	0.00999999999999999\\
234.01	0.00999999999999999\\
235.01	0.00999999999999999\\
236.01	0.00999999999999999\\
237.01	0.00999999999999999\\
238.01	0.00999999999999999\\
239.01	0.00999999999999999\\
240.01	0.00999999999999999\\
241.01	0.00999999999999999\\
242.01	0.00999999999999999\\
243.01	0.00999999999999999\\
244.01	0.00999999999999999\\
245.01	0.00999999999999999\\
246.01	0.00999999999999999\\
247.01	0.00999999999999999\\
248.01	0.00999999999999999\\
249.01	0.00999999999999999\\
250.01	0.00999999999999999\\
251.01	0.00999999999999999\\
252.01	0.00999999999999999\\
253.01	0.00999999999999999\\
254.01	0.00999999999999999\\
255.01	0.00999999999999999\\
256.01	0.00999999999999999\\
257.01	0.00999999999999999\\
258.01	0.00999999999999999\\
259.01	0.00999999999999999\\
260.01	0.00999999999999999\\
261.01	0.00999999999999999\\
262.01	0.00999999999999999\\
263.01	0.00999999999999999\\
264.01	0.00999999999999999\\
265.01	0.00999999999999999\\
266.01	0.00999999999999999\\
267.01	0.00999999999999999\\
268.01	0.00999999999999999\\
269.01	0.00999999999999999\\
270.01	0.00999999999999999\\
271.01	0.00999999999999999\\
272.01	0.00999999999999999\\
273.01	0.00999999999999999\\
274.01	0.00999999999999999\\
275.01	0.00999999999999999\\
276.01	0.00999999999999999\\
277.01	0.00999999999999999\\
278.01	0.00999999999999999\\
279.01	0.00999999999999999\\
280.01	0.00999999999999999\\
281.01	0.00999999999999999\\
282.01	0.00999999999999999\\
283.01	0.00999999999999999\\
284.01	0.00999999999999999\\
285.01	0.00999999999999999\\
286.01	0.00999999999999999\\
287.01	0.00999999999999999\\
288.01	0.00999999999999999\\
289.01	0.00999999999999999\\
290.01	0.00999999999999999\\
291.01	0.00999999999999999\\
292.01	0.00999999999999999\\
293.01	0.00999999999999999\\
294.01	0.00999999999999999\\
295.01	0.00999999999999999\\
296.01	0.00999999999999999\\
297.01	0.00999999999999999\\
298.01	0.00999999999999999\\
299.01	0.00999999999999999\\
300.01	0.00999999999999999\\
301.01	0.00999999999999999\\
302.01	0.00999999999999999\\
303.01	0.00999999999999999\\
304.01	0.00999999999999999\\
305.01	0.00999999999999999\\
306.01	0.00999999999999999\\
307.01	0.00999999999999999\\
308.01	0.00999999999999999\\
309.01	0.00999999999999999\\
310.01	0.00999999999999999\\
311.01	0.00999999999999999\\
312.01	0.00999999999999999\\
313.01	0.00999999999999999\\
314.01	0.00999999999999999\\
315.01	0.00999999999999999\\
316.01	0.00999999999999999\\
317.01	0.00999999999999999\\
318.01	0.00999999999999999\\
319.01	0.00999999999999999\\
320.01	0.00999999999999999\\
321.01	0.00999999999999999\\
322.01	0.00999999999999999\\
323.01	0.00999999999999999\\
324.01	0.00999999999999999\\
325.01	0.00999999999999999\\
326.01	0.00999999999999999\\
327.01	0.00999999999999999\\
328.01	0.00999999999999999\\
329.01	0.00999999999999999\\
330.01	0.00999999999999999\\
331.01	0.00999999999999999\\
332.01	0.00999999999999999\\
333.01	0.00999999999999999\\
334.01	0.00999999999999999\\
335.01	0.00999999999999999\\
336.01	0.00999999999999999\\
337.01	0.00999999999999999\\
338.01	0.00999999999999999\\
339.01	0.00999999999999999\\
340.01	0.00999999999999999\\
341.01	0.00999999999999999\\
342.01	0.00999999999999999\\
343.01	0.00999999999999999\\
344.01	0.00999999999999999\\
345.01	0.00999999999999999\\
346.01	0.00999999999999999\\
347.01	0.00999999999999999\\
348.01	0.00999999999999999\\
349.01	0.00999999999999999\\
350.01	0.00999999999999999\\
351.01	0.00999999999999999\\
352.01	0.00999999999999999\\
353.01	0.00999999999999999\\
354.01	0.00999999999999999\\
355.01	0.00999999999999999\\
356.01	0.00999999999999999\\
357.01	0.00999999999999999\\
358.01	0.00999999999999999\\
359.01	0.00999999999999999\\
360.01	0.00999999999999999\\
361.01	0.00999999999999999\\
362.01	0.00999999999999999\\
363.01	0.00999999999999999\\
364.01	0.00999999999999999\\
365.01	0.00999999999999999\\
366.01	0.00999999999999999\\
367.01	0.00999999999999999\\
368.01	0.00999999999999999\\
369.01	0.00999999999999999\\
370.01	0.00999999999999999\\
371.01	0.00999999999999999\\
372.01	0.00999999999999999\\
373.01	0.00999999999999999\\
374.01	0.00999999999999999\\
375.01	0.00999999999999999\\
376.01	0.00999999999999999\\
377.01	0.00999999999999999\\
378.01	0.00999999999999999\\
379.01	0.00999999999999999\\
380.01	0.00999999999999999\\
381.01	0.00999999999999999\\
382.01	0.00999999999999999\\
383.01	0.00999999999999999\\
384.01	0.00999999999999999\\
385.01	0.00999999999999999\\
386.01	0.00999999999999999\\
387.01	0.00999999999999999\\
388.01	0.00999999999999999\\
389.01	0.00999999999999999\\
390.01	0.00999999999999999\\
391.01	0.00999999999999999\\
392.01	0.00999999999999999\\
393.01	0.00999999999999999\\
394.01	0.00999999999999999\\
395.01	0.00999999999999999\\
396.01	0.00999999999999999\\
397.01	0.00999999999999999\\
398.01	0.00999999999999999\\
399.01	0.00999999999999999\\
400.01	0.00999999999999999\\
401.01	0.00999999999999999\\
402.01	0.00999999999999999\\
403.01	0.00999999999999999\\
404.01	0.00999999999999999\\
405.01	0.00999999999999999\\
406.01	0.00999999999999999\\
407.01	0.00999999999999999\\
408.01	0.00999999999999999\\
409.01	0.00999999999999999\\
410.01	0.00999999999999999\\
411.01	0.00999999999999999\\
412.01	0.00999999999999999\\
413.01	0.00999999999999999\\
414.01	0.00999999999999999\\
415.01	0.00999999999999999\\
416.01	0.00999999999999999\\
417.01	0.00999999999999999\\
418.01	0.00999999999999999\\
419.01	0.00999999999999999\\
420.01	0.00999999999999999\\
421.01	0.00999999999999999\\
422.01	0.00999999999999999\\
423.01	0.00999999999999999\\
424.01	0.00999999999999999\\
425.01	0.00999999999999999\\
426.01	0.00999999999999999\\
427.01	0.00999999999999999\\
428.01	0.00999999999999999\\
429.01	0.00999999999999999\\
430.01	0.00999999999999999\\
431.01	0.00999999999999999\\
432.01	0.00999999999999999\\
433.01	0.00999999999999999\\
434.01	0.00999999999999999\\
435.01	0.00999999999999999\\
436.01	0.00999999999999999\\
437.01	0.00999999999999999\\
438.01	0.00999999999999999\\
439.01	0.00999999999999999\\
440.01	0.00999999999999999\\
441.01	0.00999999999999999\\
442.01	0.00999999999999999\\
443.01	0.00999999999999999\\
444.01	0.00999999999999999\\
445.01	0.00999999999999999\\
446.01	0.00999999999999999\\
447.01	0.00999999999999999\\
448.01	0.00999999999999999\\
449.01	0.00999999999999999\\
450.01	0.00999999999999999\\
451.01	0.00999999999999999\\
452.01	0.00999999999999999\\
453.01	0.00999999999999999\\
454.01	0.00999999999999999\\
455.01	0.00999999999999999\\
456.01	0.00999999999999999\\
457.01	0.00999999999999999\\
458.01	0.00999999999999999\\
459.01	0.00999999999999999\\
460.01	0.00999999999999999\\
461.01	0.00999999999999999\\
462.01	0.00999999999999999\\
463.01	0.00999999999999999\\
464.01	0.00999999999999999\\
465.01	0.00999999999999999\\
466.01	0.00999999999999999\\
467.01	0.00999999999999999\\
468.01	0.00999999999999999\\
469.01	0.00999999999999999\\
470.01	0.00999999999999999\\
471.01	0.00999999999999999\\
472.01	0.00999999999999999\\
473.01	0.00999999999999999\\
474.01	0.00999999999999999\\
475.01	0.00999999999999999\\
476.01	0.00999999999999999\\
477.01	0.00999999999999999\\
478.01	0.00999999999999999\\
479.01	0.00999999999999999\\
480.01	0.00999999999999999\\
481.01	0.00999999999999999\\
482.01	0.00999999999999999\\
483.01	0.00999999999999999\\
484.01	0.00999999999999999\\
485.01	0.00999999999999999\\
486.01	0.00999999999999999\\
487.01	0.00999999999999999\\
488.01	0.00999999999999999\\
489.01	0.00999999999999999\\
490.01	0.00999999999999999\\
491.01	0.00999999999999999\\
492.01	0.00999999999999999\\
493.01	0.00999999999999999\\
494.01	0.00999999999999999\\
495.01	0.00999999999999999\\
496.01	0.00999999999999999\\
497.01	0.00999999999999999\\
498.01	0.00999999999999999\\
499.01	0.00999999999999999\\
500.01	0.00999999999999999\\
501.01	0.00999999999999999\\
502.01	0.00999999999999999\\
503.01	0.00999999999999999\\
504.01	0.00999999999999999\\
505.01	0.00999999999999999\\
506.01	0.00999999999999999\\
507.01	0.00999999999999999\\
508.01	0.00999999999999999\\
509.01	0.00999999999999999\\
510.01	0.00999999999999999\\
511.01	0.00999999999999999\\
512.01	0.00999999999999999\\
513.01	0.00999999999999999\\
514.01	0.00999999999999999\\
515.01	0.00999999999999999\\
516.01	0.00999999999999999\\
517.01	0.00999999999999999\\
518.01	0.00999999999999999\\
519.01	0.00999999999999999\\
520.01	0.00999999999999999\\
521.01	0.00999999999999999\\
522.01	0.00999999999999999\\
523.01	0.00999999999999999\\
524.01	0.00999999999999999\\
525.01	0.00999999999999999\\
526.01	0.00999999999999999\\
527.01	0.00999999999999999\\
528.01	0.00999999999999999\\
529.01	0.00999999999999999\\
530.01	0.00999999999999999\\
531.01	0.00999999999999999\\
532.01	0.00999999999999999\\
533.01	0.00999999999999999\\
534.01	0.00999999999999999\\
535.01	0.00999999999999999\\
536.01	0.00999999999999999\\
537.01	0.00999999999999999\\
538.01	0.00999999999999999\\
539.01	0.00999999999999999\\
540.01	0.00999999999999999\\
541.01	0.00999999999999999\\
542.01	0.00999999999999999\\
543.01	0.00999999999999999\\
544.01	0.00999999999999999\\
545.01	0.00999999999999999\\
546.01	0.00999999999999999\\
547.01	0.00999999999999999\\
548.01	0.00999999999999999\\
549.01	0.00999999999999999\\
550.01	0.00999999999999999\\
551.01	0.00999999999999999\\
552.01	0.00999999999999999\\
553.01	0.00999999999999999\\
554.01	0.00999999999999999\\
555.01	0.00999999999999999\\
556.01	0.00999999999999999\\
557.01	0.00999999999999999\\
558.01	0.00999999999999999\\
559.01	0.00999999999999999\\
560.01	0.00999999999999999\\
561.01	0.00999999999999999\\
562.01	0.00999999999999999\\
563.01	0.00999999999999999\\
564.01	0.00999999999999999\\
565.01	0.00999999999999999\\
566.01	0.00999999999999999\\
567.01	0.00999999999999999\\
568.01	0.00999999999999999\\
569.01	0.00999999999999999\\
570.01	0.00999999999999999\\
571.01	0.00999999999999999\\
572.01	0.00999999999999999\\
573.01	0.00999999999999999\\
574.01	0.00999999999999999\\
575.01	0.00999999999999999\\
576.01	0.00999999999999999\\
577.01	0.00999999999999999\\
578.01	0.00999999999999999\\
579.01	0.00999999999999999\\
580.01	0.00999999999999999\\
581.01	0.00999999999999999\\
582.01	0.00999999999999999\\
583.01	0.00999999999999999\\
584.01	0.00999999999999999\\
585.01	0.00999999999999999\\
586.01	0.00999999999999999\\
587.01	0.00999999999999999\\
588.01	0.00999999999999999\\
589.01	0.00999999999999999\\
590.01	0.00999999999999999\\
591.01	0.00999999999999999\\
592.01	0.00999999999999999\\
593.01	0.00999999999999999\\
594.01	0.00999999999999999\\
595.01	0.00999999999999999\\
596.01	0.00999999999999999\\
597.01	0.00951973751571587\\
598.01	0.00856167931814824\\
599.01	0.00614338742788375\\
599.02	0.00610585181164769\\
599.03	0.00606795853990916\\
599.04	0.00602970415859942\\
599.05	0.00599108518031156\\
599.06	0.00595209808397893\\
599.07	0.00591273931455056\\
599.08	0.00587300528266341\\
599.09	0.00583289236431152\\
599.1	0.00579239690051185\\
599.11	0.00575151519696703\\
599.12	0.00571024352372485\\
599.13	0.00566857811483433\\
599.14	0.00562651516799865\\
599.15	0.00558405084422464\\
599.16	0.0055411812674689\\
599.17	0.00549790252428056\\
599.18	0.00545421066344051\\
599.19	0.00541010169559728\\
599.2	0.00536557159289926\\
599.21	0.00532061628862351\\
599.22	0.00527523167680091\\
599.23	0.00522941361183776\\
599.24	0.0051831579081337\\
599.25	0.00513646033969588\\
599.26	0.00508931665044633\\
599.27	0.00504172255041564\\
599.28	0.00499367370841231\\
599.29	0.00494516575162712\\
599.3	0.00489619426523359\\
599.31	0.00484675479198464\\
599.32	0.00479684283180545\\
599.33	0.00474645384138226\\
599.34	0.00469558323374742\\
599.35	0.00464422637786028\\
599.36	0.00459237859818422\\
599.37	0.00454003517425949\\
599.38	0.00448719134027204\\
599.39	0.0044338422846182\\
599.4	0.00437998314946516\\
599.41	0.00432560903030727\\
599.42	0.00427071497551802\\
599.43	0.00421529598589786\\
599.44	0.00415934701421749\\
599.45	0.00410286296475698\\
599.46	0.00404583869284027\\
599.47	0.00398826900436541\\
599.48	0.00393014865533011\\
599.49	0.00387147235135287\\
599.5	0.00381223474718944\\
599.51	0.00375243044624474\\
599.52	0.00369205400007993\\
599.53	0.00363109990791499\\
599.54	0.0035695626161263\\
599.55	0.00350743651773955\\
599.56	0.00344471595191778\\
599.57	0.0033813952034444\\
599.58	0.00331746850220143\\
599.59	0.00325293002264254\\
599.6	0.00318777388326121\\
599.61	0.00312199414605369\\
599.62	0.00305558481597686\\
599.63	0.00298853984040087\\
599.64	0.00292085310855655\\
599.65	0.00285251845097747\\
599.66	0.00278352963893676\\
599.67	0.00271388038387843\\
599.68	0.00264356433684325\\
599.69	0.00257257508788922\\
599.7	0.00250090616550637\\
599.71	0.002428551036026\\
599.72	0.00235550310302428\\
599.73	0.00228175570672011\\
599.74	0.00220730212336715\\
599.75	0.00213213556464017\\
599.76	0.00205624917701539\\
599.77	0.0019796360411449\\
599.78	0.00190228917122514\\
599.79	0.00182420151435929\\
599.8	0.00174536594991352\\
599.81	0.00166577528886715\\
599.82	0.00158542227315651\\
599.83	0.00150429957501253\\
599.84	0.00142239979629202\\
599.85	0.00133971546780248\\
599.86	0.00125623904862046\\
599.87	0.00117196292540346\\
599.88	0.0010868794116951\\
599.89	0.00100098074722375\\
599.9	0.000914259097194382\\
599.91	0.000826706551573643\\
599.92	0.000738315124368094\\
599.93	0.000649076752895508\\
599.94	0.000558983297049216\\
599.95	0.000468026538555355\\
599.96	0.000376198180223065\\
599.97	0.000283489845187451\\
599.98	0.0001898930761453\\
599.99	9.53993345834923e-05\\
600	0\\
};
\addplot [color=red!80!mycolor19,solid,forget plot]
  table[row sep=crcr]{%
0.01	0.00999999999999999\\
1.01	0.00999999999999999\\
2.01	0.00999999999999999\\
3.01	0.00999999999999999\\
4.01	0.00999999999999999\\
5.01	0.00999999999999999\\
6.01	0.00999999999999999\\
7.01	0.00999999999999999\\
8.01	0.00999999999999999\\
9.01	0.00999999999999999\\
10.01	0.00999999999999999\\
11.01	0.00999999999999999\\
12.01	0.00999999999999999\\
13.01	0.00999999999999999\\
14.01	0.00999999999999999\\
15.01	0.00999999999999999\\
16.01	0.00999999999999999\\
17.01	0.00999999999999999\\
18.01	0.00999999999999999\\
19.01	0.00999999999999999\\
20.01	0.00999999999999999\\
21.01	0.00999999999999999\\
22.01	0.00999999999999999\\
23.01	0.00999999999999999\\
24.01	0.00999999999999999\\
25.01	0.00999999999999999\\
26.01	0.00999999999999999\\
27.01	0.00999999999999999\\
28.01	0.00999999999999999\\
29.01	0.00999999999999999\\
30.01	0.00999999999999999\\
31.01	0.00999999999999999\\
32.01	0.00999999999999999\\
33.01	0.00999999999999999\\
34.01	0.00999999999999999\\
35.01	0.00999999999999999\\
36.01	0.00999999999999999\\
37.01	0.00999999999999999\\
38.01	0.00999999999999999\\
39.01	0.00999999999999999\\
40.01	0.00999999999999999\\
41.01	0.00999999999999999\\
42.01	0.00999999999999999\\
43.01	0.00999999999999999\\
44.01	0.00999999999999999\\
45.01	0.00999999999999999\\
46.01	0.00999999999999999\\
47.01	0.00999999999999999\\
48.01	0.00999999999999999\\
49.01	0.00999999999999999\\
50.01	0.00999999999999999\\
51.01	0.00999999999999999\\
52.01	0.00999999999999999\\
53.01	0.00999999999999999\\
54.01	0.00999999999999999\\
55.01	0.00999999999999999\\
56.01	0.00999999999999999\\
57.01	0.00999999999999999\\
58.01	0.00999999999999999\\
59.01	0.00999999999999999\\
60.01	0.00999999999999999\\
61.01	0.00999999999999999\\
62.01	0.00999999999999999\\
63.01	0.00999999999999999\\
64.01	0.00999999999999999\\
65.01	0.00999999999999999\\
66.01	0.00999999999999999\\
67.01	0.00999999999999999\\
68.01	0.00999999999999999\\
69.01	0.00999999999999999\\
70.01	0.00999999999999999\\
71.01	0.00999999999999999\\
72.01	0.00999999999999999\\
73.01	0.00999999999999999\\
74.01	0.00999999999999999\\
75.01	0.00999999999999999\\
76.01	0.00999999999999999\\
77.01	0.00999999999999999\\
78.01	0.00999999999999999\\
79.01	0.00999999999999999\\
80.01	0.00999999999999999\\
81.01	0.00999999999999999\\
82.01	0.00999999999999999\\
83.01	0.00999999999999999\\
84.01	0.00999999999999999\\
85.01	0.00999999999999999\\
86.01	0.00999999999999999\\
87.01	0.00999999999999999\\
88.01	0.00999999999999999\\
89.01	0.00999999999999999\\
90.01	0.00999999999999999\\
91.01	0.00999999999999999\\
92.01	0.00999999999999999\\
93.01	0.00999999999999999\\
94.01	0.00999999999999999\\
95.01	0.00999999999999999\\
96.01	0.00999999999999999\\
97.01	0.00999999999999999\\
98.01	0.00999999999999999\\
99.01	0.00999999999999999\\
100.01	0.00999999999999999\\
101.01	0.00999999999999999\\
102.01	0.00999999999999999\\
103.01	0.00999999999999999\\
104.01	0.00999999999999999\\
105.01	0.00999999999999999\\
106.01	0.00999999999999999\\
107.01	0.00999999999999999\\
108.01	0.00999999999999999\\
109.01	0.00999999999999999\\
110.01	0.00999999999999999\\
111.01	0.00999999999999999\\
112.01	0.00999999999999999\\
113.01	0.00999999999999999\\
114.01	0.00999999999999999\\
115.01	0.00999999999999999\\
116.01	0.00999999999999999\\
117.01	0.00999999999999999\\
118.01	0.00999999999999999\\
119.01	0.00999999999999999\\
120.01	0.00999999999999999\\
121.01	0.00999999999999999\\
122.01	0.00999999999999999\\
123.01	0.00999999999999999\\
124.01	0.00999999999999999\\
125.01	0.00999999999999999\\
126.01	0.00999999999999999\\
127.01	0.00999999999999999\\
128.01	0.00999999999999999\\
129.01	0.00999999999999999\\
130.01	0.00999999999999999\\
131.01	0.00999999999999999\\
132.01	0.00999999999999999\\
133.01	0.00999999999999999\\
134.01	0.00999999999999999\\
135.01	0.00999999999999999\\
136.01	0.00999999999999999\\
137.01	0.00999999999999999\\
138.01	0.00999999999999999\\
139.01	0.00999999999999999\\
140.01	0.00999999999999999\\
141.01	0.00999999999999999\\
142.01	0.00999999999999999\\
143.01	0.00999999999999999\\
144.01	0.00999999999999999\\
145.01	0.00999999999999999\\
146.01	0.00999999999999999\\
147.01	0.00999999999999999\\
148.01	0.00999999999999999\\
149.01	0.00999999999999999\\
150.01	0.00999999999999999\\
151.01	0.00999999999999999\\
152.01	0.00999999999999999\\
153.01	0.00999999999999999\\
154.01	0.00999999999999999\\
155.01	0.00999999999999999\\
156.01	0.00999999999999999\\
157.01	0.00999999999999999\\
158.01	0.00999999999999999\\
159.01	0.00999999999999999\\
160.01	0.00999999999999999\\
161.01	0.00999999999999999\\
162.01	0.00999999999999999\\
163.01	0.00999999999999999\\
164.01	0.00999999999999999\\
165.01	0.00999999999999999\\
166.01	0.00999999999999999\\
167.01	0.00999999999999999\\
168.01	0.00999999999999999\\
169.01	0.00999999999999999\\
170.01	0.00999999999999999\\
171.01	0.00999999999999999\\
172.01	0.00999999999999999\\
173.01	0.00999999999999999\\
174.01	0.00999999999999999\\
175.01	0.00999999999999999\\
176.01	0.00999999999999999\\
177.01	0.00999999999999999\\
178.01	0.00999999999999999\\
179.01	0.00999999999999999\\
180.01	0.00999999999999999\\
181.01	0.00999999999999999\\
182.01	0.00999999999999999\\
183.01	0.00999999999999999\\
184.01	0.00999999999999999\\
185.01	0.00999999999999999\\
186.01	0.00999999999999999\\
187.01	0.00999999999999999\\
188.01	0.00999999999999999\\
189.01	0.00999999999999999\\
190.01	0.00999999999999999\\
191.01	0.00999999999999999\\
192.01	0.00999999999999999\\
193.01	0.00999999999999999\\
194.01	0.00999999999999999\\
195.01	0.00999999999999999\\
196.01	0.00999999999999999\\
197.01	0.00999999999999999\\
198.01	0.00999999999999999\\
199.01	0.00999999999999999\\
200.01	0.00999999999999999\\
201.01	0.00999999999999999\\
202.01	0.00999999999999999\\
203.01	0.00999999999999999\\
204.01	0.00999999999999999\\
205.01	0.00999999999999999\\
206.01	0.00999999999999999\\
207.01	0.00999999999999999\\
208.01	0.00999999999999999\\
209.01	0.00999999999999999\\
210.01	0.00999999999999999\\
211.01	0.00999999999999999\\
212.01	0.00999999999999999\\
213.01	0.00999999999999999\\
214.01	0.00999999999999999\\
215.01	0.00999999999999999\\
216.01	0.00999999999999999\\
217.01	0.00999999999999999\\
218.01	0.00999999999999999\\
219.01	0.00999999999999999\\
220.01	0.00999999999999999\\
221.01	0.00999999999999999\\
222.01	0.00999999999999999\\
223.01	0.00999999999999999\\
224.01	0.00999999999999999\\
225.01	0.00999999999999999\\
226.01	0.00999999999999999\\
227.01	0.00999999999999999\\
228.01	0.00999999999999999\\
229.01	0.00999999999999999\\
230.01	0.00999999999999999\\
231.01	0.00999999999999999\\
232.01	0.00999999999999999\\
233.01	0.00999999999999999\\
234.01	0.00999999999999999\\
235.01	0.00999999999999999\\
236.01	0.00999999999999999\\
237.01	0.00999999999999999\\
238.01	0.00999999999999999\\
239.01	0.00999999999999999\\
240.01	0.00999999999999999\\
241.01	0.00999999999999999\\
242.01	0.00999999999999999\\
243.01	0.00999999999999999\\
244.01	0.00999999999999999\\
245.01	0.00999999999999999\\
246.01	0.00999999999999999\\
247.01	0.00999999999999999\\
248.01	0.00999999999999999\\
249.01	0.00999999999999999\\
250.01	0.00999999999999999\\
251.01	0.00999999999999999\\
252.01	0.00999999999999999\\
253.01	0.00999999999999999\\
254.01	0.00999999999999999\\
255.01	0.00999999999999999\\
256.01	0.00999999999999999\\
257.01	0.00999999999999999\\
258.01	0.00999999999999999\\
259.01	0.00999999999999999\\
260.01	0.00999999999999999\\
261.01	0.00999999999999999\\
262.01	0.00999999999999999\\
263.01	0.00999999999999999\\
264.01	0.00999999999999999\\
265.01	0.00999999999999999\\
266.01	0.00999999999999999\\
267.01	0.00999999999999999\\
268.01	0.00999999999999999\\
269.01	0.00999999999999999\\
270.01	0.00999999999999999\\
271.01	0.00999999999999999\\
272.01	0.00999999999999999\\
273.01	0.00999999999999999\\
274.01	0.00999999999999999\\
275.01	0.00999999999999999\\
276.01	0.00999999999999999\\
277.01	0.00999999999999999\\
278.01	0.00999999999999999\\
279.01	0.00999999999999999\\
280.01	0.00999999999999999\\
281.01	0.00999999999999999\\
282.01	0.00999999999999999\\
283.01	0.00999999999999999\\
284.01	0.00999999999999999\\
285.01	0.00999999999999999\\
286.01	0.00999999999999999\\
287.01	0.00999999999999999\\
288.01	0.00999999999999999\\
289.01	0.00999999999999999\\
290.01	0.00999999999999999\\
291.01	0.00999999999999999\\
292.01	0.00999999999999999\\
293.01	0.00999999999999999\\
294.01	0.00999999999999999\\
295.01	0.00999999999999999\\
296.01	0.00999999999999999\\
297.01	0.00999999999999999\\
298.01	0.00999999999999999\\
299.01	0.00999999999999999\\
300.01	0.00999999999999999\\
301.01	0.00999999999999999\\
302.01	0.00999999999999999\\
303.01	0.00999999999999999\\
304.01	0.00999999999999999\\
305.01	0.00999999999999999\\
306.01	0.00999999999999999\\
307.01	0.00999999999999999\\
308.01	0.00999999999999999\\
309.01	0.00999999999999999\\
310.01	0.00999999999999999\\
311.01	0.00999999999999999\\
312.01	0.00999999999999999\\
313.01	0.00999999999999999\\
314.01	0.00999999999999999\\
315.01	0.00999999999999999\\
316.01	0.00999999999999999\\
317.01	0.00999999999999999\\
318.01	0.00999999999999999\\
319.01	0.00999999999999999\\
320.01	0.00999999999999999\\
321.01	0.00999999999999999\\
322.01	0.00999999999999999\\
323.01	0.00999999999999999\\
324.01	0.00999999999999999\\
325.01	0.00999999999999999\\
326.01	0.00999999999999999\\
327.01	0.00999999999999999\\
328.01	0.00999999999999999\\
329.01	0.00999999999999999\\
330.01	0.00999999999999999\\
331.01	0.00999999999999999\\
332.01	0.00999999999999999\\
333.01	0.00999999999999999\\
334.01	0.00999999999999999\\
335.01	0.00999999999999999\\
336.01	0.00999999999999999\\
337.01	0.00999999999999999\\
338.01	0.00999999999999999\\
339.01	0.00999999999999999\\
340.01	0.00999999999999999\\
341.01	0.00999999999999999\\
342.01	0.00999999999999999\\
343.01	0.00999999999999999\\
344.01	0.00999999999999999\\
345.01	0.00999999999999999\\
346.01	0.00999999999999999\\
347.01	0.00999999999999999\\
348.01	0.00999999999999999\\
349.01	0.00999999999999999\\
350.01	0.00999999999999999\\
351.01	0.00999999999999999\\
352.01	0.00999999999999999\\
353.01	0.00999999999999999\\
354.01	0.00999999999999999\\
355.01	0.00999999999999999\\
356.01	0.00999999999999999\\
357.01	0.00999999999999999\\
358.01	0.00999999999999999\\
359.01	0.00999999999999999\\
360.01	0.00999999999999999\\
361.01	0.00999999999999999\\
362.01	0.00999999999999999\\
363.01	0.00999999999999999\\
364.01	0.00999999999999999\\
365.01	0.00999999999999999\\
366.01	0.00999999999999999\\
367.01	0.00999999999999999\\
368.01	0.00999999999999999\\
369.01	0.00999999999999999\\
370.01	0.00999999999999999\\
371.01	0.00999999999999999\\
372.01	0.00999999999999999\\
373.01	0.00999999999999999\\
374.01	0.00999999999999999\\
375.01	0.00999999999999999\\
376.01	0.00999999999999999\\
377.01	0.00999999999999999\\
378.01	0.00999999999999999\\
379.01	0.00999999999999999\\
380.01	0.00999999999999999\\
381.01	0.00999999999999999\\
382.01	0.00999999999999999\\
383.01	0.00999999999999999\\
384.01	0.00999999999999999\\
385.01	0.00999999999999999\\
386.01	0.00999999999999999\\
387.01	0.00999999999999999\\
388.01	0.00999999999999999\\
389.01	0.00999999999999999\\
390.01	0.00999999999999999\\
391.01	0.00999999999999999\\
392.01	0.00999999999999999\\
393.01	0.00999999999999999\\
394.01	0.00999999999999999\\
395.01	0.00999999999999999\\
396.01	0.00999999999999999\\
397.01	0.00999999999999999\\
398.01	0.00999999999999999\\
399.01	0.00999999999999999\\
400.01	0.00999999999999999\\
401.01	0.00999999999999999\\
402.01	0.00999999999999999\\
403.01	0.00999999999999999\\
404.01	0.00999999999999999\\
405.01	0.00999999999999999\\
406.01	0.00999999999999999\\
407.01	0.00999999999999999\\
408.01	0.00999999999999999\\
409.01	0.00999999999999999\\
410.01	0.00999999999999999\\
411.01	0.00999999999999999\\
412.01	0.00999999999999999\\
413.01	0.00999999999999999\\
414.01	0.00999999999999999\\
415.01	0.00999999999999999\\
416.01	0.00999999999999999\\
417.01	0.00999999999999999\\
418.01	0.00999999999999999\\
419.01	0.00999999999999999\\
420.01	0.00999999999999999\\
421.01	0.00999999999999999\\
422.01	0.00999999999999999\\
423.01	0.00999999999999999\\
424.01	0.00999999999999999\\
425.01	0.00999999999999999\\
426.01	0.00999999999999999\\
427.01	0.00999999999999999\\
428.01	0.00999999999999999\\
429.01	0.00999999999999999\\
430.01	0.00999999999999999\\
431.01	0.00999999999999999\\
432.01	0.00999999999999999\\
433.01	0.00999999999999999\\
434.01	0.00999999999999999\\
435.01	0.00999999999999999\\
436.01	0.00999999999999999\\
437.01	0.00999999999999999\\
438.01	0.00999999999999999\\
439.01	0.00999999999999999\\
440.01	0.00999999999999999\\
441.01	0.00999999999999999\\
442.01	0.00999999999999999\\
443.01	0.00999999999999999\\
444.01	0.00999999999999999\\
445.01	0.00999999999999999\\
446.01	0.00999999999999999\\
447.01	0.00999999999999999\\
448.01	0.00999999999999999\\
449.01	0.00999999999999999\\
450.01	0.00999999999999999\\
451.01	0.00999999999999999\\
452.01	0.00999999999999999\\
453.01	0.00999999999999999\\
454.01	0.00999999999999999\\
455.01	0.00999999999999999\\
456.01	0.00999999999999999\\
457.01	0.00999999999999999\\
458.01	0.00999999999999999\\
459.01	0.00999999999999999\\
460.01	0.00999999999999999\\
461.01	0.00999999999999999\\
462.01	0.00999999999999999\\
463.01	0.00999999999999999\\
464.01	0.00999999999999999\\
465.01	0.00999999999999999\\
466.01	0.00999999999999999\\
467.01	0.00999999999999999\\
468.01	0.00999999999999999\\
469.01	0.00999999999999999\\
470.01	0.00999999999999999\\
471.01	0.00999999999999999\\
472.01	0.00999999999999999\\
473.01	0.00999999999999999\\
474.01	0.00999999999999999\\
475.01	0.00999999999999999\\
476.01	0.00999999999999999\\
477.01	0.00999999999999999\\
478.01	0.00999999999999999\\
479.01	0.00999999999999999\\
480.01	0.00999999999999999\\
481.01	0.00999999999999999\\
482.01	0.00999999999999999\\
483.01	0.00999999999999999\\
484.01	0.00999999999999999\\
485.01	0.00999999999999999\\
486.01	0.00999999999999999\\
487.01	0.00999999999999999\\
488.01	0.00999999999999999\\
489.01	0.00999999999999999\\
490.01	0.00999999999999999\\
491.01	0.00999999999999999\\
492.01	0.00999999999999999\\
493.01	0.00999999999999999\\
494.01	0.00999999999999999\\
495.01	0.00999999999999999\\
496.01	0.00999999999999999\\
497.01	0.00999999999999999\\
498.01	0.00999999999999999\\
499.01	0.00999999999999999\\
500.01	0.00999999999999999\\
501.01	0.00999999999999999\\
502.01	0.00999999999999999\\
503.01	0.00999999999999999\\
504.01	0.00999999999999999\\
505.01	0.00999999999999999\\
506.01	0.00999999999999999\\
507.01	0.00999999999999999\\
508.01	0.00999999999999999\\
509.01	0.00999999999999999\\
510.01	0.00999999999999999\\
511.01	0.00999999999999999\\
512.01	0.00999999999999999\\
513.01	0.00999999999999999\\
514.01	0.00999999999999999\\
515.01	0.00999999999999999\\
516.01	0.00999999999999999\\
517.01	0.00999999999999999\\
518.01	0.00999999999999999\\
519.01	0.00999999999999999\\
520.01	0.00999999999999999\\
521.01	0.00999999999999999\\
522.01	0.00999999999999999\\
523.01	0.00999999999999999\\
524.01	0.00999999999999999\\
525.01	0.00999999999999999\\
526.01	0.00999999999999999\\
527.01	0.00999999999999999\\
528.01	0.00999999999999999\\
529.01	0.00999999999999999\\
530.01	0.00999999999999999\\
531.01	0.00999999999999999\\
532.01	0.00999999999999999\\
533.01	0.00999999999999999\\
534.01	0.00999999999999999\\
535.01	0.00999999999999999\\
536.01	0.00999999999999999\\
537.01	0.00999999999999999\\
538.01	0.00999999999999999\\
539.01	0.00999999999999999\\
540.01	0.00999999999999999\\
541.01	0.00999999999999999\\
542.01	0.00999999999999999\\
543.01	0.00999999999999999\\
544.01	0.00999999999999999\\
545.01	0.00999999999999999\\
546.01	0.00999999999999999\\
547.01	0.00999999999999999\\
548.01	0.00999999999999999\\
549.01	0.00999999999999999\\
550.01	0.00999999999999999\\
551.01	0.00999999999999999\\
552.01	0.00999999999999999\\
553.01	0.00999999999999999\\
554.01	0.00999999999999999\\
555.01	0.00999999999999999\\
556.01	0.00999999999999999\\
557.01	0.00999999999999999\\
558.01	0.00999999999999999\\
559.01	0.00999999999999999\\
560.01	0.00999999999999999\\
561.01	0.00999999999999999\\
562.01	0.00999999999999999\\
563.01	0.00999999999999999\\
564.01	0.00999999999999999\\
565.01	0.00999999999999999\\
566.01	0.00999999999999999\\
567.01	0.00999999999999999\\
568.01	0.00999999999999999\\
569.01	0.00999999999999999\\
570.01	0.00999999999999999\\
571.01	0.00999999999999999\\
572.01	0.00999999999999999\\
573.01	0.00999999999999999\\
574.01	0.00999999999999999\\
575.01	0.00999999999999999\\
576.01	0.00999999999999999\\
577.01	0.00999999999999999\\
578.01	0.00999999999999999\\
579.01	0.00999999999999999\\
580.01	0.00999999999999999\\
581.01	0.00999999999999999\\
582.01	0.00999999999999999\\
583.01	0.00999999999999999\\
584.01	0.00999999999999999\\
585.01	0.00999999999999999\\
586.01	0.00999999999999999\\
587.01	0.00999999999999999\\
588.01	0.00999999999999999\\
589.01	0.00999999999999999\\
590.01	0.00999999999999999\\
591.01	0.00999999999999999\\
592.01	0.00999999999999999\\
593.01	0.00999999999999999\\
594.01	0.00999999999999999\\
595.01	0.00999999999999999\\
596.01	0.00999999999999999\\
597.01	0.00951939857269104\\
598.01	0.00856167931814819\\
599.01	0.00614338742788375\\
599.02	0.00610585181164768\\
599.03	0.00606795853990915\\
599.04	0.00602970415859941\\
599.05	0.00599108518031155\\
599.06	0.00595209808397892\\
599.07	0.00591273931455054\\
599.08	0.0058730052826634\\
599.09	0.00583289236431151\\
599.1	0.00579239690051182\\
599.11	0.00575151519696701\\
599.12	0.00571024352372483\\
599.13	0.00566857811483431\\
599.14	0.00562651516799864\\
599.15	0.00558405084422462\\
599.16	0.00554118126746889\\
599.17	0.00549790252428055\\
599.18	0.00545421066344051\\
599.19	0.00541010169559727\\
599.2	0.00536557159289924\\
599.21	0.00532061628862349\\
599.22	0.00527523167680091\\
599.23	0.00522941361183776\\
599.24	0.00518315790813369\\
599.25	0.00513646033969587\\
599.26	0.00508931665044633\\
599.27	0.00504172255041564\\
599.28	0.0049936737084123\\
599.29	0.00494516575162711\\
599.3	0.00489619426523357\\
599.31	0.00484675479198463\\
599.32	0.00479684283180543\\
599.33	0.00474645384138225\\
599.34	0.00469558323374741\\
599.35	0.00464422637786027\\
599.36	0.00459237859818421\\
599.37	0.00454003517425947\\
599.38	0.00448719134027202\\
599.39	0.00443384228461819\\
599.4	0.00437998314946515\\
599.41	0.00432560903030725\\
599.42	0.00427071497551801\\
599.43	0.00421529598589784\\
599.44	0.00415934701421747\\
599.45	0.00410286296475695\\
599.46	0.00404583869284025\\
599.47	0.00398826900436539\\
599.48	0.00393014865533009\\
599.49	0.00387147235135285\\
599.5	0.00381223474718943\\
599.51	0.00375243044624472\\
599.52	0.00369205400007992\\
599.53	0.00363109990791498\\
599.54	0.00356956261612628\\
599.55	0.00350743651773954\\
599.56	0.00344471595191777\\
599.57	0.0033813952034444\\
599.58	0.00331746850220143\\
599.59	0.00325293002264254\\
599.6	0.00318777388326121\\
599.61	0.00312199414605369\\
599.62	0.00305558481597686\\
599.63	0.00298853984040088\\
599.64	0.00292085310855656\\
599.65	0.00285251845097747\\
599.66	0.00278352963893676\\
599.67	0.00271388038387843\\
599.68	0.00264356433684325\\
599.69	0.00257257508788922\\
599.7	0.00250090616550637\\
599.71	0.00242855103602601\\
599.72	0.0023555031030243\\
599.73	0.00228175570672011\\
599.74	0.00220730212336715\\
599.75	0.00213213556464018\\
599.76	0.00205624917701539\\
599.77	0.00197963604114491\\
599.78	0.00190228917122514\\
599.79	0.00182420151435929\\
599.8	0.00174536594991352\\
599.81	0.00166577528886715\\
599.82	0.00158542227315651\\
599.83	0.00150429957501254\\
599.84	0.00142239979629203\\
599.85	0.00133971546780248\\
599.86	0.00125623904862047\\
599.87	0.00117196292540347\\
599.88	0.0010868794116951\\
599.89	0.00100098074722375\\
599.9	0.000914259097194382\\
599.91	0.000826706551573644\\
599.92	0.000738315124368094\\
599.93	0.00064907675289551\\
599.94	0.000558983297049212\\
599.95	0.000468026538555355\\
599.96	0.000376198180223065\\
599.97	0.000283489845187451\\
599.98	0.000189893076145302\\
599.99	9.53993345834923e-05\\
600	0\\
};
\addplot [color=red,solid,forget plot]
  table[row sep=crcr]{%
0.01	0.00999999999999999\\
1.01	0.00999999999999999\\
2.01	0.00999999999999999\\
3.01	0.00999999999999999\\
4.01	0.00999999999999999\\
5.01	0.00999999999999999\\
6.01	0.00999999999999999\\
7.01	0.00999999999999999\\
8.01	0.00999999999999999\\
9.01	0.00999999999999999\\
10.01	0.00999999999999999\\
11.01	0.00999999999999999\\
12.01	0.00999999999999999\\
13.01	0.00999999999999999\\
14.01	0.00999999999999999\\
15.01	0.00999999999999999\\
16.01	0.00999999999999999\\
17.01	0.00999999999999999\\
18.01	0.00999999999999999\\
19.01	0.00999999999999999\\
20.01	0.00999999999999999\\
21.01	0.00999999999999999\\
22.01	0.00999999999999999\\
23.01	0.00999999999999999\\
24.01	0.00999999999999999\\
25.01	0.00999999999999999\\
26.01	0.00999999999999999\\
27.01	0.00999999999999999\\
28.01	0.00999999999999999\\
29.01	0.00999999999999999\\
30.01	0.00999999999999999\\
31.01	0.00999999999999999\\
32.01	0.00999999999999999\\
33.01	0.00999999999999999\\
34.01	0.00999999999999999\\
35.01	0.00999999999999999\\
36.01	0.00999999999999999\\
37.01	0.00999999999999999\\
38.01	0.00999999999999999\\
39.01	0.00999999999999999\\
40.01	0.00999999999999999\\
41.01	0.00999999999999999\\
42.01	0.00999999999999999\\
43.01	0.00999999999999999\\
44.01	0.00999999999999999\\
45.01	0.00999999999999999\\
46.01	0.00999999999999999\\
47.01	0.00999999999999999\\
48.01	0.00999999999999999\\
49.01	0.00999999999999999\\
50.01	0.00999999999999999\\
51.01	0.00999999999999999\\
52.01	0.00999999999999999\\
53.01	0.00999999999999999\\
54.01	0.00999999999999999\\
55.01	0.00999999999999999\\
56.01	0.00999999999999999\\
57.01	0.00999999999999999\\
58.01	0.00999999999999999\\
59.01	0.00999999999999999\\
60.01	0.00999999999999999\\
61.01	0.00999999999999999\\
62.01	0.00999999999999999\\
63.01	0.00999999999999999\\
64.01	0.00999999999999999\\
65.01	0.00999999999999999\\
66.01	0.00999999999999999\\
67.01	0.00999999999999999\\
68.01	0.00999999999999999\\
69.01	0.00999999999999999\\
70.01	0.00999999999999999\\
71.01	0.00999999999999999\\
72.01	0.00999999999999999\\
73.01	0.00999999999999999\\
74.01	0.00999999999999999\\
75.01	0.00999999999999999\\
76.01	0.00999999999999999\\
77.01	0.00999999999999999\\
78.01	0.00999999999999999\\
79.01	0.00999999999999999\\
80.01	0.00999999999999999\\
81.01	0.00999999999999999\\
82.01	0.00999999999999999\\
83.01	0.00999999999999999\\
84.01	0.00999999999999999\\
85.01	0.00999999999999999\\
86.01	0.00999999999999999\\
87.01	0.00999999999999999\\
88.01	0.00999999999999999\\
89.01	0.00999999999999999\\
90.01	0.00999999999999999\\
91.01	0.00999999999999999\\
92.01	0.00999999999999999\\
93.01	0.00999999999999999\\
94.01	0.00999999999999999\\
95.01	0.00999999999999999\\
96.01	0.00999999999999999\\
97.01	0.00999999999999999\\
98.01	0.00999999999999999\\
99.01	0.00999999999999999\\
100.01	0.00999999999999999\\
101.01	0.00999999999999999\\
102.01	0.00999999999999999\\
103.01	0.00999999999999999\\
104.01	0.00999999999999999\\
105.01	0.00999999999999999\\
106.01	0.00999999999999999\\
107.01	0.00999999999999999\\
108.01	0.00999999999999999\\
109.01	0.00999999999999999\\
110.01	0.00999999999999999\\
111.01	0.00999999999999999\\
112.01	0.00999999999999999\\
113.01	0.00999999999999999\\
114.01	0.00999999999999999\\
115.01	0.00999999999999999\\
116.01	0.00999999999999999\\
117.01	0.00999999999999999\\
118.01	0.00999999999999999\\
119.01	0.00999999999999999\\
120.01	0.00999999999999999\\
121.01	0.00999999999999999\\
122.01	0.00999999999999999\\
123.01	0.00999999999999999\\
124.01	0.00999999999999999\\
125.01	0.00999999999999999\\
126.01	0.00999999999999999\\
127.01	0.00999999999999999\\
128.01	0.00999999999999999\\
129.01	0.00999999999999999\\
130.01	0.00999999999999999\\
131.01	0.00999999999999999\\
132.01	0.00999999999999999\\
133.01	0.00999999999999999\\
134.01	0.00999999999999999\\
135.01	0.00999999999999999\\
136.01	0.00999999999999999\\
137.01	0.00999999999999999\\
138.01	0.00999999999999999\\
139.01	0.00999999999999999\\
140.01	0.00999999999999999\\
141.01	0.00999999999999999\\
142.01	0.00999999999999999\\
143.01	0.00999999999999999\\
144.01	0.00999999999999999\\
145.01	0.00999999999999999\\
146.01	0.00999999999999999\\
147.01	0.00999999999999999\\
148.01	0.00999999999999999\\
149.01	0.00999999999999999\\
150.01	0.00999999999999999\\
151.01	0.00999999999999999\\
152.01	0.00999999999999999\\
153.01	0.00999999999999999\\
154.01	0.00999999999999999\\
155.01	0.00999999999999999\\
156.01	0.00999999999999999\\
157.01	0.00999999999999999\\
158.01	0.00999999999999999\\
159.01	0.00999999999999999\\
160.01	0.00999999999999999\\
161.01	0.00999999999999999\\
162.01	0.00999999999999999\\
163.01	0.00999999999999999\\
164.01	0.00999999999999999\\
165.01	0.00999999999999999\\
166.01	0.00999999999999999\\
167.01	0.00999999999999999\\
168.01	0.00999999999999999\\
169.01	0.00999999999999999\\
170.01	0.00999999999999999\\
171.01	0.00999999999999999\\
172.01	0.00999999999999999\\
173.01	0.00999999999999999\\
174.01	0.00999999999999999\\
175.01	0.00999999999999999\\
176.01	0.00999999999999999\\
177.01	0.00999999999999999\\
178.01	0.00999999999999999\\
179.01	0.00999999999999999\\
180.01	0.00999999999999999\\
181.01	0.00999999999999999\\
182.01	0.00999999999999999\\
183.01	0.00999999999999999\\
184.01	0.00999999999999999\\
185.01	0.00999999999999999\\
186.01	0.00999999999999999\\
187.01	0.00999999999999999\\
188.01	0.00999999999999999\\
189.01	0.00999999999999999\\
190.01	0.00999999999999999\\
191.01	0.00999999999999999\\
192.01	0.00999999999999999\\
193.01	0.00999999999999999\\
194.01	0.00999999999999999\\
195.01	0.00999999999999999\\
196.01	0.00999999999999999\\
197.01	0.00999999999999999\\
198.01	0.00999999999999999\\
199.01	0.00999999999999999\\
200.01	0.00999999999999999\\
201.01	0.00999999999999999\\
202.01	0.00999999999999999\\
203.01	0.00999999999999999\\
204.01	0.00999999999999999\\
205.01	0.00999999999999999\\
206.01	0.00999999999999999\\
207.01	0.00999999999999999\\
208.01	0.00999999999999999\\
209.01	0.00999999999999999\\
210.01	0.00999999999999999\\
211.01	0.00999999999999999\\
212.01	0.00999999999999999\\
213.01	0.00999999999999999\\
214.01	0.00999999999999999\\
215.01	0.00999999999999999\\
216.01	0.00999999999999999\\
217.01	0.00999999999999999\\
218.01	0.00999999999999999\\
219.01	0.00999999999999999\\
220.01	0.00999999999999999\\
221.01	0.00999999999999999\\
222.01	0.00999999999999999\\
223.01	0.00999999999999999\\
224.01	0.00999999999999999\\
225.01	0.00999999999999999\\
226.01	0.00999999999999999\\
227.01	0.00999999999999999\\
228.01	0.00999999999999999\\
229.01	0.00999999999999999\\
230.01	0.00999999999999999\\
231.01	0.00999999999999999\\
232.01	0.00999999999999999\\
233.01	0.00999999999999999\\
234.01	0.00999999999999999\\
235.01	0.00999999999999999\\
236.01	0.00999999999999999\\
237.01	0.00999999999999999\\
238.01	0.00999999999999999\\
239.01	0.00999999999999999\\
240.01	0.00999999999999999\\
241.01	0.00999999999999999\\
242.01	0.00999999999999999\\
243.01	0.00999999999999999\\
244.01	0.00999999999999999\\
245.01	0.00999999999999999\\
246.01	0.00999999999999999\\
247.01	0.00999999999999999\\
248.01	0.00999999999999999\\
249.01	0.00999999999999999\\
250.01	0.00999999999999999\\
251.01	0.00999999999999999\\
252.01	0.00999999999999999\\
253.01	0.00999999999999999\\
254.01	0.00999999999999999\\
255.01	0.00999999999999999\\
256.01	0.00999999999999999\\
257.01	0.00999999999999999\\
258.01	0.00999999999999999\\
259.01	0.00999999999999999\\
260.01	0.00999999999999999\\
261.01	0.00999999999999999\\
262.01	0.00999999999999999\\
263.01	0.00999999999999999\\
264.01	0.00999999999999999\\
265.01	0.00999999999999999\\
266.01	0.00999999999999999\\
267.01	0.00999999999999999\\
268.01	0.00999999999999999\\
269.01	0.00999999999999999\\
270.01	0.00999999999999999\\
271.01	0.00999999999999999\\
272.01	0.00999999999999999\\
273.01	0.00999999999999999\\
274.01	0.00999999999999999\\
275.01	0.00999999999999999\\
276.01	0.00999999999999999\\
277.01	0.00999999999999999\\
278.01	0.00999999999999999\\
279.01	0.00999999999999999\\
280.01	0.00999999999999999\\
281.01	0.00999999999999999\\
282.01	0.00999999999999999\\
283.01	0.00999999999999999\\
284.01	0.00999999999999999\\
285.01	0.00999999999999999\\
286.01	0.00999999999999999\\
287.01	0.00999999999999999\\
288.01	0.00999999999999999\\
289.01	0.00999999999999999\\
290.01	0.00999999999999999\\
291.01	0.00999999999999999\\
292.01	0.00999999999999999\\
293.01	0.00999999999999999\\
294.01	0.00999999999999999\\
295.01	0.00999999999999999\\
296.01	0.00999999999999999\\
297.01	0.00999999999999999\\
298.01	0.00999999999999999\\
299.01	0.00999999999999999\\
300.01	0.00999999999999999\\
301.01	0.00999999999999999\\
302.01	0.00999999999999999\\
303.01	0.00999999999999999\\
304.01	0.00999999999999999\\
305.01	0.00999999999999999\\
306.01	0.00999999999999999\\
307.01	0.00999999999999999\\
308.01	0.00999999999999999\\
309.01	0.00999999999999999\\
310.01	0.00999999999999999\\
311.01	0.00999999999999999\\
312.01	0.00999999999999999\\
313.01	0.00999999999999999\\
314.01	0.00999999999999999\\
315.01	0.00999999999999999\\
316.01	0.00999999999999999\\
317.01	0.00999999999999999\\
318.01	0.00999999999999999\\
319.01	0.00999999999999999\\
320.01	0.00999999999999999\\
321.01	0.00999999999999999\\
322.01	0.00999999999999999\\
323.01	0.00999999999999999\\
324.01	0.00999999999999999\\
325.01	0.00999999999999999\\
326.01	0.00999999999999999\\
327.01	0.00999999999999999\\
328.01	0.00999999999999999\\
329.01	0.00999999999999999\\
330.01	0.00999999999999999\\
331.01	0.00999999999999999\\
332.01	0.00999999999999999\\
333.01	0.00999999999999999\\
334.01	0.00999999999999999\\
335.01	0.00999999999999999\\
336.01	0.00999999999999999\\
337.01	0.00999999999999999\\
338.01	0.00999999999999999\\
339.01	0.00999999999999999\\
340.01	0.00999999999999999\\
341.01	0.00999999999999999\\
342.01	0.00999999999999999\\
343.01	0.00999999999999999\\
344.01	0.00999999999999999\\
345.01	0.00999999999999999\\
346.01	0.00999999999999999\\
347.01	0.00999999999999999\\
348.01	0.00999999999999999\\
349.01	0.00999999999999999\\
350.01	0.00999999999999999\\
351.01	0.00999999999999999\\
352.01	0.00999999999999999\\
353.01	0.00999999999999999\\
354.01	0.00999999999999999\\
355.01	0.00999999999999999\\
356.01	0.00999999999999999\\
357.01	0.00999999999999999\\
358.01	0.00999999999999999\\
359.01	0.00999999999999999\\
360.01	0.00999999999999999\\
361.01	0.00999999999999999\\
362.01	0.00999999999999999\\
363.01	0.00999999999999999\\
364.01	0.00999999999999999\\
365.01	0.00999999999999999\\
366.01	0.00999999999999999\\
367.01	0.00999999999999999\\
368.01	0.00999999999999999\\
369.01	0.00999999999999999\\
370.01	0.00999999999999999\\
371.01	0.00999999999999999\\
372.01	0.00999999999999999\\
373.01	0.00999999999999999\\
374.01	0.00999999999999999\\
375.01	0.00999999999999999\\
376.01	0.00999999999999999\\
377.01	0.00999999999999999\\
378.01	0.00999999999999999\\
379.01	0.00999999999999999\\
380.01	0.00999999999999999\\
381.01	0.00999999999999999\\
382.01	0.00999999999999999\\
383.01	0.00999999999999999\\
384.01	0.00999999999999999\\
385.01	0.00999999999999999\\
386.01	0.00999999999999999\\
387.01	0.00999999999999999\\
388.01	0.00999999999999999\\
389.01	0.00999999999999999\\
390.01	0.00999999999999999\\
391.01	0.00999999999999999\\
392.01	0.00999999999999999\\
393.01	0.00999999999999999\\
394.01	0.00999999999999999\\
395.01	0.00999999999999999\\
396.01	0.00999999999999999\\
397.01	0.00999999999999999\\
398.01	0.00999999999999999\\
399.01	0.00999999999999999\\
400.01	0.00999999999999999\\
401.01	0.00999999999999999\\
402.01	0.00999999999999999\\
403.01	0.00999999999999999\\
404.01	0.00999999999999999\\
405.01	0.00999999999999999\\
406.01	0.00999999999999999\\
407.01	0.00999999999999999\\
408.01	0.00999999999999999\\
409.01	0.00999999999999999\\
410.01	0.00999999999999999\\
411.01	0.00999999999999999\\
412.01	0.00999999999999999\\
413.01	0.00999999999999999\\
414.01	0.00999999999999999\\
415.01	0.00999999999999999\\
416.01	0.00999999999999999\\
417.01	0.00999999999999999\\
418.01	0.00999999999999999\\
419.01	0.00999999999999999\\
420.01	0.00999999999999999\\
421.01	0.00999999999999999\\
422.01	0.00999999999999999\\
423.01	0.00999999999999999\\
424.01	0.00999999999999999\\
425.01	0.00999999999999999\\
426.01	0.00999999999999999\\
427.01	0.00999999999999999\\
428.01	0.00999999999999999\\
429.01	0.00999999999999999\\
430.01	0.00999999999999999\\
431.01	0.00999999999999999\\
432.01	0.00999999999999999\\
433.01	0.00999999999999999\\
434.01	0.00999999999999999\\
435.01	0.00999999999999999\\
436.01	0.00999999999999999\\
437.01	0.00999999999999999\\
438.01	0.00999999999999999\\
439.01	0.00999999999999999\\
440.01	0.00999999999999999\\
441.01	0.00999999999999999\\
442.01	0.00999999999999999\\
443.01	0.00999999999999999\\
444.01	0.00999999999999999\\
445.01	0.00999999999999999\\
446.01	0.00999999999999999\\
447.01	0.00999999999999999\\
448.01	0.00999999999999999\\
449.01	0.00999999999999999\\
450.01	0.00999999999999999\\
451.01	0.00999999999999999\\
452.01	0.00999999999999999\\
453.01	0.00999999999999999\\
454.01	0.00999999999999999\\
455.01	0.00999999999999999\\
456.01	0.00999999999999999\\
457.01	0.00999999999999999\\
458.01	0.00999999999999999\\
459.01	0.00999999999999999\\
460.01	0.00999999999999999\\
461.01	0.00999999999999999\\
462.01	0.00999999999999999\\
463.01	0.00999999999999999\\
464.01	0.00999999999999999\\
465.01	0.00999999999999999\\
466.01	0.00999999999999999\\
467.01	0.00999999999999999\\
468.01	0.00999999999999999\\
469.01	0.00999999999999999\\
470.01	0.00999999999999999\\
471.01	0.00999999999999999\\
472.01	0.00999999999999999\\
473.01	0.00999999999999999\\
474.01	0.00999999999999999\\
475.01	0.00999999999999999\\
476.01	0.00999999999999999\\
477.01	0.00999999999999999\\
478.01	0.00999999999999999\\
479.01	0.00999999999999999\\
480.01	0.00999999999999999\\
481.01	0.00999999999999999\\
482.01	0.00999999999999999\\
483.01	0.00999999999999999\\
484.01	0.00999999999999999\\
485.01	0.00999999999999999\\
486.01	0.00999999999999999\\
487.01	0.00999999999999999\\
488.01	0.00999999999999999\\
489.01	0.00999999999999999\\
490.01	0.00999999999999999\\
491.01	0.00999999999999999\\
492.01	0.00999999999999999\\
493.01	0.00999999999999999\\
494.01	0.00999999999999999\\
495.01	0.00999999999999999\\
496.01	0.00999999999999999\\
497.01	0.00999999999999999\\
498.01	0.00999999999999999\\
499.01	0.00999999999999999\\
500.01	0.00999999999999999\\
501.01	0.00999999999999999\\
502.01	0.00999999999999999\\
503.01	0.00999999999999999\\
504.01	0.00999999999999999\\
505.01	0.00999999999999999\\
506.01	0.00999999999999999\\
507.01	0.00999999999999999\\
508.01	0.00999999999999999\\
509.01	0.00999999999999999\\
510.01	0.00999999999999999\\
511.01	0.00999999999999999\\
512.01	0.00999999999999999\\
513.01	0.00999999999999999\\
514.01	0.00999999999999999\\
515.01	0.00999999999999999\\
516.01	0.00999999999999999\\
517.01	0.00999999999999999\\
518.01	0.00999999999999999\\
519.01	0.00999999999999999\\
520.01	0.00999999999999999\\
521.01	0.00999999999999999\\
522.01	0.00999999999999999\\
523.01	0.00999999999999999\\
524.01	0.00999999999999999\\
525.01	0.00999999999999999\\
526.01	0.00999999999999999\\
527.01	0.00999999999999999\\
528.01	0.00999999999999999\\
529.01	0.00999999999999999\\
530.01	0.00999999999999999\\
531.01	0.00999999999999999\\
532.01	0.00999999999999999\\
533.01	0.00999999999999999\\
534.01	0.00999999999999999\\
535.01	0.00999999999999999\\
536.01	0.00999999999999999\\
537.01	0.00999999999999999\\
538.01	0.00999999999999999\\
539.01	0.00999999999999999\\
540.01	0.00999999999999999\\
541.01	0.00999999999999999\\
542.01	0.00999999999999999\\
543.01	0.00999999999999999\\
544.01	0.00999999999999999\\
545.01	0.00999999999999999\\
546.01	0.00999999999999999\\
547.01	0.00999999999999999\\
548.01	0.00999999999999999\\
549.01	0.00999999999999999\\
550.01	0.00999999999999999\\
551.01	0.00999999999999999\\
552.01	0.00999999999999999\\
553.01	0.00999999999999999\\
554.01	0.00999999999999999\\
555.01	0.00999999999999999\\
556.01	0.00999999999999999\\
557.01	0.00999999999999999\\
558.01	0.00999999999999999\\
559.01	0.00999999999999999\\
560.01	0.00999999999999999\\
561.01	0.00999999999999999\\
562.01	0.00999999999999999\\
563.01	0.00999999999999999\\
564.01	0.00999999999999999\\
565.01	0.00999999999999999\\
566.01	0.00999999999999999\\
567.01	0.00999999999999999\\
568.01	0.00999999999999999\\
569.01	0.00999999999999999\\
570.01	0.00999999999999999\\
571.01	0.00999999999999999\\
572.01	0.00999999999999999\\
573.01	0.00999999999999999\\
574.01	0.00999999999999999\\
575.01	0.00999999999999999\\
576.01	0.00999999999999999\\
577.01	0.00999999999999999\\
578.01	0.00999999999999999\\
579.01	0.00999999999999999\\
580.01	0.00999999999999999\\
581.01	0.00999999999999999\\
582.01	0.00999999999999999\\
583.01	0.00999999999999999\\
584.01	0.00999999999999999\\
585.01	0.00999999999999999\\
586.01	0.00999999999999999\\
587.01	0.00999999999999999\\
588.01	0.00999999999999999\\
589.01	0.00999999999999999\\
590.01	0.00999999999999999\\
591.01	0.00999999999999999\\
592.01	0.00999999999999999\\
593.01	0.00999999999999999\\
594.01	0.00999999999999999\\
595.01	0.00999999999999999\\
596.01	0.00999999999999999\\
597.01	0.00951919512428425\\
598.01	0.00856167931814814\\
599.01	0.00614338742788381\\
599.02	0.00610585181164774\\
599.03	0.00606795853990919\\
599.04	0.00602970415859946\\
599.05	0.0059910851803116\\
599.06	0.00595209808397897\\
599.07	0.0059127393145506\\
599.08	0.00587300528266346\\
599.09	0.00583289236431155\\
599.1	0.00579239690051187\\
599.11	0.00575151519696708\\
599.12	0.00571024352372488\\
599.13	0.00566857811483437\\
599.14	0.00562651516799868\\
599.15	0.00558405084422466\\
599.16	0.00554118126746892\\
599.17	0.00549790252428058\\
599.18	0.00545421066344054\\
599.19	0.0054101016955973\\
599.2	0.00536557159289928\\
599.21	0.00532061628862352\\
599.22	0.00527523167680093\\
599.23	0.00522941361183779\\
599.24	0.00518315790813372\\
599.25	0.0051364603396959\\
599.26	0.00508931665044637\\
599.27	0.00504172255041567\\
599.28	0.00499367370841233\\
599.29	0.00494516575162714\\
599.3	0.0048961942652336\\
599.31	0.00484675479198465\\
599.32	0.00479684283180546\\
599.33	0.00474645384138228\\
599.34	0.00469558323374743\\
599.35	0.0046442263778603\\
599.36	0.00459237859818423\\
599.37	0.0045400351742595\\
599.38	0.00448719134027205\\
599.39	0.00443384228461821\\
599.4	0.00437998314946518\\
599.41	0.00432560903030727\\
599.42	0.00427071497551804\\
599.43	0.00421529598589787\\
599.44	0.0041593470142175\\
599.45	0.00410286296475698\\
599.46	0.00404583869284028\\
599.47	0.00398826900436542\\
599.48	0.00393014865533012\\
599.49	0.00387147235135288\\
599.5	0.00381223474718946\\
599.51	0.00375243044624475\\
599.52	0.00369205400007994\\
599.53	0.003631099907915\\
599.54	0.0035695626161263\\
599.55	0.00350743651773956\\
599.56	0.00344471595191778\\
599.57	0.00338139520344442\\
599.58	0.00331746850220145\\
599.59	0.00325293002264256\\
599.6	0.00318777388326122\\
599.61	0.0031219941460537\\
599.62	0.00305558481597687\\
599.63	0.00298853984040088\\
599.64	0.00292085310855656\\
599.65	0.00285251845097748\\
599.66	0.00278352963893677\\
599.67	0.00271388038387843\\
599.68	0.00264356433684325\\
599.69	0.00257257508788923\\
599.7	0.00250090616550637\\
599.71	0.00242855103602601\\
599.72	0.00235550310302429\\
599.73	0.00228175570672011\\
599.74	0.00220730212336716\\
599.75	0.00213213556464018\\
599.76	0.0020562491770154\\
599.77	0.00197963604114491\\
599.78	0.00190228917122515\\
599.79	0.00182420151435929\\
599.8	0.00174536594991352\\
599.81	0.00166577528886715\\
599.82	0.00158542227315652\\
599.83	0.00150429957501254\\
599.84	0.00142239979629203\\
599.85	0.00133971546780248\\
599.86	0.00125623904862046\\
599.87	0.00117196292540346\\
599.88	0.0010868794116951\\
599.89	0.00100098074722375\\
599.9	0.000914259097194385\\
599.91	0.000826706551573648\\
599.92	0.000738315124368099\\
599.93	0.00064907675289551\\
599.94	0.000558983297049216\\
599.95	0.000468026538555356\\
599.96	0.000376198180223065\\
599.97	0.000283489845187453\\
599.98	0.0001898930761453\\
599.99	9.53993345834923e-05\\
600	0\\
};
\addplot [color=mycolor20,solid,forget plot]
  table[row sep=crcr]{%
0.01	0.01\\
1.01	0.01\\
2.01	0.01\\
3.01	0.01\\
4.01	0.01\\
5.01	0.01\\
6.01	0.01\\
7.01	0.01\\
8.01	0.01\\
9.01	0.01\\
10.01	0.01\\
11.01	0.01\\
12.01	0.01\\
13.01	0.01\\
14.01	0.01\\
15.01	0.01\\
16.01	0.01\\
17.01	0.01\\
18.01	0.01\\
19.01	0.01\\
20.01	0.01\\
21.01	0.01\\
22.01	0.01\\
23.01	0.01\\
24.01	0.01\\
25.01	0.01\\
26.01	0.01\\
27.01	0.01\\
28.01	0.01\\
29.01	0.01\\
30.01	0.01\\
31.01	0.01\\
32.01	0.01\\
33.01	0.01\\
34.01	0.01\\
35.01	0.01\\
36.01	0.01\\
37.01	0.01\\
38.01	0.01\\
39.01	0.01\\
40.01	0.01\\
41.01	0.01\\
42.01	0.01\\
43.01	0.01\\
44.01	0.01\\
45.01	0.01\\
46.01	0.01\\
47.01	0.01\\
48.01	0.01\\
49.01	0.01\\
50.01	0.01\\
51.01	0.01\\
52.01	0.01\\
53.01	0.01\\
54.01	0.01\\
55.01	0.01\\
56.01	0.01\\
57.01	0.01\\
58.01	0.01\\
59.01	0.01\\
60.01	0.01\\
61.01	0.01\\
62.01	0.01\\
63.01	0.01\\
64.01	0.01\\
65.01	0.01\\
66.01	0.01\\
67.01	0.01\\
68.01	0.01\\
69.01	0.01\\
70.01	0.01\\
71.01	0.01\\
72.01	0.01\\
73.01	0.01\\
74.01	0.01\\
75.01	0.01\\
76.01	0.01\\
77.01	0.01\\
78.01	0.01\\
79.01	0.01\\
80.01	0.01\\
81.01	0.01\\
82.01	0.01\\
83.01	0.01\\
84.01	0.01\\
85.01	0.01\\
86.01	0.01\\
87.01	0.01\\
88.01	0.01\\
89.01	0.01\\
90.01	0.01\\
91.01	0.01\\
92.01	0.01\\
93.01	0.01\\
94.01	0.01\\
95.01	0.01\\
96.01	0.01\\
97.01	0.01\\
98.01	0.01\\
99.01	0.01\\
100.01	0.01\\
101.01	0.01\\
102.01	0.01\\
103.01	0.01\\
104.01	0.01\\
105.01	0.01\\
106.01	0.01\\
107.01	0.01\\
108.01	0.01\\
109.01	0.01\\
110.01	0.01\\
111.01	0.01\\
112.01	0.01\\
113.01	0.01\\
114.01	0.01\\
115.01	0.01\\
116.01	0.01\\
117.01	0.01\\
118.01	0.01\\
119.01	0.01\\
120.01	0.01\\
121.01	0.01\\
122.01	0.01\\
123.01	0.01\\
124.01	0.01\\
125.01	0.01\\
126.01	0.01\\
127.01	0.01\\
128.01	0.01\\
129.01	0.01\\
130.01	0.01\\
131.01	0.01\\
132.01	0.01\\
133.01	0.01\\
134.01	0.01\\
135.01	0.01\\
136.01	0.01\\
137.01	0.01\\
138.01	0.01\\
139.01	0.01\\
140.01	0.01\\
141.01	0.01\\
142.01	0.01\\
143.01	0.01\\
144.01	0.01\\
145.01	0.01\\
146.01	0.01\\
147.01	0.01\\
148.01	0.01\\
149.01	0.01\\
150.01	0.01\\
151.01	0.01\\
152.01	0.01\\
153.01	0.01\\
154.01	0.01\\
155.01	0.01\\
156.01	0.01\\
157.01	0.01\\
158.01	0.01\\
159.01	0.01\\
160.01	0.01\\
161.01	0.01\\
162.01	0.01\\
163.01	0.01\\
164.01	0.01\\
165.01	0.01\\
166.01	0.01\\
167.01	0.01\\
168.01	0.01\\
169.01	0.01\\
170.01	0.01\\
171.01	0.01\\
172.01	0.01\\
173.01	0.01\\
174.01	0.01\\
175.01	0.01\\
176.01	0.01\\
177.01	0.01\\
178.01	0.01\\
179.01	0.01\\
180.01	0.01\\
181.01	0.01\\
182.01	0.01\\
183.01	0.01\\
184.01	0.01\\
185.01	0.01\\
186.01	0.01\\
187.01	0.01\\
188.01	0.01\\
189.01	0.01\\
190.01	0.01\\
191.01	0.01\\
192.01	0.01\\
193.01	0.01\\
194.01	0.01\\
195.01	0.01\\
196.01	0.01\\
197.01	0.01\\
198.01	0.01\\
199.01	0.01\\
200.01	0.01\\
201.01	0.01\\
202.01	0.01\\
203.01	0.01\\
204.01	0.01\\
205.01	0.01\\
206.01	0.01\\
207.01	0.01\\
208.01	0.01\\
209.01	0.01\\
210.01	0.01\\
211.01	0.01\\
212.01	0.01\\
213.01	0.01\\
214.01	0.01\\
215.01	0.01\\
216.01	0.01\\
217.01	0.01\\
218.01	0.01\\
219.01	0.01\\
220.01	0.01\\
221.01	0.01\\
222.01	0.01\\
223.01	0.01\\
224.01	0.01\\
225.01	0.01\\
226.01	0.01\\
227.01	0.01\\
228.01	0.01\\
229.01	0.01\\
230.01	0.01\\
231.01	0.01\\
232.01	0.01\\
233.01	0.01\\
234.01	0.01\\
235.01	0.01\\
236.01	0.01\\
237.01	0.01\\
238.01	0.01\\
239.01	0.01\\
240.01	0.01\\
241.01	0.01\\
242.01	0.01\\
243.01	0.01\\
244.01	0.01\\
245.01	0.01\\
246.01	0.01\\
247.01	0.01\\
248.01	0.01\\
249.01	0.01\\
250.01	0.01\\
251.01	0.01\\
252.01	0.01\\
253.01	0.01\\
254.01	0.01\\
255.01	0.01\\
256.01	0.01\\
257.01	0.01\\
258.01	0.01\\
259.01	0.01\\
260.01	0.01\\
261.01	0.01\\
262.01	0.01\\
263.01	0.01\\
264.01	0.01\\
265.01	0.01\\
266.01	0.01\\
267.01	0.01\\
268.01	0.01\\
269.01	0.01\\
270.01	0.01\\
271.01	0.01\\
272.01	0.01\\
273.01	0.01\\
274.01	0.01\\
275.01	0.01\\
276.01	0.01\\
277.01	0.01\\
278.01	0.01\\
279.01	0.01\\
280.01	0.01\\
281.01	0.01\\
282.01	0.01\\
283.01	0.01\\
284.01	0.01\\
285.01	0.01\\
286.01	0.01\\
287.01	0.01\\
288.01	0.01\\
289.01	0.01\\
290.01	0.01\\
291.01	0.01\\
292.01	0.01\\
293.01	0.01\\
294.01	0.01\\
295.01	0.01\\
296.01	0.01\\
297.01	0.01\\
298.01	0.01\\
299.01	0.01\\
300.01	0.01\\
301.01	0.01\\
302.01	0.01\\
303.01	0.01\\
304.01	0.01\\
305.01	0.01\\
306.01	0.01\\
307.01	0.01\\
308.01	0.01\\
309.01	0.01\\
310.01	0.01\\
311.01	0.01\\
312.01	0.01\\
313.01	0.01\\
314.01	0.01\\
315.01	0.01\\
316.01	0.01\\
317.01	0.01\\
318.01	0.01\\
319.01	0.01\\
320.01	0.01\\
321.01	0.01\\
322.01	0.01\\
323.01	0.01\\
324.01	0.01\\
325.01	0.01\\
326.01	0.01\\
327.01	0.01\\
328.01	0.01\\
329.01	0.01\\
330.01	0.01\\
331.01	0.01\\
332.01	0.01\\
333.01	0.01\\
334.01	0.01\\
335.01	0.01\\
336.01	0.01\\
337.01	0.01\\
338.01	0.01\\
339.01	0.01\\
340.01	0.01\\
341.01	0.01\\
342.01	0.01\\
343.01	0.01\\
344.01	0.01\\
345.01	0.01\\
346.01	0.01\\
347.01	0.01\\
348.01	0.01\\
349.01	0.01\\
350.01	0.01\\
351.01	0.01\\
352.01	0.01\\
353.01	0.01\\
354.01	0.01\\
355.01	0.01\\
356.01	0.01\\
357.01	0.01\\
358.01	0.01\\
359.01	0.01\\
360.01	0.01\\
361.01	0.01\\
362.01	0.01\\
363.01	0.01\\
364.01	0.01\\
365.01	0.01\\
366.01	0.01\\
367.01	0.01\\
368.01	0.01\\
369.01	0.01\\
370.01	0.01\\
371.01	0.01\\
372.01	0.01\\
373.01	0.01\\
374.01	0.01\\
375.01	0.01\\
376.01	0.01\\
377.01	0.01\\
378.01	0.01\\
379.01	0.01\\
380.01	0.01\\
381.01	0.01\\
382.01	0.01\\
383.01	0.01\\
384.01	0.01\\
385.01	0.01\\
386.01	0.01\\
387.01	0.01\\
388.01	0.01\\
389.01	0.01\\
390.01	0.01\\
391.01	0.01\\
392.01	0.01\\
393.01	0.01\\
394.01	0.01\\
395.01	0.01\\
396.01	0.01\\
397.01	0.01\\
398.01	0.01\\
399.01	0.01\\
400.01	0.01\\
401.01	0.01\\
402.01	0.01\\
403.01	0.01\\
404.01	0.01\\
405.01	0.01\\
406.01	0.01\\
407.01	0.01\\
408.01	0.01\\
409.01	0.01\\
410.01	0.01\\
411.01	0.01\\
412.01	0.01\\
413.01	0.01\\
414.01	0.01\\
415.01	0.01\\
416.01	0.01\\
417.01	0.01\\
418.01	0.01\\
419.01	0.01\\
420.01	0.01\\
421.01	0.01\\
422.01	0.01\\
423.01	0.01\\
424.01	0.01\\
425.01	0.01\\
426.01	0.01\\
427.01	0.01\\
428.01	0.01\\
429.01	0.01\\
430.01	0.01\\
431.01	0.01\\
432.01	0.01\\
433.01	0.01\\
434.01	0.01\\
435.01	0.01\\
436.01	0.01\\
437.01	0.01\\
438.01	0.01\\
439.01	0.01\\
440.01	0.01\\
441.01	0.01\\
442.01	0.01\\
443.01	0.01\\
444.01	0.01\\
445.01	0.01\\
446.01	0.01\\
447.01	0.01\\
448.01	0.01\\
449.01	0.01\\
450.01	0.01\\
451.01	0.01\\
452.01	0.01\\
453.01	0.01\\
454.01	0.01\\
455.01	0.01\\
456.01	0.01\\
457.01	0.01\\
458.01	0.01\\
459.01	0.01\\
460.01	0.01\\
461.01	0.01\\
462.01	0.01\\
463.01	0.01\\
464.01	0.01\\
465.01	0.01\\
466.01	0.01\\
467.01	0.01\\
468.01	0.01\\
469.01	0.01\\
470.01	0.01\\
471.01	0.01\\
472.01	0.01\\
473.01	0.01\\
474.01	0.01\\
475.01	0.01\\
476.01	0.01\\
477.01	0.01\\
478.01	0.01\\
479.01	0.01\\
480.01	0.01\\
481.01	0.01\\
482.01	0.01\\
483.01	0.01\\
484.01	0.01\\
485.01	0.01\\
486.01	0.01\\
487.01	0.01\\
488.01	0.01\\
489.01	0.01\\
490.01	0.01\\
491.01	0.01\\
492.01	0.01\\
493.01	0.01\\
494.01	0.01\\
495.01	0.01\\
496.01	0.01\\
497.01	0.01\\
498.01	0.01\\
499.01	0.01\\
500.01	0.01\\
501.01	0.01\\
502.01	0.01\\
503.01	0.01\\
504.01	0.01\\
505.01	0.01\\
506.01	0.01\\
507.01	0.01\\
508.01	0.01\\
509.01	0.01\\
510.01	0.01\\
511.01	0.01\\
512.01	0.01\\
513.01	0.01\\
514.01	0.01\\
515.01	0.01\\
516.01	0.01\\
517.01	0.01\\
518.01	0.01\\
519.01	0.01\\
520.01	0.01\\
521.01	0.01\\
522.01	0.01\\
523.01	0.01\\
524.01	0.01\\
525.01	0.01\\
526.01	0.01\\
527.01	0.01\\
528.01	0.01\\
529.01	0.01\\
530.01	0.01\\
531.01	0.01\\
532.01	0.01\\
533.01	0.01\\
534.01	0.01\\
535.01	0.01\\
536.01	0.01\\
537.01	0.01\\
538.01	0.01\\
539.01	0.01\\
540.01	0.01\\
541.01	0.01\\
542.01	0.01\\
543.01	0.01\\
544.01	0.01\\
545.01	0.01\\
546.01	0.01\\
547.01	0.01\\
548.01	0.01\\
549.01	0.01\\
550.01	0.01\\
551.01	0.01\\
552.01	0.01\\
553.01	0.01\\
554.01	0.01\\
555.01	0.01\\
556.01	0.01\\
557.01	0.01\\
558.01	0.01\\
559.01	0.01\\
560.01	0.01\\
561.01	0.01\\
562.01	0.01\\
563.01	0.01\\
564.01	0.01\\
565.01	0.01\\
566.01	0.01\\
567.01	0.01\\
568.01	0.01\\
569.01	0.01\\
570.01	0.01\\
571.01	0.01\\
572.01	0.01\\
573.01	0.01\\
574.01	0.01\\
575.01	0.01\\
576.01	0.01\\
577.01	0.01\\
578.01	0.01\\
579.01	0.01\\
580.01	0.01\\
581.01	0.01\\
582.01	0.01\\
583.01	0.01\\
584.01	0.01\\
585.01	0.01\\
586.01	0.01\\
587.01	0.01\\
588.01	0.01\\
589.01	0.01\\
590.01	0.01\\
591.01	0.01\\
592.01	0.01\\
593.01	0.01\\
594.01	0.01\\
595.01	0.01\\
596.01	0.01\\
597.01	0.00951902522560497\\
598.01	0.00856167931814816\\
599.01	0.00614338742788375\\
599.02	0.00610585181164768\\
599.03	0.00606795853990914\\
599.04	0.00602970415859939\\
599.05	0.00599108518031152\\
599.06	0.00595209808397888\\
599.07	0.00591273931455051\\
599.08	0.00587300528266336\\
599.09	0.00583289236431146\\
599.1	0.00579239690051179\\
599.11	0.00575151519696698\\
599.12	0.0057102435237248\\
599.13	0.00566857811483427\\
599.14	0.00562651516799859\\
599.15	0.00558405084422458\\
599.16	0.00554118126746885\\
599.17	0.00549790252428052\\
599.18	0.00545421066344047\\
599.19	0.00541010169559724\\
599.2	0.00536557159289922\\
599.21	0.00532061628862348\\
599.22	0.00527523167680089\\
599.23	0.00522941361183776\\
599.24	0.00518315790813368\\
599.25	0.00513646033969586\\
599.26	0.00508931665044632\\
599.27	0.00504172255041562\\
599.28	0.00499367370841229\\
599.29	0.0049451657516271\\
599.3	0.00489619426523357\\
599.31	0.00484675479198462\\
599.32	0.00479684283180543\\
599.33	0.00474645384138224\\
599.34	0.00469558323374741\\
599.35	0.00464422637786027\\
599.36	0.0045923785981842\\
599.37	0.00454003517425947\\
599.38	0.00448719134027202\\
599.39	0.00443384228461819\\
599.4	0.00437998314946514\\
599.41	0.00432560903030725\\
599.42	0.004270714975518\\
599.43	0.00421529598589785\\
599.44	0.00415934701421748\\
599.45	0.00410286296475696\\
599.46	0.00404583869284025\\
599.47	0.00398826900436538\\
599.48	0.00393014865533008\\
599.49	0.00387147235135285\\
599.5	0.00381223474718943\\
599.51	0.00375243044624472\\
599.52	0.00369205400007992\\
599.53	0.00363109990791497\\
599.54	0.00356956261612627\\
599.55	0.00350743651773952\\
599.56	0.00344471595191776\\
599.57	0.00338139520344438\\
599.58	0.00331746850220141\\
599.59	0.00325293002264252\\
599.6	0.00318777388326119\\
599.61	0.00312199414605367\\
599.62	0.00305558481597684\\
599.63	0.00298853984040086\\
599.64	0.00292085310855653\\
599.65	0.00285251845097746\\
599.66	0.00278352963893675\\
599.67	0.00271388038387841\\
599.68	0.00264356433684324\\
599.69	0.00257257508788921\\
599.7	0.00250090616550636\\
599.71	0.00242855103602599\\
599.72	0.00235550310302428\\
599.73	0.0022817557067201\\
599.74	0.00220730212336715\\
599.75	0.00213213556464017\\
599.76	0.00205624917701539\\
599.77	0.0019796360411449\\
599.78	0.00190228917122514\\
599.79	0.00182420151435928\\
599.8	0.00174536594991351\\
599.81	0.00166577528886715\\
599.82	0.0015854222731565\\
599.83	0.00150429957501253\\
599.84	0.00142239979629202\\
599.85	0.00133971546780248\\
599.86	0.00125623904862046\\
599.87	0.00117196292540346\\
599.88	0.0010868794116951\\
599.89	0.00100098074722375\\
599.9	0.00091425909719438\\
599.91	0.000826706551573646\\
599.92	0.000738315124368097\\
599.93	0.000649076752895513\\
599.94	0.000558983297049212\\
599.95	0.000468026538555355\\
599.96	0.000376198180223065\\
599.97	0.000283489845187451\\
599.98	0.0001898930761453\\
599.99	9.53993345834923e-05\\
600	0\\
};
\addplot [color=mycolor21,solid,forget plot]
  table[row sep=crcr]{%
0.01	0.01\\
1.01	0.01\\
2.01	0.01\\
3.01	0.01\\
4.01	0.01\\
5.01	0.01\\
6.01	0.01\\
7.01	0.01\\
8.01	0.01\\
9.01	0.01\\
10.01	0.01\\
11.01	0.01\\
12.01	0.01\\
13.01	0.01\\
14.01	0.01\\
15.01	0.01\\
16.01	0.01\\
17.01	0.01\\
18.01	0.01\\
19.01	0.01\\
20.01	0.01\\
21.01	0.01\\
22.01	0.01\\
23.01	0.01\\
24.01	0.01\\
25.01	0.01\\
26.01	0.01\\
27.01	0.01\\
28.01	0.01\\
29.01	0.01\\
30.01	0.01\\
31.01	0.01\\
32.01	0.01\\
33.01	0.01\\
34.01	0.01\\
35.01	0.01\\
36.01	0.01\\
37.01	0.01\\
38.01	0.01\\
39.01	0.01\\
40.01	0.01\\
41.01	0.01\\
42.01	0.01\\
43.01	0.01\\
44.01	0.01\\
45.01	0.01\\
46.01	0.01\\
47.01	0.01\\
48.01	0.01\\
49.01	0.01\\
50.01	0.01\\
51.01	0.01\\
52.01	0.01\\
53.01	0.01\\
54.01	0.01\\
55.01	0.01\\
56.01	0.01\\
57.01	0.01\\
58.01	0.01\\
59.01	0.01\\
60.01	0.01\\
61.01	0.01\\
62.01	0.01\\
63.01	0.01\\
64.01	0.01\\
65.01	0.01\\
66.01	0.01\\
67.01	0.01\\
68.01	0.01\\
69.01	0.01\\
70.01	0.01\\
71.01	0.01\\
72.01	0.01\\
73.01	0.01\\
74.01	0.01\\
75.01	0.01\\
76.01	0.01\\
77.01	0.01\\
78.01	0.01\\
79.01	0.01\\
80.01	0.01\\
81.01	0.01\\
82.01	0.01\\
83.01	0.01\\
84.01	0.01\\
85.01	0.01\\
86.01	0.01\\
87.01	0.01\\
88.01	0.01\\
89.01	0.01\\
90.01	0.01\\
91.01	0.01\\
92.01	0.01\\
93.01	0.01\\
94.01	0.01\\
95.01	0.01\\
96.01	0.01\\
97.01	0.01\\
98.01	0.01\\
99.01	0.01\\
100.01	0.01\\
101.01	0.01\\
102.01	0.01\\
103.01	0.01\\
104.01	0.01\\
105.01	0.01\\
106.01	0.01\\
107.01	0.01\\
108.01	0.01\\
109.01	0.01\\
110.01	0.01\\
111.01	0.01\\
112.01	0.01\\
113.01	0.01\\
114.01	0.01\\
115.01	0.01\\
116.01	0.01\\
117.01	0.01\\
118.01	0.01\\
119.01	0.01\\
120.01	0.01\\
121.01	0.01\\
122.01	0.01\\
123.01	0.01\\
124.01	0.01\\
125.01	0.01\\
126.01	0.01\\
127.01	0.01\\
128.01	0.01\\
129.01	0.01\\
130.01	0.01\\
131.01	0.01\\
132.01	0.01\\
133.01	0.01\\
134.01	0.01\\
135.01	0.01\\
136.01	0.01\\
137.01	0.01\\
138.01	0.01\\
139.01	0.01\\
140.01	0.01\\
141.01	0.01\\
142.01	0.01\\
143.01	0.01\\
144.01	0.01\\
145.01	0.01\\
146.01	0.01\\
147.01	0.01\\
148.01	0.01\\
149.01	0.01\\
150.01	0.01\\
151.01	0.01\\
152.01	0.01\\
153.01	0.01\\
154.01	0.01\\
155.01	0.01\\
156.01	0.01\\
157.01	0.01\\
158.01	0.01\\
159.01	0.01\\
160.01	0.01\\
161.01	0.01\\
162.01	0.01\\
163.01	0.01\\
164.01	0.01\\
165.01	0.01\\
166.01	0.01\\
167.01	0.01\\
168.01	0.01\\
169.01	0.01\\
170.01	0.01\\
171.01	0.01\\
172.01	0.01\\
173.01	0.01\\
174.01	0.01\\
175.01	0.01\\
176.01	0.01\\
177.01	0.01\\
178.01	0.01\\
179.01	0.01\\
180.01	0.01\\
181.01	0.01\\
182.01	0.01\\
183.01	0.01\\
184.01	0.01\\
185.01	0.01\\
186.01	0.01\\
187.01	0.01\\
188.01	0.01\\
189.01	0.01\\
190.01	0.01\\
191.01	0.01\\
192.01	0.01\\
193.01	0.01\\
194.01	0.01\\
195.01	0.01\\
196.01	0.01\\
197.01	0.01\\
198.01	0.01\\
199.01	0.01\\
200.01	0.01\\
201.01	0.01\\
202.01	0.01\\
203.01	0.01\\
204.01	0.01\\
205.01	0.01\\
206.01	0.01\\
207.01	0.01\\
208.01	0.01\\
209.01	0.01\\
210.01	0.01\\
211.01	0.01\\
212.01	0.01\\
213.01	0.01\\
214.01	0.01\\
215.01	0.01\\
216.01	0.01\\
217.01	0.01\\
218.01	0.01\\
219.01	0.01\\
220.01	0.01\\
221.01	0.01\\
222.01	0.01\\
223.01	0.01\\
224.01	0.01\\
225.01	0.01\\
226.01	0.01\\
227.01	0.01\\
228.01	0.01\\
229.01	0.01\\
230.01	0.01\\
231.01	0.01\\
232.01	0.01\\
233.01	0.01\\
234.01	0.01\\
235.01	0.01\\
236.01	0.01\\
237.01	0.01\\
238.01	0.01\\
239.01	0.01\\
240.01	0.01\\
241.01	0.01\\
242.01	0.01\\
243.01	0.01\\
244.01	0.01\\
245.01	0.01\\
246.01	0.01\\
247.01	0.01\\
248.01	0.01\\
249.01	0.01\\
250.01	0.01\\
251.01	0.01\\
252.01	0.01\\
253.01	0.01\\
254.01	0.01\\
255.01	0.01\\
256.01	0.01\\
257.01	0.01\\
258.01	0.01\\
259.01	0.01\\
260.01	0.01\\
261.01	0.01\\
262.01	0.01\\
263.01	0.01\\
264.01	0.01\\
265.01	0.01\\
266.01	0.01\\
267.01	0.01\\
268.01	0.01\\
269.01	0.01\\
270.01	0.01\\
271.01	0.01\\
272.01	0.01\\
273.01	0.01\\
274.01	0.01\\
275.01	0.01\\
276.01	0.01\\
277.01	0.01\\
278.01	0.01\\
279.01	0.01\\
280.01	0.01\\
281.01	0.01\\
282.01	0.01\\
283.01	0.01\\
284.01	0.01\\
285.01	0.01\\
286.01	0.01\\
287.01	0.01\\
288.01	0.01\\
289.01	0.01\\
290.01	0.01\\
291.01	0.01\\
292.01	0.01\\
293.01	0.01\\
294.01	0.01\\
295.01	0.01\\
296.01	0.01\\
297.01	0.01\\
298.01	0.01\\
299.01	0.01\\
300.01	0.01\\
301.01	0.01\\
302.01	0.01\\
303.01	0.01\\
304.01	0.01\\
305.01	0.01\\
306.01	0.01\\
307.01	0.01\\
308.01	0.01\\
309.01	0.01\\
310.01	0.01\\
311.01	0.01\\
312.01	0.01\\
313.01	0.01\\
314.01	0.01\\
315.01	0.01\\
316.01	0.01\\
317.01	0.01\\
318.01	0.01\\
319.01	0.01\\
320.01	0.01\\
321.01	0.01\\
322.01	0.01\\
323.01	0.01\\
324.01	0.01\\
325.01	0.01\\
326.01	0.01\\
327.01	0.01\\
328.01	0.01\\
329.01	0.01\\
330.01	0.01\\
331.01	0.01\\
332.01	0.01\\
333.01	0.01\\
334.01	0.01\\
335.01	0.01\\
336.01	0.01\\
337.01	0.01\\
338.01	0.01\\
339.01	0.01\\
340.01	0.01\\
341.01	0.01\\
342.01	0.01\\
343.01	0.01\\
344.01	0.01\\
345.01	0.01\\
346.01	0.01\\
347.01	0.01\\
348.01	0.01\\
349.01	0.01\\
350.01	0.01\\
351.01	0.01\\
352.01	0.01\\
353.01	0.01\\
354.01	0.01\\
355.01	0.01\\
356.01	0.01\\
357.01	0.01\\
358.01	0.01\\
359.01	0.01\\
360.01	0.01\\
361.01	0.01\\
362.01	0.01\\
363.01	0.01\\
364.01	0.01\\
365.01	0.01\\
366.01	0.01\\
367.01	0.01\\
368.01	0.01\\
369.01	0.01\\
370.01	0.01\\
371.01	0.01\\
372.01	0.01\\
373.01	0.01\\
374.01	0.01\\
375.01	0.01\\
376.01	0.01\\
377.01	0.01\\
378.01	0.01\\
379.01	0.01\\
380.01	0.01\\
381.01	0.01\\
382.01	0.01\\
383.01	0.01\\
384.01	0.01\\
385.01	0.01\\
386.01	0.01\\
387.01	0.01\\
388.01	0.01\\
389.01	0.01\\
390.01	0.01\\
391.01	0.01\\
392.01	0.01\\
393.01	0.01\\
394.01	0.01\\
395.01	0.01\\
396.01	0.01\\
397.01	0.01\\
398.01	0.01\\
399.01	0.01\\
400.01	0.01\\
401.01	0.01\\
402.01	0.01\\
403.01	0.01\\
404.01	0.01\\
405.01	0.01\\
406.01	0.01\\
407.01	0.01\\
408.01	0.01\\
409.01	0.01\\
410.01	0.01\\
411.01	0.01\\
412.01	0.01\\
413.01	0.01\\
414.01	0.01\\
415.01	0.01\\
416.01	0.01\\
417.01	0.01\\
418.01	0.01\\
419.01	0.01\\
420.01	0.01\\
421.01	0.01\\
422.01	0.01\\
423.01	0.01\\
424.01	0.01\\
425.01	0.01\\
426.01	0.01\\
427.01	0.01\\
428.01	0.01\\
429.01	0.01\\
430.01	0.01\\
431.01	0.01\\
432.01	0.01\\
433.01	0.01\\
434.01	0.01\\
435.01	0.01\\
436.01	0.01\\
437.01	0.01\\
438.01	0.01\\
439.01	0.01\\
440.01	0.01\\
441.01	0.01\\
442.01	0.01\\
443.01	0.01\\
444.01	0.01\\
445.01	0.01\\
446.01	0.01\\
447.01	0.01\\
448.01	0.01\\
449.01	0.01\\
450.01	0.01\\
451.01	0.01\\
452.01	0.01\\
453.01	0.01\\
454.01	0.01\\
455.01	0.01\\
456.01	0.01\\
457.01	0.01\\
458.01	0.01\\
459.01	0.01\\
460.01	0.01\\
461.01	0.01\\
462.01	0.01\\
463.01	0.01\\
464.01	0.01\\
465.01	0.01\\
466.01	0.01\\
467.01	0.01\\
468.01	0.01\\
469.01	0.01\\
470.01	0.01\\
471.01	0.01\\
472.01	0.01\\
473.01	0.01\\
474.01	0.01\\
475.01	0.01\\
476.01	0.01\\
477.01	0.01\\
478.01	0.01\\
479.01	0.01\\
480.01	0.01\\
481.01	0.01\\
482.01	0.01\\
483.01	0.01\\
484.01	0.01\\
485.01	0.01\\
486.01	0.01\\
487.01	0.01\\
488.01	0.01\\
489.01	0.01\\
490.01	0.01\\
491.01	0.01\\
492.01	0.01\\
493.01	0.01\\
494.01	0.01\\
495.01	0.01\\
496.01	0.01\\
497.01	0.01\\
498.01	0.01\\
499.01	0.01\\
500.01	0.01\\
501.01	0.01\\
502.01	0.01\\
503.01	0.01\\
504.01	0.01\\
505.01	0.01\\
506.01	0.01\\
507.01	0.01\\
508.01	0.01\\
509.01	0.01\\
510.01	0.01\\
511.01	0.01\\
512.01	0.01\\
513.01	0.01\\
514.01	0.01\\
515.01	0.01\\
516.01	0.01\\
517.01	0.01\\
518.01	0.01\\
519.01	0.01\\
520.01	0.01\\
521.01	0.01\\
522.01	0.01\\
523.01	0.01\\
524.01	0.01\\
525.01	0.01\\
526.01	0.01\\
527.01	0.01\\
528.01	0.01\\
529.01	0.01\\
530.01	0.01\\
531.01	0.01\\
532.01	0.01\\
533.01	0.01\\
534.01	0.01\\
535.01	0.01\\
536.01	0.01\\
537.01	0.01\\
538.01	0.01\\
539.01	0.01\\
540.01	0.01\\
541.01	0.01\\
542.01	0.01\\
543.01	0.01\\
544.01	0.01\\
545.01	0.01\\
546.01	0.01\\
547.01	0.01\\
548.01	0.01\\
549.01	0.01\\
550.01	0.01\\
551.01	0.01\\
552.01	0.01\\
553.01	0.01\\
554.01	0.01\\
555.01	0.01\\
556.01	0.01\\
557.01	0.01\\
558.01	0.01\\
559.01	0.01\\
560.01	0.01\\
561.01	0.01\\
562.01	0.01\\
563.01	0.01\\
564.01	0.01\\
565.01	0.01\\
566.01	0.01\\
567.01	0.01\\
568.01	0.01\\
569.01	0.01\\
570.01	0.01\\
571.01	0.01\\
572.01	0.01\\
573.01	0.01\\
574.01	0.01\\
575.01	0.01\\
576.01	0.01\\
577.01	0.01\\
578.01	0.01\\
579.01	0.01\\
580.01	0.01\\
581.01	0.01\\
582.01	0.01\\
583.01	0.01\\
584.01	0.01\\
585.01	0.01\\
586.01	0.01\\
587.01	0.01\\
588.01	0.01\\
589.01	0.01\\
590.01	0.01\\
591.01	0.01\\
592.01	0.01\\
593.01	0.01\\
594.01	0.01\\
595.01	0.01\\
596.01	0.01\\
597.01	0.00951888910102859\\
598.01	0.00856167931814817\\
599.01	0.00614338742788371\\
599.02	0.00610585181164764\\
599.03	0.00606795853990911\\
599.04	0.00602970415859938\\
599.05	0.00599108518031152\\
599.06	0.00595209808397889\\
599.07	0.00591273931455052\\
599.08	0.00587300528266337\\
599.09	0.00583289236431148\\
599.1	0.0057923969005118\\
599.11	0.00575151519696699\\
599.12	0.00571024352372481\\
599.13	0.0056685781148343\\
599.14	0.00562651516799862\\
599.15	0.00558405084422461\\
599.16	0.00554118126746887\\
599.17	0.00549790252428052\\
599.18	0.00545421066344048\\
599.19	0.00541010169559724\\
599.2	0.00536557159289922\\
599.21	0.00532061628862347\\
599.22	0.00527523167680088\\
599.23	0.00522941361183774\\
599.24	0.00518315790813368\\
599.25	0.00513646033969587\\
599.26	0.00508931665044633\\
599.27	0.00504172255041564\\
599.28	0.0049936737084123\\
599.29	0.0049451657516271\\
599.3	0.00489619426523356\\
599.31	0.00484675479198462\\
599.32	0.00479684283180542\\
599.33	0.00474645384138223\\
599.34	0.00469558323374739\\
599.35	0.00464422637786025\\
599.36	0.00459237859818419\\
599.37	0.00454003517425945\\
599.38	0.00448719134027201\\
599.39	0.00443384228461817\\
599.4	0.00437998314946513\\
599.41	0.00432560903030724\\
599.42	0.004270714975518\\
599.43	0.00421529598589783\\
599.44	0.00415934701421747\\
599.45	0.00410286296475694\\
599.46	0.00404583869284024\\
599.47	0.00398826900436538\\
599.48	0.00393014865533008\\
599.49	0.00387147235135285\\
599.5	0.00381223474718942\\
599.51	0.00375243044624472\\
599.52	0.00369205400007992\\
599.53	0.00363109990791498\\
599.54	0.00356956261612629\\
599.55	0.00350743651773955\\
599.56	0.00344471595191778\\
599.57	0.00338139520344441\\
599.58	0.00331746850220144\\
599.59	0.00325293002264255\\
599.6	0.00318777388326121\\
599.61	0.00312199414605369\\
599.62	0.00305558481597687\\
599.63	0.00298853984040089\\
599.64	0.00292085310855656\\
599.65	0.00285251845097748\\
599.66	0.00278352963893678\\
599.67	0.00271388038387844\\
599.68	0.00264356433684326\\
599.69	0.00257257508788923\\
599.7	0.00250090616550638\\
599.71	0.00242855103602601\\
599.72	0.00235550310302429\\
599.73	0.00228175570672011\\
599.74	0.00220730212336716\\
599.75	0.00213213556464018\\
599.76	0.0020562491770154\\
599.77	0.00197963604114491\\
599.78	0.00190228917122515\\
599.79	0.00182420151435929\\
599.8	0.00174536594991353\\
599.81	0.00166577528886715\\
599.82	0.00158542227315651\\
599.83	0.00150429957501254\\
599.84	0.00142239979629202\\
599.85	0.00133971546780248\\
599.86	0.00125623904862047\\
599.87	0.00117196292540346\\
599.88	0.00108687941169511\\
599.89	0.00100098074722376\\
599.9	0.000914259097194383\\
599.91	0.000826706551573646\\
599.92	0.000738315124368097\\
599.93	0.00064907675289551\\
599.94	0.000558983297049212\\
599.95	0.000468026538555356\\
599.96	0.000376198180223067\\
599.97	0.000283489845187451\\
599.98	0.0001898930761453\\
599.99	9.53993345834923e-05\\
600	0\\
};
\addplot [color=black!20!mycolor21,solid,forget plot]
  table[row sep=crcr]{%
0.01	0.01\\
1.01	0.01\\
2.01	0.01\\
3.01	0.01\\
4.01	0.01\\
5.01	0.01\\
6.01	0.01\\
7.01	0.01\\
8.01	0.01\\
9.01	0.01\\
10.01	0.01\\
11.01	0.01\\
12.01	0.01\\
13.01	0.01\\
14.01	0.01\\
15.01	0.01\\
16.01	0.01\\
17.01	0.01\\
18.01	0.01\\
19.01	0.01\\
20.01	0.01\\
21.01	0.01\\
22.01	0.01\\
23.01	0.01\\
24.01	0.01\\
25.01	0.01\\
26.01	0.01\\
27.01	0.01\\
28.01	0.01\\
29.01	0.01\\
30.01	0.01\\
31.01	0.01\\
32.01	0.01\\
33.01	0.01\\
34.01	0.01\\
35.01	0.01\\
36.01	0.01\\
37.01	0.01\\
38.01	0.01\\
39.01	0.01\\
40.01	0.01\\
41.01	0.01\\
42.01	0.01\\
43.01	0.01\\
44.01	0.01\\
45.01	0.01\\
46.01	0.01\\
47.01	0.01\\
48.01	0.01\\
49.01	0.01\\
50.01	0.01\\
51.01	0.01\\
52.01	0.01\\
53.01	0.01\\
54.01	0.01\\
55.01	0.01\\
56.01	0.01\\
57.01	0.01\\
58.01	0.01\\
59.01	0.01\\
60.01	0.01\\
61.01	0.01\\
62.01	0.01\\
63.01	0.01\\
64.01	0.01\\
65.01	0.01\\
66.01	0.01\\
67.01	0.01\\
68.01	0.01\\
69.01	0.01\\
70.01	0.01\\
71.01	0.01\\
72.01	0.01\\
73.01	0.01\\
74.01	0.01\\
75.01	0.01\\
76.01	0.01\\
77.01	0.01\\
78.01	0.01\\
79.01	0.01\\
80.01	0.01\\
81.01	0.01\\
82.01	0.01\\
83.01	0.01\\
84.01	0.01\\
85.01	0.01\\
86.01	0.01\\
87.01	0.01\\
88.01	0.01\\
89.01	0.01\\
90.01	0.01\\
91.01	0.01\\
92.01	0.01\\
93.01	0.01\\
94.01	0.01\\
95.01	0.01\\
96.01	0.01\\
97.01	0.01\\
98.01	0.01\\
99.01	0.01\\
100.01	0.01\\
101.01	0.01\\
102.01	0.01\\
103.01	0.01\\
104.01	0.01\\
105.01	0.01\\
106.01	0.01\\
107.01	0.01\\
108.01	0.01\\
109.01	0.01\\
110.01	0.01\\
111.01	0.01\\
112.01	0.01\\
113.01	0.01\\
114.01	0.01\\
115.01	0.01\\
116.01	0.01\\
117.01	0.01\\
118.01	0.01\\
119.01	0.01\\
120.01	0.01\\
121.01	0.01\\
122.01	0.01\\
123.01	0.01\\
124.01	0.01\\
125.01	0.01\\
126.01	0.01\\
127.01	0.01\\
128.01	0.01\\
129.01	0.01\\
130.01	0.01\\
131.01	0.01\\
132.01	0.01\\
133.01	0.01\\
134.01	0.01\\
135.01	0.01\\
136.01	0.01\\
137.01	0.01\\
138.01	0.01\\
139.01	0.01\\
140.01	0.01\\
141.01	0.01\\
142.01	0.01\\
143.01	0.01\\
144.01	0.01\\
145.01	0.01\\
146.01	0.01\\
147.01	0.01\\
148.01	0.01\\
149.01	0.01\\
150.01	0.01\\
151.01	0.01\\
152.01	0.01\\
153.01	0.01\\
154.01	0.01\\
155.01	0.01\\
156.01	0.01\\
157.01	0.01\\
158.01	0.01\\
159.01	0.01\\
160.01	0.01\\
161.01	0.01\\
162.01	0.01\\
163.01	0.01\\
164.01	0.01\\
165.01	0.01\\
166.01	0.01\\
167.01	0.01\\
168.01	0.01\\
169.01	0.01\\
170.01	0.01\\
171.01	0.01\\
172.01	0.01\\
173.01	0.01\\
174.01	0.01\\
175.01	0.01\\
176.01	0.01\\
177.01	0.01\\
178.01	0.01\\
179.01	0.01\\
180.01	0.01\\
181.01	0.01\\
182.01	0.01\\
183.01	0.01\\
184.01	0.01\\
185.01	0.01\\
186.01	0.01\\
187.01	0.01\\
188.01	0.01\\
189.01	0.01\\
190.01	0.01\\
191.01	0.01\\
192.01	0.01\\
193.01	0.01\\
194.01	0.01\\
195.01	0.01\\
196.01	0.01\\
197.01	0.01\\
198.01	0.01\\
199.01	0.01\\
200.01	0.01\\
201.01	0.01\\
202.01	0.01\\
203.01	0.01\\
204.01	0.01\\
205.01	0.01\\
206.01	0.01\\
207.01	0.01\\
208.01	0.01\\
209.01	0.01\\
210.01	0.01\\
211.01	0.01\\
212.01	0.01\\
213.01	0.01\\
214.01	0.01\\
215.01	0.01\\
216.01	0.01\\
217.01	0.01\\
218.01	0.01\\
219.01	0.01\\
220.01	0.01\\
221.01	0.01\\
222.01	0.01\\
223.01	0.01\\
224.01	0.01\\
225.01	0.01\\
226.01	0.01\\
227.01	0.01\\
228.01	0.01\\
229.01	0.01\\
230.01	0.01\\
231.01	0.01\\
232.01	0.01\\
233.01	0.01\\
234.01	0.01\\
235.01	0.01\\
236.01	0.01\\
237.01	0.01\\
238.01	0.01\\
239.01	0.01\\
240.01	0.01\\
241.01	0.01\\
242.01	0.01\\
243.01	0.01\\
244.01	0.01\\
245.01	0.01\\
246.01	0.01\\
247.01	0.01\\
248.01	0.01\\
249.01	0.01\\
250.01	0.01\\
251.01	0.01\\
252.01	0.01\\
253.01	0.01\\
254.01	0.01\\
255.01	0.01\\
256.01	0.01\\
257.01	0.01\\
258.01	0.01\\
259.01	0.01\\
260.01	0.01\\
261.01	0.01\\
262.01	0.01\\
263.01	0.01\\
264.01	0.01\\
265.01	0.01\\
266.01	0.01\\
267.01	0.01\\
268.01	0.01\\
269.01	0.01\\
270.01	0.01\\
271.01	0.01\\
272.01	0.01\\
273.01	0.01\\
274.01	0.01\\
275.01	0.01\\
276.01	0.01\\
277.01	0.01\\
278.01	0.01\\
279.01	0.01\\
280.01	0.01\\
281.01	0.01\\
282.01	0.01\\
283.01	0.01\\
284.01	0.01\\
285.01	0.01\\
286.01	0.01\\
287.01	0.01\\
288.01	0.01\\
289.01	0.01\\
290.01	0.01\\
291.01	0.01\\
292.01	0.01\\
293.01	0.01\\
294.01	0.01\\
295.01	0.01\\
296.01	0.01\\
297.01	0.01\\
298.01	0.01\\
299.01	0.01\\
300.01	0.01\\
301.01	0.01\\
302.01	0.01\\
303.01	0.01\\
304.01	0.01\\
305.01	0.01\\
306.01	0.01\\
307.01	0.01\\
308.01	0.01\\
309.01	0.01\\
310.01	0.01\\
311.01	0.01\\
312.01	0.01\\
313.01	0.01\\
314.01	0.01\\
315.01	0.01\\
316.01	0.01\\
317.01	0.01\\
318.01	0.01\\
319.01	0.01\\
320.01	0.01\\
321.01	0.01\\
322.01	0.01\\
323.01	0.01\\
324.01	0.01\\
325.01	0.01\\
326.01	0.01\\
327.01	0.01\\
328.01	0.01\\
329.01	0.01\\
330.01	0.01\\
331.01	0.01\\
332.01	0.01\\
333.01	0.01\\
334.01	0.01\\
335.01	0.01\\
336.01	0.01\\
337.01	0.01\\
338.01	0.01\\
339.01	0.01\\
340.01	0.01\\
341.01	0.01\\
342.01	0.01\\
343.01	0.01\\
344.01	0.01\\
345.01	0.01\\
346.01	0.01\\
347.01	0.01\\
348.01	0.01\\
349.01	0.01\\
350.01	0.01\\
351.01	0.01\\
352.01	0.01\\
353.01	0.01\\
354.01	0.01\\
355.01	0.01\\
356.01	0.01\\
357.01	0.01\\
358.01	0.01\\
359.01	0.01\\
360.01	0.01\\
361.01	0.01\\
362.01	0.01\\
363.01	0.01\\
364.01	0.01\\
365.01	0.01\\
366.01	0.01\\
367.01	0.01\\
368.01	0.01\\
369.01	0.01\\
370.01	0.01\\
371.01	0.01\\
372.01	0.01\\
373.01	0.01\\
374.01	0.01\\
375.01	0.01\\
376.01	0.01\\
377.01	0.01\\
378.01	0.01\\
379.01	0.01\\
380.01	0.01\\
381.01	0.01\\
382.01	0.01\\
383.01	0.01\\
384.01	0.01\\
385.01	0.01\\
386.01	0.01\\
387.01	0.01\\
388.01	0.01\\
389.01	0.01\\
390.01	0.01\\
391.01	0.01\\
392.01	0.01\\
393.01	0.01\\
394.01	0.01\\
395.01	0.01\\
396.01	0.01\\
397.01	0.01\\
398.01	0.01\\
399.01	0.01\\
400.01	0.01\\
401.01	0.01\\
402.01	0.01\\
403.01	0.01\\
404.01	0.01\\
405.01	0.01\\
406.01	0.01\\
407.01	0.01\\
408.01	0.01\\
409.01	0.01\\
410.01	0.01\\
411.01	0.01\\
412.01	0.01\\
413.01	0.01\\
414.01	0.01\\
415.01	0.01\\
416.01	0.01\\
417.01	0.01\\
418.01	0.01\\
419.01	0.01\\
420.01	0.01\\
421.01	0.01\\
422.01	0.01\\
423.01	0.01\\
424.01	0.01\\
425.01	0.01\\
426.01	0.01\\
427.01	0.01\\
428.01	0.01\\
429.01	0.01\\
430.01	0.01\\
431.01	0.01\\
432.01	0.01\\
433.01	0.01\\
434.01	0.01\\
435.01	0.01\\
436.01	0.01\\
437.01	0.01\\
438.01	0.01\\
439.01	0.01\\
440.01	0.01\\
441.01	0.01\\
442.01	0.01\\
443.01	0.01\\
444.01	0.01\\
445.01	0.01\\
446.01	0.01\\
447.01	0.01\\
448.01	0.01\\
449.01	0.01\\
450.01	0.01\\
451.01	0.01\\
452.01	0.01\\
453.01	0.01\\
454.01	0.01\\
455.01	0.01\\
456.01	0.01\\
457.01	0.01\\
458.01	0.01\\
459.01	0.01\\
460.01	0.01\\
461.01	0.01\\
462.01	0.01\\
463.01	0.01\\
464.01	0.01\\
465.01	0.01\\
466.01	0.01\\
467.01	0.01\\
468.01	0.01\\
469.01	0.01\\
470.01	0.01\\
471.01	0.01\\
472.01	0.01\\
473.01	0.01\\
474.01	0.01\\
475.01	0.01\\
476.01	0.01\\
477.01	0.01\\
478.01	0.01\\
479.01	0.01\\
480.01	0.01\\
481.01	0.01\\
482.01	0.01\\
483.01	0.01\\
484.01	0.01\\
485.01	0.01\\
486.01	0.01\\
487.01	0.01\\
488.01	0.01\\
489.01	0.01\\
490.01	0.01\\
491.01	0.01\\
492.01	0.01\\
493.01	0.01\\
494.01	0.01\\
495.01	0.01\\
496.01	0.01\\
497.01	0.01\\
498.01	0.01\\
499.01	0.01\\
500.01	0.01\\
501.01	0.01\\
502.01	0.01\\
503.01	0.01\\
504.01	0.01\\
505.01	0.01\\
506.01	0.01\\
507.01	0.01\\
508.01	0.01\\
509.01	0.01\\
510.01	0.01\\
511.01	0.01\\
512.01	0.01\\
513.01	0.01\\
514.01	0.01\\
515.01	0.01\\
516.01	0.01\\
517.01	0.01\\
518.01	0.01\\
519.01	0.01\\
520.01	0.01\\
521.01	0.01\\
522.01	0.01\\
523.01	0.01\\
524.01	0.01\\
525.01	0.01\\
526.01	0.01\\
527.01	0.01\\
528.01	0.01\\
529.01	0.01\\
530.01	0.01\\
531.01	0.01\\
532.01	0.01\\
533.01	0.01\\
534.01	0.01\\
535.01	0.01\\
536.01	0.01\\
537.01	0.01\\
538.01	0.01\\
539.01	0.01\\
540.01	0.01\\
541.01	0.01\\
542.01	0.01\\
543.01	0.01\\
544.01	0.01\\
545.01	0.01\\
546.01	0.01\\
547.01	0.01\\
548.01	0.01\\
549.01	0.01\\
550.01	0.01\\
551.01	0.01\\
552.01	0.01\\
553.01	0.01\\
554.01	0.01\\
555.01	0.01\\
556.01	0.01\\
557.01	0.01\\
558.01	0.01\\
559.01	0.01\\
560.01	0.01\\
561.01	0.01\\
562.01	0.01\\
563.01	0.01\\
564.01	0.01\\
565.01	0.01\\
566.01	0.01\\
567.01	0.01\\
568.01	0.01\\
569.01	0.01\\
570.01	0.01\\
571.01	0.01\\
572.01	0.01\\
573.01	0.01\\
574.01	0.01\\
575.01	0.01\\
576.01	0.01\\
577.01	0.01\\
578.01	0.01\\
579.01	0.01\\
580.01	0.01\\
581.01	0.01\\
582.01	0.01\\
583.01	0.01\\
584.01	0.01\\
585.01	0.01\\
586.01	0.01\\
587.01	0.01\\
588.01	0.01\\
589.01	0.01\\
590.01	0.01\\
591.01	0.01\\
592.01	0.01\\
593.01	0.01\\
594.01	0.01\\
595.01	0.01\\
596.01	0.01\\
597.01	0.00951881251103901\\
598.01	0.00856167931814826\\
599.01	0.00614338742788385\\
599.02	0.00610585181164778\\
599.03	0.00606795853990924\\
599.04	0.0060297041585995\\
599.05	0.00599108518031163\\
599.06	0.00595209808397899\\
599.07	0.00591273931455062\\
599.08	0.00587300528266347\\
599.09	0.00583289236431157\\
599.1	0.0057923969005119\\
599.11	0.00575151519696709\\
599.12	0.0057102435237249\\
599.13	0.00566857811483437\\
599.14	0.00562651516799868\\
599.15	0.00558405084422466\\
599.16	0.00554118126746892\\
599.17	0.00549790252428058\\
599.18	0.00545421066344055\\
599.19	0.00541010169559731\\
599.2	0.00536557159289929\\
599.21	0.00532061628862354\\
599.22	0.00527523167680093\\
599.23	0.00522941361183779\\
599.24	0.00518315790813371\\
599.25	0.00513646033969589\\
599.26	0.00508931665044635\\
599.27	0.00504172255041565\\
599.28	0.00499367370841233\\
599.29	0.00494516575162713\\
599.3	0.0048961942652336\\
599.31	0.00484675479198466\\
599.32	0.00479684283180548\\
599.33	0.00474645384138229\\
599.34	0.00469558323374744\\
599.35	0.00464422637786031\\
599.36	0.00459237859818424\\
599.37	0.0045400351742595\\
599.38	0.00448719134027204\\
599.39	0.00443384228461821\\
599.4	0.00437998314946518\\
599.41	0.00432560903030729\\
599.42	0.00427071497551804\\
599.43	0.00421529598589787\\
599.44	0.0041593470142175\\
599.45	0.00410286296475698\\
599.46	0.00404583869284028\\
599.47	0.00398826900436542\\
599.48	0.00393014865533013\\
599.49	0.00387147235135289\\
599.5	0.00381223474718947\\
599.51	0.00375243044624475\\
599.52	0.00369205400007994\\
599.53	0.003631099907915\\
599.54	0.0035695626161263\\
599.55	0.00350743651773956\\
599.56	0.00344471595191779\\
599.57	0.00338139520344443\\
599.58	0.00331746850220145\\
599.59	0.00325293002264255\\
599.6	0.00318777388326122\\
599.61	0.0031219941460537\\
599.62	0.00305558481597687\\
599.63	0.00298853984040088\\
599.64	0.00292085310855656\\
599.65	0.00285251845097748\\
599.66	0.00278352963893677\\
599.67	0.00271388038387843\\
599.68	0.00264356433684325\\
599.69	0.00257257508788923\\
599.7	0.00250090616550638\\
599.71	0.00242855103602601\\
599.72	0.00235550310302429\\
599.73	0.00228175570672011\\
599.74	0.00220730212336715\\
599.75	0.00213213556464018\\
599.76	0.00205624917701539\\
599.77	0.00197963604114491\\
599.78	0.00190228917122514\\
599.79	0.00182420151435929\\
599.8	0.00174536594991352\\
599.81	0.00166577528886715\\
599.82	0.00158542227315651\\
599.83	0.00150429957501253\\
599.84	0.00142239979629203\\
599.85	0.00133971546780248\\
599.86	0.00125623904862047\\
599.87	0.00117196292540346\\
599.88	0.0010868794116951\\
599.89	0.00100098074722375\\
599.9	0.000914259097194385\\
599.91	0.000826706551573646\\
599.92	0.000738315124368097\\
599.93	0.000649076752895513\\
599.94	0.000558983297049216\\
599.95	0.000468026538555355\\
599.96	0.000376198180223069\\
599.97	0.000283489845187451\\
599.98	0.000189893076145302\\
599.99	9.53993345834923e-05\\
600	0\\
};
\addplot [color=black!50!mycolor20,solid,forget plot]
  table[row sep=crcr]{%
0.01	0.01\\
1.01	0.01\\
2.01	0.01\\
3.01	0.01\\
4.01	0.01\\
5.01	0.01\\
6.01	0.01\\
7.01	0.01\\
8.01	0.01\\
9.01	0.01\\
10.01	0.01\\
11.01	0.01\\
12.01	0.01\\
13.01	0.01\\
14.01	0.01\\
15.01	0.01\\
16.01	0.01\\
17.01	0.01\\
18.01	0.01\\
19.01	0.01\\
20.01	0.01\\
21.01	0.01\\
22.01	0.01\\
23.01	0.01\\
24.01	0.01\\
25.01	0.01\\
26.01	0.01\\
27.01	0.01\\
28.01	0.01\\
29.01	0.01\\
30.01	0.01\\
31.01	0.01\\
32.01	0.01\\
33.01	0.01\\
34.01	0.01\\
35.01	0.01\\
36.01	0.01\\
37.01	0.01\\
38.01	0.01\\
39.01	0.01\\
40.01	0.01\\
41.01	0.01\\
42.01	0.01\\
43.01	0.01\\
44.01	0.01\\
45.01	0.01\\
46.01	0.01\\
47.01	0.01\\
48.01	0.01\\
49.01	0.01\\
50.01	0.01\\
51.01	0.01\\
52.01	0.01\\
53.01	0.01\\
54.01	0.01\\
55.01	0.01\\
56.01	0.01\\
57.01	0.01\\
58.01	0.01\\
59.01	0.01\\
60.01	0.01\\
61.01	0.01\\
62.01	0.01\\
63.01	0.01\\
64.01	0.01\\
65.01	0.01\\
66.01	0.01\\
67.01	0.01\\
68.01	0.01\\
69.01	0.01\\
70.01	0.01\\
71.01	0.01\\
72.01	0.01\\
73.01	0.01\\
74.01	0.01\\
75.01	0.01\\
76.01	0.01\\
77.01	0.01\\
78.01	0.01\\
79.01	0.01\\
80.01	0.01\\
81.01	0.01\\
82.01	0.01\\
83.01	0.01\\
84.01	0.01\\
85.01	0.01\\
86.01	0.01\\
87.01	0.01\\
88.01	0.01\\
89.01	0.01\\
90.01	0.01\\
91.01	0.01\\
92.01	0.01\\
93.01	0.01\\
94.01	0.01\\
95.01	0.01\\
96.01	0.01\\
97.01	0.01\\
98.01	0.01\\
99.01	0.01\\
100.01	0.01\\
101.01	0.01\\
102.01	0.01\\
103.01	0.01\\
104.01	0.01\\
105.01	0.01\\
106.01	0.01\\
107.01	0.01\\
108.01	0.01\\
109.01	0.01\\
110.01	0.01\\
111.01	0.01\\
112.01	0.01\\
113.01	0.01\\
114.01	0.01\\
115.01	0.01\\
116.01	0.01\\
117.01	0.01\\
118.01	0.01\\
119.01	0.01\\
120.01	0.01\\
121.01	0.01\\
122.01	0.01\\
123.01	0.01\\
124.01	0.01\\
125.01	0.01\\
126.01	0.01\\
127.01	0.01\\
128.01	0.01\\
129.01	0.01\\
130.01	0.01\\
131.01	0.01\\
132.01	0.01\\
133.01	0.01\\
134.01	0.01\\
135.01	0.01\\
136.01	0.01\\
137.01	0.01\\
138.01	0.01\\
139.01	0.01\\
140.01	0.01\\
141.01	0.01\\
142.01	0.01\\
143.01	0.01\\
144.01	0.01\\
145.01	0.01\\
146.01	0.01\\
147.01	0.01\\
148.01	0.01\\
149.01	0.01\\
150.01	0.01\\
151.01	0.01\\
152.01	0.01\\
153.01	0.01\\
154.01	0.01\\
155.01	0.01\\
156.01	0.01\\
157.01	0.01\\
158.01	0.01\\
159.01	0.01\\
160.01	0.01\\
161.01	0.01\\
162.01	0.01\\
163.01	0.01\\
164.01	0.01\\
165.01	0.01\\
166.01	0.01\\
167.01	0.01\\
168.01	0.01\\
169.01	0.01\\
170.01	0.01\\
171.01	0.01\\
172.01	0.01\\
173.01	0.01\\
174.01	0.01\\
175.01	0.01\\
176.01	0.01\\
177.01	0.01\\
178.01	0.01\\
179.01	0.01\\
180.01	0.01\\
181.01	0.01\\
182.01	0.01\\
183.01	0.01\\
184.01	0.01\\
185.01	0.01\\
186.01	0.01\\
187.01	0.01\\
188.01	0.01\\
189.01	0.01\\
190.01	0.01\\
191.01	0.01\\
192.01	0.01\\
193.01	0.01\\
194.01	0.01\\
195.01	0.01\\
196.01	0.01\\
197.01	0.01\\
198.01	0.01\\
199.01	0.01\\
200.01	0.01\\
201.01	0.01\\
202.01	0.01\\
203.01	0.01\\
204.01	0.01\\
205.01	0.01\\
206.01	0.01\\
207.01	0.01\\
208.01	0.01\\
209.01	0.01\\
210.01	0.01\\
211.01	0.01\\
212.01	0.01\\
213.01	0.01\\
214.01	0.01\\
215.01	0.01\\
216.01	0.01\\
217.01	0.01\\
218.01	0.01\\
219.01	0.01\\
220.01	0.01\\
221.01	0.01\\
222.01	0.01\\
223.01	0.01\\
224.01	0.01\\
225.01	0.01\\
226.01	0.01\\
227.01	0.01\\
228.01	0.01\\
229.01	0.01\\
230.01	0.01\\
231.01	0.01\\
232.01	0.01\\
233.01	0.01\\
234.01	0.01\\
235.01	0.01\\
236.01	0.01\\
237.01	0.01\\
238.01	0.01\\
239.01	0.01\\
240.01	0.01\\
241.01	0.01\\
242.01	0.01\\
243.01	0.01\\
244.01	0.01\\
245.01	0.01\\
246.01	0.01\\
247.01	0.01\\
248.01	0.01\\
249.01	0.01\\
250.01	0.01\\
251.01	0.01\\
252.01	0.01\\
253.01	0.01\\
254.01	0.01\\
255.01	0.01\\
256.01	0.01\\
257.01	0.01\\
258.01	0.01\\
259.01	0.01\\
260.01	0.01\\
261.01	0.01\\
262.01	0.01\\
263.01	0.01\\
264.01	0.01\\
265.01	0.01\\
266.01	0.01\\
267.01	0.01\\
268.01	0.01\\
269.01	0.01\\
270.01	0.01\\
271.01	0.01\\
272.01	0.01\\
273.01	0.01\\
274.01	0.01\\
275.01	0.01\\
276.01	0.01\\
277.01	0.01\\
278.01	0.01\\
279.01	0.01\\
280.01	0.01\\
281.01	0.01\\
282.01	0.01\\
283.01	0.01\\
284.01	0.01\\
285.01	0.01\\
286.01	0.01\\
287.01	0.01\\
288.01	0.01\\
289.01	0.01\\
290.01	0.01\\
291.01	0.01\\
292.01	0.01\\
293.01	0.01\\
294.01	0.01\\
295.01	0.01\\
296.01	0.01\\
297.01	0.01\\
298.01	0.01\\
299.01	0.01\\
300.01	0.01\\
301.01	0.01\\
302.01	0.01\\
303.01	0.01\\
304.01	0.01\\
305.01	0.01\\
306.01	0.01\\
307.01	0.01\\
308.01	0.01\\
309.01	0.01\\
310.01	0.01\\
311.01	0.01\\
312.01	0.01\\
313.01	0.01\\
314.01	0.01\\
315.01	0.01\\
316.01	0.01\\
317.01	0.01\\
318.01	0.01\\
319.01	0.01\\
320.01	0.01\\
321.01	0.01\\
322.01	0.01\\
323.01	0.01\\
324.01	0.01\\
325.01	0.01\\
326.01	0.01\\
327.01	0.01\\
328.01	0.01\\
329.01	0.01\\
330.01	0.01\\
331.01	0.01\\
332.01	0.01\\
333.01	0.01\\
334.01	0.01\\
335.01	0.01\\
336.01	0.01\\
337.01	0.01\\
338.01	0.01\\
339.01	0.01\\
340.01	0.01\\
341.01	0.01\\
342.01	0.01\\
343.01	0.01\\
344.01	0.01\\
345.01	0.01\\
346.01	0.01\\
347.01	0.01\\
348.01	0.01\\
349.01	0.01\\
350.01	0.01\\
351.01	0.01\\
352.01	0.01\\
353.01	0.01\\
354.01	0.01\\
355.01	0.01\\
356.01	0.01\\
357.01	0.01\\
358.01	0.01\\
359.01	0.01\\
360.01	0.01\\
361.01	0.01\\
362.01	0.01\\
363.01	0.01\\
364.01	0.01\\
365.01	0.01\\
366.01	0.01\\
367.01	0.01\\
368.01	0.01\\
369.01	0.01\\
370.01	0.01\\
371.01	0.01\\
372.01	0.01\\
373.01	0.01\\
374.01	0.01\\
375.01	0.01\\
376.01	0.01\\
377.01	0.01\\
378.01	0.01\\
379.01	0.01\\
380.01	0.01\\
381.01	0.01\\
382.01	0.01\\
383.01	0.01\\
384.01	0.01\\
385.01	0.01\\
386.01	0.01\\
387.01	0.01\\
388.01	0.01\\
389.01	0.01\\
390.01	0.01\\
391.01	0.01\\
392.01	0.01\\
393.01	0.01\\
394.01	0.01\\
395.01	0.01\\
396.01	0.01\\
397.01	0.01\\
398.01	0.01\\
399.01	0.01\\
400.01	0.01\\
401.01	0.01\\
402.01	0.01\\
403.01	0.01\\
404.01	0.01\\
405.01	0.01\\
406.01	0.01\\
407.01	0.01\\
408.01	0.01\\
409.01	0.01\\
410.01	0.01\\
411.01	0.01\\
412.01	0.01\\
413.01	0.01\\
414.01	0.01\\
415.01	0.01\\
416.01	0.01\\
417.01	0.01\\
418.01	0.01\\
419.01	0.01\\
420.01	0.01\\
421.01	0.01\\
422.01	0.01\\
423.01	0.01\\
424.01	0.01\\
425.01	0.01\\
426.01	0.01\\
427.01	0.01\\
428.01	0.01\\
429.01	0.01\\
430.01	0.01\\
431.01	0.01\\
432.01	0.01\\
433.01	0.01\\
434.01	0.01\\
435.01	0.01\\
436.01	0.01\\
437.01	0.01\\
438.01	0.01\\
439.01	0.01\\
440.01	0.01\\
441.01	0.01\\
442.01	0.01\\
443.01	0.01\\
444.01	0.01\\
445.01	0.01\\
446.01	0.01\\
447.01	0.01\\
448.01	0.01\\
449.01	0.01\\
450.01	0.01\\
451.01	0.01\\
452.01	0.01\\
453.01	0.01\\
454.01	0.01\\
455.01	0.01\\
456.01	0.01\\
457.01	0.01\\
458.01	0.01\\
459.01	0.01\\
460.01	0.01\\
461.01	0.01\\
462.01	0.01\\
463.01	0.01\\
464.01	0.01\\
465.01	0.01\\
466.01	0.01\\
467.01	0.01\\
468.01	0.01\\
469.01	0.01\\
470.01	0.01\\
471.01	0.01\\
472.01	0.01\\
473.01	0.01\\
474.01	0.01\\
475.01	0.01\\
476.01	0.01\\
477.01	0.01\\
478.01	0.01\\
479.01	0.01\\
480.01	0.01\\
481.01	0.01\\
482.01	0.01\\
483.01	0.01\\
484.01	0.01\\
485.01	0.01\\
486.01	0.01\\
487.01	0.01\\
488.01	0.01\\
489.01	0.01\\
490.01	0.01\\
491.01	0.01\\
492.01	0.01\\
493.01	0.01\\
494.01	0.01\\
495.01	0.01\\
496.01	0.01\\
497.01	0.01\\
498.01	0.01\\
499.01	0.01\\
500.01	0.01\\
501.01	0.01\\
502.01	0.01\\
503.01	0.01\\
504.01	0.01\\
505.01	0.01\\
506.01	0.01\\
507.01	0.01\\
508.01	0.01\\
509.01	0.01\\
510.01	0.01\\
511.01	0.01\\
512.01	0.01\\
513.01	0.01\\
514.01	0.01\\
515.01	0.01\\
516.01	0.01\\
517.01	0.01\\
518.01	0.01\\
519.01	0.01\\
520.01	0.01\\
521.01	0.01\\
522.01	0.01\\
523.01	0.01\\
524.01	0.01\\
525.01	0.01\\
526.01	0.01\\
527.01	0.01\\
528.01	0.01\\
529.01	0.01\\
530.01	0.01\\
531.01	0.01\\
532.01	0.01\\
533.01	0.01\\
534.01	0.01\\
535.01	0.01\\
536.01	0.01\\
537.01	0.01\\
538.01	0.01\\
539.01	0.01\\
540.01	0.01\\
541.01	0.01\\
542.01	0.01\\
543.01	0.01\\
544.01	0.01\\
545.01	0.01\\
546.01	0.01\\
547.01	0.01\\
548.01	0.01\\
549.01	0.01\\
550.01	0.01\\
551.01	0.01\\
552.01	0.01\\
553.01	0.01\\
554.01	0.01\\
555.01	0.01\\
556.01	0.01\\
557.01	0.01\\
558.01	0.01\\
559.01	0.01\\
560.01	0.01\\
561.01	0.01\\
562.01	0.01\\
563.01	0.01\\
564.01	0.01\\
565.01	0.01\\
566.01	0.01\\
567.01	0.01\\
568.01	0.01\\
569.01	0.01\\
570.01	0.01\\
571.01	0.01\\
572.01	0.01\\
573.01	0.01\\
574.01	0.01\\
575.01	0.01\\
576.01	0.01\\
577.01	0.01\\
578.01	0.01\\
579.01	0.01\\
580.01	0.01\\
581.01	0.01\\
582.01	0.01\\
583.01	0.01\\
584.01	0.01\\
585.01	0.01\\
586.01	0.01\\
587.01	0.01\\
588.01	0.01\\
589.01	0.01\\
590.01	0.01\\
591.01	0.01\\
592.01	0.01\\
593.01	0.01\\
594.01	0.01\\
595.01	0.01\\
596.01	0.01\\
597.01	0.00951875017326085\\
598.01	0.00856167931814806\\
599.01	0.00614338742788374\\
599.02	0.00610585181164767\\
599.03	0.00606795853990912\\
599.04	0.00602970415859939\\
599.05	0.00599108518031153\\
599.06	0.0059520980839789\\
599.07	0.00591273931455053\\
599.08	0.00587300528266339\\
599.09	0.00583289236431148\\
599.1	0.0057923969005118\\
599.11	0.00575151519696701\\
599.12	0.00571024352372483\\
599.13	0.00566857811483431\\
599.14	0.00562651516799864\\
599.15	0.00558405084422462\\
599.16	0.00554118126746888\\
599.17	0.00549790252428053\\
599.18	0.00545421066344048\\
599.19	0.00541010169559724\\
599.2	0.00536557159289921\\
599.21	0.00532061628862347\\
599.22	0.00527523167680088\\
599.23	0.00522941361183774\\
599.24	0.00518315790813368\\
599.25	0.00513646033969586\\
599.26	0.00508931665044631\\
599.27	0.00504172255041562\\
599.28	0.00499367370841229\\
599.29	0.00494516575162709\\
599.3	0.00489619426523356\\
599.31	0.00484675479198462\\
599.32	0.00479684283180542\\
599.33	0.00474645384138225\\
599.34	0.00469558323374741\\
599.35	0.00464422637786027\\
599.36	0.0045923785981842\\
599.37	0.00454003517425946\\
599.38	0.00448719134027202\\
599.39	0.00443384228461818\\
599.4	0.00437998314946513\\
599.41	0.00432560903030723\\
599.42	0.004270714975518\\
599.43	0.00421529598589784\\
599.44	0.00415934701421748\\
599.45	0.00410286296475696\\
599.46	0.00404583869284025\\
599.47	0.00398826900436539\\
599.48	0.00393014865533009\\
599.49	0.00387147235135285\\
599.5	0.00381223474718943\\
599.51	0.00375243044624472\\
599.52	0.00369205400007992\\
599.53	0.00363109990791497\\
599.54	0.00356956261612628\\
599.55	0.00350743651773954\\
599.56	0.00344471595191777\\
599.57	0.00338139520344439\\
599.58	0.00331746850220142\\
599.59	0.00325293002264254\\
599.6	0.0031877738832612\\
599.61	0.00312199414605368\\
599.62	0.00305558481597686\\
599.63	0.00298853984040087\\
599.64	0.00292085310855655\\
599.65	0.00285251845097747\\
599.66	0.00278352963893676\\
599.67	0.00271388038387843\\
599.68	0.00264356433684325\\
599.69	0.00257257508788922\\
599.7	0.00250090616550637\\
599.71	0.00242855103602601\\
599.72	0.0023555031030243\\
599.73	0.00228175570672012\\
599.74	0.00220730212336716\\
599.75	0.00213213556464018\\
599.76	0.0020562491770154\\
599.77	0.00197963604114491\\
599.78	0.00190228917122515\\
599.79	0.00182420151435929\\
599.8	0.00174536594991352\\
599.81	0.00166577528886716\\
599.82	0.00158542227315651\\
599.83	0.00150429957501254\\
599.84	0.00142239979629202\\
599.85	0.00133971546780248\\
599.86	0.00125623904862047\\
599.87	0.00117196292540347\\
599.88	0.00108687941169511\\
599.89	0.00100098074722376\\
599.9	0.000914259097194387\\
599.91	0.000826706551573644\\
599.92	0.000738315124368095\\
599.93	0.000649076752895513\\
599.94	0.000558983297049218\\
599.95	0.000468026538555351\\
599.96	0.000376198180223063\\
599.97	0.000283489845187451\\
599.98	0.0001898930761453\\
599.99	9.53993345834923e-05\\
600	0\\
};
\addplot [color=black!60!mycolor21,solid,forget plot]
  table[row sep=crcr]{%
0.01	0.01\\
1.01	0.01\\
2.01	0.01\\
3.01	0.01\\
4.01	0.01\\
5.01	0.01\\
6.01	0.01\\
7.01	0.01\\
8.01	0.01\\
9.01	0.01\\
10.01	0.01\\
11.01	0.01\\
12.01	0.01\\
13.01	0.01\\
14.01	0.01\\
15.01	0.01\\
16.01	0.01\\
17.01	0.01\\
18.01	0.01\\
19.01	0.01\\
20.01	0.01\\
21.01	0.01\\
22.01	0.01\\
23.01	0.01\\
24.01	0.01\\
25.01	0.01\\
26.01	0.01\\
27.01	0.01\\
28.01	0.01\\
29.01	0.01\\
30.01	0.01\\
31.01	0.01\\
32.01	0.01\\
33.01	0.01\\
34.01	0.01\\
35.01	0.01\\
36.01	0.01\\
37.01	0.01\\
38.01	0.01\\
39.01	0.01\\
40.01	0.01\\
41.01	0.01\\
42.01	0.01\\
43.01	0.01\\
44.01	0.01\\
45.01	0.01\\
46.01	0.01\\
47.01	0.01\\
48.01	0.01\\
49.01	0.01\\
50.01	0.01\\
51.01	0.01\\
52.01	0.01\\
53.01	0.01\\
54.01	0.01\\
55.01	0.01\\
56.01	0.01\\
57.01	0.01\\
58.01	0.01\\
59.01	0.01\\
60.01	0.01\\
61.01	0.01\\
62.01	0.01\\
63.01	0.01\\
64.01	0.01\\
65.01	0.01\\
66.01	0.01\\
67.01	0.01\\
68.01	0.01\\
69.01	0.01\\
70.01	0.01\\
71.01	0.01\\
72.01	0.01\\
73.01	0.01\\
74.01	0.01\\
75.01	0.01\\
76.01	0.01\\
77.01	0.01\\
78.01	0.01\\
79.01	0.01\\
80.01	0.01\\
81.01	0.01\\
82.01	0.01\\
83.01	0.01\\
84.01	0.01\\
85.01	0.01\\
86.01	0.01\\
87.01	0.01\\
88.01	0.01\\
89.01	0.01\\
90.01	0.01\\
91.01	0.01\\
92.01	0.01\\
93.01	0.01\\
94.01	0.01\\
95.01	0.01\\
96.01	0.01\\
97.01	0.01\\
98.01	0.01\\
99.01	0.01\\
100.01	0.01\\
101.01	0.01\\
102.01	0.01\\
103.01	0.01\\
104.01	0.01\\
105.01	0.01\\
106.01	0.01\\
107.01	0.01\\
108.01	0.01\\
109.01	0.01\\
110.01	0.01\\
111.01	0.01\\
112.01	0.01\\
113.01	0.01\\
114.01	0.01\\
115.01	0.01\\
116.01	0.01\\
117.01	0.01\\
118.01	0.01\\
119.01	0.01\\
120.01	0.01\\
121.01	0.01\\
122.01	0.01\\
123.01	0.01\\
124.01	0.01\\
125.01	0.01\\
126.01	0.01\\
127.01	0.01\\
128.01	0.01\\
129.01	0.01\\
130.01	0.01\\
131.01	0.01\\
132.01	0.01\\
133.01	0.01\\
134.01	0.01\\
135.01	0.01\\
136.01	0.01\\
137.01	0.01\\
138.01	0.01\\
139.01	0.01\\
140.01	0.01\\
141.01	0.01\\
142.01	0.01\\
143.01	0.01\\
144.01	0.01\\
145.01	0.01\\
146.01	0.01\\
147.01	0.01\\
148.01	0.01\\
149.01	0.01\\
150.01	0.01\\
151.01	0.01\\
152.01	0.01\\
153.01	0.01\\
154.01	0.01\\
155.01	0.01\\
156.01	0.01\\
157.01	0.01\\
158.01	0.01\\
159.01	0.01\\
160.01	0.01\\
161.01	0.01\\
162.01	0.01\\
163.01	0.01\\
164.01	0.01\\
165.01	0.01\\
166.01	0.01\\
167.01	0.01\\
168.01	0.01\\
169.01	0.01\\
170.01	0.01\\
171.01	0.01\\
172.01	0.01\\
173.01	0.01\\
174.01	0.01\\
175.01	0.01\\
176.01	0.01\\
177.01	0.01\\
178.01	0.01\\
179.01	0.01\\
180.01	0.01\\
181.01	0.01\\
182.01	0.01\\
183.01	0.01\\
184.01	0.01\\
185.01	0.01\\
186.01	0.01\\
187.01	0.01\\
188.01	0.01\\
189.01	0.01\\
190.01	0.01\\
191.01	0.01\\
192.01	0.01\\
193.01	0.01\\
194.01	0.01\\
195.01	0.01\\
196.01	0.01\\
197.01	0.01\\
198.01	0.01\\
199.01	0.01\\
200.01	0.01\\
201.01	0.01\\
202.01	0.01\\
203.01	0.01\\
204.01	0.01\\
205.01	0.01\\
206.01	0.01\\
207.01	0.01\\
208.01	0.01\\
209.01	0.01\\
210.01	0.01\\
211.01	0.01\\
212.01	0.01\\
213.01	0.01\\
214.01	0.01\\
215.01	0.01\\
216.01	0.01\\
217.01	0.01\\
218.01	0.01\\
219.01	0.01\\
220.01	0.01\\
221.01	0.01\\
222.01	0.01\\
223.01	0.01\\
224.01	0.01\\
225.01	0.01\\
226.01	0.01\\
227.01	0.01\\
228.01	0.01\\
229.01	0.01\\
230.01	0.01\\
231.01	0.01\\
232.01	0.01\\
233.01	0.01\\
234.01	0.01\\
235.01	0.01\\
236.01	0.01\\
237.01	0.01\\
238.01	0.01\\
239.01	0.01\\
240.01	0.01\\
241.01	0.01\\
242.01	0.01\\
243.01	0.01\\
244.01	0.01\\
245.01	0.01\\
246.01	0.01\\
247.01	0.01\\
248.01	0.01\\
249.01	0.01\\
250.01	0.01\\
251.01	0.01\\
252.01	0.01\\
253.01	0.01\\
254.01	0.01\\
255.01	0.01\\
256.01	0.01\\
257.01	0.01\\
258.01	0.01\\
259.01	0.01\\
260.01	0.01\\
261.01	0.01\\
262.01	0.01\\
263.01	0.01\\
264.01	0.01\\
265.01	0.01\\
266.01	0.01\\
267.01	0.01\\
268.01	0.01\\
269.01	0.01\\
270.01	0.01\\
271.01	0.01\\
272.01	0.01\\
273.01	0.01\\
274.01	0.01\\
275.01	0.01\\
276.01	0.01\\
277.01	0.01\\
278.01	0.01\\
279.01	0.01\\
280.01	0.01\\
281.01	0.01\\
282.01	0.01\\
283.01	0.01\\
284.01	0.01\\
285.01	0.01\\
286.01	0.01\\
287.01	0.01\\
288.01	0.01\\
289.01	0.01\\
290.01	0.01\\
291.01	0.01\\
292.01	0.01\\
293.01	0.01\\
294.01	0.01\\
295.01	0.01\\
296.01	0.01\\
297.01	0.01\\
298.01	0.01\\
299.01	0.01\\
300.01	0.01\\
301.01	0.01\\
302.01	0.01\\
303.01	0.01\\
304.01	0.01\\
305.01	0.01\\
306.01	0.01\\
307.01	0.01\\
308.01	0.01\\
309.01	0.01\\
310.01	0.01\\
311.01	0.01\\
312.01	0.01\\
313.01	0.01\\
314.01	0.01\\
315.01	0.01\\
316.01	0.01\\
317.01	0.01\\
318.01	0.01\\
319.01	0.01\\
320.01	0.01\\
321.01	0.01\\
322.01	0.01\\
323.01	0.01\\
324.01	0.01\\
325.01	0.01\\
326.01	0.01\\
327.01	0.01\\
328.01	0.01\\
329.01	0.01\\
330.01	0.01\\
331.01	0.01\\
332.01	0.01\\
333.01	0.01\\
334.01	0.01\\
335.01	0.01\\
336.01	0.01\\
337.01	0.01\\
338.01	0.01\\
339.01	0.01\\
340.01	0.01\\
341.01	0.01\\
342.01	0.01\\
343.01	0.01\\
344.01	0.01\\
345.01	0.01\\
346.01	0.01\\
347.01	0.01\\
348.01	0.01\\
349.01	0.01\\
350.01	0.01\\
351.01	0.01\\
352.01	0.01\\
353.01	0.01\\
354.01	0.01\\
355.01	0.01\\
356.01	0.01\\
357.01	0.01\\
358.01	0.01\\
359.01	0.01\\
360.01	0.01\\
361.01	0.01\\
362.01	0.01\\
363.01	0.01\\
364.01	0.01\\
365.01	0.01\\
366.01	0.01\\
367.01	0.01\\
368.01	0.01\\
369.01	0.01\\
370.01	0.01\\
371.01	0.01\\
372.01	0.01\\
373.01	0.01\\
374.01	0.01\\
375.01	0.01\\
376.01	0.01\\
377.01	0.01\\
378.01	0.01\\
379.01	0.01\\
380.01	0.01\\
381.01	0.01\\
382.01	0.01\\
383.01	0.01\\
384.01	0.01\\
385.01	0.01\\
386.01	0.01\\
387.01	0.01\\
388.01	0.01\\
389.01	0.01\\
390.01	0.01\\
391.01	0.01\\
392.01	0.01\\
393.01	0.01\\
394.01	0.01\\
395.01	0.01\\
396.01	0.01\\
397.01	0.01\\
398.01	0.01\\
399.01	0.01\\
400.01	0.01\\
401.01	0.01\\
402.01	0.01\\
403.01	0.01\\
404.01	0.01\\
405.01	0.01\\
406.01	0.01\\
407.01	0.01\\
408.01	0.01\\
409.01	0.01\\
410.01	0.01\\
411.01	0.01\\
412.01	0.01\\
413.01	0.01\\
414.01	0.01\\
415.01	0.01\\
416.01	0.01\\
417.01	0.01\\
418.01	0.01\\
419.01	0.01\\
420.01	0.01\\
421.01	0.01\\
422.01	0.01\\
423.01	0.01\\
424.01	0.01\\
425.01	0.01\\
426.01	0.01\\
427.01	0.01\\
428.01	0.01\\
429.01	0.01\\
430.01	0.01\\
431.01	0.01\\
432.01	0.01\\
433.01	0.01\\
434.01	0.01\\
435.01	0.01\\
436.01	0.01\\
437.01	0.01\\
438.01	0.01\\
439.01	0.01\\
440.01	0.01\\
441.01	0.01\\
442.01	0.01\\
443.01	0.01\\
444.01	0.01\\
445.01	0.01\\
446.01	0.01\\
447.01	0.01\\
448.01	0.01\\
449.01	0.01\\
450.01	0.01\\
451.01	0.01\\
452.01	0.01\\
453.01	0.01\\
454.01	0.01\\
455.01	0.01\\
456.01	0.01\\
457.01	0.01\\
458.01	0.01\\
459.01	0.01\\
460.01	0.01\\
461.01	0.01\\
462.01	0.01\\
463.01	0.01\\
464.01	0.01\\
465.01	0.01\\
466.01	0.01\\
467.01	0.01\\
468.01	0.01\\
469.01	0.01\\
470.01	0.01\\
471.01	0.01\\
472.01	0.01\\
473.01	0.01\\
474.01	0.01\\
475.01	0.01\\
476.01	0.01\\
477.01	0.01\\
478.01	0.01\\
479.01	0.01\\
480.01	0.01\\
481.01	0.01\\
482.01	0.01\\
483.01	0.01\\
484.01	0.01\\
485.01	0.01\\
486.01	0.01\\
487.01	0.01\\
488.01	0.01\\
489.01	0.01\\
490.01	0.01\\
491.01	0.01\\
492.01	0.01\\
493.01	0.01\\
494.01	0.01\\
495.01	0.01\\
496.01	0.01\\
497.01	0.01\\
498.01	0.01\\
499.01	0.01\\
500.01	0.01\\
501.01	0.01\\
502.01	0.01\\
503.01	0.01\\
504.01	0.01\\
505.01	0.01\\
506.01	0.01\\
507.01	0.01\\
508.01	0.01\\
509.01	0.01\\
510.01	0.01\\
511.01	0.01\\
512.01	0.01\\
513.01	0.01\\
514.01	0.01\\
515.01	0.01\\
516.01	0.01\\
517.01	0.01\\
518.01	0.01\\
519.01	0.01\\
520.01	0.01\\
521.01	0.01\\
522.01	0.01\\
523.01	0.01\\
524.01	0.01\\
525.01	0.01\\
526.01	0.01\\
527.01	0.01\\
528.01	0.01\\
529.01	0.01\\
530.01	0.01\\
531.01	0.01\\
532.01	0.01\\
533.01	0.01\\
534.01	0.01\\
535.01	0.01\\
536.01	0.01\\
537.01	0.01\\
538.01	0.01\\
539.01	0.01\\
540.01	0.01\\
541.01	0.01\\
542.01	0.01\\
543.01	0.01\\
544.01	0.01\\
545.01	0.01\\
546.01	0.01\\
547.01	0.01\\
548.01	0.01\\
549.01	0.01\\
550.01	0.01\\
551.01	0.01\\
552.01	0.01\\
553.01	0.01\\
554.01	0.01\\
555.01	0.01\\
556.01	0.01\\
557.01	0.01\\
558.01	0.01\\
559.01	0.01\\
560.01	0.01\\
561.01	0.01\\
562.01	0.01\\
563.01	0.01\\
564.01	0.01\\
565.01	0.01\\
566.01	0.01\\
567.01	0.01\\
568.01	0.01\\
569.01	0.01\\
570.01	0.01\\
571.01	0.01\\
572.01	0.01\\
573.01	0.01\\
574.01	0.01\\
575.01	0.01\\
576.01	0.01\\
577.01	0.01\\
578.01	0.01\\
579.01	0.01\\
580.01	0.01\\
581.01	0.01\\
582.01	0.01\\
583.01	0.01\\
584.01	0.01\\
585.01	0.01\\
586.01	0.01\\
587.01	0.01\\
588.01	0.01\\
589.01	0.01\\
590.01	0.01\\
591.01	0.01\\
592.01	0.01\\
593.01	0.01\\
594.01	0.01\\
595.01	0.01\\
596.01	0.01\\
597.01	0.00951869572166329\\
598.01	0.00856167931814812\\
599.01	0.00614338742788371\\
599.02	0.00610585181164764\\
599.03	0.0060679585399091\\
599.04	0.00602970415859936\\
599.05	0.0059910851803115\\
599.06	0.00595209808397888\\
599.07	0.00591273931455051\\
599.08	0.00587300528266336\\
599.09	0.00583289236431146\\
599.1	0.00579239690051178\\
599.11	0.00575151519696696\\
599.12	0.00571024352372478\\
599.13	0.00566857811483427\\
599.14	0.00562651516799859\\
599.15	0.00558405084422458\\
599.16	0.00554118126746885\\
599.17	0.00549790252428052\\
599.18	0.00545421066344048\\
599.19	0.00541010169559726\\
599.2	0.00536557159289924\\
599.21	0.00532061628862348\\
599.22	0.00527523167680089\\
599.23	0.00522941361183776\\
599.24	0.00518315790813368\\
599.25	0.00513646033969587\\
599.26	0.00508931665044633\\
599.27	0.00504172255041562\\
599.28	0.0049936737084123\\
599.29	0.0049451657516271\\
599.3	0.00489619426523356\\
599.31	0.00484675479198461\\
599.32	0.00479684283180541\\
599.33	0.00474645384138222\\
599.34	0.00469558323374738\\
599.35	0.00464422637786024\\
599.36	0.00459237859818417\\
599.37	0.00454003517425943\\
599.38	0.00448719134027199\\
599.39	0.00443384228461815\\
599.4	0.00437998314946512\\
599.41	0.00432560903030723\\
599.42	0.00427071497551799\\
599.43	0.00421529598589782\\
599.44	0.00415934701421745\\
599.45	0.00410286296475692\\
599.46	0.00404583869284022\\
599.47	0.00398826900436535\\
599.48	0.00393014865533006\\
599.49	0.00387147235135282\\
599.5	0.00381223474718941\\
599.51	0.0037524304462447\\
599.52	0.0036920540000799\\
599.53	0.00363109990791496\\
599.54	0.00356956261612626\\
599.55	0.00350743651773952\\
599.56	0.00344471595191776\\
599.57	0.00338139520344439\\
599.58	0.00331746850220142\\
599.59	0.00325293002264253\\
599.6	0.0031877738832612\\
599.61	0.00312199414605368\\
599.62	0.00305558481597684\\
599.63	0.00298853984040087\\
599.64	0.00292085310855654\\
599.65	0.00285251845097746\\
599.66	0.00278352963893676\\
599.67	0.00271388038387842\\
599.68	0.00264356433684324\\
599.69	0.00257257508788922\\
599.7	0.00250090616550637\\
599.71	0.002428551036026\\
599.72	0.00235550310302429\\
599.73	0.00228175570672011\\
599.74	0.00220730212336715\\
599.75	0.00213213556464018\\
599.76	0.00205624917701539\\
599.77	0.0019796360411449\\
599.78	0.00190228917122514\\
599.79	0.00182420151435928\\
599.8	0.00174536594991351\\
599.81	0.00166577528886714\\
599.82	0.00158542227315651\\
599.83	0.00150429957501253\\
599.84	0.00142239979629202\\
599.85	0.00133971546780248\\
599.86	0.00125623904862046\\
599.87	0.00117196292540346\\
599.88	0.0010868794116951\\
599.89	0.00100098074722375\\
599.9	0.00091425909719438\\
599.91	0.000826706551573643\\
599.92	0.000738315124368094\\
599.93	0.000649076752895508\\
599.94	0.000558983297049212\\
599.95	0.000468026538555355\\
599.96	0.000376198180223065\\
599.97	0.000283489845187451\\
599.98	0.0001898930761453\\
599.99	9.53993345834923e-05\\
600	0\\
};
\addplot [color=black!80!mycolor21,solid,forget plot]
  table[row sep=crcr]{%
0.01	0.01\\
1.01	0.01\\
2.01	0.01\\
3.01	0.01\\
4.01	0.01\\
5.01	0.01\\
6.01	0.01\\
7.01	0.01\\
8.01	0.01\\
9.01	0.01\\
10.01	0.01\\
11.01	0.01\\
12.01	0.01\\
13.01	0.01\\
14.01	0.01\\
15.01	0.01\\
16.01	0.01\\
17.01	0.01\\
18.01	0.01\\
19.01	0.01\\
20.01	0.01\\
21.01	0.01\\
22.01	0.01\\
23.01	0.01\\
24.01	0.01\\
25.01	0.01\\
26.01	0.01\\
27.01	0.01\\
28.01	0.01\\
29.01	0.01\\
30.01	0.01\\
31.01	0.01\\
32.01	0.01\\
33.01	0.01\\
34.01	0.01\\
35.01	0.01\\
36.01	0.01\\
37.01	0.01\\
38.01	0.01\\
39.01	0.01\\
40.01	0.01\\
41.01	0.01\\
42.01	0.01\\
43.01	0.01\\
44.01	0.01\\
45.01	0.01\\
46.01	0.01\\
47.01	0.01\\
48.01	0.01\\
49.01	0.01\\
50.01	0.01\\
51.01	0.01\\
52.01	0.01\\
53.01	0.01\\
54.01	0.01\\
55.01	0.01\\
56.01	0.01\\
57.01	0.01\\
58.01	0.01\\
59.01	0.01\\
60.01	0.01\\
61.01	0.01\\
62.01	0.01\\
63.01	0.01\\
64.01	0.01\\
65.01	0.01\\
66.01	0.01\\
67.01	0.01\\
68.01	0.01\\
69.01	0.01\\
70.01	0.01\\
71.01	0.01\\
72.01	0.01\\
73.01	0.01\\
74.01	0.01\\
75.01	0.01\\
76.01	0.01\\
77.01	0.01\\
78.01	0.01\\
79.01	0.01\\
80.01	0.01\\
81.01	0.01\\
82.01	0.01\\
83.01	0.01\\
84.01	0.01\\
85.01	0.01\\
86.01	0.01\\
87.01	0.01\\
88.01	0.01\\
89.01	0.01\\
90.01	0.01\\
91.01	0.01\\
92.01	0.01\\
93.01	0.01\\
94.01	0.01\\
95.01	0.01\\
96.01	0.01\\
97.01	0.01\\
98.01	0.01\\
99.01	0.01\\
100.01	0.01\\
101.01	0.01\\
102.01	0.01\\
103.01	0.01\\
104.01	0.01\\
105.01	0.01\\
106.01	0.01\\
107.01	0.01\\
108.01	0.01\\
109.01	0.01\\
110.01	0.01\\
111.01	0.01\\
112.01	0.01\\
113.01	0.01\\
114.01	0.01\\
115.01	0.01\\
116.01	0.01\\
117.01	0.01\\
118.01	0.01\\
119.01	0.01\\
120.01	0.01\\
121.01	0.01\\
122.01	0.01\\
123.01	0.01\\
124.01	0.01\\
125.01	0.01\\
126.01	0.01\\
127.01	0.01\\
128.01	0.01\\
129.01	0.01\\
130.01	0.01\\
131.01	0.01\\
132.01	0.01\\
133.01	0.01\\
134.01	0.01\\
135.01	0.01\\
136.01	0.01\\
137.01	0.01\\
138.01	0.01\\
139.01	0.01\\
140.01	0.01\\
141.01	0.01\\
142.01	0.01\\
143.01	0.01\\
144.01	0.01\\
145.01	0.01\\
146.01	0.01\\
147.01	0.01\\
148.01	0.01\\
149.01	0.01\\
150.01	0.01\\
151.01	0.01\\
152.01	0.01\\
153.01	0.01\\
154.01	0.01\\
155.01	0.01\\
156.01	0.01\\
157.01	0.01\\
158.01	0.01\\
159.01	0.01\\
160.01	0.01\\
161.01	0.01\\
162.01	0.01\\
163.01	0.01\\
164.01	0.01\\
165.01	0.01\\
166.01	0.01\\
167.01	0.01\\
168.01	0.01\\
169.01	0.01\\
170.01	0.01\\
171.01	0.01\\
172.01	0.01\\
173.01	0.01\\
174.01	0.01\\
175.01	0.01\\
176.01	0.01\\
177.01	0.01\\
178.01	0.01\\
179.01	0.01\\
180.01	0.01\\
181.01	0.01\\
182.01	0.01\\
183.01	0.01\\
184.01	0.01\\
185.01	0.01\\
186.01	0.01\\
187.01	0.01\\
188.01	0.01\\
189.01	0.01\\
190.01	0.01\\
191.01	0.01\\
192.01	0.01\\
193.01	0.01\\
194.01	0.01\\
195.01	0.01\\
196.01	0.01\\
197.01	0.01\\
198.01	0.01\\
199.01	0.01\\
200.01	0.01\\
201.01	0.01\\
202.01	0.01\\
203.01	0.01\\
204.01	0.01\\
205.01	0.01\\
206.01	0.01\\
207.01	0.01\\
208.01	0.01\\
209.01	0.01\\
210.01	0.01\\
211.01	0.01\\
212.01	0.01\\
213.01	0.01\\
214.01	0.01\\
215.01	0.01\\
216.01	0.01\\
217.01	0.01\\
218.01	0.01\\
219.01	0.01\\
220.01	0.01\\
221.01	0.01\\
222.01	0.01\\
223.01	0.01\\
224.01	0.01\\
225.01	0.01\\
226.01	0.01\\
227.01	0.01\\
228.01	0.01\\
229.01	0.01\\
230.01	0.01\\
231.01	0.01\\
232.01	0.01\\
233.01	0.01\\
234.01	0.01\\
235.01	0.01\\
236.01	0.01\\
237.01	0.01\\
238.01	0.01\\
239.01	0.01\\
240.01	0.01\\
241.01	0.01\\
242.01	0.01\\
243.01	0.01\\
244.01	0.01\\
245.01	0.01\\
246.01	0.01\\
247.01	0.01\\
248.01	0.01\\
249.01	0.01\\
250.01	0.01\\
251.01	0.01\\
252.01	0.01\\
253.01	0.01\\
254.01	0.01\\
255.01	0.01\\
256.01	0.01\\
257.01	0.01\\
258.01	0.01\\
259.01	0.01\\
260.01	0.01\\
261.01	0.01\\
262.01	0.01\\
263.01	0.01\\
264.01	0.01\\
265.01	0.01\\
266.01	0.01\\
267.01	0.01\\
268.01	0.01\\
269.01	0.01\\
270.01	0.01\\
271.01	0.01\\
272.01	0.01\\
273.01	0.01\\
274.01	0.01\\
275.01	0.01\\
276.01	0.01\\
277.01	0.01\\
278.01	0.01\\
279.01	0.01\\
280.01	0.01\\
281.01	0.01\\
282.01	0.01\\
283.01	0.01\\
284.01	0.01\\
285.01	0.01\\
286.01	0.01\\
287.01	0.01\\
288.01	0.01\\
289.01	0.01\\
290.01	0.01\\
291.01	0.01\\
292.01	0.01\\
293.01	0.01\\
294.01	0.01\\
295.01	0.01\\
296.01	0.01\\
297.01	0.01\\
298.01	0.01\\
299.01	0.01\\
300.01	0.01\\
301.01	0.01\\
302.01	0.01\\
303.01	0.01\\
304.01	0.01\\
305.01	0.01\\
306.01	0.01\\
307.01	0.01\\
308.01	0.01\\
309.01	0.01\\
310.01	0.01\\
311.01	0.01\\
312.01	0.01\\
313.01	0.01\\
314.01	0.01\\
315.01	0.01\\
316.01	0.01\\
317.01	0.01\\
318.01	0.01\\
319.01	0.01\\
320.01	0.01\\
321.01	0.01\\
322.01	0.01\\
323.01	0.01\\
324.01	0.01\\
325.01	0.01\\
326.01	0.01\\
327.01	0.01\\
328.01	0.01\\
329.01	0.01\\
330.01	0.01\\
331.01	0.01\\
332.01	0.01\\
333.01	0.01\\
334.01	0.01\\
335.01	0.01\\
336.01	0.01\\
337.01	0.01\\
338.01	0.01\\
339.01	0.01\\
340.01	0.01\\
341.01	0.01\\
342.01	0.01\\
343.01	0.01\\
344.01	0.01\\
345.01	0.01\\
346.01	0.01\\
347.01	0.01\\
348.01	0.01\\
349.01	0.01\\
350.01	0.01\\
351.01	0.01\\
352.01	0.01\\
353.01	0.01\\
354.01	0.01\\
355.01	0.01\\
356.01	0.01\\
357.01	0.01\\
358.01	0.01\\
359.01	0.01\\
360.01	0.01\\
361.01	0.01\\
362.01	0.01\\
363.01	0.01\\
364.01	0.01\\
365.01	0.01\\
366.01	0.01\\
367.01	0.01\\
368.01	0.01\\
369.01	0.01\\
370.01	0.01\\
371.01	0.01\\
372.01	0.01\\
373.01	0.01\\
374.01	0.01\\
375.01	0.01\\
376.01	0.01\\
377.01	0.01\\
378.01	0.01\\
379.01	0.01\\
380.01	0.01\\
381.01	0.01\\
382.01	0.01\\
383.01	0.01\\
384.01	0.01\\
385.01	0.01\\
386.01	0.01\\
387.01	0.01\\
388.01	0.01\\
389.01	0.01\\
390.01	0.01\\
391.01	0.01\\
392.01	0.01\\
393.01	0.01\\
394.01	0.01\\
395.01	0.01\\
396.01	0.01\\
397.01	0.01\\
398.01	0.01\\
399.01	0.01\\
400.01	0.01\\
401.01	0.01\\
402.01	0.01\\
403.01	0.01\\
404.01	0.01\\
405.01	0.01\\
406.01	0.01\\
407.01	0.01\\
408.01	0.01\\
409.01	0.01\\
410.01	0.01\\
411.01	0.01\\
412.01	0.01\\
413.01	0.01\\
414.01	0.01\\
415.01	0.01\\
416.01	0.01\\
417.01	0.01\\
418.01	0.01\\
419.01	0.01\\
420.01	0.01\\
421.01	0.01\\
422.01	0.01\\
423.01	0.01\\
424.01	0.01\\
425.01	0.01\\
426.01	0.01\\
427.01	0.01\\
428.01	0.01\\
429.01	0.01\\
430.01	0.01\\
431.01	0.01\\
432.01	0.01\\
433.01	0.01\\
434.01	0.01\\
435.01	0.01\\
436.01	0.01\\
437.01	0.01\\
438.01	0.01\\
439.01	0.01\\
440.01	0.01\\
441.01	0.01\\
442.01	0.01\\
443.01	0.01\\
444.01	0.01\\
445.01	0.01\\
446.01	0.01\\
447.01	0.01\\
448.01	0.01\\
449.01	0.01\\
450.01	0.01\\
451.01	0.01\\
452.01	0.01\\
453.01	0.01\\
454.01	0.01\\
455.01	0.01\\
456.01	0.01\\
457.01	0.01\\
458.01	0.01\\
459.01	0.01\\
460.01	0.01\\
461.01	0.01\\
462.01	0.01\\
463.01	0.01\\
464.01	0.01\\
465.01	0.01\\
466.01	0.01\\
467.01	0.01\\
468.01	0.01\\
469.01	0.01\\
470.01	0.01\\
471.01	0.01\\
472.01	0.01\\
473.01	0.01\\
474.01	0.01\\
475.01	0.01\\
476.01	0.01\\
477.01	0.01\\
478.01	0.01\\
479.01	0.01\\
480.01	0.01\\
481.01	0.01\\
482.01	0.01\\
483.01	0.01\\
484.01	0.01\\
485.01	0.01\\
486.01	0.01\\
487.01	0.01\\
488.01	0.01\\
489.01	0.01\\
490.01	0.01\\
491.01	0.01\\
492.01	0.01\\
493.01	0.01\\
494.01	0.01\\
495.01	0.01\\
496.01	0.01\\
497.01	0.01\\
498.01	0.01\\
499.01	0.01\\
500.01	0.01\\
501.01	0.01\\
502.01	0.01\\
503.01	0.01\\
504.01	0.01\\
505.01	0.01\\
506.01	0.01\\
507.01	0.01\\
508.01	0.01\\
509.01	0.01\\
510.01	0.01\\
511.01	0.01\\
512.01	0.01\\
513.01	0.01\\
514.01	0.01\\
515.01	0.01\\
516.01	0.01\\
517.01	0.01\\
518.01	0.01\\
519.01	0.01\\
520.01	0.01\\
521.01	0.01\\
522.01	0.01\\
523.01	0.01\\
524.01	0.01\\
525.01	0.01\\
526.01	0.01\\
527.01	0.01\\
528.01	0.01\\
529.01	0.01\\
530.01	0.01\\
531.01	0.01\\
532.01	0.01\\
533.01	0.01\\
534.01	0.01\\
535.01	0.01\\
536.01	0.01\\
537.01	0.01\\
538.01	0.01\\
539.01	0.01\\
540.01	0.01\\
541.01	0.01\\
542.01	0.01\\
543.01	0.01\\
544.01	0.01\\
545.01	0.01\\
546.01	0.01\\
547.01	0.01\\
548.01	0.01\\
549.01	0.01\\
550.01	0.01\\
551.01	0.01\\
552.01	0.01\\
553.01	0.01\\
554.01	0.01\\
555.01	0.01\\
556.01	0.01\\
557.01	0.01\\
558.01	0.01\\
559.01	0.01\\
560.01	0.01\\
561.01	0.01\\
562.01	0.01\\
563.01	0.01\\
564.01	0.01\\
565.01	0.01\\
566.01	0.01\\
567.01	0.01\\
568.01	0.01\\
569.01	0.01\\
570.01	0.01\\
571.01	0.01\\
572.01	0.01\\
573.01	0.01\\
574.01	0.01\\
575.01	0.01\\
576.01	0.01\\
577.01	0.01\\
578.01	0.01\\
579.01	0.01\\
580.01	0.01\\
581.01	0.01\\
582.01	0.01\\
583.01	0.01\\
584.01	0.01\\
585.01	0.01\\
586.01	0.01\\
587.01	0.01\\
588.01	0.01\\
589.01	0.01\\
590.01	0.01\\
591.01	0.01\\
592.01	0.01\\
593.01	0.01\\
594.01	0.01\\
595.01	0.01\\
596.01	0.01\\
597.01	0.00951867749327391\\
598.01	0.00856167931814826\\
599.01	0.00614338742788382\\
599.02	0.00610585181164775\\
599.03	0.00606795853990922\\
599.04	0.00602970415859947\\
599.05	0.0059910851803116\\
599.06	0.00595209808397896\\
599.07	0.00591273931455059\\
599.08	0.00587300528266344\\
599.09	0.00583289236431155\\
599.1	0.00579239690051189\\
599.11	0.00575151519696708\\
599.12	0.00571024352372488\\
599.13	0.00566857811483435\\
599.14	0.00562651516799866\\
599.15	0.00558405084422465\\
599.16	0.00554118126746891\\
599.17	0.00549790252428057\\
599.18	0.00545421066344052\\
599.19	0.00541010169559728\\
599.2	0.00536557159289927\\
599.21	0.00532061628862351\\
599.22	0.00527523167680091\\
599.23	0.00522941361183776\\
599.24	0.00518315790813368\\
599.25	0.00513646033969586\\
599.26	0.00508931665044633\\
599.27	0.00504172255041564\\
599.28	0.0049936737084123\\
599.29	0.0049451657516271\\
599.3	0.00489619426523357\\
599.31	0.00484675479198463\\
599.32	0.00479684283180545\\
599.33	0.00474645384138226\\
599.34	0.00469558323374742\\
599.35	0.00464422637786029\\
599.36	0.00459237859818423\\
599.37	0.00454003517425949\\
599.38	0.00448719134027203\\
599.39	0.00443384228461818\\
599.4	0.00437998314946515\\
599.41	0.00432560903030725\\
599.42	0.004270714975518\\
599.43	0.00421529598589784\\
599.44	0.00415934701421748\\
599.45	0.00410286296475696\\
599.46	0.00404583869284026\\
599.47	0.00398826900436541\\
599.48	0.0039301486553301\\
599.49	0.00387147235135286\\
599.5	0.00381223474718944\\
599.51	0.00375243044624473\\
599.52	0.00369205400007993\\
599.53	0.00363109990791499\\
599.54	0.0035695626161263\\
599.55	0.00350743651773956\\
599.56	0.00344471595191778\\
599.57	0.00338139520344441\\
599.58	0.00331746850220144\\
599.59	0.00325293002264254\\
599.6	0.00318777388326121\\
599.61	0.00312199414605369\\
599.62	0.00305558481597686\\
599.63	0.00298853984040089\\
599.64	0.00292085310855656\\
599.65	0.00285251845097748\\
599.66	0.00278352963893677\\
599.67	0.00271388038387843\\
599.68	0.00264356433684326\\
599.69	0.00257257508788923\\
599.7	0.00250090616550638\\
599.71	0.00242855103602601\\
599.72	0.0023555031030243\\
599.73	0.00228175570672012\\
599.74	0.00220730212336716\\
599.75	0.00213213556464019\\
599.76	0.00205624917701541\\
599.77	0.00197963604114492\\
599.78	0.00190228917122516\\
599.79	0.0018242015143593\\
599.8	0.00174536594991352\\
599.81	0.00166577528886715\\
599.82	0.00158542227315652\\
599.83	0.00150429957501254\\
599.84	0.00142239979629203\\
599.85	0.00133971546780248\\
599.86	0.00125623904862046\\
599.87	0.00117196292540346\\
599.88	0.0010868794116951\\
599.89	0.00100098074722376\\
599.9	0.000914259097194387\\
599.91	0.000826706551573646\\
599.92	0.000738315124368097\\
599.93	0.000649076752895515\\
599.94	0.000558983297049218\\
599.95	0.000468026538555358\\
599.96	0.000376198180223065\\
599.97	0.000283489845187453\\
599.98	0.000189893076145302\\
599.99	9.53993345834923e-05\\
600	0\\
};
\addplot [color=black,solid,forget plot]
  table[row sep=crcr]{%
0.01	0.01\\
1.01	0.01\\
2.01	0.01\\
3.01	0.01\\
4.01	0.01\\
5.01	0.01\\
6.01	0.01\\
7.01	0.01\\
8.01	0.01\\
9.01	0.01\\
10.01	0.01\\
11.01	0.01\\
12.01	0.01\\
13.01	0.01\\
14.01	0.01\\
15.01	0.01\\
16.01	0.01\\
17.01	0.01\\
18.01	0.01\\
19.01	0.01\\
20.01	0.01\\
21.01	0.01\\
22.01	0.01\\
23.01	0.01\\
24.01	0.01\\
25.01	0.01\\
26.01	0.01\\
27.01	0.01\\
28.01	0.01\\
29.01	0.01\\
30.01	0.01\\
31.01	0.01\\
32.01	0.01\\
33.01	0.01\\
34.01	0.01\\
35.01	0.01\\
36.01	0.01\\
37.01	0.01\\
38.01	0.01\\
39.01	0.01\\
40.01	0.01\\
41.01	0.01\\
42.01	0.01\\
43.01	0.01\\
44.01	0.01\\
45.01	0.01\\
46.01	0.01\\
47.01	0.01\\
48.01	0.01\\
49.01	0.01\\
50.01	0.01\\
51.01	0.01\\
52.01	0.01\\
53.01	0.01\\
54.01	0.01\\
55.01	0.01\\
56.01	0.01\\
57.01	0.01\\
58.01	0.01\\
59.01	0.01\\
60.01	0.01\\
61.01	0.01\\
62.01	0.01\\
63.01	0.01\\
64.01	0.01\\
65.01	0.01\\
66.01	0.01\\
67.01	0.01\\
68.01	0.01\\
69.01	0.01\\
70.01	0.01\\
71.01	0.01\\
72.01	0.01\\
73.01	0.01\\
74.01	0.01\\
75.01	0.01\\
76.01	0.01\\
77.01	0.01\\
78.01	0.01\\
79.01	0.01\\
80.01	0.01\\
81.01	0.01\\
82.01	0.01\\
83.01	0.01\\
84.01	0.01\\
85.01	0.01\\
86.01	0.01\\
87.01	0.01\\
88.01	0.01\\
89.01	0.01\\
90.01	0.01\\
91.01	0.01\\
92.01	0.01\\
93.01	0.01\\
94.01	0.01\\
95.01	0.01\\
96.01	0.01\\
97.01	0.01\\
98.01	0.01\\
99.01	0.01\\
100.01	0.01\\
101.01	0.01\\
102.01	0.01\\
103.01	0.01\\
104.01	0.01\\
105.01	0.01\\
106.01	0.01\\
107.01	0.01\\
108.01	0.01\\
109.01	0.01\\
110.01	0.01\\
111.01	0.01\\
112.01	0.01\\
113.01	0.01\\
114.01	0.01\\
115.01	0.01\\
116.01	0.01\\
117.01	0.01\\
118.01	0.01\\
119.01	0.01\\
120.01	0.01\\
121.01	0.01\\
122.01	0.01\\
123.01	0.01\\
124.01	0.01\\
125.01	0.01\\
126.01	0.01\\
127.01	0.01\\
128.01	0.01\\
129.01	0.01\\
130.01	0.01\\
131.01	0.01\\
132.01	0.01\\
133.01	0.01\\
134.01	0.01\\
135.01	0.01\\
136.01	0.01\\
137.01	0.01\\
138.01	0.01\\
139.01	0.01\\
140.01	0.01\\
141.01	0.01\\
142.01	0.01\\
143.01	0.01\\
144.01	0.01\\
145.01	0.01\\
146.01	0.01\\
147.01	0.01\\
148.01	0.01\\
149.01	0.01\\
150.01	0.01\\
151.01	0.01\\
152.01	0.01\\
153.01	0.01\\
154.01	0.01\\
155.01	0.01\\
156.01	0.01\\
157.01	0.01\\
158.01	0.01\\
159.01	0.01\\
160.01	0.01\\
161.01	0.01\\
162.01	0.01\\
163.01	0.01\\
164.01	0.01\\
165.01	0.01\\
166.01	0.01\\
167.01	0.01\\
168.01	0.01\\
169.01	0.01\\
170.01	0.01\\
171.01	0.01\\
172.01	0.01\\
173.01	0.01\\
174.01	0.01\\
175.01	0.01\\
176.01	0.01\\
177.01	0.01\\
178.01	0.01\\
179.01	0.01\\
180.01	0.01\\
181.01	0.01\\
182.01	0.01\\
183.01	0.01\\
184.01	0.01\\
185.01	0.01\\
186.01	0.01\\
187.01	0.01\\
188.01	0.01\\
189.01	0.01\\
190.01	0.01\\
191.01	0.01\\
192.01	0.01\\
193.01	0.01\\
194.01	0.01\\
195.01	0.01\\
196.01	0.01\\
197.01	0.01\\
198.01	0.01\\
199.01	0.01\\
200.01	0.01\\
201.01	0.01\\
202.01	0.01\\
203.01	0.01\\
204.01	0.01\\
205.01	0.01\\
206.01	0.01\\
207.01	0.01\\
208.01	0.01\\
209.01	0.01\\
210.01	0.01\\
211.01	0.01\\
212.01	0.01\\
213.01	0.01\\
214.01	0.01\\
215.01	0.01\\
216.01	0.01\\
217.01	0.01\\
218.01	0.01\\
219.01	0.01\\
220.01	0.01\\
221.01	0.01\\
222.01	0.01\\
223.01	0.01\\
224.01	0.01\\
225.01	0.01\\
226.01	0.01\\
227.01	0.01\\
228.01	0.01\\
229.01	0.01\\
230.01	0.01\\
231.01	0.01\\
232.01	0.01\\
233.01	0.01\\
234.01	0.01\\
235.01	0.01\\
236.01	0.01\\
237.01	0.01\\
238.01	0.01\\
239.01	0.01\\
240.01	0.01\\
241.01	0.01\\
242.01	0.01\\
243.01	0.01\\
244.01	0.01\\
245.01	0.01\\
246.01	0.01\\
247.01	0.01\\
248.01	0.01\\
249.01	0.01\\
250.01	0.01\\
251.01	0.01\\
252.01	0.01\\
253.01	0.01\\
254.01	0.01\\
255.01	0.01\\
256.01	0.01\\
257.01	0.01\\
258.01	0.01\\
259.01	0.01\\
260.01	0.01\\
261.01	0.01\\
262.01	0.01\\
263.01	0.01\\
264.01	0.01\\
265.01	0.01\\
266.01	0.01\\
267.01	0.01\\
268.01	0.01\\
269.01	0.01\\
270.01	0.01\\
271.01	0.01\\
272.01	0.01\\
273.01	0.01\\
274.01	0.01\\
275.01	0.01\\
276.01	0.01\\
277.01	0.01\\
278.01	0.01\\
279.01	0.01\\
280.01	0.01\\
281.01	0.01\\
282.01	0.01\\
283.01	0.01\\
284.01	0.01\\
285.01	0.01\\
286.01	0.01\\
287.01	0.01\\
288.01	0.01\\
289.01	0.01\\
290.01	0.01\\
291.01	0.01\\
292.01	0.01\\
293.01	0.01\\
294.01	0.01\\
295.01	0.01\\
296.01	0.01\\
297.01	0.01\\
298.01	0.01\\
299.01	0.01\\
300.01	0.01\\
301.01	0.01\\
302.01	0.01\\
303.01	0.01\\
304.01	0.01\\
305.01	0.01\\
306.01	0.01\\
307.01	0.01\\
308.01	0.01\\
309.01	0.01\\
310.01	0.01\\
311.01	0.01\\
312.01	0.01\\
313.01	0.01\\
314.01	0.01\\
315.01	0.01\\
316.01	0.01\\
317.01	0.01\\
318.01	0.01\\
319.01	0.01\\
320.01	0.01\\
321.01	0.01\\
322.01	0.01\\
323.01	0.01\\
324.01	0.01\\
325.01	0.01\\
326.01	0.01\\
327.01	0.01\\
328.01	0.01\\
329.01	0.01\\
330.01	0.01\\
331.01	0.01\\
332.01	0.01\\
333.01	0.01\\
334.01	0.01\\
335.01	0.01\\
336.01	0.01\\
337.01	0.01\\
338.01	0.01\\
339.01	0.01\\
340.01	0.01\\
341.01	0.01\\
342.01	0.01\\
343.01	0.01\\
344.01	0.01\\
345.01	0.01\\
346.01	0.01\\
347.01	0.01\\
348.01	0.01\\
349.01	0.01\\
350.01	0.01\\
351.01	0.01\\
352.01	0.01\\
353.01	0.01\\
354.01	0.01\\
355.01	0.01\\
356.01	0.01\\
357.01	0.01\\
358.01	0.01\\
359.01	0.01\\
360.01	0.01\\
361.01	0.01\\
362.01	0.01\\
363.01	0.01\\
364.01	0.01\\
365.01	0.01\\
366.01	0.01\\
367.01	0.01\\
368.01	0.01\\
369.01	0.01\\
370.01	0.01\\
371.01	0.01\\
372.01	0.01\\
373.01	0.01\\
374.01	0.01\\
375.01	0.01\\
376.01	0.01\\
377.01	0.01\\
378.01	0.01\\
379.01	0.01\\
380.01	0.01\\
381.01	0.01\\
382.01	0.01\\
383.01	0.01\\
384.01	0.01\\
385.01	0.01\\
386.01	0.01\\
387.01	0.01\\
388.01	0.01\\
389.01	0.01\\
390.01	0.01\\
391.01	0.01\\
392.01	0.01\\
393.01	0.01\\
394.01	0.01\\
395.01	0.01\\
396.01	0.01\\
397.01	0.01\\
398.01	0.01\\
399.01	0.01\\
400.01	0.01\\
401.01	0.01\\
402.01	0.01\\
403.01	0.01\\
404.01	0.01\\
405.01	0.01\\
406.01	0.01\\
407.01	0.01\\
408.01	0.01\\
409.01	0.01\\
410.01	0.01\\
411.01	0.01\\
412.01	0.01\\
413.01	0.01\\
414.01	0.01\\
415.01	0.01\\
416.01	0.01\\
417.01	0.01\\
418.01	0.01\\
419.01	0.01\\
420.01	0.01\\
421.01	0.01\\
422.01	0.01\\
423.01	0.01\\
424.01	0.01\\
425.01	0.01\\
426.01	0.01\\
427.01	0.01\\
428.01	0.01\\
429.01	0.01\\
430.01	0.01\\
431.01	0.01\\
432.01	0.01\\
433.01	0.01\\
434.01	0.01\\
435.01	0.01\\
436.01	0.01\\
437.01	0.01\\
438.01	0.01\\
439.01	0.01\\
440.01	0.01\\
441.01	0.01\\
442.01	0.01\\
443.01	0.01\\
444.01	0.01\\
445.01	0.01\\
446.01	0.01\\
447.01	0.01\\
448.01	0.01\\
449.01	0.01\\
450.01	0.01\\
451.01	0.01\\
452.01	0.01\\
453.01	0.01\\
454.01	0.01\\
455.01	0.01\\
456.01	0.01\\
457.01	0.01\\
458.01	0.01\\
459.01	0.01\\
460.01	0.01\\
461.01	0.01\\
462.01	0.01\\
463.01	0.01\\
464.01	0.01\\
465.01	0.01\\
466.01	0.01\\
467.01	0.01\\
468.01	0.01\\
469.01	0.01\\
470.01	0.01\\
471.01	0.01\\
472.01	0.01\\
473.01	0.01\\
474.01	0.01\\
475.01	0.01\\
476.01	0.01\\
477.01	0.01\\
478.01	0.01\\
479.01	0.01\\
480.01	0.01\\
481.01	0.01\\
482.01	0.01\\
483.01	0.01\\
484.01	0.01\\
485.01	0.01\\
486.01	0.01\\
487.01	0.01\\
488.01	0.01\\
489.01	0.01\\
490.01	0.01\\
491.01	0.01\\
492.01	0.01\\
493.01	0.01\\
494.01	0.01\\
495.01	0.01\\
496.01	0.01\\
497.01	0.01\\
498.01	0.01\\
499.01	0.01\\
500.01	0.01\\
501.01	0.01\\
502.01	0.01\\
503.01	0.01\\
504.01	0.01\\
505.01	0.01\\
506.01	0.01\\
507.01	0.01\\
508.01	0.01\\
509.01	0.01\\
510.01	0.01\\
511.01	0.01\\
512.01	0.01\\
513.01	0.01\\
514.01	0.01\\
515.01	0.01\\
516.01	0.01\\
517.01	0.01\\
518.01	0.01\\
519.01	0.01\\
520.01	0.01\\
521.01	0.01\\
522.01	0.01\\
523.01	0.01\\
524.01	0.01\\
525.01	0.01\\
526.01	0.01\\
527.01	0.01\\
528.01	0.01\\
529.01	0.01\\
530.01	0.01\\
531.01	0.01\\
532.01	0.01\\
533.01	0.01\\
534.01	0.01\\
535.01	0.01\\
536.01	0.01\\
537.01	0.01\\
538.01	0.01\\
539.01	0.01\\
540.01	0.01\\
541.01	0.01\\
542.01	0.01\\
543.01	0.01\\
544.01	0.01\\
545.01	0.01\\
546.01	0.01\\
547.01	0.01\\
548.01	0.01\\
549.01	0.01\\
550.01	0.01\\
551.01	0.01\\
552.01	0.01\\
553.01	0.01\\
554.01	0.01\\
555.01	0.01\\
556.01	0.01\\
557.01	0.01\\
558.01	0.01\\
559.01	0.01\\
560.01	0.01\\
561.01	0.01\\
562.01	0.01\\
563.01	0.01\\
564.01	0.01\\
565.01	0.01\\
566.01	0.01\\
567.01	0.01\\
568.01	0.01\\
569.01	0.01\\
570.01	0.01\\
571.01	0.01\\
572.01	0.01\\
573.01	0.01\\
574.01	0.01\\
575.01	0.01\\
576.01	0.01\\
577.01	0.01\\
578.01	0.01\\
579.01	0.01\\
580.01	0.01\\
581.01	0.01\\
582.01	0.01\\
583.01	0.01\\
584.01	0.01\\
585.01	0.01\\
586.01	0.01\\
587.01	0.01\\
588.01	0.01\\
589.01	0.01\\
590.01	0.01\\
591.01	0.01\\
592.01	0.01\\
593.01	0.01\\
594.01	0.01\\
595.01	0.01\\
596.01	0.01\\
597.01	0.00951867114582728\\
598.01	0.00856167931814814\\
599.01	0.00614338742788377\\
599.02	0.0061058518116477\\
599.03	0.00606795853990914\\
599.04	0.0060297041585994\\
599.05	0.00599108518031156\\
599.06	0.00595209808397892\\
599.07	0.00591273931455055\\
599.08	0.00587300528266342\\
599.09	0.00583289236431151\\
599.1	0.00579239690051183\\
599.11	0.00575151519696703\\
599.12	0.00571024352372485\\
599.13	0.00566857811483434\\
599.14	0.00562651516799866\\
599.15	0.00558405084422465\\
599.16	0.00554118126746891\\
599.17	0.00549790252428056\\
599.18	0.00545421066344052\\
599.19	0.00541010169559728\\
599.2	0.00536557159289925\\
599.21	0.00532061628862351\\
599.22	0.00527523167680092\\
599.23	0.00522941361183779\\
599.24	0.00518315790813372\\
599.25	0.0051364603396959\\
599.26	0.00508931665044635\\
599.27	0.00504172255041565\\
599.28	0.00499367370841231\\
599.29	0.00494516575162712\\
599.3	0.00489619426523359\\
599.31	0.00484675479198465\\
599.32	0.00479684283180545\\
599.33	0.00474645384138226\\
599.34	0.00469558323374741\\
599.35	0.00464422637786027\\
599.36	0.00459237859818421\\
599.37	0.00454003517425949\\
599.38	0.00448719134027204\\
599.39	0.00443384228461821\\
599.4	0.00437998314946516\\
599.41	0.00432560903030727\\
599.42	0.00427071497551803\\
599.43	0.00421529598589787\\
599.44	0.0041593470142175\\
599.45	0.00410286296475699\\
599.46	0.00404583869284028\\
599.47	0.00398826900436542\\
599.48	0.00393014865533013\\
599.49	0.00387147235135289\\
599.5	0.00381223474718947\\
599.51	0.00375243044624476\\
599.52	0.00369205400007994\\
599.53	0.00363109990791499\\
599.54	0.0035695626161263\\
599.55	0.00350743651773954\\
599.56	0.00344471595191778\\
599.57	0.00338139520344442\\
599.58	0.00331746850220145\\
599.59	0.00325293002264255\\
599.6	0.00318777388326121\\
599.61	0.00312199414605369\\
599.62	0.00305558481597688\\
599.63	0.00298853984040089\\
599.64	0.00292085310855655\\
599.65	0.00285251845097748\\
599.66	0.00278352963893677\\
599.67	0.00271388038387843\\
599.68	0.00264356433684324\\
599.69	0.00257257508788921\\
599.7	0.00250090616550636\\
599.71	0.002428551036026\\
599.72	0.00235550310302429\\
599.73	0.0022817557067201\\
599.74	0.00220730212336715\\
599.75	0.00213213556464017\\
599.76	0.00205624917701539\\
599.77	0.00197963604114489\\
599.78	0.00190228917122514\\
599.79	0.00182420151435928\\
599.8	0.00174536594991352\\
599.81	0.00166577528886715\\
599.82	0.0015854222731565\\
599.83	0.00150429957501253\\
599.84	0.00142239979629202\\
599.85	0.00133971546780247\\
599.86	0.00125623904862046\\
599.87	0.00117196292540346\\
599.88	0.0010868794116951\\
599.89	0.00100098074722375\\
599.9	0.00091425909719438\\
599.91	0.000826706551573643\\
599.92	0.000738315124368094\\
599.93	0.000649076752895512\\
599.94	0.000558983297049214\\
599.95	0.000468026538555351\\
599.96	0.000376198180223065\\
599.97	0.000283489845187451\\
599.98	0.0001898930761453\\
599.99	9.53993345834923e-05\\
600	0\\
};
\end{axis}
\end{tikzpicture}%
  \caption{Continuous Time}
\end{subfigure}%
\hfill%
\begin{subfigure}{.45\linewidth}
  \centering
  \setlength\figureheight{\linewidth} 
  \setlength\figurewidth{\linewidth}
  \tikzsetnextfilename{dm_dscr_z1}
  % This file was created by matlab2tikz.
%
%The latest updates can be retrieved from
%  http://www.mathworks.com/matlabcentral/fileexchange/22022-matlab2tikz-matlab2tikz
%where you can also make suggestions and rate matlab2tikz.
%
\definecolor{mycolor1}{rgb}{1.00000,0.00000,1.00000}%
%
\begin{tikzpicture}

\begin{axis}[%
width=4.822in,
height=3.803in,
at={(0.809in,0.513in)},
scale only axis,
every outer x axis line/.append style={black},
every x tick label/.append style={font=\color{black}},
xmin=0,
xmax=100,
xlabel={Time},
every outer y axis line/.append style={black},
every y tick label/.append style={font=\color{black}},
ymin=0,
ymax=0.007,
ylabel={Depth $\delta$},
axis background/.style={fill=white},
title={Z=1},
axis x line*=bottom,
axis y line*=left,
legend style={legend cell align=left,align=left,draw=black}
]
\addplot [color=green,dashed]
  table[row sep=crcr]{%
1	0\\
2	0\\
3	0\\
4	0\\
5	0\\
6	0\\
7	0\\
8	0\\
9	0\\
10	0\\
11	0\\
12	0\\
13	0\\
14	0\\
15	0\\
16	0\\
17	0\\
18	0\\
19	0\\
20	0\\
21	0\\
22	0\\
23	0\\
24	0\\
25	0\\
26	0\\
27	0\\
28	0\\
29	0\\
30	0\\
31	0\\
32	0\\
33	0\\
34	0\\
35	0\\
36	0\\
37	0\\
38	0\\
39	0\\
40	0\\
41	0\\
42	0\\
43	0\\
44	0\\
45	0\\
46	0\\
47	0\\
48	0\\
49	0\\
50	0\\
51	0\\
52	0\\
53	0\\
54	0\\
55	0\\
56	0\\
57	0\\
58	0\\
59	0\\
60	0\\
61	0\\
62	0\\
63	0\\
64	0\\
65	0\\
66	0\\
67	0\\
68	1.85029240059433e-05\\
69	8.12194848045114e-05\\
70	0.000144220454210139\\
71	0.000206673370148375\\
72	0.000269620057018997\\
73	0.000334209076241419\\
74	0.000400505889364924\\
75	0.00046916545272572\\
76	0.000540627279513955\\
77	0.000613650616114012\\
78	0.000687925889382712\\
79	0.000763855840365605\\
80	0.000841587498565912\\
81	0.000921366201697552\\
82	0.00100256321864973\\
83	0.00108582061711338\\
84	0.00117146565560667\\
85	0.00125959384911904\\
86	0.00135039530500243\\
87	0.00144401711869371\\
88	0.00154520843180213\\
89	0.00164920876586887\\
90	0.00195918604474435\\
91	0.00233453662600035\\
92	0.00271483902507165\\
93	0.00306310504106292\\
94	0.00348083529600612\\
95	0.00387557868857673\\
96	0.00425637224613168\\
97	0.00494508978473457\\
98	0.00644915222848939\\
99	0\\
100	0\\
};
\addlegendentry{$q=-4$};

\addplot [color=mycolor1,dashed]
  table[row sep=crcr]{%
1	0\\
2	0\\
3	0\\
4	0\\
5	0\\
6	0\\
7	0\\
8	0\\
9	0\\
10	0\\
11	0\\
12	0\\
13	0\\
14	0\\
15	0\\
16	0\\
17	0\\
18	0\\
19	0\\
20	0\\
21	0\\
22	0\\
23	0\\
24	0\\
25	0\\
26	0\\
27	0\\
28	0\\
29	0\\
30	0\\
31	0\\
32	0\\
33	0\\
34	0\\
35	0\\
36	0\\
37	0\\
38	0\\
39	0\\
40	0\\
41	0\\
42	0\\
43	0\\
44	0\\
45	0\\
46	0\\
47	0\\
48	0\\
49	0\\
50	0\\
51	0\\
52	0\\
53	0\\
54	0\\
55	0\\
56	0\\
57	0\\
58	0\\
59	0\\
60	0\\
61	0\\
62	0\\
63	0\\
64	0\\
65	0\\
66	0\\
67	0\\
68	0\\
69	0\\
70	0\\
71	0\\
72	4.95889084203055e-05\\
73	0.00012431494305836\\
74	0.000200668105863697\\
75	0.000278626264636334\\
76	0.000357795573190019\\
77	0.000438041326325762\\
78	0.000519182788162845\\
79	0.000601635762846073\\
80	0.000685235033147003\\
81	0.000769838844272252\\
82	0.000857717658260223\\
83	0.000948085559418393\\
84	0.00104055432656207\\
85	0.00113517959557298\\
86	0.0012321373993517\\
87	0.00133153740320272\\
88	0.0014335801464463\\
89	0.00153765325057386\\
90	0.00164414650092961\\
91	0.00175443581819706\\
92	0.00193922025435775\\
93	0.00231854302867369\\
94	0.00277384894288533\\
95	0.00331639969684156\\
96	0.00414944733954028\\
97	0.00494508978473457\\
98	0.00644915222848939\\
99	0\\
100	0\\
};
\addlegendentry{$q=-3$};

\addplot [color=red,dashed]
  table[row sep=crcr]{%
1	0\\
2	0\\
3	0\\
4	0\\
5	0\\
6	0\\
7	0\\
8	0\\
9	0\\
10	0\\
11	0\\
12	0\\
13	0\\
14	0\\
15	0\\
16	0\\
17	0\\
18	0\\
19	0\\
20	0\\
21	0\\
22	0\\
23	0\\
24	0\\
25	0\\
26	0\\
27	0\\
28	0\\
29	0\\
30	0\\
31	0\\
32	0\\
33	0\\
34	0\\
35	0\\
36	0\\
37	0\\
38	0\\
39	0\\
40	0\\
41	0\\
42	0\\
43	0\\
44	0\\
45	0\\
46	0\\
47	0\\
48	0\\
49	0\\
50	0\\
51	0\\
52	0\\
53	0\\
54	0\\
55	0\\
56	0\\
57	0\\
58	0\\
59	0\\
60	0\\
61	0\\
62	0\\
63	0\\
64	0\\
65	0\\
66	0\\
67	0\\
68	0\\
69	0\\
70	0\\
71	0\\
72	0\\
73	0\\
74	0\\
75	0\\
76	0\\
77	0\\
78	6.87298550526237e-06\\
79	0.000137922426144921\\
80	0.00027276559631549\\
81	0.000411533017358002\\
82	0.000525419142223996\\
83	0.000633331127546702\\
84	0.000743679485166376\\
85	0.000856731337429374\\
86	0.00097138239846429\\
87	0.00108659787132039\\
88	0.00120515476938989\\
89	0.00132833555177225\\
90	0.00145530220443721\\
91	0.00158497784031782\\
92	0.00171719013625745\\
93	0.00185176991879111\\
94	0.0019887406971283\\
95	0.00237859074873332\\
96	0.00327466363255641\\
97	0.0047114263257848\\
98	0.00644915222848939\\
99	0\\
100	0\\
};
\addlegendentry{$q=-2$};

\addplot [color=blue,dashed]
  table[row sep=crcr]{%
1	0\\
2	0\\
3	0\\
4	0\\
5	0\\
6	0\\
7	0\\
8	0\\
9	0\\
10	0\\
11	0\\
12	0\\
13	0\\
14	0\\
15	0\\
16	0\\
17	0\\
18	0\\
19	0\\
20	0\\
21	0\\
22	0\\
23	0\\
24	0\\
25	0\\
26	0\\
27	0\\
28	0\\
29	0\\
30	0\\
31	0\\
32	0\\
33	0\\
34	0\\
35	0\\
36	0\\
37	0\\
38	0\\
39	0\\
40	0\\
41	0\\
42	0\\
43	0\\
44	0\\
45	0\\
46	0\\
47	0\\
48	0\\
49	0\\
50	0\\
51	0\\
52	0\\
53	0\\
54	0\\
55	0\\
56	0\\
57	0\\
58	0\\
59	0\\
60	0\\
61	0\\
62	0\\
63	0\\
64	0\\
65	0\\
66	0\\
67	0\\
68	0\\
69	0\\
70	0\\
71	0\\
72	0\\
73	0\\
74	0\\
75	0\\
76	0\\
77	0\\
78	0\\
79	0\\
80	0\\
81	0\\
82	0\\
83	0\\
84	0\\
85	0\\
86	0\\
87	9.67646396375256e-05\\
88	0.000263518432389417\\
89	0.00043745130154558\\
90	0.000618980126108849\\
91	0.000808556213023978\\
92	0.00100688237761286\\
93	0.001215016437067\\
94	0.00143441817404981\\
95	0.0016668379621191\\
96	0.001918932216732\\
97	0.00343108163973639\\
98	0.00644915222848939\\
99	0\\
100	0\\
};
\addlegendentry{$q=-1$};

\addplot [color=black,solid]
  table[row sep=crcr]{%
1	0.00130729513311284\\
2	0.00130729513311284\\
3	0.00130729513311284\\
4	0.00130729513311284\\
5	0.00130729513311284\\
6	0.00130729513311284\\
7	0.00130729513311284\\
8	0.00130729513311284\\
9	0.00130729513311284\\
10	0.00130729513311284\\
11	0.00130729513311284\\
12	0.00130729513311284\\
13	0.00130729513311284\\
14	0.00130729513311284\\
15	0.00130729513311284\\
16	0.00130729513311284\\
17	0.00130729513311284\\
18	0.00130729513311284\\
19	0.00130729513311284\\
20	0.00130729513311284\\
21	0.00130729513311284\\
22	0.00130729513311284\\
23	0.00130729513311284\\
24	0.00130729513311284\\
25	0.00130729513311284\\
26	0.00130729513311284\\
27	0.00130729513311284\\
28	0.00130729513311284\\
29	0.00130729513311284\\
30	0.00130729513311284\\
31	0.00130729513311284\\
32	0.00130729513311284\\
33	0.00130729513311284\\
34	0.00130729513311284\\
35	0.00130729513311284\\
36	0.00130729513311284\\
37	0.00130729513311284\\
38	0.00130729513311284\\
39	0.00130729513311284\\
40	0.00130729513311284\\
41	0.00130729513311284\\
42	0.00130729513311284\\
43	0.00130729513311284\\
44	0.00130729513311284\\
45	0.00130729513311284\\
46	0.00130729513311284\\
47	0.00130729513311284\\
48	0.00130729513311284\\
49	0.00130729513311284\\
50	0.00130729513311284\\
51	0.00130729513311284\\
52	0.00130743933509653\\
53	0.00130771174095063\\
54	0.00130800288106717\\
55	0.00130831553418581\\
56	0.00130865279526993\\
57	0.00130901818959467\\
58	0.00130941588791552\\
59	0.00130985083272548\\
60	0.00131032890334915\\
61	0.00131085712405088\\
62	0.0013114437971423\\
63	0.00131209880499486\\
64	0.00131283383801302\\
65	0.00131366269406093\\
66	0.001268281865654\\
67	0.00120558934074198\\
68	0.00114008837180342\\
69	0.00107159435283429\\
70	0.00100371738587701\\
71	0.00091307686397967\\
72	0.000820755020401308\\
73	0.00072709671425061\\
74	0.000597688852762575\\
75	0.000425127573675561\\
76	0.00025691573944154\\
77	0.000115116394957649\\
78	0\\
79	0\\
80	0\\
81	0\\
82	0\\
83	0\\
84	0\\
85	0\\
86	0\\
87	0\\
88	0\\
89	0\\
90	0\\
91	0\\
92	0\\
93	6.01890667945395e-05\\
94	0.000193027848249216\\
95	0.000345254966843814\\
96	0.00051291918550346\\
97	0.000693902009637616\\
98	0.0037209352564381\\
99	0\\
100	0\\
};
\addlegendentry{$q=0$};

\addplot [color=blue,solid]
  table[row sep=crcr]{%
1	0.00581358135610308\\
2	0.00580939210713567\\
3	0.0058050937908805\\
4	0.00580068178815123\\
5	0.00579614923571526\\
6	0.00579148340189175\\
7	0.00578665686992641\\
8	0.00578160922725733\\
9	0.00577623006028437\\
10	0.00577051285596812\\
11	0.00576465134497519\\
12	0.00575863815967354\\
13	0.00575245686456513\\
14	0.00574605203730685\\
15	0.00573919737280845\\
16	0.00573091454042782\\
17	0.0057169168772906\\
18	0.00569580420338897\\
19	0.00567419718318176\\
20	0.00565209399607671\\
21	0.00562951307353187\\
22	0.00560651926489519\\
23	0.00558326456270097\\
24	0.00555997227965269\\
25	0.00553635559143978\\
26	0.00551187248681073\\
27	0.00548673769999326\\
28	0.00546092408142106\\
29	0.00543440275182262\\
30	0.00540714292545547\\
31	0.005379111566427\\
32	0.00535027255539921\\
33	0.00532058442113743\\
34	0.00528999386523552\\
35	0.00525841694343911\\
36	0.00522568398315746\\
37	0.00519137788987259\\
38	0.0051539925174137\\
39	0.00511251011639995\\
40	0.00507070832560639\\
41	0.00502859720807794\\
42	0.004986187178354\\
43	0.00494349403017208\\
44	0.00490053506200351\\
45	0.0048573485583307\\
46	0.0048139506071797\\
47	0.00477035920121737\\
48	0.00472659467935481\\
49	0.00468268016071996\\
50	0.00463864107990628\\
51	0.00459450008974445\\
52	0.00455025442381132\\
53	0.004505797374668\\
54	0.00446084815808306\\
55	0.00441422360663091\\
56	0.00436742740198791\\
57	0.00432030310991268\\
58	0.00427069395112626\\
59	0.00422176350660739\\
60	0.00417419375772347\\
61	0.00412806713094696\\
62	0.00408348165774458\\
63	0.00404056855403655\\
64	0.00399953744672514\\
65	0.00396078178889732\\
66	0.00389940820738929\\
67	0.00383118322172626\\
68	0.00376895805244447\\
69	0.00370368533482076\\
70	0.00363489740941967\\
71	0.00354818390084217\\
72	0.00345766952772868\\
73	0.00336330381266006\\
74	0.00323701981426088\\
75	0.00307059901634411\\
76	0.00289973615187097\\
77	0.00272390721903009\\
78	0.00254222788903001\\
79	0.00235332669377231\\
80	0.00215726864023762\\
81	0.00195454745849313\\
82	0.00174710381983143\\
83	0.0015357180979922\\
84	0.00132016242518229\\
85	0.00109998454565165\\
86	0.000874516058045971\\
87	0.000644203916859976\\
88	0.000406229553482991\\
89	0.000151190772810909\\
90	0\\
91	0\\
92	0\\
93	0\\
94	0\\
95	0\\
96	0\\
97	0\\
98	0\\
99	0\\
100	0\\
};
\addlegendentry{$q=1$};

\addplot [color=red,solid]
  table[row sep=crcr]{%
1	0.00362906338809373\\
2	0.00361360690675621\\
3	0.00359775445668861\\
4	0.00358149262474337\\
5	0.00356480474859232\\
6	0.00354766608303203\\
7	0.00353003061966084\\
8	0.00351179820079985\\
9	0.00349274207424792\\
10	0.00347241449985361\\
11	0.00345076261620851\\
12	0.0034285766280089\\
13	0.00340583157191294\\
14	0.00338247102409288\\
15	0.00335830029606151\\
16	0.0033325081775922\\
17	0.00330153554594552\\
18	0.00326403657390173\\
19	0.00322535461723309\\
20	0.00318545869303344\\
21	0.00314435073545669\\
22	0.00310211860130669\\
23	0.00305905142525516\\
24	0.00301583116084905\\
25	0.00297352900220973\\
26	0.00293110901419233\\
27	0.00288631834829097\\
28	0.00283992915578842\\
29	0.00279185536335251\\
30	0.00274200387571132\\
31	0.00269027349893577\\
32	0.00263655300096539\\
33	0.00258071675199441\\
34	0.00252261332999874\\
35	0.00246203337680799\\
36	0.0023986160134837\\
37	0.0023315742279362\\
38	0.00225892839982855\\
39	0.0021796790712447\\
40	0.00209800356065331\\
41	0.00201379943624811\\
42	0.00192695429975106\\
43	0.0018373483452371\\
44	0.00174487136101107\\
45	0.00164941211529018\\
46	0.00155090828564596\\
47	0.00144921939535191\\
48	0.00134419713333645\\
49	0.00123568565271801\\
50	0.00112352060552713\\
51	0.00100751946224541\\
52	0.000887435772310189\\
53	0.000762797589458546\\
54	0.000632390550861594\\
55	0.00049370573703769\\
56	0.000348775002135443\\
57	0.000195616353429131\\
58	2.64523120621046e-05\\
59	0\\
60	0\\
61	0\\
62	0\\
63	0\\
64	0\\
65	0\\
66	0\\
67	0\\
68	0\\
69	0\\
70	0\\
71	0\\
72	0\\
73	0\\
74	0\\
75	0\\
76	0\\
77	0\\
78	0\\
79	0\\
80	0\\
81	0\\
82	0\\
83	0\\
84	0\\
85	0\\
86	0\\
87	0\\
88	0\\
89	0\\
90	0\\
91	0\\
92	0\\
93	0\\
94	0\\
95	0\\
96	0\\
97	0\\
98	0\\
99	0\\
100	0\\
};
\addlegendentry{$q=2$};

\addplot [color=mycolor1,solid]
  table[row sep=crcr]{%
1	0.000378445566861253\\
2	0.000344890094495566\\
3	0.000310299170219903\\
4	0.000274633095907648\\
5	0.000237849294288968\\
6	0.000199900826987552\\
7	0.000160730159374424\\
8	0.00012023819830351\\
9	7.81324453553233e-05\\
10	3.30975649942788e-05\\
11	0\\
12	0\\
13	0\\
14	0\\
15	0\\
16	0\\
17	0\\
18	0\\
19	0\\
20	0\\
21	0\\
22	0\\
23	0\\
24	0\\
25	0\\
26	0\\
27	0\\
28	0\\
29	0\\
30	0\\
31	0\\
32	0\\
33	0\\
34	0\\
35	0\\
36	0\\
37	0\\
38	0\\
39	0\\
40	0\\
41	0\\
42	0\\
43	0\\
44	0\\
45	0\\
46	0\\
47	0\\
48	0\\
49	0\\
50	0\\
51	0\\
52	0\\
53	0\\
54	0\\
55	0\\
56	0\\
57	0\\
58	0\\
59	0\\
60	0\\
61	0\\
62	0\\
63	0\\
64	0\\
65	0\\
66	0\\
67	0\\
68	0\\
69	0\\
70	0\\
71	0\\
72	0\\
73	0\\
74	0\\
75	0\\
76	0\\
77	0\\
78	0\\
79	0\\
80	0\\
81	0\\
82	0\\
83	0\\
84	0\\
85	0\\
86	0\\
87	0\\
88	0\\
89	0\\
90	0\\
91	0\\
92	0\\
93	0\\
94	0\\
95	0\\
96	0\\
97	0\\
98	0\\
99	0\\
100	0\\
};
\addlegendentry{$q=3$};

\addplot [color=green,solid]
  table[row sep=crcr]{%
1	0\\
2	0\\
3	0\\
4	0\\
5	0\\
6	0\\
7	0\\
8	0\\
9	0\\
10	0\\
11	0\\
12	0\\
13	0\\
14	0\\
15	0\\
16	0\\
17	0\\
18	0\\
19	0\\
20	0\\
21	0\\
22	0\\
23	0\\
24	0\\
25	0\\
26	0\\
27	0\\
28	0\\
29	0\\
30	0\\
31	0\\
32	0\\
33	0\\
34	0\\
35	0\\
36	0\\
37	0\\
38	0\\
39	0\\
40	0\\
41	0\\
42	0\\
43	0\\
44	0\\
45	0\\
46	0\\
47	0\\
48	0\\
49	0\\
50	0\\
51	0\\
52	0\\
53	0\\
54	0\\
55	0\\
56	0\\
57	0\\
58	0\\
59	0\\
60	0\\
61	0\\
62	0\\
63	0\\
64	0\\
65	0\\
66	0\\
67	0\\
68	0\\
69	0\\
70	0\\
71	0\\
72	0\\
73	0\\
74	0\\
75	0\\
76	0\\
77	0\\
78	0\\
79	0\\
80	0\\
81	0\\
82	0\\
83	0\\
84	0\\
85	0\\
86	0\\
87	0\\
88	0\\
89	0\\
90	0\\
91	0\\
92	0\\
93	0\\
94	0\\
95	0\\
96	0\\
97	0\\
98	0\\
99	0\\
100	0\\
};
\addlegendentry{$q=4$};

\end{axis}
\end{tikzpicture}% 
  \caption{Discrete Time}
\end{subfigure}\\
\vspace{1cm}
\begin{subfigure}{.45\linewidth}
  \centering
  \setlength\figureheight{\linewidth} 
  \setlength\figurewidth{\linewidth}
  \tikzsetnextfilename{dm_cts_nFPC_z1}
  % This file was created by matlab2tikz.
%
%The latest updates can be retrieved from
%  http://www.mathworks.com/matlabcentral/fileexchange/22022-matlab2tikz-matlab2tikz
%where you can also make suggestions and rate matlab2tikz.
%
\definecolor{mycolor1}{rgb}{1.00000,0.00000,1.00000}%
%
\begin{tikzpicture}

\begin{axis}[%
width=4.564in,
height=3.803in,
at={(1.067in,0.513in)},
scale only axis,
every outer x axis line/.append style={black},
every x tick label/.append style={font=\color{black}},
xmin=0,
xmax=100,
xlabel={Time},
every outer y axis line/.append style={black},
every y tick label/.append style={font=\color{black}},
ymin=0,
ymax=0.012,
ylabel={Depth $\delta$},
axis background/.style={fill=white},
title={Z=1},
axis x line*=bottom,
axis y line*=left,
legend style={legend cell align=left,align=left,draw=black}
]
\addplot [color=green,dashed,forget plot]
  table[row sep=crcr]{%
0.01	0.01\\
0.02	0.01\\
0.03	0.01\\
0.04	0.01\\
0.05	0.01\\
0.06	0.01\\
0.07	0.01\\
0.08	0.01\\
0.09	0.01\\
0.1	0.01\\
0.11	0.01\\
0.12	0.01\\
0.13	0.01\\
0.14	0.01\\
0.15	0.01\\
0.16	0.01\\
0.17	0.01\\
0.18	0.01\\
0.19	0.01\\
0.2	0.01\\
0.21	0.01\\
0.22	0.01\\
0.23	0.01\\
0.24	0.01\\
0.25	0.01\\
0.26	0.01\\
0.27	0.01\\
0.28	0.01\\
0.29	0.01\\
0.3	0.01\\
0.31	0.01\\
0.32	0.01\\
0.33	0.01\\
0.34	0.01\\
0.35	0.01\\
0.36	0.01\\
0.37	0.01\\
0.38	0.01\\
0.39	0.01\\
0.4	0.01\\
0.41	0.01\\
0.42	0.01\\
0.43	0.01\\
0.44	0.01\\
0.45	0.01\\
0.46	0.01\\
0.47	0.01\\
0.48	0.01\\
0.49	0.01\\
0.5	0.01\\
0.51	0.01\\
0.52	0.01\\
0.53	0.01\\
0.54	0.01\\
0.55	0.01\\
0.56	0.01\\
0.57	0.01\\
0.58	0.01\\
0.59	0.01\\
0.6	0.01\\
0.61	0.01\\
0.62	0.01\\
0.63	0.01\\
0.64	0.01\\
0.65	0.01\\
0.66	0.01\\
0.67	0.01\\
0.68	0.01\\
0.69	0.01\\
0.7	0.01\\
0.71	0.01\\
0.72	0.01\\
0.73	0.01\\
0.74	0.01\\
0.75	0.01\\
0.76	0.01\\
0.77	0.01\\
0.78	0.01\\
0.79	0.01\\
0.8	0.01\\
0.81	0.01\\
0.82	0.01\\
0.83	0.01\\
0.84	0.01\\
0.85	0.01\\
0.86	0.01\\
0.87	0.01\\
0.88	0.01\\
0.89	0.01\\
0.9	0.01\\
0.91	0.01\\
0.92	0.01\\
0.93	0.01\\
0.94	0.01\\
0.95	0.01\\
0.96	0.01\\
0.97	0.01\\
0.98	0.01\\
0.99	0.01\\
1	0.01\\
1.01	0.01\\
1.02	0.01\\
1.03	0.01\\
1.04	0.01\\
1.05	0.01\\
1.06	0.01\\
1.07	0.01\\
1.08	0.01\\
1.09	0.01\\
1.1	0.01\\
1.11	0.01\\
1.12	0.01\\
1.13	0.01\\
1.14	0.01\\
1.15	0.01\\
1.16	0.01\\
1.17	0.01\\
1.18	0.01\\
1.19	0.01\\
1.2	0.01\\
1.21	0.01\\
1.22	0.01\\
1.23	0.01\\
1.24	0.01\\
1.25	0.01\\
1.26	0.01\\
1.27	0.01\\
1.28	0.01\\
1.29	0.01\\
1.3	0.01\\
1.31	0.01\\
1.32	0.01\\
1.33	0.01\\
1.34	0.01\\
1.35	0.01\\
1.36	0.01\\
1.37	0.01\\
1.38	0.01\\
1.39	0.01\\
1.4	0.01\\
1.41	0.01\\
1.42	0.01\\
1.43	0.01\\
1.44	0.01\\
1.45	0.01\\
1.46	0.01\\
1.47	0.01\\
1.48	0.01\\
1.49	0.01\\
1.5	0.01\\
1.51	0.01\\
1.52	0.01\\
1.53	0.01\\
1.54	0.01\\
1.55	0.01\\
1.56	0.01\\
1.57	0.01\\
1.58	0.01\\
1.59	0.01\\
1.6	0.01\\
1.61	0.01\\
1.62	0.01\\
1.63	0.01\\
1.64	0.01\\
1.65	0.01\\
1.66	0.01\\
1.67	0.01\\
1.68	0.01\\
1.69	0.01\\
1.7	0.01\\
1.71	0.01\\
1.72	0.01\\
1.73	0.01\\
1.74	0.01\\
1.75	0.01\\
1.76	0.01\\
1.77	0.01\\
1.78	0.01\\
1.79	0.01\\
1.8	0.01\\
1.81	0.01\\
1.82	0.01\\
1.83	0.01\\
1.84	0.01\\
1.85	0.01\\
1.86	0.01\\
1.87	0.01\\
1.88	0.01\\
1.89	0.01\\
1.9	0.01\\
1.91	0.01\\
1.92	0.01\\
1.93	0.01\\
1.94	0.01\\
1.95	0.01\\
1.96	0.01\\
1.97	0.01\\
1.98	0.01\\
1.99	0.01\\
2	0.01\\
2.01	0.01\\
2.02	0.01\\
2.03	0.01\\
2.04	0.01\\
2.05	0.01\\
2.06	0.01\\
2.07	0.01\\
2.08	0.01\\
2.09	0.01\\
2.1	0.01\\
2.11	0.01\\
2.12	0.01\\
2.13	0.01\\
2.14	0.01\\
2.15	0.01\\
2.16	0.01\\
2.17	0.01\\
2.18	0.01\\
2.19	0.01\\
2.2	0.01\\
2.21	0.01\\
2.22	0.01\\
2.23	0.01\\
2.24	0.01\\
2.25	0.01\\
2.26	0.01\\
2.27	0.01\\
2.28	0.01\\
2.29	0.01\\
2.3	0.01\\
2.31	0.01\\
2.32	0.01\\
2.33	0.01\\
2.34	0.01\\
2.35	0.01\\
2.36	0.01\\
2.37	0.01\\
2.38	0.01\\
2.39	0.01\\
2.4	0.01\\
2.41	0.01\\
2.42	0.01\\
2.43	0.01\\
2.44	0.01\\
2.45	0.01\\
2.46	0.01\\
2.47	0.01\\
2.48	0.01\\
2.49	0.01\\
2.5	0.01\\
2.51	0.01\\
2.52	0.01\\
2.53	0.01\\
2.54	0.01\\
2.55	0.01\\
2.56	0.01\\
2.57	0.01\\
2.58	0.01\\
2.59	0.01\\
2.6	0.01\\
2.61	0.01\\
2.62	0.01\\
2.63	0.01\\
2.64	0.01\\
2.65	0.01\\
2.66	0.01\\
2.67	0.01\\
2.68	0.01\\
2.69	0.01\\
2.7	0.01\\
2.71	0.01\\
2.72	0.01\\
2.73	0.01\\
2.74	0.01\\
2.75	0.01\\
2.76	0.01\\
2.77	0.01\\
2.78	0.01\\
2.79	0.01\\
2.8	0.01\\
2.81	0.01\\
2.82	0.01\\
2.83	0.01\\
2.84	0.01\\
2.85	0.01\\
2.86	0.01\\
2.87	0.01\\
2.88	0.01\\
2.89	0.01\\
2.9	0.01\\
2.91	0.01\\
2.92	0.01\\
2.93	0.01\\
2.94	0.01\\
2.95	0.01\\
2.96	0.01\\
2.97	0.01\\
2.98	0.01\\
2.99	0.01\\
3	0.01\\
3.01	0.01\\
3.02	0.01\\
3.03	0.01\\
3.04	0.01\\
3.05	0.01\\
3.06	0.01\\
3.07	0.01\\
3.08	0.01\\
3.09	0.01\\
3.1	0.01\\
3.11	0.01\\
3.12	0.01\\
3.13	0.01\\
3.14	0.01\\
3.15	0.01\\
3.16	0.01\\
3.17	0.01\\
3.18	0.01\\
3.19	0.01\\
3.2	0.01\\
3.21	0.01\\
3.22	0.01\\
3.23	0.01\\
3.24	0.01\\
3.25	0.01\\
3.26	0.01\\
3.27	0.01\\
3.28	0.01\\
3.29	0.01\\
3.3	0.01\\
3.31	0.01\\
3.32	0.01\\
3.33	0.01\\
3.34	0.01\\
3.35	0.01\\
3.36	0.01\\
3.37	0.01\\
3.38	0.01\\
3.39	0.01\\
3.4	0.01\\
3.41	0.01\\
3.42	0.01\\
3.43	0.01\\
3.44	0.01\\
3.45	0.01\\
3.46	0.01\\
3.47	0.01\\
3.48	0.01\\
3.49	0.01\\
3.5	0.01\\
3.51	0.01\\
3.52	0.01\\
3.53	0.01\\
3.54	0.01\\
3.55	0.01\\
3.56	0.01\\
3.57	0.01\\
3.58	0.01\\
3.59	0.01\\
3.6	0.01\\
3.61	0.01\\
3.62	0.01\\
3.63	0.01\\
3.64	0.01\\
3.65	0.01\\
3.66	0.01\\
3.67	0.01\\
3.68	0.01\\
3.69	0.01\\
3.7	0.01\\
3.71	0.01\\
3.72	0.01\\
3.73	0.01\\
3.74	0.01\\
3.75	0.01\\
3.76	0.01\\
3.77	0.01\\
3.78	0.01\\
3.79	0.01\\
3.8	0.01\\
3.81	0.01\\
3.82	0.01\\
3.83	0.01\\
3.84	0.01\\
3.85	0.01\\
3.86	0.01\\
3.87	0.01\\
3.88	0.01\\
3.89	0.01\\
3.9	0.01\\
3.91	0.01\\
3.92	0.01\\
3.93	0.01\\
3.94	0.01\\
3.95	0.01\\
3.96	0.01\\
3.97	0.01\\
3.98	0.01\\
3.99	0.01\\
4	0.01\\
4.01	0.01\\
4.02	0.01\\
4.03	0.01\\
4.04	0.01\\
4.05	0.01\\
4.06	0.01\\
4.07	0.01\\
4.08	0.01\\
4.09	0.01\\
4.1	0.01\\
4.11	0.01\\
4.12	0.01\\
4.13	0.01\\
4.14	0.01\\
4.15	0.01\\
4.16	0.01\\
4.17	0.01\\
4.18	0.01\\
4.19	0.01\\
4.2	0.01\\
4.21	0.01\\
4.22	0.01\\
4.23	0.01\\
4.24	0.01\\
4.25	0.01\\
4.26	0.01\\
4.27	0.01\\
4.28	0.01\\
4.29	0.01\\
4.3	0.01\\
4.31	0.01\\
4.32	0.01\\
4.33	0.01\\
4.34	0.01\\
4.35	0.01\\
4.36	0.01\\
4.37	0.01\\
4.38	0.01\\
4.39	0.01\\
4.4	0.01\\
4.41	0.01\\
4.42	0.01\\
4.43	0.01\\
4.44	0.01\\
4.45	0.01\\
4.46	0.01\\
4.47	0.01\\
4.48	0.01\\
4.49	0.01\\
4.5	0.01\\
4.51	0.01\\
4.52	0.01\\
4.53	0.01\\
4.54	0.01\\
4.55	0.01\\
4.56	0.01\\
4.57	0.01\\
4.58	0.01\\
4.59	0.01\\
4.6	0.01\\
4.61	0.01\\
4.62	0.01\\
4.63	0.01\\
4.64	0.01\\
4.65	0.01\\
4.66	0.01\\
4.67	0.01\\
4.68	0.01\\
4.69	0.01\\
4.7	0.01\\
4.71	0.01\\
4.72	0.01\\
4.73	0.01\\
4.74	0.01\\
4.75	0.01\\
4.76	0.01\\
4.77	0.01\\
4.78	0.01\\
4.79	0.01\\
4.8	0.01\\
4.81	0.01\\
4.82	0.01\\
4.83	0.01\\
4.84	0.01\\
4.85	0.01\\
4.86	0.01\\
4.87	0.01\\
4.88	0.01\\
4.89	0.01\\
4.9	0.01\\
4.91	0.01\\
4.92	0.01\\
4.93	0.01\\
4.94	0.01\\
4.95	0.01\\
4.96	0.01\\
4.97	0.01\\
4.98	0.01\\
4.99	0.01\\
5	0.01\\
5.01	0.01\\
5.02	0.01\\
5.03	0.01\\
5.04	0.01\\
5.05	0.01\\
5.06	0.01\\
5.07	0.01\\
5.08	0.01\\
5.09	0.01\\
5.1	0.01\\
5.11	0.01\\
5.12	0.01\\
5.13	0.01\\
5.14	0.01\\
5.15	0.01\\
5.16	0.01\\
5.17	0.01\\
5.18	0.01\\
5.19	0.01\\
5.2	0.01\\
5.21	0.01\\
5.22	0.01\\
5.23	0.01\\
5.24	0.01\\
5.25	0.01\\
5.26	0.01\\
5.27	0.01\\
5.28	0.01\\
5.29	0.01\\
5.3	0.01\\
5.31	0.01\\
5.32	0.01\\
5.33	0.01\\
5.34	0.01\\
5.35	0.01\\
5.36	0.01\\
5.37	0.01\\
5.38	0.01\\
5.39	0.01\\
5.4	0.01\\
5.41	0.01\\
5.42	0.01\\
5.43	0.01\\
5.44	0.01\\
5.45	0.01\\
5.46	0.01\\
5.47	0.01\\
5.48	0.01\\
5.49	0.01\\
5.5	0.01\\
5.51	0.01\\
5.52	0.01\\
5.53	0.01\\
5.54	0.01\\
5.55	0.01\\
5.56	0.01\\
5.57	0.01\\
5.58	0.01\\
5.59	0.01\\
5.6	0.01\\
5.61	0.01\\
5.62	0.01\\
5.63	0.01\\
5.64	0.01\\
5.65	0.01\\
5.66	0.01\\
5.67	0.01\\
5.68	0.01\\
5.69	0.01\\
5.7	0.01\\
5.71	0.01\\
5.72	0.01\\
5.73	0.01\\
5.74	0.01\\
5.75	0.01\\
5.76	0.01\\
5.77	0.01\\
5.78	0.01\\
5.79	0.01\\
5.8	0.01\\
5.81	0.01\\
5.82	0.01\\
5.83	0.01\\
5.84	0.01\\
5.85	0.01\\
5.86	0.01\\
5.87	0.01\\
5.88	0.01\\
5.89	0.01\\
5.9	0.01\\
5.91	0.01\\
5.92	0.01\\
5.93	0.01\\
5.94	0.01\\
5.95	0.01\\
5.96	0.01\\
5.97	0.01\\
5.98	0.01\\
5.99	0.01\\
6	0.01\\
6.01	0.01\\
6.02	0.01\\
6.03	0.01\\
6.04	0.01\\
6.05	0.01\\
6.06	0.01\\
6.07	0.01\\
6.08	0.01\\
6.09	0.01\\
6.1	0.01\\
6.11	0.01\\
6.12	0.01\\
6.13	0.01\\
6.14	0.01\\
6.15	0.01\\
6.16	0.01\\
6.17	0.01\\
6.18	0.01\\
6.19	0.01\\
6.2	0.01\\
6.21	0.01\\
6.22	0.01\\
6.23	0.01\\
6.24	0.01\\
6.25	0.01\\
6.26	0.01\\
6.27	0.01\\
6.28	0.01\\
6.29	0.01\\
6.3	0.01\\
6.31	0.01\\
6.32	0.01\\
6.33	0.01\\
6.34	0.01\\
6.35	0.01\\
6.36	0.01\\
6.37	0.01\\
6.38	0.01\\
6.39	0.01\\
6.4	0.01\\
6.41	0.01\\
6.42	0.01\\
6.43	0.01\\
6.44	0.01\\
6.45	0.01\\
6.46	0.01\\
6.47	0.01\\
6.48	0.01\\
6.49	0.01\\
6.5	0.01\\
6.51	0.01\\
6.52	0.01\\
6.53	0.01\\
6.54	0.01\\
6.55	0.01\\
6.56	0.01\\
6.57	0.01\\
6.58	0.01\\
6.59	0.01\\
6.6	0.01\\
6.61	0.01\\
6.62	0.01\\
6.63	0.01\\
6.64	0.01\\
6.65	0.01\\
6.66	0.01\\
6.67	0.01\\
6.68	0.01\\
6.69	0.01\\
6.7	0.01\\
6.71	0.01\\
6.72	0.01\\
6.73	0.01\\
6.74	0.01\\
6.75	0.01\\
6.76	0.01\\
6.77	0.01\\
6.78	0.01\\
6.79	0.01\\
6.8	0.01\\
6.81	0.01\\
6.82	0.01\\
6.83	0.01\\
6.84	0.01\\
6.85	0.01\\
6.86	0.01\\
6.87	0.01\\
6.88	0.01\\
6.89	0.01\\
6.9	0.01\\
6.91	0.01\\
6.92	0.01\\
6.93	0.01\\
6.94	0.01\\
6.95	0.01\\
6.96	0.01\\
6.97	0.01\\
6.98	0.01\\
6.99	0.01\\
7	0.01\\
7.01	0.01\\
7.02	0.01\\
7.03	0.01\\
7.04	0.01\\
7.05	0.01\\
7.06	0.01\\
7.07	0.01\\
7.08	0.01\\
7.09	0.01\\
7.1	0.01\\
7.11	0.01\\
7.12	0.01\\
7.13	0.01\\
7.14	0.01\\
7.15	0.01\\
7.16	0.01\\
7.17	0.01\\
7.18	0.01\\
7.19	0.01\\
7.2	0.01\\
7.21	0.01\\
7.22	0.01\\
7.23	0.01\\
7.24	0.01\\
7.25	0.01\\
7.26	0.01\\
7.27	0.01\\
7.28	0.01\\
7.29	0.01\\
7.3	0.01\\
7.31	0.01\\
7.32	0.01\\
7.33	0.01\\
7.34	0.01\\
7.35	0.01\\
7.36	0.01\\
7.37	0.01\\
7.38	0.01\\
7.39	0.01\\
7.4	0.01\\
7.41	0.01\\
7.42	0.01\\
7.43	0.01\\
7.44	0.01\\
7.45	0.01\\
7.46	0.01\\
7.47	0.01\\
7.48	0.01\\
7.49	0.01\\
7.5	0.01\\
7.51	0.01\\
7.52	0.01\\
7.53	0.01\\
7.54	0.01\\
7.55	0.01\\
7.56	0.01\\
7.57	0.01\\
7.58	0.01\\
7.59	0.01\\
7.6	0.01\\
7.61	0.01\\
7.62	0.01\\
7.63	0.01\\
7.64	0.01\\
7.65	0.01\\
7.66	0.01\\
7.67	0.01\\
7.68	0.01\\
7.69	0.01\\
7.7	0.01\\
7.71	0.01\\
7.72	0.01\\
7.73	0.01\\
7.74	0.01\\
7.75	0.01\\
7.76	0.01\\
7.77	0.01\\
7.78	0.01\\
7.79	0.01\\
7.8	0.01\\
7.81	0.01\\
7.82	0.01\\
7.83	0.01\\
7.84	0.01\\
7.85	0.01\\
7.86	0.01\\
7.87	0.01\\
7.88	0.01\\
7.89	0.01\\
7.9	0.01\\
7.91	0.01\\
7.92	0.01\\
7.93	0.01\\
7.94	0.01\\
7.95	0.01\\
7.96	0.01\\
7.97	0.01\\
7.98	0.01\\
7.99	0.01\\
8	0.01\\
8.01	0.01\\
8.02	0.01\\
8.03	0.01\\
8.04	0.01\\
8.05	0.01\\
8.06	0.01\\
8.07	0.01\\
8.08	0.01\\
8.09	0.01\\
8.1	0.01\\
8.11	0.01\\
8.12	0.01\\
8.13	0.01\\
8.14	0.01\\
8.15	0.01\\
8.16	0.01\\
8.17	0.01\\
8.18	0.01\\
8.19	0.01\\
8.2	0.01\\
8.21	0.01\\
8.22	0.01\\
8.23	0.01\\
8.24	0.01\\
8.25	0.01\\
8.26	0.01\\
8.27	0.01\\
8.28	0.01\\
8.29	0.01\\
8.3	0.01\\
8.31	0.01\\
8.32	0.01\\
8.33	0.01\\
8.34	0.01\\
8.35	0.01\\
8.36	0.01\\
8.37	0.01\\
8.38	0.01\\
8.39	0.01\\
8.4	0.01\\
8.41	0.01\\
8.42	0.01\\
8.43	0.01\\
8.44	0.01\\
8.45	0.01\\
8.46	0.01\\
8.47	0.01\\
8.48	0.01\\
8.49	0.01\\
8.5	0.01\\
8.51	0.01\\
8.52	0.01\\
8.53	0.01\\
8.54	0.01\\
8.55	0.01\\
8.56	0.01\\
8.57	0.01\\
8.58	0.01\\
8.59	0.01\\
8.6	0.01\\
8.61	0.01\\
8.62	0.01\\
8.63	0.01\\
8.64	0.01\\
8.65	0.01\\
8.66	0.01\\
8.67	0.01\\
8.68	0.01\\
8.69	0.01\\
8.7	0.01\\
8.71	0.01\\
8.72	0.01\\
8.73	0.01\\
8.74	0.01\\
8.75	0.01\\
8.76	0.01\\
8.77	0.01\\
8.78	0.01\\
8.79	0.01\\
8.8	0.01\\
8.81	0.01\\
8.82	0.01\\
8.83	0.01\\
8.84	0.01\\
8.85	0.01\\
8.86	0.01\\
8.87	0.01\\
8.88	0.01\\
8.89	0.01\\
8.9	0.01\\
8.91	0.01\\
8.92	0.01\\
8.93	0.01\\
8.94	0.01\\
8.95	0.01\\
8.96	0.01\\
8.97	0.01\\
8.98	0.01\\
8.99	0.01\\
9	0.01\\
9.01	0.01\\
9.02	0.01\\
9.03	0.01\\
9.04	0.01\\
9.05	0.01\\
9.06	0.01\\
9.07	0.01\\
9.08	0.01\\
9.09	0.01\\
9.1	0.01\\
9.11	0.01\\
9.12	0.01\\
9.13	0.01\\
9.14	0.01\\
9.15	0.01\\
9.16	0.01\\
9.17	0.01\\
9.18	0.01\\
9.19	0.01\\
9.2	0.01\\
9.21	0.01\\
9.22	0.01\\
9.23	0.01\\
9.24	0.01\\
9.25	0.01\\
9.26	0.01\\
9.27	0.01\\
9.28	0.01\\
9.29	0.01\\
9.3	0.01\\
9.31	0.01\\
9.32	0.01\\
9.33	0.01\\
9.34	0.01\\
9.35	0.01\\
9.36	0.01\\
9.37	0.01\\
9.38	0.01\\
9.39	0.01\\
9.4	0.01\\
9.41	0.01\\
9.42	0.01\\
9.43	0.01\\
9.44	0.01\\
9.45	0.01\\
9.46	0.01\\
9.47	0.01\\
9.48	0.01\\
9.49	0.01\\
9.5	0.01\\
9.51	0.01\\
9.52	0.01\\
9.53	0.01\\
9.54	0.01\\
9.55	0.01\\
9.56	0.01\\
9.57	0.01\\
9.58	0.01\\
9.59	0.01\\
9.6	0.01\\
9.61	0.01\\
9.62	0.01\\
9.63	0.01\\
9.64	0.01\\
9.65	0.01\\
9.66	0.01\\
9.67	0.01\\
9.68	0.01\\
9.69	0.01\\
9.7	0.01\\
9.71	0.01\\
9.72	0.01\\
9.73	0.01\\
9.74	0.01\\
9.75	0.01\\
9.76	0.01\\
9.77	0.01\\
9.78	0.01\\
9.79	0.01\\
9.8	0.01\\
9.81	0.01\\
9.82	0.01\\
9.83	0.01\\
9.84	0.01\\
9.85	0.01\\
9.86	0.01\\
9.87	0.01\\
9.88	0.01\\
9.89	0.01\\
9.9	0.01\\
9.91	0.01\\
9.92	0.01\\
9.93	0.01\\
9.94	0.01\\
9.95	0.01\\
9.96	0.01\\
9.97	0.01\\
9.98	0.01\\
9.99	0.01\\
10	0.01\\
10.01	0.01\\
10.02	0.01\\
10.03	0.01\\
10.04	0.01\\
10.05	0.01\\
10.06	0.01\\
10.07	0.01\\
10.08	0.01\\
10.09	0.01\\
10.1	0.01\\
10.11	0.01\\
10.12	0.01\\
10.13	0.01\\
10.14	0.01\\
10.15	0.01\\
10.16	0.01\\
10.17	0.01\\
10.18	0.01\\
10.19	0.01\\
10.2	0.01\\
10.21	0.01\\
10.22	0.01\\
10.23	0.01\\
10.24	0.01\\
10.25	0.01\\
10.26	0.01\\
10.27	0.01\\
10.28	0.01\\
10.29	0.01\\
10.3	0.01\\
10.31	0.01\\
10.32	0.01\\
10.33	0.01\\
10.34	0.01\\
10.35	0.01\\
10.36	0.01\\
10.37	0.01\\
10.38	0.01\\
10.39	0.01\\
10.4	0.01\\
10.41	0.01\\
10.42	0.01\\
10.43	0.01\\
10.44	0.01\\
10.45	0.01\\
10.46	0.01\\
10.47	0.01\\
10.48	0.01\\
10.49	0.01\\
10.5	0.01\\
10.51	0.01\\
10.52	0.01\\
10.53	0.01\\
10.54	0.01\\
10.55	0.01\\
10.56	0.01\\
10.57	0.01\\
10.58	0.01\\
10.59	0.01\\
10.6	0.01\\
10.61	0.01\\
10.62	0.01\\
10.63	0.01\\
10.64	0.01\\
10.65	0.01\\
10.66	0.01\\
10.67	0.01\\
10.68	0.01\\
10.69	0.01\\
10.7	0.01\\
10.71	0.01\\
10.72	0.01\\
10.73	0.01\\
10.74	0.01\\
10.75	0.01\\
10.76	0.01\\
10.77	0.01\\
10.78	0.01\\
10.79	0.01\\
10.8	0.01\\
10.81	0.01\\
10.82	0.01\\
10.83	0.01\\
10.84	0.01\\
10.85	0.01\\
10.86	0.01\\
10.87	0.01\\
10.88	0.01\\
10.89	0.01\\
10.9	0.01\\
10.91	0.01\\
10.92	0.01\\
10.93	0.01\\
10.94	0.01\\
10.95	0.01\\
10.96	0.01\\
10.97	0.01\\
10.98	0.01\\
10.99	0.01\\
11	0.01\\
11.01	0.01\\
11.02	0.01\\
11.03	0.01\\
11.04	0.01\\
11.05	0.01\\
11.06	0.01\\
11.07	0.01\\
11.08	0.01\\
11.09	0.01\\
11.1	0.01\\
11.11	0.01\\
11.12	0.01\\
11.13	0.01\\
11.14	0.01\\
11.15	0.01\\
11.16	0.01\\
11.17	0.01\\
11.18	0.01\\
11.19	0.01\\
11.2	0.01\\
11.21	0.01\\
11.22	0.01\\
11.23	0.01\\
11.24	0.01\\
11.25	0.01\\
11.26	0.01\\
11.27	0.01\\
11.28	0.01\\
11.29	0.01\\
11.3	0.01\\
11.31	0.01\\
11.32	0.01\\
11.33	0.01\\
11.34	0.01\\
11.35	0.01\\
11.36	0.01\\
11.37	0.01\\
11.38	0.01\\
11.39	0.01\\
11.4	0.01\\
11.41	0.01\\
11.42	0.01\\
11.43	0.01\\
11.44	0.01\\
11.45	0.01\\
11.46	0.01\\
11.47	0.01\\
11.48	0.01\\
11.49	0.01\\
11.5	0.01\\
11.51	0.01\\
11.52	0.01\\
11.53	0.01\\
11.54	0.01\\
11.55	0.01\\
11.56	0.01\\
11.57	0.01\\
11.58	0.01\\
11.59	0.01\\
11.6	0.01\\
11.61	0.01\\
11.62	0.01\\
11.63	0.01\\
11.64	0.01\\
11.65	0.01\\
11.66	0.01\\
11.67	0.01\\
11.68	0.01\\
11.69	0.01\\
11.7	0.01\\
11.71	0.01\\
11.72	0.01\\
11.73	0.01\\
11.74	0.01\\
11.75	0.01\\
11.76	0.01\\
11.77	0.01\\
11.78	0.01\\
11.79	0.01\\
11.8	0.01\\
11.81	0.01\\
11.82	0.01\\
11.83	0.01\\
11.84	0.01\\
11.85	0.01\\
11.86	0.01\\
11.87	0.01\\
11.88	0.01\\
11.89	0.01\\
11.9	0.01\\
11.91	0.01\\
11.92	0.01\\
11.93	0.01\\
11.94	0.01\\
11.95	0.01\\
11.96	0.01\\
11.97	0.01\\
11.98	0.01\\
11.99	0.01\\
12	0.01\\
12.01	0.01\\
12.02	0.01\\
12.03	0.01\\
12.04	0.01\\
12.05	0.01\\
12.06	0.01\\
12.07	0.01\\
12.08	0.01\\
12.09	0.01\\
12.1	0.01\\
12.11	0.01\\
12.12	0.01\\
12.13	0.01\\
12.14	0.01\\
12.15	0.01\\
12.16	0.01\\
12.17	0.01\\
12.18	0.01\\
12.19	0.01\\
12.2	0.01\\
12.21	0.01\\
12.22	0.01\\
12.23	0.01\\
12.24	0.01\\
12.25	0.01\\
12.26	0.01\\
12.27	0.01\\
12.28	0.01\\
12.29	0.01\\
12.3	0.01\\
12.31	0.01\\
12.32	0.01\\
12.33	0.01\\
12.34	0.01\\
12.35	0.01\\
12.36	0.01\\
12.37	0.01\\
12.38	0.01\\
12.39	0.01\\
12.4	0.01\\
12.41	0.01\\
12.42	0.01\\
12.43	0.01\\
12.44	0.01\\
12.45	0.01\\
12.46	0.01\\
12.47	0.01\\
12.48	0.01\\
12.49	0.01\\
12.5	0.01\\
12.51	0.01\\
12.52	0.01\\
12.53	0.01\\
12.54	0.01\\
12.55	0.01\\
12.56	0.01\\
12.57	0.01\\
12.58	0.01\\
12.59	0.01\\
12.6	0.01\\
12.61	0.01\\
12.62	0.01\\
12.63	0.01\\
12.64	0.01\\
12.65	0.01\\
12.66	0.01\\
12.67	0.01\\
12.68	0.01\\
12.69	0.01\\
12.7	0.01\\
12.71	0.01\\
12.72	0.01\\
12.73	0.01\\
12.74	0.01\\
12.75	0.01\\
12.76	0.01\\
12.77	0.01\\
12.78	0.01\\
12.79	0.01\\
12.8	0.01\\
12.81	0.01\\
12.82	0.01\\
12.83	0.01\\
12.84	0.01\\
12.85	0.01\\
12.86	0.01\\
12.87	0.01\\
12.88	0.01\\
12.89	0.01\\
12.9	0.01\\
12.91	0.01\\
12.92	0.01\\
12.93	0.01\\
12.94	0.01\\
12.95	0.01\\
12.96	0.01\\
12.97	0.01\\
12.98	0.01\\
12.99	0.01\\
13	0.01\\
13.01	0.01\\
13.02	0.01\\
13.03	0.01\\
13.04	0.01\\
13.05	0.01\\
13.06	0.01\\
13.07	0.01\\
13.08	0.01\\
13.09	0.01\\
13.1	0.01\\
13.11	0.01\\
13.12	0.01\\
13.13	0.01\\
13.14	0.01\\
13.15	0.01\\
13.16	0.01\\
13.17	0.01\\
13.18	0.01\\
13.19	0.01\\
13.2	0.01\\
13.21	0.01\\
13.22	0.01\\
13.23	0.01\\
13.24	0.01\\
13.25	0.01\\
13.26	0.01\\
13.27	0.01\\
13.28	0.01\\
13.29	0.01\\
13.3	0.01\\
13.31	0.01\\
13.32	0.01\\
13.33	0.01\\
13.34	0.01\\
13.35	0.01\\
13.36	0.01\\
13.37	0.01\\
13.38	0.01\\
13.39	0.01\\
13.4	0.01\\
13.41	0.01\\
13.42	0.01\\
13.43	0.01\\
13.44	0.01\\
13.45	0.01\\
13.46	0.01\\
13.47	0.01\\
13.48	0.01\\
13.49	0.01\\
13.5	0.01\\
13.51	0.01\\
13.52	0.01\\
13.53	0.01\\
13.54	0.01\\
13.55	0.01\\
13.56	0.01\\
13.57	0.01\\
13.58	0.01\\
13.59	0.01\\
13.6	0.01\\
13.61	0.01\\
13.62	0.01\\
13.63	0.01\\
13.64	0.01\\
13.65	0.01\\
13.66	0.01\\
13.67	0.01\\
13.68	0.01\\
13.69	0.01\\
13.7	0.01\\
13.71	0.01\\
13.72	0.01\\
13.73	0.01\\
13.74	0.01\\
13.75	0.01\\
13.76	0.01\\
13.77	0.01\\
13.78	0.01\\
13.79	0.01\\
13.8	0.01\\
13.81	0.01\\
13.82	0.01\\
13.83	0.01\\
13.84	0.01\\
13.85	0.01\\
13.86	0.01\\
13.87	0.01\\
13.88	0.01\\
13.89	0.01\\
13.9	0.01\\
13.91	0.01\\
13.92	0.01\\
13.93	0.01\\
13.94	0.01\\
13.95	0.01\\
13.96	0.01\\
13.97	0.01\\
13.98	0.01\\
13.99	0.01\\
14	0.01\\
14.01	0.01\\
14.02	0.01\\
14.03	0.01\\
14.04	0.01\\
14.05	0.01\\
14.06	0.01\\
14.07	0.01\\
14.08	0.01\\
14.09	0.01\\
14.1	0.01\\
14.11	0.01\\
14.12	0.01\\
14.13	0.01\\
14.14	0.01\\
14.15	0.01\\
14.16	0.01\\
14.17	0.01\\
14.18	0.01\\
14.19	0.01\\
14.2	0.01\\
14.21	0.01\\
14.22	0.01\\
14.23	0.01\\
14.24	0.01\\
14.25	0.01\\
14.26	0.01\\
14.27	0.01\\
14.28	0.01\\
14.29	0.01\\
14.3	0.01\\
14.31	0.01\\
14.32	0.01\\
14.33	0.01\\
14.34	0.01\\
14.35	0.01\\
14.36	0.01\\
14.37	0.01\\
14.38	0.01\\
14.39	0.01\\
14.4	0.01\\
14.41	0.01\\
14.42	0.01\\
14.43	0.01\\
14.44	0.01\\
14.45	0.01\\
14.46	0.01\\
14.47	0.01\\
14.48	0.01\\
14.49	0.01\\
14.5	0.01\\
14.51	0.01\\
14.52	0.01\\
14.53	0.01\\
14.54	0.01\\
14.55	0.01\\
14.56	0.01\\
14.57	0.01\\
14.58	0.01\\
14.59	0.01\\
14.6	0.01\\
14.61	0.01\\
14.62	0.01\\
14.63	0.01\\
14.64	0.01\\
14.65	0.01\\
14.66	0.01\\
14.67	0.01\\
14.68	0.01\\
14.69	0.01\\
14.7	0.01\\
14.71	0.01\\
14.72	0.01\\
14.73	0.01\\
14.74	0.01\\
14.75	0.01\\
14.76	0.01\\
14.77	0.01\\
14.78	0.01\\
14.79	0.01\\
14.8	0.01\\
14.81	0.01\\
14.82	0.01\\
14.83	0.01\\
14.84	0.01\\
14.85	0.01\\
14.86	0.01\\
14.87	0.01\\
14.88	0.01\\
14.89	0.01\\
14.9	0.01\\
14.91	0.01\\
14.92	0.01\\
14.93	0.01\\
14.94	0.01\\
14.95	0.01\\
14.96	0.01\\
14.97	0.01\\
14.98	0.01\\
14.99	0.01\\
15	0.01\\
15.01	0.01\\
15.02	0.01\\
15.03	0.01\\
15.04	0.01\\
15.05	0.01\\
15.06	0.01\\
15.07	0.01\\
15.08	0.01\\
15.09	0.01\\
15.1	0.01\\
15.11	0.01\\
15.12	0.01\\
15.13	0.01\\
15.14	0.01\\
15.15	0.01\\
15.16	0.01\\
15.17	0.01\\
15.18	0.01\\
15.19	0.01\\
15.2	0.01\\
15.21	0.01\\
15.22	0.01\\
15.23	0.01\\
15.24	0.01\\
15.25	0.01\\
15.26	0.01\\
15.27	0.01\\
15.28	0.01\\
15.29	0.01\\
15.3	0.01\\
15.31	0.01\\
15.32	0.01\\
15.33	0.01\\
15.34	0.01\\
15.35	0.01\\
15.36	0.01\\
15.37	0.01\\
15.38	0.01\\
15.39	0.01\\
15.4	0.01\\
15.41	0.01\\
15.42	0.01\\
15.43	0.01\\
15.44	0.01\\
15.45	0.01\\
15.46	0.01\\
15.47	0.01\\
15.48	0.01\\
15.49	0.01\\
15.5	0.01\\
15.51	0.01\\
15.52	0.01\\
15.53	0.01\\
15.54	0.01\\
15.55	0.01\\
15.56	0.01\\
15.57	0.01\\
15.58	0.01\\
15.59	0.01\\
15.6	0.01\\
15.61	0.01\\
15.62	0.01\\
15.63	0.01\\
15.64	0.01\\
15.65	0.01\\
15.66	0.01\\
15.67	0.01\\
15.68	0.01\\
15.69	0.01\\
15.7	0.01\\
15.71	0.01\\
15.72	0.01\\
15.73	0.01\\
15.74	0.01\\
15.75	0.01\\
15.76	0.01\\
15.77	0.01\\
15.78	0.01\\
15.79	0.01\\
15.8	0.01\\
15.81	0.01\\
15.82	0.01\\
15.83	0.01\\
15.84	0.01\\
15.85	0.01\\
15.86	0.01\\
15.87	0.01\\
15.88	0.01\\
15.89	0.01\\
15.9	0.01\\
15.91	0.01\\
15.92	0.01\\
15.93	0.01\\
15.94	0.01\\
15.95	0.01\\
15.96	0.01\\
15.97	0.01\\
15.98	0.01\\
15.99	0.01\\
16	0.01\\
16.01	0.01\\
16.02	0.01\\
16.03	0.01\\
16.04	0.01\\
16.05	0.01\\
16.06	0.01\\
16.07	0.01\\
16.08	0.01\\
16.09	0.01\\
16.1	0.01\\
16.11	0.01\\
16.12	0.01\\
16.13	0.01\\
16.14	0.01\\
16.15	0.01\\
16.16	0.01\\
16.17	0.01\\
16.18	0.01\\
16.19	0.01\\
16.2	0.01\\
16.21	0.01\\
16.22	0.01\\
16.23	0.01\\
16.24	0.01\\
16.25	0.01\\
16.26	0.01\\
16.27	0.01\\
16.28	0.01\\
16.29	0.01\\
16.3	0.01\\
16.31	0.01\\
16.32	0.01\\
16.33	0.01\\
16.34	0.01\\
16.35	0.01\\
16.36	0.01\\
16.37	0.01\\
16.38	0.01\\
16.39	0.01\\
16.4	0.01\\
16.41	0.01\\
16.42	0.01\\
16.43	0.01\\
16.44	0.01\\
16.45	0.01\\
16.46	0.01\\
16.47	0.01\\
16.48	0.01\\
16.49	0.01\\
16.5	0.01\\
16.51	0.01\\
16.52	0.01\\
16.53	0.01\\
16.54	0.01\\
16.55	0.01\\
16.56	0.01\\
16.57	0.01\\
16.58	0.01\\
16.59	0.01\\
16.6	0.01\\
16.61	0.01\\
16.62	0.01\\
16.63	0.01\\
16.64	0.01\\
16.65	0.01\\
16.66	0.01\\
16.67	0.01\\
16.68	0.01\\
16.69	0.01\\
16.7	0.01\\
16.71	0.01\\
16.72	0.01\\
16.73	0.01\\
16.74	0.01\\
16.75	0.01\\
16.76	0.01\\
16.77	0.01\\
16.78	0.01\\
16.79	0.01\\
16.8	0.01\\
16.81	0.01\\
16.82	0.01\\
16.83	0.01\\
16.84	0.01\\
16.85	0.01\\
16.86	0.01\\
16.87	0.01\\
16.88	0.01\\
16.89	0.01\\
16.9	0.01\\
16.91	0.01\\
16.92	0.01\\
16.93	0.01\\
16.94	0.01\\
16.95	0.01\\
16.96	0.01\\
16.97	0.01\\
16.98	0.01\\
16.99	0.01\\
17	0.01\\
17.01	0.01\\
17.02	0.01\\
17.03	0.01\\
17.04	0.01\\
17.05	0.01\\
17.06	0.01\\
17.07	0.01\\
17.08	0.01\\
17.09	0.01\\
17.1	0.01\\
17.11	0.01\\
17.12	0.01\\
17.13	0.01\\
17.14	0.01\\
17.15	0.01\\
17.16	0.01\\
17.17	0.01\\
17.18	0.01\\
17.19	0.01\\
17.2	0.01\\
17.21	0.01\\
17.22	0.01\\
17.23	0.01\\
17.24	0.01\\
17.25	0.01\\
17.26	0.01\\
17.27	0.01\\
17.28	0.01\\
17.29	0.01\\
17.3	0.01\\
17.31	0.01\\
17.32	0.01\\
17.33	0.01\\
17.34	0.01\\
17.35	0.01\\
17.36	0.01\\
17.37	0.01\\
17.38	0.01\\
17.39	0.01\\
17.4	0.01\\
17.41	0.01\\
17.42	0.01\\
17.43	0.01\\
17.44	0.01\\
17.45	0.01\\
17.46	0.01\\
17.47	0.01\\
17.48	0.01\\
17.49	0.01\\
17.5	0.01\\
17.51	0.01\\
17.52	0.01\\
17.53	0.01\\
17.54	0.01\\
17.55	0.01\\
17.56	0.01\\
17.57	0.01\\
17.58	0.01\\
17.59	0.01\\
17.6	0.01\\
17.61	0.01\\
17.62	0.01\\
17.63	0.01\\
17.64	0.01\\
17.65	0.01\\
17.66	0.01\\
17.67	0.01\\
17.68	0.01\\
17.69	0.01\\
17.7	0.01\\
17.71	0.01\\
17.72	0.01\\
17.73	0.01\\
17.74	0.01\\
17.75	0.01\\
17.76	0.01\\
17.77	0.01\\
17.78	0.01\\
17.79	0.01\\
17.8	0.01\\
17.81	0.01\\
17.82	0.01\\
17.83	0.01\\
17.84	0.01\\
17.85	0.01\\
17.86	0.01\\
17.87	0.01\\
17.88	0.01\\
17.89	0.01\\
17.9	0.01\\
17.91	0.01\\
17.92	0.01\\
17.93	0.01\\
17.94	0.01\\
17.95	0.01\\
17.96	0.01\\
17.97	0.01\\
17.98	0.01\\
17.99	0.01\\
18	0.01\\
18.01	0.01\\
18.02	0.01\\
18.03	0.01\\
18.04	0.01\\
18.05	0.01\\
18.06	0.01\\
18.07	0.01\\
18.08	0.01\\
18.09	0.01\\
18.1	0.01\\
18.11	0.01\\
18.12	0.01\\
18.13	0.01\\
18.14	0.01\\
18.15	0.01\\
18.16	0.01\\
18.17	0.01\\
18.18	0.01\\
18.19	0.01\\
18.2	0.01\\
18.21	0.01\\
18.22	0.01\\
18.23	0.01\\
18.24	0.01\\
18.25	0.01\\
18.26	0.01\\
18.27	0.01\\
18.28	0.01\\
18.29	0.01\\
18.3	0.01\\
18.31	0.01\\
18.32	0.01\\
18.33	0.01\\
18.34	0.01\\
18.35	0.01\\
18.36	0.01\\
18.37	0.01\\
18.38	0.01\\
18.39	0.01\\
18.4	0.01\\
18.41	0.01\\
18.42	0.01\\
18.43	0.01\\
18.44	0.01\\
18.45	0.01\\
18.46	0.01\\
18.47	0.01\\
18.48	0.01\\
18.49	0.01\\
18.5	0.01\\
18.51	0.01\\
18.52	0.01\\
18.53	0.01\\
18.54	0.01\\
18.55	0.01\\
18.56	0.01\\
18.57	0.01\\
18.58	0.01\\
18.59	0.01\\
18.6	0.01\\
18.61	0.01\\
18.62	0.01\\
18.63	0.01\\
18.64	0.01\\
18.65	0.01\\
18.66	0.01\\
18.67	0.01\\
18.68	0.01\\
18.69	0.01\\
18.7	0.01\\
18.71	0.01\\
18.72	0.01\\
18.73	0.01\\
18.74	0.01\\
18.75	0.01\\
18.76	0.01\\
18.77	0.01\\
18.78	0.01\\
18.79	0.01\\
18.8	0.01\\
18.81	0.01\\
18.82	0.01\\
18.83	0.01\\
18.84	0.01\\
18.85	0.01\\
18.86	0.01\\
18.87	0.01\\
18.88	0.01\\
18.89	0.01\\
18.9	0.01\\
18.91	0.01\\
18.92	0.01\\
18.93	0.01\\
18.94	0.01\\
18.95	0.01\\
18.96	0.01\\
18.97	0.01\\
18.98	0.01\\
18.99	0.01\\
19	0.01\\
19.01	0.01\\
19.02	0.01\\
19.03	0.01\\
19.04	0.01\\
19.05	0.01\\
19.06	0.01\\
19.07	0.01\\
19.08	0.01\\
19.09	0.01\\
19.1	0.01\\
19.11	0.01\\
19.12	0.01\\
19.13	0.01\\
19.14	0.01\\
19.15	0.01\\
19.16	0.01\\
19.17	0.01\\
19.18	0.01\\
19.19	0.01\\
19.2	0.01\\
19.21	0.01\\
19.22	0.01\\
19.23	0.01\\
19.24	0.01\\
19.25	0.01\\
19.26	0.01\\
19.27	0.01\\
19.28	0.01\\
19.29	0.01\\
19.3	0.01\\
19.31	0.01\\
19.32	0.01\\
19.33	0.01\\
19.34	0.01\\
19.35	0.01\\
19.36	0.01\\
19.37	0.01\\
19.38	0.01\\
19.39	0.01\\
19.4	0.01\\
19.41	0.01\\
19.42	0.01\\
19.43	0.01\\
19.44	0.01\\
19.45	0.01\\
19.46	0.01\\
19.47	0.01\\
19.48	0.01\\
19.49	0.01\\
19.5	0.01\\
19.51	0.01\\
19.52	0.01\\
19.53	0.01\\
19.54	0.01\\
19.55	0.01\\
19.56	0.01\\
19.57	0.01\\
19.58	0.01\\
19.59	0.01\\
19.6	0.01\\
19.61	0.01\\
19.62	0.01\\
19.63	0.01\\
19.64	0.01\\
19.65	0.01\\
19.66	0.01\\
19.67	0.01\\
19.68	0.01\\
19.69	0.01\\
19.7	0.01\\
19.71	0.01\\
19.72	0.01\\
19.73	0.01\\
19.74	0.01\\
19.75	0.01\\
19.76	0.01\\
19.77	0.01\\
19.78	0.01\\
19.79	0.01\\
19.8	0.01\\
19.81	0.01\\
19.82	0.01\\
19.83	0.01\\
19.84	0.01\\
19.85	0.01\\
19.86	0.01\\
19.87	0.01\\
19.88	0.01\\
19.89	0.01\\
19.9	0.01\\
19.91	0.01\\
19.92	0.01\\
19.93	0.01\\
19.94	0.01\\
19.95	0.01\\
19.96	0.01\\
19.97	0.01\\
19.98	0.01\\
19.99	0.01\\
20	0.01\\
20.01	0.01\\
20.02	0.01\\
20.03	0.01\\
20.04	0.01\\
20.05	0.01\\
20.06	0.01\\
20.07	0.01\\
20.08	0.01\\
20.09	0.01\\
20.1	0.01\\
20.11	0.01\\
20.12	0.01\\
20.13	0.01\\
20.14	0.01\\
20.15	0.01\\
20.16	0.01\\
20.17	0.01\\
20.18	0.01\\
20.19	0.01\\
20.2	0.01\\
20.21	0.01\\
20.22	0.01\\
20.23	0.01\\
20.24	0.01\\
20.25	0.01\\
20.26	0.01\\
20.27	0.01\\
20.28	0.01\\
20.29	0.01\\
20.3	0.01\\
20.31	0.01\\
20.32	0.01\\
20.33	0.01\\
20.34	0.01\\
20.35	0.01\\
20.36	0.01\\
20.37	0.01\\
20.38	0.01\\
20.39	0.01\\
20.4	0.01\\
20.41	0.01\\
20.42	0.01\\
20.43	0.01\\
20.44	0.01\\
20.45	0.01\\
20.46	0.01\\
20.47	0.01\\
20.48	0.01\\
20.49	0.01\\
20.5	0.01\\
20.51	0.01\\
20.52	0.01\\
20.53	0.01\\
20.54	0.01\\
20.55	0.01\\
20.56	0.01\\
20.57	0.01\\
20.58	0.01\\
20.59	0.01\\
20.6	0.01\\
20.61	0.01\\
20.62	0.01\\
20.63	0.01\\
20.64	0.01\\
20.65	0.01\\
20.66	0.01\\
20.67	0.01\\
20.68	0.01\\
20.69	0.01\\
20.7	0.01\\
20.71	0.01\\
20.72	0.01\\
20.73	0.01\\
20.74	0.01\\
20.75	0.01\\
20.76	0.01\\
20.77	0.01\\
20.78	0.01\\
20.79	0.01\\
20.8	0.01\\
20.81	0.01\\
20.82	0.01\\
20.83	0.01\\
20.84	0.01\\
20.85	0.01\\
20.86	0.01\\
20.87	0.01\\
20.88	0.01\\
20.89	0.01\\
20.9	0.01\\
20.91	0.01\\
20.92	0.01\\
20.93	0.01\\
20.94	0.01\\
20.95	0.01\\
20.96	0.01\\
20.97	0.01\\
20.98	0.01\\
20.99	0.01\\
21	0.01\\
21.01	0.01\\
21.02	0.01\\
21.03	0.01\\
21.04	0.01\\
21.05	0.01\\
21.06	0.01\\
21.07	0.01\\
21.08	0.01\\
21.09	0.01\\
21.1	0.01\\
21.11	0.01\\
21.12	0.01\\
21.13	0.01\\
21.14	0.01\\
21.15	0.01\\
21.16	0.01\\
21.17	0.01\\
21.18	0.01\\
21.19	0.01\\
21.2	0.01\\
21.21	0.01\\
21.22	0.01\\
21.23	0.01\\
21.24	0.01\\
21.25	0.01\\
21.26	0.01\\
21.27	0.01\\
21.28	0.01\\
21.29	0.01\\
21.3	0.01\\
21.31	0.01\\
21.32	0.01\\
21.33	0.01\\
21.34	0.01\\
21.35	0.01\\
21.36	0.01\\
21.37	0.01\\
21.38	0.01\\
21.39	0.01\\
21.4	0.01\\
21.41	0.01\\
21.42	0.01\\
21.43	0.01\\
21.44	0.01\\
21.45	0.01\\
21.46	0.01\\
21.47	0.01\\
21.48	0.01\\
21.49	0.01\\
21.5	0.01\\
21.51	0.01\\
21.52	0.01\\
21.53	0.01\\
21.54	0.01\\
21.55	0.01\\
21.56	0.01\\
21.57	0.01\\
21.58	0.01\\
21.59	0.01\\
21.6	0.01\\
21.61	0.01\\
21.62	0.01\\
21.63	0.01\\
21.64	0.01\\
21.65	0.01\\
21.66	0.01\\
21.67	0.01\\
21.68	0.01\\
21.69	0.01\\
21.7	0.01\\
21.71	0.01\\
21.72	0.01\\
21.73	0.01\\
21.74	0.01\\
21.75	0.01\\
21.76	0.01\\
21.77	0.01\\
21.78	0.01\\
21.79	0.01\\
21.8	0.01\\
21.81	0.01\\
21.82	0.01\\
21.83	0.01\\
21.84	0.01\\
21.85	0.01\\
21.86	0.01\\
21.87	0.01\\
21.88	0.01\\
21.89	0.01\\
21.9	0.01\\
21.91	0.01\\
21.92	0.01\\
21.93	0.01\\
21.94	0.01\\
21.95	0.01\\
21.96	0.01\\
21.97	0.01\\
21.98	0.01\\
21.99	0.01\\
22	0.01\\
22.01	0.01\\
22.02	0.01\\
22.03	0.01\\
22.04	0.01\\
22.05	0.01\\
22.06	0.01\\
22.07	0.01\\
22.08	0.01\\
22.09	0.01\\
22.1	0.01\\
22.11	0.01\\
22.12	0.01\\
22.13	0.01\\
22.14	0.01\\
22.15	0.01\\
22.16	0.01\\
22.17	0.01\\
22.18	0.01\\
22.19	0.01\\
22.2	0.01\\
22.21	0.01\\
22.22	0.01\\
22.23	0.01\\
22.24	0.01\\
22.25	0.01\\
22.26	0.01\\
22.27	0.01\\
22.28	0.01\\
22.29	0.01\\
22.3	0.01\\
22.31	0.01\\
22.32	0.01\\
22.33	0.01\\
22.34	0.01\\
22.35	0.01\\
22.36	0.01\\
22.37	0.01\\
22.38	0.01\\
22.39	0.01\\
22.4	0.01\\
22.41	0.01\\
22.42	0.01\\
22.43	0.01\\
22.44	0.01\\
22.45	0.01\\
22.46	0.01\\
22.47	0.01\\
22.48	0.01\\
22.49	0.01\\
22.5	0.01\\
22.51	0.01\\
22.52	0.01\\
22.53	0.01\\
22.54	0.01\\
22.55	0.01\\
22.56	0.01\\
22.57	0.01\\
22.58	0.01\\
22.59	0.01\\
22.6	0.01\\
22.61	0.01\\
22.62	0.01\\
22.63	0.01\\
22.64	0.01\\
22.65	0.01\\
22.66	0.01\\
22.67	0.01\\
22.68	0.01\\
22.69	0.01\\
22.7	0.01\\
22.71	0.01\\
22.72	0.01\\
22.73	0.01\\
22.74	0.01\\
22.75	0.01\\
22.76	0.01\\
22.77	0.01\\
22.78	0.01\\
22.79	0.01\\
22.8	0.01\\
22.81	0.01\\
22.82	0.01\\
22.83	0.01\\
22.84	0.01\\
22.85	0.01\\
22.86	0.01\\
22.87	0.01\\
22.88	0.01\\
22.89	0.01\\
22.9	0.01\\
22.91	0.01\\
22.92	0.01\\
22.93	0.01\\
22.94	0.01\\
22.95	0.01\\
22.96	0.01\\
22.97	0.01\\
22.98	0.01\\
22.99	0.01\\
23	0.01\\
23.01	0.01\\
23.02	0.01\\
23.03	0.01\\
23.04	0.01\\
23.05	0.01\\
23.06	0.01\\
23.07	0.01\\
23.08	0.01\\
23.09	0.01\\
23.1	0.01\\
23.11	0.01\\
23.12	0.01\\
23.13	0.01\\
23.14	0.01\\
23.15	0.01\\
23.16	0.01\\
23.17	0.01\\
23.18	0.01\\
23.19	0.01\\
23.2	0.01\\
23.21	0.01\\
23.22	0.01\\
23.23	0.01\\
23.24	0.01\\
23.25	0.01\\
23.26	0.01\\
23.27	0.01\\
23.28	0.01\\
23.29	0.01\\
23.3	0.01\\
23.31	0.01\\
23.32	0.01\\
23.33	0.01\\
23.34	0.01\\
23.35	0.01\\
23.36	0.01\\
23.37	0.01\\
23.38	0.01\\
23.39	0.01\\
23.4	0.01\\
23.41	0.01\\
23.42	0.01\\
23.43	0.01\\
23.44	0.01\\
23.45	0.01\\
23.46	0.01\\
23.47	0.01\\
23.48	0.01\\
23.49	0.01\\
23.5	0.01\\
23.51	0.01\\
23.52	0.01\\
23.53	0.01\\
23.54	0.01\\
23.55	0.01\\
23.56	0.01\\
23.57	0.01\\
23.58	0.01\\
23.59	0.01\\
23.6	0.01\\
23.61	0.01\\
23.62	0.01\\
23.63	0.01\\
23.64	0.01\\
23.65	0.01\\
23.66	0.01\\
23.67	0.01\\
23.68	0.01\\
23.69	0.01\\
23.7	0.01\\
23.71	0.01\\
23.72	0.01\\
23.73	0.01\\
23.74	0.01\\
23.75	0.01\\
23.76	0.01\\
23.77	0.01\\
23.78	0.01\\
23.79	0.01\\
23.8	0.01\\
23.81	0.01\\
23.82	0.01\\
23.83	0.01\\
23.84	0.01\\
23.85	0.01\\
23.86	0.01\\
23.87	0.01\\
23.88	0.01\\
23.89	0.01\\
23.9	0.01\\
23.91	0.01\\
23.92	0.01\\
23.93	0.01\\
23.94	0.01\\
23.95	0.01\\
23.96	0.01\\
23.97	0.01\\
23.98	0.01\\
23.99	0.01\\
24	0.01\\
24.01	0.01\\
24.02	0.01\\
24.03	0.01\\
24.04	0.01\\
24.05	0.01\\
24.06	0.01\\
24.07	0.01\\
24.08	0.01\\
24.09	0.01\\
24.1	0.01\\
24.11	0.01\\
24.12	0.01\\
24.13	0.01\\
24.14	0.01\\
24.15	0.01\\
24.16	0.01\\
24.17	0.01\\
24.18	0.01\\
24.19	0.01\\
24.2	0.01\\
24.21	0.01\\
24.22	0.01\\
24.23	0.01\\
24.24	0.01\\
24.25	0.01\\
24.26	0.01\\
24.27	0.01\\
24.28	0.01\\
24.29	0.01\\
24.3	0.01\\
24.31	0.01\\
24.32	0.01\\
24.33	0.01\\
24.34	0.01\\
24.35	0.01\\
24.36	0.01\\
24.37	0.01\\
24.38	0.01\\
24.39	0.01\\
24.4	0.01\\
24.41	0.01\\
24.42	0.01\\
24.43	0.01\\
24.44	0.01\\
24.45	0.01\\
24.46	0.01\\
24.47	0.01\\
24.48	0.01\\
24.49	0.01\\
24.5	0.01\\
24.51	0.01\\
24.52	0.01\\
24.53	0.01\\
24.54	0.01\\
24.55	0.01\\
24.56	0.01\\
24.57	0.01\\
24.58	0.01\\
24.59	0.01\\
24.6	0.01\\
24.61	0.01\\
24.62	0.01\\
24.63	0.01\\
24.64	0.01\\
24.65	0.01\\
24.66	0.01\\
24.67	0.01\\
24.68	0.01\\
24.69	0.01\\
24.7	0.01\\
24.71	0.01\\
24.72	0.01\\
24.73	0.01\\
24.74	0.01\\
24.75	0.01\\
24.76	0.01\\
24.77	0.01\\
24.78	0.01\\
24.79	0.01\\
24.8	0.01\\
24.81	0.01\\
24.82	0.01\\
24.83	0.01\\
24.84	0.01\\
24.85	0.01\\
24.86	0.01\\
24.87	0.01\\
24.88	0.01\\
24.89	0.01\\
24.9	0.01\\
24.91	0.01\\
24.92	0.01\\
24.93	0.01\\
24.94	0.01\\
24.95	0.01\\
24.96	0.01\\
24.97	0.01\\
24.98	0.01\\
24.99	0.01\\
25	0.01\\
25.01	0.01\\
25.02	0.01\\
25.03	0.01\\
25.04	0.01\\
25.05	0.01\\
25.06	0.01\\
25.07	0.01\\
25.08	0.01\\
25.09	0.01\\
25.1	0.01\\
25.11	0.01\\
25.12	0.01\\
25.13	0.01\\
25.14	0.01\\
25.15	0.01\\
25.16	0.01\\
25.17	0.01\\
25.18	0.01\\
25.19	0.01\\
25.2	0.01\\
25.21	0.01\\
25.22	0.01\\
25.23	0.01\\
25.24	0.01\\
25.25	0.01\\
25.26	0.01\\
25.27	0.01\\
25.28	0.01\\
25.29	0.01\\
25.3	0.01\\
25.31	0.01\\
25.32	0.01\\
25.33	0.01\\
25.34	0.01\\
25.35	0.01\\
25.36	0.01\\
25.37	0.01\\
25.38	0.01\\
25.39	0.01\\
25.4	0.01\\
25.41	0.01\\
25.42	0.01\\
25.43	0.01\\
25.44	0.01\\
25.45	0.01\\
25.46	0.01\\
25.47	0.01\\
25.48	0.01\\
25.49	0.01\\
25.5	0.01\\
25.51	0.01\\
25.52	0.01\\
25.53	0.01\\
25.54	0.01\\
25.55	0.01\\
25.56	0.01\\
25.57	0.01\\
25.58	0.01\\
25.59	0.01\\
25.6	0.01\\
25.61	0.01\\
25.62	0.01\\
25.63	0.01\\
25.64	0.01\\
25.65	0.01\\
25.66	0.01\\
25.67	0.01\\
25.68	0.01\\
25.69	0.01\\
25.7	0.01\\
25.71	0.01\\
25.72	0.01\\
25.73	0.01\\
25.74	0.01\\
25.75	0.01\\
25.76	0.01\\
25.77	0.01\\
25.78	0.01\\
25.79	0.01\\
25.8	0.01\\
25.81	0.01\\
25.82	0.01\\
25.83	0.01\\
25.84	0.01\\
25.85	0.01\\
25.86	0.01\\
25.87	0.01\\
25.88	0.01\\
25.89	0.01\\
25.9	0.01\\
25.91	0.01\\
25.92	0.01\\
25.93	0.01\\
25.94	0.01\\
25.95	0.01\\
25.96	0.01\\
25.97	0.01\\
25.98	0.01\\
25.99	0.01\\
26	0.01\\
26.01	0.01\\
26.02	0.01\\
26.03	0.01\\
26.04	0.01\\
26.05	0.01\\
26.06	0.01\\
26.07	0.01\\
26.08	0.01\\
26.09	0.01\\
26.1	0.01\\
26.11	0.01\\
26.12	0.01\\
26.13	0.01\\
26.14	0.01\\
26.15	0.01\\
26.16	0.01\\
26.17	0.01\\
26.18	0.01\\
26.19	0.01\\
26.2	0.01\\
26.21	0.01\\
26.22	0.01\\
26.23	0.01\\
26.24	0.01\\
26.25	0.01\\
26.26	0.01\\
26.27	0.01\\
26.28	0.01\\
26.29	0.01\\
26.3	0.01\\
26.31	0.01\\
26.32	0.01\\
26.33	0.01\\
26.34	0.01\\
26.35	0.01\\
26.36	0.01\\
26.37	0.01\\
26.38	0.01\\
26.39	0.01\\
26.4	0.01\\
26.41	0.01\\
26.42	0.01\\
26.43	0.01\\
26.44	0.01\\
26.45	0.01\\
26.46	0.01\\
26.47	0.01\\
26.48	0.01\\
26.49	0.01\\
26.5	0.01\\
26.51	0.01\\
26.52	0.01\\
26.53	0.01\\
26.54	0.01\\
26.55	0.01\\
26.56	0.01\\
26.57	0.01\\
26.58	0.01\\
26.59	0.01\\
26.6	0.01\\
26.61	0.01\\
26.62	0.01\\
26.63	0.01\\
26.64	0.01\\
26.65	0.01\\
26.66	0.01\\
26.67	0.01\\
26.68	0.01\\
26.69	0.01\\
26.7	0.01\\
26.71	0.01\\
26.72	0.01\\
26.73	0.01\\
26.74	0.01\\
26.75	0.01\\
26.76	0.01\\
26.77	0.01\\
26.78	0.01\\
26.79	0.01\\
26.8	0.01\\
26.81	0.01\\
26.82	0.01\\
26.83	0.01\\
26.84	0.01\\
26.85	0.01\\
26.86	0.01\\
26.87	0.01\\
26.88	0.01\\
26.89	0.01\\
26.9	0.01\\
26.91	0.01\\
26.92	0.01\\
26.93	0.01\\
26.94	0.01\\
26.95	0.01\\
26.96	0.01\\
26.97	0.01\\
26.98	0.01\\
26.99	0.01\\
27	0.01\\
27.01	0.01\\
27.02	0.01\\
27.03	0.01\\
27.04	0.01\\
27.05	0.01\\
27.06	0.01\\
27.07	0.01\\
27.08	0.01\\
27.09	0.01\\
27.1	0.01\\
27.11	0.01\\
27.12	0.01\\
27.13	0.01\\
27.14	0.01\\
27.15	0.01\\
27.16	0.01\\
27.17	0.01\\
27.18	0.01\\
27.19	0.01\\
27.2	0.01\\
27.21	0.01\\
27.22	0.01\\
27.23	0.01\\
27.24	0.01\\
27.25	0.01\\
27.26	0.01\\
27.27	0.01\\
27.28	0.01\\
27.29	0.01\\
27.3	0.01\\
27.31	0.01\\
27.32	0.01\\
27.33	0.01\\
27.34	0.01\\
27.35	0.01\\
27.36	0.01\\
27.37	0.01\\
27.38	0.01\\
27.39	0.01\\
27.4	0.01\\
27.41	0.01\\
27.42	0.01\\
27.43	0.01\\
27.44	0.01\\
27.45	0.01\\
27.46	0.01\\
27.47	0.01\\
27.48	0.01\\
27.49	0.01\\
27.5	0.01\\
27.51	0.01\\
27.52	0.01\\
27.53	0.01\\
27.54	0.01\\
27.55	0.01\\
27.56	0.01\\
27.57	0.01\\
27.58	0.01\\
27.59	0.01\\
27.6	0.01\\
27.61	0.01\\
27.62	0.01\\
27.63	0.01\\
27.64	0.01\\
27.65	0.01\\
27.66	0.01\\
27.67	0.01\\
27.68	0.01\\
27.69	0.01\\
27.7	0.01\\
27.71	0.01\\
27.72	0.01\\
27.73	0.01\\
27.74	0.01\\
27.75	0.01\\
27.76	0.01\\
27.77	0.01\\
27.78	0.01\\
27.79	0.01\\
27.8	0.01\\
27.81	0.01\\
27.82	0.01\\
27.83	0.01\\
27.84	0.01\\
27.85	0.01\\
27.86	0.01\\
27.87	0.01\\
27.88	0.01\\
27.89	0.01\\
27.9	0.01\\
27.91	0.01\\
27.92	0.01\\
27.93	0.01\\
27.94	0.01\\
27.95	0.01\\
27.96	0.01\\
27.97	0.01\\
27.98	0.01\\
27.99	0.01\\
28	0.01\\
28.01	0.01\\
28.02	0.01\\
28.03	0.01\\
28.04	0.01\\
28.05	0.01\\
28.06	0.01\\
28.07	0.01\\
28.08	0.01\\
28.09	0.01\\
28.1	0.01\\
28.11	0.01\\
28.12	0.01\\
28.13	0.01\\
28.14	0.01\\
28.15	0.01\\
28.16	0.01\\
28.17	0.01\\
28.18	0.01\\
28.19	0.01\\
28.2	0.01\\
28.21	0.01\\
28.22	0.01\\
28.23	0.01\\
28.24	0.01\\
28.25	0.01\\
28.26	0.01\\
28.27	0.01\\
28.28	0.01\\
28.29	0.01\\
28.3	0.01\\
28.31	0.01\\
28.32	0.01\\
28.33	0.01\\
28.34	0.01\\
28.35	0.01\\
28.36	0.01\\
28.37	0.01\\
28.38	0.01\\
28.39	0.01\\
28.4	0.01\\
28.41	0.01\\
28.42	0.01\\
28.43	0.01\\
28.44	0.01\\
28.45	0.01\\
28.46	0.01\\
28.47	0.01\\
28.48	0.01\\
28.49	0.01\\
28.5	0.01\\
28.51	0.01\\
28.52	0.01\\
28.53	0.01\\
28.54	0.01\\
28.55	0.01\\
28.56	0.01\\
28.57	0.01\\
28.58	0.01\\
28.59	0.01\\
28.6	0.01\\
28.61	0.01\\
28.62	0.01\\
28.63	0.01\\
28.64	0.01\\
28.65	0.01\\
28.66	0.01\\
28.67	0.01\\
28.68	0.01\\
28.69	0.01\\
28.7	0.01\\
28.71	0.01\\
28.72	0.01\\
28.73	0.01\\
28.74	0.01\\
28.75	0.01\\
28.76	0.01\\
28.77	0.01\\
28.78	0.01\\
28.79	0.01\\
28.8	0.01\\
28.81	0.01\\
28.82	0.01\\
28.83	0.01\\
28.84	0.01\\
28.85	0.01\\
28.86	0.01\\
28.87	0.01\\
28.88	0.01\\
28.89	0.01\\
28.9	0.01\\
28.91	0.01\\
28.92	0.01\\
28.93	0.01\\
28.94	0.01\\
28.95	0.01\\
28.96	0.01\\
28.97	0.01\\
28.98	0.01\\
28.99	0.01\\
29	0.01\\
29.01	0.01\\
29.02	0.01\\
29.03	0.01\\
29.04	0.01\\
29.05	0.01\\
29.06	0.01\\
29.07	0.01\\
29.08	0.01\\
29.09	0.01\\
29.1	0.01\\
29.11	0.01\\
29.12	0.01\\
29.13	0.01\\
29.14	0.01\\
29.15	0.01\\
29.16	0.01\\
29.17	0.01\\
29.18	0.01\\
29.19	0.01\\
29.2	0.01\\
29.21	0.01\\
29.22	0.01\\
29.23	0.01\\
29.24	0.01\\
29.25	0.01\\
29.26	0.01\\
29.27	0.01\\
29.28	0.01\\
29.29	0.01\\
29.3	0.01\\
29.31	0.01\\
29.32	0.01\\
29.33	0.01\\
29.34	0.01\\
29.35	0.01\\
29.36	0.01\\
29.37	0.01\\
29.38	0.01\\
29.39	0.01\\
29.4	0.01\\
29.41	0.01\\
29.42	0.01\\
29.43	0.01\\
29.44	0.01\\
29.45	0.01\\
29.46	0.01\\
29.47	0.01\\
29.48	0.01\\
29.49	0.01\\
29.5	0.01\\
29.51	0.01\\
29.52	0.01\\
29.53	0.01\\
29.54	0.01\\
29.55	0.01\\
29.56	0.01\\
29.57	0.01\\
29.58	0.01\\
29.59	0.01\\
29.6	0.01\\
29.61	0.01\\
29.62	0.01\\
29.63	0.01\\
29.64	0.01\\
29.65	0.01\\
29.66	0.01\\
29.67	0.01\\
29.68	0.01\\
29.69	0.01\\
29.7	0.01\\
29.71	0.01\\
29.72	0.01\\
29.73	0.01\\
29.74	0.01\\
29.75	0.01\\
29.76	0.01\\
29.77	0.01\\
29.78	0.01\\
29.79	0.01\\
29.8	0.01\\
29.81	0.01\\
29.82	0.01\\
29.83	0.01\\
29.84	0.01\\
29.85	0.01\\
29.86	0.01\\
29.87	0.01\\
29.88	0.01\\
29.89	0.01\\
29.9	0.01\\
29.91	0.01\\
29.92	0.01\\
29.93	0.01\\
29.94	0.01\\
29.95	0.01\\
29.96	0.01\\
29.97	0.01\\
29.98	0.01\\
29.99	0.01\\
30	0.01\\
30.01	0.01\\
30.02	0.01\\
30.03	0.01\\
30.04	0.01\\
30.05	0.01\\
30.06	0.01\\
30.07	0.01\\
30.08	0.01\\
30.09	0.01\\
30.1	0.01\\
30.11	0.01\\
30.12	0.01\\
30.13	0.01\\
30.14	0.01\\
30.15	0.01\\
30.16	0.01\\
30.17	0.01\\
30.18	0.01\\
30.19	0.01\\
30.2	0.01\\
30.21	0.01\\
30.22	0.01\\
30.23	0.01\\
30.24	0.01\\
30.25	0.01\\
30.26	0.01\\
30.27	0.01\\
30.28	0.01\\
30.29	0.01\\
30.3	0.01\\
30.31	0.01\\
30.32	0.01\\
30.33	0.01\\
30.34	0.01\\
30.35	0.01\\
30.36	0.01\\
30.37	0.01\\
30.38	0.01\\
30.39	0.01\\
30.4	0.01\\
30.41	0.01\\
30.42	0.01\\
30.43	0.01\\
30.44	0.01\\
30.45	0.01\\
30.46	0.01\\
30.47	0.01\\
30.48	0.01\\
30.49	0.01\\
30.5	0.01\\
30.51	0.01\\
30.52	0.01\\
30.53	0.01\\
30.54	0.01\\
30.55	0.01\\
30.56	0.01\\
30.57	0.01\\
30.58	0.01\\
30.59	0.01\\
30.6	0.01\\
30.61	0.01\\
30.62	0.01\\
30.63	0.01\\
30.64	0.01\\
30.65	0.01\\
30.66	0.01\\
30.67	0.01\\
30.68	0.01\\
30.69	0.01\\
30.7	0.01\\
30.71	0.01\\
30.72	0.01\\
30.73	0.01\\
30.74	0.01\\
30.75	0.01\\
30.76	0.01\\
30.77	0.01\\
30.78	0.01\\
30.79	0.01\\
30.8	0.01\\
30.81	0.01\\
30.82	0.01\\
30.83	0.01\\
30.84	0.01\\
30.85	0.01\\
30.86	0.01\\
30.87	0.01\\
30.88	0.01\\
30.89	0.01\\
30.9	0.01\\
30.91	0.01\\
30.92	0.01\\
30.93	0.01\\
30.94	0.01\\
30.95	0.01\\
30.96	0.01\\
30.97	0.01\\
30.98	0.01\\
30.99	0.01\\
31	0.01\\
31.01	0.01\\
31.02	0.01\\
31.03	0.01\\
31.04	0.01\\
31.05	0.01\\
31.06	0.01\\
31.07	0.01\\
31.08	0.01\\
31.09	0.01\\
31.1	0.01\\
31.11	0.01\\
31.12	0.01\\
31.13	0.01\\
31.14	0.01\\
31.15	0.01\\
31.16	0.01\\
31.17	0.01\\
31.18	0.01\\
31.19	0.01\\
31.2	0.01\\
31.21	0.01\\
31.22	0.01\\
31.23	0.01\\
31.24	0.01\\
31.25	0.01\\
31.26	0.01\\
31.27	0.01\\
31.28	0.01\\
31.29	0.01\\
31.3	0.01\\
31.31	0.01\\
31.32	0.01\\
31.33	0.01\\
31.34	0.01\\
31.35	0.01\\
31.36	0.01\\
31.37	0.01\\
31.38	0.01\\
31.39	0.01\\
31.4	0.01\\
31.41	0.01\\
31.42	0.01\\
31.43	0.01\\
31.44	0.01\\
31.45	0.01\\
31.46	0.01\\
31.47	0.01\\
31.48	0.01\\
31.49	0.01\\
31.5	0.01\\
31.51	0.01\\
31.52	0.01\\
31.53	0.01\\
31.54	0.01\\
31.55	0.01\\
31.56	0.01\\
31.57	0.01\\
31.58	0.01\\
31.59	0.01\\
31.6	0.01\\
31.61	0.01\\
31.62	0.01\\
31.63	0.01\\
31.64	0.01\\
31.65	0.01\\
31.66	0.01\\
31.67	0.01\\
31.68	0.01\\
31.69	0.01\\
31.7	0.01\\
31.71	0.01\\
31.72	0.01\\
31.73	0.01\\
31.74	0.01\\
31.75	0.01\\
31.76	0.01\\
31.77	0.01\\
31.78	0.01\\
31.79	0.01\\
31.8	0.01\\
31.81	0.01\\
31.82	0.01\\
31.83	0.01\\
31.84	0.01\\
31.85	0.01\\
31.86	0.01\\
31.87	0.01\\
31.88	0.01\\
31.89	0.01\\
31.9	0.01\\
31.91	0.01\\
31.92	0.01\\
31.93	0.01\\
31.94	0.01\\
31.95	0.01\\
31.96	0.01\\
31.97	0.01\\
31.98	0.01\\
31.99	0.01\\
32	0.01\\
32.01	0.01\\
32.02	0.01\\
32.03	0.01\\
32.04	0.01\\
32.05	0.01\\
32.06	0.01\\
32.07	0.01\\
32.08	0.01\\
32.09	0.01\\
32.1	0.01\\
32.11	0.01\\
32.12	0.01\\
32.13	0.01\\
32.14	0.01\\
32.15	0.01\\
32.16	0.01\\
32.17	0.01\\
32.18	0.01\\
32.19	0.01\\
32.2	0.01\\
32.21	0.01\\
32.22	0.01\\
32.23	0.01\\
32.24	0.01\\
32.25	0.01\\
32.26	0.01\\
32.27	0.01\\
32.28	0.01\\
32.29	0.01\\
32.3	0.01\\
32.31	0.01\\
32.32	0.01\\
32.33	0.01\\
32.34	0.01\\
32.35	0.01\\
32.36	0.01\\
32.37	0.01\\
32.38	0.01\\
32.39	0.01\\
32.4	0.01\\
32.41	0.01\\
32.42	0.01\\
32.43	0.01\\
32.44	0.01\\
32.45	0.01\\
32.46	0.01\\
32.47	0.01\\
32.48	0.01\\
32.49	0.01\\
32.5	0.01\\
32.51	0.01\\
32.52	0.01\\
32.53	0.01\\
32.54	0.01\\
32.55	0.01\\
32.56	0.01\\
32.57	0.01\\
32.58	0.01\\
32.59	0.01\\
32.6	0.01\\
32.61	0.01\\
32.62	0.01\\
32.63	0.01\\
32.64	0.01\\
32.65	0.01\\
32.66	0.01\\
32.67	0.01\\
32.68	0.01\\
32.69	0.01\\
32.7	0.01\\
32.71	0.01\\
32.72	0.01\\
32.73	0.01\\
32.74	0.01\\
32.75	0.01\\
32.76	0.01\\
32.77	0.01\\
32.78	0.01\\
32.79	0.01\\
32.8	0.01\\
32.81	0.01\\
32.82	0.01\\
32.83	0.01\\
32.84	0.01\\
32.85	0.01\\
32.86	0.01\\
32.87	0.01\\
32.88	0.01\\
32.89	0.01\\
32.9	0.01\\
32.91	0.01\\
32.92	0.01\\
32.93	0.01\\
32.94	0.01\\
32.95	0.01\\
32.96	0.01\\
32.97	0.01\\
32.98	0.01\\
32.99	0.01\\
33	0.01\\
33.01	0.01\\
33.02	0.01\\
33.03	0.01\\
33.04	0.01\\
33.05	0.01\\
33.06	0.01\\
33.07	0.01\\
33.08	0.01\\
33.09	0.01\\
33.1	0.01\\
33.11	0.01\\
33.12	0.01\\
33.13	0.01\\
33.14	0.01\\
33.15	0.01\\
33.16	0.01\\
33.17	0.01\\
33.18	0.01\\
33.19	0.01\\
33.2	0.01\\
33.21	0.01\\
33.22	0.01\\
33.23	0.01\\
33.24	0.01\\
33.25	0.01\\
33.26	0.01\\
33.27	0.01\\
33.28	0.01\\
33.29	0.01\\
33.3	0.01\\
33.31	0.01\\
33.32	0.01\\
33.33	0.01\\
33.34	0.01\\
33.35	0.01\\
33.36	0.01\\
33.37	0.01\\
33.38	0.01\\
33.39	0.01\\
33.4	0.01\\
33.41	0.01\\
33.42	0.01\\
33.43	0.01\\
33.44	0.01\\
33.45	0.01\\
33.46	0.01\\
33.47	0.01\\
33.48	0.01\\
33.49	0.01\\
33.5	0.01\\
33.51	0.01\\
33.52	0.01\\
33.53	0.01\\
33.54	0.01\\
33.55	0.01\\
33.56	0.01\\
33.57	0.01\\
33.58	0.01\\
33.59	0.01\\
33.6	0.01\\
33.61	0.01\\
33.62	0.01\\
33.63	0.01\\
33.64	0.01\\
33.65	0.01\\
33.66	0.01\\
33.67	0.01\\
33.68	0.01\\
33.69	0.01\\
33.7	0.01\\
33.71	0.01\\
33.72	0.01\\
33.73	0.01\\
33.74	0.01\\
33.75	0.01\\
33.76	0.01\\
33.77	0.01\\
33.78	0.01\\
33.79	0.01\\
33.8	0.01\\
33.81	0.01\\
33.82	0.01\\
33.83	0.01\\
33.84	0.01\\
33.85	0.01\\
33.86	0.01\\
33.87	0.01\\
33.88	0.01\\
33.89	0.01\\
33.9	0.01\\
33.91	0.01\\
33.92	0.01\\
33.93	0.01\\
33.94	0.01\\
33.95	0.01\\
33.96	0.01\\
33.97	0.01\\
33.98	0.01\\
33.99	0.01\\
34	0.01\\
34.01	0.01\\
34.02	0.01\\
34.03	0.01\\
34.04	0.01\\
34.05	0.01\\
34.06	0.01\\
34.07	0.01\\
34.08	0.01\\
34.09	0.01\\
34.1	0.01\\
34.11	0.01\\
34.12	0.01\\
34.13	0.01\\
34.14	0.01\\
34.15	0.01\\
34.16	0.01\\
34.17	0.01\\
34.18	0.01\\
34.19	0.01\\
34.2	0.01\\
34.21	0.01\\
34.22	0.01\\
34.23	0.01\\
34.24	0.01\\
34.25	0.01\\
34.26	0.01\\
34.27	0.01\\
34.28	0.01\\
34.29	0.01\\
34.3	0.01\\
34.31	0.01\\
34.32	0.01\\
34.33	0.01\\
34.34	0.01\\
34.35	0.01\\
34.36	0.01\\
34.37	0.01\\
34.38	0.01\\
34.39	0.01\\
34.4	0.01\\
34.41	0.01\\
34.42	0.01\\
34.43	0.01\\
34.44	0.01\\
34.45	0.01\\
34.46	0.01\\
34.47	0.01\\
34.48	0.01\\
34.49	0.01\\
34.5	0.01\\
34.51	0.01\\
34.52	0.01\\
34.53	0.01\\
34.54	0.01\\
34.55	0.01\\
34.56	0.01\\
34.57	0.01\\
34.58	0.01\\
34.59	0.01\\
34.6	0.01\\
34.61	0.01\\
34.62	0.01\\
34.63	0.01\\
34.64	0.01\\
34.65	0.01\\
34.66	0.01\\
34.67	0.01\\
34.68	0.01\\
34.69	0.01\\
34.7	0.01\\
34.71	0.01\\
34.72	0.01\\
34.73	0.01\\
34.74	0.01\\
34.75	0.01\\
34.76	0.01\\
34.77	0.01\\
34.78	0.01\\
34.79	0.01\\
34.8	0.01\\
34.81	0.01\\
34.82	0.01\\
34.83	0.01\\
34.84	0.01\\
34.85	0.01\\
34.86	0.01\\
34.87	0.01\\
34.88	0.01\\
34.89	0.01\\
34.9	0.01\\
34.91	0.01\\
34.92	0.01\\
34.93	0.01\\
34.94	0.01\\
34.95	0.01\\
34.96	0.01\\
34.97	0.01\\
34.98	0.01\\
34.99	0.01\\
35	0.01\\
35.01	0.01\\
35.02	0.01\\
35.03	0.01\\
35.04	0.01\\
35.05	0.01\\
35.06	0.01\\
35.07	0.01\\
35.08	0.01\\
35.09	0.01\\
35.1	0.01\\
35.11	0.01\\
35.12	0.01\\
35.13	0.01\\
35.14	0.01\\
35.15	0.01\\
35.16	0.01\\
35.17	0.01\\
35.18	0.01\\
35.19	0.01\\
35.2	0.01\\
35.21	0.01\\
35.22	0.01\\
35.23	0.01\\
35.24	0.01\\
35.25	0.01\\
35.26	0.01\\
35.27	0.01\\
35.28	0.01\\
35.29	0.01\\
35.3	0.01\\
35.31	0.01\\
35.32	0.01\\
35.33	0.01\\
35.34	0.01\\
35.35	0.01\\
35.36	0.01\\
35.37	0.01\\
35.38	0.01\\
35.39	0.01\\
35.4	0.01\\
35.41	0.01\\
35.42	0.01\\
35.43	0.01\\
35.44	0.01\\
35.45	0.01\\
35.46	0.01\\
35.47	0.01\\
35.48	0.01\\
35.49	0.01\\
35.5	0.01\\
35.51	0.01\\
35.52	0.01\\
35.53	0.01\\
35.54	0.01\\
35.55	0.01\\
35.56	0.01\\
35.57	0.01\\
35.58	0.01\\
35.59	0.01\\
35.6	0.01\\
35.61	0.01\\
35.62	0.01\\
35.63	0.01\\
35.64	0.01\\
35.65	0.01\\
35.66	0.01\\
35.67	0.01\\
35.68	0.01\\
35.69	0.01\\
35.7	0.01\\
35.71	0.01\\
35.72	0.01\\
35.73	0.01\\
35.74	0.01\\
35.75	0.01\\
35.76	0.01\\
35.77	0.01\\
35.78	0.01\\
35.79	0.01\\
35.8	0.01\\
35.81	0.01\\
35.82	0.01\\
35.83	0.01\\
35.84	0.01\\
35.85	0.01\\
35.86	0.01\\
35.87	0.01\\
35.88	0.01\\
35.89	0.01\\
35.9	0.01\\
35.91	0.01\\
35.92	0.01\\
35.93	0.01\\
35.94	0.01\\
35.95	0.01\\
35.96	0.01\\
35.97	0.01\\
35.98	0.01\\
35.99	0.01\\
36	0.01\\
36.01	0.01\\
36.02	0.01\\
36.03	0.01\\
36.04	0.01\\
36.05	0.01\\
36.06	0.01\\
36.07	0.01\\
36.08	0.01\\
36.09	0.01\\
36.1	0.01\\
36.11	0.01\\
36.12	0.01\\
36.13	0.01\\
36.14	0.01\\
36.15	0.01\\
36.16	0.01\\
36.17	0.01\\
36.18	0.01\\
36.19	0.01\\
36.2	0.01\\
36.21	0.01\\
36.22	0.01\\
36.23	0.01\\
36.24	0.01\\
36.25	0.01\\
36.26	0.01\\
36.27	0.01\\
36.28	0.01\\
36.29	0.01\\
36.3	0.01\\
36.31	0.01\\
36.32	0.01\\
36.33	0.01\\
36.34	0.01\\
36.35	0.01\\
36.36	0.01\\
36.37	0.01\\
36.38	0.01\\
36.39	0.01\\
36.4	0.01\\
36.41	0.01\\
36.42	0.01\\
36.43	0.01\\
36.44	0.01\\
36.45	0.01\\
36.46	0.01\\
36.47	0.01\\
36.48	0.01\\
36.49	0.01\\
36.5	0.01\\
36.51	0.01\\
36.52	0.01\\
36.53	0.01\\
36.54	0.01\\
36.55	0.01\\
36.56	0.01\\
36.57	0.01\\
36.58	0.01\\
36.59	0.01\\
36.6	0.01\\
36.61	0.01\\
36.62	0.01\\
36.63	0.01\\
36.64	0.01\\
36.65	0.01\\
36.66	0.01\\
36.67	0.01\\
36.68	0.01\\
36.69	0.01\\
36.7	0.01\\
36.71	0.01\\
36.72	0.01\\
36.73	0.01\\
36.74	0.01\\
36.75	0.01\\
36.76	0.01\\
36.77	0.01\\
36.78	0.01\\
36.79	0.01\\
36.8	0.01\\
36.81	0.01\\
36.82	0.01\\
36.83	0.01\\
36.84	0.01\\
36.85	0.01\\
36.86	0.01\\
36.87	0.01\\
36.88	0.01\\
36.89	0.01\\
36.9	0.01\\
36.91	0.01\\
36.92	0.01\\
36.93	0.01\\
36.94	0.01\\
36.95	0.01\\
36.96	0.01\\
36.97	0.01\\
36.98	0.01\\
36.99	0.01\\
37	0.01\\
37.01	0.01\\
37.02	0.01\\
37.03	0.01\\
37.04	0.01\\
37.05	0.01\\
37.06	0.01\\
37.07	0.01\\
37.08	0.01\\
37.09	0.01\\
37.1	0.01\\
37.11	0.01\\
37.12	0.01\\
37.13	0.01\\
37.14	0.01\\
37.15	0.01\\
37.16	0.01\\
37.17	0.01\\
37.18	0.01\\
37.19	0.01\\
37.2	0.01\\
37.21	0.01\\
37.22	0.01\\
37.23	0.01\\
37.24	0.01\\
37.25	0.01\\
37.26	0.01\\
37.27	0.01\\
37.28	0.01\\
37.29	0.01\\
37.3	0.01\\
37.31	0.01\\
37.32	0.01\\
37.33	0.01\\
37.34	0.01\\
37.35	0.01\\
37.36	0.01\\
37.37	0.01\\
37.38	0.01\\
37.39	0.01\\
37.4	0.01\\
37.41	0.01\\
37.42	0.01\\
37.43	0.01\\
37.44	0.01\\
37.45	0.01\\
37.46	0.01\\
37.47	0.01\\
37.48	0.01\\
37.49	0.01\\
37.5	0.01\\
37.51	0.01\\
37.52	0.01\\
37.53	0.01\\
37.54	0.01\\
37.55	0.01\\
37.56	0.01\\
37.57	0.01\\
37.58	0.01\\
37.59	0.01\\
37.6	0.01\\
37.61	0.01\\
37.62	0.01\\
37.63	0.01\\
37.64	0.01\\
37.65	0.01\\
37.66	0.01\\
37.67	0.01\\
37.68	0.01\\
37.69	0.01\\
37.7	0.01\\
37.71	0.01\\
37.72	0.01\\
37.73	0.01\\
37.74	0.01\\
37.75	0.01\\
37.76	0.01\\
37.77	0.01\\
37.78	0.01\\
37.79	0.01\\
37.8	0.01\\
37.81	0.01\\
37.82	0.01\\
37.83	0.01\\
37.84	0.01\\
37.85	0.01\\
37.86	0.01\\
37.87	0.01\\
37.88	0.01\\
37.89	0.01\\
37.9	0.01\\
37.91	0.01\\
37.92	0.01\\
37.93	0.01\\
37.94	0.01\\
37.95	0.01\\
37.96	0.01\\
37.97	0.01\\
37.98	0.01\\
37.99	0.01\\
38	0.01\\
38.01	0.01\\
38.02	0.01\\
38.03	0.01\\
38.04	0.01\\
38.05	0.01\\
38.06	0.01\\
38.07	0.01\\
38.08	0.01\\
38.09	0.01\\
38.1	0.01\\
38.11	0.01\\
38.12	0.01\\
38.13	0.01\\
38.14	0.01\\
38.15	0.01\\
38.16	0.01\\
38.17	0.01\\
38.18	0.01\\
38.19	0.01\\
38.2	0.01\\
38.21	0.01\\
38.22	0.01\\
38.23	0.01\\
38.24	0.01\\
38.25	0.01\\
38.26	0.01\\
38.27	0.01\\
38.28	0.01\\
38.29	0.01\\
38.3	0.01\\
38.31	0.01\\
38.32	0.01\\
38.33	0.01\\
38.34	0.01\\
38.35	0.01\\
38.36	0.01\\
38.37	0.01\\
38.38	0.01\\
38.39	0.01\\
38.4	0.01\\
38.41	0.01\\
38.42	0.01\\
38.43	0.01\\
38.44	0.01\\
38.45	0.01\\
38.46	0.01\\
38.47	0.01\\
38.48	0.01\\
38.49	0.01\\
38.5	0.01\\
38.51	0.01\\
38.52	0.01\\
38.53	0.01\\
38.54	0.01\\
38.55	0.01\\
38.56	0.01\\
38.57	0.01\\
38.58	0.01\\
38.59	0.01\\
38.6	0.01\\
38.61	0.01\\
38.62	0.01\\
38.63	0.01\\
38.64	0.01\\
38.65	0.01\\
38.66	0.01\\
38.67	0.01\\
38.68	0.01\\
38.69	0.01\\
38.7	0.01\\
38.71	0.01\\
38.72	0.01\\
38.73	0.01\\
38.74	0.01\\
38.75	0.01\\
38.76	0.01\\
38.77	0.01\\
38.78	0.01\\
38.79	0.01\\
38.8	0.01\\
38.81	0.01\\
38.82	0.01\\
38.83	0.01\\
38.84	0.01\\
38.85	0.01\\
38.86	0.01\\
38.87	0.01\\
38.88	0.01\\
38.89	0.01\\
38.9	0.01\\
38.91	0.01\\
38.92	0.01\\
38.93	0.01\\
38.94	0.01\\
38.95	0.01\\
38.96	0.01\\
38.97	0.01\\
38.98	0.01\\
38.99	0.01\\
39	0.01\\
39.01	0.01\\
39.02	0.01\\
39.03	0.01\\
39.04	0.01\\
39.05	0.01\\
39.06	0.01\\
39.07	0.01\\
39.08	0.01\\
39.09	0.01\\
39.1	0.01\\
39.11	0.01\\
39.12	0.01\\
39.13	0.01\\
39.14	0.01\\
39.15	0.01\\
39.16	0.01\\
39.17	0.01\\
39.18	0.01\\
39.19	0.01\\
39.2	0.01\\
39.21	0.01\\
39.22	0.01\\
39.23	0.01\\
39.24	0.01\\
39.25	0.01\\
39.26	0.01\\
39.27	0.01\\
39.28	0.01\\
39.29	0.01\\
39.3	0.01\\
39.31	0.01\\
39.32	0.01\\
39.33	0.01\\
39.34	0.01\\
39.35	0.01\\
39.36	0.01\\
39.37	0.01\\
39.38	0.01\\
39.39	0.01\\
39.4	0.01\\
39.41	0.01\\
39.42	0.01\\
39.43	0.01\\
39.44	0.01\\
39.45	0.01\\
39.46	0.01\\
39.47	0.01\\
39.48	0.01\\
39.49	0.01\\
39.5	0.01\\
39.51	0.01\\
39.52	0.01\\
39.53	0.01\\
39.54	0.01\\
39.55	0.01\\
39.56	0.01\\
39.57	0.01\\
39.58	0.01\\
39.59	0.01\\
39.6	0.01\\
39.61	0.01\\
39.62	0.01\\
39.63	0.01\\
39.64	0.01\\
39.65	0.01\\
39.66	0.01\\
39.67	0.01\\
39.68	0.01\\
39.69	0.01\\
39.7	0.01\\
39.71	0.01\\
39.72	0.01\\
39.73	0.01\\
39.74	0.01\\
39.75	0.01\\
39.76	0.01\\
39.77	0.01\\
39.78	0.01\\
39.79	0.01\\
39.8	0.01\\
39.81	0.01\\
39.82	0.01\\
39.83	0.01\\
39.84	0.01\\
39.85	0.01\\
39.86	0.01\\
39.87	0.01\\
39.88	0.01\\
39.89	0.01\\
39.9	0.01\\
39.91	0.01\\
39.92	0.01\\
39.93	0.01\\
39.94	0.01\\
39.95	0.01\\
39.96	0.01\\
39.97	0.01\\
39.98	0.01\\
39.99	0.01\\
40	0.01\\
40.01	0.01\\
};
\addplot [color=green,dashed,forget plot]
  table[row sep=crcr]{%
40.01	0.01\\
40.02	0.01\\
40.03	0.01\\
40.04	0.01\\
40.05	0.01\\
40.06	0.01\\
40.07	0.01\\
40.08	0.01\\
40.09	0.01\\
40.1	0.01\\
40.11	0.01\\
40.12	0.01\\
40.13	0.01\\
40.14	0.01\\
40.15	0.01\\
40.16	0.01\\
40.17	0.01\\
40.18	0.01\\
40.19	0.01\\
40.2	0.01\\
40.21	0.01\\
40.22	0.01\\
40.23	0.01\\
40.24	0.01\\
40.25	0.01\\
40.26	0.01\\
40.27	0.01\\
40.28	0.01\\
40.29	0.01\\
40.3	0.01\\
40.31	0.01\\
40.32	0.01\\
40.33	0.01\\
40.34	0.01\\
40.35	0.01\\
40.36	0.01\\
40.37	0.01\\
40.38	0.01\\
40.39	0.01\\
40.4	0.01\\
40.41	0.01\\
40.42	0.01\\
40.43	0.01\\
40.44	0.01\\
40.45	0.01\\
40.46	0.01\\
40.47	0.01\\
40.48	0.01\\
40.49	0.01\\
40.5	0.01\\
40.51	0.01\\
40.52	0.01\\
40.53	0.01\\
40.54	0.01\\
40.55	0.01\\
40.56	0.01\\
40.57	0.01\\
40.58	0.01\\
40.59	0.01\\
40.6	0.01\\
40.61	0.01\\
40.62	0.01\\
40.63	0.01\\
40.64	0.01\\
40.65	0.01\\
40.66	0.01\\
40.67	0.01\\
40.68	0.01\\
40.69	0.01\\
40.7	0.01\\
40.71	0.01\\
40.72	0.01\\
40.73	0.01\\
40.74	0.01\\
40.75	0.01\\
40.76	0.01\\
40.77	0.01\\
40.78	0.01\\
40.79	0.01\\
40.8	0.01\\
40.81	0.01\\
40.82	0.01\\
40.83	0.01\\
40.84	0.01\\
40.85	0.01\\
40.86	0.01\\
40.87	0.01\\
40.88	0.01\\
40.89	0.01\\
40.9	0.01\\
40.91	0.01\\
40.92	0.01\\
40.93	0.01\\
40.94	0.01\\
40.95	0.01\\
40.96	0.01\\
40.97	0.01\\
40.98	0.01\\
40.99	0.01\\
41	0.01\\
41.01	0.01\\
41.02	0.01\\
41.03	0.01\\
41.04	0.01\\
41.05	0.01\\
41.06	0.01\\
41.07	0.01\\
41.08	0.01\\
41.09	0.01\\
41.1	0.01\\
41.11	0.01\\
41.12	0.01\\
41.13	0.01\\
41.14	0.01\\
41.15	0.01\\
41.16	0.01\\
41.17	0.01\\
41.18	0.01\\
41.19	0.01\\
41.2	0.01\\
41.21	0.01\\
41.22	0.01\\
41.23	0.01\\
41.24	0.01\\
41.25	0.01\\
41.26	0.01\\
41.27	0.01\\
41.28	0.01\\
41.29	0.01\\
41.3	0.01\\
41.31	0.01\\
41.32	0.01\\
41.33	0.01\\
41.34	0.01\\
41.35	0.01\\
41.36	0.01\\
41.37	0.01\\
41.38	0.01\\
41.39	0.01\\
41.4	0.01\\
41.41	0.01\\
41.42	0.01\\
41.43	0.01\\
41.44	0.01\\
41.45	0.01\\
41.46	0.01\\
41.47	0.01\\
41.48	0.01\\
41.49	0.01\\
41.5	0.01\\
41.51	0.01\\
41.52	0.01\\
41.53	0.01\\
41.54	0.01\\
41.55	0.01\\
41.56	0.01\\
41.57	0.01\\
41.58	0.01\\
41.59	0.01\\
41.6	0.01\\
41.61	0.01\\
41.62	0.01\\
41.63	0.01\\
41.64	0.01\\
41.65	0.01\\
41.66	0.01\\
41.67	0.01\\
41.68	0.01\\
41.69	0.01\\
41.7	0.01\\
41.71	0.01\\
41.72	0.01\\
41.73	0.01\\
41.74	0.01\\
41.75	0.01\\
41.76	0.01\\
41.77	0.01\\
41.78	0.01\\
41.79	0.01\\
41.8	0.01\\
41.81	0.01\\
41.82	0.01\\
41.83	0.01\\
41.84	0.01\\
41.85	0.01\\
41.86	0.01\\
41.87	0.01\\
41.88	0.01\\
41.89	0.01\\
41.9	0.01\\
41.91	0.01\\
41.92	0.01\\
41.93	0.01\\
41.94	0.01\\
41.95	0.01\\
41.96	0.01\\
41.97	0.01\\
41.98	0.01\\
41.99	0.01\\
42	0.01\\
42.01	0.01\\
42.02	0.01\\
42.03	0.01\\
42.04	0.01\\
42.05	0.01\\
42.06	0.01\\
42.07	0.01\\
42.08	0.01\\
42.09	0.01\\
42.1	0.01\\
42.11	0.01\\
42.12	0.01\\
42.13	0.01\\
42.14	0.01\\
42.15	0.01\\
42.16	0.01\\
42.17	0.01\\
42.18	0.01\\
42.19	0.01\\
42.2	0.01\\
42.21	0.01\\
42.22	0.01\\
42.23	0.01\\
42.24	0.01\\
42.25	0.01\\
42.26	0.01\\
42.27	0.01\\
42.28	0.01\\
42.29	0.01\\
42.3	0.01\\
42.31	0.01\\
42.32	0.01\\
42.33	0.01\\
42.34	0.01\\
42.35	0.01\\
42.36	0.01\\
42.37	0.01\\
42.38	0.01\\
42.39	0.01\\
42.4	0.01\\
42.41	0.01\\
42.42	0.01\\
42.43	0.01\\
42.44	0.01\\
42.45	0.01\\
42.46	0.01\\
42.47	0.01\\
42.48	0.01\\
42.49	0.01\\
42.5	0.01\\
42.51	0.01\\
42.52	0.01\\
42.53	0.01\\
42.54	0.01\\
42.55	0.01\\
42.56	0.01\\
42.57	0.01\\
42.58	0.01\\
42.59	0.01\\
42.6	0.01\\
42.61	0.01\\
42.62	0.01\\
42.63	0.01\\
42.64	0.01\\
42.65	0.01\\
42.66	0.01\\
42.67	0.01\\
42.68	0.01\\
42.69	0.01\\
42.7	0.01\\
42.71	0.01\\
42.72	0.01\\
42.73	0.01\\
42.74	0.01\\
42.75	0.01\\
42.76	0.01\\
42.77	0.01\\
42.78	0.01\\
42.79	0.01\\
42.8	0.01\\
42.81	0.01\\
42.82	0.01\\
42.83	0.01\\
42.84	0.01\\
42.85	0.01\\
42.86	0.01\\
42.87	0.01\\
42.88	0.01\\
42.89	0.01\\
42.9	0.01\\
42.91	0.01\\
42.92	0.01\\
42.93	0.01\\
42.94	0.01\\
42.95	0.01\\
42.96	0.01\\
42.97	0.01\\
42.98	0.01\\
42.99	0.01\\
43	0.01\\
43.01	0.01\\
43.02	0.01\\
43.03	0.01\\
43.04	0.01\\
43.05	0.01\\
43.06	0.01\\
43.07	0.01\\
43.08	0.01\\
43.09	0.01\\
43.1	0.01\\
43.11	0.01\\
43.12	0.01\\
43.13	0.01\\
43.14	0.01\\
43.15	0.01\\
43.16	0.01\\
43.17	0.01\\
43.18	0.01\\
43.19	0.01\\
43.2	0.01\\
43.21	0.01\\
43.22	0.01\\
43.23	0.01\\
43.24	0.01\\
43.25	0.01\\
43.26	0.01\\
43.27	0.01\\
43.28	0.01\\
43.29	0.01\\
43.3	0.01\\
43.31	0.01\\
43.32	0.01\\
43.33	0.01\\
43.34	0.01\\
43.35	0.01\\
43.36	0.01\\
43.37	0.01\\
43.38	0.01\\
43.39	0.01\\
43.4	0.01\\
43.41	0.01\\
43.42	0.01\\
43.43	0.01\\
43.44	0.01\\
43.45	0.01\\
43.46	0.01\\
43.47	0.01\\
43.48	0.01\\
43.49	0.01\\
43.5	0.01\\
43.51	0.01\\
43.52	0.01\\
43.53	0.01\\
43.54	0.01\\
43.55	0.01\\
43.56	0.01\\
43.57	0.01\\
43.58	0.01\\
43.59	0.01\\
43.6	0.01\\
43.61	0.01\\
43.62	0.01\\
43.63	0.01\\
43.64	0.01\\
43.65	0.01\\
43.66	0.01\\
43.67	0.01\\
43.68	0.01\\
43.69	0.01\\
43.7	0.01\\
43.71	0.01\\
43.72	0.01\\
43.73	0.01\\
43.74	0.01\\
43.75	0.01\\
43.76	0.01\\
43.77	0.01\\
43.78	0.01\\
43.79	0.01\\
43.8	0.01\\
43.81	0.01\\
43.82	0.01\\
43.83	0.01\\
43.84	0.01\\
43.85	0.01\\
43.86	0.01\\
43.87	0.01\\
43.88	0.01\\
43.89	0.01\\
43.9	0.01\\
43.91	0.01\\
43.92	0.01\\
43.93	0.01\\
43.94	0.01\\
43.95	0.01\\
43.96	0.01\\
43.97	0.01\\
43.98	0.01\\
43.99	0.01\\
44	0.01\\
44.01	0.01\\
44.02	0.01\\
44.03	0.01\\
44.04	0.01\\
44.05	0.01\\
44.06	0.01\\
44.07	0.01\\
44.08	0.01\\
44.09	0.01\\
44.1	0.01\\
44.11	0.01\\
44.12	0.01\\
44.13	0.01\\
44.14	0.01\\
44.15	0.01\\
44.16	0.01\\
44.17	0.01\\
44.18	0.01\\
44.19	0.01\\
44.2	0.01\\
44.21	0.01\\
44.22	0.01\\
44.23	0.01\\
44.24	0.01\\
44.25	0.01\\
44.26	0.01\\
44.27	0.01\\
44.28	0.01\\
44.29	0.01\\
44.3	0.01\\
44.31	0.01\\
44.32	0.01\\
44.33	0.01\\
44.34	0.01\\
44.35	0.01\\
44.36	0.01\\
44.37	0.01\\
44.38	0.01\\
44.39	0.01\\
44.4	0.01\\
44.41	0.01\\
44.42	0.01\\
44.43	0.01\\
44.44	0.01\\
44.45	0.01\\
44.46	0.01\\
44.47	0.01\\
44.48	0.01\\
44.49	0.01\\
44.5	0.01\\
44.51	0.01\\
44.52	0.01\\
44.53	0.01\\
44.54	0.01\\
44.55	0.01\\
44.56	0.01\\
44.57	0.01\\
44.58	0.01\\
44.59	0.01\\
44.6	0.01\\
44.61	0.01\\
44.62	0.01\\
44.63	0.01\\
44.64	0.01\\
44.65	0.01\\
44.66	0.01\\
44.67	0.01\\
44.68	0.01\\
44.69	0.01\\
44.7	0.01\\
44.71	0.01\\
44.72	0.01\\
44.73	0.01\\
44.74	0.01\\
44.75	0.01\\
44.76	0.01\\
44.77	0.01\\
44.78	0.01\\
44.79	0.01\\
44.8	0.01\\
44.81	0.01\\
44.82	0.01\\
44.83	0.01\\
44.84	0.01\\
44.85	0.01\\
44.86	0.01\\
44.87	0.01\\
44.88	0.01\\
44.89	0.01\\
44.9	0.01\\
44.91	0.01\\
44.92	0.01\\
44.93	0.01\\
44.94	0.01\\
44.95	0.01\\
44.96	0.01\\
44.97	0.01\\
44.98	0.01\\
44.99	0.01\\
45	0.01\\
45.01	0.01\\
45.02	0.01\\
45.03	0.01\\
45.04	0.01\\
45.05	0.01\\
45.06	0.01\\
45.07	0.01\\
45.08	0.01\\
45.09	0.01\\
45.1	0.01\\
45.11	0.01\\
45.12	0.01\\
45.13	0.01\\
45.14	0.01\\
45.15	0.01\\
45.16	0.01\\
45.17	0.01\\
45.18	0.01\\
45.19	0.01\\
45.2	0.01\\
45.21	0.01\\
45.22	0.01\\
45.23	0.01\\
45.24	0.01\\
45.25	0.01\\
45.26	0.01\\
45.27	0.01\\
45.28	0.01\\
45.29	0.01\\
45.3	0.01\\
45.31	0.01\\
45.32	0.01\\
45.33	0.01\\
45.34	0.01\\
45.35	0.01\\
45.36	0.01\\
45.37	0.01\\
45.38	0.01\\
45.39	0.01\\
45.4	0.01\\
45.41	0.01\\
45.42	0.01\\
45.43	0.01\\
45.44	0.01\\
45.45	0.01\\
45.46	0.01\\
45.47	0.01\\
45.48	0.01\\
45.49	0.01\\
45.5	0.01\\
45.51	0.01\\
45.52	0.01\\
45.53	0.01\\
45.54	0.01\\
45.55	0.01\\
45.56	0.01\\
45.57	0.01\\
45.58	0.01\\
45.59	0.01\\
45.6	0.01\\
45.61	0.01\\
45.62	0.01\\
45.63	0.01\\
45.64	0.01\\
45.65	0.01\\
45.66	0.01\\
45.67	0.01\\
45.68	0.01\\
45.69	0.01\\
45.7	0.01\\
45.71	0.01\\
45.72	0.01\\
45.73	0.01\\
45.74	0.01\\
45.75	0.01\\
45.76	0.01\\
45.77	0.01\\
45.78	0.01\\
45.79	0.01\\
45.8	0.01\\
45.81	0.01\\
45.82	0.01\\
45.83	0.01\\
45.84	0.01\\
45.85	0.01\\
45.86	0.01\\
45.87	0.01\\
45.88	0.01\\
45.89	0.01\\
45.9	0.01\\
45.91	0.01\\
45.92	0.01\\
45.93	0.01\\
45.94	0.01\\
45.95	0.01\\
45.96	0.01\\
45.97	0.01\\
45.98	0.01\\
45.99	0.01\\
46	0.01\\
46.01	0.01\\
46.02	0.01\\
46.03	0.01\\
46.04	0.01\\
46.05	0.01\\
46.06	0.01\\
46.07	0.01\\
46.08	0.01\\
46.09	0.01\\
46.1	0.01\\
46.11	0.01\\
46.12	0.01\\
46.13	0.01\\
46.14	0.01\\
46.15	0.01\\
46.16	0.01\\
46.17	0.01\\
46.18	0.01\\
46.19	0.01\\
46.2	0.01\\
46.21	0.01\\
46.22	0.01\\
46.23	0.01\\
46.24	0.01\\
46.25	0.01\\
46.26	0.01\\
46.27	0.01\\
46.28	0.01\\
46.29	0.01\\
46.3	0.01\\
46.31	0.01\\
46.32	0.01\\
46.33	0.01\\
46.34	0.01\\
46.35	0.01\\
46.36	0.01\\
46.37	0.01\\
46.38	0.01\\
46.39	0.01\\
46.4	0.01\\
46.41	0.01\\
46.42	0.01\\
46.43	0.01\\
46.44	0.01\\
46.45	0.01\\
46.46	0.01\\
46.47	0.01\\
46.48	0.01\\
46.49	0.01\\
46.5	0.01\\
46.51	0.01\\
46.52	0.01\\
46.53	0.01\\
46.54	0.01\\
46.55	0.01\\
46.56	0.01\\
46.57	0.01\\
46.58	0.01\\
46.59	0.01\\
46.6	0.01\\
46.61	0.01\\
46.62	0.01\\
46.63	0.01\\
46.64	0.01\\
46.65	0.01\\
46.66	0.01\\
46.67	0.01\\
46.68	0.01\\
46.69	0.01\\
46.7	0.01\\
46.71	0.01\\
46.72	0.01\\
46.73	0.01\\
46.74	0.01\\
46.75	0.01\\
46.76	0.01\\
46.77	0.01\\
46.78	0.01\\
46.79	0.01\\
46.8	0.01\\
46.81	0.01\\
46.82	0.01\\
46.83	0.01\\
46.84	0.01\\
46.85	0.01\\
46.86	0.01\\
46.87	0.01\\
46.88	0.01\\
46.89	0.01\\
46.9	0.01\\
46.91	0.01\\
46.92	0.01\\
46.93	0.01\\
46.94	0.01\\
46.95	0.01\\
46.96	0.01\\
46.97	0.01\\
46.98	0.01\\
46.99	0.01\\
47	0.01\\
47.01	0.01\\
47.02	0.01\\
47.03	0.01\\
47.04	0.01\\
47.05	0.01\\
47.06	0.01\\
47.07	0.01\\
47.08	0.01\\
47.09	0.01\\
47.1	0.01\\
47.11	0.01\\
47.12	0.01\\
47.13	0.01\\
47.14	0.01\\
47.15	0.01\\
47.16	0.01\\
47.17	0.01\\
47.18	0.01\\
47.19	0.01\\
47.2	0.01\\
47.21	0.01\\
47.22	0.01\\
47.23	0.01\\
47.24	0.01\\
47.25	0.01\\
47.26	0.01\\
47.27	0.01\\
47.28	0.01\\
47.29	0.01\\
47.3	0.01\\
47.31	0.01\\
47.32	0.01\\
47.33	0.01\\
47.34	0.01\\
47.35	0.01\\
47.36	0.01\\
47.37	0.01\\
47.38	0.01\\
47.39	0.01\\
47.4	0.01\\
47.41	0.01\\
47.42	0.01\\
47.43	0.01\\
47.44	0.01\\
47.45	0.01\\
47.46	0.01\\
47.47	0.01\\
47.48	0.01\\
47.49	0.01\\
47.5	0.01\\
47.51	0.01\\
47.52	0.01\\
47.53	0.01\\
47.54	0.01\\
47.55	0.01\\
47.56	0.01\\
47.57	0.01\\
47.58	0.01\\
47.59	0.01\\
47.6	0.01\\
47.61	0.01\\
47.62	0.01\\
47.63	0.01\\
47.64	0.01\\
47.65	0.01\\
47.66	0.01\\
47.67	0.01\\
47.68	0.01\\
47.69	0.01\\
47.7	0.01\\
47.71	0.01\\
47.72	0.01\\
47.73	0.01\\
47.74	0.01\\
47.75	0.01\\
47.76	0.01\\
47.77	0.01\\
47.78	0.01\\
47.79	0.01\\
47.8	0.01\\
47.81	0.01\\
47.82	0.01\\
47.83	0.01\\
47.84	0.01\\
47.85	0.01\\
47.86	0.01\\
47.87	0.01\\
47.88	0.01\\
47.89	0.01\\
47.9	0.01\\
47.91	0.01\\
47.92	0.01\\
47.93	0.01\\
47.94	0.01\\
47.95	0.01\\
47.96	0.01\\
47.97	0.01\\
47.98	0.01\\
47.99	0.01\\
48	0.01\\
48.01	0.01\\
48.02	0.01\\
48.03	0.01\\
48.04	0.01\\
48.05	0.01\\
48.06	0.01\\
48.07	0.01\\
48.08	0.01\\
48.09	0.01\\
48.1	0.01\\
48.11	0.01\\
48.12	0.01\\
48.13	0.01\\
48.14	0.01\\
48.15	0.01\\
48.16	0.01\\
48.17	0.01\\
48.18	0.01\\
48.19	0.01\\
48.2	0.01\\
48.21	0.01\\
48.22	0.01\\
48.23	0.01\\
48.24	0.01\\
48.25	0.01\\
48.26	0.01\\
48.27	0.01\\
48.28	0.01\\
48.29	0.01\\
48.3	0.01\\
48.31	0.01\\
48.32	0.01\\
48.33	0.01\\
48.34	0.01\\
48.35	0.01\\
48.36	0.01\\
48.37	0.01\\
48.38	0.01\\
48.39	0.01\\
48.4	0.01\\
48.41	0.01\\
48.42	0.01\\
48.43	0.01\\
48.44	0.01\\
48.45	0.01\\
48.46	0.01\\
48.47	0.01\\
48.48	0.01\\
48.49	0.01\\
48.5	0.01\\
48.51	0.01\\
48.52	0.01\\
48.53	0.01\\
48.54	0.01\\
48.55	0.01\\
48.56	0.01\\
48.57	0.01\\
48.58	0.01\\
48.59	0.01\\
48.6	0.01\\
48.61	0.01\\
48.62	0.01\\
48.63	0.01\\
48.64	0.01\\
48.65	0.01\\
48.66	0.01\\
48.67	0.01\\
48.68	0.01\\
48.69	0.01\\
48.7	0.01\\
48.71	0.01\\
48.72	0.01\\
48.73	0.01\\
48.74	0.01\\
48.75	0.01\\
48.76	0.01\\
48.77	0.01\\
48.78	0.01\\
48.79	0.01\\
48.8	0.01\\
48.81	0.01\\
48.82	0.01\\
48.83	0.01\\
48.84	0.01\\
48.85	0.01\\
48.86	0.01\\
48.87	0.01\\
48.88	0.01\\
48.89	0.01\\
48.9	0.01\\
48.91	0.01\\
48.92	0.01\\
48.93	0.01\\
48.94	0.01\\
48.95	0.01\\
48.96	0.01\\
48.97	0.01\\
48.98	0.01\\
48.99	0.01\\
49	0.01\\
49.01	0.01\\
49.02	0.01\\
49.03	0.01\\
49.04	0.01\\
49.05	0.01\\
49.06	0.01\\
49.07	0.01\\
49.08	0.01\\
49.09	0.01\\
49.1	0.01\\
49.11	0.01\\
49.12	0.01\\
49.13	0.01\\
49.14	0.01\\
49.15	0.01\\
49.16	0.01\\
49.17	0.01\\
49.18	0.01\\
49.19	0.01\\
49.2	0.01\\
49.21	0.01\\
49.22	0.01\\
49.23	0.01\\
49.24	0.01\\
49.25	0.01\\
49.26	0.01\\
49.27	0.01\\
49.28	0.01\\
49.29	0.01\\
49.3	0.01\\
49.31	0.01\\
49.32	0.01\\
49.33	0.01\\
49.34	0.01\\
49.35	0.01\\
49.36	0.01\\
49.37	0.01\\
49.38	0.01\\
49.39	0.01\\
49.4	0.01\\
49.41	0.01\\
49.42	0.01\\
49.43	0.01\\
49.44	0.01\\
49.45	0.01\\
49.46	0.01\\
49.47	0.01\\
49.48	0.01\\
49.49	0.01\\
49.5	0.01\\
49.51	0.01\\
49.52	0.01\\
49.53	0.01\\
49.54	0.01\\
49.55	0.01\\
49.56	0.01\\
49.57	0.01\\
49.58	0.01\\
49.59	0.01\\
49.6	0.01\\
49.61	0.01\\
49.62	0.01\\
49.63	0.01\\
49.64	0.01\\
49.65	0.01\\
49.66	0.01\\
49.67	0.01\\
49.68	0.01\\
49.69	0.01\\
49.7	0.01\\
49.71	0.01\\
49.72	0.01\\
49.73	0.01\\
49.74	0.01\\
49.75	0.01\\
49.76	0.01\\
49.77	0.01\\
49.78	0.01\\
49.79	0.01\\
49.8	0.01\\
49.81	0.01\\
49.82	0.01\\
49.83	0.01\\
49.84	0.01\\
49.85	0.01\\
49.86	0.01\\
49.87	0.01\\
49.88	0.01\\
49.89	0.01\\
49.9	0.01\\
49.91	0.01\\
49.92	0.01\\
49.93	0.01\\
49.94	0.01\\
49.95	0.01\\
49.96	0.01\\
49.97	0.01\\
49.98	0.01\\
49.99	0.01\\
50	0.01\\
50.01	0.01\\
50.02	0.01\\
50.03	0.01\\
50.04	0.01\\
50.05	0.01\\
50.06	0.01\\
50.07	0.01\\
50.08	0.01\\
50.09	0.01\\
50.1	0.01\\
50.11	0.01\\
50.12	0.01\\
50.13	0.01\\
50.14	0.01\\
50.15	0.01\\
50.16	0.01\\
50.17	0.01\\
50.18	0.01\\
50.19	0.01\\
50.2	0.01\\
50.21	0.01\\
50.22	0.01\\
50.23	0.01\\
50.24	0.01\\
50.25	0.01\\
50.26	0.01\\
50.27	0.01\\
50.28	0.01\\
50.29	0.01\\
50.3	0.01\\
50.31	0.01\\
50.32	0.01\\
50.33	0.01\\
50.34	0.01\\
50.35	0.01\\
50.36	0.01\\
50.37	0.01\\
50.38	0.01\\
50.39	0.01\\
50.4	0.01\\
50.41	0.01\\
50.42	0.01\\
50.43	0.01\\
50.44	0.01\\
50.45	0.01\\
50.46	0.01\\
50.47	0.01\\
50.48	0.01\\
50.49	0.01\\
50.5	0.01\\
50.51	0.01\\
50.52	0.01\\
50.53	0.01\\
50.54	0.01\\
50.55	0.01\\
50.56	0.01\\
50.57	0.01\\
50.58	0.01\\
50.59	0.01\\
50.6	0.01\\
50.61	0.01\\
50.62	0.01\\
50.63	0.01\\
50.64	0.01\\
50.65	0.01\\
50.66	0.01\\
50.67	0.01\\
50.68	0.01\\
50.69	0.01\\
50.7	0.01\\
50.71	0.01\\
50.72	0.01\\
50.73	0.01\\
50.74	0.01\\
50.75	0.01\\
50.76	0.01\\
50.77	0.01\\
50.78	0.01\\
50.79	0.01\\
50.8	0.01\\
50.81	0.01\\
50.82	0.01\\
50.83	0.01\\
50.84	0.01\\
50.85	0.01\\
50.86	0.01\\
50.87	0.01\\
50.88	0.01\\
50.89	0.01\\
50.9	0.01\\
50.91	0.01\\
50.92	0.01\\
50.93	0.01\\
50.94	0.01\\
50.95	0.01\\
50.96	0.01\\
50.97	0.01\\
50.98	0.01\\
50.99	0.01\\
51	0.01\\
51.01	0.01\\
51.02	0.01\\
51.03	0.01\\
51.04	0.01\\
51.05	0.01\\
51.06	0.01\\
51.07	0.01\\
51.08	0.01\\
51.09	0.01\\
51.1	0.01\\
51.11	0.01\\
51.12	0.01\\
51.13	0.01\\
51.14	0.01\\
51.15	0.01\\
51.16	0.01\\
51.17	0.01\\
51.18	0.01\\
51.19	0.01\\
51.2	0.01\\
51.21	0.01\\
51.22	0.01\\
51.23	0.01\\
51.24	0.01\\
51.25	0.01\\
51.26	0.01\\
51.27	0.01\\
51.28	0.01\\
51.29	0.01\\
51.3	0.01\\
51.31	0.01\\
51.32	0.01\\
51.33	0.01\\
51.34	0.01\\
51.35	0.01\\
51.36	0.01\\
51.37	0.01\\
51.38	0.01\\
51.39	0.01\\
51.4	0.01\\
51.41	0.01\\
51.42	0.01\\
51.43	0.01\\
51.44	0.01\\
51.45	0.01\\
51.46	0.01\\
51.47	0.01\\
51.48	0.01\\
51.49	0.01\\
51.5	0.01\\
51.51	0.01\\
51.52	0.01\\
51.53	0.01\\
51.54	0.01\\
51.55	0.01\\
51.56	0.01\\
51.57	0.01\\
51.58	0.01\\
51.59	0.01\\
51.6	0.01\\
51.61	0.01\\
51.62	0.01\\
51.63	0.01\\
51.64	0.01\\
51.65	0.01\\
51.66	0.01\\
51.67	0.01\\
51.68	0.01\\
51.69	0.01\\
51.7	0.01\\
51.71	0.01\\
51.72	0.01\\
51.73	0.01\\
51.74	0.01\\
51.75	0.01\\
51.76	0.01\\
51.77	0.01\\
51.78	0.01\\
51.79	0.01\\
51.8	0.01\\
51.81	0.01\\
51.82	0.01\\
51.83	0.01\\
51.84	0.01\\
51.85	0.01\\
51.86	0.01\\
51.87	0.01\\
51.88	0.01\\
51.89	0.01\\
51.9	0.01\\
51.91	0.01\\
51.92	0.01\\
51.93	0.01\\
51.94	0.01\\
51.95	0.01\\
51.96	0.01\\
51.97	0.01\\
51.98	0.01\\
51.99	0.01\\
52	0.01\\
52.01	0.01\\
52.02	0.01\\
52.03	0.01\\
52.04	0.01\\
52.05	0.01\\
52.06	0.01\\
52.07	0.01\\
52.08	0.01\\
52.09	0.01\\
52.1	0.01\\
52.11	0.01\\
52.12	0.01\\
52.13	0.01\\
52.14	0.01\\
52.15	0.01\\
52.16	0.01\\
52.17	0.01\\
52.18	0.01\\
52.19	0.01\\
52.2	0.01\\
52.21	0.01\\
52.22	0.01\\
52.23	0.01\\
52.24	0.01\\
52.25	0.01\\
52.26	0.01\\
52.27	0.01\\
52.28	0.01\\
52.29	0.01\\
52.3	0.01\\
52.31	0.01\\
52.32	0.01\\
52.33	0.01\\
52.34	0.01\\
52.35	0.01\\
52.36	0.01\\
52.37	0.01\\
52.38	0.01\\
52.39	0.01\\
52.4	0.01\\
52.41	0.01\\
52.42	0.01\\
52.43	0.01\\
52.44	0.01\\
52.45	0.01\\
52.46	0.01\\
52.47	0.01\\
52.48	0.01\\
52.49	0.01\\
52.5	0.01\\
52.51	0.01\\
52.52	0.01\\
52.53	0.01\\
52.54	0.01\\
52.55	0.01\\
52.56	0.01\\
52.57	0.01\\
52.58	0.01\\
52.59	0.01\\
52.6	0.01\\
52.61	0.01\\
52.62	0.01\\
52.63	0.01\\
52.64	0.01\\
52.65	0.01\\
52.66	0.01\\
52.67	0.01\\
52.68	0.01\\
52.69	0.01\\
52.7	0.01\\
52.71	0.01\\
52.72	0.01\\
52.73	0.01\\
52.74	0.01\\
52.75	0.01\\
52.76	0.01\\
52.77	0.01\\
52.78	0.01\\
52.79	0.01\\
52.8	0.01\\
52.81	0.01\\
52.82	0.01\\
52.83	0.01\\
52.84	0.01\\
52.85	0.01\\
52.86	0.01\\
52.87	0.01\\
52.88	0.01\\
52.89	0.01\\
52.9	0.01\\
52.91	0.01\\
52.92	0.01\\
52.93	0.01\\
52.94	0.01\\
52.95	0.01\\
52.96	0.01\\
52.97	0.01\\
52.98	0.01\\
52.99	0.01\\
53	0.01\\
53.01	0.01\\
53.02	0.01\\
53.03	0.01\\
53.04	0.01\\
53.05	0.01\\
53.06	0.01\\
53.07	0.01\\
53.08	0.01\\
53.09	0.01\\
53.1	0.01\\
53.11	0.01\\
53.12	0.01\\
53.13	0.01\\
53.14	0.01\\
53.15	0.01\\
53.16	0.01\\
53.17	0.01\\
53.18	0.01\\
53.19	0.01\\
53.2	0.01\\
53.21	0.01\\
53.22	0.01\\
53.23	0.01\\
53.24	0.01\\
53.25	0.01\\
53.26	0.01\\
53.27	0.01\\
53.28	0.01\\
53.29	0.01\\
53.3	0.01\\
53.31	0.01\\
53.32	0.01\\
53.33	0.01\\
53.34	0.01\\
53.35	0.01\\
53.36	0.01\\
53.37	0.01\\
53.38	0.01\\
53.39	0.01\\
53.4	0.01\\
53.41	0.01\\
53.42	0.01\\
53.43	0.01\\
53.44	0.01\\
53.45	0.01\\
53.46	0.01\\
53.47	0.01\\
53.48	0.01\\
53.49	0.01\\
53.5	0.01\\
53.51	0.01\\
53.52	0.01\\
53.53	0.01\\
53.54	0.01\\
53.55	0.01\\
53.56	0.01\\
53.57	0.01\\
53.58	0.01\\
53.59	0.01\\
53.6	0.01\\
53.61	0.01\\
53.62	0.01\\
53.63	0.01\\
53.64	0.01\\
53.65	0.01\\
53.66	0.01\\
53.67	0.01\\
53.68	0.01\\
53.69	0.01\\
53.7	0.01\\
53.71	0.01\\
53.72	0.01\\
53.73	0.01\\
53.74	0.01\\
53.75	0.01\\
53.76	0.01\\
53.77	0.01\\
53.78	0.01\\
53.79	0.01\\
53.8	0.01\\
53.81	0.01\\
53.82	0.01\\
53.83	0.01\\
53.84	0.01\\
53.85	0.01\\
53.86	0.01\\
53.87	0.01\\
53.88	0.01\\
53.89	0.01\\
53.9	0.01\\
53.91	0.01\\
53.92	0.01\\
53.93	0.01\\
53.94	0.01\\
53.95	0.01\\
53.96	0.01\\
53.97	0.01\\
53.98	0.01\\
53.99	0.01\\
54	0.01\\
54.01	0.01\\
54.02	0.01\\
54.03	0.01\\
54.04	0.01\\
54.05	0.01\\
54.06	0.01\\
54.07	0.01\\
54.08	0.01\\
54.09	0.01\\
54.1	0.01\\
54.11	0.01\\
54.12	0.01\\
54.13	0.01\\
54.14	0.01\\
54.15	0.01\\
54.16	0.01\\
54.17	0.01\\
54.18	0.01\\
54.19	0.01\\
54.2	0.01\\
54.21	0.01\\
54.22	0.01\\
54.23	0.01\\
54.24	0.01\\
54.25	0.01\\
54.26	0.01\\
54.27	0.01\\
54.28	0.01\\
54.29	0.01\\
54.3	0.01\\
54.31	0.01\\
54.32	0.01\\
54.33	0.01\\
54.34	0.01\\
54.35	0.01\\
54.36	0.01\\
54.37	0.01\\
54.38	0.01\\
54.39	0.01\\
54.4	0.01\\
54.41	0.01\\
54.42	0.01\\
54.43	0.01\\
54.44	0.01\\
54.45	0.01\\
54.46	0.01\\
54.47	0.01\\
54.48	0.01\\
54.49	0.01\\
54.5	0.01\\
54.51	0.01\\
54.52	0.01\\
54.53	0.01\\
54.54	0.01\\
54.55	0.01\\
54.56	0.01\\
54.57	0.01\\
54.58	0.01\\
54.59	0.01\\
54.6	0.01\\
54.61	0.01\\
54.62	0.01\\
54.63	0.01\\
54.64	0.01\\
54.65	0.01\\
54.66	0.01\\
54.67	0.01\\
54.68	0.01\\
54.69	0.01\\
54.7	0.01\\
54.71	0.01\\
54.72	0.01\\
54.73	0.01\\
54.74	0.01\\
54.75	0.01\\
54.76	0.01\\
54.77	0.01\\
54.78	0.01\\
54.79	0.01\\
54.8	0.01\\
54.81	0.01\\
54.82	0.01\\
54.83	0.01\\
54.84	0.01\\
54.85	0.01\\
54.86	0.01\\
54.87	0.01\\
54.88	0.01\\
54.89	0.01\\
54.9	0.01\\
54.91	0.01\\
54.92	0.01\\
54.93	0.01\\
54.94	0.01\\
54.95	0.01\\
54.96	0.01\\
54.97	0.01\\
54.98	0.01\\
54.99	0.01\\
55	0.01\\
55.01	0.01\\
55.02	0.01\\
55.03	0.01\\
55.04	0.01\\
55.05	0.01\\
55.06	0.01\\
55.07	0.01\\
55.08	0.01\\
55.09	0.01\\
55.1	0.01\\
55.11	0.01\\
55.12	0.01\\
55.13	0.01\\
55.14	0.01\\
55.15	0.01\\
55.16	0.01\\
55.17	0.01\\
55.18	0.01\\
55.19	0.01\\
55.2	0.01\\
55.21	0.01\\
55.22	0.01\\
55.23	0.01\\
55.24	0.01\\
55.25	0.01\\
55.26	0.01\\
55.27	0.01\\
55.28	0.01\\
55.29	0.01\\
55.3	0.01\\
55.31	0.01\\
55.32	0.01\\
55.33	0.01\\
55.34	0.01\\
55.35	0.01\\
55.36	0.01\\
55.37	0.01\\
55.38	0.01\\
55.39	0.01\\
55.4	0.01\\
55.41	0.01\\
55.42	0.01\\
55.43	0.01\\
55.44	0.01\\
55.45	0.01\\
55.46	0.01\\
55.47	0.01\\
55.48	0.01\\
55.49	0.01\\
55.5	0.01\\
55.51	0.01\\
55.52	0.01\\
55.53	0.01\\
55.54	0.01\\
55.55	0.01\\
55.56	0.01\\
55.57	0.01\\
55.58	0.01\\
55.59	0.01\\
55.6	0.01\\
55.61	0.01\\
55.62	0.01\\
55.63	0.01\\
55.64	0.01\\
55.65	0.01\\
55.66	0.01\\
55.67	0.01\\
55.68	0.01\\
55.69	0.01\\
55.7	0.01\\
55.71	0.01\\
55.72	0.01\\
55.73	0.01\\
55.74	0.01\\
55.75	0.01\\
55.76	0.01\\
55.77	0.01\\
55.78	0.01\\
55.79	0.01\\
55.8	0.01\\
55.81	0.01\\
55.82	0.01\\
55.83	0.01\\
55.84	0.01\\
55.85	0.01\\
55.86	0.01\\
55.87	0.01\\
55.88	0.01\\
55.89	0.01\\
55.9	0.01\\
55.91	0.01\\
55.92	0.01\\
55.93	0.01\\
55.94	0.01\\
55.95	0.01\\
55.96	0.01\\
55.97	0.01\\
55.98	0.01\\
55.99	0.01\\
56	0.01\\
56.01	0.01\\
56.02	0.01\\
56.03	0.01\\
56.04	0.01\\
56.05	0.01\\
56.06	0.01\\
56.07	0.01\\
56.08	0.01\\
56.09	0.01\\
56.1	0.01\\
56.11	0.01\\
56.12	0.01\\
56.13	0.01\\
56.14	0.01\\
56.15	0.01\\
56.16	0.01\\
56.17	0.01\\
56.18	0.01\\
56.19	0.01\\
56.2	0.01\\
56.21	0.01\\
56.22	0.01\\
56.23	0.01\\
56.24	0.01\\
56.25	0.01\\
56.26	0.01\\
56.27	0.01\\
56.28	0.01\\
56.29	0.01\\
56.3	0.01\\
56.31	0.01\\
56.32	0.01\\
56.33	0.01\\
56.34	0.01\\
56.35	0.01\\
56.36	0.01\\
56.37	0.01\\
56.38	0.01\\
56.39	0.01\\
56.4	0.01\\
56.41	0.01\\
56.42	0.01\\
56.43	0.01\\
56.44	0.01\\
56.45	0.01\\
56.46	0.01\\
56.47	0.01\\
56.48	0.01\\
56.49	0.01\\
56.5	0.01\\
56.51	0.01\\
56.52	0.01\\
56.53	0.01\\
56.54	0.01\\
56.55	0.01\\
56.56	0.01\\
56.57	0.01\\
56.58	0.01\\
56.59	0.01\\
56.6	0.01\\
56.61	0.01\\
56.62	0.01\\
56.63	0.01\\
56.64	0.01\\
56.65	0.01\\
56.66	0.01\\
56.67	0.01\\
56.68	0.01\\
56.69	0.01\\
56.7	0.01\\
56.71	0.01\\
56.72	0.01\\
56.73	0.01\\
56.74	0.01\\
56.75	0.01\\
56.76	0.01\\
56.77	0.01\\
56.78	0.01\\
56.79	0.01\\
56.8	0.01\\
56.81	0.01\\
56.82	0.01\\
56.83	0.01\\
56.84	0.01\\
56.85	0.01\\
56.86	0.01\\
56.87	0.01\\
56.88	0.01\\
56.89	0.01\\
56.9	0.01\\
56.91	0.01\\
56.92	0.01\\
56.93	0.01\\
56.94	0.01\\
56.95	0.01\\
56.96	0.01\\
56.97	0.01\\
56.98	0.01\\
56.99	0.01\\
57	0.01\\
57.01	0.01\\
57.02	0.01\\
57.03	0.01\\
57.04	0.01\\
57.05	0.01\\
57.06	0.01\\
57.07	0.01\\
57.08	0.01\\
57.09	0.01\\
57.1	0.01\\
57.11	0.01\\
57.12	0.01\\
57.13	0.01\\
57.14	0.01\\
57.15	0.01\\
57.16	0.01\\
57.17	0.01\\
57.18	0.01\\
57.19	0.01\\
57.2	0.01\\
57.21	0.01\\
57.22	0.01\\
57.23	0.01\\
57.24	0.01\\
57.25	0.01\\
57.26	0.01\\
57.27	0.01\\
57.28	0.01\\
57.29	0.01\\
57.3	0.01\\
57.31	0.01\\
57.32	0.01\\
57.33	0.01\\
57.34	0.01\\
57.35	0.01\\
57.36	0.01\\
57.37	0.01\\
57.38	0.01\\
57.39	0.01\\
57.4	0.01\\
57.41	0.01\\
57.42	0.01\\
57.43	0.01\\
57.44	0.01\\
57.45	0.01\\
57.46	0.01\\
57.47	0.01\\
57.48	0.01\\
57.49	0.01\\
57.5	0.01\\
57.51	0.01\\
57.52	0.01\\
57.53	0.01\\
57.54	0.01\\
57.55	0.01\\
57.56	0.01\\
57.57	0.01\\
57.58	0.01\\
57.59	0.01\\
57.6	0.01\\
57.61	0.01\\
57.62	0.01\\
57.63	0.01\\
57.64	0.01\\
57.65	0.01\\
57.66	0.01\\
57.67	0.01\\
57.68	0.01\\
57.69	0.01\\
57.7	0.01\\
57.71	0.01\\
57.72	0.01\\
57.73	0.01\\
57.74	0.01\\
57.75	0.01\\
57.76	0.01\\
57.77	0.01\\
57.78	0.01\\
57.79	0.01\\
57.8	0.01\\
57.81	0.01\\
57.82	0.01\\
57.83	0.01\\
57.84	0.01\\
57.85	0.01\\
57.86	0.01\\
57.87	0.01\\
57.88	0.01\\
57.89	0.01\\
57.9	0.01\\
57.91	0.01\\
57.92	0.01\\
57.93	0.01\\
57.94	0.01\\
57.95	0.01\\
57.96	0.01\\
57.97	0.01\\
57.98	0.01\\
57.99	0.01\\
58	0.01\\
58.01	0.01\\
58.02	0.01\\
58.03	0.01\\
58.04	0.01\\
58.05	0.01\\
58.06	0.01\\
58.07	0.01\\
58.08	0.01\\
58.09	0.01\\
58.1	0.01\\
58.11	0.01\\
58.12	0.01\\
58.13	0.01\\
58.14	0.01\\
58.15	0.01\\
58.16	0.01\\
58.17	0.01\\
58.18	0.01\\
58.19	0.01\\
58.2	0.01\\
58.21	0.01\\
58.22	0.01\\
58.23	0.01\\
58.24	0.01\\
58.25	0.01\\
58.26	0.01\\
58.27	0.01\\
58.28	0.01\\
58.29	0.01\\
58.3	0.01\\
58.31	0.01\\
58.32	0.01\\
58.33	0.01\\
58.34	0.01\\
58.35	0.01\\
58.36	0.01\\
58.37	0.01\\
58.38	0.01\\
58.39	0.01\\
58.4	0.01\\
58.41	0.01\\
58.42	0.01\\
58.43	0.01\\
58.44	0.01\\
58.45	0.01\\
58.46	0.01\\
58.47	0.01\\
58.48	0.01\\
58.49	0.01\\
58.5	0.01\\
58.51	0.01\\
58.52	0.01\\
58.53	0.01\\
58.54	0.01\\
58.55	0.01\\
58.56	0.01\\
58.57	0.01\\
58.58	0.01\\
58.59	0.01\\
58.6	0.01\\
58.61	0.01\\
58.62	0.01\\
58.63	0.01\\
58.64	0.01\\
58.65	0.01\\
58.66	0.01\\
58.67	0.01\\
58.68	0.01\\
58.69	0.01\\
58.7	0.01\\
58.71	0.01\\
58.72	0.01\\
58.73	0.01\\
58.74	0.01\\
58.75	0.01\\
58.76	0.01\\
58.77	0.01\\
58.78	0.01\\
58.79	0.01\\
58.8	0.01\\
58.81	0.01\\
58.82	0.01\\
58.83	0.01\\
58.84	0.01\\
58.85	0.01\\
58.86	0.01\\
58.87	0.01\\
58.88	0.01\\
58.89	0.01\\
58.9	0.01\\
58.91	0.01\\
58.92	0.01\\
58.93	0.01\\
58.94	0.01\\
58.95	0.01\\
58.96	0.01\\
58.97	0.01\\
58.98	0.01\\
58.99	0.01\\
59	0.01\\
59.01	0.01\\
59.02	0.01\\
59.03	0.01\\
59.04	0.01\\
59.05	0.01\\
59.06	0.01\\
59.07	0.01\\
59.08	0.01\\
59.09	0.01\\
59.1	0.01\\
59.11	0.01\\
59.12	0.01\\
59.13	0.01\\
59.14	0.01\\
59.15	0.01\\
59.16	0.01\\
59.17	0.01\\
59.18	0.01\\
59.19	0.01\\
59.2	0.01\\
59.21	0.01\\
59.22	0.01\\
59.23	0.01\\
59.24	0.01\\
59.25	0.01\\
59.26	0.01\\
59.27	0.01\\
59.28	0.01\\
59.29	0.01\\
59.3	0.01\\
59.31	0.01\\
59.32	0.01\\
59.33	0.01\\
59.34	0.01\\
59.35	0.01\\
59.36	0.01\\
59.37	0.01\\
59.38	0.01\\
59.39	0.01\\
59.4	0.01\\
59.41	0.01\\
59.42	0.01\\
59.43	0.01\\
59.44	0.01\\
59.45	0.01\\
59.46	0.01\\
59.47	0.01\\
59.48	0.01\\
59.49	0.01\\
59.5	0.01\\
59.51	0.01\\
59.52	0.01\\
59.53	0.01\\
59.54	0.01\\
59.55	0.01\\
59.56	0.01\\
59.57	0.01\\
59.58	0.01\\
59.59	0.01\\
59.6	0.01\\
59.61	0.01\\
59.62	0.01\\
59.63	0.01\\
59.64	0.01\\
59.65	0.01\\
59.66	0.01\\
59.67	0.01\\
59.68	0.01\\
59.69	0.01\\
59.7	0.01\\
59.71	0.01\\
59.72	0.01\\
59.73	0.01\\
59.74	0.01\\
59.75	0.01\\
59.76	0.01\\
59.77	0.01\\
59.78	0.01\\
59.79	0.01\\
59.8	0.01\\
59.81	0.01\\
59.82	0.01\\
59.83	0.01\\
59.84	0.01\\
59.85	0.01\\
59.86	0.01\\
59.87	0.01\\
59.88	0.01\\
59.89	0.01\\
59.9	0.01\\
59.91	0.01\\
59.92	0.01\\
59.93	0.01\\
59.94	0.01\\
59.95	0.01\\
59.96	0.01\\
59.97	0.01\\
59.98	0.01\\
59.99	0.01\\
60	0.01\\
60.01	0.01\\
60.02	0.01\\
60.03	0.01\\
60.04	0.01\\
60.05	0.01\\
60.06	0.01\\
60.07	0.01\\
60.08	0.01\\
60.09	0.01\\
60.1	0.01\\
60.11	0.01\\
60.12	0.01\\
60.13	0.01\\
60.14	0.01\\
60.15	0.01\\
60.16	0.01\\
60.17	0.01\\
60.18	0.01\\
60.19	0.01\\
60.2	0.01\\
60.21	0.01\\
60.22	0.01\\
60.23	0.01\\
60.24	0.01\\
60.25	0.01\\
60.26	0.01\\
60.27	0.01\\
60.28	0.01\\
60.29	0.01\\
60.3	0.01\\
60.31	0.01\\
60.32	0.01\\
60.33	0.01\\
60.34	0.01\\
60.35	0.01\\
60.36	0.01\\
60.37	0.01\\
60.38	0.01\\
60.39	0.01\\
60.4	0.01\\
60.41	0.01\\
60.42	0.01\\
60.43	0.01\\
60.44	0.01\\
60.45	0.01\\
60.46	0.01\\
60.47	0.01\\
60.48	0.01\\
60.49	0.01\\
60.5	0.01\\
60.51	0.01\\
60.52	0.01\\
60.53	0.01\\
60.54	0.01\\
60.55	0.01\\
60.56	0.01\\
60.57	0.01\\
60.58	0.01\\
60.59	0.01\\
60.6	0.01\\
60.61	0.01\\
60.62	0.01\\
60.63	0.01\\
60.64	0.01\\
60.65	0.01\\
60.66	0.01\\
60.67	0.01\\
60.68	0.01\\
60.69	0.01\\
60.7	0.01\\
60.71	0.01\\
60.72	0.01\\
60.73	0.01\\
60.74	0.01\\
60.75	0.01\\
60.76	0.01\\
60.77	0.01\\
60.78	0.01\\
60.79	0.01\\
60.8	0.01\\
60.81	0.01\\
60.82	0.01\\
60.83	0.01\\
60.84	0.01\\
60.85	0.01\\
60.86	0.01\\
60.87	0.01\\
60.88	0.01\\
60.89	0.01\\
60.9	0.01\\
60.91	0.01\\
60.92	0.01\\
60.93	0.01\\
60.94	0.01\\
60.95	0.01\\
60.96	0.01\\
60.97	0.01\\
60.98	0.01\\
60.99	0.01\\
61	0.01\\
61.01	0.01\\
61.02	0.01\\
61.03	0.01\\
61.04	0.01\\
61.05	0.01\\
61.06	0.01\\
61.07	0.01\\
61.08	0.01\\
61.09	0.01\\
61.1	0.01\\
61.11	0.01\\
61.12	0.01\\
61.13	0.01\\
61.14	0.01\\
61.15	0.01\\
61.16	0.01\\
61.17	0.01\\
61.18	0.01\\
61.19	0.01\\
61.2	0.01\\
61.21	0.01\\
61.22	0.01\\
61.23	0.01\\
61.24	0.01\\
61.25	0.01\\
61.26	0.01\\
61.27	0.01\\
61.28	0.01\\
61.29	0.01\\
61.3	0.01\\
61.31	0.01\\
61.32	0.01\\
61.33	0.01\\
61.34	0.01\\
61.35	0.01\\
61.36	0.01\\
61.37	0.01\\
61.38	0.01\\
61.39	0.01\\
61.4	0.01\\
61.41	0.01\\
61.42	0.01\\
61.43	0.01\\
61.44	0.01\\
61.45	0.01\\
61.46	0.01\\
61.47	0.01\\
61.48	0.01\\
61.49	0.01\\
61.5	0.01\\
61.51	0.01\\
61.52	0.01\\
61.53	0.01\\
61.54	0.01\\
61.55	0.01\\
61.56	0.01\\
61.57	0.01\\
61.58	0.01\\
61.59	0.01\\
61.6	0.01\\
61.61	0.01\\
61.62	0.01\\
61.63	0.01\\
61.64	0.01\\
61.65	0.01\\
61.66	0.01\\
61.67	0.01\\
61.68	0.01\\
61.69	0.01\\
61.7	0.01\\
61.71	0.01\\
61.72	0.01\\
61.73	0.01\\
61.74	0.01\\
61.75	0.01\\
61.76	0.01\\
61.77	0.01\\
61.78	0.01\\
61.79	0.01\\
61.8	0.01\\
61.81	0.01\\
61.82	0.01\\
61.83	0.01\\
61.84	0.01\\
61.85	0.01\\
61.86	0.01\\
61.87	0.01\\
61.88	0.01\\
61.89	0.01\\
61.9	0.01\\
61.91	0.01\\
61.92	0.01\\
61.93	0.01\\
61.94	0.01\\
61.95	0.01\\
61.96	0.01\\
61.97	0.01\\
61.98	0.01\\
61.99	0.01\\
62	0.01\\
62.01	0.01\\
62.02	0.01\\
62.03	0.01\\
62.04	0.01\\
62.05	0.01\\
62.06	0.01\\
62.07	0.01\\
62.08	0.01\\
62.09	0.01\\
62.1	0.01\\
62.11	0.01\\
62.12	0.01\\
62.13	0.01\\
62.14	0.01\\
62.15	0.01\\
62.16	0.01\\
62.17	0.01\\
62.18	0.01\\
62.19	0.01\\
62.2	0.01\\
62.21	0.01\\
62.22	0.01\\
62.23	0.01\\
62.24	0.01\\
62.25	0.01\\
62.26	0.01\\
62.27	0.01\\
62.28	0.01\\
62.29	0.01\\
62.3	0.01\\
62.31	0.01\\
62.32	0.01\\
62.33	0.01\\
62.34	0.01\\
62.35	0.01\\
62.36	0.01\\
62.37	0.01\\
62.38	0.01\\
62.39	0.01\\
62.4	0.01\\
62.41	0.01\\
62.42	0.01\\
62.43	0.01\\
62.44	0.01\\
62.45	0.01\\
62.46	0.01\\
62.47	0.01\\
62.48	0.01\\
62.49	0.01\\
62.5	0.01\\
62.51	0.01\\
62.52	0.01\\
62.53	0.01\\
62.54	0.01\\
62.55	0.01\\
62.56	0.01\\
62.57	0.01\\
62.58	0.01\\
62.59	0.01\\
62.6	0.01\\
62.61	0.01\\
62.62	0.01\\
62.63	0.01\\
62.64	0.01\\
62.65	0.01\\
62.66	0.01\\
62.67	0.01\\
62.68	0.01\\
62.69	0.01\\
62.7	0.01\\
62.71	0.01\\
62.72	0.01\\
62.73	0.01\\
62.74	0.01\\
62.75	0.01\\
62.76	0.01\\
62.77	0.01\\
62.78	0.01\\
62.79	0.01\\
62.8	0.01\\
62.81	0.01\\
62.82	0.01\\
62.83	0.01\\
62.84	0.01\\
62.85	0.01\\
62.86	0.01\\
62.87	0.01\\
62.88	0.01\\
62.89	0.01\\
62.9	0.01\\
62.91	0.01\\
62.92	0.01\\
62.93	0.01\\
62.94	0.01\\
62.95	0.01\\
62.96	0.01\\
62.97	0.01\\
62.98	0.01\\
62.99	0.01\\
63	0.01\\
63.01	0.01\\
63.02	0.01\\
63.03	0.01\\
63.04	0.01\\
63.05	0.01\\
63.06	0.01\\
63.07	0.01\\
63.08	0.01\\
63.09	0.01\\
63.1	0.01\\
63.11	0.01\\
63.12	0.01\\
63.13	0.01\\
63.14	0.01\\
63.15	0.01\\
63.16	0.01\\
63.17	0.01\\
63.18	0.01\\
63.19	0.01\\
63.2	0.01\\
63.21	0.01\\
63.22	0.01\\
63.23	0.01\\
63.24	0.01\\
63.25	0.01\\
63.26	0.01\\
63.27	0.01\\
63.28	0.01\\
63.29	0.01\\
63.3	0.01\\
63.31	0.01\\
63.32	0.01\\
63.33	0.01\\
63.34	0.01\\
63.35	0.01\\
63.36	0.01\\
63.37	0.01\\
63.38	0.01\\
63.39	0.01\\
63.4	0.01\\
63.41	0.01\\
63.42	0.01\\
63.43	0.01\\
63.44	0.01\\
63.45	0.01\\
63.46	0.01\\
63.47	0.01\\
63.48	0.01\\
63.49	0.01\\
63.5	0.01\\
63.51	0.01\\
63.52	0.01\\
63.53	0.01\\
63.54	0.01\\
63.55	0.01\\
63.56	0.01\\
63.57	0.01\\
63.58	0.01\\
63.59	0.01\\
63.6	0.01\\
63.61	0.01\\
63.62	0.01\\
63.63	0.01\\
63.64	0.01\\
63.65	0.01\\
63.66	0.01\\
63.67	0.01\\
63.68	0.01\\
63.69	0.01\\
63.7	0.01\\
63.71	0.01\\
63.72	0.01\\
63.73	0.01\\
63.74	0.01\\
63.75	0.01\\
63.76	0.01\\
63.77	0.01\\
63.78	0.01\\
63.79	0.01\\
63.8	0.01\\
63.81	0.01\\
63.82	0.01\\
63.83	0.01\\
63.84	0.01\\
63.85	0.01\\
63.86	0.01\\
63.87	0.01\\
63.88	0.01\\
63.89	0.01\\
63.9	0.01\\
63.91	0.01\\
63.92	0.01\\
63.93	0.01\\
63.94	0.01\\
63.95	0.01\\
63.96	0.01\\
63.97	0.01\\
63.98	0.01\\
63.99	0.01\\
64	0.01\\
64.01	0.01\\
64.02	0.01\\
64.03	0.01\\
64.04	0.01\\
64.05	0.01\\
64.06	0.01\\
64.07	0.01\\
64.08	0.01\\
64.09	0.01\\
64.1	0.01\\
64.11	0.01\\
64.12	0.01\\
64.13	0.01\\
64.14	0.01\\
64.15	0.01\\
64.16	0.01\\
64.17	0.01\\
64.18	0.01\\
64.19	0.01\\
64.2	0.01\\
64.21	0.01\\
64.22	0.01\\
64.23	0.01\\
64.24	0.01\\
64.25	0.01\\
64.26	0.01\\
64.27	0.01\\
64.28	0.01\\
64.29	0.01\\
64.3	0.01\\
64.31	0.01\\
64.32	0.01\\
64.33	0.01\\
64.34	0.01\\
64.35	0.01\\
64.36	0.01\\
64.37	0.01\\
64.38	0.01\\
64.39	0.01\\
64.4	0.01\\
64.41	0.01\\
64.42	0.01\\
64.43	0.01\\
64.44	0.01\\
64.45	0.01\\
64.46	0.01\\
64.47	0.01\\
64.48	0.01\\
64.49	0.01\\
64.5	0.01\\
64.51	0.01\\
64.52	0.01\\
64.53	0.01\\
64.54	0.01\\
64.55	0.01\\
64.56	0.01\\
64.57	0.01\\
64.58	0.01\\
64.59	0.01\\
64.6	0.01\\
64.61	0.01\\
64.62	0.01\\
64.63	0.01\\
64.64	0.01\\
64.65	0.01\\
64.66	0.01\\
64.67	0.01\\
64.68	0.01\\
64.69	0.01\\
64.7	0.01\\
64.71	0.01\\
64.72	0.01\\
64.73	0.01\\
64.74	0.01\\
64.75	0.01\\
64.76	0.01\\
64.77	0.01\\
64.78	0.01\\
64.79	0.01\\
64.8	0.01\\
64.81	0.01\\
64.82	0.01\\
64.83	0.01\\
64.84	0.01\\
64.85	0.01\\
64.86	0.01\\
64.87	0.01\\
64.88	0.01\\
64.89	0.01\\
64.9	0.01\\
64.91	0.01\\
64.92	0.01\\
64.93	0.01\\
64.94	0.01\\
64.95	0.01\\
64.96	0.01\\
64.97	0.01\\
64.98	0.01\\
64.99	0.01\\
65	0.01\\
65.01	0.01\\
65.02	0.01\\
65.03	0.01\\
65.04	0.01\\
65.05	0.01\\
65.06	0.01\\
65.07	0.01\\
65.08	0.01\\
65.09	0.01\\
65.1	0.01\\
65.11	0.01\\
65.12	0.01\\
65.13	0.01\\
65.14	0.01\\
65.15	0.01\\
65.16	0.01\\
65.17	0.01\\
65.18	0.01\\
65.19	0.01\\
65.2	0.01\\
65.21	0.01\\
65.22	0.01\\
65.23	0.01\\
65.24	0.01\\
65.25	0.01\\
65.26	0.01\\
65.27	0.01\\
65.28	0.01\\
65.29	0.01\\
65.3	0.01\\
65.31	0.01\\
65.32	0.01\\
65.33	0.01\\
65.34	0.01\\
65.35	0.01\\
65.36	0.01\\
65.37	0.01\\
65.38	0.01\\
65.39	0.01\\
65.4	0.01\\
65.41	0.01\\
65.42	0.01\\
65.43	0.01\\
65.44	0.01\\
65.45	0.01\\
65.46	0.01\\
65.47	0.01\\
65.48	0.01\\
65.49	0.01\\
65.5	0.01\\
65.51	0.01\\
65.52	0.01\\
65.53	0.01\\
65.54	0.01\\
65.55	0.01\\
65.56	0.01\\
65.57	0.01\\
65.58	0.01\\
65.59	0.01\\
65.6	0.01\\
65.61	0.01\\
65.62	0.01\\
65.63	0.01\\
65.64	0.01\\
65.65	0.01\\
65.66	0.01\\
65.67	0.01\\
65.68	0.01\\
65.69	0.01\\
65.7	0.01\\
65.71	0.01\\
65.72	0.01\\
65.73	0.01\\
65.74	0.01\\
65.75	0.01\\
65.76	0.01\\
65.77	0.01\\
65.78	0.01\\
65.79	0.01\\
65.8	0.01\\
65.81	0.01\\
65.82	0.01\\
65.83	0.01\\
65.84	0.01\\
65.85	0.01\\
65.86	0.01\\
65.87	0.01\\
65.88	0.01\\
65.89	0.01\\
65.9	0.01\\
65.91	0.01\\
65.92	0.01\\
65.93	0.01\\
65.94	0.01\\
65.95	0.01\\
65.96	0.01\\
65.97	0.01\\
65.98	0.01\\
65.99	0.01\\
66	0.01\\
66.01	0.01\\
66.02	0.01\\
66.03	0.01\\
66.04	0.01\\
66.05	0.01\\
66.06	0.01\\
66.07	0.01\\
66.08	0.01\\
66.09	0.01\\
66.1	0.01\\
66.11	0.01\\
66.12	0.01\\
66.13	0.01\\
66.14	0.01\\
66.15	0.01\\
66.16	0.01\\
66.17	0.01\\
66.18	0.01\\
66.19	0.01\\
66.2	0.01\\
66.21	0.01\\
66.22	0.01\\
66.23	0.01\\
66.24	0.01\\
66.25	0.01\\
66.26	0.01\\
66.27	0.01\\
66.28	0.01\\
66.29	0.01\\
66.3	0.01\\
66.31	0.01\\
66.32	0.01\\
66.33	0.01\\
66.34	0.01\\
66.35	0.01\\
66.36	0.01\\
66.37	0.01\\
66.38	0.01\\
66.39	0.01\\
66.4	0.01\\
66.41	0.01\\
66.42	0.01\\
66.43	0.01\\
66.44	0.01\\
66.45	0.01\\
66.46	0.01\\
66.47	0.01\\
66.48	0.01\\
66.49	0.01\\
66.5	0.01\\
66.51	0.01\\
66.52	0.01\\
66.53	0.01\\
66.54	0.01\\
66.55	0.01\\
66.56	0.01\\
66.57	0.01\\
66.58	0.01\\
66.59	0.01\\
66.6	0.01\\
66.61	0.01\\
66.62	0.01\\
66.63	0.01\\
66.64	0.01\\
66.65	0.01\\
66.66	0.01\\
66.67	0.01\\
66.68	0.01\\
66.69	0.01\\
66.7	0.01\\
66.71	0.01\\
66.72	0.01\\
66.73	0.01\\
66.74	0.01\\
66.75	0.01\\
66.76	0.01\\
66.77	0.01\\
66.78	0.01\\
66.79	0.01\\
66.8	0.01\\
66.81	0.01\\
66.82	0.01\\
66.83	0.01\\
66.84	0.01\\
66.85	0.01\\
66.86	0.01\\
66.87	0.01\\
66.88	0.01\\
66.89	0.01\\
66.9	0.01\\
66.91	0.01\\
66.92	0.01\\
66.93	0.01\\
66.94	0.01\\
66.95	0.01\\
66.96	0.01\\
66.97	0.01\\
66.98	0.01\\
66.99	0.01\\
67	0.01\\
67.01	0.01\\
67.02	0.01\\
67.03	0.01\\
67.04	0.01\\
67.05	0.01\\
67.06	0.01\\
67.07	0.01\\
67.08	0.01\\
67.09	0.01\\
67.1	0.01\\
67.11	0.01\\
67.12	0.01\\
67.13	0.01\\
67.14	0.01\\
67.15	0.01\\
67.16	0.01\\
67.17	0.01\\
67.18	0.01\\
67.19	0.01\\
67.2	0.01\\
67.21	0.01\\
67.22	0.01\\
67.23	0.01\\
67.24	0.01\\
67.25	0.01\\
67.26	0.01\\
67.27	0.01\\
67.28	0.01\\
67.29	0.01\\
67.3	0.01\\
67.31	0.01\\
67.32	0.01\\
67.33	0.01\\
67.34	0.01\\
67.35	0.01\\
67.36	0.01\\
67.37	0.01\\
67.38	0.01\\
67.39	0.01\\
67.4	0.01\\
67.41	0.01\\
67.42	0.01\\
67.43	0.01\\
67.44	0.01\\
67.45	0.01\\
67.46	0.01\\
67.47	0.01\\
67.48	0.01\\
67.49	0.01\\
67.5	0.01\\
67.51	0.01\\
67.52	0.01\\
67.53	0.01\\
67.54	0.01\\
67.55	0.01\\
67.56	0.01\\
67.57	0.01\\
67.58	0.01\\
67.59	0.01\\
67.6	0.01\\
67.61	0.01\\
67.62	0.01\\
67.63	0.01\\
67.64	0.01\\
67.65	0.01\\
67.66	0.01\\
67.67	0.01\\
67.68	0.01\\
67.69	0.01\\
67.7	0.01\\
67.71	0.01\\
67.72	0.01\\
67.73	0.01\\
67.74	0.01\\
67.75	0.01\\
67.76	0.01\\
67.77	0.01\\
67.78	0.01\\
67.79	0.01\\
67.8	0.01\\
67.81	0.01\\
67.82	0.01\\
67.83	0.01\\
67.84	0.01\\
67.85	0.01\\
67.86	0.01\\
67.87	0.01\\
67.88	0.01\\
67.89	0.01\\
67.9	0.01\\
67.91	0.01\\
67.92	0.01\\
67.93	0.01\\
67.94	0.01\\
67.95	0.01\\
67.96	0.01\\
67.97	0.01\\
67.98	0.01\\
67.99	0.01\\
68	0.01\\
68.01	0.01\\
68.02	0.01\\
68.03	0.01\\
68.04	0.01\\
68.05	0.01\\
68.06	0.01\\
68.07	0.01\\
68.08	0.01\\
68.09	0.01\\
68.1	0.01\\
68.11	0.01\\
68.12	0.01\\
68.13	0.01\\
68.14	0.01\\
68.15	0.01\\
68.16	0.01\\
68.17	0.01\\
68.18	0.01\\
68.19	0.01\\
68.2	0.01\\
68.21	0.01\\
68.22	0.01\\
68.23	0.01\\
68.24	0.01\\
68.25	0.01\\
68.26	0.01\\
68.27	0.01\\
68.28	0.01\\
68.29	0.01\\
68.3	0.01\\
68.31	0.01\\
68.32	0.01\\
68.33	0.01\\
68.34	0.01\\
68.35	0.01\\
68.36	0.01\\
68.37	0.01\\
68.38	0.01\\
68.39	0.01\\
68.4	0.01\\
68.41	0.01\\
68.42	0.01\\
68.43	0.01\\
68.44	0.01\\
68.45	0.01\\
68.46	0.01\\
68.47	0.01\\
68.48	0.01\\
68.49	0.01\\
68.5	0.01\\
68.51	0.01\\
68.52	0.01\\
68.53	0.01\\
68.54	0.01\\
68.55	0.01\\
68.56	0.01\\
68.57	0.01\\
68.58	0.01\\
68.59	0.01\\
68.6	0.01\\
68.61	0.01\\
68.62	0.01\\
68.63	0.01\\
68.64	0.01\\
68.65	0.01\\
68.66	0.01\\
68.67	0.01\\
68.68	0.01\\
68.69	0.01\\
68.7	0.01\\
68.71	0.01\\
68.72	0.01\\
68.73	0.01\\
68.74	0.01\\
68.75	0.01\\
68.76	0.01\\
68.77	0.01\\
68.78	0.01\\
68.79	0.01\\
68.8	0.01\\
68.81	0.01\\
68.82	0.01\\
68.83	0.01\\
68.84	0.01\\
68.85	0.01\\
68.86	0.01\\
68.87	0.01\\
68.88	0.01\\
68.89	0.01\\
68.9	0.01\\
68.91	0.01\\
68.92	0.01\\
68.93	0.01\\
68.94	0.01\\
68.95	0.01\\
68.96	0.01\\
68.97	0.01\\
68.98	0.01\\
68.99	0.01\\
69	0.01\\
69.01	0.01\\
69.02	0.01\\
69.03	0.01\\
69.04	0.01\\
69.05	0.01\\
69.06	0.01\\
69.07	0.01\\
69.08	0.01\\
69.09	0.01\\
69.1	0.01\\
69.11	0.01\\
69.12	0.01\\
69.13	0.01\\
69.14	0.01\\
69.15	0.01\\
69.16	0.01\\
69.17	0.01\\
69.18	0.01\\
69.19	0.01\\
69.2	0.01\\
69.21	0.01\\
69.22	0.01\\
69.23	0.01\\
69.24	0.01\\
69.25	0.01\\
69.26	0.01\\
69.27	0.01\\
69.28	0.01\\
69.29	0.01\\
69.3	0.01\\
69.31	0.01\\
69.32	0.01\\
69.33	0.01\\
69.34	0.01\\
69.35	0.01\\
69.36	0.01\\
69.37	0.01\\
69.38	0.01\\
69.39	0.01\\
69.4	0.01\\
69.41	0.01\\
69.42	0.01\\
69.43	0.01\\
69.44	0.01\\
69.45	0.01\\
69.46	0.01\\
69.47	0.01\\
69.48	0.01\\
69.49	0.01\\
69.5	0.01\\
69.51	0.01\\
69.52	0.01\\
69.53	0.01\\
69.54	0.01\\
69.55	0.01\\
69.56	0.01\\
69.57	0.01\\
69.58	0.01\\
69.59	0.01\\
69.6	0.01\\
69.61	0.01\\
69.62	0.01\\
69.63	0.01\\
69.64	0.01\\
69.65	0.01\\
69.66	0.01\\
69.67	0.01\\
69.68	0.01\\
69.69	0.01\\
69.7	0.01\\
69.71	0.01\\
69.72	0.01\\
69.73	0.01\\
69.74	0.01\\
69.75	0.01\\
69.76	0.01\\
69.77	0.01\\
69.78	0.01\\
69.79	0.01\\
69.8	0.01\\
69.81	0.01\\
69.82	0.01\\
69.83	0.01\\
69.84	0.01\\
69.85	0.01\\
69.86	0.01\\
69.87	0.01\\
69.88	0.01\\
69.89	0.01\\
69.9	0.01\\
69.91	0.01\\
69.92	0.01\\
69.93	0.01\\
69.94	0.01\\
69.95	0.01\\
69.96	0.01\\
69.97	0.01\\
69.98	0.01\\
69.99	0.01\\
70	0.01\\
70.01	0.01\\
70.02	0.01\\
70.03	0.01\\
70.04	0.01\\
70.05	0.01\\
70.06	0.01\\
70.07	0.01\\
70.08	0.01\\
70.09	0.01\\
70.1	0.01\\
70.11	0.01\\
70.12	0.01\\
70.13	0.01\\
70.14	0.01\\
70.15	0.01\\
70.16	0.01\\
70.17	0.01\\
70.18	0.01\\
70.19	0.01\\
70.2	0.01\\
70.21	0.01\\
70.22	0.01\\
70.23	0.01\\
70.24	0.01\\
70.25	0.01\\
70.26	0.01\\
70.27	0.01\\
70.28	0.01\\
70.29	0.01\\
70.3	0.01\\
70.31	0.01\\
70.32	0.01\\
70.33	0.01\\
70.34	0.01\\
70.35	0.01\\
70.36	0.01\\
70.37	0.01\\
70.38	0.01\\
70.39	0.01\\
70.4	0.01\\
70.41	0.01\\
70.42	0.01\\
70.43	0.01\\
70.44	0.01\\
70.45	0.01\\
70.46	0.01\\
70.47	0.01\\
70.48	0.01\\
70.49	0.01\\
70.5	0.01\\
70.51	0.01\\
70.52	0.01\\
70.53	0.01\\
70.54	0.01\\
70.55	0.01\\
70.56	0.01\\
70.57	0.01\\
70.58	0.01\\
70.59	0.01\\
70.6	0.01\\
70.61	0.01\\
70.62	0.01\\
70.63	0.01\\
70.64	0.01\\
70.65	0.01\\
70.66	0.01\\
70.67	0.01\\
70.68	0.01\\
70.69	0.01\\
70.7	0.01\\
70.71	0.01\\
70.72	0.01\\
70.73	0.01\\
70.74	0.01\\
70.75	0.01\\
70.76	0.01\\
70.77	0.01\\
70.78	0.01\\
70.79	0.01\\
70.8	0.01\\
70.81	0.01\\
70.82	0.01\\
70.83	0.01\\
70.84	0.01\\
70.85	0.01\\
70.86	0.01\\
70.87	0.01\\
70.88	0.01\\
70.89	0.01\\
70.9	0.01\\
70.91	0.01\\
70.92	0.01\\
70.93	0.01\\
70.94	0.01\\
70.95	0.01\\
70.96	0.01\\
70.97	0.01\\
70.98	0.01\\
70.99	0.01\\
71	0.01\\
71.01	0.01\\
71.02	0.01\\
71.03	0.01\\
71.04	0.01\\
71.05	0.01\\
71.06	0.01\\
71.07	0.01\\
71.08	0.01\\
71.09	0.01\\
71.1	0.01\\
71.11	0.01\\
71.12	0.01\\
71.13	0.01\\
71.14	0.01\\
71.15	0.01\\
71.16	0.01\\
71.17	0.01\\
71.18	0.01\\
71.19	0.01\\
71.2	0.01\\
71.21	0.01\\
71.22	0.01\\
71.23	0.01\\
71.24	0.01\\
71.25	0.01\\
71.26	0.01\\
71.27	0.01\\
71.28	0.01\\
71.29	0.01\\
71.3	0.01\\
71.31	0.01\\
71.32	0.01\\
71.33	0.01\\
71.34	0.01\\
71.35	0.01\\
71.36	0.01\\
71.37	0.01\\
71.38	0.01\\
71.39	0.01\\
71.4	0.01\\
71.41	0.01\\
71.42	0.01\\
71.43	0.01\\
71.44	0.01\\
71.45	0.01\\
71.46	0.01\\
71.47	0.01\\
71.48	0.01\\
71.49	0.01\\
71.5	0.01\\
71.51	0.01\\
71.52	0.01\\
71.53	0.01\\
71.54	0.01\\
71.55	0.01\\
71.56	0.01\\
71.57	0.01\\
71.58	0.01\\
71.59	0.01\\
71.6	0.01\\
71.61	0.01\\
71.62	0.01\\
71.63	0.01\\
71.64	0.01\\
71.65	0.01\\
71.66	0.01\\
71.67	0.01\\
71.68	0.01\\
71.69	0.01\\
71.7	0.01\\
71.71	0.01\\
71.72	0.01\\
71.73	0.01\\
71.74	0.01\\
71.75	0.01\\
71.76	0.01\\
71.77	0.01\\
71.78	0.01\\
71.79	0.01\\
71.8	0.01\\
71.81	0.01\\
71.82	0.01\\
71.83	0.01\\
71.84	0.01\\
71.85	0.01\\
71.86	0.01\\
71.87	0.01\\
71.88	0.01\\
71.89	0.01\\
71.9	0.01\\
71.91	0.01\\
71.92	0.01\\
71.93	0.01\\
71.94	0.01\\
71.95	0.01\\
71.96	0.01\\
71.97	0.01\\
71.98	0.01\\
71.99	0.01\\
72	0.01\\
72.01	0.01\\
72.02	0.01\\
72.03	0.01\\
72.04	0.01\\
72.05	0.01\\
72.06	0.01\\
72.07	0.01\\
72.08	0.01\\
72.09	0.01\\
72.1	0.01\\
72.11	0.01\\
72.12	0.01\\
72.13	0.01\\
72.14	0.01\\
72.15	0.01\\
72.16	0.01\\
72.17	0.01\\
72.18	0.01\\
72.19	0.01\\
72.2	0.01\\
72.21	0.01\\
72.22	0.01\\
72.23	0.01\\
72.24	0.01\\
72.25	0.01\\
72.26	0.01\\
72.27	0.01\\
72.28	0.01\\
72.29	0.01\\
72.3	0.01\\
72.31	0.01\\
72.32	0.01\\
72.33	0.01\\
72.34	0.01\\
72.35	0.01\\
72.36	0.01\\
72.37	0.01\\
72.38	0.01\\
72.39	0.01\\
72.4	0.01\\
72.41	0.01\\
72.42	0.01\\
72.43	0.01\\
72.44	0.01\\
72.45	0.01\\
72.46	0.01\\
72.47	0.01\\
72.48	0.01\\
72.49	0.01\\
72.5	0.01\\
72.51	0.01\\
72.52	0.01\\
72.53	0.01\\
72.54	0.01\\
72.55	0.01\\
72.56	0.01\\
72.57	0.01\\
72.58	0.01\\
72.59	0.01\\
72.6	0.01\\
72.61	0.01\\
72.62	0.01\\
72.63	0.01\\
72.64	0.01\\
72.65	0.01\\
72.66	0.01\\
72.67	0.01\\
72.68	0.01\\
72.69	0.01\\
72.7	0.01\\
72.71	0.01\\
72.72	0.01\\
72.73	0.01\\
72.74	0.01\\
72.75	0.01\\
72.76	0.01\\
72.77	0.01\\
72.78	0.01\\
72.79	0.01\\
72.8	0.01\\
72.81	0.01\\
72.82	0.01\\
72.83	0.01\\
72.84	0.01\\
72.85	0.01\\
72.86	0.01\\
72.87	0.01\\
72.88	0.01\\
72.89	0.01\\
72.9	0.01\\
72.91	0.01\\
72.92	0.01\\
72.93	0.01\\
72.94	0.01\\
72.95	0.01\\
72.96	0.01\\
72.97	0.01\\
72.98	0.01\\
72.99	0.01\\
73	0.01\\
73.01	0.01\\
73.02	0.01\\
73.03	0.01\\
73.04	0.01\\
73.05	0.01\\
73.06	0.01\\
73.07	0.01\\
73.08	0.01\\
73.09	0.01\\
73.1	0.01\\
73.11	0.01\\
73.12	0.01\\
73.13	0.01\\
73.14	0.01\\
73.15	0.01\\
73.16	0.01\\
73.17	0.01\\
73.18	0.01\\
73.19	0.01\\
73.2	0.01\\
73.21	0.01\\
73.22	0.01\\
73.23	0.01\\
73.24	0.01\\
73.25	0.01\\
73.26	0.01\\
73.27	0.01\\
73.28	0.01\\
73.29	0.01\\
73.3	0.01\\
73.31	0.01\\
73.32	0.01\\
73.33	0.01\\
73.34	0.01\\
73.35	0.01\\
73.36	0.01\\
73.37	0.01\\
73.38	0.01\\
73.39	0.01\\
73.4	0.01\\
73.41	0.01\\
73.42	0.01\\
73.43	0.01\\
73.44	0.01\\
73.45	0.01\\
73.46	0.01\\
73.47	0.01\\
73.48	0.01\\
73.49	0.01\\
73.5	0.01\\
73.51	0.01\\
73.52	0.01\\
73.53	0.01\\
73.54	0.01\\
73.55	0.01\\
73.56	0.01\\
73.57	0.01\\
73.58	0.01\\
73.59	0.01\\
73.6	0.01\\
73.61	0.01\\
73.62	0.01\\
73.63	0.01\\
73.64	0.01\\
73.65	0.01\\
73.66	0.01\\
73.67	0.01\\
73.68	0.01\\
73.69	0.01\\
73.7	0.01\\
73.71	0.01\\
73.72	0.01\\
73.73	0.01\\
73.74	0.01\\
73.75	0.01\\
73.76	0.01\\
73.77	0.01\\
73.78	0.01\\
73.79	0.01\\
73.8	0.01\\
73.81	0.01\\
73.82	0.01\\
73.83	0.01\\
73.84	0.01\\
73.85	0.01\\
73.86	0.01\\
73.87	0.01\\
73.88	0.01\\
73.89	0.01\\
73.9	0.01\\
73.91	0.01\\
73.92	0.01\\
73.93	0.01\\
73.94	0.01\\
73.95	0.01\\
73.96	0.01\\
73.97	0.01\\
73.98	0.01\\
73.99	0.01\\
74	0.01\\
74.01	0.01\\
74.02	0.01\\
74.03	0.01\\
74.04	0.01\\
74.05	0.01\\
74.06	0.01\\
74.07	0.01\\
74.08	0.01\\
74.09	0.01\\
74.1	0.01\\
74.11	0.01\\
74.12	0.01\\
74.13	0.01\\
74.14	0.01\\
74.15	0.01\\
74.16	0.01\\
74.17	0.01\\
74.18	0.01\\
74.19	0.01\\
74.2	0.01\\
74.21	0.01\\
74.22	0.01\\
74.23	0.01\\
74.24	0.01\\
74.25	0.01\\
74.26	0.01\\
74.27	0.01\\
74.28	0.01\\
74.29	0.01\\
74.3	0.01\\
74.31	0.01\\
74.32	0.01\\
74.33	0.01\\
74.34	0.01\\
74.35	0.01\\
74.36	0.01\\
74.37	0.01\\
74.38	0.01\\
74.39	0.01\\
74.4	0.01\\
74.41	0.01\\
74.42	0.01\\
74.43	0.01\\
74.44	0.01\\
74.45	0.01\\
74.46	0.01\\
74.47	0.01\\
74.48	0.01\\
74.49	0.01\\
74.5	0.01\\
74.51	0.01\\
74.52	0.01\\
74.53	0.01\\
74.54	0.01\\
74.55	0.01\\
74.56	0.01\\
74.57	0.01\\
74.58	0.01\\
74.59	0.01\\
74.6	0.01\\
74.61	0.01\\
74.62	0.01\\
74.63	0.01\\
74.64	0.01\\
74.65	0.01\\
74.66	0.01\\
74.67	0.01\\
74.68	0.01\\
74.69	0.01\\
74.7	0.01\\
74.71	0.01\\
74.72	0.01\\
74.73	0.01\\
74.74	0.01\\
74.75	0.01\\
74.76	0.01\\
74.77	0.01\\
74.78	0.01\\
74.79	0.01\\
74.8	0.01\\
74.81	0.01\\
74.82	0.01\\
74.83	0.01\\
74.84	0.01\\
74.85	0.01\\
74.86	0.01\\
74.87	0.01\\
74.88	0.01\\
74.89	0.01\\
74.9	0.01\\
74.91	0.01\\
74.92	0.01\\
74.93	0.01\\
74.94	0.01\\
74.95	0.01\\
74.96	0.01\\
74.97	0.01\\
74.98	0.01\\
74.99	0.01\\
75	0.01\\
75.01	0.01\\
75.02	0.01\\
75.03	0.01\\
75.04	0.01\\
75.05	0.01\\
75.06	0.01\\
75.07	0.01\\
75.08	0.01\\
75.09	0.01\\
75.1	0.01\\
75.11	0.01\\
75.12	0.01\\
75.13	0.01\\
75.14	0.01\\
75.15	0.01\\
75.16	0.01\\
75.17	0.01\\
75.18	0.01\\
75.19	0.01\\
75.2	0.01\\
75.21	0.01\\
75.22	0.01\\
75.23	0.01\\
75.24	0.01\\
75.25	0.01\\
75.26	0.01\\
75.27	0.01\\
75.28	0.01\\
75.29	0.01\\
75.3	0.01\\
75.31	0.01\\
75.32	0.01\\
75.33	0.01\\
75.34	0.01\\
75.35	0.01\\
75.36	0.01\\
75.37	0.01\\
75.38	0.01\\
75.39	0.01\\
75.4	0.01\\
75.41	0.01\\
75.42	0.01\\
75.43	0.01\\
75.44	0.01\\
75.45	0.01\\
75.46	0.01\\
75.47	0.01\\
75.48	0.01\\
75.49	0.01\\
75.5	0.01\\
75.51	0.01\\
75.52	0.01\\
75.53	0.01\\
75.54	0.01\\
75.55	0.01\\
75.56	0.01\\
75.57	0.01\\
75.58	0.01\\
75.59	0.01\\
75.6	0.01\\
75.61	0.01\\
75.62	0.01\\
75.63	0.01\\
75.64	0.01\\
75.65	0.01\\
75.66	0.01\\
75.67	0.01\\
75.68	0.01\\
75.69	0.01\\
75.7	0.01\\
75.71	0.01\\
75.72	0.01\\
75.73	0.01\\
75.74	0.01\\
75.75	0.01\\
75.76	0.01\\
75.77	0.01\\
75.78	0.01\\
75.79	0.01\\
75.8	0.01\\
75.81	0.01\\
75.82	0.01\\
75.83	0.01\\
75.84	0.01\\
75.85	0.01\\
75.86	0.01\\
75.87	0.01\\
75.88	0.01\\
75.89	0.01\\
75.9	0.01\\
75.91	0.01\\
75.92	0.01\\
75.93	0.01\\
75.94	0.01\\
75.95	0.01\\
75.96	0.01\\
75.97	0.01\\
75.98	0.01\\
75.99	0.01\\
76	0.01\\
76.01	0.01\\
76.02	0.01\\
76.03	0.01\\
76.04	0.01\\
76.05	0.01\\
76.06	0.01\\
76.07	0.01\\
76.08	0.01\\
76.09	0.01\\
76.1	0.01\\
76.11	0.01\\
76.12	0.01\\
76.13	0.01\\
76.14	0.01\\
76.15	0.01\\
76.16	0.01\\
76.17	0.01\\
76.18	0.01\\
76.19	0.01\\
76.2	0.01\\
76.21	0.01\\
76.22	0.01\\
76.23	0.01\\
76.24	0.01\\
76.25	0.01\\
76.26	0.01\\
76.27	0.01\\
76.28	0.01\\
76.29	0.01\\
76.3	0.01\\
76.31	0.01\\
76.32	0.01\\
76.33	0.01\\
76.34	0.01\\
76.35	0.01\\
76.36	0.01\\
76.37	0.01\\
76.38	0.01\\
76.39	0.01\\
76.4	0.01\\
76.41	0.01\\
76.42	0.01\\
76.43	0.01\\
76.44	0.01\\
76.45	0.01\\
76.46	0.01\\
76.47	0.01\\
76.48	0.01\\
76.49	0.01\\
76.5	0.01\\
76.51	0.01\\
76.52	0.01\\
76.53	0.01\\
76.54	0.01\\
76.55	0.01\\
76.56	0.01\\
76.57	0.01\\
76.58	0.01\\
76.59	0.01\\
76.6	0.01\\
76.61	0.01\\
76.62	0.01\\
76.63	0.01\\
76.64	0.01\\
76.65	0.01\\
76.66	0.01\\
76.67	0.01\\
76.68	0.01\\
76.69	0.01\\
76.7	0.01\\
76.71	0.01\\
76.72	0.01\\
76.73	0.01\\
76.74	0.01\\
76.75	0.01\\
76.76	0.01\\
76.77	0.01\\
76.78	0.01\\
76.79	0.01\\
76.8	0.01\\
76.81	0.01\\
76.82	0.01\\
76.83	0.01\\
76.84	0.01\\
76.85	0.01\\
76.86	0.01\\
76.87	0.01\\
76.88	0.01\\
76.89	0.01\\
76.9	0.01\\
76.91	0.01\\
76.92	0.01\\
76.93	0.01\\
76.94	0.01\\
76.95	0.01\\
76.96	0.01\\
76.97	0.01\\
76.98	0.01\\
76.99	0.01\\
77	0.01\\
77.01	0.01\\
77.02	0.01\\
77.03	0.01\\
77.04	0.01\\
77.05	0.01\\
77.06	0.01\\
77.07	0.01\\
77.08	0.01\\
77.09	0.01\\
77.1	0.01\\
77.11	0.01\\
77.12	0.01\\
77.13	0.01\\
77.14	0.01\\
77.15	0.01\\
77.16	0.01\\
77.17	0.01\\
77.18	0.01\\
77.19	0.01\\
77.2	0.01\\
77.21	0.01\\
77.22	0.01\\
77.23	0.01\\
77.24	0.01\\
77.25	0.01\\
77.26	0.01\\
77.27	0.01\\
77.28	0.01\\
77.29	0.01\\
77.3	0.01\\
77.31	0.01\\
77.32	0.01\\
77.33	0.01\\
77.34	0.01\\
77.35	0.01\\
77.36	0.01\\
77.37	0.01\\
77.38	0.01\\
77.39	0.01\\
77.4	0.01\\
77.41	0.01\\
77.42	0.01\\
77.43	0.01\\
77.44	0.01\\
77.45	0.01\\
77.46	0.01\\
77.47	0.01\\
77.48	0.01\\
77.49	0.01\\
77.5	0.01\\
77.51	0.01\\
77.52	0.01\\
77.53	0.01\\
77.54	0.01\\
77.55	0.01\\
77.56	0.01\\
77.57	0.01\\
77.58	0.01\\
77.59	0.01\\
77.6	0.01\\
77.61	0.01\\
77.62	0.01\\
77.63	0.01\\
77.64	0.01\\
77.65	0.01\\
77.66	0.01\\
77.67	0.01\\
77.68	0.01\\
77.69	0.01\\
77.7	0.01\\
77.71	0.01\\
77.72	0.01\\
77.73	0.01\\
77.74	0.01\\
77.75	0.01\\
77.76	0.01\\
77.77	0.01\\
77.78	0.01\\
77.79	0.01\\
77.8	0.01\\
77.81	0.01\\
77.82	0.01\\
77.83	0.01\\
77.84	0.01\\
77.85	0.01\\
77.86	0.01\\
77.87	0.01\\
77.88	0.01\\
77.89	0.01\\
77.9	0.01\\
77.91	0.01\\
77.92	0.01\\
77.93	0.01\\
77.94	0.01\\
77.95	0.01\\
77.96	0.01\\
77.97	0.01\\
77.98	0.01\\
77.99	0.01\\
78	0.01\\
78.01	0.01\\
78.02	0.01\\
78.03	0.01\\
78.04	0.01\\
78.05	0.01\\
78.06	0.01\\
78.07	0.01\\
78.08	0.01\\
78.09	0.01\\
78.1	0.01\\
78.11	0.01\\
78.12	0.01\\
78.13	0.01\\
78.14	0.01\\
78.15	0.01\\
78.16	0.01\\
78.17	0.01\\
78.18	0.01\\
78.19	0.01\\
78.2	0.01\\
78.21	0.01\\
78.22	0.01\\
78.23	0.01\\
78.24	0.01\\
78.25	0.01\\
78.26	0.01\\
78.27	0.01\\
78.28	0.01\\
78.29	0.01\\
78.3	0.01\\
78.31	0.01\\
78.32	0.01\\
78.33	0.01\\
78.34	0.01\\
78.35	0.01\\
78.36	0.01\\
78.37	0.01\\
78.38	0.01\\
78.39	0.01\\
78.4	0.01\\
78.41	0.01\\
78.42	0.01\\
78.43	0.01\\
78.44	0.01\\
78.45	0.01\\
78.46	0.01\\
78.47	0.01\\
78.48	0.01\\
78.49	0.01\\
78.5	0.01\\
78.51	0.01\\
78.52	0.01\\
78.53	0.01\\
78.54	0.01\\
78.55	0.01\\
78.56	0.01\\
78.57	0.01\\
78.58	0.01\\
78.59	0.01\\
78.6	0.01\\
78.61	0.01\\
78.62	0.01\\
78.63	0.01\\
78.64	0.01\\
78.65	0.01\\
78.66	0.01\\
78.67	0.01\\
78.68	0.01\\
78.69	0.01\\
78.7	0.01\\
78.71	0.01\\
78.72	0.01\\
78.73	0.01\\
78.74	0.01\\
78.75	0.01\\
78.76	0.01\\
78.77	0.01\\
78.78	0.01\\
78.79	0.01\\
78.8	0.01\\
78.81	0.01\\
78.82	0.01\\
78.83	0.01\\
78.84	0.01\\
78.85	0.01\\
78.86	0.01\\
78.87	0.01\\
78.88	0.01\\
78.89	0.01\\
78.9	0.01\\
78.91	0.01\\
78.92	0.01\\
78.93	0.01\\
78.94	0.01\\
78.95	0.01\\
78.96	0.01\\
78.97	0.01\\
78.98	0.01\\
78.99	0.01\\
79	0.01\\
79.01	0.01\\
79.02	0.01\\
79.03	0.01\\
79.04	0.01\\
79.05	0.01\\
79.06	0.01\\
79.07	0.01\\
79.08	0.01\\
79.09	0.01\\
79.1	0.01\\
79.11	0.01\\
79.12	0.01\\
79.13	0.01\\
79.14	0.01\\
79.15	0.01\\
79.16	0.01\\
79.17	0.01\\
79.18	0.01\\
79.19	0.01\\
79.2	0.01\\
79.21	0.01\\
79.22	0.01\\
79.23	0.01\\
79.24	0.01\\
79.25	0.01\\
79.26	0.01\\
79.27	0.01\\
79.28	0.01\\
79.29	0.01\\
79.3	0.01\\
79.31	0.01\\
79.32	0.01\\
79.33	0.01\\
79.34	0.01\\
79.35	0.01\\
79.36	0.01\\
79.37	0.01\\
79.38	0.01\\
79.39	0.01\\
79.4	0.01\\
79.41	0.01\\
79.42	0.01\\
79.43	0.01\\
79.44	0.01\\
79.45	0.01\\
79.46	0.01\\
79.47	0.01\\
79.48	0.01\\
79.49	0.01\\
79.5	0.01\\
79.51	0.01\\
79.52	0.01\\
79.53	0.01\\
79.54	0.01\\
79.55	0.01\\
79.56	0.01\\
79.57	0.01\\
79.58	0.01\\
79.59	0.01\\
79.6	0.01\\
79.61	0.01\\
79.62	0.01\\
79.63	0.01\\
79.64	0.01\\
79.65	0.01\\
79.66	0.01\\
79.67	0.01\\
79.68	0.01\\
79.69	0.01\\
79.7	0.01\\
79.71	0.01\\
79.72	0.01\\
79.73	0.01\\
79.74	0.01\\
79.75	0.01\\
79.76	0.01\\
79.77	0.01\\
79.78	0.01\\
79.79	0.01\\
79.8	0.01\\
79.81	0.01\\
79.82	0.01\\
79.83	0.01\\
79.84	0.01\\
79.85	0.01\\
79.86	0.01\\
79.87	0.01\\
79.88	0.01\\
79.89	0.01\\
79.9	0.01\\
79.91	0.01\\
79.92	0.01\\
79.93	0.01\\
79.94	0.01\\
79.95	0.01\\
79.96	0.01\\
79.97	0.01\\
79.98	0.01\\
79.99	0.01\\
80	0.01\\
80.01	0.01\\
};
\addplot [color=green,dashed]
  table[row sep=crcr]{%
80.01	0.01\\
80.02	0.01\\
80.03	0.01\\
80.04	0.01\\
80.05	0.01\\
80.06	0.01\\
80.07	0.01\\
80.08	0.01\\
80.09	0.01\\
80.1	0.01\\
80.11	0.01\\
80.12	0.01\\
80.13	0.01\\
80.14	0.01\\
80.15	0.01\\
80.16	0.01\\
80.17	0.01\\
80.18	0.01\\
80.19	0.01\\
80.2	0.01\\
80.21	0.01\\
80.22	0.01\\
80.23	0.01\\
80.24	0.01\\
80.25	0.01\\
80.26	0.01\\
80.27	0.01\\
80.28	0.01\\
80.29	0.01\\
80.3	0.01\\
80.31	0.01\\
80.32	0.01\\
80.33	0.01\\
80.34	0.01\\
80.35	0.01\\
80.36	0.01\\
80.37	0.01\\
80.38	0.01\\
80.39	0.01\\
80.4	0.01\\
80.41	0.01\\
80.42	0.01\\
80.43	0.01\\
80.44	0.01\\
80.45	0.01\\
80.46	0.01\\
80.47	0.01\\
80.48	0.01\\
80.49	0.01\\
80.5	0.01\\
80.51	0.01\\
80.52	0.01\\
80.53	0.01\\
80.54	0.01\\
80.55	0.01\\
80.56	0.01\\
80.57	0.01\\
80.58	0.01\\
80.59	0.01\\
80.6	0.01\\
80.61	0.01\\
80.62	0.01\\
80.63	0.01\\
80.64	0.01\\
80.65	0.01\\
80.66	0.01\\
80.67	0.01\\
80.68	0.01\\
80.69	0.01\\
80.7	0.01\\
80.71	0.01\\
80.72	0.01\\
80.73	0.01\\
80.74	0.01\\
80.75	0.01\\
80.76	0.01\\
80.77	0.01\\
80.78	0.01\\
80.79	0.01\\
80.8	0.01\\
80.81	0.01\\
80.82	0.01\\
80.83	0.01\\
80.84	0.01\\
80.85	0.01\\
80.86	0.01\\
80.87	0.01\\
80.88	0.01\\
80.89	0.01\\
80.9	0.01\\
80.91	0.01\\
80.92	0.01\\
80.93	0.01\\
80.94	0.01\\
80.95	0.01\\
80.96	0.01\\
80.97	0.01\\
80.98	0.01\\
80.99	0.01\\
81	0.01\\
81.01	0.01\\
81.02	0.01\\
81.03	0.01\\
81.04	0.01\\
81.05	0.01\\
81.06	0.01\\
81.07	0.01\\
81.08	0.01\\
81.09	0.01\\
81.1	0.01\\
81.11	0.01\\
81.12	0.01\\
81.13	0.01\\
81.14	0.01\\
81.15	0.01\\
81.16	0.01\\
81.17	0.01\\
81.18	0.01\\
81.19	0.01\\
81.2	0.01\\
81.21	0.01\\
81.22	0.01\\
81.23	0.01\\
81.24	0.01\\
81.25	0.01\\
81.26	0.01\\
81.27	0.01\\
81.28	0.01\\
81.29	0.01\\
81.3	0.01\\
81.31	0.01\\
81.32	0.01\\
81.33	0.01\\
81.34	0.01\\
81.35	0.01\\
81.36	0.01\\
81.37	0.01\\
81.38	0.01\\
81.39	0.01\\
81.4	0.01\\
81.41	0.01\\
81.42	0.01\\
81.43	0.01\\
81.44	0.01\\
81.45	0.01\\
81.46	0.01\\
81.47	0.01\\
81.48	0.01\\
81.49	0.01\\
81.5	0.01\\
81.51	0.01\\
81.52	0.01\\
81.53	0.01\\
81.54	0.01\\
81.55	0.01\\
81.56	0.01\\
81.57	0.01\\
81.58	0.01\\
81.59	0.01\\
81.6	0.01\\
81.61	0.01\\
81.62	0.01\\
81.63	0.01\\
81.64	0.01\\
81.65	0.01\\
81.66	0.01\\
81.67	0.01\\
81.68	0.01\\
81.69	0.01\\
81.7	0.01\\
81.71	0.01\\
81.72	0.01\\
81.73	0.01\\
81.74	0.01\\
81.75	0.01\\
81.76	0.01\\
81.77	0.01\\
81.78	0.01\\
81.79	0.01\\
81.8	0.01\\
81.81	0.01\\
81.82	0.01\\
81.83	0.01\\
81.84	0.01\\
81.85	0.01\\
81.86	0.01\\
81.87	0.01\\
81.88	0.01\\
81.89	0.01\\
81.9	0.01\\
81.91	0.01\\
81.92	0.01\\
81.93	0.01\\
81.94	0.01\\
81.95	0.01\\
81.96	0.01\\
81.97	0.01\\
81.98	0.01\\
81.99	0.01\\
82	0.01\\
82.01	0.01\\
82.02	0.01\\
82.03	0.01\\
82.04	0.01\\
82.05	0.01\\
82.06	0.01\\
82.07	0.01\\
82.08	0.01\\
82.09	0.01\\
82.1	0.01\\
82.11	0.01\\
82.12	0.01\\
82.13	0.01\\
82.14	0.01\\
82.15	0.01\\
82.16	0.01\\
82.17	0.01\\
82.18	0.01\\
82.19	0.01\\
82.2	0.01\\
82.21	0.01\\
82.22	0.01\\
82.23	0.01\\
82.24	0.01\\
82.25	0.01\\
82.26	0.01\\
82.27	0.01\\
82.28	0.01\\
82.29	0.01\\
82.3	0.01\\
82.31	0.01\\
82.32	0.01\\
82.33	0.01\\
82.34	0.01\\
82.35	0.01\\
82.36	0.01\\
82.37	0.01\\
82.38	0.01\\
82.39	0.01\\
82.4	0.01\\
82.41	0.01\\
82.42	0.01\\
82.43	0.01\\
82.44	0.01\\
82.45	0.01\\
82.46	0.01\\
82.47	0.01\\
82.48	0.01\\
82.49	0.01\\
82.5	0.01\\
82.51	0.01\\
82.52	0.01\\
82.53	0.01\\
82.54	0.01\\
82.55	0.01\\
82.56	0.01\\
82.57	0.01\\
82.58	0.01\\
82.59	0.01\\
82.6	0.01\\
82.61	0.01\\
82.62	0.01\\
82.63	0.01\\
82.64	0.01\\
82.65	0.01\\
82.66	0.01\\
82.67	0.01\\
82.68	0.01\\
82.69	0.01\\
82.7	0.01\\
82.71	0.01\\
82.72	0.01\\
82.73	0.01\\
82.74	0.01\\
82.75	0.01\\
82.76	0.01\\
82.77	0.01\\
82.78	0.01\\
82.79	0.01\\
82.8	0.01\\
82.81	0.01\\
82.82	0.01\\
82.83	0.01\\
82.84	0.01\\
82.85	0.01\\
82.86	0.01\\
82.87	0.01\\
82.88	0.01\\
82.89	0.01\\
82.9	0.01\\
82.91	0.01\\
82.92	0.01\\
82.93	0.01\\
82.94	0.01\\
82.95	0.01\\
82.96	0.01\\
82.97	0.01\\
82.98	0.01\\
82.99	0.01\\
83	0.01\\
83.01	0.01\\
83.02	0.01\\
83.03	0.01\\
83.04	0.01\\
83.05	0.01\\
83.06	0.01\\
83.07	0.01\\
83.08	0.01\\
83.09	0.01\\
83.1	0.01\\
83.11	0.01\\
83.12	0.01\\
83.13	0.01\\
83.14	0.01\\
83.15	0.01\\
83.16	0.01\\
83.17	0.01\\
83.18	0.01\\
83.19	0.01\\
83.2	0.01\\
83.21	0.01\\
83.22	0.01\\
83.23	0.01\\
83.24	0.01\\
83.25	0.01\\
83.26	0.01\\
83.27	0.01\\
83.28	0.01\\
83.29	0.01\\
83.3	0.01\\
83.31	0.01\\
83.32	0.01\\
83.33	0.01\\
83.34	0.01\\
83.35	0.01\\
83.36	0.01\\
83.37	0.01\\
83.38	0.01\\
83.39	0.01\\
83.4	0.01\\
83.41	0.01\\
83.42	0.01\\
83.43	0.01\\
83.44	0.01\\
83.45	0.01\\
83.46	0.01\\
83.47	0.01\\
83.48	0.01\\
83.49	0.01\\
83.5	0.01\\
83.51	0.01\\
83.52	0.01\\
83.53	0.01\\
83.54	0.01\\
83.55	0.01\\
83.56	0.01\\
83.57	0.01\\
83.58	0.01\\
83.59	0.01\\
83.6	0.01\\
83.61	0.01\\
83.62	0.01\\
83.63	0.01\\
83.64	0.01\\
83.65	0.01\\
83.66	0.01\\
83.67	0.01\\
83.68	0.01\\
83.69	0.01\\
83.7	0.01\\
83.71	0.01\\
83.72	0.01\\
83.73	0.01\\
83.74	0.01\\
83.75	0.01\\
83.76	0.01\\
83.77	0.01\\
83.78	0.01\\
83.79	0.01\\
83.8	0.01\\
83.81	0.01\\
83.82	0.01\\
83.83	0.01\\
83.84	0.01\\
83.85	0.01\\
83.86	0.01\\
83.87	0.01\\
83.88	0.01\\
83.89	0.01\\
83.9	0.01\\
83.91	0.01\\
83.92	0.01\\
83.93	0.01\\
83.94	0.01\\
83.95	0.01\\
83.96	0.01\\
83.97	0.01\\
83.98	0.01\\
83.99	0.01\\
84	0.01\\
84.01	0.01\\
84.02	0.01\\
84.03	0.01\\
84.04	0.01\\
84.05	0.01\\
84.06	0.01\\
84.07	0.01\\
84.08	0.01\\
84.09	0.01\\
84.1	0.01\\
84.11	0.01\\
84.12	0.01\\
84.13	0.01\\
84.14	0.01\\
84.15	0.01\\
84.16	0.01\\
84.17	0.01\\
84.18	0.01\\
84.19	0.01\\
84.2	0.01\\
84.21	0.01\\
84.22	0.01\\
84.23	0.01\\
84.24	0.01\\
84.25	0.01\\
84.26	0.01\\
84.27	0.01\\
84.28	0.01\\
84.29	0.01\\
84.3	0.01\\
84.31	0.01\\
84.32	0.01\\
84.33	0.01\\
84.34	0.01\\
84.35	0.01\\
84.36	0.01\\
84.37	0.01\\
84.38	0.01\\
84.39	0.01\\
84.4	0.01\\
84.41	0.01\\
84.42	0.01\\
84.43	0.01\\
84.44	0.01\\
84.45	0.01\\
84.46	0.01\\
84.47	0.01\\
84.48	0.01\\
84.49	0.01\\
84.5	0.01\\
84.51	0.01\\
84.52	0.01\\
84.53	0.01\\
84.54	0.01\\
84.55	0.01\\
84.56	0.01\\
84.57	0.01\\
84.58	0.01\\
84.59	0.01\\
84.6	0.01\\
84.61	0.01\\
84.62	0.01\\
84.63	0.01\\
84.64	0.01\\
84.65	0.01\\
84.66	0.01\\
84.67	0.01\\
84.68	0.01\\
84.69	0.01\\
84.7	0.01\\
84.71	0.01\\
84.72	0.01\\
84.73	0.01\\
84.74	0.01\\
84.75	0.01\\
84.76	0.01\\
84.77	0.01\\
84.78	0.01\\
84.79	0.01\\
84.8	0.01\\
84.81	0.01\\
84.82	0.01\\
84.83	0.01\\
84.84	0.01\\
84.85	0.01\\
84.86	0.01\\
84.87	0.01\\
84.88	0.01\\
84.89	0.01\\
84.9	0.01\\
84.91	0.01\\
84.92	0.01\\
84.93	0.01\\
84.94	0.01\\
84.95	0.01\\
84.96	0.01\\
84.97	0.01\\
84.98	0.01\\
84.99	0.01\\
85	0.01\\
85.01	0.01\\
85.02	0.01\\
85.03	0.01\\
85.04	0.01\\
85.05	0.01\\
85.06	0.01\\
85.07	0.01\\
85.08	0.01\\
85.09	0.01\\
85.1	0.01\\
85.11	0.01\\
85.12	0.01\\
85.13	0.01\\
85.14	0.01\\
85.15	0.01\\
85.16	0.01\\
85.17	0.01\\
85.18	0.01\\
85.19	0.01\\
85.2	0.01\\
85.21	0.01\\
85.22	0.01\\
85.23	0.01\\
85.24	0.01\\
85.25	0.01\\
85.26	0.01\\
85.27	0.01\\
85.28	0.01\\
85.29	0.01\\
85.3	0.01\\
85.31	0.01\\
85.32	0.01\\
85.33	0.01\\
85.34	0.01\\
85.35	0.01\\
85.36	0.01\\
85.37	0.01\\
85.38	0.01\\
85.39	0.01\\
85.4	0.01\\
85.41	0.01\\
85.42	0.01\\
85.43	0.01\\
85.44	0.01\\
85.45	0.01\\
85.46	0.01\\
85.47	0.01\\
85.48	0.01\\
85.49	0.01\\
85.5	0.01\\
85.51	0.01\\
85.52	0.01\\
85.53	0.01\\
85.54	0.01\\
85.55	0.01\\
85.56	0.01\\
85.57	0.01\\
85.58	0.01\\
85.59	0.01\\
85.6	0.01\\
85.61	0.01\\
85.62	0.01\\
85.63	0.01\\
85.64	0.01\\
85.65	0.01\\
85.66	0.01\\
85.67	0.01\\
85.68	0.01\\
85.69	0.01\\
85.7	0.01\\
85.71	0.01\\
85.72	0.01\\
85.73	0.01\\
85.74	0.01\\
85.75	0.01\\
85.76	0.01\\
85.77	0.01\\
85.78	0.01\\
85.79	0.01\\
85.8	0.01\\
85.81	0.01\\
85.82	0.01\\
85.83	0.01\\
85.84	0.01\\
85.85	0.01\\
85.86	0.01\\
85.87	0.01\\
85.88	0.01\\
85.89	0.01\\
85.9	0.01\\
85.91	0.01\\
85.92	0.01\\
85.93	0.01\\
85.94	0.01\\
85.95	0.01\\
85.96	0.01\\
85.97	0.01\\
85.98	0.01\\
85.99	0.01\\
86	0.01\\
86.01	0.01\\
86.02	0.01\\
86.03	0.01\\
86.04	0.01\\
86.05	0.01\\
86.06	0.01\\
86.07	0.01\\
86.08	0.01\\
86.09	0.01\\
86.1	0.01\\
86.11	0.01\\
86.12	0.01\\
86.13	0.01\\
86.14	0.01\\
86.15	0.01\\
86.16	0.01\\
86.17	0.01\\
86.18	0.01\\
86.19	0.01\\
86.2	0.01\\
86.21	0.01\\
86.22	0.01\\
86.23	0.01\\
86.24	0.01\\
86.25	0.01\\
86.26	0.01\\
86.27	0.01\\
86.28	0.01\\
86.29	0.01\\
86.3	0.01\\
86.31	0.01\\
86.32	0.01\\
86.33	0.01\\
86.34	0.01\\
86.35	0.01\\
86.36	0.01\\
86.37	0.01\\
86.38	0.01\\
86.39	0.01\\
86.4	0.01\\
86.41	0.01\\
86.42	0.01\\
86.43	0.01\\
86.44	0.01\\
86.45	0.01\\
86.46	0.01\\
86.47	0.01\\
86.48	0.01\\
86.49	0.01\\
86.5	0.01\\
86.51	0.01\\
86.52	0.01\\
86.53	0.01\\
86.54	0.01\\
86.55	0.01\\
86.56	0.01\\
86.57	0.01\\
86.58	0.01\\
86.59	0.01\\
86.6	0.01\\
86.61	0.01\\
86.62	0.01\\
86.63	0.01\\
86.64	0.01\\
86.65	0.01\\
86.66	0.01\\
86.67	0.01\\
86.68	0.01\\
86.69	0.01\\
86.7	0.01\\
86.71	0.01\\
86.72	0.01\\
86.73	0.01\\
86.74	0.01\\
86.75	0.01\\
86.76	0.01\\
86.77	0.01\\
86.78	0.01\\
86.79	0.01\\
86.8	0.01\\
86.81	0.01\\
86.82	0.01\\
86.83	0.01\\
86.84	0.01\\
86.85	0.01\\
86.86	0.01\\
86.87	0.01\\
86.88	0.01\\
86.89	0.01\\
86.9	0.01\\
86.91	0.01\\
86.92	0.01\\
86.93	0.01\\
86.94	0.01\\
86.95	0.01\\
86.96	0.01\\
86.97	0.01\\
86.98	0.01\\
86.99	0.01\\
87	0.01\\
87.01	0.01\\
87.02	0.01\\
87.03	0.01\\
87.04	0.01\\
87.05	0.01\\
87.06	0.01\\
87.07	0.01\\
87.08	0.01\\
87.09	0.01\\
87.1	0.01\\
87.11	0.01\\
87.12	0.01\\
87.13	0.01\\
87.14	0.01\\
87.15	0.01\\
87.16	0.01\\
87.17	0.01\\
87.18	0.01\\
87.19	0.01\\
87.2	0.01\\
87.21	0.01\\
87.22	0.01\\
87.23	0.01\\
87.24	0.01\\
87.25	0.01\\
87.26	0.01\\
87.27	0.01\\
87.28	0.01\\
87.29	0.01\\
87.3	0.01\\
87.31	0.01\\
87.32	0.01\\
87.33	0.01\\
87.34	0.01\\
87.35	0.01\\
87.36	0.01\\
87.37	0.01\\
87.38	0.01\\
87.39	0.01\\
87.4	0.01\\
87.41	0.01\\
87.42	0.01\\
87.43	0.01\\
87.44	0.01\\
87.45	0.01\\
87.46	0.01\\
87.47	0.01\\
87.48	0.01\\
87.49	0.01\\
87.5	0.01\\
87.51	0.01\\
87.52	0.01\\
87.53	0.01\\
87.54	0.01\\
87.55	0.01\\
87.56	0.01\\
87.57	0.01\\
87.58	0.01\\
87.59	0.01\\
87.6	0.01\\
87.61	0.01\\
87.62	0.01\\
87.63	0.01\\
87.64	0.01\\
87.65	0.01\\
87.66	0.01\\
87.67	0.01\\
87.68	0.01\\
87.69	0.01\\
87.7	0.01\\
87.71	0.01\\
87.72	0.01\\
87.73	0.01\\
87.74	0.01\\
87.75	0.01\\
87.76	0.01\\
87.77	0.01\\
87.78	0.01\\
87.79	0.01\\
87.8	0.01\\
87.81	0.01\\
87.82	0.01\\
87.83	0.01\\
87.84	0.01\\
87.85	0.01\\
87.86	0.01\\
87.87	0.01\\
87.88	0.01\\
87.89	0.01\\
87.9	0.01\\
87.91	0.01\\
87.92	0.01\\
87.93	0.01\\
87.94	0.01\\
87.95	0.01\\
87.96	0.01\\
87.97	0.01\\
87.98	0.01\\
87.99	0.01\\
88	0.01\\
88.01	0.01\\
88.02	0.01\\
88.03	0.01\\
88.04	0.01\\
88.05	0.01\\
88.06	0.01\\
88.07	0.01\\
88.08	0.01\\
88.09	0.01\\
88.1	0.01\\
88.11	0.01\\
88.12	0.01\\
88.13	0.01\\
88.14	0.01\\
88.15	0.01\\
88.16	0.01\\
88.17	0.01\\
88.18	0.01\\
88.19	0.01\\
88.2	0.01\\
88.21	0.01\\
88.22	0.01\\
88.23	0.01\\
88.24	0.01\\
88.25	0.01\\
88.26	0.01\\
88.27	0.01\\
88.28	0.01\\
88.29	0.01\\
88.3	0.01\\
88.31	0.01\\
88.32	0.01\\
88.33	0.01\\
88.34	0.01\\
88.35	0.01\\
88.36	0.01\\
88.37	0.01\\
88.38	0.01\\
88.39	0.01\\
88.4	0.01\\
88.41	0.01\\
88.42	0.01\\
88.43	0.01\\
88.44	0.01\\
88.45	0.01\\
88.46	0.01\\
88.47	0.01\\
88.48	0.01\\
88.49	0.01\\
88.5	0.01\\
88.51	0.01\\
88.52	0.01\\
88.53	0.01\\
88.54	0.01\\
88.55	0.01\\
88.56	0.01\\
88.57	0.01\\
88.58	0.01\\
88.59	0.01\\
88.6	0.01\\
88.61	0.01\\
88.62	0.01\\
88.63	0.01\\
88.64	0.01\\
88.65	0.01\\
88.66	0.01\\
88.67	0.01\\
88.68	0.01\\
88.69	0.01\\
88.7	0.01\\
88.71	0.01\\
88.72	0.01\\
88.73	0.01\\
88.74	0.01\\
88.75	0.01\\
88.76	0.01\\
88.77	0.01\\
88.78	0.01\\
88.79	0.01\\
88.8	0.01\\
88.81	0.01\\
88.82	0.01\\
88.83	0.01\\
88.84	0.01\\
88.85	0.01\\
88.86	0.01\\
88.87	0.01\\
88.88	0.01\\
88.89	0.01\\
88.9	0.01\\
88.91	0.01\\
88.92	0.01\\
88.93	0.01\\
88.94	0.01\\
88.95	0.01\\
88.96	0.01\\
88.97	0.01\\
88.98	0.01\\
88.99	0.01\\
89	0.01\\
89.01	0.01\\
89.02	0.01\\
89.03	0.01\\
89.04	0.01\\
89.05	0.01\\
89.06	0.01\\
89.07	0.01\\
89.08	0.01\\
89.09	0.01\\
89.1	0.01\\
89.11	0.01\\
89.12	0.01\\
89.13	0.01\\
89.14	0.01\\
89.15	0.01\\
89.16	0.01\\
89.17	0.01\\
89.18	0.01\\
89.19	0.01\\
89.2	0.01\\
89.21	0.01\\
89.22	0.01\\
89.23	0.01\\
89.24	0.01\\
89.25	0.01\\
89.26	0.01\\
89.27	0.01\\
89.28	0.01\\
89.29	0.01\\
89.3	0.01\\
89.31	0.01\\
89.32	0.01\\
89.33	0.01\\
89.34	0.01\\
89.35	0.01\\
89.36	0.01\\
89.37	0.01\\
89.38	0.01\\
89.39	0.01\\
89.4	0.01\\
89.41	0.01\\
89.42	0.01\\
89.43	0.01\\
89.44	0.01\\
89.45	0.01\\
89.46	0.01\\
89.47	0.01\\
89.48	0.01\\
89.49	0.01\\
89.5	0.01\\
89.51	0.01\\
89.52	0.01\\
89.53	0.01\\
89.54	0.01\\
89.55	0.01\\
89.56	0.01\\
89.57	0.01\\
89.58	0.01\\
89.59	0.01\\
89.6	0.01\\
89.61	0.01\\
89.62	0.01\\
89.63	0.01\\
89.64	0.01\\
89.65	0.01\\
89.66	0.01\\
89.67	0.01\\
89.68	0.01\\
89.69	0.01\\
89.7	0.01\\
89.71	0.01\\
89.72	0.01\\
89.73	0.01\\
89.74	0.01\\
89.75	0.01\\
89.76	0.01\\
89.77	0.01\\
89.78	0.01\\
89.79	0.01\\
89.8	0.01\\
89.81	0.01\\
89.82	0.01\\
89.83	0.01\\
89.84	0.01\\
89.85	0.01\\
89.86	0.01\\
89.87	0.01\\
89.88	0.01\\
89.89	0.01\\
89.9	0.01\\
89.91	0.01\\
89.92	0.01\\
89.93	0.01\\
89.94	0.01\\
89.95	0.01\\
89.96	0.01\\
89.97	0.01\\
89.98	0.01\\
89.99	0.01\\
90	0.01\\
90.01	0.01\\
90.02	0.01\\
90.03	0.01\\
90.04	0.01\\
90.05	0.01\\
90.06	0.01\\
90.07	0.01\\
90.08	0.01\\
90.09	0.01\\
90.1	0.01\\
90.11	0.01\\
90.12	0.01\\
90.13	0.01\\
90.14	0.01\\
90.15	0.01\\
90.16	0.01\\
90.17	0.01\\
90.18	0.01\\
90.19	0.01\\
90.2	0.01\\
90.21	0.01\\
90.22	0.01\\
90.23	0.01\\
90.24	0.01\\
90.25	0.01\\
90.26	0.01\\
90.27	0.01\\
90.28	0.01\\
90.29	0.01\\
90.3	0.01\\
90.31	0.01\\
90.32	0.01\\
90.33	0.01\\
90.34	0.01\\
90.35	0.01\\
90.36	0.01\\
90.37	0.01\\
90.38	0.01\\
90.39	0.01\\
90.4	0.01\\
90.41	0.01\\
90.42	0.01\\
90.43	0.01\\
90.44	0.01\\
90.45	0.01\\
90.46	0.01\\
90.47	0.01\\
90.48	0.01\\
90.49	0.01\\
90.5	0.01\\
90.51	0.01\\
90.52	0.01\\
90.53	0.01\\
90.54	0.01\\
90.55	0.01\\
90.56	0.01\\
90.57	0.01\\
90.58	0.01\\
90.59	0.01\\
90.6	0.01\\
90.61	0.01\\
90.62	0.01\\
90.63	0.01\\
90.64	0.01\\
90.65	0.01\\
90.66	0.01\\
90.67	0.01\\
90.68	0.01\\
90.69	0.01\\
90.7	0.01\\
90.71	0.01\\
90.72	0.01\\
90.73	0.01\\
90.74	0.01\\
90.75	0.01\\
90.76	0.01\\
90.77	0.01\\
90.78	0.01\\
90.79	0.01\\
90.8	0.01\\
90.81	0.01\\
90.82	0.01\\
90.83	0.01\\
90.84	0.01\\
90.85	0.01\\
90.86	0.01\\
90.87	0.01\\
90.88	0.01\\
90.89	0.01\\
90.9	0.01\\
90.91	0.01\\
90.92	0.01\\
90.93	0.01\\
90.94	0.01\\
90.95	0.01\\
90.96	0.01\\
90.97	0.01\\
90.98	0.01\\
90.99	0.01\\
91	0.01\\
91.01	0.01\\
91.02	0.01\\
91.03	0.01\\
91.04	0.01\\
91.05	0.01\\
91.06	0.01\\
91.07	0.01\\
91.08	0.01\\
91.09	0.01\\
91.1	0.01\\
91.11	0.01\\
91.12	0.01\\
91.13	0.01\\
91.14	0.01\\
91.15	0.01\\
91.16	0.01\\
91.17	0.01\\
91.18	0.01\\
91.19	0.01\\
91.2	0.01\\
91.21	0.01\\
91.22	0.01\\
91.23	0.01\\
91.24	0.01\\
91.25	0.01\\
91.26	0.01\\
91.27	0.01\\
91.28	0.01\\
91.29	0.01\\
91.3	0.01\\
91.31	0.01\\
91.32	0.01\\
91.33	0.01\\
91.34	0.01\\
91.35	0.01\\
91.36	0.01\\
91.37	0.01\\
91.38	0.01\\
91.39	0.01\\
91.4	0.01\\
91.41	0.01\\
91.42	0.01\\
91.43	0.01\\
91.44	0.01\\
91.45	0.01\\
91.46	0.01\\
91.47	0.01\\
91.48	0.01\\
91.49	0.01\\
91.5	0.01\\
91.51	0.01\\
91.52	0.01\\
91.53	0.01\\
91.54	0.01\\
91.55	0.01\\
91.56	0.01\\
91.57	0.01\\
91.58	0.01\\
91.59	0.01\\
91.6	0.01\\
91.61	0.01\\
91.62	0.01\\
91.63	0.01\\
91.64	0.01\\
91.65	0.01\\
91.66	0.01\\
91.67	0.01\\
91.68	0.01\\
91.69	0.01\\
91.7	0.01\\
91.71	0.01\\
91.72	0.01\\
91.73	0.01\\
91.74	0.01\\
91.75	0.01\\
91.76	0.01\\
91.77	0.01\\
91.78	0.01\\
91.79	0.01\\
91.8	0.01\\
91.81	0.01\\
91.82	0.01\\
91.83	0.01\\
91.84	0.01\\
91.85	0.01\\
91.86	0.01\\
91.87	0.01\\
91.88	0.01\\
91.89	0.01\\
91.9	0.01\\
91.91	0.01\\
91.92	0.01\\
91.93	0.01\\
91.94	0.01\\
91.95	0.01\\
91.96	0.01\\
91.97	0.01\\
91.98	0.01\\
91.99	0.01\\
92	0.01\\
92.01	0.01\\
92.02	0.01\\
92.03	0.01\\
92.04	0.01\\
92.05	0.01\\
92.06	0.01\\
92.07	0.01\\
92.08	0.01\\
92.09	0.01\\
92.1	0.01\\
92.11	0.01\\
92.12	0.01\\
92.13	0.01\\
92.14	0.01\\
92.15	0.01\\
92.16	0.01\\
92.17	0.01\\
92.18	0.01\\
92.19	0.01\\
92.2	0.01\\
92.21	0.01\\
92.22	0.01\\
92.23	0.01\\
92.24	0.01\\
92.25	0.01\\
92.26	0.01\\
92.27	0.01\\
92.28	0.01\\
92.29	0.01\\
92.3	0.01\\
92.31	0.01\\
92.32	0.01\\
92.33	0.01\\
92.34	0.01\\
92.35	0.01\\
92.36	0.01\\
92.37	0.01\\
92.38	0.01\\
92.39	0.01\\
92.4	0.01\\
92.41	0.01\\
92.42	0.01\\
92.43	0.01\\
92.44	0.01\\
92.45	0.01\\
92.46	0.01\\
92.47	0.01\\
92.48	0.01\\
92.49	0.01\\
92.5	0.01\\
92.51	0.01\\
92.52	0.01\\
92.53	0.01\\
92.54	0.01\\
92.55	0.01\\
92.56	0.01\\
92.57	0.01\\
92.58	0.01\\
92.59	0.01\\
92.6	0.01\\
92.61	0.01\\
92.62	0.01\\
92.63	0.01\\
92.64	0.01\\
92.65	0.01\\
92.66	0.01\\
92.67	0.01\\
92.68	0.01\\
92.69	0.01\\
92.7	0.01\\
92.71	0.01\\
92.72	0.01\\
92.73	0.01\\
92.74	0.01\\
92.75	0.01\\
92.76	0.01\\
92.77	0.01\\
92.78	0.01\\
92.79	0.01\\
92.8	0.01\\
92.81	0.01\\
92.82	0.01\\
92.83	0.01\\
92.84	0.01\\
92.85	0.01\\
92.86	0.01\\
92.87	0.01\\
92.88	0.01\\
92.89	0.01\\
92.9	0.01\\
92.91	0.01\\
92.92	0.01\\
92.93	0.01\\
92.94	0.01\\
92.95	0.01\\
92.96	0.01\\
92.97	0.01\\
92.98	0.01\\
92.99	0.01\\
93	0.01\\
93.01	0.01\\
93.02	0.01\\
93.03	0.01\\
93.04	0.01\\
93.05	0.01\\
93.06	0.01\\
93.07	0.01\\
93.08	0.01\\
93.09	0.01\\
93.1	0.01\\
93.11	0.01\\
93.12	0.01\\
93.13	0.01\\
93.14	0.01\\
93.15	0.01\\
93.16	0.01\\
93.17	0.01\\
93.18	0.01\\
93.19	0.01\\
93.2	0.01\\
93.21	0.01\\
93.22	0.01\\
93.23	0.01\\
93.24	0.01\\
93.25	0.01\\
93.26	0.01\\
93.27	0.01\\
93.28	0.01\\
93.29	0.01\\
93.3	0.01\\
93.31	0.01\\
93.32	0.01\\
93.33	0.01\\
93.34	0.01\\
93.35	0.01\\
93.36	0.01\\
93.37	0.01\\
93.38	0.01\\
93.39	0.01\\
93.4	0.01\\
93.41	0.01\\
93.42	0.01\\
93.43	0.01\\
93.44	0.01\\
93.45	0.01\\
93.46	0.01\\
93.47	0.01\\
93.48	0.01\\
93.49	0.01\\
93.5	0.01\\
93.51	0.01\\
93.52	0.01\\
93.53	0.01\\
93.54	0.01\\
93.55	0.01\\
93.56	0.01\\
93.57	0.01\\
93.58	0.01\\
93.59	0.01\\
93.6	0.01\\
93.61	0.01\\
93.62	0.01\\
93.63	0.01\\
93.64	0.01\\
93.65	0.01\\
93.66	0.01\\
93.67	0.01\\
93.68	0.01\\
93.69	0.01\\
93.7	0.01\\
93.71	0.01\\
93.72	0.01\\
93.73	0.01\\
93.74	0.01\\
93.75	0.01\\
93.76	0.01\\
93.77	0.01\\
93.78	0.01\\
93.79	0.01\\
93.8	0.01\\
93.81	0.01\\
93.82	0.01\\
93.83	0.01\\
93.84	0.01\\
93.85	0.01\\
93.86	0.01\\
93.87	0.01\\
93.88	0.01\\
93.89	0.01\\
93.9	0.01\\
93.91	0.01\\
93.92	0.01\\
93.93	0.01\\
93.94	0.01\\
93.95	0.01\\
93.96	0.01\\
93.97	0.01\\
93.98	0.01\\
93.99	0.01\\
94	0.01\\
94.01	0.01\\
94.02	0.01\\
94.03	0.01\\
94.04	0.01\\
94.05	0.01\\
94.06	0.01\\
94.07	0.01\\
94.08	0.01\\
94.09	0.01\\
94.1	0.01\\
94.11	0.01\\
94.12	0.01\\
94.13	0.01\\
94.14	0.01\\
94.15	0.01\\
94.16	0.01\\
94.17	0.01\\
94.18	0.01\\
94.19	0.01\\
94.2	0.01\\
94.21	0.01\\
94.22	0.01\\
94.23	0.01\\
94.24	0.01\\
94.25	0.01\\
94.26	0.01\\
94.27	0.01\\
94.28	0.01\\
94.29	0.01\\
94.3	0.01\\
94.31	0.01\\
94.32	0.01\\
94.33	0.01\\
94.34	0.01\\
94.35	0.01\\
94.36	0.01\\
94.37	0.01\\
94.38	0.01\\
94.39	0.01\\
94.4	0.01\\
94.41	0.01\\
94.42	0.01\\
94.43	0.01\\
94.44	0.01\\
94.45	0.01\\
94.46	0.01\\
94.47	0.01\\
94.48	0.01\\
94.49	0.01\\
94.5	0.01\\
94.51	0.01\\
94.52	0.01\\
94.53	0.01\\
94.54	0.01\\
94.55	0.01\\
94.56	0.01\\
94.57	0.01\\
94.58	0.01\\
94.59	0.01\\
94.6	0.01\\
94.61	0.01\\
94.62	0.01\\
94.63	0.01\\
94.64	0.01\\
94.65	0.01\\
94.66	0.01\\
94.67	0.01\\
94.68	0.01\\
94.69	0.01\\
94.7	0.01\\
94.71	0.01\\
94.72	0.01\\
94.73	0.01\\
94.74	0.01\\
94.75	0.01\\
94.76	0.01\\
94.77	0.01\\
94.78	0.01\\
94.79	0.01\\
94.8	0.01\\
94.81	0.01\\
94.82	0.01\\
94.83	0.01\\
94.84	0.01\\
94.85	0.01\\
94.86	0.01\\
94.87	0.01\\
94.88	0.01\\
94.89	0.01\\
94.9	0.01\\
94.91	0.01\\
94.92	0.01\\
94.93	0.01\\
94.94	0.01\\
94.95	0.01\\
94.96	0.01\\
94.97	0.01\\
94.98	0.01\\
94.99	0.01\\
95	0.01\\
95.01	0.01\\
95.02	0.01\\
95.03	0.01\\
95.04	0.01\\
95.05	0.01\\
95.06	0.01\\
95.07	0.01\\
95.08	0.01\\
95.09	0.01\\
95.1	0.01\\
95.11	0.01\\
95.12	0.01\\
95.13	0.01\\
95.14	0.01\\
95.15	0.01\\
95.16	0.01\\
95.17	0.01\\
95.18	0.01\\
95.19	0.01\\
95.2	0.01\\
95.21	0.01\\
95.22	0.01\\
95.23	0.01\\
95.24	0.01\\
95.25	0.01\\
95.26	0.01\\
95.27	0.01\\
95.28	0.01\\
95.29	0.01\\
95.3	0.01\\
95.31	0.01\\
95.32	0.01\\
95.33	0.01\\
95.34	0.01\\
95.35	0.01\\
95.36	0.01\\
95.37	0.01\\
95.38	0.01\\
95.39	0.01\\
95.4	0.01\\
95.41	0.01\\
95.42	0.01\\
95.43	0.01\\
95.44	0.01\\
95.45	0.01\\
95.46	0.01\\
95.47	0.01\\
95.48	0.01\\
95.49	0.01\\
95.5	0.01\\
95.51	0.01\\
95.52	0.01\\
95.53	0.01\\
95.54	0.01\\
95.55	0.01\\
95.56	0.01\\
95.57	0.01\\
95.58	0.01\\
95.59	0.01\\
95.6	0.01\\
95.61	0.01\\
95.62	0.01\\
95.63	0.01\\
95.64	0.01\\
95.65	0.01\\
95.66	0.01\\
95.67	0.01\\
95.68	0.01\\
95.69	0.01\\
95.7	0.01\\
95.71	0.01\\
95.72	0.01\\
95.73	0.01\\
95.74	0.01\\
95.75	0.01\\
95.76	0.01\\
95.77	0.01\\
95.78	0.01\\
95.79	0.01\\
95.8	0.01\\
95.81	0.01\\
95.82	0.01\\
95.83	0.01\\
95.84	0.01\\
95.85	0.01\\
95.86	0.01\\
95.87	0.01\\
95.88	0.01\\
95.89	0.01\\
95.9	0.01\\
95.91	0.01\\
95.92	0.01\\
95.93	0.01\\
95.94	0.01\\
95.95	0.01\\
95.96	0.01\\
95.97	0.01\\
95.98	0.01\\
95.99	0.01\\
96	0.01\\
96.01	0.01\\
96.02	0.01\\
96.03	0.01\\
96.04	0.01\\
96.05	0.01\\
96.06	0.01\\
96.07	0.01\\
96.08	0.01\\
96.09	0.01\\
96.1	0.01\\
96.11	0.01\\
96.12	0.01\\
96.13	0.01\\
96.14	0.01\\
96.15	0.01\\
96.16	0.01\\
96.17	0.01\\
96.18	0.01\\
96.19	0.01\\
96.2	0.01\\
96.21	0.01\\
96.22	0.01\\
96.23	0.01\\
96.24	0.01\\
96.25	0.01\\
96.26	0.01\\
96.27	0.01\\
96.28	0.01\\
96.29	0.01\\
96.3	0.01\\
96.31	0.01\\
96.32	0.01\\
96.33	0.01\\
96.34	0.01\\
96.35	0.01\\
96.36	0.01\\
96.37	0.01\\
96.38	0.01\\
96.39	0.01\\
96.4	0.01\\
96.41	0.01\\
96.42	0.01\\
96.43	0.01\\
96.44	0.01\\
96.45	0.01\\
96.46	0.01\\
96.47	0.01\\
96.48	0.01\\
96.49	0.01\\
96.5	0.01\\
96.51	0.01\\
96.52	0.01\\
96.53	0.01\\
96.54	0.01\\
96.55	0.01\\
96.56	0.01\\
96.57	0.01\\
96.58	0.01\\
96.59	0.01\\
96.6	0.01\\
96.61	0.01\\
96.62	0.01\\
96.63	0.01\\
96.64	0.01\\
96.65	0.01\\
96.66	0.01\\
96.67	0.01\\
96.68	0.01\\
96.69	0.01\\
96.7	0.01\\
96.71	0.01\\
96.72	0.01\\
96.73	0.01\\
96.74	0.01\\
96.75	0.01\\
96.76	0.01\\
96.77	0.01\\
96.78	0.01\\
96.79	0.01\\
96.8	0.01\\
96.81	0.01\\
96.82	0.01\\
96.83	0.01\\
96.84	0.01\\
96.85	0.01\\
96.86	0.01\\
96.87	0.01\\
96.88	0.01\\
96.89	0.01\\
96.9	0.01\\
96.91	0.01\\
96.92	0.01\\
96.93	0.01\\
96.94	0.01\\
96.95	0.01\\
96.96	0.01\\
96.97	0.01\\
96.98	0.01\\
96.99	0.01\\
97	0.01\\
97.01	0.01\\
97.02	0.01\\
97.03	0.01\\
97.04	0.01\\
97.05	0.01\\
97.06	0.01\\
97.07	0.01\\
97.08	0.01\\
97.09	0.01\\
97.1	0.01\\
97.11	0.01\\
97.12	0.01\\
97.13	0.01\\
97.14	0.01\\
97.15	0.01\\
97.16	0.01\\
97.17	0.01\\
97.18	0.01\\
97.19	0.01\\
97.2	0.01\\
97.21	0.01\\
97.22	0.01\\
97.23	0.01\\
97.24	0.01\\
97.25	0.01\\
97.26	0.01\\
97.27	0.01\\
97.28	0.01\\
97.29	0.01\\
97.3	0.01\\
97.31	0.01\\
97.32	0.01\\
97.33	0.01\\
97.34	0.01\\
97.35	0.01\\
97.36	0.01\\
97.37	0.01\\
97.38	0.01\\
97.39	0.01\\
97.4	0.01\\
97.41	0.01\\
97.42	0.01\\
97.43	0.01\\
97.44	0.01\\
97.45	0.01\\
97.46	0.01\\
97.47	0.01\\
97.48	0.01\\
97.49	0.01\\
97.5	0.01\\
97.51	0.01\\
97.52	0.01\\
97.53	0.01\\
97.54	0.01\\
97.55	0.01\\
97.56	0.01\\
97.57	0.01\\
97.58	0.01\\
97.59	0.01\\
97.6	0.01\\
97.61	0.01\\
97.62	0.01\\
97.63	0.01\\
97.64	0.01\\
97.65	0.01\\
97.66	0.01\\
97.67	0.01\\
97.68	0.01\\
97.69	0.01\\
97.7	0.01\\
97.71	0.01\\
97.72	0.01\\
97.73	0.01\\
97.74	0.01\\
97.75	0.01\\
97.76	0.01\\
97.77	0.01\\
97.78	0.01\\
97.79	0.01\\
97.8	0.01\\
97.81	0.01\\
97.82	0.01\\
97.83	0.01\\
97.84	0.01\\
97.85	0.01\\
97.86	0.01\\
97.87	0.01\\
97.88	0.01\\
97.89	0.01\\
97.9	0.01\\
97.91	0.01\\
97.92	0.01\\
97.93	0.01\\
97.94	0.01\\
97.95	0.01\\
97.96	0.01\\
97.97	0.01\\
97.98	0.01\\
97.99	0.01\\
98	0.01\\
98.01	0.01\\
98.02	0.01\\
98.03	0.01\\
98.04	0.01\\
98.05	0.01\\
98.06	0.01\\
98.07	0.01\\
98.08	0.01\\
98.09	0.01\\
98.1	0.01\\
98.11	0.01\\
98.12	0.01\\
98.13	0.01\\
98.14	0.01\\
98.15	0.01\\
98.16	0.01\\
98.17	0.01\\
98.18	0.01\\
98.19	0.01\\
98.2	0.01\\
98.21	0.01\\
98.22	0.01\\
98.23	0.01\\
98.24	0.01\\
98.25	0.01\\
98.26	0.01\\
98.27	0.01\\
98.28	0.01\\
98.29	0.01\\
98.3	0.01\\
98.31	0.01\\
98.32	0.01\\
98.33	0.01\\
98.34	0.01\\
98.35	0.01\\
98.36	0.01\\
98.37	0.01\\
98.38	0.01\\
98.39	0.01\\
98.4	0.01\\
98.41	0.01\\
98.42	0.01\\
98.43	0.01\\
98.44	0.01\\
98.45	0.01\\
98.46	0.01\\
98.47	0.01\\
98.48	0.01\\
98.49	0.01\\
98.5	0.01\\
98.51	0.01\\
98.52	0.01\\
98.53	0.01\\
98.54	0.01\\
98.55	0.01\\
98.56	0.01\\
98.57	0.01\\
98.58	0.01\\
98.59	0.01\\
98.6	0.01\\
98.61	0.01\\
98.62	0.01\\
98.63	0.01\\
98.64	0.01\\
98.65	0.01\\
98.66	0.01\\
98.67	0.01\\
98.68	0.01\\
98.69	0.01\\
98.7	0.01\\
98.71	0.01\\
98.72	0.01\\
98.73	0.01\\
98.74	0.01\\
98.75	0.01\\
98.76	0.01\\
98.77	0.01\\
98.78	0.01\\
98.79	0.01\\
98.8	0.01\\
98.81	0.01\\
98.82	0.01\\
98.83	0.01\\
98.84	0.01\\
98.85	0.01\\
98.86	0.01\\
98.87	0.01\\
98.88	0.01\\
98.89	0.01\\
98.9	0.01\\
98.91	0.01\\
98.92	0.01\\
98.93	0.01\\
98.94	0.01\\
98.95	0.01\\
98.96	0.01\\
98.97	0.01\\
98.98	0.01\\
98.99	0.01\\
99	0.01\\
99.01	0.01\\
99.02	0.01\\
99.03	0.01\\
99.04	0.01\\
99.05	0.01\\
99.06	0.01\\
99.07	0.01\\
99.08	0.01\\
99.09	0.01\\
99.1	0.01\\
99.11	0.01\\
99.12	0.01\\
99.13	0.01\\
99.14	0.01\\
99.15	0.01\\
99.16	0.01\\
99.17	0.01\\
99.18	0.01\\
99.19	0.01\\
99.2	0.01\\
99.21	0.01\\
99.22	0.01\\
99.23	0.01\\
99.24	0.01\\
99.25	0.01\\
99.26	0.01\\
99.27	0.01\\
99.28	0.01\\
99.29	0.01\\
99.3	0.01\\
99.31	0.01\\
99.32	0.01\\
99.33	0.01\\
99.34	0.01\\
99.35	0.01\\
99.36	0.01\\
99.37	0.01\\
99.38	0.01\\
99.39	0.01\\
99.4	0.01\\
99.41	0.01\\
99.42	0.01\\
99.43	0.01\\
99.44	0.01\\
99.45	0.01\\
99.46	0.01\\
99.47	0.01\\
99.48	0.01\\
99.49	0.01\\
99.5	0.01\\
99.51	0.01\\
99.52	0.01\\
99.53	0.01\\
99.54	0.01\\
99.55	0.01\\
99.56	0.01\\
99.57	0.01\\
99.58	0.01\\
99.59	0.01\\
99.6	0.01\\
99.61	0.01\\
99.62	0.01\\
99.63	0.01\\
99.64	0.01\\
99.65	0.01\\
99.66	0.01\\
99.67	0.01\\
99.68	0.01\\
99.69	0.01\\
99.7	0.01\\
99.71	0.01\\
99.72	0.01\\
99.73	0.01\\
99.74	0.01\\
99.75	0.01\\
99.76	0.01\\
99.77	0.01\\
99.78	0.01\\
99.79	0.01\\
99.8	0.01\\
99.81	0.01\\
99.82	0.01\\
99.83	0.01\\
99.84	0.01\\
99.85	0.01\\
99.86	0.01\\
99.87	0.01\\
99.88	0.01\\
99.89	0.01\\
99.9	0.01\\
99.91	0.01\\
99.92	0.01\\
99.93	0.01\\
99.94	0.01\\
99.95	0.01\\
99.96	0.01\\
99.97	0.01\\
99.98	0.01\\
99.99	0.01\\
100	0.01\\
};
\addlegendentry{$q=-4$};

\addplot [color=mycolor1,dashed,forget plot]
  table[row sep=crcr]{%
0.01	0.01\\
0.02	0.01\\
0.03	0.01\\
0.04	0.01\\
0.05	0.01\\
0.06	0.01\\
0.07	0.01\\
0.08	0.01\\
0.09	0.01\\
0.1	0.01\\
0.11	0.01\\
0.12	0.01\\
0.13	0.01\\
0.14	0.01\\
0.15	0.01\\
0.16	0.01\\
0.17	0.01\\
0.18	0.01\\
0.19	0.01\\
0.2	0.01\\
0.21	0.01\\
0.22	0.01\\
0.23	0.01\\
0.24	0.01\\
0.25	0.01\\
0.26	0.01\\
0.27	0.01\\
0.28	0.01\\
0.29	0.01\\
0.3	0.01\\
0.31	0.01\\
0.32	0.01\\
0.33	0.01\\
0.34	0.01\\
0.35	0.01\\
0.36	0.01\\
0.37	0.01\\
0.38	0.01\\
0.39	0.01\\
0.4	0.01\\
0.41	0.01\\
0.42	0.01\\
0.43	0.01\\
0.44	0.01\\
0.45	0.01\\
0.46	0.01\\
0.47	0.01\\
0.48	0.01\\
0.49	0.01\\
0.5	0.01\\
0.51	0.01\\
0.52	0.01\\
0.53	0.01\\
0.54	0.01\\
0.55	0.01\\
0.56	0.01\\
0.57	0.01\\
0.58	0.01\\
0.59	0.01\\
0.6	0.01\\
0.61	0.01\\
0.62	0.01\\
0.63	0.01\\
0.64	0.01\\
0.65	0.01\\
0.66	0.01\\
0.67	0.01\\
0.68	0.01\\
0.69	0.01\\
0.7	0.01\\
0.71	0.01\\
0.72	0.01\\
0.73	0.01\\
0.74	0.01\\
0.75	0.01\\
0.76	0.01\\
0.77	0.01\\
0.78	0.01\\
0.79	0.01\\
0.8	0.01\\
0.81	0.01\\
0.82	0.01\\
0.83	0.01\\
0.84	0.01\\
0.85	0.01\\
0.86	0.01\\
0.87	0.01\\
0.88	0.01\\
0.89	0.01\\
0.9	0.01\\
0.91	0.01\\
0.92	0.01\\
0.93	0.01\\
0.94	0.01\\
0.95	0.01\\
0.96	0.01\\
0.97	0.01\\
0.98	0.01\\
0.99	0.01\\
1	0.01\\
1.01	0.01\\
1.02	0.01\\
1.03	0.01\\
1.04	0.01\\
1.05	0.01\\
1.06	0.01\\
1.07	0.01\\
1.08	0.01\\
1.09	0.01\\
1.1	0.01\\
1.11	0.01\\
1.12	0.01\\
1.13	0.01\\
1.14	0.01\\
1.15	0.01\\
1.16	0.01\\
1.17	0.01\\
1.18	0.01\\
1.19	0.01\\
1.2	0.01\\
1.21	0.01\\
1.22	0.01\\
1.23	0.01\\
1.24	0.01\\
1.25	0.01\\
1.26	0.01\\
1.27	0.01\\
1.28	0.01\\
1.29	0.01\\
1.3	0.01\\
1.31	0.01\\
1.32	0.01\\
1.33	0.01\\
1.34	0.01\\
1.35	0.01\\
1.36	0.01\\
1.37	0.01\\
1.38	0.01\\
1.39	0.01\\
1.4	0.01\\
1.41	0.01\\
1.42	0.01\\
1.43	0.01\\
1.44	0.01\\
1.45	0.01\\
1.46	0.01\\
1.47	0.01\\
1.48	0.01\\
1.49	0.01\\
1.5	0.01\\
1.51	0.01\\
1.52	0.01\\
1.53	0.01\\
1.54	0.01\\
1.55	0.01\\
1.56	0.01\\
1.57	0.01\\
1.58	0.01\\
1.59	0.01\\
1.6	0.01\\
1.61	0.01\\
1.62	0.01\\
1.63	0.01\\
1.64	0.01\\
1.65	0.01\\
1.66	0.01\\
1.67	0.01\\
1.68	0.01\\
1.69	0.01\\
1.7	0.01\\
1.71	0.01\\
1.72	0.01\\
1.73	0.01\\
1.74	0.01\\
1.75	0.01\\
1.76	0.01\\
1.77	0.01\\
1.78	0.01\\
1.79	0.01\\
1.8	0.01\\
1.81	0.01\\
1.82	0.01\\
1.83	0.01\\
1.84	0.01\\
1.85	0.01\\
1.86	0.01\\
1.87	0.01\\
1.88	0.01\\
1.89	0.01\\
1.9	0.01\\
1.91	0.01\\
1.92	0.01\\
1.93	0.01\\
1.94	0.01\\
1.95	0.01\\
1.96	0.01\\
1.97	0.01\\
1.98	0.01\\
1.99	0.01\\
2	0.01\\
2.01	0.01\\
2.02	0.01\\
2.03	0.01\\
2.04	0.01\\
2.05	0.01\\
2.06	0.01\\
2.07	0.01\\
2.08	0.01\\
2.09	0.01\\
2.1	0.01\\
2.11	0.01\\
2.12	0.01\\
2.13	0.01\\
2.14	0.01\\
2.15	0.01\\
2.16	0.01\\
2.17	0.01\\
2.18	0.01\\
2.19	0.01\\
2.2	0.01\\
2.21	0.01\\
2.22	0.01\\
2.23	0.01\\
2.24	0.01\\
2.25	0.01\\
2.26	0.01\\
2.27	0.01\\
2.28	0.01\\
2.29	0.01\\
2.3	0.01\\
2.31	0.01\\
2.32	0.01\\
2.33	0.01\\
2.34	0.01\\
2.35	0.01\\
2.36	0.01\\
2.37	0.01\\
2.38	0.01\\
2.39	0.01\\
2.4	0.01\\
2.41	0.01\\
2.42	0.01\\
2.43	0.01\\
2.44	0.01\\
2.45	0.01\\
2.46	0.01\\
2.47	0.01\\
2.48	0.01\\
2.49	0.01\\
2.5	0.01\\
2.51	0.01\\
2.52	0.01\\
2.53	0.01\\
2.54	0.01\\
2.55	0.01\\
2.56	0.01\\
2.57	0.01\\
2.58	0.01\\
2.59	0.01\\
2.6	0.01\\
2.61	0.01\\
2.62	0.01\\
2.63	0.01\\
2.64	0.01\\
2.65	0.01\\
2.66	0.01\\
2.67	0.01\\
2.68	0.01\\
2.69	0.01\\
2.7	0.01\\
2.71	0.01\\
2.72	0.01\\
2.73	0.01\\
2.74	0.01\\
2.75	0.01\\
2.76	0.01\\
2.77	0.01\\
2.78	0.01\\
2.79	0.01\\
2.8	0.01\\
2.81	0.01\\
2.82	0.01\\
2.83	0.01\\
2.84	0.01\\
2.85	0.01\\
2.86	0.01\\
2.87	0.01\\
2.88	0.01\\
2.89	0.01\\
2.9	0.01\\
2.91	0.01\\
2.92	0.01\\
2.93	0.01\\
2.94	0.01\\
2.95	0.01\\
2.96	0.01\\
2.97	0.01\\
2.98	0.01\\
2.99	0.01\\
3	0.01\\
3.01	0.01\\
3.02	0.01\\
3.03	0.01\\
3.04	0.01\\
3.05	0.01\\
3.06	0.01\\
3.07	0.01\\
3.08	0.01\\
3.09	0.01\\
3.1	0.01\\
3.11	0.01\\
3.12	0.01\\
3.13	0.01\\
3.14	0.01\\
3.15	0.01\\
3.16	0.01\\
3.17	0.01\\
3.18	0.01\\
3.19	0.01\\
3.2	0.01\\
3.21	0.01\\
3.22	0.01\\
3.23	0.01\\
3.24	0.01\\
3.25	0.01\\
3.26	0.01\\
3.27	0.01\\
3.28	0.01\\
3.29	0.01\\
3.3	0.01\\
3.31	0.01\\
3.32	0.01\\
3.33	0.01\\
3.34	0.01\\
3.35	0.01\\
3.36	0.01\\
3.37	0.01\\
3.38	0.01\\
3.39	0.01\\
3.4	0.01\\
3.41	0.01\\
3.42	0.01\\
3.43	0.01\\
3.44	0.01\\
3.45	0.01\\
3.46	0.01\\
3.47	0.01\\
3.48	0.01\\
3.49	0.01\\
3.5	0.01\\
3.51	0.01\\
3.52	0.01\\
3.53	0.01\\
3.54	0.01\\
3.55	0.01\\
3.56	0.01\\
3.57	0.01\\
3.58	0.01\\
3.59	0.01\\
3.6	0.01\\
3.61	0.01\\
3.62	0.01\\
3.63	0.01\\
3.64	0.01\\
3.65	0.01\\
3.66	0.01\\
3.67	0.01\\
3.68	0.01\\
3.69	0.01\\
3.7	0.01\\
3.71	0.01\\
3.72	0.01\\
3.73	0.01\\
3.74	0.01\\
3.75	0.01\\
3.76	0.01\\
3.77	0.01\\
3.78	0.01\\
3.79	0.01\\
3.8	0.01\\
3.81	0.01\\
3.82	0.01\\
3.83	0.01\\
3.84	0.01\\
3.85	0.01\\
3.86	0.01\\
3.87	0.01\\
3.88	0.01\\
3.89	0.01\\
3.9	0.01\\
3.91	0.01\\
3.92	0.01\\
3.93	0.01\\
3.94	0.01\\
3.95	0.01\\
3.96	0.01\\
3.97	0.01\\
3.98	0.01\\
3.99	0.01\\
4	0.01\\
4.01	0.01\\
4.02	0.01\\
4.03	0.01\\
4.04	0.01\\
4.05	0.01\\
4.06	0.01\\
4.07	0.01\\
4.08	0.01\\
4.09	0.01\\
4.1	0.01\\
4.11	0.01\\
4.12	0.01\\
4.13	0.01\\
4.14	0.01\\
4.15	0.01\\
4.16	0.01\\
4.17	0.01\\
4.18	0.01\\
4.19	0.01\\
4.2	0.01\\
4.21	0.01\\
4.22	0.01\\
4.23	0.01\\
4.24	0.01\\
4.25	0.01\\
4.26	0.01\\
4.27	0.01\\
4.28	0.01\\
4.29	0.01\\
4.3	0.01\\
4.31	0.01\\
4.32	0.01\\
4.33	0.01\\
4.34	0.01\\
4.35	0.01\\
4.36	0.01\\
4.37	0.01\\
4.38	0.01\\
4.39	0.01\\
4.4	0.01\\
4.41	0.01\\
4.42	0.01\\
4.43	0.01\\
4.44	0.01\\
4.45	0.01\\
4.46	0.01\\
4.47	0.01\\
4.48	0.01\\
4.49	0.01\\
4.5	0.01\\
4.51	0.01\\
4.52	0.01\\
4.53	0.01\\
4.54	0.01\\
4.55	0.01\\
4.56	0.01\\
4.57	0.01\\
4.58	0.01\\
4.59	0.01\\
4.6	0.01\\
4.61	0.01\\
4.62	0.01\\
4.63	0.01\\
4.64	0.01\\
4.65	0.01\\
4.66	0.01\\
4.67	0.01\\
4.68	0.01\\
4.69	0.01\\
4.7	0.01\\
4.71	0.01\\
4.72	0.01\\
4.73	0.01\\
4.74	0.01\\
4.75	0.01\\
4.76	0.01\\
4.77	0.01\\
4.78	0.01\\
4.79	0.01\\
4.8	0.01\\
4.81	0.01\\
4.82	0.01\\
4.83	0.01\\
4.84	0.01\\
4.85	0.01\\
4.86	0.01\\
4.87	0.01\\
4.88	0.01\\
4.89	0.01\\
4.9	0.01\\
4.91	0.01\\
4.92	0.01\\
4.93	0.01\\
4.94	0.01\\
4.95	0.01\\
4.96	0.01\\
4.97	0.01\\
4.98	0.01\\
4.99	0.01\\
5	0.01\\
5.01	0.01\\
5.02	0.01\\
5.03	0.01\\
5.04	0.01\\
5.05	0.01\\
5.06	0.01\\
5.07	0.01\\
5.08	0.01\\
5.09	0.01\\
5.1	0.01\\
5.11	0.01\\
5.12	0.01\\
5.13	0.01\\
5.14	0.01\\
5.15	0.01\\
5.16	0.01\\
5.17	0.01\\
5.18	0.01\\
5.19	0.01\\
5.2	0.01\\
5.21	0.01\\
5.22	0.01\\
5.23	0.01\\
5.24	0.01\\
5.25	0.01\\
5.26	0.01\\
5.27	0.01\\
5.28	0.01\\
5.29	0.01\\
5.3	0.01\\
5.31	0.01\\
5.32	0.01\\
5.33	0.01\\
5.34	0.01\\
5.35	0.01\\
5.36	0.01\\
5.37	0.01\\
5.38	0.01\\
5.39	0.01\\
5.4	0.01\\
5.41	0.01\\
5.42	0.01\\
5.43	0.01\\
5.44	0.01\\
5.45	0.01\\
5.46	0.01\\
5.47	0.01\\
5.48	0.01\\
5.49	0.01\\
5.5	0.01\\
5.51	0.01\\
5.52	0.01\\
5.53	0.01\\
5.54	0.01\\
5.55	0.01\\
5.56	0.01\\
5.57	0.01\\
5.58	0.01\\
5.59	0.01\\
5.6	0.01\\
5.61	0.01\\
5.62	0.01\\
5.63	0.01\\
5.64	0.01\\
5.65	0.01\\
5.66	0.01\\
5.67	0.01\\
5.68	0.01\\
5.69	0.01\\
5.7	0.01\\
5.71	0.01\\
5.72	0.01\\
5.73	0.01\\
5.74	0.01\\
5.75	0.01\\
5.76	0.01\\
5.77	0.01\\
5.78	0.01\\
5.79	0.01\\
5.8	0.01\\
5.81	0.01\\
5.82	0.01\\
5.83	0.01\\
5.84	0.01\\
5.85	0.01\\
5.86	0.01\\
5.87	0.01\\
5.88	0.01\\
5.89	0.01\\
5.9	0.01\\
5.91	0.01\\
5.92	0.01\\
5.93	0.01\\
5.94	0.01\\
5.95	0.01\\
5.96	0.01\\
5.97	0.01\\
5.98	0.01\\
5.99	0.01\\
6	0.01\\
6.01	0.01\\
6.02	0.01\\
6.03	0.01\\
6.04	0.01\\
6.05	0.01\\
6.06	0.01\\
6.07	0.01\\
6.08	0.01\\
6.09	0.01\\
6.1	0.01\\
6.11	0.01\\
6.12	0.01\\
6.13	0.01\\
6.14	0.01\\
6.15	0.01\\
6.16	0.01\\
6.17	0.01\\
6.18	0.01\\
6.19	0.01\\
6.2	0.01\\
6.21	0.01\\
6.22	0.01\\
6.23	0.01\\
6.24	0.01\\
6.25	0.01\\
6.26	0.01\\
6.27	0.01\\
6.28	0.01\\
6.29	0.01\\
6.3	0.01\\
6.31	0.01\\
6.32	0.01\\
6.33	0.01\\
6.34	0.01\\
6.35	0.01\\
6.36	0.01\\
6.37	0.01\\
6.38	0.01\\
6.39	0.01\\
6.4	0.01\\
6.41	0.01\\
6.42	0.01\\
6.43	0.01\\
6.44	0.01\\
6.45	0.01\\
6.46	0.01\\
6.47	0.01\\
6.48	0.01\\
6.49	0.01\\
6.5	0.01\\
6.51	0.01\\
6.52	0.01\\
6.53	0.01\\
6.54	0.01\\
6.55	0.01\\
6.56	0.01\\
6.57	0.01\\
6.58	0.01\\
6.59	0.01\\
6.6	0.01\\
6.61	0.01\\
6.62	0.01\\
6.63	0.01\\
6.64	0.01\\
6.65	0.01\\
6.66	0.01\\
6.67	0.01\\
6.68	0.01\\
6.69	0.01\\
6.7	0.01\\
6.71	0.01\\
6.72	0.01\\
6.73	0.01\\
6.74	0.01\\
6.75	0.01\\
6.76	0.01\\
6.77	0.01\\
6.78	0.01\\
6.79	0.01\\
6.8	0.01\\
6.81	0.01\\
6.82	0.01\\
6.83	0.01\\
6.84	0.01\\
6.85	0.01\\
6.86	0.01\\
6.87	0.01\\
6.88	0.01\\
6.89	0.01\\
6.9	0.01\\
6.91	0.01\\
6.92	0.01\\
6.93	0.01\\
6.94	0.01\\
6.95	0.01\\
6.96	0.01\\
6.97	0.01\\
6.98	0.01\\
6.99	0.01\\
7	0.01\\
7.01	0.01\\
7.02	0.01\\
7.03	0.01\\
7.04	0.01\\
7.05	0.01\\
7.06	0.01\\
7.07	0.01\\
7.08	0.01\\
7.09	0.01\\
7.1	0.01\\
7.11	0.01\\
7.12	0.01\\
7.13	0.01\\
7.14	0.01\\
7.15	0.01\\
7.16	0.01\\
7.17	0.01\\
7.18	0.01\\
7.19	0.01\\
7.2	0.01\\
7.21	0.01\\
7.22	0.01\\
7.23	0.01\\
7.24	0.01\\
7.25	0.01\\
7.26	0.01\\
7.27	0.01\\
7.28	0.01\\
7.29	0.01\\
7.3	0.01\\
7.31	0.01\\
7.32	0.01\\
7.33	0.01\\
7.34	0.01\\
7.35	0.01\\
7.36	0.01\\
7.37	0.01\\
7.38	0.01\\
7.39	0.01\\
7.4	0.01\\
7.41	0.01\\
7.42	0.01\\
7.43	0.01\\
7.44	0.01\\
7.45	0.01\\
7.46	0.01\\
7.47	0.01\\
7.48	0.01\\
7.49	0.01\\
7.5	0.01\\
7.51	0.01\\
7.52	0.01\\
7.53	0.01\\
7.54	0.01\\
7.55	0.01\\
7.56	0.01\\
7.57	0.01\\
7.58	0.01\\
7.59	0.01\\
7.6	0.01\\
7.61	0.01\\
7.62	0.01\\
7.63	0.01\\
7.64	0.01\\
7.65	0.01\\
7.66	0.01\\
7.67	0.01\\
7.68	0.01\\
7.69	0.01\\
7.7	0.01\\
7.71	0.01\\
7.72	0.01\\
7.73	0.01\\
7.74	0.01\\
7.75	0.01\\
7.76	0.01\\
7.77	0.01\\
7.78	0.01\\
7.79	0.01\\
7.8	0.01\\
7.81	0.01\\
7.82	0.01\\
7.83	0.01\\
7.84	0.01\\
7.85	0.01\\
7.86	0.01\\
7.87	0.01\\
7.88	0.01\\
7.89	0.01\\
7.9	0.01\\
7.91	0.01\\
7.92	0.01\\
7.93	0.01\\
7.94	0.01\\
7.95	0.01\\
7.96	0.01\\
7.97	0.01\\
7.98	0.01\\
7.99	0.01\\
8	0.01\\
8.01	0.01\\
8.02	0.01\\
8.03	0.01\\
8.04	0.01\\
8.05	0.01\\
8.06	0.01\\
8.07	0.01\\
8.08	0.01\\
8.09	0.01\\
8.1	0.01\\
8.11	0.01\\
8.12	0.01\\
8.13	0.01\\
8.14	0.01\\
8.15	0.01\\
8.16	0.01\\
8.17	0.01\\
8.18	0.01\\
8.19	0.01\\
8.2	0.01\\
8.21	0.01\\
8.22	0.01\\
8.23	0.01\\
8.24	0.01\\
8.25	0.01\\
8.26	0.01\\
8.27	0.01\\
8.28	0.01\\
8.29	0.01\\
8.3	0.01\\
8.31	0.01\\
8.32	0.01\\
8.33	0.01\\
8.34	0.01\\
8.35	0.01\\
8.36	0.01\\
8.37	0.01\\
8.38	0.01\\
8.39	0.01\\
8.4	0.01\\
8.41	0.01\\
8.42	0.01\\
8.43	0.01\\
8.44	0.01\\
8.45	0.01\\
8.46	0.01\\
8.47	0.01\\
8.48	0.01\\
8.49	0.01\\
8.5	0.01\\
8.51	0.01\\
8.52	0.01\\
8.53	0.01\\
8.54	0.01\\
8.55	0.01\\
8.56	0.01\\
8.57	0.01\\
8.58	0.01\\
8.59	0.01\\
8.6	0.01\\
8.61	0.01\\
8.62	0.01\\
8.63	0.01\\
8.64	0.01\\
8.65	0.01\\
8.66	0.01\\
8.67	0.01\\
8.68	0.01\\
8.69	0.01\\
8.7	0.01\\
8.71	0.01\\
8.72	0.01\\
8.73	0.01\\
8.74	0.01\\
8.75	0.01\\
8.76	0.01\\
8.77	0.01\\
8.78	0.01\\
8.79	0.01\\
8.8	0.01\\
8.81	0.01\\
8.82	0.01\\
8.83	0.01\\
8.84	0.01\\
8.85	0.01\\
8.86	0.01\\
8.87	0.01\\
8.88	0.01\\
8.89	0.01\\
8.9	0.01\\
8.91	0.01\\
8.92	0.01\\
8.93	0.01\\
8.94	0.01\\
8.95	0.01\\
8.96	0.01\\
8.97	0.01\\
8.98	0.01\\
8.99	0.01\\
9	0.01\\
9.01	0.01\\
9.02	0.01\\
9.03	0.01\\
9.04	0.01\\
9.05	0.01\\
9.06	0.01\\
9.07	0.01\\
9.08	0.01\\
9.09	0.01\\
9.1	0.01\\
9.11	0.01\\
9.12	0.01\\
9.13	0.01\\
9.14	0.01\\
9.15	0.01\\
9.16	0.01\\
9.17	0.01\\
9.18	0.01\\
9.19	0.01\\
9.2	0.01\\
9.21	0.01\\
9.22	0.01\\
9.23	0.01\\
9.24	0.01\\
9.25	0.01\\
9.26	0.01\\
9.27	0.01\\
9.28	0.01\\
9.29	0.01\\
9.3	0.01\\
9.31	0.01\\
9.32	0.01\\
9.33	0.01\\
9.34	0.01\\
9.35	0.01\\
9.36	0.01\\
9.37	0.01\\
9.38	0.01\\
9.39	0.01\\
9.4	0.01\\
9.41	0.01\\
9.42	0.01\\
9.43	0.01\\
9.44	0.01\\
9.45	0.01\\
9.46	0.01\\
9.47	0.01\\
9.48	0.01\\
9.49	0.01\\
9.5	0.01\\
9.51	0.01\\
9.52	0.01\\
9.53	0.01\\
9.54	0.01\\
9.55	0.01\\
9.56	0.01\\
9.57	0.01\\
9.58	0.01\\
9.59	0.01\\
9.6	0.01\\
9.61	0.01\\
9.62	0.01\\
9.63	0.01\\
9.64	0.01\\
9.65	0.01\\
9.66	0.01\\
9.67	0.01\\
9.68	0.01\\
9.69	0.01\\
9.7	0.01\\
9.71	0.01\\
9.72	0.01\\
9.73	0.01\\
9.74	0.01\\
9.75	0.01\\
9.76	0.01\\
9.77	0.01\\
9.78	0.01\\
9.79	0.01\\
9.8	0.01\\
9.81	0.01\\
9.82	0.01\\
9.83	0.01\\
9.84	0.01\\
9.85	0.01\\
9.86	0.01\\
9.87	0.01\\
9.88	0.01\\
9.89	0.01\\
9.9	0.01\\
9.91	0.01\\
9.92	0.01\\
9.93	0.01\\
9.94	0.01\\
9.95	0.01\\
9.96	0.01\\
9.97	0.01\\
9.98	0.01\\
9.99	0.01\\
10	0.01\\
10.01	0.01\\
10.02	0.01\\
10.03	0.01\\
10.04	0.01\\
10.05	0.01\\
10.06	0.01\\
10.07	0.01\\
10.08	0.01\\
10.09	0.01\\
10.1	0.01\\
10.11	0.01\\
10.12	0.01\\
10.13	0.01\\
10.14	0.01\\
10.15	0.01\\
10.16	0.01\\
10.17	0.01\\
10.18	0.01\\
10.19	0.01\\
10.2	0.01\\
10.21	0.01\\
10.22	0.01\\
10.23	0.01\\
10.24	0.01\\
10.25	0.01\\
10.26	0.01\\
10.27	0.01\\
10.28	0.01\\
10.29	0.01\\
10.3	0.01\\
10.31	0.01\\
10.32	0.01\\
10.33	0.01\\
10.34	0.01\\
10.35	0.01\\
10.36	0.01\\
10.37	0.01\\
10.38	0.01\\
10.39	0.01\\
10.4	0.01\\
10.41	0.01\\
10.42	0.01\\
10.43	0.01\\
10.44	0.01\\
10.45	0.01\\
10.46	0.01\\
10.47	0.01\\
10.48	0.01\\
10.49	0.01\\
10.5	0.01\\
10.51	0.01\\
10.52	0.01\\
10.53	0.01\\
10.54	0.01\\
10.55	0.01\\
10.56	0.01\\
10.57	0.01\\
10.58	0.01\\
10.59	0.01\\
10.6	0.01\\
10.61	0.01\\
10.62	0.01\\
10.63	0.01\\
10.64	0.01\\
10.65	0.01\\
10.66	0.01\\
10.67	0.01\\
10.68	0.01\\
10.69	0.01\\
10.7	0.01\\
10.71	0.01\\
10.72	0.01\\
10.73	0.01\\
10.74	0.01\\
10.75	0.01\\
10.76	0.01\\
10.77	0.01\\
10.78	0.01\\
10.79	0.01\\
10.8	0.01\\
10.81	0.01\\
10.82	0.01\\
10.83	0.01\\
10.84	0.01\\
10.85	0.01\\
10.86	0.01\\
10.87	0.01\\
10.88	0.01\\
10.89	0.01\\
10.9	0.01\\
10.91	0.01\\
10.92	0.01\\
10.93	0.01\\
10.94	0.01\\
10.95	0.01\\
10.96	0.01\\
10.97	0.01\\
10.98	0.01\\
10.99	0.01\\
11	0.01\\
11.01	0.01\\
11.02	0.01\\
11.03	0.01\\
11.04	0.01\\
11.05	0.01\\
11.06	0.01\\
11.07	0.01\\
11.08	0.01\\
11.09	0.01\\
11.1	0.01\\
11.11	0.01\\
11.12	0.01\\
11.13	0.01\\
11.14	0.01\\
11.15	0.01\\
11.16	0.01\\
11.17	0.01\\
11.18	0.01\\
11.19	0.01\\
11.2	0.01\\
11.21	0.01\\
11.22	0.01\\
11.23	0.01\\
11.24	0.01\\
11.25	0.01\\
11.26	0.01\\
11.27	0.01\\
11.28	0.01\\
11.29	0.01\\
11.3	0.01\\
11.31	0.01\\
11.32	0.01\\
11.33	0.01\\
11.34	0.01\\
11.35	0.01\\
11.36	0.01\\
11.37	0.01\\
11.38	0.01\\
11.39	0.01\\
11.4	0.01\\
11.41	0.01\\
11.42	0.01\\
11.43	0.01\\
11.44	0.01\\
11.45	0.01\\
11.46	0.01\\
11.47	0.01\\
11.48	0.01\\
11.49	0.01\\
11.5	0.01\\
11.51	0.01\\
11.52	0.01\\
11.53	0.01\\
11.54	0.01\\
11.55	0.01\\
11.56	0.01\\
11.57	0.01\\
11.58	0.01\\
11.59	0.01\\
11.6	0.01\\
11.61	0.01\\
11.62	0.01\\
11.63	0.01\\
11.64	0.01\\
11.65	0.01\\
11.66	0.01\\
11.67	0.01\\
11.68	0.01\\
11.69	0.01\\
11.7	0.01\\
11.71	0.01\\
11.72	0.01\\
11.73	0.01\\
11.74	0.01\\
11.75	0.01\\
11.76	0.01\\
11.77	0.01\\
11.78	0.01\\
11.79	0.01\\
11.8	0.01\\
11.81	0.01\\
11.82	0.01\\
11.83	0.01\\
11.84	0.01\\
11.85	0.01\\
11.86	0.01\\
11.87	0.01\\
11.88	0.01\\
11.89	0.01\\
11.9	0.01\\
11.91	0.01\\
11.92	0.01\\
11.93	0.01\\
11.94	0.01\\
11.95	0.01\\
11.96	0.01\\
11.97	0.01\\
11.98	0.01\\
11.99	0.01\\
12	0.01\\
12.01	0.01\\
12.02	0.01\\
12.03	0.01\\
12.04	0.01\\
12.05	0.01\\
12.06	0.01\\
12.07	0.01\\
12.08	0.01\\
12.09	0.01\\
12.1	0.01\\
12.11	0.01\\
12.12	0.01\\
12.13	0.01\\
12.14	0.01\\
12.15	0.01\\
12.16	0.01\\
12.17	0.01\\
12.18	0.01\\
12.19	0.01\\
12.2	0.01\\
12.21	0.01\\
12.22	0.01\\
12.23	0.01\\
12.24	0.01\\
12.25	0.01\\
12.26	0.01\\
12.27	0.01\\
12.28	0.01\\
12.29	0.01\\
12.3	0.01\\
12.31	0.01\\
12.32	0.01\\
12.33	0.01\\
12.34	0.01\\
12.35	0.01\\
12.36	0.01\\
12.37	0.01\\
12.38	0.01\\
12.39	0.01\\
12.4	0.01\\
12.41	0.01\\
12.42	0.01\\
12.43	0.01\\
12.44	0.01\\
12.45	0.01\\
12.46	0.01\\
12.47	0.01\\
12.48	0.01\\
12.49	0.01\\
12.5	0.01\\
12.51	0.01\\
12.52	0.01\\
12.53	0.01\\
12.54	0.01\\
12.55	0.01\\
12.56	0.01\\
12.57	0.01\\
12.58	0.01\\
12.59	0.01\\
12.6	0.01\\
12.61	0.01\\
12.62	0.01\\
12.63	0.01\\
12.64	0.01\\
12.65	0.01\\
12.66	0.01\\
12.67	0.01\\
12.68	0.01\\
12.69	0.01\\
12.7	0.01\\
12.71	0.01\\
12.72	0.01\\
12.73	0.01\\
12.74	0.01\\
12.75	0.01\\
12.76	0.01\\
12.77	0.01\\
12.78	0.01\\
12.79	0.01\\
12.8	0.01\\
12.81	0.01\\
12.82	0.01\\
12.83	0.01\\
12.84	0.01\\
12.85	0.01\\
12.86	0.01\\
12.87	0.01\\
12.88	0.01\\
12.89	0.01\\
12.9	0.01\\
12.91	0.01\\
12.92	0.01\\
12.93	0.01\\
12.94	0.01\\
12.95	0.01\\
12.96	0.01\\
12.97	0.01\\
12.98	0.01\\
12.99	0.01\\
13	0.01\\
13.01	0.01\\
13.02	0.01\\
13.03	0.01\\
13.04	0.01\\
13.05	0.01\\
13.06	0.01\\
13.07	0.01\\
13.08	0.01\\
13.09	0.01\\
13.1	0.01\\
13.11	0.01\\
13.12	0.01\\
13.13	0.01\\
13.14	0.01\\
13.15	0.01\\
13.16	0.01\\
13.17	0.01\\
13.18	0.01\\
13.19	0.01\\
13.2	0.01\\
13.21	0.01\\
13.22	0.01\\
13.23	0.01\\
13.24	0.01\\
13.25	0.01\\
13.26	0.01\\
13.27	0.01\\
13.28	0.01\\
13.29	0.01\\
13.3	0.01\\
13.31	0.01\\
13.32	0.01\\
13.33	0.01\\
13.34	0.01\\
13.35	0.01\\
13.36	0.01\\
13.37	0.01\\
13.38	0.01\\
13.39	0.01\\
13.4	0.01\\
13.41	0.01\\
13.42	0.01\\
13.43	0.01\\
13.44	0.01\\
13.45	0.01\\
13.46	0.01\\
13.47	0.01\\
13.48	0.01\\
13.49	0.01\\
13.5	0.01\\
13.51	0.01\\
13.52	0.01\\
13.53	0.01\\
13.54	0.01\\
13.55	0.01\\
13.56	0.01\\
13.57	0.01\\
13.58	0.01\\
13.59	0.01\\
13.6	0.01\\
13.61	0.01\\
13.62	0.01\\
13.63	0.01\\
13.64	0.01\\
13.65	0.01\\
13.66	0.01\\
13.67	0.01\\
13.68	0.01\\
13.69	0.01\\
13.7	0.01\\
13.71	0.01\\
13.72	0.01\\
13.73	0.01\\
13.74	0.01\\
13.75	0.01\\
13.76	0.01\\
13.77	0.01\\
13.78	0.01\\
13.79	0.01\\
13.8	0.01\\
13.81	0.01\\
13.82	0.01\\
13.83	0.01\\
13.84	0.01\\
13.85	0.01\\
13.86	0.01\\
13.87	0.01\\
13.88	0.01\\
13.89	0.01\\
13.9	0.01\\
13.91	0.01\\
13.92	0.01\\
13.93	0.01\\
13.94	0.01\\
13.95	0.01\\
13.96	0.01\\
13.97	0.01\\
13.98	0.01\\
13.99	0.01\\
14	0.01\\
14.01	0.01\\
14.02	0.01\\
14.03	0.01\\
14.04	0.01\\
14.05	0.01\\
14.06	0.01\\
14.07	0.01\\
14.08	0.01\\
14.09	0.01\\
14.1	0.01\\
14.11	0.01\\
14.12	0.01\\
14.13	0.01\\
14.14	0.01\\
14.15	0.01\\
14.16	0.01\\
14.17	0.01\\
14.18	0.01\\
14.19	0.01\\
14.2	0.01\\
14.21	0.01\\
14.22	0.01\\
14.23	0.01\\
14.24	0.01\\
14.25	0.01\\
14.26	0.01\\
14.27	0.01\\
14.28	0.01\\
14.29	0.01\\
14.3	0.01\\
14.31	0.01\\
14.32	0.01\\
14.33	0.01\\
14.34	0.01\\
14.35	0.01\\
14.36	0.01\\
14.37	0.01\\
14.38	0.01\\
14.39	0.01\\
14.4	0.01\\
14.41	0.01\\
14.42	0.01\\
14.43	0.01\\
14.44	0.01\\
14.45	0.01\\
14.46	0.01\\
14.47	0.01\\
14.48	0.01\\
14.49	0.01\\
14.5	0.01\\
14.51	0.01\\
14.52	0.01\\
14.53	0.01\\
14.54	0.01\\
14.55	0.01\\
14.56	0.01\\
14.57	0.01\\
14.58	0.01\\
14.59	0.01\\
14.6	0.01\\
14.61	0.01\\
14.62	0.01\\
14.63	0.01\\
14.64	0.01\\
14.65	0.01\\
14.66	0.01\\
14.67	0.01\\
14.68	0.01\\
14.69	0.01\\
14.7	0.01\\
14.71	0.01\\
14.72	0.01\\
14.73	0.01\\
14.74	0.01\\
14.75	0.01\\
14.76	0.01\\
14.77	0.01\\
14.78	0.01\\
14.79	0.01\\
14.8	0.01\\
14.81	0.01\\
14.82	0.01\\
14.83	0.01\\
14.84	0.01\\
14.85	0.01\\
14.86	0.01\\
14.87	0.01\\
14.88	0.01\\
14.89	0.01\\
14.9	0.01\\
14.91	0.01\\
14.92	0.01\\
14.93	0.01\\
14.94	0.01\\
14.95	0.01\\
14.96	0.01\\
14.97	0.01\\
14.98	0.01\\
14.99	0.01\\
15	0.01\\
15.01	0.01\\
15.02	0.01\\
15.03	0.01\\
15.04	0.01\\
15.05	0.01\\
15.06	0.01\\
15.07	0.01\\
15.08	0.01\\
15.09	0.01\\
15.1	0.01\\
15.11	0.01\\
15.12	0.01\\
15.13	0.01\\
15.14	0.01\\
15.15	0.01\\
15.16	0.01\\
15.17	0.01\\
15.18	0.01\\
15.19	0.01\\
15.2	0.01\\
15.21	0.01\\
15.22	0.01\\
15.23	0.01\\
15.24	0.01\\
15.25	0.01\\
15.26	0.01\\
15.27	0.01\\
15.28	0.01\\
15.29	0.01\\
15.3	0.01\\
15.31	0.01\\
15.32	0.01\\
15.33	0.01\\
15.34	0.01\\
15.35	0.01\\
15.36	0.01\\
15.37	0.01\\
15.38	0.01\\
15.39	0.01\\
15.4	0.01\\
15.41	0.01\\
15.42	0.01\\
15.43	0.01\\
15.44	0.01\\
15.45	0.01\\
15.46	0.01\\
15.47	0.01\\
15.48	0.01\\
15.49	0.01\\
15.5	0.01\\
15.51	0.01\\
15.52	0.01\\
15.53	0.01\\
15.54	0.01\\
15.55	0.01\\
15.56	0.01\\
15.57	0.01\\
15.58	0.01\\
15.59	0.01\\
15.6	0.01\\
15.61	0.01\\
15.62	0.01\\
15.63	0.01\\
15.64	0.01\\
15.65	0.01\\
15.66	0.01\\
15.67	0.01\\
15.68	0.01\\
15.69	0.01\\
15.7	0.01\\
15.71	0.01\\
15.72	0.01\\
15.73	0.01\\
15.74	0.01\\
15.75	0.01\\
15.76	0.01\\
15.77	0.01\\
15.78	0.01\\
15.79	0.01\\
15.8	0.01\\
15.81	0.01\\
15.82	0.01\\
15.83	0.01\\
15.84	0.01\\
15.85	0.01\\
15.86	0.01\\
15.87	0.01\\
15.88	0.01\\
15.89	0.01\\
15.9	0.01\\
15.91	0.01\\
15.92	0.01\\
15.93	0.01\\
15.94	0.01\\
15.95	0.01\\
15.96	0.01\\
15.97	0.01\\
15.98	0.01\\
15.99	0.01\\
16	0.01\\
16.01	0.01\\
16.02	0.01\\
16.03	0.01\\
16.04	0.01\\
16.05	0.01\\
16.06	0.01\\
16.07	0.01\\
16.08	0.01\\
16.09	0.01\\
16.1	0.01\\
16.11	0.01\\
16.12	0.01\\
16.13	0.01\\
16.14	0.01\\
16.15	0.01\\
16.16	0.01\\
16.17	0.01\\
16.18	0.01\\
16.19	0.01\\
16.2	0.01\\
16.21	0.01\\
16.22	0.01\\
16.23	0.01\\
16.24	0.01\\
16.25	0.01\\
16.26	0.01\\
16.27	0.01\\
16.28	0.01\\
16.29	0.01\\
16.3	0.01\\
16.31	0.01\\
16.32	0.01\\
16.33	0.01\\
16.34	0.01\\
16.35	0.01\\
16.36	0.01\\
16.37	0.01\\
16.38	0.01\\
16.39	0.01\\
16.4	0.01\\
16.41	0.01\\
16.42	0.01\\
16.43	0.01\\
16.44	0.01\\
16.45	0.01\\
16.46	0.01\\
16.47	0.01\\
16.48	0.01\\
16.49	0.01\\
16.5	0.01\\
16.51	0.01\\
16.52	0.01\\
16.53	0.01\\
16.54	0.01\\
16.55	0.01\\
16.56	0.01\\
16.57	0.01\\
16.58	0.01\\
16.59	0.01\\
16.6	0.01\\
16.61	0.01\\
16.62	0.01\\
16.63	0.01\\
16.64	0.01\\
16.65	0.01\\
16.66	0.01\\
16.67	0.01\\
16.68	0.01\\
16.69	0.01\\
16.7	0.01\\
16.71	0.01\\
16.72	0.01\\
16.73	0.01\\
16.74	0.01\\
16.75	0.01\\
16.76	0.01\\
16.77	0.01\\
16.78	0.01\\
16.79	0.01\\
16.8	0.01\\
16.81	0.01\\
16.82	0.01\\
16.83	0.01\\
16.84	0.01\\
16.85	0.01\\
16.86	0.01\\
16.87	0.01\\
16.88	0.01\\
16.89	0.01\\
16.9	0.01\\
16.91	0.01\\
16.92	0.01\\
16.93	0.01\\
16.94	0.01\\
16.95	0.01\\
16.96	0.01\\
16.97	0.01\\
16.98	0.01\\
16.99	0.01\\
17	0.01\\
17.01	0.01\\
17.02	0.01\\
17.03	0.01\\
17.04	0.01\\
17.05	0.01\\
17.06	0.01\\
17.07	0.01\\
17.08	0.01\\
17.09	0.01\\
17.1	0.01\\
17.11	0.01\\
17.12	0.01\\
17.13	0.01\\
17.14	0.01\\
17.15	0.01\\
17.16	0.01\\
17.17	0.01\\
17.18	0.01\\
17.19	0.01\\
17.2	0.01\\
17.21	0.01\\
17.22	0.01\\
17.23	0.01\\
17.24	0.01\\
17.25	0.01\\
17.26	0.01\\
17.27	0.01\\
17.28	0.01\\
17.29	0.01\\
17.3	0.01\\
17.31	0.01\\
17.32	0.01\\
17.33	0.01\\
17.34	0.01\\
17.35	0.01\\
17.36	0.01\\
17.37	0.01\\
17.38	0.01\\
17.39	0.01\\
17.4	0.01\\
17.41	0.01\\
17.42	0.01\\
17.43	0.01\\
17.44	0.01\\
17.45	0.01\\
17.46	0.01\\
17.47	0.01\\
17.48	0.01\\
17.49	0.01\\
17.5	0.01\\
17.51	0.01\\
17.52	0.01\\
17.53	0.01\\
17.54	0.01\\
17.55	0.01\\
17.56	0.01\\
17.57	0.01\\
17.58	0.01\\
17.59	0.01\\
17.6	0.01\\
17.61	0.01\\
17.62	0.01\\
17.63	0.01\\
17.64	0.01\\
17.65	0.01\\
17.66	0.01\\
17.67	0.01\\
17.68	0.01\\
17.69	0.01\\
17.7	0.01\\
17.71	0.01\\
17.72	0.01\\
17.73	0.01\\
17.74	0.01\\
17.75	0.01\\
17.76	0.01\\
17.77	0.01\\
17.78	0.01\\
17.79	0.01\\
17.8	0.01\\
17.81	0.01\\
17.82	0.01\\
17.83	0.01\\
17.84	0.01\\
17.85	0.01\\
17.86	0.01\\
17.87	0.01\\
17.88	0.01\\
17.89	0.01\\
17.9	0.01\\
17.91	0.01\\
17.92	0.01\\
17.93	0.01\\
17.94	0.01\\
17.95	0.01\\
17.96	0.01\\
17.97	0.01\\
17.98	0.01\\
17.99	0.01\\
18	0.01\\
18.01	0.01\\
18.02	0.01\\
18.03	0.01\\
18.04	0.01\\
18.05	0.01\\
18.06	0.01\\
18.07	0.01\\
18.08	0.01\\
18.09	0.01\\
18.1	0.01\\
18.11	0.01\\
18.12	0.01\\
18.13	0.01\\
18.14	0.01\\
18.15	0.01\\
18.16	0.01\\
18.17	0.01\\
18.18	0.01\\
18.19	0.01\\
18.2	0.01\\
18.21	0.01\\
18.22	0.01\\
18.23	0.01\\
18.24	0.01\\
18.25	0.01\\
18.26	0.01\\
18.27	0.01\\
18.28	0.01\\
18.29	0.01\\
18.3	0.01\\
18.31	0.01\\
18.32	0.01\\
18.33	0.01\\
18.34	0.01\\
18.35	0.01\\
18.36	0.01\\
18.37	0.01\\
18.38	0.01\\
18.39	0.01\\
18.4	0.01\\
18.41	0.01\\
18.42	0.01\\
18.43	0.01\\
18.44	0.01\\
18.45	0.01\\
18.46	0.01\\
18.47	0.01\\
18.48	0.01\\
18.49	0.01\\
18.5	0.01\\
18.51	0.01\\
18.52	0.01\\
18.53	0.01\\
18.54	0.01\\
18.55	0.01\\
18.56	0.01\\
18.57	0.01\\
18.58	0.01\\
18.59	0.01\\
18.6	0.01\\
18.61	0.01\\
18.62	0.01\\
18.63	0.01\\
18.64	0.01\\
18.65	0.01\\
18.66	0.01\\
18.67	0.01\\
18.68	0.01\\
18.69	0.01\\
18.7	0.01\\
18.71	0.01\\
18.72	0.01\\
18.73	0.01\\
18.74	0.01\\
18.75	0.01\\
18.76	0.01\\
18.77	0.01\\
18.78	0.01\\
18.79	0.01\\
18.8	0.01\\
18.81	0.01\\
18.82	0.01\\
18.83	0.01\\
18.84	0.01\\
18.85	0.01\\
18.86	0.01\\
18.87	0.01\\
18.88	0.01\\
18.89	0.01\\
18.9	0.01\\
18.91	0.01\\
18.92	0.01\\
18.93	0.01\\
18.94	0.01\\
18.95	0.01\\
18.96	0.01\\
18.97	0.01\\
18.98	0.01\\
18.99	0.01\\
19	0.01\\
19.01	0.01\\
19.02	0.01\\
19.03	0.01\\
19.04	0.01\\
19.05	0.01\\
19.06	0.01\\
19.07	0.01\\
19.08	0.01\\
19.09	0.01\\
19.1	0.01\\
19.11	0.01\\
19.12	0.01\\
19.13	0.01\\
19.14	0.01\\
19.15	0.01\\
19.16	0.01\\
19.17	0.01\\
19.18	0.01\\
19.19	0.01\\
19.2	0.01\\
19.21	0.01\\
19.22	0.01\\
19.23	0.01\\
19.24	0.01\\
19.25	0.01\\
19.26	0.01\\
19.27	0.01\\
19.28	0.01\\
19.29	0.01\\
19.3	0.01\\
19.31	0.01\\
19.32	0.01\\
19.33	0.01\\
19.34	0.01\\
19.35	0.01\\
19.36	0.01\\
19.37	0.01\\
19.38	0.01\\
19.39	0.01\\
19.4	0.01\\
19.41	0.01\\
19.42	0.01\\
19.43	0.01\\
19.44	0.01\\
19.45	0.01\\
19.46	0.01\\
19.47	0.01\\
19.48	0.01\\
19.49	0.01\\
19.5	0.01\\
19.51	0.01\\
19.52	0.01\\
19.53	0.01\\
19.54	0.01\\
19.55	0.01\\
19.56	0.01\\
19.57	0.01\\
19.58	0.01\\
19.59	0.01\\
19.6	0.01\\
19.61	0.01\\
19.62	0.01\\
19.63	0.01\\
19.64	0.01\\
19.65	0.01\\
19.66	0.01\\
19.67	0.01\\
19.68	0.01\\
19.69	0.01\\
19.7	0.01\\
19.71	0.01\\
19.72	0.01\\
19.73	0.01\\
19.74	0.01\\
19.75	0.01\\
19.76	0.01\\
19.77	0.01\\
19.78	0.01\\
19.79	0.01\\
19.8	0.01\\
19.81	0.01\\
19.82	0.01\\
19.83	0.01\\
19.84	0.01\\
19.85	0.01\\
19.86	0.01\\
19.87	0.01\\
19.88	0.01\\
19.89	0.01\\
19.9	0.01\\
19.91	0.01\\
19.92	0.01\\
19.93	0.01\\
19.94	0.01\\
19.95	0.01\\
19.96	0.01\\
19.97	0.01\\
19.98	0.01\\
19.99	0.01\\
20	0.01\\
20.01	0.01\\
20.02	0.01\\
20.03	0.01\\
20.04	0.01\\
20.05	0.01\\
20.06	0.01\\
20.07	0.01\\
20.08	0.01\\
20.09	0.01\\
20.1	0.01\\
20.11	0.01\\
20.12	0.01\\
20.13	0.01\\
20.14	0.01\\
20.15	0.01\\
20.16	0.01\\
20.17	0.01\\
20.18	0.01\\
20.19	0.01\\
20.2	0.01\\
20.21	0.01\\
20.22	0.01\\
20.23	0.01\\
20.24	0.01\\
20.25	0.01\\
20.26	0.01\\
20.27	0.01\\
20.28	0.01\\
20.29	0.01\\
20.3	0.01\\
20.31	0.01\\
20.32	0.01\\
20.33	0.01\\
20.34	0.01\\
20.35	0.01\\
20.36	0.01\\
20.37	0.01\\
20.38	0.01\\
20.39	0.01\\
20.4	0.01\\
20.41	0.01\\
20.42	0.01\\
20.43	0.01\\
20.44	0.01\\
20.45	0.01\\
20.46	0.01\\
20.47	0.01\\
20.48	0.01\\
20.49	0.01\\
20.5	0.01\\
20.51	0.01\\
20.52	0.01\\
20.53	0.01\\
20.54	0.01\\
20.55	0.01\\
20.56	0.01\\
20.57	0.01\\
20.58	0.01\\
20.59	0.01\\
20.6	0.01\\
20.61	0.01\\
20.62	0.01\\
20.63	0.01\\
20.64	0.01\\
20.65	0.01\\
20.66	0.01\\
20.67	0.01\\
20.68	0.01\\
20.69	0.01\\
20.7	0.01\\
20.71	0.01\\
20.72	0.01\\
20.73	0.01\\
20.74	0.01\\
20.75	0.01\\
20.76	0.01\\
20.77	0.01\\
20.78	0.01\\
20.79	0.01\\
20.8	0.01\\
20.81	0.01\\
20.82	0.01\\
20.83	0.01\\
20.84	0.01\\
20.85	0.01\\
20.86	0.01\\
20.87	0.01\\
20.88	0.01\\
20.89	0.01\\
20.9	0.01\\
20.91	0.01\\
20.92	0.01\\
20.93	0.01\\
20.94	0.01\\
20.95	0.01\\
20.96	0.01\\
20.97	0.01\\
20.98	0.01\\
20.99	0.01\\
21	0.01\\
21.01	0.01\\
21.02	0.01\\
21.03	0.01\\
21.04	0.01\\
21.05	0.01\\
21.06	0.01\\
21.07	0.01\\
21.08	0.01\\
21.09	0.01\\
21.1	0.01\\
21.11	0.01\\
21.12	0.01\\
21.13	0.01\\
21.14	0.01\\
21.15	0.01\\
21.16	0.01\\
21.17	0.01\\
21.18	0.01\\
21.19	0.01\\
21.2	0.01\\
21.21	0.01\\
21.22	0.01\\
21.23	0.01\\
21.24	0.01\\
21.25	0.01\\
21.26	0.01\\
21.27	0.01\\
21.28	0.01\\
21.29	0.01\\
21.3	0.01\\
21.31	0.01\\
21.32	0.01\\
21.33	0.01\\
21.34	0.01\\
21.35	0.01\\
21.36	0.01\\
21.37	0.01\\
21.38	0.01\\
21.39	0.01\\
21.4	0.01\\
21.41	0.01\\
21.42	0.01\\
21.43	0.01\\
21.44	0.01\\
21.45	0.01\\
21.46	0.01\\
21.47	0.01\\
21.48	0.01\\
21.49	0.01\\
21.5	0.01\\
21.51	0.01\\
21.52	0.01\\
21.53	0.01\\
21.54	0.01\\
21.55	0.01\\
21.56	0.01\\
21.57	0.01\\
21.58	0.01\\
21.59	0.01\\
21.6	0.01\\
21.61	0.01\\
21.62	0.01\\
21.63	0.01\\
21.64	0.01\\
21.65	0.01\\
21.66	0.01\\
21.67	0.01\\
21.68	0.01\\
21.69	0.01\\
21.7	0.01\\
21.71	0.01\\
21.72	0.01\\
21.73	0.01\\
21.74	0.01\\
21.75	0.01\\
21.76	0.01\\
21.77	0.01\\
21.78	0.01\\
21.79	0.01\\
21.8	0.01\\
21.81	0.01\\
21.82	0.01\\
21.83	0.01\\
21.84	0.01\\
21.85	0.01\\
21.86	0.01\\
21.87	0.01\\
21.88	0.01\\
21.89	0.01\\
21.9	0.01\\
21.91	0.01\\
21.92	0.01\\
21.93	0.01\\
21.94	0.01\\
21.95	0.01\\
21.96	0.01\\
21.97	0.01\\
21.98	0.01\\
21.99	0.01\\
22	0.01\\
22.01	0.01\\
22.02	0.01\\
22.03	0.01\\
22.04	0.01\\
22.05	0.01\\
22.06	0.01\\
22.07	0.01\\
22.08	0.01\\
22.09	0.01\\
22.1	0.01\\
22.11	0.01\\
22.12	0.01\\
22.13	0.01\\
22.14	0.01\\
22.15	0.01\\
22.16	0.01\\
22.17	0.01\\
22.18	0.01\\
22.19	0.01\\
22.2	0.01\\
22.21	0.01\\
22.22	0.01\\
22.23	0.01\\
22.24	0.01\\
22.25	0.01\\
22.26	0.01\\
22.27	0.01\\
22.28	0.01\\
22.29	0.01\\
22.3	0.01\\
22.31	0.01\\
22.32	0.01\\
22.33	0.01\\
22.34	0.01\\
22.35	0.01\\
22.36	0.01\\
22.37	0.01\\
22.38	0.01\\
22.39	0.01\\
22.4	0.01\\
22.41	0.01\\
22.42	0.01\\
22.43	0.01\\
22.44	0.01\\
22.45	0.01\\
22.46	0.01\\
22.47	0.01\\
22.48	0.01\\
22.49	0.01\\
22.5	0.01\\
22.51	0.01\\
22.52	0.01\\
22.53	0.01\\
22.54	0.01\\
22.55	0.01\\
22.56	0.01\\
22.57	0.01\\
22.58	0.01\\
22.59	0.01\\
22.6	0.01\\
22.61	0.01\\
22.62	0.01\\
22.63	0.01\\
22.64	0.01\\
22.65	0.01\\
22.66	0.01\\
22.67	0.01\\
22.68	0.01\\
22.69	0.01\\
22.7	0.01\\
22.71	0.01\\
22.72	0.01\\
22.73	0.01\\
22.74	0.01\\
22.75	0.01\\
22.76	0.01\\
22.77	0.01\\
22.78	0.01\\
22.79	0.01\\
22.8	0.01\\
22.81	0.01\\
22.82	0.01\\
22.83	0.01\\
22.84	0.01\\
22.85	0.01\\
22.86	0.01\\
22.87	0.01\\
22.88	0.01\\
22.89	0.01\\
22.9	0.01\\
22.91	0.01\\
22.92	0.01\\
22.93	0.01\\
22.94	0.01\\
22.95	0.01\\
22.96	0.01\\
22.97	0.01\\
22.98	0.01\\
22.99	0.01\\
23	0.01\\
23.01	0.01\\
23.02	0.01\\
23.03	0.01\\
23.04	0.01\\
23.05	0.01\\
23.06	0.01\\
23.07	0.01\\
23.08	0.01\\
23.09	0.01\\
23.1	0.01\\
23.11	0.01\\
23.12	0.01\\
23.13	0.01\\
23.14	0.01\\
23.15	0.01\\
23.16	0.01\\
23.17	0.01\\
23.18	0.01\\
23.19	0.01\\
23.2	0.01\\
23.21	0.01\\
23.22	0.01\\
23.23	0.01\\
23.24	0.01\\
23.25	0.01\\
23.26	0.01\\
23.27	0.01\\
23.28	0.01\\
23.29	0.01\\
23.3	0.01\\
23.31	0.01\\
23.32	0.01\\
23.33	0.01\\
23.34	0.01\\
23.35	0.01\\
23.36	0.01\\
23.37	0.01\\
23.38	0.01\\
23.39	0.01\\
23.4	0.01\\
23.41	0.01\\
23.42	0.01\\
23.43	0.01\\
23.44	0.01\\
23.45	0.01\\
23.46	0.01\\
23.47	0.01\\
23.48	0.01\\
23.49	0.01\\
23.5	0.01\\
23.51	0.01\\
23.52	0.01\\
23.53	0.01\\
23.54	0.01\\
23.55	0.01\\
23.56	0.01\\
23.57	0.01\\
23.58	0.01\\
23.59	0.01\\
23.6	0.01\\
23.61	0.01\\
23.62	0.01\\
23.63	0.01\\
23.64	0.01\\
23.65	0.01\\
23.66	0.01\\
23.67	0.01\\
23.68	0.01\\
23.69	0.01\\
23.7	0.01\\
23.71	0.01\\
23.72	0.01\\
23.73	0.01\\
23.74	0.01\\
23.75	0.01\\
23.76	0.01\\
23.77	0.01\\
23.78	0.01\\
23.79	0.01\\
23.8	0.01\\
23.81	0.01\\
23.82	0.01\\
23.83	0.01\\
23.84	0.01\\
23.85	0.01\\
23.86	0.01\\
23.87	0.01\\
23.88	0.01\\
23.89	0.01\\
23.9	0.01\\
23.91	0.01\\
23.92	0.01\\
23.93	0.01\\
23.94	0.01\\
23.95	0.01\\
23.96	0.01\\
23.97	0.01\\
23.98	0.01\\
23.99	0.01\\
24	0.01\\
24.01	0.01\\
24.02	0.01\\
24.03	0.01\\
24.04	0.01\\
24.05	0.01\\
24.06	0.01\\
24.07	0.01\\
24.08	0.01\\
24.09	0.01\\
24.1	0.01\\
24.11	0.01\\
24.12	0.01\\
24.13	0.01\\
24.14	0.01\\
24.15	0.01\\
24.16	0.01\\
24.17	0.01\\
24.18	0.01\\
24.19	0.01\\
24.2	0.01\\
24.21	0.01\\
24.22	0.01\\
24.23	0.01\\
24.24	0.01\\
24.25	0.01\\
24.26	0.01\\
24.27	0.01\\
24.28	0.01\\
24.29	0.01\\
24.3	0.01\\
24.31	0.01\\
24.32	0.01\\
24.33	0.01\\
24.34	0.01\\
24.35	0.01\\
24.36	0.01\\
24.37	0.01\\
24.38	0.01\\
24.39	0.01\\
24.4	0.01\\
24.41	0.01\\
24.42	0.01\\
24.43	0.01\\
24.44	0.01\\
24.45	0.01\\
24.46	0.01\\
24.47	0.01\\
24.48	0.01\\
24.49	0.01\\
24.5	0.01\\
24.51	0.01\\
24.52	0.01\\
24.53	0.01\\
24.54	0.01\\
24.55	0.01\\
24.56	0.01\\
24.57	0.01\\
24.58	0.01\\
24.59	0.01\\
24.6	0.01\\
24.61	0.01\\
24.62	0.01\\
24.63	0.01\\
24.64	0.01\\
24.65	0.01\\
24.66	0.01\\
24.67	0.01\\
24.68	0.01\\
24.69	0.01\\
24.7	0.01\\
24.71	0.01\\
24.72	0.01\\
24.73	0.01\\
24.74	0.01\\
24.75	0.01\\
24.76	0.01\\
24.77	0.01\\
24.78	0.01\\
24.79	0.01\\
24.8	0.01\\
24.81	0.01\\
24.82	0.01\\
24.83	0.01\\
24.84	0.01\\
24.85	0.01\\
24.86	0.01\\
24.87	0.01\\
24.88	0.01\\
24.89	0.01\\
24.9	0.01\\
24.91	0.01\\
24.92	0.01\\
24.93	0.01\\
24.94	0.01\\
24.95	0.01\\
24.96	0.01\\
24.97	0.01\\
24.98	0.01\\
24.99	0.01\\
25	0.01\\
25.01	0.01\\
25.02	0.01\\
25.03	0.01\\
25.04	0.01\\
25.05	0.01\\
25.06	0.01\\
25.07	0.01\\
25.08	0.01\\
25.09	0.01\\
25.1	0.01\\
25.11	0.01\\
25.12	0.01\\
25.13	0.01\\
25.14	0.01\\
25.15	0.01\\
25.16	0.01\\
25.17	0.01\\
25.18	0.01\\
25.19	0.01\\
25.2	0.01\\
25.21	0.01\\
25.22	0.01\\
25.23	0.01\\
25.24	0.01\\
25.25	0.01\\
25.26	0.01\\
25.27	0.01\\
25.28	0.01\\
25.29	0.01\\
25.3	0.01\\
25.31	0.01\\
25.32	0.01\\
25.33	0.01\\
25.34	0.01\\
25.35	0.01\\
25.36	0.01\\
25.37	0.01\\
25.38	0.01\\
25.39	0.01\\
25.4	0.01\\
25.41	0.01\\
25.42	0.01\\
25.43	0.01\\
25.44	0.01\\
25.45	0.01\\
25.46	0.01\\
25.47	0.01\\
25.48	0.01\\
25.49	0.01\\
25.5	0.01\\
25.51	0.01\\
25.52	0.01\\
25.53	0.01\\
25.54	0.01\\
25.55	0.01\\
25.56	0.01\\
25.57	0.01\\
25.58	0.01\\
25.59	0.01\\
25.6	0.01\\
25.61	0.01\\
25.62	0.01\\
25.63	0.01\\
25.64	0.01\\
25.65	0.01\\
25.66	0.01\\
25.67	0.01\\
25.68	0.01\\
25.69	0.01\\
25.7	0.01\\
25.71	0.01\\
25.72	0.01\\
25.73	0.01\\
25.74	0.01\\
25.75	0.01\\
25.76	0.01\\
25.77	0.01\\
25.78	0.01\\
25.79	0.01\\
25.8	0.01\\
25.81	0.01\\
25.82	0.01\\
25.83	0.01\\
25.84	0.01\\
25.85	0.01\\
25.86	0.01\\
25.87	0.01\\
25.88	0.01\\
25.89	0.01\\
25.9	0.01\\
25.91	0.01\\
25.92	0.01\\
25.93	0.01\\
25.94	0.01\\
25.95	0.01\\
25.96	0.01\\
25.97	0.01\\
25.98	0.01\\
25.99	0.01\\
26	0.01\\
26.01	0.01\\
26.02	0.01\\
26.03	0.01\\
26.04	0.01\\
26.05	0.01\\
26.06	0.01\\
26.07	0.01\\
26.08	0.01\\
26.09	0.01\\
26.1	0.01\\
26.11	0.01\\
26.12	0.01\\
26.13	0.01\\
26.14	0.01\\
26.15	0.01\\
26.16	0.01\\
26.17	0.01\\
26.18	0.01\\
26.19	0.01\\
26.2	0.01\\
26.21	0.01\\
26.22	0.01\\
26.23	0.01\\
26.24	0.01\\
26.25	0.01\\
26.26	0.01\\
26.27	0.01\\
26.28	0.01\\
26.29	0.01\\
26.3	0.01\\
26.31	0.01\\
26.32	0.01\\
26.33	0.01\\
26.34	0.01\\
26.35	0.01\\
26.36	0.01\\
26.37	0.01\\
26.38	0.01\\
26.39	0.01\\
26.4	0.01\\
26.41	0.01\\
26.42	0.01\\
26.43	0.01\\
26.44	0.01\\
26.45	0.01\\
26.46	0.01\\
26.47	0.01\\
26.48	0.01\\
26.49	0.01\\
26.5	0.01\\
26.51	0.01\\
26.52	0.01\\
26.53	0.01\\
26.54	0.01\\
26.55	0.01\\
26.56	0.01\\
26.57	0.01\\
26.58	0.01\\
26.59	0.01\\
26.6	0.01\\
26.61	0.01\\
26.62	0.01\\
26.63	0.01\\
26.64	0.01\\
26.65	0.01\\
26.66	0.01\\
26.67	0.01\\
26.68	0.01\\
26.69	0.01\\
26.7	0.01\\
26.71	0.01\\
26.72	0.01\\
26.73	0.01\\
26.74	0.01\\
26.75	0.01\\
26.76	0.01\\
26.77	0.01\\
26.78	0.01\\
26.79	0.01\\
26.8	0.01\\
26.81	0.01\\
26.82	0.01\\
26.83	0.01\\
26.84	0.01\\
26.85	0.01\\
26.86	0.01\\
26.87	0.01\\
26.88	0.01\\
26.89	0.01\\
26.9	0.01\\
26.91	0.01\\
26.92	0.01\\
26.93	0.01\\
26.94	0.01\\
26.95	0.01\\
26.96	0.01\\
26.97	0.01\\
26.98	0.01\\
26.99	0.01\\
27	0.01\\
27.01	0.01\\
27.02	0.01\\
27.03	0.01\\
27.04	0.01\\
27.05	0.01\\
27.06	0.01\\
27.07	0.01\\
27.08	0.01\\
27.09	0.01\\
27.1	0.01\\
27.11	0.01\\
27.12	0.01\\
27.13	0.01\\
27.14	0.01\\
27.15	0.01\\
27.16	0.01\\
27.17	0.01\\
27.18	0.01\\
27.19	0.01\\
27.2	0.01\\
27.21	0.01\\
27.22	0.01\\
27.23	0.01\\
27.24	0.01\\
27.25	0.01\\
27.26	0.01\\
27.27	0.01\\
27.28	0.01\\
27.29	0.01\\
27.3	0.01\\
27.31	0.01\\
27.32	0.01\\
27.33	0.01\\
27.34	0.01\\
27.35	0.01\\
27.36	0.01\\
27.37	0.01\\
27.38	0.01\\
27.39	0.01\\
27.4	0.01\\
27.41	0.01\\
27.42	0.01\\
27.43	0.01\\
27.44	0.01\\
27.45	0.01\\
27.46	0.01\\
27.47	0.01\\
27.48	0.01\\
27.49	0.01\\
27.5	0.01\\
27.51	0.01\\
27.52	0.01\\
27.53	0.01\\
27.54	0.01\\
27.55	0.01\\
27.56	0.01\\
27.57	0.01\\
27.58	0.01\\
27.59	0.01\\
27.6	0.01\\
27.61	0.01\\
27.62	0.01\\
27.63	0.01\\
27.64	0.01\\
27.65	0.01\\
27.66	0.01\\
27.67	0.01\\
27.68	0.01\\
27.69	0.01\\
27.7	0.01\\
27.71	0.01\\
27.72	0.01\\
27.73	0.01\\
27.74	0.01\\
27.75	0.01\\
27.76	0.01\\
27.77	0.01\\
27.78	0.01\\
27.79	0.01\\
27.8	0.01\\
27.81	0.01\\
27.82	0.01\\
27.83	0.01\\
27.84	0.01\\
27.85	0.01\\
27.86	0.01\\
27.87	0.01\\
27.88	0.01\\
27.89	0.01\\
27.9	0.01\\
27.91	0.01\\
27.92	0.01\\
27.93	0.01\\
27.94	0.01\\
27.95	0.01\\
27.96	0.01\\
27.97	0.01\\
27.98	0.01\\
27.99	0.01\\
28	0.01\\
28.01	0.01\\
28.02	0.01\\
28.03	0.01\\
28.04	0.01\\
28.05	0.01\\
28.06	0.01\\
28.07	0.01\\
28.08	0.01\\
28.09	0.01\\
28.1	0.01\\
28.11	0.01\\
28.12	0.01\\
28.13	0.01\\
28.14	0.01\\
28.15	0.01\\
28.16	0.01\\
28.17	0.01\\
28.18	0.01\\
28.19	0.01\\
28.2	0.01\\
28.21	0.01\\
28.22	0.01\\
28.23	0.01\\
28.24	0.01\\
28.25	0.01\\
28.26	0.01\\
28.27	0.01\\
28.28	0.01\\
28.29	0.01\\
28.3	0.01\\
28.31	0.01\\
28.32	0.01\\
28.33	0.01\\
28.34	0.01\\
28.35	0.01\\
28.36	0.01\\
28.37	0.01\\
28.38	0.01\\
28.39	0.01\\
28.4	0.01\\
28.41	0.01\\
28.42	0.01\\
28.43	0.01\\
28.44	0.01\\
28.45	0.01\\
28.46	0.01\\
28.47	0.01\\
28.48	0.01\\
28.49	0.01\\
28.5	0.01\\
28.51	0.01\\
28.52	0.01\\
28.53	0.01\\
28.54	0.01\\
28.55	0.01\\
28.56	0.01\\
28.57	0.01\\
28.58	0.01\\
28.59	0.01\\
28.6	0.01\\
28.61	0.01\\
28.62	0.01\\
28.63	0.01\\
28.64	0.01\\
28.65	0.01\\
28.66	0.01\\
28.67	0.01\\
28.68	0.01\\
28.69	0.01\\
28.7	0.01\\
28.71	0.01\\
28.72	0.01\\
28.73	0.01\\
28.74	0.01\\
28.75	0.01\\
28.76	0.01\\
28.77	0.01\\
28.78	0.01\\
28.79	0.01\\
28.8	0.01\\
28.81	0.01\\
28.82	0.01\\
28.83	0.01\\
28.84	0.01\\
28.85	0.01\\
28.86	0.01\\
28.87	0.01\\
28.88	0.01\\
28.89	0.01\\
28.9	0.01\\
28.91	0.01\\
28.92	0.01\\
28.93	0.01\\
28.94	0.01\\
28.95	0.01\\
28.96	0.01\\
28.97	0.01\\
28.98	0.01\\
28.99	0.01\\
29	0.01\\
29.01	0.01\\
29.02	0.01\\
29.03	0.01\\
29.04	0.01\\
29.05	0.01\\
29.06	0.01\\
29.07	0.01\\
29.08	0.01\\
29.09	0.01\\
29.1	0.01\\
29.11	0.01\\
29.12	0.01\\
29.13	0.01\\
29.14	0.01\\
29.15	0.01\\
29.16	0.01\\
29.17	0.01\\
29.18	0.01\\
29.19	0.01\\
29.2	0.01\\
29.21	0.01\\
29.22	0.01\\
29.23	0.01\\
29.24	0.01\\
29.25	0.01\\
29.26	0.01\\
29.27	0.01\\
29.28	0.01\\
29.29	0.01\\
29.3	0.01\\
29.31	0.01\\
29.32	0.01\\
29.33	0.01\\
29.34	0.01\\
29.35	0.01\\
29.36	0.01\\
29.37	0.01\\
29.38	0.01\\
29.39	0.01\\
29.4	0.01\\
29.41	0.01\\
29.42	0.01\\
29.43	0.01\\
29.44	0.01\\
29.45	0.01\\
29.46	0.01\\
29.47	0.01\\
29.48	0.01\\
29.49	0.01\\
29.5	0.01\\
29.51	0.01\\
29.52	0.01\\
29.53	0.01\\
29.54	0.01\\
29.55	0.01\\
29.56	0.01\\
29.57	0.01\\
29.58	0.01\\
29.59	0.01\\
29.6	0.01\\
29.61	0.01\\
29.62	0.01\\
29.63	0.01\\
29.64	0.01\\
29.65	0.01\\
29.66	0.01\\
29.67	0.01\\
29.68	0.01\\
29.69	0.01\\
29.7	0.01\\
29.71	0.01\\
29.72	0.01\\
29.73	0.01\\
29.74	0.01\\
29.75	0.01\\
29.76	0.01\\
29.77	0.01\\
29.78	0.01\\
29.79	0.01\\
29.8	0.01\\
29.81	0.01\\
29.82	0.01\\
29.83	0.01\\
29.84	0.01\\
29.85	0.01\\
29.86	0.01\\
29.87	0.01\\
29.88	0.01\\
29.89	0.01\\
29.9	0.01\\
29.91	0.01\\
29.92	0.01\\
29.93	0.01\\
29.94	0.01\\
29.95	0.01\\
29.96	0.01\\
29.97	0.01\\
29.98	0.01\\
29.99	0.01\\
30	0.01\\
30.01	0.01\\
30.02	0.01\\
30.03	0.01\\
30.04	0.01\\
30.05	0.01\\
30.06	0.01\\
30.07	0.01\\
30.08	0.01\\
30.09	0.01\\
30.1	0.01\\
30.11	0.01\\
30.12	0.01\\
30.13	0.01\\
30.14	0.01\\
30.15	0.01\\
30.16	0.01\\
30.17	0.01\\
30.18	0.01\\
30.19	0.01\\
30.2	0.01\\
30.21	0.01\\
30.22	0.01\\
30.23	0.01\\
30.24	0.01\\
30.25	0.01\\
30.26	0.01\\
30.27	0.01\\
30.28	0.01\\
30.29	0.01\\
30.3	0.01\\
30.31	0.01\\
30.32	0.01\\
30.33	0.01\\
30.34	0.01\\
30.35	0.01\\
30.36	0.01\\
30.37	0.01\\
30.38	0.01\\
30.39	0.01\\
30.4	0.01\\
30.41	0.01\\
30.42	0.01\\
30.43	0.01\\
30.44	0.01\\
30.45	0.01\\
30.46	0.01\\
30.47	0.01\\
30.48	0.01\\
30.49	0.01\\
30.5	0.01\\
30.51	0.01\\
30.52	0.01\\
30.53	0.01\\
30.54	0.01\\
30.55	0.01\\
30.56	0.01\\
30.57	0.01\\
30.58	0.01\\
30.59	0.01\\
30.6	0.01\\
30.61	0.01\\
30.62	0.01\\
30.63	0.01\\
30.64	0.01\\
30.65	0.01\\
30.66	0.01\\
30.67	0.01\\
30.68	0.01\\
30.69	0.01\\
30.7	0.01\\
30.71	0.01\\
30.72	0.01\\
30.73	0.01\\
30.74	0.01\\
30.75	0.01\\
30.76	0.01\\
30.77	0.01\\
30.78	0.01\\
30.79	0.01\\
30.8	0.01\\
30.81	0.01\\
30.82	0.01\\
30.83	0.01\\
30.84	0.01\\
30.85	0.01\\
30.86	0.01\\
30.87	0.01\\
30.88	0.01\\
30.89	0.01\\
30.9	0.01\\
30.91	0.01\\
30.92	0.01\\
30.93	0.01\\
30.94	0.01\\
30.95	0.01\\
30.96	0.01\\
30.97	0.01\\
30.98	0.01\\
30.99	0.01\\
31	0.01\\
31.01	0.01\\
31.02	0.01\\
31.03	0.01\\
31.04	0.01\\
31.05	0.01\\
31.06	0.01\\
31.07	0.01\\
31.08	0.01\\
31.09	0.01\\
31.1	0.01\\
31.11	0.01\\
31.12	0.01\\
31.13	0.01\\
31.14	0.01\\
31.15	0.01\\
31.16	0.01\\
31.17	0.01\\
31.18	0.01\\
31.19	0.01\\
31.2	0.01\\
31.21	0.01\\
31.22	0.01\\
31.23	0.01\\
31.24	0.01\\
31.25	0.01\\
31.26	0.01\\
31.27	0.01\\
31.28	0.01\\
31.29	0.01\\
31.3	0.01\\
31.31	0.01\\
31.32	0.01\\
31.33	0.01\\
31.34	0.01\\
31.35	0.01\\
31.36	0.01\\
31.37	0.01\\
31.38	0.01\\
31.39	0.01\\
31.4	0.01\\
31.41	0.01\\
31.42	0.01\\
31.43	0.01\\
31.44	0.01\\
31.45	0.01\\
31.46	0.01\\
31.47	0.01\\
31.48	0.01\\
31.49	0.01\\
31.5	0.01\\
31.51	0.01\\
31.52	0.01\\
31.53	0.01\\
31.54	0.01\\
31.55	0.01\\
31.56	0.01\\
31.57	0.01\\
31.58	0.01\\
31.59	0.01\\
31.6	0.01\\
31.61	0.01\\
31.62	0.01\\
31.63	0.01\\
31.64	0.01\\
31.65	0.01\\
31.66	0.01\\
31.67	0.01\\
31.68	0.01\\
31.69	0.01\\
31.7	0.01\\
31.71	0.01\\
31.72	0.01\\
31.73	0.01\\
31.74	0.01\\
31.75	0.01\\
31.76	0.01\\
31.77	0.01\\
31.78	0.01\\
31.79	0.01\\
31.8	0.01\\
31.81	0.01\\
31.82	0.01\\
31.83	0.01\\
31.84	0.01\\
31.85	0.01\\
31.86	0.01\\
31.87	0.01\\
31.88	0.01\\
31.89	0.01\\
31.9	0.01\\
31.91	0.01\\
31.92	0.01\\
31.93	0.01\\
31.94	0.01\\
31.95	0.01\\
31.96	0.01\\
31.97	0.01\\
31.98	0.01\\
31.99	0.01\\
32	0.01\\
32.01	0.01\\
32.02	0.01\\
32.03	0.01\\
32.04	0.01\\
32.05	0.01\\
32.06	0.01\\
32.07	0.01\\
32.08	0.01\\
32.09	0.01\\
32.1	0.01\\
32.11	0.01\\
32.12	0.01\\
32.13	0.01\\
32.14	0.01\\
32.15	0.01\\
32.16	0.01\\
32.17	0.01\\
32.18	0.01\\
32.19	0.01\\
32.2	0.01\\
32.21	0.01\\
32.22	0.01\\
32.23	0.01\\
32.24	0.01\\
32.25	0.01\\
32.26	0.01\\
32.27	0.01\\
32.28	0.01\\
32.29	0.01\\
32.3	0.01\\
32.31	0.01\\
32.32	0.01\\
32.33	0.01\\
32.34	0.01\\
32.35	0.01\\
32.36	0.01\\
32.37	0.01\\
32.38	0.01\\
32.39	0.01\\
32.4	0.01\\
32.41	0.01\\
32.42	0.01\\
32.43	0.01\\
32.44	0.01\\
32.45	0.01\\
32.46	0.01\\
32.47	0.01\\
32.48	0.01\\
32.49	0.01\\
32.5	0.01\\
32.51	0.01\\
32.52	0.01\\
32.53	0.01\\
32.54	0.01\\
32.55	0.01\\
32.56	0.01\\
32.57	0.01\\
32.58	0.01\\
32.59	0.01\\
32.6	0.01\\
32.61	0.01\\
32.62	0.01\\
32.63	0.01\\
32.64	0.01\\
32.65	0.01\\
32.66	0.01\\
32.67	0.01\\
32.68	0.01\\
32.69	0.01\\
32.7	0.01\\
32.71	0.01\\
32.72	0.01\\
32.73	0.01\\
32.74	0.01\\
32.75	0.01\\
32.76	0.01\\
32.77	0.01\\
32.78	0.01\\
32.79	0.01\\
32.8	0.01\\
32.81	0.01\\
32.82	0.01\\
32.83	0.01\\
32.84	0.01\\
32.85	0.01\\
32.86	0.01\\
32.87	0.01\\
32.88	0.01\\
32.89	0.01\\
32.9	0.01\\
32.91	0.01\\
32.92	0.01\\
32.93	0.01\\
32.94	0.01\\
32.95	0.01\\
32.96	0.01\\
32.97	0.01\\
32.98	0.01\\
32.99	0.01\\
33	0.01\\
33.01	0.01\\
33.02	0.01\\
33.03	0.01\\
33.04	0.01\\
33.05	0.01\\
33.06	0.01\\
33.07	0.01\\
33.08	0.01\\
33.09	0.01\\
33.1	0.01\\
33.11	0.01\\
33.12	0.01\\
33.13	0.01\\
33.14	0.01\\
33.15	0.01\\
33.16	0.01\\
33.17	0.01\\
33.18	0.01\\
33.19	0.01\\
33.2	0.01\\
33.21	0.01\\
33.22	0.01\\
33.23	0.01\\
33.24	0.01\\
33.25	0.01\\
33.26	0.01\\
33.27	0.01\\
33.28	0.01\\
33.29	0.01\\
33.3	0.01\\
33.31	0.01\\
33.32	0.01\\
33.33	0.01\\
33.34	0.01\\
33.35	0.01\\
33.36	0.01\\
33.37	0.01\\
33.38	0.01\\
33.39	0.01\\
33.4	0.01\\
33.41	0.01\\
33.42	0.01\\
33.43	0.01\\
33.44	0.01\\
33.45	0.01\\
33.46	0.01\\
33.47	0.01\\
33.48	0.01\\
33.49	0.01\\
33.5	0.01\\
33.51	0.01\\
33.52	0.01\\
33.53	0.01\\
33.54	0.01\\
33.55	0.01\\
33.56	0.01\\
33.57	0.01\\
33.58	0.01\\
33.59	0.01\\
33.6	0.01\\
33.61	0.01\\
33.62	0.01\\
33.63	0.01\\
33.64	0.01\\
33.65	0.01\\
33.66	0.01\\
33.67	0.01\\
33.68	0.01\\
33.69	0.01\\
33.7	0.01\\
33.71	0.01\\
33.72	0.01\\
33.73	0.01\\
33.74	0.01\\
33.75	0.01\\
33.76	0.01\\
33.77	0.01\\
33.78	0.01\\
33.79	0.01\\
33.8	0.01\\
33.81	0.01\\
33.82	0.01\\
33.83	0.01\\
33.84	0.01\\
33.85	0.01\\
33.86	0.01\\
33.87	0.01\\
33.88	0.01\\
33.89	0.01\\
33.9	0.01\\
33.91	0.01\\
33.92	0.01\\
33.93	0.01\\
33.94	0.01\\
33.95	0.01\\
33.96	0.01\\
33.97	0.01\\
33.98	0.01\\
33.99	0.01\\
34	0.01\\
34.01	0.01\\
34.02	0.01\\
34.03	0.01\\
34.04	0.01\\
34.05	0.01\\
34.06	0.01\\
34.07	0.01\\
34.08	0.01\\
34.09	0.01\\
34.1	0.01\\
34.11	0.01\\
34.12	0.01\\
34.13	0.01\\
34.14	0.01\\
34.15	0.01\\
34.16	0.01\\
34.17	0.01\\
34.18	0.01\\
34.19	0.01\\
34.2	0.01\\
34.21	0.01\\
34.22	0.01\\
34.23	0.01\\
34.24	0.01\\
34.25	0.01\\
34.26	0.01\\
34.27	0.01\\
34.28	0.01\\
34.29	0.01\\
34.3	0.01\\
34.31	0.01\\
34.32	0.01\\
34.33	0.01\\
34.34	0.01\\
34.35	0.01\\
34.36	0.01\\
34.37	0.01\\
34.38	0.01\\
34.39	0.01\\
34.4	0.01\\
34.41	0.01\\
34.42	0.01\\
34.43	0.01\\
34.44	0.01\\
34.45	0.01\\
34.46	0.01\\
34.47	0.01\\
34.48	0.01\\
34.49	0.01\\
34.5	0.01\\
34.51	0.01\\
34.52	0.01\\
34.53	0.01\\
34.54	0.01\\
34.55	0.01\\
34.56	0.01\\
34.57	0.01\\
34.58	0.01\\
34.59	0.01\\
34.6	0.01\\
34.61	0.01\\
34.62	0.01\\
34.63	0.01\\
34.64	0.01\\
34.65	0.01\\
34.66	0.01\\
34.67	0.01\\
34.68	0.01\\
34.69	0.01\\
34.7	0.01\\
34.71	0.01\\
34.72	0.01\\
34.73	0.01\\
34.74	0.01\\
34.75	0.01\\
34.76	0.01\\
34.77	0.01\\
34.78	0.01\\
34.79	0.01\\
34.8	0.01\\
34.81	0.01\\
34.82	0.01\\
34.83	0.01\\
34.84	0.01\\
34.85	0.01\\
34.86	0.01\\
34.87	0.01\\
34.88	0.01\\
34.89	0.01\\
34.9	0.01\\
34.91	0.01\\
34.92	0.01\\
34.93	0.01\\
34.94	0.01\\
34.95	0.01\\
34.96	0.01\\
34.97	0.01\\
34.98	0.01\\
34.99	0.01\\
35	0.01\\
35.01	0.01\\
35.02	0.01\\
35.03	0.01\\
35.04	0.01\\
35.05	0.01\\
35.06	0.01\\
35.07	0.01\\
35.08	0.01\\
35.09	0.01\\
35.1	0.01\\
35.11	0.01\\
35.12	0.01\\
35.13	0.01\\
35.14	0.01\\
35.15	0.01\\
35.16	0.01\\
35.17	0.01\\
35.18	0.01\\
35.19	0.01\\
35.2	0.01\\
35.21	0.01\\
35.22	0.01\\
35.23	0.01\\
35.24	0.01\\
35.25	0.01\\
35.26	0.01\\
35.27	0.01\\
35.28	0.01\\
35.29	0.01\\
35.3	0.01\\
35.31	0.01\\
35.32	0.01\\
35.33	0.01\\
35.34	0.01\\
35.35	0.01\\
35.36	0.01\\
35.37	0.01\\
35.38	0.01\\
35.39	0.01\\
35.4	0.01\\
35.41	0.01\\
35.42	0.01\\
35.43	0.01\\
35.44	0.01\\
35.45	0.01\\
35.46	0.01\\
35.47	0.01\\
35.48	0.01\\
35.49	0.01\\
35.5	0.01\\
35.51	0.01\\
35.52	0.01\\
35.53	0.01\\
35.54	0.01\\
35.55	0.01\\
35.56	0.01\\
35.57	0.01\\
35.58	0.01\\
35.59	0.01\\
35.6	0.01\\
35.61	0.01\\
35.62	0.01\\
35.63	0.01\\
35.64	0.01\\
35.65	0.01\\
35.66	0.01\\
35.67	0.01\\
35.68	0.01\\
35.69	0.01\\
35.7	0.01\\
35.71	0.01\\
35.72	0.01\\
35.73	0.01\\
35.74	0.01\\
35.75	0.01\\
35.76	0.01\\
35.77	0.01\\
35.78	0.01\\
35.79	0.01\\
35.8	0.01\\
35.81	0.01\\
35.82	0.01\\
35.83	0.01\\
35.84	0.01\\
35.85	0.01\\
35.86	0.01\\
35.87	0.01\\
35.88	0.01\\
35.89	0.01\\
35.9	0.01\\
35.91	0.01\\
35.92	0.01\\
35.93	0.01\\
35.94	0.01\\
35.95	0.01\\
35.96	0.01\\
35.97	0.01\\
35.98	0.01\\
35.99	0.01\\
36	0.01\\
36.01	0.01\\
36.02	0.01\\
36.03	0.01\\
36.04	0.01\\
36.05	0.01\\
36.06	0.01\\
36.07	0.01\\
36.08	0.01\\
36.09	0.01\\
36.1	0.01\\
36.11	0.01\\
36.12	0.01\\
36.13	0.01\\
36.14	0.01\\
36.15	0.01\\
36.16	0.01\\
36.17	0.01\\
36.18	0.01\\
36.19	0.01\\
36.2	0.01\\
36.21	0.01\\
36.22	0.01\\
36.23	0.01\\
36.24	0.01\\
36.25	0.01\\
36.26	0.01\\
36.27	0.01\\
36.28	0.01\\
36.29	0.01\\
36.3	0.01\\
36.31	0.01\\
36.32	0.01\\
36.33	0.01\\
36.34	0.01\\
36.35	0.01\\
36.36	0.01\\
36.37	0.01\\
36.38	0.01\\
36.39	0.01\\
36.4	0.01\\
36.41	0.01\\
36.42	0.01\\
36.43	0.01\\
36.44	0.01\\
36.45	0.01\\
36.46	0.01\\
36.47	0.01\\
36.48	0.01\\
36.49	0.01\\
36.5	0.01\\
36.51	0.01\\
36.52	0.01\\
36.53	0.01\\
36.54	0.01\\
36.55	0.01\\
36.56	0.01\\
36.57	0.01\\
36.58	0.01\\
36.59	0.01\\
36.6	0.01\\
36.61	0.01\\
36.62	0.01\\
36.63	0.01\\
36.64	0.01\\
36.65	0.01\\
36.66	0.01\\
36.67	0.01\\
36.68	0.01\\
36.69	0.01\\
36.7	0.01\\
36.71	0.01\\
36.72	0.01\\
36.73	0.01\\
36.74	0.01\\
36.75	0.01\\
36.76	0.01\\
36.77	0.01\\
36.78	0.01\\
36.79	0.01\\
36.8	0.01\\
36.81	0.01\\
36.82	0.01\\
36.83	0.01\\
36.84	0.01\\
36.85	0.01\\
36.86	0.01\\
36.87	0.01\\
36.88	0.01\\
36.89	0.01\\
36.9	0.01\\
36.91	0.01\\
36.92	0.01\\
36.93	0.01\\
36.94	0.01\\
36.95	0.01\\
36.96	0.01\\
36.97	0.01\\
36.98	0.01\\
36.99	0.01\\
37	0.01\\
37.01	0.01\\
37.02	0.01\\
37.03	0.01\\
37.04	0.01\\
37.05	0.01\\
37.06	0.01\\
37.07	0.01\\
37.08	0.01\\
37.09	0.01\\
37.1	0.01\\
37.11	0.01\\
37.12	0.01\\
37.13	0.01\\
37.14	0.01\\
37.15	0.01\\
37.16	0.01\\
37.17	0.01\\
37.18	0.01\\
37.19	0.01\\
37.2	0.01\\
37.21	0.01\\
37.22	0.01\\
37.23	0.01\\
37.24	0.01\\
37.25	0.01\\
37.26	0.01\\
37.27	0.01\\
37.28	0.01\\
37.29	0.01\\
37.3	0.01\\
37.31	0.01\\
37.32	0.01\\
37.33	0.01\\
37.34	0.01\\
37.35	0.01\\
37.36	0.01\\
37.37	0.01\\
37.38	0.01\\
37.39	0.01\\
37.4	0.01\\
37.41	0.01\\
37.42	0.01\\
37.43	0.01\\
37.44	0.01\\
37.45	0.01\\
37.46	0.01\\
37.47	0.01\\
37.48	0.01\\
37.49	0.01\\
37.5	0.01\\
37.51	0.01\\
37.52	0.01\\
37.53	0.01\\
37.54	0.01\\
37.55	0.01\\
37.56	0.01\\
37.57	0.01\\
37.58	0.01\\
37.59	0.01\\
37.6	0.01\\
37.61	0.01\\
37.62	0.01\\
37.63	0.01\\
37.64	0.01\\
37.65	0.01\\
37.66	0.01\\
37.67	0.01\\
37.68	0.01\\
37.69	0.01\\
37.7	0.01\\
37.71	0.01\\
37.72	0.01\\
37.73	0.01\\
37.74	0.01\\
37.75	0.01\\
37.76	0.01\\
37.77	0.01\\
37.78	0.01\\
37.79	0.01\\
37.8	0.01\\
37.81	0.01\\
37.82	0.01\\
37.83	0.01\\
37.84	0.01\\
37.85	0.01\\
37.86	0.01\\
37.87	0.01\\
37.88	0.01\\
37.89	0.01\\
37.9	0.01\\
37.91	0.01\\
37.92	0.01\\
37.93	0.01\\
37.94	0.01\\
37.95	0.01\\
37.96	0.01\\
37.97	0.01\\
37.98	0.01\\
37.99	0.01\\
38	0.01\\
38.01	0.01\\
38.02	0.01\\
38.03	0.01\\
38.04	0.01\\
38.05	0.01\\
38.06	0.01\\
38.07	0.01\\
38.08	0.01\\
38.09	0.01\\
38.1	0.01\\
38.11	0.01\\
38.12	0.01\\
38.13	0.01\\
38.14	0.01\\
38.15	0.01\\
38.16	0.01\\
38.17	0.01\\
38.18	0.01\\
38.19	0.01\\
38.2	0.01\\
38.21	0.01\\
38.22	0.01\\
38.23	0.01\\
38.24	0.01\\
38.25	0.01\\
38.26	0.01\\
38.27	0.01\\
38.28	0.01\\
38.29	0.01\\
38.3	0.01\\
38.31	0.01\\
38.32	0.01\\
38.33	0.01\\
38.34	0.01\\
38.35	0.01\\
38.36	0.01\\
38.37	0.01\\
38.38	0.01\\
38.39	0.01\\
38.4	0.01\\
38.41	0.01\\
38.42	0.01\\
38.43	0.01\\
38.44	0.01\\
38.45	0.01\\
38.46	0.01\\
38.47	0.01\\
38.48	0.01\\
38.49	0.01\\
38.5	0.01\\
38.51	0.01\\
38.52	0.01\\
38.53	0.01\\
38.54	0.01\\
38.55	0.01\\
38.56	0.01\\
38.57	0.01\\
38.58	0.01\\
38.59	0.01\\
38.6	0.01\\
38.61	0.01\\
38.62	0.01\\
38.63	0.01\\
38.64	0.01\\
38.65	0.01\\
38.66	0.01\\
38.67	0.01\\
38.68	0.01\\
38.69	0.01\\
38.7	0.01\\
38.71	0.01\\
38.72	0.01\\
38.73	0.01\\
38.74	0.01\\
38.75	0.01\\
38.76	0.01\\
38.77	0.01\\
38.78	0.01\\
38.79	0.01\\
38.8	0.01\\
38.81	0.01\\
38.82	0.01\\
38.83	0.01\\
38.84	0.01\\
38.85	0.01\\
38.86	0.01\\
38.87	0.01\\
38.88	0.01\\
38.89	0.01\\
38.9	0.01\\
38.91	0.01\\
38.92	0.01\\
38.93	0.01\\
38.94	0.01\\
38.95	0.01\\
38.96	0.01\\
38.97	0.01\\
38.98	0.01\\
38.99	0.01\\
39	0.01\\
39.01	0.01\\
39.02	0.01\\
39.03	0.01\\
39.04	0.01\\
39.05	0.01\\
39.06	0.01\\
39.07	0.01\\
39.08	0.01\\
39.09	0.01\\
39.1	0.01\\
39.11	0.01\\
39.12	0.01\\
39.13	0.01\\
39.14	0.01\\
39.15	0.01\\
39.16	0.01\\
39.17	0.01\\
39.18	0.01\\
39.19	0.01\\
39.2	0.01\\
39.21	0.01\\
39.22	0.01\\
39.23	0.01\\
39.24	0.01\\
39.25	0.01\\
39.26	0.01\\
39.27	0.01\\
39.28	0.01\\
39.29	0.01\\
39.3	0.01\\
39.31	0.01\\
39.32	0.01\\
39.33	0.01\\
39.34	0.01\\
39.35	0.01\\
39.36	0.01\\
39.37	0.01\\
39.38	0.01\\
39.39	0.01\\
39.4	0.01\\
39.41	0.01\\
39.42	0.01\\
39.43	0.01\\
39.44	0.01\\
39.45	0.01\\
39.46	0.01\\
39.47	0.01\\
39.48	0.01\\
39.49	0.01\\
39.5	0.01\\
39.51	0.01\\
39.52	0.01\\
39.53	0.01\\
39.54	0.01\\
39.55	0.01\\
39.56	0.01\\
39.57	0.01\\
39.58	0.01\\
39.59	0.01\\
39.6	0.01\\
39.61	0.01\\
39.62	0.01\\
39.63	0.01\\
39.64	0.01\\
39.65	0.01\\
39.66	0.01\\
39.67	0.01\\
39.68	0.01\\
39.69	0.01\\
39.7	0.01\\
39.71	0.01\\
39.72	0.01\\
39.73	0.01\\
39.74	0.01\\
39.75	0.01\\
39.76	0.01\\
39.77	0.01\\
39.78	0.01\\
39.79	0.01\\
39.8	0.01\\
39.81	0.01\\
39.82	0.01\\
39.83	0.01\\
39.84	0.01\\
39.85	0.01\\
39.86	0.01\\
39.87	0.01\\
39.88	0.01\\
39.89	0.01\\
39.9	0.01\\
39.91	0.01\\
39.92	0.01\\
39.93	0.01\\
39.94	0.01\\
39.95	0.01\\
39.96	0.01\\
39.97	0.01\\
39.98	0.01\\
39.99	0.01\\
40	0.01\\
40.01	0.01\\
};
\addplot [color=mycolor1,dashed,forget plot]
  table[row sep=crcr]{%
40.01	0.01\\
40.02	0.01\\
40.03	0.01\\
40.04	0.01\\
40.05	0.01\\
40.06	0.01\\
40.07	0.01\\
40.08	0.01\\
40.09	0.01\\
40.1	0.01\\
40.11	0.01\\
40.12	0.01\\
40.13	0.01\\
40.14	0.01\\
40.15	0.01\\
40.16	0.01\\
40.17	0.01\\
40.18	0.01\\
40.19	0.01\\
40.2	0.01\\
40.21	0.01\\
40.22	0.01\\
40.23	0.01\\
40.24	0.01\\
40.25	0.01\\
40.26	0.01\\
40.27	0.01\\
40.28	0.01\\
40.29	0.01\\
40.3	0.01\\
40.31	0.01\\
40.32	0.01\\
40.33	0.01\\
40.34	0.01\\
40.35	0.01\\
40.36	0.01\\
40.37	0.01\\
40.38	0.01\\
40.39	0.01\\
40.4	0.01\\
40.41	0.01\\
40.42	0.01\\
40.43	0.01\\
40.44	0.01\\
40.45	0.01\\
40.46	0.01\\
40.47	0.01\\
40.48	0.01\\
40.49	0.01\\
40.5	0.01\\
40.51	0.01\\
40.52	0.01\\
40.53	0.01\\
40.54	0.01\\
40.55	0.01\\
40.56	0.01\\
40.57	0.01\\
40.58	0.01\\
40.59	0.01\\
40.6	0.01\\
40.61	0.01\\
40.62	0.01\\
40.63	0.01\\
40.64	0.01\\
40.65	0.01\\
40.66	0.01\\
40.67	0.01\\
40.68	0.01\\
40.69	0.01\\
40.7	0.01\\
40.71	0.01\\
40.72	0.01\\
40.73	0.01\\
40.74	0.01\\
40.75	0.01\\
40.76	0.01\\
40.77	0.01\\
40.78	0.01\\
40.79	0.01\\
40.8	0.01\\
40.81	0.01\\
40.82	0.01\\
40.83	0.01\\
40.84	0.01\\
40.85	0.01\\
40.86	0.01\\
40.87	0.01\\
40.88	0.01\\
40.89	0.01\\
40.9	0.01\\
40.91	0.01\\
40.92	0.01\\
40.93	0.01\\
40.94	0.01\\
40.95	0.01\\
40.96	0.01\\
40.97	0.01\\
40.98	0.01\\
40.99	0.01\\
41	0.01\\
41.01	0.01\\
41.02	0.01\\
41.03	0.01\\
41.04	0.01\\
41.05	0.01\\
41.06	0.01\\
41.07	0.01\\
41.08	0.01\\
41.09	0.01\\
41.1	0.01\\
41.11	0.01\\
41.12	0.01\\
41.13	0.01\\
41.14	0.01\\
41.15	0.01\\
41.16	0.01\\
41.17	0.01\\
41.18	0.01\\
41.19	0.01\\
41.2	0.01\\
41.21	0.01\\
41.22	0.01\\
41.23	0.01\\
41.24	0.01\\
41.25	0.01\\
41.26	0.01\\
41.27	0.01\\
41.28	0.01\\
41.29	0.01\\
41.3	0.01\\
41.31	0.01\\
41.32	0.01\\
41.33	0.01\\
41.34	0.01\\
41.35	0.01\\
41.36	0.01\\
41.37	0.01\\
41.38	0.01\\
41.39	0.01\\
41.4	0.01\\
41.41	0.01\\
41.42	0.01\\
41.43	0.01\\
41.44	0.01\\
41.45	0.01\\
41.46	0.01\\
41.47	0.01\\
41.48	0.01\\
41.49	0.01\\
41.5	0.01\\
41.51	0.01\\
41.52	0.01\\
41.53	0.01\\
41.54	0.01\\
41.55	0.01\\
41.56	0.01\\
41.57	0.01\\
41.58	0.01\\
41.59	0.01\\
41.6	0.01\\
41.61	0.01\\
41.62	0.01\\
41.63	0.01\\
41.64	0.01\\
41.65	0.01\\
41.66	0.01\\
41.67	0.01\\
41.68	0.01\\
41.69	0.01\\
41.7	0.01\\
41.71	0.01\\
41.72	0.01\\
41.73	0.01\\
41.74	0.01\\
41.75	0.01\\
41.76	0.01\\
41.77	0.01\\
41.78	0.01\\
41.79	0.01\\
41.8	0.01\\
41.81	0.01\\
41.82	0.01\\
41.83	0.01\\
41.84	0.01\\
41.85	0.01\\
41.86	0.01\\
41.87	0.01\\
41.88	0.01\\
41.89	0.01\\
41.9	0.01\\
41.91	0.01\\
41.92	0.01\\
41.93	0.01\\
41.94	0.01\\
41.95	0.01\\
41.96	0.01\\
41.97	0.01\\
41.98	0.01\\
41.99	0.01\\
42	0.01\\
42.01	0.01\\
42.02	0.01\\
42.03	0.01\\
42.04	0.01\\
42.05	0.01\\
42.06	0.01\\
42.07	0.01\\
42.08	0.01\\
42.09	0.01\\
42.1	0.01\\
42.11	0.01\\
42.12	0.01\\
42.13	0.01\\
42.14	0.01\\
42.15	0.01\\
42.16	0.01\\
42.17	0.01\\
42.18	0.01\\
42.19	0.01\\
42.2	0.01\\
42.21	0.01\\
42.22	0.01\\
42.23	0.01\\
42.24	0.01\\
42.25	0.01\\
42.26	0.01\\
42.27	0.01\\
42.28	0.01\\
42.29	0.01\\
42.3	0.01\\
42.31	0.01\\
42.32	0.01\\
42.33	0.01\\
42.34	0.01\\
42.35	0.01\\
42.36	0.01\\
42.37	0.01\\
42.38	0.01\\
42.39	0.01\\
42.4	0.01\\
42.41	0.01\\
42.42	0.01\\
42.43	0.01\\
42.44	0.01\\
42.45	0.01\\
42.46	0.01\\
42.47	0.01\\
42.48	0.01\\
42.49	0.01\\
42.5	0.01\\
42.51	0.01\\
42.52	0.01\\
42.53	0.01\\
42.54	0.01\\
42.55	0.01\\
42.56	0.01\\
42.57	0.01\\
42.58	0.01\\
42.59	0.01\\
42.6	0.01\\
42.61	0.01\\
42.62	0.01\\
42.63	0.01\\
42.64	0.01\\
42.65	0.01\\
42.66	0.01\\
42.67	0.01\\
42.68	0.01\\
42.69	0.01\\
42.7	0.01\\
42.71	0.01\\
42.72	0.01\\
42.73	0.01\\
42.74	0.01\\
42.75	0.01\\
42.76	0.01\\
42.77	0.01\\
42.78	0.01\\
42.79	0.01\\
42.8	0.01\\
42.81	0.01\\
42.82	0.01\\
42.83	0.01\\
42.84	0.01\\
42.85	0.01\\
42.86	0.01\\
42.87	0.01\\
42.88	0.01\\
42.89	0.01\\
42.9	0.01\\
42.91	0.01\\
42.92	0.01\\
42.93	0.01\\
42.94	0.01\\
42.95	0.01\\
42.96	0.01\\
42.97	0.01\\
42.98	0.01\\
42.99	0.01\\
43	0.01\\
43.01	0.01\\
43.02	0.01\\
43.03	0.01\\
43.04	0.01\\
43.05	0.01\\
43.06	0.01\\
43.07	0.01\\
43.08	0.01\\
43.09	0.01\\
43.1	0.01\\
43.11	0.01\\
43.12	0.01\\
43.13	0.01\\
43.14	0.01\\
43.15	0.01\\
43.16	0.01\\
43.17	0.01\\
43.18	0.01\\
43.19	0.01\\
43.2	0.01\\
43.21	0.01\\
43.22	0.01\\
43.23	0.01\\
43.24	0.01\\
43.25	0.01\\
43.26	0.01\\
43.27	0.01\\
43.28	0.01\\
43.29	0.01\\
43.3	0.01\\
43.31	0.01\\
43.32	0.01\\
43.33	0.01\\
43.34	0.01\\
43.35	0.01\\
43.36	0.01\\
43.37	0.01\\
43.38	0.01\\
43.39	0.01\\
43.4	0.01\\
43.41	0.01\\
43.42	0.01\\
43.43	0.01\\
43.44	0.01\\
43.45	0.01\\
43.46	0.01\\
43.47	0.01\\
43.48	0.01\\
43.49	0.01\\
43.5	0.01\\
43.51	0.01\\
43.52	0.01\\
43.53	0.01\\
43.54	0.01\\
43.55	0.01\\
43.56	0.01\\
43.57	0.01\\
43.58	0.01\\
43.59	0.01\\
43.6	0.01\\
43.61	0.01\\
43.62	0.01\\
43.63	0.01\\
43.64	0.01\\
43.65	0.01\\
43.66	0.01\\
43.67	0.01\\
43.68	0.01\\
43.69	0.01\\
43.7	0.01\\
43.71	0.01\\
43.72	0.01\\
43.73	0.01\\
43.74	0.01\\
43.75	0.01\\
43.76	0.01\\
43.77	0.01\\
43.78	0.01\\
43.79	0.01\\
43.8	0.01\\
43.81	0.01\\
43.82	0.01\\
43.83	0.01\\
43.84	0.01\\
43.85	0.01\\
43.86	0.01\\
43.87	0.01\\
43.88	0.01\\
43.89	0.01\\
43.9	0.01\\
43.91	0.01\\
43.92	0.01\\
43.93	0.01\\
43.94	0.01\\
43.95	0.01\\
43.96	0.01\\
43.97	0.01\\
43.98	0.01\\
43.99	0.01\\
44	0.01\\
44.01	0.01\\
44.02	0.01\\
44.03	0.01\\
44.04	0.01\\
44.05	0.01\\
44.06	0.01\\
44.07	0.01\\
44.08	0.01\\
44.09	0.01\\
44.1	0.01\\
44.11	0.01\\
44.12	0.01\\
44.13	0.01\\
44.14	0.01\\
44.15	0.01\\
44.16	0.01\\
44.17	0.01\\
44.18	0.01\\
44.19	0.01\\
44.2	0.01\\
44.21	0.01\\
44.22	0.01\\
44.23	0.01\\
44.24	0.01\\
44.25	0.01\\
44.26	0.01\\
44.27	0.01\\
44.28	0.01\\
44.29	0.01\\
44.3	0.01\\
44.31	0.01\\
44.32	0.01\\
44.33	0.01\\
44.34	0.01\\
44.35	0.01\\
44.36	0.01\\
44.37	0.01\\
44.38	0.01\\
44.39	0.01\\
44.4	0.01\\
44.41	0.01\\
44.42	0.01\\
44.43	0.01\\
44.44	0.01\\
44.45	0.01\\
44.46	0.01\\
44.47	0.01\\
44.48	0.01\\
44.49	0.01\\
44.5	0.01\\
44.51	0.01\\
44.52	0.01\\
44.53	0.01\\
44.54	0.01\\
44.55	0.01\\
44.56	0.01\\
44.57	0.01\\
44.58	0.01\\
44.59	0.01\\
44.6	0.01\\
44.61	0.01\\
44.62	0.01\\
44.63	0.01\\
44.64	0.01\\
44.65	0.01\\
44.66	0.01\\
44.67	0.01\\
44.68	0.01\\
44.69	0.01\\
44.7	0.01\\
44.71	0.01\\
44.72	0.01\\
44.73	0.01\\
44.74	0.01\\
44.75	0.01\\
44.76	0.01\\
44.77	0.01\\
44.78	0.01\\
44.79	0.01\\
44.8	0.01\\
44.81	0.01\\
44.82	0.01\\
44.83	0.01\\
44.84	0.01\\
44.85	0.01\\
44.86	0.01\\
44.87	0.01\\
44.88	0.01\\
44.89	0.01\\
44.9	0.01\\
44.91	0.01\\
44.92	0.01\\
44.93	0.01\\
44.94	0.01\\
44.95	0.01\\
44.96	0.01\\
44.97	0.01\\
44.98	0.01\\
44.99	0.01\\
45	0.01\\
45.01	0.01\\
45.02	0.01\\
45.03	0.01\\
45.04	0.01\\
45.05	0.01\\
45.06	0.01\\
45.07	0.01\\
45.08	0.01\\
45.09	0.01\\
45.1	0.01\\
45.11	0.01\\
45.12	0.01\\
45.13	0.01\\
45.14	0.01\\
45.15	0.01\\
45.16	0.01\\
45.17	0.01\\
45.18	0.01\\
45.19	0.01\\
45.2	0.01\\
45.21	0.01\\
45.22	0.01\\
45.23	0.01\\
45.24	0.01\\
45.25	0.01\\
45.26	0.01\\
45.27	0.01\\
45.28	0.01\\
45.29	0.01\\
45.3	0.01\\
45.31	0.01\\
45.32	0.01\\
45.33	0.01\\
45.34	0.01\\
45.35	0.01\\
45.36	0.01\\
45.37	0.01\\
45.38	0.01\\
45.39	0.01\\
45.4	0.01\\
45.41	0.01\\
45.42	0.01\\
45.43	0.01\\
45.44	0.01\\
45.45	0.01\\
45.46	0.01\\
45.47	0.01\\
45.48	0.01\\
45.49	0.01\\
45.5	0.01\\
45.51	0.01\\
45.52	0.01\\
45.53	0.01\\
45.54	0.01\\
45.55	0.01\\
45.56	0.01\\
45.57	0.01\\
45.58	0.01\\
45.59	0.01\\
45.6	0.01\\
45.61	0.01\\
45.62	0.01\\
45.63	0.01\\
45.64	0.01\\
45.65	0.01\\
45.66	0.01\\
45.67	0.01\\
45.68	0.01\\
45.69	0.01\\
45.7	0.01\\
45.71	0.01\\
45.72	0.01\\
45.73	0.01\\
45.74	0.01\\
45.75	0.01\\
45.76	0.01\\
45.77	0.01\\
45.78	0.01\\
45.79	0.01\\
45.8	0.01\\
45.81	0.01\\
45.82	0.01\\
45.83	0.01\\
45.84	0.01\\
45.85	0.01\\
45.86	0.01\\
45.87	0.01\\
45.88	0.01\\
45.89	0.01\\
45.9	0.01\\
45.91	0.01\\
45.92	0.01\\
45.93	0.01\\
45.94	0.01\\
45.95	0.01\\
45.96	0.01\\
45.97	0.01\\
45.98	0.01\\
45.99	0.01\\
46	0.01\\
46.01	0.01\\
46.02	0.01\\
46.03	0.01\\
46.04	0.01\\
46.05	0.01\\
46.06	0.01\\
46.07	0.01\\
46.08	0.01\\
46.09	0.01\\
46.1	0.01\\
46.11	0.01\\
46.12	0.01\\
46.13	0.01\\
46.14	0.01\\
46.15	0.01\\
46.16	0.01\\
46.17	0.01\\
46.18	0.01\\
46.19	0.01\\
46.2	0.01\\
46.21	0.01\\
46.22	0.01\\
46.23	0.01\\
46.24	0.01\\
46.25	0.01\\
46.26	0.01\\
46.27	0.01\\
46.28	0.01\\
46.29	0.01\\
46.3	0.01\\
46.31	0.01\\
46.32	0.01\\
46.33	0.01\\
46.34	0.01\\
46.35	0.01\\
46.36	0.01\\
46.37	0.01\\
46.38	0.01\\
46.39	0.01\\
46.4	0.01\\
46.41	0.01\\
46.42	0.01\\
46.43	0.01\\
46.44	0.01\\
46.45	0.01\\
46.46	0.01\\
46.47	0.01\\
46.48	0.01\\
46.49	0.01\\
46.5	0.01\\
46.51	0.01\\
46.52	0.01\\
46.53	0.01\\
46.54	0.01\\
46.55	0.01\\
46.56	0.01\\
46.57	0.01\\
46.58	0.01\\
46.59	0.01\\
46.6	0.01\\
46.61	0.01\\
46.62	0.01\\
46.63	0.01\\
46.64	0.01\\
46.65	0.01\\
46.66	0.01\\
46.67	0.01\\
46.68	0.01\\
46.69	0.01\\
46.7	0.01\\
46.71	0.01\\
46.72	0.01\\
46.73	0.01\\
46.74	0.01\\
46.75	0.01\\
46.76	0.01\\
46.77	0.01\\
46.78	0.01\\
46.79	0.01\\
46.8	0.01\\
46.81	0.01\\
46.82	0.01\\
46.83	0.01\\
46.84	0.01\\
46.85	0.01\\
46.86	0.01\\
46.87	0.01\\
46.88	0.01\\
46.89	0.01\\
46.9	0.01\\
46.91	0.01\\
46.92	0.01\\
46.93	0.01\\
46.94	0.01\\
46.95	0.01\\
46.96	0.01\\
46.97	0.01\\
46.98	0.01\\
46.99	0.01\\
47	0.01\\
47.01	0.01\\
47.02	0.01\\
47.03	0.01\\
47.04	0.01\\
47.05	0.01\\
47.06	0.01\\
47.07	0.01\\
47.08	0.01\\
47.09	0.01\\
47.1	0.01\\
47.11	0.01\\
47.12	0.01\\
47.13	0.01\\
47.14	0.01\\
47.15	0.01\\
47.16	0.01\\
47.17	0.01\\
47.18	0.01\\
47.19	0.01\\
47.2	0.01\\
47.21	0.01\\
47.22	0.01\\
47.23	0.01\\
47.24	0.01\\
47.25	0.01\\
47.26	0.01\\
47.27	0.01\\
47.28	0.01\\
47.29	0.01\\
47.3	0.01\\
47.31	0.01\\
47.32	0.01\\
47.33	0.01\\
47.34	0.01\\
47.35	0.01\\
47.36	0.01\\
47.37	0.01\\
47.38	0.01\\
47.39	0.01\\
47.4	0.01\\
47.41	0.01\\
47.42	0.01\\
47.43	0.01\\
47.44	0.01\\
47.45	0.01\\
47.46	0.01\\
47.47	0.01\\
47.48	0.01\\
47.49	0.01\\
47.5	0.01\\
47.51	0.01\\
47.52	0.01\\
47.53	0.01\\
47.54	0.01\\
47.55	0.01\\
47.56	0.01\\
47.57	0.01\\
47.58	0.01\\
47.59	0.01\\
47.6	0.01\\
47.61	0.01\\
47.62	0.01\\
47.63	0.01\\
47.64	0.01\\
47.65	0.01\\
47.66	0.01\\
47.67	0.01\\
47.68	0.01\\
47.69	0.01\\
47.7	0.01\\
47.71	0.01\\
47.72	0.01\\
47.73	0.01\\
47.74	0.01\\
47.75	0.01\\
47.76	0.01\\
47.77	0.01\\
47.78	0.01\\
47.79	0.01\\
47.8	0.01\\
47.81	0.01\\
47.82	0.01\\
47.83	0.01\\
47.84	0.01\\
47.85	0.01\\
47.86	0.01\\
47.87	0.01\\
47.88	0.01\\
47.89	0.01\\
47.9	0.01\\
47.91	0.01\\
47.92	0.01\\
47.93	0.01\\
47.94	0.01\\
47.95	0.01\\
47.96	0.01\\
47.97	0.01\\
47.98	0.01\\
47.99	0.01\\
48	0.01\\
48.01	0.01\\
48.02	0.01\\
48.03	0.01\\
48.04	0.01\\
48.05	0.01\\
48.06	0.01\\
48.07	0.01\\
48.08	0.01\\
48.09	0.01\\
48.1	0.01\\
48.11	0.01\\
48.12	0.01\\
48.13	0.01\\
48.14	0.01\\
48.15	0.01\\
48.16	0.01\\
48.17	0.01\\
48.18	0.01\\
48.19	0.01\\
48.2	0.01\\
48.21	0.01\\
48.22	0.01\\
48.23	0.01\\
48.24	0.01\\
48.25	0.01\\
48.26	0.01\\
48.27	0.01\\
48.28	0.01\\
48.29	0.01\\
48.3	0.01\\
48.31	0.01\\
48.32	0.01\\
48.33	0.01\\
48.34	0.01\\
48.35	0.01\\
48.36	0.01\\
48.37	0.01\\
48.38	0.01\\
48.39	0.01\\
48.4	0.01\\
48.41	0.01\\
48.42	0.01\\
48.43	0.01\\
48.44	0.01\\
48.45	0.01\\
48.46	0.01\\
48.47	0.01\\
48.48	0.01\\
48.49	0.01\\
48.5	0.01\\
48.51	0.01\\
48.52	0.01\\
48.53	0.01\\
48.54	0.01\\
48.55	0.01\\
48.56	0.01\\
48.57	0.01\\
48.58	0.01\\
48.59	0.01\\
48.6	0.01\\
48.61	0.01\\
48.62	0.01\\
48.63	0.01\\
48.64	0.01\\
48.65	0.01\\
48.66	0.01\\
48.67	0.01\\
48.68	0.01\\
48.69	0.01\\
48.7	0.01\\
48.71	0.01\\
48.72	0.01\\
48.73	0.01\\
48.74	0.01\\
48.75	0.01\\
48.76	0.01\\
48.77	0.01\\
48.78	0.01\\
48.79	0.01\\
48.8	0.01\\
48.81	0.01\\
48.82	0.01\\
48.83	0.01\\
48.84	0.01\\
48.85	0.01\\
48.86	0.01\\
48.87	0.01\\
48.88	0.01\\
48.89	0.01\\
48.9	0.01\\
48.91	0.01\\
48.92	0.01\\
48.93	0.01\\
48.94	0.01\\
48.95	0.01\\
48.96	0.01\\
48.97	0.01\\
48.98	0.01\\
48.99	0.01\\
49	0.01\\
49.01	0.01\\
49.02	0.01\\
49.03	0.01\\
49.04	0.01\\
49.05	0.01\\
49.06	0.01\\
49.07	0.01\\
49.08	0.01\\
49.09	0.01\\
49.1	0.01\\
49.11	0.01\\
49.12	0.01\\
49.13	0.01\\
49.14	0.01\\
49.15	0.01\\
49.16	0.01\\
49.17	0.01\\
49.18	0.01\\
49.19	0.01\\
49.2	0.01\\
49.21	0.01\\
49.22	0.01\\
49.23	0.01\\
49.24	0.01\\
49.25	0.01\\
49.26	0.01\\
49.27	0.01\\
49.28	0.01\\
49.29	0.01\\
49.3	0.01\\
49.31	0.01\\
49.32	0.01\\
49.33	0.01\\
49.34	0.01\\
49.35	0.01\\
49.36	0.01\\
49.37	0.01\\
49.38	0.01\\
49.39	0.01\\
49.4	0.01\\
49.41	0.01\\
49.42	0.01\\
49.43	0.01\\
49.44	0.01\\
49.45	0.01\\
49.46	0.01\\
49.47	0.01\\
49.48	0.01\\
49.49	0.01\\
49.5	0.01\\
49.51	0.01\\
49.52	0.01\\
49.53	0.01\\
49.54	0.01\\
49.55	0.01\\
49.56	0.01\\
49.57	0.01\\
49.58	0.01\\
49.59	0.01\\
49.6	0.01\\
49.61	0.01\\
49.62	0.01\\
49.63	0.01\\
49.64	0.01\\
49.65	0.01\\
49.66	0.01\\
49.67	0.01\\
49.68	0.01\\
49.69	0.01\\
49.7	0.01\\
49.71	0.01\\
49.72	0.01\\
49.73	0.01\\
49.74	0.01\\
49.75	0.01\\
49.76	0.01\\
49.77	0.01\\
49.78	0.01\\
49.79	0.01\\
49.8	0.01\\
49.81	0.01\\
49.82	0.01\\
49.83	0.01\\
49.84	0.01\\
49.85	0.01\\
49.86	0.01\\
49.87	0.01\\
49.88	0.01\\
49.89	0.01\\
49.9	0.01\\
49.91	0.01\\
49.92	0.01\\
49.93	0.01\\
49.94	0.01\\
49.95	0.01\\
49.96	0.01\\
49.97	0.01\\
49.98	0.01\\
49.99	0.01\\
50	0.01\\
50.01	0.01\\
50.02	0.01\\
50.03	0.01\\
50.04	0.01\\
50.05	0.01\\
50.06	0.01\\
50.07	0.01\\
50.08	0.01\\
50.09	0.01\\
50.1	0.01\\
50.11	0.01\\
50.12	0.01\\
50.13	0.01\\
50.14	0.01\\
50.15	0.01\\
50.16	0.01\\
50.17	0.01\\
50.18	0.01\\
50.19	0.01\\
50.2	0.01\\
50.21	0.01\\
50.22	0.01\\
50.23	0.01\\
50.24	0.01\\
50.25	0.01\\
50.26	0.01\\
50.27	0.01\\
50.28	0.01\\
50.29	0.01\\
50.3	0.01\\
50.31	0.01\\
50.32	0.01\\
50.33	0.01\\
50.34	0.01\\
50.35	0.01\\
50.36	0.01\\
50.37	0.01\\
50.38	0.01\\
50.39	0.01\\
50.4	0.01\\
50.41	0.01\\
50.42	0.01\\
50.43	0.01\\
50.44	0.01\\
50.45	0.01\\
50.46	0.01\\
50.47	0.01\\
50.48	0.01\\
50.49	0.01\\
50.5	0.01\\
50.51	0.01\\
50.52	0.01\\
50.53	0.01\\
50.54	0.01\\
50.55	0.01\\
50.56	0.01\\
50.57	0.01\\
50.58	0.01\\
50.59	0.01\\
50.6	0.01\\
50.61	0.01\\
50.62	0.01\\
50.63	0.01\\
50.64	0.01\\
50.65	0.01\\
50.66	0.01\\
50.67	0.01\\
50.68	0.01\\
50.69	0.01\\
50.7	0.01\\
50.71	0.01\\
50.72	0.01\\
50.73	0.01\\
50.74	0.01\\
50.75	0.01\\
50.76	0.01\\
50.77	0.01\\
50.78	0.01\\
50.79	0.01\\
50.8	0.01\\
50.81	0.01\\
50.82	0.01\\
50.83	0.01\\
50.84	0.01\\
50.85	0.01\\
50.86	0.01\\
50.87	0.01\\
50.88	0.01\\
50.89	0.01\\
50.9	0.01\\
50.91	0.01\\
50.92	0.01\\
50.93	0.01\\
50.94	0.01\\
50.95	0.01\\
50.96	0.01\\
50.97	0.01\\
50.98	0.01\\
50.99	0.01\\
51	0.01\\
51.01	0.01\\
51.02	0.01\\
51.03	0.01\\
51.04	0.01\\
51.05	0.01\\
51.06	0.01\\
51.07	0.01\\
51.08	0.01\\
51.09	0.01\\
51.1	0.01\\
51.11	0.01\\
51.12	0.01\\
51.13	0.01\\
51.14	0.01\\
51.15	0.01\\
51.16	0.01\\
51.17	0.01\\
51.18	0.01\\
51.19	0.01\\
51.2	0.01\\
51.21	0.01\\
51.22	0.01\\
51.23	0.01\\
51.24	0.01\\
51.25	0.01\\
51.26	0.01\\
51.27	0.01\\
51.28	0.01\\
51.29	0.01\\
51.3	0.01\\
51.31	0.01\\
51.32	0.01\\
51.33	0.01\\
51.34	0.01\\
51.35	0.01\\
51.36	0.01\\
51.37	0.01\\
51.38	0.01\\
51.39	0.01\\
51.4	0.01\\
51.41	0.01\\
51.42	0.01\\
51.43	0.01\\
51.44	0.01\\
51.45	0.01\\
51.46	0.01\\
51.47	0.01\\
51.48	0.01\\
51.49	0.01\\
51.5	0.01\\
51.51	0.01\\
51.52	0.01\\
51.53	0.01\\
51.54	0.01\\
51.55	0.01\\
51.56	0.01\\
51.57	0.01\\
51.58	0.01\\
51.59	0.01\\
51.6	0.01\\
51.61	0.01\\
51.62	0.01\\
51.63	0.01\\
51.64	0.01\\
51.65	0.01\\
51.66	0.01\\
51.67	0.01\\
51.68	0.01\\
51.69	0.01\\
51.7	0.01\\
51.71	0.01\\
51.72	0.01\\
51.73	0.01\\
51.74	0.01\\
51.75	0.01\\
51.76	0.01\\
51.77	0.01\\
51.78	0.01\\
51.79	0.01\\
51.8	0.01\\
51.81	0.01\\
51.82	0.01\\
51.83	0.01\\
51.84	0.01\\
51.85	0.01\\
51.86	0.01\\
51.87	0.01\\
51.88	0.01\\
51.89	0.01\\
51.9	0.01\\
51.91	0.01\\
51.92	0.01\\
51.93	0.01\\
51.94	0.01\\
51.95	0.01\\
51.96	0.01\\
51.97	0.01\\
51.98	0.01\\
51.99	0.01\\
52	0.01\\
52.01	0.01\\
52.02	0.01\\
52.03	0.01\\
52.04	0.01\\
52.05	0.01\\
52.06	0.01\\
52.07	0.01\\
52.08	0.01\\
52.09	0.01\\
52.1	0.01\\
52.11	0.01\\
52.12	0.01\\
52.13	0.01\\
52.14	0.01\\
52.15	0.01\\
52.16	0.01\\
52.17	0.01\\
52.18	0.01\\
52.19	0.01\\
52.2	0.01\\
52.21	0.01\\
52.22	0.01\\
52.23	0.01\\
52.24	0.01\\
52.25	0.01\\
52.26	0.01\\
52.27	0.01\\
52.28	0.01\\
52.29	0.01\\
52.3	0.01\\
52.31	0.01\\
52.32	0.01\\
52.33	0.01\\
52.34	0.01\\
52.35	0.01\\
52.36	0.01\\
52.37	0.01\\
52.38	0.01\\
52.39	0.01\\
52.4	0.01\\
52.41	0.01\\
52.42	0.01\\
52.43	0.01\\
52.44	0.01\\
52.45	0.01\\
52.46	0.01\\
52.47	0.01\\
52.48	0.01\\
52.49	0.01\\
52.5	0.01\\
52.51	0.01\\
52.52	0.01\\
52.53	0.01\\
52.54	0.01\\
52.55	0.01\\
52.56	0.01\\
52.57	0.01\\
52.58	0.01\\
52.59	0.01\\
52.6	0.01\\
52.61	0.01\\
52.62	0.01\\
52.63	0.01\\
52.64	0.01\\
52.65	0.01\\
52.66	0.01\\
52.67	0.01\\
52.68	0.01\\
52.69	0.01\\
52.7	0.01\\
52.71	0.01\\
52.72	0.01\\
52.73	0.01\\
52.74	0.01\\
52.75	0.01\\
52.76	0.01\\
52.77	0.01\\
52.78	0.01\\
52.79	0.01\\
52.8	0.01\\
52.81	0.01\\
52.82	0.01\\
52.83	0.01\\
52.84	0.01\\
52.85	0.01\\
52.86	0.01\\
52.87	0.01\\
52.88	0.01\\
52.89	0.01\\
52.9	0.01\\
52.91	0.01\\
52.92	0.01\\
52.93	0.01\\
52.94	0.01\\
52.95	0.01\\
52.96	0.01\\
52.97	0.01\\
52.98	0.01\\
52.99	0.01\\
53	0.01\\
53.01	0.01\\
53.02	0.01\\
53.03	0.01\\
53.04	0.01\\
53.05	0.01\\
53.06	0.01\\
53.07	0.01\\
53.08	0.01\\
53.09	0.01\\
53.1	0.01\\
53.11	0.01\\
53.12	0.01\\
53.13	0.01\\
53.14	0.01\\
53.15	0.01\\
53.16	0.01\\
53.17	0.01\\
53.18	0.01\\
53.19	0.01\\
53.2	0.01\\
53.21	0.01\\
53.22	0.01\\
53.23	0.01\\
53.24	0.01\\
53.25	0.01\\
53.26	0.01\\
53.27	0.01\\
53.28	0.01\\
53.29	0.01\\
53.3	0.01\\
53.31	0.01\\
53.32	0.01\\
53.33	0.01\\
53.34	0.01\\
53.35	0.01\\
53.36	0.01\\
53.37	0.01\\
53.38	0.01\\
53.39	0.01\\
53.4	0.01\\
53.41	0.01\\
53.42	0.01\\
53.43	0.01\\
53.44	0.01\\
53.45	0.01\\
53.46	0.01\\
53.47	0.01\\
53.48	0.01\\
53.49	0.01\\
53.5	0.01\\
53.51	0.01\\
53.52	0.01\\
53.53	0.01\\
53.54	0.01\\
53.55	0.01\\
53.56	0.01\\
53.57	0.01\\
53.58	0.01\\
53.59	0.01\\
53.6	0.01\\
53.61	0.01\\
53.62	0.01\\
53.63	0.01\\
53.64	0.01\\
53.65	0.01\\
53.66	0.01\\
53.67	0.01\\
53.68	0.01\\
53.69	0.01\\
53.7	0.01\\
53.71	0.01\\
53.72	0.01\\
53.73	0.01\\
53.74	0.01\\
53.75	0.01\\
53.76	0.01\\
53.77	0.01\\
53.78	0.01\\
53.79	0.01\\
53.8	0.01\\
53.81	0.01\\
53.82	0.01\\
53.83	0.01\\
53.84	0.01\\
53.85	0.01\\
53.86	0.01\\
53.87	0.01\\
53.88	0.01\\
53.89	0.01\\
53.9	0.01\\
53.91	0.01\\
53.92	0.01\\
53.93	0.01\\
53.94	0.01\\
53.95	0.01\\
53.96	0.01\\
53.97	0.01\\
53.98	0.01\\
53.99	0.01\\
54	0.01\\
54.01	0.01\\
54.02	0.01\\
54.03	0.01\\
54.04	0.01\\
54.05	0.01\\
54.06	0.01\\
54.07	0.01\\
54.08	0.01\\
54.09	0.01\\
54.1	0.01\\
54.11	0.01\\
54.12	0.01\\
54.13	0.01\\
54.14	0.01\\
54.15	0.01\\
54.16	0.01\\
54.17	0.01\\
54.18	0.01\\
54.19	0.01\\
54.2	0.01\\
54.21	0.01\\
54.22	0.01\\
54.23	0.01\\
54.24	0.01\\
54.25	0.01\\
54.26	0.01\\
54.27	0.01\\
54.28	0.01\\
54.29	0.01\\
54.3	0.01\\
54.31	0.01\\
54.32	0.01\\
54.33	0.01\\
54.34	0.01\\
54.35	0.01\\
54.36	0.01\\
54.37	0.01\\
54.38	0.01\\
54.39	0.01\\
54.4	0.01\\
54.41	0.01\\
54.42	0.01\\
54.43	0.01\\
54.44	0.01\\
54.45	0.01\\
54.46	0.01\\
54.47	0.01\\
54.48	0.01\\
54.49	0.01\\
54.5	0.01\\
54.51	0.01\\
54.52	0.01\\
54.53	0.01\\
54.54	0.01\\
54.55	0.01\\
54.56	0.01\\
54.57	0.01\\
54.58	0.01\\
54.59	0.01\\
54.6	0.01\\
54.61	0.01\\
54.62	0.01\\
54.63	0.01\\
54.64	0.01\\
54.65	0.01\\
54.66	0.01\\
54.67	0.01\\
54.68	0.01\\
54.69	0.01\\
54.7	0.01\\
54.71	0.01\\
54.72	0.01\\
54.73	0.01\\
54.74	0.01\\
54.75	0.01\\
54.76	0.01\\
54.77	0.01\\
54.78	0.01\\
54.79	0.01\\
54.8	0.01\\
54.81	0.01\\
54.82	0.01\\
54.83	0.01\\
54.84	0.01\\
54.85	0.01\\
54.86	0.01\\
54.87	0.01\\
54.88	0.01\\
54.89	0.01\\
54.9	0.01\\
54.91	0.01\\
54.92	0.01\\
54.93	0.01\\
54.94	0.01\\
54.95	0.01\\
54.96	0.01\\
54.97	0.01\\
54.98	0.01\\
54.99	0.01\\
55	0.01\\
55.01	0.01\\
55.02	0.01\\
55.03	0.01\\
55.04	0.01\\
55.05	0.01\\
55.06	0.01\\
55.07	0.01\\
55.08	0.01\\
55.09	0.01\\
55.1	0.01\\
55.11	0.01\\
55.12	0.01\\
55.13	0.01\\
55.14	0.01\\
55.15	0.01\\
55.16	0.01\\
55.17	0.01\\
55.18	0.01\\
55.19	0.01\\
55.2	0.01\\
55.21	0.01\\
55.22	0.01\\
55.23	0.01\\
55.24	0.01\\
55.25	0.01\\
55.26	0.01\\
55.27	0.01\\
55.28	0.01\\
55.29	0.01\\
55.3	0.01\\
55.31	0.01\\
55.32	0.01\\
55.33	0.01\\
55.34	0.01\\
55.35	0.01\\
55.36	0.01\\
55.37	0.01\\
55.38	0.01\\
55.39	0.01\\
55.4	0.01\\
55.41	0.01\\
55.42	0.01\\
55.43	0.01\\
55.44	0.01\\
55.45	0.01\\
55.46	0.01\\
55.47	0.01\\
55.48	0.01\\
55.49	0.01\\
55.5	0.01\\
55.51	0.01\\
55.52	0.01\\
55.53	0.01\\
55.54	0.01\\
55.55	0.01\\
55.56	0.01\\
55.57	0.01\\
55.58	0.01\\
55.59	0.01\\
55.6	0.01\\
55.61	0.01\\
55.62	0.01\\
55.63	0.01\\
55.64	0.01\\
55.65	0.01\\
55.66	0.01\\
55.67	0.01\\
55.68	0.01\\
55.69	0.01\\
55.7	0.01\\
55.71	0.01\\
55.72	0.01\\
55.73	0.01\\
55.74	0.01\\
55.75	0.01\\
55.76	0.01\\
55.77	0.01\\
55.78	0.01\\
55.79	0.01\\
55.8	0.01\\
55.81	0.01\\
55.82	0.01\\
55.83	0.01\\
55.84	0.01\\
55.85	0.01\\
55.86	0.01\\
55.87	0.01\\
55.88	0.01\\
55.89	0.01\\
55.9	0.01\\
55.91	0.01\\
55.92	0.01\\
55.93	0.01\\
55.94	0.01\\
55.95	0.01\\
55.96	0.01\\
55.97	0.01\\
55.98	0.01\\
55.99	0.01\\
56	0.01\\
56.01	0.01\\
56.02	0.01\\
56.03	0.01\\
56.04	0.01\\
56.05	0.01\\
56.06	0.01\\
56.07	0.01\\
56.08	0.01\\
56.09	0.01\\
56.1	0.01\\
56.11	0.01\\
56.12	0.01\\
56.13	0.01\\
56.14	0.01\\
56.15	0.01\\
56.16	0.01\\
56.17	0.01\\
56.18	0.01\\
56.19	0.01\\
56.2	0.01\\
56.21	0.01\\
56.22	0.01\\
56.23	0.01\\
56.24	0.01\\
56.25	0.01\\
56.26	0.01\\
56.27	0.01\\
56.28	0.01\\
56.29	0.01\\
56.3	0.01\\
56.31	0.01\\
56.32	0.01\\
56.33	0.01\\
56.34	0.01\\
56.35	0.01\\
56.36	0.01\\
56.37	0.01\\
56.38	0.01\\
56.39	0.01\\
56.4	0.01\\
56.41	0.01\\
56.42	0.01\\
56.43	0.01\\
56.44	0.01\\
56.45	0.01\\
56.46	0.01\\
56.47	0.01\\
56.48	0.01\\
56.49	0.01\\
56.5	0.01\\
56.51	0.01\\
56.52	0.01\\
56.53	0.01\\
56.54	0.01\\
56.55	0.01\\
56.56	0.01\\
56.57	0.01\\
56.58	0.01\\
56.59	0.01\\
56.6	0.01\\
56.61	0.01\\
56.62	0.01\\
56.63	0.01\\
56.64	0.01\\
56.65	0.01\\
56.66	0.01\\
56.67	0.01\\
56.68	0.01\\
56.69	0.01\\
56.7	0.01\\
56.71	0.01\\
56.72	0.01\\
56.73	0.01\\
56.74	0.01\\
56.75	0.01\\
56.76	0.01\\
56.77	0.01\\
56.78	0.01\\
56.79	0.01\\
56.8	0.01\\
56.81	0.01\\
56.82	0.01\\
56.83	0.01\\
56.84	0.01\\
56.85	0.01\\
56.86	0.01\\
56.87	0.01\\
56.88	0.01\\
56.89	0.01\\
56.9	0.01\\
56.91	0.01\\
56.92	0.01\\
56.93	0.01\\
56.94	0.01\\
56.95	0.01\\
56.96	0.01\\
56.97	0.01\\
56.98	0.01\\
56.99	0.01\\
57	0.01\\
57.01	0.01\\
57.02	0.01\\
57.03	0.01\\
57.04	0.01\\
57.05	0.01\\
57.06	0.01\\
57.07	0.01\\
57.08	0.01\\
57.09	0.01\\
57.1	0.01\\
57.11	0.01\\
57.12	0.01\\
57.13	0.01\\
57.14	0.01\\
57.15	0.01\\
57.16	0.01\\
57.17	0.01\\
57.18	0.01\\
57.19	0.01\\
57.2	0.01\\
57.21	0.01\\
57.22	0.01\\
57.23	0.01\\
57.24	0.01\\
57.25	0.01\\
57.26	0.01\\
57.27	0.01\\
57.28	0.01\\
57.29	0.01\\
57.3	0.01\\
57.31	0.01\\
57.32	0.01\\
57.33	0.01\\
57.34	0.01\\
57.35	0.01\\
57.36	0.01\\
57.37	0.01\\
57.38	0.01\\
57.39	0.01\\
57.4	0.01\\
57.41	0.01\\
57.42	0.01\\
57.43	0.01\\
57.44	0.01\\
57.45	0.01\\
57.46	0.01\\
57.47	0.01\\
57.48	0.01\\
57.49	0.01\\
57.5	0.01\\
57.51	0.01\\
57.52	0.01\\
57.53	0.01\\
57.54	0.01\\
57.55	0.01\\
57.56	0.01\\
57.57	0.01\\
57.58	0.01\\
57.59	0.01\\
57.6	0.01\\
57.61	0.01\\
57.62	0.01\\
57.63	0.01\\
57.64	0.01\\
57.65	0.01\\
57.66	0.01\\
57.67	0.01\\
57.68	0.01\\
57.69	0.01\\
57.7	0.01\\
57.71	0.01\\
57.72	0.01\\
57.73	0.01\\
57.74	0.01\\
57.75	0.01\\
57.76	0.01\\
57.77	0.01\\
57.78	0.01\\
57.79	0.01\\
57.8	0.01\\
57.81	0.01\\
57.82	0.01\\
57.83	0.01\\
57.84	0.01\\
57.85	0.01\\
57.86	0.01\\
57.87	0.01\\
57.88	0.01\\
57.89	0.01\\
57.9	0.01\\
57.91	0.01\\
57.92	0.01\\
57.93	0.01\\
57.94	0.01\\
57.95	0.01\\
57.96	0.01\\
57.97	0.01\\
57.98	0.01\\
57.99	0.01\\
58	0.01\\
58.01	0.01\\
58.02	0.01\\
58.03	0.01\\
58.04	0.01\\
58.05	0.01\\
58.06	0.01\\
58.07	0.01\\
58.08	0.01\\
58.09	0.01\\
58.1	0.01\\
58.11	0.01\\
58.12	0.01\\
58.13	0.01\\
58.14	0.01\\
58.15	0.01\\
58.16	0.01\\
58.17	0.01\\
58.18	0.01\\
58.19	0.01\\
58.2	0.01\\
58.21	0.01\\
58.22	0.01\\
58.23	0.01\\
58.24	0.01\\
58.25	0.01\\
58.26	0.01\\
58.27	0.01\\
58.28	0.01\\
58.29	0.01\\
58.3	0.01\\
58.31	0.01\\
58.32	0.01\\
58.33	0.01\\
58.34	0.01\\
58.35	0.01\\
58.36	0.01\\
58.37	0.01\\
58.38	0.01\\
58.39	0.01\\
58.4	0.01\\
58.41	0.01\\
58.42	0.01\\
58.43	0.01\\
58.44	0.01\\
58.45	0.01\\
58.46	0.01\\
58.47	0.01\\
58.48	0.01\\
58.49	0.01\\
58.5	0.01\\
58.51	0.01\\
58.52	0.01\\
58.53	0.01\\
58.54	0.01\\
58.55	0.01\\
58.56	0.01\\
58.57	0.01\\
58.58	0.01\\
58.59	0.01\\
58.6	0.01\\
58.61	0.01\\
58.62	0.01\\
58.63	0.01\\
58.64	0.01\\
58.65	0.01\\
58.66	0.01\\
58.67	0.01\\
58.68	0.01\\
58.69	0.01\\
58.7	0.01\\
58.71	0.01\\
58.72	0.01\\
58.73	0.01\\
58.74	0.01\\
58.75	0.01\\
58.76	0.01\\
58.77	0.01\\
58.78	0.01\\
58.79	0.01\\
58.8	0.01\\
58.81	0.01\\
58.82	0.01\\
58.83	0.01\\
58.84	0.01\\
58.85	0.01\\
58.86	0.01\\
58.87	0.01\\
58.88	0.01\\
58.89	0.01\\
58.9	0.01\\
58.91	0.01\\
58.92	0.01\\
58.93	0.01\\
58.94	0.01\\
58.95	0.01\\
58.96	0.01\\
58.97	0.01\\
58.98	0.01\\
58.99	0.01\\
59	0.01\\
59.01	0.01\\
59.02	0.01\\
59.03	0.01\\
59.04	0.01\\
59.05	0.01\\
59.06	0.01\\
59.07	0.01\\
59.08	0.01\\
59.09	0.01\\
59.1	0.01\\
59.11	0.01\\
59.12	0.01\\
59.13	0.01\\
59.14	0.01\\
59.15	0.01\\
59.16	0.01\\
59.17	0.01\\
59.18	0.01\\
59.19	0.01\\
59.2	0.01\\
59.21	0.01\\
59.22	0.01\\
59.23	0.01\\
59.24	0.01\\
59.25	0.01\\
59.26	0.01\\
59.27	0.01\\
59.28	0.01\\
59.29	0.01\\
59.3	0.01\\
59.31	0.01\\
59.32	0.01\\
59.33	0.01\\
59.34	0.01\\
59.35	0.01\\
59.36	0.01\\
59.37	0.01\\
59.38	0.01\\
59.39	0.01\\
59.4	0.01\\
59.41	0.01\\
59.42	0.01\\
59.43	0.01\\
59.44	0.01\\
59.45	0.01\\
59.46	0.01\\
59.47	0.01\\
59.48	0.01\\
59.49	0.01\\
59.5	0.01\\
59.51	0.01\\
59.52	0.01\\
59.53	0.01\\
59.54	0.01\\
59.55	0.01\\
59.56	0.01\\
59.57	0.01\\
59.58	0.01\\
59.59	0.01\\
59.6	0.01\\
59.61	0.01\\
59.62	0.01\\
59.63	0.01\\
59.64	0.01\\
59.65	0.01\\
59.66	0.01\\
59.67	0.01\\
59.68	0.01\\
59.69	0.01\\
59.7	0.01\\
59.71	0.01\\
59.72	0.01\\
59.73	0.01\\
59.74	0.01\\
59.75	0.01\\
59.76	0.01\\
59.77	0.01\\
59.78	0.01\\
59.79	0.01\\
59.8	0.01\\
59.81	0.01\\
59.82	0.01\\
59.83	0.01\\
59.84	0.01\\
59.85	0.01\\
59.86	0.01\\
59.87	0.01\\
59.88	0.01\\
59.89	0.01\\
59.9	0.01\\
59.91	0.01\\
59.92	0.01\\
59.93	0.01\\
59.94	0.01\\
59.95	0.01\\
59.96	0.01\\
59.97	0.01\\
59.98	0.01\\
59.99	0.01\\
60	0.01\\
60.01	0.01\\
60.02	0.01\\
60.03	0.01\\
60.04	0.01\\
60.05	0.01\\
60.06	0.01\\
60.07	0.01\\
60.08	0.01\\
60.09	0.01\\
60.1	0.01\\
60.11	0.01\\
60.12	0.01\\
60.13	0.01\\
60.14	0.01\\
60.15	0.01\\
60.16	0.01\\
60.17	0.01\\
60.18	0.01\\
60.19	0.01\\
60.2	0.01\\
60.21	0.01\\
60.22	0.01\\
60.23	0.01\\
60.24	0.01\\
60.25	0.01\\
60.26	0.01\\
60.27	0.01\\
60.28	0.01\\
60.29	0.01\\
60.3	0.01\\
60.31	0.01\\
60.32	0.01\\
60.33	0.01\\
60.34	0.01\\
60.35	0.01\\
60.36	0.01\\
60.37	0.01\\
60.38	0.01\\
60.39	0.01\\
60.4	0.01\\
60.41	0.01\\
60.42	0.01\\
60.43	0.01\\
60.44	0.01\\
60.45	0.01\\
60.46	0.01\\
60.47	0.01\\
60.48	0.01\\
60.49	0.01\\
60.5	0.01\\
60.51	0.01\\
60.52	0.01\\
60.53	0.01\\
60.54	0.01\\
60.55	0.01\\
60.56	0.01\\
60.57	0.01\\
60.58	0.01\\
60.59	0.01\\
60.6	0.01\\
60.61	0.01\\
60.62	0.01\\
60.63	0.01\\
60.64	0.01\\
60.65	0.01\\
60.66	0.01\\
60.67	0.01\\
60.68	0.01\\
60.69	0.01\\
60.7	0.01\\
60.71	0.01\\
60.72	0.01\\
60.73	0.01\\
60.74	0.01\\
60.75	0.01\\
60.76	0.01\\
60.77	0.01\\
60.78	0.01\\
60.79	0.01\\
60.8	0.01\\
60.81	0.01\\
60.82	0.01\\
60.83	0.01\\
60.84	0.01\\
60.85	0.01\\
60.86	0.01\\
60.87	0.01\\
60.88	0.01\\
60.89	0.01\\
60.9	0.01\\
60.91	0.01\\
60.92	0.01\\
60.93	0.01\\
60.94	0.01\\
60.95	0.01\\
60.96	0.01\\
60.97	0.01\\
60.98	0.01\\
60.99	0.01\\
61	0.01\\
61.01	0.01\\
61.02	0.01\\
61.03	0.01\\
61.04	0.01\\
61.05	0.01\\
61.06	0.01\\
61.07	0.01\\
61.08	0.01\\
61.09	0.01\\
61.1	0.01\\
61.11	0.01\\
61.12	0.01\\
61.13	0.01\\
61.14	0.01\\
61.15	0.01\\
61.16	0.01\\
61.17	0.01\\
61.18	0.01\\
61.19	0.01\\
61.2	0.01\\
61.21	0.01\\
61.22	0.01\\
61.23	0.01\\
61.24	0.01\\
61.25	0.01\\
61.26	0.01\\
61.27	0.01\\
61.28	0.01\\
61.29	0.01\\
61.3	0.01\\
61.31	0.01\\
61.32	0.01\\
61.33	0.01\\
61.34	0.01\\
61.35	0.01\\
61.36	0.01\\
61.37	0.01\\
61.38	0.01\\
61.39	0.01\\
61.4	0.01\\
61.41	0.01\\
61.42	0.01\\
61.43	0.01\\
61.44	0.01\\
61.45	0.01\\
61.46	0.01\\
61.47	0.01\\
61.48	0.01\\
61.49	0.01\\
61.5	0.01\\
61.51	0.01\\
61.52	0.01\\
61.53	0.01\\
61.54	0.01\\
61.55	0.01\\
61.56	0.01\\
61.57	0.01\\
61.58	0.01\\
61.59	0.01\\
61.6	0.01\\
61.61	0.01\\
61.62	0.01\\
61.63	0.01\\
61.64	0.01\\
61.65	0.01\\
61.66	0.01\\
61.67	0.01\\
61.68	0.01\\
61.69	0.01\\
61.7	0.01\\
61.71	0.01\\
61.72	0.01\\
61.73	0.01\\
61.74	0.01\\
61.75	0.01\\
61.76	0.01\\
61.77	0.01\\
61.78	0.01\\
61.79	0.01\\
61.8	0.01\\
61.81	0.01\\
61.82	0.01\\
61.83	0.01\\
61.84	0.01\\
61.85	0.01\\
61.86	0.01\\
61.87	0.01\\
61.88	0.01\\
61.89	0.01\\
61.9	0.01\\
61.91	0.01\\
61.92	0.01\\
61.93	0.01\\
61.94	0.01\\
61.95	0.01\\
61.96	0.01\\
61.97	0.01\\
61.98	0.01\\
61.99	0.01\\
62	0.01\\
62.01	0.01\\
62.02	0.01\\
62.03	0.01\\
62.04	0.01\\
62.05	0.01\\
62.06	0.01\\
62.07	0.01\\
62.08	0.01\\
62.09	0.01\\
62.1	0.01\\
62.11	0.01\\
62.12	0.01\\
62.13	0.01\\
62.14	0.01\\
62.15	0.01\\
62.16	0.01\\
62.17	0.01\\
62.18	0.01\\
62.19	0.01\\
62.2	0.01\\
62.21	0.01\\
62.22	0.01\\
62.23	0.01\\
62.24	0.01\\
62.25	0.01\\
62.26	0.01\\
62.27	0.01\\
62.28	0.01\\
62.29	0.01\\
62.3	0.01\\
62.31	0.01\\
62.32	0.01\\
62.33	0.01\\
62.34	0.01\\
62.35	0.01\\
62.36	0.01\\
62.37	0.01\\
62.38	0.01\\
62.39	0.01\\
62.4	0.01\\
62.41	0.01\\
62.42	0.01\\
62.43	0.01\\
62.44	0.01\\
62.45	0.01\\
62.46	0.01\\
62.47	0.01\\
62.48	0.01\\
62.49	0.01\\
62.5	0.01\\
62.51	0.01\\
62.52	0.01\\
62.53	0.01\\
62.54	0.01\\
62.55	0.01\\
62.56	0.01\\
62.57	0.01\\
62.58	0.01\\
62.59	0.01\\
62.6	0.01\\
62.61	0.01\\
62.62	0.01\\
62.63	0.01\\
62.64	0.01\\
62.65	0.01\\
62.66	0.01\\
62.67	0.01\\
62.68	0.01\\
62.69	0.01\\
62.7	0.01\\
62.71	0.01\\
62.72	0.01\\
62.73	0.01\\
62.74	0.01\\
62.75	0.01\\
62.76	0.01\\
62.77	0.01\\
62.78	0.01\\
62.79	0.01\\
62.8	0.01\\
62.81	0.01\\
62.82	0.01\\
62.83	0.01\\
62.84	0.01\\
62.85	0.01\\
62.86	0.01\\
62.87	0.01\\
62.88	0.01\\
62.89	0.01\\
62.9	0.01\\
62.91	0.01\\
62.92	0.01\\
62.93	0.01\\
62.94	0.01\\
62.95	0.01\\
62.96	0.01\\
62.97	0.01\\
62.98	0.01\\
62.99	0.01\\
63	0.01\\
63.01	0.01\\
63.02	0.01\\
63.03	0.01\\
63.04	0.01\\
63.05	0.01\\
63.06	0.01\\
63.07	0.01\\
63.08	0.01\\
63.09	0.01\\
63.1	0.01\\
63.11	0.01\\
63.12	0.01\\
63.13	0.01\\
63.14	0.01\\
63.15	0.01\\
63.16	0.01\\
63.17	0.01\\
63.18	0.01\\
63.19	0.01\\
63.2	0.01\\
63.21	0.01\\
63.22	0.01\\
63.23	0.01\\
63.24	0.01\\
63.25	0.01\\
63.26	0.01\\
63.27	0.01\\
63.28	0.01\\
63.29	0.01\\
63.3	0.01\\
63.31	0.01\\
63.32	0.01\\
63.33	0.01\\
63.34	0.01\\
63.35	0.01\\
63.36	0.01\\
63.37	0.01\\
63.38	0.01\\
63.39	0.01\\
63.4	0.01\\
63.41	0.01\\
63.42	0.01\\
63.43	0.01\\
63.44	0.01\\
63.45	0.01\\
63.46	0.01\\
63.47	0.01\\
63.48	0.01\\
63.49	0.01\\
63.5	0.01\\
63.51	0.01\\
63.52	0.01\\
63.53	0.01\\
63.54	0.01\\
63.55	0.01\\
63.56	0.01\\
63.57	0.01\\
63.58	0.01\\
63.59	0.01\\
63.6	0.01\\
63.61	0.01\\
63.62	0.01\\
63.63	0.01\\
63.64	0.01\\
63.65	0.01\\
63.66	0.01\\
63.67	0.01\\
63.68	0.01\\
63.69	0.01\\
63.7	0.01\\
63.71	0.01\\
63.72	0.01\\
63.73	0.01\\
63.74	0.01\\
63.75	0.01\\
63.76	0.01\\
63.77	0.01\\
63.78	0.01\\
63.79	0.01\\
63.8	0.01\\
63.81	0.01\\
63.82	0.01\\
63.83	0.01\\
63.84	0.01\\
63.85	0.01\\
63.86	0.01\\
63.87	0.01\\
63.88	0.01\\
63.89	0.01\\
63.9	0.01\\
63.91	0.01\\
63.92	0.01\\
63.93	0.01\\
63.94	0.01\\
63.95	0.01\\
63.96	0.01\\
63.97	0.01\\
63.98	0.01\\
63.99	0.01\\
64	0.01\\
64.01	0.01\\
64.02	0.01\\
64.03	0.01\\
64.04	0.01\\
64.05	0.01\\
64.06	0.01\\
64.07	0.01\\
64.08	0.01\\
64.09	0.01\\
64.1	0.01\\
64.11	0.01\\
64.12	0.01\\
64.13	0.01\\
64.14	0.01\\
64.15	0.01\\
64.16	0.01\\
64.17	0.01\\
64.18	0.01\\
64.19	0.01\\
64.2	0.01\\
64.21	0.01\\
64.22	0.01\\
64.23	0.01\\
64.24	0.01\\
64.25	0.01\\
64.26	0.01\\
64.27	0.01\\
64.28	0.01\\
64.29	0.01\\
64.3	0.01\\
64.31	0.01\\
64.32	0.01\\
64.33	0.01\\
64.34	0.01\\
64.35	0.01\\
64.36	0.01\\
64.37	0.01\\
64.38	0.01\\
64.39	0.01\\
64.4	0.01\\
64.41	0.01\\
64.42	0.01\\
64.43	0.01\\
64.44	0.01\\
64.45	0.01\\
64.46	0.01\\
64.47	0.01\\
64.48	0.01\\
64.49	0.01\\
64.5	0.01\\
64.51	0.01\\
64.52	0.01\\
64.53	0.01\\
64.54	0.01\\
64.55	0.01\\
64.56	0.01\\
64.57	0.01\\
64.58	0.01\\
64.59	0.01\\
64.6	0.01\\
64.61	0.01\\
64.62	0.01\\
64.63	0.01\\
64.64	0.01\\
64.65	0.01\\
64.66	0.01\\
64.67	0.01\\
64.68	0.01\\
64.69	0.01\\
64.7	0.01\\
64.71	0.01\\
64.72	0.01\\
64.73	0.01\\
64.74	0.01\\
64.75	0.01\\
64.76	0.01\\
64.77	0.01\\
64.78	0.01\\
64.79	0.01\\
64.8	0.01\\
64.81	0.01\\
64.82	0.01\\
64.83	0.01\\
64.84	0.01\\
64.85	0.01\\
64.86	0.01\\
64.87	0.01\\
64.88	0.01\\
64.89	0.01\\
64.9	0.01\\
64.91	0.01\\
64.92	0.01\\
64.93	0.01\\
64.94	0.01\\
64.95	0.01\\
64.96	0.01\\
64.97	0.01\\
64.98	0.01\\
64.99	0.01\\
65	0.01\\
65.01	0.01\\
65.02	0.01\\
65.03	0.01\\
65.04	0.01\\
65.05	0.01\\
65.06	0.01\\
65.07	0.01\\
65.08	0.01\\
65.09	0.01\\
65.1	0.01\\
65.11	0.01\\
65.12	0.01\\
65.13	0.01\\
65.14	0.01\\
65.15	0.01\\
65.16	0.01\\
65.17	0.01\\
65.18	0.01\\
65.19	0.01\\
65.2	0.01\\
65.21	0.01\\
65.22	0.01\\
65.23	0.01\\
65.24	0.01\\
65.25	0.01\\
65.26	0.01\\
65.27	0.01\\
65.28	0.01\\
65.29	0.01\\
65.3	0.01\\
65.31	0.01\\
65.32	0.01\\
65.33	0.01\\
65.34	0.01\\
65.35	0.01\\
65.36	0.01\\
65.37	0.01\\
65.38	0.01\\
65.39	0.01\\
65.4	0.01\\
65.41	0.01\\
65.42	0.01\\
65.43	0.01\\
65.44	0.01\\
65.45	0.01\\
65.46	0.01\\
65.47	0.01\\
65.48	0.01\\
65.49	0.01\\
65.5	0.01\\
65.51	0.01\\
65.52	0.01\\
65.53	0.01\\
65.54	0.01\\
65.55	0.01\\
65.56	0.01\\
65.57	0.01\\
65.58	0.01\\
65.59	0.01\\
65.6	0.01\\
65.61	0.01\\
65.62	0.01\\
65.63	0.01\\
65.64	0.01\\
65.65	0.01\\
65.66	0.01\\
65.67	0.01\\
65.68	0.01\\
65.69	0.01\\
65.7	0.01\\
65.71	0.01\\
65.72	0.01\\
65.73	0.01\\
65.74	0.01\\
65.75	0.01\\
65.76	0.01\\
65.77	0.01\\
65.78	0.01\\
65.79	0.01\\
65.8	0.01\\
65.81	0.01\\
65.82	0.01\\
65.83	0.01\\
65.84	0.01\\
65.85	0.01\\
65.86	0.01\\
65.87	0.01\\
65.88	0.01\\
65.89	0.01\\
65.9	0.01\\
65.91	0.01\\
65.92	0.01\\
65.93	0.01\\
65.94	0.01\\
65.95	0.01\\
65.96	0.01\\
65.97	0.01\\
65.98	0.01\\
65.99	0.01\\
66	0.01\\
66.01	0.01\\
66.02	0.01\\
66.03	0.01\\
66.04	0.01\\
66.05	0.01\\
66.06	0.01\\
66.07	0.01\\
66.08	0.01\\
66.09	0.01\\
66.1	0.01\\
66.11	0.01\\
66.12	0.01\\
66.13	0.01\\
66.14	0.01\\
66.15	0.01\\
66.16	0.01\\
66.17	0.01\\
66.18	0.01\\
66.19	0.01\\
66.2	0.01\\
66.21	0.01\\
66.22	0.01\\
66.23	0.01\\
66.24	0.01\\
66.25	0.01\\
66.26	0.01\\
66.27	0.01\\
66.28	0.01\\
66.29	0.01\\
66.3	0.01\\
66.31	0.01\\
66.32	0.01\\
66.33	0.01\\
66.34	0.01\\
66.35	0.01\\
66.36	0.01\\
66.37	0.01\\
66.38	0.01\\
66.39	0.01\\
66.4	0.01\\
66.41	0.01\\
66.42	0.01\\
66.43	0.01\\
66.44	0.01\\
66.45	0.01\\
66.46	0.01\\
66.47	0.01\\
66.48	0.01\\
66.49	0.01\\
66.5	0.01\\
66.51	0.01\\
66.52	0.01\\
66.53	0.01\\
66.54	0.01\\
66.55	0.01\\
66.56	0.01\\
66.57	0.01\\
66.58	0.01\\
66.59	0.01\\
66.6	0.01\\
66.61	0.01\\
66.62	0.01\\
66.63	0.01\\
66.64	0.01\\
66.65	0.01\\
66.66	0.01\\
66.67	0.01\\
66.68	0.01\\
66.69	0.01\\
66.7	0.01\\
66.71	0.01\\
66.72	0.01\\
66.73	0.01\\
66.74	0.01\\
66.75	0.01\\
66.76	0.01\\
66.77	0.01\\
66.78	0.01\\
66.79	0.01\\
66.8	0.01\\
66.81	0.01\\
66.82	0.01\\
66.83	0.01\\
66.84	0.01\\
66.85	0.01\\
66.86	0.01\\
66.87	0.01\\
66.88	0.01\\
66.89	0.01\\
66.9	0.01\\
66.91	0.01\\
66.92	0.01\\
66.93	0.01\\
66.94	0.01\\
66.95	0.01\\
66.96	0.01\\
66.97	0.01\\
66.98	0.01\\
66.99	0.01\\
67	0.01\\
67.01	0.01\\
67.02	0.01\\
67.03	0.01\\
67.04	0.01\\
67.05	0.01\\
67.06	0.01\\
67.07	0.01\\
67.08	0.01\\
67.09	0.01\\
67.1	0.01\\
67.11	0.01\\
67.12	0.01\\
67.13	0.01\\
67.14	0.01\\
67.15	0.01\\
67.16	0.01\\
67.17	0.01\\
67.18	0.01\\
67.19	0.01\\
67.2	0.01\\
67.21	0.01\\
67.22	0.01\\
67.23	0.01\\
67.24	0.01\\
67.25	0.01\\
67.26	0.01\\
67.27	0.01\\
67.28	0.01\\
67.29	0.01\\
67.3	0.01\\
67.31	0.01\\
67.32	0.01\\
67.33	0.01\\
67.34	0.01\\
67.35	0.01\\
67.36	0.01\\
67.37	0.01\\
67.38	0.01\\
67.39	0.01\\
67.4	0.01\\
67.41	0.01\\
67.42	0.01\\
67.43	0.01\\
67.44	0.01\\
67.45	0.01\\
67.46	0.01\\
67.47	0.01\\
67.48	0.01\\
67.49	0.01\\
67.5	0.01\\
67.51	0.01\\
67.52	0.01\\
67.53	0.01\\
67.54	0.01\\
67.55	0.01\\
67.56	0.01\\
67.57	0.01\\
67.58	0.01\\
67.59	0.01\\
67.6	0.01\\
67.61	0.01\\
67.62	0.01\\
67.63	0.01\\
67.64	0.01\\
67.65	0.01\\
67.66	0.01\\
67.67	0.01\\
67.68	0.01\\
67.69	0.01\\
67.7	0.01\\
67.71	0.01\\
67.72	0.01\\
67.73	0.01\\
67.74	0.01\\
67.75	0.01\\
67.76	0.01\\
67.77	0.01\\
67.78	0.01\\
67.79	0.01\\
67.8	0.01\\
67.81	0.01\\
67.82	0.01\\
67.83	0.01\\
67.84	0.01\\
67.85	0.01\\
67.86	0.01\\
67.87	0.01\\
67.88	0.01\\
67.89	0.01\\
67.9	0.01\\
67.91	0.01\\
67.92	0.01\\
67.93	0.01\\
67.94	0.01\\
67.95	0.01\\
67.96	0.01\\
67.97	0.01\\
67.98	0.01\\
67.99	0.01\\
68	0.01\\
68.01	0.01\\
68.02	0.01\\
68.03	0.01\\
68.04	0.01\\
68.05	0.01\\
68.06	0.01\\
68.07	0.01\\
68.08	0.01\\
68.09	0.01\\
68.1	0.01\\
68.11	0.01\\
68.12	0.01\\
68.13	0.01\\
68.14	0.01\\
68.15	0.01\\
68.16	0.01\\
68.17	0.01\\
68.18	0.01\\
68.19	0.01\\
68.2	0.01\\
68.21	0.01\\
68.22	0.01\\
68.23	0.01\\
68.24	0.01\\
68.25	0.01\\
68.26	0.01\\
68.27	0.01\\
68.28	0.01\\
68.29	0.01\\
68.3	0.01\\
68.31	0.01\\
68.32	0.01\\
68.33	0.01\\
68.34	0.01\\
68.35	0.01\\
68.36	0.01\\
68.37	0.01\\
68.38	0.01\\
68.39	0.01\\
68.4	0.01\\
68.41	0.01\\
68.42	0.01\\
68.43	0.01\\
68.44	0.01\\
68.45	0.01\\
68.46	0.01\\
68.47	0.01\\
68.48	0.01\\
68.49	0.01\\
68.5	0.01\\
68.51	0.01\\
68.52	0.01\\
68.53	0.01\\
68.54	0.01\\
68.55	0.01\\
68.56	0.01\\
68.57	0.01\\
68.58	0.01\\
68.59	0.01\\
68.6	0.01\\
68.61	0.01\\
68.62	0.01\\
68.63	0.01\\
68.64	0.01\\
68.65	0.01\\
68.66	0.01\\
68.67	0.01\\
68.68	0.01\\
68.69	0.01\\
68.7	0.01\\
68.71	0.01\\
68.72	0.01\\
68.73	0.01\\
68.74	0.01\\
68.75	0.01\\
68.76	0.01\\
68.77	0.01\\
68.78	0.01\\
68.79	0.01\\
68.8	0.01\\
68.81	0.01\\
68.82	0.01\\
68.83	0.01\\
68.84	0.01\\
68.85	0.01\\
68.86	0.01\\
68.87	0.01\\
68.88	0.01\\
68.89	0.01\\
68.9	0.01\\
68.91	0.01\\
68.92	0.01\\
68.93	0.01\\
68.94	0.01\\
68.95	0.01\\
68.96	0.01\\
68.97	0.01\\
68.98	0.01\\
68.99	0.01\\
69	0.01\\
69.01	0.01\\
69.02	0.01\\
69.03	0.01\\
69.04	0.01\\
69.05	0.01\\
69.06	0.01\\
69.07	0.01\\
69.08	0.01\\
69.09	0.01\\
69.1	0.01\\
69.11	0.01\\
69.12	0.01\\
69.13	0.01\\
69.14	0.01\\
69.15	0.01\\
69.16	0.01\\
69.17	0.01\\
69.18	0.01\\
69.19	0.01\\
69.2	0.01\\
69.21	0.01\\
69.22	0.01\\
69.23	0.01\\
69.24	0.01\\
69.25	0.01\\
69.26	0.01\\
69.27	0.01\\
69.28	0.01\\
69.29	0.01\\
69.3	0.01\\
69.31	0.01\\
69.32	0.01\\
69.33	0.01\\
69.34	0.01\\
69.35	0.01\\
69.36	0.01\\
69.37	0.01\\
69.38	0.01\\
69.39	0.01\\
69.4	0.01\\
69.41	0.01\\
69.42	0.01\\
69.43	0.01\\
69.44	0.01\\
69.45	0.01\\
69.46	0.01\\
69.47	0.01\\
69.48	0.01\\
69.49	0.01\\
69.5	0.01\\
69.51	0.01\\
69.52	0.01\\
69.53	0.01\\
69.54	0.01\\
69.55	0.01\\
69.56	0.01\\
69.57	0.01\\
69.58	0.01\\
69.59	0.01\\
69.6	0.01\\
69.61	0.01\\
69.62	0.01\\
69.63	0.01\\
69.64	0.01\\
69.65	0.01\\
69.66	0.01\\
69.67	0.01\\
69.68	0.01\\
69.69	0.01\\
69.7	0.01\\
69.71	0.01\\
69.72	0.01\\
69.73	0.01\\
69.74	0.01\\
69.75	0.01\\
69.76	0.01\\
69.77	0.01\\
69.78	0.01\\
69.79	0.01\\
69.8	0.01\\
69.81	0.01\\
69.82	0.01\\
69.83	0.01\\
69.84	0.01\\
69.85	0.01\\
69.86	0.01\\
69.87	0.01\\
69.88	0.01\\
69.89	0.01\\
69.9	0.01\\
69.91	0.01\\
69.92	0.01\\
69.93	0.01\\
69.94	0.01\\
69.95	0.01\\
69.96	0.01\\
69.97	0.01\\
69.98	0.01\\
69.99	0.01\\
70	0.01\\
70.01	0.01\\
70.02	0.01\\
70.03	0.01\\
70.04	0.01\\
70.05	0.01\\
70.06	0.01\\
70.07	0.01\\
70.08	0.01\\
70.09	0.01\\
70.1	0.01\\
70.11	0.01\\
70.12	0.01\\
70.13	0.01\\
70.14	0.01\\
70.15	0.01\\
70.16	0.01\\
70.17	0.01\\
70.18	0.01\\
70.19	0.01\\
70.2	0.01\\
70.21	0.01\\
70.22	0.01\\
70.23	0.01\\
70.24	0.01\\
70.25	0.01\\
70.26	0.01\\
70.27	0.01\\
70.28	0.01\\
70.29	0.01\\
70.3	0.01\\
70.31	0.01\\
70.32	0.01\\
70.33	0.01\\
70.34	0.01\\
70.35	0.01\\
70.36	0.01\\
70.37	0.01\\
70.38	0.01\\
70.39	0.01\\
70.4	0.01\\
70.41	0.01\\
70.42	0.01\\
70.43	0.01\\
70.44	0.01\\
70.45	0.01\\
70.46	0.01\\
70.47	0.01\\
70.48	0.01\\
70.49	0.01\\
70.5	0.01\\
70.51	0.01\\
70.52	0.01\\
70.53	0.01\\
70.54	0.01\\
70.55	0.01\\
70.56	0.01\\
70.57	0.01\\
70.58	0.01\\
70.59	0.01\\
70.6	0.01\\
70.61	0.01\\
70.62	0.01\\
70.63	0.01\\
70.64	0.01\\
70.65	0.01\\
70.66	0.01\\
70.67	0.01\\
70.68	0.01\\
70.69	0.01\\
70.7	0.01\\
70.71	0.01\\
70.72	0.01\\
70.73	0.01\\
70.74	0.01\\
70.75	0.01\\
70.76	0.01\\
70.77	0.01\\
70.78	0.01\\
70.79	0.01\\
70.8	0.01\\
70.81	0.01\\
70.82	0.01\\
70.83	0.01\\
70.84	0.01\\
70.85	0.01\\
70.86	0.01\\
70.87	0.01\\
70.88	0.01\\
70.89	0.01\\
70.9	0.01\\
70.91	0.01\\
70.92	0.01\\
70.93	0.01\\
70.94	0.01\\
70.95	0.01\\
70.96	0.01\\
70.97	0.01\\
70.98	0.01\\
70.99	0.01\\
71	0.01\\
71.01	0.01\\
71.02	0.01\\
71.03	0.01\\
71.04	0.01\\
71.05	0.01\\
71.06	0.01\\
71.07	0.01\\
71.08	0.01\\
71.09	0.01\\
71.1	0.01\\
71.11	0.01\\
71.12	0.01\\
71.13	0.01\\
71.14	0.01\\
71.15	0.01\\
71.16	0.01\\
71.17	0.01\\
71.18	0.01\\
71.19	0.01\\
71.2	0.01\\
71.21	0.01\\
71.22	0.01\\
71.23	0.01\\
71.24	0.01\\
71.25	0.01\\
71.26	0.01\\
71.27	0.01\\
71.28	0.01\\
71.29	0.01\\
71.3	0.01\\
71.31	0.01\\
71.32	0.01\\
71.33	0.01\\
71.34	0.01\\
71.35	0.01\\
71.36	0.01\\
71.37	0.01\\
71.38	0.01\\
71.39	0.01\\
71.4	0.01\\
71.41	0.01\\
71.42	0.01\\
71.43	0.01\\
71.44	0.01\\
71.45	0.01\\
71.46	0.01\\
71.47	0.01\\
71.48	0.01\\
71.49	0.01\\
71.5	0.01\\
71.51	0.01\\
71.52	0.01\\
71.53	0.01\\
71.54	0.01\\
71.55	0.01\\
71.56	0.01\\
71.57	0.01\\
71.58	0.01\\
71.59	0.01\\
71.6	0.01\\
71.61	0.01\\
71.62	0.01\\
71.63	0.01\\
71.64	0.01\\
71.65	0.01\\
71.66	0.01\\
71.67	0.01\\
71.68	0.01\\
71.69	0.01\\
71.7	0.01\\
71.71	0.01\\
71.72	0.01\\
71.73	0.01\\
71.74	0.01\\
71.75	0.01\\
71.76	0.01\\
71.77	0.01\\
71.78	0.01\\
71.79	0.01\\
71.8	0.01\\
71.81	0.01\\
71.82	0.01\\
71.83	0.01\\
71.84	0.01\\
71.85	0.01\\
71.86	0.01\\
71.87	0.01\\
71.88	0.01\\
71.89	0.01\\
71.9	0.01\\
71.91	0.01\\
71.92	0.01\\
71.93	0.01\\
71.94	0.01\\
71.95	0.01\\
71.96	0.01\\
71.97	0.01\\
71.98	0.01\\
71.99	0.01\\
72	0.01\\
72.01	0.01\\
72.02	0.01\\
72.03	0.01\\
72.04	0.01\\
72.05	0.01\\
72.06	0.01\\
72.07	0.01\\
72.08	0.01\\
72.09	0.01\\
72.1	0.01\\
72.11	0.01\\
72.12	0.01\\
72.13	0.01\\
72.14	0.01\\
72.15	0.01\\
72.16	0.01\\
72.17	0.01\\
72.18	0.01\\
72.19	0.01\\
72.2	0.01\\
72.21	0.01\\
72.22	0.01\\
72.23	0.01\\
72.24	0.01\\
72.25	0.01\\
72.26	0.01\\
72.27	0.01\\
72.28	0.01\\
72.29	0.01\\
72.3	0.01\\
72.31	0.01\\
72.32	0.01\\
72.33	0.01\\
72.34	0.01\\
72.35	0.01\\
72.36	0.01\\
72.37	0.01\\
72.38	0.01\\
72.39	0.01\\
72.4	0.01\\
72.41	0.01\\
72.42	0.01\\
72.43	0.01\\
72.44	0.01\\
72.45	0.01\\
72.46	0.01\\
72.47	0.01\\
72.48	0.01\\
72.49	0.01\\
72.5	0.01\\
72.51	0.01\\
72.52	0.01\\
72.53	0.01\\
72.54	0.01\\
72.55	0.01\\
72.56	0.01\\
72.57	0.01\\
72.58	0.01\\
72.59	0.01\\
72.6	0.01\\
72.61	0.01\\
72.62	0.01\\
72.63	0.01\\
72.64	0.01\\
72.65	0.01\\
72.66	0.01\\
72.67	0.01\\
72.68	0.01\\
72.69	0.01\\
72.7	0.01\\
72.71	0.01\\
72.72	0.01\\
72.73	0.01\\
72.74	0.01\\
72.75	0.01\\
72.76	0.01\\
72.77	0.01\\
72.78	0.01\\
72.79	0.01\\
72.8	0.01\\
72.81	0.01\\
72.82	0.01\\
72.83	0.01\\
72.84	0.01\\
72.85	0.01\\
72.86	0.01\\
72.87	0.01\\
72.88	0.01\\
72.89	0.01\\
72.9	0.01\\
72.91	0.01\\
72.92	0.01\\
72.93	0.01\\
72.94	0.01\\
72.95	0.01\\
72.96	0.01\\
72.97	0.01\\
72.98	0.01\\
72.99	0.01\\
73	0.01\\
73.01	0.01\\
73.02	0.01\\
73.03	0.01\\
73.04	0.01\\
73.05	0.01\\
73.06	0.01\\
73.07	0.01\\
73.08	0.01\\
73.09	0.01\\
73.1	0.01\\
73.11	0.01\\
73.12	0.01\\
73.13	0.01\\
73.14	0.01\\
73.15	0.01\\
73.16	0.01\\
73.17	0.01\\
73.18	0.01\\
73.19	0.01\\
73.2	0.01\\
73.21	0.01\\
73.22	0.01\\
73.23	0.01\\
73.24	0.01\\
73.25	0.01\\
73.26	0.01\\
73.27	0.01\\
73.28	0.01\\
73.29	0.01\\
73.3	0.01\\
73.31	0.01\\
73.32	0.01\\
73.33	0.01\\
73.34	0.01\\
73.35	0.01\\
73.36	0.01\\
73.37	0.01\\
73.38	0.01\\
73.39	0.01\\
73.4	0.01\\
73.41	0.01\\
73.42	0.01\\
73.43	0.01\\
73.44	0.01\\
73.45	0.01\\
73.46	0.01\\
73.47	0.01\\
73.48	0.01\\
73.49	0.01\\
73.5	0.01\\
73.51	0.01\\
73.52	0.01\\
73.53	0.01\\
73.54	0.01\\
73.55	0.01\\
73.56	0.01\\
73.57	0.01\\
73.58	0.01\\
73.59	0.01\\
73.6	0.01\\
73.61	0.01\\
73.62	0.01\\
73.63	0.01\\
73.64	0.01\\
73.65	0.01\\
73.66	0.01\\
73.67	0.01\\
73.68	0.01\\
73.69	0.01\\
73.7	0.01\\
73.71	0.01\\
73.72	0.01\\
73.73	0.01\\
73.74	0.01\\
73.75	0.01\\
73.76	0.01\\
73.77	0.01\\
73.78	0.01\\
73.79	0.01\\
73.8	0.01\\
73.81	0.01\\
73.82	0.01\\
73.83	0.01\\
73.84	0.01\\
73.85	0.01\\
73.86	0.01\\
73.87	0.01\\
73.88	0.01\\
73.89	0.01\\
73.9	0.01\\
73.91	0.01\\
73.92	0.01\\
73.93	0.01\\
73.94	0.01\\
73.95	0.01\\
73.96	0.01\\
73.97	0.01\\
73.98	0.01\\
73.99	0.01\\
74	0.01\\
74.01	0.01\\
74.02	0.01\\
74.03	0.01\\
74.04	0.01\\
74.05	0.01\\
74.06	0.01\\
74.07	0.01\\
74.08	0.01\\
74.09	0.01\\
74.1	0.01\\
74.11	0.01\\
74.12	0.01\\
74.13	0.01\\
74.14	0.01\\
74.15	0.01\\
74.16	0.01\\
74.17	0.01\\
74.18	0.01\\
74.19	0.01\\
74.2	0.01\\
74.21	0.01\\
74.22	0.01\\
74.23	0.01\\
74.24	0.01\\
74.25	0.01\\
74.26	0.01\\
74.27	0.01\\
74.28	0.01\\
74.29	0.01\\
74.3	0.01\\
74.31	0.01\\
74.32	0.01\\
74.33	0.01\\
74.34	0.01\\
74.35	0.01\\
74.36	0.01\\
74.37	0.01\\
74.38	0.01\\
74.39	0.01\\
74.4	0.01\\
74.41	0.01\\
74.42	0.01\\
74.43	0.01\\
74.44	0.01\\
74.45	0.01\\
74.46	0.01\\
74.47	0.01\\
74.48	0.01\\
74.49	0.01\\
74.5	0.01\\
74.51	0.01\\
74.52	0.01\\
74.53	0.01\\
74.54	0.01\\
74.55	0.01\\
74.56	0.01\\
74.57	0.01\\
74.58	0.01\\
74.59	0.01\\
74.6	0.01\\
74.61	0.01\\
74.62	0.01\\
74.63	0.01\\
74.64	0.01\\
74.65	0.01\\
74.66	0.01\\
74.67	0.01\\
74.68	0.01\\
74.69	0.01\\
74.7	0.01\\
74.71	0.01\\
74.72	0.01\\
74.73	0.01\\
74.74	0.01\\
74.75	0.01\\
74.76	0.01\\
74.77	0.01\\
74.78	0.01\\
74.79	0.01\\
74.8	0.01\\
74.81	0.01\\
74.82	0.01\\
74.83	0.01\\
74.84	0.01\\
74.85	0.01\\
74.86	0.01\\
74.87	0.01\\
74.88	0.01\\
74.89	0.01\\
74.9	0.01\\
74.91	0.01\\
74.92	0.01\\
74.93	0.01\\
74.94	0.01\\
74.95	0.01\\
74.96	0.01\\
74.97	0.01\\
74.98	0.01\\
74.99	0.01\\
75	0.01\\
75.01	0.01\\
75.02	0.01\\
75.03	0.01\\
75.04	0.01\\
75.05	0.01\\
75.06	0.01\\
75.07	0.01\\
75.08	0.01\\
75.09	0.01\\
75.1	0.01\\
75.11	0.01\\
75.12	0.01\\
75.13	0.01\\
75.14	0.01\\
75.15	0.01\\
75.16	0.01\\
75.17	0.01\\
75.18	0.01\\
75.19	0.01\\
75.2	0.01\\
75.21	0.01\\
75.22	0.01\\
75.23	0.01\\
75.24	0.01\\
75.25	0.01\\
75.26	0.01\\
75.27	0.01\\
75.28	0.01\\
75.29	0.01\\
75.3	0.01\\
75.31	0.01\\
75.32	0.01\\
75.33	0.01\\
75.34	0.01\\
75.35	0.01\\
75.36	0.01\\
75.37	0.01\\
75.38	0.01\\
75.39	0.01\\
75.4	0.01\\
75.41	0.01\\
75.42	0.01\\
75.43	0.01\\
75.44	0.01\\
75.45	0.01\\
75.46	0.01\\
75.47	0.01\\
75.48	0.01\\
75.49	0.01\\
75.5	0.01\\
75.51	0.01\\
75.52	0.01\\
75.53	0.01\\
75.54	0.01\\
75.55	0.01\\
75.56	0.01\\
75.57	0.01\\
75.58	0.01\\
75.59	0.01\\
75.6	0.01\\
75.61	0.01\\
75.62	0.01\\
75.63	0.01\\
75.64	0.01\\
75.65	0.01\\
75.66	0.01\\
75.67	0.01\\
75.68	0.01\\
75.69	0.01\\
75.7	0.01\\
75.71	0.01\\
75.72	0.01\\
75.73	0.01\\
75.74	0.01\\
75.75	0.01\\
75.76	0.01\\
75.77	0.01\\
75.78	0.01\\
75.79	0.01\\
75.8	0.01\\
75.81	0.01\\
75.82	0.01\\
75.83	0.01\\
75.84	0.01\\
75.85	0.01\\
75.86	0.01\\
75.87	0.01\\
75.88	0.01\\
75.89	0.01\\
75.9	0.01\\
75.91	0.01\\
75.92	0.01\\
75.93	0.01\\
75.94	0.01\\
75.95	0.01\\
75.96	0.01\\
75.97	0.01\\
75.98	0.01\\
75.99	0.01\\
76	0.01\\
76.01	0.01\\
76.02	0.01\\
76.03	0.01\\
76.04	0.01\\
76.05	0.01\\
76.06	0.01\\
76.07	0.01\\
76.08	0.01\\
76.09	0.01\\
76.1	0.01\\
76.11	0.01\\
76.12	0.01\\
76.13	0.01\\
76.14	0.01\\
76.15	0.01\\
76.16	0.01\\
76.17	0.01\\
76.18	0.01\\
76.19	0.01\\
76.2	0.01\\
76.21	0.01\\
76.22	0.01\\
76.23	0.01\\
76.24	0.01\\
76.25	0.01\\
76.26	0.01\\
76.27	0.01\\
76.28	0.01\\
76.29	0.01\\
76.3	0.01\\
76.31	0.01\\
76.32	0.01\\
76.33	0.01\\
76.34	0.01\\
76.35	0.01\\
76.36	0.01\\
76.37	0.01\\
76.38	0.01\\
76.39	0.01\\
76.4	0.01\\
76.41	0.01\\
76.42	0.01\\
76.43	0.01\\
76.44	0.01\\
76.45	0.01\\
76.46	0.01\\
76.47	0.01\\
76.48	0.01\\
76.49	0.01\\
76.5	0.01\\
76.51	0.01\\
76.52	0.01\\
76.53	0.01\\
76.54	0.01\\
76.55	0.01\\
76.56	0.01\\
76.57	0.01\\
76.58	0.01\\
76.59	0.01\\
76.6	0.01\\
76.61	0.01\\
76.62	0.01\\
76.63	0.01\\
76.64	0.01\\
76.65	0.01\\
76.66	0.01\\
76.67	0.01\\
76.68	0.01\\
76.69	0.01\\
76.7	0.01\\
76.71	0.01\\
76.72	0.01\\
76.73	0.01\\
76.74	0.01\\
76.75	0.01\\
76.76	0.01\\
76.77	0.01\\
76.78	0.01\\
76.79	0.01\\
76.8	0.01\\
76.81	0.01\\
76.82	0.01\\
76.83	0.01\\
76.84	0.01\\
76.85	0.01\\
76.86	0.01\\
76.87	0.01\\
76.88	0.01\\
76.89	0.01\\
76.9	0.01\\
76.91	0.01\\
76.92	0.01\\
76.93	0.01\\
76.94	0.01\\
76.95	0.01\\
76.96	0.01\\
76.97	0.01\\
76.98	0.01\\
76.99	0.01\\
77	0.01\\
77.01	0.01\\
77.02	0.01\\
77.03	0.01\\
77.04	0.01\\
77.05	0.01\\
77.06	0.01\\
77.07	0.01\\
77.08	0.01\\
77.09	0.01\\
77.1	0.01\\
77.11	0.01\\
77.12	0.01\\
77.13	0.01\\
77.14	0.01\\
77.15	0.01\\
77.16	0.01\\
77.17	0.01\\
77.18	0.01\\
77.19	0.01\\
77.2	0.01\\
77.21	0.01\\
77.22	0.01\\
77.23	0.01\\
77.24	0.01\\
77.25	0.01\\
77.26	0.01\\
77.27	0.01\\
77.28	0.01\\
77.29	0.01\\
77.3	0.01\\
77.31	0.01\\
77.32	0.01\\
77.33	0.01\\
77.34	0.01\\
77.35	0.01\\
77.36	0.01\\
77.37	0.01\\
77.38	0.01\\
77.39	0.01\\
77.4	0.01\\
77.41	0.01\\
77.42	0.01\\
77.43	0.01\\
77.44	0.01\\
77.45	0.01\\
77.46	0.01\\
77.47	0.01\\
77.48	0.01\\
77.49	0.01\\
77.5	0.01\\
77.51	0.01\\
77.52	0.01\\
77.53	0.01\\
77.54	0.01\\
77.55	0.01\\
77.56	0.01\\
77.57	0.01\\
77.58	0.01\\
77.59	0.01\\
77.6	0.01\\
77.61	0.01\\
77.62	0.01\\
77.63	0.01\\
77.64	0.01\\
77.65	0.01\\
77.66	0.01\\
77.67	0.01\\
77.68	0.01\\
77.69	0.01\\
77.7	0.01\\
77.71	0.01\\
77.72	0.01\\
77.73	0.01\\
77.74	0.01\\
77.75	0.01\\
77.76	0.01\\
77.77	0.01\\
77.78	0.01\\
77.79	0.01\\
77.8	0.01\\
77.81	0.01\\
77.82	0.01\\
77.83	0.01\\
77.84	0.01\\
77.85	0.01\\
77.86	0.01\\
77.87	0.01\\
77.88	0.01\\
77.89	0.01\\
77.9	0.01\\
77.91	0.01\\
77.92	0.01\\
77.93	0.01\\
77.94	0.01\\
77.95	0.01\\
77.96	0.01\\
77.97	0.01\\
77.98	0.01\\
77.99	0.01\\
78	0.01\\
78.01	0.01\\
78.02	0.01\\
78.03	0.01\\
78.04	0.01\\
78.05	0.01\\
78.06	0.01\\
78.07	0.01\\
78.08	0.01\\
78.09	0.01\\
78.1	0.01\\
78.11	0.01\\
78.12	0.01\\
78.13	0.01\\
78.14	0.01\\
78.15	0.01\\
78.16	0.01\\
78.17	0.01\\
78.18	0.01\\
78.19	0.01\\
78.2	0.01\\
78.21	0.01\\
78.22	0.01\\
78.23	0.01\\
78.24	0.01\\
78.25	0.01\\
78.26	0.01\\
78.27	0.01\\
78.28	0.01\\
78.29	0.01\\
78.3	0.01\\
78.31	0.01\\
78.32	0.01\\
78.33	0.01\\
78.34	0.01\\
78.35	0.01\\
78.36	0.01\\
78.37	0.01\\
78.38	0.01\\
78.39	0.01\\
78.4	0.01\\
78.41	0.01\\
78.42	0.01\\
78.43	0.01\\
78.44	0.01\\
78.45	0.01\\
78.46	0.01\\
78.47	0.01\\
78.48	0.01\\
78.49	0.01\\
78.5	0.01\\
78.51	0.01\\
78.52	0.01\\
78.53	0.01\\
78.54	0.01\\
78.55	0.01\\
78.56	0.01\\
78.57	0.01\\
78.58	0.01\\
78.59	0.01\\
78.6	0.01\\
78.61	0.01\\
78.62	0.01\\
78.63	0.01\\
78.64	0.01\\
78.65	0.01\\
78.66	0.01\\
78.67	0.01\\
78.68	0.01\\
78.69	0.01\\
78.7	0.01\\
78.71	0.01\\
78.72	0.01\\
78.73	0.01\\
78.74	0.01\\
78.75	0.01\\
78.76	0.01\\
78.77	0.01\\
78.78	0.01\\
78.79	0.01\\
78.8	0.01\\
78.81	0.01\\
78.82	0.01\\
78.83	0.01\\
78.84	0.01\\
78.85	0.01\\
78.86	0.01\\
78.87	0.01\\
78.88	0.01\\
78.89	0.01\\
78.9	0.01\\
78.91	0.01\\
78.92	0.01\\
78.93	0.01\\
78.94	0.01\\
78.95	0.01\\
78.96	0.01\\
78.97	0.01\\
78.98	0.01\\
78.99	0.01\\
79	0.01\\
79.01	0.01\\
79.02	0.01\\
79.03	0.01\\
79.04	0.01\\
79.05	0.01\\
79.06	0.01\\
79.07	0.01\\
79.08	0.01\\
79.09	0.01\\
79.1	0.01\\
79.11	0.01\\
79.12	0.01\\
79.13	0.01\\
79.14	0.01\\
79.15	0.01\\
79.16	0.01\\
79.17	0.01\\
79.18	0.01\\
79.19	0.01\\
79.2	0.01\\
79.21	0.01\\
79.22	0.01\\
79.23	0.01\\
79.24	0.01\\
79.25	0.01\\
79.26	0.01\\
79.27	0.01\\
79.28	0.01\\
79.29	0.01\\
79.3	0.01\\
79.31	0.01\\
79.32	0.01\\
79.33	0.01\\
79.34	0.01\\
79.35	0.01\\
79.36	0.01\\
79.37	0.01\\
79.38	0.01\\
79.39	0.01\\
79.4	0.01\\
79.41	0.01\\
79.42	0.01\\
79.43	0.01\\
79.44	0.01\\
79.45	0.01\\
79.46	0.01\\
79.47	0.01\\
79.48	0.01\\
79.49	0.01\\
79.5	0.01\\
79.51	0.01\\
79.52	0.01\\
79.53	0.01\\
79.54	0.01\\
79.55	0.01\\
79.56	0.01\\
79.57	0.01\\
79.58	0.01\\
79.59	0.01\\
79.6	0.01\\
79.61	0.01\\
79.62	0.01\\
79.63	0.01\\
79.64	0.01\\
79.65	0.01\\
79.66	0.01\\
79.67	0.01\\
79.68	0.01\\
79.69	0.01\\
79.7	0.01\\
79.71	0.01\\
79.72	0.01\\
79.73	0.01\\
79.74	0.01\\
79.75	0.01\\
79.76	0.01\\
79.77	0.01\\
79.78	0.01\\
79.79	0.01\\
79.8	0.01\\
79.81	0.01\\
79.82	0.01\\
79.83	0.01\\
79.84	0.01\\
79.85	0.01\\
79.86	0.01\\
79.87	0.01\\
79.88	0.01\\
79.89	0.01\\
79.9	0.01\\
79.91	0.01\\
79.92	0.01\\
79.93	0.01\\
79.94	0.01\\
79.95	0.01\\
79.96	0.01\\
79.97	0.01\\
79.98	0.01\\
79.99	0.01\\
80	0.01\\
80.01	0.01\\
};
\addplot [color=mycolor1,dashed]
  table[row sep=crcr]{%
80.01	0.01\\
80.02	0.01\\
80.03	0.01\\
80.04	0.01\\
80.05	0.01\\
80.06	0.01\\
80.07	0.01\\
80.08	0.01\\
80.09	0.01\\
80.1	0.01\\
80.11	0.01\\
80.12	0.01\\
80.13	0.01\\
80.14	0.01\\
80.15	0.01\\
80.16	0.01\\
80.17	0.01\\
80.18	0.01\\
80.19	0.01\\
80.2	0.01\\
80.21	0.01\\
80.22	0.01\\
80.23	0.01\\
80.24	0.01\\
80.25	0.01\\
80.26	0.01\\
80.27	0.01\\
80.28	0.01\\
80.29	0.01\\
80.3	0.01\\
80.31	0.01\\
80.32	0.01\\
80.33	0.01\\
80.34	0.01\\
80.35	0.01\\
80.36	0.01\\
80.37	0.01\\
80.38	0.01\\
80.39	0.01\\
80.4	0.01\\
80.41	0.01\\
80.42	0.01\\
80.43	0.01\\
80.44	0.01\\
80.45	0.01\\
80.46	0.01\\
80.47	0.01\\
80.48	0.01\\
80.49	0.01\\
80.5	0.01\\
80.51	0.01\\
80.52	0.01\\
80.53	0.01\\
80.54	0.01\\
80.55	0.01\\
80.56	0.01\\
80.57	0.01\\
80.58	0.01\\
80.59	0.01\\
80.6	0.01\\
80.61	0.01\\
80.62	0.01\\
80.63	0.01\\
80.64	0.01\\
80.65	0.01\\
80.66	0.01\\
80.67	0.01\\
80.68	0.01\\
80.69	0.01\\
80.7	0.01\\
80.71	0.01\\
80.72	0.01\\
80.73	0.01\\
80.74	0.01\\
80.75	0.01\\
80.76	0.01\\
80.77	0.01\\
80.78	0.01\\
80.79	0.01\\
80.8	0.01\\
80.81	0.01\\
80.82	0.01\\
80.83	0.01\\
80.84	0.01\\
80.85	0.01\\
80.86	0.01\\
80.87	0.01\\
80.88	0.01\\
80.89	0.01\\
80.9	0.01\\
80.91	0.01\\
80.92	0.01\\
80.93	0.01\\
80.94	0.01\\
80.95	0.01\\
80.96	0.01\\
80.97	0.01\\
80.98	0.01\\
80.99	0.01\\
81	0.01\\
81.01	0.01\\
81.02	0.01\\
81.03	0.01\\
81.04	0.01\\
81.05	0.01\\
81.06	0.01\\
81.07	0.01\\
81.08	0.01\\
81.09	0.01\\
81.1	0.01\\
81.11	0.01\\
81.12	0.01\\
81.13	0.01\\
81.14	0.01\\
81.15	0.01\\
81.16	0.01\\
81.17	0.01\\
81.18	0.01\\
81.19	0.01\\
81.2	0.01\\
81.21	0.01\\
81.22	0.01\\
81.23	0.01\\
81.24	0.01\\
81.25	0.01\\
81.26	0.01\\
81.27	0.01\\
81.28	0.01\\
81.29	0.01\\
81.3	0.01\\
81.31	0.01\\
81.32	0.01\\
81.33	0.01\\
81.34	0.01\\
81.35	0.01\\
81.36	0.01\\
81.37	0.01\\
81.38	0.01\\
81.39	0.01\\
81.4	0.01\\
81.41	0.01\\
81.42	0.01\\
81.43	0.01\\
81.44	0.01\\
81.45	0.01\\
81.46	0.01\\
81.47	0.01\\
81.48	0.01\\
81.49	0.01\\
81.5	0.01\\
81.51	0.01\\
81.52	0.01\\
81.53	0.01\\
81.54	0.01\\
81.55	0.01\\
81.56	0.01\\
81.57	0.01\\
81.58	0.01\\
81.59	0.01\\
81.6	0.01\\
81.61	0.01\\
81.62	0.01\\
81.63	0.01\\
81.64	0.01\\
81.65	0.01\\
81.66	0.01\\
81.67	0.01\\
81.68	0.01\\
81.69	0.01\\
81.7	0.01\\
81.71	0.01\\
81.72	0.01\\
81.73	0.01\\
81.74	0.01\\
81.75	0.01\\
81.76	0.01\\
81.77	0.01\\
81.78	0.01\\
81.79	0.01\\
81.8	0.01\\
81.81	0.01\\
81.82	0.01\\
81.83	0.01\\
81.84	0.01\\
81.85	0.01\\
81.86	0.01\\
81.87	0.01\\
81.88	0.01\\
81.89	0.01\\
81.9	0.01\\
81.91	0.01\\
81.92	0.01\\
81.93	0.01\\
81.94	0.01\\
81.95	0.01\\
81.96	0.01\\
81.97	0.01\\
81.98	0.01\\
81.99	0.01\\
82	0.01\\
82.01	0.01\\
82.02	0.01\\
82.03	0.01\\
82.04	0.01\\
82.05	0.01\\
82.06	0.01\\
82.07	0.01\\
82.08	0.01\\
82.09	0.01\\
82.1	0.01\\
82.11	0.01\\
82.12	0.01\\
82.13	0.01\\
82.14	0.01\\
82.15	0.01\\
82.16	0.01\\
82.17	0.01\\
82.18	0.01\\
82.19	0.01\\
82.2	0.01\\
82.21	0.01\\
82.22	0.01\\
82.23	0.01\\
82.24	0.01\\
82.25	0.01\\
82.26	0.01\\
82.27	0.01\\
82.28	0.01\\
82.29	0.01\\
82.3	0.01\\
82.31	0.01\\
82.32	0.01\\
82.33	0.01\\
82.34	0.01\\
82.35	0.01\\
82.36	0.01\\
82.37	0.01\\
82.38	0.01\\
82.39	0.01\\
82.4	0.01\\
82.41	0.01\\
82.42	0.01\\
82.43	0.01\\
82.44	0.01\\
82.45	0.01\\
82.46	0.01\\
82.47	0.01\\
82.48	0.01\\
82.49	0.01\\
82.5	0.01\\
82.51	0.01\\
82.52	0.01\\
82.53	0.01\\
82.54	0.01\\
82.55	0.01\\
82.56	0.01\\
82.57	0.01\\
82.58	0.01\\
82.59	0.01\\
82.6	0.01\\
82.61	0.01\\
82.62	0.01\\
82.63	0.01\\
82.64	0.01\\
82.65	0.01\\
82.66	0.01\\
82.67	0.01\\
82.68	0.01\\
82.69	0.01\\
82.7	0.01\\
82.71	0.01\\
82.72	0.01\\
82.73	0.01\\
82.74	0.01\\
82.75	0.01\\
82.76	0.01\\
82.77	0.01\\
82.78	0.01\\
82.79	0.01\\
82.8	0.01\\
82.81	0.01\\
82.82	0.01\\
82.83	0.01\\
82.84	0.01\\
82.85	0.01\\
82.86	0.01\\
82.87	0.01\\
82.88	0.01\\
82.89	0.01\\
82.9	0.01\\
82.91	0.01\\
82.92	0.01\\
82.93	0.01\\
82.94	0.01\\
82.95	0.01\\
82.96	0.01\\
82.97	0.01\\
82.98	0.01\\
82.99	0.01\\
83	0.01\\
83.01	0.01\\
83.02	0.01\\
83.03	0.01\\
83.04	0.01\\
83.05	0.01\\
83.06	0.01\\
83.07	0.01\\
83.08	0.01\\
83.09	0.01\\
83.1	0.01\\
83.11	0.01\\
83.12	0.01\\
83.13	0.01\\
83.14	0.01\\
83.15	0.01\\
83.16	0.01\\
83.17	0.01\\
83.18	0.01\\
83.19	0.01\\
83.2	0.01\\
83.21	0.01\\
83.22	0.01\\
83.23	0.01\\
83.24	0.01\\
83.25	0.01\\
83.26	0.01\\
83.27	0.01\\
83.28	0.01\\
83.29	0.01\\
83.3	0.01\\
83.31	0.01\\
83.32	0.01\\
83.33	0.01\\
83.34	0.01\\
83.35	0.01\\
83.36	0.01\\
83.37	0.01\\
83.38	0.01\\
83.39	0.01\\
83.4	0.01\\
83.41	0.01\\
83.42	0.01\\
83.43	0.01\\
83.44	0.01\\
83.45	0.01\\
83.46	0.01\\
83.47	0.01\\
83.48	0.01\\
83.49	0.01\\
83.5	0.01\\
83.51	0.01\\
83.52	0.01\\
83.53	0.01\\
83.54	0.01\\
83.55	0.01\\
83.56	0.01\\
83.57	0.01\\
83.58	0.01\\
83.59	0.01\\
83.6	0.01\\
83.61	0.01\\
83.62	0.01\\
83.63	0.01\\
83.64	0.01\\
83.65	0.01\\
83.66	0.01\\
83.67	0.01\\
83.68	0.01\\
83.69	0.01\\
83.7	0.01\\
83.71	0.01\\
83.72	0.01\\
83.73	0.01\\
83.74	0.01\\
83.75	0.01\\
83.76	0.01\\
83.77	0.01\\
83.78	0.01\\
83.79	0.01\\
83.8	0.01\\
83.81	0.01\\
83.82	0.01\\
83.83	0.01\\
83.84	0.01\\
83.85	0.01\\
83.86	0.01\\
83.87	0.01\\
83.88	0.01\\
83.89	0.01\\
83.9	0.01\\
83.91	0.01\\
83.92	0.01\\
83.93	0.01\\
83.94	0.01\\
83.95	0.01\\
83.96	0.01\\
83.97	0.01\\
83.98	0.01\\
83.99	0.01\\
84	0.01\\
84.01	0.01\\
84.02	0.01\\
84.03	0.01\\
84.04	0.01\\
84.05	0.01\\
84.06	0.01\\
84.07	0.01\\
84.08	0.01\\
84.09	0.01\\
84.1	0.01\\
84.11	0.01\\
84.12	0.01\\
84.13	0.01\\
84.14	0.01\\
84.15	0.01\\
84.16	0.01\\
84.17	0.01\\
84.18	0.01\\
84.19	0.01\\
84.2	0.01\\
84.21	0.01\\
84.22	0.01\\
84.23	0.01\\
84.24	0.01\\
84.25	0.01\\
84.26	0.01\\
84.27	0.01\\
84.28	0.01\\
84.29	0.01\\
84.3	0.01\\
84.31	0.01\\
84.32	0.01\\
84.33	0.01\\
84.34	0.01\\
84.35	0.01\\
84.36	0.01\\
84.37	0.01\\
84.38	0.01\\
84.39	0.01\\
84.4	0.01\\
84.41	0.01\\
84.42	0.01\\
84.43	0.01\\
84.44	0.01\\
84.45	0.01\\
84.46	0.01\\
84.47	0.01\\
84.48	0.01\\
84.49	0.01\\
84.5	0.01\\
84.51	0.01\\
84.52	0.01\\
84.53	0.01\\
84.54	0.01\\
84.55	0.01\\
84.56	0.01\\
84.57	0.01\\
84.58	0.01\\
84.59	0.01\\
84.6	0.01\\
84.61	0.01\\
84.62	0.01\\
84.63	0.01\\
84.64	0.01\\
84.65	0.01\\
84.66	0.01\\
84.67	0.01\\
84.68	0.01\\
84.69	0.01\\
84.7	0.01\\
84.71	0.01\\
84.72	0.01\\
84.73	0.01\\
84.74	0.01\\
84.75	0.01\\
84.76	0.01\\
84.77	0.01\\
84.78	0.01\\
84.79	0.01\\
84.8	0.01\\
84.81	0.01\\
84.82	0.01\\
84.83	0.01\\
84.84	0.01\\
84.85	0.01\\
84.86	0.01\\
84.87	0.01\\
84.88	0.01\\
84.89	0.01\\
84.9	0.01\\
84.91	0.01\\
84.92	0.01\\
84.93	0.01\\
84.94	0.01\\
84.95	0.01\\
84.96	0.01\\
84.97	0.01\\
84.98	0.01\\
84.99	0.01\\
85	0.01\\
85.01	0.01\\
85.02	0.01\\
85.03	0.01\\
85.04	0.01\\
85.05	0.01\\
85.06	0.01\\
85.07	0.01\\
85.08	0.01\\
85.09	0.01\\
85.1	0.01\\
85.11	0.01\\
85.12	0.01\\
85.13	0.01\\
85.14	0.01\\
85.15	0.01\\
85.16	0.01\\
85.17	0.01\\
85.18	0.01\\
85.19	0.01\\
85.2	0.01\\
85.21	0.01\\
85.22	0.01\\
85.23	0.01\\
85.24	0.01\\
85.25	0.01\\
85.26	0.01\\
85.27	0.01\\
85.28	0.01\\
85.29	0.01\\
85.3	0.01\\
85.31	0.01\\
85.32	0.01\\
85.33	0.01\\
85.34	0.01\\
85.35	0.01\\
85.36	0.01\\
85.37	0.01\\
85.38	0.01\\
85.39	0.01\\
85.4	0.01\\
85.41	0.01\\
85.42	0.01\\
85.43	0.01\\
85.44	0.01\\
85.45	0.01\\
85.46	0.01\\
85.47	0.01\\
85.48	0.01\\
85.49	0.01\\
85.5	0.01\\
85.51	0.01\\
85.52	0.01\\
85.53	0.01\\
85.54	0.01\\
85.55	0.01\\
85.56	0.01\\
85.57	0.01\\
85.58	0.01\\
85.59	0.01\\
85.6	0.01\\
85.61	0.01\\
85.62	0.01\\
85.63	0.01\\
85.64	0.01\\
85.65	0.01\\
85.66	0.01\\
85.67	0.01\\
85.68	0.01\\
85.69	0.01\\
85.7	0.01\\
85.71	0.01\\
85.72	0.01\\
85.73	0.01\\
85.74	0.01\\
85.75	0.01\\
85.76	0.01\\
85.77	0.01\\
85.78	0.01\\
85.79	0.01\\
85.8	0.01\\
85.81	0.01\\
85.82	0.01\\
85.83	0.01\\
85.84	0.01\\
85.85	0.01\\
85.86	0.01\\
85.87	0.01\\
85.88	0.01\\
85.89	0.01\\
85.9	0.01\\
85.91	0.01\\
85.92	0.01\\
85.93	0.01\\
85.94	0.01\\
85.95	0.01\\
85.96	0.01\\
85.97	0.01\\
85.98	0.01\\
85.99	0.01\\
86	0.01\\
86.01	0.01\\
86.02	0.01\\
86.03	0.01\\
86.04	0.01\\
86.05	0.01\\
86.06	0.01\\
86.07	0.01\\
86.08	0.01\\
86.09	0.01\\
86.1	0.01\\
86.11	0.01\\
86.12	0.01\\
86.13	0.01\\
86.14	0.01\\
86.15	0.01\\
86.16	0.01\\
86.17	0.01\\
86.18	0.01\\
86.19	0.01\\
86.2	0.01\\
86.21	0.01\\
86.22	0.01\\
86.23	0.01\\
86.24	0.01\\
86.25	0.01\\
86.26	0.01\\
86.27	0.01\\
86.28	0.01\\
86.29	0.01\\
86.3	0.01\\
86.31	0.01\\
86.32	0.01\\
86.33	0.01\\
86.34	0.01\\
86.35	0.01\\
86.36	0.01\\
86.37	0.01\\
86.38	0.01\\
86.39	0.01\\
86.4	0.01\\
86.41	0.01\\
86.42	0.01\\
86.43	0.01\\
86.44	0.01\\
86.45	0.01\\
86.46	0.01\\
86.47	0.01\\
86.48	0.01\\
86.49	0.01\\
86.5	0.01\\
86.51	0.01\\
86.52	0.01\\
86.53	0.01\\
86.54	0.01\\
86.55	0.01\\
86.56	0.01\\
86.57	0.01\\
86.58	0.01\\
86.59	0.01\\
86.6	0.01\\
86.61	0.01\\
86.62	0.01\\
86.63	0.01\\
86.64	0.01\\
86.65	0.01\\
86.66	0.01\\
86.67	0.01\\
86.68	0.01\\
86.69	0.01\\
86.7	0.01\\
86.71	0.01\\
86.72	0.01\\
86.73	0.01\\
86.74	0.01\\
86.75	0.01\\
86.76	0.01\\
86.77	0.01\\
86.78	0.01\\
86.79	0.01\\
86.8	0.01\\
86.81	0.01\\
86.82	0.01\\
86.83	0.01\\
86.84	0.01\\
86.85	0.01\\
86.86	0.01\\
86.87	0.01\\
86.88	0.01\\
86.89	0.01\\
86.9	0.01\\
86.91	0.01\\
86.92	0.01\\
86.93	0.01\\
86.94	0.01\\
86.95	0.01\\
86.96	0.01\\
86.97	0.01\\
86.98	0.01\\
86.99	0.01\\
87	0.01\\
87.01	0.01\\
87.02	0.01\\
87.03	0.01\\
87.04	0.01\\
87.05	0.01\\
87.06	0.01\\
87.07	0.01\\
87.08	0.01\\
87.09	0.01\\
87.1	0.01\\
87.11	0.01\\
87.12	0.01\\
87.13	0.01\\
87.14	0.01\\
87.15	0.01\\
87.16	0.01\\
87.17	0.01\\
87.18	0.01\\
87.19	0.01\\
87.2	0.01\\
87.21	0.01\\
87.22	0.01\\
87.23	0.01\\
87.24	0.01\\
87.25	0.01\\
87.26	0.01\\
87.27	0.01\\
87.28	0.01\\
87.29	0.01\\
87.3	0.01\\
87.31	0.01\\
87.32	0.01\\
87.33	0.01\\
87.34	0.01\\
87.35	0.01\\
87.36	0.01\\
87.37	0.01\\
87.38	0.01\\
87.39	0.01\\
87.4	0.01\\
87.41	0.01\\
87.42	0.01\\
87.43	0.01\\
87.44	0.01\\
87.45	0.01\\
87.46	0.01\\
87.47	0.01\\
87.48	0.01\\
87.49	0.01\\
87.5	0.01\\
87.51	0.01\\
87.52	0.01\\
87.53	0.01\\
87.54	0.01\\
87.55	0.01\\
87.56	0.01\\
87.57	0.01\\
87.58	0.01\\
87.59	0.01\\
87.6	0.01\\
87.61	0.01\\
87.62	0.01\\
87.63	0.01\\
87.64	0.01\\
87.65	0.01\\
87.66	0.01\\
87.67	0.01\\
87.68	0.01\\
87.69	0.01\\
87.7	0.01\\
87.71	0.01\\
87.72	0.01\\
87.73	0.01\\
87.74	0.01\\
87.75	0.01\\
87.76	0.01\\
87.77	0.01\\
87.78	0.01\\
87.79	0.01\\
87.8	0.01\\
87.81	0.01\\
87.82	0.01\\
87.83	0.01\\
87.84	0.01\\
87.85	0.01\\
87.86	0.01\\
87.87	0.01\\
87.88	0.01\\
87.89	0.01\\
87.9	0.01\\
87.91	0.01\\
87.92	0.01\\
87.93	0.01\\
87.94	0.01\\
87.95	0.01\\
87.96	0.01\\
87.97	0.01\\
87.98	0.01\\
87.99	0.01\\
88	0.01\\
88.01	0.01\\
88.02	0.01\\
88.03	0.01\\
88.04	0.01\\
88.05	0.01\\
88.06	0.01\\
88.07	0.01\\
88.08	0.01\\
88.09	0.01\\
88.1	0.01\\
88.11	0.01\\
88.12	0.01\\
88.13	0.01\\
88.14	0.01\\
88.15	0.01\\
88.16	0.01\\
88.17	0.01\\
88.18	0.01\\
88.19	0.01\\
88.2	0.01\\
88.21	0.01\\
88.22	0.01\\
88.23	0.01\\
88.24	0.01\\
88.25	0.01\\
88.26	0.01\\
88.27	0.01\\
88.28	0.01\\
88.29	0.01\\
88.3	0.01\\
88.31	0.01\\
88.32	0.01\\
88.33	0.01\\
88.34	0.01\\
88.35	0.01\\
88.36	0.01\\
88.37	0.01\\
88.38	0.01\\
88.39	0.01\\
88.4	0.01\\
88.41	0.01\\
88.42	0.01\\
88.43	0.01\\
88.44	0.01\\
88.45	0.01\\
88.46	0.01\\
88.47	0.01\\
88.48	0.01\\
88.49	0.01\\
88.5	0.01\\
88.51	0.01\\
88.52	0.01\\
88.53	0.01\\
88.54	0.01\\
88.55	0.01\\
88.56	0.01\\
88.57	0.01\\
88.58	0.01\\
88.59	0.01\\
88.6	0.01\\
88.61	0.01\\
88.62	0.01\\
88.63	0.01\\
88.64	0.01\\
88.65	0.01\\
88.66	0.01\\
88.67	0.01\\
88.68	0.01\\
88.69	0.01\\
88.7	0.01\\
88.71	0.01\\
88.72	0.01\\
88.73	0.01\\
88.74	0.01\\
88.75	0.01\\
88.76	0.01\\
88.77	0.01\\
88.78	0.01\\
88.79	0.01\\
88.8	0.01\\
88.81	0.01\\
88.82	0.01\\
88.83	0.01\\
88.84	0.01\\
88.85	0.01\\
88.86	0.01\\
88.87	0.01\\
88.88	0.01\\
88.89	0.01\\
88.9	0.01\\
88.91	0.01\\
88.92	0.01\\
88.93	0.01\\
88.94	0.01\\
88.95	0.01\\
88.96	0.01\\
88.97	0.01\\
88.98	0.01\\
88.99	0.01\\
89	0.01\\
89.01	0.01\\
89.02	0.01\\
89.03	0.01\\
89.04	0.01\\
89.05	0.01\\
89.06	0.01\\
89.07	0.01\\
89.08	0.01\\
89.09	0.01\\
89.1	0.01\\
89.11	0.01\\
89.12	0.01\\
89.13	0.01\\
89.14	0.01\\
89.15	0.01\\
89.16	0.01\\
89.17	0.01\\
89.18	0.01\\
89.19	0.01\\
89.2	0.01\\
89.21	0.01\\
89.22	0.01\\
89.23	0.01\\
89.24	0.01\\
89.25	0.01\\
89.26	0.01\\
89.27	0.01\\
89.28	0.01\\
89.29	0.01\\
89.3	0.01\\
89.31	0.01\\
89.32	0.01\\
89.33	0.01\\
89.34	0.01\\
89.35	0.01\\
89.36	0.01\\
89.37	0.01\\
89.38	0.01\\
89.39	0.01\\
89.4	0.01\\
89.41	0.01\\
89.42	0.01\\
89.43	0.01\\
89.44	0.01\\
89.45	0.01\\
89.46	0.01\\
89.47	0.01\\
89.48	0.01\\
89.49	0.01\\
89.5	0.01\\
89.51	0.01\\
89.52	0.01\\
89.53	0.01\\
89.54	0.01\\
89.55	0.01\\
89.56	0.01\\
89.57	0.01\\
89.58	0.01\\
89.59	0.01\\
89.6	0.01\\
89.61	0.01\\
89.62	0.01\\
89.63	0.01\\
89.64	0.01\\
89.65	0.01\\
89.66	0.01\\
89.67	0.01\\
89.68	0.01\\
89.69	0.01\\
89.7	0.01\\
89.71	0.01\\
89.72	0.01\\
89.73	0.01\\
89.74	0.01\\
89.75	0.01\\
89.76	0.01\\
89.77	0.01\\
89.78	0.01\\
89.79	0.01\\
89.8	0.01\\
89.81	0.01\\
89.82	0.01\\
89.83	0.01\\
89.84	0.01\\
89.85	0.01\\
89.86	0.01\\
89.87	0.01\\
89.88	0.01\\
89.89	0.01\\
89.9	0.01\\
89.91	0.01\\
89.92	0.01\\
89.93	0.01\\
89.94	0.01\\
89.95	0.01\\
89.96	0.01\\
89.97	0.01\\
89.98	0.01\\
89.99	0.01\\
90	0.01\\
90.01	0.01\\
90.02	0.01\\
90.03	0.01\\
90.04	0.01\\
90.05	0.01\\
90.06	0.01\\
90.07	0.01\\
90.08	0.01\\
90.09	0.01\\
90.1	0.01\\
90.11	0.01\\
90.12	0.01\\
90.13	0.01\\
90.14	0.01\\
90.15	0.01\\
90.16	0.01\\
90.17	0.01\\
90.18	0.01\\
90.19	0.01\\
90.2	0.01\\
90.21	0.01\\
90.22	0.01\\
90.23	0.01\\
90.24	0.01\\
90.25	0.01\\
90.26	0.01\\
90.27	0.01\\
90.28	0.01\\
90.29	0.01\\
90.3	0.01\\
90.31	0.01\\
90.32	0.01\\
90.33	0.01\\
90.34	0.01\\
90.35	0.01\\
90.36	0.01\\
90.37	0.01\\
90.38	0.01\\
90.39	0.01\\
90.4	0.01\\
90.41	0.01\\
90.42	0.01\\
90.43	0.01\\
90.44	0.01\\
90.45	0.01\\
90.46	0.01\\
90.47	0.01\\
90.48	0.01\\
90.49	0.01\\
90.5	0.01\\
90.51	0.01\\
90.52	0.01\\
90.53	0.01\\
90.54	0.01\\
90.55	0.01\\
90.56	0.01\\
90.57	0.01\\
90.58	0.01\\
90.59	0.01\\
90.6	0.01\\
90.61	0.01\\
90.62	0.01\\
90.63	0.01\\
90.64	0.01\\
90.65	0.01\\
90.66	0.01\\
90.67	0.01\\
90.68	0.01\\
90.69	0.01\\
90.7	0.01\\
90.71	0.01\\
90.72	0.01\\
90.73	0.01\\
90.74	0.01\\
90.75	0.01\\
90.76	0.01\\
90.77	0.01\\
90.78	0.01\\
90.79	0.01\\
90.8	0.01\\
90.81	0.01\\
90.82	0.01\\
90.83	0.01\\
90.84	0.01\\
90.85	0.01\\
90.86	0.01\\
90.87	0.01\\
90.88	0.01\\
90.89	0.01\\
90.9	0.01\\
90.91	0.01\\
90.92	0.01\\
90.93	0.01\\
90.94	0.01\\
90.95	0.01\\
90.96	0.01\\
90.97	0.01\\
90.98	0.01\\
90.99	0.01\\
91	0.01\\
91.01	0.01\\
91.02	0.01\\
91.03	0.01\\
91.04	0.01\\
91.05	0.01\\
91.06	0.01\\
91.07	0.01\\
91.08	0.01\\
91.09	0.01\\
91.1	0.01\\
91.11	0.01\\
91.12	0.01\\
91.13	0.01\\
91.14	0.01\\
91.15	0.01\\
91.16	0.01\\
91.17	0.01\\
91.18	0.01\\
91.19	0.01\\
91.2	0.01\\
91.21	0.01\\
91.22	0.01\\
91.23	0.01\\
91.24	0.01\\
91.25	0.01\\
91.26	0.01\\
91.27	0.01\\
91.28	0.01\\
91.29	0.01\\
91.3	0.01\\
91.31	0.01\\
91.32	0.01\\
91.33	0.01\\
91.34	0.01\\
91.35	0.01\\
91.36	0.01\\
91.37	0.01\\
91.38	0.01\\
91.39	0.01\\
91.4	0.01\\
91.41	0.01\\
91.42	0.01\\
91.43	0.01\\
91.44	0.01\\
91.45	0.01\\
91.46	0.01\\
91.47	0.01\\
91.48	0.01\\
91.49	0.01\\
91.5	0.01\\
91.51	0.01\\
91.52	0.01\\
91.53	0.01\\
91.54	0.01\\
91.55	0.01\\
91.56	0.01\\
91.57	0.01\\
91.58	0.01\\
91.59	0.01\\
91.6	0.01\\
91.61	0.01\\
91.62	0.01\\
91.63	0.01\\
91.64	0.01\\
91.65	0.01\\
91.66	0.01\\
91.67	0.01\\
91.68	0.01\\
91.69	0.01\\
91.7	0.01\\
91.71	0.01\\
91.72	0.01\\
91.73	0.01\\
91.74	0.01\\
91.75	0.01\\
91.76	0.01\\
91.77	0.01\\
91.78	0.01\\
91.79	0.01\\
91.8	0.01\\
91.81	0.01\\
91.82	0.01\\
91.83	0.01\\
91.84	0.01\\
91.85	0.01\\
91.86	0.01\\
91.87	0.01\\
91.88	0.01\\
91.89	0.01\\
91.9	0.01\\
91.91	0.01\\
91.92	0.01\\
91.93	0.01\\
91.94	0.01\\
91.95	0.01\\
91.96	0.01\\
91.97	0.01\\
91.98	0.01\\
91.99	0.01\\
92	0.01\\
92.01	0.01\\
92.02	0.01\\
92.03	0.01\\
92.04	0.01\\
92.05	0.01\\
92.06	0.01\\
92.07	0.01\\
92.08	0.01\\
92.09	0.01\\
92.1	0.01\\
92.11	0.01\\
92.12	0.01\\
92.13	0.01\\
92.14	0.01\\
92.15	0.01\\
92.16	0.01\\
92.17	0.01\\
92.18	0.01\\
92.19	0.01\\
92.2	0.01\\
92.21	0.01\\
92.22	0.01\\
92.23	0.01\\
92.24	0.01\\
92.25	0.01\\
92.26	0.01\\
92.27	0.01\\
92.28	0.01\\
92.29	0.01\\
92.3	0.01\\
92.31	0.01\\
92.32	0.01\\
92.33	0.01\\
92.34	0.01\\
92.35	0.01\\
92.36	0.01\\
92.37	0.01\\
92.38	0.01\\
92.39	0.01\\
92.4	0.01\\
92.41	0.01\\
92.42	0.01\\
92.43	0.01\\
92.44	0.01\\
92.45	0.01\\
92.46	0.01\\
92.47	0.01\\
92.48	0.01\\
92.49	0.01\\
92.5	0.01\\
92.51	0.01\\
92.52	0.01\\
92.53	0.01\\
92.54	0.01\\
92.55	0.01\\
92.56	0.01\\
92.57	0.01\\
92.58	0.01\\
92.59	0.01\\
92.6	0.01\\
92.61	0.01\\
92.62	0.01\\
92.63	0.01\\
92.64	0.01\\
92.65	0.01\\
92.66	0.01\\
92.67	0.01\\
92.68	0.01\\
92.69	0.01\\
92.7	0.01\\
92.71	0.01\\
92.72	0.01\\
92.73	0.01\\
92.74	0.01\\
92.75	0.01\\
92.76	0.01\\
92.77	0.01\\
92.78	0.01\\
92.79	0.01\\
92.8	0.01\\
92.81	0.01\\
92.82	0.01\\
92.83	0.01\\
92.84	0.01\\
92.85	0.01\\
92.86	0.01\\
92.87	0.01\\
92.88	0.01\\
92.89	0.01\\
92.9	0.01\\
92.91	0.01\\
92.92	0.01\\
92.93	0.01\\
92.94	0.01\\
92.95	0.01\\
92.96	0.01\\
92.97	0.01\\
92.98	0.01\\
92.99	0.01\\
93	0.01\\
93.01	0.01\\
93.02	0.01\\
93.03	0.01\\
93.04	0.01\\
93.05	0.01\\
93.06	0.01\\
93.07	0.01\\
93.08	0.01\\
93.09	0.01\\
93.1	0.01\\
93.11	0.01\\
93.12	0.01\\
93.13	0.01\\
93.14	0.01\\
93.15	0.01\\
93.16	0.01\\
93.17	0.01\\
93.18	0.01\\
93.19	0.01\\
93.2	0.01\\
93.21	0.01\\
93.22	0.01\\
93.23	0.01\\
93.24	0.01\\
93.25	0.01\\
93.26	0.01\\
93.27	0.01\\
93.28	0.01\\
93.29	0.01\\
93.3	0.01\\
93.31	0.01\\
93.32	0.01\\
93.33	0.01\\
93.34	0.01\\
93.35	0.01\\
93.36	0.01\\
93.37	0.01\\
93.38	0.01\\
93.39	0.01\\
93.4	0.01\\
93.41	0.01\\
93.42	0.01\\
93.43	0.01\\
93.44	0.01\\
93.45	0.01\\
93.46	0.01\\
93.47	0.01\\
93.48	0.01\\
93.49	0.01\\
93.5	0.01\\
93.51	0.01\\
93.52	0.01\\
93.53	0.01\\
93.54	0.01\\
93.55	0.01\\
93.56	0.01\\
93.57	0.01\\
93.58	0.01\\
93.59	0.01\\
93.6	0.01\\
93.61	0.01\\
93.62	0.01\\
93.63	0.01\\
93.64	0.01\\
93.65	0.01\\
93.66	0.01\\
93.67	0.01\\
93.68	0.01\\
93.69	0.01\\
93.7	0.01\\
93.71	0.01\\
93.72	0.01\\
93.73	0.01\\
93.74	0.01\\
93.75	0.01\\
93.76	0.01\\
93.77	0.01\\
93.78	0.01\\
93.79	0.01\\
93.8	0.01\\
93.81	0.01\\
93.82	0.01\\
93.83	0.01\\
93.84	0.01\\
93.85	0.01\\
93.86	0.01\\
93.87	0.01\\
93.88	0.01\\
93.89	0.01\\
93.9	0.01\\
93.91	0.01\\
93.92	0.01\\
93.93	0.01\\
93.94	0.01\\
93.95	0.01\\
93.96	0.01\\
93.97	0.01\\
93.98	0.01\\
93.99	0.01\\
94	0.01\\
94.01	0.01\\
94.02	0.01\\
94.03	0.01\\
94.04	0.01\\
94.05	0.01\\
94.06	0.01\\
94.07	0.01\\
94.08	0.01\\
94.09	0.01\\
94.1	0.01\\
94.11	0.01\\
94.12	0.01\\
94.13	0.01\\
94.14	0.01\\
94.15	0.01\\
94.16	0.01\\
94.17	0.01\\
94.18	0.01\\
94.19	0.01\\
94.2	0.01\\
94.21	0.01\\
94.22	0.01\\
94.23	0.01\\
94.24	0.01\\
94.25	0.01\\
94.26	0.01\\
94.27	0.01\\
94.28	0.01\\
94.29	0.01\\
94.3	0.01\\
94.31	0.01\\
94.32	0.01\\
94.33	0.01\\
94.34	0.01\\
94.35	0.01\\
94.36	0.01\\
94.37	0.01\\
94.38	0.01\\
94.39	0.01\\
94.4	0.01\\
94.41	0.01\\
94.42	0.01\\
94.43	0.01\\
94.44	0.01\\
94.45	0.01\\
94.46	0.01\\
94.47	0.01\\
94.48	0.01\\
94.49	0.01\\
94.5	0.01\\
94.51	0.01\\
94.52	0.01\\
94.53	0.01\\
94.54	0.01\\
94.55	0.01\\
94.56	0.01\\
94.57	0.01\\
94.58	0.01\\
94.59	0.01\\
94.6	0.01\\
94.61	0.01\\
94.62	0.01\\
94.63	0.01\\
94.64	0.01\\
94.65	0.01\\
94.66	0.01\\
94.67	0.01\\
94.68	0.01\\
94.69	0.01\\
94.7	0.01\\
94.71	0.01\\
94.72	0.01\\
94.73	0.01\\
94.74	0.01\\
94.75	0.01\\
94.76	0.01\\
94.77	0.01\\
94.78	0.01\\
94.79	0.01\\
94.8	0.01\\
94.81	0.01\\
94.82	0.01\\
94.83	0.01\\
94.84	0.01\\
94.85	0.01\\
94.86	0.01\\
94.87	0.01\\
94.88	0.01\\
94.89	0.01\\
94.9	0.01\\
94.91	0.01\\
94.92	0.01\\
94.93	0.01\\
94.94	0.01\\
94.95	0.01\\
94.96	0.01\\
94.97	0.01\\
94.98	0.01\\
94.99	0.01\\
95	0.01\\
95.01	0.01\\
95.02	0.01\\
95.03	0.01\\
95.04	0.01\\
95.05	0.01\\
95.06	0.01\\
95.07	0.01\\
95.08	0.01\\
95.09	0.01\\
95.1	0.01\\
95.11	0.01\\
95.12	0.01\\
95.13	0.01\\
95.14	0.01\\
95.15	0.01\\
95.16	0.01\\
95.17	0.01\\
95.18	0.01\\
95.19	0.01\\
95.2	0.01\\
95.21	0.01\\
95.22	0.01\\
95.23	0.01\\
95.24	0.01\\
95.25	0.01\\
95.26	0.01\\
95.27	0.01\\
95.28	0.01\\
95.29	0.01\\
95.3	0.01\\
95.31	0.01\\
95.32	0.01\\
95.33	0.01\\
95.34	0.01\\
95.35	0.01\\
95.36	0.01\\
95.37	0.01\\
95.38	0.01\\
95.39	0.01\\
95.4	0.01\\
95.41	0.01\\
95.42	0.01\\
95.43	0.01\\
95.44	0.01\\
95.45	0.01\\
95.46	0.01\\
95.47	0.01\\
95.48	0.01\\
95.49	0.01\\
95.5	0.01\\
95.51	0.01\\
95.52	0.01\\
95.53	0.01\\
95.54	0.01\\
95.55	0.01\\
95.56	0.01\\
95.57	0.01\\
95.58	0.01\\
95.59	0.01\\
95.6	0.01\\
95.61	0.01\\
95.62	0.01\\
95.63	0.01\\
95.64	0.01\\
95.65	0.01\\
95.66	0.01\\
95.67	0.01\\
95.68	0.01\\
95.69	0.01\\
95.7	0.01\\
95.71	0.01\\
95.72	0.01\\
95.73	0.01\\
95.74	0.01\\
95.75	0.01\\
95.76	0.01\\
95.77	0.01\\
95.78	0.01\\
95.79	0.01\\
95.8	0.01\\
95.81	0.01\\
95.82	0.01\\
95.83	0.01\\
95.84	0.01\\
95.85	0.01\\
95.86	0.01\\
95.87	0.01\\
95.88	0.01\\
95.89	0.01\\
95.9	0.01\\
95.91	0.01\\
95.92	0.01\\
95.93	0.01\\
95.94	0.01\\
95.95	0.01\\
95.96	0.01\\
95.97	0.01\\
95.98	0.01\\
95.99	0.01\\
96	0.01\\
96.01	0.01\\
96.02	0.01\\
96.03	0.01\\
96.04	0.01\\
96.05	0.01\\
96.06	0.01\\
96.07	0.01\\
96.08	0.01\\
96.09	0.01\\
96.1	0.01\\
96.11	0.01\\
96.12	0.01\\
96.13	0.01\\
96.14	0.01\\
96.15	0.01\\
96.16	0.01\\
96.17	0.01\\
96.18	0.01\\
96.19	0.01\\
96.2	0.01\\
96.21	0.01\\
96.22	0.01\\
96.23	0.01\\
96.24	0.01\\
96.25	0.01\\
96.26	0.01\\
96.27	0.01\\
96.28	0.01\\
96.29	0.01\\
96.3	0.01\\
96.31	0.01\\
96.32	0.01\\
96.33	0.01\\
96.34	0.01\\
96.35	0.01\\
96.36	0.01\\
96.37	0.01\\
96.38	0.01\\
96.39	0.01\\
96.4	0.01\\
96.41	0.01\\
96.42	0.01\\
96.43	0.01\\
96.44	0.01\\
96.45	0.01\\
96.46	0.01\\
96.47	0.01\\
96.48	0.01\\
96.49	0.01\\
96.5	0.01\\
96.51	0.01\\
96.52	0.01\\
96.53	0.01\\
96.54	0.01\\
96.55	0.01\\
96.56	0.01\\
96.57	0.01\\
96.58	0.01\\
96.59	0.01\\
96.6	0.01\\
96.61	0.01\\
96.62	0.01\\
96.63	0.01\\
96.64	0.01\\
96.65	0.01\\
96.66	0.01\\
96.67	0.01\\
96.68	0.01\\
96.69	0.01\\
96.7	0.01\\
96.71	0.01\\
96.72	0.01\\
96.73	0.01\\
96.74	0.01\\
96.75	0.01\\
96.76	0.01\\
96.77	0.01\\
96.78	0.01\\
96.79	0.01\\
96.8	0.01\\
96.81	0.01\\
96.82	0.01\\
96.83	0.01\\
96.84	0.01\\
96.85	0.01\\
96.86	0.01\\
96.87	0.01\\
96.88	0.01\\
96.89	0.01\\
96.9	0.01\\
96.91	0.01\\
96.92	0.01\\
96.93	0.01\\
96.94	0.01\\
96.95	0.01\\
96.96	0.01\\
96.97	0.01\\
96.98	0.01\\
96.99	0.01\\
97	0.01\\
97.01	0.01\\
97.02	0.01\\
97.03	0.01\\
97.04	0.01\\
97.05	0.01\\
97.06	0.01\\
97.07	0.01\\
97.08	0.01\\
97.09	0.01\\
97.1	0.01\\
97.11	0.01\\
97.12	0.01\\
97.13	0.01\\
97.14	0.01\\
97.15	0.01\\
97.16	0.01\\
97.17	0.01\\
97.18	0.01\\
97.19	0.01\\
97.2	0.01\\
97.21	0.01\\
97.22	0.01\\
97.23	0.01\\
97.24	0.01\\
97.25	0.01\\
97.26	0.01\\
97.27	0.01\\
97.28	0.01\\
97.29	0.01\\
97.3	0.01\\
97.31	0.01\\
97.32	0.01\\
97.33	0.01\\
97.34	0.01\\
97.35	0.01\\
97.36	0.01\\
97.37	0.01\\
97.38	0.01\\
97.39	0.01\\
97.4	0.01\\
97.41	0.01\\
97.42	0.01\\
97.43	0.01\\
97.44	0.01\\
97.45	0.01\\
97.46	0.01\\
97.47	0.01\\
97.48	0.01\\
97.49	0.01\\
97.5	0.01\\
97.51	0.01\\
97.52	0.01\\
97.53	0.01\\
97.54	0.01\\
97.55	0.01\\
97.56	0.01\\
97.57	0.01\\
97.58	0.01\\
97.59	0.01\\
97.6	0.01\\
97.61	0.01\\
97.62	0.01\\
97.63	0.01\\
97.64	0.01\\
97.65	0.01\\
97.66	0.01\\
97.67	0.01\\
97.68	0.01\\
97.69	0.01\\
97.7	0.01\\
97.71	0.01\\
97.72	0.01\\
97.73	0.01\\
97.74	0.01\\
97.75	0.01\\
97.76	0.01\\
97.77	0.01\\
97.78	0.01\\
97.79	0.01\\
97.8	0.01\\
97.81	0.01\\
97.82	0.01\\
97.83	0.01\\
97.84	0.01\\
97.85	0.01\\
97.86	0.01\\
97.87	0.01\\
97.88	0.01\\
97.89	0.01\\
97.9	0.01\\
97.91	0.01\\
97.92	0.01\\
97.93	0.01\\
97.94	0.01\\
97.95	0.01\\
97.96	0.01\\
97.97	0.01\\
97.98	0.01\\
97.99	0.01\\
98	0.01\\
98.01	0.01\\
98.02	0.01\\
98.03	0.01\\
98.04	0.01\\
98.05	0.01\\
98.06	0.01\\
98.07	0.01\\
98.08	0.01\\
98.09	0.01\\
98.1	0.01\\
98.11	0.01\\
98.12	0.01\\
98.13	0.01\\
98.14	0.01\\
98.15	0.01\\
98.16	0.01\\
98.17	0.01\\
98.18	0.01\\
98.19	0.01\\
98.2	0.01\\
98.21	0.01\\
98.22	0.01\\
98.23	0.01\\
98.24	0.01\\
98.25	0.01\\
98.26	0.01\\
98.27	0.01\\
98.28	0.01\\
98.29	0.01\\
98.3	0.01\\
98.31	0.01\\
98.32	0.01\\
98.33	0.01\\
98.34	0.01\\
98.35	0.01\\
98.36	0.01\\
98.37	0.01\\
98.38	0.01\\
98.39	0.01\\
98.4	0.01\\
98.41	0.01\\
98.42	0.01\\
98.43	0.01\\
98.44	0.01\\
98.45	0.01\\
98.46	0.01\\
98.47	0.01\\
98.48	0.01\\
98.49	0.01\\
98.5	0.01\\
98.51	0.01\\
98.52	0.01\\
98.53	0.01\\
98.54	0.01\\
98.55	0.01\\
98.56	0.01\\
98.57	0.01\\
98.58	0.01\\
98.59	0.01\\
98.6	0.01\\
98.61	0.01\\
98.62	0.01\\
98.63	0.01\\
98.64	0.01\\
98.65	0.01\\
98.66	0.01\\
98.67	0.01\\
98.68	0.01\\
98.69	0.01\\
98.7	0.01\\
98.71	0.01\\
98.72	0.01\\
98.73	0.01\\
98.74	0.01\\
98.75	0.01\\
98.76	0.01\\
98.77	0.01\\
98.78	0.01\\
98.79	0.01\\
98.8	0.01\\
98.81	0.01\\
98.82	0.01\\
98.83	0.01\\
98.84	0.01\\
98.85	0.01\\
98.86	0.01\\
98.87	0.01\\
98.88	0.01\\
98.89	0.01\\
98.9	0.01\\
98.91	0.01\\
98.92	0.01\\
98.93	0.01\\
98.94	0.01\\
98.95	0.01\\
98.96	0.01\\
98.97	0.01\\
98.98	0.01\\
98.99	0.01\\
99	0.01\\
99.01	0.01\\
99.02	0.01\\
99.03	0.01\\
99.04	0.01\\
99.05	0.01\\
99.06	0.01\\
99.07	0.01\\
99.08	0.01\\
99.09	0.01\\
99.1	0.01\\
99.11	0.01\\
99.12	0.01\\
99.13	0.01\\
99.14	0.01\\
99.15	0.01\\
99.16	0.01\\
99.17	0.01\\
99.18	0.01\\
99.19	0.01\\
99.2	0.01\\
99.21	0.01\\
99.22	0.01\\
99.23	0.01\\
99.24	0.01\\
99.25	0.01\\
99.26	0.01\\
99.27	0.01\\
99.28	0.01\\
99.29	0.01\\
99.3	0.01\\
99.31	0.01\\
99.32	0.01\\
99.33	0.01\\
99.34	0.01\\
99.35	0.01\\
99.36	0.01\\
99.37	0.01\\
99.38	0.01\\
99.39	0.01\\
99.4	0.01\\
99.41	0.01\\
99.42	0.01\\
99.43	0.01\\
99.44	0.01\\
99.45	0.01\\
99.46	0.01\\
99.47	0.01\\
99.48	0.01\\
99.49	0.01\\
99.5	0.01\\
99.51	0.01\\
99.52	0.01\\
99.53	0.01\\
99.54	0.01\\
99.55	0.01\\
99.56	0.01\\
99.57	0.01\\
99.58	0.01\\
99.59	0.01\\
99.6	0.01\\
99.61	0.01\\
99.62	0.01\\
99.63	0.01\\
99.64	0.01\\
99.65	0.01\\
99.66	0.01\\
99.67	0.01\\
99.68	0.01\\
99.69	0.01\\
99.7	0.01\\
99.71	0.01\\
99.72	0.01\\
99.73	0.01\\
99.74	0.01\\
99.75	0.01\\
99.76	0.01\\
99.77	0.01\\
99.78	0.01\\
99.79	0.01\\
99.8	0.01\\
99.81	0.01\\
99.82	0.01\\
99.83	0.01\\
99.84	0.01\\
99.85	0.01\\
99.86	0.01\\
99.87	0.01\\
99.88	0.01\\
99.89	0.01\\
99.9	0.01\\
99.91	0.01\\
99.92	0.01\\
99.93	0.01\\
99.94	0.01\\
99.95	0.01\\
99.96	0.01\\
99.97	0.01\\
99.98	0.01\\
99.99	0.01\\
100	0.01\\
};
\addlegendentry{$q=-3$};

\addplot [color=red,dashed,forget plot]
  table[row sep=crcr]{%
0.01	0.01\\
0.02	0.01\\
0.03	0.01\\
0.04	0.01\\
0.05	0.01\\
0.06	0.01\\
0.07	0.01\\
0.08	0.01\\
0.09	0.01\\
0.1	0.01\\
0.11	0.01\\
0.12	0.01\\
0.13	0.01\\
0.14	0.01\\
0.15	0.01\\
0.16	0.01\\
0.17	0.01\\
0.18	0.01\\
0.19	0.01\\
0.2	0.01\\
0.21	0.01\\
0.22	0.01\\
0.23	0.01\\
0.24	0.01\\
0.25	0.01\\
0.26	0.01\\
0.27	0.01\\
0.28	0.01\\
0.29	0.01\\
0.3	0.01\\
0.31	0.01\\
0.32	0.01\\
0.33	0.01\\
0.34	0.01\\
0.35	0.01\\
0.36	0.01\\
0.37	0.01\\
0.38	0.01\\
0.39	0.01\\
0.4	0.01\\
0.41	0.01\\
0.42	0.01\\
0.43	0.01\\
0.44	0.01\\
0.45	0.01\\
0.46	0.01\\
0.47	0.01\\
0.48	0.01\\
0.49	0.01\\
0.5	0.01\\
0.51	0.01\\
0.52	0.01\\
0.53	0.01\\
0.54	0.01\\
0.55	0.01\\
0.56	0.01\\
0.57	0.01\\
0.58	0.01\\
0.59	0.01\\
0.6	0.01\\
0.61	0.01\\
0.62	0.01\\
0.63	0.01\\
0.64	0.01\\
0.65	0.01\\
0.66	0.01\\
0.67	0.01\\
0.68	0.01\\
0.69	0.01\\
0.7	0.01\\
0.71	0.01\\
0.72	0.01\\
0.73	0.01\\
0.74	0.01\\
0.75	0.01\\
0.76	0.01\\
0.77	0.01\\
0.78	0.01\\
0.79	0.01\\
0.8	0.01\\
0.81	0.01\\
0.82	0.01\\
0.83	0.01\\
0.84	0.01\\
0.85	0.01\\
0.86	0.01\\
0.87	0.01\\
0.88	0.01\\
0.89	0.01\\
0.9	0.01\\
0.91	0.01\\
0.92	0.01\\
0.93	0.01\\
0.94	0.01\\
0.95	0.01\\
0.96	0.01\\
0.97	0.01\\
0.98	0.01\\
0.99	0.01\\
1	0.01\\
1.01	0.01\\
1.02	0.01\\
1.03	0.01\\
1.04	0.01\\
1.05	0.01\\
1.06	0.01\\
1.07	0.01\\
1.08	0.01\\
1.09	0.01\\
1.1	0.01\\
1.11	0.01\\
1.12	0.01\\
1.13	0.01\\
1.14	0.01\\
1.15	0.01\\
1.16	0.01\\
1.17	0.01\\
1.18	0.01\\
1.19	0.01\\
1.2	0.01\\
1.21	0.01\\
1.22	0.01\\
1.23	0.01\\
1.24	0.01\\
1.25	0.01\\
1.26	0.01\\
1.27	0.01\\
1.28	0.01\\
1.29	0.01\\
1.3	0.01\\
1.31	0.01\\
1.32	0.01\\
1.33	0.01\\
1.34	0.01\\
1.35	0.01\\
1.36	0.01\\
1.37	0.01\\
1.38	0.01\\
1.39	0.01\\
1.4	0.01\\
1.41	0.01\\
1.42	0.01\\
1.43	0.01\\
1.44	0.01\\
1.45	0.01\\
1.46	0.01\\
1.47	0.01\\
1.48	0.01\\
1.49	0.01\\
1.5	0.01\\
1.51	0.01\\
1.52	0.01\\
1.53	0.01\\
1.54	0.01\\
1.55	0.01\\
1.56	0.01\\
1.57	0.01\\
1.58	0.01\\
1.59	0.01\\
1.6	0.01\\
1.61	0.01\\
1.62	0.01\\
1.63	0.01\\
1.64	0.01\\
1.65	0.01\\
1.66	0.01\\
1.67	0.01\\
1.68	0.01\\
1.69	0.01\\
1.7	0.01\\
1.71	0.01\\
1.72	0.01\\
1.73	0.01\\
1.74	0.01\\
1.75	0.01\\
1.76	0.01\\
1.77	0.01\\
1.78	0.01\\
1.79	0.01\\
1.8	0.01\\
1.81	0.01\\
1.82	0.01\\
1.83	0.01\\
1.84	0.01\\
1.85	0.01\\
1.86	0.01\\
1.87	0.01\\
1.88	0.01\\
1.89	0.01\\
1.9	0.01\\
1.91	0.01\\
1.92	0.01\\
1.93	0.01\\
1.94	0.01\\
1.95	0.01\\
1.96	0.01\\
1.97	0.01\\
1.98	0.01\\
1.99	0.01\\
2	0.01\\
2.01	0.01\\
2.02	0.01\\
2.03	0.01\\
2.04	0.01\\
2.05	0.01\\
2.06	0.01\\
2.07	0.01\\
2.08	0.01\\
2.09	0.01\\
2.1	0.01\\
2.11	0.01\\
2.12	0.01\\
2.13	0.01\\
2.14	0.01\\
2.15	0.01\\
2.16	0.01\\
2.17	0.01\\
2.18	0.01\\
2.19	0.01\\
2.2	0.01\\
2.21	0.01\\
2.22	0.01\\
2.23	0.01\\
2.24	0.01\\
2.25	0.01\\
2.26	0.01\\
2.27	0.01\\
2.28	0.01\\
2.29	0.01\\
2.3	0.01\\
2.31	0.01\\
2.32	0.01\\
2.33	0.01\\
2.34	0.01\\
2.35	0.01\\
2.36	0.01\\
2.37	0.01\\
2.38	0.01\\
2.39	0.01\\
2.4	0.01\\
2.41	0.01\\
2.42	0.01\\
2.43	0.01\\
2.44	0.01\\
2.45	0.01\\
2.46	0.01\\
2.47	0.01\\
2.48	0.01\\
2.49	0.01\\
2.5	0.01\\
2.51	0.01\\
2.52	0.01\\
2.53	0.01\\
2.54	0.01\\
2.55	0.01\\
2.56	0.01\\
2.57	0.01\\
2.58	0.01\\
2.59	0.01\\
2.6	0.01\\
2.61	0.01\\
2.62	0.01\\
2.63	0.01\\
2.64	0.01\\
2.65	0.01\\
2.66	0.01\\
2.67	0.01\\
2.68	0.01\\
2.69	0.01\\
2.7	0.01\\
2.71	0.01\\
2.72	0.01\\
2.73	0.01\\
2.74	0.01\\
2.75	0.01\\
2.76	0.01\\
2.77	0.01\\
2.78	0.01\\
2.79	0.01\\
2.8	0.01\\
2.81	0.01\\
2.82	0.01\\
2.83	0.01\\
2.84	0.01\\
2.85	0.01\\
2.86	0.01\\
2.87	0.01\\
2.88	0.01\\
2.89	0.01\\
2.9	0.01\\
2.91	0.01\\
2.92	0.01\\
2.93	0.01\\
2.94	0.01\\
2.95	0.01\\
2.96	0.01\\
2.97	0.01\\
2.98	0.01\\
2.99	0.01\\
3	0.01\\
3.01	0.01\\
3.02	0.01\\
3.03	0.01\\
3.04	0.01\\
3.05	0.01\\
3.06	0.01\\
3.07	0.01\\
3.08	0.01\\
3.09	0.01\\
3.1	0.01\\
3.11	0.01\\
3.12	0.01\\
3.13	0.01\\
3.14	0.01\\
3.15	0.01\\
3.16	0.01\\
3.17	0.01\\
3.18	0.01\\
3.19	0.01\\
3.2	0.01\\
3.21	0.01\\
3.22	0.01\\
3.23	0.01\\
3.24	0.01\\
3.25	0.01\\
3.26	0.01\\
3.27	0.01\\
3.28	0.01\\
3.29	0.01\\
3.3	0.01\\
3.31	0.01\\
3.32	0.01\\
3.33	0.01\\
3.34	0.01\\
3.35	0.01\\
3.36	0.01\\
3.37	0.01\\
3.38	0.01\\
3.39	0.01\\
3.4	0.01\\
3.41	0.01\\
3.42	0.01\\
3.43	0.01\\
3.44	0.01\\
3.45	0.01\\
3.46	0.01\\
3.47	0.01\\
3.48	0.01\\
3.49	0.01\\
3.5	0.01\\
3.51	0.01\\
3.52	0.01\\
3.53	0.01\\
3.54	0.01\\
3.55	0.01\\
3.56	0.01\\
3.57	0.01\\
3.58	0.01\\
3.59	0.01\\
3.6	0.01\\
3.61	0.01\\
3.62	0.01\\
3.63	0.01\\
3.64	0.01\\
3.65	0.01\\
3.66	0.01\\
3.67	0.01\\
3.68	0.01\\
3.69	0.01\\
3.7	0.01\\
3.71	0.01\\
3.72	0.01\\
3.73	0.01\\
3.74	0.01\\
3.75	0.01\\
3.76	0.01\\
3.77	0.01\\
3.78	0.01\\
3.79	0.01\\
3.8	0.01\\
3.81	0.01\\
3.82	0.01\\
3.83	0.01\\
3.84	0.01\\
3.85	0.01\\
3.86	0.01\\
3.87	0.01\\
3.88	0.01\\
3.89	0.01\\
3.9	0.01\\
3.91	0.01\\
3.92	0.01\\
3.93	0.01\\
3.94	0.01\\
3.95	0.01\\
3.96	0.01\\
3.97	0.01\\
3.98	0.01\\
3.99	0.01\\
4	0.01\\
4.01	0.01\\
4.02	0.01\\
4.03	0.01\\
4.04	0.01\\
4.05	0.01\\
4.06	0.01\\
4.07	0.01\\
4.08	0.01\\
4.09	0.01\\
4.1	0.01\\
4.11	0.01\\
4.12	0.01\\
4.13	0.01\\
4.14	0.01\\
4.15	0.01\\
4.16	0.01\\
4.17	0.01\\
4.18	0.01\\
4.19	0.01\\
4.2	0.01\\
4.21	0.01\\
4.22	0.01\\
4.23	0.01\\
4.24	0.01\\
4.25	0.01\\
4.26	0.01\\
4.27	0.01\\
4.28	0.01\\
4.29	0.01\\
4.3	0.01\\
4.31	0.01\\
4.32	0.01\\
4.33	0.01\\
4.34	0.01\\
4.35	0.01\\
4.36	0.01\\
4.37	0.01\\
4.38	0.01\\
4.39	0.01\\
4.4	0.01\\
4.41	0.01\\
4.42	0.01\\
4.43	0.01\\
4.44	0.01\\
4.45	0.01\\
4.46	0.01\\
4.47	0.01\\
4.48	0.01\\
4.49	0.01\\
4.5	0.01\\
4.51	0.01\\
4.52	0.01\\
4.53	0.01\\
4.54	0.01\\
4.55	0.01\\
4.56	0.01\\
4.57	0.01\\
4.58	0.01\\
4.59	0.01\\
4.6	0.01\\
4.61	0.01\\
4.62	0.01\\
4.63	0.01\\
4.64	0.01\\
4.65	0.01\\
4.66	0.01\\
4.67	0.01\\
4.68	0.01\\
4.69	0.01\\
4.7	0.01\\
4.71	0.01\\
4.72	0.01\\
4.73	0.01\\
4.74	0.01\\
4.75	0.01\\
4.76	0.01\\
4.77	0.01\\
4.78	0.01\\
4.79	0.01\\
4.8	0.01\\
4.81	0.01\\
4.82	0.01\\
4.83	0.01\\
4.84	0.01\\
4.85	0.01\\
4.86	0.01\\
4.87	0.01\\
4.88	0.01\\
4.89	0.01\\
4.9	0.01\\
4.91	0.01\\
4.92	0.01\\
4.93	0.01\\
4.94	0.01\\
4.95	0.01\\
4.96	0.01\\
4.97	0.01\\
4.98	0.01\\
4.99	0.01\\
5	0.01\\
5.01	0.01\\
5.02	0.01\\
5.03	0.01\\
5.04	0.01\\
5.05	0.01\\
5.06	0.01\\
5.07	0.01\\
5.08	0.01\\
5.09	0.01\\
5.1	0.01\\
5.11	0.01\\
5.12	0.01\\
5.13	0.01\\
5.14	0.01\\
5.15	0.01\\
5.16	0.01\\
5.17	0.01\\
5.18	0.01\\
5.19	0.01\\
5.2	0.01\\
5.21	0.01\\
5.22	0.01\\
5.23	0.01\\
5.24	0.01\\
5.25	0.01\\
5.26	0.01\\
5.27	0.01\\
5.28	0.01\\
5.29	0.01\\
5.3	0.01\\
5.31	0.01\\
5.32	0.01\\
5.33	0.01\\
5.34	0.01\\
5.35	0.01\\
5.36	0.01\\
5.37	0.01\\
5.38	0.01\\
5.39	0.01\\
5.4	0.01\\
5.41	0.01\\
5.42	0.01\\
5.43	0.01\\
5.44	0.01\\
5.45	0.01\\
5.46	0.01\\
5.47	0.01\\
5.48	0.01\\
5.49	0.01\\
5.5	0.01\\
5.51	0.01\\
5.52	0.01\\
5.53	0.01\\
5.54	0.01\\
5.55	0.01\\
5.56	0.01\\
5.57	0.01\\
5.58	0.01\\
5.59	0.01\\
5.6	0.01\\
5.61	0.01\\
5.62	0.01\\
5.63	0.01\\
5.64	0.01\\
5.65	0.01\\
5.66	0.01\\
5.67	0.01\\
5.68	0.01\\
5.69	0.01\\
5.7	0.01\\
5.71	0.01\\
5.72	0.01\\
5.73	0.01\\
5.74	0.01\\
5.75	0.01\\
5.76	0.01\\
5.77	0.01\\
5.78	0.01\\
5.79	0.01\\
5.8	0.01\\
5.81	0.01\\
5.82	0.01\\
5.83	0.01\\
5.84	0.01\\
5.85	0.01\\
5.86	0.01\\
5.87	0.01\\
5.88	0.01\\
5.89	0.01\\
5.9	0.01\\
5.91	0.01\\
5.92	0.01\\
5.93	0.01\\
5.94	0.01\\
5.95	0.01\\
5.96	0.01\\
5.97	0.01\\
5.98	0.01\\
5.99	0.01\\
6	0.01\\
6.01	0.01\\
6.02	0.01\\
6.03	0.01\\
6.04	0.01\\
6.05	0.01\\
6.06	0.01\\
6.07	0.01\\
6.08	0.01\\
6.09	0.01\\
6.1	0.01\\
6.11	0.01\\
6.12	0.01\\
6.13	0.01\\
6.14	0.01\\
6.15	0.01\\
6.16	0.01\\
6.17	0.01\\
6.18	0.01\\
6.19	0.01\\
6.2	0.01\\
6.21	0.01\\
6.22	0.01\\
6.23	0.01\\
6.24	0.01\\
6.25	0.01\\
6.26	0.01\\
6.27	0.01\\
6.28	0.01\\
6.29	0.01\\
6.3	0.01\\
6.31	0.01\\
6.32	0.01\\
6.33	0.01\\
6.34	0.01\\
6.35	0.01\\
6.36	0.01\\
6.37	0.01\\
6.38	0.01\\
6.39	0.01\\
6.4	0.01\\
6.41	0.01\\
6.42	0.01\\
6.43	0.01\\
6.44	0.01\\
6.45	0.01\\
6.46	0.01\\
6.47	0.01\\
6.48	0.01\\
6.49	0.01\\
6.5	0.01\\
6.51	0.01\\
6.52	0.01\\
6.53	0.01\\
6.54	0.01\\
6.55	0.01\\
6.56	0.01\\
6.57	0.01\\
6.58	0.01\\
6.59	0.01\\
6.6	0.01\\
6.61	0.01\\
6.62	0.01\\
6.63	0.01\\
6.64	0.01\\
6.65	0.01\\
6.66	0.01\\
6.67	0.01\\
6.68	0.01\\
6.69	0.01\\
6.7	0.01\\
6.71	0.01\\
6.72	0.01\\
6.73	0.01\\
6.74	0.01\\
6.75	0.01\\
6.76	0.01\\
6.77	0.01\\
6.78	0.01\\
6.79	0.01\\
6.8	0.01\\
6.81	0.01\\
6.82	0.01\\
6.83	0.01\\
6.84	0.01\\
6.85	0.01\\
6.86	0.01\\
6.87	0.01\\
6.88	0.01\\
6.89	0.01\\
6.9	0.01\\
6.91	0.01\\
6.92	0.01\\
6.93	0.01\\
6.94	0.01\\
6.95	0.01\\
6.96	0.01\\
6.97	0.01\\
6.98	0.01\\
6.99	0.01\\
7	0.01\\
7.01	0.01\\
7.02	0.01\\
7.03	0.01\\
7.04	0.01\\
7.05	0.01\\
7.06	0.01\\
7.07	0.01\\
7.08	0.01\\
7.09	0.01\\
7.1	0.01\\
7.11	0.01\\
7.12	0.01\\
7.13	0.01\\
7.14	0.01\\
7.15	0.01\\
7.16	0.01\\
7.17	0.01\\
7.18	0.01\\
7.19	0.01\\
7.2	0.01\\
7.21	0.01\\
7.22	0.01\\
7.23	0.01\\
7.24	0.01\\
7.25	0.01\\
7.26	0.01\\
7.27	0.01\\
7.28	0.01\\
7.29	0.01\\
7.3	0.01\\
7.31	0.01\\
7.32	0.01\\
7.33	0.01\\
7.34	0.01\\
7.35	0.01\\
7.36	0.01\\
7.37	0.01\\
7.38	0.01\\
7.39	0.01\\
7.4	0.01\\
7.41	0.01\\
7.42	0.01\\
7.43	0.01\\
7.44	0.01\\
7.45	0.01\\
7.46	0.01\\
7.47	0.01\\
7.48	0.01\\
7.49	0.01\\
7.5	0.01\\
7.51	0.01\\
7.52	0.01\\
7.53	0.01\\
7.54	0.01\\
7.55	0.01\\
7.56	0.01\\
7.57	0.01\\
7.58	0.01\\
7.59	0.01\\
7.6	0.01\\
7.61	0.01\\
7.62	0.01\\
7.63	0.01\\
7.64	0.01\\
7.65	0.01\\
7.66	0.01\\
7.67	0.01\\
7.68	0.01\\
7.69	0.01\\
7.7	0.01\\
7.71	0.01\\
7.72	0.01\\
7.73	0.01\\
7.74	0.01\\
7.75	0.01\\
7.76	0.01\\
7.77	0.01\\
7.78	0.01\\
7.79	0.01\\
7.8	0.01\\
7.81	0.01\\
7.82	0.01\\
7.83	0.01\\
7.84	0.01\\
7.85	0.01\\
7.86	0.01\\
7.87	0.01\\
7.88	0.01\\
7.89	0.01\\
7.9	0.01\\
7.91	0.01\\
7.92	0.01\\
7.93	0.01\\
7.94	0.01\\
7.95	0.01\\
7.96	0.01\\
7.97	0.01\\
7.98	0.01\\
7.99	0.01\\
8	0.01\\
8.01	0.01\\
8.02	0.01\\
8.03	0.01\\
8.04	0.01\\
8.05	0.01\\
8.06	0.01\\
8.07	0.01\\
8.08	0.01\\
8.09	0.01\\
8.1	0.01\\
8.11	0.01\\
8.12	0.01\\
8.13	0.01\\
8.14	0.01\\
8.15	0.01\\
8.16	0.01\\
8.17	0.01\\
8.18	0.01\\
8.19	0.01\\
8.2	0.01\\
8.21	0.01\\
8.22	0.01\\
8.23	0.01\\
8.24	0.01\\
8.25	0.01\\
8.26	0.01\\
8.27	0.01\\
8.28	0.01\\
8.29	0.01\\
8.3	0.01\\
8.31	0.01\\
8.32	0.01\\
8.33	0.01\\
8.34	0.01\\
8.35	0.01\\
8.36	0.01\\
8.37	0.01\\
8.38	0.01\\
8.39	0.01\\
8.4	0.01\\
8.41	0.01\\
8.42	0.01\\
8.43	0.01\\
8.44	0.01\\
8.45	0.01\\
8.46	0.01\\
8.47	0.01\\
8.48	0.01\\
8.49	0.01\\
8.5	0.01\\
8.51	0.01\\
8.52	0.01\\
8.53	0.01\\
8.54	0.01\\
8.55	0.01\\
8.56	0.01\\
8.57	0.01\\
8.58	0.01\\
8.59	0.01\\
8.6	0.01\\
8.61	0.01\\
8.62	0.01\\
8.63	0.01\\
8.64	0.01\\
8.65	0.01\\
8.66	0.01\\
8.67	0.01\\
8.68	0.01\\
8.69	0.01\\
8.7	0.01\\
8.71	0.01\\
8.72	0.01\\
8.73	0.01\\
8.74	0.01\\
8.75	0.01\\
8.76	0.01\\
8.77	0.01\\
8.78	0.01\\
8.79	0.01\\
8.8	0.01\\
8.81	0.01\\
8.82	0.01\\
8.83	0.01\\
8.84	0.01\\
8.85	0.01\\
8.86	0.01\\
8.87	0.01\\
8.88	0.01\\
8.89	0.01\\
8.9	0.01\\
8.91	0.01\\
8.92	0.01\\
8.93	0.01\\
8.94	0.01\\
8.95	0.01\\
8.96	0.01\\
8.97	0.01\\
8.98	0.01\\
8.99	0.01\\
9	0.01\\
9.01	0.01\\
9.02	0.01\\
9.03	0.01\\
9.04	0.01\\
9.05	0.01\\
9.06	0.01\\
9.07	0.01\\
9.08	0.01\\
9.09	0.01\\
9.1	0.01\\
9.11	0.01\\
9.12	0.01\\
9.13	0.01\\
9.14	0.01\\
9.15	0.01\\
9.16	0.01\\
9.17	0.01\\
9.18	0.01\\
9.19	0.01\\
9.2	0.01\\
9.21	0.01\\
9.22	0.01\\
9.23	0.01\\
9.24	0.01\\
9.25	0.01\\
9.26	0.01\\
9.27	0.01\\
9.28	0.01\\
9.29	0.01\\
9.3	0.01\\
9.31	0.01\\
9.32	0.01\\
9.33	0.01\\
9.34	0.01\\
9.35	0.01\\
9.36	0.01\\
9.37	0.01\\
9.38	0.01\\
9.39	0.01\\
9.4	0.01\\
9.41	0.01\\
9.42	0.01\\
9.43	0.01\\
9.44	0.01\\
9.45	0.01\\
9.46	0.01\\
9.47	0.01\\
9.48	0.01\\
9.49	0.01\\
9.5	0.01\\
9.51	0.01\\
9.52	0.01\\
9.53	0.01\\
9.54	0.01\\
9.55	0.01\\
9.56	0.01\\
9.57	0.01\\
9.58	0.01\\
9.59	0.01\\
9.6	0.01\\
9.61	0.01\\
9.62	0.01\\
9.63	0.01\\
9.64	0.01\\
9.65	0.01\\
9.66	0.01\\
9.67	0.01\\
9.68	0.01\\
9.69	0.01\\
9.7	0.01\\
9.71	0.01\\
9.72	0.01\\
9.73	0.01\\
9.74	0.01\\
9.75	0.01\\
9.76	0.01\\
9.77	0.01\\
9.78	0.01\\
9.79	0.01\\
9.8	0.01\\
9.81	0.01\\
9.82	0.01\\
9.83	0.01\\
9.84	0.01\\
9.85	0.01\\
9.86	0.01\\
9.87	0.01\\
9.88	0.01\\
9.89	0.01\\
9.9	0.01\\
9.91	0.01\\
9.92	0.01\\
9.93	0.01\\
9.94	0.01\\
9.95	0.01\\
9.96	0.01\\
9.97	0.01\\
9.98	0.01\\
9.99	0.01\\
10	0.01\\
10.01	0.01\\
10.02	0.01\\
10.03	0.01\\
10.04	0.01\\
10.05	0.01\\
10.06	0.01\\
10.07	0.01\\
10.08	0.01\\
10.09	0.01\\
10.1	0.01\\
10.11	0.01\\
10.12	0.01\\
10.13	0.01\\
10.14	0.01\\
10.15	0.01\\
10.16	0.01\\
10.17	0.01\\
10.18	0.01\\
10.19	0.01\\
10.2	0.01\\
10.21	0.01\\
10.22	0.01\\
10.23	0.01\\
10.24	0.01\\
10.25	0.01\\
10.26	0.01\\
10.27	0.01\\
10.28	0.01\\
10.29	0.01\\
10.3	0.01\\
10.31	0.01\\
10.32	0.01\\
10.33	0.01\\
10.34	0.01\\
10.35	0.01\\
10.36	0.01\\
10.37	0.01\\
10.38	0.01\\
10.39	0.01\\
10.4	0.01\\
10.41	0.01\\
10.42	0.01\\
10.43	0.01\\
10.44	0.01\\
10.45	0.01\\
10.46	0.01\\
10.47	0.01\\
10.48	0.01\\
10.49	0.01\\
10.5	0.01\\
10.51	0.01\\
10.52	0.01\\
10.53	0.01\\
10.54	0.01\\
10.55	0.01\\
10.56	0.01\\
10.57	0.01\\
10.58	0.01\\
10.59	0.01\\
10.6	0.01\\
10.61	0.01\\
10.62	0.01\\
10.63	0.01\\
10.64	0.01\\
10.65	0.01\\
10.66	0.01\\
10.67	0.01\\
10.68	0.01\\
10.69	0.01\\
10.7	0.01\\
10.71	0.01\\
10.72	0.01\\
10.73	0.01\\
10.74	0.01\\
10.75	0.01\\
10.76	0.01\\
10.77	0.01\\
10.78	0.01\\
10.79	0.01\\
10.8	0.01\\
10.81	0.01\\
10.82	0.01\\
10.83	0.01\\
10.84	0.01\\
10.85	0.01\\
10.86	0.01\\
10.87	0.01\\
10.88	0.01\\
10.89	0.01\\
10.9	0.01\\
10.91	0.01\\
10.92	0.01\\
10.93	0.01\\
10.94	0.01\\
10.95	0.01\\
10.96	0.01\\
10.97	0.01\\
10.98	0.01\\
10.99	0.01\\
11	0.01\\
11.01	0.01\\
11.02	0.01\\
11.03	0.01\\
11.04	0.01\\
11.05	0.01\\
11.06	0.01\\
11.07	0.01\\
11.08	0.01\\
11.09	0.01\\
11.1	0.01\\
11.11	0.01\\
11.12	0.01\\
11.13	0.01\\
11.14	0.01\\
11.15	0.01\\
11.16	0.01\\
11.17	0.01\\
11.18	0.01\\
11.19	0.01\\
11.2	0.01\\
11.21	0.01\\
11.22	0.01\\
11.23	0.01\\
11.24	0.01\\
11.25	0.01\\
11.26	0.01\\
11.27	0.01\\
11.28	0.01\\
11.29	0.01\\
11.3	0.01\\
11.31	0.01\\
11.32	0.01\\
11.33	0.01\\
11.34	0.01\\
11.35	0.01\\
11.36	0.01\\
11.37	0.01\\
11.38	0.01\\
11.39	0.01\\
11.4	0.01\\
11.41	0.01\\
11.42	0.01\\
11.43	0.01\\
11.44	0.01\\
11.45	0.01\\
11.46	0.01\\
11.47	0.01\\
11.48	0.01\\
11.49	0.01\\
11.5	0.01\\
11.51	0.01\\
11.52	0.01\\
11.53	0.01\\
11.54	0.01\\
11.55	0.01\\
11.56	0.01\\
11.57	0.01\\
11.58	0.01\\
11.59	0.01\\
11.6	0.01\\
11.61	0.01\\
11.62	0.01\\
11.63	0.01\\
11.64	0.01\\
11.65	0.01\\
11.66	0.01\\
11.67	0.01\\
11.68	0.01\\
11.69	0.01\\
11.7	0.01\\
11.71	0.01\\
11.72	0.01\\
11.73	0.01\\
11.74	0.01\\
11.75	0.01\\
11.76	0.01\\
11.77	0.01\\
11.78	0.01\\
11.79	0.01\\
11.8	0.01\\
11.81	0.01\\
11.82	0.01\\
11.83	0.01\\
11.84	0.01\\
11.85	0.01\\
11.86	0.01\\
11.87	0.01\\
11.88	0.01\\
11.89	0.01\\
11.9	0.01\\
11.91	0.01\\
11.92	0.01\\
11.93	0.01\\
11.94	0.01\\
11.95	0.01\\
11.96	0.01\\
11.97	0.01\\
11.98	0.01\\
11.99	0.01\\
12	0.01\\
12.01	0.01\\
12.02	0.01\\
12.03	0.01\\
12.04	0.01\\
12.05	0.01\\
12.06	0.01\\
12.07	0.01\\
12.08	0.01\\
12.09	0.01\\
12.1	0.01\\
12.11	0.01\\
12.12	0.01\\
12.13	0.01\\
12.14	0.01\\
12.15	0.01\\
12.16	0.01\\
12.17	0.01\\
12.18	0.01\\
12.19	0.01\\
12.2	0.01\\
12.21	0.01\\
12.22	0.01\\
12.23	0.01\\
12.24	0.01\\
12.25	0.01\\
12.26	0.01\\
12.27	0.01\\
12.28	0.01\\
12.29	0.01\\
12.3	0.01\\
12.31	0.01\\
12.32	0.01\\
12.33	0.01\\
12.34	0.01\\
12.35	0.01\\
12.36	0.01\\
12.37	0.01\\
12.38	0.01\\
12.39	0.01\\
12.4	0.01\\
12.41	0.01\\
12.42	0.01\\
12.43	0.01\\
12.44	0.01\\
12.45	0.01\\
12.46	0.01\\
12.47	0.01\\
12.48	0.01\\
12.49	0.01\\
12.5	0.01\\
12.51	0.01\\
12.52	0.01\\
12.53	0.01\\
12.54	0.01\\
12.55	0.01\\
12.56	0.01\\
12.57	0.01\\
12.58	0.01\\
12.59	0.01\\
12.6	0.01\\
12.61	0.01\\
12.62	0.01\\
12.63	0.01\\
12.64	0.01\\
12.65	0.01\\
12.66	0.01\\
12.67	0.01\\
12.68	0.01\\
12.69	0.01\\
12.7	0.01\\
12.71	0.01\\
12.72	0.01\\
12.73	0.01\\
12.74	0.01\\
12.75	0.01\\
12.76	0.01\\
12.77	0.01\\
12.78	0.01\\
12.79	0.01\\
12.8	0.01\\
12.81	0.01\\
12.82	0.01\\
12.83	0.01\\
12.84	0.01\\
12.85	0.01\\
12.86	0.01\\
12.87	0.01\\
12.88	0.01\\
12.89	0.01\\
12.9	0.01\\
12.91	0.01\\
12.92	0.01\\
12.93	0.01\\
12.94	0.01\\
12.95	0.01\\
12.96	0.01\\
12.97	0.01\\
12.98	0.01\\
12.99	0.01\\
13	0.01\\
13.01	0.01\\
13.02	0.01\\
13.03	0.01\\
13.04	0.01\\
13.05	0.01\\
13.06	0.01\\
13.07	0.01\\
13.08	0.01\\
13.09	0.01\\
13.1	0.01\\
13.11	0.01\\
13.12	0.01\\
13.13	0.01\\
13.14	0.01\\
13.15	0.01\\
13.16	0.01\\
13.17	0.01\\
13.18	0.01\\
13.19	0.01\\
13.2	0.01\\
13.21	0.01\\
13.22	0.01\\
13.23	0.01\\
13.24	0.01\\
13.25	0.01\\
13.26	0.01\\
13.27	0.01\\
13.28	0.01\\
13.29	0.01\\
13.3	0.01\\
13.31	0.01\\
13.32	0.01\\
13.33	0.01\\
13.34	0.01\\
13.35	0.01\\
13.36	0.01\\
13.37	0.01\\
13.38	0.01\\
13.39	0.01\\
13.4	0.01\\
13.41	0.01\\
13.42	0.01\\
13.43	0.01\\
13.44	0.01\\
13.45	0.01\\
13.46	0.01\\
13.47	0.01\\
13.48	0.01\\
13.49	0.01\\
13.5	0.01\\
13.51	0.01\\
13.52	0.01\\
13.53	0.01\\
13.54	0.01\\
13.55	0.01\\
13.56	0.01\\
13.57	0.01\\
13.58	0.01\\
13.59	0.01\\
13.6	0.01\\
13.61	0.01\\
13.62	0.01\\
13.63	0.01\\
13.64	0.01\\
13.65	0.01\\
13.66	0.01\\
13.67	0.01\\
13.68	0.01\\
13.69	0.01\\
13.7	0.01\\
13.71	0.01\\
13.72	0.01\\
13.73	0.01\\
13.74	0.01\\
13.75	0.01\\
13.76	0.01\\
13.77	0.01\\
13.78	0.01\\
13.79	0.01\\
13.8	0.01\\
13.81	0.01\\
13.82	0.01\\
13.83	0.01\\
13.84	0.01\\
13.85	0.01\\
13.86	0.01\\
13.87	0.01\\
13.88	0.01\\
13.89	0.01\\
13.9	0.01\\
13.91	0.01\\
13.92	0.01\\
13.93	0.01\\
13.94	0.01\\
13.95	0.01\\
13.96	0.01\\
13.97	0.01\\
13.98	0.01\\
13.99	0.01\\
14	0.01\\
14.01	0.01\\
14.02	0.01\\
14.03	0.01\\
14.04	0.01\\
14.05	0.01\\
14.06	0.01\\
14.07	0.01\\
14.08	0.01\\
14.09	0.01\\
14.1	0.01\\
14.11	0.01\\
14.12	0.01\\
14.13	0.01\\
14.14	0.01\\
14.15	0.01\\
14.16	0.01\\
14.17	0.01\\
14.18	0.01\\
14.19	0.01\\
14.2	0.01\\
14.21	0.01\\
14.22	0.01\\
14.23	0.01\\
14.24	0.01\\
14.25	0.01\\
14.26	0.01\\
14.27	0.01\\
14.28	0.01\\
14.29	0.01\\
14.3	0.01\\
14.31	0.01\\
14.32	0.01\\
14.33	0.01\\
14.34	0.01\\
14.35	0.01\\
14.36	0.01\\
14.37	0.01\\
14.38	0.01\\
14.39	0.01\\
14.4	0.01\\
14.41	0.01\\
14.42	0.01\\
14.43	0.01\\
14.44	0.01\\
14.45	0.01\\
14.46	0.01\\
14.47	0.01\\
14.48	0.01\\
14.49	0.01\\
14.5	0.01\\
14.51	0.01\\
14.52	0.01\\
14.53	0.01\\
14.54	0.01\\
14.55	0.01\\
14.56	0.01\\
14.57	0.01\\
14.58	0.01\\
14.59	0.01\\
14.6	0.01\\
14.61	0.01\\
14.62	0.01\\
14.63	0.01\\
14.64	0.01\\
14.65	0.01\\
14.66	0.01\\
14.67	0.01\\
14.68	0.01\\
14.69	0.01\\
14.7	0.01\\
14.71	0.01\\
14.72	0.01\\
14.73	0.01\\
14.74	0.01\\
14.75	0.01\\
14.76	0.01\\
14.77	0.01\\
14.78	0.01\\
14.79	0.01\\
14.8	0.01\\
14.81	0.01\\
14.82	0.01\\
14.83	0.01\\
14.84	0.01\\
14.85	0.01\\
14.86	0.01\\
14.87	0.01\\
14.88	0.01\\
14.89	0.01\\
14.9	0.01\\
14.91	0.01\\
14.92	0.01\\
14.93	0.01\\
14.94	0.01\\
14.95	0.01\\
14.96	0.01\\
14.97	0.01\\
14.98	0.01\\
14.99	0.01\\
15	0.01\\
15.01	0.01\\
15.02	0.01\\
15.03	0.01\\
15.04	0.01\\
15.05	0.01\\
15.06	0.01\\
15.07	0.01\\
15.08	0.01\\
15.09	0.01\\
15.1	0.01\\
15.11	0.01\\
15.12	0.01\\
15.13	0.01\\
15.14	0.01\\
15.15	0.01\\
15.16	0.01\\
15.17	0.01\\
15.18	0.01\\
15.19	0.01\\
15.2	0.01\\
15.21	0.01\\
15.22	0.01\\
15.23	0.01\\
15.24	0.01\\
15.25	0.01\\
15.26	0.01\\
15.27	0.01\\
15.28	0.01\\
15.29	0.01\\
15.3	0.01\\
15.31	0.01\\
15.32	0.01\\
15.33	0.01\\
15.34	0.01\\
15.35	0.01\\
15.36	0.01\\
15.37	0.01\\
15.38	0.01\\
15.39	0.01\\
15.4	0.01\\
15.41	0.01\\
15.42	0.01\\
15.43	0.01\\
15.44	0.01\\
15.45	0.01\\
15.46	0.01\\
15.47	0.01\\
15.48	0.01\\
15.49	0.01\\
15.5	0.01\\
15.51	0.01\\
15.52	0.01\\
15.53	0.01\\
15.54	0.01\\
15.55	0.01\\
15.56	0.01\\
15.57	0.01\\
15.58	0.01\\
15.59	0.01\\
15.6	0.01\\
15.61	0.01\\
15.62	0.01\\
15.63	0.01\\
15.64	0.01\\
15.65	0.01\\
15.66	0.01\\
15.67	0.01\\
15.68	0.01\\
15.69	0.01\\
15.7	0.01\\
15.71	0.01\\
15.72	0.01\\
15.73	0.01\\
15.74	0.01\\
15.75	0.01\\
15.76	0.01\\
15.77	0.01\\
15.78	0.01\\
15.79	0.01\\
15.8	0.01\\
15.81	0.01\\
15.82	0.01\\
15.83	0.01\\
15.84	0.01\\
15.85	0.01\\
15.86	0.01\\
15.87	0.01\\
15.88	0.01\\
15.89	0.01\\
15.9	0.01\\
15.91	0.01\\
15.92	0.01\\
15.93	0.01\\
15.94	0.01\\
15.95	0.01\\
15.96	0.01\\
15.97	0.01\\
15.98	0.01\\
15.99	0.01\\
16	0.01\\
16.01	0.01\\
16.02	0.01\\
16.03	0.01\\
16.04	0.01\\
16.05	0.01\\
16.06	0.01\\
16.07	0.01\\
16.08	0.01\\
16.09	0.01\\
16.1	0.01\\
16.11	0.01\\
16.12	0.01\\
16.13	0.01\\
16.14	0.01\\
16.15	0.01\\
16.16	0.01\\
16.17	0.01\\
16.18	0.01\\
16.19	0.01\\
16.2	0.01\\
16.21	0.01\\
16.22	0.01\\
16.23	0.01\\
16.24	0.01\\
16.25	0.01\\
16.26	0.01\\
16.27	0.01\\
16.28	0.01\\
16.29	0.01\\
16.3	0.01\\
16.31	0.01\\
16.32	0.01\\
16.33	0.01\\
16.34	0.01\\
16.35	0.01\\
16.36	0.01\\
16.37	0.01\\
16.38	0.01\\
16.39	0.01\\
16.4	0.01\\
16.41	0.01\\
16.42	0.01\\
16.43	0.01\\
16.44	0.01\\
16.45	0.01\\
16.46	0.01\\
16.47	0.01\\
16.48	0.01\\
16.49	0.01\\
16.5	0.01\\
16.51	0.01\\
16.52	0.01\\
16.53	0.01\\
16.54	0.01\\
16.55	0.01\\
16.56	0.01\\
16.57	0.01\\
16.58	0.01\\
16.59	0.01\\
16.6	0.01\\
16.61	0.01\\
16.62	0.01\\
16.63	0.01\\
16.64	0.01\\
16.65	0.01\\
16.66	0.01\\
16.67	0.01\\
16.68	0.01\\
16.69	0.01\\
16.7	0.01\\
16.71	0.01\\
16.72	0.01\\
16.73	0.01\\
16.74	0.01\\
16.75	0.01\\
16.76	0.01\\
16.77	0.01\\
16.78	0.01\\
16.79	0.01\\
16.8	0.01\\
16.81	0.01\\
16.82	0.01\\
16.83	0.01\\
16.84	0.01\\
16.85	0.01\\
16.86	0.01\\
16.87	0.01\\
16.88	0.01\\
16.89	0.01\\
16.9	0.01\\
16.91	0.01\\
16.92	0.01\\
16.93	0.01\\
16.94	0.01\\
16.95	0.01\\
16.96	0.01\\
16.97	0.01\\
16.98	0.01\\
16.99	0.01\\
17	0.01\\
17.01	0.01\\
17.02	0.01\\
17.03	0.01\\
17.04	0.01\\
17.05	0.01\\
17.06	0.01\\
17.07	0.01\\
17.08	0.01\\
17.09	0.01\\
17.1	0.01\\
17.11	0.01\\
17.12	0.01\\
17.13	0.01\\
17.14	0.01\\
17.15	0.01\\
17.16	0.01\\
17.17	0.01\\
17.18	0.01\\
17.19	0.01\\
17.2	0.01\\
17.21	0.01\\
17.22	0.01\\
17.23	0.01\\
17.24	0.01\\
17.25	0.01\\
17.26	0.01\\
17.27	0.01\\
17.28	0.01\\
17.29	0.01\\
17.3	0.01\\
17.31	0.01\\
17.32	0.01\\
17.33	0.01\\
17.34	0.01\\
17.35	0.01\\
17.36	0.01\\
17.37	0.01\\
17.38	0.01\\
17.39	0.01\\
17.4	0.01\\
17.41	0.01\\
17.42	0.01\\
17.43	0.01\\
17.44	0.01\\
17.45	0.01\\
17.46	0.01\\
17.47	0.01\\
17.48	0.01\\
17.49	0.01\\
17.5	0.01\\
17.51	0.01\\
17.52	0.01\\
17.53	0.01\\
17.54	0.01\\
17.55	0.01\\
17.56	0.01\\
17.57	0.01\\
17.58	0.01\\
17.59	0.01\\
17.6	0.01\\
17.61	0.01\\
17.62	0.01\\
17.63	0.01\\
17.64	0.01\\
17.65	0.01\\
17.66	0.01\\
17.67	0.01\\
17.68	0.01\\
17.69	0.01\\
17.7	0.01\\
17.71	0.01\\
17.72	0.01\\
17.73	0.01\\
17.74	0.01\\
17.75	0.01\\
17.76	0.01\\
17.77	0.01\\
17.78	0.01\\
17.79	0.01\\
17.8	0.01\\
17.81	0.01\\
17.82	0.01\\
17.83	0.01\\
17.84	0.01\\
17.85	0.01\\
17.86	0.01\\
17.87	0.01\\
17.88	0.01\\
17.89	0.01\\
17.9	0.01\\
17.91	0.01\\
17.92	0.01\\
17.93	0.01\\
17.94	0.01\\
17.95	0.01\\
17.96	0.01\\
17.97	0.01\\
17.98	0.01\\
17.99	0.01\\
18	0.01\\
18.01	0.01\\
18.02	0.01\\
18.03	0.01\\
18.04	0.01\\
18.05	0.01\\
18.06	0.01\\
18.07	0.01\\
18.08	0.01\\
18.09	0.01\\
18.1	0.01\\
18.11	0.01\\
18.12	0.01\\
18.13	0.01\\
18.14	0.01\\
18.15	0.01\\
18.16	0.01\\
18.17	0.01\\
18.18	0.01\\
18.19	0.01\\
18.2	0.01\\
18.21	0.01\\
18.22	0.01\\
18.23	0.01\\
18.24	0.01\\
18.25	0.01\\
18.26	0.01\\
18.27	0.01\\
18.28	0.01\\
18.29	0.01\\
18.3	0.01\\
18.31	0.01\\
18.32	0.01\\
18.33	0.01\\
18.34	0.01\\
18.35	0.01\\
18.36	0.01\\
18.37	0.01\\
18.38	0.01\\
18.39	0.01\\
18.4	0.01\\
18.41	0.01\\
18.42	0.01\\
18.43	0.01\\
18.44	0.01\\
18.45	0.01\\
18.46	0.01\\
18.47	0.01\\
18.48	0.01\\
18.49	0.01\\
18.5	0.01\\
18.51	0.01\\
18.52	0.01\\
18.53	0.01\\
18.54	0.01\\
18.55	0.01\\
18.56	0.01\\
18.57	0.01\\
18.58	0.01\\
18.59	0.01\\
18.6	0.01\\
18.61	0.01\\
18.62	0.01\\
18.63	0.01\\
18.64	0.01\\
18.65	0.01\\
18.66	0.01\\
18.67	0.01\\
18.68	0.01\\
18.69	0.01\\
18.7	0.01\\
18.71	0.01\\
18.72	0.01\\
18.73	0.01\\
18.74	0.01\\
18.75	0.01\\
18.76	0.01\\
18.77	0.01\\
18.78	0.01\\
18.79	0.01\\
18.8	0.01\\
18.81	0.01\\
18.82	0.01\\
18.83	0.01\\
18.84	0.01\\
18.85	0.01\\
18.86	0.01\\
18.87	0.01\\
18.88	0.01\\
18.89	0.01\\
18.9	0.01\\
18.91	0.01\\
18.92	0.01\\
18.93	0.01\\
18.94	0.01\\
18.95	0.01\\
18.96	0.01\\
18.97	0.01\\
18.98	0.01\\
18.99	0.01\\
19	0.01\\
19.01	0.01\\
19.02	0.01\\
19.03	0.01\\
19.04	0.01\\
19.05	0.01\\
19.06	0.01\\
19.07	0.01\\
19.08	0.01\\
19.09	0.01\\
19.1	0.01\\
19.11	0.01\\
19.12	0.01\\
19.13	0.01\\
19.14	0.01\\
19.15	0.01\\
19.16	0.01\\
19.17	0.01\\
19.18	0.01\\
19.19	0.01\\
19.2	0.01\\
19.21	0.01\\
19.22	0.01\\
19.23	0.01\\
19.24	0.01\\
19.25	0.01\\
19.26	0.01\\
19.27	0.01\\
19.28	0.01\\
19.29	0.01\\
19.3	0.01\\
19.31	0.01\\
19.32	0.01\\
19.33	0.01\\
19.34	0.01\\
19.35	0.01\\
19.36	0.01\\
19.37	0.01\\
19.38	0.01\\
19.39	0.01\\
19.4	0.01\\
19.41	0.01\\
19.42	0.01\\
19.43	0.01\\
19.44	0.01\\
19.45	0.01\\
19.46	0.01\\
19.47	0.01\\
19.48	0.01\\
19.49	0.01\\
19.5	0.01\\
19.51	0.01\\
19.52	0.01\\
19.53	0.01\\
19.54	0.01\\
19.55	0.01\\
19.56	0.01\\
19.57	0.01\\
19.58	0.01\\
19.59	0.01\\
19.6	0.01\\
19.61	0.01\\
19.62	0.01\\
19.63	0.01\\
19.64	0.01\\
19.65	0.01\\
19.66	0.01\\
19.67	0.01\\
19.68	0.01\\
19.69	0.01\\
19.7	0.01\\
19.71	0.01\\
19.72	0.01\\
19.73	0.01\\
19.74	0.01\\
19.75	0.01\\
19.76	0.01\\
19.77	0.01\\
19.78	0.01\\
19.79	0.01\\
19.8	0.01\\
19.81	0.01\\
19.82	0.01\\
19.83	0.01\\
19.84	0.01\\
19.85	0.01\\
19.86	0.01\\
19.87	0.01\\
19.88	0.01\\
19.89	0.01\\
19.9	0.01\\
19.91	0.01\\
19.92	0.01\\
19.93	0.01\\
19.94	0.01\\
19.95	0.01\\
19.96	0.01\\
19.97	0.01\\
19.98	0.01\\
19.99	0.01\\
20	0.01\\
20.01	0.01\\
20.02	0.01\\
20.03	0.01\\
20.04	0.01\\
20.05	0.01\\
20.06	0.01\\
20.07	0.01\\
20.08	0.01\\
20.09	0.01\\
20.1	0.01\\
20.11	0.01\\
20.12	0.01\\
20.13	0.01\\
20.14	0.01\\
20.15	0.01\\
20.16	0.01\\
20.17	0.01\\
20.18	0.01\\
20.19	0.01\\
20.2	0.01\\
20.21	0.01\\
20.22	0.01\\
20.23	0.01\\
20.24	0.01\\
20.25	0.01\\
20.26	0.01\\
20.27	0.01\\
20.28	0.01\\
20.29	0.01\\
20.3	0.01\\
20.31	0.01\\
20.32	0.01\\
20.33	0.01\\
20.34	0.01\\
20.35	0.01\\
20.36	0.01\\
20.37	0.01\\
20.38	0.01\\
20.39	0.01\\
20.4	0.01\\
20.41	0.01\\
20.42	0.01\\
20.43	0.01\\
20.44	0.01\\
20.45	0.01\\
20.46	0.01\\
20.47	0.01\\
20.48	0.01\\
20.49	0.01\\
20.5	0.01\\
20.51	0.01\\
20.52	0.01\\
20.53	0.01\\
20.54	0.01\\
20.55	0.01\\
20.56	0.01\\
20.57	0.01\\
20.58	0.01\\
20.59	0.01\\
20.6	0.01\\
20.61	0.01\\
20.62	0.01\\
20.63	0.01\\
20.64	0.01\\
20.65	0.01\\
20.66	0.01\\
20.67	0.01\\
20.68	0.01\\
20.69	0.01\\
20.7	0.01\\
20.71	0.01\\
20.72	0.01\\
20.73	0.01\\
20.74	0.01\\
20.75	0.01\\
20.76	0.01\\
20.77	0.01\\
20.78	0.01\\
20.79	0.01\\
20.8	0.01\\
20.81	0.01\\
20.82	0.01\\
20.83	0.01\\
20.84	0.01\\
20.85	0.01\\
20.86	0.01\\
20.87	0.01\\
20.88	0.01\\
20.89	0.01\\
20.9	0.01\\
20.91	0.01\\
20.92	0.01\\
20.93	0.01\\
20.94	0.01\\
20.95	0.01\\
20.96	0.01\\
20.97	0.01\\
20.98	0.01\\
20.99	0.01\\
21	0.01\\
21.01	0.01\\
21.02	0.01\\
21.03	0.01\\
21.04	0.01\\
21.05	0.01\\
21.06	0.01\\
21.07	0.01\\
21.08	0.01\\
21.09	0.01\\
21.1	0.01\\
21.11	0.01\\
21.12	0.01\\
21.13	0.01\\
21.14	0.01\\
21.15	0.01\\
21.16	0.01\\
21.17	0.01\\
21.18	0.01\\
21.19	0.01\\
21.2	0.01\\
21.21	0.01\\
21.22	0.01\\
21.23	0.01\\
21.24	0.01\\
21.25	0.01\\
21.26	0.01\\
21.27	0.01\\
21.28	0.01\\
21.29	0.01\\
21.3	0.01\\
21.31	0.01\\
21.32	0.01\\
21.33	0.01\\
21.34	0.01\\
21.35	0.01\\
21.36	0.01\\
21.37	0.01\\
21.38	0.01\\
21.39	0.01\\
21.4	0.01\\
21.41	0.01\\
21.42	0.01\\
21.43	0.01\\
21.44	0.01\\
21.45	0.01\\
21.46	0.01\\
21.47	0.01\\
21.48	0.01\\
21.49	0.01\\
21.5	0.01\\
21.51	0.01\\
21.52	0.01\\
21.53	0.01\\
21.54	0.01\\
21.55	0.01\\
21.56	0.01\\
21.57	0.01\\
21.58	0.01\\
21.59	0.01\\
21.6	0.01\\
21.61	0.01\\
21.62	0.01\\
21.63	0.01\\
21.64	0.01\\
21.65	0.01\\
21.66	0.01\\
21.67	0.01\\
21.68	0.01\\
21.69	0.01\\
21.7	0.01\\
21.71	0.01\\
21.72	0.01\\
21.73	0.01\\
21.74	0.01\\
21.75	0.01\\
21.76	0.01\\
21.77	0.01\\
21.78	0.01\\
21.79	0.01\\
21.8	0.01\\
21.81	0.01\\
21.82	0.01\\
21.83	0.01\\
21.84	0.01\\
21.85	0.01\\
21.86	0.01\\
21.87	0.01\\
21.88	0.01\\
21.89	0.01\\
21.9	0.01\\
21.91	0.01\\
21.92	0.01\\
21.93	0.01\\
21.94	0.01\\
21.95	0.01\\
21.96	0.01\\
21.97	0.01\\
21.98	0.01\\
21.99	0.01\\
22	0.01\\
22.01	0.01\\
22.02	0.01\\
22.03	0.01\\
22.04	0.01\\
22.05	0.01\\
22.06	0.01\\
22.07	0.01\\
22.08	0.01\\
22.09	0.01\\
22.1	0.01\\
22.11	0.01\\
22.12	0.01\\
22.13	0.01\\
22.14	0.01\\
22.15	0.01\\
22.16	0.01\\
22.17	0.01\\
22.18	0.01\\
22.19	0.01\\
22.2	0.01\\
22.21	0.01\\
22.22	0.01\\
22.23	0.01\\
22.24	0.01\\
22.25	0.01\\
22.26	0.01\\
22.27	0.01\\
22.28	0.01\\
22.29	0.01\\
22.3	0.01\\
22.31	0.01\\
22.32	0.01\\
22.33	0.01\\
22.34	0.01\\
22.35	0.01\\
22.36	0.01\\
22.37	0.01\\
22.38	0.01\\
22.39	0.01\\
22.4	0.01\\
22.41	0.01\\
22.42	0.01\\
22.43	0.01\\
22.44	0.01\\
22.45	0.01\\
22.46	0.01\\
22.47	0.01\\
22.48	0.01\\
22.49	0.01\\
22.5	0.01\\
22.51	0.01\\
22.52	0.01\\
22.53	0.01\\
22.54	0.01\\
22.55	0.01\\
22.56	0.01\\
22.57	0.01\\
22.58	0.01\\
22.59	0.01\\
22.6	0.01\\
22.61	0.01\\
22.62	0.01\\
22.63	0.01\\
22.64	0.01\\
22.65	0.01\\
22.66	0.01\\
22.67	0.01\\
22.68	0.01\\
22.69	0.01\\
22.7	0.01\\
22.71	0.01\\
22.72	0.01\\
22.73	0.01\\
22.74	0.01\\
22.75	0.01\\
22.76	0.01\\
22.77	0.01\\
22.78	0.01\\
22.79	0.01\\
22.8	0.01\\
22.81	0.01\\
22.82	0.01\\
22.83	0.01\\
22.84	0.01\\
22.85	0.01\\
22.86	0.01\\
22.87	0.01\\
22.88	0.01\\
22.89	0.01\\
22.9	0.01\\
22.91	0.01\\
22.92	0.01\\
22.93	0.01\\
22.94	0.01\\
22.95	0.01\\
22.96	0.01\\
22.97	0.01\\
22.98	0.01\\
22.99	0.01\\
23	0.01\\
23.01	0.01\\
23.02	0.01\\
23.03	0.01\\
23.04	0.01\\
23.05	0.01\\
23.06	0.01\\
23.07	0.01\\
23.08	0.01\\
23.09	0.01\\
23.1	0.01\\
23.11	0.01\\
23.12	0.01\\
23.13	0.01\\
23.14	0.01\\
23.15	0.01\\
23.16	0.01\\
23.17	0.01\\
23.18	0.01\\
23.19	0.01\\
23.2	0.01\\
23.21	0.01\\
23.22	0.01\\
23.23	0.01\\
23.24	0.01\\
23.25	0.01\\
23.26	0.01\\
23.27	0.01\\
23.28	0.01\\
23.29	0.01\\
23.3	0.01\\
23.31	0.01\\
23.32	0.01\\
23.33	0.01\\
23.34	0.01\\
23.35	0.01\\
23.36	0.01\\
23.37	0.01\\
23.38	0.01\\
23.39	0.01\\
23.4	0.01\\
23.41	0.01\\
23.42	0.01\\
23.43	0.01\\
23.44	0.01\\
23.45	0.01\\
23.46	0.01\\
23.47	0.01\\
23.48	0.01\\
23.49	0.01\\
23.5	0.01\\
23.51	0.01\\
23.52	0.01\\
23.53	0.01\\
23.54	0.01\\
23.55	0.01\\
23.56	0.01\\
23.57	0.01\\
23.58	0.01\\
23.59	0.01\\
23.6	0.01\\
23.61	0.01\\
23.62	0.01\\
23.63	0.01\\
23.64	0.01\\
23.65	0.01\\
23.66	0.01\\
23.67	0.01\\
23.68	0.01\\
23.69	0.01\\
23.7	0.01\\
23.71	0.01\\
23.72	0.01\\
23.73	0.01\\
23.74	0.01\\
23.75	0.01\\
23.76	0.01\\
23.77	0.01\\
23.78	0.01\\
23.79	0.01\\
23.8	0.01\\
23.81	0.01\\
23.82	0.01\\
23.83	0.01\\
23.84	0.01\\
23.85	0.01\\
23.86	0.01\\
23.87	0.01\\
23.88	0.01\\
23.89	0.01\\
23.9	0.01\\
23.91	0.01\\
23.92	0.01\\
23.93	0.01\\
23.94	0.01\\
23.95	0.01\\
23.96	0.01\\
23.97	0.01\\
23.98	0.01\\
23.99	0.01\\
24	0.01\\
24.01	0.01\\
24.02	0.01\\
24.03	0.01\\
24.04	0.01\\
24.05	0.01\\
24.06	0.01\\
24.07	0.01\\
24.08	0.01\\
24.09	0.01\\
24.1	0.01\\
24.11	0.01\\
24.12	0.01\\
24.13	0.01\\
24.14	0.01\\
24.15	0.01\\
24.16	0.01\\
24.17	0.01\\
24.18	0.01\\
24.19	0.01\\
24.2	0.01\\
24.21	0.01\\
24.22	0.01\\
24.23	0.01\\
24.24	0.01\\
24.25	0.01\\
24.26	0.01\\
24.27	0.01\\
24.28	0.01\\
24.29	0.01\\
24.3	0.01\\
24.31	0.01\\
24.32	0.01\\
24.33	0.01\\
24.34	0.01\\
24.35	0.01\\
24.36	0.01\\
24.37	0.01\\
24.38	0.01\\
24.39	0.01\\
24.4	0.01\\
24.41	0.01\\
24.42	0.01\\
24.43	0.01\\
24.44	0.01\\
24.45	0.01\\
24.46	0.01\\
24.47	0.01\\
24.48	0.01\\
24.49	0.01\\
24.5	0.01\\
24.51	0.01\\
24.52	0.01\\
24.53	0.01\\
24.54	0.01\\
24.55	0.01\\
24.56	0.01\\
24.57	0.01\\
24.58	0.01\\
24.59	0.01\\
24.6	0.01\\
24.61	0.01\\
24.62	0.01\\
24.63	0.01\\
24.64	0.01\\
24.65	0.01\\
24.66	0.01\\
24.67	0.01\\
24.68	0.01\\
24.69	0.01\\
24.7	0.01\\
24.71	0.01\\
24.72	0.01\\
24.73	0.01\\
24.74	0.01\\
24.75	0.01\\
24.76	0.01\\
24.77	0.01\\
24.78	0.01\\
24.79	0.01\\
24.8	0.01\\
24.81	0.01\\
24.82	0.01\\
24.83	0.01\\
24.84	0.01\\
24.85	0.01\\
24.86	0.01\\
24.87	0.01\\
24.88	0.01\\
24.89	0.01\\
24.9	0.01\\
24.91	0.01\\
24.92	0.01\\
24.93	0.01\\
24.94	0.01\\
24.95	0.01\\
24.96	0.01\\
24.97	0.01\\
24.98	0.01\\
24.99	0.01\\
25	0.01\\
25.01	0.01\\
25.02	0.01\\
25.03	0.01\\
25.04	0.01\\
25.05	0.01\\
25.06	0.01\\
25.07	0.01\\
25.08	0.01\\
25.09	0.01\\
25.1	0.01\\
25.11	0.01\\
25.12	0.01\\
25.13	0.01\\
25.14	0.01\\
25.15	0.01\\
25.16	0.01\\
25.17	0.01\\
25.18	0.01\\
25.19	0.01\\
25.2	0.01\\
25.21	0.01\\
25.22	0.01\\
25.23	0.01\\
25.24	0.01\\
25.25	0.01\\
25.26	0.01\\
25.27	0.01\\
25.28	0.01\\
25.29	0.01\\
25.3	0.01\\
25.31	0.01\\
25.32	0.01\\
25.33	0.01\\
25.34	0.01\\
25.35	0.01\\
25.36	0.01\\
25.37	0.01\\
25.38	0.01\\
25.39	0.01\\
25.4	0.01\\
25.41	0.01\\
25.42	0.01\\
25.43	0.01\\
25.44	0.01\\
25.45	0.01\\
25.46	0.01\\
25.47	0.01\\
25.48	0.01\\
25.49	0.01\\
25.5	0.01\\
25.51	0.01\\
25.52	0.01\\
25.53	0.01\\
25.54	0.01\\
25.55	0.01\\
25.56	0.01\\
25.57	0.01\\
25.58	0.01\\
25.59	0.01\\
25.6	0.01\\
25.61	0.01\\
25.62	0.01\\
25.63	0.01\\
25.64	0.01\\
25.65	0.01\\
25.66	0.01\\
25.67	0.01\\
25.68	0.01\\
25.69	0.01\\
25.7	0.01\\
25.71	0.01\\
25.72	0.01\\
25.73	0.01\\
25.74	0.01\\
25.75	0.01\\
25.76	0.01\\
25.77	0.01\\
25.78	0.01\\
25.79	0.01\\
25.8	0.01\\
25.81	0.01\\
25.82	0.01\\
25.83	0.01\\
25.84	0.01\\
25.85	0.01\\
25.86	0.01\\
25.87	0.01\\
25.88	0.01\\
25.89	0.01\\
25.9	0.01\\
25.91	0.01\\
25.92	0.01\\
25.93	0.01\\
25.94	0.01\\
25.95	0.01\\
25.96	0.01\\
25.97	0.01\\
25.98	0.01\\
25.99	0.01\\
26	0.01\\
26.01	0.01\\
26.02	0.01\\
26.03	0.01\\
26.04	0.01\\
26.05	0.01\\
26.06	0.01\\
26.07	0.01\\
26.08	0.01\\
26.09	0.01\\
26.1	0.01\\
26.11	0.01\\
26.12	0.01\\
26.13	0.01\\
26.14	0.01\\
26.15	0.01\\
26.16	0.01\\
26.17	0.01\\
26.18	0.01\\
26.19	0.01\\
26.2	0.01\\
26.21	0.01\\
26.22	0.01\\
26.23	0.01\\
26.24	0.01\\
26.25	0.01\\
26.26	0.01\\
26.27	0.01\\
26.28	0.01\\
26.29	0.01\\
26.3	0.01\\
26.31	0.01\\
26.32	0.01\\
26.33	0.01\\
26.34	0.01\\
26.35	0.01\\
26.36	0.01\\
26.37	0.01\\
26.38	0.01\\
26.39	0.01\\
26.4	0.01\\
26.41	0.01\\
26.42	0.01\\
26.43	0.01\\
26.44	0.01\\
26.45	0.01\\
26.46	0.01\\
26.47	0.01\\
26.48	0.01\\
26.49	0.01\\
26.5	0.01\\
26.51	0.01\\
26.52	0.01\\
26.53	0.01\\
26.54	0.01\\
26.55	0.01\\
26.56	0.01\\
26.57	0.01\\
26.58	0.01\\
26.59	0.01\\
26.6	0.01\\
26.61	0.01\\
26.62	0.01\\
26.63	0.01\\
26.64	0.01\\
26.65	0.01\\
26.66	0.01\\
26.67	0.01\\
26.68	0.01\\
26.69	0.01\\
26.7	0.01\\
26.71	0.01\\
26.72	0.01\\
26.73	0.01\\
26.74	0.01\\
26.75	0.01\\
26.76	0.01\\
26.77	0.01\\
26.78	0.01\\
26.79	0.01\\
26.8	0.01\\
26.81	0.01\\
26.82	0.01\\
26.83	0.01\\
26.84	0.01\\
26.85	0.01\\
26.86	0.01\\
26.87	0.01\\
26.88	0.01\\
26.89	0.01\\
26.9	0.01\\
26.91	0.01\\
26.92	0.01\\
26.93	0.01\\
26.94	0.01\\
26.95	0.01\\
26.96	0.01\\
26.97	0.01\\
26.98	0.01\\
26.99	0.01\\
27	0.01\\
27.01	0.01\\
27.02	0.01\\
27.03	0.01\\
27.04	0.01\\
27.05	0.01\\
27.06	0.01\\
27.07	0.01\\
27.08	0.01\\
27.09	0.01\\
27.1	0.01\\
27.11	0.01\\
27.12	0.01\\
27.13	0.01\\
27.14	0.01\\
27.15	0.01\\
27.16	0.01\\
27.17	0.01\\
27.18	0.01\\
27.19	0.01\\
27.2	0.01\\
27.21	0.01\\
27.22	0.01\\
27.23	0.01\\
27.24	0.01\\
27.25	0.01\\
27.26	0.01\\
27.27	0.01\\
27.28	0.01\\
27.29	0.01\\
27.3	0.01\\
27.31	0.01\\
27.32	0.01\\
27.33	0.01\\
27.34	0.01\\
27.35	0.01\\
27.36	0.01\\
27.37	0.01\\
27.38	0.01\\
27.39	0.01\\
27.4	0.01\\
27.41	0.01\\
27.42	0.01\\
27.43	0.01\\
27.44	0.01\\
27.45	0.01\\
27.46	0.01\\
27.47	0.01\\
27.48	0.01\\
27.49	0.01\\
27.5	0.01\\
27.51	0.01\\
27.52	0.01\\
27.53	0.01\\
27.54	0.01\\
27.55	0.01\\
27.56	0.01\\
27.57	0.01\\
27.58	0.01\\
27.59	0.01\\
27.6	0.01\\
27.61	0.01\\
27.62	0.01\\
27.63	0.01\\
27.64	0.01\\
27.65	0.01\\
27.66	0.01\\
27.67	0.01\\
27.68	0.01\\
27.69	0.01\\
27.7	0.01\\
27.71	0.01\\
27.72	0.01\\
27.73	0.01\\
27.74	0.01\\
27.75	0.01\\
27.76	0.01\\
27.77	0.01\\
27.78	0.01\\
27.79	0.01\\
27.8	0.01\\
27.81	0.01\\
27.82	0.01\\
27.83	0.01\\
27.84	0.01\\
27.85	0.01\\
27.86	0.01\\
27.87	0.01\\
27.88	0.01\\
27.89	0.01\\
27.9	0.01\\
27.91	0.01\\
27.92	0.01\\
27.93	0.01\\
27.94	0.01\\
27.95	0.01\\
27.96	0.01\\
27.97	0.01\\
27.98	0.01\\
27.99	0.01\\
28	0.01\\
28.01	0.01\\
28.02	0.01\\
28.03	0.01\\
28.04	0.01\\
28.05	0.01\\
28.06	0.01\\
28.07	0.01\\
28.08	0.01\\
28.09	0.01\\
28.1	0.01\\
28.11	0.01\\
28.12	0.01\\
28.13	0.01\\
28.14	0.01\\
28.15	0.01\\
28.16	0.01\\
28.17	0.01\\
28.18	0.01\\
28.19	0.01\\
28.2	0.01\\
28.21	0.01\\
28.22	0.01\\
28.23	0.01\\
28.24	0.01\\
28.25	0.01\\
28.26	0.01\\
28.27	0.01\\
28.28	0.01\\
28.29	0.01\\
28.3	0.01\\
28.31	0.01\\
28.32	0.01\\
28.33	0.01\\
28.34	0.01\\
28.35	0.01\\
28.36	0.01\\
28.37	0.01\\
28.38	0.01\\
28.39	0.01\\
28.4	0.01\\
28.41	0.01\\
28.42	0.01\\
28.43	0.01\\
28.44	0.01\\
28.45	0.01\\
28.46	0.01\\
28.47	0.01\\
28.48	0.01\\
28.49	0.01\\
28.5	0.01\\
28.51	0.01\\
28.52	0.01\\
28.53	0.01\\
28.54	0.01\\
28.55	0.01\\
28.56	0.01\\
28.57	0.01\\
28.58	0.01\\
28.59	0.01\\
28.6	0.01\\
28.61	0.01\\
28.62	0.01\\
28.63	0.01\\
28.64	0.01\\
28.65	0.01\\
28.66	0.01\\
28.67	0.01\\
28.68	0.01\\
28.69	0.01\\
28.7	0.01\\
28.71	0.01\\
28.72	0.01\\
28.73	0.01\\
28.74	0.01\\
28.75	0.01\\
28.76	0.01\\
28.77	0.01\\
28.78	0.01\\
28.79	0.01\\
28.8	0.01\\
28.81	0.01\\
28.82	0.01\\
28.83	0.01\\
28.84	0.01\\
28.85	0.01\\
28.86	0.01\\
28.87	0.01\\
28.88	0.01\\
28.89	0.01\\
28.9	0.01\\
28.91	0.01\\
28.92	0.01\\
28.93	0.01\\
28.94	0.01\\
28.95	0.01\\
28.96	0.01\\
28.97	0.01\\
28.98	0.01\\
28.99	0.01\\
29	0.01\\
29.01	0.01\\
29.02	0.01\\
29.03	0.01\\
29.04	0.01\\
29.05	0.01\\
29.06	0.01\\
29.07	0.01\\
29.08	0.01\\
29.09	0.01\\
29.1	0.01\\
29.11	0.01\\
29.12	0.01\\
29.13	0.01\\
29.14	0.01\\
29.15	0.01\\
29.16	0.01\\
29.17	0.01\\
29.18	0.01\\
29.19	0.01\\
29.2	0.01\\
29.21	0.01\\
29.22	0.01\\
29.23	0.01\\
29.24	0.01\\
29.25	0.01\\
29.26	0.01\\
29.27	0.01\\
29.28	0.01\\
29.29	0.01\\
29.3	0.01\\
29.31	0.01\\
29.32	0.01\\
29.33	0.01\\
29.34	0.01\\
29.35	0.01\\
29.36	0.01\\
29.37	0.01\\
29.38	0.01\\
29.39	0.01\\
29.4	0.01\\
29.41	0.01\\
29.42	0.01\\
29.43	0.01\\
29.44	0.01\\
29.45	0.01\\
29.46	0.01\\
29.47	0.01\\
29.48	0.01\\
29.49	0.01\\
29.5	0.01\\
29.51	0.01\\
29.52	0.01\\
29.53	0.01\\
29.54	0.01\\
29.55	0.01\\
29.56	0.01\\
29.57	0.01\\
29.58	0.01\\
29.59	0.01\\
29.6	0.01\\
29.61	0.01\\
29.62	0.01\\
29.63	0.01\\
29.64	0.01\\
29.65	0.01\\
29.66	0.01\\
29.67	0.01\\
29.68	0.01\\
29.69	0.01\\
29.7	0.01\\
29.71	0.01\\
29.72	0.01\\
29.73	0.01\\
29.74	0.01\\
29.75	0.01\\
29.76	0.01\\
29.77	0.01\\
29.78	0.01\\
29.79	0.01\\
29.8	0.01\\
29.81	0.01\\
29.82	0.01\\
29.83	0.01\\
29.84	0.01\\
29.85	0.01\\
29.86	0.01\\
29.87	0.01\\
29.88	0.01\\
29.89	0.01\\
29.9	0.01\\
29.91	0.01\\
29.92	0.01\\
29.93	0.01\\
29.94	0.01\\
29.95	0.01\\
29.96	0.01\\
29.97	0.01\\
29.98	0.01\\
29.99	0.01\\
30	0.01\\
30.01	0.01\\
30.02	0.01\\
30.03	0.01\\
30.04	0.01\\
30.05	0.01\\
30.06	0.01\\
30.07	0.01\\
30.08	0.01\\
30.09	0.01\\
30.1	0.01\\
30.11	0.01\\
30.12	0.01\\
30.13	0.01\\
30.14	0.01\\
30.15	0.01\\
30.16	0.01\\
30.17	0.01\\
30.18	0.01\\
30.19	0.01\\
30.2	0.01\\
30.21	0.01\\
30.22	0.01\\
30.23	0.01\\
30.24	0.01\\
30.25	0.01\\
30.26	0.01\\
30.27	0.01\\
30.28	0.01\\
30.29	0.01\\
30.3	0.01\\
30.31	0.01\\
30.32	0.01\\
30.33	0.01\\
30.34	0.01\\
30.35	0.01\\
30.36	0.01\\
30.37	0.01\\
30.38	0.01\\
30.39	0.01\\
30.4	0.01\\
30.41	0.01\\
30.42	0.01\\
30.43	0.01\\
30.44	0.01\\
30.45	0.01\\
30.46	0.01\\
30.47	0.01\\
30.48	0.01\\
30.49	0.01\\
30.5	0.01\\
30.51	0.01\\
30.52	0.01\\
30.53	0.01\\
30.54	0.01\\
30.55	0.01\\
30.56	0.01\\
30.57	0.01\\
30.58	0.01\\
30.59	0.01\\
30.6	0.01\\
30.61	0.01\\
30.62	0.01\\
30.63	0.01\\
30.64	0.01\\
30.65	0.01\\
30.66	0.01\\
30.67	0.01\\
30.68	0.01\\
30.69	0.01\\
30.7	0.01\\
30.71	0.01\\
30.72	0.01\\
30.73	0.01\\
30.74	0.01\\
30.75	0.01\\
30.76	0.01\\
30.77	0.01\\
30.78	0.01\\
30.79	0.01\\
30.8	0.01\\
30.81	0.01\\
30.82	0.01\\
30.83	0.01\\
30.84	0.01\\
30.85	0.01\\
30.86	0.01\\
30.87	0.01\\
30.88	0.01\\
30.89	0.01\\
30.9	0.01\\
30.91	0.01\\
30.92	0.01\\
30.93	0.01\\
30.94	0.01\\
30.95	0.01\\
30.96	0.01\\
30.97	0.01\\
30.98	0.01\\
30.99	0.01\\
31	0.01\\
31.01	0.01\\
31.02	0.01\\
31.03	0.01\\
31.04	0.01\\
31.05	0.01\\
31.06	0.01\\
31.07	0.01\\
31.08	0.01\\
31.09	0.01\\
31.1	0.01\\
31.11	0.01\\
31.12	0.01\\
31.13	0.01\\
31.14	0.01\\
31.15	0.01\\
31.16	0.01\\
31.17	0.01\\
31.18	0.01\\
31.19	0.01\\
31.2	0.01\\
31.21	0.01\\
31.22	0.01\\
31.23	0.01\\
31.24	0.01\\
31.25	0.01\\
31.26	0.01\\
31.27	0.01\\
31.28	0.01\\
31.29	0.01\\
31.3	0.01\\
31.31	0.01\\
31.32	0.01\\
31.33	0.01\\
31.34	0.01\\
31.35	0.01\\
31.36	0.01\\
31.37	0.01\\
31.38	0.01\\
31.39	0.01\\
31.4	0.01\\
31.41	0.01\\
31.42	0.01\\
31.43	0.01\\
31.44	0.01\\
31.45	0.01\\
31.46	0.01\\
31.47	0.01\\
31.48	0.01\\
31.49	0.01\\
31.5	0.01\\
31.51	0.01\\
31.52	0.01\\
31.53	0.01\\
31.54	0.01\\
31.55	0.01\\
31.56	0.01\\
31.57	0.01\\
31.58	0.01\\
31.59	0.01\\
31.6	0.01\\
31.61	0.01\\
31.62	0.01\\
31.63	0.01\\
31.64	0.01\\
31.65	0.01\\
31.66	0.01\\
31.67	0.01\\
31.68	0.01\\
31.69	0.01\\
31.7	0.01\\
31.71	0.01\\
31.72	0.01\\
31.73	0.01\\
31.74	0.01\\
31.75	0.01\\
31.76	0.01\\
31.77	0.01\\
31.78	0.01\\
31.79	0.01\\
31.8	0.01\\
31.81	0.01\\
31.82	0.01\\
31.83	0.01\\
31.84	0.01\\
31.85	0.01\\
31.86	0.01\\
31.87	0.01\\
31.88	0.01\\
31.89	0.01\\
31.9	0.01\\
31.91	0.01\\
31.92	0.01\\
31.93	0.01\\
31.94	0.01\\
31.95	0.01\\
31.96	0.01\\
31.97	0.01\\
31.98	0.01\\
31.99	0.01\\
32	0.01\\
32.01	0.01\\
32.02	0.01\\
32.03	0.01\\
32.04	0.01\\
32.05	0.01\\
32.06	0.01\\
32.07	0.01\\
32.08	0.01\\
32.09	0.01\\
32.1	0.01\\
32.11	0.01\\
32.12	0.01\\
32.13	0.01\\
32.14	0.01\\
32.15	0.01\\
32.16	0.01\\
32.17	0.01\\
32.18	0.01\\
32.19	0.01\\
32.2	0.01\\
32.21	0.01\\
32.22	0.01\\
32.23	0.01\\
32.24	0.01\\
32.25	0.01\\
32.26	0.01\\
32.27	0.01\\
32.28	0.01\\
32.29	0.01\\
32.3	0.01\\
32.31	0.01\\
32.32	0.01\\
32.33	0.01\\
32.34	0.01\\
32.35	0.01\\
32.36	0.01\\
32.37	0.01\\
32.38	0.01\\
32.39	0.01\\
32.4	0.01\\
32.41	0.01\\
32.42	0.01\\
32.43	0.01\\
32.44	0.01\\
32.45	0.01\\
32.46	0.01\\
32.47	0.01\\
32.48	0.01\\
32.49	0.01\\
32.5	0.01\\
32.51	0.01\\
32.52	0.01\\
32.53	0.01\\
32.54	0.01\\
32.55	0.01\\
32.56	0.01\\
32.57	0.01\\
32.58	0.01\\
32.59	0.01\\
32.6	0.01\\
32.61	0.01\\
32.62	0.01\\
32.63	0.01\\
32.64	0.01\\
32.65	0.01\\
32.66	0.01\\
32.67	0.01\\
32.68	0.01\\
32.69	0.01\\
32.7	0.01\\
32.71	0.01\\
32.72	0.01\\
32.73	0.01\\
32.74	0.01\\
32.75	0.01\\
32.76	0.01\\
32.77	0.01\\
32.78	0.01\\
32.79	0.01\\
32.8	0.01\\
32.81	0.01\\
32.82	0.01\\
32.83	0.01\\
32.84	0.01\\
32.85	0.01\\
32.86	0.01\\
32.87	0.01\\
32.88	0.01\\
32.89	0.01\\
32.9	0.01\\
32.91	0.01\\
32.92	0.01\\
32.93	0.01\\
32.94	0.01\\
32.95	0.01\\
32.96	0.01\\
32.97	0.01\\
32.98	0.01\\
32.99	0.01\\
33	0.01\\
33.01	0.01\\
33.02	0.01\\
33.03	0.01\\
33.04	0.01\\
33.05	0.01\\
33.06	0.01\\
33.07	0.01\\
33.08	0.01\\
33.09	0.01\\
33.1	0.01\\
33.11	0.01\\
33.12	0.01\\
33.13	0.01\\
33.14	0.01\\
33.15	0.01\\
33.16	0.01\\
33.17	0.01\\
33.18	0.01\\
33.19	0.01\\
33.2	0.01\\
33.21	0.01\\
33.22	0.01\\
33.23	0.01\\
33.24	0.01\\
33.25	0.01\\
33.26	0.01\\
33.27	0.01\\
33.28	0.01\\
33.29	0.01\\
33.3	0.01\\
33.31	0.01\\
33.32	0.01\\
33.33	0.01\\
33.34	0.01\\
33.35	0.01\\
33.36	0.01\\
33.37	0.01\\
33.38	0.01\\
33.39	0.01\\
33.4	0.01\\
33.41	0.01\\
33.42	0.01\\
33.43	0.01\\
33.44	0.01\\
33.45	0.01\\
33.46	0.01\\
33.47	0.01\\
33.48	0.01\\
33.49	0.01\\
33.5	0.01\\
33.51	0.01\\
33.52	0.01\\
33.53	0.01\\
33.54	0.01\\
33.55	0.01\\
33.56	0.01\\
33.57	0.01\\
33.58	0.01\\
33.59	0.01\\
33.6	0.01\\
33.61	0.01\\
33.62	0.01\\
33.63	0.01\\
33.64	0.01\\
33.65	0.01\\
33.66	0.01\\
33.67	0.01\\
33.68	0.01\\
33.69	0.01\\
33.7	0.01\\
33.71	0.01\\
33.72	0.01\\
33.73	0.01\\
33.74	0.01\\
33.75	0.01\\
33.76	0.01\\
33.77	0.01\\
33.78	0.01\\
33.79	0.01\\
33.8	0.01\\
33.81	0.01\\
33.82	0.01\\
33.83	0.01\\
33.84	0.01\\
33.85	0.01\\
33.86	0.01\\
33.87	0.01\\
33.88	0.01\\
33.89	0.01\\
33.9	0.01\\
33.91	0.01\\
33.92	0.01\\
33.93	0.01\\
33.94	0.01\\
33.95	0.01\\
33.96	0.01\\
33.97	0.01\\
33.98	0.01\\
33.99	0.01\\
34	0.01\\
34.01	0.01\\
34.02	0.01\\
34.03	0.01\\
34.04	0.01\\
34.05	0.01\\
34.06	0.01\\
34.07	0.01\\
34.08	0.01\\
34.09	0.01\\
34.1	0.01\\
34.11	0.01\\
34.12	0.01\\
34.13	0.01\\
34.14	0.01\\
34.15	0.01\\
34.16	0.01\\
34.17	0.01\\
34.18	0.01\\
34.19	0.01\\
34.2	0.01\\
34.21	0.01\\
34.22	0.01\\
34.23	0.01\\
34.24	0.01\\
34.25	0.01\\
34.26	0.01\\
34.27	0.01\\
34.28	0.01\\
34.29	0.01\\
34.3	0.01\\
34.31	0.01\\
34.32	0.01\\
34.33	0.01\\
34.34	0.01\\
34.35	0.01\\
34.36	0.01\\
34.37	0.01\\
34.38	0.01\\
34.39	0.01\\
34.4	0.01\\
34.41	0.01\\
34.42	0.01\\
34.43	0.01\\
34.44	0.01\\
34.45	0.01\\
34.46	0.01\\
34.47	0.01\\
34.48	0.01\\
34.49	0.01\\
34.5	0.01\\
34.51	0.01\\
34.52	0.01\\
34.53	0.01\\
34.54	0.01\\
34.55	0.01\\
34.56	0.01\\
34.57	0.01\\
34.58	0.01\\
34.59	0.01\\
34.6	0.01\\
34.61	0.01\\
34.62	0.01\\
34.63	0.01\\
34.64	0.01\\
34.65	0.01\\
34.66	0.01\\
34.67	0.01\\
34.68	0.01\\
34.69	0.01\\
34.7	0.01\\
34.71	0.01\\
34.72	0.01\\
34.73	0.01\\
34.74	0.01\\
34.75	0.01\\
34.76	0.01\\
34.77	0.01\\
34.78	0.01\\
34.79	0.01\\
34.8	0.01\\
34.81	0.01\\
34.82	0.01\\
34.83	0.01\\
34.84	0.01\\
34.85	0.01\\
34.86	0.01\\
34.87	0.01\\
34.88	0.01\\
34.89	0.01\\
34.9	0.01\\
34.91	0.01\\
34.92	0.01\\
34.93	0.01\\
34.94	0.01\\
34.95	0.01\\
34.96	0.01\\
34.97	0.01\\
34.98	0.01\\
34.99	0.01\\
35	0.01\\
35.01	0.01\\
35.02	0.01\\
35.03	0.01\\
35.04	0.01\\
35.05	0.01\\
35.06	0.01\\
35.07	0.01\\
35.08	0.01\\
35.09	0.01\\
35.1	0.01\\
35.11	0.01\\
35.12	0.01\\
35.13	0.01\\
35.14	0.01\\
35.15	0.01\\
35.16	0.01\\
35.17	0.01\\
35.18	0.01\\
35.19	0.01\\
35.2	0.01\\
35.21	0.01\\
35.22	0.01\\
35.23	0.01\\
35.24	0.01\\
35.25	0.01\\
35.26	0.01\\
35.27	0.01\\
35.28	0.01\\
35.29	0.01\\
35.3	0.01\\
35.31	0.01\\
35.32	0.01\\
35.33	0.01\\
35.34	0.01\\
35.35	0.01\\
35.36	0.01\\
35.37	0.01\\
35.38	0.01\\
35.39	0.01\\
35.4	0.01\\
35.41	0.01\\
35.42	0.01\\
35.43	0.01\\
35.44	0.01\\
35.45	0.01\\
35.46	0.01\\
35.47	0.01\\
35.48	0.01\\
35.49	0.01\\
35.5	0.01\\
35.51	0.01\\
35.52	0.01\\
35.53	0.01\\
35.54	0.01\\
35.55	0.01\\
35.56	0.01\\
35.57	0.01\\
35.58	0.01\\
35.59	0.01\\
35.6	0.01\\
35.61	0.01\\
35.62	0.01\\
35.63	0.01\\
35.64	0.01\\
35.65	0.01\\
35.66	0.01\\
35.67	0.01\\
35.68	0.01\\
35.69	0.01\\
35.7	0.01\\
35.71	0.01\\
35.72	0.01\\
35.73	0.01\\
35.74	0.01\\
35.75	0.01\\
35.76	0.01\\
35.77	0.01\\
35.78	0.01\\
35.79	0.01\\
35.8	0.01\\
35.81	0.01\\
35.82	0.01\\
35.83	0.01\\
35.84	0.01\\
35.85	0.01\\
35.86	0.01\\
35.87	0.01\\
35.88	0.01\\
35.89	0.01\\
35.9	0.01\\
35.91	0.01\\
35.92	0.01\\
35.93	0.01\\
35.94	0.01\\
35.95	0.01\\
35.96	0.01\\
35.97	0.01\\
35.98	0.01\\
35.99	0.01\\
36	0.01\\
36.01	0.01\\
36.02	0.01\\
36.03	0.01\\
36.04	0.01\\
36.05	0.01\\
36.06	0.01\\
36.07	0.01\\
36.08	0.01\\
36.09	0.01\\
36.1	0.01\\
36.11	0.01\\
36.12	0.01\\
36.13	0.01\\
36.14	0.01\\
36.15	0.01\\
36.16	0.01\\
36.17	0.01\\
36.18	0.01\\
36.19	0.01\\
36.2	0.01\\
36.21	0.01\\
36.22	0.01\\
36.23	0.01\\
36.24	0.01\\
36.25	0.01\\
36.26	0.01\\
36.27	0.01\\
36.28	0.01\\
36.29	0.01\\
36.3	0.01\\
36.31	0.01\\
36.32	0.01\\
36.33	0.01\\
36.34	0.01\\
36.35	0.01\\
36.36	0.01\\
36.37	0.01\\
36.38	0.01\\
36.39	0.01\\
36.4	0.01\\
36.41	0.01\\
36.42	0.01\\
36.43	0.01\\
36.44	0.01\\
36.45	0.01\\
36.46	0.01\\
36.47	0.01\\
36.48	0.01\\
36.49	0.01\\
36.5	0.01\\
36.51	0.01\\
36.52	0.01\\
36.53	0.01\\
36.54	0.01\\
36.55	0.01\\
36.56	0.01\\
36.57	0.01\\
36.58	0.01\\
36.59	0.01\\
36.6	0.01\\
36.61	0.01\\
36.62	0.01\\
36.63	0.01\\
36.64	0.01\\
36.65	0.01\\
36.66	0.01\\
36.67	0.01\\
36.68	0.01\\
36.69	0.01\\
36.7	0.01\\
36.71	0.01\\
36.72	0.01\\
36.73	0.01\\
36.74	0.01\\
36.75	0.01\\
36.76	0.01\\
36.77	0.01\\
36.78	0.01\\
36.79	0.01\\
36.8	0.01\\
36.81	0.01\\
36.82	0.01\\
36.83	0.01\\
36.84	0.01\\
36.85	0.01\\
36.86	0.01\\
36.87	0.01\\
36.88	0.01\\
36.89	0.01\\
36.9	0.01\\
36.91	0.01\\
36.92	0.01\\
36.93	0.01\\
36.94	0.01\\
36.95	0.01\\
36.96	0.01\\
36.97	0.01\\
36.98	0.01\\
36.99	0.01\\
37	0.01\\
37.01	0.01\\
37.02	0.01\\
37.03	0.01\\
37.04	0.01\\
37.05	0.01\\
37.06	0.01\\
37.07	0.01\\
37.08	0.01\\
37.09	0.01\\
37.1	0.01\\
37.11	0.01\\
37.12	0.01\\
37.13	0.01\\
37.14	0.01\\
37.15	0.01\\
37.16	0.01\\
37.17	0.01\\
37.18	0.01\\
37.19	0.01\\
37.2	0.01\\
37.21	0.01\\
37.22	0.01\\
37.23	0.01\\
37.24	0.01\\
37.25	0.01\\
37.26	0.01\\
37.27	0.01\\
37.28	0.01\\
37.29	0.01\\
37.3	0.01\\
37.31	0.01\\
37.32	0.01\\
37.33	0.01\\
37.34	0.01\\
37.35	0.01\\
37.36	0.01\\
37.37	0.01\\
37.38	0.01\\
37.39	0.01\\
37.4	0.01\\
37.41	0.01\\
37.42	0.01\\
37.43	0.01\\
37.44	0.01\\
37.45	0.01\\
37.46	0.01\\
37.47	0.01\\
37.48	0.01\\
37.49	0.01\\
37.5	0.01\\
37.51	0.01\\
37.52	0.01\\
37.53	0.01\\
37.54	0.01\\
37.55	0.01\\
37.56	0.01\\
37.57	0.01\\
37.58	0.01\\
37.59	0.01\\
37.6	0.01\\
37.61	0.01\\
37.62	0.01\\
37.63	0.01\\
37.64	0.01\\
37.65	0.01\\
37.66	0.01\\
37.67	0.01\\
37.68	0.01\\
37.69	0.01\\
37.7	0.01\\
37.71	0.01\\
37.72	0.01\\
37.73	0.01\\
37.74	0.01\\
37.75	0.01\\
37.76	0.01\\
37.77	0.01\\
37.78	0.01\\
37.79	0.01\\
37.8	0.01\\
37.81	0.01\\
37.82	0.01\\
37.83	0.01\\
37.84	0.01\\
37.85	0.01\\
37.86	0.01\\
37.87	0.01\\
37.88	0.01\\
37.89	0.01\\
37.9	0.01\\
37.91	0.01\\
37.92	0.01\\
37.93	0.01\\
37.94	0.01\\
37.95	0.01\\
37.96	0.01\\
37.97	0.01\\
37.98	0.01\\
37.99	0.01\\
38	0.01\\
38.01	0.01\\
38.02	0.01\\
38.03	0.01\\
38.04	0.01\\
38.05	0.01\\
38.06	0.01\\
38.07	0.01\\
38.08	0.01\\
38.09	0.01\\
38.1	0.01\\
38.11	0.01\\
38.12	0.01\\
38.13	0.01\\
38.14	0.01\\
38.15	0.01\\
38.16	0.01\\
38.17	0.01\\
38.18	0.01\\
38.19	0.01\\
38.2	0.01\\
38.21	0.01\\
38.22	0.01\\
38.23	0.01\\
38.24	0.01\\
38.25	0.01\\
38.26	0.01\\
38.27	0.01\\
38.28	0.01\\
38.29	0.01\\
38.3	0.01\\
38.31	0.01\\
38.32	0.01\\
38.33	0.01\\
38.34	0.01\\
38.35	0.01\\
38.36	0.01\\
38.37	0.01\\
38.38	0.01\\
38.39	0.01\\
38.4	0.01\\
38.41	0.01\\
38.42	0.01\\
38.43	0.01\\
38.44	0.01\\
38.45	0.01\\
38.46	0.01\\
38.47	0.01\\
38.48	0.01\\
38.49	0.01\\
38.5	0.01\\
38.51	0.01\\
38.52	0.01\\
38.53	0.01\\
38.54	0.01\\
38.55	0.01\\
38.56	0.01\\
38.57	0.01\\
38.58	0.01\\
38.59	0.01\\
38.6	0.01\\
38.61	0.01\\
38.62	0.01\\
38.63	0.01\\
38.64	0.01\\
38.65	0.01\\
38.66	0.01\\
38.67	0.01\\
38.68	0.01\\
38.69	0.01\\
38.7	0.01\\
38.71	0.01\\
38.72	0.01\\
38.73	0.01\\
38.74	0.01\\
38.75	0.01\\
38.76	0.01\\
38.77	0.01\\
38.78	0.01\\
38.79	0.01\\
38.8	0.01\\
38.81	0.01\\
38.82	0.01\\
38.83	0.01\\
38.84	0.01\\
38.85	0.01\\
38.86	0.01\\
38.87	0.01\\
38.88	0.01\\
38.89	0.01\\
38.9	0.01\\
38.91	0.01\\
38.92	0.01\\
38.93	0.01\\
38.94	0.01\\
38.95	0.01\\
38.96	0.01\\
38.97	0.01\\
38.98	0.01\\
38.99	0.01\\
39	0.01\\
39.01	0.01\\
39.02	0.01\\
39.03	0.01\\
39.04	0.01\\
39.05	0.01\\
39.06	0.01\\
39.07	0.01\\
39.08	0.01\\
39.09	0.01\\
39.1	0.01\\
39.11	0.01\\
39.12	0.01\\
39.13	0.01\\
39.14	0.01\\
39.15	0.01\\
39.16	0.01\\
39.17	0.01\\
39.18	0.01\\
39.19	0.01\\
39.2	0.01\\
39.21	0.01\\
39.22	0.01\\
39.23	0.01\\
39.24	0.01\\
39.25	0.01\\
39.26	0.01\\
39.27	0.01\\
39.28	0.01\\
39.29	0.01\\
39.3	0.01\\
39.31	0.01\\
39.32	0.01\\
39.33	0.01\\
39.34	0.01\\
39.35	0.01\\
39.36	0.01\\
39.37	0.01\\
39.38	0.01\\
39.39	0.01\\
39.4	0.01\\
39.41	0.01\\
39.42	0.01\\
39.43	0.01\\
39.44	0.01\\
39.45	0.01\\
39.46	0.01\\
39.47	0.01\\
39.48	0.01\\
39.49	0.01\\
39.5	0.01\\
39.51	0.01\\
39.52	0.01\\
39.53	0.01\\
39.54	0.01\\
39.55	0.01\\
39.56	0.01\\
39.57	0.01\\
39.58	0.01\\
39.59	0.01\\
39.6	0.01\\
39.61	0.01\\
39.62	0.01\\
39.63	0.01\\
39.64	0.01\\
39.65	0.01\\
39.66	0.01\\
39.67	0.01\\
39.68	0.01\\
39.69	0.01\\
39.7	0.01\\
39.71	0.01\\
39.72	0.01\\
39.73	0.01\\
39.74	0.01\\
39.75	0.01\\
39.76	0.01\\
39.77	0.01\\
39.78	0.01\\
39.79	0.01\\
39.8	0.01\\
39.81	0.01\\
39.82	0.01\\
39.83	0.01\\
39.84	0.01\\
39.85	0.01\\
39.86	0.01\\
39.87	0.01\\
39.88	0.01\\
39.89	0.01\\
39.9	0.01\\
39.91	0.01\\
39.92	0.01\\
39.93	0.01\\
39.94	0.01\\
39.95	0.01\\
39.96	0.01\\
39.97	0.01\\
39.98	0.01\\
39.99	0.01\\
40	0.01\\
40.01	0.01\\
};
\addplot [color=red,dashed,forget plot]
  table[row sep=crcr]{%
40.01	0.01\\
40.02	0.01\\
40.03	0.01\\
40.04	0.01\\
40.05	0.01\\
40.06	0.01\\
40.07	0.01\\
40.08	0.01\\
40.09	0.01\\
40.1	0.01\\
40.11	0.01\\
40.12	0.01\\
40.13	0.01\\
40.14	0.01\\
40.15	0.01\\
40.16	0.01\\
40.17	0.01\\
40.18	0.01\\
40.19	0.01\\
40.2	0.01\\
40.21	0.01\\
40.22	0.01\\
40.23	0.01\\
40.24	0.01\\
40.25	0.01\\
40.26	0.01\\
40.27	0.01\\
40.28	0.01\\
40.29	0.01\\
40.3	0.01\\
40.31	0.01\\
40.32	0.01\\
40.33	0.01\\
40.34	0.01\\
40.35	0.01\\
40.36	0.01\\
40.37	0.01\\
40.38	0.01\\
40.39	0.01\\
40.4	0.01\\
40.41	0.01\\
40.42	0.01\\
40.43	0.01\\
40.44	0.01\\
40.45	0.01\\
40.46	0.01\\
40.47	0.01\\
40.48	0.01\\
40.49	0.01\\
40.5	0.01\\
40.51	0.01\\
40.52	0.01\\
40.53	0.01\\
40.54	0.01\\
40.55	0.01\\
40.56	0.01\\
40.57	0.01\\
40.58	0.01\\
40.59	0.01\\
40.6	0.01\\
40.61	0.01\\
40.62	0.01\\
40.63	0.01\\
40.64	0.01\\
40.65	0.01\\
40.66	0.01\\
40.67	0.01\\
40.68	0.01\\
40.69	0.01\\
40.7	0.01\\
40.71	0.01\\
40.72	0.01\\
40.73	0.01\\
40.74	0.01\\
40.75	0.01\\
40.76	0.01\\
40.77	0.01\\
40.78	0.01\\
40.79	0.01\\
40.8	0.01\\
40.81	0.01\\
40.82	0.01\\
40.83	0.01\\
40.84	0.01\\
40.85	0.01\\
40.86	0.01\\
40.87	0.01\\
40.88	0.01\\
40.89	0.01\\
40.9	0.01\\
40.91	0.01\\
40.92	0.01\\
40.93	0.01\\
40.94	0.01\\
40.95	0.01\\
40.96	0.01\\
40.97	0.01\\
40.98	0.01\\
40.99	0.01\\
41	0.01\\
41.01	0.01\\
41.02	0.01\\
41.03	0.01\\
41.04	0.01\\
41.05	0.01\\
41.06	0.01\\
41.07	0.01\\
41.08	0.01\\
41.09	0.01\\
41.1	0.01\\
41.11	0.01\\
41.12	0.01\\
41.13	0.01\\
41.14	0.01\\
41.15	0.01\\
41.16	0.01\\
41.17	0.01\\
41.18	0.01\\
41.19	0.01\\
41.2	0.01\\
41.21	0.01\\
41.22	0.01\\
41.23	0.01\\
41.24	0.01\\
41.25	0.01\\
41.26	0.01\\
41.27	0.01\\
41.28	0.01\\
41.29	0.01\\
41.3	0.01\\
41.31	0.01\\
41.32	0.01\\
41.33	0.01\\
41.34	0.01\\
41.35	0.01\\
41.36	0.01\\
41.37	0.01\\
41.38	0.01\\
41.39	0.01\\
41.4	0.01\\
41.41	0.01\\
41.42	0.01\\
41.43	0.01\\
41.44	0.01\\
41.45	0.01\\
41.46	0.01\\
41.47	0.01\\
41.48	0.01\\
41.49	0.01\\
41.5	0.01\\
41.51	0.01\\
41.52	0.01\\
41.53	0.01\\
41.54	0.01\\
41.55	0.01\\
41.56	0.01\\
41.57	0.01\\
41.58	0.01\\
41.59	0.01\\
41.6	0.01\\
41.61	0.01\\
41.62	0.01\\
41.63	0.01\\
41.64	0.01\\
41.65	0.01\\
41.66	0.01\\
41.67	0.01\\
41.68	0.01\\
41.69	0.01\\
41.7	0.01\\
41.71	0.01\\
41.72	0.01\\
41.73	0.01\\
41.74	0.01\\
41.75	0.01\\
41.76	0.01\\
41.77	0.01\\
41.78	0.01\\
41.79	0.01\\
41.8	0.01\\
41.81	0.01\\
41.82	0.01\\
41.83	0.01\\
41.84	0.01\\
41.85	0.01\\
41.86	0.01\\
41.87	0.01\\
41.88	0.01\\
41.89	0.01\\
41.9	0.01\\
41.91	0.01\\
41.92	0.01\\
41.93	0.01\\
41.94	0.01\\
41.95	0.01\\
41.96	0.01\\
41.97	0.01\\
41.98	0.01\\
41.99	0.01\\
42	0.01\\
42.01	0.01\\
42.02	0.01\\
42.03	0.01\\
42.04	0.01\\
42.05	0.01\\
42.06	0.01\\
42.07	0.01\\
42.08	0.01\\
42.09	0.01\\
42.1	0.01\\
42.11	0.01\\
42.12	0.01\\
42.13	0.01\\
42.14	0.01\\
42.15	0.01\\
42.16	0.01\\
42.17	0.01\\
42.18	0.01\\
42.19	0.01\\
42.2	0.01\\
42.21	0.01\\
42.22	0.01\\
42.23	0.01\\
42.24	0.01\\
42.25	0.01\\
42.26	0.01\\
42.27	0.01\\
42.28	0.01\\
42.29	0.01\\
42.3	0.01\\
42.31	0.01\\
42.32	0.01\\
42.33	0.01\\
42.34	0.01\\
42.35	0.01\\
42.36	0.01\\
42.37	0.01\\
42.38	0.01\\
42.39	0.01\\
42.4	0.01\\
42.41	0.01\\
42.42	0.01\\
42.43	0.01\\
42.44	0.01\\
42.45	0.01\\
42.46	0.01\\
42.47	0.01\\
42.48	0.01\\
42.49	0.01\\
42.5	0.01\\
42.51	0.01\\
42.52	0.01\\
42.53	0.01\\
42.54	0.01\\
42.55	0.01\\
42.56	0.01\\
42.57	0.01\\
42.58	0.01\\
42.59	0.01\\
42.6	0.01\\
42.61	0.01\\
42.62	0.01\\
42.63	0.01\\
42.64	0.01\\
42.65	0.01\\
42.66	0.01\\
42.67	0.01\\
42.68	0.01\\
42.69	0.01\\
42.7	0.01\\
42.71	0.01\\
42.72	0.01\\
42.73	0.01\\
42.74	0.01\\
42.75	0.01\\
42.76	0.01\\
42.77	0.01\\
42.78	0.01\\
42.79	0.01\\
42.8	0.01\\
42.81	0.01\\
42.82	0.01\\
42.83	0.01\\
42.84	0.01\\
42.85	0.01\\
42.86	0.01\\
42.87	0.01\\
42.88	0.01\\
42.89	0.01\\
42.9	0.01\\
42.91	0.01\\
42.92	0.01\\
42.93	0.01\\
42.94	0.01\\
42.95	0.01\\
42.96	0.01\\
42.97	0.01\\
42.98	0.01\\
42.99	0.01\\
43	0.01\\
43.01	0.01\\
43.02	0.01\\
43.03	0.01\\
43.04	0.01\\
43.05	0.01\\
43.06	0.01\\
43.07	0.01\\
43.08	0.01\\
43.09	0.01\\
43.1	0.01\\
43.11	0.01\\
43.12	0.01\\
43.13	0.01\\
43.14	0.01\\
43.15	0.01\\
43.16	0.01\\
43.17	0.01\\
43.18	0.01\\
43.19	0.01\\
43.2	0.01\\
43.21	0.01\\
43.22	0.01\\
43.23	0.01\\
43.24	0.01\\
43.25	0.01\\
43.26	0.01\\
43.27	0.01\\
43.28	0.01\\
43.29	0.01\\
43.3	0.01\\
43.31	0.01\\
43.32	0.01\\
43.33	0.01\\
43.34	0.01\\
43.35	0.01\\
43.36	0.01\\
43.37	0.01\\
43.38	0.01\\
43.39	0.01\\
43.4	0.01\\
43.41	0.01\\
43.42	0.01\\
43.43	0.01\\
43.44	0.01\\
43.45	0.01\\
43.46	0.01\\
43.47	0.01\\
43.48	0.01\\
43.49	0.01\\
43.5	0.01\\
43.51	0.01\\
43.52	0.01\\
43.53	0.01\\
43.54	0.01\\
43.55	0.01\\
43.56	0.01\\
43.57	0.01\\
43.58	0.01\\
43.59	0.01\\
43.6	0.01\\
43.61	0.01\\
43.62	0.01\\
43.63	0.01\\
43.64	0.01\\
43.65	0.01\\
43.66	0.01\\
43.67	0.01\\
43.68	0.01\\
43.69	0.01\\
43.7	0.01\\
43.71	0.01\\
43.72	0.01\\
43.73	0.01\\
43.74	0.01\\
43.75	0.01\\
43.76	0.01\\
43.77	0.01\\
43.78	0.01\\
43.79	0.01\\
43.8	0.01\\
43.81	0.01\\
43.82	0.01\\
43.83	0.01\\
43.84	0.01\\
43.85	0.01\\
43.86	0.01\\
43.87	0.01\\
43.88	0.01\\
43.89	0.01\\
43.9	0.01\\
43.91	0.01\\
43.92	0.01\\
43.93	0.01\\
43.94	0.01\\
43.95	0.01\\
43.96	0.01\\
43.97	0.01\\
43.98	0.01\\
43.99	0.01\\
44	0.01\\
44.01	0.01\\
44.02	0.01\\
44.03	0.01\\
44.04	0.01\\
44.05	0.01\\
44.06	0.01\\
44.07	0.01\\
44.08	0.01\\
44.09	0.01\\
44.1	0.01\\
44.11	0.01\\
44.12	0.01\\
44.13	0.01\\
44.14	0.01\\
44.15	0.01\\
44.16	0.01\\
44.17	0.01\\
44.18	0.01\\
44.19	0.01\\
44.2	0.01\\
44.21	0.01\\
44.22	0.01\\
44.23	0.01\\
44.24	0.01\\
44.25	0.01\\
44.26	0.01\\
44.27	0.01\\
44.28	0.01\\
44.29	0.01\\
44.3	0.01\\
44.31	0.01\\
44.32	0.01\\
44.33	0.01\\
44.34	0.01\\
44.35	0.01\\
44.36	0.01\\
44.37	0.01\\
44.38	0.01\\
44.39	0.01\\
44.4	0.01\\
44.41	0.01\\
44.42	0.01\\
44.43	0.01\\
44.44	0.01\\
44.45	0.01\\
44.46	0.01\\
44.47	0.01\\
44.48	0.01\\
44.49	0.01\\
44.5	0.01\\
44.51	0.01\\
44.52	0.01\\
44.53	0.01\\
44.54	0.01\\
44.55	0.01\\
44.56	0.01\\
44.57	0.01\\
44.58	0.01\\
44.59	0.01\\
44.6	0.01\\
44.61	0.01\\
44.62	0.01\\
44.63	0.01\\
44.64	0.01\\
44.65	0.01\\
44.66	0.01\\
44.67	0.01\\
44.68	0.01\\
44.69	0.01\\
44.7	0.01\\
44.71	0.01\\
44.72	0.01\\
44.73	0.01\\
44.74	0.01\\
44.75	0.01\\
44.76	0.01\\
44.77	0.01\\
44.78	0.01\\
44.79	0.01\\
44.8	0.01\\
44.81	0.01\\
44.82	0.01\\
44.83	0.01\\
44.84	0.01\\
44.85	0.01\\
44.86	0.01\\
44.87	0.01\\
44.88	0.01\\
44.89	0.01\\
44.9	0.01\\
44.91	0.01\\
44.92	0.01\\
44.93	0.01\\
44.94	0.01\\
44.95	0.01\\
44.96	0.01\\
44.97	0.01\\
44.98	0.01\\
44.99	0.01\\
45	0.01\\
45.01	0.01\\
45.02	0.01\\
45.03	0.01\\
45.04	0.01\\
45.05	0.01\\
45.06	0.01\\
45.07	0.01\\
45.08	0.01\\
45.09	0.01\\
45.1	0.01\\
45.11	0.01\\
45.12	0.01\\
45.13	0.01\\
45.14	0.01\\
45.15	0.01\\
45.16	0.01\\
45.17	0.01\\
45.18	0.01\\
45.19	0.01\\
45.2	0.01\\
45.21	0.01\\
45.22	0.01\\
45.23	0.01\\
45.24	0.01\\
45.25	0.01\\
45.26	0.01\\
45.27	0.01\\
45.28	0.01\\
45.29	0.01\\
45.3	0.01\\
45.31	0.01\\
45.32	0.01\\
45.33	0.01\\
45.34	0.01\\
45.35	0.01\\
45.36	0.01\\
45.37	0.01\\
45.38	0.01\\
45.39	0.01\\
45.4	0.01\\
45.41	0.01\\
45.42	0.01\\
45.43	0.01\\
45.44	0.01\\
45.45	0.01\\
45.46	0.01\\
45.47	0.01\\
45.48	0.01\\
45.49	0.01\\
45.5	0.01\\
45.51	0.01\\
45.52	0.01\\
45.53	0.01\\
45.54	0.01\\
45.55	0.01\\
45.56	0.01\\
45.57	0.01\\
45.58	0.01\\
45.59	0.01\\
45.6	0.01\\
45.61	0.01\\
45.62	0.01\\
45.63	0.01\\
45.64	0.01\\
45.65	0.01\\
45.66	0.01\\
45.67	0.01\\
45.68	0.01\\
45.69	0.01\\
45.7	0.01\\
45.71	0.01\\
45.72	0.01\\
45.73	0.01\\
45.74	0.01\\
45.75	0.01\\
45.76	0.01\\
45.77	0.01\\
45.78	0.01\\
45.79	0.01\\
45.8	0.01\\
45.81	0.01\\
45.82	0.01\\
45.83	0.01\\
45.84	0.01\\
45.85	0.01\\
45.86	0.01\\
45.87	0.01\\
45.88	0.01\\
45.89	0.01\\
45.9	0.01\\
45.91	0.01\\
45.92	0.01\\
45.93	0.01\\
45.94	0.01\\
45.95	0.01\\
45.96	0.01\\
45.97	0.01\\
45.98	0.01\\
45.99	0.01\\
46	0.01\\
46.01	0.01\\
46.02	0.01\\
46.03	0.01\\
46.04	0.01\\
46.05	0.01\\
46.06	0.01\\
46.07	0.01\\
46.08	0.01\\
46.09	0.01\\
46.1	0.01\\
46.11	0.01\\
46.12	0.01\\
46.13	0.01\\
46.14	0.01\\
46.15	0.01\\
46.16	0.01\\
46.17	0.01\\
46.18	0.01\\
46.19	0.01\\
46.2	0.01\\
46.21	0.01\\
46.22	0.01\\
46.23	0.01\\
46.24	0.01\\
46.25	0.01\\
46.26	0.01\\
46.27	0.01\\
46.28	0.01\\
46.29	0.01\\
46.3	0.01\\
46.31	0.01\\
46.32	0.01\\
46.33	0.01\\
46.34	0.01\\
46.35	0.01\\
46.36	0.01\\
46.37	0.01\\
46.38	0.01\\
46.39	0.01\\
46.4	0.01\\
46.41	0.01\\
46.42	0.01\\
46.43	0.01\\
46.44	0.01\\
46.45	0.01\\
46.46	0.01\\
46.47	0.01\\
46.48	0.01\\
46.49	0.01\\
46.5	0.01\\
46.51	0.01\\
46.52	0.01\\
46.53	0.01\\
46.54	0.01\\
46.55	0.01\\
46.56	0.01\\
46.57	0.01\\
46.58	0.01\\
46.59	0.01\\
46.6	0.01\\
46.61	0.01\\
46.62	0.01\\
46.63	0.01\\
46.64	0.01\\
46.65	0.01\\
46.66	0.01\\
46.67	0.01\\
46.68	0.01\\
46.69	0.01\\
46.7	0.01\\
46.71	0.01\\
46.72	0.01\\
46.73	0.01\\
46.74	0.01\\
46.75	0.01\\
46.76	0.01\\
46.77	0.01\\
46.78	0.01\\
46.79	0.01\\
46.8	0.01\\
46.81	0.01\\
46.82	0.01\\
46.83	0.01\\
46.84	0.01\\
46.85	0.01\\
46.86	0.01\\
46.87	0.01\\
46.88	0.01\\
46.89	0.01\\
46.9	0.01\\
46.91	0.01\\
46.92	0.01\\
46.93	0.01\\
46.94	0.01\\
46.95	0.01\\
46.96	0.01\\
46.97	0.01\\
46.98	0.01\\
46.99	0.01\\
47	0.01\\
47.01	0.01\\
47.02	0.01\\
47.03	0.01\\
47.04	0.01\\
47.05	0.01\\
47.06	0.01\\
47.07	0.01\\
47.08	0.01\\
47.09	0.01\\
47.1	0.01\\
47.11	0.01\\
47.12	0.01\\
47.13	0.01\\
47.14	0.01\\
47.15	0.01\\
47.16	0.01\\
47.17	0.01\\
47.18	0.01\\
47.19	0.01\\
47.2	0.01\\
47.21	0.01\\
47.22	0.01\\
47.23	0.01\\
47.24	0.01\\
47.25	0.01\\
47.26	0.01\\
47.27	0.01\\
47.28	0.01\\
47.29	0.01\\
47.3	0.01\\
47.31	0.01\\
47.32	0.01\\
47.33	0.01\\
47.34	0.01\\
47.35	0.01\\
47.36	0.01\\
47.37	0.01\\
47.38	0.01\\
47.39	0.01\\
47.4	0.01\\
47.41	0.01\\
47.42	0.01\\
47.43	0.01\\
47.44	0.01\\
47.45	0.01\\
47.46	0.01\\
47.47	0.01\\
47.48	0.01\\
47.49	0.01\\
47.5	0.01\\
47.51	0.01\\
47.52	0.01\\
47.53	0.01\\
47.54	0.01\\
47.55	0.01\\
47.56	0.01\\
47.57	0.01\\
47.58	0.01\\
47.59	0.01\\
47.6	0.01\\
47.61	0.01\\
47.62	0.01\\
47.63	0.01\\
47.64	0.01\\
47.65	0.01\\
47.66	0.01\\
47.67	0.01\\
47.68	0.01\\
47.69	0.01\\
47.7	0.01\\
47.71	0.01\\
47.72	0.01\\
47.73	0.01\\
47.74	0.01\\
47.75	0.01\\
47.76	0.01\\
47.77	0.01\\
47.78	0.01\\
47.79	0.01\\
47.8	0.01\\
47.81	0.01\\
47.82	0.01\\
47.83	0.01\\
47.84	0.01\\
47.85	0.01\\
47.86	0.01\\
47.87	0.01\\
47.88	0.01\\
47.89	0.01\\
47.9	0.01\\
47.91	0.01\\
47.92	0.01\\
47.93	0.01\\
47.94	0.01\\
47.95	0.01\\
47.96	0.01\\
47.97	0.01\\
47.98	0.01\\
47.99	0.01\\
48	0.01\\
48.01	0.01\\
48.02	0.01\\
48.03	0.01\\
48.04	0.01\\
48.05	0.01\\
48.06	0.01\\
48.07	0.01\\
48.08	0.01\\
48.09	0.01\\
48.1	0.01\\
48.11	0.01\\
48.12	0.01\\
48.13	0.01\\
48.14	0.01\\
48.15	0.01\\
48.16	0.01\\
48.17	0.01\\
48.18	0.01\\
48.19	0.01\\
48.2	0.01\\
48.21	0.01\\
48.22	0.01\\
48.23	0.01\\
48.24	0.01\\
48.25	0.01\\
48.26	0.01\\
48.27	0.01\\
48.28	0.01\\
48.29	0.01\\
48.3	0.01\\
48.31	0.01\\
48.32	0.01\\
48.33	0.01\\
48.34	0.01\\
48.35	0.01\\
48.36	0.01\\
48.37	0.01\\
48.38	0.01\\
48.39	0.01\\
48.4	0.01\\
48.41	0.01\\
48.42	0.01\\
48.43	0.01\\
48.44	0.01\\
48.45	0.01\\
48.46	0.01\\
48.47	0.01\\
48.48	0.01\\
48.49	0.01\\
48.5	0.01\\
48.51	0.01\\
48.52	0.01\\
48.53	0.01\\
48.54	0.01\\
48.55	0.01\\
48.56	0.01\\
48.57	0.01\\
48.58	0.01\\
48.59	0.01\\
48.6	0.01\\
48.61	0.01\\
48.62	0.01\\
48.63	0.01\\
48.64	0.01\\
48.65	0.01\\
48.66	0.01\\
48.67	0.01\\
48.68	0.01\\
48.69	0.01\\
48.7	0.01\\
48.71	0.01\\
48.72	0.01\\
48.73	0.01\\
48.74	0.01\\
48.75	0.01\\
48.76	0.01\\
48.77	0.01\\
48.78	0.01\\
48.79	0.01\\
48.8	0.01\\
48.81	0.01\\
48.82	0.01\\
48.83	0.01\\
48.84	0.01\\
48.85	0.01\\
48.86	0.01\\
48.87	0.01\\
48.88	0.01\\
48.89	0.01\\
48.9	0.01\\
48.91	0.01\\
48.92	0.01\\
48.93	0.01\\
48.94	0.01\\
48.95	0.01\\
48.96	0.01\\
48.97	0.01\\
48.98	0.01\\
48.99	0.01\\
49	0.01\\
49.01	0.01\\
49.02	0.01\\
49.03	0.01\\
49.04	0.01\\
49.05	0.01\\
49.06	0.01\\
49.07	0.01\\
49.08	0.01\\
49.09	0.01\\
49.1	0.01\\
49.11	0.01\\
49.12	0.01\\
49.13	0.01\\
49.14	0.01\\
49.15	0.01\\
49.16	0.01\\
49.17	0.01\\
49.18	0.01\\
49.19	0.01\\
49.2	0.01\\
49.21	0.01\\
49.22	0.01\\
49.23	0.01\\
49.24	0.01\\
49.25	0.01\\
49.26	0.01\\
49.27	0.01\\
49.28	0.01\\
49.29	0.01\\
49.3	0.01\\
49.31	0.01\\
49.32	0.01\\
49.33	0.01\\
49.34	0.01\\
49.35	0.01\\
49.36	0.01\\
49.37	0.01\\
49.38	0.01\\
49.39	0.01\\
49.4	0.01\\
49.41	0.01\\
49.42	0.01\\
49.43	0.01\\
49.44	0.01\\
49.45	0.01\\
49.46	0.01\\
49.47	0.01\\
49.48	0.01\\
49.49	0.01\\
49.5	0.01\\
49.51	0.01\\
49.52	0.01\\
49.53	0.01\\
49.54	0.01\\
49.55	0.01\\
49.56	0.01\\
49.57	0.01\\
49.58	0.01\\
49.59	0.01\\
49.6	0.01\\
49.61	0.01\\
49.62	0.01\\
49.63	0.01\\
49.64	0.01\\
49.65	0.01\\
49.66	0.01\\
49.67	0.01\\
49.68	0.01\\
49.69	0.01\\
49.7	0.01\\
49.71	0.01\\
49.72	0.01\\
49.73	0.01\\
49.74	0.01\\
49.75	0.01\\
49.76	0.01\\
49.77	0.01\\
49.78	0.01\\
49.79	0.01\\
49.8	0.01\\
49.81	0.01\\
49.82	0.01\\
49.83	0.01\\
49.84	0.01\\
49.85	0.01\\
49.86	0.01\\
49.87	0.01\\
49.88	0.01\\
49.89	0.01\\
49.9	0.01\\
49.91	0.01\\
49.92	0.01\\
49.93	0.01\\
49.94	0.01\\
49.95	0.01\\
49.96	0.01\\
49.97	0.01\\
49.98	0.01\\
49.99	0.01\\
50	0.01\\
50.01	0.01\\
50.02	0.01\\
50.03	0.01\\
50.04	0.01\\
50.05	0.01\\
50.06	0.01\\
50.07	0.01\\
50.08	0.01\\
50.09	0.01\\
50.1	0.01\\
50.11	0.01\\
50.12	0.01\\
50.13	0.01\\
50.14	0.01\\
50.15	0.01\\
50.16	0.01\\
50.17	0.01\\
50.18	0.01\\
50.19	0.01\\
50.2	0.01\\
50.21	0.01\\
50.22	0.01\\
50.23	0.01\\
50.24	0.01\\
50.25	0.01\\
50.26	0.01\\
50.27	0.01\\
50.28	0.01\\
50.29	0.01\\
50.3	0.01\\
50.31	0.01\\
50.32	0.01\\
50.33	0.01\\
50.34	0.01\\
50.35	0.01\\
50.36	0.01\\
50.37	0.01\\
50.38	0.01\\
50.39	0.01\\
50.4	0.01\\
50.41	0.01\\
50.42	0.01\\
50.43	0.01\\
50.44	0.01\\
50.45	0.01\\
50.46	0.01\\
50.47	0.01\\
50.48	0.01\\
50.49	0.01\\
50.5	0.01\\
50.51	0.01\\
50.52	0.01\\
50.53	0.01\\
50.54	0.01\\
50.55	0.01\\
50.56	0.01\\
50.57	0.01\\
50.58	0.01\\
50.59	0.01\\
50.6	0.01\\
50.61	0.01\\
50.62	0.01\\
50.63	0.01\\
50.64	0.01\\
50.65	0.01\\
50.66	0.01\\
50.67	0.01\\
50.68	0.01\\
50.69	0.01\\
50.7	0.01\\
50.71	0.01\\
50.72	0.01\\
50.73	0.01\\
50.74	0.01\\
50.75	0.01\\
50.76	0.01\\
50.77	0.01\\
50.78	0.01\\
50.79	0.01\\
50.8	0.01\\
50.81	0.01\\
50.82	0.01\\
50.83	0.01\\
50.84	0.01\\
50.85	0.01\\
50.86	0.01\\
50.87	0.01\\
50.88	0.01\\
50.89	0.01\\
50.9	0.01\\
50.91	0.01\\
50.92	0.01\\
50.93	0.01\\
50.94	0.01\\
50.95	0.01\\
50.96	0.01\\
50.97	0.01\\
50.98	0.01\\
50.99	0.01\\
51	0.01\\
51.01	0.01\\
51.02	0.01\\
51.03	0.01\\
51.04	0.01\\
51.05	0.01\\
51.06	0.01\\
51.07	0.01\\
51.08	0.01\\
51.09	0.01\\
51.1	0.01\\
51.11	0.01\\
51.12	0.01\\
51.13	0.01\\
51.14	0.01\\
51.15	0.01\\
51.16	0.01\\
51.17	0.01\\
51.18	0.01\\
51.19	0.01\\
51.2	0.01\\
51.21	0.01\\
51.22	0.01\\
51.23	0.01\\
51.24	0.01\\
51.25	0.01\\
51.26	0.01\\
51.27	0.01\\
51.28	0.01\\
51.29	0.01\\
51.3	0.01\\
51.31	0.01\\
51.32	0.01\\
51.33	0.01\\
51.34	0.01\\
51.35	0.01\\
51.36	0.01\\
51.37	0.01\\
51.38	0.01\\
51.39	0.01\\
51.4	0.01\\
51.41	0.01\\
51.42	0.01\\
51.43	0.01\\
51.44	0.01\\
51.45	0.01\\
51.46	0.01\\
51.47	0.01\\
51.48	0.01\\
51.49	0.01\\
51.5	0.01\\
51.51	0.01\\
51.52	0.01\\
51.53	0.01\\
51.54	0.01\\
51.55	0.01\\
51.56	0.01\\
51.57	0.01\\
51.58	0.01\\
51.59	0.01\\
51.6	0.01\\
51.61	0.01\\
51.62	0.01\\
51.63	0.01\\
51.64	0.01\\
51.65	0.01\\
51.66	0.01\\
51.67	0.01\\
51.68	0.01\\
51.69	0.01\\
51.7	0.01\\
51.71	0.01\\
51.72	0.01\\
51.73	0.01\\
51.74	0.01\\
51.75	0.01\\
51.76	0.01\\
51.77	0.01\\
51.78	0.01\\
51.79	0.01\\
51.8	0.01\\
51.81	0.01\\
51.82	0.01\\
51.83	0.01\\
51.84	0.01\\
51.85	0.01\\
51.86	0.01\\
51.87	0.01\\
51.88	0.01\\
51.89	0.01\\
51.9	0.01\\
51.91	0.01\\
51.92	0.01\\
51.93	0.01\\
51.94	0.01\\
51.95	0.01\\
51.96	0.01\\
51.97	0.01\\
51.98	0.01\\
51.99	0.01\\
52	0.01\\
52.01	0.01\\
52.02	0.01\\
52.03	0.01\\
52.04	0.01\\
52.05	0.01\\
52.06	0.01\\
52.07	0.01\\
52.08	0.01\\
52.09	0.01\\
52.1	0.01\\
52.11	0.01\\
52.12	0.01\\
52.13	0.01\\
52.14	0.01\\
52.15	0.01\\
52.16	0.01\\
52.17	0.01\\
52.18	0.01\\
52.19	0.01\\
52.2	0.01\\
52.21	0.01\\
52.22	0.01\\
52.23	0.01\\
52.24	0.01\\
52.25	0.01\\
52.26	0.01\\
52.27	0.01\\
52.28	0.01\\
52.29	0.01\\
52.3	0.01\\
52.31	0.01\\
52.32	0.01\\
52.33	0.01\\
52.34	0.01\\
52.35	0.01\\
52.36	0.01\\
52.37	0.01\\
52.38	0.01\\
52.39	0.01\\
52.4	0.01\\
52.41	0.01\\
52.42	0.01\\
52.43	0.01\\
52.44	0.01\\
52.45	0.01\\
52.46	0.01\\
52.47	0.01\\
52.48	0.01\\
52.49	0.01\\
52.5	0.01\\
52.51	0.01\\
52.52	0.01\\
52.53	0.01\\
52.54	0.01\\
52.55	0.01\\
52.56	0.01\\
52.57	0.01\\
52.58	0.01\\
52.59	0.01\\
52.6	0.01\\
52.61	0.01\\
52.62	0.01\\
52.63	0.01\\
52.64	0.01\\
52.65	0.01\\
52.66	0.01\\
52.67	0.01\\
52.68	0.01\\
52.69	0.01\\
52.7	0.01\\
52.71	0.01\\
52.72	0.01\\
52.73	0.01\\
52.74	0.01\\
52.75	0.01\\
52.76	0.01\\
52.77	0.01\\
52.78	0.01\\
52.79	0.01\\
52.8	0.01\\
52.81	0.01\\
52.82	0.01\\
52.83	0.01\\
52.84	0.01\\
52.85	0.01\\
52.86	0.01\\
52.87	0.01\\
52.88	0.01\\
52.89	0.01\\
52.9	0.01\\
52.91	0.01\\
52.92	0.01\\
52.93	0.01\\
52.94	0.01\\
52.95	0.01\\
52.96	0.01\\
52.97	0.01\\
52.98	0.01\\
52.99	0.01\\
53	0.01\\
53.01	0.01\\
53.02	0.01\\
53.03	0.01\\
53.04	0.01\\
53.05	0.01\\
53.06	0.01\\
53.07	0.01\\
53.08	0.01\\
53.09	0.01\\
53.1	0.01\\
53.11	0.01\\
53.12	0.01\\
53.13	0.01\\
53.14	0.01\\
53.15	0.01\\
53.16	0.01\\
53.17	0.01\\
53.18	0.01\\
53.19	0.01\\
53.2	0.01\\
53.21	0.01\\
53.22	0.01\\
53.23	0.01\\
53.24	0.01\\
53.25	0.01\\
53.26	0.01\\
53.27	0.01\\
53.28	0.01\\
53.29	0.01\\
53.3	0.01\\
53.31	0.01\\
53.32	0.01\\
53.33	0.01\\
53.34	0.01\\
53.35	0.01\\
53.36	0.01\\
53.37	0.01\\
53.38	0.01\\
53.39	0.01\\
53.4	0.01\\
53.41	0.01\\
53.42	0.01\\
53.43	0.01\\
53.44	0.01\\
53.45	0.01\\
53.46	0.01\\
53.47	0.01\\
53.48	0.01\\
53.49	0.01\\
53.5	0.01\\
53.51	0.01\\
53.52	0.01\\
53.53	0.01\\
53.54	0.01\\
53.55	0.01\\
53.56	0.01\\
53.57	0.01\\
53.58	0.01\\
53.59	0.01\\
53.6	0.01\\
53.61	0.01\\
53.62	0.01\\
53.63	0.01\\
53.64	0.01\\
53.65	0.01\\
53.66	0.01\\
53.67	0.01\\
53.68	0.01\\
53.69	0.01\\
53.7	0.01\\
53.71	0.01\\
53.72	0.01\\
53.73	0.01\\
53.74	0.01\\
53.75	0.01\\
53.76	0.01\\
53.77	0.01\\
53.78	0.01\\
53.79	0.01\\
53.8	0.01\\
53.81	0.01\\
53.82	0.01\\
53.83	0.01\\
53.84	0.01\\
53.85	0.01\\
53.86	0.01\\
53.87	0.01\\
53.88	0.01\\
53.89	0.01\\
53.9	0.01\\
53.91	0.01\\
53.92	0.01\\
53.93	0.01\\
53.94	0.01\\
53.95	0.01\\
53.96	0.01\\
53.97	0.01\\
53.98	0.01\\
53.99	0.01\\
54	0.01\\
54.01	0.01\\
54.02	0.01\\
54.03	0.01\\
54.04	0.01\\
54.05	0.01\\
54.06	0.01\\
54.07	0.01\\
54.08	0.01\\
54.09	0.01\\
54.1	0.01\\
54.11	0.01\\
54.12	0.01\\
54.13	0.01\\
54.14	0.01\\
54.15	0.01\\
54.16	0.01\\
54.17	0.01\\
54.18	0.01\\
54.19	0.01\\
54.2	0.01\\
54.21	0.01\\
54.22	0.01\\
54.23	0.01\\
54.24	0.01\\
54.25	0.01\\
54.26	0.01\\
54.27	0.01\\
54.28	0.01\\
54.29	0.01\\
54.3	0.01\\
54.31	0.01\\
54.32	0.01\\
54.33	0.01\\
54.34	0.01\\
54.35	0.01\\
54.36	0.01\\
54.37	0.01\\
54.38	0.01\\
54.39	0.01\\
54.4	0.01\\
54.41	0.01\\
54.42	0.01\\
54.43	0.01\\
54.44	0.01\\
54.45	0.01\\
54.46	0.01\\
54.47	0.01\\
54.48	0.01\\
54.49	0.01\\
54.5	0.01\\
54.51	0.01\\
54.52	0.01\\
54.53	0.01\\
54.54	0.01\\
54.55	0.01\\
54.56	0.01\\
54.57	0.01\\
54.58	0.01\\
54.59	0.01\\
54.6	0.01\\
54.61	0.01\\
54.62	0.01\\
54.63	0.01\\
54.64	0.01\\
54.65	0.01\\
54.66	0.01\\
54.67	0.01\\
54.68	0.01\\
54.69	0.01\\
54.7	0.01\\
54.71	0.01\\
54.72	0.01\\
54.73	0.01\\
54.74	0.01\\
54.75	0.01\\
54.76	0.01\\
54.77	0.01\\
54.78	0.01\\
54.79	0.01\\
54.8	0.01\\
54.81	0.01\\
54.82	0.01\\
54.83	0.01\\
54.84	0.01\\
54.85	0.01\\
54.86	0.01\\
54.87	0.01\\
54.88	0.01\\
54.89	0.01\\
54.9	0.01\\
54.91	0.01\\
54.92	0.01\\
54.93	0.01\\
54.94	0.01\\
54.95	0.01\\
54.96	0.01\\
54.97	0.01\\
54.98	0.01\\
54.99	0.01\\
55	0.01\\
55.01	0.01\\
55.02	0.01\\
55.03	0.01\\
55.04	0.01\\
55.05	0.01\\
55.06	0.01\\
55.07	0.01\\
55.08	0.01\\
55.09	0.01\\
55.1	0.01\\
55.11	0.01\\
55.12	0.01\\
55.13	0.01\\
55.14	0.01\\
55.15	0.01\\
55.16	0.01\\
55.17	0.01\\
55.18	0.01\\
55.19	0.01\\
55.2	0.01\\
55.21	0.01\\
55.22	0.01\\
55.23	0.01\\
55.24	0.01\\
55.25	0.01\\
55.26	0.01\\
55.27	0.01\\
55.28	0.01\\
55.29	0.01\\
55.3	0.01\\
55.31	0.01\\
55.32	0.01\\
55.33	0.01\\
55.34	0.01\\
55.35	0.01\\
55.36	0.01\\
55.37	0.01\\
55.38	0.01\\
55.39	0.01\\
55.4	0.01\\
55.41	0.01\\
55.42	0.01\\
55.43	0.01\\
55.44	0.01\\
55.45	0.01\\
55.46	0.01\\
55.47	0.01\\
55.48	0.01\\
55.49	0.01\\
55.5	0.01\\
55.51	0.01\\
55.52	0.01\\
55.53	0.01\\
55.54	0.01\\
55.55	0.01\\
55.56	0.01\\
55.57	0.01\\
55.58	0.01\\
55.59	0.01\\
55.6	0.01\\
55.61	0.01\\
55.62	0.01\\
55.63	0.01\\
55.64	0.01\\
55.65	0.01\\
55.66	0.01\\
55.67	0.01\\
55.68	0.01\\
55.69	0.01\\
55.7	0.01\\
55.71	0.01\\
55.72	0.01\\
55.73	0.01\\
55.74	0.01\\
55.75	0.01\\
55.76	0.01\\
55.77	0.01\\
55.78	0.01\\
55.79	0.01\\
55.8	0.01\\
55.81	0.01\\
55.82	0.01\\
55.83	0.01\\
55.84	0.01\\
55.85	0.01\\
55.86	0.01\\
55.87	0.01\\
55.88	0.01\\
55.89	0.01\\
55.9	0.01\\
55.91	0.01\\
55.92	0.01\\
55.93	0.01\\
55.94	0.01\\
55.95	0.01\\
55.96	0.01\\
55.97	0.01\\
55.98	0.01\\
55.99	0.01\\
56	0.01\\
56.01	0.01\\
56.02	0.01\\
56.03	0.01\\
56.04	0.01\\
56.05	0.01\\
56.06	0.01\\
56.07	0.01\\
56.08	0.01\\
56.09	0.01\\
56.1	0.01\\
56.11	0.01\\
56.12	0.01\\
56.13	0.01\\
56.14	0.01\\
56.15	0.01\\
56.16	0.01\\
56.17	0.01\\
56.18	0.01\\
56.19	0.01\\
56.2	0.01\\
56.21	0.01\\
56.22	0.01\\
56.23	0.01\\
56.24	0.01\\
56.25	0.01\\
56.26	0.01\\
56.27	0.01\\
56.28	0.01\\
56.29	0.01\\
56.3	0.01\\
56.31	0.01\\
56.32	0.01\\
56.33	0.01\\
56.34	0.01\\
56.35	0.01\\
56.36	0.01\\
56.37	0.01\\
56.38	0.01\\
56.39	0.01\\
56.4	0.01\\
56.41	0.01\\
56.42	0.01\\
56.43	0.01\\
56.44	0.01\\
56.45	0.01\\
56.46	0.01\\
56.47	0.01\\
56.48	0.01\\
56.49	0.01\\
56.5	0.01\\
56.51	0.01\\
56.52	0.01\\
56.53	0.01\\
56.54	0.01\\
56.55	0.01\\
56.56	0.01\\
56.57	0.01\\
56.58	0.01\\
56.59	0.01\\
56.6	0.01\\
56.61	0.01\\
56.62	0.01\\
56.63	0.01\\
56.64	0.01\\
56.65	0.01\\
56.66	0.01\\
56.67	0.01\\
56.68	0.01\\
56.69	0.01\\
56.7	0.01\\
56.71	0.01\\
56.72	0.01\\
56.73	0.01\\
56.74	0.01\\
56.75	0.01\\
56.76	0.01\\
56.77	0.01\\
56.78	0.01\\
56.79	0.01\\
56.8	0.01\\
56.81	0.01\\
56.82	0.01\\
56.83	0.01\\
56.84	0.01\\
56.85	0.01\\
56.86	0.01\\
56.87	0.01\\
56.88	0.01\\
56.89	0.01\\
56.9	0.01\\
56.91	0.01\\
56.92	0.01\\
56.93	0.01\\
56.94	0.01\\
56.95	0.01\\
56.96	0.01\\
56.97	0.01\\
56.98	0.01\\
56.99	0.01\\
57	0.01\\
57.01	0.01\\
57.02	0.01\\
57.03	0.01\\
57.04	0.01\\
57.05	0.01\\
57.06	0.01\\
57.07	0.01\\
57.08	0.01\\
57.09	0.01\\
57.1	0.01\\
57.11	0.01\\
57.12	0.01\\
57.13	0.01\\
57.14	0.01\\
57.15	0.01\\
57.16	0.01\\
57.17	0.01\\
57.18	0.01\\
57.19	0.01\\
57.2	0.01\\
57.21	0.01\\
57.22	0.01\\
57.23	0.01\\
57.24	0.01\\
57.25	0.01\\
57.26	0.01\\
57.27	0.01\\
57.28	0.01\\
57.29	0.01\\
57.3	0.01\\
57.31	0.01\\
57.32	0.01\\
57.33	0.01\\
57.34	0.01\\
57.35	0.01\\
57.36	0.01\\
57.37	0.01\\
57.38	0.01\\
57.39	0.01\\
57.4	0.01\\
57.41	0.01\\
57.42	0.01\\
57.43	0.01\\
57.44	0.01\\
57.45	0.01\\
57.46	0.01\\
57.47	0.01\\
57.48	0.01\\
57.49	0.01\\
57.5	0.01\\
57.51	0.01\\
57.52	0.01\\
57.53	0.01\\
57.54	0.01\\
57.55	0.01\\
57.56	0.01\\
57.57	0.01\\
57.58	0.01\\
57.59	0.01\\
57.6	0.01\\
57.61	0.01\\
57.62	0.01\\
57.63	0.01\\
57.64	0.01\\
57.65	0.01\\
57.66	0.01\\
57.67	0.01\\
57.68	0.01\\
57.69	0.01\\
57.7	0.01\\
57.71	0.01\\
57.72	0.01\\
57.73	0.01\\
57.74	0.01\\
57.75	0.01\\
57.76	0.01\\
57.77	0.01\\
57.78	0.01\\
57.79	0.01\\
57.8	0.01\\
57.81	0.01\\
57.82	0.01\\
57.83	0.01\\
57.84	0.01\\
57.85	0.01\\
57.86	0.01\\
57.87	0.01\\
57.88	0.01\\
57.89	0.01\\
57.9	0.01\\
57.91	0.01\\
57.92	0.01\\
57.93	0.01\\
57.94	0.01\\
57.95	0.01\\
57.96	0.01\\
57.97	0.01\\
57.98	0.01\\
57.99	0.01\\
58	0.01\\
58.01	0.01\\
58.02	0.01\\
58.03	0.01\\
58.04	0.01\\
58.05	0.01\\
58.06	0.01\\
58.07	0.01\\
58.08	0.01\\
58.09	0.01\\
58.1	0.01\\
58.11	0.01\\
58.12	0.01\\
58.13	0.01\\
58.14	0.01\\
58.15	0.01\\
58.16	0.01\\
58.17	0.01\\
58.18	0.01\\
58.19	0.01\\
58.2	0.01\\
58.21	0.01\\
58.22	0.01\\
58.23	0.01\\
58.24	0.01\\
58.25	0.01\\
58.26	0.01\\
58.27	0.01\\
58.28	0.01\\
58.29	0.01\\
58.3	0.01\\
58.31	0.01\\
58.32	0.01\\
58.33	0.01\\
58.34	0.01\\
58.35	0.01\\
58.36	0.01\\
58.37	0.01\\
58.38	0.01\\
58.39	0.01\\
58.4	0.01\\
58.41	0.01\\
58.42	0.01\\
58.43	0.01\\
58.44	0.01\\
58.45	0.01\\
58.46	0.01\\
58.47	0.01\\
58.48	0.01\\
58.49	0.01\\
58.5	0.01\\
58.51	0.01\\
58.52	0.01\\
58.53	0.01\\
58.54	0.01\\
58.55	0.01\\
58.56	0.01\\
58.57	0.01\\
58.58	0.01\\
58.59	0.01\\
58.6	0.01\\
58.61	0.01\\
58.62	0.01\\
58.63	0.01\\
58.64	0.01\\
58.65	0.01\\
58.66	0.01\\
58.67	0.01\\
58.68	0.01\\
58.69	0.01\\
58.7	0.01\\
58.71	0.01\\
58.72	0.01\\
58.73	0.01\\
58.74	0.01\\
58.75	0.01\\
58.76	0.01\\
58.77	0.01\\
58.78	0.01\\
58.79	0.01\\
58.8	0.01\\
58.81	0.01\\
58.82	0.01\\
58.83	0.01\\
58.84	0.01\\
58.85	0.01\\
58.86	0.01\\
58.87	0.01\\
58.88	0.01\\
58.89	0.01\\
58.9	0.01\\
58.91	0.01\\
58.92	0.01\\
58.93	0.01\\
58.94	0.01\\
58.95	0.01\\
58.96	0.01\\
58.97	0.01\\
58.98	0.01\\
58.99	0.01\\
59	0.01\\
59.01	0.01\\
59.02	0.01\\
59.03	0.01\\
59.04	0.01\\
59.05	0.01\\
59.06	0.01\\
59.07	0.01\\
59.08	0.01\\
59.09	0.01\\
59.1	0.01\\
59.11	0.01\\
59.12	0.01\\
59.13	0.01\\
59.14	0.01\\
59.15	0.01\\
59.16	0.01\\
59.17	0.01\\
59.18	0.01\\
59.19	0.01\\
59.2	0.01\\
59.21	0.01\\
59.22	0.01\\
59.23	0.01\\
59.24	0.01\\
59.25	0.01\\
59.26	0.01\\
59.27	0.01\\
59.28	0.01\\
59.29	0.01\\
59.3	0.01\\
59.31	0.01\\
59.32	0.01\\
59.33	0.01\\
59.34	0.01\\
59.35	0.01\\
59.36	0.01\\
59.37	0.01\\
59.38	0.01\\
59.39	0.01\\
59.4	0.01\\
59.41	0.01\\
59.42	0.01\\
59.43	0.01\\
59.44	0.01\\
59.45	0.01\\
59.46	0.01\\
59.47	0.01\\
59.48	0.01\\
59.49	0.01\\
59.5	0.01\\
59.51	0.01\\
59.52	0.01\\
59.53	0.01\\
59.54	0.01\\
59.55	0.01\\
59.56	0.01\\
59.57	0.01\\
59.58	0.01\\
59.59	0.01\\
59.6	0.01\\
59.61	0.01\\
59.62	0.01\\
59.63	0.01\\
59.64	0.01\\
59.65	0.01\\
59.66	0.01\\
59.67	0.01\\
59.68	0.01\\
59.69	0.01\\
59.7	0.01\\
59.71	0.01\\
59.72	0.01\\
59.73	0.01\\
59.74	0.01\\
59.75	0.01\\
59.76	0.01\\
59.77	0.01\\
59.78	0.01\\
59.79	0.01\\
59.8	0.01\\
59.81	0.01\\
59.82	0.01\\
59.83	0.01\\
59.84	0.01\\
59.85	0.01\\
59.86	0.01\\
59.87	0.01\\
59.88	0.01\\
59.89	0.01\\
59.9	0.01\\
59.91	0.01\\
59.92	0.01\\
59.93	0.01\\
59.94	0.01\\
59.95	0.01\\
59.96	0.01\\
59.97	0.01\\
59.98	0.01\\
59.99	0.01\\
60	0.01\\
60.01	0.01\\
60.02	0.01\\
60.03	0.01\\
60.04	0.01\\
60.05	0.01\\
60.06	0.01\\
60.07	0.01\\
60.08	0.01\\
60.09	0.01\\
60.1	0.01\\
60.11	0.01\\
60.12	0.01\\
60.13	0.01\\
60.14	0.01\\
60.15	0.01\\
60.16	0.01\\
60.17	0.01\\
60.18	0.01\\
60.19	0.01\\
60.2	0.01\\
60.21	0.01\\
60.22	0.01\\
60.23	0.01\\
60.24	0.01\\
60.25	0.01\\
60.26	0.01\\
60.27	0.01\\
60.28	0.01\\
60.29	0.01\\
60.3	0.01\\
60.31	0.01\\
60.32	0.01\\
60.33	0.01\\
60.34	0.01\\
60.35	0.01\\
60.36	0.01\\
60.37	0.01\\
60.38	0.01\\
60.39	0.01\\
60.4	0.01\\
60.41	0.01\\
60.42	0.01\\
60.43	0.01\\
60.44	0.01\\
60.45	0.01\\
60.46	0.01\\
60.47	0.01\\
60.48	0.01\\
60.49	0.01\\
60.5	0.01\\
60.51	0.01\\
60.52	0.01\\
60.53	0.01\\
60.54	0.01\\
60.55	0.01\\
60.56	0.01\\
60.57	0.01\\
60.58	0.01\\
60.59	0.01\\
60.6	0.01\\
60.61	0.01\\
60.62	0.01\\
60.63	0.01\\
60.64	0.01\\
60.65	0.01\\
60.66	0.01\\
60.67	0.01\\
60.68	0.01\\
60.69	0.01\\
60.7	0.01\\
60.71	0.01\\
60.72	0.01\\
60.73	0.01\\
60.74	0.01\\
60.75	0.01\\
60.76	0.01\\
60.77	0.01\\
60.78	0.01\\
60.79	0.01\\
60.8	0.01\\
60.81	0.01\\
60.82	0.01\\
60.83	0.01\\
60.84	0.01\\
60.85	0.01\\
60.86	0.01\\
60.87	0.01\\
60.88	0.01\\
60.89	0.01\\
60.9	0.01\\
60.91	0.01\\
60.92	0.01\\
60.93	0.01\\
60.94	0.01\\
60.95	0.01\\
60.96	0.01\\
60.97	0.01\\
60.98	0.01\\
60.99	0.01\\
61	0.01\\
61.01	0.01\\
61.02	0.01\\
61.03	0.01\\
61.04	0.01\\
61.05	0.01\\
61.06	0.01\\
61.07	0.01\\
61.08	0.01\\
61.09	0.01\\
61.1	0.01\\
61.11	0.01\\
61.12	0.01\\
61.13	0.01\\
61.14	0.01\\
61.15	0.01\\
61.16	0.01\\
61.17	0.01\\
61.18	0.01\\
61.19	0.01\\
61.2	0.01\\
61.21	0.01\\
61.22	0.01\\
61.23	0.01\\
61.24	0.01\\
61.25	0.01\\
61.26	0.01\\
61.27	0.01\\
61.28	0.01\\
61.29	0.01\\
61.3	0.01\\
61.31	0.01\\
61.32	0.01\\
61.33	0.01\\
61.34	0.01\\
61.35	0.01\\
61.36	0.01\\
61.37	0.01\\
61.38	0.01\\
61.39	0.01\\
61.4	0.01\\
61.41	0.01\\
61.42	0.01\\
61.43	0.01\\
61.44	0.01\\
61.45	0.01\\
61.46	0.01\\
61.47	0.01\\
61.48	0.01\\
61.49	0.01\\
61.5	0.01\\
61.51	0.01\\
61.52	0.01\\
61.53	0.01\\
61.54	0.01\\
61.55	0.01\\
61.56	0.01\\
61.57	0.01\\
61.58	0.01\\
61.59	0.01\\
61.6	0.01\\
61.61	0.01\\
61.62	0.01\\
61.63	0.01\\
61.64	0.01\\
61.65	0.01\\
61.66	0.01\\
61.67	0.01\\
61.68	0.01\\
61.69	0.01\\
61.7	0.01\\
61.71	0.01\\
61.72	0.01\\
61.73	0.01\\
61.74	0.01\\
61.75	0.01\\
61.76	0.01\\
61.77	0.01\\
61.78	0.01\\
61.79	0.01\\
61.8	0.01\\
61.81	0.01\\
61.82	0.01\\
61.83	0.01\\
61.84	0.01\\
61.85	0.01\\
61.86	0.01\\
61.87	0.01\\
61.88	0.01\\
61.89	0.01\\
61.9	0.01\\
61.91	0.01\\
61.92	0.01\\
61.93	0.01\\
61.94	0.01\\
61.95	0.01\\
61.96	0.01\\
61.97	0.01\\
61.98	0.01\\
61.99	0.01\\
62	0.01\\
62.01	0.01\\
62.02	0.01\\
62.03	0.01\\
62.04	0.01\\
62.05	0.01\\
62.06	0.01\\
62.07	0.01\\
62.08	0.01\\
62.09	0.01\\
62.1	0.01\\
62.11	0.01\\
62.12	0.01\\
62.13	0.01\\
62.14	0.01\\
62.15	0.01\\
62.16	0.01\\
62.17	0.01\\
62.18	0.01\\
62.19	0.01\\
62.2	0.01\\
62.21	0.01\\
62.22	0.01\\
62.23	0.01\\
62.24	0.01\\
62.25	0.01\\
62.26	0.01\\
62.27	0.01\\
62.28	0.01\\
62.29	0.01\\
62.3	0.01\\
62.31	0.01\\
62.32	0.01\\
62.33	0.01\\
62.34	0.01\\
62.35	0.01\\
62.36	0.01\\
62.37	0.01\\
62.38	0.01\\
62.39	0.01\\
62.4	0.01\\
62.41	0.01\\
62.42	0.01\\
62.43	0.01\\
62.44	0.01\\
62.45	0.01\\
62.46	0.01\\
62.47	0.01\\
62.48	0.01\\
62.49	0.01\\
62.5	0.01\\
62.51	0.01\\
62.52	0.01\\
62.53	0.01\\
62.54	0.01\\
62.55	0.01\\
62.56	0.01\\
62.57	0.01\\
62.58	0.01\\
62.59	0.01\\
62.6	0.01\\
62.61	0.01\\
62.62	0.01\\
62.63	0.01\\
62.64	0.01\\
62.65	0.01\\
62.66	0.01\\
62.67	0.01\\
62.68	0.01\\
62.69	0.01\\
62.7	0.01\\
62.71	0.01\\
62.72	0.01\\
62.73	0.01\\
62.74	0.01\\
62.75	0.01\\
62.76	0.01\\
62.77	0.01\\
62.78	0.01\\
62.79	0.01\\
62.8	0.01\\
62.81	0.01\\
62.82	0.01\\
62.83	0.01\\
62.84	0.01\\
62.85	0.01\\
62.86	0.01\\
62.87	0.01\\
62.88	0.01\\
62.89	0.01\\
62.9	0.01\\
62.91	0.01\\
62.92	0.01\\
62.93	0.01\\
62.94	0.01\\
62.95	0.01\\
62.96	0.01\\
62.97	0.01\\
62.98	0.01\\
62.99	0.01\\
63	0.01\\
63.01	0.01\\
63.02	0.01\\
63.03	0.01\\
63.04	0.01\\
63.05	0.01\\
63.06	0.01\\
63.07	0.01\\
63.08	0.01\\
63.09	0.01\\
63.1	0.01\\
63.11	0.01\\
63.12	0.01\\
63.13	0.01\\
63.14	0.01\\
63.15	0.01\\
63.16	0.01\\
63.17	0.01\\
63.18	0.01\\
63.19	0.01\\
63.2	0.01\\
63.21	0.01\\
63.22	0.01\\
63.23	0.01\\
63.24	0.01\\
63.25	0.01\\
63.26	0.01\\
63.27	0.01\\
63.28	0.01\\
63.29	0.01\\
63.3	0.01\\
63.31	0.01\\
63.32	0.01\\
63.33	0.01\\
63.34	0.01\\
63.35	0.01\\
63.36	0.01\\
63.37	0.01\\
63.38	0.01\\
63.39	0.01\\
63.4	0.01\\
63.41	0.01\\
63.42	0.01\\
63.43	0.01\\
63.44	0.01\\
63.45	0.01\\
63.46	0.01\\
63.47	0.01\\
63.48	0.01\\
63.49	0.01\\
63.5	0.01\\
63.51	0.01\\
63.52	0.01\\
63.53	0.01\\
63.54	0.01\\
63.55	0.01\\
63.56	0.01\\
63.57	0.01\\
63.58	0.01\\
63.59	0.01\\
63.6	0.01\\
63.61	0.01\\
63.62	0.01\\
63.63	0.01\\
63.64	0.01\\
63.65	0.01\\
63.66	0.01\\
63.67	0.01\\
63.68	0.01\\
63.69	0.01\\
63.7	0.01\\
63.71	0.01\\
63.72	0.01\\
63.73	0.01\\
63.74	0.01\\
63.75	0.01\\
63.76	0.01\\
63.77	0.01\\
63.78	0.01\\
63.79	0.01\\
63.8	0.01\\
63.81	0.01\\
63.82	0.01\\
63.83	0.01\\
63.84	0.01\\
63.85	0.01\\
63.86	0.01\\
63.87	0.01\\
63.88	0.01\\
63.89	0.01\\
63.9	0.01\\
63.91	0.01\\
63.92	0.01\\
63.93	0.01\\
63.94	0.01\\
63.95	0.01\\
63.96	0.01\\
63.97	0.01\\
63.98	0.01\\
63.99	0.01\\
64	0.01\\
64.01	0.01\\
64.02	0.01\\
64.03	0.01\\
64.04	0.01\\
64.05	0.01\\
64.06	0.01\\
64.07	0.01\\
64.08	0.01\\
64.09	0.01\\
64.1	0.01\\
64.11	0.01\\
64.12	0.01\\
64.13	0.01\\
64.14	0.01\\
64.15	0.01\\
64.16	0.01\\
64.17	0.01\\
64.18	0.01\\
64.19	0.01\\
64.2	0.01\\
64.21	0.01\\
64.22	0.01\\
64.23	0.01\\
64.24	0.01\\
64.25	0.01\\
64.26	0.01\\
64.27	0.01\\
64.28	0.01\\
64.29	0.01\\
64.3	0.01\\
64.31	0.01\\
64.32	0.01\\
64.33	0.01\\
64.34	0.01\\
64.35	0.01\\
64.36	0.01\\
64.37	0.01\\
64.38	0.01\\
64.39	0.01\\
64.4	0.01\\
64.41	0.01\\
64.42	0.01\\
64.43	0.01\\
64.44	0.01\\
64.45	0.01\\
64.46	0.01\\
64.47	0.01\\
64.48	0.01\\
64.49	0.01\\
64.5	0.01\\
64.51	0.01\\
64.52	0.01\\
64.53	0.01\\
64.54	0.01\\
64.55	0.01\\
64.56	0.01\\
64.57	0.01\\
64.58	0.01\\
64.59	0.01\\
64.6	0.01\\
64.61	0.01\\
64.62	0.01\\
64.63	0.01\\
64.64	0.01\\
64.65	0.01\\
64.66	0.01\\
64.67	0.01\\
64.68	0.01\\
64.69	0.01\\
64.7	0.01\\
64.71	0.01\\
64.72	0.01\\
64.73	0.01\\
64.74	0.01\\
64.75	0.01\\
64.76	0.01\\
64.77	0.01\\
64.78	0.01\\
64.79	0.01\\
64.8	0.01\\
64.81	0.01\\
64.82	0.01\\
64.83	0.01\\
64.84	0.01\\
64.85	0.01\\
64.86	0.01\\
64.87	0.01\\
64.88	0.01\\
64.89	0.01\\
64.9	0.01\\
64.91	0.01\\
64.92	0.01\\
64.93	0.01\\
64.94	0.01\\
64.95	0.01\\
64.96	0.01\\
64.97	0.01\\
64.98	0.01\\
64.99	0.01\\
65	0.01\\
65.01	0.01\\
65.02	0.01\\
65.03	0.01\\
65.04	0.01\\
65.05	0.01\\
65.06	0.01\\
65.07	0.01\\
65.08	0.01\\
65.09	0.01\\
65.1	0.01\\
65.11	0.01\\
65.12	0.01\\
65.13	0.01\\
65.14	0.01\\
65.15	0.01\\
65.16	0.01\\
65.17	0.01\\
65.18	0.01\\
65.19	0.01\\
65.2	0.01\\
65.21	0.01\\
65.22	0.01\\
65.23	0.01\\
65.24	0.01\\
65.25	0.01\\
65.26	0.01\\
65.27	0.01\\
65.28	0.01\\
65.29	0.01\\
65.3	0.01\\
65.31	0.01\\
65.32	0.01\\
65.33	0.01\\
65.34	0.01\\
65.35	0.01\\
65.36	0.01\\
65.37	0.01\\
65.38	0.01\\
65.39	0.01\\
65.4	0.01\\
65.41	0.01\\
65.42	0.01\\
65.43	0.01\\
65.44	0.01\\
65.45	0.01\\
65.46	0.01\\
65.47	0.01\\
65.48	0.01\\
65.49	0.01\\
65.5	0.01\\
65.51	0.01\\
65.52	0.01\\
65.53	0.01\\
65.54	0.01\\
65.55	0.01\\
65.56	0.01\\
65.57	0.01\\
65.58	0.01\\
65.59	0.01\\
65.6	0.01\\
65.61	0.01\\
65.62	0.01\\
65.63	0.01\\
65.64	0.01\\
65.65	0.01\\
65.66	0.01\\
65.67	0.01\\
65.68	0.01\\
65.69	0.01\\
65.7	0.01\\
65.71	0.01\\
65.72	0.01\\
65.73	0.01\\
65.74	0.01\\
65.75	0.01\\
65.76	0.01\\
65.77	0.01\\
65.78	0.01\\
65.79	0.01\\
65.8	0.01\\
65.81	0.01\\
65.82	0.01\\
65.83	0.01\\
65.84	0.01\\
65.85	0.01\\
65.86	0.01\\
65.87	0.01\\
65.88	0.01\\
65.89	0.01\\
65.9	0.01\\
65.91	0.01\\
65.92	0.01\\
65.93	0.01\\
65.94	0.01\\
65.95	0.01\\
65.96	0.01\\
65.97	0.01\\
65.98	0.01\\
65.99	0.01\\
66	0.01\\
66.01	0.01\\
66.02	0.01\\
66.03	0.01\\
66.04	0.01\\
66.05	0.01\\
66.06	0.01\\
66.07	0.01\\
66.08	0.01\\
66.09	0.01\\
66.1	0.01\\
66.11	0.01\\
66.12	0.01\\
66.13	0.01\\
66.14	0.01\\
66.15	0.01\\
66.16	0.01\\
66.17	0.01\\
66.18	0.01\\
66.19	0.01\\
66.2	0.01\\
66.21	0.01\\
66.22	0.01\\
66.23	0.01\\
66.24	0.01\\
66.25	0.01\\
66.26	0.01\\
66.27	0.01\\
66.28	0.01\\
66.29	0.01\\
66.3	0.01\\
66.31	0.01\\
66.32	0.01\\
66.33	0.01\\
66.34	0.01\\
66.35	0.01\\
66.36	0.01\\
66.37	0.01\\
66.38	0.01\\
66.39	0.01\\
66.4	0.01\\
66.41	0.01\\
66.42	0.01\\
66.43	0.01\\
66.44	0.01\\
66.45	0.01\\
66.46	0.01\\
66.47	0.01\\
66.48	0.01\\
66.49	0.01\\
66.5	0.01\\
66.51	0.01\\
66.52	0.01\\
66.53	0.01\\
66.54	0.01\\
66.55	0.01\\
66.56	0.01\\
66.57	0.01\\
66.58	0.01\\
66.59	0.01\\
66.6	0.01\\
66.61	0.01\\
66.62	0.01\\
66.63	0.01\\
66.64	0.01\\
66.65	0.01\\
66.66	0.01\\
66.67	0.01\\
66.68	0.01\\
66.69	0.01\\
66.7	0.01\\
66.71	0.01\\
66.72	0.01\\
66.73	0.01\\
66.74	0.01\\
66.75	0.01\\
66.76	0.01\\
66.77	0.01\\
66.78	0.01\\
66.79	0.01\\
66.8	0.01\\
66.81	0.01\\
66.82	0.01\\
66.83	0.01\\
66.84	0.01\\
66.85	0.01\\
66.86	0.01\\
66.87	0.01\\
66.88	0.01\\
66.89	0.01\\
66.9	0.01\\
66.91	0.01\\
66.92	0.01\\
66.93	0.01\\
66.94	0.01\\
66.95	0.01\\
66.96	0.01\\
66.97	0.01\\
66.98	0.01\\
66.99	0.01\\
67	0.01\\
67.01	0.01\\
67.02	0.01\\
67.03	0.01\\
67.04	0.01\\
67.05	0.01\\
67.06	0.01\\
67.07	0.01\\
67.08	0.01\\
67.09	0.01\\
67.1	0.01\\
67.11	0.01\\
67.12	0.01\\
67.13	0.01\\
67.14	0.01\\
67.15	0.01\\
67.16	0.01\\
67.17	0.01\\
67.18	0.01\\
67.19	0.01\\
67.2	0.01\\
67.21	0.01\\
67.22	0.01\\
67.23	0.01\\
67.24	0.01\\
67.25	0.01\\
67.26	0.01\\
67.27	0.01\\
67.28	0.01\\
67.29	0.01\\
67.3	0.01\\
67.31	0.01\\
67.32	0.01\\
67.33	0.01\\
67.34	0.01\\
67.35	0.01\\
67.36	0.01\\
67.37	0.01\\
67.38	0.01\\
67.39	0.01\\
67.4	0.01\\
67.41	0.01\\
67.42	0.01\\
67.43	0.01\\
67.44	0.01\\
67.45	0.01\\
67.46	0.01\\
67.47	0.01\\
67.48	0.01\\
67.49	0.01\\
67.5	0.01\\
67.51	0.01\\
67.52	0.01\\
67.53	0.01\\
67.54	0.01\\
67.55	0.01\\
67.56	0.01\\
67.57	0.01\\
67.58	0.01\\
67.59	0.01\\
67.6	0.01\\
67.61	0.01\\
67.62	0.01\\
67.63	0.01\\
67.64	0.01\\
67.65	0.01\\
67.66	0.01\\
67.67	0.01\\
67.68	0.01\\
67.69	0.01\\
67.7	0.01\\
67.71	0.01\\
67.72	0.01\\
67.73	0.01\\
67.74	0.01\\
67.75	0.01\\
67.76	0.01\\
67.77	0.01\\
67.78	0.01\\
67.79	0.01\\
67.8	0.01\\
67.81	0.01\\
67.82	0.01\\
67.83	0.01\\
67.84	0.01\\
67.85	0.01\\
67.86	0.01\\
67.87	0.01\\
67.88	0.01\\
67.89	0.01\\
67.9	0.01\\
67.91	0.01\\
67.92	0.01\\
67.93	0.01\\
67.94	0.01\\
67.95	0.01\\
67.96	0.01\\
67.97	0.01\\
67.98	0.01\\
67.99	0.01\\
68	0.01\\
68.01	0.01\\
68.02	0.01\\
68.03	0.01\\
68.04	0.01\\
68.05	0.01\\
68.06	0.01\\
68.07	0.01\\
68.08	0.01\\
68.09	0.01\\
68.1	0.01\\
68.11	0.01\\
68.12	0.01\\
68.13	0.01\\
68.14	0.01\\
68.15	0.01\\
68.16	0.01\\
68.17	0.01\\
68.18	0.01\\
68.19	0.01\\
68.2	0.01\\
68.21	0.01\\
68.22	0.01\\
68.23	0.01\\
68.24	0.01\\
68.25	0.01\\
68.26	0.01\\
68.27	0.01\\
68.28	0.01\\
68.29	0.01\\
68.3	0.01\\
68.31	0.01\\
68.32	0.01\\
68.33	0.01\\
68.34	0.01\\
68.35	0.01\\
68.36	0.01\\
68.37	0.01\\
68.38	0.01\\
68.39	0.01\\
68.4	0.01\\
68.41	0.01\\
68.42	0.01\\
68.43	0.01\\
68.44	0.01\\
68.45	0.01\\
68.46	0.01\\
68.47	0.01\\
68.48	0.01\\
68.49	0.01\\
68.5	0.01\\
68.51	0.01\\
68.52	0.01\\
68.53	0.01\\
68.54	0.01\\
68.55	0.01\\
68.56	0.01\\
68.57	0.01\\
68.58	0.01\\
68.59	0.01\\
68.6	0.01\\
68.61	0.01\\
68.62	0.01\\
68.63	0.01\\
68.64	0.01\\
68.65	0.01\\
68.66	0.01\\
68.67	0.01\\
68.68	0.01\\
68.69	0.01\\
68.7	0.01\\
68.71	0.01\\
68.72	0.01\\
68.73	0.01\\
68.74	0.01\\
68.75	0.01\\
68.76	0.01\\
68.77	0.01\\
68.78	0.01\\
68.79	0.01\\
68.8	0.01\\
68.81	0.01\\
68.82	0.01\\
68.83	0.01\\
68.84	0.01\\
68.85	0.01\\
68.86	0.01\\
68.87	0.01\\
68.88	0.01\\
68.89	0.01\\
68.9	0.01\\
68.91	0.01\\
68.92	0.01\\
68.93	0.01\\
68.94	0.01\\
68.95	0.01\\
68.96	0.01\\
68.97	0.01\\
68.98	0.01\\
68.99	0.01\\
69	0.01\\
69.01	0.01\\
69.02	0.01\\
69.03	0.01\\
69.04	0.01\\
69.05	0.01\\
69.06	0.01\\
69.07	0.01\\
69.08	0.01\\
69.09	0.01\\
69.1	0.01\\
69.11	0.01\\
69.12	0.01\\
69.13	0.01\\
69.14	0.01\\
69.15	0.01\\
69.16	0.01\\
69.17	0.01\\
69.18	0.01\\
69.19	0.01\\
69.2	0.01\\
69.21	0.01\\
69.22	0.01\\
69.23	0.01\\
69.24	0.01\\
69.25	0.01\\
69.26	0.01\\
69.27	0.01\\
69.28	0.01\\
69.29	0.01\\
69.3	0.01\\
69.31	0.01\\
69.32	0.01\\
69.33	0.01\\
69.34	0.01\\
69.35	0.01\\
69.36	0.01\\
69.37	0.01\\
69.38	0.01\\
69.39	0.01\\
69.4	0.01\\
69.41	0.01\\
69.42	0.01\\
69.43	0.01\\
69.44	0.01\\
69.45	0.01\\
69.46	0.01\\
69.47	0.01\\
69.48	0.01\\
69.49	0.01\\
69.5	0.01\\
69.51	0.01\\
69.52	0.01\\
69.53	0.01\\
69.54	0.01\\
69.55	0.01\\
69.56	0.01\\
69.57	0.01\\
69.58	0.01\\
69.59	0.01\\
69.6	0.01\\
69.61	0.01\\
69.62	0.01\\
69.63	0.01\\
69.64	0.01\\
69.65	0.01\\
69.66	0.01\\
69.67	0.01\\
69.68	0.01\\
69.69	0.01\\
69.7	0.01\\
69.71	0.01\\
69.72	0.01\\
69.73	0.01\\
69.74	0.01\\
69.75	0.01\\
69.76	0.01\\
69.77	0.01\\
69.78	0.01\\
69.79	0.01\\
69.8	0.01\\
69.81	0.01\\
69.82	0.01\\
69.83	0.01\\
69.84	0.01\\
69.85	0.01\\
69.86	0.01\\
69.87	0.01\\
69.88	0.01\\
69.89	0.01\\
69.9	0.01\\
69.91	0.01\\
69.92	0.01\\
69.93	0.01\\
69.94	0.01\\
69.95	0.01\\
69.96	0.01\\
69.97	0.01\\
69.98	0.01\\
69.99	0.01\\
70	0.01\\
70.01	0.01\\
70.02	0.01\\
70.03	0.01\\
70.04	0.01\\
70.05	0.01\\
70.06	0.01\\
70.07	0.01\\
70.08	0.01\\
70.09	0.01\\
70.1	0.01\\
70.11	0.01\\
70.12	0.01\\
70.13	0.01\\
70.14	0.01\\
70.15	0.01\\
70.16	0.01\\
70.17	0.01\\
70.18	0.01\\
70.19	0.01\\
70.2	0.01\\
70.21	0.01\\
70.22	0.01\\
70.23	0.01\\
70.24	0.01\\
70.25	0.01\\
70.26	0.01\\
70.27	0.01\\
70.28	0.01\\
70.29	0.01\\
70.3	0.01\\
70.31	0.01\\
70.32	0.01\\
70.33	0.01\\
70.34	0.01\\
70.35	0.01\\
70.36	0.01\\
70.37	0.01\\
70.38	0.01\\
70.39	0.01\\
70.4	0.01\\
70.41	0.01\\
70.42	0.01\\
70.43	0.01\\
70.44	0.01\\
70.45	0.01\\
70.46	0.01\\
70.47	0.01\\
70.48	0.01\\
70.49	0.01\\
70.5	0.01\\
70.51	0.01\\
70.52	0.01\\
70.53	0.01\\
70.54	0.01\\
70.55	0.01\\
70.56	0.01\\
70.57	0.01\\
70.58	0.01\\
70.59	0.01\\
70.6	0.01\\
70.61	0.01\\
70.62	0.01\\
70.63	0.01\\
70.64	0.01\\
70.65	0.01\\
70.66	0.01\\
70.67	0.01\\
70.68	0.01\\
70.69	0.01\\
70.7	0.01\\
70.71	0.01\\
70.72	0.01\\
70.73	0.01\\
70.74	0.01\\
70.75	0.01\\
70.76	0.01\\
70.77	0.01\\
70.78	0.01\\
70.79	0.01\\
70.8	0.01\\
70.81	0.01\\
70.82	0.01\\
70.83	0.01\\
70.84	0.01\\
70.85	0.01\\
70.86	0.01\\
70.87	0.01\\
70.88	0.01\\
70.89	0.01\\
70.9	0.01\\
70.91	0.01\\
70.92	0.01\\
70.93	0.01\\
70.94	0.01\\
70.95	0.01\\
70.96	0.01\\
70.97	0.01\\
70.98	0.01\\
70.99	0.01\\
71	0.01\\
71.01	0.01\\
71.02	0.01\\
71.03	0.01\\
71.04	0.01\\
71.05	0.01\\
71.06	0.01\\
71.07	0.01\\
71.08	0.01\\
71.09	0.01\\
71.1	0.01\\
71.11	0.01\\
71.12	0.01\\
71.13	0.01\\
71.14	0.01\\
71.15	0.01\\
71.16	0.01\\
71.17	0.01\\
71.18	0.01\\
71.19	0.01\\
71.2	0.01\\
71.21	0.01\\
71.22	0.01\\
71.23	0.01\\
71.24	0.01\\
71.25	0.01\\
71.26	0.01\\
71.27	0.01\\
71.28	0.01\\
71.29	0.01\\
71.3	0.01\\
71.31	0.01\\
71.32	0.01\\
71.33	0.01\\
71.34	0.01\\
71.35	0.01\\
71.36	0.01\\
71.37	0.01\\
71.38	0.01\\
71.39	0.01\\
71.4	0.01\\
71.41	0.01\\
71.42	0.01\\
71.43	0.01\\
71.44	0.01\\
71.45	0.01\\
71.46	0.01\\
71.47	0.01\\
71.48	0.01\\
71.49	0.01\\
71.5	0.01\\
71.51	0.01\\
71.52	0.01\\
71.53	0.01\\
71.54	0.01\\
71.55	0.01\\
71.56	0.01\\
71.57	0.01\\
71.58	0.01\\
71.59	0.01\\
71.6	0.01\\
71.61	0.01\\
71.62	0.01\\
71.63	0.01\\
71.64	0.01\\
71.65	0.01\\
71.66	0.01\\
71.67	0.01\\
71.68	0.01\\
71.69	0.01\\
71.7	0.01\\
71.71	0.01\\
71.72	0.01\\
71.73	0.01\\
71.74	0.01\\
71.75	0.01\\
71.76	0.01\\
71.77	0.01\\
71.78	0.01\\
71.79	0.01\\
71.8	0.01\\
71.81	0.01\\
71.82	0.01\\
71.83	0.01\\
71.84	0.01\\
71.85	0.01\\
71.86	0.01\\
71.87	0.01\\
71.88	0.01\\
71.89	0.01\\
71.9	0.01\\
71.91	0.01\\
71.92	0.01\\
71.93	0.01\\
71.94	0.01\\
71.95	0.01\\
71.96	0.01\\
71.97	0.01\\
71.98	0.01\\
71.99	0.01\\
72	0.01\\
72.01	0.01\\
72.02	0.01\\
72.03	0.01\\
72.04	0.01\\
72.05	0.01\\
72.06	0.01\\
72.07	0.01\\
72.08	0.01\\
72.09	0.01\\
72.1	0.01\\
72.11	0.01\\
72.12	0.01\\
72.13	0.01\\
72.14	0.01\\
72.15	0.01\\
72.16	0.01\\
72.17	0.01\\
72.18	0.01\\
72.19	0.01\\
72.2	0.01\\
72.21	0.01\\
72.22	0.01\\
72.23	0.01\\
72.24	0.01\\
72.25	0.01\\
72.26	0.01\\
72.27	0.01\\
72.28	0.01\\
72.29	0.01\\
72.3	0.01\\
72.31	0.01\\
72.32	0.01\\
72.33	0.01\\
72.34	0.01\\
72.35	0.01\\
72.36	0.01\\
72.37	0.01\\
72.38	0.01\\
72.39	0.01\\
72.4	0.01\\
72.41	0.01\\
72.42	0.01\\
72.43	0.01\\
72.44	0.01\\
72.45	0.01\\
72.46	0.01\\
72.47	0.01\\
72.48	0.01\\
72.49	0.01\\
72.5	0.01\\
72.51	0.01\\
72.52	0.01\\
72.53	0.01\\
72.54	0.01\\
72.55	0.01\\
72.56	0.01\\
72.57	0.01\\
72.58	0.01\\
72.59	0.01\\
72.6	0.01\\
72.61	0.01\\
72.62	0.01\\
72.63	0.01\\
72.64	0.01\\
72.65	0.01\\
72.66	0.01\\
72.67	0.01\\
72.68	0.01\\
72.69	0.01\\
72.7	0.01\\
72.71	0.01\\
72.72	0.01\\
72.73	0.01\\
72.74	0.01\\
72.75	0.01\\
72.76	0.01\\
72.77	0.01\\
72.78	0.01\\
72.79	0.01\\
72.8	0.01\\
72.81	0.01\\
72.82	0.01\\
72.83	0.01\\
72.84	0.01\\
72.85	0.01\\
72.86	0.01\\
72.87	0.01\\
72.88	0.01\\
72.89	0.01\\
72.9	0.01\\
72.91	0.01\\
72.92	0.01\\
72.93	0.01\\
72.94	0.01\\
72.95	0.01\\
72.96	0.01\\
72.97	0.01\\
72.98	0.01\\
72.99	0.01\\
73	0.01\\
73.01	0.01\\
73.02	0.01\\
73.03	0.01\\
73.04	0.01\\
73.05	0.01\\
73.06	0.01\\
73.07	0.01\\
73.08	0.01\\
73.09	0.01\\
73.1	0.01\\
73.11	0.01\\
73.12	0.01\\
73.13	0.01\\
73.14	0.01\\
73.15	0.01\\
73.16	0.01\\
73.17	0.01\\
73.18	0.01\\
73.19	0.01\\
73.2	0.01\\
73.21	0.01\\
73.22	0.01\\
73.23	0.01\\
73.24	0.01\\
73.25	0.01\\
73.26	0.01\\
73.27	0.01\\
73.28	0.01\\
73.29	0.01\\
73.3	0.01\\
73.31	0.01\\
73.32	0.01\\
73.33	0.01\\
73.34	0.01\\
73.35	0.01\\
73.36	0.01\\
73.37	0.01\\
73.38	0.01\\
73.39	0.01\\
73.4	0.01\\
73.41	0.01\\
73.42	0.01\\
73.43	0.01\\
73.44	0.01\\
73.45	0.01\\
73.46	0.01\\
73.47	0.01\\
73.48	0.01\\
73.49	0.01\\
73.5	0.01\\
73.51	0.01\\
73.52	0.01\\
73.53	0.01\\
73.54	0.01\\
73.55	0.01\\
73.56	0.01\\
73.57	0.01\\
73.58	0.01\\
73.59	0.01\\
73.6	0.01\\
73.61	0.01\\
73.62	0.01\\
73.63	0.01\\
73.64	0.01\\
73.65	0.01\\
73.66	0.01\\
73.67	0.01\\
73.68	0.01\\
73.69	0.01\\
73.7	0.01\\
73.71	0.01\\
73.72	0.01\\
73.73	0.01\\
73.74	0.01\\
73.75	0.01\\
73.76	0.01\\
73.77	0.01\\
73.78	0.01\\
73.79	0.01\\
73.8	0.01\\
73.81	0.01\\
73.82	0.01\\
73.83	0.01\\
73.84	0.01\\
73.85	0.01\\
73.86	0.01\\
73.87	0.01\\
73.88	0.01\\
73.89	0.01\\
73.9	0.01\\
73.91	0.01\\
73.92	0.01\\
73.93	0.01\\
73.94	0.01\\
73.95	0.01\\
73.96	0.01\\
73.97	0.01\\
73.98	0.01\\
73.99	0.01\\
74	0.01\\
74.01	0.01\\
74.02	0.01\\
74.03	0.01\\
74.04	0.01\\
74.05	0.01\\
74.06	0.01\\
74.07	0.01\\
74.08	0.01\\
74.09	0.01\\
74.1	0.01\\
74.11	0.01\\
74.12	0.01\\
74.13	0.01\\
74.14	0.01\\
74.15	0.01\\
74.16	0.01\\
74.17	0.01\\
74.18	0.01\\
74.19	0.01\\
74.2	0.01\\
74.21	0.01\\
74.22	0.01\\
74.23	0.01\\
74.24	0.01\\
74.25	0.01\\
74.26	0.01\\
74.27	0.01\\
74.28	0.01\\
74.29	0.01\\
74.3	0.01\\
74.31	0.01\\
74.32	0.01\\
74.33	0.01\\
74.34	0.01\\
74.35	0.01\\
74.36	0.01\\
74.37	0.01\\
74.38	0.01\\
74.39	0.01\\
74.4	0.01\\
74.41	0.01\\
74.42	0.01\\
74.43	0.01\\
74.44	0.01\\
74.45	0.01\\
74.46	0.01\\
74.47	0.01\\
74.48	0.01\\
74.49	0.01\\
74.5	0.01\\
74.51	0.01\\
74.52	0.01\\
74.53	0.01\\
74.54	0.01\\
74.55	0.01\\
74.56	0.01\\
74.57	0.01\\
74.58	0.01\\
74.59	0.01\\
74.6	0.01\\
74.61	0.01\\
74.62	0.01\\
74.63	0.01\\
74.64	0.01\\
74.65	0.01\\
74.66	0.01\\
74.67	0.01\\
74.68	0.01\\
74.69	0.01\\
74.7	0.01\\
74.71	0.01\\
74.72	0.01\\
74.73	0.01\\
74.74	0.01\\
74.75	0.01\\
74.76	0.01\\
74.77	0.01\\
74.78	0.01\\
74.79	0.01\\
74.8	0.01\\
74.81	0.01\\
74.82	0.01\\
74.83	0.01\\
74.84	0.01\\
74.85	0.01\\
74.86	0.01\\
74.87	0.01\\
74.88	0.01\\
74.89	0.01\\
74.9	0.01\\
74.91	0.01\\
74.92	0.01\\
74.93	0.01\\
74.94	0.01\\
74.95	0.01\\
74.96	0.01\\
74.97	0.01\\
74.98	0.01\\
74.99	0.01\\
75	0.01\\
75.01	0.01\\
75.02	0.01\\
75.03	0.01\\
75.04	0.01\\
75.05	0.01\\
75.06	0.01\\
75.07	0.01\\
75.08	0.01\\
75.09	0.01\\
75.1	0.01\\
75.11	0.01\\
75.12	0.01\\
75.13	0.01\\
75.14	0.01\\
75.15	0.01\\
75.16	0.01\\
75.17	0.01\\
75.18	0.01\\
75.19	0.01\\
75.2	0.01\\
75.21	0.01\\
75.22	0.01\\
75.23	0.01\\
75.24	0.01\\
75.25	0.01\\
75.26	0.01\\
75.27	0.01\\
75.28	0.01\\
75.29	0.01\\
75.3	0.01\\
75.31	0.01\\
75.32	0.01\\
75.33	0.01\\
75.34	0.01\\
75.35	0.01\\
75.36	0.01\\
75.37	0.01\\
75.38	0.01\\
75.39	0.01\\
75.4	0.01\\
75.41	0.01\\
75.42	0.01\\
75.43	0.01\\
75.44	0.01\\
75.45	0.01\\
75.46	0.01\\
75.47	0.01\\
75.48	0.01\\
75.49	0.01\\
75.5	0.01\\
75.51	0.01\\
75.52	0.01\\
75.53	0.01\\
75.54	0.01\\
75.55	0.01\\
75.56	0.01\\
75.57	0.01\\
75.58	0.01\\
75.59	0.01\\
75.6	0.01\\
75.61	0.01\\
75.62	0.01\\
75.63	0.01\\
75.64	0.01\\
75.65	0.01\\
75.66	0.01\\
75.67	0.01\\
75.68	0.01\\
75.69	0.01\\
75.7	0.01\\
75.71	0.01\\
75.72	0.01\\
75.73	0.01\\
75.74	0.01\\
75.75	0.01\\
75.76	0.01\\
75.77	0.01\\
75.78	0.01\\
75.79	0.01\\
75.8	0.01\\
75.81	0.01\\
75.82	0.01\\
75.83	0.01\\
75.84	0.01\\
75.85	0.01\\
75.86	0.01\\
75.87	0.01\\
75.88	0.01\\
75.89	0.01\\
75.9	0.01\\
75.91	0.01\\
75.92	0.01\\
75.93	0.01\\
75.94	0.01\\
75.95	0.01\\
75.96	0.01\\
75.97	0.01\\
75.98	0.01\\
75.99	0.01\\
76	0.01\\
76.01	0.01\\
76.02	0.01\\
76.03	0.01\\
76.04	0.01\\
76.05	0.01\\
76.06	0.01\\
76.07	0.01\\
76.08	0.01\\
76.09	0.01\\
76.1	0.01\\
76.11	0.01\\
76.12	0.01\\
76.13	0.01\\
76.14	0.01\\
76.15	0.01\\
76.16	0.01\\
76.17	0.01\\
76.18	0.01\\
76.19	0.01\\
76.2	0.01\\
76.21	0.01\\
76.22	0.01\\
76.23	0.01\\
76.24	0.01\\
76.25	0.01\\
76.26	0.01\\
76.27	0.01\\
76.28	0.01\\
76.29	0.01\\
76.3	0.01\\
76.31	0.01\\
76.32	0.01\\
76.33	0.01\\
76.34	0.01\\
76.35	0.01\\
76.36	0.01\\
76.37	0.01\\
76.38	0.01\\
76.39	0.01\\
76.4	0.01\\
76.41	0.01\\
76.42	0.01\\
76.43	0.01\\
76.44	0.01\\
76.45	0.01\\
76.46	0.01\\
76.47	0.01\\
76.48	0.01\\
76.49	0.01\\
76.5	0.01\\
76.51	0.01\\
76.52	0.01\\
76.53	0.01\\
76.54	0.01\\
76.55	0.01\\
76.56	0.01\\
76.57	0.01\\
76.58	0.01\\
76.59	0.01\\
76.6	0.01\\
76.61	0.01\\
76.62	0.01\\
76.63	0.01\\
76.64	0.01\\
76.65	0.01\\
76.66	0.01\\
76.67	0.01\\
76.68	0.01\\
76.69	0.01\\
76.7	0.01\\
76.71	0.01\\
76.72	0.01\\
76.73	0.01\\
76.74	0.01\\
76.75	0.01\\
76.76	0.01\\
76.77	0.01\\
76.78	0.01\\
76.79	0.01\\
76.8	0.01\\
76.81	0.01\\
76.82	0.01\\
76.83	0.01\\
76.84	0.01\\
76.85	0.01\\
76.86	0.01\\
76.87	0.01\\
76.88	0.01\\
76.89	0.01\\
76.9	0.01\\
76.91	0.01\\
76.92	0.01\\
76.93	0.01\\
76.94	0.01\\
76.95	0.01\\
76.96	0.01\\
76.97	0.01\\
76.98	0.01\\
76.99	0.01\\
77	0.01\\
77.01	0.01\\
77.02	0.01\\
77.03	0.01\\
77.04	0.01\\
77.05	0.01\\
77.06	0.01\\
77.07	0.01\\
77.08	0.01\\
77.09	0.01\\
77.1	0.01\\
77.11	0.01\\
77.12	0.01\\
77.13	0.01\\
77.14	0.01\\
77.15	0.01\\
77.16	0.01\\
77.17	0.01\\
77.18	0.01\\
77.19	0.01\\
77.2	0.01\\
77.21	0.01\\
77.22	0.01\\
77.23	0.01\\
77.24	0.01\\
77.25	0.01\\
77.26	0.01\\
77.27	0.01\\
77.28	0.01\\
77.29	0.01\\
77.3	0.01\\
77.31	0.01\\
77.32	0.01\\
77.33	0.01\\
77.34	0.01\\
77.35	0.01\\
77.36	0.01\\
77.37	0.01\\
77.38	0.01\\
77.39	0.01\\
77.4	0.01\\
77.41	0.01\\
77.42	0.01\\
77.43	0.01\\
77.44	0.01\\
77.45	0.01\\
77.46	0.01\\
77.47	0.01\\
77.48	0.01\\
77.49	0.01\\
77.5	0.01\\
77.51	0.01\\
77.52	0.01\\
77.53	0.01\\
77.54	0.01\\
77.55	0.01\\
77.56	0.01\\
77.57	0.01\\
77.58	0.01\\
77.59	0.01\\
77.6	0.01\\
77.61	0.01\\
77.62	0.01\\
77.63	0.01\\
77.64	0.01\\
77.65	0.01\\
77.66	0.01\\
77.67	0.01\\
77.68	0.01\\
77.69	0.01\\
77.7	0.01\\
77.71	0.01\\
77.72	0.01\\
77.73	0.01\\
77.74	0.01\\
77.75	0.01\\
77.76	0.01\\
77.77	0.01\\
77.78	0.01\\
77.79	0.01\\
77.8	0.01\\
77.81	0.01\\
77.82	0.01\\
77.83	0.01\\
77.84	0.01\\
77.85	0.01\\
77.86	0.01\\
77.87	0.01\\
77.88	0.01\\
77.89	0.01\\
77.9	0.01\\
77.91	0.01\\
77.92	0.01\\
77.93	0.01\\
77.94	0.01\\
77.95	0.01\\
77.96	0.01\\
77.97	0.01\\
77.98	0.01\\
77.99	0.01\\
78	0.01\\
78.01	0.01\\
78.02	0.01\\
78.03	0.01\\
78.04	0.01\\
78.05	0.01\\
78.06	0.01\\
78.07	0.01\\
78.08	0.01\\
78.09	0.01\\
78.1	0.01\\
78.11	0.01\\
78.12	0.01\\
78.13	0.01\\
78.14	0.01\\
78.15	0.01\\
78.16	0.01\\
78.17	0.01\\
78.18	0.01\\
78.19	0.01\\
78.2	0.01\\
78.21	0.01\\
78.22	0.01\\
78.23	0.01\\
78.24	0.01\\
78.25	0.01\\
78.26	0.01\\
78.27	0.01\\
78.28	0.01\\
78.29	0.01\\
78.3	0.01\\
78.31	0.01\\
78.32	0.01\\
78.33	0.01\\
78.34	0.01\\
78.35	0.01\\
78.36	0.01\\
78.37	0.01\\
78.38	0.01\\
78.39	0.01\\
78.4	0.01\\
78.41	0.01\\
78.42	0.01\\
78.43	0.01\\
78.44	0.01\\
78.45	0.01\\
78.46	0.01\\
78.47	0.01\\
78.48	0.01\\
78.49	0.01\\
78.5	0.01\\
78.51	0.01\\
78.52	0.01\\
78.53	0.01\\
78.54	0.01\\
78.55	0.01\\
78.56	0.01\\
78.57	0.01\\
78.58	0.01\\
78.59	0.01\\
78.6	0.01\\
78.61	0.01\\
78.62	0.01\\
78.63	0.01\\
78.64	0.01\\
78.65	0.01\\
78.66	0.01\\
78.67	0.01\\
78.68	0.01\\
78.69	0.01\\
78.7	0.01\\
78.71	0.01\\
78.72	0.01\\
78.73	0.01\\
78.74	0.01\\
78.75	0.01\\
78.76	0.01\\
78.77	0.01\\
78.78	0.01\\
78.79	0.01\\
78.8	0.01\\
78.81	0.01\\
78.82	0.01\\
78.83	0.01\\
78.84	0.01\\
78.85	0.01\\
78.86	0.01\\
78.87	0.01\\
78.88	0.01\\
78.89	0.01\\
78.9	0.01\\
78.91	0.01\\
78.92	0.01\\
78.93	0.01\\
78.94	0.01\\
78.95	0.01\\
78.96	0.01\\
78.97	0.01\\
78.98	0.01\\
78.99	0.01\\
79	0.01\\
79.01	0.01\\
79.02	0.01\\
79.03	0.01\\
79.04	0.01\\
79.05	0.01\\
79.06	0.01\\
79.07	0.01\\
79.08	0.01\\
79.09	0.01\\
79.1	0.01\\
79.11	0.01\\
79.12	0.01\\
79.13	0.01\\
79.14	0.01\\
79.15	0.01\\
79.16	0.01\\
79.17	0.01\\
79.18	0.01\\
79.19	0.01\\
79.2	0.01\\
79.21	0.01\\
79.22	0.01\\
79.23	0.01\\
79.24	0.01\\
79.25	0.01\\
79.26	0.01\\
79.27	0.01\\
79.28	0.01\\
79.29	0.01\\
79.3	0.01\\
79.31	0.01\\
79.32	0.01\\
79.33	0.01\\
79.34	0.01\\
79.35	0.01\\
79.36	0.01\\
79.37	0.01\\
79.38	0.01\\
79.39	0.01\\
79.4	0.01\\
79.41	0.01\\
79.42	0.01\\
79.43	0.01\\
79.44	0.01\\
79.45	0.01\\
79.46	0.01\\
79.47	0.01\\
79.48	0.01\\
79.49	0.01\\
79.5	0.01\\
79.51	0.01\\
79.52	0.01\\
79.53	0.01\\
79.54	0.01\\
79.55	0.01\\
79.56	0.01\\
79.57	0.01\\
79.58	0.01\\
79.59	0.01\\
79.6	0.01\\
79.61	0.01\\
79.62	0.01\\
79.63	0.01\\
79.64	0.01\\
79.65	0.01\\
79.66	0.01\\
79.67	0.01\\
79.68	0.01\\
79.69	0.01\\
79.7	0.01\\
79.71	0.01\\
79.72	0.01\\
79.73	0.01\\
79.74	0.01\\
79.75	0.01\\
79.76	0.01\\
79.77	0.01\\
79.78	0.01\\
79.79	0.01\\
79.8	0.01\\
79.81	0.01\\
79.82	0.01\\
79.83	0.01\\
79.84	0.01\\
79.85	0.01\\
79.86	0.01\\
79.87	0.01\\
79.88	0.01\\
79.89	0.01\\
79.9	0.01\\
79.91	0.01\\
79.92	0.01\\
79.93	0.01\\
79.94	0.01\\
79.95	0.01\\
79.96	0.01\\
79.97	0.01\\
79.98	0.01\\
79.99	0.01\\
80	0.01\\
80.01	0.01\\
};
\addplot [color=red,dashed]
  table[row sep=crcr]{%
80.01	0.01\\
80.02	0.01\\
80.03	0.01\\
80.04	0.01\\
80.05	0.01\\
80.06	0.01\\
80.07	0.01\\
80.08	0.01\\
80.09	0.01\\
80.1	0.01\\
80.11	0.01\\
80.12	0.01\\
80.13	0.01\\
80.14	0.01\\
80.15	0.01\\
80.16	0.01\\
80.17	0.01\\
80.18	0.01\\
80.19	0.01\\
80.2	0.01\\
80.21	0.01\\
80.22	0.01\\
80.23	0.01\\
80.24	0.01\\
80.25	0.01\\
80.26	0.01\\
80.27	0.01\\
80.28	0.01\\
80.29	0.01\\
80.3	0.01\\
80.31	0.01\\
80.32	0.01\\
80.33	0.01\\
80.34	0.01\\
80.35	0.01\\
80.36	0.01\\
80.37	0.01\\
80.38	0.01\\
80.39	0.01\\
80.4	0.01\\
80.41	0.01\\
80.42	0.01\\
80.43	0.01\\
80.44	0.01\\
80.45	0.01\\
80.46	0.01\\
80.47	0.01\\
80.48	0.01\\
80.49	0.01\\
80.5	0.01\\
80.51	0.01\\
80.52	0.01\\
80.53	0.01\\
80.54	0.01\\
80.55	0.01\\
80.56	0.01\\
80.57	0.01\\
80.58	0.01\\
80.59	0.01\\
80.6	0.01\\
80.61	0.01\\
80.62	0.01\\
80.63	0.01\\
80.64	0.01\\
80.65	0.01\\
80.66	0.01\\
80.67	0.01\\
80.68	0.01\\
80.69	0.01\\
80.7	0.01\\
80.71	0.01\\
80.72	0.01\\
80.73	0.01\\
80.74	0.01\\
80.75	0.01\\
80.76	0.01\\
80.77	0.01\\
80.78	0.01\\
80.79	0.01\\
80.8	0.01\\
80.81	0.01\\
80.82	0.01\\
80.83	0.01\\
80.84	0.01\\
80.85	0.01\\
80.86	0.01\\
80.87	0.01\\
80.88	0.01\\
80.89	0.01\\
80.9	0.01\\
80.91	0.01\\
80.92	0.01\\
80.93	0.01\\
80.94	0.01\\
80.95	0.01\\
80.96	0.01\\
80.97	0.01\\
80.98	0.01\\
80.99	0.01\\
81	0.01\\
81.01	0.01\\
81.02	0.01\\
81.03	0.01\\
81.04	0.01\\
81.05	0.01\\
81.06	0.01\\
81.07	0.01\\
81.08	0.01\\
81.09	0.01\\
81.1	0.01\\
81.11	0.01\\
81.12	0.01\\
81.13	0.01\\
81.14	0.01\\
81.15	0.01\\
81.16	0.01\\
81.17	0.01\\
81.18	0.01\\
81.19	0.01\\
81.2	0.01\\
81.21	0.01\\
81.22	0.01\\
81.23	0.01\\
81.24	0.01\\
81.25	0.01\\
81.26	0.01\\
81.27	0.01\\
81.28	0.01\\
81.29	0.01\\
81.3	0.01\\
81.31	0.01\\
81.32	0.01\\
81.33	0.01\\
81.34	0.01\\
81.35	0.01\\
81.36	0.01\\
81.37	0.01\\
81.38	0.01\\
81.39	0.01\\
81.4	0.01\\
81.41	0.01\\
81.42	0.01\\
81.43	0.01\\
81.44	0.01\\
81.45	0.01\\
81.46	0.01\\
81.47	0.01\\
81.48	0.01\\
81.49	0.01\\
81.5	0.01\\
81.51	0.01\\
81.52	0.01\\
81.53	0.01\\
81.54	0.01\\
81.55	0.01\\
81.56	0.01\\
81.57	0.01\\
81.58	0.01\\
81.59	0.01\\
81.6	0.01\\
81.61	0.01\\
81.62	0.01\\
81.63	0.01\\
81.64	0.01\\
81.65	0.01\\
81.66	0.01\\
81.67	0.01\\
81.68	0.01\\
81.69	0.01\\
81.7	0.01\\
81.71	0.01\\
81.72	0.01\\
81.73	0.01\\
81.74	0.01\\
81.75	0.01\\
81.76	0.01\\
81.77	0.01\\
81.78	0.01\\
81.79	0.01\\
81.8	0.01\\
81.81	0.01\\
81.82	0.01\\
81.83	0.01\\
81.84	0.01\\
81.85	0.01\\
81.86	0.01\\
81.87	0.01\\
81.88	0.01\\
81.89	0.01\\
81.9	0.01\\
81.91	0.01\\
81.92	0.01\\
81.93	0.01\\
81.94	0.01\\
81.95	0.01\\
81.96	0.01\\
81.97	0.01\\
81.98	0.01\\
81.99	0.01\\
82	0.01\\
82.01	0.01\\
82.02	0.01\\
82.03	0.01\\
82.04	0.01\\
82.05	0.01\\
82.06	0.01\\
82.07	0.01\\
82.08	0.01\\
82.09	0.01\\
82.1	0.01\\
82.11	0.01\\
82.12	0.01\\
82.13	0.01\\
82.14	0.01\\
82.15	0.01\\
82.16	0.01\\
82.17	0.01\\
82.18	0.01\\
82.19	0.01\\
82.2	0.01\\
82.21	0.01\\
82.22	0.01\\
82.23	0.01\\
82.24	0.01\\
82.25	0.01\\
82.26	0.01\\
82.27	0.01\\
82.28	0.01\\
82.29	0.01\\
82.3	0.01\\
82.31	0.01\\
82.32	0.01\\
82.33	0.01\\
82.34	0.01\\
82.35	0.01\\
82.36	0.01\\
82.37	0.01\\
82.38	0.01\\
82.39	0.01\\
82.4	0.01\\
82.41	0.01\\
82.42	0.01\\
82.43	0.01\\
82.44	0.01\\
82.45	0.01\\
82.46	0.01\\
82.47	0.01\\
82.48	0.01\\
82.49	0.01\\
82.5	0.01\\
82.51	0.01\\
82.52	0.01\\
82.53	0.01\\
82.54	0.01\\
82.55	0.01\\
82.56	0.01\\
82.57	0.01\\
82.58	0.01\\
82.59	0.01\\
82.6	0.01\\
82.61	0.01\\
82.62	0.01\\
82.63	0.01\\
82.64	0.01\\
82.65	0.01\\
82.66	0.01\\
82.67	0.01\\
82.68	0.01\\
82.69	0.01\\
82.7	0.01\\
82.71	0.01\\
82.72	0.01\\
82.73	0.01\\
82.74	0.01\\
82.75	0.01\\
82.76	0.01\\
82.77	0.01\\
82.78	0.01\\
82.79	0.01\\
82.8	0.01\\
82.81	0.01\\
82.82	0.01\\
82.83	0.01\\
82.84	0.01\\
82.85	0.01\\
82.86	0.01\\
82.87	0.01\\
82.88	0.01\\
82.89	0.01\\
82.9	0.01\\
82.91	0.01\\
82.92	0.01\\
82.93	0.01\\
82.94	0.01\\
82.95	0.01\\
82.96	0.01\\
82.97	0.01\\
82.98	0.01\\
82.99	0.01\\
83	0.01\\
83.01	0.01\\
83.02	0.01\\
83.03	0.01\\
83.04	0.01\\
83.05	0.01\\
83.06	0.01\\
83.07	0.01\\
83.08	0.01\\
83.09	0.01\\
83.1	0.01\\
83.11	0.01\\
83.12	0.01\\
83.13	0.01\\
83.14	0.01\\
83.15	0.01\\
83.16	0.01\\
83.17	0.01\\
83.18	0.01\\
83.19	0.01\\
83.2	0.01\\
83.21	0.01\\
83.22	0.01\\
83.23	0.01\\
83.24	0.01\\
83.25	0.01\\
83.26	0.01\\
83.27	0.01\\
83.28	0.01\\
83.29	0.01\\
83.3	0.01\\
83.31	0.01\\
83.32	0.01\\
83.33	0.01\\
83.34	0.01\\
83.35	0.01\\
83.36	0.01\\
83.37	0.01\\
83.38	0.01\\
83.39	0.01\\
83.4	0.01\\
83.41	0.01\\
83.42	0.01\\
83.43	0.01\\
83.44	0.01\\
83.45	0.01\\
83.46	0.01\\
83.47	0.01\\
83.48	0.01\\
83.49	0.01\\
83.5	0.01\\
83.51	0.01\\
83.52	0.01\\
83.53	0.01\\
83.54	0.01\\
83.55	0.01\\
83.56	0.01\\
83.57	0.01\\
83.58	0.01\\
83.59	0.01\\
83.6	0.01\\
83.61	0.01\\
83.62	0.01\\
83.63	0.01\\
83.64	0.01\\
83.65	0.01\\
83.66	0.01\\
83.67	0.01\\
83.68	0.01\\
83.69	0.01\\
83.7	0.01\\
83.71	0.01\\
83.72	0.01\\
83.73	0.01\\
83.74	0.01\\
83.75	0.01\\
83.76	0.01\\
83.77	0.01\\
83.78	0.01\\
83.79	0.01\\
83.8	0.01\\
83.81	0.01\\
83.82	0.01\\
83.83	0.01\\
83.84	0.01\\
83.85	0.01\\
83.86	0.01\\
83.87	0.01\\
83.88	0.01\\
83.89	0.01\\
83.9	0.01\\
83.91	0.01\\
83.92	0.01\\
83.93	0.01\\
83.94	0.01\\
83.95	0.01\\
83.96	0.01\\
83.97	0.01\\
83.98	0.01\\
83.99	0.01\\
84	0.01\\
84.01	0.01\\
84.02	0.01\\
84.03	0.01\\
84.04	0.01\\
84.05	0.01\\
84.06	0.01\\
84.07	0.01\\
84.08	0.01\\
84.09	0.01\\
84.1	0.01\\
84.11	0.01\\
84.12	0.01\\
84.13	0.01\\
84.14	0.01\\
84.15	0.01\\
84.16	0.01\\
84.17	0.01\\
84.18	0.01\\
84.19	0.01\\
84.2	0.01\\
84.21	0.01\\
84.22	0.01\\
84.23	0.01\\
84.24	0.01\\
84.25	0.01\\
84.26	0.01\\
84.27	0.01\\
84.28	0.01\\
84.29	0.01\\
84.3	0.01\\
84.31	0.01\\
84.32	0.01\\
84.33	0.01\\
84.34	0.01\\
84.35	0.01\\
84.36	0.01\\
84.37	0.01\\
84.38	0.01\\
84.39	0.01\\
84.4	0.01\\
84.41	0.01\\
84.42	0.01\\
84.43	0.01\\
84.44	0.01\\
84.45	0.01\\
84.46	0.01\\
84.47	0.01\\
84.48	0.01\\
84.49	0.01\\
84.5	0.01\\
84.51	0.01\\
84.52	0.01\\
84.53	0.01\\
84.54	0.01\\
84.55	0.01\\
84.56	0.01\\
84.57	0.01\\
84.58	0.01\\
84.59	0.01\\
84.6	0.01\\
84.61	0.01\\
84.62	0.01\\
84.63	0.01\\
84.64	0.01\\
84.65	0.01\\
84.66	0.01\\
84.67	0.01\\
84.68	0.01\\
84.69	0.01\\
84.7	0.01\\
84.71	0.01\\
84.72	0.01\\
84.73	0.01\\
84.74	0.01\\
84.75	0.01\\
84.76	0.01\\
84.77	0.01\\
84.78	0.01\\
84.79	0.01\\
84.8	0.01\\
84.81	0.01\\
84.82	0.01\\
84.83	0.01\\
84.84	0.01\\
84.85	0.01\\
84.86	0.01\\
84.87	0.01\\
84.88	0.01\\
84.89	0.01\\
84.9	0.01\\
84.91	0.01\\
84.92	0.01\\
84.93	0.01\\
84.94	0.01\\
84.95	0.01\\
84.96	0.01\\
84.97	0.01\\
84.98	0.01\\
84.99	0.01\\
85	0.01\\
85.01	0.01\\
85.02	0.01\\
85.03	0.01\\
85.04	0.01\\
85.05	0.01\\
85.06	0.01\\
85.07	0.01\\
85.08	0.01\\
85.09	0.01\\
85.1	0.01\\
85.11	0.01\\
85.12	0.01\\
85.13	0.01\\
85.14	0.01\\
85.15	0.01\\
85.16	0.01\\
85.17	0.01\\
85.18	0.01\\
85.19	0.01\\
85.2	0.01\\
85.21	0.01\\
85.22	0.01\\
85.23	0.01\\
85.24	0.01\\
85.25	0.01\\
85.26	0.01\\
85.27	0.01\\
85.28	0.01\\
85.29	0.01\\
85.3	0.01\\
85.31	0.01\\
85.32	0.01\\
85.33	0.01\\
85.34	0.01\\
85.35	0.01\\
85.36	0.01\\
85.37	0.01\\
85.38	0.01\\
85.39	0.01\\
85.4	0.01\\
85.41	0.01\\
85.42	0.01\\
85.43	0.01\\
85.44	0.01\\
85.45	0.01\\
85.46	0.01\\
85.47	0.01\\
85.48	0.01\\
85.49	0.01\\
85.5	0.01\\
85.51	0.01\\
85.52	0.01\\
85.53	0.01\\
85.54	0.01\\
85.55	0.01\\
85.56	0.01\\
85.57	0.01\\
85.58	0.01\\
85.59	0.01\\
85.6	0.01\\
85.61	0.01\\
85.62	0.01\\
85.63	0.01\\
85.64	0.01\\
85.65	0.01\\
85.66	0.01\\
85.67	0.01\\
85.68	0.01\\
85.69	0.01\\
85.7	0.01\\
85.71	0.01\\
85.72	0.01\\
85.73	0.01\\
85.74	0.01\\
85.75	0.01\\
85.76	0.01\\
85.77	0.01\\
85.78	0.01\\
85.79	0.01\\
85.8	0.01\\
85.81	0.01\\
85.82	0.01\\
85.83	0.01\\
85.84	0.01\\
85.85	0.01\\
85.86	0.01\\
85.87	0.01\\
85.88	0.01\\
85.89	0.01\\
85.9	0.01\\
85.91	0.01\\
85.92	0.01\\
85.93	0.01\\
85.94	0.01\\
85.95	0.01\\
85.96	0.01\\
85.97	0.01\\
85.98	0.01\\
85.99	0.01\\
86	0.01\\
86.01	0.01\\
86.02	0.01\\
86.03	0.01\\
86.04	0.01\\
86.05	0.01\\
86.06	0.01\\
86.07	0.01\\
86.08	0.01\\
86.09	0.01\\
86.1	0.01\\
86.11	0.01\\
86.12	0.01\\
86.13	0.01\\
86.14	0.01\\
86.15	0.01\\
86.16	0.01\\
86.17	0.01\\
86.18	0.01\\
86.19	0.01\\
86.2	0.01\\
86.21	0.01\\
86.22	0.01\\
86.23	0.01\\
86.24	0.01\\
86.25	0.01\\
86.26	0.01\\
86.27	0.01\\
86.28	0.01\\
86.29	0.01\\
86.3	0.01\\
86.31	0.01\\
86.32	0.01\\
86.33	0.01\\
86.34	0.01\\
86.35	0.01\\
86.36	0.01\\
86.37	0.01\\
86.38	0.01\\
86.39	0.01\\
86.4	0.01\\
86.41	0.01\\
86.42	0.01\\
86.43	0.01\\
86.44	0.01\\
86.45	0.01\\
86.46	0.01\\
86.47	0.01\\
86.48	0.01\\
86.49	0.01\\
86.5	0.01\\
86.51	0.01\\
86.52	0.01\\
86.53	0.01\\
86.54	0.01\\
86.55	0.01\\
86.56	0.01\\
86.57	0.01\\
86.58	0.01\\
86.59	0.01\\
86.6	0.01\\
86.61	0.01\\
86.62	0.01\\
86.63	0.01\\
86.64	0.01\\
86.65	0.01\\
86.66	0.01\\
86.67	0.01\\
86.68	0.01\\
86.69	0.01\\
86.7	0.01\\
86.71	0.01\\
86.72	0.01\\
86.73	0.01\\
86.74	0.01\\
86.75	0.01\\
86.76	0.01\\
86.77	0.01\\
86.78	0.01\\
86.79	0.01\\
86.8	0.01\\
86.81	0.01\\
86.82	0.01\\
86.83	0.01\\
86.84	0.01\\
86.85	0.01\\
86.86	0.01\\
86.87	0.01\\
86.88	0.01\\
86.89	0.01\\
86.9	0.01\\
86.91	0.01\\
86.92	0.01\\
86.93	0.01\\
86.94	0.01\\
86.95	0.01\\
86.96	0.01\\
86.97	0.01\\
86.98	0.01\\
86.99	0.01\\
87	0.01\\
87.01	0.01\\
87.02	0.01\\
87.03	0.01\\
87.04	0.01\\
87.05	0.01\\
87.06	0.01\\
87.07	0.01\\
87.08	0.01\\
87.09	0.01\\
87.1	0.01\\
87.11	0.01\\
87.12	0.01\\
87.13	0.01\\
87.14	0.01\\
87.15	0.01\\
87.16	0.01\\
87.17	0.01\\
87.18	0.01\\
87.19	0.01\\
87.2	0.01\\
87.21	0.01\\
87.22	0.01\\
87.23	0.01\\
87.24	0.01\\
87.25	0.01\\
87.26	0.01\\
87.27	0.01\\
87.28	0.01\\
87.29	0.01\\
87.3	0.01\\
87.31	0.01\\
87.32	0.01\\
87.33	0.01\\
87.34	0.01\\
87.35	0.01\\
87.36	0.01\\
87.37	0.01\\
87.38	0.01\\
87.39	0.01\\
87.4	0.01\\
87.41	0.01\\
87.42	0.01\\
87.43	0.01\\
87.44	0.01\\
87.45	0.01\\
87.46	0.01\\
87.47	0.01\\
87.48	0.01\\
87.49	0.01\\
87.5	0.01\\
87.51	0.01\\
87.52	0.01\\
87.53	0.01\\
87.54	0.01\\
87.55	0.01\\
87.56	0.01\\
87.57	0.01\\
87.58	0.01\\
87.59	0.01\\
87.6	0.01\\
87.61	0.01\\
87.62	0.01\\
87.63	0.01\\
87.64	0.01\\
87.65	0.01\\
87.66	0.01\\
87.67	0.01\\
87.68	0.01\\
87.69	0.01\\
87.7	0.01\\
87.71	0.01\\
87.72	0.01\\
87.73	0.01\\
87.74	0.01\\
87.75	0.01\\
87.76	0.01\\
87.77	0.01\\
87.78	0.01\\
87.79	0.01\\
87.8	0.01\\
87.81	0.01\\
87.82	0.01\\
87.83	0.01\\
87.84	0.01\\
87.85	0.01\\
87.86	0.01\\
87.87	0.01\\
87.88	0.01\\
87.89	0.01\\
87.9	0.01\\
87.91	0.01\\
87.92	0.01\\
87.93	0.01\\
87.94	0.01\\
87.95	0.01\\
87.96	0.01\\
87.97	0.01\\
87.98	0.01\\
87.99	0.01\\
88	0.01\\
88.01	0.01\\
88.02	0.01\\
88.03	0.01\\
88.04	0.01\\
88.05	0.01\\
88.06	0.01\\
88.07	0.01\\
88.08	0.01\\
88.09	0.01\\
88.1	0.01\\
88.11	0.01\\
88.12	0.01\\
88.13	0.01\\
88.14	0.01\\
88.15	0.01\\
88.16	0.01\\
88.17	0.01\\
88.18	0.01\\
88.19	0.01\\
88.2	0.01\\
88.21	0.01\\
88.22	0.01\\
88.23	0.01\\
88.24	0.01\\
88.25	0.01\\
88.26	0.01\\
88.27	0.01\\
88.28	0.01\\
88.29	0.01\\
88.3	0.01\\
88.31	0.01\\
88.32	0.01\\
88.33	0.01\\
88.34	0.01\\
88.35	0.01\\
88.36	0.01\\
88.37	0.01\\
88.38	0.01\\
88.39	0.01\\
88.4	0.01\\
88.41	0.01\\
88.42	0.01\\
88.43	0.01\\
88.44	0.01\\
88.45	0.01\\
88.46	0.01\\
88.47	0.01\\
88.48	0.01\\
88.49	0.01\\
88.5	0.01\\
88.51	0.01\\
88.52	0.01\\
88.53	0.01\\
88.54	0.01\\
88.55	0.01\\
88.56	0.01\\
88.57	0.01\\
88.58	0.01\\
88.59	0.01\\
88.6	0.01\\
88.61	0.01\\
88.62	0.01\\
88.63	0.01\\
88.64	0.01\\
88.65	0.01\\
88.66	0.01\\
88.67	0.01\\
88.68	0.01\\
88.69	0.01\\
88.7	0.01\\
88.71	0.01\\
88.72	0.01\\
88.73	0.01\\
88.74	0.01\\
88.75	0.01\\
88.76	0.01\\
88.77	0.01\\
88.78	0.01\\
88.79	0.01\\
88.8	0.01\\
88.81	0.01\\
88.82	0.01\\
88.83	0.01\\
88.84	0.01\\
88.85	0.01\\
88.86	0.01\\
88.87	0.01\\
88.88	0.01\\
88.89	0.01\\
88.9	0.01\\
88.91	0.01\\
88.92	0.01\\
88.93	0.01\\
88.94	0.01\\
88.95	0.01\\
88.96	0.01\\
88.97	0.01\\
88.98	0.01\\
88.99	0.01\\
89	0.01\\
89.01	0.01\\
89.02	0.01\\
89.03	0.01\\
89.04	0.01\\
89.05	0.01\\
89.06	0.01\\
89.07	0.01\\
89.08	0.01\\
89.09	0.01\\
89.1	0.01\\
89.11	0.01\\
89.12	0.01\\
89.13	0.01\\
89.14	0.01\\
89.15	0.01\\
89.16	0.01\\
89.17	0.01\\
89.18	0.01\\
89.19	0.01\\
89.2	0.01\\
89.21	0.01\\
89.22	0.01\\
89.23	0.01\\
89.24	0.01\\
89.25	0.01\\
89.26	0.01\\
89.27	0.01\\
89.28	0.01\\
89.29	0.01\\
89.3	0.01\\
89.31	0.01\\
89.32	0.01\\
89.33	0.01\\
89.34	0.01\\
89.35	0.01\\
89.36	0.01\\
89.37	0.01\\
89.38	0.01\\
89.39	0.01\\
89.4	0.01\\
89.41	0.01\\
89.42	0.01\\
89.43	0.01\\
89.44	0.01\\
89.45	0.01\\
89.46	0.01\\
89.47	0.01\\
89.48	0.01\\
89.49	0.01\\
89.5	0.01\\
89.51	0.01\\
89.52	0.01\\
89.53	0.01\\
89.54	0.01\\
89.55	0.01\\
89.56	0.01\\
89.57	0.01\\
89.58	0.01\\
89.59	0.01\\
89.6	0.01\\
89.61	0.01\\
89.62	0.01\\
89.63	0.01\\
89.64	0.01\\
89.65	0.01\\
89.66	0.01\\
89.67	0.01\\
89.68	0.01\\
89.69	0.01\\
89.7	0.01\\
89.71	0.01\\
89.72	0.01\\
89.73	0.01\\
89.74	0.01\\
89.75	0.01\\
89.76	0.01\\
89.77	0.01\\
89.78	0.01\\
89.79	0.01\\
89.8	0.01\\
89.81	0.01\\
89.82	0.01\\
89.83	0.01\\
89.84	0.01\\
89.85	0.01\\
89.86	0.01\\
89.87	0.01\\
89.88	0.01\\
89.89	0.01\\
89.9	0.01\\
89.91	0.01\\
89.92	0.01\\
89.93	0.01\\
89.94	0.01\\
89.95	0.01\\
89.96	0.01\\
89.97	0.01\\
89.98	0.01\\
89.99	0.01\\
90	0.01\\
90.01	0.01\\
90.02	0.01\\
90.03	0.01\\
90.04	0.01\\
90.05	0.01\\
90.06	0.01\\
90.07	0.01\\
90.08	0.01\\
90.09	0.01\\
90.1	0.01\\
90.11	0.01\\
90.12	0.01\\
90.13	0.01\\
90.14	0.01\\
90.15	0.01\\
90.16	0.01\\
90.17	0.01\\
90.18	0.01\\
90.19	0.01\\
90.2	0.01\\
90.21	0.01\\
90.22	0.01\\
90.23	0.01\\
90.24	0.01\\
90.25	0.01\\
90.26	0.01\\
90.27	0.01\\
90.28	0.01\\
90.29	0.01\\
90.3	0.01\\
90.31	0.01\\
90.32	0.01\\
90.33	0.01\\
90.34	0.01\\
90.35	0.01\\
90.36	0.01\\
90.37	0.01\\
90.38	0.01\\
90.39	0.01\\
90.4	0.01\\
90.41	0.01\\
90.42	0.01\\
90.43	0.01\\
90.44	0.01\\
90.45	0.01\\
90.46	0.01\\
90.47	0.01\\
90.48	0.01\\
90.49	0.01\\
90.5	0.01\\
90.51	0.01\\
90.52	0.01\\
90.53	0.01\\
90.54	0.01\\
90.55	0.01\\
90.56	0.01\\
90.57	0.01\\
90.58	0.01\\
90.59	0.01\\
90.6	0.01\\
90.61	0.01\\
90.62	0.01\\
90.63	0.01\\
90.64	0.01\\
90.65	0.01\\
90.66	0.01\\
90.67	0.01\\
90.68	0.01\\
90.69	0.01\\
90.7	0.01\\
90.71	0.01\\
90.72	0.01\\
90.73	0.01\\
90.74	0.01\\
90.75	0.01\\
90.76	0.01\\
90.77	0.01\\
90.78	0.01\\
90.79	0.01\\
90.8	0.01\\
90.81	0.01\\
90.82	0.01\\
90.83	0.01\\
90.84	0.01\\
90.85	0.01\\
90.86	0.01\\
90.87	0.01\\
90.88	0.01\\
90.89	0.01\\
90.9	0.01\\
90.91	0.01\\
90.92	0.01\\
90.93	0.01\\
90.94	0.01\\
90.95	0.01\\
90.96	0.01\\
90.97	0.01\\
90.98	0.01\\
90.99	0.01\\
91	0.01\\
91.01	0.01\\
91.02	0.01\\
91.03	0.01\\
91.04	0.01\\
91.05	0.01\\
91.06	0.01\\
91.07	0.01\\
91.08	0.01\\
91.09	0.01\\
91.1	0.01\\
91.11	0.01\\
91.12	0.01\\
91.13	0.01\\
91.14	0.01\\
91.15	0.01\\
91.16	0.01\\
91.17	0.01\\
91.18	0.01\\
91.19	0.01\\
91.2	0.01\\
91.21	0.01\\
91.22	0.01\\
91.23	0.01\\
91.24	0.01\\
91.25	0.01\\
91.26	0.01\\
91.27	0.01\\
91.28	0.01\\
91.29	0.01\\
91.3	0.01\\
91.31	0.01\\
91.32	0.01\\
91.33	0.01\\
91.34	0.01\\
91.35	0.01\\
91.36	0.01\\
91.37	0.01\\
91.38	0.01\\
91.39	0.01\\
91.4	0.01\\
91.41	0.01\\
91.42	0.01\\
91.43	0.01\\
91.44	0.01\\
91.45	0.01\\
91.46	0.01\\
91.47	0.01\\
91.48	0.01\\
91.49	0.01\\
91.5	0.01\\
91.51	0.01\\
91.52	0.01\\
91.53	0.01\\
91.54	0.01\\
91.55	0.01\\
91.56	0.01\\
91.57	0.01\\
91.58	0.01\\
91.59	0.01\\
91.6	0.01\\
91.61	0.01\\
91.62	0.01\\
91.63	0.01\\
91.64	0.01\\
91.65	0.01\\
91.66	0.01\\
91.67	0.01\\
91.68	0.01\\
91.69	0.01\\
91.7	0.01\\
91.71	0.01\\
91.72	0.01\\
91.73	0.01\\
91.74	0.01\\
91.75	0.01\\
91.76	0.01\\
91.77	0.01\\
91.78	0.01\\
91.79	0.01\\
91.8	0.01\\
91.81	0.01\\
91.82	0.01\\
91.83	0.01\\
91.84	0.01\\
91.85	0.01\\
91.86	0.01\\
91.87	0.01\\
91.88	0.01\\
91.89	0.01\\
91.9	0.01\\
91.91	0.01\\
91.92	0.01\\
91.93	0.01\\
91.94	0.01\\
91.95	0.01\\
91.96	0.01\\
91.97	0.01\\
91.98	0.01\\
91.99	0.01\\
92	0.01\\
92.01	0.01\\
92.02	0.01\\
92.03	0.01\\
92.04	0.01\\
92.05	0.01\\
92.06	0.01\\
92.07	0.01\\
92.08	0.01\\
92.09	0.01\\
92.1	0.01\\
92.11	0.01\\
92.12	0.01\\
92.13	0.01\\
92.14	0.01\\
92.15	0.01\\
92.16	0.01\\
92.17	0.01\\
92.18	0.01\\
92.19	0.01\\
92.2	0.01\\
92.21	0.01\\
92.22	0.01\\
92.23	0.01\\
92.24	0.01\\
92.25	0.01\\
92.26	0.01\\
92.27	0.01\\
92.28	0.01\\
92.29	0.01\\
92.3	0.01\\
92.31	0.01\\
92.32	0.01\\
92.33	0.01\\
92.34	0.01\\
92.35	0.01\\
92.36	0.01\\
92.37	0.01\\
92.38	0.01\\
92.39	0.01\\
92.4	0.01\\
92.41	0.01\\
92.42	0.01\\
92.43	0.01\\
92.44	0.01\\
92.45	0.01\\
92.46	0.01\\
92.47	0.01\\
92.48	0.01\\
92.49	0.01\\
92.5	0.01\\
92.51	0.01\\
92.52	0.01\\
92.53	0.01\\
92.54	0.01\\
92.55	0.01\\
92.56	0.01\\
92.57	0.01\\
92.58	0.01\\
92.59	0.01\\
92.6	0.01\\
92.61	0.01\\
92.62	0.01\\
92.63	0.01\\
92.64	0.01\\
92.65	0.01\\
92.66	0.01\\
92.67	0.01\\
92.68	0.01\\
92.69	0.01\\
92.7	0.01\\
92.71	0.01\\
92.72	0.01\\
92.73	0.01\\
92.74	0.01\\
92.75	0.01\\
92.76	0.01\\
92.77	0.01\\
92.78	0.01\\
92.79	0.01\\
92.8	0.01\\
92.81	0.01\\
92.82	0.01\\
92.83	0.01\\
92.84	0.01\\
92.85	0.01\\
92.86	0.01\\
92.87	0.01\\
92.88	0.01\\
92.89	0.01\\
92.9	0.01\\
92.91	0.01\\
92.92	0.01\\
92.93	0.01\\
92.94	0.01\\
92.95	0.01\\
92.96	0.01\\
92.97	0.01\\
92.98	0.01\\
92.99	0.01\\
93	0.01\\
93.01	0.01\\
93.02	0.01\\
93.03	0.01\\
93.04	0.01\\
93.05	0.01\\
93.06	0.01\\
93.07	0.01\\
93.08	0.01\\
93.09	0.01\\
93.1	0.01\\
93.11	0.01\\
93.12	0.01\\
93.13	0.01\\
93.14	0.01\\
93.15	0.01\\
93.16	0.01\\
93.17	0.01\\
93.18	0.01\\
93.19	0.01\\
93.2	0.01\\
93.21	0.01\\
93.22	0.01\\
93.23	0.01\\
93.24	0.01\\
93.25	0.01\\
93.26	0.01\\
93.27	0.01\\
93.28	0.01\\
93.29	0.01\\
93.3	0.01\\
93.31	0.01\\
93.32	0.01\\
93.33	0.01\\
93.34	0.01\\
93.35	0.01\\
93.36	0.01\\
93.37	0.01\\
93.38	0.01\\
93.39	0.01\\
93.4	0.01\\
93.41	0.01\\
93.42	0.01\\
93.43	0.01\\
93.44	0.01\\
93.45	0.01\\
93.46	0.01\\
93.47	0.01\\
93.48	0.01\\
93.49	0.01\\
93.5	0.01\\
93.51	0.01\\
93.52	0.01\\
93.53	0.01\\
93.54	0.01\\
93.55	0.01\\
93.56	0.01\\
93.57	0.01\\
93.58	0.01\\
93.59	0.01\\
93.6	0.01\\
93.61	0.01\\
93.62	0.01\\
93.63	0.01\\
93.64	0.01\\
93.65	0.01\\
93.66	0.01\\
93.67	0.01\\
93.68	0.01\\
93.69	0.01\\
93.7	0.01\\
93.71	0.01\\
93.72	0.01\\
93.73	0.01\\
93.74	0.01\\
93.75	0.01\\
93.76	0.01\\
93.77	0.01\\
93.78	0.01\\
93.79	0.01\\
93.8	0.01\\
93.81	0.01\\
93.82	0.01\\
93.83	0.01\\
93.84	0.01\\
93.85	0.01\\
93.86	0.01\\
93.87	0.01\\
93.88	0.01\\
93.89	0.01\\
93.9	0.01\\
93.91	0.01\\
93.92	0.01\\
93.93	0.01\\
93.94	0.01\\
93.95	0.01\\
93.96	0.01\\
93.97	0.01\\
93.98	0.01\\
93.99	0.01\\
94	0.01\\
94.01	0.01\\
94.02	0.01\\
94.03	0.01\\
94.04	0.01\\
94.05	0.01\\
94.06	0.01\\
94.07	0.01\\
94.08	0.01\\
94.09	0.01\\
94.1	0.01\\
94.11	0.01\\
94.12	0.01\\
94.13	0.01\\
94.14	0.01\\
94.15	0.01\\
94.16	0.01\\
94.17	0.01\\
94.18	0.01\\
94.19	0.01\\
94.2	0.01\\
94.21	0.01\\
94.22	0.01\\
94.23	0.01\\
94.24	0.01\\
94.25	0.01\\
94.26	0.01\\
94.27	0.01\\
94.28	0.01\\
94.29	0.01\\
94.3	0.01\\
94.31	0.01\\
94.32	0.01\\
94.33	0.01\\
94.34	0.01\\
94.35	0.01\\
94.36	0.01\\
94.37	0.01\\
94.38	0.01\\
94.39	0.01\\
94.4	0.01\\
94.41	0.01\\
94.42	0.01\\
94.43	0.01\\
94.44	0.01\\
94.45	0.01\\
94.46	0.01\\
94.47	0.01\\
94.48	0.01\\
94.49	0.01\\
94.5	0.01\\
94.51	0.01\\
94.52	0.01\\
94.53	0.01\\
94.54	0.01\\
94.55	0.01\\
94.56	0.01\\
94.57	0.01\\
94.58	0.01\\
94.59	0.01\\
94.6	0.01\\
94.61	0.01\\
94.62	0.01\\
94.63	0.01\\
94.64	0.01\\
94.65	0.01\\
94.66	0.01\\
94.67	0.01\\
94.68	0.01\\
94.69	0.01\\
94.7	0.01\\
94.71	0.01\\
94.72	0.01\\
94.73	0.01\\
94.74	0.01\\
94.75	0.01\\
94.76	0.01\\
94.77	0.01\\
94.78	0.01\\
94.79	0.01\\
94.8	0.01\\
94.81	0.01\\
94.82	0.01\\
94.83	0.01\\
94.84	0.01\\
94.85	0.01\\
94.86	0.01\\
94.87	0.01\\
94.88	0.01\\
94.89	0.01\\
94.9	0.01\\
94.91	0.01\\
94.92	0.01\\
94.93	0.01\\
94.94	0.01\\
94.95	0.01\\
94.96	0.01\\
94.97	0.01\\
94.98	0.01\\
94.99	0.01\\
95	0.01\\
95.01	0.01\\
95.02	0.01\\
95.03	0.01\\
95.04	0.01\\
95.05	0.01\\
95.06	0.01\\
95.07	0.01\\
95.08	0.01\\
95.09	0.01\\
95.1	0.01\\
95.11	0.01\\
95.12	0.01\\
95.13	0.01\\
95.14	0.01\\
95.15	0.01\\
95.16	0.01\\
95.17	0.01\\
95.18	0.01\\
95.19	0.01\\
95.2	0.01\\
95.21	0.01\\
95.22	0.01\\
95.23	0.01\\
95.24	0.01\\
95.25	0.01\\
95.26	0.01\\
95.27	0.01\\
95.28	0.01\\
95.29	0.01\\
95.3	0.01\\
95.31	0.01\\
95.32	0.01\\
95.33	0.01\\
95.34	0.01\\
95.35	0.01\\
95.36	0.01\\
95.37	0.01\\
95.38	0.01\\
95.39	0.01\\
95.4	0.01\\
95.41	0.01\\
95.42	0.01\\
95.43	0.01\\
95.44	0.01\\
95.45	0.01\\
95.46	0.01\\
95.47	0.01\\
95.48	0.01\\
95.49	0.01\\
95.5	0.01\\
95.51	0.01\\
95.52	0.01\\
95.53	0.01\\
95.54	0.01\\
95.55	0.01\\
95.56	0.01\\
95.57	0.01\\
95.58	0.01\\
95.59	0.01\\
95.6	0.01\\
95.61	0.01\\
95.62	0.01\\
95.63	0.01\\
95.64	0.01\\
95.65	0.01\\
95.66	0.01\\
95.67	0.01\\
95.68	0.01\\
95.69	0.01\\
95.7	0.01\\
95.71	0.01\\
95.72	0.01\\
95.73	0.01\\
95.74	0.01\\
95.75	0.01\\
95.76	0.01\\
95.77	0.01\\
95.78	0.01\\
95.79	0.01\\
95.8	0.01\\
95.81	0.01\\
95.82	0.01\\
95.83	0.01\\
95.84	0.01\\
95.85	0.01\\
95.86	0.01\\
95.87	0.01\\
95.88	0.01\\
95.89	0.01\\
95.9	0.01\\
95.91	0.01\\
95.92	0.01\\
95.93	0.01\\
95.94	0.01\\
95.95	0.01\\
95.96	0.01\\
95.97	0.01\\
95.98	0.01\\
95.99	0.01\\
96	0.01\\
96.01	0.01\\
96.02	0.01\\
96.03	0.01\\
96.04	0.01\\
96.05	0.01\\
96.06	0.01\\
96.07	0.01\\
96.08	0.01\\
96.09	0.01\\
96.1	0.01\\
96.11	0.01\\
96.12	0.01\\
96.13	0.01\\
96.14	0.01\\
96.15	0.01\\
96.16	0.01\\
96.17	0.01\\
96.18	0.01\\
96.19	0.01\\
96.2	0.01\\
96.21	0.01\\
96.22	0.01\\
96.23	0.01\\
96.24	0.01\\
96.25	0.01\\
96.26	0.01\\
96.27	0.01\\
96.28	0.01\\
96.29	0.01\\
96.3	0.01\\
96.31	0.01\\
96.32	0.01\\
96.33	0.01\\
96.34	0.01\\
96.35	0.01\\
96.36	0.01\\
96.37	0.01\\
96.38	0.01\\
96.39	0.01\\
96.4	0.01\\
96.41	0.01\\
96.42	0.01\\
96.43	0.01\\
96.44	0.01\\
96.45	0.01\\
96.46	0.01\\
96.47	0.01\\
96.48	0.01\\
96.49	0.01\\
96.5	0.01\\
96.51	0.01\\
96.52	0.01\\
96.53	0.01\\
96.54	0.01\\
96.55	0.01\\
96.56	0.01\\
96.57	0.01\\
96.58	0.01\\
96.59	0.01\\
96.6	0.01\\
96.61	0.01\\
96.62	0.01\\
96.63	0.01\\
96.64	0.01\\
96.65	0.01\\
96.66	0.01\\
96.67	0.01\\
96.68	0.01\\
96.69	0.01\\
96.7	0.01\\
96.71	0.01\\
96.72	0.01\\
96.73	0.01\\
96.74	0.01\\
96.75	0.01\\
96.76	0.01\\
96.77	0.01\\
96.78	0.01\\
96.79	0.01\\
96.8	0.01\\
96.81	0.01\\
96.82	0.01\\
96.83	0.01\\
96.84	0.01\\
96.85	0.01\\
96.86	0.01\\
96.87	0.01\\
96.88	0.01\\
96.89	0.01\\
96.9	0.01\\
96.91	0.01\\
96.92	0.01\\
96.93	0.01\\
96.94	0.01\\
96.95	0.01\\
96.96	0.01\\
96.97	0.01\\
96.98	0.01\\
96.99	0.01\\
97	0.01\\
97.01	0.01\\
97.02	0.01\\
97.03	0.01\\
97.04	0.01\\
97.05	0.01\\
97.06	0.01\\
97.07	0.01\\
97.08	0.01\\
97.09	0.01\\
97.1	0.01\\
97.11	0.01\\
97.12	0.01\\
97.13	0.01\\
97.14	0.01\\
97.15	0.01\\
97.16	0.01\\
97.17	0.01\\
97.18	0.01\\
97.19	0.01\\
97.2	0.01\\
97.21	0.01\\
97.22	0.01\\
97.23	0.01\\
97.24	0.01\\
97.25	0.01\\
97.26	0.01\\
97.27	0.01\\
97.28	0.01\\
97.29	0.01\\
97.3	0.01\\
97.31	0.01\\
97.32	0.01\\
97.33	0.01\\
97.34	0.01\\
97.35	0.01\\
97.36	0.01\\
97.37	0.01\\
97.38	0.01\\
97.39	0.01\\
97.4	0.01\\
97.41	0.01\\
97.42	0.01\\
97.43	0.01\\
97.44	0.01\\
97.45	0.01\\
97.46	0.01\\
97.47	0.01\\
97.48	0.01\\
97.49	0.01\\
97.5	0.01\\
97.51	0.01\\
97.52	0.01\\
97.53	0.01\\
97.54	0.01\\
97.55	0.01\\
97.56	0.01\\
97.57	0.01\\
97.58	0.01\\
97.59	0.01\\
97.6	0.01\\
97.61	0.01\\
97.62	0.01\\
97.63	0.01\\
97.64	0.01\\
97.65	0.01\\
97.66	0.01\\
97.67	0.01\\
97.68	0.01\\
97.69	0.01\\
97.7	0.01\\
97.71	0.01\\
97.72	0.01\\
97.73	0.01\\
97.74	0.01\\
97.75	0.01\\
97.76	0.01\\
97.77	0.01\\
97.78	0.01\\
97.79	0.01\\
97.8	0.01\\
97.81	0.01\\
97.82	0.01\\
97.83	0.01\\
97.84	0.01\\
97.85	0.01\\
97.86	0.01\\
97.87	0.01\\
97.88	0.01\\
97.89	0.01\\
97.9	0.01\\
97.91	0.01\\
97.92	0.01\\
97.93	0.01\\
97.94	0.01\\
97.95	0.01\\
97.96	0.01\\
97.97	0.01\\
97.98	0.01\\
97.99	0.01\\
98	0.01\\
98.01	0.01\\
98.02	0.01\\
98.03	0.01\\
98.04	0.01\\
98.05	0.01\\
98.06	0.01\\
98.07	0.01\\
98.08	0.01\\
98.09	0.01\\
98.1	0.01\\
98.11	0.01\\
98.12	0.01\\
98.13	0.01\\
98.14	0.01\\
98.15	0.01\\
98.16	0.01\\
98.17	0.01\\
98.18	0.01\\
98.19	0.01\\
98.2	0.01\\
98.21	0.01\\
98.22	0.01\\
98.23	0.01\\
98.24	0.01\\
98.25	0.01\\
98.26	0.01\\
98.27	0.01\\
98.28	0.01\\
98.29	0.01\\
98.3	0.01\\
98.31	0.01\\
98.32	0.01\\
98.33	0.01\\
98.34	0.01\\
98.35	0.01\\
98.36	0.01\\
98.37	0.01\\
98.38	0.01\\
98.39	0.01\\
98.4	0.01\\
98.41	0.01\\
98.42	0.01\\
98.43	0.01\\
98.44	0.01\\
98.45	0.01\\
98.46	0.01\\
98.47	0.01\\
98.48	0.01\\
98.49	0.01\\
98.5	0.01\\
98.51	0.01\\
98.52	0.01\\
98.53	0.01\\
98.54	0.01\\
98.55	0.01\\
98.56	0.01\\
98.57	0.01\\
98.58	0.01\\
98.59	0.01\\
98.6	0.01\\
98.61	0.01\\
98.62	0.01\\
98.63	0.01\\
98.64	0.01\\
98.65	0.01\\
98.66	0.01\\
98.67	0.01\\
98.68	0.01\\
98.69	0.01\\
98.7	0.01\\
98.71	0.01\\
98.72	0.01\\
98.73	0.01\\
98.74	0.01\\
98.75	0.01\\
98.76	0.01\\
98.77	0.01\\
98.78	0.01\\
98.79	0.01\\
98.8	0.01\\
98.81	0.01\\
98.82	0.01\\
98.83	0.01\\
98.84	0.01\\
98.85	0.01\\
98.86	0.01\\
98.87	0.01\\
98.88	0.01\\
98.89	0.01\\
98.9	0.01\\
98.91	0.01\\
98.92	0.01\\
98.93	0.01\\
98.94	0.01\\
98.95	0.01\\
98.96	0.01\\
98.97	0.01\\
98.98	0.01\\
98.99	0.01\\
99	0.01\\
99.01	0.01\\
99.02	0.01\\
99.03	0.01\\
99.04	0.01\\
99.05	0.01\\
99.06	0.01\\
99.07	0.01\\
99.08	0.01\\
99.09	0.01\\
99.1	0.01\\
99.11	0.01\\
99.12	0.01\\
99.13	0.01\\
99.14	0.01\\
99.15	0.01\\
99.16	0.01\\
99.17	0.01\\
99.18	0.01\\
99.19	0.01\\
99.2	0.01\\
99.21	0.01\\
99.22	0.01\\
99.23	0.01\\
99.24	0.01\\
99.25	0.01\\
99.26	0.01\\
99.27	0.01\\
99.28	0.01\\
99.29	0.01\\
99.3	0.01\\
99.31	0.01\\
99.32	0.01\\
99.33	0.01\\
99.34	0.01\\
99.35	0.01\\
99.36	0.01\\
99.37	0.01\\
99.38	0.01\\
99.39	0.01\\
99.4	0.01\\
99.41	0.01\\
99.42	0.01\\
99.43	0.01\\
99.44	0.01\\
99.45	0.01\\
99.46	0.01\\
99.47	0.01\\
99.48	0.01\\
99.49	0.01\\
99.5	0.01\\
99.51	0.01\\
99.52	0.01\\
99.53	0.01\\
99.54	0.01\\
99.55	0.01\\
99.56	0.01\\
99.57	0.01\\
99.58	0.01\\
99.59	0.01\\
99.6	0.01\\
99.61	0.01\\
99.62	0.01\\
99.63	0.01\\
99.64	0.01\\
99.65	0.01\\
99.66	0.01\\
99.67	0.01\\
99.68	0.01\\
99.69	0.01\\
99.7	0.01\\
99.71	0.01\\
99.72	0.01\\
99.73	0.01\\
99.74	0.01\\
99.75	0.01\\
99.76	0.01\\
99.77	0.01\\
99.78	0.01\\
99.79	0.01\\
99.8	0.01\\
99.81	0.01\\
99.82	0.01\\
99.83	0.01\\
99.84	0.01\\
99.85	0.01\\
99.86	0.01\\
99.87	0.01\\
99.88	0.01\\
99.89	0.01\\
99.9	0.01\\
99.91	0.01\\
99.92	0.01\\
99.93	0.01\\
99.94	0.01\\
99.95	0.01\\
99.96	0.01\\
99.97	0.01\\
99.98	0.01\\
99.99	0.01\\
100	0.01\\
};
\addlegendentry{$q=-2$};

\addplot [color=blue,dashed,forget plot]
  table[row sep=crcr]{%
0.01	0.01\\
0.02	0.01\\
0.03	0.01\\
0.04	0.01\\
0.05	0.01\\
0.06	0.01\\
0.07	0.01\\
0.08	0.01\\
0.09	0.01\\
0.1	0.01\\
0.11	0.01\\
0.12	0.01\\
0.13	0.01\\
0.14	0.01\\
0.15	0.01\\
0.16	0.01\\
0.17	0.01\\
0.18	0.01\\
0.19	0.01\\
0.2	0.01\\
0.21	0.01\\
0.22	0.01\\
0.23	0.01\\
0.24	0.01\\
0.25	0.01\\
0.26	0.01\\
0.27	0.01\\
0.28	0.01\\
0.29	0.01\\
0.3	0.01\\
0.31	0.01\\
0.32	0.01\\
0.33	0.01\\
0.34	0.01\\
0.35	0.01\\
0.36	0.01\\
0.37	0.01\\
0.38	0.01\\
0.39	0.01\\
0.4	0.01\\
0.41	0.01\\
0.42	0.01\\
0.43	0.01\\
0.44	0.01\\
0.45	0.01\\
0.46	0.01\\
0.47	0.01\\
0.48	0.01\\
0.49	0.01\\
0.5	0.01\\
0.51	0.01\\
0.52	0.01\\
0.53	0.01\\
0.54	0.01\\
0.55	0.01\\
0.56	0.01\\
0.57	0.01\\
0.58	0.01\\
0.59	0.01\\
0.6	0.01\\
0.61	0.01\\
0.62	0.01\\
0.63	0.01\\
0.64	0.01\\
0.65	0.01\\
0.66	0.01\\
0.67	0.01\\
0.68	0.01\\
0.69	0.01\\
0.7	0.01\\
0.71	0.01\\
0.72	0.01\\
0.73	0.01\\
0.74	0.01\\
0.75	0.01\\
0.76	0.01\\
0.77	0.01\\
0.78	0.01\\
0.79	0.01\\
0.8	0.01\\
0.81	0.01\\
0.82	0.01\\
0.83	0.01\\
0.84	0.01\\
0.85	0.01\\
0.86	0.01\\
0.87	0.01\\
0.88	0.01\\
0.89	0.01\\
0.9	0.01\\
0.91	0.01\\
0.92	0.01\\
0.93	0.01\\
0.94	0.01\\
0.95	0.01\\
0.96	0.01\\
0.97	0.01\\
0.98	0.01\\
0.99	0.01\\
1	0.01\\
1.01	0.01\\
1.02	0.01\\
1.03	0.01\\
1.04	0.01\\
1.05	0.01\\
1.06	0.01\\
1.07	0.01\\
1.08	0.01\\
1.09	0.01\\
1.1	0.01\\
1.11	0.01\\
1.12	0.01\\
1.13	0.01\\
1.14	0.01\\
1.15	0.01\\
1.16	0.01\\
1.17	0.01\\
1.18	0.01\\
1.19	0.01\\
1.2	0.01\\
1.21	0.01\\
1.22	0.01\\
1.23	0.01\\
1.24	0.01\\
1.25	0.01\\
1.26	0.01\\
1.27	0.01\\
1.28	0.01\\
1.29	0.01\\
1.3	0.01\\
1.31	0.01\\
1.32	0.01\\
1.33	0.01\\
1.34	0.01\\
1.35	0.01\\
1.36	0.01\\
1.37	0.01\\
1.38	0.01\\
1.39	0.01\\
1.4	0.01\\
1.41	0.01\\
1.42	0.01\\
1.43	0.01\\
1.44	0.01\\
1.45	0.01\\
1.46	0.01\\
1.47	0.01\\
1.48	0.01\\
1.49	0.01\\
1.5	0.01\\
1.51	0.01\\
1.52	0.01\\
1.53	0.01\\
1.54	0.01\\
1.55	0.01\\
1.56	0.01\\
1.57	0.01\\
1.58	0.01\\
1.59	0.01\\
1.6	0.01\\
1.61	0.01\\
1.62	0.01\\
1.63	0.01\\
1.64	0.01\\
1.65	0.01\\
1.66	0.01\\
1.67	0.01\\
1.68	0.01\\
1.69	0.01\\
1.7	0.01\\
1.71	0.01\\
1.72	0.01\\
1.73	0.01\\
1.74	0.01\\
1.75	0.01\\
1.76	0.01\\
1.77	0.01\\
1.78	0.01\\
1.79	0.01\\
1.8	0.01\\
1.81	0.01\\
1.82	0.01\\
1.83	0.01\\
1.84	0.01\\
1.85	0.01\\
1.86	0.01\\
1.87	0.01\\
1.88	0.01\\
1.89	0.01\\
1.9	0.01\\
1.91	0.01\\
1.92	0.01\\
1.93	0.01\\
1.94	0.01\\
1.95	0.01\\
1.96	0.01\\
1.97	0.01\\
1.98	0.01\\
1.99	0.01\\
2	0.01\\
2.01	0.01\\
2.02	0.01\\
2.03	0.01\\
2.04	0.01\\
2.05	0.01\\
2.06	0.01\\
2.07	0.01\\
2.08	0.01\\
2.09	0.01\\
2.1	0.01\\
2.11	0.01\\
2.12	0.01\\
2.13	0.01\\
2.14	0.01\\
2.15	0.01\\
2.16	0.01\\
2.17	0.01\\
2.18	0.01\\
2.19	0.01\\
2.2	0.01\\
2.21	0.01\\
2.22	0.01\\
2.23	0.01\\
2.24	0.01\\
2.25	0.01\\
2.26	0.01\\
2.27	0.01\\
2.28	0.01\\
2.29	0.01\\
2.3	0.01\\
2.31	0.01\\
2.32	0.01\\
2.33	0.01\\
2.34	0.01\\
2.35	0.01\\
2.36	0.01\\
2.37	0.01\\
2.38	0.01\\
2.39	0.01\\
2.4	0.01\\
2.41	0.01\\
2.42	0.01\\
2.43	0.01\\
2.44	0.01\\
2.45	0.01\\
2.46	0.01\\
2.47	0.01\\
2.48	0.01\\
2.49	0.01\\
2.5	0.01\\
2.51	0.01\\
2.52	0.01\\
2.53	0.01\\
2.54	0.01\\
2.55	0.01\\
2.56	0.01\\
2.57	0.01\\
2.58	0.01\\
2.59	0.01\\
2.6	0.01\\
2.61	0.01\\
2.62	0.01\\
2.63	0.01\\
2.64	0.01\\
2.65	0.01\\
2.66	0.01\\
2.67	0.01\\
2.68	0.01\\
2.69	0.01\\
2.7	0.01\\
2.71	0.01\\
2.72	0.01\\
2.73	0.01\\
2.74	0.01\\
2.75	0.01\\
2.76	0.01\\
2.77	0.01\\
2.78	0.01\\
2.79	0.01\\
2.8	0.01\\
2.81	0.01\\
2.82	0.01\\
2.83	0.01\\
2.84	0.01\\
2.85	0.01\\
2.86	0.01\\
2.87	0.01\\
2.88	0.01\\
2.89	0.01\\
2.9	0.01\\
2.91	0.01\\
2.92	0.01\\
2.93	0.01\\
2.94	0.01\\
2.95	0.01\\
2.96	0.01\\
2.97	0.01\\
2.98	0.01\\
2.99	0.01\\
3	0.01\\
3.01	0.01\\
3.02	0.01\\
3.03	0.01\\
3.04	0.01\\
3.05	0.01\\
3.06	0.01\\
3.07	0.01\\
3.08	0.01\\
3.09	0.01\\
3.1	0.01\\
3.11	0.01\\
3.12	0.01\\
3.13	0.01\\
3.14	0.01\\
3.15	0.01\\
3.16	0.01\\
3.17	0.01\\
3.18	0.01\\
3.19	0.01\\
3.2	0.01\\
3.21	0.01\\
3.22	0.01\\
3.23	0.01\\
3.24	0.01\\
3.25	0.01\\
3.26	0.01\\
3.27	0.01\\
3.28	0.01\\
3.29	0.01\\
3.3	0.01\\
3.31	0.01\\
3.32	0.01\\
3.33	0.01\\
3.34	0.01\\
3.35	0.01\\
3.36	0.01\\
3.37	0.01\\
3.38	0.01\\
3.39	0.01\\
3.4	0.01\\
3.41	0.01\\
3.42	0.01\\
3.43	0.01\\
3.44	0.01\\
3.45	0.01\\
3.46	0.01\\
3.47	0.01\\
3.48	0.01\\
3.49	0.01\\
3.5	0.01\\
3.51	0.01\\
3.52	0.01\\
3.53	0.01\\
3.54	0.01\\
3.55	0.01\\
3.56	0.01\\
3.57	0.01\\
3.58	0.01\\
3.59	0.01\\
3.6	0.01\\
3.61	0.01\\
3.62	0.01\\
3.63	0.01\\
3.64	0.01\\
3.65	0.01\\
3.66	0.01\\
3.67	0.01\\
3.68	0.01\\
3.69	0.01\\
3.7	0.01\\
3.71	0.01\\
3.72	0.01\\
3.73	0.01\\
3.74	0.01\\
3.75	0.01\\
3.76	0.01\\
3.77	0.01\\
3.78	0.01\\
3.79	0.01\\
3.8	0.01\\
3.81	0.01\\
3.82	0.01\\
3.83	0.01\\
3.84	0.01\\
3.85	0.01\\
3.86	0.01\\
3.87	0.01\\
3.88	0.01\\
3.89	0.01\\
3.9	0.01\\
3.91	0.01\\
3.92	0.01\\
3.93	0.01\\
3.94	0.01\\
3.95	0.01\\
3.96	0.01\\
3.97	0.01\\
3.98	0.01\\
3.99	0.01\\
4	0.01\\
4.01	0.01\\
4.02	0.01\\
4.03	0.01\\
4.04	0.01\\
4.05	0.01\\
4.06	0.01\\
4.07	0.01\\
4.08	0.01\\
4.09	0.01\\
4.1	0.01\\
4.11	0.01\\
4.12	0.01\\
4.13	0.01\\
4.14	0.01\\
4.15	0.01\\
4.16	0.01\\
4.17	0.01\\
4.18	0.01\\
4.19	0.01\\
4.2	0.01\\
4.21	0.01\\
4.22	0.01\\
4.23	0.01\\
4.24	0.01\\
4.25	0.01\\
4.26	0.01\\
4.27	0.01\\
4.28	0.01\\
4.29	0.01\\
4.3	0.01\\
4.31	0.01\\
4.32	0.01\\
4.33	0.01\\
4.34	0.01\\
4.35	0.01\\
4.36	0.01\\
4.37	0.01\\
4.38	0.01\\
4.39	0.01\\
4.4	0.01\\
4.41	0.01\\
4.42	0.01\\
4.43	0.01\\
4.44	0.01\\
4.45	0.01\\
4.46	0.01\\
4.47	0.01\\
4.48	0.01\\
4.49	0.01\\
4.5	0.01\\
4.51	0.01\\
4.52	0.01\\
4.53	0.01\\
4.54	0.01\\
4.55	0.01\\
4.56	0.01\\
4.57	0.01\\
4.58	0.01\\
4.59	0.01\\
4.6	0.01\\
4.61	0.01\\
4.62	0.01\\
4.63	0.01\\
4.64	0.01\\
4.65	0.01\\
4.66	0.01\\
4.67	0.01\\
4.68	0.01\\
4.69	0.01\\
4.7	0.01\\
4.71	0.01\\
4.72	0.01\\
4.73	0.01\\
4.74	0.01\\
4.75	0.01\\
4.76	0.01\\
4.77	0.01\\
4.78	0.01\\
4.79	0.01\\
4.8	0.01\\
4.81	0.01\\
4.82	0.01\\
4.83	0.01\\
4.84	0.01\\
4.85	0.01\\
4.86	0.01\\
4.87	0.01\\
4.88	0.01\\
4.89	0.01\\
4.9	0.01\\
4.91	0.01\\
4.92	0.01\\
4.93	0.01\\
4.94	0.01\\
4.95	0.01\\
4.96	0.01\\
4.97	0.01\\
4.98	0.01\\
4.99	0.01\\
5	0.01\\
5.01	0.01\\
5.02	0.01\\
5.03	0.01\\
5.04	0.01\\
5.05	0.01\\
5.06	0.01\\
5.07	0.01\\
5.08	0.01\\
5.09	0.01\\
5.1	0.01\\
5.11	0.01\\
5.12	0.01\\
5.13	0.01\\
5.14	0.01\\
5.15	0.01\\
5.16	0.01\\
5.17	0.01\\
5.18	0.01\\
5.19	0.01\\
5.2	0.01\\
5.21	0.01\\
5.22	0.01\\
5.23	0.01\\
5.24	0.01\\
5.25	0.01\\
5.26	0.01\\
5.27	0.01\\
5.28	0.01\\
5.29	0.01\\
5.3	0.01\\
5.31	0.01\\
5.32	0.01\\
5.33	0.01\\
5.34	0.01\\
5.35	0.01\\
5.36	0.01\\
5.37	0.01\\
5.38	0.01\\
5.39	0.01\\
5.4	0.01\\
5.41	0.01\\
5.42	0.01\\
5.43	0.01\\
5.44	0.01\\
5.45	0.01\\
5.46	0.01\\
5.47	0.01\\
5.48	0.01\\
5.49	0.01\\
5.5	0.01\\
5.51	0.01\\
5.52	0.01\\
5.53	0.01\\
5.54	0.01\\
5.55	0.01\\
5.56	0.01\\
5.57	0.01\\
5.58	0.01\\
5.59	0.01\\
5.6	0.01\\
5.61	0.01\\
5.62	0.01\\
5.63	0.01\\
5.64	0.01\\
5.65	0.01\\
5.66	0.01\\
5.67	0.01\\
5.68	0.01\\
5.69	0.01\\
5.7	0.01\\
5.71	0.01\\
5.72	0.01\\
5.73	0.01\\
5.74	0.01\\
5.75	0.01\\
5.76	0.01\\
5.77	0.01\\
5.78	0.01\\
5.79	0.01\\
5.8	0.01\\
5.81	0.01\\
5.82	0.01\\
5.83	0.01\\
5.84	0.01\\
5.85	0.01\\
5.86	0.01\\
5.87	0.01\\
5.88	0.01\\
5.89	0.01\\
5.9	0.01\\
5.91	0.01\\
5.92	0.01\\
5.93	0.01\\
5.94	0.01\\
5.95	0.01\\
5.96	0.01\\
5.97	0.01\\
5.98	0.01\\
5.99	0.01\\
6	0.01\\
6.01	0.01\\
6.02	0.01\\
6.03	0.01\\
6.04	0.01\\
6.05	0.01\\
6.06	0.01\\
6.07	0.01\\
6.08	0.01\\
6.09	0.01\\
6.1	0.01\\
6.11	0.01\\
6.12	0.01\\
6.13	0.01\\
6.14	0.01\\
6.15	0.01\\
6.16	0.01\\
6.17	0.01\\
6.18	0.01\\
6.19	0.01\\
6.2	0.01\\
6.21	0.01\\
6.22	0.01\\
6.23	0.01\\
6.24	0.01\\
6.25	0.01\\
6.26	0.01\\
6.27	0.01\\
6.28	0.01\\
6.29	0.01\\
6.3	0.01\\
6.31	0.01\\
6.32	0.01\\
6.33	0.01\\
6.34	0.01\\
6.35	0.01\\
6.36	0.01\\
6.37	0.01\\
6.38	0.01\\
6.39	0.01\\
6.4	0.01\\
6.41	0.01\\
6.42	0.01\\
6.43	0.01\\
6.44	0.01\\
6.45	0.01\\
6.46	0.01\\
6.47	0.01\\
6.48	0.01\\
6.49	0.01\\
6.5	0.01\\
6.51	0.01\\
6.52	0.01\\
6.53	0.01\\
6.54	0.01\\
6.55	0.01\\
6.56	0.01\\
6.57	0.01\\
6.58	0.01\\
6.59	0.01\\
6.6	0.01\\
6.61	0.01\\
6.62	0.01\\
6.63	0.01\\
6.64	0.01\\
6.65	0.01\\
6.66	0.01\\
6.67	0.01\\
6.68	0.01\\
6.69	0.01\\
6.7	0.01\\
6.71	0.01\\
6.72	0.01\\
6.73	0.01\\
6.74	0.01\\
6.75	0.01\\
6.76	0.01\\
6.77	0.01\\
6.78	0.01\\
6.79	0.01\\
6.8	0.01\\
6.81	0.01\\
6.82	0.01\\
6.83	0.01\\
6.84	0.01\\
6.85	0.01\\
6.86	0.01\\
6.87	0.01\\
6.88	0.01\\
6.89	0.01\\
6.9	0.01\\
6.91	0.01\\
6.92	0.01\\
6.93	0.01\\
6.94	0.01\\
6.95	0.01\\
6.96	0.01\\
6.97	0.01\\
6.98	0.01\\
6.99	0.01\\
7	0.01\\
7.01	0.01\\
7.02	0.01\\
7.03	0.01\\
7.04	0.01\\
7.05	0.01\\
7.06	0.01\\
7.07	0.01\\
7.08	0.01\\
7.09	0.01\\
7.1	0.01\\
7.11	0.01\\
7.12	0.01\\
7.13	0.01\\
7.14	0.01\\
7.15	0.01\\
7.16	0.01\\
7.17	0.01\\
7.18	0.01\\
7.19	0.01\\
7.2	0.01\\
7.21	0.01\\
7.22	0.01\\
7.23	0.01\\
7.24	0.01\\
7.25	0.01\\
7.26	0.01\\
7.27	0.01\\
7.28	0.01\\
7.29	0.01\\
7.3	0.01\\
7.31	0.01\\
7.32	0.01\\
7.33	0.01\\
7.34	0.01\\
7.35	0.01\\
7.36	0.01\\
7.37	0.01\\
7.38	0.01\\
7.39	0.01\\
7.4	0.01\\
7.41	0.01\\
7.42	0.01\\
7.43	0.01\\
7.44	0.01\\
7.45	0.01\\
7.46	0.01\\
7.47	0.01\\
7.48	0.01\\
7.49	0.01\\
7.5	0.01\\
7.51	0.01\\
7.52	0.01\\
7.53	0.01\\
7.54	0.01\\
7.55	0.01\\
7.56	0.01\\
7.57	0.01\\
7.58	0.01\\
7.59	0.01\\
7.6	0.01\\
7.61	0.01\\
7.62	0.01\\
7.63	0.01\\
7.64	0.01\\
7.65	0.01\\
7.66	0.01\\
7.67	0.01\\
7.68	0.01\\
7.69	0.01\\
7.7	0.01\\
7.71	0.01\\
7.72	0.01\\
7.73	0.01\\
7.74	0.01\\
7.75	0.01\\
7.76	0.01\\
7.77	0.01\\
7.78	0.01\\
7.79	0.01\\
7.8	0.01\\
7.81	0.01\\
7.82	0.01\\
7.83	0.01\\
7.84	0.01\\
7.85	0.01\\
7.86	0.01\\
7.87	0.01\\
7.88	0.01\\
7.89	0.01\\
7.9	0.01\\
7.91	0.01\\
7.92	0.01\\
7.93	0.01\\
7.94	0.01\\
7.95	0.01\\
7.96	0.01\\
7.97	0.01\\
7.98	0.01\\
7.99	0.01\\
8	0.01\\
8.01	0.01\\
8.02	0.01\\
8.03	0.01\\
8.04	0.01\\
8.05	0.01\\
8.06	0.01\\
8.07	0.01\\
8.08	0.01\\
8.09	0.01\\
8.1	0.01\\
8.11	0.01\\
8.12	0.01\\
8.13	0.01\\
8.14	0.01\\
8.15	0.01\\
8.16	0.01\\
8.17	0.01\\
8.18	0.01\\
8.19	0.01\\
8.2	0.01\\
8.21	0.01\\
8.22	0.01\\
8.23	0.01\\
8.24	0.01\\
8.25	0.01\\
8.26	0.01\\
8.27	0.01\\
8.28	0.01\\
8.29	0.01\\
8.3	0.01\\
8.31	0.01\\
8.32	0.01\\
8.33	0.01\\
8.34	0.01\\
8.35	0.01\\
8.36	0.01\\
8.37	0.01\\
8.38	0.01\\
8.39	0.01\\
8.4	0.01\\
8.41	0.01\\
8.42	0.01\\
8.43	0.01\\
8.44	0.01\\
8.45	0.01\\
8.46	0.01\\
8.47	0.01\\
8.48	0.01\\
8.49	0.01\\
8.5	0.01\\
8.51	0.01\\
8.52	0.01\\
8.53	0.01\\
8.54	0.01\\
8.55	0.01\\
8.56	0.01\\
8.57	0.01\\
8.58	0.01\\
8.59	0.01\\
8.6	0.01\\
8.61	0.01\\
8.62	0.01\\
8.63	0.01\\
8.64	0.01\\
8.65	0.01\\
8.66	0.01\\
8.67	0.01\\
8.68	0.01\\
8.69	0.01\\
8.7	0.01\\
8.71	0.01\\
8.72	0.01\\
8.73	0.01\\
8.74	0.01\\
8.75	0.01\\
8.76	0.01\\
8.77	0.01\\
8.78	0.01\\
8.79	0.01\\
8.8	0.01\\
8.81	0.01\\
8.82	0.01\\
8.83	0.01\\
8.84	0.01\\
8.85	0.01\\
8.86	0.01\\
8.87	0.01\\
8.88	0.01\\
8.89	0.01\\
8.9	0.01\\
8.91	0.01\\
8.92	0.01\\
8.93	0.01\\
8.94	0.01\\
8.95	0.01\\
8.96	0.01\\
8.97	0.01\\
8.98	0.01\\
8.99	0.01\\
9	0.01\\
9.01	0.01\\
9.02	0.01\\
9.03	0.01\\
9.04	0.01\\
9.05	0.01\\
9.06	0.01\\
9.07	0.01\\
9.08	0.01\\
9.09	0.01\\
9.1	0.01\\
9.11	0.01\\
9.12	0.01\\
9.13	0.01\\
9.14	0.01\\
9.15	0.01\\
9.16	0.01\\
9.17	0.01\\
9.18	0.01\\
9.19	0.01\\
9.2	0.01\\
9.21	0.01\\
9.22	0.01\\
9.23	0.01\\
9.24	0.01\\
9.25	0.01\\
9.26	0.01\\
9.27	0.01\\
9.28	0.01\\
9.29	0.01\\
9.3	0.01\\
9.31	0.01\\
9.32	0.01\\
9.33	0.01\\
9.34	0.01\\
9.35	0.01\\
9.36	0.01\\
9.37	0.01\\
9.38	0.01\\
9.39	0.01\\
9.4	0.01\\
9.41	0.01\\
9.42	0.01\\
9.43	0.01\\
9.44	0.01\\
9.45	0.01\\
9.46	0.01\\
9.47	0.01\\
9.48	0.01\\
9.49	0.01\\
9.5	0.01\\
9.51	0.01\\
9.52	0.01\\
9.53	0.01\\
9.54	0.01\\
9.55	0.01\\
9.56	0.01\\
9.57	0.01\\
9.58	0.01\\
9.59	0.01\\
9.6	0.01\\
9.61	0.01\\
9.62	0.01\\
9.63	0.01\\
9.64	0.01\\
9.65	0.01\\
9.66	0.01\\
9.67	0.01\\
9.68	0.01\\
9.69	0.01\\
9.7	0.01\\
9.71	0.01\\
9.72	0.01\\
9.73	0.01\\
9.74	0.01\\
9.75	0.01\\
9.76	0.01\\
9.77	0.01\\
9.78	0.01\\
9.79	0.01\\
9.8	0.01\\
9.81	0.01\\
9.82	0.01\\
9.83	0.01\\
9.84	0.01\\
9.85	0.01\\
9.86	0.01\\
9.87	0.01\\
9.88	0.01\\
9.89	0.01\\
9.9	0.01\\
9.91	0.01\\
9.92	0.01\\
9.93	0.01\\
9.94	0.01\\
9.95	0.01\\
9.96	0.01\\
9.97	0.01\\
9.98	0.01\\
9.99	0.01\\
10	0.01\\
10.01	0.01\\
10.02	0.01\\
10.03	0.01\\
10.04	0.01\\
10.05	0.01\\
10.06	0.01\\
10.07	0.01\\
10.08	0.01\\
10.09	0.01\\
10.1	0.01\\
10.11	0.01\\
10.12	0.01\\
10.13	0.01\\
10.14	0.01\\
10.15	0.01\\
10.16	0.01\\
10.17	0.01\\
10.18	0.01\\
10.19	0.01\\
10.2	0.01\\
10.21	0.01\\
10.22	0.01\\
10.23	0.01\\
10.24	0.01\\
10.25	0.01\\
10.26	0.01\\
10.27	0.01\\
10.28	0.01\\
10.29	0.01\\
10.3	0.01\\
10.31	0.01\\
10.32	0.01\\
10.33	0.01\\
10.34	0.01\\
10.35	0.01\\
10.36	0.01\\
10.37	0.01\\
10.38	0.01\\
10.39	0.01\\
10.4	0.01\\
10.41	0.01\\
10.42	0.01\\
10.43	0.01\\
10.44	0.01\\
10.45	0.01\\
10.46	0.01\\
10.47	0.01\\
10.48	0.01\\
10.49	0.01\\
10.5	0.01\\
10.51	0.01\\
10.52	0.01\\
10.53	0.01\\
10.54	0.01\\
10.55	0.01\\
10.56	0.01\\
10.57	0.01\\
10.58	0.01\\
10.59	0.01\\
10.6	0.01\\
10.61	0.01\\
10.62	0.01\\
10.63	0.01\\
10.64	0.01\\
10.65	0.01\\
10.66	0.01\\
10.67	0.01\\
10.68	0.01\\
10.69	0.01\\
10.7	0.01\\
10.71	0.01\\
10.72	0.01\\
10.73	0.01\\
10.74	0.01\\
10.75	0.01\\
10.76	0.01\\
10.77	0.01\\
10.78	0.01\\
10.79	0.01\\
10.8	0.01\\
10.81	0.01\\
10.82	0.01\\
10.83	0.01\\
10.84	0.01\\
10.85	0.01\\
10.86	0.01\\
10.87	0.01\\
10.88	0.01\\
10.89	0.01\\
10.9	0.01\\
10.91	0.01\\
10.92	0.01\\
10.93	0.01\\
10.94	0.01\\
10.95	0.01\\
10.96	0.01\\
10.97	0.01\\
10.98	0.01\\
10.99	0.01\\
11	0.01\\
11.01	0.01\\
11.02	0.01\\
11.03	0.01\\
11.04	0.01\\
11.05	0.01\\
11.06	0.01\\
11.07	0.01\\
11.08	0.01\\
11.09	0.01\\
11.1	0.01\\
11.11	0.01\\
11.12	0.01\\
11.13	0.01\\
11.14	0.01\\
11.15	0.01\\
11.16	0.01\\
11.17	0.01\\
11.18	0.01\\
11.19	0.01\\
11.2	0.01\\
11.21	0.01\\
11.22	0.01\\
11.23	0.01\\
11.24	0.01\\
11.25	0.01\\
11.26	0.01\\
11.27	0.01\\
11.28	0.01\\
11.29	0.01\\
11.3	0.01\\
11.31	0.01\\
11.32	0.01\\
11.33	0.01\\
11.34	0.01\\
11.35	0.01\\
11.36	0.01\\
11.37	0.01\\
11.38	0.01\\
11.39	0.01\\
11.4	0.01\\
11.41	0.01\\
11.42	0.01\\
11.43	0.01\\
11.44	0.01\\
11.45	0.01\\
11.46	0.01\\
11.47	0.01\\
11.48	0.01\\
11.49	0.01\\
11.5	0.01\\
11.51	0.01\\
11.52	0.01\\
11.53	0.01\\
11.54	0.01\\
11.55	0.01\\
11.56	0.01\\
11.57	0.01\\
11.58	0.01\\
11.59	0.01\\
11.6	0.01\\
11.61	0.01\\
11.62	0.01\\
11.63	0.01\\
11.64	0.01\\
11.65	0.01\\
11.66	0.01\\
11.67	0.01\\
11.68	0.01\\
11.69	0.01\\
11.7	0.01\\
11.71	0.01\\
11.72	0.01\\
11.73	0.01\\
11.74	0.01\\
11.75	0.01\\
11.76	0.01\\
11.77	0.01\\
11.78	0.01\\
11.79	0.01\\
11.8	0.01\\
11.81	0.01\\
11.82	0.01\\
11.83	0.01\\
11.84	0.01\\
11.85	0.01\\
11.86	0.01\\
11.87	0.01\\
11.88	0.01\\
11.89	0.01\\
11.9	0.01\\
11.91	0.01\\
11.92	0.01\\
11.93	0.01\\
11.94	0.01\\
11.95	0.01\\
11.96	0.01\\
11.97	0.01\\
11.98	0.01\\
11.99	0.01\\
12	0.01\\
12.01	0.01\\
12.02	0.01\\
12.03	0.01\\
12.04	0.01\\
12.05	0.01\\
12.06	0.01\\
12.07	0.01\\
12.08	0.01\\
12.09	0.01\\
12.1	0.01\\
12.11	0.01\\
12.12	0.01\\
12.13	0.01\\
12.14	0.01\\
12.15	0.01\\
12.16	0.01\\
12.17	0.01\\
12.18	0.01\\
12.19	0.01\\
12.2	0.01\\
12.21	0.01\\
12.22	0.01\\
12.23	0.01\\
12.24	0.01\\
12.25	0.01\\
12.26	0.01\\
12.27	0.01\\
12.28	0.01\\
12.29	0.01\\
12.3	0.01\\
12.31	0.01\\
12.32	0.01\\
12.33	0.01\\
12.34	0.01\\
12.35	0.01\\
12.36	0.01\\
12.37	0.01\\
12.38	0.01\\
12.39	0.01\\
12.4	0.01\\
12.41	0.01\\
12.42	0.01\\
12.43	0.01\\
12.44	0.01\\
12.45	0.01\\
12.46	0.01\\
12.47	0.01\\
12.48	0.01\\
12.49	0.01\\
12.5	0.01\\
12.51	0.01\\
12.52	0.01\\
12.53	0.01\\
12.54	0.01\\
12.55	0.01\\
12.56	0.01\\
12.57	0.01\\
12.58	0.01\\
12.59	0.01\\
12.6	0.01\\
12.61	0.01\\
12.62	0.01\\
12.63	0.01\\
12.64	0.01\\
12.65	0.01\\
12.66	0.01\\
12.67	0.01\\
12.68	0.01\\
12.69	0.01\\
12.7	0.01\\
12.71	0.01\\
12.72	0.01\\
12.73	0.01\\
12.74	0.01\\
12.75	0.01\\
12.76	0.01\\
12.77	0.01\\
12.78	0.01\\
12.79	0.01\\
12.8	0.01\\
12.81	0.01\\
12.82	0.01\\
12.83	0.01\\
12.84	0.01\\
12.85	0.01\\
12.86	0.01\\
12.87	0.01\\
12.88	0.01\\
12.89	0.01\\
12.9	0.01\\
12.91	0.01\\
12.92	0.01\\
12.93	0.01\\
12.94	0.01\\
12.95	0.01\\
12.96	0.01\\
12.97	0.01\\
12.98	0.01\\
12.99	0.01\\
13	0.01\\
13.01	0.01\\
13.02	0.01\\
13.03	0.01\\
13.04	0.01\\
13.05	0.01\\
13.06	0.01\\
13.07	0.01\\
13.08	0.01\\
13.09	0.01\\
13.1	0.01\\
13.11	0.01\\
13.12	0.01\\
13.13	0.01\\
13.14	0.01\\
13.15	0.01\\
13.16	0.01\\
13.17	0.01\\
13.18	0.01\\
13.19	0.01\\
13.2	0.01\\
13.21	0.01\\
13.22	0.01\\
13.23	0.01\\
13.24	0.01\\
13.25	0.01\\
13.26	0.01\\
13.27	0.01\\
13.28	0.01\\
13.29	0.01\\
13.3	0.01\\
13.31	0.01\\
13.32	0.01\\
13.33	0.01\\
13.34	0.01\\
13.35	0.01\\
13.36	0.01\\
13.37	0.01\\
13.38	0.01\\
13.39	0.01\\
13.4	0.01\\
13.41	0.01\\
13.42	0.01\\
13.43	0.01\\
13.44	0.01\\
13.45	0.01\\
13.46	0.01\\
13.47	0.01\\
13.48	0.01\\
13.49	0.01\\
13.5	0.01\\
13.51	0.01\\
13.52	0.01\\
13.53	0.01\\
13.54	0.01\\
13.55	0.01\\
13.56	0.01\\
13.57	0.01\\
13.58	0.01\\
13.59	0.01\\
13.6	0.01\\
13.61	0.01\\
13.62	0.01\\
13.63	0.01\\
13.64	0.01\\
13.65	0.01\\
13.66	0.01\\
13.67	0.01\\
13.68	0.01\\
13.69	0.01\\
13.7	0.01\\
13.71	0.01\\
13.72	0.01\\
13.73	0.01\\
13.74	0.01\\
13.75	0.01\\
13.76	0.01\\
13.77	0.01\\
13.78	0.01\\
13.79	0.01\\
13.8	0.01\\
13.81	0.01\\
13.82	0.01\\
13.83	0.01\\
13.84	0.01\\
13.85	0.01\\
13.86	0.01\\
13.87	0.01\\
13.88	0.01\\
13.89	0.01\\
13.9	0.01\\
13.91	0.01\\
13.92	0.01\\
13.93	0.01\\
13.94	0.01\\
13.95	0.01\\
13.96	0.01\\
13.97	0.01\\
13.98	0.01\\
13.99	0.01\\
14	0.01\\
14.01	0.01\\
14.02	0.01\\
14.03	0.01\\
14.04	0.01\\
14.05	0.01\\
14.06	0.01\\
14.07	0.01\\
14.08	0.01\\
14.09	0.01\\
14.1	0.01\\
14.11	0.01\\
14.12	0.01\\
14.13	0.01\\
14.14	0.01\\
14.15	0.01\\
14.16	0.01\\
14.17	0.01\\
14.18	0.01\\
14.19	0.01\\
14.2	0.01\\
14.21	0.01\\
14.22	0.01\\
14.23	0.01\\
14.24	0.01\\
14.25	0.01\\
14.26	0.01\\
14.27	0.01\\
14.28	0.01\\
14.29	0.01\\
14.3	0.01\\
14.31	0.01\\
14.32	0.01\\
14.33	0.01\\
14.34	0.01\\
14.35	0.01\\
14.36	0.01\\
14.37	0.01\\
14.38	0.01\\
14.39	0.01\\
14.4	0.01\\
14.41	0.01\\
14.42	0.01\\
14.43	0.01\\
14.44	0.01\\
14.45	0.01\\
14.46	0.01\\
14.47	0.01\\
14.48	0.01\\
14.49	0.01\\
14.5	0.01\\
14.51	0.01\\
14.52	0.01\\
14.53	0.01\\
14.54	0.01\\
14.55	0.01\\
14.56	0.01\\
14.57	0.01\\
14.58	0.01\\
14.59	0.01\\
14.6	0.01\\
14.61	0.01\\
14.62	0.01\\
14.63	0.01\\
14.64	0.01\\
14.65	0.01\\
14.66	0.01\\
14.67	0.01\\
14.68	0.01\\
14.69	0.01\\
14.7	0.01\\
14.71	0.01\\
14.72	0.01\\
14.73	0.01\\
14.74	0.01\\
14.75	0.01\\
14.76	0.01\\
14.77	0.01\\
14.78	0.01\\
14.79	0.01\\
14.8	0.01\\
14.81	0.01\\
14.82	0.01\\
14.83	0.01\\
14.84	0.01\\
14.85	0.01\\
14.86	0.01\\
14.87	0.01\\
14.88	0.01\\
14.89	0.01\\
14.9	0.01\\
14.91	0.01\\
14.92	0.01\\
14.93	0.01\\
14.94	0.01\\
14.95	0.01\\
14.96	0.01\\
14.97	0.01\\
14.98	0.01\\
14.99	0.01\\
15	0.01\\
15.01	0.01\\
15.02	0.01\\
15.03	0.01\\
15.04	0.01\\
15.05	0.01\\
15.06	0.01\\
15.07	0.01\\
15.08	0.01\\
15.09	0.01\\
15.1	0.01\\
15.11	0.01\\
15.12	0.01\\
15.13	0.01\\
15.14	0.01\\
15.15	0.01\\
15.16	0.01\\
15.17	0.01\\
15.18	0.01\\
15.19	0.01\\
15.2	0.01\\
15.21	0.01\\
15.22	0.01\\
15.23	0.01\\
15.24	0.01\\
15.25	0.01\\
15.26	0.01\\
15.27	0.01\\
15.28	0.01\\
15.29	0.01\\
15.3	0.01\\
15.31	0.01\\
15.32	0.01\\
15.33	0.01\\
15.34	0.01\\
15.35	0.01\\
15.36	0.01\\
15.37	0.01\\
15.38	0.01\\
15.39	0.01\\
15.4	0.01\\
15.41	0.01\\
15.42	0.01\\
15.43	0.01\\
15.44	0.01\\
15.45	0.01\\
15.46	0.01\\
15.47	0.01\\
15.48	0.01\\
15.49	0.01\\
15.5	0.01\\
15.51	0.01\\
15.52	0.01\\
15.53	0.01\\
15.54	0.01\\
15.55	0.01\\
15.56	0.01\\
15.57	0.01\\
15.58	0.01\\
15.59	0.01\\
15.6	0.01\\
15.61	0.01\\
15.62	0.01\\
15.63	0.01\\
15.64	0.01\\
15.65	0.01\\
15.66	0.01\\
15.67	0.01\\
15.68	0.01\\
15.69	0.01\\
15.7	0.01\\
15.71	0.01\\
15.72	0.01\\
15.73	0.01\\
15.74	0.01\\
15.75	0.01\\
15.76	0.01\\
15.77	0.01\\
15.78	0.01\\
15.79	0.01\\
15.8	0.01\\
15.81	0.01\\
15.82	0.01\\
15.83	0.01\\
15.84	0.01\\
15.85	0.01\\
15.86	0.01\\
15.87	0.01\\
15.88	0.01\\
15.89	0.01\\
15.9	0.01\\
15.91	0.01\\
15.92	0.01\\
15.93	0.01\\
15.94	0.01\\
15.95	0.01\\
15.96	0.01\\
15.97	0.01\\
15.98	0.01\\
15.99	0.01\\
16	0.01\\
16.01	0.01\\
16.02	0.01\\
16.03	0.01\\
16.04	0.01\\
16.05	0.01\\
16.06	0.01\\
16.07	0.01\\
16.08	0.01\\
16.09	0.01\\
16.1	0.01\\
16.11	0.01\\
16.12	0.01\\
16.13	0.01\\
16.14	0.01\\
16.15	0.01\\
16.16	0.01\\
16.17	0.01\\
16.18	0.01\\
16.19	0.01\\
16.2	0.01\\
16.21	0.01\\
16.22	0.01\\
16.23	0.01\\
16.24	0.01\\
16.25	0.01\\
16.26	0.01\\
16.27	0.01\\
16.28	0.01\\
16.29	0.01\\
16.3	0.01\\
16.31	0.01\\
16.32	0.01\\
16.33	0.01\\
16.34	0.01\\
16.35	0.01\\
16.36	0.01\\
16.37	0.01\\
16.38	0.01\\
16.39	0.01\\
16.4	0.01\\
16.41	0.01\\
16.42	0.01\\
16.43	0.01\\
16.44	0.01\\
16.45	0.01\\
16.46	0.01\\
16.47	0.01\\
16.48	0.01\\
16.49	0.01\\
16.5	0.01\\
16.51	0.01\\
16.52	0.01\\
16.53	0.01\\
16.54	0.01\\
16.55	0.01\\
16.56	0.01\\
16.57	0.01\\
16.58	0.01\\
16.59	0.01\\
16.6	0.01\\
16.61	0.01\\
16.62	0.01\\
16.63	0.01\\
16.64	0.01\\
16.65	0.01\\
16.66	0.01\\
16.67	0.01\\
16.68	0.01\\
16.69	0.01\\
16.7	0.01\\
16.71	0.01\\
16.72	0.01\\
16.73	0.01\\
16.74	0.01\\
16.75	0.01\\
16.76	0.01\\
16.77	0.01\\
16.78	0.01\\
16.79	0.01\\
16.8	0.01\\
16.81	0.01\\
16.82	0.01\\
16.83	0.01\\
16.84	0.01\\
16.85	0.01\\
16.86	0.01\\
16.87	0.01\\
16.88	0.01\\
16.89	0.01\\
16.9	0.01\\
16.91	0.01\\
16.92	0.01\\
16.93	0.01\\
16.94	0.01\\
16.95	0.01\\
16.96	0.01\\
16.97	0.01\\
16.98	0.01\\
16.99	0.01\\
17	0.01\\
17.01	0.01\\
17.02	0.01\\
17.03	0.01\\
17.04	0.01\\
17.05	0.01\\
17.06	0.01\\
17.07	0.01\\
17.08	0.01\\
17.09	0.01\\
17.1	0.01\\
17.11	0.01\\
17.12	0.01\\
17.13	0.01\\
17.14	0.01\\
17.15	0.01\\
17.16	0.01\\
17.17	0.01\\
17.18	0.01\\
17.19	0.01\\
17.2	0.01\\
17.21	0.01\\
17.22	0.01\\
17.23	0.01\\
17.24	0.01\\
17.25	0.01\\
17.26	0.01\\
17.27	0.01\\
17.28	0.01\\
17.29	0.01\\
17.3	0.01\\
17.31	0.01\\
17.32	0.01\\
17.33	0.01\\
17.34	0.01\\
17.35	0.01\\
17.36	0.01\\
17.37	0.01\\
17.38	0.01\\
17.39	0.01\\
17.4	0.01\\
17.41	0.01\\
17.42	0.01\\
17.43	0.01\\
17.44	0.01\\
17.45	0.01\\
17.46	0.01\\
17.47	0.01\\
17.48	0.01\\
17.49	0.01\\
17.5	0.01\\
17.51	0.01\\
17.52	0.01\\
17.53	0.01\\
17.54	0.01\\
17.55	0.01\\
17.56	0.01\\
17.57	0.01\\
17.58	0.01\\
17.59	0.01\\
17.6	0.01\\
17.61	0.01\\
17.62	0.01\\
17.63	0.01\\
17.64	0.01\\
17.65	0.01\\
17.66	0.01\\
17.67	0.01\\
17.68	0.01\\
17.69	0.01\\
17.7	0.01\\
17.71	0.01\\
17.72	0.01\\
17.73	0.01\\
17.74	0.01\\
17.75	0.01\\
17.76	0.01\\
17.77	0.01\\
17.78	0.01\\
17.79	0.01\\
17.8	0.01\\
17.81	0.01\\
17.82	0.01\\
17.83	0.01\\
17.84	0.01\\
17.85	0.01\\
17.86	0.01\\
17.87	0.01\\
17.88	0.01\\
17.89	0.01\\
17.9	0.01\\
17.91	0.01\\
17.92	0.01\\
17.93	0.01\\
17.94	0.01\\
17.95	0.01\\
17.96	0.01\\
17.97	0.01\\
17.98	0.01\\
17.99	0.01\\
18	0.01\\
18.01	0.01\\
18.02	0.01\\
18.03	0.01\\
18.04	0.01\\
18.05	0.01\\
18.06	0.01\\
18.07	0.01\\
18.08	0.01\\
18.09	0.01\\
18.1	0.01\\
18.11	0.01\\
18.12	0.01\\
18.13	0.01\\
18.14	0.01\\
18.15	0.01\\
18.16	0.01\\
18.17	0.01\\
18.18	0.01\\
18.19	0.01\\
18.2	0.01\\
18.21	0.01\\
18.22	0.01\\
18.23	0.01\\
18.24	0.01\\
18.25	0.01\\
18.26	0.01\\
18.27	0.01\\
18.28	0.01\\
18.29	0.01\\
18.3	0.01\\
18.31	0.01\\
18.32	0.01\\
18.33	0.01\\
18.34	0.01\\
18.35	0.01\\
18.36	0.01\\
18.37	0.01\\
18.38	0.01\\
18.39	0.01\\
18.4	0.01\\
18.41	0.01\\
18.42	0.01\\
18.43	0.01\\
18.44	0.01\\
18.45	0.01\\
18.46	0.01\\
18.47	0.01\\
18.48	0.01\\
18.49	0.01\\
18.5	0.01\\
18.51	0.01\\
18.52	0.01\\
18.53	0.01\\
18.54	0.01\\
18.55	0.01\\
18.56	0.01\\
18.57	0.01\\
18.58	0.01\\
18.59	0.01\\
18.6	0.01\\
18.61	0.01\\
18.62	0.01\\
18.63	0.01\\
18.64	0.01\\
18.65	0.01\\
18.66	0.01\\
18.67	0.01\\
18.68	0.01\\
18.69	0.01\\
18.7	0.01\\
18.71	0.01\\
18.72	0.01\\
18.73	0.01\\
18.74	0.01\\
18.75	0.01\\
18.76	0.01\\
18.77	0.01\\
18.78	0.01\\
18.79	0.01\\
18.8	0.01\\
18.81	0.01\\
18.82	0.01\\
18.83	0.01\\
18.84	0.01\\
18.85	0.01\\
18.86	0.01\\
18.87	0.01\\
18.88	0.01\\
18.89	0.01\\
18.9	0.01\\
18.91	0.01\\
18.92	0.01\\
18.93	0.01\\
18.94	0.01\\
18.95	0.01\\
18.96	0.01\\
18.97	0.01\\
18.98	0.01\\
18.99	0.01\\
19	0.01\\
19.01	0.01\\
19.02	0.01\\
19.03	0.01\\
19.04	0.01\\
19.05	0.01\\
19.06	0.01\\
19.07	0.01\\
19.08	0.01\\
19.09	0.01\\
19.1	0.01\\
19.11	0.01\\
19.12	0.01\\
19.13	0.01\\
19.14	0.01\\
19.15	0.01\\
19.16	0.01\\
19.17	0.01\\
19.18	0.01\\
19.19	0.01\\
19.2	0.01\\
19.21	0.01\\
19.22	0.01\\
19.23	0.01\\
19.24	0.01\\
19.25	0.01\\
19.26	0.01\\
19.27	0.01\\
19.28	0.01\\
19.29	0.01\\
19.3	0.01\\
19.31	0.01\\
19.32	0.01\\
19.33	0.01\\
19.34	0.01\\
19.35	0.01\\
19.36	0.01\\
19.37	0.01\\
19.38	0.01\\
19.39	0.01\\
19.4	0.01\\
19.41	0.01\\
19.42	0.01\\
19.43	0.01\\
19.44	0.01\\
19.45	0.01\\
19.46	0.01\\
19.47	0.01\\
19.48	0.01\\
19.49	0.01\\
19.5	0.01\\
19.51	0.01\\
19.52	0.01\\
19.53	0.01\\
19.54	0.01\\
19.55	0.01\\
19.56	0.01\\
19.57	0.01\\
19.58	0.01\\
19.59	0.01\\
19.6	0.01\\
19.61	0.01\\
19.62	0.01\\
19.63	0.01\\
19.64	0.01\\
19.65	0.01\\
19.66	0.01\\
19.67	0.01\\
19.68	0.01\\
19.69	0.01\\
19.7	0.01\\
19.71	0.01\\
19.72	0.01\\
19.73	0.01\\
19.74	0.01\\
19.75	0.01\\
19.76	0.01\\
19.77	0.01\\
19.78	0.01\\
19.79	0.01\\
19.8	0.01\\
19.81	0.01\\
19.82	0.01\\
19.83	0.01\\
19.84	0.01\\
19.85	0.01\\
19.86	0.01\\
19.87	0.01\\
19.88	0.01\\
19.89	0.01\\
19.9	0.01\\
19.91	0.01\\
19.92	0.01\\
19.93	0.01\\
19.94	0.01\\
19.95	0.01\\
19.96	0.01\\
19.97	0.01\\
19.98	0.01\\
19.99	0.01\\
20	0.01\\
20.01	0.01\\
20.02	0.01\\
20.03	0.01\\
20.04	0.01\\
20.05	0.01\\
20.06	0.01\\
20.07	0.01\\
20.08	0.01\\
20.09	0.01\\
20.1	0.01\\
20.11	0.01\\
20.12	0.01\\
20.13	0.01\\
20.14	0.01\\
20.15	0.01\\
20.16	0.01\\
20.17	0.01\\
20.18	0.01\\
20.19	0.01\\
20.2	0.01\\
20.21	0.01\\
20.22	0.01\\
20.23	0.01\\
20.24	0.01\\
20.25	0.01\\
20.26	0.01\\
20.27	0.01\\
20.28	0.01\\
20.29	0.01\\
20.3	0.01\\
20.31	0.01\\
20.32	0.01\\
20.33	0.01\\
20.34	0.01\\
20.35	0.01\\
20.36	0.01\\
20.37	0.01\\
20.38	0.01\\
20.39	0.01\\
20.4	0.01\\
20.41	0.01\\
20.42	0.01\\
20.43	0.01\\
20.44	0.01\\
20.45	0.01\\
20.46	0.01\\
20.47	0.01\\
20.48	0.01\\
20.49	0.01\\
20.5	0.01\\
20.51	0.01\\
20.52	0.01\\
20.53	0.01\\
20.54	0.01\\
20.55	0.01\\
20.56	0.01\\
20.57	0.01\\
20.58	0.01\\
20.59	0.01\\
20.6	0.01\\
20.61	0.01\\
20.62	0.01\\
20.63	0.01\\
20.64	0.01\\
20.65	0.01\\
20.66	0.01\\
20.67	0.01\\
20.68	0.01\\
20.69	0.01\\
20.7	0.01\\
20.71	0.01\\
20.72	0.01\\
20.73	0.01\\
20.74	0.01\\
20.75	0.01\\
20.76	0.01\\
20.77	0.01\\
20.78	0.01\\
20.79	0.01\\
20.8	0.01\\
20.81	0.01\\
20.82	0.01\\
20.83	0.01\\
20.84	0.01\\
20.85	0.01\\
20.86	0.01\\
20.87	0.01\\
20.88	0.01\\
20.89	0.01\\
20.9	0.01\\
20.91	0.01\\
20.92	0.01\\
20.93	0.01\\
20.94	0.01\\
20.95	0.01\\
20.96	0.01\\
20.97	0.01\\
20.98	0.01\\
20.99	0.01\\
21	0.01\\
21.01	0.01\\
21.02	0.01\\
21.03	0.01\\
21.04	0.01\\
21.05	0.01\\
21.06	0.01\\
21.07	0.01\\
21.08	0.01\\
21.09	0.01\\
21.1	0.01\\
21.11	0.01\\
21.12	0.01\\
21.13	0.01\\
21.14	0.01\\
21.15	0.01\\
21.16	0.01\\
21.17	0.01\\
21.18	0.01\\
21.19	0.01\\
21.2	0.01\\
21.21	0.01\\
21.22	0.01\\
21.23	0.01\\
21.24	0.01\\
21.25	0.01\\
21.26	0.01\\
21.27	0.01\\
21.28	0.01\\
21.29	0.01\\
21.3	0.01\\
21.31	0.01\\
21.32	0.01\\
21.33	0.01\\
21.34	0.01\\
21.35	0.01\\
21.36	0.01\\
21.37	0.01\\
21.38	0.01\\
21.39	0.01\\
21.4	0.01\\
21.41	0.01\\
21.42	0.01\\
21.43	0.01\\
21.44	0.01\\
21.45	0.01\\
21.46	0.01\\
21.47	0.01\\
21.48	0.01\\
21.49	0.01\\
21.5	0.01\\
21.51	0.01\\
21.52	0.01\\
21.53	0.01\\
21.54	0.01\\
21.55	0.01\\
21.56	0.01\\
21.57	0.01\\
21.58	0.01\\
21.59	0.01\\
21.6	0.01\\
21.61	0.01\\
21.62	0.01\\
21.63	0.01\\
21.64	0.01\\
21.65	0.01\\
21.66	0.01\\
21.67	0.01\\
21.68	0.01\\
21.69	0.01\\
21.7	0.01\\
21.71	0.01\\
21.72	0.01\\
21.73	0.01\\
21.74	0.01\\
21.75	0.01\\
21.76	0.01\\
21.77	0.01\\
21.78	0.01\\
21.79	0.01\\
21.8	0.01\\
21.81	0.01\\
21.82	0.01\\
21.83	0.01\\
21.84	0.01\\
21.85	0.01\\
21.86	0.01\\
21.87	0.01\\
21.88	0.01\\
21.89	0.01\\
21.9	0.01\\
21.91	0.01\\
21.92	0.01\\
21.93	0.01\\
21.94	0.01\\
21.95	0.01\\
21.96	0.01\\
21.97	0.01\\
21.98	0.01\\
21.99	0.01\\
22	0.01\\
22.01	0.01\\
22.02	0.01\\
22.03	0.01\\
22.04	0.01\\
22.05	0.01\\
22.06	0.01\\
22.07	0.01\\
22.08	0.01\\
22.09	0.01\\
22.1	0.01\\
22.11	0.01\\
22.12	0.01\\
22.13	0.01\\
22.14	0.01\\
22.15	0.01\\
22.16	0.01\\
22.17	0.01\\
22.18	0.01\\
22.19	0.01\\
22.2	0.01\\
22.21	0.01\\
22.22	0.01\\
22.23	0.01\\
22.24	0.01\\
22.25	0.01\\
22.26	0.01\\
22.27	0.01\\
22.28	0.01\\
22.29	0.01\\
22.3	0.01\\
22.31	0.01\\
22.32	0.01\\
22.33	0.01\\
22.34	0.01\\
22.35	0.01\\
22.36	0.01\\
22.37	0.01\\
22.38	0.01\\
22.39	0.01\\
22.4	0.01\\
22.41	0.01\\
22.42	0.01\\
22.43	0.01\\
22.44	0.01\\
22.45	0.01\\
22.46	0.01\\
22.47	0.01\\
22.48	0.01\\
22.49	0.01\\
22.5	0.01\\
22.51	0.01\\
22.52	0.01\\
22.53	0.01\\
22.54	0.01\\
22.55	0.01\\
22.56	0.01\\
22.57	0.01\\
22.58	0.01\\
22.59	0.01\\
22.6	0.01\\
22.61	0.01\\
22.62	0.01\\
22.63	0.01\\
22.64	0.01\\
22.65	0.01\\
22.66	0.01\\
22.67	0.01\\
22.68	0.01\\
22.69	0.01\\
22.7	0.01\\
22.71	0.01\\
22.72	0.01\\
22.73	0.01\\
22.74	0.01\\
22.75	0.01\\
22.76	0.01\\
22.77	0.01\\
22.78	0.01\\
22.79	0.01\\
22.8	0.01\\
22.81	0.01\\
22.82	0.01\\
22.83	0.01\\
22.84	0.01\\
22.85	0.01\\
22.86	0.01\\
22.87	0.01\\
22.88	0.01\\
22.89	0.01\\
22.9	0.01\\
22.91	0.01\\
22.92	0.01\\
22.93	0.01\\
22.94	0.01\\
22.95	0.01\\
22.96	0.01\\
22.97	0.01\\
22.98	0.01\\
22.99	0.01\\
23	0.01\\
23.01	0.01\\
23.02	0.01\\
23.03	0.01\\
23.04	0.01\\
23.05	0.01\\
23.06	0.01\\
23.07	0.01\\
23.08	0.01\\
23.09	0.01\\
23.1	0.01\\
23.11	0.01\\
23.12	0.01\\
23.13	0.01\\
23.14	0.01\\
23.15	0.01\\
23.16	0.01\\
23.17	0.01\\
23.18	0.01\\
23.19	0.01\\
23.2	0.01\\
23.21	0.01\\
23.22	0.01\\
23.23	0.01\\
23.24	0.01\\
23.25	0.01\\
23.26	0.01\\
23.27	0.01\\
23.28	0.01\\
23.29	0.01\\
23.3	0.01\\
23.31	0.01\\
23.32	0.01\\
23.33	0.01\\
23.34	0.01\\
23.35	0.01\\
23.36	0.01\\
23.37	0.01\\
23.38	0.01\\
23.39	0.01\\
23.4	0.01\\
23.41	0.01\\
23.42	0.01\\
23.43	0.01\\
23.44	0.01\\
23.45	0.01\\
23.46	0.01\\
23.47	0.01\\
23.48	0.01\\
23.49	0.01\\
23.5	0.01\\
23.51	0.01\\
23.52	0.01\\
23.53	0.01\\
23.54	0.01\\
23.55	0.01\\
23.56	0.01\\
23.57	0.01\\
23.58	0.01\\
23.59	0.01\\
23.6	0.01\\
23.61	0.01\\
23.62	0.01\\
23.63	0.01\\
23.64	0.01\\
23.65	0.01\\
23.66	0.01\\
23.67	0.01\\
23.68	0.01\\
23.69	0.01\\
23.7	0.01\\
23.71	0.01\\
23.72	0.01\\
23.73	0.01\\
23.74	0.01\\
23.75	0.01\\
23.76	0.01\\
23.77	0.01\\
23.78	0.01\\
23.79	0.01\\
23.8	0.01\\
23.81	0.01\\
23.82	0.01\\
23.83	0.01\\
23.84	0.01\\
23.85	0.01\\
23.86	0.01\\
23.87	0.01\\
23.88	0.01\\
23.89	0.01\\
23.9	0.01\\
23.91	0.01\\
23.92	0.01\\
23.93	0.01\\
23.94	0.01\\
23.95	0.01\\
23.96	0.01\\
23.97	0.01\\
23.98	0.01\\
23.99	0.01\\
24	0.01\\
24.01	0.01\\
24.02	0.01\\
24.03	0.01\\
24.04	0.01\\
24.05	0.01\\
24.06	0.01\\
24.07	0.01\\
24.08	0.01\\
24.09	0.01\\
24.1	0.01\\
24.11	0.01\\
24.12	0.01\\
24.13	0.01\\
24.14	0.01\\
24.15	0.01\\
24.16	0.01\\
24.17	0.01\\
24.18	0.01\\
24.19	0.01\\
24.2	0.01\\
24.21	0.01\\
24.22	0.01\\
24.23	0.01\\
24.24	0.01\\
24.25	0.01\\
24.26	0.01\\
24.27	0.01\\
24.28	0.01\\
24.29	0.01\\
24.3	0.01\\
24.31	0.01\\
24.32	0.01\\
24.33	0.01\\
24.34	0.01\\
24.35	0.01\\
24.36	0.01\\
24.37	0.01\\
24.38	0.01\\
24.39	0.01\\
24.4	0.01\\
24.41	0.01\\
24.42	0.01\\
24.43	0.01\\
24.44	0.01\\
24.45	0.01\\
24.46	0.01\\
24.47	0.01\\
24.48	0.01\\
24.49	0.01\\
24.5	0.01\\
24.51	0.01\\
24.52	0.01\\
24.53	0.01\\
24.54	0.01\\
24.55	0.01\\
24.56	0.01\\
24.57	0.01\\
24.58	0.01\\
24.59	0.01\\
24.6	0.01\\
24.61	0.01\\
24.62	0.01\\
24.63	0.01\\
24.64	0.01\\
24.65	0.01\\
24.66	0.01\\
24.67	0.01\\
24.68	0.01\\
24.69	0.01\\
24.7	0.01\\
24.71	0.01\\
24.72	0.01\\
24.73	0.01\\
24.74	0.01\\
24.75	0.01\\
24.76	0.01\\
24.77	0.01\\
24.78	0.01\\
24.79	0.01\\
24.8	0.01\\
24.81	0.01\\
24.82	0.01\\
24.83	0.01\\
24.84	0.01\\
24.85	0.01\\
24.86	0.01\\
24.87	0.01\\
24.88	0.01\\
24.89	0.01\\
24.9	0.01\\
24.91	0.01\\
24.92	0.01\\
24.93	0.01\\
24.94	0.01\\
24.95	0.01\\
24.96	0.01\\
24.97	0.01\\
24.98	0.01\\
24.99	0.01\\
25	0.01\\
25.01	0.01\\
25.02	0.01\\
25.03	0.01\\
25.04	0.01\\
25.05	0.01\\
25.06	0.01\\
25.07	0.01\\
25.08	0.01\\
25.09	0.01\\
25.1	0.01\\
25.11	0.01\\
25.12	0.01\\
25.13	0.01\\
25.14	0.01\\
25.15	0.01\\
25.16	0.01\\
25.17	0.01\\
25.18	0.01\\
25.19	0.01\\
25.2	0.01\\
25.21	0.01\\
25.22	0.01\\
25.23	0.01\\
25.24	0.01\\
25.25	0.01\\
25.26	0.01\\
25.27	0.01\\
25.28	0.01\\
25.29	0.01\\
25.3	0.01\\
25.31	0.01\\
25.32	0.01\\
25.33	0.01\\
25.34	0.01\\
25.35	0.01\\
25.36	0.01\\
25.37	0.01\\
25.38	0.01\\
25.39	0.01\\
25.4	0.01\\
25.41	0.01\\
25.42	0.01\\
25.43	0.01\\
25.44	0.01\\
25.45	0.01\\
25.46	0.01\\
25.47	0.01\\
25.48	0.01\\
25.49	0.01\\
25.5	0.01\\
25.51	0.01\\
25.52	0.01\\
25.53	0.01\\
25.54	0.01\\
25.55	0.01\\
25.56	0.01\\
25.57	0.01\\
25.58	0.01\\
25.59	0.01\\
25.6	0.01\\
25.61	0.01\\
25.62	0.01\\
25.63	0.01\\
25.64	0.01\\
25.65	0.01\\
25.66	0.01\\
25.67	0.01\\
25.68	0.01\\
25.69	0.01\\
25.7	0.01\\
25.71	0.01\\
25.72	0.01\\
25.73	0.01\\
25.74	0.01\\
25.75	0.01\\
25.76	0.01\\
25.77	0.01\\
25.78	0.01\\
25.79	0.01\\
25.8	0.01\\
25.81	0.01\\
25.82	0.01\\
25.83	0.01\\
25.84	0.01\\
25.85	0.01\\
25.86	0.01\\
25.87	0.01\\
25.88	0.01\\
25.89	0.01\\
25.9	0.01\\
25.91	0.01\\
25.92	0.01\\
25.93	0.01\\
25.94	0.01\\
25.95	0.01\\
25.96	0.01\\
25.97	0.01\\
25.98	0.01\\
25.99	0.01\\
26	0.01\\
26.01	0.01\\
26.02	0.01\\
26.03	0.01\\
26.04	0.01\\
26.05	0.01\\
26.06	0.01\\
26.07	0.01\\
26.08	0.01\\
26.09	0.01\\
26.1	0.01\\
26.11	0.01\\
26.12	0.01\\
26.13	0.01\\
26.14	0.01\\
26.15	0.01\\
26.16	0.01\\
26.17	0.01\\
26.18	0.01\\
26.19	0.01\\
26.2	0.01\\
26.21	0.01\\
26.22	0.01\\
26.23	0.01\\
26.24	0.01\\
26.25	0.01\\
26.26	0.01\\
26.27	0.01\\
26.28	0.01\\
26.29	0.01\\
26.3	0.01\\
26.31	0.01\\
26.32	0.01\\
26.33	0.01\\
26.34	0.01\\
26.35	0.01\\
26.36	0.01\\
26.37	0.01\\
26.38	0.01\\
26.39	0.01\\
26.4	0.01\\
26.41	0.01\\
26.42	0.01\\
26.43	0.01\\
26.44	0.01\\
26.45	0.01\\
26.46	0.01\\
26.47	0.01\\
26.48	0.01\\
26.49	0.01\\
26.5	0.01\\
26.51	0.01\\
26.52	0.01\\
26.53	0.01\\
26.54	0.01\\
26.55	0.01\\
26.56	0.01\\
26.57	0.01\\
26.58	0.01\\
26.59	0.01\\
26.6	0.01\\
26.61	0.01\\
26.62	0.01\\
26.63	0.01\\
26.64	0.01\\
26.65	0.01\\
26.66	0.01\\
26.67	0.01\\
26.68	0.01\\
26.69	0.01\\
26.7	0.01\\
26.71	0.01\\
26.72	0.01\\
26.73	0.01\\
26.74	0.01\\
26.75	0.01\\
26.76	0.01\\
26.77	0.01\\
26.78	0.01\\
26.79	0.01\\
26.8	0.01\\
26.81	0.01\\
26.82	0.01\\
26.83	0.01\\
26.84	0.01\\
26.85	0.01\\
26.86	0.01\\
26.87	0.01\\
26.88	0.01\\
26.89	0.01\\
26.9	0.01\\
26.91	0.01\\
26.92	0.01\\
26.93	0.01\\
26.94	0.01\\
26.95	0.01\\
26.96	0.01\\
26.97	0.01\\
26.98	0.01\\
26.99	0.01\\
27	0.01\\
27.01	0.01\\
27.02	0.01\\
27.03	0.01\\
27.04	0.01\\
27.05	0.01\\
27.06	0.01\\
27.07	0.01\\
27.08	0.01\\
27.09	0.01\\
27.1	0.01\\
27.11	0.01\\
27.12	0.01\\
27.13	0.01\\
27.14	0.01\\
27.15	0.01\\
27.16	0.01\\
27.17	0.01\\
27.18	0.01\\
27.19	0.01\\
27.2	0.01\\
27.21	0.01\\
27.22	0.01\\
27.23	0.01\\
27.24	0.01\\
27.25	0.01\\
27.26	0.01\\
27.27	0.01\\
27.28	0.01\\
27.29	0.01\\
27.3	0.01\\
27.31	0.01\\
27.32	0.01\\
27.33	0.01\\
27.34	0.01\\
27.35	0.01\\
27.36	0.01\\
27.37	0.01\\
27.38	0.01\\
27.39	0.01\\
27.4	0.01\\
27.41	0.01\\
27.42	0.01\\
27.43	0.01\\
27.44	0.01\\
27.45	0.01\\
27.46	0.01\\
27.47	0.01\\
27.48	0.01\\
27.49	0.01\\
27.5	0.01\\
27.51	0.01\\
27.52	0.01\\
27.53	0.01\\
27.54	0.01\\
27.55	0.01\\
27.56	0.01\\
27.57	0.01\\
27.58	0.01\\
27.59	0.01\\
27.6	0.01\\
27.61	0.01\\
27.62	0.01\\
27.63	0.01\\
27.64	0.01\\
27.65	0.01\\
27.66	0.01\\
27.67	0.01\\
27.68	0.01\\
27.69	0.01\\
27.7	0.01\\
27.71	0.01\\
27.72	0.01\\
27.73	0.01\\
27.74	0.01\\
27.75	0.01\\
27.76	0.01\\
27.77	0.01\\
27.78	0.01\\
27.79	0.01\\
27.8	0.01\\
27.81	0.01\\
27.82	0.01\\
27.83	0.01\\
27.84	0.01\\
27.85	0.01\\
27.86	0.01\\
27.87	0.01\\
27.88	0.01\\
27.89	0.01\\
27.9	0.01\\
27.91	0.01\\
27.92	0.01\\
27.93	0.01\\
27.94	0.01\\
27.95	0.01\\
27.96	0.01\\
27.97	0.01\\
27.98	0.01\\
27.99	0.01\\
28	0.01\\
28.01	0.01\\
28.02	0.01\\
28.03	0.01\\
28.04	0.01\\
28.05	0.01\\
28.06	0.01\\
28.07	0.01\\
28.08	0.01\\
28.09	0.01\\
28.1	0.01\\
28.11	0.01\\
28.12	0.01\\
28.13	0.01\\
28.14	0.01\\
28.15	0.01\\
28.16	0.01\\
28.17	0.01\\
28.18	0.01\\
28.19	0.01\\
28.2	0.01\\
28.21	0.01\\
28.22	0.01\\
28.23	0.01\\
28.24	0.01\\
28.25	0.01\\
28.26	0.01\\
28.27	0.01\\
28.28	0.01\\
28.29	0.01\\
28.3	0.01\\
28.31	0.01\\
28.32	0.01\\
28.33	0.01\\
28.34	0.01\\
28.35	0.01\\
28.36	0.01\\
28.37	0.01\\
28.38	0.01\\
28.39	0.01\\
28.4	0.01\\
28.41	0.01\\
28.42	0.01\\
28.43	0.01\\
28.44	0.01\\
28.45	0.01\\
28.46	0.01\\
28.47	0.01\\
28.48	0.01\\
28.49	0.01\\
28.5	0.01\\
28.51	0.01\\
28.52	0.01\\
28.53	0.01\\
28.54	0.01\\
28.55	0.01\\
28.56	0.01\\
28.57	0.01\\
28.58	0.01\\
28.59	0.01\\
28.6	0.01\\
28.61	0.01\\
28.62	0.01\\
28.63	0.01\\
28.64	0.01\\
28.65	0.01\\
28.66	0.01\\
28.67	0.01\\
28.68	0.01\\
28.69	0.01\\
28.7	0.01\\
28.71	0.01\\
28.72	0.01\\
28.73	0.01\\
28.74	0.01\\
28.75	0.01\\
28.76	0.01\\
28.77	0.01\\
28.78	0.01\\
28.79	0.01\\
28.8	0.01\\
28.81	0.01\\
28.82	0.01\\
28.83	0.01\\
28.84	0.01\\
28.85	0.01\\
28.86	0.01\\
28.87	0.01\\
28.88	0.01\\
28.89	0.01\\
28.9	0.01\\
28.91	0.01\\
28.92	0.01\\
28.93	0.01\\
28.94	0.01\\
28.95	0.01\\
28.96	0.01\\
28.97	0.01\\
28.98	0.01\\
28.99	0.01\\
29	0.01\\
29.01	0.01\\
29.02	0.01\\
29.03	0.01\\
29.04	0.01\\
29.05	0.01\\
29.06	0.01\\
29.07	0.01\\
29.08	0.01\\
29.09	0.01\\
29.1	0.01\\
29.11	0.01\\
29.12	0.01\\
29.13	0.01\\
29.14	0.01\\
29.15	0.01\\
29.16	0.01\\
29.17	0.01\\
29.18	0.01\\
29.19	0.01\\
29.2	0.01\\
29.21	0.01\\
29.22	0.01\\
29.23	0.01\\
29.24	0.01\\
29.25	0.01\\
29.26	0.01\\
29.27	0.01\\
29.28	0.01\\
29.29	0.01\\
29.3	0.01\\
29.31	0.01\\
29.32	0.01\\
29.33	0.01\\
29.34	0.01\\
29.35	0.01\\
29.36	0.01\\
29.37	0.01\\
29.38	0.01\\
29.39	0.01\\
29.4	0.01\\
29.41	0.01\\
29.42	0.01\\
29.43	0.01\\
29.44	0.01\\
29.45	0.01\\
29.46	0.01\\
29.47	0.01\\
29.48	0.01\\
29.49	0.01\\
29.5	0.01\\
29.51	0.01\\
29.52	0.01\\
29.53	0.01\\
29.54	0.01\\
29.55	0.01\\
29.56	0.01\\
29.57	0.01\\
29.58	0.01\\
29.59	0.01\\
29.6	0.01\\
29.61	0.01\\
29.62	0.01\\
29.63	0.01\\
29.64	0.01\\
29.65	0.01\\
29.66	0.01\\
29.67	0.01\\
29.68	0.01\\
29.69	0.01\\
29.7	0.01\\
29.71	0.01\\
29.72	0.01\\
29.73	0.01\\
29.74	0.01\\
29.75	0.01\\
29.76	0.01\\
29.77	0.01\\
29.78	0.01\\
29.79	0.01\\
29.8	0.01\\
29.81	0.01\\
29.82	0.01\\
29.83	0.01\\
29.84	0.01\\
29.85	0.01\\
29.86	0.01\\
29.87	0.01\\
29.88	0.01\\
29.89	0.01\\
29.9	0.01\\
29.91	0.01\\
29.92	0.01\\
29.93	0.01\\
29.94	0.01\\
29.95	0.01\\
29.96	0.01\\
29.97	0.01\\
29.98	0.01\\
29.99	0.01\\
30	0.01\\
30.01	0.01\\
30.02	0.01\\
30.03	0.01\\
30.04	0.01\\
30.05	0.01\\
30.06	0.01\\
30.07	0.01\\
30.08	0.01\\
30.09	0.01\\
30.1	0.01\\
30.11	0.01\\
30.12	0.01\\
30.13	0.01\\
30.14	0.01\\
30.15	0.01\\
30.16	0.01\\
30.17	0.01\\
30.18	0.01\\
30.19	0.01\\
30.2	0.01\\
30.21	0.01\\
30.22	0.01\\
30.23	0.01\\
30.24	0.01\\
30.25	0.01\\
30.26	0.01\\
30.27	0.01\\
30.28	0.01\\
30.29	0.01\\
30.3	0.01\\
30.31	0.01\\
30.32	0.01\\
30.33	0.01\\
30.34	0.01\\
30.35	0.01\\
30.36	0.01\\
30.37	0.01\\
30.38	0.01\\
30.39	0.01\\
30.4	0.01\\
30.41	0.01\\
30.42	0.01\\
30.43	0.01\\
30.44	0.01\\
30.45	0.01\\
30.46	0.01\\
30.47	0.01\\
30.48	0.01\\
30.49	0.01\\
30.5	0.01\\
30.51	0.01\\
30.52	0.01\\
30.53	0.01\\
30.54	0.01\\
30.55	0.01\\
30.56	0.01\\
30.57	0.01\\
30.58	0.01\\
30.59	0.01\\
30.6	0.01\\
30.61	0.01\\
30.62	0.01\\
30.63	0.01\\
30.64	0.01\\
30.65	0.01\\
30.66	0.01\\
30.67	0.01\\
30.68	0.01\\
30.69	0.01\\
30.7	0.01\\
30.71	0.01\\
30.72	0.01\\
30.73	0.01\\
30.74	0.01\\
30.75	0.01\\
30.76	0.01\\
30.77	0.01\\
30.78	0.01\\
30.79	0.01\\
30.8	0.01\\
30.81	0.01\\
30.82	0.01\\
30.83	0.01\\
30.84	0.01\\
30.85	0.01\\
30.86	0.01\\
30.87	0.01\\
30.88	0.01\\
30.89	0.01\\
30.9	0.01\\
30.91	0.01\\
30.92	0.01\\
30.93	0.01\\
30.94	0.01\\
30.95	0.01\\
30.96	0.01\\
30.97	0.01\\
30.98	0.01\\
30.99	0.01\\
31	0.01\\
31.01	0.01\\
31.02	0.01\\
31.03	0.01\\
31.04	0.01\\
31.05	0.01\\
31.06	0.01\\
31.07	0.01\\
31.08	0.01\\
31.09	0.01\\
31.1	0.01\\
31.11	0.01\\
31.12	0.01\\
31.13	0.01\\
31.14	0.01\\
31.15	0.01\\
31.16	0.01\\
31.17	0.01\\
31.18	0.01\\
31.19	0.01\\
31.2	0.01\\
31.21	0.01\\
31.22	0.01\\
31.23	0.01\\
31.24	0.01\\
31.25	0.01\\
31.26	0.01\\
31.27	0.01\\
31.28	0.01\\
31.29	0.01\\
31.3	0.01\\
31.31	0.01\\
31.32	0.01\\
31.33	0.01\\
31.34	0.01\\
31.35	0.01\\
31.36	0.01\\
31.37	0.01\\
31.38	0.01\\
31.39	0.01\\
31.4	0.01\\
31.41	0.01\\
31.42	0.01\\
31.43	0.01\\
31.44	0.01\\
31.45	0.01\\
31.46	0.01\\
31.47	0.01\\
31.48	0.01\\
31.49	0.01\\
31.5	0.01\\
31.51	0.01\\
31.52	0.01\\
31.53	0.01\\
31.54	0.01\\
31.55	0.01\\
31.56	0.01\\
31.57	0.01\\
31.58	0.01\\
31.59	0.01\\
31.6	0.01\\
31.61	0.01\\
31.62	0.01\\
31.63	0.01\\
31.64	0.01\\
31.65	0.01\\
31.66	0.01\\
31.67	0.01\\
31.68	0.01\\
31.69	0.01\\
31.7	0.01\\
31.71	0.01\\
31.72	0.01\\
31.73	0.01\\
31.74	0.01\\
31.75	0.01\\
31.76	0.01\\
31.77	0.01\\
31.78	0.01\\
31.79	0.01\\
31.8	0.01\\
31.81	0.01\\
31.82	0.01\\
31.83	0.01\\
31.84	0.01\\
31.85	0.01\\
31.86	0.01\\
31.87	0.01\\
31.88	0.01\\
31.89	0.01\\
31.9	0.01\\
31.91	0.01\\
31.92	0.01\\
31.93	0.01\\
31.94	0.01\\
31.95	0.01\\
31.96	0.01\\
31.97	0.01\\
31.98	0.01\\
31.99	0.01\\
32	0.01\\
32.01	0.01\\
32.02	0.01\\
32.03	0.01\\
32.04	0.01\\
32.05	0.01\\
32.06	0.01\\
32.07	0.01\\
32.08	0.01\\
32.09	0.01\\
32.1	0.01\\
32.11	0.01\\
32.12	0.01\\
32.13	0.01\\
32.14	0.01\\
32.15	0.01\\
32.16	0.01\\
32.17	0.01\\
32.18	0.01\\
32.19	0.01\\
32.2	0.01\\
32.21	0.01\\
32.22	0.01\\
32.23	0.01\\
32.24	0.01\\
32.25	0.01\\
32.26	0.01\\
32.27	0.01\\
32.28	0.01\\
32.29	0.01\\
32.3	0.01\\
32.31	0.01\\
32.32	0.01\\
32.33	0.01\\
32.34	0.01\\
32.35	0.01\\
32.36	0.01\\
32.37	0.01\\
32.38	0.01\\
32.39	0.01\\
32.4	0.01\\
32.41	0.01\\
32.42	0.01\\
32.43	0.01\\
32.44	0.01\\
32.45	0.01\\
32.46	0.01\\
32.47	0.01\\
32.48	0.01\\
32.49	0.01\\
32.5	0.01\\
32.51	0.01\\
32.52	0.01\\
32.53	0.01\\
32.54	0.01\\
32.55	0.01\\
32.56	0.01\\
32.57	0.01\\
32.58	0.01\\
32.59	0.01\\
32.6	0.01\\
32.61	0.01\\
32.62	0.01\\
32.63	0.01\\
32.64	0.01\\
32.65	0.01\\
32.66	0.01\\
32.67	0.01\\
32.68	0.01\\
32.69	0.01\\
32.7	0.01\\
32.71	0.01\\
32.72	0.01\\
32.73	0.01\\
32.74	0.01\\
32.75	0.01\\
32.76	0.01\\
32.77	0.01\\
32.78	0.01\\
32.79	0.01\\
32.8	0.01\\
32.81	0.01\\
32.82	0.01\\
32.83	0.01\\
32.84	0.01\\
32.85	0.01\\
32.86	0.01\\
32.87	0.01\\
32.88	0.01\\
32.89	0.01\\
32.9	0.01\\
32.91	0.01\\
32.92	0.01\\
32.93	0.01\\
32.94	0.01\\
32.95	0.01\\
32.96	0.01\\
32.97	0.01\\
32.98	0.01\\
32.99	0.01\\
33	0.01\\
33.01	0.01\\
33.02	0.01\\
33.03	0.01\\
33.04	0.01\\
33.05	0.01\\
33.06	0.01\\
33.07	0.01\\
33.08	0.01\\
33.09	0.01\\
33.1	0.01\\
33.11	0.01\\
33.12	0.01\\
33.13	0.01\\
33.14	0.01\\
33.15	0.01\\
33.16	0.01\\
33.17	0.01\\
33.18	0.01\\
33.19	0.01\\
33.2	0.01\\
33.21	0.01\\
33.22	0.01\\
33.23	0.01\\
33.24	0.01\\
33.25	0.01\\
33.26	0.01\\
33.27	0.01\\
33.28	0.01\\
33.29	0.01\\
33.3	0.01\\
33.31	0.01\\
33.32	0.01\\
33.33	0.01\\
33.34	0.01\\
33.35	0.01\\
33.36	0.01\\
33.37	0.01\\
33.38	0.01\\
33.39	0.01\\
33.4	0.01\\
33.41	0.01\\
33.42	0.01\\
33.43	0.01\\
33.44	0.01\\
33.45	0.01\\
33.46	0.01\\
33.47	0.01\\
33.48	0.01\\
33.49	0.01\\
33.5	0.01\\
33.51	0.01\\
33.52	0.01\\
33.53	0.01\\
33.54	0.01\\
33.55	0.01\\
33.56	0.01\\
33.57	0.01\\
33.58	0.01\\
33.59	0.01\\
33.6	0.01\\
33.61	0.01\\
33.62	0.01\\
33.63	0.01\\
33.64	0.01\\
33.65	0.01\\
33.66	0.01\\
33.67	0.01\\
33.68	0.01\\
33.69	0.01\\
33.7	0.01\\
33.71	0.01\\
33.72	0.01\\
33.73	0.01\\
33.74	0.01\\
33.75	0.01\\
33.76	0.01\\
33.77	0.01\\
33.78	0.01\\
33.79	0.01\\
33.8	0.01\\
33.81	0.01\\
33.82	0.01\\
33.83	0.01\\
33.84	0.01\\
33.85	0.01\\
33.86	0.01\\
33.87	0.01\\
33.88	0.01\\
33.89	0.01\\
33.9	0.01\\
33.91	0.01\\
33.92	0.01\\
33.93	0.01\\
33.94	0.01\\
33.95	0.01\\
33.96	0.01\\
33.97	0.01\\
33.98	0.01\\
33.99	0.01\\
34	0.01\\
34.01	0.01\\
34.02	0.01\\
34.03	0.01\\
34.04	0.01\\
34.05	0.01\\
34.06	0.01\\
34.07	0.01\\
34.08	0.01\\
34.09	0.01\\
34.1	0.01\\
34.11	0.01\\
34.12	0.01\\
34.13	0.01\\
34.14	0.01\\
34.15	0.01\\
34.16	0.01\\
34.17	0.01\\
34.18	0.01\\
34.19	0.01\\
34.2	0.01\\
34.21	0.01\\
34.22	0.01\\
34.23	0.01\\
34.24	0.01\\
34.25	0.01\\
34.26	0.01\\
34.27	0.01\\
34.28	0.01\\
34.29	0.01\\
34.3	0.01\\
34.31	0.01\\
34.32	0.01\\
34.33	0.01\\
34.34	0.01\\
34.35	0.01\\
34.36	0.01\\
34.37	0.01\\
34.38	0.01\\
34.39	0.01\\
34.4	0.01\\
34.41	0.01\\
34.42	0.01\\
34.43	0.01\\
34.44	0.01\\
34.45	0.01\\
34.46	0.01\\
34.47	0.01\\
34.48	0.01\\
34.49	0.01\\
34.5	0.01\\
34.51	0.01\\
34.52	0.01\\
34.53	0.01\\
34.54	0.01\\
34.55	0.01\\
34.56	0.01\\
34.57	0.01\\
34.58	0.01\\
34.59	0.01\\
34.6	0.01\\
34.61	0.01\\
34.62	0.01\\
34.63	0.01\\
34.64	0.01\\
34.65	0.01\\
34.66	0.01\\
34.67	0.01\\
34.68	0.01\\
34.69	0.01\\
34.7	0.01\\
34.71	0.01\\
34.72	0.01\\
34.73	0.01\\
34.74	0.01\\
34.75	0.01\\
34.76	0.01\\
34.77	0.01\\
34.78	0.01\\
34.79	0.01\\
34.8	0.01\\
34.81	0.01\\
34.82	0.01\\
34.83	0.01\\
34.84	0.01\\
34.85	0.01\\
34.86	0.01\\
34.87	0.01\\
34.88	0.01\\
34.89	0.01\\
34.9	0.01\\
34.91	0.01\\
34.92	0.01\\
34.93	0.01\\
34.94	0.01\\
34.95	0.01\\
34.96	0.01\\
34.97	0.01\\
34.98	0.01\\
34.99	0.01\\
35	0.01\\
35.01	0.01\\
35.02	0.01\\
35.03	0.01\\
35.04	0.01\\
35.05	0.01\\
35.06	0.01\\
35.07	0.01\\
35.08	0.01\\
35.09	0.01\\
35.1	0.01\\
35.11	0.01\\
35.12	0.01\\
35.13	0.01\\
35.14	0.01\\
35.15	0.01\\
35.16	0.01\\
35.17	0.01\\
35.18	0.01\\
35.19	0.01\\
35.2	0.01\\
35.21	0.01\\
35.22	0.01\\
35.23	0.01\\
35.24	0.01\\
35.25	0.01\\
35.26	0.01\\
35.27	0.01\\
35.28	0.01\\
35.29	0.01\\
35.3	0.01\\
35.31	0.01\\
35.32	0.01\\
35.33	0.01\\
35.34	0.01\\
35.35	0.01\\
35.36	0.01\\
35.37	0.01\\
35.38	0.01\\
35.39	0.01\\
35.4	0.01\\
35.41	0.01\\
35.42	0.01\\
35.43	0.01\\
35.44	0.01\\
35.45	0.01\\
35.46	0.01\\
35.47	0.01\\
35.48	0.01\\
35.49	0.01\\
35.5	0.01\\
35.51	0.01\\
35.52	0.01\\
35.53	0.01\\
35.54	0.01\\
35.55	0.01\\
35.56	0.01\\
35.57	0.01\\
35.58	0.01\\
35.59	0.01\\
35.6	0.01\\
35.61	0.01\\
35.62	0.01\\
35.63	0.01\\
35.64	0.01\\
35.65	0.01\\
35.66	0.01\\
35.67	0.01\\
35.68	0.01\\
35.69	0.01\\
35.7	0.01\\
35.71	0.01\\
35.72	0.01\\
35.73	0.01\\
35.74	0.01\\
35.75	0.01\\
35.76	0.01\\
35.77	0.01\\
35.78	0.01\\
35.79	0.01\\
35.8	0.01\\
35.81	0.01\\
35.82	0.01\\
35.83	0.01\\
35.84	0.01\\
35.85	0.01\\
35.86	0.01\\
35.87	0.01\\
35.88	0.01\\
35.89	0.01\\
35.9	0.01\\
35.91	0.01\\
35.92	0.01\\
35.93	0.01\\
35.94	0.01\\
35.95	0.01\\
35.96	0.01\\
35.97	0.01\\
35.98	0.01\\
35.99	0.01\\
36	0.01\\
36.01	0.01\\
36.02	0.01\\
36.03	0.01\\
36.04	0.01\\
36.05	0.01\\
36.06	0.01\\
36.07	0.01\\
36.08	0.01\\
36.09	0.01\\
36.1	0.01\\
36.11	0.01\\
36.12	0.01\\
36.13	0.01\\
36.14	0.01\\
36.15	0.01\\
36.16	0.01\\
36.17	0.01\\
36.18	0.01\\
36.19	0.01\\
36.2	0.01\\
36.21	0.01\\
36.22	0.01\\
36.23	0.01\\
36.24	0.01\\
36.25	0.01\\
36.26	0.01\\
36.27	0.01\\
36.28	0.01\\
36.29	0.01\\
36.3	0.01\\
36.31	0.01\\
36.32	0.01\\
36.33	0.01\\
36.34	0.01\\
36.35	0.01\\
36.36	0.01\\
36.37	0.01\\
36.38	0.01\\
36.39	0.01\\
36.4	0.01\\
36.41	0.01\\
36.42	0.01\\
36.43	0.01\\
36.44	0.01\\
36.45	0.01\\
36.46	0.01\\
36.47	0.01\\
36.48	0.01\\
36.49	0.01\\
36.5	0.01\\
36.51	0.01\\
36.52	0.01\\
36.53	0.01\\
36.54	0.01\\
36.55	0.01\\
36.56	0.01\\
36.57	0.01\\
36.58	0.01\\
36.59	0.01\\
36.6	0.01\\
36.61	0.01\\
36.62	0.01\\
36.63	0.01\\
36.64	0.01\\
36.65	0.01\\
36.66	0.01\\
36.67	0.01\\
36.68	0.01\\
36.69	0.01\\
36.7	0.01\\
36.71	0.01\\
36.72	0.01\\
36.73	0.01\\
36.74	0.01\\
36.75	0.01\\
36.76	0.01\\
36.77	0.01\\
36.78	0.01\\
36.79	0.01\\
36.8	0.01\\
36.81	0.01\\
36.82	0.01\\
36.83	0.01\\
36.84	0.01\\
36.85	0.01\\
36.86	0.01\\
36.87	0.01\\
36.88	0.01\\
36.89	0.01\\
36.9	0.01\\
36.91	0.01\\
36.92	0.01\\
36.93	0.01\\
36.94	0.01\\
36.95	0.01\\
36.96	0.01\\
36.97	0.01\\
36.98	0.01\\
36.99	0.01\\
37	0.01\\
37.01	0.01\\
37.02	0.01\\
37.03	0.01\\
37.04	0.01\\
37.05	0.01\\
37.06	0.01\\
37.07	0.01\\
37.08	0.01\\
37.09	0.01\\
37.1	0.01\\
37.11	0.01\\
37.12	0.01\\
37.13	0.01\\
37.14	0.01\\
37.15	0.01\\
37.16	0.01\\
37.17	0.01\\
37.18	0.01\\
37.19	0.01\\
37.2	0.01\\
37.21	0.01\\
37.22	0.01\\
37.23	0.01\\
37.24	0.01\\
37.25	0.01\\
37.26	0.01\\
37.27	0.01\\
37.28	0.01\\
37.29	0.01\\
37.3	0.01\\
37.31	0.01\\
37.32	0.01\\
37.33	0.01\\
37.34	0.01\\
37.35	0.01\\
37.36	0.01\\
37.37	0.01\\
37.38	0.01\\
37.39	0.01\\
37.4	0.01\\
37.41	0.01\\
37.42	0.01\\
37.43	0.01\\
37.44	0.01\\
37.45	0.01\\
37.46	0.01\\
37.47	0.01\\
37.48	0.01\\
37.49	0.01\\
37.5	0.01\\
37.51	0.01\\
37.52	0.01\\
37.53	0.01\\
37.54	0.01\\
37.55	0.01\\
37.56	0.01\\
37.57	0.01\\
37.58	0.01\\
37.59	0.01\\
37.6	0.01\\
37.61	0.01\\
37.62	0.01\\
37.63	0.01\\
37.64	0.01\\
37.65	0.01\\
37.66	0.01\\
37.67	0.01\\
37.68	0.01\\
37.69	0.01\\
37.7	0.01\\
37.71	0.01\\
37.72	0.01\\
37.73	0.01\\
37.74	0.01\\
37.75	0.01\\
37.76	0.01\\
37.77	0.01\\
37.78	0.01\\
37.79	0.01\\
37.8	0.01\\
37.81	0.01\\
37.82	0.01\\
37.83	0.01\\
37.84	0.01\\
37.85	0.01\\
37.86	0.01\\
37.87	0.01\\
37.88	0.01\\
37.89	0.01\\
37.9	0.01\\
37.91	0.01\\
37.92	0.01\\
37.93	0.01\\
37.94	0.01\\
37.95	0.01\\
37.96	0.01\\
37.97	0.01\\
37.98	0.01\\
37.99	0.01\\
38	0.01\\
38.01	0.01\\
38.02	0.01\\
38.03	0.01\\
38.04	0.01\\
38.05	0.01\\
38.06	0.01\\
38.07	0.01\\
38.08	0.01\\
38.09	0.01\\
38.1	0.01\\
38.11	0.01\\
38.12	0.01\\
38.13	0.01\\
38.14	0.01\\
38.15	0.01\\
38.16	0.01\\
38.17	0.01\\
38.18	0.01\\
38.19	0.01\\
38.2	0.01\\
38.21	0.01\\
38.22	0.01\\
38.23	0.01\\
38.24	0.01\\
38.25	0.01\\
38.26	0.01\\
38.27	0.01\\
38.28	0.01\\
38.29	0.01\\
38.3	0.01\\
38.31	0.01\\
38.32	0.01\\
38.33	0.01\\
38.34	0.01\\
38.35	0.01\\
38.36	0.01\\
38.37	0.01\\
38.38	0.01\\
38.39	0.01\\
38.4	0.01\\
38.41	0.01\\
38.42	0.01\\
38.43	0.01\\
38.44	0.01\\
38.45	0.01\\
38.46	0.01\\
38.47	0.01\\
38.48	0.01\\
38.49	0.01\\
38.5	0.01\\
38.51	0.01\\
38.52	0.01\\
38.53	0.01\\
38.54	0.01\\
38.55	0.01\\
38.56	0.01\\
38.57	0.01\\
38.58	0.01\\
38.59	0.01\\
38.6	0.01\\
38.61	0.01\\
38.62	0.01\\
38.63	0.01\\
38.64	0.01\\
38.65	0.01\\
38.66	0.01\\
38.67	0.01\\
38.68	0.01\\
38.69	0.01\\
38.7	0.01\\
38.71	0.01\\
38.72	0.01\\
38.73	0.01\\
38.74	0.01\\
38.75	0.01\\
38.76	0.01\\
38.77	0.01\\
38.78	0.01\\
38.79	0.01\\
38.8	0.01\\
38.81	0.01\\
38.82	0.01\\
38.83	0.01\\
38.84	0.01\\
38.85	0.01\\
38.86	0.01\\
38.87	0.01\\
38.88	0.01\\
38.89	0.01\\
38.9	0.01\\
38.91	0.01\\
38.92	0.01\\
38.93	0.01\\
38.94	0.01\\
38.95	0.01\\
38.96	0.01\\
38.97	0.01\\
38.98	0.01\\
38.99	0.01\\
39	0.01\\
39.01	0.01\\
39.02	0.01\\
39.03	0.01\\
39.04	0.01\\
39.05	0.01\\
39.06	0.01\\
39.07	0.01\\
39.08	0.01\\
39.09	0.01\\
39.1	0.01\\
39.11	0.01\\
39.12	0.01\\
39.13	0.01\\
39.14	0.01\\
39.15	0.01\\
39.16	0.01\\
39.17	0.01\\
39.18	0.01\\
39.19	0.01\\
39.2	0.01\\
39.21	0.01\\
39.22	0.01\\
39.23	0.01\\
39.24	0.01\\
39.25	0.01\\
39.26	0.01\\
39.27	0.01\\
39.28	0.01\\
39.29	0.01\\
39.3	0.01\\
39.31	0.01\\
39.32	0.01\\
39.33	0.01\\
39.34	0.01\\
39.35	0.01\\
39.36	0.01\\
39.37	0.01\\
39.38	0.01\\
39.39	0.01\\
39.4	0.01\\
39.41	0.01\\
39.42	0.01\\
39.43	0.01\\
39.44	0.01\\
39.45	0.01\\
39.46	0.01\\
39.47	0.01\\
39.48	0.01\\
39.49	0.01\\
39.5	0.01\\
39.51	0.01\\
39.52	0.01\\
39.53	0.01\\
39.54	0.01\\
39.55	0.01\\
39.56	0.01\\
39.57	0.01\\
39.58	0.01\\
39.59	0.01\\
39.6	0.01\\
39.61	0.01\\
39.62	0.01\\
39.63	0.01\\
39.64	0.01\\
39.65	0.01\\
39.66	0.01\\
39.67	0.01\\
39.68	0.01\\
39.69	0.01\\
39.7	0.01\\
39.71	0.01\\
39.72	0.01\\
39.73	0.01\\
39.74	0.01\\
39.75	0.01\\
39.76	0.01\\
39.77	0.01\\
39.78	0.01\\
39.79	0.01\\
39.8	0.01\\
39.81	0.01\\
39.82	0.01\\
39.83	0.01\\
39.84	0.01\\
39.85	0.01\\
39.86	0.01\\
39.87	0.01\\
39.88	0.01\\
39.89	0.01\\
39.9	0.01\\
39.91	0.01\\
39.92	0.01\\
39.93	0.01\\
39.94	0.01\\
39.95	0.01\\
39.96	0.01\\
39.97	0.01\\
39.98	0.01\\
39.99	0.01\\
40	0.01\\
40.01	0.01\\
};
\addplot [color=blue,dashed,forget plot]
  table[row sep=crcr]{%
40.01	0.01\\
40.02	0.01\\
40.03	0.01\\
40.04	0.01\\
40.05	0.01\\
40.06	0.01\\
40.07	0.01\\
40.08	0.01\\
40.09	0.01\\
40.1	0.01\\
40.11	0.01\\
40.12	0.01\\
40.13	0.01\\
40.14	0.01\\
40.15	0.01\\
40.16	0.01\\
40.17	0.01\\
40.18	0.01\\
40.19	0.01\\
40.2	0.01\\
40.21	0.01\\
40.22	0.01\\
40.23	0.01\\
40.24	0.01\\
40.25	0.01\\
40.26	0.01\\
40.27	0.01\\
40.28	0.01\\
40.29	0.01\\
40.3	0.01\\
40.31	0.01\\
40.32	0.01\\
40.33	0.01\\
40.34	0.01\\
40.35	0.01\\
40.36	0.01\\
40.37	0.01\\
40.38	0.01\\
40.39	0.01\\
40.4	0.01\\
40.41	0.01\\
40.42	0.01\\
40.43	0.01\\
40.44	0.01\\
40.45	0.01\\
40.46	0.01\\
40.47	0.01\\
40.48	0.01\\
40.49	0.01\\
40.5	0.01\\
40.51	0.01\\
40.52	0.01\\
40.53	0.01\\
40.54	0.01\\
40.55	0.01\\
40.56	0.01\\
40.57	0.01\\
40.58	0.01\\
40.59	0.01\\
40.6	0.01\\
40.61	0.01\\
40.62	0.01\\
40.63	0.01\\
40.64	0.01\\
40.65	0.01\\
40.66	0.01\\
40.67	0.01\\
40.68	0.01\\
40.69	0.01\\
40.7	0.01\\
40.71	0.01\\
40.72	0.01\\
40.73	0.01\\
40.74	0.01\\
40.75	0.01\\
40.76	0.01\\
40.77	0.01\\
40.78	0.01\\
40.79	0.01\\
40.8	0.01\\
40.81	0.01\\
40.82	0.01\\
40.83	0.01\\
40.84	0.01\\
40.85	0.01\\
40.86	0.01\\
40.87	0.01\\
40.88	0.01\\
40.89	0.01\\
40.9	0.01\\
40.91	0.01\\
40.92	0.01\\
40.93	0.01\\
40.94	0.01\\
40.95	0.01\\
40.96	0.01\\
40.97	0.01\\
40.98	0.01\\
40.99	0.01\\
41	0.01\\
41.01	0.01\\
41.02	0.01\\
41.03	0.01\\
41.04	0.01\\
41.05	0.01\\
41.06	0.01\\
41.07	0.01\\
41.08	0.01\\
41.09	0.01\\
41.1	0.01\\
41.11	0.01\\
41.12	0.01\\
41.13	0.01\\
41.14	0.01\\
41.15	0.01\\
41.16	0.01\\
41.17	0.01\\
41.18	0.01\\
41.19	0.01\\
41.2	0.01\\
41.21	0.01\\
41.22	0.01\\
41.23	0.01\\
41.24	0.01\\
41.25	0.01\\
41.26	0.01\\
41.27	0.01\\
41.28	0.01\\
41.29	0.01\\
41.3	0.01\\
41.31	0.01\\
41.32	0.01\\
41.33	0.01\\
41.34	0.01\\
41.35	0.01\\
41.36	0.01\\
41.37	0.01\\
41.38	0.01\\
41.39	0.01\\
41.4	0.01\\
41.41	0.01\\
41.42	0.01\\
41.43	0.01\\
41.44	0.01\\
41.45	0.01\\
41.46	0.01\\
41.47	0.01\\
41.48	0.01\\
41.49	0.01\\
41.5	0.01\\
41.51	0.01\\
41.52	0.01\\
41.53	0.01\\
41.54	0.01\\
41.55	0.01\\
41.56	0.01\\
41.57	0.01\\
41.58	0.01\\
41.59	0.01\\
41.6	0.01\\
41.61	0.01\\
41.62	0.01\\
41.63	0.01\\
41.64	0.01\\
41.65	0.01\\
41.66	0.01\\
41.67	0.01\\
41.68	0.01\\
41.69	0.01\\
41.7	0.01\\
41.71	0.01\\
41.72	0.01\\
41.73	0.01\\
41.74	0.01\\
41.75	0.01\\
41.76	0.01\\
41.77	0.01\\
41.78	0.01\\
41.79	0.01\\
41.8	0.01\\
41.81	0.01\\
41.82	0.01\\
41.83	0.01\\
41.84	0.01\\
41.85	0.01\\
41.86	0.01\\
41.87	0.01\\
41.88	0.01\\
41.89	0.01\\
41.9	0.01\\
41.91	0.01\\
41.92	0.01\\
41.93	0.01\\
41.94	0.01\\
41.95	0.01\\
41.96	0.01\\
41.97	0.01\\
41.98	0.01\\
41.99	0.01\\
42	0.01\\
42.01	0.01\\
42.02	0.01\\
42.03	0.01\\
42.04	0.01\\
42.05	0.01\\
42.06	0.01\\
42.07	0.01\\
42.08	0.01\\
42.09	0.01\\
42.1	0.01\\
42.11	0.01\\
42.12	0.01\\
42.13	0.01\\
42.14	0.01\\
42.15	0.01\\
42.16	0.01\\
42.17	0.01\\
42.18	0.01\\
42.19	0.01\\
42.2	0.01\\
42.21	0.01\\
42.22	0.01\\
42.23	0.01\\
42.24	0.01\\
42.25	0.01\\
42.26	0.01\\
42.27	0.01\\
42.28	0.01\\
42.29	0.01\\
42.3	0.01\\
42.31	0.01\\
42.32	0.01\\
42.33	0.01\\
42.34	0.01\\
42.35	0.01\\
42.36	0.01\\
42.37	0.01\\
42.38	0.01\\
42.39	0.01\\
42.4	0.01\\
42.41	0.01\\
42.42	0.01\\
42.43	0.01\\
42.44	0.01\\
42.45	0.01\\
42.46	0.01\\
42.47	0.01\\
42.48	0.01\\
42.49	0.01\\
42.5	0.01\\
42.51	0.01\\
42.52	0.01\\
42.53	0.01\\
42.54	0.01\\
42.55	0.01\\
42.56	0.01\\
42.57	0.01\\
42.58	0.01\\
42.59	0.01\\
42.6	0.01\\
42.61	0.01\\
42.62	0.01\\
42.63	0.01\\
42.64	0.01\\
42.65	0.01\\
42.66	0.01\\
42.67	0.01\\
42.68	0.01\\
42.69	0.01\\
42.7	0.01\\
42.71	0.01\\
42.72	0.01\\
42.73	0.01\\
42.74	0.01\\
42.75	0.01\\
42.76	0.01\\
42.77	0.01\\
42.78	0.01\\
42.79	0.01\\
42.8	0.01\\
42.81	0.01\\
42.82	0.01\\
42.83	0.01\\
42.84	0.01\\
42.85	0.01\\
42.86	0.01\\
42.87	0.01\\
42.88	0.01\\
42.89	0.01\\
42.9	0.01\\
42.91	0.01\\
42.92	0.01\\
42.93	0.01\\
42.94	0.01\\
42.95	0.01\\
42.96	0.01\\
42.97	0.01\\
42.98	0.01\\
42.99	0.01\\
43	0.01\\
43.01	0.01\\
43.02	0.01\\
43.03	0.01\\
43.04	0.01\\
43.05	0.01\\
43.06	0.01\\
43.07	0.01\\
43.08	0.01\\
43.09	0.01\\
43.1	0.01\\
43.11	0.01\\
43.12	0.01\\
43.13	0.01\\
43.14	0.01\\
43.15	0.01\\
43.16	0.01\\
43.17	0.01\\
43.18	0.01\\
43.19	0.01\\
43.2	0.01\\
43.21	0.01\\
43.22	0.01\\
43.23	0.01\\
43.24	0.01\\
43.25	0.01\\
43.26	0.01\\
43.27	0.01\\
43.28	0.01\\
43.29	0.01\\
43.3	0.01\\
43.31	0.01\\
43.32	0.01\\
43.33	0.01\\
43.34	0.01\\
43.35	0.01\\
43.36	0.01\\
43.37	0.01\\
43.38	0.01\\
43.39	0.01\\
43.4	0.01\\
43.41	0.01\\
43.42	0.01\\
43.43	0.01\\
43.44	0.01\\
43.45	0.01\\
43.46	0.01\\
43.47	0.01\\
43.48	0.01\\
43.49	0.01\\
43.5	0.01\\
43.51	0.01\\
43.52	0.01\\
43.53	0.01\\
43.54	0.01\\
43.55	0.01\\
43.56	0.01\\
43.57	0.01\\
43.58	0.01\\
43.59	0.01\\
43.6	0.01\\
43.61	0.01\\
43.62	0.01\\
43.63	0.01\\
43.64	0.01\\
43.65	0.01\\
43.66	0.01\\
43.67	0.01\\
43.68	0.01\\
43.69	0.01\\
43.7	0.01\\
43.71	0.01\\
43.72	0.01\\
43.73	0.01\\
43.74	0.01\\
43.75	0.01\\
43.76	0.01\\
43.77	0.01\\
43.78	0.01\\
43.79	0.01\\
43.8	0.01\\
43.81	0.01\\
43.82	0.01\\
43.83	0.01\\
43.84	0.01\\
43.85	0.01\\
43.86	0.01\\
43.87	0.01\\
43.88	0.01\\
43.89	0.01\\
43.9	0.01\\
43.91	0.01\\
43.92	0.01\\
43.93	0.01\\
43.94	0.01\\
43.95	0.01\\
43.96	0.01\\
43.97	0.01\\
43.98	0.01\\
43.99	0.01\\
44	0.01\\
44.01	0.01\\
44.02	0.01\\
44.03	0.01\\
44.04	0.01\\
44.05	0.01\\
44.06	0.01\\
44.07	0.01\\
44.08	0.01\\
44.09	0.01\\
44.1	0.01\\
44.11	0.01\\
44.12	0.01\\
44.13	0.01\\
44.14	0.01\\
44.15	0.01\\
44.16	0.01\\
44.17	0.01\\
44.18	0.01\\
44.19	0.01\\
44.2	0.01\\
44.21	0.01\\
44.22	0.01\\
44.23	0.01\\
44.24	0.01\\
44.25	0.01\\
44.26	0.01\\
44.27	0.01\\
44.28	0.01\\
44.29	0.01\\
44.3	0.01\\
44.31	0.01\\
44.32	0.01\\
44.33	0.01\\
44.34	0.01\\
44.35	0.01\\
44.36	0.01\\
44.37	0.01\\
44.38	0.01\\
44.39	0.01\\
44.4	0.01\\
44.41	0.01\\
44.42	0.01\\
44.43	0.01\\
44.44	0.01\\
44.45	0.01\\
44.46	0.01\\
44.47	0.01\\
44.48	0.01\\
44.49	0.01\\
44.5	0.01\\
44.51	0.01\\
44.52	0.01\\
44.53	0.01\\
44.54	0.01\\
44.55	0.01\\
44.56	0.01\\
44.57	0.01\\
44.58	0.01\\
44.59	0.01\\
44.6	0.01\\
44.61	0.01\\
44.62	0.01\\
44.63	0.01\\
44.64	0.01\\
44.65	0.01\\
44.66	0.01\\
44.67	0.01\\
44.68	0.01\\
44.69	0.01\\
44.7	0.01\\
44.71	0.01\\
44.72	0.01\\
44.73	0.01\\
44.74	0.01\\
44.75	0.01\\
44.76	0.01\\
44.77	0.01\\
44.78	0.01\\
44.79	0.01\\
44.8	0.01\\
44.81	0.01\\
44.82	0.01\\
44.83	0.01\\
44.84	0.01\\
44.85	0.01\\
44.86	0.01\\
44.87	0.01\\
44.88	0.01\\
44.89	0.01\\
44.9	0.01\\
44.91	0.01\\
44.92	0.01\\
44.93	0.01\\
44.94	0.01\\
44.95	0.01\\
44.96	0.01\\
44.97	0.01\\
44.98	0.01\\
44.99	0.01\\
45	0.01\\
45.01	0.01\\
45.02	0.01\\
45.03	0.01\\
45.04	0.01\\
45.05	0.01\\
45.06	0.01\\
45.07	0.01\\
45.08	0.01\\
45.09	0.01\\
45.1	0.01\\
45.11	0.01\\
45.12	0.01\\
45.13	0.01\\
45.14	0.01\\
45.15	0.01\\
45.16	0.01\\
45.17	0.01\\
45.18	0.01\\
45.19	0.01\\
45.2	0.01\\
45.21	0.01\\
45.22	0.01\\
45.23	0.01\\
45.24	0.01\\
45.25	0.01\\
45.26	0.01\\
45.27	0.01\\
45.28	0.01\\
45.29	0.01\\
45.3	0.01\\
45.31	0.01\\
45.32	0.01\\
45.33	0.01\\
45.34	0.01\\
45.35	0.01\\
45.36	0.01\\
45.37	0.01\\
45.38	0.01\\
45.39	0.01\\
45.4	0.01\\
45.41	0.01\\
45.42	0.01\\
45.43	0.01\\
45.44	0.01\\
45.45	0.01\\
45.46	0.01\\
45.47	0.01\\
45.48	0.01\\
45.49	0.01\\
45.5	0.01\\
45.51	0.01\\
45.52	0.01\\
45.53	0.01\\
45.54	0.01\\
45.55	0.01\\
45.56	0.01\\
45.57	0.01\\
45.58	0.01\\
45.59	0.01\\
45.6	0.01\\
45.61	0.01\\
45.62	0.01\\
45.63	0.01\\
45.64	0.01\\
45.65	0.01\\
45.66	0.01\\
45.67	0.01\\
45.68	0.01\\
45.69	0.01\\
45.7	0.01\\
45.71	0.01\\
45.72	0.01\\
45.73	0.01\\
45.74	0.01\\
45.75	0.01\\
45.76	0.01\\
45.77	0.01\\
45.78	0.01\\
45.79	0.01\\
45.8	0.01\\
45.81	0.01\\
45.82	0.01\\
45.83	0.01\\
45.84	0.01\\
45.85	0.01\\
45.86	0.01\\
45.87	0.01\\
45.88	0.01\\
45.89	0.01\\
45.9	0.01\\
45.91	0.01\\
45.92	0.01\\
45.93	0.01\\
45.94	0.01\\
45.95	0.01\\
45.96	0.01\\
45.97	0.01\\
45.98	0.01\\
45.99	0.01\\
46	0.01\\
46.01	0.01\\
46.02	0.01\\
46.03	0.01\\
46.04	0.01\\
46.05	0.01\\
46.06	0.01\\
46.07	0.01\\
46.08	0.01\\
46.09	0.01\\
46.1	0.01\\
46.11	0.01\\
46.12	0.01\\
46.13	0.01\\
46.14	0.01\\
46.15	0.01\\
46.16	0.01\\
46.17	0.01\\
46.18	0.01\\
46.19	0.01\\
46.2	0.01\\
46.21	0.01\\
46.22	0.01\\
46.23	0.01\\
46.24	0.01\\
46.25	0.01\\
46.26	0.01\\
46.27	0.01\\
46.28	0.01\\
46.29	0.01\\
46.3	0.01\\
46.31	0.01\\
46.32	0.01\\
46.33	0.01\\
46.34	0.01\\
46.35	0.01\\
46.36	0.01\\
46.37	0.01\\
46.38	0.01\\
46.39	0.01\\
46.4	0.01\\
46.41	0.01\\
46.42	0.01\\
46.43	0.01\\
46.44	0.01\\
46.45	0.01\\
46.46	0.01\\
46.47	0.01\\
46.48	0.01\\
46.49	0.01\\
46.5	0.01\\
46.51	0.01\\
46.52	0.01\\
46.53	0.01\\
46.54	0.01\\
46.55	0.01\\
46.56	0.01\\
46.57	0.01\\
46.58	0.01\\
46.59	0.01\\
46.6	0.01\\
46.61	0.01\\
46.62	0.01\\
46.63	0.01\\
46.64	0.01\\
46.65	0.01\\
46.66	0.01\\
46.67	0.01\\
46.68	0.01\\
46.69	0.01\\
46.7	0.01\\
46.71	0.01\\
46.72	0.01\\
46.73	0.01\\
46.74	0.01\\
46.75	0.01\\
46.76	0.01\\
46.77	0.01\\
46.78	0.01\\
46.79	0.01\\
46.8	0.01\\
46.81	0.01\\
46.82	0.01\\
46.83	0.01\\
46.84	0.01\\
46.85	0.01\\
46.86	0.01\\
46.87	0.01\\
46.88	0.01\\
46.89	0.01\\
46.9	0.01\\
46.91	0.01\\
46.92	0.01\\
46.93	0.01\\
46.94	0.01\\
46.95	0.01\\
46.96	0.01\\
46.97	0.01\\
46.98	0.01\\
46.99	0.01\\
47	0.01\\
47.01	0.01\\
47.02	0.01\\
47.03	0.01\\
47.04	0.01\\
47.05	0.01\\
47.06	0.01\\
47.07	0.01\\
47.08	0.01\\
47.09	0.01\\
47.1	0.01\\
47.11	0.01\\
47.12	0.01\\
47.13	0.01\\
47.14	0.01\\
47.15	0.01\\
47.16	0.01\\
47.17	0.01\\
47.18	0.01\\
47.19	0.01\\
47.2	0.01\\
47.21	0.01\\
47.22	0.01\\
47.23	0.01\\
47.24	0.01\\
47.25	0.01\\
47.26	0.01\\
47.27	0.01\\
47.28	0.01\\
47.29	0.01\\
47.3	0.01\\
47.31	0.01\\
47.32	0.01\\
47.33	0.01\\
47.34	0.01\\
47.35	0.01\\
47.36	0.01\\
47.37	0.01\\
47.38	0.01\\
47.39	0.01\\
47.4	0.01\\
47.41	0.01\\
47.42	0.01\\
47.43	0.01\\
47.44	0.01\\
47.45	0.01\\
47.46	0.01\\
47.47	0.01\\
47.48	0.01\\
47.49	0.01\\
47.5	0.01\\
47.51	0.01\\
47.52	0.01\\
47.53	0.01\\
47.54	0.01\\
47.55	0.01\\
47.56	0.01\\
47.57	0.01\\
47.58	0.01\\
47.59	0.01\\
47.6	0.01\\
47.61	0.01\\
47.62	0.01\\
47.63	0.01\\
47.64	0.01\\
47.65	0.01\\
47.66	0.01\\
47.67	0.01\\
47.68	0.01\\
47.69	0.01\\
47.7	0.01\\
47.71	0.01\\
47.72	0.01\\
47.73	0.01\\
47.74	0.01\\
47.75	0.01\\
47.76	0.01\\
47.77	0.01\\
47.78	0.01\\
47.79	0.01\\
47.8	0.01\\
47.81	0.01\\
47.82	0.01\\
47.83	0.01\\
47.84	0.01\\
47.85	0.01\\
47.86	0.01\\
47.87	0.01\\
47.88	0.01\\
47.89	0.01\\
47.9	0.01\\
47.91	0.01\\
47.92	0.01\\
47.93	0.01\\
47.94	0.01\\
47.95	0.01\\
47.96	0.01\\
47.97	0.01\\
47.98	0.01\\
47.99	0.01\\
48	0.01\\
48.01	0.01\\
48.02	0.01\\
48.03	0.01\\
48.04	0.01\\
48.05	0.01\\
48.06	0.01\\
48.07	0.01\\
48.08	0.01\\
48.09	0.01\\
48.1	0.01\\
48.11	0.01\\
48.12	0.01\\
48.13	0.01\\
48.14	0.01\\
48.15	0.01\\
48.16	0.01\\
48.17	0.01\\
48.18	0.01\\
48.19	0.01\\
48.2	0.01\\
48.21	0.01\\
48.22	0.01\\
48.23	0.01\\
48.24	0.01\\
48.25	0.01\\
48.26	0.01\\
48.27	0.01\\
48.28	0.01\\
48.29	0.01\\
48.3	0.01\\
48.31	0.01\\
48.32	0.01\\
48.33	0.01\\
48.34	0.01\\
48.35	0.01\\
48.36	0.01\\
48.37	0.01\\
48.38	0.01\\
48.39	0.01\\
48.4	0.01\\
48.41	0.01\\
48.42	0.01\\
48.43	0.01\\
48.44	0.01\\
48.45	0.01\\
48.46	0.01\\
48.47	0.01\\
48.48	0.01\\
48.49	0.01\\
48.5	0.01\\
48.51	0.01\\
48.52	0.01\\
48.53	0.01\\
48.54	0.01\\
48.55	0.01\\
48.56	0.01\\
48.57	0.01\\
48.58	0.01\\
48.59	0.01\\
48.6	0.01\\
48.61	0.01\\
48.62	0.01\\
48.63	0.01\\
48.64	0.01\\
48.65	0.01\\
48.66	0.01\\
48.67	0.01\\
48.68	0.01\\
48.69	0.01\\
48.7	0.01\\
48.71	0.01\\
48.72	0.01\\
48.73	0.01\\
48.74	0.01\\
48.75	0.01\\
48.76	0.01\\
48.77	0.01\\
48.78	0.01\\
48.79	0.01\\
48.8	0.01\\
48.81	0.01\\
48.82	0.01\\
48.83	0.01\\
48.84	0.01\\
48.85	0.01\\
48.86	0.01\\
48.87	0.01\\
48.88	0.01\\
48.89	0.01\\
48.9	0.01\\
48.91	0.01\\
48.92	0.01\\
48.93	0.01\\
48.94	0.01\\
48.95	0.01\\
48.96	0.01\\
48.97	0.01\\
48.98	0.01\\
48.99	0.01\\
49	0.01\\
49.01	0.01\\
49.02	0.01\\
49.03	0.01\\
49.04	0.01\\
49.05	0.01\\
49.06	0.01\\
49.07	0.01\\
49.08	0.01\\
49.09	0.01\\
49.1	0.01\\
49.11	0.01\\
49.12	0.01\\
49.13	0.01\\
49.14	0.01\\
49.15	0.01\\
49.16	0.01\\
49.17	0.01\\
49.18	0.01\\
49.19	0.01\\
49.2	0.01\\
49.21	0.01\\
49.22	0.01\\
49.23	0.01\\
49.24	0.01\\
49.25	0.01\\
49.26	0.01\\
49.27	0.01\\
49.28	0.01\\
49.29	0.01\\
49.3	0.01\\
49.31	0.01\\
49.32	0.01\\
49.33	0.01\\
49.34	0.01\\
49.35	0.01\\
49.36	0.01\\
49.37	0.01\\
49.38	0.01\\
49.39	0.01\\
49.4	0.01\\
49.41	0.01\\
49.42	0.01\\
49.43	0.01\\
49.44	0.01\\
49.45	0.01\\
49.46	0.01\\
49.47	0.01\\
49.48	0.01\\
49.49	0.01\\
49.5	0.01\\
49.51	0.01\\
49.52	0.01\\
49.53	0.01\\
49.54	0.01\\
49.55	0.01\\
49.56	0.01\\
49.57	0.01\\
49.58	0.01\\
49.59	0.01\\
49.6	0.01\\
49.61	0.01\\
49.62	0.01\\
49.63	0.01\\
49.64	0.01\\
49.65	0.01\\
49.66	0.01\\
49.67	0.01\\
49.68	0.01\\
49.69	0.01\\
49.7	0.01\\
49.71	0.01\\
49.72	0.01\\
49.73	0.01\\
49.74	0.01\\
49.75	0.01\\
49.76	0.01\\
49.77	0.01\\
49.78	0.01\\
49.79	0.01\\
49.8	0.01\\
49.81	0.01\\
49.82	0.01\\
49.83	0.01\\
49.84	0.01\\
49.85	0.01\\
49.86	0.01\\
49.87	0.01\\
49.88	0.01\\
49.89	0.01\\
49.9	0.01\\
49.91	0.01\\
49.92	0.01\\
49.93	0.01\\
49.94	0.01\\
49.95	0.01\\
49.96	0.01\\
49.97	0.01\\
49.98	0.01\\
49.99	0.01\\
50	0.01\\
50.01	0.01\\
50.02	0.01\\
50.03	0.01\\
50.04	0.01\\
50.05	0.01\\
50.06	0.01\\
50.07	0.01\\
50.08	0.01\\
50.09	0.01\\
50.1	0.01\\
50.11	0.01\\
50.12	0.01\\
50.13	0.01\\
50.14	0.01\\
50.15	0.01\\
50.16	0.01\\
50.17	0.01\\
50.18	0.01\\
50.19	0.01\\
50.2	0.01\\
50.21	0.01\\
50.22	0.01\\
50.23	0.01\\
50.24	0.01\\
50.25	0.01\\
50.26	0.01\\
50.27	0.01\\
50.28	0.01\\
50.29	0.01\\
50.3	0.01\\
50.31	0.01\\
50.32	0.01\\
50.33	0.01\\
50.34	0.01\\
50.35	0.01\\
50.36	0.01\\
50.37	0.01\\
50.38	0.01\\
50.39	0.01\\
50.4	0.01\\
50.41	0.01\\
50.42	0.01\\
50.43	0.01\\
50.44	0.01\\
50.45	0.01\\
50.46	0.01\\
50.47	0.01\\
50.48	0.01\\
50.49	0.01\\
50.5	0.01\\
50.51	0.01\\
50.52	0.01\\
50.53	0.01\\
50.54	0.01\\
50.55	0.01\\
50.56	0.01\\
50.57	0.01\\
50.58	0.01\\
50.59	0.01\\
50.6	0.01\\
50.61	0.01\\
50.62	0.01\\
50.63	0.01\\
50.64	0.01\\
50.65	0.01\\
50.66	0.01\\
50.67	0.01\\
50.68	0.01\\
50.69	0.01\\
50.7	0.01\\
50.71	0.01\\
50.72	0.01\\
50.73	0.01\\
50.74	0.01\\
50.75	0.01\\
50.76	0.01\\
50.77	0.01\\
50.78	0.01\\
50.79	0.01\\
50.8	0.01\\
50.81	0.01\\
50.82	0.01\\
50.83	0.01\\
50.84	0.01\\
50.85	0.01\\
50.86	0.01\\
50.87	0.01\\
50.88	0.01\\
50.89	0.01\\
50.9	0.01\\
50.91	0.01\\
50.92	0.01\\
50.93	0.01\\
50.94	0.01\\
50.95	0.01\\
50.96	0.01\\
50.97	0.01\\
50.98	0.01\\
50.99	0.01\\
51	0.01\\
51.01	0.01\\
51.02	0.01\\
51.03	0.01\\
51.04	0.01\\
51.05	0.01\\
51.06	0.01\\
51.07	0.01\\
51.08	0.01\\
51.09	0.01\\
51.1	0.01\\
51.11	0.01\\
51.12	0.01\\
51.13	0.01\\
51.14	0.01\\
51.15	0.01\\
51.16	0.01\\
51.17	0.01\\
51.18	0.01\\
51.19	0.01\\
51.2	0.01\\
51.21	0.01\\
51.22	0.01\\
51.23	0.01\\
51.24	0.01\\
51.25	0.01\\
51.26	0.01\\
51.27	0.01\\
51.28	0.01\\
51.29	0.01\\
51.3	0.01\\
51.31	0.01\\
51.32	0.01\\
51.33	0.01\\
51.34	0.01\\
51.35	0.01\\
51.36	0.01\\
51.37	0.01\\
51.38	0.01\\
51.39	0.01\\
51.4	0.01\\
51.41	0.01\\
51.42	0.01\\
51.43	0.01\\
51.44	0.01\\
51.45	0.01\\
51.46	0.01\\
51.47	0.01\\
51.48	0.01\\
51.49	0.01\\
51.5	0.01\\
51.51	0.01\\
51.52	0.01\\
51.53	0.01\\
51.54	0.01\\
51.55	0.01\\
51.56	0.01\\
51.57	0.01\\
51.58	0.01\\
51.59	0.01\\
51.6	0.01\\
51.61	0.01\\
51.62	0.01\\
51.63	0.01\\
51.64	0.01\\
51.65	0.01\\
51.66	0.01\\
51.67	0.01\\
51.68	0.01\\
51.69	0.01\\
51.7	0.01\\
51.71	0.01\\
51.72	0.01\\
51.73	0.01\\
51.74	0.01\\
51.75	0.01\\
51.76	0.01\\
51.77	0.01\\
51.78	0.01\\
51.79	0.01\\
51.8	0.01\\
51.81	0.01\\
51.82	0.01\\
51.83	0.01\\
51.84	0.01\\
51.85	0.01\\
51.86	0.01\\
51.87	0.01\\
51.88	0.01\\
51.89	0.01\\
51.9	0.01\\
51.91	0.01\\
51.92	0.01\\
51.93	0.01\\
51.94	0.01\\
51.95	0.01\\
51.96	0.01\\
51.97	0.01\\
51.98	0.01\\
51.99	0.01\\
52	0.01\\
52.01	0.01\\
52.02	0.01\\
52.03	0.01\\
52.04	0.01\\
52.05	0.01\\
52.06	0.01\\
52.07	0.01\\
52.08	0.01\\
52.09	0.01\\
52.1	0.01\\
52.11	0.01\\
52.12	0.01\\
52.13	0.01\\
52.14	0.01\\
52.15	0.01\\
52.16	0.01\\
52.17	0.01\\
52.18	0.01\\
52.19	0.01\\
52.2	0.01\\
52.21	0.01\\
52.22	0.01\\
52.23	0.01\\
52.24	0.01\\
52.25	0.01\\
52.26	0.01\\
52.27	0.01\\
52.28	0.01\\
52.29	0.01\\
52.3	0.01\\
52.31	0.01\\
52.32	0.01\\
52.33	0.01\\
52.34	0.01\\
52.35	0.01\\
52.36	0.01\\
52.37	0.01\\
52.38	0.01\\
52.39	0.01\\
52.4	0.01\\
52.41	0.01\\
52.42	0.01\\
52.43	0.01\\
52.44	0.01\\
52.45	0.01\\
52.46	0.01\\
52.47	0.01\\
52.48	0.01\\
52.49	0.01\\
52.5	0.01\\
52.51	0.01\\
52.52	0.01\\
52.53	0.01\\
52.54	0.01\\
52.55	0.01\\
52.56	0.01\\
52.57	0.01\\
52.58	0.01\\
52.59	0.01\\
52.6	0.01\\
52.61	0.01\\
52.62	0.01\\
52.63	0.01\\
52.64	0.01\\
52.65	0.01\\
52.66	0.01\\
52.67	0.01\\
52.68	0.01\\
52.69	0.01\\
52.7	0.01\\
52.71	0.01\\
52.72	0.01\\
52.73	0.01\\
52.74	0.01\\
52.75	0.01\\
52.76	0.01\\
52.77	0.01\\
52.78	0.01\\
52.79	0.01\\
52.8	0.01\\
52.81	0.01\\
52.82	0.01\\
52.83	0.01\\
52.84	0.01\\
52.85	0.01\\
52.86	0.01\\
52.87	0.01\\
52.88	0.01\\
52.89	0.01\\
52.9	0.01\\
52.91	0.01\\
52.92	0.01\\
52.93	0.01\\
52.94	0.01\\
52.95	0.01\\
52.96	0.01\\
52.97	0.01\\
52.98	0.01\\
52.99	0.01\\
53	0.01\\
53.01	0.01\\
53.02	0.01\\
53.03	0.01\\
53.04	0.01\\
53.05	0.01\\
53.06	0.01\\
53.07	0.01\\
53.08	0.01\\
53.09	0.01\\
53.1	0.01\\
53.11	0.01\\
53.12	0.01\\
53.13	0.01\\
53.14	0.01\\
53.15	0.01\\
53.16	0.01\\
53.17	0.01\\
53.18	0.01\\
53.19	0.01\\
53.2	0.01\\
53.21	0.01\\
53.22	0.01\\
53.23	0.01\\
53.24	0.01\\
53.25	0.01\\
53.26	0.01\\
53.27	0.01\\
53.28	0.01\\
53.29	0.01\\
53.3	0.01\\
53.31	0.01\\
53.32	0.01\\
53.33	0.01\\
53.34	0.01\\
53.35	0.01\\
53.36	0.01\\
53.37	0.01\\
53.38	0.01\\
53.39	0.01\\
53.4	0.01\\
53.41	0.01\\
53.42	0.01\\
53.43	0.01\\
53.44	0.01\\
53.45	0.01\\
53.46	0.01\\
53.47	0.01\\
53.48	0.01\\
53.49	0.01\\
53.5	0.01\\
53.51	0.01\\
53.52	0.01\\
53.53	0.01\\
53.54	0.01\\
53.55	0.01\\
53.56	0.01\\
53.57	0.01\\
53.58	0.01\\
53.59	0.01\\
53.6	0.01\\
53.61	0.01\\
53.62	0.01\\
53.63	0.01\\
53.64	0.01\\
53.65	0.01\\
53.66	0.01\\
53.67	0.01\\
53.68	0.01\\
53.69	0.01\\
53.7	0.01\\
53.71	0.01\\
53.72	0.01\\
53.73	0.01\\
53.74	0.01\\
53.75	0.01\\
53.76	0.01\\
53.77	0.01\\
53.78	0.01\\
53.79	0.01\\
53.8	0.01\\
53.81	0.01\\
53.82	0.01\\
53.83	0.01\\
53.84	0.01\\
53.85	0.01\\
53.86	0.01\\
53.87	0.01\\
53.88	0.01\\
53.89	0.01\\
53.9	0.01\\
53.91	0.01\\
53.92	0.01\\
53.93	0.01\\
53.94	0.01\\
53.95	0.01\\
53.96	0.01\\
53.97	0.01\\
53.98	0.01\\
53.99	0.01\\
54	0.01\\
54.01	0.01\\
54.02	0.01\\
54.03	0.01\\
54.04	0.01\\
54.05	0.01\\
54.06	0.01\\
54.07	0.01\\
54.08	0.01\\
54.09	0.01\\
54.1	0.01\\
54.11	0.01\\
54.12	0.01\\
54.13	0.01\\
54.14	0.01\\
54.15	0.01\\
54.16	0.01\\
54.17	0.01\\
54.18	0.01\\
54.19	0.01\\
54.2	0.01\\
54.21	0.01\\
54.22	0.01\\
54.23	0.01\\
54.24	0.01\\
54.25	0.01\\
54.26	0.01\\
54.27	0.01\\
54.28	0.01\\
54.29	0.01\\
54.3	0.01\\
54.31	0.01\\
54.32	0.01\\
54.33	0.01\\
54.34	0.01\\
54.35	0.01\\
54.36	0.01\\
54.37	0.01\\
54.38	0.01\\
54.39	0.01\\
54.4	0.01\\
54.41	0.01\\
54.42	0.01\\
54.43	0.01\\
54.44	0.01\\
54.45	0.01\\
54.46	0.01\\
54.47	0.01\\
54.48	0.01\\
54.49	0.01\\
54.5	0.01\\
54.51	0.01\\
54.52	0.01\\
54.53	0.01\\
54.54	0.01\\
54.55	0.01\\
54.56	0.01\\
54.57	0.01\\
54.58	0.01\\
54.59	0.01\\
54.6	0.01\\
54.61	0.01\\
54.62	0.01\\
54.63	0.01\\
54.64	0.01\\
54.65	0.01\\
54.66	0.01\\
54.67	0.01\\
54.68	0.01\\
54.69	0.01\\
54.7	0.01\\
54.71	0.01\\
54.72	0.01\\
54.73	0.01\\
54.74	0.01\\
54.75	0.01\\
54.76	0.01\\
54.77	0.01\\
54.78	0.01\\
54.79	0.01\\
54.8	0.01\\
54.81	0.01\\
54.82	0.01\\
54.83	0.01\\
54.84	0.01\\
54.85	0.01\\
54.86	0.01\\
54.87	0.01\\
54.88	0.01\\
54.89	0.01\\
54.9	0.01\\
54.91	0.01\\
54.92	0.01\\
54.93	0.01\\
54.94	0.01\\
54.95	0.01\\
54.96	0.01\\
54.97	0.01\\
54.98	0.01\\
54.99	0.01\\
55	0.01\\
55.01	0.01\\
55.02	0.01\\
55.03	0.01\\
55.04	0.01\\
55.05	0.01\\
55.06	0.01\\
55.07	0.01\\
55.08	0.01\\
55.09	0.01\\
55.1	0.01\\
55.11	0.01\\
55.12	0.01\\
55.13	0.01\\
55.14	0.01\\
55.15	0.01\\
55.16	0.01\\
55.17	0.01\\
55.18	0.01\\
55.19	0.01\\
55.2	0.01\\
55.21	0.01\\
55.22	0.01\\
55.23	0.01\\
55.24	0.01\\
55.25	0.01\\
55.26	0.01\\
55.27	0.01\\
55.28	0.01\\
55.29	0.01\\
55.3	0.01\\
55.31	0.01\\
55.32	0.01\\
55.33	0.01\\
55.34	0.01\\
55.35	0.01\\
55.36	0.01\\
55.37	0.01\\
55.38	0.01\\
55.39	0.01\\
55.4	0.01\\
55.41	0.01\\
55.42	0.01\\
55.43	0.01\\
55.44	0.01\\
55.45	0.01\\
55.46	0.01\\
55.47	0.01\\
55.48	0.01\\
55.49	0.01\\
55.5	0.01\\
55.51	0.01\\
55.52	0.01\\
55.53	0.01\\
55.54	0.01\\
55.55	0.01\\
55.56	0.01\\
55.57	0.01\\
55.58	0.01\\
55.59	0.01\\
55.6	0.01\\
55.61	0.01\\
55.62	0.01\\
55.63	0.01\\
55.64	0.01\\
55.65	0.01\\
55.66	0.01\\
55.67	0.01\\
55.68	0.01\\
55.69	0.01\\
55.7	0.01\\
55.71	0.01\\
55.72	0.01\\
55.73	0.01\\
55.74	0.01\\
55.75	0.01\\
55.76	0.01\\
55.77	0.01\\
55.78	0.01\\
55.79	0.01\\
55.8	0.01\\
55.81	0.01\\
55.82	0.01\\
55.83	0.01\\
55.84	0.01\\
55.85	0.01\\
55.86	0.01\\
55.87	0.01\\
55.88	0.01\\
55.89	0.01\\
55.9	0.01\\
55.91	0.01\\
55.92	0.01\\
55.93	0.01\\
55.94	0.01\\
55.95	0.01\\
55.96	0.01\\
55.97	0.01\\
55.98	0.01\\
55.99	0.01\\
56	0.01\\
56.01	0.01\\
56.02	0.01\\
56.03	0.01\\
56.04	0.01\\
56.05	0.01\\
56.06	0.01\\
56.07	0.01\\
56.08	0.01\\
56.09	0.01\\
56.1	0.01\\
56.11	0.01\\
56.12	0.01\\
56.13	0.01\\
56.14	0.01\\
56.15	0.01\\
56.16	0.01\\
56.17	0.01\\
56.18	0.01\\
56.19	0.01\\
56.2	0.01\\
56.21	0.01\\
56.22	0.01\\
56.23	0.01\\
56.24	0.01\\
56.25	0.01\\
56.26	0.01\\
56.27	0.01\\
56.28	0.01\\
56.29	0.01\\
56.3	0.01\\
56.31	0.01\\
56.32	0.01\\
56.33	0.01\\
56.34	0.01\\
56.35	0.01\\
56.36	0.01\\
56.37	0.01\\
56.38	0.01\\
56.39	0.01\\
56.4	0.01\\
56.41	0.01\\
56.42	0.01\\
56.43	0.01\\
56.44	0.01\\
56.45	0.01\\
56.46	0.01\\
56.47	0.01\\
56.48	0.01\\
56.49	0.01\\
56.5	0.01\\
56.51	0.01\\
56.52	0.01\\
56.53	0.01\\
56.54	0.01\\
56.55	0.01\\
56.56	0.01\\
56.57	0.01\\
56.58	0.01\\
56.59	0.01\\
56.6	0.01\\
56.61	0.01\\
56.62	0.01\\
56.63	0.01\\
56.64	0.01\\
56.65	0.01\\
56.66	0.01\\
56.67	0.01\\
56.68	0.01\\
56.69	0.01\\
56.7	0.01\\
56.71	0.01\\
56.72	0.01\\
56.73	0.01\\
56.74	0.01\\
56.75	0.01\\
56.76	0.01\\
56.77	0.01\\
56.78	0.01\\
56.79	0.01\\
56.8	0.01\\
56.81	0.01\\
56.82	0.01\\
56.83	0.01\\
56.84	0.01\\
56.85	0.01\\
56.86	0.01\\
56.87	0.01\\
56.88	0.01\\
56.89	0.01\\
56.9	0.01\\
56.91	0.01\\
56.92	0.01\\
56.93	0.01\\
56.94	0.01\\
56.95	0.01\\
56.96	0.01\\
56.97	0.01\\
56.98	0.01\\
56.99	0.01\\
57	0.01\\
57.01	0.01\\
57.02	0.01\\
57.03	0.01\\
57.04	0.01\\
57.05	0.01\\
57.06	0.01\\
57.07	0.01\\
57.08	0.01\\
57.09	0.01\\
57.1	0.01\\
57.11	0.01\\
57.12	0.01\\
57.13	0.01\\
57.14	0.01\\
57.15	0.01\\
57.16	0.01\\
57.17	0.01\\
57.18	0.01\\
57.19	0.01\\
57.2	0.01\\
57.21	0.01\\
57.22	0.01\\
57.23	0.01\\
57.24	0.01\\
57.25	0.01\\
57.26	0.01\\
57.27	0.01\\
57.28	0.01\\
57.29	0.01\\
57.3	0.01\\
57.31	0.01\\
57.32	0.01\\
57.33	0.01\\
57.34	0.01\\
57.35	0.01\\
57.36	0.01\\
57.37	0.01\\
57.38	0.01\\
57.39	0.01\\
57.4	0.01\\
57.41	0.01\\
57.42	0.01\\
57.43	0.01\\
57.44	0.01\\
57.45	0.01\\
57.46	0.01\\
57.47	0.01\\
57.48	0.01\\
57.49	0.01\\
57.5	0.01\\
57.51	0.01\\
57.52	0.01\\
57.53	0.01\\
57.54	0.01\\
57.55	0.01\\
57.56	0.01\\
57.57	0.01\\
57.58	0.01\\
57.59	0.01\\
57.6	0.01\\
57.61	0.01\\
57.62	0.01\\
57.63	0.01\\
57.64	0.01\\
57.65	0.01\\
57.66	0.01\\
57.67	0.01\\
57.68	0.01\\
57.69	0.01\\
57.7	0.01\\
57.71	0.01\\
57.72	0.01\\
57.73	0.01\\
57.74	0.01\\
57.75	0.01\\
57.76	0.01\\
57.77	0.01\\
57.78	0.01\\
57.79	0.01\\
57.8	0.01\\
57.81	0.01\\
57.82	0.01\\
57.83	0.01\\
57.84	0.01\\
57.85	0.01\\
57.86	0.01\\
57.87	0.01\\
57.88	0.01\\
57.89	0.01\\
57.9	0.01\\
57.91	0.01\\
57.92	0.01\\
57.93	0.01\\
57.94	0.01\\
57.95	0.01\\
57.96	0.01\\
57.97	0.01\\
57.98	0.01\\
57.99	0.01\\
58	0.01\\
58.01	0.01\\
58.02	0.01\\
58.03	0.01\\
58.04	0.01\\
58.05	0.01\\
58.06	0.01\\
58.07	0.01\\
58.08	0.01\\
58.09	0.01\\
58.1	0.01\\
58.11	0.01\\
58.12	0.01\\
58.13	0.01\\
58.14	0.01\\
58.15	0.01\\
58.16	0.01\\
58.17	0.01\\
58.18	0.01\\
58.19	0.01\\
58.2	0.01\\
58.21	0.01\\
58.22	0.01\\
58.23	0.01\\
58.24	0.01\\
58.25	0.01\\
58.26	0.01\\
58.27	0.01\\
58.28	0.01\\
58.29	0.01\\
58.3	0.01\\
58.31	0.01\\
58.32	0.01\\
58.33	0.01\\
58.34	0.01\\
58.35	0.01\\
58.36	0.01\\
58.37	0.01\\
58.38	0.01\\
58.39	0.01\\
58.4	0.01\\
58.41	0.01\\
58.42	0.01\\
58.43	0.01\\
58.44	0.01\\
58.45	0.01\\
58.46	0.01\\
58.47	0.01\\
58.48	0.01\\
58.49	0.01\\
58.5	0.01\\
58.51	0.01\\
58.52	0.01\\
58.53	0.01\\
58.54	0.01\\
58.55	0.01\\
58.56	0.01\\
58.57	0.01\\
58.58	0.01\\
58.59	0.01\\
58.6	0.01\\
58.61	0.01\\
58.62	0.01\\
58.63	0.01\\
58.64	0.01\\
58.65	0.01\\
58.66	0.01\\
58.67	0.01\\
58.68	0.01\\
58.69	0.01\\
58.7	0.01\\
58.71	0.01\\
58.72	0.01\\
58.73	0.01\\
58.74	0.01\\
58.75	0.01\\
58.76	0.01\\
58.77	0.01\\
58.78	0.01\\
58.79	0.01\\
58.8	0.01\\
58.81	0.01\\
58.82	0.01\\
58.83	0.01\\
58.84	0.01\\
58.85	0.01\\
58.86	0.01\\
58.87	0.01\\
58.88	0.01\\
58.89	0.01\\
58.9	0.01\\
58.91	0.01\\
58.92	0.01\\
58.93	0.01\\
58.94	0.01\\
58.95	0.01\\
58.96	0.01\\
58.97	0.01\\
58.98	0.01\\
58.99	0.01\\
59	0.01\\
59.01	0.01\\
59.02	0.01\\
59.03	0.01\\
59.04	0.01\\
59.05	0.01\\
59.06	0.01\\
59.07	0.01\\
59.08	0.01\\
59.09	0.01\\
59.1	0.01\\
59.11	0.01\\
59.12	0.01\\
59.13	0.01\\
59.14	0.01\\
59.15	0.01\\
59.16	0.01\\
59.17	0.01\\
59.18	0.01\\
59.19	0.01\\
59.2	0.01\\
59.21	0.01\\
59.22	0.01\\
59.23	0.01\\
59.24	0.01\\
59.25	0.01\\
59.26	0.01\\
59.27	0.01\\
59.28	0.01\\
59.29	0.01\\
59.3	0.01\\
59.31	0.01\\
59.32	0.01\\
59.33	0.01\\
59.34	0.01\\
59.35	0.01\\
59.36	0.01\\
59.37	0.01\\
59.38	0.01\\
59.39	0.01\\
59.4	0.01\\
59.41	0.01\\
59.42	0.01\\
59.43	0.01\\
59.44	0.01\\
59.45	0.01\\
59.46	0.01\\
59.47	0.01\\
59.48	0.01\\
59.49	0.01\\
59.5	0.01\\
59.51	0.01\\
59.52	0.01\\
59.53	0.01\\
59.54	0.01\\
59.55	0.01\\
59.56	0.01\\
59.57	0.01\\
59.58	0.01\\
59.59	0.01\\
59.6	0.01\\
59.61	0.01\\
59.62	0.01\\
59.63	0.01\\
59.64	0.01\\
59.65	0.01\\
59.66	0.01\\
59.67	0.01\\
59.68	0.01\\
59.69	0.01\\
59.7	0.01\\
59.71	0.01\\
59.72	0.01\\
59.73	0.01\\
59.74	0.01\\
59.75	0.01\\
59.76	0.01\\
59.77	0.01\\
59.78	0.01\\
59.79	0.01\\
59.8	0.01\\
59.81	0.01\\
59.82	0.01\\
59.83	0.01\\
59.84	0.01\\
59.85	0.01\\
59.86	0.01\\
59.87	0.01\\
59.88	0.01\\
59.89	0.01\\
59.9	0.01\\
59.91	0.01\\
59.92	0.01\\
59.93	0.01\\
59.94	0.01\\
59.95	0.01\\
59.96	0.01\\
59.97	0.01\\
59.98	0.01\\
59.99	0.01\\
60	0.01\\
60.01	0.01\\
60.02	0.01\\
60.03	0.01\\
60.04	0.01\\
60.05	0.01\\
60.06	0.01\\
60.07	0.01\\
60.08	0.01\\
60.09	0.01\\
60.1	0.01\\
60.11	0.01\\
60.12	0.01\\
60.13	0.01\\
60.14	0.01\\
60.15	0.01\\
60.16	0.01\\
60.17	0.01\\
60.18	0.01\\
60.19	0.01\\
60.2	0.01\\
60.21	0.01\\
60.22	0.01\\
60.23	0.01\\
60.24	0.01\\
60.25	0.01\\
60.26	0.01\\
60.27	0.01\\
60.28	0.01\\
60.29	0.01\\
60.3	0.01\\
60.31	0.01\\
60.32	0.01\\
60.33	0.01\\
60.34	0.01\\
60.35	0.01\\
60.36	0.01\\
60.37	0.01\\
60.38	0.01\\
60.39	0.01\\
60.4	0.01\\
60.41	0.01\\
60.42	0.01\\
60.43	0.01\\
60.44	0.01\\
60.45	0.01\\
60.46	0.01\\
60.47	0.01\\
60.48	0.01\\
60.49	0.01\\
60.5	0.01\\
60.51	0.01\\
60.52	0.01\\
60.53	0.01\\
60.54	0.01\\
60.55	0.01\\
60.56	0.01\\
60.57	0.01\\
60.58	0.01\\
60.59	0.01\\
60.6	0.01\\
60.61	0.01\\
60.62	0.01\\
60.63	0.01\\
60.64	0.01\\
60.65	0.01\\
60.66	0.01\\
60.67	0.01\\
60.68	0.01\\
60.69	0.01\\
60.7	0.01\\
60.71	0.01\\
60.72	0.01\\
60.73	0.01\\
60.74	0.01\\
60.75	0.01\\
60.76	0.01\\
60.77	0.01\\
60.78	0.01\\
60.79	0.01\\
60.8	0.01\\
60.81	0.01\\
60.82	0.01\\
60.83	0.01\\
60.84	0.01\\
60.85	0.01\\
60.86	0.01\\
60.87	0.01\\
60.88	0.01\\
60.89	0.01\\
60.9	0.01\\
60.91	0.01\\
60.92	0.01\\
60.93	0.01\\
60.94	0.01\\
60.95	0.01\\
60.96	0.01\\
60.97	0.01\\
60.98	0.01\\
60.99	0.01\\
61	0.01\\
61.01	0.01\\
61.02	0.01\\
61.03	0.01\\
61.04	0.01\\
61.05	0.01\\
61.06	0.01\\
61.07	0.01\\
61.08	0.01\\
61.09	0.01\\
61.1	0.01\\
61.11	0.01\\
61.12	0.01\\
61.13	0.01\\
61.14	0.01\\
61.15	0.01\\
61.16	0.01\\
61.17	0.01\\
61.18	0.01\\
61.19	0.01\\
61.2	0.01\\
61.21	0.01\\
61.22	0.01\\
61.23	0.01\\
61.24	0.01\\
61.25	0.01\\
61.26	0.01\\
61.27	0.01\\
61.28	0.01\\
61.29	0.01\\
61.3	0.01\\
61.31	0.01\\
61.32	0.01\\
61.33	0.01\\
61.34	0.01\\
61.35	0.01\\
61.36	0.01\\
61.37	0.01\\
61.38	0.01\\
61.39	0.01\\
61.4	0.01\\
61.41	0.01\\
61.42	0.01\\
61.43	0.01\\
61.44	0.01\\
61.45	0.01\\
61.46	0.01\\
61.47	0.01\\
61.48	0.01\\
61.49	0.01\\
61.5	0.01\\
61.51	0.01\\
61.52	0.01\\
61.53	0.01\\
61.54	0.01\\
61.55	0.01\\
61.56	0.01\\
61.57	0.01\\
61.58	0.01\\
61.59	0.01\\
61.6	0.01\\
61.61	0.01\\
61.62	0.01\\
61.63	0.01\\
61.64	0.01\\
61.65	0.01\\
61.66	0.01\\
61.67	0.01\\
61.68	0.01\\
61.69	0.01\\
61.7	0.01\\
61.71	0.01\\
61.72	0.01\\
61.73	0.01\\
61.74	0.01\\
61.75	0.01\\
61.76	0.01\\
61.77	0.01\\
61.78	0.01\\
61.79	0.01\\
61.8	0.01\\
61.81	0.01\\
61.82	0.01\\
61.83	0.01\\
61.84	0.01\\
61.85	0.01\\
61.86	0.01\\
61.87	0.01\\
61.88	0.01\\
61.89	0.01\\
61.9	0.01\\
61.91	0.01\\
61.92	0.01\\
61.93	0.01\\
61.94	0.01\\
61.95	0.01\\
61.96	0.01\\
61.97	0.01\\
61.98	0.01\\
61.99	0.01\\
62	0.01\\
62.01	0.01\\
62.02	0.01\\
62.03	0.01\\
62.04	0.01\\
62.05	0.01\\
62.06	0.01\\
62.07	0.01\\
62.08	0.01\\
62.09	0.01\\
62.1	0.01\\
62.11	0.01\\
62.12	0.01\\
62.13	0.01\\
62.14	0.01\\
62.15	0.01\\
62.16	0.01\\
62.17	0.01\\
62.18	0.01\\
62.19	0.01\\
62.2	0.01\\
62.21	0.01\\
62.22	0.01\\
62.23	0.01\\
62.24	0.01\\
62.25	0.01\\
62.26	0.01\\
62.27	0.01\\
62.28	0.01\\
62.29	0.01\\
62.3	0.01\\
62.31	0.01\\
62.32	0.01\\
62.33	0.01\\
62.34	0.01\\
62.35	0.01\\
62.36	0.01\\
62.37	0.01\\
62.38	0.01\\
62.39	0.01\\
62.4	0.01\\
62.41	0.01\\
62.42	0.01\\
62.43	0.01\\
62.44	0.01\\
62.45	0.01\\
62.46	0.01\\
62.47	0.01\\
62.48	0.01\\
62.49	0.01\\
62.5	0.01\\
62.51	0.01\\
62.52	0.01\\
62.53	0.01\\
62.54	0.01\\
62.55	0.01\\
62.56	0.01\\
62.57	0.01\\
62.58	0.01\\
62.59	0.01\\
62.6	0.01\\
62.61	0.01\\
62.62	0.01\\
62.63	0.01\\
62.64	0.01\\
62.65	0.01\\
62.66	0.01\\
62.67	0.01\\
62.68	0.01\\
62.69	0.01\\
62.7	0.01\\
62.71	0.01\\
62.72	0.01\\
62.73	0.01\\
62.74	0.01\\
62.75	0.01\\
62.76	0.01\\
62.77	0.01\\
62.78	0.01\\
62.79	0.01\\
62.8	0.01\\
62.81	0.01\\
62.82	0.01\\
62.83	0.01\\
62.84	0.01\\
62.85	0.01\\
62.86	0.01\\
62.87	0.01\\
62.88	0.01\\
62.89	0.01\\
62.9	0.01\\
62.91	0.01\\
62.92	0.01\\
62.93	0.01\\
62.94	0.01\\
62.95	0.01\\
62.96	0.01\\
62.97	0.01\\
62.98	0.01\\
62.99	0.01\\
63	0.01\\
63.01	0.01\\
63.02	0.01\\
63.03	0.01\\
63.04	0.01\\
63.05	0.01\\
63.06	0.01\\
63.07	0.01\\
63.08	0.01\\
63.09	0.01\\
63.1	0.01\\
63.11	0.01\\
63.12	0.01\\
63.13	0.01\\
63.14	0.01\\
63.15	0.01\\
63.16	0.01\\
63.17	0.01\\
63.18	0.01\\
63.19	0.01\\
63.2	0.01\\
63.21	0.01\\
63.22	0.01\\
63.23	0.01\\
63.24	0.01\\
63.25	0.01\\
63.26	0.01\\
63.27	0.01\\
63.28	0.01\\
63.29	0.01\\
63.3	0.01\\
63.31	0.01\\
63.32	0.01\\
63.33	0.01\\
63.34	0.01\\
63.35	0.01\\
63.36	0.01\\
63.37	0.01\\
63.38	0.01\\
63.39	0.01\\
63.4	0.01\\
63.41	0.01\\
63.42	0.01\\
63.43	0.01\\
63.44	0.01\\
63.45	0.01\\
63.46	0.01\\
63.47	0.01\\
63.48	0.01\\
63.49	0.01\\
63.5	0.01\\
63.51	0.01\\
63.52	0.01\\
63.53	0.01\\
63.54	0.01\\
63.55	0.01\\
63.56	0.01\\
63.57	0.01\\
63.58	0.01\\
63.59	0.01\\
63.6	0.01\\
63.61	0.01\\
63.62	0.01\\
63.63	0.01\\
63.64	0.01\\
63.65	0.01\\
63.66	0.01\\
63.67	0.01\\
63.68	0.01\\
63.69	0.01\\
63.7	0.01\\
63.71	0.01\\
63.72	0.01\\
63.73	0.01\\
63.74	0.01\\
63.75	0.01\\
63.76	0.01\\
63.77	0.01\\
63.78	0.01\\
63.79	0.01\\
63.8	0.01\\
63.81	0.01\\
63.82	0.01\\
63.83	0.01\\
63.84	0.01\\
63.85	0.01\\
63.86	0.01\\
63.87	0.01\\
63.88	0.01\\
63.89	0.01\\
63.9	0.01\\
63.91	0.01\\
63.92	0.01\\
63.93	0.01\\
63.94	0.01\\
63.95	0.01\\
63.96	0.01\\
63.97	0.01\\
63.98	0.01\\
63.99	0.01\\
64	0.01\\
64.01	0.01\\
64.02	0.01\\
64.03	0.01\\
64.04	0.01\\
64.05	0.01\\
64.06	0.01\\
64.07	0.01\\
64.08	0.01\\
64.09	0.01\\
64.1	0.01\\
64.11	0.01\\
64.12	0.01\\
64.13	0.01\\
64.14	0.01\\
64.15	0.01\\
64.16	0.01\\
64.17	0.01\\
64.18	0.01\\
64.19	0.01\\
64.2	0.01\\
64.21	0.01\\
64.22	0.01\\
64.23	0.01\\
64.24	0.01\\
64.25	0.01\\
64.26	0.01\\
64.27	0.01\\
64.28	0.01\\
64.29	0.01\\
64.3	0.01\\
64.31	0.01\\
64.32	0.01\\
64.33	0.01\\
64.34	0.01\\
64.35	0.01\\
64.36	0.01\\
64.37	0.01\\
64.38	0.01\\
64.39	0.01\\
64.4	0.01\\
64.41	0.01\\
64.42	0.01\\
64.43	0.01\\
64.44	0.01\\
64.45	0.01\\
64.46	0.01\\
64.47	0.01\\
64.48	0.01\\
64.49	0.01\\
64.5	0.01\\
64.51	0.01\\
64.52	0.01\\
64.53	0.01\\
64.54	0.01\\
64.55	0.01\\
64.56	0.01\\
64.57	0.01\\
64.58	0.01\\
64.59	0.01\\
64.6	0.01\\
64.61	0.01\\
64.62	0.01\\
64.63	0.01\\
64.64	0.01\\
64.65	0.01\\
64.66	0.01\\
64.67	0.01\\
64.68	0.01\\
64.69	0.01\\
64.7	0.01\\
64.71	0.01\\
64.72	0.01\\
64.73	0.01\\
64.74	0.01\\
64.75	0.01\\
64.76	0.01\\
64.77	0.01\\
64.78	0.01\\
64.79	0.01\\
64.8	0.01\\
64.81	0.01\\
64.82	0.01\\
64.83	0.01\\
64.84	0.01\\
64.85	0.01\\
64.86	0.01\\
64.87	0.01\\
64.88	0.01\\
64.89	0.01\\
64.9	0.01\\
64.91	0.01\\
64.92	0.01\\
64.93	0.01\\
64.94	0.01\\
64.95	0.01\\
64.96	0.01\\
64.97	0.01\\
64.98	0.01\\
64.99	0.01\\
65	0.01\\
65.01	0.01\\
65.02	0.01\\
65.03	0.01\\
65.04	0.01\\
65.05	0.01\\
65.06	0.01\\
65.07	0.01\\
65.08	0.01\\
65.09	0.01\\
65.1	0.01\\
65.11	0.01\\
65.12	0.01\\
65.13	0.01\\
65.14	0.01\\
65.15	0.01\\
65.16	0.01\\
65.17	0.01\\
65.18	0.01\\
65.19	0.01\\
65.2	0.01\\
65.21	0.01\\
65.22	0.01\\
65.23	0.01\\
65.24	0.01\\
65.25	0.01\\
65.26	0.01\\
65.27	0.01\\
65.28	0.01\\
65.29	0.01\\
65.3	0.01\\
65.31	0.01\\
65.32	0.01\\
65.33	0.01\\
65.34	0.01\\
65.35	0.01\\
65.36	0.01\\
65.37	0.01\\
65.38	0.01\\
65.39	0.01\\
65.4	0.01\\
65.41	0.01\\
65.42	0.01\\
65.43	0.01\\
65.44	0.01\\
65.45	0.01\\
65.46	0.01\\
65.47	0.01\\
65.48	0.01\\
65.49	0.01\\
65.5	0.01\\
65.51	0.01\\
65.52	0.01\\
65.53	0.01\\
65.54	0.01\\
65.55	0.01\\
65.56	0.01\\
65.57	0.01\\
65.58	0.01\\
65.59	0.01\\
65.6	0.01\\
65.61	0.01\\
65.62	0.01\\
65.63	0.01\\
65.64	0.01\\
65.65	0.01\\
65.66	0.01\\
65.67	0.01\\
65.68	0.01\\
65.69	0.01\\
65.7	0.01\\
65.71	0.01\\
65.72	0.01\\
65.73	0.01\\
65.74	0.01\\
65.75	0.01\\
65.76	0.01\\
65.77	0.01\\
65.78	0.01\\
65.79	0.01\\
65.8	0.01\\
65.81	0.01\\
65.82	0.01\\
65.83	0.01\\
65.84	0.01\\
65.85	0.01\\
65.86	0.01\\
65.87	0.01\\
65.88	0.01\\
65.89	0.01\\
65.9	0.01\\
65.91	0.01\\
65.92	0.01\\
65.93	0.01\\
65.94	0.01\\
65.95	0.01\\
65.96	0.01\\
65.97	0.01\\
65.98	0.01\\
65.99	0.01\\
66	0.01\\
66.01	0.01\\
66.02	0.01\\
66.03	0.01\\
66.04	0.01\\
66.05	0.01\\
66.06	0.01\\
66.07	0.01\\
66.08	0.01\\
66.09	0.01\\
66.1	0.01\\
66.11	0.01\\
66.12	0.01\\
66.13	0.01\\
66.14	0.01\\
66.15	0.01\\
66.16	0.01\\
66.17	0.01\\
66.18	0.01\\
66.19	0.01\\
66.2	0.01\\
66.21	0.01\\
66.22	0.01\\
66.23	0.01\\
66.24	0.01\\
66.25	0.01\\
66.26	0.01\\
66.27	0.01\\
66.28	0.01\\
66.29	0.01\\
66.3	0.01\\
66.31	0.01\\
66.32	0.01\\
66.33	0.01\\
66.34	0.01\\
66.35	0.01\\
66.36	0.01\\
66.37	0.01\\
66.38	0.01\\
66.39	0.01\\
66.4	0.01\\
66.41	0.01\\
66.42	0.01\\
66.43	0.01\\
66.44	0.01\\
66.45	0.01\\
66.46	0.01\\
66.47	0.01\\
66.48	0.01\\
66.49	0.01\\
66.5	0.01\\
66.51	0.01\\
66.52	0.01\\
66.53	0.01\\
66.54	0.01\\
66.55	0.01\\
66.56	0.01\\
66.57	0.01\\
66.58	0.01\\
66.59	0.01\\
66.6	0.01\\
66.61	0.01\\
66.62	0.01\\
66.63	0.01\\
66.64	0.01\\
66.65	0.01\\
66.66	0.01\\
66.67	0.01\\
66.68	0.01\\
66.69	0.01\\
66.7	0.01\\
66.71	0.01\\
66.72	0.01\\
66.73	0.01\\
66.74	0.01\\
66.75	0.01\\
66.76	0.01\\
66.77	0.01\\
66.78	0.01\\
66.79	0.01\\
66.8	0.01\\
66.81	0.01\\
66.82	0.01\\
66.83	0.01\\
66.84	0.01\\
66.85	0.01\\
66.86	0.01\\
66.87	0.01\\
66.88	0.01\\
66.89	0.01\\
66.9	0.01\\
66.91	0.01\\
66.92	0.01\\
66.93	0.01\\
66.94	0.01\\
66.95	0.01\\
66.96	0.01\\
66.97	0.01\\
66.98	0.01\\
66.99	0.01\\
67	0.01\\
67.01	0.01\\
67.02	0.01\\
67.03	0.01\\
67.04	0.01\\
67.05	0.01\\
67.06	0.01\\
67.07	0.01\\
67.08	0.01\\
67.09	0.01\\
67.1	0.01\\
67.11	0.01\\
67.12	0.01\\
67.13	0.01\\
67.14	0.01\\
67.15	0.01\\
67.16	0.01\\
67.17	0.01\\
67.18	0.01\\
67.19	0.01\\
67.2	0.01\\
67.21	0.01\\
67.22	0.01\\
67.23	0.01\\
67.24	0.01\\
67.25	0.01\\
67.26	0.01\\
67.27	0.01\\
67.28	0.01\\
67.29	0.01\\
67.3	0.01\\
67.31	0.01\\
67.32	0.01\\
67.33	0.01\\
67.34	0.01\\
67.35	0.01\\
67.36	0.01\\
67.37	0.01\\
67.38	0.01\\
67.39	0.01\\
67.4	0.01\\
67.41	0.01\\
67.42	0.01\\
67.43	0.01\\
67.44	0.01\\
67.45	0.01\\
67.46	0.01\\
67.47	0.01\\
67.48	0.01\\
67.49	0.01\\
67.5	0.01\\
67.51	0.01\\
67.52	0.01\\
67.53	0.01\\
67.54	0.01\\
67.55	0.01\\
67.56	0.01\\
67.57	0.01\\
67.58	0.01\\
67.59	0.01\\
67.6	0.01\\
67.61	0.01\\
67.62	0.01\\
67.63	0.01\\
67.64	0.01\\
67.65	0.01\\
67.66	0.01\\
67.67	0.01\\
67.68	0.01\\
67.69	0.01\\
67.7	0.01\\
67.71	0.01\\
67.72	0.01\\
67.73	0.01\\
67.74	0.01\\
67.75	0.01\\
67.76	0.01\\
67.77	0.01\\
67.78	0.01\\
67.79	0.01\\
67.8	0.01\\
67.81	0.01\\
67.82	0.01\\
67.83	0.01\\
67.84	0.01\\
67.85	0.01\\
67.86	0.01\\
67.87	0.01\\
67.88	0.01\\
67.89	0.01\\
67.9	0.01\\
67.91	0.01\\
67.92	0.01\\
67.93	0.01\\
67.94	0.01\\
67.95	0.01\\
67.96	0.01\\
67.97	0.01\\
67.98	0.01\\
67.99	0.01\\
68	0.01\\
68.01	0.01\\
68.02	0.01\\
68.03	0.01\\
68.04	0.01\\
68.05	0.01\\
68.06	0.01\\
68.07	0.01\\
68.08	0.01\\
68.09	0.01\\
68.1	0.01\\
68.11	0.01\\
68.12	0.01\\
68.13	0.01\\
68.14	0.01\\
68.15	0.01\\
68.16	0.01\\
68.17	0.01\\
68.18	0.01\\
68.19	0.01\\
68.2	0.01\\
68.21	0.01\\
68.22	0.01\\
68.23	0.01\\
68.24	0.01\\
68.25	0.01\\
68.26	0.01\\
68.27	0.01\\
68.28	0.01\\
68.29	0.01\\
68.3	0.01\\
68.31	0.01\\
68.32	0.01\\
68.33	0.01\\
68.34	0.01\\
68.35	0.01\\
68.36	0.01\\
68.37	0.01\\
68.38	0.01\\
68.39	0.01\\
68.4	0.01\\
68.41	0.01\\
68.42	0.01\\
68.43	0.01\\
68.44	0.01\\
68.45	0.01\\
68.46	0.01\\
68.47	0.01\\
68.48	0.01\\
68.49	0.01\\
68.5	0.01\\
68.51	0.01\\
68.52	0.01\\
68.53	0.01\\
68.54	0.01\\
68.55	0.01\\
68.56	0.01\\
68.57	0.01\\
68.58	0.01\\
68.59	0.01\\
68.6	0.01\\
68.61	0.01\\
68.62	0.01\\
68.63	0.01\\
68.64	0.01\\
68.65	0.01\\
68.66	0.01\\
68.67	0.01\\
68.68	0.01\\
68.69	0.01\\
68.7	0.01\\
68.71	0.01\\
68.72	0.01\\
68.73	0.01\\
68.74	0.01\\
68.75	0.01\\
68.76	0.01\\
68.77	0.01\\
68.78	0.01\\
68.79	0.01\\
68.8	0.01\\
68.81	0.01\\
68.82	0.01\\
68.83	0.01\\
68.84	0.01\\
68.85	0.01\\
68.86	0.01\\
68.87	0.01\\
68.88	0.01\\
68.89	0.01\\
68.9	0.01\\
68.91	0.01\\
68.92	0.01\\
68.93	0.01\\
68.94	0.01\\
68.95	0.01\\
68.96	0.01\\
68.97	0.01\\
68.98	0.01\\
68.99	0.01\\
69	0.01\\
69.01	0.01\\
69.02	0.01\\
69.03	0.01\\
69.04	0.01\\
69.05	0.01\\
69.06	0.01\\
69.07	0.01\\
69.08	0.01\\
69.09	0.01\\
69.1	0.01\\
69.11	0.01\\
69.12	0.01\\
69.13	0.01\\
69.14	0.01\\
69.15	0.01\\
69.16	0.01\\
69.17	0.01\\
69.18	0.01\\
69.19	0.01\\
69.2	0.01\\
69.21	0.01\\
69.22	0.01\\
69.23	0.01\\
69.24	0.01\\
69.25	0.01\\
69.26	0.01\\
69.27	0.01\\
69.28	0.01\\
69.29	0.01\\
69.3	0.01\\
69.31	0.01\\
69.32	0.01\\
69.33	0.01\\
69.34	0.01\\
69.35	0.01\\
69.36	0.01\\
69.37	0.01\\
69.38	0.01\\
69.39	0.01\\
69.4	0.01\\
69.41	0.01\\
69.42	0.01\\
69.43	0.01\\
69.44	0.01\\
69.45	0.01\\
69.46	0.01\\
69.47	0.01\\
69.48	0.01\\
69.49	0.01\\
69.5	0.01\\
69.51	0.01\\
69.52	0.01\\
69.53	0.01\\
69.54	0.01\\
69.55	0.01\\
69.56	0.01\\
69.57	0.01\\
69.58	0.01\\
69.59	0.01\\
69.6	0.01\\
69.61	0.01\\
69.62	0.01\\
69.63	0.01\\
69.64	0.01\\
69.65	0.01\\
69.66	0.01\\
69.67	0.01\\
69.68	0.01\\
69.69	0.01\\
69.7	0.01\\
69.71	0.01\\
69.72	0.01\\
69.73	0.01\\
69.74	0.01\\
69.75	0.01\\
69.76	0.01\\
69.77	0.01\\
69.78	0.01\\
69.79	0.01\\
69.8	0.01\\
69.81	0.01\\
69.82	0.01\\
69.83	0.01\\
69.84	0.01\\
69.85	0.01\\
69.86	0.01\\
69.87	0.01\\
69.88	0.01\\
69.89	0.01\\
69.9	0.01\\
69.91	0.01\\
69.92	0.01\\
69.93	0.01\\
69.94	0.01\\
69.95	0.01\\
69.96	0.01\\
69.97	0.01\\
69.98	0.01\\
69.99	0.01\\
70	0.01\\
70.01	0.01\\
70.02	0.01\\
70.03	0.01\\
70.04	0.01\\
70.05	0.01\\
70.06	0.01\\
70.07	0.01\\
70.08	0.01\\
70.09	0.01\\
70.1	0.01\\
70.11	0.01\\
70.12	0.01\\
70.13	0.01\\
70.14	0.01\\
70.15	0.01\\
70.16	0.01\\
70.17	0.01\\
70.18	0.01\\
70.19	0.01\\
70.2	0.01\\
70.21	0.01\\
70.22	0.01\\
70.23	0.01\\
70.24	0.01\\
70.25	0.01\\
70.26	0.01\\
70.27	0.01\\
70.28	0.01\\
70.29	0.01\\
70.3	0.01\\
70.31	0.01\\
70.32	0.01\\
70.33	0.01\\
70.34	0.01\\
70.35	0.01\\
70.36	0.01\\
70.37	0.01\\
70.38	0.01\\
70.39	0.01\\
70.4	0.01\\
70.41	0.01\\
70.42	0.01\\
70.43	0.01\\
70.44	0.01\\
70.45	0.01\\
70.46	0.01\\
70.47	0.01\\
70.48	0.01\\
70.49	0.01\\
70.5	0.01\\
70.51	0.01\\
70.52	0.01\\
70.53	0.01\\
70.54	0.01\\
70.55	0.01\\
70.56	0.01\\
70.57	0.01\\
70.58	0.01\\
70.59	0.01\\
70.6	0.01\\
70.61	0.01\\
70.62	0.01\\
70.63	0.01\\
70.64	0.01\\
70.65	0.01\\
70.66	0.01\\
70.67	0.01\\
70.68	0.01\\
70.69	0.01\\
70.7	0.01\\
70.71	0.01\\
70.72	0.01\\
70.73	0.01\\
70.74	0.01\\
70.75	0.01\\
70.76	0.01\\
70.77	0.01\\
70.78	0.01\\
70.79	0.01\\
70.8	0.01\\
70.81	0.01\\
70.82	0.01\\
70.83	0.01\\
70.84	0.01\\
70.85	0.01\\
70.86	0.01\\
70.87	0.01\\
70.88	0.01\\
70.89	0.01\\
70.9	0.01\\
70.91	0.01\\
70.92	0.01\\
70.93	0.01\\
70.94	0.01\\
70.95	0.01\\
70.96	0.01\\
70.97	0.01\\
70.98	0.01\\
70.99	0.01\\
71	0.01\\
71.01	0.01\\
71.02	0.01\\
71.03	0.01\\
71.04	0.01\\
71.05	0.01\\
71.06	0.01\\
71.07	0.01\\
71.08	0.01\\
71.09	0.01\\
71.1	0.01\\
71.11	0.01\\
71.12	0.01\\
71.13	0.01\\
71.14	0.01\\
71.15	0.01\\
71.16	0.01\\
71.17	0.01\\
71.18	0.01\\
71.19	0.01\\
71.2	0.01\\
71.21	0.01\\
71.22	0.01\\
71.23	0.01\\
71.24	0.01\\
71.25	0.01\\
71.26	0.01\\
71.27	0.01\\
71.28	0.01\\
71.29	0.01\\
71.3	0.01\\
71.31	0.01\\
71.32	0.01\\
71.33	0.01\\
71.34	0.01\\
71.35	0.01\\
71.36	0.01\\
71.37	0.01\\
71.38	0.01\\
71.39	0.01\\
71.4	0.01\\
71.41	0.01\\
71.42	0.01\\
71.43	0.01\\
71.44	0.01\\
71.45	0.01\\
71.46	0.01\\
71.47	0.01\\
71.48	0.01\\
71.49	0.01\\
71.5	0.01\\
71.51	0.01\\
71.52	0.01\\
71.53	0.01\\
71.54	0.01\\
71.55	0.01\\
71.56	0.01\\
71.57	0.01\\
71.58	0.01\\
71.59	0.01\\
71.6	0.01\\
71.61	0.01\\
71.62	0.01\\
71.63	0.01\\
71.64	0.01\\
71.65	0.01\\
71.66	0.01\\
71.67	0.01\\
71.68	0.01\\
71.69	0.01\\
71.7	0.01\\
71.71	0.01\\
71.72	0.01\\
71.73	0.01\\
71.74	0.01\\
71.75	0.01\\
71.76	0.01\\
71.77	0.01\\
71.78	0.01\\
71.79	0.01\\
71.8	0.01\\
71.81	0.01\\
71.82	0.01\\
71.83	0.01\\
71.84	0.01\\
71.85	0.01\\
71.86	0.01\\
71.87	0.01\\
71.88	0.01\\
71.89	0.01\\
71.9	0.01\\
71.91	0.01\\
71.92	0.01\\
71.93	0.01\\
71.94	0.01\\
71.95	0.01\\
71.96	0.01\\
71.97	0.01\\
71.98	0.01\\
71.99	0.01\\
72	0.01\\
72.01	0.01\\
72.02	0.01\\
72.03	0.01\\
72.04	0.01\\
72.05	0.01\\
72.06	0.01\\
72.07	0.01\\
72.08	0.01\\
72.09	0.01\\
72.1	0.01\\
72.11	0.01\\
72.12	0.01\\
72.13	0.01\\
72.14	0.01\\
72.15	0.01\\
72.16	0.01\\
72.17	0.01\\
72.18	0.01\\
72.19	0.01\\
72.2	0.01\\
72.21	0.01\\
72.22	0.01\\
72.23	0.01\\
72.24	0.01\\
72.25	0.01\\
72.26	0.01\\
72.27	0.01\\
72.28	0.01\\
72.29	0.01\\
72.3	0.01\\
72.31	0.01\\
72.32	0.01\\
72.33	0.01\\
72.34	0.01\\
72.35	0.01\\
72.36	0.01\\
72.37	0.01\\
72.38	0.01\\
72.39	0.01\\
72.4	0.01\\
72.41	0.01\\
72.42	0.01\\
72.43	0.01\\
72.44	0.01\\
72.45	0.01\\
72.46	0.01\\
72.47	0.01\\
72.48	0.01\\
72.49	0.01\\
72.5	0.01\\
72.51	0.01\\
72.52	0.01\\
72.53	0.01\\
72.54	0.01\\
72.55	0.01\\
72.56	0.01\\
72.57	0.01\\
72.58	0.01\\
72.59	0.01\\
72.6	0.01\\
72.61	0.01\\
72.62	0.01\\
72.63	0.01\\
72.64	0.01\\
72.65	0.01\\
72.66	0.01\\
72.67	0.01\\
72.68	0.01\\
72.69	0.01\\
72.7	0.01\\
72.71	0.01\\
72.72	0.01\\
72.73	0.01\\
72.74	0.01\\
72.75	0.01\\
72.76	0.01\\
72.77	0.01\\
72.78	0.01\\
72.79	0.01\\
72.8	0.01\\
72.81	0.01\\
72.82	0.01\\
72.83	0.01\\
72.84	0.01\\
72.85	0.01\\
72.86	0.01\\
72.87	0.01\\
72.88	0.01\\
72.89	0.01\\
72.9	0.01\\
72.91	0.01\\
72.92	0.01\\
72.93	0.01\\
72.94	0.01\\
72.95	0.01\\
72.96	0.01\\
72.97	0.01\\
72.98	0.01\\
72.99	0.01\\
73	0.01\\
73.01	0.01\\
73.02	0.01\\
73.03	0.01\\
73.04	0.01\\
73.05	0.01\\
73.06	0.01\\
73.07	0.01\\
73.08	0.01\\
73.09	0.01\\
73.1	0.01\\
73.11	0.01\\
73.12	0.01\\
73.13	0.01\\
73.14	0.01\\
73.15	0.01\\
73.16	0.01\\
73.17	0.01\\
73.18	0.01\\
73.19	0.01\\
73.2	0.01\\
73.21	0.01\\
73.22	0.01\\
73.23	0.01\\
73.24	0.01\\
73.25	0.01\\
73.26	0.01\\
73.27	0.01\\
73.28	0.01\\
73.29	0.01\\
73.3	0.01\\
73.31	0.01\\
73.32	0.01\\
73.33	0.01\\
73.34	0.01\\
73.35	0.01\\
73.36	0.01\\
73.37	0.01\\
73.38	0.01\\
73.39	0.01\\
73.4	0.01\\
73.41	0.01\\
73.42	0.01\\
73.43	0.01\\
73.44	0.01\\
73.45	0.01\\
73.46	0.01\\
73.47	0.01\\
73.48	0.01\\
73.49	0.01\\
73.5	0.01\\
73.51	0.01\\
73.52	0.01\\
73.53	0.01\\
73.54	0.01\\
73.55	0.01\\
73.56	0.01\\
73.57	0.01\\
73.58	0.01\\
73.59	0.01\\
73.6	0.01\\
73.61	0.01\\
73.62	0.01\\
73.63	0.01\\
73.64	0.01\\
73.65	0.01\\
73.66	0.01\\
73.67	0.01\\
73.68	0.01\\
73.69	0.01\\
73.7	0.01\\
73.71	0.01\\
73.72	0.01\\
73.73	0.01\\
73.74	0.01\\
73.75	0.01\\
73.76	0.01\\
73.77	0.01\\
73.78	0.01\\
73.79	0.01\\
73.8	0.01\\
73.81	0.01\\
73.82	0.01\\
73.83	0.01\\
73.84	0.01\\
73.85	0.01\\
73.86	0.01\\
73.87	0.01\\
73.88	0.01\\
73.89	0.01\\
73.9	0.01\\
73.91	0.01\\
73.92	0.01\\
73.93	0.01\\
73.94	0.01\\
73.95	0.01\\
73.96	0.01\\
73.97	0.01\\
73.98	0.01\\
73.99	0.01\\
74	0.01\\
74.01	0.01\\
74.02	0.01\\
74.03	0.01\\
74.04	0.01\\
74.05	0.01\\
74.06	0.01\\
74.07	0.01\\
74.08	0.01\\
74.09	0.01\\
74.1	0.01\\
74.11	0.01\\
74.12	0.01\\
74.13	0.01\\
74.14	0.01\\
74.15	0.01\\
74.16	0.01\\
74.17	0.01\\
74.18	0.01\\
74.19	0.01\\
74.2	0.01\\
74.21	0.01\\
74.22	0.01\\
74.23	0.01\\
74.24	0.01\\
74.25	0.01\\
74.26	0.01\\
74.27	0.01\\
74.28	0.01\\
74.29	0.01\\
74.3	0.01\\
74.31	0.01\\
74.32	0.01\\
74.33	0.01\\
74.34	0.01\\
74.35	0.01\\
74.36	0.01\\
74.37	0.01\\
74.38	0.01\\
74.39	0.01\\
74.4	0.01\\
74.41	0.01\\
74.42	0.01\\
74.43	0.01\\
74.44	0.01\\
74.45	0.01\\
74.46	0.01\\
74.47	0.01\\
74.48	0.01\\
74.49	0.01\\
74.5	0.01\\
74.51	0.01\\
74.52	0.01\\
74.53	0.01\\
74.54	0.01\\
74.55	0.01\\
74.56	0.01\\
74.57	0.01\\
74.58	0.01\\
74.59	0.01\\
74.6	0.01\\
74.61	0.01\\
74.62	0.01\\
74.63	0.01\\
74.64	0.01\\
74.65	0.01\\
74.66	0.01\\
74.67	0.01\\
74.68	0.01\\
74.69	0.01\\
74.7	0.01\\
74.71	0.01\\
74.72	0.01\\
74.73	0.01\\
74.74	0.01\\
74.75	0.01\\
74.76	0.01\\
74.77	0.01\\
74.78	0.01\\
74.79	0.01\\
74.8	0.01\\
74.81	0.01\\
74.82	0.01\\
74.83	0.01\\
74.84	0.01\\
74.85	0.01\\
74.86	0.01\\
74.87	0.01\\
74.88	0.01\\
74.89	0.01\\
74.9	0.01\\
74.91	0.01\\
74.92	0.01\\
74.93	0.01\\
74.94	0.01\\
74.95	0.01\\
74.96	0.01\\
74.97	0.01\\
74.98	0.01\\
74.99	0.01\\
75	0.01\\
75.01	0.01\\
75.02	0.01\\
75.03	0.01\\
75.04	0.01\\
75.05	0.01\\
75.06	0.01\\
75.07	0.01\\
75.08	0.01\\
75.09	0.01\\
75.1	0.01\\
75.11	0.01\\
75.12	0.01\\
75.13	0.01\\
75.14	0.01\\
75.15	0.01\\
75.16	0.01\\
75.17	0.01\\
75.18	0.01\\
75.19	0.01\\
75.2	0.01\\
75.21	0.01\\
75.22	0.01\\
75.23	0.01\\
75.24	0.01\\
75.25	0.01\\
75.26	0.01\\
75.27	0.01\\
75.28	0.01\\
75.29	0.01\\
75.3	0.01\\
75.31	0.01\\
75.32	0.01\\
75.33	0.01\\
75.34	0.01\\
75.35	0.01\\
75.36	0.01\\
75.37	0.01\\
75.38	0.01\\
75.39	0.01\\
75.4	0.01\\
75.41	0.01\\
75.42	0.01\\
75.43	0.01\\
75.44	0.01\\
75.45	0.01\\
75.46	0.01\\
75.47	0.01\\
75.48	0.01\\
75.49	0.01\\
75.5	0.01\\
75.51	0.01\\
75.52	0.01\\
75.53	0.01\\
75.54	0.01\\
75.55	0.01\\
75.56	0.01\\
75.57	0.01\\
75.58	0.01\\
75.59	0.01\\
75.6	0.01\\
75.61	0.01\\
75.62	0.01\\
75.63	0.01\\
75.64	0.01\\
75.65	0.01\\
75.66	0.01\\
75.67	0.01\\
75.68	0.01\\
75.69	0.01\\
75.7	0.01\\
75.71	0.01\\
75.72	0.01\\
75.73	0.01\\
75.74	0.01\\
75.75	0.01\\
75.76	0.01\\
75.77	0.01\\
75.78	0.01\\
75.79	0.01\\
75.8	0.01\\
75.81	0.01\\
75.82	0.01\\
75.83	0.01\\
75.84	0.01\\
75.85	0.01\\
75.86	0.01\\
75.87	0.01\\
75.88	0.01\\
75.89	0.01\\
75.9	0.01\\
75.91	0.01\\
75.92	0.01\\
75.93	0.01\\
75.94	0.01\\
75.95	0.01\\
75.96	0.01\\
75.97	0.01\\
75.98	0.01\\
75.99	0.01\\
76	0.01\\
76.01	0.01\\
76.02	0.01\\
76.03	0.01\\
76.04	0.01\\
76.05	0.01\\
76.06	0.01\\
76.07	0.01\\
76.08	0.01\\
76.09	0.01\\
76.1	0.01\\
76.11	0.01\\
76.12	0.01\\
76.13	0.01\\
76.14	0.01\\
76.15	0.01\\
76.16	0.01\\
76.17	0.01\\
76.18	0.01\\
76.19	0.01\\
76.2	0.01\\
76.21	0.01\\
76.22	0.01\\
76.23	0.01\\
76.24	0.01\\
76.25	0.01\\
76.26	0.01\\
76.27	0.01\\
76.28	0.01\\
76.29	0.01\\
76.3	0.01\\
76.31	0.01\\
76.32	0.01\\
76.33	0.01\\
76.34	0.01\\
76.35	0.01\\
76.36	0.01\\
76.37	0.01\\
76.38	0.01\\
76.39	0.01\\
76.4	0.01\\
76.41	0.01\\
76.42	0.01\\
76.43	0.01\\
76.44	0.01\\
76.45	0.01\\
76.46	0.01\\
76.47	0.01\\
76.48	0.01\\
76.49	0.01\\
76.5	0.01\\
76.51	0.01\\
76.52	0.01\\
76.53	0.01\\
76.54	0.01\\
76.55	0.01\\
76.56	0.01\\
76.57	0.01\\
76.58	0.01\\
76.59	0.01\\
76.6	0.01\\
76.61	0.01\\
76.62	0.01\\
76.63	0.01\\
76.64	0.01\\
76.65	0.01\\
76.66	0.01\\
76.67	0.01\\
76.68	0.01\\
76.69	0.01\\
76.7	0.01\\
76.71	0.01\\
76.72	0.01\\
76.73	0.01\\
76.74	0.01\\
76.75	0.01\\
76.76	0.01\\
76.77	0.01\\
76.78	0.01\\
76.79	0.01\\
76.8	0.01\\
76.81	0.01\\
76.82	0.01\\
76.83	0.01\\
76.84	0.01\\
76.85	0.01\\
76.86	0.01\\
76.87	0.01\\
76.88	0.01\\
76.89	0.01\\
76.9	0.01\\
76.91	0.01\\
76.92	0.01\\
76.93	0.01\\
76.94	0.01\\
76.95	0.01\\
76.96	0.01\\
76.97	0.01\\
76.98	0.01\\
76.99	0.01\\
77	0.01\\
77.01	0.01\\
77.02	0.01\\
77.03	0.01\\
77.04	0.01\\
77.05	0.01\\
77.06	0.01\\
77.07	0.01\\
77.08	0.01\\
77.09	0.01\\
77.1	0.01\\
77.11	0.01\\
77.12	0.01\\
77.13	0.01\\
77.14	0.01\\
77.15	0.01\\
77.16	0.01\\
77.17	0.01\\
77.18	0.01\\
77.19	0.01\\
77.2	0.01\\
77.21	0.01\\
77.22	0.01\\
77.23	0.01\\
77.24	0.01\\
77.25	0.01\\
77.26	0.01\\
77.27	0.01\\
77.28	0.01\\
77.29	0.01\\
77.3	0.01\\
77.31	0.01\\
77.32	0.01\\
77.33	0.01\\
77.34	0.01\\
77.35	0.01\\
77.36	0.01\\
77.37	0.01\\
77.38	0.01\\
77.39	0.01\\
77.4	0.01\\
77.41	0.01\\
77.42	0.01\\
77.43	0.01\\
77.44	0.01\\
77.45	0.01\\
77.46	0.01\\
77.47	0.01\\
77.48	0.01\\
77.49	0.01\\
77.5	0.01\\
77.51	0.01\\
77.52	0.01\\
77.53	0.01\\
77.54	0.01\\
77.55	0.01\\
77.56	0.01\\
77.57	0.01\\
77.58	0.01\\
77.59	0.01\\
77.6	0.01\\
77.61	0.01\\
77.62	0.01\\
77.63	0.01\\
77.64	0.01\\
77.65	0.01\\
77.66	0.01\\
77.67	0.01\\
77.68	0.01\\
77.69	0.01\\
77.7	0.01\\
77.71	0.01\\
77.72	0.01\\
77.73	0.01\\
77.74	0.01\\
77.75	0.01\\
77.76	0.01\\
77.77	0.01\\
77.78	0.01\\
77.79	0.01\\
77.8	0.01\\
77.81	0.01\\
77.82	0.01\\
77.83	0.01\\
77.84	0.01\\
77.85	0.01\\
77.86	0.01\\
77.87	0.01\\
77.88	0.01\\
77.89	0.01\\
77.9	0.01\\
77.91	0.01\\
77.92	0.01\\
77.93	0.01\\
77.94	0.01\\
77.95	0.01\\
77.96	0.01\\
77.97	0.01\\
77.98	0.01\\
77.99	0.01\\
78	0.01\\
78.01	0.01\\
78.02	0.01\\
78.03	0.01\\
78.04	0.01\\
78.05	0.01\\
78.06	0.01\\
78.07	0.01\\
78.08	0.01\\
78.09	0.01\\
78.1	0.01\\
78.11	0.01\\
78.12	0.01\\
78.13	0.01\\
78.14	0.01\\
78.15	0.01\\
78.16	0.01\\
78.17	0.01\\
78.18	0.01\\
78.19	0.01\\
78.2	0.01\\
78.21	0.01\\
78.22	0.01\\
78.23	0.01\\
78.24	0.01\\
78.25	0.01\\
78.26	0.01\\
78.27	0.01\\
78.28	0.01\\
78.29	0.01\\
78.3	0.01\\
78.31	0.01\\
78.32	0.01\\
78.33	0.01\\
78.34	0.01\\
78.35	0.01\\
78.36	0.01\\
78.37	0.01\\
78.38	0.01\\
78.39	0.01\\
78.4	0.01\\
78.41	0.01\\
78.42	0.01\\
78.43	0.01\\
78.44	0.01\\
78.45	0.01\\
78.46	0.01\\
78.47	0.01\\
78.48	0.01\\
78.49	0.01\\
78.5	0.01\\
78.51	0.01\\
78.52	0.01\\
78.53	0.01\\
78.54	0.01\\
78.55	0.01\\
78.56	0.01\\
78.57	0.01\\
78.58	0.01\\
78.59	0.01\\
78.6	0.01\\
78.61	0.01\\
78.62	0.01\\
78.63	0.01\\
78.64	0.01\\
78.65	0.01\\
78.66	0.01\\
78.67	0.01\\
78.68	0.01\\
78.69	0.01\\
78.7	0.01\\
78.71	0.01\\
78.72	0.01\\
78.73	0.01\\
78.74	0.01\\
78.75	0.01\\
78.76	0.01\\
78.77	0.01\\
78.78	0.01\\
78.79	0.01\\
78.8	0.01\\
78.81	0.01\\
78.82	0.01\\
78.83	0.01\\
78.84	0.01\\
78.85	0.01\\
78.86	0.01\\
78.87	0.01\\
78.88	0.01\\
78.89	0.01\\
78.9	0.01\\
78.91	0.01\\
78.92	0.01\\
78.93	0.01\\
78.94	0.01\\
78.95	0.01\\
78.96	0.01\\
78.97	0.01\\
78.98	0.01\\
78.99	0.01\\
79	0.01\\
79.01	0.01\\
79.02	0.01\\
79.03	0.01\\
79.04	0.01\\
79.05	0.01\\
79.06	0.01\\
79.07	0.01\\
79.08	0.01\\
79.09	0.01\\
79.1	0.01\\
79.11	0.01\\
79.12	0.01\\
79.13	0.01\\
79.14	0.01\\
79.15	0.01\\
79.16	0.01\\
79.17	0.01\\
79.18	0.01\\
79.19	0.01\\
79.2	0.01\\
79.21	0.01\\
79.22	0.01\\
79.23	0.01\\
79.24	0.01\\
79.25	0.01\\
79.26	0.01\\
79.27	0.01\\
79.28	0.01\\
79.29	0.01\\
79.3	0.01\\
79.31	0.01\\
79.32	0.01\\
79.33	0.01\\
79.34	0.01\\
79.35	0.01\\
79.36	0.01\\
79.37	0.01\\
79.38	0.01\\
79.39	0.01\\
79.4	0.01\\
79.41	0.01\\
79.42	0.01\\
79.43	0.01\\
79.44	0.01\\
79.45	0.01\\
79.46	0.01\\
79.47	0.01\\
79.48	0.01\\
79.49	0.01\\
79.5	0.01\\
79.51	0.01\\
79.52	0.01\\
79.53	0.01\\
79.54	0.01\\
79.55	0.01\\
79.56	0.01\\
79.57	0.01\\
79.58	0.01\\
79.59	0.01\\
79.6	0.01\\
79.61	0.01\\
79.62	0.01\\
79.63	0.01\\
79.64	0.01\\
79.65	0.01\\
79.66	0.01\\
79.67	0.01\\
79.68	0.01\\
79.69	0.01\\
79.7	0.01\\
79.71	0.01\\
79.72	0.01\\
79.73	0.01\\
79.74	0.01\\
79.75	0.01\\
79.76	0.01\\
79.77	0.01\\
79.78	0.01\\
79.79	0.01\\
79.8	0.01\\
79.81	0.01\\
79.82	0.01\\
79.83	0.01\\
79.84	0.01\\
79.85	0.01\\
79.86	0.01\\
79.87	0.01\\
79.88	0.01\\
79.89	0.01\\
79.9	0.01\\
79.91	0.01\\
79.92	0.01\\
79.93	0.01\\
79.94	0.01\\
79.95	0.01\\
79.96	0.01\\
79.97	0.01\\
79.98	0.01\\
79.99	0.01\\
80	0.01\\
80.01	0.01\\
};
\addplot [color=blue,dashed]
  table[row sep=crcr]{%
80.01	0.01\\
80.02	0.01\\
80.03	0.01\\
80.04	0.01\\
80.05	0.01\\
80.06	0.01\\
80.07	0.01\\
80.08	0.01\\
80.09	0.01\\
80.1	0.01\\
80.11	0.01\\
80.12	0.01\\
80.13	0.01\\
80.14	0.01\\
80.15	0.01\\
80.16	0.01\\
80.17	0.01\\
80.18	0.01\\
80.19	0.01\\
80.2	0.01\\
80.21	0.01\\
80.22	0.01\\
80.23	0.01\\
80.24	0.01\\
80.25	0.01\\
80.26	0.01\\
80.27	0.01\\
80.28	0.01\\
80.29	0.01\\
80.3	0.01\\
80.31	0.01\\
80.32	0.01\\
80.33	0.01\\
80.34	0.01\\
80.35	0.01\\
80.36	0.01\\
80.37	0.01\\
80.38	0.01\\
80.39	0.01\\
80.4	0.01\\
80.41	0.01\\
80.42	0.01\\
80.43	0.01\\
80.44	0.01\\
80.45	0.01\\
80.46	0.01\\
80.47	0.01\\
80.48	0.01\\
80.49	0.01\\
80.5	0.01\\
80.51	0.01\\
80.52	0.01\\
80.53	0.01\\
80.54	0.01\\
80.55	0.01\\
80.56	0.01\\
80.57	0.01\\
80.58	0.01\\
80.59	0.01\\
80.6	0.01\\
80.61	0.01\\
80.62	0.01\\
80.63	0.01\\
80.64	0.01\\
80.65	0.01\\
80.66	0.01\\
80.67	0.01\\
80.68	0.01\\
80.69	0.01\\
80.7	0.01\\
80.71	0.01\\
80.72	0.01\\
80.73	0.01\\
80.74	0.01\\
80.75	0.01\\
80.76	0.01\\
80.77	0.01\\
80.78	0.01\\
80.79	0.01\\
80.8	0.01\\
80.81	0.01\\
80.82	0.01\\
80.83	0.01\\
80.84	0.01\\
80.85	0.01\\
80.86	0.01\\
80.87	0.01\\
80.88	0.01\\
80.89	0.01\\
80.9	0.01\\
80.91	0.01\\
80.92	0.01\\
80.93	0.01\\
80.94	0.01\\
80.95	0.01\\
80.96	0.01\\
80.97	0.01\\
80.98	0.01\\
80.99	0.01\\
81	0.01\\
81.01	0.01\\
81.02	0.01\\
81.03	0.01\\
81.04	0.01\\
81.05	0.01\\
81.06	0.01\\
81.07	0.01\\
81.08	0.01\\
81.09	0.01\\
81.1	0.01\\
81.11	0.01\\
81.12	0.01\\
81.13	0.01\\
81.14	0.01\\
81.15	0.01\\
81.16	0.01\\
81.17	0.01\\
81.18	0.01\\
81.19	0.01\\
81.2	0.01\\
81.21	0.01\\
81.22	0.01\\
81.23	0.01\\
81.24	0.01\\
81.25	0.01\\
81.26	0.01\\
81.27	0.01\\
81.28	0.01\\
81.29	0.01\\
81.3	0.01\\
81.31	0.01\\
81.32	0.01\\
81.33	0.01\\
81.34	0.01\\
81.35	0.01\\
81.36	0.01\\
81.37	0.01\\
81.38	0.01\\
81.39	0.01\\
81.4	0.01\\
81.41	0.01\\
81.42	0.01\\
81.43	0.01\\
81.44	0.01\\
81.45	0.01\\
81.46	0.01\\
81.47	0.01\\
81.48	0.01\\
81.49	0.01\\
81.5	0.01\\
81.51	0.01\\
81.52	0.01\\
81.53	0.01\\
81.54	0.01\\
81.55	0.01\\
81.56	0.01\\
81.57	0.01\\
81.58	0.01\\
81.59	0.01\\
81.6	0.01\\
81.61	0.01\\
81.62	0.01\\
81.63	0.01\\
81.64	0.01\\
81.65	0.01\\
81.66	0.01\\
81.67	0.01\\
81.68	0.01\\
81.69	0.01\\
81.7	0.01\\
81.71	0.01\\
81.72	0.01\\
81.73	0.01\\
81.74	0.01\\
81.75	0.01\\
81.76	0.01\\
81.77	0.01\\
81.78	0.01\\
81.79	0.01\\
81.8	0.01\\
81.81	0.01\\
81.82	0.01\\
81.83	0.01\\
81.84	0.01\\
81.85	0.01\\
81.86	0.01\\
81.87	0.01\\
81.88	0.01\\
81.89	0.01\\
81.9	0.01\\
81.91	0.01\\
81.92	0.01\\
81.93	0.01\\
81.94	0.01\\
81.95	0.01\\
81.96	0.01\\
81.97	0.01\\
81.98	0.01\\
81.99	0.01\\
82	0.01\\
82.01	0.01\\
82.02	0.01\\
82.03	0.01\\
82.04	0.01\\
82.05	0.01\\
82.06	0.01\\
82.07	0.01\\
82.08	0.01\\
82.09	0.01\\
82.1	0.01\\
82.11	0.01\\
82.12	0.01\\
82.13	0.01\\
82.14	0.01\\
82.15	0.01\\
82.16	0.01\\
82.17	0.01\\
82.18	0.01\\
82.19	0.01\\
82.2	0.01\\
82.21	0.01\\
82.22	0.01\\
82.23	0.01\\
82.24	0.01\\
82.25	0.01\\
82.26	0.01\\
82.27	0.01\\
82.28	0.01\\
82.29	0.01\\
82.3	0.01\\
82.31	0.01\\
82.32	0.01\\
82.33	0.01\\
82.34	0.01\\
82.35	0.01\\
82.36	0.01\\
82.37	0.01\\
82.38	0.01\\
82.39	0.01\\
82.4	0.01\\
82.41	0.01\\
82.42	0.01\\
82.43	0.01\\
82.44	0.01\\
82.45	0.01\\
82.46	0.01\\
82.47	0.01\\
82.48	0.01\\
82.49	0.01\\
82.5	0.01\\
82.51	0.01\\
82.52	0.01\\
82.53	0.01\\
82.54	0.01\\
82.55	0.01\\
82.56	0.01\\
82.57	0.01\\
82.58	0.01\\
82.59	0.01\\
82.6	0.01\\
82.61	0.01\\
82.62	0.01\\
82.63	0.01\\
82.64	0.01\\
82.65	0.01\\
82.66	0.01\\
82.67	0.01\\
82.68	0.01\\
82.69	0.01\\
82.7	0.01\\
82.71	0.01\\
82.72	0.01\\
82.73	0.01\\
82.74	0.01\\
82.75	0.01\\
82.76	0.01\\
82.77	0.01\\
82.78	0.01\\
82.79	0.01\\
82.8	0.01\\
82.81	0.01\\
82.82	0.01\\
82.83	0.01\\
82.84	0.01\\
82.85	0.01\\
82.86	0.01\\
82.87	0.01\\
82.88	0.01\\
82.89	0.01\\
82.9	0.01\\
82.91	0.01\\
82.92	0.01\\
82.93	0.01\\
82.94	0.01\\
82.95	0.01\\
82.96	0.01\\
82.97	0.01\\
82.98	0.01\\
82.99	0.01\\
83	0.01\\
83.01	0.01\\
83.02	0.01\\
83.03	0.01\\
83.04	0.01\\
83.05	0.01\\
83.06	0.01\\
83.07	0.01\\
83.08	0.01\\
83.09	0.01\\
83.1	0.01\\
83.11	0.01\\
83.12	0.01\\
83.13	0.01\\
83.14	0.01\\
83.15	0.01\\
83.16	0.01\\
83.17	0.01\\
83.18	0.01\\
83.19	0.01\\
83.2	0.01\\
83.21	0.01\\
83.22	0.01\\
83.23	0.01\\
83.24	0.01\\
83.25	0.01\\
83.26	0.01\\
83.27	0.01\\
83.28	0.01\\
83.29	0.01\\
83.3	0.01\\
83.31	0.01\\
83.32	0.01\\
83.33	0.01\\
83.34	0.01\\
83.35	0.01\\
83.36	0.01\\
83.37	0.01\\
83.38	0.01\\
83.39	0.01\\
83.4	0.01\\
83.41	0.01\\
83.42	0.01\\
83.43	0.01\\
83.44	0.01\\
83.45	0.01\\
83.46	0.01\\
83.47	0.01\\
83.48	0.01\\
83.49	0.01\\
83.5	0.01\\
83.51	0.01\\
83.52	0.01\\
83.53	0.01\\
83.54	0.01\\
83.55	0.01\\
83.56	0.01\\
83.57	0.01\\
83.58	0.01\\
83.59	0.01\\
83.6	0.01\\
83.61	0.01\\
83.62	0.01\\
83.63	0.01\\
83.64	0.01\\
83.65	0.01\\
83.66	0.01\\
83.67	0.01\\
83.68	0.01\\
83.69	0.01\\
83.7	0.01\\
83.71	0.01\\
83.72	0.01\\
83.73	0.01\\
83.74	0.01\\
83.75	0.01\\
83.76	0.01\\
83.77	0.01\\
83.78	0.01\\
83.79	0.01\\
83.8	0.01\\
83.81	0.01\\
83.82	0.01\\
83.83	0.01\\
83.84	0.01\\
83.85	0.01\\
83.86	0.01\\
83.87	0.01\\
83.88	0.01\\
83.89	0.01\\
83.9	0.01\\
83.91	0.01\\
83.92	0.01\\
83.93	0.01\\
83.94	0.01\\
83.95	0.01\\
83.96	0.01\\
83.97	0.01\\
83.98	0.01\\
83.99	0.01\\
84	0.01\\
84.01	0.01\\
84.02	0.01\\
84.03	0.01\\
84.04	0.01\\
84.05	0.01\\
84.06	0.01\\
84.07	0.01\\
84.08	0.01\\
84.09	0.01\\
84.1	0.01\\
84.11	0.01\\
84.12	0.01\\
84.13	0.01\\
84.14	0.01\\
84.15	0.01\\
84.16	0.01\\
84.17	0.01\\
84.18	0.01\\
84.19	0.01\\
84.2	0.01\\
84.21	0.01\\
84.22	0.01\\
84.23	0.01\\
84.24	0.01\\
84.25	0.01\\
84.26	0.01\\
84.27	0.01\\
84.28	0.01\\
84.29	0.01\\
84.3	0.01\\
84.31	0.01\\
84.32	0.01\\
84.33	0.01\\
84.34	0.01\\
84.35	0.01\\
84.36	0.01\\
84.37	0.01\\
84.38	0.01\\
84.39	0.01\\
84.4	0.01\\
84.41	0.01\\
84.42	0.01\\
84.43	0.01\\
84.44	0.01\\
84.45	0.01\\
84.46	0.01\\
84.47	0.01\\
84.48	0.01\\
84.49	0.01\\
84.5	0.01\\
84.51	0.01\\
84.52	0.01\\
84.53	0.01\\
84.54	0.01\\
84.55	0.01\\
84.56	0.01\\
84.57	0.01\\
84.58	0.01\\
84.59	0.01\\
84.6	0.01\\
84.61	0.01\\
84.62	0.01\\
84.63	0.01\\
84.64	0.01\\
84.65	0.01\\
84.66	0.01\\
84.67	0.01\\
84.68	0.01\\
84.69	0.01\\
84.7	0.01\\
84.71	0.01\\
84.72	0.01\\
84.73	0.01\\
84.74	0.01\\
84.75	0.01\\
84.76	0.01\\
84.77	0.01\\
84.78	0.01\\
84.79	0.01\\
84.8	0.01\\
84.81	0.01\\
84.82	0.01\\
84.83	0.01\\
84.84	0.01\\
84.85	0.01\\
84.86	0.01\\
84.87	0.01\\
84.88	0.01\\
84.89	0.01\\
84.9	0.01\\
84.91	0.01\\
84.92	0.01\\
84.93	0.01\\
84.94	0.01\\
84.95	0.01\\
84.96	0.01\\
84.97	0.01\\
84.98	0.01\\
84.99	0.01\\
85	0.01\\
85.01	0.01\\
85.02	0.01\\
85.03	0.01\\
85.04	0.01\\
85.05	0.01\\
85.06	0.01\\
85.07	0.01\\
85.08	0.01\\
85.09	0.01\\
85.1	0.01\\
85.11	0.01\\
85.12	0.01\\
85.13	0.01\\
85.14	0.01\\
85.15	0.01\\
85.16	0.01\\
85.17	0.01\\
85.18	0.01\\
85.19	0.01\\
85.2	0.01\\
85.21	0.01\\
85.22	0.01\\
85.23	0.01\\
85.24	0.01\\
85.25	0.01\\
85.26	0.01\\
85.27	0.01\\
85.28	0.01\\
85.29	0.01\\
85.3	0.01\\
85.31	0.01\\
85.32	0.01\\
85.33	0.01\\
85.34	0.01\\
85.35	0.01\\
85.36	0.01\\
85.37	0.01\\
85.38	0.01\\
85.39	0.01\\
85.4	0.01\\
85.41	0.01\\
85.42	0.01\\
85.43	0.01\\
85.44	0.01\\
85.45	0.01\\
85.46	0.01\\
85.47	0.01\\
85.48	0.01\\
85.49	0.01\\
85.5	0.01\\
85.51	0.01\\
85.52	0.01\\
85.53	0.01\\
85.54	0.01\\
85.55	0.01\\
85.56	0.01\\
85.57	0.01\\
85.58	0.01\\
85.59	0.01\\
85.6	0.01\\
85.61	0.01\\
85.62	0.01\\
85.63	0.01\\
85.64	0.01\\
85.65	0.01\\
85.66	0.01\\
85.67	0.01\\
85.68	0.01\\
85.69	0.01\\
85.7	0.01\\
85.71	0.01\\
85.72	0.01\\
85.73	0.01\\
85.74	0.01\\
85.75	0.01\\
85.76	0.01\\
85.77	0.01\\
85.78	0.01\\
85.79	0.01\\
85.8	0.01\\
85.81	0.01\\
85.82	0.01\\
85.83	0.01\\
85.84	0.01\\
85.85	0.01\\
85.86	0.01\\
85.87	0.01\\
85.88	0.01\\
85.89	0.01\\
85.9	0.01\\
85.91	0.01\\
85.92	0.01\\
85.93	0.01\\
85.94	0.01\\
85.95	0.01\\
85.96	0.01\\
85.97	0.01\\
85.98	0.01\\
85.99	0.01\\
86	0.01\\
86.01	0.01\\
86.02	0.01\\
86.03	0.01\\
86.04	0.01\\
86.05	0.01\\
86.06	0.01\\
86.07	0.01\\
86.08	0.01\\
86.09	0.01\\
86.1	0.01\\
86.11	0.01\\
86.12	0.01\\
86.13	0.01\\
86.14	0.01\\
86.15	0.01\\
86.16	0.01\\
86.17	0.01\\
86.18	0.01\\
86.19	0.01\\
86.2	0.01\\
86.21	0.01\\
86.22	0.01\\
86.23	0.01\\
86.24	0.01\\
86.25	0.01\\
86.26	0.01\\
86.27	0.01\\
86.28	0.01\\
86.29	0.01\\
86.3	0.01\\
86.31	0.01\\
86.32	0.01\\
86.33	0.01\\
86.34	0.01\\
86.35	0.01\\
86.36	0.01\\
86.37	0.01\\
86.38	0.01\\
86.39	0.01\\
86.4	0.01\\
86.41	0.01\\
86.42	0.01\\
86.43	0.01\\
86.44	0.01\\
86.45	0.01\\
86.46	0.01\\
86.47	0.01\\
86.48	0.01\\
86.49	0.01\\
86.5	0.01\\
86.51	0.01\\
86.52	0.01\\
86.53	0.01\\
86.54	0.01\\
86.55	0.01\\
86.56	0.01\\
86.57	0.01\\
86.58	0.01\\
86.59	0.01\\
86.6	0.01\\
86.61	0.01\\
86.62	0.01\\
86.63	0.01\\
86.64	0.01\\
86.65	0.01\\
86.66	0.01\\
86.67	0.01\\
86.68	0.01\\
86.69	0.01\\
86.7	0.01\\
86.71	0.01\\
86.72	0.01\\
86.73	0.01\\
86.74	0.01\\
86.75	0.01\\
86.76	0.01\\
86.77	0.01\\
86.78	0.01\\
86.79	0.01\\
86.8	0.01\\
86.81	0.01\\
86.82	0.01\\
86.83	0.01\\
86.84	0.01\\
86.85	0.01\\
86.86	0.01\\
86.87	0.01\\
86.88	0.01\\
86.89	0.01\\
86.9	0.01\\
86.91	0.01\\
86.92	0.01\\
86.93	0.01\\
86.94	0.01\\
86.95	0.01\\
86.96	0.01\\
86.97	0.01\\
86.98	0.01\\
86.99	0.01\\
87	0.01\\
87.01	0.01\\
87.02	0.01\\
87.03	0.01\\
87.04	0.01\\
87.05	0.01\\
87.06	0.01\\
87.07	0.01\\
87.08	0.01\\
87.09	0.01\\
87.1	0.01\\
87.11	0.01\\
87.12	0.01\\
87.13	0.01\\
87.14	0.01\\
87.15	0.01\\
87.16	0.01\\
87.17	0.01\\
87.18	0.01\\
87.19	0.01\\
87.2	0.01\\
87.21	0.01\\
87.22	0.01\\
87.23	0.01\\
87.24	0.01\\
87.25	0.01\\
87.26	0.01\\
87.27	0.01\\
87.28	0.01\\
87.29	0.01\\
87.3	0.01\\
87.31	0.01\\
87.32	0.01\\
87.33	0.01\\
87.34	0.01\\
87.35	0.01\\
87.36	0.01\\
87.37	0.01\\
87.38	0.01\\
87.39	0.01\\
87.4	0.01\\
87.41	0.01\\
87.42	0.01\\
87.43	0.01\\
87.44	0.01\\
87.45	0.01\\
87.46	0.01\\
87.47	0.01\\
87.48	0.01\\
87.49	0.01\\
87.5	0.01\\
87.51	0.01\\
87.52	0.01\\
87.53	0.01\\
87.54	0.01\\
87.55	0.01\\
87.56	0.01\\
87.57	0.01\\
87.58	0.01\\
87.59	0.01\\
87.6	0.01\\
87.61	0.01\\
87.62	0.01\\
87.63	0.01\\
87.64	0.01\\
87.65	0.01\\
87.66	0.01\\
87.67	0.01\\
87.68	0.01\\
87.69	0.01\\
87.7	0.01\\
87.71	0.01\\
87.72	0.01\\
87.73	0.01\\
87.74	0.01\\
87.75	0.01\\
87.76	0.01\\
87.77	0.01\\
87.78	0.01\\
87.79	0.01\\
87.8	0.01\\
87.81	0.01\\
87.82	0.01\\
87.83	0.01\\
87.84	0.01\\
87.85	0.01\\
87.86	0.01\\
87.87	0.01\\
87.88	0.01\\
87.89	0.01\\
87.9	0.01\\
87.91	0.01\\
87.92	0.01\\
87.93	0.01\\
87.94	0.01\\
87.95	0.01\\
87.96	0.01\\
87.97	0.01\\
87.98	0.01\\
87.99	0.01\\
88	0.01\\
88.01	0.01\\
88.02	0.01\\
88.03	0.01\\
88.04	0.01\\
88.05	0.01\\
88.06	0.01\\
88.07	0.01\\
88.08	0.01\\
88.09	0.01\\
88.1	0.01\\
88.11	0.01\\
88.12	0.01\\
88.13	0.01\\
88.14	0.01\\
88.15	0.01\\
88.16	0.01\\
88.17	0.01\\
88.18	0.01\\
88.19	0.01\\
88.2	0.01\\
88.21	0.01\\
88.22	0.01\\
88.23	0.01\\
88.24	0.01\\
88.25	0.01\\
88.26	0.01\\
88.27	0.01\\
88.28	0.01\\
88.29	0.01\\
88.3	0.01\\
88.31	0.01\\
88.32	0.01\\
88.33	0.01\\
88.34	0.01\\
88.35	0.01\\
88.36	0.01\\
88.37	0.01\\
88.38	0.01\\
88.39	0.01\\
88.4	0.01\\
88.41	0.01\\
88.42	0.01\\
88.43	0.01\\
88.44	0.01\\
88.45	0.01\\
88.46	0.01\\
88.47	0.01\\
88.48	0.01\\
88.49	0.01\\
88.5	0.01\\
88.51	0.01\\
88.52	0.01\\
88.53	0.01\\
88.54	0.01\\
88.55	0.01\\
88.56	0.01\\
88.57	0.01\\
88.58	0.01\\
88.59	0.01\\
88.6	0.01\\
88.61	0.01\\
88.62	0.01\\
88.63	0.01\\
88.64	0.01\\
88.65	0.01\\
88.66	0.01\\
88.67	0.01\\
88.68	0.01\\
88.69	0.01\\
88.7	0.01\\
88.71	0.01\\
88.72	0.01\\
88.73	0.01\\
88.74	0.01\\
88.75	0.01\\
88.76	0.01\\
88.77	0.01\\
88.78	0.01\\
88.79	0.01\\
88.8	0.01\\
88.81	0.01\\
88.82	0.01\\
88.83	0.01\\
88.84	0.01\\
88.85	0.01\\
88.86	0.01\\
88.87	0.01\\
88.88	0.01\\
88.89	0.01\\
88.9	0.01\\
88.91	0.01\\
88.92	0.01\\
88.93	0.01\\
88.94	0.01\\
88.95	0.01\\
88.96	0.01\\
88.97	0.01\\
88.98	0.01\\
88.99	0.01\\
89	0.01\\
89.01	0.01\\
89.02	0.01\\
89.03	0.01\\
89.04	0.01\\
89.05	0.01\\
89.06	0.01\\
89.07	0.01\\
89.08	0.01\\
89.09	0.01\\
89.1	0.01\\
89.11	0.01\\
89.12	0.01\\
89.13	0.01\\
89.14	0.01\\
89.15	0.01\\
89.16	0.01\\
89.17	0.01\\
89.18	0.01\\
89.19	0.01\\
89.2	0.01\\
89.21	0.01\\
89.22	0.01\\
89.23	0.01\\
89.24	0.01\\
89.25	0.01\\
89.26	0.01\\
89.27	0.01\\
89.28	0.01\\
89.29	0.01\\
89.3	0.01\\
89.31	0.01\\
89.32	0.01\\
89.33	0.01\\
89.34	0.01\\
89.35	0.01\\
89.36	0.01\\
89.37	0.01\\
89.38	0.01\\
89.39	0.01\\
89.4	0.01\\
89.41	0.01\\
89.42	0.01\\
89.43	0.01\\
89.44	0.01\\
89.45	0.01\\
89.46	0.01\\
89.47	0.01\\
89.48	0.01\\
89.49	0.01\\
89.5	0.01\\
89.51	0.01\\
89.52	0.01\\
89.53	0.01\\
89.54	0.01\\
89.55	0.01\\
89.56	0.01\\
89.57	0.01\\
89.58	0.01\\
89.59	0.01\\
89.6	0.01\\
89.61	0.01\\
89.62	0.01\\
89.63	0.01\\
89.64	0.01\\
89.65	0.01\\
89.66	0.01\\
89.67	0.01\\
89.68	0.01\\
89.69	0.01\\
89.7	0.01\\
89.71	0.01\\
89.72	0.01\\
89.73	0.01\\
89.74	0.01\\
89.75	0.01\\
89.76	0.01\\
89.77	0.01\\
89.78	0.01\\
89.79	0.01\\
89.8	0.01\\
89.81	0.01\\
89.82	0.01\\
89.83	0.01\\
89.84	0.01\\
89.85	0.01\\
89.86	0.01\\
89.87	0.01\\
89.88	0.01\\
89.89	0.01\\
89.9	0.01\\
89.91	0.01\\
89.92	0.01\\
89.93	0.01\\
89.94	0.01\\
89.95	0.01\\
89.96	0.01\\
89.97	0.01\\
89.98	0.01\\
89.99	0.01\\
90	0.01\\
90.01	0.01\\
90.02	0.01\\
90.03	0.01\\
90.04	0.01\\
90.05	0.01\\
90.06	0.01\\
90.07	0.01\\
90.08	0.01\\
90.09	0.01\\
90.1	0.01\\
90.11	0.01\\
90.12	0.01\\
90.13	0.01\\
90.14	0.01\\
90.15	0.01\\
90.16	0.01\\
90.17	0.01\\
90.18	0.01\\
90.19	0.01\\
90.2	0.01\\
90.21	0.01\\
90.22	0.01\\
90.23	0.01\\
90.24	0.01\\
90.25	0.01\\
90.26	0.01\\
90.27	0.01\\
90.28	0.01\\
90.29	0.01\\
90.3	0.01\\
90.31	0.01\\
90.32	0.01\\
90.33	0.01\\
90.34	0.01\\
90.35	0.01\\
90.36	0.01\\
90.37	0.01\\
90.38	0.01\\
90.39	0.01\\
90.4	0.01\\
90.41	0.01\\
90.42	0.01\\
90.43	0.01\\
90.44	0.01\\
90.45	0.01\\
90.46	0.01\\
90.47	0.01\\
90.48	0.01\\
90.49	0.01\\
90.5	0.01\\
90.51	0.01\\
90.52	0.01\\
90.53	0.01\\
90.54	0.01\\
90.55	0.01\\
90.56	0.01\\
90.57	0.01\\
90.58	0.01\\
90.59	0.01\\
90.6	0.01\\
90.61	0.01\\
90.62	0.01\\
90.63	0.01\\
90.64	0.01\\
90.65	0.01\\
90.66	0.01\\
90.67	0.01\\
90.68	0.01\\
90.69	0.01\\
90.7	0.01\\
90.71	0.01\\
90.72	0.01\\
90.73	0.01\\
90.74	0.01\\
90.75	0.01\\
90.76	0.01\\
90.77	0.01\\
90.78	0.01\\
90.79	0.01\\
90.8	0.01\\
90.81	0.01\\
90.82	0.01\\
90.83	0.01\\
90.84	0.01\\
90.85	0.01\\
90.86	0.01\\
90.87	0.01\\
90.88	0.01\\
90.89	0.01\\
90.9	0.01\\
90.91	0.01\\
90.92	0.01\\
90.93	0.01\\
90.94	0.01\\
90.95	0.01\\
90.96	0.01\\
90.97	0.01\\
90.98	0.01\\
90.99	0.01\\
91	0.01\\
91.01	0.01\\
91.02	0.01\\
91.03	0.01\\
91.04	0.01\\
91.05	0.01\\
91.06	0.01\\
91.07	0.01\\
91.08	0.01\\
91.09	0.01\\
91.1	0.01\\
91.11	0.01\\
91.12	0.01\\
91.13	0.01\\
91.14	0.01\\
91.15	0.01\\
91.16	0.01\\
91.17	0.01\\
91.18	0.01\\
91.19	0.01\\
91.2	0.01\\
91.21	0.01\\
91.22	0.01\\
91.23	0.01\\
91.24	0.01\\
91.25	0.01\\
91.26	0.01\\
91.27	0.01\\
91.28	0.01\\
91.29	0.01\\
91.3	0.01\\
91.31	0.01\\
91.32	0.01\\
91.33	0.01\\
91.34	0.01\\
91.35	0.01\\
91.36	0.01\\
91.37	0.01\\
91.38	0.01\\
91.39	0.01\\
91.4	0.01\\
91.41	0.01\\
91.42	0.01\\
91.43	0.01\\
91.44	0.01\\
91.45	0.01\\
91.46	0.01\\
91.47	0.01\\
91.48	0.01\\
91.49	0.01\\
91.5	0.01\\
91.51	0.01\\
91.52	0.01\\
91.53	0.01\\
91.54	0.01\\
91.55	0.01\\
91.56	0.01\\
91.57	0.01\\
91.58	0.01\\
91.59	0.01\\
91.6	0.01\\
91.61	0.01\\
91.62	0.01\\
91.63	0.01\\
91.64	0.01\\
91.65	0.01\\
91.66	0.01\\
91.67	0.01\\
91.68	0.01\\
91.69	0.01\\
91.7	0.01\\
91.71	0.01\\
91.72	0.01\\
91.73	0.01\\
91.74	0.01\\
91.75	0.01\\
91.76	0.01\\
91.77	0.01\\
91.78	0.01\\
91.79	0.01\\
91.8	0.01\\
91.81	0.01\\
91.82	0.01\\
91.83	0.01\\
91.84	0.01\\
91.85	0.01\\
91.86	0.01\\
91.87	0.01\\
91.88	0.01\\
91.89	0.01\\
91.9	0.01\\
91.91	0.01\\
91.92	0.01\\
91.93	0.01\\
91.94	0.01\\
91.95	0.01\\
91.96	0.01\\
91.97	0.01\\
91.98	0.01\\
91.99	0.01\\
92	0.01\\
92.01	0.01\\
92.02	0.01\\
92.03	0.01\\
92.04	0.01\\
92.05	0.01\\
92.06	0.01\\
92.07	0.01\\
92.08	0.01\\
92.09	0.01\\
92.1	0.01\\
92.11	0.01\\
92.12	0.01\\
92.13	0.01\\
92.14	0.01\\
92.15	0.01\\
92.16	0.01\\
92.17	0.01\\
92.18	0.01\\
92.19	0.01\\
92.2	0.01\\
92.21	0.01\\
92.22	0.01\\
92.23	0.01\\
92.24	0.01\\
92.25	0.01\\
92.26	0.01\\
92.27	0.01\\
92.28	0.01\\
92.29	0.01\\
92.3	0.01\\
92.31	0.01\\
92.32	0.01\\
92.33	0.01\\
92.34	0.01\\
92.35	0.01\\
92.36	0.01\\
92.37	0.01\\
92.38	0.01\\
92.39	0.01\\
92.4	0.01\\
92.41	0.01\\
92.42	0.01\\
92.43	0.01\\
92.44	0.01\\
92.45	0.01\\
92.46	0.01\\
92.47	0.01\\
92.48	0.01\\
92.49	0.01\\
92.5	0.01\\
92.51	0.01\\
92.52	0.01\\
92.53	0.01\\
92.54	0.01\\
92.55	0.01\\
92.56	0.01\\
92.57	0.01\\
92.58	0.01\\
92.59	0.01\\
92.6	0.01\\
92.61	0.01\\
92.62	0.01\\
92.63	0.01\\
92.64	0.01\\
92.65	0.01\\
92.66	0.01\\
92.67	0.01\\
92.68	0.01\\
92.69	0.01\\
92.7	0.01\\
92.71	0.01\\
92.72	0.01\\
92.73	0.01\\
92.74	0.01\\
92.75	0.01\\
92.76	0.01\\
92.77	0.01\\
92.78	0.01\\
92.79	0.01\\
92.8	0.01\\
92.81	0.01\\
92.82	0.01\\
92.83	0.01\\
92.84	0.01\\
92.85	0.01\\
92.86	0.01\\
92.87	0.01\\
92.88	0.01\\
92.89	0.01\\
92.9	0.01\\
92.91	0.01\\
92.92	0.01\\
92.93	0.01\\
92.94	0.01\\
92.95	0.01\\
92.96	0.01\\
92.97	0.01\\
92.98	0.01\\
92.99	0.01\\
93	0.01\\
93.01	0.01\\
93.02	0.01\\
93.03	0.01\\
93.04	0.01\\
93.05	0.01\\
93.06	0.01\\
93.07	0.01\\
93.08	0.01\\
93.09	0.01\\
93.1	0.01\\
93.11	0.01\\
93.12	0.01\\
93.13	0.01\\
93.14	0.01\\
93.15	0.01\\
93.16	0.01\\
93.17	0.01\\
93.18	0.01\\
93.19	0.01\\
93.2	0.01\\
93.21	0.01\\
93.22	0.01\\
93.23	0.01\\
93.24	0.01\\
93.25	0.01\\
93.26	0.01\\
93.27	0.01\\
93.28	0.01\\
93.29	0.01\\
93.3	0.01\\
93.31	0.01\\
93.32	0.01\\
93.33	0.01\\
93.34	0.01\\
93.35	0.01\\
93.36	0.01\\
93.37	0.01\\
93.38	0.01\\
93.39	0.01\\
93.4	0.01\\
93.41	0.01\\
93.42	0.01\\
93.43	0.01\\
93.44	0.01\\
93.45	0.01\\
93.46	0.01\\
93.47	0.01\\
93.48	0.01\\
93.49	0.01\\
93.5	0.01\\
93.51	0.01\\
93.52	0.01\\
93.53	0.01\\
93.54	0.01\\
93.55	0.01\\
93.56	0.01\\
93.57	0.01\\
93.58	0.01\\
93.59	0.01\\
93.6	0.01\\
93.61	0.01\\
93.62	0.01\\
93.63	0.01\\
93.64	0.01\\
93.65	0.01\\
93.66	0.01\\
93.67	0.01\\
93.68	0.01\\
93.69	0.01\\
93.7	0.01\\
93.71	0.01\\
93.72	0.01\\
93.73	0.01\\
93.74	0.01\\
93.75	0.01\\
93.76	0.01\\
93.77	0.01\\
93.78	0.01\\
93.79	0.01\\
93.8	0.01\\
93.81	0.01\\
93.82	0.01\\
93.83	0.01\\
93.84	0.01\\
93.85	0.01\\
93.86	0.01\\
93.87	0.01\\
93.88	0.01\\
93.89	0.01\\
93.9	0.01\\
93.91	0.01\\
93.92	0.01\\
93.93	0.01\\
93.94	0.01\\
93.95	0.01\\
93.96	0.01\\
93.97	0.01\\
93.98	0.01\\
93.99	0.01\\
94	0.01\\
94.01	0.01\\
94.02	0.01\\
94.03	0.01\\
94.04	0.01\\
94.05	0.01\\
94.06	0.01\\
94.07	0.01\\
94.08	0.01\\
94.09	0.01\\
94.1	0.01\\
94.11	0.01\\
94.12	0.01\\
94.13	0.01\\
94.14	0.01\\
94.15	0.01\\
94.16	0.01\\
94.17	0.01\\
94.18	0.01\\
94.19	0.01\\
94.2	0.01\\
94.21	0.01\\
94.22	0.01\\
94.23	0.01\\
94.24	0.01\\
94.25	0.01\\
94.26	0.01\\
94.27	0.01\\
94.28	0.01\\
94.29	0.01\\
94.3	0.01\\
94.31	0.01\\
94.32	0.01\\
94.33	0.01\\
94.34	0.01\\
94.35	0.01\\
94.36	0.01\\
94.37	0.01\\
94.38	0.01\\
94.39	0.01\\
94.4	0.01\\
94.41	0.01\\
94.42	0.01\\
94.43	0.01\\
94.44	0.01\\
94.45	0.01\\
94.46	0.01\\
94.47	0.01\\
94.48	0.01\\
94.49	0.01\\
94.5	0.01\\
94.51	0.01\\
94.52	0.01\\
94.53	0.01\\
94.54	0.01\\
94.55	0.01\\
94.56	0.01\\
94.57	0.01\\
94.58	0.01\\
94.59	0.01\\
94.6	0.01\\
94.61	0.01\\
94.62	0.01\\
94.63	0.01\\
94.64	0.01\\
94.65	0.01\\
94.66	0.01\\
94.67	0.01\\
94.68	0.01\\
94.69	0.01\\
94.7	0.01\\
94.71	0.01\\
94.72	0.01\\
94.73	0.01\\
94.74	0.01\\
94.75	0.01\\
94.76	0.01\\
94.77	0.01\\
94.78	0.01\\
94.79	0.01\\
94.8	0.01\\
94.81	0.01\\
94.82	0.01\\
94.83	0.01\\
94.84	0.01\\
94.85	0.01\\
94.86	0.01\\
94.87	0.01\\
94.88	0.01\\
94.89	0.01\\
94.9	0.01\\
94.91	0.01\\
94.92	0.01\\
94.93	0.01\\
94.94	0.01\\
94.95	0.01\\
94.96	0.01\\
94.97	0.01\\
94.98	0.01\\
94.99	0.01\\
95	0.01\\
95.01	0.01\\
95.02	0.01\\
95.03	0.01\\
95.04	0.01\\
95.05	0.01\\
95.06	0.01\\
95.07	0.01\\
95.08	0.01\\
95.09	0.01\\
95.1	0.01\\
95.11	0.01\\
95.12	0.01\\
95.13	0.01\\
95.14	0.01\\
95.15	0.01\\
95.16	0.01\\
95.17	0.01\\
95.18	0.01\\
95.19	0.01\\
95.2	0.01\\
95.21	0.01\\
95.22	0.01\\
95.23	0.01\\
95.24	0.01\\
95.25	0.01\\
95.26	0.01\\
95.27	0.01\\
95.28	0.01\\
95.29	0.01\\
95.3	0.01\\
95.31	0.01\\
95.32	0.01\\
95.33	0.01\\
95.34	0.01\\
95.35	0.01\\
95.36	0.01\\
95.37	0.01\\
95.38	0.01\\
95.39	0.01\\
95.4	0.01\\
95.41	0.01\\
95.42	0.01\\
95.43	0.01\\
95.44	0.01\\
95.45	0.01\\
95.46	0.01\\
95.47	0.01\\
95.48	0.01\\
95.49	0.01\\
95.5	0.01\\
95.51	0.01\\
95.52	0.01\\
95.53	0.01\\
95.54	0.01\\
95.55	0.01\\
95.56	0.01\\
95.57	0.01\\
95.58	0.01\\
95.59	0.01\\
95.6	0.01\\
95.61	0.01\\
95.62	0.01\\
95.63	0.01\\
95.64	0.01\\
95.65	0.01\\
95.66	0.01\\
95.67	0.01\\
95.68	0.01\\
95.69	0.01\\
95.7	0.01\\
95.71	0.01\\
95.72	0.01\\
95.73	0.01\\
95.74	0.01\\
95.75	0.01\\
95.76	0.01\\
95.77	0.01\\
95.78	0.01\\
95.79	0.01\\
95.8	0.01\\
95.81	0.01\\
95.82	0.01\\
95.83	0.01\\
95.84	0.01\\
95.85	0.01\\
95.86	0.01\\
95.87	0.01\\
95.88	0.01\\
95.89	0.01\\
95.9	0.01\\
95.91	0.01\\
95.92	0.01\\
95.93	0.01\\
95.94	0.01\\
95.95	0.01\\
95.96	0.01\\
95.97	0.01\\
95.98	0.01\\
95.99	0.01\\
96	0.01\\
96.01	0.01\\
96.02	0.01\\
96.03	0.01\\
96.04	0.01\\
96.05	0.01\\
96.06	0.01\\
96.07	0.01\\
96.08	0.01\\
96.09	0.01\\
96.1	0.01\\
96.11	0.01\\
96.12	0.01\\
96.13	0.01\\
96.14	0.01\\
96.15	0.01\\
96.16	0.01\\
96.17	0.01\\
96.18	0.01\\
96.19	0.01\\
96.2	0.01\\
96.21	0.01\\
96.22	0.01\\
96.23	0.01\\
96.24	0.01\\
96.25	0.01\\
96.26	0.01\\
96.27	0.01\\
96.28	0.01\\
96.29	0.01\\
96.3	0.01\\
96.31	0.01\\
96.32	0.01\\
96.33	0.01\\
96.34	0.01\\
96.35	0.01\\
96.36	0.01\\
96.37	0.01\\
96.38	0.01\\
96.39	0.01\\
96.4	0.01\\
96.41	0.01\\
96.42	0.01\\
96.43	0.01\\
96.44	0.01\\
96.45	0.01\\
96.46	0.01\\
96.47	0.01\\
96.48	0.01\\
96.49	0.01\\
96.5	0.01\\
96.51	0.01\\
96.52	0.01\\
96.53	0.01\\
96.54	0.01\\
96.55	0.01\\
96.56	0.01\\
96.57	0.01\\
96.58	0.01\\
96.59	0.01\\
96.6	0.01\\
96.61	0.01\\
96.62	0.01\\
96.63	0.01\\
96.64	0.01\\
96.65	0.01\\
96.66	0.01\\
96.67	0.01\\
96.68	0.01\\
96.69	0.01\\
96.7	0.01\\
96.71	0.01\\
96.72	0.01\\
96.73	0.01\\
96.74	0.01\\
96.75	0.01\\
96.76	0.01\\
96.77	0.01\\
96.78	0.01\\
96.79	0.01\\
96.8	0.01\\
96.81	0.01\\
96.82	0.01\\
96.83	0.01\\
96.84	0.01\\
96.85	0.01\\
96.86	0.01\\
96.87	0.01\\
96.88	0.01\\
96.89	0.01\\
96.9	0.01\\
96.91	0.01\\
96.92	0.01\\
96.93	0.01\\
96.94	0.01\\
96.95	0.01\\
96.96	0.01\\
96.97	0.01\\
96.98	0.01\\
96.99	0.01\\
97	0.01\\
97.01	0.01\\
97.02	0.01\\
97.03	0.01\\
97.04	0.01\\
97.05	0.01\\
97.06	0.01\\
97.07	0.01\\
97.08	0.01\\
97.09	0.01\\
97.1	0.01\\
97.11	0.01\\
97.12	0.01\\
97.13	0.01\\
97.14	0.01\\
97.15	0.01\\
97.16	0.01\\
97.17	0.01\\
97.18	0.01\\
97.19	0.01\\
97.2	0.01\\
97.21	0.01\\
97.22	0.01\\
97.23	0.01\\
97.24	0.01\\
97.25	0.01\\
97.26	0.01\\
97.27	0.01\\
97.28	0.01\\
97.29	0.01\\
97.3	0.01\\
97.31	0.01\\
97.32	0.01\\
97.33	0.01\\
97.34	0.01\\
97.35	0.01\\
97.36	0.01\\
97.37	0.01\\
97.38	0.01\\
97.39	0.01\\
97.4	0.01\\
97.41	0.01\\
97.42	0.01\\
97.43	0.01\\
97.44	0.01\\
97.45	0.01\\
97.46	0.01\\
97.47	0.01\\
97.48	0.01\\
97.49	0.01\\
97.5	0.01\\
97.51	0.01\\
97.52	0.01\\
97.53	0.01\\
97.54	0.01\\
97.55	0.01\\
97.56	0.01\\
97.57	0.01\\
97.58	0.01\\
97.59	0.01\\
97.6	0.01\\
97.61	0.01\\
97.62	0.01\\
97.63	0.01\\
97.64	0.01\\
97.65	0.01\\
97.66	0.01\\
97.67	0.01\\
97.68	0.01\\
97.69	0.01\\
97.7	0.01\\
97.71	0.01\\
97.72	0.01\\
97.73	0.01\\
97.74	0.01\\
97.75	0.01\\
97.76	0.01\\
97.77	0.01\\
97.78	0.01\\
97.79	0.01\\
97.8	0.01\\
97.81	0.01\\
97.82	0.01\\
97.83	0.01\\
97.84	0.01\\
97.85	0.01\\
97.86	0.01\\
97.87	0.01\\
97.88	0.01\\
97.89	0.01\\
97.9	0.01\\
97.91	0.01\\
97.92	0.01\\
97.93	0.01\\
97.94	0.01\\
97.95	0.01\\
97.96	0.01\\
97.97	0.01\\
97.98	0.01\\
97.99	0.01\\
98	0.01\\
98.01	0.01\\
98.02	0.01\\
98.03	0.01\\
98.04	0.01\\
98.05	0.01\\
98.06	0.01\\
98.07	0.01\\
98.08	0.01\\
98.09	0.01\\
98.1	0.01\\
98.11	0.01\\
98.12	0.01\\
98.13	0.01\\
98.14	0.01\\
98.15	0.01\\
98.16	0.01\\
98.17	0.01\\
98.18	0.01\\
98.19	0.01\\
98.2	0.01\\
98.21	0.01\\
98.22	0.01\\
98.23	0.01\\
98.24	0.01\\
98.25	0.01\\
98.26	0.01\\
98.27	0.01\\
98.28	0.01\\
98.29	0.01\\
98.3	0.01\\
98.31	0.01\\
98.32	0.01\\
98.33	0.01\\
98.34	0.01\\
98.35	0.01\\
98.36	0.01\\
98.37	0.01\\
98.38	0.01\\
98.39	0.01\\
98.4	0.01\\
98.41	0.01\\
98.42	0.01\\
98.43	0.01\\
98.44	0.01\\
98.45	0.01\\
98.46	0.01\\
98.47	0.01\\
98.48	0.01\\
98.49	0.01\\
98.5	0.01\\
98.51	0.01\\
98.52	0.01\\
98.53	0.01\\
98.54	0.01\\
98.55	0.01\\
98.56	0.01\\
98.57	0.01\\
98.58	0.01\\
98.59	0.01\\
98.6	0.01\\
98.61	0.01\\
98.62	0.01\\
98.63	0.01\\
98.64	0.01\\
98.65	0.01\\
98.66	0.01\\
98.67	0.01\\
98.68	0.01\\
98.69	0.01\\
98.7	0.01\\
98.71	0.01\\
98.72	0.01\\
98.73	0.01\\
98.74	0.01\\
98.75	0.01\\
98.76	0.01\\
98.77	0.01\\
98.78	0.01\\
98.79	0.01\\
98.8	0.01\\
98.81	0.01\\
98.82	0.01\\
98.83	0.01\\
98.84	0.01\\
98.85	0.01\\
98.86	0.01\\
98.87	0.01\\
98.88	0.01\\
98.89	0.01\\
98.9	0.01\\
98.91	0.01\\
98.92	0.01\\
98.93	0.01\\
98.94	0.01\\
98.95	0.01\\
98.96	0.01\\
98.97	0.01\\
98.98	0.01\\
98.99	0.01\\
99	0.01\\
99.01	0.01\\
99.02	0.01\\
99.03	0.01\\
99.04	0.01\\
99.05	0.01\\
99.06	0.01\\
99.07	0.01\\
99.08	0.01\\
99.09	0.01\\
99.1	0.01\\
99.11	0.01\\
99.12	0.01\\
99.13	0.01\\
99.14	0.01\\
99.15	0.01\\
99.16	0.01\\
99.17	0.01\\
99.18	0.01\\
99.19	0.01\\
99.2	0.01\\
99.21	0.01\\
99.22	0.01\\
99.23	0.01\\
99.24	0.01\\
99.25	0.01\\
99.26	0.01\\
99.27	0.01\\
99.28	0.01\\
99.29	0.01\\
99.3	0.01\\
99.31	0.01\\
99.32	0.01\\
99.33	0.01\\
99.34	0.01\\
99.35	0.01\\
99.36	0.01\\
99.37	0.01\\
99.38	0.01\\
99.39	0.01\\
99.4	0.01\\
99.41	0.01\\
99.42	0.01\\
99.43	0.01\\
99.44	0.01\\
99.45	0.01\\
99.46	0.01\\
99.47	0.01\\
99.48	0.01\\
99.49	0.01\\
99.5	0.01\\
99.51	0.01\\
99.52	0.01\\
99.53	0.01\\
99.54	0.01\\
99.55	0.01\\
99.56	0.01\\
99.57	0.01\\
99.58	0.01\\
99.59	0.01\\
99.6	0.01\\
99.61	0.01\\
99.62	0.01\\
99.63	0.01\\
99.64	0.01\\
99.65	0.01\\
99.66	0.01\\
99.67	0.01\\
99.68	0.01\\
99.69	0.01\\
99.7	0.01\\
99.71	0.01\\
99.72	0.01\\
99.73	0.01\\
99.74	0.01\\
99.75	0.01\\
99.76	0.01\\
99.77	0.01\\
99.78	0.01\\
99.79	0.01\\
99.8	0.01\\
99.81	0.01\\
99.82	0.01\\
99.83	0.01\\
99.84	0.01\\
99.85	0.01\\
99.86	0.01\\
99.87	0.01\\
99.88	0.01\\
99.89	0.01\\
99.9	0.01\\
99.91	0.01\\
99.92	0.01\\
99.93	0.01\\
99.94	0.01\\
99.95	0.01\\
99.96	0.01\\
99.97	0.01\\
99.98	0.01\\
99.99	0.01\\
100	0.01\\
};
\addlegendentry{$q=-1$};

\addplot [color=black,solid,forget plot]
  table[row sep=crcr]{%
0.01	0.00798727870709108\\
0.02	0.00798727870950488\\
0.03	0.00798727871192002\\
0.04	0.00798727871433649\\
0.05	0.0079872787167543\\
0.06	0.00798727871917344\\
0.07	0.00798727872159392\\
0.08	0.00798727872401573\\
0.09	0.00798727872643889\\
0.1	0.00798727872886338\\
0.11	0.00798727873128922\\
0.12	0.00798727873371639\\
0.13	0.00798727873614491\\
0.14	0.00798727873857477\\
0.15	0.00798727874100598\\
0.16	0.00798727874343853\\
0.17	0.00798727874587243\\
0.18	0.00798727874830768\\
0.19	0.00798727875074427\\
0.2	0.00798727875318222\\
0.21	0.00798727875562151\\
0.22	0.00798727875806216\\
0.23	0.00798727876050416\\
0.24	0.00798727876294751\\
0.25	0.00798727876539222\\
0.26	0.00798727876783828\\
0.27	0.00798727877028571\\
0.28	0.00798727877273448\\
0.29	0.00798727877518462\\
0.3	0.00798727877763612\\
0.31	0.00798727878008897\\
0.32	0.00798727878254319\\
0.33	0.00798727878499878\\
0.34	0.00798727878745572\\
0.35	0.00798727878991403\\
0.36	0.00798727879237371\\
0.37	0.00798727879483476\\
0.38	0.00798727879729717\\
0.39	0.00798727879976095\\
0.4	0.0079872788022261\\
0.41	0.00798727880469263\\
0.42	0.00798727880716052\\
0.43	0.00798727880962979\\
0.44	0.00798727881210043\\
0.45	0.00798727881457245\\
0.46	0.00798727881704584\\
0.47	0.00798727881952061\\
0.48	0.00798727882199676\\
0.49	0.00798727882447429\\
0.5	0.0079872788269532\\
0.51	0.00798727882943349\\
0.52	0.00798727883191517\\
0.53	0.00798727883439823\\
0.54	0.00798727883688267\\
0.55	0.0079872788393685\\
0.56	0.00798727884185572\\
0.57	0.00798727884434432\\
0.58	0.00798727884683431\\
0.59	0.00798727884932569\\
0.6	0.00798727885181847\\
0.61	0.00798727885431263\\
0.62	0.00798727885680819\\
0.63	0.00798727885930514\\
0.64	0.00798727886180349\\
0.65	0.00798727886430324\\
0.66	0.00798727886680438\\
0.67	0.00798727886930692\\
0.68	0.00798727887181086\\
0.69	0.0079872788743162\\
0.7	0.00798727887682295\\
0.71	0.00798727887933109\\
0.72	0.00798727888184064\\
0.73	0.0079872788843516\\
0.74	0.00798727888686396\\
0.75	0.00798727888937773\\
0.76	0.0079872788918929\\
0.77	0.00798727889440949\\
0.78	0.00798727889692748\\
0.79	0.00798727889944689\\
0.8	0.00798727890196771\\
0.81	0.00798727890448995\\
0.82	0.00798727890701359\\
0.83	0.00798727890953866\\
0.84	0.00798727891206514\\
0.85	0.00798727891459304\\
0.86	0.00798727891712235\\
0.87	0.00798727891965309\\
0.88	0.00798727892218525\\
0.89	0.00798727892471883\\
0.9	0.00798727892725384\\
0.91	0.00798727892979027\\
0.92	0.00798727893232812\\
0.93	0.0079872789348674\\
0.94	0.00798727893740812\\
0.95	0.00798727893995025\\
0.96	0.00798727894249382\\
0.97	0.00798727894503882\\
0.98	0.00798727894758525\\
0.99	0.00798727895013312\\
1	0.00798727895268242\\
1.01	0.00798727895523315\\
1.02	0.00798727895778533\\
1.03	0.00798727896033894\\
1.04	0.00798727896289398\\
1.05	0.00798727896545047\\
1.06	0.0079872789680084\\
1.07	0.00798727897056778\\
1.08	0.00798727897312859\\
1.09	0.00798727897569085\\
1.1	0.00798727897825456\\
1.11	0.00798727898081971\\
1.12	0.00798727898338631\\
1.13	0.00798727898595436\\
1.14	0.00798727898852386\\
1.15	0.00798727899109481\\
1.16	0.00798727899366721\\
1.17	0.00798727899624107\\
1.18	0.00798727899881638\\
1.19	0.00798727900139314\\
1.2	0.00798727900397137\\
1.21	0.00798727900655105\\
1.22	0.00798727900913219\\
1.23	0.00798727901171479\\
1.24	0.00798727901429885\\
1.25	0.00798727901688438\\
1.26	0.00798727901947137\\
1.27	0.00798727902205982\\
1.28	0.00798727902464974\\
1.29	0.00798727902724112\\
1.3	0.00798727902983398\\
1.31	0.0079872790324283\\
1.32	0.0079872790350241\\
1.33	0.00798727903762136\\
1.34	0.0079872790402201\\
1.35	0.00798727904282032\\
1.36	0.007987279045422\\
1.37	0.00798727904802517\\
1.38	0.00798727905062981\\
1.39	0.00798727905323593\\
1.4	0.00798727905584353\\
1.41	0.00798727905845262\\
1.42	0.00798727906106318\\
1.43	0.00798727906367523\\
1.44	0.00798727906628876\\
1.45	0.00798727906890377\\
1.46	0.00798727907152028\\
1.47	0.00798727907413827\\
1.48	0.00798727907675775\\
1.49	0.00798727907937872\\
1.5	0.00798727908200119\\
1.51	0.00798727908462514\\
1.52	0.00798727908725059\\
1.53	0.00798727908987753\\
1.54	0.00798727909250598\\
1.55	0.00798727909513591\\
1.56	0.00798727909776735\\
1.57	0.00798727910040029\\
1.58	0.00798727910303472\\
1.59	0.00798727910567066\\
1.6	0.0079872791083081\\
1.61	0.00798727911094705\\
1.62	0.0079872791135875\\
1.63	0.00798727911622946\\
1.64	0.00798727911887293\\
1.65	0.0079872791215179\\
1.66	0.00798727912416439\\
1.67	0.00798727912681239\\
1.68	0.0079872791294619\\
1.69	0.00798727913211292\\
1.7	0.00798727913476546\\
1.71	0.00798727913741952\\
1.72	0.00798727914007509\\
1.73	0.00798727914273219\\
1.74	0.0079872791453908\\
1.75	0.00798727914805093\\
1.76	0.00798727915071259\\
1.77	0.00798727915337577\\
1.78	0.00798727915604047\\
1.79	0.00798727915870671\\
1.8	0.00798727916137446\\
1.81	0.00798727916404375\\
1.82	0.00798727916671457\\
1.83	0.00798727916938691\\
1.84	0.00798727917206079\\
1.85	0.0079872791747362\\
1.86	0.00798727917741315\\
1.87	0.00798727918009164\\
1.88	0.00798727918277166\\
1.89	0.00798727918545321\\
1.9	0.00798727918813631\\
1.91	0.00798727919082095\\
1.92	0.00798727919350713\\
1.93	0.00798727919619485\\
1.94	0.00798727919888412\\
1.95	0.00798727920157493\\
1.96	0.00798727920426729\\
1.97	0.0079872792069612\\
1.98	0.00798727920965666\\
1.99	0.00798727921235366\\
2	0.00798727921505222\\
2.01	0.00798727921775234\\
2.02	0.007987279220454\\
2.03	0.00798727922315723\\
2.04	0.00798727922586201\\
2.05	0.00798727922856834\\
2.06	0.00798727923127624\\
2.07	0.0079872792339857\\
2.08	0.00798727923669671\\
2.09	0.00798727923940929\\
2.1	0.00798727924212344\\
2.11	0.00798727924483915\\
2.12	0.00798727924755643\\
2.13	0.00798727925027528\\
2.14	0.00798727925299569\\
2.15	0.00798727925571768\\
2.16	0.00798727925844124\\
2.17	0.00798727926116637\\
2.18	0.00798727926389307\\
2.19	0.00798727926662135\\
2.2	0.00798727926935121\\
2.21	0.00798727927208265\\
2.22	0.00798727927481566\\
2.23	0.00798727927755026\\
2.24	0.00798727928028644\\
2.25	0.0079872792830242\\
2.26	0.00798727928576354\\
2.27	0.00798727928850448\\
2.28	0.007987279291247\\
2.29	0.0079872792939911\\
2.3	0.0079872792967368\\
2.31	0.00798727929948409\\
2.32	0.00798727930223297\\
2.33	0.00798727930498344\\
2.34	0.00798727930773551\\
2.35	0.00798727931048917\\
2.36	0.00798727931324444\\
2.37	0.0079872793160013\\
2.38	0.00798727931875976\\
2.39	0.00798727932151982\\
2.4	0.00798727932428148\\
2.41	0.00798727932704475\\
2.42	0.00798727932980962\\
2.43	0.0079872793325761\\
2.44	0.00798727933534419\\
2.45	0.00798727933811389\\
2.46	0.00798727934088519\\
2.47	0.00798727934365811\\
2.48	0.00798727934643265\\
2.49	0.00798727934920879\\
2.5	0.00798727935198655\\
2.51	0.00798727935476593\\
2.52	0.00798727935754693\\
2.53	0.00798727936032955\\
2.54	0.00798727936311379\\
2.55	0.00798727936589964\\
2.56	0.00798727936868713\\
2.57	0.00798727937147624\\
2.58	0.00798727937426697\\
2.59	0.00798727937705933\\
2.6	0.00798727937985332\\
2.61	0.00798727938264895\\
2.62	0.0079872793854462\\
2.63	0.00798727938824509\\
2.64	0.0079872793910456\\
2.65	0.00798727939384776\\
2.66	0.00798727939665155\\
2.67	0.00798727939945698\\
2.68	0.00798727940226405\\
2.69	0.00798727940507276\\
2.7	0.00798727940788312\\
2.71	0.00798727941069511\\
2.72	0.00798727941350875\\
2.73	0.00798727941632404\\
2.74	0.00798727941914098\\
2.75	0.00798727942195956\\
2.76	0.0079872794247798\\
2.77	0.00798727942760168\\
2.78	0.00798727943042522\\
2.79	0.00798727943325042\\
2.8	0.00798727943607727\\
2.81	0.00798727943890577\\
2.82	0.00798727944173594\\
2.83	0.00798727944456776\\
2.84	0.00798727944740125\\
2.85	0.0079872794502364\\
2.86	0.00798727945307321\\
2.87	0.00798727945591169\\
2.88	0.00798727945875184\\
2.89	0.00798727946159365\\
2.9	0.00798727946443713\\
2.91	0.00798727946728229\\
2.92	0.00798727947012911\\
2.93	0.00798727947297761\\
2.94	0.00798727947582779\\
2.95	0.00798727947867964\\
2.96	0.00798727948153316\\
2.97	0.00798727948438837\\
2.98	0.00798727948724526\\
2.99	0.00798727949010383\\
3	0.00798727949296408\\
3.01	0.00798727949582602\\
3.02	0.00798727949868965\\
3.03	0.00798727950155496\\
3.04	0.00798727950442196\\
3.05	0.00798727950729065\\
3.06	0.00798727951016103\\
3.07	0.00798727951303311\\
3.08	0.00798727951590688\\
3.09	0.00798727951878234\\
3.1	0.00798727952165951\\
3.11	0.00798727952453837\\
3.12	0.00798727952741893\\
3.13	0.00798727953030119\\
3.14	0.00798727953318516\\
3.15	0.00798727953607083\\
3.16	0.00798727953895821\\
3.17	0.00798727954184729\\
3.18	0.00798727954473808\\
3.19	0.00798727954763059\\
3.2	0.0079872795505248\\
3.21	0.00798727955342073\\
3.22	0.00798727955631837\\
3.23	0.00798727955921772\\
3.24	0.0079872795621188\\
3.25	0.00798727956502159\\
3.26	0.00798727956792611\\
3.27	0.00798727957083234\\
3.28	0.0079872795737403\\
3.29	0.00798727957664998\\
3.3	0.00798727957956139\\
3.31	0.00798727958247452\\
3.32	0.00798727958538939\\
3.33	0.00798727958830598\\
3.34	0.00798727959122431\\
3.35	0.00798727959414436\\
3.36	0.00798727959706616\\
3.37	0.00798727959998969\\
3.38	0.00798727960291495\\
3.39	0.00798727960584196\\
3.4	0.0079872796087707\\
3.41	0.00798727961170119\\
3.42	0.00798727961463342\\
3.43	0.0079872796175674\\
3.44	0.00798727962050312\\
3.45	0.00798727962344059\\
3.46	0.00798727962637981\\
3.47	0.00798727962932078\\
3.48	0.0079872796322635\\
3.49	0.00798727963520797\\
3.5	0.0079872796381542\\
3.51	0.00798727964110219\\
3.52	0.00798727964405193\\
3.53	0.00798727964700343\\
3.54	0.0079872796499567\\
3.55	0.00798727965291172\\
3.56	0.00798727965586852\\
3.57	0.00798727965882707\\
3.58	0.00798727966178739\\
3.59	0.00798727966474948\\
3.6	0.00798727966771335\\
3.61	0.00798727967067898\\
3.62	0.00798727967364639\\
3.63	0.00798727967661556\\
3.64	0.00798727967958652\\
3.65	0.00798727968255925\\
3.66	0.00798727968553376\\
3.67	0.00798727968851006\\
3.68	0.00798727969148813\\
3.69	0.00798727969446799\\
3.7	0.00798727969744963\\
3.71	0.00798727970043306\\
3.72	0.00798727970341827\\
3.73	0.00798727970640528\\
3.74	0.00798727970939408\\
3.75	0.00798727971238467\\
3.76	0.00798727971537705\\
3.77	0.00798727971837123\\
3.78	0.0079872797213672\\
3.79	0.00798727972436498\\
3.8	0.00798727972736455\\
3.81	0.00798727973036593\\
3.82	0.00798727973336911\\
3.83	0.00798727973637409\\
3.84	0.00798727973938088\\
3.85	0.00798727974238948\\
3.86	0.00798727974539989\\
3.87	0.0079872797484121\\
3.88	0.00798727975142613\\
3.89	0.00798727975444198\\
3.9	0.00798727975745964\\
3.91	0.00798727976047912\\
3.92	0.00798727976350041\\
3.93	0.00798727976652353\\
3.94	0.00798727976954846\\
3.95	0.00798727977257522\\
3.96	0.00798727977560381\\
3.97	0.00798727977863422\\
3.98	0.00798727978166646\\
3.99	0.00798727978470053\\
4	0.00798727978773643\\
4.01	0.00798727979077416\\
4.02	0.00798727979381372\\
4.03	0.00798727979685513\\
4.04	0.00798727979989837\\
4.05	0.00798727980294344\\
4.06	0.00798727980599036\\
4.07	0.00798727980903912\\
4.08	0.00798727981208973\\
4.09	0.00798727981514218\\
4.1	0.00798727981819648\\
4.11	0.00798727982125262\\
4.12	0.00798727982431061\\
4.13	0.00798727982737046\\
4.14	0.00798727983043216\\
4.15	0.00798727983349571\\
4.16	0.00798727983656112\\
4.17	0.00798727983962839\\
4.18	0.00798727984269752\\
4.19	0.00798727984576851\\
4.2	0.00798727984884136\\
4.21	0.00798727985191608\\
4.22	0.00798727985499266\\
4.23	0.00798727985807111\\
4.24	0.00798727986115143\\
4.25	0.00798727986423362\\
4.26	0.00798727986731768\\
4.27	0.00798727987040361\\
4.28	0.00798727987349142\\
4.29	0.00798727987658111\\
4.3	0.00798727987967268\\
4.31	0.00798727988276613\\
4.32	0.00798727988586146\\
4.33	0.00798727988895867\\
4.34	0.00798727989205777\\
4.35	0.00798727989515875\\
4.36	0.00798727989826163\\
4.37	0.00798727990136639\\
4.38	0.00798727990447305\\
4.39	0.0079872799075816\\
4.4	0.00798727991069204\\
4.41	0.00798727991380438\\
4.42	0.00798727991691862\\
4.43	0.00798727992003476\\
4.44	0.0079872799231528\\
4.45	0.00798727992627275\\
4.46	0.0079872799293946\\
4.47	0.00798727993251836\\
4.48	0.00798727993564402\\
4.49	0.0079872799387716\\
4.5	0.00798727994190108\\
4.51	0.00798727994503248\\
4.52	0.0079872799481658\\
4.53	0.00798727995130103\\
4.54	0.00798727995443818\\
4.55	0.00798727995757725\\
4.56	0.00798727996071824\\
4.57	0.00798727996386116\\
4.58	0.007987279967006\\
4.59	0.00798727997015276\\
4.6	0.00798727997330146\\
4.61	0.00798727997645208\\
4.62	0.00798727997960464\\
4.63	0.00798727998275913\\
4.64	0.00798727998591555\\
4.65	0.00798727998907391\\
4.66	0.00798727999223421\\
4.67	0.00798727999539646\\
4.68	0.00798727999856064\\
4.69	0.00798728000172676\\
4.7	0.00798728000489483\\
4.71	0.00798728000806485\\
4.72	0.00798728001123682\\
4.73	0.00798728001441074\\
4.74	0.00798728001758661\\
4.75	0.00798728002076443\\
4.76	0.00798728002394421\\
4.77	0.00798728002712594\\
4.78	0.00798728003030963\\
4.79	0.00798728003349529\\
4.8	0.00798728003668291\\
4.81	0.00798728003987249\\
4.82	0.00798728004306403\\
4.83	0.00798728004625755\\
4.84	0.00798728004945303\\
4.85	0.00798728005265049\\
4.86	0.00798728005584992\\
4.87	0.00798728005905132\\
4.88	0.0079872800622547\\
4.89	0.00798728006546005\\
4.9	0.00798728006866739\\
4.91	0.00798728007187671\\
4.92	0.007987280075088\\
4.93	0.00798728007830129\\
4.94	0.00798728008151656\\
4.95	0.00798728008473382\\
4.96	0.00798728008795307\\
4.97	0.00798728009117431\\
4.98	0.00798728009439755\\
4.99	0.00798728009762278\\
5	0.00798728010085001\\
5.01	0.00798728010407924\\
5.02	0.00798728010731046\\
5.03	0.00798728011054369\\
5.04	0.00798728011377893\\
5.05	0.00798728011701617\\
5.06	0.00798728012025542\\
5.07	0.00798728012349667\\
5.08	0.00798728012673994\\
5.09	0.00798728012998523\\
5.1	0.00798728013323252\\
5.11	0.00798728013648184\\
5.12	0.00798728013973317\\
5.13	0.00798728014298652\\
5.14	0.00798728014624189\\
5.15	0.00798728014949929\\
5.16	0.00798728015275872\\
5.17	0.00798728015602017\\
5.18	0.00798728015928365\\
5.19	0.00798728016254916\\
5.2	0.0079872801658167\\
5.21	0.00798728016908628\\
5.22	0.0079872801723579\\
5.23	0.00798728017563155\\
5.24	0.00798728017890724\\
5.25	0.00798728018218498\\
5.26	0.00798728018546476\\
5.27	0.00798728018874658\\
5.28	0.00798728019203046\\
5.29	0.00798728019531638\\
5.3	0.00798728019860435\\
5.31	0.00798728020189437\\
5.32	0.00798728020518645\\
5.33	0.00798728020848059\\
5.34	0.00798728021177678\\
5.35	0.00798728021507503\\
5.36	0.00798728021837535\\
5.37	0.00798728022167773\\
5.38	0.00798728022498217\\
5.39	0.00798728022828869\\
5.4	0.00798728023159727\\
5.41	0.00798728023490792\\
5.42	0.00798728023822065\\
5.43	0.00798728024153545\\
5.44	0.00798728024485233\\
5.45	0.00798728024817128\\
5.46	0.00798728025149232\\
5.47	0.00798728025481544\\
5.48	0.00798728025814064\\
5.49	0.00798728026146793\\
5.5	0.00798728026479731\\
5.51	0.00798728026812878\\
5.52	0.00798728027146233\\
5.53	0.00798728027479798\\
5.54	0.00798728027813573\\
5.55	0.00798728028147558\\
5.56	0.00798728028481752\\
5.57	0.00798728028816156\\
5.58	0.00798728029150771\\
5.59	0.00798728029485596\\
5.6	0.00798728029820632\\
5.61	0.00798728030155879\\
5.62	0.00798728030491337\\
5.63	0.00798728030827006\\
5.64	0.00798728031162886\\
5.65	0.00798728031498978\\
5.66	0.00798728031835282\\
5.67	0.00798728032171798\\
5.68	0.00798728032508526\\
5.69	0.00798728032845467\\
5.7	0.0079872803318262\\
5.71	0.00798728033519985\\
5.72	0.00798728033857564\\
5.73	0.00798728034195356\\
5.74	0.00798728034533361\\
5.75	0.0079872803487158\\
5.76	0.00798728035210012\\
5.77	0.00798728035548659\\
5.78	0.00798728035887519\\
5.79	0.00798728036226594\\
5.8	0.00798728036565883\\
5.81	0.00798728036905387\\
5.82	0.00798728037245106\\
5.83	0.0079872803758504\\
5.84	0.00798728037925189\\
5.85	0.00798728038265554\\
5.86	0.00798728038606134\\
5.87	0.0079872803894693\\
5.88	0.00798728039287942\\
5.89	0.00798728039629171\\
5.9	0.00798728039970616\\
5.91	0.00798728040312277\\
5.92	0.00798728040654156\\
5.93	0.00798728040996251\\
5.94	0.00798728041338564\\
5.95	0.00798728041681094\\
5.96	0.00798728042023842\\
5.97	0.00798728042366807\\
5.98	0.00798728042709991\\
5.99	0.00798728043053392\\
6	0.00798728043397013\\
6.01	0.00798728043740851\\
6.02	0.00798728044084909\\
6.03	0.00798728044429186\\
6.04	0.00798728044773681\\
6.05	0.00798728045118396\\
6.06	0.00798728045463331\\
6.07	0.00798728045808486\\
6.08	0.00798728046153861\\
6.09	0.00798728046499455\\
6.1	0.00798728046845271\\
6.11	0.00798728047191306\\
6.12	0.00798728047537563\\
6.13	0.00798728047884041\\
6.14	0.0079872804823074\\
6.15	0.0079872804857766\\
6.16	0.00798728048924802\\
6.17	0.00798728049272165\\
6.18	0.00798728049619751\\
6.19	0.00798728049967559\\
6.2	0.0079872805031559\\
6.21	0.00798728050663843\\
6.22	0.00798728051012319\\
6.23	0.00798728051361017\\
6.24	0.0079872805170994\\
6.25	0.00798728052059085\\
6.26	0.00798728052408454\\
6.27	0.00798728052758048\\
6.28	0.00798728053107865\\
6.29	0.00798728053457906\\
6.3	0.00798728053808172\\
6.31	0.00798728054158663\\
6.32	0.00798728054509378\\
6.33	0.00798728054860318\\
6.34	0.00798728055211484\\
6.35	0.00798728055562875\\
6.36	0.00798728055914492\\
6.37	0.00798728056266335\\
6.38	0.00798728056618404\\
6.39	0.007987280569707\\
6.4	0.00798728057323222\\
6.41	0.0079872805767597\\
6.42	0.00798728058028946\\
6.43	0.00798728058382149\\
6.44	0.00798728058735579\\
6.45	0.00798728059089237\\
6.46	0.00798728059443122\\
6.47	0.00798728059797236\\
6.48	0.00798728060151577\\
6.49	0.00798728060506147\\
6.5	0.00798728060860946\\
6.51	0.00798728061215974\\
6.52	0.0079872806157123\\
6.53	0.00798728061926716\\
6.54	0.00798728062282431\\
6.55	0.00798728062638376\\
6.56	0.00798728062994551\\
6.57	0.00798728063350956\\
6.58	0.00798728063707592\\
6.59	0.00798728064064457\\
6.6	0.00798728064421554\\
6.61	0.00798728064778882\\
6.62	0.0079872806513644\\
6.63	0.0079872806549423\\
6.64	0.00798728065852252\\
6.65	0.00798728066210506\\
6.66	0.00798728066568991\\
6.67	0.00798728066927709\\
6.68	0.00798728067286659\\
6.69	0.00798728067645842\\
6.7	0.00798728068005258\\
6.71	0.00798728068364907\\
6.72	0.00798728068724789\\
6.73	0.00798728069084904\\
6.74	0.00798728069445254\\
6.75	0.00798728069805837\\
6.76	0.00798728070166655\\
6.77	0.00798728070527707\\
6.78	0.00798728070888994\\
6.79	0.00798728071250515\\
6.8	0.00798728071612271\\
6.81	0.00798728071974263\\
6.82	0.0079872807233649\\
6.83	0.00798728072698953\\
6.84	0.00798728073061652\\
6.85	0.00798728073424587\\
6.86	0.00798728073787758\\
6.87	0.00798728074151167\\
6.88	0.00798728074514811\\
6.89	0.00798728074878693\\
6.9	0.00798728075242812\\
6.91	0.00798728075607169\\
6.92	0.00798728075971763\\
6.93	0.00798728076336595\\
6.94	0.00798728076701665\\
6.95	0.00798728077066973\\
6.96	0.00798728077432521\\
6.97	0.00798728077798306\\
6.98	0.00798728078164331\\
6.99	0.00798728078530596\\
7	0.00798728078897099\\
7.01	0.00798728079263842\\
7.02	0.00798728079630825\\
7.03	0.00798728079998049\\
7.04	0.00798728080365512\\
7.05	0.00798728080733217\\
7.06	0.00798728081101162\\
7.07	0.00798728081469348\\
7.08	0.00798728081837775\\
7.09	0.00798728082206444\\
7.1	0.00798728082575354\\
7.11	0.00798728082944507\\
7.12	0.00798728083313902\\
7.13	0.00798728083683539\\
7.14	0.00798728084053418\\
7.15	0.00798728084423541\\
7.16	0.00798728084793906\\
7.17	0.00798728085164515\\
7.18	0.00798728085535367\\
7.19	0.00798728085906463\\
7.2	0.00798728086277803\\
7.21	0.00798728086649387\\
7.22	0.00798728087021216\\
7.23	0.00798728087393289\\
7.24	0.00798728087765608\\
7.25	0.00798728088138171\\
7.26	0.0079872808851098\\
7.27	0.00798728088884034\\
7.28	0.00798728089257334\\
7.29	0.0079872808963088\\
7.3	0.00798728090004672\\
7.31	0.00798728090378711\\
7.32	0.00798728090752997\\
7.33	0.00798728091127529\\
7.34	0.00798728091502309\\
7.35	0.00798728091877336\\
7.36	0.00798728092252611\\
7.37	0.00798728092628134\\
7.38	0.00798728093003905\\
7.39	0.00798728093379924\\
7.4	0.00798728093756192\\
7.41	0.00798728094132709\\
7.42	0.00798728094509474\\
7.43	0.0079872809488649\\
7.44	0.00798728095263754\\
7.45	0.00798728095641268\\
7.46	0.00798728096019033\\
7.47	0.00798728096397047\\
7.48	0.00798728096775313\\
7.49	0.00798728097153828\\
7.5	0.00798728097532595\\
7.51	0.00798728097911613\\
7.52	0.00798728098290882\\
7.53	0.00798728098670403\\
7.54	0.00798728099050176\\
7.55	0.00798728099430201\\
7.56	0.00798728099810479\\
7.57	0.00798728100191009\\
7.58	0.00798728100571791\\
7.59	0.00798728100952827\\
7.6	0.00798728101334117\\
7.61	0.0079872810171566\\
7.62	0.00798728102097456\\
7.63	0.00798728102479507\\
7.64	0.00798728102861812\\
7.65	0.00798728103244371\\
7.66	0.00798728103627186\\
7.67	0.00798728104010255\\
7.68	0.0079872810439358\\
7.69	0.0079872810477716\\
7.7	0.00798728105160996\\
7.71	0.00798728105545087\\
7.72	0.00798728105929435\\
7.73	0.0079872810631404\\
7.74	0.00798728106698901\\
7.75	0.00798728107084019\\
7.76	0.00798728107469394\\
7.77	0.00798728107855027\\
7.78	0.00798728108240918\\
7.79	0.00798728108627066\\
7.8	0.00798728109013472\\
7.81	0.00798728109400137\\
7.82	0.00798728109787061\\
7.83	0.00798728110174243\\
7.84	0.00798728110561685\\
7.85	0.00798728110949386\\
7.86	0.00798728111337347\\
7.87	0.00798728111725567\\
7.88	0.00798728112114048\\
7.89	0.00798728112502789\\
7.9	0.00798728112891791\\
7.91	0.00798728113281054\\
7.92	0.00798728113670578\\
7.93	0.00798728114060363\\
7.94	0.0079872811445041\\
7.95	0.00798728114840718\\
7.96	0.00798728115231289\\
7.97	0.00798728115622122\\
7.98	0.00798728116013218\\
7.99	0.00798728116404577\\
8	0.00798728116796199\\
8.01	0.00798728117188084\\
8.02	0.00798728117580232\\
8.03	0.00798728117972645\\
8.04	0.00798728118365322\\
8.05	0.00798728118758263\\
8.06	0.00798728119151469\\
8.07	0.0079872811954494\\
8.08	0.00798728119938676\\
8.09	0.00798728120332677\\
8.1	0.00798728120726944\\
8.11	0.00798728121121477\\
8.12	0.00798728121516276\\
8.13	0.00798728121911341\\
8.14	0.00798728122306674\\
8.15	0.00798728122702273\\
8.16	0.00798728123098139\\
8.17	0.00798728123494272\\
8.18	0.00798728123890674\\
8.19	0.00798728124287343\\
8.2	0.00798728124684281\\
8.21	0.00798728125081486\\
8.22	0.00798728125478961\\
8.23	0.00798728125876705\\
8.24	0.00798728126274718\\
8.25	0.00798728126673\\
8.26	0.00798728127071552\\
8.27	0.00798728127470374\\
8.28	0.00798728127869467\\
8.29	0.00798728128268829\\
8.3	0.00798728128668463\\
8.31	0.00798728129068368\\
8.32	0.00798728129468544\\
8.33	0.00798728129868991\\
8.34	0.00798728130269711\\
8.35	0.00798728130670702\\
8.36	0.00798728131071966\\
8.37	0.00798728131473503\\
8.38	0.00798728131875312\\
8.39	0.00798728132277394\\
8.4	0.0079872813267975\\
8.41	0.0079872813308238\\
8.42	0.00798728133485283\\
8.43	0.00798728133888461\\
8.44	0.00798728134291913\\
8.45	0.00798728134695639\\
8.46	0.00798728135099641\\
8.47	0.00798728135503918\\
8.48	0.00798728135908471\\
8.49	0.00798728136313299\\
8.5	0.00798728136718403\\
8.51	0.00798728137123784\\
8.52	0.00798728137529441\\
8.53	0.00798728137935375\\
8.54	0.00798728138341586\\
8.55	0.00798728138748074\\
8.56	0.0079872813915484\\
8.57	0.00798728139561884\\
8.58	0.00798728139969206\\
8.59	0.00798728140376807\\
8.6	0.00798728140784686\\
8.61	0.00798728141192845\\
8.62	0.00798728141601282\\
8.63	0.00798728142009999\\
8.64	0.00798728142418996\\
8.65	0.00798728142828273\\
8.66	0.0079872814323783\\
8.67	0.00798728143647668\\
8.68	0.00798728144057786\\
8.69	0.00798728144468186\\
8.7	0.00798728144878868\\
8.71	0.0079872814528983\\
8.72	0.00798728145701075\\
8.73	0.00798728146112602\\
8.74	0.00798728146524412\\
8.75	0.00798728146936504\\
8.76	0.0079872814734888\\
8.77	0.00798728147761538\\
8.78	0.00798728148174481\\
8.79	0.00798728148587707\\
8.8	0.00798728149001217\\
8.81	0.00798728149415012\\
8.82	0.00798728149829091\\
8.83	0.00798728150243455\\
8.84	0.00798728150658105\\
8.85	0.0079872815107304\\
8.86	0.00798728151488261\\
8.87	0.00798728151903768\\
8.88	0.00798728152319562\\
8.89	0.00798728152735641\\
8.9	0.00798728153152008\\
8.91	0.00798728153568663\\
8.92	0.00798728153985604\\
8.93	0.00798728154402833\\
8.94	0.00798728154820351\\
8.95	0.00798728155238156\\
8.96	0.0079872815565625\\
8.97	0.00798728156074633\\
8.98	0.00798728156493306\\
8.99	0.00798728156912267\\
9	0.00798728157331518\\
9.01	0.0079872815775106\\
9.02	0.00798728158170891\\
9.03	0.00798728158591013\\
9.04	0.00798728159011426\\
9.05	0.0079872815943213\\
9.06	0.00798728159853126\\
9.07	0.00798728160274413\\
9.08	0.00798728160695992\\
9.09	0.00798728161117863\\
9.1	0.00798728161540027\\
9.11	0.00798728161962483\\
9.12	0.00798728162385233\\
9.13	0.00798728162808276\\
9.14	0.00798728163231613\\
9.15	0.00798728163655244\\
9.16	0.00798728164079169\\
9.17	0.00798728164503388\\
9.18	0.00798728164927903\\
9.19	0.00798728165352712\\
9.2	0.00798728165777817\\
9.21	0.00798728166203217\\
9.22	0.00798728166628914\\
9.23	0.00798728167054906\\
9.24	0.00798728167481196\\
9.25	0.00798728167907782\\
9.26	0.00798728168334665\\
9.27	0.00798728168761846\\
9.28	0.00798728169189324\\
9.29	0.007987281696171\\
9.3	0.00798728170045175\\
9.31	0.00798728170473548\\
9.32	0.0079872817090222\\
9.33	0.00798728171331191\\
9.34	0.00798728171760462\\
9.35	0.00798728172190032\\
9.36	0.00798728172619902\\
9.37	0.00798728173050073\\
9.38	0.00798728173480544\\
9.39	0.00798728173911316\\
9.4	0.0079872817434239\\
9.41	0.00798728174773764\\
9.42	0.00798728175205441\\
9.43	0.0079872817563742\\
9.44	0.00798728176069701\\
9.45	0.00798728176502285\\
9.46	0.00798728176935172\\
9.47	0.00798728177368362\\
9.48	0.00798728177801855\\
9.49	0.00798728178235653\\
9.5	0.00798728178669755\\
9.51	0.00798728179104161\\
9.52	0.00798728179538872\\
9.53	0.00798728179973887\\
9.54	0.00798728180409209\\
9.55	0.00798728180844836\\
9.56	0.00798728181280768\\
9.57	0.00798728181717008\\
9.58	0.00798728182153553\\
9.59	0.00798728182590406\\
9.6	0.00798728183027565\\
9.61	0.00798728183465033\\
9.62	0.00798728183902807\\
9.63	0.0079872818434089\\
9.64	0.00798728184779281\\
9.65	0.00798728185217981\\
9.66	0.0079872818565699\\
9.67	0.00798728186096308\\
9.68	0.00798728186535936\\
9.69	0.00798728186975874\\
9.7	0.00798728187416121\\
9.71	0.00798728187856679\\
9.72	0.00798728188297548\\
9.73	0.00798728188738728\\
9.74	0.0079872818918022\\
9.75	0.00798728189622023\\
9.76	0.00798728190064138\\
9.77	0.00798728190506566\\
9.78	0.00798728190949306\\
9.79	0.00798728191392359\\
9.8	0.00798728191835725\\
9.81	0.00798728192279404\\
9.82	0.00798728192723398\\
9.83	0.00798728193167706\\
9.84	0.00798728193612328\\
9.85	0.00798728194057265\\
9.86	0.00798728194502517\\
9.87	0.00798728194948084\\
9.88	0.00798728195393968\\
9.89	0.00798728195840167\\
9.9	0.00798728196286682\\
9.91	0.00798728196733514\\
9.92	0.00798728197180663\\
9.93	0.00798728197628129\\
9.94	0.00798728198075913\\
9.95	0.00798728198524015\\
9.96	0.00798728198972435\\
9.97	0.00798728199421173\\
9.98	0.0079872819987023\\
9.99	0.00798728200319606\\
10	0.00798728200769302\\
10.01	0.00798728201219318\\
10.02	0.00798728201669653\\
10.03	0.00798728202120309\\
10.04	0.00798728202571286\\
10.05	0.00798728203022583\\
10.06	0.00798728203474202\\
10.07	0.00798728203926143\\
10.08	0.00798728204378405\\
10.09	0.0079872820483099\\
10.1	0.00798728205283898\\
10.11	0.00798728205737128\\
10.12	0.00798728206190681\\
10.13	0.00798728206644559\\
10.14	0.0079872820709876\\
10.15	0.00798728207553285\\
10.16	0.00798728208008134\\
10.17	0.00798728208463309\\
10.18	0.00798728208918809\\
10.19	0.00798728209374634\\
10.2	0.00798728209830785\\
10.21	0.00798728210287262\\
10.22	0.00798728210744065\\
10.23	0.00798728211201195\\
10.24	0.00798728211658652\\
10.25	0.00798728212116437\\
10.26	0.00798728212574549\\
10.27	0.0079872821303299\\
10.28	0.00798728213491759\\
10.29	0.00798728213950856\\
10.3	0.00798728214410283\\
10.31	0.00798728214870038\\
10.32	0.00798728215330124\\
10.33	0.00798728215790539\\
10.34	0.00798728216251285\\
10.35	0.00798728216712361\\
10.36	0.00798728217173768\\
10.37	0.00798728217635507\\
10.38	0.00798728218097577\\
10.39	0.0079872821855998\\
10.4	0.00798728219022714\\
10.41	0.00798728219485781\\
10.42	0.00798728219949181\\
10.43	0.00798728220412914\\
10.44	0.00798728220876981\\
10.45	0.00798728221341382\\
10.46	0.00798728221806117\\
10.47	0.00798728222271186\\
10.48	0.00798728222736591\\
10.49	0.00798728223202331\\
10.5	0.00798728223668406\\
10.51	0.00798728224134817\\
10.52	0.00798728224601564\\
10.53	0.00798728225068648\\
10.54	0.00798728225536069\\
10.55	0.00798728226003827\\
10.56	0.00798728226471923\\
10.57	0.00798728226940356\\
10.58	0.00798728227409128\\
10.59	0.00798728227878238\\
10.6	0.00798728228347687\\
10.61	0.00798728228817476\\
10.62	0.00798728229287604\\
10.63	0.00798728229758072\\
10.64	0.0079872823022888\\
10.65	0.00798728230700028\\
10.66	0.00798728231171518\\
10.67	0.00798728231643349\\
10.68	0.00798728232115521\\
10.69	0.00798728232588035\\
10.7	0.00798728233060892\\
10.71	0.00798728233534091\\
10.72	0.00798728234007633\\
10.73	0.00798728234481519\\
10.74	0.00798728234955748\\
10.75	0.0079872823543032\\
10.76	0.00798728235905238\\
10.77	0.00798728236380499\\
10.78	0.00798728236856106\\
10.79	0.00798728237332058\\
10.8	0.00798728237808356\\
10.81	0.00798728238285\\
10.82	0.0079872823876199\\
10.83	0.00798728239239327\\
10.84	0.00798728239717011\\
10.85	0.00798728240195042\\
10.86	0.00798728240673421\\
10.87	0.00798728241152148\\
10.88	0.00798728241631223\\
10.89	0.00798728242110647\\
10.9	0.00798728242590421\\
10.91	0.00798728243070543\\
10.92	0.00798728243551016\\
10.93	0.00798728244031838\\
10.94	0.00798728244513011\\
10.95	0.00798728244994535\\
10.96	0.0079872824547641\\
10.97	0.00798728245958637\\
10.98	0.00798728246441215\\
10.99	0.00798728246924146\\
11	0.00798728247407429\\
11.01	0.00798728247891065\\
11.02	0.00798728248375054\\
11.03	0.00798728248859397\\
11.04	0.00798728249344094\\
11.05	0.00798728249829145\\
11.06	0.00798728250314551\\
11.07	0.00798728250800312\\
11.08	0.00798728251286428\\
11.09	0.007987282517729\\
11.1	0.00798728252259728\\
11.11	0.00798728252746912\\
11.12	0.00798728253234453\\
11.13	0.00798728253722352\\
11.14	0.00798728254210608\\
11.15	0.00798728254699221\\
11.16	0.00798728255188193\\
11.17	0.00798728255677523\\
11.18	0.00798728256167212\\
11.19	0.00798728256657261\\
11.2	0.00798728257147669\\
11.21	0.00798728257638437\\
11.22	0.00798728258129565\\
11.23	0.00798728258621054\\
11.24	0.00798728259112904\\
11.25	0.00798728259605115\\
11.26	0.00798728260097689\\
11.27	0.00798728260590624\\
11.28	0.00798728261083922\\
11.29	0.00798728261577582\\
11.3	0.00798728262071606\\
11.31	0.00798728262565993\\
11.32	0.00798728263060744\\
11.33	0.0079872826355586\\
11.34	0.0079872826405134\\
11.35	0.00798728264547184\\
11.36	0.00798728265043395\\
11.37	0.00798728265539971\\
11.38	0.00798728266036913\\
11.39	0.00798728266534221\\
11.4	0.00798728267031897\\
11.41	0.00798728267529939\\
11.42	0.00798728268028349\\
11.43	0.00798728268527127\\
11.44	0.00798728269026273\\
11.45	0.00798728269525788\\
11.46	0.00798728270025671\\
11.47	0.00798728270525925\\
11.48	0.00798728271026547\\
11.49	0.0079872827152754\\
11.5	0.00798728272028903\\
11.51	0.00798728272530637\\
11.52	0.00798728273032742\\
11.53	0.00798728273535219\\
11.54	0.00798728274038068\\
11.55	0.00798728274541288\\
11.56	0.00798728275044882\\
11.57	0.00798728275548848\\
11.58	0.00798728276053188\\
11.59	0.00798728276557902\\
11.6	0.0079872827706299\\
11.61	0.00798728277568452\\
11.62	0.00798728278074289\\
11.63	0.00798728278580501\\
11.64	0.00798728279087089\\
11.65	0.00798728279594053\\
11.66	0.00798728280101393\\
11.67	0.0079872828060911\\
11.68	0.00798728281117204\\
11.69	0.00798728281625676\\
11.7	0.00798728282134525\\
11.71	0.00798728282643753\\
11.72	0.00798728283153359\\
11.73	0.00798728283663344\\
11.74	0.00798728284173708\\
11.75	0.00798728284684453\\
11.76	0.00798728285195577\\
11.77	0.00798728285707081\\
11.78	0.00798728286218967\\
11.79	0.00798728286731233\\
11.8	0.00798728287243882\\
11.81	0.00798728287756912\\
11.82	0.00798728288270325\\
11.83	0.0079872828878412\\
11.84	0.00798728289298298\\
11.85	0.0079872828981286\\
11.86	0.00798728290327806\\
11.87	0.00798728290843136\\
11.88	0.0079872829135885\\
11.89	0.0079872829187495\\
11.9	0.00798728292391435\\
11.91	0.00798728292908305\\
11.92	0.00798728293425562\\
11.93	0.00798728293943205\\
11.94	0.00798728294461236\\
11.95	0.00798728294979653\\
11.96	0.00798728295498458\\
11.97	0.00798728296017651\\
11.98	0.00798728296537233\\
11.99	0.00798728297057204\\
12	0.00798728297577563\\
12.01	0.00798728298098312\\
12.02	0.00798728298619452\\
12.03	0.00798728299140981\\
12.04	0.00798728299662901\\
12.05	0.00798728300185213\\
12.06	0.00798728300707916\\
12.07	0.00798728301231011\\
12.08	0.00798728301754498\\
12.09	0.00798728302278377\\
12.1	0.0079872830280265\\
12.11	0.00798728303327316\\
12.12	0.00798728303852377\\
12.13	0.00798728304377831\\
12.14	0.0079872830490368\\
12.15	0.00798728305429924\\
12.16	0.00798728305956563\\
12.17	0.00798728306483598\\
12.18	0.00798728307011029\\
12.19	0.00798728307538857\\
12.2	0.00798728308067082\\
12.21	0.00798728308595704\\
12.22	0.00798728309124723\\
12.23	0.00798728309654142\\
12.24	0.00798728310183958\\
12.25	0.00798728310714173\\
12.26	0.00798728311244788\\
12.27	0.00798728311775802\\
12.28	0.00798728312307216\\
12.29	0.00798728312839032\\
12.3	0.00798728313371247\\
12.31	0.00798728313903864\\
12.32	0.00798728314436883\\
12.33	0.00798728314970304\\
12.34	0.00798728315504127\\
12.35	0.00798728316038353\\
12.36	0.00798728316572983\\
12.37	0.00798728317108015\\
12.38	0.00798728317643452\\
12.39	0.00798728318179294\\
12.4	0.0079872831871554\\
12.41	0.00798728319252192\\
12.42	0.00798728319789249\\
12.43	0.00798728320326712\\
12.44	0.00798728320864582\\
12.45	0.00798728321402858\\
12.46	0.00798728321941542\\
12.47	0.00798728322480633\\
12.48	0.00798728323020133\\
12.49	0.00798728323560041\\
12.5	0.00798728324100357\\
12.51	0.00798728324641083\\
12.52	0.00798728325182219\\
12.53	0.00798728325723765\\
12.54	0.00798728326265721\\
12.55	0.00798728326808087\\
12.56	0.00798728327350866\\
12.57	0.00798728327894055\\
12.58	0.00798728328437657\\
12.59	0.00798728328981672\\
12.6	0.00798728329526099\\
12.61	0.00798728330070939\\
12.62	0.00798728330616194\\
12.63	0.00798728331161862\\
12.64	0.00798728331707945\\
12.65	0.00798728332254442\\
12.66	0.00798728332801355\\
12.67	0.00798728333348684\\
12.68	0.00798728333896429\\
12.69	0.0079872833444459\\
12.7	0.00798728334993168\\
12.71	0.00798728335542164\\
12.72	0.00798728336091577\\
12.73	0.00798728336641409\\
12.74	0.00798728337191659\\
12.75	0.00798728337742328\\
12.76	0.00798728338293416\\
12.77	0.00798728338844925\\
12.78	0.00798728339396853\\
12.79	0.00798728339949202\\
12.8	0.00798728340501972\\
12.81	0.00798728341055164\\
12.82	0.00798728341608777\\
12.83	0.00798728342162813\\
12.84	0.00798728342717272\\
12.85	0.00798728343272153\\
12.86	0.00798728343827458\\
12.87	0.00798728344383187\\
12.88	0.00798728344939341\\
12.89	0.00798728345495919\\
12.9	0.00798728346052922\\
12.91	0.00798728346610351\\
12.92	0.00798728347168206\\
12.93	0.00798728347726487\\
12.94	0.00798728348285195\\
12.95	0.0079872834884433\\
12.96	0.00798728349403893\\
12.97	0.00798728349963884\\
12.98	0.00798728350524304\\
12.99	0.00798728351085153\\
13	0.0079872835164643\\
13.01	0.00798728352208138\\
13.02	0.00798728352770276\\
13.03	0.00798728353332844\\
13.04	0.00798728353895843\\
13.05	0.00798728354459274\\
13.06	0.00798728355023137\\
13.07	0.00798728355587432\\
13.08	0.0079872835615216\\
13.09	0.00798728356717321\\
13.1	0.00798728357282915\\
13.11	0.00798728357848943\\
13.12	0.00798728358415406\\
13.13	0.00798728358982303\\
13.14	0.00798728359549636\\
13.15	0.00798728360117405\\
13.16	0.00798728360685609\\
13.17	0.0079872836125425\\
13.18	0.00798728361823328\\
13.19	0.00798728362392843\\
13.2	0.00798728362962796\\
13.21	0.00798728363533187\\
13.22	0.00798728364104017\\
13.23	0.00798728364675286\\
13.24	0.00798728365246994\\
13.25	0.00798728365819143\\
13.26	0.00798728366391731\\
13.27	0.00798728366964761\\
13.28	0.00798728367538231\\
13.29	0.00798728368112143\\
13.3	0.00798728368686498\\
13.31	0.00798728369261294\\
13.32	0.00798728369836534\\
13.33	0.00798728370412216\\
13.34	0.00798728370988343\\
13.35	0.00798728371564914\\
13.36	0.00798728372141929\\
13.37	0.0079872837271939\\
13.38	0.00798728373297296\\
13.39	0.00798728373875648\\
13.4	0.00798728374454446\\
13.41	0.00798728375033691\\
13.42	0.00798728375613383\\
13.43	0.00798728376193523\\
13.44	0.00798728376774111\\
13.45	0.00798728377355147\\
13.46	0.00798728377936632\\
13.47	0.00798728378518567\\
13.48	0.00798728379100951\\
13.49	0.00798728379683786\\
13.5	0.00798728380267071\\
13.51	0.00798728380850808\\
13.52	0.00798728381434996\\
13.53	0.00798728382019636\\
13.54	0.00798728382604728\\
13.55	0.00798728383190273\\
13.56	0.00798728383776272\\
13.57	0.00798728384362724\\
13.58	0.0079872838494963\\
13.59	0.00798728385536991\\
13.6	0.00798728386124807\\
13.61	0.00798728386713078\\
13.62	0.00798728387301805\\
13.63	0.00798728387890989\\
13.64	0.0079872838848063\\
13.65	0.00798728389070727\\
13.66	0.00798728389661282\\
13.67	0.00798728390252296\\
13.68	0.00798728390843768\\
13.69	0.00798728391435699\\
13.7	0.00798728392028089\\
13.71	0.00798728392620939\\
13.72	0.00798728393214249\\
13.73	0.00798728393808021\\
13.74	0.00798728394402253\\
13.75	0.00798728394996947\\
13.76	0.00798728395592103\\
13.77	0.00798728396187721\\
13.78	0.00798728396783803\\
13.79	0.00798728397380348\\
13.8	0.00798728397977356\\
13.81	0.00798728398574829\\
13.82	0.00798728399172767\\
13.83	0.0079872839977117\\
13.84	0.00798728400370038\\
13.85	0.00798728400969373\\
13.86	0.00798728401569174\\
13.87	0.00798728402169442\\
13.88	0.00798728402770177\\
13.89	0.0079872840337138\\
13.9	0.00798728403973052\\
13.91	0.00798728404575192\\
13.92	0.00798728405177801\\
13.93	0.0079872840578088\\
13.94	0.00798728406384429\\
13.95	0.00798728406988449\\
13.96	0.0079872840759294\\
13.97	0.00798728408197901\\
13.98	0.00798728408803335\\
13.99	0.00798728409409241\\
14	0.0079872841001562\\
14.01	0.00798728410622472\\
14.02	0.00798728411229798\\
14.03	0.00798728411837598\\
14.04	0.00798728412445872\\
14.05	0.00798728413054621\\
14.06	0.00798728413663846\\
14.07	0.00798728414273547\\
14.08	0.00798728414883724\\
14.09	0.00798728415494378\\
14.1	0.00798728416105509\\
14.11	0.00798728416717118\\
14.12	0.00798728417329205\\
14.13	0.00798728417941771\\
14.14	0.00798728418554815\\
14.15	0.0079872841916834\\
14.16	0.00798728419782344\\
14.17	0.00798728420396828\\
14.18	0.00798728421011794\\
14.19	0.00798728421627241\\
14.2	0.0079872842224317\\
14.21	0.0079872842285958\\
14.22	0.00798728423476474\\
14.23	0.00798728424093851\\
14.24	0.00798728424711711\\
14.25	0.00798728425330055\\
14.26	0.00798728425948884\\
14.27	0.00798728426568198\\
14.28	0.00798728427187998\\
14.29	0.00798728427808283\\
14.3	0.00798728428429054\\
14.31	0.00798728429050313\\
14.32	0.00798728429672058\\
14.33	0.00798728430294292\\
14.34	0.00798728430917013\\
14.35	0.00798728431540224\\
14.36	0.00798728432163923\\
14.37	0.00798728432788112\\
14.38	0.0079872843341279\\
14.39	0.0079872843403796\\
14.4	0.0079872843466362\\
14.41	0.00798728435289772\\
14.42	0.00798728435916415\\
14.43	0.00798728436543551\\
14.44	0.00798728437171179\\
14.45	0.00798728437799301\\
14.46	0.00798728438427916\\
14.47	0.00798728439057026\\
14.48	0.0079872843968663\\
14.49	0.0079872844031673\\
14.5	0.00798728440947324\\
14.51	0.00798728441578415\\
14.52	0.00798728442210003\\
14.53	0.00798728442842087\\
14.54	0.00798728443474668\\
14.55	0.00798728444107748\\
14.56	0.00798728444741325\\
14.57	0.00798728445375402\\
14.58	0.00798728446009978\\
14.59	0.00798728446645053\\
14.6	0.00798728447280629\\
14.61	0.00798728447916705\\
14.62	0.00798728448553282\\
14.63	0.00798728449190361\\
14.64	0.00798728449827942\\
14.65	0.00798728450466025\\
14.66	0.00798728451104612\\
14.67	0.00798728451743701\\
14.68	0.00798728452383295\\
14.69	0.00798728453023393\\
14.7	0.00798728453663996\\
14.71	0.00798728454305104\\
14.72	0.00798728454946717\\
14.73	0.00798728455588837\\
14.74	0.00798728456231464\\
14.75	0.00798728456874598\\
14.76	0.0079872845751824\\
14.77	0.00798728458162389\\
14.78	0.00798728458807047\\
14.79	0.00798728459452214\\
14.8	0.00798728460097891\\
14.81	0.00798728460744078\\
14.82	0.00798728461390774\\
14.83	0.00798728462037982\\
14.84	0.00798728462685701\\
14.85	0.00798728463333933\\
14.86	0.00798728463982676\\
14.87	0.00798728464631932\\
14.88	0.00798728465281702\\
14.89	0.00798728465931985\\
14.9	0.00798728466582782\\
14.91	0.00798728467234094\\
14.92	0.00798728467885921\\
14.93	0.00798728468538264\\
14.94	0.00798728469191122\\
14.95	0.00798728469844497\\
14.96	0.00798728470498389\\
14.97	0.00798728471152799\\
14.98	0.00798728471807727\\
14.99	0.00798728472463173\\
15	0.00798728473119138\\
15.01	0.00798728473775622\\
15.02	0.00798728474432626\\
15.03	0.00798728475090151\\
15.04	0.00798728475748196\\
15.05	0.00798728476406763\\
15.06	0.00798728477065851\\
15.07	0.00798728477725461\\
15.08	0.00798728478385594\\
15.09	0.00798728479046251\\
15.1	0.00798728479707431\\
15.11	0.00798728480369135\\
15.12	0.00798728481031363\\
15.13	0.00798728481694117\\
15.14	0.00798728482357396\\
15.15	0.00798728483021201\\
15.16	0.00798728483685533\\
15.17	0.00798728484350391\\
15.18	0.00798728485015777\\
15.19	0.00798728485681691\\
15.2	0.00798728486348133\\
15.21	0.00798728487015104\\
15.22	0.00798728487682604\\
15.23	0.00798728488350635\\
15.24	0.00798728489019195\\
15.25	0.00798728489688286\\
15.26	0.00798728490357909\\
15.27	0.00798728491028063\\
15.28	0.00798728491698749\\
15.29	0.00798728492369968\\
15.3	0.0079872849304172\\
15.31	0.00798728493714006\\
15.32	0.00798728494386825\\
15.33	0.0079872849506018\\
15.34	0.00798728495734069\\
15.35	0.00798728496408494\\
15.36	0.00798728497083455\\
15.37	0.00798728497758952\\
15.38	0.00798728498434986\\
15.39	0.00798728499111558\\
15.4	0.00798728499788668\\
15.41	0.00798728500466316\\
15.42	0.00798728501144503\\
15.43	0.00798728501823229\\
15.44	0.00798728502502495\\
15.45	0.00798728503182301\\
15.46	0.00798728503862649\\
15.47	0.00798728504543537\\
15.48	0.00798728505224967\\
15.49	0.0079872850590694\\
15.5	0.00798728506589455\\
15.51	0.00798728507272513\\
15.52	0.00798728507956115\\
15.53	0.00798728508640261\\
15.54	0.00798728509324951\\
15.55	0.00798728510010187\\
15.56	0.00798728510695968\\
15.57	0.00798728511382296\\
15.58	0.0079872851206917\\
15.59	0.00798728512756591\\
15.6	0.00798728513444559\\
15.61	0.00798728514133075\\
15.62	0.0079872851482214\\
15.63	0.00798728515511754\\
15.64	0.00798728516201917\\
15.65	0.00798728516892631\\
15.66	0.00798728517583894\\
15.67	0.00798728518275708\\
15.68	0.00798728518968074\\
15.69	0.00798728519660992\\
15.7	0.00798728520354462\\
15.71	0.00798728521048484\\
15.72	0.0079872852174306\\
15.73	0.0079872852243819\\
15.74	0.00798728523133874\\
15.75	0.00798728523830112\\
15.76	0.00798728524526906\\
15.77	0.00798728525224256\\
15.78	0.00798728525922161\\
15.79	0.00798728526620624\\
15.8	0.00798728527319643\\
15.81	0.0079872852801922\\
15.82	0.00798728528719355\\
15.83	0.00798728529420049\\
15.84	0.00798728530121302\\
15.85	0.00798728530823114\\
15.86	0.00798728531525486\\
15.87	0.00798728532228419\\
15.88	0.00798728532931913\\
15.89	0.00798728533635969\\
15.9	0.00798728534340586\\
15.91	0.00798728535045765\\
15.92	0.00798728535751508\\
15.93	0.00798728536457814\\
15.94	0.00798728537164684\\
15.95	0.00798728537872118\\
15.96	0.00798728538580117\\
15.97	0.00798728539288682\\
15.98	0.00798728539997812\\
15.99	0.00798728540707509\\
16	0.00798728541417772\\
16.01	0.00798728542128603\\
16.02	0.00798728542840001\\
16.03	0.00798728543551968\\
16.04	0.00798728544264503\\
16.05	0.00798728544977608\\
16.06	0.00798728545691282\\
16.07	0.00798728546405526\\
16.08	0.00798728547120341\\
16.09	0.00798728547835728\\
16.1	0.00798728548551685\\
16.11	0.00798728549268215\\
16.12	0.00798728549985318\\
16.13	0.00798728550702993\\
16.14	0.00798728551421243\\
16.15	0.00798728552140066\\
16.16	0.00798728552859463\\
16.17	0.00798728553579436\\
16.18	0.00798728554299984\\
16.19	0.00798728555021109\\
16.2	0.00798728555742809\\
16.21	0.00798728556465087\\
16.22	0.00798728557187942\\
16.23	0.00798728557911375\\
16.24	0.00798728558635386\\
16.25	0.00798728559359976\\
16.26	0.00798728560085146\\
16.27	0.00798728560810895\\
16.28	0.00798728561537225\\
16.29	0.00798728562264135\\
16.3	0.00798728562991627\\
16.31	0.007987285637197\\
16.32	0.00798728564448356\\
16.33	0.00798728565177595\\
16.34	0.00798728565907417\\
16.35	0.00798728566637822\\
16.36	0.00798728567368812\\
16.37	0.00798728568100386\\
16.38	0.00798728568832546\\
16.39	0.00798728569565291\\
16.4	0.00798728570298622\\
16.41	0.0079872857103254\\
16.42	0.00798728571767045\\
16.43	0.00798728572502138\\
16.44	0.00798728573237818\\
16.45	0.00798728573974088\\
16.46	0.00798728574710946\\
16.47	0.00798728575448394\\
16.48	0.00798728576186431\\
16.49	0.0079872857692506\\
16.5	0.00798728577664279\\
16.51	0.00798728578404089\\
16.52	0.00798728579144492\\
16.53	0.00798728579885487\\
16.54	0.00798728580627075\\
16.55	0.00798728581369256\\
16.56	0.00798728582112031\\
16.57	0.00798728582855401\\
16.58	0.00798728583599365\\
16.59	0.00798728584343924\\
16.6	0.0079872858508908\\
16.61	0.00798728585834831\\
16.62	0.0079872858658118\\
16.63	0.00798728587328125\\
16.64	0.00798728588075668\\
16.65	0.0079872858882381\\
16.66	0.0079872858957255\\
16.67	0.00798728590321889\\
16.68	0.00798728591071828\\
16.69	0.00798728591822367\\
16.7	0.00798728592573507\\
16.71	0.00798728593325248\\
16.72	0.0079872859407759\\
16.73	0.00798728594830535\\
16.74	0.00798728595584081\\
16.75	0.00798728596338231\\
16.76	0.00798728597092985\\
16.77	0.00798728597848342\\
16.78	0.00798728598604304\\
16.79	0.00798728599360871\\
16.8	0.00798728600118043\\
16.81	0.00798728600875821\\
16.82	0.00798728601634206\\
16.83	0.00798728602393197\\
16.84	0.00798728603152795\\
16.85	0.00798728603913002\\
16.86	0.00798728604673817\\
16.87	0.0079872860543524\\
16.88	0.00798728606197273\\
16.89	0.00798728606959915\\
16.9	0.00798728607723168\\
16.91	0.00798728608487031\\
16.92	0.00798728609251506\\
16.93	0.00798728610016592\\
16.94	0.0079872861078229\\
16.95	0.00798728611548601\\
16.96	0.00798728612315525\\
16.97	0.00798728613083063\\
16.98	0.00798728613851215\\
16.99	0.00798728614619981\\
17	0.00798728615389362\\
17.01	0.00798728616159358\\
17.02	0.00798728616929971\\
17.03	0.007987286177012\\
17.04	0.00798728618473046\\
17.05	0.00798728619245509\\
17.06	0.0079872862001859\\
17.07	0.00798728620792289\\
17.08	0.00798728621566607\\
17.09	0.00798728622341545\\
17.1	0.00798728623117102\\
17.11	0.0079872862389328\\
17.12	0.00798728624670078\\
17.13	0.00798728625447497\\
17.14	0.00798728626225538\\
17.15	0.00798728627004201\\
17.16	0.00798728627783487\\
17.17	0.00798728628563396\\
17.18	0.00798728629343929\\
17.19	0.00798728630125085\\
17.2	0.00798728630906866\\
17.21	0.00798728631689272\\
17.22	0.00798728632472303\\
17.23	0.00798728633255961\\
17.24	0.00798728634040245\\
17.25	0.00798728634825155\\
17.26	0.00798728635610693\\
17.27	0.00798728636396859\\
17.28	0.00798728637183653\\
17.29	0.00798728637971076\\
17.3	0.00798728638759128\\
17.31	0.00798728639547809\\
17.32	0.00798728640337121\\
17.33	0.00798728641127064\\
17.34	0.00798728641917638\\
17.35	0.00798728642708843\\
17.36	0.0079872864350068\\
17.37	0.0079872864429315\\
17.38	0.00798728645086253\\
17.39	0.0079872864587999\\
17.4	0.0079872864667436\\
17.41	0.00798728647469365\\
17.42	0.00798728648265004\\
17.43	0.00798728649061279\\
17.44	0.0079872864985819\\
17.45	0.00798728650655737\\
17.46	0.00798728651453921\\
17.47	0.00798728652252742\\
17.48	0.007987286530522\\
17.49	0.00798728653852297\\
17.5	0.00798728654653033\\
17.51	0.00798728655454407\\
17.52	0.00798728656256421\\
17.53	0.00798728657059075\\
17.54	0.0079872865786237\\
17.55	0.00798728658666305\\
17.56	0.00798728659470882\\
17.57	0.00798728660276101\\
17.58	0.00798728661081962\\
17.59	0.00798728661888466\\
17.6	0.00798728662695614\\
17.61	0.00798728663503405\\
17.62	0.0079872866431184\\
17.63	0.00798728665120919\\
17.64	0.00798728665930644\\
17.65	0.00798728666741015\\
17.66	0.00798728667552031\\
17.67	0.00798728668363695\\
17.68	0.00798728669176004\\
17.69	0.00798728669988962\\
17.7	0.00798728670802567\\
17.71	0.00798728671616821\\
17.72	0.00798728672431723\\
17.73	0.00798728673247275\\
17.74	0.00798728674063477\\
17.75	0.00798728674880328\\
17.76	0.00798728675697831\\
17.77	0.00798728676515984\\
17.78	0.00798728677334789\\
17.79	0.00798728678154246\\
17.8	0.00798728678974355\\
17.81	0.00798728679795118\\
17.82	0.00798728680616534\\
17.83	0.00798728681438603\\
17.84	0.00798728682261327\\
17.85	0.00798728683084706\\
17.86	0.00798728683908739\\
17.87	0.00798728684733429\\
17.88	0.00798728685558775\\
17.89	0.00798728686384777\\
17.9	0.00798728687211436\\
17.91	0.00798728688038753\\
17.92	0.00798728688866727\\
17.93	0.0079872868969536\\
17.94	0.00798728690524652\\
17.95	0.00798728691354603\\
17.96	0.00798728692185214\\
17.97	0.00798728693016485\\
17.98	0.00798728693848417\\
17.99	0.00798728694681009\\
18	0.00798728695514263\\
18.01	0.0079872869634818\\
18.02	0.00798728697182758\\
18.03	0.00798728698018\\
18.04	0.00798728698853904\\
18.05	0.00798728699690473\\
18.06	0.00798728700527706\\
18.07	0.00798728701365603\\
18.08	0.00798728702204166\\
18.09	0.00798728703043394\\
18.1	0.00798728703883288\\
18.11	0.00798728704723848\\
18.12	0.00798728705565076\\
18.13	0.00798728706406971\\
18.14	0.00798728707249533\\
18.15	0.00798728708092764\\
18.16	0.00798728708936663\\
18.17	0.00798728709781231\\
18.18	0.00798728710626469\\
18.19	0.00798728711472377\\
18.2	0.00798728712318955\\
18.21	0.00798728713166205\\
18.22	0.00798728714014125\\
18.23	0.00798728714862717\\
18.24	0.00798728715711982\\
18.25	0.00798728716561919\\
18.26	0.00798728717412529\\
18.27	0.00798728718263812\\
18.28	0.00798728719115769\\
18.29	0.00798728719968401\\
18.3	0.00798728720821707\\
18.31	0.00798728721675689\\
18.32	0.00798728722530346\\
18.33	0.0079872872338568\\
18.34	0.0079872872424169\\
18.35	0.00798728725098376\\
18.36	0.00798728725955741\\
18.37	0.00798728726813783\\
18.38	0.00798728727672503\\
18.39	0.00798728728531902\\
18.4	0.0079872872939198\\
18.41	0.00798728730252737\\
18.42	0.00798728731114175\\
18.43	0.00798728731976292\\
18.44	0.00798728732839091\\
18.45	0.00798728733702571\\
18.46	0.00798728734566732\\
18.47	0.00798728735431576\\
18.48	0.00798728736297102\\
18.49	0.00798728737163311\\
18.5	0.00798728738030203\\
18.51	0.00798728738897779\\
18.52	0.00798728739766039\\
18.53	0.00798728740634984\\
18.54	0.00798728741504614\\
18.55	0.00798728742374929\\
18.56	0.0079872874324593\\
18.57	0.00798728744117618\\
18.58	0.00798728744989992\\
18.59	0.00798728745863053\\
18.6	0.00798728746736802\\
18.61	0.00798728747611239\\
18.62	0.00798728748486364\\
18.63	0.00798728749362177\\
18.64	0.0079872875023868\\
18.65	0.00798728751115873\\
18.66	0.00798728751993756\\
18.67	0.00798728752872329\\
18.68	0.00798728753751593\\
18.69	0.00798728754631548\\
18.7	0.00798728755512195\\
18.71	0.00798728756393534\\
18.72	0.00798728757275565\\
18.73	0.00798728758158289\\
18.74	0.00798728759041707\\
18.75	0.00798728759925818\\
18.76	0.00798728760810623\\
18.77	0.00798728761696122\\
18.78	0.00798728762582317\\
18.79	0.00798728763469207\\
18.8	0.00798728764356793\\
18.81	0.00798728765245074\\
18.82	0.00798728766134053\\
18.83	0.00798728767023728\\
18.84	0.007987287679141\\
18.85	0.0079872876880517\\
18.86	0.00798728769696938\\
18.87	0.00798728770589405\\
18.88	0.0079872877148257\\
18.89	0.00798728772376435\\
18.9	0.00798728773271\\
18.91	0.00798728774166264\\
18.92	0.00798728775062229\\
18.93	0.00798728775958895\\
18.94	0.00798728776856262\\
18.95	0.00798728777754331\\
18.96	0.00798728778653102\\
18.97	0.00798728779552575\\
18.98	0.0079872878045275\\
18.99	0.00798728781353629\\
19	0.00798728782255212\\
19.01	0.00798728783157499\\
19.02	0.00798728784060489\\
19.03	0.00798728784964185\\
19.04	0.00798728785868585\\
19.05	0.00798728786773692\\
19.06	0.00798728787679504\\
19.07	0.00798728788586021\\
19.08	0.00798728789493246\\
19.09	0.00798728790401178\\
19.1	0.00798728791309817\\
19.11	0.00798728792219163\\
19.12	0.00798728793129218\\
19.13	0.00798728794039981\\
19.14	0.00798728794951454\\
19.15	0.00798728795863635\\
19.16	0.00798728796776526\\
19.17	0.00798728797690126\\
19.18	0.00798728798604437\\
19.19	0.00798728799519459\\
19.2	0.00798728800435192\\
19.21	0.00798728801351636\\
19.22	0.00798728802268792\\
19.23	0.0079872880318666\\
19.24	0.0079872880410524\\
19.25	0.00798728805024533\\
19.26	0.0079872880594454\\
19.27	0.0079872880686526\\
19.28	0.00798728807786694\\
19.29	0.00798728808708842\\
19.3	0.00798728809631705\\
19.31	0.00798728810555282\\
19.32	0.00798728811479575\\
19.33	0.00798728812404584\\
19.34	0.00798728813330309\\
19.35	0.0079872881425675\\
19.36	0.00798728815183908\\
19.37	0.00798728816111782\\
19.38	0.00798728817040375\\
19.39	0.00798728817969684\\
19.4	0.00798728818899713\\
19.41	0.00798728819830459\\
19.42	0.00798728820761924\\
19.43	0.00798728821694108\\
19.44	0.00798728822627012\\
19.45	0.00798728823560635\\
19.46	0.00798728824494979\\
19.47	0.00798728825430043\\
19.48	0.00798728826365827\\
19.49	0.00798728827302333\\
19.5	0.0079872882823956\\
19.51	0.00798728829177509\\
19.52	0.0079872883011618\\
19.53	0.00798728831055573\\
19.54	0.00798728831995689\\
19.55	0.00798728832936528\\
19.56	0.0079872883387809\\
19.57	0.00798728834820376\\
19.58	0.00798728835763386\\
19.59	0.0079872883670712\\
19.6	0.00798728837651579\\
19.61	0.00798728838596763\\
19.62	0.00798728839542672\\
19.63	0.00798728840489306\\
19.64	0.00798728841436667\\
19.65	0.00798728842384753\\
19.66	0.00798728843333566\\
19.67	0.00798728844283106\\
19.68	0.00798728845233372\\
19.69	0.00798728846184366\\
19.7	0.00798728847136088\\
19.71	0.00798728848088537\\
19.72	0.00798728849041715\\
19.73	0.00798728849995621\\
19.74	0.00798728850950256\\
19.75	0.0079872885190562\\
19.76	0.00798728852861714\\
19.77	0.00798728853818537\\
19.78	0.0079872885477609\\
19.79	0.00798728855734374\\
19.8	0.00798728856693388\\
19.81	0.00798728857653132\\
19.82	0.00798728858613608\\
19.83	0.00798728859574815\\
19.84	0.00798728860536754\\
19.85	0.00798728861499425\\
19.86	0.00798728862462827\\
19.87	0.00798728863426963\\
19.88	0.00798728864391831\\
19.89	0.00798728865357431\\
19.9	0.00798728866323765\\
19.91	0.00798728867290833\\
19.92	0.00798728868258634\\
19.93	0.00798728869227169\\
19.94	0.00798728870196438\\
19.95	0.00798728871166442\\
19.96	0.00798728872137181\\
19.97	0.00798728873108655\\
19.98	0.00798728874080863\\
19.99	0.00798728875053807\\
20	0.00798728876027487\\
20.01	0.00798728877001903\\
20.02	0.00798728877977055\\
20.03	0.00798728878952944\\
20.04	0.00798728879929569\\
20.05	0.00798728880906931\\
20.06	0.0079872888188503\\
20.07	0.00798728882863866\\
20.08	0.0079872888384344\\
20.09	0.00798728884823751\\
20.1	0.00798728885804801\\
20.11	0.00798728886786589\\
20.12	0.00798728887769115\\
20.13	0.0079872888875238\\
20.14	0.00798728889736383\\
20.15	0.00798728890721126\\
20.16	0.00798728891706608\\
20.17	0.00798728892692829\\
20.18	0.0079872889367979\\
20.19	0.00798728894667491\\
20.2	0.00798728895655931\\
20.21	0.00798728896645112\\
20.22	0.00798728897635033\\
20.23	0.00798728898625695\\
20.24	0.00798728899617098\\
20.25	0.00798728900609241\\
20.26	0.00798728901602126\\
20.27	0.00798728902595752\\
20.28	0.00798728903590119\\
20.29	0.00798728904585228\\
20.3	0.00798728905581079\\
20.31	0.00798728906577672\\
20.32	0.00798728907575006\\
20.33	0.00798728908573083\\
20.34	0.00798728909571903\\
20.35	0.00798728910571465\\
20.36	0.0079872891157177\\
20.37	0.00798728912572817\\
20.38	0.00798728913574608\\
20.39	0.00798728914577141\\
20.4	0.00798728915580418\\
20.41	0.00798728916584439\\
20.42	0.00798728917589202\\
20.43	0.0079872891859471\\
20.44	0.00798728919600961\\
20.45	0.00798728920607956\\
20.46	0.00798728921615696\\
20.47	0.00798728922624179\\
20.48	0.00798728923633407\\
20.49	0.00798728924643378\\
20.5	0.00798728925654095\\
20.51	0.00798728926665556\\
20.52	0.00798728927677761\\
20.53	0.00798728928690711\\
20.54	0.00798728929704406\\
20.55	0.00798728930718846\\
20.56	0.00798728931734031\\
20.57	0.00798728932749961\\
20.58	0.00798728933766636\\
20.59	0.00798728934784056\\
20.6	0.00798728935802222\\
20.61	0.00798728936821133\\
20.62	0.00798728937840789\\
20.63	0.0079872893886119\\
20.64	0.00798728939882338\\
20.65	0.0079872894090423\\
20.66	0.00798728941926868\\
20.67	0.00798728942950252\\
20.68	0.00798728943974382\\
20.69	0.00798728944999257\\
20.7	0.00798728946024878\\
20.71	0.00798728947051244\\
20.72	0.00798728948078357\\
20.73	0.00798728949106215\\
20.74	0.00798728950134818\\
20.75	0.00798728951164168\\
20.76	0.00798728952194263\\
20.77	0.00798728953225104\\
20.78	0.00798728954256691\\
20.79	0.00798728955289024\\
20.8	0.00798728956322102\\
20.81	0.00798728957355926\\
20.82	0.00798728958390496\\
20.83	0.00798728959425811\\
20.84	0.00798728960461872\\
20.85	0.00798728961498679\\
20.86	0.00798728962536231\\
20.87	0.00798728963574528\\
20.88	0.00798728964613571\\
20.89	0.0079872896565336\\
20.9	0.00798728966693893\\
20.91	0.00798728967735172\\
20.92	0.00798728968777196\\
20.93	0.00798728969819965\\
20.94	0.00798728970863479\\
20.95	0.00798728971907738\\
20.96	0.00798728972952742\\
20.97	0.0079872897399849\\
20.98	0.00798728975044983\\
20.99	0.0079872897609222\\
21	0.00798728977140202\\
21.01	0.00798728978188928\\
21.02	0.00798728979238398\\
21.03	0.00798728980288612\\
21.04	0.0079872898133957\\
21.05	0.00798728982391272\\
21.06	0.00798728983443717\\
21.07	0.00798728984496906\\
21.08	0.00798728985550838\\
21.09	0.00798728986605513\\
21.1	0.0079872898766093\\
21.11	0.00798728988717091\\
21.12	0.00798728989773994\\
21.13	0.00798728990831639\\
21.14	0.00798728991890026\\
21.15	0.00798728992949155\\
21.16	0.00798728994009026\\
21.17	0.00798728995069639\\
21.18	0.00798728996130993\\
21.19	0.00798728997193088\\
21.2	0.00798728998255923\\
21.21	0.00798728999319499\\
21.22	0.00798729000383816\\
21.23	0.00798729001448872\\
21.24	0.00798729002514669\\
21.25	0.00798729003581205\\
21.26	0.0079872900464848\\
21.27	0.00798729005716494\\
21.28	0.00798729006785247\\
21.29	0.00798729007854738\\
21.3	0.00798729008924967\\
21.31	0.00798729009995934\\
21.32	0.00798729011067638\\
21.33	0.0079872901214008\\
21.34	0.00798729013213258\\
21.35	0.00798729014287173\\
21.36	0.00798729015361824\\
21.37	0.00798729016437211\\
21.38	0.00798729017513333\\
21.39	0.0079872901859019\\
21.4	0.00798729019667781\\
21.41	0.00798729020746107\\
21.42	0.00798729021825167\\
21.43	0.0079872902290496\\
21.44	0.00798729023985487\\
21.45	0.00798729025066746\\
21.46	0.00798729026148737\\
21.47	0.0079872902723146\\
21.48	0.00798729028314915\\
21.49	0.007987290293991\\
21.5	0.00798729030484016\\
21.51	0.00798729031569661\\
21.52	0.00798729032656037\\
21.53	0.00798729033743141\\
21.54	0.00798729034830974\\
21.55	0.00798729035919535\\
21.56	0.00798729037008823\\
21.57	0.00798729038098838\\
21.58	0.00798729039189581\\
21.59	0.00798729040281048\\
21.6	0.00798729041373242\\
21.61	0.0079872904246616\\
21.62	0.00798729043559802\\
21.63	0.00798729044654169\\
21.64	0.00798729045749258\\
21.65	0.0079872904684507\\
21.66	0.00798729047941604\\
21.67	0.0079872904903886\\
21.68	0.00798729050136836\\
21.69	0.00798729051235532\\
21.7	0.00798729052334948\\
21.71	0.00798729053435083\\
21.72	0.00798729054535936\\
21.73	0.00798729055637506\\
21.74	0.00798729056739794\\
21.75	0.00798729057842797\\
21.76	0.00798729058946517\\
21.77	0.0079872906005095\\
21.78	0.00798729061156098\\
21.79	0.0079872906226196\\
21.8	0.00798729063368534\\
21.81	0.0079872906447582\\
21.82	0.00798729065583817\\
21.83	0.00798729066692525\\
21.84	0.00798729067801942\\
21.85	0.00798729068912068\\
21.86	0.00798729070022902\\
21.87	0.00798729071134443\\
21.88	0.00798729072246691\\
21.89	0.00798729073359645\\
21.9	0.00798729074473303\\
21.91	0.00798729075587665\\
21.92	0.0079872907670273\\
21.93	0.00798729077818497\\
21.94	0.00798729078934966\\
21.95	0.00798729080052135\\
21.96	0.00798729081170003\\
21.97	0.00798729082288571\\
21.98	0.00798729083407836\\
21.99	0.00798729084527797\\
22	0.00798729085648455\\
22.01	0.00798729086769807\\
22.02	0.00798729087891853\\
22.03	0.00798729089014592\\
22.04	0.00798729090138023\\
22.05	0.00798729091262145\\
22.06	0.00798729092386957\\
22.07	0.00798729093512458\\
22.08	0.00798729094638646\\
22.09	0.00798729095765521\\
22.1	0.00798729096893082\\
22.11	0.00798729098021328\\
22.12	0.00798729099150257\\
22.13	0.00798729100279869\\
22.14	0.00798729101410161\\
22.15	0.00798729102541134\\
22.16	0.00798729103672787\\
22.17	0.00798729104805117\\
22.18	0.00798729105938123\\
22.19	0.00798729107071806\\
22.2	0.00798729108206163\\
22.21	0.00798729109341193\\
22.22	0.00798729110476895\\
22.23	0.00798729111613268\\
22.24	0.0079872911275031\\
22.25	0.0079872911388802\\
22.26	0.00798729115026398\\
22.27	0.00798729116165441\\
22.28	0.00798729117305149\\
22.29	0.0079872911844552\\
22.3	0.00798729119586553\\
22.31	0.00798729120728246\\
22.32	0.00798729121870599\\
22.33	0.00798729123013609\\
22.34	0.00798729124157276\\
22.35	0.00798729125301598\\
22.36	0.00798729126446573\\
22.37	0.00798729127592201\\
22.38	0.0079872912873848\\
22.39	0.00798729129885408\\
22.4	0.00798729131032985\\
22.41	0.00798729132181207\\
22.42	0.00798729133330075\\
22.43	0.00798729134479587\\
22.44	0.0079872913562974\\
22.45	0.00798729136780535\\
22.46	0.00798729137931968\\
22.47	0.00798729139084038\\
22.48	0.00798729140236745\\
22.49	0.00798729141390087\\
22.5	0.00798729142544061\\
22.51	0.00798729143698666\\
22.52	0.00798729144853901\\
22.53	0.00798729146009764\\
22.54	0.00798729147166253\\
22.55	0.00798729148323367\\
22.56	0.00798729149481105\\
22.57	0.00798729150639464\\
22.58	0.00798729151798442\\
22.59	0.00798729152958039\\
22.6	0.00798729154118251\\
22.61	0.00798729155279079\\
22.62	0.00798729156440519\\
22.63	0.00798729157602571\\
22.64	0.00798729158765232\\
22.65	0.00798729159928501\\
22.66	0.00798729161092375\\
22.67	0.00798729162256854\\
22.68	0.00798729163421934\\
22.69	0.00798729164587616\\
22.7	0.00798729165753895\\
22.71	0.00798729166920772\\
22.72	0.00798729168088243\\
22.73	0.00798729169256307\\
22.74	0.00798729170424962\\
22.75	0.00798729171594206\\
22.76	0.00798729172764038\\
22.77	0.00798729173934455\\
22.78	0.00798729175105455\\
22.79	0.00798729176277037\\
22.8	0.00798729177449198\\
22.81	0.00798729178621937\\
22.82	0.00798729179795251\\
22.83	0.00798729180969138\\
22.84	0.00798729182143597\\
22.85	0.00798729183318626\\
22.86	0.00798729184494222\\
22.87	0.00798729185670383\\
22.88	0.00798729186847107\\
22.89	0.00798729188024393\\
22.9	0.00798729189202238\\
22.91	0.00798729190380639\\
22.92	0.00798729191559596\\
22.93	0.00798729192739106\\
22.94	0.00798729193919166\\
22.95	0.00798729195099774\\
22.96	0.00798729196280929\\
22.97	0.00798729197462629\\
22.98	0.0079872919864487\\
22.99	0.00798729199827651\\
23	0.0079872920101097\\
23.01	0.00798729202194824\\
23.02	0.00798729203379211\\
23.03	0.0079872920456413\\
23.04	0.00798729205749577\\
23.05	0.00798729206935551\\
23.06	0.00798729208122049\\
23.07	0.00798729209309068\\
23.08	0.00798729210496608\\
23.09	0.00798729211684666\\
23.1	0.00798729212873238\\
23.11	0.00798729214062323\\
23.12	0.00798729215251919\\
23.13	0.00798729216442023\\
23.14	0.00798729217632634\\
23.15	0.00798729218823748\\
23.16	0.00798729220015363\\
23.17	0.00798729221207477\\
23.18	0.00798729222400088\\
23.19	0.00798729223593193\\
23.2	0.00798729224786789\\
23.21	0.00798729225980876\\
23.22	0.0079872922717545\\
23.23	0.00798729228370508\\
23.24	0.00798729229566049\\
23.25	0.0079872923076207\\
23.26	0.00798729231958568\\
23.27	0.00798729233155542\\
23.28	0.00798729234352989\\
23.29	0.00798729235550906\\
23.3	0.00798729236749292\\
23.31	0.00798729237948143\\
23.32	0.00798729239147457\\
23.33	0.00798729240347232\\
23.34	0.00798729241547465\\
23.35	0.00798729242748155\\
23.36	0.00798729243949298\\
23.37	0.00798729245150893\\
23.38	0.00798729246352936\\
23.39	0.00798729247555425\\
23.4	0.00798729248758359\\
23.41	0.00798729249961734\\
23.42	0.00798729251165548\\
23.43	0.00798729252369799\\
23.44	0.00798729253574484\\
23.45	0.00798729254779602\\
23.46	0.00798729255985148\\
23.47	0.00798729257191122\\
23.48	0.00798729258397521\\
23.49	0.00798729259604342\\
23.5	0.00798729260811582\\
23.51	0.00798729262019241\\
23.52	0.00798729263227314\\
23.53	0.00798729264435801\\
23.54	0.00798729265644698\\
23.55	0.00798729266854003\\
23.56	0.00798729268063713\\
23.57	0.00798729269273828\\
23.58	0.00798729270484343\\
23.59	0.00798729271695257\\
23.6	0.00798729272906567\\
23.61	0.00798729274118272\\
23.62	0.00798729275330369\\
23.63	0.00798729276542855\\
23.64	0.00798729277755728\\
23.65	0.00798729278968987\\
23.66	0.00798729280182628\\
23.67	0.00798729281396651\\
23.68	0.00798729282611051\\
23.69	0.00798729283825828\\
23.7	0.00798729285040978\\
23.71	0.00798729286256501\\
23.72	0.00798729287472393\\
23.73	0.00798729288688653\\
23.74	0.00798729289905279\\
23.75	0.00798729291122267\\
23.76	0.00798729292339617\\
23.77	0.00798729293557327\\
23.78	0.00798729294775393\\
23.79	0.00798729295993815\\
23.8	0.0079872929721259\\
23.81	0.00798729298431716\\
23.82	0.00798729299651192\\
23.83	0.00798729300871015\\
23.84	0.00798729302091184\\
23.85	0.00798729303311696\\
23.86	0.0079872930453255\\
23.87	0.00798729305753744\\
23.88	0.00798729306975276\\
23.89	0.00798729308197145\\
23.9	0.00798729309419349\\
23.91	0.00798729310641885\\
23.92	0.00798729311864753\\
23.93	0.00798729313087951\\
23.94	0.00798729314311476\\
23.95	0.00798729315535329\\
23.96	0.00798729316759506\\
23.97	0.00798729317984007\\
23.98	0.0079872931920883\\
23.99	0.00798729320433974\\
24	0.00798729321659436\\
24.01	0.00798729322885217\\
24.02	0.00798729324111314\\
24.03	0.00798729325337727\\
24.04	0.00798729326564454\\
24.05	0.00798729327791493\\
24.06	0.00798729329018845\\
24.07	0.00798729330246506\\
24.08	0.00798729331474478\\
24.09	0.00798729332702758\\
24.1	0.00798729333931345\\
24.11	0.0079872933516024\\
24.12	0.00798729336389439\\
24.13	0.00798729337618944\\
24.14	0.00798729338848753\\
24.15	0.00798729340078865\\
24.16	0.0079872934130928\\
24.17	0.00798729342539997\\
24.18	0.00798729343771016\\
24.19	0.00798729345002335\\
24.2	0.00798729346233955\\
24.21	0.00798729347465875\\
24.22	0.00798729348698094\\
24.23	0.00798729349930613\\
24.24	0.0079872935116343\\
24.25	0.00798729352396547\\
24.26	0.00798729353629962\\
24.27	0.00798729354863676\\
24.28	0.00798729356097688\\
24.29	0.00798729357331999\\
24.3	0.00798729358566608\\
24.31	0.00798729359801517\\
24.32	0.00798729361036724\\
24.33	0.0079872936227223\\
24.34	0.00798729363508037\\
24.35	0.00798729364744143\\
24.36	0.00798729365980549\\
24.37	0.00798729367217257\\
24.38	0.00798729368454266\\
24.39	0.00798729369691577\\
24.4	0.00798729370929191\\
24.41	0.00798729372167109\\
24.42	0.00798729373405331\\
24.43	0.00798729374643859\\
24.44	0.00798729375882693\\
24.45	0.00798729377121835\\
24.46	0.00798729378361285\\
24.47	0.00798729379601044\\
24.48	0.00798729380841115\\
24.49	0.00798729382081499\\
24.5	0.00798729383322196\\
24.51	0.00798729384563208\\
24.52	0.00798729385804536\\
24.53	0.00798729387046183\\
24.54	0.0079872938828815\\
24.55	0.00798729389530439\\
24.56	0.00798729390773051\\
24.57	0.00798729392015988\\
24.58	0.00798729393259253\\
24.59	0.00798729394502847\\
24.6	0.00798729395746772\\
24.61	0.0079872939699103\\
24.62	0.00798729398235625\\
24.63	0.00798729399480557\\
24.64	0.00798729400725829\\
24.65	0.00798729401971444\\
24.66	0.00798729403217404\\
24.67	0.00798729404463712\\
24.68	0.0079872940571037\\
24.69	0.00798729406957381\\
24.7	0.00798729408204747\\
24.71	0.00798729409452472\\
24.72	0.00798729410700558\\
24.73	0.00798729411949008\\
24.74	0.00798729413197824\\
24.75	0.00798729414447011\\
24.76	0.00798729415696571\\
24.77	0.00798729416946507\\
24.78	0.00798729418196822\\
24.79	0.00798729419447519\\
24.8	0.00798729420698602\\
24.81	0.00798729421950074\\
24.82	0.00798729423201939\\
24.83	0.00798729424454199\\
24.84	0.00798729425706857\\
24.85	0.00798729426959919\\
24.86	0.00798729428213386\\
24.87	0.00798729429467262\\
24.88	0.00798729430721551\\
24.89	0.00798729431976256\\
24.9	0.00798729433231382\\
24.91	0.0079872943448693\\
24.92	0.00798729435742906\\
24.93	0.00798729436999312\\
24.94	0.00798729438256152\\
24.95	0.0079872943951343\\
24.96	0.00798729440771149\\
24.97	0.00798729442029314\\
24.98	0.00798729443287926\\
24.99	0.00798729444546991\\
25	0.00798729445806511\\
25.01	0.00798729447066491\\
25.02	0.00798729448326934\\
25.03	0.00798729449587843\\
25.04	0.00798729450849223\\
25.05	0.00798729452111076\\
25.06	0.00798729453373407\\
25.07	0.00798729454636219\\
25.08	0.00798729455899516\\
25.09	0.00798729457163301\\
25.1	0.00798729458427578\\
25.11	0.0079872945969235\\
25.12	0.00798729460957621\\
25.13	0.00798729462223395\\
25.14	0.00798729463489676\\
25.15	0.00798729464756466\\
25.16	0.00798729466023769\\
25.17	0.0079872946729159\\
25.18	0.00798729468559931\\
25.19	0.00798729469828795\\
25.2	0.00798729471098187\\
25.21	0.00798729472368111\\
25.22	0.00798729473638568\\
25.23	0.00798729474909563\\
25.24	0.007987294761811\\
25.25	0.00798729477453181\\
25.26	0.0079872947872581\\
25.27	0.0079872947999899\\
25.28	0.00798729481272725\\
25.29	0.00798729482547018\\
25.3	0.00798729483821872\\
25.31	0.00798729485097291\\
25.32	0.00798729486373278\\
25.33	0.00798729487649835\\
25.34	0.00798729488926967\\
25.35	0.00798729490204675\\
25.36	0.00798729491482964\\
25.37	0.00798729492761836\\
25.38	0.00798729494041295\\
25.39	0.00798729495321343\\
25.4	0.00798729496601983\\
25.41	0.00798729497883218\\
25.42	0.00798729499165052\\
25.43	0.00798729500447486\\
25.44	0.00798729501730524\\
25.45	0.00798729503014169\\
25.46	0.00798729504298422\\
25.47	0.00798729505583287\\
25.48	0.00798729506868767\\
25.49	0.00798729508154864\\
25.5	0.00798729509441579\\
25.51	0.00798729510728917\\
25.52	0.0079872951201688\\
25.53	0.00798729513305469\\
25.54	0.00798729514594686\\
25.55	0.00798729515884536\\
25.56	0.00798729517175018\\
25.57	0.00798729518466136\\
25.58	0.00798729519757893\\
25.59	0.00798729521050288\\
25.6	0.00798729522343326\\
25.61	0.00798729523637007\\
25.62	0.00798729524931334\\
25.63	0.00798729526226309\\
25.64	0.00798729527521932\\
25.65	0.00798729528818207\\
25.66	0.00798729530115134\\
25.67	0.00798729531412716\\
25.68	0.00798729532710953\\
25.69	0.00798729534009848\\
25.7	0.00798729535309401\\
25.71	0.00798729536609614\\
25.72	0.00798729537910489\\
25.73	0.00798729539212027\\
25.74	0.00798729540514228\\
25.75	0.00798729541817095\\
25.76	0.00798729543120628\\
25.77	0.00798729544424828\\
25.78	0.00798729545729696\\
25.79	0.00798729547035234\\
25.8	0.00798729548341442\\
25.81	0.00798729549648321\\
25.82	0.00798729550955871\\
25.83	0.00798729552264094\\
25.84	0.00798729553572991\\
25.85	0.00798729554882562\\
25.86	0.00798729556192807\\
25.87	0.00798729557503727\\
25.88	0.00798729558815323\\
25.89	0.00798729560127596\\
25.9	0.00798729561440545\\
25.91	0.00798729562754172\\
25.92	0.00798729564068476\\
25.93	0.00798729565383459\\
25.94	0.0079872956669912\\
25.95	0.0079872956801546\\
25.96	0.00798729569332479\\
25.97	0.00798729570650178\\
25.98	0.00798729571968558\\
25.99	0.00798729573287617\\
26	0.00798729574607358\\
26.01	0.0079872957592778\\
26.02	0.00798729577248883\\
26.03	0.00798729578570668\\
26.04	0.00798729579893135\\
26.05	0.00798729581216285\\
26.06	0.00798729582540118\\
26.07	0.00798729583864635\\
26.08	0.00798729585189835\\
26.09	0.00798729586515719\\
26.1	0.00798729587842288\\
26.11	0.00798729589169542\\
26.12	0.00798729590497481\\
26.13	0.00798729591826106\\
26.14	0.00798729593155416\\
26.15	0.00798729594485413\\
26.16	0.00798729595816097\\
26.17	0.00798729597147467\\
26.18	0.00798729598479525\\
26.19	0.00798729599812271\\
26.2	0.00798729601145706\\
26.21	0.00798729602479828\\
26.22	0.0079872960381464\\
26.23	0.00798729605150141\\
26.24	0.00798729606486332\\
26.25	0.00798729607823212\\
26.26	0.00798729609160784\\
26.27	0.00798729610499046\\
26.28	0.00798729611837999\\
26.29	0.00798729613177643\\
26.3	0.0079872961451798\\
26.31	0.00798729615859009\\
26.32	0.00798729617200731\\
26.33	0.00798729618543146\\
26.34	0.00798729619886254\\
26.35	0.00798729621230056\\
26.36	0.00798729622574552\\
26.37	0.00798729623919743\\
26.38	0.00798729625265628\\
26.39	0.00798729626612209\\
26.4	0.00798729627959486\\
26.41	0.00798729629307459\\
26.42	0.00798729630656128\\
26.43	0.00798729632005494\\
26.44	0.00798729633355557\\
26.45	0.00798729634706318\\
26.46	0.00798729636057777\\
26.47	0.00798729637409934\\
26.48	0.00798729638762789\\
26.49	0.00798729640116344\\
26.5	0.00798729641470599\\
26.51	0.00798729642825553\\
26.52	0.00798729644181207\\
26.53	0.00798729645537562\\
26.54	0.00798729646894618\\
26.55	0.00798729648252376\\
26.56	0.00798729649610835\\
26.57	0.00798729650969996\\
26.58	0.0079872965232986\\
26.59	0.00798729653690427\\
26.6	0.00798729655051697\\
26.61	0.00798729656413671\\
26.62	0.00798729657776348\\
26.63	0.00798729659139731\\
26.64	0.00798729660503818\\
26.65	0.0079872966186861\\
26.66	0.00798729663234108\\
26.67	0.00798729664600312\\
26.68	0.00798729665967222\\
26.69	0.00798729667334839\\
26.7	0.00798729668703163\\
26.71	0.00798729670072194\\
26.72	0.00798729671441934\\
26.73	0.00798729672812381\\
26.74	0.00798729674183538\\
26.75	0.00798729675555403\\
26.76	0.00798729676927978\\
26.77	0.00798729678301263\\
26.78	0.00798729679675258\\
26.79	0.00798729681049963\\
26.8	0.0079872968242538\\
26.81	0.00798729683801508\\
26.82	0.00798729685178347\\
26.83	0.00798729686555899\\
26.84	0.00798729687934163\\
26.85	0.00798729689313141\\
26.86	0.00798729690692831\\
26.87	0.00798729692073236\\
26.88	0.00798729693454354\\
26.89	0.00798729694836187\\
26.9	0.00798729696218735\\
26.91	0.00798729697601998\\
26.92	0.00798729698985977\\
26.93	0.00798729700370671\\
26.94	0.00798729701756083\\
26.95	0.00798729703142211\\
26.96	0.00798729704529056\\
26.97	0.00798729705916619\\
26.98	0.007987297073049\\
26.99	0.007987297086939\\
27	0.00798729710083618\\
27.01	0.00798729711474055\\
27.02	0.00798729712865212\\
27.03	0.00798729714257089\\
27.04	0.00798729715649687\\
27.05	0.00798729717043005\\
27.06	0.00798729718437044\\
27.07	0.00798729719831805\\
27.08	0.00798729721227288\\
27.09	0.00798729722623493\\
27.1	0.00798729724020421\\
27.11	0.00798729725418073\\
27.12	0.00798729726816447\\
27.13	0.00798729728215546\\
27.14	0.00798729729615369\\
27.15	0.00798729731015917\\
27.16	0.0079872973241719\\
27.17	0.00798729733819188\\
27.18	0.00798729735221913\\
27.19	0.00798729736625364\\
27.2	0.00798729738029541\\
27.21	0.00798729739434446\\
27.22	0.00798729740840078\\
27.23	0.00798729742246438\\
27.24	0.00798729743653526\\
27.25	0.00798729745061343\\
27.26	0.0079872974646989\\
27.27	0.00798729747879166\\
27.28	0.00798729749289171\\
27.29	0.00798729750699908\\
27.3	0.00798729752111375\\
27.31	0.00798729753523572\\
27.32	0.00798729754936502\\
27.33	0.00798729756350164\\
27.34	0.00798729757764558\\
27.35	0.00798729759179684\\
27.36	0.00798729760595544\\
27.37	0.00798729762012138\\
27.38	0.00798729763429465\\
27.39	0.00798729764847527\\
27.4	0.00798729766266323\\
27.41	0.00798729767685855\\
27.42	0.00798729769106123\\
27.43	0.00798729770527126\\
27.44	0.00798729771948866\\
27.45	0.00798729773371342\\
27.46	0.00798729774794556\\
27.47	0.00798729776218508\\
27.48	0.00798729777643197\\
27.49	0.00798729779068625\\
27.5	0.00798729780494792\\
27.51	0.00798729781921698\\
27.52	0.00798729783349344\\
27.53	0.0079872978477773\\
27.54	0.00798729786206856\\
27.55	0.00798729787636723\\
27.56	0.00798729789067332\\
27.57	0.00798729790498682\\
27.58	0.00798729791930774\\
27.59	0.00798729793363609\\
27.6	0.00798729794797187\\
27.61	0.00798729796231508\\
27.62	0.00798729797666574\\
27.63	0.00798729799102383\\
27.64	0.00798729800538937\\
27.65	0.00798729801976235\\
27.66	0.0079872980341428\\
27.67	0.0079872980485307\\
27.68	0.00798729806292606\\
27.69	0.00798729807732889\\
27.7	0.00798729809173919\\
27.71	0.00798729810615697\\
27.72	0.00798729812058223\\
27.73	0.00798729813501496\\
27.74	0.00798729814945519\\
27.75	0.00798729816390291\\
27.76	0.00798729817835812\\
27.77	0.00798729819282084\\
27.78	0.00798729820729106\\
27.79	0.00798729822176878\\
27.8	0.00798729823625402\\
27.81	0.00798729825074678\\
27.82	0.00798729826524705\\
27.83	0.00798729827975485\\
27.84	0.00798729829427018\\
27.85	0.00798729830879305\\
27.86	0.00798729832332345\\
27.87	0.00798729833786139\\
27.88	0.00798729835240688\\
27.89	0.00798729836695992\\
27.9	0.00798729838152052\\
27.91	0.00798729839608867\\
27.92	0.00798729841066438\\
27.93	0.00798729842524767\\
27.94	0.00798729843983852\\
27.95	0.00798729845443695\\
27.96	0.00798729846904296\\
27.97	0.00798729848365655\\
27.98	0.00798729849827773\\
27.99	0.00798729851290651\\
28	0.00798729852754287\\
28.01	0.00798729854218685\\
28.02	0.00798729855683842\\
28.03	0.00798729857149761\\
28.04	0.00798729858616441\\
28.05	0.00798729860083882\\
28.06	0.00798729861552086\\
28.07	0.00798729863021053\\
28.08	0.00798729864490783\\
28.09	0.00798729865961276\\
28.1	0.00798729867432533\\
28.11	0.00798729868904554\\
28.12	0.0079872987037734\\
28.13	0.00798729871850892\\
28.14	0.00798729873325209\\
28.15	0.00798729874800292\\
28.16	0.00798729876276141\\
28.17	0.00798729877752758\\
28.18	0.00798729879230142\\
28.19	0.00798729880708293\\
28.2	0.00798729882187213\\
28.21	0.00798729883666902\\
28.22	0.00798729885147359\\
28.23	0.00798729886628586\\
28.24	0.00798729888110583\\
28.25	0.00798729889593351\\
28.26	0.00798729891076889\\
28.27	0.00798729892561198\\
28.28	0.00798729894046279\\
28.29	0.00798729895532133\\
28.3	0.00798729897018758\\
28.31	0.00798729898506157\\
28.32	0.00798729899994329\\
28.33	0.00798729901483275\\
28.34	0.00798729902972995\\
28.35	0.0079872990446349\\
28.36	0.0079872990595476\\
28.37	0.00798729907446806\\
28.38	0.00798729908939627\\
28.39	0.00798729910433226\\
28.4	0.00798729911927601\\
28.41	0.00798729913422753\\
28.42	0.00798729914918683\\
28.43	0.00798729916415391\\
28.44	0.00798729917912878\\
28.45	0.00798729919411144\\
28.46	0.0079872992091019\\
28.47	0.00798729922410015\\
28.48	0.00798729923910621\\
28.49	0.00798729925412008\\
28.5	0.00798729926914176\\
28.51	0.00798729928417126\\
28.52	0.00798729929920858\\
28.53	0.00798729931425373\\
28.54	0.0079872993293067\\
28.55	0.00798729934436751\\
28.56	0.00798729935943616\\
28.57	0.00798729937451266\\
28.58	0.007987299389597\\
28.59	0.00798729940468919\\
28.6	0.00798729941978924\\
28.61	0.00798729943489716\\
28.62	0.00798729945001294\\
28.63	0.00798729946513659\\
28.64	0.00798729948026811\\
28.65	0.00798729949540752\\
28.66	0.0079872995105548\\
28.67	0.00798729952570998\\
28.68	0.00798729954087305\\
28.69	0.00798729955604401\\
28.7	0.00798729957122288\\
28.71	0.00798729958640965\\
28.72	0.00798729960160434\\
28.73	0.00798729961680694\\
28.74	0.00798729963201746\\
28.75	0.0079872996472359\\
28.76	0.00798729966246227\\
28.77	0.00798729967769658\\
28.78	0.00798729969293882\\
28.79	0.007987299708189\\
28.8	0.00798729972344713\\
28.81	0.00798729973871322\\
28.82	0.00798729975398725\\
28.83	0.00798729976926925\\
28.84	0.00798729978455921\\
28.85	0.00798729979985715\\
28.86	0.00798729981516305\\
28.87	0.00798729983047694\\
28.88	0.0079872998457988\\
28.89	0.00798729986112866\\
28.9	0.0079872998764665\\
28.91	0.00798729989181234\\
28.92	0.00798729990716619\\
28.93	0.00798729992252803\\
28.94	0.00798729993789789\\
28.95	0.00798729995327577\\
28.96	0.00798729996866166\\
28.97	0.00798729998405558\\
28.98	0.00798729999945752\\
28.99	0.0079873000148675\\
29	0.00798730003028551\\
29.01	0.00798730004571157\\
29.02	0.00798730006114568\\
29.03	0.00798730007658783\\
29.04	0.00798730009203804\\
29.05	0.00798730010749631\\
29.06	0.00798730012296265\\
29.07	0.00798730013843706\\
29.08	0.00798730015391954\\
29.09	0.0079873001694101\\
29.1	0.00798730018490874\\
29.11	0.00798730020041547\\
29.12	0.0079873002159303\\
29.13	0.00798730023145322\\
29.14	0.00798730024698424\\
29.15	0.00798730026252337\\
29.16	0.00798730027807062\\
29.17	0.00798730029362597\\
29.18	0.00798730030918945\\
29.19	0.00798730032476106\\
29.2	0.00798730034034079\\
29.21	0.00798730035592866\\
29.22	0.00798730037152466\\
29.23	0.00798730038712882\\
29.24	0.00798730040274112\\
29.25	0.00798730041836157\\
29.26	0.00798730043399018\\
29.27	0.00798730044962695\\
29.28	0.00798730046527189\\
29.29	0.00798730048092501\\
29.3	0.0079873004965863\\
29.31	0.00798730051225576\\
29.32	0.00798730052793342\\
29.33	0.00798730054361927\\
29.34	0.00798730055931331\\
29.35	0.00798730057501556\\
29.36	0.007987300590726\\
29.37	0.00798730060644466\\
29.38	0.00798730062217153\\
29.39	0.00798730063790662\\
29.4	0.00798730065364994\\
29.41	0.00798730066940148\\
29.42	0.00798730068516126\\
29.43	0.00798730070092927\\
29.44	0.00798730071670553\\
29.45	0.00798730073249003\\
29.46	0.00798730074828279\\
29.47	0.0079873007640838\\
29.48	0.00798730077989308\\
29.49	0.00798730079571062\\
29.5	0.00798730081153643\\
29.51	0.00798730082737052\\
29.52	0.00798730084321289\\
29.53	0.00798730085906355\\
29.54	0.00798730087492249\\
29.55	0.00798730089078973\\
29.56	0.00798730090666527\\
29.57	0.00798730092254912\\
29.58	0.00798730093844127\\
29.59	0.00798730095434174\\
29.6	0.00798730097025053\\
29.61	0.00798730098616764\\
29.62	0.00798730100209308\\
29.63	0.00798730101802686\\
29.64	0.00798730103396897\\
29.65	0.00798730104991942\\
29.66	0.00798730106587823\\
29.67	0.00798730108184539\\
29.68	0.0079873010978209\\
29.69	0.00798730111380478\\
29.7	0.00798730112979702\\
29.71	0.00798730114579764\\
29.72	0.00798730116180663\\
29.73	0.00798730117782401\\
29.74	0.00798730119384977\\
29.75	0.00798730120988392\\
29.76	0.00798730122592647\\
29.77	0.00798730124197742\\
29.78	0.00798730125803678\\
29.79	0.00798730127410455\\
29.8	0.00798730129018073\\
29.81	0.00798730130626534\\
29.82	0.00798730132235837\\
29.83	0.00798730133845983\\
29.84	0.00798730135456973\\
29.85	0.00798730137068807\\
29.86	0.00798730138681485\\
29.87	0.00798730140295008\\
29.88	0.00798730141909377\\
29.89	0.00798730143524592\\
29.9	0.00798730145140653\\
29.91	0.00798730146757562\\
29.92	0.00798730148375317\\
29.93	0.00798730149993921\\
29.94	0.00798730151613373\\
29.95	0.00798730153233675\\
29.96	0.00798730154854825\\
29.97	0.00798730156476826\\
29.98	0.00798730158099677\\
29.99	0.00798730159723379\\
30	0.00798730161347932\\
30.01	0.00798730162973338\\
30.02	0.00798730164599595\\
30.03	0.00798730166226706\\
30.04	0.0079873016785467\\
30.05	0.00798730169483488\\
30.06	0.00798730171113161\\
30.07	0.00798730172743688\\
30.08	0.00798730174375071\\
30.09	0.0079873017600731\\
30.1	0.00798730177640405\\
30.11	0.00798730179274357\\
30.12	0.00798730180909166\\
30.13	0.00798730182544834\\
30.14	0.00798730184181359\\
30.15	0.00798730185818744\\
30.16	0.00798730187456988\\
30.17	0.00798730189096092\\
30.18	0.00798730190736057\\
30.19	0.00798730192376882\\
30.2	0.00798730194018569\\
30.21	0.00798730195661118\\
30.22	0.00798730197304529\\
30.23	0.00798730198948803\\
30.24	0.00798730200593941\\
30.25	0.00798730202239943\\
30.26	0.00798730203886809\\
30.27	0.0079873020553454\\
30.28	0.00798730207183136\\
30.29	0.00798730208832599\\
30.3	0.00798730210482928\\
30.31	0.00798730212134124\\
30.32	0.00798730213786187\\
30.33	0.00798730215439119\\
30.34	0.00798730217092919\\
30.35	0.00798730218747588\\
30.36	0.00798730220403126\\
30.37	0.00798730222059535\\
30.38	0.00798730223716814\\
30.39	0.00798730225374964\\
30.4	0.00798730227033986\\
30.41	0.0079873022869388\\
30.42	0.00798730230354647\\
30.43	0.00798730232016287\\
30.44	0.007987302336788\\
30.45	0.00798730235342188\\
30.46	0.0079873023700645\\
30.47	0.00798730238671587\\
30.48	0.007987302403376\\
30.49	0.0079873024200449\\
30.5	0.00798730243672256\\
30.51	0.00798730245340899\\
30.52	0.0079873024701042\\
30.53	0.00798730248680819\\
30.54	0.00798730250352098\\
30.55	0.00798730252024255\\
30.56	0.00798730253697292\\
30.57	0.0079873025537121\\
30.58	0.00798730257046008\\
30.59	0.00798730258721688\\
30.6	0.0079873026039825\\
30.61	0.00798730262075694\\
30.62	0.00798730263754022\\
30.63	0.00798730265433232\\
30.64	0.00798730267113327\\
30.65	0.00798730268794306\\
30.66	0.0079873027047617\\
30.67	0.0079873027215892\\
30.68	0.00798730273842556\\
30.69	0.00798730275527078\\
30.7	0.00798730277212488\\
30.71	0.00798730278898785\\
30.72	0.0079873028058597\\
30.73	0.00798730282274044\\
30.74	0.00798730283963008\\
30.75	0.0079873028565286\\
30.76	0.00798730287343604\\
30.77	0.00798730289035238\\
30.78	0.00798730290727763\\
30.79	0.00798730292421181\\
30.8	0.00798730294115491\\
30.81	0.00798730295810693\\
30.82	0.00798730297506789\\
30.83	0.00798730299203779\\
30.84	0.00798730300901664\\
30.85	0.00798730302600444\\
30.86	0.00798730304300119\\
30.87	0.0079873030600069\\
30.88	0.00798730307702158\\
30.89	0.00798730309404523\\
30.9	0.00798730311107786\\
30.91	0.00798730312811947\\
30.92	0.00798730314517007\\
30.93	0.00798730316222966\\
30.94	0.00798730317929824\\
30.95	0.00798730319637584\\
30.96	0.00798730321346244\\
30.97	0.00798730323055805\\
30.98	0.00798730324766269\\
30.99	0.00798730326477635\\
31	0.00798730328189904\\
31.01	0.00798730329903076\\
31.02	0.00798730331617153\\
31.03	0.00798730333332135\\
31.04	0.00798730335048021\\
31.05	0.00798730336764814\\
31.06	0.00798730338482512\\
31.07	0.00798730340201117\\
31.08	0.0079873034192063\\
31.09	0.00798730343641051\\
31.1	0.0079873034536238\\
31.11	0.00798730347084618\\
31.12	0.00798730348807765\\
31.13	0.00798730350531822\\
31.14	0.00798730352256791\\
31.15	0.0079873035398267\\
31.16	0.00798730355709461\\
31.17	0.00798730357437164\\
31.18	0.0079873035916578\\
31.19	0.00798730360895309\\
31.2	0.00798730362625752\\
31.21	0.00798730364357109\\
31.22	0.00798730366089381\\
31.23	0.00798730367822569\\
31.24	0.00798730369556673\\
31.25	0.00798730371291693\\
31.26	0.00798730373027631\\
31.27	0.00798730374764486\\
31.28	0.0079873037650226\\
31.29	0.00798730378240952\\
31.3	0.00798730379980563\\
31.31	0.00798730381721095\\
31.32	0.00798730383462547\\
31.33	0.00798730385204919\\
31.34	0.00798730386948214\\
31.35	0.0079873038869243\\
31.36	0.00798730390437569\\
31.37	0.00798730392183631\\
31.38	0.00798730393930617\\
31.39	0.00798730395678527\\
31.4	0.00798730397427362\\
31.41	0.00798730399177122\\
31.42	0.00798730400927809\\
31.43	0.00798730402679421\\
31.44	0.00798730404431961\\
31.45	0.00798730406185429\\
31.46	0.00798730407939824\\
31.47	0.00798730409695148\\
31.48	0.00798730411451402\\
31.49	0.00798730413208585\\
31.5	0.00798730414966699\\
31.51	0.00798730416725744\\
31.52	0.0079873041848572\\
31.53	0.00798730420246628\\
31.54	0.00798730422008469\\
31.55	0.00798730423771243\\
31.56	0.0079873042553495\\
31.57	0.00798730427299593\\
31.58	0.00798730429065169\\
31.59	0.00798730430831682\\
31.6	0.0079873043259913\\
31.61	0.00798730434367515\\
31.62	0.00798730436136837\\
31.63	0.00798730437907096\\
31.64	0.00798730439678294\\
31.65	0.0079873044145043\\
31.66	0.00798730443223506\\
31.67	0.00798730444997522\\
31.68	0.00798730446772478\\
31.69	0.00798730448548376\\
31.7	0.00798730450325215\\
31.71	0.00798730452102996\\
31.72	0.0079873045388172\\
31.73	0.00798730455661387\\
31.74	0.00798730457441998\\
31.75	0.00798730459223554\\
31.76	0.00798730461006054\\
31.77	0.007987304627895\\
31.78	0.00798730464573892\\
31.79	0.00798730466359231\\
31.8	0.00798730468145517\\
31.81	0.00798730469932751\\
31.82	0.00798730471720934\\
31.83	0.00798730473510065\\
31.84	0.00798730475300147\\
31.85	0.00798730477091178\\
31.86	0.0079873047888316\\
31.87	0.00798730480676093\\
31.88	0.00798730482469978\\
31.89	0.00798730484264816\\
31.9	0.00798730486060606\\
31.91	0.0079873048785735\\
31.92	0.00798730489655049\\
31.93	0.00798730491453702\\
31.94	0.0079873049325331\\
31.95	0.00798730495053874\\
31.96	0.00798730496855395\\
31.97	0.00798730498657873\\
31.98	0.00798730500461308\\
31.99	0.00798730502265702\\
32	0.00798730504071054\\
32.01	0.00798730505877365\\
32.02	0.00798730507684637\\
32.03	0.00798730509492869\\
32.04	0.00798730511302062\\
32.05	0.00798730513112217\\
32.06	0.00798730514923334\\
32.07	0.00798730516735414\\
32.08	0.00798730518548457\\
32.09	0.00798730520362465\\
32.1	0.00798730522177437\\
32.11	0.00798730523993374\\
32.12	0.00798730525810276\\
32.13	0.00798730527628146\\
32.14	0.00798730529446981\\
32.15	0.00798730531266785\\
32.16	0.00798730533087557\\
32.17	0.00798730534909297\\
32.18	0.00798730536732006\\
32.19	0.00798730538555686\\
32.2	0.00798730540380336\\
32.21	0.00798730542205956\\
32.22	0.00798730544032549\\
32.23	0.00798730545860113\\
32.24	0.0079873054768865\\
32.25	0.00798730549518161\\
32.26	0.00798730551348646\\
32.27	0.00798730553180105\\
32.28	0.00798730555012539\\
32.29	0.00798730556845949\\
32.3	0.00798730558680336\\
32.31	0.00798730560515699\\
32.32	0.0079873056235204\\
32.33	0.00798730564189358\\
32.34	0.00798730566027656\\
32.35	0.00798730567866933\\
32.36	0.00798730569707189\\
32.37	0.00798730571548426\\
32.38	0.00798730573390644\\
32.39	0.00798730575233844\\
32.4	0.00798730577078026\\
32.41	0.00798730578923191\\
32.42	0.00798730580769339\\
32.43	0.00798730582616472\\
32.44	0.00798730584464588\\
32.45	0.00798730586313691\\
32.46	0.00798730588163779\\
32.47	0.00798730590014853\\
32.48	0.00798730591866914\\
32.49	0.00798730593719963\\
32.5	0.00798730595574001\\
32.51	0.00798730597429027\\
32.52	0.00798730599285042\\
32.53	0.00798730601142048\\
32.54	0.00798730603000044\\
32.55	0.00798730604859031\\
32.56	0.0079873060671901\\
32.57	0.00798730608579981\\
32.58	0.00798730610441946\\
32.59	0.00798730612304904\\
32.6	0.00798730614168856\\
32.61	0.00798730616033803\\
32.62	0.00798730617899746\\
32.63	0.00798730619766685\\
32.64	0.0079873062163462\\
32.65	0.00798730623503553\\
32.66	0.00798730625373483\\
32.67	0.00798730627244412\\
32.68	0.0079873062911634\\
32.69	0.00798730630989267\\
32.7	0.00798730632863195\\
32.71	0.00798730634738124\\
32.72	0.00798730636614054\\
32.73	0.00798730638490986\\
32.74	0.00798730640368921\\
32.75	0.0079873064224786\\
32.76	0.00798730644127802\\
32.77	0.00798730646008749\\
32.78	0.00798730647890701\\
32.79	0.00798730649773658\\
32.8	0.00798730651657622\\
32.81	0.00798730653542593\\
32.82	0.00798730655428572\\
32.83	0.00798730657315559\\
32.84	0.00798730659203555\\
32.85	0.0079873066109256\\
32.86	0.00798730662982575\\
32.87	0.00798730664873601\\
32.88	0.00798730666765638\\
32.89	0.00798730668658687\\
32.9	0.00798730670552748\\
32.91	0.00798730672447823\\
32.92	0.00798730674343911\\
32.93	0.00798730676241014\\
32.94	0.00798730678139132\\
32.95	0.00798730680038265\\
32.96	0.00798730681938414\\
32.97	0.0079873068383958\\
32.98	0.00798730685741764\\
32.99	0.00798730687644966\\
33	0.00798730689549186\\
33.01	0.00798730691454426\\
33.02	0.00798730693360686\\
33.03	0.00798730695267966\\
33.04	0.00798730697176267\\
33.05	0.00798730699085591\\
33.06	0.00798730700995936\\
33.07	0.00798730702907305\\
33.08	0.00798730704819697\\
33.09	0.00798730706733113\\
33.1	0.00798730708647555\\
33.11	0.00798730710563022\\
33.12	0.00798730712479515\\
33.13	0.00798730714397035\\
33.14	0.00798730716315582\\
33.15	0.00798730718235157\\
33.16	0.00798730720155761\\
33.17	0.00798730722077394\\
33.18	0.00798730724000057\\
33.19	0.0079873072592375\\
33.2	0.00798730727848475\\
33.21	0.00798730729774232\\
33.22	0.0079873073170102\\
33.23	0.00798730733628842\\
33.24	0.00798730735557698\\
33.25	0.00798730737487588\\
33.26	0.00798730739418512\\
33.27	0.00798730741350473\\
33.28	0.00798730743283469\\
33.29	0.00798730745217502\\
33.3	0.00798730747152573\\
33.31	0.00798730749088682\\
33.32	0.00798730751025829\\
33.33	0.00798730752964016\\
33.34	0.00798730754903243\\
33.35	0.0079873075684351\\
33.36	0.00798730758784819\\
33.37	0.0079873076072717\\
33.38	0.00798730762670562\\
33.39	0.00798730764614999\\
33.4	0.00798730766560479\\
33.41	0.00798730768507003\\
33.42	0.00798730770454572\\
33.43	0.00798730772403187\\
33.44	0.00798730774352848\\
33.45	0.00798730776303557\\
33.46	0.00798730778255313\\
33.47	0.00798730780208117\\
33.48	0.0079873078216197\\
33.49	0.00798730784116872\\
33.5	0.00798730786072825\\
33.51	0.00798730788029828\\
33.52	0.00798730789987883\\
33.53	0.0079873079194699\\
33.54	0.00798730793907149\\
33.55	0.00798730795868362\\
33.56	0.00798730797830629\\
33.57	0.0079873079979395\\
33.58	0.00798730801758327\\
33.59	0.0079873080372376\\
33.6	0.00798730805690249\\
33.61	0.00798730807657795\\
33.62	0.007987308096264\\
33.63	0.00798730811596062\\
33.64	0.00798730813566784\\
33.65	0.00798730815538565\\
33.66	0.00798730817511407\\
33.67	0.0079873081948531\\
33.68	0.00798730821460275\\
33.69	0.00798730823436302\\
33.7	0.00798730825413392\\
33.71	0.00798730827391546\\
33.72	0.00798730829370764\\
33.73	0.00798730831351047\\
33.74	0.00798730833332395\\
33.75	0.0079873083531481\\
33.76	0.00798730837298291\\
33.77	0.0079873083928284\\
33.78	0.00798730841268457\\
33.79	0.00798730843255143\\
33.8	0.00798730845242899\\
33.81	0.00798730847231724\\
33.82	0.00798730849221621\\
33.83	0.00798730851212589\\
33.84	0.00798730853204629\\
33.85	0.00798730855197741\\
33.86	0.00798730857191927\\
33.87	0.00798730859187187\\
33.88	0.00798730861183522\\
33.89	0.00798730863180932\\
33.9	0.00798730865179418\\
33.91	0.0079873086717898\\
33.92	0.0079873086917962\\
33.93	0.00798730871181338\\
33.94	0.00798730873184134\\
33.95	0.0079873087518801\\
33.96	0.00798730877192966\\
33.97	0.00798730879199002\\
33.98	0.00798730881206119\\
33.99	0.00798730883214318\\
34	0.007987308852236\\
34.01	0.00798730887233965\\
34.02	0.00798730889245414\\
34.03	0.00798730891257947\\
34.04	0.00798730893271566\\
34.05	0.0079873089528627\\
34.06	0.00798730897302061\\
34.07	0.00798730899318939\\
34.08	0.00798730901336904\\
34.09	0.00798730903355958\\
34.1	0.00798730905376102\\
34.11	0.00798730907397335\\
34.12	0.00798730909419658\\
34.13	0.00798730911443073\\
34.14	0.00798730913467579\\
34.15	0.00798730915493178\\
34.16	0.00798730917519869\\
34.17	0.00798730919547655\\
34.18	0.00798730921576535\\
34.19	0.0079873092360651\\
34.2	0.00798730925637581\\
34.21	0.00798730927669748\\
34.22	0.00798730929703012\\
34.23	0.00798730931737374\\
34.24	0.00798730933772834\\
34.25	0.00798730935809394\\
34.26	0.00798730937847053\\
34.27	0.00798730939885812\\
34.28	0.00798730941925673\\
34.29	0.00798730943966635\\
34.3	0.007987309460087\\
34.31	0.00798730948051867\\
34.32	0.00798730950096139\\
34.33	0.00798730952141515\\
34.34	0.00798730954187995\\
34.35	0.00798730956235582\\
34.36	0.00798730958284275\\
34.37	0.00798730960334075\\
34.38	0.00798730962384982\\
34.39	0.00798730964436998\\
34.4	0.00798730966490124\\
34.41	0.00798730968544359\\
34.42	0.00798730970599704\\
34.43	0.0079873097265616\\
34.44	0.00798730974713728\\
34.45	0.00798730976772409\\
34.46	0.00798730978832202\\
34.47	0.0079873098089311\\
34.48	0.00798730982955131\\
34.49	0.00798730985018268\\
34.5	0.00798730987082521\\
34.51	0.0079873098914789\\
34.52	0.00798730991214376\\
34.53	0.0079873099328198\\
34.54	0.00798730995350703\\
34.55	0.00798730997420545\\
34.56	0.00798730999491506\\
34.57	0.00798731001563588\\
34.58	0.00798731003636791\\
34.59	0.00798731005711116\\
34.6	0.00798731007786564\\
34.61	0.00798731009863134\\
34.62	0.00798731011940829\\
34.63	0.00798731014019648\\
34.64	0.00798731016099592\\
34.65	0.00798731018180662\\
34.66	0.00798731020262859\\
34.67	0.00798731022346183\\
34.68	0.00798731024430635\\
34.69	0.00798731026516216\\
34.7	0.00798731028602926\\
34.71	0.00798731030690766\\
34.72	0.00798731032779736\\
34.73	0.00798731034869838\\
34.74	0.00798731036961072\\
34.75	0.00798731039053438\\
34.76	0.00798731041146938\\
34.77	0.00798731043241572\\
34.78	0.00798731045337341\\
34.79	0.00798731047434245\\
34.8	0.00798731049532285\\
34.81	0.00798731051631462\\
34.82	0.00798731053731777\\
34.83	0.00798731055833229\\
34.84	0.00798731057935821\\
34.85	0.00798731060039552\\
34.86	0.00798731062144423\\
34.87	0.00798731064250435\\
34.88	0.00798731066357588\\
34.89	0.00798731068465884\\
34.9	0.00798731070575323\\
34.91	0.00798731072685905\\
34.92	0.00798731074797631\\
34.93	0.00798731076910503\\
34.94	0.0079873107902452\\
34.95	0.00798731081139684\\
34.96	0.00798731083255994\\
34.97	0.00798731085373452\\
34.98	0.00798731087492059\\
34.99	0.00798731089611815\\
35	0.00798731091732721\\
35.01	0.00798731093854777\\
35.02	0.00798731095977984\\
35.03	0.00798731098102343\\
35.04	0.00798731100227855\\
35.05	0.00798731102354519\\
35.06	0.00798731104482338\\
35.07	0.00798731106611311\\
35.08	0.0079873110874144\\
35.09	0.00798731110872724\\
35.1	0.00798731113005166\\
35.11	0.00798731115138764\\
35.12	0.00798731117273521\\
35.13	0.00798731119409436\\
35.14	0.00798731121546511\\
35.15	0.00798731123684746\\
35.16	0.00798731125824141\\
35.17	0.00798731127964698\\
35.18	0.00798731130106418\\
35.19	0.007987311322493\\
35.2	0.00798731134393346\\
35.21	0.00798731136538556\\
35.22	0.00798731138684931\\
35.23	0.00798731140832471\\
35.24	0.00798731142981178\\
35.25	0.00798731145131052\\
35.26	0.00798731147282094\\
35.27	0.00798731149434304\\
35.28	0.00798731151587683\\
35.29	0.00798731153742232\\
35.3	0.00798731155897952\\
35.31	0.00798731158054843\\
35.32	0.00798731160212905\\
35.33	0.00798731162372141\\
35.34	0.00798731164532549\\
35.35	0.00798731166694132\\
35.36	0.00798731168856889\\
35.37	0.00798731171020821\\
35.38	0.0079873117318593\\
35.39	0.00798731175352215\\
35.4	0.00798731177519678\\
35.41	0.00798731179688319\\
35.42	0.00798731181858139\\
35.43	0.00798731184029138\\
35.44	0.00798731186201317\\
35.45	0.00798731188374678\\
35.46	0.0079873119054922\\
35.47	0.00798731192724945\\
35.48	0.00798731194901852\\
35.49	0.00798731197079943\\
35.5	0.00798731199259219\\
35.51	0.0079873120143968\\
35.52	0.00798731203621327\\
35.53	0.0079873120580416\\
35.54	0.0079873120798818\\
35.55	0.00798731210173389\\
35.56	0.00798731212359786\\
35.57	0.00798731214547372\\
35.58	0.00798731216736148\\
35.59	0.00798731218926115\\
35.6	0.00798731221117274\\
35.61	0.00798731223309625\\
35.62	0.00798731225503168\\
35.63	0.00798731227697905\\
35.64	0.00798731229893836\\
35.65	0.00798731232090963\\
35.66	0.00798731234289284\\
35.67	0.00798731236488802\\
35.68	0.00798731238689518\\
35.69	0.00798731240891431\\
35.7	0.00798731243094542\\
35.71	0.00798731245298853\\
35.72	0.00798731247504363\\
35.73	0.00798731249711074\\
35.74	0.00798731251918986\\
35.75	0.00798731254128101\\
35.76	0.00798731256338417\\
35.77	0.00798731258549938\\
35.78	0.00798731260762662\\
35.79	0.00798731262976591\\
35.8	0.00798731265191726\\
35.81	0.00798731267408067\\
35.82	0.00798731269625615\\
35.83	0.0079873127184437\\
35.84	0.00798731274064334\\
35.85	0.00798731276285506\\
35.86	0.00798731278507889\\
35.87	0.00798731280731482\\
35.88	0.00798731282956286\\
35.89	0.00798731285182302\\
35.9	0.0079873128740953\\
35.91	0.00798731289637972\\
35.92	0.00798731291867627\\
35.93	0.00798731294098498\\
35.94	0.00798731296330583\\
35.95	0.00798731298563885\\
35.96	0.00798731300798404\\
35.97	0.0079873130303414\\
35.98	0.00798731305271094\\
35.99	0.00798731307509268\\
36	0.0079873130974866\\
36.01	0.00798731311989274\\
36.02	0.00798731314231108\\
36.03	0.00798731316474164\\
36.04	0.00798731318718443\\
36.05	0.00798731320963944\\
36.06	0.0079873132321067\\
36.07	0.0079873132545862\\
36.08	0.00798731327707795\\
36.09	0.00798731329958197\\
36.1	0.00798731332209825\\
36.11	0.00798731334462681\\
36.12	0.00798731336716765\\
36.13	0.00798731338972077\\
36.14	0.0079873134122862\\
36.15	0.00798731343486392\\
36.16	0.00798731345745396\\
36.17	0.00798731348005631\\
36.18	0.00798731350267099\\
36.19	0.00798731352529799\\
36.2	0.00798731354793734\\
36.21	0.00798731357058903\\
36.22	0.00798731359325308\\
36.23	0.00798731361592948\\
36.24	0.00798731363861825\\
36.25	0.0079873136613194\\
36.26	0.00798731368403293\\
36.27	0.00798731370675884\\
36.28	0.00798731372949715\\
36.29	0.00798731375224786\\
36.3	0.00798731377501099\\
36.31	0.00798731379778653\\
36.32	0.00798731382057449\\
36.33	0.00798731384337488\\
36.34	0.00798731386618772\\
36.35	0.007987313889013\\
36.36	0.00798731391185073\\
36.37	0.00798731393470092\\
36.38	0.00798731395756358\\
36.39	0.00798731398043871\\
36.4	0.00798731400332632\\
36.41	0.00798731402622642\\
36.42	0.00798731404913902\\
36.43	0.00798731407206412\\
36.44	0.00798731409500173\\
36.45	0.00798731411795186\\
36.46	0.0079873141409145\\
36.47	0.00798731416388969\\
36.48	0.00798731418687741\\
36.49	0.00798731420987767\\
36.5	0.00798731423289049\\
36.51	0.00798731425591587\\
36.52	0.00798731427895382\\
36.53	0.00798731430200434\\
36.54	0.00798731432506744\\
36.55	0.00798731434814313\\
36.56	0.00798731437123141\\
36.57	0.0079873143943323\\
36.58	0.0079873144174458\\
36.59	0.00798731444057192\\
36.6	0.00798731446371066\\
36.61	0.00798731448686203\\
36.62	0.00798731451002604\\
36.63	0.0079873145332027\\
36.64	0.00798731455639201\\
36.65	0.00798731457959398\\
36.66	0.00798731460280862\\
36.67	0.00798731462603593\\
36.68	0.00798731464927593\\
36.69	0.00798731467252861\\
36.7	0.00798731469579399\\
36.71	0.00798731471907208\\
36.72	0.00798731474236287\\
36.73	0.00798731476566638\\
36.74	0.00798731478898262\\
36.75	0.00798731481231159\\
36.76	0.0079873148356533\\
36.77	0.00798731485900776\\
36.78	0.00798731488237497\\
36.79	0.00798731490575494\\
36.8	0.00798731492914768\\
36.81	0.00798731495255319\\
36.82	0.00798731497597149\\
36.83	0.00798731499940258\\
36.84	0.00798731502284647\\
36.85	0.00798731504630316\\
36.86	0.00798731506977266\\
36.87	0.00798731509325498\\
36.88	0.00798731511675013\\
36.89	0.0079873151402581\\
36.9	0.00798731516377892\\
36.91	0.00798731518731259\\
36.92	0.00798731521085911\\
36.93	0.0079873152344185\\
36.94	0.00798731525799075\\
36.95	0.00798731528157588\\
36.96	0.00798731530517389\\
36.97	0.0079873153287848\\
36.98	0.0079873153524086\\
36.99	0.00798731537604531\\
37	0.00798731539969493\\
37.01	0.00798731542335747\\
37.02	0.00798731544703294\\
37.03	0.00798731547072134\\
37.04	0.00798731549442268\\
37.05	0.00798731551813697\\
37.06	0.00798731554186422\\
37.07	0.00798731556560443\\
37.08	0.00798731558935761\\
37.09	0.00798731561312377\\
37.1	0.00798731563690292\\
37.11	0.00798731566069505\\
37.12	0.00798731568450019\\
37.13	0.00798731570831833\\
37.14	0.00798731573214948\\
37.15	0.00798731575599366\\
37.16	0.00798731577985086\\
37.17	0.0079873158037211\\
37.18	0.00798731582760439\\
37.19	0.00798731585150072\\
37.2	0.00798731587541011\\
37.21	0.00798731589933256\\
37.22	0.00798731592326809\\
37.23	0.00798731594721669\\
37.24	0.00798731597117838\\
37.25	0.00798731599515317\\
37.26	0.00798731601914105\\
37.27	0.00798731604314205\\
37.28	0.00798731606715615\\
37.29	0.00798731609118338\\
37.3	0.00798731611522374\\
37.31	0.00798731613927724\\
37.32	0.00798731616334388\\
37.33	0.00798731618742367\\
37.34	0.00798731621151662\\
37.35	0.00798731623562274\\
37.36	0.00798731625974203\\
37.37	0.0079873162838745\\
37.38	0.00798731630802015\\
37.39	0.007987316332179\\
37.4	0.00798731635635106\\
37.41	0.00798731638053632\\
37.42	0.0079873164047348\\
37.43	0.0079873164289465\\
37.44	0.00798731645317144\\
37.45	0.00798731647740961\\
37.46	0.00798731650166103\\
37.47	0.0079873165259257\\
37.48	0.00798731655020363\\
37.49	0.00798731657449483\\
37.5	0.0079873165987993\\
37.51	0.00798731662311706\\
37.52	0.0079873166474481\\
37.53	0.00798731667179244\\
37.54	0.00798731669615008\\
37.55	0.00798731672052104\\
37.56	0.00798731674490531\\
37.57	0.00798731676930291\\
37.58	0.00798731679371384\\
37.59	0.00798731681813811\\
37.6	0.00798731684257572\\
37.61	0.00798731686702669\\
37.62	0.00798731689149103\\
37.63	0.00798731691596873\\
37.64	0.00798731694045981\\
37.65	0.00798731696496427\\
37.66	0.00798731698948212\\
37.67	0.00798731701401337\\
37.68	0.00798731703855802\\
37.69	0.00798731706311609\\
37.7	0.00798731708768757\\
37.71	0.00798731711227249\\
37.72	0.00798731713687084\\
37.73	0.00798731716148262\\
37.74	0.00798731718610786\\
37.75	0.00798731721074656\\
37.76	0.00798731723539871\\
37.77	0.00798731726006434\\
37.78	0.00798731728474344\\
37.79	0.00798731730943603\\
37.8	0.00798731733414212\\
37.81	0.0079873173588617\\
37.82	0.00798731738359479\\
37.83	0.00798731740834139\\
37.84	0.00798731743310151\\
37.85	0.00798731745787516\\
37.86	0.00798731748266235\\
37.87	0.00798731750746308\\
37.88	0.00798731753227736\\
37.89	0.0079873175571052\\
37.9	0.0079873175819466\\
37.91	0.00798731760680157\\
37.92	0.00798731763167013\\
37.93	0.00798731765655227\\
37.94	0.007987317681448\\
37.95	0.00798731770635734\\
37.96	0.00798731773128028\\
37.97	0.00798731775621684\\
37.98	0.00798731778116702\\
37.99	0.00798731780613083\\
38	0.00798731783110828\\
38.01	0.00798731785609938\\
38.02	0.00798731788110412\\
38.03	0.00798731790612253\\
38.04	0.0079873179311546\\
38.05	0.00798731795620035\\
38.06	0.00798731798125977\\
38.07	0.00798731800633288\\
38.08	0.00798731803141969\\
38.09	0.0079873180565202\\
38.1	0.00798731808163443\\
38.11	0.00798731810676236\\
38.12	0.00798731813190403\\
38.13	0.00798731815705942\\
38.14	0.00798731818222856\\
38.15	0.00798731820741143\\
38.16	0.00798731823260807\\
38.17	0.00798731825781846\\
38.18	0.00798731828304262\\
38.19	0.00798731830828056\\
38.2	0.00798731833353228\\
38.21	0.00798731835879779\\
38.22	0.00798731838407709\\
38.23	0.0079873184093702\\
38.24	0.00798731843467712\\
38.25	0.00798731845999786\\
38.26	0.00798731848533242\\
38.27	0.00798731851068082\\
38.28	0.00798731853604306\\
38.29	0.00798731856141915\\
38.3	0.00798731858680909\\
38.31	0.00798731861221289\\
38.32	0.00798731863763056\\
38.33	0.00798731866306211\\
38.34	0.00798731868850754\\
38.35	0.00798731871396686\\
38.36	0.00798731873944009\\
38.37	0.00798731876492722\\
38.38	0.00798731879042825\\
38.39	0.00798731881594321\\
38.4	0.0079873188414721\\
38.41	0.00798731886701492\\
38.42	0.00798731889257169\\
38.43	0.0079873189181424\\
38.44	0.00798731894372707\\
38.45	0.0079873189693257\\
38.46	0.0079873189949383\\
38.47	0.00798731902056488\\
38.48	0.00798731904620545\\
38.49	0.00798731907186\\
38.5	0.00798731909752856\\
38.51	0.00798731912321112\\
38.52	0.0079873191489077\\
38.53	0.0079873191746183\\
38.54	0.00798731920034293\\
38.55	0.00798731922608159\\
38.56	0.0079873192518343\\
38.57	0.00798731927760105\\
38.58	0.00798731930338187\\
38.59	0.00798731932917675\\
38.6	0.0079873193549857\\
38.61	0.00798731938080872\\
38.62	0.00798731940664584\\
38.63	0.00798731943249705\\
38.64	0.00798731945836236\\
38.65	0.00798731948424177\\
38.66	0.00798731951013531\\
38.67	0.00798731953604296\\
38.68	0.00798731956196475\\
38.69	0.00798731958790067\\
38.7	0.00798731961385073\\
38.71	0.00798731963981495\\
38.72	0.00798731966579332\\
38.73	0.00798731969178586\\
38.74	0.00798731971779258\\
38.75	0.00798731974381347\\
38.76	0.00798731976984855\\
38.77	0.00798731979589782\\
38.78	0.0079873198219613\\
38.79	0.00798731984803898\\
38.8	0.00798731987413089\\
38.81	0.00798731990023701\\
38.82	0.00798731992635736\\
38.83	0.00798731995249195\\
38.84	0.00798731997864079\\
38.85	0.00798732000480388\\
38.86	0.00798732003098123\\
38.87	0.00798732005717284\\
38.88	0.00798732008337873\\
38.89	0.0079873201095989\\
38.9	0.00798732013583336\\
38.91	0.00798732016208211\\
38.92	0.00798732018834516\\
38.93	0.00798732021462253\\
38.94	0.00798732024091421\\
38.95	0.00798732026722021\\
38.96	0.00798732029354055\\
38.97	0.00798732031987522\\
38.98	0.00798732034622424\\
38.99	0.00798732037258761\\
39	0.00798732039896534\\
39.01	0.00798732042535744\\
39.02	0.00798732045176391\\
39.03	0.00798732047818476\\
39.04	0.00798732050462001\\
39.05	0.00798732053106964\\
39.06	0.00798732055753369\\
39.07	0.00798732058401213\\
39.08	0.007987320610505\\
39.09	0.00798732063701229\\
39.1	0.00798732066353402\\
39.11	0.00798732069007018\\
39.12	0.00798732071662079\\
39.13	0.00798732074318585\\
39.14	0.00798732076976537\\
39.15	0.00798732079635936\\
39.16	0.00798732082296782\\
39.17	0.00798732084959076\\
39.18	0.00798732087622819\\
39.19	0.00798732090288012\\
39.2	0.00798732092954655\\
39.21	0.00798732095622749\\
39.22	0.00798732098292295\\
39.23	0.00798732100963294\\
39.24	0.00798732103635745\\
39.25	0.0079873210630965\\
39.26	0.0079873210898501\\
39.27	0.00798732111661825\\
39.28	0.00798732114340097\\
39.29	0.00798732117019824\\
39.3	0.0079873211970101\\
39.31	0.00798732122383653\\
39.32	0.00798732125067756\\
39.33	0.00798732127753318\\
39.34	0.0079873213044034\\
39.35	0.00798732133128823\\
39.36	0.00798732135818768\\
39.37	0.00798732138510175\\
39.38	0.00798732141203046\\
39.39	0.0079873214389738\\
39.4	0.00798732146593179\\
39.41	0.00798732149290443\\
39.42	0.00798732151989173\\
39.43	0.0079873215468937\\
39.44	0.00798732157391034\\
39.45	0.00798732160094166\\
39.46	0.00798732162798767\\
39.47	0.00798732165504838\\
39.48	0.00798732168212379\\
39.49	0.0079873217092139\\
39.5	0.00798732173631874\\
39.51	0.0079873217634383\\
39.52	0.00798732179057259\\
39.53	0.00798732181772161\\
39.54	0.00798732184488538\\
39.55	0.00798732187206391\\
39.56	0.00798732189925719\\
39.57	0.00798732192646524\\
39.58	0.00798732195368806\\
39.59	0.00798732198092566\\
39.6	0.00798732200817805\\
39.61	0.00798732203544524\\
39.62	0.00798732206272722\\
39.63	0.00798732209002401\\
39.64	0.00798732211733562\\
39.65	0.00798732214466206\\
39.66	0.00798732217200332\\
39.67	0.00798732219935942\\
39.68	0.00798732222673036\\
39.69	0.00798732225411615\\
39.7	0.0079873222815168\\
39.71	0.00798732230893232\\
39.72	0.00798732233636271\\
39.73	0.00798732236380798\\
39.74	0.00798732239126813\\
39.75	0.00798732241874318\\
39.76	0.00798732244623312\\
39.77	0.00798732247373798\\
39.78	0.00798732250125775\\
39.79	0.00798732252879244\\
39.8	0.00798732255634206\\
39.81	0.00798732258390661\\
39.82	0.00798732261148611\\
39.83	0.00798732263908055\\
39.84	0.00798732266668995\\
39.85	0.00798732269431432\\
39.86	0.00798732272195366\\
39.87	0.00798732274960797\\
39.88	0.00798732277727727\\
39.89	0.00798732280496156\\
39.9	0.00798732283266085\\
39.91	0.00798732286037514\\
39.92	0.00798732288810445\\
39.93	0.00798732291584877\\
39.94	0.00798732294360812\\
39.95	0.00798732297138251\\
39.96	0.00798732299917194\\
39.97	0.00798732302697641\\
39.98	0.00798732305479594\\
39.99	0.00798732308263053\\
40	0.00798732311048019\\
40.01	0.00798732313834492\\
};
\addplot [color=black,solid,forget plot]
  table[row sep=crcr]{%
40.01	0.00798732313834492\\
40.02	0.00798732316622474\\
40.03	0.00798732319411964\\
40.04	0.00798732322202964\\
40.05	0.00798732324995475\\
40.06	0.00798732327789497\\
40.07	0.0079873233058503\\
40.08	0.00798732333382076\\
40.09	0.00798732336180634\\
40.1	0.00798732338980707\\
40.11	0.00798732341782294\\
40.12	0.00798732344585397\\
40.13	0.00798732347390015\\
40.14	0.0079873235019615\\
40.15	0.00798732353003802\\
40.16	0.00798732355812972\\
40.17	0.00798732358623661\\
40.18	0.00798732361435869\\
40.19	0.00798732364249597\\
40.2	0.00798732367064846\\
40.21	0.00798732369881617\\
40.22	0.00798732372699909\\
40.23	0.00798732375519725\\
40.24	0.00798732378341064\\
40.25	0.00798732381163927\\
40.26	0.00798732383988315\\
40.27	0.00798732386814229\\
40.28	0.00798732389641669\\
40.29	0.00798732392470636\\
40.3	0.00798732395301131\\
40.31	0.00798732398133154\\
40.32	0.00798732400966706\\
40.33	0.00798732403801788\\
40.34	0.007987324066384\\
40.35	0.00798732409476544\\
40.36	0.00798732412316219\\
40.37	0.00798732415157427\\
40.38	0.00798732418000168\\
40.39	0.00798732420844443\\
40.4	0.00798732423690252\\
40.41	0.00798732426537597\\
40.42	0.00798732429386477\\
40.43	0.00798732432236894\\
40.44	0.00798732435088849\\
40.45	0.00798732437942342\\
40.46	0.00798732440797373\\
40.47	0.00798732443653944\\
40.48	0.00798732446512054\\
40.49	0.00798732449371706\\
40.5	0.00798732452232899\\
40.51	0.00798732455095634\\
40.52	0.00798732457959912\\
40.53	0.00798732460825733\\
40.54	0.00798732463693098\\
40.55	0.00798732466562009\\
40.56	0.00798732469432465\\
40.57	0.00798732472304467\\
40.58	0.00798732475178016\\
40.59	0.00798732478053112\\
40.6	0.00798732480929757\\
40.61	0.00798732483807951\\
40.62	0.00798732486687694\\
40.63	0.00798732489568988\\
40.64	0.00798732492451833\\
40.65	0.00798732495336229\\
40.66	0.00798732498222178\\
40.67	0.0079873250110968\\
40.68	0.00798732503998736\\
40.69	0.00798732506889346\\
40.7	0.00798732509781511\\
40.71	0.00798732512675232\\
40.72	0.00798732515570509\\
40.73	0.00798732518467343\\
40.74	0.00798732521365735\\
40.75	0.00798732524265686\\
40.76	0.00798732527167196\\
40.77	0.00798732530070265\\
40.78	0.00798732532974895\\
40.79	0.00798732535881086\\
40.8	0.0079873253878884\\
40.81	0.00798732541698155\\
40.82	0.00798732544609034\\
40.83	0.00798732547521477\\
40.84	0.00798732550435484\\
40.85	0.00798732553351057\\
40.86	0.00798732556268195\\
40.87	0.007987325591869\\
40.88	0.00798732562107173\\
40.89	0.00798732565029013\\
40.9	0.00798732567952421\\
40.91	0.007987325708774\\
40.92	0.00798732573803948\\
40.93	0.00798732576732066\\
40.94	0.00798732579661756\\
40.95	0.00798732582593018\\
40.96	0.00798732585525853\\
40.97	0.00798732588460261\\
40.98	0.00798732591396243\\
40.99	0.00798732594333799\\
41	0.00798732597272931\\
41.01	0.00798732600213639\\
41.02	0.00798732603155923\\
41.03	0.00798732606099785\\
41.04	0.00798732609045225\\
41.05	0.00798732611992243\\
41.06	0.00798732614940841\\
41.07	0.00798732617891019\\
41.08	0.00798732620842778\\
41.09	0.00798732623796117\\
41.1	0.00798732626751039\\
41.11	0.00798732629707544\\
41.12	0.00798732632665632\\
41.13	0.00798732635625304\\
41.14	0.0079873263858656\\
41.15	0.00798732641549402\\
41.16	0.0079873264451383\\
41.17	0.00798732647479845\\
41.18	0.00798732650447447\\
41.19	0.00798732653416637\\
41.2	0.00798732656387415\\
41.21	0.00798732659359783\\
41.22	0.00798732662333741\\
41.23	0.0079873266530929\\
41.24	0.0079873266828643\\
41.25	0.00798732671265162\\
41.26	0.00798732674245486\\
41.27	0.00798732677227404\\
41.28	0.00798732680210916\\
41.29	0.00798732683196022\\
41.3	0.00798732686182724\\
41.31	0.00798732689171022\\
41.32	0.00798732692160916\\
41.33	0.00798732695152408\\
41.34	0.00798732698145497\\
41.35	0.00798732701140186\\
41.36	0.00798732704136473\\
41.37	0.0079873270713436\\
41.38	0.00798732710133848\\
41.39	0.00798732713134937\\
41.4	0.00798732716137628\\
41.41	0.00798732719141922\\
41.42	0.00798732722147819\\
41.43	0.0079873272515532\\
41.44	0.00798732728164425\\
41.45	0.00798732731175135\\
41.46	0.00798732734187451\\
41.47	0.00798732737201374\\
41.48	0.00798732740216904\\
41.49	0.00798732743234042\\
41.5	0.00798732746252788\\
41.51	0.00798732749273143\\
41.52	0.00798732752295108\\
41.53	0.00798732755318683\\
41.54	0.0079873275834387\\
41.55	0.00798732761370668\\
41.56	0.00798732764399079\\
41.57	0.00798732767429102\\
41.58	0.0079873277046074\\
41.59	0.00798732773493991\\
41.6	0.00798732776528858\\
41.61	0.0079873277956534\\
41.62	0.00798732782603439\\
41.63	0.00798732785643155\\
41.64	0.00798732788684488\\
41.65	0.00798732791727439\\
41.66	0.0079873279477201\\
41.67	0.007987327978182\\
41.68	0.0079873280086601\\
41.69	0.00798732803915441\\
41.7	0.00798732806966493\\
41.71	0.00798732810019168\\
41.72	0.00798732813073465\\
41.73	0.00798732816129386\\
41.74	0.00798732819186931\\
41.75	0.00798732822246101\\
41.76	0.00798732825306896\\
41.77	0.00798732828369317\\
41.78	0.00798732831433365\\
41.79	0.0079873283449904\\
41.8	0.00798732837566343\\
41.81	0.00798732840635275\\
41.82	0.00798732843705836\\
41.83	0.00798732846778027\\
41.84	0.00798732849851849\\
41.85	0.00798732852927301\\
41.86	0.00798732856004386\\
41.87	0.00798732859083103\\
41.88	0.00798732862163453\\
41.89	0.00798732865245438\\
41.9	0.00798732868329056\\
41.91	0.0079873287141431\\
41.92	0.00798732874501199\\
41.93	0.00798732877589725\\
41.94	0.00798732880679887\\
41.95	0.00798732883771688\\
41.96	0.00798732886865126\\
41.97	0.00798732889960204\\
41.98	0.00798732893056921\\
41.99	0.00798732896155278\\
42	0.00798732899255276\\
42.01	0.00798732902356916\\
42.02	0.00798732905460198\\
42.03	0.00798732908565123\\
42.04	0.0079873291167169\\
42.05	0.00798732914779903\\
42.06	0.00798732917889759\\
42.07	0.00798732921001262\\
42.08	0.0079873292411441\\
42.09	0.00798732927229204\\
42.1	0.00798732930345646\\
42.11	0.00798732933463736\\
42.12	0.00798732936583475\\
42.13	0.00798732939704862\\
42.14	0.007987329428279\\
42.15	0.00798732945952588\\
42.16	0.00798732949078926\\
42.17	0.00798732952206917\\
42.18	0.0079873295533656\\
42.19	0.00798732958467856\\
42.2	0.00798732961600806\\
42.21	0.0079873296473541\\
42.22	0.00798732967871668\\
42.23	0.00798732971009583\\
42.24	0.00798732974149153\\
42.25	0.00798732977290381\\
42.26	0.00798732980433265\\
42.27	0.00798732983577808\\
42.28	0.0079873298672401\\
42.29	0.00798732989871871\\
42.3	0.00798732993021391\\
42.31	0.00798732996172573\\
42.32	0.00798732999325416\\
42.33	0.0079873300247992\\
42.34	0.00798733005636088\\
42.35	0.00798733008793918\\
42.36	0.00798733011953412\\
42.37	0.0079873301511457\\
42.38	0.00798733018277394\\
42.39	0.00798733021441883\\
42.4	0.00798733024608038\\
42.41	0.00798733027775861\\
42.42	0.00798733030945351\\
42.43	0.00798733034116509\\
42.44	0.00798733037289336\\
42.45	0.00798733040463832\\
42.46	0.00798733043639999\\
42.47	0.00798733046817836\\
42.48	0.00798733049997345\\
42.49	0.00798733053178526\\
42.5	0.00798733056361379\\
42.51	0.00798733059545905\\
42.52	0.00798733062732106\\
42.53	0.0079873306591998\\
42.54	0.00798733069109531\\
42.55	0.00798733072300756\\
42.56	0.00798733075493658\\
42.57	0.00798733078688238\\
42.58	0.00798733081884494\\
42.59	0.0079873308508243\\
42.6	0.00798733088282044\\
42.61	0.00798733091483337\\
42.62	0.00798733094686311\\
42.63	0.00798733097890965\\
42.64	0.00798733101097301\\
42.65	0.00798733104305319\\
42.66	0.0079873310751502\\
42.67	0.00798733110726404\\
42.68	0.00798733113939472\\
42.69	0.00798733117154224\\
42.7	0.00798733120370662\\
42.71	0.00798733123588785\\
42.72	0.00798733126808595\\
42.73	0.00798733130030092\\
42.74	0.00798733133253277\\
42.75	0.0079873313647815\\
42.76	0.00798733139704711\\
42.77	0.00798733142932963\\
42.78	0.00798733146162904\\
42.79	0.00798733149394537\\
42.8	0.00798733152627861\\
42.81	0.00798733155862877\\
42.82	0.00798733159099585\\
42.83	0.00798733162337988\\
42.84	0.00798733165578083\\
42.85	0.00798733168819874\\
42.86	0.0079873317206336\\
42.87	0.00798733175308541\\
42.88	0.00798733178555419\\
42.89	0.00798733181803994\\
42.9	0.00798733185054267\\
42.91	0.00798733188306238\\
42.92	0.00798733191559908\\
42.93	0.00798733194815278\\
42.94	0.00798733198072347\\
42.95	0.00798733201331118\\
42.96	0.0079873320459159\\
42.97	0.00798733207853764\\
42.98	0.00798733211117641\\
42.99	0.00798733214383221\\
43	0.00798733217650505\\
43.01	0.00798733220919494\\
43.02	0.00798733224190188\\
43.03	0.00798733227462588\\
43.04	0.00798733230736694\\
43.05	0.00798733234012507\\
43.06	0.00798733237290028\\
43.07	0.00798733240569257\\
43.08	0.00798733243850195\\
43.09	0.00798733247132843\\
43.1	0.00798733250417201\\
43.11	0.00798733253703269\\
43.12	0.0079873325699105\\
43.13	0.00798733260280542\\
43.14	0.00798733263571747\\
43.15	0.00798733266864665\\
43.16	0.00798733270159297\\
43.17	0.00798733273455644\\
43.18	0.00798733276753705\\
43.19	0.00798733280053483\\
43.2	0.00798733283354977\\
43.21	0.00798733286658188\\
43.22	0.00798733289963117\\
43.23	0.00798733293269764\\
43.24	0.00798733296578131\\
43.25	0.00798733299888216\\
43.26	0.00798733303200022\\
43.27	0.00798733306513549\\
43.28	0.00798733309828797\\
43.29	0.00798733313145767\\
43.3	0.00798733316464459\\
43.31	0.00798733319784876\\
43.32	0.00798733323107016\\
43.33	0.0079873332643088\\
43.34	0.0079873332975647\\
43.35	0.00798733333083785\\
43.36	0.00798733336412827\\
43.37	0.00798733339743596\\
43.38	0.00798733343076093\\
43.39	0.00798733346410318\\
43.4	0.00798733349746272\\
43.41	0.00798733353083956\\
43.42	0.00798733356423369\\
43.43	0.00798733359764514\\
43.44	0.0079873336310739\\
43.45	0.00798733366451998\\
43.46	0.00798733369798338\\
43.47	0.00798733373146412\\
43.48	0.0079873337649622\\
43.49	0.00798733379847763\\
43.5	0.00798733383201041\\
43.51	0.00798733386556054\\
43.52	0.00798733389912804\\
43.53	0.00798733393271291\\
43.54	0.00798733396631516\\
43.55	0.00798733399993479\\
43.56	0.00798733403357182\\
43.57	0.00798733406722623\\
43.58	0.00798733410089805\\
43.59	0.00798733413458728\\
43.6	0.00798733416829393\\
43.61	0.00798733420201799\\
43.62	0.00798733423575948\\
43.63	0.00798733426951841\\
43.64	0.00798733430329478\\
43.65	0.00798733433708859\\
43.66	0.00798733437089985\\
43.67	0.00798733440472858\\
43.68	0.00798733443857477\\
43.69	0.00798733447243843\\
43.7	0.00798733450631957\\
43.71	0.00798733454021819\\
43.72	0.0079873345741343\\
43.73	0.00798733460806792\\
43.74	0.00798733464201903\\
43.75	0.00798733467598766\\
43.76	0.0079873347099738\\
43.77	0.00798733474397746\\
43.78	0.00798733477799865\\
43.79	0.00798733481203738\\
43.8	0.00798733484609364\\
43.81	0.00798733488016746\\
43.82	0.00798733491425883\\
43.83	0.00798733494836776\\
43.84	0.00798733498249425\\
43.85	0.00798733501663833\\
43.86	0.00798733505079998\\
43.87	0.00798733508497921\\
43.88	0.00798733511917604\\
43.89	0.00798733515339047\\
43.9	0.0079873351876225\\
43.91	0.00798733522187214\\
43.92	0.0079873352561394\\
43.93	0.00798733529042429\\
43.94	0.0079873353247268\\
43.95	0.00798733535904696\\
43.96	0.00798733539338475\\
43.97	0.0079873354277402\\
43.98	0.0079873354621133\\
43.99	0.00798733549650407\\
44	0.00798733553091251\\
44.01	0.00798733556533862\\
44.02	0.00798733559978241\\
44.03	0.00798733563424389\\
44.04	0.00798733566872307\\
44.05	0.00798733570321995\\
44.06	0.00798733573773453\\
44.07	0.00798733577226684\\
44.08	0.00798733580681686\\
44.09	0.00798733584138461\\
44.1	0.00798733587597009\\
44.11	0.00798733591057331\\
44.12	0.00798733594519428\\
44.13	0.00798733597983301\\
44.14	0.00798733601448949\\
44.15	0.00798733604916374\\
44.16	0.00798733608385576\\
44.17	0.00798733611856556\\
44.18	0.00798733615329315\\
44.19	0.00798733618803853\\
44.2	0.00798733622280171\\
44.21	0.0079873362575827\\
44.22	0.0079873362923815\\
44.23	0.00798733632719811\\
44.24	0.00798733636203255\\
44.25	0.00798733639688483\\
44.26	0.00798733643175494\\
44.27	0.0079873364666429\\
44.28	0.0079873365015487\\
44.29	0.00798733653647237\\
44.3	0.0079873365714139\\
44.31	0.00798733660637331\\
44.32	0.00798733664135059\\
44.33	0.00798733667634575\\
44.34	0.00798733671135881\\
44.35	0.00798733674638976\\
44.36	0.00798733678143862\\
44.37	0.0079873368165054\\
44.38	0.00798733685159008\\
44.39	0.0079873368866927\\
44.4	0.00798733692181324\\
44.41	0.00798733695695173\\
44.42	0.00798733699210815\\
44.43	0.00798733702728253\\
44.44	0.00798733706247487\\
44.45	0.00798733709768517\\
44.46	0.00798733713291344\\
44.47	0.0079873371681597\\
44.48	0.00798733720342393\\
44.49	0.00798733723870616\\
44.5	0.00798733727400639\\
44.51	0.00798733730932462\\
44.52	0.00798733734466087\\
44.53	0.00798733738001513\\
44.54	0.00798733741538742\\
44.55	0.00798733745077774\\
44.56	0.00798733748618611\\
44.57	0.00798733752161251\\
44.58	0.00798733755705697\\
44.59	0.0079873375925195\\
44.6	0.00798733762800008\\
44.61	0.00798733766349874\\
44.62	0.00798733769901548\\
44.63	0.00798733773455031\\
44.64	0.00798733777010324\\
44.65	0.00798733780567426\\
44.66	0.00798733784126339\\
44.67	0.00798733787687064\\
44.68	0.00798733791249601\\
44.69	0.00798733794813951\\
44.7	0.00798733798380114\\
44.71	0.00798733801948092\\
44.72	0.00798733805517884\\
44.73	0.00798733809089493\\
44.74	0.00798733812662917\\
44.75	0.00798733816238159\\
44.76	0.00798733819815218\\
44.77	0.00798733823394096\\
44.78	0.00798733826974794\\
44.79	0.00798733830557311\\
44.8	0.00798733834141648\\
44.81	0.00798733837727807\\
44.82	0.00798733841315788\\
44.83	0.00798733844905592\\
44.84	0.00798733848497219\\
44.85	0.0079873385209067\\
44.86	0.00798733855685946\\
44.87	0.00798733859283048\\
44.88	0.00798733862881976\\
44.89	0.00798733866482731\\
44.9	0.00798733870085313\\
44.91	0.00798733873689724\\
44.92	0.00798733877295965\\
44.93	0.00798733880904035\\
44.94	0.00798733884513936\\
44.95	0.00798733888125668\\
44.96	0.00798733891739232\\
44.97	0.00798733895354629\\
44.98	0.00798733898971859\\
44.99	0.00798733902590924\\
45	0.00798733906211824\\
45.01	0.00798733909834559\\
45.02	0.00798733913459131\\
45.03	0.0079873391708554\\
45.04	0.00798733920713787\\
45.05	0.00798733924343872\\
45.06	0.00798733927975797\\
45.07	0.00798733931609562\\
45.08	0.00798733935245168\\
45.09	0.00798733938882616\\
45.1	0.00798733942521906\\
45.11	0.00798733946163039\\
45.12	0.00798733949806015\\
45.13	0.00798733953450837\\
45.14	0.00798733957097504\\
45.15	0.00798733960746017\\
45.16	0.00798733964396376\\
45.17	0.00798733968048584\\
45.18	0.00798733971702639\\
45.19	0.00798733975358544\\
45.2	0.00798733979016299\\
45.21	0.00798733982675904\\
45.22	0.00798733986337361\\
45.23	0.0079873399000067\\
45.24	0.00798733993665832\\
45.25	0.00798733997332848\\
45.26	0.00798734001001718\\
45.27	0.00798734004672444\\
45.28	0.00798734008345025\\
45.29	0.00798734012019463\\
45.3	0.00798734015695759\\
45.31	0.00798734019373913\\
45.32	0.00798734023053926\\
45.33	0.007987340267358\\
45.34	0.00798734030419533\\
45.35	0.00798734034105129\\
45.36	0.00798734037792586\\
45.37	0.00798734041481907\\
45.38	0.00798734045173092\\
45.39	0.00798734048866141\\
45.4	0.00798734052561055\\
45.41	0.00798734056257836\\
45.42	0.00798734059956484\\
45.43	0.00798734063657\\
45.44	0.00798734067359384\\
45.45	0.00798734071063638\\
45.46	0.00798734074769762\\
45.47	0.00798734078477757\\
45.48	0.00798734082187624\\
45.49	0.00798734085899363\\
45.5	0.00798734089612976\\
45.51	0.00798734093328463\\
45.52	0.00798734097045825\\
45.53	0.00798734100765064\\
45.54	0.00798734104486179\\
45.55	0.00798734108209171\\
45.56	0.00798734111934042\\
45.57	0.00798734115660792\\
45.58	0.00798734119389422\\
45.59	0.00798734123119933\\
45.6	0.00798734126852325\\
45.61	0.00798734130586601\\
45.62	0.00798734134322759\\
45.63	0.00798734138060802\\
45.64	0.00798734141800729\\
45.65	0.00798734145542543\\
45.66	0.00798734149286243\\
45.67	0.00798734153031831\\
45.68	0.00798734156779307\\
45.69	0.00798734160528673\\
45.7	0.00798734164279928\\
45.71	0.00798734168033075\\
45.72	0.00798734171788113\\
45.73	0.00798734175545044\\
45.74	0.00798734179303869\\
45.75	0.00798734183064587\\
45.76	0.00798734186827202\\
45.77	0.00798734190591712\\
45.78	0.00798734194358119\\
45.79	0.00798734198126424\\
45.8	0.00798734201896628\\
45.81	0.00798734205668732\\
45.82	0.00798734209442736\\
45.83	0.00798734213218641\\
45.84	0.00798734216996449\\
45.85	0.00798734220776159\\
45.86	0.00798734224557774\\
45.87	0.00798734228341294\\
45.88	0.00798734232126719\\
45.89	0.00798734235914052\\
45.9	0.00798734239703291\\
45.91	0.0079873424349444\\
45.92	0.00798734247287497\\
45.93	0.00798734251082465\\
45.94	0.00798734254879344\\
45.95	0.00798734258678136\\
45.96	0.0079873426247884\\
45.97	0.00798734266281458\\
45.98	0.00798734270085991\\
45.99	0.0079873427389244\\
46	0.00798734277700806\\
46.01	0.00798734281511089\\
46.02	0.00798734285323291\\
46.03	0.00798734289137412\\
46.04	0.00798734292953453\\
46.05	0.00798734296771416\\
46.06	0.00798734300591301\\
46.07	0.0079873430441311\\
46.08	0.00798734308236842\\
46.09	0.007987343120625\\
46.1	0.00798734315890083\\
46.11	0.00798734319719594\\
46.12	0.00798734323551032\\
46.13	0.00798734327384399\\
46.14	0.00798734331219697\\
46.15	0.00798734335056924\\
46.16	0.00798734338896084\\
46.17	0.00798734342737177\\
46.18	0.00798734346580203\\
46.19	0.00798734350425163\\
46.2	0.0079873435427206\\
46.21	0.00798734358120893\\
46.22	0.00798734361971663\\
46.23	0.00798734365824372\\
46.24	0.00798734369679021\\
46.25	0.0079873437353561\\
46.26	0.00798734377394141\\
46.27	0.00798734381254614\\
46.28	0.00798734385117031\\
46.29	0.00798734388981391\\
46.3	0.00798734392847698\\
46.31	0.00798734396715951\\
46.32	0.00798734400586152\\
46.33	0.00798734404458301\\
46.34	0.00798734408332399\\
46.35	0.00798734412208448\\
46.36	0.00798734416086449\\
46.37	0.00798734419966402\\
46.38	0.00798734423848309\\
46.39	0.0079873442773217\\
46.4	0.00798734431617987\\
46.41	0.00798734435505761\\
46.42	0.00798734439395492\\
46.43	0.00798734443287182\\
46.44	0.00798734447180832\\
46.45	0.00798734451076442\\
46.46	0.00798734454974015\\
46.47	0.0079873445887355\\
46.48	0.00798734462775049\\
46.49	0.00798734466678513\\
46.5	0.00798734470583944\\
46.51	0.00798734474491341\\
46.52	0.00798734478400707\\
46.53	0.00798734482312041\\
46.54	0.00798734486225347\\
46.55	0.00798734490140623\\
46.56	0.00798734494057872\\
46.57	0.00798734497977094\\
46.58	0.00798734501898291\\
46.59	0.00798734505821464\\
46.6	0.00798734509746613\\
46.61	0.00798734513673741\\
46.62	0.00798734517602847\\
46.63	0.00798734521533933\\
46.64	0.00798734525467001\\
46.65	0.0079873452940205\\
46.66	0.00798734533339083\\
46.67	0.00798734537278101\\
46.68	0.00798734541219104\\
46.69	0.00798734545162094\\
46.7	0.00798734549107071\\
46.71	0.00798734553054038\\
46.72	0.00798734557002994\\
46.73	0.00798734560953942\\
46.74	0.00798734564906881\\
46.75	0.00798734568861815\\
46.76	0.00798734572818743\\
46.77	0.00798734576777666\\
46.78	0.00798734580738587\\
46.79	0.00798734584701505\\
46.8	0.00798734588666422\\
46.81	0.0079873459263334\\
46.82	0.0079873459660226\\
46.83	0.00798734600573181\\
46.84	0.00798734604546107\\
46.85	0.00798734608521037\\
46.86	0.00798734612497974\\
46.87	0.00798734616476917\\
46.88	0.0079873462045787\\
46.89	0.00798734624440831\\
46.9	0.00798734628425804\\
46.91	0.00798734632412788\\
46.92	0.00798734636401786\\
46.93	0.00798734640392798\\
46.94	0.00798734644385825\\
46.95	0.0079873464838087\\
46.96	0.00798734652377932\\
46.97	0.00798734656377013\\
46.98	0.00798734660378114\\
46.99	0.00798734664381237\\
47	0.00798734668386383\\
47.01	0.00798734672393553\\
47.02	0.00798734676402748\\
47.03	0.00798734680413969\\
47.04	0.00798734684427218\\
47.05	0.00798734688442496\\
47.06	0.00798734692459803\\
47.07	0.00798734696479143\\
47.08	0.00798734700500514\\
47.09	0.0079873470452392\\
47.1	0.0079873470854936\\
47.11	0.00798734712576837\\
47.12	0.00798734716606351\\
47.13	0.00798734720637904\\
47.14	0.00798734724671497\\
47.15	0.00798734728707131\\
47.16	0.00798734732744808\\
47.17	0.00798734736784529\\
47.18	0.00798734740826295\\
47.19	0.00798734744870107\\
47.2	0.00798734748915966\\
47.21	0.00798734752963875\\
47.22	0.00798734757013834\\
47.23	0.00798734761065844\\
47.24	0.00798734765119907\\
47.25	0.00798734769176025\\
47.26	0.00798734773234197\\
47.27	0.00798734777294426\\
47.28	0.00798734781356713\\
47.29	0.00798734785421059\\
47.3	0.00798734789487466\\
47.31	0.00798734793555935\\
47.32	0.00798734797626467\\
47.33	0.00798734801699063\\
47.34	0.00798734805773726\\
47.35	0.00798734809850455\\
47.36	0.00798734813929253\\
47.37	0.0079873481801012\\
47.38	0.00798734822093059\\
47.39	0.0079873482617807\\
47.4	0.00798734830265155\\
47.41	0.00798734834354316\\
47.42	0.00798734838445552\\
47.43	0.00798734842538867\\
47.44	0.00798734846634261\\
47.45	0.00798734850731735\\
47.46	0.00798734854831291\\
47.47	0.00798734858932931\\
47.48	0.00798734863036655\\
47.49	0.00798734867142465\\
47.5	0.00798734871250362\\
47.51	0.00798734875360348\\
47.52	0.00798734879472425\\
47.53	0.00798734883586593\\
47.54	0.00798734887702853\\
47.55	0.00798734891821208\\
47.56	0.00798734895941659\\
47.57	0.00798734900064206\\
47.58	0.00798734904188853\\
47.59	0.00798734908315599\\
47.6	0.00798734912444446\\
47.61	0.00798734916575396\\
47.62	0.00798734920708449\\
47.63	0.00798734924843609\\
47.64	0.00798734928980875\\
47.65	0.00798734933120249\\
47.66	0.00798734937261734\\
47.67	0.00798734941405329\\
47.68	0.00798734945551037\\
47.69	0.00798734949698859\\
47.7	0.00798734953848796\\
47.71	0.0079873495800085\\
47.72	0.00798734962155022\\
47.73	0.00798734966311315\\
47.74	0.00798734970469728\\
47.75	0.00798734974630264\\
47.76	0.00798734978792923\\
47.77	0.00798734982957709\\
47.78	0.00798734987124621\\
47.79	0.00798734991293662\\
47.8	0.00798734995464833\\
47.81	0.00798734999638134\\
47.82	0.00798735003813569\\
47.83	0.00798735007991138\\
47.84	0.00798735012170843\\
47.85	0.00798735016352685\\
47.86	0.00798735020536665\\
47.87	0.00798735024722786\\
47.88	0.00798735028911049\\
47.89	0.00798735033101454\\
47.9	0.00798735037294004\\
47.91	0.00798735041488701\\
47.92	0.00798735045685544\\
47.93	0.00798735049884537\\
47.94	0.00798735054085681\\
47.95	0.00798735058288976\\
47.96	0.00798735062494425\\
47.97	0.00798735066702029\\
47.98	0.0079873507091179\\
47.99	0.00798735075123709\\
48	0.00798735079337787\\
48.01	0.00798735083554027\\
48.02	0.00798735087772429\\
48.03	0.00798735091992996\\
48.04	0.00798735096215728\\
48.05	0.00798735100440627\\
48.06	0.00798735104667695\\
48.07	0.00798735108896933\\
48.08	0.00798735113128343\\
48.09	0.00798735117361927\\
48.1	0.00798735121597685\\
48.11	0.00798735125835619\\
48.12	0.00798735130075732\\
48.13	0.00798735134318024\\
48.14	0.00798735138562497\\
48.15	0.00798735142809153\\
48.16	0.00798735147057993\\
48.17	0.00798735151309018\\
48.18	0.00798735155562231\\
48.19	0.00798735159817632\\
48.2	0.00798735164075224\\
48.21	0.00798735168335008\\
48.22	0.00798735172596985\\
48.23	0.00798735176861157\\
48.24	0.00798735181127526\\
48.25	0.00798735185396093\\
48.26	0.0079873518966686\\
48.27	0.00798735193939828\\
48.28	0.00798735198214999\\
48.29	0.00798735202492374\\
48.3	0.00798735206771956\\
48.31	0.00798735211053745\\
48.32	0.00798735215337744\\
48.33	0.00798735219623953\\
48.34	0.00798735223912374\\
48.35	0.0079873522820301\\
48.36	0.00798735232495862\\
48.37	0.0079873523679093\\
48.38	0.00798735241088217\\
48.39	0.00798735245387725\\
48.4	0.00798735249689455\\
48.41	0.00798735253993408\\
48.42	0.00798735258299586\\
48.43	0.00798735262607992\\
48.44	0.00798735266918625\\
48.45	0.00798735271231489\\
48.46	0.00798735275546585\\
48.47	0.00798735279863913\\
48.48	0.00798735284183477\\
48.49	0.00798735288505277\\
48.5	0.00798735292829315\\
48.51	0.00798735297155593\\
48.52	0.00798735301484113\\
48.53	0.00798735305814875\\
48.54	0.00798735310147882\\
48.55	0.00798735314483135\\
48.56	0.00798735318820637\\
48.57	0.00798735323160387\\
48.58	0.0079873532750239\\
48.59	0.00798735331846644\\
48.6	0.00798735336193154\\
48.61	0.00798735340541919\\
48.62	0.00798735344892942\\
48.63	0.00798735349246225\\
48.64	0.00798735353601769\\
48.65	0.00798735357959575\\
48.66	0.00798735362319646\\
48.67	0.00798735366681983\\
48.68	0.00798735371046588\\
48.69	0.00798735375413462\\
48.7	0.00798735379782607\\
48.71	0.00798735384154024\\
48.72	0.00798735388527717\\
48.73	0.00798735392903685\\
48.74	0.00798735397281931\\
48.75	0.00798735401662456\\
48.76	0.00798735406045262\\
48.77	0.00798735410430351\\
48.78	0.00798735414817725\\
48.79	0.00798735419207385\\
48.8	0.00798735423599332\\
48.81	0.00798735427993569\\
48.82	0.00798735432390097\\
48.83	0.00798735436788918\\
48.84	0.00798735441190034\\
48.85	0.00798735445593446\\
48.86	0.00798735449999156\\
48.87	0.00798735454407165\\
48.88	0.00798735458817476\\
48.89	0.00798735463230091\\
48.9	0.0079873546764501\\
48.91	0.00798735472062235\\
48.92	0.00798735476481769\\
48.93	0.00798735480903612\\
48.94	0.00798735485327767\\
48.95	0.00798735489754236\\
48.96	0.0079873549418302\\
48.97	0.0079873549861412\\
48.98	0.00798735503047539\\
48.99	0.00798735507483279\\
49	0.0079873551192134\\
49.01	0.00798735516361725\\
49.02	0.00798735520804436\\
49.03	0.00798735525249474\\
49.04	0.00798735529696841\\
49.05	0.00798735534146538\\
49.06	0.00798735538598568\\
49.07	0.00798735543052933\\
49.08	0.00798735547509633\\
49.09	0.00798735551968671\\
49.1	0.00798735556430048\\
49.11	0.00798735560893767\\
49.12	0.00798735565359828\\
49.13	0.00798735569828234\\
49.14	0.00798735574298987\\
49.15	0.00798735578772088\\
49.16	0.00798735583247539\\
49.17	0.00798735587725342\\
49.18	0.00798735592205499\\
49.19	0.00798735596688011\\
49.2	0.0079873560117288\\
49.21	0.00798735605660108\\
49.22	0.00798735610149696\\
49.23	0.00798735614641648\\
49.24	0.00798735619135963\\
49.25	0.00798735623632645\\
49.26	0.00798735628131694\\
49.27	0.00798735632633113\\
49.28	0.00798735637136903\\
49.29	0.00798735641643067\\
49.3	0.00798735646151606\\
49.31	0.00798735650662522\\
49.32	0.00798735655175816\\
49.33	0.00798735659691491\\
49.34	0.00798735664209548\\
49.35	0.00798735668729989\\
49.36	0.00798735673252816\\
49.37	0.00798735677778031\\
49.38	0.00798735682305635\\
49.39	0.00798735686835631\\
49.4	0.0079873569136802\\
49.41	0.00798735695902804\\
49.42	0.00798735700439985\\
49.43	0.00798735704979565\\
49.44	0.00798735709521545\\
49.45	0.00798735714065928\\
49.46	0.00798735718612715\\
49.47	0.00798735723161908\\
49.48	0.00798735727713508\\
49.49	0.00798735732267519\\
49.5	0.00798735736823941\\
49.51	0.00798735741382777\\
49.52	0.00798735745944028\\
49.53	0.00798735750507697\\
49.54	0.00798735755073784\\
49.55	0.00798735759642292\\
49.56	0.00798735764213223\\
49.57	0.00798735768786579\\
49.58	0.00798735773362361\\
49.59	0.00798735777940572\\
49.6	0.00798735782521213\\
49.61	0.00798735787104286\\
49.62	0.00798735791689793\\
49.63	0.00798735796277736\\
49.64	0.00798735800868117\\
49.65	0.00798735805460938\\
49.66	0.007987358100562\\
49.67	0.00798735814653906\\
49.68	0.00798735819254057\\
49.69	0.00798735823856655\\
49.7	0.00798735828461703\\
49.71	0.00798735833069202\\
49.72	0.00798735837679154\\
49.73	0.0079873584229156\\
49.74	0.00798735846906424\\
49.75	0.00798735851523746\\
49.76	0.00798735856143529\\
49.77	0.00798735860765775\\
49.78	0.00798735865390485\\
49.79	0.00798735870017662\\
49.8	0.00798735874647307\\
49.81	0.00798735879279422\\
49.82	0.0079873588391401\\
49.83	0.00798735888551072\\
49.84	0.0079873589319061\\
49.85	0.00798735897832626\\
49.86	0.00798735902477122\\
49.87	0.007987359071241\\
49.88	0.00798735911773562\\
49.89	0.0079873591642551\\
49.9	0.00798735921079946\\
49.91	0.00798735925736872\\
49.92	0.00798735930396289\\
49.93	0.007987359350582\\
49.94	0.00798735939722607\\
49.95	0.00798735944389512\\
49.96	0.00798735949058916\\
49.97	0.00798735953730822\\
49.98	0.00798735958405231\\
49.99	0.00798735963082146\\
50	0.00798735967761569\\
50.01	0.00798735972443501\\
50.02	0.00798735977127945\\
50.03	0.00798735981814902\\
50.04	0.00798735986504375\\
50.05	0.00798735991196365\\
50.06	0.00798735995890875\\
50.07	0.00798736000587907\\
50.08	0.00798736005287462\\
50.09	0.00798736009989543\\
50.1	0.00798736014694151\\
50.11	0.00798736019401289\\
50.12	0.00798736024110958\\
50.13	0.00798736028823162\\
50.14	0.007987360335379\\
50.15	0.00798736038255177\\
50.16	0.00798736042974994\\
50.17	0.00798736047697352\\
50.18	0.00798736052422254\\
50.19	0.00798736057149702\\
50.2	0.00798736061879697\\
50.21	0.00798736066612243\\
50.22	0.0079873607134734\\
50.23	0.00798736076084992\\
50.24	0.007987360808252\\
50.25	0.00798736085567966\\
50.26	0.00798736090313292\\
50.27	0.00798736095061181\\
50.28	0.00798736099811634\\
50.29	0.00798736104564653\\
50.3	0.0079873610932024\\
50.31	0.00798736114078399\\
50.32	0.0079873611883913\\
50.33	0.00798736123602435\\
50.34	0.00798736128368318\\
50.35	0.0079873613313678\\
50.36	0.00798736137907822\\
50.37	0.00798736142681447\\
50.38	0.00798736147457658\\
50.39	0.00798736152236456\\
50.4	0.00798736157017844\\
50.41	0.00798736161801823\\
50.42	0.00798736166588395\\
50.43	0.00798736171377564\\
50.44	0.0079873617616933\\
50.45	0.00798736180963696\\
50.46	0.00798736185760664\\
50.47	0.00798736190560236\\
50.48	0.00798736195362415\\
50.49	0.00798736200167202\\
50.5	0.007987362049746\\
50.51	0.0079873620978461\\
50.52	0.00798736214597236\\
50.53	0.00798736219412478\\
50.54	0.0079873622423034\\
50.55	0.00798736229050823\\
50.56	0.00798736233873929\\
50.57	0.00798736238699661\\
50.58	0.00798736243528021\\
50.59	0.00798736248359011\\
50.6	0.00798736253192633\\
50.61	0.00798736258028889\\
50.62	0.00798736262867782\\
50.63	0.00798736267709313\\
50.64	0.00798736272553485\\
50.65	0.007987362774003\\
50.66	0.0079873628224976\\
50.67	0.00798736287101868\\
50.68	0.00798736291956625\\
50.69	0.00798736296814034\\
50.7	0.00798736301674097\\
50.71	0.00798736306536817\\
50.72	0.00798736311402194\\
50.73	0.00798736316270232\\
50.74	0.00798736321140933\\
50.75	0.00798736326014299\\
50.76	0.00798736330890332\\
50.77	0.00798736335769035\\
50.78	0.00798736340650409\\
50.79	0.00798736345534457\\
50.8	0.00798736350421181\\
50.81	0.00798736355310584\\
50.82	0.00798736360202667\\
50.83	0.00798736365097433\\
50.84	0.00798736369994885\\
50.85	0.00798736374895023\\
50.86	0.00798736379797851\\
50.87	0.00798736384703371\\
50.88	0.00798736389611585\\
50.89	0.00798736394522496\\
50.9	0.00798736399436105\\
50.91	0.00798736404352415\\
50.92	0.00798736409271428\\
50.93	0.00798736414193147\\
50.94	0.00798736419117573\\
50.95	0.0079873642404471\\
50.96	0.00798736428974558\\
50.97	0.00798736433907122\\
50.98	0.00798736438842402\\
50.99	0.00798736443780401\\
51	0.00798736448721122\\
51.01	0.00798736453664567\\
51.02	0.00798736458610737\\
51.03	0.00798736463559637\\
51.04	0.00798736468511266\\
51.05	0.00798736473465629\\
51.06	0.00798736478422728\\
51.07	0.00798736483382564\\
51.08	0.0079873648834514\\
51.09	0.00798736493310458\\
51.1	0.00798736498278521\\
51.11	0.00798736503249331\\
51.12	0.0079873650822289\\
51.13	0.00798736513199202\\
51.14	0.00798736518178267\\
51.15	0.00798736523160088\\
51.16	0.00798736528144669\\
51.17	0.0079873653313201\\
51.18	0.00798736538122115\\
51.19	0.00798736543114986\\
51.2	0.00798736548110626\\
51.21	0.00798736553109035\\
51.22	0.00798736558110218\\
51.23	0.00798736563114176\\
51.24	0.00798736568120912\\
51.25	0.00798736573130428\\
51.26	0.00798736578142727\\
51.27	0.0079873658315781\\
51.28	0.00798736588175681\\
51.29	0.00798736593196342\\
51.3	0.00798736598219795\\
51.31	0.00798736603246042\\
51.32	0.00798736608275086\\
51.33	0.0079873661330693\\
51.34	0.00798736618341576\\
51.35	0.00798736623379026\\
51.36	0.00798736628419283\\
51.37	0.00798736633462348\\
51.38	0.00798736638508226\\
51.39	0.00798736643556917\\
51.4	0.00798736648608425\\
51.41	0.00798736653662752\\
51.42	0.007987366587199\\
51.43	0.00798736663779872\\
51.44	0.0079873666884267\\
51.45	0.00798736673908297\\
51.46	0.00798736678976756\\
51.47	0.00798736684048048\\
51.48	0.00798736689122176\\
51.49	0.00798736694199143\\
51.5	0.00798736699278951\\
51.51	0.00798736704361602\\
51.52	0.007987367094471\\
51.53	0.00798736714535446\\
51.54	0.00798736719626644\\
51.55	0.00798736724720695\\
51.56	0.00798736729817602\\
51.57	0.00798736734917368\\
51.58	0.00798736740019995\\
51.59	0.00798736745125486\\
51.6	0.00798736750233843\\
51.61	0.00798736755345069\\
51.62	0.00798736760459166\\
51.63	0.00798736765576137\\
51.64	0.00798736770695984\\
51.65	0.0079873677581871\\
51.66	0.00798736780944318\\
51.67	0.0079873678607281\\
51.68	0.00798736791204188\\
51.69	0.00798736796338455\\
51.7	0.00798736801475614\\
51.71	0.00798736806615668\\
51.72	0.00798736811758618\\
51.73	0.00798736816904468\\
51.74	0.00798736822053219\\
51.75	0.00798736827204875\\
51.76	0.00798736832359439\\
51.77	0.00798736837516912\\
51.78	0.00798736842677297\\
51.79	0.00798736847840598\\
51.8	0.00798736853006816\\
51.81	0.00798736858175954\\
51.82	0.00798736863348016\\
51.83	0.00798736868523002\\
51.84	0.00798736873700917\\
51.85	0.00798736878881762\\
51.86	0.00798736884065541\\
51.87	0.00798736889252255\\
51.88	0.00798736894441908\\
51.89	0.00798736899634503\\
51.9	0.00798736904830041\\
51.91	0.00798736910028526\\
51.92	0.0079873691522996\\
51.93	0.00798736920434346\\
51.94	0.00798736925641687\\
51.95	0.00798736930851984\\
51.96	0.00798736936065242\\
51.97	0.00798736941281462\\
51.98	0.00798736946500648\\
51.99	0.00798736951722801\\
52	0.00798736956947925\\
52.01	0.00798736962176022\\
52.02	0.00798736967407095\\
52.03	0.00798736972641148\\
52.04	0.00798736977878181\\
52.05	0.00798736983118199\\
52.06	0.00798736988361203\\
52.07	0.00798736993607197\\
52.08	0.00798736998856184\\
52.09	0.00798737004108165\\
52.1	0.00798737009363145\\
52.11	0.00798737014621125\\
52.12	0.00798737019882108\\
52.13	0.00798737025146097\\
52.14	0.00798737030413095\\
52.15	0.00798737035683105\\
52.16	0.00798737040956129\\
52.17	0.0079873704623217\\
52.18	0.00798737051511231\\
52.19	0.00798737056793315\\
52.2	0.00798737062078424\\
52.21	0.00798737067366561\\
52.22	0.00798737072657729\\
52.23	0.00798737077951931\\
52.24	0.0079873708324917\\
52.25	0.00798737088549448\\
52.26	0.00798737093852768\\
52.27	0.00798737099159133\\
52.28	0.00798737104468547\\
52.29	0.0079873710978101\\
52.3	0.00798737115096528\\
52.31	0.00798737120415102\\
52.32	0.00798737125736734\\
52.33	0.00798737131061429\\
52.34	0.00798737136389189\\
52.35	0.00798737141720017\\
52.36	0.00798737147053915\\
52.37	0.00798737152390887\\
52.38	0.00798737157730935\\
52.39	0.00798737163074062\\
52.4	0.00798737168420271\\
52.41	0.00798737173769565\\
52.42	0.00798737179121948\\
52.43	0.0079873718447742\\
52.44	0.00798737189835987\\
52.45	0.0079873719519765\\
52.46	0.00798737200562412\\
52.47	0.00798737205930277\\
52.48	0.00798737211301247\\
52.49	0.00798737216675325\\
52.5	0.00798737222052514\\
52.51	0.00798737227432818\\
52.52	0.00798737232816238\\
52.53	0.00798737238202778\\
52.54	0.00798737243592441\\
52.55	0.0079873724898523\\
52.56	0.00798737254381148\\
52.57	0.00798737259780197\\
52.58	0.00798737265182382\\
52.59	0.00798737270587703\\
52.6	0.00798737275996166\\
52.61	0.00798737281407772\\
52.62	0.00798737286822524\\
52.63	0.00798737292240427\\
52.64	0.00798737297661482\\
52.65	0.00798737303085692\\
52.66	0.00798737308513061\\
52.67	0.00798737313943592\\
52.68	0.00798737319377287\\
52.69	0.0079873732481415\\
52.7	0.00798737330254184\\
52.71	0.00798737335697391\\
52.72	0.00798737341143775\\
52.73	0.0079873734659334\\
52.74	0.00798737352046086\\
52.75	0.00798737357502019\\
52.76	0.00798737362961141\\
52.77	0.00798737368423454\\
52.78	0.00798737373888963\\
52.79	0.0079873737935767\\
52.8	0.00798737384829578\\
52.81	0.0079873739030469\\
52.82	0.0079873739578301\\
52.83	0.00798737401264541\\
52.84	0.00798737406749284\\
52.85	0.00798737412237245\\
52.86	0.00798737417728426\\
52.87	0.00798737423222829\\
52.88	0.00798737428720459\\
52.89	0.00798737434221317\\
52.9	0.00798737439725408\\
52.91	0.00798737445232735\\
52.92	0.007987374507433\\
52.93	0.00798737456257107\\
52.94	0.00798737461774158\\
52.95	0.00798737467294458\\
52.96	0.00798737472818009\\
52.97	0.00798737478344815\\
52.98	0.00798737483874878\\
52.99	0.00798737489408201\\
53	0.00798737494944789\\
53.01	0.00798737500484644\\
53.02	0.00798737506027769\\
53.03	0.00798737511574168\\
53.04	0.00798737517123843\\
53.05	0.00798737522676799\\
53.06	0.00798737528233037\\
53.07	0.00798737533792562\\
53.08	0.00798737539355377\\
53.09	0.00798737544921484\\
53.1	0.00798737550490888\\
53.11	0.00798737556063591\\
53.12	0.00798737561639597\\
53.13	0.00798737567218908\\
53.14	0.00798737572801529\\
53.15	0.00798737578387462\\
53.16	0.00798737583976711\\
53.17	0.00798737589569279\\
53.18	0.00798737595165169\\
53.19	0.00798737600764385\\
53.2	0.00798737606366929\\
53.21	0.00798737611972806\\
53.22	0.00798737617582018\\
53.23	0.00798737623194569\\
53.24	0.00798737628810463\\
53.25	0.00798737634429701\\
53.26	0.00798737640052289\\
53.27	0.00798737645678228\\
53.28	0.00798737651307524\\
53.29	0.00798737656940177\\
53.3	0.00798737662576194\\
53.31	0.00798737668215575\\
53.32	0.00798737673858326\\
53.33	0.00798737679504449\\
53.34	0.00798737685153948\\
53.35	0.00798737690806826\\
53.36	0.00798737696463086\\
53.37	0.00798737702122732\\
53.38	0.00798737707785767\\
53.39	0.00798737713452196\\
53.4	0.0079873771912202\\
53.41	0.00798737724795244\\
53.42	0.00798737730471871\\
53.43	0.00798737736151904\\
53.44	0.00798737741835347\\
53.45	0.00798737747522204\\
53.46	0.00798737753212477\\
53.47	0.0079873775890617\\
53.48	0.00798737764603288\\
53.49	0.00798737770303832\\
53.5	0.00798737776007807\\
53.51	0.00798737781715217\\
53.52	0.00798737787426063\\
53.53	0.00798737793140351\\
53.54	0.00798737798858084\\
53.55	0.00798737804579265\\
53.56	0.00798737810303898\\
53.57	0.00798737816031985\\
53.58	0.00798737821763532\\
53.59	0.00798737827498541\\
53.6	0.00798737833237016\\
53.61	0.0079873783897896\\
53.62	0.00798737844724377\\
53.63	0.0079873785047327\\
53.64	0.00798737856225644\\
53.65	0.00798737861981501\\
53.66	0.00798737867740846\\
53.67	0.00798737873503681\\
53.68	0.00798737879270011\\
53.69	0.00798737885039839\\
53.7	0.00798737890813168\\
53.71	0.00798737896590003\\
53.72	0.00798737902370346\\
53.73	0.00798737908154202\\
53.74	0.00798737913941575\\
53.75	0.00798737919732466\\
53.76	0.00798737925526882\\
53.77	0.00798737931324824\\
53.78	0.00798737937126298\\
53.79	0.00798737942931305\\
53.8	0.00798737948739851\\
53.81	0.00798737954551939\\
53.82	0.00798737960367572\\
53.83	0.00798737966186754\\
53.84	0.00798737972009489\\
53.85	0.00798737977835781\\
53.86	0.00798737983665633\\
53.87	0.00798737989499049\\
53.88	0.00798737995336033\\
53.89	0.00798738001176588\\
53.9	0.00798738007020718\\
53.91	0.00798738012868428\\
53.92	0.0079873801871972\\
53.93	0.00798738024574599\\
53.94	0.00798738030433068\\
53.95	0.00798738036295131\\
53.96	0.00798738042160792\\
53.97	0.00798738048030054\\
53.98	0.00798738053902922\\
53.99	0.007987380597794\\
54	0.0079873806565949\\
54.01	0.00798738071543197\\
54.02	0.00798738077430525\\
54.03	0.00798738083321478\\
54.04	0.00798738089216058\\
54.05	0.00798738095114272\\
54.06	0.00798738101016121\\
54.07	0.0079873810692161\\
54.08	0.00798738112830743\\
54.09	0.00798738118743523\\
54.1	0.00798738124659956\\
54.11	0.00798738130580043\\
54.12	0.0079873813650379\\
54.13	0.00798738142431201\\
54.14	0.00798738148362278\\
54.15	0.00798738154297027\\
54.16	0.00798738160235451\\
54.17	0.00798738166177554\\
54.18	0.0079873817212334\\
54.19	0.00798738178072812\\
54.2	0.00798738184025976\\
54.21	0.00798738189982834\\
54.22	0.00798738195943391\\
54.23	0.00798738201907651\\
54.24	0.00798738207875617\\
54.25	0.00798738213847295\\
54.26	0.00798738219822687\\
54.27	0.00798738225801798\\
54.28	0.00798738231784631\\
54.29	0.00798738237771192\\
54.3	0.00798738243761483\\
54.31	0.0079873824975551\\
54.32	0.00798738255753275\\
54.33	0.00798738261754783\\
54.34	0.00798738267760038\\
54.35	0.00798738273769044\\
54.36	0.00798738279781806\\
54.37	0.00798738285798327\\
54.38	0.00798738291818611\\
54.39	0.00798738297842663\\
54.4	0.00798738303870486\\
54.41	0.00798738309902085\\
54.42	0.00798738315937464\\
54.43	0.00798738321976627\\
54.44	0.00798738328019578\\
54.45	0.00798738334066322\\
54.46	0.00798738340116862\\
54.47	0.00798738346171202\\
54.48	0.00798738352229347\\
54.49	0.00798738358291301\\
54.5	0.00798738364357069\\
54.51	0.00798738370426653\\
54.52	0.0079873837650006\\
54.53	0.00798738382577292\\
54.54	0.00798738388658354\\
54.55	0.0079873839474325\\
54.56	0.00798738400831984\\
54.57	0.00798738406924562\\
54.58	0.00798738413020986\\
54.59	0.00798738419121262\\
54.6	0.00798738425225393\\
54.61	0.00798738431333384\\
54.62	0.00798738437445239\\
54.63	0.00798738443560962\\
54.64	0.00798738449680558\\
54.65	0.00798738455804031\\
54.66	0.00798738461931385\\
54.67	0.00798738468062625\\
54.68	0.00798738474197755\\
54.69	0.00798738480336779\\
54.7	0.00798738486479702\\
54.71	0.00798738492626528\\
54.72	0.00798738498777261\\
54.73	0.00798738504931906\\
54.74	0.00798738511090467\\
54.75	0.00798738517252949\\
54.76	0.00798738523419356\\
54.77	0.00798738529589692\\
54.78	0.00798738535763962\\
54.79	0.0079873854194217\\
54.8	0.00798738548124321\\
54.81	0.00798738554310419\\
54.82	0.00798738560500468\\
54.83	0.00798738566694474\\
54.84	0.0079873857289244\\
54.85	0.00798738579094371\\
54.86	0.00798738585300271\\
54.87	0.00798738591510146\\
54.88	0.00798738597723999\\
54.89	0.00798738603941835\\
54.9	0.00798738610163659\\
54.91	0.00798738616389475\\
54.92	0.00798738622619287\\
54.93	0.00798738628853101\\
54.94	0.0079873863509092\\
54.95	0.0079873864133275\\
54.96	0.00798738647578595\\
54.97	0.00798738653828459\\
54.98	0.00798738660082347\\
54.99	0.00798738666340264\\
55	0.00798738672602214\\
55.01	0.00798738678868203\\
55.02	0.00798738685138233\\
55.03	0.00798738691412311\\
55.04	0.00798738697690441\\
55.05	0.00798738703972627\\
55.06	0.00798738710258875\\
55.07	0.00798738716549189\\
55.08	0.00798738722843573\\
55.09	0.00798738729142032\\
55.1	0.00798738735444572\\
55.11	0.00798738741751196\\
55.12	0.00798738748061909\\
55.13	0.00798738754376717\\
55.14	0.00798738760695624\\
55.15	0.00798738767018634\\
55.16	0.00798738773345753\\
55.17	0.00798738779676985\\
55.18	0.00798738786012335\\
55.19	0.00798738792351808\\
55.2	0.00798738798695409\\
55.21	0.00798738805043142\\
55.22	0.00798738811395012\\
55.23	0.00798738817751024\\
55.24	0.00798738824111183\\
55.25	0.00798738830475494\\
55.26	0.00798738836843962\\
55.27	0.00798738843216591\\
55.28	0.00798738849593386\\
55.29	0.00798738855974353\\
55.3	0.00798738862359496\\
55.31	0.00798738868748821\\
55.32	0.00798738875142331\\
55.33	0.00798738881540032\\
55.34	0.00798738887941929\\
55.35	0.00798738894348027\\
55.36	0.0079873890075833\\
55.37	0.00798738907172845\\
55.38	0.00798738913591575\\
55.39	0.00798738920014526\\
55.4	0.00798738926441703\\
55.41	0.0079873893287311\\
55.42	0.00798738939308754\\
55.43	0.00798738945748638\\
55.44	0.00798738952192768\\
55.45	0.00798738958641148\\
55.46	0.00798738965093785\\
55.47	0.00798738971550683\\
55.48	0.00798738978011847\\
55.49	0.00798738984477283\\
55.5	0.00798738990946994\\
55.51	0.00798738997420987\\
55.52	0.00798739003899267\\
55.53	0.00798739010381838\\
55.54	0.00798739016868707\\
55.55	0.00798739023359877\\
55.56	0.00798739029855354\\
55.57	0.00798739036355144\\
55.58	0.00798739042859252\\
55.59	0.00798739049367682\\
55.6	0.0079873905588044\\
55.61	0.00798739062397531\\
55.62	0.00798739068918961\\
55.63	0.00798739075444734\\
55.64	0.00798739081974856\\
55.65	0.00798739088509333\\
55.66	0.00798739095048169\\
55.67	0.00798739101591369\\
55.68	0.0079873910813894\\
55.69	0.00798739114690886\\
55.7	0.00798739121247212\\
55.71	0.00798739127807925\\
55.72	0.00798739134373029\\
55.73	0.0079873914094253\\
55.74	0.00798739147516432\\
55.75	0.00798739154094742\\
55.76	0.00798739160677465\\
55.77	0.00798739167264606\\
55.78	0.00798739173856171\\
55.79	0.00798739180452165\\
55.8	0.00798739187052593\\
55.81	0.00798739193657461\\
55.82	0.00798739200266775\\
55.83	0.00798739206880539\\
55.84	0.0079873921349876\\
55.85	0.00798739220121442\\
55.86	0.00798739226748592\\
55.87	0.00798739233380214\\
55.88	0.00798739240016315\\
55.89	0.00798739246656899\\
55.9	0.00798739253301973\\
55.91	0.00798739259951542\\
55.92	0.00798739266605612\\
55.93	0.00798739273264187\\
55.94	0.00798739279927275\\
55.95	0.00798739286594879\\
55.96	0.00798739293267007\\
55.97	0.00798739299943663\\
55.98	0.00798739306624853\\
55.99	0.00798739313310583\\
56	0.00798739320000858\\
56.01	0.00798739326695685\\
56.02	0.00798739333395069\\
56.03	0.00798739340099015\\
56.04	0.00798739346807529\\
56.05	0.00798739353520618\\
56.06	0.00798739360238286\\
56.07	0.00798739366960539\\
56.08	0.00798739373687384\\
56.09	0.00798739380418826\\
56.1	0.00798739387154871\\
56.11	0.00798739393895524\\
56.12	0.00798739400640792\\
56.13	0.0079873940739068\\
56.14	0.00798739414145194\\
56.15	0.0079873942090434\\
56.16	0.00798739427668124\\
56.17	0.00798739434436551\\
56.18	0.00798739441209628\\
56.19	0.0079873944798736\\
56.2	0.00798739454769754\\
56.21	0.00798739461556815\\
56.22	0.00798739468348549\\
56.23	0.00798739475144962\\
56.24	0.00798739481946061\\
56.25	0.0079873948875185\\
56.26	0.00798739495562337\\
56.27	0.00798739502377526\\
56.28	0.00798739509197425\\
56.29	0.00798739516022039\\
56.3	0.00798739522851374\\
56.31	0.00798739529685437\\
56.32	0.00798739536524232\\
56.33	0.00798739543367767\\
56.34	0.00798739550216047\\
56.35	0.00798739557069079\\
56.36	0.00798739563926869\\
56.37	0.00798739570789422\\
56.38	0.00798739577656746\\
56.39	0.00798739584528845\\
56.4	0.00798739591405727\\
56.41	0.00798739598287397\\
56.42	0.00798739605173862\\
56.43	0.00798739612065128\\
56.44	0.00798739618961201\\
56.45	0.00798739625862088\\
56.46	0.00798739632767794\\
56.47	0.00798739639678326\\
56.48	0.0079873964659369\\
56.49	0.00798739653513892\\
56.5	0.0079873966043894\\
56.51	0.00798739667368838\\
56.52	0.00798739674303594\\
56.53	0.00798739681243214\\
56.54	0.00798739688187703\\
56.55	0.0079873969513707\\
56.56	0.00798739702091319\\
56.57	0.00798739709050458\\
56.58	0.00798739716014492\\
56.59	0.00798739722983429\\
56.6	0.00798739729957274\\
56.61	0.00798739736936035\\
56.62	0.00798739743919717\\
56.63	0.00798739750908327\\
56.64	0.00798739757901872\\
56.65	0.00798739764900358\\
56.66	0.00798739771903791\\
56.67	0.00798739778912179\\
56.68	0.00798739785925528\\
56.69	0.00798739792943844\\
56.7	0.00798739799967134\\
56.71	0.00798739806995405\\
56.72	0.00798739814028663\\
56.73	0.00798739821066914\\
56.74	0.00798739828110167\\
56.75	0.00798739835158426\\
56.76	0.007987398422117\\
56.77	0.00798739849269994\\
56.78	0.00798739856333315\\
56.79	0.0079873986340167\\
56.8	0.00798739870475067\\
56.81	0.0079873987755351\\
56.82	0.00798739884637008\\
56.83	0.00798739891725568\\
56.84	0.00798739898819195\\
56.85	0.00798739905917897\\
56.86	0.00798739913021681\\
56.87	0.00798739920130553\\
56.88	0.0079873992724452\\
56.89	0.0079873993436359\\
56.9	0.00798739941487769\\
56.91	0.00798739948617064\\
56.92	0.00798739955751482\\
56.93	0.0079873996289103\\
56.94	0.00798739970035715\\
56.95	0.00798739977185544\\
56.96	0.00798739984340524\\
56.97	0.00798739991500661\\
56.98	0.00798739998665964\\
56.99	0.00798740005836439\\
57	0.00798740013012093\\
57.01	0.00798740020192933\\
57.02	0.00798740027378966\\
57.03	0.007987400345702\\
57.04	0.00798740041766641\\
57.05	0.00798740048968297\\
57.06	0.00798740056175175\\
57.07	0.00798740063387282\\
57.08	0.00798740070604625\\
57.09	0.00798740077827211\\
57.1	0.00798740085055049\\
57.11	0.00798740092288144\\
57.12	0.00798740099526504\\
57.13	0.00798740106770137\\
57.14	0.0079874011401905\\
57.15	0.0079874012127325\\
57.16	0.00798740128532744\\
57.17	0.0079874013579754\\
57.18	0.00798740143067645\\
57.19	0.00798740150343067\\
57.2	0.00798740157623813\\
57.21	0.0079874016490989\\
57.22	0.00798740172201306\\
57.23	0.00798740179498068\\
57.24	0.00798740186800185\\
57.25	0.00798740194107662\\
57.26	0.00798740201420508\\
57.27	0.00798740208738731\\
57.28	0.00798740216062337\\
57.29	0.00798740223391335\\
57.3	0.00798740230725732\\
57.31	0.00798740238065536\\
57.32	0.00798740245410753\\
57.33	0.00798740252761393\\
57.34	0.00798740260117462\\
57.35	0.00798740267478969\\
57.36	0.0079874027484592\\
57.37	0.00798740282218324\\
57.38	0.00798740289596189\\
57.39	0.00798740296979522\\
57.4	0.0079874030436833\\
57.41	0.00798740311762623\\
57.42	0.00798740319162406\\
57.43	0.00798740326567689\\
57.44	0.0079874033397848\\
57.45	0.00798740341394785\\
57.46	0.00798740348816614\\
57.47	0.00798740356243973\\
57.48	0.00798740363676871\\
57.49	0.00798740371115315\\
57.5	0.00798740378559315\\
57.51	0.00798740386008877\\
57.52	0.0079874039346401\\
57.53	0.00798740400924721\\
57.54	0.00798740408391019\\
57.55	0.00798740415862912\\
57.56	0.00798740423340408\\
57.57	0.00798740430823515\\
57.58	0.00798740438312241\\
57.59	0.00798740445806595\\
57.6	0.00798740453306583\\
57.61	0.00798740460812216\\
57.62	0.007987404683235\\
57.63	0.00798740475840444\\
57.64	0.00798740483363057\\
57.65	0.00798740490891346\\
57.66	0.0079874049842532\\
57.67	0.00798740505964987\\
57.68	0.00798740513510356\\
57.69	0.00798740521061434\\
57.7	0.00798740528618231\\
57.71	0.00798740536180754\\
57.72	0.00798740543749013\\
57.73	0.00798740551323015\\
57.74	0.00798740558902768\\
57.75	0.00798740566488283\\
57.76	0.00798740574079566\\
57.77	0.00798740581676626\\
57.78	0.00798740589279473\\
57.79	0.00798740596888114\\
57.8	0.00798740604502558\\
57.81	0.00798740612122814\\
57.82	0.00798740619748891\\
57.83	0.00798740627380796\\
57.84	0.00798740635018539\\
57.85	0.00798740642662129\\
57.86	0.00798740650311573\\
57.87	0.00798740657966882\\
57.88	0.00798740665628063\\
57.89	0.00798740673295126\\
57.9	0.00798740680968079\\
57.91	0.00798740688646931\\
57.92	0.00798740696331691\\
57.93	0.00798740704022368\\
57.94	0.00798740711718971\\
57.95	0.00798740719421508\\
57.96	0.00798740727129989\\
57.97	0.00798740734844423\\
57.98	0.00798740742564818\\
57.99	0.00798740750291184\\
58	0.00798740758023529\\
58.01	0.00798740765761864\\
58.02	0.00798740773506196\\
58.03	0.00798740781256535\\
58.04	0.0079874078901289\\
58.05	0.00798740796775271\\
58.06	0.00798740804543686\\
58.07	0.00798740812318145\\
58.08	0.00798740820098657\\
58.09	0.00798740827885231\\
58.1	0.00798740835677877\\
58.11	0.00798740843476603\\
58.12	0.0079874085128142\\
58.13	0.00798740859092336\\
58.14	0.00798740866909362\\
58.15	0.00798740874732505\\
58.16	0.00798740882561777\\
58.17	0.00798740890397186\\
58.18	0.00798740898238741\\
58.19	0.00798740906086453\\
58.2	0.00798740913940331\\
58.21	0.00798740921800384\\
58.22	0.00798740929666622\\
58.23	0.00798740937539055\\
58.24	0.00798740945417693\\
58.25	0.00798740953302544\\
58.26	0.00798740961193619\\
58.27	0.00798740969090928\\
58.28	0.0079874097699448\\
58.29	0.00798740984904284\\
58.3	0.00798740992820352\\
58.31	0.00798741000742692\\
58.32	0.00798741008671315\\
58.33	0.00798741016606231\\
58.34	0.00798741024547448\\
58.35	0.00798741032494979\\
58.36	0.00798741040448831\\
58.37	0.00798741048409016\\
58.38	0.00798741056375543\\
58.39	0.00798741064348423\\
58.4	0.00798741072327665\\
58.41	0.0079874108031328\\
58.42	0.00798741088305277\\
58.43	0.00798741096303668\\
58.44	0.00798741104308462\\
58.45	0.00798741112319669\\
58.46	0.00798741120337301\\
58.47	0.00798741128361366\\
58.48	0.00798741136391875\\
58.49	0.0079874114442884\\
58.5	0.00798741152472269\\
58.51	0.00798741160522174\\
58.52	0.00798741168578565\\
58.53	0.00798741176641452\\
58.54	0.00798741184710846\\
58.55	0.00798741192786758\\
58.56	0.00798741200869198\\
58.57	0.00798741208958177\\
58.58	0.00798741217053705\\
58.59	0.00798741225155793\\
58.6	0.00798741233264452\\
58.61	0.00798741241379692\\
58.62	0.00798741249501525\\
58.63	0.0079874125762996\\
58.64	0.00798741265765009\\
58.65	0.00798741273906683\\
58.66	0.00798741282054993\\
58.67	0.00798741290209948\\
58.68	0.00798741298371562\\
58.69	0.00798741306539844\\
58.7	0.00798741314714805\\
58.71	0.00798741322896457\\
58.72	0.0079874133108481\\
58.73	0.00798741339279876\\
58.74	0.00798741347481666\\
58.75	0.00798741355690191\\
58.76	0.00798741363905462\\
58.77	0.00798741372127491\\
58.78	0.00798741380356288\\
58.79	0.00798741388591866\\
58.8	0.00798741396834235\\
58.81	0.00798741405083407\\
58.82	0.00798741413339393\\
58.83	0.00798741421602206\\
58.84	0.00798741429871855\\
58.85	0.00798741438148353\\
58.86	0.00798741446431712\\
58.87	0.00798741454721943\\
58.88	0.00798741463019057\\
58.89	0.00798741471323067\\
58.9	0.00798741479633983\\
58.91	0.00798741487951819\\
58.92	0.00798741496276585\\
58.93	0.00798741504608293\\
58.94	0.00798741512946956\\
58.95	0.00798741521292585\\
58.96	0.00798741529645192\\
58.97	0.00798741538004789\\
58.98	0.00798741546371388\\
58.99	0.00798741554745002\\
59	0.00798741563125641\\
59.01	0.00798741571513319\\
59.02	0.00798741579908048\\
59.03	0.00798741588309839\\
59.04	0.00798741596718705\\
59.05	0.00798741605134659\\
59.06	0.00798741613557712\\
59.07	0.00798741621987878\\
59.08	0.00798741630425167\\
59.09	0.00798741638869594\\
59.1	0.00798741647321169\\
59.11	0.00798741655779907\\
59.12	0.00798741664245819\\
59.13	0.00798741672718918\\
59.14	0.00798741681199217\\
59.15	0.00798741689686729\\
59.16	0.00798741698181465\\
59.17	0.00798741706683439\\
59.18	0.00798741715192664\\
59.19	0.00798741723709153\\
59.2	0.00798741732232918\\
59.21	0.00798741740763973\\
59.22	0.0079874174930233\\
59.23	0.00798741757848002\\
59.24	0.00798741766401003\\
59.25	0.00798741774961346\\
59.26	0.00798741783529044\\
59.27	0.0079874179210411\\
59.28	0.00798741800686557\\
59.29	0.00798741809276398\\
59.3	0.00798741817873648\\
59.31	0.00798741826478319\\
59.32	0.00798741835090425\\
59.33	0.00798741843709979\\
59.34	0.00798741852336995\\
59.35	0.00798741860971487\\
59.36	0.00798741869613467\\
59.37	0.00798741878262951\\
59.38	0.0079874188691995\\
59.39	0.0079874189558448\\
59.4	0.00798741904256554\\
59.41	0.00798741912936186\\
59.42	0.0079874192162339\\
59.43	0.00798741930318179\\
59.44	0.00798741939020569\\
59.45	0.00798741947730572\\
59.46	0.00798741956448203\\
59.47	0.00798741965173476\\
59.48	0.00798741973906406\\
59.49	0.00798741982647006\\
59.5	0.00798741991395291\\
59.51	0.00798742000151276\\
59.52	0.00798742008914974\\
59.53	0.007987420176864\\
59.54	0.0079874202646557\\
59.55	0.00798742035252496\\
59.56	0.00798742044047194\\
59.57	0.00798742052849679\\
59.58	0.00798742061659965\\
59.59	0.00798742070478067\\
59.6	0.00798742079304001\\
59.61	0.0079874208813778\\
59.62	0.00798742096979419\\
59.63	0.00798742105828935\\
59.64	0.00798742114686342\\
59.65	0.00798742123551654\\
59.66	0.00798742132424888\\
59.67	0.00798742141306058\\
59.68	0.0079874215019518\\
59.69	0.00798742159092269\\
59.7	0.0079874216799734\\
59.71	0.00798742176910409\\
59.72	0.00798742185831491\\
59.73	0.00798742194760603\\
59.74	0.00798742203697759\\
59.75	0.00798742212642975\\
59.76	0.00798742221596267\\
59.77	0.00798742230557651\\
59.78	0.00798742239527143\\
59.79	0.00798742248504759\\
59.8	0.00798742257490514\\
59.81	0.00798742266484425\\
59.82	0.00798742275486508\\
59.83	0.00798742284496779\\
59.84	0.00798742293515254\\
59.85	0.0079874230254195\\
59.86	0.00798742311576883\\
59.87	0.00798742320620069\\
59.88	0.00798742329671525\\
59.89	0.00798742338731267\\
59.9	0.00798742347799313\\
59.91	0.00798742356875678\\
59.92	0.0079874236596038\\
59.93	0.00798742375053435\\
59.94	0.00798742384154861\\
59.95	0.00798742393264673\\
59.96	0.0079874240238289\\
59.97	0.00798742411509529\\
59.98	0.00798742420644605\\
59.99	0.00798742429788138\\
60	0.00798742438940144\\
60.01	0.0079874244810064\\
60.02	0.00798742457269644\\
60.03	0.00798742466447173\\
60.04	0.00798742475633246\\
60.05	0.00798742484827879\\
60.06	0.0079874249403109\\
60.07	0.00798742503242898\\
60.08	0.00798742512463319\\
60.09	0.00798742521692373\\
60.1	0.00798742530930077\\
60.11	0.00798742540176449\\
60.12	0.00798742549431507\\
60.13	0.00798742558695269\\
60.14	0.00798742567967755\\
60.15	0.00798742577248982\\
60.16	0.00798742586538969\\
60.17	0.00798742595837734\\
60.18	0.00798742605145296\\
60.19	0.00798742614461674\\
60.2	0.00798742623786886\\
60.21	0.00798742633120951\\
60.22	0.00798742642463889\\
60.23	0.00798742651815719\\
60.24	0.00798742661176458\\
60.25	0.00798742670546128\\
60.26	0.00798742679924746\\
60.27	0.00798742689312333\\
60.28	0.00798742698708907\\
60.29	0.00798742708114489\\
60.3	0.00798742717529098\\
60.31	0.00798742726952753\\
60.32	0.00798742736385475\\
60.33	0.00798742745827283\\
60.34	0.00798742755278197\\
60.35	0.00798742764738238\\
60.36	0.00798742774207425\\
60.37	0.00798742783685778\\
60.38	0.00798742793173319\\
60.39	0.00798742802670067\\
60.4	0.00798742812176043\\
60.41	0.00798742821691268\\
60.42	0.00798742831215761\\
60.43	0.00798742840749545\\
60.44	0.00798742850292639\\
60.45	0.00798742859845066\\
60.46	0.00798742869406845\\
60.47	0.00798742878977998\\
60.48	0.00798742888558546\\
60.49	0.00798742898148511\\
60.5	0.00798742907747914\\
60.51	0.00798742917356777\\
60.52	0.0079874292697512\\
60.53	0.00798742936602967\\
60.54	0.00798742946240338\\
60.55	0.00798742955887256\\
60.56	0.00798742965543743\\
60.57	0.00798742975209821\\
60.58	0.00798742984885512\\
60.59	0.00798742994570839\\
60.6	0.00798743004265823\\
60.61	0.00798743013970488\\
60.62	0.00798743023684856\\
60.63	0.0079874303340895\\
60.64	0.00798743043142793\\
60.65	0.00798743052886408\\
60.66	0.00798743062639817\\
60.67	0.00798743072403045\\
60.68	0.00798743082176114\\
60.69	0.00798743091959048\\
60.7	0.00798743101751871\\
60.71	0.00798743111554606\\
60.72	0.00798743121367276\\
60.73	0.00798743131189906\\
60.74	0.0079874314102252\\
60.75	0.00798743150865141\\
60.76	0.00798743160717795\\
60.77	0.00798743170580505\\
60.78	0.00798743180453296\\
60.79	0.00798743190336192\\
60.8	0.00798743200229218\\
60.81	0.007987432101324\\
60.82	0.00798743220045761\\
60.83	0.00798743229969327\\
60.84	0.00798743239903124\\
60.85	0.00798743249847176\\
60.86	0.00798743259801508\\
60.87	0.00798743269766148\\
60.88	0.0079874327974112\\
60.89	0.00798743289726449\\
60.9	0.00798743299722163\\
60.91	0.00798743309728287\\
60.92	0.00798743319744847\\
60.93	0.0079874332977187\\
60.94	0.00798743339809382\\
60.95	0.0079874334985741\\
60.96	0.0079874335991598\\
60.97	0.0079874336998512\\
60.98	0.00798743380064856\\
60.99	0.00798743390155216\\
61	0.00798743400256226\\
61.01	0.00798743410367915\\
61.02	0.0079874342049031\\
61.03	0.00798743430623439\\
61.04	0.0079874344076733\\
61.05	0.00798743450922009\\
61.06	0.00798743461087507\\
61.07	0.00798743471263851\\
61.08	0.00798743481451069\\
61.09	0.0079874349164919\\
61.1	0.00798743501858244\\
61.11	0.00798743512078258\\
61.12	0.00798743522309262\\
61.13	0.00798743532551285\\
61.14	0.00798743542804356\\
61.15	0.00798743553068505\\
61.16	0.00798743563343762\\
61.17	0.00798743573630157\\
61.18	0.00798743583927719\\
61.19	0.00798743594236478\\
61.2	0.00798743604556466\\
61.21	0.00798743614887712\\
61.22	0.00798743625230247\\
61.23	0.00798743635584102\\
61.24	0.00798743645949308\\
61.25	0.00798743656325896\\
61.26	0.00798743666713897\\
61.27	0.00798743677113343\\
61.28	0.00798743687524265\\
61.29	0.00798743697946696\\
61.3	0.00798743708380667\\
61.31	0.00798743718826211\\
61.32	0.00798743729283359\\
61.33	0.00798743739752145\\
61.34	0.00798743750232601\\
61.35	0.0079874376072476\\
61.36	0.00798743771228655\\
61.37	0.00798743781744319\\
61.38	0.00798743792271787\\
61.39	0.0079874380281109\\
61.4	0.00798743813362264\\
61.41	0.00798743823925342\\
61.42	0.00798743834500358\\
61.43	0.00798743845087348\\
61.44	0.00798743855686345\\
61.45	0.00798743866297384\\
61.46	0.00798743876920499\\
61.47	0.00798743887555728\\
61.48	0.00798743898203104\\
61.49	0.00798743908862664\\
61.5	0.00798743919534442\\
61.51	0.00798743930218476\\
61.52	0.00798743940914801\\
61.53	0.00798743951623454\\
61.54	0.0079874396234447\\
61.55	0.00798743973077888\\
61.56	0.00798743983823745\\
61.57	0.00798743994582076\\
61.58	0.0079874400535292\\
61.59	0.00798744016136315\\
61.6	0.00798744026932298\\
61.61	0.00798744037740908\\
61.62	0.00798744048562183\\
61.63	0.00798744059396161\\
61.64	0.00798744070242881\\
61.65	0.00798744081102382\\
61.66	0.00798744091974704\\
61.67	0.00798744102859886\\
61.68	0.00798744113757968\\
61.69	0.00798744124668989\\
61.7	0.0079874413559299\\
61.71	0.00798744146530011\\
61.72	0.00798744157480093\\
61.73	0.00798744168443276\\
61.74	0.00798744179419603\\
61.75	0.00798744190409114\\
61.76	0.00798744201411851\\
61.77	0.00798744212427855\\
61.78	0.0079874422345717\\
61.79	0.00798744234499837\\
61.8	0.00798744245555899\\
61.81	0.00798744256625399\\
61.82	0.0079874426770838\\
61.83	0.00798744278804886\\
61.84	0.00798744289914959\\
61.85	0.00798744301038645\\
61.86	0.00798744312175988\\
61.87	0.00798744323327031\\
61.88	0.0079874433449182\\
61.89	0.00798744345670399\\
61.9	0.00798744356862815\\
61.91	0.00798744368069112\\
61.92	0.00798744379289336\\
61.93	0.00798744390523535\\
61.94	0.00798744401771753\\
61.95	0.00798744413034038\\
61.96	0.00798744424310437\\
61.97	0.00798744435600997\\
61.98	0.00798744446905767\\
61.99	0.00798744458224792\\
62	0.00798744469558123\\
62.01	0.00798744480905807\\
62.02	0.00798744492267893\\
62.03	0.0079874450364443\\
62.04	0.00798744515035468\\
62.05	0.00798744526441057\\
62.06	0.00798744537861246\\
62.07	0.00798744549296086\\
62.08	0.00798744560745628\\
62.09	0.00798744572209922\\
62.1	0.0079874458368902\\
62.11	0.00798744595182973\\
62.12	0.00798744606691834\\
62.13	0.00798744618215655\\
62.14	0.00798744629754488\\
62.15	0.00798744641308386\\
62.16	0.00798744652877403\\
62.17	0.00798744664461593\\
62.18	0.00798744676061009\\
62.19	0.00798744687675706\\
62.2	0.00798744699305738\\
62.21	0.0079874471095116\\
62.22	0.00798744722612029\\
62.23	0.00798744734288399\\
62.24	0.00798744745980326\\
62.25	0.00798744757687868\\
62.26	0.00798744769411081\\
62.27	0.00798744781150023\\
62.28	0.0079874479290475\\
62.29	0.00798744804675321\\
62.3	0.00798744816461794\\
62.31	0.00798744828264228\\
62.32	0.00798744840082681\\
62.33	0.00798744851917215\\
62.34	0.00798744863767887\\
62.35	0.0079874487563476\\
62.36	0.00798744887517892\\
62.37	0.00798744899417346\\
62.38	0.00798744911333183\\
62.39	0.00798744923265464\\
62.4	0.00798744935214252\\
62.41	0.0079874494717961\\
62.42	0.00798744959161601\\
62.43	0.00798744971160287\\
62.44	0.00798744983175734\\
62.45	0.00798744995208006\\
62.46	0.00798745007257167\\
62.47	0.00798745019323282\\
62.48	0.00798745031406418\\
62.49	0.0079874504350664\\
62.5	0.00798745055624014\\
62.51	0.00798745067758609\\
62.52	0.00798745079910491\\
62.53	0.00798745092079728\\
62.54	0.00798745104266388\\
62.55	0.0079874511647054\\
62.56	0.00798745128692254\\
62.57	0.00798745140931599\\
62.58	0.00798745153188646\\
62.59	0.00798745165463464\\
62.6	0.00798745177756126\\
62.61	0.00798745190066702\\
62.62	0.00798745202395265\\
62.63	0.00798745214741888\\
62.64	0.00798745227106643\\
62.65	0.00798745239489604\\
62.66	0.00798745251890845\\
62.67	0.00798745264310441\\
62.68	0.00798745276748467\\
62.69	0.00798745289204998\\
62.7	0.0079874530168011\\
62.71	0.0079874531417388\\
62.72	0.00798745326686386\\
62.73	0.00798745339217704\\
62.74	0.00798745351767913\\
62.75	0.00798745364337092\\
62.76	0.0079874537692532\\
62.77	0.00798745389532677\\
62.78	0.00798745402159243\\
62.79	0.00798745414805099\\
62.8	0.00798745427470326\\
62.81	0.00798745440155007\\
62.82	0.00798745452859224\\
62.83	0.0079874546558306\\
62.84	0.00798745478326598\\
62.85	0.00798745491089925\\
62.86	0.00798745503873123\\
62.87	0.00798745516676278\\
62.88	0.00798745529499477\\
62.89	0.00798745542342807\\
62.9	0.00798745555206353\\
62.91	0.00798745568090205\\
62.92	0.00798745580994451\\
62.93	0.00798745593919179\\
62.94	0.0079874560686448\\
62.95	0.00798745619830443\\
62.96	0.0079874563281716\\
62.97	0.00798745645824722\\
62.98	0.00798745658853221\\
62.99	0.00798745671902751\\
63	0.00798745684973404\\
63.01	0.00798745698065275\\
63.02	0.00798745711178458\\
63.03	0.00798745724313049\\
63.04	0.00798745737469144\\
63.05	0.0079874575064684\\
63.06	0.00798745763846234\\
63.07	0.00798745777067424\\
63.08	0.00798745790310509\\
63.09	0.00798745803575589\\
63.1	0.00798745816862764\\
63.11	0.00798745830172134\\
63.12	0.00798745843503802\\
63.13	0.00798745856857869\\
63.14	0.00798745870234438\\
63.15	0.00798745883633614\\
63.16	0.007987458970555\\
63.17	0.00798745910500202\\
63.18	0.00798745923967826\\
63.19	0.00798745937458478\\
63.2	0.00798745950972266\\
63.21	0.00798745964509298\\
63.22	0.00798745978069683\\
63.23	0.00798745991653531\\
63.24	0.00798746005260952\\
63.25	0.00798746018892057\\
63.26	0.00798746032546958\\
63.27	0.00798746046225768\\
63.28	0.00798746059928601\\
63.29	0.00798746073655571\\
63.3	0.00798746087406794\\
63.31	0.00798746101182384\\
63.32	0.0079874611498246\\
63.33	0.00798746128807138\\
63.34	0.00798746142656537\\
63.35	0.00798746156530777\\
63.36	0.00798746170429978\\
63.37	0.0079874618435426\\
63.38	0.00798746198303745\\
63.39	0.00798746212278557\\
63.4	0.00798746226278818\\
63.41	0.00798746240304653\\
63.42	0.00798746254356188\\
63.43	0.00798746268433547\\
63.44	0.0079874628253686\\
63.45	0.00798746296666253\\
63.46	0.00798746310821856\\
63.47	0.00798746325003797\\
63.48	0.00798746339212209\\
63.49	0.00798746353447222\\
63.5	0.00798746367708969\\
63.51	0.00798746381997584\\
63.52	0.007987463963132\\
63.53	0.00798746410655954\\
63.54	0.00798746425025981\\
63.55	0.00798746439423419\\
63.56	0.00798746453848406\\
63.57	0.00798746468301081\\
63.58	0.00798746482781585\\
63.59	0.00798746497290058\\
63.6	0.00798746511826643\\
63.61	0.00798746526391483\\
63.62	0.00798746540984722\\
63.63	0.00798746555606505\\
63.64	0.00798746570256979\\
63.65	0.00798746584936291\\
63.66	0.00798746599644588\\
63.67	0.00798746614382021\\
63.68	0.0079874662914874\\
63.69	0.00798746643944896\\
63.7	0.00798746658770641\\
63.71	0.0079874667362613\\
63.72	0.00798746688511516\\
63.73	0.00798746703426956\\
63.74	0.00798746718372607\\
63.75	0.00798746733348626\\
63.76	0.00798746748355172\\
63.77	0.00798746763392405\\
63.78	0.00798746778460488\\
63.79	0.00798746793559581\\
63.8	0.0079874680868985\\
63.81	0.00798746823851457\\
63.82	0.0079874683904457\\
63.83	0.00798746854269356\\
63.84	0.00798746869525981\\
63.85	0.00798746884814617\\
63.86	0.00798746900135432\\
63.87	0.007987469154886\\
63.88	0.00798746930874293\\
63.89	0.00798746946292685\\
63.9	0.00798746961743951\\
63.91	0.00798746977228268\\
63.92	0.00798746992745814\\
63.93	0.00798747008296768\\
63.94	0.0079874702388131\\
63.95	0.00798747039499622\\
63.96	0.00798747055151887\\
63.97	0.00798747070838288\\
63.98	0.00798747086559011\\
63.99	0.00798747102314243\\
64	0.00798747118104172\\
64.01	0.00798747133928986\\
64.02	0.00798747149788877\\
64.03	0.00798747165684036\\
64.04	0.00798747181614657\\
64.05	0.00798747197580934\\
64.06	0.00798747213583063\\
64.07	0.00798747229621241\\
64.08	0.00798747245695667\\
64.09	0.00798747261806542\\
64.1	0.00798747277954066\\
64.11	0.00798747294138441\\
64.12	0.00798747310359874\\
64.13	0.00798747326618568\\
64.14	0.00798747342914731\\
64.15	0.00798747359248571\\
64.16	0.00798747375620298\\
64.17	0.00798747392030124\\
64.18	0.00798747408478261\\
64.19	0.00798747424964924\\
64.2	0.00798747441490328\\
64.21	0.00798747458054689\\
64.22	0.00798747474658228\\
64.23	0.00798747491301164\\
64.24	0.00798747507983718\\
64.25	0.00798747524706114\\
64.26	0.00798747541468577\\
64.27	0.00798747558271332\\
64.28	0.00798747575114607\\
64.29	0.00798747591998632\\
64.3	0.00798747608923638\\
64.31	0.00798747625889857\\
64.32	0.00798747642897522\\
64.33	0.0079874765994687\\
64.34	0.00798747677038137\\
64.35	0.00798747694171562\\
64.36	0.00798747711347387\\
64.37	0.00798747728565851\\
64.38	0.00798747745827201\\
64.39	0.00798747763131679\\
64.4	0.00798747780479534\\
64.41	0.00798747797871015\\
64.42	0.0079874781530637\\
64.43	0.00798747832785853\\
64.44	0.00798747850309716\\
64.45	0.00798747867878215\\
64.46	0.00798747885491607\\
64.47	0.00798747903150151\\
64.48	0.00798747920854107\\
64.49	0.00798747938603737\\
64.5	0.00798747956399305\\
64.51	0.00798747974241077\\
64.52	0.00798747992129321\\
64.53	0.00798748010064304\\
64.54	0.00798748028046299\\
64.55	0.00798748046075577\\
64.56	0.00798748064152414\\
64.57	0.00798748082277086\\
64.58	0.00798748100449871\\
64.59	0.00798748118671049\\
64.6	0.00798748136940901\\
64.61	0.00798748155259712\\
64.62	0.00798748173627767\\
64.63	0.00798748192045352\\
64.64	0.00798748210512758\\
64.65	0.00798748229030276\\
64.66	0.00798748247598198\\
64.67	0.00798748266216819\\
64.68	0.00798748284886436\\
64.69	0.00798748303607349\\
64.7	0.00798748322379856\\
64.71	0.00798748341204262\\
64.72	0.0079874836008087\\
64.73	0.00798748379009987\\
64.74	0.00798748397991921\\
64.75	0.00798748417026983\\
64.76	0.00798748436115485\\
64.77	0.00798748455257742\\
64.78	0.00798748474454069\\
64.79	0.00798748493704785\\
64.8	0.00798748513010211\\
64.81	0.00798748532370669\\
64.82	0.00798748551786483\\
64.83	0.0079874857125798\\
64.84	0.00798748590785488\\
64.85	0.00798748610369338\\
64.86	0.00798748630009862\\
64.87	0.00798748649707396\\
64.88	0.00798748669462275\\
64.89	0.0079874868927484\\
64.9	0.0079874870914543\\
64.91	0.0079874872907439\\
64.92	0.00798748749062063\\
64.93	0.00798748769108799\\
64.94	0.00798748789214945\\
64.95	0.00798748809380854\\
64.96	0.00798748829606879\\
64.97	0.00798748849893377\\
64.98	0.00798748870240706\\
64.99	0.00798748890649225\\
65	0.00798748911119297\\
65.01	0.00798748931651288\\
65.02	0.00798748952245563\\
65.03	0.00798748972902493\\
65.04	0.00798748993622447\\
65.05	0.007987490144058\\
65.06	0.00798749035252928\\
65.07	0.00798749056164208\\
65.08	0.00798749077140021\\
65.09	0.00798749098180749\\
65.1	0.00798749119286778\\
65.11	0.00798749140458493\\
65.12	0.00798749161696284\\
65.13	0.00798749183000543\\
65.14	0.00798749204371664\\
65.15	0.00798749225810043\\
65.16	0.00798749247316078\\
65.17	0.0079874926889017\\
65.18	0.00798749290532722\\
65.19	0.0079874931224414\\
65.2	0.00798749334024831\\
65.21	0.00798749355875206\\
65.22	0.00798749377795676\\
65.23	0.00798749399786657\\
65.24	0.00798749421848565\\
65.25	0.00798749443981821\\
65.26	0.00798749466186846\\
65.27	0.00798749488464064\\
65.28	0.00798749510813902\\
65.29	0.00798749533236788\\
65.3	0.00798749555733155\\
65.31	0.00798749578303435\\
65.32	0.00798749600948066\\
65.33	0.00798749623667484\\
65.34	0.00798749646462133\\
65.35	0.00798749669332454\\
65.36	0.00798749692278893\\
65.37	0.00798749715301899\\
65.38	0.00798749738401923\\
65.39	0.00798749761579417\\
65.4	0.00798749784834836\\
65.41	0.00798749808168639\\
65.42	0.00798749831581286\\
65.43	0.0079874985507324\\
65.44	0.00798749878644965\\
65.45	0.0079874990229693\\
65.46	0.00798749926029605\\
65.47	0.00798749949843462\\
65.48	0.00798749973738975\\
65.49	0.00798749997716623\\
65.5	0.00798750021776886\\
65.51	0.00798750045920245\\
65.52	0.00798750070147186\\
65.53	0.00798750094458195\\
65.54	0.00798750118853763\\
65.55	0.00798750143334381\\
65.56	0.00798750167900545\\
65.57	0.00798750192552751\\
65.58	0.00798750217291499\\
65.59	0.0079875024211729\\
65.6	0.0079875026703063\\
65.61	0.00798750292032025\\
65.62	0.00798750317121985\\
65.63	0.00798750342301021\\
65.64	0.00798750367569648\\
65.65	0.00798750392928382\\
65.66	0.00798750418377743\\
65.67	0.00798750443918252\\
65.68	0.00798750469550433\\
65.69	0.00798750495274813\\
65.7	0.00798750521091921\\
65.71	0.00798750547002288\\
65.72	0.00798750573006448\\
65.73	0.00798750599104938\\
65.74	0.00798750625298295\\
65.75	0.00798750651587061\\
65.76	0.0079875067797178\\
65.77	0.00798750704452997\\
65.78	0.00798750731031261\\
65.79	0.00798750757707122\\
65.8	0.00798750784481134\\
65.81	0.00798750811353852\\
65.82	0.00798750838325834\\
65.83	0.0079875086539764\\
65.84	0.00798750892569833\\
65.85	0.00798750919842977\\
65.86	0.0079875094721764\\
65.87	0.00798750974694392\\
65.88	0.00798751002273805\\
65.89	0.00798751029956452\\
65.9	0.00798751057742911\\
65.91	0.0079875108563376\\
65.92	0.0079875111362958\\
65.93	0.00798751141730956\\
65.94	0.00798751169938472\\
65.95	0.00798751198252716\\
65.96	0.0079875122667428\\
65.97	0.00798751255203754\\
65.98	0.00798751283841735\\
65.99	0.00798751312588818\\
66	0.00798751341445602\\
66.01	0.0079875137041269\\
66.02	0.00798751399490684\\
66.03	0.0079875142868019\\
66.04	0.00798751457981815\\
66.05	0.00798751487396169\\
66.06	0.00798751516923864\\
66.07	0.00798751546565514\\
66.08	0.00798751576321735\\
66.09	0.00798751606193145\\
66.1	0.00798751636180363\\
66.11	0.00798751666284012\\
66.12	0.00798751696504716\\
66.13	0.00798751726843101\\
66.14	0.00798751757299795\\
66.15	0.00798751787875427\\
66.16	0.00798751818570629\\
66.17	0.00798751849386036\\
66.18	0.00798751880322281\\
66.19	0.00798751911380003\\
66.2	0.00798751942559841\\
66.21	0.00798751973862435\\
66.22	0.00798752005288429\\
66.23	0.00798752036838465\\
66.24	0.00798752068513192\\
66.25	0.00798752100313255\\
66.26	0.00798752132239305\\
66.27	0.00798752164291992\\
66.28	0.0079875219647197\\
66.29	0.00798752228779892\\
66.3	0.00798752261216413\\
66.31	0.00798752293782192\\
66.32	0.00798752326477886\\
66.33	0.00798752359304156\\
66.34	0.00798752392261663\\
66.35	0.0079875242535107\\
66.36	0.00798752458573042\\
66.37	0.00798752491928243\\
66.38	0.0079875252541734\\
66.39	0.00798752559041002\\
66.4	0.00798752592799897\\
66.41	0.00798752626694696\\
66.42	0.0079875266072607\\
66.43	0.00798752694894692\\
66.44	0.00798752729201235\\
66.45	0.00798752763646374\\
66.46	0.00798752798230784\\
66.47	0.00798752832955142\\
66.48	0.00798752867820125\\
66.49	0.00798752902826411\\
66.5	0.0079875293797468\\
66.51	0.0079875297326561\\
66.52	0.00798753008699882\\
66.53	0.00798753044278177\\
66.54	0.00798753080001176\\
66.55	0.00798753115869563\\
66.56	0.00798753151884018\\
66.57	0.00798753188045227\\
66.58	0.00798753224353871\\
66.59	0.00798753260810635\\
66.6	0.00798753297416204\\
66.61	0.0079875333417126\\
66.62	0.0079875337107649\\
66.63	0.00798753408132578\\
66.64	0.00798753445340208\\
66.65	0.00798753482700066\\
66.66	0.00798753520212837\\
66.67	0.00798753557879204\\
66.68	0.00798753595699854\\
66.69	0.00798753633675469\\
66.7	0.00798753671806735\\
66.71	0.00798753710094335\\
66.72	0.00798753748538953\\
66.73	0.00798753787141271\\
66.74	0.00798753825901972\\
66.75	0.00798753864821739\\
66.76	0.00798753903901251\\
66.77	0.00798753943141191\\
66.78	0.00798753982542236\\
66.79	0.00798754022105068\\
66.8	0.00798754061830362\\
66.81	0.00798754101718798\\
66.82	0.00798754141771049\\
66.83	0.00798754181987792\\
66.84	0.007987542223697\\
66.85	0.00798754262917445\\
66.86	0.00798754303631698\\
66.87	0.00798754344513129\\
66.88	0.00798754385562406\\
66.89	0.00798754426780195\\
66.9	0.00798754468167161\\
66.91	0.00798754509723968\\
66.92	0.00798754551451276\\
66.93	0.00798754593349745\\
66.94	0.00798754635420032\\
66.95	0.00798754677662793\\
66.96	0.0079875472007868\\
66.97	0.00798754762668346\\
66.98	0.00798754805432437\\
66.99	0.00798754848371601\\
67	0.00798754891486481\\
67.01	0.00798754934777719\\
67.02	0.00798754978245952\\
67.03	0.00798755021891817\\
67.04	0.00798755065715946\\
67.05	0.00798755109718969\\
67.06	0.00798755153901512\\
67.07	0.00798755198264201\\
67.08	0.00798755242807654\\
67.09	0.0079875528753249\\
67.1	0.00798755332439322\\
67.11	0.0079875537752876\\
67.12	0.0079875542280141\\
67.13	0.00798755468257875\\
67.14	0.00798755513898756\\
67.15	0.00798755559724645\\
67.16	0.00798755605736136\\
67.17	0.00798755651933813\\
67.18	0.00798755698318262\\
67.19	0.00798755744890058\\
67.2	0.00798755791649777\\
67.21	0.00798755838597988\\
67.22	0.00798755885735255\\
67.23	0.00798755933062139\\
67.24	0.00798755980579195\\
67.25	0.00798756028286972\\
67.26	0.00798756076186015\\
67.27	0.00798756124276865\\
67.28	0.00798756172560057\\
67.29	0.00798756221036118\\
67.3	0.00798756269705572\\
67.31	0.00798756318568938\\
67.32	0.00798756367626727\\
67.33	0.00798756416879446\\
67.34	0.00798756466327594\\
67.35	0.00798756515971667\\
67.36	0.00798756565812151\\
67.37	0.00798756615849529\\
67.38	0.00798756666084275\\
67.39	0.00798756716516857\\
67.4	0.00798756767147739\\
67.41	0.00798756817977373\\
67.42	0.0079875686900621\\
67.43	0.00798756920234689\\
67.44	0.00798756971663244\\
67.45	0.00798757023292302\\
67.46	0.00798757075122282\\
67.47	0.00798757127153595\\
67.48	0.00798757179386645\\
67.49	0.00798757231821829\\
67.5	0.00798757284459534\\
67.51	0.0079875733730014\\
67.52	0.00798757390344021\\
67.53	0.00798757443591538\\
67.54	0.00798757497043049\\
67.55	0.00798757550698899\\
67.56	0.00798757604559427\\
67.57	0.00798757658624962\\
67.58	0.00798757712895825\\
67.59	0.00798757767372327\\
67.6	0.00798757822054772\\
67.61	0.00798757876943451\\
67.62	0.00798757932038651\\
67.63	0.00798757987340643\\
67.64	0.00798758042849694\\
67.65	0.00798758098566059\\
67.66	0.00798758154489983\\
67.67	0.00798758210621702\\
67.68	0.0079875826696144\\
67.69	0.00798758323509413\\
67.7	0.00798758380265827\\
67.71	0.00798758437230874\\
67.72	0.0079875849440474\\
67.73	0.00798758551787598\\
67.74	0.00798758609379611\\
67.75	0.0079875866718093\\
67.76	0.00798758725191696\\
67.77	0.0079875878341204\\
67.78	0.00798758841842079\\
67.79	0.00798758900481922\\
67.8	0.00798758959331665\\
67.81	0.00798759018391392\\
67.82	0.00798759077661176\\
67.83	0.0079875913714108\\
67.84	0.00798759196831152\\
67.85	0.00798759256731431\\
67.86	0.00798759316841943\\
67.87	0.00798759377162702\\
67.88	0.00798759437693709\\
67.89	0.00798759498434956\\
67.9	0.00798759559386419\\
67.91	0.00798759620548064\\
67.92	0.00798759681919844\\
67.93	0.007987597435017\\
67.94	0.00798759805293559\\
67.95	0.00798759867295337\\
67.96	0.00798759929506937\\
67.97	0.00798759991928249\\
67.98	0.00798760054559151\\
67.99	0.00798760117399506\\
68	0.00798760180449168\\
68.01	0.00798760243707976\\
68.02	0.00798760307175754\\
68.03	0.00798760370852318\\
68.04	0.00798760434737466\\
68.05	0.00798760498830987\\
68.06	0.00798760563132655\\
68.07	0.00798760627642232\\
68.08	0.00798760692359465\\
68.09	0.0079876075728409\\
68.1	0.00798760822415831\\
68.11	0.00798760887754397\\
68.12	0.00798760953299484\\
68.13	0.00798761019050776\\
68.14	0.00798761085007944\\
68.15	0.00798761151170647\\
68.16	0.0079876121753853\\
68.17	0.00798761284111226\\
68.18	0.00798761350888354\\
68.19	0.00798761417869523\\
68.2	0.00798761485054327\\
68.21	0.00798761552442349\\
68.22	0.0079876162003316\\
68.23	0.00798761687826317\\
68.24	0.00798761755821367\\
68.25	0.00798761824017844\\
68.26	0.0079876189241527\\
68.27	0.00798761961013155\\
68.28	0.00798762029811\\
68.29	0.00798762098808292\\
68.3	0.00798762168004508\\
68.31	0.00798762237399113\\
68.32	0.00798762306991562\\
68.33	0.00798762376781299\\
68.34	0.0079876244676776\\
68.35	0.00798762516950367\\
68.36	0.00798762587328535\\
68.37	0.00798762657901667\\
68.38	0.00798762728669161\\
68.39	0.00798762799630401\\
68.4	0.00798762870784764\\
68.41	0.00798762942131621\\
68.42	0.00798763013670332\\
68.43	0.00798763085400251\\
68.44	0.00798763157320723\\
68.45	0.00798763229431087\\
68.46	0.00798763301730676\\
68.47	0.00798763374218816\\
68.48	0.00798763446894829\\
68.49	0.00798763519758028\\
68.5	0.00798763592807726\\
68.51	0.00798763666043228\\
68.52	0.00798763739463837\\
68.53	0.00798763813068852\\
68.54	0.0079876388685757\\
68.55	0.00798763960829286\\
68.56	0.00798764034983293\\
68.57	0.00798764109318883\\
68.58	0.00798764183835348\\
68.59	0.0079876425853198\\
68.6	0.00798764333408074\\
68.61	0.00798764408462923\\
68.62	0.00798764483695827\\
68.63	0.00798764559106087\\
68.64	0.00798764634693008\\
68.65	0.007987647104559\\
68.66	0.0079876478639408\\
68.67	0.0079876486250687\\
68.68	0.00798764938793601\\
68.69	0.00798765015253612\\
68.7	0.00798765091886252\\
68.71	0.00798765168690879\\
68.72	0.00798765245666864\\
68.73	0.0079876532281359\\
68.74	0.00798765400130453\\
68.75	0.00798765477616866\\
68.76	0.00798765555272256\\
68.77	0.00798765633096068\\
68.78	0.00798765711087764\\
68.79	0.00798765789246829\\
68.8	0.00798765867572765\\
68.81	0.007987659460651\\
68.82	0.00798766024723383\\
68.83	0.00798766103547191\\
68.84	0.00798766182536124\\
68.85	0.00798766261689814\\
68.86	0.0079876634100792\\
68.87	0.00798766420490135\\
68.88	0.00798766500136184\\
68.89	0.00798766579945825\\
68.9	0.00798766659918856\\
68.91	0.0079876674005511\\
68.92	0.00798766820354463\\
68.93	0.00798766900816832\\
68.94	0.00798766981442178\\
68.95	0.00798767062230508\\
68.96	0.00798767143181879\\
68.97	0.00798767224296396\\
68.98	0.00798767305574218\\
68.99	0.00798767387015559\\
69	0.00798767468620688\\
69.01	0.00798767550389938\\
69.02	0.00798767632323691\\
69.03	0.00798767714422326\\
69.04	0.00798767796686219\\
69.05	0.00798767879115736\\
69.06	0.00798767961711244\\
69.07	0.00798768044473098\\
69.08	0.00798768127401653\\
69.09	0.00798768210497256\\
69.1	0.00798768293760247\\
69.11	0.00798768377190964\\
69.12	0.00798768460789734\\
69.13	0.00798768544556881\\
69.14	0.00798768628492723\\
69.15	0.00798768712597571\\
69.16	0.00798768796871727\\
69.17	0.0079876888131549\\
69.18	0.00798768965929151\\
69.19	0.00798769050712992\\
69.2	0.00798769135667291\\
69.21	0.00798769220792316\\
69.22	0.00798769306088329\\
69.23	0.00798769391555585\\
69.24	0.00798769477194329\\
69.25	0.00798769563004801\\
69.26	0.00798769648987231\\
69.27	0.00798769735141841\\
69.28	0.00798769821468846\\
69.29	0.00798769907968451\\
69.3	0.00798769994640853\\
69.31	0.0079877008148624\\
69.32	0.00798770168504791\\
69.33	0.00798770255696676\\
69.34	0.00798770343062055\\
69.35	0.00798770430601079\\
69.36	0.00798770518313891\\
69.37	0.00798770606200622\\
69.38	0.00798770694261393\\
69.39	0.00798770782496315\\
69.4	0.00798770870905489\\
69.41	0.00798770959489007\\
69.42	0.00798771048246948\\
69.43	0.0079877113717938\\
69.44	0.00798771226286362\\
69.45	0.00798771315567939\\
69.46	0.00798771405024148\\
69.47	0.0079877149465501\\
69.48	0.00798771584460538\\
69.49	0.00798771674440731\\
69.5	0.00798771764595576\\
69.51	0.00798771854925047\\
69.52	0.00798771945429108\\
69.53	0.00798772036107707\\
69.54	0.0079877212696078\\
69.55	0.0079877221798825\\
69.56	0.00798772309190027\\
69.57	0.00798772400566006\\
69.58	0.0079877249211607\\
69.59	0.00798772583840087\\
69.6	0.00798772675737909\\
69.61	0.00798772767809377\\
69.62	0.00798772860054313\\
69.63	0.00798772952472529\\
69.64	0.00798773045063817\\
69.65	0.00798773137827958\\
69.66	0.00798773230764713\\
69.67	0.00798773323873831\\
69.68	0.00798773417155043\\
69.69	0.00798773510608064\\
69.7	0.00798773604232592\\
69.71	0.0079877369802831\\
69.72	0.00798773791994883\\
69.73	0.00798773886131958\\
69.74	0.00798773980439165\\
69.75	0.00798774074916117\\
69.76	0.00798774169562409\\
69.77	0.00798774264377617\\
69.78	0.00798774359361299\\
69.79	0.00798774454512996\\
69.8	0.00798774549832227\\
69.81	0.00798774645318495\\
69.82	0.00798774740971282\\
69.83	0.00798774836790049\\
69.84	0.00798774932774241\\
69.85	0.00798775028923279\\
69.86	0.00798775125236567\\
69.87	0.00798775221713485\\
69.88	0.00798775318353396\\
69.89	0.00798775415155639\\
69.9	0.00798775512119531\\
69.91	0.00798775609244371\\
69.92	0.00798775706529434\\
69.93	0.00798775803973972\\
69.94	0.00798775901577216\\
69.95	0.00798775999338375\\
69.96	0.00798776097256633\\
69.97	0.00798776195331152\\
69.98	0.00798776293561071\\
69.99	0.00798776391945505\\
70	0.00798776490483546\\
70.01	0.0079877658917426\\
70.02	0.0079877668801669\\
70.03	0.00798776787009853\\
70.04	0.00798776886152744\\
70.05	0.00798776985444329\\
70.06	0.00798777084883551\\
70.07	0.00798777184469326\\
70.08	0.00798777284200546\\
70.09	0.00798777384076075\\
70.1	0.0079877748409475\\
70.11	0.00798777584255384\\
70.12	0.0079877768455676\\
70.13	0.00798777784997636\\
70.14	0.0079877788557674\\
70.15	0.00798777986292776\\
70.16	0.00798778087144416\\
70.17	0.00798778188130307\\
70.18	0.00798778289249065\\
70.19	0.0079877839049928\\
70.2	0.00798778491879509\\
70.21	0.00798778593388283\\
70.22	0.00798778695024104\\
70.23	0.00798778796785442\\
70.24	0.00798778898670737\\
70.25	0.00798779000678401\\
70.26	0.00798779102806814\\
70.27	0.00798779205054326\\
70.28	0.00798779307419256\\
70.29	0.0079877940989989\\
70.3	0.00798779512494486\\
70.31	0.00798779615201268\\
70.32	0.0079877971801843\\
70.33	0.00798779820944132\\
70.34	0.00798779923976502\\
70.35	0.00798780027113638\\
70.36	0.00798780130353604\\
70.37	0.00798780233694429\\
70.38	0.00798780337134111\\
70.39	0.00798780440670617\\
70.4	0.00798780544301876\\
70.41	0.00798780648025787\\
70.42	0.00798780751840213\\
70.43	0.00798780855742985\\
70.44	0.00798780959731899\\
70.45	0.00798781063804715\\
70.46	0.00798781167959162\\
70.47	0.00798781272192933\\
70.48	0.00798781376503685\\
70.49	0.00798781480889041\\
70.5	0.0079878158534659\\
70.51	0.00798781689873885\\
70.52	0.00798781794468445\\
70.53	0.00798781899127751\\
70.54	0.00798782003849251\\
70.55	0.00798782108630357\\
70.56	0.00798782213468446\\
70.57	0.00798782318360857\\
70.58	0.00798782423304897\\
70.59	0.00798782528297833\\
70.6	0.007987826333369\\
70.61	0.00798782738419296\\
70.62	0.00798782843542181\\
70.63	0.00798782948702682\\
70.64	0.00798783053897889\\
70.65	0.00798783159124856\\
70.66	0.00798783264380601\\
70.67	0.00798783369662108\\
70.68	0.00798783474966322\\
70.69	0.00798783580290155\\
70.7	0.00798783685630483\\
70.71	0.00798783790984145\\
70.72	0.00798783896347946\\
70.73	0.00798784001718656\\
70.74	0.00798784107093008\\
70.75	0.00798784212467702\\
70.76	0.00798784317839403\\
70.77	0.00798784423204738\\
70.78	0.00798784528560305\\
70.79	0.00798784633902663\\
70.8	0.0079878473922834\\
70.81	0.00798784844533828\\
70.82	0.00798784949815588\\
70.83	0.00798785055070045\\
70.84	0.00798785160293593\\
70.85	0.00798785265482593\\
70.86	0.00798785370633373\\
70.87	0.00798785475742231\\
70.88	0.00798785580805433\\
70.89	0.00798785685819211\\
70.9	0.00798785790779771\\
70.91	0.00798785895683286\\
70.92	0.007987860005259\\
70.93	0.00798786105303728\\
70.94	0.00798786210012857\\
70.95	0.00798786314649346\\
70.96	0.00798786419209225\\
70.97	0.00798786523688499\\
70.98	0.00798786628083147\\
70.99	0.00798786732389121\\
71	0.0079878683660235\\
71.01	0.00798786940718739\\
71.02	0.00798787044734169\\
71.03	0.00798787148644499\\
71.04	0.00798787252445566\\
71.05	0.00798787356133188\\
71.06	0.00798787459703162\\
71.07	0.00798787563151267\\
71.08	0.00798787666473265\\
71.09	0.00798787769664899\\
71.1	0.00798787872721898\\
71.11	0.00798787975639978\\
71.12	0.00798788078414839\\
71.13	0.00798788181042171\\
71.14	0.00798788283517652\\
71.15	0.00798788385836951\\
71.16	0.00798788487995728\\
71.17	0.00798788589989637\\
71.18	0.00798788691814327\\
71.19	0.00798788793465442\\
71.2	0.00798788894938625\\
71.21	0.00798788996229517\\
71.22	0.00798789097333761\\
71.23	0.00798789198247\\
71.24	0.00798789298964885\\
71.25	0.0079878939948307\\
71.26	0.0079878949979722\\
71.27	0.00798789599903005\\
71.28	0.00798789699796114\\
71.29	0.00798789799472243\\
71.3	0.00798789898927107\\
71.31	0.00798789998156441\\
71.32	0.00798790097155997\\
71.33	0.00798790195921552\\
71.34	0.00798790294448908\\
71.35	0.00798790392733892\\
71.36	0.00798790490772365\\
71.37	0.00798790588560219\\
71.38	0.00798790686093379\\
71.39	0.00798790783367813\\
71.4	0.00798790880379525\\
71.41	0.00798790977124567\\
71.42	0.00798791073599035\\
71.43	0.00798791169799076\\
71.44	0.00798791265720891\\
71.45	0.00798791361360735\\
71.46	0.00798791456714925\\
71.47	0.0079879155177984\\
71.48	0.00798791646551926\\
71.49	0.007987917410277\\
71.5	0.00798791835203753\\
71.51	0.00798791929076751\\
71.52	0.00798792022643447\\
71.53	0.00798792115900675\\
71.54	0.00798792208845363\\
71.55	0.0079879230147453\\
71.56	0.00798792393785297\\
71.57	0.00798792485774884\\
71.58	0.00798792577440624\\
71.59	0.00798792668779958\\
71.6	0.00798792759790447\\
71.61	0.00798792850469773\\
71.62	0.00798792940815746\\
71.63	0.00798793030826306\\
71.64	0.00798793120499392\\
71.65	0.00798793209832947\\
71.66	0.00798793298824912\\
71.67	0.00798793387473233\\
71.68	0.00798793475775859\\
71.69	0.00798793563730738\\
71.7	0.00798793651335823\\
71.71	0.0079879373858907\\
71.72	0.00798793825488437\\
71.73	0.00798793912031887\\
71.74	0.00798793998217384\\
71.75	0.00798794084042898\\
71.76	0.00798794169506404\\
71.77	0.00798794254605878\\
71.78	0.00798794339339305\\
71.79	0.00798794423704672\\
71.8	0.00798794507699971\\
71.81	0.00798794591323202\\
71.82	0.00798794674572368\\
71.83	0.00798794757445481\\
71.84	0.00798794839940557\\
71.85	0.00798794922055619\\
71.86	0.00798795003788699\\
71.87	0.00798795085137834\\
71.88	0.00798795166101068\\
71.89	0.00798795246676456\\
71.9	0.0079879532686206\\
71.91	0.00798795406655947\\
71.92	0.00798795486056199\\
71.93	0.00798795565060902\\
71.94	0.00798795643668154\\
71.95	0.00798795721876062\\
71.96	0.00798795799682744\\
71.97	0.00798795877086327\\
71.98	0.00798795954084951\\
71.99	0.00798796030676765\\
72	0.00798796106859931\\
72.01	0.00798796182632621\\
72.02	0.00798796257993023\\
72.03	0.00798796332939334\\
72.04	0.00798796407469765\\
72.05	0.00798796481582541\\
72.06	0.00798796555275901\\
72.07	0.00798796628548096\\
72.08	0.00798796701397395\\
72.09	0.00798796773822079\\
72.1	0.00798796845820444\\
72.11	0.00798796917390805\\
72.12	0.00798796988531489\\
72.13	0.00798797059240843\\
72.14	0.00798797129517229\\
72.15	0.00798797199359027\\
72.16	0.00798797268764633\\
72.17	0.00798797337732464\\
72.18	0.00798797406260954\\
72.19	0.00798797474348555\\
72.2	0.00798797541993741\\
72.21	0.00798797609195003\\
72.22	0.00798797675950855\\
72.23	0.00798797742259829\\
72.24	0.00798797808120482\\
72.25	0.00798797873531388\\
72.26	0.00798797938491146\\
72.27	0.00798798002998378\\
72.28	0.00798798067051727\\
72.29	0.0079879813064986\\
72.3	0.00798798193791469\\
72.31	0.0079879825647527\\
72.32	0.00798798318700004\\
72.33	0.00798798380464436\\
72.34	0.00798798441767359\\
72.35	0.0079879850260759\\
72.36	0.00798798562983975\\
72.37	0.00798798622895386\\
72.38	0.00798798682340724\\
72.39	0.00798798741318916\\
72.4	0.0079879879982892\\
72.41	0.00798798857869724\\
72.42	0.00798798915440342\\
72.43	0.00798798972539822\\
72.44	0.00798799029167241\\
72.45	0.00798799085321708\\
72.46	0.00798799141002365\\
72.47	0.00798799196208383\\
72.48	0.0079879925093897\\
72.49	0.00798799305193363\\
72.5	0.00798799358970838\\
72.51	0.007987994122707\\
72.52	0.00798799465092295\\
72.53	0.00798799517434998\\
72.54	0.00798799569298225\\
72.55	0.00798799620681427\\
72.56	0.00798799671584091\\
72.57	0.00798799722005744\\
72.58	0.00798799771945948\\
72.59	0.00798799821404307\\
72.6	0.00798799870380461\\
72.61	0.00798799918874094\\
72.62	0.00798799966884925\\
72.63	0.00798800014412719\\
72.64	0.00798800061457278\\
72.65	0.00798800108018449\\
72.66	0.0079880015409612\\
72.67	0.00798800199690223\\
72.68	0.00798800244800731\\
72.69	0.00798800289427665\\
72.7	0.00798800333571087\\
72.71	0.00798800377231106\\
72.72	0.00798800420407876\\
72.73	0.00798800463101597\\
72.74	0.00798800505312517\\
72.75	0.00798800547040929\\
72.76	0.00798800588287175\\
72.77	0.00798800629051646\\
72.78	0.00798800669334779\\
72.79	0.00798800709137065\\
72.8	0.00798800748459038\\
72.81	0.00798800787301289\\
72.82	0.00798800825664455\\
72.83	0.00798800863549227\\
72.84	0.00798800900956346\\
72.85	0.00798800937886606\\
72.86	0.00798800974340854\\
72.87	0.00798801010319988\\
72.88	0.00798801045824962\\
72.89	0.00798801080856783\\
72.9	0.00798801115416514\\
72.91	0.00798801149505269\\
72.92	0.00798801183124222\\
72.93	0.007988012162746\\
72.94	0.00798801248957686\\
72.95	0.0079880128117482\\
72.96	0.00798801312927401\\
72.97	0.00798801344216882\\
72.98	0.00798801375044776\\
72.99	0.00798801405412653\\
73	0.00798801435322141\\
73.01	0.00798801464774929\\
73.02	0.00798801493772763\\
73.03	0.00798801522317448\\
73.04	0.0079880155041085\\
73.05	0.00798801578054894\\
73.06	0.00798801605251566\\
73.07	0.00798801632002912\\
73.08	0.00798801658311038\\
73.09	0.00798801684178112\\
73.1	0.00798801709606363\\
73.11	0.00798801734598081\\
73.12	0.00798801759155616\\
73.13	0.00798801783281381\\
73.14	0.00798801806977851\\
73.15	0.00798801830247562\\
73.16	0.00798801853093112\\
73.17	0.0079880187551716\\
73.18	0.00798801897522428\\
73.19	0.007988019191117\\
73.2	0.00798801940287821\\
73.21	0.00798801961053698\\
73.22	0.007988019814123\\
73.23	0.00798802001366658\\
73.24	0.00798802020919864\\
73.25	0.00798802040075071\\
73.26	0.00798802058835493\\
73.27	0.00798802077204406\\
73.28	0.00798802095185146\\
73.29	0.00798802112781109\\
73.3	0.00798802129995752\\
73.31	0.00798802146832589\\
73.32	0.00798802163295197\\
73.33	0.0079880217938721\\
73.34	0.00798802195112319\\
73.35	0.00798802210474275\\
73.36	0.00798802225476885\\
73.37	0.00798802240124014\\
73.38	0.0079880225441958\\
73.39	0.00798802268367558\\
73.4	0.0079880228197198\\
73.41	0.00798802295236927\\
73.42	0.00798802308166536\\
73.43	0.00798802320764996\\
73.44	0.00798802333036546\\
73.45	0.00798802344985475\\
73.46	0.00798802356616121\\
73.47	0.0079880236793287\\
73.48	0.00798802378940156\\
73.49	0.00798802389642457\\
73.5	0.00798802400044294\\
73.51	0.00798802410150235\\
73.52	0.00798802419964885\\
73.53	0.0079880242949289\\
73.54	0.00798802438738937\\
73.55	0.00798802447707746\\
73.56	0.00798802456404074\\
73.57	0.00798802464832711\\
73.58	0.00798802472998478\\
73.59	0.00798802480906225\\
73.6	0.00798802488560831\\
73.61	0.00798802495967199\\
73.62	0.00798802503130255\\
73.63	0.00798802510054947\\
73.64	0.0079880251674624\\
73.65	0.00798802523209116\\
73.66	0.00798802529448569\\
73.67	0.00798802535469607\\
73.68	0.00798802541277243\\
73.69	0.00798802546876495\\
73.7	0.00798802552272387\\
73.71	0.00798802557469938\\
73.72	0.00798802562474166\\
73.73	0.00798802567290079\\
73.74	0.00798802571922677\\
73.75	0.00798802576376944\\
73.76	0.00798802580657847\\
73.77	0.00798802584770332\\
73.78	0.00798802588719319\\
73.79	0.00798802592509697\\
73.8	0.00798802596146324\\
73.81	0.00798802599634018\\
73.82	0.00798802602977557\\
73.83	0.00798802606181669\\
73.84	0.00798802609251033\\
73.85	0.0079880261219027\\
73.86	0.0079880261500394\\
73.87	0.00798802617696536\\
73.88	0.0079880262027248\\
73.89	0.00798802622736115\\
73.9	0.00798802625091702\\
73.91	0.00798802627343412\\
73.92	0.00798802629495321\\
73.93	0.00798802631551403\\
73.94	0.00798802633515527\\
73.95	0.00798802635391446\\
73.96	0.00798802637182789\\
73.97	0.00798802638893063\\
73.98	0.00798802640525634\\
73.99	0.0079880264208373\\
74	0.00798802643570432\\
74.01	0.00798802644988812\\
74.02	0.00798802646341924\\
74.03	0.00798802647632804\\
74.04	0.00798802648864468\\
74.05	0.00798802650039909\\
74.06	0.00798802651162094\\
74.07	0.00798802652233961\\
74.08	0.00798802653258419\\
74.09	0.00798802654238341\\
74.1	0.00798802655176561\\
74.11	0.00798802656075878\\
74.12	0.00798802656939043\\
74.13	0.00798802657768763\\
74.14	0.00798802658567693\\
74.15	0.00798802659338435\\
74.16	0.00798802660083536\\
74.17	0.00798802660805479\\
74.18	0.00798802661506685\\
74.19	0.00798802662189504\\
74.2	0.00798802662856217\\
74.21	0.00798802663509024\\
74.22	0.00798802664150047\\
74.23	0.00798802664781323\\
74.24	0.00798802665404796\\
74.25	0.00798802666022318\\
74.26	0.00798802666635641\\
74.27	0.00798802667246413\\
74.28	0.00798802667856173\\
74.29	0.00798802668466343\\
74.3	0.00798802669078229\\
74.31	0.00798802669693009\\
74.32	0.00798802670311731\\
74.33	0.00798802670935305\\
74.34	0.007988026715645\\
74.35	0.00798802672199934\\
74.36	0.00798802672842073\\
74.37	0.00798802673491259\\
74.38	0.00798802674147775\\
74.39	0.00798802674811833\\
74.4	0.00798802675483581\\
74.41	0.00798802676163107\\
74.42	0.00798802676850502\\
74.43	0.00798802677545855\\
74.44	0.0079880267824926\\
74.45	0.00798802678960808\\
74.46	0.00798802679680594\\
74.47	0.00798802680408713\\
74.48	0.00798802681145262\\
74.49	0.00798802681890337\\
74.5	0.00798802682644037\\
74.51	0.00798802683406463\\
74.52	0.00798802684177713\\
74.53	0.00798802684957891\\
74.54	0.00798802685747099\\
74.55	0.00798802686545442\\
74.56	0.00798802687353026\\
74.57	0.00798802688169956\\
74.58	0.00798802688996341\\
74.59	0.00798802689832289\\
74.6	0.00798802690677912\\
74.61	0.00798802691533321\\
74.62	0.00798802692398629\\
74.63	0.00798802693273951\\
74.64	0.007988026941594\\
74.65	0.00798802695055096\\
74.66	0.00798802695961156\\
74.67	0.00798802696877699\\
74.68	0.00798802697804846\\
74.69	0.00798802698742721\\
74.7	0.00798802699691446\\
74.71	0.00798802700651147\\
74.72	0.00798802701621951\\
74.73	0.00798802702603985\\
74.74	0.0079880270359738\\
74.75	0.00798802704602266\\
74.76	0.00798802705618776\\
74.77	0.00798802706647044\\
74.78	0.00798802707687205\\
74.79	0.00798802708739398\\
74.8	0.0079880270980376\\
74.81	0.00798802710880432\\
74.82	0.00798802711969557\\
74.83	0.00798802713071277\\
74.84	0.00798802714185738\\
74.85	0.00798802715313087\\
74.86	0.00798802716453472\\
74.87	0.00798802717607045\\
74.88	0.00798802718773957\\
74.89	0.00798802719954361\\
74.9	0.00798802721148414\\
74.91	0.00798802722356273\\
74.92	0.00798802723578097\\
74.93	0.00798802724814047\\
74.94	0.00798802726064286\\
74.95	0.00798802727328979\\
74.96	0.00798802728608292\\
74.97	0.00798802729902395\\
74.98	0.00798802731211456\\
74.99	0.0079880273253565\\
75	0.0079880273387515\\
75.01	0.00798802735230133\\
75.02	0.00798802736600777\\
75.03	0.00798802737987264\\
75.04	0.00798802739389774\\
75.05	0.00798802740808494\\
75.06	0.0079880274224361\\
75.07	0.00798802743695311\\
75.08	0.00798802745163788\\
75.09	0.00798802746649234\\
75.1	0.00798802748151846\\
75.11	0.0079880274967182\\
75.12	0.00798802751209357\\
75.13	0.00798802752764659\\
75.14	0.00798802754337932\\
75.15	0.00798802755929382\\
75.16	0.00798802757539218\\
75.17	0.00798802759167652\\
75.18	0.00798802760814899\\
75.19	0.00798802762481176\\
75.2	0.00798802764166701\\
75.21	0.00798802765871696\\
75.22	0.00798802767596386\\
75.23	0.00798802769340997\\
75.24	0.0079880277110576\\
75.25	0.00798802772890905\\
75.26	0.00798802774696668\\
75.27	0.00798802776523287\\
75.28	0.00798802778371001\\
75.29	0.00798802780240054\\
75.3	0.0079880278213069\\
75.31	0.0079880278404316\\
75.32	0.00798802785977713\\
75.33	0.00798802787934606\\
75.34	0.00798802789914093\\
75.35	0.00798802791916437\\
75.36	0.007988027939419\\
75.37	0.00798802795990747\\
75.38	0.0079880279806325\\
75.39	0.00798802800159679\\
75.4	0.0079880280228031\\
75.41	0.00798802804425422\\
75.42	0.00798802806595297\\
75.43	0.00798802808790219\\
75.44	0.00798802811010477\\
75.45	0.00798802813256363\\
75.46	0.00798802815528171\\
75.47	0.00798802817826201\\
75.48	0.00798802820150753\\
75.49	0.00798802822502132\\
75.5	0.00798802824880649\\
75.51	0.00798802827286614\\
75.52	0.00798802829720344\\
75.53	0.00798802832182159\\
75.54	0.00798802834672381\\
75.55	0.00798802837191337\\
75.56	0.00798802839739357\\
75.57	0.00798802842316778\\
75.58	0.00798802844923935\\
75.59	0.00798802847561172\\
75.6	0.00798802850228835\\
75.61	0.00798802852927273\\
75.62	0.00798802855656841\\
75.63	0.00798802858417897\\
75.64	0.00798802861210802\\
75.65	0.00798802864035924\\
75.66	0.00798802866893632\\
75.67	0.00798802869784302\\
75.68	0.00798802872708312\\
75.69	0.00798802875666046\\
75.7	0.00798802878657891\\
75.71	0.00798802881684241\\
75.72	0.00798802884745491\\
75.73	0.00798802887842044\\
75.74	0.00798802890974304\\
75.75	0.00798802894142683\\
75.76	0.00798802897347595\\
75.77	0.00798802900589461\\
75.78	0.00798802903868706\\
75.79	0.00798802907185759\\
75.8	0.00798802910541054\\
75.81	0.00798802913935032\\
75.82	0.00798802917368137\\
75.83	0.00798802920840818\\
75.84	0.00798802924353531\\
75.85	0.00798802927906736\\
75.86	0.00798802931500898\\
75.87	0.00798802935136487\\
75.88	0.0079880293881398\\
75.89	0.00798802942533858\\
75.9	0.00798802946296608\\
75.91	0.00798802950102724\\
75.92	0.00798802953952702\\
75.93	0.00798802957847047\\
75.94	0.0079880296178627\\
75.95	0.00798802965770884\\
75.96	0.00798802969801412\\
75.97	0.00798802973878382\\
75.98	0.00798802978002326\\
75.99	0.00798802982173785\\
76	0.00798802986393303\\
76.01	0.00798802990661433\\
76.02	0.00798802994978734\\
76.03	0.0079880299934577\\
76.04	0.00798803003763112\\
76.05	0.00798803008231337\\
76.06	0.00798803012751031\\
76.07	0.00798803017322785\\
76.08	0.00798803021947195\\
76.09	0.00798803026624867\\
76.1	0.00798803031356412\\
76.11	0.00798803036142448\\
76.12	0.00798803040983602\\
76.13	0.00798803045880505\\
76.14	0.00798803050833798\\
76.15	0.00798803055844129\\
76.16	0.00798803060912152\\
76.17	0.00798803066038528\\
76.18	0.00798803071223929\\
76.19	0.00798803076469031\\
76.2	0.0079880308177452\\
76.21	0.00798803087141088\\
76.22	0.00798803092569437\\
76.23	0.00798803098060274\\
76.24	0.00798803103614319\\
76.25	0.00798803109232295\\
76.26	0.00798803114914937\\
76.27	0.00798803120662986\\
76.28	0.00798803126477192\\
76.29	0.00798803132358315\\
76.3	0.00798803138307122\\
76.31	0.0079880314432439\\
76.32	0.00798803150410905\\
76.33	0.00798803156567459\\
76.34	0.00798803162794858\\
76.35	0.00798803169093913\\
76.36	0.00798803175465445\\
76.37	0.00798803181910288\\
76.38	0.0079880318842928\\
76.39	0.00798803195023273\\
76.4	0.00798803201693127\\
76.41	0.0079880320843971\\
76.42	0.00798803215263904\\
76.43	0.00798803222166598\\
76.44	0.00798803229148693\\
76.45	0.00798803236211097\\
76.46	0.00798803243354733\\
76.47	0.00798803250580531\\
76.48	0.00798803257889433\\
76.49	0.00798803265282392\\
76.5	0.00798803272760372\\
76.51	0.00798803280324346\\
76.52	0.00798803287975301\\
76.53	0.00798803295714233\\
76.54	0.0079880330354215\\
76.55	0.00798803311460073\\
76.56	0.00798803319469033\\
76.57	0.00798803327570072\\
76.58	0.00798803335764247\\
76.59	0.00798803344052624\\
76.6	0.00798803352436282\\
76.61	0.00798803360916314\\
76.62	0.00798803369493823\\
76.63	0.00798803378169925\\
76.64	0.00798803386945751\\
76.65	0.00798803395822443\\
76.66	0.00798803404801156\\
76.67	0.00798803413883059\\
76.68	0.00798803423069333\\
76.69	0.00798803432361174\\
76.7	0.0079880344175979\\
76.71	0.00798803451266406\\
76.72	0.00798803460882257\\
76.73	0.00798803470608594\\
76.74	0.00798803480446683\\
76.75	0.00798803490397802\\
76.76	0.00798803500463247\\
76.77	0.00798803510644325\\
76.78	0.00798803520942362\\
76.79	0.00798803531358695\\
76.8	0.00798803541894679\\
76.81	0.00798803552551683\\
76.82	0.00798803563331094\\
76.83	0.00798803574234311\\
76.84	0.00798803585262752\\
76.85	0.0079880359641785\\
76.86	0.00798803607701055\\
76.87	0.00798803619113832\\
76.88	0.00798803630657664\\
76.89	0.00798803642334052\\
76.9	0.00798803654144511\\
76.91	0.00798803666090575\\
76.92	0.00798803678173797\\
76.93	0.00798803690395745\\
76.94	0.00798803702758006\\
76.95	0.00798803715262186\\
76.96	0.00798803727909908\\
76.97	0.00798803740702814\\
76.98	0.00798803753642564\\
76.99	0.00798803766730838\\
77	0.00798803779969334\\
77.01	0.00798803793359772\\
77.02	0.00798803806903887\\
77.03	0.00798803820603439\\
77.04	0.00798803834460203\\
77.05	0.00798803848475978\\
77.06	0.00798803862652581\\
77.07	0.00798803876991852\\
77.08	0.00798803891495651\\
77.09	0.00798803906165857\\
77.1	0.00798803921004374\\
77.11	0.00798803936013126\\
77.12	0.00798803951194058\\
77.13	0.00798803966549138\\
77.14	0.00798803982080357\\
77.15	0.00798803997789728\\
77.16	0.00798804013679287\\
77.17	0.00798804029751093\\
77.18	0.00798804046007228\\
77.19	0.007988040624498\\
77.2	0.00798804079080938\\
77.21	0.00798804095902797\\
77.22	0.00798804112917555\\
77.23	0.00798804130127417\\
77.24	0.00798804147534611\\
77.25	0.00798804165141393\\
77.26	0.0079880418295004\\
77.27	0.0079880420096286\\
77.28	0.00798804219182185\\
77.29	0.00798804237610372\\
77.3	0.00798804256249808\\
77.31	0.00798804275102905\\
77.32	0.00798804294172102\\
77.33	0.00798804313459868\\
77.34	0.00798804332968697\\
77.35	0.00798804352701116\\
77.36	0.00798804372659675\\
77.37	0.00798804392846957\\
77.38	0.00798804413265574\\
77.39	0.00798804433918165\\
77.4	0.00798804454807402\\
77.41	0.00798804475935986\\
77.42	0.00798804497306649\\
77.43	0.00798804518922154\\
77.44	0.00798804540785295\\
77.45	0.00798804562898899\\
77.46	0.00798804585265823\\
77.47	0.00798804607888959\\
77.48	0.0079880463077123\\
77.49	0.00798804653915592\\
77.5	0.00798804677325036\\
77.51	0.00798804701002586\\
77.52	0.00798804724951302\\
77.53	0.00798804749174275\\
77.54	0.00798804773674636\\
77.55	0.00798804798455547\\
77.56	0.00798804823520209\\
77.57	0.00798804848871859\\
77.58	0.00798804874513769\\
77.59	0.00798804900449249\\
77.6	0.00798804926681648\\
77.61	0.00798804953214351\\
77.62	0.00798804980050782\\
77.63	0.00798805007194404\\
77.64	0.00798805034648721\\
77.65	0.00798805062417273\\
77.66	0.00798805090503645\\
77.67	0.00798805118911457\\
77.68	0.00798805147644376\\
77.69	0.00798805176706105\\
77.7	0.00798805206100394\\
77.71	0.00798805235831032\\
77.72	0.00798805265901852\\
77.73	0.00798805296316733\\
77.74	0.00798805327079593\\
77.75	0.00798805358194398\\
77.76	0.0079880538966516\\
77.77	0.00798805421495932\\
77.78	0.00798805453690818\\
77.79	0.00798805486253964\\
77.8	0.00798805519189567\\
77.81	0.00798805552501869\\
77.82	0.00798805586195161\\
77.83	0.00798805620273782\\
77.84	0.00798805654742122\\
77.85	0.00798805689604618\\
77.86	0.0079880572486576\\
77.87	0.00798805760530087\\
77.88	0.0079880579660219\\
77.89	0.00798805833086712\\
77.9	0.0079880586998835\\
77.91	0.00798805907311852\\
77.92	0.00798805945062021\\
77.93	0.00798805983243715\\
77.94	0.00798806021861844\\
77.95	0.00798806060921378\\
77.96	0.00798806100427341\\
77.97	0.00798806140384813\\
77.98	0.00798806180798933\\
77.99	0.00798806221674898\\
78	0.00798806263017964\\
78.01	0.00798806304833445\\
78.02	0.00798806347126719\\
78.03	0.00798806389903219\\
78.04	0.00798806433168445\\
78.05	0.00798806476927956\\
78.06	0.00798806521187376\\
78.07	0.00798806565952391\\
78.08	0.00798806611228752\\
78.09	0.00798806657022275\\
78.1	0.00798806703338843\\
78.11	0.00798806750184403\\
78.12	0.00798806797564971\\
78.13	0.00798806845486632\\
78.14	0.00798806893955538\\
78.15	0.00798806942977912\\
78.16	0.00798806992560046\\
78.17	0.00798807042708304\\
78.18	0.00798807093429122\\
78.19	0.0079880714472901\\
78.2	0.00798807196614548\\
78.21	0.00798807249092395\\
78.22	0.00798807302169282\\
78.23	0.00798807355852017\\
78.24	0.00798807410147486\\
78.25	0.00798807465062652\\
78.26	0.00798807520604556\\
78.27	0.0079880757678032\\
78.28	0.00798807633597146\\
78.29	0.00798807691062317\\
78.3	0.00798807749183199\\
78.31	0.0079880780796724\\
78.32	0.00798807867421975\\
78.33	0.0079880792755502\\
78.34	0.0079880798837408\\
78.35	0.00798808049886947\\
78.36	0.007988081121015\\
78.37	0.00798808175025707\\
78.38	0.00798808238667627\\
78.39	0.0079880830303541\\
78.4	0.00798808368137296\\
78.41	0.00798808433981621\\
78.42	0.00798808500576813\\
78.43	0.00798808567931396\\
78.44	0.0079880863605399\\
78.45	0.00798808704953314\\
78.46	0.00798808774638182\\
78.47	0.0079880884511751\\
78.48	0.00798808916400314\\
78.49	0.00798808988495713\\
78.5	0.00798809061412927\\
78.51	0.00798809135161282\\
78.52	0.00798809209750206\\
78.53	0.00798809285189238\\
78.54	0.0079880936148802\\
78.55	0.00798809438656307\\
78.56	0.00798809516703961\\
78.57	0.00798809595640956\\
78.58	0.0079880967547738\\
78.59	0.00798809756223434\\
78.6	0.00798809837889432\\
78.61	0.00798809920485807\\
78.62	0.00798810004023109\\
78.63	0.00798810088512006\\
78.64	0.00798810173963289\\
78.65	0.00798810260387866\\
78.66	0.00798810347796773\\
78.67	0.00798810436201167\\
78.68	0.00798810525612332\\
78.69	0.00798810616041679\\
78.7	0.00798810707500749\\
78.71	0.0079881080000121\\
78.72	0.00798810893554864\\
78.73	0.00798810988173646\\
78.74	0.00798811083869625\\
78.75	0.00798811180655005\\
78.76	0.00798811278542128\\
78.77	0.00798811377543476\\
78.78	0.00798811477671672\\
78.79	0.00798811578939479\\
78.8	0.00798811681359804\\
78.81	0.00798811784945703\\
78.82	0.00798811889710374\\
78.83	0.00798811995667167\\
78.84	0.00798812102829582\\
78.85	0.00798812211211269\\
78.86	0.00798812320826034\\
78.87	0.00798812431687837\\
78.88	0.00798812543810795\\
78.89	0.00798812657209186\\
78.9	0.00798812771897446\\
78.91	0.00798812887890176\\
78.92	0.00798813005202139\\
78.93	0.00798813123848265\\
78.94	0.00798813243843653\\
78.95	0.00798813365203569\\
78.96	0.00798813487943456\\
78.97	0.00798813612078925\\
78.98	0.00798813737625766\\
78.99	0.00798813864599946\\
79	0.00798813993017611\\
79.01	0.00798814122895089\\
79.02	0.00798814254248891\\
79.03	0.00798814387095714\\
79.04	0.00798814521452444\\
79.05	0.00798814657336156\\
79.06	0.00798814794764115\\
79.07	0.00798814933753782\\
79.08	0.00798815074322816\\
79.09	0.0079881521648907\\
79.1	0.00798815360270602\\
79.11	0.00798815505685671\\
79.12	0.0079881565275274\\
79.13	0.00798815801490481\\
79.14	0.00798815951917776\\
79.15	0.00798816104053719\\
79.16	0.00798816257917618\\
79.17	0.00798816413528997\\
79.18	0.00798816570907601\\
79.19	0.00798816730073396\\
79.2	0.00798816891046571\\
79.21	0.00798817053847545\\
79.22	0.00798817218496963\\
79.23	0.00798817385015703\\
79.24	0.00798817553424877\\
79.25	0.00798817723745835\\
79.26	0.00798817896000165\\
79.27	0.007988180702097\\
79.28	0.00798818246396515\\
79.29	0.00798818424582933\\
79.3	0.0079881860479153\\
79.31	0.00798818787045134\\
79.32	0.00798818971366828\\
79.33	0.00798819157779955\\
79.34	0.00798819346308119\\
79.35	0.00798819536975191\\
79.36	0.00798819729805306\\
79.37	0.00798819924822874\\
79.38	0.00798820122052574\\
79.39	0.00798820321519367\\
79.4	0.00798820523248489\\
79.41	0.00798820727265461\\
79.42	0.00798820933596091\\
79.43	0.00798821142266474\\
79.44	0.00798821353303001\\
79.45	0.00798821566732354\\
79.46	0.00798821782581518\\
79.47	0.00798822000877778\\
79.48	0.00798822221648727\\
79.49	0.00798822444922265\\
79.5	0.00798822670726604\\
79.51	0.00798822899090275\\
79.52	0.00798823130042126\\
79.53	0.00798823363611329\\
79.54	0.00798823599827383\\
79.55	0.00798823838720116\\
79.56	0.00798824080319692\\
79.57	0.0079882432465661\\
79.58	0.00798824571761713\\
79.59	0.00798824821666189\\
79.6	0.00798825074401572\\
79.61	0.00798825329999754\\
79.62	0.00798825588492979\\
79.63	0.00798825849913854\\
79.64	0.00798826114295352\\
79.65	0.00798826381670812\\
79.66	0.00798826652073949\\
79.67	0.00798826925538853\\
79.68	0.00798827202099997\\
79.69	0.00798827481792237\\
79.7	0.00798827764650821\\
79.71	0.0079882805071139\\
79.72	0.00798828340009985\\
79.73	0.00798828632583048\\
79.74	0.0079882892846743\\
79.75	0.00798829227700392\\
79.76	0.00798829530319614\\
79.77	0.00798829836363195\\
79.78	0.00798830145869661\\
79.79	0.00798830458877967\\
79.8	0.00798830775427505\\
79.81	0.00798831095558107\\
79.82	0.00798831419310047\\
79.83	0.00798831746724054\\
79.84	0.00798832077841307\\
79.85	0.00798832412703448\\
79.86	0.00798832751352582\\
79.87	0.00798833093831285\\
79.88	0.0079883344018261\\
79.89	0.00798833790450088\\
79.9	0.00798834144677736\\
79.91	0.00798834502910065\\
79.92	0.0079883486519208\\
79.93	0.00798835231569289\\
79.94	0.00798835602087709\\
79.95	0.0079883597679387\\
79.96	0.00798836355734819\\
79.97	0.0079883673895813\\
79.98	0.00798837126511908\\
79.99	0.00798837518444794\\
80	0.00798837914805969\\
80.01	0.00798838315645166\\
};
\addplot [color=black,solid]
  table[row sep=crcr]{%
80.01	0.00798838315645166\\
80.02	0.00798838721012671\\
80.03	0.00798839130959331\\
80.04	0.0079883954553656\\
80.05	0.00798839964796343\\
80.06	0.00798840388791248\\
80.07	0.00798840817574426\\
80.08	0.00798841251199622\\
80.09	0.00798841689721178\\
80.1	0.00798842133194044\\
80.11	0.00798842581673781\\
80.12	0.00798843035216568\\
80.13	0.00798843493879211\\
80.14	0.00798843957719148\\
80.15	0.00798844426794456\\
80.16	0.00798844901163862\\
80.17	0.00798845380886742\\
80.18	0.00798845866023137\\
80.19	0.00798846356633754\\
80.2	0.00798846852779978\\
80.21	0.00798847354523874\\
80.22	0.00798847861928201\\
80.23	0.00798848375056415\\
80.24	0.00798848893972677\\
80.25	0.00798849418741863\\
80.26	0.00798849949429572\\
80.27	0.00798850486102129\\
80.28	0.007988510288266\\
80.29	0.00798851577670795\\
80.3	0.0079885213270328\\
80.31	0.00798852693993382\\
80.32	0.00798853261611198\\
80.33	0.00798853835627605\\
80.34	0.00798854416114268\\
80.35	0.0079885500314365\\
80.36	0.00798855596789017\\
80.37	0.00798856197124449\\
80.38	0.0079885680422485\\
80.39	0.00798857418165958\\
80.4	0.00798858039024349\\
80.41	0.00798858666877452\\
80.42	0.00798859301803554\\
80.43	0.00798859943881814\\
80.44	0.00798860593192267\\
80.45	0.00798861249815839\\
80.46	0.00798861913834352\\
80.47	0.0079886258533054\\
80.48	0.00798863264388052\\
80.49	0.00798863951091467\\
80.5	0.00798864645526301\\
80.51	0.00798865347779021\\
80.52	0.00798866057937051\\
80.53	0.00798866776088788\\
80.54	0.00798867502323606\\
80.55	0.00798868236731873\\
80.56	0.00798868979404957\\
80.57	0.00798869730435242\\
80.58	0.00798870489916133\\
80.59	0.00798871257942074\\
80.6	0.00798872034608552\\
80.61	0.00798872820012116\\
80.62	0.00798873614250384\\
80.63	0.00798874417422054\\
80.64	0.0079887522962692\\
80.65	0.0079887605096588\\
80.66	0.00798876881540951\\
80.67	0.00798877721455279\\
80.68	0.00798878570813153\\
80.69	0.00798879429720017\\
80.7	0.00798880298282484\\
80.71	0.00798881176608343\\
80.72	0.00798882064806582\\
80.73	0.0079888296298739\\
80.74	0.00798883871262181\\
80.75	0.00798884789743598\\
80.76	0.00798885718545532\\
80.77	0.00798886657783134\\
80.78	0.00798887607572829\\
80.79	0.00798888568032331\\
80.8	0.00798889539280654\\
80.81	0.00798890521438129\\
80.82	0.00798891514626419\\
80.83	0.0079889251896853\\
80.84	0.00798893534588831\\
80.85	0.00798894561613062\\
80.86	0.00798895600168356\\
80.87	0.00798896650383251\\
80.88	0.00798897712387703\\
80.89	0.00798898786313106\\
80.9	0.00798899872292305\\
80.91	0.00798900970459613\\
80.92	0.00798902080950827\\
80.93	0.00798903203903242\\
80.94	0.00798904339455671\\
80.95	0.00798905487748459\\
80.96	0.00798906648923502\\
80.97	0.00798907823124259\\
80.98	0.00798909010495776\\
80.99	0.00798910211184698\\
81	0.0079891142533929\\
81.01	0.0079891265310945\\
81.02	0.00798913894646733\\
81.03	0.00798915150104364\\
81.04	0.00798916419637259\\
81.05	0.00798917703402042\\
81.06	0.00798919001557064\\
81.07	0.00798920314262424\\
81.08	0.00798921641679983\\
81.09	0.00798922983973389\\
81.1	0.00798924341308093\\
81.11	0.00798925713851367\\
81.12	0.00798927101772331\\
81.13	0.00798928505241965\\
81.14	0.00798929924433132\\
81.15	0.00798931359520604\\
81.16	0.00798932810681071\\
81.17	0.00798934278093176\\
81.18	0.00798935761937524\\
81.19	0.0079893726239671\\
81.2	0.00798938779655341\\
81.21	0.00798940313900053\\
81.22	0.00798941865319539\\
81.23	0.00798943434104566\\
81.24	0.00798945020448003\\
81.25	0.00798946624544837\\
81.26	0.00798948246592204\\
81.27	0.00798949886789407\\
81.28	0.00798951545337938\\
81.29	0.0079895322244151\\
81.3	0.00798954918306071\\
81.31	0.00798956633139837\\
81.32	0.0079895836715331\\
81.33	0.00798960120559306\\
81.34	0.00798961893572981\\
81.35	0.00798963686411855\\
81.36	0.00798965499295837\\
81.37	0.00798967332447251\\
81.38	0.00798969186090864\\
81.39	0.00798971060453913\\
81.4	0.00798972955766128\\
81.41	0.00798974872259762\\
81.42	0.00798976810169617\\
81.43	0.00798978769733075\\
81.44	0.00798980751190119\\
81.45	0.00798982754783369\\
81.46	0.00798984780758107\\
81.47	0.00798986829362304\\
81.48	0.00798988900846652\\
81.49	0.00798990995464593\\
81.5	0.00798993113472348\\
81.51	0.00798995255128947\\
81.52	0.00798997420696261\\
81.53	0.0079899961043903\\
81.54	0.00799001824624898\\
81.55	0.0079900406352444\\
81.56	0.00799006327411197\\
81.57	0.00799008616561706\\
81.58	0.00799010931255535\\
81.59	0.00799013271775312\\
81.6	0.00799015638406763\\
81.61	0.00799018031438739\\
81.62	0.00799020451163257\\
81.63	0.00799022897875529\\
81.64	0.00799025371873999\\
81.65	0.00799027873460376\\
81.66	0.00799030402939673\\
81.67	0.00799032960620236\\
81.68	0.00799035546813788\\
81.69	0.0079903816183546\\
81.7	0.0079904080600383\\
81.71	0.00799043479640959\\
81.72	0.00799046183072429\\
81.73	0.00799048916627381\\
81.74	0.00799051680638556\\
81.75	0.00799054475442328\\
81.76	0.00799057301378749\\
81.77	0.00799060158791586\\
81.78	0.00799063048028359\\
81.79	0.00799065969440388\\
81.8	0.00799068923382825\\
81.81	0.00799071910214703\\
81.82	0.00799074930298974\\
81.83	0.00799077984002552\\
81.84	0.00799081071696356\\
81.85	0.00799084193755352\\
81.86	0.00799087350558598\\
81.87	0.00799090542489288\\
81.88	0.00799093769934794\\
81.89	0.00799097033286715\\
81.9	0.00799100332940918\\
81.91	0.0079910366929759\\
81.92	0.00799107042761276\\
81.93	0.00799110453740934\\
81.94	0.00799113902649977\\
81.95	0.00799117389906324\\
81.96	0.00799120915932446\\
81.97	0.00799124481155419\\
81.98	0.00799128086006967\\
81.99	0.00799131730923517\\
82	0.00799135416346249\\
82.01	0.00799139142721145\\
82.02	0.00799142910499041\\
82.03	0.00799146720135681\\
82.04	0.00799150572091768\\
82.05	0.00799154466833017\\
82.06	0.00799158404830208\\
82.07	0.00799162386559245\\
82.08	0.00799166412501205\\
82.09	0.00799170483142395\\
82.1	0.00799174598974412\\
82.11	0.00799178760494194\\
82.12	0.0079918296820408\\
82.13	0.0079918722261187\\
82.14	0.00799191524230879\\
82.15	0.00799195873579998\\
82.16	0.00799200271183755\\
82.17	0.00799204717572374\\
82.18	0.00799209213281835\\
82.19	0.00799213758853939\\
82.2	0.00799218354836365\\
82.21	0.00799223001782737\\
82.22	0.00799227700252688\\
82.23	0.00799232450811919\\
82.24	0.0079923725403227\\
82.25	0.00799242110491783\\
82.26	0.00799247020774766\\
82.27	0.00799251985471865\\
82.28	0.00799257005180126\\
82.29	0.00799262080503067\\
82.3	0.00799267212050747\\
82.31	0.00799272400439834\\
82.32	0.00799277646293677\\
82.33	0.00799282950242376\\
82.34	0.00799288312922854\\
82.35	0.00799293734978932\\
82.36	0.00799299217061401\\
82.37	0.00799304759828094\\
82.38	0.00799310363943964\\
82.39	0.0079931603008116\\
82.4	0.00799321758919103\\
82.41	0.0079932755114456\\
82.42	0.00799333407451728\\
82.43	0.00799339328542309\\
82.44	0.00799345315125592\\
82.45	0.00799351367918533\\
82.46	0.00799357487645834\\
82.47	0.00799363675040031\\
82.48	0.00799369930841572\\
82.49	0.00799376255798903\\
82.5	0.00799382650668554\\
82.51	0.00799389116215222\\
82.52	0.00799395653211863\\
82.53	0.00799402262439772\\
82.54	0.0079940894468868\\
82.55	0.00799415700756836\\
82.56	0.00799422531451103\\
82.57	0.00799429437587046\\
82.58	0.00799436419989026\\
82.59	0.00799443479490292\\
82.6	0.00799450616933075\\
82.61	0.00799457833168687\\
82.62	0.00799465129057611\\
82.63	0.00799472505469604\\
82.64	0.00799479963283791\\
82.65	0.00799487503388766\\
82.66	0.00799495126682692\\
82.67	0.00799502805767265\\
82.68	0.00799510528795258\\
82.69	0.00799518296218263\\
82.7	0.00799526108492959\\
82.71	0.00799533966081167\\
82.72	0.0079954186944991\\
82.73	0.00799549819071473\\
82.74	0.00799557815423463\\
82.75	0.00799565858988867\\
82.76	0.00799573950256119\\
82.77	0.00799582089719155\\
82.78	0.00799590277877484\\
82.79	0.00799598515236244\\
82.8	0.00799606802306271\\
82.81	0.00799615139604162\\
82.82	0.00799623527652341\\
82.83	0.00799631966979129\\
82.84	0.00799640458118805\\
82.85	0.00799649001611678\\
82.86	0.00799657598004156\\
82.87	0.00799666247848815\\
82.88	0.00799674951704468\\
82.89	0.00799683710136238\\
82.9	0.0079969252371563\\
82.91	0.00799701393020602\\
82.92	0.00799710318635643\\
82.93	0.00799719301151841\\
82.94	0.00799728341166964\\
82.95	0.00799737439285533\\
82.96	0.00799746596118902\\
82.97	0.00799755812285333\\
82.98	0.00799765088410076\\
82.99	0.0079977442512545\\
83	0.00799783823070922\\
83.01	0.0079979328289319\\
83.02	0.00799802805246262\\
83.03	0.00799812390791546\\
83.04	0.00799822040197929\\
83.05	0.00799831754141864\\
83.06	0.00799841533307457\\
83.07	0.00799851378386552\\
83.08	0.00799861290078825\\
83.09	0.00799871269091865\\
83.1	0.00799881316141273\\
83.11	0.00799891431950746\\
83.12	0.00799901617252176\\
83.13	0.00799911872785737\\
83.14	0.00799922199299984\\
83.15	0.0079993259755195\\
83.16	0.00799943068307236\\
83.17	0.00799953612340117\\
83.18	0.00799964230433634\\
83.19	0.00799974923379699\\
83.2	0.00799985691979193\\
83.21	0.00799996537042072\\
83.22	0.00800007459387466\\
83.23	0.00800018459843787\\
83.24	0.00800029539248833\\
83.25	0.00800040698449898\\
83.26	0.00800051938303876\\
83.27	0.00800063259677375\\
83.28	0.00800074663446825\\
83.29	0.00800086150498592\\
83.3	0.00800097721729087\\
83.31	0.00800109378044888\\
83.32	0.0080012112036285\\
83.33	0.00800132949610224\\
83.34	0.00800144866724777\\
83.35	0.00800156872654909\\
83.36	0.00800168968359778\\
83.37	0.00800181154809421\\
83.38	0.00800193432984876\\
83.39	0.00800205803878312\\
83.4	0.00800218268493151\\
83.41	0.00800230827844202\\
83.42	0.00800243482957785\\
83.43	0.00800256234871868\\
83.44	0.00800269084636194\\
83.45	0.00800282033312421\\
83.46	0.00800295081974255\\
83.47	0.00800308231707586\\
83.48	0.00800321483610631\\
83.49	0.00800334838794071\\
83.5	0.00800348298381195\\
83.51	0.00800361863508044\\
83.52	0.00800375535323555\\
83.53	0.0080038931498971\\
83.54	0.00800403203681682\\
83.55	0.0080041720258799\\
83.56	0.00800431312910646\\
83.57	0.00800445535865313\\
83.58	0.00800459872681457\\
83.59	0.0080047432460251\\
83.6	0.00800488892886022\\
83.61	0.00800503578803829\\
83.62	0.00800518383642211\\
83.63	0.00800533308702057\\
83.64	0.00800548355299038\\
83.65	0.00800563524763765\\
83.66	0.00800578818441968\\
83.67	0.00800594237694665\\
83.68	0.00800609783898337\\
83.69	0.00800625458445104\\
83.7	0.00800641262742903\\
83.71	0.00800657198215668\\
83.72	0.00800673266303516\\
83.73	0.00800689468462926\\
83.74	0.00800705806166928\\
83.75	0.00800722280905295\\
83.76	0.00800738894184728\\
83.77	0.00800755647529054\\
83.78	0.00800772542479415\\
83.79	0.00800789580594476\\
83.8	0.00800806763450612\\
83.81	0.0080082409264212\\
83.82	0.00800841569781419\\
83.83	0.00800859196499256\\
83.84	0.00800876974444916\\
83.85	0.00800894905286434\\
83.86	0.00800912990710808\\
83.87	0.00800931232424216\\
83.88	0.00800949632152231\\
83.89	0.00800968191640049\\
83.9	0.00800986912652705\\
83.91	0.00801005796975305\\
83.92	0.0080102484641325\\
83.93	0.00801044062792473\\
83.94	0.00801063447959668\\
83.95	0.00801083003782528\\
83.96	0.00801102732149984\\
83.97	0.00801122634972451\\
83.98	0.00801142714182067\\
83.99	0.00801162971732942\\
84	0.00801183409601413\\
84.01	0.00801204029786291\\
84.02	0.00801224834309121\\
84.03	0.00801245825214439\\
84.04	0.00801267004570036\\
84.05	0.00801288374467222\\
84.06	0.00801309937021093\\
84.07	0.00801331694370803\\
84.08	0.00801353648679839\\
84.09	0.00801375802136296\\
84.1	0.0080139815695316\\
84.11	0.00801420715368587\\
84.12	0.00801443479646198\\
84.13	0.00801466452075359\\
84.14	0.00801489634971484\\
84.15	0.00801513030676323\\
84.16	0.00801536641558269\\
84.17	0.00801560470012657\\
84.18	0.00801584518462073\\
84.19	0.00801608789356665\\
84.2	0.00801633285174455\\
84.21	0.00801658008421656\\
84.22	0.00801682961632998\\
84.23	0.00801708147372047\\
84.24	0.00801733568231535\\
84.25	0.00801759226833696\\
84.26	0.00801785125830597\\
84.27	0.0080181126790448\\
84.28	0.00801837655768105\\
84.29	0.00801864292165101\\
84.3	0.00801891179870311\\
84.31	0.00801918321690154\\
84.32	0.00801945720462979\\
84.33	0.00801973379059434\\
84.34	0.00802001300382831\\
84.35	0.00802029487369516\\
84.36	0.00802057942989249\\
84.37	0.00802086670245584\\
84.38	0.00802115672176251\\
84.39	0.00802144951853547\\
84.4	0.0080217451238473\\
84.41	0.00802204356912418\\
84.42	0.00802234488614989\\
84.43	0.00802264910706989\\
84.44	0.00802295626439544\\
84.45	0.00802326639100779\\
84.46	0.00802357952016236\\
84.47	0.008023895685493\\
84.48	0.00802421492101633\\
84.49	0.00802453726113607\\
84.5	0.00802486274064744\\
84.51	0.00802519139474167\\
84.52	0.00802552325901044\\
84.53	0.00802585836945048\\
84.54	0.00802619676246817\\
84.55	0.00802653847488423\\
84.56	0.00802688354393838\\
84.57	0.00802723200729417\\
84.58	0.00802758390304376\\
84.59	0.00802793926971283\\
84.6	0.00802829814626551\\
84.61	0.00802866057210935\\
84.62	0.0080290265871004\\
84.63	0.00802939623154828\\
84.64	0.00802976954622138\\
84.65	0.00803014657235206\\
84.66	0.00803052735164194\\
84.67	0.00803091192626723\\
84.68	0.00803130033888416\\
84.69	0.00803169263263441\\
84.7	0.00803208885115065\\
84.71	0.00803248903856215\\
84.72	0.00803289323950039\\
84.73	0.00803330149910483\\
84.74	0.00803371386302865\\
84.75	0.0080341303774446\\
84.76	0.00803455108905094\\
84.77	0.00803497604507742\\
84.78	0.0080354052932913\\
84.79	0.00803583888200348\\
84.8	0.00803627686007473\\
84.81	0.00803671927692187\\
84.82	0.00803716618252418\\
84.83	0.00803761762742975\\
84.84	0.00803807366276199\\
84.85	0.00803853434022613\\
84.86	0.0080389997121159\\
84.87	0.00803946983132019\\
84.88	0.00803994475132985\\
84.89	0.00804042452624452\\
84.9	0.00804090921077957\\
84.91	0.00804139886027312\\
84.92	0.0080418935306931\\
84.93	0.00804239327864447\\
84.94	0.0080428981613764\\
84.95	0.00804340823678967\\
84.96	0.00804392356344406\\
84.97	0.00804444420056584\\
84.98	0.00804497020805539\\
84.99	0.00804550164649483\\
85	0.00804603857715582\\
85.01	0.00804658106200741\\
85.02	0.00804712916372394\\
85.03	0.0080476829456931\\
85.04	0.00804824247202406\\
85.05	0.00804880780755565\\
85.06	0.00804937901786472\\
85.07	0.00804995616927446\\
85.08	0.008050539328863\\
85.09	0.00805112856447192\\
85.1	0.00805172394471498\\
85.11	0.00805232553898692\\
85.12	0.00805293341747232\\
85.13	0.00805354765115463\\
85.14	0.00805416831182525\\
85.15	0.00805479547209272\\
85.16	0.00805542920539203\\
85.17	0.00805606958599405\\
85.18	0.00805671668901504\\
85.19	0.00805737059042625\\
85.2	0.00805803136706369\\
85.21	0.00805869909663797\\
85.22	0.00805937385774425\\
85.23	0.00806005572987234\\
85.24	0.00806074479341685\\
85.25	0.00806144112968752\\
85.26	0.00806214482091966\\
85.27	0.00806285595028466\\
85.28	0.00806357460190066\\
85.29	0.00806430086084336\\
85.3	0.00806503481315688\\
85.31	0.00806577654586485\\
85.32	0.00806652614698154\\
85.33	0.00806728370552312\\
85.34	0.00806804931151913\\
85.35	0.00806882305602399\\
85.36	0.00806960503112867\\
85.37	0.00807039532997255\\
85.38	0.0080711940467553\\
85.39	0.00807200127674903\\
85.4	0.00807281711631044\\
85.41	0.00807364166289326\\
85.42	0.00807447501506067\\
85.43	0.00807531727249804\\
85.44	0.00807616853602562\\
85.45	0.00807702890761156\\
85.46	0.00807789849038494\\
85.47	0.00807877738864904\\
85.48	0.0080796657078947\\
85.49	0.00808056355481388\\
85.5	0.00808147103731336\\
85.51	0.00808238826452856\\
85.52	0.00808331534683759\\
85.53	0.00808425239587537\\
85.54	0.00808519821853273\\
85.55	0.00808615012403515\\
85.56	0.00808710817486022\\
85.57	0.00808807243419314\\
85.58	0.00808904296593495\\
85.59	0.00809001983471082\\
85.6	0.00809100310587836\\
85.61	0.00809199284553618\\
85.62	0.00809298912053241\\
85.63	0.00809399054050093\\
85.64	0.00809499698539106\\
85.65	0.00809600850901311\\
85.66	0.00809702516580704\\
85.67	0.00809804701084992\\
85.68	0.00809907409986352\\
85.69	0.00810010648922189\\
85.7	0.00810114423595922\\
85.71	0.00810218739777758\\
85.72	0.00810323603305492\\
85.73	0.00810429020085304\\
85.74	0.0081053499609258\\
85.75	0.00810641537372725\\
85.76	0.00810748650042002\\
85.77	0.0081085634028837\\
85.78	0.00810964614372339\\
85.79	0.0081107347862783\\
85.8	0.0081118293946305\\
85.81	0.00811293003361373\\
85.82	0.00811403676882235\\
85.83	0.00811514966662039\\
85.84	0.0081162687941507\\
85.85	0.00811739421934419\\
85.86	0.00811852601092926\\
85.87	0.00811966423844122\\
85.88	0.00812080897223194\\
85.89	0.00812196028347955\\
85.9	0.00812311824419826\\
85.91	0.00812428292724834\\
85.92	0.00812545440634615\\
85.93	0.00812663275607438\\
85.94	0.00812781805189233\\
85.95	0.00812901037014635\\
85.96	0.00813020978808041\\
85.97	0.00813141638384679\\
85.98	0.00813263023651691\\
85.99	0.00813385142609226\\
86	0.00813508003351547\\
86.01	0.00813631614068157\\
86.02	0.0081375598304493\\
86.03	0.00813881118665258\\
86.04	0.0081400702941122\\
86.05	0.00814133723864752\\
86.06	0.00814261210708838\\
86.07	0.00814389498728721\\
86.08	0.00814518596813113\\
86.09	0.00814648513955438\\
86.1	0.00814779259255075\\
86.11	0.00814910841918623\\
86.12	0.00815043271261184\\
86.13	0.00815176556707651\\
86.14	0.00815310707794025\\
86.15	0.00815445734168737\\
86.16	0.00815581645593992\\
86.17	0.00815718451947124\\
86.18	0.00815856163221977\\
86.19	0.00815994789530289\\
86.2	0.00816134341103106\\
86.21	0.00816274828292202\\
86.22	0.00816416261571527\\
86.23	0.00816558651538661\\
86.24	0.00816702008916296\\
86.25	0.0081684634455373\\
86.26	0.00816991669428376\\
86.27	0.00817137994647302\\
86.28	0.00817285331448775\\
86.29	0.0081743369120383\\
86.3	0.00817583085417862\\
86.31	0.0081773352573223\\
86.32	0.00817885023925885\\
86.33	0.00818037591917016\\
86.34	0.00818191241764715\\
86.35	0.00818345985670668\\
86.36	0.00818501835980856\\
86.37	0.00818658805187288\\
86.38	0.00818816905929747\\
86.39	0.0081897615099756\\
86.4	0.00819136553331391\\
86.41	0.00819298126025053\\
86.42	0.00819460882327342\\
86.43	0.008196248356439\\
86.44	0.00819789999539086\\
86.45	0.00819956387737889\\
86.46	0.00820124014127848\\
86.47	0.00820292892761002\\
86.48	0.00820463037855866\\
86.49	0.00820634463799427\\
86.5	0.00820807185149166\\
86.51	0.00820981216635101\\
86.52	0.00821156573161867\\
86.53	0.00821333269810802\\
86.54	0.00821511321842079\\
86.55	0.00821690744696845\\
86.56	0.00821871553999402\\
86.57	0.00822053765559407\\
86.58	0.00822237395374096\\
86.59	0.00822422459630543\\
86.6	0.00822608974707939\\
86.61	0.00822796957179908\\
86.62	0.00822986423816839\\
86.63	0.0082317739158826\\
86.64	0.00823369877665228\\
86.65	0.00823563899422759\\
86.66	0.00823759474442285\\
86.67	0.00823956620514134\\
86.68	0.0082415535564005\\
86.69	0.0082435569803574\\
86.7	0.00824557666133451\\
86.71	0.00824761278584581\\
86.72	0.00824966554262319\\
86.73	0.00825173512264323\\
86.74	0.00825382171915424\\
86.75	0.00825592552770368\\
86.76	0.00825804674616592\\
86.77	0.00826018557477028\\
86.78	0.00826234221612952\\
86.79	0.0082645168752686\\
86.8	0.00826670975965379\\
86.81	0.0082689210792222\\
86.82	0.00827115104641163\\
86.83	0.00827339987619083\\
86.84	0.00827566778609002\\
86.85	0.00827795499623198\\
86.86	0.00828026172936335\\
86.87	0.00828258821088637\\
86.88	0.0082849346688911\\
86.89	0.00828730133418788\\
86.9	0.00828968844034034\\
86.91	0.00829209622369872\\
86.92	0.00829452492343368\\
86.93	0.00829697478157045\\
86.94	0.00829943654167261\\
86.95	0.00830190979100486\\
86.96	0.00830439465712111\\
86.97	0.00830689126912073\\
86.98	0.00830939975766765\\
86.99	0.00831192025500965\\
87	0.00831445289499796\\
87.01	0.00831699781310706\\
87.02	0.00831955514645475\\
87.03	0.00832212503382246\\
87.04	0.00832470761567584\\
87.05	0.00832730303418556\\
87.06	0.00832991143324844\\
87.07	0.00833253295850877\\
87.08	0.00833516775737999\\
87.09	0.00833781597906654\\
87.1	0.00834047777458606\\
87.11	0.00834315329679186\\
87.12	0.00834584270039563\\
87.13	0.0083485461419905\\
87.14	0.00835126378007434\\
87.15	0.00835399577507338\\
87.16	0.00835674228936613\\
87.17	0.00835950348730756\\
87.18	0.00836227953525367\\
87.19	0.0083650706015863\\
87.2	0.00836787685673826\\
87.21	0.00837069847321883\\
87.22	0.0083735356256395\\
87.23	0.00837638849074014\\
87.24	0.0083792572474154\\
87.25	0.0083821420767415\\
87.26	0.0083850431620034\\
87.27	0.00838796068872217\\
87.28	0.00839089484468292\\
87.29	0.00839384581996285\\
87.3	0.00839681380695989\\
87.31	0.0083997990004215\\
87.32	0.00840280159747399\\
87.33	0.0084058217976521\\
87.34	0.00840885980292906\\
87.35	0.00841191581774694\\
87.36	0.00841499004904745\\
87.37	0.00841808270630309\\
87.38	0.00842119400154874\\
87.39	0.00842432414941365\\
87.4	0.00842747336715378\\
87.41	0.0084306418746846\\
87.42	0.00843382989461434\\
87.43	0.00843703765227762\\
87.44	0.00844026537576949\\
87.45	0.00844351329597998\\
87.46	0.00844678164662901\\
87.47	0.00845007066430185\\
87.48	0.00845338058848489\\
87.49	0.00845671166160205\\
87.5	0.00846006412905147\\
87.51	0.00846343823924283\\
87.52	0.00846683424363507\\
87.53	0.00847025239677458\\
87.54	0.00847369295633396\\
87.55	0.00847715618315117\\
87.56	0.00848064234126932\\
87.57	0.00848415169797685\\
87.58	0.00848768452384829\\
87.59	0.00849124109278555\\
87.6	0.00849482168205971\\
87.61	0.00849842657235338\\
87.62	0.00850205604780363\\
87.63	0.0085057103960454\\
87.64	0.00850938990825552\\
87.65	0.00851309487919735\\
87.66	0.0085168256072659\\
87.67	0.00852058239453363\\
87.68	0.00852436554679673\\
87.69	0.00852817537362218\\
87.7	0.00853201218839522\\
87.71	0.00853587630836759\\
87.72	0.00853976805470633\\
87.73	0.00854368775254323\\
87.74	0.00854760914293455\\
87.75	0.00855153144821591\\
87.76	0.00855545466615325\\
87.77	0.00855937879447253\\
87.78	0.00856330383085922\\
87.79	0.00856722977295773\\
87.8	0.00857115661837094\\
87.81	0.0085750843646596\\
87.82	0.00857901300934182\\
87.83	0.0085829425498925\\
87.84	0.00858687298374275\\
87.85	0.00859080430827938\\
87.86	0.00859473652084423\\
87.87	0.00859866961873369\\
87.88	0.00860260359919803\\
87.89	0.00860653845944082\\
87.9	0.00861047419661835\\
87.91	0.00861441080783899\\
87.92	0.00861834829016258\\
87.93	0.00862228664059979\\
87.94	0.00862622585611147\\
87.95	0.00863016593360806\\
87.96	0.00863410686994885\\
87.97	0.00863804866194139\\
87.98	0.00864199130634078\\
87.99	0.008645934799849\\
88	0.00864987913911423\\
88.01	0.00865382432073014\\
88.02	0.00865777034123518\\
88.03	0.00866171719711187\\
88.04	0.0086656648847861\\
88.05	0.00866961340062635\\
88.06	0.00867356274094298\\
88.07	0.00867751290198746\\
88.08	0.00868146387995162\\
88.09	0.00868541567096688\\
88.1	0.00868936827110345\\
88.11	0.00869332167636955\\
88.12	0.0086972758827106\\
88.13	0.00870123088600844\\
88.14	0.00870518668208045\\
88.15	0.00870914326667876\\
88.16	0.00871310063548941\\
88.17	0.00871705878413146\\
88.18	0.00872101770815614\\
88.19	0.008724977403046\\
88.2	0.00872893786421397\\
88.21	0.00873289908700253\\
88.22	0.0087368610666827\\
88.23	0.00874082379845325\\
88.24	0.00874478727743962\\
88.25	0.0087487514986931\\
88.26	0.00875271645718979\\
88.27	0.00875668214782967\\
88.28	0.0087606485654356\\
88.29	0.00876461570475233\\
88.3	0.00876858356044549\\
88.31	0.00877255212710056\\
88.32	0.00877652139922187\\
88.33	0.00878049137123151\\
88.34	0.00878446203746828\\
88.35	0.00878843339218664\\
88.36	0.0087924054295556\\
88.37	0.00879637814365761\\
88.38	0.00880035152848748\\
88.39	0.00880432557795121\\
88.4	0.00880830028586485\\
88.41	0.00881227564595337\\
88.42	0.00881625165184946\\
88.43	0.00882022829709234\\
88.44	0.00882420557512658\\
88.45	0.00882818347930085\\
88.46	0.00883216200286669\\
88.47	0.00883614113897727\\
88.48	0.00884012088068612\\
88.49	0.00884410122094585\\
88.5	0.00884808215260682\\
88.51	0.00885206366841587\\
88.52	0.00885604576101494\\
88.53	0.00886002842293976\\
88.54	0.00886401164661844\\
88.55	0.00886799542437013\\
88.56	0.00887197974840359\\
88.57	0.00887596461081574\\
88.58	0.00887995000359029\\
88.59	0.00888393591859623\\
88.6	0.00888792234758636\\
88.61	0.0088919092821958\\
88.62	0.00889589671394047\\
88.63	0.00889988463421559\\
88.64	0.00890387303429406\\
88.65	0.00890786190532494\\
88.66	0.00891185123833184\\
88.67	0.0089158410242113\\
88.68	0.00891983125373114\\
88.69	0.00892382191752886\\
88.7	0.00892781300610988\\
88.71	0.00893180450984591\\
88.72	0.0089357964189732\\
88.73	0.0089397887235908\\
88.74	0.00894378141365879\\
88.75	0.00894777447899649\\
88.76	0.00895176790928068\\
88.77	0.00895576169404371\\
88.78	0.00895975582267169\\
88.79	0.0089637502844026\\
88.8	0.00896774506832435\\
88.81	0.0089717401633729\\
88.82	0.00897573555833026\\
88.83	0.00897973124182255\\
88.84	0.00898372720231795\\
88.85	0.00898772342812471\\
88.86	0.00899171990738908\\
88.87	0.0089957166280932\\
88.88	0.00899971357805303\\
88.89	0.00900371074491618\\
88.9	0.00900770811615976\\
88.91	0.0090117056790882\\
88.92	0.00901570342083099\\
88.93	0.00901970132834046\\
88.94	0.0090236993883895\\
88.95	0.00902769758756924\\
88.96	0.00903169591228673\\
88.97	0.00903569434876255\\
88.98	0.00903969288302843\\
88.99	0.0090436915009248\\
89	0.00904769018809837\\
89.01	0.00905168892999959\\
89.02	0.00905568771188015\\
89.03	0.00905968651879043\\
89.04	0.00906368533557691\\
89.05	0.00906768414687951\\
89.06	0.009071682937129\\
89.07	0.00907568169054425\\
89.08	0.00907968039112954\\
89.09	0.00908367902267179\\
89.1	0.00908767756873776\\
89.11	0.00909167601267121\\
89.12	0.00909567433759007\\
89.13	0.00909967252638349\\
89.14	0.00910367056170893\\
89.15	0.00910766842598914\\
89.16	0.00911166610140922\\
89.17	0.00911566356991346\\
89.18	0.00911966081320233\\
89.19	0.00912365781272928\\
89.2	0.00912765454969763\\
89.21	0.00913165100505729\\
89.22	0.00913564715950153\\
89.23	0.00913964299346366\\
89.24	0.00914363848711374\\
89.25	0.00914763362035513\\
89.26	0.00915162837282107\\
89.27	0.00915562272387126\\
89.28	0.00915961665258826\\
89.29	0.00916361013777401\\
89.3	0.00916760315794615\\
89.31	0.00917159569133438\\
89.32	0.00917558771587681\\
89.33	0.00917957920921613\\
89.34	0.00918357014869585\\
89.35	0.00918756051135645\\
89.36	0.0091915502739315\\
89.37	0.00919553941284365\\
89.38	0.00919952790420072\\
89.39	0.00920351572379155\\
89.4	0.00920750284708198\\
89.41	0.00921148924921066\\
89.42	0.00921547490498482\\
89.43	0.00921945978887605\\
89.44	0.00922344387501595\\
89.45	0.00922742713719176\\
89.46	0.00923140954884196\\
89.47	0.00923539108305171\\
89.48	0.0092393717125484\\
89.49	0.00924335140969697\\
89.5	0.00924733014649527\\
89.51	0.00925130789456936\\
89.52	0.00925528462516868\\
89.53	0.00925926030916124\\
89.54	0.00926323491702872\\
89.55	0.00926720841886143\\
89.56	0.00927118078435338\\
89.57	0.00927515198279708\\
89.58	0.00927912198307842\\
89.59	0.00928309075367144\\
89.6	0.00928705826263302\\
89.61	0.00929102447759751\\
89.62	0.00929498936577128\\
89.63	0.00929895289392723\\
89.64	0.00930291502839922\\
89.65	0.00930687573507637\\
89.66	0.00931083497939741\\
89.67	0.00931479272634479\\
89.68	0.0093187489404389\\
89.69	0.00932270358573205\\
89.7	0.00932665662580248\\
89.71	0.00933060802374826\\
89.72	0.00933455774218107\\
89.73	0.00933850574321999\\
89.74	0.00934245198848512\\
89.75	0.00934639643909117\\
89.76	0.00935033905564096\\
89.77	0.0093542797982188\\
89.78	0.00935821862638385\\
89.79	0.00936215549916333\\
89.8	0.00936609037504568\\
89.81	0.00937002321197365\\
89.82	0.0093739539673372\\
89.83	0.00937788259796648\\
89.84	0.00938180906012452\\
89.85	0.0093857333095\\
89.86	0.00938965530119983\\
89.87	0.00939357498974165\\
89.88	0.00939749232904624\\
89.89	0.00940140727242984\\
89.9	0.00940531977259638\\
89.91	0.00940922978162955\\
89.92	0.00941313725098485\\
89.93	0.00941704213148148\\
89.94	0.00942094437329416\\
89.95	0.00942484392594481\\
89.96	0.00942874073829415\\
89.97	0.00943263475853317\\
89.98	0.00943652593417453\\
89.99	0.00944041421204381\\
90	0.00944429953827062\\
90.01	0.00944818185827971\\
90.02	0.00945206111678181\\
90.03	0.0094559372577645\\
90.04	0.00945981022448285\\
90.05	0.00946367995944998\\
90.06	0.00946754640442754\\
90.07	0.00947140950041599\\
90.08	0.0094752691876448\\
90.09	0.00947912540556253\\
90.1	0.00948297809282677\\
90.11	0.00948682718729392\\
90.12	0.0094906726260089\\
90.13	0.00949451434519467\\
90.14	0.00949835228024166\\
90.15	0.00950218636569701\\
90.16	0.00950601653525375\\
90.17	0.00950984272173976\\
90.18	0.00951366485710662\\
90.19	0.00951748287241834\\
90.2	0.0095212966978399\\
90.21	0.0095251062626257\\
90.22	0.00952891149510776\\
90.23	0.00953271232268393\\
90.24	0.00953650867180575\\
90.25	0.00954030046796634\\
90.26	0.00954408763568798\\
90.27	0.00954787009850965\\
90.28	0.00955164777897434\\
90.29	0.00955542059861622\\
90.3	0.00955918847794761\\
90.31	0.00956295133644587\\
90.32	0.00956670909254001\\
90.33	0.00957046166359719\\
90.34	0.00957420896590908\\
90.35	0.00957795091467794\\
90.36	0.0095816874240026\\
90.37	0.00958541840686427\\
90.38	0.00958914377511212\\
90.39	0.00959286343944866\\
90.4	0.00959657730941502\\
90.41	0.00960028529337595\\
90.42	0.00960398729850466\\
90.43	0.00960768323076749\\
90.44	0.00961137299490832\\
90.45	0.00961505649443286\\
90.46	0.00961873363159268\\
90.47	0.00962240430736903\\
90.48	0.00962606842145652\\
90.49	0.00962972587224646\\
90.5	0.00963337655681019\\
90.51	0.00963702037088194\\
90.52	0.00964065720884171\\
90.53	0.00964428696369775\\
90.54	0.00964790952706894\\
90.55	0.00965152478916686\\
90.56	0.00965513263877766\\
90.57	0.00965873296324372\\
90.58	0.00966232564844503\\
90.59	0.00966591057878037\\
90.6	0.00966948763714822\\
90.61	0.00967305670492742\\
90.62	0.00967661766195765\\
90.63	0.00968017038651954\\
90.64	0.00968371475531464\\
90.65	0.00968725064344505\\
90.66	0.00969077792439288\\
90.67	0.0096942964699993\\
90.68	0.00969780615044349\\
90.69	0.0097013068342212\\
90.7	0.00970479838812309\\
90.71	0.00970828067721276\\
90.72	0.00971175356480456\\
90.73	0.009715216912441\\
90.74	0.00971867057987001\\
90.75	0.0097221144250218\\
90.76	0.00972554830398549\\
90.77	0.00972897207098539\\
90.78	0.00973238557835702\\
90.79	0.00973578867652279\\
90.8	0.00973918121396741\\
90.81	0.00974256303721292\\
90.82	0.00974593399079345\\
90.83	0.00974929391722969\\
90.84	0.00975264265700292\\
90.85	0.00975598004852882\\
90.86	0.0097593059281309\\
90.87	0.00976262013001358\\
90.88	0.00976592248623497\\
90.89	0.00976921282667921\\
90.9	0.00977249097902859\\
90.91	0.00977575676873521\\
90.92	0.00977901001899229\\
90.93	0.00978225055070519\\
90.94	0.00978547818246195\\
90.95	0.00978869273050354\\
90.96	0.0097918940086937\\
90.97	0.00979508182848841\\
90.98	0.00979825599890491\\
90.99	0.00980141632649044\\
91	0.00980456261529044\\
91.01	0.00980769466681652\\
91.02	0.00981081228001382\\
91.03	0.00981391525122813\\
91.04	0.00981700337417249\\
91.05	0.00982007643989341\\
91.06	0.00982313423673664\\
91.07	0.00982617655031253\\
91.08	0.00982920316346091\\
91.09	0.00983221385621555\\
91.1	0.00983520840576818\\
91.11	0.00983818658643206\\
91.12	0.00984114816960504\\
91.13	0.00984409292373216\\
91.14	0.00984702061426787\\
91.15	0.00984993100363763\\
91.16	0.00985282385119915\\
91.17	0.00985569891320305\\
91.18	0.00985855594275306\\
91.19	0.00986139468976573\\
91.2	0.00986421490092961\\
91.21	0.00986701631966392\\
91.22	0.0098697986860767\\
91.23	0.00987256173692245\\
91.24	0.00987530520555916\\
91.25	0.00987802882190494\\
91.26	0.00988073231239397\\
91.27	0.00988341539993196\\
91.28	0.00988607780385104\\
91.29	0.00988871923986405\\
91.3	0.00989133942001836\\
91.31	0.00989393805264896\\
91.32	0.00989651484233108\\
91.33	0.00989906948983214\\
91.34	0.0099016016920632\\
91.35	0.00990411114202966\\
91.36	0.00990659752878144\\
91.37	0.00990906053736254\\
91.38	0.00991149984875993\\
91.39	0.00991391513985178\\
91.4	0.00991630608335514\\
91.41	0.00991867234777286\\
91.42	0.0099210135973399\\
91.43	0.009923329491969\\
91.44	0.0099256196871956\\
91.45	0.00992788383412212\\
91.46	0.00993012157936156\\
91.47	0.00993233256498038\\
91.48	0.00993451642844063\\
91.49	0.00993667280254148\\
91.5	0.00993880131535986\\
91.51	0.00994090159019053\\
91.52	0.0099429732454853\\
91.53	0.00994501589479151\\
91.54	0.00994702914668981\\
91.55	0.00994901260473111\\
91.56	0.00995096586737279\\
91.57	0.0099528885279141\\
91.58	0.00995478017443075\\
91.59	0.00995664038970878\\
91.6	0.00995846875117749\\
91.61	0.00996026483084165\\
91.62	0.00996202819521284\\
91.63	0.00996375840523993\\
91.64	0.00996545501623875\\
91.65	0.00996711757782088\\
91.66	0.00996874563382153\\
91.67	0.00997033872222662\\
91.68	0.00997189637509891\\
91.69	0.00997341811850326\\
91.7	0.00997490347243093\\
91.71	0.00997635195072307\\
91.72	0.00997776306099314\\
91.73	0.00997913630454848\\
91.74	0.00998047117631092\\
91.75	0.00998176716473642\\
91.76	0.00998302375173371\\
91.77	0.00998424041258199\\
91.78	0.00998541661584763\\
91.79	0.00998655182329983\\
91.8	0.00998764548982532\\
91.81	0.009988697063342\\
91.82	0.00998970598471153\\
91.83	0.00999067168765091\\
91.84	0.00999159359864302\\
91.85	0.00999247113684597\\
91.86	0.00999330371400154\\
91.87	0.00999409073434241\\
91.88	0.0099948315944983\\
91.89	0.00999552568340103\\
91.9	0.00999617238218844\\
91.91	0.00999677106410716\\
91.92	0.00999732109441422\\
91.93	0.00999782183027756\\
91.94	0.00999827262067526\\
91.95	0.00999867280629367\\
91.96	0.00999902171942435\\
91.97	0.00999931868385968\\
91.98	0.0099995630147874\\
91.99	0.00999975401868384\\
92	0.00999989099320585\\
92.01	0.00999997322708161\\
92.02	0.01\\
92.03	0.01\\
92.04	0.01\\
92.05	0.01\\
92.06	0.01\\
92.07	0.01\\
92.08	0.01\\
92.09	0.01\\
92.1	0.01\\
92.11	0.01\\
92.12	0.01\\
92.13	0.01\\
92.14	0.01\\
92.15	0.01\\
92.16	0.01\\
92.17	0.01\\
92.18	0.01\\
92.19	0.01\\
92.2	0.01\\
92.21	0.01\\
92.22	0.01\\
92.23	0.01\\
92.24	0.01\\
92.25	0.01\\
92.26	0.01\\
92.27	0.01\\
92.28	0.01\\
92.29	0.01\\
92.3	0.01\\
92.31	0.01\\
92.32	0.01\\
92.33	0.01\\
92.34	0.01\\
92.35	0.01\\
92.36	0.01\\
92.37	0.01\\
92.38	0.01\\
92.39	0.01\\
92.4	0.01\\
92.41	0.01\\
92.42	0.01\\
92.43	0.01\\
92.44	0.01\\
92.45	0.01\\
92.46	0.01\\
92.47	0.01\\
92.48	0.01\\
92.49	0.01\\
92.5	0.01\\
92.51	0.01\\
92.52	0.01\\
92.53	0.01\\
92.54	0.01\\
92.55	0.01\\
92.56	0.01\\
92.57	0.01\\
92.58	0.01\\
92.59	0.01\\
92.6	0.01\\
92.61	0.01\\
92.62	0.01\\
92.63	0.01\\
92.64	0.01\\
92.65	0.01\\
92.66	0.01\\
92.67	0.01\\
92.68	0.01\\
92.69	0.01\\
92.7	0.01\\
92.71	0.01\\
92.72	0.01\\
92.73	0.01\\
92.74	0.01\\
92.75	0.01\\
92.76	0.01\\
92.77	0.01\\
92.78	0.01\\
92.79	0.01\\
92.8	0.01\\
92.81	0.01\\
92.82	0.01\\
92.83	0.01\\
92.84	0.01\\
92.85	0.01\\
92.86	0.01\\
92.87	0.01\\
92.88	0.01\\
92.89	0.01\\
92.9	0.01\\
92.91	0.01\\
92.92	0.01\\
92.93	0.01\\
92.94	0.01\\
92.95	0.01\\
92.96	0.01\\
92.97	0.01\\
92.98	0.01\\
92.99	0.01\\
93	0.01\\
93.01	0.01\\
93.02	0.01\\
93.03	0.01\\
93.04	0.01\\
93.05	0.01\\
93.06	0.01\\
93.07	0.01\\
93.08	0.01\\
93.09	0.01\\
93.1	0.01\\
93.11	0.01\\
93.12	0.01\\
93.13	0.01\\
93.14	0.01\\
93.15	0.01\\
93.16	0.01\\
93.17	0.01\\
93.18	0.01\\
93.19	0.01\\
93.2	0.01\\
93.21	0.01\\
93.22	0.01\\
93.23	0.01\\
93.24	0.01\\
93.25	0.01\\
93.26	0.01\\
93.27	0.01\\
93.28	0.01\\
93.29	0.01\\
93.3	0.01\\
93.31	0.01\\
93.32	0.01\\
93.33	0.01\\
93.34	0.01\\
93.35	0.01\\
93.36	0.01\\
93.37	0.01\\
93.38	0.01\\
93.39	0.01\\
93.4	0.01\\
93.41	0.01\\
93.42	0.01\\
93.43	0.01\\
93.44	0.01\\
93.45	0.01\\
93.46	0.01\\
93.47	0.01\\
93.48	0.01\\
93.49	0.01\\
93.5	0.01\\
93.51	0.01\\
93.52	0.01\\
93.53	0.01\\
93.54	0.01\\
93.55	0.01\\
93.56	0.01\\
93.57	0.01\\
93.58	0.01\\
93.59	0.01\\
93.6	0.01\\
93.61	0.01\\
93.62	0.01\\
93.63	0.01\\
93.64	0.01\\
93.65	0.01\\
93.66	0.01\\
93.67	0.01\\
93.68	0.01\\
93.69	0.01\\
93.7	0.01\\
93.71	0.01\\
93.72	0.01\\
93.73	0.01\\
93.74	0.01\\
93.75	0.01\\
93.76	0.01\\
93.77	0.01\\
93.78	0.01\\
93.79	0.01\\
93.8	0.01\\
93.81	0.01\\
93.82	0.01\\
93.83	0.01\\
93.84	0.01\\
93.85	0.01\\
93.86	0.01\\
93.87	0.01\\
93.88	0.01\\
93.89	0.01\\
93.9	0.01\\
93.91	0.01\\
93.92	0.01\\
93.93	0.01\\
93.94	0.01\\
93.95	0.01\\
93.96	0.01\\
93.97	0.01\\
93.98	0.01\\
93.99	0.01\\
94	0.01\\
94.01	0.01\\
94.02	0.01\\
94.03	0.01\\
94.04	0.01\\
94.05	0.01\\
94.06	0.01\\
94.07	0.01\\
94.08	0.01\\
94.09	0.01\\
94.1	0.01\\
94.11	0.01\\
94.12	0.01\\
94.13	0.01\\
94.14	0.01\\
94.15	0.01\\
94.16	0.01\\
94.17	0.01\\
94.18	0.01\\
94.19	0.01\\
94.2	0.01\\
94.21	0.01\\
94.22	0.01\\
94.23	0.01\\
94.24	0.01\\
94.25	0.01\\
94.26	0.01\\
94.27	0.01\\
94.28	0.01\\
94.29	0.01\\
94.3	0.01\\
94.31	0.01\\
94.32	0.01\\
94.33	0.01\\
94.34	0.01\\
94.35	0.01\\
94.36	0.01\\
94.37	0.01\\
94.38	0.01\\
94.39	0.01\\
94.4	0.01\\
94.41	0.01\\
94.42	0.01\\
94.43	0.01\\
94.44	0.01\\
94.45	0.01\\
94.46	0.01\\
94.47	0.01\\
94.48	0.01\\
94.49	0.01\\
94.5	0.01\\
94.51	0.01\\
94.52	0.01\\
94.53	0.01\\
94.54	0.01\\
94.55	0.01\\
94.56	0.01\\
94.57	0.01\\
94.58	0.01\\
94.59	0.01\\
94.6	0.01\\
94.61	0.01\\
94.62	0.01\\
94.63	0.01\\
94.64	0.01\\
94.65	0.01\\
94.66	0.01\\
94.67	0.01\\
94.68	0.01\\
94.69	0.01\\
94.7	0.01\\
94.71	0.01\\
94.72	0.01\\
94.73	0.01\\
94.74	0.01\\
94.75	0.01\\
94.76	0.01\\
94.77	0.01\\
94.78	0.01\\
94.79	0.01\\
94.8	0.01\\
94.81	0.01\\
94.82	0.01\\
94.83	0.01\\
94.84	0.01\\
94.85	0.01\\
94.86	0.01\\
94.87	0.01\\
94.88	0.01\\
94.89	0.01\\
94.9	0.01\\
94.91	0.01\\
94.92	0.01\\
94.93	0.01\\
94.94	0.01\\
94.95	0.01\\
94.96	0.01\\
94.97	0.01\\
94.98	0.01\\
94.99	0.01\\
95	0.01\\
95.01	0.01\\
95.02	0.01\\
95.03	0.01\\
95.04	0.01\\
95.05	0.01\\
95.06	0.01\\
95.07	0.01\\
95.08	0.01\\
95.09	0.01\\
95.1	0.01\\
95.11	0.01\\
95.12	0.01\\
95.13	0.01\\
95.14	0.01\\
95.15	0.01\\
95.16	0.01\\
95.17	0.01\\
95.18	0.01\\
95.19	0.01\\
95.2	0.01\\
95.21	0.01\\
95.22	0.01\\
95.23	0.01\\
95.24	0.01\\
95.25	0.01\\
95.26	0.01\\
95.27	0.01\\
95.28	0.01\\
95.29	0.01\\
95.3	0.01\\
95.31	0.01\\
95.32	0.01\\
95.33	0.01\\
95.34	0.01\\
95.35	0.01\\
95.36	0.01\\
95.37	0.01\\
95.38	0.01\\
95.39	0.01\\
95.4	0.01\\
95.41	0.01\\
95.42	0.01\\
95.43	0.01\\
95.44	0.01\\
95.45	0.01\\
95.46	0.01\\
95.47	0.01\\
95.48	0.01\\
95.49	0.01\\
95.5	0.01\\
95.51	0.01\\
95.52	0.01\\
95.53	0.01\\
95.54	0.01\\
95.55	0.01\\
95.56	0.01\\
95.57	0.01\\
95.58	0.01\\
95.59	0.01\\
95.6	0.01\\
95.61	0.01\\
95.62	0.01\\
95.63	0.01\\
95.64	0.01\\
95.65	0.01\\
95.66	0.01\\
95.67	0.01\\
95.68	0.01\\
95.69	0.01\\
95.7	0.01\\
95.71	0.01\\
95.72	0.01\\
95.73	0.01\\
95.74	0.01\\
95.75	0.01\\
95.76	0.01\\
95.77	0.01\\
95.78	0.01\\
95.79	0.01\\
95.8	0.01\\
95.81	0.01\\
95.82	0.01\\
95.83	0.01\\
95.84	0.01\\
95.85	0.01\\
95.86	0.01\\
95.87	0.01\\
95.88	0.01\\
95.89	0.01\\
95.9	0.01\\
95.91	0.01\\
95.92	0.01\\
95.93	0.01\\
95.94	0.01\\
95.95	0.01\\
95.96	0.01\\
95.97	0.01\\
95.98	0.01\\
95.99	0.01\\
96	0.01\\
96.01	0.01\\
96.02	0.01\\
96.03	0.01\\
96.04	0.01\\
96.05	0.01\\
96.06	0.01\\
96.07	0.01\\
96.08	0.01\\
96.09	0.01\\
96.1	0.01\\
96.11	0.01\\
96.12	0.01\\
96.13	0.01\\
96.14	0.01\\
96.15	0.01\\
96.16	0.01\\
96.17	0.01\\
96.18	0.01\\
96.19	0.01\\
96.2	0.01\\
96.21	0.01\\
96.22	0.01\\
96.23	0.01\\
96.24	0.01\\
96.25	0.01\\
96.26	0.01\\
96.27	0.01\\
96.28	0.01\\
96.29	0.01\\
96.3	0.01\\
96.31	0.01\\
96.32	0.01\\
96.33	0.01\\
96.34	0.01\\
96.35	0.01\\
96.36	0.01\\
96.37	0.01\\
96.38	0.01\\
96.39	0.01\\
96.4	0.01\\
96.41	0.01\\
96.42	0.01\\
96.43	0.01\\
96.44	0.01\\
96.45	0.01\\
96.46	0.01\\
96.47	0.01\\
96.48	0.01\\
96.49	0.01\\
96.5	0.01\\
96.51	0.01\\
96.52	0.01\\
96.53	0.01\\
96.54	0.01\\
96.55	0.01\\
96.56	0.01\\
96.57	0.01\\
96.58	0.01\\
96.59	0.01\\
96.6	0.01\\
96.61	0.01\\
96.62	0.01\\
96.63	0.01\\
96.64	0.01\\
96.65	0.01\\
96.66	0.01\\
96.67	0.01\\
96.68	0.01\\
96.69	0.01\\
96.7	0.01\\
96.71	0.01\\
96.72	0.01\\
96.73	0.01\\
96.74	0.01\\
96.75	0.01\\
96.76	0.01\\
96.77	0.01\\
96.78	0.01\\
96.79	0.01\\
96.8	0.01\\
96.81	0.01\\
96.82	0.01\\
96.83	0.01\\
96.84	0.01\\
96.85	0.01\\
96.86	0.01\\
96.87	0.01\\
96.88	0.01\\
96.89	0.01\\
96.9	0.01\\
96.91	0.01\\
96.92	0.01\\
96.93	0.01\\
96.94	0.01\\
96.95	0.01\\
96.96	0.01\\
96.97	0.01\\
96.98	0.01\\
96.99	0.01\\
97	0.01\\
97.01	0.01\\
97.02	0.01\\
97.03	0.01\\
97.04	0.01\\
97.05	0.01\\
97.06	0.01\\
97.07	0.01\\
97.08	0.01\\
97.09	0.01\\
97.1	0.01\\
97.11	0.01\\
97.12	0.01\\
97.13	0.01\\
97.14	0.01\\
97.15	0.01\\
97.16	0.01\\
97.17	0.01\\
97.18	0.01\\
97.19	0.01\\
97.2	0.01\\
97.21	0.01\\
97.22	0.01\\
97.23	0.01\\
97.24	0.01\\
97.25	0.01\\
97.26	0.01\\
97.27	0.01\\
97.28	0.01\\
97.29	0.01\\
97.3	0.01\\
97.31	0.01\\
97.32	0.01\\
97.33	0.01\\
97.34	0.01\\
97.35	0.01\\
97.36	0.01\\
97.37	0.01\\
97.38	0.01\\
97.39	0.01\\
97.4	0.01\\
97.41	0.01\\
97.42	0.01\\
97.43	0.01\\
97.44	0.01\\
97.45	0.01\\
97.46	0.01\\
97.47	0.01\\
97.48	0.01\\
97.49	0.01\\
97.5	0.01\\
97.51	0.01\\
97.52	0.01\\
97.53	0.01\\
97.54	0.01\\
97.55	0.01\\
97.56	0.01\\
97.57	0.01\\
97.58	0.01\\
97.59	0.01\\
97.6	0.01\\
97.61	0.01\\
97.62	0.01\\
97.63	0.01\\
97.64	0.01\\
97.65	0.01\\
97.66	0.01\\
97.67	0.01\\
97.68	0.01\\
97.69	0.01\\
97.7	0.01\\
97.71	0.01\\
97.72	0.01\\
97.73	0.01\\
97.74	0.01\\
97.75	0.01\\
97.76	0.01\\
97.77	0.01\\
97.78	0.01\\
97.79	0.01\\
97.8	0.01\\
97.81	0.01\\
97.82	0.01\\
97.83	0.01\\
97.84	0.01\\
97.85	0.01\\
97.86	0.01\\
97.87	0.01\\
97.88	0.01\\
97.89	0.01\\
97.9	0.01\\
97.91	0.01\\
97.92	0.01\\
97.93	0.01\\
97.94	0.01\\
97.95	0.01\\
97.96	0.01\\
97.97	0.01\\
97.98	0.01\\
97.99	0.01\\
98	0.01\\
98.01	0.01\\
98.02	0.01\\
98.03	0.01\\
98.04	0.01\\
98.05	0.01\\
98.06	0.01\\
98.07	0.01\\
98.08	0.01\\
98.09	0.01\\
98.1	0.01\\
98.11	0.01\\
98.12	0.01\\
98.13	0.01\\
98.14	0.01\\
98.15	0.01\\
98.16	0.01\\
98.17	0.01\\
98.18	0.01\\
98.19	0.01\\
98.2	0.01\\
98.21	0.01\\
98.22	0.01\\
98.23	0.01\\
98.24	0.01\\
98.25	0.01\\
98.26	0.01\\
98.27	0.01\\
98.28	0.01\\
98.29	0.01\\
98.3	0.01\\
98.31	0.01\\
98.32	0.01\\
98.33	0.01\\
98.34	0.01\\
98.35	0.01\\
98.36	0.01\\
98.37	0.01\\
98.38	0.01\\
98.39	0.01\\
98.4	0.01\\
98.41	0.01\\
98.42	0.01\\
98.43	0.01\\
98.44	0.01\\
98.45	0.01\\
98.46	0.01\\
98.47	0.01\\
98.48	0.01\\
98.49	0.01\\
98.5	0.01\\
98.51	0.01\\
98.52	0.01\\
98.53	0.01\\
98.54	0.01\\
98.55	0.01\\
98.56	0.01\\
98.57	0.01\\
98.58	0.01\\
98.59	0.01\\
98.6	0.01\\
98.61	0.01\\
98.62	0.01\\
98.63	0.01\\
98.64	0.01\\
98.65	0.01\\
98.66	0.01\\
98.67	0.01\\
98.68	0.01\\
98.69	0.01\\
98.7	0.01\\
98.71	0.01\\
98.72	0.01\\
98.73	0.01\\
98.74	0.01\\
98.75	0.01\\
98.76	0.01\\
98.77	0.01\\
98.78	0.01\\
98.79	0.01\\
98.8	0.01\\
98.81	0.01\\
98.82	0.01\\
98.83	0.01\\
98.84	0.01\\
98.85	0.01\\
98.86	0.01\\
98.87	0.01\\
98.88	0.01\\
98.89	0.01\\
98.9	0.01\\
98.91	0.01\\
98.92	0.01\\
98.93	0.01\\
98.94	0.01\\
98.95	0.01\\
98.96	0.01\\
98.97	0.01\\
98.98	0.01\\
98.99	0.01\\
99	0.01\\
99.01	0.01\\
99.02	0.01\\
99.03	0.01\\
99.04	0.01\\
99.05	0.01\\
99.06	0.01\\
99.07	0.01\\
99.08	0.01\\
99.09	0.01\\
99.1	0.01\\
99.11	0.01\\
99.12	0.01\\
99.13	0.01\\
99.14	0.01\\
99.15	0.01\\
99.16	0.01\\
99.17	0.01\\
99.18	0.01\\
99.19	0.01\\
99.2	0.01\\
99.21	0.01\\
99.22	0.01\\
99.23	0.01\\
99.24	0.01\\
99.25	0.01\\
99.26	0.01\\
99.27	0.01\\
99.28	0.01\\
99.29	0.01\\
99.3	0.01\\
99.31	0.01\\
99.32	0.01\\
99.33	0.01\\
99.34	0.01\\
99.35	0.01\\
99.36	0.01\\
99.37	0.01\\
99.38	0.01\\
99.39	0.01\\
99.4	0.01\\
99.41	0.01\\
99.42	0.01\\
99.43	0.01\\
99.44	0.01\\
99.45	0.01\\
99.46	0.01\\
99.47	0.01\\
99.48	0.01\\
99.49	0.01\\
99.5	0.01\\
99.51	0.01\\
99.52	0.01\\
99.53	0.01\\
99.54	0.01\\
99.55	0.01\\
99.56	0.01\\
99.57	0.01\\
99.58	0.01\\
99.59	0.01\\
99.6	0.01\\
99.61	0.01\\
99.62	0.01\\
99.63	0.01\\
99.64	0.01\\
99.65	0.01\\
99.66	0.01\\
99.67	0.01\\
99.68	0.01\\
99.69	0.01\\
99.7	0.01\\
99.71	0.01\\
99.72	0.01\\
99.73	0.01\\
99.74	0.01\\
99.75	0.01\\
99.76	0.01\\
99.77	0.01\\
99.78	0.01\\
99.79	0.01\\
99.8	0.01\\
99.81	0.01\\
99.82	0.01\\
99.83	0.01\\
99.84	0.01\\
99.85	0.01\\
99.86	0.01\\
99.87	0.01\\
99.88	0.01\\
99.89	0.01\\
99.9	0.01\\
99.91	0.01\\
99.92	0.01\\
99.93	0.01\\
99.94	0.01\\
99.95	0.01\\
99.96	0.01\\
99.97	0.01\\
99.98	0.01\\
99.99	0.01\\
100	0.01\\
};
\addlegendentry{$q=0$};

\addplot [color=blue,solid,forget plot]
  table[row sep=crcr]{%
0.01	0.01\\
0.02	0.01\\
0.03	0.01\\
0.04	0.01\\
0.05	0.01\\
0.06	0.01\\
0.07	0.01\\
0.08	0.01\\
0.09	0.01\\
0.1	0.01\\
0.11	0.01\\
0.12	0.01\\
0.13	0.01\\
0.14	0.01\\
0.15	0.01\\
0.16	0.01\\
0.17	0.01\\
0.18	0.01\\
0.19	0.01\\
0.2	0.01\\
0.21	0.01\\
0.22	0.01\\
0.23	0.01\\
0.24	0.01\\
0.25	0.01\\
0.26	0.01\\
0.27	0.01\\
0.28	0.01\\
0.29	0.01\\
0.3	0.01\\
0.31	0.01\\
0.32	0.01\\
0.33	0.01\\
0.34	0.01\\
0.35	0.01\\
0.36	0.01\\
0.37	0.01\\
0.38	0.01\\
0.39	0.01\\
0.4	0.01\\
0.41	0.01\\
0.42	0.01\\
0.43	0.01\\
0.44	0.01\\
0.45	0.01\\
0.46	0.01\\
0.47	0.01\\
0.48	0.01\\
0.49	0.01\\
0.5	0.01\\
0.51	0.01\\
0.52	0.01\\
0.53	0.01\\
0.54	0.01\\
0.55	0.01\\
0.56	0.01\\
0.57	0.01\\
0.58	0.01\\
0.59	0.01\\
0.6	0.01\\
0.61	0.01\\
0.62	0.01\\
0.63	0.01\\
0.64	0.01\\
0.65	0.01\\
0.66	0.01\\
0.67	0.01\\
0.68	0.01\\
0.69	0.01\\
0.7	0.01\\
0.71	0.01\\
0.72	0.01\\
0.73	0.01\\
0.74	0.01\\
0.75	0.01\\
0.76	0.01\\
0.77	0.01\\
0.78	0.01\\
0.79	0.01\\
0.8	0.01\\
0.81	0.01\\
0.82	0.01\\
0.83	0.01\\
0.84	0.01\\
0.85	0.01\\
0.86	0.01\\
0.87	0.01\\
0.88	0.01\\
0.89	0.01\\
0.9	0.01\\
0.91	0.01\\
0.92	0.01\\
0.93	0.01\\
0.94	0.01\\
0.95	0.01\\
0.96	0.01\\
0.97	0.01\\
0.98	0.01\\
0.99	0.01\\
1	0.01\\
1.01	0.01\\
1.02	0.01\\
1.03	0.01\\
1.04	0.01\\
1.05	0.01\\
1.06	0.01\\
1.07	0.01\\
1.08	0.01\\
1.09	0.01\\
1.1	0.01\\
1.11	0.01\\
1.12	0.01\\
1.13	0.01\\
1.14	0.01\\
1.15	0.01\\
1.16	0.01\\
1.17	0.01\\
1.18	0.01\\
1.19	0.01\\
1.2	0.01\\
1.21	0.01\\
1.22	0.01\\
1.23	0.01\\
1.24	0.01\\
1.25	0.01\\
1.26	0.01\\
1.27	0.01\\
1.28	0.01\\
1.29	0.01\\
1.3	0.01\\
1.31	0.01\\
1.32	0.01\\
1.33	0.01\\
1.34	0.01\\
1.35	0.01\\
1.36	0.01\\
1.37	0.01\\
1.38	0.01\\
1.39	0.01\\
1.4	0.01\\
1.41	0.01\\
1.42	0.01\\
1.43	0.01\\
1.44	0.01\\
1.45	0.01\\
1.46	0.01\\
1.47	0.01\\
1.48	0.01\\
1.49	0.01\\
1.5	0.01\\
1.51	0.01\\
1.52	0.01\\
1.53	0.01\\
1.54	0.01\\
1.55	0.01\\
1.56	0.01\\
1.57	0.01\\
1.58	0.01\\
1.59	0.01\\
1.6	0.01\\
1.61	0.01\\
1.62	0.01\\
1.63	0.01\\
1.64	0.01\\
1.65	0.01\\
1.66	0.01\\
1.67	0.01\\
1.68	0.01\\
1.69	0.01\\
1.7	0.01\\
1.71	0.01\\
1.72	0.01\\
1.73	0.01\\
1.74	0.01\\
1.75	0.01\\
1.76	0.01\\
1.77	0.01\\
1.78	0.01\\
1.79	0.01\\
1.8	0.01\\
1.81	0.01\\
1.82	0.01\\
1.83	0.01\\
1.84	0.01\\
1.85	0.01\\
1.86	0.01\\
1.87	0.01\\
1.88	0.01\\
1.89	0.01\\
1.9	0.01\\
1.91	0.01\\
1.92	0.01\\
1.93	0.01\\
1.94	0.01\\
1.95	0.01\\
1.96	0.01\\
1.97	0.01\\
1.98	0.01\\
1.99	0.01\\
2	0.01\\
2.01	0.01\\
2.02	0.01\\
2.03	0.01\\
2.04	0.01\\
2.05	0.01\\
2.06	0.01\\
2.07	0.01\\
2.08	0.01\\
2.09	0.01\\
2.1	0.01\\
2.11	0.01\\
2.12	0.01\\
2.13	0.01\\
2.14	0.01\\
2.15	0.01\\
2.16	0.01\\
2.17	0.01\\
2.18	0.01\\
2.19	0.01\\
2.2	0.01\\
2.21	0.01\\
2.22	0.01\\
2.23	0.01\\
2.24	0.01\\
2.25	0.01\\
2.26	0.01\\
2.27	0.01\\
2.28	0.01\\
2.29	0.01\\
2.3	0.01\\
2.31	0.01\\
2.32	0.01\\
2.33	0.01\\
2.34	0.01\\
2.35	0.01\\
2.36	0.01\\
2.37	0.01\\
2.38	0.01\\
2.39	0.01\\
2.4	0.01\\
2.41	0.01\\
2.42	0.01\\
2.43	0.01\\
2.44	0.01\\
2.45	0.01\\
2.46	0.01\\
2.47	0.01\\
2.48	0.01\\
2.49	0.01\\
2.5	0.01\\
2.51	0.01\\
2.52	0.01\\
2.53	0.01\\
2.54	0.01\\
2.55	0.01\\
2.56	0.01\\
2.57	0.01\\
2.58	0.01\\
2.59	0.01\\
2.6	0.01\\
2.61	0.01\\
2.62	0.01\\
2.63	0.01\\
2.64	0.01\\
2.65	0.01\\
2.66	0.01\\
2.67	0.01\\
2.68	0.01\\
2.69	0.01\\
2.7	0.01\\
2.71	0.01\\
2.72	0.01\\
2.73	0.01\\
2.74	0.01\\
2.75	0.01\\
2.76	0.01\\
2.77	0.01\\
2.78	0.01\\
2.79	0.01\\
2.8	0.01\\
2.81	0.01\\
2.82	0.01\\
2.83	0.01\\
2.84	0.01\\
2.85	0.01\\
2.86	0.01\\
2.87	0.01\\
2.88	0.01\\
2.89	0.01\\
2.9	0.01\\
2.91	0.01\\
2.92	0.01\\
2.93	0.01\\
2.94	0.01\\
2.95	0.01\\
2.96	0.01\\
2.97	0.01\\
2.98	0.01\\
2.99	0.01\\
3	0.01\\
3.01	0.01\\
3.02	0.01\\
3.03	0.01\\
3.04	0.01\\
3.05	0.01\\
3.06	0.01\\
3.07	0.01\\
3.08	0.01\\
3.09	0.01\\
3.1	0.01\\
3.11	0.01\\
3.12	0.01\\
3.13	0.01\\
3.14	0.01\\
3.15	0.01\\
3.16	0.01\\
3.17	0.01\\
3.18	0.01\\
3.19	0.01\\
3.2	0.01\\
3.21	0.01\\
3.22	0.01\\
3.23	0.01\\
3.24	0.01\\
3.25	0.01\\
3.26	0.01\\
3.27	0.01\\
3.28	0.01\\
3.29	0.01\\
3.3	0.01\\
3.31	0.01\\
3.32	0.01\\
3.33	0.01\\
3.34	0.01\\
3.35	0.01\\
3.36	0.01\\
3.37	0.01\\
3.38	0.01\\
3.39	0.01\\
3.4	0.01\\
3.41	0.01\\
3.42	0.01\\
3.43	0.01\\
3.44	0.01\\
3.45	0.01\\
3.46	0.01\\
3.47	0.01\\
3.48	0.01\\
3.49	0.01\\
3.5	0.01\\
3.51	0.01\\
3.52	0.01\\
3.53	0.01\\
3.54	0.01\\
3.55	0.01\\
3.56	0.01\\
3.57	0.01\\
3.58	0.01\\
3.59	0.01\\
3.6	0.01\\
3.61	0.01\\
3.62	0.01\\
3.63	0.01\\
3.64	0.01\\
3.65	0.01\\
3.66	0.01\\
3.67	0.01\\
3.68	0.01\\
3.69	0.01\\
3.7	0.01\\
3.71	0.01\\
3.72	0.01\\
3.73	0.01\\
3.74	0.01\\
3.75	0.01\\
3.76	0.01\\
3.77	0.01\\
3.78	0.01\\
3.79	0.01\\
3.8	0.01\\
3.81	0.01\\
3.82	0.01\\
3.83	0.01\\
3.84	0.01\\
3.85	0.01\\
3.86	0.01\\
3.87	0.01\\
3.88	0.01\\
3.89	0.01\\
3.9	0.01\\
3.91	0.01\\
3.92	0.01\\
3.93	0.01\\
3.94	0.01\\
3.95	0.01\\
3.96	0.01\\
3.97	0.01\\
3.98	0.01\\
3.99	0.01\\
4	0.01\\
4.01	0.01\\
4.02	0.01\\
4.03	0.01\\
4.04	0.01\\
4.05	0.01\\
4.06	0.01\\
4.07	0.01\\
4.08	0.01\\
4.09	0.01\\
4.1	0.01\\
4.11	0.01\\
4.12	0.01\\
4.13	0.01\\
4.14	0.01\\
4.15	0.01\\
4.16	0.01\\
4.17	0.01\\
4.18	0.01\\
4.19	0.01\\
4.2	0.01\\
4.21	0.01\\
4.22	0.01\\
4.23	0.01\\
4.24	0.01\\
4.25	0.01\\
4.26	0.01\\
4.27	0.01\\
4.28	0.01\\
4.29	0.01\\
4.3	0.01\\
4.31	0.01\\
4.32	0.01\\
4.33	0.01\\
4.34	0.01\\
4.35	0.01\\
4.36	0.01\\
4.37	0.01\\
4.38	0.01\\
4.39	0.01\\
4.4	0.01\\
4.41	0.01\\
4.42	0.01\\
4.43	0.01\\
4.44	0.01\\
4.45	0.01\\
4.46	0.01\\
4.47	0.01\\
4.48	0.01\\
4.49	0.01\\
4.5	0.01\\
4.51	0.01\\
4.52	0.01\\
4.53	0.01\\
4.54	0.01\\
4.55	0.01\\
4.56	0.01\\
4.57	0.01\\
4.58	0.01\\
4.59	0.01\\
4.6	0.01\\
4.61	0.01\\
4.62	0.01\\
4.63	0.01\\
4.64	0.01\\
4.65	0.01\\
4.66	0.01\\
4.67	0.01\\
4.68	0.01\\
4.69	0.01\\
4.7	0.01\\
4.71	0.01\\
4.72	0.01\\
4.73	0.01\\
4.74	0.01\\
4.75	0.01\\
4.76	0.01\\
4.77	0.01\\
4.78	0.01\\
4.79	0.01\\
4.8	0.01\\
4.81	0.01\\
4.82	0.01\\
4.83	0.01\\
4.84	0.01\\
4.85	0.01\\
4.86	0.01\\
4.87	0.01\\
4.88	0.01\\
4.89	0.01\\
4.9	0.01\\
4.91	0.01\\
4.92	0.01\\
4.93	0.01\\
4.94	0.01\\
4.95	0.01\\
4.96	0.01\\
4.97	0.01\\
4.98	0.01\\
4.99	0.01\\
5	0.01\\
5.01	0.01\\
5.02	0.01\\
5.03	0.01\\
5.04	0.01\\
5.05	0.01\\
5.06	0.01\\
5.07	0.01\\
5.08	0.01\\
5.09	0.01\\
5.1	0.01\\
5.11	0.01\\
5.12	0.01\\
5.13	0.01\\
5.14	0.01\\
5.15	0.01\\
5.16	0.01\\
5.17	0.01\\
5.18	0.01\\
5.19	0.01\\
5.2	0.01\\
5.21	0.01\\
5.22	0.01\\
5.23	0.01\\
5.24	0.01\\
5.25	0.01\\
5.26	0.01\\
5.27	0.01\\
5.28	0.01\\
5.29	0.01\\
5.3	0.01\\
5.31	0.01\\
5.32	0.01\\
5.33	0.01\\
5.34	0.01\\
5.35	0.01\\
5.36	0.01\\
5.37	0.01\\
5.38	0.01\\
5.39	0.01\\
5.4	0.01\\
5.41	0.01\\
5.42	0.01\\
5.43	0.01\\
5.44	0.01\\
5.45	0.01\\
5.46	0.01\\
5.47	0.01\\
5.48	0.01\\
5.49	0.01\\
5.5	0.01\\
5.51	0.01\\
5.52	0.01\\
5.53	0.01\\
5.54	0.01\\
5.55	0.01\\
5.56	0.01\\
5.57	0.01\\
5.58	0.01\\
5.59	0.01\\
5.6	0.01\\
5.61	0.01\\
5.62	0.01\\
5.63	0.01\\
5.64	0.01\\
5.65	0.01\\
5.66	0.01\\
5.67	0.01\\
5.68	0.01\\
5.69	0.01\\
5.7	0.01\\
5.71	0.01\\
5.72	0.01\\
5.73	0.01\\
5.74	0.01\\
5.75	0.01\\
5.76	0.01\\
5.77	0.01\\
5.78	0.01\\
5.79	0.01\\
5.8	0.01\\
5.81	0.01\\
5.82	0.01\\
5.83	0.01\\
5.84	0.01\\
5.85	0.01\\
5.86	0.01\\
5.87	0.01\\
5.88	0.01\\
5.89	0.01\\
5.9	0.01\\
5.91	0.01\\
5.92	0.01\\
5.93	0.01\\
5.94	0.01\\
5.95	0.01\\
5.96	0.01\\
5.97	0.01\\
5.98	0.01\\
5.99	0.01\\
6	0.01\\
6.01	0.01\\
6.02	0.01\\
6.03	0.01\\
6.04	0.01\\
6.05	0.01\\
6.06	0.01\\
6.07	0.01\\
6.08	0.01\\
6.09	0.01\\
6.1	0.01\\
6.11	0.01\\
6.12	0.01\\
6.13	0.01\\
6.14	0.01\\
6.15	0.01\\
6.16	0.01\\
6.17	0.01\\
6.18	0.01\\
6.19	0.01\\
6.2	0.01\\
6.21	0.01\\
6.22	0.01\\
6.23	0.01\\
6.24	0.01\\
6.25	0.01\\
6.26	0.01\\
6.27	0.01\\
6.28	0.01\\
6.29	0.01\\
6.3	0.01\\
6.31	0.01\\
6.32	0.01\\
6.33	0.01\\
6.34	0.01\\
6.35	0.01\\
6.36	0.01\\
6.37	0.01\\
6.38	0.01\\
6.39	0.01\\
6.4	0.01\\
6.41	0.01\\
6.42	0.01\\
6.43	0.01\\
6.44	0.01\\
6.45	0.01\\
6.46	0.01\\
6.47	0.01\\
6.48	0.01\\
6.49	0.01\\
6.5	0.01\\
6.51	0.01\\
6.52	0.01\\
6.53	0.01\\
6.54	0.01\\
6.55	0.01\\
6.56	0.01\\
6.57	0.01\\
6.58	0.01\\
6.59	0.01\\
6.6	0.01\\
6.61	0.01\\
6.62	0.01\\
6.63	0.01\\
6.64	0.01\\
6.65	0.01\\
6.66	0.01\\
6.67	0.01\\
6.68	0.01\\
6.69	0.01\\
6.7	0.01\\
6.71	0.01\\
6.72	0.01\\
6.73	0.01\\
6.74	0.01\\
6.75	0.01\\
6.76	0.01\\
6.77	0.01\\
6.78	0.01\\
6.79	0.01\\
6.8	0.01\\
6.81	0.01\\
6.82	0.01\\
6.83	0.01\\
6.84	0.01\\
6.85	0.01\\
6.86	0.01\\
6.87	0.01\\
6.88	0.01\\
6.89	0.01\\
6.9	0.01\\
6.91	0.01\\
6.92	0.01\\
6.93	0.01\\
6.94	0.01\\
6.95	0.01\\
6.96	0.01\\
6.97	0.01\\
6.98	0.01\\
6.99	0.01\\
7	0.01\\
7.01	0.01\\
7.02	0.01\\
7.03	0.01\\
7.04	0.01\\
7.05	0.01\\
7.06	0.01\\
7.07	0.01\\
7.08	0.01\\
7.09	0.01\\
7.1	0.01\\
7.11	0.01\\
7.12	0.01\\
7.13	0.01\\
7.14	0.01\\
7.15	0.01\\
7.16	0.01\\
7.17	0.01\\
7.18	0.01\\
7.19	0.01\\
7.2	0.01\\
7.21	0.01\\
7.22	0.01\\
7.23	0.01\\
7.24	0.01\\
7.25	0.01\\
7.26	0.01\\
7.27	0.01\\
7.28	0.01\\
7.29	0.01\\
7.3	0.01\\
7.31	0.01\\
7.32	0.01\\
7.33	0.01\\
7.34	0.01\\
7.35	0.01\\
7.36	0.01\\
7.37	0.01\\
7.38	0.01\\
7.39	0.01\\
7.4	0.01\\
7.41	0.01\\
7.42	0.01\\
7.43	0.01\\
7.44	0.01\\
7.45	0.01\\
7.46	0.01\\
7.47	0.01\\
7.48	0.01\\
7.49	0.01\\
7.5	0.01\\
7.51	0.01\\
7.52	0.01\\
7.53	0.01\\
7.54	0.01\\
7.55	0.01\\
7.56	0.01\\
7.57	0.01\\
7.58	0.01\\
7.59	0.01\\
7.6	0.01\\
7.61	0.01\\
7.62	0.01\\
7.63	0.01\\
7.64	0.01\\
7.65	0.01\\
7.66	0.01\\
7.67	0.01\\
7.68	0.01\\
7.69	0.01\\
7.7	0.01\\
7.71	0.01\\
7.72	0.01\\
7.73	0.01\\
7.74	0.01\\
7.75	0.01\\
7.76	0.01\\
7.77	0.01\\
7.78	0.01\\
7.79	0.01\\
7.8	0.01\\
7.81	0.01\\
7.82	0.01\\
7.83	0.01\\
7.84	0.01\\
7.85	0.01\\
7.86	0.01\\
7.87	0.01\\
7.88	0.01\\
7.89	0.01\\
7.9	0.01\\
7.91	0.01\\
7.92	0.01\\
7.93	0.01\\
7.94	0.01\\
7.95	0.01\\
7.96	0.01\\
7.97	0.01\\
7.98	0.01\\
7.99	0.01\\
8	0.01\\
8.01	0.01\\
8.02	0.01\\
8.03	0.01\\
8.04	0.01\\
8.05	0.01\\
8.06	0.01\\
8.07	0.01\\
8.08	0.01\\
8.09	0.01\\
8.1	0.01\\
8.11	0.01\\
8.12	0.01\\
8.13	0.01\\
8.14	0.01\\
8.15	0.01\\
8.16	0.01\\
8.17	0.01\\
8.18	0.01\\
8.19	0.01\\
8.2	0.01\\
8.21	0.01\\
8.22	0.01\\
8.23	0.01\\
8.24	0.01\\
8.25	0.01\\
8.26	0.01\\
8.27	0.01\\
8.28	0.01\\
8.29	0.01\\
8.3	0.01\\
8.31	0.01\\
8.32	0.01\\
8.33	0.01\\
8.34	0.01\\
8.35	0.01\\
8.36	0.01\\
8.37	0.01\\
8.38	0.01\\
8.39	0.01\\
8.4	0.01\\
8.41	0.01\\
8.42	0.01\\
8.43	0.01\\
8.44	0.01\\
8.45	0.01\\
8.46	0.01\\
8.47	0.01\\
8.48	0.01\\
8.49	0.01\\
8.5	0.01\\
8.51	0.01\\
8.52	0.01\\
8.53	0.01\\
8.54	0.01\\
8.55	0.01\\
8.56	0.01\\
8.57	0.01\\
8.58	0.01\\
8.59	0.01\\
8.6	0.01\\
8.61	0.01\\
8.62	0.01\\
8.63	0.01\\
8.64	0.01\\
8.65	0.01\\
8.66	0.01\\
8.67	0.01\\
8.68	0.01\\
8.69	0.01\\
8.7	0.01\\
8.71	0.01\\
8.72	0.01\\
8.73	0.01\\
8.74	0.01\\
8.75	0.01\\
8.76	0.01\\
8.77	0.01\\
8.78	0.01\\
8.79	0.01\\
8.8	0.01\\
8.81	0.01\\
8.82	0.01\\
8.83	0.01\\
8.84	0.01\\
8.85	0.01\\
8.86	0.01\\
8.87	0.01\\
8.88	0.01\\
8.89	0.01\\
8.9	0.01\\
8.91	0.01\\
8.92	0.01\\
8.93	0.01\\
8.94	0.01\\
8.95	0.01\\
8.96	0.01\\
8.97	0.01\\
8.98	0.01\\
8.99	0.01\\
9	0.01\\
9.01	0.01\\
9.02	0.01\\
9.03	0.01\\
9.04	0.01\\
9.05	0.01\\
9.06	0.01\\
9.07	0.01\\
9.08	0.01\\
9.09	0.01\\
9.1	0.01\\
9.11	0.01\\
9.12	0.01\\
9.13	0.01\\
9.14	0.01\\
9.15	0.01\\
9.16	0.01\\
9.17	0.01\\
9.18	0.01\\
9.19	0.01\\
9.2	0.01\\
9.21	0.01\\
9.22	0.01\\
9.23	0.01\\
9.24	0.01\\
9.25	0.01\\
9.26	0.01\\
9.27	0.01\\
9.28	0.01\\
9.29	0.01\\
9.3	0.01\\
9.31	0.01\\
9.32	0.01\\
9.33	0.01\\
9.34	0.01\\
9.35	0.01\\
9.36	0.01\\
9.37	0.01\\
9.38	0.01\\
9.39	0.01\\
9.4	0.01\\
9.41	0.01\\
9.42	0.01\\
9.43	0.01\\
9.44	0.01\\
9.45	0.01\\
9.46	0.01\\
9.47	0.01\\
9.48	0.01\\
9.49	0.01\\
9.5	0.01\\
9.51	0.01\\
9.52	0.01\\
9.53	0.01\\
9.54	0.01\\
9.55	0.01\\
9.56	0.01\\
9.57	0.01\\
9.58	0.01\\
9.59	0.01\\
9.6	0.01\\
9.61	0.01\\
9.62	0.01\\
9.63	0.01\\
9.64	0.01\\
9.65	0.01\\
9.66	0.01\\
9.67	0.01\\
9.68	0.01\\
9.69	0.01\\
9.7	0.01\\
9.71	0.01\\
9.72	0.01\\
9.73	0.01\\
9.74	0.01\\
9.75	0.01\\
9.76	0.01\\
9.77	0.01\\
9.78	0.01\\
9.79	0.01\\
9.8	0.01\\
9.81	0.01\\
9.82	0.01\\
9.83	0.01\\
9.84	0.01\\
9.85	0.01\\
9.86	0.01\\
9.87	0.01\\
9.88	0.01\\
9.89	0.01\\
9.9	0.01\\
9.91	0.01\\
9.92	0.01\\
9.93	0.01\\
9.94	0.01\\
9.95	0.01\\
9.96	0.01\\
9.97	0.01\\
9.98	0.01\\
9.99	0.01\\
10	0.01\\
10.01	0.01\\
10.02	0.01\\
10.03	0.01\\
10.04	0.01\\
10.05	0.01\\
10.06	0.01\\
10.07	0.01\\
10.08	0.01\\
10.09	0.01\\
10.1	0.01\\
10.11	0.01\\
10.12	0.01\\
10.13	0.01\\
10.14	0.01\\
10.15	0.01\\
10.16	0.01\\
10.17	0.01\\
10.18	0.01\\
10.19	0.01\\
10.2	0.01\\
10.21	0.01\\
10.22	0.01\\
10.23	0.01\\
10.24	0.01\\
10.25	0.01\\
10.26	0.01\\
10.27	0.01\\
10.28	0.01\\
10.29	0.01\\
10.3	0.01\\
10.31	0.01\\
10.32	0.01\\
10.33	0.01\\
10.34	0.01\\
10.35	0.01\\
10.36	0.01\\
10.37	0.01\\
10.38	0.01\\
10.39	0.01\\
10.4	0.01\\
10.41	0.01\\
10.42	0.01\\
10.43	0.01\\
10.44	0.01\\
10.45	0.01\\
10.46	0.01\\
10.47	0.01\\
10.48	0.01\\
10.49	0.01\\
10.5	0.01\\
10.51	0.01\\
10.52	0.01\\
10.53	0.01\\
10.54	0.01\\
10.55	0.01\\
10.56	0.01\\
10.57	0.01\\
10.58	0.01\\
10.59	0.01\\
10.6	0.01\\
10.61	0.01\\
10.62	0.01\\
10.63	0.01\\
10.64	0.01\\
10.65	0.01\\
10.66	0.01\\
10.67	0.01\\
10.68	0.01\\
10.69	0.01\\
10.7	0.01\\
10.71	0.01\\
10.72	0.01\\
10.73	0.01\\
10.74	0.01\\
10.75	0.01\\
10.76	0.01\\
10.77	0.01\\
10.78	0.01\\
10.79	0.01\\
10.8	0.01\\
10.81	0.01\\
10.82	0.01\\
10.83	0.01\\
10.84	0.01\\
10.85	0.01\\
10.86	0.01\\
10.87	0.01\\
10.88	0.01\\
10.89	0.01\\
10.9	0.01\\
10.91	0.01\\
10.92	0.01\\
10.93	0.01\\
10.94	0.01\\
10.95	0.01\\
10.96	0.01\\
10.97	0.01\\
10.98	0.01\\
10.99	0.01\\
11	0.01\\
11.01	0.01\\
11.02	0.01\\
11.03	0.01\\
11.04	0.01\\
11.05	0.01\\
11.06	0.01\\
11.07	0.01\\
11.08	0.01\\
11.09	0.01\\
11.1	0.01\\
11.11	0.01\\
11.12	0.01\\
11.13	0.01\\
11.14	0.01\\
11.15	0.01\\
11.16	0.01\\
11.17	0.01\\
11.18	0.01\\
11.19	0.01\\
11.2	0.01\\
11.21	0.01\\
11.22	0.01\\
11.23	0.01\\
11.24	0.01\\
11.25	0.01\\
11.26	0.01\\
11.27	0.01\\
11.28	0.01\\
11.29	0.01\\
11.3	0.01\\
11.31	0.01\\
11.32	0.01\\
11.33	0.01\\
11.34	0.01\\
11.35	0.01\\
11.36	0.01\\
11.37	0.01\\
11.38	0.01\\
11.39	0.01\\
11.4	0.01\\
11.41	0.01\\
11.42	0.01\\
11.43	0.01\\
11.44	0.01\\
11.45	0.01\\
11.46	0.01\\
11.47	0.01\\
11.48	0.01\\
11.49	0.01\\
11.5	0.01\\
11.51	0.01\\
11.52	0.01\\
11.53	0.01\\
11.54	0.01\\
11.55	0.01\\
11.56	0.01\\
11.57	0.01\\
11.58	0.01\\
11.59	0.01\\
11.6	0.01\\
11.61	0.01\\
11.62	0.01\\
11.63	0.01\\
11.64	0.01\\
11.65	0.01\\
11.66	0.01\\
11.67	0.01\\
11.68	0.01\\
11.69	0.01\\
11.7	0.01\\
11.71	0.01\\
11.72	0.01\\
11.73	0.01\\
11.74	0.01\\
11.75	0.01\\
11.76	0.01\\
11.77	0.01\\
11.78	0.01\\
11.79	0.01\\
11.8	0.01\\
11.81	0.01\\
11.82	0.01\\
11.83	0.01\\
11.84	0.01\\
11.85	0.01\\
11.86	0.01\\
11.87	0.01\\
11.88	0.01\\
11.89	0.01\\
11.9	0.01\\
11.91	0.01\\
11.92	0.01\\
11.93	0.01\\
11.94	0.01\\
11.95	0.01\\
11.96	0.01\\
11.97	0.01\\
11.98	0.01\\
11.99	0.01\\
12	0.01\\
12.01	0.01\\
12.02	0.01\\
12.03	0.01\\
12.04	0.01\\
12.05	0.01\\
12.06	0.01\\
12.07	0.01\\
12.08	0.01\\
12.09	0.01\\
12.1	0.01\\
12.11	0.01\\
12.12	0.01\\
12.13	0.01\\
12.14	0.01\\
12.15	0.01\\
12.16	0.01\\
12.17	0.01\\
12.18	0.01\\
12.19	0.01\\
12.2	0.01\\
12.21	0.01\\
12.22	0.01\\
12.23	0.01\\
12.24	0.01\\
12.25	0.01\\
12.26	0.01\\
12.27	0.01\\
12.28	0.01\\
12.29	0.01\\
12.3	0.01\\
12.31	0.01\\
12.32	0.01\\
12.33	0.01\\
12.34	0.01\\
12.35	0.01\\
12.36	0.01\\
12.37	0.01\\
12.38	0.01\\
12.39	0.01\\
12.4	0.01\\
12.41	0.01\\
12.42	0.01\\
12.43	0.01\\
12.44	0.01\\
12.45	0.01\\
12.46	0.01\\
12.47	0.01\\
12.48	0.01\\
12.49	0.01\\
12.5	0.01\\
12.51	0.01\\
12.52	0.01\\
12.53	0.01\\
12.54	0.01\\
12.55	0.01\\
12.56	0.01\\
12.57	0.01\\
12.58	0.01\\
12.59	0.01\\
12.6	0.01\\
12.61	0.01\\
12.62	0.01\\
12.63	0.01\\
12.64	0.01\\
12.65	0.01\\
12.66	0.01\\
12.67	0.01\\
12.68	0.01\\
12.69	0.01\\
12.7	0.01\\
12.71	0.01\\
12.72	0.01\\
12.73	0.01\\
12.74	0.01\\
12.75	0.01\\
12.76	0.01\\
12.77	0.01\\
12.78	0.01\\
12.79	0.01\\
12.8	0.01\\
12.81	0.01\\
12.82	0.01\\
12.83	0.01\\
12.84	0.01\\
12.85	0.01\\
12.86	0.01\\
12.87	0.01\\
12.88	0.01\\
12.89	0.01\\
12.9	0.01\\
12.91	0.01\\
12.92	0.01\\
12.93	0.01\\
12.94	0.01\\
12.95	0.01\\
12.96	0.01\\
12.97	0.01\\
12.98	0.01\\
12.99	0.01\\
13	0.01\\
13.01	0.01\\
13.02	0.01\\
13.03	0.01\\
13.04	0.01\\
13.05	0.01\\
13.06	0.01\\
13.07	0.01\\
13.08	0.01\\
13.09	0.01\\
13.1	0.01\\
13.11	0.01\\
13.12	0.01\\
13.13	0.01\\
13.14	0.01\\
13.15	0.01\\
13.16	0.01\\
13.17	0.01\\
13.18	0.01\\
13.19	0.01\\
13.2	0.01\\
13.21	0.01\\
13.22	0.01\\
13.23	0.01\\
13.24	0.01\\
13.25	0.01\\
13.26	0.01\\
13.27	0.01\\
13.28	0.01\\
13.29	0.01\\
13.3	0.01\\
13.31	0.01\\
13.32	0.01\\
13.33	0.01\\
13.34	0.01\\
13.35	0.01\\
13.36	0.01\\
13.37	0.01\\
13.38	0.01\\
13.39	0.01\\
13.4	0.01\\
13.41	0.01\\
13.42	0.01\\
13.43	0.01\\
13.44	0.01\\
13.45	0.01\\
13.46	0.01\\
13.47	0.01\\
13.48	0.01\\
13.49	0.01\\
13.5	0.01\\
13.51	0.01\\
13.52	0.01\\
13.53	0.01\\
13.54	0.01\\
13.55	0.01\\
13.56	0.01\\
13.57	0.01\\
13.58	0.01\\
13.59	0.01\\
13.6	0.01\\
13.61	0.01\\
13.62	0.01\\
13.63	0.01\\
13.64	0.01\\
13.65	0.01\\
13.66	0.01\\
13.67	0.01\\
13.68	0.01\\
13.69	0.01\\
13.7	0.01\\
13.71	0.01\\
13.72	0.01\\
13.73	0.01\\
13.74	0.01\\
13.75	0.01\\
13.76	0.01\\
13.77	0.01\\
13.78	0.01\\
13.79	0.01\\
13.8	0.01\\
13.81	0.01\\
13.82	0.01\\
13.83	0.01\\
13.84	0.01\\
13.85	0.01\\
13.86	0.01\\
13.87	0.01\\
13.88	0.01\\
13.89	0.01\\
13.9	0.01\\
13.91	0.01\\
13.92	0.01\\
13.93	0.01\\
13.94	0.01\\
13.95	0.01\\
13.96	0.01\\
13.97	0.01\\
13.98	0.01\\
13.99	0.01\\
14	0.01\\
14.01	0.01\\
14.02	0.01\\
14.03	0.01\\
14.04	0.01\\
14.05	0.01\\
14.06	0.01\\
14.07	0.01\\
14.08	0.01\\
14.09	0.01\\
14.1	0.01\\
14.11	0.01\\
14.12	0.01\\
14.13	0.01\\
14.14	0.01\\
14.15	0.01\\
14.16	0.01\\
14.17	0.01\\
14.18	0.01\\
14.19	0.01\\
14.2	0.01\\
14.21	0.01\\
14.22	0.01\\
14.23	0.01\\
14.24	0.01\\
14.25	0.01\\
14.26	0.01\\
14.27	0.01\\
14.28	0.01\\
14.29	0.01\\
14.3	0.01\\
14.31	0.01\\
14.32	0.01\\
14.33	0.01\\
14.34	0.01\\
14.35	0.01\\
14.36	0.01\\
14.37	0.01\\
14.38	0.01\\
14.39	0.01\\
14.4	0.01\\
14.41	0.01\\
14.42	0.01\\
14.43	0.01\\
14.44	0.01\\
14.45	0.01\\
14.46	0.01\\
14.47	0.01\\
14.48	0.01\\
14.49	0.01\\
14.5	0.01\\
14.51	0.01\\
14.52	0.01\\
14.53	0.01\\
14.54	0.01\\
14.55	0.01\\
14.56	0.01\\
14.57	0.01\\
14.58	0.01\\
14.59	0.01\\
14.6	0.01\\
14.61	0.01\\
14.62	0.01\\
14.63	0.01\\
14.64	0.01\\
14.65	0.01\\
14.66	0.01\\
14.67	0.01\\
14.68	0.01\\
14.69	0.01\\
14.7	0.01\\
14.71	0.01\\
14.72	0.01\\
14.73	0.01\\
14.74	0.01\\
14.75	0.01\\
14.76	0.01\\
14.77	0.01\\
14.78	0.01\\
14.79	0.01\\
14.8	0.01\\
14.81	0.01\\
14.82	0.01\\
14.83	0.01\\
14.84	0.01\\
14.85	0.01\\
14.86	0.01\\
14.87	0.01\\
14.88	0.01\\
14.89	0.01\\
14.9	0.01\\
14.91	0.01\\
14.92	0.01\\
14.93	0.01\\
14.94	0.01\\
14.95	0.01\\
14.96	0.01\\
14.97	0.01\\
14.98	0.01\\
14.99	0.01\\
15	0.01\\
15.01	0.01\\
15.02	0.01\\
15.03	0.01\\
15.04	0.01\\
15.05	0.01\\
15.06	0.01\\
15.07	0.01\\
15.08	0.01\\
15.09	0.01\\
15.1	0.01\\
15.11	0.01\\
15.12	0.01\\
15.13	0.01\\
15.14	0.01\\
15.15	0.01\\
15.16	0.01\\
15.17	0.01\\
15.18	0.01\\
15.19	0.01\\
15.2	0.01\\
15.21	0.01\\
15.22	0.01\\
15.23	0.01\\
15.24	0.01\\
15.25	0.01\\
15.26	0.01\\
15.27	0.01\\
15.28	0.01\\
15.29	0.01\\
15.3	0.01\\
15.31	0.01\\
15.32	0.01\\
15.33	0.01\\
15.34	0.01\\
15.35	0.01\\
15.36	0.01\\
15.37	0.01\\
15.38	0.01\\
15.39	0.01\\
15.4	0.01\\
15.41	0.01\\
15.42	0.01\\
15.43	0.01\\
15.44	0.01\\
15.45	0.01\\
15.46	0.01\\
15.47	0.01\\
15.48	0.01\\
15.49	0.01\\
15.5	0.01\\
15.51	0.01\\
15.52	0.01\\
15.53	0.01\\
15.54	0.01\\
15.55	0.01\\
15.56	0.01\\
15.57	0.01\\
15.58	0.01\\
15.59	0.01\\
15.6	0.01\\
15.61	0.01\\
15.62	0.01\\
15.63	0.01\\
15.64	0.01\\
15.65	0.01\\
15.66	0.01\\
15.67	0.01\\
15.68	0.01\\
15.69	0.01\\
15.7	0.01\\
15.71	0.01\\
15.72	0.01\\
15.73	0.01\\
15.74	0.01\\
15.75	0.01\\
15.76	0.01\\
15.77	0.01\\
15.78	0.01\\
15.79	0.01\\
15.8	0.01\\
15.81	0.01\\
15.82	0.01\\
15.83	0.01\\
15.84	0.01\\
15.85	0.01\\
15.86	0.01\\
15.87	0.01\\
15.88	0.01\\
15.89	0.01\\
15.9	0.01\\
15.91	0.01\\
15.92	0.01\\
15.93	0.01\\
15.94	0.01\\
15.95	0.01\\
15.96	0.01\\
15.97	0.01\\
15.98	0.01\\
15.99	0.01\\
16	0.01\\
16.01	0.01\\
16.02	0.01\\
16.03	0.01\\
16.04	0.01\\
16.05	0.01\\
16.06	0.01\\
16.07	0.01\\
16.08	0.01\\
16.09	0.01\\
16.1	0.01\\
16.11	0.01\\
16.12	0.01\\
16.13	0.01\\
16.14	0.01\\
16.15	0.01\\
16.16	0.01\\
16.17	0.01\\
16.18	0.01\\
16.19	0.01\\
16.2	0.01\\
16.21	0.01\\
16.22	0.01\\
16.23	0.01\\
16.24	0.01\\
16.25	0.01\\
16.26	0.01\\
16.27	0.01\\
16.28	0.01\\
16.29	0.01\\
16.3	0.01\\
16.31	0.01\\
16.32	0.01\\
16.33	0.01\\
16.34	0.01\\
16.35	0.01\\
16.36	0.01\\
16.37	0.01\\
16.38	0.01\\
16.39	0.01\\
16.4	0.01\\
16.41	0.01\\
16.42	0.01\\
16.43	0.01\\
16.44	0.01\\
16.45	0.01\\
16.46	0.01\\
16.47	0.01\\
16.48	0.01\\
16.49	0.01\\
16.5	0.01\\
16.51	0.01\\
16.52	0.01\\
16.53	0.01\\
16.54	0.01\\
16.55	0.01\\
16.56	0.01\\
16.57	0.01\\
16.58	0.01\\
16.59	0.01\\
16.6	0.01\\
16.61	0.01\\
16.62	0.01\\
16.63	0.01\\
16.64	0.01\\
16.65	0.01\\
16.66	0.01\\
16.67	0.01\\
16.68	0.01\\
16.69	0.01\\
16.7	0.01\\
16.71	0.01\\
16.72	0.01\\
16.73	0.01\\
16.74	0.01\\
16.75	0.01\\
16.76	0.01\\
16.77	0.01\\
16.78	0.01\\
16.79	0.01\\
16.8	0.01\\
16.81	0.01\\
16.82	0.01\\
16.83	0.01\\
16.84	0.01\\
16.85	0.01\\
16.86	0.01\\
16.87	0.01\\
16.88	0.01\\
16.89	0.01\\
16.9	0.01\\
16.91	0.01\\
16.92	0.01\\
16.93	0.01\\
16.94	0.01\\
16.95	0.01\\
16.96	0.01\\
16.97	0.01\\
16.98	0.01\\
16.99	0.01\\
17	0.01\\
17.01	0.01\\
17.02	0.01\\
17.03	0.01\\
17.04	0.01\\
17.05	0.01\\
17.06	0.01\\
17.07	0.01\\
17.08	0.01\\
17.09	0.01\\
17.1	0.01\\
17.11	0.01\\
17.12	0.01\\
17.13	0.01\\
17.14	0.01\\
17.15	0.01\\
17.16	0.01\\
17.17	0.01\\
17.18	0.01\\
17.19	0.01\\
17.2	0.01\\
17.21	0.01\\
17.22	0.01\\
17.23	0.01\\
17.24	0.01\\
17.25	0.01\\
17.26	0.01\\
17.27	0.01\\
17.28	0.01\\
17.29	0.01\\
17.3	0.01\\
17.31	0.01\\
17.32	0.01\\
17.33	0.01\\
17.34	0.01\\
17.35	0.01\\
17.36	0.01\\
17.37	0.01\\
17.38	0.01\\
17.39	0.01\\
17.4	0.01\\
17.41	0.01\\
17.42	0.01\\
17.43	0.01\\
17.44	0.01\\
17.45	0.01\\
17.46	0.01\\
17.47	0.01\\
17.48	0.01\\
17.49	0.01\\
17.5	0.01\\
17.51	0.01\\
17.52	0.01\\
17.53	0.01\\
17.54	0.01\\
17.55	0.01\\
17.56	0.01\\
17.57	0.01\\
17.58	0.01\\
17.59	0.01\\
17.6	0.01\\
17.61	0.01\\
17.62	0.01\\
17.63	0.01\\
17.64	0.01\\
17.65	0.01\\
17.66	0.01\\
17.67	0.01\\
17.68	0.01\\
17.69	0.01\\
17.7	0.01\\
17.71	0.01\\
17.72	0.01\\
17.73	0.01\\
17.74	0.01\\
17.75	0.01\\
17.76	0.01\\
17.77	0.01\\
17.78	0.01\\
17.79	0.01\\
17.8	0.01\\
17.81	0.01\\
17.82	0.01\\
17.83	0.01\\
17.84	0.01\\
17.85	0.01\\
17.86	0.01\\
17.87	0.01\\
17.88	0.01\\
17.89	0.01\\
17.9	0.01\\
17.91	0.01\\
17.92	0.01\\
17.93	0.01\\
17.94	0.01\\
17.95	0.01\\
17.96	0.01\\
17.97	0.01\\
17.98	0.01\\
17.99	0.01\\
18	0.01\\
18.01	0.01\\
18.02	0.01\\
18.03	0.01\\
18.04	0.01\\
18.05	0.01\\
18.06	0.01\\
18.07	0.01\\
18.08	0.01\\
18.09	0.01\\
18.1	0.01\\
18.11	0.01\\
18.12	0.01\\
18.13	0.01\\
18.14	0.01\\
18.15	0.01\\
18.16	0.01\\
18.17	0.01\\
18.18	0.01\\
18.19	0.01\\
18.2	0.01\\
18.21	0.01\\
18.22	0.01\\
18.23	0.01\\
18.24	0.01\\
18.25	0.01\\
18.26	0.01\\
18.27	0.01\\
18.28	0.01\\
18.29	0.01\\
18.3	0.01\\
18.31	0.01\\
18.32	0.01\\
18.33	0.01\\
18.34	0.01\\
18.35	0.01\\
18.36	0.01\\
18.37	0.01\\
18.38	0.01\\
18.39	0.01\\
18.4	0.01\\
18.41	0.01\\
18.42	0.01\\
18.43	0.01\\
18.44	0.01\\
18.45	0.01\\
18.46	0.01\\
18.47	0.01\\
18.48	0.01\\
18.49	0.01\\
18.5	0.01\\
18.51	0.01\\
18.52	0.01\\
18.53	0.01\\
18.54	0.01\\
18.55	0.01\\
18.56	0.01\\
18.57	0.01\\
18.58	0.01\\
18.59	0.01\\
18.6	0.01\\
18.61	0.01\\
18.62	0.01\\
18.63	0.01\\
18.64	0.01\\
18.65	0.01\\
18.66	0.01\\
18.67	0.01\\
18.68	0.01\\
18.69	0.01\\
18.7	0.01\\
18.71	0.01\\
18.72	0.01\\
18.73	0.01\\
18.74	0.01\\
18.75	0.01\\
18.76	0.01\\
18.77	0.01\\
18.78	0.01\\
18.79	0.01\\
18.8	0.01\\
18.81	0.01\\
18.82	0.01\\
18.83	0.01\\
18.84	0.01\\
18.85	0.01\\
18.86	0.01\\
18.87	0.01\\
18.88	0.01\\
18.89	0.01\\
18.9	0.01\\
18.91	0.01\\
18.92	0.01\\
18.93	0.01\\
18.94	0.01\\
18.95	0.01\\
18.96	0.01\\
18.97	0.01\\
18.98	0.01\\
18.99	0.01\\
19	0.01\\
19.01	0.01\\
19.02	0.01\\
19.03	0.01\\
19.04	0.01\\
19.05	0.01\\
19.06	0.01\\
19.07	0.01\\
19.08	0.01\\
19.09	0.01\\
19.1	0.01\\
19.11	0.01\\
19.12	0.01\\
19.13	0.01\\
19.14	0.01\\
19.15	0.01\\
19.16	0.01\\
19.17	0.01\\
19.18	0.01\\
19.19	0.01\\
19.2	0.01\\
19.21	0.01\\
19.22	0.01\\
19.23	0.01\\
19.24	0.01\\
19.25	0.01\\
19.26	0.01\\
19.27	0.01\\
19.28	0.01\\
19.29	0.01\\
19.3	0.01\\
19.31	0.01\\
19.32	0.01\\
19.33	0.01\\
19.34	0.01\\
19.35	0.01\\
19.36	0.01\\
19.37	0.01\\
19.38	0.01\\
19.39	0.01\\
19.4	0.01\\
19.41	0.01\\
19.42	0.01\\
19.43	0.01\\
19.44	0.01\\
19.45	0.01\\
19.46	0.01\\
19.47	0.01\\
19.48	0.01\\
19.49	0.01\\
19.5	0.01\\
19.51	0.01\\
19.52	0.01\\
19.53	0.01\\
19.54	0.01\\
19.55	0.01\\
19.56	0.01\\
19.57	0.01\\
19.58	0.01\\
19.59	0.01\\
19.6	0.01\\
19.61	0.01\\
19.62	0.01\\
19.63	0.01\\
19.64	0.01\\
19.65	0.01\\
19.66	0.01\\
19.67	0.01\\
19.68	0.01\\
19.69	0.01\\
19.7	0.01\\
19.71	0.01\\
19.72	0.01\\
19.73	0.01\\
19.74	0.01\\
19.75	0.01\\
19.76	0.01\\
19.77	0.01\\
19.78	0.01\\
19.79	0.01\\
19.8	0.01\\
19.81	0.01\\
19.82	0.01\\
19.83	0.01\\
19.84	0.01\\
19.85	0.01\\
19.86	0.01\\
19.87	0.01\\
19.88	0.01\\
19.89	0.01\\
19.9	0.01\\
19.91	0.01\\
19.92	0.01\\
19.93	0.01\\
19.94	0.01\\
19.95	0.01\\
19.96	0.01\\
19.97	0.01\\
19.98	0.01\\
19.99	0.01\\
20	0.01\\
20.01	0.01\\
20.02	0.01\\
20.03	0.01\\
20.04	0.01\\
20.05	0.01\\
20.06	0.01\\
20.07	0.01\\
20.08	0.01\\
20.09	0.01\\
20.1	0.01\\
20.11	0.01\\
20.12	0.01\\
20.13	0.01\\
20.14	0.01\\
20.15	0.01\\
20.16	0.01\\
20.17	0.01\\
20.18	0.01\\
20.19	0.01\\
20.2	0.01\\
20.21	0.01\\
20.22	0.01\\
20.23	0.01\\
20.24	0.01\\
20.25	0.01\\
20.26	0.01\\
20.27	0.01\\
20.28	0.01\\
20.29	0.01\\
20.3	0.01\\
20.31	0.01\\
20.32	0.01\\
20.33	0.01\\
20.34	0.01\\
20.35	0.01\\
20.36	0.01\\
20.37	0.01\\
20.38	0.01\\
20.39	0.01\\
20.4	0.01\\
20.41	0.01\\
20.42	0.01\\
20.43	0.01\\
20.44	0.01\\
20.45	0.01\\
20.46	0.01\\
20.47	0.01\\
20.48	0.01\\
20.49	0.01\\
20.5	0.01\\
20.51	0.01\\
20.52	0.01\\
20.53	0.01\\
20.54	0.01\\
20.55	0.01\\
20.56	0.01\\
20.57	0.01\\
20.58	0.01\\
20.59	0.01\\
20.6	0.01\\
20.61	0.01\\
20.62	0.01\\
20.63	0.01\\
20.64	0.01\\
20.65	0.01\\
20.66	0.01\\
20.67	0.01\\
20.68	0.01\\
20.69	0.01\\
20.7	0.01\\
20.71	0.01\\
20.72	0.01\\
20.73	0.01\\
20.74	0.01\\
20.75	0.01\\
20.76	0.01\\
20.77	0.01\\
20.78	0.01\\
20.79	0.01\\
20.8	0.01\\
20.81	0.01\\
20.82	0.01\\
20.83	0.01\\
20.84	0.01\\
20.85	0.01\\
20.86	0.01\\
20.87	0.01\\
20.88	0.01\\
20.89	0.01\\
20.9	0.01\\
20.91	0.01\\
20.92	0.01\\
20.93	0.01\\
20.94	0.01\\
20.95	0.01\\
20.96	0.01\\
20.97	0.01\\
20.98	0.01\\
20.99	0.01\\
21	0.01\\
21.01	0.01\\
21.02	0.01\\
21.03	0.01\\
21.04	0.01\\
21.05	0.01\\
21.06	0.01\\
21.07	0.01\\
21.08	0.01\\
21.09	0.01\\
21.1	0.01\\
21.11	0.01\\
21.12	0.01\\
21.13	0.01\\
21.14	0.01\\
21.15	0.01\\
21.16	0.01\\
21.17	0.01\\
21.18	0.01\\
21.19	0.01\\
21.2	0.01\\
21.21	0.01\\
21.22	0.01\\
21.23	0.01\\
21.24	0.01\\
21.25	0.01\\
21.26	0.01\\
21.27	0.01\\
21.28	0.01\\
21.29	0.01\\
21.3	0.01\\
21.31	0.01\\
21.32	0.01\\
21.33	0.01\\
21.34	0.01\\
21.35	0.01\\
21.36	0.01\\
21.37	0.01\\
21.38	0.01\\
21.39	0.01\\
21.4	0.01\\
21.41	0.01\\
21.42	0.01\\
21.43	0.01\\
21.44	0.01\\
21.45	0.01\\
21.46	0.01\\
21.47	0.01\\
21.48	0.01\\
21.49	0.01\\
21.5	0.01\\
21.51	0.01\\
21.52	0.01\\
21.53	0.01\\
21.54	0.01\\
21.55	0.01\\
21.56	0.01\\
21.57	0.01\\
21.58	0.01\\
21.59	0.01\\
21.6	0.01\\
21.61	0.01\\
21.62	0.01\\
21.63	0.01\\
21.64	0.01\\
21.65	0.01\\
21.66	0.01\\
21.67	0.01\\
21.68	0.01\\
21.69	0.01\\
21.7	0.01\\
21.71	0.01\\
21.72	0.01\\
21.73	0.01\\
21.74	0.01\\
21.75	0.01\\
21.76	0.01\\
21.77	0.01\\
21.78	0.01\\
21.79	0.01\\
21.8	0.01\\
21.81	0.01\\
21.82	0.01\\
21.83	0.01\\
21.84	0.01\\
21.85	0.01\\
21.86	0.01\\
21.87	0.01\\
21.88	0.01\\
21.89	0.01\\
21.9	0.01\\
21.91	0.01\\
21.92	0.01\\
21.93	0.01\\
21.94	0.01\\
21.95	0.01\\
21.96	0.01\\
21.97	0.01\\
21.98	0.01\\
21.99	0.01\\
22	0.01\\
22.01	0.01\\
22.02	0.01\\
22.03	0.01\\
22.04	0.01\\
22.05	0.01\\
22.06	0.01\\
22.07	0.01\\
22.08	0.01\\
22.09	0.01\\
22.1	0.01\\
22.11	0.01\\
22.12	0.01\\
22.13	0.01\\
22.14	0.01\\
22.15	0.01\\
22.16	0.01\\
22.17	0.01\\
22.18	0.01\\
22.19	0.01\\
22.2	0.01\\
22.21	0.01\\
22.22	0.01\\
22.23	0.01\\
22.24	0.01\\
22.25	0.01\\
22.26	0.01\\
22.27	0.01\\
22.28	0.01\\
22.29	0.01\\
22.3	0.01\\
22.31	0.01\\
22.32	0.01\\
22.33	0.01\\
22.34	0.01\\
22.35	0.01\\
22.36	0.01\\
22.37	0.01\\
22.38	0.01\\
22.39	0.01\\
22.4	0.01\\
22.41	0.01\\
22.42	0.01\\
22.43	0.01\\
22.44	0.01\\
22.45	0.01\\
22.46	0.01\\
22.47	0.01\\
22.48	0.01\\
22.49	0.01\\
22.5	0.01\\
22.51	0.01\\
22.52	0.01\\
22.53	0.01\\
22.54	0.01\\
22.55	0.01\\
22.56	0.01\\
22.57	0.01\\
22.58	0.01\\
22.59	0.01\\
22.6	0.01\\
22.61	0.01\\
22.62	0.01\\
22.63	0.01\\
22.64	0.01\\
22.65	0.01\\
22.66	0.01\\
22.67	0.01\\
22.68	0.01\\
22.69	0.01\\
22.7	0.01\\
22.71	0.01\\
22.72	0.01\\
22.73	0.01\\
22.74	0.01\\
22.75	0.01\\
22.76	0.01\\
22.77	0.01\\
22.78	0.01\\
22.79	0.01\\
22.8	0.01\\
22.81	0.01\\
22.82	0.01\\
22.83	0.01\\
22.84	0.01\\
22.85	0.01\\
22.86	0.01\\
22.87	0.01\\
22.88	0.01\\
22.89	0.01\\
22.9	0.01\\
22.91	0.01\\
22.92	0.01\\
22.93	0.01\\
22.94	0.01\\
22.95	0.01\\
22.96	0.01\\
22.97	0.01\\
22.98	0.01\\
22.99	0.01\\
23	0.01\\
23.01	0.01\\
23.02	0.01\\
23.03	0.01\\
23.04	0.01\\
23.05	0.01\\
23.06	0.01\\
23.07	0.01\\
23.08	0.01\\
23.09	0.01\\
23.1	0.01\\
23.11	0.01\\
23.12	0.01\\
23.13	0.01\\
23.14	0.01\\
23.15	0.01\\
23.16	0.01\\
23.17	0.01\\
23.18	0.01\\
23.19	0.01\\
23.2	0.01\\
23.21	0.01\\
23.22	0.01\\
23.23	0.01\\
23.24	0.01\\
23.25	0.01\\
23.26	0.01\\
23.27	0.01\\
23.28	0.01\\
23.29	0.01\\
23.3	0.01\\
23.31	0.01\\
23.32	0.01\\
23.33	0.01\\
23.34	0.01\\
23.35	0.01\\
23.36	0.01\\
23.37	0.01\\
23.38	0.01\\
23.39	0.01\\
23.4	0.01\\
23.41	0.01\\
23.42	0.01\\
23.43	0.01\\
23.44	0.01\\
23.45	0.01\\
23.46	0.01\\
23.47	0.01\\
23.48	0.01\\
23.49	0.01\\
23.5	0.01\\
23.51	0.01\\
23.52	0.01\\
23.53	0.01\\
23.54	0.01\\
23.55	0.01\\
23.56	0.01\\
23.57	0.01\\
23.58	0.01\\
23.59	0.01\\
23.6	0.01\\
23.61	0.01\\
23.62	0.01\\
23.63	0.01\\
23.64	0.01\\
23.65	0.01\\
23.66	0.01\\
23.67	0.01\\
23.68	0.01\\
23.69	0.01\\
23.7	0.01\\
23.71	0.01\\
23.72	0.01\\
23.73	0.01\\
23.74	0.01\\
23.75	0.01\\
23.76	0.01\\
23.77	0.01\\
23.78	0.01\\
23.79	0.01\\
23.8	0.01\\
23.81	0.01\\
23.82	0.01\\
23.83	0.01\\
23.84	0.01\\
23.85	0.01\\
23.86	0.01\\
23.87	0.01\\
23.88	0.01\\
23.89	0.01\\
23.9	0.01\\
23.91	0.01\\
23.92	0.01\\
23.93	0.01\\
23.94	0.01\\
23.95	0.01\\
23.96	0.01\\
23.97	0.01\\
23.98	0.01\\
23.99	0.01\\
24	0.01\\
24.01	0.01\\
24.02	0.01\\
24.03	0.01\\
24.04	0.01\\
24.05	0.01\\
24.06	0.01\\
24.07	0.01\\
24.08	0.01\\
24.09	0.01\\
24.1	0.01\\
24.11	0.01\\
24.12	0.01\\
24.13	0.01\\
24.14	0.01\\
24.15	0.01\\
24.16	0.01\\
24.17	0.01\\
24.18	0.01\\
24.19	0.01\\
24.2	0.01\\
24.21	0.01\\
24.22	0.01\\
24.23	0.01\\
24.24	0.01\\
24.25	0.01\\
24.26	0.01\\
24.27	0.01\\
24.28	0.01\\
24.29	0.01\\
24.3	0.01\\
24.31	0.01\\
24.32	0.01\\
24.33	0.01\\
24.34	0.01\\
24.35	0.01\\
24.36	0.01\\
24.37	0.01\\
24.38	0.01\\
24.39	0.01\\
24.4	0.01\\
24.41	0.01\\
24.42	0.01\\
24.43	0.01\\
24.44	0.01\\
24.45	0.01\\
24.46	0.01\\
24.47	0.01\\
24.48	0.01\\
24.49	0.01\\
24.5	0.01\\
24.51	0.01\\
24.52	0.01\\
24.53	0.01\\
24.54	0.01\\
24.55	0.01\\
24.56	0.01\\
24.57	0.01\\
24.58	0.01\\
24.59	0.01\\
24.6	0.01\\
24.61	0.01\\
24.62	0.01\\
24.63	0.01\\
24.64	0.01\\
24.65	0.01\\
24.66	0.01\\
24.67	0.01\\
24.68	0.01\\
24.69	0.01\\
24.7	0.01\\
24.71	0.01\\
24.72	0.01\\
24.73	0.01\\
24.74	0.01\\
24.75	0.01\\
24.76	0.01\\
24.77	0.01\\
24.78	0.01\\
24.79	0.01\\
24.8	0.01\\
24.81	0.01\\
24.82	0.01\\
24.83	0.01\\
24.84	0.01\\
24.85	0.01\\
24.86	0.01\\
24.87	0.01\\
24.88	0.01\\
24.89	0.01\\
24.9	0.01\\
24.91	0.01\\
24.92	0.01\\
24.93	0.01\\
24.94	0.01\\
24.95	0.01\\
24.96	0.01\\
24.97	0.01\\
24.98	0.01\\
24.99	0.01\\
25	0.01\\
25.01	0.01\\
25.02	0.01\\
25.03	0.01\\
25.04	0.01\\
25.05	0.01\\
25.06	0.01\\
25.07	0.01\\
25.08	0.01\\
25.09	0.01\\
25.1	0.01\\
25.11	0.01\\
25.12	0.01\\
25.13	0.01\\
25.14	0.01\\
25.15	0.01\\
25.16	0.01\\
25.17	0.01\\
25.18	0.01\\
25.19	0.01\\
25.2	0.01\\
25.21	0.01\\
25.22	0.01\\
25.23	0.01\\
25.24	0.01\\
25.25	0.01\\
25.26	0.01\\
25.27	0.01\\
25.28	0.01\\
25.29	0.01\\
25.3	0.01\\
25.31	0.01\\
25.32	0.01\\
25.33	0.01\\
25.34	0.01\\
25.35	0.01\\
25.36	0.01\\
25.37	0.01\\
25.38	0.01\\
25.39	0.01\\
25.4	0.01\\
25.41	0.01\\
25.42	0.01\\
25.43	0.01\\
25.44	0.01\\
25.45	0.01\\
25.46	0.01\\
25.47	0.01\\
25.48	0.01\\
25.49	0.01\\
25.5	0.01\\
25.51	0.01\\
25.52	0.01\\
25.53	0.01\\
25.54	0.01\\
25.55	0.01\\
25.56	0.01\\
25.57	0.01\\
25.58	0.01\\
25.59	0.01\\
25.6	0.01\\
25.61	0.01\\
25.62	0.01\\
25.63	0.01\\
25.64	0.01\\
25.65	0.01\\
25.66	0.01\\
25.67	0.01\\
25.68	0.01\\
25.69	0.01\\
25.7	0.01\\
25.71	0.01\\
25.72	0.01\\
25.73	0.01\\
25.74	0.01\\
25.75	0.01\\
25.76	0.01\\
25.77	0.01\\
25.78	0.01\\
25.79	0.01\\
25.8	0.01\\
25.81	0.01\\
25.82	0.01\\
25.83	0.01\\
25.84	0.01\\
25.85	0.01\\
25.86	0.01\\
25.87	0.01\\
25.88	0.01\\
25.89	0.01\\
25.9	0.01\\
25.91	0.01\\
25.92	0.01\\
25.93	0.01\\
25.94	0.01\\
25.95	0.01\\
25.96	0.01\\
25.97	0.01\\
25.98	0.01\\
25.99	0.01\\
26	0.01\\
26.01	0.01\\
26.02	0.01\\
26.03	0.01\\
26.04	0.01\\
26.05	0.01\\
26.06	0.01\\
26.07	0.01\\
26.08	0.01\\
26.09	0.01\\
26.1	0.01\\
26.11	0.01\\
26.12	0.01\\
26.13	0.01\\
26.14	0.01\\
26.15	0.01\\
26.16	0.01\\
26.17	0.01\\
26.18	0.01\\
26.19	0.01\\
26.2	0.01\\
26.21	0.01\\
26.22	0.01\\
26.23	0.01\\
26.24	0.01\\
26.25	0.01\\
26.26	0.01\\
26.27	0.01\\
26.28	0.01\\
26.29	0.01\\
26.3	0.01\\
26.31	0.01\\
26.32	0.01\\
26.33	0.01\\
26.34	0.01\\
26.35	0.01\\
26.36	0.01\\
26.37	0.01\\
26.38	0.01\\
26.39	0.01\\
26.4	0.01\\
26.41	0.01\\
26.42	0.01\\
26.43	0.01\\
26.44	0.01\\
26.45	0.01\\
26.46	0.01\\
26.47	0.01\\
26.48	0.01\\
26.49	0.01\\
26.5	0.01\\
26.51	0.01\\
26.52	0.01\\
26.53	0.01\\
26.54	0.01\\
26.55	0.01\\
26.56	0.01\\
26.57	0.01\\
26.58	0.01\\
26.59	0.01\\
26.6	0.01\\
26.61	0.01\\
26.62	0.01\\
26.63	0.01\\
26.64	0.01\\
26.65	0.01\\
26.66	0.01\\
26.67	0.01\\
26.68	0.01\\
26.69	0.01\\
26.7	0.01\\
26.71	0.01\\
26.72	0.01\\
26.73	0.01\\
26.74	0.01\\
26.75	0.01\\
26.76	0.01\\
26.77	0.01\\
26.78	0.01\\
26.79	0.01\\
26.8	0.01\\
26.81	0.01\\
26.82	0.01\\
26.83	0.01\\
26.84	0.01\\
26.85	0.01\\
26.86	0.01\\
26.87	0.01\\
26.88	0.01\\
26.89	0.01\\
26.9	0.01\\
26.91	0.01\\
26.92	0.01\\
26.93	0.01\\
26.94	0.01\\
26.95	0.01\\
26.96	0.01\\
26.97	0.01\\
26.98	0.01\\
26.99	0.01\\
27	0.01\\
27.01	0.01\\
27.02	0.01\\
27.03	0.01\\
27.04	0.01\\
27.05	0.01\\
27.06	0.01\\
27.07	0.01\\
27.08	0.01\\
27.09	0.01\\
27.1	0.01\\
27.11	0.01\\
27.12	0.01\\
27.13	0.01\\
27.14	0.01\\
27.15	0.01\\
27.16	0.01\\
27.17	0.01\\
27.18	0.01\\
27.19	0.01\\
27.2	0.01\\
27.21	0.01\\
27.22	0.01\\
27.23	0.01\\
27.24	0.01\\
27.25	0.01\\
27.26	0.01\\
27.27	0.01\\
27.28	0.01\\
27.29	0.01\\
27.3	0.01\\
27.31	0.01\\
27.32	0.01\\
27.33	0.01\\
27.34	0.01\\
27.35	0.01\\
27.36	0.01\\
27.37	0.01\\
27.38	0.01\\
27.39	0.01\\
27.4	0.01\\
27.41	0.01\\
27.42	0.01\\
27.43	0.01\\
27.44	0.01\\
27.45	0.01\\
27.46	0.01\\
27.47	0.01\\
27.48	0.01\\
27.49	0.01\\
27.5	0.01\\
27.51	0.01\\
27.52	0.01\\
27.53	0.01\\
27.54	0.01\\
27.55	0.01\\
27.56	0.01\\
27.57	0.01\\
27.58	0.01\\
27.59	0.01\\
27.6	0.01\\
27.61	0.01\\
27.62	0.01\\
27.63	0.01\\
27.64	0.01\\
27.65	0.01\\
27.66	0.01\\
27.67	0.01\\
27.68	0.01\\
27.69	0.01\\
27.7	0.01\\
27.71	0.01\\
27.72	0.01\\
27.73	0.01\\
27.74	0.01\\
27.75	0.01\\
27.76	0.01\\
27.77	0.01\\
27.78	0.01\\
27.79	0.01\\
27.8	0.01\\
27.81	0.01\\
27.82	0.01\\
27.83	0.01\\
27.84	0.01\\
27.85	0.01\\
27.86	0.01\\
27.87	0.01\\
27.88	0.01\\
27.89	0.01\\
27.9	0.01\\
27.91	0.01\\
27.92	0.01\\
27.93	0.01\\
27.94	0.01\\
27.95	0.01\\
27.96	0.01\\
27.97	0.01\\
27.98	0.01\\
27.99	0.01\\
28	0.01\\
28.01	0.01\\
28.02	0.01\\
28.03	0.01\\
28.04	0.01\\
28.05	0.01\\
28.06	0.01\\
28.07	0.01\\
28.08	0.01\\
28.09	0.01\\
28.1	0.01\\
28.11	0.01\\
28.12	0.01\\
28.13	0.01\\
28.14	0.01\\
28.15	0.01\\
28.16	0.01\\
28.17	0.01\\
28.18	0.01\\
28.19	0.01\\
28.2	0.01\\
28.21	0.01\\
28.22	0.01\\
28.23	0.01\\
28.24	0.01\\
28.25	0.01\\
28.26	0.01\\
28.27	0.01\\
28.28	0.01\\
28.29	0.01\\
28.3	0.01\\
28.31	0.01\\
28.32	0.01\\
28.33	0.01\\
28.34	0.01\\
28.35	0.01\\
28.36	0.01\\
28.37	0.01\\
28.38	0.01\\
28.39	0.01\\
28.4	0.01\\
28.41	0.01\\
28.42	0.01\\
28.43	0.01\\
28.44	0.01\\
28.45	0.01\\
28.46	0.01\\
28.47	0.01\\
28.48	0.01\\
28.49	0.01\\
28.5	0.01\\
28.51	0.01\\
28.52	0.01\\
28.53	0.01\\
28.54	0.01\\
28.55	0.01\\
28.56	0.01\\
28.57	0.01\\
28.58	0.01\\
28.59	0.01\\
28.6	0.01\\
28.61	0.01\\
28.62	0.01\\
28.63	0.01\\
28.64	0.01\\
28.65	0.01\\
28.66	0.01\\
28.67	0.01\\
28.68	0.01\\
28.69	0.01\\
28.7	0.01\\
28.71	0.01\\
28.72	0.01\\
28.73	0.01\\
28.74	0.01\\
28.75	0.01\\
28.76	0.01\\
28.77	0.01\\
28.78	0.01\\
28.79	0.01\\
28.8	0.01\\
28.81	0.01\\
28.82	0.01\\
28.83	0.01\\
28.84	0.01\\
28.85	0.01\\
28.86	0.01\\
28.87	0.01\\
28.88	0.01\\
28.89	0.01\\
28.9	0.01\\
28.91	0.01\\
28.92	0.01\\
28.93	0.01\\
28.94	0.01\\
28.95	0.01\\
28.96	0.01\\
28.97	0.01\\
28.98	0.01\\
28.99	0.01\\
29	0.01\\
29.01	0.01\\
29.02	0.01\\
29.03	0.01\\
29.04	0.01\\
29.05	0.01\\
29.06	0.01\\
29.07	0.01\\
29.08	0.01\\
29.09	0.01\\
29.1	0.01\\
29.11	0.01\\
29.12	0.01\\
29.13	0.01\\
29.14	0.01\\
29.15	0.01\\
29.16	0.01\\
29.17	0.01\\
29.18	0.01\\
29.19	0.01\\
29.2	0.01\\
29.21	0.01\\
29.22	0.01\\
29.23	0.01\\
29.24	0.01\\
29.25	0.01\\
29.26	0.01\\
29.27	0.01\\
29.28	0.01\\
29.29	0.01\\
29.3	0.01\\
29.31	0.01\\
29.32	0.01\\
29.33	0.01\\
29.34	0.01\\
29.35	0.01\\
29.36	0.01\\
29.37	0.01\\
29.38	0.01\\
29.39	0.01\\
29.4	0.01\\
29.41	0.01\\
29.42	0.01\\
29.43	0.01\\
29.44	0.01\\
29.45	0.01\\
29.46	0.01\\
29.47	0.01\\
29.48	0.01\\
29.49	0.01\\
29.5	0.01\\
29.51	0.01\\
29.52	0.01\\
29.53	0.01\\
29.54	0.01\\
29.55	0.01\\
29.56	0.01\\
29.57	0.01\\
29.58	0.01\\
29.59	0.01\\
29.6	0.01\\
29.61	0.01\\
29.62	0.01\\
29.63	0.01\\
29.64	0.01\\
29.65	0.01\\
29.66	0.01\\
29.67	0.01\\
29.68	0.01\\
29.69	0.01\\
29.7	0.01\\
29.71	0.01\\
29.72	0.01\\
29.73	0.01\\
29.74	0.01\\
29.75	0.01\\
29.76	0.01\\
29.77	0.01\\
29.78	0.01\\
29.79	0.01\\
29.8	0.01\\
29.81	0.01\\
29.82	0.01\\
29.83	0.01\\
29.84	0.01\\
29.85	0.01\\
29.86	0.01\\
29.87	0.01\\
29.88	0.01\\
29.89	0.01\\
29.9	0.01\\
29.91	0.01\\
29.92	0.01\\
29.93	0.01\\
29.94	0.01\\
29.95	0.01\\
29.96	0.01\\
29.97	0.01\\
29.98	0.01\\
29.99	0.01\\
30	0.01\\
30.01	0.01\\
30.02	0.01\\
30.03	0.01\\
30.04	0.01\\
30.05	0.01\\
30.06	0.01\\
30.07	0.01\\
30.08	0.01\\
30.09	0.01\\
30.1	0.01\\
30.11	0.01\\
30.12	0.01\\
30.13	0.01\\
30.14	0.01\\
30.15	0.01\\
30.16	0.01\\
30.17	0.01\\
30.18	0.01\\
30.19	0.01\\
30.2	0.01\\
30.21	0.01\\
30.22	0.01\\
30.23	0.01\\
30.24	0.01\\
30.25	0.01\\
30.26	0.01\\
30.27	0.01\\
30.28	0.01\\
30.29	0.01\\
30.3	0.01\\
30.31	0.01\\
30.32	0.01\\
30.33	0.01\\
30.34	0.01\\
30.35	0.01\\
30.36	0.01\\
30.37	0.01\\
30.38	0.01\\
30.39	0.01\\
30.4	0.01\\
30.41	0.01\\
30.42	0.01\\
30.43	0.01\\
30.44	0.01\\
30.45	0.01\\
30.46	0.01\\
30.47	0.01\\
30.48	0.01\\
30.49	0.01\\
30.5	0.01\\
30.51	0.01\\
30.52	0.01\\
30.53	0.01\\
30.54	0.01\\
30.55	0.01\\
30.56	0.01\\
30.57	0.01\\
30.58	0.01\\
30.59	0.01\\
30.6	0.01\\
30.61	0.01\\
30.62	0.01\\
30.63	0.01\\
30.64	0.01\\
30.65	0.01\\
30.66	0.01\\
30.67	0.01\\
30.68	0.01\\
30.69	0.01\\
30.7	0.01\\
30.71	0.01\\
30.72	0.01\\
30.73	0.01\\
30.74	0.01\\
30.75	0.01\\
30.76	0.01\\
30.77	0.01\\
30.78	0.01\\
30.79	0.01\\
30.8	0.01\\
30.81	0.01\\
30.82	0.01\\
30.83	0.01\\
30.84	0.01\\
30.85	0.01\\
30.86	0.01\\
30.87	0.01\\
30.88	0.01\\
30.89	0.01\\
30.9	0.01\\
30.91	0.01\\
30.92	0.01\\
30.93	0.01\\
30.94	0.01\\
30.95	0.01\\
30.96	0.01\\
30.97	0.01\\
30.98	0.01\\
30.99	0.01\\
31	0.01\\
31.01	0.01\\
31.02	0.01\\
31.03	0.01\\
31.04	0.01\\
31.05	0.01\\
31.06	0.01\\
31.07	0.01\\
31.08	0.01\\
31.09	0.01\\
31.1	0.01\\
31.11	0.01\\
31.12	0.01\\
31.13	0.01\\
31.14	0.01\\
31.15	0.01\\
31.16	0.01\\
31.17	0.01\\
31.18	0.01\\
31.19	0.01\\
31.2	0.01\\
31.21	0.01\\
31.22	0.01\\
31.23	0.01\\
31.24	0.01\\
31.25	0.01\\
31.26	0.01\\
31.27	0.01\\
31.28	0.01\\
31.29	0.01\\
31.3	0.01\\
31.31	0.01\\
31.32	0.01\\
31.33	0.01\\
31.34	0.01\\
31.35	0.01\\
31.36	0.01\\
31.37	0.01\\
31.38	0.01\\
31.39	0.01\\
31.4	0.01\\
31.41	0.01\\
31.42	0.01\\
31.43	0.01\\
31.44	0.01\\
31.45	0.01\\
31.46	0.01\\
31.47	0.01\\
31.48	0.01\\
31.49	0.01\\
31.5	0.01\\
31.51	0.01\\
31.52	0.01\\
31.53	0.01\\
31.54	0.01\\
31.55	0.01\\
31.56	0.01\\
31.57	0.01\\
31.58	0.01\\
31.59	0.01\\
31.6	0.01\\
31.61	0.01\\
31.62	0.01\\
31.63	0.01\\
31.64	0.01\\
31.65	0.01\\
31.66	0.01\\
31.67	0.01\\
31.68	0.01\\
31.69	0.01\\
31.7	0.01\\
31.71	0.01\\
31.72	0.01\\
31.73	0.01\\
31.74	0.01\\
31.75	0.01\\
31.76	0.01\\
31.77	0.01\\
31.78	0.01\\
31.79	0.01\\
31.8	0.01\\
31.81	0.01\\
31.82	0.01\\
31.83	0.01\\
31.84	0.01\\
31.85	0.01\\
31.86	0.01\\
31.87	0.01\\
31.88	0.01\\
31.89	0.01\\
31.9	0.01\\
31.91	0.01\\
31.92	0.01\\
31.93	0.01\\
31.94	0.01\\
31.95	0.01\\
31.96	0.01\\
31.97	0.01\\
31.98	0.01\\
31.99	0.01\\
32	0.01\\
32.01	0.01\\
32.02	0.01\\
32.03	0.01\\
32.04	0.01\\
32.05	0.01\\
32.06	0.01\\
32.07	0.01\\
32.08	0.01\\
32.09	0.01\\
32.1	0.01\\
32.11	0.01\\
32.12	0.01\\
32.13	0.01\\
32.14	0.01\\
32.15	0.01\\
32.16	0.01\\
32.17	0.01\\
32.18	0.01\\
32.19	0.01\\
32.2	0.01\\
32.21	0.01\\
32.22	0.01\\
32.23	0.01\\
32.24	0.01\\
32.25	0.01\\
32.26	0.01\\
32.27	0.01\\
32.28	0.01\\
32.29	0.01\\
32.3	0.01\\
32.31	0.01\\
32.32	0.01\\
32.33	0.01\\
32.34	0.01\\
32.35	0.01\\
32.36	0.01\\
32.37	0.01\\
32.38	0.01\\
32.39	0.01\\
32.4	0.01\\
32.41	0.01\\
32.42	0.01\\
32.43	0.01\\
32.44	0.01\\
32.45	0.01\\
32.46	0.01\\
32.47	0.01\\
32.48	0.01\\
32.49	0.01\\
32.5	0.01\\
32.51	0.01\\
32.52	0.01\\
32.53	0.01\\
32.54	0.01\\
32.55	0.01\\
32.56	0.01\\
32.57	0.01\\
32.58	0.01\\
32.59	0.01\\
32.6	0.01\\
32.61	0.01\\
32.62	0.01\\
32.63	0.01\\
32.64	0.01\\
32.65	0.01\\
32.66	0.01\\
32.67	0.01\\
32.68	0.01\\
32.69	0.01\\
32.7	0.01\\
32.71	0.01\\
32.72	0.01\\
32.73	0.01\\
32.74	0.01\\
32.75	0.01\\
32.76	0.01\\
32.77	0.01\\
32.78	0.01\\
32.79	0.01\\
32.8	0.01\\
32.81	0.01\\
32.82	0.01\\
32.83	0.01\\
32.84	0.01\\
32.85	0.01\\
32.86	0.01\\
32.87	0.01\\
32.88	0.01\\
32.89	0.01\\
32.9	0.01\\
32.91	0.01\\
32.92	0.01\\
32.93	0.01\\
32.94	0.01\\
32.95	0.01\\
32.96	0.01\\
32.97	0.01\\
32.98	0.01\\
32.99	0.01\\
33	0.01\\
33.01	0.01\\
33.02	0.01\\
33.03	0.01\\
33.04	0.01\\
33.05	0.01\\
33.06	0.01\\
33.07	0.01\\
33.08	0.01\\
33.09	0.01\\
33.1	0.01\\
33.11	0.01\\
33.12	0.01\\
33.13	0.01\\
33.14	0.01\\
33.15	0.01\\
33.16	0.01\\
33.17	0.01\\
33.18	0.01\\
33.19	0.01\\
33.2	0.01\\
33.21	0.01\\
33.22	0.01\\
33.23	0.01\\
33.24	0.01\\
33.25	0.01\\
33.26	0.01\\
33.27	0.01\\
33.28	0.01\\
33.29	0.01\\
33.3	0.01\\
33.31	0.01\\
33.32	0.01\\
33.33	0.01\\
33.34	0.01\\
33.35	0.01\\
33.36	0.01\\
33.37	0.01\\
33.38	0.01\\
33.39	0.01\\
33.4	0.01\\
33.41	0.01\\
33.42	0.01\\
33.43	0.01\\
33.44	0.01\\
33.45	0.01\\
33.46	0.01\\
33.47	0.01\\
33.48	0.01\\
33.49	0.01\\
33.5	0.01\\
33.51	0.01\\
33.52	0.01\\
33.53	0.01\\
33.54	0.01\\
33.55	0.01\\
33.56	0.01\\
33.57	0.01\\
33.58	0.01\\
33.59	0.01\\
33.6	0.01\\
33.61	0.01\\
33.62	0.01\\
33.63	0.01\\
33.64	0.01\\
33.65	0.01\\
33.66	0.01\\
33.67	0.01\\
33.68	0.01\\
33.69	0.01\\
33.7	0.01\\
33.71	0.01\\
33.72	0.01\\
33.73	0.01\\
33.74	0.01\\
33.75	0.01\\
33.76	0.01\\
33.77	0.01\\
33.78	0.01\\
33.79	0.01\\
33.8	0.01\\
33.81	0.01\\
33.82	0.01\\
33.83	0.01\\
33.84	0.01\\
33.85	0.01\\
33.86	0.01\\
33.87	0.01\\
33.88	0.01\\
33.89	0.01\\
33.9	0.01\\
33.91	0.01\\
33.92	0.01\\
33.93	0.01\\
33.94	0.01\\
33.95	0.01\\
33.96	0.01\\
33.97	0.01\\
33.98	0.01\\
33.99	0.01\\
34	0.01\\
34.01	0.01\\
34.02	0.01\\
34.03	0.01\\
34.04	0.01\\
34.05	0.01\\
34.06	0.01\\
34.07	0.01\\
34.08	0.01\\
34.09	0.01\\
34.1	0.01\\
34.11	0.01\\
34.12	0.01\\
34.13	0.01\\
34.14	0.01\\
34.15	0.01\\
34.16	0.01\\
34.17	0.01\\
34.18	0.01\\
34.19	0.01\\
34.2	0.01\\
34.21	0.01\\
34.22	0.01\\
34.23	0.01\\
34.24	0.01\\
34.25	0.01\\
34.26	0.01\\
34.27	0.01\\
34.28	0.01\\
34.29	0.01\\
34.3	0.01\\
34.31	0.01\\
34.32	0.01\\
34.33	0.01\\
34.34	0.01\\
34.35	0.01\\
34.36	0.01\\
34.37	0.01\\
34.38	0.01\\
34.39	0.01\\
34.4	0.01\\
34.41	0.01\\
34.42	0.01\\
34.43	0.01\\
34.44	0.01\\
34.45	0.01\\
34.46	0.01\\
34.47	0.01\\
34.48	0.01\\
34.49	0.01\\
34.5	0.01\\
34.51	0.01\\
34.52	0.01\\
34.53	0.01\\
34.54	0.01\\
34.55	0.01\\
34.56	0.01\\
34.57	0.01\\
34.58	0.01\\
34.59	0.01\\
34.6	0.01\\
34.61	0.01\\
34.62	0.01\\
34.63	0.01\\
34.64	0.01\\
34.65	0.01\\
34.66	0.01\\
34.67	0.01\\
34.68	0.01\\
34.69	0.01\\
34.7	0.01\\
34.71	0.01\\
34.72	0.01\\
34.73	0.01\\
34.74	0.01\\
34.75	0.01\\
34.76	0.01\\
34.77	0.01\\
34.78	0.01\\
34.79	0.01\\
34.8	0.01\\
34.81	0.01\\
34.82	0.01\\
34.83	0.01\\
34.84	0.01\\
34.85	0.01\\
34.86	0.01\\
34.87	0.01\\
34.88	0.01\\
34.89	0.01\\
34.9	0.01\\
34.91	0.01\\
34.92	0.01\\
34.93	0.01\\
34.94	0.01\\
34.95	0.01\\
34.96	0.01\\
34.97	0.01\\
34.98	0.01\\
34.99	0.01\\
35	0.01\\
35.01	0.01\\
35.02	0.01\\
35.03	0.01\\
35.04	0.01\\
35.05	0.01\\
35.06	0.01\\
35.07	0.01\\
35.08	0.01\\
35.09	0.01\\
35.1	0.01\\
35.11	0.01\\
35.12	0.01\\
35.13	0.01\\
35.14	0.01\\
35.15	0.01\\
35.16	0.01\\
35.17	0.01\\
35.18	0.01\\
35.19	0.01\\
35.2	0.01\\
35.21	0.01\\
35.22	0.01\\
35.23	0.01\\
35.24	0.01\\
35.25	0.01\\
35.26	0.01\\
35.27	0.01\\
35.28	0.01\\
35.29	0.01\\
35.3	0.01\\
35.31	0.01\\
35.32	0.01\\
35.33	0.01\\
35.34	0.01\\
35.35	0.01\\
35.36	0.01\\
35.37	0.01\\
35.38	0.01\\
35.39	0.01\\
35.4	0.01\\
35.41	0.01\\
35.42	0.01\\
35.43	0.01\\
35.44	0.01\\
35.45	0.01\\
35.46	0.01\\
35.47	0.01\\
35.48	0.01\\
35.49	0.01\\
35.5	0.01\\
35.51	0.01\\
35.52	0.01\\
35.53	0.01\\
35.54	0.01\\
35.55	0.01\\
35.56	0.01\\
35.57	0.01\\
35.58	0.01\\
35.59	0.01\\
35.6	0.01\\
35.61	0.01\\
35.62	0.01\\
35.63	0.01\\
35.64	0.01\\
35.65	0.01\\
35.66	0.01\\
35.67	0.01\\
35.68	0.01\\
35.69	0.01\\
35.7	0.01\\
35.71	0.01\\
35.72	0.01\\
35.73	0.01\\
35.74	0.01\\
35.75	0.01\\
35.76	0.01\\
35.77	0.01\\
35.78	0.01\\
35.79	0.01\\
35.8	0.01\\
35.81	0.01\\
35.82	0.01\\
35.83	0.01\\
35.84	0.01\\
35.85	0.01\\
35.86	0.01\\
35.87	0.01\\
35.88	0.01\\
35.89	0.01\\
35.9	0.01\\
35.91	0.01\\
35.92	0.01\\
35.93	0.01\\
35.94	0.01\\
35.95	0.01\\
35.96	0.01\\
35.97	0.01\\
35.98	0.01\\
35.99	0.01\\
36	0.01\\
36.01	0.01\\
36.02	0.01\\
36.03	0.01\\
36.04	0.01\\
36.05	0.01\\
36.06	0.01\\
36.07	0.01\\
36.08	0.01\\
36.09	0.01\\
36.1	0.01\\
36.11	0.01\\
36.12	0.01\\
36.13	0.01\\
36.14	0.01\\
36.15	0.01\\
36.16	0.01\\
36.17	0.01\\
36.18	0.01\\
36.19	0.01\\
36.2	0.01\\
36.21	0.01\\
36.22	0.01\\
36.23	0.01\\
36.24	0.01\\
36.25	0.01\\
36.26	0.01\\
36.27	0.01\\
36.28	0.01\\
36.29	0.01\\
36.3	0.01\\
36.31	0.01\\
36.32	0.01\\
36.33	0.01\\
36.34	0.01\\
36.35	0.01\\
36.36	0.01\\
36.37	0.01\\
36.38	0.01\\
36.39	0.01\\
36.4	0.01\\
36.41	0.01\\
36.42	0.01\\
36.43	0.01\\
36.44	0.01\\
36.45	0.01\\
36.46	0.01\\
36.47	0.01\\
36.48	0.01\\
36.49	0.01\\
36.5	0.01\\
36.51	0.01\\
36.52	0.01\\
36.53	0.01\\
36.54	0.01\\
36.55	0.01\\
36.56	0.01\\
36.57	0.01\\
36.58	0.01\\
36.59	0.01\\
36.6	0.01\\
36.61	0.01\\
36.62	0.01\\
36.63	0.01\\
36.64	0.01\\
36.65	0.01\\
36.66	0.01\\
36.67	0.01\\
36.68	0.01\\
36.69	0.01\\
36.7	0.01\\
36.71	0.01\\
36.72	0.01\\
36.73	0.01\\
36.74	0.01\\
36.75	0.01\\
36.76	0.01\\
36.77	0.01\\
36.78	0.01\\
36.79	0.01\\
36.8	0.01\\
36.81	0.01\\
36.82	0.01\\
36.83	0.01\\
36.84	0.01\\
36.85	0.01\\
36.86	0.01\\
36.87	0.01\\
36.88	0.01\\
36.89	0.01\\
36.9	0.01\\
36.91	0.01\\
36.92	0.01\\
36.93	0.01\\
36.94	0.01\\
36.95	0.01\\
36.96	0.01\\
36.97	0.01\\
36.98	0.01\\
36.99	0.01\\
37	0.01\\
37.01	0.01\\
37.02	0.01\\
37.03	0.01\\
37.04	0.01\\
37.05	0.01\\
37.06	0.01\\
37.07	0.01\\
37.08	0.01\\
37.09	0.01\\
37.1	0.01\\
37.11	0.01\\
37.12	0.01\\
37.13	0.01\\
37.14	0.01\\
37.15	0.01\\
37.16	0.01\\
37.17	0.01\\
37.18	0.01\\
37.19	0.01\\
37.2	0.01\\
37.21	0.01\\
37.22	0.01\\
37.23	0.01\\
37.24	0.01\\
37.25	0.01\\
37.26	0.01\\
37.27	0.01\\
37.28	0.01\\
37.29	0.01\\
37.3	0.01\\
37.31	0.01\\
37.32	0.01\\
37.33	0.01\\
37.34	0.01\\
37.35	0.01\\
37.36	0.01\\
37.37	0.01\\
37.38	0.01\\
37.39	0.01\\
37.4	0.01\\
37.41	0.01\\
37.42	0.01\\
37.43	0.01\\
37.44	0.01\\
37.45	0.01\\
37.46	0.01\\
37.47	0.01\\
37.48	0.01\\
37.49	0.01\\
37.5	0.01\\
37.51	0.01\\
37.52	0.01\\
37.53	0.01\\
37.54	0.01\\
37.55	0.01\\
37.56	0.01\\
37.57	0.01\\
37.58	0.01\\
37.59	0.01\\
37.6	0.01\\
37.61	0.01\\
37.62	0.01\\
37.63	0.01\\
37.64	0.01\\
37.65	0.01\\
37.66	0.01\\
37.67	0.01\\
37.68	0.01\\
37.69	0.01\\
37.7	0.01\\
37.71	0.01\\
37.72	0.01\\
37.73	0.01\\
37.74	0.01\\
37.75	0.01\\
37.76	0.01\\
37.77	0.01\\
37.78	0.01\\
37.79	0.01\\
37.8	0.01\\
37.81	0.01\\
37.82	0.01\\
37.83	0.01\\
37.84	0.01\\
37.85	0.01\\
37.86	0.01\\
37.87	0.01\\
37.88	0.01\\
37.89	0.01\\
37.9	0.01\\
37.91	0.01\\
37.92	0.01\\
37.93	0.01\\
37.94	0.01\\
37.95	0.01\\
37.96	0.01\\
37.97	0.01\\
37.98	0.01\\
37.99	0.01\\
38	0.01\\
38.01	0.01\\
38.02	0.01\\
38.03	0.01\\
38.04	0.01\\
38.05	0.01\\
38.06	0.01\\
38.07	0.01\\
38.08	0.01\\
38.09	0.01\\
38.1	0.01\\
38.11	0.01\\
38.12	0.01\\
38.13	0.01\\
38.14	0.01\\
38.15	0.01\\
38.16	0.01\\
38.17	0.01\\
38.18	0.01\\
38.19	0.01\\
38.2	0.01\\
38.21	0.01\\
38.22	0.01\\
38.23	0.01\\
38.24	0.01\\
38.25	0.01\\
38.26	0.01\\
38.27	0.01\\
38.28	0.01\\
38.29	0.01\\
38.3	0.01\\
38.31	0.01\\
38.32	0.01\\
38.33	0.01\\
38.34	0.01\\
38.35	0.01\\
38.36	0.01\\
38.37	0.01\\
38.38	0.01\\
38.39	0.01\\
38.4	0.01\\
38.41	0.01\\
38.42	0.01\\
38.43	0.01\\
38.44	0.01\\
38.45	0.01\\
38.46	0.01\\
38.47	0.01\\
38.48	0.01\\
38.49	0.01\\
38.5	0.01\\
38.51	0.01\\
38.52	0.01\\
38.53	0.01\\
38.54	0.01\\
38.55	0.01\\
38.56	0.01\\
38.57	0.01\\
38.58	0.01\\
38.59	0.01\\
38.6	0.01\\
38.61	0.01\\
38.62	0.01\\
38.63	0.01\\
38.64	0.01\\
38.65	0.01\\
38.66	0.01\\
38.67	0.01\\
38.68	0.01\\
38.69	0.01\\
38.7	0.01\\
38.71	0.01\\
38.72	0.01\\
38.73	0.01\\
38.74	0.01\\
38.75	0.01\\
38.76	0.01\\
38.77	0.01\\
38.78	0.01\\
38.79	0.01\\
38.8	0.01\\
38.81	0.01\\
38.82	0.01\\
38.83	0.01\\
38.84	0.01\\
38.85	0.01\\
38.86	0.01\\
38.87	0.01\\
38.88	0.01\\
38.89	0.01\\
38.9	0.01\\
38.91	0.01\\
38.92	0.01\\
38.93	0.01\\
38.94	0.01\\
38.95	0.01\\
38.96	0.01\\
38.97	0.01\\
38.98	0.01\\
38.99	0.01\\
39	0.01\\
39.01	0.01\\
39.02	0.01\\
39.03	0.01\\
39.04	0.01\\
39.05	0.01\\
39.06	0.01\\
39.07	0.01\\
39.08	0.01\\
39.09	0.01\\
39.1	0.01\\
39.11	0.01\\
39.12	0.01\\
39.13	0.01\\
39.14	0.01\\
39.15	0.01\\
39.16	0.01\\
39.17	0.01\\
39.18	0.01\\
39.19	0.01\\
39.2	0.01\\
39.21	0.01\\
39.22	0.01\\
39.23	0.01\\
39.24	0.01\\
39.25	0.01\\
39.26	0.01\\
39.27	0.01\\
39.28	0.01\\
39.29	0.01\\
39.3	0.01\\
39.31	0.01\\
39.32	0.01\\
39.33	0.01\\
39.34	0.01\\
39.35	0.01\\
39.36	0.01\\
39.37	0.01\\
39.38	0.01\\
39.39	0.01\\
39.4	0.01\\
39.41	0.01\\
39.42	0.01\\
39.43	0.01\\
39.44	0.01\\
39.45	0.01\\
39.46	0.01\\
39.47	0.01\\
39.48	0.01\\
39.49	0.01\\
39.5	0.01\\
39.51	0.01\\
39.52	0.01\\
39.53	0.01\\
39.54	0.01\\
39.55	0.01\\
39.56	0.01\\
39.57	0.01\\
39.58	0.01\\
39.59	0.01\\
39.6	0.01\\
39.61	0.01\\
39.62	0.01\\
39.63	0.01\\
39.64	0.01\\
39.65	0.01\\
39.66	0.01\\
39.67	0.01\\
39.68	0.01\\
39.69	0.01\\
39.7	0.01\\
39.71	0.01\\
39.72	0.01\\
39.73	0.01\\
39.74	0.01\\
39.75	0.01\\
39.76	0.01\\
39.77	0.01\\
39.78	0.01\\
39.79	0.01\\
39.8	0.01\\
39.81	0.01\\
39.82	0.01\\
39.83	0.01\\
39.84	0.01\\
39.85	0.01\\
39.86	0.01\\
39.87	0.01\\
39.88	0.01\\
39.89	0.01\\
39.9	0.01\\
39.91	0.01\\
39.92	0.01\\
39.93	0.01\\
39.94	0.01\\
39.95	0.01\\
39.96	0.01\\
39.97	0.01\\
39.98	0.01\\
39.99	0.01\\
40	0.01\\
40.01	0.01\\
};
\addplot [color=blue,solid,forget plot]
  table[row sep=crcr]{%
40.01	0.01\\
40.02	0.01\\
40.03	0.01\\
40.04	0.01\\
40.05	0.01\\
40.06	0.01\\
40.07	0.01\\
40.08	0.01\\
40.09	0.01\\
40.1	0.01\\
40.11	0.01\\
40.12	0.01\\
40.13	0.01\\
40.14	0.01\\
40.15	0.01\\
40.16	0.01\\
40.17	0.01\\
40.18	0.01\\
40.19	0.01\\
40.2	0.01\\
40.21	0.01\\
40.22	0.01\\
40.23	0.01\\
40.24	0.01\\
40.25	0.01\\
40.26	0.01\\
40.27	0.01\\
40.28	0.01\\
40.29	0.01\\
40.3	0.01\\
40.31	0.01\\
40.32	0.01\\
40.33	0.01\\
40.34	0.01\\
40.35	0.01\\
40.36	0.01\\
40.37	0.01\\
40.38	0.01\\
40.39	0.01\\
40.4	0.01\\
40.41	0.01\\
40.42	0.01\\
40.43	0.01\\
40.44	0.01\\
40.45	0.01\\
40.46	0.01\\
40.47	0.01\\
40.48	0.01\\
40.49	0.01\\
40.5	0.01\\
40.51	0.01\\
40.52	0.01\\
40.53	0.01\\
40.54	0.01\\
40.55	0.01\\
40.56	0.01\\
40.57	0.01\\
40.58	0.01\\
40.59	0.01\\
40.6	0.01\\
40.61	0.01\\
40.62	0.01\\
40.63	0.01\\
40.64	0.01\\
40.65	0.01\\
40.66	0.01\\
40.67	0.01\\
40.68	0.01\\
40.69	0.01\\
40.7	0.01\\
40.71	0.01\\
40.72	0.01\\
40.73	0.01\\
40.74	0.01\\
40.75	0.01\\
40.76	0.01\\
40.77	0.01\\
40.78	0.01\\
40.79	0.01\\
40.8	0.01\\
40.81	0.01\\
40.82	0.01\\
40.83	0.01\\
40.84	0.01\\
40.85	0.01\\
40.86	0.01\\
40.87	0.01\\
40.88	0.01\\
40.89	0.01\\
40.9	0.01\\
40.91	0.01\\
40.92	0.01\\
40.93	0.01\\
40.94	0.01\\
40.95	0.01\\
40.96	0.01\\
40.97	0.01\\
40.98	0.01\\
40.99	0.01\\
41	0.01\\
41.01	0.01\\
41.02	0.01\\
41.03	0.01\\
41.04	0.01\\
41.05	0.01\\
41.06	0.01\\
41.07	0.01\\
41.08	0.01\\
41.09	0.01\\
41.1	0.01\\
41.11	0.01\\
41.12	0.01\\
41.13	0.01\\
41.14	0.01\\
41.15	0.01\\
41.16	0.01\\
41.17	0.01\\
41.18	0.01\\
41.19	0.01\\
41.2	0.01\\
41.21	0.01\\
41.22	0.01\\
41.23	0.01\\
41.24	0.01\\
41.25	0.01\\
41.26	0.01\\
41.27	0.01\\
41.28	0.01\\
41.29	0.01\\
41.3	0.01\\
41.31	0.01\\
41.32	0.01\\
41.33	0.01\\
41.34	0.01\\
41.35	0.01\\
41.36	0.01\\
41.37	0.01\\
41.38	0.01\\
41.39	0.01\\
41.4	0.01\\
41.41	0.01\\
41.42	0.01\\
41.43	0.01\\
41.44	0.01\\
41.45	0.01\\
41.46	0.01\\
41.47	0.01\\
41.48	0.01\\
41.49	0.01\\
41.5	0.01\\
41.51	0.01\\
41.52	0.01\\
41.53	0.01\\
41.54	0.01\\
41.55	0.01\\
41.56	0.01\\
41.57	0.01\\
41.58	0.01\\
41.59	0.01\\
41.6	0.01\\
41.61	0.01\\
41.62	0.01\\
41.63	0.01\\
41.64	0.01\\
41.65	0.01\\
41.66	0.01\\
41.67	0.01\\
41.68	0.01\\
41.69	0.01\\
41.7	0.01\\
41.71	0.01\\
41.72	0.01\\
41.73	0.01\\
41.74	0.01\\
41.75	0.01\\
41.76	0.01\\
41.77	0.01\\
41.78	0.01\\
41.79	0.01\\
41.8	0.01\\
41.81	0.01\\
41.82	0.01\\
41.83	0.01\\
41.84	0.01\\
41.85	0.01\\
41.86	0.01\\
41.87	0.01\\
41.88	0.01\\
41.89	0.01\\
41.9	0.01\\
41.91	0.01\\
41.92	0.01\\
41.93	0.01\\
41.94	0.01\\
41.95	0.01\\
41.96	0.01\\
41.97	0.01\\
41.98	0.01\\
41.99	0.01\\
42	0.01\\
42.01	0.01\\
42.02	0.01\\
42.03	0.01\\
42.04	0.01\\
42.05	0.01\\
42.06	0.01\\
42.07	0.01\\
42.08	0.01\\
42.09	0.01\\
42.1	0.01\\
42.11	0.01\\
42.12	0.01\\
42.13	0.01\\
42.14	0.01\\
42.15	0.01\\
42.16	0.01\\
42.17	0.01\\
42.18	0.01\\
42.19	0.01\\
42.2	0.01\\
42.21	0.01\\
42.22	0.01\\
42.23	0.01\\
42.24	0.01\\
42.25	0.01\\
42.26	0.01\\
42.27	0.01\\
42.28	0.01\\
42.29	0.01\\
42.3	0.01\\
42.31	0.01\\
42.32	0.01\\
42.33	0.01\\
42.34	0.01\\
42.35	0.01\\
42.36	0.01\\
42.37	0.01\\
42.38	0.01\\
42.39	0.01\\
42.4	0.01\\
42.41	0.01\\
42.42	0.01\\
42.43	0.01\\
42.44	0.01\\
42.45	0.01\\
42.46	0.01\\
42.47	0.01\\
42.48	0.01\\
42.49	0.01\\
42.5	0.01\\
42.51	0.01\\
42.52	0.01\\
42.53	0.01\\
42.54	0.01\\
42.55	0.01\\
42.56	0.01\\
42.57	0.01\\
42.58	0.01\\
42.59	0.01\\
42.6	0.01\\
42.61	0.01\\
42.62	0.01\\
42.63	0.01\\
42.64	0.01\\
42.65	0.01\\
42.66	0.01\\
42.67	0.01\\
42.68	0.01\\
42.69	0.01\\
42.7	0.01\\
42.71	0.01\\
42.72	0.01\\
42.73	0.01\\
42.74	0.01\\
42.75	0.01\\
42.76	0.01\\
42.77	0.01\\
42.78	0.01\\
42.79	0.01\\
42.8	0.01\\
42.81	0.01\\
42.82	0.01\\
42.83	0.01\\
42.84	0.01\\
42.85	0.01\\
42.86	0.01\\
42.87	0.01\\
42.88	0.01\\
42.89	0.01\\
42.9	0.01\\
42.91	0.01\\
42.92	0.01\\
42.93	0.01\\
42.94	0.01\\
42.95	0.01\\
42.96	0.01\\
42.97	0.01\\
42.98	0.01\\
42.99	0.01\\
43	0.01\\
43.01	0.01\\
43.02	0.01\\
43.03	0.01\\
43.04	0.01\\
43.05	0.01\\
43.06	0.01\\
43.07	0.01\\
43.08	0.01\\
43.09	0.01\\
43.1	0.01\\
43.11	0.01\\
43.12	0.01\\
43.13	0.01\\
43.14	0.01\\
43.15	0.01\\
43.16	0.01\\
43.17	0.01\\
43.18	0.01\\
43.19	0.01\\
43.2	0.01\\
43.21	0.01\\
43.22	0.01\\
43.23	0.01\\
43.24	0.01\\
43.25	0.01\\
43.26	0.01\\
43.27	0.01\\
43.28	0.01\\
43.29	0.01\\
43.3	0.01\\
43.31	0.01\\
43.32	0.01\\
43.33	0.01\\
43.34	0.01\\
43.35	0.01\\
43.36	0.01\\
43.37	0.01\\
43.38	0.01\\
43.39	0.01\\
43.4	0.01\\
43.41	0.01\\
43.42	0.01\\
43.43	0.01\\
43.44	0.01\\
43.45	0.01\\
43.46	0.01\\
43.47	0.01\\
43.48	0.01\\
43.49	0.01\\
43.5	0.01\\
43.51	0.01\\
43.52	0.01\\
43.53	0.01\\
43.54	0.01\\
43.55	0.01\\
43.56	0.01\\
43.57	0.01\\
43.58	0.01\\
43.59	0.01\\
43.6	0.01\\
43.61	0.01\\
43.62	0.01\\
43.63	0.01\\
43.64	0.01\\
43.65	0.01\\
43.66	0.01\\
43.67	0.01\\
43.68	0.01\\
43.69	0.01\\
43.7	0.01\\
43.71	0.01\\
43.72	0.01\\
43.73	0.01\\
43.74	0.01\\
43.75	0.01\\
43.76	0.01\\
43.77	0.01\\
43.78	0.01\\
43.79	0.01\\
43.8	0.01\\
43.81	0.01\\
43.82	0.01\\
43.83	0.01\\
43.84	0.01\\
43.85	0.01\\
43.86	0.01\\
43.87	0.01\\
43.88	0.01\\
43.89	0.01\\
43.9	0.01\\
43.91	0.01\\
43.92	0.01\\
43.93	0.01\\
43.94	0.01\\
43.95	0.01\\
43.96	0.01\\
43.97	0.01\\
43.98	0.01\\
43.99	0.01\\
44	0.01\\
44.01	0.01\\
44.02	0.01\\
44.03	0.01\\
44.04	0.01\\
44.05	0.01\\
44.06	0.01\\
44.07	0.01\\
44.08	0.01\\
44.09	0.01\\
44.1	0.01\\
44.11	0.01\\
44.12	0.01\\
44.13	0.01\\
44.14	0.01\\
44.15	0.01\\
44.16	0.01\\
44.17	0.01\\
44.18	0.01\\
44.19	0.01\\
44.2	0.01\\
44.21	0.01\\
44.22	0.01\\
44.23	0.01\\
44.24	0.01\\
44.25	0.01\\
44.26	0.01\\
44.27	0.01\\
44.28	0.01\\
44.29	0.01\\
44.3	0.01\\
44.31	0.01\\
44.32	0.01\\
44.33	0.01\\
44.34	0.01\\
44.35	0.01\\
44.36	0.01\\
44.37	0.01\\
44.38	0.01\\
44.39	0.01\\
44.4	0.01\\
44.41	0.01\\
44.42	0.01\\
44.43	0.01\\
44.44	0.01\\
44.45	0.01\\
44.46	0.01\\
44.47	0.01\\
44.48	0.01\\
44.49	0.01\\
44.5	0.01\\
44.51	0.01\\
44.52	0.01\\
44.53	0.01\\
44.54	0.01\\
44.55	0.01\\
44.56	0.01\\
44.57	0.01\\
44.58	0.01\\
44.59	0.01\\
44.6	0.01\\
44.61	0.01\\
44.62	0.01\\
44.63	0.01\\
44.64	0.01\\
44.65	0.01\\
44.66	0.01\\
44.67	0.01\\
44.68	0.01\\
44.69	0.01\\
44.7	0.01\\
44.71	0.01\\
44.72	0.01\\
44.73	0.01\\
44.74	0.01\\
44.75	0.01\\
44.76	0.01\\
44.77	0.01\\
44.78	0.01\\
44.79	0.01\\
44.8	0.01\\
44.81	0.01\\
44.82	0.01\\
44.83	0.01\\
44.84	0.01\\
44.85	0.01\\
44.86	0.01\\
44.87	0.01\\
44.88	0.01\\
44.89	0.01\\
44.9	0.01\\
44.91	0.01\\
44.92	0.01\\
44.93	0.01\\
44.94	0.01\\
44.95	0.01\\
44.96	0.01\\
44.97	0.01\\
44.98	0.01\\
44.99	0.01\\
45	0.01\\
45.01	0.01\\
45.02	0.01\\
45.03	0.01\\
45.04	0.01\\
45.05	0.01\\
45.06	0.01\\
45.07	0.01\\
45.08	0.01\\
45.09	0.01\\
45.1	0.01\\
45.11	0.01\\
45.12	0.01\\
45.13	0.01\\
45.14	0.01\\
45.15	0.01\\
45.16	0.01\\
45.17	0.01\\
45.18	0.01\\
45.19	0.01\\
45.2	0.01\\
45.21	0.01\\
45.22	0.01\\
45.23	0.01\\
45.24	0.01\\
45.25	0.01\\
45.26	0.01\\
45.27	0.01\\
45.28	0.01\\
45.29	0.01\\
45.3	0.01\\
45.31	0.01\\
45.32	0.01\\
45.33	0.01\\
45.34	0.01\\
45.35	0.01\\
45.36	0.01\\
45.37	0.01\\
45.38	0.01\\
45.39	0.01\\
45.4	0.01\\
45.41	0.01\\
45.42	0.01\\
45.43	0.01\\
45.44	0.01\\
45.45	0.01\\
45.46	0.01\\
45.47	0.01\\
45.48	0.01\\
45.49	0.01\\
45.5	0.01\\
45.51	0.01\\
45.52	0.01\\
45.53	0.01\\
45.54	0.01\\
45.55	0.01\\
45.56	0.01\\
45.57	0.01\\
45.58	0.01\\
45.59	0.01\\
45.6	0.01\\
45.61	0.01\\
45.62	0.01\\
45.63	0.01\\
45.64	0.01\\
45.65	0.01\\
45.66	0.01\\
45.67	0.01\\
45.68	0.01\\
45.69	0.01\\
45.7	0.01\\
45.71	0.01\\
45.72	0.01\\
45.73	0.01\\
45.74	0.01\\
45.75	0.01\\
45.76	0.01\\
45.77	0.01\\
45.78	0.01\\
45.79	0.01\\
45.8	0.01\\
45.81	0.01\\
45.82	0.01\\
45.83	0.01\\
45.84	0.01\\
45.85	0.01\\
45.86	0.01\\
45.87	0.01\\
45.88	0.01\\
45.89	0.01\\
45.9	0.01\\
45.91	0.01\\
45.92	0.01\\
45.93	0.01\\
45.94	0.01\\
45.95	0.01\\
45.96	0.01\\
45.97	0.01\\
45.98	0.01\\
45.99	0.01\\
46	0.01\\
46.01	0.01\\
46.02	0.01\\
46.03	0.01\\
46.04	0.01\\
46.05	0.01\\
46.06	0.01\\
46.07	0.01\\
46.08	0.01\\
46.09	0.01\\
46.1	0.01\\
46.11	0.01\\
46.12	0.01\\
46.13	0.01\\
46.14	0.01\\
46.15	0.01\\
46.16	0.01\\
46.17	0.01\\
46.18	0.01\\
46.19	0.01\\
46.2	0.01\\
46.21	0.01\\
46.22	0.01\\
46.23	0.01\\
46.24	0.01\\
46.25	0.01\\
46.26	0.01\\
46.27	0.01\\
46.28	0.01\\
46.29	0.01\\
46.3	0.01\\
46.31	0.01\\
46.32	0.01\\
46.33	0.01\\
46.34	0.01\\
46.35	0.01\\
46.36	0.01\\
46.37	0.01\\
46.38	0.01\\
46.39	0.01\\
46.4	0.01\\
46.41	0.01\\
46.42	0.01\\
46.43	0.01\\
46.44	0.01\\
46.45	0.01\\
46.46	0.01\\
46.47	0.01\\
46.48	0.01\\
46.49	0.01\\
46.5	0.01\\
46.51	0.01\\
46.52	0.01\\
46.53	0.01\\
46.54	0.01\\
46.55	0.01\\
46.56	0.01\\
46.57	0.01\\
46.58	0.01\\
46.59	0.01\\
46.6	0.01\\
46.61	0.01\\
46.62	0.01\\
46.63	0.01\\
46.64	0.01\\
46.65	0.01\\
46.66	0.01\\
46.67	0.01\\
46.68	0.01\\
46.69	0.01\\
46.7	0.01\\
46.71	0.01\\
46.72	0.01\\
46.73	0.01\\
46.74	0.01\\
46.75	0.01\\
46.76	0.01\\
46.77	0.01\\
46.78	0.01\\
46.79	0.01\\
46.8	0.01\\
46.81	0.01\\
46.82	0.01\\
46.83	0.01\\
46.84	0.01\\
46.85	0.01\\
46.86	0.01\\
46.87	0.01\\
46.88	0.01\\
46.89	0.01\\
46.9	0.01\\
46.91	0.01\\
46.92	0.01\\
46.93	0.01\\
46.94	0.01\\
46.95	0.01\\
46.96	0.01\\
46.97	0.01\\
46.98	0.01\\
46.99	0.01\\
47	0.01\\
47.01	0.01\\
47.02	0.01\\
47.03	0.01\\
47.04	0.01\\
47.05	0.01\\
47.06	0.01\\
47.07	0.01\\
47.08	0.01\\
47.09	0.01\\
47.1	0.01\\
47.11	0.01\\
47.12	0.01\\
47.13	0.01\\
47.14	0.01\\
47.15	0.01\\
47.16	0.01\\
47.17	0.01\\
47.18	0.01\\
47.19	0.01\\
47.2	0.01\\
47.21	0.01\\
47.22	0.01\\
47.23	0.01\\
47.24	0.01\\
47.25	0.01\\
47.26	0.01\\
47.27	0.01\\
47.28	0.01\\
47.29	0.01\\
47.3	0.01\\
47.31	0.01\\
47.32	0.01\\
47.33	0.01\\
47.34	0.01\\
47.35	0.01\\
47.36	0.01\\
47.37	0.01\\
47.38	0.01\\
47.39	0.01\\
47.4	0.01\\
47.41	0.01\\
47.42	0.01\\
47.43	0.01\\
47.44	0.01\\
47.45	0.01\\
47.46	0.01\\
47.47	0.01\\
47.48	0.01\\
47.49	0.01\\
47.5	0.01\\
47.51	0.01\\
47.52	0.01\\
47.53	0.01\\
47.54	0.01\\
47.55	0.01\\
47.56	0.01\\
47.57	0.01\\
47.58	0.01\\
47.59	0.01\\
47.6	0.01\\
47.61	0.01\\
47.62	0.01\\
47.63	0.01\\
47.64	0.01\\
47.65	0.01\\
47.66	0.01\\
47.67	0.01\\
47.68	0.01\\
47.69	0.01\\
47.7	0.01\\
47.71	0.01\\
47.72	0.01\\
47.73	0.01\\
47.74	0.01\\
47.75	0.01\\
47.76	0.01\\
47.77	0.01\\
47.78	0.01\\
47.79	0.01\\
47.8	0.01\\
47.81	0.01\\
47.82	0.01\\
47.83	0.01\\
47.84	0.01\\
47.85	0.01\\
47.86	0.01\\
47.87	0.01\\
47.88	0.01\\
47.89	0.01\\
47.9	0.01\\
47.91	0.01\\
47.92	0.01\\
47.93	0.01\\
47.94	0.01\\
47.95	0.01\\
47.96	0.01\\
47.97	0.01\\
47.98	0.01\\
47.99	0.01\\
48	0.01\\
48.01	0.01\\
48.02	0.01\\
48.03	0.01\\
48.04	0.01\\
48.05	0.01\\
48.06	0.01\\
48.07	0.01\\
48.08	0.01\\
48.09	0.01\\
48.1	0.01\\
48.11	0.01\\
48.12	0.01\\
48.13	0.01\\
48.14	0.01\\
48.15	0.01\\
48.16	0.01\\
48.17	0.01\\
48.18	0.01\\
48.19	0.01\\
48.2	0.01\\
48.21	0.01\\
48.22	0.01\\
48.23	0.01\\
48.24	0.01\\
48.25	0.01\\
48.26	0.01\\
48.27	0.01\\
48.28	0.01\\
48.29	0.01\\
48.3	0.01\\
48.31	0.01\\
48.32	0.01\\
48.33	0.01\\
48.34	0.01\\
48.35	0.01\\
48.36	0.01\\
48.37	0.01\\
48.38	0.01\\
48.39	0.01\\
48.4	0.01\\
48.41	0.01\\
48.42	0.01\\
48.43	0.01\\
48.44	0.01\\
48.45	0.01\\
48.46	0.01\\
48.47	0.01\\
48.48	0.01\\
48.49	0.01\\
48.5	0.01\\
48.51	0.01\\
48.52	0.01\\
48.53	0.01\\
48.54	0.01\\
48.55	0.01\\
48.56	0.01\\
48.57	0.01\\
48.58	0.01\\
48.59	0.01\\
48.6	0.01\\
48.61	0.01\\
48.62	0.01\\
48.63	0.01\\
48.64	0.01\\
48.65	0.01\\
48.66	0.01\\
48.67	0.01\\
48.68	0.01\\
48.69	0.01\\
48.7	0.01\\
48.71	0.01\\
48.72	0.01\\
48.73	0.01\\
48.74	0.01\\
48.75	0.01\\
48.76	0.01\\
48.77	0.01\\
48.78	0.01\\
48.79	0.01\\
48.8	0.01\\
48.81	0.01\\
48.82	0.01\\
48.83	0.01\\
48.84	0.01\\
48.85	0.01\\
48.86	0.01\\
48.87	0.01\\
48.88	0.01\\
48.89	0.01\\
48.9	0.01\\
48.91	0.01\\
48.92	0.01\\
48.93	0.01\\
48.94	0.01\\
48.95	0.01\\
48.96	0.01\\
48.97	0.01\\
48.98	0.01\\
48.99	0.01\\
49	0.01\\
49.01	0.01\\
49.02	0.01\\
49.03	0.01\\
49.04	0.01\\
49.05	0.01\\
49.06	0.01\\
49.07	0.01\\
49.08	0.01\\
49.09	0.01\\
49.1	0.01\\
49.11	0.01\\
49.12	0.01\\
49.13	0.01\\
49.14	0.01\\
49.15	0.01\\
49.16	0.01\\
49.17	0.01\\
49.18	0.01\\
49.19	0.01\\
49.2	0.01\\
49.21	0.01\\
49.22	0.01\\
49.23	0.01\\
49.24	0.01\\
49.25	0.01\\
49.26	0.01\\
49.27	0.01\\
49.28	0.01\\
49.29	0.01\\
49.3	0.01\\
49.31	0.01\\
49.32	0.01\\
49.33	0.01\\
49.34	0.01\\
49.35	0.01\\
49.36	0.01\\
49.37	0.01\\
49.38	0.01\\
49.39	0.01\\
49.4	0.01\\
49.41	0.01\\
49.42	0.01\\
49.43	0.01\\
49.44	0.01\\
49.45	0.01\\
49.46	0.01\\
49.47	0.01\\
49.48	0.01\\
49.49	0.01\\
49.5	0.01\\
49.51	0.01\\
49.52	0.01\\
49.53	0.01\\
49.54	0.01\\
49.55	0.01\\
49.56	0.01\\
49.57	0.01\\
49.58	0.01\\
49.59	0.01\\
49.6	0.01\\
49.61	0.01\\
49.62	0.01\\
49.63	0.01\\
49.64	0.01\\
49.65	0.01\\
49.66	0.01\\
49.67	0.01\\
49.68	0.01\\
49.69	0.01\\
49.7	0.01\\
49.71	0.01\\
49.72	0.01\\
49.73	0.01\\
49.74	0.01\\
49.75	0.01\\
49.76	0.01\\
49.77	0.01\\
49.78	0.01\\
49.79	0.01\\
49.8	0.01\\
49.81	0.01\\
49.82	0.01\\
49.83	0.01\\
49.84	0.01\\
49.85	0.01\\
49.86	0.01\\
49.87	0.01\\
49.88	0.01\\
49.89	0.01\\
49.9	0.01\\
49.91	0.01\\
49.92	0.01\\
49.93	0.01\\
49.94	0.01\\
49.95	0.01\\
49.96	0.01\\
49.97	0.01\\
49.98	0.01\\
49.99	0.01\\
50	0.01\\
50.01	0.01\\
50.02	0.01\\
50.03	0.01\\
50.04	0.01\\
50.05	0.01\\
50.06	0.01\\
50.07	0.01\\
50.08	0.01\\
50.09	0.01\\
50.1	0.01\\
50.11	0.01\\
50.12	0.01\\
50.13	0.01\\
50.14	0.01\\
50.15	0.01\\
50.16	0.01\\
50.17	0.01\\
50.18	0.01\\
50.19	0.01\\
50.2	0.01\\
50.21	0.01\\
50.22	0.01\\
50.23	0.01\\
50.24	0.01\\
50.25	0.01\\
50.26	0.01\\
50.27	0.01\\
50.28	0.01\\
50.29	0.01\\
50.3	0.01\\
50.31	0.01\\
50.32	0.01\\
50.33	0.01\\
50.34	0.01\\
50.35	0.01\\
50.36	0.01\\
50.37	0.01\\
50.38	0.01\\
50.39	0.01\\
50.4	0.01\\
50.41	0.01\\
50.42	0.01\\
50.43	0.01\\
50.44	0.01\\
50.45	0.01\\
50.46	0.01\\
50.47	0.01\\
50.48	0.01\\
50.49	0.01\\
50.5	0.01\\
50.51	0.01\\
50.52	0.01\\
50.53	0.01\\
50.54	0.01\\
50.55	0.01\\
50.56	0.01\\
50.57	0.01\\
50.58	0.01\\
50.59	0.01\\
50.6	0.01\\
50.61	0.01\\
50.62	0.01\\
50.63	0.01\\
50.64	0.01\\
50.65	0.01\\
50.66	0.01\\
50.67	0.01\\
50.68	0.01\\
50.69	0.01\\
50.7	0.01\\
50.71	0.01\\
50.72	0.01\\
50.73	0.01\\
50.74	0.01\\
50.75	0.01\\
50.76	0.01\\
50.77	0.01\\
50.78	0.01\\
50.79	0.01\\
50.8	0.01\\
50.81	0.01\\
50.82	0.01\\
50.83	0.01\\
50.84	0.01\\
50.85	0.01\\
50.86	0.01\\
50.87	0.01\\
50.88	0.01\\
50.89	0.01\\
50.9	0.01\\
50.91	0.01\\
50.92	0.01\\
50.93	0.01\\
50.94	0.01\\
50.95	0.01\\
50.96	0.01\\
50.97	0.01\\
50.98	0.01\\
50.99	0.01\\
51	0.01\\
51.01	0.01\\
51.02	0.01\\
51.03	0.01\\
51.04	0.01\\
51.05	0.01\\
51.06	0.01\\
51.07	0.01\\
51.08	0.01\\
51.09	0.01\\
51.1	0.01\\
51.11	0.01\\
51.12	0.01\\
51.13	0.01\\
51.14	0.01\\
51.15	0.01\\
51.16	0.01\\
51.17	0.01\\
51.18	0.01\\
51.19	0.01\\
51.2	0.01\\
51.21	0.01\\
51.22	0.01\\
51.23	0.01\\
51.24	0.01\\
51.25	0.01\\
51.26	0.01\\
51.27	0.01\\
51.28	0.01\\
51.29	0.01\\
51.3	0.01\\
51.31	0.01\\
51.32	0.01\\
51.33	0.01\\
51.34	0.01\\
51.35	0.01\\
51.36	0.01\\
51.37	0.01\\
51.38	0.01\\
51.39	0.01\\
51.4	0.01\\
51.41	0.01\\
51.42	0.01\\
51.43	0.01\\
51.44	0.01\\
51.45	0.01\\
51.46	0.01\\
51.47	0.01\\
51.48	0.01\\
51.49	0.01\\
51.5	0.01\\
51.51	0.01\\
51.52	0.01\\
51.53	0.01\\
51.54	0.01\\
51.55	0.01\\
51.56	0.01\\
51.57	0.01\\
51.58	0.01\\
51.59	0.01\\
51.6	0.01\\
51.61	0.01\\
51.62	0.01\\
51.63	0.01\\
51.64	0.01\\
51.65	0.01\\
51.66	0.01\\
51.67	0.01\\
51.68	0.01\\
51.69	0.01\\
51.7	0.01\\
51.71	0.01\\
51.72	0.01\\
51.73	0.01\\
51.74	0.01\\
51.75	0.01\\
51.76	0.01\\
51.77	0.01\\
51.78	0.01\\
51.79	0.01\\
51.8	0.01\\
51.81	0.01\\
51.82	0.01\\
51.83	0.01\\
51.84	0.01\\
51.85	0.01\\
51.86	0.01\\
51.87	0.01\\
51.88	0.01\\
51.89	0.01\\
51.9	0.01\\
51.91	0.01\\
51.92	0.01\\
51.93	0.01\\
51.94	0.01\\
51.95	0.01\\
51.96	0.01\\
51.97	0.01\\
51.98	0.01\\
51.99	0.01\\
52	0.01\\
52.01	0.01\\
52.02	0.01\\
52.03	0.01\\
52.04	0.01\\
52.05	0.01\\
52.06	0.01\\
52.07	0.01\\
52.08	0.01\\
52.09	0.01\\
52.1	0.01\\
52.11	0.01\\
52.12	0.01\\
52.13	0.01\\
52.14	0.01\\
52.15	0.01\\
52.16	0.01\\
52.17	0.01\\
52.18	0.01\\
52.19	0.01\\
52.2	0.01\\
52.21	0.01\\
52.22	0.01\\
52.23	0.01\\
52.24	0.01\\
52.25	0.01\\
52.26	0.01\\
52.27	0.01\\
52.28	0.01\\
52.29	0.01\\
52.3	0.01\\
52.31	0.01\\
52.32	0.01\\
52.33	0.01\\
52.34	0.01\\
52.35	0.01\\
52.36	0.01\\
52.37	0.01\\
52.38	0.01\\
52.39	0.01\\
52.4	0.01\\
52.41	0.01\\
52.42	0.01\\
52.43	0.01\\
52.44	0.01\\
52.45	0.01\\
52.46	0.01\\
52.47	0.01\\
52.48	0.01\\
52.49	0.01\\
52.5	0.01\\
52.51	0.01\\
52.52	0.01\\
52.53	0.01\\
52.54	0.01\\
52.55	0.01\\
52.56	0.01\\
52.57	0.01\\
52.58	0.01\\
52.59	0.01\\
52.6	0.01\\
52.61	0.01\\
52.62	0.01\\
52.63	0.01\\
52.64	0.01\\
52.65	0.01\\
52.66	0.01\\
52.67	0.01\\
52.68	0.01\\
52.69	0.01\\
52.7	0.01\\
52.71	0.01\\
52.72	0.01\\
52.73	0.01\\
52.74	0.01\\
52.75	0.01\\
52.76	0.01\\
52.77	0.01\\
52.78	0.01\\
52.79	0.01\\
52.8	0.01\\
52.81	0.01\\
52.82	0.01\\
52.83	0.01\\
52.84	0.01\\
52.85	0.01\\
52.86	0.01\\
52.87	0.01\\
52.88	0.01\\
52.89	0.01\\
52.9	0.01\\
52.91	0.01\\
52.92	0.01\\
52.93	0.01\\
52.94	0.01\\
52.95	0.01\\
52.96	0.01\\
52.97	0.01\\
52.98	0.01\\
52.99	0.01\\
53	0.01\\
53.01	0.01\\
53.02	0.01\\
53.03	0.01\\
53.04	0.01\\
53.05	0.01\\
53.06	0.01\\
53.07	0.01\\
53.08	0.01\\
53.09	0.01\\
53.1	0.01\\
53.11	0.01\\
53.12	0.01\\
53.13	0.01\\
53.14	0.01\\
53.15	0.01\\
53.16	0.01\\
53.17	0.01\\
53.18	0.01\\
53.19	0.01\\
53.2	0.01\\
53.21	0.01\\
53.22	0.01\\
53.23	0.01\\
53.24	0.01\\
53.25	0.01\\
53.26	0.01\\
53.27	0.01\\
53.28	0.01\\
53.29	0.01\\
53.3	0.01\\
53.31	0.01\\
53.32	0.01\\
53.33	0.01\\
53.34	0.01\\
53.35	0.01\\
53.36	0.01\\
53.37	0.01\\
53.38	0.01\\
53.39	0.01\\
53.4	0.01\\
53.41	0.01\\
53.42	0.01\\
53.43	0.01\\
53.44	0.01\\
53.45	0.01\\
53.46	0.01\\
53.47	0.01\\
53.48	0.01\\
53.49	0.01\\
53.5	0.01\\
53.51	0.01\\
53.52	0.01\\
53.53	0.01\\
53.54	0.01\\
53.55	0.01\\
53.56	0.01\\
53.57	0.01\\
53.58	0.01\\
53.59	0.01\\
53.6	0.01\\
53.61	0.01\\
53.62	0.01\\
53.63	0.01\\
53.64	0.01\\
53.65	0.01\\
53.66	0.01\\
53.67	0.01\\
53.68	0.01\\
53.69	0.01\\
53.7	0.01\\
53.71	0.01\\
53.72	0.01\\
53.73	0.01\\
53.74	0.01\\
53.75	0.01\\
53.76	0.01\\
53.77	0.01\\
53.78	0.01\\
53.79	0.01\\
53.8	0.01\\
53.81	0.01\\
53.82	0.01\\
53.83	0.01\\
53.84	0.01\\
53.85	0.01\\
53.86	0.01\\
53.87	0.01\\
53.88	0.01\\
53.89	0.01\\
53.9	0.01\\
53.91	0.01\\
53.92	0.01\\
53.93	0.01\\
53.94	0.01\\
53.95	0.01\\
53.96	0.01\\
53.97	0.01\\
53.98	0.01\\
53.99	0.01\\
54	0.01\\
54.01	0.01\\
54.02	0.01\\
54.03	0.01\\
54.04	0.01\\
54.05	0.01\\
54.06	0.01\\
54.07	0.01\\
54.08	0.01\\
54.09	0.01\\
54.1	0.01\\
54.11	0.01\\
54.12	0.01\\
54.13	0.01\\
54.14	0.01\\
54.15	0.01\\
54.16	0.01\\
54.17	0.01\\
54.18	0.01\\
54.19	0.01\\
54.2	0.01\\
54.21	0.01\\
54.22	0.01\\
54.23	0.01\\
54.24	0.01\\
54.25	0.01\\
54.26	0.01\\
54.27	0.01\\
54.28	0.01\\
54.29	0.01\\
54.3	0.01\\
54.31	0.01\\
54.32	0.01\\
54.33	0.01\\
54.34	0.01\\
54.35	0.01\\
54.36	0.01\\
54.37	0.01\\
54.38	0.01\\
54.39	0.01\\
54.4	0.01\\
54.41	0.01\\
54.42	0.01\\
54.43	0.01\\
54.44	0.01\\
54.45	0.01\\
54.46	0.01\\
54.47	0.01\\
54.48	0.01\\
54.49	0.01\\
54.5	0.01\\
54.51	0.01\\
54.52	0.01\\
54.53	0.01\\
54.54	0.01\\
54.55	0.01\\
54.56	0.01\\
54.57	0.01\\
54.58	0.01\\
54.59	0.01\\
54.6	0.01\\
54.61	0.01\\
54.62	0.01\\
54.63	0.01\\
54.64	0.01\\
54.65	0.01\\
54.66	0.01\\
54.67	0.01\\
54.68	0.01\\
54.69	0.01\\
54.7	0.01\\
54.71	0.01\\
54.72	0.01\\
54.73	0.01\\
54.74	0.01\\
54.75	0.01\\
54.76	0.01\\
54.77	0.01\\
54.78	0.01\\
54.79	0.01\\
54.8	0.01\\
54.81	0.01\\
54.82	0.01\\
54.83	0.01\\
54.84	0.01\\
54.85	0.01\\
54.86	0.01\\
54.87	0.01\\
54.88	0.01\\
54.89	0.01\\
54.9	0.01\\
54.91	0.01\\
54.92	0.01\\
54.93	0.01\\
54.94	0.01\\
54.95	0.01\\
54.96	0.01\\
54.97	0.01\\
54.98	0.01\\
54.99	0.01\\
55	0.01\\
55.01	0.01\\
55.02	0.01\\
55.03	0.01\\
55.04	0.01\\
55.05	0.01\\
55.06	0.01\\
55.07	0.01\\
55.08	0.01\\
55.09	0.01\\
55.1	0.01\\
55.11	0.01\\
55.12	0.01\\
55.13	0.01\\
55.14	0.01\\
55.15	0.01\\
55.16	0.01\\
55.17	0.01\\
55.18	0.01\\
55.19	0.01\\
55.2	0.01\\
55.21	0.01\\
55.22	0.01\\
55.23	0.01\\
55.24	0.01\\
55.25	0.01\\
55.26	0.01\\
55.27	0.01\\
55.28	0.01\\
55.29	0.01\\
55.3	0.01\\
55.31	0.01\\
55.32	0.01\\
55.33	0.01\\
55.34	0.01\\
55.35	0.01\\
55.36	0.01\\
55.37	0.01\\
55.38	0.01\\
55.39	0.01\\
55.4	0.01\\
55.41	0.01\\
55.42	0.01\\
55.43	0.01\\
55.44	0.01\\
55.45	0.01\\
55.46	0.01\\
55.47	0.01\\
55.48	0.01\\
55.49	0.01\\
55.5	0.01\\
55.51	0.01\\
55.52	0.01\\
55.53	0.01\\
55.54	0.01\\
55.55	0.01\\
55.56	0.01\\
55.57	0.01\\
55.58	0.01\\
55.59	0.01\\
55.6	0.01\\
55.61	0.01\\
55.62	0.01\\
55.63	0.01\\
55.64	0.01\\
55.65	0.01\\
55.66	0.01\\
55.67	0.01\\
55.68	0.01\\
55.69	0.01\\
55.7	0.01\\
55.71	0.01\\
55.72	0.01\\
55.73	0.01\\
55.74	0.01\\
55.75	0.01\\
55.76	0.01\\
55.77	0.01\\
55.78	0.01\\
55.79	0.01\\
55.8	0.01\\
55.81	0.01\\
55.82	0.01\\
55.83	0.01\\
55.84	0.01\\
55.85	0.01\\
55.86	0.01\\
55.87	0.01\\
55.88	0.01\\
55.89	0.01\\
55.9	0.01\\
55.91	0.01\\
55.92	0.01\\
55.93	0.01\\
55.94	0.01\\
55.95	0.01\\
55.96	0.01\\
55.97	0.01\\
55.98	0.01\\
55.99	0.01\\
56	0.01\\
56.01	0.01\\
56.02	0.01\\
56.03	0.01\\
56.04	0.01\\
56.05	0.01\\
56.06	0.01\\
56.07	0.01\\
56.08	0.01\\
56.09	0.01\\
56.1	0.01\\
56.11	0.01\\
56.12	0.01\\
56.13	0.01\\
56.14	0.01\\
56.15	0.01\\
56.16	0.01\\
56.17	0.01\\
56.18	0.01\\
56.19	0.01\\
56.2	0.01\\
56.21	0.01\\
56.22	0.01\\
56.23	0.01\\
56.24	0.01\\
56.25	0.01\\
56.26	0.01\\
56.27	0.01\\
56.28	0.01\\
56.29	0.01\\
56.3	0.01\\
56.31	0.01\\
56.32	0.01\\
56.33	0.01\\
56.34	0.01\\
56.35	0.01\\
56.36	0.01\\
56.37	0.01\\
56.38	0.01\\
56.39	0.01\\
56.4	0.01\\
56.41	0.01\\
56.42	0.01\\
56.43	0.01\\
56.44	0.01\\
56.45	0.01\\
56.46	0.01\\
56.47	0.01\\
56.48	0.01\\
56.49	0.01\\
56.5	0.01\\
56.51	0.01\\
56.52	0.01\\
56.53	0.01\\
56.54	0.01\\
56.55	0.01\\
56.56	0.01\\
56.57	0.01\\
56.58	0.01\\
56.59	0.01\\
56.6	0.01\\
56.61	0.01\\
56.62	0.01\\
56.63	0.01\\
56.64	0.01\\
56.65	0.01\\
56.66	0.01\\
56.67	0.01\\
56.68	0.01\\
56.69	0.01\\
56.7	0.01\\
56.71	0.01\\
56.72	0.01\\
56.73	0.01\\
56.74	0.01\\
56.75	0.01\\
56.76	0.01\\
56.77	0.01\\
56.78	0.01\\
56.79	0.01\\
56.8	0.01\\
56.81	0.01\\
56.82	0.01\\
56.83	0.01\\
56.84	0.01\\
56.85	0.01\\
56.86	0.01\\
56.87	0.01\\
56.88	0.01\\
56.89	0.01\\
56.9	0.01\\
56.91	0.01\\
56.92	0.01\\
56.93	0.01\\
56.94	0.01\\
56.95	0.01\\
56.96	0.01\\
56.97	0.01\\
56.98	0.01\\
56.99	0.01\\
57	0.01\\
57.01	0.01\\
57.02	0.01\\
57.03	0.01\\
57.04	0.01\\
57.05	0.01\\
57.06	0.01\\
57.07	0.01\\
57.08	0.01\\
57.09	0.01\\
57.1	0.01\\
57.11	0.01\\
57.12	0.01\\
57.13	0.01\\
57.14	0.01\\
57.15	0.01\\
57.16	0.01\\
57.17	0.01\\
57.18	0.01\\
57.19	0.01\\
57.2	0.01\\
57.21	0.01\\
57.22	0.01\\
57.23	0.01\\
57.24	0.01\\
57.25	0.01\\
57.26	0.01\\
57.27	0.01\\
57.28	0.01\\
57.29	0.01\\
57.3	0.01\\
57.31	0.01\\
57.32	0.01\\
57.33	0.01\\
57.34	0.01\\
57.35	0.01\\
57.36	0.01\\
57.37	0.01\\
57.38	0.01\\
57.39	0.01\\
57.4	0.01\\
57.41	0.01\\
57.42	0.01\\
57.43	0.01\\
57.44	0.01\\
57.45	0.01\\
57.46	0.01\\
57.47	0.01\\
57.48	0.01\\
57.49	0.01\\
57.5	0.01\\
57.51	0.01\\
57.52	0.01\\
57.53	0.01\\
57.54	0.01\\
57.55	0.01\\
57.56	0.01\\
57.57	0.01\\
57.58	0.01\\
57.59	0.01\\
57.6	0.01\\
57.61	0.01\\
57.62	0.01\\
57.63	0.01\\
57.64	0.01\\
57.65	0.01\\
57.66	0.01\\
57.67	0.01\\
57.68	0.01\\
57.69	0.01\\
57.7	0.01\\
57.71	0.01\\
57.72	0.01\\
57.73	0.01\\
57.74	0.01\\
57.75	0.01\\
57.76	0.01\\
57.77	0.01\\
57.78	0.01\\
57.79	0.01\\
57.8	0.01\\
57.81	0.01\\
57.82	0.01\\
57.83	0.01\\
57.84	0.01\\
57.85	0.01\\
57.86	0.01\\
57.87	0.01\\
57.88	0.01\\
57.89	0.01\\
57.9	0.01\\
57.91	0.01\\
57.92	0.01\\
57.93	0.01\\
57.94	0.01\\
57.95	0.01\\
57.96	0.01\\
57.97	0.01\\
57.98	0.01\\
57.99	0.01\\
58	0.01\\
58.01	0.01\\
58.02	0.01\\
58.03	0.01\\
58.04	0.01\\
58.05	0.01\\
58.06	0.01\\
58.07	0.01\\
58.08	0.01\\
58.09	0.01\\
58.1	0.01\\
58.11	0.01\\
58.12	0.01\\
58.13	0.01\\
58.14	0.01\\
58.15	0.01\\
58.16	0.01\\
58.17	0.01\\
58.18	0.01\\
58.19	0.01\\
58.2	0.01\\
58.21	0.01\\
58.22	0.01\\
58.23	0.01\\
58.24	0.01\\
58.25	0.01\\
58.26	0.01\\
58.27	0.01\\
58.28	0.01\\
58.29	0.01\\
58.3	0.01\\
58.31	0.01\\
58.32	0.01\\
58.33	0.01\\
58.34	0.01\\
58.35	0.01\\
58.36	0.01\\
58.37	0.01\\
58.38	0.01\\
58.39	0.01\\
58.4	0.01\\
58.41	0.01\\
58.42	0.01\\
58.43	0.01\\
58.44	0.01\\
58.45	0.01\\
58.46	0.01\\
58.47	0.01\\
58.48	0.01\\
58.49	0.01\\
58.5	0.01\\
58.51	0.01\\
58.52	0.01\\
58.53	0.01\\
58.54	0.01\\
58.55	0.01\\
58.56	0.01\\
58.57	0.01\\
58.58	0.01\\
58.59	0.01\\
58.6	0.01\\
58.61	0.01\\
58.62	0.01\\
58.63	0.01\\
58.64	0.01\\
58.65	0.01\\
58.66	0.01\\
58.67	0.01\\
58.68	0.01\\
58.69	0.01\\
58.7	0.01\\
58.71	0.01\\
58.72	0.01\\
58.73	0.01\\
58.74	0.01\\
58.75	0.01\\
58.76	0.01\\
58.77	0.01\\
58.78	0.01\\
58.79	0.01\\
58.8	0.01\\
58.81	0.01\\
58.82	0.01\\
58.83	0.01\\
58.84	0.01\\
58.85	0.01\\
58.86	0.01\\
58.87	0.01\\
58.88	0.01\\
58.89	0.01\\
58.9	0.01\\
58.91	0.01\\
58.92	0.01\\
58.93	0.01\\
58.94	0.01\\
58.95	0.01\\
58.96	0.01\\
58.97	0.01\\
58.98	0.01\\
58.99	0.01\\
59	0.01\\
59.01	0.01\\
59.02	0.01\\
59.03	0.01\\
59.04	0.01\\
59.05	0.01\\
59.06	0.01\\
59.07	0.01\\
59.08	0.01\\
59.09	0.01\\
59.1	0.01\\
59.11	0.01\\
59.12	0.01\\
59.13	0.01\\
59.14	0.01\\
59.15	0.01\\
59.16	0.01\\
59.17	0.01\\
59.18	0.01\\
59.19	0.01\\
59.2	0.01\\
59.21	0.01\\
59.22	0.01\\
59.23	0.01\\
59.24	0.01\\
59.25	0.01\\
59.26	0.01\\
59.27	0.01\\
59.28	0.01\\
59.29	0.01\\
59.3	0.01\\
59.31	0.01\\
59.32	0.01\\
59.33	0.01\\
59.34	0.01\\
59.35	0.01\\
59.36	0.01\\
59.37	0.01\\
59.38	0.01\\
59.39	0.01\\
59.4	0.01\\
59.41	0.01\\
59.42	0.01\\
59.43	0.01\\
59.44	0.01\\
59.45	0.01\\
59.46	0.01\\
59.47	0.01\\
59.48	0.01\\
59.49	0.01\\
59.5	0.01\\
59.51	0.01\\
59.52	0.01\\
59.53	0.01\\
59.54	0.01\\
59.55	0.01\\
59.56	0.01\\
59.57	0.01\\
59.58	0.01\\
59.59	0.01\\
59.6	0.01\\
59.61	0.01\\
59.62	0.01\\
59.63	0.01\\
59.64	0.01\\
59.65	0.01\\
59.66	0.01\\
59.67	0.01\\
59.68	0.01\\
59.69	0.01\\
59.7	0.01\\
59.71	0.01\\
59.72	0.01\\
59.73	0.01\\
59.74	0.01\\
59.75	0.01\\
59.76	0.01\\
59.77	0.01\\
59.78	0.01\\
59.79	0.01\\
59.8	0.01\\
59.81	0.01\\
59.82	0.01\\
59.83	0.01\\
59.84	0.01\\
59.85	0.01\\
59.86	0.01\\
59.87	0.01\\
59.88	0.01\\
59.89	0.01\\
59.9	0.01\\
59.91	0.01\\
59.92	0.01\\
59.93	0.01\\
59.94	0.01\\
59.95	0.01\\
59.96	0.01\\
59.97	0.01\\
59.98	0.01\\
59.99	0.01\\
60	0.01\\
60.01	0.01\\
60.02	0.01\\
60.03	0.01\\
60.04	0.01\\
60.05	0.01\\
60.06	0.01\\
60.07	0.01\\
60.08	0.01\\
60.09	0.01\\
60.1	0.01\\
60.11	0.01\\
60.12	0.01\\
60.13	0.01\\
60.14	0.01\\
60.15	0.01\\
60.16	0.01\\
60.17	0.01\\
60.18	0.01\\
60.19	0.01\\
60.2	0.01\\
60.21	0.01\\
60.22	0.01\\
60.23	0.01\\
60.24	0.01\\
60.25	0.01\\
60.26	0.01\\
60.27	0.01\\
60.28	0.01\\
60.29	0.01\\
60.3	0.01\\
60.31	0.01\\
60.32	0.01\\
60.33	0.01\\
60.34	0.01\\
60.35	0.01\\
60.36	0.01\\
60.37	0.01\\
60.38	0.01\\
60.39	0.01\\
60.4	0.01\\
60.41	0.01\\
60.42	0.01\\
60.43	0.01\\
60.44	0.01\\
60.45	0.01\\
60.46	0.01\\
60.47	0.01\\
60.48	0.01\\
60.49	0.01\\
60.5	0.01\\
60.51	0.01\\
60.52	0.01\\
60.53	0.01\\
60.54	0.01\\
60.55	0.01\\
60.56	0.01\\
60.57	0.01\\
60.58	0.01\\
60.59	0.01\\
60.6	0.01\\
60.61	0.01\\
60.62	0.01\\
60.63	0.01\\
60.64	0.01\\
60.65	0.01\\
60.66	0.01\\
60.67	0.01\\
60.68	0.01\\
60.69	0.01\\
60.7	0.01\\
60.71	0.01\\
60.72	0.01\\
60.73	0.01\\
60.74	0.01\\
60.75	0.01\\
60.76	0.01\\
60.77	0.01\\
60.78	0.01\\
60.79	0.01\\
60.8	0.01\\
60.81	0.01\\
60.82	0.01\\
60.83	0.01\\
60.84	0.01\\
60.85	0.01\\
60.86	0.01\\
60.87	0.01\\
60.88	0.01\\
60.89	0.01\\
60.9	0.01\\
60.91	0.01\\
60.92	0.01\\
60.93	0.01\\
60.94	0.01\\
60.95	0.01\\
60.96	0.01\\
60.97	0.01\\
60.98	0.01\\
60.99	0.01\\
61	0.01\\
61.01	0.01\\
61.02	0.01\\
61.03	0.01\\
61.04	0.01\\
61.05	0.01\\
61.06	0.01\\
61.07	0.01\\
61.08	0.01\\
61.09	0.01\\
61.1	0.01\\
61.11	0.01\\
61.12	0.01\\
61.13	0.01\\
61.14	0.01\\
61.15	0.01\\
61.16	0.01\\
61.17	0.01\\
61.18	0.01\\
61.19	0.01\\
61.2	0.01\\
61.21	0.01\\
61.22	0.01\\
61.23	0.01\\
61.24	0.01\\
61.25	0.01\\
61.26	0.01\\
61.27	0.01\\
61.28	0.01\\
61.29	0.01\\
61.3	0.01\\
61.31	0.01\\
61.32	0.01\\
61.33	0.01\\
61.34	0.01\\
61.35	0.01\\
61.36	0.01\\
61.37	0.01\\
61.38	0.01\\
61.39	0.01\\
61.4	0.01\\
61.41	0.01\\
61.42	0.01\\
61.43	0.01\\
61.44	0.01\\
61.45	0.01\\
61.46	0.01\\
61.47	0.01\\
61.48	0.01\\
61.49	0.01\\
61.5	0.01\\
61.51	0.01\\
61.52	0.01\\
61.53	0.01\\
61.54	0.01\\
61.55	0.01\\
61.56	0.01\\
61.57	0.01\\
61.58	0.01\\
61.59	0.01\\
61.6	0.01\\
61.61	0.01\\
61.62	0.01\\
61.63	0.01\\
61.64	0.01\\
61.65	0.01\\
61.66	0.01\\
61.67	0.01\\
61.68	0.01\\
61.69	0.01\\
61.7	0.01\\
61.71	0.01\\
61.72	0.01\\
61.73	0.01\\
61.74	0.01\\
61.75	0.01\\
61.76	0.01\\
61.77	0.01\\
61.78	0.01\\
61.79	0.01\\
61.8	0.01\\
61.81	0.01\\
61.82	0.01\\
61.83	0.01\\
61.84	0.01\\
61.85	0.01\\
61.86	0.01\\
61.87	0.01\\
61.88	0.01\\
61.89	0.01\\
61.9	0.01\\
61.91	0.01\\
61.92	0.01\\
61.93	0.01\\
61.94	0.01\\
61.95	0.01\\
61.96	0.01\\
61.97	0.01\\
61.98	0.01\\
61.99	0.01\\
62	0.01\\
62.01	0.01\\
62.02	0.01\\
62.03	0.01\\
62.04	0.01\\
62.05	0.01\\
62.06	0.01\\
62.07	0.01\\
62.08	0.01\\
62.09	0.01\\
62.1	0.01\\
62.11	0.01\\
62.12	0.01\\
62.13	0.01\\
62.14	0.01\\
62.15	0.01\\
62.16	0.01\\
62.17	0.01\\
62.18	0.01\\
62.19	0.01\\
62.2	0.01\\
62.21	0.01\\
62.22	0.01\\
62.23	0.01\\
62.24	0.01\\
62.25	0.01\\
62.26	0.01\\
62.27	0.01\\
62.28	0.01\\
62.29	0.01\\
62.3	0.01\\
62.31	0.01\\
62.32	0.01\\
62.33	0.01\\
62.34	0.01\\
62.35	0.01\\
62.36	0.01\\
62.37	0.01\\
62.38	0.01\\
62.39	0.01\\
62.4	0.01\\
62.41	0.01\\
62.42	0.01\\
62.43	0.01\\
62.44	0.01\\
62.45	0.01\\
62.46	0.01\\
62.47	0.01\\
62.48	0.01\\
62.49	0.01\\
62.5	0.01\\
62.51	0.01\\
62.52	0.01\\
62.53	0.01\\
62.54	0.01\\
62.55	0.01\\
62.56	0.01\\
62.57	0.01\\
62.58	0.01\\
62.59	0.01\\
62.6	0.01\\
62.61	0.01\\
62.62	0.01\\
62.63	0.01\\
62.64	0.01\\
62.65	0.01\\
62.66	0.01\\
62.67	0.01\\
62.68	0.01\\
62.69	0.01\\
62.7	0.01\\
62.71	0.01\\
62.72	0.01\\
62.73	0.01\\
62.74	0.01\\
62.75	0.01\\
62.76	0.01\\
62.77	0.01\\
62.78	0.01\\
62.79	0.01\\
62.8	0.01\\
62.81	0.01\\
62.82	0.01\\
62.83	0.01\\
62.84	0.01\\
62.85	0.01\\
62.86	0.01\\
62.87	0.01\\
62.88	0.01\\
62.89	0.01\\
62.9	0.01\\
62.91	0.01\\
62.92	0.01\\
62.93	0.01\\
62.94	0.01\\
62.95	0.01\\
62.96	0.01\\
62.97	0.01\\
62.98	0.01\\
62.99	0.01\\
63	0.01\\
63.01	0.01\\
63.02	0.01\\
63.03	0.01\\
63.04	0.01\\
63.05	0.01\\
63.06	0.01\\
63.07	0.01\\
63.08	0.01\\
63.09	0.01\\
63.1	0.01\\
63.11	0.01\\
63.12	0.01\\
63.13	0.01\\
63.14	0.01\\
63.15	0.01\\
63.16	0.01\\
63.17	0.01\\
63.18	0.01\\
63.19	0.01\\
63.2	0.01\\
63.21	0.01\\
63.22	0.01\\
63.23	0.01\\
63.24	0.01\\
63.25	0.01\\
63.26	0.01\\
63.27	0.01\\
63.28	0.01\\
63.29	0.01\\
63.3	0.01\\
63.31	0.01\\
63.32	0.01\\
63.33	0.01\\
63.34	0.01\\
63.35	0.01\\
63.36	0.01\\
63.37	0.01\\
63.38	0.01\\
63.39	0.01\\
63.4	0.01\\
63.41	0.01\\
63.42	0.01\\
63.43	0.01\\
63.44	0.01\\
63.45	0.01\\
63.46	0.01\\
63.47	0.01\\
63.48	0.01\\
63.49	0.01\\
63.5	0.01\\
63.51	0.01\\
63.52	0.01\\
63.53	0.01\\
63.54	0.01\\
63.55	0.01\\
63.56	0.01\\
63.57	0.01\\
63.58	0.01\\
63.59	0.01\\
63.6	0.01\\
63.61	0.01\\
63.62	0.01\\
63.63	0.01\\
63.64	0.01\\
63.65	0.01\\
63.66	0.01\\
63.67	0.01\\
63.68	0.01\\
63.69	0.01\\
63.7	0.01\\
63.71	0.01\\
63.72	0.01\\
63.73	0.01\\
63.74	0.01\\
63.75	0.01\\
63.76	0.01\\
63.77	0.01\\
63.78	0.01\\
63.79	0.01\\
63.8	0.01\\
63.81	0.01\\
63.82	0.01\\
63.83	0.01\\
63.84	0.01\\
63.85	0.01\\
63.86	0.01\\
63.87	0.01\\
63.88	0.01\\
63.89	0.01\\
63.9	0.01\\
63.91	0.01\\
63.92	0.01\\
63.93	0.01\\
63.94	0.01\\
63.95	0.01\\
63.96	0.01\\
63.97	0.01\\
63.98	0.01\\
63.99	0.01\\
64	0.01\\
64.01	0.01\\
64.02	0.01\\
64.03	0.01\\
64.04	0.01\\
64.05	0.01\\
64.06	0.01\\
64.07	0.01\\
64.08	0.01\\
64.09	0.01\\
64.1	0.01\\
64.11	0.01\\
64.12	0.01\\
64.13	0.01\\
64.14	0.01\\
64.15	0.01\\
64.16	0.01\\
64.17	0.01\\
64.18	0.01\\
64.19	0.01\\
64.2	0.01\\
64.21	0.01\\
64.22	0.01\\
64.23	0.01\\
64.24	0.01\\
64.25	0.01\\
64.26	0.01\\
64.27	0.01\\
64.28	0.01\\
64.29	0.01\\
64.3	0.01\\
64.31	0.01\\
64.32	0.01\\
64.33	0.01\\
64.34	0.01\\
64.35	0.01\\
64.36	0.01\\
64.37	0.01\\
64.38	0.01\\
64.39	0.01\\
64.4	0.01\\
64.41	0.01\\
64.42	0.01\\
64.43	0.01\\
64.44	0.01\\
64.45	0.01\\
64.46	0.01\\
64.47	0.01\\
64.48	0.01\\
64.49	0.01\\
64.5	0.01\\
64.51	0.01\\
64.52	0.01\\
64.53	0.01\\
64.54	0.01\\
64.55	0.01\\
64.56	0.01\\
64.57	0.01\\
64.58	0.01\\
64.59	0.01\\
64.6	0.01\\
64.61	0.01\\
64.62	0.01\\
64.63	0.01\\
64.64	0.01\\
64.65	0.01\\
64.66	0.01\\
64.67	0.01\\
64.68	0.01\\
64.69	0.01\\
64.7	0.01\\
64.71	0.01\\
64.72	0.01\\
64.73	0.01\\
64.74	0.01\\
64.75	0.01\\
64.76	0.01\\
64.77	0.01\\
64.78	0.01\\
64.79	0.01\\
64.8	0.01\\
64.81	0.01\\
64.82	0.01\\
64.83	0.01\\
64.84	0.01\\
64.85	0.01\\
64.86	0.01\\
64.87	0.01\\
64.88	0.01\\
64.89	0.01\\
64.9	0.01\\
64.91	0.01\\
64.92	0.01\\
64.93	0.01\\
64.94	0.01\\
64.95	0.01\\
64.96	0.01\\
64.97	0.01\\
64.98	0.01\\
64.99	0.01\\
65	0.01\\
65.01	0.01\\
65.02	0.01\\
65.03	0.01\\
65.04	0.01\\
65.05	0.01\\
65.06	0.01\\
65.07	0.01\\
65.08	0.01\\
65.09	0.01\\
65.1	0.01\\
65.11	0.01\\
65.12	0.01\\
65.13	0.01\\
65.14	0.01\\
65.15	0.01\\
65.16	0.01\\
65.17	0.01\\
65.18	0.01\\
65.19	0.01\\
65.2	0.01\\
65.21	0.01\\
65.22	0.01\\
65.23	0.01\\
65.24	0.01\\
65.25	0.01\\
65.26	0.01\\
65.27	0.01\\
65.28	0.01\\
65.29	0.01\\
65.3	0.01\\
65.31	0.01\\
65.32	0.01\\
65.33	0.01\\
65.34	0.01\\
65.35	0.01\\
65.36	0.01\\
65.37	0.01\\
65.38	0.01\\
65.39	0.01\\
65.4	0.01\\
65.41	0.01\\
65.42	0.01\\
65.43	0.01\\
65.44	0.01\\
65.45	0.01\\
65.46	0.01\\
65.47	0.01\\
65.48	0.01\\
65.49	0.01\\
65.5	0.01\\
65.51	0.01\\
65.52	0.01\\
65.53	0.01\\
65.54	0.01\\
65.55	0.01\\
65.56	0.01\\
65.57	0.01\\
65.58	0.01\\
65.59	0.01\\
65.6	0.01\\
65.61	0.01\\
65.62	0.01\\
65.63	0.01\\
65.64	0.01\\
65.65	0.01\\
65.66	0.01\\
65.67	0.01\\
65.68	0.01\\
65.69	0.01\\
65.7	0.01\\
65.71	0.01\\
65.72	0.01\\
65.73	0.01\\
65.74	0.01\\
65.75	0.01\\
65.76	0.01\\
65.77	0.01\\
65.78	0.01\\
65.79	0.01\\
65.8	0.01\\
65.81	0.01\\
65.82	0.01\\
65.83	0.01\\
65.84	0.01\\
65.85	0.01\\
65.86	0.01\\
65.87	0.01\\
65.88	0.01\\
65.89	0.01\\
65.9	0.01\\
65.91	0.01\\
65.92	0.01\\
65.93	0.01\\
65.94	0.01\\
65.95	0.01\\
65.96	0.01\\
65.97	0.01\\
65.98	0.01\\
65.99	0.01\\
66	0.01\\
66.01	0.01\\
66.02	0.01\\
66.03	0.01\\
66.04	0.01\\
66.05	0.01\\
66.06	0.01\\
66.07	0.01\\
66.08	0.01\\
66.09	0.01\\
66.1	0.01\\
66.11	0.01\\
66.12	0.01\\
66.13	0.01\\
66.14	0.01\\
66.15	0.01\\
66.16	0.01\\
66.17	0.01\\
66.18	0.01\\
66.19	0.01\\
66.2	0.01\\
66.21	0.01\\
66.22	0.01\\
66.23	0.01\\
66.24	0.01\\
66.25	0.01\\
66.26	0.01\\
66.27	0.01\\
66.28	0.01\\
66.29	0.01\\
66.3	0.01\\
66.31	0.01\\
66.32	0.01\\
66.33	0.01\\
66.34	0.01\\
66.35	0.01\\
66.36	0.01\\
66.37	0.01\\
66.38	0.01\\
66.39	0.01\\
66.4	0.01\\
66.41	0.01\\
66.42	0.01\\
66.43	0.01\\
66.44	0.01\\
66.45	0.01\\
66.46	0.01\\
66.47	0.01\\
66.48	0.01\\
66.49	0.01\\
66.5	0.01\\
66.51	0.01\\
66.52	0.01\\
66.53	0.01\\
66.54	0.01\\
66.55	0.01\\
66.56	0.01\\
66.57	0.01\\
66.58	0.01\\
66.59	0.01\\
66.6	0.01\\
66.61	0.01\\
66.62	0.01\\
66.63	0.01\\
66.64	0.01\\
66.65	0.01\\
66.66	0.01\\
66.67	0.01\\
66.68	0.01\\
66.69	0.01\\
66.7	0.01\\
66.71	0.01\\
66.72	0.01\\
66.73	0.01\\
66.74	0.01\\
66.75	0.01\\
66.76	0.01\\
66.77	0.01\\
66.78	0.01\\
66.79	0.01\\
66.8	0.01\\
66.81	0.01\\
66.82	0.01\\
66.83	0.01\\
66.84	0.01\\
66.85	0.01\\
66.86	0.01\\
66.87	0.01\\
66.88	0.01\\
66.89	0.01\\
66.9	0.01\\
66.91	0.01\\
66.92	0.01\\
66.93	0.01\\
66.94	0.01\\
66.95	0.01\\
66.96	0.01\\
66.97	0.01\\
66.98	0.01\\
66.99	0.01\\
67	0.01\\
67.01	0.01\\
67.02	0.01\\
67.03	0.01\\
67.04	0.01\\
67.05	0.01\\
67.06	0.01\\
67.07	0.01\\
67.08	0.01\\
67.09	0.01\\
67.1	0.01\\
67.11	0.01\\
67.12	0.01\\
67.13	0.01\\
67.14	0.01\\
67.15	0.01\\
67.16	0.01\\
67.17	0.01\\
67.18	0.01\\
67.19	0.01\\
67.2	0.01\\
67.21	0.01\\
67.22	0.01\\
67.23	0.01\\
67.24	0.01\\
67.25	0.01\\
67.26	0.01\\
67.27	0.01\\
67.28	0.01\\
67.29	0.01\\
67.3	0.01\\
67.31	0.01\\
67.32	0.01\\
67.33	0.01\\
67.34	0.01\\
67.35	0.01\\
67.36	0.01\\
67.37	0.01\\
67.38	0.01\\
67.39	0.01\\
67.4	0.01\\
67.41	0.01\\
67.42	0.01\\
67.43	0.01\\
67.44	0.01\\
67.45	0.01\\
67.46	0.01\\
67.47	0.01\\
67.48	0.01\\
67.49	0.01\\
67.5	0.01\\
67.51	0.01\\
67.52	0.01\\
67.53	0.01\\
67.54	0.01\\
67.55	0.01\\
67.56	0.01\\
67.57	0.01\\
67.58	0.01\\
67.59	0.01\\
67.6	0.01\\
67.61	0.01\\
67.62	0.01\\
67.63	0.01\\
67.64	0.01\\
67.65	0.01\\
67.66	0.01\\
67.67	0.01\\
67.68	0.01\\
67.69	0.01\\
67.7	0.01\\
67.71	0.01\\
67.72	0.01\\
67.73	0.01\\
67.74	0.01\\
67.75	0.01\\
67.76	0.01\\
67.77	0.01\\
67.78	0.01\\
67.79	0.01\\
67.8	0.01\\
67.81	0.01\\
67.82	0.01\\
67.83	0.01\\
67.84	0.01\\
67.85	0.01\\
67.86	0.01\\
67.87	0.01\\
67.88	0.01\\
67.89	0.01\\
67.9	0.01\\
67.91	0.01\\
67.92	0.01\\
67.93	0.01\\
67.94	0.01\\
67.95	0.01\\
67.96	0.01\\
67.97	0.01\\
67.98	0.01\\
67.99	0.01\\
68	0.01\\
68.01	0.01\\
68.02	0.01\\
68.03	0.01\\
68.04	0.01\\
68.05	0.01\\
68.06	0.01\\
68.07	0.01\\
68.08	0.01\\
68.09	0.01\\
68.1	0.01\\
68.11	0.01\\
68.12	0.01\\
68.13	0.01\\
68.14	0.01\\
68.15	0.01\\
68.16	0.01\\
68.17	0.01\\
68.18	0.01\\
68.19	0.01\\
68.2	0.01\\
68.21	0.01\\
68.22	0.01\\
68.23	0.01\\
68.24	0.01\\
68.25	0.01\\
68.26	0.01\\
68.27	0.01\\
68.28	0.01\\
68.29	0.01\\
68.3	0.01\\
68.31	0.01\\
68.32	0.01\\
68.33	0.01\\
68.34	0.01\\
68.35	0.01\\
68.36	0.01\\
68.37	0.01\\
68.38	0.01\\
68.39	0.01\\
68.4	0.01\\
68.41	0.01\\
68.42	0.01\\
68.43	0.01\\
68.44	0.01\\
68.45	0.01\\
68.46	0.01\\
68.47	0.01\\
68.48	0.01\\
68.49	0.01\\
68.5	0.01\\
68.51	0.01\\
68.52	0.01\\
68.53	0.01\\
68.54	0.01\\
68.55	0.01\\
68.56	0.01\\
68.57	0.01\\
68.58	0.01\\
68.59	0.01\\
68.6	0.01\\
68.61	0.01\\
68.62	0.01\\
68.63	0.01\\
68.64	0.01\\
68.65	0.01\\
68.66	0.01\\
68.67	0.01\\
68.68	0.01\\
68.69	0.01\\
68.7	0.01\\
68.71	0.01\\
68.72	0.01\\
68.73	0.01\\
68.74	0.01\\
68.75	0.01\\
68.76	0.01\\
68.77	0.01\\
68.78	0.01\\
68.79	0.01\\
68.8	0.01\\
68.81	0.01\\
68.82	0.01\\
68.83	0.01\\
68.84	0.01\\
68.85	0.01\\
68.86	0.01\\
68.87	0.01\\
68.88	0.01\\
68.89	0.01\\
68.9	0.01\\
68.91	0.01\\
68.92	0.01\\
68.93	0.01\\
68.94	0.01\\
68.95	0.01\\
68.96	0.01\\
68.97	0.01\\
68.98	0.01\\
68.99	0.01\\
69	0.01\\
69.01	0.01\\
69.02	0.01\\
69.03	0.01\\
69.04	0.01\\
69.05	0.01\\
69.06	0.01\\
69.07	0.01\\
69.08	0.01\\
69.09	0.01\\
69.1	0.01\\
69.11	0.01\\
69.12	0.01\\
69.13	0.01\\
69.14	0.01\\
69.15	0.01\\
69.16	0.01\\
69.17	0.01\\
69.18	0.01\\
69.19	0.01\\
69.2	0.01\\
69.21	0.01\\
69.22	0.01\\
69.23	0.01\\
69.24	0.01\\
69.25	0.01\\
69.26	0.01\\
69.27	0.01\\
69.28	0.01\\
69.29	0.01\\
69.3	0.01\\
69.31	0.01\\
69.32	0.01\\
69.33	0.01\\
69.34	0.01\\
69.35	0.01\\
69.36	0.01\\
69.37	0.01\\
69.38	0.01\\
69.39	0.01\\
69.4	0.01\\
69.41	0.01\\
69.42	0.01\\
69.43	0.01\\
69.44	0.01\\
69.45	0.01\\
69.46	0.01\\
69.47	0.01\\
69.48	0.01\\
69.49	0.01\\
69.5	0.01\\
69.51	0.01\\
69.52	0.01\\
69.53	0.01\\
69.54	0.01\\
69.55	0.01\\
69.56	0.01\\
69.57	0.01\\
69.58	0.01\\
69.59	0.01\\
69.6	0.01\\
69.61	0.01\\
69.62	0.01\\
69.63	0.01\\
69.64	0.01\\
69.65	0.01\\
69.66	0.01\\
69.67	0.01\\
69.68	0.01\\
69.69	0.01\\
69.7	0.01\\
69.71	0.01\\
69.72	0.01\\
69.73	0.01\\
69.74	0.01\\
69.75	0.01\\
69.76	0.01\\
69.77	0.01\\
69.78	0.01\\
69.79	0.01\\
69.8	0.01\\
69.81	0.01\\
69.82	0.01\\
69.83	0.01\\
69.84	0.01\\
69.85	0.01\\
69.86	0.01\\
69.87	0.01\\
69.88	0.01\\
69.89	0.01\\
69.9	0.01\\
69.91	0.01\\
69.92	0.01\\
69.93	0.01\\
69.94	0.01\\
69.95	0.01\\
69.96	0.01\\
69.97	0.01\\
69.98	0.01\\
69.99	0.01\\
70	0.01\\
70.01	0.01\\
70.02	0.01\\
70.03	0.01\\
70.04	0.01\\
70.05	0.01\\
70.06	0.01\\
70.07	0.01\\
70.08	0.01\\
70.09	0.01\\
70.1	0.01\\
70.11	0.01\\
70.12	0.01\\
70.13	0.01\\
70.14	0.01\\
70.15	0.01\\
70.16	0.01\\
70.17	0.01\\
70.18	0.01\\
70.19	0.01\\
70.2	0.01\\
70.21	0.01\\
70.22	0.01\\
70.23	0.01\\
70.24	0.01\\
70.25	0.01\\
70.26	0.01\\
70.27	0.01\\
70.28	0.01\\
70.29	0.01\\
70.3	0.01\\
70.31	0.01\\
70.32	0.01\\
70.33	0.01\\
70.34	0.01\\
70.35	0.01\\
70.36	0.01\\
70.37	0.01\\
70.38	0.01\\
70.39	0.01\\
70.4	0.01\\
70.41	0.01\\
70.42	0.01\\
70.43	0.01\\
70.44	0.01\\
70.45	0.01\\
70.46	0.01\\
70.47	0.01\\
70.48	0.01\\
70.49	0.01\\
70.5	0.01\\
70.51	0.01\\
70.52	0.01\\
70.53	0.01\\
70.54	0.01\\
70.55	0.01\\
70.56	0.01\\
70.57	0.01\\
70.58	0.01\\
70.59	0.01\\
70.6	0.01\\
70.61	0.01\\
70.62	0.01\\
70.63	0.01\\
70.64	0.01\\
70.65	0.01\\
70.66	0.01\\
70.67	0.01\\
70.68	0.01\\
70.69	0.01\\
70.7	0.01\\
70.71	0.01\\
70.72	0.01\\
70.73	0.01\\
70.74	0.01\\
70.75	0.01\\
70.76	0.01\\
70.77	0.01\\
70.78	0.01\\
70.79	0.01\\
70.8	0.01\\
70.81	0.01\\
70.82	0.01\\
70.83	0.01\\
70.84	0.01\\
70.85	0.01\\
70.86	0.01\\
70.87	0.01\\
70.88	0.01\\
70.89	0.01\\
70.9	0.01\\
70.91	0.01\\
70.92	0.01\\
70.93	0.01\\
70.94	0.01\\
70.95	0.01\\
70.96	0.01\\
70.97	0.01\\
70.98	0.01\\
70.99	0.01\\
71	0.01\\
71.01	0.01\\
71.02	0.01\\
71.03	0.01\\
71.04	0.01\\
71.05	0.01\\
71.06	0.01\\
71.07	0.01\\
71.08	0.01\\
71.09	0.01\\
71.1	0.01\\
71.11	0.01\\
71.12	0.01\\
71.13	0.01\\
71.14	0.01\\
71.15	0.01\\
71.16	0.01\\
71.17	0.01\\
71.18	0.01\\
71.19	0.01\\
71.2	0.01\\
71.21	0.01\\
71.22	0.01\\
71.23	0.01\\
71.24	0.01\\
71.25	0.01\\
71.26	0.01\\
71.27	0.01\\
71.28	0.01\\
71.29	0.01\\
71.3	0.01\\
71.31	0.01\\
71.32	0.01\\
71.33	0.01\\
71.34	0.01\\
71.35	0.01\\
71.36	0.01\\
71.37	0.01\\
71.38	0.01\\
71.39	0.01\\
71.4	0.01\\
71.41	0.01\\
71.42	0.01\\
71.43	0.01\\
71.44	0.01\\
71.45	0.01\\
71.46	0.01\\
71.47	0.01\\
71.48	0.01\\
71.49	0.01\\
71.5	0.01\\
71.51	0.01\\
71.52	0.01\\
71.53	0.01\\
71.54	0.01\\
71.55	0.01\\
71.56	0.01\\
71.57	0.01\\
71.58	0.01\\
71.59	0.01\\
71.6	0.01\\
71.61	0.01\\
71.62	0.01\\
71.63	0.01\\
71.64	0.01\\
71.65	0.01\\
71.66	0.01\\
71.67	0.01\\
71.68	0.01\\
71.69	0.01\\
71.7	0.01\\
71.71	0.01\\
71.72	0.01\\
71.73	0.01\\
71.74	0.01\\
71.75	0.01\\
71.76	0.01\\
71.77	0.01\\
71.78	0.01\\
71.79	0.01\\
71.8	0.01\\
71.81	0.01\\
71.82	0.01\\
71.83	0.01\\
71.84	0.01\\
71.85	0.01\\
71.86	0.01\\
71.87	0.01\\
71.88	0.01\\
71.89	0.01\\
71.9	0.01\\
71.91	0.01\\
71.92	0.01\\
71.93	0.01\\
71.94	0.01\\
71.95	0.01\\
71.96	0.01\\
71.97	0.01\\
71.98	0.01\\
71.99	0.01\\
72	0.01\\
72.01	0.01\\
72.02	0.01\\
72.03	0.01\\
72.04	0.01\\
72.05	0.01\\
72.06	0.01\\
72.07	0.01\\
72.08	0.01\\
72.09	0.01\\
72.1	0.01\\
72.11	0.01\\
72.12	0.01\\
72.13	0.01\\
72.14	0.01\\
72.15	0.01\\
72.16	0.01\\
72.17	0.01\\
72.18	0.01\\
72.19	0.01\\
72.2	0.01\\
72.21	0.01\\
72.22	0.01\\
72.23	0.01\\
72.24	0.01\\
72.25	0.01\\
72.26	0.01\\
72.27	0.01\\
72.28	0.01\\
72.29	0.01\\
72.3	0.01\\
72.31	0.01\\
72.32	0.01\\
72.33	0.01\\
72.34	0.01\\
72.35	0.01\\
72.36	0.01\\
72.37	0.01\\
72.38	0.01\\
72.39	0.01\\
72.4	0.01\\
72.41	0.01\\
72.42	0.01\\
72.43	0.01\\
72.44	0.01\\
72.45	0.01\\
72.46	0.01\\
72.47	0.01\\
72.48	0.01\\
72.49	0.01\\
72.5	0.01\\
72.51	0.01\\
72.52	0.01\\
72.53	0.01\\
72.54	0.01\\
72.55	0.01\\
72.56	0.01\\
72.57	0.01\\
72.58	0.01\\
72.59	0.01\\
72.6	0.01\\
72.61	0.01\\
72.62	0.01\\
72.63	0.01\\
72.64	0.01\\
72.65	0.01\\
72.66	0.01\\
72.67	0.01\\
72.68	0.01\\
72.69	0.01\\
72.7	0.01\\
72.71	0.01\\
72.72	0.01\\
72.73	0.01\\
72.74	0.01\\
72.75	0.01\\
72.76	0.01\\
72.77	0.01\\
72.78	0.01\\
72.79	0.01\\
72.8	0.01\\
72.81	0.01\\
72.82	0.01\\
72.83	0.01\\
72.84	0.01\\
72.85	0.01\\
72.86	0.01\\
72.87	0.01\\
72.88	0.01\\
72.89	0.01\\
72.9	0.01\\
72.91	0.01\\
72.92	0.01\\
72.93	0.01\\
72.94	0.01\\
72.95	0.01\\
72.96	0.01\\
72.97	0.01\\
72.98	0.01\\
72.99	0.01\\
73	0.01\\
73.01	0.01\\
73.02	0.01\\
73.03	0.01\\
73.04	0.01\\
73.05	0.01\\
73.06	0.01\\
73.07	0.01\\
73.08	0.01\\
73.09	0.01\\
73.1	0.01\\
73.11	0.01\\
73.12	0.01\\
73.13	0.01\\
73.14	0.01\\
73.15	0.01\\
73.16	0.01\\
73.17	0.01\\
73.18	0.01\\
73.19	0.01\\
73.2	0.01\\
73.21	0.01\\
73.22	0.01\\
73.23	0.01\\
73.24	0.01\\
73.25	0.01\\
73.26	0.01\\
73.27	0.01\\
73.28	0.01\\
73.29	0.01\\
73.3	0.01\\
73.31	0.01\\
73.32	0.01\\
73.33	0.01\\
73.34	0.01\\
73.35	0.01\\
73.36	0.01\\
73.37	0.01\\
73.38	0.01\\
73.39	0.01\\
73.4	0.01\\
73.41	0.01\\
73.42	0.01\\
73.43	0.01\\
73.44	0.01\\
73.45	0.01\\
73.46	0.01\\
73.47	0.01\\
73.48	0.01\\
73.49	0.01\\
73.5	0.01\\
73.51	0.01\\
73.52	0.01\\
73.53	0.01\\
73.54	0.01\\
73.55	0.01\\
73.56	0.01\\
73.57	0.01\\
73.58	0.01\\
73.59	0.01\\
73.6	0.01\\
73.61	0.01\\
73.62	0.01\\
73.63	0.01\\
73.64	0.01\\
73.65	0.01\\
73.66	0.01\\
73.67	0.01\\
73.68	0.01\\
73.69	0.01\\
73.7	0.01\\
73.71	0.01\\
73.72	0.01\\
73.73	0.01\\
73.74	0.01\\
73.75	0.01\\
73.76	0.01\\
73.77	0.01\\
73.78	0.01\\
73.79	0.01\\
73.8	0.01\\
73.81	0.01\\
73.82	0.01\\
73.83	0.01\\
73.84	0.01\\
73.85	0.01\\
73.86	0.01\\
73.87	0.01\\
73.88	0.01\\
73.89	0.01\\
73.9	0.01\\
73.91	0.01\\
73.92	0.01\\
73.93	0.01\\
73.94	0.01\\
73.95	0.01\\
73.96	0.01\\
73.97	0.01\\
73.98	0.01\\
73.99	0.01\\
74	0.01\\
74.01	0.01\\
74.02	0.01\\
74.03	0.01\\
74.04	0.01\\
74.05	0.01\\
74.06	0.01\\
74.07	0.01\\
74.08	0.01\\
74.09	0.01\\
74.1	0.01\\
74.11	0.01\\
74.12	0.01\\
74.13	0.01\\
74.14	0.01\\
74.15	0.01\\
74.16	0.01\\
74.17	0.01\\
74.18	0.01\\
74.19	0.01\\
74.2	0.01\\
74.21	0.01\\
74.22	0.01\\
74.23	0.01\\
74.24	0.01\\
74.25	0.01\\
74.26	0.01\\
74.27	0.01\\
74.28	0.01\\
74.29	0.01\\
74.3	0.01\\
74.31	0.01\\
74.32	0.01\\
74.33	0.01\\
74.34	0.01\\
74.35	0.01\\
74.36	0.01\\
74.37	0.01\\
74.38	0.01\\
74.39	0.01\\
74.4	0.01\\
74.41	0.01\\
74.42	0.01\\
74.43	0.01\\
74.44	0.01\\
74.45	0.01\\
74.46	0.01\\
74.47	0.01\\
74.48	0.01\\
74.49	0.01\\
74.5	0.01\\
74.51	0.01\\
74.52	0.01\\
74.53	0.01\\
74.54	0.01\\
74.55	0.01\\
74.56	0.01\\
74.57	0.01\\
74.58	0.01\\
74.59	0.01\\
74.6	0.01\\
74.61	0.01\\
74.62	0.01\\
74.63	0.01\\
74.64	0.01\\
74.65	0.01\\
74.66	0.01\\
74.67	0.01\\
74.68	0.01\\
74.69	0.01\\
74.7	0.01\\
74.71	0.01\\
74.72	0.01\\
74.73	0.01\\
74.74	0.01\\
74.75	0.01\\
74.76	0.01\\
74.77	0.01\\
74.78	0.01\\
74.79	0.01\\
74.8	0.01\\
74.81	0.01\\
74.82	0.01\\
74.83	0.01\\
74.84	0.01\\
74.85	0.01\\
74.86	0.01\\
74.87	0.01\\
74.88	0.01\\
74.89	0.01\\
74.9	0.01\\
74.91	0.01\\
74.92	0.01\\
74.93	0.01\\
74.94	0.01\\
74.95	0.01\\
74.96	0.01\\
74.97	0.01\\
74.98	0.01\\
74.99	0.01\\
75	0.01\\
75.01	0.01\\
75.02	0.01\\
75.03	0.01\\
75.04	0.01\\
75.05	0.01\\
75.06	0.01\\
75.07	0.01\\
75.08	0.01\\
75.09	0.01\\
75.1	0.01\\
75.11	0.01\\
75.12	0.01\\
75.13	0.01\\
75.14	0.01\\
75.15	0.01\\
75.16	0.01\\
75.17	0.01\\
75.18	0.01\\
75.19	0.01\\
75.2	0.01\\
75.21	0.01\\
75.22	0.01\\
75.23	0.01\\
75.24	0.01\\
75.25	0.01\\
75.26	0.01\\
75.27	0.01\\
75.28	0.01\\
75.29	0.01\\
75.3	0.01\\
75.31	0.01\\
75.32	0.01\\
75.33	0.01\\
75.34	0.01\\
75.35	0.01\\
75.36	0.01\\
75.37	0.01\\
75.38	0.01\\
75.39	0.01\\
75.4	0.01\\
75.41	0.01\\
75.42	0.01\\
75.43	0.01\\
75.44	0.01\\
75.45	0.01\\
75.46	0.01\\
75.47	0.01\\
75.48	0.01\\
75.49	0.01\\
75.5	0.01\\
75.51	0.01\\
75.52	0.01\\
75.53	0.01\\
75.54	0.01\\
75.55	0.01\\
75.56	0.01\\
75.57	0.01\\
75.58	0.01\\
75.59	0.01\\
75.6	0.01\\
75.61	0.01\\
75.62	0.01\\
75.63	0.01\\
75.64	0.01\\
75.65	0.01\\
75.66	0.01\\
75.67	0.01\\
75.68	0.01\\
75.69	0.01\\
75.7	0.01\\
75.71	0.01\\
75.72	0.01\\
75.73	0.01\\
75.74	0.01\\
75.75	0.01\\
75.76	0.01\\
75.77	0.01\\
75.78	0.01\\
75.79	0.01\\
75.8	0.01\\
75.81	0.01\\
75.82	0.01\\
75.83	0.01\\
75.84	0.01\\
75.85	0.01\\
75.86	0.01\\
75.87	0.01\\
75.88	0.01\\
75.89	0.01\\
75.9	0.01\\
75.91	0.01\\
75.92	0.01\\
75.93	0.01\\
75.94	0.01\\
75.95	0.01\\
75.96	0.01\\
75.97	0.01\\
75.98	0.01\\
75.99	0.01\\
76	0.01\\
76.01	0.01\\
76.02	0.01\\
76.03	0.01\\
76.04	0.01\\
76.05	0.01\\
76.06	0.01\\
76.07	0.01\\
76.08	0.01\\
76.09	0.01\\
76.1	0.01\\
76.11	0.01\\
76.12	0.01\\
76.13	0.01\\
76.14	0.01\\
76.15	0.01\\
76.16	0.01\\
76.17	0.01\\
76.18	0.01\\
76.19	0.01\\
76.2	0.01\\
76.21	0.01\\
76.22	0.01\\
76.23	0.01\\
76.24	0.01\\
76.25	0.01\\
76.26	0.01\\
76.27	0.01\\
76.28	0.01\\
76.29	0.01\\
76.3	0.01\\
76.31	0.01\\
76.32	0.01\\
76.33	0.01\\
76.34	0.01\\
76.35	0.01\\
76.36	0.01\\
76.37	0.01\\
76.38	0.01\\
76.39	0.01\\
76.4	0.01\\
76.41	0.01\\
76.42	0.01\\
76.43	0.01\\
76.44	0.01\\
76.45	0.01\\
76.46	0.01\\
76.47	0.01\\
76.48	0.01\\
76.49	0.01\\
76.5	0.01\\
76.51	0.01\\
76.52	0.01\\
76.53	0.01\\
76.54	0.01\\
76.55	0.01\\
76.56	0.01\\
76.57	0.01\\
76.58	0.01\\
76.59	0.01\\
76.6	0.01\\
76.61	0.01\\
76.62	0.01\\
76.63	0.01\\
76.64	0.01\\
76.65	0.01\\
76.66	0.01\\
76.67	0.01\\
76.68	0.01\\
76.69	0.01\\
76.7	0.01\\
76.71	0.01\\
76.72	0.01\\
76.73	0.01\\
76.74	0.01\\
76.75	0.01\\
76.76	0.01\\
76.77	0.01\\
76.78	0.01\\
76.79	0.01\\
76.8	0.01\\
76.81	0.01\\
76.82	0.01\\
76.83	0.01\\
76.84	0.01\\
76.85	0.01\\
76.86	0.01\\
76.87	0.01\\
76.88	0.01\\
76.89	0.01\\
76.9	0.01\\
76.91	0.01\\
76.92	0.01\\
76.93	0.01\\
76.94	0.01\\
76.95	0.01\\
76.96	0.01\\
76.97	0.01\\
76.98	0.01\\
76.99	0.01\\
77	0.01\\
77.01	0.01\\
77.02	0.01\\
77.03	0.01\\
77.04	0.01\\
77.05	0.01\\
77.06	0.01\\
77.07	0.01\\
77.08	0.01\\
77.09	0.01\\
77.1	0.01\\
77.11	0.01\\
77.12	0.01\\
77.13	0.01\\
77.14	0.01\\
77.15	0.01\\
77.16	0.01\\
77.17	0.01\\
77.18	0.01\\
77.19	0.01\\
77.2	0.01\\
77.21	0.01\\
77.22	0.01\\
77.23	0.01\\
77.24	0.01\\
77.25	0.01\\
77.26	0.01\\
77.27	0.01\\
77.28	0.01\\
77.29	0.01\\
77.3	0.01\\
77.31	0.01\\
77.32	0.01\\
77.33	0.01\\
77.34	0.01\\
77.35	0.01\\
77.36	0.01\\
77.37	0.01\\
77.38	0.01\\
77.39	0.01\\
77.4	0.01\\
77.41	0.01\\
77.42	0.01\\
77.43	0.01\\
77.44	0.01\\
77.45	0.01\\
77.46	0.01\\
77.47	0.01\\
77.48	0.01\\
77.49	0.01\\
77.5	0.01\\
77.51	0.01\\
77.52	0.01\\
77.53	0.01\\
77.54	0.01\\
77.55	0.01\\
77.56	0.01\\
77.57	0.01\\
77.58	0.01\\
77.59	0.01\\
77.6	0.01\\
77.61	0.01\\
77.62	0.01\\
77.63	0.01\\
77.64	0.01\\
77.65	0.01\\
77.66	0.01\\
77.67	0.01\\
77.68	0.01\\
77.69	0.01\\
77.7	0.01\\
77.71	0.01\\
77.72	0.01\\
77.73	0.01\\
77.74	0.01\\
77.75	0.01\\
77.76	0.01\\
77.77	0.01\\
77.78	0.01\\
77.79	0.01\\
77.8	0.01\\
77.81	0.01\\
77.82	0.01\\
77.83	0.01\\
77.84	0.01\\
77.85	0.01\\
77.86	0.01\\
77.87	0.01\\
77.88	0.01\\
77.89	0.01\\
77.9	0.01\\
77.91	0.01\\
77.92	0.01\\
77.93	0.01\\
77.94	0.01\\
77.95	0.01\\
77.96	0.01\\
77.97	0.01\\
77.98	0.01\\
77.99	0.01\\
78	0.01\\
78.01	0.01\\
78.02	0.01\\
78.03	0.01\\
78.04	0.01\\
78.05	0.01\\
78.06	0.01\\
78.07	0.01\\
78.08	0.01\\
78.09	0.01\\
78.1	0.01\\
78.11	0.01\\
78.12	0.01\\
78.13	0.01\\
78.14	0.01\\
78.15	0.01\\
78.16	0.01\\
78.17	0.01\\
78.18	0.01\\
78.19	0.01\\
78.2	0.01\\
78.21	0.01\\
78.22	0.01\\
78.23	0.01\\
78.24	0.01\\
78.25	0.01\\
78.26	0.01\\
78.27	0.01\\
78.28	0.01\\
78.29	0.01\\
78.3	0.01\\
78.31	0.01\\
78.32	0.01\\
78.33	0.01\\
78.34	0.01\\
78.35	0.01\\
78.36	0.01\\
78.37	0.01\\
78.38	0.01\\
78.39	0.01\\
78.4	0.01\\
78.41	0.01\\
78.42	0.01\\
78.43	0.01\\
78.44	0.01\\
78.45	0.01\\
78.46	0.01\\
78.47	0.01\\
78.48	0.01\\
78.49	0.01\\
78.5	0.01\\
78.51	0.01\\
78.52	0.01\\
78.53	0.01\\
78.54	0.01\\
78.55	0.01\\
78.56	0.01\\
78.57	0.01\\
78.58	0.01\\
78.59	0.01\\
78.6	0.01\\
78.61	0.01\\
78.62	0.01\\
78.63	0.01\\
78.64	0.01\\
78.65	0.01\\
78.66	0.01\\
78.67	0.01\\
78.68	0.01\\
78.69	0.01\\
78.7	0.01\\
78.71	0.01\\
78.72	0.01\\
78.73	0.01\\
78.74	0.01\\
78.75	0.01\\
78.76	0.01\\
78.77	0.01\\
78.78	0.01\\
78.79	0.01\\
78.8	0.01\\
78.81	0.01\\
78.82	0.01\\
78.83	0.01\\
78.84	0.01\\
78.85	0.01\\
78.86	0.01\\
78.87	0.01\\
78.88	0.01\\
78.89	0.01\\
78.9	0.01\\
78.91	0.01\\
78.92	0.01\\
78.93	0.01\\
78.94	0.01\\
78.95	0.01\\
78.96	0.01\\
78.97	0.01\\
78.98	0.01\\
78.99	0.01\\
79	0.01\\
79.01	0.01\\
79.02	0.01\\
79.03	0.01\\
79.04	0.01\\
79.05	0.01\\
79.06	0.01\\
79.07	0.01\\
79.08	0.01\\
79.09	0.01\\
79.1	0.01\\
79.11	0.01\\
79.12	0.01\\
79.13	0.01\\
79.14	0.01\\
79.15	0.01\\
79.16	0.01\\
79.17	0.01\\
79.18	0.01\\
79.19	0.01\\
79.2	0.01\\
79.21	0.01\\
79.22	0.01\\
79.23	0.01\\
79.24	0.01\\
79.25	0.01\\
79.26	0.01\\
79.27	0.01\\
79.28	0.01\\
79.29	0.01\\
79.3	0.01\\
79.31	0.01\\
79.32	0.01\\
79.33	0.01\\
79.34	0.01\\
79.35	0.01\\
79.36	0.01\\
79.37	0.01\\
79.38	0.01\\
79.39	0.01\\
79.4	0.01\\
79.41	0.01\\
79.42	0.01\\
79.43	0.01\\
79.44	0.01\\
79.45	0.01\\
79.46	0.01\\
79.47	0.01\\
79.48	0.01\\
79.49	0.01\\
79.5	0.01\\
79.51	0.01\\
79.52	0.01\\
79.53	0.01\\
79.54	0.01\\
79.55	0.01\\
79.56	0.01\\
79.57	0.01\\
79.58	0.01\\
79.59	0.01\\
79.6	0.01\\
79.61	0.01\\
79.62	0.01\\
79.63	0.01\\
79.64	0.01\\
79.65	0.01\\
79.66	0.01\\
79.67	0.01\\
79.68	0.01\\
79.69	0.01\\
79.7	0.01\\
79.71	0.01\\
79.72	0.01\\
79.73	0.01\\
79.74	0.01\\
79.75	0.01\\
79.76	0.01\\
79.77	0.01\\
79.78	0.01\\
79.79	0.01\\
79.8	0.01\\
79.81	0.01\\
79.82	0.01\\
79.83	0.01\\
79.84	0.01\\
79.85	0.01\\
79.86	0.01\\
79.87	0.01\\
79.88	0.01\\
79.89	0.01\\
79.9	0.01\\
79.91	0.01\\
79.92	0.01\\
79.93	0.01\\
79.94	0.01\\
79.95	0.01\\
79.96	0.01\\
79.97	0.01\\
79.98	0.01\\
79.99	0.01\\
80	0.01\\
80.01	0.01\\
};
\addplot [color=blue,solid]
  table[row sep=crcr]{%
80.01	0.01\\
80.02	0.01\\
80.03	0.01\\
80.04	0.01\\
80.05	0.01\\
80.06	0.01\\
80.07	0.01\\
80.08	0.01\\
80.09	0.01\\
80.1	0.01\\
80.11	0.01\\
80.12	0.01\\
80.13	0.01\\
80.14	0.01\\
80.15	0.01\\
80.16	0.01\\
80.17	0.01\\
80.18	0.01\\
80.19	0.01\\
80.2	0.01\\
80.21	0.01\\
80.22	0.01\\
80.23	0.01\\
80.24	0.01\\
80.25	0.01\\
80.26	0.01\\
80.27	0.01\\
80.28	0.01\\
80.29	0.01\\
80.3	0.01\\
80.31	0.01\\
80.32	0.01\\
80.33	0.01\\
80.34	0.01\\
80.35	0.01\\
80.36	0.01\\
80.37	0.01\\
80.38	0.01\\
80.39	0.01\\
80.4	0.01\\
80.41	0.01\\
80.42	0.01\\
80.43	0.01\\
80.44	0.01\\
80.45	0.01\\
80.46	0.01\\
80.47	0.01\\
80.48	0.01\\
80.49	0.01\\
80.5	0.01\\
80.51	0.01\\
80.52	0.01\\
80.53	0.01\\
80.54	0.01\\
80.55	0.01\\
80.56	0.01\\
80.57	0.01\\
80.58	0.01\\
80.59	0.01\\
80.6	0.01\\
80.61	0.01\\
80.62	0.01\\
80.63	0.01\\
80.64	0.01\\
80.65	0.01\\
80.66	0.01\\
80.67	0.01\\
80.68	0.01\\
80.69	0.01\\
80.7	0.01\\
80.71	0.01\\
80.72	0.01\\
80.73	0.01\\
80.74	0.01\\
80.75	0.01\\
80.76	0.01\\
80.77	0.01\\
80.78	0.01\\
80.79	0.01\\
80.8	0.01\\
80.81	0.01\\
80.82	0.01\\
80.83	0.01\\
80.84	0.01\\
80.85	0.01\\
80.86	0.01\\
80.87	0.01\\
80.88	0.01\\
80.89	0.01\\
80.9	0.01\\
80.91	0.01\\
80.92	0.01\\
80.93	0.01\\
80.94	0.01\\
80.95	0.01\\
80.96	0.01\\
80.97	0.01\\
80.98	0.01\\
80.99	0.01\\
81	0.01\\
81.01	0.01\\
81.02	0.01\\
81.03	0.01\\
81.04	0.01\\
81.05	0.01\\
81.06	0.01\\
81.07	0.01\\
81.08	0.01\\
81.09	0.01\\
81.1	0.01\\
81.11	0.01\\
81.12	0.01\\
81.13	0.01\\
81.14	0.01\\
81.15	0.01\\
81.16	0.01\\
81.17	0.01\\
81.18	0.01\\
81.19	0.01\\
81.2	0.01\\
81.21	0.01\\
81.22	0.01\\
81.23	0.01\\
81.24	0.01\\
81.25	0.01\\
81.26	0.01\\
81.27	0.01\\
81.28	0.01\\
81.29	0.01\\
81.3	0.01\\
81.31	0.01\\
81.32	0.01\\
81.33	0.01\\
81.34	0.01\\
81.35	0.01\\
81.36	0.01\\
81.37	0.01\\
81.38	0.01\\
81.39	0.01\\
81.4	0.01\\
81.41	0.01\\
81.42	0.01\\
81.43	0.01\\
81.44	0.01\\
81.45	0.01\\
81.46	0.01\\
81.47	0.01\\
81.48	0.01\\
81.49	0.01\\
81.5	0.01\\
81.51	0.01\\
81.52	0.01\\
81.53	0.01\\
81.54	0.01\\
81.55	0.01\\
81.56	0.01\\
81.57	0.01\\
81.58	0.01\\
81.59	0.01\\
81.6	0.01\\
81.61	0.01\\
81.62	0.01\\
81.63	0.01\\
81.64	0.01\\
81.65	0.01\\
81.66	0.01\\
81.67	0.01\\
81.68	0.01\\
81.69	0.01\\
81.7	0.01\\
81.71	0.01\\
81.72	0.01\\
81.73	0.01\\
81.74	0.01\\
81.75	0.01\\
81.76	0.01\\
81.77	0.01\\
81.78	0.01\\
81.79	0.01\\
81.8	0.01\\
81.81	0.01\\
81.82	0.01\\
81.83	0.01\\
81.84	0.01\\
81.85	0.01\\
81.86	0.01\\
81.87	0.01\\
81.88	0.01\\
81.89	0.01\\
81.9	0.01\\
81.91	0.01\\
81.92	0.01\\
81.93	0.01\\
81.94	0.01\\
81.95	0.01\\
81.96	0.01\\
81.97	0.01\\
81.98	0.01\\
81.99	0.01\\
82	0.01\\
82.01	0.01\\
82.02	0.01\\
82.03	0.01\\
82.04	0.01\\
82.05	0.01\\
82.06	0.01\\
82.07	0.01\\
82.08	0.01\\
82.09	0.01\\
82.1	0.01\\
82.11	0.01\\
82.12	0.01\\
82.13	0.01\\
82.14	0.01\\
82.15	0.01\\
82.16	0.01\\
82.17	0.01\\
82.18	0.01\\
82.19	0.01\\
82.2	0.01\\
82.21	0.01\\
82.22	0.01\\
82.23	0.01\\
82.24	0.01\\
82.25	0.01\\
82.26	0.01\\
82.27	0.01\\
82.28	0.01\\
82.29	0.01\\
82.3	0.01\\
82.31	0.01\\
82.32	0.01\\
82.33	0.01\\
82.34	0.01\\
82.35	0.01\\
82.36	0.01\\
82.37	0.01\\
82.38	0.01\\
82.39	0.01\\
82.4	0.01\\
82.41	0.01\\
82.42	0.01\\
82.43	0.01\\
82.44	0.01\\
82.45	0.01\\
82.46	0.01\\
82.47	0.01\\
82.48	0.01\\
82.49	0.01\\
82.5	0.01\\
82.51	0.01\\
82.52	0.01\\
82.53	0.01\\
82.54	0.01\\
82.55	0.01\\
82.56	0.01\\
82.57	0.01\\
82.58	0.01\\
82.59	0.01\\
82.6	0.01\\
82.61	0.01\\
82.62	0.01\\
82.63	0.01\\
82.64	0.01\\
82.65	0.01\\
82.66	0.01\\
82.67	0.01\\
82.68	0.01\\
82.69	0.01\\
82.7	0.01\\
82.71	0.01\\
82.72	0.01\\
82.73	0.01\\
82.74	0.01\\
82.75	0.01\\
82.76	0.01\\
82.77	0.01\\
82.78	0.01\\
82.79	0.01\\
82.8	0.01\\
82.81	0.01\\
82.82	0.01\\
82.83	0.01\\
82.84	0.01\\
82.85	0.01\\
82.86	0.01\\
82.87	0.01\\
82.88	0.01\\
82.89	0.01\\
82.9	0.01\\
82.91	0.01\\
82.92	0.01\\
82.93	0.01\\
82.94	0.01\\
82.95	0.01\\
82.96	0.01\\
82.97	0.01\\
82.98	0.01\\
82.99	0.01\\
83	0.01\\
83.01	0.01\\
83.02	0.01\\
83.03	0.01\\
83.04	0.01\\
83.05	0.01\\
83.06	0.01\\
83.07	0.01\\
83.08	0.01\\
83.09	0.01\\
83.1	0.01\\
83.11	0.01\\
83.12	0.01\\
83.13	0.01\\
83.14	0.01\\
83.15	0.01\\
83.16	0.01\\
83.17	0.01\\
83.18	0.01\\
83.19	0.01\\
83.2	0.01\\
83.21	0.01\\
83.22	0.01\\
83.23	0.01\\
83.24	0.01\\
83.25	0.01\\
83.26	0.01\\
83.27	0.01\\
83.28	0.01\\
83.29	0.01\\
83.3	0.01\\
83.31	0.01\\
83.32	0.01\\
83.33	0.01\\
83.34	0.01\\
83.35	0.01\\
83.36	0.01\\
83.37	0.01\\
83.38	0.01\\
83.39	0.01\\
83.4	0.01\\
83.41	0.01\\
83.42	0.01\\
83.43	0.01\\
83.44	0.01\\
83.45	0.01\\
83.46	0.01\\
83.47	0.01\\
83.48	0.01\\
83.49	0.01\\
83.5	0.01\\
83.51	0.01\\
83.52	0.01\\
83.53	0.01\\
83.54	0.01\\
83.55	0.01\\
83.56	0.01\\
83.57	0.01\\
83.58	0.01\\
83.59	0.01\\
83.6	0.01\\
83.61	0.01\\
83.62	0.01\\
83.63	0.01\\
83.64	0.01\\
83.65	0.01\\
83.66	0.01\\
83.67	0.01\\
83.68	0.01\\
83.69	0.01\\
83.7	0.01\\
83.71	0.01\\
83.72	0.01\\
83.73	0.01\\
83.74	0.01\\
83.75	0.01\\
83.76	0.01\\
83.77	0.01\\
83.78	0.01\\
83.79	0.01\\
83.8	0.01\\
83.81	0.01\\
83.82	0.01\\
83.83	0.01\\
83.84	0.01\\
83.85	0.01\\
83.86	0.01\\
83.87	0.01\\
83.88	0.01\\
83.89	0.01\\
83.9	0.01\\
83.91	0.01\\
83.92	0.01\\
83.93	0.01\\
83.94	0.01\\
83.95	0.01\\
83.96	0.01\\
83.97	0.01\\
83.98	0.01\\
83.99	0.01\\
84	0.01\\
84.01	0.01\\
84.02	0.01\\
84.03	0.01\\
84.04	0.01\\
84.05	0.01\\
84.06	0.01\\
84.07	0.01\\
84.08	0.01\\
84.09	0.01\\
84.1	0.01\\
84.11	0.01\\
84.12	0.01\\
84.13	0.01\\
84.14	0.01\\
84.15	0.01\\
84.16	0.01\\
84.17	0.01\\
84.18	0.01\\
84.19	0.01\\
84.2	0.01\\
84.21	0.01\\
84.22	0.01\\
84.23	0.01\\
84.24	0.01\\
84.25	0.01\\
84.26	0.01\\
84.27	0.01\\
84.28	0.01\\
84.29	0.01\\
84.3	0.01\\
84.31	0.01\\
84.32	0.01\\
84.33	0.01\\
84.34	0.01\\
84.35	0.01\\
84.36	0.01\\
84.37	0.01\\
84.38	0.01\\
84.39	0.01\\
84.4	0.01\\
84.41	0.01\\
84.42	0.01\\
84.43	0.01\\
84.44	0.01\\
84.45	0.01\\
84.46	0.01\\
84.47	0.01\\
84.48	0.01\\
84.49	0.01\\
84.5	0.01\\
84.51	0.01\\
84.52	0.01\\
84.53	0.01\\
84.54	0.01\\
84.55	0.01\\
84.56	0.01\\
84.57	0.01\\
84.58	0.01\\
84.59	0.01\\
84.6	0.01\\
84.61	0.01\\
84.62	0.01\\
84.63	0.01\\
84.64	0.01\\
84.65	0.01\\
84.66	0.01\\
84.67	0.01\\
84.68	0.01\\
84.69	0.01\\
84.7	0.01\\
84.71	0.01\\
84.72	0.01\\
84.73	0.01\\
84.74	0.01\\
84.75	0.01\\
84.76	0.01\\
84.77	0.01\\
84.78	0.01\\
84.79	0.01\\
84.8	0.01\\
84.81	0.01\\
84.82	0.01\\
84.83	0.01\\
84.84	0.01\\
84.85	0.01\\
84.86	0.01\\
84.87	0.01\\
84.88	0.01\\
84.89	0.01\\
84.9	0.01\\
84.91	0.01\\
84.92	0.01\\
84.93	0.01\\
84.94	0.01\\
84.95	0.01\\
84.96	0.01\\
84.97	0.01\\
84.98	0.01\\
84.99	0.01\\
85	0.01\\
85.01	0.01\\
85.02	0.01\\
85.03	0.01\\
85.04	0.01\\
85.05	0.01\\
85.06	0.01\\
85.07	0.01\\
85.08	0.01\\
85.09	0.01\\
85.1	0.01\\
85.11	0.01\\
85.12	0.01\\
85.13	0.01\\
85.14	0.01\\
85.15	0.01\\
85.16	0.01\\
85.17	0.01\\
85.18	0.01\\
85.19	0.01\\
85.2	0.01\\
85.21	0.01\\
85.22	0.01\\
85.23	0.01\\
85.24	0.01\\
85.25	0.01\\
85.26	0.01\\
85.27	0.01\\
85.28	0.01\\
85.29	0.01\\
85.3	0.01\\
85.31	0.01\\
85.32	0.01\\
85.33	0.01\\
85.34	0.01\\
85.35	0.01\\
85.36	0.01\\
85.37	0.01\\
85.38	0.01\\
85.39	0.01\\
85.4	0.01\\
85.41	0.01\\
85.42	0.01\\
85.43	0.01\\
85.44	0.01\\
85.45	0.01\\
85.46	0.01\\
85.47	0.01\\
85.48	0.01\\
85.49	0.01\\
85.5	0.01\\
85.51	0.01\\
85.52	0.01\\
85.53	0.01\\
85.54	0.01\\
85.55	0.01\\
85.56	0.01\\
85.57	0.01\\
85.58	0.01\\
85.59	0.01\\
85.6	0.01\\
85.61	0.01\\
85.62	0.01\\
85.63	0.01\\
85.64	0.01\\
85.65	0.01\\
85.66	0.01\\
85.67	0.01\\
85.68	0.01\\
85.69	0.01\\
85.7	0.01\\
85.71	0.01\\
85.72	0.01\\
85.73	0.01\\
85.74	0.01\\
85.75	0.01\\
85.76	0.01\\
85.77	0.01\\
85.78	0.01\\
85.79	0.01\\
85.8	0.01\\
85.81	0.01\\
85.82	0.01\\
85.83	0.01\\
85.84	0.01\\
85.85	0.01\\
85.86	0.01\\
85.87	0.01\\
85.88	0.01\\
85.89	0.01\\
85.9	0.01\\
85.91	0.01\\
85.92	0.01\\
85.93	0.01\\
85.94	0.01\\
85.95	0.01\\
85.96	0.01\\
85.97	0.01\\
85.98	0.01\\
85.99	0.01\\
86	0.01\\
86.01	0.01\\
86.02	0.01\\
86.03	0.01\\
86.04	0.01\\
86.05	0.01\\
86.06	0.01\\
86.07	0.01\\
86.08	0.01\\
86.09	0.01\\
86.1	0.01\\
86.11	0.01\\
86.12	0.01\\
86.13	0.01\\
86.14	0.01\\
86.15	0.01\\
86.16	0.01\\
86.17	0.01\\
86.18	0.01\\
86.19	0.01\\
86.2	0.01\\
86.21	0.01\\
86.22	0.01\\
86.23	0.01\\
86.24	0.01\\
86.25	0.01\\
86.26	0.01\\
86.27	0.01\\
86.28	0.01\\
86.29	0.01\\
86.3	0.01\\
86.31	0.01\\
86.32	0.01\\
86.33	0.01\\
86.34	0.01\\
86.35	0.01\\
86.36	0.01\\
86.37	0.01\\
86.38	0.01\\
86.39	0.01\\
86.4	0.01\\
86.41	0.01\\
86.42	0.01\\
86.43	0.01\\
86.44	0.01\\
86.45	0.01\\
86.46	0.01\\
86.47	0.01\\
86.48	0.01\\
86.49	0.01\\
86.5	0.01\\
86.51	0.01\\
86.52	0.01\\
86.53	0.01\\
86.54	0.01\\
86.55	0.01\\
86.56	0.01\\
86.57	0.01\\
86.58	0.01\\
86.59	0.01\\
86.6	0.01\\
86.61	0.01\\
86.62	0.01\\
86.63	0.01\\
86.64	0.01\\
86.65	0.01\\
86.66	0.01\\
86.67	0.01\\
86.68	0.01\\
86.69	0.01\\
86.7	0.01\\
86.71	0.01\\
86.72	0.01\\
86.73	0.01\\
86.74	0.01\\
86.75	0.01\\
86.76	0.01\\
86.77	0.01\\
86.78	0.01\\
86.79	0.01\\
86.8	0.01\\
86.81	0.01\\
86.82	0.01\\
86.83	0.01\\
86.84	0.01\\
86.85	0.01\\
86.86	0.01\\
86.87	0.01\\
86.88	0.01\\
86.89	0.01\\
86.9	0.01\\
86.91	0.01\\
86.92	0.01\\
86.93	0.01\\
86.94	0.01\\
86.95	0.01\\
86.96	0.01\\
86.97	0.01\\
86.98	0.01\\
86.99	0.01\\
87	0.01\\
87.01	0.01\\
87.02	0.01\\
87.03	0.01\\
87.04	0.01\\
87.05	0.01\\
87.06	0.01\\
87.07	0.01\\
87.08	0.01\\
87.09	0.01\\
87.1	0.01\\
87.11	0.01\\
87.12	0.01\\
87.13	0.01\\
87.14	0.01\\
87.15	0.01\\
87.16	0.01\\
87.17	0.01\\
87.18	0.01\\
87.19	0.01\\
87.2	0.01\\
87.21	0.01\\
87.22	0.01\\
87.23	0.01\\
87.24	0.01\\
87.25	0.01\\
87.26	0.01\\
87.27	0.01\\
87.28	0.01\\
87.29	0.01\\
87.3	0.01\\
87.31	0.01\\
87.32	0.01\\
87.33	0.01\\
87.34	0.01\\
87.35	0.01\\
87.36	0.01\\
87.37	0.01\\
87.38	0.01\\
87.39	0.01\\
87.4	0.01\\
87.41	0.01\\
87.42	0.01\\
87.43	0.01\\
87.44	0.01\\
87.45	0.01\\
87.46	0.01\\
87.47	0.01\\
87.48	0.01\\
87.49	0.01\\
87.5	0.01\\
87.51	0.01\\
87.52	0.01\\
87.53	0.01\\
87.54	0.01\\
87.55	0.01\\
87.56	0.01\\
87.57	0.01\\
87.58	0.01\\
87.59	0.01\\
87.6	0.01\\
87.61	0.01\\
87.62	0.01\\
87.63	0.01\\
87.64	0.01\\
87.65	0.01\\
87.66	0.01\\
87.67	0.01\\
87.68	0.01\\
87.69	0.01\\
87.7	0.01\\
87.71	0.01\\
87.72	0.01\\
87.73	0.01\\
87.74	0.01\\
87.75	0.01\\
87.76	0.01\\
87.77	0.01\\
87.78	0.01\\
87.79	0.01\\
87.8	0.01\\
87.81	0.01\\
87.82	0.01\\
87.83	0.01\\
87.84	0.01\\
87.85	0.01\\
87.86	0.01\\
87.87	0.01\\
87.88	0.01\\
87.89	0.01\\
87.9	0.01\\
87.91	0.01\\
87.92	0.01\\
87.93	0.01\\
87.94	0.01\\
87.95	0.01\\
87.96	0.01\\
87.97	0.01\\
87.98	0.01\\
87.99	0.01\\
88	0.01\\
88.01	0.01\\
88.02	0.01\\
88.03	0.01\\
88.04	0.01\\
88.05	0.01\\
88.06	0.01\\
88.07	0.01\\
88.08	0.01\\
88.09	0.01\\
88.1	0.01\\
88.11	0.01\\
88.12	0.01\\
88.13	0.01\\
88.14	0.01\\
88.15	0.01\\
88.16	0.01\\
88.17	0.01\\
88.18	0.01\\
88.19	0.01\\
88.2	0.01\\
88.21	0.01\\
88.22	0.01\\
88.23	0.01\\
88.24	0.01\\
88.25	0.01\\
88.26	0.01\\
88.27	0.01\\
88.28	0.01\\
88.29	0.01\\
88.3	0.01\\
88.31	0.01\\
88.32	0.01\\
88.33	0.01\\
88.34	0.01\\
88.35	0.01\\
88.36	0.01\\
88.37	0.01\\
88.38	0.01\\
88.39	0.01\\
88.4	0.01\\
88.41	0.01\\
88.42	0.01\\
88.43	0.01\\
88.44	0.01\\
88.45	0.01\\
88.46	0.01\\
88.47	0.01\\
88.48	0.01\\
88.49	0.01\\
88.5	0.01\\
88.51	0.01\\
88.52	0.01\\
88.53	0.01\\
88.54	0.01\\
88.55	0.01\\
88.56	0.01\\
88.57	0.01\\
88.58	0.01\\
88.59	0.01\\
88.6	0.01\\
88.61	0.01\\
88.62	0.01\\
88.63	0.01\\
88.64	0.01\\
88.65	0.01\\
88.66	0.01\\
88.67	0.01\\
88.68	0.01\\
88.69	0.01\\
88.7	0.01\\
88.71	0.01\\
88.72	0.01\\
88.73	0.01\\
88.74	0.01\\
88.75	0.01\\
88.76	0.01\\
88.77	0.01\\
88.78	0.01\\
88.79	0.01\\
88.8	0.01\\
88.81	0.01\\
88.82	0.01\\
88.83	0.01\\
88.84	0.01\\
88.85	0.01\\
88.86	0.01\\
88.87	0.01\\
88.88	0.01\\
88.89	0.01\\
88.9	0.01\\
88.91	0.01\\
88.92	0.01\\
88.93	0.01\\
88.94	0.01\\
88.95	0.01\\
88.96	0.01\\
88.97	0.01\\
88.98	0.01\\
88.99	0.01\\
89	0.01\\
89.01	0.01\\
89.02	0.01\\
89.03	0.01\\
89.04	0.01\\
89.05	0.01\\
89.06	0.01\\
89.07	0.01\\
89.08	0.01\\
89.09	0.01\\
89.1	0.01\\
89.11	0.01\\
89.12	0.01\\
89.13	0.01\\
89.14	0.01\\
89.15	0.01\\
89.16	0.01\\
89.17	0.01\\
89.18	0.01\\
89.19	0.01\\
89.2	0.01\\
89.21	0.01\\
89.22	0.01\\
89.23	0.01\\
89.24	0.01\\
89.25	0.01\\
89.26	0.01\\
89.27	0.01\\
89.28	0.01\\
89.29	0.01\\
89.3	0.01\\
89.31	0.01\\
89.32	0.01\\
89.33	0.01\\
89.34	0.01\\
89.35	0.01\\
89.36	0.01\\
89.37	0.01\\
89.38	0.01\\
89.39	0.01\\
89.4	0.01\\
89.41	0.01\\
89.42	0.01\\
89.43	0.01\\
89.44	0.01\\
89.45	0.01\\
89.46	0.01\\
89.47	0.01\\
89.48	0.01\\
89.49	0.01\\
89.5	0.01\\
89.51	0.01\\
89.52	0.01\\
89.53	0.01\\
89.54	0.01\\
89.55	0.01\\
89.56	0.01\\
89.57	0.01\\
89.58	0.01\\
89.59	0.01\\
89.6	0.01\\
89.61	0.01\\
89.62	0.01\\
89.63	0.01\\
89.64	0.01\\
89.65	0.01\\
89.66	0.01\\
89.67	0.01\\
89.68	0.01\\
89.69	0.01\\
89.7	0.01\\
89.71	0.01\\
89.72	0.01\\
89.73	0.01\\
89.74	0.01\\
89.75	0.01\\
89.76	0.01\\
89.77	0.01\\
89.78	0.01\\
89.79	0.01\\
89.8	0.01\\
89.81	0.01\\
89.82	0.01\\
89.83	0.01\\
89.84	0.01\\
89.85	0.01\\
89.86	0.01\\
89.87	0.01\\
89.88	0.01\\
89.89	0.01\\
89.9	0.01\\
89.91	0.01\\
89.92	0.01\\
89.93	0.01\\
89.94	0.01\\
89.95	0.01\\
89.96	0.01\\
89.97	0.01\\
89.98	0.01\\
89.99	0.01\\
90	0.01\\
90.01	0.01\\
90.02	0.01\\
90.03	0.01\\
90.04	0.01\\
90.05	0.01\\
90.06	0.01\\
90.07	0.01\\
90.08	0.01\\
90.09	0.01\\
90.1	0.01\\
90.11	0.01\\
90.12	0.01\\
90.13	0.01\\
90.14	0.01\\
90.15	0.01\\
90.16	0.01\\
90.17	0.01\\
90.18	0.01\\
90.19	0.01\\
90.2	0.01\\
90.21	0.01\\
90.22	0.01\\
90.23	0.01\\
90.24	0.01\\
90.25	0.01\\
90.26	0.01\\
90.27	0.01\\
90.28	0.01\\
90.29	0.01\\
90.3	0.01\\
90.31	0.01\\
90.32	0.01\\
90.33	0.01\\
90.34	0.01\\
90.35	0.01\\
90.36	0.01\\
90.37	0.01\\
90.38	0.01\\
90.39	0.01\\
90.4	0.01\\
90.41	0.01\\
90.42	0.01\\
90.43	0.01\\
90.44	0.01\\
90.45	0.01\\
90.46	0.01\\
90.47	0.01\\
90.48	0.01\\
90.49	0.01\\
90.5	0.01\\
90.51	0.01\\
90.52	0.01\\
90.53	0.01\\
90.54	0.01\\
90.55	0.01\\
90.56	0.01\\
90.57	0.01\\
90.58	0.01\\
90.59	0.01\\
90.6	0.01\\
90.61	0.01\\
90.62	0.01\\
90.63	0.01\\
90.64	0.01\\
90.65	0.01\\
90.66	0.01\\
90.67	0.01\\
90.68	0.01\\
90.69	0.01\\
90.7	0.01\\
90.71	0.01\\
90.72	0.01\\
90.73	0.01\\
90.74	0.01\\
90.75	0.01\\
90.76	0.01\\
90.77	0.01\\
90.78	0.01\\
90.79	0.01\\
90.8	0.01\\
90.81	0.01\\
90.82	0.01\\
90.83	0.01\\
90.84	0.01\\
90.85	0.01\\
90.86	0.01\\
90.87	0.01\\
90.88	0.01\\
90.89	0.01\\
90.9	0.01\\
90.91	0.01\\
90.92	0.01\\
90.93	0.01\\
90.94	0.01\\
90.95	0.01\\
90.96	0.01\\
90.97	0.01\\
90.98	0.01\\
90.99	0.01\\
91	0.01\\
91.01	0.01\\
91.02	0.01\\
91.03	0.01\\
91.04	0.01\\
91.05	0.01\\
91.06	0.01\\
91.07	0.01\\
91.08	0.01\\
91.09	0.01\\
91.1	0.01\\
91.11	0.01\\
91.12	0.01\\
91.13	0.01\\
91.14	0.01\\
91.15	0.01\\
91.16	0.01\\
91.17	0.01\\
91.18	0.01\\
91.19	0.01\\
91.2	0.01\\
91.21	0.01\\
91.22	0.01\\
91.23	0.01\\
91.24	0.01\\
91.25	0.01\\
91.26	0.01\\
91.27	0.01\\
91.28	0.01\\
91.29	0.01\\
91.3	0.01\\
91.31	0.01\\
91.32	0.01\\
91.33	0.01\\
91.34	0.01\\
91.35	0.01\\
91.36	0.01\\
91.37	0.01\\
91.38	0.01\\
91.39	0.01\\
91.4	0.01\\
91.41	0.01\\
91.42	0.01\\
91.43	0.01\\
91.44	0.01\\
91.45	0.01\\
91.46	0.01\\
91.47	0.01\\
91.48	0.01\\
91.49	0.01\\
91.5	0.01\\
91.51	0.01\\
91.52	0.01\\
91.53	0.01\\
91.54	0.01\\
91.55	0.01\\
91.56	0.01\\
91.57	0.01\\
91.58	0.01\\
91.59	0.01\\
91.6	0.01\\
91.61	0.01\\
91.62	0.01\\
91.63	0.01\\
91.64	0.01\\
91.65	0.01\\
91.66	0.01\\
91.67	0.01\\
91.68	0.01\\
91.69	0.01\\
91.7	0.01\\
91.71	0.01\\
91.72	0.01\\
91.73	0.01\\
91.74	0.01\\
91.75	0.01\\
91.76	0.01\\
91.77	0.01\\
91.78	0.01\\
91.79	0.01\\
91.8	0.01\\
91.81	0.01\\
91.82	0.01\\
91.83	0.01\\
91.84	0.01\\
91.85	0.01\\
91.86	0.01\\
91.87	0.01\\
91.88	0.01\\
91.89	0.01\\
91.9	0.01\\
91.91	0.01\\
91.92	0.01\\
91.93	0.01\\
91.94	0.01\\
91.95	0.01\\
91.96	0.01\\
91.97	0.01\\
91.98	0.01\\
91.99	0.01\\
92	0.01\\
92.01	0.01\\
92.02	0.01\\
92.03	0.01\\
92.04	0.01\\
92.05	0.01\\
92.06	0.01\\
92.07	0.01\\
92.08	0.01\\
92.09	0.01\\
92.1	0.01\\
92.11	0.01\\
92.12	0.01\\
92.13	0.01\\
92.14	0.01\\
92.15	0.01\\
92.16	0.01\\
92.17	0.01\\
92.18	0.01\\
92.19	0.01\\
92.2	0.01\\
92.21	0.01\\
92.22	0.01\\
92.23	0.01\\
92.24	0.01\\
92.25	0.01\\
92.26	0.01\\
92.27	0.01\\
92.28	0.01\\
92.29	0.01\\
92.3	0.01\\
92.31	0.01\\
92.32	0.01\\
92.33	0.01\\
92.34	0.01\\
92.35	0.01\\
92.36	0.01\\
92.37	0.01\\
92.38	0.01\\
92.39	0.01\\
92.4	0.01\\
92.41	0.01\\
92.42	0.01\\
92.43	0.01\\
92.44	0.01\\
92.45	0.01\\
92.46	0.01\\
92.47	0.01\\
92.48	0.01\\
92.49	0.01\\
92.5	0.01\\
92.51	0.01\\
92.52	0.01\\
92.53	0.01\\
92.54	0.01\\
92.55	0.01\\
92.56	0.01\\
92.57	0.01\\
92.58	0.01\\
92.59	0.01\\
92.6	0.01\\
92.61	0.01\\
92.62	0.01\\
92.63	0.01\\
92.64	0.01\\
92.65	0.01\\
92.66	0.01\\
92.67	0.01\\
92.68	0.01\\
92.69	0.01\\
92.7	0.01\\
92.71	0.01\\
92.72	0.01\\
92.73	0.01\\
92.74	0.01\\
92.75	0.01\\
92.76	0.01\\
92.77	0.01\\
92.78	0.01\\
92.79	0.01\\
92.8	0.01\\
92.81	0.01\\
92.82	0.01\\
92.83	0.01\\
92.84	0.01\\
92.85	0.01\\
92.86	0.01\\
92.87	0.01\\
92.88	0.01\\
92.89	0.01\\
92.9	0.01\\
92.91	0.01\\
92.92	0.01\\
92.93	0.01\\
92.94	0.01\\
92.95	0.01\\
92.96	0.01\\
92.97	0.01\\
92.98	0.01\\
92.99	0.01\\
93	0.01\\
93.01	0.01\\
93.02	0.01\\
93.03	0.01\\
93.04	0.01\\
93.05	0.01\\
93.06	0.01\\
93.07	0.01\\
93.08	0.01\\
93.09	0.01\\
93.1	0.01\\
93.11	0.01\\
93.12	0.01\\
93.13	0.01\\
93.14	0.01\\
93.15	0.01\\
93.16	0.01\\
93.17	0.01\\
93.18	0.01\\
93.19	0.01\\
93.2	0.01\\
93.21	0.01\\
93.22	0.01\\
93.23	0.01\\
93.24	0.01\\
93.25	0.01\\
93.26	0.01\\
93.27	0.01\\
93.28	0.01\\
93.29	0.01\\
93.3	0.01\\
93.31	0.01\\
93.32	0.01\\
93.33	0.01\\
93.34	0.01\\
93.35	0.01\\
93.36	0.01\\
93.37	0.01\\
93.38	0.01\\
93.39	0.01\\
93.4	0.01\\
93.41	0.01\\
93.42	0.01\\
93.43	0.01\\
93.44	0.01\\
93.45	0.01\\
93.46	0.01\\
93.47	0.01\\
93.48	0.01\\
93.49	0.01\\
93.5	0.01\\
93.51	0.01\\
93.52	0.01\\
93.53	0.01\\
93.54	0.01\\
93.55	0.01\\
93.56	0.01\\
93.57	0.01\\
93.58	0.01\\
93.59	0.01\\
93.6	0.01\\
93.61	0.01\\
93.62	0.01\\
93.63	0.01\\
93.64	0.01\\
93.65	0.01\\
93.66	0.01\\
93.67	0.01\\
93.68	0.01\\
93.69	0.01\\
93.7	0.01\\
93.71	0.01\\
93.72	0.01\\
93.73	0.01\\
93.74	0.01\\
93.75	0.01\\
93.76	0.01\\
93.77	0.01\\
93.78	0.01\\
93.79	0.01\\
93.8	0.01\\
93.81	0.01\\
93.82	0.01\\
93.83	0.01\\
93.84	0.01\\
93.85	0.01\\
93.86	0.01\\
93.87	0.01\\
93.88	0.01\\
93.89	0.01\\
93.9	0.01\\
93.91	0.01\\
93.92	0.01\\
93.93	0.01\\
93.94	0.01\\
93.95	0.01\\
93.96	0.01\\
93.97	0.01\\
93.98	0.01\\
93.99	0.01\\
94	0.01\\
94.01	0.01\\
94.02	0.01\\
94.03	0.01\\
94.04	0.01\\
94.05	0.01\\
94.06	0.01\\
94.07	0.01\\
94.08	0.01\\
94.09	0.01\\
94.1	0.01\\
94.11	0.01\\
94.12	0.01\\
94.13	0.01\\
94.14	0.01\\
94.15	0.01\\
94.16	0.01\\
94.17	0.01\\
94.18	0.01\\
94.19	0.01\\
94.2	0.01\\
94.21	0.01\\
94.22	0.01\\
94.23	0.01\\
94.24	0.01\\
94.25	0.01\\
94.26	0.01\\
94.27	0.01\\
94.28	0.01\\
94.29	0.01\\
94.3	0.01\\
94.31	0.01\\
94.32	0.01\\
94.33	0.01\\
94.34	0.01\\
94.35	0.01\\
94.36	0.01\\
94.37	0.01\\
94.38	0.01\\
94.39	0.01\\
94.4	0.01\\
94.41	0.01\\
94.42	0.01\\
94.43	0.01\\
94.44	0.01\\
94.45	0.01\\
94.46	0.01\\
94.47	0.01\\
94.48	0.01\\
94.49	0.01\\
94.5	0.01\\
94.51	0.01\\
94.52	0.01\\
94.53	0.01\\
94.54	0.01\\
94.55	0.01\\
94.56	0.01\\
94.57	0.01\\
94.58	0.01\\
94.59	0.01\\
94.6	0.01\\
94.61	0.01\\
94.62	0.01\\
94.63	0.01\\
94.64	0.01\\
94.65	0.01\\
94.66	0.01\\
94.67	0.01\\
94.68	0.01\\
94.69	0.01\\
94.7	0.01\\
94.71	0.01\\
94.72	0.01\\
94.73	0.01\\
94.74	0.01\\
94.75	0.01\\
94.76	0.01\\
94.77	0.01\\
94.78	0.01\\
94.79	0.01\\
94.8	0.01\\
94.81	0.01\\
94.82	0.01\\
94.83	0.01\\
94.84	0.01\\
94.85	0.01\\
94.86	0.01\\
94.87	0.01\\
94.88	0.01\\
94.89	0.01\\
94.9	0.01\\
94.91	0.01\\
94.92	0.01\\
94.93	0.01\\
94.94	0.01\\
94.95	0.01\\
94.96	0.01\\
94.97	0.01\\
94.98	0.01\\
94.99	0.01\\
95	0.01\\
95.01	0.01\\
95.02	0.01\\
95.03	0.01\\
95.04	0.01\\
95.05	0.01\\
95.06	0.01\\
95.07	0.01\\
95.08	0.01\\
95.09	0.01\\
95.1	0.01\\
95.11	0.01\\
95.12	0.01\\
95.13	0.01\\
95.14	0.01\\
95.15	0.01\\
95.16	0.01\\
95.17	0.01\\
95.18	0.01\\
95.19	0.01\\
95.2	0.01\\
95.21	0.01\\
95.22	0.01\\
95.23	0.01\\
95.24	0.01\\
95.25	0.01\\
95.26	0.01\\
95.27	0.01\\
95.28	0.01\\
95.29	0.01\\
95.3	0.01\\
95.31	0.01\\
95.32	0.01\\
95.33	0.01\\
95.34	0.01\\
95.35	0.01\\
95.36	0.01\\
95.37	0.01\\
95.38	0.01\\
95.39	0.01\\
95.4	0.01\\
95.41	0.01\\
95.42	0.01\\
95.43	0.01\\
95.44	0.01\\
95.45	0.01\\
95.46	0.01\\
95.47	0.01\\
95.48	0.01\\
95.49	0.01\\
95.5	0.01\\
95.51	0.01\\
95.52	0.01\\
95.53	0.01\\
95.54	0.01\\
95.55	0.01\\
95.56	0.01\\
95.57	0.01\\
95.58	0.01\\
95.59	0.01\\
95.6	0.01\\
95.61	0.01\\
95.62	0.01\\
95.63	0.01\\
95.64	0.01\\
95.65	0.01\\
95.66	0.01\\
95.67	0.01\\
95.68	0.01\\
95.69	0.01\\
95.7	0.01\\
95.71	0.01\\
95.72	0.01\\
95.73	0.01\\
95.74	0.01\\
95.75	0.01\\
95.76	0.01\\
95.77	0.01\\
95.78	0.01\\
95.79	0.01\\
95.8	0.01\\
95.81	0.01\\
95.82	0.01\\
95.83	0.01\\
95.84	0.01\\
95.85	0.01\\
95.86	0.01\\
95.87	0.01\\
95.88	0.01\\
95.89	0.01\\
95.9	0.01\\
95.91	0.01\\
95.92	0.01\\
95.93	0.01\\
95.94	0.01\\
95.95	0.01\\
95.96	0.01\\
95.97	0.01\\
95.98	0.01\\
95.99	0.01\\
96	0.01\\
96.01	0.01\\
96.02	0.01\\
96.03	0.01\\
96.04	0.01\\
96.05	0.01\\
96.06	0.01\\
96.07	0.01\\
96.08	0.01\\
96.09	0.01\\
96.1	0.01\\
96.11	0.01\\
96.12	0.01\\
96.13	0.01\\
96.14	0.01\\
96.15	0.01\\
96.16	0.01\\
96.17	0.01\\
96.18	0.01\\
96.19	0.01\\
96.2	0.01\\
96.21	0.01\\
96.22	0.01\\
96.23	0.01\\
96.24	0.01\\
96.25	0.01\\
96.26	0.01\\
96.27	0.01\\
96.28	0.01\\
96.29	0.01\\
96.3	0.01\\
96.31	0.01\\
96.32	0.01\\
96.33	0.01\\
96.34	0.01\\
96.35	0.01\\
96.36	0.01\\
96.37	0.01\\
96.38	0.01\\
96.39	0.01\\
96.4	0.01\\
96.41	0.01\\
96.42	0.01\\
96.43	0.01\\
96.44	0.01\\
96.45	0.01\\
96.46	0.01\\
96.47	0.01\\
96.48	0.01\\
96.49	0.01\\
96.5	0.01\\
96.51	0.01\\
96.52	0.01\\
96.53	0.01\\
96.54	0.01\\
96.55	0.01\\
96.56	0.01\\
96.57	0.01\\
96.58	0.01\\
96.59	0.01\\
96.6	0.01\\
96.61	0.01\\
96.62	0.01\\
96.63	0.01\\
96.64	0.01\\
96.65	0.01\\
96.66	0.01\\
96.67	0.01\\
96.68	0.01\\
96.69	0.01\\
96.7	0.01\\
96.71	0.01\\
96.72	0.01\\
96.73	0.01\\
96.74	0.01\\
96.75	0.01\\
96.76	0.01\\
96.77	0.01\\
96.78	0.01\\
96.79	0.01\\
96.8	0.01\\
96.81	0.01\\
96.82	0.01\\
96.83	0.01\\
96.84	0.01\\
96.85	0.01\\
96.86	0.01\\
96.87	0.01\\
96.88	0.01\\
96.89	0.01\\
96.9	0.01\\
96.91	0.01\\
96.92	0.01\\
96.93	0.01\\
96.94	0.01\\
96.95	0.01\\
96.96	0.01\\
96.97	0.01\\
96.98	0.01\\
96.99	0.01\\
97	0.01\\
97.01	0.01\\
97.02	0.01\\
97.03	0.01\\
97.04	0.01\\
97.05	0.01\\
97.06	0.01\\
97.07	0.01\\
97.08	0.01\\
97.09	0.01\\
97.1	0.01\\
97.11	0.01\\
97.12	0.01\\
97.13	0.01\\
97.14	0.01\\
97.15	0.01\\
97.16	0.01\\
97.17	0.01\\
97.18	0.01\\
97.19	0.01\\
97.2	0.01\\
97.21	0.01\\
97.22	0.01\\
97.23	0.01\\
97.24	0.01\\
97.25	0.01\\
97.26	0.01\\
97.27	0.01\\
97.28	0.01\\
97.29	0.01\\
97.3	0.01\\
97.31	0.01\\
97.32	0.01\\
97.33	0.01\\
97.34	0.01\\
97.35	0.01\\
97.36	0.01\\
97.37	0.01\\
97.38	0.01\\
97.39	0.01\\
97.4	0.01\\
97.41	0.01\\
97.42	0.01\\
97.43	0.01\\
97.44	0.01\\
97.45	0.01\\
97.46	0.01\\
97.47	0.01\\
97.48	0.01\\
97.49	0.01\\
97.5	0.01\\
97.51	0.01\\
97.52	0.01\\
97.53	0.01\\
97.54	0.01\\
97.55	0.01\\
97.56	0.01\\
97.57	0.01\\
97.58	0.01\\
97.59	0.01\\
97.6	0.01\\
97.61	0.01\\
97.62	0.01\\
97.63	0.01\\
97.64	0.01\\
97.65	0.01\\
97.66	0.01\\
97.67	0.01\\
97.68	0.01\\
97.69	0.01\\
97.7	0.01\\
97.71	0.01\\
97.72	0.01\\
97.73	0.01\\
97.74	0.01\\
97.75	0.01\\
97.76	0.01\\
97.77	0.01\\
97.78	0.01\\
97.79	0.01\\
97.8	0.01\\
97.81	0.01\\
97.82	0.01\\
97.83	0.01\\
97.84	0.01\\
97.85	0.01\\
97.86	0.01\\
97.87	0.01\\
97.88	0.01\\
97.89	0.01\\
97.9	0.01\\
97.91	0.01\\
97.92	0.01\\
97.93	0.01\\
97.94	0.01\\
97.95	0.01\\
97.96	0.01\\
97.97	0.01\\
97.98	0.01\\
97.99	0.01\\
98	0.01\\
98.01	0.01\\
98.02	0.01\\
98.03	0.01\\
98.04	0.01\\
98.05	0.01\\
98.06	0.01\\
98.07	0.01\\
98.08	0.01\\
98.09	0.01\\
98.1	0.01\\
98.11	0.01\\
98.12	0.01\\
98.13	0.01\\
98.14	0.01\\
98.15	0.01\\
98.16	0.01\\
98.17	0.01\\
98.18	0.01\\
98.19	0.01\\
98.2	0.01\\
98.21	0.01\\
98.22	0.01\\
98.23	0.01\\
98.24	0.01\\
98.25	0.01\\
98.26	0.01\\
98.27	0.01\\
98.28	0.01\\
98.29	0.01\\
98.3	0.01\\
98.31	0.01\\
98.32	0.01\\
98.33	0.01\\
98.34	0.01\\
98.35	0.01\\
98.36	0.01\\
98.37	0.01\\
98.38	0.01\\
98.39	0.01\\
98.4	0.01\\
98.41	0.01\\
98.42	0.01\\
98.43	0.01\\
98.44	0.01\\
98.45	0.01\\
98.46	0.01\\
98.47	0.01\\
98.48	0.01\\
98.49	0.01\\
98.5	0.01\\
98.51	0.01\\
98.52	0.01\\
98.53	0.01\\
98.54	0.01\\
98.55	0.01\\
98.56	0.01\\
98.57	0.01\\
98.58	0.01\\
98.59	0.01\\
98.6	0.01\\
98.61	0.01\\
98.62	0.01\\
98.63	0.01\\
98.64	0.01\\
98.65	0.01\\
98.66	0.01\\
98.67	0.01\\
98.68	0.01\\
98.69	0.00996545564155078\\
98.7	0.00991968012276111\\
98.71	0.00987359391987314\\
98.72	0.00982719422543803\\
98.73	0.00978047821033394\\
98.74	0.00973344557609966\\
98.75	0.00968609552470258\\
98.76	0.00963842520565945\\
98.77	0.00959043173792486\\
98.78	0.00954211220948754\\
98.79	0.00949347115953559\\
98.8	0.00944450788058834\\
98.81	0.00939521952608289\\
98.82	0.00934560321907486\\
98.83	0.00929565605183323\\
98.84	0.00924537508542838\\
98.85	0.00919475734931255\\
98.86	0.0091437998408936\\
98.87	0.00909249952510205\\
98.88	0.00904085333395115\\
98.89	0.00898885816609\\
98.9	0.00893651088634938\\
98.91	0.00888380832528046\\
98.92	0.00883074727868599\\
98.93	0.00877732450714391\\
98.94	0.00872353673552342\\
98.95	0.00866938065249306\\
98.96	0.00861485291002093\\
98.97	0.00855995012286675\\
98.98	0.00850466886806567\\
98.99	0.00844900568440366\\
99	0.00839295707188433\\
99.01	0.00833651949118704\\
99.02	0.00827968936311606\\
99.03	0.00822246306804074\\
99.04	0.0081648369453264\\
99.05	0.00810680729275582\\
99.06	0.0080483703659412\\
99.07	0.00798952237772625\\
99.08	0.0079302594974818\\
99.09	0.00787057785040668\\
99.1	0.00781047351689714\\
99.11	0.00774994253190569\\
99.12	0.00768898088355802\\
99.13	0.00762758451302687\\
99.14	0.00756574931403476\\
99.15	0.00750347113216843\\
99.16	0.00744074576418163\\
99.17	0.0073775689572861\\
99.18	0.00731393640843054\\
99.19	0.00724984376356714\\
99.2	0.00718528661690569\\
99.21	0.00712026051015475\\
99.22	0.00705476093174978\\
99.23	0.00698878331606789\\
99.24	0.00692232304262898\\
99.25	0.00685537543528293\\
99.26	0.0067879357613825\\
99.27	0.00671999923094184\\
99.28	0.00665156099578\\
99.29	0.00658261614864939\\
99.3	0.00651315972234868\\
99.31	0.00644318668881984\\
99.32	0.00637269195822908\\
99.33	0.00630167037803112\\
99.34	0.00623011673201661\\
99.35	0.00615802573934219\\
99.36	0.00608539205354286\\
99.37	0.00601221026152622\\
99.38	0.00593847488254819\\
99.39	0.00586418036716968\\
99.4	0.00578932109619394\\
99.41	0.00571389137958388\\
99.42	0.00563788545535915\\
99.43	0.00556129748847227\\
99.44	0.00548412156966339\\
99.45	0.00540635171429312\\
99.46	0.005327981861153\\
99.47	0.00524900587125281\\
99.48	0.00516941752658434\\
99.49	0.00508921052886098\\
99.5	0.0050083784982324\\
99.51	0.00492691497197378\\
99.52	0.00484481340314884\\
99.53	0.0047620671592461\\
99.54	0.00467866952078743\\
99.55	0.0045946136799084\\
99.56	0.00450989273890942\\
99.57	0.00442449970877707\\
99.58	0.00433842750767462\\
99.59	0.00425166895940098\\
99.6	0.00416421679181713\\
99.61	0.00407606363523917\\
99.62	0.00398720202079703\\
99.63	0.00389762437875774\\
99.64	0.00380732303681222\\
99.65	0.00371629021832458\\
99.66	0.0036245180405427\\
99.67	0.003531998512769\\
99.68	0.00343872353449005\\
99.69	0.00334468489346397\\
99.7	0.00324987426376405\\
99.71	0.00315428320370294\\
99.72	0.00305790315377877\\
99.73	0.00296072543457911\\
99.74	0.00286274124463556\\
99.75	0.00276394165822706\\
99.76	0.00266431762313054\\
99.77	0.00256385995831677\\
99.78	0.00246255935158997\\
99.79	0.00236040635716888\\
99.8	0.00225739139320756\\
99.81	0.00215350473925367\\
99.82	0.00204873653364207\\
99.83	0.00194307677082145\\
99.84	0.00183651529861168\\
99.85	0.00172904181538915\\
99.86	0.00162064586719775\\
99.87	0.00151131684478261\\
99.88	0.00140104398054366\\
99.89	0.00128981634540623\\
99.9	0.0011776228456053\\
99.91	0.00106445221938031\\
99.92	0.000950293033576939\\
99.93	0.00083513368015232\\
99.94	0.000718962372579865\\
99.95	0.000601767142149755\\
99.96	0.000483535834160888\\
99.97	0.000364256103999944\\
99.98	0.000243915413102962\\
99.99	0.000122501024794594\\
100	0\\
};
\addlegendentry{$q=1$};

\addplot [color=red,solid,forget plot]
  table[row sep=crcr]{%
0.01	0.01\\
0.02	0.01\\
0.03	0.01\\
0.04	0.01\\
0.05	0.01\\
0.06	0.01\\
0.07	0.01\\
0.08	0.01\\
0.09	0.01\\
0.1	0.01\\
0.11	0.01\\
0.12	0.01\\
0.13	0.01\\
0.14	0.01\\
0.15	0.01\\
0.16	0.01\\
0.17	0.01\\
0.18	0.01\\
0.19	0.01\\
0.2	0.01\\
0.21	0.01\\
0.22	0.01\\
0.23	0.01\\
0.24	0.01\\
0.25	0.01\\
0.26	0.01\\
0.27	0.01\\
0.28	0.01\\
0.29	0.01\\
0.3	0.01\\
0.31	0.01\\
0.32	0.01\\
0.33	0.01\\
0.34	0.01\\
0.35	0.01\\
0.36	0.01\\
0.37	0.01\\
0.38	0.01\\
0.39	0.01\\
0.4	0.01\\
0.41	0.01\\
0.42	0.01\\
0.43	0.01\\
0.44	0.01\\
0.45	0.01\\
0.46	0.01\\
0.47	0.01\\
0.48	0.01\\
0.49	0.01\\
0.5	0.01\\
0.51	0.01\\
0.52	0.01\\
0.53	0.01\\
0.54	0.01\\
0.55	0.01\\
0.56	0.01\\
0.57	0.01\\
0.58	0.01\\
0.59	0.01\\
0.6	0.01\\
0.61	0.01\\
0.62	0.01\\
0.63	0.01\\
0.64	0.01\\
0.65	0.01\\
0.66	0.01\\
0.67	0.01\\
0.68	0.01\\
0.69	0.01\\
0.7	0.01\\
0.71	0.01\\
0.72	0.01\\
0.73	0.01\\
0.74	0.01\\
0.75	0.01\\
0.76	0.01\\
0.77	0.01\\
0.78	0.01\\
0.79	0.01\\
0.8	0.01\\
0.81	0.01\\
0.82	0.01\\
0.83	0.01\\
0.84	0.01\\
0.85	0.01\\
0.86	0.01\\
0.87	0.01\\
0.88	0.01\\
0.89	0.01\\
0.9	0.01\\
0.91	0.01\\
0.92	0.01\\
0.93	0.01\\
0.94	0.01\\
0.95	0.01\\
0.96	0.01\\
0.97	0.01\\
0.98	0.01\\
0.99	0.01\\
1	0.01\\
1.01	0.01\\
1.02	0.01\\
1.03	0.01\\
1.04	0.01\\
1.05	0.01\\
1.06	0.01\\
1.07	0.01\\
1.08	0.01\\
1.09	0.01\\
1.1	0.01\\
1.11	0.01\\
1.12	0.01\\
1.13	0.01\\
1.14	0.01\\
1.15	0.01\\
1.16	0.01\\
1.17	0.01\\
1.18	0.01\\
1.19	0.01\\
1.2	0.01\\
1.21	0.01\\
1.22	0.01\\
1.23	0.01\\
1.24	0.01\\
1.25	0.01\\
1.26	0.01\\
1.27	0.01\\
1.28	0.01\\
1.29	0.01\\
1.3	0.01\\
1.31	0.01\\
1.32	0.01\\
1.33	0.01\\
1.34	0.01\\
1.35	0.01\\
1.36	0.01\\
1.37	0.01\\
1.38	0.01\\
1.39	0.01\\
1.4	0.01\\
1.41	0.01\\
1.42	0.01\\
1.43	0.01\\
1.44	0.01\\
1.45	0.01\\
1.46	0.01\\
1.47	0.01\\
1.48	0.01\\
1.49	0.01\\
1.5	0.01\\
1.51	0.01\\
1.52	0.01\\
1.53	0.01\\
1.54	0.01\\
1.55	0.01\\
1.56	0.01\\
1.57	0.01\\
1.58	0.01\\
1.59	0.01\\
1.6	0.01\\
1.61	0.01\\
1.62	0.01\\
1.63	0.01\\
1.64	0.01\\
1.65	0.01\\
1.66	0.01\\
1.67	0.01\\
1.68	0.01\\
1.69	0.01\\
1.7	0.01\\
1.71	0.01\\
1.72	0.01\\
1.73	0.01\\
1.74	0.01\\
1.75	0.01\\
1.76	0.01\\
1.77	0.01\\
1.78	0.01\\
1.79	0.01\\
1.8	0.01\\
1.81	0.01\\
1.82	0.01\\
1.83	0.01\\
1.84	0.01\\
1.85	0.01\\
1.86	0.01\\
1.87	0.01\\
1.88	0.01\\
1.89	0.01\\
1.9	0.01\\
1.91	0.01\\
1.92	0.01\\
1.93	0.01\\
1.94	0.01\\
1.95	0.01\\
1.96	0.01\\
1.97	0.01\\
1.98	0.01\\
1.99	0.01\\
2	0.01\\
2.01	0.01\\
2.02	0.01\\
2.03	0.01\\
2.04	0.01\\
2.05	0.01\\
2.06	0.01\\
2.07	0.01\\
2.08	0.01\\
2.09	0.01\\
2.1	0.01\\
2.11	0.01\\
2.12	0.01\\
2.13	0.01\\
2.14	0.01\\
2.15	0.01\\
2.16	0.01\\
2.17	0.01\\
2.18	0.01\\
2.19	0.01\\
2.2	0.01\\
2.21	0.01\\
2.22	0.01\\
2.23	0.01\\
2.24	0.01\\
2.25	0.01\\
2.26	0.01\\
2.27	0.01\\
2.28	0.01\\
2.29	0.01\\
2.3	0.01\\
2.31	0.01\\
2.32	0.01\\
2.33	0.01\\
2.34	0.01\\
2.35	0.01\\
2.36	0.01\\
2.37	0.01\\
2.38	0.01\\
2.39	0.01\\
2.4	0.01\\
2.41	0.01\\
2.42	0.01\\
2.43	0.01\\
2.44	0.01\\
2.45	0.01\\
2.46	0.01\\
2.47	0.01\\
2.48	0.01\\
2.49	0.01\\
2.5	0.01\\
2.51	0.01\\
2.52	0.01\\
2.53	0.01\\
2.54	0.01\\
2.55	0.01\\
2.56	0.01\\
2.57	0.01\\
2.58	0.01\\
2.59	0.01\\
2.6	0.01\\
2.61	0.01\\
2.62	0.01\\
2.63	0.01\\
2.64	0.01\\
2.65	0.01\\
2.66	0.01\\
2.67	0.01\\
2.68	0.01\\
2.69	0.01\\
2.7	0.01\\
2.71	0.01\\
2.72	0.01\\
2.73	0.01\\
2.74	0.01\\
2.75	0.01\\
2.76	0.01\\
2.77	0.01\\
2.78	0.01\\
2.79	0.01\\
2.8	0.01\\
2.81	0.01\\
2.82	0.01\\
2.83	0.01\\
2.84	0.01\\
2.85	0.01\\
2.86	0.01\\
2.87	0.01\\
2.88	0.01\\
2.89	0.01\\
2.9	0.01\\
2.91	0.01\\
2.92	0.01\\
2.93	0.01\\
2.94	0.01\\
2.95	0.01\\
2.96	0.01\\
2.97	0.01\\
2.98	0.01\\
2.99	0.01\\
3	0.01\\
3.01	0.01\\
3.02	0.01\\
3.03	0.01\\
3.04	0.01\\
3.05	0.01\\
3.06	0.01\\
3.07	0.01\\
3.08	0.01\\
3.09	0.01\\
3.1	0.01\\
3.11	0.01\\
3.12	0.01\\
3.13	0.01\\
3.14	0.01\\
3.15	0.01\\
3.16	0.01\\
3.17	0.01\\
3.18	0.01\\
3.19	0.01\\
3.2	0.01\\
3.21	0.01\\
3.22	0.01\\
3.23	0.01\\
3.24	0.01\\
3.25	0.01\\
3.26	0.01\\
3.27	0.01\\
3.28	0.01\\
3.29	0.01\\
3.3	0.01\\
3.31	0.01\\
3.32	0.01\\
3.33	0.01\\
3.34	0.01\\
3.35	0.01\\
3.36	0.01\\
3.37	0.01\\
3.38	0.01\\
3.39	0.01\\
3.4	0.01\\
3.41	0.01\\
3.42	0.01\\
3.43	0.01\\
3.44	0.01\\
3.45	0.01\\
3.46	0.01\\
3.47	0.01\\
3.48	0.01\\
3.49	0.01\\
3.5	0.01\\
3.51	0.01\\
3.52	0.01\\
3.53	0.01\\
3.54	0.01\\
3.55	0.01\\
3.56	0.01\\
3.57	0.01\\
3.58	0.01\\
3.59	0.01\\
3.6	0.01\\
3.61	0.01\\
3.62	0.01\\
3.63	0.01\\
3.64	0.01\\
3.65	0.01\\
3.66	0.01\\
3.67	0.01\\
3.68	0.01\\
3.69	0.01\\
3.7	0.01\\
3.71	0.01\\
3.72	0.01\\
3.73	0.01\\
3.74	0.01\\
3.75	0.01\\
3.76	0.01\\
3.77	0.01\\
3.78	0.01\\
3.79	0.01\\
3.8	0.01\\
3.81	0.01\\
3.82	0.01\\
3.83	0.01\\
3.84	0.01\\
3.85	0.01\\
3.86	0.01\\
3.87	0.01\\
3.88	0.01\\
3.89	0.01\\
3.9	0.01\\
3.91	0.01\\
3.92	0.01\\
3.93	0.01\\
3.94	0.01\\
3.95	0.01\\
3.96	0.01\\
3.97	0.01\\
3.98	0.01\\
3.99	0.01\\
4	0.01\\
4.01	0.01\\
4.02	0.01\\
4.03	0.01\\
4.04	0.01\\
4.05	0.01\\
4.06	0.01\\
4.07	0.01\\
4.08	0.01\\
4.09	0.01\\
4.1	0.01\\
4.11	0.01\\
4.12	0.01\\
4.13	0.01\\
4.14	0.01\\
4.15	0.01\\
4.16	0.01\\
4.17	0.01\\
4.18	0.01\\
4.19	0.01\\
4.2	0.01\\
4.21	0.01\\
4.22	0.01\\
4.23	0.01\\
4.24	0.01\\
4.25	0.01\\
4.26	0.01\\
4.27	0.01\\
4.28	0.01\\
4.29	0.01\\
4.3	0.01\\
4.31	0.01\\
4.32	0.01\\
4.33	0.01\\
4.34	0.01\\
4.35	0.01\\
4.36	0.01\\
4.37	0.01\\
4.38	0.01\\
4.39	0.01\\
4.4	0.01\\
4.41	0.01\\
4.42	0.01\\
4.43	0.01\\
4.44	0.01\\
4.45	0.01\\
4.46	0.01\\
4.47	0.01\\
4.48	0.01\\
4.49	0.01\\
4.5	0.01\\
4.51	0.01\\
4.52	0.01\\
4.53	0.01\\
4.54	0.01\\
4.55	0.01\\
4.56	0.01\\
4.57	0.01\\
4.58	0.01\\
4.59	0.01\\
4.6	0.01\\
4.61	0.01\\
4.62	0.01\\
4.63	0.01\\
4.64	0.01\\
4.65	0.01\\
4.66	0.01\\
4.67	0.01\\
4.68	0.01\\
4.69	0.01\\
4.7	0.01\\
4.71	0.01\\
4.72	0.01\\
4.73	0.01\\
4.74	0.01\\
4.75	0.01\\
4.76	0.01\\
4.77	0.01\\
4.78	0.01\\
4.79	0.01\\
4.8	0.01\\
4.81	0.01\\
4.82	0.01\\
4.83	0.01\\
4.84	0.01\\
4.85	0.01\\
4.86	0.01\\
4.87	0.01\\
4.88	0.01\\
4.89	0.01\\
4.9	0.01\\
4.91	0.01\\
4.92	0.01\\
4.93	0.01\\
4.94	0.01\\
4.95	0.01\\
4.96	0.01\\
4.97	0.01\\
4.98	0.01\\
4.99	0.01\\
5	0.01\\
5.01	0.01\\
5.02	0.01\\
5.03	0.01\\
5.04	0.01\\
5.05	0.01\\
5.06	0.01\\
5.07	0.01\\
5.08	0.01\\
5.09	0.01\\
5.1	0.01\\
5.11	0.01\\
5.12	0.01\\
5.13	0.01\\
5.14	0.01\\
5.15	0.01\\
5.16	0.01\\
5.17	0.01\\
5.18	0.01\\
5.19	0.01\\
5.2	0.01\\
5.21	0.01\\
5.22	0.01\\
5.23	0.01\\
5.24	0.01\\
5.25	0.01\\
5.26	0.01\\
5.27	0.01\\
5.28	0.01\\
5.29	0.01\\
5.3	0.01\\
5.31	0.01\\
5.32	0.01\\
5.33	0.01\\
5.34	0.01\\
5.35	0.01\\
5.36	0.01\\
5.37	0.01\\
5.38	0.01\\
5.39	0.01\\
5.4	0.01\\
5.41	0.01\\
5.42	0.01\\
5.43	0.01\\
5.44	0.01\\
5.45	0.01\\
5.46	0.01\\
5.47	0.01\\
5.48	0.01\\
5.49	0.01\\
5.5	0.01\\
5.51	0.01\\
5.52	0.01\\
5.53	0.01\\
5.54	0.01\\
5.55	0.01\\
5.56	0.01\\
5.57	0.01\\
5.58	0.01\\
5.59	0.01\\
5.6	0.01\\
5.61	0.01\\
5.62	0.01\\
5.63	0.01\\
5.64	0.01\\
5.65	0.01\\
5.66	0.01\\
5.67	0.01\\
5.68	0.01\\
5.69	0.01\\
5.7	0.01\\
5.71	0.01\\
5.72	0.01\\
5.73	0.01\\
5.74	0.01\\
5.75	0.01\\
5.76	0.01\\
5.77	0.01\\
5.78	0.01\\
5.79	0.01\\
5.8	0.01\\
5.81	0.01\\
5.82	0.01\\
5.83	0.01\\
5.84	0.01\\
5.85	0.01\\
5.86	0.01\\
5.87	0.01\\
5.88	0.01\\
5.89	0.01\\
5.9	0.01\\
5.91	0.01\\
5.92	0.01\\
5.93	0.01\\
5.94	0.01\\
5.95	0.01\\
5.96	0.01\\
5.97	0.01\\
5.98	0.01\\
5.99	0.01\\
6	0.01\\
6.01	0.01\\
6.02	0.01\\
6.03	0.01\\
6.04	0.01\\
6.05	0.01\\
6.06	0.01\\
6.07	0.01\\
6.08	0.01\\
6.09	0.01\\
6.1	0.01\\
6.11	0.01\\
6.12	0.01\\
6.13	0.01\\
6.14	0.01\\
6.15	0.01\\
6.16	0.01\\
6.17	0.01\\
6.18	0.01\\
6.19	0.01\\
6.2	0.01\\
6.21	0.01\\
6.22	0.01\\
6.23	0.01\\
6.24	0.01\\
6.25	0.01\\
6.26	0.01\\
6.27	0.01\\
6.28	0.01\\
6.29	0.01\\
6.3	0.01\\
6.31	0.01\\
6.32	0.01\\
6.33	0.01\\
6.34	0.01\\
6.35	0.01\\
6.36	0.01\\
6.37	0.01\\
6.38	0.01\\
6.39	0.01\\
6.4	0.01\\
6.41	0.01\\
6.42	0.01\\
6.43	0.01\\
6.44	0.01\\
6.45	0.01\\
6.46	0.01\\
6.47	0.01\\
6.48	0.01\\
6.49	0.01\\
6.5	0.01\\
6.51	0.01\\
6.52	0.01\\
6.53	0.01\\
6.54	0.01\\
6.55	0.01\\
6.56	0.01\\
6.57	0.01\\
6.58	0.01\\
6.59	0.01\\
6.6	0.01\\
6.61	0.01\\
6.62	0.01\\
6.63	0.01\\
6.64	0.01\\
6.65	0.01\\
6.66	0.01\\
6.67	0.01\\
6.68	0.01\\
6.69	0.01\\
6.7	0.01\\
6.71	0.01\\
6.72	0.01\\
6.73	0.01\\
6.74	0.01\\
6.75	0.01\\
6.76	0.01\\
6.77	0.01\\
6.78	0.01\\
6.79	0.01\\
6.8	0.01\\
6.81	0.01\\
6.82	0.01\\
6.83	0.01\\
6.84	0.01\\
6.85	0.01\\
6.86	0.01\\
6.87	0.01\\
6.88	0.01\\
6.89	0.01\\
6.9	0.01\\
6.91	0.01\\
6.92	0.01\\
6.93	0.01\\
6.94	0.01\\
6.95	0.01\\
6.96	0.01\\
6.97	0.01\\
6.98	0.01\\
6.99	0.01\\
7	0.01\\
7.01	0.01\\
7.02	0.01\\
7.03	0.01\\
7.04	0.01\\
7.05	0.01\\
7.06	0.01\\
7.07	0.01\\
7.08	0.01\\
7.09	0.01\\
7.1	0.01\\
7.11	0.01\\
7.12	0.01\\
7.13	0.01\\
7.14	0.01\\
7.15	0.01\\
7.16	0.01\\
7.17	0.01\\
7.18	0.01\\
7.19	0.01\\
7.2	0.01\\
7.21	0.01\\
7.22	0.01\\
7.23	0.01\\
7.24	0.01\\
7.25	0.01\\
7.26	0.01\\
7.27	0.01\\
7.28	0.01\\
7.29	0.01\\
7.3	0.01\\
7.31	0.01\\
7.32	0.01\\
7.33	0.01\\
7.34	0.01\\
7.35	0.01\\
7.36	0.01\\
7.37	0.01\\
7.38	0.01\\
7.39	0.01\\
7.4	0.01\\
7.41	0.01\\
7.42	0.01\\
7.43	0.01\\
7.44	0.01\\
7.45	0.01\\
7.46	0.01\\
7.47	0.01\\
7.48	0.01\\
7.49	0.01\\
7.5	0.01\\
7.51	0.01\\
7.52	0.01\\
7.53	0.01\\
7.54	0.01\\
7.55	0.01\\
7.56	0.01\\
7.57	0.01\\
7.58	0.01\\
7.59	0.01\\
7.6	0.01\\
7.61	0.01\\
7.62	0.01\\
7.63	0.01\\
7.64	0.01\\
7.65	0.01\\
7.66	0.01\\
7.67	0.01\\
7.68	0.01\\
7.69	0.01\\
7.7	0.01\\
7.71	0.01\\
7.72	0.01\\
7.73	0.01\\
7.74	0.01\\
7.75	0.01\\
7.76	0.01\\
7.77	0.01\\
7.78	0.01\\
7.79	0.01\\
7.8	0.01\\
7.81	0.01\\
7.82	0.01\\
7.83	0.01\\
7.84	0.01\\
7.85	0.01\\
7.86	0.01\\
7.87	0.01\\
7.88	0.01\\
7.89	0.01\\
7.9	0.01\\
7.91	0.01\\
7.92	0.01\\
7.93	0.01\\
7.94	0.01\\
7.95	0.01\\
7.96	0.01\\
7.97	0.01\\
7.98	0.01\\
7.99	0.01\\
8	0.01\\
8.01	0.01\\
8.02	0.01\\
8.03	0.01\\
8.04	0.01\\
8.05	0.01\\
8.06	0.01\\
8.07	0.01\\
8.08	0.01\\
8.09	0.01\\
8.1	0.01\\
8.11	0.01\\
8.12	0.01\\
8.13	0.01\\
8.14	0.01\\
8.15	0.01\\
8.16	0.01\\
8.17	0.01\\
8.18	0.01\\
8.19	0.01\\
8.2	0.01\\
8.21	0.01\\
8.22	0.01\\
8.23	0.01\\
8.24	0.01\\
8.25	0.01\\
8.26	0.01\\
8.27	0.01\\
8.28	0.01\\
8.29	0.01\\
8.3	0.01\\
8.31	0.01\\
8.32	0.01\\
8.33	0.01\\
8.34	0.01\\
8.35	0.01\\
8.36	0.01\\
8.37	0.01\\
8.38	0.01\\
8.39	0.01\\
8.4	0.01\\
8.41	0.01\\
8.42	0.01\\
8.43	0.01\\
8.44	0.01\\
8.45	0.01\\
8.46	0.01\\
8.47	0.01\\
8.48	0.01\\
8.49	0.01\\
8.5	0.01\\
8.51	0.01\\
8.52	0.01\\
8.53	0.01\\
8.54	0.01\\
8.55	0.01\\
8.56	0.01\\
8.57	0.01\\
8.58	0.01\\
8.59	0.01\\
8.6	0.01\\
8.61	0.01\\
8.62	0.01\\
8.63	0.01\\
8.64	0.01\\
8.65	0.01\\
8.66	0.01\\
8.67	0.01\\
8.68	0.01\\
8.69	0.01\\
8.7	0.01\\
8.71	0.01\\
8.72	0.01\\
8.73	0.01\\
8.74	0.01\\
8.75	0.01\\
8.76	0.01\\
8.77	0.01\\
8.78	0.01\\
8.79	0.01\\
8.8	0.01\\
8.81	0.01\\
8.82	0.01\\
8.83	0.01\\
8.84	0.01\\
8.85	0.01\\
8.86	0.01\\
8.87	0.01\\
8.88	0.01\\
8.89	0.01\\
8.9	0.01\\
8.91	0.01\\
8.92	0.01\\
8.93	0.01\\
8.94	0.01\\
8.95	0.01\\
8.96	0.01\\
8.97	0.01\\
8.98	0.01\\
8.99	0.01\\
9	0.01\\
9.01	0.01\\
9.02	0.01\\
9.03	0.01\\
9.04	0.01\\
9.05	0.01\\
9.06	0.01\\
9.07	0.01\\
9.08	0.01\\
9.09	0.01\\
9.1	0.01\\
9.11	0.01\\
9.12	0.01\\
9.13	0.01\\
9.14	0.01\\
9.15	0.01\\
9.16	0.01\\
9.17	0.01\\
9.18	0.01\\
9.19	0.01\\
9.2	0.01\\
9.21	0.01\\
9.22	0.01\\
9.23	0.01\\
9.24	0.01\\
9.25	0.01\\
9.26	0.01\\
9.27	0.01\\
9.28	0.01\\
9.29	0.01\\
9.3	0.01\\
9.31	0.01\\
9.32	0.01\\
9.33	0.01\\
9.34	0.01\\
9.35	0.01\\
9.36	0.01\\
9.37	0.01\\
9.38	0.01\\
9.39	0.01\\
9.4	0.01\\
9.41	0.01\\
9.42	0.01\\
9.43	0.01\\
9.44	0.01\\
9.45	0.01\\
9.46	0.01\\
9.47	0.01\\
9.48	0.01\\
9.49	0.01\\
9.5	0.01\\
9.51	0.01\\
9.52	0.01\\
9.53	0.01\\
9.54	0.01\\
9.55	0.01\\
9.56	0.01\\
9.57	0.01\\
9.58	0.01\\
9.59	0.01\\
9.6	0.01\\
9.61	0.01\\
9.62	0.01\\
9.63	0.01\\
9.64	0.01\\
9.65	0.01\\
9.66	0.01\\
9.67	0.01\\
9.68	0.01\\
9.69	0.01\\
9.7	0.01\\
9.71	0.01\\
9.72	0.01\\
9.73	0.01\\
9.74	0.01\\
9.75	0.01\\
9.76	0.01\\
9.77	0.01\\
9.78	0.01\\
9.79	0.01\\
9.8	0.01\\
9.81	0.01\\
9.82	0.01\\
9.83	0.01\\
9.84	0.01\\
9.85	0.01\\
9.86	0.01\\
9.87	0.01\\
9.88	0.01\\
9.89	0.01\\
9.9	0.01\\
9.91	0.01\\
9.92	0.01\\
9.93	0.01\\
9.94	0.01\\
9.95	0.01\\
9.96	0.01\\
9.97	0.01\\
9.98	0.01\\
9.99	0.01\\
10	0.01\\
10.01	0.01\\
10.02	0.01\\
10.03	0.01\\
10.04	0.01\\
10.05	0.01\\
10.06	0.01\\
10.07	0.01\\
10.08	0.01\\
10.09	0.01\\
10.1	0.01\\
10.11	0.01\\
10.12	0.01\\
10.13	0.01\\
10.14	0.01\\
10.15	0.01\\
10.16	0.01\\
10.17	0.01\\
10.18	0.01\\
10.19	0.01\\
10.2	0.01\\
10.21	0.01\\
10.22	0.01\\
10.23	0.01\\
10.24	0.01\\
10.25	0.01\\
10.26	0.01\\
10.27	0.01\\
10.28	0.01\\
10.29	0.01\\
10.3	0.01\\
10.31	0.01\\
10.32	0.01\\
10.33	0.01\\
10.34	0.01\\
10.35	0.01\\
10.36	0.01\\
10.37	0.01\\
10.38	0.01\\
10.39	0.01\\
10.4	0.01\\
10.41	0.01\\
10.42	0.01\\
10.43	0.01\\
10.44	0.01\\
10.45	0.01\\
10.46	0.01\\
10.47	0.01\\
10.48	0.01\\
10.49	0.01\\
10.5	0.01\\
10.51	0.01\\
10.52	0.01\\
10.53	0.01\\
10.54	0.01\\
10.55	0.01\\
10.56	0.01\\
10.57	0.01\\
10.58	0.01\\
10.59	0.01\\
10.6	0.01\\
10.61	0.01\\
10.62	0.01\\
10.63	0.01\\
10.64	0.01\\
10.65	0.01\\
10.66	0.01\\
10.67	0.01\\
10.68	0.01\\
10.69	0.01\\
10.7	0.01\\
10.71	0.01\\
10.72	0.01\\
10.73	0.01\\
10.74	0.01\\
10.75	0.01\\
10.76	0.01\\
10.77	0.01\\
10.78	0.01\\
10.79	0.01\\
10.8	0.01\\
10.81	0.01\\
10.82	0.01\\
10.83	0.01\\
10.84	0.01\\
10.85	0.01\\
10.86	0.01\\
10.87	0.01\\
10.88	0.01\\
10.89	0.01\\
10.9	0.01\\
10.91	0.01\\
10.92	0.01\\
10.93	0.01\\
10.94	0.01\\
10.95	0.01\\
10.96	0.01\\
10.97	0.01\\
10.98	0.01\\
10.99	0.01\\
11	0.01\\
11.01	0.01\\
11.02	0.01\\
11.03	0.01\\
11.04	0.01\\
11.05	0.01\\
11.06	0.01\\
11.07	0.01\\
11.08	0.01\\
11.09	0.01\\
11.1	0.01\\
11.11	0.01\\
11.12	0.01\\
11.13	0.01\\
11.14	0.01\\
11.15	0.01\\
11.16	0.01\\
11.17	0.01\\
11.18	0.01\\
11.19	0.01\\
11.2	0.01\\
11.21	0.01\\
11.22	0.01\\
11.23	0.01\\
11.24	0.01\\
11.25	0.01\\
11.26	0.01\\
11.27	0.01\\
11.28	0.01\\
11.29	0.01\\
11.3	0.01\\
11.31	0.01\\
11.32	0.01\\
11.33	0.01\\
11.34	0.01\\
11.35	0.01\\
11.36	0.01\\
11.37	0.01\\
11.38	0.01\\
11.39	0.01\\
11.4	0.01\\
11.41	0.01\\
11.42	0.01\\
11.43	0.01\\
11.44	0.01\\
11.45	0.01\\
11.46	0.01\\
11.47	0.01\\
11.48	0.01\\
11.49	0.01\\
11.5	0.01\\
11.51	0.01\\
11.52	0.01\\
11.53	0.01\\
11.54	0.01\\
11.55	0.01\\
11.56	0.01\\
11.57	0.01\\
11.58	0.01\\
11.59	0.01\\
11.6	0.01\\
11.61	0.01\\
11.62	0.01\\
11.63	0.01\\
11.64	0.01\\
11.65	0.01\\
11.66	0.01\\
11.67	0.01\\
11.68	0.01\\
11.69	0.01\\
11.7	0.01\\
11.71	0.01\\
11.72	0.01\\
11.73	0.01\\
11.74	0.01\\
11.75	0.01\\
11.76	0.01\\
11.77	0.01\\
11.78	0.01\\
11.79	0.01\\
11.8	0.01\\
11.81	0.01\\
11.82	0.01\\
11.83	0.01\\
11.84	0.01\\
11.85	0.01\\
11.86	0.01\\
11.87	0.01\\
11.88	0.01\\
11.89	0.01\\
11.9	0.01\\
11.91	0.01\\
11.92	0.01\\
11.93	0.01\\
11.94	0.01\\
11.95	0.01\\
11.96	0.01\\
11.97	0.01\\
11.98	0.01\\
11.99	0.01\\
12	0.01\\
12.01	0.01\\
12.02	0.01\\
12.03	0.01\\
12.04	0.01\\
12.05	0.01\\
12.06	0.01\\
12.07	0.01\\
12.08	0.01\\
12.09	0.01\\
12.1	0.01\\
12.11	0.01\\
12.12	0.01\\
12.13	0.01\\
12.14	0.01\\
12.15	0.01\\
12.16	0.01\\
12.17	0.01\\
12.18	0.01\\
12.19	0.01\\
12.2	0.01\\
12.21	0.01\\
12.22	0.01\\
12.23	0.01\\
12.24	0.01\\
12.25	0.01\\
12.26	0.01\\
12.27	0.01\\
12.28	0.01\\
12.29	0.01\\
12.3	0.01\\
12.31	0.01\\
12.32	0.01\\
12.33	0.01\\
12.34	0.01\\
12.35	0.01\\
12.36	0.01\\
12.37	0.01\\
12.38	0.01\\
12.39	0.01\\
12.4	0.01\\
12.41	0.01\\
12.42	0.01\\
12.43	0.01\\
12.44	0.01\\
12.45	0.01\\
12.46	0.01\\
12.47	0.01\\
12.48	0.01\\
12.49	0.01\\
12.5	0.01\\
12.51	0.01\\
12.52	0.01\\
12.53	0.01\\
12.54	0.01\\
12.55	0.01\\
12.56	0.01\\
12.57	0.01\\
12.58	0.01\\
12.59	0.01\\
12.6	0.01\\
12.61	0.01\\
12.62	0.01\\
12.63	0.01\\
12.64	0.01\\
12.65	0.01\\
12.66	0.01\\
12.67	0.01\\
12.68	0.01\\
12.69	0.01\\
12.7	0.01\\
12.71	0.01\\
12.72	0.01\\
12.73	0.01\\
12.74	0.01\\
12.75	0.01\\
12.76	0.01\\
12.77	0.01\\
12.78	0.01\\
12.79	0.01\\
12.8	0.01\\
12.81	0.01\\
12.82	0.01\\
12.83	0.01\\
12.84	0.01\\
12.85	0.01\\
12.86	0.01\\
12.87	0.01\\
12.88	0.01\\
12.89	0.01\\
12.9	0.01\\
12.91	0.01\\
12.92	0.01\\
12.93	0.01\\
12.94	0.01\\
12.95	0.01\\
12.96	0.01\\
12.97	0.01\\
12.98	0.01\\
12.99	0.01\\
13	0.01\\
13.01	0.01\\
13.02	0.01\\
13.03	0.01\\
13.04	0.01\\
13.05	0.01\\
13.06	0.01\\
13.07	0.01\\
13.08	0.01\\
13.09	0.01\\
13.1	0.01\\
13.11	0.01\\
13.12	0.01\\
13.13	0.01\\
13.14	0.01\\
13.15	0.01\\
13.16	0.01\\
13.17	0.01\\
13.18	0.01\\
13.19	0.01\\
13.2	0.01\\
13.21	0.01\\
13.22	0.01\\
13.23	0.01\\
13.24	0.01\\
13.25	0.01\\
13.26	0.01\\
13.27	0.01\\
13.28	0.01\\
13.29	0.01\\
13.3	0.01\\
13.31	0.01\\
13.32	0.01\\
13.33	0.01\\
13.34	0.01\\
13.35	0.01\\
13.36	0.01\\
13.37	0.01\\
13.38	0.01\\
13.39	0.01\\
13.4	0.01\\
13.41	0.01\\
13.42	0.01\\
13.43	0.01\\
13.44	0.01\\
13.45	0.01\\
13.46	0.01\\
13.47	0.01\\
13.48	0.01\\
13.49	0.01\\
13.5	0.01\\
13.51	0.01\\
13.52	0.01\\
13.53	0.01\\
13.54	0.01\\
13.55	0.01\\
13.56	0.01\\
13.57	0.01\\
13.58	0.01\\
13.59	0.01\\
13.6	0.01\\
13.61	0.01\\
13.62	0.01\\
13.63	0.01\\
13.64	0.01\\
13.65	0.01\\
13.66	0.01\\
13.67	0.01\\
13.68	0.01\\
13.69	0.01\\
13.7	0.01\\
13.71	0.01\\
13.72	0.01\\
13.73	0.01\\
13.74	0.01\\
13.75	0.01\\
13.76	0.01\\
13.77	0.01\\
13.78	0.01\\
13.79	0.01\\
13.8	0.01\\
13.81	0.01\\
13.82	0.01\\
13.83	0.01\\
13.84	0.01\\
13.85	0.01\\
13.86	0.01\\
13.87	0.01\\
13.88	0.01\\
13.89	0.01\\
13.9	0.01\\
13.91	0.01\\
13.92	0.01\\
13.93	0.01\\
13.94	0.01\\
13.95	0.01\\
13.96	0.01\\
13.97	0.01\\
13.98	0.01\\
13.99	0.01\\
14	0.01\\
14.01	0.01\\
14.02	0.01\\
14.03	0.01\\
14.04	0.01\\
14.05	0.01\\
14.06	0.01\\
14.07	0.01\\
14.08	0.01\\
14.09	0.01\\
14.1	0.01\\
14.11	0.01\\
14.12	0.01\\
14.13	0.01\\
14.14	0.01\\
14.15	0.01\\
14.16	0.01\\
14.17	0.01\\
14.18	0.01\\
14.19	0.01\\
14.2	0.01\\
14.21	0.01\\
14.22	0.01\\
14.23	0.01\\
14.24	0.01\\
14.25	0.01\\
14.26	0.01\\
14.27	0.01\\
14.28	0.01\\
14.29	0.01\\
14.3	0.01\\
14.31	0.01\\
14.32	0.01\\
14.33	0.01\\
14.34	0.01\\
14.35	0.01\\
14.36	0.01\\
14.37	0.01\\
14.38	0.01\\
14.39	0.01\\
14.4	0.01\\
14.41	0.01\\
14.42	0.01\\
14.43	0.01\\
14.44	0.01\\
14.45	0.01\\
14.46	0.01\\
14.47	0.01\\
14.48	0.01\\
14.49	0.01\\
14.5	0.01\\
14.51	0.01\\
14.52	0.01\\
14.53	0.01\\
14.54	0.01\\
14.55	0.01\\
14.56	0.01\\
14.57	0.01\\
14.58	0.01\\
14.59	0.01\\
14.6	0.01\\
14.61	0.01\\
14.62	0.01\\
14.63	0.01\\
14.64	0.01\\
14.65	0.01\\
14.66	0.01\\
14.67	0.01\\
14.68	0.01\\
14.69	0.01\\
14.7	0.01\\
14.71	0.01\\
14.72	0.01\\
14.73	0.01\\
14.74	0.01\\
14.75	0.01\\
14.76	0.01\\
14.77	0.01\\
14.78	0.01\\
14.79	0.01\\
14.8	0.01\\
14.81	0.01\\
14.82	0.01\\
14.83	0.01\\
14.84	0.01\\
14.85	0.01\\
14.86	0.01\\
14.87	0.01\\
14.88	0.01\\
14.89	0.01\\
14.9	0.01\\
14.91	0.01\\
14.92	0.01\\
14.93	0.01\\
14.94	0.01\\
14.95	0.01\\
14.96	0.01\\
14.97	0.01\\
14.98	0.01\\
14.99	0.01\\
15	0.01\\
15.01	0.01\\
15.02	0.01\\
15.03	0.01\\
15.04	0.01\\
15.05	0.01\\
15.06	0.01\\
15.07	0.01\\
15.08	0.01\\
15.09	0.01\\
15.1	0.01\\
15.11	0.01\\
15.12	0.01\\
15.13	0.01\\
15.14	0.01\\
15.15	0.01\\
15.16	0.01\\
15.17	0.01\\
15.18	0.01\\
15.19	0.01\\
15.2	0.01\\
15.21	0.01\\
15.22	0.01\\
15.23	0.01\\
15.24	0.01\\
15.25	0.01\\
15.26	0.01\\
15.27	0.01\\
15.28	0.01\\
15.29	0.01\\
15.3	0.01\\
15.31	0.01\\
15.32	0.01\\
15.33	0.01\\
15.34	0.01\\
15.35	0.01\\
15.36	0.01\\
15.37	0.01\\
15.38	0.01\\
15.39	0.01\\
15.4	0.01\\
15.41	0.01\\
15.42	0.01\\
15.43	0.01\\
15.44	0.01\\
15.45	0.01\\
15.46	0.01\\
15.47	0.01\\
15.48	0.01\\
15.49	0.01\\
15.5	0.01\\
15.51	0.01\\
15.52	0.01\\
15.53	0.01\\
15.54	0.01\\
15.55	0.01\\
15.56	0.01\\
15.57	0.01\\
15.58	0.01\\
15.59	0.01\\
15.6	0.01\\
15.61	0.01\\
15.62	0.01\\
15.63	0.01\\
15.64	0.01\\
15.65	0.01\\
15.66	0.01\\
15.67	0.01\\
15.68	0.01\\
15.69	0.01\\
15.7	0.01\\
15.71	0.01\\
15.72	0.01\\
15.73	0.01\\
15.74	0.01\\
15.75	0.01\\
15.76	0.01\\
15.77	0.01\\
15.78	0.01\\
15.79	0.01\\
15.8	0.01\\
15.81	0.01\\
15.82	0.01\\
15.83	0.01\\
15.84	0.01\\
15.85	0.01\\
15.86	0.01\\
15.87	0.01\\
15.88	0.01\\
15.89	0.01\\
15.9	0.01\\
15.91	0.01\\
15.92	0.01\\
15.93	0.01\\
15.94	0.01\\
15.95	0.01\\
15.96	0.01\\
15.97	0.01\\
15.98	0.01\\
15.99	0.01\\
16	0.01\\
16.01	0.01\\
16.02	0.01\\
16.03	0.01\\
16.04	0.01\\
16.05	0.01\\
16.06	0.01\\
16.07	0.01\\
16.08	0.01\\
16.09	0.01\\
16.1	0.01\\
16.11	0.01\\
16.12	0.01\\
16.13	0.01\\
16.14	0.01\\
16.15	0.01\\
16.16	0.01\\
16.17	0.01\\
16.18	0.01\\
16.19	0.01\\
16.2	0.01\\
16.21	0.01\\
16.22	0.01\\
16.23	0.01\\
16.24	0.01\\
16.25	0.01\\
16.26	0.01\\
16.27	0.01\\
16.28	0.01\\
16.29	0.01\\
16.3	0.01\\
16.31	0.01\\
16.32	0.01\\
16.33	0.01\\
16.34	0.01\\
16.35	0.01\\
16.36	0.01\\
16.37	0.01\\
16.38	0.01\\
16.39	0.01\\
16.4	0.01\\
16.41	0.01\\
16.42	0.01\\
16.43	0.01\\
16.44	0.01\\
16.45	0.01\\
16.46	0.01\\
16.47	0.01\\
16.48	0.01\\
16.49	0.01\\
16.5	0.01\\
16.51	0.01\\
16.52	0.01\\
16.53	0.01\\
16.54	0.01\\
16.55	0.01\\
16.56	0.01\\
16.57	0.01\\
16.58	0.01\\
16.59	0.01\\
16.6	0.01\\
16.61	0.01\\
16.62	0.01\\
16.63	0.01\\
16.64	0.01\\
16.65	0.01\\
16.66	0.01\\
16.67	0.01\\
16.68	0.01\\
16.69	0.01\\
16.7	0.01\\
16.71	0.01\\
16.72	0.01\\
16.73	0.01\\
16.74	0.01\\
16.75	0.01\\
16.76	0.01\\
16.77	0.01\\
16.78	0.01\\
16.79	0.01\\
16.8	0.01\\
16.81	0.01\\
16.82	0.01\\
16.83	0.01\\
16.84	0.01\\
16.85	0.01\\
16.86	0.01\\
16.87	0.01\\
16.88	0.01\\
16.89	0.01\\
16.9	0.01\\
16.91	0.01\\
16.92	0.01\\
16.93	0.01\\
16.94	0.01\\
16.95	0.01\\
16.96	0.01\\
16.97	0.01\\
16.98	0.01\\
16.99	0.01\\
17	0.01\\
17.01	0.01\\
17.02	0.01\\
17.03	0.01\\
17.04	0.01\\
17.05	0.01\\
17.06	0.01\\
17.07	0.01\\
17.08	0.01\\
17.09	0.01\\
17.1	0.01\\
17.11	0.01\\
17.12	0.01\\
17.13	0.01\\
17.14	0.01\\
17.15	0.01\\
17.16	0.01\\
17.17	0.01\\
17.18	0.01\\
17.19	0.01\\
17.2	0.01\\
17.21	0.01\\
17.22	0.01\\
17.23	0.01\\
17.24	0.01\\
17.25	0.01\\
17.26	0.01\\
17.27	0.01\\
17.28	0.01\\
17.29	0.01\\
17.3	0.01\\
17.31	0.01\\
17.32	0.01\\
17.33	0.01\\
17.34	0.01\\
17.35	0.01\\
17.36	0.01\\
17.37	0.01\\
17.38	0.01\\
17.39	0.01\\
17.4	0.01\\
17.41	0.01\\
17.42	0.01\\
17.43	0.01\\
17.44	0.01\\
17.45	0.01\\
17.46	0.01\\
17.47	0.01\\
17.48	0.01\\
17.49	0.01\\
17.5	0.01\\
17.51	0.01\\
17.52	0.01\\
17.53	0.01\\
17.54	0.01\\
17.55	0.01\\
17.56	0.01\\
17.57	0.01\\
17.58	0.01\\
17.59	0.01\\
17.6	0.01\\
17.61	0.01\\
17.62	0.01\\
17.63	0.01\\
17.64	0.01\\
17.65	0.01\\
17.66	0.01\\
17.67	0.01\\
17.68	0.01\\
17.69	0.01\\
17.7	0.01\\
17.71	0.01\\
17.72	0.01\\
17.73	0.01\\
17.74	0.01\\
17.75	0.01\\
17.76	0.01\\
17.77	0.01\\
17.78	0.01\\
17.79	0.01\\
17.8	0.01\\
17.81	0.01\\
17.82	0.01\\
17.83	0.01\\
17.84	0.01\\
17.85	0.01\\
17.86	0.01\\
17.87	0.01\\
17.88	0.01\\
17.89	0.01\\
17.9	0.01\\
17.91	0.01\\
17.92	0.01\\
17.93	0.01\\
17.94	0.01\\
17.95	0.01\\
17.96	0.01\\
17.97	0.01\\
17.98	0.01\\
17.99	0.01\\
18	0.01\\
18.01	0.01\\
18.02	0.01\\
18.03	0.01\\
18.04	0.01\\
18.05	0.01\\
18.06	0.01\\
18.07	0.01\\
18.08	0.01\\
18.09	0.01\\
18.1	0.01\\
18.11	0.01\\
18.12	0.01\\
18.13	0.01\\
18.14	0.01\\
18.15	0.01\\
18.16	0.01\\
18.17	0.01\\
18.18	0.01\\
18.19	0.01\\
18.2	0.01\\
18.21	0.01\\
18.22	0.01\\
18.23	0.01\\
18.24	0.01\\
18.25	0.01\\
18.26	0.01\\
18.27	0.01\\
18.28	0.01\\
18.29	0.01\\
18.3	0.01\\
18.31	0.01\\
18.32	0.01\\
18.33	0.01\\
18.34	0.01\\
18.35	0.01\\
18.36	0.01\\
18.37	0.01\\
18.38	0.01\\
18.39	0.01\\
18.4	0.01\\
18.41	0.01\\
18.42	0.01\\
18.43	0.01\\
18.44	0.01\\
18.45	0.01\\
18.46	0.01\\
18.47	0.01\\
18.48	0.01\\
18.49	0.01\\
18.5	0.01\\
18.51	0.01\\
18.52	0.01\\
18.53	0.01\\
18.54	0.01\\
18.55	0.01\\
18.56	0.01\\
18.57	0.01\\
18.58	0.01\\
18.59	0.01\\
18.6	0.01\\
18.61	0.01\\
18.62	0.01\\
18.63	0.01\\
18.64	0.01\\
18.65	0.01\\
18.66	0.01\\
18.67	0.01\\
18.68	0.01\\
18.69	0.01\\
18.7	0.01\\
18.71	0.01\\
18.72	0.01\\
18.73	0.01\\
18.74	0.01\\
18.75	0.01\\
18.76	0.01\\
18.77	0.01\\
18.78	0.01\\
18.79	0.01\\
18.8	0.01\\
18.81	0.01\\
18.82	0.01\\
18.83	0.01\\
18.84	0.01\\
18.85	0.01\\
18.86	0.01\\
18.87	0.01\\
18.88	0.01\\
18.89	0.01\\
18.9	0.01\\
18.91	0.01\\
18.92	0.01\\
18.93	0.01\\
18.94	0.01\\
18.95	0.01\\
18.96	0.01\\
18.97	0.01\\
18.98	0.01\\
18.99	0.01\\
19	0.01\\
19.01	0.01\\
19.02	0.01\\
19.03	0.01\\
19.04	0.01\\
19.05	0.01\\
19.06	0.01\\
19.07	0.01\\
19.08	0.01\\
19.09	0.01\\
19.1	0.01\\
19.11	0.01\\
19.12	0.01\\
19.13	0.01\\
19.14	0.01\\
19.15	0.01\\
19.16	0.01\\
19.17	0.01\\
19.18	0.01\\
19.19	0.01\\
19.2	0.01\\
19.21	0.01\\
19.22	0.01\\
19.23	0.01\\
19.24	0.01\\
19.25	0.01\\
19.26	0.01\\
19.27	0.01\\
19.28	0.01\\
19.29	0.01\\
19.3	0.01\\
19.31	0.01\\
19.32	0.01\\
19.33	0.01\\
19.34	0.01\\
19.35	0.01\\
19.36	0.01\\
19.37	0.01\\
19.38	0.01\\
19.39	0.01\\
19.4	0.01\\
19.41	0.01\\
19.42	0.01\\
19.43	0.01\\
19.44	0.01\\
19.45	0.01\\
19.46	0.01\\
19.47	0.01\\
19.48	0.01\\
19.49	0.01\\
19.5	0.01\\
19.51	0.01\\
19.52	0.01\\
19.53	0.01\\
19.54	0.01\\
19.55	0.01\\
19.56	0.01\\
19.57	0.01\\
19.58	0.01\\
19.59	0.01\\
19.6	0.01\\
19.61	0.01\\
19.62	0.01\\
19.63	0.01\\
19.64	0.01\\
19.65	0.01\\
19.66	0.01\\
19.67	0.01\\
19.68	0.01\\
19.69	0.01\\
19.7	0.01\\
19.71	0.01\\
19.72	0.01\\
19.73	0.01\\
19.74	0.01\\
19.75	0.01\\
19.76	0.01\\
19.77	0.01\\
19.78	0.01\\
19.79	0.01\\
19.8	0.01\\
19.81	0.01\\
19.82	0.01\\
19.83	0.01\\
19.84	0.01\\
19.85	0.01\\
19.86	0.01\\
19.87	0.01\\
19.88	0.01\\
19.89	0.01\\
19.9	0.01\\
19.91	0.01\\
19.92	0.01\\
19.93	0.01\\
19.94	0.01\\
19.95	0.01\\
19.96	0.01\\
19.97	0.01\\
19.98	0.01\\
19.99	0.01\\
20	0.01\\
20.01	0.01\\
20.02	0.01\\
20.03	0.01\\
20.04	0.01\\
20.05	0.01\\
20.06	0.01\\
20.07	0.01\\
20.08	0.01\\
20.09	0.01\\
20.1	0.01\\
20.11	0.01\\
20.12	0.01\\
20.13	0.01\\
20.14	0.01\\
20.15	0.01\\
20.16	0.01\\
20.17	0.01\\
20.18	0.01\\
20.19	0.01\\
20.2	0.01\\
20.21	0.01\\
20.22	0.01\\
20.23	0.01\\
20.24	0.01\\
20.25	0.01\\
20.26	0.01\\
20.27	0.01\\
20.28	0.01\\
20.29	0.01\\
20.3	0.01\\
20.31	0.01\\
20.32	0.01\\
20.33	0.01\\
20.34	0.01\\
20.35	0.01\\
20.36	0.01\\
20.37	0.01\\
20.38	0.01\\
20.39	0.01\\
20.4	0.01\\
20.41	0.01\\
20.42	0.01\\
20.43	0.01\\
20.44	0.01\\
20.45	0.01\\
20.46	0.01\\
20.47	0.01\\
20.48	0.01\\
20.49	0.01\\
20.5	0.01\\
20.51	0.01\\
20.52	0.01\\
20.53	0.01\\
20.54	0.01\\
20.55	0.01\\
20.56	0.01\\
20.57	0.01\\
20.58	0.01\\
20.59	0.01\\
20.6	0.01\\
20.61	0.01\\
20.62	0.01\\
20.63	0.01\\
20.64	0.01\\
20.65	0.01\\
20.66	0.01\\
20.67	0.01\\
20.68	0.01\\
20.69	0.01\\
20.7	0.01\\
20.71	0.01\\
20.72	0.01\\
20.73	0.01\\
20.74	0.01\\
20.75	0.01\\
20.76	0.01\\
20.77	0.01\\
20.78	0.01\\
20.79	0.01\\
20.8	0.01\\
20.81	0.01\\
20.82	0.01\\
20.83	0.01\\
20.84	0.01\\
20.85	0.01\\
20.86	0.01\\
20.87	0.01\\
20.88	0.01\\
20.89	0.01\\
20.9	0.01\\
20.91	0.01\\
20.92	0.01\\
20.93	0.01\\
20.94	0.01\\
20.95	0.01\\
20.96	0.01\\
20.97	0.01\\
20.98	0.01\\
20.99	0.01\\
21	0.01\\
21.01	0.01\\
21.02	0.01\\
21.03	0.01\\
21.04	0.01\\
21.05	0.01\\
21.06	0.01\\
21.07	0.01\\
21.08	0.01\\
21.09	0.01\\
21.1	0.01\\
21.11	0.01\\
21.12	0.01\\
21.13	0.01\\
21.14	0.01\\
21.15	0.01\\
21.16	0.01\\
21.17	0.01\\
21.18	0.01\\
21.19	0.01\\
21.2	0.01\\
21.21	0.01\\
21.22	0.01\\
21.23	0.01\\
21.24	0.01\\
21.25	0.01\\
21.26	0.01\\
21.27	0.01\\
21.28	0.01\\
21.29	0.01\\
21.3	0.01\\
21.31	0.01\\
21.32	0.01\\
21.33	0.01\\
21.34	0.01\\
21.35	0.01\\
21.36	0.01\\
21.37	0.01\\
21.38	0.01\\
21.39	0.01\\
21.4	0.01\\
21.41	0.01\\
21.42	0.01\\
21.43	0.01\\
21.44	0.01\\
21.45	0.01\\
21.46	0.01\\
21.47	0.01\\
21.48	0.01\\
21.49	0.01\\
21.5	0.01\\
21.51	0.01\\
21.52	0.01\\
21.53	0.01\\
21.54	0.01\\
21.55	0.01\\
21.56	0.01\\
21.57	0.01\\
21.58	0.01\\
21.59	0.01\\
21.6	0.01\\
21.61	0.01\\
21.62	0.01\\
21.63	0.01\\
21.64	0.01\\
21.65	0.01\\
21.66	0.01\\
21.67	0.01\\
21.68	0.01\\
21.69	0.01\\
21.7	0.01\\
21.71	0.01\\
21.72	0.01\\
21.73	0.01\\
21.74	0.01\\
21.75	0.01\\
21.76	0.01\\
21.77	0.01\\
21.78	0.01\\
21.79	0.01\\
21.8	0.01\\
21.81	0.01\\
21.82	0.01\\
21.83	0.01\\
21.84	0.01\\
21.85	0.01\\
21.86	0.01\\
21.87	0.01\\
21.88	0.01\\
21.89	0.01\\
21.9	0.01\\
21.91	0.01\\
21.92	0.01\\
21.93	0.01\\
21.94	0.01\\
21.95	0.01\\
21.96	0.01\\
21.97	0.01\\
21.98	0.01\\
21.99	0.01\\
22	0.01\\
22.01	0.01\\
22.02	0.01\\
22.03	0.01\\
22.04	0.01\\
22.05	0.01\\
22.06	0.01\\
22.07	0.01\\
22.08	0.01\\
22.09	0.01\\
22.1	0.01\\
22.11	0.01\\
22.12	0.01\\
22.13	0.01\\
22.14	0.01\\
22.15	0.01\\
22.16	0.01\\
22.17	0.01\\
22.18	0.01\\
22.19	0.01\\
22.2	0.01\\
22.21	0.01\\
22.22	0.01\\
22.23	0.01\\
22.24	0.01\\
22.25	0.01\\
22.26	0.01\\
22.27	0.01\\
22.28	0.01\\
22.29	0.01\\
22.3	0.01\\
22.31	0.01\\
22.32	0.01\\
22.33	0.01\\
22.34	0.01\\
22.35	0.01\\
22.36	0.01\\
22.37	0.01\\
22.38	0.01\\
22.39	0.01\\
22.4	0.01\\
22.41	0.01\\
22.42	0.01\\
22.43	0.01\\
22.44	0.01\\
22.45	0.01\\
22.46	0.01\\
22.47	0.01\\
22.48	0.01\\
22.49	0.01\\
22.5	0.01\\
22.51	0.01\\
22.52	0.01\\
22.53	0.01\\
22.54	0.01\\
22.55	0.01\\
22.56	0.01\\
22.57	0.01\\
22.58	0.01\\
22.59	0.01\\
22.6	0.01\\
22.61	0.01\\
22.62	0.01\\
22.63	0.01\\
22.64	0.01\\
22.65	0.01\\
22.66	0.01\\
22.67	0.01\\
22.68	0.01\\
22.69	0.01\\
22.7	0.01\\
22.71	0.01\\
22.72	0.01\\
22.73	0.01\\
22.74	0.01\\
22.75	0.01\\
22.76	0.01\\
22.77	0.01\\
22.78	0.01\\
22.79	0.01\\
22.8	0.01\\
22.81	0.01\\
22.82	0.01\\
22.83	0.01\\
22.84	0.01\\
22.85	0.01\\
22.86	0.01\\
22.87	0.01\\
22.88	0.01\\
22.89	0.01\\
22.9	0.01\\
22.91	0.01\\
22.92	0.01\\
22.93	0.01\\
22.94	0.01\\
22.95	0.01\\
22.96	0.01\\
22.97	0.01\\
22.98	0.01\\
22.99	0.01\\
23	0.01\\
23.01	0.01\\
23.02	0.01\\
23.03	0.01\\
23.04	0.01\\
23.05	0.01\\
23.06	0.01\\
23.07	0.01\\
23.08	0.01\\
23.09	0.01\\
23.1	0.01\\
23.11	0.01\\
23.12	0.01\\
23.13	0.01\\
23.14	0.01\\
23.15	0.01\\
23.16	0.01\\
23.17	0.01\\
23.18	0.01\\
23.19	0.01\\
23.2	0.01\\
23.21	0.01\\
23.22	0.01\\
23.23	0.01\\
23.24	0.01\\
23.25	0.01\\
23.26	0.01\\
23.27	0.01\\
23.28	0.01\\
23.29	0.01\\
23.3	0.01\\
23.31	0.01\\
23.32	0.01\\
23.33	0.01\\
23.34	0.01\\
23.35	0.01\\
23.36	0.01\\
23.37	0.01\\
23.38	0.01\\
23.39	0.01\\
23.4	0.01\\
23.41	0.01\\
23.42	0.01\\
23.43	0.01\\
23.44	0.01\\
23.45	0.01\\
23.46	0.01\\
23.47	0.01\\
23.48	0.01\\
23.49	0.01\\
23.5	0.01\\
23.51	0.01\\
23.52	0.01\\
23.53	0.01\\
23.54	0.01\\
23.55	0.01\\
23.56	0.01\\
23.57	0.01\\
23.58	0.01\\
23.59	0.01\\
23.6	0.01\\
23.61	0.01\\
23.62	0.01\\
23.63	0.01\\
23.64	0.01\\
23.65	0.01\\
23.66	0.01\\
23.67	0.01\\
23.68	0.01\\
23.69	0.01\\
23.7	0.01\\
23.71	0.01\\
23.72	0.01\\
23.73	0.01\\
23.74	0.01\\
23.75	0.01\\
23.76	0.01\\
23.77	0.01\\
23.78	0.01\\
23.79	0.01\\
23.8	0.01\\
23.81	0.01\\
23.82	0.01\\
23.83	0.01\\
23.84	0.01\\
23.85	0.01\\
23.86	0.01\\
23.87	0.01\\
23.88	0.01\\
23.89	0.01\\
23.9	0.01\\
23.91	0.01\\
23.92	0.01\\
23.93	0.01\\
23.94	0.01\\
23.95	0.01\\
23.96	0.01\\
23.97	0.01\\
23.98	0.01\\
23.99	0.01\\
24	0.01\\
24.01	0.01\\
24.02	0.01\\
24.03	0.01\\
24.04	0.01\\
24.05	0.01\\
24.06	0.01\\
24.07	0.01\\
24.08	0.01\\
24.09	0.01\\
24.1	0.01\\
24.11	0.01\\
24.12	0.01\\
24.13	0.01\\
24.14	0.01\\
24.15	0.01\\
24.16	0.01\\
24.17	0.01\\
24.18	0.01\\
24.19	0.01\\
24.2	0.01\\
24.21	0.01\\
24.22	0.01\\
24.23	0.01\\
24.24	0.01\\
24.25	0.01\\
24.26	0.01\\
24.27	0.01\\
24.28	0.01\\
24.29	0.01\\
24.3	0.01\\
24.31	0.01\\
24.32	0.01\\
24.33	0.01\\
24.34	0.01\\
24.35	0.01\\
24.36	0.01\\
24.37	0.01\\
24.38	0.01\\
24.39	0.01\\
24.4	0.01\\
24.41	0.01\\
24.42	0.01\\
24.43	0.01\\
24.44	0.01\\
24.45	0.01\\
24.46	0.01\\
24.47	0.01\\
24.48	0.01\\
24.49	0.01\\
24.5	0.01\\
24.51	0.01\\
24.52	0.01\\
24.53	0.01\\
24.54	0.01\\
24.55	0.01\\
24.56	0.01\\
24.57	0.01\\
24.58	0.01\\
24.59	0.01\\
24.6	0.01\\
24.61	0.01\\
24.62	0.01\\
24.63	0.01\\
24.64	0.01\\
24.65	0.01\\
24.66	0.01\\
24.67	0.01\\
24.68	0.01\\
24.69	0.01\\
24.7	0.01\\
24.71	0.01\\
24.72	0.01\\
24.73	0.01\\
24.74	0.01\\
24.75	0.01\\
24.76	0.01\\
24.77	0.01\\
24.78	0.01\\
24.79	0.01\\
24.8	0.01\\
24.81	0.01\\
24.82	0.01\\
24.83	0.01\\
24.84	0.01\\
24.85	0.01\\
24.86	0.01\\
24.87	0.01\\
24.88	0.01\\
24.89	0.01\\
24.9	0.01\\
24.91	0.01\\
24.92	0.01\\
24.93	0.01\\
24.94	0.01\\
24.95	0.01\\
24.96	0.01\\
24.97	0.01\\
24.98	0.01\\
24.99	0.01\\
25	0.01\\
25.01	0.01\\
25.02	0.01\\
25.03	0.01\\
25.04	0.01\\
25.05	0.01\\
25.06	0.01\\
25.07	0.01\\
25.08	0.01\\
25.09	0.01\\
25.1	0.01\\
25.11	0.01\\
25.12	0.01\\
25.13	0.01\\
25.14	0.01\\
25.15	0.01\\
25.16	0.01\\
25.17	0.01\\
25.18	0.01\\
25.19	0.01\\
25.2	0.01\\
25.21	0.01\\
25.22	0.01\\
25.23	0.01\\
25.24	0.01\\
25.25	0.01\\
25.26	0.01\\
25.27	0.01\\
25.28	0.01\\
25.29	0.01\\
25.3	0.01\\
25.31	0.01\\
25.32	0.01\\
25.33	0.01\\
25.34	0.01\\
25.35	0.01\\
25.36	0.01\\
25.37	0.01\\
25.38	0.01\\
25.39	0.01\\
25.4	0.01\\
25.41	0.01\\
25.42	0.01\\
25.43	0.01\\
25.44	0.01\\
25.45	0.01\\
25.46	0.01\\
25.47	0.01\\
25.48	0.01\\
25.49	0.01\\
25.5	0.01\\
25.51	0.01\\
25.52	0.01\\
25.53	0.01\\
25.54	0.01\\
25.55	0.01\\
25.56	0.01\\
25.57	0.01\\
25.58	0.01\\
25.59	0.01\\
25.6	0.01\\
25.61	0.01\\
25.62	0.01\\
25.63	0.01\\
25.64	0.01\\
25.65	0.01\\
25.66	0.01\\
25.67	0.01\\
25.68	0.01\\
25.69	0.01\\
25.7	0.01\\
25.71	0.01\\
25.72	0.01\\
25.73	0.01\\
25.74	0.01\\
25.75	0.01\\
25.76	0.01\\
25.77	0.01\\
25.78	0.01\\
25.79	0.01\\
25.8	0.01\\
25.81	0.01\\
25.82	0.01\\
25.83	0.01\\
25.84	0.01\\
25.85	0.01\\
25.86	0.01\\
25.87	0.01\\
25.88	0.01\\
25.89	0.01\\
25.9	0.01\\
25.91	0.01\\
25.92	0.01\\
25.93	0.01\\
25.94	0.01\\
25.95	0.01\\
25.96	0.01\\
25.97	0.01\\
25.98	0.01\\
25.99	0.01\\
26	0.01\\
26.01	0.01\\
26.02	0.01\\
26.03	0.01\\
26.04	0.01\\
26.05	0.01\\
26.06	0.01\\
26.07	0.01\\
26.08	0.01\\
26.09	0.01\\
26.1	0.01\\
26.11	0.01\\
26.12	0.01\\
26.13	0.01\\
26.14	0.01\\
26.15	0.01\\
26.16	0.01\\
26.17	0.01\\
26.18	0.01\\
26.19	0.01\\
26.2	0.01\\
26.21	0.01\\
26.22	0.01\\
26.23	0.01\\
26.24	0.01\\
26.25	0.01\\
26.26	0.01\\
26.27	0.01\\
26.28	0.01\\
26.29	0.01\\
26.3	0.01\\
26.31	0.01\\
26.32	0.01\\
26.33	0.01\\
26.34	0.01\\
26.35	0.01\\
26.36	0.01\\
26.37	0.01\\
26.38	0.01\\
26.39	0.01\\
26.4	0.01\\
26.41	0.01\\
26.42	0.01\\
26.43	0.01\\
26.44	0.01\\
26.45	0.01\\
26.46	0.01\\
26.47	0.01\\
26.48	0.01\\
26.49	0.01\\
26.5	0.01\\
26.51	0.01\\
26.52	0.01\\
26.53	0.01\\
26.54	0.01\\
26.55	0.01\\
26.56	0.01\\
26.57	0.01\\
26.58	0.01\\
26.59	0.01\\
26.6	0.01\\
26.61	0.01\\
26.62	0.01\\
26.63	0.01\\
26.64	0.01\\
26.65	0.01\\
26.66	0.01\\
26.67	0.01\\
26.68	0.01\\
26.69	0.01\\
26.7	0.01\\
26.71	0.01\\
26.72	0.01\\
26.73	0.01\\
26.74	0.01\\
26.75	0.01\\
26.76	0.01\\
26.77	0.01\\
26.78	0.01\\
26.79	0.01\\
26.8	0.01\\
26.81	0.01\\
26.82	0.01\\
26.83	0.01\\
26.84	0.01\\
26.85	0.01\\
26.86	0.01\\
26.87	0.01\\
26.88	0.01\\
26.89	0.01\\
26.9	0.01\\
26.91	0.01\\
26.92	0.01\\
26.93	0.01\\
26.94	0.01\\
26.95	0.01\\
26.96	0.01\\
26.97	0.01\\
26.98	0.01\\
26.99	0.01\\
27	0.01\\
27.01	0.01\\
27.02	0.01\\
27.03	0.01\\
27.04	0.01\\
27.05	0.01\\
27.06	0.01\\
27.07	0.01\\
27.08	0.01\\
27.09	0.01\\
27.1	0.01\\
27.11	0.01\\
27.12	0.01\\
27.13	0.01\\
27.14	0.01\\
27.15	0.01\\
27.16	0.01\\
27.17	0.01\\
27.18	0.01\\
27.19	0.01\\
27.2	0.01\\
27.21	0.01\\
27.22	0.01\\
27.23	0.01\\
27.24	0.01\\
27.25	0.01\\
27.26	0.01\\
27.27	0.01\\
27.28	0.01\\
27.29	0.01\\
27.3	0.01\\
27.31	0.01\\
27.32	0.01\\
27.33	0.01\\
27.34	0.01\\
27.35	0.01\\
27.36	0.01\\
27.37	0.01\\
27.38	0.01\\
27.39	0.01\\
27.4	0.01\\
27.41	0.01\\
27.42	0.01\\
27.43	0.01\\
27.44	0.01\\
27.45	0.01\\
27.46	0.01\\
27.47	0.01\\
27.48	0.01\\
27.49	0.01\\
27.5	0.01\\
27.51	0.01\\
27.52	0.01\\
27.53	0.01\\
27.54	0.01\\
27.55	0.01\\
27.56	0.01\\
27.57	0.01\\
27.58	0.01\\
27.59	0.01\\
27.6	0.01\\
27.61	0.01\\
27.62	0.01\\
27.63	0.01\\
27.64	0.01\\
27.65	0.01\\
27.66	0.01\\
27.67	0.01\\
27.68	0.01\\
27.69	0.01\\
27.7	0.01\\
27.71	0.01\\
27.72	0.01\\
27.73	0.01\\
27.74	0.01\\
27.75	0.01\\
27.76	0.01\\
27.77	0.01\\
27.78	0.01\\
27.79	0.01\\
27.8	0.01\\
27.81	0.01\\
27.82	0.01\\
27.83	0.01\\
27.84	0.01\\
27.85	0.01\\
27.86	0.01\\
27.87	0.01\\
27.88	0.01\\
27.89	0.01\\
27.9	0.01\\
27.91	0.01\\
27.92	0.01\\
27.93	0.01\\
27.94	0.01\\
27.95	0.01\\
27.96	0.01\\
27.97	0.01\\
27.98	0.01\\
27.99	0.01\\
28	0.01\\
28.01	0.01\\
28.02	0.01\\
28.03	0.01\\
28.04	0.01\\
28.05	0.01\\
28.06	0.01\\
28.07	0.01\\
28.08	0.01\\
28.09	0.01\\
28.1	0.01\\
28.11	0.01\\
28.12	0.01\\
28.13	0.01\\
28.14	0.01\\
28.15	0.01\\
28.16	0.01\\
28.17	0.01\\
28.18	0.01\\
28.19	0.01\\
28.2	0.01\\
28.21	0.01\\
28.22	0.01\\
28.23	0.01\\
28.24	0.01\\
28.25	0.01\\
28.26	0.01\\
28.27	0.01\\
28.28	0.01\\
28.29	0.01\\
28.3	0.01\\
28.31	0.01\\
28.32	0.01\\
28.33	0.01\\
28.34	0.01\\
28.35	0.01\\
28.36	0.01\\
28.37	0.01\\
28.38	0.01\\
28.39	0.01\\
28.4	0.01\\
28.41	0.01\\
28.42	0.01\\
28.43	0.01\\
28.44	0.01\\
28.45	0.01\\
28.46	0.01\\
28.47	0.01\\
28.48	0.01\\
28.49	0.01\\
28.5	0.01\\
28.51	0.01\\
28.52	0.01\\
28.53	0.01\\
28.54	0.01\\
28.55	0.01\\
28.56	0.01\\
28.57	0.01\\
28.58	0.01\\
28.59	0.01\\
28.6	0.01\\
28.61	0.01\\
28.62	0.01\\
28.63	0.01\\
28.64	0.01\\
28.65	0.01\\
28.66	0.01\\
28.67	0.01\\
28.68	0.01\\
28.69	0.01\\
28.7	0.01\\
28.71	0.01\\
28.72	0.01\\
28.73	0.01\\
28.74	0.01\\
28.75	0.01\\
28.76	0.01\\
28.77	0.01\\
28.78	0.01\\
28.79	0.01\\
28.8	0.01\\
28.81	0.01\\
28.82	0.01\\
28.83	0.01\\
28.84	0.01\\
28.85	0.01\\
28.86	0.01\\
28.87	0.01\\
28.88	0.01\\
28.89	0.01\\
28.9	0.01\\
28.91	0.01\\
28.92	0.01\\
28.93	0.01\\
28.94	0.01\\
28.95	0.01\\
28.96	0.01\\
28.97	0.01\\
28.98	0.01\\
28.99	0.01\\
29	0.01\\
29.01	0.01\\
29.02	0.01\\
29.03	0.01\\
29.04	0.01\\
29.05	0.01\\
29.06	0.01\\
29.07	0.01\\
29.08	0.01\\
29.09	0.01\\
29.1	0.01\\
29.11	0.01\\
29.12	0.01\\
29.13	0.01\\
29.14	0.01\\
29.15	0.01\\
29.16	0.01\\
29.17	0.01\\
29.18	0.01\\
29.19	0.01\\
29.2	0.01\\
29.21	0.01\\
29.22	0.01\\
29.23	0.01\\
29.24	0.01\\
29.25	0.01\\
29.26	0.01\\
29.27	0.01\\
29.28	0.01\\
29.29	0.01\\
29.3	0.01\\
29.31	0.01\\
29.32	0.01\\
29.33	0.01\\
29.34	0.01\\
29.35	0.01\\
29.36	0.01\\
29.37	0.01\\
29.38	0.01\\
29.39	0.01\\
29.4	0.01\\
29.41	0.01\\
29.42	0.01\\
29.43	0.01\\
29.44	0.01\\
29.45	0.01\\
29.46	0.01\\
29.47	0.01\\
29.48	0.01\\
29.49	0.01\\
29.5	0.01\\
29.51	0.01\\
29.52	0.01\\
29.53	0.01\\
29.54	0.01\\
29.55	0.01\\
29.56	0.01\\
29.57	0.01\\
29.58	0.01\\
29.59	0.01\\
29.6	0.01\\
29.61	0.01\\
29.62	0.01\\
29.63	0.01\\
29.64	0.01\\
29.65	0.01\\
29.66	0.01\\
29.67	0.01\\
29.68	0.01\\
29.69	0.01\\
29.7	0.01\\
29.71	0.01\\
29.72	0.01\\
29.73	0.01\\
29.74	0.01\\
29.75	0.01\\
29.76	0.01\\
29.77	0.01\\
29.78	0.01\\
29.79	0.01\\
29.8	0.01\\
29.81	0.01\\
29.82	0.01\\
29.83	0.01\\
29.84	0.01\\
29.85	0.01\\
29.86	0.01\\
29.87	0.01\\
29.88	0.01\\
29.89	0.01\\
29.9	0.01\\
29.91	0.01\\
29.92	0.01\\
29.93	0.01\\
29.94	0.01\\
29.95	0.01\\
29.96	0.01\\
29.97	0.01\\
29.98	0.01\\
29.99	0.01\\
30	0.01\\
30.01	0.01\\
30.02	0.01\\
30.03	0.01\\
30.04	0.01\\
30.05	0.01\\
30.06	0.01\\
30.07	0.01\\
30.08	0.01\\
30.09	0.01\\
30.1	0.01\\
30.11	0.01\\
30.12	0.01\\
30.13	0.01\\
30.14	0.01\\
30.15	0.01\\
30.16	0.01\\
30.17	0.01\\
30.18	0.01\\
30.19	0.01\\
30.2	0.01\\
30.21	0.01\\
30.22	0.01\\
30.23	0.01\\
30.24	0.01\\
30.25	0.01\\
30.26	0.01\\
30.27	0.01\\
30.28	0.01\\
30.29	0.01\\
30.3	0.01\\
30.31	0.01\\
30.32	0.01\\
30.33	0.01\\
30.34	0.01\\
30.35	0.01\\
30.36	0.01\\
30.37	0.01\\
30.38	0.01\\
30.39	0.01\\
30.4	0.01\\
30.41	0.01\\
30.42	0.01\\
30.43	0.01\\
30.44	0.01\\
30.45	0.01\\
30.46	0.01\\
30.47	0.01\\
30.48	0.01\\
30.49	0.01\\
30.5	0.01\\
30.51	0.01\\
30.52	0.01\\
30.53	0.01\\
30.54	0.01\\
30.55	0.01\\
30.56	0.01\\
30.57	0.01\\
30.58	0.01\\
30.59	0.01\\
30.6	0.01\\
30.61	0.01\\
30.62	0.01\\
30.63	0.01\\
30.64	0.01\\
30.65	0.01\\
30.66	0.01\\
30.67	0.01\\
30.68	0.01\\
30.69	0.01\\
30.7	0.01\\
30.71	0.01\\
30.72	0.01\\
30.73	0.01\\
30.74	0.01\\
30.75	0.01\\
30.76	0.01\\
30.77	0.01\\
30.78	0.01\\
30.79	0.01\\
30.8	0.01\\
30.81	0.01\\
30.82	0.01\\
30.83	0.01\\
30.84	0.01\\
30.85	0.01\\
30.86	0.01\\
30.87	0.01\\
30.88	0.01\\
30.89	0.01\\
30.9	0.01\\
30.91	0.01\\
30.92	0.01\\
30.93	0.01\\
30.94	0.01\\
30.95	0.01\\
30.96	0.01\\
30.97	0.01\\
30.98	0.01\\
30.99	0.01\\
31	0.01\\
31.01	0.01\\
31.02	0.01\\
31.03	0.01\\
31.04	0.01\\
31.05	0.01\\
31.06	0.01\\
31.07	0.01\\
31.08	0.01\\
31.09	0.01\\
31.1	0.01\\
31.11	0.01\\
31.12	0.01\\
31.13	0.01\\
31.14	0.01\\
31.15	0.01\\
31.16	0.01\\
31.17	0.01\\
31.18	0.01\\
31.19	0.01\\
31.2	0.01\\
31.21	0.01\\
31.22	0.01\\
31.23	0.01\\
31.24	0.01\\
31.25	0.01\\
31.26	0.01\\
31.27	0.01\\
31.28	0.01\\
31.29	0.01\\
31.3	0.01\\
31.31	0.01\\
31.32	0.01\\
31.33	0.01\\
31.34	0.01\\
31.35	0.01\\
31.36	0.01\\
31.37	0.01\\
31.38	0.01\\
31.39	0.01\\
31.4	0.01\\
31.41	0.01\\
31.42	0.01\\
31.43	0.01\\
31.44	0.01\\
31.45	0.01\\
31.46	0.01\\
31.47	0.01\\
31.48	0.01\\
31.49	0.01\\
31.5	0.01\\
31.51	0.01\\
31.52	0.01\\
31.53	0.01\\
31.54	0.01\\
31.55	0.01\\
31.56	0.01\\
31.57	0.01\\
31.58	0.01\\
31.59	0.01\\
31.6	0.01\\
31.61	0.01\\
31.62	0.01\\
31.63	0.01\\
31.64	0.01\\
31.65	0.01\\
31.66	0.01\\
31.67	0.01\\
31.68	0.01\\
31.69	0.01\\
31.7	0.01\\
31.71	0.01\\
31.72	0.01\\
31.73	0.01\\
31.74	0.01\\
31.75	0.01\\
31.76	0.01\\
31.77	0.01\\
31.78	0.01\\
31.79	0.01\\
31.8	0.01\\
31.81	0.01\\
31.82	0.01\\
31.83	0.01\\
31.84	0.01\\
31.85	0.01\\
31.86	0.01\\
31.87	0.01\\
31.88	0.01\\
31.89	0.01\\
31.9	0.01\\
31.91	0.01\\
31.92	0.01\\
31.93	0.01\\
31.94	0.01\\
31.95	0.01\\
31.96	0.01\\
31.97	0.01\\
31.98	0.01\\
31.99	0.01\\
32	0.01\\
32.01	0.01\\
32.02	0.01\\
32.03	0.01\\
32.04	0.01\\
32.05	0.01\\
32.06	0.01\\
32.07	0.01\\
32.08	0.01\\
32.09	0.01\\
32.1	0.01\\
32.11	0.01\\
32.12	0.01\\
32.13	0.01\\
32.14	0.01\\
32.15	0.01\\
32.16	0.01\\
32.17	0.01\\
32.18	0.01\\
32.19	0.01\\
32.2	0.01\\
32.21	0.01\\
32.22	0.01\\
32.23	0.01\\
32.24	0.01\\
32.25	0.01\\
32.26	0.01\\
32.27	0.01\\
32.28	0.01\\
32.29	0.01\\
32.3	0.01\\
32.31	0.01\\
32.32	0.01\\
32.33	0.01\\
32.34	0.01\\
32.35	0.01\\
32.36	0.01\\
32.37	0.01\\
32.38	0.01\\
32.39	0.01\\
32.4	0.01\\
32.41	0.01\\
32.42	0.01\\
32.43	0.01\\
32.44	0.01\\
32.45	0.01\\
32.46	0.01\\
32.47	0.01\\
32.48	0.01\\
32.49	0.01\\
32.5	0.01\\
32.51	0.01\\
32.52	0.01\\
32.53	0.01\\
32.54	0.01\\
32.55	0.01\\
32.56	0.01\\
32.57	0.01\\
32.58	0.01\\
32.59	0.01\\
32.6	0.01\\
32.61	0.01\\
32.62	0.01\\
32.63	0.01\\
32.64	0.01\\
32.65	0.01\\
32.66	0.01\\
32.67	0.01\\
32.68	0.01\\
32.69	0.01\\
32.7	0.01\\
32.71	0.01\\
32.72	0.01\\
32.73	0.01\\
32.74	0.01\\
32.75	0.01\\
32.76	0.01\\
32.77	0.01\\
32.78	0.01\\
32.79	0.01\\
32.8	0.01\\
32.81	0.01\\
32.82	0.01\\
32.83	0.01\\
32.84	0.01\\
32.85	0.01\\
32.86	0.01\\
32.87	0.01\\
32.88	0.01\\
32.89	0.01\\
32.9	0.01\\
32.91	0.01\\
32.92	0.01\\
32.93	0.01\\
32.94	0.01\\
32.95	0.01\\
32.96	0.01\\
32.97	0.01\\
32.98	0.01\\
32.99	0.01\\
33	0.01\\
33.01	0.01\\
33.02	0.01\\
33.03	0.01\\
33.04	0.01\\
33.05	0.01\\
33.06	0.01\\
33.07	0.01\\
33.08	0.01\\
33.09	0.01\\
33.1	0.01\\
33.11	0.01\\
33.12	0.01\\
33.13	0.01\\
33.14	0.01\\
33.15	0.01\\
33.16	0.01\\
33.17	0.01\\
33.18	0.01\\
33.19	0.01\\
33.2	0.01\\
33.21	0.01\\
33.22	0.01\\
33.23	0.01\\
33.24	0.01\\
33.25	0.01\\
33.26	0.01\\
33.27	0.01\\
33.28	0.01\\
33.29	0.01\\
33.3	0.01\\
33.31	0.01\\
33.32	0.01\\
33.33	0.01\\
33.34	0.01\\
33.35	0.01\\
33.36	0.01\\
33.37	0.01\\
33.38	0.01\\
33.39	0.01\\
33.4	0.01\\
33.41	0.01\\
33.42	0.01\\
33.43	0.01\\
33.44	0.01\\
33.45	0.01\\
33.46	0.01\\
33.47	0.01\\
33.48	0.01\\
33.49	0.01\\
33.5	0.01\\
33.51	0.01\\
33.52	0.01\\
33.53	0.01\\
33.54	0.01\\
33.55	0.01\\
33.56	0.01\\
33.57	0.01\\
33.58	0.01\\
33.59	0.01\\
33.6	0.01\\
33.61	0.01\\
33.62	0.01\\
33.63	0.01\\
33.64	0.01\\
33.65	0.01\\
33.66	0.01\\
33.67	0.01\\
33.68	0.01\\
33.69	0.01\\
33.7	0.01\\
33.71	0.01\\
33.72	0.01\\
33.73	0.01\\
33.74	0.01\\
33.75	0.01\\
33.76	0.01\\
33.77	0.01\\
33.78	0.01\\
33.79	0.01\\
33.8	0.01\\
33.81	0.01\\
33.82	0.01\\
33.83	0.01\\
33.84	0.01\\
33.85	0.01\\
33.86	0.01\\
33.87	0.01\\
33.88	0.01\\
33.89	0.01\\
33.9	0.01\\
33.91	0.01\\
33.92	0.01\\
33.93	0.01\\
33.94	0.01\\
33.95	0.01\\
33.96	0.01\\
33.97	0.01\\
33.98	0.01\\
33.99	0.01\\
34	0.01\\
34.01	0.01\\
34.02	0.01\\
34.03	0.01\\
34.04	0.01\\
34.05	0.01\\
34.06	0.01\\
34.07	0.01\\
34.08	0.01\\
34.09	0.01\\
34.1	0.01\\
34.11	0.01\\
34.12	0.01\\
34.13	0.01\\
34.14	0.01\\
34.15	0.01\\
34.16	0.01\\
34.17	0.01\\
34.18	0.01\\
34.19	0.01\\
34.2	0.01\\
34.21	0.01\\
34.22	0.01\\
34.23	0.01\\
34.24	0.01\\
34.25	0.01\\
34.26	0.01\\
34.27	0.01\\
34.28	0.01\\
34.29	0.01\\
34.3	0.01\\
34.31	0.01\\
34.32	0.01\\
34.33	0.01\\
34.34	0.01\\
34.35	0.01\\
34.36	0.01\\
34.37	0.01\\
34.38	0.01\\
34.39	0.01\\
34.4	0.01\\
34.41	0.01\\
34.42	0.01\\
34.43	0.01\\
34.44	0.01\\
34.45	0.01\\
34.46	0.01\\
34.47	0.01\\
34.48	0.01\\
34.49	0.01\\
34.5	0.01\\
34.51	0.01\\
34.52	0.01\\
34.53	0.01\\
34.54	0.01\\
34.55	0.01\\
34.56	0.01\\
34.57	0.01\\
34.58	0.01\\
34.59	0.01\\
34.6	0.01\\
34.61	0.01\\
34.62	0.01\\
34.63	0.01\\
34.64	0.01\\
34.65	0.01\\
34.66	0.01\\
34.67	0.01\\
34.68	0.01\\
34.69	0.01\\
34.7	0.01\\
34.71	0.01\\
34.72	0.01\\
34.73	0.01\\
34.74	0.01\\
34.75	0.01\\
34.76	0.01\\
34.77	0.01\\
34.78	0.01\\
34.79	0.01\\
34.8	0.01\\
34.81	0.01\\
34.82	0.01\\
34.83	0.01\\
34.84	0.01\\
34.85	0.01\\
34.86	0.01\\
34.87	0.01\\
34.88	0.01\\
34.89	0.01\\
34.9	0.01\\
34.91	0.01\\
34.92	0.01\\
34.93	0.01\\
34.94	0.01\\
34.95	0.01\\
34.96	0.01\\
34.97	0.01\\
34.98	0.01\\
34.99	0.01\\
35	0.01\\
35.01	0.01\\
35.02	0.01\\
35.03	0.01\\
35.04	0.01\\
35.05	0.01\\
35.06	0.01\\
35.07	0.01\\
35.08	0.01\\
35.09	0.01\\
35.1	0.01\\
35.11	0.01\\
35.12	0.01\\
35.13	0.01\\
35.14	0.01\\
35.15	0.01\\
35.16	0.01\\
35.17	0.01\\
35.18	0.01\\
35.19	0.01\\
35.2	0.01\\
35.21	0.01\\
35.22	0.01\\
35.23	0.01\\
35.24	0.01\\
35.25	0.01\\
35.26	0.01\\
35.27	0.01\\
35.28	0.01\\
35.29	0.01\\
35.3	0.01\\
35.31	0.01\\
35.32	0.01\\
35.33	0.01\\
35.34	0.01\\
35.35	0.01\\
35.36	0.01\\
35.37	0.01\\
35.38	0.01\\
35.39	0.01\\
35.4	0.01\\
35.41	0.01\\
35.42	0.01\\
35.43	0.01\\
35.44	0.01\\
35.45	0.01\\
35.46	0.01\\
35.47	0.01\\
35.48	0.01\\
35.49	0.01\\
35.5	0.01\\
35.51	0.01\\
35.52	0.01\\
35.53	0.01\\
35.54	0.01\\
35.55	0.01\\
35.56	0.01\\
35.57	0.01\\
35.58	0.01\\
35.59	0.01\\
35.6	0.01\\
35.61	0.01\\
35.62	0.01\\
35.63	0.01\\
35.64	0.01\\
35.65	0.01\\
35.66	0.01\\
35.67	0.01\\
35.68	0.01\\
35.69	0.01\\
35.7	0.01\\
35.71	0.01\\
35.72	0.01\\
35.73	0.01\\
35.74	0.01\\
35.75	0.01\\
35.76	0.01\\
35.77	0.01\\
35.78	0.01\\
35.79	0.01\\
35.8	0.01\\
35.81	0.01\\
35.82	0.01\\
35.83	0.01\\
35.84	0.01\\
35.85	0.01\\
35.86	0.01\\
35.87	0.01\\
35.88	0.01\\
35.89	0.01\\
35.9	0.01\\
35.91	0.01\\
35.92	0.01\\
35.93	0.01\\
35.94	0.01\\
35.95	0.01\\
35.96	0.01\\
35.97	0.01\\
35.98	0.01\\
35.99	0.01\\
36	0.01\\
36.01	0.01\\
36.02	0.01\\
36.03	0.01\\
36.04	0.01\\
36.05	0.01\\
36.06	0.01\\
36.07	0.01\\
36.08	0.01\\
36.09	0.01\\
36.1	0.01\\
36.11	0.01\\
36.12	0.01\\
36.13	0.01\\
36.14	0.01\\
36.15	0.01\\
36.16	0.01\\
36.17	0.01\\
36.18	0.01\\
36.19	0.01\\
36.2	0.01\\
36.21	0.01\\
36.22	0.01\\
36.23	0.01\\
36.24	0.01\\
36.25	0.01\\
36.26	0.01\\
36.27	0.01\\
36.28	0.01\\
36.29	0.01\\
36.3	0.01\\
36.31	0.01\\
36.32	0.01\\
36.33	0.01\\
36.34	0.01\\
36.35	0.01\\
36.36	0.01\\
36.37	0.01\\
36.38	0.01\\
36.39	0.01\\
36.4	0.01\\
36.41	0.01\\
36.42	0.01\\
36.43	0.01\\
36.44	0.01\\
36.45	0.01\\
36.46	0.01\\
36.47	0.01\\
36.48	0.01\\
36.49	0.01\\
36.5	0.01\\
36.51	0.01\\
36.52	0.01\\
36.53	0.01\\
36.54	0.01\\
36.55	0.01\\
36.56	0.01\\
36.57	0.01\\
36.58	0.01\\
36.59	0.01\\
36.6	0.01\\
36.61	0.01\\
36.62	0.01\\
36.63	0.01\\
36.64	0.01\\
36.65	0.01\\
36.66	0.01\\
36.67	0.01\\
36.68	0.01\\
36.69	0.01\\
36.7	0.01\\
36.71	0.01\\
36.72	0.01\\
36.73	0.01\\
36.74	0.01\\
36.75	0.01\\
36.76	0.01\\
36.77	0.01\\
36.78	0.01\\
36.79	0.01\\
36.8	0.01\\
36.81	0.01\\
36.82	0.01\\
36.83	0.01\\
36.84	0.01\\
36.85	0.01\\
36.86	0.01\\
36.87	0.01\\
36.88	0.01\\
36.89	0.01\\
36.9	0.01\\
36.91	0.01\\
36.92	0.01\\
36.93	0.01\\
36.94	0.01\\
36.95	0.01\\
36.96	0.01\\
36.97	0.01\\
36.98	0.01\\
36.99	0.01\\
37	0.01\\
37.01	0.01\\
37.02	0.01\\
37.03	0.01\\
37.04	0.01\\
37.05	0.01\\
37.06	0.01\\
37.07	0.01\\
37.08	0.01\\
37.09	0.01\\
37.1	0.01\\
37.11	0.01\\
37.12	0.01\\
37.13	0.01\\
37.14	0.01\\
37.15	0.01\\
37.16	0.01\\
37.17	0.01\\
37.18	0.01\\
37.19	0.01\\
37.2	0.01\\
37.21	0.01\\
37.22	0.01\\
37.23	0.01\\
37.24	0.01\\
37.25	0.01\\
37.26	0.01\\
37.27	0.01\\
37.28	0.01\\
37.29	0.01\\
37.3	0.01\\
37.31	0.01\\
37.32	0.01\\
37.33	0.01\\
37.34	0.01\\
37.35	0.01\\
37.36	0.01\\
37.37	0.01\\
37.38	0.01\\
37.39	0.01\\
37.4	0.01\\
37.41	0.01\\
37.42	0.01\\
37.43	0.01\\
37.44	0.01\\
37.45	0.01\\
37.46	0.01\\
37.47	0.01\\
37.48	0.01\\
37.49	0.01\\
37.5	0.01\\
37.51	0.01\\
37.52	0.01\\
37.53	0.01\\
37.54	0.01\\
37.55	0.01\\
37.56	0.01\\
37.57	0.01\\
37.58	0.01\\
37.59	0.01\\
37.6	0.01\\
37.61	0.01\\
37.62	0.01\\
37.63	0.01\\
37.64	0.01\\
37.65	0.01\\
37.66	0.01\\
37.67	0.01\\
37.68	0.01\\
37.69	0.01\\
37.7	0.01\\
37.71	0.01\\
37.72	0.01\\
37.73	0.01\\
37.74	0.01\\
37.75	0.01\\
37.76	0.01\\
37.77	0.01\\
37.78	0.01\\
37.79	0.01\\
37.8	0.01\\
37.81	0.01\\
37.82	0.01\\
37.83	0.01\\
37.84	0.01\\
37.85	0.01\\
37.86	0.01\\
37.87	0.01\\
37.88	0.01\\
37.89	0.01\\
37.9	0.01\\
37.91	0.01\\
37.92	0.01\\
37.93	0.01\\
37.94	0.01\\
37.95	0.01\\
37.96	0.01\\
37.97	0.01\\
37.98	0.01\\
37.99	0.01\\
38	0.01\\
38.01	0.01\\
38.02	0.01\\
38.03	0.01\\
38.04	0.01\\
38.05	0.01\\
38.06	0.01\\
38.07	0.01\\
38.08	0.01\\
38.09	0.01\\
38.1	0.01\\
38.11	0.01\\
38.12	0.01\\
38.13	0.01\\
38.14	0.01\\
38.15	0.01\\
38.16	0.01\\
38.17	0.01\\
38.18	0.01\\
38.19	0.01\\
38.2	0.01\\
38.21	0.01\\
38.22	0.01\\
38.23	0.01\\
38.24	0.01\\
38.25	0.01\\
38.26	0.01\\
38.27	0.01\\
38.28	0.01\\
38.29	0.01\\
38.3	0.01\\
38.31	0.01\\
38.32	0.01\\
38.33	0.01\\
38.34	0.01\\
38.35	0.01\\
38.36	0.01\\
38.37	0.01\\
38.38	0.01\\
38.39	0.01\\
38.4	0.01\\
38.41	0.01\\
38.42	0.01\\
38.43	0.01\\
38.44	0.01\\
38.45	0.01\\
38.46	0.01\\
38.47	0.01\\
38.48	0.01\\
38.49	0.01\\
38.5	0.01\\
38.51	0.01\\
38.52	0.01\\
38.53	0.01\\
38.54	0.01\\
38.55	0.01\\
38.56	0.01\\
38.57	0.01\\
38.58	0.01\\
38.59	0.01\\
38.6	0.01\\
38.61	0.01\\
38.62	0.01\\
38.63	0.01\\
38.64	0.01\\
38.65	0.01\\
38.66	0.01\\
38.67	0.01\\
38.68	0.01\\
38.69	0.01\\
38.7	0.01\\
38.71	0.01\\
38.72	0.01\\
38.73	0.01\\
38.74	0.01\\
38.75	0.01\\
38.76	0.01\\
38.77	0.01\\
38.78	0.01\\
38.79	0.01\\
38.8	0.01\\
38.81	0.01\\
38.82	0.01\\
38.83	0.01\\
38.84	0.01\\
38.85	0.01\\
38.86	0.01\\
38.87	0.01\\
38.88	0.01\\
38.89	0.01\\
38.9	0.01\\
38.91	0.01\\
38.92	0.01\\
38.93	0.01\\
38.94	0.01\\
38.95	0.01\\
38.96	0.01\\
38.97	0.01\\
38.98	0.01\\
38.99	0.01\\
39	0.01\\
39.01	0.01\\
39.02	0.01\\
39.03	0.01\\
39.04	0.01\\
39.05	0.01\\
39.06	0.01\\
39.07	0.01\\
39.08	0.01\\
39.09	0.01\\
39.1	0.01\\
39.11	0.01\\
39.12	0.01\\
39.13	0.01\\
39.14	0.01\\
39.15	0.01\\
39.16	0.01\\
39.17	0.01\\
39.18	0.01\\
39.19	0.01\\
39.2	0.01\\
39.21	0.01\\
39.22	0.01\\
39.23	0.01\\
39.24	0.01\\
39.25	0.01\\
39.26	0.01\\
39.27	0.01\\
39.28	0.01\\
39.29	0.01\\
39.3	0.01\\
39.31	0.01\\
39.32	0.01\\
39.33	0.01\\
39.34	0.01\\
39.35	0.01\\
39.36	0.01\\
39.37	0.01\\
39.38	0.01\\
39.39	0.01\\
39.4	0.01\\
39.41	0.01\\
39.42	0.01\\
39.43	0.01\\
39.44	0.01\\
39.45	0.01\\
39.46	0.01\\
39.47	0.01\\
39.48	0.01\\
39.49	0.01\\
39.5	0.01\\
39.51	0.01\\
39.52	0.01\\
39.53	0.01\\
39.54	0.01\\
39.55	0.01\\
39.56	0.01\\
39.57	0.01\\
39.58	0.01\\
39.59	0.01\\
39.6	0.01\\
39.61	0.01\\
39.62	0.01\\
39.63	0.01\\
39.64	0.01\\
39.65	0.01\\
39.66	0.01\\
39.67	0.01\\
39.68	0.01\\
39.69	0.01\\
39.7	0.01\\
39.71	0.01\\
39.72	0.01\\
39.73	0.01\\
39.74	0.01\\
39.75	0.01\\
39.76	0.01\\
39.77	0.01\\
39.78	0.01\\
39.79	0.01\\
39.8	0.01\\
39.81	0.01\\
39.82	0.01\\
39.83	0.01\\
39.84	0.01\\
39.85	0.01\\
39.86	0.01\\
39.87	0.01\\
39.88	0.01\\
39.89	0.01\\
39.9	0.01\\
39.91	0.01\\
39.92	0.01\\
39.93	0.01\\
39.94	0.01\\
39.95	0.01\\
39.96	0.01\\
39.97	0.01\\
39.98	0.01\\
39.99	0.01\\
40	0.01\\
40.01	0.01\\
};
\addplot [color=red,solid,forget plot]
  table[row sep=crcr]{%
40.01	0.01\\
40.02	0.01\\
40.03	0.01\\
40.04	0.01\\
40.05	0.01\\
40.06	0.01\\
40.07	0.01\\
40.08	0.01\\
40.09	0.01\\
40.1	0.01\\
40.11	0.01\\
40.12	0.01\\
40.13	0.01\\
40.14	0.01\\
40.15	0.01\\
40.16	0.01\\
40.17	0.01\\
40.18	0.01\\
40.19	0.01\\
40.2	0.01\\
40.21	0.01\\
40.22	0.01\\
40.23	0.01\\
40.24	0.01\\
40.25	0.01\\
40.26	0.01\\
40.27	0.01\\
40.28	0.01\\
40.29	0.01\\
40.3	0.01\\
40.31	0.01\\
40.32	0.01\\
40.33	0.01\\
40.34	0.01\\
40.35	0.01\\
40.36	0.01\\
40.37	0.01\\
40.38	0.01\\
40.39	0.01\\
40.4	0.01\\
40.41	0.01\\
40.42	0.01\\
40.43	0.01\\
40.44	0.01\\
40.45	0.01\\
40.46	0.01\\
40.47	0.01\\
40.48	0.01\\
40.49	0.01\\
40.5	0.01\\
40.51	0.01\\
40.52	0.01\\
40.53	0.01\\
40.54	0.01\\
40.55	0.01\\
40.56	0.01\\
40.57	0.01\\
40.58	0.01\\
40.59	0.01\\
40.6	0.01\\
40.61	0.01\\
40.62	0.01\\
40.63	0.01\\
40.64	0.01\\
40.65	0.01\\
40.66	0.01\\
40.67	0.01\\
40.68	0.01\\
40.69	0.01\\
40.7	0.01\\
40.71	0.01\\
40.72	0.01\\
40.73	0.01\\
40.74	0.01\\
40.75	0.01\\
40.76	0.01\\
40.77	0.01\\
40.78	0.01\\
40.79	0.01\\
40.8	0.01\\
40.81	0.01\\
40.82	0.01\\
40.83	0.01\\
40.84	0.01\\
40.85	0.01\\
40.86	0.01\\
40.87	0.01\\
40.88	0.01\\
40.89	0.01\\
40.9	0.01\\
40.91	0.01\\
40.92	0.01\\
40.93	0.01\\
40.94	0.01\\
40.95	0.01\\
40.96	0.01\\
40.97	0.01\\
40.98	0.01\\
40.99	0.01\\
41	0.01\\
41.01	0.01\\
41.02	0.01\\
41.03	0.01\\
41.04	0.01\\
41.05	0.01\\
41.06	0.01\\
41.07	0.01\\
41.08	0.01\\
41.09	0.01\\
41.1	0.01\\
41.11	0.01\\
41.12	0.01\\
41.13	0.01\\
41.14	0.01\\
41.15	0.01\\
41.16	0.01\\
41.17	0.01\\
41.18	0.01\\
41.19	0.01\\
41.2	0.01\\
41.21	0.01\\
41.22	0.01\\
41.23	0.01\\
41.24	0.01\\
41.25	0.01\\
41.26	0.01\\
41.27	0.01\\
41.28	0.01\\
41.29	0.01\\
41.3	0.01\\
41.31	0.01\\
41.32	0.01\\
41.33	0.01\\
41.34	0.01\\
41.35	0.01\\
41.36	0.01\\
41.37	0.01\\
41.38	0.01\\
41.39	0.01\\
41.4	0.01\\
41.41	0.01\\
41.42	0.01\\
41.43	0.01\\
41.44	0.01\\
41.45	0.01\\
41.46	0.01\\
41.47	0.01\\
41.48	0.01\\
41.49	0.01\\
41.5	0.01\\
41.51	0.01\\
41.52	0.01\\
41.53	0.01\\
41.54	0.01\\
41.55	0.01\\
41.56	0.01\\
41.57	0.01\\
41.58	0.01\\
41.59	0.01\\
41.6	0.01\\
41.61	0.01\\
41.62	0.01\\
41.63	0.01\\
41.64	0.01\\
41.65	0.01\\
41.66	0.01\\
41.67	0.01\\
41.68	0.01\\
41.69	0.01\\
41.7	0.01\\
41.71	0.01\\
41.72	0.01\\
41.73	0.01\\
41.74	0.01\\
41.75	0.01\\
41.76	0.01\\
41.77	0.01\\
41.78	0.01\\
41.79	0.01\\
41.8	0.01\\
41.81	0.01\\
41.82	0.01\\
41.83	0.01\\
41.84	0.01\\
41.85	0.01\\
41.86	0.01\\
41.87	0.01\\
41.88	0.01\\
41.89	0.01\\
41.9	0.01\\
41.91	0.01\\
41.92	0.01\\
41.93	0.01\\
41.94	0.01\\
41.95	0.01\\
41.96	0.01\\
41.97	0.01\\
41.98	0.01\\
41.99	0.01\\
42	0.01\\
42.01	0.01\\
42.02	0.01\\
42.03	0.01\\
42.04	0.01\\
42.05	0.01\\
42.06	0.01\\
42.07	0.01\\
42.08	0.01\\
42.09	0.01\\
42.1	0.01\\
42.11	0.01\\
42.12	0.01\\
42.13	0.01\\
42.14	0.01\\
42.15	0.01\\
42.16	0.01\\
42.17	0.01\\
42.18	0.01\\
42.19	0.01\\
42.2	0.01\\
42.21	0.01\\
42.22	0.01\\
42.23	0.01\\
42.24	0.01\\
42.25	0.01\\
42.26	0.01\\
42.27	0.01\\
42.28	0.01\\
42.29	0.01\\
42.3	0.01\\
42.31	0.01\\
42.32	0.01\\
42.33	0.01\\
42.34	0.01\\
42.35	0.01\\
42.36	0.01\\
42.37	0.01\\
42.38	0.01\\
42.39	0.01\\
42.4	0.01\\
42.41	0.01\\
42.42	0.01\\
42.43	0.01\\
42.44	0.01\\
42.45	0.01\\
42.46	0.01\\
42.47	0.01\\
42.48	0.01\\
42.49	0.01\\
42.5	0.01\\
42.51	0.01\\
42.52	0.01\\
42.53	0.01\\
42.54	0.01\\
42.55	0.01\\
42.56	0.01\\
42.57	0.01\\
42.58	0.01\\
42.59	0.01\\
42.6	0.01\\
42.61	0.01\\
42.62	0.01\\
42.63	0.01\\
42.64	0.01\\
42.65	0.01\\
42.66	0.01\\
42.67	0.01\\
42.68	0.01\\
42.69	0.01\\
42.7	0.01\\
42.71	0.01\\
42.72	0.01\\
42.73	0.01\\
42.74	0.01\\
42.75	0.01\\
42.76	0.01\\
42.77	0.01\\
42.78	0.01\\
42.79	0.01\\
42.8	0.01\\
42.81	0.01\\
42.82	0.01\\
42.83	0.01\\
42.84	0.01\\
42.85	0.01\\
42.86	0.01\\
42.87	0.01\\
42.88	0.01\\
42.89	0.01\\
42.9	0.01\\
42.91	0.01\\
42.92	0.01\\
42.93	0.01\\
42.94	0.01\\
42.95	0.01\\
42.96	0.01\\
42.97	0.01\\
42.98	0.01\\
42.99	0.01\\
43	0.01\\
43.01	0.01\\
43.02	0.01\\
43.03	0.01\\
43.04	0.01\\
43.05	0.01\\
43.06	0.01\\
43.07	0.01\\
43.08	0.01\\
43.09	0.01\\
43.1	0.01\\
43.11	0.01\\
43.12	0.01\\
43.13	0.01\\
43.14	0.01\\
43.15	0.01\\
43.16	0.01\\
43.17	0.01\\
43.18	0.01\\
43.19	0.01\\
43.2	0.01\\
43.21	0.01\\
43.22	0.01\\
43.23	0.01\\
43.24	0.01\\
43.25	0.01\\
43.26	0.01\\
43.27	0.01\\
43.28	0.01\\
43.29	0.01\\
43.3	0.01\\
43.31	0.01\\
43.32	0.01\\
43.33	0.01\\
43.34	0.01\\
43.35	0.01\\
43.36	0.01\\
43.37	0.01\\
43.38	0.01\\
43.39	0.01\\
43.4	0.01\\
43.41	0.01\\
43.42	0.01\\
43.43	0.01\\
43.44	0.01\\
43.45	0.01\\
43.46	0.01\\
43.47	0.01\\
43.48	0.01\\
43.49	0.01\\
43.5	0.01\\
43.51	0.01\\
43.52	0.01\\
43.53	0.01\\
43.54	0.01\\
43.55	0.01\\
43.56	0.01\\
43.57	0.01\\
43.58	0.01\\
43.59	0.01\\
43.6	0.01\\
43.61	0.01\\
43.62	0.01\\
43.63	0.01\\
43.64	0.01\\
43.65	0.01\\
43.66	0.01\\
43.67	0.01\\
43.68	0.01\\
43.69	0.01\\
43.7	0.01\\
43.71	0.01\\
43.72	0.01\\
43.73	0.01\\
43.74	0.01\\
43.75	0.01\\
43.76	0.01\\
43.77	0.01\\
43.78	0.01\\
43.79	0.01\\
43.8	0.01\\
43.81	0.01\\
43.82	0.01\\
43.83	0.01\\
43.84	0.01\\
43.85	0.01\\
43.86	0.01\\
43.87	0.01\\
43.88	0.01\\
43.89	0.01\\
43.9	0.01\\
43.91	0.01\\
43.92	0.01\\
43.93	0.01\\
43.94	0.01\\
43.95	0.01\\
43.96	0.01\\
43.97	0.01\\
43.98	0.01\\
43.99	0.01\\
44	0.01\\
44.01	0.01\\
44.02	0.01\\
44.03	0.01\\
44.04	0.01\\
44.05	0.01\\
44.06	0.01\\
44.07	0.01\\
44.08	0.01\\
44.09	0.01\\
44.1	0.01\\
44.11	0.01\\
44.12	0.01\\
44.13	0.01\\
44.14	0.01\\
44.15	0.01\\
44.16	0.01\\
44.17	0.01\\
44.18	0.01\\
44.19	0.01\\
44.2	0.01\\
44.21	0.01\\
44.22	0.01\\
44.23	0.01\\
44.24	0.01\\
44.25	0.01\\
44.26	0.01\\
44.27	0.01\\
44.28	0.01\\
44.29	0.01\\
44.3	0.01\\
44.31	0.01\\
44.32	0.01\\
44.33	0.01\\
44.34	0.01\\
44.35	0.01\\
44.36	0.01\\
44.37	0.01\\
44.38	0.01\\
44.39	0.01\\
44.4	0.01\\
44.41	0.01\\
44.42	0.01\\
44.43	0.01\\
44.44	0.01\\
44.45	0.01\\
44.46	0.01\\
44.47	0.01\\
44.48	0.01\\
44.49	0.01\\
44.5	0.01\\
44.51	0.01\\
44.52	0.01\\
44.53	0.01\\
44.54	0.01\\
44.55	0.01\\
44.56	0.01\\
44.57	0.01\\
44.58	0.01\\
44.59	0.01\\
44.6	0.01\\
44.61	0.01\\
44.62	0.01\\
44.63	0.01\\
44.64	0.01\\
44.65	0.01\\
44.66	0.01\\
44.67	0.01\\
44.68	0.01\\
44.69	0.01\\
44.7	0.01\\
44.71	0.01\\
44.72	0.01\\
44.73	0.01\\
44.74	0.01\\
44.75	0.01\\
44.76	0.01\\
44.77	0.01\\
44.78	0.01\\
44.79	0.01\\
44.8	0.01\\
44.81	0.01\\
44.82	0.01\\
44.83	0.01\\
44.84	0.01\\
44.85	0.01\\
44.86	0.01\\
44.87	0.01\\
44.88	0.01\\
44.89	0.01\\
44.9	0.01\\
44.91	0.01\\
44.92	0.01\\
44.93	0.01\\
44.94	0.01\\
44.95	0.01\\
44.96	0.01\\
44.97	0.01\\
44.98	0.01\\
44.99	0.01\\
45	0.01\\
45.01	0.01\\
45.02	0.01\\
45.03	0.01\\
45.04	0.01\\
45.05	0.01\\
45.06	0.01\\
45.07	0.01\\
45.08	0.01\\
45.09	0.01\\
45.1	0.01\\
45.11	0.01\\
45.12	0.01\\
45.13	0.01\\
45.14	0.01\\
45.15	0.01\\
45.16	0.01\\
45.17	0.01\\
45.18	0.01\\
45.19	0.01\\
45.2	0.01\\
45.21	0.01\\
45.22	0.01\\
45.23	0.01\\
45.24	0.01\\
45.25	0.01\\
45.26	0.01\\
45.27	0.01\\
45.28	0.01\\
45.29	0.01\\
45.3	0.01\\
45.31	0.01\\
45.32	0.01\\
45.33	0.01\\
45.34	0.01\\
45.35	0.01\\
45.36	0.01\\
45.37	0.01\\
45.38	0.01\\
45.39	0.01\\
45.4	0.01\\
45.41	0.01\\
45.42	0.01\\
45.43	0.01\\
45.44	0.01\\
45.45	0.01\\
45.46	0.01\\
45.47	0.01\\
45.48	0.01\\
45.49	0.01\\
45.5	0.01\\
45.51	0.01\\
45.52	0.01\\
45.53	0.01\\
45.54	0.01\\
45.55	0.01\\
45.56	0.01\\
45.57	0.01\\
45.58	0.01\\
45.59	0.01\\
45.6	0.01\\
45.61	0.01\\
45.62	0.01\\
45.63	0.01\\
45.64	0.01\\
45.65	0.01\\
45.66	0.01\\
45.67	0.01\\
45.68	0.01\\
45.69	0.01\\
45.7	0.01\\
45.71	0.01\\
45.72	0.01\\
45.73	0.01\\
45.74	0.01\\
45.75	0.01\\
45.76	0.01\\
45.77	0.01\\
45.78	0.01\\
45.79	0.01\\
45.8	0.01\\
45.81	0.01\\
45.82	0.01\\
45.83	0.01\\
45.84	0.01\\
45.85	0.01\\
45.86	0.01\\
45.87	0.01\\
45.88	0.01\\
45.89	0.01\\
45.9	0.01\\
45.91	0.01\\
45.92	0.01\\
45.93	0.01\\
45.94	0.01\\
45.95	0.01\\
45.96	0.01\\
45.97	0.01\\
45.98	0.01\\
45.99	0.01\\
46	0.01\\
46.01	0.01\\
46.02	0.01\\
46.03	0.01\\
46.04	0.01\\
46.05	0.01\\
46.06	0.01\\
46.07	0.01\\
46.08	0.01\\
46.09	0.01\\
46.1	0.01\\
46.11	0.01\\
46.12	0.01\\
46.13	0.01\\
46.14	0.01\\
46.15	0.01\\
46.16	0.01\\
46.17	0.01\\
46.18	0.01\\
46.19	0.01\\
46.2	0.01\\
46.21	0.01\\
46.22	0.01\\
46.23	0.01\\
46.24	0.01\\
46.25	0.01\\
46.26	0.01\\
46.27	0.01\\
46.28	0.01\\
46.29	0.01\\
46.3	0.01\\
46.31	0.01\\
46.32	0.01\\
46.33	0.01\\
46.34	0.01\\
46.35	0.01\\
46.36	0.01\\
46.37	0.01\\
46.38	0.01\\
46.39	0.01\\
46.4	0.01\\
46.41	0.01\\
46.42	0.01\\
46.43	0.01\\
46.44	0.01\\
46.45	0.01\\
46.46	0.01\\
46.47	0.01\\
46.48	0.01\\
46.49	0.01\\
46.5	0.01\\
46.51	0.01\\
46.52	0.01\\
46.53	0.01\\
46.54	0.01\\
46.55	0.01\\
46.56	0.01\\
46.57	0.01\\
46.58	0.01\\
46.59	0.01\\
46.6	0.01\\
46.61	0.01\\
46.62	0.01\\
46.63	0.01\\
46.64	0.01\\
46.65	0.01\\
46.66	0.01\\
46.67	0.01\\
46.68	0.01\\
46.69	0.01\\
46.7	0.01\\
46.71	0.01\\
46.72	0.01\\
46.73	0.01\\
46.74	0.01\\
46.75	0.01\\
46.76	0.01\\
46.77	0.01\\
46.78	0.01\\
46.79	0.01\\
46.8	0.01\\
46.81	0.01\\
46.82	0.01\\
46.83	0.01\\
46.84	0.01\\
46.85	0.01\\
46.86	0.01\\
46.87	0.01\\
46.88	0.01\\
46.89	0.01\\
46.9	0.01\\
46.91	0.01\\
46.92	0.01\\
46.93	0.01\\
46.94	0.01\\
46.95	0.01\\
46.96	0.01\\
46.97	0.01\\
46.98	0.01\\
46.99	0.01\\
47	0.01\\
47.01	0.01\\
47.02	0.01\\
47.03	0.01\\
47.04	0.01\\
47.05	0.01\\
47.06	0.01\\
47.07	0.01\\
47.08	0.01\\
47.09	0.01\\
47.1	0.01\\
47.11	0.01\\
47.12	0.01\\
47.13	0.01\\
47.14	0.01\\
47.15	0.01\\
47.16	0.01\\
47.17	0.01\\
47.18	0.01\\
47.19	0.01\\
47.2	0.01\\
47.21	0.01\\
47.22	0.01\\
47.23	0.01\\
47.24	0.01\\
47.25	0.01\\
47.26	0.01\\
47.27	0.01\\
47.28	0.01\\
47.29	0.01\\
47.3	0.01\\
47.31	0.01\\
47.32	0.01\\
47.33	0.01\\
47.34	0.01\\
47.35	0.01\\
47.36	0.01\\
47.37	0.01\\
47.38	0.01\\
47.39	0.01\\
47.4	0.01\\
47.41	0.01\\
47.42	0.01\\
47.43	0.01\\
47.44	0.01\\
47.45	0.01\\
47.46	0.01\\
47.47	0.01\\
47.48	0.01\\
47.49	0.01\\
47.5	0.01\\
47.51	0.01\\
47.52	0.01\\
47.53	0.01\\
47.54	0.01\\
47.55	0.01\\
47.56	0.01\\
47.57	0.01\\
47.58	0.01\\
47.59	0.01\\
47.6	0.01\\
47.61	0.01\\
47.62	0.01\\
47.63	0.01\\
47.64	0.01\\
47.65	0.01\\
47.66	0.01\\
47.67	0.01\\
47.68	0.01\\
47.69	0.01\\
47.7	0.01\\
47.71	0.01\\
47.72	0.01\\
47.73	0.01\\
47.74	0.01\\
47.75	0.01\\
47.76	0.01\\
47.77	0.01\\
47.78	0.01\\
47.79	0.01\\
47.8	0.01\\
47.81	0.01\\
47.82	0.01\\
47.83	0.01\\
47.84	0.01\\
47.85	0.01\\
47.86	0.01\\
47.87	0.01\\
47.88	0.01\\
47.89	0.01\\
47.9	0.01\\
47.91	0.01\\
47.92	0.01\\
47.93	0.01\\
47.94	0.01\\
47.95	0.01\\
47.96	0.01\\
47.97	0.01\\
47.98	0.01\\
47.99	0.01\\
48	0.01\\
48.01	0.01\\
48.02	0.01\\
48.03	0.01\\
48.04	0.01\\
48.05	0.01\\
48.06	0.01\\
48.07	0.01\\
48.08	0.01\\
48.09	0.01\\
48.1	0.01\\
48.11	0.01\\
48.12	0.01\\
48.13	0.01\\
48.14	0.01\\
48.15	0.01\\
48.16	0.01\\
48.17	0.01\\
48.18	0.01\\
48.19	0.01\\
48.2	0.01\\
48.21	0.01\\
48.22	0.01\\
48.23	0.01\\
48.24	0.01\\
48.25	0.01\\
48.26	0.01\\
48.27	0.01\\
48.28	0.01\\
48.29	0.01\\
48.3	0.01\\
48.31	0.01\\
48.32	0.01\\
48.33	0.01\\
48.34	0.01\\
48.35	0.01\\
48.36	0.01\\
48.37	0.01\\
48.38	0.01\\
48.39	0.01\\
48.4	0.01\\
48.41	0.01\\
48.42	0.01\\
48.43	0.01\\
48.44	0.01\\
48.45	0.01\\
48.46	0.01\\
48.47	0.01\\
48.48	0.01\\
48.49	0.01\\
48.5	0.01\\
48.51	0.01\\
48.52	0.01\\
48.53	0.01\\
48.54	0.01\\
48.55	0.01\\
48.56	0.01\\
48.57	0.01\\
48.58	0.01\\
48.59	0.01\\
48.6	0.01\\
48.61	0.01\\
48.62	0.01\\
48.63	0.01\\
48.64	0.01\\
48.65	0.01\\
48.66	0.01\\
48.67	0.01\\
48.68	0.01\\
48.69	0.01\\
48.7	0.01\\
48.71	0.01\\
48.72	0.01\\
48.73	0.01\\
48.74	0.01\\
48.75	0.01\\
48.76	0.01\\
48.77	0.01\\
48.78	0.01\\
48.79	0.01\\
48.8	0.01\\
48.81	0.01\\
48.82	0.01\\
48.83	0.01\\
48.84	0.01\\
48.85	0.01\\
48.86	0.01\\
48.87	0.01\\
48.88	0.01\\
48.89	0.01\\
48.9	0.01\\
48.91	0.01\\
48.92	0.01\\
48.93	0.01\\
48.94	0.01\\
48.95	0.01\\
48.96	0.01\\
48.97	0.01\\
48.98	0.01\\
48.99	0.01\\
49	0.01\\
49.01	0.01\\
49.02	0.01\\
49.03	0.01\\
49.04	0.01\\
49.05	0.01\\
49.06	0.01\\
49.07	0.01\\
49.08	0.01\\
49.09	0.01\\
49.1	0.01\\
49.11	0.01\\
49.12	0.01\\
49.13	0.01\\
49.14	0.01\\
49.15	0.01\\
49.16	0.01\\
49.17	0.01\\
49.18	0.01\\
49.19	0.01\\
49.2	0.01\\
49.21	0.01\\
49.22	0.01\\
49.23	0.01\\
49.24	0.01\\
49.25	0.01\\
49.26	0.01\\
49.27	0.01\\
49.28	0.01\\
49.29	0.01\\
49.3	0.01\\
49.31	0.01\\
49.32	0.01\\
49.33	0.01\\
49.34	0.01\\
49.35	0.01\\
49.36	0.01\\
49.37	0.01\\
49.38	0.01\\
49.39	0.01\\
49.4	0.01\\
49.41	0.01\\
49.42	0.01\\
49.43	0.01\\
49.44	0.01\\
49.45	0.01\\
49.46	0.01\\
49.47	0.01\\
49.48	0.01\\
49.49	0.01\\
49.5	0.01\\
49.51	0.01\\
49.52	0.01\\
49.53	0.01\\
49.54	0.01\\
49.55	0.01\\
49.56	0.01\\
49.57	0.01\\
49.58	0.01\\
49.59	0.01\\
49.6	0.01\\
49.61	0.01\\
49.62	0.01\\
49.63	0.01\\
49.64	0.01\\
49.65	0.01\\
49.66	0.01\\
49.67	0.01\\
49.68	0.01\\
49.69	0.01\\
49.7	0.01\\
49.71	0.01\\
49.72	0.01\\
49.73	0.01\\
49.74	0.01\\
49.75	0.01\\
49.76	0.01\\
49.77	0.01\\
49.78	0.01\\
49.79	0.01\\
49.8	0.01\\
49.81	0.01\\
49.82	0.01\\
49.83	0.01\\
49.84	0.01\\
49.85	0.01\\
49.86	0.01\\
49.87	0.01\\
49.88	0.01\\
49.89	0.01\\
49.9	0.01\\
49.91	0.01\\
49.92	0.01\\
49.93	0.01\\
49.94	0.01\\
49.95	0.01\\
49.96	0.01\\
49.97	0.01\\
49.98	0.01\\
49.99	0.01\\
50	0.01\\
50.01	0.01\\
50.02	0.01\\
50.03	0.01\\
50.04	0.01\\
50.05	0.01\\
50.06	0.01\\
50.07	0.01\\
50.08	0.01\\
50.09	0.01\\
50.1	0.01\\
50.11	0.01\\
50.12	0.01\\
50.13	0.01\\
50.14	0.01\\
50.15	0.01\\
50.16	0.01\\
50.17	0.01\\
50.18	0.01\\
50.19	0.01\\
50.2	0.01\\
50.21	0.01\\
50.22	0.01\\
50.23	0.01\\
50.24	0.01\\
50.25	0.01\\
50.26	0.01\\
50.27	0.01\\
50.28	0.01\\
50.29	0.01\\
50.3	0.01\\
50.31	0.01\\
50.32	0.01\\
50.33	0.01\\
50.34	0.01\\
50.35	0.01\\
50.36	0.01\\
50.37	0.01\\
50.38	0.01\\
50.39	0.01\\
50.4	0.01\\
50.41	0.01\\
50.42	0.01\\
50.43	0.01\\
50.44	0.01\\
50.45	0.01\\
50.46	0.01\\
50.47	0.01\\
50.48	0.01\\
50.49	0.01\\
50.5	0.01\\
50.51	0.01\\
50.52	0.01\\
50.53	0.01\\
50.54	0.01\\
50.55	0.01\\
50.56	0.01\\
50.57	0.01\\
50.58	0.01\\
50.59	0.01\\
50.6	0.01\\
50.61	0.01\\
50.62	0.01\\
50.63	0.01\\
50.64	0.01\\
50.65	0.01\\
50.66	0.01\\
50.67	0.01\\
50.68	0.01\\
50.69	0.01\\
50.7	0.01\\
50.71	0.01\\
50.72	0.01\\
50.73	0.01\\
50.74	0.01\\
50.75	0.01\\
50.76	0.01\\
50.77	0.01\\
50.78	0.01\\
50.79	0.01\\
50.8	0.01\\
50.81	0.01\\
50.82	0.01\\
50.83	0.01\\
50.84	0.01\\
50.85	0.01\\
50.86	0.01\\
50.87	0.01\\
50.88	0.01\\
50.89	0.01\\
50.9	0.01\\
50.91	0.01\\
50.92	0.01\\
50.93	0.01\\
50.94	0.01\\
50.95	0.01\\
50.96	0.01\\
50.97	0.01\\
50.98	0.01\\
50.99	0.01\\
51	0.01\\
51.01	0.01\\
51.02	0.01\\
51.03	0.01\\
51.04	0.01\\
51.05	0.01\\
51.06	0.01\\
51.07	0.01\\
51.08	0.01\\
51.09	0.01\\
51.1	0.01\\
51.11	0.01\\
51.12	0.01\\
51.13	0.01\\
51.14	0.01\\
51.15	0.01\\
51.16	0.01\\
51.17	0.01\\
51.18	0.01\\
51.19	0.01\\
51.2	0.01\\
51.21	0.01\\
51.22	0.01\\
51.23	0.01\\
51.24	0.01\\
51.25	0.01\\
51.26	0.01\\
51.27	0.01\\
51.28	0.01\\
51.29	0.01\\
51.3	0.01\\
51.31	0.01\\
51.32	0.01\\
51.33	0.01\\
51.34	0.01\\
51.35	0.01\\
51.36	0.01\\
51.37	0.01\\
51.38	0.01\\
51.39	0.01\\
51.4	0.01\\
51.41	0.01\\
51.42	0.01\\
51.43	0.01\\
51.44	0.01\\
51.45	0.01\\
51.46	0.01\\
51.47	0.01\\
51.48	0.01\\
51.49	0.01\\
51.5	0.01\\
51.51	0.01\\
51.52	0.01\\
51.53	0.01\\
51.54	0.01\\
51.55	0.01\\
51.56	0.01\\
51.57	0.01\\
51.58	0.01\\
51.59	0.01\\
51.6	0.01\\
51.61	0.01\\
51.62	0.01\\
51.63	0.01\\
51.64	0.01\\
51.65	0.01\\
51.66	0.01\\
51.67	0.01\\
51.68	0.01\\
51.69	0.01\\
51.7	0.01\\
51.71	0.01\\
51.72	0.01\\
51.73	0.01\\
51.74	0.01\\
51.75	0.01\\
51.76	0.01\\
51.77	0.01\\
51.78	0.01\\
51.79	0.01\\
51.8	0.01\\
51.81	0.01\\
51.82	0.01\\
51.83	0.01\\
51.84	0.01\\
51.85	0.01\\
51.86	0.01\\
51.87	0.01\\
51.88	0.01\\
51.89	0.01\\
51.9	0.01\\
51.91	0.01\\
51.92	0.01\\
51.93	0.01\\
51.94	0.01\\
51.95	0.01\\
51.96	0.01\\
51.97	0.01\\
51.98	0.01\\
51.99	0.01\\
52	0.01\\
52.01	0.01\\
52.02	0.01\\
52.03	0.01\\
52.04	0.01\\
52.05	0.01\\
52.06	0.01\\
52.07	0.01\\
52.08	0.01\\
52.09	0.01\\
52.1	0.01\\
52.11	0.01\\
52.12	0.01\\
52.13	0.01\\
52.14	0.01\\
52.15	0.01\\
52.16	0.01\\
52.17	0.01\\
52.18	0.01\\
52.19	0.01\\
52.2	0.01\\
52.21	0.01\\
52.22	0.01\\
52.23	0.01\\
52.24	0.01\\
52.25	0.01\\
52.26	0.01\\
52.27	0.01\\
52.28	0.01\\
52.29	0.01\\
52.3	0.01\\
52.31	0.01\\
52.32	0.01\\
52.33	0.01\\
52.34	0.01\\
52.35	0.01\\
52.36	0.01\\
52.37	0.01\\
52.38	0.01\\
52.39	0.01\\
52.4	0.01\\
52.41	0.01\\
52.42	0.01\\
52.43	0.01\\
52.44	0.01\\
52.45	0.01\\
52.46	0.01\\
52.47	0.01\\
52.48	0.01\\
52.49	0.01\\
52.5	0.01\\
52.51	0.01\\
52.52	0.01\\
52.53	0.01\\
52.54	0.01\\
52.55	0.01\\
52.56	0.01\\
52.57	0.01\\
52.58	0.01\\
52.59	0.01\\
52.6	0.01\\
52.61	0.01\\
52.62	0.01\\
52.63	0.01\\
52.64	0.01\\
52.65	0.01\\
52.66	0.01\\
52.67	0.01\\
52.68	0.01\\
52.69	0.01\\
52.7	0.01\\
52.71	0.01\\
52.72	0.01\\
52.73	0.01\\
52.74	0.01\\
52.75	0.01\\
52.76	0.01\\
52.77	0.01\\
52.78	0.01\\
52.79	0.01\\
52.8	0.01\\
52.81	0.01\\
52.82	0.01\\
52.83	0.01\\
52.84	0.01\\
52.85	0.01\\
52.86	0.01\\
52.87	0.01\\
52.88	0.01\\
52.89	0.01\\
52.9	0.01\\
52.91	0.01\\
52.92	0.01\\
52.93	0.01\\
52.94	0.01\\
52.95	0.01\\
52.96	0.01\\
52.97	0.01\\
52.98	0.01\\
52.99	0.01\\
53	0.01\\
53.01	0.01\\
53.02	0.01\\
53.03	0.01\\
53.04	0.01\\
53.05	0.01\\
53.06	0.01\\
53.07	0.01\\
53.08	0.01\\
53.09	0.01\\
53.1	0.01\\
53.11	0.01\\
53.12	0.01\\
53.13	0.01\\
53.14	0.01\\
53.15	0.01\\
53.16	0.01\\
53.17	0.01\\
53.18	0.01\\
53.19	0.01\\
53.2	0.01\\
53.21	0.01\\
53.22	0.01\\
53.23	0.01\\
53.24	0.01\\
53.25	0.01\\
53.26	0.01\\
53.27	0.01\\
53.28	0.01\\
53.29	0.01\\
53.3	0.01\\
53.31	0.01\\
53.32	0.01\\
53.33	0.01\\
53.34	0.01\\
53.35	0.01\\
53.36	0.01\\
53.37	0.01\\
53.38	0.01\\
53.39	0.01\\
53.4	0.01\\
53.41	0.01\\
53.42	0.01\\
53.43	0.01\\
53.44	0.01\\
53.45	0.01\\
53.46	0.01\\
53.47	0.01\\
53.48	0.01\\
53.49	0.01\\
53.5	0.01\\
53.51	0.01\\
53.52	0.01\\
53.53	0.01\\
53.54	0.01\\
53.55	0.01\\
53.56	0.01\\
53.57	0.01\\
53.58	0.01\\
53.59	0.01\\
53.6	0.01\\
53.61	0.01\\
53.62	0.01\\
53.63	0.01\\
53.64	0.01\\
53.65	0.01\\
53.66	0.01\\
53.67	0.01\\
53.68	0.01\\
53.69	0.01\\
53.7	0.01\\
53.71	0.01\\
53.72	0.01\\
53.73	0.01\\
53.74	0.01\\
53.75	0.01\\
53.76	0.01\\
53.77	0.01\\
53.78	0.01\\
53.79	0.01\\
53.8	0.01\\
53.81	0.01\\
53.82	0.01\\
53.83	0.01\\
53.84	0.01\\
53.85	0.01\\
53.86	0.01\\
53.87	0.01\\
53.88	0.01\\
53.89	0.01\\
53.9	0.01\\
53.91	0.01\\
53.92	0.01\\
53.93	0.01\\
53.94	0.01\\
53.95	0.01\\
53.96	0.01\\
53.97	0.01\\
53.98	0.01\\
53.99	0.01\\
54	0.01\\
54.01	0.01\\
54.02	0.01\\
54.03	0.01\\
54.04	0.01\\
54.05	0.01\\
54.06	0.01\\
54.07	0.01\\
54.08	0.01\\
54.09	0.01\\
54.1	0.01\\
54.11	0.01\\
54.12	0.01\\
54.13	0.01\\
54.14	0.01\\
54.15	0.01\\
54.16	0.01\\
54.17	0.01\\
54.18	0.01\\
54.19	0.01\\
54.2	0.01\\
54.21	0.01\\
54.22	0.01\\
54.23	0.01\\
54.24	0.01\\
54.25	0.01\\
54.26	0.01\\
54.27	0.01\\
54.28	0.01\\
54.29	0.01\\
54.3	0.01\\
54.31	0.01\\
54.32	0.01\\
54.33	0.01\\
54.34	0.01\\
54.35	0.01\\
54.36	0.01\\
54.37	0.01\\
54.38	0.01\\
54.39	0.01\\
54.4	0.01\\
54.41	0.01\\
54.42	0.01\\
54.43	0.01\\
54.44	0.01\\
54.45	0.01\\
54.46	0.01\\
54.47	0.01\\
54.48	0.01\\
54.49	0.01\\
54.5	0.01\\
54.51	0.01\\
54.52	0.01\\
54.53	0.01\\
54.54	0.01\\
54.55	0.01\\
54.56	0.01\\
54.57	0.01\\
54.58	0.01\\
54.59	0.01\\
54.6	0.01\\
54.61	0.01\\
54.62	0.01\\
54.63	0.01\\
54.64	0.01\\
54.65	0.01\\
54.66	0.01\\
54.67	0.01\\
54.68	0.01\\
54.69	0.01\\
54.7	0.01\\
54.71	0.01\\
54.72	0.01\\
54.73	0.01\\
54.74	0.01\\
54.75	0.01\\
54.76	0.01\\
54.77	0.01\\
54.78	0.01\\
54.79	0.01\\
54.8	0.01\\
54.81	0.01\\
54.82	0.01\\
54.83	0.01\\
54.84	0.01\\
54.85	0.01\\
54.86	0.01\\
54.87	0.01\\
54.88	0.01\\
54.89	0.01\\
54.9	0.01\\
54.91	0.01\\
54.92	0.01\\
54.93	0.01\\
54.94	0.01\\
54.95	0.01\\
54.96	0.01\\
54.97	0.01\\
54.98	0.01\\
54.99	0.01\\
55	0.01\\
55.01	0.01\\
55.02	0.01\\
55.03	0.01\\
55.04	0.01\\
55.05	0.01\\
55.06	0.01\\
55.07	0.01\\
55.08	0.01\\
55.09	0.01\\
55.1	0.01\\
55.11	0.01\\
55.12	0.01\\
55.13	0.01\\
55.14	0.01\\
55.15	0.01\\
55.16	0.01\\
55.17	0.01\\
55.18	0.01\\
55.19	0.01\\
55.2	0.01\\
55.21	0.01\\
55.22	0.01\\
55.23	0.01\\
55.24	0.01\\
55.25	0.01\\
55.26	0.01\\
55.27	0.01\\
55.28	0.01\\
55.29	0.01\\
55.3	0.01\\
55.31	0.01\\
55.32	0.01\\
55.33	0.01\\
55.34	0.01\\
55.35	0.01\\
55.36	0.01\\
55.37	0.01\\
55.38	0.01\\
55.39	0.01\\
55.4	0.01\\
55.41	0.01\\
55.42	0.01\\
55.43	0.01\\
55.44	0.01\\
55.45	0.01\\
55.46	0.01\\
55.47	0.01\\
55.48	0.01\\
55.49	0.01\\
55.5	0.01\\
55.51	0.01\\
55.52	0.01\\
55.53	0.01\\
55.54	0.01\\
55.55	0.01\\
55.56	0.01\\
55.57	0.01\\
55.58	0.01\\
55.59	0.01\\
55.6	0.01\\
55.61	0.01\\
55.62	0.01\\
55.63	0.01\\
55.64	0.01\\
55.65	0.01\\
55.66	0.01\\
55.67	0.01\\
55.68	0.01\\
55.69	0.01\\
55.7	0.01\\
55.71	0.01\\
55.72	0.01\\
55.73	0.01\\
55.74	0.01\\
55.75	0.01\\
55.76	0.01\\
55.77	0.01\\
55.78	0.01\\
55.79	0.01\\
55.8	0.01\\
55.81	0.01\\
55.82	0.01\\
55.83	0.01\\
55.84	0.01\\
55.85	0.01\\
55.86	0.01\\
55.87	0.01\\
55.88	0.01\\
55.89	0.01\\
55.9	0.01\\
55.91	0.01\\
55.92	0.01\\
55.93	0.01\\
55.94	0.01\\
55.95	0.01\\
55.96	0.01\\
55.97	0.01\\
55.98	0.01\\
55.99	0.01\\
56	0.01\\
56.01	0.01\\
56.02	0.01\\
56.03	0.01\\
56.04	0.01\\
56.05	0.01\\
56.06	0.01\\
56.07	0.01\\
56.08	0.01\\
56.09	0.01\\
56.1	0.01\\
56.11	0.01\\
56.12	0.01\\
56.13	0.01\\
56.14	0.01\\
56.15	0.01\\
56.16	0.01\\
56.17	0.01\\
56.18	0.01\\
56.19	0.01\\
56.2	0.01\\
56.21	0.01\\
56.22	0.01\\
56.23	0.01\\
56.24	0.01\\
56.25	0.01\\
56.26	0.01\\
56.27	0.01\\
56.28	0.01\\
56.29	0.01\\
56.3	0.01\\
56.31	0.01\\
56.32	0.01\\
56.33	0.01\\
56.34	0.01\\
56.35	0.01\\
56.36	0.01\\
56.37	0.01\\
56.38	0.01\\
56.39	0.01\\
56.4	0.01\\
56.41	0.01\\
56.42	0.01\\
56.43	0.01\\
56.44	0.01\\
56.45	0.01\\
56.46	0.01\\
56.47	0.01\\
56.48	0.01\\
56.49	0.01\\
56.5	0.01\\
56.51	0.01\\
56.52	0.01\\
56.53	0.01\\
56.54	0.01\\
56.55	0.01\\
56.56	0.01\\
56.57	0.01\\
56.58	0.01\\
56.59	0.01\\
56.6	0.01\\
56.61	0.01\\
56.62	0.01\\
56.63	0.01\\
56.64	0.01\\
56.65	0.01\\
56.66	0.01\\
56.67	0.01\\
56.68	0.01\\
56.69	0.01\\
56.7	0.01\\
56.71	0.01\\
56.72	0.01\\
56.73	0.01\\
56.74	0.01\\
56.75	0.01\\
56.76	0.01\\
56.77	0.01\\
56.78	0.01\\
56.79	0.01\\
56.8	0.01\\
56.81	0.01\\
56.82	0.01\\
56.83	0.01\\
56.84	0.01\\
56.85	0.01\\
56.86	0.01\\
56.87	0.01\\
56.88	0.01\\
56.89	0.01\\
56.9	0.01\\
56.91	0.01\\
56.92	0.01\\
56.93	0.01\\
56.94	0.01\\
56.95	0.01\\
56.96	0.01\\
56.97	0.01\\
56.98	0.01\\
56.99	0.01\\
57	0.01\\
57.01	0.01\\
57.02	0.01\\
57.03	0.01\\
57.04	0.01\\
57.05	0.01\\
57.06	0.01\\
57.07	0.01\\
57.08	0.01\\
57.09	0.01\\
57.1	0.01\\
57.11	0.01\\
57.12	0.01\\
57.13	0.01\\
57.14	0.01\\
57.15	0.01\\
57.16	0.01\\
57.17	0.01\\
57.18	0.01\\
57.19	0.01\\
57.2	0.01\\
57.21	0.01\\
57.22	0.01\\
57.23	0.01\\
57.24	0.01\\
57.25	0.01\\
57.26	0.01\\
57.27	0.01\\
57.28	0.01\\
57.29	0.01\\
57.3	0.01\\
57.31	0.01\\
57.32	0.01\\
57.33	0.01\\
57.34	0.01\\
57.35	0.01\\
57.36	0.01\\
57.37	0.01\\
57.38	0.01\\
57.39	0.01\\
57.4	0.01\\
57.41	0.01\\
57.42	0.01\\
57.43	0.01\\
57.44	0.01\\
57.45	0.01\\
57.46	0.01\\
57.47	0.01\\
57.48	0.01\\
57.49	0.01\\
57.5	0.01\\
57.51	0.01\\
57.52	0.01\\
57.53	0.01\\
57.54	0.01\\
57.55	0.01\\
57.56	0.01\\
57.57	0.01\\
57.58	0.01\\
57.59	0.01\\
57.6	0.01\\
57.61	0.01\\
57.62	0.01\\
57.63	0.01\\
57.64	0.01\\
57.65	0.01\\
57.66	0.01\\
57.67	0.01\\
57.68	0.01\\
57.69	0.01\\
57.7	0.01\\
57.71	0.01\\
57.72	0.01\\
57.73	0.01\\
57.74	0.01\\
57.75	0.01\\
57.76	0.01\\
57.77	0.01\\
57.78	0.01\\
57.79	0.01\\
57.8	0.01\\
57.81	0.01\\
57.82	0.01\\
57.83	0.01\\
57.84	0.01\\
57.85	0.01\\
57.86	0.01\\
57.87	0.01\\
57.88	0.01\\
57.89	0.01\\
57.9	0.01\\
57.91	0.01\\
57.92	0.01\\
57.93	0.01\\
57.94	0.01\\
57.95	0.01\\
57.96	0.01\\
57.97	0.01\\
57.98	0.01\\
57.99	0.01\\
58	0.01\\
58.01	0.01\\
58.02	0.01\\
58.03	0.01\\
58.04	0.01\\
58.05	0.01\\
58.06	0.01\\
58.07	0.01\\
58.08	0.01\\
58.09	0.01\\
58.1	0.01\\
58.11	0.01\\
58.12	0.01\\
58.13	0.01\\
58.14	0.01\\
58.15	0.01\\
58.16	0.01\\
58.17	0.01\\
58.18	0.01\\
58.19	0.01\\
58.2	0.01\\
58.21	0.01\\
58.22	0.01\\
58.23	0.01\\
58.24	0.01\\
58.25	0.01\\
58.26	0.01\\
58.27	0.01\\
58.28	0.01\\
58.29	0.01\\
58.3	0.01\\
58.31	0.01\\
58.32	0.01\\
58.33	0.01\\
58.34	0.01\\
58.35	0.01\\
58.36	0.01\\
58.37	0.01\\
58.38	0.01\\
58.39	0.01\\
58.4	0.01\\
58.41	0.01\\
58.42	0.01\\
58.43	0.01\\
58.44	0.01\\
58.45	0.01\\
58.46	0.01\\
58.47	0.01\\
58.48	0.01\\
58.49	0.01\\
58.5	0.01\\
58.51	0.01\\
58.52	0.01\\
58.53	0.01\\
58.54	0.01\\
58.55	0.01\\
58.56	0.01\\
58.57	0.01\\
58.58	0.01\\
58.59	0.01\\
58.6	0.01\\
58.61	0.01\\
58.62	0.01\\
58.63	0.01\\
58.64	0.01\\
58.65	0.01\\
58.66	0.01\\
58.67	0.01\\
58.68	0.01\\
58.69	0.01\\
58.7	0.01\\
58.71	0.01\\
58.72	0.01\\
58.73	0.01\\
58.74	0.01\\
58.75	0.01\\
58.76	0.01\\
58.77	0.01\\
58.78	0.01\\
58.79	0.01\\
58.8	0.01\\
58.81	0.01\\
58.82	0.01\\
58.83	0.01\\
58.84	0.01\\
58.85	0.01\\
58.86	0.01\\
58.87	0.01\\
58.88	0.01\\
58.89	0.01\\
58.9	0.01\\
58.91	0.01\\
58.92	0.01\\
58.93	0.01\\
58.94	0.01\\
58.95	0.01\\
58.96	0.01\\
58.97	0.01\\
58.98	0.01\\
58.99	0.01\\
59	0.01\\
59.01	0.01\\
59.02	0.01\\
59.03	0.01\\
59.04	0.01\\
59.05	0.01\\
59.06	0.01\\
59.07	0.01\\
59.08	0.01\\
59.09	0.01\\
59.1	0.01\\
59.11	0.01\\
59.12	0.01\\
59.13	0.01\\
59.14	0.01\\
59.15	0.01\\
59.16	0.01\\
59.17	0.01\\
59.18	0.01\\
59.19	0.01\\
59.2	0.01\\
59.21	0.01\\
59.22	0.01\\
59.23	0.01\\
59.24	0.01\\
59.25	0.01\\
59.26	0.01\\
59.27	0.01\\
59.28	0.01\\
59.29	0.01\\
59.3	0.01\\
59.31	0.01\\
59.32	0.01\\
59.33	0.01\\
59.34	0.01\\
59.35	0.01\\
59.36	0.01\\
59.37	0.01\\
59.38	0.01\\
59.39	0.01\\
59.4	0.01\\
59.41	0.01\\
59.42	0.01\\
59.43	0.01\\
59.44	0.01\\
59.45	0.01\\
59.46	0.01\\
59.47	0.01\\
59.48	0.01\\
59.49	0.01\\
59.5	0.01\\
59.51	0.01\\
59.52	0.01\\
59.53	0.01\\
59.54	0.01\\
59.55	0.01\\
59.56	0.01\\
59.57	0.01\\
59.58	0.01\\
59.59	0.01\\
59.6	0.01\\
59.61	0.01\\
59.62	0.01\\
59.63	0.01\\
59.64	0.01\\
59.65	0.01\\
59.66	0.01\\
59.67	0.01\\
59.68	0.01\\
59.69	0.01\\
59.7	0.01\\
59.71	0.01\\
59.72	0.01\\
59.73	0.01\\
59.74	0.01\\
59.75	0.01\\
59.76	0.01\\
59.77	0.01\\
59.78	0.01\\
59.79	0.01\\
59.8	0.01\\
59.81	0.01\\
59.82	0.01\\
59.83	0.01\\
59.84	0.01\\
59.85	0.01\\
59.86	0.01\\
59.87	0.01\\
59.88	0.01\\
59.89	0.01\\
59.9	0.01\\
59.91	0.01\\
59.92	0.01\\
59.93	0.01\\
59.94	0.01\\
59.95	0.01\\
59.96	0.01\\
59.97	0.01\\
59.98	0.01\\
59.99	0.01\\
60	0.01\\
60.01	0.01\\
60.02	0.01\\
60.03	0.01\\
60.04	0.01\\
60.05	0.01\\
60.06	0.01\\
60.07	0.01\\
60.08	0.01\\
60.09	0.01\\
60.1	0.01\\
60.11	0.01\\
60.12	0.01\\
60.13	0.01\\
60.14	0.01\\
60.15	0.01\\
60.16	0.01\\
60.17	0.01\\
60.18	0.01\\
60.19	0.01\\
60.2	0.01\\
60.21	0.01\\
60.22	0.01\\
60.23	0.01\\
60.24	0.01\\
60.25	0.01\\
60.26	0.01\\
60.27	0.01\\
60.28	0.01\\
60.29	0.01\\
60.3	0.01\\
60.31	0.01\\
60.32	0.01\\
60.33	0.01\\
60.34	0.01\\
60.35	0.01\\
60.36	0.01\\
60.37	0.01\\
60.38	0.01\\
60.39	0.01\\
60.4	0.01\\
60.41	0.01\\
60.42	0.01\\
60.43	0.01\\
60.44	0.01\\
60.45	0.01\\
60.46	0.01\\
60.47	0.01\\
60.48	0.01\\
60.49	0.01\\
60.5	0.01\\
60.51	0.01\\
60.52	0.01\\
60.53	0.01\\
60.54	0.01\\
60.55	0.01\\
60.56	0.01\\
60.57	0.01\\
60.58	0.01\\
60.59	0.01\\
60.6	0.01\\
60.61	0.01\\
60.62	0.01\\
60.63	0.01\\
60.64	0.01\\
60.65	0.01\\
60.66	0.01\\
60.67	0.01\\
60.68	0.01\\
60.69	0.01\\
60.7	0.01\\
60.71	0.01\\
60.72	0.01\\
60.73	0.01\\
60.74	0.01\\
60.75	0.01\\
60.76	0.01\\
60.77	0.01\\
60.78	0.01\\
60.79	0.01\\
60.8	0.01\\
60.81	0.01\\
60.82	0.01\\
60.83	0.01\\
60.84	0.01\\
60.85	0.01\\
60.86	0.01\\
60.87	0.01\\
60.88	0.01\\
60.89	0.01\\
60.9	0.01\\
60.91	0.01\\
60.92	0.01\\
60.93	0.01\\
60.94	0.01\\
60.95	0.01\\
60.96	0.01\\
60.97	0.01\\
60.98	0.01\\
60.99	0.01\\
61	0.01\\
61.01	0.01\\
61.02	0.01\\
61.03	0.01\\
61.04	0.01\\
61.05	0.01\\
61.06	0.01\\
61.07	0.01\\
61.08	0.01\\
61.09	0.01\\
61.1	0.01\\
61.11	0.01\\
61.12	0.01\\
61.13	0.01\\
61.14	0.01\\
61.15	0.01\\
61.16	0.01\\
61.17	0.01\\
61.18	0.01\\
61.19	0.01\\
61.2	0.01\\
61.21	0.01\\
61.22	0.01\\
61.23	0.01\\
61.24	0.01\\
61.25	0.01\\
61.26	0.01\\
61.27	0.01\\
61.28	0.01\\
61.29	0.01\\
61.3	0.01\\
61.31	0.01\\
61.32	0.01\\
61.33	0.01\\
61.34	0.01\\
61.35	0.01\\
61.36	0.01\\
61.37	0.01\\
61.38	0.01\\
61.39	0.01\\
61.4	0.01\\
61.41	0.01\\
61.42	0.01\\
61.43	0.01\\
61.44	0.01\\
61.45	0.01\\
61.46	0.01\\
61.47	0.01\\
61.48	0.01\\
61.49	0.01\\
61.5	0.01\\
61.51	0.01\\
61.52	0.01\\
61.53	0.01\\
61.54	0.01\\
61.55	0.01\\
61.56	0.01\\
61.57	0.01\\
61.58	0.01\\
61.59	0.01\\
61.6	0.01\\
61.61	0.01\\
61.62	0.01\\
61.63	0.01\\
61.64	0.01\\
61.65	0.01\\
61.66	0.01\\
61.67	0.01\\
61.68	0.01\\
61.69	0.01\\
61.7	0.01\\
61.71	0.01\\
61.72	0.01\\
61.73	0.01\\
61.74	0.01\\
61.75	0.01\\
61.76	0.01\\
61.77	0.01\\
61.78	0.01\\
61.79	0.01\\
61.8	0.01\\
61.81	0.01\\
61.82	0.01\\
61.83	0.01\\
61.84	0.01\\
61.85	0.01\\
61.86	0.01\\
61.87	0.01\\
61.88	0.01\\
61.89	0.01\\
61.9	0.01\\
61.91	0.01\\
61.92	0.01\\
61.93	0.01\\
61.94	0.01\\
61.95	0.01\\
61.96	0.01\\
61.97	0.01\\
61.98	0.01\\
61.99	0.01\\
62	0.01\\
62.01	0.01\\
62.02	0.01\\
62.03	0.01\\
62.04	0.01\\
62.05	0.01\\
62.06	0.01\\
62.07	0.01\\
62.08	0.01\\
62.09	0.01\\
62.1	0.01\\
62.11	0.01\\
62.12	0.01\\
62.13	0.01\\
62.14	0.01\\
62.15	0.01\\
62.16	0.01\\
62.17	0.01\\
62.18	0.01\\
62.19	0.01\\
62.2	0.01\\
62.21	0.01\\
62.22	0.01\\
62.23	0.01\\
62.24	0.01\\
62.25	0.01\\
62.26	0.01\\
62.27	0.01\\
62.28	0.01\\
62.29	0.01\\
62.3	0.01\\
62.31	0.01\\
62.32	0.01\\
62.33	0.01\\
62.34	0.01\\
62.35	0.01\\
62.36	0.01\\
62.37	0.01\\
62.38	0.01\\
62.39	0.01\\
62.4	0.01\\
62.41	0.01\\
62.42	0.01\\
62.43	0.01\\
62.44	0.01\\
62.45	0.01\\
62.46	0.01\\
62.47	0.01\\
62.48	0.01\\
62.49	0.01\\
62.5	0.01\\
62.51	0.01\\
62.52	0.01\\
62.53	0.01\\
62.54	0.01\\
62.55	0.01\\
62.56	0.01\\
62.57	0.01\\
62.58	0.01\\
62.59	0.01\\
62.6	0.01\\
62.61	0.01\\
62.62	0.01\\
62.63	0.01\\
62.64	0.01\\
62.65	0.01\\
62.66	0.01\\
62.67	0.01\\
62.68	0.01\\
62.69	0.01\\
62.7	0.01\\
62.71	0.01\\
62.72	0.01\\
62.73	0.01\\
62.74	0.01\\
62.75	0.01\\
62.76	0.01\\
62.77	0.01\\
62.78	0.01\\
62.79	0.01\\
62.8	0.01\\
62.81	0.01\\
62.82	0.01\\
62.83	0.01\\
62.84	0.01\\
62.85	0.01\\
62.86	0.01\\
62.87	0.01\\
62.88	0.01\\
62.89	0.01\\
62.9	0.01\\
62.91	0.01\\
62.92	0.01\\
62.93	0.01\\
62.94	0.01\\
62.95	0.01\\
62.96	0.01\\
62.97	0.01\\
62.98	0.01\\
62.99	0.01\\
63	0.01\\
63.01	0.01\\
63.02	0.01\\
63.03	0.01\\
63.04	0.01\\
63.05	0.01\\
63.06	0.01\\
63.07	0.01\\
63.08	0.01\\
63.09	0.01\\
63.1	0.01\\
63.11	0.01\\
63.12	0.01\\
63.13	0.01\\
63.14	0.01\\
63.15	0.01\\
63.16	0.01\\
63.17	0.01\\
63.18	0.01\\
63.19	0.01\\
63.2	0.01\\
63.21	0.01\\
63.22	0.01\\
63.23	0.01\\
63.24	0.01\\
63.25	0.01\\
63.26	0.01\\
63.27	0.01\\
63.28	0.01\\
63.29	0.01\\
63.3	0.01\\
63.31	0.01\\
63.32	0.01\\
63.33	0.01\\
63.34	0.01\\
63.35	0.01\\
63.36	0.01\\
63.37	0.01\\
63.38	0.01\\
63.39	0.01\\
63.4	0.01\\
63.41	0.01\\
63.42	0.01\\
63.43	0.01\\
63.44	0.01\\
63.45	0.01\\
63.46	0.01\\
63.47	0.01\\
63.48	0.01\\
63.49	0.01\\
63.5	0.01\\
63.51	0.01\\
63.52	0.01\\
63.53	0.01\\
63.54	0.01\\
63.55	0.01\\
63.56	0.01\\
63.57	0.01\\
63.58	0.01\\
63.59	0.01\\
63.6	0.01\\
63.61	0.01\\
63.62	0.01\\
63.63	0.01\\
63.64	0.01\\
63.65	0.01\\
63.66	0.01\\
63.67	0.01\\
63.68	0.01\\
63.69	0.01\\
63.7	0.01\\
63.71	0.01\\
63.72	0.01\\
63.73	0.01\\
63.74	0.01\\
63.75	0.01\\
63.76	0.01\\
63.77	0.01\\
63.78	0.01\\
63.79	0.01\\
63.8	0.01\\
63.81	0.01\\
63.82	0.01\\
63.83	0.01\\
63.84	0.01\\
63.85	0.01\\
63.86	0.01\\
63.87	0.01\\
63.88	0.01\\
63.89	0.01\\
63.9	0.01\\
63.91	0.01\\
63.92	0.01\\
63.93	0.01\\
63.94	0.01\\
63.95	0.01\\
63.96	0.01\\
63.97	0.01\\
63.98	0.01\\
63.99	0.01\\
64	0.01\\
64.01	0.01\\
64.02	0.01\\
64.03	0.01\\
64.04	0.01\\
64.05	0.01\\
64.06	0.01\\
64.07	0.01\\
64.08	0.01\\
64.09	0.01\\
64.1	0.01\\
64.11	0.01\\
64.12	0.01\\
64.13	0.01\\
64.14	0.01\\
64.15	0.01\\
64.16	0.01\\
64.17	0.01\\
64.18	0.01\\
64.19	0.01\\
64.2	0.01\\
64.21	0.01\\
64.22	0.01\\
64.23	0.01\\
64.24	0.01\\
64.25	0.01\\
64.26	0.01\\
64.27	0.01\\
64.28	0.01\\
64.29	0.01\\
64.3	0.01\\
64.31	0.01\\
64.32	0.01\\
64.33	0.01\\
64.34	0.01\\
64.35	0.01\\
64.36	0.01\\
64.37	0.01\\
64.38	0.01\\
64.39	0.01\\
64.4	0.01\\
64.41	0.01\\
64.42	0.01\\
64.43	0.01\\
64.44	0.01\\
64.45	0.01\\
64.46	0.01\\
64.47	0.01\\
64.48	0.01\\
64.49	0.01\\
64.5	0.01\\
64.51	0.01\\
64.52	0.01\\
64.53	0.01\\
64.54	0.01\\
64.55	0.01\\
64.56	0.01\\
64.57	0.01\\
64.58	0.01\\
64.59	0.01\\
64.6	0.01\\
64.61	0.01\\
64.62	0.01\\
64.63	0.01\\
64.64	0.01\\
64.65	0.01\\
64.66	0.01\\
64.67	0.01\\
64.68	0.01\\
64.69	0.01\\
64.7	0.01\\
64.71	0.01\\
64.72	0.01\\
64.73	0.01\\
64.74	0.01\\
64.75	0.01\\
64.76	0.01\\
64.77	0.01\\
64.78	0.01\\
64.79	0.01\\
64.8	0.01\\
64.81	0.01\\
64.82	0.01\\
64.83	0.01\\
64.84	0.01\\
64.85	0.01\\
64.86	0.01\\
64.87	0.01\\
64.88	0.01\\
64.89	0.01\\
64.9	0.01\\
64.91	0.01\\
64.92	0.01\\
64.93	0.01\\
64.94	0.01\\
64.95	0.01\\
64.96	0.01\\
64.97	0.01\\
64.98	0.01\\
64.99	0.01\\
65	0.01\\
65.01	0.01\\
65.02	0.01\\
65.03	0.01\\
65.04	0.01\\
65.05	0.01\\
65.06	0.01\\
65.07	0.01\\
65.08	0.01\\
65.09	0.01\\
65.1	0.01\\
65.11	0.01\\
65.12	0.01\\
65.13	0.01\\
65.14	0.01\\
65.15	0.01\\
65.16	0.01\\
65.17	0.01\\
65.18	0.01\\
65.19	0.01\\
65.2	0.01\\
65.21	0.01\\
65.22	0.01\\
65.23	0.01\\
65.24	0.01\\
65.25	0.01\\
65.26	0.01\\
65.27	0.01\\
65.28	0.01\\
65.29	0.01\\
65.3	0.01\\
65.31	0.01\\
65.32	0.01\\
65.33	0.01\\
65.34	0.01\\
65.35	0.01\\
65.36	0.01\\
65.37	0.01\\
65.38	0.01\\
65.39	0.01\\
65.4	0.01\\
65.41	0.01\\
65.42	0.01\\
65.43	0.01\\
65.44	0.01\\
65.45	0.01\\
65.46	0.01\\
65.47	0.01\\
65.48	0.01\\
65.49	0.01\\
65.5	0.01\\
65.51	0.01\\
65.52	0.01\\
65.53	0.01\\
65.54	0.01\\
65.55	0.01\\
65.56	0.01\\
65.57	0.01\\
65.58	0.01\\
65.59	0.01\\
65.6	0.01\\
65.61	0.01\\
65.62	0.01\\
65.63	0.01\\
65.64	0.01\\
65.65	0.01\\
65.66	0.01\\
65.67	0.01\\
65.68	0.01\\
65.69	0.01\\
65.7	0.01\\
65.71	0.01\\
65.72	0.01\\
65.73	0.01\\
65.74	0.01\\
65.75	0.01\\
65.76	0.01\\
65.77	0.01\\
65.78	0.01\\
65.79	0.01\\
65.8	0.01\\
65.81	0.01\\
65.82	0.01\\
65.83	0.01\\
65.84	0.01\\
65.85	0.01\\
65.86	0.01\\
65.87	0.01\\
65.88	0.01\\
65.89	0.01\\
65.9	0.01\\
65.91	0.01\\
65.92	0.01\\
65.93	0.01\\
65.94	0.01\\
65.95	0.01\\
65.96	0.01\\
65.97	0.01\\
65.98	0.01\\
65.99	0.01\\
66	0.01\\
66.01	0.01\\
66.02	0.01\\
66.03	0.01\\
66.04	0.01\\
66.05	0.01\\
66.06	0.01\\
66.07	0.01\\
66.08	0.01\\
66.09	0.01\\
66.1	0.01\\
66.11	0.01\\
66.12	0.01\\
66.13	0.01\\
66.14	0.01\\
66.15	0.01\\
66.16	0.01\\
66.17	0.01\\
66.18	0.01\\
66.19	0.01\\
66.2	0.01\\
66.21	0.01\\
66.22	0.01\\
66.23	0.01\\
66.24	0.01\\
66.25	0.01\\
66.26	0.01\\
66.27	0.01\\
66.28	0.01\\
66.29	0.01\\
66.3	0.01\\
66.31	0.01\\
66.32	0.01\\
66.33	0.01\\
66.34	0.01\\
66.35	0.01\\
66.36	0.01\\
66.37	0.01\\
66.38	0.01\\
66.39	0.01\\
66.4	0.01\\
66.41	0.01\\
66.42	0.01\\
66.43	0.01\\
66.44	0.01\\
66.45	0.01\\
66.46	0.01\\
66.47	0.01\\
66.48	0.01\\
66.49	0.01\\
66.5	0.01\\
66.51	0.01\\
66.52	0.01\\
66.53	0.01\\
66.54	0.01\\
66.55	0.01\\
66.56	0.01\\
66.57	0.01\\
66.58	0.01\\
66.59	0.01\\
66.6	0.01\\
66.61	0.01\\
66.62	0.01\\
66.63	0.01\\
66.64	0.01\\
66.65	0.01\\
66.66	0.01\\
66.67	0.01\\
66.68	0.01\\
66.69	0.01\\
66.7	0.01\\
66.71	0.01\\
66.72	0.01\\
66.73	0.01\\
66.74	0.01\\
66.75	0.01\\
66.76	0.01\\
66.77	0.01\\
66.78	0.01\\
66.79	0.01\\
66.8	0.01\\
66.81	0.01\\
66.82	0.01\\
66.83	0.01\\
66.84	0.01\\
66.85	0.01\\
66.86	0.01\\
66.87	0.01\\
66.88	0.01\\
66.89	0.01\\
66.9	0.01\\
66.91	0.01\\
66.92	0.01\\
66.93	0.01\\
66.94	0.01\\
66.95	0.01\\
66.96	0.01\\
66.97	0.01\\
66.98	0.01\\
66.99	0.01\\
67	0.01\\
67.01	0.01\\
67.02	0.01\\
67.03	0.01\\
67.04	0.01\\
67.05	0.01\\
67.06	0.01\\
67.07	0.01\\
67.08	0.01\\
67.09	0.01\\
67.1	0.01\\
67.11	0.01\\
67.12	0.01\\
67.13	0.01\\
67.14	0.01\\
67.15	0.01\\
67.16	0.01\\
67.17	0.01\\
67.18	0.01\\
67.19	0.01\\
67.2	0.01\\
67.21	0.01\\
67.22	0.01\\
67.23	0.01\\
67.24	0.01\\
67.25	0.01\\
67.26	0.01\\
67.27	0.01\\
67.28	0.01\\
67.29	0.01\\
67.3	0.01\\
67.31	0.01\\
67.32	0.01\\
67.33	0.01\\
67.34	0.01\\
67.35	0.01\\
67.36	0.01\\
67.37	0.01\\
67.38	0.01\\
67.39	0.01\\
67.4	0.01\\
67.41	0.01\\
67.42	0.01\\
67.43	0.01\\
67.44	0.01\\
67.45	0.01\\
67.46	0.01\\
67.47	0.01\\
67.48	0.01\\
67.49	0.01\\
67.5	0.01\\
67.51	0.01\\
67.52	0.01\\
67.53	0.01\\
67.54	0.01\\
67.55	0.01\\
67.56	0.01\\
67.57	0.01\\
67.58	0.01\\
67.59	0.01\\
67.6	0.01\\
67.61	0.01\\
67.62	0.01\\
67.63	0.01\\
67.64	0.01\\
67.65	0.01\\
67.66	0.01\\
67.67	0.01\\
67.68	0.01\\
67.69	0.01\\
67.7	0.01\\
67.71	0.01\\
67.72	0.01\\
67.73	0.01\\
67.74	0.01\\
67.75	0.01\\
67.76	0.01\\
67.77	0.01\\
67.78	0.01\\
67.79	0.01\\
67.8	0.01\\
67.81	0.01\\
67.82	0.01\\
67.83	0.01\\
67.84	0.01\\
67.85	0.01\\
67.86	0.01\\
67.87	0.01\\
67.88	0.01\\
67.89	0.01\\
67.9	0.01\\
67.91	0.01\\
67.92	0.01\\
67.93	0.01\\
67.94	0.01\\
67.95	0.01\\
67.96	0.01\\
67.97	0.01\\
67.98	0.01\\
67.99	0.01\\
68	0.01\\
68.01	0.01\\
68.02	0.01\\
68.03	0.01\\
68.04	0.01\\
68.05	0.01\\
68.06	0.01\\
68.07	0.01\\
68.08	0.01\\
68.09	0.01\\
68.1	0.01\\
68.11	0.01\\
68.12	0.01\\
68.13	0.01\\
68.14	0.01\\
68.15	0.01\\
68.16	0.01\\
68.17	0.01\\
68.18	0.01\\
68.19	0.01\\
68.2	0.01\\
68.21	0.01\\
68.22	0.01\\
68.23	0.01\\
68.24	0.01\\
68.25	0.01\\
68.26	0.01\\
68.27	0.01\\
68.28	0.01\\
68.29	0.01\\
68.3	0.01\\
68.31	0.01\\
68.32	0.01\\
68.33	0.01\\
68.34	0.01\\
68.35	0.01\\
68.36	0.01\\
68.37	0.01\\
68.38	0.01\\
68.39	0.01\\
68.4	0.01\\
68.41	0.01\\
68.42	0.01\\
68.43	0.01\\
68.44	0.01\\
68.45	0.01\\
68.46	0.01\\
68.47	0.01\\
68.48	0.01\\
68.49	0.01\\
68.5	0.01\\
68.51	0.01\\
68.52	0.01\\
68.53	0.01\\
68.54	0.01\\
68.55	0.01\\
68.56	0.01\\
68.57	0.01\\
68.58	0.01\\
68.59	0.01\\
68.6	0.01\\
68.61	0.01\\
68.62	0.01\\
68.63	0.01\\
68.64	0.01\\
68.65	0.01\\
68.66	0.01\\
68.67	0.01\\
68.68	0.01\\
68.69	0.01\\
68.7	0.01\\
68.71	0.01\\
68.72	0.01\\
68.73	0.01\\
68.74	0.01\\
68.75	0.01\\
68.76	0.01\\
68.77	0.01\\
68.78	0.01\\
68.79	0.01\\
68.8	0.01\\
68.81	0.01\\
68.82	0.01\\
68.83	0.01\\
68.84	0.01\\
68.85	0.01\\
68.86	0.01\\
68.87	0.01\\
68.88	0.01\\
68.89	0.01\\
68.9	0.01\\
68.91	0.01\\
68.92	0.01\\
68.93	0.01\\
68.94	0.01\\
68.95	0.01\\
68.96	0.01\\
68.97	0.01\\
68.98	0.01\\
68.99	0.01\\
69	0.01\\
69.01	0.01\\
69.02	0.01\\
69.03	0.01\\
69.04	0.01\\
69.05	0.01\\
69.06	0.01\\
69.07	0.01\\
69.08	0.01\\
69.09	0.01\\
69.1	0.01\\
69.11	0.01\\
69.12	0.01\\
69.13	0.01\\
69.14	0.01\\
69.15	0.01\\
69.16	0.01\\
69.17	0.01\\
69.18	0.01\\
69.19	0.01\\
69.2	0.01\\
69.21	0.01\\
69.22	0.01\\
69.23	0.01\\
69.24	0.01\\
69.25	0.01\\
69.26	0.01\\
69.27	0.01\\
69.28	0.01\\
69.29	0.01\\
69.3	0.01\\
69.31	0.01\\
69.32	0.01\\
69.33	0.01\\
69.34	0.01\\
69.35	0.01\\
69.36	0.01\\
69.37	0.01\\
69.38	0.01\\
69.39	0.01\\
69.4	0.01\\
69.41	0.01\\
69.42	0.01\\
69.43	0.01\\
69.44	0.01\\
69.45	0.01\\
69.46	0.01\\
69.47	0.01\\
69.48	0.01\\
69.49	0.01\\
69.5	0.01\\
69.51	0.01\\
69.52	0.01\\
69.53	0.01\\
69.54	0.01\\
69.55	0.01\\
69.56	0.01\\
69.57	0.01\\
69.58	0.01\\
69.59	0.01\\
69.6	0.01\\
69.61	0.01\\
69.62	0.01\\
69.63	0.01\\
69.64	0.01\\
69.65	0.01\\
69.66	0.01\\
69.67	0.01\\
69.68	0.01\\
69.69	0.01\\
69.7	0.01\\
69.71	0.01\\
69.72	0.01\\
69.73	0.01\\
69.74	0.01\\
69.75	0.01\\
69.76	0.01\\
69.77	0.01\\
69.78	0.01\\
69.79	0.01\\
69.8	0.01\\
69.81	0.01\\
69.82	0.01\\
69.83	0.01\\
69.84	0.01\\
69.85	0.01\\
69.86	0.01\\
69.87	0.01\\
69.88	0.01\\
69.89	0.01\\
69.9	0.01\\
69.91	0.01\\
69.92	0.01\\
69.93	0.01\\
69.94	0.01\\
69.95	0.01\\
69.96	0.01\\
69.97	0.01\\
69.98	0.01\\
69.99	0.01\\
70	0.01\\
70.01	0.01\\
70.02	0.01\\
70.03	0.01\\
70.04	0.01\\
70.05	0.01\\
70.06	0.01\\
70.07	0.01\\
70.08	0.01\\
70.09	0.01\\
70.1	0.01\\
70.11	0.01\\
70.12	0.01\\
70.13	0.01\\
70.14	0.01\\
70.15	0.01\\
70.16	0.01\\
70.17	0.01\\
70.18	0.01\\
70.19	0.01\\
70.2	0.01\\
70.21	0.01\\
70.22	0.01\\
70.23	0.01\\
70.24	0.01\\
70.25	0.01\\
70.26	0.01\\
70.27	0.01\\
70.28	0.01\\
70.29	0.01\\
70.3	0.01\\
70.31	0.01\\
70.32	0.01\\
70.33	0.01\\
70.34	0.01\\
70.35	0.01\\
70.36	0.01\\
70.37	0.01\\
70.38	0.01\\
70.39	0.01\\
70.4	0.01\\
70.41	0.01\\
70.42	0.01\\
70.43	0.01\\
70.44	0.01\\
70.45	0.01\\
70.46	0.01\\
70.47	0.01\\
70.48	0.01\\
70.49	0.01\\
70.5	0.01\\
70.51	0.01\\
70.52	0.01\\
70.53	0.01\\
70.54	0.01\\
70.55	0.01\\
70.56	0.01\\
70.57	0.01\\
70.58	0.01\\
70.59	0.01\\
70.6	0.01\\
70.61	0.01\\
70.62	0.01\\
70.63	0.01\\
70.64	0.01\\
70.65	0.01\\
70.66	0.01\\
70.67	0.01\\
70.68	0.01\\
70.69	0.01\\
70.7	0.01\\
70.71	0.01\\
70.72	0.01\\
70.73	0.01\\
70.74	0.01\\
70.75	0.01\\
70.76	0.01\\
70.77	0.01\\
70.78	0.01\\
70.79	0.01\\
70.8	0.01\\
70.81	0.01\\
70.82	0.01\\
70.83	0.01\\
70.84	0.01\\
70.85	0.01\\
70.86	0.01\\
70.87	0.01\\
70.88	0.01\\
70.89	0.01\\
70.9	0.01\\
70.91	0.01\\
70.92	0.01\\
70.93	0.01\\
70.94	0.01\\
70.95	0.01\\
70.96	0.01\\
70.97	0.01\\
70.98	0.01\\
70.99	0.01\\
71	0.01\\
71.01	0.01\\
71.02	0.01\\
71.03	0.01\\
71.04	0.01\\
71.05	0.01\\
71.06	0.01\\
71.07	0.01\\
71.08	0.01\\
71.09	0.01\\
71.1	0.01\\
71.11	0.01\\
71.12	0.01\\
71.13	0.01\\
71.14	0.01\\
71.15	0.01\\
71.16	0.01\\
71.17	0.01\\
71.18	0.01\\
71.19	0.01\\
71.2	0.01\\
71.21	0.01\\
71.22	0.01\\
71.23	0.01\\
71.24	0.01\\
71.25	0.01\\
71.26	0.01\\
71.27	0.01\\
71.28	0.01\\
71.29	0.01\\
71.3	0.01\\
71.31	0.01\\
71.32	0.01\\
71.33	0.01\\
71.34	0.01\\
71.35	0.01\\
71.36	0.01\\
71.37	0.01\\
71.38	0.01\\
71.39	0.01\\
71.4	0.01\\
71.41	0.01\\
71.42	0.01\\
71.43	0.01\\
71.44	0.01\\
71.45	0.01\\
71.46	0.01\\
71.47	0.01\\
71.48	0.01\\
71.49	0.01\\
71.5	0.01\\
71.51	0.01\\
71.52	0.01\\
71.53	0.01\\
71.54	0.01\\
71.55	0.01\\
71.56	0.01\\
71.57	0.01\\
71.58	0.01\\
71.59	0.01\\
71.6	0.01\\
71.61	0.01\\
71.62	0.01\\
71.63	0.01\\
71.64	0.01\\
71.65	0.01\\
71.66	0.01\\
71.67	0.01\\
71.68	0.01\\
71.69	0.01\\
71.7	0.01\\
71.71	0.01\\
71.72	0.01\\
71.73	0.01\\
71.74	0.01\\
71.75	0.01\\
71.76	0.01\\
71.77	0.01\\
71.78	0.01\\
71.79	0.01\\
71.8	0.01\\
71.81	0.01\\
71.82	0.01\\
71.83	0.01\\
71.84	0.01\\
71.85	0.01\\
71.86	0.01\\
71.87	0.01\\
71.88	0.01\\
71.89	0.01\\
71.9	0.01\\
71.91	0.01\\
71.92	0.01\\
71.93	0.01\\
71.94	0.01\\
71.95	0.01\\
71.96	0.01\\
71.97	0.01\\
71.98	0.01\\
71.99	0.01\\
72	0.01\\
72.01	0.01\\
72.02	0.01\\
72.03	0.01\\
72.04	0.01\\
72.05	0.01\\
72.06	0.01\\
72.07	0.01\\
72.08	0.01\\
72.09	0.01\\
72.1	0.01\\
72.11	0.01\\
72.12	0.01\\
72.13	0.01\\
72.14	0.01\\
72.15	0.01\\
72.16	0.01\\
72.17	0.01\\
72.18	0.01\\
72.19	0.01\\
72.2	0.01\\
72.21	0.01\\
72.22	0.01\\
72.23	0.01\\
72.24	0.01\\
72.25	0.01\\
72.26	0.01\\
72.27	0.01\\
72.28	0.01\\
72.29	0.01\\
72.3	0.01\\
72.31	0.01\\
72.32	0.01\\
72.33	0.01\\
72.34	0.01\\
72.35	0.01\\
72.36	0.01\\
72.37	0.01\\
72.38	0.01\\
72.39	0.01\\
72.4	0.01\\
72.41	0.01\\
72.42	0.01\\
72.43	0.01\\
72.44	0.01\\
72.45	0.01\\
72.46	0.01\\
72.47	0.01\\
72.48	0.01\\
72.49	0.01\\
72.5	0.01\\
72.51	0.01\\
72.52	0.01\\
72.53	0.01\\
72.54	0.01\\
72.55	0.01\\
72.56	0.01\\
72.57	0.01\\
72.58	0.01\\
72.59	0.01\\
72.6	0.01\\
72.61	0.01\\
72.62	0.01\\
72.63	0.01\\
72.64	0.01\\
72.65	0.01\\
72.66	0.01\\
72.67	0.01\\
72.68	0.01\\
72.69	0.01\\
72.7	0.01\\
72.71	0.01\\
72.72	0.01\\
72.73	0.01\\
72.74	0.01\\
72.75	0.01\\
72.76	0.01\\
72.77	0.01\\
72.78	0.01\\
72.79	0.01\\
72.8	0.01\\
72.81	0.01\\
72.82	0.01\\
72.83	0.01\\
72.84	0.01\\
72.85	0.01\\
72.86	0.01\\
72.87	0.01\\
72.88	0.01\\
72.89	0.01\\
72.9	0.01\\
72.91	0.01\\
72.92	0.01\\
72.93	0.01\\
72.94	0.01\\
72.95	0.01\\
72.96	0.01\\
72.97	0.01\\
72.98	0.01\\
72.99	0.01\\
73	0.01\\
73.01	0.01\\
73.02	0.01\\
73.03	0.01\\
73.04	0.01\\
73.05	0.01\\
73.06	0.01\\
73.07	0.01\\
73.08	0.01\\
73.09	0.01\\
73.1	0.01\\
73.11	0.01\\
73.12	0.01\\
73.13	0.01\\
73.14	0.01\\
73.15	0.01\\
73.16	0.01\\
73.17	0.01\\
73.18	0.01\\
73.19	0.01\\
73.2	0.01\\
73.21	0.01\\
73.22	0.01\\
73.23	0.01\\
73.24	0.01\\
73.25	0.01\\
73.26	0.01\\
73.27	0.01\\
73.28	0.01\\
73.29	0.01\\
73.3	0.01\\
73.31	0.01\\
73.32	0.01\\
73.33	0.01\\
73.34	0.01\\
73.35	0.01\\
73.36	0.01\\
73.37	0.01\\
73.38	0.01\\
73.39	0.01\\
73.4	0.01\\
73.41	0.01\\
73.42	0.01\\
73.43	0.01\\
73.44	0.01\\
73.45	0.01\\
73.46	0.01\\
73.47	0.01\\
73.48	0.01\\
73.49	0.01\\
73.5	0.01\\
73.51	0.01\\
73.52	0.01\\
73.53	0.01\\
73.54	0.01\\
73.55	0.01\\
73.56	0.01\\
73.57	0.01\\
73.58	0.01\\
73.59	0.01\\
73.6	0.01\\
73.61	0.01\\
73.62	0.01\\
73.63	0.01\\
73.64	0.01\\
73.65	0.01\\
73.66	0.01\\
73.67	0.01\\
73.68	0.01\\
73.69	0.01\\
73.7	0.01\\
73.71	0.01\\
73.72	0.01\\
73.73	0.01\\
73.74	0.01\\
73.75	0.01\\
73.76	0.01\\
73.77	0.01\\
73.78	0.01\\
73.79	0.01\\
73.8	0.01\\
73.81	0.01\\
73.82	0.01\\
73.83	0.01\\
73.84	0.01\\
73.85	0.01\\
73.86	0.01\\
73.87	0.01\\
73.88	0.01\\
73.89	0.01\\
73.9	0.01\\
73.91	0.01\\
73.92	0.01\\
73.93	0.01\\
73.94	0.01\\
73.95	0.01\\
73.96	0.01\\
73.97	0.01\\
73.98	0.01\\
73.99	0.01\\
74	0.01\\
74.01	0.01\\
74.02	0.01\\
74.03	0.01\\
74.04	0.01\\
74.05	0.01\\
74.06	0.01\\
74.07	0.01\\
74.08	0.01\\
74.09	0.01\\
74.1	0.01\\
74.11	0.01\\
74.12	0.01\\
74.13	0.01\\
74.14	0.01\\
74.15	0.01\\
74.16	0.01\\
74.17	0.01\\
74.18	0.01\\
74.19	0.01\\
74.2	0.01\\
74.21	0.01\\
74.22	0.01\\
74.23	0.01\\
74.24	0.01\\
74.25	0.01\\
74.26	0.01\\
74.27	0.01\\
74.28	0.01\\
74.29	0.01\\
74.3	0.01\\
74.31	0.01\\
74.32	0.01\\
74.33	0.01\\
74.34	0.01\\
74.35	0.01\\
74.36	0.01\\
74.37	0.01\\
74.38	0.01\\
74.39	0.01\\
74.4	0.01\\
74.41	0.01\\
74.42	0.01\\
74.43	0.01\\
74.44	0.01\\
74.45	0.01\\
74.46	0.01\\
74.47	0.01\\
74.48	0.01\\
74.49	0.01\\
74.5	0.01\\
74.51	0.01\\
74.52	0.01\\
74.53	0.01\\
74.54	0.01\\
74.55	0.01\\
74.56	0.01\\
74.57	0.01\\
74.58	0.01\\
74.59	0.01\\
74.6	0.01\\
74.61	0.01\\
74.62	0.01\\
74.63	0.01\\
74.64	0.01\\
74.65	0.01\\
74.66	0.01\\
74.67	0.01\\
74.68	0.01\\
74.69	0.01\\
74.7	0.01\\
74.71	0.01\\
74.72	0.01\\
74.73	0.01\\
74.74	0.01\\
74.75	0.01\\
74.76	0.01\\
74.77	0.01\\
74.78	0.01\\
74.79	0.01\\
74.8	0.01\\
74.81	0.01\\
74.82	0.01\\
74.83	0.01\\
74.84	0.01\\
74.85	0.01\\
74.86	0.01\\
74.87	0.01\\
74.88	0.01\\
74.89	0.01\\
74.9	0.01\\
74.91	0.01\\
74.92	0.01\\
74.93	0.01\\
74.94	0.01\\
74.95	0.01\\
74.96	0.01\\
74.97	0.01\\
74.98	0.01\\
74.99	0.01\\
75	0.01\\
75.01	0.01\\
75.02	0.01\\
75.03	0.01\\
75.04	0.01\\
75.05	0.01\\
75.06	0.01\\
75.07	0.01\\
75.08	0.01\\
75.09	0.01\\
75.1	0.01\\
75.11	0.01\\
75.12	0.01\\
75.13	0.01\\
75.14	0.01\\
75.15	0.01\\
75.16	0.01\\
75.17	0.01\\
75.18	0.01\\
75.19	0.01\\
75.2	0.01\\
75.21	0.01\\
75.22	0.01\\
75.23	0.01\\
75.24	0.01\\
75.25	0.01\\
75.26	0.01\\
75.27	0.01\\
75.28	0.01\\
75.29	0.01\\
75.3	0.01\\
75.31	0.01\\
75.32	0.01\\
75.33	0.01\\
75.34	0.01\\
75.35	0.01\\
75.36	0.01\\
75.37	0.01\\
75.38	0.01\\
75.39	0.01\\
75.4	0.01\\
75.41	0.01\\
75.42	0.01\\
75.43	0.01\\
75.44	0.01\\
75.45	0.01\\
75.46	0.01\\
75.47	0.01\\
75.48	0.01\\
75.49	0.01\\
75.5	0.01\\
75.51	0.01\\
75.52	0.01\\
75.53	0.01\\
75.54	0.01\\
75.55	0.01\\
75.56	0.01\\
75.57	0.01\\
75.58	0.01\\
75.59	0.01\\
75.6	0.01\\
75.61	0.01\\
75.62	0.01\\
75.63	0.01\\
75.64	0.01\\
75.65	0.01\\
75.66	0.01\\
75.67	0.01\\
75.68	0.01\\
75.69	0.01\\
75.7	0.01\\
75.71	0.01\\
75.72	0.01\\
75.73	0.01\\
75.74	0.01\\
75.75	0.01\\
75.76	0.01\\
75.77	0.01\\
75.78	0.01\\
75.79	0.01\\
75.8	0.01\\
75.81	0.01\\
75.82	0.01\\
75.83	0.01\\
75.84	0.01\\
75.85	0.01\\
75.86	0.01\\
75.87	0.01\\
75.88	0.01\\
75.89	0.01\\
75.9	0.01\\
75.91	0.01\\
75.92	0.01\\
75.93	0.01\\
75.94	0.01\\
75.95	0.01\\
75.96	0.01\\
75.97	0.01\\
75.98	0.01\\
75.99	0.01\\
76	0.01\\
76.01	0.01\\
76.02	0.01\\
76.03	0.01\\
76.04	0.01\\
76.05	0.01\\
76.06	0.01\\
76.07	0.01\\
76.08	0.01\\
76.09	0.01\\
76.1	0.01\\
76.11	0.01\\
76.12	0.01\\
76.13	0.01\\
76.14	0.01\\
76.15	0.01\\
76.16	0.01\\
76.17	0.01\\
76.18	0.01\\
76.19	0.01\\
76.2	0.01\\
76.21	0.01\\
76.22	0.01\\
76.23	0.01\\
76.24	0.01\\
76.25	0.01\\
76.26	0.01\\
76.27	0.01\\
76.28	0.01\\
76.29	0.01\\
76.3	0.01\\
76.31	0.01\\
76.32	0.01\\
76.33	0.01\\
76.34	0.01\\
76.35	0.01\\
76.36	0.01\\
76.37	0.01\\
76.38	0.01\\
76.39	0.01\\
76.4	0.01\\
76.41	0.01\\
76.42	0.01\\
76.43	0.01\\
76.44	0.01\\
76.45	0.01\\
76.46	0.01\\
76.47	0.01\\
76.48	0.01\\
76.49	0.01\\
76.5	0.01\\
76.51	0.01\\
76.52	0.01\\
76.53	0.01\\
76.54	0.01\\
76.55	0.01\\
76.56	0.01\\
76.57	0.01\\
76.58	0.01\\
76.59	0.01\\
76.6	0.01\\
76.61	0.01\\
76.62	0.01\\
76.63	0.01\\
76.64	0.01\\
76.65	0.01\\
76.66	0.01\\
76.67	0.01\\
76.68	0.01\\
76.69	0.01\\
76.7	0.01\\
76.71	0.01\\
76.72	0.01\\
76.73	0.01\\
76.74	0.01\\
76.75	0.01\\
76.76	0.01\\
76.77	0.01\\
76.78	0.01\\
76.79	0.01\\
76.8	0.01\\
76.81	0.01\\
76.82	0.01\\
76.83	0.01\\
76.84	0.01\\
76.85	0.01\\
76.86	0.01\\
76.87	0.01\\
76.88	0.01\\
76.89	0.01\\
76.9	0.01\\
76.91	0.01\\
76.92	0.01\\
76.93	0.01\\
76.94	0.01\\
76.95	0.01\\
76.96	0.01\\
76.97	0.01\\
76.98	0.01\\
76.99	0.01\\
77	0.01\\
77.01	0.01\\
77.02	0.01\\
77.03	0.01\\
77.04	0.01\\
77.05	0.01\\
77.06	0.01\\
77.07	0.01\\
77.08	0.01\\
77.09	0.01\\
77.1	0.01\\
77.11	0.01\\
77.12	0.01\\
77.13	0.01\\
77.14	0.01\\
77.15	0.01\\
77.16	0.01\\
77.17	0.01\\
77.18	0.01\\
77.19	0.01\\
77.2	0.01\\
77.21	0.01\\
77.22	0.01\\
77.23	0.01\\
77.24	0.01\\
77.25	0.01\\
77.26	0.01\\
77.27	0.01\\
77.28	0.01\\
77.29	0.01\\
77.3	0.01\\
77.31	0.01\\
77.32	0.01\\
77.33	0.01\\
77.34	0.01\\
77.35	0.01\\
77.36	0.01\\
77.37	0.01\\
77.38	0.01\\
77.39	0.01\\
77.4	0.01\\
77.41	0.01\\
77.42	0.01\\
77.43	0.01\\
77.44	0.01\\
77.45	0.01\\
77.46	0.01\\
77.47	0.01\\
77.48	0.01\\
77.49	0.01\\
77.5	0.01\\
77.51	0.01\\
77.52	0.01\\
77.53	0.01\\
77.54	0.01\\
77.55	0.01\\
77.56	0.01\\
77.57	0.01\\
77.58	0.01\\
77.59	0.01\\
77.6	0.01\\
77.61	0.01\\
77.62	0.01\\
77.63	0.01\\
77.64	0.01\\
77.65	0.01\\
77.66	0.01\\
77.67	0.01\\
77.68	0.01\\
77.69	0.01\\
77.7	0.01\\
77.71	0.01\\
77.72	0.01\\
77.73	0.01\\
77.74	0.01\\
77.75	0.01\\
77.76	0.01\\
77.77	0.01\\
77.78	0.01\\
77.79	0.01\\
77.8	0.01\\
77.81	0.01\\
77.82	0.01\\
77.83	0.01\\
77.84	0.01\\
77.85	0.01\\
77.86	0.01\\
77.87	0.01\\
77.88	0.01\\
77.89	0.01\\
77.9	0.01\\
77.91	0.01\\
77.92	0.01\\
77.93	0.01\\
77.94	0.01\\
77.95	0.01\\
77.96	0.01\\
77.97	0.01\\
77.98	0.01\\
77.99	0.01\\
78	0.01\\
78.01	0.01\\
78.02	0.01\\
78.03	0.01\\
78.04	0.01\\
78.05	0.01\\
78.06	0.01\\
78.07	0.01\\
78.08	0.01\\
78.09	0.01\\
78.1	0.01\\
78.11	0.01\\
78.12	0.01\\
78.13	0.01\\
78.14	0.01\\
78.15	0.01\\
78.16	0.01\\
78.17	0.01\\
78.18	0.01\\
78.19	0.01\\
78.2	0.01\\
78.21	0.01\\
78.22	0.01\\
78.23	0.01\\
78.24	0.01\\
78.25	0.01\\
78.26	0.01\\
78.27	0.01\\
78.28	0.01\\
78.29	0.01\\
78.3	0.01\\
78.31	0.01\\
78.32	0.01\\
78.33	0.01\\
78.34	0.01\\
78.35	0.01\\
78.36	0.01\\
78.37	0.01\\
78.38	0.01\\
78.39	0.01\\
78.4	0.01\\
78.41	0.01\\
78.42	0.01\\
78.43	0.01\\
78.44	0.01\\
78.45	0.01\\
78.46	0.01\\
78.47	0.01\\
78.48	0.01\\
78.49	0.01\\
78.5	0.01\\
78.51	0.01\\
78.52	0.01\\
78.53	0.01\\
78.54	0.01\\
78.55	0.01\\
78.56	0.01\\
78.57	0.01\\
78.58	0.01\\
78.59	0.01\\
78.6	0.01\\
78.61	0.01\\
78.62	0.01\\
78.63	0.01\\
78.64	0.01\\
78.65	0.01\\
78.66	0.01\\
78.67	0.01\\
78.68	0.01\\
78.69	0.01\\
78.7	0.01\\
78.71	0.01\\
78.72	0.01\\
78.73	0.01\\
78.74	0.01\\
78.75	0.01\\
78.76	0.01\\
78.77	0.01\\
78.78	0.01\\
78.79	0.01\\
78.8	0.01\\
78.81	0.01\\
78.82	0.01\\
78.83	0.01\\
78.84	0.01\\
78.85	0.01\\
78.86	0.01\\
78.87	0.01\\
78.88	0.01\\
78.89	0.01\\
78.9	0.01\\
78.91	0.01\\
78.92	0.01\\
78.93	0.01\\
78.94	0.01\\
78.95	0.01\\
78.96	0.01\\
78.97	0.01\\
78.98	0.01\\
78.99	0.01\\
79	0.01\\
79.01	0.01\\
79.02	0.01\\
79.03	0.01\\
79.04	0.01\\
79.05	0.01\\
79.06	0.01\\
79.07	0.01\\
79.08	0.01\\
79.09	0.01\\
79.1	0.01\\
79.11	0.01\\
79.12	0.01\\
79.13	0.01\\
79.14	0.01\\
79.15	0.01\\
79.16	0.01\\
79.17	0.01\\
79.18	0.01\\
79.19	0.01\\
79.2	0.01\\
79.21	0.01\\
79.22	0.01\\
79.23	0.01\\
79.24	0.01\\
79.25	0.01\\
79.26	0.01\\
79.27	0.01\\
79.28	0.01\\
79.29	0.01\\
79.3	0.01\\
79.31	0.01\\
79.32	0.01\\
79.33	0.01\\
79.34	0.01\\
79.35	0.01\\
79.36	0.01\\
79.37	0.01\\
79.38	0.01\\
79.39	0.01\\
79.4	0.01\\
79.41	0.01\\
79.42	0.01\\
79.43	0.01\\
79.44	0.01\\
79.45	0.01\\
79.46	0.01\\
79.47	0.01\\
79.48	0.01\\
79.49	0.01\\
79.5	0.01\\
79.51	0.01\\
79.52	0.01\\
79.53	0.01\\
79.54	0.01\\
79.55	0.01\\
79.56	0.01\\
79.57	0.01\\
79.58	0.01\\
79.59	0.01\\
79.6	0.01\\
79.61	0.01\\
79.62	0.01\\
79.63	0.01\\
79.64	0.01\\
79.65	0.01\\
79.66	0.01\\
79.67	0.01\\
79.68	0.01\\
79.69	0.01\\
79.7	0.01\\
79.71	0.01\\
79.72	0.01\\
79.73	0.01\\
79.74	0.01\\
79.75	0.01\\
79.76	0.01\\
79.77	0.01\\
79.78	0.01\\
79.79	0.01\\
79.8	0.01\\
79.81	0.01\\
79.82	0.01\\
79.83	0.01\\
79.84	0.01\\
79.85	0.01\\
79.86	0.01\\
79.87	0.01\\
79.88	0.01\\
79.89	0.01\\
79.9	0.01\\
79.91	0.01\\
79.92	0.01\\
79.93	0.01\\
79.94	0.01\\
79.95	0.01\\
79.96	0.01\\
79.97	0.01\\
79.98	0.01\\
79.99	0.01\\
80	0.01\\
80.01	0.01\\
};
\addplot [color=red,solid]
  table[row sep=crcr]{%
80.01	0.01\\
80.02	0.01\\
80.03	0.01\\
80.04	0.01\\
80.05	0.01\\
80.06	0.01\\
80.07	0.01\\
80.08	0.01\\
80.09	0.01\\
80.1	0.01\\
80.11	0.01\\
80.12	0.01\\
80.13	0.01\\
80.14	0.01\\
80.15	0.01\\
80.16	0.01\\
80.17	0.01\\
80.18	0.01\\
80.19	0.01\\
80.2	0.01\\
80.21	0.01\\
80.22	0.01\\
80.23	0.01\\
80.24	0.01\\
80.25	0.01\\
80.26	0.01\\
80.27	0.01\\
80.28	0.01\\
80.29	0.01\\
80.3	0.01\\
80.31	0.01\\
80.32	0.01\\
80.33	0.01\\
80.34	0.01\\
80.35	0.01\\
80.36	0.01\\
80.37	0.01\\
80.38	0.01\\
80.39	0.01\\
80.4	0.01\\
80.41	0.01\\
80.42	0.01\\
80.43	0.01\\
80.44	0.01\\
80.45	0.01\\
80.46	0.01\\
80.47	0.01\\
80.48	0.01\\
80.49	0.01\\
80.5	0.01\\
80.51	0.01\\
80.52	0.01\\
80.53	0.01\\
80.54	0.01\\
80.55	0.01\\
80.56	0.01\\
80.57	0.01\\
80.58	0.01\\
80.59	0.01\\
80.6	0.01\\
80.61	0.01\\
80.62	0.01\\
80.63	0.01\\
80.64	0.01\\
80.65	0.01\\
80.66	0.01\\
80.67	0.01\\
80.68	0.01\\
80.69	0.01\\
80.7	0.01\\
80.71	0.01\\
80.72	0.01\\
80.73	0.01\\
80.74	0.01\\
80.75	0.01\\
80.76	0.01\\
80.77	0.01\\
80.78	0.01\\
80.79	0.01\\
80.8	0.01\\
80.81	0.01\\
80.82	0.01\\
80.83	0.01\\
80.84	0.01\\
80.85	0.01\\
80.86	0.01\\
80.87	0.01\\
80.88	0.01\\
80.89	0.01\\
80.9	0.01\\
80.91	0.01\\
80.92	0.01\\
80.93	0.01\\
80.94	0.01\\
80.95	0.01\\
80.96	0.01\\
80.97	0.01\\
80.98	0.01\\
80.99	0.01\\
81	0.01\\
81.01	0.01\\
81.02	0.01\\
81.03	0.01\\
81.04	0.01\\
81.05	0.01\\
81.06	0.01\\
81.07	0.01\\
81.08	0.01\\
81.09	0.01\\
81.1	0.01\\
81.11	0.01\\
81.12	0.01\\
81.13	0.01\\
81.14	0.01\\
81.15	0.01\\
81.16	0.01\\
81.17	0.01\\
81.18	0.01\\
81.19	0.01\\
81.2	0.01\\
81.21	0.01\\
81.22	0.01\\
81.23	0.01\\
81.24	0.01\\
81.25	0.01\\
81.26	0.01\\
81.27	0.01\\
81.28	0.01\\
81.29	0.01\\
81.3	0.01\\
81.31	0.01\\
81.32	0.01\\
81.33	0.01\\
81.34	0.01\\
81.35	0.01\\
81.36	0.01\\
81.37	0.01\\
81.38	0.01\\
81.39	0.01\\
81.4	0.01\\
81.41	0.01\\
81.42	0.01\\
81.43	0.01\\
81.44	0.01\\
81.45	0.01\\
81.46	0.01\\
81.47	0.01\\
81.48	0.01\\
81.49	0.01\\
81.5	0.01\\
81.51	0.01\\
81.52	0.01\\
81.53	0.01\\
81.54	0.01\\
81.55	0.01\\
81.56	0.01\\
81.57	0.01\\
81.58	0.01\\
81.59	0.01\\
81.6	0.01\\
81.61	0.01\\
81.62	0.01\\
81.63	0.01\\
81.64	0.01\\
81.65	0.01\\
81.66	0.01\\
81.67	0.01\\
81.68	0.01\\
81.69	0.01\\
81.7	0.01\\
81.71	0.01\\
81.72	0.01\\
81.73	0.01\\
81.74	0.01\\
81.75	0.01\\
81.76	0.01\\
81.77	0.01\\
81.78	0.01\\
81.79	0.01\\
81.8	0.01\\
81.81	0.01\\
81.82	0.01\\
81.83	0.01\\
81.84	0.01\\
81.85	0.01\\
81.86	0.01\\
81.87	0.01\\
81.88	0.01\\
81.89	0.01\\
81.9	0.01\\
81.91	0.01\\
81.92	0.01\\
81.93	0.01\\
81.94	0.01\\
81.95	0.01\\
81.96	0.01\\
81.97	0.01\\
81.98	0.01\\
81.99	0.01\\
82	0.01\\
82.01	0.01\\
82.02	0.01\\
82.03	0.01\\
82.04	0.01\\
82.05	0.01\\
82.06	0.01\\
82.07	0.01\\
82.08	0.01\\
82.09	0.01\\
82.1	0.01\\
82.11	0.01\\
82.12	0.01\\
82.13	0.01\\
82.14	0.01\\
82.15	0.01\\
82.16	0.01\\
82.17	0.01\\
82.18	0.01\\
82.19	0.01\\
82.2	0.01\\
82.21	0.01\\
82.22	0.01\\
82.23	0.01\\
82.24	0.01\\
82.25	0.01\\
82.26	0.01\\
82.27	0.01\\
82.28	0.01\\
82.29	0.01\\
82.3	0.01\\
82.31	0.01\\
82.32	0.01\\
82.33	0.01\\
82.34	0.01\\
82.35	0.01\\
82.36	0.01\\
82.37	0.01\\
82.38	0.01\\
82.39	0.01\\
82.4	0.01\\
82.41	0.01\\
82.42	0.01\\
82.43	0.01\\
82.44	0.01\\
82.45	0.01\\
82.46	0.01\\
82.47	0.01\\
82.48	0.01\\
82.49	0.01\\
82.5	0.01\\
82.51	0.01\\
82.52	0.01\\
82.53	0.01\\
82.54	0.01\\
82.55	0.01\\
82.56	0.01\\
82.57	0.01\\
82.58	0.01\\
82.59	0.01\\
82.6	0.01\\
82.61	0.01\\
82.62	0.01\\
82.63	0.01\\
82.64	0.01\\
82.65	0.01\\
82.66	0.01\\
82.67	0.01\\
82.68	0.01\\
82.69	0.01\\
82.7	0.01\\
82.71	0.01\\
82.72	0.01\\
82.73	0.01\\
82.74	0.01\\
82.75	0.01\\
82.76	0.01\\
82.77	0.01\\
82.78	0.01\\
82.79	0.01\\
82.8	0.01\\
82.81	0.01\\
82.82	0.01\\
82.83	0.01\\
82.84	0.01\\
82.85	0.01\\
82.86	0.01\\
82.87	0.01\\
82.88	0.01\\
82.89	0.01\\
82.9	0.01\\
82.91	0.01\\
82.92	0.01\\
82.93	0.01\\
82.94	0.01\\
82.95	0.01\\
82.96	0.01\\
82.97	0.01\\
82.98	0.01\\
82.99	0.01\\
83	0.01\\
83.01	0.01\\
83.02	0.01\\
83.03	0.01\\
83.04	0.01\\
83.05	0.01\\
83.06	0.01\\
83.07	0.01\\
83.08	0.01\\
83.09	0.01\\
83.1	0.01\\
83.11	0.01\\
83.12	0.01\\
83.13	0.01\\
83.14	0.01\\
83.15	0.01\\
83.16	0.01\\
83.17	0.01\\
83.18	0.01\\
83.19	0.01\\
83.2	0.01\\
83.21	0.01\\
83.22	0.01\\
83.23	0.01\\
83.24	0.01\\
83.25	0.01\\
83.26	0.01\\
83.27	0.01\\
83.28	0.01\\
83.29	0.01\\
83.3	0.01\\
83.31	0.01\\
83.32	0.01\\
83.33	0.01\\
83.34	0.01\\
83.35	0.01\\
83.36	0.01\\
83.37	0.01\\
83.38	0.01\\
83.39	0.01\\
83.4	0.01\\
83.41	0.01\\
83.42	0.01\\
83.43	0.01\\
83.44	0.01\\
83.45	0.01\\
83.46	0.01\\
83.47	0.01\\
83.48	0.01\\
83.49	0.01\\
83.5	0.01\\
83.51	0.01\\
83.52	0.01\\
83.53	0.01\\
83.54	0.01\\
83.55	0.01\\
83.56	0.01\\
83.57	0.01\\
83.58	0.01\\
83.59	0.01\\
83.6	0.01\\
83.61	0.01\\
83.62	0.01\\
83.63	0.01\\
83.64	0.01\\
83.65	0.01\\
83.66	0.01\\
83.67	0.01\\
83.68	0.01\\
83.69	0.01\\
83.7	0.01\\
83.71	0.01\\
83.72	0.01\\
83.73	0.01\\
83.74	0.01\\
83.75	0.01\\
83.76	0.01\\
83.77	0.01\\
83.78	0.01\\
83.79	0.01\\
83.8	0.01\\
83.81	0.01\\
83.82	0.01\\
83.83	0.01\\
83.84	0.01\\
83.85	0.01\\
83.86	0.01\\
83.87	0.01\\
83.88	0.01\\
83.89	0.01\\
83.9	0.01\\
83.91	0.01\\
83.92	0.01\\
83.93	0.01\\
83.94	0.01\\
83.95	0.01\\
83.96	0.01\\
83.97	0.01\\
83.98	0.01\\
83.99	0.01\\
84	0.01\\
84.01	0.01\\
84.02	0.01\\
84.03	0.01\\
84.04	0.01\\
84.05	0.01\\
84.06	0.01\\
84.07	0.01\\
84.08	0.01\\
84.09	0.01\\
84.1	0.01\\
84.11	0.01\\
84.12	0.01\\
84.13	0.01\\
84.14	0.01\\
84.15	0.01\\
84.16	0.01\\
84.17	0.01\\
84.18	0.01\\
84.19	0.01\\
84.2	0.01\\
84.21	0.01\\
84.22	0.01\\
84.23	0.01\\
84.24	0.01\\
84.25	0.01\\
84.26	0.01\\
84.27	0.01\\
84.28	0.01\\
84.29	0.01\\
84.3	0.01\\
84.31	0.01\\
84.32	0.01\\
84.33	0.01\\
84.34	0.01\\
84.35	0.01\\
84.36	0.01\\
84.37	0.01\\
84.38	0.01\\
84.39	0.01\\
84.4	0.01\\
84.41	0.01\\
84.42	0.01\\
84.43	0.01\\
84.44	0.01\\
84.45	0.01\\
84.46	0.01\\
84.47	0.01\\
84.48	0.01\\
84.49	0.01\\
84.5	0.01\\
84.51	0.01\\
84.52	0.01\\
84.53	0.01\\
84.54	0.01\\
84.55	0.01\\
84.56	0.01\\
84.57	0.01\\
84.58	0.01\\
84.59	0.01\\
84.6	0.01\\
84.61	0.01\\
84.62	0.01\\
84.63	0.01\\
84.64	0.01\\
84.65	0.01\\
84.66	0.01\\
84.67	0.01\\
84.68	0.01\\
84.69	0.01\\
84.7	0.01\\
84.71	0.01\\
84.72	0.01\\
84.73	0.01\\
84.74	0.01\\
84.75	0.01\\
84.76	0.01\\
84.77	0.01\\
84.78	0.01\\
84.79	0.01\\
84.8	0.01\\
84.81	0.01\\
84.82	0.01\\
84.83	0.01\\
84.84	0.01\\
84.85	0.01\\
84.86	0.01\\
84.87	0.01\\
84.88	0.01\\
84.89	0.01\\
84.9	0.01\\
84.91	0.01\\
84.92	0.01\\
84.93	0.01\\
84.94	0.01\\
84.95	0.01\\
84.96	0.01\\
84.97	0.01\\
84.98	0.01\\
84.99	0.01\\
85	0.01\\
85.01	0.01\\
85.02	0.01\\
85.03	0.01\\
85.04	0.01\\
85.05	0.01\\
85.06	0.01\\
85.07	0.01\\
85.08	0.01\\
85.09	0.01\\
85.1	0.01\\
85.11	0.01\\
85.12	0.01\\
85.13	0.01\\
85.14	0.01\\
85.15	0.01\\
85.16	0.01\\
85.17	0.01\\
85.18	0.01\\
85.19	0.01\\
85.2	0.01\\
85.21	0.01\\
85.22	0.01\\
85.23	0.01\\
85.24	0.01\\
85.25	0.01\\
85.26	0.01\\
85.27	0.01\\
85.28	0.01\\
85.29	0.01\\
85.3	0.01\\
85.31	0.01\\
85.32	0.01\\
85.33	0.01\\
85.34	0.01\\
85.35	0.01\\
85.36	0.01\\
85.37	0.01\\
85.38	0.01\\
85.39	0.01\\
85.4	0.01\\
85.41	0.01\\
85.42	0.01\\
85.43	0.01\\
85.44	0.01\\
85.45	0.01\\
85.46	0.01\\
85.47	0.01\\
85.48	0.01\\
85.49	0.01\\
85.5	0.01\\
85.51	0.01\\
85.52	0.01\\
85.53	0.01\\
85.54	0.01\\
85.55	0.01\\
85.56	0.01\\
85.57	0.01\\
85.58	0.01\\
85.59	0.01\\
85.6	0.01\\
85.61	0.01\\
85.62	0.01\\
85.63	0.01\\
85.64	0.01\\
85.65	0.01\\
85.66	0.01\\
85.67	0.01\\
85.68	0.01\\
85.69	0.01\\
85.7	0.01\\
85.71	0.01\\
85.72	0.01\\
85.73	0.01\\
85.74	0.01\\
85.75	0.01\\
85.76	0.01\\
85.77	0.01\\
85.78	0.01\\
85.79	0.01\\
85.8	0.01\\
85.81	0.01\\
85.82	0.01\\
85.83	0.01\\
85.84	0.01\\
85.85	0.01\\
85.86	0.01\\
85.87	0.01\\
85.88	0.01\\
85.89	0.01\\
85.9	0.01\\
85.91	0.01\\
85.92	0.01\\
85.93	0.01\\
85.94	0.01\\
85.95	0.01\\
85.96	0.01\\
85.97	0.01\\
85.98	0.01\\
85.99	0.01\\
86	0.01\\
86.01	0.01\\
86.02	0.01\\
86.03	0.01\\
86.04	0.01\\
86.05	0.01\\
86.06	0.01\\
86.07	0.01\\
86.08	0.01\\
86.09	0.01\\
86.1	0.01\\
86.11	0.01\\
86.12	0.01\\
86.13	0.01\\
86.14	0.01\\
86.15	0.01\\
86.16	0.01\\
86.17	0.01\\
86.18	0.01\\
86.19	0.01\\
86.2	0.01\\
86.21	0.01\\
86.22	0.01\\
86.23	0.01\\
86.24	0.01\\
86.25	0.01\\
86.26	0.01\\
86.27	0.01\\
86.28	0.01\\
86.29	0.01\\
86.3	0.01\\
86.31	0.01\\
86.32	0.01\\
86.33	0.01\\
86.34	0.01\\
86.35	0.01\\
86.36	0.01\\
86.37	0.01\\
86.38	0.01\\
86.39	0.01\\
86.4	0.01\\
86.41	0.01\\
86.42	0.01\\
86.43	0.01\\
86.44	0.01\\
86.45	0.01\\
86.46	0.01\\
86.47	0.01\\
86.48	0.01\\
86.49	0.01\\
86.5	0.01\\
86.51	0.01\\
86.52	0.01\\
86.53	0.01\\
86.54	0.01\\
86.55	0.01\\
86.56	0.01\\
86.57	0.01\\
86.58	0.01\\
86.59	0.01\\
86.6	0.01\\
86.61	0.01\\
86.62	0.01\\
86.63	0.01\\
86.64	0.01\\
86.65	0.01\\
86.66	0.01\\
86.67	0.01\\
86.68	0.01\\
86.69	0.01\\
86.7	0.01\\
86.71	0.01\\
86.72	0.01\\
86.73	0.01\\
86.74	0.01\\
86.75	0.01\\
86.76	0.01\\
86.77	0.01\\
86.78	0.01\\
86.79	0.01\\
86.8	0.01\\
86.81	0.01\\
86.82	0.01\\
86.83	0.01\\
86.84	0.01\\
86.85	0.01\\
86.86	0.01\\
86.87	0.01\\
86.88	0.01\\
86.89	0.01\\
86.9	0.01\\
86.91	0.01\\
86.92	0.01\\
86.93	0.01\\
86.94	0.01\\
86.95	0.01\\
86.96	0.01\\
86.97	0.01\\
86.98	0.01\\
86.99	0.01\\
87	0.01\\
87.01	0.01\\
87.02	0.01\\
87.03	0.01\\
87.04	0.01\\
87.05	0.01\\
87.06	0.01\\
87.07	0.01\\
87.08	0.01\\
87.09	0.01\\
87.1	0.01\\
87.11	0.01\\
87.12	0.01\\
87.13	0.01\\
87.14	0.01\\
87.15	0.01\\
87.16	0.01\\
87.17	0.01\\
87.18	0.01\\
87.19	0.01\\
87.2	0.01\\
87.21	0.01\\
87.22	0.01\\
87.23	0.01\\
87.24	0.01\\
87.25	0.01\\
87.26	0.01\\
87.27	0.01\\
87.28	0.01\\
87.29	0.01\\
87.3	0.01\\
87.31	0.01\\
87.32	0.01\\
87.33	0.01\\
87.34	0.01\\
87.35	0.01\\
87.36	0.01\\
87.37	0.01\\
87.38	0.01\\
87.39	0.01\\
87.4	0.01\\
87.41	0.01\\
87.42	0.01\\
87.43	0.01\\
87.44	0.01\\
87.45	0.01\\
87.46	0.01\\
87.47	0.01\\
87.48	0.01\\
87.49	0.01\\
87.5	0.01\\
87.51	0.01\\
87.52	0.01\\
87.53	0.01\\
87.54	0.01\\
87.55	0.01\\
87.56	0.01\\
87.57	0.01\\
87.58	0.01\\
87.59	0.01\\
87.6	0.01\\
87.61	0.01\\
87.62	0.01\\
87.63	0.01\\
87.64	0.01\\
87.65	0.01\\
87.66	0.01\\
87.67	0.01\\
87.68	0.01\\
87.69	0.01\\
87.7	0.01\\
87.71	0.01\\
87.72	0.01\\
87.73	0.01\\
87.74	0.01\\
87.75	0.01\\
87.76	0.01\\
87.77	0.01\\
87.78	0.01\\
87.79	0.01\\
87.8	0.01\\
87.81	0.01\\
87.82	0.01\\
87.83	0.01\\
87.84	0.01\\
87.85	0.01\\
87.86	0.01\\
87.87	0.01\\
87.88	0.01\\
87.89	0.01\\
87.9	0.01\\
87.91	0.01\\
87.92	0.01\\
87.93	0.01\\
87.94	0.01\\
87.95	0.01\\
87.96	0.01\\
87.97	0.01\\
87.98	0.01\\
87.99	0.01\\
88	0.01\\
88.01	0.01\\
88.02	0.01\\
88.03	0.01\\
88.04	0.01\\
88.05	0.01\\
88.06	0.01\\
88.07	0.01\\
88.08	0.01\\
88.09	0.01\\
88.1	0.01\\
88.11	0.01\\
88.12	0.01\\
88.13	0.01\\
88.14	0.01\\
88.15	0.01\\
88.16	0.01\\
88.17	0.01\\
88.18	0.01\\
88.19	0.01\\
88.2	0.01\\
88.21	0.01\\
88.22	0.01\\
88.23	0.01\\
88.24	0.01\\
88.25	0.01\\
88.26	0.01\\
88.27	0.01\\
88.28	0.01\\
88.29	0.01\\
88.3	0.01\\
88.31	0.01\\
88.32	0.01\\
88.33	0.01\\
88.34	0.01\\
88.35	0.01\\
88.36	0.01\\
88.37	0.01\\
88.38	0.01\\
88.39	0.01\\
88.4	0.01\\
88.41	0.01\\
88.42	0.01\\
88.43	0.01\\
88.44	0.01\\
88.45	0.01\\
88.46	0.01\\
88.47	0.01\\
88.48	0.01\\
88.49	0.01\\
88.5	0.01\\
88.51	0.01\\
88.52	0.01\\
88.53	0.01\\
88.54	0.01\\
88.55	0.01\\
88.56	0.01\\
88.57	0.01\\
88.58	0.01\\
88.59	0.01\\
88.6	0.01\\
88.61	0.01\\
88.62	0.01\\
88.63	0.01\\
88.64	0.01\\
88.65	0.01\\
88.66	0.01\\
88.67	0.01\\
88.68	0.01\\
88.69	0.01\\
88.7	0.01\\
88.71	0.01\\
88.72	0.01\\
88.73	0.01\\
88.74	0.01\\
88.75	0.01\\
88.76	0.01\\
88.77	0.01\\
88.78	0.01\\
88.79	0.01\\
88.8	0.01\\
88.81	0.01\\
88.82	0.01\\
88.83	0.01\\
88.84	0.01\\
88.85	0.01\\
88.86	0.01\\
88.87	0.01\\
88.88	0.01\\
88.89	0.01\\
88.9	0.01\\
88.91	0.01\\
88.92	0.01\\
88.93	0.01\\
88.94	0.01\\
88.95	0.01\\
88.96	0.01\\
88.97	0.01\\
88.98	0.01\\
88.99	0.01\\
89	0.01\\
89.01	0.01\\
89.02	0.01\\
89.03	0.01\\
89.04	0.01\\
89.05	0.01\\
89.06	0.01\\
89.07	0.01\\
89.08	0.01\\
89.09	0.01\\
89.1	0.01\\
89.11	0.01\\
89.12	0.01\\
89.13	0.01\\
89.14	0.01\\
89.15	0.01\\
89.16	0.01\\
89.17	0.01\\
89.18	0.01\\
89.19	0.01\\
89.2	0.01\\
89.21	0.01\\
89.22	0.01\\
89.23	0.01\\
89.24	0.01\\
89.25	0.01\\
89.26	0.01\\
89.27	0.01\\
89.28	0.01\\
89.29	0.01\\
89.3	0.01\\
89.31	0.01\\
89.32	0.01\\
89.33	0.01\\
89.34	0.01\\
89.35	0.01\\
89.36	0.01\\
89.37	0.01\\
89.38	0.01\\
89.39	0.01\\
89.4	0.01\\
89.41	0.01\\
89.42	0.01\\
89.43	0.01\\
89.44	0.01\\
89.45	0.01\\
89.46	0.01\\
89.47	0.01\\
89.48	0.01\\
89.49	0.01\\
89.5	0.01\\
89.51	0.01\\
89.52	0.01\\
89.53	0.01\\
89.54	0.01\\
89.55	0.01\\
89.56	0.01\\
89.57	0.01\\
89.58	0.01\\
89.59	0.01\\
89.6	0.01\\
89.61	0.01\\
89.62	0.01\\
89.63	0.01\\
89.64	0.01\\
89.65	0.01\\
89.66	0.01\\
89.67	0.01\\
89.68	0.01\\
89.69	0.01\\
89.7	0.01\\
89.71	0.01\\
89.72	0.01\\
89.73	0.01\\
89.74	0.01\\
89.75	0.01\\
89.76	0.01\\
89.77	0.01\\
89.78	0.01\\
89.79	0.01\\
89.8	0.01\\
89.81	0.01\\
89.82	0.01\\
89.83	0.01\\
89.84	0.01\\
89.85	0.01\\
89.86	0.01\\
89.87	0.01\\
89.88	0.01\\
89.89	0.01\\
89.9	0.01\\
89.91	0.01\\
89.92	0.01\\
89.93	0.01\\
89.94	0.01\\
89.95	0.01\\
89.96	0.01\\
89.97	0.01\\
89.98	0.01\\
89.99	0.01\\
90	0.01\\
90.01	0.01\\
90.02	0.01\\
90.03	0.01\\
90.04	0.01\\
90.05	0.01\\
90.06	0.01\\
90.07	0.01\\
90.08	0.01\\
90.09	0.01\\
90.1	0.01\\
90.11	0.01\\
90.12	0.01\\
90.13	0.01\\
90.14	0.01\\
90.15	0.01\\
90.16	0.01\\
90.17	0.01\\
90.18	0.01\\
90.19	0.01\\
90.2	0.01\\
90.21	0.01\\
90.22	0.01\\
90.23	0.01\\
90.24	0.01\\
90.25	0.01\\
90.26	0.01\\
90.27	0.01\\
90.28	0.01\\
90.29	0.01\\
90.3	0.01\\
90.31	0.01\\
90.32	0.01\\
90.33	0.01\\
90.34	0.01\\
90.35	0.01\\
90.36	0.01\\
90.37	0.01\\
90.38	0.01\\
90.39	0.01\\
90.4	0.01\\
90.41	0.01\\
90.42	0.01\\
90.43	0.01\\
90.44	0.01\\
90.45	0.01\\
90.46	0.01\\
90.47	0.01\\
90.48	0.01\\
90.49	0.01\\
90.5	0.01\\
90.51	0.01\\
90.52	0.01\\
90.53	0.01\\
90.54	0.01\\
90.55	0.01\\
90.56	0.01\\
90.57	0.01\\
90.58	0.01\\
90.59	0.01\\
90.6	0.01\\
90.61	0.01\\
90.62	0.01\\
90.63	0.01\\
90.64	0.01\\
90.65	0.01\\
90.66	0.01\\
90.67	0.01\\
90.68	0.01\\
90.69	0.01\\
90.7	0.01\\
90.71	0.01\\
90.72	0.01\\
90.73	0.01\\
90.74	0.01\\
90.75	0.01\\
90.76	0.01\\
90.77	0.01\\
90.78	0.01\\
90.79	0.01\\
90.8	0.01\\
90.81	0.01\\
90.82	0.01\\
90.83	0.01\\
90.84	0.01\\
90.85	0.01\\
90.86	0.01\\
90.87	0.01\\
90.88	0.01\\
90.89	0.01\\
90.9	0.01\\
90.91	0.01\\
90.92	0.01\\
90.93	0.01\\
90.94	0.01\\
90.95	0.01\\
90.96	0.01\\
90.97	0.01\\
90.98	0.01\\
90.99	0.01\\
91	0.01\\
91.01	0.01\\
91.02	0.01\\
91.03	0.01\\
91.04	0.01\\
91.05	0.01\\
91.06	0.01\\
91.07	0.01\\
91.08	0.01\\
91.09	0.01\\
91.1	0.01\\
91.11	0.01\\
91.12	0.01\\
91.13	0.01\\
91.14	0.01\\
91.15	0.01\\
91.16	0.01\\
91.17	0.01\\
91.18	0.01\\
91.19	0.01\\
91.2	0.01\\
91.21	0.01\\
91.22	0.01\\
91.23	0.01\\
91.24	0.01\\
91.25	0.01\\
91.26	0.01\\
91.27	0.01\\
91.28	0.01\\
91.29	0.01\\
91.3	0.01\\
91.31	0.01\\
91.32	0.01\\
91.33	0.01\\
91.34	0.01\\
91.35	0.01\\
91.36	0.01\\
91.37	0.01\\
91.38	0.01\\
91.39	0.01\\
91.4	0.01\\
91.41	0.01\\
91.42	0.01\\
91.43	0.01\\
91.44	0.01\\
91.45	0.01\\
91.46	0.01\\
91.47	0.01\\
91.48	0.01\\
91.49	0.01\\
91.5	0.01\\
91.51	0.01\\
91.52	0.01\\
91.53	0.01\\
91.54	0.01\\
91.55	0.01\\
91.56	0.01\\
91.57	0.01\\
91.58	0.01\\
91.59	0.01\\
91.6	0.01\\
91.61	0.01\\
91.62	0.01\\
91.63	0.01\\
91.64	0.01\\
91.65	0.01\\
91.66	0.01\\
91.67	0.01\\
91.68	0.01\\
91.69	0.01\\
91.7	0.01\\
91.71	0.01\\
91.72	0.01\\
91.73	0.01\\
91.74	0.01\\
91.75	0.01\\
91.76	0.01\\
91.77	0.01\\
91.78	0.01\\
91.79	0.01\\
91.8	0.01\\
91.81	0.01\\
91.82	0.01\\
91.83	0.01\\
91.84	0.01\\
91.85	0.01\\
91.86	0.01\\
91.87	0.01\\
91.88	0.01\\
91.89	0.01\\
91.9	0.01\\
91.91	0.01\\
91.92	0.01\\
91.93	0.01\\
91.94	0.01\\
91.95	0.01\\
91.96	0.01\\
91.97	0.01\\
91.98	0.01\\
91.99	0.01\\
92	0.01\\
92.01	0.01\\
92.02	0.01\\
92.03	0.01\\
92.04	0.01\\
92.05	0.01\\
92.06	0.01\\
92.07	0.01\\
92.08	0.01\\
92.09	0.01\\
92.1	0.01\\
92.11	0.01\\
92.12	0.01\\
92.13	0.01\\
92.14	0.01\\
92.15	0.01\\
92.16	0.01\\
92.17	0.01\\
92.18	0.01\\
92.19	0.01\\
92.2	0.01\\
92.21	0.01\\
92.22	0.01\\
92.23	0.01\\
92.24	0.01\\
92.25	0.01\\
92.26	0.01\\
92.27	0.01\\
92.28	0.01\\
92.29	0.01\\
92.3	0.01\\
92.31	0.01\\
92.32	0.01\\
92.33	0.01\\
92.34	0.01\\
92.35	0.01\\
92.36	0.01\\
92.37	0.01\\
92.38	0.01\\
92.39	0.01\\
92.4	0.01\\
92.41	0.01\\
92.42	0.01\\
92.43	0.01\\
92.44	0.01\\
92.45	0.01\\
92.46	0.01\\
92.47	0.01\\
92.48	0.01\\
92.49	0.01\\
92.5	0.01\\
92.51	0.01\\
92.52	0.01\\
92.53	0.01\\
92.54	0.01\\
92.55	0.01\\
92.56	0.01\\
92.57	0.01\\
92.58	0.01\\
92.59	0.01\\
92.6	0.01\\
92.61	0.01\\
92.62	0.01\\
92.63	0.01\\
92.64	0.01\\
92.65	0.01\\
92.66	0.01\\
92.67	0.01\\
92.68	0.01\\
92.69	0.01\\
92.7	0.01\\
92.71	0.01\\
92.72	0.01\\
92.73	0.01\\
92.74	0.01\\
92.75	0.01\\
92.76	0.01\\
92.77	0.01\\
92.78	0.01\\
92.79	0.01\\
92.8	0.01\\
92.81	0.01\\
92.82	0.01\\
92.83	0.01\\
92.84	0.01\\
92.85	0.01\\
92.86	0.01\\
92.87	0.01\\
92.88	0.01\\
92.89	0.01\\
92.9	0.01\\
92.91	0.01\\
92.92	0.01\\
92.93	0.01\\
92.94	0.01\\
92.95	0.01\\
92.96	0.01\\
92.97	0.01\\
92.98	0.01\\
92.99	0.01\\
93	0.01\\
93.01	0.01\\
93.02	0.01\\
93.03	0.01\\
93.04	0.01\\
93.05	0.01\\
93.06	0.01\\
93.07	0.01\\
93.08	0.01\\
93.09	0.01\\
93.1	0.01\\
93.11	0.01\\
93.12	0.01\\
93.13	0.01\\
93.14	0.01\\
93.15	0.01\\
93.16	0.01\\
93.17	0.01\\
93.18	0.01\\
93.19	0.01\\
93.2	0.01\\
93.21	0.01\\
93.22	0.01\\
93.23	0.01\\
93.24	0.01\\
93.25	0.01\\
93.26	0.01\\
93.27	0.01\\
93.28	0.01\\
93.29	0.01\\
93.3	0.01\\
93.31	0.01\\
93.32	0.01\\
93.33	0.01\\
93.34	0.01\\
93.35	0.01\\
93.36	0.01\\
93.37	0.01\\
93.38	0.01\\
93.39	0.01\\
93.4	0.01\\
93.41	0.01\\
93.42	0.01\\
93.43	0.01\\
93.44	0.01\\
93.45	0.01\\
93.46	0.01\\
93.47	0.01\\
93.48	0.01\\
93.49	0.01\\
93.5	0.01\\
93.51	0.01\\
93.52	0.01\\
93.53	0.01\\
93.54	0.01\\
93.55	0.01\\
93.56	0.01\\
93.57	0.01\\
93.58	0.01\\
93.59	0.01\\
93.6	0.01\\
93.61	0.01\\
93.62	0.01\\
93.63	0.01\\
93.64	0.01\\
93.65	0.01\\
93.66	0.01\\
93.67	0.01\\
93.68	0.01\\
93.69	0.01\\
93.7	0.01\\
93.71	0.01\\
93.72	0.01\\
93.73	0.01\\
93.74	0.01\\
93.75	0.01\\
93.76	0.01\\
93.77	0.01\\
93.78	0.01\\
93.79	0.01\\
93.8	0.01\\
93.81	0.01\\
93.82	0.01\\
93.83	0.01\\
93.84	0.01\\
93.85	0.01\\
93.86	0.01\\
93.87	0.01\\
93.88	0.01\\
93.89	0.01\\
93.9	0.01\\
93.91	0.01\\
93.92	0.01\\
93.93	0.01\\
93.94	0.01\\
93.95	0.01\\
93.96	0.01\\
93.97	0.01\\
93.98	0.01\\
93.99	0.01\\
94	0.01\\
94.01	0.01\\
94.02	0.01\\
94.03	0.01\\
94.04	0.01\\
94.05	0.01\\
94.06	0.01\\
94.07	0.01\\
94.08	0.01\\
94.09	0.01\\
94.1	0.01\\
94.11	0.01\\
94.12	0.01\\
94.13	0.01\\
94.14	0.01\\
94.15	0.01\\
94.16	0.01\\
94.17	0.01\\
94.18	0.01\\
94.19	0.01\\
94.2	0.01\\
94.21	0.01\\
94.22	0.01\\
94.23	0.01\\
94.24	0.01\\
94.25	0.01\\
94.26	0.01\\
94.27	0.01\\
94.28	0.01\\
94.29	0.01\\
94.3	0.01\\
94.31	0.01\\
94.32	0.01\\
94.33	0.01\\
94.34	0.01\\
94.35	0.01\\
94.36	0.01\\
94.37	0.01\\
94.38	0.01\\
94.39	0.01\\
94.4	0.01\\
94.41	0.01\\
94.42	0.01\\
94.43	0.01\\
94.44	0.01\\
94.45	0.01\\
94.46	0.01\\
94.47	0.01\\
94.48	0.01\\
94.49	0.01\\
94.5	0.01\\
94.51	0.01\\
94.52	0.01\\
94.53	0.01\\
94.54	0.01\\
94.55	0.01\\
94.56	0.01\\
94.57	0.01\\
94.58	0.01\\
94.59	0.01\\
94.6	0.01\\
94.61	0.01\\
94.62	0.01\\
94.63	0.01\\
94.64	0.01\\
94.65	0.01\\
94.66	0.01\\
94.67	0.01\\
94.68	0.01\\
94.69	0.01\\
94.7	0.01\\
94.71	0.01\\
94.72	0.01\\
94.73	0.01\\
94.74	0.01\\
94.75	0.01\\
94.76	0.01\\
94.77	0.01\\
94.78	0.01\\
94.79	0.01\\
94.8	0.01\\
94.81	0.01\\
94.82	0.01\\
94.83	0.01\\
94.84	0.01\\
94.85	0.01\\
94.86	0.01\\
94.87	0.01\\
94.88	0.01\\
94.89	0.01\\
94.9	0.01\\
94.91	0.01\\
94.92	0.01\\
94.93	0.01\\
94.94	0.01\\
94.95	0.01\\
94.96	0.01\\
94.97	0.01\\
94.98	0.01\\
94.99	0.01\\
95	0.01\\
95.01	0.01\\
95.02	0.01\\
95.03	0.01\\
95.04	0.01\\
95.05	0.01\\
95.06	0.01\\
95.07	0.01\\
95.08	0.01\\
95.09	0.01\\
95.1	0.01\\
95.11	0.01\\
95.12	0.01\\
95.13	0.01\\
95.14	0.01\\
95.15	0.01\\
95.16	0.01\\
95.17	0.01\\
95.18	0.01\\
95.19	0.01\\
95.2	0.01\\
95.21	0.01\\
95.22	0.01\\
95.23	0.01\\
95.24	0.01\\
95.25	0.01\\
95.26	0.01\\
95.27	0.01\\
95.28	0.01\\
95.29	0.01\\
95.3	0.01\\
95.31	0.01\\
95.32	0.01\\
95.33	0.01\\
95.34	0.01\\
95.35	0.01\\
95.36	0.01\\
95.37	0.01\\
95.38	0.01\\
95.39	0.01\\
95.4	0.01\\
95.41	0.01\\
95.42	0.01\\
95.43	0.01\\
95.44	0.01\\
95.45	0.01\\
95.46	0.01\\
95.47	0.01\\
95.48	0.01\\
95.49	0.01\\
95.5	0.01\\
95.51	0.01\\
95.52	0.01\\
95.53	0.01\\
95.54	0.01\\
95.55	0.01\\
95.56	0.01\\
95.57	0.01\\
95.58	0.01\\
95.59	0.01\\
95.6	0.01\\
95.61	0.01\\
95.62	0.01\\
95.63	0.01\\
95.64	0.01\\
95.65	0.01\\
95.66	0.01\\
95.67	0.01\\
95.68	0.01\\
95.69	0.01\\
95.7	0.01\\
95.71	0.01\\
95.72	0.01\\
95.73	0.01\\
95.74	0.01\\
95.75	0.01\\
95.76	0.01\\
95.77	0.01\\
95.78	0.01\\
95.79	0.01\\
95.8	0.01\\
95.81	0.01\\
95.82	0.01\\
95.83	0.01\\
95.84	0.01\\
95.85	0.01\\
95.86	0.01\\
95.87	0.01\\
95.88	0.01\\
95.89	0.01\\
95.9	0.01\\
95.91	0.01\\
95.92	0.01\\
95.93	0.01\\
95.94	0.01\\
95.95	0.01\\
95.96	0.01\\
95.97	0.01\\
95.98	0.01\\
95.99	0.01\\
96	0.01\\
96.01	0.01\\
96.02	0.01\\
96.03	0.01\\
96.04	0.01\\
96.05	0.01\\
96.06	0.01\\
96.07	0.01\\
96.08	0.01\\
96.09	0.01\\
96.1	0.01\\
96.11	0.01\\
96.12	0.01\\
96.13	0.01\\
96.14	0.01\\
96.15	0.01\\
96.16	0.01\\
96.17	0.01\\
96.18	0.01\\
96.19	0.01\\
96.2	0.01\\
96.21	0.01\\
96.22	0.01\\
96.23	0.01\\
96.24	0.01\\
96.25	0.01\\
96.26	0.01\\
96.27	0.01\\
96.28	0.01\\
96.29	0.01\\
96.3	0.01\\
96.31	0.01\\
96.32	0.01\\
96.33	0.01\\
96.34	0.01\\
96.35	0.01\\
96.36	0.01\\
96.37	0.01\\
96.38	0.01\\
96.39	0.01\\
96.4	0.01\\
96.41	0.01\\
96.42	0.01\\
96.43	0.01\\
96.44	0.01\\
96.45	0.01\\
96.46	0.01\\
96.47	0.01\\
96.48	0.01\\
96.49	0.01\\
96.5	0.01\\
96.51	0.01\\
96.52	0.01\\
96.53	0.01\\
96.54	0.01\\
96.55	0.01\\
96.56	0.01\\
96.57	0.01\\
96.58	0.01\\
96.59	0.01\\
96.6	0.01\\
96.61	0.01\\
96.62	0.01\\
96.63	0.01\\
96.64	0.01\\
96.65	0.01\\
96.66	0.01\\
96.67	0.01\\
96.68	0.01\\
96.69	0.01\\
96.7	0.01\\
96.71	0.01\\
96.72	0.01\\
96.73	0.01\\
96.74	0.01\\
96.75	0.01\\
96.76	0.01\\
96.77	0.01\\
96.78	0.01\\
96.79	0.01\\
96.8	0.01\\
96.81	0.01\\
96.82	0.01\\
96.83	0.01\\
96.84	0.01\\
96.85	0.01\\
96.86	0.01\\
96.87	0.01\\
96.88	0.01\\
96.89	0.01\\
96.9	0.01\\
96.91	0.01\\
96.92	0.01\\
96.93	0.01\\
96.94	0.01\\
96.95	0.01\\
96.96	0.01\\
96.97	0.01\\
96.98	0.01\\
96.99	0.01\\
97	0.01\\
97.01	0.01\\
97.02	0.01\\
97.03	0.01\\
97.04	0.01\\
97.05	0.01\\
97.06	0.01\\
97.07	0.01\\
97.08	0.01\\
97.09	0.01\\
97.1	0.01\\
97.11	0.01\\
97.12	0.01\\
97.13	0.01\\
97.14	0.01\\
97.15	0.01\\
97.16	0.01\\
97.17	0.01\\
97.18	0.01\\
97.19	0.01\\
97.2	0.01\\
97.21	0.01\\
97.22	0.01\\
97.23	0.01\\
97.24	0.01\\
97.25	0.01\\
97.26	0.01\\
97.27	0.01\\
97.28	0.01\\
97.29	0.01\\
97.3	0.01\\
97.31	0.01\\
97.32	0.01\\
97.33	0.01\\
97.34	0.01\\
97.35	0.01\\
97.36	0.01\\
97.37	0.01\\
97.38	0.01\\
97.39	0.01\\
97.4	0.01\\
97.41	0.01\\
97.42	0.01\\
97.43	0.01\\
97.44	0.01\\
97.45	0.01\\
97.46	0.01\\
97.47	0.01\\
97.48	0.01\\
97.49	0.01\\
97.5	0.01\\
97.51	0.01\\
97.52	0.01\\
97.53	0.01\\
97.54	0.01\\
97.55	0.01\\
97.56	0.01\\
97.57	0.01\\
97.58	0.01\\
97.59	0.01\\
97.6	0.01\\
97.61	0.01\\
97.62	0.01\\
97.63	0.01\\
97.64	0.01\\
97.65	0.01\\
97.66	0.01\\
97.67	0.01\\
97.68	0.01\\
97.69	0.01\\
97.7	0.01\\
97.71	0.01\\
97.72	0.01\\
97.73	0.01\\
97.74	0.01\\
97.75	0.01\\
97.76	0.01\\
97.77	0.01\\
97.78	0.01\\
97.79	0.01\\
97.8	0.01\\
97.81	0.01\\
97.82	0.01\\
97.83	0.01\\
97.84	0.01\\
97.85	0.01\\
97.86	0.01\\
97.87	0.01\\
97.88	0.01\\
97.89	0.01\\
97.9	0.01\\
97.91	0.01\\
97.92	0.01\\
97.93	0.01\\
97.94	0.01\\
97.95	0.01\\
97.96	0.01\\
97.97	0.01\\
97.98	0.01\\
97.99	0.01\\
98	0.01\\
98.01	0.01\\
98.02	0.01\\
98.03	0.01\\
98.04	0.01\\
98.05	0.01\\
98.06	0.01\\
98.07	0.01\\
98.08	0.01\\
98.09	0.01\\
98.1	0.01\\
98.11	0.01\\
98.12	0.01\\
98.13	0.01\\
98.14	0.01\\
98.15	0.01\\
98.16	0.01\\
98.17	0.01\\
98.18	0.01\\
98.19	0.01\\
98.2	0.01\\
98.21	0.01\\
98.22	0.01\\
98.23	0.01\\
98.24	0.01\\
98.25	0.01\\
98.26	0.01\\
98.27	0.01\\
98.28	0.01\\
98.29	0.01\\
98.3	0.01\\
98.31	0.01\\
98.32	0.01\\
98.33	0.01\\
98.34	0.01\\
98.35	0.01\\
98.36	0.01\\
98.37	0.00994200189967468\\
98.38	0.00988149358642676\\
98.39	0.00982053227918336\\
98.4	0.00975911360852013\\
98.41	0.00969723315730262\\
98.42	0.00963488646006192\\
98.43	0.00957207823952274\\
98.44	0.00950880697546138\\
98.45	0.0094450682680312\\
98.46	0.00938085766971558\\
98.47	0.00931617068469812\\
98.48	0.00925100276822203\\
98.49	0.00918534932593885\\
98.5	0.00911920571324596\\
98.51	0.00905256723461284\\
98.52	0.00898542914289573\\
98.53	0.00891778663864044\\
98.54	0.00884963486937308\\
98.55	0.00878096892887842\\
98.56	0.00871178385646553\\
98.57	0.00864207463622052\\
98.58	0.00857183619624606\\
98.59	0.00850106340788731\\
98.6	0.00842975108494393\\
98.61	0.00835789398286802\\
98.62	0.00828548679794739\\
98.63	0.00821252416647407\\
98.64	0.0081390006638975\\
98.65	0.00806491080396212\\
98.66	0.00799024903783184\\
98.67	0.00791500975319625\\
98.68	0.0078391872733592\\
98.69	0.00779733688559829\\
98.7	0.00776614542144277\\
98.71	0.0077346872050461\\
98.72	0.00770295979320297\\
98.73	0.00767096071275983\\
98.74	0.00763868490384814\\
98.75	0.00760612775047667\\
98.76	0.00757328663550659\\
98.77	0.00754015891816841\\
98.78	0.00750674193386365\\
98.79	0.00747302550100462\\
98.8	0.00743900462524496\\
98.81	0.00740467639412652\\
98.82	0.00737003786772924\\
98.83	0.00733508607772653\\
98.84	0.00729981802727104\\
98.85	0.00726423069114149\\
98.86	0.00722832101548301\\
98.87	0.00719208591754541\\
98.88	0.00715552228541941\\
98.89	0.00711862697777067\\
98.9	0.00708139682357191\\
98.91	0.00704382862183277\\
98.92	0.00700591914132766\\
98.93	0.00696766512032152\\
98.94	0.00692906326629346\\
98.95	0.00689011025565822\\
98.96	0.00685080273348568\\
98.97	0.00681113731321816\\
98.98	0.00677111057638558\\
98.99	0.00673071907231868\\
99	0.00668995931785992\\
99.01	0.00664882779707248\\
99.02	0.00660732096094699\\
99.03	0.00656543522710637\\
99.04	0.00652316697950833\\
99.05	0.00648051256814606\\
99.06	0.00643746830874661\\
99.07	0.00639403048246739\\
99.08	0.00635019537614513\\
99.09	0.00630595927587717\\
99.1	0.00626131843376204\\
99.11	0.00621626906760943\\
99.12	0.00617080736138312\\
99.13	0.00612492946437186\\
99.14	0.00607863149071768\\
99.15	0.00603190951911964\\
99.16	0.00598475959253574\\
99.17	0.00593717771788343\\
99.18	0.00588915986573838\\
99.19	0.00584070197003187\\
99.2	0.00579179992774656\\
99.21	0.00574244959861087\\
99.22	0.00569264680479192\\
99.23	0.0056423873305871\\
99.24	0.00559166692211428\\
99.25	0.0055404812870008\\
99.26	0.00548882609407117\\
99.27	0.00543669697303366\\
99.28	0.0053840895141657\\
99.29	0.00533099926799834\\
99.3	0.0052774217449996\\
99.31	0.00522335241525704\\
99.32	0.00516878670815941\\
99.33	0.00511372001207753\\
99.34	0.00505814767404461\\
99.35	0.0050020649994359\\
99.36	0.0049454672516478\\
99.37	0.00488834965177681\\
99.38	0.00483070737829788\\
99.39	0.00477253556674291\\
99.4	0.00471382930937896\\
99.41	0.0046545836548867\\
99.42	0.00459479360803896\\
99.43	0.0045344541293797\\
99.44	0.00447356013490343\\
99.45	0.00441210649573532\\
99.46	0.00435008803781211\\
99.47	0.00428749954156401\\
99.48	0.00422433574159776\\
99.49	0.00416059132638106\\
99.5	0.00409626093792865\\
99.51	0.00403133917149\\
99.52	0.00396582057523918\\
99.53	0.00389969964996693\\
99.54	0.00383297084877514\\
99.55	0.00376562857677427\\
99.56	0.00369766719078361\\
99.57	0.00362908099903511\\
99.58	0.00355986426088066\\
99.59	0.00349001118650345\\
99.6	0.0034195159366337\\
99.61	0.00334837262222938\\
99.62	0.00327657530418136\\
99.63	0.00320411799304407\\
99.64	0.00313099464877195\\
99.65	0.00305719918046246\\
99.66	0.00298272544610582\\
99.67	0.00290756725234218\\
99.68	0.00283171835422667\\
99.69	0.00275517245500294\\
99.7	0.00267792320588561\\
99.71	0.00259996424989131\\
99.72	0.0025212891819963\\
99.73	0.0024418915455014\\
99.74	0.00236176483187052\\
99.75	0.00228090248058198\\
99.76	0.00219929787899328\\
99.77	0.00211694436222034\\
99.78	0.00203383521303203\\
99.79	0.00194996366176096\\
99.8	0.00186532288623169\\
99.81	0.00177990601170719\\
99.82	0.0016937061108549\\
99.83	0.00160671620373351\\
99.84	0.00151892925780174\\
99.85	0.00143033818795045\\
99.86	0.00134093585655955\\
99.87	0.00125071507358113\\
99.88	0.00115966859665052\\
99.89	0.00106778913122683\\
99.9	0.000975069330764865\\
99.91	0.000881501796920222\\
99.92	0.000787079079789596\\
99.93	0.00069179367818842\\
99.94	0.000595638039968045\\
99.95	0.000498604562374856\\
99.96	0.000400685592453825\\
99.97	0.000301873427499166\\
99.98	0.000202160315554915\\
99.99	0.000101538455968433\\
100	0\\
};
\addlegendentry{$q=2$};

\addplot [color=mycolor1,solid,forget plot]
  table[row sep=crcr]{%
0.01	0.01\\
0.02	0.01\\
0.03	0.01\\
0.04	0.01\\
0.05	0.01\\
0.06	0.01\\
0.07	0.01\\
0.08	0.01\\
0.09	0.01\\
0.1	0.01\\
0.11	0.01\\
0.12	0.01\\
0.13	0.01\\
0.14	0.01\\
0.15	0.01\\
0.16	0.01\\
0.17	0.01\\
0.18	0.01\\
0.19	0.01\\
0.2	0.01\\
0.21	0.01\\
0.22	0.01\\
0.23	0.01\\
0.24	0.01\\
0.25	0.01\\
0.26	0.01\\
0.27	0.01\\
0.28	0.01\\
0.29	0.01\\
0.3	0.01\\
0.31	0.01\\
0.32	0.01\\
0.33	0.01\\
0.34	0.01\\
0.35	0.01\\
0.36	0.01\\
0.37	0.01\\
0.38	0.01\\
0.39	0.01\\
0.4	0.01\\
0.41	0.01\\
0.42	0.01\\
0.43	0.01\\
0.44	0.01\\
0.45	0.01\\
0.46	0.01\\
0.47	0.01\\
0.48	0.01\\
0.49	0.01\\
0.5	0.01\\
0.51	0.01\\
0.52	0.01\\
0.53	0.01\\
0.54	0.01\\
0.55	0.01\\
0.56	0.01\\
0.57	0.01\\
0.58	0.01\\
0.59	0.01\\
0.6	0.01\\
0.61	0.01\\
0.62	0.01\\
0.63	0.01\\
0.64	0.01\\
0.65	0.01\\
0.66	0.01\\
0.67	0.01\\
0.68	0.01\\
0.69	0.01\\
0.7	0.01\\
0.71	0.01\\
0.72	0.01\\
0.73	0.01\\
0.74	0.01\\
0.75	0.01\\
0.76	0.01\\
0.77	0.01\\
0.78	0.01\\
0.79	0.01\\
0.8	0.01\\
0.81	0.01\\
0.82	0.01\\
0.83	0.01\\
0.84	0.01\\
0.85	0.01\\
0.86	0.01\\
0.87	0.01\\
0.88	0.01\\
0.89	0.01\\
0.9	0.01\\
0.91	0.01\\
0.92	0.01\\
0.93	0.01\\
0.94	0.01\\
0.95	0.01\\
0.96	0.01\\
0.97	0.01\\
0.98	0.01\\
0.99	0.01\\
1	0.01\\
1.01	0.01\\
1.02	0.01\\
1.03	0.01\\
1.04	0.01\\
1.05	0.01\\
1.06	0.01\\
1.07	0.01\\
1.08	0.01\\
1.09	0.01\\
1.1	0.01\\
1.11	0.01\\
1.12	0.01\\
1.13	0.01\\
1.14	0.01\\
1.15	0.01\\
1.16	0.01\\
1.17	0.01\\
1.18	0.01\\
1.19	0.01\\
1.2	0.01\\
1.21	0.01\\
1.22	0.01\\
1.23	0.01\\
1.24	0.01\\
1.25	0.01\\
1.26	0.01\\
1.27	0.01\\
1.28	0.01\\
1.29	0.01\\
1.3	0.01\\
1.31	0.01\\
1.32	0.01\\
1.33	0.01\\
1.34	0.01\\
1.35	0.01\\
1.36	0.01\\
1.37	0.01\\
1.38	0.01\\
1.39	0.01\\
1.4	0.01\\
1.41	0.01\\
1.42	0.01\\
1.43	0.01\\
1.44	0.01\\
1.45	0.01\\
1.46	0.01\\
1.47	0.01\\
1.48	0.01\\
1.49	0.01\\
1.5	0.01\\
1.51	0.01\\
1.52	0.01\\
1.53	0.01\\
1.54	0.01\\
1.55	0.01\\
1.56	0.01\\
1.57	0.01\\
1.58	0.01\\
1.59	0.01\\
1.6	0.01\\
1.61	0.01\\
1.62	0.01\\
1.63	0.01\\
1.64	0.01\\
1.65	0.01\\
1.66	0.01\\
1.67	0.01\\
1.68	0.01\\
1.69	0.01\\
1.7	0.01\\
1.71	0.01\\
1.72	0.01\\
1.73	0.01\\
1.74	0.01\\
1.75	0.01\\
1.76	0.01\\
1.77	0.01\\
1.78	0.01\\
1.79	0.01\\
1.8	0.01\\
1.81	0.01\\
1.82	0.01\\
1.83	0.01\\
1.84	0.01\\
1.85	0.01\\
1.86	0.01\\
1.87	0.01\\
1.88	0.01\\
1.89	0.01\\
1.9	0.01\\
1.91	0.01\\
1.92	0.01\\
1.93	0.01\\
1.94	0.01\\
1.95	0.01\\
1.96	0.01\\
1.97	0.01\\
1.98	0.01\\
1.99	0.01\\
2	0.01\\
2.01	0.01\\
2.02	0.01\\
2.03	0.01\\
2.04	0.01\\
2.05	0.01\\
2.06	0.01\\
2.07	0.01\\
2.08	0.01\\
2.09	0.01\\
2.1	0.01\\
2.11	0.01\\
2.12	0.01\\
2.13	0.01\\
2.14	0.01\\
2.15	0.01\\
2.16	0.01\\
2.17	0.01\\
2.18	0.01\\
2.19	0.01\\
2.2	0.01\\
2.21	0.01\\
2.22	0.01\\
2.23	0.01\\
2.24	0.01\\
2.25	0.01\\
2.26	0.01\\
2.27	0.01\\
2.28	0.01\\
2.29	0.01\\
2.3	0.01\\
2.31	0.01\\
2.32	0.01\\
2.33	0.01\\
2.34	0.01\\
2.35	0.01\\
2.36	0.01\\
2.37	0.01\\
2.38	0.01\\
2.39	0.01\\
2.4	0.01\\
2.41	0.01\\
2.42	0.01\\
2.43	0.01\\
2.44	0.01\\
2.45	0.01\\
2.46	0.01\\
2.47	0.01\\
2.48	0.01\\
2.49	0.01\\
2.5	0.01\\
2.51	0.01\\
2.52	0.01\\
2.53	0.01\\
2.54	0.01\\
2.55	0.01\\
2.56	0.01\\
2.57	0.01\\
2.58	0.01\\
2.59	0.01\\
2.6	0.01\\
2.61	0.01\\
2.62	0.01\\
2.63	0.01\\
2.64	0.01\\
2.65	0.01\\
2.66	0.01\\
2.67	0.01\\
2.68	0.01\\
2.69	0.01\\
2.7	0.01\\
2.71	0.01\\
2.72	0.01\\
2.73	0.01\\
2.74	0.01\\
2.75	0.01\\
2.76	0.01\\
2.77	0.01\\
2.78	0.01\\
2.79	0.01\\
2.8	0.01\\
2.81	0.01\\
2.82	0.01\\
2.83	0.01\\
2.84	0.01\\
2.85	0.01\\
2.86	0.01\\
2.87	0.01\\
2.88	0.01\\
2.89	0.01\\
2.9	0.01\\
2.91	0.01\\
2.92	0.01\\
2.93	0.01\\
2.94	0.01\\
2.95	0.01\\
2.96	0.01\\
2.97	0.01\\
2.98	0.01\\
2.99	0.01\\
3	0.01\\
3.01	0.01\\
3.02	0.01\\
3.03	0.01\\
3.04	0.01\\
3.05	0.01\\
3.06	0.01\\
3.07	0.01\\
3.08	0.01\\
3.09	0.01\\
3.1	0.01\\
3.11	0.01\\
3.12	0.01\\
3.13	0.01\\
3.14	0.01\\
3.15	0.01\\
3.16	0.01\\
3.17	0.01\\
3.18	0.01\\
3.19	0.01\\
3.2	0.01\\
3.21	0.01\\
3.22	0.01\\
3.23	0.01\\
3.24	0.01\\
3.25	0.01\\
3.26	0.01\\
3.27	0.01\\
3.28	0.01\\
3.29	0.01\\
3.3	0.01\\
3.31	0.01\\
3.32	0.01\\
3.33	0.01\\
3.34	0.01\\
3.35	0.01\\
3.36	0.01\\
3.37	0.01\\
3.38	0.01\\
3.39	0.01\\
3.4	0.01\\
3.41	0.01\\
3.42	0.01\\
3.43	0.01\\
3.44	0.01\\
3.45	0.01\\
3.46	0.01\\
3.47	0.01\\
3.48	0.01\\
3.49	0.01\\
3.5	0.01\\
3.51	0.01\\
3.52	0.01\\
3.53	0.01\\
3.54	0.01\\
3.55	0.01\\
3.56	0.01\\
3.57	0.01\\
3.58	0.01\\
3.59	0.01\\
3.6	0.01\\
3.61	0.01\\
3.62	0.01\\
3.63	0.01\\
3.64	0.01\\
3.65	0.01\\
3.66	0.01\\
3.67	0.01\\
3.68	0.01\\
3.69	0.01\\
3.7	0.01\\
3.71	0.01\\
3.72	0.01\\
3.73	0.01\\
3.74	0.01\\
3.75	0.01\\
3.76	0.01\\
3.77	0.01\\
3.78	0.01\\
3.79	0.01\\
3.8	0.01\\
3.81	0.01\\
3.82	0.01\\
3.83	0.01\\
3.84	0.01\\
3.85	0.01\\
3.86	0.01\\
3.87	0.01\\
3.88	0.01\\
3.89	0.01\\
3.9	0.01\\
3.91	0.01\\
3.92	0.01\\
3.93	0.01\\
3.94	0.01\\
3.95	0.01\\
3.96	0.01\\
3.97	0.01\\
3.98	0.01\\
3.99	0.01\\
4	0.01\\
4.01	0.01\\
4.02	0.01\\
4.03	0.01\\
4.04	0.01\\
4.05	0.01\\
4.06	0.01\\
4.07	0.01\\
4.08	0.01\\
4.09	0.01\\
4.1	0.01\\
4.11	0.01\\
4.12	0.01\\
4.13	0.01\\
4.14	0.01\\
4.15	0.01\\
4.16	0.01\\
4.17	0.01\\
4.18	0.01\\
4.19	0.01\\
4.2	0.01\\
4.21	0.01\\
4.22	0.01\\
4.23	0.01\\
4.24	0.01\\
4.25	0.01\\
4.26	0.01\\
4.27	0.01\\
4.28	0.01\\
4.29	0.01\\
4.3	0.01\\
4.31	0.01\\
4.32	0.01\\
4.33	0.01\\
4.34	0.01\\
4.35	0.01\\
4.36	0.01\\
4.37	0.01\\
4.38	0.01\\
4.39	0.01\\
4.4	0.01\\
4.41	0.01\\
4.42	0.01\\
4.43	0.01\\
4.44	0.01\\
4.45	0.01\\
4.46	0.01\\
4.47	0.01\\
4.48	0.01\\
4.49	0.01\\
4.5	0.01\\
4.51	0.01\\
4.52	0.01\\
4.53	0.01\\
4.54	0.01\\
4.55	0.01\\
4.56	0.01\\
4.57	0.01\\
4.58	0.01\\
4.59	0.01\\
4.6	0.01\\
4.61	0.01\\
4.62	0.01\\
4.63	0.01\\
4.64	0.01\\
4.65	0.01\\
4.66	0.01\\
4.67	0.01\\
4.68	0.01\\
4.69	0.01\\
4.7	0.01\\
4.71	0.01\\
4.72	0.01\\
4.73	0.01\\
4.74	0.01\\
4.75	0.01\\
4.76	0.01\\
4.77	0.01\\
4.78	0.01\\
4.79	0.01\\
4.8	0.01\\
4.81	0.01\\
4.82	0.01\\
4.83	0.01\\
4.84	0.01\\
4.85	0.01\\
4.86	0.01\\
4.87	0.01\\
4.88	0.01\\
4.89	0.01\\
4.9	0.01\\
4.91	0.01\\
4.92	0.01\\
4.93	0.01\\
4.94	0.01\\
4.95	0.01\\
4.96	0.01\\
4.97	0.01\\
4.98	0.01\\
4.99	0.01\\
5	0.01\\
5.01	0.01\\
5.02	0.01\\
5.03	0.01\\
5.04	0.01\\
5.05	0.01\\
5.06	0.01\\
5.07	0.01\\
5.08	0.01\\
5.09	0.01\\
5.1	0.01\\
5.11	0.01\\
5.12	0.01\\
5.13	0.01\\
5.14	0.01\\
5.15	0.01\\
5.16	0.01\\
5.17	0.01\\
5.18	0.01\\
5.19	0.01\\
5.2	0.01\\
5.21	0.01\\
5.22	0.01\\
5.23	0.01\\
5.24	0.01\\
5.25	0.01\\
5.26	0.01\\
5.27	0.01\\
5.28	0.01\\
5.29	0.01\\
5.3	0.01\\
5.31	0.01\\
5.32	0.01\\
5.33	0.01\\
5.34	0.01\\
5.35	0.01\\
5.36	0.01\\
5.37	0.01\\
5.38	0.01\\
5.39	0.01\\
5.4	0.01\\
5.41	0.01\\
5.42	0.01\\
5.43	0.01\\
5.44	0.01\\
5.45	0.01\\
5.46	0.01\\
5.47	0.01\\
5.48	0.01\\
5.49	0.01\\
5.5	0.01\\
5.51	0.01\\
5.52	0.01\\
5.53	0.01\\
5.54	0.01\\
5.55	0.01\\
5.56	0.01\\
5.57	0.01\\
5.58	0.01\\
5.59	0.01\\
5.6	0.01\\
5.61	0.01\\
5.62	0.01\\
5.63	0.01\\
5.64	0.01\\
5.65	0.01\\
5.66	0.01\\
5.67	0.01\\
5.68	0.01\\
5.69	0.01\\
5.7	0.01\\
5.71	0.01\\
5.72	0.01\\
5.73	0.01\\
5.74	0.01\\
5.75	0.01\\
5.76	0.01\\
5.77	0.01\\
5.78	0.01\\
5.79	0.01\\
5.8	0.01\\
5.81	0.01\\
5.82	0.01\\
5.83	0.01\\
5.84	0.01\\
5.85	0.01\\
5.86	0.01\\
5.87	0.01\\
5.88	0.01\\
5.89	0.01\\
5.9	0.01\\
5.91	0.01\\
5.92	0.01\\
5.93	0.01\\
5.94	0.01\\
5.95	0.01\\
5.96	0.01\\
5.97	0.01\\
5.98	0.01\\
5.99	0.01\\
6	0.01\\
6.01	0.01\\
6.02	0.01\\
6.03	0.01\\
6.04	0.01\\
6.05	0.01\\
6.06	0.01\\
6.07	0.01\\
6.08	0.01\\
6.09	0.01\\
6.1	0.01\\
6.11	0.01\\
6.12	0.01\\
6.13	0.01\\
6.14	0.01\\
6.15	0.01\\
6.16	0.01\\
6.17	0.01\\
6.18	0.01\\
6.19	0.01\\
6.2	0.01\\
6.21	0.01\\
6.22	0.01\\
6.23	0.01\\
6.24	0.01\\
6.25	0.01\\
6.26	0.01\\
6.27	0.01\\
6.28	0.01\\
6.29	0.01\\
6.3	0.01\\
6.31	0.01\\
6.32	0.01\\
6.33	0.01\\
6.34	0.01\\
6.35	0.01\\
6.36	0.01\\
6.37	0.01\\
6.38	0.01\\
6.39	0.01\\
6.4	0.01\\
6.41	0.01\\
6.42	0.01\\
6.43	0.01\\
6.44	0.01\\
6.45	0.01\\
6.46	0.01\\
6.47	0.01\\
6.48	0.01\\
6.49	0.01\\
6.5	0.01\\
6.51	0.01\\
6.52	0.01\\
6.53	0.01\\
6.54	0.01\\
6.55	0.01\\
6.56	0.01\\
6.57	0.01\\
6.58	0.01\\
6.59	0.01\\
6.6	0.01\\
6.61	0.01\\
6.62	0.01\\
6.63	0.01\\
6.64	0.01\\
6.65	0.01\\
6.66	0.01\\
6.67	0.01\\
6.68	0.01\\
6.69	0.01\\
6.7	0.01\\
6.71	0.01\\
6.72	0.01\\
6.73	0.01\\
6.74	0.01\\
6.75	0.01\\
6.76	0.01\\
6.77	0.01\\
6.78	0.01\\
6.79	0.01\\
6.8	0.01\\
6.81	0.01\\
6.82	0.01\\
6.83	0.01\\
6.84	0.01\\
6.85	0.01\\
6.86	0.01\\
6.87	0.01\\
6.88	0.01\\
6.89	0.01\\
6.9	0.01\\
6.91	0.01\\
6.92	0.01\\
6.93	0.01\\
6.94	0.01\\
6.95	0.01\\
6.96	0.01\\
6.97	0.01\\
6.98	0.01\\
6.99	0.01\\
7	0.01\\
7.01	0.01\\
7.02	0.01\\
7.03	0.01\\
7.04	0.01\\
7.05	0.01\\
7.06	0.01\\
7.07	0.01\\
7.08	0.01\\
7.09	0.01\\
7.1	0.01\\
7.11	0.01\\
7.12	0.01\\
7.13	0.01\\
7.14	0.01\\
7.15	0.01\\
7.16	0.01\\
7.17	0.01\\
7.18	0.01\\
7.19	0.01\\
7.2	0.01\\
7.21	0.01\\
7.22	0.01\\
7.23	0.01\\
7.24	0.01\\
7.25	0.01\\
7.26	0.01\\
7.27	0.01\\
7.28	0.01\\
7.29	0.01\\
7.3	0.01\\
7.31	0.01\\
7.32	0.01\\
7.33	0.01\\
7.34	0.01\\
7.35	0.01\\
7.36	0.01\\
7.37	0.01\\
7.38	0.01\\
7.39	0.01\\
7.4	0.01\\
7.41	0.01\\
7.42	0.01\\
7.43	0.01\\
7.44	0.01\\
7.45	0.01\\
7.46	0.01\\
7.47	0.01\\
7.48	0.01\\
7.49	0.01\\
7.5	0.01\\
7.51	0.01\\
7.52	0.01\\
7.53	0.01\\
7.54	0.01\\
7.55	0.01\\
7.56	0.01\\
7.57	0.01\\
7.58	0.01\\
7.59	0.01\\
7.6	0.01\\
7.61	0.01\\
7.62	0.01\\
7.63	0.01\\
7.64	0.01\\
7.65	0.01\\
7.66	0.01\\
7.67	0.01\\
7.68	0.01\\
7.69	0.01\\
7.7	0.01\\
7.71	0.01\\
7.72	0.01\\
7.73	0.01\\
7.74	0.01\\
7.75	0.01\\
7.76	0.01\\
7.77	0.01\\
7.78	0.01\\
7.79	0.01\\
7.8	0.01\\
7.81	0.01\\
7.82	0.01\\
7.83	0.01\\
7.84	0.01\\
7.85	0.01\\
7.86	0.01\\
7.87	0.01\\
7.88	0.01\\
7.89	0.01\\
7.9	0.01\\
7.91	0.01\\
7.92	0.01\\
7.93	0.01\\
7.94	0.01\\
7.95	0.01\\
7.96	0.01\\
7.97	0.01\\
7.98	0.01\\
7.99	0.01\\
8	0.01\\
8.01	0.01\\
8.02	0.01\\
8.03	0.01\\
8.04	0.01\\
8.05	0.01\\
8.06	0.01\\
8.07	0.01\\
8.08	0.01\\
8.09	0.01\\
8.1	0.01\\
8.11	0.01\\
8.12	0.01\\
8.13	0.01\\
8.14	0.01\\
8.15	0.01\\
8.16	0.01\\
8.17	0.01\\
8.18	0.01\\
8.19	0.01\\
8.2	0.01\\
8.21	0.01\\
8.22	0.01\\
8.23	0.01\\
8.24	0.01\\
8.25	0.01\\
8.26	0.01\\
8.27	0.01\\
8.28	0.01\\
8.29	0.01\\
8.3	0.01\\
8.31	0.01\\
8.32	0.01\\
8.33	0.01\\
8.34	0.01\\
8.35	0.01\\
8.36	0.01\\
8.37	0.01\\
8.38	0.01\\
8.39	0.01\\
8.4	0.01\\
8.41	0.01\\
8.42	0.01\\
8.43	0.01\\
8.44	0.01\\
8.45	0.01\\
8.46	0.01\\
8.47	0.01\\
8.48	0.01\\
8.49	0.01\\
8.5	0.01\\
8.51	0.01\\
8.52	0.01\\
8.53	0.01\\
8.54	0.01\\
8.55	0.01\\
8.56	0.01\\
8.57	0.01\\
8.58	0.01\\
8.59	0.01\\
8.6	0.01\\
8.61	0.01\\
8.62	0.01\\
8.63	0.01\\
8.64	0.01\\
8.65	0.01\\
8.66	0.01\\
8.67	0.01\\
8.68	0.01\\
8.69	0.01\\
8.7	0.01\\
8.71	0.01\\
8.72	0.01\\
8.73	0.01\\
8.74	0.01\\
8.75	0.01\\
8.76	0.01\\
8.77	0.01\\
8.78	0.01\\
8.79	0.01\\
8.8	0.01\\
8.81	0.01\\
8.82	0.01\\
8.83	0.01\\
8.84	0.01\\
8.85	0.01\\
8.86	0.01\\
8.87	0.01\\
8.88	0.01\\
8.89	0.01\\
8.9	0.01\\
8.91	0.01\\
8.92	0.01\\
8.93	0.01\\
8.94	0.01\\
8.95	0.01\\
8.96	0.01\\
8.97	0.01\\
8.98	0.01\\
8.99	0.01\\
9	0.01\\
9.01	0.01\\
9.02	0.01\\
9.03	0.01\\
9.04	0.01\\
9.05	0.01\\
9.06	0.01\\
9.07	0.01\\
9.08	0.01\\
9.09	0.01\\
9.1	0.01\\
9.11	0.01\\
9.12	0.01\\
9.13	0.01\\
9.14	0.01\\
9.15	0.01\\
9.16	0.01\\
9.17	0.01\\
9.18	0.01\\
9.19	0.01\\
9.2	0.01\\
9.21	0.01\\
9.22	0.01\\
9.23	0.01\\
9.24	0.01\\
9.25	0.01\\
9.26	0.01\\
9.27	0.01\\
9.28	0.01\\
9.29	0.01\\
9.3	0.01\\
9.31	0.01\\
9.32	0.01\\
9.33	0.01\\
9.34	0.01\\
9.35	0.01\\
9.36	0.01\\
9.37	0.01\\
9.38	0.01\\
9.39	0.01\\
9.4	0.01\\
9.41	0.01\\
9.42	0.01\\
9.43	0.01\\
9.44	0.01\\
9.45	0.01\\
9.46	0.01\\
9.47	0.01\\
9.48	0.01\\
9.49	0.01\\
9.5	0.01\\
9.51	0.01\\
9.52	0.01\\
9.53	0.01\\
9.54	0.01\\
9.55	0.01\\
9.56	0.01\\
9.57	0.01\\
9.58	0.01\\
9.59	0.01\\
9.6	0.01\\
9.61	0.01\\
9.62	0.01\\
9.63	0.01\\
9.64	0.01\\
9.65	0.01\\
9.66	0.01\\
9.67	0.01\\
9.68	0.01\\
9.69	0.01\\
9.7	0.01\\
9.71	0.01\\
9.72	0.01\\
9.73	0.01\\
9.74	0.01\\
9.75	0.01\\
9.76	0.01\\
9.77	0.01\\
9.78	0.01\\
9.79	0.01\\
9.8	0.01\\
9.81	0.01\\
9.82	0.01\\
9.83	0.01\\
9.84	0.01\\
9.85	0.01\\
9.86	0.01\\
9.87	0.01\\
9.88	0.01\\
9.89	0.01\\
9.9	0.01\\
9.91	0.01\\
9.92	0.01\\
9.93	0.01\\
9.94	0.01\\
9.95	0.01\\
9.96	0.01\\
9.97	0.01\\
9.98	0.01\\
9.99	0.01\\
10	0.01\\
10.01	0.01\\
10.02	0.01\\
10.03	0.01\\
10.04	0.01\\
10.05	0.01\\
10.06	0.01\\
10.07	0.01\\
10.08	0.01\\
10.09	0.01\\
10.1	0.01\\
10.11	0.01\\
10.12	0.01\\
10.13	0.01\\
10.14	0.01\\
10.15	0.01\\
10.16	0.01\\
10.17	0.01\\
10.18	0.01\\
10.19	0.01\\
10.2	0.01\\
10.21	0.01\\
10.22	0.01\\
10.23	0.01\\
10.24	0.01\\
10.25	0.01\\
10.26	0.01\\
10.27	0.01\\
10.28	0.01\\
10.29	0.01\\
10.3	0.01\\
10.31	0.01\\
10.32	0.01\\
10.33	0.01\\
10.34	0.01\\
10.35	0.01\\
10.36	0.01\\
10.37	0.01\\
10.38	0.01\\
10.39	0.01\\
10.4	0.01\\
10.41	0.01\\
10.42	0.01\\
10.43	0.01\\
10.44	0.01\\
10.45	0.01\\
10.46	0.01\\
10.47	0.01\\
10.48	0.01\\
10.49	0.01\\
10.5	0.01\\
10.51	0.01\\
10.52	0.01\\
10.53	0.01\\
10.54	0.01\\
10.55	0.01\\
10.56	0.01\\
10.57	0.01\\
10.58	0.01\\
10.59	0.01\\
10.6	0.01\\
10.61	0.01\\
10.62	0.01\\
10.63	0.01\\
10.64	0.01\\
10.65	0.01\\
10.66	0.01\\
10.67	0.01\\
10.68	0.01\\
10.69	0.01\\
10.7	0.01\\
10.71	0.01\\
10.72	0.01\\
10.73	0.01\\
10.74	0.01\\
10.75	0.01\\
10.76	0.01\\
10.77	0.01\\
10.78	0.01\\
10.79	0.01\\
10.8	0.01\\
10.81	0.01\\
10.82	0.01\\
10.83	0.01\\
10.84	0.01\\
10.85	0.01\\
10.86	0.01\\
10.87	0.01\\
10.88	0.01\\
10.89	0.01\\
10.9	0.01\\
10.91	0.01\\
10.92	0.01\\
10.93	0.01\\
10.94	0.01\\
10.95	0.01\\
10.96	0.01\\
10.97	0.01\\
10.98	0.01\\
10.99	0.01\\
11	0.01\\
11.01	0.01\\
11.02	0.01\\
11.03	0.01\\
11.04	0.01\\
11.05	0.01\\
11.06	0.01\\
11.07	0.01\\
11.08	0.01\\
11.09	0.01\\
11.1	0.01\\
11.11	0.01\\
11.12	0.01\\
11.13	0.01\\
11.14	0.01\\
11.15	0.01\\
11.16	0.01\\
11.17	0.01\\
11.18	0.01\\
11.19	0.01\\
11.2	0.01\\
11.21	0.01\\
11.22	0.01\\
11.23	0.01\\
11.24	0.01\\
11.25	0.01\\
11.26	0.01\\
11.27	0.01\\
11.28	0.01\\
11.29	0.01\\
11.3	0.01\\
11.31	0.01\\
11.32	0.01\\
11.33	0.01\\
11.34	0.01\\
11.35	0.01\\
11.36	0.01\\
11.37	0.01\\
11.38	0.01\\
11.39	0.01\\
11.4	0.01\\
11.41	0.01\\
11.42	0.01\\
11.43	0.01\\
11.44	0.01\\
11.45	0.01\\
11.46	0.01\\
11.47	0.01\\
11.48	0.01\\
11.49	0.01\\
11.5	0.01\\
11.51	0.01\\
11.52	0.01\\
11.53	0.01\\
11.54	0.01\\
11.55	0.01\\
11.56	0.01\\
11.57	0.01\\
11.58	0.01\\
11.59	0.01\\
11.6	0.01\\
11.61	0.01\\
11.62	0.01\\
11.63	0.01\\
11.64	0.01\\
11.65	0.01\\
11.66	0.01\\
11.67	0.01\\
11.68	0.01\\
11.69	0.01\\
11.7	0.01\\
11.71	0.01\\
11.72	0.01\\
11.73	0.01\\
11.74	0.01\\
11.75	0.01\\
11.76	0.01\\
11.77	0.01\\
11.78	0.01\\
11.79	0.01\\
11.8	0.01\\
11.81	0.01\\
11.82	0.01\\
11.83	0.01\\
11.84	0.01\\
11.85	0.01\\
11.86	0.01\\
11.87	0.01\\
11.88	0.01\\
11.89	0.01\\
11.9	0.01\\
11.91	0.01\\
11.92	0.01\\
11.93	0.01\\
11.94	0.01\\
11.95	0.01\\
11.96	0.01\\
11.97	0.01\\
11.98	0.01\\
11.99	0.01\\
12	0.01\\
12.01	0.01\\
12.02	0.01\\
12.03	0.01\\
12.04	0.01\\
12.05	0.01\\
12.06	0.01\\
12.07	0.01\\
12.08	0.01\\
12.09	0.01\\
12.1	0.01\\
12.11	0.01\\
12.12	0.01\\
12.13	0.01\\
12.14	0.01\\
12.15	0.01\\
12.16	0.01\\
12.17	0.01\\
12.18	0.01\\
12.19	0.01\\
12.2	0.01\\
12.21	0.01\\
12.22	0.01\\
12.23	0.01\\
12.24	0.01\\
12.25	0.01\\
12.26	0.01\\
12.27	0.01\\
12.28	0.01\\
12.29	0.01\\
12.3	0.01\\
12.31	0.01\\
12.32	0.01\\
12.33	0.01\\
12.34	0.01\\
12.35	0.01\\
12.36	0.01\\
12.37	0.01\\
12.38	0.01\\
12.39	0.01\\
12.4	0.01\\
12.41	0.01\\
12.42	0.01\\
12.43	0.01\\
12.44	0.01\\
12.45	0.01\\
12.46	0.01\\
12.47	0.01\\
12.48	0.01\\
12.49	0.01\\
12.5	0.01\\
12.51	0.01\\
12.52	0.01\\
12.53	0.01\\
12.54	0.01\\
12.55	0.01\\
12.56	0.01\\
12.57	0.01\\
12.58	0.01\\
12.59	0.01\\
12.6	0.01\\
12.61	0.01\\
12.62	0.01\\
12.63	0.01\\
12.64	0.01\\
12.65	0.01\\
12.66	0.01\\
12.67	0.01\\
12.68	0.01\\
12.69	0.01\\
12.7	0.01\\
12.71	0.01\\
12.72	0.01\\
12.73	0.01\\
12.74	0.01\\
12.75	0.01\\
12.76	0.01\\
12.77	0.01\\
12.78	0.01\\
12.79	0.01\\
12.8	0.01\\
12.81	0.01\\
12.82	0.01\\
12.83	0.01\\
12.84	0.01\\
12.85	0.01\\
12.86	0.01\\
12.87	0.01\\
12.88	0.01\\
12.89	0.01\\
12.9	0.01\\
12.91	0.01\\
12.92	0.01\\
12.93	0.01\\
12.94	0.01\\
12.95	0.01\\
12.96	0.01\\
12.97	0.01\\
12.98	0.01\\
12.99	0.01\\
13	0.01\\
13.01	0.01\\
13.02	0.01\\
13.03	0.01\\
13.04	0.01\\
13.05	0.01\\
13.06	0.01\\
13.07	0.01\\
13.08	0.01\\
13.09	0.01\\
13.1	0.01\\
13.11	0.01\\
13.12	0.01\\
13.13	0.01\\
13.14	0.01\\
13.15	0.01\\
13.16	0.01\\
13.17	0.01\\
13.18	0.01\\
13.19	0.01\\
13.2	0.01\\
13.21	0.01\\
13.22	0.01\\
13.23	0.01\\
13.24	0.01\\
13.25	0.01\\
13.26	0.01\\
13.27	0.01\\
13.28	0.01\\
13.29	0.01\\
13.3	0.01\\
13.31	0.01\\
13.32	0.01\\
13.33	0.01\\
13.34	0.01\\
13.35	0.01\\
13.36	0.01\\
13.37	0.01\\
13.38	0.01\\
13.39	0.01\\
13.4	0.01\\
13.41	0.01\\
13.42	0.01\\
13.43	0.01\\
13.44	0.01\\
13.45	0.01\\
13.46	0.01\\
13.47	0.01\\
13.48	0.01\\
13.49	0.01\\
13.5	0.01\\
13.51	0.01\\
13.52	0.01\\
13.53	0.01\\
13.54	0.01\\
13.55	0.01\\
13.56	0.01\\
13.57	0.01\\
13.58	0.01\\
13.59	0.01\\
13.6	0.01\\
13.61	0.01\\
13.62	0.01\\
13.63	0.01\\
13.64	0.01\\
13.65	0.01\\
13.66	0.01\\
13.67	0.01\\
13.68	0.01\\
13.69	0.01\\
13.7	0.01\\
13.71	0.01\\
13.72	0.01\\
13.73	0.01\\
13.74	0.01\\
13.75	0.01\\
13.76	0.01\\
13.77	0.01\\
13.78	0.01\\
13.79	0.01\\
13.8	0.01\\
13.81	0.01\\
13.82	0.01\\
13.83	0.01\\
13.84	0.01\\
13.85	0.01\\
13.86	0.01\\
13.87	0.01\\
13.88	0.01\\
13.89	0.01\\
13.9	0.01\\
13.91	0.01\\
13.92	0.01\\
13.93	0.01\\
13.94	0.01\\
13.95	0.01\\
13.96	0.01\\
13.97	0.01\\
13.98	0.01\\
13.99	0.01\\
14	0.01\\
14.01	0.01\\
14.02	0.01\\
14.03	0.01\\
14.04	0.01\\
14.05	0.01\\
14.06	0.01\\
14.07	0.01\\
14.08	0.01\\
14.09	0.01\\
14.1	0.01\\
14.11	0.01\\
14.12	0.01\\
14.13	0.01\\
14.14	0.01\\
14.15	0.01\\
14.16	0.01\\
14.17	0.01\\
14.18	0.01\\
14.19	0.01\\
14.2	0.01\\
14.21	0.01\\
14.22	0.01\\
14.23	0.01\\
14.24	0.01\\
14.25	0.01\\
14.26	0.01\\
14.27	0.01\\
14.28	0.01\\
14.29	0.01\\
14.3	0.01\\
14.31	0.01\\
14.32	0.01\\
14.33	0.01\\
14.34	0.01\\
14.35	0.01\\
14.36	0.01\\
14.37	0.01\\
14.38	0.01\\
14.39	0.01\\
14.4	0.01\\
14.41	0.01\\
14.42	0.01\\
14.43	0.01\\
14.44	0.01\\
14.45	0.01\\
14.46	0.01\\
14.47	0.01\\
14.48	0.01\\
14.49	0.01\\
14.5	0.01\\
14.51	0.01\\
14.52	0.01\\
14.53	0.01\\
14.54	0.01\\
14.55	0.01\\
14.56	0.01\\
14.57	0.01\\
14.58	0.01\\
14.59	0.01\\
14.6	0.01\\
14.61	0.01\\
14.62	0.01\\
14.63	0.01\\
14.64	0.01\\
14.65	0.01\\
14.66	0.01\\
14.67	0.01\\
14.68	0.01\\
14.69	0.01\\
14.7	0.01\\
14.71	0.01\\
14.72	0.01\\
14.73	0.01\\
14.74	0.01\\
14.75	0.01\\
14.76	0.01\\
14.77	0.01\\
14.78	0.01\\
14.79	0.01\\
14.8	0.01\\
14.81	0.01\\
14.82	0.01\\
14.83	0.01\\
14.84	0.01\\
14.85	0.01\\
14.86	0.01\\
14.87	0.01\\
14.88	0.01\\
14.89	0.01\\
14.9	0.01\\
14.91	0.01\\
14.92	0.01\\
14.93	0.01\\
14.94	0.01\\
14.95	0.01\\
14.96	0.01\\
14.97	0.01\\
14.98	0.01\\
14.99	0.01\\
15	0.01\\
15.01	0.01\\
15.02	0.01\\
15.03	0.01\\
15.04	0.01\\
15.05	0.01\\
15.06	0.01\\
15.07	0.01\\
15.08	0.01\\
15.09	0.01\\
15.1	0.01\\
15.11	0.01\\
15.12	0.01\\
15.13	0.01\\
15.14	0.01\\
15.15	0.01\\
15.16	0.01\\
15.17	0.01\\
15.18	0.01\\
15.19	0.01\\
15.2	0.01\\
15.21	0.01\\
15.22	0.01\\
15.23	0.01\\
15.24	0.01\\
15.25	0.01\\
15.26	0.01\\
15.27	0.01\\
15.28	0.01\\
15.29	0.01\\
15.3	0.01\\
15.31	0.01\\
15.32	0.01\\
15.33	0.01\\
15.34	0.01\\
15.35	0.01\\
15.36	0.01\\
15.37	0.01\\
15.38	0.01\\
15.39	0.01\\
15.4	0.01\\
15.41	0.01\\
15.42	0.01\\
15.43	0.01\\
15.44	0.01\\
15.45	0.01\\
15.46	0.01\\
15.47	0.01\\
15.48	0.01\\
15.49	0.01\\
15.5	0.01\\
15.51	0.01\\
15.52	0.01\\
15.53	0.01\\
15.54	0.01\\
15.55	0.01\\
15.56	0.01\\
15.57	0.01\\
15.58	0.01\\
15.59	0.01\\
15.6	0.01\\
15.61	0.01\\
15.62	0.01\\
15.63	0.01\\
15.64	0.01\\
15.65	0.01\\
15.66	0.01\\
15.67	0.01\\
15.68	0.01\\
15.69	0.01\\
15.7	0.01\\
15.71	0.01\\
15.72	0.01\\
15.73	0.01\\
15.74	0.01\\
15.75	0.01\\
15.76	0.01\\
15.77	0.01\\
15.78	0.01\\
15.79	0.01\\
15.8	0.01\\
15.81	0.01\\
15.82	0.01\\
15.83	0.01\\
15.84	0.01\\
15.85	0.01\\
15.86	0.01\\
15.87	0.01\\
15.88	0.01\\
15.89	0.01\\
15.9	0.01\\
15.91	0.01\\
15.92	0.01\\
15.93	0.01\\
15.94	0.01\\
15.95	0.01\\
15.96	0.01\\
15.97	0.01\\
15.98	0.01\\
15.99	0.01\\
16	0.01\\
16.01	0.01\\
16.02	0.01\\
16.03	0.01\\
16.04	0.01\\
16.05	0.01\\
16.06	0.01\\
16.07	0.01\\
16.08	0.01\\
16.09	0.01\\
16.1	0.01\\
16.11	0.01\\
16.12	0.01\\
16.13	0.01\\
16.14	0.01\\
16.15	0.01\\
16.16	0.01\\
16.17	0.01\\
16.18	0.01\\
16.19	0.01\\
16.2	0.01\\
16.21	0.01\\
16.22	0.01\\
16.23	0.01\\
16.24	0.01\\
16.25	0.01\\
16.26	0.01\\
16.27	0.01\\
16.28	0.01\\
16.29	0.01\\
16.3	0.01\\
16.31	0.01\\
16.32	0.01\\
16.33	0.01\\
16.34	0.01\\
16.35	0.01\\
16.36	0.01\\
16.37	0.01\\
16.38	0.01\\
16.39	0.01\\
16.4	0.01\\
16.41	0.01\\
16.42	0.01\\
16.43	0.01\\
16.44	0.01\\
16.45	0.01\\
16.46	0.01\\
16.47	0.01\\
16.48	0.01\\
16.49	0.01\\
16.5	0.01\\
16.51	0.01\\
16.52	0.01\\
16.53	0.01\\
16.54	0.01\\
16.55	0.01\\
16.56	0.01\\
16.57	0.01\\
16.58	0.01\\
16.59	0.01\\
16.6	0.01\\
16.61	0.01\\
16.62	0.01\\
16.63	0.01\\
16.64	0.01\\
16.65	0.01\\
16.66	0.01\\
16.67	0.01\\
16.68	0.01\\
16.69	0.01\\
16.7	0.01\\
16.71	0.01\\
16.72	0.01\\
16.73	0.01\\
16.74	0.01\\
16.75	0.01\\
16.76	0.01\\
16.77	0.01\\
16.78	0.01\\
16.79	0.01\\
16.8	0.01\\
16.81	0.01\\
16.82	0.01\\
16.83	0.01\\
16.84	0.01\\
16.85	0.01\\
16.86	0.01\\
16.87	0.01\\
16.88	0.01\\
16.89	0.01\\
16.9	0.01\\
16.91	0.01\\
16.92	0.01\\
16.93	0.01\\
16.94	0.01\\
16.95	0.01\\
16.96	0.01\\
16.97	0.01\\
16.98	0.01\\
16.99	0.01\\
17	0.01\\
17.01	0.01\\
17.02	0.01\\
17.03	0.01\\
17.04	0.01\\
17.05	0.01\\
17.06	0.01\\
17.07	0.01\\
17.08	0.01\\
17.09	0.01\\
17.1	0.01\\
17.11	0.01\\
17.12	0.01\\
17.13	0.01\\
17.14	0.01\\
17.15	0.01\\
17.16	0.01\\
17.17	0.01\\
17.18	0.01\\
17.19	0.01\\
17.2	0.01\\
17.21	0.01\\
17.22	0.01\\
17.23	0.01\\
17.24	0.01\\
17.25	0.01\\
17.26	0.01\\
17.27	0.01\\
17.28	0.01\\
17.29	0.01\\
17.3	0.01\\
17.31	0.01\\
17.32	0.01\\
17.33	0.01\\
17.34	0.01\\
17.35	0.01\\
17.36	0.01\\
17.37	0.01\\
17.38	0.01\\
17.39	0.01\\
17.4	0.01\\
17.41	0.01\\
17.42	0.01\\
17.43	0.01\\
17.44	0.01\\
17.45	0.01\\
17.46	0.01\\
17.47	0.01\\
17.48	0.01\\
17.49	0.01\\
17.5	0.01\\
17.51	0.01\\
17.52	0.01\\
17.53	0.01\\
17.54	0.01\\
17.55	0.01\\
17.56	0.01\\
17.57	0.01\\
17.58	0.01\\
17.59	0.01\\
17.6	0.01\\
17.61	0.01\\
17.62	0.01\\
17.63	0.01\\
17.64	0.01\\
17.65	0.01\\
17.66	0.01\\
17.67	0.01\\
17.68	0.01\\
17.69	0.01\\
17.7	0.01\\
17.71	0.01\\
17.72	0.01\\
17.73	0.01\\
17.74	0.01\\
17.75	0.01\\
17.76	0.01\\
17.77	0.01\\
17.78	0.01\\
17.79	0.01\\
17.8	0.01\\
17.81	0.01\\
17.82	0.01\\
17.83	0.01\\
17.84	0.01\\
17.85	0.01\\
17.86	0.01\\
17.87	0.01\\
17.88	0.01\\
17.89	0.01\\
17.9	0.01\\
17.91	0.01\\
17.92	0.01\\
17.93	0.01\\
17.94	0.01\\
17.95	0.01\\
17.96	0.01\\
17.97	0.01\\
17.98	0.01\\
17.99	0.01\\
18	0.01\\
18.01	0.01\\
18.02	0.01\\
18.03	0.01\\
18.04	0.01\\
18.05	0.01\\
18.06	0.01\\
18.07	0.01\\
18.08	0.01\\
18.09	0.01\\
18.1	0.01\\
18.11	0.01\\
18.12	0.01\\
18.13	0.01\\
18.14	0.01\\
18.15	0.01\\
18.16	0.01\\
18.17	0.01\\
18.18	0.01\\
18.19	0.01\\
18.2	0.01\\
18.21	0.01\\
18.22	0.01\\
18.23	0.01\\
18.24	0.01\\
18.25	0.01\\
18.26	0.01\\
18.27	0.01\\
18.28	0.01\\
18.29	0.01\\
18.3	0.01\\
18.31	0.01\\
18.32	0.01\\
18.33	0.01\\
18.34	0.01\\
18.35	0.01\\
18.36	0.01\\
18.37	0.01\\
18.38	0.01\\
18.39	0.01\\
18.4	0.01\\
18.41	0.01\\
18.42	0.01\\
18.43	0.01\\
18.44	0.01\\
18.45	0.01\\
18.46	0.01\\
18.47	0.01\\
18.48	0.01\\
18.49	0.01\\
18.5	0.01\\
18.51	0.01\\
18.52	0.01\\
18.53	0.01\\
18.54	0.01\\
18.55	0.01\\
18.56	0.01\\
18.57	0.01\\
18.58	0.01\\
18.59	0.01\\
18.6	0.01\\
18.61	0.01\\
18.62	0.01\\
18.63	0.01\\
18.64	0.01\\
18.65	0.01\\
18.66	0.01\\
18.67	0.01\\
18.68	0.01\\
18.69	0.01\\
18.7	0.01\\
18.71	0.01\\
18.72	0.01\\
18.73	0.01\\
18.74	0.01\\
18.75	0.01\\
18.76	0.01\\
18.77	0.01\\
18.78	0.01\\
18.79	0.01\\
18.8	0.01\\
18.81	0.01\\
18.82	0.01\\
18.83	0.01\\
18.84	0.01\\
18.85	0.01\\
18.86	0.01\\
18.87	0.01\\
18.88	0.01\\
18.89	0.01\\
18.9	0.01\\
18.91	0.01\\
18.92	0.01\\
18.93	0.01\\
18.94	0.01\\
18.95	0.01\\
18.96	0.01\\
18.97	0.01\\
18.98	0.01\\
18.99	0.01\\
19	0.01\\
19.01	0.01\\
19.02	0.01\\
19.03	0.01\\
19.04	0.01\\
19.05	0.01\\
19.06	0.01\\
19.07	0.01\\
19.08	0.01\\
19.09	0.01\\
19.1	0.01\\
19.11	0.01\\
19.12	0.01\\
19.13	0.01\\
19.14	0.01\\
19.15	0.01\\
19.16	0.01\\
19.17	0.01\\
19.18	0.01\\
19.19	0.01\\
19.2	0.01\\
19.21	0.01\\
19.22	0.01\\
19.23	0.01\\
19.24	0.01\\
19.25	0.01\\
19.26	0.01\\
19.27	0.01\\
19.28	0.01\\
19.29	0.01\\
19.3	0.01\\
19.31	0.01\\
19.32	0.01\\
19.33	0.01\\
19.34	0.01\\
19.35	0.01\\
19.36	0.01\\
19.37	0.01\\
19.38	0.01\\
19.39	0.01\\
19.4	0.01\\
19.41	0.01\\
19.42	0.01\\
19.43	0.01\\
19.44	0.01\\
19.45	0.01\\
19.46	0.01\\
19.47	0.01\\
19.48	0.01\\
19.49	0.01\\
19.5	0.01\\
19.51	0.01\\
19.52	0.01\\
19.53	0.01\\
19.54	0.01\\
19.55	0.01\\
19.56	0.01\\
19.57	0.01\\
19.58	0.01\\
19.59	0.01\\
19.6	0.01\\
19.61	0.01\\
19.62	0.01\\
19.63	0.01\\
19.64	0.01\\
19.65	0.01\\
19.66	0.01\\
19.67	0.01\\
19.68	0.01\\
19.69	0.01\\
19.7	0.01\\
19.71	0.01\\
19.72	0.01\\
19.73	0.01\\
19.74	0.01\\
19.75	0.01\\
19.76	0.01\\
19.77	0.01\\
19.78	0.01\\
19.79	0.01\\
19.8	0.01\\
19.81	0.01\\
19.82	0.01\\
19.83	0.01\\
19.84	0.01\\
19.85	0.01\\
19.86	0.01\\
19.87	0.01\\
19.88	0.01\\
19.89	0.01\\
19.9	0.01\\
19.91	0.01\\
19.92	0.01\\
19.93	0.01\\
19.94	0.01\\
19.95	0.01\\
19.96	0.01\\
19.97	0.01\\
19.98	0.01\\
19.99	0.01\\
20	0.01\\
20.01	0.01\\
20.02	0.01\\
20.03	0.01\\
20.04	0.01\\
20.05	0.01\\
20.06	0.01\\
20.07	0.01\\
20.08	0.01\\
20.09	0.01\\
20.1	0.01\\
20.11	0.01\\
20.12	0.01\\
20.13	0.01\\
20.14	0.01\\
20.15	0.01\\
20.16	0.01\\
20.17	0.01\\
20.18	0.01\\
20.19	0.01\\
20.2	0.01\\
20.21	0.01\\
20.22	0.01\\
20.23	0.01\\
20.24	0.01\\
20.25	0.01\\
20.26	0.01\\
20.27	0.01\\
20.28	0.01\\
20.29	0.01\\
20.3	0.01\\
20.31	0.01\\
20.32	0.01\\
20.33	0.01\\
20.34	0.01\\
20.35	0.01\\
20.36	0.01\\
20.37	0.01\\
20.38	0.01\\
20.39	0.01\\
20.4	0.01\\
20.41	0.01\\
20.42	0.01\\
20.43	0.01\\
20.44	0.01\\
20.45	0.01\\
20.46	0.01\\
20.47	0.01\\
20.48	0.01\\
20.49	0.01\\
20.5	0.01\\
20.51	0.01\\
20.52	0.01\\
20.53	0.01\\
20.54	0.01\\
20.55	0.01\\
20.56	0.01\\
20.57	0.01\\
20.58	0.01\\
20.59	0.01\\
20.6	0.01\\
20.61	0.01\\
20.62	0.01\\
20.63	0.01\\
20.64	0.01\\
20.65	0.01\\
20.66	0.01\\
20.67	0.01\\
20.68	0.01\\
20.69	0.01\\
20.7	0.01\\
20.71	0.01\\
20.72	0.01\\
20.73	0.01\\
20.74	0.01\\
20.75	0.01\\
20.76	0.01\\
20.77	0.01\\
20.78	0.01\\
20.79	0.01\\
20.8	0.01\\
20.81	0.01\\
20.82	0.01\\
20.83	0.01\\
20.84	0.01\\
20.85	0.01\\
20.86	0.01\\
20.87	0.01\\
20.88	0.01\\
20.89	0.01\\
20.9	0.01\\
20.91	0.01\\
20.92	0.01\\
20.93	0.01\\
20.94	0.01\\
20.95	0.01\\
20.96	0.01\\
20.97	0.01\\
20.98	0.01\\
20.99	0.01\\
21	0.01\\
21.01	0.01\\
21.02	0.01\\
21.03	0.01\\
21.04	0.01\\
21.05	0.01\\
21.06	0.01\\
21.07	0.01\\
21.08	0.01\\
21.09	0.01\\
21.1	0.01\\
21.11	0.01\\
21.12	0.01\\
21.13	0.01\\
21.14	0.01\\
21.15	0.01\\
21.16	0.01\\
21.17	0.01\\
21.18	0.01\\
21.19	0.01\\
21.2	0.01\\
21.21	0.01\\
21.22	0.01\\
21.23	0.01\\
21.24	0.01\\
21.25	0.01\\
21.26	0.01\\
21.27	0.01\\
21.28	0.01\\
21.29	0.01\\
21.3	0.01\\
21.31	0.01\\
21.32	0.01\\
21.33	0.01\\
21.34	0.01\\
21.35	0.01\\
21.36	0.01\\
21.37	0.01\\
21.38	0.01\\
21.39	0.01\\
21.4	0.01\\
21.41	0.01\\
21.42	0.01\\
21.43	0.01\\
21.44	0.01\\
21.45	0.01\\
21.46	0.01\\
21.47	0.01\\
21.48	0.01\\
21.49	0.01\\
21.5	0.01\\
21.51	0.01\\
21.52	0.01\\
21.53	0.01\\
21.54	0.01\\
21.55	0.01\\
21.56	0.01\\
21.57	0.01\\
21.58	0.01\\
21.59	0.01\\
21.6	0.01\\
21.61	0.01\\
21.62	0.01\\
21.63	0.01\\
21.64	0.01\\
21.65	0.01\\
21.66	0.01\\
21.67	0.01\\
21.68	0.01\\
21.69	0.01\\
21.7	0.01\\
21.71	0.01\\
21.72	0.01\\
21.73	0.01\\
21.74	0.01\\
21.75	0.01\\
21.76	0.01\\
21.77	0.01\\
21.78	0.01\\
21.79	0.01\\
21.8	0.01\\
21.81	0.01\\
21.82	0.01\\
21.83	0.01\\
21.84	0.01\\
21.85	0.01\\
21.86	0.01\\
21.87	0.01\\
21.88	0.01\\
21.89	0.01\\
21.9	0.01\\
21.91	0.01\\
21.92	0.01\\
21.93	0.01\\
21.94	0.01\\
21.95	0.01\\
21.96	0.01\\
21.97	0.01\\
21.98	0.01\\
21.99	0.01\\
22	0.01\\
22.01	0.01\\
22.02	0.01\\
22.03	0.01\\
22.04	0.01\\
22.05	0.01\\
22.06	0.01\\
22.07	0.01\\
22.08	0.01\\
22.09	0.01\\
22.1	0.01\\
22.11	0.01\\
22.12	0.01\\
22.13	0.01\\
22.14	0.01\\
22.15	0.01\\
22.16	0.01\\
22.17	0.01\\
22.18	0.01\\
22.19	0.01\\
22.2	0.01\\
22.21	0.01\\
22.22	0.01\\
22.23	0.01\\
22.24	0.01\\
22.25	0.01\\
22.26	0.01\\
22.27	0.01\\
22.28	0.01\\
22.29	0.01\\
22.3	0.01\\
22.31	0.01\\
22.32	0.01\\
22.33	0.01\\
22.34	0.01\\
22.35	0.01\\
22.36	0.01\\
22.37	0.01\\
22.38	0.01\\
22.39	0.01\\
22.4	0.01\\
22.41	0.01\\
22.42	0.01\\
22.43	0.01\\
22.44	0.01\\
22.45	0.01\\
22.46	0.01\\
22.47	0.01\\
22.48	0.01\\
22.49	0.01\\
22.5	0.01\\
22.51	0.01\\
22.52	0.01\\
22.53	0.01\\
22.54	0.01\\
22.55	0.01\\
22.56	0.01\\
22.57	0.01\\
22.58	0.01\\
22.59	0.01\\
22.6	0.01\\
22.61	0.01\\
22.62	0.01\\
22.63	0.01\\
22.64	0.01\\
22.65	0.01\\
22.66	0.01\\
22.67	0.01\\
22.68	0.01\\
22.69	0.01\\
22.7	0.01\\
22.71	0.01\\
22.72	0.01\\
22.73	0.01\\
22.74	0.01\\
22.75	0.01\\
22.76	0.01\\
22.77	0.01\\
22.78	0.01\\
22.79	0.01\\
22.8	0.01\\
22.81	0.01\\
22.82	0.01\\
22.83	0.01\\
22.84	0.01\\
22.85	0.01\\
22.86	0.01\\
22.87	0.01\\
22.88	0.01\\
22.89	0.01\\
22.9	0.01\\
22.91	0.01\\
22.92	0.01\\
22.93	0.01\\
22.94	0.01\\
22.95	0.01\\
22.96	0.01\\
22.97	0.01\\
22.98	0.01\\
22.99	0.01\\
23	0.01\\
23.01	0.01\\
23.02	0.01\\
23.03	0.01\\
23.04	0.01\\
23.05	0.01\\
23.06	0.01\\
23.07	0.01\\
23.08	0.01\\
23.09	0.01\\
23.1	0.01\\
23.11	0.01\\
23.12	0.01\\
23.13	0.01\\
23.14	0.01\\
23.15	0.01\\
23.16	0.01\\
23.17	0.01\\
23.18	0.01\\
23.19	0.01\\
23.2	0.01\\
23.21	0.01\\
23.22	0.01\\
23.23	0.01\\
23.24	0.01\\
23.25	0.01\\
23.26	0.01\\
23.27	0.01\\
23.28	0.01\\
23.29	0.01\\
23.3	0.01\\
23.31	0.01\\
23.32	0.01\\
23.33	0.01\\
23.34	0.01\\
23.35	0.01\\
23.36	0.01\\
23.37	0.01\\
23.38	0.01\\
23.39	0.01\\
23.4	0.01\\
23.41	0.01\\
23.42	0.01\\
23.43	0.01\\
23.44	0.01\\
23.45	0.01\\
23.46	0.01\\
23.47	0.01\\
23.48	0.01\\
23.49	0.01\\
23.5	0.01\\
23.51	0.01\\
23.52	0.01\\
23.53	0.01\\
23.54	0.01\\
23.55	0.01\\
23.56	0.01\\
23.57	0.01\\
23.58	0.01\\
23.59	0.01\\
23.6	0.01\\
23.61	0.01\\
23.62	0.01\\
23.63	0.01\\
23.64	0.01\\
23.65	0.01\\
23.66	0.01\\
23.67	0.01\\
23.68	0.01\\
23.69	0.01\\
23.7	0.01\\
23.71	0.01\\
23.72	0.01\\
23.73	0.01\\
23.74	0.01\\
23.75	0.01\\
23.76	0.01\\
23.77	0.01\\
23.78	0.01\\
23.79	0.01\\
23.8	0.01\\
23.81	0.01\\
23.82	0.01\\
23.83	0.01\\
23.84	0.01\\
23.85	0.01\\
23.86	0.01\\
23.87	0.01\\
23.88	0.01\\
23.89	0.01\\
23.9	0.01\\
23.91	0.01\\
23.92	0.01\\
23.93	0.01\\
23.94	0.01\\
23.95	0.01\\
23.96	0.01\\
23.97	0.01\\
23.98	0.01\\
23.99	0.01\\
24	0.01\\
24.01	0.01\\
24.02	0.01\\
24.03	0.01\\
24.04	0.01\\
24.05	0.01\\
24.06	0.01\\
24.07	0.01\\
24.08	0.01\\
24.09	0.01\\
24.1	0.01\\
24.11	0.01\\
24.12	0.01\\
24.13	0.01\\
24.14	0.01\\
24.15	0.01\\
24.16	0.01\\
24.17	0.01\\
24.18	0.01\\
24.19	0.01\\
24.2	0.01\\
24.21	0.01\\
24.22	0.01\\
24.23	0.01\\
24.24	0.01\\
24.25	0.01\\
24.26	0.01\\
24.27	0.01\\
24.28	0.01\\
24.29	0.01\\
24.3	0.01\\
24.31	0.01\\
24.32	0.01\\
24.33	0.01\\
24.34	0.01\\
24.35	0.01\\
24.36	0.01\\
24.37	0.01\\
24.38	0.01\\
24.39	0.01\\
24.4	0.01\\
24.41	0.01\\
24.42	0.01\\
24.43	0.01\\
24.44	0.01\\
24.45	0.01\\
24.46	0.01\\
24.47	0.01\\
24.48	0.01\\
24.49	0.01\\
24.5	0.01\\
24.51	0.01\\
24.52	0.01\\
24.53	0.01\\
24.54	0.01\\
24.55	0.01\\
24.56	0.01\\
24.57	0.01\\
24.58	0.01\\
24.59	0.01\\
24.6	0.01\\
24.61	0.01\\
24.62	0.01\\
24.63	0.01\\
24.64	0.01\\
24.65	0.01\\
24.66	0.01\\
24.67	0.01\\
24.68	0.01\\
24.69	0.01\\
24.7	0.01\\
24.71	0.01\\
24.72	0.01\\
24.73	0.01\\
24.74	0.01\\
24.75	0.01\\
24.76	0.01\\
24.77	0.01\\
24.78	0.01\\
24.79	0.01\\
24.8	0.01\\
24.81	0.01\\
24.82	0.01\\
24.83	0.01\\
24.84	0.01\\
24.85	0.01\\
24.86	0.01\\
24.87	0.01\\
24.88	0.01\\
24.89	0.01\\
24.9	0.01\\
24.91	0.01\\
24.92	0.01\\
24.93	0.01\\
24.94	0.01\\
24.95	0.01\\
24.96	0.01\\
24.97	0.01\\
24.98	0.01\\
24.99	0.01\\
25	0.01\\
25.01	0.01\\
25.02	0.01\\
25.03	0.01\\
25.04	0.01\\
25.05	0.01\\
25.06	0.01\\
25.07	0.01\\
25.08	0.01\\
25.09	0.01\\
25.1	0.01\\
25.11	0.01\\
25.12	0.01\\
25.13	0.01\\
25.14	0.01\\
25.15	0.01\\
25.16	0.01\\
25.17	0.01\\
25.18	0.01\\
25.19	0.01\\
25.2	0.01\\
25.21	0.01\\
25.22	0.01\\
25.23	0.01\\
25.24	0.01\\
25.25	0.01\\
25.26	0.01\\
25.27	0.01\\
25.28	0.01\\
25.29	0.01\\
25.3	0.01\\
25.31	0.01\\
25.32	0.01\\
25.33	0.01\\
25.34	0.01\\
25.35	0.01\\
25.36	0.01\\
25.37	0.01\\
25.38	0.01\\
25.39	0.01\\
25.4	0.01\\
25.41	0.01\\
25.42	0.01\\
25.43	0.01\\
25.44	0.01\\
25.45	0.01\\
25.46	0.01\\
25.47	0.01\\
25.48	0.01\\
25.49	0.01\\
25.5	0.01\\
25.51	0.01\\
25.52	0.01\\
25.53	0.01\\
25.54	0.01\\
25.55	0.01\\
25.56	0.01\\
25.57	0.01\\
25.58	0.01\\
25.59	0.01\\
25.6	0.01\\
25.61	0.01\\
25.62	0.01\\
25.63	0.01\\
25.64	0.01\\
25.65	0.01\\
25.66	0.01\\
25.67	0.01\\
25.68	0.01\\
25.69	0.01\\
25.7	0.01\\
25.71	0.01\\
25.72	0.01\\
25.73	0.01\\
25.74	0.01\\
25.75	0.01\\
25.76	0.01\\
25.77	0.01\\
25.78	0.01\\
25.79	0.01\\
25.8	0.01\\
25.81	0.01\\
25.82	0.01\\
25.83	0.01\\
25.84	0.01\\
25.85	0.01\\
25.86	0.01\\
25.87	0.01\\
25.88	0.01\\
25.89	0.01\\
25.9	0.01\\
25.91	0.01\\
25.92	0.01\\
25.93	0.01\\
25.94	0.01\\
25.95	0.01\\
25.96	0.01\\
25.97	0.01\\
25.98	0.01\\
25.99	0.01\\
26	0.01\\
26.01	0.01\\
26.02	0.01\\
26.03	0.01\\
26.04	0.01\\
26.05	0.01\\
26.06	0.01\\
26.07	0.01\\
26.08	0.01\\
26.09	0.01\\
26.1	0.01\\
26.11	0.01\\
26.12	0.01\\
26.13	0.01\\
26.14	0.01\\
26.15	0.01\\
26.16	0.01\\
26.17	0.01\\
26.18	0.01\\
26.19	0.01\\
26.2	0.01\\
26.21	0.01\\
26.22	0.01\\
26.23	0.01\\
26.24	0.01\\
26.25	0.01\\
26.26	0.01\\
26.27	0.01\\
26.28	0.01\\
26.29	0.01\\
26.3	0.01\\
26.31	0.01\\
26.32	0.01\\
26.33	0.01\\
26.34	0.01\\
26.35	0.01\\
26.36	0.01\\
26.37	0.01\\
26.38	0.01\\
26.39	0.01\\
26.4	0.01\\
26.41	0.01\\
26.42	0.01\\
26.43	0.01\\
26.44	0.01\\
26.45	0.01\\
26.46	0.01\\
26.47	0.01\\
26.48	0.01\\
26.49	0.01\\
26.5	0.01\\
26.51	0.01\\
26.52	0.01\\
26.53	0.01\\
26.54	0.01\\
26.55	0.01\\
26.56	0.01\\
26.57	0.01\\
26.58	0.01\\
26.59	0.01\\
26.6	0.01\\
26.61	0.01\\
26.62	0.01\\
26.63	0.01\\
26.64	0.01\\
26.65	0.01\\
26.66	0.01\\
26.67	0.01\\
26.68	0.01\\
26.69	0.01\\
26.7	0.01\\
26.71	0.01\\
26.72	0.01\\
26.73	0.01\\
26.74	0.01\\
26.75	0.01\\
26.76	0.01\\
26.77	0.01\\
26.78	0.01\\
26.79	0.01\\
26.8	0.01\\
26.81	0.01\\
26.82	0.01\\
26.83	0.01\\
26.84	0.01\\
26.85	0.01\\
26.86	0.01\\
26.87	0.01\\
26.88	0.01\\
26.89	0.01\\
26.9	0.01\\
26.91	0.01\\
26.92	0.01\\
26.93	0.01\\
26.94	0.01\\
26.95	0.01\\
26.96	0.01\\
26.97	0.01\\
26.98	0.01\\
26.99	0.01\\
27	0.01\\
27.01	0.01\\
27.02	0.01\\
27.03	0.01\\
27.04	0.01\\
27.05	0.01\\
27.06	0.01\\
27.07	0.01\\
27.08	0.01\\
27.09	0.01\\
27.1	0.01\\
27.11	0.01\\
27.12	0.01\\
27.13	0.01\\
27.14	0.01\\
27.15	0.01\\
27.16	0.01\\
27.17	0.01\\
27.18	0.01\\
27.19	0.01\\
27.2	0.01\\
27.21	0.01\\
27.22	0.01\\
27.23	0.01\\
27.24	0.01\\
27.25	0.01\\
27.26	0.01\\
27.27	0.01\\
27.28	0.01\\
27.29	0.01\\
27.3	0.01\\
27.31	0.01\\
27.32	0.01\\
27.33	0.01\\
27.34	0.01\\
27.35	0.01\\
27.36	0.01\\
27.37	0.01\\
27.38	0.01\\
27.39	0.01\\
27.4	0.01\\
27.41	0.01\\
27.42	0.01\\
27.43	0.01\\
27.44	0.01\\
27.45	0.01\\
27.46	0.01\\
27.47	0.01\\
27.48	0.01\\
27.49	0.01\\
27.5	0.01\\
27.51	0.01\\
27.52	0.01\\
27.53	0.01\\
27.54	0.01\\
27.55	0.01\\
27.56	0.01\\
27.57	0.01\\
27.58	0.01\\
27.59	0.01\\
27.6	0.01\\
27.61	0.01\\
27.62	0.01\\
27.63	0.01\\
27.64	0.01\\
27.65	0.01\\
27.66	0.01\\
27.67	0.01\\
27.68	0.01\\
27.69	0.01\\
27.7	0.01\\
27.71	0.01\\
27.72	0.01\\
27.73	0.01\\
27.74	0.01\\
27.75	0.01\\
27.76	0.01\\
27.77	0.01\\
27.78	0.01\\
27.79	0.01\\
27.8	0.01\\
27.81	0.01\\
27.82	0.01\\
27.83	0.01\\
27.84	0.01\\
27.85	0.01\\
27.86	0.01\\
27.87	0.01\\
27.88	0.01\\
27.89	0.01\\
27.9	0.01\\
27.91	0.01\\
27.92	0.01\\
27.93	0.01\\
27.94	0.01\\
27.95	0.01\\
27.96	0.01\\
27.97	0.01\\
27.98	0.01\\
27.99	0.01\\
28	0.01\\
28.01	0.01\\
28.02	0.01\\
28.03	0.01\\
28.04	0.01\\
28.05	0.01\\
28.06	0.01\\
28.07	0.01\\
28.08	0.01\\
28.09	0.01\\
28.1	0.01\\
28.11	0.01\\
28.12	0.01\\
28.13	0.01\\
28.14	0.01\\
28.15	0.01\\
28.16	0.01\\
28.17	0.01\\
28.18	0.01\\
28.19	0.01\\
28.2	0.01\\
28.21	0.01\\
28.22	0.01\\
28.23	0.01\\
28.24	0.01\\
28.25	0.01\\
28.26	0.01\\
28.27	0.01\\
28.28	0.01\\
28.29	0.01\\
28.3	0.01\\
28.31	0.01\\
28.32	0.01\\
28.33	0.01\\
28.34	0.01\\
28.35	0.01\\
28.36	0.01\\
28.37	0.01\\
28.38	0.01\\
28.39	0.01\\
28.4	0.01\\
28.41	0.01\\
28.42	0.01\\
28.43	0.01\\
28.44	0.01\\
28.45	0.01\\
28.46	0.01\\
28.47	0.01\\
28.48	0.01\\
28.49	0.01\\
28.5	0.01\\
28.51	0.01\\
28.52	0.01\\
28.53	0.01\\
28.54	0.01\\
28.55	0.01\\
28.56	0.01\\
28.57	0.01\\
28.58	0.01\\
28.59	0.01\\
28.6	0.01\\
28.61	0.01\\
28.62	0.01\\
28.63	0.01\\
28.64	0.01\\
28.65	0.01\\
28.66	0.01\\
28.67	0.01\\
28.68	0.01\\
28.69	0.01\\
28.7	0.01\\
28.71	0.01\\
28.72	0.01\\
28.73	0.01\\
28.74	0.01\\
28.75	0.01\\
28.76	0.01\\
28.77	0.01\\
28.78	0.01\\
28.79	0.01\\
28.8	0.01\\
28.81	0.01\\
28.82	0.01\\
28.83	0.01\\
28.84	0.01\\
28.85	0.01\\
28.86	0.01\\
28.87	0.01\\
28.88	0.01\\
28.89	0.01\\
28.9	0.01\\
28.91	0.01\\
28.92	0.01\\
28.93	0.01\\
28.94	0.01\\
28.95	0.01\\
28.96	0.01\\
28.97	0.01\\
28.98	0.01\\
28.99	0.01\\
29	0.01\\
29.01	0.01\\
29.02	0.01\\
29.03	0.01\\
29.04	0.01\\
29.05	0.01\\
29.06	0.01\\
29.07	0.01\\
29.08	0.01\\
29.09	0.01\\
29.1	0.01\\
29.11	0.01\\
29.12	0.01\\
29.13	0.01\\
29.14	0.01\\
29.15	0.01\\
29.16	0.01\\
29.17	0.01\\
29.18	0.01\\
29.19	0.01\\
29.2	0.01\\
29.21	0.01\\
29.22	0.01\\
29.23	0.01\\
29.24	0.01\\
29.25	0.01\\
29.26	0.01\\
29.27	0.01\\
29.28	0.01\\
29.29	0.01\\
29.3	0.01\\
29.31	0.01\\
29.32	0.01\\
29.33	0.01\\
29.34	0.01\\
29.35	0.01\\
29.36	0.01\\
29.37	0.01\\
29.38	0.01\\
29.39	0.01\\
29.4	0.01\\
29.41	0.01\\
29.42	0.01\\
29.43	0.01\\
29.44	0.01\\
29.45	0.01\\
29.46	0.01\\
29.47	0.01\\
29.48	0.01\\
29.49	0.01\\
29.5	0.01\\
29.51	0.01\\
29.52	0.01\\
29.53	0.01\\
29.54	0.01\\
29.55	0.01\\
29.56	0.01\\
29.57	0.01\\
29.58	0.01\\
29.59	0.01\\
29.6	0.01\\
29.61	0.01\\
29.62	0.01\\
29.63	0.01\\
29.64	0.01\\
29.65	0.01\\
29.66	0.01\\
29.67	0.01\\
29.68	0.01\\
29.69	0.01\\
29.7	0.01\\
29.71	0.01\\
29.72	0.01\\
29.73	0.01\\
29.74	0.01\\
29.75	0.01\\
29.76	0.01\\
29.77	0.01\\
29.78	0.01\\
29.79	0.01\\
29.8	0.01\\
29.81	0.01\\
29.82	0.01\\
29.83	0.01\\
29.84	0.01\\
29.85	0.01\\
29.86	0.01\\
29.87	0.01\\
29.88	0.01\\
29.89	0.01\\
29.9	0.01\\
29.91	0.01\\
29.92	0.01\\
29.93	0.01\\
29.94	0.01\\
29.95	0.01\\
29.96	0.01\\
29.97	0.01\\
29.98	0.01\\
29.99	0.01\\
30	0.01\\
30.01	0.01\\
30.02	0.01\\
30.03	0.01\\
30.04	0.01\\
30.05	0.01\\
30.06	0.01\\
30.07	0.01\\
30.08	0.01\\
30.09	0.01\\
30.1	0.01\\
30.11	0.01\\
30.12	0.01\\
30.13	0.01\\
30.14	0.01\\
30.15	0.01\\
30.16	0.01\\
30.17	0.01\\
30.18	0.01\\
30.19	0.01\\
30.2	0.01\\
30.21	0.01\\
30.22	0.01\\
30.23	0.01\\
30.24	0.01\\
30.25	0.01\\
30.26	0.01\\
30.27	0.01\\
30.28	0.01\\
30.29	0.01\\
30.3	0.01\\
30.31	0.01\\
30.32	0.01\\
30.33	0.01\\
30.34	0.01\\
30.35	0.01\\
30.36	0.01\\
30.37	0.01\\
30.38	0.01\\
30.39	0.01\\
30.4	0.01\\
30.41	0.01\\
30.42	0.01\\
30.43	0.01\\
30.44	0.01\\
30.45	0.01\\
30.46	0.01\\
30.47	0.01\\
30.48	0.01\\
30.49	0.01\\
30.5	0.01\\
30.51	0.01\\
30.52	0.01\\
30.53	0.01\\
30.54	0.01\\
30.55	0.01\\
30.56	0.01\\
30.57	0.01\\
30.58	0.01\\
30.59	0.01\\
30.6	0.01\\
30.61	0.01\\
30.62	0.01\\
30.63	0.01\\
30.64	0.01\\
30.65	0.01\\
30.66	0.01\\
30.67	0.01\\
30.68	0.01\\
30.69	0.01\\
30.7	0.01\\
30.71	0.01\\
30.72	0.01\\
30.73	0.01\\
30.74	0.01\\
30.75	0.01\\
30.76	0.01\\
30.77	0.01\\
30.78	0.01\\
30.79	0.01\\
30.8	0.01\\
30.81	0.01\\
30.82	0.01\\
30.83	0.01\\
30.84	0.01\\
30.85	0.01\\
30.86	0.01\\
30.87	0.01\\
30.88	0.01\\
30.89	0.01\\
30.9	0.01\\
30.91	0.01\\
30.92	0.01\\
30.93	0.01\\
30.94	0.01\\
30.95	0.01\\
30.96	0.01\\
30.97	0.01\\
30.98	0.01\\
30.99	0.01\\
31	0.01\\
31.01	0.01\\
31.02	0.01\\
31.03	0.01\\
31.04	0.01\\
31.05	0.01\\
31.06	0.01\\
31.07	0.01\\
31.08	0.01\\
31.09	0.01\\
31.1	0.01\\
31.11	0.01\\
31.12	0.01\\
31.13	0.01\\
31.14	0.01\\
31.15	0.01\\
31.16	0.01\\
31.17	0.01\\
31.18	0.01\\
31.19	0.01\\
31.2	0.01\\
31.21	0.01\\
31.22	0.01\\
31.23	0.01\\
31.24	0.01\\
31.25	0.01\\
31.26	0.01\\
31.27	0.01\\
31.28	0.01\\
31.29	0.01\\
31.3	0.01\\
31.31	0.01\\
31.32	0.01\\
31.33	0.01\\
31.34	0.01\\
31.35	0.01\\
31.36	0.01\\
31.37	0.01\\
31.38	0.01\\
31.39	0.01\\
31.4	0.01\\
31.41	0.01\\
31.42	0.01\\
31.43	0.01\\
31.44	0.01\\
31.45	0.01\\
31.46	0.01\\
31.47	0.01\\
31.48	0.01\\
31.49	0.01\\
31.5	0.01\\
31.51	0.01\\
31.52	0.01\\
31.53	0.01\\
31.54	0.01\\
31.55	0.01\\
31.56	0.01\\
31.57	0.01\\
31.58	0.01\\
31.59	0.01\\
31.6	0.01\\
31.61	0.01\\
31.62	0.01\\
31.63	0.01\\
31.64	0.01\\
31.65	0.01\\
31.66	0.01\\
31.67	0.01\\
31.68	0.01\\
31.69	0.01\\
31.7	0.01\\
31.71	0.01\\
31.72	0.01\\
31.73	0.01\\
31.74	0.01\\
31.75	0.01\\
31.76	0.01\\
31.77	0.01\\
31.78	0.01\\
31.79	0.01\\
31.8	0.01\\
31.81	0.01\\
31.82	0.01\\
31.83	0.01\\
31.84	0.01\\
31.85	0.01\\
31.86	0.01\\
31.87	0.01\\
31.88	0.01\\
31.89	0.01\\
31.9	0.01\\
31.91	0.01\\
31.92	0.01\\
31.93	0.01\\
31.94	0.01\\
31.95	0.01\\
31.96	0.01\\
31.97	0.01\\
31.98	0.01\\
31.99	0.01\\
32	0.01\\
32.01	0.01\\
32.02	0.01\\
32.03	0.01\\
32.04	0.01\\
32.05	0.01\\
32.06	0.01\\
32.07	0.01\\
32.08	0.01\\
32.09	0.01\\
32.1	0.01\\
32.11	0.01\\
32.12	0.01\\
32.13	0.01\\
32.14	0.01\\
32.15	0.01\\
32.16	0.01\\
32.17	0.01\\
32.18	0.01\\
32.19	0.01\\
32.2	0.01\\
32.21	0.01\\
32.22	0.01\\
32.23	0.01\\
32.24	0.01\\
32.25	0.01\\
32.26	0.01\\
32.27	0.01\\
32.28	0.01\\
32.29	0.01\\
32.3	0.01\\
32.31	0.01\\
32.32	0.01\\
32.33	0.01\\
32.34	0.01\\
32.35	0.01\\
32.36	0.01\\
32.37	0.01\\
32.38	0.01\\
32.39	0.01\\
32.4	0.01\\
32.41	0.01\\
32.42	0.01\\
32.43	0.01\\
32.44	0.01\\
32.45	0.01\\
32.46	0.01\\
32.47	0.01\\
32.48	0.01\\
32.49	0.01\\
32.5	0.01\\
32.51	0.01\\
32.52	0.01\\
32.53	0.01\\
32.54	0.01\\
32.55	0.01\\
32.56	0.01\\
32.57	0.01\\
32.58	0.01\\
32.59	0.01\\
32.6	0.01\\
32.61	0.01\\
32.62	0.01\\
32.63	0.01\\
32.64	0.01\\
32.65	0.01\\
32.66	0.01\\
32.67	0.01\\
32.68	0.01\\
32.69	0.01\\
32.7	0.01\\
32.71	0.01\\
32.72	0.01\\
32.73	0.01\\
32.74	0.01\\
32.75	0.01\\
32.76	0.01\\
32.77	0.01\\
32.78	0.01\\
32.79	0.01\\
32.8	0.01\\
32.81	0.01\\
32.82	0.01\\
32.83	0.01\\
32.84	0.01\\
32.85	0.01\\
32.86	0.01\\
32.87	0.01\\
32.88	0.01\\
32.89	0.01\\
32.9	0.01\\
32.91	0.01\\
32.92	0.01\\
32.93	0.01\\
32.94	0.01\\
32.95	0.01\\
32.96	0.01\\
32.97	0.01\\
32.98	0.01\\
32.99	0.01\\
33	0.01\\
33.01	0.01\\
33.02	0.01\\
33.03	0.01\\
33.04	0.01\\
33.05	0.01\\
33.06	0.01\\
33.07	0.01\\
33.08	0.01\\
33.09	0.01\\
33.1	0.01\\
33.11	0.01\\
33.12	0.01\\
33.13	0.01\\
33.14	0.01\\
33.15	0.01\\
33.16	0.01\\
33.17	0.01\\
33.18	0.01\\
33.19	0.01\\
33.2	0.01\\
33.21	0.01\\
33.22	0.01\\
33.23	0.01\\
33.24	0.01\\
33.25	0.01\\
33.26	0.01\\
33.27	0.01\\
33.28	0.01\\
33.29	0.01\\
33.3	0.01\\
33.31	0.01\\
33.32	0.01\\
33.33	0.01\\
33.34	0.01\\
33.35	0.01\\
33.36	0.01\\
33.37	0.01\\
33.38	0.01\\
33.39	0.01\\
33.4	0.01\\
33.41	0.01\\
33.42	0.01\\
33.43	0.01\\
33.44	0.01\\
33.45	0.01\\
33.46	0.01\\
33.47	0.01\\
33.48	0.01\\
33.49	0.01\\
33.5	0.01\\
33.51	0.01\\
33.52	0.01\\
33.53	0.01\\
33.54	0.01\\
33.55	0.01\\
33.56	0.01\\
33.57	0.01\\
33.58	0.01\\
33.59	0.01\\
33.6	0.01\\
33.61	0.01\\
33.62	0.01\\
33.63	0.01\\
33.64	0.01\\
33.65	0.01\\
33.66	0.01\\
33.67	0.01\\
33.68	0.01\\
33.69	0.01\\
33.7	0.01\\
33.71	0.01\\
33.72	0.01\\
33.73	0.01\\
33.74	0.01\\
33.75	0.01\\
33.76	0.01\\
33.77	0.01\\
33.78	0.01\\
33.79	0.01\\
33.8	0.01\\
33.81	0.01\\
33.82	0.01\\
33.83	0.01\\
33.84	0.01\\
33.85	0.01\\
33.86	0.01\\
33.87	0.01\\
33.88	0.01\\
33.89	0.01\\
33.9	0.01\\
33.91	0.01\\
33.92	0.01\\
33.93	0.01\\
33.94	0.01\\
33.95	0.01\\
33.96	0.01\\
33.97	0.01\\
33.98	0.01\\
33.99	0.01\\
34	0.01\\
34.01	0.01\\
34.02	0.01\\
34.03	0.01\\
34.04	0.01\\
34.05	0.01\\
34.06	0.01\\
34.07	0.01\\
34.08	0.01\\
34.09	0.01\\
34.1	0.01\\
34.11	0.01\\
34.12	0.01\\
34.13	0.01\\
34.14	0.01\\
34.15	0.01\\
34.16	0.01\\
34.17	0.01\\
34.18	0.01\\
34.19	0.01\\
34.2	0.01\\
34.21	0.01\\
34.22	0.01\\
34.23	0.01\\
34.24	0.01\\
34.25	0.01\\
34.26	0.01\\
34.27	0.01\\
34.28	0.01\\
34.29	0.01\\
34.3	0.01\\
34.31	0.01\\
34.32	0.01\\
34.33	0.01\\
34.34	0.01\\
34.35	0.01\\
34.36	0.01\\
34.37	0.01\\
34.38	0.01\\
34.39	0.01\\
34.4	0.01\\
34.41	0.01\\
34.42	0.01\\
34.43	0.01\\
34.44	0.01\\
34.45	0.01\\
34.46	0.01\\
34.47	0.01\\
34.48	0.01\\
34.49	0.01\\
34.5	0.01\\
34.51	0.01\\
34.52	0.01\\
34.53	0.01\\
34.54	0.01\\
34.55	0.01\\
34.56	0.01\\
34.57	0.01\\
34.58	0.01\\
34.59	0.01\\
34.6	0.01\\
34.61	0.01\\
34.62	0.01\\
34.63	0.01\\
34.64	0.01\\
34.65	0.01\\
34.66	0.01\\
34.67	0.01\\
34.68	0.01\\
34.69	0.01\\
34.7	0.01\\
34.71	0.01\\
34.72	0.01\\
34.73	0.01\\
34.74	0.01\\
34.75	0.01\\
34.76	0.01\\
34.77	0.01\\
34.78	0.01\\
34.79	0.01\\
34.8	0.01\\
34.81	0.01\\
34.82	0.01\\
34.83	0.01\\
34.84	0.01\\
34.85	0.01\\
34.86	0.01\\
34.87	0.01\\
34.88	0.01\\
34.89	0.01\\
34.9	0.01\\
34.91	0.01\\
34.92	0.01\\
34.93	0.01\\
34.94	0.01\\
34.95	0.01\\
34.96	0.01\\
34.97	0.01\\
34.98	0.01\\
34.99	0.01\\
35	0.01\\
35.01	0.01\\
35.02	0.01\\
35.03	0.01\\
35.04	0.01\\
35.05	0.01\\
35.06	0.01\\
35.07	0.01\\
35.08	0.01\\
35.09	0.01\\
35.1	0.01\\
35.11	0.01\\
35.12	0.01\\
35.13	0.01\\
35.14	0.01\\
35.15	0.01\\
35.16	0.01\\
35.17	0.01\\
35.18	0.01\\
35.19	0.01\\
35.2	0.01\\
35.21	0.01\\
35.22	0.01\\
35.23	0.01\\
35.24	0.01\\
35.25	0.01\\
35.26	0.01\\
35.27	0.01\\
35.28	0.01\\
35.29	0.01\\
35.3	0.01\\
35.31	0.01\\
35.32	0.01\\
35.33	0.01\\
35.34	0.01\\
35.35	0.01\\
35.36	0.01\\
35.37	0.01\\
35.38	0.01\\
35.39	0.01\\
35.4	0.01\\
35.41	0.01\\
35.42	0.01\\
35.43	0.01\\
35.44	0.01\\
35.45	0.01\\
35.46	0.01\\
35.47	0.01\\
35.48	0.01\\
35.49	0.01\\
35.5	0.01\\
35.51	0.01\\
35.52	0.01\\
35.53	0.01\\
35.54	0.01\\
35.55	0.01\\
35.56	0.01\\
35.57	0.01\\
35.58	0.01\\
35.59	0.01\\
35.6	0.01\\
35.61	0.01\\
35.62	0.01\\
35.63	0.01\\
35.64	0.01\\
35.65	0.01\\
35.66	0.01\\
35.67	0.01\\
35.68	0.01\\
35.69	0.01\\
35.7	0.01\\
35.71	0.01\\
35.72	0.01\\
35.73	0.01\\
35.74	0.01\\
35.75	0.01\\
35.76	0.01\\
35.77	0.01\\
35.78	0.01\\
35.79	0.01\\
35.8	0.01\\
35.81	0.01\\
35.82	0.01\\
35.83	0.01\\
35.84	0.01\\
35.85	0.01\\
35.86	0.01\\
35.87	0.01\\
35.88	0.01\\
35.89	0.01\\
35.9	0.01\\
35.91	0.01\\
35.92	0.01\\
35.93	0.01\\
35.94	0.01\\
35.95	0.01\\
35.96	0.01\\
35.97	0.01\\
35.98	0.01\\
35.99	0.01\\
36	0.01\\
36.01	0.01\\
36.02	0.01\\
36.03	0.01\\
36.04	0.01\\
36.05	0.01\\
36.06	0.01\\
36.07	0.01\\
36.08	0.01\\
36.09	0.01\\
36.1	0.01\\
36.11	0.01\\
36.12	0.01\\
36.13	0.01\\
36.14	0.01\\
36.15	0.01\\
36.16	0.01\\
36.17	0.01\\
36.18	0.01\\
36.19	0.01\\
36.2	0.01\\
36.21	0.01\\
36.22	0.01\\
36.23	0.01\\
36.24	0.01\\
36.25	0.01\\
36.26	0.01\\
36.27	0.01\\
36.28	0.01\\
36.29	0.01\\
36.3	0.01\\
36.31	0.01\\
36.32	0.01\\
36.33	0.01\\
36.34	0.01\\
36.35	0.01\\
36.36	0.01\\
36.37	0.01\\
36.38	0.01\\
36.39	0.01\\
36.4	0.01\\
36.41	0.01\\
36.42	0.01\\
36.43	0.01\\
36.44	0.01\\
36.45	0.01\\
36.46	0.01\\
36.47	0.01\\
36.48	0.01\\
36.49	0.01\\
36.5	0.01\\
36.51	0.01\\
36.52	0.01\\
36.53	0.01\\
36.54	0.01\\
36.55	0.01\\
36.56	0.01\\
36.57	0.01\\
36.58	0.01\\
36.59	0.01\\
36.6	0.01\\
36.61	0.01\\
36.62	0.01\\
36.63	0.01\\
36.64	0.01\\
36.65	0.01\\
36.66	0.01\\
36.67	0.01\\
36.68	0.01\\
36.69	0.01\\
36.7	0.01\\
36.71	0.01\\
36.72	0.01\\
36.73	0.01\\
36.74	0.01\\
36.75	0.01\\
36.76	0.01\\
36.77	0.01\\
36.78	0.01\\
36.79	0.01\\
36.8	0.01\\
36.81	0.01\\
36.82	0.01\\
36.83	0.01\\
36.84	0.01\\
36.85	0.01\\
36.86	0.01\\
36.87	0.01\\
36.88	0.01\\
36.89	0.01\\
36.9	0.01\\
36.91	0.01\\
36.92	0.01\\
36.93	0.01\\
36.94	0.01\\
36.95	0.01\\
36.96	0.01\\
36.97	0.01\\
36.98	0.01\\
36.99	0.01\\
37	0.01\\
37.01	0.01\\
37.02	0.01\\
37.03	0.01\\
37.04	0.01\\
37.05	0.01\\
37.06	0.01\\
37.07	0.01\\
37.08	0.01\\
37.09	0.01\\
37.1	0.01\\
37.11	0.01\\
37.12	0.01\\
37.13	0.01\\
37.14	0.01\\
37.15	0.01\\
37.16	0.01\\
37.17	0.01\\
37.18	0.01\\
37.19	0.01\\
37.2	0.01\\
37.21	0.01\\
37.22	0.01\\
37.23	0.01\\
37.24	0.01\\
37.25	0.01\\
37.26	0.01\\
37.27	0.01\\
37.28	0.01\\
37.29	0.01\\
37.3	0.01\\
37.31	0.01\\
37.32	0.01\\
37.33	0.01\\
37.34	0.01\\
37.35	0.01\\
37.36	0.01\\
37.37	0.01\\
37.38	0.01\\
37.39	0.01\\
37.4	0.01\\
37.41	0.01\\
37.42	0.01\\
37.43	0.01\\
37.44	0.01\\
37.45	0.01\\
37.46	0.01\\
37.47	0.01\\
37.48	0.01\\
37.49	0.01\\
37.5	0.01\\
37.51	0.01\\
37.52	0.01\\
37.53	0.01\\
37.54	0.01\\
37.55	0.01\\
37.56	0.01\\
37.57	0.01\\
37.58	0.01\\
37.59	0.01\\
37.6	0.01\\
37.61	0.01\\
37.62	0.01\\
37.63	0.01\\
37.64	0.01\\
37.65	0.01\\
37.66	0.01\\
37.67	0.01\\
37.68	0.01\\
37.69	0.01\\
37.7	0.01\\
37.71	0.01\\
37.72	0.01\\
37.73	0.01\\
37.74	0.01\\
37.75	0.01\\
37.76	0.01\\
37.77	0.01\\
37.78	0.01\\
37.79	0.01\\
37.8	0.01\\
37.81	0.01\\
37.82	0.01\\
37.83	0.01\\
37.84	0.01\\
37.85	0.01\\
37.86	0.01\\
37.87	0.01\\
37.88	0.01\\
37.89	0.01\\
37.9	0.01\\
37.91	0.01\\
37.92	0.01\\
37.93	0.01\\
37.94	0.01\\
37.95	0.01\\
37.96	0.01\\
37.97	0.01\\
37.98	0.01\\
37.99	0.01\\
38	0.01\\
38.01	0.01\\
38.02	0.01\\
38.03	0.01\\
38.04	0.01\\
38.05	0.01\\
38.06	0.01\\
38.07	0.01\\
38.08	0.01\\
38.09	0.01\\
38.1	0.01\\
38.11	0.01\\
38.12	0.01\\
38.13	0.01\\
38.14	0.01\\
38.15	0.01\\
38.16	0.01\\
38.17	0.01\\
38.18	0.01\\
38.19	0.01\\
38.2	0.01\\
38.21	0.01\\
38.22	0.01\\
38.23	0.01\\
38.24	0.01\\
38.25	0.01\\
38.26	0.01\\
38.27	0.01\\
38.28	0.01\\
38.29	0.01\\
38.3	0.01\\
38.31	0.01\\
38.32	0.01\\
38.33	0.01\\
38.34	0.01\\
38.35	0.01\\
38.36	0.01\\
38.37	0.01\\
38.38	0.01\\
38.39	0.01\\
38.4	0.01\\
38.41	0.01\\
38.42	0.01\\
38.43	0.01\\
38.44	0.01\\
38.45	0.01\\
38.46	0.01\\
38.47	0.01\\
38.48	0.01\\
38.49	0.01\\
38.5	0.01\\
38.51	0.01\\
38.52	0.01\\
38.53	0.01\\
38.54	0.01\\
38.55	0.01\\
38.56	0.01\\
38.57	0.01\\
38.58	0.01\\
38.59	0.01\\
38.6	0.01\\
38.61	0.01\\
38.62	0.01\\
38.63	0.01\\
38.64	0.01\\
38.65	0.01\\
38.66	0.01\\
38.67	0.01\\
38.68	0.01\\
38.69	0.01\\
38.7	0.01\\
38.71	0.01\\
38.72	0.01\\
38.73	0.01\\
38.74	0.01\\
38.75	0.01\\
38.76	0.01\\
38.77	0.01\\
38.78	0.01\\
38.79	0.01\\
38.8	0.01\\
38.81	0.01\\
38.82	0.01\\
38.83	0.01\\
38.84	0.01\\
38.85	0.01\\
38.86	0.01\\
38.87	0.01\\
38.88	0.01\\
38.89	0.01\\
38.9	0.01\\
38.91	0.01\\
38.92	0.01\\
38.93	0.01\\
38.94	0.01\\
38.95	0.01\\
38.96	0.01\\
38.97	0.01\\
38.98	0.01\\
38.99	0.01\\
39	0.01\\
39.01	0.01\\
39.02	0.01\\
39.03	0.01\\
39.04	0.01\\
39.05	0.01\\
39.06	0.01\\
39.07	0.01\\
39.08	0.01\\
39.09	0.01\\
39.1	0.01\\
39.11	0.01\\
39.12	0.01\\
39.13	0.01\\
39.14	0.01\\
39.15	0.01\\
39.16	0.01\\
39.17	0.01\\
39.18	0.01\\
39.19	0.01\\
39.2	0.01\\
39.21	0.01\\
39.22	0.01\\
39.23	0.01\\
39.24	0.01\\
39.25	0.01\\
39.26	0.01\\
39.27	0.01\\
39.28	0.01\\
39.29	0.01\\
39.3	0.01\\
39.31	0.01\\
39.32	0.01\\
39.33	0.01\\
39.34	0.01\\
39.35	0.01\\
39.36	0.01\\
39.37	0.01\\
39.38	0.01\\
39.39	0.01\\
39.4	0.01\\
39.41	0.01\\
39.42	0.01\\
39.43	0.01\\
39.44	0.01\\
39.45	0.01\\
39.46	0.01\\
39.47	0.01\\
39.48	0.01\\
39.49	0.01\\
39.5	0.01\\
39.51	0.01\\
39.52	0.01\\
39.53	0.01\\
39.54	0.01\\
39.55	0.01\\
39.56	0.01\\
39.57	0.01\\
39.58	0.01\\
39.59	0.01\\
39.6	0.01\\
39.61	0.01\\
39.62	0.01\\
39.63	0.01\\
39.64	0.01\\
39.65	0.01\\
39.66	0.01\\
39.67	0.01\\
39.68	0.01\\
39.69	0.01\\
39.7	0.01\\
39.71	0.01\\
39.72	0.01\\
39.73	0.01\\
39.74	0.01\\
39.75	0.01\\
39.76	0.01\\
39.77	0.01\\
39.78	0.01\\
39.79	0.01\\
39.8	0.01\\
39.81	0.01\\
39.82	0.01\\
39.83	0.01\\
39.84	0.01\\
39.85	0.01\\
39.86	0.01\\
39.87	0.01\\
39.88	0.01\\
39.89	0.01\\
39.9	0.01\\
39.91	0.01\\
39.92	0.01\\
39.93	0.01\\
39.94	0.01\\
39.95	0.01\\
39.96	0.01\\
39.97	0.01\\
39.98	0.01\\
39.99	0.01\\
40	0.01\\
40.01	0.01\\
};
\addplot [color=mycolor1,solid,forget plot]
  table[row sep=crcr]{%
40.01	0.01\\
40.02	0.01\\
40.03	0.01\\
40.04	0.01\\
40.05	0.01\\
40.06	0.01\\
40.07	0.01\\
40.08	0.01\\
40.09	0.01\\
40.1	0.01\\
40.11	0.01\\
40.12	0.01\\
40.13	0.01\\
40.14	0.01\\
40.15	0.01\\
40.16	0.01\\
40.17	0.01\\
40.18	0.01\\
40.19	0.01\\
40.2	0.01\\
40.21	0.01\\
40.22	0.01\\
40.23	0.01\\
40.24	0.01\\
40.25	0.01\\
40.26	0.01\\
40.27	0.01\\
40.28	0.01\\
40.29	0.01\\
40.3	0.01\\
40.31	0.01\\
40.32	0.01\\
40.33	0.01\\
40.34	0.01\\
40.35	0.01\\
40.36	0.01\\
40.37	0.01\\
40.38	0.01\\
40.39	0.01\\
40.4	0.01\\
40.41	0.01\\
40.42	0.01\\
40.43	0.01\\
40.44	0.01\\
40.45	0.01\\
40.46	0.01\\
40.47	0.01\\
40.48	0.01\\
40.49	0.01\\
40.5	0.01\\
40.51	0.01\\
40.52	0.01\\
40.53	0.01\\
40.54	0.01\\
40.55	0.01\\
40.56	0.01\\
40.57	0.01\\
40.58	0.01\\
40.59	0.01\\
40.6	0.01\\
40.61	0.01\\
40.62	0.01\\
40.63	0.01\\
40.64	0.01\\
40.65	0.01\\
40.66	0.01\\
40.67	0.01\\
40.68	0.01\\
40.69	0.01\\
40.7	0.01\\
40.71	0.01\\
40.72	0.01\\
40.73	0.01\\
40.74	0.01\\
40.75	0.01\\
40.76	0.01\\
40.77	0.01\\
40.78	0.01\\
40.79	0.01\\
40.8	0.01\\
40.81	0.01\\
40.82	0.01\\
40.83	0.01\\
40.84	0.01\\
40.85	0.01\\
40.86	0.01\\
40.87	0.01\\
40.88	0.01\\
40.89	0.01\\
40.9	0.01\\
40.91	0.01\\
40.92	0.01\\
40.93	0.01\\
40.94	0.01\\
40.95	0.01\\
40.96	0.01\\
40.97	0.01\\
40.98	0.01\\
40.99	0.01\\
41	0.01\\
41.01	0.01\\
41.02	0.01\\
41.03	0.01\\
41.04	0.01\\
41.05	0.01\\
41.06	0.01\\
41.07	0.01\\
41.08	0.01\\
41.09	0.01\\
41.1	0.01\\
41.11	0.01\\
41.12	0.01\\
41.13	0.01\\
41.14	0.01\\
41.15	0.01\\
41.16	0.01\\
41.17	0.01\\
41.18	0.01\\
41.19	0.01\\
41.2	0.01\\
41.21	0.01\\
41.22	0.01\\
41.23	0.01\\
41.24	0.01\\
41.25	0.01\\
41.26	0.01\\
41.27	0.01\\
41.28	0.01\\
41.29	0.01\\
41.3	0.01\\
41.31	0.01\\
41.32	0.01\\
41.33	0.01\\
41.34	0.01\\
41.35	0.01\\
41.36	0.01\\
41.37	0.01\\
41.38	0.01\\
41.39	0.01\\
41.4	0.01\\
41.41	0.01\\
41.42	0.01\\
41.43	0.01\\
41.44	0.01\\
41.45	0.01\\
41.46	0.01\\
41.47	0.01\\
41.48	0.01\\
41.49	0.01\\
41.5	0.01\\
41.51	0.01\\
41.52	0.01\\
41.53	0.01\\
41.54	0.01\\
41.55	0.01\\
41.56	0.01\\
41.57	0.01\\
41.58	0.01\\
41.59	0.01\\
41.6	0.01\\
41.61	0.01\\
41.62	0.01\\
41.63	0.01\\
41.64	0.01\\
41.65	0.01\\
41.66	0.01\\
41.67	0.01\\
41.68	0.01\\
41.69	0.01\\
41.7	0.01\\
41.71	0.01\\
41.72	0.01\\
41.73	0.01\\
41.74	0.01\\
41.75	0.01\\
41.76	0.01\\
41.77	0.01\\
41.78	0.01\\
41.79	0.01\\
41.8	0.01\\
41.81	0.01\\
41.82	0.01\\
41.83	0.01\\
41.84	0.01\\
41.85	0.01\\
41.86	0.01\\
41.87	0.01\\
41.88	0.01\\
41.89	0.01\\
41.9	0.01\\
41.91	0.01\\
41.92	0.01\\
41.93	0.01\\
41.94	0.01\\
41.95	0.01\\
41.96	0.01\\
41.97	0.01\\
41.98	0.01\\
41.99	0.01\\
42	0.01\\
42.01	0.01\\
42.02	0.01\\
42.03	0.01\\
42.04	0.01\\
42.05	0.01\\
42.06	0.01\\
42.07	0.01\\
42.08	0.01\\
42.09	0.01\\
42.1	0.01\\
42.11	0.01\\
42.12	0.01\\
42.13	0.01\\
42.14	0.01\\
42.15	0.01\\
42.16	0.01\\
42.17	0.01\\
42.18	0.01\\
42.19	0.01\\
42.2	0.01\\
42.21	0.01\\
42.22	0.01\\
42.23	0.01\\
42.24	0.01\\
42.25	0.01\\
42.26	0.01\\
42.27	0.01\\
42.28	0.01\\
42.29	0.01\\
42.3	0.01\\
42.31	0.01\\
42.32	0.01\\
42.33	0.01\\
42.34	0.01\\
42.35	0.01\\
42.36	0.01\\
42.37	0.01\\
42.38	0.01\\
42.39	0.01\\
42.4	0.01\\
42.41	0.01\\
42.42	0.01\\
42.43	0.01\\
42.44	0.01\\
42.45	0.01\\
42.46	0.01\\
42.47	0.01\\
42.48	0.01\\
42.49	0.01\\
42.5	0.01\\
42.51	0.01\\
42.52	0.01\\
42.53	0.01\\
42.54	0.01\\
42.55	0.01\\
42.56	0.01\\
42.57	0.01\\
42.58	0.01\\
42.59	0.01\\
42.6	0.01\\
42.61	0.01\\
42.62	0.01\\
42.63	0.01\\
42.64	0.01\\
42.65	0.01\\
42.66	0.01\\
42.67	0.01\\
42.68	0.01\\
42.69	0.01\\
42.7	0.01\\
42.71	0.01\\
42.72	0.01\\
42.73	0.01\\
42.74	0.01\\
42.75	0.01\\
42.76	0.01\\
42.77	0.01\\
42.78	0.01\\
42.79	0.01\\
42.8	0.01\\
42.81	0.01\\
42.82	0.01\\
42.83	0.01\\
42.84	0.01\\
42.85	0.01\\
42.86	0.01\\
42.87	0.01\\
42.88	0.01\\
42.89	0.01\\
42.9	0.01\\
42.91	0.01\\
42.92	0.01\\
42.93	0.01\\
42.94	0.01\\
42.95	0.01\\
42.96	0.01\\
42.97	0.01\\
42.98	0.01\\
42.99	0.01\\
43	0.01\\
43.01	0.01\\
43.02	0.01\\
43.03	0.01\\
43.04	0.01\\
43.05	0.01\\
43.06	0.01\\
43.07	0.01\\
43.08	0.01\\
43.09	0.01\\
43.1	0.01\\
43.11	0.01\\
43.12	0.01\\
43.13	0.01\\
43.14	0.01\\
43.15	0.01\\
43.16	0.01\\
43.17	0.01\\
43.18	0.01\\
43.19	0.01\\
43.2	0.01\\
43.21	0.01\\
43.22	0.01\\
43.23	0.01\\
43.24	0.01\\
43.25	0.01\\
43.26	0.01\\
43.27	0.01\\
43.28	0.01\\
43.29	0.01\\
43.3	0.01\\
43.31	0.01\\
43.32	0.01\\
43.33	0.01\\
43.34	0.01\\
43.35	0.01\\
43.36	0.01\\
43.37	0.01\\
43.38	0.01\\
43.39	0.01\\
43.4	0.01\\
43.41	0.01\\
43.42	0.01\\
43.43	0.01\\
43.44	0.01\\
43.45	0.01\\
43.46	0.01\\
43.47	0.01\\
43.48	0.01\\
43.49	0.01\\
43.5	0.01\\
43.51	0.01\\
43.52	0.01\\
43.53	0.01\\
43.54	0.01\\
43.55	0.01\\
43.56	0.01\\
43.57	0.01\\
43.58	0.01\\
43.59	0.01\\
43.6	0.01\\
43.61	0.01\\
43.62	0.01\\
43.63	0.01\\
43.64	0.01\\
43.65	0.01\\
43.66	0.01\\
43.67	0.01\\
43.68	0.01\\
43.69	0.01\\
43.7	0.01\\
43.71	0.01\\
43.72	0.01\\
43.73	0.01\\
43.74	0.01\\
43.75	0.01\\
43.76	0.01\\
43.77	0.01\\
43.78	0.01\\
43.79	0.01\\
43.8	0.01\\
43.81	0.01\\
43.82	0.01\\
43.83	0.01\\
43.84	0.01\\
43.85	0.01\\
43.86	0.01\\
43.87	0.01\\
43.88	0.01\\
43.89	0.01\\
43.9	0.01\\
43.91	0.01\\
43.92	0.01\\
43.93	0.01\\
43.94	0.01\\
43.95	0.01\\
43.96	0.01\\
43.97	0.01\\
43.98	0.01\\
43.99	0.01\\
44	0.01\\
44.01	0.01\\
44.02	0.01\\
44.03	0.01\\
44.04	0.01\\
44.05	0.01\\
44.06	0.01\\
44.07	0.01\\
44.08	0.01\\
44.09	0.01\\
44.1	0.01\\
44.11	0.01\\
44.12	0.01\\
44.13	0.01\\
44.14	0.01\\
44.15	0.01\\
44.16	0.01\\
44.17	0.01\\
44.18	0.01\\
44.19	0.01\\
44.2	0.01\\
44.21	0.01\\
44.22	0.01\\
44.23	0.01\\
44.24	0.01\\
44.25	0.01\\
44.26	0.01\\
44.27	0.01\\
44.28	0.01\\
44.29	0.01\\
44.3	0.01\\
44.31	0.01\\
44.32	0.01\\
44.33	0.01\\
44.34	0.01\\
44.35	0.01\\
44.36	0.01\\
44.37	0.01\\
44.38	0.01\\
44.39	0.01\\
44.4	0.01\\
44.41	0.01\\
44.42	0.01\\
44.43	0.01\\
44.44	0.01\\
44.45	0.01\\
44.46	0.01\\
44.47	0.01\\
44.48	0.01\\
44.49	0.01\\
44.5	0.01\\
44.51	0.01\\
44.52	0.01\\
44.53	0.01\\
44.54	0.01\\
44.55	0.01\\
44.56	0.01\\
44.57	0.01\\
44.58	0.01\\
44.59	0.01\\
44.6	0.01\\
44.61	0.01\\
44.62	0.01\\
44.63	0.01\\
44.64	0.01\\
44.65	0.01\\
44.66	0.01\\
44.67	0.01\\
44.68	0.01\\
44.69	0.01\\
44.7	0.01\\
44.71	0.01\\
44.72	0.01\\
44.73	0.01\\
44.74	0.01\\
44.75	0.01\\
44.76	0.01\\
44.77	0.01\\
44.78	0.01\\
44.79	0.01\\
44.8	0.01\\
44.81	0.01\\
44.82	0.01\\
44.83	0.01\\
44.84	0.01\\
44.85	0.01\\
44.86	0.01\\
44.87	0.01\\
44.88	0.01\\
44.89	0.01\\
44.9	0.01\\
44.91	0.01\\
44.92	0.01\\
44.93	0.01\\
44.94	0.01\\
44.95	0.01\\
44.96	0.01\\
44.97	0.01\\
44.98	0.01\\
44.99	0.01\\
45	0.01\\
45.01	0.01\\
45.02	0.01\\
45.03	0.01\\
45.04	0.01\\
45.05	0.01\\
45.06	0.01\\
45.07	0.01\\
45.08	0.01\\
45.09	0.01\\
45.1	0.01\\
45.11	0.01\\
45.12	0.01\\
45.13	0.01\\
45.14	0.01\\
45.15	0.01\\
45.16	0.01\\
45.17	0.01\\
45.18	0.01\\
45.19	0.01\\
45.2	0.01\\
45.21	0.01\\
45.22	0.01\\
45.23	0.01\\
45.24	0.01\\
45.25	0.01\\
45.26	0.01\\
45.27	0.01\\
45.28	0.01\\
45.29	0.01\\
45.3	0.01\\
45.31	0.01\\
45.32	0.01\\
45.33	0.01\\
45.34	0.01\\
45.35	0.01\\
45.36	0.01\\
45.37	0.01\\
45.38	0.01\\
45.39	0.01\\
45.4	0.01\\
45.41	0.01\\
45.42	0.01\\
45.43	0.01\\
45.44	0.01\\
45.45	0.01\\
45.46	0.01\\
45.47	0.01\\
45.48	0.01\\
45.49	0.01\\
45.5	0.01\\
45.51	0.01\\
45.52	0.01\\
45.53	0.01\\
45.54	0.01\\
45.55	0.01\\
45.56	0.01\\
45.57	0.01\\
45.58	0.01\\
45.59	0.01\\
45.6	0.01\\
45.61	0.01\\
45.62	0.01\\
45.63	0.01\\
45.64	0.01\\
45.65	0.01\\
45.66	0.01\\
45.67	0.01\\
45.68	0.01\\
45.69	0.01\\
45.7	0.01\\
45.71	0.01\\
45.72	0.01\\
45.73	0.01\\
45.74	0.01\\
45.75	0.01\\
45.76	0.01\\
45.77	0.01\\
45.78	0.01\\
45.79	0.01\\
45.8	0.01\\
45.81	0.01\\
45.82	0.01\\
45.83	0.01\\
45.84	0.01\\
45.85	0.01\\
45.86	0.01\\
45.87	0.01\\
45.88	0.01\\
45.89	0.01\\
45.9	0.01\\
45.91	0.01\\
45.92	0.01\\
45.93	0.01\\
45.94	0.01\\
45.95	0.01\\
45.96	0.01\\
45.97	0.01\\
45.98	0.01\\
45.99	0.01\\
46	0.01\\
46.01	0.01\\
46.02	0.01\\
46.03	0.01\\
46.04	0.01\\
46.05	0.01\\
46.06	0.01\\
46.07	0.01\\
46.08	0.01\\
46.09	0.01\\
46.1	0.01\\
46.11	0.01\\
46.12	0.01\\
46.13	0.01\\
46.14	0.01\\
46.15	0.01\\
46.16	0.01\\
46.17	0.01\\
46.18	0.01\\
46.19	0.01\\
46.2	0.01\\
46.21	0.01\\
46.22	0.01\\
46.23	0.01\\
46.24	0.01\\
46.25	0.01\\
46.26	0.01\\
46.27	0.01\\
46.28	0.01\\
46.29	0.01\\
46.3	0.01\\
46.31	0.01\\
46.32	0.01\\
46.33	0.01\\
46.34	0.01\\
46.35	0.01\\
46.36	0.01\\
46.37	0.01\\
46.38	0.01\\
46.39	0.01\\
46.4	0.01\\
46.41	0.01\\
46.42	0.01\\
46.43	0.01\\
46.44	0.01\\
46.45	0.01\\
46.46	0.01\\
46.47	0.01\\
46.48	0.01\\
46.49	0.01\\
46.5	0.01\\
46.51	0.01\\
46.52	0.01\\
46.53	0.01\\
46.54	0.01\\
46.55	0.01\\
46.56	0.01\\
46.57	0.01\\
46.58	0.01\\
46.59	0.01\\
46.6	0.01\\
46.61	0.01\\
46.62	0.01\\
46.63	0.01\\
46.64	0.01\\
46.65	0.01\\
46.66	0.01\\
46.67	0.01\\
46.68	0.01\\
46.69	0.01\\
46.7	0.01\\
46.71	0.01\\
46.72	0.01\\
46.73	0.01\\
46.74	0.01\\
46.75	0.01\\
46.76	0.01\\
46.77	0.01\\
46.78	0.01\\
46.79	0.01\\
46.8	0.01\\
46.81	0.01\\
46.82	0.01\\
46.83	0.01\\
46.84	0.01\\
46.85	0.01\\
46.86	0.01\\
46.87	0.01\\
46.88	0.01\\
46.89	0.01\\
46.9	0.01\\
46.91	0.01\\
46.92	0.01\\
46.93	0.01\\
46.94	0.01\\
46.95	0.01\\
46.96	0.01\\
46.97	0.01\\
46.98	0.01\\
46.99	0.01\\
47	0.01\\
47.01	0.01\\
47.02	0.01\\
47.03	0.01\\
47.04	0.01\\
47.05	0.01\\
47.06	0.01\\
47.07	0.01\\
47.08	0.01\\
47.09	0.01\\
47.1	0.01\\
47.11	0.01\\
47.12	0.01\\
47.13	0.01\\
47.14	0.01\\
47.15	0.01\\
47.16	0.01\\
47.17	0.01\\
47.18	0.01\\
47.19	0.01\\
47.2	0.01\\
47.21	0.01\\
47.22	0.01\\
47.23	0.01\\
47.24	0.01\\
47.25	0.01\\
47.26	0.01\\
47.27	0.01\\
47.28	0.01\\
47.29	0.01\\
47.3	0.01\\
47.31	0.01\\
47.32	0.01\\
47.33	0.01\\
47.34	0.01\\
47.35	0.01\\
47.36	0.01\\
47.37	0.01\\
47.38	0.01\\
47.39	0.01\\
47.4	0.01\\
47.41	0.01\\
47.42	0.01\\
47.43	0.01\\
47.44	0.01\\
47.45	0.01\\
47.46	0.01\\
47.47	0.01\\
47.48	0.01\\
47.49	0.01\\
47.5	0.01\\
47.51	0.01\\
47.52	0.01\\
47.53	0.01\\
47.54	0.01\\
47.55	0.01\\
47.56	0.01\\
47.57	0.01\\
47.58	0.01\\
47.59	0.01\\
47.6	0.01\\
47.61	0.01\\
47.62	0.01\\
47.63	0.01\\
47.64	0.01\\
47.65	0.01\\
47.66	0.01\\
47.67	0.01\\
47.68	0.01\\
47.69	0.01\\
47.7	0.01\\
47.71	0.01\\
47.72	0.01\\
47.73	0.01\\
47.74	0.01\\
47.75	0.01\\
47.76	0.01\\
47.77	0.01\\
47.78	0.01\\
47.79	0.01\\
47.8	0.01\\
47.81	0.01\\
47.82	0.01\\
47.83	0.01\\
47.84	0.01\\
47.85	0.01\\
47.86	0.01\\
47.87	0.01\\
47.88	0.01\\
47.89	0.01\\
47.9	0.01\\
47.91	0.01\\
47.92	0.01\\
47.93	0.01\\
47.94	0.01\\
47.95	0.01\\
47.96	0.01\\
47.97	0.01\\
47.98	0.01\\
47.99	0.01\\
48	0.01\\
48.01	0.01\\
48.02	0.01\\
48.03	0.01\\
48.04	0.01\\
48.05	0.01\\
48.06	0.01\\
48.07	0.01\\
48.08	0.01\\
48.09	0.01\\
48.1	0.01\\
48.11	0.01\\
48.12	0.01\\
48.13	0.01\\
48.14	0.01\\
48.15	0.01\\
48.16	0.01\\
48.17	0.01\\
48.18	0.01\\
48.19	0.01\\
48.2	0.01\\
48.21	0.01\\
48.22	0.01\\
48.23	0.01\\
48.24	0.01\\
48.25	0.01\\
48.26	0.01\\
48.27	0.01\\
48.28	0.01\\
48.29	0.01\\
48.3	0.01\\
48.31	0.01\\
48.32	0.01\\
48.33	0.01\\
48.34	0.01\\
48.35	0.01\\
48.36	0.01\\
48.37	0.01\\
48.38	0.01\\
48.39	0.01\\
48.4	0.01\\
48.41	0.01\\
48.42	0.01\\
48.43	0.01\\
48.44	0.01\\
48.45	0.01\\
48.46	0.01\\
48.47	0.01\\
48.48	0.01\\
48.49	0.01\\
48.5	0.01\\
48.51	0.01\\
48.52	0.01\\
48.53	0.01\\
48.54	0.01\\
48.55	0.01\\
48.56	0.01\\
48.57	0.01\\
48.58	0.01\\
48.59	0.01\\
48.6	0.01\\
48.61	0.01\\
48.62	0.01\\
48.63	0.01\\
48.64	0.01\\
48.65	0.01\\
48.66	0.01\\
48.67	0.01\\
48.68	0.01\\
48.69	0.01\\
48.7	0.01\\
48.71	0.01\\
48.72	0.01\\
48.73	0.01\\
48.74	0.01\\
48.75	0.01\\
48.76	0.01\\
48.77	0.01\\
48.78	0.01\\
48.79	0.01\\
48.8	0.01\\
48.81	0.01\\
48.82	0.01\\
48.83	0.01\\
48.84	0.01\\
48.85	0.01\\
48.86	0.01\\
48.87	0.01\\
48.88	0.01\\
48.89	0.01\\
48.9	0.01\\
48.91	0.01\\
48.92	0.01\\
48.93	0.01\\
48.94	0.01\\
48.95	0.01\\
48.96	0.01\\
48.97	0.01\\
48.98	0.01\\
48.99	0.01\\
49	0.01\\
49.01	0.01\\
49.02	0.01\\
49.03	0.01\\
49.04	0.01\\
49.05	0.01\\
49.06	0.01\\
49.07	0.01\\
49.08	0.01\\
49.09	0.01\\
49.1	0.01\\
49.11	0.01\\
49.12	0.01\\
49.13	0.01\\
49.14	0.01\\
49.15	0.01\\
49.16	0.01\\
49.17	0.01\\
49.18	0.01\\
49.19	0.01\\
49.2	0.01\\
49.21	0.01\\
49.22	0.01\\
49.23	0.01\\
49.24	0.01\\
49.25	0.01\\
49.26	0.01\\
49.27	0.01\\
49.28	0.01\\
49.29	0.01\\
49.3	0.01\\
49.31	0.01\\
49.32	0.01\\
49.33	0.01\\
49.34	0.01\\
49.35	0.01\\
49.36	0.01\\
49.37	0.01\\
49.38	0.01\\
49.39	0.01\\
49.4	0.01\\
49.41	0.01\\
49.42	0.01\\
49.43	0.01\\
49.44	0.01\\
49.45	0.01\\
49.46	0.01\\
49.47	0.01\\
49.48	0.01\\
49.49	0.01\\
49.5	0.01\\
49.51	0.01\\
49.52	0.01\\
49.53	0.01\\
49.54	0.01\\
49.55	0.01\\
49.56	0.01\\
49.57	0.01\\
49.58	0.01\\
49.59	0.01\\
49.6	0.01\\
49.61	0.01\\
49.62	0.01\\
49.63	0.01\\
49.64	0.01\\
49.65	0.01\\
49.66	0.01\\
49.67	0.01\\
49.68	0.01\\
49.69	0.01\\
49.7	0.01\\
49.71	0.01\\
49.72	0.01\\
49.73	0.01\\
49.74	0.01\\
49.75	0.01\\
49.76	0.01\\
49.77	0.01\\
49.78	0.01\\
49.79	0.01\\
49.8	0.01\\
49.81	0.01\\
49.82	0.01\\
49.83	0.01\\
49.84	0.01\\
49.85	0.01\\
49.86	0.01\\
49.87	0.01\\
49.88	0.01\\
49.89	0.01\\
49.9	0.01\\
49.91	0.01\\
49.92	0.01\\
49.93	0.01\\
49.94	0.01\\
49.95	0.01\\
49.96	0.01\\
49.97	0.01\\
49.98	0.01\\
49.99	0.01\\
50	0.01\\
50.01	0.01\\
50.02	0.01\\
50.03	0.01\\
50.04	0.01\\
50.05	0.01\\
50.06	0.01\\
50.07	0.01\\
50.08	0.01\\
50.09	0.01\\
50.1	0.01\\
50.11	0.01\\
50.12	0.01\\
50.13	0.01\\
50.14	0.01\\
50.15	0.01\\
50.16	0.01\\
50.17	0.01\\
50.18	0.01\\
50.19	0.01\\
50.2	0.01\\
50.21	0.01\\
50.22	0.01\\
50.23	0.01\\
50.24	0.01\\
50.25	0.01\\
50.26	0.01\\
50.27	0.01\\
50.28	0.01\\
50.29	0.01\\
50.3	0.01\\
50.31	0.01\\
50.32	0.01\\
50.33	0.01\\
50.34	0.01\\
50.35	0.01\\
50.36	0.01\\
50.37	0.01\\
50.38	0.01\\
50.39	0.01\\
50.4	0.01\\
50.41	0.01\\
50.42	0.01\\
50.43	0.01\\
50.44	0.01\\
50.45	0.01\\
50.46	0.01\\
50.47	0.01\\
50.48	0.01\\
50.49	0.01\\
50.5	0.01\\
50.51	0.01\\
50.52	0.01\\
50.53	0.01\\
50.54	0.01\\
50.55	0.01\\
50.56	0.01\\
50.57	0.01\\
50.58	0.01\\
50.59	0.01\\
50.6	0.01\\
50.61	0.01\\
50.62	0.01\\
50.63	0.01\\
50.64	0.01\\
50.65	0.01\\
50.66	0.01\\
50.67	0.01\\
50.68	0.01\\
50.69	0.01\\
50.7	0.01\\
50.71	0.01\\
50.72	0.01\\
50.73	0.01\\
50.74	0.01\\
50.75	0.01\\
50.76	0.01\\
50.77	0.01\\
50.78	0.01\\
50.79	0.01\\
50.8	0.01\\
50.81	0.01\\
50.82	0.01\\
50.83	0.01\\
50.84	0.01\\
50.85	0.01\\
50.86	0.01\\
50.87	0.01\\
50.88	0.01\\
50.89	0.01\\
50.9	0.01\\
50.91	0.01\\
50.92	0.01\\
50.93	0.01\\
50.94	0.01\\
50.95	0.01\\
50.96	0.01\\
50.97	0.01\\
50.98	0.01\\
50.99	0.01\\
51	0.01\\
51.01	0.01\\
51.02	0.01\\
51.03	0.01\\
51.04	0.01\\
51.05	0.01\\
51.06	0.01\\
51.07	0.01\\
51.08	0.01\\
51.09	0.01\\
51.1	0.01\\
51.11	0.01\\
51.12	0.01\\
51.13	0.01\\
51.14	0.01\\
51.15	0.01\\
51.16	0.01\\
51.17	0.01\\
51.18	0.01\\
51.19	0.01\\
51.2	0.01\\
51.21	0.01\\
51.22	0.01\\
51.23	0.01\\
51.24	0.01\\
51.25	0.01\\
51.26	0.01\\
51.27	0.01\\
51.28	0.01\\
51.29	0.01\\
51.3	0.01\\
51.31	0.01\\
51.32	0.01\\
51.33	0.01\\
51.34	0.01\\
51.35	0.01\\
51.36	0.01\\
51.37	0.01\\
51.38	0.01\\
51.39	0.01\\
51.4	0.01\\
51.41	0.01\\
51.42	0.01\\
51.43	0.01\\
51.44	0.01\\
51.45	0.01\\
51.46	0.01\\
51.47	0.01\\
51.48	0.01\\
51.49	0.01\\
51.5	0.01\\
51.51	0.01\\
51.52	0.01\\
51.53	0.01\\
51.54	0.01\\
51.55	0.01\\
51.56	0.01\\
51.57	0.01\\
51.58	0.01\\
51.59	0.01\\
51.6	0.01\\
51.61	0.01\\
51.62	0.01\\
51.63	0.01\\
51.64	0.01\\
51.65	0.01\\
51.66	0.01\\
51.67	0.01\\
51.68	0.01\\
51.69	0.01\\
51.7	0.01\\
51.71	0.01\\
51.72	0.01\\
51.73	0.01\\
51.74	0.01\\
51.75	0.01\\
51.76	0.01\\
51.77	0.01\\
51.78	0.01\\
51.79	0.01\\
51.8	0.01\\
51.81	0.01\\
51.82	0.01\\
51.83	0.01\\
51.84	0.01\\
51.85	0.01\\
51.86	0.01\\
51.87	0.01\\
51.88	0.01\\
51.89	0.01\\
51.9	0.01\\
51.91	0.01\\
51.92	0.01\\
51.93	0.01\\
51.94	0.01\\
51.95	0.01\\
51.96	0.01\\
51.97	0.01\\
51.98	0.01\\
51.99	0.01\\
52	0.01\\
52.01	0.01\\
52.02	0.01\\
52.03	0.01\\
52.04	0.01\\
52.05	0.01\\
52.06	0.01\\
52.07	0.01\\
52.08	0.01\\
52.09	0.01\\
52.1	0.01\\
52.11	0.01\\
52.12	0.01\\
52.13	0.01\\
52.14	0.01\\
52.15	0.01\\
52.16	0.01\\
52.17	0.01\\
52.18	0.01\\
52.19	0.01\\
52.2	0.01\\
52.21	0.01\\
52.22	0.01\\
52.23	0.01\\
52.24	0.01\\
52.25	0.01\\
52.26	0.01\\
52.27	0.01\\
52.28	0.01\\
52.29	0.01\\
52.3	0.01\\
52.31	0.01\\
52.32	0.01\\
52.33	0.01\\
52.34	0.01\\
52.35	0.01\\
52.36	0.01\\
52.37	0.01\\
52.38	0.01\\
52.39	0.01\\
52.4	0.01\\
52.41	0.01\\
52.42	0.01\\
52.43	0.01\\
52.44	0.01\\
52.45	0.01\\
52.46	0.01\\
52.47	0.01\\
52.48	0.01\\
52.49	0.01\\
52.5	0.01\\
52.51	0.01\\
52.52	0.01\\
52.53	0.01\\
52.54	0.01\\
52.55	0.01\\
52.56	0.01\\
52.57	0.01\\
52.58	0.01\\
52.59	0.01\\
52.6	0.01\\
52.61	0.01\\
52.62	0.01\\
52.63	0.01\\
52.64	0.01\\
52.65	0.01\\
52.66	0.01\\
52.67	0.01\\
52.68	0.01\\
52.69	0.01\\
52.7	0.01\\
52.71	0.01\\
52.72	0.01\\
52.73	0.01\\
52.74	0.01\\
52.75	0.01\\
52.76	0.01\\
52.77	0.01\\
52.78	0.01\\
52.79	0.01\\
52.8	0.01\\
52.81	0.01\\
52.82	0.01\\
52.83	0.01\\
52.84	0.01\\
52.85	0.01\\
52.86	0.01\\
52.87	0.01\\
52.88	0.01\\
52.89	0.01\\
52.9	0.01\\
52.91	0.01\\
52.92	0.01\\
52.93	0.01\\
52.94	0.01\\
52.95	0.01\\
52.96	0.01\\
52.97	0.01\\
52.98	0.01\\
52.99	0.01\\
53	0.01\\
53.01	0.01\\
53.02	0.01\\
53.03	0.01\\
53.04	0.01\\
53.05	0.01\\
53.06	0.01\\
53.07	0.01\\
53.08	0.01\\
53.09	0.01\\
53.1	0.01\\
53.11	0.01\\
53.12	0.01\\
53.13	0.01\\
53.14	0.01\\
53.15	0.01\\
53.16	0.01\\
53.17	0.01\\
53.18	0.01\\
53.19	0.01\\
53.2	0.01\\
53.21	0.01\\
53.22	0.01\\
53.23	0.01\\
53.24	0.01\\
53.25	0.01\\
53.26	0.01\\
53.27	0.01\\
53.28	0.01\\
53.29	0.01\\
53.3	0.01\\
53.31	0.01\\
53.32	0.01\\
53.33	0.01\\
53.34	0.01\\
53.35	0.01\\
53.36	0.01\\
53.37	0.01\\
53.38	0.01\\
53.39	0.01\\
53.4	0.01\\
53.41	0.01\\
53.42	0.01\\
53.43	0.01\\
53.44	0.01\\
53.45	0.01\\
53.46	0.01\\
53.47	0.01\\
53.48	0.01\\
53.49	0.01\\
53.5	0.01\\
53.51	0.01\\
53.52	0.01\\
53.53	0.01\\
53.54	0.01\\
53.55	0.01\\
53.56	0.01\\
53.57	0.01\\
53.58	0.01\\
53.59	0.01\\
53.6	0.01\\
53.61	0.01\\
53.62	0.01\\
53.63	0.01\\
53.64	0.01\\
53.65	0.01\\
53.66	0.01\\
53.67	0.01\\
53.68	0.01\\
53.69	0.01\\
53.7	0.01\\
53.71	0.01\\
53.72	0.01\\
53.73	0.01\\
53.74	0.01\\
53.75	0.01\\
53.76	0.01\\
53.77	0.01\\
53.78	0.01\\
53.79	0.01\\
53.8	0.01\\
53.81	0.01\\
53.82	0.01\\
53.83	0.01\\
53.84	0.01\\
53.85	0.01\\
53.86	0.01\\
53.87	0.01\\
53.88	0.01\\
53.89	0.01\\
53.9	0.01\\
53.91	0.01\\
53.92	0.01\\
53.93	0.01\\
53.94	0.01\\
53.95	0.01\\
53.96	0.01\\
53.97	0.01\\
53.98	0.01\\
53.99	0.01\\
54	0.01\\
54.01	0.01\\
54.02	0.01\\
54.03	0.01\\
54.04	0.01\\
54.05	0.01\\
54.06	0.01\\
54.07	0.01\\
54.08	0.01\\
54.09	0.01\\
54.1	0.01\\
54.11	0.01\\
54.12	0.01\\
54.13	0.01\\
54.14	0.01\\
54.15	0.01\\
54.16	0.01\\
54.17	0.01\\
54.18	0.01\\
54.19	0.01\\
54.2	0.01\\
54.21	0.01\\
54.22	0.01\\
54.23	0.01\\
54.24	0.01\\
54.25	0.01\\
54.26	0.01\\
54.27	0.01\\
54.28	0.01\\
54.29	0.01\\
54.3	0.01\\
54.31	0.01\\
54.32	0.01\\
54.33	0.01\\
54.34	0.01\\
54.35	0.01\\
54.36	0.01\\
54.37	0.01\\
54.38	0.01\\
54.39	0.01\\
54.4	0.01\\
54.41	0.01\\
54.42	0.01\\
54.43	0.01\\
54.44	0.01\\
54.45	0.01\\
54.46	0.01\\
54.47	0.01\\
54.48	0.01\\
54.49	0.01\\
54.5	0.01\\
54.51	0.01\\
54.52	0.01\\
54.53	0.01\\
54.54	0.01\\
54.55	0.01\\
54.56	0.01\\
54.57	0.01\\
54.58	0.01\\
54.59	0.01\\
54.6	0.01\\
54.61	0.01\\
54.62	0.01\\
54.63	0.01\\
54.64	0.01\\
54.65	0.01\\
54.66	0.01\\
54.67	0.01\\
54.68	0.01\\
54.69	0.01\\
54.7	0.01\\
54.71	0.01\\
54.72	0.01\\
54.73	0.01\\
54.74	0.01\\
54.75	0.01\\
54.76	0.01\\
54.77	0.01\\
54.78	0.01\\
54.79	0.01\\
54.8	0.01\\
54.81	0.01\\
54.82	0.01\\
54.83	0.01\\
54.84	0.01\\
54.85	0.01\\
54.86	0.01\\
54.87	0.01\\
54.88	0.01\\
54.89	0.01\\
54.9	0.01\\
54.91	0.01\\
54.92	0.01\\
54.93	0.01\\
54.94	0.01\\
54.95	0.01\\
54.96	0.01\\
54.97	0.01\\
54.98	0.01\\
54.99	0.01\\
55	0.01\\
55.01	0.01\\
55.02	0.01\\
55.03	0.01\\
55.04	0.01\\
55.05	0.01\\
55.06	0.01\\
55.07	0.01\\
55.08	0.01\\
55.09	0.01\\
55.1	0.01\\
55.11	0.01\\
55.12	0.01\\
55.13	0.01\\
55.14	0.01\\
55.15	0.01\\
55.16	0.01\\
55.17	0.01\\
55.18	0.01\\
55.19	0.01\\
55.2	0.01\\
55.21	0.01\\
55.22	0.01\\
55.23	0.01\\
55.24	0.01\\
55.25	0.01\\
55.26	0.01\\
55.27	0.01\\
55.28	0.01\\
55.29	0.01\\
55.3	0.01\\
55.31	0.01\\
55.32	0.01\\
55.33	0.01\\
55.34	0.01\\
55.35	0.01\\
55.36	0.01\\
55.37	0.01\\
55.38	0.01\\
55.39	0.01\\
55.4	0.01\\
55.41	0.01\\
55.42	0.01\\
55.43	0.01\\
55.44	0.01\\
55.45	0.01\\
55.46	0.01\\
55.47	0.01\\
55.48	0.01\\
55.49	0.01\\
55.5	0.01\\
55.51	0.01\\
55.52	0.01\\
55.53	0.01\\
55.54	0.01\\
55.55	0.01\\
55.56	0.01\\
55.57	0.01\\
55.58	0.01\\
55.59	0.01\\
55.6	0.01\\
55.61	0.01\\
55.62	0.01\\
55.63	0.01\\
55.64	0.01\\
55.65	0.01\\
55.66	0.01\\
55.67	0.01\\
55.68	0.01\\
55.69	0.01\\
55.7	0.01\\
55.71	0.01\\
55.72	0.01\\
55.73	0.01\\
55.74	0.01\\
55.75	0.01\\
55.76	0.01\\
55.77	0.01\\
55.78	0.01\\
55.79	0.01\\
55.8	0.01\\
55.81	0.01\\
55.82	0.01\\
55.83	0.01\\
55.84	0.01\\
55.85	0.01\\
55.86	0.01\\
55.87	0.01\\
55.88	0.01\\
55.89	0.01\\
55.9	0.01\\
55.91	0.01\\
55.92	0.01\\
55.93	0.01\\
55.94	0.01\\
55.95	0.01\\
55.96	0.01\\
55.97	0.01\\
55.98	0.01\\
55.99	0.01\\
56	0.01\\
56.01	0.01\\
56.02	0.01\\
56.03	0.01\\
56.04	0.01\\
56.05	0.01\\
56.06	0.01\\
56.07	0.01\\
56.08	0.01\\
56.09	0.01\\
56.1	0.01\\
56.11	0.01\\
56.12	0.01\\
56.13	0.01\\
56.14	0.01\\
56.15	0.01\\
56.16	0.01\\
56.17	0.01\\
56.18	0.01\\
56.19	0.01\\
56.2	0.01\\
56.21	0.01\\
56.22	0.01\\
56.23	0.01\\
56.24	0.01\\
56.25	0.01\\
56.26	0.01\\
56.27	0.01\\
56.28	0.01\\
56.29	0.01\\
56.3	0.01\\
56.31	0.01\\
56.32	0.01\\
56.33	0.01\\
56.34	0.01\\
56.35	0.01\\
56.36	0.01\\
56.37	0.01\\
56.38	0.01\\
56.39	0.01\\
56.4	0.01\\
56.41	0.01\\
56.42	0.01\\
56.43	0.01\\
56.44	0.01\\
56.45	0.01\\
56.46	0.01\\
56.47	0.01\\
56.48	0.01\\
56.49	0.01\\
56.5	0.01\\
56.51	0.01\\
56.52	0.01\\
56.53	0.01\\
56.54	0.01\\
56.55	0.01\\
56.56	0.01\\
56.57	0.01\\
56.58	0.01\\
56.59	0.01\\
56.6	0.01\\
56.61	0.01\\
56.62	0.01\\
56.63	0.01\\
56.64	0.01\\
56.65	0.01\\
56.66	0.01\\
56.67	0.01\\
56.68	0.01\\
56.69	0.01\\
56.7	0.01\\
56.71	0.01\\
56.72	0.01\\
56.73	0.01\\
56.74	0.01\\
56.75	0.01\\
56.76	0.01\\
56.77	0.01\\
56.78	0.01\\
56.79	0.01\\
56.8	0.01\\
56.81	0.01\\
56.82	0.01\\
56.83	0.01\\
56.84	0.01\\
56.85	0.01\\
56.86	0.01\\
56.87	0.01\\
56.88	0.01\\
56.89	0.01\\
56.9	0.01\\
56.91	0.01\\
56.92	0.01\\
56.93	0.01\\
56.94	0.01\\
56.95	0.01\\
56.96	0.01\\
56.97	0.01\\
56.98	0.01\\
56.99	0.01\\
57	0.01\\
57.01	0.01\\
57.02	0.01\\
57.03	0.01\\
57.04	0.01\\
57.05	0.01\\
57.06	0.01\\
57.07	0.01\\
57.08	0.01\\
57.09	0.01\\
57.1	0.01\\
57.11	0.01\\
57.12	0.01\\
57.13	0.01\\
57.14	0.01\\
57.15	0.01\\
57.16	0.01\\
57.17	0.01\\
57.18	0.01\\
57.19	0.01\\
57.2	0.01\\
57.21	0.01\\
57.22	0.01\\
57.23	0.01\\
57.24	0.01\\
57.25	0.01\\
57.26	0.01\\
57.27	0.01\\
57.28	0.01\\
57.29	0.01\\
57.3	0.01\\
57.31	0.01\\
57.32	0.01\\
57.33	0.01\\
57.34	0.01\\
57.35	0.01\\
57.36	0.01\\
57.37	0.01\\
57.38	0.01\\
57.39	0.01\\
57.4	0.01\\
57.41	0.01\\
57.42	0.01\\
57.43	0.01\\
57.44	0.01\\
57.45	0.01\\
57.46	0.01\\
57.47	0.01\\
57.48	0.01\\
57.49	0.01\\
57.5	0.01\\
57.51	0.01\\
57.52	0.01\\
57.53	0.01\\
57.54	0.01\\
57.55	0.01\\
57.56	0.01\\
57.57	0.01\\
57.58	0.01\\
57.59	0.01\\
57.6	0.01\\
57.61	0.01\\
57.62	0.01\\
57.63	0.01\\
57.64	0.01\\
57.65	0.01\\
57.66	0.01\\
57.67	0.01\\
57.68	0.01\\
57.69	0.01\\
57.7	0.01\\
57.71	0.01\\
57.72	0.01\\
57.73	0.01\\
57.74	0.01\\
57.75	0.01\\
57.76	0.01\\
57.77	0.01\\
57.78	0.01\\
57.79	0.01\\
57.8	0.01\\
57.81	0.01\\
57.82	0.01\\
57.83	0.01\\
57.84	0.01\\
57.85	0.01\\
57.86	0.01\\
57.87	0.01\\
57.88	0.01\\
57.89	0.01\\
57.9	0.01\\
57.91	0.01\\
57.92	0.01\\
57.93	0.01\\
57.94	0.01\\
57.95	0.01\\
57.96	0.01\\
57.97	0.01\\
57.98	0.01\\
57.99	0.01\\
58	0.01\\
58.01	0.01\\
58.02	0.01\\
58.03	0.01\\
58.04	0.01\\
58.05	0.01\\
58.06	0.01\\
58.07	0.01\\
58.08	0.01\\
58.09	0.01\\
58.1	0.01\\
58.11	0.01\\
58.12	0.01\\
58.13	0.01\\
58.14	0.01\\
58.15	0.01\\
58.16	0.01\\
58.17	0.01\\
58.18	0.01\\
58.19	0.01\\
58.2	0.01\\
58.21	0.01\\
58.22	0.01\\
58.23	0.01\\
58.24	0.01\\
58.25	0.01\\
58.26	0.01\\
58.27	0.01\\
58.28	0.01\\
58.29	0.01\\
58.3	0.01\\
58.31	0.01\\
58.32	0.01\\
58.33	0.01\\
58.34	0.01\\
58.35	0.01\\
58.36	0.01\\
58.37	0.01\\
58.38	0.01\\
58.39	0.01\\
58.4	0.01\\
58.41	0.01\\
58.42	0.01\\
58.43	0.01\\
58.44	0.01\\
58.45	0.01\\
58.46	0.01\\
58.47	0.01\\
58.48	0.01\\
58.49	0.01\\
58.5	0.01\\
58.51	0.01\\
58.52	0.01\\
58.53	0.01\\
58.54	0.01\\
58.55	0.01\\
58.56	0.01\\
58.57	0.01\\
58.58	0.01\\
58.59	0.01\\
58.6	0.01\\
58.61	0.01\\
58.62	0.01\\
58.63	0.01\\
58.64	0.01\\
58.65	0.01\\
58.66	0.01\\
58.67	0.01\\
58.68	0.01\\
58.69	0.01\\
58.7	0.01\\
58.71	0.01\\
58.72	0.01\\
58.73	0.01\\
58.74	0.01\\
58.75	0.01\\
58.76	0.01\\
58.77	0.01\\
58.78	0.01\\
58.79	0.01\\
58.8	0.01\\
58.81	0.01\\
58.82	0.01\\
58.83	0.01\\
58.84	0.01\\
58.85	0.01\\
58.86	0.01\\
58.87	0.01\\
58.88	0.01\\
58.89	0.01\\
58.9	0.01\\
58.91	0.01\\
58.92	0.01\\
58.93	0.01\\
58.94	0.01\\
58.95	0.01\\
58.96	0.01\\
58.97	0.01\\
58.98	0.01\\
58.99	0.01\\
59	0.01\\
59.01	0.01\\
59.02	0.01\\
59.03	0.01\\
59.04	0.01\\
59.05	0.01\\
59.06	0.01\\
59.07	0.01\\
59.08	0.01\\
59.09	0.01\\
59.1	0.01\\
59.11	0.01\\
59.12	0.01\\
59.13	0.01\\
59.14	0.01\\
59.15	0.01\\
59.16	0.01\\
59.17	0.01\\
59.18	0.01\\
59.19	0.01\\
59.2	0.01\\
59.21	0.01\\
59.22	0.01\\
59.23	0.01\\
59.24	0.01\\
59.25	0.01\\
59.26	0.01\\
59.27	0.01\\
59.28	0.01\\
59.29	0.01\\
59.3	0.01\\
59.31	0.01\\
59.32	0.01\\
59.33	0.01\\
59.34	0.01\\
59.35	0.01\\
59.36	0.01\\
59.37	0.01\\
59.38	0.01\\
59.39	0.01\\
59.4	0.01\\
59.41	0.01\\
59.42	0.01\\
59.43	0.01\\
59.44	0.01\\
59.45	0.01\\
59.46	0.01\\
59.47	0.01\\
59.48	0.01\\
59.49	0.01\\
59.5	0.01\\
59.51	0.01\\
59.52	0.01\\
59.53	0.01\\
59.54	0.01\\
59.55	0.01\\
59.56	0.01\\
59.57	0.01\\
59.58	0.01\\
59.59	0.01\\
59.6	0.01\\
59.61	0.01\\
59.62	0.01\\
59.63	0.01\\
59.64	0.01\\
59.65	0.01\\
59.66	0.01\\
59.67	0.01\\
59.68	0.01\\
59.69	0.01\\
59.7	0.01\\
59.71	0.01\\
59.72	0.01\\
59.73	0.01\\
59.74	0.01\\
59.75	0.01\\
59.76	0.01\\
59.77	0.01\\
59.78	0.01\\
59.79	0.01\\
59.8	0.01\\
59.81	0.01\\
59.82	0.01\\
59.83	0.01\\
59.84	0.01\\
59.85	0.01\\
59.86	0.01\\
59.87	0.01\\
59.88	0.01\\
59.89	0.01\\
59.9	0.01\\
59.91	0.01\\
59.92	0.01\\
59.93	0.01\\
59.94	0.01\\
59.95	0.01\\
59.96	0.01\\
59.97	0.01\\
59.98	0.01\\
59.99	0.01\\
60	0.01\\
60.01	0.01\\
60.02	0.01\\
60.03	0.01\\
60.04	0.01\\
60.05	0.01\\
60.06	0.01\\
60.07	0.01\\
60.08	0.01\\
60.09	0.01\\
60.1	0.01\\
60.11	0.01\\
60.12	0.01\\
60.13	0.01\\
60.14	0.01\\
60.15	0.01\\
60.16	0.01\\
60.17	0.01\\
60.18	0.01\\
60.19	0.01\\
60.2	0.01\\
60.21	0.01\\
60.22	0.01\\
60.23	0.01\\
60.24	0.01\\
60.25	0.01\\
60.26	0.01\\
60.27	0.01\\
60.28	0.01\\
60.29	0.01\\
60.3	0.01\\
60.31	0.01\\
60.32	0.01\\
60.33	0.01\\
60.34	0.01\\
60.35	0.01\\
60.36	0.01\\
60.37	0.01\\
60.38	0.01\\
60.39	0.01\\
60.4	0.01\\
60.41	0.01\\
60.42	0.01\\
60.43	0.01\\
60.44	0.01\\
60.45	0.01\\
60.46	0.01\\
60.47	0.01\\
60.48	0.01\\
60.49	0.01\\
60.5	0.01\\
60.51	0.01\\
60.52	0.01\\
60.53	0.01\\
60.54	0.01\\
60.55	0.01\\
60.56	0.01\\
60.57	0.01\\
60.58	0.01\\
60.59	0.01\\
60.6	0.01\\
60.61	0.01\\
60.62	0.01\\
60.63	0.01\\
60.64	0.01\\
60.65	0.01\\
60.66	0.01\\
60.67	0.01\\
60.68	0.01\\
60.69	0.01\\
60.7	0.01\\
60.71	0.01\\
60.72	0.01\\
60.73	0.01\\
60.74	0.01\\
60.75	0.01\\
60.76	0.01\\
60.77	0.01\\
60.78	0.01\\
60.79	0.01\\
60.8	0.01\\
60.81	0.01\\
60.82	0.01\\
60.83	0.01\\
60.84	0.01\\
60.85	0.01\\
60.86	0.01\\
60.87	0.01\\
60.88	0.01\\
60.89	0.01\\
60.9	0.01\\
60.91	0.01\\
60.92	0.01\\
60.93	0.01\\
60.94	0.01\\
60.95	0.01\\
60.96	0.01\\
60.97	0.01\\
60.98	0.01\\
60.99	0.01\\
61	0.01\\
61.01	0.01\\
61.02	0.01\\
61.03	0.01\\
61.04	0.01\\
61.05	0.01\\
61.06	0.01\\
61.07	0.01\\
61.08	0.01\\
61.09	0.01\\
61.1	0.01\\
61.11	0.01\\
61.12	0.01\\
61.13	0.01\\
61.14	0.01\\
61.15	0.01\\
61.16	0.01\\
61.17	0.01\\
61.18	0.01\\
61.19	0.01\\
61.2	0.01\\
61.21	0.01\\
61.22	0.01\\
61.23	0.01\\
61.24	0.01\\
61.25	0.01\\
61.26	0.01\\
61.27	0.01\\
61.28	0.01\\
61.29	0.01\\
61.3	0.01\\
61.31	0.01\\
61.32	0.01\\
61.33	0.01\\
61.34	0.01\\
61.35	0.01\\
61.36	0.01\\
61.37	0.01\\
61.38	0.01\\
61.39	0.01\\
61.4	0.01\\
61.41	0.01\\
61.42	0.01\\
61.43	0.01\\
61.44	0.01\\
61.45	0.01\\
61.46	0.01\\
61.47	0.01\\
61.48	0.01\\
61.49	0.01\\
61.5	0.01\\
61.51	0.01\\
61.52	0.01\\
61.53	0.01\\
61.54	0.01\\
61.55	0.01\\
61.56	0.01\\
61.57	0.01\\
61.58	0.01\\
61.59	0.01\\
61.6	0.01\\
61.61	0.01\\
61.62	0.01\\
61.63	0.01\\
61.64	0.01\\
61.65	0.01\\
61.66	0.01\\
61.67	0.01\\
61.68	0.01\\
61.69	0.01\\
61.7	0.01\\
61.71	0.01\\
61.72	0.01\\
61.73	0.01\\
61.74	0.01\\
61.75	0.01\\
61.76	0.01\\
61.77	0.01\\
61.78	0.01\\
61.79	0.01\\
61.8	0.01\\
61.81	0.01\\
61.82	0.01\\
61.83	0.01\\
61.84	0.01\\
61.85	0.01\\
61.86	0.01\\
61.87	0.01\\
61.88	0.01\\
61.89	0.01\\
61.9	0.01\\
61.91	0.01\\
61.92	0.01\\
61.93	0.01\\
61.94	0.01\\
61.95	0.01\\
61.96	0.01\\
61.97	0.01\\
61.98	0.01\\
61.99	0.01\\
62	0.01\\
62.01	0.01\\
62.02	0.01\\
62.03	0.01\\
62.04	0.01\\
62.05	0.01\\
62.06	0.01\\
62.07	0.01\\
62.08	0.01\\
62.09	0.01\\
62.1	0.01\\
62.11	0.01\\
62.12	0.01\\
62.13	0.01\\
62.14	0.01\\
62.15	0.01\\
62.16	0.01\\
62.17	0.01\\
62.18	0.01\\
62.19	0.01\\
62.2	0.01\\
62.21	0.01\\
62.22	0.01\\
62.23	0.01\\
62.24	0.01\\
62.25	0.01\\
62.26	0.01\\
62.27	0.01\\
62.28	0.01\\
62.29	0.01\\
62.3	0.01\\
62.31	0.01\\
62.32	0.01\\
62.33	0.01\\
62.34	0.01\\
62.35	0.01\\
62.36	0.01\\
62.37	0.01\\
62.38	0.01\\
62.39	0.01\\
62.4	0.01\\
62.41	0.01\\
62.42	0.01\\
62.43	0.01\\
62.44	0.01\\
62.45	0.01\\
62.46	0.01\\
62.47	0.01\\
62.48	0.01\\
62.49	0.01\\
62.5	0.01\\
62.51	0.01\\
62.52	0.01\\
62.53	0.01\\
62.54	0.01\\
62.55	0.01\\
62.56	0.01\\
62.57	0.01\\
62.58	0.01\\
62.59	0.01\\
62.6	0.01\\
62.61	0.01\\
62.62	0.01\\
62.63	0.01\\
62.64	0.01\\
62.65	0.01\\
62.66	0.01\\
62.67	0.01\\
62.68	0.01\\
62.69	0.01\\
62.7	0.01\\
62.71	0.01\\
62.72	0.01\\
62.73	0.01\\
62.74	0.01\\
62.75	0.01\\
62.76	0.01\\
62.77	0.01\\
62.78	0.01\\
62.79	0.01\\
62.8	0.01\\
62.81	0.01\\
62.82	0.01\\
62.83	0.01\\
62.84	0.01\\
62.85	0.01\\
62.86	0.01\\
62.87	0.01\\
62.88	0.01\\
62.89	0.01\\
62.9	0.01\\
62.91	0.01\\
62.92	0.01\\
62.93	0.01\\
62.94	0.01\\
62.95	0.01\\
62.96	0.01\\
62.97	0.01\\
62.98	0.01\\
62.99	0.01\\
63	0.01\\
63.01	0.01\\
63.02	0.01\\
63.03	0.01\\
63.04	0.01\\
63.05	0.01\\
63.06	0.01\\
63.07	0.01\\
63.08	0.01\\
63.09	0.01\\
63.1	0.01\\
63.11	0.01\\
63.12	0.01\\
63.13	0.01\\
63.14	0.01\\
63.15	0.01\\
63.16	0.01\\
63.17	0.01\\
63.18	0.01\\
63.19	0.01\\
63.2	0.01\\
63.21	0.01\\
63.22	0.01\\
63.23	0.01\\
63.24	0.01\\
63.25	0.01\\
63.26	0.01\\
63.27	0.01\\
63.28	0.01\\
63.29	0.01\\
63.3	0.01\\
63.31	0.01\\
63.32	0.01\\
63.33	0.01\\
63.34	0.01\\
63.35	0.01\\
63.36	0.01\\
63.37	0.01\\
63.38	0.01\\
63.39	0.01\\
63.4	0.01\\
63.41	0.01\\
63.42	0.01\\
63.43	0.01\\
63.44	0.01\\
63.45	0.01\\
63.46	0.01\\
63.47	0.01\\
63.48	0.01\\
63.49	0.01\\
63.5	0.01\\
63.51	0.01\\
63.52	0.01\\
63.53	0.01\\
63.54	0.01\\
63.55	0.01\\
63.56	0.01\\
63.57	0.01\\
63.58	0.01\\
63.59	0.01\\
63.6	0.01\\
63.61	0.01\\
63.62	0.01\\
63.63	0.01\\
63.64	0.01\\
63.65	0.01\\
63.66	0.01\\
63.67	0.01\\
63.68	0.01\\
63.69	0.01\\
63.7	0.01\\
63.71	0.01\\
63.72	0.01\\
63.73	0.01\\
63.74	0.01\\
63.75	0.01\\
63.76	0.01\\
63.77	0.01\\
63.78	0.01\\
63.79	0.01\\
63.8	0.01\\
63.81	0.01\\
63.82	0.01\\
63.83	0.01\\
63.84	0.01\\
63.85	0.01\\
63.86	0.01\\
63.87	0.01\\
63.88	0.01\\
63.89	0.01\\
63.9	0.01\\
63.91	0.01\\
63.92	0.01\\
63.93	0.01\\
63.94	0.01\\
63.95	0.01\\
63.96	0.01\\
63.97	0.01\\
63.98	0.01\\
63.99	0.01\\
64	0.01\\
64.01	0.01\\
64.02	0.01\\
64.03	0.01\\
64.04	0.01\\
64.05	0.01\\
64.06	0.01\\
64.07	0.01\\
64.08	0.01\\
64.09	0.01\\
64.1	0.01\\
64.11	0.01\\
64.12	0.01\\
64.13	0.01\\
64.14	0.01\\
64.15	0.01\\
64.16	0.01\\
64.17	0.01\\
64.18	0.01\\
64.19	0.01\\
64.2	0.01\\
64.21	0.01\\
64.22	0.01\\
64.23	0.01\\
64.24	0.01\\
64.25	0.01\\
64.26	0.01\\
64.27	0.01\\
64.28	0.01\\
64.29	0.01\\
64.3	0.01\\
64.31	0.01\\
64.32	0.01\\
64.33	0.01\\
64.34	0.01\\
64.35	0.01\\
64.36	0.01\\
64.37	0.01\\
64.38	0.01\\
64.39	0.01\\
64.4	0.01\\
64.41	0.01\\
64.42	0.01\\
64.43	0.01\\
64.44	0.01\\
64.45	0.01\\
64.46	0.01\\
64.47	0.01\\
64.48	0.01\\
64.49	0.01\\
64.5	0.01\\
64.51	0.01\\
64.52	0.01\\
64.53	0.01\\
64.54	0.01\\
64.55	0.01\\
64.56	0.01\\
64.57	0.01\\
64.58	0.01\\
64.59	0.01\\
64.6	0.01\\
64.61	0.01\\
64.62	0.01\\
64.63	0.01\\
64.64	0.01\\
64.65	0.01\\
64.66	0.01\\
64.67	0.01\\
64.68	0.01\\
64.69	0.01\\
64.7	0.01\\
64.71	0.01\\
64.72	0.01\\
64.73	0.01\\
64.74	0.01\\
64.75	0.01\\
64.76	0.01\\
64.77	0.01\\
64.78	0.01\\
64.79	0.01\\
64.8	0.01\\
64.81	0.01\\
64.82	0.01\\
64.83	0.01\\
64.84	0.01\\
64.85	0.01\\
64.86	0.01\\
64.87	0.01\\
64.88	0.01\\
64.89	0.01\\
64.9	0.01\\
64.91	0.01\\
64.92	0.01\\
64.93	0.01\\
64.94	0.01\\
64.95	0.01\\
64.96	0.01\\
64.97	0.01\\
64.98	0.01\\
64.99	0.01\\
65	0.01\\
65.01	0.01\\
65.02	0.01\\
65.03	0.01\\
65.04	0.01\\
65.05	0.01\\
65.06	0.01\\
65.07	0.01\\
65.08	0.01\\
65.09	0.01\\
65.1	0.01\\
65.11	0.01\\
65.12	0.01\\
65.13	0.01\\
65.14	0.01\\
65.15	0.01\\
65.16	0.01\\
65.17	0.01\\
65.18	0.01\\
65.19	0.01\\
65.2	0.01\\
65.21	0.01\\
65.22	0.01\\
65.23	0.01\\
65.24	0.01\\
65.25	0.01\\
65.26	0.01\\
65.27	0.01\\
65.28	0.01\\
65.29	0.01\\
65.3	0.01\\
65.31	0.01\\
65.32	0.01\\
65.33	0.01\\
65.34	0.01\\
65.35	0.01\\
65.36	0.01\\
65.37	0.01\\
65.38	0.01\\
65.39	0.01\\
65.4	0.01\\
65.41	0.01\\
65.42	0.01\\
65.43	0.01\\
65.44	0.01\\
65.45	0.01\\
65.46	0.01\\
65.47	0.01\\
65.48	0.01\\
65.49	0.01\\
65.5	0.01\\
65.51	0.01\\
65.52	0.01\\
65.53	0.01\\
65.54	0.01\\
65.55	0.01\\
65.56	0.01\\
65.57	0.01\\
65.58	0.01\\
65.59	0.01\\
65.6	0.01\\
65.61	0.01\\
65.62	0.01\\
65.63	0.01\\
65.64	0.01\\
65.65	0.01\\
65.66	0.01\\
65.67	0.01\\
65.68	0.01\\
65.69	0.01\\
65.7	0.01\\
65.71	0.01\\
65.72	0.01\\
65.73	0.01\\
65.74	0.01\\
65.75	0.01\\
65.76	0.01\\
65.77	0.01\\
65.78	0.01\\
65.79	0.01\\
65.8	0.01\\
65.81	0.01\\
65.82	0.01\\
65.83	0.01\\
65.84	0.01\\
65.85	0.01\\
65.86	0.01\\
65.87	0.01\\
65.88	0.01\\
65.89	0.01\\
65.9	0.01\\
65.91	0.01\\
65.92	0.01\\
65.93	0.01\\
65.94	0.01\\
65.95	0.01\\
65.96	0.01\\
65.97	0.01\\
65.98	0.01\\
65.99	0.01\\
66	0.01\\
66.01	0.01\\
66.02	0.01\\
66.03	0.01\\
66.04	0.01\\
66.05	0.01\\
66.06	0.01\\
66.07	0.01\\
66.08	0.01\\
66.09	0.01\\
66.1	0.01\\
66.11	0.01\\
66.12	0.01\\
66.13	0.01\\
66.14	0.01\\
66.15	0.01\\
66.16	0.01\\
66.17	0.01\\
66.18	0.01\\
66.19	0.01\\
66.2	0.01\\
66.21	0.01\\
66.22	0.01\\
66.23	0.01\\
66.24	0.01\\
66.25	0.01\\
66.26	0.01\\
66.27	0.01\\
66.28	0.01\\
66.29	0.01\\
66.3	0.01\\
66.31	0.01\\
66.32	0.01\\
66.33	0.01\\
66.34	0.01\\
66.35	0.01\\
66.36	0.01\\
66.37	0.01\\
66.38	0.01\\
66.39	0.01\\
66.4	0.01\\
66.41	0.01\\
66.42	0.01\\
66.43	0.01\\
66.44	0.01\\
66.45	0.01\\
66.46	0.01\\
66.47	0.01\\
66.48	0.01\\
66.49	0.01\\
66.5	0.01\\
66.51	0.01\\
66.52	0.01\\
66.53	0.01\\
66.54	0.01\\
66.55	0.01\\
66.56	0.01\\
66.57	0.01\\
66.58	0.01\\
66.59	0.01\\
66.6	0.01\\
66.61	0.01\\
66.62	0.01\\
66.63	0.01\\
66.64	0.01\\
66.65	0.01\\
66.66	0.01\\
66.67	0.01\\
66.68	0.01\\
66.69	0.01\\
66.7	0.01\\
66.71	0.01\\
66.72	0.01\\
66.73	0.01\\
66.74	0.01\\
66.75	0.01\\
66.76	0.01\\
66.77	0.01\\
66.78	0.01\\
66.79	0.01\\
66.8	0.01\\
66.81	0.01\\
66.82	0.01\\
66.83	0.01\\
66.84	0.01\\
66.85	0.01\\
66.86	0.01\\
66.87	0.01\\
66.88	0.01\\
66.89	0.01\\
66.9	0.01\\
66.91	0.01\\
66.92	0.01\\
66.93	0.01\\
66.94	0.01\\
66.95	0.01\\
66.96	0.01\\
66.97	0.01\\
66.98	0.01\\
66.99	0.01\\
67	0.01\\
67.01	0.01\\
67.02	0.01\\
67.03	0.01\\
67.04	0.01\\
67.05	0.01\\
67.06	0.01\\
67.07	0.01\\
67.08	0.01\\
67.09	0.01\\
67.1	0.01\\
67.11	0.01\\
67.12	0.01\\
67.13	0.01\\
67.14	0.01\\
67.15	0.01\\
67.16	0.01\\
67.17	0.01\\
67.18	0.01\\
67.19	0.01\\
67.2	0.01\\
67.21	0.01\\
67.22	0.01\\
67.23	0.01\\
67.24	0.01\\
67.25	0.01\\
67.26	0.01\\
67.27	0.01\\
67.28	0.01\\
67.29	0.01\\
67.3	0.01\\
67.31	0.01\\
67.32	0.01\\
67.33	0.01\\
67.34	0.01\\
67.35	0.01\\
67.36	0.01\\
67.37	0.01\\
67.38	0.01\\
67.39	0.01\\
67.4	0.01\\
67.41	0.01\\
67.42	0.01\\
67.43	0.01\\
67.44	0.01\\
67.45	0.01\\
67.46	0.01\\
67.47	0.01\\
67.48	0.01\\
67.49	0.01\\
67.5	0.01\\
67.51	0.01\\
67.52	0.01\\
67.53	0.01\\
67.54	0.01\\
67.55	0.01\\
67.56	0.01\\
67.57	0.01\\
67.58	0.01\\
67.59	0.01\\
67.6	0.01\\
67.61	0.01\\
67.62	0.01\\
67.63	0.01\\
67.64	0.01\\
67.65	0.01\\
67.66	0.01\\
67.67	0.01\\
67.68	0.01\\
67.69	0.01\\
67.7	0.01\\
67.71	0.01\\
67.72	0.01\\
67.73	0.01\\
67.74	0.01\\
67.75	0.01\\
67.76	0.01\\
67.77	0.01\\
67.78	0.01\\
67.79	0.01\\
67.8	0.01\\
67.81	0.01\\
67.82	0.01\\
67.83	0.01\\
67.84	0.01\\
67.85	0.01\\
67.86	0.01\\
67.87	0.01\\
67.88	0.01\\
67.89	0.01\\
67.9	0.01\\
67.91	0.01\\
67.92	0.01\\
67.93	0.01\\
67.94	0.01\\
67.95	0.01\\
67.96	0.01\\
67.97	0.01\\
67.98	0.01\\
67.99	0.01\\
68	0.01\\
68.01	0.01\\
68.02	0.01\\
68.03	0.01\\
68.04	0.01\\
68.05	0.01\\
68.06	0.01\\
68.07	0.01\\
68.08	0.01\\
68.09	0.01\\
68.1	0.01\\
68.11	0.01\\
68.12	0.01\\
68.13	0.01\\
68.14	0.01\\
68.15	0.01\\
68.16	0.01\\
68.17	0.01\\
68.18	0.01\\
68.19	0.01\\
68.2	0.01\\
68.21	0.01\\
68.22	0.01\\
68.23	0.01\\
68.24	0.01\\
68.25	0.01\\
68.26	0.01\\
68.27	0.01\\
68.28	0.01\\
68.29	0.01\\
68.3	0.01\\
68.31	0.01\\
68.32	0.01\\
68.33	0.01\\
68.34	0.01\\
68.35	0.01\\
68.36	0.01\\
68.37	0.01\\
68.38	0.01\\
68.39	0.01\\
68.4	0.01\\
68.41	0.01\\
68.42	0.01\\
68.43	0.01\\
68.44	0.01\\
68.45	0.01\\
68.46	0.01\\
68.47	0.01\\
68.48	0.01\\
68.49	0.01\\
68.5	0.01\\
68.51	0.01\\
68.52	0.01\\
68.53	0.01\\
68.54	0.01\\
68.55	0.01\\
68.56	0.01\\
68.57	0.01\\
68.58	0.01\\
68.59	0.01\\
68.6	0.01\\
68.61	0.01\\
68.62	0.01\\
68.63	0.01\\
68.64	0.01\\
68.65	0.01\\
68.66	0.01\\
68.67	0.01\\
68.68	0.01\\
68.69	0.01\\
68.7	0.01\\
68.71	0.01\\
68.72	0.01\\
68.73	0.01\\
68.74	0.01\\
68.75	0.01\\
68.76	0.01\\
68.77	0.01\\
68.78	0.01\\
68.79	0.01\\
68.8	0.01\\
68.81	0.01\\
68.82	0.01\\
68.83	0.01\\
68.84	0.01\\
68.85	0.01\\
68.86	0.01\\
68.87	0.01\\
68.88	0.01\\
68.89	0.01\\
68.9	0.01\\
68.91	0.01\\
68.92	0.01\\
68.93	0.01\\
68.94	0.01\\
68.95	0.01\\
68.96	0.01\\
68.97	0.01\\
68.98	0.01\\
68.99	0.01\\
69	0.01\\
69.01	0.01\\
69.02	0.01\\
69.03	0.01\\
69.04	0.01\\
69.05	0.01\\
69.06	0.01\\
69.07	0.01\\
69.08	0.01\\
69.09	0.01\\
69.1	0.01\\
69.11	0.01\\
69.12	0.01\\
69.13	0.01\\
69.14	0.01\\
69.15	0.01\\
69.16	0.01\\
69.17	0.01\\
69.18	0.01\\
69.19	0.01\\
69.2	0.01\\
69.21	0.01\\
69.22	0.01\\
69.23	0.01\\
69.24	0.01\\
69.25	0.01\\
69.26	0.01\\
69.27	0.01\\
69.28	0.01\\
69.29	0.01\\
69.3	0.01\\
69.31	0.01\\
69.32	0.01\\
69.33	0.01\\
69.34	0.01\\
69.35	0.01\\
69.36	0.01\\
69.37	0.01\\
69.38	0.01\\
69.39	0.01\\
69.4	0.01\\
69.41	0.01\\
69.42	0.01\\
69.43	0.01\\
69.44	0.01\\
69.45	0.01\\
69.46	0.01\\
69.47	0.01\\
69.48	0.01\\
69.49	0.01\\
69.5	0.01\\
69.51	0.01\\
69.52	0.01\\
69.53	0.01\\
69.54	0.01\\
69.55	0.01\\
69.56	0.01\\
69.57	0.01\\
69.58	0.01\\
69.59	0.01\\
69.6	0.01\\
69.61	0.01\\
69.62	0.01\\
69.63	0.01\\
69.64	0.01\\
69.65	0.01\\
69.66	0.01\\
69.67	0.01\\
69.68	0.01\\
69.69	0.01\\
69.7	0.01\\
69.71	0.01\\
69.72	0.01\\
69.73	0.01\\
69.74	0.01\\
69.75	0.01\\
69.76	0.01\\
69.77	0.01\\
69.78	0.01\\
69.79	0.01\\
69.8	0.01\\
69.81	0.01\\
69.82	0.01\\
69.83	0.01\\
69.84	0.01\\
69.85	0.01\\
69.86	0.01\\
69.87	0.01\\
69.88	0.01\\
69.89	0.01\\
69.9	0.01\\
69.91	0.01\\
69.92	0.01\\
69.93	0.01\\
69.94	0.01\\
69.95	0.01\\
69.96	0.01\\
69.97	0.01\\
69.98	0.01\\
69.99	0.01\\
70	0.01\\
70.01	0.01\\
70.02	0.01\\
70.03	0.01\\
70.04	0.01\\
70.05	0.01\\
70.06	0.01\\
70.07	0.01\\
70.08	0.01\\
70.09	0.01\\
70.1	0.01\\
70.11	0.01\\
70.12	0.01\\
70.13	0.01\\
70.14	0.01\\
70.15	0.01\\
70.16	0.01\\
70.17	0.01\\
70.18	0.01\\
70.19	0.01\\
70.2	0.01\\
70.21	0.01\\
70.22	0.01\\
70.23	0.01\\
70.24	0.01\\
70.25	0.01\\
70.26	0.01\\
70.27	0.01\\
70.28	0.01\\
70.29	0.01\\
70.3	0.01\\
70.31	0.01\\
70.32	0.01\\
70.33	0.01\\
70.34	0.01\\
70.35	0.01\\
70.36	0.01\\
70.37	0.01\\
70.38	0.01\\
70.39	0.01\\
70.4	0.01\\
70.41	0.01\\
70.42	0.01\\
70.43	0.01\\
70.44	0.01\\
70.45	0.01\\
70.46	0.01\\
70.47	0.01\\
70.48	0.01\\
70.49	0.01\\
70.5	0.01\\
70.51	0.01\\
70.52	0.01\\
70.53	0.01\\
70.54	0.01\\
70.55	0.01\\
70.56	0.01\\
70.57	0.01\\
70.58	0.01\\
70.59	0.01\\
70.6	0.01\\
70.61	0.01\\
70.62	0.01\\
70.63	0.01\\
70.64	0.01\\
70.65	0.01\\
70.66	0.01\\
70.67	0.01\\
70.68	0.01\\
70.69	0.01\\
70.7	0.01\\
70.71	0.01\\
70.72	0.01\\
70.73	0.01\\
70.74	0.01\\
70.75	0.01\\
70.76	0.01\\
70.77	0.01\\
70.78	0.01\\
70.79	0.01\\
70.8	0.01\\
70.81	0.01\\
70.82	0.01\\
70.83	0.01\\
70.84	0.01\\
70.85	0.01\\
70.86	0.01\\
70.87	0.01\\
70.88	0.01\\
70.89	0.01\\
70.9	0.01\\
70.91	0.01\\
70.92	0.01\\
70.93	0.01\\
70.94	0.01\\
70.95	0.01\\
70.96	0.01\\
70.97	0.01\\
70.98	0.01\\
70.99	0.01\\
71	0.01\\
71.01	0.01\\
71.02	0.01\\
71.03	0.01\\
71.04	0.01\\
71.05	0.01\\
71.06	0.01\\
71.07	0.01\\
71.08	0.01\\
71.09	0.01\\
71.1	0.01\\
71.11	0.01\\
71.12	0.01\\
71.13	0.01\\
71.14	0.01\\
71.15	0.01\\
71.16	0.01\\
71.17	0.01\\
71.18	0.01\\
71.19	0.01\\
71.2	0.01\\
71.21	0.01\\
71.22	0.01\\
71.23	0.01\\
71.24	0.01\\
71.25	0.01\\
71.26	0.01\\
71.27	0.01\\
71.28	0.01\\
71.29	0.01\\
71.3	0.01\\
71.31	0.01\\
71.32	0.01\\
71.33	0.01\\
71.34	0.01\\
71.35	0.01\\
71.36	0.01\\
71.37	0.01\\
71.38	0.01\\
71.39	0.01\\
71.4	0.01\\
71.41	0.01\\
71.42	0.01\\
71.43	0.01\\
71.44	0.01\\
71.45	0.01\\
71.46	0.01\\
71.47	0.01\\
71.48	0.01\\
71.49	0.01\\
71.5	0.01\\
71.51	0.01\\
71.52	0.01\\
71.53	0.01\\
71.54	0.01\\
71.55	0.01\\
71.56	0.01\\
71.57	0.01\\
71.58	0.01\\
71.59	0.01\\
71.6	0.01\\
71.61	0.01\\
71.62	0.01\\
71.63	0.01\\
71.64	0.01\\
71.65	0.01\\
71.66	0.01\\
71.67	0.01\\
71.68	0.01\\
71.69	0.01\\
71.7	0.01\\
71.71	0.01\\
71.72	0.01\\
71.73	0.01\\
71.74	0.01\\
71.75	0.01\\
71.76	0.01\\
71.77	0.01\\
71.78	0.01\\
71.79	0.01\\
71.8	0.01\\
71.81	0.01\\
71.82	0.01\\
71.83	0.01\\
71.84	0.01\\
71.85	0.01\\
71.86	0.01\\
71.87	0.01\\
71.88	0.01\\
71.89	0.01\\
71.9	0.01\\
71.91	0.01\\
71.92	0.01\\
71.93	0.01\\
71.94	0.01\\
71.95	0.01\\
71.96	0.01\\
71.97	0.01\\
71.98	0.01\\
71.99	0.01\\
72	0.01\\
72.01	0.01\\
72.02	0.01\\
72.03	0.01\\
72.04	0.01\\
72.05	0.01\\
72.06	0.01\\
72.07	0.01\\
72.08	0.01\\
72.09	0.01\\
72.1	0.01\\
72.11	0.01\\
72.12	0.01\\
72.13	0.01\\
72.14	0.01\\
72.15	0.01\\
72.16	0.01\\
72.17	0.01\\
72.18	0.01\\
72.19	0.01\\
72.2	0.01\\
72.21	0.01\\
72.22	0.01\\
72.23	0.01\\
72.24	0.01\\
72.25	0.01\\
72.26	0.01\\
72.27	0.01\\
72.28	0.01\\
72.29	0.01\\
72.3	0.01\\
72.31	0.01\\
72.32	0.01\\
72.33	0.01\\
72.34	0.01\\
72.35	0.01\\
72.36	0.01\\
72.37	0.01\\
72.38	0.01\\
72.39	0.01\\
72.4	0.01\\
72.41	0.01\\
72.42	0.01\\
72.43	0.01\\
72.44	0.01\\
72.45	0.01\\
72.46	0.01\\
72.47	0.01\\
72.48	0.01\\
72.49	0.01\\
72.5	0.01\\
72.51	0.01\\
72.52	0.01\\
72.53	0.01\\
72.54	0.01\\
72.55	0.01\\
72.56	0.01\\
72.57	0.01\\
72.58	0.01\\
72.59	0.01\\
72.6	0.01\\
72.61	0.01\\
72.62	0.01\\
72.63	0.01\\
72.64	0.01\\
72.65	0.01\\
72.66	0.01\\
72.67	0.01\\
72.68	0.01\\
72.69	0.01\\
72.7	0.01\\
72.71	0.01\\
72.72	0.01\\
72.73	0.01\\
72.74	0.01\\
72.75	0.01\\
72.76	0.01\\
72.77	0.01\\
72.78	0.01\\
72.79	0.01\\
72.8	0.01\\
72.81	0.01\\
72.82	0.01\\
72.83	0.01\\
72.84	0.01\\
72.85	0.01\\
72.86	0.01\\
72.87	0.01\\
72.88	0.01\\
72.89	0.01\\
72.9	0.01\\
72.91	0.01\\
72.92	0.01\\
72.93	0.01\\
72.94	0.01\\
72.95	0.01\\
72.96	0.01\\
72.97	0.01\\
72.98	0.01\\
72.99	0.01\\
73	0.01\\
73.01	0.01\\
73.02	0.01\\
73.03	0.01\\
73.04	0.01\\
73.05	0.01\\
73.06	0.01\\
73.07	0.01\\
73.08	0.01\\
73.09	0.01\\
73.1	0.01\\
73.11	0.01\\
73.12	0.01\\
73.13	0.01\\
73.14	0.01\\
73.15	0.01\\
73.16	0.01\\
73.17	0.01\\
73.18	0.01\\
73.19	0.01\\
73.2	0.01\\
73.21	0.01\\
73.22	0.01\\
73.23	0.01\\
73.24	0.01\\
73.25	0.01\\
73.26	0.01\\
73.27	0.01\\
73.28	0.01\\
73.29	0.01\\
73.3	0.01\\
73.31	0.01\\
73.32	0.01\\
73.33	0.01\\
73.34	0.01\\
73.35	0.01\\
73.36	0.01\\
73.37	0.01\\
73.38	0.01\\
73.39	0.01\\
73.4	0.01\\
73.41	0.01\\
73.42	0.01\\
73.43	0.01\\
73.44	0.01\\
73.45	0.01\\
73.46	0.01\\
73.47	0.01\\
73.48	0.01\\
73.49	0.01\\
73.5	0.01\\
73.51	0.01\\
73.52	0.01\\
73.53	0.01\\
73.54	0.01\\
73.55	0.01\\
73.56	0.01\\
73.57	0.01\\
73.58	0.01\\
73.59	0.01\\
73.6	0.01\\
73.61	0.01\\
73.62	0.01\\
73.63	0.01\\
73.64	0.01\\
73.65	0.01\\
73.66	0.01\\
73.67	0.01\\
73.68	0.01\\
73.69	0.01\\
73.7	0.01\\
73.71	0.01\\
73.72	0.01\\
73.73	0.01\\
73.74	0.01\\
73.75	0.01\\
73.76	0.01\\
73.77	0.01\\
73.78	0.01\\
73.79	0.01\\
73.8	0.01\\
73.81	0.01\\
73.82	0.01\\
73.83	0.01\\
73.84	0.01\\
73.85	0.01\\
73.86	0.01\\
73.87	0.01\\
73.88	0.01\\
73.89	0.01\\
73.9	0.01\\
73.91	0.01\\
73.92	0.01\\
73.93	0.01\\
73.94	0.01\\
73.95	0.01\\
73.96	0.01\\
73.97	0.01\\
73.98	0.01\\
73.99	0.01\\
74	0.01\\
74.01	0.01\\
74.02	0.01\\
74.03	0.01\\
74.04	0.01\\
74.05	0.01\\
74.06	0.01\\
74.07	0.01\\
74.08	0.01\\
74.09	0.01\\
74.1	0.01\\
74.11	0.01\\
74.12	0.01\\
74.13	0.01\\
74.14	0.01\\
74.15	0.01\\
74.16	0.01\\
74.17	0.01\\
74.18	0.01\\
74.19	0.01\\
74.2	0.01\\
74.21	0.01\\
74.22	0.01\\
74.23	0.01\\
74.24	0.01\\
74.25	0.01\\
74.26	0.01\\
74.27	0.01\\
74.28	0.01\\
74.29	0.01\\
74.3	0.01\\
74.31	0.01\\
74.32	0.01\\
74.33	0.01\\
74.34	0.01\\
74.35	0.01\\
74.36	0.01\\
74.37	0.01\\
74.38	0.01\\
74.39	0.01\\
74.4	0.01\\
74.41	0.01\\
74.42	0.01\\
74.43	0.01\\
74.44	0.01\\
74.45	0.01\\
74.46	0.01\\
74.47	0.01\\
74.48	0.01\\
74.49	0.01\\
74.5	0.01\\
74.51	0.01\\
74.52	0.01\\
74.53	0.01\\
74.54	0.01\\
74.55	0.01\\
74.56	0.01\\
74.57	0.01\\
74.58	0.01\\
74.59	0.01\\
74.6	0.01\\
74.61	0.01\\
74.62	0.01\\
74.63	0.01\\
74.64	0.01\\
74.65	0.01\\
74.66	0.01\\
74.67	0.01\\
74.68	0.01\\
74.69	0.01\\
74.7	0.01\\
74.71	0.01\\
74.72	0.01\\
74.73	0.01\\
74.74	0.01\\
74.75	0.01\\
74.76	0.01\\
74.77	0.01\\
74.78	0.01\\
74.79	0.01\\
74.8	0.01\\
74.81	0.01\\
74.82	0.01\\
74.83	0.01\\
74.84	0.01\\
74.85	0.01\\
74.86	0.01\\
74.87	0.01\\
74.88	0.01\\
74.89	0.01\\
74.9	0.01\\
74.91	0.01\\
74.92	0.01\\
74.93	0.01\\
74.94	0.01\\
74.95	0.01\\
74.96	0.01\\
74.97	0.01\\
74.98	0.01\\
74.99	0.01\\
75	0.01\\
75.01	0.01\\
75.02	0.01\\
75.03	0.01\\
75.04	0.01\\
75.05	0.01\\
75.06	0.01\\
75.07	0.01\\
75.08	0.01\\
75.09	0.01\\
75.1	0.01\\
75.11	0.01\\
75.12	0.01\\
75.13	0.01\\
75.14	0.01\\
75.15	0.01\\
75.16	0.01\\
75.17	0.01\\
75.18	0.01\\
75.19	0.01\\
75.2	0.01\\
75.21	0.01\\
75.22	0.01\\
75.23	0.01\\
75.24	0.01\\
75.25	0.01\\
75.26	0.01\\
75.27	0.01\\
75.28	0.01\\
75.29	0.01\\
75.3	0.01\\
75.31	0.01\\
75.32	0.01\\
75.33	0.01\\
75.34	0.01\\
75.35	0.01\\
75.36	0.01\\
75.37	0.01\\
75.38	0.01\\
75.39	0.01\\
75.4	0.01\\
75.41	0.01\\
75.42	0.01\\
75.43	0.01\\
75.44	0.01\\
75.45	0.01\\
75.46	0.01\\
75.47	0.01\\
75.48	0.01\\
75.49	0.01\\
75.5	0.01\\
75.51	0.01\\
75.52	0.01\\
75.53	0.01\\
75.54	0.01\\
75.55	0.01\\
75.56	0.01\\
75.57	0.01\\
75.58	0.01\\
75.59	0.01\\
75.6	0.01\\
75.61	0.01\\
75.62	0.01\\
75.63	0.01\\
75.64	0.01\\
75.65	0.01\\
75.66	0.01\\
75.67	0.01\\
75.68	0.01\\
75.69	0.01\\
75.7	0.01\\
75.71	0.01\\
75.72	0.01\\
75.73	0.01\\
75.74	0.01\\
75.75	0.01\\
75.76	0.01\\
75.77	0.01\\
75.78	0.01\\
75.79	0.01\\
75.8	0.01\\
75.81	0.01\\
75.82	0.01\\
75.83	0.01\\
75.84	0.01\\
75.85	0.01\\
75.86	0.01\\
75.87	0.01\\
75.88	0.01\\
75.89	0.01\\
75.9	0.01\\
75.91	0.01\\
75.92	0.01\\
75.93	0.01\\
75.94	0.01\\
75.95	0.01\\
75.96	0.01\\
75.97	0.01\\
75.98	0.01\\
75.99	0.01\\
76	0.01\\
76.01	0.01\\
76.02	0.01\\
76.03	0.01\\
76.04	0.01\\
76.05	0.01\\
76.06	0.01\\
76.07	0.01\\
76.08	0.01\\
76.09	0.01\\
76.1	0.01\\
76.11	0.01\\
76.12	0.01\\
76.13	0.01\\
76.14	0.01\\
76.15	0.01\\
76.16	0.01\\
76.17	0.01\\
76.18	0.01\\
76.19	0.01\\
76.2	0.01\\
76.21	0.01\\
76.22	0.01\\
76.23	0.01\\
76.24	0.01\\
76.25	0.01\\
76.26	0.01\\
76.27	0.01\\
76.28	0.01\\
76.29	0.01\\
76.3	0.01\\
76.31	0.01\\
76.32	0.01\\
76.33	0.01\\
76.34	0.01\\
76.35	0.01\\
76.36	0.01\\
76.37	0.01\\
76.38	0.01\\
76.39	0.01\\
76.4	0.01\\
76.41	0.01\\
76.42	0.01\\
76.43	0.01\\
76.44	0.01\\
76.45	0.01\\
76.46	0.01\\
76.47	0.01\\
76.48	0.01\\
76.49	0.01\\
76.5	0.01\\
76.51	0.01\\
76.52	0.01\\
76.53	0.01\\
76.54	0.01\\
76.55	0.01\\
76.56	0.01\\
76.57	0.01\\
76.58	0.01\\
76.59	0.01\\
76.6	0.01\\
76.61	0.01\\
76.62	0.01\\
76.63	0.01\\
76.64	0.01\\
76.65	0.01\\
76.66	0.01\\
76.67	0.01\\
76.68	0.01\\
76.69	0.01\\
76.7	0.01\\
76.71	0.01\\
76.72	0.01\\
76.73	0.01\\
76.74	0.01\\
76.75	0.01\\
76.76	0.01\\
76.77	0.01\\
76.78	0.01\\
76.79	0.01\\
76.8	0.01\\
76.81	0.01\\
76.82	0.01\\
76.83	0.01\\
76.84	0.01\\
76.85	0.01\\
76.86	0.01\\
76.87	0.01\\
76.88	0.01\\
76.89	0.01\\
76.9	0.01\\
76.91	0.01\\
76.92	0.01\\
76.93	0.01\\
76.94	0.01\\
76.95	0.01\\
76.96	0.01\\
76.97	0.01\\
76.98	0.01\\
76.99	0.01\\
77	0.01\\
77.01	0.01\\
77.02	0.01\\
77.03	0.01\\
77.04	0.01\\
77.05	0.01\\
77.06	0.01\\
77.07	0.01\\
77.08	0.01\\
77.09	0.01\\
77.1	0.01\\
77.11	0.01\\
77.12	0.01\\
77.13	0.01\\
77.14	0.01\\
77.15	0.01\\
77.16	0.01\\
77.17	0.01\\
77.18	0.01\\
77.19	0.01\\
77.2	0.01\\
77.21	0.01\\
77.22	0.01\\
77.23	0.01\\
77.24	0.01\\
77.25	0.01\\
77.26	0.01\\
77.27	0.01\\
77.28	0.01\\
77.29	0.01\\
77.3	0.01\\
77.31	0.01\\
77.32	0.01\\
77.33	0.01\\
77.34	0.01\\
77.35	0.01\\
77.36	0.01\\
77.37	0.01\\
77.38	0.01\\
77.39	0.01\\
77.4	0.01\\
77.41	0.01\\
77.42	0.01\\
77.43	0.01\\
77.44	0.01\\
77.45	0.01\\
77.46	0.01\\
77.47	0.01\\
77.48	0.01\\
77.49	0.01\\
77.5	0.01\\
77.51	0.01\\
77.52	0.01\\
77.53	0.01\\
77.54	0.01\\
77.55	0.01\\
77.56	0.01\\
77.57	0.01\\
77.58	0.01\\
77.59	0.01\\
77.6	0.01\\
77.61	0.01\\
77.62	0.01\\
77.63	0.01\\
77.64	0.01\\
77.65	0.01\\
77.66	0.01\\
77.67	0.01\\
77.68	0.01\\
77.69	0.01\\
77.7	0.01\\
77.71	0.01\\
77.72	0.01\\
77.73	0.01\\
77.74	0.01\\
77.75	0.01\\
77.76	0.01\\
77.77	0.01\\
77.78	0.01\\
77.79	0.01\\
77.8	0.01\\
77.81	0.01\\
77.82	0.01\\
77.83	0.01\\
77.84	0.01\\
77.85	0.01\\
77.86	0.01\\
77.87	0.01\\
77.88	0.01\\
77.89	0.01\\
77.9	0.01\\
77.91	0.01\\
77.92	0.01\\
77.93	0.01\\
77.94	0.01\\
77.95	0.01\\
77.96	0.01\\
77.97	0.01\\
77.98	0.01\\
77.99	0.01\\
78	0.01\\
78.01	0.01\\
78.02	0.01\\
78.03	0.01\\
78.04	0.01\\
78.05	0.01\\
78.06	0.01\\
78.07	0.01\\
78.08	0.01\\
78.09	0.01\\
78.1	0.01\\
78.11	0.01\\
78.12	0.01\\
78.13	0.01\\
78.14	0.01\\
78.15	0.01\\
78.16	0.01\\
78.17	0.01\\
78.18	0.01\\
78.19	0.01\\
78.2	0.01\\
78.21	0.01\\
78.22	0.01\\
78.23	0.01\\
78.24	0.01\\
78.25	0.01\\
78.26	0.01\\
78.27	0.01\\
78.28	0.01\\
78.29	0.01\\
78.3	0.01\\
78.31	0.01\\
78.32	0.01\\
78.33	0.01\\
78.34	0.01\\
78.35	0.01\\
78.36	0.01\\
78.37	0.01\\
78.38	0.01\\
78.39	0.01\\
78.4	0.01\\
78.41	0.01\\
78.42	0.01\\
78.43	0.01\\
78.44	0.01\\
78.45	0.01\\
78.46	0.01\\
78.47	0.01\\
78.48	0.01\\
78.49	0.01\\
78.5	0.01\\
78.51	0.01\\
78.52	0.01\\
78.53	0.01\\
78.54	0.01\\
78.55	0.01\\
78.56	0.01\\
78.57	0.01\\
78.58	0.01\\
78.59	0.01\\
78.6	0.01\\
78.61	0.01\\
78.62	0.01\\
78.63	0.01\\
78.64	0.01\\
78.65	0.01\\
78.66	0.01\\
78.67	0.01\\
78.68	0.01\\
78.69	0.01\\
78.7	0.01\\
78.71	0.01\\
78.72	0.01\\
78.73	0.01\\
78.74	0.01\\
78.75	0.01\\
78.76	0.01\\
78.77	0.01\\
78.78	0.01\\
78.79	0.01\\
78.8	0.01\\
78.81	0.01\\
78.82	0.01\\
78.83	0.01\\
78.84	0.01\\
78.85	0.01\\
78.86	0.01\\
78.87	0.01\\
78.88	0.01\\
78.89	0.01\\
78.9	0.01\\
78.91	0.01\\
78.92	0.01\\
78.93	0.01\\
78.94	0.01\\
78.95	0.01\\
78.96	0.01\\
78.97	0.01\\
78.98	0.01\\
78.99	0.01\\
79	0.01\\
79.01	0.01\\
79.02	0.01\\
79.03	0.01\\
79.04	0.01\\
79.05	0.01\\
79.06	0.01\\
79.07	0.01\\
79.08	0.01\\
79.09	0.01\\
79.1	0.01\\
79.11	0.01\\
79.12	0.01\\
79.13	0.01\\
79.14	0.01\\
79.15	0.01\\
79.16	0.01\\
79.17	0.01\\
79.18	0.01\\
79.19	0.01\\
79.2	0.01\\
79.21	0.01\\
79.22	0.01\\
79.23	0.01\\
79.24	0.01\\
79.25	0.01\\
79.26	0.01\\
79.27	0.01\\
79.28	0.01\\
79.29	0.01\\
79.3	0.01\\
79.31	0.01\\
79.32	0.01\\
79.33	0.01\\
79.34	0.01\\
79.35	0.01\\
79.36	0.01\\
79.37	0.01\\
79.38	0.01\\
79.39	0.01\\
79.4	0.01\\
79.41	0.01\\
79.42	0.01\\
79.43	0.01\\
79.44	0.01\\
79.45	0.01\\
79.46	0.01\\
79.47	0.01\\
79.48	0.01\\
79.49	0.01\\
79.5	0.01\\
79.51	0.01\\
79.52	0.01\\
79.53	0.01\\
79.54	0.01\\
79.55	0.01\\
79.56	0.01\\
79.57	0.01\\
79.58	0.01\\
79.59	0.01\\
79.6	0.01\\
79.61	0.01\\
79.62	0.01\\
79.63	0.01\\
79.64	0.01\\
79.65	0.01\\
79.66	0.01\\
79.67	0.01\\
79.68	0.01\\
79.69	0.01\\
79.7	0.01\\
79.71	0.01\\
79.72	0.01\\
79.73	0.01\\
79.74	0.01\\
79.75	0.01\\
79.76	0.01\\
79.77	0.01\\
79.78	0.01\\
79.79	0.01\\
79.8	0.01\\
79.81	0.01\\
79.82	0.01\\
79.83	0.01\\
79.84	0.01\\
79.85	0.01\\
79.86	0.01\\
79.87	0.01\\
79.88	0.01\\
79.89	0.01\\
79.9	0.01\\
79.91	0.01\\
79.92	0.01\\
79.93	0.01\\
79.94	0.01\\
79.95	0.01\\
79.96	0.01\\
79.97	0.01\\
79.98	0.01\\
79.99	0.01\\
80	0.01\\
80.01	0.01\\
};
\addplot [color=mycolor1,solid]
  table[row sep=crcr]{%
80.01	0.01\\
80.02	0.01\\
80.03	0.01\\
80.04	0.01\\
80.05	0.01\\
80.06	0.01\\
80.07	0.01\\
80.08	0.01\\
80.09	0.01\\
80.1	0.01\\
80.11	0.01\\
80.12	0.01\\
80.13	0.01\\
80.14	0.01\\
80.15	0.01\\
80.16	0.01\\
80.17	0.01\\
80.18	0.01\\
80.19	0.01\\
80.2	0.01\\
80.21	0.01\\
80.22	0.01\\
80.23	0.01\\
80.24	0.01\\
80.25	0.01\\
80.26	0.01\\
80.27	0.01\\
80.28	0.01\\
80.29	0.01\\
80.3	0.01\\
80.31	0.01\\
80.32	0.01\\
80.33	0.01\\
80.34	0.01\\
80.35	0.01\\
80.36	0.01\\
80.37	0.01\\
80.38	0.01\\
80.39	0.01\\
80.4	0.01\\
80.41	0.01\\
80.42	0.01\\
80.43	0.01\\
80.44	0.01\\
80.45	0.01\\
80.46	0.01\\
80.47	0.01\\
80.48	0.01\\
80.49	0.01\\
80.5	0.01\\
80.51	0.01\\
80.52	0.01\\
80.53	0.01\\
80.54	0.01\\
80.55	0.01\\
80.56	0.01\\
80.57	0.01\\
80.58	0.01\\
80.59	0.01\\
80.6	0.01\\
80.61	0.01\\
80.62	0.01\\
80.63	0.01\\
80.64	0.01\\
80.65	0.01\\
80.66	0.01\\
80.67	0.01\\
80.68	0.01\\
80.69	0.01\\
80.7	0.01\\
80.71	0.01\\
80.72	0.01\\
80.73	0.01\\
80.74	0.01\\
80.75	0.01\\
80.76	0.01\\
80.77	0.01\\
80.78	0.01\\
80.79	0.01\\
80.8	0.01\\
80.81	0.01\\
80.82	0.01\\
80.83	0.01\\
80.84	0.01\\
80.85	0.01\\
80.86	0.01\\
80.87	0.01\\
80.88	0.01\\
80.89	0.01\\
80.9	0.01\\
80.91	0.01\\
80.92	0.01\\
80.93	0.01\\
80.94	0.01\\
80.95	0.01\\
80.96	0.01\\
80.97	0.01\\
80.98	0.01\\
80.99	0.01\\
81	0.01\\
81.01	0.01\\
81.02	0.01\\
81.03	0.01\\
81.04	0.01\\
81.05	0.01\\
81.06	0.01\\
81.07	0.01\\
81.08	0.01\\
81.09	0.01\\
81.1	0.01\\
81.11	0.01\\
81.12	0.01\\
81.13	0.01\\
81.14	0.01\\
81.15	0.01\\
81.16	0.01\\
81.17	0.01\\
81.18	0.01\\
81.19	0.01\\
81.2	0.01\\
81.21	0.01\\
81.22	0.01\\
81.23	0.01\\
81.24	0.01\\
81.25	0.01\\
81.26	0.01\\
81.27	0.01\\
81.28	0.01\\
81.29	0.01\\
81.3	0.01\\
81.31	0.01\\
81.32	0.01\\
81.33	0.01\\
81.34	0.01\\
81.35	0.01\\
81.36	0.01\\
81.37	0.01\\
81.38	0.01\\
81.39	0.01\\
81.4	0.01\\
81.41	0.01\\
81.42	0.01\\
81.43	0.01\\
81.44	0.01\\
81.45	0.01\\
81.46	0.01\\
81.47	0.01\\
81.48	0.01\\
81.49	0.01\\
81.5	0.01\\
81.51	0.01\\
81.52	0.01\\
81.53	0.01\\
81.54	0.01\\
81.55	0.01\\
81.56	0.01\\
81.57	0.01\\
81.58	0.01\\
81.59	0.01\\
81.6	0.01\\
81.61	0.01\\
81.62	0.01\\
81.63	0.01\\
81.64	0.01\\
81.65	0.01\\
81.66	0.01\\
81.67	0.01\\
81.68	0.01\\
81.69	0.01\\
81.7	0.01\\
81.71	0.01\\
81.72	0.01\\
81.73	0.01\\
81.74	0.01\\
81.75	0.01\\
81.76	0.01\\
81.77	0.01\\
81.78	0.01\\
81.79	0.01\\
81.8	0.01\\
81.81	0.01\\
81.82	0.01\\
81.83	0.01\\
81.84	0.01\\
81.85	0.01\\
81.86	0.01\\
81.87	0.01\\
81.88	0.01\\
81.89	0.01\\
81.9	0.01\\
81.91	0.01\\
81.92	0.01\\
81.93	0.01\\
81.94	0.01\\
81.95	0.01\\
81.96	0.01\\
81.97	0.01\\
81.98	0.01\\
81.99	0.01\\
82	0.01\\
82.01	0.01\\
82.02	0.01\\
82.03	0.01\\
82.04	0.01\\
82.05	0.01\\
82.06	0.01\\
82.07	0.01\\
82.08	0.01\\
82.09	0.01\\
82.1	0.01\\
82.11	0.01\\
82.12	0.01\\
82.13	0.01\\
82.14	0.01\\
82.15	0.01\\
82.16	0.01\\
82.17	0.01\\
82.18	0.01\\
82.19	0.01\\
82.2	0.01\\
82.21	0.01\\
82.22	0.01\\
82.23	0.01\\
82.24	0.01\\
82.25	0.01\\
82.26	0.01\\
82.27	0.01\\
82.28	0.01\\
82.29	0.01\\
82.3	0.01\\
82.31	0.01\\
82.32	0.01\\
82.33	0.01\\
82.34	0.01\\
82.35	0.01\\
82.36	0.01\\
82.37	0.01\\
82.38	0.01\\
82.39	0.01\\
82.4	0.01\\
82.41	0.01\\
82.42	0.01\\
82.43	0.01\\
82.44	0.01\\
82.45	0.01\\
82.46	0.01\\
82.47	0.01\\
82.48	0.01\\
82.49	0.01\\
82.5	0.01\\
82.51	0.01\\
82.52	0.01\\
82.53	0.01\\
82.54	0.01\\
82.55	0.01\\
82.56	0.01\\
82.57	0.01\\
82.58	0.01\\
82.59	0.01\\
82.6	0.01\\
82.61	0.01\\
82.62	0.01\\
82.63	0.01\\
82.64	0.01\\
82.65	0.01\\
82.66	0.01\\
82.67	0.01\\
82.68	0.01\\
82.69	0.01\\
82.7	0.01\\
82.71	0.01\\
82.72	0.01\\
82.73	0.01\\
82.74	0.01\\
82.75	0.01\\
82.76	0.01\\
82.77	0.01\\
82.78	0.01\\
82.79	0.01\\
82.8	0.01\\
82.81	0.01\\
82.82	0.01\\
82.83	0.01\\
82.84	0.01\\
82.85	0.01\\
82.86	0.01\\
82.87	0.01\\
82.88	0.01\\
82.89	0.01\\
82.9	0.01\\
82.91	0.01\\
82.92	0.01\\
82.93	0.01\\
82.94	0.01\\
82.95	0.01\\
82.96	0.01\\
82.97	0.01\\
82.98	0.01\\
82.99	0.01\\
83	0.01\\
83.01	0.01\\
83.02	0.01\\
83.03	0.01\\
83.04	0.01\\
83.05	0.01\\
83.06	0.01\\
83.07	0.01\\
83.08	0.01\\
83.09	0.01\\
83.1	0.01\\
83.11	0.01\\
83.12	0.01\\
83.13	0.01\\
83.14	0.01\\
83.15	0.01\\
83.16	0.01\\
83.17	0.01\\
83.18	0.01\\
83.19	0.01\\
83.2	0.01\\
83.21	0.01\\
83.22	0.01\\
83.23	0.01\\
83.24	0.01\\
83.25	0.01\\
83.26	0.01\\
83.27	0.01\\
83.28	0.01\\
83.29	0.01\\
83.3	0.01\\
83.31	0.01\\
83.32	0.01\\
83.33	0.01\\
83.34	0.01\\
83.35	0.01\\
83.36	0.01\\
83.37	0.01\\
83.38	0.01\\
83.39	0.01\\
83.4	0.01\\
83.41	0.01\\
83.42	0.01\\
83.43	0.01\\
83.44	0.01\\
83.45	0.01\\
83.46	0.01\\
83.47	0.01\\
83.48	0.01\\
83.49	0.01\\
83.5	0.01\\
83.51	0.01\\
83.52	0.01\\
83.53	0.01\\
83.54	0.01\\
83.55	0.01\\
83.56	0.01\\
83.57	0.01\\
83.58	0.01\\
83.59	0.01\\
83.6	0.01\\
83.61	0.01\\
83.62	0.01\\
83.63	0.01\\
83.64	0.01\\
83.65	0.01\\
83.66	0.01\\
83.67	0.01\\
83.68	0.01\\
83.69	0.01\\
83.7	0.01\\
83.71	0.01\\
83.72	0.01\\
83.73	0.01\\
83.74	0.01\\
83.75	0.01\\
83.76	0.01\\
83.77	0.01\\
83.78	0.01\\
83.79	0.01\\
83.8	0.01\\
83.81	0.01\\
83.82	0.01\\
83.83	0.01\\
83.84	0.01\\
83.85	0.01\\
83.86	0.01\\
83.87	0.01\\
83.88	0.01\\
83.89	0.01\\
83.9	0.01\\
83.91	0.01\\
83.92	0.01\\
83.93	0.01\\
83.94	0.01\\
83.95	0.01\\
83.96	0.01\\
83.97	0.01\\
83.98	0.01\\
83.99	0.01\\
84	0.01\\
84.01	0.01\\
84.02	0.01\\
84.03	0.01\\
84.04	0.01\\
84.05	0.01\\
84.06	0.01\\
84.07	0.01\\
84.08	0.01\\
84.09	0.01\\
84.1	0.01\\
84.11	0.01\\
84.12	0.01\\
84.13	0.01\\
84.14	0.01\\
84.15	0.01\\
84.16	0.01\\
84.17	0.01\\
84.18	0.01\\
84.19	0.01\\
84.2	0.01\\
84.21	0.01\\
84.22	0.01\\
84.23	0.01\\
84.24	0.01\\
84.25	0.01\\
84.26	0.01\\
84.27	0.01\\
84.28	0.01\\
84.29	0.01\\
84.3	0.01\\
84.31	0.01\\
84.32	0.01\\
84.33	0.01\\
84.34	0.01\\
84.35	0.01\\
84.36	0.01\\
84.37	0.01\\
84.38	0.01\\
84.39	0.01\\
84.4	0.01\\
84.41	0.01\\
84.42	0.01\\
84.43	0.01\\
84.44	0.01\\
84.45	0.01\\
84.46	0.01\\
84.47	0.01\\
84.48	0.01\\
84.49	0.01\\
84.5	0.01\\
84.51	0.01\\
84.52	0.01\\
84.53	0.01\\
84.54	0.01\\
84.55	0.01\\
84.56	0.01\\
84.57	0.01\\
84.58	0.01\\
84.59	0.01\\
84.6	0.01\\
84.61	0.01\\
84.62	0.01\\
84.63	0.01\\
84.64	0.01\\
84.65	0.01\\
84.66	0.01\\
84.67	0.01\\
84.68	0.01\\
84.69	0.01\\
84.7	0.01\\
84.71	0.01\\
84.72	0.01\\
84.73	0.01\\
84.74	0.01\\
84.75	0.01\\
84.76	0.01\\
84.77	0.01\\
84.78	0.01\\
84.79	0.01\\
84.8	0.01\\
84.81	0.01\\
84.82	0.01\\
84.83	0.01\\
84.84	0.01\\
84.85	0.01\\
84.86	0.01\\
84.87	0.01\\
84.88	0.01\\
84.89	0.01\\
84.9	0.01\\
84.91	0.01\\
84.92	0.01\\
84.93	0.01\\
84.94	0.01\\
84.95	0.01\\
84.96	0.01\\
84.97	0.01\\
84.98	0.01\\
84.99	0.01\\
85	0.01\\
85.01	0.01\\
85.02	0.01\\
85.03	0.01\\
85.04	0.01\\
85.05	0.01\\
85.06	0.01\\
85.07	0.01\\
85.08	0.01\\
85.09	0.01\\
85.1	0.01\\
85.11	0.01\\
85.12	0.01\\
85.13	0.01\\
85.14	0.01\\
85.15	0.01\\
85.16	0.01\\
85.17	0.01\\
85.18	0.01\\
85.19	0.01\\
85.2	0.01\\
85.21	0.01\\
85.22	0.01\\
85.23	0.01\\
85.24	0.01\\
85.25	0.01\\
85.26	0.01\\
85.27	0.01\\
85.28	0.01\\
85.29	0.01\\
85.3	0.01\\
85.31	0.01\\
85.32	0.01\\
85.33	0.01\\
85.34	0.01\\
85.35	0.01\\
85.36	0.01\\
85.37	0.01\\
85.38	0.01\\
85.39	0.01\\
85.4	0.01\\
85.41	0.01\\
85.42	0.01\\
85.43	0.01\\
85.44	0.01\\
85.45	0.01\\
85.46	0.01\\
85.47	0.01\\
85.48	0.01\\
85.49	0.01\\
85.5	0.01\\
85.51	0.01\\
85.52	0.01\\
85.53	0.01\\
85.54	0.01\\
85.55	0.01\\
85.56	0.01\\
85.57	0.01\\
85.58	0.01\\
85.59	0.01\\
85.6	0.01\\
85.61	0.01\\
85.62	0.01\\
85.63	0.01\\
85.64	0.01\\
85.65	0.01\\
85.66	0.01\\
85.67	0.01\\
85.68	0.01\\
85.69	0.01\\
85.7	0.01\\
85.71	0.01\\
85.72	0.01\\
85.73	0.01\\
85.74	0.01\\
85.75	0.01\\
85.76	0.01\\
85.77	0.01\\
85.78	0.01\\
85.79	0.01\\
85.8	0.01\\
85.81	0.01\\
85.82	0.01\\
85.83	0.01\\
85.84	0.01\\
85.85	0.01\\
85.86	0.01\\
85.87	0.01\\
85.88	0.01\\
85.89	0.01\\
85.9	0.01\\
85.91	0.01\\
85.92	0.01\\
85.93	0.01\\
85.94	0.01\\
85.95	0.01\\
85.96	0.01\\
85.97	0.01\\
85.98	0.01\\
85.99	0.01\\
86	0.01\\
86.01	0.01\\
86.02	0.01\\
86.03	0.01\\
86.04	0.01\\
86.05	0.01\\
86.06	0.01\\
86.07	0.01\\
86.08	0.01\\
86.09	0.01\\
86.1	0.01\\
86.11	0.01\\
86.12	0.01\\
86.13	0.01\\
86.14	0.01\\
86.15	0.01\\
86.16	0.01\\
86.17	0.01\\
86.18	0.01\\
86.19	0.01\\
86.2	0.01\\
86.21	0.01\\
86.22	0.01\\
86.23	0.01\\
86.24	0.01\\
86.25	0.01\\
86.26	0.01\\
86.27	0.01\\
86.28	0.01\\
86.29	0.01\\
86.3	0.01\\
86.31	0.01\\
86.32	0.01\\
86.33	0.01\\
86.34	0.01\\
86.35	0.01\\
86.36	0.01\\
86.37	0.01\\
86.38	0.01\\
86.39	0.01\\
86.4	0.01\\
86.41	0.01\\
86.42	0.01\\
86.43	0.01\\
86.44	0.01\\
86.45	0.01\\
86.46	0.01\\
86.47	0.01\\
86.48	0.01\\
86.49	0.01\\
86.5	0.01\\
86.51	0.01\\
86.52	0.01\\
86.53	0.01\\
86.54	0.01\\
86.55	0.01\\
86.56	0.01\\
86.57	0.01\\
86.58	0.01\\
86.59	0.01\\
86.6	0.01\\
86.61	0.01\\
86.62	0.01\\
86.63	0.01\\
86.64	0.01\\
86.65	0.01\\
86.66	0.01\\
86.67	0.01\\
86.68	0.01\\
86.69	0.01\\
86.7	0.01\\
86.71	0.01\\
86.72	0.01\\
86.73	0.01\\
86.74	0.01\\
86.75	0.01\\
86.76	0.01\\
86.77	0.01\\
86.78	0.01\\
86.79	0.01\\
86.8	0.01\\
86.81	0.01\\
86.82	0.01\\
86.83	0.01\\
86.84	0.01\\
86.85	0.01\\
86.86	0.01\\
86.87	0.01\\
86.88	0.01\\
86.89	0.01\\
86.9	0.01\\
86.91	0.01\\
86.92	0.01\\
86.93	0.01\\
86.94	0.01\\
86.95	0.01\\
86.96	0.01\\
86.97	0.01\\
86.98	0.01\\
86.99	0.01\\
87	0.01\\
87.01	0.01\\
87.02	0.01\\
87.03	0.01\\
87.04	0.01\\
87.05	0.01\\
87.06	0.01\\
87.07	0.01\\
87.08	0.01\\
87.09	0.01\\
87.1	0.01\\
87.11	0.01\\
87.12	0.01\\
87.13	0.01\\
87.14	0.01\\
87.15	0.01\\
87.16	0.01\\
87.17	0.01\\
87.18	0.01\\
87.19	0.01\\
87.2	0.01\\
87.21	0.01\\
87.22	0.01\\
87.23	0.01\\
87.24	0.01\\
87.25	0.01\\
87.26	0.01\\
87.27	0.01\\
87.28	0.01\\
87.29	0.01\\
87.3	0.01\\
87.31	0.01\\
87.32	0.01\\
87.33	0.01\\
87.34	0.01\\
87.35	0.01\\
87.36	0.01\\
87.37	0.01\\
87.38	0.01\\
87.39	0.01\\
87.4	0.01\\
87.41	0.01\\
87.42	0.01\\
87.43	0.01\\
87.44	0.01\\
87.45	0.01\\
87.46	0.01\\
87.47	0.01\\
87.48	0.01\\
87.49	0.01\\
87.5	0.01\\
87.51	0.01\\
87.52	0.01\\
87.53	0.01\\
87.54	0.01\\
87.55	0.01\\
87.56	0.01\\
87.57	0.01\\
87.58	0.01\\
87.59	0.01\\
87.6	0.01\\
87.61	0.01\\
87.62	0.01\\
87.63	0.01\\
87.64	0.01\\
87.65	0.01\\
87.66	0.01\\
87.67	0.01\\
87.68	0.01\\
87.69	0.01\\
87.7	0.01\\
87.71	0.01\\
87.72	0.01\\
87.73	0.01\\
87.74	0.01\\
87.75	0.01\\
87.76	0.01\\
87.77	0.01\\
87.78	0.01\\
87.79	0.01\\
87.8	0.01\\
87.81	0.01\\
87.82	0.01\\
87.83	0.01\\
87.84	0.01\\
87.85	0.01\\
87.86	0.01\\
87.87	0.01\\
87.88	0.01\\
87.89	0.01\\
87.9	0.01\\
87.91	0.01\\
87.92	0.01\\
87.93	0.01\\
87.94	0.01\\
87.95	0.01\\
87.96	0.01\\
87.97	0.01\\
87.98	0.01\\
87.99	0.01\\
88	0.01\\
88.01	0.01\\
88.02	0.01\\
88.03	0.01\\
88.04	0.01\\
88.05	0.01\\
88.06	0.01\\
88.07	0.01\\
88.08	0.01\\
88.09	0.01\\
88.1	0.01\\
88.11	0.01\\
88.12	0.01\\
88.13	0.01\\
88.14	0.01\\
88.15	0.01\\
88.16	0.01\\
88.17	0.01\\
88.18	0.01\\
88.19	0.01\\
88.2	0.01\\
88.21	0.01\\
88.22	0.01\\
88.23	0.01\\
88.24	0.01\\
88.25	0.01\\
88.26	0.01\\
88.27	0.01\\
88.28	0.01\\
88.29	0.01\\
88.3	0.01\\
88.31	0.01\\
88.32	0.01\\
88.33	0.01\\
88.34	0.01\\
88.35	0.01\\
88.36	0.01\\
88.37	0.01\\
88.38	0.01\\
88.39	0.01\\
88.4	0.01\\
88.41	0.01\\
88.42	0.01\\
88.43	0.01\\
88.44	0.01\\
88.45	0.01\\
88.46	0.01\\
88.47	0.01\\
88.48	0.01\\
88.49	0.01\\
88.5	0.01\\
88.51	0.01\\
88.52	0.01\\
88.53	0.01\\
88.54	0.01\\
88.55	0.01\\
88.56	0.01\\
88.57	0.01\\
88.58	0.01\\
88.59	0.01\\
88.6	0.01\\
88.61	0.01\\
88.62	0.01\\
88.63	0.01\\
88.64	0.01\\
88.65	0.01\\
88.66	0.01\\
88.67	0.01\\
88.68	0.01\\
88.69	0.01\\
88.7	0.01\\
88.71	0.01\\
88.72	0.01\\
88.73	0.01\\
88.74	0.01\\
88.75	0.01\\
88.76	0.01\\
88.77	0.01\\
88.78	0.01\\
88.79	0.01\\
88.8	0.01\\
88.81	0.01\\
88.82	0.01\\
88.83	0.01\\
88.84	0.01\\
88.85	0.01\\
88.86	0.01\\
88.87	0.01\\
88.88	0.01\\
88.89	0.01\\
88.9	0.01\\
88.91	0.01\\
88.92	0.01\\
88.93	0.01\\
88.94	0.01\\
88.95	0.01\\
88.96	0.01\\
88.97	0.01\\
88.98	0.01\\
88.99	0.01\\
89	0.01\\
89.01	0.01\\
89.02	0.01\\
89.03	0.01\\
89.04	0.01\\
89.05	0.01\\
89.06	0.01\\
89.07	0.01\\
89.08	0.01\\
89.09	0.01\\
89.1	0.01\\
89.11	0.01\\
89.12	0.01\\
89.13	0.01\\
89.14	0.01\\
89.15	0.01\\
89.16	0.01\\
89.17	0.01\\
89.18	0.01\\
89.19	0.01\\
89.2	0.01\\
89.21	0.01\\
89.22	0.01\\
89.23	0.01\\
89.24	0.01\\
89.25	0.01\\
89.26	0.01\\
89.27	0.01\\
89.28	0.01\\
89.29	0.01\\
89.3	0.01\\
89.31	0.01\\
89.32	0.01\\
89.33	0.01\\
89.34	0.01\\
89.35	0.01\\
89.36	0.01\\
89.37	0.01\\
89.38	0.01\\
89.39	0.01\\
89.4	0.01\\
89.41	0.01\\
89.42	0.01\\
89.43	0.01\\
89.44	0.01\\
89.45	0.01\\
89.46	0.01\\
89.47	0.01\\
89.48	0.01\\
89.49	0.01\\
89.5	0.01\\
89.51	0.01\\
89.52	0.01\\
89.53	0.01\\
89.54	0.01\\
89.55	0.01\\
89.56	0.01\\
89.57	0.01\\
89.58	0.01\\
89.59	0.01\\
89.6	0.01\\
89.61	0.01\\
89.62	0.01\\
89.63	0.01\\
89.64	0.01\\
89.65	0.01\\
89.66	0.01\\
89.67	0.01\\
89.68	0.01\\
89.69	0.01\\
89.7	0.01\\
89.71	0.01\\
89.72	0.01\\
89.73	0.01\\
89.74	0.01\\
89.75	0.01\\
89.76	0.01\\
89.77	0.01\\
89.78	0.01\\
89.79	0.01\\
89.8	0.01\\
89.81	0.01\\
89.82	0.01\\
89.83	0.01\\
89.84	0.01\\
89.85	0.01\\
89.86	0.01\\
89.87	0.01\\
89.88	0.01\\
89.89	0.01\\
89.9	0.01\\
89.91	0.01\\
89.92	0.01\\
89.93	0.01\\
89.94	0.01\\
89.95	0.01\\
89.96	0.01\\
89.97	0.01\\
89.98	0.01\\
89.99	0.01\\
90	0.01\\
90.01	0.01\\
90.02	0.01\\
90.03	0.01\\
90.04	0.01\\
90.05	0.01\\
90.06	0.01\\
90.07	0.01\\
90.08	0.01\\
90.09	0.01\\
90.1	0.01\\
90.11	0.01\\
90.12	0.01\\
90.13	0.01\\
90.14	0.01\\
90.15	0.01\\
90.16	0.01\\
90.17	0.01\\
90.18	0.01\\
90.19	0.01\\
90.2	0.01\\
90.21	0.01\\
90.22	0.01\\
90.23	0.01\\
90.24	0.01\\
90.25	0.01\\
90.26	0.01\\
90.27	0.01\\
90.28	0.01\\
90.29	0.01\\
90.3	0.01\\
90.31	0.01\\
90.32	0.01\\
90.33	0.01\\
90.34	0.01\\
90.35	0.01\\
90.36	0.01\\
90.37	0.01\\
90.38	0.01\\
90.39	0.01\\
90.4	0.01\\
90.41	0.01\\
90.42	0.01\\
90.43	0.01\\
90.44	0.01\\
90.45	0.01\\
90.46	0.01\\
90.47	0.01\\
90.48	0.01\\
90.49	0.01\\
90.5	0.01\\
90.51	0.01\\
90.52	0.01\\
90.53	0.01\\
90.54	0.01\\
90.55	0.01\\
90.56	0.01\\
90.57	0.01\\
90.58	0.01\\
90.59	0.01\\
90.6	0.01\\
90.61	0.01\\
90.62	0.01\\
90.63	0.01\\
90.64	0.01\\
90.65	0.01\\
90.66	0.01\\
90.67	0.01\\
90.68	0.01\\
90.69	0.01\\
90.7	0.01\\
90.71	0.01\\
90.72	0.01\\
90.73	0.01\\
90.74	0.01\\
90.75	0.01\\
90.76	0.01\\
90.77	0.01\\
90.78	0.01\\
90.79	0.01\\
90.8	0.01\\
90.81	0.01\\
90.82	0.01\\
90.83	0.01\\
90.84	0.01\\
90.85	0.01\\
90.86	0.01\\
90.87	0.01\\
90.88	0.01\\
90.89	0.01\\
90.9	0.01\\
90.91	0.01\\
90.92	0.01\\
90.93	0.01\\
90.94	0.01\\
90.95	0.01\\
90.96	0.01\\
90.97	0.01\\
90.98	0.01\\
90.99	0.01\\
91	0.01\\
91.01	0.01\\
91.02	0.01\\
91.03	0.01\\
91.04	0.01\\
91.05	0.01\\
91.06	0.01\\
91.07	0.01\\
91.08	0.01\\
91.09	0.01\\
91.1	0.01\\
91.11	0.01\\
91.12	0.01\\
91.13	0.01\\
91.14	0.01\\
91.15	0.01\\
91.16	0.01\\
91.17	0.01\\
91.18	0.01\\
91.19	0.01\\
91.2	0.01\\
91.21	0.01\\
91.22	0.01\\
91.23	0.01\\
91.24	0.01\\
91.25	0.01\\
91.26	0.01\\
91.27	0.01\\
91.28	0.01\\
91.29	0.01\\
91.3	0.01\\
91.31	0.01\\
91.32	0.01\\
91.33	0.01\\
91.34	0.01\\
91.35	0.01\\
91.36	0.01\\
91.37	0.01\\
91.38	0.01\\
91.39	0.01\\
91.4	0.01\\
91.41	0.01\\
91.42	0.01\\
91.43	0.01\\
91.44	0.01\\
91.45	0.01\\
91.46	0.01\\
91.47	0.01\\
91.48	0.01\\
91.49	0.01\\
91.5	0.01\\
91.51	0.01\\
91.52	0.01\\
91.53	0.01\\
91.54	0.01\\
91.55	0.01\\
91.56	0.01\\
91.57	0.01\\
91.58	0.01\\
91.59	0.01\\
91.6	0.01\\
91.61	0.01\\
91.62	0.01\\
91.63	0.01\\
91.64	0.01\\
91.65	0.01\\
91.66	0.01\\
91.67	0.01\\
91.68	0.01\\
91.69	0.01\\
91.7	0.01\\
91.71	0.01\\
91.72	0.01\\
91.73	0.01\\
91.74	0.01\\
91.75	0.01\\
91.76	0.01\\
91.77	0.01\\
91.78	0.01\\
91.79	0.01\\
91.8	0.01\\
91.81	0.01\\
91.82	0.01\\
91.83	0.01\\
91.84	0.01\\
91.85	0.01\\
91.86	0.01\\
91.87	0.01\\
91.88	0.01\\
91.89	0.01\\
91.9	0.01\\
91.91	0.01\\
91.92	0.01\\
91.93	0.01\\
91.94	0.01\\
91.95	0.01\\
91.96	0.01\\
91.97	0.01\\
91.98	0.01\\
91.99	0.01\\
92	0.01\\
92.01	0.01\\
92.02	0.01\\
92.03	0.01\\
92.04	0.01\\
92.05	0.01\\
92.06	0.01\\
92.07	0.01\\
92.08	0.01\\
92.09	0.01\\
92.1	0.01\\
92.11	0.01\\
92.12	0.01\\
92.13	0.01\\
92.14	0.01\\
92.15	0.01\\
92.16	0.01\\
92.17	0.01\\
92.18	0.01\\
92.19	0.01\\
92.2	0.01\\
92.21	0.01\\
92.22	0.01\\
92.23	0.01\\
92.24	0.01\\
92.25	0.01\\
92.26	0.01\\
92.27	0.01\\
92.28	0.01\\
92.29	0.01\\
92.3	0.01\\
92.31	0.01\\
92.32	0.01\\
92.33	0.01\\
92.34	0.01\\
92.35	0.01\\
92.36	0.01\\
92.37	0.01\\
92.38	0.01\\
92.39	0.01\\
92.4	0.01\\
92.41	0.01\\
92.42	0.01\\
92.43	0.01\\
92.44	0.01\\
92.45	0.01\\
92.46	0.01\\
92.47	0.01\\
92.48	0.01\\
92.49	0.01\\
92.5	0.01\\
92.51	0.01\\
92.52	0.01\\
92.53	0.01\\
92.54	0.01\\
92.55	0.01\\
92.56	0.01\\
92.57	0.01\\
92.58	0.01\\
92.59	0.01\\
92.6	0.01\\
92.61	0.01\\
92.62	0.01\\
92.63	0.01\\
92.64	0.01\\
92.65	0.01\\
92.66	0.01\\
92.67	0.01\\
92.68	0.01\\
92.69	0.01\\
92.7	0.01\\
92.71	0.01\\
92.72	0.01\\
92.73	0.01\\
92.74	0.01\\
92.75	0.01\\
92.76	0.01\\
92.77	0.01\\
92.78	0.01\\
92.79	0.01\\
92.8	0.01\\
92.81	0.01\\
92.82	0.01\\
92.83	0.01\\
92.84	0.01\\
92.85	0.01\\
92.86	0.01\\
92.87	0.01\\
92.88	0.01\\
92.89	0.01\\
92.9	0.01\\
92.91	0.01\\
92.92	0.01\\
92.93	0.01\\
92.94	0.01\\
92.95	0.01\\
92.96	0.01\\
92.97	0.01\\
92.98	0.01\\
92.99	0.01\\
93	0.01\\
93.01	0.01\\
93.02	0.01\\
93.03	0.01\\
93.04	0.01\\
93.05	0.01\\
93.06	0.01\\
93.07	0.01\\
93.08	0.01\\
93.09	0.01\\
93.1	0.01\\
93.11	0.01\\
93.12	0.01\\
93.13	0.01\\
93.14	0.01\\
93.15	0.01\\
93.16	0.01\\
93.17	0.01\\
93.18	0.01\\
93.19	0.01\\
93.2	0.01\\
93.21	0.01\\
93.22	0.01\\
93.23	0.01\\
93.24	0.01\\
93.25	0.01\\
93.26	0.01\\
93.27	0.01\\
93.28	0.01\\
93.29	0.01\\
93.3	0.01\\
93.31	0.01\\
93.32	0.01\\
93.33	0.01\\
93.34	0.01\\
93.35	0.01\\
93.36	0.01\\
93.37	0.01\\
93.38	0.01\\
93.39	0.01\\
93.4	0.01\\
93.41	0.01\\
93.42	0.01\\
93.43	0.01\\
93.44	0.01\\
93.45	0.01\\
93.46	0.01\\
93.47	0.01\\
93.48	0.01\\
93.49	0.01\\
93.5	0.01\\
93.51	0.01\\
93.52	0.01\\
93.53	0.01\\
93.54	0.01\\
93.55	0.01\\
93.56	0.01\\
93.57	0.01\\
93.58	0.01\\
93.59	0.01\\
93.6	0.01\\
93.61	0.01\\
93.62	0.01\\
93.63	0.01\\
93.64	0.01\\
93.65	0.01\\
93.66	0.01\\
93.67	0.01\\
93.68	0.01\\
93.69	0.01\\
93.7	0.01\\
93.71	0.01\\
93.72	0.01\\
93.73	0.01\\
93.74	0.01\\
93.75	0.01\\
93.76	0.01\\
93.77	0.01\\
93.78	0.01\\
93.79	0.01\\
93.8	0.01\\
93.81	0.01\\
93.82	0.01\\
93.83	0.01\\
93.84	0.01\\
93.85	0.01\\
93.86	0.01\\
93.87	0.01\\
93.88	0.01\\
93.89	0.01\\
93.9	0.01\\
93.91	0.01\\
93.92	0.01\\
93.93	0.01\\
93.94	0.01\\
93.95	0.01\\
93.96	0.01\\
93.97	0.01\\
93.98	0.01\\
93.99	0.01\\
94	0.01\\
94.01	0.01\\
94.02	0.01\\
94.03	0.01\\
94.04	0.01\\
94.05	0.01\\
94.06	0.01\\
94.07	0.01\\
94.08	0.01\\
94.09	0.01\\
94.1	0.01\\
94.11	0.01\\
94.12	0.01\\
94.13	0.01\\
94.14	0.01\\
94.15	0.01\\
94.16	0.01\\
94.17	0.01\\
94.18	0.01\\
94.19	0.01\\
94.2	0.01\\
94.21	0.01\\
94.22	0.01\\
94.23	0.01\\
94.24	0.01\\
94.25	0.01\\
94.26	0.01\\
94.27	0.01\\
94.28	0.01\\
94.29	0.01\\
94.3	0.01\\
94.31	0.01\\
94.32	0.01\\
94.33	0.01\\
94.34	0.01\\
94.35	0.01\\
94.36	0.01\\
94.37	0.01\\
94.38	0.01\\
94.39	0.01\\
94.4	0.01\\
94.41	0.01\\
94.42	0.01\\
94.43	0.01\\
94.44	0.01\\
94.45	0.01\\
94.46	0.01\\
94.47	0.01\\
94.48	0.01\\
94.49	0.01\\
94.5	0.01\\
94.51	0.01\\
94.52	0.01\\
94.53	0.01\\
94.54	0.01\\
94.55	0.01\\
94.56	0.01\\
94.57	0.01\\
94.58	0.01\\
94.59	0.01\\
94.6	0.01\\
94.61	0.01\\
94.62	0.01\\
94.63	0.01\\
94.64	0.01\\
94.65	0.01\\
94.66	0.01\\
94.67	0.01\\
94.68	0.01\\
94.69	0.01\\
94.7	0.01\\
94.71	0.01\\
94.72	0.01\\
94.73	0.01\\
94.74	0.01\\
94.75	0.01\\
94.76	0.01\\
94.77	0.01\\
94.78	0.01\\
94.79	0.01\\
94.8	0.01\\
94.81	0.01\\
94.82	0.01\\
94.83	0.01\\
94.84	0.01\\
94.85	0.01\\
94.86	0.01\\
94.87	0.01\\
94.88	0.01\\
94.89	0.01\\
94.9	0.01\\
94.91	0.01\\
94.92	0.01\\
94.93	0.01\\
94.94	0.01\\
94.95	0.01\\
94.96	0.01\\
94.97	0.01\\
94.98	0.01\\
94.99	0.01\\
95	0.01\\
95.01	0.01\\
95.02	0.01\\
95.03	0.01\\
95.04	0.01\\
95.05	0.01\\
95.06	0.01\\
95.07	0.01\\
95.08	0.01\\
95.09	0.01\\
95.1	0.01\\
95.11	0.01\\
95.12	0.01\\
95.13	0.01\\
95.14	0.01\\
95.15	0.01\\
95.16	0.01\\
95.17	0.01\\
95.18	0.01\\
95.19	0.01\\
95.2	0.01\\
95.21	0.01\\
95.22	0.01\\
95.23	0.01\\
95.24	0.01\\
95.25	0.01\\
95.26	0.01\\
95.27	0.01\\
95.28	0.01\\
95.29	0.01\\
95.3	0.01\\
95.31	0.01\\
95.32	0.01\\
95.33	0.01\\
95.34	0.01\\
95.35	0.01\\
95.36	0.01\\
95.37	0.01\\
95.38	0.01\\
95.39	0.01\\
95.4	0.01\\
95.41	0.01\\
95.42	0.01\\
95.43	0.01\\
95.44	0.01\\
95.45	0.01\\
95.46	0.01\\
95.47	0.01\\
95.48	0.01\\
95.49	0.01\\
95.5	0.01\\
95.51	0.01\\
95.52	0.01\\
95.53	0.01\\
95.54	0.01\\
95.55	0.01\\
95.56	0.01\\
95.57	0.01\\
95.58	0.01\\
95.59	0.01\\
95.6	0.01\\
95.61	0.01\\
95.62	0.01\\
95.63	0.01\\
95.64	0.01\\
95.65	0.01\\
95.66	0.01\\
95.67	0.01\\
95.68	0.01\\
95.69	0.01\\
95.7	0.01\\
95.71	0.01\\
95.72	0.01\\
95.73	0.01\\
95.74	0.01\\
95.75	0.01\\
95.76	0.01\\
95.77	0.01\\
95.78	0.01\\
95.79	0.01\\
95.8	0.01\\
95.81	0.01\\
95.82	0.01\\
95.83	0.01\\
95.84	0.01\\
95.85	0.01\\
95.86	0.01\\
95.87	0.01\\
95.88	0.01\\
95.89	0.01\\
95.9	0.01\\
95.91	0.01\\
95.92	0.01\\
95.93	0.01\\
95.94	0.01\\
95.95	0.01\\
95.96	0.01\\
95.97	0.01\\
95.98	0.01\\
95.99	0.01\\
96	0.01\\
96.01	0.01\\
96.02	0.01\\
96.03	0.01\\
96.04	0.01\\
96.05	0.01\\
96.06	0.01\\
96.07	0.01\\
96.08	0.01\\
96.09	0.01\\
96.1	0.01\\
96.11	0.01\\
96.12	0.01\\
96.13	0.01\\
96.14	0.01\\
96.15	0.01\\
96.16	0.01\\
96.17	0.01\\
96.18	0.01\\
96.19	0.01\\
96.2	0.01\\
96.21	0.01\\
96.22	0.01\\
96.23	0.01\\
96.24	0.01\\
96.25	0.01\\
96.26	0.01\\
96.27	0.01\\
96.28	0.01\\
96.29	0.01\\
96.3	0.01\\
96.31	0.01\\
96.32	0.01\\
96.33	0.01\\
96.34	0.01\\
96.35	0.01\\
96.36	0.01\\
96.37	0.01\\
96.38	0.01\\
96.39	0.01\\
96.4	0.01\\
96.41	0.01\\
96.42	0.01\\
96.43	0.01\\
96.44	0.01\\
96.45	0.01\\
96.46	0.01\\
96.47	0.01\\
96.48	0.01\\
96.49	0.01\\
96.5	0.01\\
96.51	0.01\\
96.52	0.01\\
96.53	0.01\\
96.54	0.01\\
96.55	0.01\\
96.56	0.01\\
96.57	0.01\\
96.58	0.01\\
96.59	0.01\\
96.6	0.01\\
96.61	0.01\\
96.62	0.01\\
96.63	0.01\\
96.64	0.01\\
96.65	0.01\\
96.66	0.01\\
96.67	0.01\\
96.68	0.01\\
96.69	0.01\\
96.7	0.01\\
96.71	0.01\\
96.72	0.01\\
96.73	0.01\\
96.74	0.01\\
96.75	0.01\\
96.76	0.01\\
96.77	0.01\\
96.78	0.01\\
96.79	0.01\\
96.8	0.01\\
96.81	0.01\\
96.82	0.01\\
96.83	0.01\\
96.84	0.01\\
96.85	0.01\\
96.86	0.01\\
96.87	0.01\\
96.88	0.01\\
96.89	0.01\\
96.9	0.01\\
96.91	0.01\\
96.92	0.01\\
96.93	0.01\\
96.94	0.01\\
96.95	0.01\\
96.96	0.01\\
96.97	0.01\\
96.98	0.01\\
96.99	0.01\\
97	0.01\\
97.01	0.01\\
97.02	0.01\\
97.03	0.01\\
97.04	0.01\\
97.05	0.01\\
97.06	0.01\\
97.07	0.01\\
97.08	0.01\\
97.09	0.01\\
97.1	0.01\\
97.11	0.01\\
97.12	0.01\\
97.13	0.01\\
97.14	0.01\\
97.15	0.01\\
97.16	0.01\\
97.17	0.01\\
97.18	0.01\\
97.19	0.01\\
97.2	0.01\\
97.21	0.01\\
97.22	0.01\\
97.23	0.01\\
97.24	0.01\\
97.25	0.01\\
97.26	0.01\\
97.27	0.01\\
97.28	0.01\\
97.29	0.01\\
97.3	0.01\\
97.31	0.01\\
97.32	0.01\\
97.33	0.01\\
97.34	0.01\\
97.35	0.01\\
97.36	0.01\\
97.37	0.01\\
97.38	0.01\\
97.39	0.01\\
97.4	0.01\\
97.41	0.01\\
97.42	0.01\\
97.43	0.01\\
97.44	0.01\\
97.45	0.01\\
97.46	0.01\\
97.47	0.01\\
97.48	0.01\\
97.49	0.01\\
97.5	0.01\\
97.51	0.01\\
97.52	0.01\\
97.53	0.01\\
97.54	0.01\\
97.55	0.01\\
97.56	0.01\\
97.57	0.01\\
97.58	0.01\\
97.59	0.01\\
97.6	0.01\\
97.61	0.01\\
97.62	0.01\\
97.63	0.01\\
97.64	0.01\\
97.65	0.01\\
97.66	0.01\\
97.67	0.01\\
97.68	0.01\\
97.69	0.01\\
97.7	0.01\\
97.71	0.01\\
97.72	0.01\\
97.73	0.01\\
97.74	0.01\\
97.75	0.01\\
97.76	0.01\\
97.77	0.01\\
97.78	0.01\\
97.79	0.01\\
97.8	0.01\\
97.81	0.01\\
97.82	0.01\\
97.83	0.01\\
97.84	0.01\\
97.85	0.01\\
97.86	0.01\\
97.87	0.01\\
97.88	0.01\\
97.89	0.01\\
97.9	0.01\\
97.91	0.01\\
97.92	0.01\\
97.93	0.01\\
97.94	0.01\\
97.95	0.01\\
97.96	0.01\\
97.97	0.01\\
97.98	0.01\\
97.99	0.01\\
98	0.01\\
98.01	0.01\\
98.02	0.01\\
98.03	0.01\\
98.04	0.01\\
98.05	0.01\\
98.06	0.01\\
98.07	0.01\\
98.08	0.01\\
98.09	0.01\\
98.1	0.01\\
98.11	0.01\\
98.12	0.01\\
98.13	0.01\\
98.14	0.01\\
98.15	0.01\\
98.16	0.01\\
98.17	0.01\\
98.18	0.01\\
98.19	0.00994129080068528\\
98.2	0.00986833933171061\\
98.21	0.00979481633259846\\
98.22	0.00972072772781104\\
98.23	0.00964606798886552\\
98.24	0.0095708315261914\\
98.25	0.00949501268829988\\
98.26	0.0094186057609382\\
98.27	0.0093416049662285\\
98.28	0.009264004461791\\
98.29	0.00918579833985082\\
98.3	0.00910698062420501\\
98.31	0.00902754527090133\\
98.32	0.00894748616770274\\
98.33	0.00886679713292877\\
98.34	0.00878547191258998\\
98.35	0.00870350418137153\\
98.36	0.00862088754160808\\
98.37	0.00859563949916318\\
98.38	0.0085722673523513\\
98.39	0.00854870811048133\\
98.4	0.00852496031668337\\
98.41	0.00850102250672303\\
98.42	0.0084768932090755\\
98.43	0.00845256169615563\\
98.44	0.00842802342245719\\
98.45	0.00840327666416255\\
98.46	0.00837831968659693\\
98.47	0.00835315074425895\\
98.48	0.00832776808085526\\
98.49	0.00830216992933915\\
98.5	0.00827635451195351\\
98.51	0.00825032004027816\\
98.52	0.0082240647152819\\
98.53	0.00819758672737922\\
98.54	0.00817088425649252\\
98.55	0.00814395547212306\\
98.56	0.00811679853342339\\
98.57	0.00808941158927525\\
98.58	0.0080617927783733\\
98.59	0.0080339402293148\\
98.6	0.00800585206069563\\
98.61	0.00797752638121268\\
98.62	0.00794896128977298\\
98.63	0.00792015487560991\\
98.64	0.00789110521840648\\
98.65	0.00786181038842636\\
98.66	0.00783226844556139\\
98.67	0.00780247744016606\\
98.68	0.00777243541358119\\
98.69	0.00774212371938256\\
98.7	0.0077115345323865\\
98.71	0.00768066518036141\\
98.72	0.0076495129650221\\
98.73	0.00761807516177867\\
98.74	0.00758634902334191\\
98.75	0.00755433177852406\\
98.76	0.00752202062929114\\
98.77	0.00748941275049575\\
98.78	0.00745650528960692\\
98.79	0.00742329537683375\\
98.8	0.00738978011787402\\
98.81	0.007355956590778\\
98.82	0.00732182184568431\\
98.83	0.00728737290524804\\
98.84	0.00725260677532317\\
98.85	0.00721752047072853\\
98.86	0.00718211097824068\\
98.87	0.00714637525633141\\
98.88	0.00711031023490265\\
98.89	0.00707391281501909\\
98.9	0.00703717986863816\\
98.91	0.00700010823833766\\
98.92	0.00696269473704068\\
98.93	0.00692493614773813\\
98.94	0.00688682922320853\\
98.95	0.00684837068573539\\
98.96	0.00680955722682175\\
98.97	0.00677038550690228\\
98.98	0.0067308521550525\\
98.99	0.00669095376869546\\
99	0.00665068691330562\\
99.01	0.00661004812211003\\
99.02	0.00656903389578665\\
99.03	0.00652764070215999\\
99.04	0.00648586497589378\\
99.05	0.00644370311818096\\
99.06	0.00640115149643064\\
99.07	0.00635820644395225\\
99.08	0.00631486425961415\\
99.09	0.00627112120750197\\
99.1	0.00622697351659171\\
99.11	0.00618241738041986\\
99.12	0.00613744895674713\\
99.13	0.0060920643672213\\
99.14	0.00604625969703757\\
99.15	0.00600003099459582\\
99.16	0.00595337427115449\\
99.17	0.0059062855004812\\
99.18	0.00585876061850007\\
99.19	0.00581079552293571\\
99.2	0.00576238607295369\\
99.21	0.00571352808879781\\
99.22	0.00566421735142363\\
99.23	0.00561444960212877\\
99.24	0.00556422054217949\\
99.25	0.00551352583243382\\
99.26	0.00546236109296097\\
99.27	0.00541072190265723\\
99.28	0.00535860379885804\\
99.29	0.0053060022769465\\
99.3	0.00525291278995796\\
99.31	0.00519933074818091\\
99.32	0.00514525151875404\\
99.33	0.00509067042525927\\
99.34	0.00503558274731109\\
99.35	0.00497998372014168\\
99.36	0.00492386853418223\\
99.37	0.00486723233464002\\
99.38	0.00481007022107158\\
99.39	0.00475237724695148\\
99.4	0.00469414841923717\\
99.41	0.00463537869792936\\
99.42	0.00457606299562822\\
99.43	0.00451619617708523\\
99.44	0.00445577305875056\\
99.45	0.00439478840831612\\
99.46	0.00433323694425396\\
99.47	0.00427111333535026\\
99.48	0.00420841220023459\\
99.49	0.00414512810690458\\
99.5	0.00408125557224581\\
99.51	0.00401678906154691\\
99.52	0.00395172298800988\\
99.53	0.00388605171225538\\
99.54	0.00381976954182313\\
99.55	0.00375287073066724\\
99.56	0.00368534947864641\\
99.57	0.00361719993100894\\
99.58	0.0035484161778725\\
99.59	0.00347899225369861\\
99.6	0.00340892213676156\\
99.61	0.00333819977061685\\
99.62	0.00326681905321176\\
99.63	0.00319477382533651\\
99.64	0.00312205787008127\\
99.65	0.00304866491228757\\
99.66	0.00297458861799396\\
99.67	0.00289982259387588\\
99.68	0.00282436038667954\\
99.69	0.00274819548264983\\
99.7	0.00267132130695209\\
99.71	0.00259373122305209\\
99.72	0.00251541853212007\\
99.73	0.00243637647243111\\
99.74	0.0023565982187586\\
99.75	0.00227607688176103\\
99.76	0.0021948055073617\\
99.77	0.00211277707612147\\
99.78	0.00202998450260408\\
99.79	0.00194642063473418\\
99.8	0.00186207825314774\\
99.81	0.00177695007053471\\
99.82	0.00169102873097375\\
99.83	0.00160430680925891\\
99.84	0.00151677681021786\\
99.85	0.00142843116802176\\
99.86	0.00133926224548628\\
99.87	0.0012492623333637\\
99.88	0.00115842364962579\\
99.89	0.00106673833873724\\
99.9	0.000974198470919339\\
99.91	0.000880796041403625\\
99.92	0.000786522969675284\\
99.93	0.000691371098705833\\
99.94	0.000595332194174888\\
99.95	0.000498397943680602\\
99.96	0.000400559955938384\\
99.97	0.000301809759967542\\
99.98	0.000202138804265397\\
99.99	0.000101538455968433\\
100	0\\
};
\addlegendentry{$q=3$};

\addplot [color=green,solid,forget plot]
  table[row sep=crcr]{%
0.01	0.01\\
0.02	0.01\\
0.03	0.01\\
0.04	0.01\\
0.05	0.01\\
0.06	0.01\\
0.07	0.01\\
0.08	0.01\\
0.09	0.01\\
0.1	0.01\\
0.11	0.01\\
0.12	0.01\\
0.13	0.01\\
0.14	0.01\\
0.15	0.01\\
0.16	0.01\\
0.17	0.01\\
0.18	0.01\\
0.19	0.01\\
0.2	0.01\\
0.21	0.01\\
0.22	0.01\\
0.23	0.01\\
0.24	0.01\\
0.25	0.01\\
0.26	0.01\\
0.27	0.01\\
0.28	0.01\\
0.29	0.01\\
0.3	0.01\\
0.31	0.01\\
0.32	0.01\\
0.33	0.01\\
0.34	0.01\\
0.35	0.01\\
0.36	0.01\\
0.37	0.01\\
0.38	0.01\\
0.39	0.01\\
0.4	0.01\\
0.41	0.01\\
0.42	0.01\\
0.43	0.01\\
0.44	0.01\\
0.45	0.01\\
0.46	0.01\\
0.47	0.01\\
0.48	0.01\\
0.49	0.01\\
0.5	0.01\\
0.51	0.01\\
0.52	0.01\\
0.53	0.01\\
0.54	0.01\\
0.55	0.01\\
0.56	0.01\\
0.57	0.01\\
0.58	0.01\\
0.59	0.01\\
0.6	0.01\\
0.61	0.01\\
0.62	0.01\\
0.63	0.01\\
0.64	0.01\\
0.65	0.01\\
0.66	0.01\\
0.67	0.01\\
0.68	0.01\\
0.69	0.01\\
0.7	0.01\\
0.71	0.01\\
0.72	0.01\\
0.73	0.01\\
0.74	0.01\\
0.75	0.01\\
0.76	0.01\\
0.77	0.01\\
0.78	0.01\\
0.79	0.01\\
0.8	0.01\\
0.81	0.01\\
0.82	0.01\\
0.83	0.01\\
0.84	0.01\\
0.85	0.01\\
0.86	0.01\\
0.87	0.01\\
0.88	0.01\\
0.89	0.01\\
0.9	0.01\\
0.91	0.01\\
0.92	0.01\\
0.93	0.01\\
0.94	0.01\\
0.95	0.01\\
0.96	0.01\\
0.97	0.01\\
0.98	0.01\\
0.99	0.01\\
1	0.01\\
1.01	0.01\\
1.02	0.01\\
1.03	0.01\\
1.04	0.01\\
1.05	0.01\\
1.06	0.01\\
1.07	0.01\\
1.08	0.01\\
1.09	0.01\\
1.1	0.01\\
1.11	0.01\\
1.12	0.01\\
1.13	0.01\\
1.14	0.01\\
1.15	0.01\\
1.16	0.01\\
1.17	0.01\\
1.18	0.01\\
1.19	0.01\\
1.2	0.01\\
1.21	0.01\\
1.22	0.01\\
1.23	0.01\\
1.24	0.01\\
1.25	0.01\\
1.26	0.01\\
1.27	0.01\\
1.28	0.01\\
1.29	0.01\\
1.3	0.01\\
1.31	0.01\\
1.32	0.01\\
1.33	0.01\\
1.34	0.01\\
1.35	0.01\\
1.36	0.01\\
1.37	0.01\\
1.38	0.01\\
1.39	0.01\\
1.4	0.01\\
1.41	0.01\\
1.42	0.01\\
1.43	0.01\\
1.44	0.01\\
1.45	0.01\\
1.46	0.01\\
1.47	0.01\\
1.48	0.01\\
1.49	0.01\\
1.5	0.01\\
1.51	0.01\\
1.52	0.01\\
1.53	0.01\\
1.54	0.01\\
1.55	0.01\\
1.56	0.01\\
1.57	0.01\\
1.58	0.01\\
1.59	0.01\\
1.6	0.01\\
1.61	0.01\\
1.62	0.01\\
1.63	0.01\\
1.64	0.01\\
1.65	0.01\\
1.66	0.01\\
1.67	0.01\\
1.68	0.01\\
1.69	0.01\\
1.7	0.01\\
1.71	0.01\\
1.72	0.01\\
1.73	0.01\\
1.74	0.01\\
1.75	0.01\\
1.76	0.01\\
1.77	0.01\\
1.78	0.01\\
1.79	0.01\\
1.8	0.01\\
1.81	0.01\\
1.82	0.01\\
1.83	0.01\\
1.84	0.01\\
1.85	0.01\\
1.86	0.01\\
1.87	0.01\\
1.88	0.01\\
1.89	0.01\\
1.9	0.01\\
1.91	0.01\\
1.92	0.01\\
1.93	0.01\\
1.94	0.01\\
1.95	0.01\\
1.96	0.01\\
1.97	0.01\\
1.98	0.01\\
1.99	0.01\\
2	0.01\\
2.01	0.01\\
2.02	0.01\\
2.03	0.01\\
2.04	0.01\\
2.05	0.01\\
2.06	0.01\\
2.07	0.01\\
2.08	0.01\\
2.09	0.01\\
2.1	0.01\\
2.11	0.01\\
2.12	0.01\\
2.13	0.01\\
2.14	0.01\\
2.15	0.01\\
2.16	0.01\\
2.17	0.01\\
2.18	0.01\\
2.19	0.01\\
2.2	0.01\\
2.21	0.01\\
2.22	0.01\\
2.23	0.01\\
2.24	0.01\\
2.25	0.01\\
2.26	0.01\\
2.27	0.01\\
2.28	0.01\\
2.29	0.01\\
2.3	0.01\\
2.31	0.01\\
2.32	0.01\\
2.33	0.01\\
2.34	0.01\\
2.35	0.01\\
2.36	0.01\\
2.37	0.01\\
2.38	0.01\\
2.39	0.01\\
2.4	0.01\\
2.41	0.01\\
2.42	0.01\\
2.43	0.01\\
2.44	0.01\\
2.45	0.01\\
2.46	0.01\\
2.47	0.01\\
2.48	0.01\\
2.49	0.01\\
2.5	0.01\\
2.51	0.01\\
2.52	0.01\\
2.53	0.01\\
2.54	0.01\\
2.55	0.01\\
2.56	0.01\\
2.57	0.01\\
2.58	0.01\\
2.59	0.01\\
2.6	0.01\\
2.61	0.01\\
2.62	0.01\\
2.63	0.01\\
2.64	0.01\\
2.65	0.01\\
2.66	0.01\\
2.67	0.01\\
2.68	0.01\\
2.69	0.01\\
2.7	0.01\\
2.71	0.01\\
2.72	0.01\\
2.73	0.01\\
2.74	0.01\\
2.75	0.01\\
2.76	0.01\\
2.77	0.01\\
2.78	0.01\\
2.79	0.01\\
2.8	0.01\\
2.81	0.01\\
2.82	0.01\\
2.83	0.01\\
2.84	0.01\\
2.85	0.01\\
2.86	0.01\\
2.87	0.01\\
2.88	0.01\\
2.89	0.01\\
2.9	0.01\\
2.91	0.01\\
2.92	0.01\\
2.93	0.01\\
2.94	0.01\\
2.95	0.01\\
2.96	0.01\\
2.97	0.01\\
2.98	0.01\\
2.99	0.01\\
3	0.01\\
3.01	0.01\\
3.02	0.01\\
3.03	0.01\\
3.04	0.01\\
3.05	0.01\\
3.06	0.01\\
3.07	0.01\\
3.08	0.01\\
3.09	0.01\\
3.1	0.01\\
3.11	0.01\\
3.12	0.01\\
3.13	0.01\\
3.14	0.01\\
3.15	0.01\\
3.16	0.01\\
3.17	0.01\\
3.18	0.01\\
3.19	0.01\\
3.2	0.01\\
3.21	0.01\\
3.22	0.01\\
3.23	0.01\\
3.24	0.01\\
3.25	0.01\\
3.26	0.01\\
3.27	0.01\\
3.28	0.01\\
3.29	0.01\\
3.3	0.01\\
3.31	0.01\\
3.32	0.01\\
3.33	0.01\\
3.34	0.01\\
3.35	0.01\\
3.36	0.01\\
3.37	0.01\\
3.38	0.01\\
3.39	0.01\\
3.4	0.01\\
3.41	0.01\\
3.42	0.01\\
3.43	0.01\\
3.44	0.01\\
3.45	0.01\\
3.46	0.01\\
3.47	0.01\\
3.48	0.01\\
3.49	0.01\\
3.5	0.01\\
3.51	0.01\\
3.52	0.01\\
3.53	0.01\\
3.54	0.01\\
3.55	0.01\\
3.56	0.01\\
3.57	0.01\\
3.58	0.01\\
3.59	0.01\\
3.6	0.01\\
3.61	0.01\\
3.62	0.01\\
3.63	0.01\\
3.64	0.01\\
3.65	0.01\\
3.66	0.01\\
3.67	0.01\\
3.68	0.01\\
3.69	0.01\\
3.7	0.01\\
3.71	0.01\\
3.72	0.01\\
3.73	0.01\\
3.74	0.01\\
3.75	0.01\\
3.76	0.01\\
3.77	0.01\\
3.78	0.01\\
3.79	0.01\\
3.8	0.01\\
3.81	0.01\\
3.82	0.01\\
3.83	0.01\\
3.84	0.01\\
3.85	0.01\\
3.86	0.01\\
3.87	0.01\\
3.88	0.01\\
3.89	0.01\\
3.9	0.01\\
3.91	0.01\\
3.92	0.01\\
3.93	0.01\\
3.94	0.01\\
3.95	0.01\\
3.96	0.01\\
3.97	0.01\\
3.98	0.01\\
3.99	0.01\\
4	0.01\\
4.01	0.01\\
4.02	0.01\\
4.03	0.01\\
4.04	0.01\\
4.05	0.01\\
4.06	0.01\\
4.07	0.01\\
4.08	0.01\\
4.09	0.01\\
4.1	0.01\\
4.11	0.01\\
4.12	0.01\\
4.13	0.01\\
4.14	0.01\\
4.15	0.01\\
4.16	0.01\\
4.17	0.01\\
4.18	0.01\\
4.19	0.01\\
4.2	0.01\\
4.21	0.01\\
4.22	0.01\\
4.23	0.01\\
4.24	0.01\\
4.25	0.01\\
4.26	0.01\\
4.27	0.01\\
4.28	0.01\\
4.29	0.01\\
4.3	0.01\\
4.31	0.01\\
4.32	0.01\\
4.33	0.01\\
4.34	0.01\\
4.35	0.01\\
4.36	0.01\\
4.37	0.01\\
4.38	0.01\\
4.39	0.01\\
4.4	0.01\\
4.41	0.01\\
4.42	0.01\\
4.43	0.01\\
4.44	0.01\\
4.45	0.01\\
4.46	0.01\\
4.47	0.01\\
4.48	0.01\\
4.49	0.01\\
4.5	0.01\\
4.51	0.01\\
4.52	0.01\\
4.53	0.01\\
4.54	0.01\\
4.55	0.01\\
4.56	0.01\\
4.57	0.01\\
4.58	0.01\\
4.59	0.01\\
4.6	0.01\\
4.61	0.01\\
4.62	0.01\\
4.63	0.01\\
4.64	0.01\\
4.65	0.01\\
4.66	0.01\\
4.67	0.01\\
4.68	0.01\\
4.69	0.01\\
4.7	0.01\\
4.71	0.01\\
4.72	0.01\\
4.73	0.01\\
4.74	0.01\\
4.75	0.01\\
4.76	0.01\\
4.77	0.01\\
4.78	0.01\\
4.79	0.01\\
4.8	0.01\\
4.81	0.01\\
4.82	0.01\\
4.83	0.01\\
4.84	0.01\\
4.85	0.01\\
4.86	0.01\\
4.87	0.01\\
4.88	0.01\\
4.89	0.01\\
4.9	0.01\\
4.91	0.01\\
4.92	0.01\\
4.93	0.01\\
4.94	0.01\\
4.95	0.01\\
4.96	0.01\\
4.97	0.01\\
4.98	0.01\\
4.99	0.01\\
5	0.01\\
5.01	0.01\\
5.02	0.01\\
5.03	0.01\\
5.04	0.01\\
5.05	0.01\\
5.06	0.01\\
5.07	0.01\\
5.08	0.01\\
5.09	0.01\\
5.1	0.01\\
5.11	0.01\\
5.12	0.01\\
5.13	0.01\\
5.14	0.01\\
5.15	0.01\\
5.16	0.01\\
5.17	0.01\\
5.18	0.01\\
5.19	0.01\\
5.2	0.01\\
5.21	0.01\\
5.22	0.01\\
5.23	0.01\\
5.24	0.01\\
5.25	0.01\\
5.26	0.01\\
5.27	0.01\\
5.28	0.01\\
5.29	0.01\\
5.3	0.01\\
5.31	0.01\\
5.32	0.01\\
5.33	0.01\\
5.34	0.01\\
5.35	0.01\\
5.36	0.01\\
5.37	0.01\\
5.38	0.01\\
5.39	0.01\\
5.4	0.01\\
5.41	0.01\\
5.42	0.01\\
5.43	0.01\\
5.44	0.01\\
5.45	0.01\\
5.46	0.01\\
5.47	0.01\\
5.48	0.01\\
5.49	0.01\\
5.5	0.01\\
5.51	0.01\\
5.52	0.01\\
5.53	0.01\\
5.54	0.01\\
5.55	0.01\\
5.56	0.01\\
5.57	0.01\\
5.58	0.01\\
5.59	0.01\\
5.6	0.01\\
5.61	0.01\\
5.62	0.01\\
5.63	0.01\\
5.64	0.01\\
5.65	0.01\\
5.66	0.01\\
5.67	0.01\\
5.68	0.01\\
5.69	0.01\\
5.7	0.01\\
5.71	0.01\\
5.72	0.01\\
5.73	0.01\\
5.74	0.01\\
5.75	0.01\\
5.76	0.01\\
5.77	0.01\\
5.78	0.01\\
5.79	0.01\\
5.8	0.01\\
5.81	0.01\\
5.82	0.01\\
5.83	0.01\\
5.84	0.01\\
5.85	0.01\\
5.86	0.01\\
5.87	0.01\\
5.88	0.01\\
5.89	0.01\\
5.9	0.01\\
5.91	0.01\\
5.92	0.01\\
5.93	0.01\\
5.94	0.01\\
5.95	0.01\\
5.96	0.01\\
5.97	0.01\\
5.98	0.01\\
5.99	0.01\\
6	0.01\\
6.01	0.01\\
6.02	0.01\\
6.03	0.01\\
6.04	0.01\\
6.05	0.01\\
6.06	0.01\\
6.07	0.01\\
6.08	0.01\\
6.09	0.01\\
6.1	0.01\\
6.11	0.01\\
6.12	0.01\\
6.13	0.01\\
6.14	0.01\\
6.15	0.01\\
6.16	0.01\\
6.17	0.01\\
6.18	0.01\\
6.19	0.01\\
6.2	0.01\\
6.21	0.01\\
6.22	0.01\\
6.23	0.01\\
6.24	0.01\\
6.25	0.01\\
6.26	0.01\\
6.27	0.01\\
6.28	0.01\\
6.29	0.01\\
6.3	0.01\\
6.31	0.01\\
6.32	0.01\\
6.33	0.01\\
6.34	0.01\\
6.35	0.01\\
6.36	0.01\\
6.37	0.01\\
6.38	0.01\\
6.39	0.01\\
6.4	0.01\\
6.41	0.01\\
6.42	0.01\\
6.43	0.01\\
6.44	0.01\\
6.45	0.01\\
6.46	0.01\\
6.47	0.01\\
6.48	0.01\\
6.49	0.01\\
6.5	0.01\\
6.51	0.01\\
6.52	0.01\\
6.53	0.01\\
6.54	0.01\\
6.55	0.01\\
6.56	0.01\\
6.57	0.01\\
6.58	0.01\\
6.59	0.01\\
6.6	0.01\\
6.61	0.01\\
6.62	0.01\\
6.63	0.01\\
6.64	0.01\\
6.65	0.01\\
6.66	0.01\\
6.67	0.01\\
6.68	0.01\\
6.69	0.01\\
6.7	0.01\\
6.71	0.01\\
6.72	0.01\\
6.73	0.01\\
6.74	0.01\\
6.75	0.01\\
6.76	0.01\\
6.77	0.01\\
6.78	0.01\\
6.79	0.01\\
6.8	0.01\\
6.81	0.01\\
6.82	0.01\\
6.83	0.01\\
6.84	0.01\\
6.85	0.01\\
6.86	0.01\\
6.87	0.01\\
6.88	0.01\\
6.89	0.01\\
6.9	0.01\\
6.91	0.01\\
6.92	0.01\\
6.93	0.01\\
6.94	0.01\\
6.95	0.01\\
6.96	0.01\\
6.97	0.01\\
6.98	0.01\\
6.99	0.01\\
7	0.01\\
7.01	0.01\\
7.02	0.01\\
7.03	0.01\\
7.04	0.01\\
7.05	0.01\\
7.06	0.01\\
7.07	0.01\\
7.08	0.01\\
7.09	0.01\\
7.1	0.01\\
7.11	0.01\\
7.12	0.01\\
7.13	0.01\\
7.14	0.01\\
7.15	0.01\\
7.16	0.01\\
7.17	0.01\\
7.18	0.01\\
7.19	0.01\\
7.2	0.01\\
7.21	0.01\\
7.22	0.01\\
7.23	0.01\\
7.24	0.01\\
7.25	0.01\\
7.26	0.01\\
7.27	0.01\\
7.28	0.01\\
7.29	0.01\\
7.3	0.01\\
7.31	0.01\\
7.32	0.01\\
7.33	0.01\\
7.34	0.01\\
7.35	0.01\\
7.36	0.01\\
7.37	0.01\\
7.38	0.01\\
7.39	0.01\\
7.4	0.01\\
7.41	0.01\\
7.42	0.01\\
7.43	0.01\\
7.44	0.01\\
7.45	0.01\\
7.46	0.01\\
7.47	0.01\\
7.48	0.01\\
7.49	0.01\\
7.5	0.01\\
7.51	0.01\\
7.52	0.01\\
7.53	0.01\\
7.54	0.01\\
7.55	0.01\\
7.56	0.01\\
7.57	0.01\\
7.58	0.01\\
7.59	0.01\\
7.6	0.01\\
7.61	0.01\\
7.62	0.01\\
7.63	0.01\\
7.64	0.01\\
7.65	0.01\\
7.66	0.01\\
7.67	0.01\\
7.68	0.01\\
7.69	0.01\\
7.7	0.01\\
7.71	0.01\\
7.72	0.01\\
7.73	0.01\\
7.74	0.01\\
7.75	0.01\\
7.76	0.01\\
7.77	0.01\\
7.78	0.01\\
7.79	0.01\\
7.8	0.01\\
7.81	0.01\\
7.82	0.01\\
7.83	0.01\\
7.84	0.01\\
7.85	0.01\\
7.86	0.01\\
7.87	0.01\\
7.88	0.01\\
7.89	0.01\\
7.9	0.01\\
7.91	0.01\\
7.92	0.01\\
7.93	0.01\\
7.94	0.01\\
7.95	0.01\\
7.96	0.01\\
7.97	0.01\\
7.98	0.01\\
7.99	0.01\\
8	0.01\\
8.01	0.01\\
8.02	0.01\\
8.03	0.01\\
8.04	0.01\\
8.05	0.01\\
8.06	0.01\\
8.07	0.01\\
8.08	0.01\\
8.09	0.01\\
8.1	0.01\\
8.11	0.01\\
8.12	0.01\\
8.13	0.01\\
8.14	0.01\\
8.15	0.01\\
8.16	0.01\\
8.17	0.01\\
8.18	0.01\\
8.19	0.01\\
8.2	0.01\\
8.21	0.01\\
8.22	0.01\\
8.23	0.01\\
8.24	0.01\\
8.25	0.01\\
8.26	0.01\\
8.27	0.01\\
8.28	0.01\\
8.29	0.01\\
8.3	0.01\\
8.31	0.01\\
8.32	0.01\\
8.33	0.01\\
8.34	0.01\\
8.35	0.01\\
8.36	0.01\\
8.37	0.01\\
8.38	0.01\\
8.39	0.01\\
8.4	0.01\\
8.41	0.01\\
8.42	0.01\\
8.43	0.01\\
8.44	0.01\\
8.45	0.01\\
8.46	0.01\\
8.47	0.01\\
8.48	0.01\\
8.49	0.01\\
8.5	0.01\\
8.51	0.01\\
8.52	0.01\\
8.53	0.01\\
8.54	0.01\\
8.55	0.01\\
8.56	0.01\\
8.57	0.01\\
8.58	0.01\\
8.59	0.01\\
8.6	0.01\\
8.61	0.01\\
8.62	0.01\\
8.63	0.01\\
8.64	0.01\\
8.65	0.01\\
8.66	0.01\\
8.67	0.01\\
8.68	0.01\\
8.69	0.01\\
8.7	0.01\\
8.71	0.01\\
8.72	0.01\\
8.73	0.01\\
8.74	0.01\\
8.75	0.01\\
8.76	0.01\\
8.77	0.01\\
8.78	0.01\\
8.79	0.01\\
8.8	0.01\\
8.81	0.01\\
8.82	0.01\\
8.83	0.01\\
8.84	0.01\\
8.85	0.01\\
8.86	0.01\\
8.87	0.01\\
8.88	0.01\\
8.89	0.01\\
8.9	0.01\\
8.91	0.01\\
8.92	0.01\\
8.93	0.01\\
8.94	0.01\\
8.95	0.01\\
8.96	0.01\\
8.97	0.01\\
8.98	0.01\\
8.99	0.01\\
9	0.01\\
9.01	0.01\\
9.02	0.01\\
9.03	0.01\\
9.04	0.01\\
9.05	0.01\\
9.06	0.01\\
9.07	0.01\\
9.08	0.01\\
9.09	0.01\\
9.1	0.01\\
9.11	0.01\\
9.12	0.01\\
9.13	0.01\\
9.14	0.01\\
9.15	0.01\\
9.16	0.01\\
9.17	0.01\\
9.18	0.01\\
9.19	0.01\\
9.2	0.01\\
9.21	0.01\\
9.22	0.01\\
9.23	0.01\\
9.24	0.01\\
9.25	0.01\\
9.26	0.01\\
9.27	0.01\\
9.28	0.01\\
9.29	0.01\\
9.3	0.01\\
9.31	0.01\\
9.32	0.01\\
9.33	0.01\\
9.34	0.01\\
9.35	0.01\\
9.36	0.01\\
9.37	0.01\\
9.38	0.01\\
9.39	0.01\\
9.4	0.01\\
9.41	0.01\\
9.42	0.01\\
9.43	0.01\\
9.44	0.01\\
9.45	0.01\\
9.46	0.01\\
9.47	0.01\\
9.48	0.01\\
9.49	0.01\\
9.5	0.01\\
9.51	0.01\\
9.52	0.01\\
9.53	0.01\\
9.54	0.01\\
9.55	0.01\\
9.56	0.01\\
9.57	0.01\\
9.58	0.01\\
9.59	0.01\\
9.6	0.01\\
9.61	0.01\\
9.62	0.01\\
9.63	0.01\\
9.64	0.01\\
9.65	0.01\\
9.66	0.01\\
9.67	0.01\\
9.68	0.01\\
9.69	0.01\\
9.7	0.01\\
9.71	0.01\\
9.72	0.01\\
9.73	0.01\\
9.74	0.01\\
9.75	0.01\\
9.76	0.01\\
9.77	0.01\\
9.78	0.01\\
9.79	0.01\\
9.8	0.01\\
9.81	0.01\\
9.82	0.01\\
9.83	0.01\\
9.84	0.01\\
9.85	0.01\\
9.86	0.01\\
9.87	0.01\\
9.88	0.01\\
9.89	0.01\\
9.9	0.01\\
9.91	0.01\\
9.92	0.01\\
9.93	0.01\\
9.94	0.01\\
9.95	0.01\\
9.96	0.01\\
9.97	0.01\\
9.98	0.01\\
9.99	0.01\\
10	0.01\\
10.01	0.01\\
10.02	0.01\\
10.03	0.01\\
10.04	0.01\\
10.05	0.01\\
10.06	0.01\\
10.07	0.01\\
10.08	0.01\\
10.09	0.01\\
10.1	0.01\\
10.11	0.01\\
10.12	0.01\\
10.13	0.01\\
10.14	0.01\\
10.15	0.01\\
10.16	0.01\\
10.17	0.01\\
10.18	0.01\\
10.19	0.01\\
10.2	0.01\\
10.21	0.01\\
10.22	0.01\\
10.23	0.01\\
10.24	0.01\\
10.25	0.01\\
10.26	0.01\\
10.27	0.01\\
10.28	0.01\\
10.29	0.01\\
10.3	0.01\\
10.31	0.01\\
10.32	0.01\\
10.33	0.01\\
10.34	0.01\\
10.35	0.01\\
10.36	0.01\\
10.37	0.01\\
10.38	0.01\\
10.39	0.01\\
10.4	0.01\\
10.41	0.01\\
10.42	0.01\\
10.43	0.01\\
10.44	0.01\\
10.45	0.01\\
10.46	0.01\\
10.47	0.01\\
10.48	0.01\\
10.49	0.01\\
10.5	0.01\\
10.51	0.01\\
10.52	0.01\\
10.53	0.01\\
10.54	0.01\\
10.55	0.01\\
10.56	0.01\\
10.57	0.01\\
10.58	0.01\\
10.59	0.01\\
10.6	0.01\\
10.61	0.01\\
10.62	0.01\\
10.63	0.01\\
10.64	0.01\\
10.65	0.01\\
10.66	0.01\\
10.67	0.01\\
10.68	0.01\\
10.69	0.01\\
10.7	0.01\\
10.71	0.01\\
10.72	0.01\\
10.73	0.01\\
10.74	0.01\\
10.75	0.01\\
10.76	0.01\\
10.77	0.01\\
10.78	0.01\\
10.79	0.01\\
10.8	0.01\\
10.81	0.01\\
10.82	0.01\\
10.83	0.01\\
10.84	0.01\\
10.85	0.01\\
10.86	0.01\\
10.87	0.01\\
10.88	0.01\\
10.89	0.01\\
10.9	0.01\\
10.91	0.01\\
10.92	0.01\\
10.93	0.01\\
10.94	0.01\\
10.95	0.01\\
10.96	0.01\\
10.97	0.01\\
10.98	0.01\\
10.99	0.01\\
11	0.01\\
11.01	0.01\\
11.02	0.01\\
11.03	0.01\\
11.04	0.01\\
11.05	0.01\\
11.06	0.01\\
11.07	0.01\\
11.08	0.01\\
11.09	0.01\\
11.1	0.01\\
11.11	0.01\\
11.12	0.01\\
11.13	0.01\\
11.14	0.01\\
11.15	0.01\\
11.16	0.01\\
11.17	0.01\\
11.18	0.01\\
11.19	0.01\\
11.2	0.01\\
11.21	0.01\\
11.22	0.01\\
11.23	0.01\\
11.24	0.01\\
11.25	0.01\\
11.26	0.01\\
11.27	0.01\\
11.28	0.01\\
11.29	0.01\\
11.3	0.01\\
11.31	0.01\\
11.32	0.01\\
11.33	0.01\\
11.34	0.01\\
11.35	0.01\\
11.36	0.01\\
11.37	0.01\\
11.38	0.01\\
11.39	0.01\\
11.4	0.01\\
11.41	0.01\\
11.42	0.01\\
11.43	0.01\\
11.44	0.01\\
11.45	0.01\\
11.46	0.01\\
11.47	0.01\\
11.48	0.01\\
11.49	0.01\\
11.5	0.01\\
11.51	0.01\\
11.52	0.01\\
11.53	0.01\\
11.54	0.01\\
11.55	0.01\\
11.56	0.01\\
11.57	0.01\\
11.58	0.01\\
11.59	0.01\\
11.6	0.01\\
11.61	0.01\\
11.62	0.01\\
11.63	0.01\\
11.64	0.01\\
11.65	0.01\\
11.66	0.01\\
11.67	0.01\\
11.68	0.01\\
11.69	0.01\\
11.7	0.01\\
11.71	0.01\\
11.72	0.01\\
11.73	0.01\\
11.74	0.01\\
11.75	0.01\\
11.76	0.01\\
11.77	0.01\\
11.78	0.01\\
11.79	0.01\\
11.8	0.01\\
11.81	0.01\\
11.82	0.01\\
11.83	0.01\\
11.84	0.01\\
11.85	0.01\\
11.86	0.01\\
11.87	0.01\\
11.88	0.01\\
11.89	0.01\\
11.9	0.01\\
11.91	0.01\\
11.92	0.01\\
11.93	0.01\\
11.94	0.01\\
11.95	0.01\\
11.96	0.01\\
11.97	0.01\\
11.98	0.01\\
11.99	0.01\\
12	0.01\\
12.01	0.01\\
12.02	0.01\\
12.03	0.01\\
12.04	0.01\\
12.05	0.01\\
12.06	0.01\\
12.07	0.01\\
12.08	0.01\\
12.09	0.01\\
12.1	0.01\\
12.11	0.01\\
12.12	0.01\\
12.13	0.01\\
12.14	0.01\\
12.15	0.01\\
12.16	0.01\\
12.17	0.01\\
12.18	0.01\\
12.19	0.01\\
12.2	0.01\\
12.21	0.01\\
12.22	0.01\\
12.23	0.01\\
12.24	0.01\\
12.25	0.01\\
12.26	0.01\\
12.27	0.01\\
12.28	0.01\\
12.29	0.01\\
12.3	0.01\\
12.31	0.01\\
12.32	0.01\\
12.33	0.01\\
12.34	0.01\\
12.35	0.01\\
12.36	0.01\\
12.37	0.01\\
12.38	0.01\\
12.39	0.01\\
12.4	0.01\\
12.41	0.01\\
12.42	0.01\\
12.43	0.01\\
12.44	0.01\\
12.45	0.01\\
12.46	0.01\\
12.47	0.01\\
12.48	0.01\\
12.49	0.01\\
12.5	0.01\\
12.51	0.01\\
12.52	0.01\\
12.53	0.01\\
12.54	0.01\\
12.55	0.01\\
12.56	0.01\\
12.57	0.01\\
12.58	0.01\\
12.59	0.01\\
12.6	0.01\\
12.61	0.01\\
12.62	0.01\\
12.63	0.01\\
12.64	0.01\\
12.65	0.01\\
12.66	0.01\\
12.67	0.01\\
12.68	0.01\\
12.69	0.01\\
12.7	0.01\\
12.71	0.01\\
12.72	0.01\\
12.73	0.01\\
12.74	0.01\\
12.75	0.01\\
12.76	0.01\\
12.77	0.01\\
12.78	0.01\\
12.79	0.01\\
12.8	0.01\\
12.81	0.01\\
12.82	0.01\\
12.83	0.01\\
12.84	0.01\\
12.85	0.01\\
12.86	0.01\\
12.87	0.01\\
12.88	0.01\\
12.89	0.01\\
12.9	0.01\\
12.91	0.01\\
12.92	0.01\\
12.93	0.01\\
12.94	0.01\\
12.95	0.01\\
12.96	0.01\\
12.97	0.01\\
12.98	0.01\\
12.99	0.01\\
13	0.01\\
13.01	0.01\\
13.02	0.01\\
13.03	0.01\\
13.04	0.01\\
13.05	0.01\\
13.06	0.01\\
13.07	0.01\\
13.08	0.01\\
13.09	0.01\\
13.1	0.01\\
13.11	0.01\\
13.12	0.01\\
13.13	0.01\\
13.14	0.01\\
13.15	0.01\\
13.16	0.01\\
13.17	0.01\\
13.18	0.01\\
13.19	0.01\\
13.2	0.01\\
13.21	0.01\\
13.22	0.01\\
13.23	0.01\\
13.24	0.01\\
13.25	0.01\\
13.26	0.01\\
13.27	0.01\\
13.28	0.01\\
13.29	0.01\\
13.3	0.01\\
13.31	0.01\\
13.32	0.01\\
13.33	0.01\\
13.34	0.01\\
13.35	0.01\\
13.36	0.01\\
13.37	0.01\\
13.38	0.01\\
13.39	0.01\\
13.4	0.01\\
13.41	0.01\\
13.42	0.01\\
13.43	0.01\\
13.44	0.01\\
13.45	0.01\\
13.46	0.01\\
13.47	0.01\\
13.48	0.01\\
13.49	0.01\\
13.5	0.01\\
13.51	0.01\\
13.52	0.01\\
13.53	0.01\\
13.54	0.01\\
13.55	0.01\\
13.56	0.01\\
13.57	0.01\\
13.58	0.01\\
13.59	0.01\\
13.6	0.01\\
13.61	0.01\\
13.62	0.01\\
13.63	0.01\\
13.64	0.01\\
13.65	0.01\\
13.66	0.01\\
13.67	0.01\\
13.68	0.01\\
13.69	0.01\\
13.7	0.01\\
13.71	0.01\\
13.72	0.01\\
13.73	0.01\\
13.74	0.01\\
13.75	0.01\\
13.76	0.01\\
13.77	0.01\\
13.78	0.01\\
13.79	0.01\\
13.8	0.01\\
13.81	0.01\\
13.82	0.01\\
13.83	0.01\\
13.84	0.01\\
13.85	0.01\\
13.86	0.01\\
13.87	0.01\\
13.88	0.01\\
13.89	0.01\\
13.9	0.01\\
13.91	0.01\\
13.92	0.01\\
13.93	0.01\\
13.94	0.01\\
13.95	0.01\\
13.96	0.01\\
13.97	0.01\\
13.98	0.01\\
13.99	0.01\\
14	0.01\\
14.01	0.01\\
14.02	0.01\\
14.03	0.01\\
14.04	0.01\\
14.05	0.01\\
14.06	0.01\\
14.07	0.01\\
14.08	0.01\\
14.09	0.01\\
14.1	0.01\\
14.11	0.01\\
14.12	0.01\\
14.13	0.01\\
14.14	0.01\\
14.15	0.01\\
14.16	0.01\\
14.17	0.01\\
14.18	0.01\\
14.19	0.01\\
14.2	0.01\\
14.21	0.01\\
14.22	0.01\\
14.23	0.01\\
14.24	0.01\\
14.25	0.01\\
14.26	0.01\\
14.27	0.01\\
14.28	0.01\\
14.29	0.01\\
14.3	0.01\\
14.31	0.01\\
14.32	0.01\\
14.33	0.01\\
14.34	0.01\\
14.35	0.01\\
14.36	0.01\\
14.37	0.01\\
14.38	0.01\\
14.39	0.01\\
14.4	0.01\\
14.41	0.01\\
14.42	0.01\\
14.43	0.01\\
14.44	0.01\\
14.45	0.01\\
14.46	0.01\\
14.47	0.01\\
14.48	0.01\\
14.49	0.01\\
14.5	0.01\\
14.51	0.01\\
14.52	0.01\\
14.53	0.01\\
14.54	0.01\\
14.55	0.01\\
14.56	0.01\\
14.57	0.01\\
14.58	0.01\\
14.59	0.01\\
14.6	0.01\\
14.61	0.01\\
14.62	0.01\\
14.63	0.01\\
14.64	0.01\\
14.65	0.01\\
14.66	0.01\\
14.67	0.01\\
14.68	0.01\\
14.69	0.01\\
14.7	0.01\\
14.71	0.01\\
14.72	0.01\\
14.73	0.01\\
14.74	0.01\\
14.75	0.01\\
14.76	0.01\\
14.77	0.01\\
14.78	0.01\\
14.79	0.01\\
14.8	0.01\\
14.81	0.01\\
14.82	0.01\\
14.83	0.01\\
14.84	0.01\\
14.85	0.01\\
14.86	0.01\\
14.87	0.01\\
14.88	0.01\\
14.89	0.01\\
14.9	0.01\\
14.91	0.01\\
14.92	0.01\\
14.93	0.01\\
14.94	0.01\\
14.95	0.01\\
14.96	0.01\\
14.97	0.01\\
14.98	0.01\\
14.99	0.01\\
15	0.01\\
15.01	0.01\\
15.02	0.01\\
15.03	0.01\\
15.04	0.01\\
15.05	0.01\\
15.06	0.01\\
15.07	0.01\\
15.08	0.01\\
15.09	0.01\\
15.1	0.01\\
15.11	0.01\\
15.12	0.01\\
15.13	0.01\\
15.14	0.01\\
15.15	0.01\\
15.16	0.01\\
15.17	0.01\\
15.18	0.01\\
15.19	0.01\\
15.2	0.01\\
15.21	0.01\\
15.22	0.01\\
15.23	0.01\\
15.24	0.01\\
15.25	0.01\\
15.26	0.01\\
15.27	0.01\\
15.28	0.01\\
15.29	0.01\\
15.3	0.01\\
15.31	0.01\\
15.32	0.01\\
15.33	0.01\\
15.34	0.01\\
15.35	0.01\\
15.36	0.01\\
15.37	0.01\\
15.38	0.01\\
15.39	0.01\\
15.4	0.01\\
15.41	0.01\\
15.42	0.01\\
15.43	0.01\\
15.44	0.01\\
15.45	0.01\\
15.46	0.01\\
15.47	0.01\\
15.48	0.01\\
15.49	0.01\\
15.5	0.01\\
15.51	0.01\\
15.52	0.01\\
15.53	0.01\\
15.54	0.01\\
15.55	0.01\\
15.56	0.01\\
15.57	0.01\\
15.58	0.01\\
15.59	0.01\\
15.6	0.01\\
15.61	0.01\\
15.62	0.01\\
15.63	0.01\\
15.64	0.01\\
15.65	0.01\\
15.66	0.01\\
15.67	0.01\\
15.68	0.01\\
15.69	0.01\\
15.7	0.01\\
15.71	0.01\\
15.72	0.01\\
15.73	0.01\\
15.74	0.01\\
15.75	0.01\\
15.76	0.01\\
15.77	0.01\\
15.78	0.01\\
15.79	0.01\\
15.8	0.01\\
15.81	0.01\\
15.82	0.01\\
15.83	0.01\\
15.84	0.01\\
15.85	0.01\\
15.86	0.01\\
15.87	0.01\\
15.88	0.01\\
15.89	0.01\\
15.9	0.01\\
15.91	0.01\\
15.92	0.01\\
15.93	0.01\\
15.94	0.01\\
15.95	0.01\\
15.96	0.01\\
15.97	0.01\\
15.98	0.01\\
15.99	0.01\\
16	0.01\\
16.01	0.01\\
16.02	0.01\\
16.03	0.01\\
16.04	0.01\\
16.05	0.01\\
16.06	0.01\\
16.07	0.01\\
16.08	0.01\\
16.09	0.01\\
16.1	0.01\\
16.11	0.01\\
16.12	0.01\\
16.13	0.01\\
16.14	0.01\\
16.15	0.01\\
16.16	0.01\\
16.17	0.01\\
16.18	0.01\\
16.19	0.01\\
16.2	0.01\\
16.21	0.01\\
16.22	0.01\\
16.23	0.01\\
16.24	0.01\\
16.25	0.01\\
16.26	0.01\\
16.27	0.01\\
16.28	0.01\\
16.29	0.01\\
16.3	0.01\\
16.31	0.01\\
16.32	0.01\\
16.33	0.01\\
16.34	0.01\\
16.35	0.01\\
16.36	0.01\\
16.37	0.01\\
16.38	0.01\\
16.39	0.01\\
16.4	0.01\\
16.41	0.01\\
16.42	0.01\\
16.43	0.01\\
16.44	0.01\\
16.45	0.01\\
16.46	0.01\\
16.47	0.01\\
16.48	0.01\\
16.49	0.01\\
16.5	0.01\\
16.51	0.01\\
16.52	0.01\\
16.53	0.01\\
16.54	0.01\\
16.55	0.01\\
16.56	0.01\\
16.57	0.01\\
16.58	0.01\\
16.59	0.01\\
16.6	0.01\\
16.61	0.01\\
16.62	0.01\\
16.63	0.01\\
16.64	0.01\\
16.65	0.01\\
16.66	0.01\\
16.67	0.01\\
16.68	0.01\\
16.69	0.01\\
16.7	0.01\\
16.71	0.01\\
16.72	0.01\\
16.73	0.01\\
16.74	0.01\\
16.75	0.01\\
16.76	0.01\\
16.77	0.01\\
16.78	0.01\\
16.79	0.01\\
16.8	0.01\\
16.81	0.01\\
16.82	0.01\\
16.83	0.01\\
16.84	0.01\\
16.85	0.01\\
16.86	0.01\\
16.87	0.01\\
16.88	0.01\\
16.89	0.01\\
16.9	0.01\\
16.91	0.01\\
16.92	0.01\\
16.93	0.01\\
16.94	0.01\\
16.95	0.01\\
16.96	0.01\\
16.97	0.01\\
16.98	0.01\\
16.99	0.01\\
17	0.01\\
17.01	0.01\\
17.02	0.01\\
17.03	0.01\\
17.04	0.01\\
17.05	0.01\\
17.06	0.01\\
17.07	0.01\\
17.08	0.01\\
17.09	0.01\\
17.1	0.01\\
17.11	0.01\\
17.12	0.01\\
17.13	0.01\\
17.14	0.01\\
17.15	0.01\\
17.16	0.01\\
17.17	0.01\\
17.18	0.01\\
17.19	0.01\\
17.2	0.01\\
17.21	0.01\\
17.22	0.01\\
17.23	0.01\\
17.24	0.01\\
17.25	0.01\\
17.26	0.01\\
17.27	0.01\\
17.28	0.01\\
17.29	0.01\\
17.3	0.01\\
17.31	0.01\\
17.32	0.01\\
17.33	0.01\\
17.34	0.01\\
17.35	0.01\\
17.36	0.01\\
17.37	0.01\\
17.38	0.01\\
17.39	0.01\\
17.4	0.01\\
17.41	0.01\\
17.42	0.01\\
17.43	0.01\\
17.44	0.01\\
17.45	0.01\\
17.46	0.01\\
17.47	0.01\\
17.48	0.01\\
17.49	0.01\\
17.5	0.01\\
17.51	0.01\\
17.52	0.01\\
17.53	0.01\\
17.54	0.01\\
17.55	0.01\\
17.56	0.01\\
17.57	0.01\\
17.58	0.01\\
17.59	0.01\\
17.6	0.01\\
17.61	0.01\\
17.62	0.01\\
17.63	0.01\\
17.64	0.01\\
17.65	0.01\\
17.66	0.01\\
17.67	0.01\\
17.68	0.01\\
17.69	0.01\\
17.7	0.01\\
17.71	0.01\\
17.72	0.01\\
17.73	0.01\\
17.74	0.01\\
17.75	0.01\\
17.76	0.01\\
17.77	0.01\\
17.78	0.01\\
17.79	0.01\\
17.8	0.01\\
17.81	0.01\\
17.82	0.01\\
17.83	0.01\\
17.84	0.01\\
17.85	0.01\\
17.86	0.01\\
17.87	0.01\\
17.88	0.01\\
17.89	0.01\\
17.9	0.01\\
17.91	0.01\\
17.92	0.01\\
17.93	0.01\\
17.94	0.01\\
17.95	0.01\\
17.96	0.01\\
17.97	0.01\\
17.98	0.01\\
17.99	0.01\\
18	0.01\\
18.01	0.01\\
18.02	0.01\\
18.03	0.01\\
18.04	0.01\\
18.05	0.01\\
18.06	0.01\\
18.07	0.01\\
18.08	0.01\\
18.09	0.01\\
18.1	0.01\\
18.11	0.01\\
18.12	0.01\\
18.13	0.01\\
18.14	0.01\\
18.15	0.01\\
18.16	0.01\\
18.17	0.01\\
18.18	0.01\\
18.19	0.01\\
18.2	0.01\\
18.21	0.01\\
18.22	0.01\\
18.23	0.01\\
18.24	0.01\\
18.25	0.01\\
18.26	0.01\\
18.27	0.01\\
18.28	0.01\\
18.29	0.01\\
18.3	0.01\\
18.31	0.01\\
18.32	0.01\\
18.33	0.01\\
18.34	0.01\\
18.35	0.01\\
18.36	0.01\\
18.37	0.01\\
18.38	0.01\\
18.39	0.01\\
18.4	0.01\\
18.41	0.01\\
18.42	0.01\\
18.43	0.01\\
18.44	0.01\\
18.45	0.01\\
18.46	0.01\\
18.47	0.01\\
18.48	0.01\\
18.49	0.01\\
18.5	0.01\\
18.51	0.01\\
18.52	0.01\\
18.53	0.01\\
18.54	0.01\\
18.55	0.01\\
18.56	0.01\\
18.57	0.01\\
18.58	0.01\\
18.59	0.01\\
18.6	0.01\\
18.61	0.01\\
18.62	0.01\\
18.63	0.01\\
18.64	0.01\\
18.65	0.01\\
18.66	0.01\\
18.67	0.01\\
18.68	0.01\\
18.69	0.01\\
18.7	0.01\\
18.71	0.01\\
18.72	0.01\\
18.73	0.01\\
18.74	0.01\\
18.75	0.01\\
18.76	0.01\\
18.77	0.01\\
18.78	0.01\\
18.79	0.01\\
18.8	0.01\\
18.81	0.01\\
18.82	0.01\\
18.83	0.01\\
18.84	0.01\\
18.85	0.01\\
18.86	0.01\\
18.87	0.01\\
18.88	0.01\\
18.89	0.01\\
18.9	0.01\\
18.91	0.01\\
18.92	0.01\\
18.93	0.01\\
18.94	0.01\\
18.95	0.01\\
18.96	0.01\\
18.97	0.01\\
18.98	0.01\\
18.99	0.01\\
19	0.01\\
19.01	0.01\\
19.02	0.01\\
19.03	0.01\\
19.04	0.01\\
19.05	0.01\\
19.06	0.01\\
19.07	0.01\\
19.08	0.01\\
19.09	0.01\\
19.1	0.01\\
19.11	0.01\\
19.12	0.01\\
19.13	0.01\\
19.14	0.01\\
19.15	0.01\\
19.16	0.01\\
19.17	0.01\\
19.18	0.01\\
19.19	0.01\\
19.2	0.01\\
19.21	0.01\\
19.22	0.01\\
19.23	0.01\\
19.24	0.01\\
19.25	0.01\\
19.26	0.01\\
19.27	0.01\\
19.28	0.01\\
19.29	0.01\\
19.3	0.01\\
19.31	0.01\\
19.32	0.01\\
19.33	0.01\\
19.34	0.01\\
19.35	0.01\\
19.36	0.01\\
19.37	0.01\\
19.38	0.01\\
19.39	0.01\\
19.4	0.01\\
19.41	0.01\\
19.42	0.01\\
19.43	0.01\\
19.44	0.01\\
19.45	0.01\\
19.46	0.01\\
19.47	0.01\\
19.48	0.01\\
19.49	0.01\\
19.5	0.01\\
19.51	0.01\\
19.52	0.01\\
19.53	0.01\\
19.54	0.01\\
19.55	0.01\\
19.56	0.01\\
19.57	0.01\\
19.58	0.01\\
19.59	0.01\\
19.6	0.01\\
19.61	0.01\\
19.62	0.01\\
19.63	0.01\\
19.64	0.01\\
19.65	0.01\\
19.66	0.01\\
19.67	0.01\\
19.68	0.01\\
19.69	0.01\\
19.7	0.01\\
19.71	0.01\\
19.72	0.01\\
19.73	0.01\\
19.74	0.01\\
19.75	0.01\\
19.76	0.01\\
19.77	0.01\\
19.78	0.01\\
19.79	0.01\\
19.8	0.01\\
19.81	0.01\\
19.82	0.01\\
19.83	0.01\\
19.84	0.01\\
19.85	0.01\\
19.86	0.01\\
19.87	0.01\\
19.88	0.01\\
19.89	0.01\\
19.9	0.01\\
19.91	0.01\\
19.92	0.01\\
19.93	0.01\\
19.94	0.01\\
19.95	0.01\\
19.96	0.01\\
19.97	0.01\\
19.98	0.01\\
19.99	0.01\\
20	0.01\\
20.01	0.01\\
20.02	0.01\\
20.03	0.01\\
20.04	0.01\\
20.05	0.01\\
20.06	0.01\\
20.07	0.01\\
20.08	0.01\\
20.09	0.01\\
20.1	0.01\\
20.11	0.01\\
20.12	0.01\\
20.13	0.01\\
20.14	0.01\\
20.15	0.01\\
20.16	0.01\\
20.17	0.01\\
20.18	0.01\\
20.19	0.01\\
20.2	0.01\\
20.21	0.01\\
20.22	0.01\\
20.23	0.01\\
20.24	0.01\\
20.25	0.01\\
20.26	0.01\\
20.27	0.01\\
20.28	0.01\\
20.29	0.01\\
20.3	0.01\\
20.31	0.01\\
20.32	0.01\\
20.33	0.01\\
20.34	0.01\\
20.35	0.01\\
20.36	0.01\\
20.37	0.01\\
20.38	0.01\\
20.39	0.01\\
20.4	0.01\\
20.41	0.01\\
20.42	0.01\\
20.43	0.01\\
20.44	0.01\\
20.45	0.01\\
20.46	0.01\\
20.47	0.01\\
20.48	0.01\\
20.49	0.01\\
20.5	0.01\\
20.51	0.01\\
20.52	0.01\\
20.53	0.01\\
20.54	0.01\\
20.55	0.01\\
20.56	0.01\\
20.57	0.01\\
20.58	0.01\\
20.59	0.01\\
20.6	0.01\\
20.61	0.01\\
20.62	0.01\\
20.63	0.01\\
20.64	0.01\\
20.65	0.01\\
20.66	0.01\\
20.67	0.01\\
20.68	0.01\\
20.69	0.01\\
20.7	0.01\\
20.71	0.01\\
20.72	0.01\\
20.73	0.01\\
20.74	0.01\\
20.75	0.01\\
20.76	0.01\\
20.77	0.01\\
20.78	0.01\\
20.79	0.01\\
20.8	0.01\\
20.81	0.01\\
20.82	0.01\\
20.83	0.01\\
20.84	0.01\\
20.85	0.01\\
20.86	0.01\\
20.87	0.01\\
20.88	0.01\\
20.89	0.01\\
20.9	0.01\\
20.91	0.01\\
20.92	0.01\\
20.93	0.01\\
20.94	0.01\\
20.95	0.01\\
20.96	0.01\\
20.97	0.01\\
20.98	0.01\\
20.99	0.01\\
21	0.01\\
21.01	0.01\\
21.02	0.01\\
21.03	0.01\\
21.04	0.01\\
21.05	0.01\\
21.06	0.01\\
21.07	0.01\\
21.08	0.01\\
21.09	0.01\\
21.1	0.01\\
21.11	0.01\\
21.12	0.01\\
21.13	0.01\\
21.14	0.01\\
21.15	0.01\\
21.16	0.01\\
21.17	0.01\\
21.18	0.01\\
21.19	0.01\\
21.2	0.01\\
21.21	0.01\\
21.22	0.01\\
21.23	0.01\\
21.24	0.01\\
21.25	0.01\\
21.26	0.01\\
21.27	0.01\\
21.28	0.01\\
21.29	0.01\\
21.3	0.01\\
21.31	0.01\\
21.32	0.01\\
21.33	0.01\\
21.34	0.01\\
21.35	0.01\\
21.36	0.01\\
21.37	0.01\\
21.38	0.01\\
21.39	0.01\\
21.4	0.01\\
21.41	0.01\\
21.42	0.01\\
21.43	0.01\\
21.44	0.01\\
21.45	0.01\\
21.46	0.01\\
21.47	0.01\\
21.48	0.01\\
21.49	0.01\\
21.5	0.01\\
21.51	0.01\\
21.52	0.01\\
21.53	0.01\\
21.54	0.01\\
21.55	0.01\\
21.56	0.01\\
21.57	0.01\\
21.58	0.01\\
21.59	0.01\\
21.6	0.01\\
21.61	0.01\\
21.62	0.01\\
21.63	0.01\\
21.64	0.01\\
21.65	0.01\\
21.66	0.01\\
21.67	0.01\\
21.68	0.01\\
21.69	0.01\\
21.7	0.01\\
21.71	0.01\\
21.72	0.01\\
21.73	0.01\\
21.74	0.01\\
21.75	0.01\\
21.76	0.01\\
21.77	0.01\\
21.78	0.01\\
21.79	0.01\\
21.8	0.01\\
21.81	0.01\\
21.82	0.01\\
21.83	0.01\\
21.84	0.01\\
21.85	0.01\\
21.86	0.01\\
21.87	0.01\\
21.88	0.01\\
21.89	0.01\\
21.9	0.01\\
21.91	0.01\\
21.92	0.01\\
21.93	0.01\\
21.94	0.01\\
21.95	0.01\\
21.96	0.01\\
21.97	0.01\\
21.98	0.01\\
21.99	0.01\\
22	0.01\\
22.01	0.01\\
22.02	0.01\\
22.03	0.01\\
22.04	0.01\\
22.05	0.01\\
22.06	0.01\\
22.07	0.01\\
22.08	0.01\\
22.09	0.01\\
22.1	0.01\\
22.11	0.01\\
22.12	0.01\\
22.13	0.01\\
22.14	0.01\\
22.15	0.01\\
22.16	0.01\\
22.17	0.01\\
22.18	0.01\\
22.19	0.01\\
22.2	0.01\\
22.21	0.01\\
22.22	0.01\\
22.23	0.01\\
22.24	0.01\\
22.25	0.01\\
22.26	0.01\\
22.27	0.01\\
22.28	0.01\\
22.29	0.01\\
22.3	0.01\\
22.31	0.01\\
22.32	0.01\\
22.33	0.01\\
22.34	0.01\\
22.35	0.01\\
22.36	0.01\\
22.37	0.01\\
22.38	0.01\\
22.39	0.01\\
22.4	0.01\\
22.41	0.01\\
22.42	0.01\\
22.43	0.01\\
22.44	0.01\\
22.45	0.01\\
22.46	0.01\\
22.47	0.01\\
22.48	0.01\\
22.49	0.01\\
22.5	0.01\\
22.51	0.01\\
22.52	0.01\\
22.53	0.01\\
22.54	0.01\\
22.55	0.01\\
22.56	0.01\\
22.57	0.01\\
22.58	0.01\\
22.59	0.01\\
22.6	0.01\\
22.61	0.01\\
22.62	0.01\\
22.63	0.01\\
22.64	0.01\\
22.65	0.01\\
22.66	0.01\\
22.67	0.01\\
22.68	0.01\\
22.69	0.01\\
22.7	0.01\\
22.71	0.01\\
22.72	0.01\\
22.73	0.01\\
22.74	0.01\\
22.75	0.01\\
22.76	0.01\\
22.77	0.01\\
22.78	0.01\\
22.79	0.01\\
22.8	0.01\\
22.81	0.01\\
22.82	0.01\\
22.83	0.01\\
22.84	0.01\\
22.85	0.01\\
22.86	0.01\\
22.87	0.01\\
22.88	0.01\\
22.89	0.01\\
22.9	0.01\\
22.91	0.01\\
22.92	0.01\\
22.93	0.01\\
22.94	0.01\\
22.95	0.01\\
22.96	0.01\\
22.97	0.01\\
22.98	0.01\\
22.99	0.01\\
23	0.01\\
23.01	0.01\\
23.02	0.01\\
23.03	0.01\\
23.04	0.01\\
23.05	0.01\\
23.06	0.01\\
23.07	0.01\\
23.08	0.01\\
23.09	0.01\\
23.1	0.01\\
23.11	0.01\\
23.12	0.01\\
23.13	0.01\\
23.14	0.01\\
23.15	0.01\\
23.16	0.01\\
23.17	0.01\\
23.18	0.01\\
23.19	0.01\\
23.2	0.01\\
23.21	0.01\\
23.22	0.01\\
23.23	0.01\\
23.24	0.01\\
23.25	0.01\\
23.26	0.01\\
23.27	0.01\\
23.28	0.01\\
23.29	0.01\\
23.3	0.01\\
23.31	0.01\\
23.32	0.01\\
23.33	0.01\\
23.34	0.01\\
23.35	0.01\\
23.36	0.01\\
23.37	0.01\\
23.38	0.01\\
23.39	0.01\\
23.4	0.01\\
23.41	0.01\\
23.42	0.01\\
23.43	0.01\\
23.44	0.01\\
23.45	0.01\\
23.46	0.01\\
23.47	0.01\\
23.48	0.01\\
23.49	0.01\\
23.5	0.01\\
23.51	0.01\\
23.52	0.01\\
23.53	0.01\\
23.54	0.01\\
23.55	0.01\\
23.56	0.01\\
23.57	0.01\\
23.58	0.01\\
23.59	0.01\\
23.6	0.01\\
23.61	0.01\\
23.62	0.01\\
23.63	0.01\\
23.64	0.01\\
23.65	0.01\\
23.66	0.01\\
23.67	0.01\\
23.68	0.01\\
23.69	0.01\\
23.7	0.01\\
23.71	0.01\\
23.72	0.01\\
23.73	0.01\\
23.74	0.01\\
23.75	0.01\\
23.76	0.01\\
23.77	0.01\\
23.78	0.01\\
23.79	0.01\\
23.8	0.01\\
23.81	0.01\\
23.82	0.01\\
23.83	0.01\\
23.84	0.01\\
23.85	0.01\\
23.86	0.01\\
23.87	0.01\\
23.88	0.01\\
23.89	0.01\\
23.9	0.01\\
23.91	0.01\\
23.92	0.01\\
23.93	0.01\\
23.94	0.01\\
23.95	0.01\\
23.96	0.01\\
23.97	0.01\\
23.98	0.01\\
23.99	0.01\\
24	0.01\\
24.01	0.01\\
24.02	0.01\\
24.03	0.01\\
24.04	0.01\\
24.05	0.01\\
24.06	0.01\\
24.07	0.01\\
24.08	0.01\\
24.09	0.01\\
24.1	0.01\\
24.11	0.01\\
24.12	0.01\\
24.13	0.01\\
24.14	0.01\\
24.15	0.01\\
24.16	0.01\\
24.17	0.01\\
24.18	0.01\\
24.19	0.01\\
24.2	0.01\\
24.21	0.01\\
24.22	0.01\\
24.23	0.01\\
24.24	0.01\\
24.25	0.01\\
24.26	0.01\\
24.27	0.01\\
24.28	0.01\\
24.29	0.01\\
24.3	0.01\\
24.31	0.01\\
24.32	0.01\\
24.33	0.01\\
24.34	0.01\\
24.35	0.01\\
24.36	0.01\\
24.37	0.01\\
24.38	0.01\\
24.39	0.01\\
24.4	0.01\\
24.41	0.01\\
24.42	0.01\\
24.43	0.01\\
24.44	0.01\\
24.45	0.01\\
24.46	0.01\\
24.47	0.01\\
24.48	0.01\\
24.49	0.01\\
24.5	0.01\\
24.51	0.01\\
24.52	0.01\\
24.53	0.01\\
24.54	0.01\\
24.55	0.01\\
24.56	0.01\\
24.57	0.01\\
24.58	0.01\\
24.59	0.01\\
24.6	0.01\\
24.61	0.01\\
24.62	0.01\\
24.63	0.01\\
24.64	0.01\\
24.65	0.01\\
24.66	0.01\\
24.67	0.01\\
24.68	0.01\\
24.69	0.01\\
24.7	0.01\\
24.71	0.01\\
24.72	0.01\\
24.73	0.01\\
24.74	0.01\\
24.75	0.01\\
24.76	0.01\\
24.77	0.01\\
24.78	0.01\\
24.79	0.01\\
24.8	0.01\\
24.81	0.01\\
24.82	0.01\\
24.83	0.01\\
24.84	0.01\\
24.85	0.01\\
24.86	0.01\\
24.87	0.01\\
24.88	0.01\\
24.89	0.01\\
24.9	0.01\\
24.91	0.01\\
24.92	0.01\\
24.93	0.01\\
24.94	0.01\\
24.95	0.01\\
24.96	0.01\\
24.97	0.01\\
24.98	0.01\\
24.99	0.01\\
25	0.01\\
25.01	0.01\\
25.02	0.01\\
25.03	0.01\\
25.04	0.01\\
25.05	0.01\\
25.06	0.01\\
25.07	0.01\\
25.08	0.01\\
25.09	0.01\\
25.1	0.01\\
25.11	0.01\\
25.12	0.01\\
25.13	0.01\\
25.14	0.01\\
25.15	0.01\\
25.16	0.01\\
25.17	0.01\\
25.18	0.01\\
25.19	0.01\\
25.2	0.01\\
25.21	0.01\\
25.22	0.01\\
25.23	0.01\\
25.24	0.01\\
25.25	0.01\\
25.26	0.01\\
25.27	0.01\\
25.28	0.01\\
25.29	0.01\\
25.3	0.01\\
25.31	0.01\\
25.32	0.01\\
25.33	0.01\\
25.34	0.01\\
25.35	0.01\\
25.36	0.01\\
25.37	0.01\\
25.38	0.01\\
25.39	0.01\\
25.4	0.01\\
25.41	0.01\\
25.42	0.01\\
25.43	0.01\\
25.44	0.01\\
25.45	0.01\\
25.46	0.01\\
25.47	0.01\\
25.48	0.01\\
25.49	0.01\\
25.5	0.01\\
25.51	0.01\\
25.52	0.01\\
25.53	0.01\\
25.54	0.01\\
25.55	0.01\\
25.56	0.01\\
25.57	0.01\\
25.58	0.01\\
25.59	0.01\\
25.6	0.01\\
25.61	0.01\\
25.62	0.01\\
25.63	0.01\\
25.64	0.01\\
25.65	0.01\\
25.66	0.01\\
25.67	0.01\\
25.68	0.01\\
25.69	0.01\\
25.7	0.01\\
25.71	0.01\\
25.72	0.01\\
25.73	0.01\\
25.74	0.01\\
25.75	0.01\\
25.76	0.01\\
25.77	0.01\\
25.78	0.01\\
25.79	0.01\\
25.8	0.01\\
25.81	0.01\\
25.82	0.01\\
25.83	0.01\\
25.84	0.01\\
25.85	0.01\\
25.86	0.01\\
25.87	0.01\\
25.88	0.01\\
25.89	0.01\\
25.9	0.01\\
25.91	0.01\\
25.92	0.01\\
25.93	0.01\\
25.94	0.01\\
25.95	0.01\\
25.96	0.01\\
25.97	0.01\\
25.98	0.01\\
25.99	0.01\\
26	0.01\\
26.01	0.01\\
26.02	0.01\\
26.03	0.01\\
26.04	0.01\\
26.05	0.01\\
26.06	0.01\\
26.07	0.01\\
26.08	0.01\\
26.09	0.01\\
26.1	0.01\\
26.11	0.01\\
26.12	0.01\\
26.13	0.01\\
26.14	0.01\\
26.15	0.01\\
26.16	0.01\\
26.17	0.01\\
26.18	0.01\\
26.19	0.01\\
26.2	0.01\\
26.21	0.01\\
26.22	0.01\\
26.23	0.01\\
26.24	0.01\\
26.25	0.01\\
26.26	0.01\\
26.27	0.01\\
26.28	0.01\\
26.29	0.01\\
26.3	0.01\\
26.31	0.01\\
26.32	0.01\\
26.33	0.01\\
26.34	0.01\\
26.35	0.01\\
26.36	0.01\\
26.37	0.01\\
26.38	0.01\\
26.39	0.01\\
26.4	0.01\\
26.41	0.01\\
26.42	0.01\\
26.43	0.01\\
26.44	0.01\\
26.45	0.01\\
26.46	0.01\\
26.47	0.01\\
26.48	0.01\\
26.49	0.01\\
26.5	0.01\\
26.51	0.01\\
26.52	0.01\\
26.53	0.01\\
26.54	0.01\\
26.55	0.01\\
26.56	0.01\\
26.57	0.01\\
26.58	0.01\\
26.59	0.01\\
26.6	0.01\\
26.61	0.01\\
26.62	0.01\\
26.63	0.01\\
26.64	0.01\\
26.65	0.01\\
26.66	0.01\\
26.67	0.01\\
26.68	0.01\\
26.69	0.01\\
26.7	0.01\\
26.71	0.01\\
26.72	0.01\\
26.73	0.01\\
26.74	0.01\\
26.75	0.01\\
26.76	0.01\\
26.77	0.01\\
26.78	0.01\\
26.79	0.01\\
26.8	0.01\\
26.81	0.01\\
26.82	0.01\\
26.83	0.01\\
26.84	0.01\\
26.85	0.01\\
26.86	0.01\\
26.87	0.01\\
26.88	0.01\\
26.89	0.01\\
26.9	0.01\\
26.91	0.01\\
26.92	0.01\\
26.93	0.01\\
26.94	0.01\\
26.95	0.01\\
26.96	0.01\\
26.97	0.01\\
26.98	0.01\\
26.99	0.01\\
27	0.01\\
27.01	0.01\\
27.02	0.01\\
27.03	0.01\\
27.04	0.01\\
27.05	0.01\\
27.06	0.01\\
27.07	0.01\\
27.08	0.01\\
27.09	0.01\\
27.1	0.01\\
27.11	0.01\\
27.12	0.01\\
27.13	0.01\\
27.14	0.01\\
27.15	0.01\\
27.16	0.01\\
27.17	0.01\\
27.18	0.01\\
27.19	0.01\\
27.2	0.01\\
27.21	0.01\\
27.22	0.01\\
27.23	0.01\\
27.24	0.01\\
27.25	0.01\\
27.26	0.01\\
27.27	0.01\\
27.28	0.01\\
27.29	0.01\\
27.3	0.01\\
27.31	0.01\\
27.32	0.01\\
27.33	0.01\\
27.34	0.01\\
27.35	0.01\\
27.36	0.01\\
27.37	0.01\\
27.38	0.01\\
27.39	0.01\\
27.4	0.01\\
27.41	0.01\\
27.42	0.01\\
27.43	0.01\\
27.44	0.01\\
27.45	0.01\\
27.46	0.01\\
27.47	0.01\\
27.48	0.01\\
27.49	0.01\\
27.5	0.01\\
27.51	0.01\\
27.52	0.01\\
27.53	0.01\\
27.54	0.01\\
27.55	0.01\\
27.56	0.01\\
27.57	0.01\\
27.58	0.01\\
27.59	0.01\\
27.6	0.01\\
27.61	0.01\\
27.62	0.01\\
27.63	0.01\\
27.64	0.01\\
27.65	0.01\\
27.66	0.01\\
27.67	0.01\\
27.68	0.01\\
27.69	0.01\\
27.7	0.01\\
27.71	0.01\\
27.72	0.01\\
27.73	0.01\\
27.74	0.01\\
27.75	0.01\\
27.76	0.01\\
27.77	0.01\\
27.78	0.01\\
27.79	0.01\\
27.8	0.01\\
27.81	0.01\\
27.82	0.01\\
27.83	0.01\\
27.84	0.01\\
27.85	0.01\\
27.86	0.01\\
27.87	0.01\\
27.88	0.01\\
27.89	0.01\\
27.9	0.01\\
27.91	0.01\\
27.92	0.01\\
27.93	0.01\\
27.94	0.01\\
27.95	0.01\\
27.96	0.01\\
27.97	0.01\\
27.98	0.01\\
27.99	0.01\\
28	0.01\\
28.01	0.01\\
28.02	0.01\\
28.03	0.01\\
28.04	0.01\\
28.05	0.01\\
28.06	0.01\\
28.07	0.01\\
28.08	0.01\\
28.09	0.01\\
28.1	0.01\\
28.11	0.01\\
28.12	0.01\\
28.13	0.01\\
28.14	0.01\\
28.15	0.01\\
28.16	0.01\\
28.17	0.01\\
28.18	0.01\\
28.19	0.01\\
28.2	0.01\\
28.21	0.01\\
28.22	0.01\\
28.23	0.01\\
28.24	0.01\\
28.25	0.01\\
28.26	0.01\\
28.27	0.01\\
28.28	0.01\\
28.29	0.01\\
28.3	0.01\\
28.31	0.01\\
28.32	0.01\\
28.33	0.01\\
28.34	0.01\\
28.35	0.01\\
28.36	0.01\\
28.37	0.01\\
28.38	0.01\\
28.39	0.01\\
28.4	0.01\\
28.41	0.01\\
28.42	0.01\\
28.43	0.01\\
28.44	0.01\\
28.45	0.01\\
28.46	0.01\\
28.47	0.01\\
28.48	0.01\\
28.49	0.01\\
28.5	0.01\\
28.51	0.01\\
28.52	0.01\\
28.53	0.01\\
28.54	0.01\\
28.55	0.01\\
28.56	0.01\\
28.57	0.01\\
28.58	0.01\\
28.59	0.01\\
28.6	0.01\\
28.61	0.01\\
28.62	0.01\\
28.63	0.01\\
28.64	0.01\\
28.65	0.01\\
28.66	0.01\\
28.67	0.01\\
28.68	0.01\\
28.69	0.01\\
28.7	0.01\\
28.71	0.01\\
28.72	0.01\\
28.73	0.01\\
28.74	0.01\\
28.75	0.01\\
28.76	0.01\\
28.77	0.01\\
28.78	0.01\\
28.79	0.01\\
28.8	0.01\\
28.81	0.01\\
28.82	0.01\\
28.83	0.01\\
28.84	0.01\\
28.85	0.01\\
28.86	0.01\\
28.87	0.01\\
28.88	0.01\\
28.89	0.01\\
28.9	0.01\\
28.91	0.01\\
28.92	0.01\\
28.93	0.01\\
28.94	0.01\\
28.95	0.01\\
28.96	0.01\\
28.97	0.01\\
28.98	0.01\\
28.99	0.01\\
29	0.01\\
29.01	0.01\\
29.02	0.01\\
29.03	0.01\\
29.04	0.01\\
29.05	0.01\\
29.06	0.01\\
29.07	0.01\\
29.08	0.01\\
29.09	0.01\\
29.1	0.01\\
29.11	0.01\\
29.12	0.01\\
29.13	0.01\\
29.14	0.01\\
29.15	0.01\\
29.16	0.01\\
29.17	0.01\\
29.18	0.01\\
29.19	0.01\\
29.2	0.01\\
29.21	0.01\\
29.22	0.01\\
29.23	0.01\\
29.24	0.01\\
29.25	0.01\\
29.26	0.01\\
29.27	0.01\\
29.28	0.01\\
29.29	0.01\\
29.3	0.01\\
29.31	0.01\\
29.32	0.01\\
29.33	0.01\\
29.34	0.01\\
29.35	0.01\\
29.36	0.01\\
29.37	0.01\\
29.38	0.01\\
29.39	0.01\\
29.4	0.01\\
29.41	0.01\\
29.42	0.01\\
29.43	0.01\\
29.44	0.01\\
29.45	0.01\\
29.46	0.01\\
29.47	0.01\\
29.48	0.01\\
29.49	0.01\\
29.5	0.01\\
29.51	0.01\\
29.52	0.01\\
29.53	0.01\\
29.54	0.01\\
29.55	0.01\\
29.56	0.01\\
29.57	0.01\\
29.58	0.01\\
29.59	0.01\\
29.6	0.01\\
29.61	0.01\\
29.62	0.01\\
29.63	0.01\\
29.64	0.01\\
29.65	0.01\\
29.66	0.01\\
29.67	0.01\\
29.68	0.01\\
29.69	0.01\\
29.7	0.01\\
29.71	0.01\\
29.72	0.01\\
29.73	0.01\\
29.74	0.01\\
29.75	0.01\\
29.76	0.01\\
29.77	0.01\\
29.78	0.01\\
29.79	0.01\\
29.8	0.01\\
29.81	0.01\\
29.82	0.01\\
29.83	0.01\\
29.84	0.01\\
29.85	0.01\\
29.86	0.01\\
29.87	0.01\\
29.88	0.01\\
29.89	0.01\\
29.9	0.01\\
29.91	0.01\\
29.92	0.01\\
29.93	0.01\\
29.94	0.01\\
29.95	0.01\\
29.96	0.01\\
29.97	0.01\\
29.98	0.01\\
29.99	0.01\\
30	0.01\\
30.01	0.01\\
30.02	0.01\\
30.03	0.01\\
30.04	0.01\\
30.05	0.01\\
30.06	0.01\\
30.07	0.01\\
30.08	0.01\\
30.09	0.01\\
30.1	0.01\\
30.11	0.01\\
30.12	0.01\\
30.13	0.01\\
30.14	0.01\\
30.15	0.01\\
30.16	0.01\\
30.17	0.01\\
30.18	0.01\\
30.19	0.01\\
30.2	0.01\\
30.21	0.01\\
30.22	0.01\\
30.23	0.01\\
30.24	0.01\\
30.25	0.01\\
30.26	0.01\\
30.27	0.01\\
30.28	0.01\\
30.29	0.01\\
30.3	0.01\\
30.31	0.01\\
30.32	0.01\\
30.33	0.01\\
30.34	0.01\\
30.35	0.01\\
30.36	0.01\\
30.37	0.01\\
30.38	0.01\\
30.39	0.01\\
30.4	0.01\\
30.41	0.01\\
30.42	0.01\\
30.43	0.01\\
30.44	0.01\\
30.45	0.01\\
30.46	0.01\\
30.47	0.01\\
30.48	0.01\\
30.49	0.01\\
30.5	0.01\\
30.51	0.01\\
30.52	0.01\\
30.53	0.01\\
30.54	0.01\\
30.55	0.01\\
30.56	0.01\\
30.57	0.01\\
30.58	0.01\\
30.59	0.01\\
30.6	0.01\\
30.61	0.01\\
30.62	0.01\\
30.63	0.01\\
30.64	0.01\\
30.65	0.01\\
30.66	0.01\\
30.67	0.01\\
30.68	0.01\\
30.69	0.01\\
30.7	0.01\\
30.71	0.01\\
30.72	0.01\\
30.73	0.01\\
30.74	0.01\\
30.75	0.01\\
30.76	0.01\\
30.77	0.01\\
30.78	0.01\\
30.79	0.01\\
30.8	0.01\\
30.81	0.01\\
30.82	0.01\\
30.83	0.01\\
30.84	0.01\\
30.85	0.01\\
30.86	0.01\\
30.87	0.01\\
30.88	0.01\\
30.89	0.01\\
30.9	0.01\\
30.91	0.01\\
30.92	0.01\\
30.93	0.01\\
30.94	0.01\\
30.95	0.01\\
30.96	0.01\\
30.97	0.01\\
30.98	0.01\\
30.99	0.01\\
31	0.01\\
31.01	0.01\\
31.02	0.01\\
31.03	0.01\\
31.04	0.01\\
31.05	0.01\\
31.06	0.01\\
31.07	0.01\\
31.08	0.01\\
31.09	0.01\\
31.1	0.01\\
31.11	0.01\\
31.12	0.01\\
31.13	0.01\\
31.14	0.01\\
31.15	0.01\\
31.16	0.01\\
31.17	0.01\\
31.18	0.01\\
31.19	0.01\\
31.2	0.01\\
31.21	0.01\\
31.22	0.01\\
31.23	0.01\\
31.24	0.01\\
31.25	0.01\\
31.26	0.01\\
31.27	0.01\\
31.28	0.01\\
31.29	0.01\\
31.3	0.01\\
31.31	0.01\\
31.32	0.01\\
31.33	0.01\\
31.34	0.01\\
31.35	0.01\\
31.36	0.01\\
31.37	0.01\\
31.38	0.01\\
31.39	0.01\\
31.4	0.01\\
31.41	0.01\\
31.42	0.01\\
31.43	0.01\\
31.44	0.01\\
31.45	0.01\\
31.46	0.01\\
31.47	0.01\\
31.48	0.01\\
31.49	0.01\\
31.5	0.01\\
31.51	0.01\\
31.52	0.01\\
31.53	0.01\\
31.54	0.01\\
31.55	0.01\\
31.56	0.01\\
31.57	0.01\\
31.58	0.01\\
31.59	0.01\\
31.6	0.01\\
31.61	0.01\\
31.62	0.01\\
31.63	0.01\\
31.64	0.01\\
31.65	0.01\\
31.66	0.01\\
31.67	0.01\\
31.68	0.01\\
31.69	0.01\\
31.7	0.01\\
31.71	0.01\\
31.72	0.01\\
31.73	0.01\\
31.74	0.01\\
31.75	0.01\\
31.76	0.01\\
31.77	0.01\\
31.78	0.01\\
31.79	0.01\\
31.8	0.01\\
31.81	0.01\\
31.82	0.01\\
31.83	0.01\\
31.84	0.01\\
31.85	0.01\\
31.86	0.01\\
31.87	0.01\\
31.88	0.01\\
31.89	0.01\\
31.9	0.01\\
31.91	0.01\\
31.92	0.01\\
31.93	0.01\\
31.94	0.01\\
31.95	0.01\\
31.96	0.01\\
31.97	0.01\\
31.98	0.01\\
31.99	0.01\\
32	0.01\\
32.01	0.01\\
32.02	0.01\\
32.03	0.01\\
32.04	0.01\\
32.05	0.01\\
32.06	0.01\\
32.07	0.01\\
32.08	0.01\\
32.09	0.01\\
32.1	0.01\\
32.11	0.01\\
32.12	0.01\\
32.13	0.01\\
32.14	0.01\\
32.15	0.01\\
32.16	0.01\\
32.17	0.01\\
32.18	0.01\\
32.19	0.01\\
32.2	0.01\\
32.21	0.01\\
32.22	0.01\\
32.23	0.01\\
32.24	0.01\\
32.25	0.01\\
32.26	0.01\\
32.27	0.01\\
32.28	0.01\\
32.29	0.01\\
32.3	0.01\\
32.31	0.01\\
32.32	0.01\\
32.33	0.01\\
32.34	0.01\\
32.35	0.01\\
32.36	0.01\\
32.37	0.01\\
32.38	0.01\\
32.39	0.01\\
32.4	0.01\\
32.41	0.01\\
32.42	0.01\\
32.43	0.01\\
32.44	0.01\\
32.45	0.01\\
32.46	0.01\\
32.47	0.01\\
32.48	0.01\\
32.49	0.01\\
32.5	0.01\\
32.51	0.01\\
32.52	0.01\\
32.53	0.01\\
32.54	0.01\\
32.55	0.01\\
32.56	0.01\\
32.57	0.01\\
32.58	0.01\\
32.59	0.01\\
32.6	0.01\\
32.61	0.01\\
32.62	0.01\\
32.63	0.01\\
32.64	0.01\\
32.65	0.01\\
32.66	0.01\\
32.67	0.01\\
32.68	0.01\\
32.69	0.01\\
32.7	0.01\\
32.71	0.01\\
32.72	0.01\\
32.73	0.01\\
32.74	0.01\\
32.75	0.01\\
32.76	0.01\\
32.77	0.01\\
32.78	0.01\\
32.79	0.01\\
32.8	0.01\\
32.81	0.01\\
32.82	0.01\\
32.83	0.01\\
32.84	0.01\\
32.85	0.01\\
32.86	0.01\\
32.87	0.01\\
32.88	0.01\\
32.89	0.01\\
32.9	0.01\\
32.91	0.01\\
32.92	0.01\\
32.93	0.01\\
32.94	0.01\\
32.95	0.01\\
32.96	0.01\\
32.97	0.01\\
32.98	0.01\\
32.99	0.01\\
33	0.01\\
33.01	0.01\\
33.02	0.01\\
33.03	0.01\\
33.04	0.01\\
33.05	0.01\\
33.06	0.01\\
33.07	0.01\\
33.08	0.01\\
33.09	0.01\\
33.1	0.01\\
33.11	0.01\\
33.12	0.01\\
33.13	0.01\\
33.14	0.01\\
33.15	0.01\\
33.16	0.01\\
33.17	0.01\\
33.18	0.01\\
33.19	0.01\\
33.2	0.01\\
33.21	0.01\\
33.22	0.01\\
33.23	0.01\\
33.24	0.01\\
33.25	0.01\\
33.26	0.01\\
33.27	0.01\\
33.28	0.01\\
33.29	0.01\\
33.3	0.01\\
33.31	0.01\\
33.32	0.01\\
33.33	0.01\\
33.34	0.01\\
33.35	0.01\\
33.36	0.01\\
33.37	0.01\\
33.38	0.01\\
33.39	0.01\\
33.4	0.01\\
33.41	0.01\\
33.42	0.01\\
33.43	0.01\\
33.44	0.01\\
33.45	0.01\\
33.46	0.01\\
33.47	0.01\\
33.48	0.01\\
33.49	0.01\\
33.5	0.01\\
33.51	0.01\\
33.52	0.01\\
33.53	0.01\\
33.54	0.01\\
33.55	0.01\\
33.56	0.01\\
33.57	0.01\\
33.58	0.01\\
33.59	0.01\\
33.6	0.01\\
33.61	0.01\\
33.62	0.01\\
33.63	0.01\\
33.64	0.01\\
33.65	0.01\\
33.66	0.01\\
33.67	0.01\\
33.68	0.01\\
33.69	0.01\\
33.7	0.01\\
33.71	0.01\\
33.72	0.01\\
33.73	0.01\\
33.74	0.01\\
33.75	0.01\\
33.76	0.01\\
33.77	0.01\\
33.78	0.01\\
33.79	0.01\\
33.8	0.01\\
33.81	0.01\\
33.82	0.01\\
33.83	0.01\\
33.84	0.01\\
33.85	0.01\\
33.86	0.01\\
33.87	0.01\\
33.88	0.01\\
33.89	0.01\\
33.9	0.01\\
33.91	0.01\\
33.92	0.01\\
33.93	0.01\\
33.94	0.01\\
33.95	0.01\\
33.96	0.01\\
33.97	0.01\\
33.98	0.01\\
33.99	0.01\\
34	0.01\\
34.01	0.01\\
34.02	0.01\\
34.03	0.01\\
34.04	0.01\\
34.05	0.01\\
34.06	0.01\\
34.07	0.01\\
34.08	0.01\\
34.09	0.01\\
34.1	0.01\\
34.11	0.01\\
34.12	0.01\\
34.13	0.01\\
34.14	0.01\\
34.15	0.01\\
34.16	0.01\\
34.17	0.01\\
34.18	0.01\\
34.19	0.01\\
34.2	0.01\\
34.21	0.01\\
34.22	0.01\\
34.23	0.01\\
34.24	0.01\\
34.25	0.01\\
34.26	0.01\\
34.27	0.01\\
34.28	0.01\\
34.29	0.01\\
34.3	0.01\\
34.31	0.01\\
34.32	0.01\\
34.33	0.01\\
34.34	0.01\\
34.35	0.01\\
34.36	0.01\\
34.37	0.01\\
34.38	0.01\\
34.39	0.01\\
34.4	0.01\\
34.41	0.01\\
34.42	0.01\\
34.43	0.01\\
34.44	0.01\\
34.45	0.01\\
34.46	0.01\\
34.47	0.01\\
34.48	0.01\\
34.49	0.01\\
34.5	0.01\\
34.51	0.01\\
34.52	0.01\\
34.53	0.01\\
34.54	0.01\\
34.55	0.01\\
34.56	0.01\\
34.57	0.01\\
34.58	0.01\\
34.59	0.01\\
34.6	0.01\\
34.61	0.01\\
34.62	0.01\\
34.63	0.01\\
34.64	0.01\\
34.65	0.01\\
34.66	0.01\\
34.67	0.01\\
34.68	0.01\\
34.69	0.01\\
34.7	0.01\\
34.71	0.01\\
34.72	0.01\\
34.73	0.01\\
34.74	0.01\\
34.75	0.01\\
34.76	0.01\\
34.77	0.01\\
34.78	0.01\\
34.79	0.01\\
34.8	0.01\\
34.81	0.01\\
34.82	0.01\\
34.83	0.01\\
34.84	0.01\\
34.85	0.01\\
34.86	0.01\\
34.87	0.01\\
34.88	0.01\\
34.89	0.01\\
34.9	0.01\\
34.91	0.01\\
34.92	0.01\\
34.93	0.01\\
34.94	0.01\\
34.95	0.01\\
34.96	0.01\\
34.97	0.01\\
34.98	0.01\\
34.99	0.01\\
35	0.01\\
35.01	0.01\\
35.02	0.01\\
35.03	0.01\\
35.04	0.01\\
35.05	0.01\\
35.06	0.01\\
35.07	0.01\\
35.08	0.01\\
35.09	0.01\\
35.1	0.01\\
35.11	0.01\\
35.12	0.01\\
35.13	0.01\\
35.14	0.01\\
35.15	0.01\\
35.16	0.01\\
35.17	0.01\\
35.18	0.01\\
35.19	0.01\\
35.2	0.01\\
35.21	0.01\\
35.22	0.01\\
35.23	0.01\\
35.24	0.01\\
35.25	0.01\\
35.26	0.01\\
35.27	0.01\\
35.28	0.01\\
35.29	0.01\\
35.3	0.01\\
35.31	0.01\\
35.32	0.01\\
35.33	0.01\\
35.34	0.01\\
35.35	0.01\\
35.36	0.01\\
35.37	0.01\\
35.38	0.01\\
35.39	0.01\\
35.4	0.01\\
35.41	0.01\\
35.42	0.01\\
35.43	0.01\\
35.44	0.01\\
35.45	0.01\\
35.46	0.01\\
35.47	0.01\\
35.48	0.01\\
35.49	0.01\\
35.5	0.01\\
35.51	0.01\\
35.52	0.01\\
35.53	0.01\\
35.54	0.01\\
35.55	0.01\\
35.56	0.01\\
35.57	0.01\\
35.58	0.01\\
35.59	0.01\\
35.6	0.01\\
35.61	0.01\\
35.62	0.01\\
35.63	0.01\\
35.64	0.01\\
35.65	0.01\\
35.66	0.01\\
35.67	0.01\\
35.68	0.01\\
35.69	0.01\\
35.7	0.01\\
35.71	0.01\\
35.72	0.01\\
35.73	0.01\\
35.74	0.01\\
35.75	0.01\\
35.76	0.01\\
35.77	0.01\\
35.78	0.01\\
35.79	0.01\\
35.8	0.01\\
35.81	0.01\\
35.82	0.01\\
35.83	0.01\\
35.84	0.01\\
35.85	0.01\\
35.86	0.01\\
35.87	0.01\\
35.88	0.01\\
35.89	0.01\\
35.9	0.01\\
35.91	0.01\\
35.92	0.01\\
35.93	0.01\\
35.94	0.01\\
35.95	0.01\\
35.96	0.01\\
35.97	0.01\\
35.98	0.01\\
35.99	0.01\\
36	0.01\\
36.01	0.01\\
36.02	0.01\\
36.03	0.01\\
36.04	0.01\\
36.05	0.01\\
36.06	0.01\\
36.07	0.01\\
36.08	0.01\\
36.09	0.01\\
36.1	0.01\\
36.11	0.01\\
36.12	0.01\\
36.13	0.01\\
36.14	0.01\\
36.15	0.01\\
36.16	0.01\\
36.17	0.01\\
36.18	0.01\\
36.19	0.01\\
36.2	0.01\\
36.21	0.01\\
36.22	0.01\\
36.23	0.01\\
36.24	0.01\\
36.25	0.01\\
36.26	0.01\\
36.27	0.01\\
36.28	0.01\\
36.29	0.01\\
36.3	0.01\\
36.31	0.01\\
36.32	0.01\\
36.33	0.01\\
36.34	0.01\\
36.35	0.01\\
36.36	0.01\\
36.37	0.01\\
36.38	0.01\\
36.39	0.01\\
36.4	0.01\\
36.41	0.01\\
36.42	0.01\\
36.43	0.01\\
36.44	0.01\\
36.45	0.01\\
36.46	0.01\\
36.47	0.01\\
36.48	0.01\\
36.49	0.01\\
36.5	0.01\\
36.51	0.01\\
36.52	0.01\\
36.53	0.01\\
36.54	0.01\\
36.55	0.01\\
36.56	0.01\\
36.57	0.01\\
36.58	0.01\\
36.59	0.01\\
36.6	0.01\\
36.61	0.01\\
36.62	0.01\\
36.63	0.01\\
36.64	0.01\\
36.65	0.01\\
36.66	0.01\\
36.67	0.01\\
36.68	0.01\\
36.69	0.01\\
36.7	0.01\\
36.71	0.01\\
36.72	0.01\\
36.73	0.01\\
36.74	0.01\\
36.75	0.01\\
36.76	0.01\\
36.77	0.01\\
36.78	0.01\\
36.79	0.01\\
36.8	0.01\\
36.81	0.01\\
36.82	0.01\\
36.83	0.01\\
36.84	0.01\\
36.85	0.01\\
36.86	0.01\\
36.87	0.01\\
36.88	0.01\\
36.89	0.01\\
36.9	0.01\\
36.91	0.01\\
36.92	0.01\\
36.93	0.01\\
36.94	0.01\\
36.95	0.01\\
36.96	0.01\\
36.97	0.01\\
36.98	0.01\\
36.99	0.01\\
37	0.01\\
37.01	0.01\\
37.02	0.01\\
37.03	0.01\\
37.04	0.01\\
37.05	0.01\\
37.06	0.01\\
37.07	0.01\\
37.08	0.01\\
37.09	0.01\\
37.1	0.01\\
37.11	0.01\\
37.12	0.01\\
37.13	0.01\\
37.14	0.01\\
37.15	0.01\\
37.16	0.01\\
37.17	0.01\\
37.18	0.01\\
37.19	0.01\\
37.2	0.01\\
37.21	0.01\\
37.22	0.01\\
37.23	0.01\\
37.24	0.01\\
37.25	0.01\\
37.26	0.01\\
37.27	0.01\\
37.28	0.01\\
37.29	0.01\\
37.3	0.01\\
37.31	0.01\\
37.32	0.01\\
37.33	0.01\\
37.34	0.01\\
37.35	0.01\\
37.36	0.01\\
37.37	0.01\\
37.38	0.01\\
37.39	0.01\\
37.4	0.01\\
37.41	0.01\\
37.42	0.01\\
37.43	0.01\\
37.44	0.01\\
37.45	0.01\\
37.46	0.01\\
37.47	0.01\\
37.48	0.01\\
37.49	0.01\\
37.5	0.01\\
37.51	0.01\\
37.52	0.01\\
37.53	0.01\\
37.54	0.01\\
37.55	0.01\\
37.56	0.01\\
37.57	0.01\\
37.58	0.01\\
37.59	0.01\\
37.6	0.01\\
37.61	0.01\\
37.62	0.01\\
37.63	0.01\\
37.64	0.01\\
37.65	0.01\\
37.66	0.01\\
37.67	0.01\\
37.68	0.01\\
37.69	0.01\\
37.7	0.01\\
37.71	0.01\\
37.72	0.01\\
37.73	0.01\\
37.74	0.01\\
37.75	0.01\\
37.76	0.01\\
37.77	0.01\\
37.78	0.01\\
37.79	0.01\\
37.8	0.01\\
37.81	0.01\\
37.82	0.01\\
37.83	0.01\\
37.84	0.01\\
37.85	0.01\\
37.86	0.01\\
37.87	0.01\\
37.88	0.01\\
37.89	0.01\\
37.9	0.01\\
37.91	0.01\\
37.92	0.01\\
37.93	0.01\\
37.94	0.01\\
37.95	0.01\\
37.96	0.01\\
37.97	0.01\\
37.98	0.01\\
37.99	0.01\\
38	0.01\\
38.01	0.01\\
38.02	0.01\\
38.03	0.01\\
38.04	0.01\\
38.05	0.01\\
38.06	0.01\\
38.07	0.01\\
38.08	0.01\\
38.09	0.01\\
38.1	0.01\\
38.11	0.01\\
38.12	0.01\\
38.13	0.01\\
38.14	0.01\\
38.15	0.01\\
38.16	0.01\\
38.17	0.01\\
38.18	0.01\\
38.19	0.01\\
38.2	0.01\\
38.21	0.01\\
38.22	0.01\\
38.23	0.01\\
38.24	0.01\\
38.25	0.01\\
38.26	0.01\\
38.27	0.01\\
38.28	0.01\\
38.29	0.01\\
38.3	0.01\\
38.31	0.01\\
38.32	0.01\\
38.33	0.01\\
38.34	0.01\\
38.35	0.01\\
38.36	0.01\\
38.37	0.01\\
38.38	0.01\\
38.39	0.01\\
38.4	0.01\\
38.41	0.01\\
38.42	0.01\\
38.43	0.01\\
38.44	0.01\\
38.45	0.01\\
38.46	0.01\\
38.47	0.01\\
38.48	0.01\\
38.49	0.01\\
38.5	0.01\\
38.51	0.01\\
38.52	0.01\\
38.53	0.01\\
38.54	0.01\\
38.55	0.01\\
38.56	0.01\\
38.57	0.01\\
38.58	0.01\\
38.59	0.01\\
38.6	0.01\\
38.61	0.01\\
38.62	0.01\\
38.63	0.01\\
38.64	0.01\\
38.65	0.01\\
38.66	0.01\\
38.67	0.01\\
38.68	0.01\\
38.69	0.01\\
38.7	0.01\\
38.71	0.01\\
38.72	0.01\\
38.73	0.01\\
38.74	0.01\\
38.75	0.01\\
38.76	0.01\\
38.77	0.01\\
38.78	0.01\\
38.79	0.01\\
38.8	0.01\\
38.81	0.01\\
38.82	0.01\\
38.83	0.01\\
38.84	0.01\\
38.85	0.01\\
38.86	0.01\\
38.87	0.01\\
38.88	0.01\\
38.89	0.01\\
38.9	0.01\\
38.91	0.01\\
38.92	0.01\\
38.93	0.01\\
38.94	0.01\\
38.95	0.01\\
38.96	0.01\\
38.97	0.01\\
38.98	0.01\\
38.99	0.01\\
39	0.01\\
39.01	0.01\\
39.02	0.01\\
39.03	0.01\\
39.04	0.01\\
39.05	0.01\\
39.06	0.01\\
39.07	0.01\\
39.08	0.01\\
39.09	0.01\\
39.1	0.01\\
39.11	0.01\\
39.12	0.01\\
39.13	0.01\\
39.14	0.01\\
39.15	0.01\\
39.16	0.01\\
39.17	0.01\\
39.18	0.01\\
39.19	0.01\\
39.2	0.01\\
39.21	0.01\\
39.22	0.01\\
39.23	0.01\\
39.24	0.01\\
39.25	0.01\\
39.26	0.01\\
39.27	0.01\\
39.28	0.01\\
39.29	0.01\\
39.3	0.01\\
39.31	0.01\\
39.32	0.01\\
39.33	0.01\\
39.34	0.01\\
39.35	0.01\\
39.36	0.01\\
39.37	0.01\\
39.38	0.01\\
39.39	0.01\\
39.4	0.01\\
39.41	0.01\\
39.42	0.01\\
39.43	0.01\\
39.44	0.01\\
39.45	0.01\\
39.46	0.01\\
39.47	0.01\\
39.48	0.01\\
39.49	0.01\\
39.5	0.01\\
39.51	0.01\\
39.52	0.01\\
39.53	0.01\\
39.54	0.01\\
39.55	0.01\\
39.56	0.01\\
39.57	0.01\\
39.58	0.01\\
39.59	0.01\\
39.6	0.01\\
39.61	0.01\\
39.62	0.01\\
39.63	0.01\\
39.64	0.01\\
39.65	0.01\\
39.66	0.01\\
39.67	0.01\\
39.68	0.01\\
39.69	0.01\\
39.7	0.01\\
39.71	0.01\\
39.72	0.01\\
39.73	0.01\\
39.74	0.01\\
39.75	0.01\\
39.76	0.01\\
39.77	0.01\\
39.78	0.01\\
39.79	0.01\\
39.8	0.01\\
39.81	0.01\\
39.82	0.01\\
39.83	0.01\\
39.84	0.01\\
39.85	0.01\\
39.86	0.01\\
39.87	0.01\\
39.88	0.01\\
39.89	0.01\\
39.9	0.01\\
39.91	0.01\\
39.92	0.01\\
39.93	0.01\\
39.94	0.01\\
39.95	0.01\\
39.96	0.01\\
39.97	0.01\\
39.98	0.01\\
39.99	0.01\\
40	0.01\\
40.01	0.01\\
};
\addplot [color=green,solid,forget plot]
  table[row sep=crcr]{%
40.01	0.01\\
40.02	0.01\\
40.03	0.01\\
40.04	0.01\\
40.05	0.01\\
40.06	0.01\\
40.07	0.01\\
40.08	0.01\\
40.09	0.01\\
40.1	0.01\\
40.11	0.01\\
40.12	0.01\\
40.13	0.01\\
40.14	0.01\\
40.15	0.01\\
40.16	0.01\\
40.17	0.01\\
40.18	0.01\\
40.19	0.01\\
40.2	0.01\\
40.21	0.01\\
40.22	0.01\\
40.23	0.01\\
40.24	0.01\\
40.25	0.01\\
40.26	0.01\\
40.27	0.01\\
40.28	0.01\\
40.29	0.01\\
40.3	0.01\\
40.31	0.01\\
40.32	0.01\\
40.33	0.01\\
40.34	0.01\\
40.35	0.01\\
40.36	0.01\\
40.37	0.01\\
40.38	0.01\\
40.39	0.01\\
40.4	0.01\\
40.41	0.01\\
40.42	0.01\\
40.43	0.01\\
40.44	0.01\\
40.45	0.01\\
40.46	0.01\\
40.47	0.01\\
40.48	0.01\\
40.49	0.01\\
40.5	0.01\\
40.51	0.01\\
40.52	0.01\\
40.53	0.01\\
40.54	0.01\\
40.55	0.01\\
40.56	0.01\\
40.57	0.01\\
40.58	0.01\\
40.59	0.01\\
40.6	0.01\\
40.61	0.01\\
40.62	0.01\\
40.63	0.01\\
40.64	0.01\\
40.65	0.01\\
40.66	0.01\\
40.67	0.01\\
40.68	0.01\\
40.69	0.01\\
40.7	0.01\\
40.71	0.01\\
40.72	0.01\\
40.73	0.01\\
40.74	0.01\\
40.75	0.01\\
40.76	0.01\\
40.77	0.01\\
40.78	0.01\\
40.79	0.01\\
40.8	0.01\\
40.81	0.01\\
40.82	0.01\\
40.83	0.01\\
40.84	0.01\\
40.85	0.01\\
40.86	0.01\\
40.87	0.01\\
40.88	0.01\\
40.89	0.01\\
40.9	0.01\\
40.91	0.01\\
40.92	0.01\\
40.93	0.01\\
40.94	0.01\\
40.95	0.01\\
40.96	0.01\\
40.97	0.01\\
40.98	0.01\\
40.99	0.01\\
41	0.01\\
41.01	0.01\\
41.02	0.01\\
41.03	0.01\\
41.04	0.01\\
41.05	0.01\\
41.06	0.01\\
41.07	0.01\\
41.08	0.01\\
41.09	0.01\\
41.1	0.01\\
41.11	0.01\\
41.12	0.01\\
41.13	0.01\\
41.14	0.01\\
41.15	0.01\\
41.16	0.01\\
41.17	0.01\\
41.18	0.01\\
41.19	0.01\\
41.2	0.01\\
41.21	0.01\\
41.22	0.01\\
41.23	0.01\\
41.24	0.01\\
41.25	0.01\\
41.26	0.01\\
41.27	0.01\\
41.28	0.01\\
41.29	0.01\\
41.3	0.01\\
41.31	0.01\\
41.32	0.01\\
41.33	0.01\\
41.34	0.01\\
41.35	0.01\\
41.36	0.01\\
41.37	0.01\\
41.38	0.01\\
41.39	0.01\\
41.4	0.01\\
41.41	0.01\\
41.42	0.01\\
41.43	0.01\\
41.44	0.01\\
41.45	0.01\\
41.46	0.01\\
41.47	0.01\\
41.48	0.01\\
41.49	0.01\\
41.5	0.01\\
41.51	0.01\\
41.52	0.01\\
41.53	0.01\\
41.54	0.01\\
41.55	0.01\\
41.56	0.01\\
41.57	0.01\\
41.58	0.01\\
41.59	0.01\\
41.6	0.01\\
41.61	0.01\\
41.62	0.01\\
41.63	0.01\\
41.64	0.01\\
41.65	0.01\\
41.66	0.01\\
41.67	0.01\\
41.68	0.01\\
41.69	0.01\\
41.7	0.01\\
41.71	0.01\\
41.72	0.01\\
41.73	0.01\\
41.74	0.01\\
41.75	0.01\\
41.76	0.01\\
41.77	0.01\\
41.78	0.01\\
41.79	0.01\\
41.8	0.01\\
41.81	0.01\\
41.82	0.01\\
41.83	0.01\\
41.84	0.01\\
41.85	0.01\\
41.86	0.01\\
41.87	0.01\\
41.88	0.01\\
41.89	0.01\\
41.9	0.01\\
41.91	0.01\\
41.92	0.01\\
41.93	0.01\\
41.94	0.01\\
41.95	0.01\\
41.96	0.01\\
41.97	0.01\\
41.98	0.01\\
41.99	0.01\\
42	0.01\\
42.01	0.01\\
42.02	0.01\\
42.03	0.01\\
42.04	0.01\\
42.05	0.01\\
42.06	0.01\\
42.07	0.01\\
42.08	0.01\\
42.09	0.01\\
42.1	0.01\\
42.11	0.01\\
42.12	0.01\\
42.13	0.01\\
42.14	0.01\\
42.15	0.01\\
42.16	0.01\\
42.17	0.01\\
42.18	0.01\\
42.19	0.01\\
42.2	0.01\\
42.21	0.01\\
42.22	0.01\\
42.23	0.01\\
42.24	0.01\\
42.25	0.01\\
42.26	0.01\\
42.27	0.01\\
42.28	0.01\\
42.29	0.01\\
42.3	0.01\\
42.31	0.01\\
42.32	0.01\\
42.33	0.01\\
42.34	0.01\\
42.35	0.01\\
42.36	0.01\\
42.37	0.01\\
42.38	0.01\\
42.39	0.01\\
42.4	0.01\\
42.41	0.01\\
42.42	0.01\\
42.43	0.01\\
42.44	0.01\\
42.45	0.01\\
42.46	0.01\\
42.47	0.01\\
42.48	0.01\\
42.49	0.01\\
42.5	0.01\\
42.51	0.01\\
42.52	0.01\\
42.53	0.01\\
42.54	0.01\\
42.55	0.01\\
42.56	0.01\\
42.57	0.01\\
42.58	0.01\\
42.59	0.01\\
42.6	0.01\\
42.61	0.01\\
42.62	0.01\\
42.63	0.01\\
42.64	0.01\\
42.65	0.01\\
42.66	0.01\\
42.67	0.01\\
42.68	0.01\\
42.69	0.01\\
42.7	0.01\\
42.71	0.01\\
42.72	0.01\\
42.73	0.01\\
42.74	0.01\\
42.75	0.01\\
42.76	0.01\\
42.77	0.01\\
42.78	0.01\\
42.79	0.01\\
42.8	0.01\\
42.81	0.01\\
42.82	0.01\\
42.83	0.01\\
42.84	0.01\\
42.85	0.01\\
42.86	0.01\\
42.87	0.01\\
42.88	0.01\\
42.89	0.01\\
42.9	0.01\\
42.91	0.01\\
42.92	0.01\\
42.93	0.01\\
42.94	0.01\\
42.95	0.01\\
42.96	0.01\\
42.97	0.01\\
42.98	0.01\\
42.99	0.01\\
43	0.01\\
43.01	0.01\\
43.02	0.01\\
43.03	0.01\\
43.04	0.01\\
43.05	0.01\\
43.06	0.01\\
43.07	0.01\\
43.08	0.01\\
43.09	0.01\\
43.1	0.01\\
43.11	0.01\\
43.12	0.01\\
43.13	0.01\\
43.14	0.01\\
43.15	0.01\\
43.16	0.01\\
43.17	0.01\\
43.18	0.01\\
43.19	0.01\\
43.2	0.01\\
43.21	0.01\\
43.22	0.01\\
43.23	0.01\\
43.24	0.01\\
43.25	0.01\\
43.26	0.01\\
43.27	0.01\\
43.28	0.01\\
43.29	0.01\\
43.3	0.01\\
43.31	0.01\\
43.32	0.01\\
43.33	0.01\\
43.34	0.01\\
43.35	0.01\\
43.36	0.01\\
43.37	0.01\\
43.38	0.01\\
43.39	0.01\\
43.4	0.01\\
43.41	0.01\\
43.42	0.01\\
43.43	0.01\\
43.44	0.01\\
43.45	0.01\\
43.46	0.01\\
43.47	0.01\\
43.48	0.01\\
43.49	0.01\\
43.5	0.01\\
43.51	0.01\\
43.52	0.01\\
43.53	0.01\\
43.54	0.01\\
43.55	0.01\\
43.56	0.01\\
43.57	0.01\\
43.58	0.01\\
43.59	0.01\\
43.6	0.01\\
43.61	0.01\\
43.62	0.01\\
43.63	0.01\\
43.64	0.01\\
43.65	0.01\\
43.66	0.01\\
43.67	0.01\\
43.68	0.01\\
43.69	0.01\\
43.7	0.01\\
43.71	0.01\\
43.72	0.01\\
43.73	0.01\\
43.74	0.01\\
43.75	0.01\\
43.76	0.01\\
43.77	0.01\\
43.78	0.01\\
43.79	0.01\\
43.8	0.01\\
43.81	0.01\\
43.82	0.01\\
43.83	0.01\\
43.84	0.01\\
43.85	0.01\\
43.86	0.01\\
43.87	0.01\\
43.88	0.01\\
43.89	0.01\\
43.9	0.01\\
43.91	0.01\\
43.92	0.01\\
43.93	0.01\\
43.94	0.01\\
43.95	0.01\\
43.96	0.01\\
43.97	0.01\\
43.98	0.01\\
43.99	0.01\\
44	0.01\\
44.01	0.01\\
44.02	0.01\\
44.03	0.01\\
44.04	0.01\\
44.05	0.01\\
44.06	0.01\\
44.07	0.01\\
44.08	0.01\\
44.09	0.01\\
44.1	0.01\\
44.11	0.01\\
44.12	0.01\\
44.13	0.01\\
44.14	0.01\\
44.15	0.01\\
44.16	0.01\\
44.17	0.01\\
44.18	0.01\\
44.19	0.01\\
44.2	0.01\\
44.21	0.01\\
44.22	0.01\\
44.23	0.01\\
44.24	0.01\\
44.25	0.01\\
44.26	0.01\\
44.27	0.01\\
44.28	0.01\\
44.29	0.01\\
44.3	0.01\\
44.31	0.01\\
44.32	0.01\\
44.33	0.01\\
44.34	0.01\\
44.35	0.01\\
44.36	0.01\\
44.37	0.01\\
44.38	0.01\\
44.39	0.01\\
44.4	0.01\\
44.41	0.01\\
44.42	0.01\\
44.43	0.01\\
44.44	0.01\\
44.45	0.01\\
44.46	0.01\\
44.47	0.01\\
44.48	0.01\\
44.49	0.01\\
44.5	0.01\\
44.51	0.01\\
44.52	0.01\\
44.53	0.01\\
44.54	0.01\\
44.55	0.01\\
44.56	0.01\\
44.57	0.01\\
44.58	0.01\\
44.59	0.01\\
44.6	0.01\\
44.61	0.01\\
44.62	0.01\\
44.63	0.01\\
44.64	0.01\\
44.65	0.01\\
44.66	0.01\\
44.67	0.01\\
44.68	0.01\\
44.69	0.01\\
44.7	0.01\\
44.71	0.01\\
44.72	0.01\\
44.73	0.01\\
44.74	0.01\\
44.75	0.01\\
44.76	0.01\\
44.77	0.01\\
44.78	0.01\\
44.79	0.01\\
44.8	0.01\\
44.81	0.01\\
44.82	0.01\\
44.83	0.01\\
44.84	0.01\\
44.85	0.01\\
44.86	0.01\\
44.87	0.01\\
44.88	0.01\\
44.89	0.01\\
44.9	0.01\\
44.91	0.01\\
44.92	0.01\\
44.93	0.01\\
44.94	0.01\\
44.95	0.01\\
44.96	0.01\\
44.97	0.01\\
44.98	0.01\\
44.99	0.01\\
45	0.01\\
45.01	0.01\\
45.02	0.01\\
45.03	0.01\\
45.04	0.01\\
45.05	0.01\\
45.06	0.01\\
45.07	0.01\\
45.08	0.01\\
45.09	0.01\\
45.1	0.01\\
45.11	0.01\\
45.12	0.01\\
45.13	0.01\\
45.14	0.01\\
45.15	0.01\\
45.16	0.01\\
45.17	0.01\\
45.18	0.01\\
45.19	0.01\\
45.2	0.01\\
45.21	0.01\\
45.22	0.01\\
45.23	0.01\\
45.24	0.01\\
45.25	0.01\\
45.26	0.01\\
45.27	0.01\\
45.28	0.01\\
45.29	0.01\\
45.3	0.01\\
45.31	0.01\\
45.32	0.01\\
45.33	0.01\\
45.34	0.01\\
45.35	0.01\\
45.36	0.01\\
45.37	0.01\\
45.38	0.01\\
45.39	0.01\\
45.4	0.01\\
45.41	0.01\\
45.42	0.01\\
45.43	0.01\\
45.44	0.01\\
45.45	0.01\\
45.46	0.01\\
45.47	0.01\\
45.48	0.01\\
45.49	0.01\\
45.5	0.01\\
45.51	0.01\\
45.52	0.01\\
45.53	0.01\\
45.54	0.01\\
45.55	0.01\\
45.56	0.01\\
45.57	0.01\\
45.58	0.01\\
45.59	0.01\\
45.6	0.01\\
45.61	0.01\\
45.62	0.01\\
45.63	0.01\\
45.64	0.01\\
45.65	0.01\\
45.66	0.01\\
45.67	0.01\\
45.68	0.01\\
45.69	0.01\\
45.7	0.01\\
45.71	0.01\\
45.72	0.01\\
45.73	0.01\\
45.74	0.01\\
45.75	0.01\\
45.76	0.01\\
45.77	0.01\\
45.78	0.01\\
45.79	0.01\\
45.8	0.01\\
45.81	0.01\\
45.82	0.01\\
45.83	0.01\\
45.84	0.01\\
45.85	0.01\\
45.86	0.01\\
45.87	0.01\\
45.88	0.01\\
45.89	0.01\\
45.9	0.01\\
45.91	0.01\\
45.92	0.01\\
45.93	0.01\\
45.94	0.01\\
45.95	0.01\\
45.96	0.01\\
45.97	0.01\\
45.98	0.01\\
45.99	0.01\\
46	0.01\\
46.01	0.01\\
46.02	0.01\\
46.03	0.01\\
46.04	0.01\\
46.05	0.01\\
46.06	0.01\\
46.07	0.01\\
46.08	0.01\\
46.09	0.01\\
46.1	0.01\\
46.11	0.01\\
46.12	0.01\\
46.13	0.01\\
46.14	0.01\\
46.15	0.01\\
46.16	0.01\\
46.17	0.01\\
46.18	0.01\\
46.19	0.01\\
46.2	0.01\\
46.21	0.01\\
46.22	0.01\\
46.23	0.01\\
46.24	0.01\\
46.25	0.01\\
46.26	0.01\\
46.27	0.01\\
46.28	0.01\\
46.29	0.01\\
46.3	0.01\\
46.31	0.01\\
46.32	0.01\\
46.33	0.01\\
46.34	0.01\\
46.35	0.01\\
46.36	0.01\\
46.37	0.01\\
46.38	0.01\\
46.39	0.01\\
46.4	0.01\\
46.41	0.01\\
46.42	0.01\\
46.43	0.01\\
46.44	0.01\\
46.45	0.01\\
46.46	0.01\\
46.47	0.01\\
46.48	0.01\\
46.49	0.01\\
46.5	0.01\\
46.51	0.01\\
46.52	0.01\\
46.53	0.01\\
46.54	0.01\\
46.55	0.01\\
46.56	0.01\\
46.57	0.01\\
46.58	0.01\\
46.59	0.01\\
46.6	0.01\\
46.61	0.01\\
46.62	0.01\\
46.63	0.01\\
46.64	0.01\\
46.65	0.01\\
46.66	0.01\\
46.67	0.01\\
46.68	0.01\\
46.69	0.01\\
46.7	0.01\\
46.71	0.01\\
46.72	0.01\\
46.73	0.01\\
46.74	0.01\\
46.75	0.01\\
46.76	0.01\\
46.77	0.01\\
46.78	0.01\\
46.79	0.01\\
46.8	0.01\\
46.81	0.01\\
46.82	0.01\\
46.83	0.01\\
46.84	0.01\\
46.85	0.01\\
46.86	0.01\\
46.87	0.01\\
46.88	0.01\\
46.89	0.01\\
46.9	0.01\\
46.91	0.01\\
46.92	0.01\\
46.93	0.01\\
46.94	0.01\\
46.95	0.01\\
46.96	0.01\\
46.97	0.01\\
46.98	0.01\\
46.99	0.01\\
47	0.01\\
47.01	0.01\\
47.02	0.01\\
47.03	0.01\\
47.04	0.01\\
47.05	0.01\\
47.06	0.01\\
47.07	0.01\\
47.08	0.01\\
47.09	0.01\\
47.1	0.01\\
47.11	0.01\\
47.12	0.01\\
47.13	0.01\\
47.14	0.01\\
47.15	0.01\\
47.16	0.01\\
47.17	0.01\\
47.18	0.01\\
47.19	0.01\\
47.2	0.01\\
47.21	0.01\\
47.22	0.01\\
47.23	0.01\\
47.24	0.01\\
47.25	0.01\\
47.26	0.01\\
47.27	0.01\\
47.28	0.01\\
47.29	0.01\\
47.3	0.01\\
47.31	0.01\\
47.32	0.01\\
47.33	0.01\\
47.34	0.01\\
47.35	0.01\\
47.36	0.01\\
47.37	0.01\\
47.38	0.01\\
47.39	0.01\\
47.4	0.01\\
47.41	0.01\\
47.42	0.01\\
47.43	0.01\\
47.44	0.01\\
47.45	0.01\\
47.46	0.01\\
47.47	0.01\\
47.48	0.01\\
47.49	0.01\\
47.5	0.01\\
47.51	0.01\\
47.52	0.01\\
47.53	0.01\\
47.54	0.01\\
47.55	0.01\\
47.56	0.01\\
47.57	0.01\\
47.58	0.01\\
47.59	0.01\\
47.6	0.01\\
47.61	0.01\\
47.62	0.01\\
47.63	0.01\\
47.64	0.01\\
47.65	0.01\\
47.66	0.01\\
47.67	0.01\\
47.68	0.01\\
47.69	0.01\\
47.7	0.01\\
47.71	0.01\\
47.72	0.01\\
47.73	0.01\\
47.74	0.01\\
47.75	0.01\\
47.76	0.01\\
47.77	0.01\\
47.78	0.01\\
47.79	0.01\\
47.8	0.01\\
47.81	0.01\\
47.82	0.01\\
47.83	0.01\\
47.84	0.01\\
47.85	0.01\\
47.86	0.01\\
47.87	0.01\\
47.88	0.01\\
47.89	0.01\\
47.9	0.01\\
47.91	0.01\\
47.92	0.01\\
47.93	0.01\\
47.94	0.01\\
47.95	0.01\\
47.96	0.01\\
47.97	0.01\\
47.98	0.01\\
47.99	0.01\\
48	0.01\\
48.01	0.01\\
48.02	0.01\\
48.03	0.01\\
48.04	0.01\\
48.05	0.01\\
48.06	0.01\\
48.07	0.01\\
48.08	0.01\\
48.09	0.01\\
48.1	0.01\\
48.11	0.01\\
48.12	0.01\\
48.13	0.01\\
48.14	0.01\\
48.15	0.01\\
48.16	0.01\\
48.17	0.01\\
48.18	0.01\\
48.19	0.01\\
48.2	0.01\\
48.21	0.01\\
48.22	0.01\\
48.23	0.01\\
48.24	0.01\\
48.25	0.01\\
48.26	0.01\\
48.27	0.01\\
48.28	0.01\\
48.29	0.01\\
48.3	0.01\\
48.31	0.01\\
48.32	0.01\\
48.33	0.01\\
48.34	0.01\\
48.35	0.01\\
48.36	0.01\\
48.37	0.01\\
48.38	0.01\\
48.39	0.01\\
48.4	0.01\\
48.41	0.01\\
48.42	0.01\\
48.43	0.01\\
48.44	0.01\\
48.45	0.01\\
48.46	0.01\\
48.47	0.01\\
48.48	0.01\\
48.49	0.01\\
48.5	0.01\\
48.51	0.01\\
48.52	0.01\\
48.53	0.01\\
48.54	0.01\\
48.55	0.01\\
48.56	0.01\\
48.57	0.01\\
48.58	0.01\\
48.59	0.01\\
48.6	0.01\\
48.61	0.01\\
48.62	0.01\\
48.63	0.01\\
48.64	0.01\\
48.65	0.01\\
48.66	0.01\\
48.67	0.01\\
48.68	0.01\\
48.69	0.01\\
48.7	0.01\\
48.71	0.01\\
48.72	0.01\\
48.73	0.01\\
48.74	0.01\\
48.75	0.01\\
48.76	0.01\\
48.77	0.01\\
48.78	0.01\\
48.79	0.01\\
48.8	0.01\\
48.81	0.01\\
48.82	0.01\\
48.83	0.01\\
48.84	0.01\\
48.85	0.01\\
48.86	0.01\\
48.87	0.01\\
48.88	0.01\\
48.89	0.01\\
48.9	0.01\\
48.91	0.01\\
48.92	0.01\\
48.93	0.01\\
48.94	0.01\\
48.95	0.01\\
48.96	0.01\\
48.97	0.01\\
48.98	0.01\\
48.99	0.01\\
49	0.01\\
49.01	0.01\\
49.02	0.01\\
49.03	0.01\\
49.04	0.01\\
49.05	0.01\\
49.06	0.01\\
49.07	0.01\\
49.08	0.01\\
49.09	0.01\\
49.1	0.01\\
49.11	0.01\\
49.12	0.01\\
49.13	0.01\\
49.14	0.01\\
49.15	0.01\\
49.16	0.01\\
49.17	0.01\\
49.18	0.01\\
49.19	0.01\\
49.2	0.01\\
49.21	0.01\\
49.22	0.01\\
49.23	0.01\\
49.24	0.01\\
49.25	0.01\\
49.26	0.01\\
49.27	0.01\\
49.28	0.01\\
49.29	0.01\\
49.3	0.01\\
49.31	0.01\\
49.32	0.01\\
49.33	0.01\\
49.34	0.01\\
49.35	0.01\\
49.36	0.01\\
49.37	0.01\\
49.38	0.01\\
49.39	0.01\\
49.4	0.01\\
49.41	0.01\\
49.42	0.01\\
49.43	0.01\\
49.44	0.01\\
49.45	0.01\\
49.46	0.01\\
49.47	0.01\\
49.48	0.01\\
49.49	0.01\\
49.5	0.01\\
49.51	0.01\\
49.52	0.01\\
49.53	0.01\\
49.54	0.01\\
49.55	0.01\\
49.56	0.01\\
49.57	0.01\\
49.58	0.01\\
49.59	0.01\\
49.6	0.01\\
49.61	0.01\\
49.62	0.01\\
49.63	0.01\\
49.64	0.01\\
49.65	0.01\\
49.66	0.01\\
49.67	0.01\\
49.68	0.01\\
49.69	0.01\\
49.7	0.01\\
49.71	0.01\\
49.72	0.01\\
49.73	0.01\\
49.74	0.01\\
49.75	0.01\\
49.76	0.01\\
49.77	0.01\\
49.78	0.01\\
49.79	0.01\\
49.8	0.01\\
49.81	0.01\\
49.82	0.01\\
49.83	0.01\\
49.84	0.01\\
49.85	0.01\\
49.86	0.01\\
49.87	0.01\\
49.88	0.01\\
49.89	0.01\\
49.9	0.01\\
49.91	0.01\\
49.92	0.01\\
49.93	0.01\\
49.94	0.01\\
49.95	0.01\\
49.96	0.01\\
49.97	0.01\\
49.98	0.01\\
49.99	0.01\\
50	0.01\\
50.01	0.01\\
50.02	0.01\\
50.03	0.01\\
50.04	0.01\\
50.05	0.01\\
50.06	0.01\\
50.07	0.01\\
50.08	0.01\\
50.09	0.01\\
50.1	0.01\\
50.11	0.01\\
50.12	0.01\\
50.13	0.01\\
50.14	0.01\\
50.15	0.01\\
50.16	0.01\\
50.17	0.01\\
50.18	0.01\\
50.19	0.01\\
50.2	0.01\\
50.21	0.01\\
50.22	0.01\\
50.23	0.01\\
50.24	0.01\\
50.25	0.01\\
50.26	0.01\\
50.27	0.01\\
50.28	0.01\\
50.29	0.01\\
50.3	0.01\\
50.31	0.01\\
50.32	0.01\\
50.33	0.01\\
50.34	0.01\\
50.35	0.01\\
50.36	0.01\\
50.37	0.01\\
50.38	0.01\\
50.39	0.01\\
50.4	0.01\\
50.41	0.01\\
50.42	0.01\\
50.43	0.01\\
50.44	0.01\\
50.45	0.01\\
50.46	0.01\\
50.47	0.01\\
50.48	0.01\\
50.49	0.01\\
50.5	0.01\\
50.51	0.01\\
50.52	0.01\\
50.53	0.01\\
50.54	0.01\\
50.55	0.01\\
50.56	0.01\\
50.57	0.01\\
50.58	0.01\\
50.59	0.01\\
50.6	0.01\\
50.61	0.01\\
50.62	0.01\\
50.63	0.01\\
50.64	0.01\\
50.65	0.01\\
50.66	0.01\\
50.67	0.01\\
50.68	0.01\\
50.69	0.01\\
50.7	0.01\\
50.71	0.01\\
50.72	0.01\\
50.73	0.01\\
50.74	0.01\\
50.75	0.01\\
50.76	0.01\\
50.77	0.01\\
50.78	0.01\\
50.79	0.01\\
50.8	0.01\\
50.81	0.01\\
50.82	0.01\\
50.83	0.01\\
50.84	0.01\\
50.85	0.01\\
50.86	0.01\\
50.87	0.01\\
50.88	0.01\\
50.89	0.01\\
50.9	0.01\\
50.91	0.01\\
50.92	0.01\\
50.93	0.01\\
50.94	0.01\\
50.95	0.01\\
50.96	0.01\\
50.97	0.01\\
50.98	0.01\\
50.99	0.01\\
51	0.01\\
51.01	0.01\\
51.02	0.01\\
51.03	0.01\\
51.04	0.01\\
51.05	0.01\\
51.06	0.01\\
51.07	0.01\\
51.08	0.01\\
51.09	0.01\\
51.1	0.01\\
51.11	0.01\\
51.12	0.01\\
51.13	0.01\\
51.14	0.01\\
51.15	0.01\\
51.16	0.01\\
51.17	0.01\\
51.18	0.01\\
51.19	0.01\\
51.2	0.01\\
51.21	0.01\\
51.22	0.01\\
51.23	0.01\\
51.24	0.01\\
51.25	0.01\\
51.26	0.01\\
51.27	0.01\\
51.28	0.01\\
51.29	0.01\\
51.3	0.01\\
51.31	0.01\\
51.32	0.01\\
51.33	0.01\\
51.34	0.01\\
51.35	0.01\\
51.36	0.01\\
51.37	0.01\\
51.38	0.01\\
51.39	0.01\\
51.4	0.01\\
51.41	0.01\\
51.42	0.01\\
51.43	0.01\\
51.44	0.01\\
51.45	0.01\\
51.46	0.01\\
51.47	0.01\\
51.48	0.01\\
51.49	0.01\\
51.5	0.01\\
51.51	0.01\\
51.52	0.01\\
51.53	0.01\\
51.54	0.01\\
51.55	0.01\\
51.56	0.01\\
51.57	0.01\\
51.58	0.01\\
51.59	0.01\\
51.6	0.01\\
51.61	0.01\\
51.62	0.01\\
51.63	0.01\\
51.64	0.01\\
51.65	0.01\\
51.66	0.01\\
51.67	0.01\\
51.68	0.01\\
51.69	0.01\\
51.7	0.01\\
51.71	0.01\\
51.72	0.01\\
51.73	0.01\\
51.74	0.01\\
51.75	0.01\\
51.76	0.01\\
51.77	0.01\\
51.78	0.01\\
51.79	0.01\\
51.8	0.01\\
51.81	0.01\\
51.82	0.01\\
51.83	0.01\\
51.84	0.01\\
51.85	0.01\\
51.86	0.01\\
51.87	0.01\\
51.88	0.01\\
51.89	0.01\\
51.9	0.01\\
51.91	0.01\\
51.92	0.01\\
51.93	0.01\\
51.94	0.01\\
51.95	0.01\\
51.96	0.01\\
51.97	0.01\\
51.98	0.01\\
51.99	0.01\\
52	0.01\\
52.01	0.01\\
52.02	0.01\\
52.03	0.01\\
52.04	0.01\\
52.05	0.01\\
52.06	0.01\\
52.07	0.01\\
52.08	0.01\\
52.09	0.01\\
52.1	0.01\\
52.11	0.01\\
52.12	0.01\\
52.13	0.01\\
52.14	0.01\\
52.15	0.01\\
52.16	0.01\\
52.17	0.01\\
52.18	0.01\\
52.19	0.01\\
52.2	0.01\\
52.21	0.01\\
52.22	0.01\\
52.23	0.01\\
52.24	0.01\\
52.25	0.01\\
52.26	0.01\\
52.27	0.01\\
52.28	0.01\\
52.29	0.01\\
52.3	0.01\\
52.31	0.01\\
52.32	0.01\\
52.33	0.01\\
52.34	0.01\\
52.35	0.01\\
52.36	0.01\\
52.37	0.01\\
52.38	0.01\\
52.39	0.01\\
52.4	0.01\\
52.41	0.01\\
52.42	0.01\\
52.43	0.01\\
52.44	0.01\\
52.45	0.01\\
52.46	0.01\\
52.47	0.01\\
52.48	0.01\\
52.49	0.01\\
52.5	0.01\\
52.51	0.01\\
52.52	0.01\\
52.53	0.01\\
52.54	0.01\\
52.55	0.01\\
52.56	0.01\\
52.57	0.01\\
52.58	0.01\\
52.59	0.01\\
52.6	0.01\\
52.61	0.01\\
52.62	0.01\\
52.63	0.01\\
52.64	0.01\\
52.65	0.01\\
52.66	0.01\\
52.67	0.01\\
52.68	0.01\\
52.69	0.01\\
52.7	0.01\\
52.71	0.01\\
52.72	0.01\\
52.73	0.01\\
52.74	0.01\\
52.75	0.01\\
52.76	0.01\\
52.77	0.01\\
52.78	0.01\\
52.79	0.01\\
52.8	0.01\\
52.81	0.01\\
52.82	0.01\\
52.83	0.01\\
52.84	0.01\\
52.85	0.01\\
52.86	0.01\\
52.87	0.01\\
52.88	0.01\\
52.89	0.01\\
52.9	0.01\\
52.91	0.01\\
52.92	0.01\\
52.93	0.01\\
52.94	0.01\\
52.95	0.01\\
52.96	0.01\\
52.97	0.01\\
52.98	0.01\\
52.99	0.01\\
53	0.01\\
53.01	0.01\\
53.02	0.01\\
53.03	0.01\\
53.04	0.01\\
53.05	0.01\\
53.06	0.01\\
53.07	0.01\\
53.08	0.01\\
53.09	0.01\\
53.1	0.01\\
53.11	0.01\\
53.12	0.01\\
53.13	0.01\\
53.14	0.01\\
53.15	0.01\\
53.16	0.01\\
53.17	0.01\\
53.18	0.01\\
53.19	0.01\\
53.2	0.01\\
53.21	0.01\\
53.22	0.01\\
53.23	0.01\\
53.24	0.01\\
53.25	0.01\\
53.26	0.01\\
53.27	0.01\\
53.28	0.01\\
53.29	0.01\\
53.3	0.01\\
53.31	0.01\\
53.32	0.01\\
53.33	0.01\\
53.34	0.01\\
53.35	0.01\\
53.36	0.01\\
53.37	0.01\\
53.38	0.01\\
53.39	0.01\\
53.4	0.01\\
53.41	0.01\\
53.42	0.01\\
53.43	0.01\\
53.44	0.01\\
53.45	0.01\\
53.46	0.01\\
53.47	0.01\\
53.48	0.01\\
53.49	0.01\\
53.5	0.01\\
53.51	0.01\\
53.52	0.01\\
53.53	0.01\\
53.54	0.01\\
53.55	0.01\\
53.56	0.01\\
53.57	0.01\\
53.58	0.01\\
53.59	0.01\\
53.6	0.01\\
53.61	0.01\\
53.62	0.01\\
53.63	0.01\\
53.64	0.01\\
53.65	0.01\\
53.66	0.01\\
53.67	0.01\\
53.68	0.01\\
53.69	0.01\\
53.7	0.01\\
53.71	0.01\\
53.72	0.01\\
53.73	0.01\\
53.74	0.01\\
53.75	0.01\\
53.76	0.01\\
53.77	0.01\\
53.78	0.01\\
53.79	0.01\\
53.8	0.01\\
53.81	0.01\\
53.82	0.01\\
53.83	0.01\\
53.84	0.01\\
53.85	0.01\\
53.86	0.01\\
53.87	0.01\\
53.88	0.01\\
53.89	0.01\\
53.9	0.01\\
53.91	0.01\\
53.92	0.01\\
53.93	0.01\\
53.94	0.01\\
53.95	0.01\\
53.96	0.01\\
53.97	0.01\\
53.98	0.01\\
53.99	0.01\\
54	0.01\\
54.01	0.01\\
54.02	0.01\\
54.03	0.01\\
54.04	0.01\\
54.05	0.01\\
54.06	0.01\\
54.07	0.01\\
54.08	0.01\\
54.09	0.01\\
54.1	0.01\\
54.11	0.01\\
54.12	0.01\\
54.13	0.01\\
54.14	0.01\\
54.15	0.01\\
54.16	0.01\\
54.17	0.01\\
54.18	0.01\\
54.19	0.01\\
54.2	0.01\\
54.21	0.01\\
54.22	0.01\\
54.23	0.01\\
54.24	0.01\\
54.25	0.01\\
54.26	0.01\\
54.27	0.01\\
54.28	0.01\\
54.29	0.01\\
54.3	0.01\\
54.31	0.01\\
54.32	0.01\\
54.33	0.01\\
54.34	0.01\\
54.35	0.01\\
54.36	0.01\\
54.37	0.01\\
54.38	0.01\\
54.39	0.01\\
54.4	0.01\\
54.41	0.01\\
54.42	0.01\\
54.43	0.01\\
54.44	0.01\\
54.45	0.01\\
54.46	0.01\\
54.47	0.01\\
54.48	0.01\\
54.49	0.01\\
54.5	0.01\\
54.51	0.01\\
54.52	0.01\\
54.53	0.01\\
54.54	0.01\\
54.55	0.01\\
54.56	0.01\\
54.57	0.01\\
54.58	0.01\\
54.59	0.01\\
54.6	0.01\\
54.61	0.01\\
54.62	0.01\\
54.63	0.01\\
54.64	0.01\\
54.65	0.01\\
54.66	0.01\\
54.67	0.01\\
54.68	0.01\\
54.69	0.01\\
54.7	0.01\\
54.71	0.01\\
54.72	0.01\\
54.73	0.01\\
54.74	0.01\\
54.75	0.01\\
54.76	0.01\\
54.77	0.01\\
54.78	0.01\\
54.79	0.01\\
54.8	0.01\\
54.81	0.01\\
54.82	0.01\\
54.83	0.01\\
54.84	0.01\\
54.85	0.01\\
54.86	0.01\\
54.87	0.01\\
54.88	0.01\\
54.89	0.01\\
54.9	0.01\\
54.91	0.01\\
54.92	0.01\\
54.93	0.01\\
54.94	0.01\\
54.95	0.01\\
54.96	0.01\\
54.97	0.01\\
54.98	0.01\\
54.99	0.01\\
55	0.01\\
55.01	0.01\\
55.02	0.01\\
55.03	0.01\\
55.04	0.01\\
55.05	0.01\\
55.06	0.01\\
55.07	0.01\\
55.08	0.01\\
55.09	0.01\\
55.1	0.01\\
55.11	0.01\\
55.12	0.01\\
55.13	0.01\\
55.14	0.01\\
55.15	0.01\\
55.16	0.01\\
55.17	0.01\\
55.18	0.01\\
55.19	0.01\\
55.2	0.01\\
55.21	0.01\\
55.22	0.01\\
55.23	0.01\\
55.24	0.01\\
55.25	0.01\\
55.26	0.01\\
55.27	0.01\\
55.28	0.01\\
55.29	0.01\\
55.3	0.01\\
55.31	0.01\\
55.32	0.01\\
55.33	0.01\\
55.34	0.01\\
55.35	0.01\\
55.36	0.01\\
55.37	0.01\\
55.38	0.01\\
55.39	0.01\\
55.4	0.01\\
55.41	0.01\\
55.42	0.01\\
55.43	0.01\\
55.44	0.01\\
55.45	0.01\\
55.46	0.01\\
55.47	0.01\\
55.48	0.01\\
55.49	0.01\\
55.5	0.01\\
55.51	0.01\\
55.52	0.01\\
55.53	0.01\\
55.54	0.01\\
55.55	0.01\\
55.56	0.01\\
55.57	0.01\\
55.58	0.01\\
55.59	0.01\\
55.6	0.01\\
55.61	0.01\\
55.62	0.01\\
55.63	0.01\\
55.64	0.01\\
55.65	0.01\\
55.66	0.01\\
55.67	0.01\\
55.68	0.01\\
55.69	0.01\\
55.7	0.01\\
55.71	0.01\\
55.72	0.01\\
55.73	0.01\\
55.74	0.01\\
55.75	0.01\\
55.76	0.01\\
55.77	0.01\\
55.78	0.01\\
55.79	0.01\\
55.8	0.01\\
55.81	0.01\\
55.82	0.01\\
55.83	0.01\\
55.84	0.01\\
55.85	0.01\\
55.86	0.01\\
55.87	0.01\\
55.88	0.01\\
55.89	0.01\\
55.9	0.01\\
55.91	0.01\\
55.92	0.01\\
55.93	0.01\\
55.94	0.01\\
55.95	0.01\\
55.96	0.01\\
55.97	0.01\\
55.98	0.01\\
55.99	0.01\\
56	0.01\\
56.01	0.01\\
56.02	0.01\\
56.03	0.01\\
56.04	0.01\\
56.05	0.01\\
56.06	0.01\\
56.07	0.01\\
56.08	0.01\\
56.09	0.01\\
56.1	0.01\\
56.11	0.01\\
56.12	0.01\\
56.13	0.01\\
56.14	0.01\\
56.15	0.01\\
56.16	0.01\\
56.17	0.01\\
56.18	0.01\\
56.19	0.01\\
56.2	0.01\\
56.21	0.01\\
56.22	0.01\\
56.23	0.01\\
56.24	0.01\\
56.25	0.01\\
56.26	0.01\\
56.27	0.01\\
56.28	0.01\\
56.29	0.01\\
56.3	0.01\\
56.31	0.01\\
56.32	0.01\\
56.33	0.01\\
56.34	0.01\\
56.35	0.01\\
56.36	0.01\\
56.37	0.01\\
56.38	0.01\\
56.39	0.01\\
56.4	0.01\\
56.41	0.01\\
56.42	0.01\\
56.43	0.01\\
56.44	0.01\\
56.45	0.01\\
56.46	0.01\\
56.47	0.01\\
56.48	0.01\\
56.49	0.01\\
56.5	0.01\\
56.51	0.01\\
56.52	0.01\\
56.53	0.01\\
56.54	0.01\\
56.55	0.01\\
56.56	0.01\\
56.57	0.01\\
56.58	0.01\\
56.59	0.01\\
56.6	0.01\\
56.61	0.01\\
56.62	0.01\\
56.63	0.01\\
56.64	0.01\\
56.65	0.01\\
56.66	0.01\\
56.67	0.01\\
56.68	0.01\\
56.69	0.01\\
56.7	0.01\\
56.71	0.01\\
56.72	0.01\\
56.73	0.01\\
56.74	0.01\\
56.75	0.01\\
56.76	0.01\\
56.77	0.01\\
56.78	0.01\\
56.79	0.01\\
56.8	0.01\\
56.81	0.01\\
56.82	0.01\\
56.83	0.01\\
56.84	0.01\\
56.85	0.01\\
56.86	0.01\\
56.87	0.01\\
56.88	0.01\\
56.89	0.01\\
56.9	0.01\\
56.91	0.01\\
56.92	0.01\\
56.93	0.01\\
56.94	0.01\\
56.95	0.01\\
56.96	0.01\\
56.97	0.01\\
56.98	0.01\\
56.99	0.01\\
57	0.01\\
57.01	0.01\\
57.02	0.01\\
57.03	0.01\\
57.04	0.01\\
57.05	0.01\\
57.06	0.01\\
57.07	0.01\\
57.08	0.01\\
57.09	0.01\\
57.1	0.01\\
57.11	0.01\\
57.12	0.01\\
57.13	0.01\\
57.14	0.01\\
57.15	0.01\\
57.16	0.01\\
57.17	0.01\\
57.18	0.01\\
57.19	0.01\\
57.2	0.01\\
57.21	0.01\\
57.22	0.01\\
57.23	0.01\\
57.24	0.01\\
57.25	0.01\\
57.26	0.01\\
57.27	0.01\\
57.28	0.01\\
57.29	0.01\\
57.3	0.01\\
57.31	0.01\\
57.32	0.01\\
57.33	0.01\\
57.34	0.01\\
57.35	0.01\\
57.36	0.01\\
57.37	0.01\\
57.38	0.01\\
57.39	0.01\\
57.4	0.01\\
57.41	0.01\\
57.42	0.01\\
57.43	0.01\\
57.44	0.01\\
57.45	0.01\\
57.46	0.01\\
57.47	0.01\\
57.48	0.01\\
57.49	0.01\\
57.5	0.01\\
57.51	0.01\\
57.52	0.01\\
57.53	0.01\\
57.54	0.01\\
57.55	0.01\\
57.56	0.01\\
57.57	0.01\\
57.58	0.01\\
57.59	0.01\\
57.6	0.01\\
57.61	0.01\\
57.62	0.01\\
57.63	0.01\\
57.64	0.01\\
57.65	0.01\\
57.66	0.01\\
57.67	0.01\\
57.68	0.01\\
57.69	0.01\\
57.7	0.01\\
57.71	0.01\\
57.72	0.01\\
57.73	0.01\\
57.74	0.01\\
57.75	0.01\\
57.76	0.01\\
57.77	0.01\\
57.78	0.01\\
57.79	0.01\\
57.8	0.01\\
57.81	0.01\\
57.82	0.01\\
57.83	0.01\\
57.84	0.01\\
57.85	0.01\\
57.86	0.01\\
57.87	0.01\\
57.88	0.01\\
57.89	0.01\\
57.9	0.01\\
57.91	0.01\\
57.92	0.01\\
57.93	0.01\\
57.94	0.01\\
57.95	0.01\\
57.96	0.01\\
57.97	0.01\\
57.98	0.01\\
57.99	0.01\\
58	0.01\\
58.01	0.01\\
58.02	0.01\\
58.03	0.01\\
58.04	0.01\\
58.05	0.01\\
58.06	0.01\\
58.07	0.01\\
58.08	0.01\\
58.09	0.01\\
58.1	0.01\\
58.11	0.01\\
58.12	0.01\\
58.13	0.01\\
58.14	0.01\\
58.15	0.01\\
58.16	0.01\\
58.17	0.01\\
58.18	0.01\\
58.19	0.01\\
58.2	0.01\\
58.21	0.01\\
58.22	0.01\\
58.23	0.01\\
58.24	0.01\\
58.25	0.01\\
58.26	0.01\\
58.27	0.01\\
58.28	0.01\\
58.29	0.01\\
58.3	0.01\\
58.31	0.01\\
58.32	0.01\\
58.33	0.01\\
58.34	0.01\\
58.35	0.01\\
58.36	0.01\\
58.37	0.01\\
58.38	0.01\\
58.39	0.01\\
58.4	0.01\\
58.41	0.01\\
58.42	0.01\\
58.43	0.01\\
58.44	0.01\\
58.45	0.01\\
58.46	0.01\\
58.47	0.01\\
58.48	0.01\\
58.49	0.01\\
58.5	0.01\\
58.51	0.01\\
58.52	0.01\\
58.53	0.01\\
58.54	0.01\\
58.55	0.01\\
58.56	0.01\\
58.57	0.01\\
58.58	0.01\\
58.59	0.01\\
58.6	0.01\\
58.61	0.01\\
58.62	0.01\\
58.63	0.01\\
58.64	0.01\\
58.65	0.01\\
58.66	0.01\\
58.67	0.01\\
58.68	0.01\\
58.69	0.01\\
58.7	0.01\\
58.71	0.01\\
58.72	0.01\\
58.73	0.01\\
58.74	0.01\\
58.75	0.01\\
58.76	0.01\\
58.77	0.01\\
58.78	0.01\\
58.79	0.01\\
58.8	0.01\\
58.81	0.01\\
58.82	0.01\\
58.83	0.01\\
58.84	0.01\\
58.85	0.01\\
58.86	0.01\\
58.87	0.01\\
58.88	0.01\\
58.89	0.01\\
58.9	0.01\\
58.91	0.01\\
58.92	0.01\\
58.93	0.01\\
58.94	0.01\\
58.95	0.01\\
58.96	0.01\\
58.97	0.01\\
58.98	0.01\\
58.99	0.01\\
59	0.01\\
59.01	0.01\\
59.02	0.01\\
59.03	0.01\\
59.04	0.01\\
59.05	0.01\\
59.06	0.01\\
59.07	0.01\\
59.08	0.01\\
59.09	0.01\\
59.1	0.01\\
59.11	0.01\\
59.12	0.01\\
59.13	0.01\\
59.14	0.01\\
59.15	0.01\\
59.16	0.01\\
59.17	0.01\\
59.18	0.01\\
59.19	0.01\\
59.2	0.01\\
59.21	0.01\\
59.22	0.01\\
59.23	0.01\\
59.24	0.01\\
59.25	0.01\\
59.26	0.01\\
59.27	0.01\\
59.28	0.01\\
59.29	0.01\\
59.3	0.01\\
59.31	0.01\\
59.32	0.01\\
59.33	0.01\\
59.34	0.01\\
59.35	0.01\\
59.36	0.01\\
59.37	0.01\\
59.38	0.01\\
59.39	0.01\\
59.4	0.01\\
59.41	0.01\\
59.42	0.01\\
59.43	0.01\\
59.44	0.01\\
59.45	0.01\\
59.46	0.01\\
59.47	0.01\\
59.48	0.01\\
59.49	0.01\\
59.5	0.01\\
59.51	0.01\\
59.52	0.01\\
59.53	0.01\\
59.54	0.01\\
59.55	0.01\\
59.56	0.01\\
59.57	0.01\\
59.58	0.01\\
59.59	0.01\\
59.6	0.01\\
59.61	0.01\\
59.62	0.01\\
59.63	0.01\\
59.64	0.01\\
59.65	0.01\\
59.66	0.01\\
59.67	0.01\\
59.68	0.01\\
59.69	0.01\\
59.7	0.01\\
59.71	0.01\\
59.72	0.01\\
59.73	0.01\\
59.74	0.01\\
59.75	0.01\\
59.76	0.01\\
59.77	0.01\\
59.78	0.01\\
59.79	0.01\\
59.8	0.01\\
59.81	0.01\\
59.82	0.01\\
59.83	0.01\\
59.84	0.01\\
59.85	0.01\\
59.86	0.01\\
59.87	0.01\\
59.88	0.01\\
59.89	0.01\\
59.9	0.01\\
59.91	0.01\\
59.92	0.01\\
59.93	0.01\\
59.94	0.01\\
59.95	0.01\\
59.96	0.01\\
59.97	0.01\\
59.98	0.01\\
59.99	0.01\\
60	0.01\\
60.01	0.01\\
60.02	0.01\\
60.03	0.01\\
60.04	0.01\\
60.05	0.01\\
60.06	0.01\\
60.07	0.01\\
60.08	0.01\\
60.09	0.01\\
60.1	0.01\\
60.11	0.01\\
60.12	0.01\\
60.13	0.01\\
60.14	0.01\\
60.15	0.01\\
60.16	0.01\\
60.17	0.01\\
60.18	0.01\\
60.19	0.01\\
60.2	0.01\\
60.21	0.01\\
60.22	0.01\\
60.23	0.01\\
60.24	0.01\\
60.25	0.01\\
60.26	0.01\\
60.27	0.01\\
60.28	0.01\\
60.29	0.01\\
60.3	0.01\\
60.31	0.01\\
60.32	0.01\\
60.33	0.01\\
60.34	0.01\\
60.35	0.01\\
60.36	0.01\\
60.37	0.01\\
60.38	0.01\\
60.39	0.01\\
60.4	0.01\\
60.41	0.01\\
60.42	0.01\\
60.43	0.01\\
60.44	0.01\\
60.45	0.01\\
60.46	0.01\\
60.47	0.01\\
60.48	0.01\\
60.49	0.01\\
60.5	0.01\\
60.51	0.01\\
60.52	0.01\\
60.53	0.01\\
60.54	0.01\\
60.55	0.01\\
60.56	0.01\\
60.57	0.01\\
60.58	0.01\\
60.59	0.01\\
60.6	0.01\\
60.61	0.01\\
60.62	0.01\\
60.63	0.01\\
60.64	0.01\\
60.65	0.01\\
60.66	0.01\\
60.67	0.01\\
60.68	0.01\\
60.69	0.01\\
60.7	0.01\\
60.71	0.01\\
60.72	0.01\\
60.73	0.01\\
60.74	0.01\\
60.75	0.01\\
60.76	0.01\\
60.77	0.01\\
60.78	0.01\\
60.79	0.01\\
60.8	0.01\\
60.81	0.01\\
60.82	0.01\\
60.83	0.01\\
60.84	0.01\\
60.85	0.01\\
60.86	0.01\\
60.87	0.01\\
60.88	0.01\\
60.89	0.01\\
60.9	0.01\\
60.91	0.01\\
60.92	0.01\\
60.93	0.01\\
60.94	0.01\\
60.95	0.01\\
60.96	0.01\\
60.97	0.01\\
60.98	0.01\\
60.99	0.01\\
61	0.01\\
61.01	0.01\\
61.02	0.01\\
61.03	0.01\\
61.04	0.01\\
61.05	0.01\\
61.06	0.01\\
61.07	0.01\\
61.08	0.01\\
61.09	0.01\\
61.1	0.01\\
61.11	0.01\\
61.12	0.01\\
61.13	0.01\\
61.14	0.01\\
61.15	0.01\\
61.16	0.01\\
61.17	0.01\\
61.18	0.01\\
61.19	0.01\\
61.2	0.01\\
61.21	0.01\\
61.22	0.01\\
61.23	0.01\\
61.24	0.01\\
61.25	0.01\\
61.26	0.01\\
61.27	0.01\\
61.28	0.01\\
61.29	0.01\\
61.3	0.01\\
61.31	0.01\\
61.32	0.01\\
61.33	0.01\\
61.34	0.01\\
61.35	0.01\\
61.36	0.01\\
61.37	0.01\\
61.38	0.01\\
61.39	0.01\\
61.4	0.01\\
61.41	0.01\\
61.42	0.01\\
61.43	0.01\\
61.44	0.01\\
61.45	0.01\\
61.46	0.01\\
61.47	0.01\\
61.48	0.01\\
61.49	0.01\\
61.5	0.01\\
61.51	0.01\\
61.52	0.01\\
61.53	0.01\\
61.54	0.01\\
61.55	0.01\\
61.56	0.01\\
61.57	0.01\\
61.58	0.01\\
61.59	0.01\\
61.6	0.01\\
61.61	0.01\\
61.62	0.01\\
61.63	0.01\\
61.64	0.01\\
61.65	0.01\\
61.66	0.01\\
61.67	0.01\\
61.68	0.01\\
61.69	0.01\\
61.7	0.01\\
61.71	0.01\\
61.72	0.01\\
61.73	0.01\\
61.74	0.01\\
61.75	0.01\\
61.76	0.01\\
61.77	0.01\\
61.78	0.01\\
61.79	0.01\\
61.8	0.01\\
61.81	0.01\\
61.82	0.01\\
61.83	0.01\\
61.84	0.01\\
61.85	0.01\\
61.86	0.01\\
61.87	0.01\\
61.88	0.01\\
61.89	0.01\\
61.9	0.01\\
61.91	0.01\\
61.92	0.01\\
61.93	0.01\\
61.94	0.01\\
61.95	0.01\\
61.96	0.01\\
61.97	0.01\\
61.98	0.01\\
61.99	0.01\\
62	0.01\\
62.01	0.01\\
62.02	0.01\\
62.03	0.01\\
62.04	0.01\\
62.05	0.01\\
62.06	0.01\\
62.07	0.01\\
62.08	0.01\\
62.09	0.01\\
62.1	0.01\\
62.11	0.01\\
62.12	0.01\\
62.13	0.01\\
62.14	0.01\\
62.15	0.01\\
62.16	0.01\\
62.17	0.01\\
62.18	0.01\\
62.19	0.01\\
62.2	0.01\\
62.21	0.01\\
62.22	0.01\\
62.23	0.01\\
62.24	0.01\\
62.25	0.01\\
62.26	0.01\\
62.27	0.01\\
62.28	0.01\\
62.29	0.01\\
62.3	0.01\\
62.31	0.01\\
62.32	0.01\\
62.33	0.01\\
62.34	0.01\\
62.35	0.01\\
62.36	0.01\\
62.37	0.01\\
62.38	0.01\\
62.39	0.01\\
62.4	0.01\\
62.41	0.01\\
62.42	0.01\\
62.43	0.01\\
62.44	0.01\\
62.45	0.01\\
62.46	0.01\\
62.47	0.01\\
62.48	0.01\\
62.49	0.01\\
62.5	0.01\\
62.51	0.01\\
62.52	0.01\\
62.53	0.01\\
62.54	0.01\\
62.55	0.01\\
62.56	0.01\\
62.57	0.01\\
62.58	0.01\\
62.59	0.01\\
62.6	0.01\\
62.61	0.01\\
62.62	0.01\\
62.63	0.01\\
62.64	0.01\\
62.65	0.01\\
62.66	0.01\\
62.67	0.01\\
62.68	0.01\\
62.69	0.01\\
62.7	0.01\\
62.71	0.01\\
62.72	0.01\\
62.73	0.01\\
62.74	0.01\\
62.75	0.01\\
62.76	0.01\\
62.77	0.01\\
62.78	0.01\\
62.79	0.01\\
62.8	0.01\\
62.81	0.01\\
62.82	0.01\\
62.83	0.01\\
62.84	0.01\\
62.85	0.01\\
62.86	0.01\\
62.87	0.01\\
62.88	0.01\\
62.89	0.01\\
62.9	0.01\\
62.91	0.01\\
62.92	0.01\\
62.93	0.01\\
62.94	0.01\\
62.95	0.01\\
62.96	0.01\\
62.97	0.01\\
62.98	0.01\\
62.99	0.01\\
63	0.01\\
63.01	0.01\\
63.02	0.01\\
63.03	0.01\\
63.04	0.01\\
63.05	0.01\\
63.06	0.01\\
63.07	0.01\\
63.08	0.01\\
63.09	0.01\\
63.1	0.01\\
63.11	0.01\\
63.12	0.01\\
63.13	0.01\\
63.14	0.01\\
63.15	0.01\\
63.16	0.01\\
63.17	0.01\\
63.18	0.01\\
63.19	0.01\\
63.2	0.01\\
63.21	0.01\\
63.22	0.01\\
63.23	0.01\\
63.24	0.01\\
63.25	0.01\\
63.26	0.01\\
63.27	0.01\\
63.28	0.01\\
63.29	0.01\\
63.3	0.01\\
63.31	0.01\\
63.32	0.01\\
63.33	0.01\\
63.34	0.01\\
63.35	0.01\\
63.36	0.01\\
63.37	0.01\\
63.38	0.01\\
63.39	0.01\\
63.4	0.01\\
63.41	0.01\\
63.42	0.01\\
63.43	0.01\\
63.44	0.01\\
63.45	0.01\\
63.46	0.01\\
63.47	0.01\\
63.48	0.01\\
63.49	0.01\\
63.5	0.01\\
63.51	0.01\\
63.52	0.01\\
63.53	0.01\\
63.54	0.01\\
63.55	0.01\\
63.56	0.01\\
63.57	0.01\\
63.58	0.01\\
63.59	0.01\\
63.6	0.01\\
63.61	0.01\\
63.62	0.01\\
63.63	0.01\\
63.64	0.01\\
63.65	0.01\\
63.66	0.01\\
63.67	0.01\\
63.68	0.01\\
63.69	0.01\\
63.7	0.01\\
63.71	0.01\\
63.72	0.01\\
63.73	0.01\\
63.74	0.01\\
63.75	0.01\\
63.76	0.01\\
63.77	0.01\\
63.78	0.01\\
63.79	0.01\\
63.8	0.01\\
63.81	0.01\\
63.82	0.01\\
63.83	0.01\\
63.84	0.01\\
63.85	0.01\\
63.86	0.01\\
63.87	0.01\\
63.88	0.01\\
63.89	0.01\\
63.9	0.01\\
63.91	0.01\\
63.92	0.01\\
63.93	0.01\\
63.94	0.01\\
63.95	0.01\\
63.96	0.01\\
63.97	0.01\\
63.98	0.01\\
63.99	0.01\\
64	0.01\\
64.01	0.01\\
64.02	0.01\\
64.03	0.01\\
64.04	0.01\\
64.05	0.01\\
64.06	0.01\\
64.07	0.01\\
64.08	0.01\\
64.09	0.01\\
64.1	0.01\\
64.11	0.01\\
64.12	0.01\\
64.13	0.01\\
64.14	0.01\\
64.15	0.01\\
64.16	0.01\\
64.17	0.01\\
64.18	0.01\\
64.19	0.01\\
64.2	0.01\\
64.21	0.01\\
64.22	0.01\\
64.23	0.01\\
64.24	0.01\\
64.25	0.01\\
64.26	0.01\\
64.27	0.01\\
64.28	0.01\\
64.29	0.01\\
64.3	0.01\\
64.31	0.01\\
64.32	0.01\\
64.33	0.01\\
64.34	0.01\\
64.35	0.01\\
64.36	0.01\\
64.37	0.01\\
64.38	0.01\\
64.39	0.01\\
64.4	0.01\\
64.41	0.01\\
64.42	0.01\\
64.43	0.01\\
64.44	0.01\\
64.45	0.01\\
64.46	0.01\\
64.47	0.01\\
64.48	0.01\\
64.49	0.01\\
64.5	0.01\\
64.51	0.01\\
64.52	0.01\\
64.53	0.01\\
64.54	0.01\\
64.55	0.01\\
64.56	0.01\\
64.57	0.01\\
64.58	0.01\\
64.59	0.01\\
64.6	0.01\\
64.61	0.01\\
64.62	0.01\\
64.63	0.01\\
64.64	0.01\\
64.65	0.01\\
64.66	0.01\\
64.67	0.01\\
64.68	0.01\\
64.69	0.01\\
64.7	0.01\\
64.71	0.01\\
64.72	0.01\\
64.73	0.01\\
64.74	0.01\\
64.75	0.01\\
64.76	0.01\\
64.77	0.01\\
64.78	0.01\\
64.79	0.01\\
64.8	0.01\\
64.81	0.01\\
64.82	0.01\\
64.83	0.01\\
64.84	0.01\\
64.85	0.01\\
64.86	0.01\\
64.87	0.01\\
64.88	0.01\\
64.89	0.01\\
64.9	0.01\\
64.91	0.01\\
64.92	0.01\\
64.93	0.01\\
64.94	0.01\\
64.95	0.01\\
64.96	0.01\\
64.97	0.01\\
64.98	0.01\\
64.99	0.01\\
65	0.01\\
65.01	0.01\\
65.02	0.01\\
65.03	0.01\\
65.04	0.01\\
65.05	0.01\\
65.06	0.01\\
65.07	0.01\\
65.08	0.01\\
65.09	0.01\\
65.1	0.01\\
65.11	0.01\\
65.12	0.01\\
65.13	0.01\\
65.14	0.01\\
65.15	0.01\\
65.16	0.01\\
65.17	0.01\\
65.18	0.01\\
65.19	0.01\\
65.2	0.01\\
65.21	0.01\\
65.22	0.01\\
65.23	0.01\\
65.24	0.01\\
65.25	0.01\\
65.26	0.01\\
65.27	0.01\\
65.28	0.01\\
65.29	0.01\\
65.3	0.01\\
65.31	0.01\\
65.32	0.01\\
65.33	0.01\\
65.34	0.01\\
65.35	0.01\\
65.36	0.01\\
65.37	0.01\\
65.38	0.01\\
65.39	0.01\\
65.4	0.01\\
65.41	0.01\\
65.42	0.01\\
65.43	0.01\\
65.44	0.01\\
65.45	0.01\\
65.46	0.01\\
65.47	0.01\\
65.48	0.01\\
65.49	0.01\\
65.5	0.01\\
65.51	0.01\\
65.52	0.01\\
65.53	0.01\\
65.54	0.01\\
65.55	0.01\\
65.56	0.01\\
65.57	0.01\\
65.58	0.01\\
65.59	0.01\\
65.6	0.01\\
65.61	0.01\\
65.62	0.01\\
65.63	0.01\\
65.64	0.01\\
65.65	0.01\\
65.66	0.01\\
65.67	0.01\\
65.68	0.01\\
65.69	0.01\\
65.7	0.01\\
65.71	0.01\\
65.72	0.01\\
65.73	0.01\\
65.74	0.01\\
65.75	0.01\\
65.76	0.01\\
65.77	0.01\\
65.78	0.01\\
65.79	0.01\\
65.8	0.01\\
65.81	0.01\\
65.82	0.01\\
65.83	0.01\\
65.84	0.01\\
65.85	0.01\\
65.86	0.01\\
65.87	0.01\\
65.88	0.01\\
65.89	0.01\\
65.9	0.01\\
65.91	0.01\\
65.92	0.01\\
65.93	0.01\\
65.94	0.01\\
65.95	0.01\\
65.96	0.01\\
65.97	0.01\\
65.98	0.01\\
65.99	0.01\\
66	0.01\\
66.01	0.01\\
66.02	0.01\\
66.03	0.01\\
66.04	0.01\\
66.05	0.01\\
66.06	0.01\\
66.07	0.01\\
66.08	0.01\\
66.09	0.01\\
66.1	0.01\\
66.11	0.01\\
66.12	0.01\\
66.13	0.01\\
66.14	0.01\\
66.15	0.01\\
66.16	0.01\\
66.17	0.01\\
66.18	0.01\\
66.19	0.01\\
66.2	0.01\\
66.21	0.01\\
66.22	0.01\\
66.23	0.01\\
66.24	0.01\\
66.25	0.01\\
66.26	0.01\\
66.27	0.01\\
66.28	0.01\\
66.29	0.01\\
66.3	0.01\\
66.31	0.01\\
66.32	0.01\\
66.33	0.01\\
66.34	0.01\\
66.35	0.01\\
66.36	0.01\\
66.37	0.01\\
66.38	0.01\\
66.39	0.01\\
66.4	0.01\\
66.41	0.01\\
66.42	0.01\\
66.43	0.01\\
66.44	0.01\\
66.45	0.01\\
66.46	0.01\\
66.47	0.01\\
66.48	0.01\\
66.49	0.01\\
66.5	0.01\\
66.51	0.01\\
66.52	0.01\\
66.53	0.01\\
66.54	0.01\\
66.55	0.01\\
66.56	0.01\\
66.57	0.01\\
66.58	0.01\\
66.59	0.01\\
66.6	0.01\\
66.61	0.01\\
66.62	0.01\\
66.63	0.01\\
66.64	0.01\\
66.65	0.01\\
66.66	0.01\\
66.67	0.01\\
66.68	0.01\\
66.69	0.01\\
66.7	0.01\\
66.71	0.01\\
66.72	0.01\\
66.73	0.01\\
66.74	0.01\\
66.75	0.01\\
66.76	0.01\\
66.77	0.01\\
66.78	0.01\\
66.79	0.01\\
66.8	0.01\\
66.81	0.01\\
66.82	0.01\\
66.83	0.01\\
66.84	0.01\\
66.85	0.01\\
66.86	0.01\\
66.87	0.01\\
66.88	0.01\\
66.89	0.01\\
66.9	0.01\\
66.91	0.01\\
66.92	0.01\\
66.93	0.01\\
66.94	0.01\\
66.95	0.01\\
66.96	0.01\\
66.97	0.01\\
66.98	0.01\\
66.99	0.01\\
67	0.01\\
67.01	0.01\\
67.02	0.01\\
67.03	0.01\\
67.04	0.01\\
67.05	0.01\\
67.06	0.01\\
67.07	0.01\\
67.08	0.01\\
67.09	0.01\\
67.1	0.01\\
67.11	0.01\\
67.12	0.01\\
67.13	0.01\\
67.14	0.01\\
67.15	0.01\\
67.16	0.01\\
67.17	0.01\\
67.18	0.01\\
67.19	0.01\\
67.2	0.01\\
67.21	0.01\\
67.22	0.01\\
67.23	0.01\\
67.24	0.01\\
67.25	0.01\\
67.26	0.01\\
67.27	0.01\\
67.28	0.01\\
67.29	0.01\\
67.3	0.01\\
67.31	0.01\\
67.32	0.01\\
67.33	0.01\\
67.34	0.01\\
67.35	0.01\\
67.36	0.01\\
67.37	0.01\\
67.38	0.01\\
67.39	0.01\\
67.4	0.01\\
67.41	0.01\\
67.42	0.01\\
67.43	0.01\\
67.44	0.01\\
67.45	0.01\\
67.46	0.01\\
67.47	0.01\\
67.48	0.01\\
67.49	0.01\\
67.5	0.01\\
67.51	0.01\\
67.52	0.01\\
67.53	0.01\\
67.54	0.01\\
67.55	0.01\\
67.56	0.01\\
67.57	0.01\\
67.58	0.01\\
67.59	0.01\\
67.6	0.01\\
67.61	0.01\\
67.62	0.01\\
67.63	0.01\\
67.64	0.01\\
67.65	0.01\\
67.66	0.01\\
67.67	0.01\\
67.68	0.01\\
67.69	0.01\\
67.7	0.01\\
67.71	0.01\\
67.72	0.01\\
67.73	0.01\\
67.74	0.01\\
67.75	0.01\\
67.76	0.01\\
67.77	0.01\\
67.78	0.01\\
67.79	0.01\\
67.8	0.01\\
67.81	0.01\\
67.82	0.01\\
67.83	0.01\\
67.84	0.01\\
67.85	0.01\\
67.86	0.01\\
67.87	0.01\\
67.88	0.01\\
67.89	0.01\\
67.9	0.01\\
67.91	0.01\\
67.92	0.01\\
67.93	0.01\\
67.94	0.01\\
67.95	0.01\\
67.96	0.01\\
67.97	0.01\\
67.98	0.01\\
67.99	0.01\\
68	0.01\\
68.01	0.01\\
68.02	0.01\\
68.03	0.01\\
68.04	0.01\\
68.05	0.01\\
68.06	0.01\\
68.07	0.01\\
68.08	0.01\\
68.09	0.01\\
68.1	0.01\\
68.11	0.01\\
68.12	0.01\\
68.13	0.01\\
68.14	0.01\\
68.15	0.01\\
68.16	0.01\\
68.17	0.01\\
68.18	0.01\\
68.19	0.01\\
68.2	0.01\\
68.21	0.01\\
68.22	0.01\\
68.23	0.01\\
68.24	0.01\\
68.25	0.01\\
68.26	0.01\\
68.27	0.01\\
68.28	0.01\\
68.29	0.01\\
68.3	0.01\\
68.31	0.01\\
68.32	0.01\\
68.33	0.01\\
68.34	0.01\\
68.35	0.01\\
68.36	0.01\\
68.37	0.01\\
68.38	0.01\\
68.39	0.01\\
68.4	0.01\\
68.41	0.01\\
68.42	0.01\\
68.43	0.01\\
68.44	0.01\\
68.45	0.01\\
68.46	0.01\\
68.47	0.01\\
68.48	0.01\\
68.49	0.01\\
68.5	0.01\\
68.51	0.01\\
68.52	0.01\\
68.53	0.01\\
68.54	0.01\\
68.55	0.01\\
68.56	0.01\\
68.57	0.01\\
68.58	0.01\\
68.59	0.01\\
68.6	0.01\\
68.61	0.01\\
68.62	0.01\\
68.63	0.01\\
68.64	0.01\\
68.65	0.01\\
68.66	0.01\\
68.67	0.01\\
68.68	0.01\\
68.69	0.01\\
68.7	0.01\\
68.71	0.01\\
68.72	0.01\\
68.73	0.01\\
68.74	0.01\\
68.75	0.01\\
68.76	0.01\\
68.77	0.01\\
68.78	0.01\\
68.79	0.01\\
68.8	0.01\\
68.81	0.01\\
68.82	0.01\\
68.83	0.01\\
68.84	0.01\\
68.85	0.01\\
68.86	0.01\\
68.87	0.01\\
68.88	0.01\\
68.89	0.01\\
68.9	0.01\\
68.91	0.01\\
68.92	0.01\\
68.93	0.01\\
68.94	0.01\\
68.95	0.01\\
68.96	0.01\\
68.97	0.01\\
68.98	0.01\\
68.99	0.01\\
69	0.01\\
69.01	0.01\\
69.02	0.01\\
69.03	0.01\\
69.04	0.01\\
69.05	0.01\\
69.06	0.01\\
69.07	0.01\\
69.08	0.01\\
69.09	0.01\\
69.1	0.01\\
69.11	0.01\\
69.12	0.01\\
69.13	0.01\\
69.14	0.01\\
69.15	0.01\\
69.16	0.01\\
69.17	0.01\\
69.18	0.01\\
69.19	0.01\\
69.2	0.01\\
69.21	0.01\\
69.22	0.01\\
69.23	0.01\\
69.24	0.01\\
69.25	0.01\\
69.26	0.01\\
69.27	0.01\\
69.28	0.01\\
69.29	0.01\\
69.3	0.01\\
69.31	0.01\\
69.32	0.01\\
69.33	0.01\\
69.34	0.01\\
69.35	0.01\\
69.36	0.01\\
69.37	0.01\\
69.38	0.01\\
69.39	0.01\\
69.4	0.01\\
69.41	0.01\\
69.42	0.01\\
69.43	0.01\\
69.44	0.01\\
69.45	0.01\\
69.46	0.01\\
69.47	0.01\\
69.48	0.01\\
69.49	0.01\\
69.5	0.01\\
69.51	0.01\\
69.52	0.01\\
69.53	0.01\\
69.54	0.01\\
69.55	0.01\\
69.56	0.01\\
69.57	0.01\\
69.58	0.01\\
69.59	0.01\\
69.6	0.01\\
69.61	0.01\\
69.62	0.01\\
69.63	0.01\\
69.64	0.01\\
69.65	0.01\\
69.66	0.01\\
69.67	0.01\\
69.68	0.01\\
69.69	0.01\\
69.7	0.01\\
69.71	0.01\\
69.72	0.01\\
69.73	0.01\\
69.74	0.01\\
69.75	0.01\\
69.76	0.01\\
69.77	0.01\\
69.78	0.01\\
69.79	0.01\\
69.8	0.01\\
69.81	0.01\\
69.82	0.01\\
69.83	0.01\\
69.84	0.01\\
69.85	0.01\\
69.86	0.01\\
69.87	0.01\\
69.88	0.01\\
69.89	0.01\\
69.9	0.01\\
69.91	0.01\\
69.92	0.01\\
69.93	0.01\\
69.94	0.01\\
69.95	0.01\\
69.96	0.01\\
69.97	0.01\\
69.98	0.01\\
69.99	0.01\\
70	0.01\\
70.01	0.01\\
70.02	0.01\\
70.03	0.01\\
70.04	0.01\\
70.05	0.01\\
70.06	0.01\\
70.07	0.01\\
70.08	0.01\\
70.09	0.01\\
70.1	0.01\\
70.11	0.01\\
70.12	0.01\\
70.13	0.01\\
70.14	0.01\\
70.15	0.01\\
70.16	0.01\\
70.17	0.01\\
70.18	0.01\\
70.19	0.01\\
70.2	0.01\\
70.21	0.01\\
70.22	0.01\\
70.23	0.01\\
70.24	0.01\\
70.25	0.01\\
70.26	0.01\\
70.27	0.01\\
70.28	0.01\\
70.29	0.01\\
70.3	0.01\\
70.31	0.01\\
70.32	0.01\\
70.33	0.01\\
70.34	0.01\\
70.35	0.01\\
70.36	0.01\\
70.37	0.01\\
70.38	0.01\\
70.39	0.01\\
70.4	0.01\\
70.41	0.01\\
70.42	0.01\\
70.43	0.01\\
70.44	0.01\\
70.45	0.01\\
70.46	0.01\\
70.47	0.01\\
70.48	0.01\\
70.49	0.01\\
70.5	0.01\\
70.51	0.01\\
70.52	0.01\\
70.53	0.01\\
70.54	0.01\\
70.55	0.01\\
70.56	0.01\\
70.57	0.01\\
70.58	0.01\\
70.59	0.01\\
70.6	0.01\\
70.61	0.01\\
70.62	0.01\\
70.63	0.01\\
70.64	0.01\\
70.65	0.01\\
70.66	0.01\\
70.67	0.01\\
70.68	0.01\\
70.69	0.01\\
70.7	0.01\\
70.71	0.01\\
70.72	0.01\\
70.73	0.01\\
70.74	0.01\\
70.75	0.01\\
70.76	0.01\\
70.77	0.01\\
70.78	0.01\\
70.79	0.01\\
70.8	0.01\\
70.81	0.01\\
70.82	0.01\\
70.83	0.01\\
70.84	0.01\\
70.85	0.01\\
70.86	0.01\\
70.87	0.01\\
70.88	0.01\\
70.89	0.01\\
70.9	0.01\\
70.91	0.01\\
70.92	0.01\\
70.93	0.01\\
70.94	0.01\\
70.95	0.01\\
70.96	0.01\\
70.97	0.01\\
70.98	0.01\\
70.99	0.01\\
71	0.01\\
71.01	0.01\\
71.02	0.01\\
71.03	0.01\\
71.04	0.01\\
71.05	0.01\\
71.06	0.01\\
71.07	0.01\\
71.08	0.01\\
71.09	0.01\\
71.1	0.01\\
71.11	0.01\\
71.12	0.01\\
71.13	0.01\\
71.14	0.01\\
71.15	0.01\\
71.16	0.01\\
71.17	0.01\\
71.18	0.01\\
71.19	0.01\\
71.2	0.01\\
71.21	0.01\\
71.22	0.01\\
71.23	0.01\\
71.24	0.01\\
71.25	0.01\\
71.26	0.01\\
71.27	0.01\\
71.28	0.01\\
71.29	0.01\\
71.3	0.01\\
71.31	0.01\\
71.32	0.01\\
71.33	0.01\\
71.34	0.01\\
71.35	0.01\\
71.36	0.01\\
71.37	0.01\\
71.38	0.01\\
71.39	0.01\\
71.4	0.01\\
71.41	0.01\\
71.42	0.01\\
71.43	0.01\\
71.44	0.01\\
71.45	0.01\\
71.46	0.01\\
71.47	0.01\\
71.48	0.01\\
71.49	0.01\\
71.5	0.01\\
71.51	0.01\\
71.52	0.01\\
71.53	0.01\\
71.54	0.01\\
71.55	0.01\\
71.56	0.01\\
71.57	0.01\\
71.58	0.01\\
71.59	0.01\\
71.6	0.01\\
71.61	0.01\\
71.62	0.01\\
71.63	0.01\\
71.64	0.01\\
71.65	0.01\\
71.66	0.01\\
71.67	0.01\\
71.68	0.01\\
71.69	0.01\\
71.7	0.01\\
71.71	0.01\\
71.72	0.01\\
71.73	0.01\\
71.74	0.01\\
71.75	0.01\\
71.76	0.01\\
71.77	0.01\\
71.78	0.01\\
71.79	0.01\\
71.8	0.01\\
71.81	0.01\\
71.82	0.01\\
71.83	0.01\\
71.84	0.01\\
71.85	0.01\\
71.86	0.01\\
71.87	0.01\\
71.88	0.01\\
71.89	0.01\\
71.9	0.01\\
71.91	0.01\\
71.92	0.01\\
71.93	0.01\\
71.94	0.01\\
71.95	0.01\\
71.96	0.01\\
71.97	0.01\\
71.98	0.01\\
71.99	0.01\\
72	0.01\\
72.01	0.01\\
72.02	0.01\\
72.03	0.01\\
72.04	0.01\\
72.05	0.01\\
72.06	0.01\\
72.07	0.01\\
72.08	0.01\\
72.09	0.01\\
72.1	0.01\\
72.11	0.01\\
72.12	0.01\\
72.13	0.01\\
72.14	0.01\\
72.15	0.01\\
72.16	0.01\\
72.17	0.01\\
72.18	0.01\\
72.19	0.01\\
72.2	0.01\\
72.21	0.01\\
72.22	0.01\\
72.23	0.01\\
72.24	0.01\\
72.25	0.01\\
72.26	0.01\\
72.27	0.01\\
72.28	0.01\\
72.29	0.01\\
72.3	0.01\\
72.31	0.01\\
72.32	0.01\\
72.33	0.01\\
72.34	0.01\\
72.35	0.01\\
72.36	0.01\\
72.37	0.01\\
72.38	0.01\\
72.39	0.01\\
72.4	0.01\\
72.41	0.01\\
72.42	0.01\\
72.43	0.01\\
72.44	0.01\\
72.45	0.01\\
72.46	0.01\\
72.47	0.01\\
72.48	0.01\\
72.49	0.01\\
72.5	0.01\\
72.51	0.01\\
72.52	0.01\\
72.53	0.01\\
72.54	0.01\\
72.55	0.01\\
72.56	0.01\\
72.57	0.01\\
72.58	0.01\\
72.59	0.01\\
72.6	0.01\\
72.61	0.01\\
72.62	0.01\\
72.63	0.01\\
72.64	0.01\\
72.65	0.01\\
72.66	0.01\\
72.67	0.01\\
72.68	0.01\\
72.69	0.01\\
72.7	0.01\\
72.71	0.01\\
72.72	0.01\\
72.73	0.01\\
72.74	0.01\\
72.75	0.01\\
72.76	0.01\\
72.77	0.01\\
72.78	0.01\\
72.79	0.01\\
72.8	0.01\\
72.81	0.01\\
72.82	0.01\\
72.83	0.01\\
72.84	0.01\\
72.85	0.01\\
72.86	0.01\\
72.87	0.01\\
72.88	0.01\\
72.89	0.01\\
72.9	0.01\\
72.91	0.01\\
72.92	0.01\\
72.93	0.01\\
72.94	0.01\\
72.95	0.01\\
72.96	0.01\\
72.97	0.01\\
72.98	0.01\\
72.99	0.01\\
73	0.01\\
73.01	0.01\\
73.02	0.01\\
73.03	0.01\\
73.04	0.01\\
73.05	0.01\\
73.06	0.01\\
73.07	0.01\\
73.08	0.01\\
73.09	0.01\\
73.1	0.01\\
73.11	0.01\\
73.12	0.01\\
73.13	0.01\\
73.14	0.01\\
73.15	0.01\\
73.16	0.01\\
73.17	0.01\\
73.18	0.01\\
73.19	0.01\\
73.2	0.01\\
73.21	0.01\\
73.22	0.01\\
73.23	0.01\\
73.24	0.01\\
73.25	0.01\\
73.26	0.01\\
73.27	0.01\\
73.28	0.01\\
73.29	0.01\\
73.3	0.01\\
73.31	0.01\\
73.32	0.01\\
73.33	0.01\\
73.34	0.01\\
73.35	0.01\\
73.36	0.01\\
73.37	0.01\\
73.38	0.01\\
73.39	0.01\\
73.4	0.01\\
73.41	0.01\\
73.42	0.01\\
73.43	0.01\\
73.44	0.01\\
73.45	0.01\\
73.46	0.01\\
73.47	0.01\\
73.48	0.01\\
73.49	0.01\\
73.5	0.01\\
73.51	0.01\\
73.52	0.01\\
73.53	0.01\\
73.54	0.01\\
73.55	0.01\\
73.56	0.01\\
73.57	0.01\\
73.58	0.01\\
73.59	0.01\\
73.6	0.01\\
73.61	0.01\\
73.62	0.01\\
73.63	0.01\\
73.64	0.01\\
73.65	0.01\\
73.66	0.01\\
73.67	0.01\\
73.68	0.01\\
73.69	0.01\\
73.7	0.01\\
73.71	0.01\\
73.72	0.01\\
73.73	0.01\\
73.74	0.01\\
73.75	0.01\\
73.76	0.01\\
73.77	0.01\\
73.78	0.01\\
73.79	0.01\\
73.8	0.01\\
73.81	0.01\\
73.82	0.01\\
73.83	0.01\\
73.84	0.01\\
73.85	0.01\\
73.86	0.01\\
73.87	0.01\\
73.88	0.01\\
73.89	0.01\\
73.9	0.01\\
73.91	0.01\\
73.92	0.01\\
73.93	0.01\\
73.94	0.01\\
73.95	0.01\\
73.96	0.01\\
73.97	0.01\\
73.98	0.01\\
73.99	0.01\\
74	0.01\\
74.01	0.01\\
74.02	0.01\\
74.03	0.01\\
74.04	0.01\\
74.05	0.01\\
74.06	0.01\\
74.07	0.01\\
74.08	0.01\\
74.09	0.01\\
74.1	0.01\\
74.11	0.01\\
74.12	0.01\\
74.13	0.01\\
74.14	0.01\\
74.15	0.01\\
74.16	0.01\\
74.17	0.01\\
74.18	0.01\\
74.19	0.01\\
74.2	0.01\\
74.21	0.01\\
74.22	0.01\\
74.23	0.01\\
74.24	0.01\\
74.25	0.01\\
74.26	0.01\\
74.27	0.01\\
74.28	0.01\\
74.29	0.01\\
74.3	0.01\\
74.31	0.01\\
74.32	0.01\\
74.33	0.01\\
74.34	0.01\\
74.35	0.01\\
74.36	0.01\\
74.37	0.01\\
74.38	0.01\\
74.39	0.01\\
74.4	0.01\\
74.41	0.01\\
74.42	0.01\\
74.43	0.01\\
74.44	0.01\\
74.45	0.01\\
74.46	0.01\\
74.47	0.01\\
74.48	0.01\\
74.49	0.01\\
74.5	0.01\\
74.51	0.01\\
74.52	0.01\\
74.53	0.01\\
74.54	0.01\\
74.55	0.01\\
74.56	0.01\\
74.57	0.01\\
74.58	0.01\\
74.59	0.01\\
74.6	0.01\\
74.61	0.01\\
74.62	0.01\\
74.63	0.01\\
74.64	0.01\\
74.65	0.01\\
74.66	0.01\\
74.67	0.01\\
74.68	0.01\\
74.69	0.01\\
74.7	0.01\\
74.71	0.01\\
74.72	0.01\\
74.73	0.01\\
74.74	0.01\\
74.75	0.01\\
74.76	0.01\\
74.77	0.01\\
74.78	0.01\\
74.79	0.01\\
74.8	0.01\\
74.81	0.01\\
74.82	0.01\\
74.83	0.01\\
74.84	0.01\\
74.85	0.01\\
74.86	0.01\\
74.87	0.01\\
74.88	0.01\\
74.89	0.01\\
74.9	0.01\\
74.91	0.01\\
74.92	0.01\\
74.93	0.01\\
74.94	0.01\\
74.95	0.01\\
74.96	0.01\\
74.97	0.01\\
74.98	0.01\\
74.99	0.01\\
75	0.01\\
75.01	0.01\\
75.02	0.01\\
75.03	0.01\\
75.04	0.01\\
75.05	0.01\\
75.06	0.01\\
75.07	0.01\\
75.08	0.01\\
75.09	0.01\\
75.1	0.01\\
75.11	0.01\\
75.12	0.01\\
75.13	0.01\\
75.14	0.01\\
75.15	0.01\\
75.16	0.01\\
75.17	0.01\\
75.18	0.01\\
75.19	0.01\\
75.2	0.01\\
75.21	0.01\\
75.22	0.01\\
75.23	0.01\\
75.24	0.01\\
75.25	0.01\\
75.26	0.01\\
75.27	0.01\\
75.28	0.01\\
75.29	0.01\\
75.3	0.01\\
75.31	0.01\\
75.32	0.01\\
75.33	0.01\\
75.34	0.01\\
75.35	0.01\\
75.36	0.01\\
75.37	0.01\\
75.38	0.01\\
75.39	0.01\\
75.4	0.01\\
75.41	0.01\\
75.42	0.01\\
75.43	0.01\\
75.44	0.01\\
75.45	0.01\\
75.46	0.01\\
75.47	0.01\\
75.48	0.01\\
75.49	0.01\\
75.5	0.01\\
75.51	0.01\\
75.52	0.01\\
75.53	0.01\\
75.54	0.01\\
75.55	0.01\\
75.56	0.01\\
75.57	0.01\\
75.58	0.01\\
75.59	0.01\\
75.6	0.01\\
75.61	0.01\\
75.62	0.01\\
75.63	0.01\\
75.64	0.01\\
75.65	0.01\\
75.66	0.01\\
75.67	0.01\\
75.68	0.01\\
75.69	0.01\\
75.7	0.01\\
75.71	0.01\\
75.72	0.01\\
75.73	0.01\\
75.74	0.01\\
75.75	0.01\\
75.76	0.01\\
75.77	0.01\\
75.78	0.01\\
75.79	0.01\\
75.8	0.01\\
75.81	0.01\\
75.82	0.01\\
75.83	0.01\\
75.84	0.01\\
75.85	0.01\\
75.86	0.01\\
75.87	0.01\\
75.88	0.01\\
75.89	0.01\\
75.9	0.01\\
75.91	0.01\\
75.92	0.01\\
75.93	0.01\\
75.94	0.01\\
75.95	0.01\\
75.96	0.01\\
75.97	0.01\\
75.98	0.01\\
75.99	0.01\\
76	0.01\\
76.01	0.01\\
76.02	0.01\\
76.03	0.01\\
76.04	0.01\\
76.05	0.01\\
76.06	0.01\\
76.07	0.01\\
76.08	0.01\\
76.09	0.01\\
76.1	0.01\\
76.11	0.01\\
76.12	0.01\\
76.13	0.01\\
76.14	0.01\\
76.15	0.01\\
76.16	0.01\\
76.17	0.01\\
76.18	0.01\\
76.19	0.01\\
76.2	0.01\\
76.21	0.01\\
76.22	0.01\\
76.23	0.01\\
76.24	0.01\\
76.25	0.01\\
76.26	0.01\\
76.27	0.01\\
76.28	0.01\\
76.29	0.01\\
76.3	0.01\\
76.31	0.01\\
76.32	0.01\\
76.33	0.01\\
76.34	0.01\\
76.35	0.01\\
76.36	0.01\\
76.37	0.01\\
76.38	0.01\\
76.39	0.01\\
76.4	0.01\\
76.41	0.01\\
76.42	0.01\\
76.43	0.01\\
76.44	0.01\\
76.45	0.01\\
76.46	0.01\\
76.47	0.01\\
76.48	0.01\\
76.49	0.01\\
76.5	0.01\\
76.51	0.01\\
76.52	0.01\\
76.53	0.01\\
76.54	0.01\\
76.55	0.01\\
76.56	0.01\\
76.57	0.01\\
76.58	0.01\\
76.59	0.01\\
76.6	0.01\\
76.61	0.01\\
76.62	0.01\\
76.63	0.01\\
76.64	0.01\\
76.65	0.01\\
76.66	0.01\\
76.67	0.01\\
76.68	0.01\\
76.69	0.01\\
76.7	0.01\\
76.71	0.01\\
76.72	0.01\\
76.73	0.01\\
76.74	0.01\\
76.75	0.01\\
76.76	0.01\\
76.77	0.01\\
76.78	0.01\\
76.79	0.01\\
76.8	0.01\\
76.81	0.01\\
76.82	0.01\\
76.83	0.01\\
76.84	0.01\\
76.85	0.01\\
76.86	0.01\\
76.87	0.01\\
76.88	0.01\\
76.89	0.01\\
76.9	0.01\\
76.91	0.01\\
76.92	0.01\\
76.93	0.01\\
76.94	0.01\\
76.95	0.01\\
76.96	0.01\\
76.97	0.01\\
76.98	0.01\\
76.99	0.01\\
77	0.01\\
77.01	0.01\\
77.02	0.01\\
77.03	0.01\\
77.04	0.01\\
77.05	0.01\\
77.06	0.01\\
77.07	0.01\\
77.08	0.01\\
77.09	0.01\\
77.1	0.01\\
77.11	0.01\\
77.12	0.01\\
77.13	0.01\\
77.14	0.01\\
77.15	0.01\\
77.16	0.01\\
77.17	0.01\\
77.18	0.01\\
77.19	0.01\\
77.2	0.01\\
77.21	0.01\\
77.22	0.01\\
77.23	0.01\\
77.24	0.01\\
77.25	0.01\\
77.26	0.01\\
77.27	0.01\\
77.28	0.01\\
77.29	0.01\\
77.3	0.01\\
77.31	0.01\\
77.32	0.01\\
77.33	0.01\\
77.34	0.01\\
77.35	0.01\\
77.36	0.01\\
77.37	0.01\\
77.38	0.01\\
77.39	0.01\\
77.4	0.01\\
77.41	0.01\\
77.42	0.01\\
77.43	0.01\\
77.44	0.01\\
77.45	0.01\\
77.46	0.01\\
77.47	0.01\\
77.48	0.01\\
77.49	0.01\\
77.5	0.01\\
77.51	0.01\\
77.52	0.01\\
77.53	0.01\\
77.54	0.01\\
77.55	0.01\\
77.56	0.01\\
77.57	0.01\\
77.58	0.01\\
77.59	0.01\\
77.6	0.01\\
77.61	0.01\\
77.62	0.01\\
77.63	0.01\\
77.64	0.01\\
77.65	0.01\\
77.66	0.01\\
77.67	0.01\\
77.68	0.01\\
77.69	0.01\\
77.7	0.01\\
77.71	0.01\\
77.72	0.01\\
77.73	0.01\\
77.74	0.01\\
77.75	0.01\\
77.76	0.01\\
77.77	0.01\\
77.78	0.01\\
77.79	0.01\\
77.8	0.01\\
77.81	0.01\\
77.82	0.01\\
77.83	0.01\\
77.84	0.01\\
77.85	0.01\\
77.86	0.01\\
77.87	0.01\\
77.88	0.01\\
77.89	0.01\\
77.9	0.01\\
77.91	0.01\\
77.92	0.01\\
77.93	0.01\\
77.94	0.01\\
77.95	0.01\\
77.96	0.01\\
77.97	0.01\\
77.98	0.01\\
77.99	0.01\\
78	0.01\\
78.01	0.01\\
78.02	0.01\\
78.03	0.01\\
78.04	0.01\\
78.05	0.01\\
78.06	0.01\\
78.07	0.01\\
78.08	0.01\\
78.09	0.01\\
78.1	0.01\\
78.11	0.01\\
78.12	0.01\\
78.13	0.01\\
78.14	0.01\\
78.15	0.01\\
78.16	0.01\\
78.17	0.01\\
78.18	0.01\\
78.19	0.01\\
78.2	0.01\\
78.21	0.01\\
78.22	0.01\\
78.23	0.01\\
78.24	0.01\\
78.25	0.01\\
78.26	0.01\\
78.27	0.01\\
78.28	0.01\\
78.29	0.01\\
78.3	0.01\\
78.31	0.01\\
78.32	0.01\\
78.33	0.01\\
78.34	0.01\\
78.35	0.01\\
78.36	0.01\\
78.37	0.01\\
78.38	0.01\\
78.39	0.01\\
78.4	0.01\\
78.41	0.01\\
78.42	0.01\\
78.43	0.01\\
78.44	0.01\\
78.45	0.01\\
78.46	0.01\\
78.47	0.01\\
78.48	0.01\\
78.49	0.01\\
78.5	0.01\\
78.51	0.01\\
78.52	0.01\\
78.53	0.01\\
78.54	0.01\\
78.55	0.01\\
78.56	0.01\\
78.57	0.01\\
78.58	0.01\\
78.59	0.01\\
78.6	0.01\\
78.61	0.01\\
78.62	0.01\\
78.63	0.01\\
78.64	0.01\\
78.65	0.01\\
78.66	0.01\\
78.67	0.01\\
78.68	0.01\\
78.69	0.01\\
78.7	0.01\\
78.71	0.01\\
78.72	0.01\\
78.73	0.01\\
78.74	0.01\\
78.75	0.01\\
78.76	0.01\\
78.77	0.01\\
78.78	0.01\\
78.79	0.01\\
78.8	0.01\\
78.81	0.01\\
78.82	0.01\\
78.83	0.01\\
78.84	0.01\\
78.85	0.01\\
78.86	0.01\\
78.87	0.01\\
78.88	0.01\\
78.89	0.01\\
78.9	0.01\\
78.91	0.01\\
78.92	0.01\\
78.93	0.01\\
78.94	0.01\\
78.95	0.01\\
78.96	0.01\\
78.97	0.01\\
78.98	0.01\\
78.99	0.01\\
79	0.01\\
79.01	0.01\\
79.02	0.01\\
79.03	0.01\\
79.04	0.01\\
79.05	0.01\\
79.06	0.01\\
79.07	0.01\\
79.08	0.01\\
79.09	0.01\\
79.1	0.01\\
79.11	0.01\\
79.12	0.01\\
79.13	0.01\\
79.14	0.01\\
79.15	0.01\\
79.16	0.01\\
79.17	0.01\\
79.18	0.01\\
79.19	0.01\\
79.2	0.01\\
79.21	0.01\\
79.22	0.01\\
79.23	0.01\\
79.24	0.01\\
79.25	0.01\\
79.26	0.01\\
79.27	0.01\\
79.28	0.01\\
79.29	0.01\\
79.3	0.01\\
79.31	0.01\\
79.32	0.01\\
79.33	0.01\\
79.34	0.01\\
79.35	0.01\\
79.36	0.01\\
79.37	0.01\\
79.38	0.01\\
79.39	0.01\\
79.4	0.01\\
79.41	0.01\\
79.42	0.01\\
79.43	0.01\\
79.44	0.01\\
79.45	0.01\\
79.46	0.01\\
79.47	0.01\\
79.48	0.01\\
79.49	0.01\\
79.5	0.01\\
79.51	0.01\\
79.52	0.01\\
79.53	0.01\\
79.54	0.01\\
79.55	0.01\\
79.56	0.01\\
79.57	0.01\\
79.58	0.01\\
79.59	0.01\\
79.6	0.01\\
79.61	0.01\\
79.62	0.01\\
79.63	0.01\\
79.64	0.01\\
79.65	0.01\\
79.66	0.01\\
79.67	0.01\\
79.68	0.01\\
79.69	0.01\\
79.7	0.01\\
79.71	0.01\\
79.72	0.01\\
79.73	0.01\\
79.74	0.01\\
79.75	0.01\\
79.76	0.01\\
79.77	0.01\\
79.78	0.01\\
79.79	0.01\\
79.8	0.01\\
79.81	0.01\\
79.82	0.01\\
79.83	0.01\\
79.84	0.01\\
79.85	0.01\\
79.86	0.01\\
79.87	0.01\\
79.88	0.01\\
79.89	0.01\\
79.9	0.01\\
79.91	0.01\\
79.92	0.01\\
79.93	0.01\\
79.94	0.01\\
79.95	0.01\\
79.96	0.01\\
79.97	0.01\\
79.98	0.01\\
79.99	0.01\\
80	0.01\\
80.01	0.01\\
};
\addplot [color=green,solid]
  table[row sep=crcr]{%
80.01	0.01\\
80.02	0.01\\
80.03	0.01\\
80.04	0.01\\
80.05	0.01\\
80.06	0.01\\
80.07	0.01\\
80.08	0.01\\
80.09	0.01\\
80.1	0.01\\
80.11	0.01\\
80.12	0.01\\
80.13	0.01\\
80.14	0.01\\
80.15	0.01\\
80.16	0.01\\
80.17	0.01\\
80.18	0.01\\
80.19	0.01\\
80.2	0.01\\
80.21	0.01\\
80.22	0.01\\
80.23	0.01\\
80.24	0.01\\
80.25	0.01\\
80.26	0.01\\
80.27	0.01\\
80.28	0.01\\
80.29	0.01\\
80.3	0.01\\
80.31	0.01\\
80.32	0.01\\
80.33	0.01\\
80.34	0.01\\
80.35	0.01\\
80.36	0.01\\
80.37	0.01\\
80.38	0.01\\
80.39	0.01\\
80.4	0.01\\
80.41	0.01\\
80.42	0.01\\
80.43	0.01\\
80.44	0.01\\
80.45	0.01\\
80.46	0.01\\
80.47	0.01\\
80.48	0.01\\
80.49	0.01\\
80.5	0.01\\
80.51	0.01\\
80.52	0.01\\
80.53	0.01\\
80.54	0.01\\
80.55	0.01\\
80.56	0.01\\
80.57	0.01\\
80.58	0.01\\
80.59	0.01\\
80.6	0.01\\
80.61	0.01\\
80.62	0.01\\
80.63	0.01\\
80.64	0.01\\
80.65	0.01\\
80.66	0.01\\
80.67	0.01\\
80.68	0.01\\
80.69	0.01\\
80.7	0.01\\
80.71	0.01\\
80.72	0.01\\
80.73	0.01\\
80.74	0.01\\
80.75	0.01\\
80.76	0.01\\
80.77	0.01\\
80.78	0.01\\
80.79	0.01\\
80.8	0.01\\
80.81	0.01\\
80.82	0.01\\
80.83	0.01\\
80.84	0.01\\
80.85	0.01\\
80.86	0.01\\
80.87	0.01\\
80.88	0.01\\
80.89	0.01\\
80.9	0.01\\
80.91	0.01\\
80.92	0.01\\
80.93	0.01\\
80.94	0.01\\
80.95	0.01\\
80.96	0.01\\
80.97	0.01\\
80.98	0.01\\
80.99	0.01\\
81	0.01\\
81.01	0.01\\
81.02	0.01\\
81.03	0.01\\
81.04	0.01\\
81.05	0.01\\
81.06	0.01\\
81.07	0.01\\
81.08	0.01\\
81.09	0.01\\
81.1	0.01\\
81.11	0.01\\
81.12	0.01\\
81.13	0.01\\
81.14	0.01\\
81.15	0.01\\
81.16	0.01\\
81.17	0.01\\
81.18	0.01\\
81.19	0.01\\
81.2	0.01\\
81.21	0.01\\
81.22	0.01\\
81.23	0.01\\
81.24	0.01\\
81.25	0.01\\
81.26	0.01\\
81.27	0.01\\
81.28	0.01\\
81.29	0.01\\
81.3	0.01\\
81.31	0.01\\
81.32	0.01\\
81.33	0.01\\
81.34	0.01\\
81.35	0.01\\
81.36	0.01\\
81.37	0.01\\
81.38	0.01\\
81.39	0.01\\
81.4	0.01\\
81.41	0.01\\
81.42	0.01\\
81.43	0.01\\
81.44	0.01\\
81.45	0.01\\
81.46	0.01\\
81.47	0.01\\
81.48	0.01\\
81.49	0.01\\
81.5	0.01\\
81.51	0.01\\
81.52	0.01\\
81.53	0.01\\
81.54	0.01\\
81.55	0.01\\
81.56	0.01\\
81.57	0.01\\
81.58	0.01\\
81.59	0.01\\
81.6	0.01\\
81.61	0.01\\
81.62	0.01\\
81.63	0.01\\
81.64	0.01\\
81.65	0.01\\
81.66	0.01\\
81.67	0.01\\
81.68	0.01\\
81.69	0.01\\
81.7	0.01\\
81.71	0.01\\
81.72	0.01\\
81.73	0.01\\
81.74	0.01\\
81.75	0.01\\
81.76	0.01\\
81.77	0.01\\
81.78	0.01\\
81.79	0.01\\
81.8	0.01\\
81.81	0.01\\
81.82	0.01\\
81.83	0.01\\
81.84	0.01\\
81.85	0.01\\
81.86	0.01\\
81.87	0.01\\
81.88	0.01\\
81.89	0.01\\
81.9	0.01\\
81.91	0.01\\
81.92	0.01\\
81.93	0.01\\
81.94	0.01\\
81.95	0.01\\
81.96	0.01\\
81.97	0.01\\
81.98	0.01\\
81.99	0.01\\
82	0.01\\
82.01	0.01\\
82.02	0.01\\
82.03	0.01\\
82.04	0.01\\
82.05	0.01\\
82.06	0.01\\
82.07	0.01\\
82.08	0.01\\
82.09	0.01\\
82.1	0.01\\
82.11	0.01\\
82.12	0.01\\
82.13	0.01\\
82.14	0.01\\
82.15	0.01\\
82.16	0.01\\
82.17	0.01\\
82.18	0.01\\
82.19	0.01\\
82.2	0.01\\
82.21	0.01\\
82.22	0.01\\
82.23	0.01\\
82.24	0.01\\
82.25	0.01\\
82.26	0.01\\
82.27	0.01\\
82.28	0.01\\
82.29	0.01\\
82.3	0.01\\
82.31	0.01\\
82.32	0.01\\
82.33	0.01\\
82.34	0.01\\
82.35	0.01\\
82.36	0.01\\
82.37	0.01\\
82.38	0.01\\
82.39	0.01\\
82.4	0.01\\
82.41	0.01\\
82.42	0.01\\
82.43	0.01\\
82.44	0.01\\
82.45	0.01\\
82.46	0.01\\
82.47	0.01\\
82.48	0.01\\
82.49	0.01\\
82.5	0.01\\
82.51	0.01\\
82.52	0.01\\
82.53	0.01\\
82.54	0.01\\
82.55	0.01\\
82.56	0.01\\
82.57	0.01\\
82.58	0.01\\
82.59	0.01\\
82.6	0.01\\
82.61	0.01\\
82.62	0.01\\
82.63	0.01\\
82.64	0.01\\
82.65	0.01\\
82.66	0.01\\
82.67	0.01\\
82.68	0.01\\
82.69	0.01\\
82.7	0.01\\
82.71	0.01\\
82.72	0.01\\
82.73	0.01\\
82.74	0.01\\
82.75	0.01\\
82.76	0.01\\
82.77	0.01\\
82.78	0.01\\
82.79	0.01\\
82.8	0.01\\
82.81	0.01\\
82.82	0.01\\
82.83	0.01\\
82.84	0.01\\
82.85	0.01\\
82.86	0.01\\
82.87	0.01\\
82.88	0.01\\
82.89	0.01\\
82.9	0.01\\
82.91	0.01\\
82.92	0.01\\
82.93	0.01\\
82.94	0.01\\
82.95	0.01\\
82.96	0.01\\
82.97	0.01\\
82.98	0.01\\
82.99	0.01\\
83	0.01\\
83.01	0.01\\
83.02	0.01\\
83.03	0.01\\
83.04	0.01\\
83.05	0.01\\
83.06	0.01\\
83.07	0.01\\
83.08	0.01\\
83.09	0.01\\
83.1	0.01\\
83.11	0.01\\
83.12	0.01\\
83.13	0.01\\
83.14	0.01\\
83.15	0.01\\
83.16	0.01\\
83.17	0.01\\
83.18	0.01\\
83.19	0.01\\
83.2	0.01\\
83.21	0.01\\
83.22	0.01\\
83.23	0.01\\
83.24	0.01\\
83.25	0.01\\
83.26	0.01\\
83.27	0.01\\
83.28	0.01\\
83.29	0.01\\
83.3	0.01\\
83.31	0.01\\
83.32	0.01\\
83.33	0.01\\
83.34	0.01\\
83.35	0.01\\
83.36	0.01\\
83.37	0.01\\
83.38	0.01\\
83.39	0.01\\
83.4	0.01\\
83.41	0.01\\
83.42	0.01\\
83.43	0.01\\
83.44	0.01\\
83.45	0.01\\
83.46	0.01\\
83.47	0.01\\
83.48	0.01\\
83.49	0.01\\
83.5	0.01\\
83.51	0.01\\
83.52	0.01\\
83.53	0.01\\
83.54	0.01\\
83.55	0.01\\
83.56	0.01\\
83.57	0.01\\
83.58	0.01\\
83.59	0.01\\
83.6	0.01\\
83.61	0.01\\
83.62	0.01\\
83.63	0.01\\
83.64	0.01\\
83.65	0.01\\
83.66	0.01\\
83.67	0.01\\
83.68	0.01\\
83.69	0.01\\
83.7	0.01\\
83.71	0.01\\
83.72	0.01\\
83.73	0.01\\
83.74	0.01\\
83.75	0.01\\
83.76	0.01\\
83.77	0.01\\
83.78	0.01\\
83.79	0.01\\
83.8	0.01\\
83.81	0.01\\
83.82	0.01\\
83.83	0.01\\
83.84	0.01\\
83.85	0.01\\
83.86	0.01\\
83.87	0.01\\
83.88	0.01\\
83.89	0.01\\
83.9	0.01\\
83.91	0.01\\
83.92	0.01\\
83.93	0.01\\
83.94	0.01\\
83.95	0.01\\
83.96	0.01\\
83.97	0.01\\
83.98	0.01\\
83.99	0.01\\
84	0.01\\
84.01	0.01\\
84.02	0.01\\
84.03	0.01\\
84.04	0.01\\
84.05	0.01\\
84.06	0.01\\
84.07	0.01\\
84.08	0.01\\
84.09	0.01\\
84.1	0.01\\
84.11	0.01\\
84.12	0.01\\
84.13	0.01\\
84.14	0.01\\
84.15	0.01\\
84.16	0.01\\
84.17	0.01\\
84.18	0.01\\
84.19	0.01\\
84.2	0.01\\
84.21	0.01\\
84.22	0.01\\
84.23	0.01\\
84.24	0.01\\
84.25	0.01\\
84.26	0.01\\
84.27	0.01\\
84.28	0.01\\
84.29	0.01\\
84.3	0.01\\
84.31	0.01\\
84.32	0.01\\
84.33	0.01\\
84.34	0.01\\
84.35	0.01\\
84.36	0.01\\
84.37	0.01\\
84.38	0.01\\
84.39	0.01\\
84.4	0.01\\
84.41	0.01\\
84.42	0.01\\
84.43	0.01\\
84.44	0.01\\
84.45	0.01\\
84.46	0.01\\
84.47	0.01\\
84.48	0.01\\
84.49	0.01\\
84.5	0.01\\
84.51	0.01\\
84.52	0.01\\
84.53	0.01\\
84.54	0.01\\
84.55	0.01\\
84.56	0.01\\
84.57	0.01\\
84.58	0.01\\
84.59	0.01\\
84.6	0.01\\
84.61	0.01\\
84.62	0.01\\
84.63	0.01\\
84.64	0.01\\
84.65	0.01\\
84.66	0.01\\
84.67	0.01\\
84.68	0.01\\
84.69	0.01\\
84.7	0.01\\
84.71	0.01\\
84.72	0.01\\
84.73	0.01\\
84.74	0.01\\
84.75	0.01\\
84.76	0.01\\
84.77	0.01\\
84.78	0.01\\
84.79	0.01\\
84.8	0.01\\
84.81	0.01\\
84.82	0.01\\
84.83	0.01\\
84.84	0.01\\
84.85	0.01\\
84.86	0.01\\
84.87	0.01\\
84.88	0.01\\
84.89	0.01\\
84.9	0.01\\
84.91	0.01\\
84.92	0.01\\
84.93	0.01\\
84.94	0.01\\
84.95	0.01\\
84.96	0.01\\
84.97	0.01\\
84.98	0.01\\
84.99	0.01\\
85	0.01\\
85.01	0.01\\
85.02	0.01\\
85.03	0.01\\
85.04	0.01\\
85.05	0.01\\
85.06	0.01\\
85.07	0.01\\
85.08	0.01\\
85.09	0.01\\
85.1	0.01\\
85.11	0.01\\
85.12	0.01\\
85.13	0.01\\
85.14	0.01\\
85.15	0.01\\
85.16	0.01\\
85.17	0.01\\
85.18	0.01\\
85.19	0.01\\
85.2	0.01\\
85.21	0.01\\
85.22	0.01\\
85.23	0.01\\
85.24	0.01\\
85.25	0.01\\
85.26	0.01\\
85.27	0.01\\
85.28	0.01\\
85.29	0.01\\
85.3	0.01\\
85.31	0.01\\
85.32	0.01\\
85.33	0.01\\
85.34	0.01\\
85.35	0.01\\
85.36	0.01\\
85.37	0.01\\
85.38	0.01\\
85.39	0.01\\
85.4	0.01\\
85.41	0.01\\
85.42	0.01\\
85.43	0.01\\
85.44	0.01\\
85.45	0.01\\
85.46	0.01\\
85.47	0.01\\
85.48	0.01\\
85.49	0.01\\
85.5	0.01\\
85.51	0.01\\
85.52	0.01\\
85.53	0.01\\
85.54	0.01\\
85.55	0.01\\
85.56	0.01\\
85.57	0.01\\
85.58	0.01\\
85.59	0.01\\
85.6	0.01\\
85.61	0.01\\
85.62	0.01\\
85.63	0.01\\
85.64	0.01\\
85.65	0.01\\
85.66	0.01\\
85.67	0.01\\
85.68	0.01\\
85.69	0.01\\
85.7	0.01\\
85.71	0.01\\
85.72	0.01\\
85.73	0.01\\
85.74	0.01\\
85.75	0.01\\
85.76	0.01\\
85.77	0.01\\
85.78	0.01\\
85.79	0.01\\
85.8	0.01\\
85.81	0.01\\
85.82	0.01\\
85.83	0.01\\
85.84	0.01\\
85.85	0.01\\
85.86	0.01\\
85.87	0.01\\
85.88	0.01\\
85.89	0.01\\
85.9	0.01\\
85.91	0.01\\
85.92	0.01\\
85.93	0.01\\
85.94	0.01\\
85.95	0.01\\
85.96	0.01\\
85.97	0.01\\
85.98	0.01\\
85.99	0.01\\
86	0.01\\
86.01	0.01\\
86.02	0.01\\
86.03	0.01\\
86.04	0.01\\
86.05	0.01\\
86.06	0.01\\
86.07	0.01\\
86.08	0.01\\
86.09	0.01\\
86.1	0.01\\
86.11	0.01\\
86.12	0.01\\
86.13	0.01\\
86.14	0.01\\
86.15	0.01\\
86.16	0.01\\
86.17	0.01\\
86.18	0.01\\
86.19	0.01\\
86.2	0.01\\
86.21	0.01\\
86.22	0.01\\
86.23	0.01\\
86.24	0.01\\
86.25	0.01\\
86.26	0.01\\
86.27	0.01\\
86.28	0.01\\
86.29	0.01\\
86.3	0.01\\
86.31	0.01\\
86.32	0.01\\
86.33	0.01\\
86.34	0.01\\
86.35	0.01\\
86.36	0.01\\
86.37	0.01\\
86.38	0.01\\
86.39	0.01\\
86.4	0.01\\
86.41	0.01\\
86.42	0.01\\
86.43	0.01\\
86.44	0.01\\
86.45	0.01\\
86.46	0.01\\
86.47	0.01\\
86.48	0.01\\
86.49	0.01\\
86.5	0.01\\
86.51	0.01\\
86.52	0.01\\
86.53	0.01\\
86.54	0.01\\
86.55	0.01\\
86.56	0.01\\
86.57	0.01\\
86.58	0.01\\
86.59	0.01\\
86.6	0.01\\
86.61	0.01\\
86.62	0.01\\
86.63	0.01\\
86.64	0.01\\
86.65	0.01\\
86.66	0.01\\
86.67	0.01\\
86.68	0.01\\
86.69	0.01\\
86.7	0.01\\
86.71	0.01\\
86.72	0.01\\
86.73	0.01\\
86.74	0.01\\
86.75	0.01\\
86.76	0.01\\
86.77	0.01\\
86.78	0.01\\
86.79	0.01\\
86.8	0.01\\
86.81	0.01\\
86.82	0.01\\
86.83	0.01\\
86.84	0.01\\
86.85	0.01\\
86.86	0.01\\
86.87	0.01\\
86.88	0.01\\
86.89	0.01\\
86.9	0.01\\
86.91	0.01\\
86.92	0.01\\
86.93	0.01\\
86.94	0.01\\
86.95	0.01\\
86.96	0.01\\
86.97	0.01\\
86.98	0.01\\
86.99	0.01\\
87	0.01\\
87.01	0.01\\
87.02	0.01\\
87.03	0.01\\
87.04	0.01\\
87.05	0.01\\
87.06	0.01\\
87.07	0.01\\
87.08	0.01\\
87.09	0.01\\
87.1	0.01\\
87.11	0.01\\
87.12	0.01\\
87.13	0.01\\
87.14	0.01\\
87.15	0.01\\
87.16	0.01\\
87.17	0.01\\
87.18	0.01\\
87.19	0.01\\
87.2	0.01\\
87.21	0.01\\
87.22	0.01\\
87.23	0.01\\
87.24	0.01\\
87.25	0.01\\
87.26	0.01\\
87.27	0.01\\
87.28	0.01\\
87.29	0.01\\
87.3	0.01\\
87.31	0.01\\
87.32	0.01\\
87.33	0.01\\
87.34	0.01\\
87.35	0.01\\
87.36	0.01\\
87.37	0.01\\
87.38	0.01\\
87.39	0.01\\
87.4	0.01\\
87.41	0.01\\
87.42	0.01\\
87.43	0.01\\
87.44	0.01\\
87.45	0.01\\
87.46	0.01\\
87.47	0.01\\
87.48	0.01\\
87.49	0.01\\
87.5	0.01\\
87.51	0.01\\
87.52	0.01\\
87.53	0.01\\
87.54	0.01\\
87.55	0.01\\
87.56	0.01\\
87.57	0.01\\
87.58	0.01\\
87.59	0.01\\
87.6	0.01\\
87.61	0.01\\
87.62	0.01\\
87.63	0.01\\
87.64	0.01\\
87.65	0.01\\
87.66	0.01\\
87.67	0.01\\
87.68	0.01\\
87.69	0.01\\
87.7	0.01\\
87.71	0.01\\
87.72	0.01\\
87.73	0.01\\
87.74	0.01\\
87.75	0.01\\
87.76	0.01\\
87.77	0.01\\
87.78	0.01\\
87.79	0.01\\
87.8	0.01\\
87.81	0.01\\
87.82	0.01\\
87.83	0.01\\
87.84	0.01\\
87.85	0.01\\
87.86	0.01\\
87.87	0.01\\
87.88	0.01\\
87.89	0.01\\
87.9	0.01\\
87.91	0.01\\
87.92	0.01\\
87.93	0.01\\
87.94	0.01\\
87.95	0.01\\
87.96	0.01\\
87.97	0.01\\
87.98	0.01\\
87.99	0.01\\
88	0.01\\
88.01	0.01\\
88.02	0.01\\
88.03	0.01\\
88.04	0.01\\
88.05	0.01\\
88.06	0.01\\
88.07	0.01\\
88.08	0.01\\
88.09	0.01\\
88.1	0.01\\
88.11	0.01\\
88.12	0.01\\
88.13	0.01\\
88.14	0.01\\
88.15	0.01\\
88.16	0.01\\
88.17	0.01\\
88.18	0.01\\
88.19	0.01\\
88.2	0.01\\
88.21	0.01\\
88.22	0.01\\
88.23	0.01\\
88.24	0.01\\
88.25	0.01\\
88.26	0.01\\
88.27	0.01\\
88.28	0.01\\
88.29	0.01\\
88.3	0.01\\
88.31	0.01\\
88.32	0.01\\
88.33	0.01\\
88.34	0.01\\
88.35	0.01\\
88.36	0.01\\
88.37	0.01\\
88.38	0.01\\
88.39	0.01\\
88.4	0.01\\
88.41	0.01\\
88.42	0.01\\
88.43	0.01\\
88.44	0.01\\
88.45	0.01\\
88.46	0.01\\
88.47	0.01\\
88.48	0.01\\
88.49	0.01\\
88.5	0.01\\
88.51	0.01\\
88.52	0.01\\
88.53	0.01\\
88.54	0.01\\
88.55	0.01\\
88.56	0.01\\
88.57	0.01\\
88.58	0.01\\
88.59	0.01\\
88.6	0.01\\
88.61	0.01\\
88.62	0.01\\
88.63	0.01\\
88.64	0.01\\
88.65	0.01\\
88.66	0.01\\
88.67	0.01\\
88.68	0.01\\
88.69	0.01\\
88.7	0.01\\
88.71	0.01\\
88.72	0.01\\
88.73	0.01\\
88.74	0.01\\
88.75	0.01\\
88.76	0.01\\
88.77	0.01\\
88.78	0.01\\
88.79	0.01\\
88.8	0.01\\
88.81	0.01\\
88.82	0.01\\
88.83	0.01\\
88.84	0.01\\
88.85	0.01\\
88.86	0.01\\
88.87	0.01\\
88.88	0.01\\
88.89	0.01\\
88.9	0.01\\
88.91	0.01\\
88.92	0.01\\
88.93	0.01\\
88.94	0.01\\
88.95	0.01\\
88.96	0.01\\
88.97	0.01\\
88.98	0.01\\
88.99	0.01\\
89	0.01\\
89.01	0.01\\
89.02	0.01\\
89.03	0.01\\
89.04	0.01\\
89.05	0.01\\
89.06	0.01\\
89.07	0.01\\
89.08	0.01\\
89.09	0.01\\
89.1	0.01\\
89.11	0.01\\
89.12	0.01\\
89.13	0.01\\
89.14	0.01\\
89.15	0.01\\
89.16	0.01\\
89.17	0.01\\
89.18	0.01\\
89.19	0.01\\
89.2	0.01\\
89.21	0.01\\
89.22	0.01\\
89.23	0.01\\
89.24	0.01\\
89.25	0.01\\
89.26	0.01\\
89.27	0.01\\
89.28	0.01\\
89.29	0.01\\
89.3	0.01\\
89.31	0.01\\
89.32	0.01\\
89.33	0.01\\
89.34	0.01\\
89.35	0.01\\
89.36	0.01\\
89.37	0.01\\
89.38	0.01\\
89.39	0.01\\
89.4	0.01\\
89.41	0.01\\
89.42	0.01\\
89.43	0.01\\
89.44	0.01\\
89.45	0.01\\
89.46	0.01\\
89.47	0.01\\
89.48	0.01\\
89.49	0.01\\
89.5	0.01\\
89.51	0.01\\
89.52	0.01\\
89.53	0.01\\
89.54	0.01\\
89.55	0.01\\
89.56	0.01\\
89.57	0.01\\
89.58	0.01\\
89.59	0.01\\
89.6	0.01\\
89.61	0.01\\
89.62	0.01\\
89.63	0.01\\
89.64	0.01\\
89.65	0.01\\
89.66	0.01\\
89.67	0.01\\
89.68	0.01\\
89.69	0.01\\
89.7	0.01\\
89.71	0.01\\
89.72	0.01\\
89.73	0.01\\
89.74	0.01\\
89.75	0.01\\
89.76	0.01\\
89.77	0.01\\
89.78	0.01\\
89.79	0.01\\
89.8	0.01\\
89.81	0.01\\
89.82	0.01\\
89.83	0.01\\
89.84	0.01\\
89.85	0.01\\
89.86	0.01\\
89.87	0.01\\
89.88	0.01\\
89.89	0.01\\
89.9	0.01\\
89.91	0.01\\
89.92	0.01\\
89.93	0.01\\
89.94	0.01\\
89.95	0.01\\
89.96	0.01\\
89.97	0.01\\
89.98	0.01\\
89.99	0.01\\
90	0.01\\
90.01	0.01\\
90.02	0.01\\
90.03	0.01\\
90.04	0.01\\
90.05	0.01\\
90.06	0.01\\
90.07	0.01\\
90.08	0.01\\
90.09	0.01\\
90.1	0.01\\
90.11	0.01\\
90.12	0.01\\
90.13	0.01\\
90.14	0.01\\
90.15	0.01\\
90.16	0.01\\
90.17	0.01\\
90.18	0.01\\
90.19	0.01\\
90.2	0.01\\
90.21	0.01\\
90.22	0.01\\
90.23	0.01\\
90.24	0.01\\
90.25	0.01\\
90.26	0.01\\
90.27	0.01\\
90.28	0.01\\
90.29	0.01\\
90.3	0.01\\
90.31	0.01\\
90.32	0.01\\
90.33	0.01\\
90.34	0.01\\
90.35	0.01\\
90.36	0.01\\
90.37	0.01\\
90.38	0.01\\
90.39	0.01\\
90.4	0.01\\
90.41	0.01\\
90.42	0.01\\
90.43	0.01\\
90.44	0.01\\
90.45	0.01\\
90.46	0.01\\
90.47	0.01\\
90.48	0.01\\
90.49	0.01\\
90.5	0.01\\
90.51	0.01\\
90.52	0.01\\
90.53	0.01\\
90.54	0.01\\
90.55	0.01\\
90.56	0.01\\
90.57	0.01\\
90.58	0.01\\
90.59	0.01\\
90.6	0.01\\
90.61	0.01\\
90.62	0.01\\
90.63	0.01\\
90.64	0.01\\
90.65	0.01\\
90.66	0.01\\
90.67	0.01\\
90.68	0.01\\
90.69	0.01\\
90.7	0.01\\
90.71	0.01\\
90.72	0.01\\
90.73	0.01\\
90.74	0.01\\
90.75	0.01\\
90.76	0.01\\
90.77	0.01\\
90.78	0.01\\
90.79	0.01\\
90.8	0.01\\
90.81	0.01\\
90.82	0.01\\
90.83	0.01\\
90.84	0.01\\
90.85	0.01\\
90.86	0.01\\
90.87	0.01\\
90.88	0.01\\
90.89	0.01\\
90.9	0.01\\
90.91	0.01\\
90.92	0.01\\
90.93	0.01\\
90.94	0.01\\
90.95	0.01\\
90.96	0.01\\
90.97	0.01\\
90.98	0.01\\
90.99	0.01\\
91	0.01\\
91.01	0.01\\
91.02	0.01\\
91.03	0.01\\
91.04	0.01\\
91.05	0.01\\
91.06	0.01\\
91.07	0.01\\
91.08	0.01\\
91.09	0.01\\
91.1	0.01\\
91.11	0.01\\
91.12	0.01\\
91.13	0.01\\
91.14	0.01\\
91.15	0.01\\
91.16	0.01\\
91.17	0.01\\
91.18	0.01\\
91.19	0.01\\
91.2	0.01\\
91.21	0.01\\
91.22	0.01\\
91.23	0.01\\
91.24	0.01\\
91.25	0.01\\
91.26	0.01\\
91.27	0.01\\
91.28	0.01\\
91.29	0.01\\
91.3	0.01\\
91.31	0.01\\
91.32	0.01\\
91.33	0.01\\
91.34	0.01\\
91.35	0.01\\
91.36	0.01\\
91.37	0.01\\
91.38	0.01\\
91.39	0.01\\
91.4	0.01\\
91.41	0.01\\
91.42	0.01\\
91.43	0.01\\
91.44	0.01\\
91.45	0.01\\
91.46	0.01\\
91.47	0.01\\
91.48	0.01\\
91.49	0.01\\
91.5	0.01\\
91.51	0.01\\
91.52	0.01\\
91.53	0.01\\
91.54	0.01\\
91.55	0.01\\
91.56	0.01\\
91.57	0.01\\
91.58	0.01\\
91.59	0.01\\
91.6	0.01\\
91.61	0.01\\
91.62	0.01\\
91.63	0.01\\
91.64	0.01\\
91.65	0.01\\
91.66	0.01\\
91.67	0.01\\
91.68	0.01\\
91.69	0.01\\
91.7	0.01\\
91.71	0.01\\
91.72	0.01\\
91.73	0.01\\
91.74	0.01\\
91.75	0.01\\
91.76	0.01\\
91.77	0.01\\
91.78	0.01\\
91.79	0.01\\
91.8	0.01\\
91.81	0.01\\
91.82	0.01\\
91.83	0.01\\
91.84	0.01\\
91.85	0.01\\
91.86	0.01\\
91.87	0.01\\
91.88	0.01\\
91.89	0.01\\
91.9	0.01\\
91.91	0.01\\
91.92	0.01\\
91.93	0.01\\
91.94	0.01\\
91.95	0.01\\
91.96	0.01\\
91.97	0.01\\
91.98	0.01\\
91.99	0.01\\
92	0.01\\
92.01	0.01\\
92.02	0.01\\
92.03	0.01\\
92.04	0.01\\
92.05	0.01\\
92.06	0.01\\
92.07	0.01\\
92.08	0.01\\
92.09	0.01\\
92.1	0.01\\
92.11	0.01\\
92.12	0.01\\
92.13	0.01\\
92.14	0.01\\
92.15	0.01\\
92.16	0.01\\
92.17	0.01\\
92.18	0.01\\
92.19	0.01\\
92.2	0.01\\
92.21	0.01\\
92.22	0.01\\
92.23	0.01\\
92.24	0.01\\
92.25	0.01\\
92.26	0.01\\
92.27	0.01\\
92.28	0.01\\
92.29	0.01\\
92.3	0.01\\
92.31	0.01\\
92.32	0.01\\
92.33	0.01\\
92.34	0.01\\
92.35	0.01\\
92.36	0.01\\
92.37	0.01\\
92.38	0.01\\
92.39	0.01\\
92.4	0.01\\
92.41	0.01\\
92.42	0.01\\
92.43	0.01\\
92.44	0.01\\
92.45	0.01\\
92.46	0.01\\
92.47	0.01\\
92.48	0.01\\
92.49	0.01\\
92.5	0.01\\
92.51	0.01\\
92.52	0.01\\
92.53	0.01\\
92.54	0.01\\
92.55	0.01\\
92.56	0.01\\
92.57	0.01\\
92.58	0.01\\
92.59	0.01\\
92.6	0.01\\
92.61	0.01\\
92.62	0.01\\
92.63	0.01\\
92.64	0.01\\
92.65	0.01\\
92.66	0.01\\
92.67	0.01\\
92.68	0.01\\
92.69	0.01\\
92.7	0.01\\
92.71	0.01\\
92.72	0.01\\
92.73	0.01\\
92.74	0.01\\
92.75	0.01\\
92.76	0.01\\
92.77	0.01\\
92.78	0.01\\
92.79	0.01\\
92.8	0.01\\
92.81	0.01\\
92.82	0.01\\
92.83	0.01\\
92.84	0.01\\
92.85	0.01\\
92.86	0.01\\
92.87	0.01\\
92.88	0.01\\
92.89	0.01\\
92.9	0.01\\
92.91	0.01\\
92.92	0.01\\
92.93	0.01\\
92.94	0.01\\
92.95	0.01\\
92.96	0.01\\
92.97	0.01\\
92.98	0.01\\
92.99	0.01\\
93	0.01\\
93.01	0.01\\
93.02	0.01\\
93.03	0.01\\
93.04	0.01\\
93.05	0.01\\
93.06	0.01\\
93.07	0.01\\
93.08	0.01\\
93.09	0.01\\
93.1	0.01\\
93.11	0.01\\
93.12	0.01\\
93.13	0.01\\
93.14	0.01\\
93.15	0.01\\
93.16	0.01\\
93.17	0.01\\
93.18	0.01\\
93.19	0.01\\
93.2	0.01\\
93.21	0.01\\
93.22	0.01\\
93.23	0.01\\
93.24	0.01\\
93.25	0.01\\
93.26	0.01\\
93.27	0.01\\
93.28	0.01\\
93.29	0.01\\
93.3	0.01\\
93.31	0.01\\
93.32	0.01\\
93.33	0.01\\
93.34	0.01\\
93.35	0.01\\
93.36	0.01\\
93.37	0.01\\
93.38	0.01\\
93.39	0.01\\
93.4	0.01\\
93.41	0.01\\
93.42	0.01\\
93.43	0.01\\
93.44	0.01\\
93.45	0.01\\
93.46	0.01\\
93.47	0.01\\
93.48	0.01\\
93.49	0.01\\
93.5	0.01\\
93.51	0.01\\
93.52	0.01\\
93.53	0.01\\
93.54	0.01\\
93.55	0.01\\
93.56	0.01\\
93.57	0.01\\
93.58	0.01\\
93.59	0.01\\
93.6	0.01\\
93.61	0.01\\
93.62	0.01\\
93.63	0.01\\
93.64	0.01\\
93.65	0.01\\
93.66	0.01\\
93.67	0.01\\
93.68	0.01\\
93.69	0.01\\
93.7	0.01\\
93.71	0.01\\
93.72	0.01\\
93.73	0.01\\
93.74	0.01\\
93.75	0.01\\
93.76	0.01\\
93.77	0.01\\
93.78	0.01\\
93.79	0.01\\
93.8	0.01\\
93.81	0.01\\
93.82	0.01\\
93.83	0.01\\
93.84	0.01\\
93.85	0.01\\
93.86	0.01\\
93.87	0.01\\
93.88	0.01\\
93.89	0.01\\
93.9	0.01\\
93.91	0.01\\
93.92	0.01\\
93.93	0.01\\
93.94	0.01\\
93.95	0.01\\
93.96	0.01\\
93.97	0.01\\
93.98	0.01\\
93.99	0.01\\
94	0.01\\
94.01	0.01\\
94.02	0.01\\
94.03	0.01\\
94.04	0.01\\
94.05	0.01\\
94.06	0.01\\
94.07	0.01\\
94.08	0.01\\
94.09	0.01\\
94.1	0.01\\
94.11	0.01\\
94.12	0.01\\
94.13	0.01\\
94.14	0.01\\
94.15	0.01\\
94.16	0.01\\
94.17	0.01\\
94.18	0.01\\
94.19	0.01\\
94.2	0.01\\
94.21	0.01\\
94.22	0.01\\
94.23	0.01\\
94.24	0.01\\
94.25	0.01\\
94.26	0.01\\
94.27	0.01\\
94.28	0.01\\
94.29	0.01\\
94.3	0.01\\
94.31	0.01\\
94.32	0.01\\
94.33	0.01\\
94.34	0.01\\
94.35	0.01\\
94.36	0.01\\
94.37	0.01\\
94.38	0.01\\
94.39	0.01\\
94.4	0.01\\
94.41	0.01\\
94.42	0.01\\
94.43	0.01\\
94.44	0.01\\
94.45	0.01\\
94.46	0.01\\
94.47	0.01\\
94.48	0.01\\
94.49	0.01\\
94.5	0.01\\
94.51	0.01\\
94.52	0.01\\
94.53	0.01\\
94.54	0.01\\
94.55	0.01\\
94.56	0.01\\
94.57	0.01\\
94.58	0.01\\
94.59	0.01\\
94.6	0.01\\
94.61	0.01\\
94.62	0.01\\
94.63	0.01\\
94.64	0.01\\
94.65	0.01\\
94.66	0.01\\
94.67	0.01\\
94.68	0.01\\
94.69	0.01\\
94.7	0.01\\
94.71	0.01\\
94.72	0.01\\
94.73	0.01\\
94.74	0.01\\
94.75	0.01\\
94.76	0.01\\
94.77	0.01\\
94.78	0.01\\
94.79	0.01\\
94.8	0.01\\
94.81	0.01\\
94.82	0.01\\
94.83	0.01\\
94.84	0.01\\
94.85	0.01\\
94.86	0.01\\
94.87	0.01\\
94.88	0.01\\
94.89	0.01\\
94.9	0.01\\
94.91	0.01\\
94.92	0.01\\
94.93	0.01\\
94.94	0.01\\
94.95	0.01\\
94.96	0.01\\
94.97	0.01\\
94.98	0.01\\
94.99	0.01\\
95	0.01\\
95.01	0.01\\
95.02	0.01\\
95.03	0.01\\
95.04	0.01\\
95.05	0.01\\
95.06	0.01\\
95.07	0.01\\
95.08	0.01\\
95.09	0.01\\
95.1	0.01\\
95.11	0.01\\
95.12	0.01\\
95.13	0.01\\
95.14	0.01\\
95.15	0.01\\
95.16	0.01\\
95.17	0.01\\
95.18	0.01\\
95.19	0.01\\
95.2	0.01\\
95.21	0.01\\
95.22	0.01\\
95.23	0.01\\
95.24	0.01\\
95.25	0.01\\
95.26	0.01\\
95.27	0.01\\
95.28	0.01\\
95.29	0.01\\
95.3	0.01\\
95.31	0.01\\
95.32	0.01\\
95.33	0.01\\
95.34	0.01\\
95.35	0.01\\
95.36	0.01\\
95.37	0.01\\
95.38	0.01\\
95.39	0.01\\
95.4	0.01\\
95.41	0.01\\
95.42	0.01\\
95.43	0.01\\
95.44	0.01\\
95.45	0.01\\
95.46	0.01\\
95.47	0.01\\
95.48	0.01\\
95.49	0.01\\
95.5	0.01\\
95.51	0.01\\
95.52	0.01\\
95.53	0.01\\
95.54	0.01\\
95.55	0.01\\
95.56	0.01\\
95.57	0.01\\
95.58	0.01\\
95.59	0.01\\
95.6	0.01\\
95.61	0.01\\
95.62	0.01\\
95.63	0.01\\
95.64	0.01\\
95.65	0.01\\
95.66	0.01\\
95.67	0.01\\
95.68	0.01\\
95.69	0.01\\
95.7	0.01\\
95.71	0.01\\
95.72	0.01\\
95.73	0.01\\
95.74	0.01\\
95.75	0.01\\
95.76	0.01\\
95.77	0.01\\
95.78	0.01\\
95.79	0.01\\
95.8	0.01\\
95.81	0.01\\
95.82	0.01\\
95.83	0.01\\
95.84	0.01\\
95.85	0.01\\
95.86	0.01\\
95.87	0.01\\
95.88	0.01\\
95.89	0.01\\
95.9	0.01\\
95.91	0.01\\
95.92	0.01\\
95.93	0.01\\
95.94	0.01\\
95.95	0.01\\
95.96	0.01\\
95.97	0.01\\
95.98	0.01\\
95.99	0.01\\
96	0.01\\
96.01	0.01\\
96.02	0.01\\
96.03	0.01\\
96.04	0.01\\
96.05	0.01\\
96.06	0.01\\
96.07	0.01\\
96.08	0.01\\
96.09	0.01\\
96.1	0.01\\
96.11	0.01\\
96.12	0.01\\
96.13	0.01\\
96.14	0.01\\
96.15	0.01\\
96.16	0.01\\
96.17	0.01\\
96.18	0.01\\
96.19	0.01\\
96.2	0.01\\
96.21	0.01\\
96.22	0.01\\
96.23	0.01\\
96.24	0.01\\
96.25	0.01\\
96.26	0.01\\
96.27	0.01\\
96.28	0.01\\
96.29	0.01\\
96.3	0.01\\
96.31	0.01\\
96.32	0.01\\
96.33	0.01\\
96.34	0.01\\
96.35	0.01\\
96.36	0.01\\
96.37	0.01\\
96.38	0.01\\
96.39	0.01\\
96.4	0.01\\
96.41	0.01\\
96.42	0.01\\
96.43	0.01\\
96.44	0.01\\
96.45	0.01\\
96.46	0.01\\
96.47	0.01\\
96.48	0.01\\
96.49	0.01\\
96.5	0.01\\
96.51	0.01\\
96.52	0.01\\
96.53	0.01\\
96.54	0.01\\
96.55	0.01\\
96.56	0.01\\
96.57	0.01\\
96.58	0.01\\
96.59	0.01\\
96.6	0.01\\
96.61	0.01\\
96.62	0.01\\
96.63	0.01\\
96.64	0.01\\
96.65	0.01\\
96.66	0.01\\
96.67	0.01\\
96.68	0.01\\
96.69	0.01\\
96.7	0.01\\
96.71	0.01\\
96.72	0.01\\
96.73	0.01\\
96.74	0.01\\
96.75	0.01\\
96.76	0.01\\
96.77	0.01\\
96.78	0.01\\
96.79	0.01\\
96.8	0.01\\
96.81	0.01\\
96.82	0.01\\
96.83	0.01\\
96.84	0.01\\
96.85	0.01\\
96.86	0.01\\
96.87	0.01\\
96.88	0.01\\
96.89	0.01\\
96.9	0.01\\
96.91	0.01\\
96.92	0.01\\
96.93	0.01\\
96.94	0.01\\
96.95	0.01\\
96.96	0.01\\
96.97	0.01\\
96.98	0.01\\
96.99	0.01\\
97	0.01\\
97.01	0.01\\
97.02	0.01\\
97.03	0.01\\
97.04	0.01\\
97.05	0.01\\
97.06	0.01\\
97.07	0.01\\
97.08	0.01\\
97.09	0.01\\
97.1	0.01\\
97.11	0.01\\
97.12	0.01\\
97.13	0.01\\
97.14	0.01\\
97.15	0.01\\
97.16	0.01\\
97.17	0.01\\
97.18	0.01\\
97.19	0.01\\
97.2	0.01\\
97.21	0.01\\
97.22	0.01\\
97.23	0.01\\
97.24	0.01\\
97.25	0.01\\
97.26	0.01\\
97.27	0.01\\
97.28	0.01\\
97.29	0.01\\
97.3	0.01\\
97.31	0.01\\
97.32	0.01\\
97.33	0.01\\
97.34	0.01\\
97.35	0.01\\
97.36	0.01\\
97.37	0.01\\
97.38	0.01\\
97.39	0.01\\
97.4	0.01\\
97.41	0.01\\
97.42	0.01\\
97.43	0.01\\
97.44	0.01\\
97.45	0.01\\
97.46	0.01\\
97.47	0.01\\
97.48	0.01\\
97.49	0.01\\
97.5	0.01\\
97.51	0.01\\
97.52	0.01\\
97.53	0.01\\
97.54	0.01\\
97.55	0.01\\
97.56	0.01\\
97.57	0.01\\
97.58	0.01\\
97.59	0.01\\
97.6	0.01\\
97.61	0.01\\
97.62	0.01\\
97.63	0.01\\
97.64	0.01\\
97.65	0.01\\
97.66	0.01\\
97.67	0.01\\
97.68	0.01\\
97.69	0.01\\
97.7	0.01\\
97.71	0.01\\
97.72	0.01\\
97.73	0.01\\
97.74	0.01\\
97.75	0.01\\
97.76	0.01\\
97.77	0.01\\
97.78	0.01\\
97.79	0.01\\
97.8	0.01\\
97.81	0.01\\
97.82	0.01\\
97.83	0.01\\
97.84	0.01\\
97.85	0.01\\
97.86	0.01\\
97.87	0.01\\
97.88	0.01\\
97.89	0.01\\
97.9	0.01\\
97.91	0.01\\
97.92	0.01\\
97.93	0.01\\
97.94	0.01\\
97.95	0.01\\
97.96	0.01\\
97.97	0.01\\
97.98	0.01\\
97.99	0.01\\
98	0.01\\
98.01	0.01\\
98.02	0.01\\
98.03	0.01\\
98.04	0.01\\
98.05	0.01\\
98.06	0.01\\
98.07	0.00996678579161849\\
98.08	0.00988240711581398\\
98.09	0.00979736226266535\\
98.1	0.00971164459380387\\
98.11	0.00962524739564042\\
98.12	0.00953816387829279\\
98.13	0.00945038717449195\\
98.14	0.00936191033846656\\
98.15	0.00927272634480508\\
98.16	0.00918282808729498\\
98.17	0.00909220837773839\\
98.18	0.00900085994474335\\
98.19	0.00896750987243684\\
98.2	0.00894769055351984\\
98.21	0.00892772473492099\\
98.22	0.00890759990364751\\
98.23	0.00888731493758286\\
98.24	0.00886686871302045\\
98.25	0.00884626010487904\\
98.26	0.00882548798692697\\
98.27	0.00880455123201535\\
98.28	0.00878344871232064\\
98.29	0.00876217929959689\\
98.3	0.00874074186756428\\
98.31	0.00871913529057698\\
98.32	0.00869735844349814\\
98.33	0.00867541020219135\\
98.34	0.00865328944571452\\
98.35	0.00863099505465544\\
98.36	0.00860852591146884\\
98.37	0.00858585501270448\\
98.38	0.00856297950185548\\
98.39	0.00853989740668253\\
98.4	0.00851660673552339\\
98.41	0.00849310547708979\\
98.42	0.00846939160026171\\
98.43	0.00844546306557663\\
98.44	0.00842131781746397\\
98.45	0.00839695378052997\\
98.46	0.00837236885936255\\
98.47	0.00834756093833781\\
98.48	0.00832252788141995\\
98.49	0.00829726753195918\\
98.5	0.00827177771248734\\
98.51	0.00824605622451141\\
98.52	0.0082201008483047\\
98.53	0.00819390934269583\\
98.54	0.00816747944468203\\
98.55	0.00814080886783424\\
98.56	0.00811389530359031\\
98.57	0.00808673642103387\\
98.58	0.00805932986667078\\
98.59	0.00803167326420301\\
98.6	0.00800376421430013\\
98.61	0.00797560029436814\\
98.62	0.00794717905831582\\
98.63	0.00791849803631834\\
98.64	0.00788955473457836\\
98.65	0.00786034663508427\\
98.66	0.00783087119645829\\
98.67	0.00780112585302229\\
98.68	0.0077711080141796\\
98.69	0.00774081507224251\\
98.7	0.00771024439711441\\
98.71	0.00767939333359036\\
98.72	0.00764825920111835\\
98.73	0.00761683929355836\\
98.74	0.0075851308789345\\
98.75	0.00755313119918636\\
98.76	0.00752083746992069\\
98.77	0.00748824688016081\\
98.78	0.00745535659209369\\
98.79	0.00742216374080337\\
98.8	0.00738866543400933\\
98.81	0.00735485875180528\\
98.82	0.00732074074639563\\
98.83	0.00728630844182654\\
98.84	0.00725155885533516\\
98.85	0.00721648900203891\\
98.86	0.00718109586897355\\
98.87	0.00714537641482925\\
98.88	0.00710932756968403\\
98.89	0.00707294623473479\\
98.9	0.00703622928202585\\
98.91	0.0069991735541748\\
98.92	0.00696177586409594\\
98.93	0.00692403299472094\\
98.94	0.0068859416987171\\
98.95	0.00684749869820272\\
98.96	0.00680870068446\\
98.97	0.00676954431764508\\
98.98	0.00673002622649548\\
98.99	0.00669014300803473\\
99	0.00664989122727423\\
99.01	0.0066092674169123\\
99.02	0.00656826807703044\\
99.03	0.00652688967478666\\
99.04	0.00648512864410604\\
99.05	0.0064429813853683\\
99.06	0.00640044426509238\\
99.07	0.00635751361561826\\
99.08	0.00631418573478554\\
99.09	0.00627045688560921\\
99.1	0.0062263232959523\\
99.11	0.00618178115819542\\
99.12	0.0061368266289033\\
99.13	0.00609145582848807\\
99.14	0.00604566484086953\\
99.15	0.00599944971313209\\
99.16	0.00595280645517858\\
99.17	0.00590573103938076\\
99.18	0.00585821940022658\\
99.19	0.00581026743396408\\
99.2	0.00576187099824206\\
99.21	0.00571302591174716\\
99.22	0.0056637279538378\\
99.23	0.00561397286417443\\
99.24	0.00556375634234653\\
99.25	0.00551307404749593\\
99.26	0.00546192159793664\\
99.27	0.00541029457077126\\
99.28	0.0053581885015035\\
99.29	0.00530559888364735\\
99.3	0.00525252116833238\\
99.31	0.00519895076390542\\
99.32	0.00514488303552852\\
99.33	0.00509031330477306\\
99.34	0.00503523684921011\\
99.35	0.00497964890199689\\
99.36	0.00492354465145946\\
99.37	0.0048669192406713\\
99.38	0.00480976776702815\\
99.39	0.00475208528181867\\
99.4	0.00469386678979122\\
99.41	0.00463510724871647\\
99.42	0.00457580156894593\\
99.43	0.00451594461296638\\
99.44	0.00445553119495003\\
99.45	0.00439455608030059\\
99.46	0.00433301398519493\\
99.47	0.00427089957612052\\
99.48	0.00420820746940855\\
99.49	0.00414493223076259\\
99.5	0.00408106837478292\\
99.51	0.00401661036448631\\
99.52	0.00395155261082136\\
99.53	0.00388588947217923\\
99.54	0.00381961525389985\\
99.55	0.00375272420777347\\
99.56	0.0036852105315375\\
99.57	0.00361706836836867\\
99.58	0.00354829180637043\\
99.59	0.0034788748780555\\
99.6	0.00340881155982366\\
99.61	0.00333809579475858\\
99.62	0.00326672147880132\\
99.63	0.00319468245081585\\
99.64	0.00312197249205576\\
99.65	0.00304858532562616\\
99.66	0.00297451461594047\\
99.67	0.00289975396817214\\
99.68	0.00282429692770135\\
99.69	0.00274813697955651\\
99.7	0.00267126754785049\\
99.71	0.00259368199521177\\
99.72	0.0025153736222102\\
99.73	0.00243633566677747\\
99.74	0.00235656130362214\\
99.75	0.00227604364363933\\
99.76	0.00219477573331486\\
99.77	0.00211275055412382\\
99.78	0.00202996102192371\\
99.79	0.0019463999863418\\
99.8	0.00186206023015686\\
99.81	0.00177693446867524\\
99.82	0.00169101534910107\\
99.83	0.00160429544990068\\
99.84	0.00151676728016119\\
99.85	0.00142842327894306\\
99.86	0.00133925581462683\\
99.87	0.00124925718425371\\
99.88	0.00115841961286012\\
99.89	0.0010667352528062\\
99.9	0.000974196183098004\\
99.91	0.000880794408703629\\
99.92	0.000786521859862974\\
99.93	0.00069137039139123\\
99.94	0.000595331781975987\\
99.95	0.000498397733467977\\
99.96	0.000400559870165304\\
99.97	0.000301809738091234\\
99.98	0.000202138804265397\\
99.99	0.000101538455968435\\
100	0\\
};
\addlegendentry{$q=4$};

\end{axis}
\end{tikzpicture}% 
  \caption{Continuous Time w/ nFPC}
\end{subfigure}%
\hfill%
\begin{subfigure}{.45\linewidth}
  \centering
  \setlength\figureheight{\linewidth} 
  \setlength\figurewidth{\linewidth}
  \tikzsetnextfilename{dm_dscr_nFPC_z1}
  % This file was created by matlab2tikz.
%
%The latest updates can be retrieved from
%  http://www.mathworks.com/matlabcentral/fileexchange/22022-matlab2tikz-matlab2tikz
%where you can also make suggestions and rate matlab2tikz.
%
\definecolor{mycolor1}{rgb}{0.00000,1.00000,0.14286}%
\definecolor{mycolor2}{rgb}{0.00000,1.00000,0.28571}%
\definecolor{mycolor3}{rgb}{0.00000,1.00000,0.42857}%
\definecolor{mycolor4}{rgb}{0.00000,1.00000,0.57143}%
\definecolor{mycolor5}{rgb}{0.00000,1.00000,0.71429}%
\definecolor{mycolor6}{rgb}{0.00000,1.00000,0.85714}%
\definecolor{mycolor7}{rgb}{0.00000,1.00000,1.00000}%
\definecolor{mycolor8}{rgb}{0.00000,0.87500,1.00000}%
\definecolor{mycolor9}{rgb}{0.00000,0.62500,1.00000}%
\definecolor{mycolor10}{rgb}{0.12500,0.00000,1.00000}%
\definecolor{mycolor11}{rgb}{0.25000,0.00000,1.00000}%
\definecolor{mycolor12}{rgb}{0.37500,0.00000,1.00000}%
\definecolor{mycolor13}{rgb}{0.50000,0.00000,1.00000}%
\definecolor{mycolor14}{rgb}{0.62500,0.00000,1.00000}%
\definecolor{mycolor15}{rgb}{0.75000,0.00000,1.00000}%
\definecolor{mycolor16}{rgb}{0.87500,0.00000,1.00000}%
\definecolor{mycolor17}{rgb}{1.00000,0.00000,1.00000}%
\definecolor{mycolor18}{rgb}{1.00000,0.00000,0.87500}%
\definecolor{mycolor19}{rgb}{1.00000,0.00000,0.62500}%
\definecolor{mycolor20}{rgb}{0.85714,0.00000,0.00000}%
\definecolor{mycolor21}{rgb}{0.71429,0.00000,0.00000}%
%
\begin{tikzpicture}[trim axis left, trim axis right]

\begin{axis}[%
width=\figurewidth,
height=\figureheight,
at={(0\figurewidth,0\figureheight)},
scale only axis,
every outer x axis line/.append style={black},
every x tick label/.append style={font=\color{black}},
xmin=0,
xmax=600,
every outer y axis line/.append style={black},
every y tick label/.append style={font=\color{black}},
ymin=0,
ymax=0.014,
axis background/.style={fill=white},
axis x line*=bottom,
axis y line*=left,
yticklabel style={
        /pgf/number format/fixed,
        /pgf/number format/precision=3
},
scaled y ticks=false
]
\addplot [color=green,solid,forget plot]
  table[row sep=crcr]{%
1	0\\
2	0\\
3	0\\
4	0\\
5	0\\
6	0\\
7	0\\
8	0\\
9	0\\
10	0\\
11	0\\
12	0\\
13	0\\
14	0\\
15	0\\
16	0\\
17	0\\
18	0\\
19	0\\
20	0\\
21	0\\
22	0\\
23	0\\
24	0\\
25	0\\
26	0\\
27	0\\
28	0\\
29	0\\
30	0\\
31	0\\
32	0\\
33	0\\
34	0\\
35	0\\
36	0\\
37	0\\
38	0\\
39	0\\
40	0\\
41	0\\
42	0\\
43	0\\
44	0\\
45	0\\
46	0\\
47	0\\
48	0\\
49	0\\
50	0\\
51	0\\
52	0\\
53	0\\
54	0\\
55	0\\
56	0\\
57	0\\
58	0\\
59	0\\
60	0\\
61	0\\
62	0\\
63	0\\
64	0\\
65	0\\
66	0\\
67	0\\
68	0\\
69	0\\
70	0\\
71	0\\
72	0\\
73	0\\
74	0\\
75	0\\
76	0\\
77	0\\
78	0\\
79	0\\
80	0\\
81	0\\
82	0\\
83	0\\
84	0\\
85	0\\
86	0\\
87	0\\
88	0\\
89	0\\
90	0\\
91	0\\
92	0\\
93	0\\
94	0\\
95	0\\
96	0\\
97	0\\
98	0\\
99	0\\
100	0\\
101	0\\
102	0\\
103	0\\
104	0\\
105	0\\
106	0\\
107	0\\
108	0\\
109	0\\
110	0\\
111	0\\
112	0\\
113	0\\
114	0\\
115	0\\
116	0\\
117	0\\
118	0\\
119	0\\
120	0\\
121	0\\
122	0\\
123	0\\
124	0\\
125	0\\
126	0\\
127	0\\
128	0\\
129	0\\
130	0\\
131	0\\
132	0\\
133	0\\
134	0\\
135	0\\
136	0\\
137	0\\
138	0\\
139	0\\
140	0\\
141	0\\
142	0\\
143	0\\
144	0\\
145	0\\
146	0\\
147	0\\
148	0\\
149	0\\
150	0\\
151	0\\
152	0\\
153	0\\
154	0\\
155	0\\
156	0\\
157	0\\
158	0\\
159	0\\
160	0\\
161	0\\
162	0\\
163	0\\
164	0\\
165	0\\
166	0\\
167	0\\
168	0\\
169	0\\
170	0\\
171	0\\
172	0\\
173	0\\
174	0\\
175	0\\
176	0\\
177	0\\
178	0\\
179	0\\
180	0\\
181	0\\
182	0\\
183	0\\
184	0\\
185	0\\
186	0\\
187	0\\
188	0\\
189	0\\
190	0\\
191	0\\
192	0\\
193	0\\
194	0\\
195	0\\
196	0\\
197	0\\
198	0\\
199	0\\
200	0\\
201	0\\
202	0\\
203	0\\
204	0\\
205	0\\
206	0\\
207	0\\
208	0\\
209	0\\
210	0\\
211	0\\
212	0\\
213	0\\
214	0\\
215	0\\
216	0\\
217	0\\
218	0\\
219	0\\
220	0\\
221	0\\
222	0\\
223	0\\
224	0\\
225	0\\
226	0\\
227	0\\
228	0\\
229	0\\
230	0\\
231	0\\
232	0\\
233	0\\
234	0\\
235	0\\
236	0\\
237	0\\
238	0\\
239	0\\
240	0\\
241	0\\
242	0\\
243	0\\
244	0\\
245	0\\
246	0\\
247	0\\
248	0\\
249	0\\
250	0\\
251	0\\
252	0\\
253	0\\
254	0\\
255	0\\
256	0\\
257	0\\
258	0\\
259	0\\
260	0\\
261	0\\
262	0\\
263	0\\
264	0\\
265	0\\
266	0\\
267	0\\
268	0\\
269	0\\
270	0\\
271	0\\
272	0\\
273	0\\
274	0\\
275	0\\
276	0\\
277	0\\
278	0\\
279	0\\
280	0\\
281	0\\
282	0\\
283	0\\
284	0\\
285	0\\
286	0\\
287	0\\
288	0\\
289	0\\
290	0\\
291	0\\
292	0\\
293	0\\
294	0\\
295	0\\
296	0\\
297	0\\
298	0\\
299	0\\
300	0\\
301	0\\
302	0\\
303	0\\
304	0\\
305	0\\
306	0\\
307	0\\
308	0\\
309	0\\
310	0\\
311	0\\
312	0\\
313	0\\
314	0\\
315	0\\
316	0\\
317	0\\
318	0\\
319	0\\
320	0\\
321	0\\
322	0\\
323	0\\
324	0\\
325	0\\
326	0\\
327	0\\
328	0\\
329	0\\
330	0\\
331	0\\
332	0\\
333	0\\
334	0\\
335	0\\
336	0\\
337	0\\
338	0\\
339	0\\
340	0\\
341	0\\
342	0\\
343	0\\
344	0\\
345	0\\
346	0\\
347	0\\
348	0\\
349	0\\
350	0\\
351	0\\
352	0\\
353	0\\
354	0\\
355	0\\
356	0\\
357	0\\
358	0\\
359	0\\
360	0\\
361	0\\
362	0\\
363	0\\
364	0\\
365	0\\
366	0\\
367	0\\
368	0\\
369	0\\
370	0\\
371	0\\
372	0\\
373	0\\
374	0\\
375	0\\
376	0\\
377	0\\
378	0\\
379	0\\
380	0\\
381	0\\
382	0\\
383	0\\
384	0\\
385	0\\
386	0\\
387	0\\
388	0\\
389	0\\
390	0\\
391	0\\
392	0\\
393	0\\
394	0\\
395	0\\
396	0\\
397	0\\
398	0\\
399	0\\
400	0\\
401	0\\
402	0\\
403	0\\
404	0\\
405	0\\
406	0\\
407	0\\
408	0\\
409	0\\
410	0\\
411	0\\
412	0\\
413	0\\
414	0\\
415	0\\
416	0\\
417	0\\
418	0\\
419	0\\
420	0\\
421	0\\
422	0\\
423	0\\
424	0\\
425	0\\
426	0\\
427	0\\
428	0\\
429	0\\
430	0\\
431	0\\
432	0\\
433	0\\
434	0\\
435	0\\
436	0\\
437	0\\
438	0\\
439	0\\
440	0\\
441	0\\
442	0\\
443	0\\
444	0\\
445	0\\
446	0\\
447	0\\
448	0\\
449	0\\
450	0\\
451	0\\
452	0\\
453	0\\
454	0\\
455	0\\
456	0\\
457	0\\
458	0\\
459	0\\
460	0\\
461	0\\
462	0\\
463	0\\
464	0\\
465	0\\
466	0\\
467	0\\
468	0\\
469	0\\
470	0\\
471	0\\
472	0\\
473	0\\
474	0\\
475	0\\
476	0\\
477	0\\
478	0\\
479	0\\
480	0\\
481	0\\
482	0\\
483	0\\
484	0\\
485	0\\
486	0\\
487	0\\
488	0\\
489	0\\
490	0\\
491	0\\
492	0\\
493	0\\
494	0\\
495	0\\
496	0\\
497	0\\
498	0\\
499	0\\
500	0\\
501	0\\
502	0\\
503	0\\
504	0\\
505	0\\
506	0\\
507	0\\
508	0\\
509	0\\
510	0\\
511	0\\
512	0\\
513	0\\
514	0\\
515	0\\
516	0\\
517	0\\
518	0\\
519	0\\
520	0\\
521	0\\
522	0\\
523	0\\
524	0\\
525	0\\
526	0\\
527	0\\
528	0\\
529	0\\
530	0\\
531	0\\
532	0\\
533	0\\
534	0\\
535	0\\
536	0\\
537	0\\
538	0\\
539	1.95527641225376e-05\\
540	4.5298850109529e-05\\
541	7.16515787047177e-05\\
542	9.87260037163583e-05\\
543	0.000126597716204775\\
544	0.000154828213300624\\
545	0.000183516284188123\\
546	0.000212652451694723\\
547	0.000242445887004352\\
548	0.000272977208656028\\
549	0.000304273916678095\\
550	0.000336360067194731\\
551	0.000369260162993882\\
552	0.000402999655010358\\
553	0.000437605098067932\\
554	0.000473104294736734\\
555	0.000509526629431762\\
556	0.000547836300656453\\
557	0.00058668916353381\\
558	0.00062520296878696\\
559	0.000663720153911229\\
560	0.00070323559382204\\
561	0.000743809025123238\\
562	0.000785486013167821\\
563	0.000865379867363339\\
564	0.00126850553815116\\
565	0.00149693684286139\\
566	0.00155900187926996\\
567	0.00162212034359134\\
568	0.00168634680537145\\
569	0.00175171104531127\\
570	0.00181824424140386\\
571	0.00188597918426003\\
572	0.00195495039828782\\
573	0.00202519426953807\\
574	0.0020967491834911\\
575	0.00216965567387592\\
576	0.00224395658348052\\
577	0.00231969723802151\\
578	0.0023969256343423\\
579	0.00247569264466642\\
580	0.00255605223972909\\
581	0.00263806173637132\\
582	0.00272178208245877\\
583	0.00280727821124775\\
584	0.00289461954867252\\
585	0.00298388089415187\\
586	0.00307514426197734\\
587	0.00316850325005475\\
588	0.00326407412228819\\
589	0.00336202479534929\\
590	0.00346265165082637\\
591	0.00356658418066895\\
592	0.00367533141522135\\
593	0.00379274227372929\\
594	0.0039289098819834\\
595	0.00411043972693672\\
596	0.00440815817174734\\
597	0.00501118703192877\\
598	0.00642488516645657\\
599	0\\
600	0\\
};
\addplot [color=mycolor1,solid,forget plot]
  table[row sep=crcr]{%
1	0\\
2	0\\
3	0\\
4	0\\
5	0\\
6	0\\
7	0\\
8	0\\
9	0\\
10	0\\
11	0\\
12	0\\
13	0\\
14	0\\
15	0\\
16	0\\
17	0\\
18	0\\
19	0\\
20	0\\
21	0\\
22	0\\
23	0\\
24	0\\
25	0\\
26	0\\
27	0\\
28	0\\
29	0\\
30	0\\
31	0\\
32	0\\
33	0\\
34	0\\
35	0\\
36	0\\
37	0\\
38	0\\
39	0\\
40	0\\
41	0\\
42	0\\
43	0\\
44	0\\
45	0\\
46	0\\
47	0\\
48	0\\
49	0\\
50	0\\
51	0\\
52	0\\
53	0\\
54	0\\
55	0\\
56	0\\
57	0\\
58	0\\
59	0\\
60	0\\
61	0\\
62	0\\
63	0\\
64	0\\
65	0\\
66	0\\
67	0\\
68	0\\
69	0\\
70	0\\
71	0\\
72	0\\
73	0\\
74	0\\
75	0\\
76	0\\
77	0\\
78	0\\
79	0\\
80	0\\
81	0\\
82	0\\
83	0\\
84	0\\
85	0\\
86	0\\
87	0\\
88	0\\
89	0\\
90	0\\
91	0\\
92	0\\
93	0\\
94	0\\
95	0\\
96	0\\
97	0\\
98	0\\
99	0\\
100	0\\
101	0\\
102	0\\
103	0\\
104	0\\
105	0\\
106	0\\
107	0\\
108	0\\
109	0\\
110	0\\
111	0\\
112	0\\
113	0\\
114	0\\
115	0\\
116	0\\
117	0\\
118	0\\
119	0\\
120	0\\
121	0\\
122	0\\
123	0\\
124	0\\
125	0\\
126	0\\
127	0\\
128	0\\
129	0\\
130	0\\
131	0\\
132	0\\
133	0\\
134	0\\
135	0\\
136	0\\
137	0\\
138	0\\
139	0\\
140	0\\
141	0\\
142	0\\
143	0\\
144	0\\
145	0\\
146	0\\
147	0\\
148	0\\
149	0\\
150	0\\
151	0\\
152	0\\
153	0\\
154	0\\
155	0\\
156	0\\
157	0\\
158	0\\
159	0\\
160	0\\
161	0\\
162	0\\
163	0\\
164	0\\
165	0\\
166	0\\
167	0\\
168	0\\
169	0\\
170	0\\
171	0\\
172	0\\
173	0\\
174	0\\
175	0\\
176	0\\
177	0\\
178	0\\
179	0\\
180	0\\
181	0\\
182	0\\
183	0\\
184	0\\
185	0\\
186	0\\
187	0\\
188	0\\
189	0\\
190	0\\
191	0\\
192	0\\
193	0\\
194	0\\
195	0\\
196	0\\
197	0\\
198	0\\
199	0\\
200	0\\
201	0\\
202	0\\
203	0\\
204	0\\
205	0\\
206	0\\
207	0\\
208	0\\
209	0\\
210	0\\
211	0\\
212	0\\
213	0\\
214	0\\
215	0\\
216	0\\
217	0\\
218	0\\
219	0\\
220	0\\
221	0\\
222	0\\
223	0\\
224	0\\
225	0\\
226	0\\
227	0\\
228	0\\
229	0\\
230	0\\
231	0\\
232	0\\
233	0\\
234	0\\
235	0\\
236	0\\
237	0\\
238	0\\
239	0\\
240	0\\
241	0\\
242	0\\
243	0\\
244	0\\
245	0\\
246	0\\
247	0\\
248	0\\
249	0\\
250	0\\
251	0\\
252	0\\
253	0\\
254	0\\
255	0\\
256	0\\
257	0\\
258	0\\
259	0\\
260	0\\
261	0\\
262	0\\
263	0\\
264	0\\
265	0\\
266	0\\
267	0\\
268	0\\
269	0\\
270	0\\
271	0\\
272	0\\
273	0\\
274	0\\
275	0\\
276	0\\
277	0\\
278	0\\
279	0\\
280	0\\
281	0\\
282	0\\
283	0\\
284	0\\
285	0\\
286	0\\
287	0\\
288	0\\
289	0\\
290	0\\
291	0\\
292	0\\
293	0\\
294	0\\
295	0\\
296	0\\
297	0\\
298	0\\
299	0\\
300	0\\
301	0\\
302	0\\
303	0\\
304	0\\
305	0\\
306	0\\
307	0\\
308	0\\
309	0\\
310	0\\
311	0\\
312	0\\
313	0\\
314	0\\
315	0\\
316	0\\
317	0\\
318	0\\
319	0\\
320	0\\
321	0\\
322	0\\
323	0\\
324	0\\
325	0\\
326	0\\
327	0\\
328	0\\
329	0\\
330	0\\
331	0\\
332	0\\
333	0\\
334	0\\
335	0\\
336	0\\
337	0\\
338	0\\
339	0\\
340	0\\
341	0\\
342	0\\
343	0\\
344	0\\
345	0\\
346	0\\
347	0\\
348	0\\
349	0\\
350	0\\
351	0\\
352	0\\
353	0\\
354	0\\
355	0\\
356	0\\
357	0\\
358	0\\
359	0\\
360	0\\
361	0\\
362	0\\
363	0\\
364	0\\
365	0\\
366	0\\
367	0\\
368	0\\
369	0\\
370	0\\
371	0\\
372	0\\
373	0\\
374	0\\
375	0\\
376	0\\
377	0\\
378	0\\
379	0\\
380	0\\
381	0\\
382	0\\
383	0\\
384	0\\
385	0\\
386	0\\
387	0\\
388	0\\
389	0\\
390	0\\
391	0\\
392	0\\
393	0\\
394	0\\
395	0\\
396	0\\
397	0\\
398	0\\
399	0\\
400	0\\
401	0\\
402	0\\
403	0\\
404	0\\
405	0\\
406	0\\
407	0\\
408	0\\
409	0\\
410	0\\
411	0\\
412	0\\
413	0\\
414	0\\
415	0\\
416	0\\
417	0\\
418	0\\
419	0\\
420	0\\
421	0\\
422	0\\
423	0\\
424	0\\
425	0\\
426	0\\
427	0\\
428	0\\
429	0\\
430	0\\
431	0\\
432	0\\
433	0\\
434	0\\
435	0\\
436	0\\
437	0\\
438	0\\
439	0\\
440	0\\
441	0\\
442	0\\
443	0\\
444	0\\
445	0\\
446	0\\
447	0\\
448	0\\
449	0\\
450	0\\
451	0\\
452	0\\
453	0\\
454	0\\
455	0\\
456	0\\
457	0\\
458	0\\
459	0\\
460	0\\
461	0\\
462	0\\
463	0\\
464	0\\
465	0\\
466	0\\
467	0\\
468	0\\
469	0\\
470	0\\
471	0\\
472	0\\
473	0\\
474	0\\
475	0\\
476	0\\
477	0\\
478	0\\
479	0\\
480	0\\
481	0\\
482	0\\
483	0\\
484	0\\
485	0\\
486	0\\
487	0\\
488	0\\
489	0\\
490	0\\
491	0\\
492	0\\
493	0\\
494	0\\
495	0\\
496	0\\
497	0\\
498	0\\
499	0\\
500	0\\
501	0\\
502	0\\
503	0\\
504	0\\
505	0\\
506	0\\
507	0\\
508	0\\
509	0\\
510	0\\
511	0\\
512	0\\
513	0\\
514	0\\
515	0\\
516	0\\
517	0\\
518	0\\
519	0\\
520	0\\
521	0\\
522	0\\
523	0\\
524	0\\
525	0\\
526	0\\
527	0\\
528	0\\
529	0\\
530	0\\
531	0\\
532	0\\
533	0\\
534	0\\
535	0\\
536	0\\
537	0\\
538	0\\
539	1.37376150958274e-05\\
540	3.93767248397528e-05\\
541	6.56239748089781e-05\\
542	9.24933624265191e-05\\
543	0.000119999855829102\\
544	0.000148363699355699\\
545	0.000177377641200535\\
546	0.000206774145788462\\
547	0.000236720771405114\\
548	0.000267152374965129\\
549	0.00029829523360433\\
550	0.000330215002525051\\
551	0.000362941523011504\\
552	0.000396501057709032\\
553	0.000430920114469899\\
554	0.000466226213023714\\
555	0.000502448117088533\\
556	0.000539615657443634\\
557	0.000578457004892515\\
558	0.000618368790079295\\
559	0.000658014352548393\\
560	0.0006974686375873\\
561	0.000737812075177008\\
562	0.000779241486122308\\
563	0.000821805097695177\\
564	0.000946087951490863\\
565	0.00132524351514214\\
566	0.00155876358354603\\
567	0.0016221192151821\\
568	0.00168634675322425\\
569	0.00175171103773517\\
570	0.00181824423981386\\
571	0.00188597918379922\\
572	0.00195495039810385\\
573	0.00202519426944808\\
574	0.0020967491834472\\
575	0.00216965567385279\\
576	0.00224395658346852\\
577	0.00231969723801546\\
578	0.00239692563433929\\
579	0.00247569264466521\\
580	0.00255605223972864\\
581	0.00263806173637133\\
582	0.00272178208245879\\
583	0.00280727821124775\\
584	0.00289461954867253\\
585	0.00298388089415188\\
586	0.00307514426197735\\
587	0.00316850325005475\\
588	0.00326407412228818\\
589	0.0033620247953493\\
590	0.00346265165082638\\
591	0.00356658418066896\\
592	0.00367533141522136\\
593	0.00379274227372929\\
594	0.00392890988198341\\
595	0.00411043972693672\\
596	0.00440815817174735\\
597	0.00501118703192878\\
598	0.00642488516645657\\
599	0\\
600	0\\
};
\addplot [color=mycolor2,solid,forget plot]
  table[row sep=crcr]{%
1	0\\
2	0\\
3	0\\
4	0\\
5	0\\
6	0\\
7	0\\
8	0\\
9	0\\
10	0\\
11	0\\
12	0\\
13	0\\
14	0\\
15	0\\
16	0\\
17	0\\
18	0\\
19	0\\
20	0\\
21	0\\
22	0\\
23	0\\
24	0\\
25	0\\
26	0\\
27	0\\
28	0\\
29	0\\
30	0\\
31	0\\
32	0\\
33	0\\
34	0\\
35	0\\
36	0\\
37	0\\
38	0\\
39	0\\
40	0\\
41	0\\
42	0\\
43	0\\
44	0\\
45	0\\
46	0\\
47	0\\
48	0\\
49	0\\
50	0\\
51	0\\
52	0\\
53	0\\
54	0\\
55	0\\
56	0\\
57	0\\
58	0\\
59	0\\
60	0\\
61	0\\
62	0\\
63	0\\
64	0\\
65	0\\
66	0\\
67	0\\
68	0\\
69	0\\
70	0\\
71	0\\
72	0\\
73	0\\
74	0\\
75	0\\
76	0\\
77	0\\
78	0\\
79	0\\
80	0\\
81	0\\
82	0\\
83	0\\
84	0\\
85	0\\
86	0\\
87	0\\
88	0\\
89	0\\
90	0\\
91	0\\
92	0\\
93	0\\
94	0\\
95	0\\
96	0\\
97	0\\
98	0\\
99	0\\
100	0\\
101	0\\
102	0\\
103	0\\
104	0\\
105	0\\
106	0\\
107	0\\
108	0\\
109	0\\
110	0\\
111	0\\
112	0\\
113	0\\
114	0\\
115	0\\
116	0\\
117	0\\
118	0\\
119	0\\
120	0\\
121	0\\
122	0\\
123	0\\
124	0\\
125	0\\
126	0\\
127	0\\
128	0\\
129	0\\
130	0\\
131	0\\
132	0\\
133	0\\
134	0\\
135	0\\
136	0\\
137	0\\
138	0\\
139	0\\
140	0\\
141	0\\
142	0\\
143	0\\
144	0\\
145	0\\
146	0\\
147	0\\
148	0\\
149	0\\
150	0\\
151	0\\
152	0\\
153	0\\
154	0\\
155	0\\
156	0\\
157	0\\
158	0\\
159	0\\
160	0\\
161	0\\
162	0\\
163	0\\
164	0\\
165	0\\
166	0\\
167	0\\
168	0\\
169	0\\
170	0\\
171	0\\
172	0\\
173	0\\
174	0\\
175	0\\
176	0\\
177	0\\
178	0\\
179	0\\
180	0\\
181	0\\
182	0\\
183	0\\
184	0\\
185	0\\
186	0\\
187	0\\
188	0\\
189	0\\
190	0\\
191	0\\
192	0\\
193	0\\
194	0\\
195	0\\
196	0\\
197	0\\
198	0\\
199	0\\
200	0\\
201	0\\
202	0\\
203	0\\
204	0\\
205	0\\
206	0\\
207	0\\
208	0\\
209	0\\
210	0\\
211	0\\
212	0\\
213	0\\
214	0\\
215	0\\
216	0\\
217	0\\
218	0\\
219	0\\
220	0\\
221	0\\
222	0\\
223	0\\
224	0\\
225	0\\
226	0\\
227	0\\
228	0\\
229	0\\
230	0\\
231	0\\
232	0\\
233	0\\
234	0\\
235	0\\
236	0\\
237	0\\
238	0\\
239	0\\
240	0\\
241	0\\
242	0\\
243	0\\
244	0\\
245	0\\
246	0\\
247	0\\
248	0\\
249	0\\
250	0\\
251	0\\
252	0\\
253	0\\
254	0\\
255	0\\
256	0\\
257	0\\
258	0\\
259	0\\
260	0\\
261	0\\
262	0\\
263	0\\
264	0\\
265	0\\
266	0\\
267	0\\
268	0\\
269	0\\
270	0\\
271	0\\
272	0\\
273	0\\
274	0\\
275	0\\
276	0\\
277	0\\
278	0\\
279	0\\
280	0\\
281	0\\
282	0\\
283	0\\
284	0\\
285	0\\
286	0\\
287	0\\
288	0\\
289	0\\
290	0\\
291	0\\
292	0\\
293	0\\
294	0\\
295	0\\
296	0\\
297	0\\
298	0\\
299	0\\
300	0\\
301	0\\
302	0\\
303	0\\
304	0\\
305	0\\
306	0\\
307	0\\
308	0\\
309	0\\
310	0\\
311	0\\
312	0\\
313	0\\
314	0\\
315	0\\
316	0\\
317	0\\
318	0\\
319	0\\
320	0\\
321	0\\
322	0\\
323	0\\
324	0\\
325	0\\
326	0\\
327	0\\
328	0\\
329	0\\
330	0\\
331	0\\
332	0\\
333	0\\
334	0\\
335	0\\
336	0\\
337	0\\
338	0\\
339	0\\
340	0\\
341	0\\
342	0\\
343	0\\
344	0\\
345	0\\
346	0\\
347	0\\
348	0\\
349	0\\
350	0\\
351	0\\
352	0\\
353	0\\
354	0\\
355	0\\
356	0\\
357	0\\
358	0\\
359	0\\
360	0\\
361	0\\
362	0\\
363	0\\
364	0\\
365	0\\
366	0\\
367	0\\
368	0\\
369	0\\
370	0\\
371	0\\
372	0\\
373	0\\
374	0\\
375	0\\
376	0\\
377	0\\
378	0\\
379	0\\
380	0\\
381	0\\
382	0\\
383	0\\
384	0\\
385	0\\
386	0\\
387	0\\
388	0\\
389	0\\
390	0\\
391	0\\
392	0\\
393	0\\
394	0\\
395	0\\
396	0\\
397	0\\
398	0\\
399	0\\
400	0\\
401	0\\
402	0\\
403	0\\
404	0\\
405	0\\
406	0\\
407	0\\
408	0\\
409	0\\
410	0\\
411	0\\
412	0\\
413	0\\
414	0\\
415	0\\
416	0\\
417	0\\
418	0\\
419	0\\
420	0\\
421	0\\
422	0\\
423	0\\
424	0\\
425	0\\
426	0\\
427	0\\
428	0\\
429	0\\
430	0\\
431	0\\
432	0\\
433	0\\
434	0\\
435	0\\
436	0\\
437	0\\
438	0\\
439	0\\
440	0\\
441	0\\
442	0\\
443	0\\
444	0\\
445	0\\
446	0\\
447	0\\
448	0\\
449	0\\
450	0\\
451	0\\
452	0\\
453	0\\
454	0\\
455	0\\
456	0\\
457	0\\
458	0\\
459	0\\
460	0\\
461	0\\
462	0\\
463	0\\
464	0\\
465	0\\
466	0\\
467	0\\
468	0\\
469	0\\
470	0\\
471	0\\
472	0\\
473	0\\
474	0\\
475	0\\
476	0\\
477	0\\
478	0\\
479	0\\
480	0\\
481	0\\
482	0\\
483	0\\
484	0\\
485	0\\
486	0\\
487	0\\
488	0\\
489	0\\
490	0\\
491	0\\
492	0\\
493	0\\
494	0\\
495	0\\
496	0\\
497	0\\
498	0\\
499	0\\
500	0\\
501	0\\
502	0\\
503	0\\
504	0\\
505	0\\
506	0\\
507	0\\
508	0\\
509	0\\
510	0\\
511	0\\
512	0\\
513	0\\
514	0\\
515	0\\
516	0\\
517	0\\
518	0\\
519	0\\
520	0\\
521	0\\
522	0\\
523	0\\
524	0\\
525	0\\
526	0\\
527	0\\
528	0\\
529	0\\
530	0\\
531	0\\
532	0\\
533	0\\
534	0\\
535	0\\
536	0\\
537	0\\
538	0\\
539	6.8109836946266e-06\\
540	3.2333550628703e-05\\
541	5.84639163104344e-05\\
542	8.52179610591799e-05\\
543	0.000112610529629426\\
544	0.000140656593656372\\
545	0.000169372996211992\\
546	0.000199020774807513\\
547	0.000229285591716688\\
548	0.00025996464680747\\
549	0.000291246962245206\\
550	0.000323061482549782\\
551	0.000355620274831104\\
552	0.000388995178994957\\
553	0.000423221419650544\\
554	0.000458328095444309\\
555	0.000494344032598358\\
556	0.000531299105939698\\
557	0.000569224327827854\\
558	0.000608484400997936\\
559	0.000649558790110594\\
560	0.000690283174206512\\
561	0.000731090493962589\\
562	0.000772286971587725\\
563	0.000814588754116958\\
564	0.000858056127894556\\
565	0.000992351393135649\\
566	0.00136836372751276\\
567	0.00162000548846045\\
568	0.00168633694558227\\
569	0.00175171060027825\\
570	0.001818244178077\\
571	0.00188597917131468\\
572	0.00195495039465754\\
573	0.00202519426812732\\
574	0.00209674918280833\\
575	0.00216965567354768\\
576	0.00224395658330853\\
577	0.00231969723793263\\
578	0.00239692563429661\\
579	0.00247569264464508\\
580	0.00255605223972036\\
581	0.00263806173636886\\
582	0.00272178208245834\\
583	0.00280727821124774\\
584	0.00289461954867251\\
585	0.00298388089415188\\
586	0.00307514426197732\\
587	0.00316850325005475\\
588	0.00326407412228818\\
589	0.0033620247953493\\
590	0.00346265165082637\\
591	0.00356658418066895\\
592	0.00367533141522134\\
593	0.00379274227372929\\
594	0.00392890988198339\\
595	0.00411043972693671\\
596	0.00440815817174734\\
597	0.00501118703192877\\
598	0.00642488516645657\\
599	0\\
600	0\\
};
\addplot [color=mycolor3,solid,forget plot]
  table[row sep=crcr]{%
1	0\\
2	0\\
3	0\\
4	0\\
5	0\\
6	0\\
7	0\\
8	0\\
9	0\\
10	0\\
11	0\\
12	0\\
13	0\\
14	0\\
15	0\\
16	0\\
17	0\\
18	0\\
19	0\\
20	0\\
21	0\\
22	0\\
23	0\\
24	0\\
25	0\\
26	0\\
27	0\\
28	0\\
29	0\\
30	0\\
31	0\\
32	0\\
33	0\\
34	0\\
35	0\\
36	0\\
37	0\\
38	0\\
39	0\\
40	0\\
41	0\\
42	0\\
43	0\\
44	0\\
45	0\\
46	0\\
47	0\\
48	0\\
49	0\\
50	0\\
51	0\\
52	0\\
53	0\\
54	0\\
55	0\\
56	0\\
57	0\\
58	0\\
59	0\\
60	0\\
61	0\\
62	0\\
63	0\\
64	0\\
65	0\\
66	0\\
67	0\\
68	0\\
69	0\\
70	0\\
71	0\\
72	0\\
73	0\\
74	0\\
75	0\\
76	0\\
77	0\\
78	0\\
79	0\\
80	0\\
81	0\\
82	0\\
83	0\\
84	0\\
85	0\\
86	0\\
87	0\\
88	0\\
89	0\\
90	0\\
91	0\\
92	0\\
93	0\\
94	0\\
95	0\\
96	0\\
97	0\\
98	0\\
99	0\\
100	0\\
101	0\\
102	0\\
103	0\\
104	0\\
105	0\\
106	0\\
107	0\\
108	0\\
109	0\\
110	0\\
111	0\\
112	0\\
113	0\\
114	0\\
115	0\\
116	0\\
117	0\\
118	0\\
119	0\\
120	0\\
121	0\\
122	0\\
123	0\\
124	0\\
125	0\\
126	0\\
127	0\\
128	0\\
129	0\\
130	0\\
131	0\\
132	0\\
133	0\\
134	0\\
135	0\\
136	0\\
137	0\\
138	0\\
139	0\\
140	0\\
141	0\\
142	0\\
143	0\\
144	0\\
145	0\\
146	0\\
147	0\\
148	0\\
149	0\\
150	0\\
151	0\\
152	0\\
153	0\\
154	0\\
155	0\\
156	0\\
157	0\\
158	0\\
159	0\\
160	0\\
161	0\\
162	0\\
163	0\\
164	0\\
165	0\\
166	0\\
167	0\\
168	0\\
169	0\\
170	0\\
171	0\\
172	0\\
173	0\\
174	0\\
175	0\\
176	0\\
177	0\\
178	0\\
179	0\\
180	0\\
181	0\\
182	0\\
183	0\\
184	0\\
185	0\\
186	0\\
187	0\\
188	0\\
189	0\\
190	0\\
191	0\\
192	0\\
193	0\\
194	0\\
195	0\\
196	0\\
197	0\\
198	0\\
199	0\\
200	0\\
201	0\\
202	0\\
203	0\\
204	0\\
205	0\\
206	0\\
207	0\\
208	0\\
209	0\\
210	0\\
211	0\\
212	0\\
213	0\\
214	0\\
215	0\\
216	0\\
217	0\\
218	0\\
219	0\\
220	0\\
221	0\\
222	0\\
223	0\\
224	0\\
225	0\\
226	0\\
227	0\\
228	0\\
229	0\\
230	0\\
231	0\\
232	0\\
233	0\\
234	0\\
235	0\\
236	0\\
237	0\\
238	0\\
239	0\\
240	0\\
241	0\\
242	0\\
243	0\\
244	0\\
245	0\\
246	0\\
247	0\\
248	0\\
249	0\\
250	0\\
251	0\\
252	0\\
253	0\\
254	0\\
255	0\\
256	0\\
257	0\\
258	0\\
259	0\\
260	0\\
261	0\\
262	0\\
263	0\\
264	0\\
265	0\\
266	0\\
267	0\\
268	0\\
269	0\\
270	0\\
271	0\\
272	0\\
273	0\\
274	0\\
275	0\\
276	0\\
277	0\\
278	0\\
279	0\\
280	0\\
281	0\\
282	0\\
283	0\\
284	0\\
285	0\\
286	0\\
287	0\\
288	0\\
289	0\\
290	0\\
291	0\\
292	0\\
293	0\\
294	0\\
295	0\\
296	0\\
297	0\\
298	0\\
299	0\\
300	0\\
301	0\\
302	0\\
303	0\\
304	0\\
305	0\\
306	0\\
307	0\\
308	0\\
309	0\\
310	0\\
311	0\\
312	0\\
313	0\\
314	0\\
315	0\\
316	0\\
317	0\\
318	0\\
319	0\\
320	0\\
321	0\\
322	0\\
323	0\\
324	0\\
325	0\\
326	0\\
327	0\\
328	0\\
329	0\\
330	0\\
331	0\\
332	0\\
333	0\\
334	0\\
335	0\\
336	0\\
337	0\\
338	0\\
339	0\\
340	0\\
341	0\\
342	0\\
343	0\\
344	0\\
345	0\\
346	0\\
347	0\\
348	0\\
349	0\\
350	0\\
351	0\\
352	0\\
353	0\\
354	0\\
355	0\\
356	0\\
357	0\\
358	0\\
359	0\\
360	0\\
361	0\\
362	0\\
363	0\\
364	0\\
365	0\\
366	0\\
367	0\\
368	0\\
369	0\\
370	0\\
371	0\\
372	0\\
373	0\\
374	0\\
375	0\\
376	0\\
377	0\\
378	0\\
379	0\\
380	0\\
381	0\\
382	0\\
383	0\\
384	0\\
385	0\\
386	0\\
387	0\\
388	0\\
389	0\\
390	0\\
391	0\\
392	0\\
393	0\\
394	0\\
395	0\\
396	0\\
397	0\\
398	0\\
399	0\\
400	0\\
401	0\\
402	0\\
403	0\\
404	0\\
405	0\\
406	0\\
407	0\\
408	0\\
409	0\\
410	0\\
411	0\\
412	0\\
413	0\\
414	0\\
415	0\\
416	0\\
417	0\\
418	0\\
419	0\\
420	0\\
421	0\\
422	0\\
423	0\\
424	0\\
425	0\\
426	0\\
427	0\\
428	0\\
429	0\\
430	0\\
431	0\\
432	0\\
433	0\\
434	0\\
435	0\\
436	0\\
437	0\\
438	0\\
439	0\\
440	0\\
441	0\\
442	0\\
443	0\\
444	0\\
445	0\\
446	0\\
447	0\\
448	0\\
449	0\\
450	0\\
451	0\\
452	0\\
453	0\\
454	0\\
455	0\\
456	0\\
457	0\\
458	0\\
459	0\\
460	0\\
461	0\\
462	0\\
463	0\\
464	0\\
465	0\\
466	0\\
467	0\\
468	0\\
469	0\\
470	0\\
471	0\\
472	0\\
473	0\\
474	0\\
475	0\\
476	0\\
477	0\\
478	0\\
479	0\\
480	0\\
481	0\\
482	0\\
483	0\\
484	0\\
485	0\\
486	0\\
487	0\\
488	0\\
489	0\\
490	0\\
491	0\\
492	0\\
493	0\\
494	0\\
495	0\\
496	0\\
497	0\\
498	0\\
499	0\\
500	0\\
501	0\\
502	0\\
503	0\\
504	0\\
505	0\\
506	0\\
507	0\\
508	0\\
509	0\\
510	0\\
511	0\\
512	0\\
513	0\\
514	0\\
515	0\\
516	0\\
517	0\\
518	0\\
519	0\\
520	0\\
521	0\\
522	0\\
523	0\\
524	0\\
525	0\\
526	0\\
527	0\\
528	0\\
529	0\\
530	0\\
531	0\\
532	0\\
533	0\\
534	0\\
535	0\\
536	0\\
537	0\\
538	0\\
539	0\\
540	2.45283203031442e-05\\
541	5.05305641840116e-05\\
542	7.7151130188513e-05\\
543	0.000104409775999626\\
544	0.000132322707189817\\
545	0.000160906048590909\\
546	0.000190176201071463\\
547	0.000220151473717069\\
548	0.000251106602530764\\
549	0.000282689628183228\\
550	0.00031473882969767\\
551	0.000347426881058723\\
552	0.000380710692019464\\
553	0.000414748901432071\\
554	0.000449641734149903\\
555	0.000485433046687176\\
556	0.000522155524764944\\
557	0.000559840336988405\\
558	0.000598520034534356\\
559	0.00063823058415102\\
560	0.000680175666195175\\
561	0.000722313760005713\\
562	0.000764531447846155\\
563	0.000806672231281403\\
564	0.000849854681485079\\
565	0.000894229023457877\\
566	0.00101511639260571\\
567	0.00139640472273508\\
568	0.00166773442770165\\
569	0.00175162603362615\\
570	0.00181824053534768\\
571	0.00188597867140269\\
572	0.00195495029706772\\
573	0.0020251942424614\\
574	0.00209674917340268\\
575	0.00216965566904425\\
576	0.00224395658121081\\
577	0.00231969723683898\\
578	0.0023969256337334\\
579	0.00247569264435266\\
580	0.00255605223958102\\
581	0.00263806173630904\\
582	0.00272178208244085\\
583	0.0028072782112451\\
584	0.00289461954867251\\
585	0.00298388089415188\\
586	0.00307514426197733\\
587	0.00316850325005475\\
588	0.00326407412228818\\
589	0.00336202479534929\\
590	0.00346265165082637\\
591	0.00356658418066895\\
592	0.00367533141522135\\
593	0.00379274227372929\\
594	0.0039289098819834\\
595	0.00411043972693672\\
596	0.00440815817174734\\
597	0.00501118703192877\\
598	0.00642488516645657\\
599	0\\
600	0\\
};
\addplot [color=mycolor4,solid,forget plot]
  table[row sep=crcr]{%
1	0\\
2	0\\
3	0\\
4	0\\
5	0\\
6	0\\
7	0\\
8	0\\
9	0\\
10	0\\
11	0\\
12	0\\
13	0\\
14	0\\
15	0\\
16	0\\
17	0\\
18	0\\
19	0\\
20	0\\
21	0\\
22	0\\
23	0\\
24	0\\
25	0\\
26	0\\
27	0\\
28	0\\
29	0\\
30	0\\
31	0\\
32	0\\
33	0\\
34	0\\
35	0\\
36	0\\
37	0\\
38	0\\
39	0\\
40	0\\
41	0\\
42	0\\
43	0\\
44	0\\
45	0\\
46	0\\
47	0\\
48	0\\
49	0\\
50	0\\
51	0\\
52	0\\
53	0\\
54	0\\
55	0\\
56	0\\
57	0\\
58	0\\
59	0\\
60	0\\
61	0\\
62	0\\
63	0\\
64	0\\
65	0\\
66	0\\
67	0\\
68	0\\
69	0\\
70	0\\
71	0\\
72	0\\
73	0\\
74	0\\
75	0\\
76	0\\
77	0\\
78	0\\
79	0\\
80	0\\
81	0\\
82	0\\
83	0\\
84	0\\
85	0\\
86	0\\
87	0\\
88	0\\
89	0\\
90	0\\
91	0\\
92	0\\
93	0\\
94	0\\
95	0\\
96	0\\
97	0\\
98	0\\
99	0\\
100	0\\
101	0\\
102	0\\
103	0\\
104	0\\
105	0\\
106	0\\
107	0\\
108	0\\
109	0\\
110	0\\
111	0\\
112	0\\
113	0\\
114	0\\
115	0\\
116	0\\
117	0\\
118	0\\
119	0\\
120	0\\
121	0\\
122	0\\
123	0\\
124	0\\
125	0\\
126	0\\
127	0\\
128	0\\
129	0\\
130	0\\
131	0\\
132	0\\
133	0\\
134	0\\
135	0\\
136	0\\
137	0\\
138	0\\
139	0\\
140	0\\
141	0\\
142	0\\
143	0\\
144	0\\
145	0\\
146	0\\
147	0\\
148	0\\
149	0\\
150	0\\
151	0\\
152	0\\
153	0\\
154	0\\
155	0\\
156	0\\
157	0\\
158	0\\
159	0\\
160	0\\
161	0\\
162	0\\
163	0\\
164	0\\
165	0\\
166	0\\
167	0\\
168	0\\
169	0\\
170	0\\
171	0\\
172	0\\
173	0\\
174	0\\
175	0\\
176	0\\
177	0\\
178	0\\
179	0\\
180	0\\
181	0\\
182	0\\
183	0\\
184	0\\
185	0\\
186	0\\
187	0\\
188	0\\
189	0\\
190	0\\
191	0\\
192	0\\
193	0\\
194	0\\
195	0\\
196	0\\
197	0\\
198	0\\
199	0\\
200	0\\
201	0\\
202	0\\
203	0\\
204	0\\
205	0\\
206	0\\
207	0\\
208	0\\
209	0\\
210	0\\
211	0\\
212	0\\
213	0\\
214	0\\
215	0\\
216	0\\
217	0\\
218	0\\
219	0\\
220	0\\
221	0\\
222	0\\
223	0\\
224	0\\
225	0\\
226	0\\
227	0\\
228	0\\
229	0\\
230	0\\
231	0\\
232	0\\
233	0\\
234	0\\
235	0\\
236	0\\
237	0\\
238	0\\
239	0\\
240	0\\
241	0\\
242	0\\
243	0\\
244	0\\
245	0\\
246	0\\
247	0\\
248	0\\
249	0\\
250	0\\
251	0\\
252	0\\
253	0\\
254	0\\
255	0\\
256	0\\
257	0\\
258	0\\
259	0\\
260	0\\
261	0\\
262	0\\
263	0\\
264	0\\
265	0\\
266	0\\
267	0\\
268	0\\
269	0\\
270	0\\
271	0\\
272	0\\
273	0\\
274	0\\
275	0\\
276	0\\
277	0\\
278	0\\
279	0\\
280	0\\
281	0\\
282	0\\
283	0\\
284	0\\
285	0\\
286	0\\
287	0\\
288	0\\
289	0\\
290	0\\
291	0\\
292	0\\
293	0\\
294	0\\
295	0\\
296	0\\
297	0\\
298	0\\
299	0\\
300	0\\
301	0\\
302	0\\
303	0\\
304	0\\
305	0\\
306	0\\
307	0\\
308	0\\
309	0\\
310	0\\
311	0\\
312	0\\
313	0\\
314	0\\
315	0\\
316	0\\
317	0\\
318	0\\
319	0\\
320	0\\
321	0\\
322	0\\
323	0\\
324	0\\
325	0\\
326	0\\
327	0\\
328	0\\
329	0\\
330	0\\
331	0\\
332	0\\
333	0\\
334	0\\
335	0\\
336	0\\
337	0\\
338	0\\
339	0\\
340	0\\
341	0\\
342	0\\
343	0\\
344	0\\
345	0\\
346	0\\
347	0\\
348	0\\
349	0\\
350	0\\
351	0\\
352	0\\
353	0\\
354	0\\
355	0\\
356	0\\
357	0\\
358	0\\
359	0\\
360	0\\
361	0\\
362	0\\
363	0\\
364	0\\
365	0\\
366	0\\
367	0\\
368	0\\
369	0\\
370	0\\
371	0\\
372	0\\
373	0\\
374	0\\
375	0\\
376	0\\
377	0\\
378	0\\
379	0\\
380	0\\
381	0\\
382	0\\
383	0\\
384	0\\
385	0\\
386	0\\
387	0\\
388	0\\
389	0\\
390	0\\
391	0\\
392	0\\
393	0\\
394	0\\
395	0\\
396	0\\
397	0\\
398	0\\
399	0\\
400	0\\
401	0\\
402	0\\
403	0\\
404	0\\
405	0\\
406	0\\
407	0\\
408	0\\
409	0\\
410	0\\
411	0\\
412	0\\
413	0\\
414	0\\
415	0\\
416	0\\
417	0\\
418	0\\
419	0\\
420	0\\
421	0\\
422	0\\
423	0\\
424	0\\
425	0\\
426	0\\
427	0\\
428	0\\
429	0\\
430	0\\
431	0\\
432	0\\
433	0\\
434	0\\
435	0\\
436	0\\
437	0\\
438	0\\
439	0\\
440	0\\
441	0\\
442	0\\
443	0\\
444	0\\
445	0\\
446	0\\
447	0\\
448	0\\
449	0\\
450	0\\
451	0\\
452	0\\
453	0\\
454	0\\
455	0\\
456	0\\
457	0\\
458	0\\
459	0\\
460	0\\
461	0\\
462	0\\
463	0\\
464	0\\
465	0\\
466	0\\
467	0\\
468	0\\
469	0\\
470	0\\
471	0\\
472	0\\
473	0\\
474	0\\
475	0\\
476	0\\
477	0\\
478	0\\
479	0\\
480	0\\
481	0\\
482	0\\
483	0\\
484	0\\
485	0\\
486	0\\
487	0\\
488	0\\
489	0\\
490	0\\
491	0\\
492	0\\
493	0\\
494	0\\
495	0\\
496	0\\
497	0\\
498	0\\
499	0\\
500	0\\
501	0\\
502	0\\
503	0\\
504	0\\
505	0\\
506	0\\
507	0\\
508	0\\
509	0\\
510	0\\
511	0\\
512	0\\
513	0\\
514	0\\
515	0\\
516	0\\
517	0\\
518	0\\
519	0\\
520	0\\
521	0\\
522	0\\
523	0\\
524	0\\
525	0\\
526	0\\
527	0\\
528	0\\
529	0\\
530	0\\
531	0\\
532	0\\
533	0\\
534	0\\
535	0\\
536	0\\
537	0\\
538	0\\
539	0\\
540	1.57801742067035e-05\\
541	4.17209330634838e-05\\
542	6.82521651627544e-05\\
543	9.53742265120166e-05\\
544	0.000123140289849838\\
545	0.000151571903397981\\
546	0.000180687612467819\\
547	0.000210507578298351\\
548	0.000241049124398555\\
549	0.000272332494289059\\
550	0.000304617640894825\\
551	0.000337587086677401\\
552	0.000371095663930957\\
553	0.000405260843325515\\
554	0.000440102586296729\\
555	0.000475683908134913\\
556	0.000512157386629192\\
557	0.000549580513042214\\
558	0.000587986730254416\\
559	0.000627411471795976\\
560	0.00066788979602979\\
561	0.000710048071235101\\
562	0.0007536191055055\\
563	0.000796982169508077\\
564	0.000840690466772013\\
565	0.000884782514078633\\
566	0.000930038853850304\\
567	0.00101562515945223\\
568	0.00140105236470372\\
569	0.00167444769533238\\
570	0.00181751715607354\\
571	0.00188594856162508\\
572	0.00195494627428676\\
573	0.00202519348264098\\
574	0.00209674898273335\\
575	0.00216965560246406\\
576	0.00224395654960764\\
577	0.00231969722252966\\
578	0.0023969256263201\\
579	0.00247569264056459\\
580	0.00255605223759985\\
581	0.00263806173535728\\
582	0.00272178208202001\\
583	0.00280727821111812\\
584	0.00289461954865299\\
585	0.00298388089415187\\
586	0.00307514426197734\\
587	0.00316850325005476\\
588	0.00326407412228819\\
589	0.0033620247953493\\
590	0.00346265165082637\\
591	0.00356658418066895\\
592	0.00367533141522135\\
593	0.0037927422737293\\
594	0.0039289098819834\\
595	0.00411043972693671\\
596	0.00440815817174734\\
597	0.00501118703192877\\
598	0.00642488516645657\\
599	0\\
600	0\\
};
\addplot [color=mycolor5,solid,forget plot]
  table[row sep=crcr]{%
1	0\\
2	0\\
3	0\\
4	0\\
5	0\\
6	0\\
7	0\\
8	0\\
9	0\\
10	0\\
11	0\\
12	0\\
13	0\\
14	0\\
15	0\\
16	0\\
17	0\\
18	0\\
19	0\\
20	0\\
21	0\\
22	0\\
23	0\\
24	0\\
25	0\\
26	0\\
27	0\\
28	0\\
29	0\\
30	0\\
31	0\\
32	0\\
33	0\\
34	0\\
35	0\\
36	0\\
37	0\\
38	0\\
39	0\\
40	0\\
41	0\\
42	0\\
43	0\\
44	0\\
45	0\\
46	0\\
47	0\\
48	0\\
49	0\\
50	0\\
51	0\\
52	0\\
53	0\\
54	0\\
55	0\\
56	0\\
57	0\\
58	0\\
59	0\\
60	0\\
61	0\\
62	0\\
63	0\\
64	0\\
65	0\\
66	0\\
67	0\\
68	0\\
69	0\\
70	0\\
71	0\\
72	0\\
73	0\\
74	0\\
75	0\\
76	0\\
77	0\\
78	0\\
79	0\\
80	0\\
81	0\\
82	0\\
83	0\\
84	0\\
85	0\\
86	0\\
87	0\\
88	0\\
89	0\\
90	0\\
91	0\\
92	0\\
93	0\\
94	0\\
95	0\\
96	0\\
97	0\\
98	0\\
99	0\\
100	0\\
101	0\\
102	0\\
103	0\\
104	0\\
105	0\\
106	0\\
107	0\\
108	0\\
109	0\\
110	0\\
111	0\\
112	0\\
113	0\\
114	0\\
115	0\\
116	0\\
117	0\\
118	0\\
119	0\\
120	0\\
121	0\\
122	0\\
123	0\\
124	0\\
125	0\\
126	0\\
127	0\\
128	0\\
129	0\\
130	0\\
131	0\\
132	0\\
133	0\\
134	0\\
135	0\\
136	0\\
137	0\\
138	0\\
139	0\\
140	0\\
141	0\\
142	0\\
143	0\\
144	0\\
145	0\\
146	0\\
147	0\\
148	0\\
149	0\\
150	0\\
151	0\\
152	0\\
153	0\\
154	0\\
155	0\\
156	0\\
157	0\\
158	0\\
159	0\\
160	0\\
161	0\\
162	0\\
163	0\\
164	0\\
165	0\\
166	0\\
167	0\\
168	0\\
169	0\\
170	0\\
171	0\\
172	0\\
173	0\\
174	0\\
175	0\\
176	0\\
177	0\\
178	0\\
179	0\\
180	0\\
181	0\\
182	0\\
183	0\\
184	0\\
185	0\\
186	0\\
187	0\\
188	0\\
189	0\\
190	0\\
191	0\\
192	0\\
193	0\\
194	0\\
195	0\\
196	0\\
197	0\\
198	0\\
199	0\\
200	0\\
201	0\\
202	0\\
203	0\\
204	0\\
205	0\\
206	0\\
207	0\\
208	0\\
209	0\\
210	0\\
211	0\\
212	0\\
213	0\\
214	0\\
215	0\\
216	0\\
217	0\\
218	0\\
219	0\\
220	0\\
221	0\\
222	0\\
223	0\\
224	0\\
225	0\\
226	0\\
227	0\\
228	0\\
229	0\\
230	0\\
231	0\\
232	0\\
233	0\\
234	0\\
235	0\\
236	0\\
237	0\\
238	0\\
239	0\\
240	0\\
241	0\\
242	0\\
243	0\\
244	0\\
245	0\\
246	0\\
247	0\\
248	0\\
249	0\\
250	0\\
251	0\\
252	0\\
253	0\\
254	0\\
255	0\\
256	0\\
257	0\\
258	0\\
259	0\\
260	0\\
261	0\\
262	0\\
263	0\\
264	0\\
265	0\\
266	0\\
267	0\\
268	0\\
269	0\\
270	0\\
271	0\\
272	0\\
273	0\\
274	0\\
275	0\\
276	0\\
277	0\\
278	0\\
279	0\\
280	0\\
281	0\\
282	0\\
283	0\\
284	0\\
285	0\\
286	0\\
287	0\\
288	0\\
289	0\\
290	0\\
291	0\\
292	0\\
293	0\\
294	0\\
295	0\\
296	0\\
297	0\\
298	0\\
299	0\\
300	0\\
301	0\\
302	0\\
303	0\\
304	0\\
305	0\\
306	0\\
307	0\\
308	0\\
309	0\\
310	0\\
311	0\\
312	0\\
313	0\\
314	0\\
315	0\\
316	0\\
317	0\\
318	0\\
319	0\\
320	0\\
321	0\\
322	0\\
323	0\\
324	0\\
325	0\\
326	0\\
327	0\\
328	0\\
329	0\\
330	0\\
331	0\\
332	0\\
333	0\\
334	0\\
335	0\\
336	0\\
337	0\\
338	0\\
339	0\\
340	0\\
341	0\\
342	0\\
343	0\\
344	0\\
345	0\\
346	0\\
347	0\\
348	0\\
349	0\\
350	0\\
351	0\\
352	0\\
353	0\\
354	0\\
355	0\\
356	0\\
357	0\\
358	0\\
359	0\\
360	0\\
361	0\\
362	0\\
363	0\\
364	0\\
365	0\\
366	0\\
367	0\\
368	0\\
369	0\\
370	0\\
371	0\\
372	0\\
373	0\\
374	0\\
375	0\\
376	0\\
377	0\\
378	0\\
379	0\\
380	0\\
381	0\\
382	0\\
383	0\\
384	0\\
385	0\\
386	0\\
387	0\\
388	0\\
389	0\\
390	0\\
391	0\\
392	0\\
393	0\\
394	0\\
395	0\\
396	0\\
397	0\\
398	0\\
399	0\\
400	0\\
401	0\\
402	0\\
403	0\\
404	0\\
405	0\\
406	0\\
407	0\\
408	0\\
409	0\\
410	0\\
411	0\\
412	0\\
413	0\\
414	0\\
415	0\\
416	0\\
417	0\\
418	0\\
419	0\\
420	0\\
421	0\\
422	0\\
423	0\\
424	0\\
425	0\\
426	0\\
427	0\\
428	0\\
429	0\\
430	0\\
431	0\\
432	0\\
433	0\\
434	0\\
435	0\\
436	0\\
437	0\\
438	0\\
439	0\\
440	0\\
441	0\\
442	0\\
443	0\\
444	0\\
445	0\\
446	0\\
447	0\\
448	0\\
449	0\\
450	0\\
451	0\\
452	0\\
453	0\\
454	0\\
455	0\\
456	0\\
457	0\\
458	0\\
459	0\\
460	0\\
461	0\\
462	0\\
463	0\\
464	0\\
465	0\\
466	0\\
467	0\\
468	0\\
469	0\\
470	0\\
471	0\\
472	0\\
473	0\\
474	0\\
475	0\\
476	0\\
477	0\\
478	0\\
479	0\\
480	0\\
481	0\\
482	0\\
483	0\\
484	0\\
485	0\\
486	0\\
487	0\\
488	0\\
489	0\\
490	0\\
491	0\\
492	0\\
493	0\\
494	0\\
495	0\\
496	0\\
497	0\\
498	0\\
499	0\\
500	0\\
501	0\\
502	0\\
503	0\\
504	0\\
505	0\\
506	0\\
507	0\\
508	0\\
509	0\\
510	0\\
511	0\\
512	0\\
513	0\\
514	0\\
515	0\\
516	0\\
517	0\\
518	0\\
519	0\\
520	0\\
521	0\\
522	0\\
523	0\\
524	0\\
525	0\\
526	0\\
527	0\\
528	0\\
529	0\\
530	0\\
531	0\\
532	0\\
533	0\\
534	0\\
535	0\\
536	0\\
537	0\\
538	0\\
539	0\\
540	5.67779163272291e-06\\
541	3.16144865046584e-05\\
542	5.81200036704712e-05\\
543	8.51921790854403e-05\\
544	0.00011288255189914\\
545	0.000141212350434799\\
546	0.000170189245831659\\
547	0.000199839716824939\\
548	0.000230207356742342\\
549	0.000261309781838766\\
550	0.000293170189040735\\
551	0.000325810513721937\\
552	0.000359445923410564\\
553	0.000393869735973008\\
554	0.000428926673279003\\
555	0.000464639289161119\\
556	0.000501125425574296\\
557	0.00053830708211444\\
558	0.000576434759629047\\
559	0.000615550561680942\\
560	0.000655704173597092\\
561	0.000696934033709092\\
562	0.000739282894716805\\
563	0.000783943930247253\\
564	0.000828931810503418\\
565	0.000874071000879365\\
566	0.000919506942758544\\
567	0.000965676871722533\\
568	0.00101311411328194\\
569	0.00136536274941772\\
570	0.00166568078770398\\
571	0.00187981036938081\\
572	0.00195469927586354\\
573	0.00202516132606539\\
574	0.00209674309368475\\
575	0.00216965418970599\\
576	0.00224395608129885\\
577	0.00231969700168746\\
578	0.00239692552947569\\
579	0.00247569259072984\\
580	0.00255605221240535\\
581	0.00263806172208846\\
582	0.00272178207560973\\
583	0.00280727820819447\\
584	0.00289461954774638\\
585	0.00298388089400896\\
586	0.00307514426197733\\
587	0.00316850325005475\\
588	0.00326407412228819\\
589	0.0033620247953493\\
590	0.00346265165082638\\
591	0.00356658418066895\\
592	0.00367533141522135\\
593	0.00379274227372929\\
594	0.00392890988198341\\
595	0.00411043972693671\\
596	0.00440815817174734\\
597	0.00501118703192877\\
598	0.00642488516645657\\
599	0\\
600	0\\
};
\addplot [color=mycolor6,solid,forget plot]
  table[row sep=crcr]{%
1	0\\
2	0\\
3	0\\
4	0\\
5	0\\
6	0\\
7	0\\
8	0\\
9	0\\
10	0\\
11	0\\
12	0\\
13	0\\
14	0\\
15	0\\
16	0\\
17	0\\
18	0\\
19	0\\
20	0\\
21	0\\
22	0\\
23	0\\
24	0\\
25	0\\
26	0\\
27	0\\
28	0\\
29	0\\
30	0\\
31	0\\
32	0\\
33	0\\
34	0\\
35	0\\
36	0\\
37	0\\
38	0\\
39	0\\
40	0\\
41	0\\
42	0\\
43	0\\
44	0\\
45	0\\
46	0\\
47	0\\
48	0\\
49	0\\
50	0\\
51	0\\
52	0\\
53	0\\
54	0\\
55	0\\
56	0\\
57	0\\
58	0\\
59	0\\
60	0\\
61	0\\
62	0\\
63	0\\
64	0\\
65	0\\
66	0\\
67	0\\
68	0\\
69	0\\
70	0\\
71	0\\
72	0\\
73	0\\
74	0\\
75	0\\
76	0\\
77	0\\
78	0\\
79	0\\
80	0\\
81	0\\
82	0\\
83	0\\
84	0\\
85	0\\
86	0\\
87	0\\
88	0\\
89	0\\
90	0\\
91	0\\
92	0\\
93	0\\
94	0\\
95	0\\
96	0\\
97	0\\
98	0\\
99	0\\
100	0\\
101	0\\
102	0\\
103	0\\
104	0\\
105	0\\
106	0\\
107	0\\
108	0\\
109	0\\
110	0\\
111	0\\
112	0\\
113	0\\
114	0\\
115	0\\
116	0\\
117	0\\
118	0\\
119	0\\
120	0\\
121	0\\
122	0\\
123	0\\
124	0\\
125	0\\
126	0\\
127	0\\
128	0\\
129	0\\
130	0\\
131	0\\
132	0\\
133	0\\
134	0\\
135	0\\
136	0\\
137	0\\
138	0\\
139	0\\
140	0\\
141	0\\
142	0\\
143	0\\
144	0\\
145	0\\
146	0\\
147	0\\
148	0\\
149	0\\
150	0\\
151	0\\
152	0\\
153	0\\
154	0\\
155	0\\
156	0\\
157	0\\
158	0\\
159	0\\
160	0\\
161	0\\
162	0\\
163	0\\
164	0\\
165	0\\
166	0\\
167	0\\
168	0\\
169	0\\
170	0\\
171	0\\
172	0\\
173	0\\
174	0\\
175	0\\
176	0\\
177	0\\
178	0\\
179	0\\
180	0\\
181	0\\
182	0\\
183	0\\
184	0\\
185	0\\
186	0\\
187	0\\
188	0\\
189	0\\
190	0\\
191	0\\
192	0\\
193	0\\
194	0\\
195	0\\
196	0\\
197	0\\
198	0\\
199	0\\
200	0\\
201	0\\
202	0\\
203	0\\
204	0\\
205	0\\
206	0\\
207	0\\
208	0\\
209	0\\
210	0\\
211	0\\
212	0\\
213	0\\
214	0\\
215	0\\
216	0\\
217	0\\
218	0\\
219	0\\
220	0\\
221	0\\
222	0\\
223	0\\
224	0\\
225	0\\
226	0\\
227	0\\
228	0\\
229	0\\
230	0\\
231	0\\
232	0\\
233	0\\
234	0\\
235	0\\
236	0\\
237	0\\
238	0\\
239	0\\
240	0\\
241	0\\
242	0\\
243	0\\
244	0\\
245	0\\
246	0\\
247	0\\
248	0\\
249	0\\
250	0\\
251	0\\
252	0\\
253	0\\
254	0\\
255	0\\
256	0\\
257	0\\
258	0\\
259	0\\
260	0\\
261	0\\
262	0\\
263	0\\
264	0\\
265	0\\
266	0\\
267	0\\
268	0\\
269	0\\
270	0\\
271	0\\
272	0\\
273	0\\
274	0\\
275	0\\
276	0\\
277	0\\
278	0\\
279	0\\
280	0\\
281	0\\
282	0\\
283	0\\
284	0\\
285	0\\
286	0\\
287	0\\
288	0\\
289	0\\
290	0\\
291	0\\
292	0\\
293	0\\
294	0\\
295	0\\
296	0\\
297	0\\
298	0\\
299	0\\
300	0\\
301	0\\
302	0\\
303	0\\
304	0\\
305	0\\
306	0\\
307	0\\
308	0\\
309	0\\
310	0\\
311	0\\
312	0\\
313	0\\
314	0\\
315	0\\
316	0\\
317	0\\
318	0\\
319	0\\
320	0\\
321	0\\
322	0\\
323	0\\
324	0\\
325	0\\
326	0\\
327	0\\
328	0\\
329	0\\
330	0\\
331	0\\
332	0\\
333	0\\
334	0\\
335	0\\
336	0\\
337	0\\
338	0\\
339	0\\
340	0\\
341	0\\
342	0\\
343	0\\
344	0\\
345	0\\
346	0\\
347	0\\
348	0\\
349	0\\
350	0\\
351	0\\
352	0\\
353	0\\
354	0\\
355	0\\
356	0\\
357	0\\
358	0\\
359	0\\
360	0\\
361	0\\
362	0\\
363	0\\
364	0\\
365	0\\
366	0\\
367	0\\
368	0\\
369	0\\
370	0\\
371	0\\
372	0\\
373	0\\
374	0\\
375	0\\
376	0\\
377	0\\
378	0\\
379	0\\
380	0\\
381	0\\
382	0\\
383	0\\
384	0\\
385	0\\
386	0\\
387	0\\
388	0\\
389	0\\
390	0\\
391	0\\
392	0\\
393	0\\
394	0\\
395	0\\
396	0\\
397	0\\
398	0\\
399	0\\
400	0\\
401	0\\
402	0\\
403	0\\
404	0\\
405	0\\
406	0\\
407	0\\
408	0\\
409	0\\
410	0\\
411	0\\
412	0\\
413	0\\
414	0\\
415	0\\
416	0\\
417	0\\
418	0\\
419	0\\
420	0\\
421	0\\
422	0\\
423	0\\
424	0\\
425	0\\
426	0\\
427	0\\
428	0\\
429	0\\
430	0\\
431	0\\
432	0\\
433	0\\
434	0\\
435	0\\
436	0\\
437	0\\
438	0\\
439	0\\
440	0\\
441	0\\
442	0\\
443	0\\
444	0\\
445	0\\
446	0\\
447	0\\
448	0\\
449	0\\
450	0\\
451	0\\
452	0\\
453	0\\
454	0\\
455	0\\
456	0\\
457	0\\
458	0\\
459	0\\
460	0\\
461	0\\
462	0\\
463	0\\
464	0\\
465	0\\
466	0\\
467	0\\
468	0\\
469	0\\
470	0\\
471	0\\
472	0\\
473	0\\
474	0\\
475	0\\
476	0\\
477	0\\
478	0\\
479	0\\
480	0\\
481	0\\
482	0\\
483	0\\
484	0\\
485	0\\
486	0\\
487	0\\
488	0\\
489	0\\
490	0\\
491	0\\
492	0\\
493	0\\
494	0\\
495	0\\
496	0\\
497	0\\
498	0\\
499	0\\
500	0\\
501	0\\
502	0\\
503	0\\
504	0\\
505	0\\
506	0\\
507	0\\
508	0\\
509	0\\
510	0\\
511	0\\
512	0\\
513	0\\
514	0\\
515	0\\
516	0\\
517	0\\
518	0\\
519	0\\
520	0\\
521	0\\
522	0\\
523	0\\
524	0\\
525	0\\
526	0\\
527	0\\
528	0\\
529	0\\
530	0\\
531	0\\
532	0\\
533	0\\
534	0\\
535	0\\
536	0\\
537	0\\
538	0\\
539	0\\
540	0\\
541	1.99792072081089e-05\\
542	4.63727525165937e-05\\
543	7.34212262876605e-05\\
544	0.000101084696376327\\
545	0.000129373228443587\\
546	0.000158292503572227\\
547	0.000187861500603815\\
548	0.000218122519240035\\
549	0.00024909286214484\\
550	0.000280775360288263\\
551	0.000313225176815619\\
552	0.000346470143581298\\
553	0.00038053214301298\\
554	0.000415556128944413\\
555	0.000451524249748653\\
556	0.00048824208518114\\
557	0.000525592789678638\\
558	0.000563780851805092\\
559	0.000602776718153804\\
560	0.00064265699652178\\
561	0.000683574413147921\\
562	0.000725585257827588\\
563	0.000768736465274667\\
564	0.000813472406506441\\
565	0.000859930031446736\\
566	0.000906633225702796\\
567	0.000953771917621697\\
568	0.00100124266870638\\
569	0.00104967338643609\\
570	0.0013043068138501\\
571	0.00163930374758022\\
572	0.00190309861278883\\
573	0.00202315071336391\\
574	0.00209648783523968\\
575	0.00216960877787299\\
576	0.00224394564255947\\
577	0.00231969373006616\\
578	0.00239692399237447\\
579	0.00247569194043164\\
580	0.00255605188006998\\
581	0.00263806155640494\\
582	0.0027217819877466\\
583	0.0028072781656434\\
584	0.00289461952769629\\
585	0.00298388088761337\\
586	0.00307514426094347\\
587	0.00316850325005476\\
588	0.00326407412228819\\
589	0.00336202479534929\\
590	0.00346265165082637\\
591	0.00356658418066895\\
592	0.00367533141522134\\
593	0.00379274227372929\\
594	0.0039289098819834\\
595	0.00411043972693671\\
596	0.00440815817174734\\
597	0.00501118703192877\\
598	0.00642488516645657\\
599	0\\
600	0\\
};
\addplot [color=mycolor7,solid,forget plot]
  table[row sep=crcr]{%
1	0\\
2	0\\
3	0\\
4	0\\
5	0\\
6	0\\
7	0\\
8	0\\
9	0\\
10	0\\
11	0\\
12	0\\
13	0\\
14	0\\
15	0\\
16	0\\
17	0\\
18	0\\
19	0\\
20	0\\
21	0\\
22	0\\
23	0\\
24	0\\
25	0\\
26	0\\
27	0\\
28	0\\
29	0\\
30	0\\
31	0\\
32	0\\
33	0\\
34	0\\
35	0\\
36	0\\
37	0\\
38	0\\
39	0\\
40	0\\
41	0\\
42	0\\
43	0\\
44	0\\
45	0\\
46	0\\
47	0\\
48	0\\
49	0\\
50	0\\
51	0\\
52	0\\
53	0\\
54	0\\
55	0\\
56	0\\
57	0\\
58	0\\
59	0\\
60	0\\
61	0\\
62	0\\
63	0\\
64	0\\
65	0\\
66	0\\
67	0\\
68	0\\
69	0\\
70	0\\
71	0\\
72	0\\
73	0\\
74	0\\
75	0\\
76	0\\
77	0\\
78	0\\
79	0\\
80	0\\
81	0\\
82	0\\
83	0\\
84	0\\
85	0\\
86	0\\
87	0\\
88	0\\
89	0\\
90	0\\
91	0\\
92	0\\
93	0\\
94	0\\
95	0\\
96	0\\
97	0\\
98	0\\
99	0\\
100	0\\
101	0\\
102	0\\
103	0\\
104	0\\
105	0\\
106	0\\
107	0\\
108	0\\
109	0\\
110	0\\
111	0\\
112	0\\
113	0\\
114	0\\
115	0\\
116	0\\
117	0\\
118	0\\
119	0\\
120	0\\
121	0\\
122	0\\
123	0\\
124	0\\
125	0\\
126	0\\
127	0\\
128	0\\
129	0\\
130	0\\
131	0\\
132	0\\
133	0\\
134	0\\
135	0\\
136	0\\
137	0\\
138	0\\
139	0\\
140	0\\
141	0\\
142	0\\
143	0\\
144	0\\
145	0\\
146	0\\
147	0\\
148	0\\
149	0\\
150	0\\
151	0\\
152	0\\
153	0\\
154	0\\
155	0\\
156	0\\
157	0\\
158	0\\
159	0\\
160	0\\
161	0\\
162	0\\
163	0\\
164	0\\
165	0\\
166	0\\
167	0\\
168	0\\
169	0\\
170	0\\
171	0\\
172	0\\
173	0\\
174	0\\
175	0\\
176	0\\
177	0\\
178	0\\
179	0\\
180	0\\
181	0\\
182	0\\
183	0\\
184	0\\
185	0\\
186	0\\
187	0\\
188	0\\
189	0\\
190	0\\
191	0\\
192	0\\
193	0\\
194	0\\
195	0\\
196	0\\
197	0\\
198	0\\
199	0\\
200	0\\
201	0\\
202	0\\
203	0\\
204	0\\
205	0\\
206	0\\
207	0\\
208	0\\
209	0\\
210	0\\
211	0\\
212	0\\
213	0\\
214	0\\
215	0\\
216	0\\
217	0\\
218	0\\
219	0\\
220	0\\
221	0\\
222	0\\
223	0\\
224	0\\
225	0\\
226	0\\
227	0\\
228	0\\
229	0\\
230	0\\
231	0\\
232	0\\
233	0\\
234	0\\
235	0\\
236	0\\
237	0\\
238	0\\
239	0\\
240	0\\
241	0\\
242	0\\
243	0\\
244	0\\
245	0\\
246	0\\
247	0\\
248	0\\
249	0\\
250	0\\
251	0\\
252	0\\
253	0\\
254	0\\
255	0\\
256	0\\
257	0\\
258	0\\
259	0\\
260	0\\
261	0\\
262	0\\
263	0\\
264	0\\
265	0\\
266	0\\
267	0\\
268	0\\
269	0\\
270	0\\
271	0\\
272	0\\
273	0\\
274	0\\
275	0\\
276	0\\
277	0\\
278	0\\
279	0\\
280	0\\
281	0\\
282	0\\
283	0\\
284	0\\
285	0\\
286	0\\
287	0\\
288	0\\
289	0\\
290	0\\
291	0\\
292	0\\
293	0\\
294	0\\
295	0\\
296	0\\
297	0\\
298	0\\
299	0\\
300	0\\
301	0\\
302	0\\
303	0\\
304	0\\
305	0\\
306	0\\
307	0\\
308	0\\
309	0\\
310	0\\
311	0\\
312	0\\
313	0\\
314	0\\
315	0\\
316	0\\
317	0\\
318	0\\
319	0\\
320	0\\
321	0\\
322	0\\
323	0\\
324	0\\
325	0\\
326	0\\
327	0\\
328	0\\
329	0\\
330	0\\
331	0\\
332	0\\
333	0\\
334	0\\
335	0\\
336	0\\
337	0\\
338	0\\
339	0\\
340	0\\
341	0\\
342	0\\
343	0\\
344	0\\
345	0\\
346	0\\
347	0\\
348	0\\
349	0\\
350	0\\
351	0\\
352	0\\
353	0\\
354	0\\
355	0\\
356	0\\
357	0\\
358	0\\
359	0\\
360	0\\
361	0\\
362	0\\
363	0\\
364	0\\
365	0\\
366	0\\
367	0\\
368	0\\
369	0\\
370	0\\
371	0\\
372	0\\
373	0\\
374	0\\
375	0\\
376	0\\
377	0\\
378	0\\
379	0\\
380	0\\
381	0\\
382	0\\
383	0\\
384	0\\
385	0\\
386	0\\
387	0\\
388	0\\
389	0\\
390	0\\
391	0\\
392	0\\
393	0\\
394	0\\
395	0\\
396	0\\
397	0\\
398	0\\
399	0\\
400	0\\
401	0\\
402	0\\
403	0\\
404	0\\
405	0\\
406	0\\
407	0\\
408	0\\
409	0\\
410	0\\
411	0\\
412	0\\
413	0\\
414	0\\
415	0\\
416	0\\
417	0\\
418	0\\
419	0\\
420	0\\
421	0\\
422	0\\
423	0\\
424	0\\
425	0\\
426	0\\
427	0\\
428	0\\
429	0\\
430	0\\
431	0\\
432	0\\
433	0\\
434	0\\
435	0\\
436	0\\
437	0\\
438	0\\
439	0\\
440	0\\
441	0\\
442	0\\
443	0\\
444	0\\
445	0\\
446	0\\
447	0\\
448	0\\
449	0\\
450	0\\
451	0\\
452	0\\
453	0\\
454	0\\
455	0\\
456	0\\
457	0\\
458	0\\
459	0\\
460	0\\
461	0\\
462	0\\
463	0\\
464	0\\
465	0\\
466	0\\
467	0\\
468	0\\
469	0\\
470	0\\
471	0\\
472	0\\
473	0\\
474	0\\
475	0\\
476	0\\
477	0\\
478	0\\
479	0\\
480	0\\
481	0\\
482	0\\
483	0\\
484	0\\
485	0\\
486	0\\
487	0\\
488	0\\
489	0\\
490	0\\
491	0\\
492	0\\
493	0\\
494	0\\
495	0\\
496	0\\
497	0\\
498	0\\
499	0\\
500	0\\
501	0\\
502	0\\
503	0\\
504	0\\
505	0\\
506	0\\
507	0\\
508	0\\
509	0\\
510	0\\
511	0\\
512	0\\
513	0\\
514	0\\
515	0\\
516	0\\
517	0\\
518	0\\
519	0\\
520	0\\
521	0\\
522	0\\
523	0\\
524	0\\
525	0\\
526	0\\
527	0\\
528	0\\
529	0\\
530	0\\
531	0\\
532	0\\
533	0\\
534	0\\
535	0\\
536	0\\
537	0\\
538	0\\
539	0\\
540	0\\
541	6.76881387420085e-06\\
542	3.30164558502364e-05\\
543	5.98611904804831e-05\\
544	8.7334079349987e-05\\
545	0.000115457856912956\\
546	0.000144277166012184\\
547	0.000173793755482025\\
548	0.000203981423139992\\
549	0.000234866214895428\\
550	0.000266443042311468\\
551	0.00029876073672643\\
552	0.000331847287909978\\
553	0.000365721898928622\\
554	0.000400391322970895\\
555	0.000435922616984166\\
556	0.000472350252675263\\
557	0.000509932211154596\\
558	0.000548310927633405\\
559	0.000587456131965322\\
560	0.000627403331006158\\
561	0.000668274334551636\\
562	0.000710017142892241\\
563	0.000752802951629782\\
564	0.000796728665476221\\
565	0.000841860817331624\\
566	0.000888906921322659\\
567	0.000937293589642778\\
568	0.000985960275563987\\
569	0.0010352558317612\\
570	0.00108500923535093\\
571	0.00122235714586162\\
572	0.00157519138430146\\
573	0.0018518308403065\\
574	0.00208025282367438\\
575	0.00216759714565874\\
576	0.00224359741223466\\
577	0.0023196168261704\\
578	0.00239690130315676\\
579	0.00247568128212639\\
580	0.00255604754780296\\
581	0.00263805935698486\\
582	0.00272178091061339\\
583	0.00280727759029156\\
584	0.00289461924949013\\
585	0.00298388075192783\\
586	0.00307514421637755\\
587	0.00316850324266257\\
588	0.00326407412228818\\
589	0.00336202479534929\\
590	0.00346265165082637\\
591	0.00356658418066895\\
592	0.00367533141522135\\
593	0.00379274227372929\\
594	0.0039289098819834\\
595	0.00411043972693672\\
596	0.00440815817174734\\
597	0.00501118703192877\\
598	0.00642488516645657\\
599	0\\
600	0\\
};
\addplot [color=mycolor8,solid,forget plot]
  table[row sep=crcr]{%
1	0\\
2	0\\
3	0\\
4	0\\
5	0\\
6	0\\
7	0\\
8	0\\
9	0\\
10	0\\
11	0\\
12	0\\
13	0\\
14	0\\
15	0\\
16	0\\
17	0\\
18	0\\
19	0\\
20	0\\
21	0\\
22	0\\
23	0\\
24	0\\
25	0\\
26	0\\
27	0\\
28	0\\
29	0\\
30	0\\
31	0\\
32	0\\
33	0\\
34	0\\
35	0\\
36	0\\
37	0\\
38	0\\
39	0\\
40	0\\
41	0\\
42	0\\
43	0\\
44	0\\
45	0\\
46	0\\
47	0\\
48	0\\
49	0\\
50	0\\
51	0\\
52	0\\
53	0\\
54	0\\
55	0\\
56	0\\
57	0\\
58	0\\
59	0\\
60	0\\
61	0\\
62	0\\
63	0\\
64	0\\
65	0\\
66	0\\
67	0\\
68	0\\
69	0\\
70	0\\
71	0\\
72	0\\
73	0\\
74	0\\
75	0\\
76	0\\
77	0\\
78	0\\
79	0\\
80	0\\
81	0\\
82	0\\
83	0\\
84	0\\
85	0\\
86	0\\
87	0\\
88	0\\
89	0\\
90	0\\
91	0\\
92	0\\
93	0\\
94	0\\
95	0\\
96	0\\
97	0\\
98	0\\
99	0\\
100	0\\
101	0\\
102	0\\
103	0\\
104	0\\
105	0\\
106	0\\
107	0\\
108	0\\
109	0\\
110	0\\
111	0\\
112	0\\
113	0\\
114	0\\
115	0\\
116	0\\
117	0\\
118	0\\
119	0\\
120	0\\
121	0\\
122	0\\
123	0\\
124	0\\
125	0\\
126	0\\
127	0\\
128	0\\
129	0\\
130	0\\
131	0\\
132	0\\
133	0\\
134	0\\
135	0\\
136	0\\
137	0\\
138	0\\
139	0\\
140	0\\
141	0\\
142	0\\
143	0\\
144	0\\
145	0\\
146	0\\
147	0\\
148	0\\
149	0\\
150	0\\
151	0\\
152	0\\
153	0\\
154	0\\
155	0\\
156	0\\
157	0\\
158	0\\
159	0\\
160	0\\
161	0\\
162	0\\
163	0\\
164	0\\
165	0\\
166	0\\
167	0\\
168	0\\
169	0\\
170	0\\
171	0\\
172	0\\
173	0\\
174	0\\
175	0\\
176	0\\
177	0\\
178	0\\
179	0\\
180	0\\
181	0\\
182	0\\
183	0\\
184	0\\
185	0\\
186	0\\
187	0\\
188	0\\
189	0\\
190	0\\
191	0\\
192	0\\
193	0\\
194	0\\
195	0\\
196	0\\
197	0\\
198	0\\
199	0\\
200	0\\
201	0\\
202	0\\
203	0\\
204	0\\
205	0\\
206	0\\
207	0\\
208	0\\
209	0\\
210	0\\
211	0\\
212	0\\
213	0\\
214	0\\
215	0\\
216	0\\
217	0\\
218	0\\
219	0\\
220	0\\
221	0\\
222	0\\
223	0\\
224	0\\
225	0\\
226	0\\
227	0\\
228	0\\
229	0\\
230	0\\
231	0\\
232	0\\
233	0\\
234	0\\
235	0\\
236	0\\
237	0\\
238	0\\
239	0\\
240	0\\
241	0\\
242	0\\
243	0\\
244	0\\
245	0\\
246	0\\
247	0\\
248	0\\
249	0\\
250	0\\
251	0\\
252	0\\
253	0\\
254	0\\
255	0\\
256	0\\
257	0\\
258	0\\
259	0\\
260	0\\
261	0\\
262	0\\
263	0\\
264	0\\
265	0\\
266	0\\
267	0\\
268	0\\
269	0\\
270	0\\
271	0\\
272	0\\
273	0\\
274	0\\
275	0\\
276	0\\
277	0\\
278	0\\
279	0\\
280	0\\
281	0\\
282	0\\
283	0\\
284	0\\
285	0\\
286	0\\
287	0\\
288	0\\
289	0\\
290	0\\
291	0\\
292	0\\
293	0\\
294	0\\
295	0\\
296	0\\
297	0\\
298	0\\
299	0\\
300	0\\
301	0\\
302	0\\
303	0\\
304	0\\
305	0\\
306	0\\
307	0\\
308	0\\
309	0\\
310	0\\
311	0\\
312	0\\
313	0\\
314	0\\
315	0\\
316	0\\
317	0\\
318	0\\
319	0\\
320	0\\
321	0\\
322	0\\
323	0\\
324	0\\
325	0\\
326	0\\
327	0\\
328	0\\
329	0\\
330	0\\
331	0\\
332	0\\
333	0\\
334	0\\
335	0\\
336	0\\
337	0\\
338	0\\
339	0\\
340	0\\
341	0\\
342	0\\
343	0\\
344	0\\
345	0\\
346	0\\
347	0\\
348	0\\
349	0\\
350	0\\
351	0\\
352	0\\
353	0\\
354	0\\
355	0\\
356	0\\
357	0\\
358	0\\
359	0\\
360	0\\
361	0\\
362	0\\
363	0\\
364	0\\
365	0\\
366	0\\
367	0\\
368	0\\
369	0\\
370	0\\
371	0\\
372	0\\
373	0\\
374	0\\
375	0\\
376	0\\
377	0\\
378	0\\
379	0\\
380	0\\
381	0\\
382	0\\
383	0\\
384	0\\
385	0\\
386	0\\
387	0\\
388	0\\
389	0\\
390	0\\
391	0\\
392	0\\
393	0\\
394	0\\
395	0\\
396	0\\
397	0\\
398	0\\
399	0\\
400	0\\
401	0\\
402	0\\
403	0\\
404	0\\
405	0\\
406	0\\
407	0\\
408	0\\
409	0\\
410	0\\
411	0\\
412	0\\
413	0\\
414	0\\
415	0\\
416	0\\
417	0\\
418	0\\
419	0\\
420	0\\
421	0\\
422	0\\
423	0\\
424	0\\
425	0\\
426	0\\
427	0\\
428	0\\
429	0\\
430	0\\
431	0\\
432	0\\
433	0\\
434	0\\
435	0\\
436	0\\
437	0\\
438	0\\
439	0\\
440	0\\
441	0\\
442	0\\
443	0\\
444	0\\
445	0\\
446	0\\
447	0\\
448	0\\
449	0\\
450	0\\
451	0\\
452	0\\
453	0\\
454	0\\
455	0\\
456	0\\
457	0\\
458	0\\
459	0\\
460	0\\
461	0\\
462	0\\
463	0\\
464	0\\
465	0\\
466	0\\
467	0\\
468	0\\
469	0\\
470	0\\
471	0\\
472	0\\
473	0\\
474	0\\
475	0\\
476	0\\
477	0\\
478	0\\
479	0\\
480	0\\
481	0\\
482	0\\
483	0\\
484	0\\
485	0\\
486	0\\
487	0\\
488	0\\
489	0\\
490	0\\
491	0\\
492	0\\
493	0\\
494	0\\
495	0\\
496	0\\
497	0\\
498	0\\
499	0\\
500	0\\
501	0\\
502	0\\
503	0\\
504	0\\
505	0\\
506	0\\
507	0\\
508	0\\
509	0\\
510	0\\
511	0\\
512	0\\
513	0\\
514	0\\
515	0\\
516	0\\
517	0\\
518	0\\
519	0\\
520	0\\
521	0\\
522	0\\
523	0\\
524	0\\
525	0\\
526	0\\
527	0\\
528	0\\
529	0\\
530	0\\
531	0\\
532	0\\
533	0\\
534	0\\
535	0\\
536	0\\
537	0\\
538	0\\
539	0\\
540	0\\
541	0\\
542	1.77994563498494e-05\\
543	4.44690418034569e-05\\
544	7.17504882016815e-05\\
545	9.96653124370045e-05\\
546	0.000128247224210947\\
547	0.000157508764958893\\
548	0.000187477130071588\\
549	0.000218174077853842\\
550	0.00024965718467097\\
551	0.000281882535055856\\
552	0.000314854645582878\\
553	0.000348597040202349\\
554	0.000383112689282817\\
555	0.00041846178624356\\
556	0.000454668625667846\\
557	0.000491753884948208\\
558	0.000529732799945901\\
559	0.000568792410221848\\
560	0.000608906511575984\\
561	0.000649904672830707\\
562	0.000691724311605649\\
563	0.000734489579471478\\
564	0.000778277104886136\\
565	0.00082303473115782\\
566	0.000868959770003612\\
567	0.000916145326157647\\
568	0.000965414946849161\\
569	0.0010158882089616\\
570	0.00106677817567325\\
571	0.00111839899996879\\
572	0.00117070764637285\\
573	0.00145155287937048\\
574	0.00177963609047075\\
575	0.00203805474797816\\
576	0.00222786492014083\\
577	0.00231696253384747\\
578	0.00239633649011414\\
579	0.00247552515937604\\
580	0.00255597390481131\\
581	0.0026380307249917\\
582	0.00272176645872677\\
583	0.00280727066893297\\
584	0.00289461552253682\\
585	0.00298387896151858\\
586	0.00307514331064204\\
587	0.00316850293601066\\
588	0.00326407407006435\\
589	0.00336202479534929\\
590	0.00346265165082637\\
591	0.00356658418066895\\
592	0.00367533141522135\\
593	0.0037927422737293\\
594	0.00392890988198341\\
595	0.00411043972693672\\
596	0.00440815817174735\\
597	0.00501118703192877\\
598	0.00642488516645657\\
599	0\\
600	0\\
};
\addplot [color=blue!25!mycolor7,solid,forget plot]
  table[row sep=crcr]{%
1	0\\
2	0\\
3	0\\
4	0\\
5	0\\
6	0\\
7	0\\
8	0\\
9	0\\
10	0\\
11	0\\
12	0\\
13	0\\
14	0\\
15	0\\
16	0\\
17	0\\
18	0\\
19	0\\
20	0\\
21	0\\
22	0\\
23	0\\
24	0\\
25	0\\
26	0\\
27	0\\
28	0\\
29	0\\
30	0\\
31	0\\
32	0\\
33	0\\
34	0\\
35	0\\
36	0\\
37	0\\
38	0\\
39	0\\
40	0\\
41	0\\
42	0\\
43	0\\
44	0\\
45	0\\
46	0\\
47	0\\
48	0\\
49	0\\
50	0\\
51	0\\
52	0\\
53	0\\
54	0\\
55	0\\
56	0\\
57	0\\
58	0\\
59	0\\
60	0\\
61	0\\
62	0\\
63	0\\
64	0\\
65	0\\
66	0\\
67	0\\
68	0\\
69	0\\
70	0\\
71	0\\
72	0\\
73	0\\
74	0\\
75	0\\
76	0\\
77	0\\
78	0\\
79	0\\
80	0\\
81	0\\
82	0\\
83	0\\
84	0\\
85	0\\
86	0\\
87	0\\
88	0\\
89	0\\
90	0\\
91	0\\
92	0\\
93	0\\
94	0\\
95	0\\
96	0\\
97	0\\
98	0\\
99	0\\
100	0\\
101	0\\
102	0\\
103	0\\
104	0\\
105	0\\
106	0\\
107	0\\
108	0\\
109	0\\
110	0\\
111	0\\
112	0\\
113	0\\
114	0\\
115	0\\
116	0\\
117	0\\
118	0\\
119	0\\
120	0\\
121	0\\
122	0\\
123	0\\
124	0\\
125	0\\
126	0\\
127	0\\
128	0\\
129	0\\
130	0\\
131	0\\
132	0\\
133	0\\
134	0\\
135	0\\
136	0\\
137	0\\
138	0\\
139	0\\
140	0\\
141	0\\
142	0\\
143	0\\
144	0\\
145	0\\
146	0\\
147	0\\
148	0\\
149	0\\
150	0\\
151	0\\
152	0\\
153	0\\
154	0\\
155	0\\
156	0\\
157	0\\
158	0\\
159	0\\
160	0\\
161	0\\
162	0\\
163	0\\
164	0\\
165	0\\
166	0\\
167	0\\
168	0\\
169	0\\
170	0\\
171	0\\
172	0\\
173	0\\
174	0\\
175	0\\
176	0\\
177	0\\
178	0\\
179	0\\
180	0\\
181	0\\
182	0\\
183	0\\
184	0\\
185	0\\
186	0\\
187	0\\
188	0\\
189	0\\
190	0\\
191	0\\
192	0\\
193	0\\
194	0\\
195	0\\
196	0\\
197	0\\
198	0\\
199	0\\
200	0\\
201	0\\
202	0\\
203	0\\
204	0\\
205	0\\
206	0\\
207	0\\
208	0\\
209	0\\
210	0\\
211	0\\
212	0\\
213	0\\
214	0\\
215	0\\
216	0\\
217	0\\
218	0\\
219	0\\
220	0\\
221	0\\
222	0\\
223	0\\
224	0\\
225	0\\
226	0\\
227	0\\
228	0\\
229	0\\
230	0\\
231	0\\
232	0\\
233	0\\
234	0\\
235	0\\
236	0\\
237	0\\
238	0\\
239	0\\
240	0\\
241	0\\
242	0\\
243	0\\
244	0\\
245	0\\
246	0\\
247	0\\
248	0\\
249	0\\
250	0\\
251	0\\
252	0\\
253	0\\
254	0\\
255	0\\
256	0\\
257	0\\
258	0\\
259	0\\
260	0\\
261	0\\
262	0\\
263	0\\
264	0\\
265	0\\
266	0\\
267	0\\
268	0\\
269	0\\
270	0\\
271	0\\
272	0\\
273	0\\
274	0\\
275	0\\
276	0\\
277	0\\
278	0\\
279	0\\
280	0\\
281	0\\
282	0\\
283	0\\
284	0\\
285	0\\
286	0\\
287	0\\
288	0\\
289	0\\
290	0\\
291	0\\
292	0\\
293	0\\
294	0\\
295	0\\
296	0\\
297	0\\
298	0\\
299	0\\
300	0\\
301	0\\
302	0\\
303	0\\
304	0\\
305	0\\
306	0\\
307	0\\
308	0\\
309	0\\
310	0\\
311	0\\
312	0\\
313	0\\
314	0\\
315	0\\
316	0\\
317	0\\
318	0\\
319	0\\
320	0\\
321	0\\
322	0\\
323	0\\
324	0\\
325	0\\
326	0\\
327	0\\
328	0\\
329	0\\
330	0\\
331	0\\
332	0\\
333	0\\
334	0\\
335	0\\
336	0\\
337	0\\
338	0\\
339	0\\
340	0\\
341	0\\
342	0\\
343	0\\
344	0\\
345	0\\
346	0\\
347	0\\
348	0\\
349	0\\
350	0\\
351	0\\
352	0\\
353	0\\
354	0\\
355	0\\
356	0\\
357	0\\
358	0\\
359	0\\
360	0\\
361	0\\
362	0\\
363	0\\
364	0\\
365	0\\
366	0\\
367	0\\
368	0\\
369	0\\
370	0\\
371	0\\
372	0\\
373	0\\
374	0\\
375	0\\
376	0\\
377	0\\
378	0\\
379	0\\
380	0\\
381	0\\
382	0\\
383	0\\
384	0\\
385	0\\
386	0\\
387	0\\
388	0\\
389	0\\
390	0\\
391	0\\
392	0\\
393	0\\
394	0\\
395	0\\
396	0\\
397	0\\
398	0\\
399	0\\
400	0\\
401	0\\
402	0\\
403	0\\
404	0\\
405	0\\
406	0\\
407	0\\
408	0\\
409	0\\
410	0\\
411	0\\
412	0\\
413	0\\
414	0\\
415	0\\
416	0\\
417	0\\
418	0\\
419	0\\
420	0\\
421	0\\
422	0\\
423	0\\
424	0\\
425	0\\
426	0\\
427	0\\
428	0\\
429	0\\
430	0\\
431	0\\
432	0\\
433	0\\
434	0\\
435	0\\
436	0\\
437	0\\
438	0\\
439	0\\
440	0\\
441	0\\
442	0\\
443	0\\
444	0\\
445	0\\
446	0\\
447	0\\
448	0\\
449	0\\
450	0\\
451	0\\
452	0\\
453	0\\
454	0\\
455	0\\
456	0\\
457	0\\
458	0\\
459	0\\
460	0\\
461	0\\
462	0\\
463	0\\
464	0\\
465	0\\
466	0\\
467	0\\
468	0\\
469	0\\
470	0\\
471	0\\
472	0\\
473	0\\
474	0\\
475	0\\
476	0\\
477	0\\
478	0\\
479	0\\
480	0\\
481	0\\
482	0\\
483	0\\
484	0\\
485	0\\
486	0\\
487	0\\
488	0\\
489	0\\
490	0\\
491	0\\
492	0\\
493	0\\
494	0\\
495	0\\
496	0\\
497	0\\
498	0\\
499	0\\
500	0\\
501	0\\
502	0\\
503	0\\
504	0\\
505	0\\
506	0\\
507	0\\
508	0\\
509	0\\
510	0\\
511	0\\
512	0\\
513	0\\
514	0\\
515	0\\
516	0\\
517	0\\
518	0\\
519	0\\
520	0\\
521	0\\
522	0\\
523	0\\
524	0\\
525	0\\
526	0\\
527	0\\
528	0\\
529	0\\
530	0\\
531	0\\
532	0\\
533	0\\
534	0\\
535	0\\
536	0\\
537	0\\
538	0\\
539	0\\
540	0\\
541	0\\
542	6.11660056486632e-07\\
543	2.701048695734e-05\\
544	5.40246794947436e-05\\
545	8.16731202693553e-05\\
546	0.000109974832324423\\
547	0.000138958562454918\\
548	0.00016865085298941\\
549	0.000199069928818032\\
550	0.000230234589378527\\
551	0.00026216528174636\\
552	0.000294879489820562\\
553	0.000328399703159835\\
554	0.000362790215985211\\
555	0.00039799532627114\\
556	0.000434040398948903\\
557	0.000470950458242298\\
558	0.000508722127414138\\
559	0.00054742699063001\\
560	0.000587094676229419\\
561	0.000627754501768088\\
562	0.000669596686142599\\
563	0.000712473576967077\\
564	0.000756336153151856\\
565	0.000801114955820075\\
566	0.000846955852793247\\
567	0.000893934541161798\\
568	0.000942022580164056\\
569	0.00099137139308498\\
570	0.00104281851857678\\
571	0.00109557130842463\\
572	0.00114898652968873\\
573	0.00120314728583361\\
574	0.00131396473799518\\
575	0.00163850815209253\\
576	0.00193799204557724\\
577	0.00219526366421932\\
578	0.00237623984097034\\
579	0.00247139029517359\\
580	0.00255490860490003\\
581	0.00263752357935765\\
582	0.00272157874905703\\
583	0.00280717633476708\\
584	0.00289457157399351\\
585	0.00298385506818827\\
586	0.0030751319777398\\
587	0.00316849697477764\\
588	0.00326407198718924\\
589	0.00336202443091807\\
590	0.00346265165082637\\
591	0.00356658418066895\\
592	0.00367533141522134\\
593	0.00379274227372929\\
594	0.00392890988198339\\
595	0.00411043972693671\\
596	0.00440815817174734\\
597	0.00501118703192877\\
598	0.00642488516645657\\
599	0\\
600	0\\
};
\addplot [color=mycolor9,solid,forget plot]
  table[row sep=crcr]{%
1	0\\
2	0\\
3	0\\
4	0\\
5	0\\
6	0\\
7	0\\
8	0\\
9	0\\
10	0\\
11	0\\
12	0\\
13	0\\
14	0\\
15	0\\
16	0\\
17	0\\
18	0\\
19	0\\
20	0\\
21	0\\
22	0\\
23	0\\
24	0\\
25	0\\
26	0\\
27	0\\
28	0\\
29	0\\
30	0\\
31	0\\
32	0\\
33	0\\
34	0\\
35	0\\
36	0\\
37	0\\
38	0\\
39	0\\
40	0\\
41	0\\
42	0\\
43	0\\
44	0\\
45	0\\
46	0\\
47	0\\
48	0\\
49	0\\
50	0\\
51	0\\
52	0\\
53	0\\
54	0\\
55	0\\
56	0\\
57	0\\
58	0\\
59	0\\
60	0\\
61	0\\
62	0\\
63	0\\
64	0\\
65	0\\
66	0\\
67	0\\
68	0\\
69	0\\
70	0\\
71	0\\
72	0\\
73	0\\
74	0\\
75	0\\
76	0\\
77	0\\
78	0\\
79	0\\
80	0\\
81	0\\
82	0\\
83	0\\
84	0\\
85	0\\
86	0\\
87	0\\
88	0\\
89	0\\
90	0\\
91	0\\
92	0\\
93	0\\
94	0\\
95	0\\
96	0\\
97	0\\
98	0\\
99	0\\
100	0\\
101	0\\
102	0\\
103	0\\
104	0\\
105	0\\
106	0\\
107	0\\
108	0\\
109	0\\
110	0\\
111	0\\
112	0\\
113	0\\
114	0\\
115	0\\
116	0\\
117	0\\
118	0\\
119	0\\
120	0\\
121	0\\
122	0\\
123	0\\
124	0\\
125	0\\
126	0\\
127	0\\
128	0\\
129	0\\
130	0\\
131	0\\
132	0\\
133	0\\
134	0\\
135	0\\
136	0\\
137	0\\
138	0\\
139	0\\
140	0\\
141	0\\
142	0\\
143	0\\
144	0\\
145	0\\
146	0\\
147	0\\
148	0\\
149	0\\
150	0\\
151	0\\
152	0\\
153	0\\
154	0\\
155	0\\
156	0\\
157	0\\
158	0\\
159	0\\
160	0\\
161	0\\
162	0\\
163	0\\
164	0\\
165	0\\
166	0\\
167	0\\
168	0\\
169	0\\
170	0\\
171	0\\
172	0\\
173	0\\
174	0\\
175	0\\
176	0\\
177	0\\
178	0\\
179	0\\
180	0\\
181	0\\
182	0\\
183	0\\
184	0\\
185	0\\
186	0\\
187	0\\
188	0\\
189	0\\
190	0\\
191	0\\
192	0\\
193	0\\
194	0\\
195	0\\
196	0\\
197	0\\
198	0\\
199	0\\
200	0\\
201	0\\
202	0\\
203	0\\
204	0\\
205	0\\
206	0\\
207	0\\
208	0\\
209	0\\
210	0\\
211	0\\
212	0\\
213	0\\
214	0\\
215	0\\
216	0\\
217	0\\
218	0\\
219	0\\
220	0\\
221	0\\
222	0\\
223	0\\
224	0\\
225	0\\
226	0\\
227	0\\
228	0\\
229	0\\
230	0\\
231	0\\
232	0\\
233	0\\
234	0\\
235	0\\
236	0\\
237	0\\
238	0\\
239	0\\
240	0\\
241	0\\
242	0\\
243	0\\
244	0\\
245	0\\
246	0\\
247	0\\
248	0\\
249	0\\
250	0\\
251	0\\
252	0\\
253	0\\
254	0\\
255	0\\
256	0\\
257	0\\
258	0\\
259	0\\
260	0\\
261	0\\
262	0\\
263	0\\
264	0\\
265	0\\
266	0\\
267	0\\
268	0\\
269	0\\
270	0\\
271	0\\
272	0\\
273	0\\
274	0\\
275	0\\
276	0\\
277	0\\
278	0\\
279	0\\
280	0\\
281	0\\
282	0\\
283	0\\
284	0\\
285	0\\
286	0\\
287	0\\
288	0\\
289	0\\
290	0\\
291	0\\
292	0\\
293	0\\
294	0\\
295	0\\
296	0\\
297	0\\
298	0\\
299	0\\
300	0\\
301	0\\
302	0\\
303	0\\
304	0\\
305	0\\
306	0\\
307	0\\
308	0\\
309	0\\
310	0\\
311	0\\
312	0\\
313	0\\
314	0\\
315	0\\
316	0\\
317	0\\
318	0\\
319	0\\
320	0\\
321	0\\
322	0\\
323	0\\
324	0\\
325	0\\
326	0\\
327	0\\
328	0\\
329	0\\
330	0\\
331	0\\
332	0\\
333	0\\
334	0\\
335	0\\
336	0\\
337	0\\
338	0\\
339	0\\
340	0\\
341	0\\
342	0\\
343	0\\
344	0\\
345	0\\
346	0\\
347	0\\
348	0\\
349	0\\
350	0\\
351	0\\
352	0\\
353	0\\
354	0\\
355	0\\
356	0\\
357	0\\
358	0\\
359	0\\
360	0\\
361	0\\
362	0\\
363	0\\
364	0\\
365	0\\
366	0\\
367	0\\
368	0\\
369	0\\
370	0\\
371	0\\
372	0\\
373	0\\
374	0\\
375	0\\
376	0\\
377	0\\
378	0\\
379	0\\
380	0\\
381	0\\
382	0\\
383	0\\
384	0\\
385	0\\
386	0\\
387	0\\
388	0\\
389	0\\
390	0\\
391	0\\
392	0\\
393	0\\
394	0\\
395	0\\
396	0\\
397	0\\
398	0\\
399	0\\
400	0\\
401	0\\
402	0\\
403	0\\
404	0\\
405	0\\
406	0\\
407	0\\
408	0\\
409	0\\
410	0\\
411	0\\
412	0\\
413	0\\
414	0\\
415	0\\
416	0\\
417	0\\
418	0\\
419	0\\
420	0\\
421	0\\
422	0\\
423	0\\
424	0\\
425	0\\
426	0\\
427	0\\
428	0\\
429	0\\
430	0\\
431	0\\
432	0\\
433	0\\
434	0\\
435	0\\
436	0\\
437	0\\
438	0\\
439	0\\
440	0\\
441	0\\
442	0\\
443	0\\
444	0\\
445	0\\
446	0\\
447	0\\
448	0\\
449	0\\
450	0\\
451	0\\
452	0\\
453	0\\
454	0\\
455	0\\
456	0\\
457	0\\
458	0\\
459	0\\
460	0\\
461	0\\
462	0\\
463	0\\
464	0\\
465	0\\
466	0\\
467	0\\
468	0\\
469	0\\
470	0\\
471	0\\
472	0\\
473	0\\
474	0\\
475	0\\
476	0\\
477	0\\
478	0\\
479	0\\
480	0\\
481	0\\
482	0\\
483	0\\
484	0\\
485	0\\
486	0\\
487	0\\
488	0\\
489	0\\
490	0\\
491	0\\
492	0\\
493	0\\
494	0\\
495	0\\
496	0\\
497	0\\
498	0\\
499	0\\
500	0\\
501	0\\
502	0\\
503	0\\
504	0\\
505	0\\
506	0\\
507	0\\
508	0\\
509	0\\
510	0\\
511	0\\
512	0\\
513	0\\
514	0\\
515	0\\
516	0\\
517	0\\
518	0\\
519	0\\
520	0\\
521	0\\
522	0\\
523	0\\
524	0\\
525	0\\
526	0\\
527	0\\
528	0\\
529	0\\
530	0\\
531	0\\
532	0\\
533	0\\
534	0\\
535	0\\
536	0\\
537	0\\
538	0\\
539	0\\
540	0\\
541	0\\
542	0\\
543	7.19050061245648e-06\\
544	3.38811395008324e-05\\
545	6.11995138419879e-05\\
546	8.91653520785335e-05\\
547	0.000117798896383763\\
548	0.0001471207591222\\
549	0.000177163848134191\\
550	0.000207954234817611\\
551	0.000239511711651359\\
552	0.000271863001773644\\
553	0.000305030078629033\\
554	0.000339024023324224\\
555	0.00037386790771121\\
556	0.000409585805568485\\
557	0.000446207799016567\\
558	0.00048378405404937\\
559	0.000522274282432132\\
560	0.000561707740752937\\
561	0.000602117791849755\\
562	0.000643509855471186\\
563	0.000685944198495142\\
564	0.000729459648213524\\
565	0.000774285458757708\\
566	0.000820194696357515\\
567	0.000867170853535688\\
568	0.000915208070350155\\
569	0.00096440285599\\
570	0.00101486356606715\\
571	0.00106662268470533\\
572	0.00112017026618547\\
573	0.00117541917271621\\
574	0.00123165468208164\\
575	0.00128853771559882\\
576	0.00145855591256311\\
577	0.00177609907745312\\
578	0.00206426274766966\\
579	0.00232087238374528\\
580	0.00252475298378954\\
581	0.00263031863751007\\
582	0.00271809688623168\\
583	0.00280595565159673\\
584	0.00289395943431418\\
585	0.00298357938090699\\
586	0.00307498027389778\\
587	0.00316842647759219\\
588	0.00326403331960031\\
589	0.00336201047031448\\
590	0.00346264913972014\\
591	0.00356658418066895\\
592	0.00367533141522134\\
593	0.00379274227372928\\
594	0.00392890988198339\\
595	0.00411043972693671\\
596	0.00440815817174734\\
597	0.00501118703192877\\
598	0.00642488516645657\\
599	0\\
600	0\\
};
\addplot [color=blue!50!mycolor7,solid,forget plot]
  table[row sep=crcr]{%
1	0\\
2	0\\
3	0\\
4	0\\
5	0\\
6	0\\
7	0\\
8	0\\
9	0\\
10	0\\
11	0\\
12	0\\
13	0\\
14	0\\
15	0\\
16	0\\
17	0\\
18	0\\
19	0\\
20	0\\
21	0\\
22	0\\
23	0\\
24	0\\
25	0\\
26	0\\
27	0\\
28	0\\
29	0\\
30	0\\
31	0\\
32	0\\
33	0\\
34	0\\
35	0\\
36	0\\
37	0\\
38	0\\
39	0\\
40	0\\
41	0\\
42	0\\
43	0\\
44	0\\
45	0\\
46	0\\
47	0\\
48	0\\
49	0\\
50	0\\
51	0\\
52	0\\
53	0\\
54	0\\
55	0\\
56	0\\
57	0\\
58	0\\
59	0\\
60	0\\
61	0\\
62	0\\
63	0\\
64	0\\
65	0\\
66	0\\
67	0\\
68	0\\
69	0\\
70	0\\
71	0\\
72	0\\
73	0\\
74	0\\
75	0\\
76	0\\
77	0\\
78	0\\
79	0\\
80	0\\
81	0\\
82	0\\
83	0\\
84	0\\
85	0\\
86	0\\
87	0\\
88	0\\
89	0\\
90	0\\
91	0\\
92	0\\
93	0\\
94	0\\
95	0\\
96	0\\
97	0\\
98	0\\
99	0\\
100	0\\
101	0\\
102	0\\
103	0\\
104	0\\
105	0\\
106	0\\
107	0\\
108	0\\
109	0\\
110	0\\
111	0\\
112	0\\
113	0\\
114	0\\
115	0\\
116	0\\
117	0\\
118	0\\
119	0\\
120	0\\
121	0\\
122	0\\
123	0\\
124	0\\
125	0\\
126	0\\
127	0\\
128	0\\
129	0\\
130	0\\
131	0\\
132	0\\
133	0\\
134	0\\
135	0\\
136	0\\
137	0\\
138	0\\
139	0\\
140	0\\
141	0\\
142	0\\
143	0\\
144	0\\
145	0\\
146	0\\
147	0\\
148	0\\
149	0\\
150	0\\
151	0\\
152	0\\
153	0\\
154	0\\
155	0\\
156	0\\
157	0\\
158	0\\
159	0\\
160	0\\
161	0\\
162	0\\
163	0\\
164	0\\
165	0\\
166	0\\
167	0\\
168	0\\
169	0\\
170	0\\
171	0\\
172	0\\
173	0\\
174	0\\
175	0\\
176	0\\
177	0\\
178	0\\
179	0\\
180	0\\
181	0\\
182	0\\
183	0\\
184	0\\
185	0\\
186	0\\
187	0\\
188	0\\
189	0\\
190	0\\
191	0\\
192	0\\
193	0\\
194	0\\
195	0\\
196	0\\
197	0\\
198	0\\
199	0\\
200	0\\
201	0\\
202	0\\
203	0\\
204	0\\
205	0\\
206	0\\
207	0\\
208	0\\
209	0\\
210	0\\
211	0\\
212	0\\
213	0\\
214	0\\
215	0\\
216	0\\
217	0\\
218	0\\
219	0\\
220	0\\
221	0\\
222	0\\
223	0\\
224	0\\
225	0\\
226	0\\
227	0\\
228	0\\
229	0\\
230	0\\
231	0\\
232	0\\
233	0\\
234	0\\
235	0\\
236	0\\
237	0\\
238	0\\
239	0\\
240	0\\
241	0\\
242	0\\
243	0\\
244	0\\
245	0\\
246	0\\
247	0\\
248	0\\
249	0\\
250	0\\
251	0\\
252	0\\
253	0\\
254	0\\
255	0\\
256	0\\
257	0\\
258	0\\
259	0\\
260	0\\
261	0\\
262	0\\
263	0\\
264	0\\
265	0\\
266	0\\
267	0\\
268	0\\
269	0\\
270	0\\
271	0\\
272	0\\
273	0\\
274	0\\
275	0\\
276	0\\
277	0\\
278	0\\
279	0\\
280	0\\
281	0\\
282	0\\
283	0\\
284	0\\
285	0\\
286	0\\
287	0\\
288	0\\
289	0\\
290	0\\
291	0\\
292	0\\
293	0\\
294	0\\
295	0\\
296	0\\
297	0\\
298	0\\
299	0\\
300	0\\
301	0\\
302	0\\
303	0\\
304	0\\
305	0\\
306	0\\
307	0\\
308	0\\
309	0\\
310	0\\
311	0\\
312	0\\
313	0\\
314	0\\
315	0\\
316	0\\
317	0\\
318	0\\
319	0\\
320	0\\
321	0\\
322	0\\
323	0\\
324	0\\
325	0\\
326	0\\
327	0\\
328	0\\
329	0\\
330	0\\
331	0\\
332	0\\
333	0\\
334	0\\
335	0\\
336	0\\
337	0\\
338	0\\
339	0\\
340	0\\
341	0\\
342	0\\
343	0\\
344	0\\
345	0\\
346	0\\
347	0\\
348	0\\
349	0\\
350	0\\
351	0\\
352	0\\
353	0\\
354	0\\
355	0\\
356	0\\
357	0\\
358	0\\
359	0\\
360	0\\
361	0\\
362	0\\
363	0\\
364	0\\
365	0\\
366	0\\
367	0\\
368	0\\
369	0\\
370	0\\
371	0\\
372	0\\
373	0\\
374	0\\
375	0\\
376	0\\
377	0\\
378	0\\
379	0\\
380	0\\
381	0\\
382	0\\
383	0\\
384	0\\
385	0\\
386	0\\
387	0\\
388	0\\
389	0\\
390	0\\
391	0\\
392	0\\
393	0\\
394	0\\
395	0\\
396	0\\
397	0\\
398	0\\
399	0\\
400	0\\
401	0\\
402	0\\
403	0\\
404	0\\
405	0\\
406	0\\
407	0\\
408	0\\
409	0\\
410	0\\
411	0\\
412	0\\
413	0\\
414	0\\
415	0\\
416	0\\
417	0\\
418	0\\
419	0\\
420	0\\
421	0\\
422	0\\
423	0\\
424	0\\
425	0\\
426	0\\
427	0\\
428	0\\
429	0\\
430	0\\
431	0\\
432	0\\
433	0\\
434	0\\
435	0\\
436	0\\
437	0\\
438	0\\
439	0\\
440	0\\
441	0\\
442	0\\
443	0\\
444	0\\
445	0\\
446	0\\
447	0\\
448	0\\
449	0\\
450	0\\
451	0\\
452	0\\
453	0\\
454	0\\
455	0\\
456	0\\
457	0\\
458	0\\
459	0\\
460	0\\
461	0\\
462	0\\
463	0\\
464	0\\
465	0\\
466	0\\
467	0\\
468	0\\
469	0\\
470	0\\
471	0\\
472	0\\
473	0\\
474	0\\
475	0\\
476	0\\
477	0\\
478	0\\
479	0\\
480	0\\
481	0\\
482	0\\
483	0\\
484	0\\
485	0\\
486	0\\
487	0\\
488	0\\
489	0\\
490	0\\
491	0\\
492	0\\
493	0\\
494	0\\
495	0\\
496	0\\
497	0\\
498	0\\
499	0\\
500	0\\
501	0\\
502	0\\
503	0\\
504	0\\
505	0\\
506	0\\
507	0\\
508	0\\
509	0\\
510	0\\
511	0\\
512	0\\
513	0\\
514	0\\
515	0\\
516	0\\
517	0\\
518	0\\
519	0\\
520	0\\
521	0\\
522	0\\
523	0\\
524	0\\
525	0\\
526	0\\
527	0\\
528	0\\
529	0\\
530	0\\
531	0\\
532	0\\
533	0\\
534	0\\
535	0\\
536	0\\
537	0\\
538	0\\
539	0\\
540	0\\
541	0\\
542	0\\
543	0\\
544	1.05627776958363e-05\\
545	3.75304772690883e-05\\
546	6.51312718678484e-05\\
547	9.33863630317708e-05\\
548	0.000122317550313663\\
549	0.00015194702081486\\
550	0.000182308666181335\\
551	0.00021342974811385\\
552	0.000245331657737849\\
553	0.000278036116023763\\
554	0.000311565519271933\\
555	0.000345943015786943\\
556	0.000381192545748294\\
557	0.000417339269889016\\
558	0.000454422980542451\\
559	0.00049245289508931\\
560	0.00053145713940947\\
561	0.000571465034329317\\
562	0.000612547638993974\\
563	0.000654664794599318\\
564	0.000697840632010934\\
565	0.000742117586682551\\
566	0.000787524528073936\\
567	0.000834101450324763\\
568	0.000882052764535135\\
569	0.000931270793437822\\
570	0.000981668890123943\\
571	0.00103328290710104\\
572	0.00108611346496739\\
573	0.00114036010457888\\
574	0.00119618222241278\\
575	0.00125424290438048\\
576	0.00131331951945714\\
577	0.00137341947531424\\
578	0.00155650097528339\\
579	0.001863805897274\\
580	0.00215685965284502\\
581	0.00241246675897658\\
582	0.0026698348651085\\
583	0.00278212107437134\\
584	0.00288608491603381\\
585	0.00297962692672493\\
586	0.00307327221777276\\
587	0.00316747165964346\\
588	0.00326360274193222\\
589	0.00336176338818523\\
590	0.00346255683773199\\
591	0.00356656710209231\\
592	0.00367533141522134\\
593	0.00379274227372929\\
594	0.00392890988198339\\
595	0.00411043972693671\\
596	0.00440815817174734\\
597	0.00501118703192877\\
598	0.00642488516645657\\
599	0\\
600	0\\
};
\addplot [color=blue!40!mycolor9,solid,forget plot]
  table[row sep=crcr]{%
1	0\\
2	0\\
3	0\\
4	0\\
5	0\\
6	0\\
7	0\\
8	0\\
9	0\\
10	0\\
11	0\\
12	0\\
13	0\\
14	0\\
15	0\\
16	0\\
17	0\\
18	0\\
19	0\\
20	0\\
21	0\\
22	0\\
23	0\\
24	0\\
25	0\\
26	0\\
27	0\\
28	0\\
29	0\\
30	0\\
31	0\\
32	0\\
33	0\\
34	0\\
35	0\\
36	0\\
37	0\\
38	0\\
39	0\\
40	0\\
41	0\\
42	0\\
43	0\\
44	0\\
45	0\\
46	0\\
47	0\\
48	0\\
49	0\\
50	0\\
51	0\\
52	0\\
53	0\\
54	0\\
55	0\\
56	0\\
57	0\\
58	0\\
59	0\\
60	0\\
61	0\\
62	0\\
63	0\\
64	0\\
65	0\\
66	0\\
67	0\\
68	0\\
69	0\\
70	0\\
71	0\\
72	0\\
73	0\\
74	0\\
75	0\\
76	0\\
77	0\\
78	0\\
79	0\\
80	0\\
81	0\\
82	0\\
83	0\\
84	0\\
85	0\\
86	0\\
87	0\\
88	0\\
89	0\\
90	0\\
91	0\\
92	0\\
93	0\\
94	0\\
95	0\\
96	0\\
97	0\\
98	0\\
99	0\\
100	0\\
101	0\\
102	0\\
103	0\\
104	0\\
105	0\\
106	0\\
107	0\\
108	0\\
109	0\\
110	0\\
111	0\\
112	0\\
113	0\\
114	0\\
115	0\\
116	0\\
117	0\\
118	0\\
119	0\\
120	0\\
121	0\\
122	0\\
123	0\\
124	0\\
125	0\\
126	0\\
127	0\\
128	0\\
129	0\\
130	0\\
131	0\\
132	0\\
133	0\\
134	0\\
135	0\\
136	0\\
137	0\\
138	0\\
139	0\\
140	0\\
141	0\\
142	0\\
143	0\\
144	0\\
145	0\\
146	0\\
147	0\\
148	0\\
149	0\\
150	0\\
151	0\\
152	0\\
153	0\\
154	0\\
155	0\\
156	0\\
157	0\\
158	0\\
159	0\\
160	0\\
161	0\\
162	0\\
163	0\\
164	0\\
165	0\\
166	0\\
167	0\\
168	0\\
169	0\\
170	0\\
171	0\\
172	0\\
173	0\\
174	0\\
175	0\\
176	0\\
177	0\\
178	0\\
179	0\\
180	0\\
181	0\\
182	0\\
183	0\\
184	0\\
185	0\\
186	0\\
187	0\\
188	0\\
189	0\\
190	0\\
191	0\\
192	0\\
193	0\\
194	0\\
195	0\\
196	0\\
197	0\\
198	0\\
199	0\\
200	0\\
201	0\\
202	0\\
203	0\\
204	0\\
205	0\\
206	0\\
207	0\\
208	0\\
209	0\\
210	0\\
211	0\\
212	0\\
213	0\\
214	0\\
215	0\\
216	0\\
217	0\\
218	0\\
219	0\\
220	0\\
221	0\\
222	0\\
223	0\\
224	0\\
225	0\\
226	0\\
227	0\\
228	0\\
229	0\\
230	0\\
231	0\\
232	0\\
233	0\\
234	0\\
235	0\\
236	0\\
237	0\\
238	0\\
239	0\\
240	0\\
241	0\\
242	0\\
243	0\\
244	0\\
245	0\\
246	0\\
247	0\\
248	0\\
249	0\\
250	0\\
251	0\\
252	0\\
253	0\\
254	0\\
255	0\\
256	0\\
257	0\\
258	0\\
259	0\\
260	0\\
261	0\\
262	0\\
263	0\\
264	0\\
265	0\\
266	0\\
267	0\\
268	0\\
269	0\\
270	0\\
271	0\\
272	0\\
273	0\\
274	0\\
275	0\\
276	0\\
277	0\\
278	0\\
279	0\\
280	0\\
281	0\\
282	0\\
283	0\\
284	0\\
285	0\\
286	0\\
287	0\\
288	0\\
289	0\\
290	0\\
291	0\\
292	0\\
293	0\\
294	0\\
295	0\\
296	0\\
297	0\\
298	0\\
299	0\\
300	0\\
301	0\\
302	0\\
303	0\\
304	0\\
305	0\\
306	0\\
307	0\\
308	0\\
309	0\\
310	0\\
311	0\\
312	0\\
313	0\\
314	0\\
315	0\\
316	0\\
317	0\\
318	0\\
319	0\\
320	0\\
321	0\\
322	0\\
323	0\\
324	0\\
325	0\\
326	0\\
327	0\\
328	0\\
329	0\\
330	0\\
331	0\\
332	0\\
333	0\\
334	0\\
335	0\\
336	0\\
337	0\\
338	0\\
339	0\\
340	0\\
341	0\\
342	0\\
343	0\\
344	0\\
345	0\\
346	0\\
347	0\\
348	0\\
349	0\\
350	0\\
351	0\\
352	0\\
353	0\\
354	0\\
355	0\\
356	0\\
357	0\\
358	0\\
359	0\\
360	0\\
361	0\\
362	0\\
363	0\\
364	0\\
365	0\\
366	0\\
367	0\\
368	0\\
369	0\\
370	0\\
371	0\\
372	0\\
373	0\\
374	0\\
375	0\\
376	0\\
377	0\\
378	0\\
379	0\\
380	0\\
381	0\\
382	0\\
383	0\\
384	0\\
385	0\\
386	0\\
387	0\\
388	0\\
389	0\\
390	0\\
391	0\\
392	0\\
393	0\\
394	0\\
395	0\\
396	0\\
397	0\\
398	0\\
399	0\\
400	0\\
401	0\\
402	0\\
403	0\\
404	0\\
405	0\\
406	0\\
407	0\\
408	0\\
409	0\\
410	0\\
411	0\\
412	0\\
413	0\\
414	0\\
415	0\\
416	0\\
417	0\\
418	0\\
419	0\\
420	0\\
421	0\\
422	0\\
423	0\\
424	0\\
425	0\\
426	0\\
427	0\\
428	0\\
429	0\\
430	0\\
431	0\\
432	0\\
433	0\\
434	0\\
435	0\\
436	0\\
437	0\\
438	0\\
439	0\\
440	0\\
441	0\\
442	0\\
443	0\\
444	0\\
445	0\\
446	0\\
447	0\\
448	0\\
449	0\\
450	0\\
451	0\\
452	0\\
453	0\\
454	0\\
455	0\\
456	0\\
457	0\\
458	0\\
459	0\\
460	0\\
461	0\\
462	0\\
463	0\\
464	0\\
465	0\\
466	0\\
467	0\\
468	0\\
469	0\\
470	0\\
471	0\\
472	0\\
473	0\\
474	0\\
475	0\\
476	0\\
477	0\\
478	0\\
479	0\\
480	0\\
481	0\\
482	0\\
483	0\\
484	0\\
485	0\\
486	0\\
487	0\\
488	0\\
489	0\\
490	0\\
491	0\\
492	0\\
493	0\\
494	0\\
495	0\\
496	0\\
497	0\\
498	0\\
499	0\\
500	0\\
501	0\\
502	0\\
503	0\\
504	0\\
505	0\\
506	0\\
507	0\\
508	0\\
509	0\\
510	0\\
511	0\\
512	0\\
513	0\\
514	0\\
515	0\\
516	0\\
517	0\\
518	0\\
519	0\\
520	0\\
521	0\\
522	0\\
523	0\\
524	0\\
525	0\\
526	0\\
527	0\\
528	0\\
529	0\\
530	0\\
531	0\\
532	0\\
533	0\\
534	0\\
535	0\\
536	0\\
537	0\\
538	0\\
539	0\\
540	0\\
541	0\\
542	0\\
543	0\\
544	0\\
545	8.42502412765558e-06\\
546	3.5851947618167e-05\\
547	6.38863155815981e-05\\
548	9.25528110374975e-05\\
549	0.000121877183064925\\
550	0.000151885824432349\\
551	0.000182616850935656\\
552	0.000214101549367516\\
553	0.000246366661718656\\
554	0.000279437159386335\\
555	0.000313337079486906\\
556	0.000348091356271088\\
557	0.000383725850763514\\
558	0.000420267380630395\\
559	0.000457743782503094\\
560	0.000496183903776342\\
561	0.000535617637525919\\
562	0.000576075963802652\\
563	0.00061759716129537\\
564	0.000660217067372072\\
565	0.000703959760100343\\
566	0.000748880601024024\\
567	0.000795008079541074\\
568	0.000842337955796574\\
569	0.00089091219139234\\
570	0.000940779053632871\\
571	0.000992018378974306\\
572	0.00104480256110106\\
573	0.00109890666525564\\
574	0.00115440158344068\\
575	0.0012113083263446\\
576	0.0012696758029509\\
577	0.00133007165307071\\
578	0.00139239776289234\\
579	0.00145595886199769\\
580	0.00160714494923117\\
581	0.001899590420333\\
582	0.00221326571071451\\
583	0.00246366209819108\\
584	0.00272385949410461\\
585	0.00292924427885274\\
586	0.00304785416029217\\
587	0.00315702143525764\\
588	0.00325763804119556\\
589	0.00335918365979025\\
590	0.00346100219613368\\
591	0.0035659653520596\\
592	0.00367521681883594\\
593	0.00379274227372929\\
594	0.0039289098819834\\
595	0.00411043972693671\\
596	0.00440815817174734\\
597	0.00501118703192877\\
598	0.00642488516645657\\
599	0\\
600	0\\
};
\addplot [color=blue!75!mycolor7,solid,forget plot]
  table[row sep=crcr]{%
1	0\\
2	0\\
3	0\\
4	0\\
5	0\\
6	0\\
7	0\\
8	0\\
9	0\\
10	0\\
11	0\\
12	0\\
13	0\\
14	0\\
15	0\\
16	0\\
17	0\\
18	0\\
19	0\\
20	0\\
21	0\\
22	0\\
23	0\\
24	0\\
25	0\\
26	0\\
27	0\\
28	0\\
29	0\\
30	0\\
31	0\\
32	0\\
33	0\\
34	0\\
35	0\\
36	0\\
37	0\\
38	0\\
39	0\\
40	0\\
41	0\\
42	0\\
43	0\\
44	0\\
45	0\\
46	0\\
47	0\\
48	0\\
49	0\\
50	0\\
51	0\\
52	0\\
53	0\\
54	0\\
55	0\\
56	0\\
57	0\\
58	0\\
59	0\\
60	0\\
61	0\\
62	0\\
63	0\\
64	0\\
65	0\\
66	0\\
67	0\\
68	0\\
69	0\\
70	0\\
71	0\\
72	0\\
73	0\\
74	0\\
75	0\\
76	0\\
77	0\\
78	0\\
79	0\\
80	0\\
81	0\\
82	0\\
83	0\\
84	0\\
85	0\\
86	0\\
87	0\\
88	0\\
89	0\\
90	0\\
91	0\\
92	0\\
93	0\\
94	0\\
95	0\\
96	0\\
97	0\\
98	0\\
99	0\\
100	0\\
101	0\\
102	0\\
103	0\\
104	0\\
105	0\\
106	0\\
107	0\\
108	0\\
109	0\\
110	0\\
111	0\\
112	0\\
113	0\\
114	0\\
115	0\\
116	0\\
117	0\\
118	0\\
119	0\\
120	0\\
121	0\\
122	0\\
123	0\\
124	0\\
125	0\\
126	0\\
127	0\\
128	0\\
129	0\\
130	0\\
131	0\\
132	0\\
133	0\\
134	0\\
135	0\\
136	0\\
137	0\\
138	0\\
139	0\\
140	0\\
141	0\\
142	0\\
143	0\\
144	0\\
145	0\\
146	0\\
147	0\\
148	0\\
149	0\\
150	0\\
151	0\\
152	0\\
153	0\\
154	0\\
155	0\\
156	0\\
157	0\\
158	0\\
159	0\\
160	0\\
161	0\\
162	0\\
163	0\\
164	0\\
165	0\\
166	0\\
167	0\\
168	0\\
169	0\\
170	0\\
171	0\\
172	0\\
173	0\\
174	0\\
175	0\\
176	0\\
177	0\\
178	0\\
179	0\\
180	0\\
181	0\\
182	0\\
183	0\\
184	0\\
185	0\\
186	0\\
187	0\\
188	0\\
189	0\\
190	0\\
191	0\\
192	0\\
193	0\\
194	0\\
195	0\\
196	0\\
197	0\\
198	0\\
199	0\\
200	0\\
201	0\\
202	0\\
203	0\\
204	0\\
205	0\\
206	0\\
207	0\\
208	0\\
209	0\\
210	0\\
211	0\\
212	0\\
213	0\\
214	0\\
215	0\\
216	0\\
217	0\\
218	0\\
219	0\\
220	0\\
221	0\\
222	0\\
223	0\\
224	0\\
225	0\\
226	0\\
227	0\\
228	0\\
229	0\\
230	0\\
231	0\\
232	0\\
233	0\\
234	0\\
235	0\\
236	0\\
237	0\\
238	0\\
239	0\\
240	0\\
241	0\\
242	0\\
243	0\\
244	0\\
245	0\\
246	0\\
247	0\\
248	0\\
249	0\\
250	0\\
251	0\\
252	0\\
253	0\\
254	0\\
255	0\\
256	0\\
257	0\\
258	0\\
259	0\\
260	0\\
261	0\\
262	0\\
263	0\\
264	0\\
265	0\\
266	0\\
267	0\\
268	0\\
269	0\\
270	0\\
271	0\\
272	0\\
273	0\\
274	0\\
275	0\\
276	0\\
277	0\\
278	0\\
279	0\\
280	0\\
281	0\\
282	0\\
283	0\\
284	0\\
285	0\\
286	0\\
287	0\\
288	0\\
289	0\\
290	0\\
291	0\\
292	0\\
293	0\\
294	0\\
295	0\\
296	0\\
297	0\\
298	0\\
299	0\\
300	0\\
301	0\\
302	0\\
303	0\\
304	0\\
305	0\\
306	0\\
307	0\\
308	0\\
309	0\\
310	0\\
311	0\\
312	0\\
313	0\\
314	0\\
315	0\\
316	0\\
317	0\\
318	0\\
319	0\\
320	0\\
321	0\\
322	0\\
323	0\\
324	0\\
325	0\\
326	0\\
327	0\\
328	0\\
329	0\\
330	0\\
331	0\\
332	0\\
333	0\\
334	0\\
335	0\\
336	0\\
337	0\\
338	0\\
339	0\\
340	0\\
341	0\\
342	0\\
343	0\\
344	0\\
345	0\\
346	0\\
347	0\\
348	0\\
349	0\\
350	0\\
351	0\\
352	0\\
353	0\\
354	0\\
355	0\\
356	0\\
357	0\\
358	0\\
359	0\\
360	0\\
361	0\\
362	0\\
363	0\\
364	0\\
365	0\\
366	0\\
367	0\\
368	0\\
369	0\\
370	0\\
371	0\\
372	0\\
373	0\\
374	0\\
375	0\\
376	0\\
377	0\\
378	0\\
379	0\\
380	0\\
381	0\\
382	0\\
383	0\\
384	0\\
385	0\\
386	0\\
387	0\\
388	0\\
389	0\\
390	0\\
391	0\\
392	0\\
393	0\\
394	0\\
395	0\\
396	0\\
397	0\\
398	0\\
399	0\\
400	0\\
401	0\\
402	0\\
403	0\\
404	0\\
405	0\\
406	0\\
407	0\\
408	0\\
409	0\\
410	0\\
411	0\\
412	0\\
413	0\\
414	0\\
415	0\\
416	0\\
417	0\\
418	0\\
419	0\\
420	0\\
421	0\\
422	0\\
423	0\\
424	0\\
425	0\\
426	0\\
427	0\\
428	0\\
429	0\\
430	0\\
431	0\\
432	0\\
433	0\\
434	0\\
435	0\\
436	0\\
437	0\\
438	0\\
439	0\\
440	0\\
441	0\\
442	0\\
443	0\\
444	0\\
445	0\\
446	0\\
447	0\\
448	0\\
449	0\\
450	0\\
451	0\\
452	0\\
453	0\\
454	0\\
455	0\\
456	0\\
457	0\\
458	0\\
459	0\\
460	0\\
461	0\\
462	0\\
463	0\\
464	0\\
465	0\\
466	0\\
467	0\\
468	0\\
469	0\\
470	0\\
471	0\\
472	0\\
473	0\\
474	0\\
475	0\\
476	0\\
477	0\\
478	0\\
479	0\\
480	0\\
481	0\\
482	0\\
483	0\\
484	0\\
485	0\\
486	0\\
487	0\\
488	0\\
489	0\\
490	0\\
491	0\\
492	0\\
493	0\\
494	0\\
495	0\\
496	0\\
497	0\\
498	0\\
499	0\\
500	0\\
501	0\\
502	0\\
503	0\\
504	0\\
505	0\\
506	0\\
507	0\\
508	0\\
509	0\\
510	0\\
511	0\\
512	0\\
513	0\\
514	0\\
515	0\\
516	0\\
517	0\\
518	0\\
519	0\\
520	0\\
521	0\\
522	0\\
523	0\\
524	0\\
525	0\\
526	0\\
527	0\\
528	0\\
529	0\\
530	0\\
531	0\\
532	0\\
533	0\\
534	0\\
535	0\\
536	0\\
537	0\\
538	0\\
539	0\\
540	0\\
541	0\\
542	0\\
543	0\\
544	0\\
545	0\\
546	0\\
547	2.01299093932324e-05\\
548	4.97180538846728e-05\\
549	7.97809066500282e-05\\
550	0.000110346808772033\\
551	0.000141449544534373\\
552	0.000173138917481353\\
553	0.000205467689239195\\
554	0.000238497409334684\\
555	0.000272296809122876\\
556	0.000306891010061317\\
557	0.000342306630210377\\
558	0.000378571867476797\\
559	0.000415716567682697\\
560	0.000453772268996213\\
561	0.000492772207995115\\
562	0.000532751271474939\\
563	0.000573745863855526\\
564	0.000615793687390193\\
565	0.000658933366835831\\
566	0.000703203836142204\\
567	0.00074864369322039\\
568	0.000795293618467427\\
569	0.000843202487028982\\
570	0.000892415829948461\\
571	0.000942997007940762\\
572	0.000994962562184872\\
573	0.0010483381877448\\
574	0.00110318229396275\\
575	0.00115964812226496\\
576	0.00121775736770317\\
577	0.00127738546819808\\
578	0.00133863477863552\\
579	0.00140157076039025\\
580	0.001466687528102\\
581	0.00153378580448497\\
582	0.00160982745781871\\
583	0.00189508529007164\\
584	0.00219138160047944\\
585	0.00247449504394679\\
586	0.0027294908751432\\
587	0.00299438591461619\\
588	0.00319452135862246\\
589	0.0033221530258415\\
590	0.00344585440730691\\
591	0.00355633781678836\\
592	0.00367135035665401\\
593	0.00379198418312597\\
594	0.0039289098819834\\
595	0.00411043972693671\\
596	0.00440815817174734\\
597	0.00501118703192877\\
598	0.00642488516645657\\
599	0\\
600	0\\
};
\addplot [color=blue!80!mycolor9,solid,forget plot]
  table[row sep=crcr]{%
1	0\\
2	0\\
3	0\\
4	0\\
5	0\\
6	0\\
7	0\\
8	0\\
9	0\\
10	0\\
11	0\\
12	0\\
13	0\\
14	0\\
15	0\\
16	0\\
17	0\\
18	0\\
19	0\\
20	0\\
21	0\\
22	0\\
23	0\\
24	0\\
25	0\\
26	0\\
27	0\\
28	0\\
29	0\\
30	0\\
31	0\\
32	0\\
33	0\\
34	0\\
35	0\\
36	0\\
37	0\\
38	0\\
39	0\\
40	0\\
41	0\\
42	0\\
43	0\\
44	0\\
45	0\\
46	0\\
47	0\\
48	0\\
49	0\\
50	0\\
51	0\\
52	0\\
53	0\\
54	0\\
55	0\\
56	0\\
57	0\\
58	0\\
59	0\\
60	0\\
61	0\\
62	0\\
63	0\\
64	0\\
65	0\\
66	0\\
67	0\\
68	0\\
69	0\\
70	0\\
71	0\\
72	0\\
73	0\\
74	0\\
75	0\\
76	0\\
77	0\\
78	0\\
79	0\\
80	0\\
81	0\\
82	0\\
83	0\\
84	0\\
85	0\\
86	0\\
87	0\\
88	0\\
89	0\\
90	0\\
91	0\\
92	0\\
93	0\\
94	0\\
95	0\\
96	0\\
97	0\\
98	0\\
99	0\\
100	0\\
101	0\\
102	0\\
103	0\\
104	0\\
105	0\\
106	0\\
107	0\\
108	0\\
109	0\\
110	0\\
111	0\\
112	0\\
113	0\\
114	0\\
115	0\\
116	0\\
117	0\\
118	0\\
119	0\\
120	0\\
121	0\\
122	0\\
123	0\\
124	0\\
125	0\\
126	0\\
127	0\\
128	0\\
129	0\\
130	0\\
131	0\\
132	0\\
133	0\\
134	0\\
135	0\\
136	0\\
137	0\\
138	0\\
139	0\\
140	0\\
141	0\\
142	0\\
143	0\\
144	0\\
145	0\\
146	0\\
147	0\\
148	0\\
149	0\\
150	0\\
151	0\\
152	0\\
153	0\\
154	0\\
155	0\\
156	0\\
157	0\\
158	0\\
159	0\\
160	0\\
161	0\\
162	0\\
163	0\\
164	0\\
165	0\\
166	0\\
167	0\\
168	0\\
169	0\\
170	0\\
171	0\\
172	0\\
173	0\\
174	0\\
175	0\\
176	0\\
177	0\\
178	0\\
179	0\\
180	0\\
181	0\\
182	0\\
183	0\\
184	0\\
185	0\\
186	0\\
187	0\\
188	0\\
189	0\\
190	0\\
191	0\\
192	0\\
193	0\\
194	0\\
195	0\\
196	0\\
197	0\\
198	0\\
199	0\\
200	0\\
201	0\\
202	0\\
203	0\\
204	0\\
205	0\\
206	0\\
207	0\\
208	0\\
209	0\\
210	0\\
211	0\\
212	0\\
213	0\\
214	0\\
215	0\\
216	0\\
217	0\\
218	0\\
219	0\\
220	0\\
221	0\\
222	0\\
223	0\\
224	0\\
225	0\\
226	0\\
227	0\\
228	0\\
229	0\\
230	0\\
231	0\\
232	0\\
233	0\\
234	0\\
235	0\\
236	0\\
237	0\\
238	0\\
239	0\\
240	0\\
241	0\\
242	0\\
243	0\\
244	0\\
245	0\\
246	0\\
247	0\\
248	0\\
249	0\\
250	0\\
251	0\\
252	0\\
253	0\\
254	0\\
255	0\\
256	0\\
257	0\\
258	0\\
259	0\\
260	0\\
261	0\\
262	0\\
263	0\\
264	0\\
265	0\\
266	0\\
267	0\\
268	0\\
269	0\\
270	0\\
271	0\\
272	0\\
273	0\\
274	0\\
275	0\\
276	0\\
277	0\\
278	0\\
279	0\\
280	0\\
281	0\\
282	0\\
283	0\\
284	0\\
285	0\\
286	0\\
287	0\\
288	0\\
289	0\\
290	0\\
291	0\\
292	0\\
293	0\\
294	0\\
295	0\\
296	0\\
297	0\\
298	0\\
299	0\\
300	0\\
301	0\\
302	0\\
303	0\\
304	0\\
305	0\\
306	0\\
307	0\\
308	0\\
309	0\\
310	0\\
311	0\\
312	0\\
313	0\\
314	0\\
315	0\\
316	0\\
317	0\\
318	0\\
319	0\\
320	0\\
321	0\\
322	0\\
323	0\\
324	0\\
325	0\\
326	0\\
327	0\\
328	0\\
329	0\\
330	0\\
331	0\\
332	0\\
333	0\\
334	0\\
335	0\\
336	0\\
337	0\\
338	0\\
339	0\\
340	0\\
341	0\\
342	0\\
343	0\\
344	0\\
345	0\\
346	0\\
347	0\\
348	0\\
349	0\\
350	0\\
351	0\\
352	0\\
353	0\\
354	0\\
355	0\\
356	0\\
357	0\\
358	0\\
359	0\\
360	0\\
361	0\\
362	0\\
363	0\\
364	0\\
365	0\\
366	0\\
367	0\\
368	0\\
369	0\\
370	0\\
371	0\\
372	0\\
373	0\\
374	0\\
375	0\\
376	0\\
377	0\\
378	0\\
379	0\\
380	0\\
381	0\\
382	0\\
383	0\\
384	0\\
385	0\\
386	0\\
387	0\\
388	0\\
389	0\\
390	0\\
391	0\\
392	0\\
393	0\\
394	0\\
395	0\\
396	0\\
397	0\\
398	0\\
399	0\\
400	0\\
401	0\\
402	0\\
403	0\\
404	0\\
405	0\\
406	0\\
407	0\\
408	0\\
409	0\\
410	0\\
411	0\\
412	0\\
413	0\\
414	0\\
415	0\\
416	0\\
417	0\\
418	0\\
419	0\\
420	0\\
421	0\\
422	0\\
423	0\\
424	0\\
425	0\\
426	0\\
427	0\\
428	0\\
429	0\\
430	0\\
431	0\\
432	0\\
433	0\\
434	0\\
435	0\\
436	0\\
437	0\\
438	0\\
439	0\\
440	0\\
441	0\\
442	0\\
443	0\\
444	0\\
445	0\\
446	0\\
447	0\\
448	0\\
449	0\\
450	0\\
451	0\\
452	0\\
453	0\\
454	0\\
455	0\\
456	0\\
457	0\\
458	0\\
459	0\\
460	0\\
461	0\\
462	0\\
463	0\\
464	0\\
465	0\\
466	0\\
467	0\\
468	0\\
469	0\\
470	0\\
471	0\\
472	0\\
473	0\\
474	0\\
475	0\\
476	0\\
477	0\\
478	0\\
479	0\\
480	0\\
481	0\\
482	0\\
483	0\\
484	0\\
485	0\\
486	0\\
487	0\\
488	0\\
489	0\\
490	0\\
491	0\\
492	0\\
493	0\\
494	0\\
495	0\\
496	0\\
497	0\\
498	0\\
499	0\\
500	0\\
501	0\\
502	0\\
503	0\\
504	0\\
505	0\\
506	0\\
507	0\\
508	0\\
509	0\\
510	0\\
511	0\\
512	0\\
513	0\\
514	0\\
515	0\\
516	0\\
517	0\\
518	0\\
519	0\\
520	0\\
521	0\\
522	0\\
523	0\\
524	0\\
525	0\\
526	0\\
527	0\\
528	0\\
529	0\\
530	0\\
531	0\\
532	0\\
533	0\\
534	0\\
535	0\\
536	0\\
537	0\\
538	0\\
539	0\\
540	0\\
541	0\\
542	0\\
543	0\\
544	0\\
545	0\\
546	0\\
547	0\\
548	0\\
549	0\\
550	2.12039339737889e-05\\
551	5.84050361664144e-05\\
552	9.54010943067283e-05\\
553	0.000132096979318932\\
554	0.000168388025923539\\
555	0.000204174616590858\\
556	0.000240606685821581\\
557	0.000277698174390048\\
558	0.000315465232300687\\
559	0.000353926827843293\\
560	0.000393105503571735\\
561	0.000433028368811907\\
562	0.000473728260877721\\
563	0.000515245178121573\\
564	0.000557628044582771\\
565	0.000600936877891869\\
566	0.000645245449218651\\
567	0.000690644491509104\\
568	0.000737174475960431\\
569	0.000784875954349228\\
570	0.000833792095532848\\
571	0.000883968867581733\\
572	0.00093545555079639\\
573	0.000988305024199897\\
574	0.00104257376943079\\
575	0.00109832432258913\\
576	0.00115564793896351\\
577	0.00121458956000823\\
578	0.0012751791441012\\
579	0.00133755345032623\\
580	0.00140183211419833\\
581	0.00146787349040303\\
582	0.00153576033905729\\
583	0.00160587300607344\\
584	0.00167839160337079\\
585	0.00183894072237567\\
586	0.00211968813859839\\
587	0.00241319308767275\\
588	0.00268990374966985\\
589	0.0029472638427176\\
590	0.00321775581666242\\
591	0.00346925274976254\\
592	0.00361272863110318\\
593	0.00376752264049869\\
594	0.00392397211113342\\
595	0.00411043972693671\\
596	0.00440815817174734\\
597	0.00501118703192877\\
598	0.00642488516645657\\
599	0\\
600	0\\
};
\addplot [color=blue,solid,forget plot]
  table[row sep=crcr]{%
1	0\\
2	0\\
3	0\\
4	0\\
5	0\\
6	0\\
7	0\\
8	0\\
9	0\\
10	0\\
11	0\\
12	0\\
13	0\\
14	0\\
15	0\\
16	0\\
17	0\\
18	0\\
19	0\\
20	0\\
21	0\\
22	0\\
23	0\\
24	0\\
25	0\\
26	0\\
27	0\\
28	0\\
29	0\\
30	0\\
31	0\\
32	0\\
33	0\\
34	0\\
35	0\\
36	0\\
37	0\\
38	0\\
39	0\\
40	0\\
41	0\\
42	0\\
43	0\\
44	0\\
45	0\\
46	0\\
47	0\\
48	0\\
49	0\\
50	0\\
51	0\\
52	0\\
53	0\\
54	0\\
55	0\\
56	0\\
57	0\\
58	0\\
59	0\\
60	0\\
61	0\\
62	0\\
63	0\\
64	0\\
65	0\\
66	0\\
67	0\\
68	0\\
69	0\\
70	0\\
71	0\\
72	0\\
73	0\\
74	0\\
75	0\\
76	0\\
77	0\\
78	0\\
79	0\\
80	0\\
81	0\\
82	0\\
83	0\\
84	0\\
85	0\\
86	0\\
87	0\\
88	0\\
89	0\\
90	0\\
91	0\\
92	0\\
93	0\\
94	0\\
95	0\\
96	0\\
97	0\\
98	0\\
99	0\\
100	0\\
101	0\\
102	0\\
103	0\\
104	0\\
105	0\\
106	0\\
107	0\\
108	0\\
109	0\\
110	0\\
111	0\\
112	0\\
113	0\\
114	0\\
115	0\\
116	0\\
117	0\\
118	0\\
119	0\\
120	0\\
121	0\\
122	0\\
123	0\\
124	0\\
125	0\\
126	0\\
127	0\\
128	0\\
129	0\\
130	0\\
131	0\\
132	0\\
133	0\\
134	0\\
135	0\\
136	0\\
137	0\\
138	0\\
139	0\\
140	0\\
141	0\\
142	0\\
143	0\\
144	0\\
145	0\\
146	0\\
147	0\\
148	0\\
149	0\\
150	0\\
151	0\\
152	0\\
153	0\\
154	0\\
155	0\\
156	0\\
157	0\\
158	0\\
159	0\\
160	0\\
161	0\\
162	0\\
163	0\\
164	0\\
165	0\\
166	0\\
167	0\\
168	0\\
169	0\\
170	0\\
171	0\\
172	0\\
173	0\\
174	0\\
175	0\\
176	0\\
177	0\\
178	0\\
179	0\\
180	0\\
181	0\\
182	0\\
183	0\\
184	0\\
185	0\\
186	0\\
187	0\\
188	0\\
189	0\\
190	0\\
191	0\\
192	0\\
193	0\\
194	0\\
195	0\\
196	0\\
197	0\\
198	0\\
199	0\\
200	0\\
201	0\\
202	0\\
203	0\\
204	0\\
205	0\\
206	0\\
207	0\\
208	0\\
209	0\\
210	0\\
211	0\\
212	0\\
213	0\\
214	0\\
215	0\\
216	0\\
217	0\\
218	0\\
219	0\\
220	0\\
221	0\\
222	0\\
223	0\\
224	0\\
225	0\\
226	0\\
227	0\\
228	0\\
229	0\\
230	0\\
231	0\\
232	0\\
233	0\\
234	0\\
235	0\\
236	0\\
237	0\\
238	0\\
239	0\\
240	0\\
241	0\\
242	0\\
243	0\\
244	0\\
245	0\\
246	0\\
247	0\\
248	0\\
249	0\\
250	0\\
251	0\\
252	0\\
253	0\\
254	0\\
255	0\\
256	0\\
257	0\\
258	0\\
259	0\\
260	0\\
261	0\\
262	0\\
263	0\\
264	0\\
265	0\\
266	0\\
267	0\\
268	0\\
269	0\\
270	0\\
271	0\\
272	0\\
273	0\\
274	0\\
275	0\\
276	0\\
277	0\\
278	0\\
279	0\\
280	0\\
281	0\\
282	0\\
283	0\\
284	0\\
285	0\\
286	0\\
287	0\\
288	0\\
289	0\\
290	0\\
291	0\\
292	0\\
293	0\\
294	0\\
295	0\\
296	0\\
297	0\\
298	0\\
299	0\\
300	0\\
301	0\\
302	0\\
303	0\\
304	0\\
305	0\\
306	0\\
307	0\\
308	0\\
309	0\\
310	0\\
311	0\\
312	0\\
313	0\\
314	0\\
315	0\\
316	0\\
317	0\\
318	0\\
319	0\\
320	0\\
321	0\\
322	0\\
323	0\\
324	0\\
325	0\\
326	0\\
327	0\\
328	0\\
329	0\\
330	0\\
331	0\\
332	0\\
333	0\\
334	0\\
335	0\\
336	0\\
337	0\\
338	0\\
339	0\\
340	0\\
341	0\\
342	0\\
343	0\\
344	0\\
345	0\\
346	0\\
347	0\\
348	0\\
349	0\\
350	0\\
351	0\\
352	0\\
353	0\\
354	0\\
355	0\\
356	0\\
357	0\\
358	0\\
359	0\\
360	0\\
361	0\\
362	0\\
363	0\\
364	0\\
365	0\\
366	0\\
367	0\\
368	0\\
369	0\\
370	0\\
371	0\\
372	0\\
373	0\\
374	0\\
375	0\\
376	0\\
377	0\\
378	0\\
379	0\\
380	0\\
381	0\\
382	0\\
383	0\\
384	0\\
385	0\\
386	0\\
387	0\\
388	0\\
389	0\\
390	0\\
391	0\\
392	0\\
393	0\\
394	0\\
395	0\\
396	0\\
397	0\\
398	0\\
399	0\\
400	0\\
401	0\\
402	0\\
403	0\\
404	0\\
405	0\\
406	0\\
407	0\\
408	0\\
409	0\\
410	0\\
411	0\\
412	0\\
413	0\\
414	0\\
415	0\\
416	0\\
417	0\\
418	0\\
419	0\\
420	0\\
421	0\\
422	0\\
423	0\\
424	0\\
425	0\\
426	0\\
427	0\\
428	0\\
429	0\\
430	0\\
431	0\\
432	0\\
433	0\\
434	0\\
435	0\\
436	0\\
437	0\\
438	0\\
439	0\\
440	0\\
441	0\\
442	0\\
443	0\\
444	0\\
445	0\\
446	0\\
447	0\\
448	0\\
449	0\\
450	0\\
451	0\\
452	0\\
453	0\\
454	0\\
455	0\\
456	0\\
457	0\\
458	0\\
459	0\\
460	0\\
461	0\\
462	0\\
463	0\\
464	0\\
465	0\\
466	0\\
467	0\\
468	0\\
469	0\\
470	0\\
471	0\\
472	0\\
473	0\\
474	0\\
475	0\\
476	0\\
477	0\\
478	0\\
479	0\\
480	0\\
481	0\\
482	0\\
483	0\\
484	0\\
485	0\\
486	0\\
487	0\\
488	0\\
489	0\\
490	0\\
491	0\\
492	0\\
493	0\\
494	0\\
495	0\\
496	0\\
497	0\\
498	0\\
499	0\\
500	0\\
501	0\\
502	0\\
503	0\\
504	0\\
505	0\\
506	0\\
507	0\\
508	0\\
509	0\\
510	0\\
511	0\\
512	0\\
513	0\\
514	0\\
515	0\\
516	0\\
517	0\\
518	0\\
519	0\\
520	0\\
521	0\\
522	0\\
523	0\\
524	0\\
525	0\\
526	0\\
527	0\\
528	0\\
529	0\\
530	0\\
531	0\\
532	0\\
533	0\\
534	0\\
535	0\\
536	0\\
537	0\\
538	0\\
539	0\\
540	0\\
541	0\\
542	0\\
543	0\\
544	0\\
545	0\\
546	0\\
547	0\\
548	0\\
549	0\\
550	0\\
551	0\\
552	0\\
553	0\\
554	0\\
555	1.91719487111611e-05\\
556	6.51916416175343e-05\\
557	0.000111751777533538\\
558	0.000158812301352876\\
559	0.000206325699048006\\
560	0.000254245520571927\\
561	0.000302505267507824\\
562	0.000351027678648465\\
563	0.000399723539571551\\
564	0.000448490252455449\\
565	0.000497210249201547\\
566	0.000545749008894194\\
567	0.00059395425796242\\
568	0.000643093628485419\\
569	0.000693247352774227\\
570	0.000744444297273541\\
571	0.000796717437008355\\
572	0.000850102988926268\\
573	0.000904635303798376\\
574	0.00096035463702777\\
575	0.00101730890690584\\
576	0.0010755559393491\\
577	0.0011351665831607\\
578	0.00119622852090244\\
579	0.00125885100335975\\
580	0.00132315112991813\\
581	0.00138919553049845\\
582	0.00145709190305839\\
583	0.00152687282332745\\
584	0.00159873739035049\\
585	0.00167260810899154\\
586	0.00174851745901825\\
587	0.00182712492469281\\
588	0.0020052356741675\\
589	0.00228207394121861\\
590	0.0025666370940843\\
591	0.00286021103851328\\
592	0.00312453732154687\\
593	0.00341862421613178\\
594	0.00377220464640417\\
595	0.00407888616628598\\
596	0.00440815817174734\\
597	0.00501118703192877\\
598	0.00642488516645657\\
599	0\\
600	0\\
};
\addplot [color=mycolor10,solid,forget plot]
  table[row sep=crcr]{%
1	0\\
2	0\\
3	0\\
4	0\\
5	0\\
6	0\\
7	0\\
8	0\\
9	0\\
10	0\\
11	0\\
12	0\\
13	0\\
14	0\\
15	0\\
16	0\\
17	0\\
18	0\\
19	0\\
20	0\\
21	0\\
22	0\\
23	0\\
24	0\\
25	0\\
26	0\\
27	0\\
28	0\\
29	0\\
30	0\\
31	0\\
32	0\\
33	0\\
34	0\\
35	0\\
36	0\\
37	0\\
38	0\\
39	0\\
40	0\\
41	0\\
42	0\\
43	0\\
44	0\\
45	0\\
46	0\\
47	0\\
48	0\\
49	0\\
50	0\\
51	0\\
52	0\\
53	0\\
54	0\\
55	0\\
56	0\\
57	0\\
58	0\\
59	0\\
60	0\\
61	0\\
62	0\\
63	0\\
64	0\\
65	0\\
66	0\\
67	0\\
68	0\\
69	0\\
70	0\\
71	0\\
72	0\\
73	0\\
74	0\\
75	0\\
76	0\\
77	0\\
78	0\\
79	0\\
80	0\\
81	0\\
82	0\\
83	0\\
84	0\\
85	0\\
86	0\\
87	0\\
88	0\\
89	0\\
90	0\\
91	0\\
92	0\\
93	0\\
94	0\\
95	0\\
96	0\\
97	0\\
98	0\\
99	0\\
100	0\\
101	0\\
102	0\\
103	0\\
104	0\\
105	0\\
106	0\\
107	0\\
108	0\\
109	0\\
110	0\\
111	0\\
112	0\\
113	0\\
114	0\\
115	0\\
116	0\\
117	0\\
118	0\\
119	0\\
120	0\\
121	0\\
122	0\\
123	0\\
124	0\\
125	0\\
126	0\\
127	0\\
128	0\\
129	0\\
130	0\\
131	0\\
132	0\\
133	0\\
134	0\\
135	0\\
136	0\\
137	0\\
138	0\\
139	0\\
140	0\\
141	0\\
142	0\\
143	0\\
144	0\\
145	0\\
146	0\\
147	0\\
148	0\\
149	0\\
150	0\\
151	0\\
152	0\\
153	0\\
154	0\\
155	0\\
156	0\\
157	0\\
158	0\\
159	0\\
160	0\\
161	0\\
162	0\\
163	0\\
164	0\\
165	0\\
166	0\\
167	0\\
168	0\\
169	0\\
170	0\\
171	0\\
172	0\\
173	0\\
174	0\\
175	0\\
176	0\\
177	0\\
178	0\\
179	0\\
180	0\\
181	0\\
182	0\\
183	0\\
184	0\\
185	0\\
186	0\\
187	0\\
188	0\\
189	0\\
190	0\\
191	0\\
192	0\\
193	0\\
194	0\\
195	0\\
196	0\\
197	0\\
198	0\\
199	0\\
200	0\\
201	0\\
202	0\\
203	0\\
204	0\\
205	0\\
206	0\\
207	0\\
208	0\\
209	0\\
210	0\\
211	0\\
212	0\\
213	0\\
214	0\\
215	0\\
216	0\\
217	0\\
218	0\\
219	0\\
220	0\\
221	0\\
222	0\\
223	0\\
224	0\\
225	0\\
226	0\\
227	0\\
228	0\\
229	0\\
230	0\\
231	0\\
232	0\\
233	0\\
234	0\\
235	0\\
236	0\\
237	0\\
238	0\\
239	0\\
240	0\\
241	0\\
242	0\\
243	0\\
244	0\\
245	0\\
246	0\\
247	0\\
248	0\\
249	0\\
250	0\\
251	0\\
252	0\\
253	0\\
254	0\\
255	0\\
256	0\\
257	0\\
258	0\\
259	0\\
260	0\\
261	0\\
262	0\\
263	0\\
264	0\\
265	0\\
266	0\\
267	0\\
268	0\\
269	0\\
270	0\\
271	0\\
272	0\\
273	0\\
274	0\\
275	0\\
276	0\\
277	0\\
278	0\\
279	0\\
280	0\\
281	0\\
282	0\\
283	0\\
284	0\\
285	0\\
286	0\\
287	0\\
288	0\\
289	0\\
290	0\\
291	0\\
292	0\\
293	0\\
294	0\\
295	0\\
296	0\\
297	0\\
298	0\\
299	0\\
300	0\\
301	0\\
302	0\\
303	0\\
304	0\\
305	0\\
306	0\\
307	0\\
308	0\\
309	0\\
310	0\\
311	0\\
312	0\\
313	0\\
314	0\\
315	0\\
316	0\\
317	0\\
318	0\\
319	0\\
320	0\\
321	0\\
322	0\\
323	0\\
324	0\\
325	0\\
326	0\\
327	0\\
328	0\\
329	0\\
330	0\\
331	0\\
332	0\\
333	0\\
334	0\\
335	0\\
336	0\\
337	0\\
338	0\\
339	0\\
340	0\\
341	0\\
342	0\\
343	0\\
344	0\\
345	0\\
346	0\\
347	0\\
348	0\\
349	0\\
350	0\\
351	0\\
352	0\\
353	0\\
354	0\\
355	0\\
356	0\\
357	0\\
358	0\\
359	0\\
360	0\\
361	0\\
362	0\\
363	0\\
364	0\\
365	0\\
366	0\\
367	0\\
368	0\\
369	0\\
370	0\\
371	0\\
372	0\\
373	0\\
374	0\\
375	0\\
376	0\\
377	0\\
378	0\\
379	0\\
380	0\\
381	0\\
382	0\\
383	0\\
384	0\\
385	0\\
386	0\\
387	0\\
388	0\\
389	0\\
390	0\\
391	0\\
392	0\\
393	0\\
394	0\\
395	0\\
396	0\\
397	0\\
398	0\\
399	0\\
400	0\\
401	0\\
402	0\\
403	0\\
404	0\\
405	0\\
406	0\\
407	0\\
408	0\\
409	0\\
410	0\\
411	0\\
412	0\\
413	0\\
414	0\\
415	0\\
416	0\\
417	0\\
418	0\\
419	0\\
420	0\\
421	0\\
422	0\\
423	0\\
424	0\\
425	0\\
426	0\\
427	0\\
428	0\\
429	0\\
430	0\\
431	0\\
432	0\\
433	0\\
434	0\\
435	0\\
436	0\\
437	0\\
438	0\\
439	0\\
440	0\\
441	0\\
442	0\\
443	0\\
444	0\\
445	0\\
446	0\\
447	0\\
448	0\\
449	0\\
450	0\\
451	0\\
452	0\\
453	0\\
454	0\\
455	0\\
456	0\\
457	0\\
458	0\\
459	0\\
460	0\\
461	0\\
462	0\\
463	0\\
464	0\\
465	0\\
466	0\\
467	0\\
468	0\\
469	0\\
470	0\\
471	0\\
472	0\\
473	0\\
474	0\\
475	0\\
476	0\\
477	0\\
478	0\\
479	0\\
480	0\\
481	0\\
482	0\\
483	0\\
484	0\\
485	0\\
486	0\\
487	0\\
488	0\\
489	0\\
490	0\\
491	0\\
492	0\\
493	0\\
494	0\\
495	0\\
496	0\\
497	0\\
498	0\\
499	0\\
500	0\\
501	0\\
502	0\\
503	0\\
504	0\\
505	0\\
506	0\\
507	0\\
508	0\\
509	0\\
510	0\\
511	0\\
512	0\\
513	0\\
514	0\\
515	0\\
516	0\\
517	0\\
518	0\\
519	0\\
520	0\\
521	0\\
522	0\\
523	0\\
524	0\\
525	0\\
526	0\\
527	0\\
528	0\\
529	0\\
530	0\\
531	0\\
532	0\\
533	0\\
534	0\\
535	0\\
536	0\\
537	0\\
538	0\\
539	0\\
540	0\\
541	0\\
542	0\\
543	0\\
544	0\\
545	0\\
546	0\\
547	0\\
548	0\\
549	0\\
550	0\\
551	0\\
552	0\\
553	0\\
554	0\\
555	0\\
556	0\\
557	0\\
558	0\\
559	0\\
560	0\\
561	0\\
562	0\\
563	0\\
564	6.38099865833115e-05\\
565	0.000154253607646718\\
566	0.000245683738061437\\
567	0.000337871408199173\\
568	0.000399914179502145\\
569	0.000461561946766633\\
570	0.00052410013148521\\
571	0.000587469940331828\\
572	0.000651657865042651\\
573	0.000716829079787043\\
574	0.000782921384134708\\
575	0.000849856385345376\\
576	0.000917537151764115\\
577	0.000985844560883639\\
578	0.00105463204067166\\
579	0.00112371753012356\\
580	0.00119324941240428\\
581	0.00126444776909183\\
582	0.00133734372880463\\
583	0.00141196851593021\\
584	0.00148835333464273\\
585	0.00156653037063608\\
586	0.00164653267396629\\
587	0.00172839481985068\\
588	0.00181217273894298\\
589	0.00189801407111492\\
590	0.00198587444025641\\
591	0.00211599393384859\\
592	0.00239672135457705\\
593	0.00268923432027998\\
594	0.003042228405148\\
595	0.00347514822745281\\
596	0.00421265278177493\\
597	0.00501118703192877\\
598	0.00642488516645657\\
599	0\\
600	0\\
};
\addplot [color=mycolor11,solid,forget plot]
  table[row sep=crcr]{%
1	0\\
2	0\\
3	0\\
4	0\\
5	0\\
6	0\\
7	0\\
8	0\\
9	0\\
10	0\\
11	0\\
12	0\\
13	0\\
14	0\\
15	0\\
16	0\\
17	0\\
18	0\\
19	0\\
20	0\\
21	0\\
22	0\\
23	0\\
24	0\\
25	0\\
26	0\\
27	0\\
28	0\\
29	0\\
30	0\\
31	0\\
32	0\\
33	0\\
34	0\\
35	0\\
36	0\\
37	0\\
38	0\\
39	0\\
40	0\\
41	0\\
42	0\\
43	0\\
44	0\\
45	0\\
46	0\\
47	0\\
48	0\\
49	0\\
50	0\\
51	0\\
52	0\\
53	0\\
54	0\\
55	0\\
56	0\\
57	0\\
58	0\\
59	0\\
60	0\\
61	0\\
62	0\\
63	0\\
64	0\\
65	0\\
66	0\\
67	0\\
68	0\\
69	0\\
70	0\\
71	0\\
72	0\\
73	0\\
74	0\\
75	0\\
76	0\\
77	0\\
78	0\\
79	0\\
80	0\\
81	0\\
82	0\\
83	0\\
84	0\\
85	0\\
86	0\\
87	0\\
88	0\\
89	0\\
90	0\\
91	0\\
92	0\\
93	0\\
94	0\\
95	0\\
96	0\\
97	0\\
98	0\\
99	0\\
100	0\\
101	0\\
102	0\\
103	0\\
104	0\\
105	0\\
106	0\\
107	0\\
108	0\\
109	0\\
110	0\\
111	0\\
112	0\\
113	0\\
114	0\\
115	0\\
116	0\\
117	0\\
118	0\\
119	0\\
120	0\\
121	0\\
122	0\\
123	0\\
124	0\\
125	0\\
126	0\\
127	0\\
128	0\\
129	0\\
130	0\\
131	0\\
132	0\\
133	0\\
134	0\\
135	0\\
136	0\\
137	0\\
138	0\\
139	0\\
140	0\\
141	0\\
142	0\\
143	0\\
144	0\\
145	0\\
146	0\\
147	0\\
148	0\\
149	0\\
150	0\\
151	0\\
152	0\\
153	0\\
154	0\\
155	0\\
156	0\\
157	0\\
158	0\\
159	0\\
160	0\\
161	0\\
162	0\\
163	0\\
164	0\\
165	0\\
166	0\\
167	0\\
168	0\\
169	0\\
170	0\\
171	0\\
172	0\\
173	0\\
174	0\\
175	0\\
176	0\\
177	0\\
178	0\\
179	0\\
180	0\\
181	0\\
182	0\\
183	0\\
184	0\\
185	0\\
186	0\\
187	0\\
188	0\\
189	0\\
190	0\\
191	0\\
192	0\\
193	0\\
194	0\\
195	0\\
196	0\\
197	0\\
198	0\\
199	0\\
200	0\\
201	0\\
202	0\\
203	0\\
204	0\\
205	0\\
206	0\\
207	0\\
208	0\\
209	0\\
210	0\\
211	0\\
212	0\\
213	0\\
214	0\\
215	0\\
216	0\\
217	0\\
218	0\\
219	0\\
220	0\\
221	0\\
222	0\\
223	0\\
224	0\\
225	0\\
226	0\\
227	0\\
228	0\\
229	0\\
230	0\\
231	0\\
232	0\\
233	0\\
234	0\\
235	0\\
236	0\\
237	0\\
238	0\\
239	0\\
240	0\\
241	0\\
242	0\\
243	0\\
244	0\\
245	0\\
246	0\\
247	0\\
248	0\\
249	0\\
250	0\\
251	0\\
252	0\\
253	0\\
254	0\\
255	0\\
256	0\\
257	0\\
258	0\\
259	0\\
260	0\\
261	0\\
262	0\\
263	0\\
264	0\\
265	0\\
266	0\\
267	0\\
268	0\\
269	0\\
270	0\\
271	0\\
272	0\\
273	0\\
274	0\\
275	0\\
276	0\\
277	0\\
278	0\\
279	0\\
280	0\\
281	0\\
282	0\\
283	0\\
284	0\\
285	0\\
286	0\\
287	0\\
288	0\\
289	0\\
290	0\\
291	0\\
292	0\\
293	0\\
294	0\\
295	0\\
296	0\\
297	0\\
298	0\\
299	0\\
300	0\\
301	0\\
302	0\\
303	0\\
304	0\\
305	0\\
306	0\\
307	0\\
308	0\\
309	0\\
310	0\\
311	0\\
312	0\\
313	0\\
314	0\\
315	0\\
316	0\\
317	0\\
318	0\\
319	0\\
320	0\\
321	0\\
322	0\\
323	0\\
324	0\\
325	0\\
326	0\\
327	0\\
328	0\\
329	0\\
330	0\\
331	0\\
332	0\\
333	0\\
334	0\\
335	0\\
336	0\\
337	0\\
338	0\\
339	0\\
340	0\\
341	0\\
342	0\\
343	0\\
344	0\\
345	0\\
346	0\\
347	0\\
348	0\\
349	0\\
350	0\\
351	0\\
352	0\\
353	0\\
354	0\\
355	0\\
356	0\\
357	0\\
358	0\\
359	0\\
360	0\\
361	0\\
362	0\\
363	0\\
364	0\\
365	0\\
366	0\\
367	0\\
368	0\\
369	0\\
370	0\\
371	0\\
372	0\\
373	0\\
374	0\\
375	0\\
376	0\\
377	0\\
378	0\\
379	0\\
380	0\\
381	0\\
382	0\\
383	0\\
384	0\\
385	0\\
386	0\\
387	0\\
388	0\\
389	0\\
390	0\\
391	0\\
392	0\\
393	0\\
394	0\\
395	0\\
396	0\\
397	0\\
398	0\\
399	0\\
400	0\\
401	0\\
402	0\\
403	0\\
404	0\\
405	0\\
406	0\\
407	0\\
408	0\\
409	0\\
410	0\\
411	0\\
412	0\\
413	0\\
414	0\\
415	0\\
416	0\\
417	0\\
418	0\\
419	0\\
420	0\\
421	0\\
422	0\\
423	0\\
424	0\\
425	0\\
426	0\\
427	0\\
428	0\\
429	0\\
430	0\\
431	0\\
432	0\\
433	0\\
434	0\\
435	0\\
436	0\\
437	0\\
438	0\\
439	0\\
440	0\\
441	0\\
442	0\\
443	0\\
444	0\\
445	0\\
446	0\\
447	0\\
448	0\\
449	0\\
450	0\\
451	0\\
452	0\\
453	0\\
454	0\\
455	0\\
456	0\\
457	0\\
458	0\\
459	0\\
460	0\\
461	0\\
462	0\\
463	0\\
464	0\\
465	0\\
466	0\\
467	0\\
468	0\\
469	0\\
470	0\\
471	0\\
472	0\\
473	0\\
474	0\\
475	0\\
476	0\\
477	0\\
478	0\\
479	0\\
480	0\\
481	0\\
482	0\\
483	0\\
484	0\\
485	0\\
486	0\\
487	0\\
488	0\\
489	0\\
490	0\\
491	0\\
492	0\\
493	0\\
494	0\\
495	0\\
496	0\\
497	0\\
498	0\\
499	0\\
500	0\\
501	0\\
502	0\\
503	0\\
504	0\\
505	0\\
506	0\\
507	0\\
508	0\\
509	0\\
510	0\\
511	0\\
512	0\\
513	0\\
514	0\\
515	0\\
516	0\\
517	0\\
518	0\\
519	0\\
520	0\\
521	0\\
522	0\\
523	0\\
524	0\\
525	0\\
526	0\\
527	0\\
528	0\\
529	0\\
530	0\\
531	0\\
532	0\\
533	0\\
534	0\\
535	0\\
536	0\\
537	0\\
538	0\\
539	0\\
540	0\\
541	0\\
542	0\\
543	0\\
544	0\\
545	0\\
546	0\\
547	0\\
548	0\\
549	0\\
550	0\\
551	0\\
552	0\\
553	0\\
554	0\\
555	0\\
556	0\\
557	0\\
558	0\\
559	0\\
560	0\\
561	0\\
562	0\\
563	0\\
564	0\\
565	0\\
566	0\\
567	0\\
568	0\\
569	0\\
570	0\\
571	0\\
572	0\\
573	4.04709118961138e-05\\
574	0.000146135873699704\\
575	0.000254345046972783\\
576	0.000365036226188813\\
577	0.000477991922738407\\
578	0.000592630547932805\\
579	0.000709819060000769\\
580	0.000822303469911085\\
581	0.000910282263958814\\
582	0.00100058720699734\\
583	0.00109325009451827\\
584	0.00118829487178716\\
585	0.0012857341382992\\
586	0.00138559035948951\\
587	0.00148783487413243\\
588	0.00159241035075032\\
589	0.0016992221476438\\
590	0.00180812756853647\\
591	0.00191892108123612\\
592	0.00203131802967084\\
593	0.0021449343752569\\
594	0.00225924019373236\\
595	0.0025990765750797\\
596	0.00331669012914152\\
597	0.00466497804100969\\
598	0.00642488516645657\\
599	0\\
600	0\\
};
\addplot [color=mycolor12,solid,forget plot]
  table[row sep=crcr]{%
1	0\\
2	0\\
3	0\\
4	0\\
5	0\\
6	0\\
7	0\\
8	0\\
9	0\\
10	0\\
11	0\\
12	0\\
13	0\\
14	0\\
15	0\\
16	0\\
17	0\\
18	0\\
19	0\\
20	0\\
21	0\\
22	0\\
23	0\\
24	0\\
25	0\\
26	0\\
27	0\\
28	0\\
29	0\\
30	0\\
31	0\\
32	0\\
33	0\\
34	0\\
35	0\\
36	0\\
37	0\\
38	0\\
39	0\\
40	0\\
41	0\\
42	0\\
43	0\\
44	0\\
45	0\\
46	0\\
47	0\\
48	0\\
49	0\\
50	0\\
51	0\\
52	0\\
53	0\\
54	0\\
55	0\\
56	0\\
57	0\\
58	0\\
59	0\\
60	0\\
61	0\\
62	0\\
63	0\\
64	0\\
65	0\\
66	0\\
67	0\\
68	0\\
69	0\\
70	0\\
71	0\\
72	0\\
73	0\\
74	0\\
75	0\\
76	0\\
77	0\\
78	0\\
79	0\\
80	0\\
81	0\\
82	0\\
83	0\\
84	0\\
85	0\\
86	0\\
87	0\\
88	0\\
89	0\\
90	0\\
91	0\\
92	0\\
93	0\\
94	0\\
95	0\\
96	0\\
97	0\\
98	0\\
99	0\\
100	0\\
101	0\\
102	0\\
103	0\\
104	0\\
105	0\\
106	0\\
107	0\\
108	0\\
109	0\\
110	0\\
111	0\\
112	0\\
113	0\\
114	0\\
115	0\\
116	0\\
117	0\\
118	0\\
119	0\\
120	0\\
121	0\\
122	0\\
123	0\\
124	0\\
125	0\\
126	0\\
127	0\\
128	0\\
129	0\\
130	0\\
131	0\\
132	0\\
133	0\\
134	0\\
135	0\\
136	0\\
137	0\\
138	0\\
139	0\\
140	0\\
141	0\\
142	0\\
143	0\\
144	0\\
145	0\\
146	0\\
147	0\\
148	0\\
149	0\\
150	0\\
151	0\\
152	0\\
153	0\\
154	0\\
155	0\\
156	0\\
157	0\\
158	0\\
159	0\\
160	0\\
161	0\\
162	0\\
163	0\\
164	0\\
165	0\\
166	0\\
167	0\\
168	0\\
169	0\\
170	0\\
171	0\\
172	0\\
173	0\\
174	0\\
175	0\\
176	0\\
177	0\\
178	0\\
179	0\\
180	0\\
181	0\\
182	0\\
183	0\\
184	0\\
185	0\\
186	0\\
187	0\\
188	0\\
189	0\\
190	0\\
191	0\\
192	0\\
193	0\\
194	0\\
195	0\\
196	0\\
197	0\\
198	0\\
199	0\\
200	0\\
201	0\\
202	0\\
203	0\\
204	0\\
205	0\\
206	0\\
207	0\\
208	0\\
209	0\\
210	0\\
211	0\\
212	0\\
213	0\\
214	0\\
215	0\\
216	0\\
217	0\\
218	0\\
219	0\\
220	0\\
221	0\\
222	0\\
223	0\\
224	0\\
225	0\\
226	0\\
227	0\\
228	0\\
229	0\\
230	0\\
231	0\\
232	0\\
233	0\\
234	0\\
235	0\\
236	0\\
237	0\\
238	0\\
239	0\\
240	0\\
241	0\\
242	0\\
243	0\\
244	0\\
245	0\\
246	0\\
247	0\\
248	0\\
249	0\\
250	0\\
251	0\\
252	0\\
253	0\\
254	0\\
255	0\\
256	0\\
257	0\\
258	0\\
259	0\\
260	0\\
261	0\\
262	0\\
263	0\\
264	0\\
265	0\\
266	0\\
267	0\\
268	0\\
269	0\\
270	0\\
271	0\\
272	0\\
273	0\\
274	0\\
275	0\\
276	0\\
277	0\\
278	0\\
279	0\\
280	0\\
281	0\\
282	0\\
283	0\\
284	0\\
285	0\\
286	0\\
287	0\\
288	0\\
289	0\\
290	0\\
291	0\\
292	0\\
293	0\\
294	0\\
295	0\\
296	0\\
297	0\\
298	0\\
299	0\\
300	0\\
301	0\\
302	0\\
303	0\\
304	0\\
305	0\\
306	0\\
307	0\\
308	0\\
309	0\\
310	0\\
311	0\\
312	0\\
313	0\\
314	0\\
315	0\\
316	0\\
317	0\\
318	0\\
319	0\\
320	0\\
321	0\\
322	0\\
323	0\\
324	0\\
325	0\\
326	0\\
327	0\\
328	0\\
329	0\\
330	0\\
331	0\\
332	0\\
333	0\\
334	0\\
335	0\\
336	0\\
337	0\\
338	0\\
339	0\\
340	0\\
341	0\\
342	0\\
343	0\\
344	0\\
345	0\\
346	0\\
347	0\\
348	0\\
349	0\\
350	0\\
351	0\\
352	0\\
353	0\\
354	0\\
355	0\\
356	0\\
357	0\\
358	0\\
359	0\\
360	0\\
361	0\\
362	0\\
363	0\\
364	0\\
365	0\\
366	0\\
367	0\\
368	0\\
369	0\\
370	0\\
371	0\\
372	0\\
373	0\\
374	0\\
375	0\\
376	0\\
377	0\\
378	0\\
379	0\\
380	0\\
381	0\\
382	0\\
383	0\\
384	0\\
385	0\\
386	0\\
387	0\\
388	0\\
389	0\\
390	0\\
391	0\\
392	0\\
393	0\\
394	0\\
395	0\\
396	0\\
397	0\\
398	0\\
399	0\\
400	0\\
401	0\\
402	0\\
403	0\\
404	0\\
405	0\\
406	0\\
407	0\\
408	0\\
409	0\\
410	0\\
411	0\\
412	0\\
413	0\\
414	0\\
415	0\\
416	0\\
417	0\\
418	0\\
419	0\\
420	0\\
421	0\\
422	0\\
423	0\\
424	0\\
425	0\\
426	0\\
427	0\\
428	0\\
429	0\\
430	0\\
431	0\\
432	0\\
433	0\\
434	0\\
435	0\\
436	0\\
437	0\\
438	0\\
439	0\\
440	0\\
441	0\\
442	0\\
443	0\\
444	0\\
445	0\\
446	0\\
447	0\\
448	0\\
449	0\\
450	0\\
451	0\\
452	0\\
453	0\\
454	0\\
455	0\\
456	0\\
457	0\\
458	0\\
459	0\\
460	0\\
461	0\\
462	0\\
463	0\\
464	0\\
465	0\\
466	0\\
467	0\\
468	0\\
469	0\\
470	0\\
471	0\\
472	0\\
473	0\\
474	0\\
475	0\\
476	0\\
477	0\\
478	0\\
479	0\\
480	0\\
481	0\\
482	0\\
483	0\\
484	0\\
485	0\\
486	0\\
487	0\\
488	0\\
489	0\\
490	0\\
491	0\\
492	0\\
493	0\\
494	0\\
495	0\\
496	0\\
497	0\\
498	0\\
499	0\\
500	0\\
501	0\\
502	0\\
503	0\\
504	0\\
505	0\\
506	0\\
507	0\\
508	0\\
509	0\\
510	0\\
511	0\\
512	0\\
513	0\\
514	0\\
515	0\\
516	0\\
517	0\\
518	0\\
519	0\\
520	0\\
521	0\\
522	0\\
523	0\\
524	0\\
525	0\\
526	0\\
527	0\\
528	0\\
529	0\\
530	0\\
531	0\\
532	0\\
533	0\\
534	0\\
535	0\\
536	0\\
537	0\\
538	0\\
539	0\\
540	0\\
541	0\\
542	0\\
543	0\\
544	0\\
545	0\\
546	0\\
547	0\\
548	0\\
549	0\\
550	0\\
551	0\\
552	0\\
553	0\\
554	0\\
555	0\\
556	0\\
557	0\\
558	0\\
559	0\\
560	0\\
561	0\\
562	0\\
563	0\\
564	0\\
565	0\\
566	0\\
567	0\\
568	0\\
569	0\\
570	0\\
571	0\\
572	0\\
573	0\\
574	0\\
575	0\\
576	0\\
577	0\\
578	0\\
579	0\\
580	0\\
581	0\\
582	0\\
583	0\\
584	0\\
585	0.00013774192519883\\
586	0.000284032666100375\\
587	0.000435941232690743\\
588	0.000593865412635933\\
589	0.000758286829767887\\
590	0.000929880038087214\\
591	0.00110901579535917\\
592	0.0012962244624629\\
593	0.00149227987384335\\
594	0.00169817885362274\\
595	0.0019152966930724\\
596	0.00214579709295045\\
597	0.00338203886614602\\
598	0.00642488516645657\\
599	0\\
600	0\\
};
\addplot [color=mycolor13,solid,forget plot]
  table[row sep=crcr]{%
1	0.0013780462490059\\
2	0.0013780462490059\\
3	0.0013780462490059\\
4	0.0013780462490059\\
5	0.0013780462490059\\
6	0.0013780462490059\\
7	0.0013780462490059\\
8	0.0013780462490059\\
9	0.0013780462490059\\
10	0.0013780462490059\\
11	0.0013780462490059\\
12	0.0013780462490059\\
13	0.0013780462490059\\
14	0.0013780462490059\\
15	0.0013780462490059\\
16	0.0013780462490059\\
17	0.0013780462490059\\
18	0.0013780462490059\\
19	0.0013780462490059\\
20	0.0013780462490059\\
21	0.0013780462490059\\
22	0.0013780462490059\\
23	0.0013780462490059\\
24	0.0013780462490059\\
25	0.0013780462490059\\
26	0.0013780462490059\\
27	0.0013780462490059\\
28	0.0013780462490059\\
29	0.0013780462490059\\
30	0.0013780462490059\\
31	0.0013780462490059\\
32	0.0013780462490059\\
33	0.0013780462490059\\
34	0.0013780462490059\\
35	0.0013780462490059\\
36	0.0013780462490059\\
37	0.0013780462490059\\
38	0.0013780462490059\\
39	0.0013780462490059\\
40	0.0013780462490059\\
41	0.0013780462490059\\
42	0.0013780462490059\\
43	0.0013780462490059\\
44	0.0013780462490059\\
45	0.0013780462490059\\
46	0.0013780462490059\\
47	0.0013780462490059\\
48	0.0013780462490059\\
49	0.0013780462490059\\
50	0.0013780462490059\\
51	0.0013780462490059\\
52	0.0013780462490059\\
53	0.0013780462490059\\
54	0.0013780462490059\\
55	0.0013780462490059\\
56	0.0013780462490059\\
57	0.0013780462490059\\
58	0.0013780462490059\\
59	0.0013780462490059\\
60	0.0013780462490059\\
61	0.0013780462490059\\
62	0.0013780462490059\\
63	0.0013780462490059\\
64	0.0013780462490059\\
65	0.0013780462490059\\
66	0.0013780462490059\\
67	0.0013780462490059\\
68	0.0013780462490059\\
69	0.0013780462490059\\
70	0.0013780462490059\\
71	0.0013780462490059\\
72	0.0013780462490059\\
73	0.0013780462490059\\
74	0.0013780462490059\\
75	0.0013780462490059\\
76	0.0013780462490059\\
77	0.0013780462490059\\
78	0.0013780462490059\\
79	0.0013780462490059\\
80	0.0013780462490059\\
81	0.0013780462490059\\
82	0.0013780462490059\\
83	0.0013780462490059\\
84	0.0013780462490059\\
85	0.0013780462490059\\
86	0.0013780462490059\\
87	0.0013780462490059\\
88	0.0013780462490059\\
89	0.0013780462490059\\
90	0.0013780462490059\\
91	0.0013780462490059\\
92	0.0013780462490059\\
93	0.0013780462490059\\
94	0.0013780462490059\\
95	0.0013780462490059\\
96	0.0013780462490059\\
97	0.0013780462490059\\
98	0.0013780462490059\\
99	0.0013780462490059\\
100	0.0013780462490059\\
101	0.0013780462490059\\
102	0.0013780462490059\\
103	0.0013780462490059\\
104	0.0013780462490059\\
105	0.0013780462490059\\
106	0.0013780462490059\\
107	0.0013780462490059\\
108	0.0013780462490059\\
109	0.0013780462490059\\
110	0.0013780462490059\\
111	0.0013780462490059\\
112	0.0013780462490059\\
113	0.0013780462490059\\
114	0.0013780462490059\\
115	0.0013780462490059\\
116	0.0013780462490059\\
117	0.0013780462490059\\
118	0.0013780462490059\\
119	0.0013780462490059\\
120	0.0013780462490059\\
121	0.0013780462490059\\
122	0.0013780462490059\\
123	0.0013780462490059\\
124	0.0013780462490059\\
125	0.0013780462490059\\
126	0.0013780462490059\\
127	0.0013780462490059\\
128	0.0013780462490059\\
129	0.0013780462490059\\
130	0.0013780462490059\\
131	0.0013780462490059\\
132	0.0013780462490059\\
133	0.0013780462490059\\
134	0.0013780462490059\\
135	0.0013780462490059\\
136	0.0013780462490059\\
137	0.0013780462490059\\
138	0.0013780462490059\\
139	0.0013780462490059\\
140	0.0013780462490059\\
141	0.0013780462490059\\
142	0.0013780462490059\\
143	0.0013780462490059\\
144	0.0013780462490059\\
145	0.0013780462490059\\
146	0.0013780462490059\\
147	0.0013780462490059\\
148	0.0013780462490059\\
149	0.0013780462490059\\
150	0.0013780462490059\\
151	0.0013780462490059\\
152	0.0013780462490059\\
153	0.0013780462490059\\
154	0.0013780462490059\\
155	0.0013780462490059\\
156	0.0013780462490059\\
157	0.0013780462490059\\
158	0.0013780462490059\\
159	0.0013780462490059\\
160	0.0013780462490059\\
161	0.0013780462490059\\
162	0.0013780462490059\\
163	0.0013780462490059\\
164	0.0013780462490059\\
165	0.0013780462490059\\
166	0.0013780462490059\\
167	0.0013780462490059\\
168	0.0013780462490059\\
169	0.0013780462490059\\
170	0.0013780462490059\\
171	0.0013780462490059\\
172	0.0013780462490059\\
173	0.0013780462490059\\
174	0.0013780462490059\\
175	0.0013780462490059\\
176	0.0013780462490059\\
177	0.0013780462490059\\
178	0.0013780462490059\\
179	0.0013780462490059\\
180	0.0013780462490059\\
181	0.0013780462490059\\
182	0.0013780462490059\\
183	0.0013780462490059\\
184	0.0013780462490059\\
185	0.0013780462490059\\
186	0.0013780462490059\\
187	0.0013780462490059\\
188	0.0013780462490059\\
189	0.0013780462490059\\
190	0.0013780462490059\\
191	0.0013780462490059\\
192	0.0013780462490059\\
193	0.0013780462490059\\
194	0.0013780462490059\\
195	0.0013780462490059\\
196	0.0013780462490059\\
197	0.0013780462490059\\
198	0.0013780462490059\\
199	0.0013780462490059\\
200	0.0013780462490059\\
201	0.0013780462490059\\
202	0.0013780462490059\\
203	0.0013780462490059\\
204	0.0013780462490059\\
205	0.0013780462490059\\
206	0.0013780462490059\\
207	0.0013780462490059\\
208	0.0013780462490059\\
209	0.0013780462490059\\
210	0.0013780462490059\\
211	0.0013780462490059\\
212	0.0013780462490059\\
213	0.0013780462490059\\
214	0.0013780462490059\\
215	0.0013780462490059\\
216	0.0013780462490059\\
217	0.0013780462490059\\
218	0.0013780462490059\\
219	0.0013780462490059\\
220	0.0013780462490059\\
221	0.0013780462490059\\
222	0.0013780462490059\\
223	0.0013780462490059\\
224	0.0013780462490059\\
225	0.0013780462490059\\
226	0.0013780462490059\\
227	0.0013780462490059\\
228	0.0013780462490059\\
229	0.0013780462490059\\
230	0.0013780462490059\\
231	0.0013780462490059\\
232	0.0013780462490059\\
233	0.0013780462490059\\
234	0.0013780462490059\\
235	0.0013780462490059\\
236	0.0013780462490059\\
237	0.0013780462490059\\
238	0.0013780462490059\\
239	0.0013780462490059\\
240	0.0013780462490059\\
241	0.0013780462490059\\
242	0.0013780462490059\\
243	0.0013780462490059\\
244	0.0013780462490059\\
245	0.0013780462490059\\
246	0.0013780462490059\\
247	0.0013780462490059\\
248	0.0013780462490059\\
249	0.0013780462490059\\
250	0.0013780462490059\\
251	0.0013780462490059\\
252	0.0013780462490059\\
253	0.0013780462490059\\
254	0.0013780462490059\\
255	0.0013780462490059\\
256	0.0013780462490059\\
257	0.0013780462490059\\
258	0.0013780462490059\\
259	0.0013780462490059\\
260	0.0013780462490059\\
261	0.0013780462490059\\
262	0.0013780462490059\\
263	0.0013780462490059\\
264	0.0013780462490059\\
265	0.0013780462490059\\
266	0.0013780462490059\\
267	0.0013780462490059\\
268	0.0013780462490059\\
269	0.0013780462490059\\
270	0.0013780462490059\\
271	0.0013780462490059\\
272	0.0013780462490059\\
273	0.0013780462490059\\
274	0.0013780462490059\\
275	0.0013780462490059\\
276	0.0013780462490059\\
277	0.0013780462490059\\
278	0.0013780462490059\\
279	0.0013780462490059\\
280	0.0013780462490059\\
281	0.0013780462490059\\
282	0.0013780462490059\\
283	0.0013780462490059\\
284	0.0013780462490059\\
285	0.0013780462490059\\
286	0.0013780462490059\\
287	0.0013780462490059\\
288	0.0013780462490059\\
289	0.0013780462490059\\
290	0.0013780462490059\\
291	0.0013780462490059\\
292	0.0013780462490059\\
293	0.0013780462490059\\
294	0.0013780462490059\\
295	0.0013780462490059\\
296	0.0013780462490059\\
297	0.0013780462490059\\
298	0.0013780462490059\\
299	0.0013780462490059\\
300	0.0013780462490059\\
301	0.0013780462490059\\
302	0.0013780462490059\\
303	0.0013780462490059\\
304	0.0013780462490059\\
305	0.0013780462490059\\
306	0.0013780462490059\\
307	0.0013780462490059\\
308	0.0013780462490059\\
309	0.0013780462490059\\
310	0.0013780462490059\\
311	0.0013780462490059\\
312	0.0013780462490059\\
313	0.0013780462490059\\
314	0.0013780462490059\\
315	0.0013780462490059\\
316	0.0013780462490059\\
317	0.0013780462490059\\
318	0.0013780462490059\\
319	0.0013780462490059\\
320	0.0013780462490059\\
321	0.0013780462490059\\
322	0.0013780462490059\\
323	0.0013780462490059\\
324	0.0013780462490059\\
325	0.0013780462490059\\
326	0.0013780462490059\\
327	0.0013780462490059\\
328	0.0013780462490059\\
329	0.0013780462490059\\
330	0.0013780462490059\\
331	0.0013780462490059\\
332	0.0013780462490059\\
333	0.0013780462490059\\
334	0.0013780462490059\\
335	0.0013780462490059\\
336	0.0013780462490059\\
337	0.0013780462490059\\
338	0.0013780462490059\\
339	0.0013780462490059\\
340	0.0013780462490059\\
341	0.0013780462490059\\
342	0.0013780462490059\\
343	0.0013780462490059\\
344	0.0013780462490059\\
345	0.0013780462490059\\
346	0.0013780462490059\\
347	0.0013780462490059\\
348	0.0013780462490059\\
349	0.0013780462490059\\
350	0.0013780462490059\\
351	0.0013780462490059\\
352	0.0013780462490059\\
353	0.0013780462490059\\
354	0.0013780462490059\\
355	0.0013780462490059\\
356	0.0013780462490059\\
357	0.0013780462490059\\
358	0.0013780462490059\\
359	0.0013780462490059\\
360	0.0013780462490059\\
361	0.0013780462490059\\
362	0.0013780462490059\\
363	0.0013780462490059\\
364	0.0013780462490059\\
365	0.0013780462490059\\
366	0.0013780462490059\\
367	0.0013780462490059\\
368	0.0013780462490059\\
369	0.0013780462490059\\
370	0.0013780462490059\\
371	0.0013780462490059\\
372	0.0013780462490059\\
373	0.0013780462490059\\
374	0.0013780462490059\\
375	0.0013780462490059\\
376	0.0013780462490059\\
377	0.0013780462490059\\
378	0.0013780462490059\\
379	0.0013780462490059\\
380	0.0013780462490059\\
381	0.0013780462490059\\
382	0.0013780462490059\\
383	0.0013780462490059\\
384	0.0013780462490059\\
385	0.0013780462490059\\
386	0.0013780462490059\\
387	0.0013780462490059\\
388	0.0013780462490059\\
389	0.0013780462490059\\
390	0.0013780462490059\\
391	0.0013780462490059\\
392	0.0013780462490059\\
393	0.0013780462490059\\
394	0.0013780462490059\\
395	0.0013780462490059\\
396	0.0013780462490059\\
397	0.0013780462490059\\
398	0.0013780462490059\\
399	0.0013780462490059\\
400	0.0013780462490059\\
401	0.0013780462490059\\
402	0.0013780462490059\\
403	0.0013780462490059\\
404	0.0013780462490059\\
405	0.0013780462490059\\
406	0.0013780462490059\\
407	0.0013780462490059\\
408	0.0013780462490059\\
409	0.0013780462490059\\
410	0.0013780462490059\\
411	0.0013780462490059\\
412	0.0013780462490059\\
413	0.0013780462490059\\
414	0.0013780462490059\\
415	0.0013780462490059\\
416	0.0013780462490059\\
417	0.0013780462490059\\
418	0.0013780462490059\\
419	0.0013780462490059\\
420	0.0013780462490059\\
421	0.0013780462490059\\
422	0.0013780462490059\\
423	0.0013780462490059\\
424	0.0013780462490059\\
425	0.0013780462490059\\
426	0.0013780462490059\\
427	0.0013780462490059\\
428	0.0013780462490059\\
429	0.0013780462490059\\
430	0.0013780462490059\\
431	0.0013780462490059\\
432	0.0013780462490059\\
433	0.0013780462490059\\
434	0.0013780462490059\\
435	0.0013780462490059\\
436	0.0013780462490059\\
437	0.0013780462490059\\
438	0.0013780462490059\\
439	0.0013780462490059\\
440	0.0013780462490059\\
441	0.0013780462490059\\
442	0.0013780462490059\\
443	0.0013780462490059\\
444	0.0013780462490059\\
445	0.0013780462490059\\
446	0.0013780462490059\\
447	0.0013780462490059\\
448	0.0013780462490059\\
449	0.0013780462490059\\
450	0.0013780462490059\\
451	0.0013780462490059\\
452	0.0013780462490059\\
453	0.0013780462490059\\
454	0.0013780462490059\\
455	0.0013780462490059\\
456	0.0013780462490059\\
457	0.0013780462490059\\
458	0.0013780462490059\\
459	0.0013780462490059\\
460	0.0013780462490059\\
461	0.0013780462490059\\
462	0.0013780462490059\\
463	0.0013780462490059\\
464	0.0013780462490059\\
465	0.0013780462490059\\
466	0.0013780462490059\\
467	0.0013780462490059\\
468	0.0013780462490059\\
469	0.0013780462490059\\
470	0.0013780462490059\\
471	0.0013780462490059\\
472	0.0013780462490059\\
473	0.0013780462490059\\
474	0.0013780462490059\\
475	0.0013780462490059\\
476	0.0013780462490059\\
477	0.0013780462490059\\
478	0.0013780462490059\\
479	0.0013780462490059\\
480	0.0013780462490059\\
481	0.0013780462490059\\
482	0.0013780462490059\\
483	0.0013780462490059\\
484	0.0013780462490059\\
485	0.0013780462490059\\
486	0.0013780462490059\\
487	0.0013780462490059\\
488	0.0013780462490059\\
489	0.0013780462490059\\
490	0.0013780462490059\\
491	0.0013780462490059\\
492	0.0013780462490059\\
493	0.0013780462490059\\
494	0.0013780462490059\\
495	0.0013780462490059\\
496	0.0013780462490059\\
497	0.0013780462490059\\
498	0.0013780462490059\\
499	0.0013780462490059\\
500	0.0013780462490059\\
501	0.0013780462490059\\
502	0.0013780462490059\\
503	0.0013780462490059\\
504	0.0013780462490059\\
505	0.0013780462490059\\
506	0.0013780462490059\\
507	0.0013780462490059\\
508	0.0013780462490059\\
509	0.0013780462490059\\
510	0.0013780462490059\\
511	0.0013780462490059\\
512	0.0013780462490059\\
513	0.0013780462490059\\
514	0.0013780462490059\\
515	0.0013780462490059\\
516	0.0013780462490059\\
517	0.0013780462490059\\
518	0.0013780462490059\\
519	0.0013780462490059\\
520	0.0013780462490059\\
521	0.0013780462490059\\
522	0.0013780462490059\\
523	0.0013780462490059\\
524	0.0013780462490059\\
525	0.0013780462490059\\
526	0.0013780462490059\\
527	0.0013780462490059\\
528	0.0013780462490059\\
529	0.0013780462490059\\
530	0.0013780462490059\\
531	0.0013780462490059\\
532	0.0013780462490059\\
533	0.0013780462490059\\
534	0.0013780462490059\\
535	0.0013780462490059\\
536	0.0013780462490059\\
537	0.0013780462490059\\
538	0.0013780462490059\\
539	0.0013780462490059\\
540	0.0013780462490059\\
541	0.0013780462490059\\
542	0.0013780462490059\\
543	0.0013780462490059\\
544	0.0013780462490059\\
545	0.0013780462490059\\
546	0.0013780462490059\\
547	0.0013780462490059\\
548	0.0013780462490059\\
549	0.0013780462490059\\
550	0.0013780462490059\\
551	0.0013780462490059\\
552	0.0013780462490059\\
553	0.0013780462490059\\
554	0.0013780462490059\\
555	0.0013780462490059\\
556	0.0013780462490059\\
557	0.0013780462490059\\
558	0.0013780462490059\\
559	0.0013780462490059\\
560	0.0013780462490059\\
561	0.0013780462490059\\
562	0.0013780462490059\\
563	0.0013780462490059\\
564	0.0013780462490059\\
565	0.00136279015443986\\
566	0.00125827733001416\\
567	0.00115323348010358\\
568	0.00104696223543405\\
569	0.000939078664145309\\
570	0.000831870952691528\\
571	0.000725446290622915\\
572	0.000624556670773367\\
573	0.00053496420760559\\
574	0.000390445210304555\\
575	0.000253767653105659\\
576	0.000124813533253903\\
577	3.73689878182394e-06\\
578	0\\
579	0\\
580	0\\
581	0\\
582	0\\
583	0\\
584	0\\
585	0\\
586	0\\
587	0\\
588	0\\
589	0\\
590	0\\
591	0\\
592	6.50501568911646e-05\\
593	0.000171129827925065\\
594	0.000294096086145507\\
595	0.000433306443467276\\
596	0.000584745759468008\\
597	0.000739344758132349\\
598	0.00345717029705862\\
599	0\\
600	0\\
};
\addplot [color=mycolor14,solid,forget plot]
  table[row sep=crcr]{%
1	0.00634705736923244\\
2	0.00634705707591813\\
3	0.00634705677729562\\
4	0.00634705647326904\\
5	0.00634705616374076\\
6	0.00634705584861141\\
7	0.00634705552777983\\
8	0.00634705520114302\\
9	0.00634705486859616\\
10	0.00634705453003251\\
11	0.00634705418534343\\
12	0.00634705383441831\\
13	0.00634705347714455\\
14	0.00634705311340754\\
15	0.00634705274309059\\
16	0.00634705236607491\\
17	0.0063470519822396\\
18	0.00634705159146155\\
19	0.00634705119361545\\
20	0.00634705078857375\\
21	0.00634705037620657\\
22	0.00634704995638174\\
23	0.00634704952896467\\
24	0.00634704909381839\\
25	0.00634704865080343\\
26	0.00634704819977784\\
27	0.00634704774059711\\
28	0.00634704727311413\\
29	0.00634704679717915\\
30	0.00634704631263973\\
31	0.00634704581934069\\
32	0.00634704531712405\\
33	0.00634704480582902\\
34	0.00634704428529189\\
35	0.00634704375534603\\
36	0.00634704321582182\\
37	0.00634704266654658\\
38	0.00634704210734454\\
39	0.00634704153803677\\
40	0.00634704095844114\\
41	0.00634704036837222\\
42	0.00634703976764128\\
43	0.00634703915605621\\
44	0.00634703853342143\\
45	0.00634703789953786\\
46	0.00634703725420284\\
47	0.0063470365972101\\
48	0.00634703592834965\\
49	0.00634703524740774\\
50	0.00634703455416678\\
51	0.00634703384840529\\
52	0.00634703312989782\\
53	0.00634703239841486\\
54	0.00634703165372281\\
55	0.00634703089558388\\
56	0.006347030123756\\
57	0.00634702933799278\\
58	0.00634702853804341\\
59	0.00634702772365259\\
60	0.00634702689456044\\
61	0.00634702605050245\\
62	0.00634702519120935\\
63	0.00634702431640704\\
64	0.00634702342581654\\
65	0.00634702251915386\\
66	0.00634702159612992\\
67	0.0063470206564505\\
68	0.00634701969981608\\
69	0.00634701872592179\\
70	0.0063470177344573\\
71	0.00634701672510676\\
72	0.00634701569754863\\
73	0.00634701465145565\\
74	0.00634701358649468\\
75	0.00634701250232665\\
76	0.00634701139860642\\
77	0.00634701027498266\\
78	0.0063470091310978\\
79	0.00634700796658785\\
80	0.00634700678108232\\
81	0.00634700557420411\\
82	0.00634700434556936\\
83	0.00634700309478739\\
84	0.00634700182146052\\
85	0.00634700052518396\\
86	0.00634699920554571\\
87	0.0063469978621264\\
88	0.00634699649449918\\
89	0.00634699510222959\\
90	0.00634699368487539\\
91	0.00634699224198647\\
92	0.00634699077310468\\
93	0.0063469892777637\\
94	0.00634698775548889\\
95	0.00634698620579713\\
96	0.00634698462819671\\
97	0.00634698302218713\\
98	0.00634698138725896\\
99	0.00634697972289372\\
100	0.00634697802856366\\
101	0.00634697630373162\\
102	0.00634697454785088\\
103	0.00634697276036498\\
104	0.00634697094070752\\
105	0.00634696908830205\\
106	0.0063469672025618\\
107	0.00634696528288959\\
108	0.0063469633286776\\
109	0.00634696133930715\\
110	0.00634695931414859\\
111	0.00634695725256103\\
112	0.00634695515389217\\
113	0.00634695301747813\\
114	0.00634695084264318\\
115	0.00634694862869959\\
116	0.00634694637494736\\
117	0.00634694408067408\\
118	0.00634694174515463\\
119	0.006346939367651\\
120	0.00634693694741208\\
121	0.00634693448367337\\
122	0.00634693197565678\\
123	0.00634692942257041\\
124	0.00634692682360826\\
125	0.00634692417795001\\
126	0.00634692148476076\\
127	0.00634691874319076\\
128	0.00634691595237519\\
129	0.00634691311143384\\
130	0.00634691021947086\\
131	0.0063469072755745\\
132	0.00634690427881681\\
133	0.00634690122825337\\
134	0.00634689812292295\\
135	0.00634689496184731\\
136	0.00634689174403079\\
137	0.00634688846846009\\
138	0.00634688513410389\\
139	0.0063468817399126\\
140	0.00634687828481799\\
141	0.00634687476773284\\
142	0.0063468711875507\\
143	0.00634686754314543\\
144	0.00634686383337096\\
145	0.00634686005706085\\
146	0.00634685621302799\\
147	0.00634685230006422\\
148	0.00634684831693993\\
149	0.00634684426240374\\
150	0.00634684013518206\\
151	0.00634683593397872\\
152	0.00634683165747458\\
153	0.0063468273043271\\
154	0.00634682287316997\\
155	0.00634681836261263\\
156	0.0063468137712399\\
157	0.00634680909761152\\
158	0.00634680434026169\\
159	0.00634679949769865\\
160	0.00634679456840423\\
161	0.00634678955083332\\
162	0.00634678444341348\\
163	0.00634677924454438\\
164	0.00634677395259738\\
165	0.00634676856591495\\
166	0.00634676308281026\\
167	0.00634675750156654\\
168	0.00634675182043668\\
169	0.00634674603764258\\
170	0.00634674015137471\\
171	0.00634673415979143\\
172	0.00634672806101856\\
173	0.00634672185314869\\
174	0.00634671553424066\\
175	0.00634670910231894\\
176	0.00634670255537304\\
177	0.00634669589135686\\
178	0.00634668910818808\\
179	0.00634668220374752\\
180	0.0063466751758785\\
181	0.00634666802238616\\
182	0.00634666074103676\\
183	0.00634665332955706\\
184	0.00634664578563356\\
185	0.00634663810691181\\
186	0.00634663029099569\\
187	0.00634662233544665\\
188	0.006346614237783\\
189	0.00634660599547909\\
190	0.00634659760596459\\
191	0.00634658906662365\\
192	0.00634658037479413\\
193	0.00634657152776673\\
194	0.00634656252278421\\
195	0.00634655335704053\\
196	0.00634654402767992\\
197	0.00634653453179609\\
198	0.00634652486643127\\
199	0.0063465150285753\\
200	0.00634650501516476\\
201	0.00634649482308192\\
202	0.00634648444915386\\
203	0.00634647389015147\\
204	0.00634646314278842\\
205	0.00634645220372016\\
206	0.00634644106954291\\
207	0.00634642973679255\\
208	0.00634641820194359\\
209	0.00634640646140806\\
210	0.0063463945115344\\
211	0.00634638234860632\\
212	0.00634636996884166\\
213	0.00634635736839118\\
214	0.00634634454333739\\
215	0.00634633148969334\\
216	0.00634631820340137\\
217	0.00634630468033181\\
218	0.00634629091628174\\
219	0.00634627690697368\\
220	0.00634626264805419\\
221	0.0063462481350926\\
222	0.00634623336357957\\
223	0.00634621832892568\\
224	0.00634620302646004\\
225	0.00634618745142875\\
226	0.00634617159899349\\
227	0.00634615546422996\\
228	0.00634613904212636\\
229	0.00634612232758177\\
230	0.00634610531540462\\
231	0.006346088000311\\
232	0.00634607037692303\\
233	0.00634605243976716\\
234	0.00634603418327243\\
235	0.00634601560176877\\
236	0.00634599668948512\\
237	0.00634597744054773\\
238	0.00634595784897819\\
239	0.00634593790869163\\
240	0.00634591761349475\\
241	0.00634589695708388\\
242	0.00634587593304298\\
243	0.00634585453484162\\
244	0.00634583275583291\\
245	0.00634581058925138\\
246	0.00634578802821086\\
247	0.00634576506570227\\
248	0.00634574169459141\\
249	0.00634571790761672\\
250	0.00634569369738691\\
251	0.00634566905637868\\
252	0.00634564397693429\\
253	0.00634561845125914\\
254	0.00634559247141927\\
255	0.00634556602933886\\
256	0.00634553911679764\\
257	0.00634551172542829\\
258	0.00634548384671375\\
259	0.00634545547198453\\
260	0.00634542659241593\\
261	0.00634539719902521\\
262	0.00634536728266877\\
263	0.00634533683403918\\
264	0.00634530584366226\\
265	0.00634527430189401\\
266	0.00634524219891754\\
267	0.00634520952473997\\
268	0.00634517626918919\\
269	0.00634514242191067\\
270	0.00634510797236414\\
271	0.00634507290982028\\
272	0.00634503722335723\\
273	0.00634500090185709\\
274	0.00634496393400233\\
275	0.00634492630827227\\
276	0.00634488801293938\\
277	0.0063448490360656\\
278	0.00634480936549852\\
279	0.0063447689888675\\
280	0.00634472789357982\\
281	0.00634468606681663\\
282	0.00634464349552894\\
283	0.00634460016643353\\
284	0.00634455606600872\\
285	0.0063445111804902\\
286	0.00634446549586669\\
287	0.00634441899787558\\
288	0.00634437167199854\\
289	0.00634432350345699\\
290	0.00634427447720758\\
291	0.00634422457793762\\
292	0.00634417379006036\\
293	0.0063441220977103\\
294	0.00634406948473844\\
295	0.00634401593470744\\
296	0.00634396143088677\\
297	0.00634390595624778\\
298	0.00634384949345877\\
299	0.00634379202488\\
300	0.00634373353255865\\
301	0.00634367399822384\\
302	0.00634361340328147\\
303	0.00634355172880927\\
304	0.00634348895555162\\
305	0.00634342506391455\\
306	0.0063433600339606\\
307	0.00634329384540378\\
308	0.00634322647760438\\
309	0.00634315790956398\\
310	0.00634308811992051\\
311	0.00634301708694385\\
312	0.00634294478853137\\
313	0.00634287120220282\\
314	0.0063427963050959\\
315	0.00634272007396187\\
316	0.00634264248516149\\
317	0.00634256351466105\\
318	0.00634248313802878\\
319	0.0063424013304314\\
320	0.00634231806663113\\
321	0.00634223332098303\\
322	0.00634214706743279\\
323	0.00634205927951495\\
324	0.00634196993035181\\
325	0.00634187899265279\\
326	0.00634178643871461\\
327	0.00634169224042198\\
328	0.00634159636924884\\
329	0.0063414987962596\\
330	0.00634139949210948\\
331	0.00634129842704191\\
332	0.00634119557088033\\
333	0.00634109089301242\\
334	0.00634098436237369\\
335	0.00634087594746614\\
336	0.00634076561650549\\
337	0.00634065333780671\\
338	0.00634053908012841\\
339	0.00634042281148967\\
340	0.0063403044970337\\
341	0.00634018409871652\\
342	0.00634006157113215\\
343	0.00633993685233052\\
344	0.00633980984603337\\
345	0.00633968039571144\\
346	0.00633954827735552\\
347	0.00633941331734338\\
348	0.00633927576477506\\
349	0.00633913590350577\\
350	0.00633899369546465\\
351	0.00633884910201422\\
352	0.00633870208394732\\
353	0.00633855260148489\\
354	0.00633840061427464\\
355	0.00633824608138921\\
356	0.0063380889613211\\
357	0.00633792921197611\\
358	0.00633776679067756\\
359	0.00633760165416524\\
360	0.00633743375859476\\
361	0.00633726305953701\\
362	0.00633708951197813\\
363	0.0063369130703195\\
364	0.00633673368837812\\
365	0.00633655131938696\\
366	0.00633636591599548\\
367	0.0063361774302701\\
368	0.00633598581369455\\
369	0.00633579101716998\\
370	0.00633559299101447\\
371	0.00633539168496156\\
372	0.00633518704815668\\
373	0.00633497902915057\\
374	0.00633476757589027\\
375	0.00633455263571372\\
376	0.00633433415536172\\
377	0.0063341120810081\\
378	0.00633388635826129\\
379	0.00633365693222053\\
380	0.00633342374756986\\
381	0.00633318674867619\\
382	0.00633294587952931\\
383	0.00633270108308434\\
384	0.00633245229929575\\
385	0.00633219946197002\\
386	0.00633194249927014\\
387	0.00633168134762682\\
388	0.00633141594441352\\
389	0.00633114622526075\\
390	0.00633087212386279\\
391	0.00633059357183474\\
392	0.00633031049874965\\
393	0.00633002283268106\\
394	0.00632973050203992\\
395	0.00632943344055455\\
396	0.00632913159953022\\
397	0.00632882497602954\\
398	0.00632851367297699\\
399	0.00632819801383586\\
400	0.00632787871871675\\
401	0.00632755703509414\\
402	0.00632723434369704\\
403	0.00632691022043075\\
404	0.00632658003640803\\
405	0.00632624367882569\\
406	0.00632590102995405\\
407	0.00632555196672422\\
408	0.00632519636035737\\
409	0.00632483407667918\\
410	0.00632446497754873\\
411	0.00632408892036726\\
412	0.00632370574530547\\
413	0.0063233152870529\\
414	0.00632291737454029\\
415	0.00632251183064566\\
416	0.00632209847188187\\
417	0.00632167710805953\\
418	0.00632124754191151\\
419	0.00632080956865844\\
420	0.00632036297551085\\
421	0.00631990754118923\\
422	0.00631944303564629\\
423	0.00631896921980199\\
424	0.00631848584474056\\
425	0.00631799265114201\\
426	0.00631748936867251\\
427	0.00631697571532665\\
428	0.00631645139670927\\
429	0.00631591610523212\\
430	0.00631536951917585\\
431	0.00631481130152096\\
432	0.0063142410983804\\
433	0.00631365853681534\\
434	0.00631306322201186\\
435	0.00631245473496562\\
436	0.00631183263545229\\
437	0.00631119648081164\\
438	0.00631054585902489\\
439	0.00630988033286531\\
440	0.00630919937951012\\
441	0.00630850244862298\\
442	0.00630778896062646\\
443	0.00630705830485506\\
444	0.00630630983758596\\
445	0.00630554287987795\\
446	0.00630475671460947\\
447	0.00630395057978746\\
448	0.00630312364952706\\
449	0.00630227499286354\\
450	0.00630140346092567\\
451	0.00630050737731262\\
452	0.00629958369754549\\
453	0.00629862575552163\\
454	0.00629761732521094\\
455	0.0062965172609967\\
456	0.00629522043131399\\
457	0.00629346826565445\\
458	0.00629058504385814\\
459	0.0062873984227585\\
460	0.00628414460028995\\
461	0.00628080773056847\\
462	0.00627737393016521\\
463	0.00627388356313013\\
464	0.00627033595250894\\
465	0.00626672881184384\\
466	0.00626305953089114\\
467	0.00625932510476634\\
468	0.00625552214579273\\
469	0.0062516473709978\\
470	0.00624769936191294\\
471	0.00624367602190079\\
472	0.00623957699600573\\
473	0.00623540724679653\\
474	0.00623118227627135\\
475	0.00622693215559939\\
476	0.00622268415786011\\
477	0.00621837197795093\\
478	0.00621391858504133\\
479	0.00620931339029792\\
480	0.00620454353957697\\
481	0.00619959164428468\\
482	0.0061944298689289\\
483	0.00618900452079593\\
484	0.00618319741935313\\
485	0.00617674152536256\\
486	0.00616948027811247\\
487	0.00616210182564511\\
488	0.00615460347021238\\
489	0.00614698244999517\\
490	0.00613923586265543\\
491	0.00613136059652419\\
492	0.0061233530577195\\
493	0.00611520758579711\\
494	0.00610691552533023\\
495	0.00609848237957893\\
496	0.00608990449054164\\
497	0.00608117670253305\\
498	0.00607229238821892\\
499	0.00606324736646706\\
500	0.00605403705724364\\
501	0.00604465615743606\\
502	0.00603509786883715\\
503	0.00602535233006483\\
504	0.00601540563529222\\
505	0.00600525878536194\\
506	0.00599488587033331\\
507	0.00598422125494825\\
508	0.00597312631007669\\
509	0.00596172141658392\\
510	0.00595018659363723\\
511	0.0059384445843753\\
512	0.00592631108743226\\
513	0.00591346613950271\\
514	0.00590030505807963\\
515	0.00588707956134307\\
516	0.00587379765610018\\
517	0.00586048147386627\\
518	0.00584718801494137\\
519	0.00583405829195754\\
520	0.00582141659957693\\
521	0.00580989012708187\\
522	0.00580008189998949\\
523	0.00578994549122259\\
524	0.00577913384386583\\
525	0.00576657393904838\\
526	0.00574897062406128\\
527	0.00570949806036672\\
528	0.00566622002656595\\
529	0.00562168348095022\\
530	0.00557566290236694\\
531	0.00552774050188853\\
532	0.00547700589151219\\
533	0.00541874111612479\\
534	0.00535561486132491\\
535	0.00529235810299489\\
536	0.00522897636497744\\
537	0.00516547212723779\\
538	0.00510184062425884\\
539	0.00503805005930613\\
540	0.0049739785608551\\
541	0.00490919741497271\\
542	0.00484167426677888\\
543	0.0047750592880797\\
544	0.00470938112578479\\
545	0.00464466913499734\\
546	0.00458095349021259\\
547	0.00451826420357372\\
548	0.00445662654337949\\
549	0.00439604589693207\\
550	0.00433645820630629\\
551	0.00427756785872679\\
552	0.00421831745834273\\
553	0.00415492967898489\\
554	0.0040927650695408\\
555	0.00403290499091029\\
556	0.00397537262231718\\
557	0.00392025786722828\\
558	0.00386761667012469\\
559	0.00381749584867485\\
560	0.0037699710534416\\
561	0.00372520839465025\\
562	0.00368359691573897\\
563	0.00364607931640829\\
564	0.00361495875009759\\
565	0.00358407587802502\\
566	0.00350180811894708\\
567	0.00341627094766928\\
568	0.00332212922189586\\
569	0.00322378609530446\\
570	0.00312258660790876\\
571	0.00301842823208778\\
572	0.00291083880439416\\
573	0.00279199464559352\\
574	0.00262090950248387\\
575	0.00244700241101585\\
576	0.00227017807912024\\
577	0.00209042395100323\\
578	0.0019079951662922\\
579	0.00172275411076606\\
580	0.00153451315287464\\
581	0.00134245890765045\\
582	0.00114797364967888\\
583	0.000951303160335593\\
584	0.000752186592492726\\
585	0.000549776472909163\\
586	0.000341218417480763\\
587	0.000117616391847642\\
588	0\\
589	0\\
590	0\\
591	0\\
592	0\\
593	0\\
594	0\\
595	0\\
596	0\\
597	0\\
598	0\\
599	0\\
600	0\\
};
\addplot [color=mycolor15,solid,forget plot]
  table[row sep=crcr]{%
1	0.0050928946278291\\
2	0.00509289365214289\\
3	0.00509289265879869\\
4	0.00509289164747749\\
5	0.00509289061785456\\
6	0.00509288956959928\\
7	0.00509288850237511\\
8	0.00509288741583942\\
9	0.00509288630964345\\
10	0.00509288518343212\\
11	0.00509288403684399\\
12	0.00509288286951108\\
13	0.00509288168105882\\
14	0.00509288047110587\\
15	0.00509287923926405\\
16	0.00509287798513818\\
17	0.00509287670832597\\
18	0.00509287540841791\\
19	0.0050928740849971\\
20	0.00509287273763914\\
21	0.00509287136591202\\
22	0.00509286996937594\\
23	0.00509286854758319\\
24	0.00509286710007804\\
25	0.00509286562639651\\
26	0.00509286412606633\\
27	0.00509286259860671\\
28	0.00509286104352824\\
29	0.0050928594603327\\
30	0.00509285784851292\\
31	0.0050928562075526\\
32	0.0050928545369262\\
33	0.0050928528360987\\
34	0.00509285110452549\\
35	0.00509284934165218\\
36	0.00509284754691442\\
37	0.0050928457197377\\
38	0.00509284385953725\\
39	0.00509284196571773\\
40	0.00509284003767317\\
41	0.0050928380747867\\
42	0.00509283607643036\\
43	0.00509283404196496\\
44	0.0050928319707398\\
45	0.00509282986209254\\
46	0.00509282771534892\\
47	0.00509282552982261\\
48	0.00509282330481497\\
49	0.00509282103961481\\
50	0.0050928187334982\\
51	0.00509281638572822\\
52	0.00509281399555476\\
53	0.00509281156221422\\
54	0.00509280908492936\\
55	0.00509280656290897\\
56	0.00509280399534768\\
57	0.00509280138142569\\
58	0.0050927987203085\\
59	0.00509279601114666\\
60	0.00509279325307551\\
61	0.0050927904452149\\
62	0.00509278758666891\\
63	0.0050927846765256\\
64	0.00509278171385666\\
65	0.0050927786977172\\
66	0.00509277562714539\\
67	0.00509277250116219\\
68	0.00509276931877105\\
69	0.00509276607895757\\
70	0.0050927627806892\\
71	0.00509275942291493\\
72	0.00509275600456492\\
73	0.00509275252455024\\
74	0.00509274898176244\\
75	0.00509274537507326\\
76	0.00509274170333429\\
77	0.00509273796537655\\
78	0.00509273416001019\\
79	0.00509273028602408\\
80	0.00509272634218544\\
81	0.00509272232723947\\
82	0.00509271823990895\\
83	0.00509271407889383\\
84	0.00509270984287085\\
85	0.00509270553049308\\
86	0.00509270114038959\\
87	0.00509269667116491\\
88	0.0050926921213987\\
89	0.00509268748964523\\
90	0.00509268277443298\\
91	0.00509267797426414\\
92	0.00509267308761421\\
93	0.00509266811293145\\
94	0.00509266304863643\\
95	0.00509265789312156\\
96	0.00509265264475058\\
97	0.00509264730185801\\
98	0.0050926418627487\\
99	0.00509263632569725\\
100	0.00509263068894751\\
101	0.00509262495071201\\
102	0.00509261910917141\\
103	0.00509261316247397\\
104	0.00509260710873491\\
105	0.0050926009460359\\
106	0.00509259467242439\\
107	0.00509258828591309\\
108	0.00509258178447927\\
109	0.00509257516606419\\
110	0.00509256842857244\\
111	0.00509256156987129\\
112	0.00509255458779004\\
113	0.00509254748011936\\
114	0.00509254024461056\\
115	0.00509253287897496\\
116	0.00509252538088311\\
117	0.00509251774796413\\
118	0.00509250997780495\\
119	0.00509250206794957\\
120	0.0050924940158983\\
121	0.00509248581910697\\
122	0.00509247747498618\\
123	0.00509246898090049\\
124	0.00509246033416757\\
125	0.00509245153205742\\
126	0.0050924425717915\\
127	0.00509243345054189\\
128	0.00509242416543039\\
129	0.00509241471352766\\
130	0.0050924050918523\\
131	0.00509239529736995\\
132	0.00509238532699234\\
133	0.00509237517757631\\
134	0.00509236484592291\\
135	0.00509235432877635\\
136	0.00509234362282302\\
137	0.0050923327246905\\
138	0.00509232163094645\\
139	0.00509231033809762\\
140	0.00509229884258875\\
141	0.00509228714080147\\
142	0.00509227522905317\\
143	0.00509226310359592\\
144	0.00509225076061526\\
145	0.00509223819622906\\
146	0.00509222540648632\\
147	0.00509221238736592\\
148	0.00509219913477545\\
149	0.00509218564454988\\
150	0.00509217191245034\\
151	0.00509215793416276\\
152	0.00509214370529661\\
153	0.00509212922138346\\
154	0.00509211447787569\\
155	0.00509209947014505\\
156	0.00509208419348126\\
157	0.00509206864309053\\
158	0.0050920528140941\\
159	0.00509203670152675\\
160	0.00509202030033528\\
161	0.00509200360537694\\
162	0.00509198661141783\\
163	0.00509196931313133\\
164	0.00509195170509646\\
165	0.00509193378179618\\
166	0.00509191553761573\\
167	0.00509189696684087\\
168	0.00509187806365615\\
169	0.00509185882214313\\
170	0.00509183923627852\\
171	0.00509181929993237\\
172	0.00509179900686614\\
173	0.00509177835073085\\
174	0.00509175732506505\\
175	0.0050917359232929\\
176	0.00509171413872208\\
177	0.0050916919645418\\
178	0.00509166939382065\\
179	0.00509164641950451\\
180	0.00509162303441435\\
181	0.00509159923124401\\
182	0.005091575002558\\
183	0.00509155034078917\\
184	0.00509152523823638\\
185	0.00509149968706214\\
186	0.00509147367929021\\
187	0.00509144720680311\\
188	0.00509142026133964\\
189	0.00509139283449233\\
190	0.00509136491770483\\
191	0.00509133650226931\\
192	0.00509130757932373\\
193	0.00509127813984914\\
194	0.00509124817466685\\
195	0.00509121767443567\\
196	0.00509118662964895\\
197	0.00509115503063168\\
198	0.00509112286753751\\
199	0.00509109013034565\\
200	0.00509105680885785\\
201	0.00509102289269518\\
202	0.00509098837129484\\
203	0.00509095323390692\\
204	0.005090917469591\\
205	0.00509088106721283\\
206	0.00509084401544081\\
207	0.00509080630274256\\
208	0.00509076791738125\\
209	0.00509072884741202\\
210	0.00509068908067825\\
211	0.00509064860480777\\
212	0.00509060740720903\\
213	0.00509056547506718\\
214	0.00509052279534006\\
215	0.00509047935475418\\
216	0.00509043513980055\\
217	0.00509039013673047\\
218	0.00509034433155126\\
219	0.00509029771002189\\
220	0.0050902502576485\\
221	0.00509020195967992\\
222	0.00509015280110303\\
223	0.00509010276663806\\
224	0.00509005184073383\\
225	0.00509000000756285\\
226	0.00508994725101637\\
227	0.00508989355469935\\
228	0.00508983890192529\\
229	0.00508978327571099\\
230	0.00508972665877122\\
231	0.0050896690335133\\
232	0.00508961038203154\\
233	0.00508955068610163\\
234	0.00508948992717488\\
235	0.00508942808637236\\
236	0.00508936514447898\\
237	0.00508930108193742\\
238	0.00508923587884193\\
239	0.00508916951493205\\
240	0.00508910196958621\\
241	0.00508903322181522\\
242	0.00508896325025557\\
243	0.00508889203316274\\
244	0.00508881954840423\\
245	0.00508874577345261\\
246	0.00508867068537832\\
247	0.0050885942608424\\
248	0.0050885164760891\\
249	0.0050884373069383\\
250	0.00508835672877782\\
251	0.00508827471655561\\
252	0.00508819124477175\\
253	0.00508810628747034\\
254	0.00508801981823121\\
255	0.00508793181016152\\
256	0.00508784223588716\\
257	0.00508775106754401\\
258	0.00508765827676908\\
259	0.0050875638346914\\
260	0.00508746771192282\\
261	0.0050873698785486\\
262	0.00508727030411788\\
263	0.0050871689576339\\
264	0.00508706580754409\\
265	0.00508696082172997\\
266	0.00508685396749688\\
267	0.00508674521156349\\
268	0.00508663452005114\\
269	0.00508652185847301\\
270	0.00508640719172311\\
271	0.00508629048406506\\
272	0.00508617169912081\\
273	0.0050860507998589\\
274	0.00508592774858237\\
275	0.00508580250691664\\
276	0.00508567503579743\\
277	0.00508554529545836\\
278	0.00508541324541817\\
279	0.00508527884446789\\
280	0.00508514205065768\\
281	0.0050850028212835\\
282	0.00508486111287353\\
283	0.00508471688117443\\
284	0.00508457008113731\\
285	0.00508442066690353\\
286	0.00508426859179028\\
287	0.00508411380827591\\
288	0.00508395626798505\\
289	0.00508379592167355\\
290	0.00508363271921317\\
291	0.00508346660957611\\
292	0.00508329754081924\\
293	0.00508312546006827\\
294	0.00508295031350161\\
295	0.00508277204633408\\
296	0.00508259060280047\\
297	0.0050824059261389\\
298	0.00508221795857403\\
299	0.00508202664130014\\
300	0.00508183191446406\\
301	0.00508163371714798\\
302	0.00508143198735219\\
303	0.00508122666197778\\
304	0.00508101767680916\\
305	0.00508080496649673\\
306	0.00508058846453944\\
307	0.00508036810326731\\
308	0.005080143813824\\
309	0.00507991552614912\\
310	0.0050796831689607\\
311	0.00507944666973814\\
312	0.00507920595470679\\
313	0.00507896094882321\\
314	0.00507871157575707\\
315	0.00507845775787563\\
316	0.00507819941622867\\
317	0.00507793647053405\\
318	0.00507766883916395\\
319	0.00507739643913194\\
320	0.00507711918608097\\
321	0.00507683699427252\\
322	0.00507654977657687\\
323	0.00507625744446493\\
324	0.00507595990800155\\
325	0.00507565707584084\\
326	0.00507534885522337\\
327	0.00507503515197581\\
328	0.00507471587051285\\
329	0.00507439091384142\\
330	0.00507406018356655\\
331	0.00507372357989712\\
332	0.00507338100164791\\
333	0.00507303234623229\\
334	0.00507267750964002\\
335	0.00507231638640881\\
336	0.00507194886965547\\
337	0.00507157485138313\\
338	0.00507119422348189\\
339	0.00507080687923724\\
340	0.00507041271038931\\
341	0.00507001160023853\\
342	0.00506960342422547\\
343	0.00506918803930844\\
344	0.00506876525882263\\
345	0.0050683347975141\\
346	0.00506789616893734\\
347	0.00506744856867696\\
348	0.00506699109119338\\
349	0.00506652456440268\\
350	0.0050660501975954\\
351	0.00506556786113188\\
352	0.0050650774234284\\
353	0.00506457875094549\\
354	0.00506407170817921\\
355	0.00506355615765624\\
356	0.00506303195992941\\
357	0.00506249897356119\\
358	0.00506195705509287\\
359	0.00506140605906212\\
360	0.00506084583799855\\
361	0.0050602762424204\\
362	0.00505969712083228\\
363	0.00505910831972386\\
364	0.00505850968356943\\
365	0.00505790105482817\\
366	0.00505728227394488\\
367	0.00505665317935105\\
368	0.00505601360746598\\
369	0.00505536339269754\\
370	0.00505470236744231\\
371	0.00505403036208429\\
372	0.00505334720499054\\
373	0.00505265272250077\\
374	0.00505194673890596\\
375	0.00505122907641344\\
376	0.00505049955511283\\
377	0.00504975799299523\\
378	0.00504900420605909\\
379	0.00504823800823944\\
380	0.00504745921149303\\
381	0.00504666762595958\\
382	0.00504586306014246\\
383	0.00504504532078459\\
384	0.00504421421144329\\
385	0.00504336952770216\\
386	0.00504251104749455\\
387	0.00504163852689711\\
388	0.00504075174925823\\
389	0.0050398505012029\\
390	0.00503893456352778\\
391	0.00503800371050569\\
392	0.00503705770926875\\
393	0.00503609631957294\\
394	0.00503511929472342\\
395	0.00503412638559884\\
396	0.00503311735245732\\
397	0.00503209199545522\\
398	0.00503105022820599\\
399	0.00502999224452367\\
400	0.00502891886763283\\
401	0.00502783218799178\\
402	0.00502673639327816\\
403	0.00502563753529514\\
404	0.00502453721029399\\
405	0.00502341626063081\\
406	0.00502227430254632\\
407	0.00502111093574222\\
408	0.00501992574202896\\
409	0.00501871828365271\\
410	0.00501748810334104\\
411	0.00501623472918392\\
412	0.005014957679507\\
413	0.00501365640947008\\
414	0.00501233035555447\\
415	0.00501097893461764\\
416	0.00500960154289323\\
417	0.00500819755493185\\
418	0.00500676632246699\\
419	0.0050053071731649\\
420	0.00500381940917676\\
421	0.0050023023054178\\
422	0.00500075510777094\\
423	0.00499917703210561\\
424	0.00499756726373227\\
425	0.00499592495449141\\
426	0.0049942492208264\\
427	0.0049925391417214\\
428	0.00499079375648711\\
429	0.00498901206236689\\
430	0.00498719301191277\\
431	0.00498533551003465\\
432	0.00498343841053814\\
433	0.00498150051182483\\
434	0.00497952055128766\\
435	0.00497749719812845\\
436	0.00497542904612344\\
437	0.00497331461496192\\
438	0.00497115238815756\\
439	0.00496894092847353\\
440	0.00496667873037992\\
441	0.00496436401488461\\
442	0.00496199490964288\\
443	0.00495956944314037\\
444	0.00495708553846413\\
445	0.00495454100667428\\
446	0.00495193353985444\\
447	0.0049492607032568\\
448	0.00494651992026552\\
449	0.00494370842112155\\
450	0.00494082314810791\\
451	0.00493786049973932\\
452	0.00493481563215763\\
453	0.00493168056649731\\
454	0.00492843908689531\\
455	0.0049250531026715\\
456	0.00492142645084939\\
457	0.00491730494594139\\
458	0.00491208803876988\\
459	0.00490239900538334\\
460	0.00489149692452099\\
461	0.00488036978683903\\
462	0.0048689610936534\\
463	0.00485718523999083\\
464	0.00484520977544474\\
465	0.00483303324960802\\
466	0.0048206477514237\\
467	0.00480804428243009\\
468	0.0047952124548571\\
469	0.00478214021535745\\
470	0.00476881468701787\\
471	0.00475523106750443\\
472	0.00474138065763046\\
473	0.00472725851969782\\
474	0.00471287490790399\\
475	0.00469827434654054\\
476	0.00468356348793886\\
477	0.00466889699916233\\
478	0.00465405026924246\\
479	0.00463871239017449\\
480	0.00462284724770989\\
481	0.00460641219022524\\
482	0.00458935239600222\\
483	0.00457158616356049\\
484	0.0045529655160597\\
485	0.0045331696140612\\
486	0.00451120800814773\\
487	0.00448604223992208\\
488	0.00446045227420199\\
489	0.00443442838203219\\
490	0.00440796066997701\\
491	0.00438103880866148\\
492	0.00435365189049488\\
493	0.00432578813853256\\
494	0.00429743538016299\\
495	0.00426858429943873\\
496	0.00423922283061158\\
497	0.00420933758393212\\
498	0.00417891159500239\\
499	0.00414792434834379\\
500	0.00411636181189269\\
501	0.00408420885099331\\
502	0.0040514483152974\\
503	0.00401805870402856\\
504	0.00398400845011968\\
505	0.00394924445222225\\
506	0.00391378241291492\\
507	0.00387754851258355\\
508	0.00384034278626771\\
509	0.00380159692797307\\
510	0.00376171459281732\\
511	0.00372140524121531\\
512	0.00368042783019527\\
513	0.00363816404977489\\
514	0.00359328710604181\\
515	0.00354725849525978\\
516	0.00350104439854883\\
517	0.0034546796590398\\
518	0.0034082466826054\\
519	0.00336194673080687\\
520	0.00331627115305347\\
521	0.00327235601772224\\
522	0.00323245081899571\\
523	0.00319880639447882\\
524	0.00316409774624987\\
525	0.00312711704512529\\
526	0.00308409729865228\\
527	0.00302322049211327\\
528	0.00295799210260085\\
529	0.00289023238240467\\
530	0.00281957843518099\\
531	0.0027452515402818\\
532	0.00266566068754643\\
533	0.0025771460147355\\
534	0.00248201373084442\\
535	0.00238431533313198\\
536	0.00228393946136431\\
537	0.00218076254050936\\
538	0.00207463993340179\\
539	0.00196537226973244\\
540	0.00185259646353105\\
541	0.00173543479749105\\
542	0.00161171504284901\\
543	0.00148532245630597\\
544	0.0013561727032017\\
545	0.00122418802429166\\
546	0.00108928435702427\\
547	0.00095136984335789\\
548	0.000810335032611802\\
549	0.000666015326258075\\
550	0.00051806260568251\\
551	0.000365504845979632\\
552	0.000205227792137007\\
553	2.68864273520115e-05\\
554	0\\
555	0\\
556	0\\
557	0\\
558	0\\
559	0\\
560	0\\
561	0\\
562	0\\
563	0\\
564	0\\
565	0\\
566	0\\
567	0\\
568	0\\
569	0\\
570	0\\
571	0\\
572	0\\
573	0\\
574	0\\
575	0\\
576	0\\
577	0\\
578	0\\
579	0\\
580	0\\
581	0\\
582	0\\
583	0\\
584	0\\
585	0\\
586	0\\
587	0\\
588	0\\
589	0\\
590	0\\
591	0\\
592	0\\
593	0\\
594	0\\
595	0\\
596	0\\
597	0\\
598	0\\
599	0\\
600	0\\
};
\addplot [color=mycolor16,solid,forget plot]
  table[row sep=crcr]{%
1	0.00344286792547887\\
2	0.0034428627298262\\
3	0.00344285744013911\\
4	0.00344285205471872\\
5	0.00344284657183551\\
6	0.00344284098972885\\
7	0.00344283530660638\\
8	0.00344282952064345\\
9	0.00344282362998257\\
10	0.00344281763273278\\
11	0.00344281152696909\\
12	0.00344280531073186\\
13	0.00344279898202617\\
14	0.00344279253882116\\
15	0.00344278597904946\\
16	0.00344277930060647\\
17	0.00344277250134973\\
18	0.00344276557909821\\
19	0.00344275853163165\\
20	0.00344275135668987\\
21	0.00344274405197199\\
22	0.00344273661513577\\
23	0.00344272904379685\\
24	0.00344272133552797\\
25	0.00344271348785824\\
26	0.00344270549827233\\
27	0.00344269736420971\\
28	0.00344268908306379\\
29	0.00344268065218116\\
30	0.00344267206886071\\
31	0.00344266333035277\\
32	0.00344265443385828\\
33	0.00344264537652789\\
34	0.00344263615546104\\
35	0.00344262676770509\\
36	0.00344261721025434\\
37	0.00344260748004913\\
38	0.00344259757397484\\
39	0.00344258748886092\\
40	0.0034425772214799\\
41	0.00344256676854639\\
42	0.00344255612671598\\
43	0.00344254529258426\\
44	0.00344253426268573\\
45	0.00344252303349268\\
46	0.0034425116014141\\
47	0.00344249996279459\\
48	0.00344248811391314\\
49	0.00344247605098199\\
50	0.00344246377014548\\
51	0.00344245126747877\\
52	0.00344243853898668\\
53	0.00344242558060235\\
54	0.00344241238818607\\
55	0.00344239895752387\\
56	0.00344238528432629\\
57	0.00344237136422697\\
58	0.00344235719278133\\
59	0.00344234276546515\\
60	0.00344232807767315\\
61	0.00344231312471755\\
62	0.00344229790182662\\
63	0.00344228240414316\\
64	0.00344226662672301\\
65	0.00344225056453347\\
66	0.00344223421245173\\
67	0.00344221756526331\\
68	0.00344220061766035\\
69	0.00344218336424005\\
70	0.00344216579950289\\
71	0.00344214791785095\\
72	0.00344212971358619\\
73	0.00344211118090861\\
74	0.00344209231391449\\
75	0.00344207310659451\\
76	0.00344205355283191\\
77	0.00344203364640054\\
78	0.00344201338096296\\
79	0.00344199275006843\\
80	0.00344197174715092\\
81	0.00344195036552705\\
82	0.00344192859839402\\
83	0.00344190643882747\\
84	0.00344188387977934\\
85	0.00344186091407569\\
86	0.00344183753441441\\
87	0.00344181373336301\\
88	0.00344178950335629\\
89	0.00344176483669394\\
90	0.00344173972553823\\
91	0.00344171416191148\\
92	0.00344168813769367\\
93	0.00344166164461982\\
94	0.00344163467427752\\
95	0.00344160721810423\\
96	0.0034415792673847\\
97	0.0034415508132482\\
98	0.00344152184666578\\
99	0.00344149235844749\\
100	0.00344146233923953\\
101	0.00344143177952131\\
102	0.00344140066960253\\
103	0.00344136899962017\\
104	0.00344133675953543\\
105	0.00344130393913062\\
106	0.00344127052800598\\
107	0.00344123651557647\\
108	0.00344120189106851\\
109	0.00344116664351662\\
110	0.00344113076176001\\
111	0.00344109423443917\\
112	0.00344105704999233\\
113	0.00344101919665188\\
114	0.00344098066244072\\
115	0.00344094143516861\\
116	0.00344090150242832\\
117	0.00344086085159187\\
118	0.00344081946980659\\
119	0.00344077734399114\\
120	0.0034407344608315\\
121	0.00344069080677685\\
122	0.00344064636803533\\
123	0.00344060113056991\\
124	0.00344055508009391\\
125	0.00344050820206669\\
126	0.00344046048168914\\
127	0.0034404119038991\\
128	0.00344036245336672\\
129	0.00344031211448975\\
130	0.00344026087138871\\
131	0.003440208707902\\
132	0.00344015560758095\\
133	0.0034401015536847\\
134	0.00344004652917507\\
135	0.00343999051671133\\
136	0.00343993349864483\\
137	0.00343987545701357\\
138	0.0034398163735367\\
139	0.00343975622960885\\
140	0.00343969500629445\\
141	0.00343963268432187\\
142	0.00343956924407749\\
143	0.0034395046655997\\
144	0.00343943892857273\\
145	0.00343937201232041\\
146	0.00343930389579982\\
147	0.00343923455759482\\
148	0.00343916397590946\\
149	0.00343909212856127\\
150	0.0034390189929745\\
151	0.00343894454617309\\
152	0.00343886876477369\\
153	0.00343879162497845\\
154	0.00343871310256769\\
155	0.00343863317289249\\
156	0.00343855181086712\\
157	0.0034384689909613\\
158	0.00343838468719239\\
159	0.00343829887311741\\
160	0.00343821152182493\\
161	0.00343812260592678\\
162	0.00343803209754964\\
163	0.00343793996832654\\
164	0.00343784618938807\\
165	0.00343775073135359\\
166	0.00343765356432215\\
167	0.00343755465786337\\
168	0.00343745398100804\\
169	0.00343735150223867\\
170	0.00343724718947977\\
171	0.00343714101008803\\
172	0.0034370329308423\\
173	0.00343692291793336\\
174	0.00343681093695358\\
175	0.00343669695288634\\
176	0.00343658093009526\\
177	0.00343646283231331\\
178	0.00343634262263159\\
179	0.0034362202634881\\
180	0.0034360957166561\\
181	0.00343596894323245\\
182	0.00343583990362558\\
183	0.00343570855754343\\
184	0.00343557486398096\\
185	0.00343543878120763\\
186	0.00343530026675451\\
187	0.00343515927740128\\
188	0.00343501576916291\\
189	0.00343486969727613\\
190	0.00343472101618568\\
191	0.00343456967953029\\
192	0.0034344156401284\\
193	0.00343425884996365\\
194	0.0034340992601701\\
195	0.00343393682101718\\
196	0.00343377148189436\\
197	0.00343360319129557\\
198	0.00343343189680334\\
199	0.0034332575450726\\
200	0.00343308008181429\\
201	0.00343289945177855\\
202	0.00343271559873771\\
203	0.00343252846546893\\
204	0.00343233799373655\\
205	0.00343214412427403\\
206	0.00343194679676573\\
207	0.00343174594982822\\
208	0.00343154152099128\\
209	0.00343133344667862\\
210	0.00343112166218817\\
211	0.00343090610167205\\
212	0.00343068669811622\\
213	0.00343046338331963\\
214	0.00343023608787317\\
215	0.00343000474113806\\
216	0.00342976927122399\\
217	0.00342952960496678\\
218	0.00342928566790563\\
219	0.00342903738426005\\
220	0.00342878467690622\\
221	0.00342852746735306\\
222	0.00342826567571775\\
223	0.00342799922070092\\
224	0.00342772801956126\\
225	0.00342745198808977\\
226	0.00342717104058347\\
227	0.00342688508981867\\
228	0.00342659404702375\\
229	0.0034262978218514\\
230	0.00342599632235041\\
231	0.00342568945493688\\
232	0.00342537712436497\\
233	0.00342505923369709\\
234	0.00342473568427346\\
235	0.0034244063756813\\
236	0.00342407120572325\\
237	0.00342373007038532\\
238	0.0034233828638043\\
239	0.0034230294782344\\
240	0.00342266980401349\\
241	0.00342230372952856\\
242	0.0034219311411806\\
243	0.00342155192334889\\
244	0.00342116595835452\\
245	0.00342077312642337\\
246	0.00342037330564831\\
247	0.00341996637195078\\
248	0.0034195521990416\\
249	0.00341913065838114\\
250	0.00341870161913868\\
251	0.00341826494815112\\
252	0.00341782050988081\\
253	0.00341736816637278\\
254	0.00341690777721102\\
255	0.00341643919947406\\
256	0.00341596228768975\\
257	0.00341547689378915\\
258	0.00341498286705964\\
259	0.00341448005409716\\
260	0.00341396829875758\\
261	0.00341344744210718\\
262	0.00341291732237227\\
263	0.00341237777488788\\
264	0.00341182863204548\\
265	0.00341126972323985\\
266	0.00341070087481491\\
267	0.0034101219100086\\
268	0.00340953264889679\\
269	0.00340893290833617\\
270	0.00340832250190613\\
271	0.00340770123984972\\
272	0.00340706892901375\\
273	0.00340642537278796\\
274	0.00340577037104266\\
275	0.00340510372006404\\
276	0.00340442521248928\\
277	0.00340373463724212\\
278	0.00340303177946615\\
279	0.00340231642045696\\
280	0.00340158833759315\\
281	0.00340084730426624\\
282	0.00340009308980937\\
283	0.00339932545942495\\
284	0.00339854417411105\\
285	0.00339774899058674\\
286	0.00339693966121618\\
287	0.00339611593393218\\
288	0.0033952775521561\\
289	0.0033944242547214\\
290	0.00339355577579138\\
291	0.00339267184477818\\
292	0.00339177218626017\\
293	0.00339085651989832\\
294	0.00338992456035151\\
295	0.00338897601719089\\
296	0.00338801059481318\\
297	0.00338702799235318\\
298	0.00338602790359528\\
299	0.00338501001688428\\
300	0.00338397401503539\\
301	0.0033829195752436\\
302	0.00338184636899246\\
303	0.00338075406196239\\
304	0.00337964231393863\\
305	0.00337851077871895\\
306	0.00337735910402129\\
307	0.00337618693139139\\
308	0.00337499389611056\\
309	0.00337377962710332\\
310	0.00337254374684477\\
311	0.00337128587126779\\
312	0.00337000560967265\\
313	0.00336870256464419\\
314	0.00336737633197372\\
315	0.00336602650056444\\
316	0.00336465265234928\\
317	0.00336325436221154\\
318	0.00336183119790872\\
319	0.00336038272000004\\
320	0.0033589084817784\\
321	0.00335740802920719\\
322	0.00335588090086295\\
323	0.00335432662788449\\
324	0.00335274473392965\\
325	0.00335113473514055\\
326	0.00334949614011866\\
327	0.00334782844991109\\
328	0.0033461311580096\\
329	0.00334440375036411\\
330	0.00334264570541242\\
331	0.00334085649412775\\
332	0.00333903558008377\\
333	0.00333718241953204\\
334	0.00333529646147408\\
335	0.00333337714768695\\
336	0.00333142391265077\\
337	0.00332943618348289\\
338	0.00332741338083925\\
339	0.00332535492385863\\
340	0.00332326024002062\\
341	0.0033211287537288\\
342	0.00331895985646738\\
343	0.00331675292394826\\
344	0.00331450728973434\\
345	0.00331222216048866\\
346	0.00330989635828335\\
347	0.00330752761579761\\
348	0.00330511112844656\\
349	0.00330263941456339\\
350	0.00330011690336177\\
351	0.00329755216031542\\
352	0.00329494449277718\\
353	0.00329229319792663\\
354	0.00328959756271786\\
355	0.00328685686384107\\
356	0.00328407036770518\\
357	0.00328123733043031\\
358	0.00327835699778407\\
359	0.00327542860503736\\
360	0.00327245137706731\\
361	0.00326942452835106\\
362	0.00326634726296647\\
363	0.00326321877459954\\
364	0.00326003824655825\\
365	0.00325680485179212\\
366	0.00325351775291687\\
367	0.00325017610224328\\
368	0.00324677904180891\\
369	0.00324332570341137\\
370	0.00323981520864163\\
371	0.00323624666891518\\
372	0.0032326191854982\\
373	0.00322893184952281\\
374	0.00322518374197804\\
375	0.00322137393365025\\
376	0.00321750148498322\\
377	0.00321356544589632\\
378	0.00320956485581192\\
379	0.00320549874406713\\
380	0.00320136612929491\\
381	0.00319716601939614\\
382	0.00319289741177549\\
383	0.00318855929398756\\
384	0.00318415064427167\\
385	0.00317967042827671\\
386	0.00317511757946406\\
387	0.00317049094444681\\
388	0.00316578925304264\\
389	0.00316101137882746\\
390	0.00315615621444637\\
391	0.00315122262383991\\
392	0.00314620943821516\\
393	0.00314111545160758\\
394	0.00313593941641775\\
395	0.00313068004017766\\
396	0.00312533598732079\\
397	0.00311990589695682\\
398	0.00311438844816519\\
399	0.00310878256139036\\
400	0.00310308797710424\\
401	0.00309730682743635\\
402	0.00309144755358902\\
403	0.00308553294398344\\
404	0.00307960777443576\\
405	0.00307370305247482\\
406	0.00306768880373024\\
407	0.0030615630565163\\
408	0.00305532375840407\\
409	0.0030489687691679\\
410	0.0030424958507213\\
411	0.00303590266293422\\
412	0.00302918678566082\\
413	0.00302234573389074\\
414	0.0030153766995041\\
415	0.00300827678087732\\
416	0.00300104297822203\\
417	0.00299367218865368\\
418	0.00298616120097232\\
419	0.00297850669012333\\
420	0.00297070521120253\\
421	0.00296275319266422\\
422	0.0029546469282637\\
423	0.00294638256839115\\
424	0.0029379561147681\\
425	0.00292936341705382\\
426	0.0029206001594658\\
427	0.00291166185151239\\
428	0.00290254381812317\\
429	0.00289324118913661\\
430	0.00288374888809989\\
431	0.00287406162032386\\
432	0.00286417386009392\\
433	0.00285407983679185\\
434	0.00284377351921915\\
435	0.00283324859613197\\
436	0.00282249844842589\\
437	0.00281151610819758\\
438	0.00280029423099783\\
439	0.00278882526319504\\
440	0.0027771019541476\\
441	0.00276511668582589\\
442	0.00275286060858101\\
443	0.00274032442670874\\
444	0.00272749837285342\\
445	0.00271437218112794\\
446	0.00270093505952688\\
447	0.0026871756644159\\
448	0.00267308208307675\\
449	0.00265864181460575\\
450	0.00264384168309426\\
451	0.00262866779095149\\
452	0.00261310541108611\\
453	0.00259713867596304\\
454	0.00258074962229649\\
455	0.00256391505904901\\
456	0.00254659623316061\\
457	0.0025286955854366\\
458	0.00250987886118515\\
459	0.00248796942335234\\
460	0.00246493431717584\\
461	0.00244142489791835\\
462	0.00241733485852565\\
463	0.00239232361604092\\
464	0.00236569411982794\\
465	0.00233856268707768\\
466	0.00231095686283289\\
467	0.00228286699853765\\
468	0.0022542819120225\\
469	0.00222518696093254\\
470	0.00219555975230399\\
471	0.00216537201516941\\
472	0.00213465618073841\\
473	0.00210341748129568\\
474	0.0020716692733659\\
475	0.0020394910917054\\
476	0.00200716271405622\\
477	0.0019754970633491\\
478	0.00194601804071996\\
479	0.00191765058876644\\
480	0.00188845229302002\\
481	0.00185837562380705\\
482	0.0018273670959291\\
483	0.00179536361596924\\
484	0.00176227937297323\\
485	0.00172794340871963\\
486	0.00169145087460265\\
487	0.00165134325332256\\
488	0.00161026073528203\\
489	0.00156816972602933\\
490	0.00152503488596659\\
491	0.00148081997832478\\
492	0.0014354873087811\\
493	0.00138899963751852\\
494	0.00134132222049671\\
495	0.00129240247908461\\
496	0.00124219552545212\\
497	0.00119065577442196\\
498	0.00113773508033689\\
499	0.00108336920262801\\
500	0.00102747393336099\\
501	0.000969990569839575\\
502	0.000910857118539138\\
503	0.000850006359514929\\
504	0.000787355549633204\\
505	0.00072274295263537\\
506	0.000655701571956031\\
507	0.00058665174801009\\
508	0.000515530777339095\\
509	0.000441372331336617\\
510	0.000364445223296124\\
511	0.000285746035416068\\
512	0.000205074854842097\\
513	0.000121623393036926\\
514	3.08717202713595e-05\\
515	0\\
516	0\\
517	0\\
518	0\\
519	0\\
520	0\\
521	0\\
522	0\\
523	0\\
524	0\\
525	0\\
526	0\\
527	0\\
528	0\\
529	0\\
530	0\\
531	0\\
532	0\\
533	0\\
534	0\\
535	0\\
536	0\\
537	0\\
538	0\\
539	0\\
540	0\\
541	0\\
542	0\\
543	0\\
544	0\\
545	0\\
546	0\\
547	0\\
548	0\\
549	0\\
550	0\\
551	0\\
552	0\\
553	0\\
554	0\\
555	0\\
556	0\\
557	0\\
558	0\\
559	0\\
560	0\\
561	0\\
562	0\\
563	0\\
564	0\\
565	0\\
566	0\\
567	0\\
568	0\\
569	0\\
570	0\\
571	0\\
572	0\\
573	0\\
574	0\\
575	0\\
576	0\\
577	0\\
578	0\\
579	0\\
580	0\\
581	0\\
582	0\\
583	0\\
584	0\\
585	0\\
586	0\\
587	0\\
588	0\\
589	0\\
590	0\\
591	0\\
592	0\\
593	0\\
594	0\\
595	0\\
596	0\\
597	0\\
598	0\\
599	0\\
600	0\\
};
\addplot [color=mycolor17,solid,forget plot]
  table[row sep=crcr]{%
1	0.00189995238561724\\
2	0.0018999437136439\\
3	0.00189993488465496\\
4	0.00189992589581167\\
5	0.00189991674422404\\
6	0.00189990742695\\
7	0.0018998979409944\\
8	0.00189988828330809\\
9	0.00189987845078691\\
10	0.00189986844027078\\
11	0.00189985824854262\\
12	0.00189984787232735\\
13	0.00189983730829086\\
14	0.00189982655303896\\
15	0.00189981560311625\\
16	0.00189980445500509\\
17	0.00189979310512444\\
18	0.00189978154982872\\
19	0.00189976978540668\\
20	0.00189975780808019\\
21	0.0018997456140031\\
22	0.00189973319925991\\
23	0.00189972055986466\\
24	0.00189970769175955\\
25	0.00189969459081374\\
26	0.00189968125282197\\
27	0.00189966767350329\\
28	0.00189965384849965\\
29	0.00189963977337455\\
30	0.00189962544361162\\
31	0.00189961085461321\\
32	0.00189959600169889\\
33	0.00189958088010403\\
34	0.00189956548497822\\
35	0.00189954981138381\\
36	0.00189953385429427\\
37	0.00189951760859267\\
38	0.00189950106906999\\
39	0.00189948423042356\\
40	0.0018994670872553\\
41	0.00189944963407005\\
42	0.00189943186527383\\
43	0.00189941377517207\\
44	0.00189939535796783\\
45	0.00189937660775992\\
46	0.00189935751854108\\
47	0.00189933808419607\\
48	0.00189931829849972\\
49	0.001899298155115\\
50	0.00189927764759099\\
51	0.00189925676936085\\
52	0.00189923551373976\\
53	0.00189921387392279\\
54	0.00189919184298278\\
55	0.00189916941386813\\
56	0.0018991465794006\\
57	0.00189912333227302\\
58	0.00189909966504699\\
59	0.00189907557015058\\
60	0.0018990510398759\\
61	0.00189902606637667\\
62	0.00189900064166575\\
63	0.00189897475761267\\
64	0.00189894840594101\\
65	0.00189892157822584\\
66	0.00189889426589102\\
67	0.00189886646020656\\
68	0.00189883815228581\\
69	0.00189880933308273\\
70	0.00189877999338901\\
71	0.00189875012383114\\
72	0.00189871971486753\\
73	0.00189868875678549\\
74	0.00189865723969814\\
75	0.00189862515354133\\
76	0.00189859248807052\\
77	0.00189855923285746\\
78	0.00189852537728702\\
79	0.00189849091055379\\
80	0.00189845582165872\\
81	0.00189842009940564\\
82	0.00189838373239779\\
83	0.00189834670903417\\
84	0.00189830901750601\\
85	0.00189827064579293\\
86	0.00189823158165929\\
87	0.00189819181265027\\
88	0.00189815132608804\\
89	0.00189811010906769\\
90	0.00189806814845327\\
91	0.00189802543087364\\
92	0.00189798194271827\\
93	0.00189793767013299\\
94	0.00189789259901563\\
95	0.00189784671501165\\
96	0.00189780000350958\\
97	0.00189775244963649\\
98	0.00189770403825333\\
99	0.00189765475395014\\
100	0.00189760458104132\\
101	0.00189755350356063\\
102	0.00189750150525623\\
103	0.00189744856958562\\
104	0.00189739467971041\\
105	0.00189733981849109\\
106	0.00189728396848168\\
107	0.00189722711192423\\
108	0.0018971692307433\\
109	0.00189711030654029\\
110	0.00189705032058771\\
111	0.00189698925382329\\
112	0.00189692708684406\\
113	0.00189686379990026\\
114	0.00189679937288919\\
115	0.00189673378534888\\
116	0.00189666701645176\\
117	0.00189659904499809\\
118	0.00189652984940941\\
119	0.00189645940772171\\
120	0.00189638769757864\\
121	0.00189631469622449\\
122	0.0018962403804971\\
123	0.00189616472682061\\
124	0.00189608771119812\\
125	0.00189600930920417\\
126	0.00189592949597712\\
127	0.00189584824621138\\
128	0.00189576553414956\\
129	0.00189568133357432\\
130	0.00189559561780031\\
131	0.00189550835966573\\
132	0.00189541953152391\\
133	0.00189532910523467\\
134	0.00189523705215551\\
135	0.00189514334313268\\
136	0.00189504794849211\\
137	0.00189495083803007\\
138	0.0018948519810038\\
139	0.0018947513461219\\
140	0.00189464890153454\\
141	0.00189454461482352\\
142	0.00189443845299216\\
143	0.00189433038245497\\
144	0.00189422036902718\\
145	0.00189410837791406\\
146	0.00189399437370003\\
147	0.00189387832033764\\
148	0.00189376018113623\\
149	0.00189363991875057\\
150	0.00189351749516907\\
151	0.00189339287170204\\
152	0.00189326600896943\\
153	0.00189313686688867\\
154	0.00189300540466206\\
155	0.001892871580764\\
156	0.00189273535292805\\
157	0.00189259667813369\\
158	0.00189245551259283\\
159	0.00189231181173616\\
160	0.00189216553019915\\
161	0.00189201662180786\\
162	0.00189186503956452\\
163	0.00189171073563275\\
164	0.00189155366132262\\
165	0.00189139376707535\\
166	0.00189123100244783\\
167	0.00189106531609673\\
168	0.00189089665576248\\
169	0.00189072496825281\\
170	0.00189055019942608\\
171	0.00189037229417427\\
172	0.00189019119640573\\
173	0.00189000684902745\\
174	0.00188981919392722\\
175	0.00188962817195532\\
176	0.00188943372290589\\
177	0.00188923578549803\\
178	0.00188903429735649\\
179	0.00188882919499201\\
180	0.00188862041378136\\
181	0.00188840788794691\\
182	0.00188819155053594\\
183	0.00188797133339948\\
184	0.0018877471671708\\
185	0.00188751898124351\\
186	0.00188728670374925\\
187	0.00188705026153492\\
188	0.00188680958013959\\
189	0.00188656458377089\\
190	0.001886315195281\\
191	0.0018860613361422\\
192	0.00188580292642196\\
193	0.00188553988475757\\
194	0.00188527212833025\\
195	0.00188499957283889\\
196	0.00188472213247319\\
197	0.00188443971988633\\
198	0.00188415224616716\\
199	0.00188385962081185\\
200	0.00188356175169499\\
201	0.00188325854504014\\
202	0.00188294990538991\\
203	0.00188263573557535\\
204	0.00188231593668489\\
205	0.00188199040803256\\
206	0.00188165904712573\\
207	0.0018813217496322\\
208	0.00188097840934658\\
209	0.00188062891815617\\
210	0.00188027316600614\\
211	0.00187991104086394\\
212	0.00187954242868321\\
213	0.00187916721336689\\
214	0.00187878527672963\\
215	0.00187839649845949\\
216	0.00187800075607892\\
217	0.00187759792490499\\
218	0.0018771878780088\\
219	0.0018767704861742\\
220	0.00187634561785559\\
221	0.00187591313913506\\
222	0.0018754729136785\\
223	0.00187502480269105\\
224	0.00187456866487158\\
225	0.00187410435636625\\
226	0.00187363173072126\\
227	0.00187315063883462\\
228	0.00187266092890697\\
229	0.00187216244639142\\
230	0.0018716550339425\\
231	0.00187113853136392\\
232	0.00187061277555552\\
233	0.00187007760045899\\
234	0.0018695328370026\\
235	0.00186897831304481\\
236	0.0018684138533168\\
237	0.00186783927936376\\
238	0.00186725440948511\\
239	0.0018666590586735\\
240	0.00186605303855252\\
241	0.00186543615731325\\
242	0.0018648082196495\\
243	0.00186416902669172\\
244	0.00186351837593962\\
245	0.00186285606119344\\
246	0.00186218187248376\\
247	0.00186149559599996\\
248	0.0018607970140172\\
249	0.00186008590482191\\
250	0.00185936204263577\\
251	0.00185862519753818\\
252	0.00185787513538704\\
253	0.00185711161773808\\
254	0.00185633440176234\\
255	0.00185554324016213\\
256	0.00185473788108515\\
257	0.00185391806803691\\
258	0.00185308353979131\\
259	0.00185223403029937\\
260	0.00185136926859611\\
261	0.00185048897870546\\
262	0.00184959287954318\\
263	0.00184868068481791\\
264	0.00184775210292986\\
265	0.00184680683686772\\
266	0.00184584458410315\\
267	0.0018448650364832\\
268	0.00184386788012037\\
269	0.00184285279528038\\
270	0.00184181945626755\\
271	0.00184076753130771\\
272	0.00183969668242903\\
273	0.00183860656534083\\
274	0.0018374968293106\\
275	0.00183636711703653\\
276	0.00183521706451204\\
277	0.00183404630089088\\
278	0.00183285444835665\\
279	0.00183164112198385\\
280	0.00183040592959619\\
281	0.00182914847162169\\
282	0.00182786834094467\\
283	0.00182656512275454\\
284	0.0018252383943913\\
285	0.00182388772518768\\
286	0.00182251267630782\\
287	0.00182111280058288\\
288	0.0018196876423409\\
289	0.00181823673723737\\
290	0.00181675961207759\\
291	0.00181525578463748\\
292	0.00181372476348\\
293	0.00181216604776756\\
294	0.00181057912707035\\
295	0.00180896348117056\\
296	0.00180731857986229\\
297	0.00180564388274715\\
298	0.00180393883902548\\
299	0.00180220288728304\\
300	0.0018004354552731\\
301	0.00179863595969402\\
302	0.00179680380596191\\
303	0.00179493838797864\\
304	0.00179303908789491\\
305	0.00179110527586846\\
306	0.0017891363098173\\
307	0.00178713153516803\\
308	0.00178509028459898\\
309	0.00178301187777807\\
310	0.00178089562109412\\
311	0.00177874080737972\\
312	0.00177654671562439\\
313	0.00177431261068608\\
314	0.00177203774302466\\
315	0.00176972134844518\\
316	0.00176736264775351\\
317	0.00176496084645114\\
318	0.00176251513442429\\
319	0.00176002468562693\\
320	0.00175748865775794\\
321	0.00175490619193196\\
322	0.00175227641234405\\
323	0.00174959842592766\\
324	0.0017468713220059\\
325	0.00174409417193571\\
326	0.00174126602874465\\
327	0.00173838592675978\\
328	0.00173545288122827\\
329	0.00173246588792904\\
330	0.00172942392277472\\
331	0.0017263259414026\\
332	0.00172317087875161\\
333	0.00171995764861706\\
334	0.00171668514315773\\
335	0.00171335223228456\\
336	0.00170995776275204\\
337	0.0017065005565978\\
338	0.00170297940867129\\
339	0.00169939308558086\\
340	0.00169574034130069\\
341	0.00169201997073723\\
342	0.0016882307717239\\
343	0.00168437138920443\\
344	0.00168044040580117\\
345	0.00167643625454292\\
346	0.00167235691241596\\
347	0.00166819882434387\\
348	0.00166395318686893\\
349	0.00165959423041928\\
350	0.00165506096071799\\
351	0.00165037745262354\\
352	0.00164561111846711\\
353	0.00164076042838519\\
354	0.00163582382245267\\
355	0.00163079971010773\\
356	0.00162568646959972\\
357	0.0016204824475013\\
358	0.00161518595826011\\
359	0.00160979528345505\\
360	0.00160430867059299\\
361	0.00159872433309202\\
362	0.00159304044970079\\
363	0.00158725516392286\\
364	0.00158136658344821\\
365	0.00157537277959397\\
366	0.00156927178675731\\
367	0.00156306160188326\\
368	0.00155674018395112\\
369	0.00155030545348344\\
370	0.00154375529208251\\
371	0.00153708754200058\\
372	0.00153030000575121\\
373	0.00152339044577001\\
374	0.00151635658412768\\
375	0.00150919610227706\\
376	0.00150190664075992\\
377	0.00149448579874129\\
378	0.00148693113347432\\
379	0.00147924016103994\\
380	0.00147141035952659\\
381	0.00146343916733997\\
382	0.00145532398482506\\
383	0.00144706217760111\\
384	0.00143865108289257\\
385	0.00143008801900107\\
386	0.00142137028793421\\
387	0.00141249511799778\\
388	0.00140345938882302\\
389	0.00139425929207139\\
390	0.00138489183209175\\
391	0.00137535422876928\\
392	0.00136564368716252\\
393	0.00135575740034698\\
394	0.00134569255308525\\
395	0.00133544632776129\\
396	0.00132501591701025\\
397	0.00131439855660885\\
398	0.00130359161994592\\
399	0.00129259289934463\\
400	0.0012814014516963\\
401	0.0012700201346766\\
402	0.00125846313425795\\
403	0.00124677777455407\\
404	0.00123510367046939\\
405	0.00122379391174075\\
406	0.00121331495775664\\
407	0.00120261856048567\\
408	0.00119170054973482\\
409	0.001180556648175\\
410	0.00116918246115891\\
411	0.00115757344745319\\
412	0.00114572489073096\\
413	0.0011336319868634\\
414	0.00112128991524153\\
415	0.00110869242013178\\
416	0.00109583303037424\\
417	0.00108270504925331\\
418	0.00106930154381129\\
419	0.00105561533360234\\
420	0.00104163897886202\\
421	0.00102736476786103\\
422	0.0010127847024471\\
423	0.000997890479493102\\
424	0.000982673469571751\\
425	0.000967124709500084\\
426	0.000951234904115045\\
427	0.000934994389298189\\
428	0.000918393112486593\\
429	0.000901420611983254\\
430	0.000884065994998693\\
431	0.000866317914362073\\
432	0.000848164543847834\\
433	0.00082959355204378\\
434	0.000810592074483101\\
435	0.000791146682864857\\
436	0.000771243346814465\\
437	0.000750867373037908\\
438	0.000730003280363605\\
439	0.000708634566704641\\
440	0.000686744057229842\\
441	0.00066431680296066\\
442	0.000641337978190259\\
443	0.000617788359155828\\
444	0.000593647776015725\\
445	0.000568895056808599\\
446	0.000543507967089672\\
447	0.000517463145007269\\
448	0.000490736039535908\\
449	0.000463300891635387\\
450	0.000435130742918302\\
451	0.000406197158852362\\
452	0.000376470248027807\\
453	0.000345918755290155\\
454	0.000314510300852118\\
455	0.00028221207140335\\
456	0.000248992681573801\\
457	0.000214839035451101\\
458	0.000179918523768187\\
459	0.000151007207087082\\
460	0.000123087413694654\\
461	9.45158457683247e-05\\
462	6.51988540410677e-05\\
463	3.47459928719172e-05\\
464	1.0433240778689e-06\\
465	0\\
466	0\\
467	0\\
468	0\\
469	0\\
470	0\\
471	0\\
472	0\\
473	0\\
474	0\\
475	0\\
476	0\\
477	0\\
478	0\\
479	0\\
480	0\\
481	0\\
482	0\\
483	0\\
484	0\\
485	0\\
486	0\\
487	0\\
488	0\\
489	0\\
490	0\\
491	0\\
492	0\\
493	0\\
494	0\\
495	0\\
496	0\\
497	0\\
498	0\\
499	0\\
500	0\\
501	0\\
502	0\\
503	0\\
504	0\\
505	0\\
506	0\\
507	0\\
508	0\\
509	0\\
510	0\\
511	0\\
512	0\\
513	0\\
514	0\\
515	0\\
516	0\\
517	0\\
518	0\\
519	0\\
520	0\\
521	0\\
522	0\\
523	0\\
524	0\\
525	0\\
526	0\\
527	0\\
528	0\\
529	0\\
530	0\\
531	0\\
532	0\\
533	0\\
534	0\\
535	0\\
536	0\\
537	0\\
538	0\\
539	0\\
540	0\\
541	0\\
542	0\\
543	0\\
544	0\\
545	0\\
546	0\\
547	0\\
548	0\\
549	0\\
550	0\\
551	0\\
552	0\\
553	0\\
554	0\\
555	0\\
556	0\\
557	0\\
558	0\\
559	0\\
560	0\\
561	0\\
562	0\\
563	0\\
564	0\\
565	0\\
566	0\\
567	0\\
568	0\\
569	0\\
570	0\\
571	0\\
572	0\\
573	0\\
574	0\\
575	0\\
576	0\\
577	0\\
578	0\\
579	0\\
580	0\\
581	0\\
582	0\\
583	0\\
584	0\\
585	0\\
586	0\\
587	0\\
588	0\\
589	0\\
590	0\\
591	0\\
592	0\\
593	0\\
594	0\\
595	0\\
596	0\\
597	0\\
598	0\\
599	0\\
600	0\\
};
\addplot [color=mycolor18,solid,forget plot]
  table[row sep=crcr]{%
1	0.000185619121464617\\
2	0.000185612706249818\\
3	0.000185606174741565\\
4	0.00018559952483321\\
5	0.000185592754379972\\
6	0.000185585861198282\\
7	0.000185578843065035\\
8	0.000185571697716924\\
9	0.000185564422849684\\
10	0.000185557016117363\\
11	0.000185549475131572\\
12	0.000185541797460714\\
13	0.00018553398062921\\
14	0.000185526022116701\\
15	0.000185517919357245\\
16	0.00018550966973849\\
17	0.000185501270600826\\
18	0.000185492719236572\\
19	0.000185484012889063\\
20	0.000185475148751796\\
21	0.000185466123967508\\
22	0.0001854569356273\\
23	0.000185447580769662\\
24	0.000185438056379553\\
25	0.000185428359387416\\
26	0.000185418486668229\\
27	0.000185408435040469\\
28	0.000185398201265113\\
29	0.000185387782044588\\
30	0.00018537717402175\\
31	0.000185366373778767\\
32	0.000185355377836053\\
33	0.00018534418265115\\
34	0.000185332784617606\\
35	0.000185321180063799\\
36	0.000185309365251772\\
37	0.000185297336376068\\
38	0.000185285089562478\\
39	0.000185272620866811\\
40	0.000185259926273652\\
41	0.000185247001695066\\
42	0.000185233842969297\\
43	0.000185220445859436\\
44	0.000185206806052073\\
45	0.000185192919155937\\
46	0.000185178780700454\\
47	0.000185164386134368\\
48	0.000185149730824251\\
49	0.00018513481005306\\
50	0.000185119619018611\\
51	0.000185104152832056\\
52	0.000185088406516322\\
53	0.000185072375004529\\
54	0.000185056053138373\\
55	0.000185039435666491\\
56	0.000185022517242776\\
57	0.000185005292424688\\
58	0.000184987755671518\\
59	0.000184969901342615\\
60	0.000184951723695606\\
61	0.00018493321688456\\
62	0.000184914374958128\\
63	0.000184895191857661\\
64	0.000184875661415266\\
65	0.000184855777351865\\
66	0.000184835533275179\\
67	0.000184814922677716\\
68	0.000184793938934687\\
69	0.000184772575301912\\
70	0.000184750824913678\\
71	0.000184728680780546\\
72	0.000184706135787143\\
73	0.000184683182689907\\
74	0.000184659814114774\\
75	0.000184636022554853\\
76	0.000184611800368026\\
77	0.000184587139774541\\
78	0.00018456203285453\\
79	0.000184536471545518\\
80	0.000184510447639833\\
81	0.000184483952782041\\
82	0.000184456978466267\\
83	0.000184429516033519\\
84	0.000184401556668935\\
85	0.000184373091398991\\
86	0.00018434411108867\\
87	0.000184314606438536\\
88	0.000184284567981833\\
89	0.000184253986081458\\
90	0.000184222850926915\\
91	0.000184191152531218\\
92	0.000184158880727728\\
93	0.00018412602516692\\
94	0.000184092575313138\\
95	0.000184058520441248\\
96	0.000184023849633228\\
97	0.000183988551774742\\
98	0.000183952615551622\\
99	0.000183916029446278\\
100	0.000183878781734071\\
101	0.000183840860479602\\
102	0.00018380225353293\\
103	0.000183762948525753\\
104	0.000183722932867498\\
105	0.000183682193741331\\
106	0.000183640718100111\\
107	0.000183598492662304\\
108	0.000183555503907729\\
109	0.000183511738073346\\
110	0.000183467181148893\\
111	0.000183421818872449\\
112	0.000183375636725948\\
113	0.000183328619930616\\
114	0.000183280753442277\\
115	0.000183232021946637\\
116	0.000183182409854425\\
117	0.000183131901296501\\
118	0.000183080480118849\\
119	0.000183028129877477\\
120	0.000182974833833243\\
121	0.000182920574946577\\
122	0.000182865335872101\\
123	0.000182809098953183\\
124	0.000182751846216364\\
125	0.000182693559365692\\
126	0.000182634219776969\\
127	0.000182573808491879\\
128	0.000182512306212025\\
129	0.000182449693292846\\
130	0.000182385949737433\\
131	0.000182321055190257\\
132	0.000182254988930714\\
133	0.000182187729866642\\
134	0.000182119256527671\\
135	0.000182049547058475\\
136	0.000181978579211895\\
137	0.000181906330341912\\
138	0.000181832777396575\\
139	0.000181757896910689\\
140	0.000181681664998503\\
141	0.000181604057346123\\
142	0.000181525049203929\\
143	0.000181444615378757\\
144	0.000181362730225995\\
145	0.000181279367641508\\
146	0.00018119450105343\\
147	0.00018110810341383\\
148	0.000181020147190189\\
149	0.000180930604356744\\
150	0.000180839446385702\\
151	0.000180746644238246\\
152	0.000180652168355429\\
153	0.000180555988648863\\
154	0.000180458074491286\\
155	0.000180358394706937\\
156	0.000180256917561722\\
157	0.000180153610753286\\
158	0.000180048441400838\\
159	0.000179941376034829\\
160	0.000179832380586422\\
161	0.000179721420376791\\
162	0.00017960846010624\\
163	0.000179493463843082\\
164	0.000179376395012368\\
165	0.000179257216384383\\
166	0.000179135890062951\\
167	0.000179012377473527\\
168	0.000178886639351057\\
169	0.000178758635727662\\
170	0.00017862832592005\\
171	0.00017849566851675\\
172	0.000178360621365075\\
173	0.00017822314155788\\
174	0.000178083185420068\\
175	0.000177940708494852\\
176	0.000177795665529789\\
177	0.000177648010462527\\
178	0.000177497696406333\\
179	0.000177344675635338\\
180	0.000177188899569507\\
181	0.000177030318759379\\
182	0.000176868882870475\\
183	0.00017670454066747\\
184	0.000176537239998063\\
185	0.000176366927776539\\
186	0.000176193549967079\\
187	0.000176017051566696\\
188	0.000175837376587934\\
189	0.000175654468041207\\
190	0.000175468267916847\\
191	0.000175278717166785\\
192	0.000175085755685953\\
193	0.000174889322293305\\
194	0.000174689354712498\\
195	0.00017448578955225\\
196	0.000174278562286291\\
197	0.000174067607233022\\
198	0.00017385285753469\\
199	0.000173634245136292\\
200	0.000173411700764045\\
201	0.000173185153903429\\
202	0.000172954532776905\\
203	0.000172719764321141\\
204	0.000172480774163874\\
205	0.000172237486600319\\
206	0.000171989824569154\\
207	0.000171737709628052\\
208	0.000171481061928769\\
209	0.000171219800191751\\
210	0.000170953841680305\\
211	0.000170683102174234\\
212	0.000170407495942966\\
213	0.000170126935718529\\
214	0.00016984133266724\\
215	0.000169550596361666\\
216	0.000169254634751657\\
217	0.000168953354134901\\
218	0.000168646659126945\\
219	0.000168334452630624\\
220	0.00016801663580494\\
221	0.000167693108033341\\
222	0.000167363766891356\\
223	0.000167028508113683\\
224	0.000166687225560611\\
225	0.000166339811183783\\
226	0.000165986154991326\\
227	0.000165626145012311\\
228	0.000165259667260515\\
229	0.000164886605697482\\
230	0.000164506842194881\\
231	0.000164120256496109\\
232	0.000163726726177173\\
233	0.000163326126606785\\
234	0.00016291833090568\\
235	0.00016250320990514\\
236	0.000162080632104701\\
237	0.000161650463628998\\
238	0.000161212568183803\\
239	0.000160766807011131\\
240	0.000160313038843497\\
241	0.000159851119857239\\
242	0.000159380903624893\\
243	0.000158902241066628\\
244	0.000158414980400708\\
245	0.0001579189670929\\
246	0.000157414043804938\\
247	0.000156900050341865\\
248	0.000156376823598324\\
249	0.0001558441975038\\
250	0.000155302002966622\\
251	0.000154750067816939\\
252	0.000154188216748465\\
253	0.000153616271258996\\
254	0.000153034049589741\\
255	0.000152441366663364\\
256	0.000151838034020765\\
257	0.000151223859756513\\
258	0.000150598648452949\\
259	0.000149962201112916\\
260	0.000149314315091053\\
261	0.000148654784023675\\
262	0.000147983397757145\\
263	0.000147299942275301\\
264	0.000146604199622824\\
265	0.000145895947831796\\
266	0.00014517496084202\\
267	0.000144441008421891\\
268	0.000143693856087161\\
269	0.000142933265017849\\
270	0.000142158991973321\\
271	0.000141370789205363\\
272	0.000140568404369321\\
273	0.000139751580433838\\
274	0.000138920055590997\\
275	0.000138073563169448\\
276	0.000137211831543223\\
277	0.00013633458400787\\
278	0.000135441538663405\\
279	0.000134532408336163\\
280	0.000133606900471857\\
281	0.000132664717025979\\
282	0.000131705554351407\\
283	0.00013072910308324\\
284	0.000129735048020551\\
285	0.000128723068005081\\
286	0.000127692835796773\\
287	0.00012664401794577\\
288	0.000125576274661166\\
289	0.000124489259675891\\
290	0.000123382620107833\\
291	0.000122255996316973\\
292	0.000121109021758344\\
293	0.000119941322830555\\
294	0.000118752518719733\\
295	0.000117542221238646\\
296	0.000116310034660682\\
297	0.000115055555548544\\
298	0.00011377837257723\\
299	0.000112478066351131\\
300	0.000111154209214821\\
301	0.000109806365057201\\
302	0.000108434089108701\\
303	0.000107036927731037\\
304	0.000105614418199145\\
305	0.000104166088474863\\
306	0.000102691456971949\\
307	0.00010119003231192\\
308	9.96613130704809e-05\\
309	9.81047875139772e-05\\
310	9.65199333248541e-05\\
311	9.49062173128027e-05\\
312	9.32630951025622e-05\\
313	9.15900107841814e-05\\
314	8.98863965375841e-05\\
315	8.81516723557525e-05\\
316	8.63852459281013e-05\\
317	8.4586512040245e-05\\
318	8.27548521819326e-05\\
319	8.08896341350972e-05\\
320	7.8990211540252e-05\\
321	7.70559234394652e-05\\
322	7.50860937938349e-05\\
323	7.30800309730493e-05\\
324	7.10370272144234e-05\\
325	6.89563580484116e-05\\
326	6.68372816871838e-05\\
327	6.46790383722825e-05\\
328	6.24808496769509e-05\\
329	6.02419177578719e-05\\
330	5.79614245501665e-05\\
331	5.56385308980601e-05\\
332	5.32723756105729e-05\\
333	5.0862074424749e-05\\
334	4.84067188397502e-05\\
335	4.59053747312909e-05\\
336	4.33570805044987e-05\\
337	4.0760844132857e-05\\
338	3.81156374116781e-05\\
339	3.54203838004354e-05\\
340	3.26739372049016e-05\\
341	2.98751043135728e-05\\
342	2.70229872211389e-05\\
343	2.41169932287785e-05\\
344	2.11558559878758e-05\\
345	1.81381856318131e-05\\
346	1.50622635655824e-05\\
347	1.19252419144098e-05\\
348	8.71976743864639e-06\\
349	5.41806378954149e-06\\
350	1.88170861355089e-06\\
351	0\\
352	0\\
353	0\\
354	0\\
355	0\\
356	0\\
357	0\\
358	0\\
359	0\\
360	0\\
361	0\\
362	0\\
363	0\\
364	0\\
365	0\\
366	0\\
367	0\\
368	0\\
369	0\\
370	0\\
371	0\\
372	0\\
373	0\\
374	0\\
375	0\\
376	0\\
377	0\\
378	0\\
379	0\\
380	0\\
381	0\\
382	0\\
383	0\\
384	0\\
385	0\\
386	0\\
387	0\\
388	0\\
389	0\\
390	0\\
391	0\\
392	0\\
393	0\\
394	0\\
395	0\\
396	0\\
397	0\\
398	0\\
399	0\\
400	0\\
401	0\\
402	0\\
403	0\\
404	0\\
405	0\\
406	0\\
407	0\\
408	0\\
409	0\\
410	0\\
411	0\\
412	0\\
413	0\\
414	0\\
415	0\\
416	0\\
417	0\\
418	0\\
419	0\\
420	0\\
421	0\\
422	0\\
423	0\\
424	0\\
425	0\\
426	0\\
427	0\\
428	0\\
429	0\\
430	0\\
431	0\\
432	0\\
433	0\\
434	0\\
435	0\\
436	0\\
437	0\\
438	0\\
439	0\\
440	0\\
441	0\\
442	0\\
443	0\\
444	0\\
445	0\\
446	0\\
447	0\\
448	0\\
449	0\\
450	0\\
451	0\\
452	0\\
453	0\\
454	0\\
455	0\\
456	0\\
457	0\\
458	0\\
459	0\\
460	0\\
461	0\\
462	0\\
463	0\\
464	0\\
465	0\\
466	0\\
467	0\\
468	0\\
469	0\\
470	0\\
471	0\\
472	0\\
473	0\\
474	0\\
475	0\\
476	0\\
477	0\\
478	0\\
479	0\\
480	0\\
481	0\\
482	0\\
483	0\\
484	0\\
485	0\\
486	0\\
487	0\\
488	0\\
489	0\\
490	0\\
491	0\\
492	0\\
493	0\\
494	0\\
495	0\\
496	0\\
497	0\\
498	0\\
499	0\\
500	0\\
501	0\\
502	0\\
503	0\\
504	0\\
505	0\\
506	0\\
507	0\\
508	0\\
509	0\\
510	0\\
511	0\\
512	0\\
513	0\\
514	0\\
515	0\\
516	0\\
517	0\\
518	0\\
519	0\\
520	0\\
521	0\\
522	0\\
523	0\\
524	0\\
525	0\\
526	0\\
527	0\\
528	0\\
529	0\\
530	0\\
531	0\\
532	0\\
533	0\\
534	0\\
535	0\\
536	0\\
537	0\\
538	0\\
539	0\\
540	0\\
541	0\\
542	0\\
543	0\\
544	0\\
545	0\\
546	0\\
547	0\\
548	0\\
549	0\\
550	0\\
551	0\\
552	0\\
553	0\\
554	0\\
555	0\\
556	0\\
557	0\\
558	0\\
559	0\\
560	0\\
561	0\\
562	0\\
563	0\\
564	0\\
565	0\\
566	0\\
567	0\\
568	0\\
569	0\\
570	0\\
571	0\\
572	0\\
573	0\\
574	0\\
575	0\\
576	0\\
577	0\\
578	0\\
579	0\\
580	0\\
581	0\\
582	0\\
583	0\\
584	0\\
585	0\\
586	0\\
587	0\\
588	0\\
589	0\\
590	0\\
591	0\\
592	0\\
593	0\\
594	0\\
595	0\\
596	0\\
597	0\\
598	0\\
599	0\\
600	0\\
};
\addplot [color=red!25!mycolor17,solid,forget plot]
  table[row sep=crcr]{%
1	0\\
2	0\\
3	0\\
4	0\\
5	0\\
6	0\\
7	0\\
8	0\\
9	0\\
10	0\\
11	0\\
12	0\\
13	0\\
14	0\\
15	0\\
16	0\\
17	0\\
18	0\\
19	0\\
20	0\\
21	0\\
22	0\\
23	0\\
24	0\\
25	0\\
26	0\\
27	0\\
28	0\\
29	0\\
30	0\\
31	0\\
32	0\\
33	0\\
34	0\\
35	0\\
36	0\\
37	0\\
38	0\\
39	0\\
40	0\\
41	0\\
42	0\\
43	0\\
44	0\\
45	0\\
46	0\\
47	0\\
48	0\\
49	0\\
50	0\\
51	0\\
52	0\\
53	0\\
54	0\\
55	0\\
56	0\\
57	0\\
58	0\\
59	0\\
60	0\\
61	0\\
62	0\\
63	0\\
64	0\\
65	0\\
66	0\\
67	0\\
68	0\\
69	0\\
70	0\\
71	0\\
72	0\\
73	0\\
74	0\\
75	0\\
76	0\\
77	0\\
78	0\\
79	0\\
80	0\\
81	0\\
82	0\\
83	0\\
84	0\\
85	0\\
86	0\\
87	0\\
88	0\\
89	0\\
90	0\\
91	0\\
92	0\\
93	0\\
94	0\\
95	0\\
96	0\\
97	0\\
98	0\\
99	0\\
100	0\\
101	0\\
102	0\\
103	0\\
104	0\\
105	0\\
106	0\\
107	0\\
108	0\\
109	0\\
110	0\\
111	0\\
112	0\\
113	0\\
114	0\\
115	0\\
116	0\\
117	0\\
118	0\\
119	0\\
120	0\\
121	0\\
122	0\\
123	0\\
124	0\\
125	0\\
126	0\\
127	0\\
128	0\\
129	0\\
130	0\\
131	0\\
132	0\\
133	0\\
134	0\\
135	0\\
136	0\\
137	0\\
138	0\\
139	0\\
140	0\\
141	0\\
142	0\\
143	0\\
144	0\\
145	0\\
146	0\\
147	0\\
148	0\\
149	0\\
150	0\\
151	0\\
152	0\\
153	0\\
154	0\\
155	0\\
156	0\\
157	0\\
158	0\\
159	0\\
160	0\\
161	0\\
162	0\\
163	0\\
164	0\\
165	0\\
166	0\\
167	0\\
168	0\\
169	0\\
170	0\\
171	0\\
172	0\\
173	0\\
174	0\\
175	0\\
176	0\\
177	0\\
178	0\\
179	0\\
180	0\\
181	0\\
182	0\\
183	0\\
184	0\\
185	0\\
186	0\\
187	0\\
188	0\\
189	0\\
190	0\\
191	0\\
192	0\\
193	0\\
194	0\\
195	0\\
196	0\\
197	0\\
198	0\\
199	0\\
200	0\\
201	0\\
202	0\\
203	0\\
204	0\\
205	0\\
206	0\\
207	0\\
208	0\\
209	0\\
210	0\\
211	0\\
212	0\\
213	0\\
214	0\\
215	0\\
216	0\\
217	0\\
218	0\\
219	0\\
220	0\\
221	0\\
222	0\\
223	0\\
224	0\\
225	0\\
226	0\\
227	0\\
228	0\\
229	0\\
230	0\\
231	0\\
232	0\\
233	0\\
234	0\\
235	0\\
236	0\\
237	0\\
238	0\\
239	0\\
240	0\\
241	0\\
242	0\\
243	0\\
244	0\\
245	0\\
246	0\\
247	0\\
248	0\\
249	0\\
250	0\\
251	0\\
252	0\\
253	0\\
254	0\\
255	0\\
256	0\\
257	0\\
258	0\\
259	0\\
260	0\\
261	0\\
262	0\\
263	0\\
264	0\\
265	0\\
266	0\\
267	0\\
268	0\\
269	0\\
270	0\\
271	0\\
272	0\\
273	0\\
274	0\\
275	0\\
276	0\\
277	0\\
278	0\\
279	0\\
280	0\\
281	0\\
282	0\\
283	0\\
284	0\\
285	0\\
286	0\\
287	0\\
288	0\\
289	0\\
290	0\\
291	0\\
292	0\\
293	0\\
294	0\\
295	0\\
296	0\\
297	0\\
298	0\\
299	0\\
300	0\\
301	0\\
302	0\\
303	0\\
304	0\\
305	0\\
306	0\\
307	0\\
308	0\\
309	0\\
310	0\\
311	0\\
312	0\\
313	0\\
314	0\\
315	0\\
316	0\\
317	0\\
318	0\\
319	0\\
320	0\\
321	0\\
322	0\\
323	0\\
324	0\\
325	0\\
326	0\\
327	0\\
328	0\\
329	0\\
330	0\\
331	0\\
332	0\\
333	0\\
334	0\\
335	0\\
336	0\\
337	0\\
338	0\\
339	0\\
340	0\\
341	0\\
342	0\\
343	0\\
344	0\\
345	0\\
346	0\\
347	0\\
348	0\\
349	0\\
350	0\\
351	0\\
352	0\\
353	0\\
354	0\\
355	0\\
356	0\\
357	0\\
358	0\\
359	0\\
360	0\\
361	0\\
362	0\\
363	0\\
364	0\\
365	0\\
366	0\\
367	0\\
368	0\\
369	0\\
370	0\\
371	0\\
372	0\\
373	0\\
374	0\\
375	0\\
376	0\\
377	0\\
378	0\\
379	0\\
380	0\\
381	0\\
382	0\\
383	0\\
384	0\\
385	0\\
386	0\\
387	0\\
388	0\\
389	0\\
390	0\\
391	0\\
392	0\\
393	0\\
394	0\\
395	0\\
396	0\\
397	0\\
398	0\\
399	0\\
400	0\\
401	0\\
402	0\\
403	0\\
404	0\\
405	0\\
406	0\\
407	0\\
408	0\\
409	0\\
410	0\\
411	0\\
412	0\\
413	0\\
414	0\\
415	0\\
416	0\\
417	0\\
418	0\\
419	0\\
420	0\\
421	0\\
422	0\\
423	0\\
424	0\\
425	0\\
426	0\\
427	0\\
428	0\\
429	0\\
430	0\\
431	0\\
432	0\\
433	0\\
434	0\\
435	0\\
436	0\\
437	0\\
438	0\\
439	0\\
440	0\\
441	0\\
442	0\\
443	0\\
444	0\\
445	0\\
446	0\\
447	0\\
448	0\\
449	0\\
450	0\\
451	0\\
452	0\\
453	0\\
454	0\\
455	0\\
456	0\\
457	0\\
458	0\\
459	0\\
460	0\\
461	0\\
462	0\\
463	0\\
464	0\\
465	0\\
466	0\\
467	0\\
468	0\\
469	0\\
470	0\\
471	0\\
472	0\\
473	0\\
474	0\\
475	0\\
476	0\\
477	0\\
478	0\\
479	0\\
480	0\\
481	0\\
482	0\\
483	0\\
484	0\\
485	0\\
486	0\\
487	0\\
488	0\\
489	0\\
490	0\\
491	0\\
492	0\\
493	0\\
494	0\\
495	0\\
496	0\\
497	0\\
498	0\\
499	0\\
500	0\\
501	0\\
502	0\\
503	0\\
504	0\\
505	0\\
506	0\\
507	0\\
508	0\\
509	0\\
510	0\\
511	0\\
512	0\\
513	0\\
514	0\\
515	0\\
516	0\\
517	0\\
518	0\\
519	0\\
520	0\\
521	0\\
522	0\\
523	0\\
524	0\\
525	0\\
526	0\\
527	0\\
528	0\\
529	0\\
530	0\\
531	0\\
532	0\\
533	0\\
534	0\\
535	0\\
536	0\\
537	0\\
538	0\\
539	0\\
540	0\\
541	0\\
542	0\\
543	0\\
544	0\\
545	0\\
546	0\\
547	0\\
548	0\\
549	0\\
550	0\\
551	0\\
552	0\\
553	0\\
554	0\\
555	0\\
556	0\\
557	0\\
558	0\\
559	0\\
560	0\\
561	0\\
562	0\\
563	0\\
564	0\\
565	0\\
566	0\\
567	0\\
568	0\\
569	0\\
570	0\\
571	0\\
572	0\\
573	0\\
574	0\\
575	0\\
576	0\\
577	0\\
578	0\\
579	0\\
580	0\\
581	0\\
582	0\\
583	0\\
584	0\\
585	0\\
586	0\\
587	0\\
588	0\\
589	0\\
590	0\\
591	0\\
592	0\\
593	0\\
594	0\\
595	0\\
596	0\\
597	0\\
598	0\\
599	0\\
600	0\\
};
\addplot [color=mycolor19,solid,forget plot]
  table[row sep=crcr]{%
1	0\\
2	0\\
3	0\\
4	0\\
5	0\\
6	0\\
7	0\\
8	0\\
9	0\\
10	0\\
11	0\\
12	0\\
13	0\\
14	0\\
15	0\\
16	0\\
17	0\\
18	0\\
19	0\\
20	0\\
21	0\\
22	0\\
23	0\\
24	0\\
25	0\\
26	0\\
27	0\\
28	0\\
29	0\\
30	0\\
31	0\\
32	0\\
33	0\\
34	0\\
35	0\\
36	0\\
37	0\\
38	0\\
39	0\\
40	0\\
41	0\\
42	0\\
43	0\\
44	0\\
45	0\\
46	0\\
47	0\\
48	0\\
49	0\\
50	0\\
51	0\\
52	0\\
53	0\\
54	0\\
55	0\\
56	0\\
57	0\\
58	0\\
59	0\\
60	0\\
61	0\\
62	0\\
63	0\\
64	0\\
65	0\\
66	0\\
67	0\\
68	0\\
69	0\\
70	0\\
71	0\\
72	0\\
73	0\\
74	0\\
75	0\\
76	0\\
77	0\\
78	0\\
79	0\\
80	0\\
81	0\\
82	0\\
83	0\\
84	0\\
85	0\\
86	0\\
87	0\\
88	0\\
89	0\\
90	0\\
91	0\\
92	0\\
93	0\\
94	0\\
95	0\\
96	0\\
97	0\\
98	0\\
99	0\\
100	0\\
101	0\\
102	0\\
103	0\\
104	0\\
105	0\\
106	0\\
107	0\\
108	0\\
109	0\\
110	0\\
111	0\\
112	0\\
113	0\\
114	0\\
115	0\\
116	0\\
117	0\\
118	0\\
119	0\\
120	0\\
121	0\\
122	0\\
123	0\\
124	0\\
125	0\\
126	0\\
127	0\\
128	0\\
129	0\\
130	0\\
131	0\\
132	0\\
133	0\\
134	0\\
135	0\\
136	0\\
137	0\\
138	0\\
139	0\\
140	0\\
141	0\\
142	0\\
143	0\\
144	0\\
145	0\\
146	0\\
147	0\\
148	0\\
149	0\\
150	0\\
151	0\\
152	0\\
153	0\\
154	0\\
155	0\\
156	0\\
157	0\\
158	0\\
159	0\\
160	0\\
161	0\\
162	0\\
163	0\\
164	0\\
165	0\\
166	0\\
167	0\\
168	0\\
169	0\\
170	0\\
171	0\\
172	0\\
173	0\\
174	0\\
175	0\\
176	0\\
177	0\\
178	0\\
179	0\\
180	0\\
181	0\\
182	0\\
183	0\\
184	0\\
185	0\\
186	0\\
187	0\\
188	0\\
189	0\\
190	0\\
191	0\\
192	0\\
193	0\\
194	0\\
195	0\\
196	0\\
197	0\\
198	0\\
199	0\\
200	0\\
201	0\\
202	0\\
203	0\\
204	0\\
205	0\\
206	0\\
207	0\\
208	0\\
209	0\\
210	0\\
211	0\\
212	0\\
213	0\\
214	0\\
215	0\\
216	0\\
217	0\\
218	0\\
219	0\\
220	0\\
221	0\\
222	0\\
223	0\\
224	0\\
225	0\\
226	0\\
227	0\\
228	0\\
229	0\\
230	0\\
231	0\\
232	0\\
233	0\\
234	0\\
235	0\\
236	0\\
237	0\\
238	0\\
239	0\\
240	0\\
241	0\\
242	0\\
243	0\\
244	0\\
245	0\\
246	0\\
247	0\\
248	0\\
249	0\\
250	0\\
251	0\\
252	0\\
253	0\\
254	0\\
255	0\\
256	0\\
257	0\\
258	0\\
259	0\\
260	0\\
261	0\\
262	0\\
263	0\\
264	0\\
265	0\\
266	0\\
267	0\\
268	0\\
269	0\\
270	0\\
271	0\\
272	0\\
273	0\\
274	0\\
275	0\\
276	0\\
277	0\\
278	0\\
279	0\\
280	0\\
281	0\\
282	0\\
283	0\\
284	0\\
285	0\\
286	0\\
287	0\\
288	0\\
289	0\\
290	0\\
291	0\\
292	0\\
293	0\\
294	0\\
295	0\\
296	0\\
297	0\\
298	0\\
299	0\\
300	0\\
301	0\\
302	0\\
303	0\\
304	0\\
305	0\\
306	0\\
307	0\\
308	0\\
309	0\\
310	0\\
311	0\\
312	0\\
313	0\\
314	0\\
315	0\\
316	0\\
317	0\\
318	0\\
319	0\\
320	0\\
321	0\\
322	0\\
323	0\\
324	0\\
325	0\\
326	0\\
327	0\\
328	0\\
329	0\\
330	0\\
331	0\\
332	0\\
333	0\\
334	0\\
335	0\\
336	0\\
337	0\\
338	0\\
339	0\\
340	0\\
341	0\\
342	0\\
343	0\\
344	0\\
345	0\\
346	0\\
347	0\\
348	0\\
349	0\\
350	0\\
351	0\\
352	0\\
353	0\\
354	0\\
355	0\\
356	0\\
357	0\\
358	0\\
359	0\\
360	0\\
361	0\\
362	0\\
363	0\\
364	0\\
365	0\\
366	0\\
367	0\\
368	0\\
369	0\\
370	0\\
371	0\\
372	0\\
373	0\\
374	0\\
375	0\\
376	0\\
377	0\\
378	0\\
379	0\\
380	0\\
381	0\\
382	0\\
383	0\\
384	0\\
385	0\\
386	0\\
387	0\\
388	0\\
389	0\\
390	0\\
391	0\\
392	0\\
393	0\\
394	0\\
395	0\\
396	0\\
397	0\\
398	0\\
399	0\\
400	0\\
401	0\\
402	0\\
403	0\\
404	0\\
405	0\\
406	0\\
407	0\\
408	0\\
409	0\\
410	0\\
411	0\\
412	0\\
413	0\\
414	0\\
415	0\\
416	0\\
417	0\\
418	0\\
419	0\\
420	0\\
421	0\\
422	0\\
423	0\\
424	0\\
425	0\\
426	0\\
427	0\\
428	0\\
429	0\\
430	0\\
431	0\\
432	0\\
433	0\\
434	0\\
435	0\\
436	0\\
437	0\\
438	0\\
439	0\\
440	0\\
441	0\\
442	0\\
443	0\\
444	0\\
445	0\\
446	0\\
447	0\\
448	0\\
449	0\\
450	0\\
451	0\\
452	0\\
453	0\\
454	0\\
455	0\\
456	0\\
457	0\\
458	0\\
459	0\\
460	0\\
461	0\\
462	0\\
463	0\\
464	0\\
465	0\\
466	0\\
467	0\\
468	0\\
469	0\\
470	0\\
471	0\\
472	0\\
473	0\\
474	0\\
475	0\\
476	0\\
477	0\\
478	0\\
479	0\\
480	0\\
481	0\\
482	0\\
483	0\\
484	0\\
485	0\\
486	0\\
487	0\\
488	0\\
489	0\\
490	0\\
491	0\\
492	0\\
493	0\\
494	0\\
495	0\\
496	0\\
497	0\\
498	0\\
499	0\\
500	0\\
501	0\\
502	0\\
503	0\\
504	0\\
505	0\\
506	0\\
507	0\\
508	0\\
509	0\\
510	0\\
511	0\\
512	0\\
513	0\\
514	0\\
515	0\\
516	0\\
517	0\\
518	0\\
519	0\\
520	0\\
521	0\\
522	0\\
523	0\\
524	0\\
525	0\\
526	0\\
527	0\\
528	0\\
529	0\\
530	0\\
531	0\\
532	0\\
533	0\\
534	0\\
535	0\\
536	0\\
537	0\\
538	0\\
539	0\\
540	0\\
541	0\\
542	0\\
543	0\\
544	0\\
545	0\\
546	0\\
547	0\\
548	0\\
549	0\\
550	0\\
551	0\\
552	0\\
553	0\\
554	0\\
555	0\\
556	0\\
557	0\\
558	0\\
559	0\\
560	0\\
561	0\\
562	0\\
563	0\\
564	0\\
565	0\\
566	0\\
567	0\\
568	0\\
569	0\\
570	0\\
571	0\\
572	0\\
573	0\\
574	0\\
575	0\\
576	0\\
577	0\\
578	0\\
579	0\\
580	0\\
581	0\\
582	0\\
583	0\\
584	0\\
585	0\\
586	0\\
587	0\\
588	0\\
589	0\\
590	0\\
591	0\\
592	0\\
593	0\\
594	0\\
595	0\\
596	0\\
597	0\\
598	0\\
599	0\\
600	0\\
};
\addplot [color=red!50!mycolor17,solid,forget plot]
  table[row sep=crcr]{%
1	0\\
2	0\\
3	0\\
4	0\\
5	0\\
6	0\\
7	0\\
8	0\\
9	0\\
10	0\\
11	0\\
12	0\\
13	0\\
14	0\\
15	0\\
16	0\\
17	0\\
18	0\\
19	0\\
20	0\\
21	0\\
22	0\\
23	0\\
24	0\\
25	0\\
26	0\\
27	0\\
28	0\\
29	0\\
30	0\\
31	0\\
32	0\\
33	0\\
34	0\\
35	0\\
36	0\\
37	0\\
38	0\\
39	0\\
40	0\\
41	0\\
42	0\\
43	0\\
44	0\\
45	0\\
46	0\\
47	0\\
48	0\\
49	0\\
50	0\\
51	0\\
52	0\\
53	0\\
54	0\\
55	0\\
56	0\\
57	0\\
58	0\\
59	0\\
60	0\\
61	0\\
62	0\\
63	0\\
64	0\\
65	0\\
66	0\\
67	0\\
68	0\\
69	0\\
70	0\\
71	0\\
72	0\\
73	0\\
74	0\\
75	0\\
76	0\\
77	0\\
78	0\\
79	0\\
80	0\\
81	0\\
82	0\\
83	0\\
84	0\\
85	0\\
86	0\\
87	0\\
88	0\\
89	0\\
90	0\\
91	0\\
92	0\\
93	0\\
94	0\\
95	0\\
96	0\\
97	0\\
98	0\\
99	0\\
100	0\\
101	0\\
102	0\\
103	0\\
104	0\\
105	0\\
106	0\\
107	0\\
108	0\\
109	0\\
110	0\\
111	0\\
112	0\\
113	0\\
114	0\\
115	0\\
116	0\\
117	0\\
118	0\\
119	0\\
120	0\\
121	0\\
122	0\\
123	0\\
124	0\\
125	0\\
126	0\\
127	0\\
128	0\\
129	0\\
130	0\\
131	0\\
132	0\\
133	0\\
134	0\\
135	0\\
136	0\\
137	0\\
138	0\\
139	0\\
140	0\\
141	0\\
142	0\\
143	0\\
144	0\\
145	0\\
146	0\\
147	0\\
148	0\\
149	0\\
150	0\\
151	0\\
152	0\\
153	0\\
154	0\\
155	0\\
156	0\\
157	0\\
158	0\\
159	0\\
160	0\\
161	0\\
162	0\\
163	0\\
164	0\\
165	0\\
166	0\\
167	0\\
168	0\\
169	0\\
170	0\\
171	0\\
172	0\\
173	0\\
174	0\\
175	0\\
176	0\\
177	0\\
178	0\\
179	0\\
180	0\\
181	0\\
182	0\\
183	0\\
184	0\\
185	0\\
186	0\\
187	0\\
188	0\\
189	0\\
190	0\\
191	0\\
192	0\\
193	0\\
194	0\\
195	0\\
196	0\\
197	0\\
198	0\\
199	0\\
200	0\\
201	0\\
202	0\\
203	0\\
204	0\\
205	0\\
206	0\\
207	0\\
208	0\\
209	0\\
210	0\\
211	0\\
212	0\\
213	0\\
214	0\\
215	0\\
216	0\\
217	0\\
218	0\\
219	0\\
220	0\\
221	0\\
222	0\\
223	0\\
224	0\\
225	0\\
226	0\\
227	0\\
228	0\\
229	0\\
230	0\\
231	0\\
232	0\\
233	0\\
234	0\\
235	0\\
236	0\\
237	0\\
238	0\\
239	0\\
240	0\\
241	0\\
242	0\\
243	0\\
244	0\\
245	0\\
246	0\\
247	0\\
248	0\\
249	0\\
250	0\\
251	0\\
252	0\\
253	0\\
254	0\\
255	0\\
256	0\\
257	0\\
258	0\\
259	0\\
260	0\\
261	0\\
262	0\\
263	0\\
264	0\\
265	0\\
266	0\\
267	0\\
268	0\\
269	0\\
270	0\\
271	0\\
272	0\\
273	0\\
274	0\\
275	0\\
276	0\\
277	0\\
278	0\\
279	0\\
280	0\\
281	0\\
282	0\\
283	0\\
284	0\\
285	0\\
286	0\\
287	0\\
288	0\\
289	0\\
290	0\\
291	0\\
292	0\\
293	0\\
294	0\\
295	0\\
296	0\\
297	0\\
298	0\\
299	0\\
300	0\\
301	0\\
302	0\\
303	0\\
304	0\\
305	0\\
306	0\\
307	0\\
308	0\\
309	0\\
310	0\\
311	0\\
312	0\\
313	0\\
314	0\\
315	0\\
316	0\\
317	0\\
318	0\\
319	0\\
320	0\\
321	0\\
322	0\\
323	0\\
324	0\\
325	0\\
326	0\\
327	0\\
328	0\\
329	0\\
330	0\\
331	0\\
332	0\\
333	0\\
334	0\\
335	0\\
336	0\\
337	0\\
338	0\\
339	0\\
340	0\\
341	0\\
342	0\\
343	0\\
344	0\\
345	0\\
346	0\\
347	0\\
348	0\\
349	0\\
350	0\\
351	0\\
352	0\\
353	0\\
354	0\\
355	0\\
356	0\\
357	0\\
358	0\\
359	0\\
360	0\\
361	0\\
362	0\\
363	0\\
364	0\\
365	0\\
366	0\\
367	0\\
368	0\\
369	0\\
370	0\\
371	0\\
372	0\\
373	0\\
374	0\\
375	0\\
376	0\\
377	0\\
378	0\\
379	0\\
380	0\\
381	0\\
382	0\\
383	0\\
384	0\\
385	0\\
386	0\\
387	0\\
388	0\\
389	0\\
390	0\\
391	0\\
392	0\\
393	0\\
394	0\\
395	0\\
396	0\\
397	0\\
398	0\\
399	0\\
400	0\\
401	0\\
402	0\\
403	0\\
404	0\\
405	0\\
406	0\\
407	0\\
408	0\\
409	0\\
410	0\\
411	0\\
412	0\\
413	0\\
414	0\\
415	0\\
416	0\\
417	0\\
418	0\\
419	0\\
420	0\\
421	0\\
422	0\\
423	0\\
424	0\\
425	0\\
426	0\\
427	0\\
428	0\\
429	0\\
430	0\\
431	0\\
432	0\\
433	0\\
434	0\\
435	0\\
436	0\\
437	0\\
438	0\\
439	0\\
440	0\\
441	0\\
442	0\\
443	0\\
444	0\\
445	0\\
446	0\\
447	0\\
448	0\\
449	0\\
450	0\\
451	0\\
452	0\\
453	0\\
454	0\\
455	0\\
456	0\\
457	0\\
458	0\\
459	0\\
460	0\\
461	0\\
462	0\\
463	0\\
464	0\\
465	0\\
466	0\\
467	0\\
468	0\\
469	0\\
470	0\\
471	0\\
472	0\\
473	0\\
474	0\\
475	0\\
476	0\\
477	0\\
478	0\\
479	0\\
480	0\\
481	0\\
482	0\\
483	0\\
484	0\\
485	0\\
486	0\\
487	0\\
488	0\\
489	0\\
490	0\\
491	0\\
492	0\\
493	0\\
494	0\\
495	0\\
496	0\\
497	0\\
498	0\\
499	0\\
500	0\\
501	0\\
502	0\\
503	0\\
504	0\\
505	0\\
506	0\\
507	0\\
508	0\\
509	0\\
510	0\\
511	0\\
512	0\\
513	0\\
514	0\\
515	0\\
516	0\\
517	0\\
518	0\\
519	0\\
520	0\\
521	0\\
522	0\\
523	0\\
524	0\\
525	0\\
526	0\\
527	0\\
528	0\\
529	0\\
530	0\\
531	0\\
532	0\\
533	0\\
534	0\\
535	0\\
536	0\\
537	0\\
538	0\\
539	0\\
540	0\\
541	0\\
542	0\\
543	0\\
544	0\\
545	0\\
546	0\\
547	0\\
548	0\\
549	0\\
550	0\\
551	0\\
552	0\\
553	0\\
554	0\\
555	0\\
556	0\\
557	0\\
558	0\\
559	0\\
560	0\\
561	0\\
562	0\\
563	0\\
564	0\\
565	0\\
566	0\\
567	0\\
568	0\\
569	0\\
570	0\\
571	0\\
572	0\\
573	0\\
574	0\\
575	0\\
576	0\\
577	0\\
578	0\\
579	0\\
580	0\\
581	0\\
582	0\\
583	0\\
584	0\\
585	0\\
586	0\\
587	0\\
588	0\\
589	0\\
590	0\\
591	0\\
592	0\\
593	0\\
594	0\\
595	0\\
596	0\\
597	0\\
598	0\\
599	0\\
600	0\\
};
\addplot [color=red!40!mycolor19,solid,forget plot]
  table[row sep=crcr]{%
1	0\\
2	0\\
3	0\\
4	0\\
5	0\\
6	0\\
7	0\\
8	0\\
9	0\\
10	0\\
11	0\\
12	0\\
13	0\\
14	0\\
15	0\\
16	0\\
17	0\\
18	0\\
19	0\\
20	0\\
21	0\\
22	0\\
23	0\\
24	0\\
25	0\\
26	0\\
27	0\\
28	0\\
29	0\\
30	0\\
31	0\\
32	0\\
33	0\\
34	0\\
35	0\\
36	0\\
37	0\\
38	0\\
39	0\\
40	0\\
41	0\\
42	0\\
43	0\\
44	0\\
45	0\\
46	0\\
47	0\\
48	0\\
49	0\\
50	0\\
51	0\\
52	0\\
53	0\\
54	0\\
55	0\\
56	0\\
57	0\\
58	0\\
59	0\\
60	0\\
61	0\\
62	0\\
63	0\\
64	0\\
65	0\\
66	0\\
67	0\\
68	0\\
69	0\\
70	0\\
71	0\\
72	0\\
73	0\\
74	0\\
75	0\\
76	0\\
77	0\\
78	0\\
79	0\\
80	0\\
81	0\\
82	0\\
83	0\\
84	0\\
85	0\\
86	0\\
87	0\\
88	0\\
89	0\\
90	0\\
91	0\\
92	0\\
93	0\\
94	0\\
95	0\\
96	0\\
97	0\\
98	0\\
99	0\\
100	0\\
101	0\\
102	0\\
103	0\\
104	0\\
105	0\\
106	0\\
107	0\\
108	0\\
109	0\\
110	0\\
111	0\\
112	0\\
113	0\\
114	0\\
115	0\\
116	0\\
117	0\\
118	0\\
119	0\\
120	0\\
121	0\\
122	0\\
123	0\\
124	0\\
125	0\\
126	0\\
127	0\\
128	0\\
129	0\\
130	0\\
131	0\\
132	0\\
133	0\\
134	0\\
135	0\\
136	0\\
137	0\\
138	0\\
139	0\\
140	0\\
141	0\\
142	0\\
143	0\\
144	0\\
145	0\\
146	0\\
147	0\\
148	0\\
149	0\\
150	0\\
151	0\\
152	0\\
153	0\\
154	0\\
155	0\\
156	0\\
157	0\\
158	0\\
159	0\\
160	0\\
161	0\\
162	0\\
163	0\\
164	0\\
165	0\\
166	0\\
167	0\\
168	0\\
169	0\\
170	0\\
171	0\\
172	0\\
173	0\\
174	0\\
175	0\\
176	0\\
177	0\\
178	0\\
179	0\\
180	0\\
181	0\\
182	0\\
183	0\\
184	0\\
185	0\\
186	0\\
187	0\\
188	0\\
189	0\\
190	0\\
191	0\\
192	0\\
193	0\\
194	0\\
195	0\\
196	0\\
197	0\\
198	0\\
199	0\\
200	0\\
201	0\\
202	0\\
203	0\\
204	0\\
205	0\\
206	0\\
207	0\\
208	0\\
209	0\\
210	0\\
211	0\\
212	0\\
213	0\\
214	0\\
215	0\\
216	0\\
217	0\\
218	0\\
219	0\\
220	0\\
221	0\\
222	0\\
223	0\\
224	0\\
225	0\\
226	0\\
227	0\\
228	0\\
229	0\\
230	0\\
231	0\\
232	0\\
233	0\\
234	0\\
235	0\\
236	0\\
237	0\\
238	0\\
239	0\\
240	0\\
241	0\\
242	0\\
243	0\\
244	0\\
245	0\\
246	0\\
247	0\\
248	0\\
249	0\\
250	0\\
251	0\\
252	0\\
253	0\\
254	0\\
255	0\\
256	0\\
257	0\\
258	0\\
259	0\\
260	0\\
261	0\\
262	0\\
263	0\\
264	0\\
265	0\\
266	0\\
267	0\\
268	0\\
269	0\\
270	0\\
271	0\\
272	0\\
273	0\\
274	0\\
275	0\\
276	0\\
277	0\\
278	0\\
279	0\\
280	0\\
281	0\\
282	0\\
283	0\\
284	0\\
285	0\\
286	0\\
287	0\\
288	0\\
289	0\\
290	0\\
291	0\\
292	0\\
293	0\\
294	0\\
295	0\\
296	0\\
297	0\\
298	0\\
299	0\\
300	0\\
301	0\\
302	0\\
303	0\\
304	0\\
305	0\\
306	0\\
307	0\\
308	0\\
309	0\\
310	0\\
311	0\\
312	0\\
313	0\\
314	0\\
315	0\\
316	0\\
317	0\\
318	0\\
319	0\\
320	0\\
321	0\\
322	0\\
323	0\\
324	0\\
325	0\\
326	0\\
327	0\\
328	0\\
329	0\\
330	0\\
331	0\\
332	0\\
333	0\\
334	0\\
335	0\\
336	0\\
337	0\\
338	0\\
339	0\\
340	0\\
341	0\\
342	0\\
343	0\\
344	0\\
345	0\\
346	0\\
347	0\\
348	0\\
349	0\\
350	0\\
351	0\\
352	0\\
353	0\\
354	0\\
355	0\\
356	0\\
357	0\\
358	0\\
359	0\\
360	0\\
361	0\\
362	0\\
363	0\\
364	0\\
365	0\\
366	0\\
367	0\\
368	0\\
369	0\\
370	0\\
371	0\\
372	0\\
373	0\\
374	0\\
375	0\\
376	0\\
377	0\\
378	0\\
379	0\\
380	0\\
381	0\\
382	0\\
383	0\\
384	0\\
385	0\\
386	0\\
387	0\\
388	0\\
389	0\\
390	0\\
391	0\\
392	0\\
393	0\\
394	0\\
395	0\\
396	0\\
397	0\\
398	0\\
399	0\\
400	0\\
401	0\\
402	0\\
403	0\\
404	0\\
405	0\\
406	0\\
407	0\\
408	0\\
409	0\\
410	0\\
411	0\\
412	0\\
413	0\\
414	0\\
415	0\\
416	0\\
417	0\\
418	0\\
419	0\\
420	0\\
421	0\\
422	0\\
423	0\\
424	0\\
425	0\\
426	0\\
427	0\\
428	0\\
429	0\\
430	0\\
431	0\\
432	0\\
433	0\\
434	0\\
435	0\\
436	0\\
437	0\\
438	0\\
439	0\\
440	0\\
441	0\\
442	0\\
443	0\\
444	0\\
445	0\\
446	0\\
447	0\\
448	0\\
449	0\\
450	0\\
451	0\\
452	0\\
453	0\\
454	0\\
455	0\\
456	0\\
457	0\\
458	0\\
459	0\\
460	0\\
461	0\\
462	0\\
463	0\\
464	0\\
465	0\\
466	0\\
467	0\\
468	0\\
469	0\\
470	0\\
471	0\\
472	0\\
473	0\\
474	0\\
475	0\\
476	0\\
477	0\\
478	0\\
479	0\\
480	0\\
481	0\\
482	0\\
483	0\\
484	0\\
485	0\\
486	0\\
487	0\\
488	0\\
489	0\\
490	0\\
491	0\\
492	0\\
493	0\\
494	0\\
495	0\\
496	0\\
497	0\\
498	0\\
499	0\\
500	0\\
501	0\\
502	0\\
503	0\\
504	0\\
505	0\\
506	0\\
507	0\\
508	0\\
509	0\\
510	0\\
511	0\\
512	0\\
513	0\\
514	0\\
515	0\\
516	0\\
517	0\\
518	0\\
519	0\\
520	0\\
521	0\\
522	0\\
523	0\\
524	0\\
525	0\\
526	0\\
527	0\\
528	0\\
529	0\\
530	0\\
531	0\\
532	0\\
533	0\\
534	0\\
535	0\\
536	0\\
537	0\\
538	0\\
539	0\\
540	0\\
541	0\\
542	0\\
543	0\\
544	0\\
545	0\\
546	0\\
547	0\\
548	0\\
549	0\\
550	0\\
551	0\\
552	0\\
553	0\\
554	0\\
555	0\\
556	0\\
557	0\\
558	0\\
559	0\\
560	0\\
561	0\\
562	0\\
563	0\\
564	0\\
565	0\\
566	0\\
567	0\\
568	0\\
569	0\\
570	0\\
571	0\\
572	0\\
573	0\\
574	0\\
575	0\\
576	0\\
577	0\\
578	0\\
579	0\\
580	0\\
581	0\\
582	0\\
583	0\\
584	0\\
585	0\\
586	0\\
587	0\\
588	0\\
589	0\\
590	0\\
591	0\\
592	0\\
593	0\\
594	0\\
595	0\\
596	0\\
597	0\\
598	0\\
599	0\\
600	0\\
};
\addplot [color=red!75!mycolor17,solid,forget plot]
  table[row sep=crcr]{%
1	0\\
2	0\\
3	0\\
4	0\\
5	0\\
6	0\\
7	0\\
8	0\\
9	0\\
10	0\\
11	0\\
12	0\\
13	0\\
14	0\\
15	0\\
16	0\\
17	0\\
18	0\\
19	0\\
20	0\\
21	0\\
22	0\\
23	0\\
24	0\\
25	0\\
26	0\\
27	0\\
28	0\\
29	0\\
30	0\\
31	0\\
32	0\\
33	0\\
34	0\\
35	0\\
36	0\\
37	0\\
38	0\\
39	0\\
40	0\\
41	0\\
42	0\\
43	0\\
44	0\\
45	0\\
46	0\\
47	0\\
48	0\\
49	0\\
50	0\\
51	0\\
52	0\\
53	0\\
54	0\\
55	0\\
56	0\\
57	0\\
58	0\\
59	0\\
60	0\\
61	0\\
62	0\\
63	0\\
64	0\\
65	0\\
66	0\\
67	0\\
68	0\\
69	0\\
70	0\\
71	0\\
72	0\\
73	0\\
74	0\\
75	0\\
76	0\\
77	0\\
78	0\\
79	0\\
80	0\\
81	0\\
82	0\\
83	0\\
84	0\\
85	0\\
86	0\\
87	0\\
88	0\\
89	0\\
90	0\\
91	0\\
92	0\\
93	0\\
94	0\\
95	0\\
96	0\\
97	0\\
98	0\\
99	0\\
100	0\\
101	0\\
102	0\\
103	0\\
104	0\\
105	0\\
106	0\\
107	0\\
108	0\\
109	0\\
110	0\\
111	0\\
112	0\\
113	0\\
114	0\\
115	0\\
116	0\\
117	0\\
118	0\\
119	0\\
120	0\\
121	0\\
122	0\\
123	0\\
124	0\\
125	0\\
126	0\\
127	0\\
128	0\\
129	0\\
130	0\\
131	0\\
132	0\\
133	0\\
134	0\\
135	0\\
136	0\\
137	0\\
138	0\\
139	0\\
140	0\\
141	0\\
142	0\\
143	0\\
144	0\\
145	0\\
146	0\\
147	0\\
148	0\\
149	0\\
150	0\\
151	0\\
152	0\\
153	0\\
154	0\\
155	0\\
156	0\\
157	0\\
158	0\\
159	0\\
160	0\\
161	0\\
162	0\\
163	0\\
164	0\\
165	0\\
166	0\\
167	0\\
168	0\\
169	0\\
170	0\\
171	0\\
172	0\\
173	0\\
174	0\\
175	0\\
176	0\\
177	0\\
178	0\\
179	0\\
180	0\\
181	0\\
182	0\\
183	0\\
184	0\\
185	0\\
186	0\\
187	0\\
188	0\\
189	0\\
190	0\\
191	0\\
192	0\\
193	0\\
194	0\\
195	0\\
196	0\\
197	0\\
198	0\\
199	0\\
200	0\\
201	0\\
202	0\\
203	0\\
204	0\\
205	0\\
206	0\\
207	0\\
208	0\\
209	0\\
210	0\\
211	0\\
212	0\\
213	0\\
214	0\\
215	0\\
216	0\\
217	0\\
218	0\\
219	0\\
220	0\\
221	0\\
222	0\\
223	0\\
224	0\\
225	0\\
226	0\\
227	0\\
228	0\\
229	0\\
230	0\\
231	0\\
232	0\\
233	0\\
234	0\\
235	0\\
236	0\\
237	0\\
238	0\\
239	0\\
240	0\\
241	0\\
242	0\\
243	0\\
244	0\\
245	0\\
246	0\\
247	0\\
248	0\\
249	0\\
250	0\\
251	0\\
252	0\\
253	0\\
254	0\\
255	0\\
256	0\\
257	0\\
258	0\\
259	0\\
260	0\\
261	0\\
262	0\\
263	0\\
264	0\\
265	0\\
266	0\\
267	0\\
268	0\\
269	0\\
270	0\\
271	0\\
272	0\\
273	0\\
274	0\\
275	0\\
276	0\\
277	0\\
278	0\\
279	0\\
280	0\\
281	0\\
282	0\\
283	0\\
284	0\\
285	0\\
286	0\\
287	0\\
288	0\\
289	0\\
290	0\\
291	0\\
292	0\\
293	0\\
294	0\\
295	0\\
296	0\\
297	0\\
298	0\\
299	0\\
300	0\\
301	0\\
302	0\\
303	0\\
304	0\\
305	0\\
306	0\\
307	0\\
308	0\\
309	0\\
310	0\\
311	0\\
312	0\\
313	0\\
314	0\\
315	0\\
316	0\\
317	0\\
318	0\\
319	0\\
320	0\\
321	0\\
322	0\\
323	0\\
324	0\\
325	0\\
326	0\\
327	0\\
328	0\\
329	0\\
330	0\\
331	0\\
332	0\\
333	0\\
334	0\\
335	0\\
336	0\\
337	0\\
338	0\\
339	0\\
340	0\\
341	0\\
342	0\\
343	0\\
344	0\\
345	0\\
346	0\\
347	0\\
348	0\\
349	0\\
350	0\\
351	0\\
352	0\\
353	0\\
354	0\\
355	0\\
356	0\\
357	0\\
358	0\\
359	0\\
360	0\\
361	0\\
362	0\\
363	0\\
364	0\\
365	0\\
366	0\\
367	0\\
368	0\\
369	0\\
370	0\\
371	0\\
372	0\\
373	0\\
374	0\\
375	0\\
376	0\\
377	0\\
378	0\\
379	0\\
380	0\\
381	0\\
382	0\\
383	0\\
384	0\\
385	0\\
386	0\\
387	0\\
388	0\\
389	0\\
390	0\\
391	0\\
392	0\\
393	0\\
394	0\\
395	0\\
396	0\\
397	0\\
398	0\\
399	0\\
400	0\\
401	0\\
402	0\\
403	0\\
404	0\\
405	0\\
406	0\\
407	0\\
408	0\\
409	0\\
410	0\\
411	0\\
412	0\\
413	0\\
414	0\\
415	0\\
416	0\\
417	0\\
418	0\\
419	0\\
420	0\\
421	0\\
422	0\\
423	0\\
424	0\\
425	0\\
426	0\\
427	0\\
428	0\\
429	0\\
430	0\\
431	0\\
432	0\\
433	0\\
434	0\\
435	0\\
436	0\\
437	0\\
438	0\\
439	0\\
440	0\\
441	0\\
442	0\\
443	0\\
444	0\\
445	0\\
446	0\\
447	0\\
448	0\\
449	0\\
450	0\\
451	0\\
452	0\\
453	0\\
454	0\\
455	0\\
456	0\\
457	0\\
458	0\\
459	0\\
460	0\\
461	0\\
462	0\\
463	0\\
464	0\\
465	0\\
466	0\\
467	0\\
468	0\\
469	0\\
470	0\\
471	0\\
472	0\\
473	0\\
474	0\\
475	0\\
476	0\\
477	0\\
478	0\\
479	0\\
480	0\\
481	0\\
482	0\\
483	0\\
484	0\\
485	0\\
486	0\\
487	0\\
488	0\\
489	0\\
490	0\\
491	0\\
492	0\\
493	0\\
494	0\\
495	0\\
496	0\\
497	0\\
498	0\\
499	0\\
500	0\\
501	0\\
502	0\\
503	0\\
504	0\\
505	0\\
506	0\\
507	0\\
508	0\\
509	0\\
510	0\\
511	0\\
512	0\\
513	0\\
514	0\\
515	0\\
516	0\\
517	0\\
518	0\\
519	0\\
520	0\\
521	0\\
522	0\\
523	0\\
524	0\\
525	0\\
526	0\\
527	0\\
528	0\\
529	0\\
530	0\\
531	0\\
532	0\\
533	0\\
534	0\\
535	0\\
536	0\\
537	0\\
538	0\\
539	0\\
540	0\\
541	0\\
542	0\\
543	0\\
544	0\\
545	0\\
546	0\\
547	0\\
548	0\\
549	0\\
550	0\\
551	0\\
552	0\\
553	0\\
554	0\\
555	0\\
556	0\\
557	0\\
558	0\\
559	0\\
560	0\\
561	0\\
562	0\\
563	0\\
564	0\\
565	0\\
566	0\\
567	0\\
568	0\\
569	0\\
570	0\\
571	0\\
572	0\\
573	0\\
574	0\\
575	0\\
576	0\\
577	0\\
578	0\\
579	0\\
580	0\\
581	0\\
582	0\\
583	0\\
584	0\\
585	0\\
586	0\\
587	0\\
588	0\\
589	0\\
590	0\\
591	0\\
592	0\\
593	0\\
594	0\\
595	0\\
596	0\\
597	0\\
598	0\\
599	0\\
600	0\\
};
\addplot [color=red!80!mycolor19,solid,forget plot]
  table[row sep=crcr]{%
1	0\\
2	0\\
3	0\\
4	0\\
5	0\\
6	0\\
7	0\\
8	0\\
9	0\\
10	0\\
11	0\\
12	0\\
13	0\\
14	0\\
15	0\\
16	0\\
17	0\\
18	0\\
19	0\\
20	0\\
21	0\\
22	0\\
23	0\\
24	0\\
25	0\\
26	0\\
27	0\\
28	0\\
29	0\\
30	0\\
31	0\\
32	0\\
33	0\\
34	0\\
35	0\\
36	0\\
37	0\\
38	0\\
39	0\\
40	0\\
41	0\\
42	0\\
43	0\\
44	0\\
45	0\\
46	0\\
47	0\\
48	0\\
49	0\\
50	0\\
51	0\\
52	0\\
53	0\\
54	0\\
55	0\\
56	0\\
57	0\\
58	0\\
59	0\\
60	0\\
61	0\\
62	0\\
63	0\\
64	0\\
65	0\\
66	0\\
67	0\\
68	0\\
69	0\\
70	0\\
71	0\\
72	0\\
73	0\\
74	0\\
75	0\\
76	0\\
77	0\\
78	0\\
79	0\\
80	0\\
81	0\\
82	0\\
83	0\\
84	0\\
85	0\\
86	0\\
87	0\\
88	0\\
89	0\\
90	0\\
91	0\\
92	0\\
93	0\\
94	0\\
95	0\\
96	0\\
97	0\\
98	0\\
99	0\\
100	0\\
101	0\\
102	0\\
103	0\\
104	0\\
105	0\\
106	0\\
107	0\\
108	0\\
109	0\\
110	0\\
111	0\\
112	0\\
113	0\\
114	0\\
115	0\\
116	0\\
117	0\\
118	0\\
119	0\\
120	0\\
121	0\\
122	0\\
123	0\\
124	0\\
125	0\\
126	0\\
127	0\\
128	0\\
129	0\\
130	0\\
131	0\\
132	0\\
133	0\\
134	0\\
135	0\\
136	0\\
137	0\\
138	0\\
139	0\\
140	0\\
141	0\\
142	0\\
143	0\\
144	0\\
145	0\\
146	0\\
147	0\\
148	0\\
149	0\\
150	0\\
151	0\\
152	0\\
153	0\\
154	0\\
155	0\\
156	0\\
157	0\\
158	0\\
159	0\\
160	0\\
161	0\\
162	0\\
163	0\\
164	0\\
165	0\\
166	0\\
167	0\\
168	0\\
169	0\\
170	0\\
171	0\\
172	0\\
173	0\\
174	0\\
175	0\\
176	0\\
177	0\\
178	0\\
179	0\\
180	0\\
181	0\\
182	0\\
183	0\\
184	0\\
185	0\\
186	0\\
187	0\\
188	0\\
189	0\\
190	0\\
191	0\\
192	0\\
193	0\\
194	0\\
195	0\\
196	0\\
197	0\\
198	0\\
199	0\\
200	0\\
201	0\\
202	0\\
203	0\\
204	0\\
205	0\\
206	0\\
207	0\\
208	0\\
209	0\\
210	0\\
211	0\\
212	0\\
213	0\\
214	0\\
215	0\\
216	0\\
217	0\\
218	0\\
219	0\\
220	0\\
221	0\\
222	0\\
223	0\\
224	0\\
225	0\\
226	0\\
227	0\\
228	0\\
229	0\\
230	0\\
231	0\\
232	0\\
233	0\\
234	0\\
235	0\\
236	0\\
237	0\\
238	0\\
239	0\\
240	0\\
241	0\\
242	0\\
243	0\\
244	0\\
245	0\\
246	0\\
247	0\\
248	0\\
249	0\\
250	0\\
251	0\\
252	0\\
253	0\\
254	0\\
255	0\\
256	0\\
257	0\\
258	0\\
259	0\\
260	0\\
261	0\\
262	0\\
263	0\\
264	0\\
265	0\\
266	0\\
267	0\\
268	0\\
269	0\\
270	0\\
271	0\\
272	0\\
273	0\\
274	0\\
275	0\\
276	0\\
277	0\\
278	0\\
279	0\\
280	0\\
281	0\\
282	0\\
283	0\\
284	0\\
285	0\\
286	0\\
287	0\\
288	0\\
289	0\\
290	0\\
291	0\\
292	0\\
293	0\\
294	0\\
295	0\\
296	0\\
297	0\\
298	0\\
299	0\\
300	0\\
301	0\\
302	0\\
303	0\\
304	0\\
305	0\\
306	0\\
307	0\\
308	0\\
309	0\\
310	0\\
311	0\\
312	0\\
313	0\\
314	0\\
315	0\\
316	0\\
317	0\\
318	0\\
319	0\\
320	0\\
321	0\\
322	0\\
323	0\\
324	0\\
325	0\\
326	0\\
327	0\\
328	0\\
329	0\\
330	0\\
331	0\\
332	0\\
333	0\\
334	0\\
335	0\\
336	0\\
337	0\\
338	0\\
339	0\\
340	0\\
341	0\\
342	0\\
343	0\\
344	0\\
345	0\\
346	0\\
347	0\\
348	0\\
349	0\\
350	0\\
351	0\\
352	0\\
353	0\\
354	0\\
355	0\\
356	0\\
357	0\\
358	0\\
359	0\\
360	0\\
361	0\\
362	0\\
363	0\\
364	0\\
365	0\\
366	0\\
367	0\\
368	0\\
369	0\\
370	0\\
371	0\\
372	0\\
373	0\\
374	0\\
375	0\\
376	0\\
377	0\\
378	0\\
379	0\\
380	0\\
381	0\\
382	0\\
383	0\\
384	0\\
385	0\\
386	0\\
387	0\\
388	0\\
389	0\\
390	0\\
391	0\\
392	0\\
393	0\\
394	0\\
395	0\\
396	0\\
397	0\\
398	0\\
399	0\\
400	0\\
401	0\\
402	0\\
403	0\\
404	0\\
405	0\\
406	0\\
407	0\\
408	0\\
409	0\\
410	0\\
411	0\\
412	0\\
413	0\\
414	0\\
415	0\\
416	0\\
417	0\\
418	0\\
419	0\\
420	0\\
421	0\\
422	0\\
423	0\\
424	0\\
425	0\\
426	0\\
427	0\\
428	0\\
429	0\\
430	0\\
431	0\\
432	0\\
433	0\\
434	0\\
435	0\\
436	0\\
437	0\\
438	0\\
439	0\\
440	0\\
441	0\\
442	0\\
443	0\\
444	0\\
445	0\\
446	0\\
447	0\\
448	0\\
449	0\\
450	0\\
451	0\\
452	0\\
453	0\\
454	0\\
455	0\\
456	0\\
457	0\\
458	0\\
459	0\\
460	0\\
461	0\\
462	0\\
463	0\\
464	0\\
465	0\\
466	0\\
467	0\\
468	0\\
469	0\\
470	0\\
471	0\\
472	0\\
473	0\\
474	0\\
475	0\\
476	0\\
477	0\\
478	0\\
479	0\\
480	0\\
481	0\\
482	0\\
483	0\\
484	0\\
485	0\\
486	0\\
487	0\\
488	0\\
489	0\\
490	0\\
491	0\\
492	0\\
493	0\\
494	0\\
495	0\\
496	0\\
497	0\\
498	0\\
499	0\\
500	0\\
501	0\\
502	0\\
503	0\\
504	0\\
505	0\\
506	0\\
507	0\\
508	0\\
509	0\\
510	0\\
511	0\\
512	0\\
513	0\\
514	0\\
515	0\\
516	0\\
517	0\\
518	0\\
519	0\\
520	0\\
521	0\\
522	0\\
523	0\\
524	0\\
525	0\\
526	0\\
527	0\\
528	0\\
529	0\\
530	0\\
531	0\\
532	0\\
533	0\\
534	0\\
535	0\\
536	0\\
537	0\\
538	0\\
539	0\\
540	0\\
541	0\\
542	0\\
543	0\\
544	0\\
545	0\\
546	0\\
547	0\\
548	0\\
549	0\\
550	0\\
551	0\\
552	0\\
553	0\\
554	0\\
555	0\\
556	0\\
557	0\\
558	0\\
559	0\\
560	0\\
561	0\\
562	0\\
563	0\\
564	0\\
565	0\\
566	0\\
567	0\\
568	0\\
569	0\\
570	0\\
571	0\\
572	0\\
573	0\\
574	0\\
575	0\\
576	0\\
577	0\\
578	0\\
579	0\\
580	0\\
581	0\\
582	0\\
583	0\\
584	0\\
585	0\\
586	0\\
587	0\\
588	0\\
589	0\\
590	0\\
591	0\\
592	0\\
593	0\\
594	0\\
595	0\\
596	0\\
597	0\\
598	0\\
599	0\\
600	0\\
};
\addplot [color=red,solid,forget plot]
  table[row sep=crcr]{%
1	0\\
2	0\\
3	0\\
4	0\\
5	0\\
6	0\\
7	0\\
8	0\\
9	0\\
10	0\\
11	0\\
12	0\\
13	0\\
14	0\\
15	0\\
16	0\\
17	0\\
18	0\\
19	0\\
20	0\\
21	0\\
22	0\\
23	0\\
24	0\\
25	0\\
26	0\\
27	0\\
28	0\\
29	0\\
30	0\\
31	0\\
32	0\\
33	0\\
34	0\\
35	0\\
36	0\\
37	0\\
38	0\\
39	0\\
40	0\\
41	0\\
42	0\\
43	0\\
44	0\\
45	0\\
46	0\\
47	0\\
48	0\\
49	0\\
50	0\\
51	0\\
52	0\\
53	0\\
54	0\\
55	0\\
56	0\\
57	0\\
58	0\\
59	0\\
60	0\\
61	0\\
62	0\\
63	0\\
64	0\\
65	0\\
66	0\\
67	0\\
68	0\\
69	0\\
70	0\\
71	0\\
72	0\\
73	0\\
74	0\\
75	0\\
76	0\\
77	0\\
78	0\\
79	0\\
80	0\\
81	0\\
82	0\\
83	0\\
84	0\\
85	0\\
86	0\\
87	0\\
88	0\\
89	0\\
90	0\\
91	0\\
92	0\\
93	0\\
94	0\\
95	0\\
96	0\\
97	0\\
98	0\\
99	0\\
100	0\\
101	0\\
102	0\\
103	0\\
104	0\\
105	0\\
106	0\\
107	0\\
108	0\\
109	0\\
110	0\\
111	0\\
112	0\\
113	0\\
114	0\\
115	0\\
116	0\\
117	0\\
118	0\\
119	0\\
120	0\\
121	0\\
122	0\\
123	0\\
124	0\\
125	0\\
126	0\\
127	0\\
128	0\\
129	0\\
130	0\\
131	0\\
132	0\\
133	0\\
134	0\\
135	0\\
136	0\\
137	0\\
138	0\\
139	0\\
140	0\\
141	0\\
142	0\\
143	0\\
144	0\\
145	0\\
146	0\\
147	0\\
148	0\\
149	0\\
150	0\\
151	0\\
152	0\\
153	0\\
154	0\\
155	0\\
156	0\\
157	0\\
158	0\\
159	0\\
160	0\\
161	0\\
162	0\\
163	0\\
164	0\\
165	0\\
166	0\\
167	0\\
168	0\\
169	0\\
170	0\\
171	0\\
172	0\\
173	0\\
174	0\\
175	0\\
176	0\\
177	0\\
178	0\\
179	0\\
180	0\\
181	0\\
182	0\\
183	0\\
184	0\\
185	0\\
186	0\\
187	0\\
188	0\\
189	0\\
190	0\\
191	0\\
192	0\\
193	0\\
194	0\\
195	0\\
196	0\\
197	0\\
198	0\\
199	0\\
200	0\\
201	0\\
202	0\\
203	0\\
204	0\\
205	0\\
206	0\\
207	0\\
208	0\\
209	0\\
210	0\\
211	0\\
212	0\\
213	0\\
214	0\\
215	0\\
216	0\\
217	0\\
218	0\\
219	0\\
220	0\\
221	0\\
222	0\\
223	0\\
224	0\\
225	0\\
226	0\\
227	0\\
228	0\\
229	0\\
230	0\\
231	0\\
232	0\\
233	0\\
234	0\\
235	0\\
236	0\\
237	0\\
238	0\\
239	0\\
240	0\\
241	0\\
242	0\\
243	0\\
244	0\\
245	0\\
246	0\\
247	0\\
248	0\\
249	0\\
250	0\\
251	0\\
252	0\\
253	0\\
254	0\\
255	0\\
256	0\\
257	0\\
258	0\\
259	0\\
260	0\\
261	0\\
262	0\\
263	0\\
264	0\\
265	0\\
266	0\\
267	0\\
268	0\\
269	0\\
270	0\\
271	0\\
272	0\\
273	0\\
274	0\\
275	0\\
276	0\\
277	0\\
278	0\\
279	0\\
280	0\\
281	0\\
282	0\\
283	0\\
284	0\\
285	0\\
286	0\\
287	0\\
288	0\\
289	0\\
290	0\\
291	0\\
292	0\\
293	0\\
294	0\\
295	0\\
296	0\\
297	0\\
298	0\\
299	0\\
300	0\\
301	0\\
302	0\\
303	0\\
304	0\\
305	0\\
306	0\\
307	0\\
308	0\\
309	0\\
310	0\\
311	0\\
312	0\\
313	0\\
314	0\\
315	0\\
316	0\\
317	0\\
318	0\\
319	0\\
320	0\\
321	0\\
322	0\\
323	0\\
324	0\\
325	0\\
326	0\\
327	0\\
328	0\\
329	0\\
330	0\\
331	0\\
332	0\\
333	0\\
334	0\\
335	0\\
336	0\\
337	0\\
338	0\\
339	0\\
340	0\\
341	0\\
342	0\\
343	0\\
344	0\\
345	0\\
346	0\\
347	0\\
348	0\\
349	0\\
350	0\\
351	0\\
352	0\\
353	0\\
354	0\\
355	0\\
356	0\\
357	0\\
358	0\\
359	0\\
360	0\\
361	0\\
362	0\\
363	0\\
364	0\\
365	0\\
366	0\\
367	0\\
368	0\\
369	0\\
370	0\\
371	0\\
372	0\\
373	0\\
374	0\\
375	0\\
376	0\\
377	0\\
378	0\\
379	0\\
380	0\\
381	0\\
382	0\\
383	0\\
384	0\\
385	0\\
386	0\\
387	0\\
388	0\\
389	0\\
390	0\\
391	0\\
392	0\\
393	0\\
394	0\\
395	0\\
396	0\\
397	0\\
398	0\\
399	0\\
400	0\\
401	0\\
402	0\\
403	0\\
404	0\\
405	0\\
406	0\\
407	0\\
408	0\\
409	0\\
410	0\\
411	0\\
412	0\\
413	0\\
414	0\\
415	0\\
416	0\\
417	0\\
418	0\\
419	0\\
420	0\\
421	0\\
422	0\\
423	0\\
424	0\\
425	0\\
426	0\\
427	0\\
428	0\\
429	0\\
430	0\\
431	0\\
432	0\\
433	0\\
434	0\\
435	0\\
436	0\\
437	0\\
438	0\\
439	0\\
440	0\\
441	0\\
442	0\\
443	0\\
444	0\\
445	0\\
446	0\\
447	0\\
448	0\\
449	0\\
450	0\\
451	0\\
452	0\\
453	0\\
454	0\\
455	0\\
456	0\\
457	0\\
458	0\\
459	0\\
460	0\\
461	0\\
462	0\\
463	0\\
464	0\\
465	0\\
466	0\\
467	0\\
468	0\\
469	0\\
470	0\\
471	0\\
472	0\\
473	0\\
474	0\\
475	0\\
476	0\\
477	0\\
478	0\\
479	0\\
480	0\\
481	0\\
482	0\\
483	0\\
484	0\\
485	0\\
486	0\\
487	0\\
488	0\\
489	0\\
490	0\\
491	0\\
492	0\\
493	0\\
494	0\\
495	0\\
496	0\\
497	0\\
498	0\\
499	0\\
500	0\\
501	0\\
502	0\\
503	0\\
504	0\\
505	0\\
506	0\\
507	0\\
508	0\\
509	0\\
510	0\\
511	0\\
512	0\\
513	0\\
514	0\\
515	0\\
516	0\\
517	0\\
518	0\\
519	0\\
520	0\\
521	0\\
522	0\\
523	0\\
524	0\\
525	0\\
526	0\\
527	0\\
528	0\\
529	0\\
530	0\\
531	0\\
532	0\\
533	0\\
534	0\\
535	0\\
536	0\\
537	0\\
538	0\\
539	0\\
540	0\\
541	0\\
542	0\\
543	0\\
544	0\\
545	0\\
546	0\\
547	0\\
548	0\\
549	0\\
550	0\\
551	0\\
552	0\\
553	0\\
554	0\\
555	0\\
556	0\\
557	0\\
558	0\\
559	0\\
560	0\\
561	0\\
562	0\\
563	0\\
564	0\\
565	0\\
566	0\\
567	0\\
568	0\\
569	0\\
570	0\\
571	0\\
572	0\\
573	0\\
574	0\\
575	0\\
576	0\\
577	0\\
578	0\\
579	0\\
580	0\\
581	0\\
582	0\\
583	0\\
584	0\\
585	0\\
586	0\\
587	0\\
588	0\\
589	0\\
590	0\\
591	0\\
592	0\\
593	0\\
594	0\\
595	0\\
596	0\\
597	0\\
598	0\\
599	0\\
600	0\\
};
\addplot [color=mycolor20,solid,forget plot]
  table[row sep=crcr]{%
1	0\\
2	0\\
3	0\\
4	0\\
5	0\\
6	0\\
7	0\\
8	0\\
9	0\\
10	0\\
11	0\\
12	0\\
13	0\\
14	0\\
15	0\\
16	0\\
17	0\\
18	0\\
19	0\\
20	0\\
21	0\\
22	0\\
23	0\\
24	0\\
25	0\\
26	0\\
27	0\\
28	0\\
29	0\\
30	0\\
31	0\\
32	0\\
33	0\\
34	0\\
35	0\\
36	0\\
37	0\\
38	0\\
39	0\\
40	0\\
41	0\\
42	0\\
43	0\\
44	0\\
45	0\\
46	0\\
47	0\\
48	0\\
49	0\\
50	0\\
51	0\\
52	0\\
53	0\\
54	0\\
55	0\\
56	0\\
57	0\\
58	0\\
59	0\\
60	0\\
61	0\\
62	0\\
63	0\\
64	0\\
65	0\\
66	0\\
67	0\\
68	0\\
69	0\\
70	0\\
71	0\\
72	0\\
73	0\\
74	0\\
75	0\\
76	0\\
77	0\\
78	0\\
79	0\\
80	0\\
81	0\\
82	0\\
83	0\\
84	0\\
85	0\\
86	0\\
87	0\\
88	0\\
89	0\\
90	0\\
91	0\\
92	0\\
93	0\\
94	0\\
95	0\\
96	0\\
97	0\\
98	0\\
99	0\\
100	0\\
101	0\\
102	0\\
103	0\\
104	0\\
105	0\\
106	0\\
107	0\\
108	0\\
109	0\\
110	0\\
111	0\\
112	0\\
113	0\\
114	0\\
115	0\\
116	0\\
117	0\\
118	0\\
119	0\\
120	0\\
121	0\\
122	0\\
123	0\\
124	0\\
125	0\\
126	0\\
127	0\\
128	0\\
129	0\\
130	0\\
131	0\\
132	0\\
133	0\\
134	0\\
135	0\\
136	0\\
137	0\\
138	0\\
139	0\\
140	0\\
141	0\\
142	0\\
143	0\\
144	0\\
145	0\\
146	0\\
147	0\\
148	0\\
149	0\\
150	0\\
151	0\\
152	0\\
153	0\\
154	0\\
155	0\\
156	0\\
157	0\\
158	0\\
159	0\\
160	0\\
161	0\\
162	0\\
163	0\\
164	0\\
165	0\\
166	0\\
167	0\\
168	0\\
169	0\\
170	0\\
171	0\\
172	0\\
173	0\\
174	0\\
175	0\\
176	0\\
177	0\\
178	0\\
179	0\\
180	0\\
181	0\\
182	0\\
183	0\\
184	0\\
185	0\\
186	0\\
187	0\\
188	0\\
189	0\\
190	0\\
191	0\\
192	0\\
193	0\\
194	0\\
195	0\\
196	0\\
197	0\\
198	0\\
199	0\\
200	0\\
201	0\\
202	0\\
203	0\\
204	0\\
205	0\\
206	0\\
207	0\\
208	0\\
209	0\\
210	0\\
211	0\\
212	0\\
213	0\\
214	0\\
215	0\\
216	0\\
217	0\\
218	0\\
219	0\\
220	0\\
221	0\\
222	0\\
223	0\\
224	0\\
225	0\\
226	0\\
227	0\\
228	0\\
229	0\\
230	0\\
231	0\\
232	0\\
233	0\\
234	0\\
235	0\\
236	0\\
237	0\\
238	0\\
239	0\\
240	0\\
241	0\\
242	0\\
243	0\\
244	0\\
245	0\\
246	0\\
247	0\\
248	0\\
249	0\\
250	0\\
251	0\\
252	0\\
253	0\\
254	0\\
255	0\\
256	0\\
257	0\\
258	0\\
259	0\\
260	0\\
261	0\\
262	0\\
263	0\\
264	0\\
265	0\\
266	0\\
267	0\\
268	0\\
269	0\\
270	0\\
271	0\\
272	0\\
273	0\\
274	0\\
275	0\\
276	0\\
277	0\\
278	0\\
279	0\\
280	0\\
281	0\\
282	0\\
283	0\\
284	0\\
285	0\\
286	0\\
287	0\\
288	0\\
289	0\\
290	0\\
291	0\\
292	0\\
293	0\\
294	0\\
295	0\\
296	0\\
297	0\\
298	0\\
299	0\\
300	0\\
301	0\\
302	0\\
303	0\\
304	0\\
305	0\\
306	0\\
307	0\\
308	0\\
309	0\\
310	0\\
311	0\\
312	0\\
313	0\\
314	0\\
315	0\\
316	0\\
317	0\\
318	0\\
319	0\\
320	0\\
321	0\\
322	0\\
323	0\\
324	0\\
325	0\\
326	0\\
327	0\\
328	0\\
329	0\\
330	0\\
331	0\\
332	0\\
333	0\\
334	0\\
335	0\\
336	0\\
337	0\\
338	0\\
339	0\\
340	0\\
341	0\\
342	0\\
343	0\\
344	0\\
345	0\\
346	0\\
347	0\\
348	0\\
349	0\\
350	0\\
351	0\\
352	0\\
353	0\\
354	0\\
355	0\\
356	0\\
357	0\\
358	0\\
359	0\\
360	0\\
361	0\\
362	0\\
363	0\\
364	0\\
365	0\\
366	0\\
367	0\\
368	0\\
369	0\\
370	0\\
371	0\\
372	0\\
373	0\\
374	0\\
375	0\\
376	0\\
377	0\\
378	0\\
379	0\\
380	0\\
381	0\\
382	0\\
383	0\\
384	0\\
385	0\\
386	0\\
387	0\\
388	0\\
389	0\\
390	0\\
391	0\\
392	0\\
393	0\\
394	0\\
395	0\\
396	0\\
397	0\\
398	0\\
399	0\\
400	0\\
401	0\\
402	0\\
403	0\\
404	0\\
405	0\\
406	0\\
407	0\\
408	0\\
409	0\\
410	0\\
411	0\\
412	0\\
413	0\\
414	0\\
415	0\\
416	0\\
417	0\\
418	0\\
419	0\\
420	0\\
421	0\\
422	0\\
423	0\\
424	0\\
425	0\\
426	0\\
427	0\\
428	0\\
429	0\\
430	0\\
431	0\\
432	0\\
433	0\\
434	0\\
435	0\\
436	0\\
437	0\\
438	0\\
439	0\\
440	0\\
441	0\\
442	0\\
443	0\\
444	0\\
445	0\\
446	0\\
447	0\\
448	0\\
449	0\\
450	0\\
451	0\\
452	0\\
453	0\\
454	0\\
455	0\\
456	0\\
457	0\\
458	0\\
459	0\\
460	0\\
461	0\\
462	0\\
463	0\\
464	0\\
465	0\\
466	0\\
467	0\\
468	0\\
469	0\\
470	0\\
471	0\\
472	0\\
473	0\\
474	0\\
475	0\\
476	0\\
477	0\\
478	0\\
479	0\\
480	0\\
481	0\\
482	0\\
483	0\\
484	0\\
485	0\\
486	0\\
487	0\\
488	0\\
489	0\\
490	0\\
491	0\\
492	0\\
493	0\\
494	0\\
495	0\\
496	0\\
497	0\\
498	0\\
499	0\\
500	0\\
501	0\\
502	0\\
503	0\\
504	0\\
505	0\\
506	0\\
507	0\\
508	0\\
509	0\\
510	0\\
511	0\\
512	0\\
513	0\\
514	0\\
515	0\\
516	0\\
517	0\\
518	0\\
519	0\\
520	0\\
521	0\\
522	0\\
523	0\\
524	0\\
525	0\\
526	0\\
527	0\\
528	0\\
529	0\\
530	0\\
531	0\\
532	0\\
533	0\\
534	0\\
535	0\\
536	0\\
537	0\\
538	0\\
539	0\\
540	0\\
541	0\\
542	0\\
543	0\\
544	0\\
545	0\\
546	0\\
547	0\\
548	0\\
549	0\\
550	0\\
551	0\\
552	0\\
553	0\\
554	0\\
555	0\\
556	0\\
557	0\\
558	0\\
559	0\\
560	0\\
561	0\\
562	0\\
563	0\\
564	0\\
565	0\\
566	0\\
567	0\\
568	0\\
569	0\\
570	0\\
571	0\\
572	0\\
573	0\\
574	0\\
575	0\\
576	0\\
577	0\\
578	0\\
579	0\\
580	0\\
581	0\\
582	0\\
583	0\\
584	0\\
585	0\\
586	0\\
587	0\\
588	0\\
589	0\\
590	0\\
591	0\\
592	0\\
593	0\\
594	0\\
595	0\\
596	0\\
597	0\\
598	0\\
599	0\\
600	0\\
};
\addplot [color=mycolor21,solid,forget plot]
  table[row sep=crcr]{%
1	0\\
2	0\\
3	0\\
4	0\\
5	0\\
6	0\\
7	0\\
8	0\\
9	0\\
10	0\\
11	0\\
12	0\\
13	0\\
14	0\\
15	0\\
16	0\\
17	0\\
18	0\\
19	0\\
20	0\\
21	0\\
22	0\\
23	0\\
24	0\\
25	0\\
26	0\\
27	0\\
28	0\\
29	0\\
30	0\\
31	0\\
32	0\\
33	0\\
34	0\\
35	0\\
36	0\\
37	0\\
38	0\\
39	0\\
40	0\\
41	0\\
42	0\\
43	0\\
44	0\\
45	0\\
46	0\\
47	0\\
48	0\\
49	0\\
50	0\\
51	0\\
52	0\\
53	0\\
54	0\\
55	0\\
56	0\\
57	0\\
58	0\\
59	0\\
60	0\\
61	0\\
62	0\\
63	0\\
64	0\\
65	0\\
66	0\\
67	0\\
68	0\\
69	0\\
70	0\\
71	0\\
72	0\\
73	0\\
74	0\\
75	0\\
76	0\\
77	0\\
78	0\\
79	0\\
80	0\\
81	0\\
82	0\\
83	0\\
84	0\\
85	0\\
86	0\\
87	0\\
88	0\\
89	0\\
90	0\\
91	0\\
92	0\\
93	0\\
94	0\\
95	0\\
96	0\\
97	0\\
98	0\\
99	0\\
100	0\\
101	0\\
102	0\\
103	0\\
104	0\\
105	0\\
106	0\\
107	0\\
108	0\\
109	0\\
110	0\\
111	0\\
112	0\\
113	0\\
114	0\\
115	0\\
116	0\\
117	0\\
118	0\\
119	0\\
120	0\\
121	0\\
122	0\\
123	0\\
124	0\\
125	0\\
126	0\\
127	0\\
128	0\\
129	0\\
130	0\\
131	0\\
132	0\\
133	0\\
134	0\\
135	0\\
136	0\\
137	0\\
138	0\\
139	0\\
140	0\\
141	0\\
142	0\\
143	0\\
144	0\\
145	0\\
146	0\\
147	0\\
148	0\\
149	0\\
150	0\\
151	0\\
152	0\\
153	0\\
154	0\\
155	0\\
156	0\\
157	0\\
158	0\\
159	0\\
160	0\\
161	0\\
162	0\\
163	0\\
164	0\\
165	0\\
166	0\\
167	0\\
168	0\\
169	0\\
170	0\\
171	0\\
172	0\\
173	0\\
174	0\\
175	0\\
176	0\\
177	0\\
178	0\\
179	0\\
180	0\\
181	0\\
182	0\\
183	0\\
184	0\\
185	0\\
186	0\\
187	0\\
188	0\\
189	0\\
190	0\\
191	0\\
192	0\\
193	0\\
194	0\\
195	0\\
196	0\\
197	0\\
198	0\\
199	0\\
200	0\\
201	0\\
202	0\\
203	0\\
204	0\\
205	0\\
206	0\\
207	0\\
208	0\\
209	0\\
210	0\\
211	0\\
212	0\\
213	0\\
214	0\\
215	0\\
216	0\\
217	0\\
218	0\\
219	0\\
220	0\\
221	0\\
222	0\\
223	0\\
224	0\\
225	0\\
226	0\\
227	0\\
228	0\\
229	0\\
230	0\\
231	0\\
232	0\\
233	0\\
234	0\\
235	0\\
236	0\\
237	0\\
238	0\\
239	0\\
240	0\\
241	0\\
242	0\\
243	0\\
244	0\\
245	0\\
246	0\\
247	0\\
248	0\\
249	0\\
250	0\\
251	0\\
252	0\\
253	0\\
254	0\\
255	0\\
256	0\\
257	0\\
258	0\\
259	0\\
260	0\\
261	0\\
262	0\\
263	0\\
264	0\\
265	0\\
266	0\\
267	0\\
268	0\\
269	0\\
270	0\\
271	0\\
272	0\\
273	0\\
274	0\\
275	0\\
276	0\\
277	0\\
278	0\\
279	0\\
280	0\\
281	0\\
282	0\\
283	0\\
284	0\\
285	0\\
286	0\\
287	0\\
288	0\\
289	0\\
290	0\\
291	0\\
292	0\\
293	0\\
294	0\\
295	0\\
296	0\\
297	0\\
298	0\\
299	0\\
300	0\\
301	0\\
302	0\\
303	0\\
304	0\\
305	0\\
306	0\\
307	0\\
308	0\\
309	0\\
310	0\\
311	0\\
312	0\\
313	0\\
314	0\\
315	0\\
316	0\\
317	0\\
318	0\\
319	0\\
320	0\\
321	0\\
322	0\\
323	0\\
324	0\\
325	0\\
326	0\\
327	0\\
328	0\\
329	0\\
330	0\\
331	0\\
332	0\\
333	0\\
334	0\\
335	0\\
336	0\\
337	0\\
338	0\\
339	0\\
340	0\\
341	0\\
342	0\\
343	0\\
344	0\\
345	0\\
346	0\\
347	0\\
348	0\\
349	0\\
350	0\\
351	0\\
352	0\\
353	0\\
354	0\\
355	0\\
356	0\\
357	0\\
358	0\\
359	0\\
360	0\\
361	0\\
362	0\\
363	0\\
364	0\\
365	0\\
366	0\\
367	0\\
368	0\\
369	0\\
370	0\\
371	0\\
372	0\\
373	0\\
374	0\\
375	0\\
376	0\\
377	0\\
378	0\\
379	0\\
380	0\\
381	0\\
382	0\\
383	0\\
384	0\\
385	0\\
386	0\\
387	0\\
388	0\\
389	0\\
390	0\\
391	0\\
392	0\\
393	0\\
394	0\\
395	0\\
396	0\\
397	0\\
398	0\\
399	0\\
400	0\\
401	0\\
402	0\\
403	0\\
404	0\\
405	0\\
406	0\\
407	0\\
408	0\\
409	0\\
410	0\\
411	0\\
412	0\\
413	0\\
414	0\\
415	0\\
416	0\\
417	0\\
418	0\\
419	0\\
420	0\\
421	0\\
422	0\\
423	0\\
424	0\\
425	0\\
426	0\\
427	0\\
428	0\\
429	0\\
430	0\\
431	0\\
432	0\\
433	0\\
434	0\\
435	0\\
436	0\\
437	0\\
438	0\\
439	0\\
440	0\\
441	0\\
442	0\\
443	0\\
444	0\\
445	0\\
446	0\\
447	0\\
448	0\\
449	0\\
450	0\\
451	0\\
452	0\\
453	0\\
454	0\\
455	0\\
456	0\\
457	0\\
458	0\\
459	0\\
460	0\\
461	0\\
462	0\\
463	0\\
464	0\\
465	0\\
466	0\\
467	0\\
468	0\\
469	0\\
470	0\\
471	0\\
472	0\\
473	0\\
474	0\\
475	0\\
476	0\\
477	0\\
478	0\\
479	0\\
480	0\\
481	0\\
482	0\\
483	0\\
484	0\\
485	0\\
486	0\\
487	0\\
488	0\\
489	0\\
490	0\\
491	0\\
492	0\\
493	0\\
494	0\\
495	0\\
496	0\\
497	0\\
498	0\\
499	0\\
500	0\\
501	0\\
502	0\\
503	0\\
504	0\\
505	0\\
506	0\\
507	0\\
508	0\\
509	0\\
510	0\\
511	0\\
512	0\\
513	0\\
514	0\\
515	0\\
516	0\\
517	0\\
518	0\\
519	0\\
520	0\\
521	0\\
522	0\\
523	0\\
524	0\\
525	0\\
526	0\\
527	0\\
528	0\\
529	0\\
530	0\\
531	0\\
532	0\\
533	0\\
534	0\\
535	0\\
536	0\\
537	0\\
538	0\\
539	0\\
540	0\\
541	0\\
542	0\\
543	0\\
544	0\\
545	0\\
546	0\\
547	0\\
548	0\\
549	0\\
550	0\\
551	0\\
552	0\\
553	0\\
554	0\\
555	0\\
556	0\\
557	0\\
558	0\\
559	0\\
560	0\\
561	0\\
562	0\\
563	0\\
564	0\\
565	0\\
566	0\\
567	0\\
568	0\\
569	0\\
570	0\\
571	0\\
572	0\\
573	0\\
574	0\\
575	0\\
576	0\\
577	0\\
578	0\\
579	0\\
580	0\\
581	0\\
582	0\\
583	0\\
584	0\\
585	0\\
586	0\\
587	0\\
588	0\\
589	0\\
590	0\\
591	0\\
592	0\\
593	0\\
594	0\\
595	0\\
596	0\\
597	0\\
598	0\\
599	0\\
600	0\\
};
\addplot [color=black!20!mycolor21,solid,forget plot]
  table[row sep=crcr]{%
1	0\\
2	0\\
3	0\\
4	0\\
5	0\\
6	0\\
7	0\\
8	0\\
9	0\\
10	0\\
11	0\\
12	0\\
13	0\\
14	0\\
15	0\\
16	0\\
17	0\\
18	0\\
19	0\\
20	0\\
21	0\\
22	0\\
23	0\\
24	0\\
25	0\\
26	0\\
27	0\\
28	0\\
29	0\\
30	0\\
31	0\\
32	0\\
33	0\\
34	0\\
35	0\\
36	0\\
37	0\\
38	0\\
39	0\\
40	0\\
41	0\\
42	0\\
43	0\\
44	0\\
45	0\\
46	0\\
47	0\\
48	0\\
49	0\\
50	0\\
51	0\\
52	0\\
53	0\\
54	0\\
55	0\\
56	0\\
57	0\\
58	0\\
59	0\\
60	0\\
61	0\\
62	0\\
63	0\\
64	0\\
65	0\\
66	0\\
67	0\\
68	0\\
69	0\\
70	0\\
71	0\\
72	0\\
73	0\\
74	0\\
75	0\\
76	0\\
77	0\\
78	0\\
79	0\\
80	0\\
81	0\\
82	0\\
83	0\\
84	0\\
85	0\\
86	0\\
87	0\\
88	0\\
89	0\\
90	0\\
91	0\\
92	0\\
93	0\\
94	0\\
95	0\\
96	0\\
97	0\\
98	0\\
99	0\\
100	0\\
101	0\\
102	0\\
103	0\\
104	0\\
105	0\\
106	0\\
107	0\\
108	0\\
109	0\\
110	0\\
111	0\\
112	0\\
113	0\\
114	0\\
115	0\\
116	0\\
117	0\\
118	0\\
119	0\\
120	0\\
121	0\\
122	0\\
123	0\\
124	0\\
125	0\\
126	0\\
127	0\\
128	0\\
129	0\\
130	0\\
131	0\\
132	0\\
133	0\\
134	0\\
135	0\\
136	0\\
137	0\\
138	0\\
139	0\\
140	0\\
141	0\\
142	0\\
143	0\\
144	0\\
145	0\\
146	0\\
147	0\\
148	0\\
149	0\\
150	0\\
151	0\\
152	0\\
153	0\\
154	0\\
155	0\\
156	0\\
157	0\\
158	0\\
159	0\\
160	0\\
161	0\\
162	0\\
163	0\\
164	0\\
165	0\\
166	0\\
167	0\\
168	0\\
169	0\\
170	0\\
171	0\\
172	0\\
173	0\\
174	0\\
175	0\\
176	0\\
177	0\\
178	0\\
179	0\\
180	0\\
181	0\\
182	0\\
183	0\\
184	0\\
185	0\\
186	0\\
187	0\\
188	0\\
189	0\\
190	0\\
191	0\\
192	0\\
193	0\\
194	0\\
195	0\\
196	0\\
197	0\\
198	0\\
199	0\\
200	0\\
201	0\\
202	0\\
203	0\\
204	0\\
205	0\\
206	0\\
207	0\\
208	0\\
209	0\\
210	0\\
211	0\\
212	0\\
213	0\\
214	0\\
215	0\\
216	0\\
217	0\\
218	0\\
219	0\\
220	0\\
221	0\\
222	0\\
223	0\\
224	0\\
225	0\\
226	0\\
227	0\\
228	0\\
229	0\\
230	0\\
231	0\\
232	0\\
233	0\\
234	0\\
235	0\\
236	0\\
237	0\\
238	0\\
239	0\\
240	0\\
241	0\\
242	0\\
243	0\\
244	0\\
245	0\\
246	0\\
247	0\\
248	0\\
249	0\\
250	0\\
251	0\\
252	0\\
253	0\\
254	0\\
255	0\\
256	0\\
257	0\\
258	0\\
259	0\\
260	0\\
261	0\\
262	0\\
263	0\\
264	0\\
265	0\\
266	0\\
267	0\\
268	0\\
269	0\\
270	0\\
271	0\\
272	0\\
273	0\\
274	0\\
275	0\\
276	0\\
277	0\\
278	0\\
279	0\\
280	0\\
281	0\\
282	0\\
283	0\\
284	0\\
285	0\\
286	0\\
287	0\\
288	0\\
289	0\\
290	0\\
291	0\\
292	0\\
293	0\\
294	0\\
295	0\\
296	0\\
297	0\\
298	0\\
299	0\\
300	0\\
301	0\\
302	0\\
303	0\\
304	0\\
305	0\\
306	0\\
307	0\\
308	0\\
309	0\\
310	0\\
311	0\\
312	0\\
313	0\\
314	0\\
315	0\\
316	0\\
317	0\\
318	0\\
319	0\\
320	0\\
321	0\\
322	0\\
323	0\\
324	0\\
325	0\\
326	0\\
327	0\\
328	0\\
329	0\\
330	0\\
331	0\\
332	0\\
333	0\\
334	0\\
335	0\\
336	0\\
337	0\\
338	0\\
339	0\\
340	0\\
341	0\\
342	0\\
343	0\\
344	0\\
345	0\\
346	0\\
347	0\\
348	0\\
349	0\\
350	0\\
351	0\\
352	0\\
353	0\\
354	0\\
355	0\\
356	0\\
357	0\\
358	0\\
359	0\\
360	0\\
361	0\\
362	0\\
363	0\\
364	0\\
365	0\\
366	0\\
367	0\\
368	0\\
369	0\\
370	0\\
371	0\\
372	0\\
373	0\\
374	0\\
375	0\\
376	0\\
377	0\\
378	0\\
379	0\\
380	0\\
381	0\\
382	0\\
383	0\\
384	0\\
385	0\\
386	0\\
387	0\\
388	0\\
389	0\\
390	0\\
391	0\\
392	0\\
393	0\\
394	0\\
395	0\\
396	0\\
397	0\\
398	0\\
399	0\\
400	0\\
401	0\\
402	0\\
403	0\\
404	0\\
405	0\\
406	0\\
407	0\\
408	0\\
409	0\\
410	0\\
411	0\\
412	0\\
413	0\\
414	0\\
415	0\\
416	0\\
417	0\\
418	0\\
419	0\\
420	0\\
421	0\\
422	0\\
423	0\\
424	0\\
425	0\\
426	0\\
427	0\\
428	0\\
429	0\\
430	0\\
431	0\\
432	0\\
433	0\\
434	0\\
435	0\\
436	0\\
437	0\\
438	0\\
439	0\\
440	0\\
441	0\\
442	0\\
443	0\\
444	0\\
445	0\\
446	0\\
447	0\\
448	0\\
449	0\\
450	0\\
451	0\\
452	0\\
453	0\\
454	0\\
455	0\\
456	0\\
457	0\\
458	0\\
459	0\\
460	0\\
461	0\\
462	0\\
463	0\\
464	0\\
465	0\\
466	0\\
467	0\\
468	0\\
469	0\\
470	0\\
471	0\\
472	0\\
473	0\\
474	0\\
475	0\\
476	0\\
477	0\\
478	0\\
479	0\\
480	0\\
481	0\\
482	0\\
483	0\\
484	0\\
485	0\\
486	0\\
487	0\\
488	0\\
489	0\\
490	0\\
491	0\\
492	0\\
493	0\\
494	0\\
495	0\\
496	0\\
497	0\\
498	0\\
499	0\\
500	0\\
501	0\\
502	0\\
503	0\\
504	0\\
505	0\\
506	0\\
507	0\\
508	0\\
509	0\\
510	0\\
511	0\\
512	0\\
513	0\\
514	0\\
515	0\\
516	0\\
517	0\\
518	0\\
519	0\\
520	0\\
521	0\\
522	0\\
523	0\\
524	0\\
525	0\\
526	0\\
527	0\\
528	0\\
529	0\\
530	0\\
531	0\\
532	0\\
533	0\\
534	0\\
535	0\\
536	0\\
537	0\\
538	0\\
539	0\\
540	0\\
541	0\\
542	0\\
543	0\\
544	0\\
545	0\\
546	0\\
547	0\\
548	0\\
549	0\\
550	0\\
551	0\\
552	0\\
553	0\\
554	0\\
555	0\\
556	0\\
557	0\\
558	0\\
559	0\\
560	0\\
561	0\\
562	0\\
563	0\\
564	0\\
565	0\\
566	0\\
567	0\\
568	0\\
569	0\\
570	0\\
571	0\\
572	0\\
573	0\\
574	0\\
575	0\\
576	0\\
577	0\\
578	0\\
579	0\\
580	0\\
581	0\\
582	0\\
583	0\\
584	0\\
585	0\\
586	0\\
587	0\\
588	0\\
589	0\\
590	0\\
591	0\\
592	0\\
593	0\\
594	0\\
595	0\\
596	0\\
597	0\\
598	0\\
599	0\\
600	0\\
};
\addplot [color=black!50!mycolor20,solid,forget plot]
  table[row sep=crcr]{%
1	0\\
2	0\\
3	0\\
4	0\\
5	0\\
6	0\\
7	0\\
8	0\\
9	0\\
10	0\\
11	0\\
12	0\\
13	0\\
14	0\\
15	0\\
16	0\\
17	0\\
18	0\\
19	0\\
20	0\\
21	0\\
22	0\\
23	0\\
24	0\\
25	0\\
26	0\\
27	0\\
28	0\\
29	0\\
30	0\\
31	0\\
32	0\\
33	0\\
34	0\\
35	0\\
36	0\\
37	0\\
38	0\\
39	0\\
40	0\\
41	0\\
42	0\\
43	0\\
44	0\\
45	0\\
46	0\\
47	0\\
48	0\\
49	0\\
50	0\\
51	0\\
52	0\\
53	0\\
54	0\\
55	0\\
56	0\\
57	0\\
58	0\\
59	0\\
60	0\\
61	0\\
62	0\\
63	0\\
64	0\\
65	0\\
66	0\\
67	0\\
68	0\\
69	0\\
70	0\\
71	0\\
72	0\\
73	0\\
74	0\\
75	0\\
76	0\\
77	0\\
78	0\\
79	0\\
80	0\\
81	0\\
82	0\\
83	0\\
84	0\\
85	0\\
86	0\\
87	0\\
88	0\\
89	0\\
90	0\\
91	0\\
92	0\\
93	0\\
94	0\\
95	0\\
96	0\\
97	0\\
98	0\\
99	0\\
100	0\\
101	0\\
102	0\\
103	0\\
104	0\\
105	0\\
106	0\\
107	0\\
108	0\\
109	0\\
110	0\\
111	0\\
112	0\\
113	0\\
114	0\\
115	0\\
116	0\\
117	0\\
118	0\\
119	0\\
120	0\\
121	0\\
122	0\\
123	0\\
124	0\\
125	0\\
126	0\\
127	0\\
128	0\\
129	0\\
130	0\\
131	0\\
132	0\\
133	0\\
134	0\\
135	0\\
136	0\\
137	0\\
138	0\\
139	0\\
140	0\\
141	0\\
142	0\\
143	0\\
144	0\\
145	0\\
146	0\\
147	0\\
148	0\\
149	0\\
150	0\\
151	0\\
152	0\\
153	0\\
154	0\\
155	0\\
156	0\\
157	0\\
158	0\\
159	0\\
160	0\\
161	0\\
162	0\\
163	0\\
164	0\\
165	0\\
166	0\\
167	0\\
168	0\\
169	0\\
170	0\\
171	0\\
172	0\\
173	0\\
174	0\\
175	0\\
176	0\\
177	0\\
178	0\\
179	0\\
180	0\\
181	0\\
182	0\\
183	0\\
184	0\\
185	0\\
186	0\\
187	0\\
188	0\\
189	0\\
190	0\\
191	0\\
192	0\\
193	0\\
194	0\\
195	0\\
196	0\\
197	0\\
198	0\\
199	0\\
200	0\\
201	0\\
202	0\\
203	0\\
204	0\\
205	0\\
206	0\\
207	0\\
208	0\\
209	0\\
210	0\\
211	0\\
212	0\\
213	0\\
214	0\\
215	0\\
216	0\\
217	0\\
218	0\\
219	0\\
220	0\\
221	0\\
222	0\\
223	0\\
224	0\\
225	0\\
226	0\\
227	0\\
228	0\\
229	0\\
230	0\\
231	0\\
232	0\\
233	0\\
234	0\\
235	0\\
236	0\\
237	0\\
238	0\\
239	0\\
240	0\\
241	0\\
242	0\\
243	0\\
244	0\\
245	0\\
246	0\\
247	0\\
248	0\\
249	0\\
250	0\\
251	0\\
252	0\\
253	0\\
254	0\\
255	0\\
256	0\\
257	0\\
258	0\\
259	0\\
260	0\\
261	0\\
262	0\\
263	0\\
264	0\\
265	0\\
266	0\\
267	0\\
268	0\\
269	0\\
270	0\\
271	0\\
272	0\\
273	0\\
274	0\\
275	0\\
276	0\\
277	0\\
278	0\\
279	0\\
280	0\\
281	0\\
282	0\\
283	0\\
284	0\\
285	0\\
286	0\\
287	0\\
288	0\\
289	0\\
290	0\\
291	0\\
292	0\\
293	0\\
294	0\\
295	0\\
296	0\\
297	0\\
298	0\\
299	0\\
300	0\\
301	0\\
302	0\\
303	0\\
304	0\\
305	0\\
306	0\\
307	0\\
308	0\\
309	0\\
310	0\\
311	0\\
312	0\\
313	0\\
314	0\\
315	0\\
316	0\\
317	0\\
318	0\\
319	0\\
320	0\\
321	0\\
322	0\\
323	0\\
324	0\\
325	0\\
326	0\\
327	0\\
328	0\\
329	0\\
330	0\\
331	0\\
332	0\\
333	0\\
334	0\\
335	0\\
336	0\\
337	0\\
338	0\\
339	0\\
340	0\\
341	0\\
342	0\\
343	0\\
344	0\\
345	0\\
346	0\\
347	0\\
348	0\\
349	0\\
350	0\\
351	0\\
352	0\\
353	0\\
354	0\\
355	0\\
356	0\\
357	0\\
358	0\\
359	0\\
360	0\\
361	0\\
362	0\\
363	0\\
364	0\\
365	0\\
366	0\\
367	0\\
368	0\\
369	0\\
370	0\\
371	0\\
372	0\\
373	0\\
374	0\\
375	0\\
376	0\\
377	0\\
378	0\\
379	0\\
380	0\\
381	0\\
382	0\\
383	0\\
384	0\\
385	0\\
386	0\\
387	0\\
388	0\\
389	0\\
390	0\\
391	0\\
392	0\\
393	0\\
394	0\\
395	0\\
396	0\\
397	0\\
398	0\\
399	0\\
400	0\\
401	0\\
402	0\\
403	0\\
404	0\\
405	0\\
406	0\\
407	0\\
408	0\\
409	0\\
410	0\\
411	0\\
412	0\\
413	0\\
414	0\\
415	0\\
416	0\\
417	0\\
418	0\\
419	0\\
420	0\\
421	0\\
422	0\\
423	0\\
424	0\\
425	0\\
426	0\\
427	0\\
428	0\\
429	0\\
430	0\\
431	0\\
432	0\\
433	0\\
434	0\\
435	0\\
436	0\\
437	0\\
438	0\\
439	0\\
440	0\\
441	0\\
442	0\\
443	0\\
444	0\\
445	0\\
446	0\\
447	0\\
448	0\\
449	0\\
450	0\\
451	0\\
452	0\\
453	0\\
454	0\\
455	0\\
456	0\\
457	0\\
458	0\\
459	0\\
460	0\\
461	0\\
462	0\\
463	0\\
464	0\\
465	0\\
466	0\\
467	0\\
468	0\\
469	0\\
470	0\\
471	0\\
472	0\\
473	0\\
474	0\\
475	0\\
476	0\\
477	0\\
478	0\\
479	0\\
480	0\\
481	0\\
482	0\\
483	0\\
484	0\\
485	0\\
486	0\\
487	0\\
488	0\\
489	0\\
490	0\\
491	0\\
492	0\\
493	0\\
494	0\\
495	0\\
496	0\\
497	0\\
498	0\\
499	0\\
500	0\\
501	0\\
502	0\\
503	0\\
504	0\\
505	0\\
506	0\\
507	0\\
508	0\\
509	0\\
510	0\\
511	0\\
512	0\\
513	0\\
514	0\\
515	0\\
516	0\\
517	0\\
518	0\\
519	0\\
520	0\\
521	0\\
522	0\\
523	0\\
524	0\\
525	0\\
526	0\\
527	0\\
528	0\\
529	0\\
530	0\\
531	0\\
532	0\\
533	0\\
534	0\\
535	0\\
536	0\\
537	0\\
538	0\\
539	0\\
540	0\\
541	0\\
542	0\\
543	0\\
544	0\\
545	0\\
546	0\\
547	0\\
548	0\\
549	0\\
550	0\\
551	0\\
552	0\\
553	0\\
554	0\\
555	0\\
556	0\\
557	0\\
558	0\\
559	0\\
560	0\\
561	0\\
562	0\\
563	0\\
564	0\\
565	0\\
566	0\\
567	0\\
568	0\\
569	0\\
570	0\\
571	0\\
572	0\\
573	0\\
574	0\\
575	0\\
576	0\\
577	0\\
578	0\\
579	0\\
580	0\\
581	0\\
582	0\\
583	0\\
584	0\\
585	0\\
586	0\\
587	0\\
588	0\\
589	0\\
590	0\\
591	0\\
592	0\\
593	0\\
594	0\\
595	0\\
596	0\\
597	0\\
598	0\\
599	0\\
600	0\\
};
\addplot [color=black!60!mycolor21,solid,forget plot]
  table[row sep=crcr]{%
1	0\\
2	0\\
3	0\\
4	0\\
5	0\\
6	0\\
7	0\\
8	0\\
9	0\\
10	0\\
11	0\\
12	0\\
13	0\\
14	0\\
15	0\\
16	0\\
17	0\\
18	0\\
19	0\\
20	0\\
21	0\\
22	0\\
23	0\\
24	0\\
25	0\\
26	0\\
27	0\\
28	0\\
29	0\\
30	0\\
31	0\\
32	0\\
33	0\\
34	0\\
35	0\\
36	0\\
37	0\\
38	0\\
39	0\\
40	0\\
41	0\\
42	0\\
43	0\\
44	0\\
45	0\\
46	0\\
47	0\\
48	0\\
49	0\\
50	0\\
51	0\\
52	0\\
53	0\\
54	0\\
55	0\\
56	0\\
57	0\\
58	0\\
59	0\\
60	0\\
61	0\\
62	0\\
63	0\\
64	0\\
65	0\\
66	0\\
67	0\\
68	0\\
69	0\\
70	0\\
71	0\\
72	0\\
73	0\\
74	0\\
75	0\\
76	0\\
77	0\\
78	0\\
79	0\\
80	0\\
81	0\\
82	0\\
83	0\\
84	0\\
85	0\\
86	0\\
87	0\\
88	0\\
89	0\\
90	0\\
91	0\\
92	0\\
93	0\\
94	0\\
95	0\\
96	0\\
97	0\\
98	0\\
99	0\\
100	0\\
101	0\\
102	0\\
103	0\\
104	0\\
105	0\\
106	0\\
107	0\\
108	0\\
109	0\\
110	0\\
111	0\\
112	0\\
113	0\\
114	0\\
115	0\\
116	0\\
117	0\\
118	0\\
119	0\\
120	0\\
121	0\\
122	0\\
123	0\\
124	0\\
125	0\\
126	0\\
127	0\\
128	0\\
129	0\\
130	0\\
131	0\\
132	0\\
133	0\\
134	0\\
135	0\\
136	0\\
137	0\\
138	0\\
139	0\\
140	0\\
141	0\\
142	0\\
143	0\\
144	0\\
145	0\\
146	0\\
147	0\\
148	0\\
149	0\\
150	0\\
151	0\\
152	0\\
153	0\\
154	0\\
155	0\\
156	0\\
157	0\\
158	0\\
159	0\\
160	0\\
161	0\\
162	0\\
163	0\\
164	0\\
165	0\\
166	0\\
167	0\\
168	0\\
169	0\\
170	0\\
171	0\\
172	0\\
173	0\\
174	0\\
175	0\\
176	0\\
177	0\\
178	0\\
179	0\\
180	0\\
181	0\\
182	0\\
183	0\\
184	0\\
185	0\\
186	0\\
187	0\\
188	0\\
189	0\\
190	0\\
191	0\\
192	0\\
193	0\\
194	0\\
195	0\\
196	0\\
197	0\\
198	0\\
199	0\\
200	0\\
201	0\\
202	0\\
203	0\\
204	0\\
205	0\\
206	0\\
207	0\\
208	0\\
209	0\\
210	0\\
211	0\\
212	0\\
213	0\\
214	0\\
215	0\\
216	0\\
217	0\\
218	0\\
219	0\\
220	0\\
221	0\\
222	0\\
223	0\\
224	0\\
225	0\\
226	0\\
227	0\\
228	0\\
229	0\\
230	0\\
231	0\\
232	0\\
233	0\\
234	0\\
235	0\\
236	0\\
237	0\\
238	0\\
239	0\\
240	0\\
241	0\\
242	0\\
243	0\\
244	0\\
245	0\\
246	0\\
247	0\\
248	0\\
249	0\\
250	0\\
251	0\\
252	0\\
253	0\\
254	0\\
255	0\\
256	0\\
257	0\\
258	0\\
259	0\\
260	0\\
261	0\\
262	0\\
263	0\\
264	0\\
265	0\\
266	0\\
267	0\\
268	0\\
269	0\\
270	0\\
271	0\\
272	0\\
273	0\\
274	0\\
275	0\\
276	0\\
277	0\\
278	0\\
279	0\\
280	0\\
281	0\\
282	0\\
283	0\\
284	0\\
285	0\\
286	0\\
287	0\\
288	0\\
289	0\\
290	0\\
291	0\\
292	0\\
293	0\\
294	0\\
295	0\\
296	0\\
297	0\\
298	0\\
299	0\\
300	0\\
301	0\\
302	0\\
303	0\\
304	0\\
305	0\\
306	0\\
307	0\\
308	0\\
309	0\\
310	0\\
311	0\\
312	0\\
313	0\\
314	0\\
315	0\\
316	0\\
317	0\\
318	0\\
319	0\\
320	0\\
321	0\\
322	0\\
323	0\\
324	0\\
325	0\\
326	0\\
327	0\\
328	0\\
329	0\\
330	0\\
331	0\\
332	0\\
333	0\\
334	0\\
335	0\\
336	0\\
337	0\\
338	0\\
339	0\\
340	0\\
341	0\\
342	0\\
343	0\\
344	0\\
345	0\\
346	0\\
347	0\\
348	0\\
349	0\\
350	0\\
351	0\\
352	0\\
353	0\\
354	0\\
355	0\\
356	0\\
357	0\\
358	0\\
359	0\\
360	0\\
361	0\\
362	0\\
363	0\\
364	0\\
365	0\\
366	0\\
367	0\\
368	0\\
369	0\\
370	0\\
371	0\\
372	0\\
373	0\\
374	0\\
375	0\\
376	0\\
377	0\\
378	0\\
379	0\\
380	0\\
381	0\\
382	0\\
383	0\\
384	0\\
385	0\\
386	0\\
387	0\\
388	0\\
389	0\\
390	0\\
391	0\\
392	0\\
393	0\\
394	0\\
395	0\\
396	0\\
397	0\\
398	0\\
399	0\\
400	0\\
401	0\\
402	0\\
403	0\\
404	0\\
405	0\\
406	0\\
407	0\\
408	0\\
409	0\\
410	0\\
411	0\\
412	0\\
413	0\\
414	0\\
415	0\\
416	0\\
417	0\\
418	0\\
419	0\\
420	0\\
421	0\\
422	0\\
423	0\\
424	0\\
425	0\\
426	0\\
427	0\\
428	0\\
429	0\\
430	0\\
431	0\\
432	0\\
433	0\\
434	0\\
435	0\\
436	0\\
437	0\\
438	0\\
439	0\\
440	0\\
441	0\\
442	0\\
443	0\\
444	0\\
445	0\\
446	0\\
447	0\\
448	0\\
449	0\\
450	0\\
451	0\\
452	0\\
453	0\\
454	0\\
455	0\\
456	0\\
457	0\\
458	0\\
459	0\\
460	0\\
461	0\\
462	0\\
463	0\\
464	0\\
465	0\\
466	0\\
467	0\\
468	0\\
469	0\\
470	0\\
471	0\\
472	0\\
473	0\\
474	0\\
475	0\\
476	0\\
477	0\\
478	0\\
479	0\\
480	0\\
481	0\\
482	0\\
483	0\\
484	0\\
485	0\\
486	0\\
487	0\\
488	0\\
489	0\\
490	0\\
491	0\\
492	0\\
493	0\\
494	0\\
495	0\\
496	0\\
497	0\\
498	0\\
499	0\\
500	0\\
501	0\\
502	0\\
503	0\\
504	0\\
505	0\\
506	0\\
507	0\\
508	0\\
509	0\\
510	0\\
511	0\\
512	0\\
513	0\\
514	0\\
515	0\\
516	0\\
517	0\\
518	0\\
519	0\\
520	0\\
521	0\\
522	0\\
523	0\\
524	0\\
525	0\\
526	0\\
527	0\\
528	0\\
529	0\\
530	0\\
531	0\\
532	0\\
533	0\\
534	0\\
535	0\\
536	0\\
537	0\\
538	0\\
539	0\\
540	0\\
541	0\\
542	0\\
543	0\\
544	0\\
545	0\\
546	0\\
547	0\\
548	0\\
549	0\\
550	0\\
551	0\\
552	0\\
553	0\\
554	0\\
555	0\\
556	0\\
557	0\\
558	0\\
559	0\\
560	0\\
561	0\\
562	0\\
563	0\\
564	0\\
565	0\\
566	0\\
567	0\\
568	0\\
569	0\\
570	0\\
571	0\\
572	0\\
573	0\\
574	0\\
575	0\\
576	0\\
577	0\\
578	0\\
579	0\\
580	0\\
581	0\\
582	0\\
583	0\\
584	0\\
585	0\\
586	0\\
587	0\\
588	0\\
589	0\\
590	0\\
591	0\\
592	0\\
593	0\\
594	0\\
595	0\\
596	0\\
597	0\\
598	0\\
599	0\\
600	0\\
};
\addplot [color=black!80!mycolor21,solid,forget plot]
  table[row sep=crcr]{%
1	0\\
2	0\\
3	0\\
4	0\\
5	0\\
6	0\\
7	0\\
8	0\\
9	0\\
10	0\\
11	0\\
12	0\\
13	0\\
14	0\\
15	0\\
16	0\\
17	0\\
18	0\\
19	0\\
20	0\\
21	0\\
22	0\\
23	0\\
24	0\\
25	0\\
26	0\\
27	0\\
28	0\\
29	0\\
30	0\\
31	0\\
32	0\\
33	0\\
34	0\\
35	0\\
36	0\\
37	0\\
38	0\\
39	0\\
40	0\\
41	0\\
42	0\\
43	0\\
44	0\\
45	0\\
46	0\\
47	0\\
48	0\\
49	0\\
50	0\\
51	0\\
52	0\\
53	0\\
54	0\\
55	0\\
56	0\\
57	0\\
58	0\\
59	0\\
60	0\\
61	0\\
62	0\\
63	0\\
64	0\\
65	0\\
66	0\\
67	0\\
68	0\\
69	0\\
70	0\\
71	0\\
72	0\\
73	0\\
74	0\\
75	0\\
76	0\\
77	0\\
78	0\\
79	0\\
80	0\\
81	0\\
82	0\\
83	0\\
84	0\\
85	0\\
86	0\\
87	0\\
88	0\\
89	0\\
90	0\\
91	0\\
92	0\\
93	0\\
94	0\\
95	0\\
96	0\\
97	0\\
98	0\\
99	0\\
100	0\\
101	0\\
102	0\\
103	0\\
104	0\\
105	0\\
106	0\\
107	0\\
108	0\\
109	0\\
110	0\\
111	0\\
112	0\\
113	0\\
114	0\\
115	0\\
116	0\\
117	0\\
118	0\\
119	0\\
120	0\\
121	0\\
122	0\\
123	0\\
124	0\\
125	0\\
126	0\\
127	0\\
128	0\\
129	0\\
130	0\\
131	0\\
132	0\\
133	0\\
134	0\\
135	0\\
136	0\\
137	0\\
138	0\\
139	0\\
140	0\\
141	0\\
142	0\\
143	0\\
144	0\\
145	0\\
146	0\\
147	0\\
148	0\\
149	0\\
150	0\\
151	0\\
152	0\\
153	0\\
154	0\\
155	0\\
156	0\\
157	0\\
158	0\\
159	0\\
160	0\\
161	0\\
162	0\\
163	0\\
164	0\\
165	0\\
166	0\\
167	0\\
168	0\\
169	0\\
170	0\\
171	0\\
172	0\\
173	0\\
174	0\\
175	0\\
176	0\\
177	0\\
178	0\\
179	0\\
180	0\\
181	0\\
182	0\\
183	0\\
184	0\\
185	0\\
186	0\\
187	0\\
188	0\\
189	0\\
190	0\\
191	0\\
192	0\\
193	0\\
194	0\\
195	0\\
196	0\\
197	0\\
198	0\\
199	0\\
200	0\\
201	0\\
202	0\\
203	0\\
204	0\\
205	0\\
206	0\\
207	0\\
208	0\\
209	0\\
210	0\\
211	0\\
212	0\\
213	0\\
214	0\\
215	0\\
216	0\\
217	0\\
218	0\\
219	0\\
220	0\\
221	0\\
222	0\\
223	0\\
224	0\\
225	0\\
226	0\\
227	0\\
228	0\\
229	0\\
230	0\\
231	0\\
232	0\\
233	0\\
234	0\\
235	0\\
236	0\\
237	0\\
238	0\\
239	0\\
240	0\\
241	0\\
242	0\\
243	0\\
244	0\\
245	0\\
246	0\\
247	0\\
248	0\\
249	0\\
250	0\\
251	0\\
252	0\\
253	0\\
254	0\\
255	0\\
256	0\\
257	0\\
258	0\\
259	0\\
260	0\\
261	0\\
262	0\\
263	0\\
264	0\\
265	0\\
266	0\\
267	0\\
268	0\\
269	0\\
270	0\\
271	0\\
272	0\\
273	0\\
274	0\\
275	0\\
276	0\\
277	0\\
278	0\\
279	0\\
280	0\\
281	0\\
282	0\\
283	0\\
284	0\\
285	0\\
286	0\\
287	0\\
288	0\\
289	0\\
290	0\\
291	0\\
292	0\\
293	0\\
294	0\\
295	0\\
296	0\\
297	0\\
298	0\\
299	0\\
300	0\\
301	0\\
302	0\\
303	0\\
304	0\\
305	0\\
306	0\\
307	0\\
308	0\\
309	0\\
310	0\\
311	0\\
312	0\\
313	0\\
314	0\\
315	0\\
316	0\\
317	0\\
318	0\\
319	0\\
320	0\\
321	0\\
322	0\\
323	0\\
324	0\\
325	0\\
326	0\\
327	0\\
328	0\\
329	0\\
330	0\\
331	0\\
332	0\\
333	0\\
334	0\\
335	0\\
336	0\\
337	0\\
338	0\\
339	0\\
340	0\\
341	0\\
342	0\\
343	0\\
344	0\\
345	0\\
346	0\\
347	0\\
348	0\\
349	0\\
350	0\\
351	0\\
352	0\\
353	0\\
354	0\\
355	0\\
356	0\\
357	0\\
358	0\\
359	0\\
360	0\\
361	0\\
362	0\\
363	0\\
364	0\\
365	0\\
366	0\\
367	0\\
368	0\\
369	0\\
370	0\\
371	0\\
372	0\\
373	0\\
374	0\\
375	0\\
376	0\\
377	0\\
378	0\\
379	0\\
380	0\\
381	0\\
382	0\\
383	0\\
384	0\\
385	0\\
386	0\\
387	0\\
388	0\\
389	0\\
390	0\\
391	0\\
392	0\\
393	0\\
394	0\\
395	0\\
396	0\\
397	0\\
398	0\\
399	0\\
400	0\\
401	0\\
402	0\\
403	0\\
404	0\\
405	0\\
406	0\\
407	0\\
408	0\\
409	0\\
410	0\\
411	0\\
412	0\\
413	0\\
414	0\\
415	0\\
416	0\\
417	0\\
418	0\\
419	0\\
420	0\\
421	0\\
422	0\\
423	0\\
424	0\\
425	0\\
426	0\\
427	0\\
428	0\\
429	0\\
430	0\\
431	0\\
432	0\\
433	0\\
434	0\\
435	0\\
436	0\\
437	0\\
438	0\\
439	0\\
440	0\\
441	0\\
442	0\\
443	0\\
444	0\\
445	0\\
446	0\\
447	0\\
448	0\\
449	0\\
450	0\\
451	0\\
452	0\\
453	0\\
454	0\\
455	0\\
456	0\\
457	0\\
458	0\\
459	0\\
460	0\\
461	0\\
462	0\\
463	0\\
464	0\\
465	0\\
466	0\\
467	0\\
468	0\\
469	0\\
470	0\\
471	0\\
472	0\\
473	0\\
474	0\\
475	0\\
476	0\\
477	0\\
478	0\\
479	0\\
480	0\\
481	0\\
482	0\\
483	0\\
484	0\\
485	0\\
486	0\\
487	0\\
488	0\\
489	0\\
490	0\\
491	0\\
492	0\\
493	0\\
494	0\\
495	0\\
496	0\\
497	0\\
498	0\\
499	0\\
500	0\\
501	0\\
502	0\\
503	0\\
504	0\\
505	0\\
506	0\\
507	0\\
508	0\\
509	0\\
510	0\\
511	0\\
512	0\\
513	0\\
514	0\\
515	0\\
516	0\\
517	0\\
518	0\\
519	0\\
520	0\\
521	0\\
522	0\\
523	0\\
524	0\\
525	0\\
526	0\\
527	0\\
528	0\\
529	0\\
530	0\\
531	0\\
532	0\\
533	0\\
534	0\\
535	0\\
536	0\\
537	0\\
538	0\\
539	0\\
540	0\\
541	0\\
542	0\\
543	0\\
544	0\\
545	0\\
546	0\\
547	0\\
548	0\\
549	0\\
550	0\\
551	0\\
552	0\\
553	0\\
554	0\\
555	0\\
556	0\\
557	0\\
558	0\\
559	0\\
560	0\\
561	0\\
562	0\\
563	0\\
564	0\\
565	0\\
566	0\\
567	0\\
568	0\\
569	0\\
570	0\\
571	0\\
572	0\\
573	0\\
574	0\\
575	0\\
576	0\\
577	0\\
578	0\\
579	0\\
580	0\\
581	0\\
582	0\\
583	0\\
584	0\\
585	0\\
586	0\\
587	0\\
588	0\\
589	0\\
590	0\\
591	0\\
592	0\\
593	0\\
594	0\\
595	0\\
596	0\\
597	0\\
598	0\\
599	0\\
600	0\\
};
\addplot [color=black,solid,forget plot]
  table[row sep=crcr]{%
1	0\\
2	0\\
3	0\\
4	0\\
5	0\\
6	0\\
7	0\\
8	0\\
9	0\\
10	0\\
11	0\\
12	0\\
13	0\\
14	0\\
15	0\\
16	0\\
17	0\\
18	0\\
19	0\\
20	0\\
21	0\\
22	0\\
23	0\\
24	0\\
25	0\\
26	0\\
27	0\\
28	0\\
29	0\\
30	0\\
31	0\\
32	0\\
33	0\\
34	0\\
35	0\\
36	0\\
37	0\\
38	0\\
39	0\\
40	0\\
41	0\\
42	0\\
43	0\\
44	0\\
45	0\\
46	0\\
47	0\\
48	0\\
49	0\\
50	0\\
51	0\\
52	0\\
53	0\\
54	0\\
55	0\\
56	0\\
57	0\\
58	0\\
59	0\\
60	0\\
61	0\\
62	0\\
63	0\\
64	0\\
65	0\\
66	0\\
67	0\\
68	0\\
69	0\\
70	0\\
71	0\\
72	0\\
73	0\\
74	0\\
75	0\\
76	0\\
77	0\\
78	0\\
79	0\\
80	0\\
81	0\\
82	0\\
83	0\\
84	0\\
85	0\\
86	0\\
87	0\\
88	0\\
89	0\\
90	0\\
91	0\\
92	0\\
93	0\\
94	0\\
95	0\\
96	0\\
97	0\\
98	0\\
99	0\\
100	0\\
101	0\\
102	0\\
103	0\\
104	0\\
105	0\\
106	0\\
107	0\\
108	0\\
109	0\\
110	0\\
111	0\\
112	0\\
113	0\\
114	0\\
115	0\\
116	0\\
117	0\\
118	0\\
119	0\\
120	0\\
121	0\\
122	0\\
123	0\\
124	0\\
125	0\\
126	0\\
127	0\\
128	0\\
129	0\\
130	0\\
131	0\\
132	0\\
133	0\\
134	0\\
135	0\\
136	0\\
137	0\\
138	0\\
139	0\\
140	0\\
141	0\\
142	0\\
143	0\\
144	0\\
145	0\\
146	0\\
147	0\\
148	0\\
149	0\\
150	0\\
151	0\\
152	0\\
153	0\\
154	0\\
155	0\\
156	0\\
157	0\\
158	0\\
159	0\\
160	0\\
161	0\\
162	0\\
163	0\\
164	0\\
165	0\\
166	0\\
167	0\\
168	0\\
169	0\\
170	0\\
171	0\\
172	0\\
173	0\\
174	0\\
175	0\\
176	0\\
177	0\\
178	0\\
179	0\\
180	0\\
181	0\\
182	0\\
183	0\\
184	0\\
185	0\\
186	0\\
187	0\\
188	0\\
189	0\\
190	0\\
191	0\\
192	0\\
193	0\\
194	0\\
195	0\\
196	0\\
197	0\\
198	0\\
199	0\\
200	0\\
201	0\\
202	0\\
203	0\\
204	0\\
205	0\\
206	0\\
207	0\\
208	0\\
209	0\\
210	0\\
211	0\\
212	0\\
213	0\\
214	0\\
215	0\\
216	0\\
217	0\\
218	0\\
219	0\\
220	0\\
221	0\\
222	0\\
223	0\\
224	0\\
225	0\\
226	0\\
227	0\\
228	0\\
229	0\\
230	0\\
231	0\\
232	0\\
233	0\\
234	0\\
235	0\\
236	0\\
237	0\\
238	0\\
239	0\\
240	0\\
241	0\\
242	0\\
243	0\\
244	0\\
245	0\\
246	0\\
247	0\\
248	0\\
249	0\\
250	0\\
251	0\\
252	0\\
253	0\\
254	0\\
255	0\\
256	0\\
257	0\\
258	0\\
259	0\\
260	0\\
261	0\\
262	0\\
263	0\\
264	0\\
265	0\\
266	0\\
267	0\\
268	0\\
269	0\\
270	0\\
271	0\\
272	0\\
273	0\\
274	0\\
275	0\\
276	0\\
277	0\\
278	0\\
279	0\\
280	0\\
281	0\\
282	0\\
283	0\\
284	0\\
285	0\\
286	0\\
287	0\\
288	0\\
289	0\\
290	0\\
291	0\\
292	0\\
293	0\\
294	0\\
295	0\\
296	0\\
297	0\\
298	0\\
299	0\\
300	0\\
301	0\\
302	0\\
303	0\\
304	0\\
305	0\\
306	0\\
307	0\\
308	0\\
309	0\\
310	0\\
311	0\\
312	0\\
313	0\\
314	0\\
315	0\\
316	0\\
317	0\\
318	0\\
319	0\\
320	0\\
321	0\\
322	0\\
323	0\\
324	0\\
325	0\\
326	0\\
327	0\\
328	0\\
329	0\\
330	0\\
331	0\\
332	0\\
333	0\\
334	0\\
335	0\\
336	0\\
337	0\\
338	0\\
339	0\\
340	0\\
341	0\\
342	0\\
343	0\\
344	0\\
345	0\\
346	0\\
347	0\\
348	0\\
349	0\\
350	0\\
351	0\\
352	0\\
353	0\\
354	0\\
355	0\\
356	0\\
357	0\\
358	0\\
359	0\\
360	0\\
361	0\\
362	0\\
363	0\\
364	0\\
365	0\\
366	0\\
367	0\\
368	0\\
369	0\\
370	0\\
371	0\\
372	0\\
373	0\\
374	0\\
375	0\\
376	0\\
377	0\\
378	0\\
379	0\\
380	0\\
381	0\\
382	0\\
383	0\\
384	0\\
385	0\\
386	0\\
387	0\\
388	0\\
389	0\\
390	0\\
391	0\\
392	0\\
393	0\\
394	0\\
395	0\\
396	0\\
397	0\\
398	0\\
399	0\\
400	0\\
401	0\\
402	0\\
403	0\\
404	0\\
405	0\\
406	0\\
407	0\\
408	0\\
409	0\\
410	0\\
411	0\\
412	0\\
413	0\\
414	0\\
415	0\\
416	0\\
417	0\\
418	0\\
419	0\\
420	0\\
421	0\\
422	0\\
423	0\\
424	0\\
425	0\\
426	0\\
427	0\\
428	0\\
429	0\\
430	0\\
431	0\\
432	0\\
433	0\\
434	0\\
435	0\\
436	0\\
437	0\\
438	0\\
439	0\\
440	0\\
441	0\\
442	0\\
443	0\\
444	0\\
445	0\\
446	0\\
447	0\\
448	0\\
449	0\\
450	0\\
451	0\\
452	0\\
453	0\\
454	0\\
455	0\\
456	0\\
457	0\\
458	0\\
459	0\\
460	0\\
461	0\\
462	0\\
463	0\\
464	0\\
465	0\\
466	0\\
467	0\\
468	0\\
469	0\\
470	0\\
471	0\\
472	0\\
473	0\\
474	0\\
475	0\\
476	0\\
477	0\\
478	0\\
479	0\\
480	0\\
481	0\\
482	0\\
483	0\\
484	0\\
485	0\\
486	0\\
487	0\\
488	0\\
489	0\\
490	0\\
491	0\\
492	0\\
493	0\\
494	0\\
495	0\\
496	0\\
497	0\\
498	0\\
499	0\\
500	0\\
501	0\\
502	0\\
503	0\\
504	0\\
505	0\\
506	0\\
507	0\\
508	0\\
509	0\\
510	0\\
511	0\\
512	0\\
513	0\\
514	0\\
515	0\\
516	0\\
517	0\\
518	0\\
519	0\\
520	0\\
521	0\\
522	0\\
523	0\\
524	0\\
525	0\\
526	0\\
527	0\\
528	0\\
529	0\\
530	0\\
531	0\\
532	0\\
533	0\\
534	0\\
535	0\\
536	0\\
537	0\\
538	0\\
539	0\\
540	0\\
541	0\\
542	0\\
543	0\\
544	0\\
545	0\\
546	0\\
547	0\\
548	0\\
549	0\\
550	0\\
551	0\\
552	0\\
553	0\\
554	0\\
555	0\\
556	0\\
557	0\\
558	0\\
559	0\\
560	0\\
561	0\\
562	0\\
563	0\\
564	0\\
565	0\\
566	0\\
567	0\\
568	0\\
569	0\\
570	0\\
571	0\\
572	0\\
573	0\\
574	0\\
575	0\\
576	0\\
577	0\\
578	0\\
579	0\\
580	0\\
581	0\\
582	0\\
583	0\\
584	0\\
585	0\\
586	0\\
587	0\\
588	0\\
589	0\\
590	0\\
591	0\\
592	0\\
593	0\\
594	0\\
595	0\\
596	0\\
597	0\\
598	0\\
599	0\\
600	0\\
};
\end{axis}
\end{tikzpicture}%
 
  \caption{Discrete Time w/ nFPC}
\end{subfigure}\\

\leavevmode\smash{\makebox[0pt]{\hspace{-7em}% HORIZONTAL POSITION           
  \rotatebox[origin=l]{90}{\hspace{20em}% VERTICAL POSITION
    Depth $\delta^-$}%
}}\hspace{0pt plus 1filll}\null

Time (s)

\vspace{1cm}
\begin{subfigure}{\linewidth}
  \centering
  \tikzsetnextfilename{deltalegend}
  \documentclass{article}
\usepackage{pgfplots}
\usetikzlibrary{backgrounds}
\pgfplotsset{compat=newest}  
\newlength\figureheight 
\newlength\figurewidth 

\begin{document}
%
%\begin{figure}
%  \centering
%  \setlength\figureheight{\linewidth} 
%  \setlength\figurewidth{\linewidth}
%  \input{/home/anton/Documents/masc/ml/thesis/tikz/ORCL_comp4stoch.tikz}
%  \caption{Backtest strategy comparison}
%  \label{fig:insample}
%\end{figure}
\definecolor{mycolor1}{rgb}{1.00000,0.00000,1.00000}%
\begin{tikzpicture}[framed]
    \begingroup
    % inits/clears the lists (which might be populated from previous
    % axes):
    \csname pgfplots@init@cleared@structures\endcsname
    \pgfplotsset{legend style={at={(0,1)},anchor=north west},legend columns=-1,legend style={draw=none,column sep=1ex},legend entries={$q=-4$,$q=-3$,$q=-2$,$q=-1$}}%
    
    \csname pgfplots@addlegendimage\endcsname{thick,green,dashed,sharp plot}
    \csname pgfplots@addlegendimage\endcsname{thick,mycolor1,dashed,sharp plot}
    \csname pgfplots@addlegendimage\endcsname{thick,red,dashed,sharp plot}
    \csname pgfplots@addlegendimage\endcsname{thick,blue,dashed,sharp plot}

    % draws the legend:
    \csname pgfplots@createlegend\endcsname
    \endgroup

    \begingroup
    % inits/clears the lists (which might be populated from previous
    % axes):
    \csname pgfplots@init@cleared@structures\endcsname
    \pgfplotsset{legend style={at={(3.45,0.5)},anchor=north west},legend columns=-1,legend style={draw=none,column sep=1ex},legend entries={$q=0$}}%

    \csname pgfplots@addlegendimage\endcsname{thick,black,sharp plot}

    % draws the legend:
    \csname pgfplots@createlegend\endcsname
    \endgroup

    \begingroup
    % inits/clears the lists (which might be populated from previous
    % axes):
    \csname pgfplots@init@cleared@structures\endcsname
    \pgfplotsset{legend style={at={(0,0)},anchor=north west},legend columns=-1,legend style={draw=none,column sep=1ex},legend entries={$q=+4$,$q=+3$,$q=+2$,$q=+1$}}%
    
    \csname pgfplots@addlegendimage\endcsname{thick,green,sharp plot}
    \csname pgfplots@addlegendimage\endcsname{thick,mycolor1,sharp plot}
    \csname pgfplots@addlegendimage\endcsname{thick,red,sharp plot}
    \csname pgfplots@addlegendimage\endcsname{thick,blue,sharp plot}

    % draws the legend:
    \csname pgfplots@createlegend\endcsname
    \endgroup
\end{tikzpicture}

\end{document} 
\end{subfigure}%
  \caption{Optimal sell depths $\delta^{-}$ for Markov state $Z=(\rho = -1, \Delta S = -1)$, implying heavy imbalance in favor of sell pressure, and having previously seen a downward price change. We expect the midprice to fall.}
  \label{fig:comp_dm_z1}
\end{figure}

\begin{figure}
\centering
\begin{subfigure}{.45\linewidth}
  \centering
  \setlength\figureheight{\linewidth} 
  \setlength\figurewidth{\linewidth}
  \tikzsetnextfilename{dm_cts_z8}
  % This file was created by matlab2tikz.
%
%The latest updates can be retrieved from
%  http://www.mathworks.com/matlabcentral/fileexchange/22022-matlab2tikz-matlab2tikz
%where you can also make suggestions and rate matlab2tikz.
%
\definecolor{mycolor1}{rgb}{0.00000,1.00000,0.14286}%
\definecolor{mycolor2}{rgb}{0.00000,1.00000,0.28571}%
\definecolor{mycolor3}{rgb}{0.00000,1.00000,0.42857}%
\definecolor{mycolor4}{rgb}{0.00000,1.00000,0.57143}%
\definecolor{mycolor5}{rgb}{0.00000,1.00000,0.71429}%
\definecolor{mycolor6}{rgb}{0.00000,1.00000,0.85714}%
\definecolor{mycolor7}{rgb}{0.00000,1.00000,1.00000}%
\definecolor{mycolor8}{rgb}{0.00000,0.87500,1.00000}%
\definecolor{mycolor9}{rgb}{0.00000,0.62500,1.00000}%
\definecolor{mycolor10}{rgb}{0.12500,0.00000,1.00000}%
\definecolor{mycolor11}{rgb}{0.25000,0.00000,1.00000}%
\definecolor{mycolor12}{rgb}{0.37500,0.00000,1.00000}%
\definecolor{mycolor13}{rgb}{0.50000,0.00000,1.00000}%
\definecolor{mycolor14}{rgb}{0.62500,0.00000,1.00000}%
\definecolor{mycolor15}{rgb}{0.75000,0.00000,1.00000}%
\definecolor{mycolor16}{rgb}{0.87500,0.00000,1.00000}%
\definecolor{mycolor17}{rgb}{1.00000,0.00000,1.00000}%
\definecolor{mycolor18}{rgb}{1.00000,0.00000,0.87500}%
\definecolor{mycolor19}{rgb}{1.00000,0.00000,0.62500}%
\definecolor{mycolor20}{rgb}{0.85714,0.00000,0.00000}%
\definecolor{mycolor21}{rgb}{0.71429,0.00000,0.00000}%
%
\begin{tikzpicture}[trim axis left, trim axis right]

\begin{axis}[%
width=\figurewidth,
height=\figureheight,
at={(0\figurewidth,0\figureheight)},
scale only axis,
every outer x axis line/.append style={black},
every x tick label/.append style={font=\color{black}},
xmin=0,
xmax=600,
every outer y axis line/.append style={black},
every y tick label/.append style={font=\color{black}},
ymin=0,
ymax=0.014,
axis background/.style={fill=white},
axis x line*=bottom,
axis y line*=left,
yticklabel style={
        /pgf/number format/fixed,
        /pgf/number format/precision=3
},
scaled y ticks=false
]
\addplot [color=green,solid,forget plot]
  table[row sep=crcr]{%
0.01	0.00503700863926798\\
1.01	0.0050370095564963\\
2.01	0.00503701049210395\\
3.01	0.0050370114464563\\
4.01	0.00503701241992682\\
5.01	0.00503701341289536\\
6.01	0.00503701442575027\\
7.01	0.00503701545888663\\
8.01	0.00503701651270766\\
9.01	0.00503701758762468\\
10.01	0.00503701868405695\\
11.01	0.00503701980243221\\
12.01	0.00503702094318642\\
13.01	0.00503702210676399\\
14.01	0.00503702329361844\\
15.01	0.00503702450421181\\
16.01	0.0050370257390156\\
17.01	0.00503702699851016\\
18.01	0.00503702828318539\\
19.01	0.00503702959354146\\
20.01	0.00503703093008762\\
21.01	0.00503703229334332\\
22.01	0.00503703368383834\\
23.01	0.00503703510211298\\
24.01	0.00503703654871751\\
25.01	0.00503703802421365\\
26.01	0.00503703952917398\\
27.01	0.00503704106418201\\
28.01	0.00503704262983323\\
29.01	0.00503704422673434\\
30.01	0.00503704585550413\\
31.01	0.00503704751677359\\
32.01	0.00503704921118597\\
33.01	0.00503705093939702\\
34.01	0.00503705270207538\\
35.01	0.00503705449990291\\
36.01	0.00503705633357451\\
37.01	0.0050370582037991\\
38.01	0.00503706011129942\\
39.01	0.00503706205681168\\
40.01	0.00503706404108728\\
41.01	0.00503706606489202\\
42.01	0.00503706812900628\\
43.01	0.00503707023422634\\
44.01	0.00503707238136358\\
45.01	0.0050370745712455\\
46.01	0.00503707680471538\\
47.01	0.00503707908263312\\
48.01	0.00503708140587569\\
49.01	0.00503708377533685\\
50.01	0.00503708619192778\\
51.01	0.00503708865657709\\
52.01	0.0050370911702326\\
53.01	0.00503709373385891\\
54.01	0.00503709634844085\\
55.01	0.00503709901498176\\
56.01	0.00503710173450458\\
57.01	0.00503710450805214\\
58.01	0.00503710733668758\\
59.01	0.00503711022149461\\
60.01	0.00503711316357836\\
61.01	0.00503711616406474\\
62.01	0.00503711922410222\\
63.01	0.00503712234486101\\
64.01	0.00503712552753433\\
65.01	0.0050371287733387\\
66.01	0.00503713208351358\\
67.01	0.00503713545932324\\
68.01	0.00503713890205601\\
69.01	0.00503714241302494\\
70.01	0.00503714599356911\\
71.01	0.00503714964505306\\
72.01	0.00503715336886795\\
73.01	0.00503715716643199\\
74.01	0.00503716103919033\\
75.01	0.00503716498861637\\
76.01	0.00503716901621196\\
77.01	0.00503717312350804\\
78.01	0.00503717731206504\\
79.01	0.00503718158347313\\
80.01	0.00503718593935376\\
81.01	0.00503719038135941\\
82.01	0.00503719491117427\\
83.01	0.00503719953051524\\
84.01	0.00503720424113244\\
85.01	0.00503720904480873\\
86.01	0.00503721394336243\\
87.01	0.00503721893864667\\
88.01	0.00503722403254947\\
89.01	0.00503722922699576\\
90.01	0.00503723452394766\\
91.01	0.00503723992540453\\
92.01	0.00503724543340422\\
93.01	0.00503725105002359\\
94.01	0.00503725677737982\\
95.01	0.00503726261763057\\
96.01	0.00503726857297443\\
97.01	0.0050372746456524\\
98.01	0.00503728083794855\\
99.01	0.00503728715219068\\
100.01	0.00503729359075072\\
101.01	0.00503730015604673\\
102.01	0.00503730685054238\\
103.01	0.00503731367674878\\
104.01	0.00503732063722473\\
105.01	0.00503732773457818\\
106.01	0.00503733497146674\\
107.01	0.00503734235059897\\
108.01	0.00503734987473498\\
109.01	0.00503735754668719\\
110.01	0.00503736536932195\\
111.01	0.00503737334556062\\
112.01	0.00503738147837934\\
113.01	0.00503738977081139\\
114.01	0.00503739822594776\\
115.01	0.00503740684693799\\
116.01	0.00503741563699175\\
117.01	0.00503742459937948\\
118.01	0.00503743373743401\\
119.01	0.00503744305455114\\
120.01	0.00503745255419133\\
121.01	0.00503746223988039\\
122.01	0.00503747211521105\\
123.01	0.00503748218384462\\
124.01	0.00503749244951107\\
125.01	0.00503750291601124\\
126.01	0.0050375135872181\\
127.01	0.00503752446707711\\
128.01	0.00503753555960904\\
129.01	0.00503754686891023\\
130.01	0.00503755839915435\\
131.01	0.00503757015459461\\
132.01	0.00503758213956329\\
133.01	0.00503759435847503\\
134.01	0.00503760681582774\\
135.01	0.00503761951620343\\
136.01	0.00503763246427077\\
137.01	0.00503764566478632\\
138.01	0.00503765912259596\\
139.01	0.00503767284263604\\
140.01	0.00503768682993702\\
141.01	0.00503770108962251\\
142.01	0.0050377156269131\\
143.01	0.00503773044712685\\
144.01	0.00503774555568146\\
145.01	0.00503776095809616\\
146.01	0.00503777665999397\\
147.01	0.00503779266710264\\
148.01	0.00503780898525674\\
149.01	0.00503782562040073\\
150.01	0.00503784257858947\\
151.01	0.00503785986599095\\
152.01	0.00503787748888843\\
153.01	0.00503789545368261\\
154.01	0.00503791376689258\\
155.01	0.00503793243515976\\
156.01	0.00503795146524905\\
157.01	0.005037970864051\\
158.01	0.00503799063858401\\
159.01	0.00503801079599758\\
160.01	0.00503803134357367\\
161.01	0.00503805228872931\\
162.01	0.00503807363901934\\
163.01	0.00503809540213851\\
164.01	0.00503811758592421\\
165.01	0.00503814019835902\\
166.01	0.00503816324757281\\
167.01	0.00503818674184675\\
168.01	0.00503821068961455\\
169.01	0.00503823509946524\\
170.01	0.00503825998014745\\
171.01	0.00503828534057066\\
172.01	0.0050383111898083\\
173.01	0.00503833753710173\\
174.01	0.00503836439186212\\
175.01	0.00503839176367404\\
176.01	0.00503841966229843\\
177.01	0.00503844809767484\\
178.01	0.00503847707992691\\
179.01	0.00503850661936314\\
180.01	0.00503853672648163\\
181.01	0.00503856741197301\\
182.01	0.00503859868672335\\
183.01	0.00503863056181852\\
184.01	0.00503866304854753\\
185.01	0.00503869615840566\\
186.01	0.00503872990309845\\
187.01	0.00503876429454511\\
188.01	0.00503879934488299\\
189.01	0.00503883506647076\\
190.01	0.00503887147189222\\
191.01	0.00503890857396075\\
192.01	0.00503894638572358\\
193.01	0.00503898492046473\\
194.01	0.00503902419171071\\
195.01	0.00503906421323363\\
196.01	0.00503910499905608\\
197.01	0.00503914656345553\\
198.01	0.00503918892096808\\
199.01	0.00503923208639412\\
200.01	0.0050392760748024\\
201.01	0.0050393209015345\\
202.01	0.00503936658221077\\
203.01	0.00503941313273334\\
204.01	0.00503946056929269\\
205.01	0.0050395089083721\\
206.01	0.00503955816675239\\
207.01	0.00503960836151857\\
208.01	0.00503965951006308\\
209.01	0.0050397116300932\\
210.01	0.00503976473963538\\
211.01	0.00503981885704116\\
212.01	0.00503987400099298\\
213.01	0.00503993019050977\\
214.01	0.00503998744495318\\
215.01	0.00504004578403339\\
216.01	0.00504010522781573\\
217.01	0.00504016579672581\\
218.01	0.00504022751155711\\
219.01	0.00504029039347662\\
220.01	0.00504035446403217\\
221.01	0.00504041974515843\\
222.01	0.00504048625918432\\
223.01	0.00504055402883963\\
224.01	0.00504062307726214\\
225.01	0.00504069342800531\\
226.01	0.005040765105045\\
227.01	0.0050408381327873\\
228.01	0.00504091253607636\\
229.01	0.00504098834020186\\
230.01	0.0050410655709073\\
231.01	0.00504114425439822\\
232.01	0.00504122441734989\\
233.01	0.00504130608691561\\
234.01	0.0050413892907368\\
235.01	0.00504147405694988\\
236.01	0.00504156041419651\\
237.01	0.00504164839163246\\
238.01	0.00504173801893631\\
239.01	0.00504182932631966\\
240.01	0.00504192234453665\\
241.01	0.00504201710489405\\
242.01	0.00504211363926055\\
243.01	0.00504221198007862\\
244.01	0.00504231216037326\\
245.01	0.00504241421376425\\
246.01	0.00504251817447671\\
247.01	0.00504262407735151\\
248.01	0.00504273195785787\\
249.01	0.00504284185210524\\
250.01	0.00504295379685413\\
251.01	0.00504306782952962\\
252.01	0.00504318398823272\\
253.01	0.00504330231175437\\
254.01	0.00504342283958797\\
255.01	0.00504354561194312\\
256.01	0.00504367066975919\\
257.01	0.00504379805471941\\
258.01	0.00504392780926499\\
259.01	0.00504405997661062\\
260.01	0.00504419460075892\\
261.01	0.00504433172651703\\
262.01	0.00504447139951025\\
263.01	0.00504461366620026\\
264.01	0.00504475857390158\\
265.01	0.0050449061707972\\
266.01	0.00504505650595788\\
267.01	0.00504520962935875\\
268.01	0.00504536559189853\\
269.01	0.00504552444541745\\
270.01	0.00504568624271717\\
271.01	0.0050458510375806\\
272.01	0.00504601888479148\\
273.01	0.00504618984015631\\
274.01	0.00504636396052505\\
275.01	0.00504654130381252\\
276.01	0.00504672192902139\\
277.01	0.00504690589626572\\
278.01	0.00504709326679303\\
279.01	0.00504728410301017\\
280.01	0.0050474784685072\\
281.01	0.00504767642808349\\
282.01	0.00504787804777293\\
283.01	0.00504808339487145\\
284.01	0.00504829253796438\\
285.01	0.00504850554695443\\
286.01	0.00504872249309057\\
287.01	0.00504894344899795\\
288.01	0.00504916848870818\\
289.01	0.00504939768769\\
290.01	0.00504963112288198\\
291.01	0.00504986887272421\\
292.01	0.00505011101719227\\
293.01	0.00505035763783143\\
294.01	0.00505060881779185\\
295.01	0.00505086464186458\\
296.01	0.00505112519651726\\
297.01	0.00505139056993306\\
298.01	0.00505166085204839\\
299.01	0.00505193613459221\\
300.01	0.00505221651112603\\
301.01	0.00505250207708521\\
302.01	0.00505279292981986\\
303.01	0.00505308916863922\\
304.01	0.00505339089485271\\
305.01	0.00505369821181673\\
306.01	0.00505401122497884\\
307.01	0.00505433004192291\\
308.01	0.00505465477241729\\
309.01	0.00505498552846184\\
310.01	0.00505532242433601\\
311.01	0.00505566557664891\\
312.01	0.00505601510438797\\
313.01	0.00505637112897089\\
314.01	0.00505673377429641\\
315.01	0.00505710316679639\\
316.01	0.00505747943548865\\
317.01	0.0050578627120301\\
318.01	0.0050582531307706\\
319.01	0.00505865082880838\\
320.01	0.00505905594604403\\
321.01	0.00505946862523698\\
322.01	0.00505988901206109\\
323.01	0.00506031725516179\\
324.01	0.0050607535062126\\
325.01	0.00506119791997345\\
326.01	0.00506165065434799\\
327.01	0.00506211187044248\\
328.01	0.00506258173262414\\
329.01	0.00506306040858084\\
330.01	0.00506354806938109\\
331.01	0.00506404488953358\\
332.01	0.00506455104704839\\
333.01	0.00506506672349805\\
334.01	0.0050655921040795\\
335.01	0.00506612737767675\\
336.01	0.00506667273692421\\
337.01	0.00506722837827064\\
338.01	0.00506779450204467\\
339.01	0.00506837131252053\\
340.01	0.00506895901798665\\
341.01	0.00506955783081389\\
342.01	0.00507016796752677\\
343.01	0.00507078964887561\\
344.01	0.00507142309991152\\
345.01	0.00507206855006312\\
346.01	0.00507272623321661\\
347.01	0.00507339638779827\\
348.01	0.00507407925685907\\
349.01	0.00507477508816512\\
350.01	0.00507548413428993\\
351.01	0.00507620665271169\\
352.01	0.00507694290591373\\
353.01	0.00507769316149151\\
354.01	0.00507845769226323\\
355.01	0.0050792367763855\\
356.01	0.00508003069747492\\
357.01	0.00508083974473519\\
358.01	0.0050816642130905\\
359.01	0.00508250440332336\\
360.01	0.00508336062221926\\
361.01	0.00508423318271767\\
362.01	0.00508512240406883\\
363.01	0.00508602861199517\\
364.01	0.00508695213885965\\
365.01	0.00508789332383966\\
366.01	0.00508885251310515\\
367.01	0.00508983006000338\\
368.01	0.00509082632524624\\
369.01	0.00509184167710397\\
370.01	0.00509287649160212\\
371.01	0.00509393115272302\\
372.01	0.00509500605261015\\
373.01	0.00509610159177717\\
374.01	0.00509721817931821\\
375.01	0.00509835623312491\\
376.01	0.00509951618010449\\
377.01	0.00510069845640224\\
378.01	0.00510190350762779\\
379.01	0.00510313178908584\\
380.01	0.0051043837660103\\
381.01	0.00510565991380521\\
382.01	0.00510696071828762\\
383.01	0.00510828667593788\\
384.01	0.00510963829415476\\
385.01	0.00511101609151532\\
386.01	0.00511242059804085\\
387.01	0.00511385235546691\\
388.01	0.00511531191752215\\
389.01	0.00511679985020833\\
390.01	0.00511831673208837\\
391.01	0.0051198631545798\\
392.01	0.00512143972225235\\
393.01	0.00512304705313218\\
394.01	0.00512468577901079\\
395.01	0.00512635654575985\\
396.01	0.00512806001365004\\
397.01	0.00512979685767662\\
398.01	0.00513156776788948\\
399.01	0.0051333734497285\\
400.01	0.00513521462436362\\
401.01	0.00513709202904053\\
402.01	0.00513900641743134\\
403.01	0.00514095855999003\\
404.01	0.0051429492443123\\
405.01	0.00514497927550131\\
406.01	0.00514704947653821\\
407.01	0.00514916068865695\\
408.01	0.0051513137717242\\
409.01	0.00515350960462505\\
410.01	0.00515574908565231\\
411.01	0.005158033132902\\
412.01	0.00516036268467382\\
413.01	0.00516273869987566\\
414.01	0.00516516215843577\\
415.01	0.00516763406171819\\
416.01	0.00517015543294619\\
417.01	0.00517272731762952\\
418.01	0.00517535078399994\\
419.01	0.0051780269234516\\
420.01	0.00518075685099006\\
421.01	0.00518354170568777\\
422.01	0.00518638265114635\\
423.01	0.00518928087596993\\
424.01	0.00519223759424342\\
425.01	0.00519525404602364\\
426.01	0.00519833149783658\\
427.01	0.00520147124318873\\
428.01	0.00520467460308546\\
429.01	0.00520794292656374\\
430.01	0.00521127759123499\\
431.01	0.00521468000384159\\
432.01	0.00521815160082545\\
433.01	0.00522169384891116\\
434.01	0.00522530824570214\\
435.01	0.00522899632029247\\
436.01	0.00523275963389283\\
437.01	0.00523659978047139\\
438.01	0.00524051838741119\\
439.01	0.0052445171161831\\
440.01	0.00524859766303384\\
441.01	0.00525276175969047\\
442.01	0.00525701117408137\\
443.01	0.00526134771107234\\
444.01	0.00526577321321897\\
445.01	0.00527028956153447\\
446.01	0.00527489867627398\\
447.01	0.00527960251773282\\
448.01	0.00528440308706105\\
449.01	0.0052893024270928\\
450.01	0.00529430262319144\\
451.01	0.00529940580410992\\
452.01	0.00530461414286562\\
453.01	0.00530992985763248\\
454.01	0.00531535521264853\\
455.01	0.00532089251913995\\
456.01	0.00532654413626477\\
457.01	0.00533231247207272\\
458.01	0.00533819998448709\\
459.01	0.00534420918230508\\
460.01	0.00535034262622169\\
461.01	0.00535660292987603\\
462.01	0.00536299276092225\\
463.01	0.00536951484212522\\
464.01	0.00537617195248202\\
465.01	0.00538296692837173\\
466.01	0.00538990266473194\\
467.01	0.00539698211626374\\
468.01	0.00540420829866611\\
469.01	0.00541158428989791\\
470.01	0.00541911323147179\\
471.01	0.00542679832977569\\
472.01	0.00543464285742695\\
473.01	0.00544265015465428\\
474.01	0.00545082363071245\\
475.01	0.00545916676532751\\
476.01	0.00546768311017262\\
477.01	0.00547637629037634\\
478.01	0.00548525000606218\\
479.01	0.00549430803392156\\
480.01	0.00550355422881765\\
481.01	0.00551299252542421\\
482.01	0.00552262693989639\\
483.01	0.005532461571577\\
484.01	0.0055425006047365\\
485.01	0.00555274831034856\\
486.01	0.00556320904790121\\
487.01	0.0055738872672441\\
488.01	0.00558478751047235\\
489.01	0.00559591441384842\\
490.01	0.00560727270975982\\
491.01	0.00561886722871592\\
492.01	0.00563070290138235\\
493.01	0.00564278476065359\\
494.01	0.00565511794376293\\
495.01	0.00566770769443155\\
496.01	0.00568055936505529\\
497.01	0.00569367841892915\\
498.01	0.00570707043250933\\
499.01	0.00572074109771233\\
500.01	0.00573469622425185\\
501.01	0.00574894174201072\\
502.01	0.0057634837034497\\
503.01	0.00577832828605053\\
504.01	0.00579348179479285\\
505.01	0.00580895066466464\\
506.01	0.00582474146320375\\
507.01	0.00584086089306883\\
508.01	0.00585731579463925\\
509.01	0.00587411314864009\\
510.01	0.00589126007879083\\
511.01	0.00590876385447447\\
512.01	0.00592663189342303\\
513.01	0.00594487176441871\\
514.01	0.00596349119000116\\
515.01	0.00598249804918133\\
516.01	0.00600190038015398\\
517.01	0.00602170638300233\\
518.01	0.00604192442238996\\
519.01	0.00606256303023013\\
520.01	0.006083630908326\\
521.01	0.0061051369309701\\
522.01	0.0061270901474948\\
523.01	0.00614949978475927\\
524.01	0.00617237524956166\\
525.01	0.0061957261309602\\
526.01	0.00621956220248883\\
527.01	0.0062438934242455\\
528.01	0.00626872994483725\\
529.01	0.00629408210315595\\
530.01	0.0063199604299614\\
531.01	0.00634637564924387\\
532.01	0.00637333867933546\\
533.01	0.00640086063373574\\
534.01	0.00642895282161518\\
535.01	0.00645762674795358\\
536.01	0.00648689411326941\\
537.01	0.00651676681288722\\
538.01	0.00654725693568947\\
539.01	0.00657837676228909\\
540.01	0.0066101387625562\\
541.01	0.00664255559242306\\
542.01	0.00667564008988554\\
543.01	0.00670940527010898\\
544.01	0.00674386431953959\\
545.01	0.00677903058890984\\
546.01	0.00681491758501747\\
547.01	0.00685153896114556\\
548.01	0.00688890850597589\\
549.01	0.00692704013083688\\
550.01	0.00696594785510919\\
551.01	0.00700564578959692\\
552.01	0.00704614811765319\\
553.01	0.00708746907382994\\
554.01	0.00712962291979972\\
555.01	0.00717262391727598\\
556.01	0.00721648629763107\\
557.01	0.00726122422788934\\
558.01	0.00730685177274028\\
559.01	0.00735338285219186\\
560.01	0.00740083119445041\\
561.01	0.00744921028358358\\
562.01	0.00749853330148946\\
563.01	0.00754881306366298\\
564.01	0.0076000619482173\\
565.01	0.00765229181758725\\
566.01	0.00770551393231232\\
567.01	0.00775973885627156\\
568.01	0.00781497635272441\\
569.01	0.00787123527049889\\
570.01	0.00792852341966897\\
571.01	0.00798684743608042\\
572.01	0.0080462126341193\\
573.01	0.00810662284718149\\
574.01	0.0081680802554003\\
575.01	0.00823058520033459\\
576.01	0.00829413598651979\\
577.01	0.00835872867006437\\
578.01	0.00842435683484077\\
579.01	0.0084910113573111\\
580.01	0.00855868016166438\\
581.01	0.00862734796776696\\
582.01	0.00869699603548486\\
583.01	0.0087676019102873\\
584.01	0.00883913917675417\\
585.01	0.00891157722877423\\
586.01	0.00898488106795022\\
587.01	0.00905901114515074\\
588.01	0.00913392326443695\\
589.01	0.00920956857395242\\
590.01	0.00928589367504629\\
591.01	0.00936284088921878\\
592.01	0.0094403487328199\\
593.01	0.00951835266226869\\
594.01	0.00959678616847739\\
595.01	0.00967558231888201\\
596.01	0.00975467586988015\\
597.01	0.00983387214752878\\
598.01	0.0099086620184807\\
599.01	0.00997087280416276\\
599.02	0.00997138072163725\\
599.03	0.00997188557625799\\
599.04	0.00997238733820211\\
599.05	0.00997288597735272\\
599.06	0.00997338146329604\\
599.07	0.00997387376531845\\
599.08	0.00997436285240349\\
599.09	0.00997484869322892\\
599.1	0.00997533125616361\\
599.11	0.00997581050926455\\
599.12	0.00997628642027372\\
599.13	0.00997675895661497\\
599.14	0.0099772280853909\\
599.15	0.00997769377337961\\
599.16	0.00997815598703157\\
599.17	0.00997861469246631\\
599.18	0.00997906985546915\\
599.19	0.00997952144148792\\
599.2	0.00997996941562957\\
599.21	0.00998041374265684\\
599.22	0.0099808543869848\\
599.23	0.00998129131267744\\
599.24	0.00998172448344418\\
599.25	0.00998215386263636\\
599.26	0.00998257941258354\\
599.27	0.00998300109279825\\
599.28	0.00998341886238981\\
599.29	0.00998383268006024\\
599.3	0.00998424250410032\\
599.31	0.00998464829238543\\
599.32	0.00998505000237151\\
599.33	0.00998544759109087\\
599.34	0.00998584101514799\\
599.35	0.00998623023071532\\
599.36	0.00998661519352895\\
599.37	0.00998699585888436\\
599.38	0.00998737218163197\\
599.39	0.00998774411617281\\
599.4	0.00998811161645403\\
599.41	0.00998847463596443\\
599.42	0.0099888331277299\\
599.43	0.00998918704430885\\
599.44	0.00998953633778757\\
599.45	0.00998988095977558\\
599.46	0.00999022086140088\\
599.47	0.00999055599330522\\
599.48	0.00999088630563925\\
599.49	0.00999121174805768\\
599.5	0.00999153226971435\\
599.51	0.0099918478192573\\
599.52	0.00999215834482374\\
599.53	0.00999246379403503\\
599.54	0.0099927641139915\\
599.55	0.00999305925126738\\
599.56	0.00999334915190553\\
599.57	0.0099936337614122\\
599.58	0.00999391302475173\\
599.59	0.00999418688634118\\
599.6	0.00999445529004489\\
599.61	0.00999471817916904\\
599.62	0.00999497549645613\\
599.63	0.00999522718407935\\
599.64	0.009995473183637\\
599.65	0.00999571343614678\\
599.66	0.00999594788204004\\
599.67	0.00999617646115598\\
599.68	0.0099963991127358\\
599.69	0.00999661577541675\\
599.7	0.00999682638722618\\
599.71	0.00999703088557551\\
599.72	0.00999722920725411\\
599.73	0.00999742128842315\\
599.74	0.00999760706460942\\
599.75	0.00999778647069897\\
599.76	0.00999795944093086\\
599.77	0.0099981259088907\\
599.78	0.0099982858075042\\
599.79	0.00999843906903064\\
599.8	0.00999858562505627\\
599.81	0.00999872540648766\\
599.82	0.00999885834354498\\
599.83	0.00999898436575515\\
599.84	0.00999910340194508\\
599.85	0.00999921538023465\\
599.86	0.00999932022802977\\
599.87	0.00999941787201528\\
599.88	0.00999950823814785\\
599.89	0.00999959125164875\\
599.9	0.00999966683699656\\
599.91	0.00999973491791987\\
599.92	0.0099997954173898\\
599.93	0.00999984825761255\\
599.94	0.00999989336002181\\
599.95	0.00999993064527112\\
599.96	0.00999996003322615\\
599.97	0.00999998144295691\\
599.98	0.00999999479272987\\
599.99	0.01\\
600	0.01\\
};
\addplot [color=mycolor1,solid,forget plot]
  table[row sep=crcr]{%
0.01	0.00503201950426191\\
1.01	0.00503202043624351\\
2.01	0.00503202138695444\\
3.01	0.00503202235676875\\
4.01	0.00503202334606778\\
5.01	0.00503202435524097\\
6.01	0.00503202538468486\\
7.01	0.00503202643480416\\
8.01	0.0050320275060117\\
9.01	0.00503202859872831\\
10.01	0.00503202971338315\\
11.01	0.00503203085041398\\
12.01	0.00503203201026672\\
13.01	0.00503203319339726\\
14.01	0.005032034400269\\
15.01	0.00503203563135604\\
16.01	0.00503203688714068\\
17.01	0.00503203816811504\\
18.01	0.00503203947478162\\
19.01	0.00503204080765182\\
20.01	0.00503204216724792\\
21.01	0.00503204355410197\\
22.01	0.00503204496875709\\
23.01	0.00503204641176672\\
24.01	0.00503204788369532\\
25.01	0.00503204938511864\\
26.01	0.00503205091662345\\
27.01	0.00503205247880859\\
28.01	0.00503205407228416\\
29.01	0.00503205569767265\\
30.01	0.00503205735560877\\
31.01	0.0050320590467394\\
32.01	0.00503206077172466\\
33.01	0.00503206253123722\\
34.01	0.00503206432596379\\
35.01	0.00503206615660365\\
36.01	0.00503206802387021\\
37.01	0.00503206992849109\\
38.01	0.0050320718712078\\
39.01	0.00503207385277674\\
40.01	0.00503207587396953\\
41.01	0.00503207793557234\\
42.01	0.0050320800383872\\
43.01	0.00503208218323159\\
44.01	0.00503208437093899\\
45.01	0.00503208660235979\\
46.01	0.00503208887836051\\
47.01	0.00503209119982492\\
48.01	0.0050320935676543\\
49.01	0.00503209598276713\\
50.01	0.00503209844610009\\
51.01	0.00503210095860906\\
52.01	0.00503210352126715\\
53.01	0.00503210613506777\\
54.01	0.00503210880102339\\
55.01	0.00503211152016583\\
56.01	0.00503211429354786\\
57.01	0.00503211712224252\\
58.01	0.00503212000734368\\
59.01	0.0050321229499671\\
60.01	0.00503212595124972\\
61.01	0.00503212901235128\\
62.01	0.00503213213445397\\
63.01	0.00503213531876315\\
64.01	0.00503213856650761\\
65.01	0.00503214187894002\\
66.01	0.00503214525733772\\
67.01	0.00503214870300306\\
68.01	0.00503215221726366\\
69.01	0.00503215580147326\\
70.01	0.00503215945701149\\
71.01	0.00503216318528557\\
72.01	0.00503216698772946\\
73.01	0.00503217086580512\\
74.01	0.00503217482100379\\
75.01	0.005032178854845\\
76.01	0.00503218296887792\\
77.01	0.00503218716468218\\
78.01	0.00503219144386768\\
79.01	0.00503219580807637\\
80.01	0.00503220025898121\\
81.01	0.00503220479828845\\
82.01	0.00503220942773705\\
83.01	0.00503221414910002\\
84.01	0.00503221896418446\\
85.01	0.00503222387483279\\
86.01	0.00503222888292315\\
87.01	0.00503223399036967\\
88.01	0.00503223919912457\\
89.01	0.00503224451117715\\
90.01	0.00503224992855547\\
91.01	0.00503225545332672\\
92.01	0.0050322610875983\\
93.01	0.00503226683351865\\
94.01	0.00503227269327697\\
95.01	0.00503227866910558\\
96.01	0.00503228476327977\\
97.01	0.00503229097811875\\
98.01	0.00503229731598653\\
99.01	0.00503230377929245\\
100.01	0.00503231037049318\\
101.01	0.00503231709209198\\
102.01	0.0050323239466408\\
103.01	0.00503233093674088\\
104.01	0.00503233806504341\\
105.01	0.00503234533425098\\
106.01	0.00503235274711802\\
107.01	0.00503236030645168\\
108.01	0.00503236801511339\\
109.01	0.00503237587602001\\
110.01	0.00503238389214397\\
111.01	0.00503239206651512\\
112.01	0.00503240040222153\\
113.01	0.00503240890241055\\
114.01	0.00503241757029019\\
115.01	0.00503242640912957\\
116.01	0.00503243542226084\\
117.01	0.00503244461308017\\
118.01	0.00503245398504823\\
119.01	0.00503246354169318\\
120.01	0.005032473286609\\
121.01	0.00503248322346042\\
122.01	0.00503249335598105\\
123.01	0.00503250368797589\\
124.01	0.00503251422332365\\
125.01	0.00503252496597636\\
126.01	0.00503253591996161\\
127.01	0.00503254708938419\\
128.01	0.00503255847842715\\
129.01	0.0050325700913533\\
130.01	0.00503258193250708\\
131.01	0.00503259400631453\\
132.01	0.00503260631728797\\
133.01	0.00503261887002382\\
134.01	0.00503263166920667\\
135.01	0.00503264471961051\\
136.01	0.00503265802610012\\
137.01	0.00503267159363238\\
138.01	0.00503268542725815\\
139.01	0.00503269953212504\\
140.01	0.00503271391347715\\
141.01	0.00503272857665937\\
142.01	0.00503274352711694\\
143.01	0.00503275877039866\\
144.01	0.00503277431215837\\
145.01	0.005032790158157\\
146.01	0.00503280631426428\\
147.01	0.00503282278646114\\
148.01	0.00503283958084161\\
149.01	0.00503285670361445\\
150.01	0.00503287416110612\\
151.01	0.00503289195976253\\
152.01	0.0050329101061502\\
153.01	0.00503292860696047\\
154.01	0.00503294746901059\\
155.01	0.00503296669924563\\
156.01	0.00503298630474189\\
157.01	0.0050330062927082\\
158.01	0.00503302667048941\\
159.01	0.0050330474455681\\
160.01	0.00503306862556732\\
161.01	0.00503309021825343\\
162.01	0.00503311223153793\\
163.01	0.00503313467348104\\
164.01	0.00503315755229414\\
165.01	0.00503318087634153\\
166.01	0.00503320465414483\\
167.01	0.00503322889438429\\
168.01	0.00503325360590273\\
169.01	0.00503327879770787\\
170.01	0.00503330447897571\\
171.01	0.0050333306590534\\
172.01	0.00503335734746257\\
173.01	0.00503338455390173\\
174.01	0.00503341228824994\\
175.01	0.00503344056057023\\
176.01	0.00503346938111257\\
177.01	0.00503349876031783\\
178.01	0.00503352870881993\\
179.01	0.00503355923745083\\
180.01	0.0050335903572428\\
181.01	0.00503362207943275\\
182.01	0.0050336544154656\\
183.01	0.00503368737699827\\
184.01	0.00503372097590274\\
185.01	0.00503375522427091\\
186.01	0.00503379013441749\\
187.01	0.00503382571888486\\
188.01	0.00503386199044611\\
189.01	0.00503389896211015\\
190.01	0.00503393664712548\\
191.01	0.00503397505898429\\
192.01	0.00503401421142635\\
193.01	0.00503405411844454\\
194.01	0.00503409479428832\\
195.01	0.00503413625346871\\
196.01	0.00503417851076279\\
197.01	0.00503422158121802\\
198.01	0.00503426548015751\\
199.01	0.00503431022318523\\
200.01	0.00503435582618919\\
201.01	0.00503440230534838\\
202.01	0.00503444967713718\\
203.01	0.00503449795832997\\
204.01	0.00503454716600721\\
205.01	0.00503459731756009\\
206.01	0.00503464843069654\\
207.01	0.00503470052344598\\
208.01	0.00503475361416586\\
209.01	0.00503480772154641\\
210.01	0.00503486286461741\\
211.01	0.00503491906275336\\
212.01	0.00503497633567945\\
213.01	0.00503503470347832\\
214.01	0.00503509418659531\\
215.01	0.0050351548058459\\
216.01	0.00503521658242043\\
217.01	0.00503527953789218\\
218.01	0.00503534369422357\\
219.01	0.00503540907377234\\
220.01	0.00503547569929882\\
221.01	0.00503554359397246\\
222.01	0.00503561278137921\\
223.01	0.00503568328552928\\
224.01	0.00503575513086363\\
225.01	0.00503582834226052\\
226.01	0.00503590294504541\\
227.01	0.00503597896499697\\
228.01	0.00503605642835475\\
229.01	0.00503613536182793\\
230.01	0.0050362157926031\\
231.01	0.00503629774835188\\
232.01	0.00503638125724035\\
233.01	0.00503646634793695\\
234.01	0.00503655304962039\\
235.01	0.00503664139199012\\
236.01	0.00503673140527389\\
237.01	0.00503682312023715\\
238.01	0.00503691656819238\\
239.01	0.00503701178100915\\
240.01	0.00503710879112251\\
241.01	0.00503720763154361\\
242.01	0.00503730833586963\\
243.01	0.00503741093829291\\
244.01	0.00503751547361275\\
245.01	0.00503762197724492\\
246.01	0.00503773048523254\\
247.01	0.00503784103425756\\
248.01	0.00503795366165097\\
249.01	0.0050380684054043\\
250.01	0.00503818530418168\\
251.01	0.00503830439733127\\
252.01	0.00503842572489746\\
253.01	0.00503854932763265\\
254.01	0.00503867524701027\\
255.01	0.00503880352523717\\
256.01	0.00503893420526672\\
257.01	0.00503906733081197\\
258.01	0.0050392029463599\\
259.01	0.00503934109718421\\
260.01	0.00503948182936034\\
261.01	0.00503962518977843\\
262.01	0.00503977122616076\\
263.01	0.00503991998707443\\
264.01	0.00504007152194746\\
265.01	0.0050402258810856\\
266.01	0.0050403831156868\\
267.01	0.0050405432778593\\
268.01	0.00504070642063711\\
269.01	0.00504087259799862\\
270.01	0.0050410418648838\\
271.01	0.00504121427721224\\
272.01	0.00504138989190245\\
273.01	0.0050415687668901\\
274.01	0.00504175096114795\\
275.01	0.00504193653470614\\
276.01	0.00504212554867273\\
277.01	0.00504231806525504\\
278.01	0.00504251414778109\\
279.01	0.00504271386072164\\
280.01	0.00504291726971426\\
281.01	0.00504312444158475\\
282.01	0.0050433354443741\\
283.01	0.00504355034736101\\
284.01	0.00504376922108885\\
285.01	0.00504399213739071\\
286.01	0.0050442191694183\\
287.01	0.00504445039166762\\
288.01	0.0050446858800087\\
289.01	0.00504492571171544\\
290.01	0.00504516996549426\\
291.01	0.00504541872151704\\
292.01	0.00504567206145204\\
293.01	0.00504593006849774\\
294.01	0.00504619282741553\\
295.01	0.00504646042456572\\
296.01	0.00504673294794333\\
297.01	0.00504701048721468\\
298.01	0.00504729313375593\\
299.01	0.00504758098069216\\
300.01	0.00504787412293708\\
301.01	0.00504817265723545\\
302.01	0.00504847668220459\\
303.01	0.00504878629837929\\
304.01	0.00504910160825628\\
305.01	0.0050494227163401\\
306.01	0.00504974972919078\\
307.01	0.00505008275547381\\
308.01	0.00505042190600799\\
309.01	0.00505076729381831\\
310.01	0.00505111903418812\\
311.01	0.0050514772447123\\
312.01	0.00505184204535378\\
313.01	0.00505221355849816\\
314.01	0.00505259190901264\\
315.01	0.00505297722430356\\
316.01	0.00505336963437772\\
317.01	0.00505376927190221\\
318.01	0.00505417627226815\\
319.01	0.00505459077365233\\
320.01	0.0050550129170826\\
321.01	0.00505544284650278\\
322.01	0.00505588070883858\\
323.01	0.00505632665406469\\
324.01	0.00505678083527231\\
325.01	0.00505724340873737\\
326.01	0.00505771453398907\\
327.01	0.00505819437387861\\
328.01	0.00505868309464988\\
329.01	0.00505918086600726\\
330.01	0.00505968786118643\\
331.01	0.00506020425702283\\
332.01	0.00506073023402123\\
333.01	0.00506126597642489\\
334.01	0.0050618116722832\\
335.01	0.00506236751351999\\
336.01	0.00506293369599944\\
337.01	0.0050635104195933\\
338.01	0.00506409788824524\\
339.01	0.00506469631003512\\
340.01	0.00506530589724219\\
341.01	0.00506592686640633\\
342.01	0.00506655943838927\\
343.01	0.00506720383843395\\
344.01	0.00506786029622303\\
345.01	0.00506852904593738\\
346.01	0.00506921032631205\\
347.01	0.00506990438069319\\
348.01	0.00507061145709439\\
349.01	0.00507133180825325\\
350.01	0.00507206569168783\\
351.01	0.0050728133697556\\
352.01	0.00507357510971221\\
353.01	0.0050743511837737\\
354.01	0.00507514186918129\\
355.01	0.0050759474482705\\
356.01	0.00507676820854269\\
357.01	0.00507760444274413\\
358.01	0.00507845644895008\\
359.01	0.00507932453065577\\
360.01	0.00508020899687583\\
361.01	0.00508111016225213\\
362.01	0.0050820283471692\\
363.01	0.0050829638778836\\
364.01	0.00508391708665953\\
365.01	0.00508488831191731\\
366.01	0.00508587789839304\\
367.01	0.00508688619730879\\
368.01	0.00508791356655353\\
369.01	0.00508896037087408\\
370.01	0.00509002698207614\\
371.01	0.00509111377923371\\
372.01	0.00509222114890623\\
373.01	0.00509334948536214\\
374.01	0.00509449919080782\\
375.01	0.00509567067562119\\
376.01	0.00509686435858675\\
377.01	0.00509808066713371\\
378.01	0.00509932003757677\\
379.01	0.00510058291535667\\
380.01	0.00510186975528346\\
381.01	0.00510318102178189\\
382.01	0.00510451718913998\\
383.01	0.00510587874176205\\
384.01	0.00510726617442703\\
385.01	0.00510867999255332\\
386.01	0.0051101207124698\\
387.01	0.00511158886169583\\
388.01	0.00511308497922627\\
389.01	0.00511460961582658\\
390.01	0.00511616333433346\\
391.01	0.00511774670996351\\
392.01	0.00511936033063096\\
393.01	0.00512100479727096\\
394.01	0.00512268072417241\\
395.01	0.00512438873931648\\
396.01	0.00512612948472518\\
397.01	0.0051279036168155\\
398.01	0.00512971180676239\\
399.01	0.00513155474086818\\
400.01	0.00513343312093995\\
401.01	0.00513534766467416\\
402.01	0.0051372991060481\\
403.01	0.00513928819571818\\
404.01	0.00514131570142495\\
405.01	0.00514338240840512\\
406.01	0.00514548911980908\\
407.01	0.00514763665712437\\
408.01	0.00514982586060625\\
409.01	0.00515205758971196\\
410.01	0.00515433272354165\\
411.01	0.00515665216128347\\
412.01	0.00515901682266326\\
413.01	0.00516142764839961\\
414.01	0.00516388560066147\\
415.01	0.00516639166353125\\
416.01	0.00516894684347027\\
417.01	0.00517155216978837\\
418.01	0.0051742086951167\\
419.01	0.00517691749588425\\
420.01	0.00517967967279609\\
421.01	0.00518249635131593\\
422.01	0.00518536868215205\\
423.01	0.00518829784174599\\
424.01	0.00519128503276489\\
425.01	0.00519433148459841\\
426.01	0.00519743845385992\\
427.01	0.00520060722489213\\
428.01	0.00520383911027941\\
429.01	0.00520713545136559\\
430.01	0.00521049761877979\\
431.01	0.0052139270129701\\
432.01	0.00521742506474721\\
433.01	0.00522099323583805\\
434.01	0.00522463301945253\\
435.01	0.00522834594086097\\
436.01	0.00523213355798921\\
437.01	0.00523599746202724\\
438.01	0.0052399392780561\\
439.01	0.0052439606656921\\
440.01	0.00524806331975258\\
441.01	0.00525224897094034\\
442.01	0.00525651938654981\\
443.01	0.00526087637119615\\
444.01	0.00526532176756633\\
445.01	0.00526985745719233\\
446.01	0.00527448536124653\\
447.01	0.00527920744135943\\
448.01	0.00528402570045861\\
449.01	0.00528894218362613\\
450.01	0.00529395897897561\\
451.01	0.00529907821854578\\
452.01	0.00530430207920979\\
453.01	0.00530963278359722\\
454.01	0.00531507260102934\\
455.01	0.00532062384846525\\
456.01	0.00532628889145623\\
457.01	0.00533207014511108\\
458.01	0.00533797007507009\\
459.01	0.00534399119848931\\
460.01	0.0053501360850358\\
461.01	0.0053564073578961\\
462.01	0.00536280769480051\\
463.01	0.00536933982906571\\
464.01	0.0053760065506588\\
465.01	0.00538281070728559\\
466.01	0.00538975520550605\\
467.01	0.00539684301187904\\
468.01	0.00540407715413795\\
469.01	0.00541146072239986\\
470.01	0.00541899687040678\\
471.01	0.00542668881680251\\
472.01	0.00543453984644367\\
473.01	0.0054425533117462\\
474.01	0.00545073263406792\\
475.01	0.00545908130512543\\
476.01	0.00546760288844794\\
477.01	0.00547630102086585\\
478.01	0.0054851794140342\\
479.01	0.00549424185599155\\
480.01	0.0055034922127532\\
481.01	0.00551293442993825\\
482.01	0.00552257253443218\\
483.01	0.00553241063608223\\
484.01	0.00554245292942729\\
485.01	0.00555270369546343\\
486.01	0.00556316730344353\\
487.01	0.00557384821271302\\
488.01	0.00558475097458304\\
489.01	0.00559588023423913\\
490.01	0.00560724073269073\\
491.01	0.00561883730875751\\
492.01	0.00563067490109745\\
493.01	0.00564275855027375\\
494.01	0.00565509340086368\\
495.01	0.00566768470360667\\
496.01	0.00568053781759384\\
497.01	0.00569365821249661\\
498.01	0.00570705147083497\\
499.01	0.00572072329028427\\
500.01	0.00573467948601941\\
501.01	0.00574892599309709\\
502.01	0.00576346886887259\\
503.01	0.00577831429545238\\
504.01	0.00579346858218174\\
505.01	0.0058089381681635\\
506.01	0.00582472962480907\\
507.01	0.00584084965842006\\
508.01	0.00585730511279649\\
509.01	0.00587410297187216\\
510.01	0.00589125036237267\\
511.01	0.00590875455649485\\
512.01	0.00592662297460268\\
513.01	0.00594486318793586\\
514.01	0.00596348292132978\\
515.01	0.0059824900559373\\
516.01	0.00600189263195081\\
517.01	0.00602169885131572\\
518.01	0.00604191708043086\\
519.01	0.00606255585282599\\
520.01	0.00608362387180911\\
521.01	0.00610513001307378\\
522.01	0.00612708332725578\\
523.01	0.00614949304242679\\
524.01	0.00617236856651236\\
525.01	0.006195719489619\\
526.01	0.00621955558625398\\
527.01	0.00624388681741989\\
528.01	0.00626872333256309\\
529.01	0.00629407547135422\\
530.01	0.00631995376527479\\
531.01	0.00634636893898359\\
532.01	0.00637333191143073\\
533.01	0.00640085379668688\\
534.01	0.00642894590444845\\
535.01	0.00645761974017953\\
536.01	0.00648688700484237\\
537.01	0.00651675959416793\\
538.01	0.00654724959740859\\
539.01	0.00657836929551299\\
540.01	0.00661013115865439\\
541.01	0.00664254784303731\\
542.01	0.0066756321869002\\
543.01	0.00670939720562297\\
544.01	0.00674385608583911\\
545.01	0.00677902217844316\\
546.01	0.00681490899037128\\
547.01	0.00685153017502185\\
548.01	0.00688889952117103\\
549.01	0.00692703094022179\\
550.01	0.00696593845161156\\
551.01	0.00700563616618453\\
552.01	0.00704613826731923\\
553.01	0.00708745898958019\\
554.01	0.00712961259464204\\
555.01	0.00717261334421132\\
556.01	0.0072164754696476\\
557.01	0.00726121313795847\\
558.01	0.00730684041381582\\
559.01	0.0073533712172122\\
560.01	0.00740081927634337\\
561.01	0.00744919807527555\\
562.01	0.00749852079591776\\
563.01	0.00754880025379286\\
564.01	0.00760004882706339\\
565.01	0.00765227837824005\\
566.01	0.0077055001679697\\
567.01	0.00775972476027739\\
568.01	0.00781496191861332\\
569.01	0.00787122049204771\\
570.01	0.00792850829095721\\
571.01	0.00798683195155923\\
572.01	0.00804619678868979\\
573.01	0.00810660663628399\\
574.01	0.00816806367511503\\
575.01	0.0082305682474943\\
576.01	0.00829411865883634\\
577.01	0.00835871096626884\\
578.01	0.00842433875483839\\
579.01	0.00849099290235135\\
580.01	0.00855866133452602\\
581.01	0.008627328772957\\
582.01	0.00869697647944997\\
583.01	0.00876758200163454\\
584.01	0.00883911892647557\\
585.01	0.00891155665046791\\
586.01	0.0089848601780259\\
587.01	0.00905898996300263\\
588.01	0.00913390181256045\\
589.01	0.00920954687797058\\
590.01	0.00928587176359865\\
591.01	0.0093628187936477\\
592.01	0.00944032648656475\\
593.01	0.00951833029984481\\
594.01	0.00959676372387219\\
595.01	0.00967555982313978\\
596.01	0.00975465334756447\\
597.01	0.00983385942577576\\
598.01	0.00990866201848071\\
599.01	0.00997087280416276\\
599.02	0.00997138072163725\\
599.03	0.009971885576258\\
599.04	0.00997238733820211\\
599.05	0.00997288597735272\\
599.06	0.00997338146329604\\
599.07	0.00997387376531845\\
599.08	0.00997436285240349\\
599.09	0.00997484869322892\\
599.1	0.00997533125616361\\
599.11	0.00997581050926455\\
599.12	0.00997628642027372\\
599.13	0.00997675895661498\\
599.14	0.0099772280853909\\
599.15	0.00997769377337961\\
599.16	0.00997815598703157\\
599.17	0.00997861469246631\\
599.18	0.00997906985546915\\
599.19	0.00997952144148792\\
599.2	0.00997996941562957\\
599.21	0.00998041374265684\\
599.22	0.0099808543869848\\
599.23	0.00998129131267744\\
599.24	0.00998172448344418\\
599.25	0.00998215386263636\\
599.26	0.00998257941258353\\
599.27	0.00998300109279825\\
599.28	0.00998341886238981\\
599.29	0.00998383268006024\\
599.3	0.00998424250410032\\
599.31	0.00998464829238543\\
599.32	0.00998505000237151\\
599.33	0.00998544759109087\\
599.34	0.00998584101514799\\
599.35	0.00998623023071532\\
599.36	0.00998661519352896\\
599.37	0.00998699585888436\\
599.38	0.00998737218163197\\
599.39	0.00998774411617281\\
599.4	0.00998811161645403\\
599.41	0.00998847463596443\\
599.42	0.0099888331277299\\
599.43	0.00998918704430885\\
599.44	0.00998953633778757\\
599.45	0.00998988095977558\\
599.46	0.00999022086140088\\
599.47	0.00999055599330522\\
599.48	0.00999088630563925\\
599.49	0.00999121174805768\\
599.5	0.00999153226971435\\
599.51	0.0099918478192573\\
599.52	0.00999215834482374\\
599.53	0.00999246379403503\\
599.54	0.0099927641139915\\
599.55	0.00999305925126738\\
599.56	0.00999334915190553\\
599.57	0.0099936337614122\\
599.58	0.00999391302475173\\
599.59	0.00999418688634118\\
599.6	0.00999445529004489\\
599.61	0.00999471817916904\\
599.62	0.00999497549645613\\
599.63	0.00999522718407934\\
599.64	0.009995473183637\\
599.65	0.00999571343614678\\
599.66	0.00999594788204004\\
599.67	0.00999617646115599\\
599.68	0.0099963991127358\\
599.69	0.00999661577541675\\
599.7	0.00999682638722618\\
599.71	0.00999703088557551\\
599.72	0.00999722920725411\\
599.73	0.00999742128842315\\
599.74	0.00999760706460942\\
599.75	0.00999778647069897\\
599.76	0.00999795944093086\\
599.77	0.0099981259088907\\
599.78	0.0099982858075042\\
599.79	0.00999843906903064\\
599.8	0.00999858562505627\\
599.81	0.00999872540648767\\
599.82	0.00999885834354498\\
599.83	0.00999898436575515\\
599.84	0.00999910340194508\\
599.85	0.00999921538023465\\
599.86	0.00999932022802977\\
599.87	0.00999941787201528\\
599.88	0.00999950823814785\\
599.89	0.00999959125164875\\
599.9	0.00999966683699656\\
599.91	0.00999973491791987\\
599.92	0.0099997954173898\\
599.93	0.00999984825761255\\
599.94	0.00999989336002181\\
599.95	0.00999993064527112\\
599.96	0.00999996003322615\\
599.97	0.00999998144295691\\
599.98	0.00999999479272987\\
599.99	0.01\\
600	0.01\\
};
\addplot [color=mycolor2,solid,forget plot]
  table[row sep=crcr]{%
0.01	0.00502120505551015\\
1.01	0.0050212060149824\\
2.01	0.00502120699383299\\
3.01	0.00502120799245165\\
4.01	0.00502120901123571\\
5.01	0.00502121005059042\\
6.01	0.00502121111092924\\
7.01	0.00502121219267379\\
8.01	0.00502121329625405\\
9.01	0.00502121442210887\\
10.01	0.00502121557068572\\
11.01	0.00502121674244047\\
12.01	0.00502121793783911\\
13.01	0.0050212191573559\\
14.01	0.00502122040147554\\
15.01	0.00502122167069138\\
16.01	0.00502122296550742\\
17.01	0.00502122428643781\\
18.01	0.00502122563400595\\
19.01	0.0050212270087466\\
20.01	0.00502122841120478\\
21.01	0.00502122984193676\\
22.01	0.00502123130150919\\
23.01	0.00502123279050046\\
24.01	0.00502123430950069\\
25.01	0.00502123585911135\\
26.01	0.00502123743994609\\
27.01	0.00502123905263077\\
28.01	0.00502124069780355\\
29.01	0.00502124237611543\\
30.01	0.00502124408823032\\
31.01	0.00502124583482541\\
32.01	0.00502124761659137\\
33.01	0.00502124943423254\\
34.01	0.0050212512884672\\
35.01	0.00502125318002777\\
36.01	0.00502125510966178\\
37.01	0.0050212570781312\\
38.01	0.00502125908621306\\
39.01	0.00502126113470028\\
40.01	0.00502126322440086\\
41.01	0.00502126535613939\\
42.01	0.00502126753075657\\
43.01	0.00502126974910991\\
44.01	0.00502127201207395\\
45.01	0.00502127432054039\\
46.01	0.00502127667541896\\
47.01	0.00502127907763728\\
48.01	0.00502128152814088\\
49.01	0.0050212840278949\\
50.01	0.00502128657788314\\
51.01	0.00502128917910871\\
52.01	0.00502129183259509\\
53.01	0.0050212945393856\\
54.01	0.00502129730054438\\
55.01	0.00502130011715682\\
56.01	0.00502130299032957\\
57.01	0.00502130592119115\\
58.01	0.00502130891089264\\
59.01	0.00502131196060768\\
60.01	0.00502131507153325\\
61.01	0.00502131824489013\\
62.01	0.00502132148192286\\
63.01	0.00502132478390088\\
64.01	0.00502132815211859\\
65.01	0.00502133158789611\\
66.01	0.00502133509257949\\
67.01	0.00502133866754131\\
68.01	0.00502134231418158\\
69.01	0.00502134603392737\\
70.01	0.0050213498282346\\
71.01	0.00502135369858719\\
72.01	0.00502135764649908\\
73.01	0.00502136167351383\\
74.01	0.00502136578120502\\
75.01	0.00502136997117758\\
76.01	0.00502137424506807\\
77.01	0.00502137860454521\\
78.01	0.00502138305131081\\
79.01	0.00502138758709988\\
80.01	0.00502139221368189\\
81.01	0.00502139693286069\\
82.01	0.00502140174647622\\
83.01	0.0050214066564041\\
84.01	0.00502141166455702\\
85.01	0.00502141677288556\\
86.01	0.00502142198337838\\
87.01	0.00502142729806312\\
88.01	0.00502143271900729\\
89.01	0.00502143824831927\\
90.01	0.00502144388814851\\
91.01	0.00502144964068685\\
92.01	0.00502145550816918\\
93.01	0.00502146149287397\\
94.01	0.00502146759712493\\
95.01	0.00502147382329066\\
96.01	0.00502148017378673\\
97.01	0.00502148665107566\\
98.01	0.00502149325766843\\
99.01	0.00502149999612531\\
100.01	0.00502150686905647\\
101.01	0.0050215138791232\\
102.01	0.00502152102903899\\
103.01	0.00502152832157038\\
104.01	0.00502153575953804\\
105.01	0.00502154334581734\\
106.01	0.00502155108334049\\
107.01	0.00502155897509676\\
108.01	0.00502156702413392\\
109.01	0.00502157523355901\\
110.01	0.00502158360653984\\
111.01	0.00502159214630588\\
112.01	0.00502160085614986\\
113.01	0.00502160973942875\\
114.01	0.00502161879956461\\
115.01	0.00502162804004638\\
116.01	0.0050216374644311\\
117.01	0.00502164707634492\\
118.01	0.00502165687948466\\
119.01	0.00502166687761887\\
120.01	0.00502167707459003\\
121.01	0.00502168747431439\\
122.01	0.00502169808078512\\
123.01	0.00502170889807277\\
124.01	0.00502171993032674\\
125.01	0.00502173118177726\\
126.01	0.00502174265673663\\
127.01	0.00502175435960078\\
128.01	0.00502176629485096\\
129.01	0.00502177846705515\\
130.01	0.00502179088086999\\
131.01	0.0050218035410424\\
132.01	0.00502181645241076\\
133.01	0.00502182961990796\\
134.01	0.00502184304856201\\
135.01	0.00502185674349784\\
136.01	0.00502187070993954\\
137.01	0.00502188495321224\\
138.01	0.00502189947874446\\
139.01	0.00502191429206877\\
140.01	0.00502192939882506\\
141.01	0.00502194480476151\\
142.01	0.00502196051573778\\
143.01	0.0050219765377259\\
144.01	0.00502199287681351\\
145.01	0.00502200953920504\\
146.01	0.00502202653122452\\
147.01	0.00502204385931768\\
148.01	0.00502206153005422\\
149.01	0.00502207955013015\\
150.01	0.00502209792636996\\
151.01	0.0050221166657293\\
152.01	0.0050221357752977\\
153.01	0.00502215526229993\\
154.01	0.00502217513409978\\
155.01	0.00502219539820223\\
156.01	0.00502221606225553\\
157.01	0.00502223713405503\\
158.01	0.0050222586215449\\
159.01	0.00502228053282122\\
160.01	0.00502230287613462\\
161.01	0.00502232565989372\\
162.01	0.00502234889266762\\
163.01	0.0050223725831888\\
164.01	0.00502239674035607\\
165.01	0.00502242137323838\\
166.01	0.005022446491077\\
167.01	0.00502247210328907\\
168.01	0.00502249821947123\\
169.01	0.00502252484940274\\
170.01	0.00502255200304808\\
171.01	0.00502257969056166\\
172.01	0.00502260792228945\\
173.01	0.00502263670877492\\
174.01	0.00502266606076066\\
175.01	0.00502269598919311\\
176.01	0.00502272650522542\\
177.01	0.00502275762022222\\
178.01	0.00502278934576256\\
179.01	0.00502282169364454\\
180.01	0.00502285467588862\\
181.01	0.00502288830474204\\
182.01	0.00502292259268314\\
183.01	0.00502295755242499\\
184.01	0.00502299319692021\\
185.01	0.00502302953936454\\
186.01	0.00502306659320215\\
187.01	0.00502310437212936\\
188.01	0.00502314289009939\\
189.01	0.00502318216132727\\
190.01	0.00502322220029433\\
191.01	0.00502326302175243\\
192.01	0.00502330464072964\\
193.01	0.00502334707253499\\
194.01	0.00502339033276278\\
195.01	0.00502343443729858\\
196.01	0.00502347940232345\\
197.01	0.00502352524432037\\
198.01	0.00502357198007852\\
199.01	0.00502361962669878\\
200.01	0.00502366820160027\\
201.01	0.0050237177225248\\
202.01	0.00502376820754208\\
203.01	0.00502381967505755\\
204.01	0.00502387214381636\\
205.01	0.00502392563291001\\
206.01	0.00502398016178238\\
207.01	0.00502403575023566\\
208.01	0.00502409241843702\\
209.01	0.00502415018692418\\
210.01	0.00502420907661233\\
211.01	0.00502426910880058\\
212.01	0.00502433030517863\\
213.01	0.00502439268783314\\
214.01	0.00502445627925474\\
215.01	0.005024521102345\\
216.01	0.00502458718042365\\
217.01	0.00502465453723511\\
218.01	0.00502472319695607\\
219.01	0.00502479318420265\\
220.01	0.00502486452403789\\
221.01	0.00502493724197978\\
222.01	0.0050250113640079\\
223.01	0.00502508691657197\\
224.01	0.00502516392659885\\
225.01	0.0050252424215021\\
226.01	0.00502532242918824\\
227.01	0.00502540397806611\\
228.01	0.00502548709705464\\
229.01	0.00502557181559166\\
230.01	0.00502565816364192\\
231.01	0.00502574617170654\\
232.01	0.005025835870831\\
233.01	0.00502592729261411\\
234.01	0.00502602046921832\\
235.01	0.0050261154333765\\
236.01	0.00502621221840324\\
237.01	0.00502631085820338\\
238.01	0.00502641138728205\\
239.01	0.00502651384075328\\
240.01	0.00502661825435088\\
241.01	0.00502672466443788\\
242.01	0.00502683310801658\\
243.01	0.00502694362273876\\
244.01	0.0050270562469161\\
245.01	0.00502717101953057\\
246.01	0.00502728798024471\\
247.01	0.00502740716941269\\
248.01	0.0050275286280913\\
249.01	0.005027652398051\\
250.01	0.00502777852178654\\
251.01	0.00502790704252892\\
252.01	0.00502803800425672\\
253.01	0.00502817145170778\\
254.01	0.00502830743039084\\
255.01	0.00502844598659744\\
256.01	0.00502858716741425\\
257.01	0.00502873102073589\\
258.01	0.00502887759527657\\
259.01	0.00502902694058344\\
260.01	0.005029179107049\\
261.01	0.00502933414592505\\
262.01	0.0050294921093347\\
263.01	0.00502965305028719\\
264.01	0.00502981702269076\\
265.01	0.0050299840813669\\
266.01	0.00503015428206463\\
267.01	0.00503032768147485\\
268.01	0.00503050433724536\\
269.01	0.00503068430799553\\
270.01	0.00503086765333136\\
271.01	0.00503105443386162\\
272.01	0.00503124471121294\\
273.01	0.00503143854804688\\
274.01	0.00503163600807607\\
275.01	0.00503183715608083\\
276.01	0.00503204205792718\\
277.01	0.00503225078058336\\
278.01	0.00503246339213911\\
279.01	0.00503267996182363\\
280.01	0.00503290056002414\\
281.01	0.00503312525830624\\
282.01	0.00503335412943298\\
283.01	0.0050335872473862\\
284.01	0.00503382468738714\\
285.01	0.00503406652591845\\
286.01	0.00503431284074593\\
287.01	0.00503456371094249\\
288.01	0.00503481921691118\\
289.01	0.00503507944040973\\
290.01	0.00503534446457649\\
291.01	0.00503561437395558\\
292.01	0.00503588925452533\\
293.01	0.00503616919372432\\
294.01	0.00503645428048247\\
295.01	0.00503674460524972\\
296.01	0.00503704026002744\\
297.01	0.00503734133840114\\
298.01	0.00503764793557371\\
299.01	0.00503796014839986\\
300.01	0.00503827807542342\\
301.01	0.00503860181691385\\
302.01	0.00503893147490631\\
303.01	0.00503926715324135\\
304.01	0.00503960895760799\\
305.01	0.00503995699558779\\
306.01	0.0050403113767003\\
307.01	0.00504067221245086\\
308.01	0.00504103961638069\\
309.01	0.00504141370411773\\
310.01	0.00504179459343115\\
311.01	0.00504218240428678\\
312.01	0.00504257725890531\\
313.01	0.00504297928182318\\
314.01	0.00504338859995435\\
315.01	0.00504380534265644\\
316.01	0.00504422964179752\\
317.01	0.00504466163182693\\
318.01	0.00504510144984702\\
319.01	0.00504554923568865\\
320.01	0.00504600513198888\\
321.01	0.00504646928427072\\
322.01	0.00504694184102603\\
323.01	0.00504742295380053\\
324.01	0.00504791277728073\\
325.01	0.00504841146938398\\
326.01	0.00504891919135017\\
327.01	0.00504943610783593\\
328.01	0.00504996238700892\\
329.01	0.00505049820064668\\
330.01	0.00505104372423423\\
331.01	0.00505159913706429\\
332.01	0.00505216462233776\\
333.01	0.0050527403672645\\
334.01	0.00505332656316482\\
335.01	0.00505392340556957\\
336.01	0.00505453109432047\\
337.01	0.00505514983366831\\
338.01	0.00505577983236974\\
339.01	0.00505642130378112\\
340.01	0.00505707446594919\\
341.01	0.00505773954169936\\
342.01	0.00505841675871721\\
343.01	0.00505910634962739\\
344.01	0.005059808552065\\
345.01	0.0050605236087411\\
346.01	0.00506125176750191\\
347.01	0.00506199328137963\\
348.01	0.00506274840863571\\
349.01	0.00506351741279459\\
350.01	0.00506430056267015\\
351.01	0.00506509813238215\\
352.01	0.00506591040136409\\
353.01	0.00506673765436222\\
354.01	0.00506758018142595\\
355.01	0.00506843827789061\\
356.01	0.00506931224435403\\
357.01	0.00507020238664743\\
358.01	0.00507110901580171\\
359.01	0.00507203244801305\\
360.01	0.00507297300460754\\
361.01	0.0050739310120094\\
362.01	0.00507490680171485\\
363.01	0.00507590071027472\\
364.01	0.00507691307928929\\
365.01	0.00507794425541877\\
366.01	0.00507899459041269\\
367.01	0.00508006444116119\\
368.01	0.00508115416977156\\
369.01	0.00508226414367324\\
370.01	0.00508339473575106\\
371.01	0.00508454632450858\\
372.01	0.00508571929426307\\
373.01	0.00508691403536712\\
374.01	0.00508813094445901\\
375.01	0.00508937042473226\\
376.01	0.00509063288622533\\
377.01	0.00509191874612255\\
378.01	0.00509322842905835\\
379.01	0.00509456236742253\\
380.01	0.0050959210016579\\
381.01	0.00509730478054768\\
382.01	0.00509871416149152\\
383.01	0.00510014961076941\\
384.01	0.00510161160379846\\
385.01	0.00510310062538774\\
386.01	0.00510461716999747\\
387.01	0.00510616174200523\\
388.01	0.00510773485598404\\
389.01	0.0051093370369904\\
390.01	0.00511096882086284\\
391.01	0.00511263075453339\\
392.01	0.00511432339634935\\
393.01	0.00511604731640772\\
394.01	0.00511780309690188\\
395.01	0.0051195913324808\\
396.01	0.00512141263062116\\
397.01	0.00512326761201211\\
398.01	0.00512515691095339\\
399.01	0.0051270811757666\\
400.01	0.00512904106921996\\
401.01	0.00513103726896684\\
402.01	0.00513307046799628\\
403.01	0.00513514137509836\\
404.01	0.00513725071534207\\
405.01	0.00513939923056583\\
406.01	0.00514158767988127\\
407.01	0.00514381684018901\\
408.01	0.00514608750670632\\
409.01	0.00514840049350604\\
410.01	0.00515075663406668\\
411.01	0.00515315678183237\\
412.01	0.00515560181078219\\
413.01	0.00515809261600762\\
414.01	0.00516063011429847\\
415.01	0.00516321524473404\\
416.01	0.00516584896928096\\
417.01	0.00516853227339409\\
418.01	0.00517126616662132\\
419.01	0.0051740516832099\\
420.01	0.00517688988271299\\
421.01	0.00517978185059545\\
422.01	0.0051827286988378\\
423.01	0.00518573156653616\\
424.01	0.00518879162049825\\
425.01	0.0051919100558333\\
426.01	0.00519508809653527\\
427.01	0.00519832699605861\\
428.01	0.005201628037886\\
429.01	0.00520499253608783\\
430.01	0.0052084218358726\\
431.01	0.00521191731412986\\
432.01	0.00521548037996471\\
433.01	0.00521911247522648\\
434.01	0.00522281507503175\\
435.01	0.00522658968828514\\
436.01	0.005230437858198\\
437.01	0.00523436116281045\\
438.01	0.00523836121551853\\
439.01	0.0052424396656111\\
440.01	0.00524659819881913\\
441.01	0.00525083853788445\\
442.01	0.00525516244315059\\
443.01	0.00525957171318059\\
444.01	0.00526406818540756\\
445.01	0.0052686537368216\\
446.01	0.0052733302846973\\
447.01	0.00527809978736513\\
448.01	0.00528296424502851\\
449.01	0.00528792570062933\\
450.01	0.00529298624076026\\
451.01	0.00529814799662302\\
452.01	0.00530341314502969\\
453.01	0.0053087839094425\\
454.01	0.00531426256104464\\
455.01	0.00531985141983556\\
456.01	0.00532555285574184\\
457.01	0.0053313692897333\\
458.01	0.00533730319493714\\
459.01	0.00534335709773871\\
460.01	0.00534953357886434\\
461.01	0.00535583527443989\\
462.01	0.00536226487702372\\
463.01	0.00536882513661484\\
464.01	0.00537551886164229\\
465.01	0.00538234891994208\\
466.01	0.0053893182397339\\
467.01	0.0053964298106058\\
468.01	0.00540368668452083\\
469.01	0.00541109197685086\\
470.01	0.0054186488674464\\
471.01	0.00542636060174487\\
472.01	0.00543423049192164\\
473.01	0.00544226191808676\\
474.01	0.00545045832952867\\
475.01	0.00545882324600859\\
476.01	0.00546736025910498\\
477.01	0.00547607303360939\\
478.01	0.00548496530897408\\
479.01	0.00549404090080845\\
480.01	0.00550330370242497\\
481.01	0.00551275768643106\\
482.01	0.00552240690636421\\
483.01	0.00553225549836771\\
484.01	0.00554230768290534\\
485.01	0.00555256776651032\\
486.01	0.00556304014356669\\
487.01	0.00557372929812386\\
488.01	0.00558463980574016\\
489.01	0.00559577633535974\\
490.01	0.00560714365122079\\
491.01	0.00561874661479977\\
492.01	0.00563059018679275\\
493.01	0.00564267942913979\\
494.01	0.00565501950709213\\
495.01	0.00566761569132816\\
496.01	0.00568047336011608\\
497.01	0.00569359800152755\\
498.01	0.00570699521569809\\
499.01	0.00572067071713506\\
500.01	0.0057346303370701\\
501.01	0.00574888002585278\\
502.01	0.00576342585538613\\
503.01	0.00577827402159808\\
504.01	0.00579343084694856\\
505.01	0.00580890278297078\\
506.01	0.00582469641284311\\
507.01	0.0058408184539908\\
508.01	0.00585727576071554\\
509.01	0.00587407532685036\\
510.01	0.00589122428843934\\
511.01	0.00590872992643619\\
512.01	0.00592659966942349\\
513.01	0.0059448410963448\\
514.01	0.00596346193924731\\
515.01	0.00598247008602941\\
516.01	0.00600187358318675\\
517.01	0.00602168063855298\\
518.01	0.0060418996240226\\
519.01	0.00606253907825346\\
520.01	0.00608360770933741\\
521.01	0.00610511439742895\\
522.01	0.00612706819732234\\
523.01	0.00614947834096546\\
524.01	0.00617235423989611\\
525.01	0.00619570548758692\\
526.01	0.00621954186168196\\
527.01	0.0062438733261071\\
528.01	0.00626871003303373\\
529.01	0.00629406232467292\\
530.01	0.00631994073487588\\
531.01	0.00634635599051178\\
532.01	0.00637331901259404\\
533.01	0.00640084091711891\\
534.01	0.00642893301558123\\
535.01	0.00645760681512446\\
536.01	0.00648687401827952\\
537.01	0.00651674652224228\\
538.01	0.00654723641763366\\
539.01	0.00657835598668036\\
540.01	0.00661011770074879\\
541.01	0.00664253421715752\\
542.01	0.00667561837518443\\
543.01	0.00670938319117929\\
544.01	0.00674384185268058\\
545.01	0.00677900771142645\\
546.01	0.00681489427513986\\
547.01	0.00685151519795303\\
548.01	0.00688888426932745\\
549.01	0.00692701540130676\\
550.01	0.00696592261392825\\
551.01	0.00700562001860003\\
552.01	0.00704612179923281\\
553.01	0.00708744219089684\\
554.01	0.00712959545574979\\
555.01	0.00717259585596489\\
556.01	0.00721645762335742\\
557.01	0.00726119492538552\\
558.01	0.00730682182717329\\
559.01	0.00735335224917305\\
560.01	0.00740079992005648\\
561.01	0.0074491783243876\\
562.01	0.0074985006446055\\
563.01	0.00754877969680245\\
564.01	0.00760002785975936\\
565.01	0.00765225699666378\\
566.01	0.00770547836890862\\
567.01	0.00775970254134385\\
568.01	0.00781493927833469\\
569.01	0.00787119742996866\\
570.01	0.00792848480775271\\
571.01	0.00798680804915989\\
572.01	0.00804617247041955\\
573.01	0.00810658190700895\\
574.01	0.00816803854140444\\
575.01	0.00823054271779196\\
576.01	0.00829409274364186\\
577.01	0.00835868467832643\\
578.01	0.00842431210933126\\
579.01	0.00849096591709788\\
580.01	0.0085586340301731\\
581.01	0.00862730117316442\\
582.01	0.00869694861105789\\
583.01	0.0087675538948034\\
584.01	0.00883909061478503\\
585.01	0.00891152817095808\\
586.01	0.00898483157115984\\
587.01	0.00905896127252333\\
588.01	0.00913387308520954\\
589.01	0.00920951816302855\\
590.01	0.00928584311219743\\
591.01	0.00936279025779631\\
592.01	0.00944029811781628\\
593.01	0.00951830214751989\\
594.01	0.00959673583273945\\
595.01	0.00967553223043691\\
596.01	0.00975462607922802\\
597.01	0.00983384409165019\\
598.01	0.00990866201848071\\
599.01	0.00997087280416276\\
599.02	0.00997138072163725\\
599.03	0.00997188557625799\\
599.04	0.00997238733820211\\
599.05	0.00997288597735272\\
599.06	0.00997338146329604\\
599.07	0.00997387376531845\\
599.08	0.00997436285240349\\
599.09	0.00997484869322892\\
599.1	0.00997533125616361\\
599.11	0.00997581050926455\\
599.12	0.00997628642027372\\
599.13	0.00997675895661497\\
599.14	0.0099772280853909\\
599.15	0.00997769377337961\\
599.16	0.00997815598703157\\
599.17	0.00997861469246631\\
599.18	0.00997906985546915\\
599.19	0.00997952144148792\\
599.2	0.00997996941562957\\
599.21	0.00998041374265684\\
599.22	0.0099808543869848\\
599.23	0.00998129131267744\\
599.24	0.00998172448344418\\
599.25	0.00998215386263636\\
599.26	0.00998257941258354\\
599.27	0.00998300109279825\\
599.28	0.00998341886238981\\
599.29	0.00998383268006024\\
599.3	0.00998424250410032\\
599.31	0.00998464829238543\\
599.32	0.00998505000237151\\
599.33	0.00998544759109087\\
599.34	0.00998584101514799\\
599.35	0.00998623023071532\\
599.36	0.00998661519352895\\
599.37	0.00998699585888436\\
599.38	0.00998737218163197\\
599.39	0.00998774411617281\\
599.4	0.00998811161645403\\
599.41	0.00998847463596443\\
599.42	0.0099888331277299\\
599.43	0.00998918704430885\\
599.44	0.00998953633778757\\
599.45	0.00998988095977558\\
599.46	0.00999022086140088\\
599.47	0.00999055599330522\\
599.48	0.00999088630563925\\
599.49	0.00999121174805768\\
599.5	0.00999153226971435\\
599.51	0.0099918478192573\\
599.52	0.00999215834482374\\
599.53	0.00999246379403503\\
599.54	0.0099927641139915\\
599.55	0.00999305925126738\\
599.56	0.00999334915190553\\
599.57	0.0099936337614122\\
599.58	0.00999391302475173\\
599.59	0.00999418688634118\\
599.6	0.00999445529004489\\
599.61	0.00999471817916904\\
599.62	0.00999497549645613\\
599.63	0.00999522718407935\\
599.64	0.009995473183637\\
599.65	0.00999571343614678\\
599.66	0.00999594788204004\\
599.67	0.00999617646115598\\
599.68	0.0099963991127358\\
599.69	0.00999661577541675\\
599.7	0.00999682638722618\\
599.71	0.00999703088557551\\
599.72	0.00999722920725411\\
599.73	0.00999742128842315\\
599.74	0.00999760706460942\\
599.75	0.00999778647069897\\
599.76	0.00999795944093086\\
599.77	0.0099981259088907\\
599.78	0.0099982858075042\\
599.79	0.00999843906903064\\
599.8	0.00999858562505627\\
599.81	0.00999872540648767\\
599.82	0.00999885834354498\\
599.83	0.00999898436575515\\
599.84	0.00999910340194508\\
599.85	0.00999921538023465\\
599.86	0.00999932022802977\\
599.87	0.00999941787201528\\
599.88	0.00999950823814785\\
599.89	0.00999959125164875\\
599.9	0.00999966683699656\\
599.91	0.00999973491791987\\
599.92	0.0099997954173898\\
599.93	0.00999984825761255\\
599.94	0.00999989336002181\\
599.95	0.00999993064527112\\
599.96	0.00999996003322615\\
599.97	0.00999998144295691\\
599.98	0.00999999479272987\\
599.99	0.01\\
600	0.01\\
};
\addplot [color=mycolor3,solid,forget plot]
  table[row sep=crcr]{%
0.01	0.0050016280518921\\
1.01	0.00500162904994192\\
2.01	0.00500163006827589\\
3.01	0.00500163110730512\\
4.01	0.00500163216744873\\
5.01	0.00500163324913459\\
6.01	0.00500163435279902\\
7.01	0.00500163547888721\\
8.01	0.00500163662785354\\
9.01	0.00500163780016109\\
10.01	0.00500163899628275\\
11.01	0.00500164021670073\\
12.01	0.0050016414619069\\
13.01	0.00500164273240315\\
14.01	0.00500164402870142\\
15.01	0.00500164535132418\\
16.01	0.00500164670080426\\
17.01	0.00500164807768509\\
18.01	0.00500164948252126\\
19.01	0.00500165091587854\\
20.01	0.00500165237833407\\
21.01	0.00500165387047644\\
22.01	0.00500165539290649\\
23.01	0.00500165694623676\\
24.01	0.00500165853109226\\
25.01	0.00500166014811082\\
26.01	0.00500166179794266\\
27.01	0.00500166348125161\\
28.01	0.0050016651987143\\
29.01	0.00500166695102154\\
30.01	0.00500166873887767\\
31.01	0.0050016705630014\\
32.01	0.00500167242412572\\
33.01	0.00500167432299854\\
34.01	0.0050016762603827\\
35.01	0.00500167823705644\\
36.01	0.00500168025381365\\
37.01	0.00500168231146401\\
38.01	0.00500168441083404\\
39.01	0.00500168655276619\\
40.01	0.00500168873812026\\
41.01	0.00500169096777317\\
42.01	0.00500169324261952\\
43.01	0.00500169556357194\\
44.01	0.00500169793156127\\
45.01	0.00500170034753701\\
46.01	0.00500170281246805\\
47.01	0.00500170532734243\\
48.01	0.00500170789316817\\
49.01	0.00500171051097331\\
50.01	0.00500171318180696\\
51.01	0.00500171590673891\\
52.01	0.00500171868686054\\
53.01	0.00500172152328528\\
54.01	0.00500172441714855\\
55.01	0.00500172736960895\\
56.01	0.00500173038184804\\
57.01	0.00500173345507127\\
58.01	0.00500173659050805\\
59.01	0.00500173978941269\\
60.01	0.00500174305306444\\
61.01	0.00500174638276826\\
62.01	0.00500174977985532\\
63.01	0.0050017532456833\\
64.01	0.00500175678163739\\
65.01	0.00500176038913015\\
66.01	0.00500176406960282\\
67.01	0.00500176782452535\\
68.01	0.00500177165539681\\
69.01	0.00500177556374673\\
70.01	0.00500177955113498\\
71.01	0.00500178361915304\\
72.01	0.00500178776942375\\
73.01	0.00500179200360257\\
74.01	0.00500179632337816\\
75.01	0.00500180073047306\\
76.01	0.00500180522664397\\
77.01	0.00500180981368298\\
78.01	0.00500181449341772\\
79.01	0.00500181926771266\\
80.01	0.0050018241384694\\
81.01	0.00500182910762757\\
82.01	0.00500183417716536\\
83.01	0.0050018393491007\\
84.01	0.00500184462549172\\
85.01	0.00500185000843775\\
86.01	0.00500185550007982\\
87.01	0.00500186110260194\\
88.01	0.00500186681823142\\
89.01	0.00500187264924018\\
90.01	0.00500187859794544\\
91.01	0.00500188466671074\\
92.01	0.00500189085794667\\
93.01	0.00500189717411179\\
94.01	0.00500190361771366\\
95.01	0.00500191019131004\\
96.01	0.00500191689750928\\
97.01	0.00500192373897206\\
98.01	0.00500193071841186\\
99.01	0.00500193783859608\\
100.01	0.00500194510234757\\
101.01	0.00500195251254507\\
102.01	0.00500196007212453\\
103.01	0.0050019677840805\\
104.01	0.00500197565146715\\
105.01	0.0050019836773994\\
106.01	0.00500199186505378\\
107.01	0.00500200021767026\\
108.01	0.00500200873855308\\
109.01	0.00500201743107212\\
110.01	0.00500202629866431\\
111.01	0.00500203534483475\\
112.01	0.00500204457315789\\
113.01	0.00500205398727957\\
114.01	0.00500206359091773\\
115.01	0.00500207338786419\\
116.01	0.00500208338198575\\
117.01	0.00500209357722599\\
118.01	0.00500210397760696\\
119.01	0.00500211458722999\\
120.01	0.00500212541027791\\
121.01	0.00500213645101627\\
122.01	0.00500214771379517\\
123.01	0.00500215920305093\\
124.01	0.00500217092330749\\
125.01	0.00500218287917836\\
126.01	0.00500219507536824\\
127.01	0.00500220751667511\\
128.01	0.00500222020799147\\
129.01	0.0050022331543068\\
130.01	0.00500224636070901\\
131.01	0.00500225983238653\\
132.01	0.00500227357463011\\
133.01	0.00500228759283523\\
134.01	0.00500230189250367\\
135.01	0.00500231647924548\\
136.01	0.00500233135878173\\
137.01	0.00500234653694627\\
138.01	0.00500236201968751\\
139.01	0.00500237781307142\\
140.01	0.00500239392328344\\
141.01	0.00500241035663061\\
142.01	0.00500242711954409\\
143.01	0.00500244421858152\\
144.01	0.00500246166042968\\
145.01	0.00500247945190674\\
146.01	0.00500249759996471\\
147.01	0.00500251611169228\\
148.01	0.00500253499431723\\
149.01	0.0050025542552094\\
150.01	0.00500257390188305\\
151.01	0.00500259394199984\\
152.01	0.0050026143833716\\
153.01	0.00500263523396326\\
154.01	0.00500265650189594\\
155.01	0.00500267819544939\\
156.01	0.00500270032306574\\
157.01	0.005002722893352\\
158.01	0.00500274591508375\\
159.01	0.0050027693972076\\
160.01	0.00500279334884582\\
161.01	0.0050028177792977\\
162.01	0.00500284269804473\\
163.01	0.00500286811475308\\
164.01	0.00500289403927738\\
165.01	0.00500292048166448\\
166.01	0.00500294745215671\\
167.01	0.00500297496119584\\
168.01	0.0050030030194267\\
169.01	0.00500303163770119\\
170.01	0.00500306082708216\\
171.01	0.00500309059884714\\
172.01	0.00500312096449286\\
173.01	0.00500315193573883\\
174.01	0.00500318352453204\\
175.01	0.00500321574305071\\
176.01	0.005003248603709\\
177.01	0.00500328211916144\\
178.01	0.00500331630230723\\
179.01	0.00500335116629466\\
180.01	0.00500338672452627\\
181.01	0.00500342299066324\\
182.01	0.00500345997863021\\
183.01	0.00500349770262021\\
184.01	0.00500353617709963\\
185.01	0.00500357541681333\\
186.01	0.00500361543678975\\
187.01	0.00500365625234608\\
188.01	0.00500369787909372\\
189.01	0.00500374033294362\\
190.01	0.00500378363011175\\
191.01	0.00500382778712488\\
192.01	0.00500387282082589\\
193.01	0.0050039187483798\\
194.01	0.00500396558727978\\
195.01	0.00500401335535278\\
196.01	0.00500406207076591\\
197.01	0.00500411175203229\\
198.01	0.00500416241801774\\
199.01	0.00500421408794673\\
200.01	0.00500426678140897\\
201.01	0.00500432051836609\\
202.01	0.00500437531915836\\
203.01	0.00500443120451104\\
204.01	0.00500448819554175\\
205.01	0.00500454631376729\\
206.01	0.00500460558111075\\
207.01	0.00500466601990832\\
208.01	0.00500472765291744\\
209.01	0.00500479050332339\\
210.01	0.00500485459474725\\
211.01	0.00500491995125339\\
212.01	0.00500498659735742\\
213.01	0.00500505455803364\\
214.01	0.0050051238587237\\
215.01	0.00500519452534379\\
216.01	0.0050052665842935\\
217.01	0.0050053400624642\\
218.01	0.0050054149872469\\
219.01	0.00500549138654141\\
220.01	0.00500556928876456\\
221.01	0.00500564872285925\\
222.01	0.00500572971830321\\
223.01	0.00500581230511794\\
224.01	0.00500589651387844\\
225.01	0.00500598237572125\\
226.01	0.00500606992235524\\
227.01	0.00500615918607009\\
228.01	0.00500625019974606\\
229.01	0.00500634299686419\\
230.01	0.00500643761151554\\
231.01	0.00500653407841152\\
232.01	0.00500663243289368\\
233.01	0.00500673271094423\\
234.01	0.00500683494919568\\
235.01	0.00500693918494234\\
236.01	0.00500704545614977\\
237.01	0.0050071538014658\\
238.01	0.00500726426023136\\
239.01	0.00500737687249117\\
240.01	0.00500749167900487\\
241.01	0.00500760872125791\\
242.01	0.00500772804147255\\
243.01	0.00500784968261975\\
244.01	0.00500797368842978\\
245.01	0.00500810010340398\\
246.01	0.0050082289728267\\
247.01	0.00500836034277632\\
248.01	0.00500849426013733\\
249.01	0.00500863077261224\\
250.01	0.00500876992873327\\
251.01	0.00500891177787448\\
252.01	0.00500905637026401\\
253.01	0.00500920375699584\\
254.01	0.00500935399004249\\
255.01	0.0050095071222674\\
256.01	0.00500966320743702\\
257.01	0.00500982230023315\\
258.01	0.00500998445626602\\
259.01	0.00501014973208669\\
260.01	0.00501031818519961\\
261.01	0.00501048987407515\\
262.01	0.00501066485816323\\
263.01	0.00501084319790523\\
264.01	0.00501102495474763\\
265.01	0.00501121019115461\\
266.01	0.00501139897062119\\
267.01	0.00501159135768638\\
268.01	0.00501178741794621\\
269.01	0.00501198721806694\\
270.01	0.00501219082579847\\
271.01	0.00501239830998744\\
272.01	0.00501260974059091\\
273.01	0.00501282518868935\\
274.01	0.00501304472650031\\
275.01	0.0050132684273924\\
276.01	0.00501349636589818\\
277.01	0.00501372861772856\\
278.01	0.00501396525978616\\
279.01	0.00501420637017932\\
280.01	0.00501445202823633\\
281.01	0.00501470231451923\\
282.01	0.00501495731083836\\
283.01	0.00501521710026688\\
284.01	0.00501548176715487\\
285.01	0.00501575139714465\\
286.01	0.00501602607718557\\
287.01	0.00501630589554915\\
288.01	0.00501659094184514\\
289.01	0.00501688130703674\\
290.01	0.00501717708345732\\
291.01	0.00501747836482667\\
292.01	0.00501778524626806\\
293.01	0.00501809782432616\\
294.01	0.00501841619698477\\
295.01	0.00501874046368529\\
296.01	0.00501907072534664\\
297.01	0.00501940708438503\\
298.01	0.00501974964473477\\
299.01	0.00502009851187034\\
300.01	0.00502045379282888\\
301.01	0.00502081559623435\\
302.01	0.00502118403232309\\
303.01	0.00502155921296964\\
304.01	0.00502194125171489\\
305.01	0.00502233026379545\\
306.01	0.00502272636617557\\
307.01	0.00502312967757916\\
308.01	0.00502354031852612\\
309.01	0.00502395841136923\\
310.01	0.00502438408033382\\
311.01	0.00502481745156044\\
312.01	0.00502525865315002\\
313.01	0.00502570781521222\\
314.01	0.00502616506991675\\
315.01	0.00502663055154874\\
316.01	0.00502710439656688\\
317.01	0.00502758674366617\\
318.01	0.00502807773384525\\
319.01	0.00502857751047717\\
320.01	0.00502908621938583\\
321.01	0.00502960400892682\\
322.01	0.00503013103007399\\
323.01	0.00503066743651082\\
324.01	0.0050312133847286\\
325.01	0.00503176903412964\\
326.01	0.00503233454713715\\
327.01	0.00503291008931162\\
328.01	0.00503349582947334\\
329.01	0.00503409193983205\\
330.01	0.00503469859612314\\
331.01	0.005035315977751\\
332.01	0.00503594426793894\\
333.01	0.00503658365388542\\
334.01	0.00503723432692794\\
335.01	0.00503789648271236\\
336.01	0.00503857032136812\\
337.01	0.00503925604768955\\
338.01	0.00503995387132164\\
339.01	0.00504066400694936\\
340.01	0.00504138667449157\\
341.01	0.00504212209929511\\
342.01	0.00504287051233135\\
343.01	0.00504363215039098\\
344.01	0.00504440725627697\\
345.01	0.00504519607899295\\
346.01	0.00504599887392596\\
347.01	0.00504681590301874\\
348.01	0.00504764743493299\\
349.01	0.00504849374519711\\
350.01	0.00504935511633733\\
351.01	0.00505023183798901\\
352.01	0.00505112420698506\\
353.01	0.00505203252741686\\
354.01	0.00505295711066694\\
355.01	0.0050538982754085\\
356.01	0.00505485634756882\\
357.01	0.00505583166025526\\
358.01	0.00505682455364158\\
359.01	0.00505783537481175\\
360.01	0.00505886447756317\\
361.01	0.00505991222216809\\
362.01	0.0050609789750954\\
363.01	0.00506206510869642\\
364.01	0.00506317100085968\\
365.01	0.00506429703464094\\
366.01	0.00506544359787774\\
367.01	0.00506661108279842\\
368.01	0.00506779988563986\\
369.01	0.00506901040628567\\
370.01	0.00507024304794354\\
371.01	0.00507149821687677\\
372.01	0.00507277632220685\\
373.01	0.00507407777580396\\
374.01	0.0050754029922775\\
375.01	0.00507675238907809\\
376.01	0.0050781263867148\\
377.01	0.00507952540908478\\
378.01	0.00508094988390589\\
379.01	0.00508240024323037\\
380.01	0.0050838769240124\\
381.01	0.00508538036869029\\
382.01	0.00508691102574624\\
383.01	0.00508846935020078\\
384.01	0.00509005580401857\\
385.01	0.00509167085641588\\
386.01	0.0050933149840912\\
387.01	0.0050949886714239\\
388.01	0.00509669241067484\\
389.01	0.00509842670220022\\
390.01	0.00510019205467784\\
391.01	0.00510198898534765\\
392.01	0.00510381802026706\\
393.01	0.00510567969458225\\
394.01	0.0051075745528162\\
395.01	0.0051095031491746\\
396.01	0.0051114660478695\\
397.01	0.00511346382346355\\
398.01	0.00511549706123428\\
399.01	0.00511756635755942\\
400.01	0.00511967232032567\\
401.01	0.00512181556935933\\
402.01	0.00512399673688282\\
403.01	0.00512621646799463\\
404.01	0.00512847542117596\\
405.01	0.00513077426882418\\
406.01	0.00513311369781239\\
407.01	0.00513549441007806\\
408.01	0.00513791712323855\\
409.01	0.00514038257123667\\
410.01	0.00514289150501269\\
411.01	0.00514544469320723\\
412.01	0.00514804292289033\\
413.01	0.0051506870003202\\
414.01	0.00515337775172785\\
415.01	0.00515611602412834\\
416.01	0.005158902686157\\
417.01	0.00516173862892902\\
418.01	0.00516462476692038\\
419.01	0.00516756203886763\\
420.01	0.00517055140868483\\
421.01	0.00517359386639368\\
422.01	0.00517669042906353\\
423.01	0.00517984214175886\\
424.01	0.00518305007848732\\
425.01	0.00518631534314723\\
426.01	0.0051896390704676\\
427.01	0.00519302242693516\\
428.01	0.00519646661170586\\
429.01	0.00519997285749194\\
430.01	0.0052035424314217\\
431.01	0.00520717663586574\\
432.01	0.00521087680922418\\
433.01	0.00521464432667026\\
434.01	0.00521848060084656\\
435.01	0.00522238708250908\\
436.01	0.00522636526111817\\
437.01	0.00523041666537323\\
438.01	0.00523454286369303\\
439.01	0.00523874546464077\\
440.01	0.00524302611730065\\
441.01	0.00524738651160832\\
442.01	0.00525182837864539\\
443.01	0.00525635349090743\\
444.01	0.00526096366255775\\
445.01	0.00526566074968343\\
446.01	0.00527044665056876\\
447.01	0.00527532330600824\\
448.01	0.00528029269967614\\
449.01	0.00528535685857795\\
450.01	0.00529051785360128\\
451.01	0.00529577780018857\\
452.01	0.00530113885914815\\
453.01	0.00530660323761675\\
454.01	0.00531217319018241\\
455.01	0.005317851020168\\
456.01	0.00532363908106857\\
457.01	0.0053295397781255\\
458.01	0.00533555557001072\\
459.01	0.005341688970586\\
460.01	0.00534794255069161\\
461.01	0.00535431893991593\\
462.01	0.00536082082829249\\
463.01	0.00536745096787895\\
464.01	0.00537421217417684\\
465.01	0.00538110732737072\\
466.01	0.00538813937338328\\
467.01	0.00539531132476963\\
468.01	0.00540262626148751\\
469.01	0.00541008733159785\\
470.01	0.00541769775194074\\
471.01	0.00542546080881408\\
472.01	0.0054333798586714\\
473.01	0.00544145832885058\\
474.01	0.00544969971834659\\
475.01	0.00545810759864\\
476.01	0.00546668561459478\\
477.01	0.00547543748543522\\
478.01	0.00548436700581247\\
479.01	0.00549347804696805\\
480.01	0.00550277455799925\\
481.01	0.00551226056722783\\
482.01	0.0055219401836714\\
483.01	0.00553181759861202\\
484.01	0.00554189708725276\\
485.01	0.00555218301045088\\
486.01	0.00556267981651445\\
487.01	0.00557339204304411\\
488.01	0.00558432431880738\\
489.01	0.00559548136562726\\
490.01	0.0056068680002765\\
491.01	0.00561848913636992\\
492.01	0.00563034978625274\\
493.01	0.0056424550628901\\
494.01	0.00565481018177132\\
495.01	0.00566742046284165\\
496.01	0.00568029133248303\\
497.01	0.00569342832555668\\
498.01	0.00570683708752279\\
499.01	0.0057205233766401\\
500.01	0.00573449306624481\\
501.01	0.00574875214710439\\
502.01	0.00576330672983821\\
503.01	0.00577816304739932\\
504.01	0.00579332745760815\\
505.01	0.00580880644572997\\
506.01	0.00582460662709058\\
507.01	0.00584073474972176\\
508.01	0.00585719769703225\\
509.01	0.00587400249050098\\
510.01	0.00589115629238746\\
511.01	0.00590866640846227\\
512.01	0.00592654029075282\\
513.01	0.00594478554030617\\
514.01	0.00596340990996545\\
515.01	0.0059824213071583\\
516.01	0.00600182779669088\\
517.01	0.00602163760354031\\
518.01	0.00604185911563766\\
519.01	0.00606250088662946\\
520.01	0.00608357163860813\\
521.01	0.00610508026480032\\
522.01	0.00612703583220095\\
523.01	0.00614944758414021\\
524.01	0.00617232494277072\\
525.01	0.00619567751146097\\
526.01	0.00621951507707789\\
527.01	0.00624384761214108\\
528.01	0.00626868527682911\\
529.01	0.00629403842081533\\
530.01	0.00631991758490741\\
531.01	0.00634633350246382\\
532.01	0.00637329710055433\\
533.01	0.00640081950083352\\
534.01	0.00642891202008611\\
535.01	0.00645758617040558\\
536.01	0.00648685365895875\\
537.01	0.00651672638728716\\
538.01	0.00654721645008825\\
539.01	0.00657833613341638\\
540.01	0.00661009791223496\\
541.01	0.00664251444724473\\
542.01	0.00667559858090686\\
543.01	0.00670936333256887\\
544.01	0.00674382189259384\\
545.01	0.00677898761538313\\
546.01	0.00681487401117106\\
547.01	0.00685149473646011\\
548.01	0.00688886358294855\\
549.01	0.00692699446479224\\
550.01	0.00696590140402426\\
551.01	0.00700559851393984\\
552.01	0.0070460999802358\\
553.01	0.00708742003967412\\
554.01	0.00712957295601932\\
555.01	0.00717257299297346\\
556.01	0.00721643438381055\\
557.01	0.00726117129738661\\
558.01	0.00730679780017085\\
559.01	0.00735332781391837\\
560.01	0.00740077506856995\\
561.01	0.00744915304993686\\
562.01	0.0074984749416927\\
563.01	0.00754875356116406\\
564.01	0.00760000128837737\\
565.01	0.00765222998778898\\
566.01	0.00770545092209638\\
567.01	0.00775967465750245\\
568.01	0.0078149109597867\\
569.01	0.00787116868052467\\
570.01	0.00792845563279784\\
571.01	0.0079867784557522\\
572.01	0.00804614246739978\\
573.01	0.00810655150512116\\
574.01	0.00816800775342491\\
575.01	0.00823051155866612\\
576.01	0.00829406123062584\\
577.01	0.00835865283113084\\
578.01	0.00842427995026335\\
579.01	0.00849093347119724\\
580.01	0.00855860132533547\\
581.01	0.00862726824024612\\
582.01	0.00869691548395165\\
583.01	0.00876752061047502\\
584.01	0.00883905721325699\\
585.01	0.00891149469522319\\
586.01	0.00898479806700421\\
587.01	0.00905892778823375\\
588.01	0.0091338396711344\\
589.01	0.00920948487095468\\
590.01	0.00928580999449757\\
591.01	0.00936275736629181\\
592.01	0.00944026550228819\\
593.01	0.00951826985378823\\
594.01	0.00959670390021289\\
595.01	0.00967550068901628\\
596.01	0.00975459494542325\\
597.01	0.00983382663319375\\
598.01	0.00990866201848071\\
599.01	0.00997087280416276\\
599.02	0.00997138072163725\\
599.03	0.009971885576258\\
599.04	0.00997238733820211\\
599.05	0.00997288597735272\\
599.06	0.00997338146329604\\
599.07	0.00997387376531845\\
599.08	0.00997436285240349\\
599.09	0.00997484869322892\\
599.1	0.00997533125616361\\
599.11	0.00997581050926455\\
599.12	0.00997628642027372\\
599.13	0.00997675895661498\\
599.14	0.0099772280853909\\
599.15	0.00997769377337961\\
599.16	0.00997815598703157\\
599.17	0.00997861469246631\\
599.18	0.00997906985546915\\
599.19	0.00997952144148792\\
599.2	0.00997996941562957\\
599.21	0.00998041374265684\\
599.22	0.0099808543869848\\
599.23	0.00998129131267744\\
599.24	0.00998172448344418\\
599.25	0.00998215386263636\\
599.26	0.00998257941258353\\
599.27	0.00998300109279825\\
599.28	0.00998341886238981\\
599.29	0.00998383268006024\\
599.3	0.00998424250410032\\
599.31	0.00998464829238543\\
599.32	0.00998505000237151\\
599.33	0.00998544759109087\\
599.34	0.00998584101514799\\
599.35	0.00998623023071532\\
599.36	0.00998661519352895\\
599.37	0.00998699585888436\\
599.38	0.00998737218163197\\
599.39	0.00998774411617281\\
599.4	0.00998811161645403\\
599.41	0.00998847463596443\\
599.42	0.0099888331277299\\
599.43	0.00998918704430885\\
599.44	0.00998953633778757\\
599.45	0.00998988095977558\\
599.46	0.00999022086140088\\
599.47	0.00999055599330522\\
599.48	0.00999088630563925\\
599.49	0.00999121174805768\\
599.5	0.00999153226971435\\
599.51	0.0099918478192573\\
599.52	0.00999215834482374\\
599.53	0.00999246379403503\\
599.54	0.0099927641139915\\
599.55	0.00999305925126738\\
599.56	0.00999334915190553\\
599.57	0.0099936337614122\\
599.58	0.00999391302475173\\
599.59	0.00999418688634118\\
599.6	0.00999445529004489\\
599.61	0.00999471817916904\\
599.62	0.00999497549645613\\
599.63	0.00999522718407934\\
599.64	0.009995473183637\\
599.65	0.00999571343614678\\
599.66	0.00999594788204004\\
599.67	0.00999617646115599\\
599.68	0.0099963991127358\\
599.69	0.00999661577541675\\
599.7	0.00999682638722618\\
599.71	0.00999703088557551\\
599.72	0.00999722920725411\\
599.73	0.00999742128842315\\
599.74	0.00999760706460942\\
599.75	0.00999778647069897\\
599.76	0.00999795944093086\\
599.77	0.0099981259088907\\
599.78	0.0099982858075042\\
599.79	0.00999843906903064\\
599.8	0.00999858562505627\\
599.81	0.00999872540648767\\
599.82	0.00999885834354498\\
599.83	0.00999898436575515\\
599.84	0.00999910340194508\\
599.85	0.00999921538023465\\
599.86	0.00999932022802977\\
599.87	0.00999941787201528\\
599.88	0.00999950823814785\\
599.89	0.00999959125164875\\
599.9	0.00999966683699656\\
599.91	0.00999973491791987\\
599.92	0.0099997954173898\\
599.93	0.00999984825761255\\
599.94	0.00999989336002181\\
599.95	0.00999993064527112\\
599.96	0.00999996003322615\\
599.97	0.00999998144295691\\
599.98	0.00999999479272987\\
599.99	0.01\\
600	0.01\\
};
\addplot [color=mycolor4,solid,forget plot]
  table[row sep=crcr]{%
0.01	0.00496755019417924\\
1.01	0.0049675512405473\\
2.01	0.00496755230832586\\
3.01	0.00496755339795235\\
4.01	0.00496755450987339\\
5.01	0.00496755564454414\\
6.01	0.00496755680242982\\
7.01	0.00496755798400442\\
8.01	0.00496755918975203\\
9.01	0.00496756042016645\\
10.01	0.0049675616757516\\
11.01	0.0049675629570216\\
12.01	0.00496756426450115\\
13.01	0.00496756559872562\\
14.01	0.00496756696024118\\
15.01	0.00496756834960534\\
16.01	0.00496756976738685\\
17.01	0.00496757121416612\\
18.01	0.0049675726905352\\
19.01	0.00496757419709846\\
20.01	0.00496757573447246\\
21.01	0.00496757730328634\\
22.01	0.00496757890418198\\
23.01	0.00496758053781456\\
24.01	0.00496758220485252\\
25.01	0.0049675839059775\\
26.01	0.00496758564188582\\
27.01	0.00496758741328719\\
28.01	0.00496758922090651\\
29.01	0.00496759106548292\\
30.01	0.0049675929477707\\
31.01	0.00496759486853966\\
32.01	0.00496759682857525\\
33.01	0.00496759882867888\\
34.01	0.00496760086966812\\
35.01	0.00496760295237748\\
36.01	0.00496760507765827\\
37.01	0.00496760724637924\\
38.01	0.00496760945942672\\
39.01	0.00496761171770513\\
40.01	0.00496761402213719\\
41.01	0.0049676163736647\\
42.01	0.00496761877324842\\
43.01	0.00496762122186858\\
44.01	0.0049676237205255\\
45.01	0.00496762627023999\\
46.01	0.00496762887205328\\
47.01	0.00496763152702795\\
48.01	0.0049676342362482\\
49.01	0.00496763700082062\\
50.01	0.00496763982187355\\
51.01	0.00496764270055901\\
52.01	0.00496764563805219\\
53.01	0.004967648635552\\
54.01	0.00496765169428197\\
55.01	0.00496765481549043\\
56.01	0.00496765800045121\\
57.01	0.00496766125046365\\
58.01	0.00496766456685403\\
59.01	0.00496766795097521\\
60.01	0.0049676714042078\\
61.01	0.0049676749279603\\
62.01	0.00496767852366987\\
63.01	0.00496768219280302\\
64.01	0.0049676859368558\\
65.01	0.00496768975735483\\
66.01	0.00496769365585771\\
67.01	0.00496769763395352\\
68.01	0.00496770169326417\\
69.01	0.00496770583544404\\
70.01	0.00496771006218102\\
71.01	0.00496771437519765\\
72.01	0.0049677187762512\\
73.01	0.00496772326713463\\
74.01	0.00496772784967736\\
75.01	0.00496773252574605\\
76.01	0.00496773729724513\\
77.01	0.00496774216611764\\
78.01	0.00496774713434632\\
79.01	0.00496775220395372\\
80.01	0.0049677573770037\\
81.01	0.004967762655602\\
82.01	0.00496776804189705\\
83.01	0.00496777353808089\\
84.01	0.00496777914638992\\
85.01	0.00496778486910593\\
86.01	0.00496779070855712\\
87.01	0.00496779666711872\\
88.01	0.0049678027472143\\
89.01	0.0049678089513164\\
90.01	0.00496781528194765\\
91.01	0.00496782174168196\\
92.01	0.00496782833314528\\
93.01	0.00496783505901684\\
94.01	0.00496784192203031\\
95.01	0.00496784892497434\\
96.01	0.00496785607069442\\
97.01	0.00496786336209352\\
98.01	0.00496787080213341\\
99.01	0.00496787839383603\\
100.01	0.00496788614028412\\
101.01	0.00496789404462306\\
102.01	0.00496790211006197\\
103.01	0.00496791033987477\\
104.01	0.00496791873740164\\
105.01	0.00496792730605047\\
106.01	0.00496793604929797\\
107.01	0.00496794497069106\\
108.01	0.0049679540738487\\
109.01	0.00496796336246286\\
110.01	0.00496797284030015\\
111.01	0.00496798251120333\\
112.01	0.00496799237909287\\
113.01	0.00496800244796837\\
114.01	0.0049680127219105\\
115.01	0.004968023205082\\
116.01	0.00496803390173005\\
117.01	0.00496804481618734\\
118.01	0.00496805595287435\\
119.01	0.00496806731630038\\
120.01	0.00496807891106613\\
121.01	0.00496809074186499\\
122.01	0.00496810281348509\\
123.01	0.00496811513081118\\
124.01	0.00496812769882653\\
125.01	0.00496814052261496\\
126.01	0.00496815360736277\\
127.01	0.00496816695836078\\
128.01	0.00496818058100656\\
129.01	0.00496819448080647\\
130.01	0.00496820866337745\\
131.01	0.00496822313445001\\
132.01	0.00496823789987005\\
133.01	0.00496825296560104\\
134.01	0.00496826833772641\\
135.01	0.00496828402245247\\
136.01	0.00496830002610989\\
137.01	0.00496831635515706\\
138.01	0.00496833301618216\\
139.01	0.0049683500159058\\
140.01	0.0049683673611837\\
141.01	0.0049683850590095\\
142.01	0.00496840311651728\\
143.01	0.00496842154098463\\
144.01	0.00496844033983493\\
145.01	0.00496845952064086\\
146.01	0.0049684790911274\\
147.01	0.00496849905917386\\
148.01	0.00496851943281836\\
149.01	0.00496854022025964\\
150.01	0.00496856142986134\\
151.01	0.00496858307015448\\
152.01	0.00496860514984107\\
153.01	0.00496862767779743\\
154.01	0.00496865066307772\\
155.01	0.00496867411491754\\
156.01	0.004968698042737\\
157.01	0.00496872245614495\\
158.01	0.00496874736494229\\
159.01	0.00496877277912618\\
160.01	0.0049687987088929\\
161.01	0.00496882516464336\\
162.01	0.00496885215698574\\
163.01	0.00496887969674001\\
164.01	0.00496890779494216\\
165.01	0.00496893646284843\\
166.01	0.00496896571193949\\
167.01	0.00496899555392501\\
168.01	0.00496902600074794\\
169.01	0.00496905706458917\\
170.01	0.00496908875787194\\
171.01	0.00496912109326691\\
172.01	0.0049691540836969\\
173.01	0.00496918774234145\\
174.01	0.00496922208264239\\
175.01	0.00496925711830809\\
176.01	0.00496929286331974\\
177.01	0.00496932933193543\\
178.01	0.00496936653869663\\
179.01	0.00496940449843285\\
180.01	0.00496944322626755\\
181.01	0.00496948273762356\\
182.01	0.0049695230482293\\
183.01	0.0049695641741242\\
184.01	0.00496960613166485\\
185.01	0.00496964893753113\\
186.01	0.00496969260873233\\
187.01	0.00496973716261352\\
188.01	0.00496978261686203\\
189.01	0.00496982898951351\\
190.01	0.0049698762989593\\
191.01	0.00496992456395258\\
192.01	0.00496997380361565\\
193.01	0.00497002403744673\\
194.01	0.00497007528532704\\
195.01	0.00497012756752824\\
196.01	0.00497018090471969\\
197.01	0.00497023531797575\\
198.01	0.00497029082878378\\
199.01	0.00497034745905163\\
200.01	0.00497040523111546\\
201.01	0.00497046416774798\\
202.01	0.00497052429216663\\
203.01	0.00497058562804158\\
204.01	0.00497064819950424\\
205.01	0.00497071203115605\\
206.01	0.00497077714807698\\
207.01	0.0049708435758344\\
208.01	0.0049709113404918\\
209.01	0.00497098046861857\\
210.01	0.00497105098729865\\
211.01	0.00497112292414026\\
212.01	0.00497119630728528\\
213.01	0.00497127116541907\\
214.01	0.00497134752778053\\
215.01	0.00497142542417174\\
216.01	0.00497150488496835\\
217.01	0.00497158594112979\\
218.01	0.00497166862420986\\
219.01	0.00497175296636723\\
220.01	0.00497183900037647\\
221.01	0.00497192675963833\\
222.01	0.00497201627819194\\
223.01	0.00497210759072489\\
224.01	0.00497220073258527\\
225.01	0.00497229573979346\\
226.01	0.00497239264905308\\
227.01	0.00497249149776331\\
228.01	0.00497259232403127\\
229.01	0.00497269516668333\\
230.01	0.00497280006527831\\
231.01	0.00497290706011921\\
232.01	0.00497301619226651\\
233.01	0.00497312750355068\\
234.01	0.00497324103658504\\
235.01	0.00497335683477877\\
236.01	0.00497347494235049\\
237.01	0.00497359540434152\\
238.01	0.00497371826662901\\
239.01	0.00497384357594036\\
240.01	0.00497397137986627\\
241.01	0.00497410172687499\\
242.01	0.0049742346663265\\
243.01	0.00497437024848646\\
244.01	0.00497450852454065\\
245.01	0.00497464954660931\\
246.01	0.00497479336776174\\
247.01	0.00497494004203085\\
248.01	0.00497508962442793\\
249.01	0.00497524217095781\\
250.01	0.00497539773863309\\
251.01	0.00497555638549001\\
252.01	0.00497571817060258\\
253.01	0.00497588315409864\\
254.01	0.0049760513971745\\
255.01	0.00497622296211021\\
256.01	0.00497639791228509\\
257.01	0.00497657631219313\\
258.01	0.00497675822745788\\
259.01	0.00497694372484804\\
260.01	0.00497713287229314\\
261.01	0.00497732573889832\\
262.01	0.00497752239496002\\
263.01	0.00497772291198105\\
264.01	0.00497792736268579\\
265.01	0.00497813582103557\\
266.01	0.0049783483622434\\
267.01	0.00497856506278914\\
268.01	0.00497878600043404\\
269.01	0.00497901125423583\\
270.01	0.00497924090456307\\
271.01	0.00497947503310968\\
272.01	0.00497971372290859\\
273.01	0.00497995705834638\\
274.01	0.00498020512517662\\
275.01	0.00498045801053344\\
276.01	0.00498071580294483\\
277.01	0.00498097859234527\\
278.01	0.00498124647008918\\
279.01	0.00498151952896214\\
280.01	0.00498179786319315\\
281.01	0.00498208156846677\\
282.01	0.00498237074193363\\
283.01	0.00498266548222114\\
284.01	0.00498296588944425\\
285.01	0.00498327206521498\\
286.01	0.00498358411265164\\
287.01	0.00498390213638834\\
288.01	0.00498422624258228\\
289.01	0.00498455653892249\\
290.01	0.00498489313463672\\
291.01	0.0049852361404982\\
292.01	0.00498558566883212\\
293.01	0.00498594183352066\\
294.01	0.00498630475000855\\
295.01	0.0049866745353076\\
296.01	0.0049870513080004\\
297.01	0.00498743518824385\\
298.01	0.00498782629777228\\
299.01	0.00498822475989989\\
300.01	0.00498863069952266\\
301.01	0.00498904424312037\\
302.01	0.00498946551875726\\
303.01	0.00498989465608378\\
304.01	0.00499033178633712\\
305.01	0.00499077704234251\\
306.01	0.00499123055851343\\
307.01	0.00499169247085336\\
308.01	0.00499216291695664\\
309.01	0.00499264203601017\\
310.01	0.00499312996879676\\
311.01	0.00499362685769801\\
312.01	0.00499413284669889\\
313.01	0.0049946480813938\\
314.01	0.0049951727089943\\
315.01	0.00499570687833882\\
316.01	0.00499625073990526\\
317.01	0.00499680444582594\\
318.01	0.00499736814990544\\
319.01	0.00499794200764365\\
320.01	0.00499852617626204\\
321.01	0.00499912081473511\\
322.01	0.00499972608382755\\
323.01	0.00500034214613818\\
324.01	0.00500096916614998\\
325.01	0.00500160731028902\\
326.01	0.00500225674699113\\
327.01	0.0050029176467794\\
328.01	0.00500359018235161\\
329.01	0.00500427452867935\\
330.01	0.00500497086312048\\
331.01	0.00500567936554518\\
332.01	0.00500640021847691\\
333.01	0.00500713360725059\\
334.01	0.00500787972018706\\
335.01	0.00500863874878736\\
336.01	0.00500941088794696\\
337.01	0.00501019633619076\\
338.01	0.00501099529593065\\
339.01	0.00501180797374665\\
340.01	0.0050126345806918\\
341.01	0.0050134753326232\\
342.01	0.00501433045055741\\
343.01	0.00501520016105224\\
344.01	0.00501608469661437\\
345.01	0.0050169842961323\\
346.01	0.0050178992053329\\
347.01	0.00501882967726136\\
348.01	0.00501977597278112\\
349.01	0.00502073836109178\\
350.01	0.00502171712025976\\
351.01	0.0050227125377582\\
352.01	0.00502372491100889\\
353.01	0.00502475454791997\\
354.01	0.00502580176740999\\
355.01	0.00502686689990904\\
356.01	0.00502795028782525\\
357.01	0.00502905228596373\\
358.01	0.00503017326188299\\
359.01	0.0050313135961752\\
360.01	0.00503247368265083\\
361.01	0.00503365392841168\\
362.01	0.00503485475379381\\
363.01	0.00503607659216244\\
364.01	0.00503731988954131\\
365.01	0.00503858510406292\\
366.01	0.00503987270522489\\
367.01	0.00504118317294898\\
368.01	0.005042516996438\\
369.01	0.00504387467284059\\
370.01	0.00504525670573876\\
371.01	0.00504666360348999\\
372.01	0.00504809587746541\\
373.01	0.00504955404024386\\
374.01	0.00505103860383685\\
375.01	0.0050525500780346\\
376.01	0.0050540889689753\\
377.01	0.0050556557780482\\
378.01	0.00505725100123884\\
379.01	0.0050588751290104\\
380.01	0.0050605286467818\\
381.01	0.00506221203601141\\
382.01	0.00506392577581308\\
383.01	0.00506567034493595\\
384.01	0.00506744622382201\\
385.01	0.00506925389636752\\
386.01	0.00507109385100737\\
387.01	0.00507296658105322\\
388.01	0.0050748725847341\\
389.01	0.00507681236521759\\
390.01	0.00507878643064001\\
391.01	0.00508079529414431\\
392.01	0.00508283947392883\\
393.01	0.00508491949330739\\
394.01	0.0050870358807824\\
395.01	0.00508918917013286\\
396.01	0.00509137990051904\\
397.01	0.00509360861660459\\
398.01	0.00509587586870029\\
399.01	0.00509818221292972\\
400.01	0.00510052821141903\\
401.01	0.00510291443251547\\
402.01	0.005105341451034\\
403.01	0.00510780984853788\\
404.01	0.00511032021365325\\
405.01	0.00511287314242206\\
406.01	0.00511546923869705\\
407.01	0.00511810911457931\\
408.01	0.00512079339090511\\
409.01	0.00512352269778132\\
410.01	0.00512629767517689\\
411.01	0.00512911897356907\\
412.01	0.00513198725465157\\
413.01	0.00513490319210574\\
414.01	0.00513786747243726\\
415.01	0.00514088079588267\\
416.01	0.00514394387738673\\
417.01	0.00514705744765431\\
418.01	0.00515022225427615\\
419.01	0.00515343906293231\\
420.01	0.00515670865867184\\
421.01	0.0051600318472699\\
422.01	0.00516340945666009\\
423.01	0.00516684233844038\\
424.01	0.00517033136945048\\
425.01	0.00517387745341429\\
426.01	0.00517748152264334\\
427.01	0.00518114453979394\\
428.01	0.00518486749966721\\
429.01	0.00518865143104338\\
430.01	0.00519249739853767\\
431.01	0.00519640650446129\\
432.01	0.00520037989067396\\
433.01	0.00520441874040711\\
434.01	0.00520852428003953\\
435.01	0.00521269778080247\\
436.01	0.00521694056039095\\
437.01	0.0052212539844582\\
438.01	0.00522563946796589\\
439.01	0.00523009847636763\\
440.01	0.00523463252660038\\
441.01	0.00523924318786104\\
442.01	0.00524393208215018\\
443.01	0.00524870088456564\\
444.01	0.00525355132333839\\
445.01	0.00525848517960635\\
446.01	0.00526350428693462\\
447.01	0.00526861053059645\\
448.01	0.00527380584664806\\
449.01	0.00527909222083719\\
450.01	0.00528447168740513\\
451.01	0.00528994632785534\\
452.01	0.00529551826977522\\
453.01	0.00530118968581521\\
454.01	0.0053069627929332\\
455.01	0.0053128398520253\\
456.01	0.00531882316805671\\
457.01	0.00532491509080059\\
458.01	0.00533111801626969\\
459.01	0.00533743438889314\\
460.01	0.00534386670444288\\
461.01	0.00535041751365456\\
462.01	0.00535708942641741\\
463.01	0.00536388511633094\\
464.01	0.0053708073253608\\
465.01	0.00537785886827462\\
466.01	0.0053850426365309\\
467.01	0.00539236160134585\\
468.01	0.00539981881579216\\
469.01	0.00540741741598878\\
470.01	0.00541516062166209\\
471.01	0.00542305173640308\\
472.01	0.00543109414774612\\
473.01	0.00543929132709937\\
474.01	0.00544764682954967\\
475.01	0.00545616429357375\\
476.01	0.00546484744069134\\
477.01	0.00547370007510466\\
478.01	0.00548272608336978\\
479.01	0.00549192943415226\\
480.01	0.00550131417811883\\
481.01	0.00551088444801663\\
482.01	0.00552064445898752\\
483.01	0.00553059850915716\\
484.01	0.00554075098052845\\
485.01	0.00555110634019344\\
486.01	0.00556166914186174\\
487.01	0.00557244402768026\\
488.01	0.0055834357303013\\
489.01	0.00559464907513395\\
490.01	0.00560608898269654\\
491.01	0.00561776047097949\\
492.01	0.0056296686577266\\
493.01	0.00564181876255559\\
494.01	0.00565421610886396\\
495.01	0.00566686612550699\\
496.01	0.00567977434828099\\
497.01	0.00569294642129253\\
498.01	0.00570638809832281\\
499.01	0.00572010524429577\\
500.01	0.0057341038369189\\
501.01	0.00574838996851797\\
502.01	0.00576296984806557\\
503.01	0.00577784980339512\\
504.01	0.00579303628358452\\
505.01	0.00580853586148709\\
506.01	0.00582435523638102\\
507.01	0.00584050123670836\\
508.01	0.0058569808228702\\
509.01	0.00587380109004865\\
510.01	0.00589096927103139\\
511.01	0.00590849273902176\\
512.01	0.00592637901042681\\
513.01	0.00594463574762613\\
514.01	0.00596327076173313\\
515.01	0.00598229201536413\\
516.01	0.00600170762542973\\
517.01	0.0060215258659571\\
518.01	0.00604175517093986\\
519.01	0.00606240413720248\\
520.01	0.00608348152725796\\
521.01	0.00610499627213941\\
522.01	0.00612695747418071\\
523.01	0.00614937440973107\\
524.01	0.0061722565317805\\
525.01	0.00619561347248153\\
526.01	0.00621945504555035\\
527.01	0.00624379124853081\\
528.01	0.00626863226490573\\
529.01	0.00629398846603425\\
530.01	0.0063198704128945\\
531.01	0.00634628885760222\\
532.01	0.00637325474467523\\
533.01	0.00640077921200628\\
534.01	0.00642887359150497\\
535.01	0.00645754940936542\\
536.01	0.00648681838591415\\
537.01	0.00651669243498613\\
538.01	0.00654718366277532\\
539.01	0.00657830436609863\\
540.01	0.00661006703000588\\
541.01	0.00664248432466176\\
542.01	0.00667556910141767\\
543.01	0.00670933438798234\\
544.01	0.00674379338259237\\
545.01	0.00677895944707115\\
546.01	0.00681484609865593\\
547.01	0.00685146700046133\\
548.01	0.00688883595043263\\
549.01	0.00692696686862976\\
550.01	0.00696587378266646\\
551.01	0.00700557081111293\\
552.01	0.00704607214465176\\
553.01	0.00708739202475621\\
554.01	0.0071295447196411\\
555.01	0.00717254449721021\\
556.01	0.00721640559470368\\
557.01	0.00726114218471995\\
558.01	0.00730676833725931\\
559.01	0.00735329797740891\\
560.01	0.00740074483825644\\
561.01	0.0074491224085882\\
562.01	0.00749844387489634\\
563.01	0.00754872205718511\\
564.01	0.00759996933803445\\
565.01	0.00765219758434847\\
566.01	0.00770541806118489\\
567.01	0.00775964133703997\\
568.01	0.00781487717993988\\
569.01	0.00787113444368163\\
570.01	0.00792842094356467\\
571.01	0.00798674332097159\\
572.01	0.00804610689619114\\
573.01	0.00810651550894171\\
574.01	0.00816797134615157\\
575.01	0.00823047475669562\\
576.01	0.00829402405299119\\
577.01	0.00835861529963226\\
578.01	0.00842424208960891\\
579.01	0.00849089530914933\\
580.01	0.00855856289285696\\
581.01	0.00862722957163953\\
582.01	0.00869687661698168\\
583.01	0.00876748158646345\\
584.01	0.00883901807713575\\
585.01	0.00891145549552764\\
586.01	0.00898475885578457\\
587.01	0.00905888862085642\\
588.01	0.00913380060593759\\
589.01	0.00920944596871416\\
590.01	0.00928577131764595\\
591.01	0.00936271897782057\\
592.01	0.00944022746424321\\
593.01	0.00951823222524592\\
594.01	0.00959666673459481\\
595.01	0.00967546403056262\\
596.01	0.00975455882459834\\
597.01	0.00983380640043464\\
598.01	0.00990866201848071\\
599.01	0.00997087280416276\\
599.02	0.00997138072163725\\
599.03	0.009971885576258\\
599.04	0.00997238733820211\\
599.05	0.00997288597735272\\
599.06	0.00997338146329604\\
599.07	0.00997387376531845\\
599.08	0.00997436285240349\\
599.09	0.00997484869322892\\
599.1	0.00997533125616361\\
599.11	0.00997581050926455\\
599.12	0.00997628642027372\\
599.13	0.00997675895661497\\
599.14	0.0099772280853909\\
599.15	0.00997769377337961\\
599.16	0.00997815598703157\\
599.17	0.00997861469246631\\
599.18	0.00997906985546915\\
599.19	0.00997952144148792\\
599.2	0.00997996941562957\\
599.21	0.00998041374265684\\
599.22	0.0099808543869848\\
599.23	0.00998129131267744\\
599.24	0.00998172448344418\\
599.25	0.00998215386263636\\
599.26	0.00998257941258354\\
599.27	0.00998300109279825\\
599.28	0.00998341886238981\\
599.29	0.00998383268006024\\
599.3	0.00998424250410032\\
599.31	0.00998464829238543\\
599.32	0.00998505000237151\\
599.33	0.00998544759109087\\
599.34	0.00998584101514799\\
599.35	0.00998623023071532\\
599.36	0.00998661519352896\\
599.37	0.00998699585888436\\
599.38	0.00998737218163197\\
599.39	0.00998774411617281\\
599.4	0.00998811161645403\\
599.41	0.00998847463596443\\
599.42	0.0099888331277299\\
599.43	0.00998918704430885\\
599.44	0.00998953633778757\\
599.45	0.00998988095977558\\
599.46	0.00999022086140088\\
599.47	0.00999055599330522\\
599.48	0.00999088630563925\\
599.49	0.00999121174805768\\
599.5	0.00999153226971435\\
599.51	0.0099918478192573\\
599.52	0.00999215834482374\\
599.53	0.00999246379403503\\
599.54	0.0099927641139915\\
599.55	0.00999305925126738\\
599.56	0.00999334915190553\\
599.57	0.0099936337614122\\
599.58	0.00999391302475173\\
599.59	0.00999418688634118\\
599.6	0.00999445529004489\\
599.61	0.00999471817916904\\
599.62	0.00999497549645613\\
599.63	0.00999522718407934\\
599.64	0.009995473183637\\
599.65	0.00999571343614678\\
599.66	0.00999594788204004\\
599.67	0.00999617646115598\\
599.68	0.0099963991127358\\
599.69	0.00999661577541675\\
599.7	0.00999682638722618\\
599.71	0.00999703088557551\\
599.72	0.00999722920725411\\
599.73	0.00999742128842315\\
599.74	0.00999760706460942\\
599.75	0.00999778647069897\\
599.76	0.00999795944093086\\
599.77	0.0099981259088907\\
599.78	0.0099982858075042\\
599.79	0.00999843906903064\\
599.8	0.00999858562505627\\
599.81	0.00999872540648767\\
599.82	0.00999885834354498\\
599.83	0.00999898436575515\\
599.84	0.00999910340194508\\
599.85	0.00999921538023465\\
599.86	0.00999932022802977\\
599.87	0.00999941787201528\\
599.88	0.00999950823814785\\
599.89	0.00999959125164875\\
599.9	0.00999966683699656\\
599.91	0.00999973491791987\\
599.92	0.0099997954173898\\
599.93	0.00999984825761255\\
599.94	0.00999989336002181\\
599.95	0.00999993064527112\\
599.96	0.00999996003322615\\
599.97	0.00999998144295691\\
599.98	0.00999999479272987\\
599.99	0.01\\
600	0.01\\
};
\addplot [color=mycolor5,solid,forget plot]
  table[row sep=crcr]{%
0.01	0.00490905749971206\\
1.01	0.00490905860197839\\
2.01	0.00490905972694872\\
3.01	0.00490906087509092\\
4.01	0.00490906204688202\\
5.01	0.00490906324280954\\
6.01	0.00490906446337047\\
7.01	0.00490906570907243\\
8.01	0.00490906698043314\\
9.01	0.00490906827798124\\
10.01	0.00490906960225622\\
11.01	0.00490907095380874\\
12.01	0.00490907233320062\\
13.01	0.00490907374100548\\
14.01	0.00490907517780848\\
15.01	0.00490907664420701\\
16.01	0.00490907814081074\\
17.01	0.00490907966824188\\
18.01	0.00490908122713578\\
19.01	0.0049090828181402\\
20.01	0.00490908444191679\\
21.01	0.00490908609914072\\
22.01	0.00490908779050088\\
23.01	0.00490908951670066\\
24.01	0.00490909127845751\\
25.01	0.00490909307650426\\
26.01	0.00490909491158817\\
27.01	0.00490909678447236\\
28.01	0.00490909869593543\\
29.01	0.00490910064677222\\
30.01	0.00490910263779389\\
31.01	0.00490910466982809\\
32.01	0.00490910674371986\\
33.01	0.00490910886033126\\
34.01	0.00490911102054242\\
35.01	0.00490911322525142\\
36.01	0.00490911547537501\\
37.01	0.0049091177718488\\
38.01	0.0049091201156274\\
39.01	0.00490912250768544\\
40.01	0.00490912494901743\\
41.01	0.00490912744063825\\
42.01	0.00490912998358396\\
43.01	0.0049091325789118\\
44.01	0.00490913522770087\\
45.01	0.00490913793105228\\
46.01	0.00490914069009012\\
47.01	0.00490914350596146\\
48.01	0.00490914637983708\\
49.01	0.00490914931291167\\
50.01	0.00490915230640494\\
51.01	0.0049091553615612\\
52.01	0.00490915847965061\\
53.01	0.00490916166196961\\
54.01	0.00490916490984129\\
55.01	0.00490916822461597\\
56.01	0.00490917160767151\\
57.01	0.00490917506041461\\
58.01	0.00490917858428062\\
59.01	0.00490918218073452\\
60.01	0.00490918585127144\\
61.01	0.00490918959741736\\
62.01	0.00490919342072964\\
63.01	0.00490919732279765\\
64.01	0.00490920130524378\\
65.01	0.00490920536972335\\
66.01	0.00490920951792605\\
67.01	0.00490921375157641\\
68.01	0.00490921807243412\\
69.01	0.00490922248229551\\
70.01	0.00490922698299386\\
71.01	0.00490923157639973\\
72.01	0.00490923626442272\\
73.01	0.00490924104901155\\
74.01	0.00490924593215501\\
75.01	0.00490925091588273\\
76.01	0.00490925600226642\\
77.01	0.00490926119342014\\
78.01	0.00490926649150164\\
79.01	0.0049092718987131\\
80.01	0.00490927741730177\\
81.01	0.00490928304956145\\
82.01	0.00490928879783287\\
83.01	0.00490929466450511\\
84.01	0.00490930065201656\\
85.01	0.00490930676285549\\
86.01	0.00490931299956143\\
87.01	0.00490931936472637\\
88.01	0.00490932586099546\\
89.01	0.0049093324910683\\
90.01	0.00490933925770003\\
91.01	0.00490934616370248\\
92.01	0.00490935321194527\\
93.01	0.00490936040535726\\
94.01	0.00490936774692727\\
95.01	0.00490937523970579\\
96.01	0.00490938288680609\\
97.01	0.00490939069140538\\
98.01	0.00490939865674618\\
99.01	0.00490940678613766\\
100.01	0.00490941508295725\\
101.01	0.00490942355065171\\
102.01	0.00490943219273871\\
103.01	0.00490944101280809\\
104.01	0.00490945001452382\\
105.01	0.00490945920162493\\
106.01	0.00490946857792736\\
107.01	0.00490947814732567\\
108.01	0.00490948791379411\\
109.01	0.00490949788138902\\
110.01	0.0049095080542499\\
111.01	0.0049095184366014\\
112.01	0.00490952903275502\\
113.01	0.00490953984711079\\
114.01	0.00490955088415899\\
115.01	0.00490956214848251\\
116.01	0.00490957364475825\\
117.01	0.00490958537775917\\
118.01	0.00490959735235603\\
119.01	0.00490960957352014\\
120.01	0.00490962204632456\\
121.01	0.00490963477594673\\
122.01	0.00490964776767029\\
123.01	0.00490966102688729\\
124.01	0.00490967455910074\\
125.01	0.00490968836992635\\
126.01	0.00490970246509535\\
127.01	0.00490971685045637\\
128.01	0.00490973153197836\\
129.01	0.00490974651575266\\
130.01	0.00490976180799554\\
131.01	0.00490977741505096\\
132.01	0.0049097933433928\\
133.01	0.00490980959962782\\
134.01	0.00490982619049835\\
135.01	0.00490984312288513\\
136.01	0.00490986040380985\\
137.01	0.00490987804043802\\
138.01	0.00490989604008256\\
139.01	0.00490991441020583\\
140.01	0.00490993315842334\\
141.01	0.00490995229250674\\
142.01	0.00490997182038666\\
143.01	0.00490999175015648\\
144.01	0.00491001209007534\\
145.01	0.00491003284857165\\
146.01	0.00491005403424595\\
147.01	0.00491007565587574\\
148.01	0.00491009772241746\\
149.01	0.0049101202430113\\
150.01	0.00491014322698433\\
151.01	0.00491016668385458\\
152.01	0.00491019062333502\\
153.01	0.00491021505533708\\
154.01	0.00491023998997492\\
155.01	0.00491026543756963\\
156.01	0.00491029140865341\\
157.01	0.00491031791397366\\
158.01	0.00491034496449737\\
159.01	0.00491037257141587\\
160.01	0.00491040074614905\\
161.01	0.00491042950034998\\
162.01	0.00491045884590979\\
163.01	0.00491048879496239\\
164.01	0.00491051935988952\\
165.01	0.0049105505533255\\
166.01	0.00491058238816239\\
167.01	0.00491061487755542\\
168.01	0.00491064803492815\\
169.01	0.00491068187397732\\
170.01	0.0049107164086796\\
171.01	0.00491075165329606\\
172.01	0.00491078762237854\\
173.01	0.00491082433077526\\
174.01	0.00491086179363659\\
175.01	0.00491090002642164\\
176.01	0.004910939044904\\
177.01	0.00491097886517806\\
178.01	0.00491101950366565\\
179.01	0.00491106097712254\\
180.01	0.00491110330264509\\
181.01	0.00491114649767722\\
182.01	0.00491119058001678\\
183.01	0.00491123556782333\\
184.01	0.00491128147962502\\
185.01	0.00491132833432602\\
186.01	0.00491137615121377\\
187.01	0.00491142494996694\\
188.01	0.00491147475066307\\
189.01	0.0049115255737865\\
190.01	0.00491157744023633\\
191.01	0.00491163037133473\\
192.01	0.0049116843888356\\
193.01	0.00491173951493254\\
194.01	0.00491179577226811\\
195.01	0.00491185318394217\\
196.01	0.00491191177352146\\
197.01	0.00491197156504844\\
198.01	0.00491203258305056\\
199.01	0.00491209485255008\\
200.01	0.00491215839907388\\
201.01	0.00491222324866275\\
202.01	0.00491228942788217\\
203.01	0.00491235696383208\\
204.01	0.00491242588415776\\
205.01	0.00491249621705986\\
206.01	0.00491256799130579\\
207.01	0.00491264123624047\\
208.01	0.0049127159817977\\
209.01	0.00491279225851149\\
210.01	0.00491287009752768\\
211.01	0.00491294953061565\\
212.01	0.00491303059018065\\
213.01	0.00491311330927587\\
214.01	0.00491319772161466\\
215.01	0.00491328386158379\\
216.01	0.00491337176425578\\
217.01	0.00491346146540234\\
218.01	0.00491355300150757\\
219.01	0.0049136464097818\\
220.01	0.00491374172817487\\
221.01	0.00491383899539088\\
222.01	0.00491393825090128\\
223.01	0.00491403953496078\\
224.01	0.00491414288862115\\
225.01	0.00491424835374625\\
226.01	0.00491435597302772\\
227.01	0.00491446578999996\\
228.01	0.00491457784905612\\
229.01	0.00491469219546373\\
230.01	0.00491480887538116\\
231.01	0.00491492793587396\\
232.01	0.00491504942493175\\
233.01	0.00491517339148445\\
234.01	0.0049152998854203\\
235.01	0.00491542895760302\\
236.01	0.00491556065988899\\
237.01	0.00491569504514588\\
238.01	0.00491583216727048\\
239.01	0.00491597208120721\\
240.01	0.00491611484296665\\
241.01	0.0049162605096445\\
242.01	0.00491640913944097\\
243.01	0.00491656079167989\\
244.01	0.00491671552682842\\
245.01	0.00491687340651699\\
246.01	0.00491703449355928\\
247.01	0.00491719885197298\\
248.01	0.00491736654699965\\
249.01	0.00491753764512574\\
250.01	0.00491771221410402\\
251.01	0.00491789032297398\\
252.01	0.00491807204208403\\
253.01	0.00491825744311238\\
254.01	0.00491844659908905\\
255.01	0.00491863958441796\\
256.01	0.00491883647489887\\
257.01	0.00491903734774932\\
258.01	0.00491924228162745\\
259.01	0.00491945135665417\\
260.01	0.00491966465443579\\
261.01	0.0049198822580869\\
262.01	0.00492010425225277\\
263.01	0.0049203307231323\\
264.01	0.004920561758501\\
265.01	0.00492079744773356\\
266.01	0.0049210378818268\\
267.01	0.00492128315342236\\
268.01	0.00492153335682981\\
269.01	0.00492178858804879\\
270.01	0.00492204894479158\\
271.01	0.00492231452650571\\
272.01	0.00492258543439587\\
273.01	0.00492286177144605\\
274.01	0.00492314364244112\\
275.01	0.00492343115398804\\
276.01	0.00492372441453698\\
277.01	0.00492402353440209\\
278.01	0.0049243286257809\\
279.01	0.00492463980277491\\
280.01	0.00492495718140823\\
281.01	0.00492528087964596\\
282.01	0.00492561101741196\\
283.01	0.00492594771660614\\
284.01	0.00492629110112015\\
285.01	0.0049266412968533\\
286.01	0.00492699843172646\\
287.01	0.00492736263569535\\
288.01	0.00492773404076295\\
289.01	0.00492811278099051\\
290.01	0.00492849899250662\\
291.01	0.00492889281351602\\
292.01	0.00492929438430578\\
293.01	0.00492970384725112\\
294.01	0.00493012134681801\\
295.01	0.0049305470295654\\
296.01	0.00493098104414412\\
297.01	0.00493142354129508\\
298.01	0.00493187467384421\\
299.01	0.00493233459669588\\
300.01	0.0049328034668234\\
301.01	0.004933281443257\\
302.01	0.00493376868706956\\
303.01	0.00493426536135884\\
304.01	0.00493477163122784\\
305.01	0.00493528766376086\\
306.01	0.00493581362799739\\
307.01	0.00493634969490228\\
308.01	0.00493689603733261\\
309.01	0.00493745283000145\\
310.01	0.00493802024943731\\
311.01	0.00493859847394061\\
312.01	0.00493918768353642\\
313.01	0.0049397880599234\\
314.01	0.00494039978641915\\
315.01	0.00494102304790169\\
316.01	0.0049416580307475\\
317.01	0.00494230492276611\\
318.01	0.00494296391313159\\
319.01	0.00494363519231056\\
320.01	0.00494431895198727\\
321.01	0.00494501538498655\\
322.01	0.00494572468519489\\
323.01	0.00494644704747903\\
324.01	0.00494718266760483\\
325.01	0.00494793174215515\\
326.01	0.00494869446844877\\
327.01	0.00494947104446108\\
328.01	0.00495026166874761\\
329.01	0.00495106654037195\\
330.01	0.00495188585883968\\
331.01	0.0049527198240398\\
332.01	0.00495356863619648\\
333.01	0.00495443249583227\\
334.01	0.0049553116037469\\
335.01	0.00495620616101282\\
336.01	0.00495711636899286\\
337.01	0.00495804242938172\\
338.01	0.00495898454427672\\
339.01	0.00495994291628156\\
340.01	0.00496091774864825\\
341.01	0.00496190924546254\\
342.01	0.00496291761187816\\
343.01	0.00496394305440709\\
344.01	0.00496498578127114\\
345.01	0.00496604600282269\\
346.01	0.0049671239320417\\
347.01	0.00496821978511656\\
348.01	0.00496933378211588\\
349.01	0.00497046614775908\\
350.01	0.00497161711229437\\
351.01	0.00497278691248819\\
352.01	0.00497397579273538\\
353.01	0.004975184006293\\
354.01	0.0049764118166411\\
355.01	0.00497765949897297\\
356.01	0.00497892734181093\\
357.01	0.00498021564874448\\
358.01	0.00498152474027903\\
359.01	0.00498285495578051\\
360.01	0.00498420665549278\\
361.01	0.00498558022259997\\
362.01	0.00498697606529111\\
363.01	0.00498839461877956\\
364.01	0.00498983634721632\\
365.01	0.00499130174541977\\
366.01	0.00499279134033873\\
367.01	0.00499430569214429\\
368.01	0.00499584539483784\\
369.01	0.00499741107625103\\
370.01	0.00499900339730523\\
371.01	0.00500062305039763\\
372.01	0.00500227075678872\\
373.01	0.0050039472628861\\
374.01	0.00500565333535642\\
375.01	0.00500738975505523\\
376.01	0.0050091573098522\\
377.01	0.00501095678654535\\
378.01	0.00501278896220887\\
379.01	0.00501465459550808\\
380.01	0.00501655441872212\\
381.01	0.00501848913142981\\
382.01	0.00502045939698054\\
383.01	0.00502246584291349\\
384.01	0.00502450906625633\\
385.01	0.00502658964389123\\
386.01	0.00502870814599819\\
387.01	0.00503086514465358\\
388.01	0.00503306121496463\\
389.01	0.00503529693482319\\
390.01	0.00503757288463709\\
391.01	0.00503988964705035\\
392.01	0.00504224780665428\\
393.01	0.0050446479496886\\
394.01	0.0050470906637351\\
395.01	0.00504957653740315\\
396.01	0.00505210616001035\\
397.01	0.00505468012125833\\
398.01	0.00505729901090658\\
399.01	0.00505996341844481\\
400.01	0.00506267393276891\\
401.01	0.00506543114186003\\
402.01	0.00506823563247182\\
403.01	0.00507108798982919\\
404.01	0.00507398879734204\\
405.01	0.00507693863633824\\
406.01	0.00507993808582029\\
407.01	0.00508298772225244\\
408.01	0.00508608811938221\\
409.01	0.00508923984810478\\
410.01	0.00509244347637487\\
411.01	0.00509569956917674\\
412.01	0.00509900868855837\\
413.01	0.00510237139373884\\
414.01	0.00510578824130067\\
415.01	0.00510925978547561\\
416.01	0.0051127865785352\\
417.01	0.00511636917129822\\
418.01	0.00512000811376698\\
419.01	0.00512370395590563\\
420.01	0.00512745724857272\\
421.01	0.00513126854462224\\
422.01	0.00513513840018684\\
423.01	0.00513906737615664\\
424.01	0.00514305603986628\\
425.01	0.00514710496700421\\
426.01	0.00515121474375491\\
427.01	0.00515538596918477\\
428.01	0.00515961925788036\\
429.01	0.00516391524284567\\
430.01	0.00516827457866102\\
431.01	0.00517269794490338\\
432.01	0.00517718604982302\\
433.01	0.00518173963426656\\
434.01	0.00518635947582922\\
435.01	0.00519104639321241\\
436.01	0.00519580125075539\\
437.01	0.00520062496309655\\
438.01	0.00520551849991628\\
439.01	0.0052104828906948\\
440.01	0.00521551922941084\\
441.01	0.00522062867909467\\
442.01	0.00522581247613339\\
443.01	0.00523107193421826\\
444.01	0.00523640844780934\\
445.01	0.00524182349498784\\
446.01	0.00524731863955634\\
447.01	0.0052528955322519\\
448.01	0.00525855591093484\\
449.01	0.00526430159963491\\
450.01	0.00527013450635724\\
451.01	0.00527605661958334\\
452.01	0.00528207000345424\\
453.01	0.00528817679168332\\
454.01	0.00529437918033213\\
455.01	0.00530067941967452\\
456.01	0.00530707980549043\\
457.01	0.00531358267025249\\
458.01	0.00532019037479467\\
459.01	0.00532690530116499\\
460.01	0.00533372984745722\\
461.01	0.00534066642544533\\
462.01	0.00534771746179938\\
463.01	0.00535488540348701\\
464.01	0.00536217272763402\\
465.01	0.00536958195559539\\
466.01	0.00537711567026543\\
467.01	0.00538477653479423\\
468.01	0.00539256731002991\\
469.01	0.00540049086756045\\
470.01	0.00540855019647486\\
471.01	0.00541674840612985\\
472.01	0.00542508872779501\\
473.01	0.00543357451562931\\
474.01	0.00544220924694998\\
475.01	0.00545099652176344\\
476.01	0.00545994006154736\\
477.01	0.00546904370729777\\
478.01	0.00547831141688372\\
479.01	0.00548774726178407\\
480.01	0.00549735542331861\\
481.01	0.005507140188521\\
482.01	0.00551710594584133\\
483.01	0.00552725718089843\\
484.01	0.00553759847253217\\
485.01	0.00554813448942038\\
486.01	0.0055588699875275\\
487.01	0.00556980980863281\\
488.01	0.00558095888013528\\
489.01	0.00559232221625795\\
490.01	0.00560390492066476\\
491.01	0.00561571219036122\\
492.01	0.00562774932059444\\
493.01	0.00564002171030427\\
494.01	0.0056525348675442\\
495.01	0.00566529441422003\\
496.01	0.00567830608954503\\
497.01	0.00569157575182161\\
498.01	0.00570510937855931\\
499.01	0.00571891306547014\\
500.01	0.0057329930251922\\
501.01	0.0057473555862806\\
502.01	0.00576200719262769\\
503.01	0.00577695440339838\\
504.01	0.00579220389354149\\
505.01	0.0058077624549108\\
506.01	0.00582363699799712\\
507.01	0.00583983455422826\\
508.01	0.00585636227876019\\
509.01	0.00587322745364077\\
510.01	0.00589043749120116\\
511.01	0.00590799993751283\\
512.01	0.00592592247575652\\
513.01	0.00594421292937826\\
514.01	0.00596287926496326\\
515.01	0.00598192959483183\\
516.01	0.00600137217944308\\
517.01	0.00602121542974826\\
518.01	0.00604146790964455\\
519.01	0.00606213833861916\\
520.01	0.00608323559458114\\
521.01	0.00610476871683385\\
522.01	0.0061267469091303\\
523.01	0.00614917954274147\\
524.01	0.00617207615947925\\
525.01	0.00619544647461083\\
526.01	0.00621930037961972\\
527.01	0.00624364794477872\\
528.01	0.0062684994215141\\
529.01	0.00629386524455719\\
530.01	0.00631975603388037\\
531.01	0.00634618259641703\\
532.01	0.00637315592754992\\
533.01	0.00640068721233276\\
534.01	0.00642878782640081\\
535.01	0.00645746933650783\\
536.01	0.00648674350063502\\
537.01	0.00651662226761109\\
538.01	0.00654711777618407\\
539.01	0.00657824235348439\\
540.01	0.00661000851281471\\
541.01	0.00664242895069685\\
542.01	0.00667551654309728\\
543.01	0.00670928434074414\\
544.01	0.00674374556343396\\
545.01	0.00677891359321937\\
546.01	0.00681480196635297\\
547.01	0.00685142436385382\\
548.01	0.00688879460055168\\
549.01	0.00692692661244959\\
550.01	0.00696583444223101\\
551.01	0.00700553222272127\\
552.01	0.00704603415809471\\
553.01	0.00708735450259801\\
554.01	0.00712950753653874\\
555.01	0.00717250753926676\\
556.01	0.00721636875885068\\
557.01	0.00726110537812294\\
558.01	0.00730673147674622\\
559.01	0.00735326098891716\\
560.01	0.00740070765629824\\
561.01	0.00744908497573459\\
562.01	0.00749840614127946\\
563.01	0.00754868398002041\\
564.01	0.00759993088116463\\
565.01	0.00765215871781027\\
566.01	0.00770537876080272\\
567.01	0.00775960158404742\\
568.01	0.00781483696063322\\
569.01	0.00787109374910713\\
570.01	0.00792837976924348\\
571.01	0.00798670166666285\\
572.01	0.00804606476569629\\
573.01	0.00810647290995056\\
574.01	0.00816792829013013\\
575.01	0.00823043125881488\\
576.01	0.00829398013209581\\
577.01	0.00835857097824435\\
578.01	0.00842419739396354\\
579.01	0.00849085026925531\\
580.01	0.00855851754257446\\
581.01	0.0086271839487638\\
582.01	0.00869683076332219\\
583.01	0.00876743554790117\\
584.01	0.00883897190364118\\
585.01	0.00891140924111708\\
586.01	0.00898471257838705\\
587.01	0.00905884238205851\\
588.01	0.0091337544705663\\
589.01	0.00920940000420826\\
590.01	0.00928572559315688\\
591.01	0.00936267356296942\\
592.01	0.00944018242744499\\
593.01	0.00951818763149205\\
594.01	0.00959662264256165\\
595.01	0.00967542048888486\\
596.01	0.00975451586711174\\
597.01	0.00983378252435979\\
598.01	0.00990866201848071\\
599.01	0.00997087280416276\\
599.02	0.00997138072163725\\
599.03	0.00997188557625799\\
599.04	0.00997238733820211\\
599.05	0.00997288597735272\\
599.06	0.00997338146329604\\
599.07	0.00997387376531845\\
599.08	0.00997436285240349\\
599.09	0.00997484869322892\\
599.1	0.00997533125616361\\
599.11	0.00997581050926455\\
599.12	0.00997628642027372\\
599.13	0.00997675895661497\\
599.14	0.0099772280853909\\
599.15	0.00997769377337961\\
599.16	0.00997815598703157\\
599.17	0.00997861469246631\\
599.18	0.00997906985546915\\
599.19	0.00997952144148792\\
599.2	0.00997996941562957\\
599.21	0.00998041374265684\\
599.22	0.0099808543869848\\
599.23	0.00998129131267744\\
599.24	0.00998172448344418\\
599.25	0.00998215386263636\\
599.26	0.00998257941258354\\
599.27	0.00998300109279825\\
599.28	0.00998341886238981\\
599.29	0.00998383268006024\\
599.3	0.00998424250410032\\
599.31	0.00998464829238543\\
599.32	0.00998505000237151\\
599.33	0.00998544759109087\\
599.34	0.00998584101514799\\
599.35	0.00998623023071532\\
599.36	0.00998661519352895\\
599.37	0.00998699585888436\\
599.38	0.00998737218163197\\
599.39	0.00998774411617281\\
599.4	0.00998811161645403\\
599.41	0.00998847463596443\\
599.42	0.0099888331277299\\
599.43	0.00998918704430885\\
599.44	0.00998953633778757\\
599.45	0.00998988095977558\\
599.46	0.00999022086140088\\
599.47	0.00999055599330522\\
599.48	0.00999088630563925\\
599.49	0.00999121174805768\\
599.5	0.00999153226971435\\
599.51	0.0099918478192573\\
599.52	0.00999215834482374\\
599.53	0.00999246379403503\\
599.54	0.0099927641139915\\
599.55	0.00999305925126738\\
599.56	0.00999334915190553\\
599.57	0.0099936337614122\\
599.58	0.00999391302475173\\
599.59	0.00999418688634118\\
599.6	0.00999445529004489\\
599.61	0.00999471817916904\\
599.62	0.00999497549645613\\
599.63	0.00999522718407935\\
599.64	0.009995473183637\\
599.65	0.00999571343614678\\
599.66	0.00999594788204004\\
599.67	0.00999617646115598\\
599.68	0.0099963991127358\\
599.69	0.00999661577541675\\
599.7	0.00999682638722618\\
599.71	0.00999703088557551\\
599.72	0.00999722920725411\\
599.73	0.00999742128842315\\
599.74	0.00999760706460942\\
599.75	0.00999778647069897\\
599.76	0.00999795944093086\\
599.77	0.0099981259088907\\
599.78	0.0099982858075042\\
599.79	0.00999843906903064\\
599.8	0.00999858562505627\\
599.81	0.00999872540648766\\
599.82	0.00999885834354498\\
599.83	0.00999898436575515\\
599.84	0.00999910340194508\\
599.85	0.00999921538023465\\
599.86	0.00999932022802977\\
599.87	0.00999941787201528\\
599.88	0.00999950823814785\\
599.89	0.00999959125164875\\
599.9	0.00999966683699656\\
599.91	0.00999973491791987\\
599.92	0.0099997954173898\\
599.93	0.00999984825761255\\
599.94	0.00999989336002181\\
599.95	0.00999993064527112\\
599.96	0.00999996003322615\\
599.97	0.00999998144295691\\
599.98	0.00999999479272987\\
599.99	0.01\\
600	0.01\\
};
\addplot [color=mycolor6,solid,forget plot]
  table[row sep=crcr]{%
0.01	0.00481014954134405\\
1.01	0.00481015070071254\\
2.01	0.00481015188410282\\
3.01	0.00481015309201312\\
4.01	0.0048101543249523\\
5.01	0.00481015558343964\\
6.01	0.0048101568680053\\
7.01	0.00481015817919046\\
8.01	0.00481015951754755\\
9.01	0.00481016088364039\\
10.01	0.00481016227804448\\
11.01	0.00481016370134746\\
12.01	0.00481016515414917\\
13.01	0.00481016663706167\\
14.01	0.0048101681507102\\
15.01	0.00481016969573249\\
16.01	0.0048101712727798\\
17.01	0.00481017288251699\\
18.01	0.00481017452562242\\
19.01	0.00481017620278913\\
20.01	0.00481017791472395\\
21.01	0.00481017966214872\\
22.01	0.00481018144580031\\
23.01	0.00481018326643066\\
24.01	0.00481018512480773\\
25.01	0.00481018702171504\\
26.01	0.00481018895795289\\
27.01	0.00481019093433795\\
28.01	0.00481019295170368\\
29.01	0.00481019501090106\\
30.01	0.00481019711279888\\
31.01	0.00481019925828392\\
32.01	0.00481020144826122\\
33.01	0.00481020368365494\\
34.01	0.0048102059654084\\
35.01	0.00481020829448453\\
36.01	0.00481021067186604\\
37.01	0.00481021309855627\\
38.01	0.00481021557557966\\
39.01	0.00481021810398161\\
40.01	0.0048102206848295\\
41.01	0.00481022331921278\\
42.01	0.00481022600824366\\
43.01	0.00481022875305748\\
44.01	0.00481023155481318\\
45.01	0.00481023441469378\\
46.01	0.00481023733390705\\
47.01	0.00481024031368568\\
48.01	0.00481024335528829\\
49.01	0.00481024645999943\\
50.01	0.00481024962913047\\
51.01	0.00481025286402029\\
52.01	0.00481025616603531\\
53.01	0.00481025953657056\\
54.01	0.00481026297705017\\
55.01	0.0048102664889277\\
56.01	0.00481027007368723\\
57.01	0.00481027373284361\\
58.01	0.0048102774679431\\
59.01	0.00481028128056438\\
60.01	0.00481028517231888\\
61.01	0.00481028914485161\\
62.01	0.0048102931998418\\
63.01	0.00481029733900368\\
64.01	0.00481030156408702\\
65.01	0.00481030587687843\\
66.01	0.00481031027920126\\
67.01	0.00481031477291716\\
68.01	0.00481031935992633\\
69.01	0.00481032404216857\\
70.01	0.00481032882162426\\
71.01	0.00481033370031486\\
72.01	0.00481033868030385\\
73.01	0.00481034376369797\\
74.01	0.00481034895264766\\
75.01	0.00481035424934825\\
76.01	0.00481035965604056\\
77.01	0.00481036517501233\\
78.01	0.00481037080859879\\
79.01	0.00481037655918383\\
80.01	0.00481038242920138\\
81.01	0.00481038842113546\\
82.01	0.00481039453752234\\
83.01	0.00481040078095088\\
84.01	0.00481040715406386\\
85.01	0.00481041365955931\\
86.01	0.00481042030019124\\
87.01	0.00481042707877116\\
88.01	0.00481043399816905\\
89.01	0.00481044106131481\\
90.01	0.00481044827119945\\
91.01	0.00481045563087631\\
92.01	0.0048104631434622\\
93.01	0.00481047081213884\\
94.01	0.00481047864015437\\
95.01	0.00481048663082471\\
96.01	0.00481049478753461\\
97.01	0.00481050311373948\\
98.01	0.00481051161296683\\
99.01	0.00481052028881776\\
100.01	0.00481052914496813\\
101.01	0.00481053818517066\\
102.01	0.00481054741325607\\
103.01	0.00481055683313504\\
104.01	0.00481056644879967\\
105.01	0.00481057626432532\\
106.01	0.00481058628387246\\
107.01	0.004810596511688\\
108.01	0.00481060695210768\\
109.01	0.00481061760955709\\
110.01	0.00481062848855444\\
111.01	0.00481063959371215\\
112.01	0.00481065092973859\\
113.01	0.00481066250144034\\
114.01	0.00481067431372413\\
115.01	0.0048106863715992\\
116.01	0.00481069868017873\\
117.01	0.00481071124468282\\
118.01	0.00481072407044042\\
119.01	0.00481073716289135\\
120.01	0.00481075052758896\\
121.01	0.00481076417020221\\
122.01	0.0048107780965182\\
123.01	0.00481079231244502\\
124.01	0.00481080682401336\\
125.01	0.0048108216373801\\
126.01	0.00481083675883011\\
127.01	0.0048108521947795\\
128.01	0.0048108679517779\\
129.01	0.00481088403651141\\
130.01	0.0048109004558058\\
131.01	0.00481091721662877\\
132.01	0.00481093432609353\\
133.01	0.00481095179146135\\
134.01	0.00481096962014495\\
135.01	0.00481098781971125\\
136.01	0.0048110063978853\\
137.01	0.0048110253625528\\
138.01	0.00481104472176394\\
139.01	0.00481106448373662\\
140.01	0.00481108465686003\\
141.01	0.00481110524969797\\
142.01	0.00481112627099319\\
143.01	0.00481114772967009\\
144.01	0.0048111696348393\\
145.01	0.00481119199580117\\
146.01	0.00481121482204984\\
147.01	0.00481123812327723\\
148.01	0.00481126190937731\\
149.01	0.00481128619045\\
150.01	0.00481131097680572\\
151.01	0.00481133627896967\\
152.01	0.00481136210768619\\
153.01	0.00481138847392362\\
154.01	0.0048114153888786\\
155.01	0.00481144286398083\\
156.01	0.00481147091089847\\
157.01	0.00481149954154214\\
158.01	0.00481152876807094\\
159.01	0.00481155860289674\\
160.01	0.00481158905869027\\
161.01	0.00481162014838588\\
162.01	0.00481165188518705\\
163.01	0.00481168428257234\\
164.01	0.00481171735430082\\
165.01	0.00481175111441771\\
166.01	0.00481178557726118\\
167.01	0.0048118207574671\\
168.01	0.00481185666997605\\
169.01	0.00481189333003998\\
170.01	0.00481193075322769\\
171.01	0.00481196895543208\\
172.01	0.00481200795287645\\
173.01	0.00481204776212178\\
174.01	0.00481208840007345\\
175.01	0.00481212988398846\\
176.01	0.00481217223148248\\
177.01	0.00481221546053767\\
178.01	0.00481225958950987\\
179.01	0.00481230463713658\\
180.01	0.00481235062254469\\
181.01	0.00481239756525864\\
182.01	0.00481244548520862\\
183.01	0.00481249440273894\\
184.01	0.0048125443386166\\
185.01	0.00481259531403974\\
186.01	0.00481264735064701\\
187.01	0.00481270047052633\\
188.01	0.00481275469622423\\
189.01	0.00481281005075523\\
190.01	0.0048128665576118\\
191.01	0.0048129242407738\\
192.01	0.00481298312471849\\
193.01	0.00481304323443138\\
194.01	0.00481310459541592\\
195.01	0.00481316723370447\\
196.01	0.00481323117586938\\
197.01	0.00481329644903362\\
198.01	0.00481336308088273\\
199.01	0.00481343109967566\\
200.01	0.00481350053425693\\
201.01	0.00481357141406868\\
202.01	0.00481364376916264\\
203.01	0.00481371763021285\\
204.01	0.0048137930285282\\
205.01	0.00481386999606585\\
206.01	0.0048139485654437\\
207.01	0.00481402876995489\\
208.01	0.00481411064358053\\
209.01	0.00481419422100451\\
210.01	0.00481427953762747\\
211.01	0.00481436662958172\\
212.01	0.00481445553374579\\
213.01	0.00481454628775976\\
214.01	0.00481463893004085\\
215.01	0.00481473349979893\\
216.01	0.00481483003705285\\
217.01	0.004814928582647\\
218.01	0.00481502917826748\\
219.01	0.00481513186645949\\
220.01	0.00481523669064492\\
221.01	0.00481534369513964\\
222.01	0.00481545292517209\\
223.01	0.00481556442690091\\
224.01	0.0048156782474344\\
225.01	0.00481579443484924\\
226.01	0.00481591303820997\\
227.01	0.0048160341075892\\
228.01	0.00481615769408726\\
229.01	0.00481628384985322\\
230.01	0.00481641262810552\\
231.01	0.0048165440831537\\
232.01	0.00481667827041972\\
233.01	0.00481681524646065\\
234.01	0.00481695506899074\\
235.01	0.00481709779690459\\
236.01	0.00481724349030065\\
237.01	0.00481739221050485\\
238.01	0.0048175440200949\\
239.01	0.00481769898292503\\
240.01	0.00481785716415088\\
241.01	0.00481801863025545\\
242.01	0.00481818344907474\\
243.01	0.00481835168982407\\
244.01	0.00481852342312557\\
245.01	0.00481869872103495\\
246.01	0.00481887765706959\\
247.01	0.00481906030623662\\
248.01	0.00481924674506191\\
249.01	0.0048194370516192\\
250.01	0.00481963130555982\\
251.01	0.00481982958814241\\
252.01	0.00482003198226424\\
253.01	0.00482023857249204\\
254.01	0.00482044944509324\\
255.01	0.00482066468806844\\
256.01	0.00482088439118374\\
257.01	0.00482110864600375\\
258.01	0.00482133754592508\\
259.01	0.00482157118621034\\
260.01	0.0048218096640221\\
261.01	0.0048220530784582\\
262.01	0.00482230153058678\\
263.01	0.00482255512348202\\
264.01	0.00482281396226006\\
265.01	0.00482307815411577\\
266.01	0.00482334780835965\\
267.01	0.00482362303645502\\
268.01	0.00482390395205565\\
269.01	0.00482419067104402\\
270.01	0.00482448331156941\\
271.01	0.00482478199408652\\
272.01	0.00482508684139468\\
273.01	0.00482539797867679\\
274.01	0.00482571553353858\\
275.01	0.0048260396360484\\
276.01	0.00482637041877684\\
277.01	0.00482670801683639\\
278.01	0.0048270525679215\\
279.01	0.0048274042123484\\
280.01	0.00482776309309454\\
281.01	0.00482812935583878\\
282.01	0.00482850314900027\\
283.01	0.00482888462377816\\
284.01	0.00482927393419044\\
285.01	0.00482967123711206\\
286.01	0.00483007669231331\\
287.01	0.00483049046249705\\
288.01	0.00483091271333523\\
289.01	0.00483134361350471\\
290.01	0.00483178333472237\\
291.01	0.00483223205177861\\
292.01	0.00483268994257\\
293.01	0.00483315718813013\\
294.01	0.00483363397265954\\
295.01	0.00483412048355308\\
296.01	0.00483461691142635\\
297.01	0.0048351234501388\\
298.01	0.00483564029681527\\
299.01	0.00483616765186451\\
300.01	0.00483670571899504\\
301.01	0.00483725470522769\\
302.01	0.00483781482090476\\
303.01	0.004838386279695\\
304.01	0.00483896929859511\\
305.01	0.00483956409792581\\
306.01	0.00484017090132424\\
307.01	0.00484078993572963\\
308.01	0.00484142143136446\\
309.01	0.00484206562170798\\
310.01	0.00484272274346456\\
311.01	0.00484339303652351\\
312.01	0.00484407674391166\\
313.01	0.00484477411173703\\
314.01	0.00484548538912395\\
315.01	0.00484621082813835\\
316.01	0.00484695068370171\\
317.01	0.00484770521349595\\
318.01	0.00484847467785354\\
319.01	0.00484925933963751\\
320.01	0.00485005946410659\\
321.01	0.00485087531876649\\
322.01	0.00485170717320614\\
323.01	0.00485255529891743\\
324.01	0.0048534199690986\\
325.01	0.0048543014584399\\
326.01	0.00485520004289085\\
327.01	0.00485611599940768\\
328.01	0.00485704960568156\\
329.01	0.00485800113984573\\
330.01	0.00485897088016164\\
331.01	0.00485995910468322\\
332.01	0.0048609660908997\\
333.01	0.00486199211535643\\
334.01	0.00486303745325403\\
335.01	0.00486410237802692\\
336.01	0.00486518716090109\\
337.01	0.00486629207043406\\
338.01	0.00486741737203751\\
339.01	0.00486856332748696\\
340.01	0.00486973019442055\\
341.01	0.00487091822583183\\
342.01	0.00487212766956283\\
343.01	0.00487335876780355\\
344.01	0.00487461175660618\\
345.01	0.00487588686542503\\
346.01	0.00487718431669312\\
347.01	0.00487850432545109\\
348.01	0.00487984709904424\\
349.01	0.00488121283690827\\
350.01	0.00488260173046647\\
351.01	0.00488401396316547\\
352.01	0.00488544971067974\\
353.01	0.00488690914131983\\
354.01	0.00488839241668426\\
355.01	0.00488989969259808\\
356.01	0.00489143112038896\\
357.01	0.00489298684855277\\
358.01	0.00489456702486797\\
359.01	0.00489617179902103\\
360.01	0.00489780132580704\\
361.01	0.00489945576897079\\
362.01	0.00490113530575392\\
363.01	0.00490284013220595\\
364.01	0.0049045704693079\\
365.01	0.00490632656994297\\
366.01	0.00490810872672136\\
367.01	0.00490991728063166\\
368.01	0.00491175263044519\\
369.01	0.00491361524273022\\
370.01	0.0049155056622537\\
371.01	0.00491742452243475\\
372.01	0.00491937255538659\\
373.01	0.00492135060091869\\
374.01	0.00492335961368574\\
375.01	0.00492540066746114\\
376.01	0.00492747495529421\\
377.01	0.00492958378410742\\
378.01	0.00493172856214356\\
379.01	0.00493391077765498\\
380.01	0.00493613196744747\\
381.01	0.00493839367453429\\
382.01	0.00494069739548804\\
383.01	0.00494304452051163\\
384.01	0.00494543627341122\\
385.01	0.0049478736656129\\
386.01	0.00495035751517274\\
387.01	0.00495288858523196\\
388.01	0.00495546764281225\\
389.01	0.00495809545977625\\
390.01	0.00496077281234379\\
391.01	0.00496350048057054\\
392.01	0.00496627924778534\\
393.01	0.00496910989998675\\
394.01	0.00497199322519443\\
395.01	0.00497493001275608\\
396.01	0.0049779210526074\\
397.01	0.00498096713448271\\
398.01	0.00498406904707539\\
399.01	0.00498722757714884\\
400.01	0.00499044350859192\\
401.01	0.00499371762142346\\
402.01	0.0049970506907422\\
403.01	0.00500044348562147\\
404.01	0.0050038967679506\\
405.01	0.00500741129122183\\
406.01	0.00501098779926511\\
407.01	0.0050146270249312\\
408.01	0.00501832968872651\\
409.01	0.00502209649740228\\
410.01	0.00502592814250234\\
411.01	0.00502982529887465\\
412.01	0.00503378862315343\\
413.01	0.00503781875221904\\
414.01	0.00504191630164617\\
415.01	0.00504608186414991\\
416.01	0.00505031600804466\\
417.01	0.00505461927572939\\
418.01	0.00505899218221903\\
419.01	0.0050634352137407\\
420.01	0.00506794882641948\\
421.01	0.00507253344507952\\
422.01	0.00507718946219067\\
423.01	0.00508191723699541\\
424.01	0.0050867170948518\\
425.01	0.00509158932683731\\
426.01	0.00509653418965687\\
427.01	0.00510155190590753\\
428.01	0.00510664266475552\\
429.01	0.00511180662308321\\
430.01	0.00511704390717192\\
431.01	0.0051223546149865\\
432.01	0.00512773881913381\\
433.01	0.00513319657056684\\
434.01	0.00513872790310789\\
435.01	0.00514433283886527\\
436.01	0.00515001139461045\\
437.01	0.00515576358918318\\
438.01	0.00516158945197483\\
439.01	0.00516748903253372\\
440.01	0.00517346241131204\\
441.01	0.00517950971155169\\
442.01	0.00518563111227167\\
443.01	0.0051918268622821\\
444.01	0.00519809729509522\\
445.01	0.00520444284454731\\
446.01	0.00521086406087209\\
447.01	0.00521736162688377\\
448.01	0.00522393637383599\\
449.01	0.00523058929641744\\
450.01	0.005237321566239\\
451.01	0.0052441345430564\\
452.01	0.00525102978286511\\
453.01	0.00525800904192067\\
454.01	0.00526507427567648\\
455.01	0.00527222763162986\\
456.01	0.00527947143513814\\
457.01	0.00528680816744723\\
458.01	0.00529424043550476\\
459.01	0.00530177093365197\\
460.01	0.00530940239803307\\
461.01	0.00531713755558978\\
462.01	0.00532497907080687\\
463.01	0.0053329294949338\\
464.01	0.0053409912240989\\
465.01	0.00534916647429395\\
466.01	0.00535745728213249\\
467.01	0.00536586553963554\\
468.01	0.00537439306754374\\
469.01	0.00538304172107079\\
470.01	0.00539181348436453\\
471.01	0.00540071050516695\\
472.01	0.00540973510763672\\
473.01	0.00541888980385066\\
474.01	0.00542817730434367\\
475.01	0.00543760052735374\\
476.01	0.00544716260641621\\
477.01	0.0054568668959398\\
478.01	0.00546671697440167\\
479.01	0.00547671664481987\\
480.01	0.00548686993220562\\
481.01	0.00549718107777489\\
482.01	0.00550765452980419\\
483.01	0.00551829493116719\\
484.01	0.00552910710377401\\
485.01	0.00554009603037293\\
486.01	0.00555126683443509\\
487.01	0.00556262475914229\\
488.01	0.00557417514679082\\
489.01	0.00558592342019337\\
490.01	0.00559787506784526\\
491.01	0.00561003563466669\\
492.01	0.00562241071994367\\
493.01	0.00563500598358412\\
494.01	0.00564782716088615\\
495.01	0.00566088008463324\\
496.01	0.0056741707115195\\
497.01	0.00568770514791241\\
498.01	0.00570148966840113\\
499.01	0.00571553072125539\\
500.01	0.0057298349224963\\
501.01	0.00574440904645443\\
502.01	0.00575926001579341\\
503.01	0.00577439489153707\\
504.01	0.00578982086358361\\
505.01	0.00580554524222467\\
506.01	0.00582157545117937\\
507.01	0.00583791902261407\\
508.01	0.00585458359451178\\
509.01	0.00587157691059846\\
510.01	0.00588890682281078\\
511.01	0.00590658129601832\\
512.01	0.00592460841441665\\
513.01	0.0059429963887318\\
514.01	0.00596175356318732\\
515.01	0.00598088842117465\\
516.01	0.00600040958882743\\
517.01	0.00602032583630079\\
518.01	0.00604064607744553\\
519.01	0.00606137936926249\\
520.01	0.0060825349121271\\
521.01	0.00610412205098902\\
522.01	0.00612615027750338\\
523.01	0.00614862923297689\\
524.01	0.00617156871194051\\
525.01	0.00619497866610692\\
526.01	0.00621886920843724\\
527.01	0.00624325061704201\\
528.01	0.00626813333868453\\
529.01	0.00629352799173694\\
530.01	0.00631944536855689\\
531.01	0.00634589643737332\\
532.01	0.00637289234385549\\
533.01	0.00640044441253146\\
534.01	0.00642856414810232\\
535.01	0.00645726323657654\\
536.01	0.00648655354609155\\
537.01	0.00651644712729382\\
538.01	0.0065469562131532\\
539.01	0.00657809321810346\\
540.01	0.00660987073641983\\
541.01	0.00664230153976614\\
542.01	0.00667539857385353\\
543.01	0.00670917495415527\\
544.01	0.0067436439606068\\
545.01	0.00677881903119029\\
546.01	0.00681471375427505\\
547.01	0.00685134185956257\\
548.01	0.00688871720747539\\
549.01	0.00692685377682434\\
550.01	0.00696576565057831\\
551.01	0.00700546699954922\\
552.01	0.00704597206379078\\
553.01	0.00708729513148984\\
554.01	0.00712945051510259\\
555.01	0.00717245252446584\\
556.01	0.00721631543658319\\
557.01	0.00726105346176332\\
558.01	0.00730668070575833\\
559.01	0.00735321112752477\\
560.01	0.0074006584921987\\
561.01	0.00744903631884569\\
562.01	0.00749835782251346\\
563.01	0.00754863585008057\\
564.01	0.00759988280936222\\
565.01	0.00765211059090243\\
566.01	0.00770533048185168\\
567.01	0.0077595530713051\\
568.01	0.00781478814645575\\
569.01	0.00787104457890665\\
570.01	0.00792833020048318\\
571.01	0.0079866516679058\\
572.01	0.00804601431571488\\
573.01	0.00810642199690455\\
574.01	0.0081678769108208\\
575.01	0.00823037941802071\\
576.01	0.00829392784199252\\
577.01	0.0083585182579118\\
578.01	0.00842414426897704\\
579.01	0.00849079677135603\\
580.01	0.00855846370941088\\
581.01	0.00862712982369113\\
582.01	0.0086967763952409\\
583.01	0.00876738099111419\\
584.01	0.00883891721770022\\
585.01	0.00891135449062501\\
586.01	0.00898465783271585\\
587.01	0.00905878771493402\\
588.01	0.00913369995946198\\
589.01	0.00920934572947927\\
590.01	0.00928567163683064\\
591.01	0.00936262000709249\\
592.01	0.00944012935186362\\
593.01	0.00951813511091667\\
594.01	0.00959657074273127\\
595.01	0.00967536926160508\\
596.01	0.00975446534388769\\
597.01	0.0098337539663526\\
598.01	0.00990866201848071\\
599.01	0.00997087280416276\\
599.02	0.00997138072163725\\
599.03	0.009971885576258\\
599.04	0.00997238733820211\\
599.05	0.00997288597735272\\
599.06	0.00997338146329604\\
599.07	0.00997387376531845\\
599.08	0.00997436285240349\\
599.09	0.00997484869322892\\
599.1	0.00997533125616361\\
599.11	0.00997581050926455\\
599.12	0.00997628642027372\\
599.13	0.00997675895661497\\
599.14	0.0099772280853909\\
599.15	0.00997769377337961\\
599.16	0.00997815598703157\\
599.17	0.00997861469246631\\
599.18	0.00997906985546915\\
599.19	0.00997952144148792\\
599.2	0.00997996941562957\\
599.21	0.00998041374265684\\
599.22	0.0099808543869848\\
599.23	0.00998129131267744\\
599.24	0.00998172448344418\\
599.25	0.00998215386263636\\
599.26	0.00998257941258353\\
599.27	0.00998300109279825\\
599.28	0.00998341886238981\\
599.29	0.00998383268006024\\
599.3	0.00998424250410032\\
599.31	0.00998464829238543\\
599.32	0.00998505000237151\\
599.33	0.00998544759109087\\
599.34	0.00998584101514799\\
599.35	0.00998623023071532\\
599.36	0.00998661519352895\\
599.37	0.00998699585888436\\
599.38	0.00998737218163197\\
599.39	0.00998774411617281\\
599.4	0.00998811161645403\\
599.41	0.00998847463596443\\
599.42	0.0099888331277299\\
599.43	0.00998918704430885\\
599.44	0.00998953633778757\\
599.45	0.00998988095977558\\
599.46	0.00999022086140088\\
599.47	0.00999055599330522\\
599.48	0.00999088630563925\\
599.49	0.00999121174805768\\
599.5	0.00999153226971435\\
599.51	0.0099918478192573\\
599.52	0.00999215834482374\\
599.53	0.00999246379403503\\
599.54	0.0099927641139915\\
599.55	0.00999305925126738\\
599.56	0.00999334915190553\\
599.57	0.0099936337614122\\
599.58	0.00999391302475173\\
599.59	0.00999418688634118\\
599.6	0.00999445529004489\\
599.61	0.00999471817916904\\
599.62	0.00999497549645613\\
599.63	0.00999522718407934\\
599.64	0.009995473183637\\
599.65	0.00999571343614678\\
599.66	0.00999594788204004\\
599.67	0.00999617646115599\\
599.68	0.0099963991127358\\
599.69	0.00999661577541675\\
599.7	0.00999682638722618\\
599.71	0.00999703088557551\\
599.72	0.00999722920725411\\
599.73	0.00999742128842315\\
599.74	0.00999760706460942\\
599.75	0.00999778647069897\\
599.76	0.00999795944093086\\
599.77	0.0099981259088907\\
599.78	0.0099982858075042\\
599.79	0.00999843906903064\\
599.8	0.00999858562505627\\
599.81	0.00999872540648767\\
599.82	0.00999885834354498\\
599.83	0.00999898436575515\\
599.84	0.00999910340194508\\
599.85	0.00999921538023465\\
599.86	0.00999932022802977\\
599.87	0.00999941787201528\\
599.88	0.00999950823814785\\
599.89	0.00999959125164875\\
599.9	0.00999966683699656\\
599.91	0.00999973491791987\\
599.92	0.0099997954173898\\
599.93	0.00999984825761255\\
599.94	0.00999989336002181\\
599.95	0.00999993064527112\\
599.96	0.00999996003322615\\
599.97	0.00999998144295691\\
599.98	0.00999999479272987\\
599.99	0.01\\
600	0.01\\
};
\addplot [color=mycolor7,solid,forget plot]
  table[row sep=crcr]{%
0.01	0.00464834445259417\\
1.01	0.00464834564822231\\
2.01	0.00464834686873988\\
3.01	0.0046483481146663\\
4.01	0.00464834938653171\\
5.01	0.00464835068487754\\
6.01	0.00464835201025615\\
7.01	0.00464835336323188\\
8.01	0.00464835474438058\\
9.01	0.00464835615429034\\
10.01	0.00464835759356162\\
11.01	0.00464835906280725\\
12.01	0.00464836056265287\\
13.01	0.00464836209373737\\
14.01	0.00464836365671296\\
15.01	0.00464836525224556\\
16.01	0.00464836688101485\\
17.01	0.00464836854371492\\
18.01	0.00464837024105429\\
19.01	0.00464837197375634\\
20.01	0.00464837374255971\\
21.01	0.00464837554821833\\
22.01	0.0046483773915021\\
23.01	0.00464837927319689\\
24.01	0.00464838119410517\\
25.01	0.00464838315504608\\
26.01	0.00464838515685599\\
27.01	0.00464838720038867\\
28.01	0.0046483892865162\\
29.01	0.00464839141612844\\
30.01	0.0046483935901341\\
31.01	0.00464839580946098\\
32.01	0.00464839807505615\\
33.01	0.00464840038788687\\
34.01	0.00464840274894025\\
35.01	0.00464840515922424\\
36.01	0.00464840761976822\\
37.01	0.00464841013162275\\
38.01	0.00464841269586058\\
39.01	0.00464841531357703\\
40.01	0.00464841798589027\\
41.01	0.00464842071394191\\
42.01	0.00464842349889752\\
43.01	0.00464842634194722\\
44.01	0.0046484292443059\\
45.01	0.00464843220721413\\
46.01	0.00464843523193825\\
47.01	0.00464843831977162\\
48.01	0.00464844147203429\\
49.01	0.00464844469007421\\
50.01	0.00464844797526748\\
51.01	0.0046484513290191\\
52.01	0.00464845475276383\\
53.01	0.00464845824796627\\
54.01	0.00464846181612165\\
55.01	0.00464846545875701\\
56.01	0.00464846917743099\\
57.01	0.00464847297373523\\
58.01	0.00464847684929467\\
59.01	0.00464848080576842\\
60.01	0.00464848484485038\\
61.01	0.00464848896826998\\
62.01	0.00464849317779312\\
63.01	0.00464849747522268\\
64.01	0.0046485018623994\\
65.01	0.00464850634120256\\
66.01	0.00464851091355128\\
67.01	0.00464851558140465\\
68.01	0.00464852034676305\\
69.01	0.00464852521166893\\
70.01	0.00464853017820752\\
71.01	0.00464853524850789\\
72.01	0.00464854042474395\\
73.01	0.00464854570913519\\
74.01	0.00464855110394777\\
75.01	0.00464855661149552\\
76.01	0.00464856223414092\\
77.01	0.00464856797429605\\
78.01	0.00464857383442366\\
79.01	0.00464857981703851\\
80.01	0.00464858592470777\\
81.01	0.00464859216005296\\
82.01	0.00464859852575081\\
83.01	0.00464860502453399\\
84.01	0.00464861165919278\\
85.01	0.00464861843257618\\
86.01	0.0046486253475931\\
87.01	0.0046486324072136\\
88.01	0.00464863961447035\\
89.01	0.00464864697245969\\
90.01	0.00464865448434311\\
91.01	0.00464866215334847\\
92.01	0.00464866998277179\\
93.01	0.00464867797597848\\
94.01	0.0046486861364045\\
95.01	0.00464869446755817\\
96.01	0.00464870297302199\\
97.01	0.00464871165645349\\
98.01	0.00464872052158731\\
99.01	0.00464872957223671\\
100.01	0.00464873881229512\\
101.01	0.00464874824573808\\
102.01	0.00464875787662486\\
103.01	0.00464876770910008\\
104.01	0.00464877774739562\\
105.01	0.00464878799583257\\
106.01	0.00464879845882279\\
107.01	0.00464880914087151\\
108.01	0.00464882004657829\\
109.01	0.00464883118064019\\
110.01	0.00464884254785258\\
111.01	0.00464885415311218\\
112.01	0.00464886600141887\\
113.01	0.00464887809787795\\
114.01	0.00464889044770213\\
115.01	0.00464890305621371\\
116.01	0.00464891592884785\\
117.01	0.00464892907115352\\
118.01	0.00464894248879703\\
119.01	0.00464895618756395\\
120.01	0.00464897017336184\\
121.01	0.00464898445222281\\
122.01	0.0046489990303062\\
123.01	0.00464901391390093\\
124.01	0.00464902910942883\\
125.01	0.00464904462344685\\
126.01	0.00464906046265044\\
127.01	0.00464907663387601\\
128.01	0.00464909314410422\\
129.01	0.00464911000046296\\
130.01	0.00464912721023041\\
131.01	0.00464914478083822\\
132.01	0.00464916271987477\\
133.01	0.00464918103508856\\
134.01	0.00464919973439162\\
135.01	0.00464921882586271\\
136.01	0.0046492383177511\\
137.01	0.00464925821848014\\
138.01	0.00464927853665087\\
139.01	0.00464929928104579\\
140.01	0.0046493204606327\\
141.01	0.00464934208456876\\
142.01	0.00464936416220404\\
143.01	0.00464938670308643\\
144.01	0.00464940971696495\\
145.01	0.00464943321379452\\
146.01	0.0046494572037403\\
147.01	0.00464948169718185\\
148.01	0.00464950670471789\\
149.01	0.00464953223717089\\
150.01	0.00464955830559181\\
151.01	0.00464958492126498\\
152.01	0.00464961209571292\\
153.01	0.00464963984070133\\
154.01	0.00464966816824472\\
155.01	0.004649697090611\\
156.01	0.00464972662032705\\
157.01	0.00464975677018478\\
158.01	0.00464978755324576\\
159.01	0.0046498189828479\\
160.01	0.00464985107261046\\
161.01	0.00464988383644051\\
162.01	0.00464991728853896\\
163.01	0.0046499514434067\\
164.01	0.00464998631585091\\
165.01	0.00465002192099168\\
166.01	0.00465005827426837\\
167.01	0.00465009539144685\\
168.01	0.00465013328862596\\
169.01	0.00465017198224481\\
170.01	0.00465021148908977\\
171.01	0.00465025182630202\\
172.01	0.0046502930113851\\
173.01	0.0046503350622123\\
174.01	0.0046503779970348\\
175.01	0.00465042183448953\\
176.01	0.00465046659360724\\
177.01	0.00465051229382124\\
178.01	0.00465055895497543\\
179.01	0.00465060659733316\\
180.01	0.00465065524158635\\
181.01	0.00465070490886434\\
182.01	0.0046507556207433\\
183.01	0.00465080739925518\\
184.01	0.00465086026689822\\
185.01	0.00465091424664639\\
186.01	0.00465096936195923\\
187.01	0.00465102563679255\\
188.01	0.00465108309560855\\
189.01	0.00465114176338727\\
190.01	0.00465120166563647\\
191.01	0.00465126282840389\\
192.01	0.00465132527828799\\
193.01	0.00465138904244983\\
194.01	0.00465145414862494\\
195.01	0.00465152062513614\\
196.01	0.00465158850090458\\
197.01	0.00465165780546387\\
198.01	0.00465172856897206\\
199.01	0.00465180082222541\\
200.01	0.0046518745966718\\
201.01	0.00465194992442446\\
202.01	0.00465202683827612\\
203.01	0.0046521053717135\\
204.01	0.00465218555893182\\
205.01	0.00465226743484987\\
206.01	0.00465235103512567\\
207.01	0.0046524363961717\\
208.01	0.0046525235551711\\
209.01	0.00465261255009412\\
210.01	0.00465270341971445\\
211.01	0.00465279620362667\\
212.01	0.00465289094226341\\
213.01	0.00465298767691322\\
214.01	0.00465308644973861\\
215.01	0.00465318730379469\\
216.01	0.00465329028304812\\
217.01	0.00465339543239622\\
218.01	0.00465350279768705\\
219.01	0.00465361242573939\\
220.01	0.00465372436436307\\
221.01	0.00465383866238059\\
222.01	0.00465395536964798\\
223.01	0.00465407453707718\\
224.01	0.00465419621665816\\
225.01	0.00465432046148176\\
226.01	0.00465444732576344\\
227.01	0.00465457686486646\\
228.01	0.00465470913532673\\
229.01	0.00465484419487774\\
230.01	0.00465498210247563\\
231.01	0.00465512291832523\\
232.01	0.00465526670390652\\
233.01	0.00465541352200186\\
234.01	0.00465556343672337\\
235.01	0.00465571651354121\\
236.01	0.00465587281931247\\
237.01	0.0046560324223103\\
238.01	0.00465619539225437\\
239.01	0.00465636180034099\\
240.01	0.00465653171927498\\
241.01	0.00465670522330118\\
242.01	0.00465688238823729\\
243.01	0.00465706329150724\\
244.01	0.00465724801217507\\
245.01	0.00465743663097969\\
246.01	0.00465762923037064\\
247.01	0.00465782589454374\\
248.01	0.00465802670947869\\
249.01	0.0046582317629764\\
250.01	0.00465844114469749\\
251.01	0.00465865494620205\\
252.01	0.0046588732609894\\
253.01	0.00465909618453894\\
254.01	0.00465932381435288\\
255.01	0.00465955624999767\\
256.01	0.00465979359314861\\
257.01	0.00466003594763367\\
258.01	0.00466028341947942\\
259.01	0.00466053611695691\\
260.01	0.00466079415062931\\
261.01	0.00466105763339973\\
262.01	0.00466132668056077\\
263.01	0.00466160140984465\\
264.01	0.00466188194147421\\
265.01	0.0046621683982155\\
266.01	0.00466246090543077\\
267.01	0.00466275959113274\\
268.01	0.00466306458604027\\
269.01	0.00466337602363437\\
270.01	0.00466369404021644\\
271.01	0.00466401877496626\\
272.01	0.00466435037000225\\
273.01	0.00466468897044188\\
274.01	0.00466503472446445\\
275.01	0.00466538778337351\\
276.01	0.00466574830166176\\
277.01	0.00466611643707615\\
278.01	0.00466649235068493\\
279.01	0.00466687620694507\\
280.01	0.00466726817377179\\
281.01	0.00466766842260811\\
282.01	0.00466807712849658\\
283.01	0.00466849447015159\\
284.01	0.00466892063003287\\
285.01	0.00466935579442034\\
286.01	0.00466980015348983\\
287.01	0.00467025390138999\\
288.01	0.00467071723632011\\
289.01	0.00467119036060909\\
290.01	0.00467167348079517\\
291.01	0.00467216680770687\\
292.01	0.00467267055654463\\
293.01	0.00467318494696309\\
294.01	0.00467371020315407\\
295.01	0.00467424655393085\\
296.01	0.0046747942328116\\
297.01	0.00467535347810466\\
298.01	0.00467592453299313\\
299.01	0.0046765076456199\\
300.01	0.00467710306917288\\
301.01	0.0046777110619695\\
302.01	0.00467833188754116\\
303.01	0.00467896581471694\\
304.01	0.00467961311770609\\
305.01	0.0046802740761803\\
306.01	0.00468094897535276\\
307.01	0.00468163810605712\\
308.01	0.0046823417648227\\
309.01	0.00468306025394839\\
310.01	0.00468379388157174\\
311.01	0.0046845429617358\\
312.01	0.00468530781445062\\
313.01	0.0046860887657502\\
314.01	0.00468688614774332\\
315.01	0.00468770029865807\\
316.01	0.004688531562879\\
317.01	0.00468938029097496\\
318.01	0.00469024683971839\\
319.01	0.0046911315720933\\
320.01	0.00469203485729067\\
321.01	0.00469295707069094\\
322.01	0.00469389859383011\\
323.01	0.00469485981435\\
324.01	0.00469584112592815\\
325.01	0.00469684292818687\\
326.01	0.00469786562657859\\
327.01	0.00469890963224451\\
328.01	0.00469997536184386\\
329.01	0.00470106323735046\\
330.01	0.00470217368581355\\
331.01	0.00470330713907838\\
332.01	0.00470446403346253\\
333.01	0.00470564480938383\\
334.01	0.0047068499109341\\
335.01	0.00470807978539418\\
336.01	0.00470933488268387\\
337.01	0.00471061565474067\\
338.01	0.00471192255482068\\
339.01	0.00471325603671385\\
340.01	0.00471461655386707\\
341.01	0.00471600455840571\\
342.01	0.00471742050004591\\
343.01	0.00471886482488885\\
344.01	0.0047203379740879\\
345.01	0.0047218403823789\\
346.01	0.0047233724764662\\
347.01	0.00472493467325462\\
348.01	0.00472652737792041\\
349.01	0.00472815098181462\\
350.01	0.00472980586019424\\
351.01	0.00473149236977929\\
352.01	0.00473321084613727\\
353.01	0.00473496160090216\\
354.01	0.00473674491883976\\
355.01	0.0047385610547811\\
356.01	0.00474041023045556\\
357.01	0.00474229263126748\\
358.01	0.00474420840307992\\
359.01	0.00474615764908549\\
360.01	0.00474814042687612\\
361.01	0.00475015674584822\\
362.01	0.00475220656512587\\
363.01	0.00475428979222349\\
364.01	0.00475640628273112\\
365.01	0.00475855584136699\\
366.01	0.00476073822481682\\
367.01	0.00476295314686802\\
368.01	0.00476520028643795\\
369.01	0.00476747929920025\\
370.01	0.00476978983361274\\
371.01	0.00477213155224933\\
372.01	0.00477450415940651\\
373.01	0.00477690743598608\\
374.01	0.00477934128259452\\
375.01	0.00478180577161051\\
376.01	0.00478430120855663\\
377.01	0.00478682820236268\\
378.01	0.00478938774284642\\
379.01	0.00479198128171347\\
380.01	0.00479461081023144\\
381.01	0.00479727892195163\\
382.01	0.00479998884170042\\
383.01	0.00480274439151386\\
384.01	0.00480554984876812\\
385.01	0.00480840960580881\\
386.01	0.00481132663638383\\
387.01	0.0048143021350969\\
388.01	0.00481733718293635\\
389.01	0.00482043287460097\\
390.01	0.00482359031817933\\
391.01	0.00482681063478051\\
392.01	0.00483009495811014\\
393.01	0.00483344443398835\\
394.01	0.00483686021980408\\
395.01	0.00484034348390203\\
396.01	0.00484389540489522\\
397.01	0.00484751717089908\\
398.01	0.00485120997867994\\
399.01	0.00485497503271119\\
400.01	0.00485881354413251\\
401.01	0.0048627267296011\\
402.01	0.00486671581003152\\
403.01	0.00487078200921302\\
404.01	0.00487492655229703\\
405.01	0.00487915066414647\\
406.01	0.00488345556753692\\
407.01	0.00488784248120079\\
408.01	0.00489231261770388\\
409.01	0.00489686718114493\\
410.01	0.00490150736466798\\
411.01	0.00490623434777614\\
412.01	0.00491104929343796\\
413.01	0.00491595334497498\\
414.01	0.0049209476227208\\
415.01	0.00492603322044267\\
416.01	0.00493121120151686\\
417.01	0.00493648259485009\\
418.01	0.00494184839054121\\
419.01	0.00494730953527837\\
420.01	0.00495286692747109\\
421.01	0.00495852141211699\\
422.01	0.00496427377540952\\
423.01	0.00497012473909408\\
424.01	0.00497607495459103\\
425.01	0.00498212499690281\\
426.01	0.00498827535833897\\
427.01	0.00499452644209535\\
428.01	0.00500087855573869\\
429.01	0.00500733190466011\\
430.01	0.00501388658557486\\
431.01	0.00502054258016595\\
432.01	0.00502729974898626\\
433.01	0.00503415782575883\\
434.01	0.00504111641224168\\
435.01	0.00504817497385036\\
436.01	0.00505533283626679\\
437.01	0.00506258918329626\\
438.01	0.00506994305627546\\
439.01	0.00507739335537467\\
440.01	0.00508493884317916\\
441.01	0.00509257815097973\\
442.01	0.00510030978824515\\
443.01	0.00510813215578212\\
444.01	0.00511604356312524\\
445.01	0.00512404225070982\\
446.01	0.00513212641738443\\
447.01	0.00514029425378412\\
448.01	0.00514854398202269\\
449.01	0.00515687390203937\\
450.01	0.00516528244474876\\
451.01	0.0051737682318643\\
452.01	0.00518233014188449\\
453.01	0.00519096738119998\\
454.01	0.00519967955859978\\
455.01	0.00520846676056924\\
456.01	0.00521732962368783\\
457.01	0.00522626939911519\\
458.01	0.00523528800263791\\
459.01	0.00524438804207301\\
460.01	0.00525357281213593\\
461.01	0.00526284624540482\\
462.01	0.00527221280719085\\
463.01	0.00528167732264719\\
464.01	0.00529124472744835\\
465.01	0.00530091974060239\\
466.01	0.0053107064721192\\
467.01	0.00532060800344458\\
468.01	0.00533062602087934\\
469.01	0.00534076072198664\\
470.01	0.00535101162228553\\
471.01	0.00536137827533253\\
472.01	0.0053718603558916\\
473.01	0.00538245769485676\\
474.01	0.00539317031688376\\
475.01	0.0054039984803502\\
476.01	0.00541494271909335\\
477.01	0.00542600388516891\\
478.01	0.00543718319164449\\
479.01	0.00544848225418493\\
480.01	0.00545990312991866\\
481.01	0.00547144835179354\\
482.01	0.00548312095635677\\
483.01	0.00549492450265347\\
484.01	0.00550686307978838\\
485.01	0.00551894130066258\\
486.01	0.00553116427957259\\
487.01	0.00554353759180509\\
488.01	0.00555606721418994\\
489.01	0.00556875944688499\\
490.01	0.00558162081857003\\
491.01	0.0055946579797886\\
492.01	0.00560787759240927\\
493.01	0.00562128622692798\\
494.01	0.00563489028319811\\
495.01	0.00564869595329159\\
496.01	0.00566270924591865\\
497.01	0.00567693608728224\\
498.01	0.00569138249822781\\
499.01	0.00570605479287023\\
500.01	0.00572095966238625\\
501.01	0.00573610416924741\\
502.01	0.00575149572772143\\
503.01	0.00576714207787485\\
504.01	0.00578305125354241\\
505.01	0.00579923154518509\\
506.01	0.00581569145909275\\
507.01	0.00583243967495409\\
508.01	0.00584948500437797\\
509.01	0.00586683635341503\\
510.01	0.00588450269239885\\
511.01	0.00590249303634946\\
512.01	0.00592081643858883\\
513.01	0.00593948199892681\\
514.01	0.00595849888564073\\
515.01	0.00597787636749476\\
516.01	0.00599762384849975\\
517.01	0.0060177508949207\\
518.01	0.00603826724368112\\
519.01	0.00605918279183401\\
520.01	0.00608050758068721\\
521.01	0.00610225178171825\\
522.01	0.006124425685514\\
523.01	0.00614703969441464\\
524.01	0.00617010431929348\\
525.01	0.00619363018054338\\
526.01	0.00621762801289089\\
527.01	0.00624210867316072\\
528.01	0.00626708314964169\\
529.01	0.00629256257138429\\
530.01	0.00631855821573893\\
531.01	0.00634508151289306\\
532.01	0.00637214404718316\\
533.01	0.00639975755639791\\
534.01	0.00642793393108233\\
535.01	0.00645668521483311\\
536.01	0.00648602360550507\\
537.01	0.00651596145698026\\
538.01	0.00654651128106563\\
539.01	0.00657768574904344\\
540.01	0.0066094976924182\\
541.01	0.0066419601024892\\
542.01	0.00667508612853107\\
543.01	0.00670888907454924\\
544.01	0.00674338239474022\\
545.01	0.00677857968783409\\
546.01	0.00681449469037443\\
547.01	0.00685114126877864\\
548.01	0.00688853340991637\\
549.01	0.00692668520993533\\
550.01	0.00696561086108091\\
551.01	0.00700532463627988\\
552.01	0.00704584087128132\\
553.01	0.00708717394416014\\
554.01	0.00712933825198563\\
555.01	0.00717234818441871\\
556.01	0.00721621809395735\\
557.01	0.00726096226249309\\
558.01	0.00730659486381002\\
559.01	0.00735312992163711\\
560.01	0.00740058126284496\\
561.01	0.00744896246535642\\
562.01	0.00749828680030934\\
563.01	0.00754856716798113\\
564.01	0.00759981602694357\\
565.01	0.00765204531588342\\
566.01	0.00770526636749114\\
567.01	0.00775948981379601\\
568.01	0.00781472548230801\\
569.01	0.00787098228231643\\
570.01	0.00792826808069656\\
571.01	0.00798658956658903\\
572.01	0.00804595210435032\\
573.01	0.00810635957423606\\
574.01	0.00816781420037203\\
575.01	0.0082303163657128\\
576.01	0.00829386441388749\\
577.01	0.00835845443810524\\
578.01	0.00842408005766179\\
579.01	0.00849073218307326\\
580.01	0.00855839877149774\\
581.01	0.00862706457492604\\
582.01	0.00869671088467823\\
583.01	0.00876731527708814\\
584.01	0.00883885136696868\\
585.01	0.00891128857761011\\
586.01	0.00898459193878753\\
587.01	0.00905872192767049\\
588.01	0.00913363437180803\\
589.01	0.00920928043871075\\
590.01	0.00928560674321885\\
591.01	0.00936255561214592\\
592.01	0.00944006555600452\\
593.01	0.00951807201042488\\
594.01	0.00959650842575684\\
595.01	0.00967530780301053\\
596.01	0.00975440479863077\\
597.01	0.00983371536751284\\
598.01	0.00990866201848071\\
599.01	0.00997087280416276\\
599.02	0.00997138072163725\\
599.03	0.009971885576258\\
599.04	0.00997238733820211\\
599.05	0.00997288597735272\\
599.06	0.00997338146329604\\
599.07	0.00997387376531845\\
599.08	0.00997436285240349\\
599.09	0.00997484869322892\\
599.1	0.00997533125616361\\
599.11	0.00997581050926455\\
599.12	0.00997628642027372\\
599.13	0.00997675895661497\\
599.14	0.0099772280853909\\
599.15	0.00997769377337961\\
599.16	0.00997815598703157\\
599.17	0.00997861469246631\\
599.18	0.00997906985546915\\
599.19	0.00997952144148792\\
599.2	0.00997996941562957\\
599.21	0.00998041374265684\\
599.22	0.0099808543869848\\
599.23	0.00998129131267744\\
599.24	0.00998172448344418\\
599.25	0.00998215386263636\\
599.26	0.00998257941258354\\
599.27	0.00998300109279825\\
599.28	0.00998341886238981\\
599.29	0.00998383268006024\\
599.3	0.00998424250410032\\
599.31	0.00998464829238543\\
599.32	0.00998505000237151\\
599.33	0.00998544759109087\\
599.34	0.00998584101514799\\
599.35	0.00998623023071532\\
599.36	0.00998661519352896\\
599.37	0.00998699585888436\\
599.38	0.00998737218163197\\
599.39	0.00998774411617281\\
599.4	0.00998811161645403\\
599.41	0.00998847463596443\\
599.42	0.0099888331277299\\
599.43	0.00998918704430885\\
599.44	0.00998953633778757\\
599.45	0.00998988095977558\\
599.46	0.00999022086140088\\
599.47	0.00999055599330522\\
599.48	0.00999088630563925\\
599.49	0.00999121174805768\\
599.5	0.00999153226971435\\
599.51	0.0099918478192573\\
599.52	0.00999215834482375\\
599.53	0.00999246379403503\\
599.54	0.0099927641139915\\
599.55	0.00999305925126738\\
599.56	0.00999334915190553\\
599.57	0.0099936337614122\\
599.58	0.00999391302475173\\
599.59	0.00999418688634118\\
599.6	0.00999445529004489\\
599.61	0.00999471817916904\\
599.62	0.00999497549645613\\
599.63	0.00999522718407935\\
599.64	0.009995473183637\\
599.65	0.00999571343614678\\
599.66	0.00999594788204004\\
599.67	0.00999617646115598\\
599.68	0.0099963991127358\\
599.69	0.00999661577541675\\
599.7	0.00999682638722618\\
599.71	0.00999703088557551\\
599.72	0.00999722920725411\\
599.73	0.00999742128842315\\
599.74	0.00999760706460942\\
599.75	0.00999778647069897\\
599.76	0.00999795944093086\\
599.77	0.0099981259088907\\
599.78	0.0099982858075042\\
599.79	0.00999843906903064\\
599.8	0.00999858562505627\\
599.81	0.00999872540648767\\
599.82	0.00999885834354498\\
599.83	0.00999898436575515\\
599.84	0.00999910340194508\\
599.85	0.00999921538023465\\
599.86	0.00999932022802977\\
599.87	0.00999941787201528\\
599.88	0.00999950823814785\\
599.89	0.00999959125164875\\
599.9	0.00999966683699656\\
599.91	0.00999973491791987\\
599.92	0.0099997954173898\\
599.93	0.00999984825761255\\
599.94	0.00999989336002181\\
599.95	0.00999993064527112\\
599.96	0.00999996003322615\\
599.97	0.00999998144295691\\
599.98	0.00999999479272987\\
599.99	0.01\\
600	0.01\\
};
\addplot [color=mycolor8,solid,forget plot]
  table[row sep=crcr]{%
0.01	0.00440626308531185\\
1.01	0.00440626422127448\\
2.01	0.00440626538095373\\
3.01	0.0044062665648462\\
4.01	0.00440626777345891\\
5.01	0.00440626900730949\\
6.01	0.00440627026692658\\
7.01	0.00440627155284969\\
8.01	0.00440627286562986\\
9.01	0.00440627420582962\\
10.01	0.0044062755740233\\
11.01	0.00440627697079723\\
12.01	0.00440627839675041\\
13.01	0.00440627985249417\\
14.01	0.00440628133865258\\
15.01	0.004406282855863\\
16.01	0.00440628440477632\\
17.01	0.00440628598605691\\
18.01	0.0044062876003831\\
19.01	0.00440628924844742\\
20.01	0.0044062909309573\\
21.01	0.00440629264863485\\
22.01	0.00440629440221726\\
23.01	0.00440629619245744\\
24.01	0.00440629802012393\\
25.01	0.00440629988600161\\
26.01	0.00440630179089169\\
27.01	0.0044063037356125\\
28.01	0.00440630572099936\\
29.01	0.0044063077479054\\
30.01	0.00440630981720147\\
31.01	0.00440631192977683\\
32.01	0.0044063140865396\\
33.01	0.00440631628841669\\
34.01	0.00440631853635493\\
35.01	0.00440632083132096\\
36.01	0.00440632317430162\\
37.01	0.00440632556630473\\
38.01	0.00440632800835908\\
39.01	0.00440633050151536\\
40.01	0.00440633304684623\\
41.01	0.00440633564544717\\
42.01	0.00440633829843672\\
43.01	0.00440634100695664\\
44.01	0.00440634377217311\\
45.01	0.0044063465952769\\
46.01	0.0044063494774837\\
47.01	0.00440635242003476\\
48.01	0.00440635542419768\\
49.01	0.00440635849126676\\
50.01	0.00440636162256351\\
51.01	0.0044063648194374\\
52.01	0.00440636808326623\\
53.01	0.00440637141545691\\
54.01	0.00440637481744614\\
55.01	0.00440637829070072\\
56.01	0.00440638183671864\\
57.01	0.00440638545702918\\
58.01	0.00440638915319415\\
59.01	0.00440639292680821\\
60.01	0.0044063967794996\\
61.01	0.00440640071293111\\
62.01	0.0044064047288004\\
63.01	0.00440640882884109\\
64.01	0.00440641301482347\\
65.01	0.00440641728855513\\
66.01	0.00440642165188182\\
67.01	0.00440642610668807\\
68.01	0.0044064306548985\\
69.01	0.0044064352984782\\
70.01	0.00440644003943368\\
71.01	0.00440644487981423\\
72.01	0.00440644982171214\\
73.01	0.00440645486726378\\
74.01	0.00440646001865092\\
75.01	0.00440646527810121\\
76.01	0.00440647064788945\\
77.01	0.00440647613033868\\
78.01	0.00440648172782072\\
79.01	0.00440648744275766\\
80.01	0.00440649327762303\\
81.01	0.00440649923494237\\
82.01	0.00440650531729445\\
83.01	0.004406511527313\\
84.01	0.00440651786768717\\
85.01	0.00440652434116291\\
86.01	0.0044065309505447\\
87.01	0.00440653769869574\\
88.01	0.00440654458854012\\
89.01	0.00440655162306375\\
90.01	0.00440655880531568\\
91.01	0.00440656613840951\\
92.01	0.00440657362552467\\
93.01	0.00440658126990784\\
94.01	0.00440658907487454\\
95.01	0.00440659704381034\\
96.01	0.00440660518017254\\
97.01	0.00440661348749162\\
98.01	0.00440662196937291\\
99.01	0.00440663062949775\\
100.01	0.00440663947162587\\
101.01	0.0044066484995962\\
102.01	0.00440665771732905\\
103.01	0.00440666712882812\\
104.01	0.00440667673818164\\
105.01	0.00440668654956447\\
106.01	0.00440669656723994\\
107.01	0.00440670679556165\\
108.01	0.00440671723897552\\
109.01	0.00440672790202165\\
110.01	0.00440673878933663\\
111.01	0.00440674990565476\\
112.01	0.00440676125581123\\
113.01	0.00440677284474333\\
114.01	0.00440678467749334\\
115.01	0.00440679675921023\\
116.01	0.0044068090951522\\
117.01	0.00440682169068929\\
118.01	0.00440683455130486\\
119.01	0.00440684768259891\\
120.01	0.00440686109029029\\
121.01	0.00440687478021913\\
122.01	0.00440688875834937\\
123.01	0.00440690303077154\\
124.01	0.00440691760370533\\
125.01	0.00440693248350259\\
126.01	0.00440694767664984\\
127.01	0.00440696318977146\\
128.01	0.00440697902963225\\
129.01	0.00440699520314074\\
130.01	0.00440701171735211\\
131.01	0.00440702857947142\\
132.01	0.00440704579685672\\
133.01	0.00440706337702228\\
134.01	0.00440708132764198\\
135.01	0.00440709965655289\\
136.01	0.00440711837175821\\
137.01	0.0044071374814314\\
138.01	0.00440715699391962\\
139.01	0.00440717691774739\\
140.01	0.00440719726162048\\
141.01	0.00440721803442939\\
142.01	0.0044072392452541\\
143.01	0.00440726090336699\\
144.01	0.00440728301823819\\
145.01	0.00440730559953875\\
146.01	0.00440732865714562\\
147.01	0.00440735220114562\\
148.01	0.00440737624184034\\
149.01	0.0044074007897504\\
150.01	0.00440742585562011\\
151.01	0.00440745145042237\\
152.01	0.00440747758536389\\
153.01	0.00440750427188961\\
154.01	0.00440753152168795\\
155.01	0.00440755934669618\\
156.01	0.00440758775910581\\
157.01	0.00440761677136776\\
158.01	0.00440764639619827\\
159.01	0.00440767664658397\\
160.01	0.00440770753578837\\
161.01	0.00440773907735721\\
162.01	0.00440777128512501\\
163.01	0.00440780417322095\\
164.01	0.00440783775607496\\
165.01	0.00440787204842495\\
166.01	0.00440790706532249\\
167.01	0.00440794282213992\\
168.01	0.00440797933457754\\
169.01	0.00440801661867002\\
170.01	0.00440805469079412\\
171.01	0.00440809356767564\\
172.01	0.0044081332663972\\
173.01	0.00440817380440565\\
174.01	0.00440821519951994\\
175.01	0.00440825746993932\\
176.01	0.00440830063425132\\
177.01	0.00440834471143978\\
178.01	0.00440838972089398\\
179.01	0.0044084356824169\\
180.01	0.0044084826162344\\
181.01	0.00440853054300401\\
182.01	0.00440857948382432\\
183.01	0.00440862946024473\\
184.01	0.00440868049427444\\
185.01	0.00440873260839318\\
186.01	0.00440878582556084\\
187.01	0.00440884016922796\\
188.01	0.00440889566334617\\
189.01	0.00440895233237887\\
190.01	0.00440901020131275\\
191.01	0.00440906929566838\\
192.01	0.00440912964151229\\
193.01	0.00440919126546833\\
194.01	0.00440925419473025\\
195.01	0.00440931845707281\\
196.01	0.00440938408086572\\
197.01	0.00440945109508563\\
198.01	0.00440951952932955\\
199.01	0.00440958941382805\\
200.01	0.00440966077945887\\
201.01	0.00440973365776112\\
202.01	0.00440980808094972\\
203.01	0.00440988408192954\\
204.01	0.00440996169431064\\
205.01	0.00441004095242328\\
206.01	0.00441012189133402\\
207.01	0.00441020454686088\\
208.01	0.00441028895559045\\
209.01	0.0044103751548937\\
210.01	0.00441046318294396\\
211.01	0.0044105530787333\\
212.01	0.00441064488209088\\
213.01	0.0044107386337011\\
214.01	0.00441083437512203\\
215.01	0.00441093214880439\\
216.01	0.00441103199811108\\
217.01	0.00441113396733675\\
218.01	0.00441123810172846\\
219.01	0.00441134444750611\\
220.01	0.00441145305188383\\
221.01	0.00441156396309121\\
222.01	0.00441167723039612\\
223.01	0.0044117929041269\\
224.01	0.00441191103569558\\
225.01	0.00441203167762177\\
226.01	0.00441215488355656\\
227.01	0.00441228070830745\\
228.01	0.00441240920786376\\
229.01	0.00441254043942216\\
230.01	0.00441267446141355\\
231.01	0.00441281133352994\\
232.01	0.00441295111675212\\
233.01	0.00441309387337799\\
234.01	0.00441323966705155\\
235.01	0.00441338856279257\\
236.01	0.00441354062702679\\
237.01	0.00441369592761702\\
238.01	0.00441385453389485\\
239.01	0.00441401651669299\\
240.01	0.00441418194837837\\
241.01	0.00441435090288646\\
242.01	0.00441452345575567\\
243.01	0.00441469968416305\\
244.01	0.00441487966696047\\
245.01	0.00441506348471231\\
246.01	0.00441525121973288\\
247.01	0.00441544295612602\\
248.01	0.0044156387798247\\
249.01	0.00441583877863176\\
250.01	0.00441604304226205\\
251.01	0.00441625166238495\\
252.01	0.00441646473266812\\
253.01	0.00441668234882274\\
254.01	0.00441690460864883\\
255.01	0.00441713161208299\\
256.01	0.00441736346124592\\
257.01	0.00441760026049218\\
258.01	0.00441784211646047\\
259.01	0.00441808913812528\\
260.01	0.00441834143685001\\
261.01	0.00441859912644132\\
262.01	0.00441886232320449\\
263.01	0.00441913114600051\\
264.01	0.00441940571630475\\
265.01	0.00441968615826624\\
266.01	0.00441997259876925\\
267.01	0.00442026516749622\\
268.01	0.00442056399699201\\
269.01	0.00442086922273013\\
270.01	0.0044211809831799\\
271.01	0.00442149941987666\\
272.01	0.00442182467749223\\
273.01	0.00442215690390849\\
274.01	0.00442249625029138\\
275.01	0.00442284287116884\\
276.01	0.00442319692450853\\
277.01	0.00442355857179931\\
278.01	0.00442392797813407\\
279.01	0.00442430531229437\\
280.01	0.00442469074683832\\
281.01	0.00442508445819004\\
282.01	0.0044254866267318\\
283.01	0.00442589743689811\\
284.01	0.0044263170772733\\
285.01	0.00442674574069093\\
286.01	0.00442718362433586\\
287.01	0.00442763092984971\\
288.01	0.004428087863439\\
289.01	0.00442855463598554\\
290.01	0.00442903146316111\\
291.01	0.00442951856554408\\
292.01	0.00443001616874006\\
293.01	0.00443052450350619\\
294.01	0.00443104380587758\\
295.01	0.00443157431729853\\
296.01	0.00443211628475747\\
297.01	0.00443266996092481\\
298.01	0.00443323560429575\\
299.01	0.00443381347933643\\
300.01	0.00443440385663482\\
301.01	0.00443500701305513\\
302.01	0.00443562323189807\\
303.01	0.00443625280306416\\
304.01	0.00443689602322331\\
305.01	0.00443755319598767\\
306.01	0.00443822463209094\\
307.01	0.00443891064957251\\
308.01	0.0044396115739662\\
309.01	0.00444032773849545\\
310.01	0.00444105948427355\\
311.01	0.00444180716050999\\
312.01	0.00444257112472224\\
313.01	0.00444335174295431\\
314.01	0.00444414939000094\\
315.01	0.00444496444963761\\
316.01	0.00444579731485817\\
317.01	0.00444664838811785\\
318.01	0.0044475180815835\\
319.01	0.00444840681738912\\
320.01	0.00444931502789954\\
321.01	0.00445024315597932\\
322.01	0.0044511916552685\\
323.01	0.00445216099046462\\
324.01	0.00445315163761006\\
325.01	0.0044541640843857\\
326.01	0.00445519883040953\\
327.01	0.00445625638753941\\
328.01	0.0044573372801805\\
329.01	0.00445844204559541\\
330.01	0.00445957123421578\\
331.01	0.00446072540995518\\
332.01	0.00446190515052064\\
333.01	0.00446311104772117\\
334.01	0.00446434370777075\\
335.01	0.00446560375158356\\
336.01	0.0044668918150572\\
337.01	0.00446820854933985\\
338.01	0.00446955462107711\\
339.01	0.00447093071263302\\
340.01	0.00447233752227598\\
341.01	0.00447377576432524\\
342.01	0.00447524616924487\\
343.01	0.00447674948367514\\
344.01	0.00447828647038827\\
345.01	0.00447985790815145\\
346.01	0.00448146459147909\\
347.01	0.0044831073302514\\
348.01	0.0044847869491738\\
349.01	0.00448650428704578\\
350.01	0.00448826019580327\\
351.01	0.00449005553929223\\
352.01	0.00449189119172355\\
353.01	0.00449376803575088\\
354.01	0.00449568696010501\\
355.01	0.00449764885670435\\
356.01	0.00449965461715125\\
357.01	0.00450170512850824\\
358.01	0.0045038012682342\\
359.01	0.00450594389814001\\
360.01	0.00450813385720904\\
361.01	0.00451037195310497\\
362.01	0.00451265895217097\\
363.01	0.00451499556770892\\
364.01	0.00451738244630782\\
365.01	0.00451982015198824\\
366.01	0.00452230914792892\\
367.01	0.00452484977556643\\
368.01	0.00452744223090766\\
369.01	0.00453008653798694\\
370.01	0.00453278251955661\\
371.01	0.00453552976534603\\
372.01	0.00453832759860198\\
373.01	0.00454117504219044\\
374.01	0.00454407078637301\\
375.01	0.00454701316158006\\
376.01	0.00455000012124152\\
377.01	0.00455302924221239\\
378.01	0.00455609775384222\\
379.01	0.00455920261167952\\
380.01	0.00456234063874891\\
381.01	0.00456550876703976\\
382.01	0.00456870442538744\\
383.01	0.00457192613875745\\
384.01	0.00457517443009159\\
385.01	0.00457845526305113\\
386.01	0.0045817974796559\\
387.01	0.00458520946202622\\
388.01	0.00458869264373901\\
389.01	0.00459224848572395\\
390.01	0.00459587847657204\\
391.01	0.00459958413282364\\
392.01	0.00460336699923236\\
393.01	0.00460722864900213\\
394.01	0.00461117068399346\\
395.01	0.00461519473489354\\
396.01	0.00461930246134645\\
397.01	0.00462349555203738\\
398.01	0.00462777572472496\\
399.01	0.0046321447262158\\
400.01	0.00463660433227225\\
401.01	0.00464115634744748\\
402.01	0.00464580260483666\\
403.01	0.00465054496573504\\
404.01	0.00465538531919338\\
405.01	0.00466032558145577\\
406.01	0.00466536769526826\\
407.01	0.00467051362904262\\
408.01	0.0046757653758587\\
409.01	0.00468112495228747\\
410.01	0.00468659439701416\\
411.01	0.00469217576923976\\
412.01	0.00469787114683555\\
413.01	0.00470368262422466\\
414.01	0.00470961230996053\\
415.01	0.00471566232396985\\
416.01	0.00472183479442334\\
417.01	0.00472813185419708\\
418.01	0.00473455563688004\\
419.01	0.00474110827228083\\
420.01	0.00474779188138451\\
421.01	0.00475460857070206\\
422.01	0.00476156042595463\\
423.01	0.0047686495050266\\
424.01	0.00477587783011849\\
425.01	0.00478324737902673\\
426.01	0.00479076007546984\\
427.01	0.00479841777838035\\
428.01	0.00480622227007421\\
429.01	0.00481417524321116\\
430.01	0.00482227828645251\\
431.01	0.00483053286872971\\
432.01	0.00483894032203472\\
433.01	0.00484750182265346\\
434.01	0.00485621837077113\\
435.01	0.00486509076839704\\
436.01	0.00487411959558005\\
437.01	0.00488330518491606\\
438.01	0.00489264759439711\\
439.01	0.00490214657870898\\
440.01	0.00491180155916121\\
441.01	0.00492161159253393\\
442.01	0.00493157533925231\\
443.01	0.00494169103145929\\
444.01	0.00495195644175643\\
445.01	0.00496236885363065\\
446.01	0.00497292503488503\\
447.01	0.00498362121575813\\
448.01	0.00499445307385384\\
449.01	0.00500541572852012\\
450.01	0.00501650374791437\\
451.01	0.00502771117267692\\
452.01	0.00503903156088574\\
453.01	0.00505045805977243\\
454.01	0.00506198351047833\\
455.01	0.00507360059285489\\
456.01	0.00508530201781257\\
457.01	0.00509708077479794\\
458.01	0.00510893044128637\\
459.01	0.00512084555923308\\
460.01	0.00513282207947717\\
461.01	0.00514485786805845\\
462.01	0.00515695325669842\\
463.01	0.00516911160101229\\
464.01	0.00518133978105872\\
465.01	0.00519364853487436\\
466.01	0.00520605244994091\\
467.01	0.00521856934055163\\
468.01	0.00523121859675642\\
469.01	0.00524401443556734\\
470.01	0.00525695595333581\\
471.01	0.00527003895708189\\
472.01	0.0052832589415233\\
473.01	0.00529661111701986\\
474.01	0.00531009044759288\\
475.01	0.00532369170046118\\
476.01	0.00533740950855351\\
477.01	0.0053512384473979\\
478.01	0.00536517312763309\\
479.01	0.00537920830408881\\
480.01	0.00539333900192615\\
481.01	0.00540756065978887\\
482.01	0.00542186928905644\\
483.01	0.00543626164696971\\
484.01	0.00545073541969959\\
485.01	0.00546528940926458\\
486.01	0.00547992371550203\\
487.01	0.00549463990104536\\
488.01	0.00550944112351877\\
489.01	0.00552433221510523\\
490.01	0.00553931968568339\\
491.01	0.00555441162261239\\
492.01	0.00556961745924792\\
493.01	0.00558494758736217\\
494.01	0.0056004127988741\\
495.01	0.00561602356490333\\
496.01	0.00563178920303264\\
497.01	0.00564771705855759\\
498.01	0.00566381198048466\\
499.01	0.00568007724959183\\
500.01	0.00569651658050933\\
501.01	0.00571313456604973\\
502.01	0.00572993672726638\\
503.01	0.00574692954501591\\
504.01	0.0057641204693081\\
505.01	0.00578151790174496\\
506.01	0.00579913114680775\\
507.01	0.00581697032881338\\
508.01	0.00583504627322513\\
509.01	0.00585337035387697\\
510.01	0.00587195431173997\\
511.01	0.0058908100562076\\
512.01	0.00590994946638894\\
513.01	0.00592938421704352\\
514.01	0.00594912566032621\\
515.01	0.005969184797927\\
516.01	0.00598957237400632\\
517.01	0.00601029909958091\\
518.01	0.00603137594750737\\
519.01	0.00605281429317925\\
520.01	0.00607462587650683\\
521.01	0.00609682273422721\\
522.01	0.00611941712827492\\
523.01	0.00614242147440214\\
524.01	0.00616584827601105\\
525.01	0.00618971006870538\\
526.01	0.00621401938102251\\
527.01	0.00623878871588064\\
528.01	0.00626403055516378\\
529.01	0.00628975738633661\\
530.01	0.00631598174502204\\
531.01	0.0063427162616497\\
532.01	0.00636997369532064\\
533.01	0.00639776693951445\\
534.01	0.00642610900545185\\
535.01	0.00645501300392987\\
536.01	0.00648449213262774\\
537.01	0.00651455966959223\\
538.01	0.00654522897277823\\
539.01	0.00657651348475956\\
540.01	0.00660842674092671\\
541.01	0.00664098237881587\\
542.01	0.00667419414588415\\
543.01	0.00670807590334808\\
544.01	0.00674264162486408\\
545.01	0.00677790539086053\\
546.01	0.00681388138114321\\
547.01	0.00685058386763962\\
548.01	0.00688802720720362\\
549.01	0.00692622583376501\\
550.01	0.00696519424901141\\
551.01	0.00700494701080015\\
552.01	0.00704549871862432\\
553.01	0.00708686399568258\\
554.01	0.00712905746738202\\
555.01	0.00717209373633373\\
556.01	0.00721598735393222\\
557.01	0.00726075278837836\\
558.01	0.00730640438870886\\
559.01	0.0073529563442945\\
560.01	0.0074004226392819\\
561.01	0.00744881700148302\\
562.01	0.0074981528452527\\
563.01	0.00754844320790874\\
564.01	0.00759970067923708\\
565.01	0.00765193732357281\\
566.01	0.00770516459387936\\
567.01	0.00775939323719052\\
568.01	0.0078146331907652\\
569.01	0.00787089346830959\\
570.01	0.00792818203563622\\
571.01	0.00798650567515498\\
572.01	0.00804586983862646\\
573.01	0.00810627848766415\\
574.01	0.00816773392156107\\
575.01	0.00823023659215375\\
576.01	0.00829378490563244\\
577.01	0.00835837501148391\\
578.01	0.00842400057911648\\
579.01	0.00849065256320018\\
580.01	0.00855831895938081\\
581.01	0.00862698455284182\\
582.01	0.00869663066323415\\
583.01	0.00876723489083508\\
584.01	0.00883877087050155\\
585.01	0.00891120804214121\\
586.01	0.00898451144914387\\
587.01	0.00905864157963132\\
588.01	0.00913355426966545\\
589.01	0.00920920069290074\\
590.01	0.0092855274678398\\
591.01	0.00936247692215228\\
592.01	0.00943998756383507\\
593.01	0.00951799482179795\\
594.01	0.00959643213433056\\
595.01	0.00967523248356592\\
596.01	0.00975433049838055\\
597.01	0.00983365065339071\\
598.01	0.00990866201848071\\
599.01	0.00997087280416276\\
599.02	0.00997138072163725\\
599.03	0.009971885576258\\
599.04	0.00997238733820211\\
599.05	0.00997288597735272\\
599.06	0.00997338146329604\\
599.07	0.00997387376531845\\
599.08	0.00997436285240349\\
599.09	0.00997484869322892\\
599.1	0.00997533125616361\\
599.11	0.00997581050926455\\
599.12	0.00997628642027372\\
599.13	0.00997675895661498\\
599.14	0.0099772280853909\\
599.15	0.00997769377337961\\
599.16	0.00997815598703157\\
599.17	0.00997861469246631\\
599.18	0.00997906985546915\\
599.19	0.00997952144148792\\
599.2	0.00997996941562957\\
599.21	0.00998041374265684\\
599.22	0.0099808543869848\\
599.23	0.00998129131267744\\
599.24	0.00998172448344418\\
599.25	0.00998215386263636\\
599.26	0.00998257941258353\\
599.27	0.00998300109279825\\
599.28	0.00998341886238981\\
599.29	0.00998383268006024\\
599.3	0.00998424250410032\\
599.31	0.00998464829238543\\
599.32	0.00998505000237151\\
599.33	0.00998544759109087\\
599.34	0.00998584101514799\\
599.35	0.00998623023071532\\
599.36	0.00998661519352895\\
599.37	0.00998699585888436\\
599.38	0.00998737218163197\\
599.39	0.00998774411617281\\
599.4	0.00998811161645403\\
599.41	0.00998847463596443\\
599.42	0.0099888331277299\\
599.43	0.00998918704430885\\
599.44	0.00998953633778757\\
599.45	0.00998988095977558\\
599.46	0.00999022086140088\\
599.47	0.00999055599330522\\
599.48	0.00999088630563925\\
599.49	0.00999121174805768\\
599.5	0.00999153226971435\\
599.51	0.0099918478192573\\
599.52	0.00999215834482374\\
599.53	0.00999246379403503\\
599.54	0.0099927641139915\\
599.55	0.00999305925126738\\
599.56	0.00999334915190553\\
599.57	0.0099936337614122\\
599.58	0.00999391302475173\\
599.59	0.00999418688634118\\
599.6	0.00999445529004489\\
599.61	0.00999471817916904\\
599.62	0.00999497549645613\\
599.63	0.00999522718407934\\
599.64	0.009995473183637\\
599.65	0.00999571343614678\\
599.66	0.00999594788204004\\
599.67	0.00999617646115599\\
599.68	0.0099963991127358\\
599.69	0.00999661577541675\\
599.7	0.00999682638722618\\
599.71	0.00999703088557551\\
599.72	0.00999722920725411\\
599.73	0.00999742128842315\\
599.74	0.00999760706460942\\
599.75	0.00999778647069897\\
599.76	0.00999795944093086\\
599.77	0.0099981259088907\\
599.78	0.0099982858075042\\
599.79	0.00999843906903064\\
599.8	0.00999858562505627\\
599.81	0.00999872540648767\\
599.82	0.00999885834354498\\
599.83	0.00999898436575515\\
599.84	0.00999910340194508\\
599.85	0.00999921538023465\\
599.86	0.00999932022802977\\
599.87	0.00999941787201528\\
599.88	0.00999950823814785\\
599.89	0.00999959125164875\\
599.9	0.00999966683699656\\
599.91	0.00999973491791987\\
599.92	0.0099997954173898\\
599.93	0.00999984825761255\\
599.94	0.00999989336002181\\
599.95	0.00999993064527112\\
599.96	0.00999996003322615\\
599.97	0.00999998144295691\\
599.98	0.00999999479272987\\
599.99	0.01\\
600	0.01\\
};
\addplot [color=blue!25!mycolor7,solid,forget plot]
  table[row sep=crcr]{%
0.01	0.00413645223902454\\
1.01	0.00413645300246634\\
2.01	0.00413645378183859\\
3.01	0.00413645457747443\\
4.01	0.0041364553897143\\
5.01	0.00413645621890542\\
6.01	0.00413645706540267\\
7.01	0.00413645792956811\\
8.01	0.00413645881177166\\
9.01	0.00413645971239082\\
10.01	0.00413646063181095\\
11.01	0.00413646157042567\\
12.01	0.00413646252863661\\
13.01	0.00413646350685418\\
14.01	0.00413646450549729\\
15.01	0.00413646552499354\\
16.01	0.00413646656577953\\
17.01	0.00413646762830092\\
18.01	0.00413646871301321\\
19.01	0.00413646982038111\\
20.01	0.00413647095087882\\
21.01	0.00413647210499092\\
22.01	0.00413647328321212\\
23.01	0.00413647448604743\\
24.01	0.0041364757140125\\
25.01	0.0041364769676339\\
26.01	0.00413647824744907\\
27.01	0.00413647955400687\\
28.01	0.00413648088786767\\
29.01	0.0041364822496036\\
30.01	0.00413648363979888\\
31.01	0.00413648505905004\\
32.01	0.00413648650796607\\
33.01	0.00413648798716877\\
34.01	0.00413648949729303\\
35.01	0.00413649103898717\\
36.01	0.00413649261291286\\
37.01	0.00413649421974589\\
38.01	0.00413649586017656\\
39.01	0.00413649753490919\\
40.01	0.0041364992446631\\
41.01	0.00413650099017301\\
42.01	0.00413650277218841\\
43.01	0.00413650459147534\\
44.01	0.00413650644881562\\
45.01	0.00413650834500721\\
46.01	0.00413651028086549\\
47.01	0.00413651225722261\\
48.01	0.00413651427492816\\
49.01	0.00413651633484986\\
50.01	0.0041365184378737\\
51.01	0.00413652058490413\\
52.01	0.00413652277686458\\
53.01	0.0041365250146983\\
54.01	0.00413652729936799\\
55.01	0.00413652963185687\\
56.01	0.00413653201316876\\
57.01	0.00413653444432874\\
58.01	0.00413653692638334\\
59.01	0.00413653946040121\\
60.01	0.00413654204747346\\
61.01	0.00413654468871429\\
62.01	0.00413654738526122\\
63.01	0.00413655013827585\\
64.01	0.00413655294894416\\
65.01	0.00413655581847719\\
66.01	0.00413655874811143\\
67.01	0.00413656173910959\\
68.01	0.00413656479276098\\
69.01	0.00413656791038176\\
70.01	0.00413657109331628\\
71.01	0.00413657434293679\\
72.01	0.00413657766064482\\
73.01	0.00413658104787133\\
74.01	0.00413658450607736\\
75.01	0.0041365880367549\\
76.01	0.00413659164142743\\
77.01	0.0041365953216502\\
78.01	0.0041365990790117\\
79.01	0.00413660291513383\\
80.01	0.00413660683167236\\
81.01	0.00413661083031852\\
82.01	0.00413661491279878\\
83.01	0.00413661908087617\\
84.01	0.00413662333635101\\
85.01	0.00413662768106166\\
86.01	0.00413663211688484\\
87.01	0.00413663664573743\\
88.01	0.00413664126957637\\
89.01	0.00413664599039996\\
90.01	0.00413665081024897\\
91.01	0.00413665573120693\\
92.01	0.00413666075540151\\
93.01	0.00413666588500525\\
94.01	0.00413667112223645\\
95.01	0.00413667646936043\\
96.01	0.00413668192869014\\
97.01	0.00413668750258765\\
98.01	0.00413669319346451\\
99.01	0.00413669900378365\\
100.01	0.00413670493605961\\
101.01	0.00413671099286049\\
102.01	0.00413671717680839\\
103.01	0.00413672349058052\\
104.01	0.00413672993691108\\
105.01	0.00413673651859182\\
106.01	0.0041367432384736\\
107.01	0.00413675009946735\\
108.01	0.00413675710454571\\
109.01	0.00413676425674391\\
110.01	0.00413677155916145\\
111.01	0.00413677901496344\\
112.01	0.0041367866273818\\
113.01	0.00413679439971668\\
114.01	0.00413680233533802\\
115.01	0.00413681043768732\\
116.01	0.00413681871027842\\
117.01	0.0041368271566995\\
118.01	0.00413683578061494\\
119.01	0.00413684458576649\\
120.01	0.00413685357597467\\
121.01	0.00413686275514111\\
122.01	0.00413687212725002\\
123.01	0.00413688169636965\\
124.01	0.00413689146665469\\
125.01	0.00413690144234731\\
126.01	0.00413691162777957\\
127.01	0.00413692202737526\\
128.01	0.00413693264565181\\
129.01	0.00413694348722218\\
130.01	0.00413695455679676\\
131.01	0.00413696585918608\\
132.01	0.00413697739930197\\
133.01	0.0041369891821606\\
134.01	0.00413700121288418\\
135.01	0.00413701349670324\\
136.01	0.00413702603895943\\
137.01	0.0041370388451072\\
138.01	0.00413705192071638\\
139.01	0.00413706527147512\\
140.01	0.00413707890319162\\
141.01	0.00413709282179756\\
142.01	0.00413710703334966\\
143.01	0.00413712154403365\\
144.01	0.00413713636016562\\
145.01	0.00413715148819595\\
146.01	0.00413716693471143\\
147.01	0.00413718270643845\\
148.01	0.00413719881024602\\
149.01	0.0041372152531487\\
150.01	0.00413723204230966\\
151.01	0.00413724918504406\\
152.01	0.00413726668882178\\
153.01	0.0041372845612714\\
154.01	0.00413730281018307\\
155.01	0.00413732144351218\\
156.01	0.00413734046938279\\
157.01	0.00413735989609126\\
158.01	0.00413737973210988\\
159.01	0.00413739998609051\\
160.01	0.00413742066686876\\
161.01	0.00413744178346762\\
162.01	0.0041374633451015\\
163.01	0.00413748536118023\\
164.01	0.0041375078413135\\
165.01	0.00413753079531483\\
166.01	0.00413755423320596\\
167.01	0.00413757816522158\\
168.01	0.00413760260181349\\
169.01	0.0041376275536554\\
170.01	0.0041376530316477\\
171.01	0.00413767904692233\\
172.01	0.00413770561084746\\
173.01	0.00413773273503304\\
174.01	0.00413776043133533\\
175.01	0.00413778871186272\\
176.01	0.00413781758898084\\
177.01	0.00413784707531809\\
178.01	0.0041378771837714\\
179.01	0.00413790792751159\\
180.01	0.00413793931998961\\
181.01	0.00413797137494241\\
182.01	0.00413800410639917\\
183.01	0.00413803752868733\\
184.01	0.0041380716564393\\
185.01	0.00413810650459848\\
186.01	0.00413814208842662\\
187.01	0.0041381784235099\\
188.01	0.00413821552576664\\
189.01	0.00413825341145374\\
190.01	0.00413829209717424\\
191.01	0.00413833159988498\\
192.01	0.00413837193690397\\
193.01	0.00413841312591806\\
194.01	0.00413845518499083\\
195.01	0.00413849813257113\\
196.01	0.00413854198750095\\
197.01	0.00413858676902401\\
198.01	0.0041386324967943\\
199.01	0.00413867919088515\\
200.01	0.0041387268717981\\
201.01	0.00413877556047228\\
202.01	0.00413882527829372\\
203.01	0.00413887604710475\\
204.01	0.0041389278892144\\
205.01	0.00413898082740843\\
206.01	0.00413903488495899\\
207.01	0.00413909008563577\\
208.01	0.0041391464537166\\
209.01	0.00413920401399849\\
210.01	0.00413926279180882\\
211.01	0.00413932281301681\\
212.01	0.00413938410404531\\
213.01	0.00413944669188301\\
214.01	0.00413951060409641\\
215.01	0.00413957586884263\\
216.01	0.00413964251488226\\
217.01	0.00413971057159257\\
218.01	0.00413978006898054\\
219.01	0.00413985103769716\\
220.01	0.00413992350905104\\
221.01	0.00413999751502343\\
222.01	0.00414007308828199\\
223.01	0.00414015026219657\\
224.01	0.00414022907085443\\
225.01	0.00414030954907614\\
226.01	0.0041403917324311\\
227.01	0.00414047565725467\\
228.01	0.00414056136066472\\
229.01	0.00414064888057912\\
230.01	0.0041407382557329\\
231.01	0.00414082952569686\\
232.01	0.00414092273089588\\
233.01	0.00414101791262771\\
234.01	0.00414111511308253\\
235.01	0.00414121437536263\\
236.01	0.00414131574350272\\
237.01	0.00414141926249077\\
238.01	0.00414152497828936\\
239.01	0.00414163293785735\\
240.01	0.00414174318917206\\
241.01	0.00414185578125228\\
242.01	0.00414197076418168\\
243.01	0.00414208818913275\\
244.01	0.00414220810839117\\
245.01	0.00414233057538117\\
246.01	0.00414245564469116\\
247.01	0.00414258337210011\\
248.01	0.00414271381460452\\
249.01	0.0041428470304464\\
250.01	0.00414298307914145\\
251.01	0.00414312202150835\\
252.01	0.00414326391969813\\
253.01	0.00414340883722551\\
254.01	0.00414355683899971\\
255.01	0.00414370799135663\\
256.01	0.0041438623620923\\
257.01	0.00414402002049635\\
258.01	0.00414418103738673\\
259.01	0.00414434548514565\\
260.01	0.00414451343775553\\
261.01	0.00414468497083743\\
262.01	0.00414486016168879\\
263.01	0.00414503908932352\\
264.01	0.0041452218345121\\
265.01	0.00414540847982364\\
266.01	0.00414559910966864\\
267.01	0.00414579381034308\\
268.01	0.00414599267007339\\
269.01	0.00414619577906319\\
270.01	0.00414640322954103\\
271.01	0.00414661511580915\\
272.01	0.00414683153429449\\
273.01	0.00414705258360024\\
274.01	0.00414727836455965\\
275.01	0.00414750898029047\\
276.01	0.00414774453625195\\
277.01	0.00414798514030318\\
278.01	0.00414823090276233\\
279.01	0.00414848193646948\\
280.01	0.0041487383568493\\
281.01	0.00414900028197704\\
282.01	0.00414926783264594\\
283.01	0.00414954113243737\\
284.01	0.00414982030779178\\
285.01	0.00415010548808338\\
286.01	0.00415039680569666\\
287.01	0.00415069439610491\\
288.01	0.00415099839795185\\
289.01	0.0041513089531358\\
290.01	0.00415162620689657\\
291.01	0.00415195030790523\\
292.01	0.00415228140835719\\
293.01	0.00415261966406762\\
294.01	0.00415296523457169\\
295.01	0.00415331828322656\\
296.01	0.00415367897731833\\
297.01	0.00415404748817211\\
298.01	0.0041544239912662\\
299.01	0.00415480866635073\\
300.01	0.00415520169757005\\
301.01	0.00415560327359081\\
302.01	0.00415601358773349\\
303.01	0.00415643283811012\\
304.01	0.00415686122776703\\
305.01	0.00415729896483309\\
306.01	0.00415774626267394\\
307.01	0.0041582033400524\\
308.01	0.00415867042129625\\
309.01	0.00415914773647154\\
310.01	0.00415963552156495\\
311.01	0.00416013401867235\\
312.01	0.00416064347619738\\
313.01	0.00416116414905673\\
314.01	0.0041616962988969\\
315.01	0.0041622401943196\\
316.01	0.00416279611111757\\
317.01	0.0041633643325221\\
318.01	0.00416394514946271\\
319.01	0.00416453886083896\\
320.01	0.00416514577380607\\
321.01	0.00416576620407566\\
322.01	0.00416640047623104\\
323.01	0.00416704892406029\\
324.01	0.00416771189090702\\
325.01	0.00416838973004008\\
326.01	0.00416908280504448\\
327.01	0.00416979149023516\\
328.01	0.0041705161710944\\
329.01	0.00417125724473598\\
330.01	0.00417201512039816\\
331.01	0.00417279021996729\\
332.01	0.00417358297853551\\
333.01	0.00417439384499517\\
334.01	0.00417522328267431\\
335.01	0.00417607177001537\\
336.01	0.00417693980130271\\
337.01	0.00417782788744424\\
338.01	0.00417873655681118\\
339.01	0.00417966635614245\\
340.01	0.00418061785152207\\
341.01	0.00418159162943539\\
342.01	0.00418258829791465\\
343.01	0.00418360848778307\\
344.01	0.00418465285400919\\
345.01	0.0041857220771854\\
346.01	0.00418681686514386\\
347.01	0.00418793795472863\\
348.01	0.00418908611374225\\
349.01	0.00419026214308968\\
350.01	0.00419146687914455\\
351.01	0.0041927011963669\\
352.01	0.00419396601020493\\
353.01	0.00419526228031948\\
354.01	0.00419659101417189\\
355.01	0.00419795327102597\\
356.01	0.00419935016641657\\
357.01	0.00420078287714831\\
358.01	0.00420225264688966\\
359.01	0.00420376079244129\\
360.01	0.00420530871075799\\
361.01	0.00420689788681412\\
362.01	0.0042085299024041\\
363.01	0.0042102064459706\\
364.01	0.0042119293235499\\
365.01	0.00421370047090788\\
366.01	0.00421552196691868\\
367.01	0.00421739604819074\\
368.01	0.00421932512487386\\
369.01	0.00422131179747372\\
370.01	0.0042233588743284\\
371.01	0.00422546938915921\\
372.01	0.00422764661774756\\
373.01	0.00422989409227113\\
374.01	0.00423221561109659\\
375.01	0.00423461524077492\\
376.01	0.00423709730550745\\
377.01	0.00423966635727248\\
378.01	0.00424232711689224\\
379.01	0.00424508437225969\\
380.01	0.00424794281429006\\
381.01	0.00425090678330682\\
382.01	0.0042539798876709\\
383.01	0.00425716444136141\\
384.01	0.00426046064632322\\
385.01	0.00426386332049204\\
386.01	0.0042673453438255\\
387.01	0.00427090121902333\\
388.01	0.0042745325387893\\
389.01	0.00427824093201318\\
390.01	0.00428202806470326\\
391.01	0.00428589564094661\\
392.01	0.0042898454039004\\
393.01	0.00429387913681263\\
394.01	0.0042979986640759\\
395.01	0.00430220585231304\\
396.01	0.00430650261149555\\
397.01	0.00431089089609782\\
398.01	0.00431537270628513\\
399.01	0.00431995008913716\\
400.01	0.00432462513990922\\
401.01	0.00432940000332797\\
402.01	0.0043342768749257\\
403.01	0.00433925800241129\\
404.01	0.00434434568707703\\
405.01	0.00434954228524249\\
406.01	0.00435485020973335\\
407.01	0.00436027193139435\\
408.01	0.00436580998063417\\
409.01	0.00437146694899987\\
410.01	0.0043772454907771\\
411.01	0.00438314832461265\\
412.01	0.00438917823515285\\
413.01	0.00439533807469185\\
414.01	0.00440163076482111\\
415.01	0.00440805929806949\\
416.01	0.00441462673952259\\
417.01	0.0044213362284047\\
418.01	0.00442819097960613\\
419.01	0.00443519428513427\\
420.01	0.00444234951546026\\
421.01	0.00444966012073209\\
422.01	0.00445712963181495\\
423.01	0.00446476166111412\\
424.01	0.00447255990312809\\
425.01	0.00448052813466672\\
426.01	0.00448867021466183\\
427.01	0.00449699008347793\\
428.01	0.00450549176162084\\
429.01	0.0045141793477156\\
430.01	0.00452305701561066\\
431.01	0.00453212901043024\\
432.01	0.00454139964337136\\
433.01	0.00455087328500416\\
434.01	0.00456055435678991\\
435.01	0.00457044732048396\\
436.01	0.0045805566650301\\
437.01	0.00459088689049059\\
438.01	0.00460144248847508\\
439.01	0.0046122279184453\\
440.01	0.00462324757917482\\
441.01	0.00463450577452604\\
442.01	0.00464600667258623\\
443.01	0.00465775425706393\\
444.01	0.00466975226970179\\
445.01	0.00468200414230838\\
446.01	0.00469451291685857\\
447.01	0.00470728115197526\\
448.01	0.00472031081399361\\
449.01	0.00473360315075769\\
450.01	0.00474715854634261\\
451.01	0.00476097635509615\\
452.01	0.00477505471382782\\
453.01	0.00478939033177071\\
454.01	0.00480397825926507\\
455.01	0.00481881163820742\\
456.01	0.00483388144051397\\
457.01	0.00484917620564145\\
458.01	0.00486468179525587\\
459.01	0.00488038119337847\\
460.01	0.00489625439505458\\
461.01	0.00491227844760229\\
462.01	0.00492842773829337\\
463.01	0.00494467466437917\\
464.01	0.00496099088050544\\
465.01	0.00497734940141575\\
466.01	0.00499372795367321\\
467.01	0.00501011413174371\\
468.01	0.0050265133947317\\
469.01	0.00504306504391844\\
470.01	0.0050598695421144\\
471.01	0.00507692297182101\\
472.01	0.00509422048189447\\
473.01	0.00511175620985842\\
474.01	0.00512952320483855\\
475.01	0.0051475133534548\\
476.01	0.00516571731177395\\
477.01	0.00518412444739152\\
478.01	0.00520272279690785\\
479.01	0.00522149904552283\\
480.01	0.00524043853601257\\
481.01	0.00525952531454729\\
482.01	0.00527874222508858\\
483.01	0.00529807106619583\\
484.01	0.00531749282573263\\
485.01	0.00533698801071057\\
486.01	0.0053565370906694\\
487.01	0.00537612107299407\\
488.01	0.00539572222640368\\
489.01	0.0054153249630382\\
490.01	0.00543491687789839\\
491.01	0.00545448992356011\\
492.01	0.00547404166447779\\
493.01	0.00549357650763554\\
494.01	0.00551310673116578\\
495.01	0.00553265301150467\\
496.01	0.00555224397090921\\
497.01	0.00557191400421153\\
498.01	0.00559169612135395\\
499.01	0.00561159653081639\\
500.01	0.0056316070469707\\
501.01	0.00565172015606437\\
502.01	0.00567192950010262\\
503.01	0.0056922301992143\\
504.01	0.00571261918965564\\
505.01	0.00573309556467543\\
506.01	0.00575366090006868\\
507.01	0.00577431953984645\\
508.01	0.00579507881024807\\
509.01	0.00581594912282832\\
510.01	0.00583694392057781\\
511.01	0.00585807941681996\\
512.01	0.00587937407805668\\
513.01	0.00590084781407138\\
514.01	0.00592252086944919\\
515.01	0.00594441247270583\\
516.01	0.00596653941150554\\
517.01	0.00598891490954435\\
518.01	0.00601154936829866\\
519.01	0.00603445364319628\\
520.01	0.00605764003084672\\
521.01	0.00608112221141253\\
522.01	0.00610491512470457\\
523.01	0.00612903478785888\\
524.01	0.00615349805588999\\
525.01	0.00617832233322134\\
526.01	0.00620352525356942\\
527.01	0.00622912435644384\\
528.01	0.0062551368000985\\
529.01	0.00628157916136718\\
530.01	0.00630846737794379\\
531.01	0.00633581688014802\\
532.01	0.00636364292158714\\
533.01	0.00639196096373634\\
534.01	0.00642078677220527\\
535.01	0.00645013631912694\\
536.01	0.00648002567432941\\
537.01	0.00651047090700989\\
538.01	0.00654148800648144\\
539.01	0.00657309283000611\\
540.01	0.00660530108354772\\
541.01	0.00663812833697643\\
542.01	0.00667159006859329\\
543.01	0.00670570172519962\\
544.01	0.00674047877514396\\
545.01	0.00677593672836852\\
546.01	0.00681209111868202\\
547.01	0.00684895747719921\\
548.01	0.00688655131297821\\
549.01	0.00692488810181778\\
550.01	0.00696398328188797\\
551.01	0.00700385225360844\\
552.01	0.00704451038011664\\
553.01	0.00708597298416507\\
554.01	0.00712825533781296\\
555.01	0.00717137264321357\\
556.01	0.00721534000599387\\
557.01	0.00726017240483853\\
558.01	0.00730588465886365\\
559.01	0.00735249139193321\\
560.01	0.0074000069925621\\
561.01	0.00744844556806246\\
562.01	0.00749782089178799\\
563.01	0.00754814634269311\\
564.01	0.00759943483685668\\
565.01	0.00765169875093423\\
566.01	0.00770494983745968\\
567.01	0.0077591991315086\\
568.01	0.0078144568479118\\
569.01	0.00787073226819176\\
570.01	0.00792803361649234\\
571.01	0.00798636792389737\\
572.01	0.00804574088066074\\
573.01	0.00810615667597409\\
574.01	0.00816761782497417\\
575.01	0.0082301249827724\\
576.01	0.00829367674544343\\
577.01	0.00835826943818673\\
578.01	0.00842389689126298\\
579.01	0.00849055020480459\\
580.01	0.00855821750423072\\
581.01	0.00862688368879396\\
582.01	0.00869653017680659\\
583.01	0.00876713465240274\\
584.01	0.00883867082037334\\
585.01	0.00891110817774947\\
586.01	0.00898441181351145\\
587.01	0.0090585422512016\\
588.01	0.00913345535348011\\
589.01	0.00920910231300336\\
590.01	0.00928542976066912\\
591.01	0.00936238003058418\\
592.01	0.00943989163144331\\
593.01	0.00951789998683473\\
594.01	0.00959633852288064\\
595.01	0.00967514020129159\\
596.01	0.00975423962023931\\
597.01	0.00983357023250315\\
598.01	0.00990866201848071\\
599.01	0.00997087280416276\\
599.02	0.00997138072163725\\
599.03	0.00997188557625799\\
599.04	0.00997238733820211\\
599.05	0.00997288597735272\\
599.06	0.00997338146329604\\
599.07	0.00997387376531845\\
599.08	0.00997436285240349\\
599.09	0.00997484869322892\\
599.1	0.00997533125616361\\
599.11	0.00997581050926455\\
599.12	0.00997628642027372\\
599.13	0.00997675895661497\\
599.14	0.0099772280853909\\
599.15	0.00997769377337961\\
599.16	0.00997815598703157\\
599.17	0.00997861469246631\\
599.18	0.00997906985546915\\
599.19	0.00997952144148792\\
599.2	0.00997996941562957\\
599.21	0.00998041374265684\\
599.22	0.0099808543869848\\
599.23	0.00998129131267744\\
599.24	0.00998172448344418\\
599.25	0.00998215386263636\\
599.26	0.00998257941258354\\
599.27	0.00998300109279825\\
599.28	0.00998341886238981\\
599.29	0.00998383268006024\\
599.3	0.00998424250410032\\
599.31	0.00998464829238543\\
599.32	0.00998505000237151\\
599.33	0.00998544759109087\\
599.34	0.00998584101514799\\
599.35	0.00998623023071532\\
599.36	0.00998661519352896\\
599.37	0.00998699585888436\\
599.38	0.00998737218163197\\
599.39	0.00998774411617281\\
599.4	0.00998811161645403\\
599.41	0.00998847463596443\\
599.42	0.0099888331277299\\
599.43	0.00998918704430885\\
599.44	0.00998953633778757\\
599.45	0.00998988095977558\\
599.46	0.00999022086140088\\
599.47	0.00999055599330522\\
599.48	0.00999088630563925\\
599.49	0.00999121174805768\\
599.5	0.00999153226971435\\
599.51	0.0099918478192573\\
599.52	0.00999215834482374\\
599.53	0.00999246379403503\\
599.54	0.0099927641139915\\
599.55	0.00999305925126738\\
599.56	0.00999334915190553\\
599.57	0.0099936337614122\\
599.58	0.00999391302475173\\
599.59	0.00999418688634118\\
599.6	0.00999445529004489\\
599.61	0.00999471817916904\\
599.62	0.00999497549645613\\
599.63	0.00999522718407934\\
599.64	0.009995473183637\\
599.65	0.00999571343614678\\
599.66	0.00999594788204004\\
599.67	0.00999617646115598\\
599.68	0.0099963991127358\\
599.69	0.00999661577541675\\
599.7	0.00999682638722618\\
599.71	0.00999703088557551\\
599.72	0.00999722920725411\\
599.73	0.00999742128842315\\
599.74	0.00999760706460942\\
599.75	0.00999778647069897\\
599.76	0.00999795944093086\\
599.77	0.0099981259088907\\
599.78	0.0099982858075042\\
599.79	0.00999843906903064\\
599.8	0.00999858562505627\\
599.81	0.00999872540648767\\
599.82	0.00999885834354498\\
599.83	0.00999898436575515\\
599.84	0.00999910340194508\\
599.85	0.00999921538023465\\
599.86	0.00999932022802977\\
599.87	0.00999941787201528\\
599.88	0.00999950823814785\\
599.89	0.00999959125164875\\
599.9	0.00999966683699656\\
599.91	0.00999973491791987\\
599.92	0.0099997954173898\\
599.93	0.00999984825761255\\
599.94	0.00999989336002181\\
599.95	0.00999993064527112\\
599.96	0.00999996003322615\\
599.97	0.00999998144295691\\
599.98	0.00999999479272987\\
599.99	0.01\\
600	0.01\\
};
\addplot [color=mycolor9,solid,forget plot]
  table[row sep=crcr]{%
0.01	0.00400079225488101\\
1.01	0.00400079282381871\\
2.01	0.00400079340463527\\
3.01	0.00400079399757922\\
4.01	0.00400079460290462\\
5.01	0.00400079522087064\\
6.01	0.00400079585174193\\
7.01	0.00400079649578867\\
8.01	0.00400079715328687\\
9.01	0.0040007978245182\\
10.01	0.00400079850977031\\
11.01	0.00400079920933689\\
12.01	0.00400079992351778\\
13.01	0.00400080065261907\\
14.01	0.00400080139695326\\
15.01	0.00400080215683971\\
16.01	0.00400080293260409\\
17.01	0.00400080372457927\\
18.01	0.00400080453310467\\
19.01	0.00400080535852729\\
20.01	0.00400080620120127\\
21.01	0.0040008070614881\\
22.01	0.00400080793975696\\
23.01	0.00400080883638475\\
24.01	0.00400080975175637\\
25.01	0.00400081068626465\\
26.01	0.00400081164031092\\
27.01	0.00400081261430475\\
28.01	0.00400081360866447\\
29.01	0.00400081462381706\\
30.01	0.00400081566019855\\
31.01	0.00400081671825407\\
32.01	0.00400081779843835\\
33.01	0.00400081890121556\\
34.01	0.00400082002705943\\
35.01	0.00400082117645384\\
36.01	0.00400082234989289\\
37.01	0.00400082354788113\\
38.01	0.00400082477093339\\
39.01	0.00400082601957574\\
40.01	0.0040008272943452\\
41.01	0.00400082859578976\\
42.01	0.00400082992446962\\
43.01	0.0040008312809561\\
44.01	0.00400083266583293\\
45.01	0.00400083407969611\\
46.01	0.00400083552315401\\
47.01	0.00400083699682792\\
48.01	0.00400083850135223\\
49.01	0.00400084003737461\\
50.01	0.00400084160555644\\
51.01	0.00400084320657304\\
52.01	0.00400084484111402\\
53.01	0.00400084650988335\\
54.01	0.00400084821359988\\
55.01	0.00400084995299756\\
56.01	0.00400085172882597\\
57.01	0.00400085354185044\\
58.01	0.0040008553928523\\
59.01	0.00400085728262939\\
60.01	0.00400085921199654\\
61.01	0.00400086118178546\\
62.01	0.00400086319284565\\
63.01	0.00400086524604443\\
64.01	0.00400086734226725\\
65.01	0.0040008694824184\\
66.01	0.00400087166742125\\
67.01	0.00400087389821861\\
68.01	0.00400087617577295\\
69.01	0.0040008785010673\\
70.01	0.0040008808751055\\
71.01	0.00400088329891232\\
72.01	0.00400088577353426\\
73.01	0.00400088830003978\\
74.01	0.00400089087951998\\
75.01	0.00400089351308901\\
76.01	0.0040008962018844\\
77.01	0.00400089894706777\\
78.01	0.00400090174982534\\
79.01	0.00400090461136811\\
80.01	0.00400090753293288\\
81.01	0.00400091051578234\\
82.01	0.00400091356120622\\
83.01	0.00400091667052092\\
84.01	0.00400091984507096\\
85.01	0.00400092308622907\\
86.01	0.00400092639539736\\
87.01	0.00400092977400694\\
88.01	0.00400093322351967\\
89.01	0.00400093674542806\\
90.01	0.00400094034125598\\
91.01	0.00400094401255949\\
92.01	0.00400094776092788\\
93.01	0.00400095158798371\\
94.01	0.00400095549538389\\
95.01	0.00400095948482035\\
96.01	0.00400096355802075\\
97.01	0.00400096771674907\\
98.01	0.00400097196280687\\
99.01	0.00400097629803352\\
100.01	0.00400098072430735\\
101.01	0.00400098524354638\\
102.01	0.00400098985770925\\
103.01	0.00400099456879582\\
104.01	0.00400099937884846\\
105.01	0.0040010042899525\\
106.01	0.00400100930423727\\
107.01	0.00400101442387746\\
108.01	0.00400101965109331\\
109.01	0.00400102498815241\\
110.01	0.00400103043737005\\
111.01	0.00400103600111069\\
112.01	0.00400104168178841\\
113.01	0.00400104748186875\\
114.01	0.00400105340386914\\
115.01	0.00400105945036039\\
116.01	0.00400106562396758\\
117.01	0.00400107192737163\\
118.01	0.00400107836330978\\
119.01	0.00400108493457729\\
120.01	0.00400109164402875\\
121.01	0.00400109849457905\\
122.01	0.00400110548920472\\
123.01	0.00400111263094542\\
124.01	0.00400111992290493\\
125.01	0.00400112736825299\\
126.01	0.00400113497022629\\
127.01	0.00400114273213006\\
128.01	0.00400115065733947\\
129.01	0.00400115874930135\\
130.01	0.00400116701153531\\
131.01	0.00400117544763559\\
132.01	0.00400118406127273\\
133.01	0.00400119285619467\\
134.01	0.00400120183622904\\
135.01	0.00400121100528446\\
136.01	0.0040012203673522\\
137.01	0.0040012299265086\\
138.01	0.00400123968691588\\
139.01	0.00400124965282467\\
140.01	0.00400125982857562\\
141.01	0.00400127021860153\\
142.01	0.00400128082742915\\
143.01	0.00400129165968113\\
144.01	0.00400130272007827\\
145.01	0.00400131401344138\\
146.01	0.00400132554469354\\
147.01	0.00400133731886239\\
148.01	0.00400134934108201\\
149.01	0.00400136161659541\\
150.01	0.00400137415075716\\
151.01	0.00400138694903507\\
152.01	0.00400140001701337\\
153.01	0.00400141336039447\\
154.01	0.00400142698500212\\
155.01	0.00400144089678368\\
156.01	0.00400145510181263\\
157.01	0.00400146960629152\\
158.01	0.00400148441655476\\
159.01	0.00400149953907127\\
160.01	0.00400151498044732\\
161.01	0.00400153074742948\\
162.01	0.00400154684690811\\
163.01	0.00400156328591966\\
164.01	0.00400158007165034\\
165.01	0.00400159721143896\\
166.01	0.00400161471278061\\
167.01	0.00400163258332972\\
168.01	0.00400165083090352\\
169.01	0.00400166946348571\\
170.01	0.00400168848922954\\
171.01	0.0040017079164623\\
172.01	0.00400172775368818\\
173.01	0.00400174800959244\\
174.01	0.00400176869304568\\
175.01	0.00400178981310693\\
176.01	0.00400181137902857\\
177.01	0.00400183340025986\\
178.01	0.00400185588645174\\
179.01	0.00400187884746085\\
180.01	0.004001902293354\\
181.01	0.00400192623441251\\
182.01	0.00400195068113719\\
183.01	0.00400197564425293\\
184.01	0.00400200113471347\\
185.01	0.00400202716370644\\
186.01	0.00400205374265833\\
187.01	0.00400208088323978\\
188.01	0.00400210859737063\\
189.01	0.00400213689722557\\
190.01	0.00400216579523964\\
191.01	0.00400219530411368\\
192.01	0.00400222543682022\\
193.01	0.00400225620660934\\
194.01	0.00400228762701485\\
195.01	0.00400231971186019\\
196.01	0.00400235247526493\\
197.01	0.00400238593165128\\
198.01	0.00400242009575025\\
199.01	0.00400245498260905\\
200.01	0.00400249060759751\\
201.01	0.00400252698641519\\
202.01	0.0040025641350986\\
203.01	0.00400260207002887\\
204.01	0.00400264080793888\\
205.01	0.00400268036592107\\
206.01	0.00400272076143546\\
207.01	0.00400276201231771\\
208.01	0.0040028041367873\\
209.01	0.00400284715345606\\
210.01	0.00400289108133653\\
211.01	0.0040029359398511\\
212.01	0.00400298174884117\\
213.01	0.00400302852857604\\
214.01	0.00400307629976253\\
215.01	0.00400312508355492\\
216.01	0.00400317490156429\\
217.01	0.00400322577586912\\
218.01	0.00400327772902583\\
219.01	0.00400333078407886\\
220.01	0.0040033849645722\\
221.01	0.00400344029455958\\
222.01	0.00400349679861733\\
223.01	0.0040035545018547\\
224.01	0.00400361342992643\\
225.01	0.00400367360904473\\
226.01	0.00400373506599244\\
227.01	0.00400379782813527\\
228.01	0.00400386192343491\\
229.01	0.00400392738046294\\
230.01	0.0040039942284144\\
231.01	0.00400406249712196\\
232.01	0.00400413221707023\\
233.01	0.00400420341941109\\
234.01	0.00400427613597826\\
235.01	0.00400435039930348\\
236.01	0.00400442624263209\\
237.01	0.00400450369993956\\
238.01	0.00400458280594816\\
239.01	0.00400466359614446\\
240.01	0.00400474610679673\\
241.01	0.00400483037497298\\
242.01	0.00400491643855967\\
243.01	0.00400500433628072\\
244.01	0.00400509410771696\\
245.01	0.00400518579332625\\
246.01	0.00400527943446398\\
247.01	0.00400537507340401\\
248.01	0.00400547275336068\\
249.01	0.00400557251851065\\
250.01	0.00400567441401559\\
251.01	0.00400577848604586\\
252.01	0.0040058847818048\\
253.01	0.00400599334955253\\
254.01	0.00400610423863214\\
255.01	0.00400621749949535\\
256.01	0.00400633318372889\\
257.01	0.00400645134408252\\
258.01	0.00400657203449714\\
259.01	0.0040066953101334\\
260.01	0.00400682122740204\\
261.01	0.00400694984399408\\
262.01	0.00400708121891265\\
263.01	0.00400721541250496\\
264.01	0.0040073524864961\\
265.01	0.00400749250402305\\
266.01	0.00400763552967005\\
267.01	0.00400778162950483\\
268.01	0.00400793087111605\\
269.01	0.0040080833236515\\
270.01	0.00400823905785821\\
271.01	0.00400839814612274\\
272.01	0.00400856066251321\\
273.01	0.00400872668282293\\
274.01	0.00400889628461461\\
275.01	0.00400906954726678\\
276.01	0.00400924655202069\\
277.01	0.00400942738202939\\
278.01	0.00400961212240829\\
279.01	0.00400980086028671\\
280.01	0.00400999368486213\\
281.01	0.00401019068745494\\
282.01	0.00401039196156595\\
283.01	0.00401059760293537\\
284.01	0.00401080770960394\\
285.01	0.00401102238197561\\
286.01	0.00401124172288256\\
287.01	0.00401146583765299\\
288.01	0.00401169483418017\\
289.01	0.00401192882299489\\
290.01	0.00401216791733948\\
291.01	0.00401241223324518\\
292.01	0.00401266188961152\\
293.01	0.00401291700828935\\
294.01	0.00401317771416633\\
295.01	0.00401344413525533\\
296.01	0.00401371640278654\\
297.01	0.00401399465130289\\
298.01	0.0040142790187586\\
299.01	0.00401456964662193\\
300.01	0.00401486667998169\\
301.01	0.00401517026765781\\
302.01	0.00401548056231579\\
303.01	0.00401579772058686\\
304.01	0.00401612190319132\\
305.01	0.00401645327506792\\
306.01	0.00401679200550804\\
307.01	0.00401713826829567\\
308.01	0.00401749224185236\\
309.01	0.00401785410938972\\
310.01	0.00401822405906641\\
311.01	0.0040186022841539\\
312.01	0.00401898898320749\\
313.01	0.00401938436024643\\
314.01	0.00401978862494078\\
315.01	0.00402020199280703\\
316.01	0.0040206246854131\\
317.01	0.00402105693059175\\
318.01	0.00402149896266428\\
319.01	0.00402195102267505\\
320.01	0.00402241335863672\\
321.01	0.00402288622578666\\
322.01	0.00402336988685704\\
323.01	0.00402386461235664\\
324.01	0.00402437068086765\\
325.01	0.00402488837935699\\
326.01	0.00402541800350273\\
327.01	0.00402595985803753\\
328.01	0.00402651425711005\\
329.01	0.00402708152466413\\
330.01	0.00402766199483845\\
331.01	0.0040282560123867\\
332.01	0.00402886393312019\\
333.01	0.00402948612437388\\
334.01	0.00403012296549693\\
335.01	0.00403077484837001\\
336.01	0.00403144217794987\\
337.01	0.00403212537284327\\
338.01	0.00403282486591065\\
339.01	0.00403354110490371\\
340.01	0.00403427455313465\\
341.01	0.00403502569018123\\
342.01	0.00403579501262745\\
343.01	0.00403658303484196\\
344.01	0.00403739028979434\\
345.01	0.00403821732990874\\
346.01	0.00403906472795776\\
347.01	0.00403993307799215\\
348.01	0.0040408229963084\\
349.01	0.00404173512244888\\
350.01	0.00404267012023247\\
351.01	0.00404362867880947\\
352.01	0.00404461151373312\\
353.01	0.00404561936803739\\
354.01	0.00404665301330805\\
355.01	0.00404771325072806\\
356.01	0.00404880091207617\\
357.01	0.00404991686064848\\
358.01	0.00405106199206755\\
359.01	0.00405223723493328\\
360.01	0.00405344355125863\\
361.01	0.00405468193662199\\
362.01	0.00405595341995024\\
363.01	0.00405725906282824\\
364.01	0.00405859995821199\\
365.01	0.00405997722839344\\
366.01	0.00406139202204257\\
367.01	0.00406284551011538\\
368.01	0.00406433888039251\\
369.01	0.00406587333037041\\
370.01	0.00406745005820723\\
371.01	0.00406907025139548\\
372.01	0.00407073507283152\\
373.01	0.00407244564396998\\
374.01	0.00407420302481734\\
375.01	0.00407600819065625\\
376.01	0.00407786200563846\\
377.01	0.00407976519380208\\
378.01	0.00408171830873572\\
379.01	0.00408372170414938\\
380.01	0.00408577550919545\\
381.01	0.00408787961475132\\
382.01	0.00409003368038524\\
383.01	0.00409223717687271\\
384.01	0.00409448948661441\\
385.01	0.00409679010368409\\
386.01	0.00409913926378894\\
387.01	0.00410153796860382\\
388.01	0.00410398729457805\\
389.01	0.00410648834409962\\
390.01	0.00410904224630156\\
391.01	0.00411165015790546\\
392.01	0.00411431326410397\\
393.01	0.00411703277948469\\
394.01	0.00411980994899869\\
395.01	0.00412264604897609\\
396.01	0.00412554238819288\\
397.01	0.00412850030899023\\
398.01	0.0041315211884525\\
399.01	0.00413460643964725\\
400.01	0.00413775751292993\\
401.01	0.00414097589732168\\
402.01	0.00414426312196102\\
403.01	0.00414762075763914\\
404.01	0.00415105041842263\\
405.01	0.00415455376337169\\
406.01	0.00415813249836025\\
407.01	0.00416178837800729\\
408.01	0.00416552320772841\\
409.01	0.0041693388459159\\
410.01	0.00417323720626121\\
411.01	0.00417722026022958\\
412.01	0.00418129003970098\\
413.01	0.00418544863979343\\
414.01	0.00418969822188282\\
415.01	0.00419404101684077\\
416.01	0.00419847932850825\\
417.01	0.00420301553742987\\
418.01	0.00420765210487349\\
419.01	0.00421239157716445\\
420.01	0.00421723659036616\\
421.01	0.00422218987534434\\
422.01	0.00422725426325424\\
423.01	0.00423243269149921\\
424.01	0.00423772821021023\\
425.01	0.00424314398930774\\
426.01	0.0042486833262106\\
427.01	0.00425434965427011\\
428.01	0.00426014655201326\\
429.01	0.00426607775329521\\
430.01	0.00427214715847105\\
431.01	0.00427835884671442\\
432.01	0.00428471708962851\\
433.01	0.00429122636631332\\
434.01	0.00429789138007754\\
435.01	0.00430471707700844\\
436.01	0.00431170866664313\\
437.01	0.00431887164501668\\
438.01	0.00432621182039702\\
439.01	0.00433373534206047\\
440.01	0.00434144873249774\\
441.01	0.00434935892349288\\
442.01	0.00435747329655405\\
443.01	0.00436579972822673\\
444.01	0.00437434664085377\\
445.01	0.00438312305937479\\
446.01	0.00439213867476621\\
447.01	0.00440140391469866\\
448.01	0.00441093002191358\\
449.01	0.00442072914067407\\
450.01	0.00443081441138396\\
451.01	0.00444120007304674\\
452.01	0.00445190157258024\\
453.01	0.00446293567901471\\
454.01	0.00447432059913423\\
455.01	0.00448607608899386\\
456.01	0.00449822355266942\\
457.01	0.00451078611521795\\
458.01	0.00452378865060674\\
459.01	0.00453725773660674\\
460.01	0.00455122149634399\\
461.01	0.00456570926899146\\
462.01	0.00458075102808025\\
463.01	0.00459637643250742\\
464.01	0.00461261334892561\\
465.01	0.00462948561986687\\
466.01	0.00464700976283944\\
467.01	0.00466519016233086\\
468.01	0.004684011890909\\
469.01	0.00470332858609318\\
470.01	0.00472304126243356\\
471.01	0.00474315610270316\\
472.01	0.00476367892864757\\
473.01	0.0047846150933752\\
474.01	0.00480596935236996\\
475.01	0.00482774570927307\\
476.01	0.00484994723197578\\
477.01	0.00487257583379748\\
478.01	0.00489563201320967\\
479.01	0.00491911454605194\\
480.01	0.00494302015020838\\
481.01	0.00496734311061718\\
482.01	0.00499207483835325\\
483.01	0.00501720336700964\\
484.01	0.00504271278708326\\
485.01	0.0050685826294028\\
486.01	0.00509478721452446\\
487.01	0.00512129499896691\\
488.01	0.00514806797106716\\
489.01	0.00517506118207071\\
490.01	0.00520222254669445\\
491.01	0.00522949311615405\\
492.01	0.00525680805092996\\
493.01	0.0052840985061533\\
494.01	0.00531129494555366\\
495.01	0.00533833263621925\\
496.01	0.005365160328844\\
497.01	0.00539175353915022\\
498.01	0.00541822952524772\\
499.01	0.00544489036454156\\
500.01	0.00547172941255036\\
501.01	0.00549871773255214\\
502.01	0.00552582449500173\\
503.01	0.00555301731921735\\
504.01	0.00558026276534436\\
505.01	0.00560752700751449\\
506.01	0.00563477672031228\\
507.01	0.00566198020904772\\
508.01	0.00568910880783696\\
509.01	0.00571613855510147\\
510.01	0.00574305212934537\\
511.01	0.00576984098227466\\
512.01	0.00579650753175162\\
513.01	0.00582306715959326\\
514.01	0.00584954957850267\\
515.01	0.00587599885929561\\
516.01	0.00590247100249808\\
517.01	0.00592902591337369\\
518.01	0.00595568984274648\\
519.01	0.00598245979477857\\
520.01	0.00600933477015154\\
521.01	0.00603631748495947\\
522.01	0.00606341483578303\\
523.01	0.00609063826637363\\
524.01	0.00611800397038684\\
525.01	0.00614553284666085\\
526.01	0.00617325011730924\\
527.01	0.00620118453538543\\
528.01	0.00622936712841125\\
529.01	0.00625782947214204\\
530.01	0.00628660159347257\\
531.01	0.00631570979295472\\
532.01	0.00634517511355617\\
533.01	0.00637501547558893\\
534.01	0.00640524991762986\\
535.01	0.00643589910486549\\
536.01	0.00646698502056707\\
537.01	0.00649853057532576\\
538.01	0.006530559152647\\
539.01	0.00656309412379254\\
540.01	0.00659615838242811\\
541.01	0.0066297739676066\\
542.01	0.0066639618567796\\
543.01	0.00669874200887199\\
544.01	0.00673413370293458\\
545.01	0.00677015607463429\\
546.01	0.00680682837909413\\
547.01	0.00684416988246932\\
548.01	0.00688219971110505\\
549.01	0.00692093672659515\\
550.01	0.00696039943911544\\
551.01	0.00700060596789915\\
552.01	0.00704157405109965\\
553.01	0.00708332109679305\\
554.01	0.00712586425334399\\
555.01	0.0071692204641202\\
556.01	0.00721340647046484\\
557.01	0.00725843876908058\\
558.01	0.00730433356515286\\
559.01	0.0073511067342187\\
560.01	0.00739877379108111\\
561.01	0.00744734986172339\\
562.01	0.00749684965260175\\
563.01	0.00754728741102208\\
564.01	0.00759867687142414\\
565.01	0.00765103118587222\\
566.01	0.00770436284199182\\
567.01	0.00775868357290257\\
568.01	0.007814004259616\\
569.01	0.00787033482401756\\
570.01	0.0079276841103895\\
571.01	0.0079860597537612\\
572.01	0.00804546803401966\\
573.01	0.00810591371553068\\
574.01	0.00816739987270856\\
575.01	0.00822992770213072\\
576.01	0.00829349632141887\\
577.01	0.00835810255497104\\
578.01	0.00842374070703372\\
579.01	0.00849040232326687\\
580.01	0.0085580759427603\\
581.01	0.00862674684338341\\
582.01	0.00869639678438916\\
583.01	0.00876700375140449\\
584.01	0.00883854171047707\\
585.01	0.0089109803798979\\
586.01	0.00898428503117805\\
587.01	0.00905841633390994\\
588.01	0.00913333026343812\\
589.01	0.00920897809551196\\
590.01	0.00928530651866069\\
591.01	0.00936225790326229\\
592.01	0.00943977077658321\\
593.01	0.00951778056592611\\
594.01	0.00959622068800083\\
595.01	0.00967502408241994\\
596.01	0.00975412531166755\\
597.01	0.00983346338244348\\
598.01	0.0099086620184807\\
599.01	0.00997087280416276\\
599.02	0.00997138072163725\\
599.03	0.009971885576258\\
599.04	0.00997238733820211\\
599.05	0.00997288597735272\\
599.06	0.00997338146329604\\
599.07	0.00997387376531845\\
599.08	0.00997436285240349\\
599.09	0.00997484869322892\\
599.1	0.00997533125616361\\
599.11	0.00997581050926455\\
599.12	0.00997628642027372\\
599.13	0.00997675895661497\\
599.14	0.0099772280853909\\
599.15	0.00997769377337961\\
599.16	0.00997815598703157\\
599.17	0.00997861469246631\\
599.18	0.00997906985546915\\
599.19	0.00997952144148792\\
599.2	0.00997996941562957\\
599.21	0.00998041374265684\\
599.22	0.0099808543869848\\
599.23	0.00998129131267744\\
599.24	0.00998172448344418\\
599.25	0.00998215386263636\\
599.26	0.00998257941258353\\
599.27	0.00998300109279825\\
599.28	0.00998341886238981\\
599.29	0.00998383268006024\\
599.3	0.00998424250410032\\
599.31	0.00998464829238543\\
599.32	0.00998505000237151\\
599.33	0.00998544759109087\\
599.34	0.00998584101514799\\
599.35	0.00998623023071532\\
599.36	0.00998661519352895\\
599.37	0.00998699585888436\\
599.38	0.00998737218163197\\
599.39	0.00998774411617281\\
599.4	0.00998811161645403\\
599.41	0.00998847463596443\\
599.42	0.0099888331277299\\
599.43	0.00998918704430885\\
599.44	0.00998953633778757\\
599.45	0.00998988095977558\\
599.46	0.00999022086140088\\
599.47	0.00999055599330522\\
599.48	0.00999088630563925\\
599.49	0.00999121174805768\\
599.5	0.00999153226971435\\
599.51	0.0099918478192573\\
599.52	0.00999215834482374\\
599.53	0.00999246379403503\\
599.54	0.0099927641139915\\
599.55	0.00999305925126738\\
599.56	0.00999334915190553\\
599.57	0.0099936337614122\\
599.58	0.00999391302475173\\
599.59	0.00999418688634118\\
599.6	0.00999445529004489\\
599.61	0.00999471817916904\\
599.62	0.00999497549645613\\
599.63	0.00999522718407935\\
599.64	0.009995473183637\\
599.65	0.00999571343614678\\
599.66	0.00999594788204004\\
599.67	0.00999617646115599\\
599.68	0.0099963991127358\\
599.69	0.00999661577541675\\
599.7	0.00999682638722618\\
599.71	0.00999703088557551\\
599.72	0.00999722920725411\\
599.73	0.00999742128842315\\
599.74	0.00999760706460942\\
599.75	0.00999778647069897\\
599.76	0.00999795944093086\\
599.77	0.0099981259088907\\
599.78	0.0099982858075042\\
599.79	0.00999843906903064\\
599.8	0.00999858562505627\\
599.81	0.00999872540648766\\
599.82	0.00999885834354498\\
599.83	0.00999898436575515\\
599.84	0.00999910340194508\\
599.85	0.00999921538023465\\
599.86	0.00999932022802977\\
599.87	0.00999941787201528\\
599.88	0.00999950823814785\\
599.89	0.00999959125164875\\
599.9	0.00999966683699656\\
599.91	0.00999973491791987\\
599.92	0.0099997954173898\\
599.93	0.00999984825761255\\
599.94	0.00999989336002181\\
599.95	0.00999993064527112\\
599.96	0.00999996003322615\\
599.97	0.00999998144295691\\
599.98	0.00999999479272987\\
599.99	0.01\\
600	0.01\\
};
\addplot [color=blue!50!mycolor7,solid,forget plot]
  table[row sep=crcr]{%
0.01	0.00386858074314488\\
1.01	0.00386858122147336\\
2.01	0.00386858170981677\\
3.01	0.00386858220838564\\
4.01	0.0038685827173946\\
5.01	0.00386858323706303\\
6.01	0.00386858376761477\\
7.01	0.00386858430927869\\
8.01	0.00386858486228814\\
9.01	0.0038685854268815\\
10.01	0.00386858600330241\\
11.01	0.00386858659179941\\
12.01	0.00386858719262637\\
13.01	0.00386858780604245\\
14.01	0.00386858843231234\\
15.01	0.00386858907170616\\
16.01	0.00386858972449997\\
17.01	0.00386859039097541\\
18.01	0.00386859107142044\\
19.01	0.00386859176612868\\
20.01	0.00386859247540022\\
21.01	0.0038685931995414\\
22.01	0.00386859393886488\\
23.01	0.00386859469369029\\
24.01	0.00386859546434348\\
25.01	0.00386859625115768\\
26.01	0.00386859705447302\\
27.01	0.00386859787463661\\
28.01	0.00386859871200307\\
29.01	0.00386859956693467\\
30.01	0.00386860043980092\\
31.01	0.00386860133097955\\
32.01	0.00386860224085585\\
33.01	0.00386860316982367\\
34.01	0.00386860411828495\\
35.01	0.0038686050866502\\
36.01	0.00386860607533868\\
37.01	0.00386860708477837\\
38.01	0.00386860811540639\\
39.01	0.00386860916766897\\
40.01	0.00386861024202195\\
41.01	0.00386861133893077\\
42.01	0.00386861245887059\\
43.01	0.00386861360232681\\
44.01	0.00386861476979494\\
45.01	0.00386861596178115\\
46.01	0.00386861717880201\\
47.01	0.00386861842138536\\
48.01	0.00386861969007007\\
49.01	0.00386862098540641\\
50.01	0.00386862230795627\\
51.01	0.00386862365829355\\
52.01	0.00386862503700421\\
53.01	0.00386862644468644\\
54.01	0.00386862788195155\\
55.01	0.00386862934942353\\
56.01	0.00386863084773952\\
57.01	0.0038686323775501\\
58.01	0.00386863393951985\\
59.01	0.0038686355343274\\
60.01	0.0038686371626656\\
61.01	0.00386863882524223\\
62.01	0.0038686405227797\\
63.01	0.00386864225601603\\
64.01	0.00386864402570479\\
65.01	0.00386864583261561\\
66.01	0.00386864767753425\\
67.01	0.00386864956126334\\
68.01	0.00386865148462237\\
69.01	0.00386865344844835\\
70.01	0.00386865545359587\\
71.01	0.00386865750093775\\
72.01	0.00386865959136522\\
73.01	0.00386866172578866\\
74.01	0.00386866390513758\\
75.01	0.00386866613036128\\
76.01	0.00386866840242901\\
77.01	0.00386867072233096\\
78.01	0.00386867309107809\\
79.01	0.0038686755097028\\
80.01	0.00386867797925949\\
81.01	0.00386868050082491\\
82.01	0.00386868307549844\\
83.01	0.00386868570440344\\
84.01	0.00386868838868656\\
85.01	0.0038686911295191\\
86.01	0.00386869392809693\\
87.01	0.00386869678564163\\
88.01	0.00386869970340065\\
89.01	0.00386870268264786\\
90.01	0.00386870572468433\\
91.01	0.00386870883083892\\
92.01	0.00386871200246827\\
93.01	0.00386871524095812\\
94.01	0.00386871854772387\\
95.01	0.00386872192421076\\
96.01	0.00386872537189478\\
97.01	0.00386872889228353\\
98.01	0.00386873248691645\\
99.01	0.00386873615736574\\
100.01	0.00386873990523717\\
101.01	0.00386874373217061\\
102.01	0.00386874763984076\\
103.01	0.00386875162995819\\
104.01	0.00386875570426959\\
105.01	0.00386875986455903\\
106.01	0.00386876411264855\\
107.01	0.00386876845039894\\
108.01	0.0038687728797106\\
109.01	0.00386877740252452\\
110.01	0.00386878202082291\\
111.01	0.00386878673663024\\
112.01	0.00386879155201419\\
113.01	0.00386879646908631\\
114.01	0.00386880149000345\\
115.01	0.00386880661696813\\
116.01	0.00386881185223024\\
117.01	0.00386881719808707\\
118.01	0.00386882265688533\\
119.01	0.00386882823102166\\
120.01	0.00386883392294382\\
121.01	0.00386883973515184\\
122.01	0.00386884567019892\\
123.01	0.00386885173069293\\
124.01	0.00386885791929734\\
125.01	0.00386886423873228\\
126.01	0.00386887069177644\\
127.01	0.00386887728126729\\
128.01	0.00386888401010324\\
129.01	0.00386889088124455\\
130.01	0.00386889789771479\\
131.01	0.00386890506260176\\
132.01	0.0038689123790597\\
133.01	0.00386891985031\\
134.01	0.00386892747964292\\
135.01	0.00386893527041925\\
136.01	0.00386894322607161\\
137.01	0.00386895135010566\\
138.01	0.00386895964610259\\
139.01	0.00386896811771983\\
140.01	0.00386897676869327\\
141.01	0.0038689856028387\\
142.01	0.0038689946240536\\
143.01	0.00386900383631873\\
144.01	0.00386901324370049\\
145.01	0.00386902285035202\\
146.01	0.00386903266051551\\
147.01	0.00386904267852417\\
148.01	0.00386905290880393\\
149.01	0.00386906335587582\\
150.01	0.0038690740243575\\
151.01	0.00386908491896602\\
152.01	0.00386909604451913\\
153.01	0.00386910740593839\\
154.01	0.00386911900825068\\
155.01	0.003869130856591\\
156.01	0.00386914295620451\\
157.01	0.00386915531244907\\
158.01	0.00386916793079733\\
159.01	0.00386918081684007\\
160.01	0.00386919397628767\\
161.01	0.00386920741497371\\
162.01	0.00386922113885681\\
163.01	0.00386923515402379\\
164.01	0.00386924946669284\\
165.01	0.00386926408321553\\
166.01	0.00386927901008027\\
167.01	0.00386929425391519\\
168.01	0.00386930982149108\\
169.01	0.0038693257197249\\
170.01	0.00386934195568241\\
171.01	0.00386935853658167\\
172.01	0.0038693754697964\\
173.01	0.00386939276285952\\
174.01	0.00386941042346642\\
175.01	0.00386942845947846\\
176.01	0.00386944687892686\\
177.01	0.00386946569001652\\
178.01	0.00386948490112914\\
179.01	0.00386950452082786\\
180.01	0.00386952455786106\\
181.01	0.00386954502116617\\
182.01	0.00386956591987418\\
183.01	0.00386958726331355\\
184.01	0.00386960906101484\\
185.01	0.00386963132271513\\
186.01	0.00386965405836236\\
187.01	0.00386967727812036\\
188.01	0.0038697009923734\\
189.01	0.0038697252117309\\
190.01	0.00386974994703293\\
191.01	0.00386977520935472\\
192.01	0.00386980101001223\\
193.01	0.00386982736056742\\
194.01	0.00386985427283396\\
195.01	0.00386988175888232\\
196.01	0.00386990983104603\\
197.01	0.00386993850192668\\
198.01	0.00386996778440132\\
199.01	0.00386999769162739\\
200.01	0.00387002823704935\\
201.01	0.00387005943440528\\
202.01	0.00387009129773321\\
203.01	0.003870123841378\\
204.01	0.00387015707999833\\
205.01	0.00387019102857358\\
206.01	0.00387022570241097\\
207.01	0.00387026111715313\\
208.01	0.00387029728878577\\
209.01	0.00387033423364494\\
210.01	0.00387037196842557\\
211.01	0.00387041051018915\\
212.01	0.00387044987637228\\
213.01	0.00387049008479498\\
214.01	0.0038705311536697\\
215.01	0.00387057310160979\\
216.01	0.00387061594763929\\
217.01	0.00387065971120169\\
218.01	0.00387070441216977\\
219.01	0.00387075007085552\\
220.01	0.00387079670802004\\
221.01	0.00387084434488412\\
222.01	0.0038708930031381\\
223.01	0.00387094270495389\\
224.01	0.00387099347299485\\
225.01	0.00387104533042792\\
226.01	0.0038710983009351\\
227.01	0.00387115240872488\\
228.01	0.00387120767854547\\
229.01	0.00387126413569633\\
230.01	0.00387132180604194\\
231.01	0.003871380716024\\
232.01	0.0038714408926759\\
233.01	0.00387150236363576\\
234.01	0.00387156515716117\\
235.01	0.00387162930214342\\
236.01	0.0038716948281229\\
237.01	0.00387176176530395\\
238.01	0.00387183014457057\\
239.01	0.00387189999750268\\
240.01	0.00387197135639282\\
241.01	0.00387204425426282\\
242.01	0.00387211872488141\\
243.01	0.00387219480278177\\
244.01	0.00387227252328017\\
245.01	0.00387235192249445\\
246.01	0.00387243303736348\\
247.01	0.0038725159056671\\
248.01	0.00387260056604614\\
249.01	0.00387268705802328\\
250.01	0.00387277542202481\\
251.01	0.00387286569940221\\
252.01	0.00387295793245487\\
253.01	0.00387305216445341\\
254.01	0.00387314843966292\\
255.01	0.00387324680336815\\
256.01	0.0038733473018983\\
257.01	0.00387344998265256\\
258.01	0.00387355489412662\\
259.01	0.00387366208594028\\
260.01	0.00387377160886529\\
261.01	0.00387388351485358\\
262.01	0.00387399785706723\\
263.01	0.00387411468990886\\
264.01	0.00387423406905245\\
265.01	0.00387435605147568\\
266.01	0.00387448069549264\\
267.01	0.00387460806078781\\
268.01	0.00387473820845088\\
269.01	0.00387487120101261\\
270.01	0.00387500710248137\\
271.01	0.00387514597838114\\
272.01	0.00387528789579067\\
273.01	0.00387543292338315\\
274.01	0.00387558113146752\\
275.01	0.00387573259203067\\
276.01	0.00387588737878133\\
277.01	0.0038760455671947\\
278.01	0.00387620723455863\\
279.01	0.00387637246002108\\
280.01	0.0038765413246388\\
281.01	0.0038767139114279\\
282.01	0.00387689030541528\\
283.01	0.00387707059369199\\
284.01	0.00387725486546811\\
285.01	0.00387744321212886\\
286.01	0.0038776357272933\\
287.01	0.00387783250687326\\
288.01	0.00387803364913558\\
289.01	0.00387823925476498\\
290.01	0.00387844942692965\\
291.01	0.00387866427134811\\
292.01	0.0038788838963589\\
293.01	0.00387910841299116\\
294.01	0.00387933793503815\\
295.01	0.00387957257913316\\
296.01	0.00387981246482685\\
297.01	0.00388005771466718\\
298.01	0.00388030845428254\\
299.01	0.00388056481246622\\
300.01	0.00388082692126358\\
301.01	0.00388109491606271\\
302.01	0.00388136893568684\\
303.01	0.00388164912248991\\
304.01	0.00388193562245431\\
305.01	0.00388222858529291\\
306.01	0.00388252816455287\\
307.01	0.00388283451772257\\
308.01	0.00388314780634237\\
309.01	0.00388346819611796\\
310.01	0.00388379585703746\\
311.01	0.00388413096349123\\
312.01	0.00388447369439597\\
313.01	0.00388482423332162\\
314.01	0.003885182768622\\
315.01	0.00388554949356952\\
316.01	0.0038859246064932\\
317.01	0.00388630831092017\\
318.01	0.00388670081572171\\
319.01	0.00388710233526245\\
320.01	0.00388751308955389\\
321.01	0.00388793330441166\\
322.01	0.00388836321161637\\
323.01	0.00388880304907896\\
324.01	0.0038892530610095\\
325.01	0.00388971349808938\\
326.01	0.0038901846176479\\
327.01	0.00389066668384264\\
328.01	0.00389115996784191\\
329.01	0.00389166474801204\\
330.01	0.00389218131010673\\
331.01	0.00389270994745958\\
332.01	0.00389325096117865\\
333.01	0.00389380466034321\\
334.01	0.00389437136220222\\
335.01	0.00389495139237302\\
336.01	0.00389554508504091\\
337.01	0.00389615278315711\\
338.01	0.00389677483863729\\
339.01	0.00389741161255426\\
340.01	0.00389806347532944\\
341.01	0.00389873080691741\\
342.01	0.00389941399698415\\
343.01	0.0039001134450754\\
344.01	0.00390082956077403\\
345.01	0.00390156276384374\\
346.01	0.00390231348435555\\
347.01	0.00390308216279489\\
348.01	0.00390386925014351\\
349.01	0.00390467520793431\\
350.01	0.00390550050827339\\
351.01	0.00390634563382347\\
352.01	0.00390721107774351\\
353.01	0.0039080973435775\\
354.01	0.00390900494508627\\
355.01	0.00390993440601353\\
356.01	0.00391088625977783\\
357.01	0.00391186104908239\\
358.01	0.00391285932543314\\
359.01	0.00391388164855558\\
360.01	0.00391492858570175\\
361.01	0.00391600071083822\\
362.01	0.00391709860370817\\
363.01	0.00391822284876306\\
364.01	0.0039193740339603\\
365.01	0.00392055274943226\\
366.01	0.00392175958603289\\
367.01	0.00392299513378343\\
368.01	0.00392425998024432\\
369.01	0.00392555470886186\\
370.01	0.0039268798973519\\
371.01	0.00392823611621257\\
372.01	0.00392962392748478\\
373.01	0.00393104388391531\\
374.01	0.00393249652871842\\
375.01	0.00393398239617307\\
376.01	0.00393550201333963\\
377.01	0.00393705590321207\\
378.01	0.0039386445896398\\
379.01	0.0039402686043306\\
380.01	0.00394192849615328\\
381.01	0.00394362484274391\\
382.01	0.0039453582640135\\
383.01	0.00394712943642735\\
384.01	0.00394893910572075\\
385.01	0.00395078809372049\\
386.01	0.00395267728833708\\
387.01	0.00395460761274427\\
388.01	0.00395658001442778\\
389.01	0.00395859546571165\\
390.01	0.00396065496457514\\
391.01	0.0039627595355054\\
392.01	0.0039649102303875\\
393.01	0.00396710812943429\\
394.01	0.00396935434215751\\
395.01	0.00397165000838327\\
396.01	0.00397399629931416\\
397.01	0.0039763944186407\\
398.01	0.00397884560370544\\
399.01	0.00398135112672141\\
400.01	0.00398391229605037\\
401.01	0.00398653045754237\\
402.01	0.00398920699594149\\
403.01	0.00399194333636183\\
404.01	0.00399474094583709\\
405.01	0.00399760133494935\\
406.01	0.00400052605954149\\
407.01	0.00400351672251856\\
408.01	0.00400657497574333\\
409.01	0.0040097025220328\\
410.01	0.0040129011172612\\
411.01	0.00401617257257576\\
412.01	0.00401951875673466\\
413.01	0.00402294159857138\\
414.01	0.00402644308959658\\
415.01	0.00403002528674379\\
416.01	0.00403369031526886\\
417.01	0.00403744037181229\\
418.01	0.00404127772763399\\
419.01	0.00404520473203011\\
420.01	0.00404922381594372\\
421.01	0.00405333749577814\\
422.01	0.00405754837742431\\
423.01	0.00406185916051362\\
424.01	0.00406627264290478\\
425.01	0.00407079172541597\\
426.01	0.00407541941681033\\
427.01	0.0040801588390427\\
428.01	0.00408501323277298\\
429.01	0.00408998596314955\\
430.01	0.00409508052586133\\
431.01	0.00410030055345542\\
432.01	0.00410564982190721\\
433.01	0.00411113225742545\\
434.01	0.00411675194346493\\
435.01	0.00412251312790421\\
436.01	0.00412842023033567\\
437.01	0.00413447784938988\\
438.01	0.00414069076999928\\
439.01	0.00414706397047156\\
440.01	0.00415360262921131\\
441.01	0.00416031213088211\\
442.01	0.00416719807175235\\
443.01	0.00417426626390027\\
444.01	0.00418152273788162\\
445.01	0.00418897374337146\\
446.01	0.00419662574718474\\
447.01	0.00420448542795937\\
448.01	0.00421255966664272\\
449.01	0.00422085553176207\\
450.01	0.00422938025828778\\
451.01	0.0042381412187074\\
452.01	0.00424714588475078\\
453.01	0.00425640177802644\\
454.01	0.00426591640771269\\
455.01	0.0042756971934038\\
456.01	0.00428575137133055\\
457.01	0.00429608588254402\\
458.01	0.004306707242431\\
459.01	0.00431762139232646\\
460.01	0.00432883353633232\\
461.01	0.00434034797020055\\
462.01	0.00435216791495188\\
463.01	0.00436429537673048\\
464.01	0.00437673106759208\\
465.01	0.00438947444140124\\
466.01	0.00440252392748826\\
467.01	0.00441587748602856\\
468.01	0.00442953366884713\\
469.01	0.00444349452215706\\
470.01	0.00445776866461103\\
471.01	0.00447236656830442\\
472.01	0.00448729958656169\\
473.01	0.0045025800846319\\
474.01	0.00451822158966254\\
475.01	0.0045342389610175\\
476.01	0.00455064858089547\\
477.01	0.00456746856324833\\
478.01	0.00458471897573415\\
479.01	0.00460242206412984\\
480.01	0.00462060245982847\\
481.01	0.00463928734521334\\
482.01	0.00465850675671346\\
483.01	0.00467829418219433\\
484.01	0.00469868696563419\\
485.01	0.00471972658808292\\
486.01	0.00474145882382038\\
487.01	0.00476393367448086\\
488.01	0.00478720493526354\\
489.01	0.00481132917779667\\
490.01	0.00483636383485271\\
491.01	0.00486236422364897\\
492.01	0.0048893823801086\\
493.01	0.00491746667181199\\
494.01	0.00494665806710051\\
495.01	0.00497698424535896\\
496.01	0.00500845104106382\\
497.01	0.00504103014834045\\
498.01	0.00507454796308061\\
499.01	0.00510865853264725\\
500.01	0.00514332840972172\\
501.01	0.00517853973412254\\
502.01	0.00521426915877894\\
503.01	0.00525048700255988\\
504.01	0.00528715633502748\\
505.01	0.00532423201429295\\
506.01	0.00536165971570869\\
507.01	0.00539937501559937\\
508.01	0.00543730263400269\\
509.01	0.00547535599460236\\
510.01	0.00551343733817398\\
511.01	0.00555143873823\\
512.01	0.00558924452729373\\
513.01	0.00562673586838863\\
514.01	0.00566379852542793\\
515.01	0.00570033533525394\\
516.01	0.00573628551453981\\
517.01	0.00577173599208902\\
518.01	0.00580717753185216\\
519.01	0.005842642439381\\
520.01	0.00587808362130181\\
521.01	0.00591345447457746\\
522.01	0.00594871041434022\\
523.01	0.0059838108718764\\
524.01	0.00601872188111489\\
525.01	0.00605341932537118\\
526.01	0.00608789247573951\\
527.01	0.0061221474258837\\
528.01	0.00615621015398192\\
529.01	0.00619012854416304\\
530.01	0.00622397222772657\\
531.01	0.00625782842397304\\
532.01	0.00629178452997938\\
533.01	0.00632587055864795\\
534.01	0.00636009328503807\\
535.01	0.00639446549593649\\
536.01	0.00642900656512509\\
537.01	0.00646374254740046\\
538.01	0.00649870585890901\\
539.01	0.00653393442389336\\
540.01	0.00656947018386905\\
541.01	0.00660535691740503\\
542.01	0.00664163743538566\\
543.01	0.00667835044132781\\
544.01	0.00671552776347704\\
545.01	0.00675319498961828\\
546.01	0.00679137757058428\\
547.01	0.00683010261535099\\
548.01	0.00686939836821782\\
549.01	0.00690929355948017\\
550.01	0.00694981670841965\\
551.01	0.0069909954519336\\
552.01	0.00703285600150236\\
553.01	0.0070754228525436\\
554.01	0.00711871886371888\\
555.01	0.0071627657603584\\
556.01	0.00720758481787483\\
557.01	0.00725319703853558\\
558.01	0.00729962296373621\\
559.01	0.00734688249195777\\
560.01	0.0073949947477996\\
561.01	0.00744397801310743\\
562.01	0.00749384972275357\\
563.01	0.00754462651135841\\
564.01	0.00759632427494854\\
565.01	0.00764895819286281\\
566.01	0.00770254266877708\\
567.01	0.00775709122588741\\
568.01	0.00781261640385305\\
569.01	0.00786912966267725\\
570.01	0.00792664128800444\\
571.01	0.00798516028951031\\
572.01	0.00804469428287664\\
573.01	0.0081052493475248\\
574.01	0.0081668298578033\\
575.01	0.00822943829317042\\
576.01	0.00829307503476616\\
577.01	0.00835773814990506\\
578.01	0.00842342316335163\\
579.01	0.00849012281477226\\
580.01	0.00855782680318764\\
581.01	0.00862652152130265\\
582.01	0.0086961897849244\\
583.01	0.0087668105646243\\
584.01	0.00883835872788432\\
585.01	0.00891080480112841\\
586.01	0.00898411476341371\\
587.01	0.0090582498869396\\
588.01	0.00913316664376106\\
589.01	0.00920881670326823\\
590.01	0.00928514705127742\\
591.01	0.00936210026927741\\
592.01	0.00943961502205398\\
593.01	0.00951762681432733\\
594.01	0.00959606909286117\\
595.01	0.0096748747904041\\
596.01	0.00975397843263973\\
597.01	0.00983331896004518\\
598.01	0.0099086620184803\\
599.01	0.00997087280416276\\
599.02	0.00997138072163725\\
599.03	0.00997188557625799\\
599.04	0.00997238733820211\\
599.05	0.00997288597735272\\
599.06	0.00997338146329604\\
599.07	0.00997387376531845\\
599.08	0.00997436285240349\\
599.09	0.00997484869322892\\
599.1	0.00997533125616361\\
599.11	0.00997581050926455\\
599.12	0.00997628642027372\\
599.13	0.00997675895661498\\
599.14	0.0099772280853909\\
599.15	0.00997769377337961\\
599.16	0.00997815598703157\\
599.17	0.00997861469246631\\
599.18	0.00997906985546915\\
599.19	0.00997952144148792\\
599.2	0.00997996941562957\\
599.21	0.00998041374265684\\
599.22	0.0099808543869848\\
599.23	0.00998129131267744\\
599.24	0.00998172448344418\\
599.25	0.00998215386263636\\
599.26	0.00998257941258354\\
599.27	0.00998300109279825\\
599.28	0.00998341886238981\\
599.29	0.00998383268006024\\
599.3	0.00998424250410032\\
599.31	0.00998464829238543\\
599.32	0.00998505000237151\\
599.33	0.00998544759109087\\
599.34	0.00998584101514799\\
599.35	0.00998623023071532\\
599.36	0.00998661519352895\\
599.37	0.00998699585888436\\
599.38	0.00998737218163197\\
599.39	0.00998774411617281\\
599.4	0.00998811161645403\\
599.41	0.00998847463596444\\
599.42	0.0099888331277299\\
599.43	0.00998918704430885\\
599.44	0.00998953633778757\\
599.45	0.00998988095977558\\
599.46	0.00999022086140088\\
599.47	0.00999055599330522\\
599.48	0.00999088630563925\\
599.49	0.00999121174805768\\
599.5	0.00999153226971435\\
599.51	0.0099918478192573\\
599.52	0.00999215834482374\\
599.53	0.00999246379403503\\
599.54	0.0099927641139915\\
599.55	0.00999305925126738\\
599.56	0.00999334915190553\\
599.57	0.0099936337614122\\
599.58	0.00999391302475173\\
599.59	0.00999418688634118\\
599.6	0.00999445529004489\\
599.61	0.00999471817916904\\
599.62	0.00999497549645613\\
599.63	0.00999522718407934\\
599.64	0.009995473183637\\
599.65	0.00999571343614678\\
599.66	0.00999594788204004\\
599.67	0.00999617646115598\\
599.68	0.0099963991127358\\
599.69	0.00999661577541675\\
599.7	0.00999682638722618\\
599.71	0.00999703088557551\\
599.72	0.00999722920725411\\
599.73	0.00999742128842315\\
599.74	0.00999760706460942\\
599.75	0.00999778647069897\\
599.76	0.00999795944093086\\
599.77	0.0099981259088907\\
599.78	0.0099982858075042\\
599.79	0.00999843906903064\\
599.8	0.00999858562505627\\
599.81	0.00999872540648767\\
599.82	0.00999885834354498\\
599.83	0.00999898436575515\\
599.84	0.00999910340194508\\
599.85	0.00999921538023465\\
599.86	0.00999932022802977\\
599.87	0.00999941787201528\\
599.88	0.00999950823814785\\
599.89	0.00999959125164875\\
599.9	0.00999966683699656\\
599.91	0.00999973491791987\\
599.92	0.0099997954173898\\
599.93	0.00999984825761255\\
599.94	0.00999989336002181\\
599.95	0.00999993064527112\\
599.96	0.00999996003322615\\
599.97	0.00999998144295691\\
599.98	0.00999999479272987\\
599.99	0.01\\
600	0.01\\
};
\addplot [color=blue!40!mycolor9,solid,forget plot]
  table[row sep=crcr]{%
0.01	0.00356935369650614\\
1.01	0.0035693541789565\\
2.01	0.00356935467156345\\
3.01	0.00356935517454188\\
4.01	0.003569355688111\\
5.01	0.00356935621249488\\
6.01	0.00356935674792234\\
7.01	0.00356935729462701\\
8.01	0.00356935785284748\\
9.01	0.00356935842282738\\
10.01	0.00356935900481559\\
11.01	0.00356935959906622\\
12.01	0.00356936020583877\\
13.01	0.00356936082539841\\
14.01	0.00356936145801577\\
15.01	0.00356936210396751\\
16.01	0.00356936276353592\\
17.01	0.00356936343700942\\
18.01	0.00356936412468239\\
19.01	0.00356936482685569\\
20.01	0.00356936554383659\\
21.01	0.00356936627593862\\
22.01	0.00356936702348254\\
23.01	0.00356936778679537\\
24.01	0.00356936856621158\\
25.01	0.00356936936207233\\
26.01	0.00356937017472626\\
27.01	0.00356937100452949\\
28.01	0.00356937185184578\\
29.01	0.00356937271704624\\
30.01	0.00356937360051053\\
31.01	0.00356937450262593\\
32.01	0.0035693754237882\\
33.01	0.00356937636440138\\
34.01	0.0035693773248783\\
35.01	0.00356937830564054\\
36.01	0.00356937930711877\\
37.01	0.00356938032975258\\
38.01	0.00356938137399115\\
39.01	0.00356938244029334\\
40.01	0.00356938352912765\\
41.01	0.00356938464097266\\
42.01	0.00356938577631713\\
43.01	0.00356938693566027\\
44.01	0.00356938811951205\\
45.01	0.00356938932839322\\
46.01	0.00356939056283574\\
47.01	0.00356939182338285\\
48.01	0.00356939311058957\\
49.01	0.0035693944250226\\
50.01	0.00356939576726092\\
51.01	0.00356939713789588\\
52.01	0.0035693985375314\\
53.01	0.0035693999667844\\
54.01	0.00356940142628485\\
55.01	0.00356940291667644\\
56.01	0.00356940443861649\\
57.01	0.00356940599277643\\
58.01	0.00356940757984218\\
59.01	0.00356940920051417\\
60.01	0.00356941085550791\\
61.01	0.00356941254555425\\
62.01	0.00356941427139984\\
63.01	0.00356941603380698\\
64.01	0.00356941783355474\\
65.01	0.00356941967143858\\
66.01	0.00356942154827111\\
67.01	0.00356942346488234\\
68.01	0.00356942542212033\\
69.01	0.00356942742085103\\
70.01	0.00356942946195912\\
71.01	0.0035694315463482\\
72.01	0.0035694336749415\\
73.01	0.00356943584868165\\
74.01	0.00356943806853179\\
75.01	0.00356944033547574\\
76.01	0.00356944265051855\\
77.01	0.00356944501468662\\
78.01	0.00356944742902841\\
79.01	0.00356944989461534\\
80.01	0.00356945241254138\\
81.01	0.00356945498392423\\
82.01	0.00356945760990569\\
83.01	0.00356946029165188\\
84.01	0.00356946303035427\\
85.01	0.00356946582722982\\
86.01	0.00356946868352175\\
87.01	0.0035694716005001\\
88.01	0.00356947457946204\\
89.01	0.00356947762173299\\
90.01	0.00356948072866653\\
91.01	0.0035694839016457\\
92.01	0.00356948714208323\\
93.01	0.00356949045142237\\
94.01	0.00356949383113733\\
95.01	0.0035694972827342\\
96.01	0.00356950080775159\\
97.01	0.00356950440776119\\
98.01	0.00356950808436861\\
99.01	0.00356951183921431\\
100.01	0.00356951567397391\\
101.01	0.00356951959035923\\
102.01	0.00356952359011911\\
103.01	0.00356952767504026\\
104.01	0.00356953184694772\\
105.01	0.00356953610770608\\
106.01	0.00356954045922046\\
107.01	0.00356954490343674\\
108.01	0.00356954944234325\\
109.01	0.0035695540779709\\
110.01	0.00356955881239492\\
111.01	0.00356956364773511\\
112.01	0.00356956858615737\\
113.01	0.00356957362987451\\
114.01	0.00356957878114701\\
115.01	0.00356958404228451\\
116.01	0.00356958941564639\\
117.01	0.00356959490364335\\
118.01	0.00356960050873852\\
119.01	0.00356960623344788\\
120.01	0.00356961208034247\\
121.01	0.00356961805204864\\
122.01	0.00356962415125012\\
123.01	0.00356963038068857\\
124.01	0.00356963674316516\\
125.01	0.00356964324154208\\
126.01	0.00356964987874328\\
127.01	0.00356965665775656\\
128.01	0.00356966358163448\\
129.01	0.00356967065349578\\
130.01	0.00356967787652725\\
131.01	0.00356968525398468\\
132.01	0.00356969278919485\\
133.01	0.0035697004855569\\
134.01	0.00356970834654367\\
135.01	0.00356971637570373\\
136.01	0.00356972457666296\\
137.01	0.00356973295312622\\
138.01	0.0035697415088789\\
139.01	0.00356975024778874\\
140.01	0.00356975917380811\\
141.01	0.00356976829097526\\
142.01	0.00356977760341658\\
143.01	0.00356978711534861\\
144.01	0.00356979683107955\\
145.01	0.00356980675501176\\
146.01	0.00356981689164382\\
147.01	0.00356982724557248\\
148.01	0.00356983782149475\\
149.01	0.00356984862421054\\
150.01	0.00356985965862446\\
151.01	0.00356987092974825\\
152.01	0.00356988244270362\\
153.01	0.00356989420272411\\
154.01	0.00356990621515779\\
155.01	0.00356991848546973\\
156.01	0.00356993101924495\\
157.01	0.00356994382219063\\
158.01	0.00356995690013892\\
159.01	0.00356997025904997\\
160.01	0.00356998390501463\\
161.01	0.00356999784425741\\
162.01	0.00357001208313922\\
163.01	0.00357002662816094\\
164.01	0.00357004148596586\\
165.01	0.00357005666334344\\
166.01	0.00357007216723233\\
167.01	0.00357008800472368\\
168.01	0.00357010418306469\\
169.01	0.00357012070966183\\
170.01	0.00357013759208466\\
171.01	0.00357015483806954\\
172.01	0.00357017245552338\\
173.01	0.003570190452527\\
174.01	0.00357020883733943\\
175.01	0.00357022761840199\\
176.01	0.00357024680434229\\
177.01	0.00357026640397823\\
178.01	0.00357028642632264\\
179.01	0.0035703068805874\\
180.01	0.00357032777618782\\
181.01	0.00357034912274765\\
182.01	0.00357037093010336\\
183.01	0.00357039320830949\\
184.01	0.00357041596764265\\
185.01	0.00357043921860747\\
186.01	0.00357046297194154\\
187.01	0.00357048723862011\\
188.01	0.00357051202986224\\
189.01	0.00357053735713599\\
190.01	0.00357056323216381\\
191.01	0.00357058966692904\\
192.01	0.00357061667368105\\
193.01	0.00357064426494189\\
194.01	0.00357067245351184\\
195.01	0.00357070125247674\\
196.01	0.00357073067521334\\
197.01	0.00357076073539701\\
198.01	0.0035707914470078\\
199.01	0.00357082282433764\\
200.01	0.00357085488199765\\
201.01	0.00357088763492534\\
202.01	0.0035709210983919\\
203.01	0.00357095528801027\\
204.01	0.00357099021974247\\
205.01	0.00357102590990781\\
206.01	0.00357106237519164\\
207.01	0.00357109963265299\\
208.01	0.0035711376997334\\
209.01	0.00357117659426605\\
210.01	0.0035712163344844\\
211.01	0.00357125693903206\\
212.01	0.00357129842697129\\
213.01	0.00357134081779364\\
214.01	0.00357138413142935\\
215.01	0.00357142838825765\\
216.01	0.0035714736091173\\
217.01	0.00357151981531727\\
218.01	0.0035715670286473\\
219.01	0.00357161527138948\\
220.01	0.00357166456632982\\
221.01	0.00357171493676962\\
222.01	0.00357176640653782\\
223.01	0.00357181900000292\\
224.01	0.00357187274208633\\
225.01	0.00357192765827472\\
226.01	0.00357198377463296\\
227.01	0.00357204111781907\\
228.01	0.00357209971509662\\
229.01	0.00357215959434987\\
230.01	0.00357222078409804\\
231.01	0.00357228331351042\\
232.01	0.00357234721242168\\
233.01	0.00357241251134764\\
234.01	0.00357247924150125\\
235.01	0.0035725474348093\\
236.01	0.00357261712392903\\
237.01	0.00357268834226578\\
238.01	0.00357276112399071\\
239.01	0.00357283550405872\\
240.01	0.00357291151822756\\
241.01	0.00357298920307673\\
242.01	0.00357306859602675\\
243.01	0.00357314973536013\\
244.01	0.00357323266024112\\
245.01	0.00357331741073738\\
246.01	0.0035734040278412\\
247.01	0.00357349255349197\\
248.01	0.00357358303059889\\
249.01	0.00357367550306435\\
250.01	0.00357377001580761\\
251.01	0.00357386661478948\\
252.01	0.00357396534703736\\
253.01	0.00357406626067107\\
254.01	0.00357416940492954\\
255.01	0.00357427483019705\\
256.01	0.0035743825880318\\
257.01	0.00357449273119416\\
258.01	0.00357460531367555\\
259.01	0.00357472039072872\\
260.01	0.00357483801889809\\
261.01	0.00357495825605129\\
262.01	0.00357508116141174\\
263.01	0.00357520679559105\\
264.01	0.0035753352206235\\
265.01	0.0035754665000003\\
266.01	0.00357560069870552\\
267.01	0.00357573788325241\\
268.01	0.00357587812172097\\
269.01	0.00357602148379602\\
270.01	0.00357616804080665\\
271.01	0.0035763178657669\\
272.01	0.00357647103341641\\
273.01	0.00357662762026308\\
274.01	0.00357678770462662\\
275.01	0.0035769513666829\\
276.01	0.00357711868850955\\
277.01	0.00357728975413245\\
278.01	0.0035774646495737\\
279.01	0.00357764346290101\\
280.01	0.0035778262842775\\
281.01	0.00357801320601334\\
282.01	0.0035782043226186\\
283.01	0.00357839973085708\\
284.01	0.00357859952980157\\
285.01	0.00357880382089096\\
286.01	0.00357901270798767\\
287.01	0.00357922629743751\\
288.01	0.00357944469813003\\
289.01	0.00357966802156129\\
290.01	0.003579896381897\\
291.01	0.00358012989603849\\
292.01	0.00358036868368844\\
293.01	0.00358061286742002\\
294.01	0.00358086257274595\\
295.01	0.0035811179281903\\
296.01	0.00358137906536138\\
297.01	0.00358164611902638\\
298.01	0.00358191922718731\\
299.01	0.00358219853115907\\
300.01	0.00358248417564937\\
301.01	0.00358277630883916\\
302.01	0.00358307508246659\\
303.01	0.0035833806519107\\
304.01	0.0035836931762786\\
305.01	0.00358401281849324\\
306.01	0.00358433974538321\\
307.01	0.00358467412777473\\
308.01	0.00358501614058483\\
309.01	0.00358536596291655\\
310.01	0.00358572377815553\\
311.01	0.00358608977406931\\
312.01	0.00358646414290688\\
313.01	0.0035868470815013\\
314.01	0.00358723879137322\\
315.01	0.00358763947883652\\
316.01	0.00358804935510501\\
317.01	0.00358846863640165\\
318.01	0.00358889754406856\\
319.01	0.00358933630467896\\
320.01	0.00358978515015054\\
321.01	0.00359024431786005\\
322.01	0.00359071405076002\\
323.01	0.00359119459749599\\
324.01	0.00359168621252533\\
325.01	0.00359218915623769\\
326.01	0.00359270369507596\\
327.01	0.00359323010165832\\
328.01	0.00359376865490199\\
329.01	0.00359431964014711\\
330.01	0.00359488334928176\\
331.01	0.0035954600808672\\
332.01	0.00359605014026447\\
333.01	0.00359665383976118\\
334.01	0.00359727149869771\\
335.01	0.00359790344359505\\
336.01	0.00359855000828187\\
337.01	0.00359921153402205\\
338.01	0.00359988836964127\\
339.01	0.00360058087165412\\
340.01	0.00360128940439082\\
341.01	0.00360201434012253\\
342.01	0.00360275605918707\\
343.01	0.00360351495011342\\
344.01	0.00360429140974572\\
345.01	0.00360508584336625\\
346.01	0.00360589866481749\\
347.01	0.00360673029662416\\
348.01	0.00360758117011349\\
349.01	0.0036084517255366\\
350.01	0.00360934241218824\\
351.01	0.00361025368852749\\
352.01	0.00361118602229971\\
353.01	0.0036121398906598\\
354.01	0.0036131157802978\\
355.01	0.0036141141875686\\
356.01	0.00361513561862792\\
357.01	0.00361618058957471\\
358.01	0.00361724962660467\\
359.01	0.00361834326617612\\
360.01	0.00361946205519366\\
361.01	0.0036206065512127\\
362.01	0.00362177732267134\\
363.01	0.00362297494915492\\
364.01	0.00362420002170184\\
365.01	0.00362545314315821\\
366.01	0.00362673492859153\\
367.01	0.00362804600577428\\
368.01	0.00362938701574894\\
369.01	0.00363075861348704\\
370.01	0.00363216146865467\\
371.01	0.00363359626649709\\
372.01	0.00363506370885205\\
373.01	0.0036365645152997\\
374.01	0.00363809942445012\\
375.01	0.00363966919536403\\
376.01	0.00364127460908717\\
377.01	0.00364291647026863\\
378.01	0.00364459560880954\\
379.01	0.0036463128814699\\
380.01	0.00364806917333388\\
381.01	0.00364986539901338\\
382.01	0.00365170250345209\\
383.01	0.00365358146220757\\
384.01	0.00365550328113723\\
385.01	0.00365746899556441\\
386.01	0.00365947966931873\\
387.01	0.00366153639456435\\
388.01	0.00366364029248849\\
389.01	0.00366579251415143\\
390.01	0.00366799424136459\\
391.01	0.00367024668759425\\
392.01	0.00367255109889375\\
393.01	0.00367490875486343\\
394.01	0.00367732096963948\\
395.01	0.00367978909291161\\
396.01	0.00368231451097226\\
397.01	0.00368489864779506\\
398.01	0.00368754296614511\\
399.01	0.00369024896872178\\
400.01	0.00369301819933206\\
401.01	0.00369585224409821\\
402.01	0.003698752732698\\
403.01	0.003701721339638\\
404.01	0.00370475978556105\\
405.01	0.00370786983858665\\
406.01	0.0037110533156855\\
407.01	0.00371431208408747\\
408.01	0.00371764806272219\\
409.01	0.00372106322369291\\
410.01	0.00372455959378086\\
411.01	0.00372813925598164\\
412.01	0.00373180435106996\\
413.01	0.00373555707919232\\
414.01	0.00373939970148496\\
415.01	0.00374333454171481\\
416.01	0.00374736398794043\\
417.01	0.00375149049418813\\
418.01	0.00375571658214024\\
419.01	0.00376004484283006\\
420.01	0.00376447793833572\\
421.01	0.00376901860346752\\
422.01	0.00377366964744026\\
423.01	0.00377843395551854\\
424.01	0.00378331449062667\\
425.01	0.00378831429490753\\
426.01	0.00379343649121612\\
427.01	0.0037986842845302\\
428.01	0.00380406096325764\\
429.01	0.00380956990041652\\
430.01	0.00381521455466308\\
431.01	0.00382099847113472\\
432.01	0.00382692528207579\\
433.01	0.00383299870720497\\
434.01	0.00383922255378029\\
435.01	0.00384560071631176\\
436.01	0.00385213717586352\\
437.01	0.00385883599888315\\
438.01	0.00386570133548747\\
439.01	0.00387273741712474\\
440.01	0.00387994855352876\\
441.01	0.00388733912886977\\
442.01	0.00389491359700014\\
443.01	0.00390267647568868\\
444.01	0.0039106323397283\\
445.01	0.00391878581280282\\
446.01	0.00392714155799803\\
447.01	0.00393570426684846\\
448.01	0.00394447864682365\\
449.01	0.003953469407182\\
450.01	0.00396268124315192\\
451.01	0.00397211881845402\\
452.01	0.00398178674624092\\
453.01	0.00399168956863305\\
454.01	0.0040018317351434\\
455.01	0.00401221758044108\\
456.01	0.00402285130209053\\
457.01	0.00403373693912723\\
458.01	0.00404487835258123\\
459.01	0.00405627920932505\\
460.01	0.00406794297087019\\
461.01	0.00407987288890881\\
462.01	0.00409207200940272\\
463.01	0.00410454318670068\\
464.01	0.00411728910826407\\
465.01	0.00413031232870952\\
466.01	0.00414361530839268\\
467.01	0.00415720044562973\\
468.01	0.00417107008107845\\
469.01	0.0041852264260813\\
470.01	0.0041996713398888\\
471.01	0.00421440617148017\\
472.01	0.00422943173661071\\
473.01	0.00424474833547754\\
474.01	0.00426035582131055\\
475.01	0.00427625374672212\\
476.01	0.00429244162620634\\
477.01	0.00430891936952346\\
478.01	0.00432568796411437\\
479.01	0.00434275051881251\\
480.01	0.00436011382294717\\
481.01	0.00437778925709713\\
482.01	0.00439578057004691\\
483.01	0.00441408517387566\\
484.01	0.00443270191090969\\
485.01	0.00445163272657016\\
486.01	0.00447088499173598\\
487.01	0.00449047488246039\\
488.01	0.00451043223607692\\
489.01	0.0045308074601669\\
490.01	0.00455168128593993\\
491.01	0.00457315422736754\\
492.01	0.00459526008907832\\
493.01	0.0046180086043931\\
494.01	0.00464140788574584\\
495.01	0.00466546479504304\\
496.01	0.00469018583076554\\
497.01	0.00471557883003723\\
498.01	0.00474165656460672\\
499.01	0.00476844518523685\\
500.01	0.00479597870150315\\
501.01	0.00482429399915098\\
502.01	0.00485343100202368\\
503.01	0.0048834328864565\\
504.01	0.00491434627623896\\
505.01	0.00494622140426403\\
506.01	0.00497911220262801\\
507.01	0.00501307619044987\\
508.01	0.00504817404249849\\
509.01	0.00508446872486107\\
510.01	0.00512202399523604\\
511.01	0.00516090197633048\\
512.01	0.00520115939160515\\
513.01	0.0052428418864781\\
514.01	0.0052859756268194\\
515.01	0.00533055504474222\\
516.01	0.00537652515399289\\
517.01	0.00542367506389353\\
518.01	0.00547140676547196\\
519.01	0.0055195941845584\\
520.01	0.00556818235367103\\
521.01	0.00561710597037564\\
522.01	0.00566628677968869\\
523.01	0.00571562983441296\\
524.01	0.00576501816609789\\
525.01	0.00581430985838814\\
526.01	0.00586334494528416\\
527.01	0.00591195183244157\\
528.01	0.00595995539606151\\
529.01	0.00600718933154067\\
530.01	0.00605351576115472\\
531.01	0.00609885554544969\\
532.01	0.00614347792173343\\
533.01	0.00618796876807136\\
534.01	0.00623229035747514\\
535.01	0.00627638553022832\\
536.01	0.00632020584718589\\
537.01	0.00636371544694091\\
538.01	0.00640689538419013\\
539.01	0.00644974820779094\\
540.01	0.00649230232892437\\
541.01	0.00653461539186144\\
542.01	0.00657677531915586\\
543.01	0.00661889690504216\\
544.01	0.00666110859425139\\
545.01	0.00670348372223335\\
546.01	0.00674604040021065\\
547.01	0.00678880258411917\\
548.01	0.00683180313316688\\
549.01	0.00687508335988836\\
550.01	0.00691869185821967\\
551.01	0.00696268244849074\\
552.01	0.00700711107313386\\
553.01	0.0070520317686205\\
554.01	0.0070974922108247\\
555.01	0.00714353011714465\\
556.01	0.00719017612575834\\
557.01	0.0072374611684107\\
558.01	0.00728541715086486\\
559.01	0.00733407609564801\\
560.01	0.00738346924975821\\
561.01	0.00743362623227398\\
562.01	0.00748457434096044\\
563.01	0.00753633822330531\\
564.01	0.00758894009010531\\
565.01	0.00764240049063996\\
566.01	0.00769673903354819\\
567.01	0.00775197432695984\\
568.01	0.00780812368455662\\
569.01	0.00786520288823177\\
570.01	0.00792322603603263\\
571.01	0.00798220548225692\\
572.01	0.00804215184843037\\
573.01	0.00810307405076856\\
574.01	0.00816497926209387\\
575.01	0.0082278727525132\\
576.01	0.00829175766618825\\
577.01	0.00835663479827209\\
578.01	0.00842250237713441\\
579.01	0.00848935584384751\\
580.01	0.00855718761791611\\
581.01	0.00862598683882235\\
582.01	0.0086957390795814\\
583.01	0.00876642604196961\\
584.01	0.00883802525422727\\
585.01	0.00891050978858627\\
586.01	0.00898384801195749\\
587.01	0.00905800338572644\\
588.01	0.00913293433529357\\
589.01	0.00920859421638682\\
590.01	0.00928493141303536\\
591.01	0.0093618896103455\\
592.01	0.00943940829294075\\
593.01	0.00951742352873278\\
594.01	0.00959586911059195\\
595.01	0.0096746781463672\\
596.01	0.00975378521137988\\
597.01	0.00983312920843281\\
598.01	0.00990866201846347\\
599.01	0.00997087280416267\\
599.02	0.00997138072163717\\
599.03	0.00997188557625792\\
599.04	0.00997238733820205\\
599.05	0.00997288597735266\\
599.06	0.00997338146329598\\
599.07	0.00997387376531839\\
599.08	0.00997436285240344\\
599.09	0.00997484869322887\\
599.1	0.00997533125616357\\
599.11	0.00997581050926451\\
599.12	0.00997628642027369\\
599.13	0.00997675895661494\\
599.14	0.00997722808539087\\
599.15	0.00997769377337959\\
599.16	0.00997815598703155\\
599.17	0.00997861469246629\\
599.18	0.00997906985546913\\
599.19	0.0099795214414879\\
599.2	0.00997996941562956\\
599.21	0.00998041374265682\\
599.22	0.00998085438698478\\
599.23	0.00998129131267743\\
599.24	0.00998172448344417\\
599.25	0.00998215386263635\\
599.26	0.00998257941258352\\
599.27	0.00998300109279824\\
599.28	0.0099834188623898\\
599.29	0.00998383268006024\\
599.3	0.00998424250410031\\
599.31	0.00998464829238542\\
599.32	0.0099850500023715\\
599.33	0.00998544759109086\\
599.34	0.00998584101514799\\
599.35	0.00998623023071531\\
599.36	0.00998661519352895\\
599.37	0.00998699585888435\\
599.38	0.00998737218163196\\
599.39	0.00998774411617281\\
599.4	0.00998811161645403\\
599.41	0.00998847463596443\\
599.42	0.0099888331277299\\
599.43	0.00998918704430885\\
599.44	0.00998953633778757\\
599.45	0.00998988095977558\\
599.46	0.00999022086140088\\
599.47	0.00999055599330522\\
599.48	0.00999088630563925\\
599.49	0.00999121174805768\\
599.5	0.00999153226971434\\
599.51	0.0099918478192573\\
599.52	0.00999215834482374\\
599.53	0.00999246379403503\\
599.54	0.0099927641139915\\
599.55	0.00999305925126738\\
599.56	0.00999334915190552\\
599.57	0.0099936337614122\\
599.58	0.00999391302475173\\
599.59	0.00999418688634118\\
599.6	0.00999445529004489\\
599.61	0.00999471817916904\\
599.62	0.00999497549645613\\
599.63	0.00999522718407935\\
599.64	0.009995473183637\\
599.65	0.00999571343614678\\
599.66	0.00999594788204004\\
599.67	0.00999617646115599\\
599.68	0.0099963991127358\\
599.69	0.00999661577541675\\
599.7	0.00999682638722618\\
599.71	0.00999703088557551\\
599.72	0.00999722920725411\\
599.73	0.00999742128842315\\
599.74	0.00999760706460942\\
599.75	0.00999778647069897\\
599.76	0.00999795944093086\\
599.77	0.0099981259088907\\
599.78	0.0099982858075042\\
599.79	0.00999843906903064\\
599.8	0.00999858562505627\\
599.81	0.00999872540648767\\
599.82	0.00999885834354498\\
599.83	0.00999898436575515\\
599.84	0.00999910340194508\\
599.85	0.00999921538023465\\
599.86	0.00999932022802977\\
599.87	0.00999941787201528\\
599.88	0.00999950823814785\\
599.89	0.00999959125164875\\
599.9	0.00999966683699656\\
599.91	0.00999973491791987\\
599.92	0.0099997954173898\\
599.93	0.00999984825761255\\
599.94	0.00999989336002181\\
599.95	0.00999993064527112\\
599.96	0.00999996003322615\\
599.97	0.00999998144295691\\
599.98	0.00999999479272987\\
599.99	0.01\\
600	0.01\\
};
\addplot [color=blue!75!mycolor7,solid,forget plot]
  table[row sep=crcr]{%
0.01	0.00270247467503103\\
1.01	0.00270247533429627\\
2.01	0.00270247600752455\\
3.01	0.00270247669501311\\
4.01	0.00270247739706596\\
5.01	0.00270247811399338\\
6.01	0.00270247884611237\\
7.01	0.00270247959374664\\
8.01	0.00270248035722696\\
9.01	0.00270248113689134\\
10.01	0.00270248193308464\\
11.01	0.00270248274615945\\
12.01	0.00270248357647591\\
13.01	0.00270248442440169\\
14.01	0.00270248529031265\\
15.01	0.0027024861745924\\
16.01	0.00270248707763303\\
17.01	0.00270248799983511\\
18.01	0.00270248894160763\\
19.01	0.00270248990336849\\
20.01	0.00270249088554445\\
21.01	0.00270249188857176\\
22.01	0.00270249291289565\\
23.01	0.00270249395897115\\
24.01	0.00270249502726301\\
25.01	0.0027024961182462\\
26.01	0.00270249723240559\\
27.01	0.00270249837023682\\
28.01	0.00270249953224588\\
29.01	0.00270250071895002\\
30.01	0.00270250193087732\\
31.01	0.00270250316856732\\
32.01	0.0027025044325714\\
33.01	0.00270250572345257\\
34.01	0.0027025070417861\\
35.01	0.00270250838815958\\
36.01	0.00270250976317325\\
37.01	0.00270251116744064\\
38.01	0.00270251260158821\\
39.01	0.00270251406625591\\
40.01	0.00270251556209768\\
41.01	0.00270251708978153\\
42.01	0.00270251864998984\\
43.01	0.0027025202434199\\
44.01	0.00270252187078381\\
45.01	0.00270252353280925\\
46.01	0.00270252523023966\\
47.01	0.00270252696383441\\
48.01	0.00270252873436937\\
49.01	0.00270253054263721\\
50.01	0.00270253238944777\\
51.01	0.00270253427562822\\
52.01	0.00270253620202393\\
53.01	0.00270253816949839\\
54.01	0.00270254017893395\\
55.01	0.00270254223123183\\
56.01	0.00270254432731296\\
57.01	0.00270254646811821\\
58.01	0.00270254865460874\\
59.01	0.00270255088776668\\
60.01	0.00270255316859545\\
61.01	0.00270255549812008\\
62.01	0.00270255787738798\\
63.01	0.00270256030746932\\
64.01	0.00270256278945737\\
65.01	0.00270256532446929\\
66.01	0.00270256791364634\\
67.01	0.00270257055815473\\
68.01	0.00270257325918585\\
69.01	0.00270257601795711\\
70.01	0.00270257883571216\\
71.01	0.00270258171372205\\
72.01	0.00270258465328506\\
73.01	0.00270258765572813\\
74.01	0.00270259072240689\\
75.01	0.00270259385470629\\
76.01	0.00270259705404162\\
77.01	0.00270260032185902\\
78.01	0.00270260365963614\\
79.01	0.00270260706888252\\
80.01	0.00270261055114097\\
81.01	0.00270261410798775\\
82.01	0.00270261774103349\\
83.01	0.00270262145192399\\
84.01	0.00270262524234094\\
85.01	0.00270262911400267\\
86.01	0.00270263306866504\\
87.01	0.00270263710812213\\
88.01	0.00270264123420753\\
89.01	0.00270264544879445\\
90.01	0.00270264975379733\\
91.01	0.0027026541511723\\
92.01	0.00270265864291827\\
93.01	0.0027026632310779\\
94.01	0.00270266791773846\\
95.01	0.00270267270503299\\
96.01	0.00270267759514119\\
97.01	0.00270268259029019\\
98.01	0.00270268769275631\\
99.01	0.00270269290486544\\
100.01	0.00270269822899444\\
101.01	0.00270270366757241\\
102.01	0.00270270922308156\\
103.01	0.00270271489805848\\
104.01	0.00270272069509551\\
105.01	0.00270272661684193\\
106.01	0.00270273266600483\\
107.01	0.00270273884535099\\
108.01	0.00270274515770798\\
109.01	0.00270275160596534\\
110.01	0.00270275819307605\\
111.01	0.00270276492205816\\
112.01	0.00270277179599596\\
113.01	0.00270277881804147\\
114.01	0.00270278599141625\\
115.01	0.00270279331941242\\
116.01	0.00270280080539494\\
117.01	0.00270280845280267\\
118.01	0.00270281626514995\\
119.01	0.00270282424602916\\
120.01	0.00270283239911125\\
121.01	0.00270284072814832\\
122.01	0.00270284923697524\\
123.01	0.00270285792951139\\
124.01	0.00270286680976269\\
125.01	0.00270287588182341\\
126.01	0.00270288514987828\\
127.01	0.00270289461820446\\
128.01	0.00270290429117342\\
129.01	0.00270291417325338\\
130.01	0.00270292426901132\\
131.01	0.00270293458311537\\
132.01	0.00270294512033654\\
133.01	0.00270295588555169\\
134.01	0.0027029668837457\\
135.01	0.00270297812001373\\
136.01	0.0027029895995636\\
137.01	0.00270300132771877\\
138.01	0.00270301330992052\\
139.01	0.0027030255517311\\
140.01	0.00270303805883587\\
141.01	0.00270305083704628\\
142.01	0.00270306389230295\\
143.01	0.00270307723067827\\
144.01	0.00270309085837955\\
145.01	0.00270310478175213\\
146.01	0.00270311900728221\\
147.01	0.00270313354160019\\
148.01	0.00270314839148419\\
149.01	0.0027031635638628\\
150.01	0.00270317906581906\\
151.01	0.00270319490459373\\
152.01	0.00270321108758879\\
153.01	0.00270322762237106\\
154.01	0.00270324451667618\\
155.01	0.00270326177841231\\
156.01	0.0027032794156638\\
157.01	0.00270329743669538\\
158.01	0.00270331584995673\\
159.01	0.00270333466408561\\
160.01	0.00270335388791323\\
161.01	0.00270337353046794\\
162.01	0.00270339360097988\\
163.01	0.00270341410888571\\
164.01	0.00270343506383317\\
165.01	0.00270345647568597\\
166.01	0.00270347835452865\\
167.01	0.00270350071067162\\
168.01	0.00270352355465635\\
169.01	0.00270354689726052\\
170.01	0.00270357074950371\\
171.01	0.00270359512265228\\
172.01	0.00270362002822581\\
173.01	0.00270364547800236\\
174.01	0.00270367148402458\\
175.01	0.00270369805860558\\
176.01	0.00270372521433532\\
177.01	0.00270375296408672\\
178.01	0.00270378132102239\\
179.01	0.0027038102986011\\
180.01	0.00270383991058468\\
181.01	0.00270387017104476\\
182.01	0.00270390109437012\\
183.01	0.00270393269527364\\
184.01	0.00270396498880028\\
185.01	0.00270399799033418\\
186.01	0.0027040317156065\\
187.01	0.00270406618070387\\
188.01	0.00270410140207589\\
189.01	0.00270413739654395\\
190.01	0.00270417418130976\\
191.01	0.00270421177396382\\
192.01	0.00270425019249451\\
193.01	0.00270428945529744\\
194.01	0.00270432958118458\\
195.01	0.00270437058939391\\
196.01	0.00270441249959952\\
197.01	0.00270445533192122\\
198.01	0.0027044991069352\\
199.01	0.00270454384568428\\
200.01	0.00270458956968893\\
201.01	0.00270463630095816\\
202.01	0.0027046840620009\\
203.01	0.00270473287583753\\
204.01	0.00270478276601178\\
205.01	0.00270483375660273\\
206.01	0.00270488587223702\\
207.01	0.00270493913810216\\
208.01	0.00270499357995892\\
209.01	0.00270504922415492\\
210.01	0.0027051060976382\\
211.01	0.00270516422797092\\
212.01	0.00270522364334395\\
213.01	0.00270528437259129\\
214.01	0.002705346445205\\
215.01	0.00270540989135065\\
216.01	0.00270547474188272\\
217.01	0.00270554102836068\\
218.01	0.00270560878306568\\
219.01	0.00270567803901699\\
220.01	0.00270574882998924\\
221.01	0.00270582119053019\\
222.01	0.00270589515597848\\
223.01	0.00270597076248241\\
224.01	0.00270604804701824\\
225.01	0.00270612704741006\\
226.01	0.00270620780234948\\
227.01	0.00270629035141534\\
228.01	0.00270637473509498\\
229.01	0.0027064609948053\\
230.01	0.00270654917291427\\
231.01	0.00270663931276335\\
232.01	0.00270673145869008\\
233.01	0.0027068256560514\\
234.01	0.00270692195124765\\
235.01	0.00270702039174651\\
236.01	0.00270712102610837\\
237.01	0.00270722390401164\\
238.01	0.00270732907627895\\
239.01	0.00270743659490411\\
240.01	0.0027075465130791\\
241.01	0.00270765888522241\\
242.01	0.00270777376700783\\
243.01	0.00270789121539326\\
244.01	0.00270801128865152\\
245.01	0.00270813404640037\\
246.01	0.00270825954963483\\
247.01	0.00270838786075876\\
248.01	0.00270851904361798\\
249.01	0.00270865316353411\\
250.01	0.00270879028733922\\
251.01	0.00270893048341113\\
252.01	0.00270907382170958\\
253.01	0.00270922037381315\\
254.01	0.00270937021295713\\
255.01	0.00270952341407281\\
256.01	0.00270968005382667\\
257.01	0.00270984021066105\\
258.01	0.00271000396483605\\
259.01	0.00271017139847174\\
260.01	0.00271034259559182\\
261.01	0.00271051764216831\\
262.01	0.00271069662616669\\
263.01	0.00271087963759313\\
264.01	0.00271106676854173\\
265.01	0.00271125811324362\\
266.01	0.00271145376811681\\
267.01	0.0027116538318173\\
268.01	0.00271185840529134\\
269.01	0.00271206759182897\\
270.01	0.00271228149711895\\
271.01	0.00271250022930422\\
272.01	0.00271272389903977\\
273.01	0.0027129526195508\\
274.01	0.00271318650669294\\
275.01	0.00271342567901324\\
276.01	0.00271367025781309\\
277.01	0.00271392036721234\\
278.01	0.00271417613421482\\
279.01	0.00271443768877547\\
280.01	0.00271470516386876\\
281.01	0.00271497869555914\\
282.01	0.0027152584230726\\
283.01	0.00271554448886996\\
284.01	0.00271583703872232\\
285.01	0.00271613622178692\\
286.01	0.00271644219068644\\
287.01	0.00271675510158866\\
288.01	0.00271707511428881\\
289.01	0.00271740239229304\\
290.01	0.00271773710290442\\
291.01	0.00271807941731017\\
292.01	0.00271842951067162\\
293.01	0.00271878756221533\\
294.01	0.00271915375532732\\
295.01	0.0027195282776481\\
296.01	0.00271991132117057\\
297.01	0.00272030308234024\\
298.01	0.00272070376215712\\
299.01	0.00272111356628033\\
300.01	0.00272153270513456\\
301.01	0.00272196139401958\\
302.01	0.00272239985322109\\
303.01	0.00272284830812539\\
304.01	0.00272330698933557\\
305.01	0.00272377613279026\\
306.01	0.00272425597988599\\
307.01	0.00272474677760093\\
308.01	0.00272524877862255\\
309.01	0.00272576224147726\\
310.01	0.00272628743066379\\
311.01	0.00272682461678848\\
312.01	0.00272737407670492\\
313.01	0.00272793609365547\\
314.01	0.00272851095741694\\
315.01	0.0027290989644492\\
316.01	0.00272970041804706\\
317.01	0.00273031562849621\\
318.01	0.00273094491323209\\
319.01	0.00273158859700301\\
320.01	0.00273224701203695\\
321.01	0.0027329204982125\\
322.01	0.00273360940323363\\
323.01	0.00273431408280884\\
324.01	0.00273503490083495\\
325.01	0.00273577222958521\\
326.01	0.00273652644990233\\
327.01	0.00273729795139636\\
328.01	0.00273808713264778\\
329.01	0.00273889440141571\\
330.01	0.0027397201748519\\
331.01	0.00274056487972089\\
332.01	0.00274142895262519\\
333.01	0.00274231284023753\\
334.01	0.00274321699953993\\
335.01	0.00274414189806882\\
336.01	0.00274508801416797\\
337.01	0.00274605583724873\\
338.01	0.00274704586805798\\
339.01	0.00274805861895508\\
340.01	0.00274909461419602\\
341.01	0.00275015439022826\\
342.01	0.00275123849599369\\
343.01	0.00275234749324232\\
344.01	0.00275348195685582\\
345.01	0.00275464247518248\\
346.01	0.00275582965038318\\
347.01	0.00275704409878906\\
348.01	0.00275828645127285\\
349.01	0.00275955735363188\\
350.01	0.00276085746698575\\
351.01	0.00276218746818835\\
352.01	0.00276354805025458\\
353.01	0.00276493992280309\\
354.01	0.00276636381251567\\
355.01	0.00276782046361405\\
356.01	0.00276931063835421\\
357.01	0.00277083511754045\\
358.01	0.00277239470105849\\
359.01	0.00277399020842996\\
360.01	0.0027756224793874\\
361.01	0.00277729237447215\\
362.01	0.00277900077565511\\
363.01	0.00278074858698127\\
364.01	0.0027825367352384\\
365.01	0.00278436617065053\\
366.01	0.00278623786759656\\
367.01	0.00278815282535375\\
368.01	0.00279011206886554\\
369.01	0.00279211664953301\\
370.01	0.00279416764602901\\
371.01	0.002796266165132\\
372.01	0.00279841334257751\\
373.01	0.00280061034392369\\
374.01	0.00280285836542661\\
375.01	0.00280515863492006\\
376.01	0.00280751241269528\\
377.01	0.00280992099237441\\
378.01	0.00281238570177332\\
379.01	0.00281490790375073\\
380.01	0.00281748899704232\\
381.01	0.00282013041708387\\
382.01	0.00282283363683311\\
383.01	0.00282560016760509\\
384.01	0.00282843155994579\\
385.01	0.00283132940456879\\
386.01	0.00283429533337751\\
387.01	0.00283733102056727\\
388.01	0.00284043818377587\\
389.01	0.00284361858526741\\
390.01	0.00284687403314821\\
391.01	0.00285020638261679\\
392.01	0.00285361753724704\\
393.01	0.00285710945030708\\
394.01	0.00286068412611297\\
395.01	0.00286434362142014\\
396.01	0.00286809004685051\\
397.01	0.00287192556835872\\
398.01	0.00287585240873706\\
399.01	0.00287987284915896\\
400.01	0.00288398923076455\\
401.01	0.0028882039562859\\
402.01	0.00289251949171476\\
403.01	0.00289693836801229\\
404.01	0.00290146318286257\\
405.01	0.00290609660246954\\
406.01	0.00291084136339893\\
407.01	0.00291570027446451\\
408.01	0.00292067621866084\\
409.01	0.00292577215514149\\
410.01	0.00293099112124419\\
411.01	0.00293633623456342\\
412.01	0.0029418106950693\\
413.01	0.00294741778727514\\
414.01	0.00295316088245261\\
415.01	0.00295904344089451\\
416.01	0.0029650690142259\\
417.01	0.00297124124776349\\
418.01	0.00297756388292232\\
419.01	0.0029840407596702\\
420.01	0.00299067581902952\\
421.01	0.00299747310562577\\
422.01	0.0030044367702815\\
423.01	0.00301157107265605\\
424.01	0.00301888038392892\\
425.01	0.00302636918952615\\
426.01	0.00303404209188808\\
427.01	0.00304190381327682\\
428.01	0.00304995919862031\\
429.01	0.00305821321839271\\
430.01	0.00306667097152531\\
431.01	0.00307533768834659\\
432.01	0.0030842187335459\\
433.01	0.00309331960915568\\
434.01	0.00310264595754579\\
435.01	0.00311220356442228\\
436.01	0.00312199836182031\\
437.01	0.00313203643107907\\
438.01	0.00314232400578267\\
439.01	0.00315286747464798\\
440.01	0.00316367338433383\\
441.01	0.00317474844213898\\
442.01	0.00318609951854902\\
443.01	0.00319773364957773\\
444.01	0.00320965803883519\\
445.01	0.00322188005923436\\
446.01	0.00323440725422265\\
447.01	0.00324724733839032\\
448.01	0.00326040819726636\\
449.01	0.0032738978860566\\
450.01	0.00328772462700695\\
451.01	0.00330189680498473\\
452.01	0.00331642296075368\\
453.01	0.00333131178126965\\
454.01	0.00334657208613542\\
455.01	0.00336221280911142\\
456.01	0.00337824297328074\\
457.01	0.00339467165808292\\
458.01	0.00341150795596004\\
459.01	0.0034287609157695\\
460.01	0.00344643946939566\\
461.01	0.00346455233711562\\
462.01	0.00348310790622741\\
463.01	0.00350211407624967\\
464.01	0.00352157806272528\\
465.01	0.00354150615061395\\
466.01	0.00356190338840462\\
467.01	0.00358277321850817\\
468.01	0.00360411705825751\\
469.01	0.00362593384889105\\
470.01	0.0036482193487216\\
471.01	0.00367096518644888\\
472.01	0.00369415773920015\\
473.01	0.00371777671173217\\
474.01	0.00374179332594245\\
475.01	0.00376616800608949\\
476.01	0.0037908474080284\\
477.01	0.00381576058788272\\
478.01	0.0038408140226982\\
479.01	0.00386588508688511\\
480.01	0.00389081376234303\\
481.01	0.00391560395586256\\
482.01	0.00394074112993927\\
483.01	0.00396620506208416\\
484.01	0.00399191528349777\\
485.01	0.00401776461077528\\
486.01	0.00404361142247187\\
487.01	0.00406926963312608\\
488.01	0.00409449564862648\\
489.01	0.00411897135558789\\
490.01	0.00414228188906971\\
491.01	0.00416507254868226\\
492.01	0.00418846898558721\\
493.01	0.00421248742566058\\
494.01	0.00423714381453705\\
495.01	0.00426245500524932\\
496.01	0.00428843889910915\\
497.01	0.00431511453216295\\
498.01	0.00434250203415325\\
499.01	0.00437062228851151\\
500.01	0.00439949642593\\
501.01	0.0044291454734143\\
502.01	0.00445958989481118\\
503.01	0.00449084888929291\\
504.01	0.00452293933632795\\
505.01	0.00455587427112269\\
506.01	0.00458966202379704\\
507.01	0.00462430698135259\\
508.01	0.00465980928779375\\
509.01	0.00469616417384544\\
510.01	0.0047333613215893\\
511.01	0.00477138434497626\\
512.01	0.0048102105109371\\
513.01	0.00484981089078331\\
514.01	0.00489015122506367\\
515.01	0.00493119391866482\\
516.01	0.00497290177298885\\
517.01	0.00501524473887385\\
518.01	0.00505821584207151\\
519.01	0.00510183294455671\\
520.01	0.00514613671083838\\
521.01	0.00519120211992206\\
522.01	0.00523715530414039\\
523.01	0.00528419723132036\\
524.01	0.0053326260424256\\
525.01	0.00538267931871982\\
526.01	0.00543447963477701\\
527.01	0.00548809594440868\\
528.01	0.00554356281551956\\
529.01	0.00560087933159319\\
530.01	0.0056599898495372\\
531.01	0.00572075662910118\\
532.01	0.00578268190428666\\
533.01	0.00584498022083387\\
534.01	0.00590749574688542\\
535.01	0.00597007948607768\\
536.01	0.00603256371658051\\
537.01	0.00609476253691515\\
538.01	0.00615647411242743\\
539.01	0.00621748539801371\\
540.01	0.00627758036634497\\
541.01	0.00633655347722414\\
542.01	0.0063942310381018\\
543.01	0.00645050344564633\\
544.01	0.00650549634327943\\
545.01	0.00656008411933888\\
546.01	0.00661435765076117\\
547.01	0.00666825438150137\\
548.01	0.00672172896272959\\
549.01	0.00677475834979131\\
550.01	0.00682734663152718\\
551.01	0.0068795317334653\\
552.01	0.00693139203259288\\
553.01	0.00698304779253108\\
554.01	0.00703465559071412\\
555.01	0.00708638066172824\\
556.01	0.00713829638688321\\
557.01	0.00719043503387839\\
558.01	0.00724283853598048\\
559.01	0.00729555771687146\\
560.01	0.00734865055334202\\
561.01	0.00740218005541401\\
562.01	0.0074562106142838\\
563.01	0.00751080248597193\\
564.01	0.00756600583168564\\
565.01	0.00762185735902811\\
566.01	0.00767838795473285\\
567.01	0.00773562904201511\\
568.01	0.00779361189922099\\
569.01	0.00785236643307945\\
570.01	0.00791191989642814\\
571.01	0.00797229587071889\\
572.01	0.00803351381482855\\
573.01	0.00809558940754691\\
574.01	0.00815853564761196\\
575.01	0.00822236373310811\\
576.01	0.00828708286240763\\
577.01	0.00835269977667169\\
578.01	0.00841921839411719\\
579.01	0.00848663956920399\\
580.01	0.00855496097497914\\
581.01	0.00862417706125201\\
582.01	0.00869427898579362\\
583.01	0.00876525439463777\\
584.01	0.00883708706023637\\
585.01	0.00890975649722161\\
586.01	0.00898323760124607\\
587.01	0.00905750032875147\\
588.01	0.00913250943662461\\
589.01	0.00920822430118101\\
590.01	0.00928459884902855\\
591.01	0.00936158165898825\\
592.01	0.00943911632123402\\
593.01	0.00951714214254039\\
594.01	0.00959559528261798\\
595.01	0.00967441041276702\\
596.01	0.00975352299930713\\
597.01	0.00983287233211095\\
598.01	0.00990866201774166\\
599.01	0.0099708728041569\\
599.02	0.00997138072163179\\
599.03	0.0099718855762529\\
599.04	0.00997238733819737\\
599.05	0.0099728859773483\\
599.06	0.00997338146329193\\
599.07	0.00997387376531463\\
599.08	0.00997436285239995\\
599.09	0.00997484869322563\\
599.1	0.00997533125616056\\
599.11	0.00997581050926173\\
599.12	0.00997628642027111\\
599.13	0.00997675895661256\\
599.14	0.00997722808538867\\
599.15	0.00997769377337755\\
599.16	0.00997815598702968\\
599.17	0.00997861469246456\\
599.18	0.00997906985546755\\
599.19	0.00997952144148644\\
599.2	0.00997996941562822\\
599.21	0.0099804137426556\\
599.22	0.00998085438698366\\
599.23	0.0099812913126764\\
599.24	0.00998172448344324\\
599.25	0.0099821538626355\\
599.26	0.00998257941258275\\
599.27	0.00998300109279754\\
599.28	0.00998341886238916\\
599.29	0.00998383268005965\\
599.3	0.00998424250409978\\
599.31	0.00998464829238495\\
599.32	0.00998505000237108\\
599.33	0.00998544759109047\\
599.34	0.00998584101514764\\
599.35	0.009986230230715\\
599.36	0.00998661519352867\\
599.37	0.0099869958588841\\
599.38	0.00998737218163174\\
599.39	0.00998774411617261\\
599.4	0.00998811161645385\\
599.41	0.00998847463596427\\
599.42	0.00998883312772976\\
599.43	0.00998918704430873\\
599.44	0.00998953633778746\\
599.45	0.00998988095977548\\
599.46	0.0099902208614008\\
599.47	0.00999055599330515\\
599.48	0.00999088630563919\\
599.49	0.00999121174805762\\
599.5	0.0099915322697143\\
599.51	0.00999184781925726\\
599.52	0.00999215834482371\\
599.53	0.009992463794035\\
599.54	0.00999276411399148\\
599.55	0.00999305925126736\\
599.56	0.00999334915190551\\
599.57	0.00999363376141218\\
599.58	0.00999391302475172\\
599.59	0.00999418688634117\\
599.6	0.00999445529004488\\
599.61	0.00999471817916904\\
599.62	0.00999497549645612\\
599.63	0.00999522718407934\\
599.64	0.00999547318363699\\
599.65	0.00999571343614677\\
599.66	0.00999594788204004\\
599.67	0.00999617646115598\\
599.68	0.0099963991127358\\
599.69	0.00999661577541675\\
599.7	0.00999682638722618\\
599.71	0.00999703088557551\\
599.72	0.00999722920725411\\
599.73	0.00999742128842315\\
599.74	0.00999760706460942\\
599.75	0.00999778647069897\\
599.76	0.00999795944093086\\
599.77	0.0099981259088907\\
599.78	0.0099982858075042\\
599.79	0.00999843906903064\\
599.8	0.00999858562505627\\
599.81	0.00999872540648767\\
599.82	0.00999885834354498\\
599.83	0.00999898436575515\\
599.84	0.00999910340194508\\
599.85	0.00999921538023465\\
599.86	0.00999932022802977\\
599.87	0.00999941787201528\\
599.88	0.00999950823814785\\
599.89	0.00999959125164875\\
599.9	0.00999966683699656\\
599.91	0.00999973491791987\\
599.92	0.0099997954173898\\
599.93	0.00999984825761255\\
599.94	0.00999989336002181\\
599.95	0.00999993064527112\\
599.96	0.00999996003322615\\
599.97	0.00999998144295691\\
599.98	0.00999999479272987\\
599.99	0.01\\
600	0.01\\
};
\addplot [color=blue!80!mycolor9,solid,forget plot]
  table[row sep=crcr]{%
0.01	0.000977852700539981\\
1.01	0.000977853634422134\\
2.01	0.000977854588141196\\
3.01	0.000977855562121314\\
4.01	0.000977856556795567\\
5.01	0.000977857572606509\\
6.01	0.000977858610006203\\
7.01	0.000977859669456425\\
8.01	0.000977860751428916\\
9.01	0.000977861856405533\\
10.01	0.000977862984878675\\
11.01	0.000977864137351257\\
12.01	0.000977865314337065\\
13.01	0.000977866516360955\\
14.01	0.000977867743959124\\
15.01	0.000977868997679321\\
16.01	0.000977870278081203\\
17.01	0.00097787158573636\\
18.01	0.000977872921228896\\
19.01	0.00097787428515541\\
20.01	0.000977875678125496\\
21.01	0.000977877100761824\\
22.01	0.000977878553700559\\
23.01	0.000977880037591639\\
24.01	0.000977881553099115\\
25.01	0.000977883100901272\\
26.01	0.000977884681691195\\
27.01	0.000977886296176872\\
28.01	0.000977887945081642\\
29.01	0.000977889629144538\\
30.01	0.000977891349120574\\
31.01	0.000977893105781148\\
32.01	0.000977894899914282\\
33.01	0.000977896732325137\\
34.01	0.000977898603836395\\
35.01	0.000977900515288496\\
36.01	0.000977902467540072\\
37.01	0.000977904461468414\\
38.01	0.00097790649796984\\
39.01	0.000977908577960193\\
40.01	0.000977910702375052\\
41.01	0.000977912872170437\\
42.01	0.00097791508832306\\
43.01	0.00097791735183083\\
44.01	0.000977919663713349\\
45.01	0.000977922025012408\\
46.01	0.000977924436792413\\
47.01	0.000977926900140917\\
48.01	0.000977929416169157\\
49.01	0.000977931986012488\\
50.01	0.000977934610830962\\
51.01	0.000977937291809956\\
52.01	0.000977940030160482\\
53.01	0.000977942827120125\\
54.01	0.000977945683953263\\
55.01	0.000977948601951856\\
56.01	0.000977951582436049\\
57.01	0.000977954626754755\\
58.01	0.00097795773628627\\
59.01	0.000977960912438923\\
60.01	0.000977964156651745\\
61.01	0.000977967470395219\\
62.01	0.000977970855171879\\
63.01	0.000977974312517095\\
64.01	0.000977977843999757\\
65.01	0.000977981451222902\\
66.01	0.000977985135824741\\
67.01	0.000977988899479122\\
68.01	0.000977992743896573\\
69.01	0.000977996670825036\\
70.01	0.000978000682050673\\
71.01	0.000978004779398585\\
72.01	0.000978008964733969\\
73.01	0.000978013239962737\\
74.01	0.000978017607032526\\
75.01	0.000978022067933695\\
76.01	0.000978026624700125\\
77.01	0.000978031279410277\\
78.01	0.00097803603418814\\
79.01	0.000978040891204311\\
80.01	0.000978045852676905\\
81.01	0.000978050920872761\\
82.01	0.000978056098108464\\
83.01	0.000978061386751328\\
84.01	0.000978066789220709\\
85.01	0.00097807230798911\\
86.01	0.000978077945583244\\
87.01	0.000978083704585513\\
88.01	0.000978089587634796\\
89.01	0.000978095597428223\\
90.01	0.000978101736722178\\
91.01	0.000978108008333575\\
92.01	0.000978114415141458\\
93.01	0.000978120960088147\\
94.01	0.000978127646180729\\
95.01	0.000978134476492525\\
96.01	0.000978141454164527\\
97.01	0.000978148582407093\\
98.01	0.000978155864501125\\
99.01	0.000978163303799975\\
100.01	0.000978170903731014\\
101.01	0.000978178667797105\\
102.01	0.000978186599578532\\
103.01	0.000978194702734539\\
104.01	0.000978202981005198\\
105.01	0.000978211438213146\\
106.01	0.000978220078265618\\
107.01	0.000978228905156135\\
108.01	0.000978237922966425\\
109.01	0.000978247135868679\\
110.01	0.000978256548127249\\
111.01	0.000978266164100912\\
112.01	0.000978275988244924\\
113.01	0.000978286025113302\\
114.01	0.000978296279360795\\
115.01	0.000978306755745513\\
116.01	0.000978317459130768\\
117.01	0.000978328394487936\\
118.01	0.00097833956689862\\
119.01	0.00097835098155704\\
120.01	0.000978362643772886\\
121.01	0.00097837455897367\\
122.01	0.000978386732707413\\
123.01	0.000978399170645523\\
124.01	0.00097841187858537\\
125.01	0.000978424862453239\\
126.01	0.000978438128307189\\
127.01	0.000978451682340092\\
128.01	0.000978465530882731\\
129.01	0.000978479680406708\\
130.01	0.000978494137527749\\
131.01	0.000978508909009045\\
132.01	0.000978524001764527\\
133.01	0.000978539422862211\\
134.01	0.000978555179527725\\
135.01	0.000978571279147918\\
136.01	0.000978587729274553\\
137.01	0.000978604537627846\\
138.01	0.0009786217121006\\
139.01	0.00097863926076177\\
140.01	0.000978657191860723\\
141.01	0.000978675513831254\\
142.01	0.00097869423529575\\
143.01	0.000978713365069467\\
144.01	0.000978732912164923\\
145.01	0.000978752885796436\\
146.01	0.000978773295384684\\
147.01	0.000978794150561374\\
148.01	0.000978815461174058\\
149.01	0.0009788372372911\\
150.01	0.00097885948920659\\
151.01	0.000978882227445584\\
152.01	0.000978905462769339\\
153.01	0.000978929206180701\\
154.01	0.000978953468929587\\
155.01	0.000978978262518707\\
156.01	0.000979003598709241\\
157.01	0.000979029489526842\\
158.01	0.000979055947267465\\
159.01	0.000979082984503894\\
160.01	0.000979110614091706\\
161.01	0.000979138849175894\\
162.01	0.000979167703197628\\
163.01	0.00097919718990077\\
164.01	0.000979227323338959\\
165.01	0.000979258117882749\\
166.01	0.00097928958822667\\
167.01	0.000979321749396848\\
168.01	0.000979354616758486\\
169.01	0.000979388206023696\\
170.01	0.000979422533259416\\
171.01	0.000979457614895658\\
172.01	0.000979493467733579\\
173.01	0.000979530108954289\\
174.01	0.00097956755612745\\
175.01	0.000979605827220183\\
176.01	0.000979644940606206\\
177.01	0.000979684915075218\\
178.01	0.000979725769842385\\
179.01	0.000979767524558271\\
180.01	0.000979810199318601\\
181.01	0.000979853814674782\\
182.01	0.000979898391644148\\
183.01	0.000979943951720866\\
184.01	0.000979990516886818\\
185.01	0.000980038109622841\\
186.01	0.000980086752920183\\
187.01	0.000980136470292346\\
188.01	0.000980187285787068\\
189.01	0.00098023922399873\\
190.01	0.000980292310080719\\
191.01	0.000980346569758483\\
192.01	0.000980402029342894\\
193.01	0.000980458715743308\\
194.01	0.000980516656481835\\
195.01	0.000980575879707124\\
196.01	0.000980636414208966\\
197.01	0.000980698289433195\\
198.01	0.000980761535496581\\
199.01	0.000980826183202589\\
200.01	0.000980892264057037\\
201.01	0.000980959810284399\\
202.01	0.000981028854844412\\
203.01	0.00098109943144898\\
204.01	0.000981171574579554\\
205.01	0.000981245319505095\\
206.01	0.000981320702300066\\
207.01	0.00098139775986307\\
208.01	0.000981476529935958\\
209.01	0.000981557051123218\\
210.01	0.000981639362912061\\
211.01	0.000981723505692672\\
212.01	0.000981809520779184\\
213.01	0.000981897450430896\\
214.01	0.00098198733787426\\
215.01	0.000982079227325149\\
216.01	0.000982173164011738\\
217.01	0.000982269194197881\\
218.01	0.000982367365207112\\
219.01	0.000982467725447043\\
220.01	0.00098257032443437\\
221.01	0.000982675212820631\\
222.01	0.000982782442418388\\
223.01	0.00098289206622787\\
224.01	0.000983004138464708\\
225.01	0.000983118714587679\\
226.01	0.000983235851327688\\
227.01	0.000983355606716904\\
228.01	0.000983478040118967\\
229.01	0.000983603212259584\\
230.01	0.000983731185257984\\
231.01	0.000983862022659251\\
232.01	0.000983995789466889\\
233.01	0.000984132552176664\\
234.01	0.000984272378810921\\
235.01	0.000984415338953833\\
236.01	0.000984561503787354\\
237.01	0.000984710946128009\\
238.01	0.000984863740464532\\
239.01	0.000985019962996402\\
240.01	0.000985179691673145\\
241.01	0.000985343006234672\\
242.01	0.00098550998825242\\
243.01	0.000985680721171445\\
244.01	0.000985855290353568\\
245.01	0.000986033783121336\\
246.01	0.000986216288803014\\
247.01	0.000986402898778722\\
248.01	0.000986593706527553\\
249.01	0.000986788807675575\\
250.01	0.000986988300045274\\
251.01	0.000987192283705654\\
252.01	0.000987400861023835\\
253.01	0.000987614136717815\\
254.01	0.00098783221791017\\
255.01	0.000988055214183013\\
256.01	0.000988283237634266\\
257.01	0.000988516402935354\\
258.01	0.000988754827389832\\
259.01	0.000988998630993609\\
260.01	0.000989247936496365\\
261.01	0.000989502869464469\\
262.01	0.000989763558345291\\
263.01	0.000990030134532631\\
264.01	0.000990302732434202\\
265.01	0.00099058148954017\\
266.01	0.000990866546493385\\
267.01	0.000991158047161086\\
268.01	0.000991456138708338\\
269.01	0.000991760971673181\\
270.01	0.000992072700043062\\
271.01	0.000992391481333565\\
272.01	0.000992717476668362\\
273.01	0.000993050850861391\\
274.01	0.000993391772500428\\
275.01	0.000993740414033099\\
276.01	0.000994096951854213\\
277.01	0.000994461566395627\\
278.01	0.000994834442217797\\
279.01	0.00099521576810329\\
280.01	0.000995605737152967\\
281.01	0.000996004546883714\\
282.01	0.000996412399328725\\
283.01	0.000996829501140058\\
284.01	0.000997256063693289\\
285.01	0.00099769230319513\\
286.01	0.000998138440792759\\
287.01	0.00099859470268631\\
288.01	0.000999061320243591\\
289.01	0.000999538530117548\\
290.01	0.00100002657436661\\
291.01	0.00100052570057756\\
292.01	0.00100103616199158\\
293.01	0.00100155821763315\\
294.01	0.00100209213244173\\
295.01	0.00100263817740722\\
296.01	0.00100319662970799\\
297.01	0.00100376777285256\\
298.01	0.00100435189682466\\
299.01	0.00100494929823174\\
300.01	0.00100556028045719\\
301.01	0.00100618515381621\\
302.01	0.00100682423571566\\
303.01	0.00100747785081769\\
304.01	0.00100814633120754\\
305.01	0.00100883001656578\\
306.01	0.00100952925434449\\
307.01	0.00101024439994826\\
308.01	0.00101097581691967\\
309.01	0.00101172387712957\\
310.01	0.00101248896097245\\
311.01	0.00101327145756668\\
312.01	0.00101407176496008\\
313.01	0.0010148902903412\\
314.01	0.00101572745025582\\
315.01	0.00101658367082933\\
316.01	0.00101745938799547\\
317.01	0.00101835504773071\\
318.01	0.00101927110629547\\
319.01	0.00102020803048173\\
320.01	0.00102116629786739\\
321.01	0.00102214639707782\\
322.01	0.00102314882805461\\
323.01	0.00102417410233171\\
324.01	0.00102522274331941\\
325.01	0.00102629528659612\\
326.01	0.00102739228020864\\
327.01	0.0010285142849805\\
328.01	0.00102966187482923\\
329.01	0.00103083563709257\\
330.01	0.00103203617286379\\
331.01	0.00103326409733641\\
332.01	0.001034520040159\\
333.01	0.00103580464579946\\
334.01	0.00103711857391993\\
335.01	0.00103846249976194\\
336.01	0.00103983711454256\\
337.01	0.00104124312586123\\
338.01	0.00104268125811837\\
339.01	0.00104415225294487\\
340.01	0.00104565686964395\\
341.01	0.00104719588564468\\
342.01	0.00104877009696788\\
343.01	0.00105038031870457\\
344.01	0.00105202738550707\\
345.01	0.00105371215209317\\
346.01	0.00105543549376341\\
347.01	0.00105719830693203\\
348.01	0.00105900150967126\\
349.01	0.00106084604226982\\
350.01	0.00106273286780535\\
351.01	0.00106466297273126\\
352.01	0.0010666373674783\\
353.01	0.00106865708707073\\
354.01	0.00107072319175775\\
355.01	0.00107283676766021\\
356.01	0.00107499892743283\\
357.01	0.00107721081094243\\
358.01	0.0010794735859621\\
359.01	0.00108178844888184\\
360.01	0.00108415662543624\\
361.01	0.00108657937144881\\
362.01	0.00108905797359376\\
363.01	0.00109159375017568\\
364.01	0.0010941880519273\\
365.01	0.00109684226282558\\
366.01	0.00109955780092683\\
367.01	0.00110233611922096\\
368.01	0.00110517870650565\\
369.01	0.00110808708828062\\
370.01	0.00111106282766278\\
371.01	0.00111410752632268\\
372.01	0.001117222825443\\
373.01	0.00112041040669993\\
374.01	0.00112367199326822\\
375.01	0.0011270093508514\\
376.01	0.00113042428873812\\
377.01	0.00113391866088624\\
378.01	0.00113749436703708\\
379.01	0.00114115335386133\\
380.01	0.00114489761613958\\
381.01	0.00114872919797993\\
382.01	0.00115265019407521\\
383.01	0.00115666275100261\\
384.01	0.00116076906856723\\
385.01	0.00116497140119103\\
386.01	0.00116927205934747\\
387.01	0.0011736734110416\\
388.01	0.00117817788333639\\
389.01	0.00118278796392682\\
390.01	0.00118750620276415\\
391.01	0.00119233521373234\\
392.01	0.00119727767637918\\
393.01	0.00120233633770445\\
394.01	0.00120751401400818\\
395.01	0.00121281359280129\\
396.01	0.00121823803478228\\
397.01	0.00122379037588293\\
398.01	0.00122947372938625\\
399.01	0.00123529128812105\\
400.01	0.00124124632673615\\
401.01	0.0012473422040592\\
402.01	0.00125358236554408\\
403.01	0.0012599703458121\\
404.01	0.00126650977129149\\
405.01	0.00127320436296136\\
406.01	0.00128005793920519\\
407.01	0.00128707441878058\\
408.01	0.00129425782391166\\
409.01	0.00130161228351112\\
410.01	0.00130914203653993\\
411.01	0.00131685143551185\\
412.01	0.00132474495015258\\
413.01	0.00133282717122183\\
414.01	0.00134110281450882\\
415.01	0.00134957672501127\\
416.01	0.00135825388130991\\
417.01	0.00136713940015008\\
418.01	0.00137623854124377\\
419.01	0.00138555671230634\\
420.01	0.00139509947434261\\
421.01	0.00140487254719901\\
422.01	0.00141488181539902\\
423.01	0.00142513333428104\\
424.01	0.00143563333645922\\
425.01	0.00144638823862921\\
426.01	0.00145740464874334\\
427.01	0.00146868937358098\\
428.01	0.00148024942674369\\
429.01	0.00149209203710532\\
430.01	0.00150422465775273\\
431.01	0.00151665497545458\\
432.01	0.00152939092070017\\
433.01	0.00154244067835579\\
434.01	0.00155581269899002\\
435.01	0.00156951571092695\\
436.01	0.00158355873309278\\
437.01	0.00159795108873086\\
438.01	0.00161270242006914\\
439.01	0.00162782270403704\\
440.01	0.00164332226914268\\
441.01	0.00165921181363873\\
442.01	0.00167550242512398\\
443.01	0.00169220560175436\\
444.01	0.00170933327526415\\
445.01	0.00172689783603421\\
446.01	0.00174491216048532\\
447.01	0.00176338964112644\\
448.01	0.00178234421964791\\
449.01	0.0018017904235238\\
450.01	0.00182174340667754\\
451.01	0.00184221899487168\\
452.01	0.0018632337366148\\
453.01	0.00188480496053472\\
454.01	0.00190695084035944\\
455.01	0.0019296904688766\\
456.01	0.00195304394251991\\
457.01	0.00197703245856441\\
458.01	0.0020016784273107\\
459.01	0.00202700560210914\\
460.01	0.00205303923061743\\
461.01	0.0020798062312703\\
462.01	0.00210733539945335\\
463.01	0.00213565764792342\\
464.01	0.00216480628436238\\
465.01	0.00219481732160913\\
466.01	0.00222572979017696\\
467.01	0.00225758593534509\\
468.01	0.0022904309084886\\
469.01	0.00232431367757898\\
470.01	0.00235929070988804\\
471.01	0.00239542522304157\\
472.01	0.00243278801778327\\
473.01	0.00247145889961402\\
474.01	0.00251152843611212\\
475.01	0.00255310014955491\\
476.01	0.00259629329740276\\
477.01	0.00264124651736053\\
478.01	0.00268812296161684\\
479.01	0.00273711620747841\\
480.01	0.00278845274894908\\
481.01	0.00281809485375913\\
482.01	0.0028376904564289\\
483.01	0.00285824238600901\\
484.01	0.00287986986848995\\
485.01	0.00290271791097438\\
486.01	0.00292696401033721\\
487.01	0.00295282676371047\\
488.01	0.00298057695173238\\
489.01	0.0030105518446611\\
490.01	0.00304317371901346\\
491.01	0.00307779714308335\\
492.01	0.00311332206216349\\
493.01	0.00314976706439995\\
494.01	0.00318715176354267\\
495.01	0.00322549583333774\\
496.01	0.00326481914899174\\
497.01	0.00330514204057174\\
498.01	0.00334648571336444\\
499.01	0.00338887291518974\\
500.01	0.0034323289608935\\
501.01	0.00347688324818993\\
502.01	0.00352257144897241\\
503.01	0.00356943863488352\\
504.01	0.00361754369033495\\
505.01	0.00366696072314491\\
506.01	0.00371773565702078\\
507.01	0.00376988797378531\\
508.01	0.00382343127861638\\
509.01	0.00387837187091133\\
510.01	0.00393470653278658\\
511.01	0.00399241970744769\\
512.01	0.00405147988646246\\
513.01	0.00411183496672934\\
514.01	0.0041734062597363\\
515.01	0.00423608073040822\\
516.01	0.00429970090066552\\
517.01	0.00436405165869855\\
518.01	0.00442884288366734\\
519.01	0.00449368678115327\\
520.01	0.00455806900599307\\
521.01	0.00462131128482143\\
522.01	0.0046825227041285\\
523.01	0.00474053607862413\\
524.01	0.00479468739552544\\
525.01	0.00484775780185083\\
526.01	0.00490099915561815\\
527.01	0.00495569460839533\\
528.01	0.005011842622378\\
529.01	0.00506943446390541\\
530.01	0.00512845593205903\\
531.01	0.00518889007741343\\
532.01	0.00525072401889197\\
533.01	0.00531396417467579\\
534.01	0.00537862213181257\\
535.01	0.00544470528652228\\
536.01	0.00551221652046107\\
537.01	0.00558115353795019\\
538.01	0.00565150795059695\\
539.01	0.00572326408073132\\
540.01	0.00579639733595701\\
541.01	0.0058708719582017\\
542.01	0.00594663896412132\\
543.01	0.006023640817064\\
544.01	0.00610169629838397\\
545.01	0.0061799446459423\\
546.01	0.00625806301414588\\
547.01	0.0063358244001688\\
548.01	0.00641299880601097\\
549.01	0.00648937082796436\\
550.01	0.00656473392672981\\
551.01	0.00663879945807873\\
552.01	0.00671127018401746\\
553.01	0.00678189713821906\\
554.01	0.00685053104426765\\
555.01	0.00691756867300738\\
556.01	0.00698399436934487\\
557.01	0.00704980690070247\\
558.01	0.0071149853986964\\
559.01	0.00717953865275502\\
560.01	0.00724349912485966\\
561.01	0.00730692267095094\\
562.01	0.00736991220277727\\
563.01	0.00743262416484024\\
564.01	0.00749525849673979\\
565.01	0.00755799012829877\\
566.01	0.00762087397561231\\
567.01	0.00768395414595888\\
568.01	0.00774728354746372\\
569.01	0.00781092427126301\\
570.01	0.0078749452850801\\
571.01	0.00793941585126919\\
572.01	0.00800439739752404\\
573.01	0.00806993562399001\\
574.01	0.00813605795685841\\
575.01	0.00820278543061315\\
576.01	0.00827013997487067\\
577.01	0.00833814309414947\\
578.01	0.0084068139649348\\
579.01	0.00847616764322697\\
580.01	0.00854621376178146\\
581.01	0.00861695616792148\\
582.01	0.00868839387545263\\
583.01	0.00876052282070629\\
584.01	0.00883333648987411\\
585.01	0.0089068255990963\\
586.01	0.00898097767697374\\
587.01	0.00905577659279556\\
588.01	0.00913120213809157\\
589.01	0.00920722970013294\\
590.01	0.00928382994806625\\
591.01	0.00936096842126078\\
592.01	0.00943860510469645\\
593.01	0.00951669429991475\\
594.01	0.00959518502752448\\
595.01	0.00967402215932641\\
596.01	0.00975314847738075\\
597.01	0.00983250783722213\\
598.01	0.00990866197942267\\
599.01	0.00997087280370268\\
599.02	0.00997138072120452\\
599.03	0.00997188557585126\\
599.04	0.00997238733782006\\
599.05	0.0099728859769941\\
599.06	0.00997338146295965\\
599.07	0.00997387376500313\\
599.08	0.00997436285210814\\
599.09	0.00997484869295248\\
599.1	0.00997533125590507\\
599.11	0.00997581050902294\\
599.12	0.00997628642004811\\
599.13	0.00997675895640448\\
599.14	0.00997722808519467\\
599.15	0.00997769377319684\\
599.16	0.00997815598686149\\
599.17	0.00997861469230817\\
599.18	0.00997906985532226\\
599.19	0.0099795214413516\\
599.2	0.0099799694155032\\
599.21	0.0099804137425398\\
599.22	0.00998085438687651\\
599.23	0.00998129131257736\\
599.24	0.00998172448335179\\
599.25	0.00998215386255116\\
599.26	0.00998257941250506\\
599.27	0.00998300109272606\\
599.28	0.00998341886232347\\
599.29	0.00998383267999937\\
599.3	0.00998424250404453\\
599.31	0.00998464829233438\\
599.32	0.00998505000232485\\
599.33	0.00998544759104829\\
599.34	0.00998584101510919\\
599.35	0.00998623023068002\\
599.36	0.00998661519349689\\
599.37	0.00998699585885528\\
599.38	0.00998737218160564\\
599.39	0.00998774411614902\\
599.4	0.00998811161643258\\
599.41	0.00998847463594512\\
599.42	0.00998883312771255\\
599.43	0.00998918704429328\\
599.44	0.00998953633777364\\
599.45	0.00998988095976314\\
599.46	0.00999022086138981\\
599.47	0.00999055599329538\\
599.48	0.00999088630563053\\
599.49	0.00999121174804997\\
599.5	0.00999153226970755\\
599.51	0.00999184781925132\\
599.52	0.0099921583448185\\
599.53	0.00999246379403045\\
599.54	0.00999276411398751\\
599.55	0.00999305925126392\\
599.56	0.00999334915190253\\
599.57	0.00999363376140962\\
599.58	0.00999391302474952\\
599.59	0.00999418688633929\\
599.6	0.00999445529004328\\
599.61	0.00999471817916768\\
599.62	0.00999497549645498\\
599.63	0.00999522718407838\\
599.64	0.00999547318363619\\
599.65	0.00999571343614611\\
599.66	0.00999594788203949\\
599.67	0.00999617646115553\\
599.68	0.00999639911273543\\
599.69	0.00999661577541645\\
599.7	0.00999682638722594\\
599.71	0.00999703088557532\\
599.72	0.00999722920725396\\
599.73	0.00999742128842304\\
599.74	0.00999760706460933\\
599.75	0.0099977864706989\\
599.76	0.00999795944093081\\
599.77	0.00999812590889066\\
599.78	0.00999828580750417\\
599.79	0.00999843906903062\\
599.8	0.00999858562505626\\
599.81	0.00999872540648766\\
599.82	0.00999885834354497\\
599.83	0.00999898436575515\\
599.84	0.00999910340194508\\
599.85	0.00999921538023465\\
599.86	0.00999932022802977\\
599.87	0.00999941787201528\\
599.88	0.00999950823814785\\
599.89	0.00999959125164875\\
599.9	0.00999966683699656\\
599.91	0.00999973491791987\\
599.92	0.0099997954173898\\
599.93	0.00999984825761255\\
599.94	0.00999989336002181\\
599.95	0.00999993064527112\\
599.96	0.00999996003322615\\
599.97	0.00999998144295691\\
599.98	0.00999999479272987\\
599.99	0.01\\
600	0.01\\
};
\addplot [color=blue,solid,forget plot]
  table[row sep=crcr]{%
0.01	0\\
1.01	0\\
2.01	0\\
3.01	0\\
4.01	0\\
5.01	0\\
6.01	0\\
7.01	0\\
8.01	0\\
9.01	0\\
10.01	0\\
11.01	0\\
12.01	0\\
13.01	0\\
14.01	0\\
15.01	0\\
16.01	0\\
17.01	0\\
18.01	0\\
19.01	0\\
20.01	0\\
21.01	0\\
22.01	0\\
23.01	0\\
24.01	0\\
25.01	0\\
26.01	0\\
27.01	0\\
28.01	0\\
29.01	0\\
30.01	0\\
31.01	0\\
32.01	0\\
33.01	0\\
34.01	0\\
35.01	0\\
36.01	0\\
37.01	0\\
38.01	0\\
39.01	0\\
40.01	0\\
41.01	0\\
42.01	0\\
43.01	0\\
44.01	0\\
45.01	0\\
46.01	0\\
47.01	0\\
48.01	0\\
49.01	0\\
50.01	0\\
51.01	0\\
52.01	0\\
53.01	0\\
54.01	0\\
55.01	0\\
56.01	0\\
57.01	0\\
58.01	0\\
59.01	0\\
60.01	0\\
61.01	0\\
62.01	0\\
63.01	0\\
64.01	0\\
65.01	0\\
66.01	0\\
67.01	0\\
68.01	0\\
69.01	0\\
70.01	0\\
71.01	0\\
72.01	0\\
73.01	0\\
74.01	0\\
75.01	0\\
76.01	0\\
77.01	0\\
78.01	0\\
79.01	0\\
80.01	0\\
81.01	0\\
82.01	0\\
83.01	0\\
84.01	0\\
85.01	0\\
86.01	0\\
87.01	0\\
88.01	0\\
89.01	0\\
90.01	0\\
91.01	0\\
92.01	0\\
93.01	0\\
94.01	0\\
95.01	0\\
96.01	0\\
97.01	0\\
98.01	0\\
99.01	0\\
100.01	0\\
101.01	0\\
102.01	0\\
103.01	0\\
104.01	0\\
105.01	0\\
106.01	0\\
107.01	0\\
108.01	0\\
109.01	0\\
110.01	0\\
111.01	0\\
112.01	0\\
113.01	0\\
114.01	0\\
115.01	0\\
116.01	0\\
117.01	0\\
118.01	0\\
119.01	0\\
120.01	0\\
121.01	0\\
122.01	0\\
123.01	0\\
124.01	0\\
125.01	0\\
126.01	0\\
127.01	0\\
128.01	0\\
129.01	0\\
130.01	0\\
131.01	0\\
132.01	0\\
133.01	0\\
134.01	0\\
135.01	0\\
136.01	0\\
137.01	0\\
138.01	0\\
139.01	0\\
140.01	0\\
141.01	0\\
142.01	0\\
143.01	0\\
144.01	0\\
145.01	0\\
146.01	0\\
147.01	0\\
148.01	0\\
149.01	0\\
150.01	0\\
151.01	0\\
152.01	0\\
153.01	0\\
154.01	0\\
155.01	0\\
156.01	0\\
157.01	0\\
158.01	0\\
159.01	0\\
160.01	0\\
161.01	0\\
162.01	0\\
163.01	0\\
164.01	0\\
165.01	0\\
166.01	0\\
167.01	0\\
168.01	0\\
169.01	0\\
170.01	0\\
171.01	0\\
172.01	0\\
173.01	0\\
174.01	0\\
175.01	0\\
176.01	0\\
177.01	0\\
178.01	0\\
179.01	0\\
180.01	0\\
181.01	0\\
182.01	0\\
183.01	0\\
184.01	0\\
185.01	0\\
186.01	0\\
187.01	0\\
188.01	0\\
189.01	0\\
190.01	0\\
191.01	0\\
192.01	0\\
193.01	0\\
194.01	0\\
195.01	0\\
196.01	0\\
197.01	0\\
198.01	0\\
199.01	0\\
200.01	0\\
201.01	0\\
202.01	0\\
203.01	0\\
204.01	0\\
205.01	0\\
206.01	0\\
207.01	0\\
208.01	0\\
209.01	0\\
210.01	0\\
211.01	0\\
212.01	0\\
213.01	0\\
214.01	0\\
215.01	0\\
216.01	0\\
217.01	0\\
218.01	0\\
219.01	0\\
220.01	0\\
221.01	0\\
222.01	0\\
223.01	0\\
224.01	0\\
225.01	0\\
226.01	0\\
227.01	0\\
228.01	0\\
229.01	0\\
230.01	0\\
231.01	0\\
232.01	0\\
233.01	0\\
234.01	0\\
235.01	0\\
236.01	0\\
237.01	0\\
238.01	0\\
239.01	0\\
240.01	0\\
241.01	0\\
242.01	0\\
243.01	0\\
244.01	0\\
245.01	0\\
246.01	0\\
247.01	0\\
248.01	0\\
249.01	0\\
250.01	0\\
251.01	0\\
252.01	0\\
253.01	0\\
254.01	0\\
255.01	0\\
256.01	0\\
257.01	0\\
258.01	0\\
259.01	0\\
260.01	0\\
261.01	0\\
262.01	0\\
263.01	0\\
264.01	0\\
265.01	0\\
266.01	0\\
267.01	0\\
268.01	0\\
269.01	0\\
270.01	0\\
271.01	0\\
272.01	0\\
273.01	0\\
274.01	0\\
275.01	0\\
276.01	0\\
277.01	0\\
278.01	0\\
279.01	0\\
280.01	0\\
281.01	0\\
282.01	0\\
283.01	0\\
284.01	0\\
285.01	0\\
286.01	0\\
287.01	0\\
288.01	0\\
289.01	0\\
290.01	0\\
291.01	0\\
292.01	0\\
293.01	0\\
294.01	0\\
295.01	0\\
296.01	0\\
297.01	0\\
298.01	0\\
299.01	0\\
300.01	0\\
301.01	0\\
302.01	0\\
303.01	0\\
304.01	0\\
305.01	0\\
306.01	0\\
307.01	0\\
308.01	0\\
309.01	0\\
310.01	0\\
311.01	0\\
312.01	0\\
313.01	0\\
314.01	0\\
315.01	0\\
316.01	0\\
317.01	0\\
318.01	0\\
319.01	0\\
320.01	0\\
321.01	0\\
322.01	0\\
323.01	0\\
324.01	0\\
325.01	0\\
326.01	0\\
327.01	0\\
328.01	0\\
329.01	0\\
330.01	0\\
331.01	0\\
332.01	0\\
333.01	0\\
334.01	0\\
335.01	0\\
336.01	0\\
337.01	0\\
338.01	0\\
339.01	0\\
340.01	0\\
341.01	0\\
342.01	0\\
343.01	0\\
344.01	0\\
345.01	0\\
346.01	0\\
347.01	0\\
348.01	0\\
349.01	0\\
350.01	0\\
351.01	0\\
352.01	0\\
353.01	0\\
354.01	0\\
355.01	0\\
356.01	0\\
357.01	0\\
358.01	0\\
359.01	0\\
360.01	0\\
361.01	0\\
362.01	0\\
363.01	0\\
364.01	0\\
365.01	0\\
366.01	0\\
367.01	0\\
368.01	0\\
369.01	0\\
370.01	0\\
371.01	0\\
372.01	0\\
373.01	0\\
374.01	0\\
375.01	0\\
376.01	0\\
377.01	0\\
378.01	0\\
379.01	0\\
380.01	0\\
381.01	0\\
382.01	0\\
383.01	0\\
384.01	0\\
385.01	0\\
386.01	0\\
387.01	0\\
388.01	0\\
389.01	0\\
390.01	0\\
391.01	0\\
392.01	0\\
393.01	0\\
394.01	0\\
395.01	0\\
396.01	0\\
397.01	0\\
398.01	0\\
399.01	0\\
400.01	0\\
401.01	0\\
402.01	0\\
403.01	0\\
404.01	0\\
405.01	0\\
406.01	0\\
407.01	0\\
408.01	0\\
409.01	0\\
410.01	0\\
411.01	0\\
412.01	0\\
413.01	0\\
414.01	0\\
415.01	0\\
416.01	0\\
417.01	0\\
418.01	0\\
419.01	0\\
420.01	0\\
421.01	0\\
422.01	0\\
423.01	0\\
424.01	0\\
425.01	0\\
426.01	0\\
427.01	0\\
428.01	0\\
429.01	0\\
430.01	0\\
431.01	0\\
432.01	0\\
433.01	0\\
434.01	0\\
435.01	0\\
436.01	0\\
437.01	0\\
438.01	0\\
439.01	0\\
440.01	0\\
441.01	0\\
442.01	0\\
443.01	0\\
444.01	0\\
445.01	0\\
446.01	0\\
447.01	0\\
448.01	0\\
449.01	0\\
450.01	0\\
451.01	0\\
452.01	0\\
453.01	0\\
454.01	0\\
455.01	0\\
456.01	0\\
457.01	0\\
458.01	0\\
459.01	0\\
460.01	0\\
461.01	0\\
462.01	0\\
463.01	0\\
464.01	0\\
465.01	0\\
466.01	0\\
467.01	0\\
468.01	0\\
469.01	0\\
470.01	0\\
471.01	0\\
472.01	0\\
473.01	0\\
474.01	0\\
475.01	0\\
476.01	0\\
477.01	0\\
478.01	0\\
479.01	0\\
480.01	0\\
481.01	2.40966981226928e-05\\
482.01	6.02289196438011e-05\\
483.01	9.7493952585008e-05\\
484.01	0.000135932639077595\\
485.01	0.000175584678312843\\
486.01	0.000216487278826116\\
487.01	0.000258673283014495\\
488.01	0.000302168582289146\\
489.01	0.000346988580984628\\
490.01	0.000393133387252401\\
491.01	0.000440595944165442\\
492.01	0.000489414579453555\\
493.01	0.000539644681450131\\
494.01	0.000591344781451793\\
495.01	0.000644576616767242\\
496.01	0.000699405125739842\\
497.01	0.000755898342664832\\
498.01	0.000814127149159516\\
499.01	0.000874164822961092\\
500.01	0.000936086304175951\\
501.01	0.000999967071387352\\
502.01	0.00106588148366777\\
503.01	0.00113390039593665\\
504.01	0.00120408779034123\\
505.01	0.00127650080732459\\
506.01	0.00135123384775706\\
507.01	0.0014284158805291\\
508.01	0.00150818998615476\\
509.01	0.00159071524288537\\
510.01	0.00167616935849231\\
511.01	0.00176475183676328\\
512.01	0.00185668780418655\\
513.01	0.00195223265526322\\
514.01	0.00205167771738155\\
515.01	0.00215535719120302\\
516.01	0.00226365669406223\\
517.01	0.00237702382739495\\
518.01	0.00249598131293208\\
519.01	0.00262114340450183\\
520.01	0.00275323647292278\\
521.01	0.00289312492573644\\
522.01	0.00304184399238899\\
523.01	0.00320064139004236\\
524.01	0.00337017847018786\\
525.01	0.0035477461113088\\
526.01	0.0036295125370641\\
527.01	0.0037108927160592\\
528.01	0.00379456895392187\\
529.01	0.00388056948962633\\
530.01	0.00396891275840736\\
531.01	0.0040596039183237\\
532.01	0.00415263008682738\\
533.01	0.00424795372319946\\
534.01	0.00434550378487202\\
535.01	0.00444515876139365\\
536.01	0.0045467365748054\\
537.01	0.0046499857123393\\
538.01	0.00475456809396017\\
539.01	0.00486003862294787\\
540.01	0.00496581932279654\\
541.01	0.00507116688167566\\
542.01	0.00517506758599043\\
543.01	0.00527604888611295\\
544.01	0.00537221070707498\\
545.01	0.00546672253319988\\
546.01	0.00556220625235845\\
547.01	0.0056580924038923\\
548.01	0.0057535950559224\\
549.01	0.00584763774075489\\
550.01	0.00594170309216324\\
551.01	0.00603796482129104\\
552.01	0.00613623750855436\\
553.01	0.00623623587041443\\
554.01	0.00633753842167594\\
555.01	0.00643917690572694\\
556.01	0.00653969205861012\\
557.01	0.00663870292660752\\
558.01	0.00673595224939041\\
559.01	0.00683132728671051\\
560.01	0.00692492472489956\\
561.01	0.00701661563739462\\
562.01	0.00710600980656644\\
563.01	0.00719278358715181\\
564.01	0.00727677208888639\\
565.01	0.00735899562298691\\
566.01	0.00744011141004657\\
567.01	0.00752017065907459\\
568.01	0.00759925750663107\\
569.01	0.00767738576415853\\
570.01	0.00775462864345971\\
571.01	0.00783113453106925\\
572.01	0.00790711791574717\\
573.01	0.00798282551971004\\
574.01	0.00805842551950888\\
575.01	0.00813393580891355\\
576.01	0.00820937321546196\\
577.01	0.00828477503550238\\
578.01	0.00836019599880814\\
579.01	0.00843570205622002\\
580.01	0.00851136051977954\\
581.01	0.0085872267444925\\
582.01	0.00866332969700664\\
583.01	0.00873967262081048\\
584.01	0.00881625188759261\\
585.01	0.00889306268112347\\
586.01	0.00897010042272349\\
587.01	0.00904736164167313\\
588.01	0.00912484327393588\\
589.01	0.00920254246997357\\
590.01	0.00928045754230262\\
591.01	0.00935858943802847\\
592.01	0.00943694120142937\\
593.01	0.00951551569543855\\
594.01	0.00959431296143754\\
595.01	0.00967332770823237\\
596.01	0.00975254751940146\\
597.01	0.00983195256120927\\
598.01	0.00990865761438377\\
599.01	0.00997087276439033\\
599.02	0.009971380683866\\
599.03	0.00997188554040684\\
599.04	0.00997238730419245\\
599.05	0.00997288594510838\\
599.06	0.00997338143274323\\
599.07	0.00997387373638571\\
599.08	0.00997436282502164\\
599.09	0.00997484866733099\\
599.1	0.00997533123168485\\
599.11	0.0099758104861423\\
599.12	0.00997628639844743\\
599.13	0.00997675893602614\\
599.14	0.00997722806598298\\
599.15	0.00997769375509804\\
599.16	0.00997815596982367\\
599.17	0.00997861467628126\\
599.18	0.00997906984025795\\
599.19	0.00997952142720333\\
599.2	0.00997996940222608\\
599.21	0.00998041373009062\\
599.22	0.00998085437521367\\
599.23	0.00998129130166082\\
599.24	0.00998172447314304\\
599.25	0.0099821538530132\\
599.26	0.00998257940360238\\
599.27	0.00998300108442457\\
599.28	0.00998341885459047\\
599.29	0.00998383267280348\\
599.3	0.00998424249735567\\
599.31	0.00998464828612373\\
599.32	0.00998504999656486\\
599.33	0.00998544758571256\\
599.34	0.00998584101017251\\
599.35	0.00998623022611829\\
599.36	0.00998661518928712\\
599.37	0.00998699585497554\\
599.38	0.00998737217803502\\
599.39	0.0099877441128676\\
599.4	0.00998811161342143\\
599.41	0.00998847463318623\\
599.42	0.00998883312518882\\
599.43	0.00998918704198849\\
599.44	0.00998953633567239\\
599.45	0.00998988095785088\\
599.46	0.00999022085965275\\
599.47	0.0099905559917205\\
599.48	0.00999088630420554\\
599.49	0.00999121174676329\\
599.5	0.00999153226854827\\
599.51	0.00999184781820919\\
599.52	0.00999215834388389\\
599.53	0.00999246379319433\\
599.54	0.00999276411324143\\
599.55	0.00999305925059996\\
599.56	0.00999334915131334\\
599.57	0.00999363376088831\\
599.58	0.0099939130242897\\
599.59	0.00999418688593504\\
599.6	0.0099944552896891\\
599.61	0.00999471817885849\\
599.62	0.00999497549618609\\
599.63	0.00999522718384549\\
599.64	0.00999547318343534\\
599.65	0.00999571343597366\\
599.66	0.00999594788189215\\
599.67	0.00999617646103028\\
599.68	0.00999639911262954\\
599.69	0.00999661577532744\\
599.7	0.00999682638715159\\
599.71	0.00999703088551362\\
599.72	0.00999722920720313\\
599.73	0.00999742128838149\\
599.74	0.00999760706457565\\
599.75	0.00999778647067185\\
599.76	0.00999795944090929\\
599.77	0.00999812590887373\\
599.78	0.00999828580749101\\
599.79	0.00999843906902052\\
599.8	0.00999858562504862\\
599.81	0.00999872540648197\\
599.82	0.00999885834354081\\
599.83	0.00999898436575217\\
599.84	0.009999103401943\\
599.85	0.00999921538023323\\
599.86	0.00999932022802883\\
599.87	0.00999941787201469\\
599.88	0.00999950823814749\\
599.89	0.00999959125164854\\
599.9	0.00999966683699645\\
599.91	0.00999973491791981\\
599.92	0.00999979541738977\\
599.93	0.00999984825761254\\
599.94	0.00999989336002181\\
599.95	0.00999993064527112\\
599.96	0.00999996003322615\\
599.97	0.00999998144295691\\
599.98	0.00999999479272987\\
599.99	0.01\\
600	0.01\\
};
\addplot [color=mycolor10,solid,forget plot]
  table[row sep=crcr]{%
0.01	0\\
1.01	0\\
2.01	0\\
3.01	0\\
4.01	0\\
5.01	0\\
6.01	0\\
7.01	0\\
8.01	0\\
9.01	0\\
10.01	0\\
11.01	0\\
12.01	0\\
13.01	0\\
14.01	0\\
15.01	0\\
16.01	0\\
17.01	0\\
18.01	0\\
19.01	0\\
20.01	0\\
21.01	0\\
22.01	0\\
23.01	0\\
24.01	0\\
25.01	0\\
26.01	0\\
27.01	0\\
28.01	0\\
29.01	0\\
30.01	0\\
31.01	0\\
32.01	0\\
33.01	0\\
34.01	0\\
35.01	0\\
36.01	0\\
37.01	0\\
38.01	0\\
39.01	0\\
40.01	0\\
41.01	0\\
42.01	0\\
43.01	0\\
44.01	0\\
45.01	0\\
46.01	0\\
47.01	0\\
48.01	0\\
49.01	0\\
50.01	0\\
51.01	0\\
52.01	0\\
53.01	0\\
54.01	0\\
55.01	0\\
56.01	0\\
57.01	0\\
58.01	0\\
59.01	0\\
60.01	0\\
61.01	0\\
62.01	0\\
63.01	0\\
64.01	0\\
65.01	0\\
66.01	0\\
67.01	0\\
68.01	0\\
69.01	0\\
70.01	0\\
71.01	0\\
72.01	0\\
73.01	0\\
74.01	0\\
75.01	0\\
76.01	0\\
77.01	0\\
78.01	0\\
79.01	0\\
80.01	0\\
81.01	0\\
82.01	0\\
83.01	0\\
84.01	0\\
85.01	0\\
86.01	0\\
87.01	0\\
88.01	0\\
89.01	0\\
90.01	0\\
91.01	0\\
92.01	0\\
93.01	0\\
94.01	0\\
95.01	0\\
96.01	0\\
97.01	0\\
98.01	0\\
99.01	0\\
100.01	0\\
101.01	0\\
102.01	0\\
103.01	0\\
104.01	0\\
105.01	0\\
106.01	0\\
107.01	0\\
108.01	0\\
109.01	0\\
110.01	0\\
111.01	0\\
112.01	0\\
113.01	0\\
114.01	0\\
115.01	0\\
116.01	0\\
117.01	0\\
118.01	0\\
119.01	0\\
120.01	0\\
121.01	0\\
122.01	0\\
123.01	0\\
124.01	0\\
125.01	0\\
126.01	0\\
127.01	0\\
128.01	0\\
129.01	0\\
130.01	0\\
131.01	0\\
132.01	0\\
133.01	0\\
134.01	0\\
135.01	0\\
136.01	0\\
137.01	0\\
138.01	0\\
139.01	0\\
140.01	0\\
141.01	0\\
142.01	0\\
143.01	0\\
144.01	0\\
145.01	0\\
146.01	0\\
147.01	0\\
148.01	0\\
149.01	0\\
150.01	0\\
151.01	0\\
152.01	0\\
153.01	0\\
154.01	0\\
155.01	0\\
156.01	0\\
157.01	0\\
158.01	0\\
159.01	0\\
160.01	0\\
161.01	0\\
162.01	0\\
163.01	0\\
164.01	0\\
165.01	0\\
166.01	0\\
167.01	0\\
168.01	0\\
169.01	0\\
170.01	0\\
171.01	0\\
172.01	0\\
173.01	0\\
174.01	0\\
175.01	0\\
176.01	0\\
177.01	0\\
178.01	0\\
179.01	0\\
180.01	0\\
181.01	0\\
182.01	0\\
183.01	0\\
184.01	0\\
185.01	0\\
186.01	0\\
187.01	0\\
188.01	0\\
189.01	0\\
190.01	0\\
191.01	0\\
192.01	0\\
193.01	0\\
194.01	0\\
195.01	0\\
196.01	0\\
197.01	0\\
198.01	0\\
199.01	0\\
200.01	0\\
201.01	0\\
202.01	0\\
203.01	0\\
204.01	0\\
205.01	0\\
206.01	0\\
207.01	0\\
208.01	0\\
209.01	0\\
210.01	0\\
211.01	0\\
212.01	0\\
213.01	0\\
214.01	0\\
215.01	0\\
216.01	0\\
217.01	0\\
218.01	0\\
219.01	0\\
220.01	0\\
221.01	0\\
222.01	0\\
223.01	0\\
224.01	0\\
225.01	0\\
226.01	0\\
227.01	0\\
228.01	0\\
229.01	0\\
230.01	0\\
231.01	0\\
232.01	0\\
233.01	0\\
234.01	0\\
235.01	0\\
236.01	0\\
237.01	0\\
238.01	0\\
239.01	0\\
240.01	0\\
241.01	0\\
242.01	0\\
243.01	0\\
244.01	0\\
245.01	0\\
246.01	0\\
247.01	0\\
248.01	0\\
249.01	0\\
250.01	0\\
251.01	0\\
252.01	0\\
253.01	0\\
254.01	0\\
255.01	0\\
256.01	0\\
257.01	0\\
258.01	0\\
259.01	0\\
260.01	0\\
261.01	0\\
262.01	0\\
263.01	0\\
264.01	0\\
265.01	0\\
266.01	0\\
267.01	0\\
268.01	0\\
269.01	0\\
270.01	0\\
271.01	0\\
272.01	0\\
273.01	0\\
274.01	0\\
275.01	0\\
276.01	0\\
277.01	0\\
278.01	0\\
279.01	0\\
280.01	0\\
281.01	0\\
282.01	0\\
283.01	0\\
284.01	0\\
285.01	0\\
286.01	0\\
287.01	0\\
288.01	0\\
289.01	0\\
290.01	0\\
291.01	0\\
292.01	0\\
293.01	0\\
294.01	0\\
295.01	0\\
296.01	0\\
297.01	0\\
298.01	0\\
299.01	0\\
300.01	0\\
301.01	0\\
302.01	0\\
303.01	0\\
304.01	0\\
305.01	0\\
306.01	0\\
307.01	0\\
308.01	0\\
309.01	0\\
310.01	0\\
311.01	0\\
312.01	0\\
313.01	0\\
314.01	0\\
315.01	0\\
316.01	0\\
317.01	0\\
318.01	0\\
319.01	0\\
320.01	0\\
321.01	0\\
322.01	0\\
323.01	0\\
324.01	0\\
325.01	0\\
326.01	0\\
327.01	0\\
328.01	0\\
329.01	0\\
330.01	0\\
331.01	0\\
332.01	0\\
333.01	0\\
334.01	0\\
335.01	0\\
336.01	0\\
337.01	0\\
338.01	0\\
339.01	0\\
340.01	0\\
341.01	0\\
342.01	0\\
343.01	0\\
344.01	0\\
345.01	0\\
346.01	0\\
347.01	0\\
348.01	0\\
349.01	0\\
350.01	0\\
351.01	0\\
352.01	0\\
353.01	0\\
354.01	0\\
355.01	0\\
356.01	0\\
357.01	0\\
358.01	0\\
359.01	0\\
360.01	0\\
361.01	0\\
362.01	0\\
363.01	0\\
364.01	0\\
365.01	0\\
366.01	0\\
367.01	0\\
368.01	0\\
369.01	0\\
370.01	0\\
371.01	0\\
372.01	0\\
373.01	0\\
374.01	0\\
375.01	0\\
376.01	0\\
377.01	0\\
378.01	0\\
379.01	0\\
380.01	0\\
381.01	0\\
382.01	0\\
383.01	0\\
384.01	0\\
385.01	0\\
386.01	0\\
387.01	0\\
388.01	0\\
389.01	0\\
390.01	0\\
391.01	0\\
392.01	0\\
393.01	0\\
394.01	0\\
395.01	0\\
396.01	0\\
397.01	0\\
398.01	0\\
399.01	0\\
400.01	0\\
401.01	0\\
402.01	0\\
403.01	0\\
404.01	0\\
405.01	0\\
406.01	0\\
407.01	0\\
408.01	0\\
409.01	0\\
410.01	0\\
411.01	0\\
412.01	0\\
413.01	0\\
414.01	0\\
415.01	0\\
416.01	0\\
417.01	0\\
418.01	0\\
419.01	0\\
420.01	0\\
421.01	0\\
422.01	0\\
423.01	0\\
424.01	0\\
425.01	0\\
426.01	0\\
427.01	0\\
428.01	0\\
429.01	0\\
430.01	0\\
431.01	0\\
432.01	0\\
433.01	0\\
434.01	0\\
435.01	0\\
436.01	0\\
437.01	0\\
438.01	0\\
439.01	0\\
440.01	0\\
441.01	0\\
442.01	0\\
443.01	0\\
444.01	0\\
445.01	0\\
446.01	0\\
447.01	0\\
448.01	0\\
449.01	0\\
450.01	0\\
451.01	0\\
452.01	0\\
453.01	0\\
454.01	0\\
455.01	0\\
456.01	0\\
457.01	0\\
458.01	0\\
459.01	0\\
460.01	0\\
461.01	0\\
462.01	0\\
463.01	0\\
464.01	0\\
465.01	0\\
466.01	0\\
467.01	0\\
468.01	0\\
469.01	0\\
470.01	0\\
471.01	0\\
472.01	0\\
473.01	0\\
474.01	0\\
475.01	0\\
476.01	0\\
477.01	0\\
478.01	0\\
479.01	0\\
480.01	0\\
481.01	0\\
482.01	0\\
483.01	0\\
484.01	0\\
485.01	0\\
486.01	0\\
487.01	0\\
488.01	0\\
489.01	0\\
490.01	0\\
491.01	0\\
492.01	0\\
493.01	0\\
494.01	0\\
495.01	0\\
496.01	0\\
497.01	0\\
498.01	0\\
499.01	0\\
500.01	0\\
501.01	0\\
502.01	0\\
503.01	0\\
504.01	0\\
505.01	0\\
506.01	0\\
507.01	0\\
508.01	0\\
509.01	0\\
510.01	0\\
511.01	0\\
512.01	0\\
513.01	0\\
514.01	0\\
515.01	0\\
516.01	0\\
517.01	0\\
518.01	0\\
519.01	0\\
520.01	0\\
521.01	0\\
522.01	0\\
523.01	0\\
524.01	0\\
525.01	0\\
526.01	0.000102771456501963\\
527.01	0.000211857262483357\\
528.01	0.000324838345335268\\
529.01	0.000441966255021686\\
530.01	0.000563519085941433\\
531.01	0.000689805325113445\\
532.01	0.000821168123154713\\
533.01	0.000957989226061683\\
534.01	0.00110069167556698\\
535.01	0.00124976707994646\\
536.01	0.00140579226223662\\
537.01	0.00156943036835827\\
538.01	0.00174142679592595\\
539.01	0.0019226250528362\\
540.01	0.00211398992820282\\
541.01	0.00231662768418949\\
542.01	0.00253187134003461\\
543.01	0.00276144006226553\\
544.01	0.00300736772181577\\
545.01	0.00326661567286447\\
546.01	0.00353699838186093\\
547.01	0.00381957633107237\\
548.01	0.00411560683043353\\
549.01	0.00442528142982412\\
550.01	0.00456342194511371\\
551.01	0.00470501977180571\\
552.01	0.00484979652204437\\
553.01	0.00499735440866724\\
554.01	0.00514714434222182\\
555.01	0.00529842819406245\\
556.01	0.00545023857511322\\
557.01	0.00560127475048913\\
558.01	0.00574977686249003\\
559.01	0.00589337759184622\\
560.01	0.00603144302138599\\
561.01	0.00617116532806554\\
562.01	0.00631279169818998\\
563.01	0.00645550940284293\\
564.01	0.00659814217849057\\
565.01	0.00673791067745923\\
566.01	0.0068719890954267\\
567.01	0.00699830368032526\\
568.01	0.00712095802857818\\
569.01	0.00724086556742704\\
570.01	0.00735745019262945\\
571.01	0.00747029237527268\\
572.01	0.00757928265660215\\
573.01	0.00768484450268323\\
574.01	0.00778859580472871\\
575.01	0.0078909719541174\\
576.01	0.00799171655542844\\
577.01	0.00809063074719604\\
578.01	0.00818760060298669\\
579.01	0.0082826282025498\\
580.01	0.00837586389043174\\
581.01	0.00846763456953067\\
582.01	0.00855842645664378\\
583.01	0.0086485452226052\\
584.01	0.0087380156008509\\
585.01	0.00882682224290697\\
586.01	0.00891491374940033\\
587.01	0.00900224063450917\\
588.01	0.00908876407379003\\
589.01	0.0091744495113924\\
590.01	0.00925926074743837\\
591.01	0.00934317099294246\\
592.01	0.0094261874329281\\
593.01	0.00950835739313356\\
594.01	0.0095897680639356\\
595.01	0.00967054251212487\\
596.01	0.00975083116190794\\
597.01	0.00983079836948479\\
598.01	0.00990857284332932\\
599.01	0.00997086936970872\\
599.02	0.0099713774254769\\
599.03	0.00997188241419166\\
599.04	0.00997238430611518\\
599.05	0.00997288307121482\\
599.06	0.00997337867916017\\
599.07	0.00997387109932015\\
599.08	0.00997436030076002\\
599.09	0.00997484625223838\\
599.1	0.00997532892220413\\
599.11	0.0099758082787934\\
599.12	0.00997628428982652\\
599.13	0.00997675692280479\\
599.14	0.00997722614490744\\
599.15	0.00997769192298835\\
599.16	0.0099781542235729\\
599.17	0.0099786130128547\\
599.18	0.00997906825669229\\
599.19	0.00997951992060585\\
599.2	0.00997996796977384\\
599.21	0.00998041236902966\\
599.22	0.00998085308285816\\
599.23	0.00998129007539227\\
599.24	0.00998172331040948\\
599.25	0.00998215275132834\\
599.26	0.00998257836054724\\
599.27	0.00998300009764374\\
599.28	0.00998341792179137\\
599.29	0.00998383179175552\\
599.3	0.00998424166588949\\
599.31	0.00998464750213039\\
599.32	0.00998504925799502\\
599.33	0.00998544689057571\\
599.34	0.00998584035653615\\
599.35	0.00998622961210713\\
599.36	0.00998661461308229\\
599.37	0.00998699531481375\\
599.38	0.00998737167220779\\
599.39	0.00998774363972043\\
599.4	0.00998811117135299\\
599.41	0.00998847422064756\\
599.42	0.0099888327406825\\
599.43	0.00998918668406783\\
599.44	0.00998953600294064\\
599.45	0.00998988064896035\\
599.46	0.00999022057330405\\
599.47	0.0099905557266617\\
599.48	0.00999088605923132\\
599.49	0.00999121152071412\\
599.5	0.0099915320603096\\
599.51	0.00999184762671058\\
599.52	0.0099921581680982\\
599.53	0.00999246363213685\\
599.54	0.00999276396596905\\
599.55	0.00999305911621034\\
599.56	0.00999334902894398\\
599.57	0.0099936336497158\\
599.58	0.00999391292352879\\
599.59	0.00999418679483778\\
599.6	0.009994455207544\\
599.61	0.00999471810498963\\
599.62	0.00999497542995225\\
599.63	0.00999522712463928\\
599.64	0.00999547313068229\\
599.65	0.00999571338913138\\
599.66	0.00999594784044938\\
599.67	0.00999617642450606\\
599.68	0.00999639908057226\\
599.69	0.00999661574731397\\
599.7	0.00999682636278636\\
599.71	0.00999703086442773\\
599.72	0.00999722918905342\\
599.73	0.00999742127284961\\
599.74	0.00999760705136717\\
599.75	0.00999778645951533\\
599.76	0.00999795943155535\\
599.77	0.00999812590109411\\
599.78	0.00999828580107764\\
599.79	0.0099984390637846\\
599.8	0.00999858562081968\\
599.81	0.00999872540310692\\
599.82	0.009998858340883\\
599.83	0.00999898436369044\\
599.84	0.00999910340037074\\
599.85	0.00999921537905747\\
599.86	0.00999932022716924\\
599.87	0.00999941787140266\\
599.88	0.0099995082377252\\
599.89	0.00999959125136801\\
599.9	0.00999966683681859\\
599.91	0.00999973491781351\\
599.92	0.00999979541733096\\
599.93	0.00999984825758326\\
599.94	0.00999989336000932\\
599.95	0.00999993064526697\\
599.96	0.00999996003322533\\
599.97	0.00999998144295691\\
599.98	0.00999999479272987\\
599.99	0.01\\
600	0.01\\
};
\addplot [color=mycolor11,solid,forget plot]
  table[row sep=crcr]{%
0.01	0\\
1.01	0\\
2.01	0\\
3.01	0\\
4.01	0\\
5.01	0\\
6.01	0\\
7.01	0\\
8.01	0\\
9.01	0\\
10.01	0\\
11.01	0\\
12.01	0\\
13.01	0\\
14.01	0\\
15.01	0\\
16.01	0\\
17.01	0\\
18.01	0\\
19.01	0\\
20.01	0\\
21.01	0\\
22.01	0\\
23.01	0\\
24.01	0\\
25.01	0\\
26.01	0\\
27.01	0\\
28.01	0\\
29.01	0\\
30.01	0\\
31.01	0\\
32.01	0\\
33.01	0\\
34.01	0\\
35.01	0\\
36.01	0\\
37.01	0\\
38.01	0\\
39.01	0\\
40.01	0\\
41.01	0\\
42.01	0\\
43.01	0\\
44.01	0\\
45.01	0\\
46.01	0\\
47.01	0\\
48.01	0\\
49.01	0\\
50.01	0\\
51.01	0\\
52.01	0\\
53.01	0\\
54.01	0\\
55.01	0\\
56.01	0\\
57.01	0\\
58.01	0\\
59.01	0\\
60.01	0\\
61.01	0\\
62.01	0\\
63.01	0\\
64.01	0\\
65.01	0\\
66.01	0\\
67.01	0\\
68.01	0\\
69.01	0\\
70.01	0\\
71.01	0\\
72.01	0\\
73.01	0\\
74.01	0\\
75.01	0\\
76.01	0\\
77.01	0\\
78.01	0\\
79.01	0\\
80.01	0\\
81.01	0\\
82.01	0\\
83.01	0\\
84.01	0\\
85.01	0\\
86.01	0\\
87.01	0\\
88.01	0\\
89.01	0\\
90.01	0\\
91.01	0\\
92.01	0\\
93.01	0\\
94.01	0\\
95.01	0\\
96.01	0\\
97.01	0\\
98.01	0\\
99.01	0\\
100.01	0\\
101.01	0\\
102.01	0\\
103.01	0\\
104.01	0\\
105.01	0\\
106.01	0\\
107.01	0\\
108.01	0\\
109.01	0\\
110.01	0\\
111.01	0\\
112.01	0\\
113.01	0\\
114.01	0\\
115.01	0\\
116.01	0\\
117.01	0\\
118.01	0\\
119.01	0\\
120.01	0\\
121.01	0\\
122.01	0\\
123.01	0\\
124.01	0\\
125.01	0\\
126.01	0\\
127.01	0\\
128.01	0\\
129.01	0\\
130.01	0\\
131.01	0\\
132.01	0\\
133.01	0\\
134.01	0\\
135.01	0\\
136.01	0\\
137.01	0\\
138.01	0\\
139.01	0\\
140.01	0\\
141.01	0\\
142.01	0\\
143.01	0\\
144.01	0\\
145.01	0\\
146.01	0\\
147.01	0\\
148.01	0\\
149.01	0\\
150.01	0\\
151.01	0\\
152.01	0\\
153.01	0\\
154.01	0\\
155.01	0\\
156.01	0\\
157.01	0\\
158.01	0\\
159.01	0\\
160.01	0\\
161.01	0\\
162.01	0\\
163.01	0\\
164.01	0\\
165.01	0\\
166.01	0\\
167.01	0\\
168.01	0\\
169.01	0\\
170.01	0\\
171.01	0\\
172.01	0\\
173.01	0\\
174.01	0\\
175.01	0\\
176.01	0\\
177.01	0\\
178.01	0\\
179.01	0\\
180.01	0\\
181.01	0\\
182.01	0\\
183.01	0\\
184.01	0\\
185.01	0\\
186.01	0\\
187.01	0\\
188.01	0\\
189.01	0\\
190.01	0\\
191.01	0\\
192.01	0\\
193.01	0\\
194.01	0\\
195.01	0\\
196.01	0\\
197.01	0\\
198.01	0\\
199.01	0\\
200.01	0\\
201.01	0\\
202.01	0\\
203.01	0\\
204.01	0\\
205.01	0\\
206.01	0\\
207.01	0\\
208.01	0\\
209.01	0\\
210.01	0\\
211.01	0\\
212.01	0\\
213.01	0\\
214.01	0\\
215.01	0\\
216.01	0\\
217.01	0\\
218.01	0\\
219.01	0\\
220.01	0\\
221.01	0\\
222.01	0\\
223.01	0\\
224.01	0\\
225.01	0\\
226.01	0\\
227.01	0\\
228.01	0\\
229.01	0\\
230.01	0\\
231.01	0\\
232.01	0\\
233.01	0\\
234.01	0\\
235.01	0\\
236.01	0\\
237.01	0\\
238.01	0\\
239.01	0\\
240.01	0\\
241.01	0\\
242.01	0\\
243.01	0\\
244.01	0\\
245.01	0\\
246.01	0\\
247.01	0\\
248.01	0\\
249.01	0\\
250.01	0\\
251.01	0\\
252.01	0\\
253.01	0\\
254.01	0\\
255.01	0\\
256.01	0\\
257.01	0\\
258.01	0\\
259.01	0\\
260.01	0\\
261.01	0\\
262.01	0\\
263.01	0\\
264.01	0\\
265.01	0\\
266.01	0\\
267.01	0\\
268.01	0\\
269.01	0\\
270.01	0\\
271.01	0\\
272.01	0\\
273.01	0\\
274.01	0\\
275.01	0\\
276.01	0\\
277.01	0\\
278.01	0\\
279.01	0\\
280.01	0\\
281.01	0\\
282.01	0\\
283.01	0\\
284.01	0\\
285.01	0\\
286.01	0\\
287.01	0\\
288.01	0\\
289.01	0\\
290.01	0\\
291.01	0\\
292.01	0\\
293.01	0\\
294.01	0\\
295.01	0\\
296.01	0\\
297.01	0\\
298.01	0\\
299.01	0\\
300.01	0\\
301.01	0\\
302.01	0\\
303.01	0\\
304.01	0\\
305.01	0\\
306.01	0\\
307.01	0\\
308.01	0\\
309.01	0\\
310.01	0\\
311.01	0\\
312.01	0\\
313.01	0\\
314.01	0\\
315.01	0\\
316.01	0\\
317.01	0\\
318.01	0\\
319.01	0\\
320.01	0\\
321.01	0\\
322.01	0\\
323.01	0\\
324.01	0\\
325.01	0\\
326.01	0\\
327.01	0\\
328.01	0\\
329.01	0\\
330.01	0\\
331.01	0\\
332.01	0\\
333.01	0\\
334.01	0\\
335.01	0\\
336.01	0\\
337.01	0\\
338.01	0\\
339.01	0\\
340.01	0\\
341.01	0\\
342.01	0\\
343.01	0\\
344.01	0\\
345.01	0\\
346.01	0\\
347.01	0\\
348.01	0\\
349.01	0\\
350.01	0\\
351.01	0\\
352.01	0\\
353.01	0\\
354.01	0\\
355.01	0\\
356.01	0\\
357.01	0\\
358.01	0\\
359.01	0\\
360.01	0\\
361.01	0\\
362.01	0\\
363.01	0\\
364.01	0\\
365.01	0\\
366.01	0\\
367.01	0\\
368.01	0\\
369.01	0\\
370.01	0\\
371.01	0\\
372.01	0\\
373.01	0\\
374.01	0\\
375.01	0\\
376.01	0\\
377.01	0\\
378.01	0\\
379.01	0\\
380.01	0\\
381.01	0\\
382.01	0\\
383.01	0\\
384.01	0\\
385.01	0\\
386.01	0\\
387.01	0\\
388.01	0\\
389.01	0\\
390.01	0\\
391.01	0\\
392.01	0\\
393.01	0\\
394.01	0\\
395.01	0\\
396.01	0\\
397.01	0\\
398.01	0\\
399.01	0\\
400.01	0\\
401.01	0\\
402.01	0\\
403.01	0\\
404.01	0\\
405.01	0\\
406.01	0\\
407.01	0\\
408.01	0\\
409.01	0\\
410.01	0\\
411.01	0\\
412.01	0\\
413.01	0\\
414.01	0\\
415.01	0\\
416.01	0\\
417.01	0\\
418.01	0\\
419.01	0\\
420.01	0\\
421.01	0\\
422.01	0\\
423.01	0\\
424.01	0\\
425.01	0\\
426.01	0\\
427.01	0\\
428.01	0\\
429.01	0\\
430.01	0\\
431.01	0\\
432.01	0\\
433.01	0\\
434.01	0\\
435.01	0\\
436.01	0\\
437.01	0\\
438.01	0\\
439.01	0\\
440.01	0\\
441.01	0\\
442.01	0\\
443.01	0\\
444.01	0\\
445.01	0\\
446.01	0\\
447.01	0\\
448.01	0\\
449.01	0\\
450.01	0\\
451.01	0\\
452.01	0\\
453.01	0\\
454.01	0\\
455.01	0\\
456.01	0\\
457.01	0\\
458.01	0\\
459.01	0\\
460.01	0\\
461.01	0\\
462.01	0\\
463.01	0\\
464.01	0\\
465.01	0\\
466.01	0\\
467.01	0\\
468.01	0\\
469.01	0\\
470.01	0\\
471.01	0\\
472.01	0\\
473.01	0\\
474.01	0\\
475.01	0\\
476.01	0\\
477.01	0\\
478.01	0\\
479.01	0\\
480.01	0\\
481.01	0\\
482.01	0\\
483.01	0\\
484.01	0\\
485.01	0\\
486.01	0\\
487.01	0\\
488.01	0\\
489.01	0\\
490.01	0\\
491.01	0\\
492.01	0\\
493.01	0\\
494.01	0\\
495.01	0\\
496.01	0\\
497.01	0\\
498.01	0\\
499.01	0\\
500.01	0\\
501.01	0\\
502.01	0\\
503.01	0\\
504.01	0\\
505.01	0\\
506.01	0\\
507.01	0\\
508.01	0\\
509.01	0\\
510.01	0\\
511.01	0\\
512.01	0\\
513.01	0\\
514.01	0\\
515.01	0\\
516.01	0\\
517.01	0\\
518.01	0\\
519.01	0\\
520.01	0\\
521.01	0\\
522.01	0\\
523.01	0\\
524.01	0\\
525.01	0\\
526.01	0\\
527.01	0\\
528.01	0\\
529.01	0\\
530.01	0\\
531.01	0\\
532.01	0\\
533.01	0\\
534.01	0\\
535.01	0\\
536.01	0\\
537.01	0\\
538.01	0\\
539.01	0\\
540.01	0\\
541.01	0\\
542.01	0\\
543.01	0\\
544.01	0\\
545.01	0\\
546.01	0\\
547.01	0\\
548.01	0\\
549.01	1.3150276878017e-06\\
550.01	0.00018803212887869\\
551.01	0.000383425867251676\\
552.01	0.000588371123930863\\
553.01	0.000803879090139129\\
554.01	0.0010311236007458\\
555.01	0.00127147336229072\\
556.01	0.00152653133905252\\
557.01	0.00179818381624366\\
558.01	0.00208866277061284\\
559.01	0.00240062447287979\\
560.01	0.0027347493126044\\
561.01	0.00308407921609736\\
562.01	0.00344878121259429\\
563.01	0.0038298958305099\\
564.01	0.00422869759739618\\
565.01	0.00464705663062651\\
566.01	0.00508738754435159\\
567.01	0.00544973364055435\\
568.01	0.0056475991624296\\
569.01	0.00584743223013972\\
570.01	0.00604748900368386\\
571.01	0.00624531985607343\\
572.01	0.00643752645254089\\
573.01	0.0066221749281171\\
574.01	0.00680685556915087\\
575.01	0.00699128669887769\\
576.01	0.00717427665445123\\
577.01	0.00735438709921835\\
578.01	0.00752989876416336\\
579.01	0.00769877883743499\\
580.01	0.0078586552305761\\
581.01	0.00800680768344173\\
582.01	0.00814164657742384\\
583.01	0.00827128831769647\\
584.01	0.00839855560001231\\
585.01	0.0085240333202042\\
586.01	0.00864790941724483\\
587.01	0.0087698867552374\\
588.01	0.00888970311215465\\
589.01	0.00900714619236641\\
590.01	0.00912201300988473\\
591.01	0.00923390318299627\\
592.01	0.00934234304250483\\
593.01	0.00944692082238488\\
594.01	0.00954732441074327\\
595.01	0.00964338499452397\\
596.01	0.00973512605081985\\
597.01	0.00982281460498365\\
598.01	0.00990633521989775\\
599.01	0.00997061088345206\\
599.02	0.00997112665722618\\
599.03	0.00997163921121569\\
599.04	0.00997214851709544\\
599.05	0.00997265454624966\\
599.06	0.00997315726976898\\
599.07	0.00997365665844757\\
599.08	0.00997415268278013\\
599.09	0.0099746453129589\\
599.1	0.00997513451887065\\
599.11	0.0099756202700936\\
599.12	0.00997610253589435\\
599.13	0.00997658128522475\\
599.14	0.00997705648671877\\
599.15	0.0099775281086893\\
599.16	0.00997799611912494\\
599.17	0.00997846048568677\\
599.18	0.00997892117570508\\
599.19	0.00997937815617603\\
599.2	0.00997983139375833\\
599.21	0.00998028085476989\\
599.22	0.00998072650518434\\
599.23	0.00998116831062769\\
599.24	0.00998160623637473\\
599.25	0.00998204024734566\\
599.26	0.00998247030788024\\
599.27	0.00998289637903019\\
599.28	0.0099833184214462\\
599.29	0.00998373639537388\\
599.3	0.0099841502606498\\
599.31	0.00998455997669731\\
599.32	0.00998496550252246\\
599.33	0.00998536679670981\\
599.34	0.00998576381741821\\
599.35	0.00998615652237657\\
599.36	0.00998654486887954\\
599.37	0.00998692881378318\\
599.38	0.00998730831350056\\
599.39	0.00998768332399733\\
599.4	0.00998805380078729\\
599.41	0.00998841969892779\\
599.42	0.00998878097301526\\
599.43	0.00998913757718051\\
599.44	0.00998948946508413\\
599.45	0.00998983658991178\\
599.46	0.00999017890437104\\
599.47	0.00999051636068749\\
599.48	0.00999084891059986\\
599.49	0.00999117650535519\\
599.5	0.00999149909570389\\
599.51	0.00999181663189475\\
599.52	0.00999212906366997\\
599.53	0.00999243634026004\\
599.54	0.00999273841037864\\
599.55	0.00999303522221745\\
599.56	0.00999332672344092\\
599.57	0.00999361286118097\\
599.58	0.00999389358203168\\
599.59	0.00999416883204387\\
599.6	0.00999443855671962\\
599.61	0.00999470270100682\\
599.62	0.00999496120929356\\
599.63	0.00999521402540248\\
599.64	0.00999546109258514\\
599.65	0.00999570235351621\\
599.66	0.00999593775028769\\
599.67	0.00999616722440306\\
599.68	0.00999639071677128\\
599.69	0.00999660816770083\\
599.7	0.00999681951689364\\
599.71	0.00999702470343895\\
599.72	0.00999722366580711\\
599.73	0.0099974163418433\\
599.74	0.00999760266876119\\
599.75	0.00999778258313656\\
599.76	0.00999795602090076\\
599.77	0.00999812291733421\\
599.78	0.00999828320705971\\
599.79	0.00999843682403578\\
599.8	0.00999858370154984\\
599.81	0.00999872377221136\\
599.82	0.00999885696794489\\
599.83	0.00999898321998304\\
599.84	0.00999910245885937\\
599.85	0.00999921461440118\\
599.86	0.00999931961572222\\
599.87	0.00999941739121527\\
599.88	0.00999950786854473\\
599.89	0.00999959097463901\\
599.9	0.00999966663568284\\
599.91	0.00999973477710955\\
599.92	0.00999979532359316\\
599.93	0.00999984819904042\\
599.94	0.00999989332658271\\
599.95	0.00999993062856786\\
599.96	0.00999996002655181\\
599.97	0.00999998144129022\\
599.98	0.00999999479272987\\
599.99	0.01\\
600	0.01\\
};
\addplot [color=mycolor12,solid,forget plot]
  table[row sep=crcr]{%
0.01	0\\
1.01	0\\
2.01	0\\
3.01	0\\
4.01	0\\
5.01	0\\
6.01	0\\
7.01	0\\
8.01	0\\
9.01	0\\
10.01	0\\
11.01	0\\
12.01	0\\
13.01	0\\
14.01	0\\
15.01	0\\
16.01	0\\
17.01	0\\
18.01	0\\
19.01	0\\
20.01	0\\
21.01	0\\
22.01	0\\
23.01	0\\
24.01	0\\
25.01	0\\
26.01	0\\
27.01	0\\
28.01	0\\
29.01	0\\
30.01	0\\
31.01	0\\
32.01	0\\
33.01	0\\
34.01	0\\
35.01	0\\
36.01	0\\
37.01	0\\
38.01	0\\
39.01	0\\
40.01	0\\
41.01	0\\
42.01	0\\
43.01	0\\
44.01	0\\
45.01	0\\
46.01	0\\
47.01	0\\
48.01	0\\
49.01	0\\
50.01	0\\
51.01	0\\
52.01	0\\
53.01	0\\
54.01	0\\
55.01	0\\
56.01	0\\
57.01	0\\
58.01	0\\
59.01	0\\
60.01	0\\
61.01	0\\
62.01	0\\
63.01	0\\
64.01	0\\
65.01	0\\
66.01	0\\
67.01	0\\
68.01	0\\
69.01	0\\
70.01	0\\
71.01	0\\
72.01	0\\
73.01	0\\
74.01	0\\
75.01	0\\
76.01	0\\
77.01	0\\
78.01	0\\
79.01	0\\
80.01	0\\
81.01	0\\
82.01	0\\
83.01	0\\
84.01	0\\
85.01	0\\
86.01	0\\
87.01	0\\
88.01	0\\
89.01	0\\
90.01	0\\
91.01	0\\
92.01	0\\
93.01	0\\
94.01	0\\
95.01	0\\
96.01	0\\
97.01	0\\
98.01	0\\
99.01	0\\
100.01	0\\
101.01	0\\
102.01	0\\
103.01	0\\
104.01	0\\
105.01	0\\
106.01	0\\
107.01	0\\
108.01	0\\
109.01	0\\
110.01	0\\
111.01	0\\
112.01	0\\
113.01	0\\
114.01	0\\
115.01	0\\
116.01	0\\
117.01	0\\
118.01	0\\
119.01	0\\
120.01	0\\
121.01	0\\
122.01	0\\
123.01	0\\
124.01	0\\
125.01	0\\
126.01	0\\
127.01	0\\
128.01	0\\
129.01	0\\
130.01	0\\
131.01	0\\
132.01	0\\
133.01	0\\
134.01	0\\
135.01	0\\
136.01	0\\
137.01	0\\
138.01	0\\
139.01	0\\
140.01	0\\
141.01	0\\
142.01	0\\
143.01	0\\
144.01	0\\
145.01	0\\
146.01	0\\
147.01	0\\
148.01	0\\
149.01	0\\
150.01	0\\
151.01	0\\
152.01	0\\
153.01	0\\
154.01	0\\
155.01	0\\
156.01	0\\
157.01	0\\
158.01	0\\
159.01	0\\
160.01	0\\
161.01	0\\
162.01	0\\
163.01	0\\
164.01	0\\
165.01	0\\
166.01	0\\
167.01	0\\
168.01	0\\
169.01	0\\
170.01	0\\
171.01	0\\
172.01	0\\
173.01	0\\
174.01	0\\
175.01	0\\
176.01	0\\
177.01	0\\
178.01	0\\
179.01	0\\
180.01	0\\
181.01	0\\
182.01	0\\
183.01	0\\
184.01	0\\
185.01	0\\
186.01	0\\
187.01	0\\
188.01	0\\
189.01	0\\
190.01	0\\
191.01	0\\
192.01	0\\
193.01	0\\
194.01	0\\
195.01	0\\
196.01	0\\
197.01	0\\
198.01	0\\
199.01	0\\
200.01	0\\
201.01	0\\
202.01	0\\
203.01	0\\
204.01	0\\
205.01	0\\
206.01	0\\
207.01	0\\
208.01	0\\
209.01	0\\
210.01	0\\
211.01	0\\
212.01	0\\
213.01	0\\
214.01	0\\
215.01	0\\
216.01	0\\
217.01	0\\
218.01	0\\
219.01	0\\
220.01	0\\
221.01	0\\
222.01	0\\
223.01	0\\
224.01	0\\
225.01	0\\
226.01	0\\
227.01	0\\
228.01	0\\
229.01	0\\
230.01	0\\
231.01	0\\
232.01	0\\
233.01	0\\
234.01	0\\
235.01	0\\
236.01	0\\
237.01	0\\
238.01	0\\
239.01	0\\
240.01	0\\
241.01	0\\
242.01	0\\
243.01	0\\
244.01	0\\
245.01	0\\
246.01	0\\
247.01	0\\
248.01	0\\
249.01	0\\
250.01	0\\
251.01	0\\
252.01	0\\
253.01	0\\
254.01	0\\
255.01	0\\
256.01	0\\
257.01	0\\
258.01	0\\
259.01	0\\
260.01	0\\
261.01	0\\
262.01	0\\
263.01	0\\
264.01	0\\
265.01	0\\
266.01	0\\
267.01	0\\
268.01	0\\
269.01	0\\
270.01	0\\
271.01	0\\
272.01	0\\
273.01	0\\
274.01	0\\
275.01	0\\
276.01	0\\
277.01	0\\
278.01	0\\
279.01	0\\
280.01	0\\
281.01	0\\
282.01	0\\
283.01	0\\
284.01	0\\
285.01	0\\
286.01	0\\
287.01	0\\
288.01	0\\
289.01	0\\
290.01	0\\
291.01	0\\
292.01	0\\
293.01	0\\
294.01	0\\
295.01	0\\
296.01	0\\
297.01	0\\
298.01	0\\
299.01	0\\
300.01	0\\
301.01	0\\
302.01	0\\
303.01	0\\
304.01	0\\
305.01	0\\
306.01	0\\
307.01	0\\
308.01	0\\
309.01	0\\
310.01	0\\
311.01	0\\
312.01	0\\
313.01	0\\
314.01	0\\
315.01	0\\
316.01	0\\
317.01	0\\
318.01	0\\
319.01	0\\
320.01	0\\
321.01	0\\
322.01	0\\
323.01	0\\
324.01	0\\
325.01	0\\
326.01	0\\
327.01	0\\
328.01	0\\
329.01	0\\
330.01	0\\
331.01	0\\
332.01	0\\
333.01	0\\
334.01	0\\
335.01	0\\
336.01	0\\
337.01	0\\
338.01	0\\
339.01	0\\
340.01	0\\
341.01	0\\
342.01	0\\
343.01	0\\
344.01	0\\
345.01	0\\
346.01	0\\
347.01	0\\
348.01	0\\
349.01	0\\
350.01	0\\
351.01	0\\
352.01	0\\
353.01	0\\
354.01	0\\
355.01	0\\
356.01	0\\
357.01	0\\
358.01	0\\
359.01	0\\
360.01	0\\
361.01	0\\
362.01	0\\
363.01	0\\
364.01	0\\
365.01	0\\
366.01	0\\
367.01	0\\
368.01	0\\
369.01	0\\
370.01	0\\
371.01	0\\
372.01	0\\
373.01	0\\
374.01	0\\
375.01	0\\
376.01	0\\
377.01	0\\
378.01	0\\
379.01	0\\
380.01	0\\
381.01	0\\
382.01	0\\
383.01	0\\
384.01	0\\
385.01	0\\
386.01	0\\
387.01	0\\
388.01	0\\
389.01	0\\
390.01	0\\
391.01	0\\
392.01	0\\
393.01	0\\
394.01	0\\
395.01	0\\
396.01	0\\
397.01	0\\
398.01	0\\
399.01	0\\
400.01	0\\
401.01	0\\
402.01	0\\
403.01	0\\
404.01	0\\
405.01	0\\
406.01	0\\
407.01	0\\
408.01	0\\
409.01	0\\
410.01	0\\
411.01	0\\
412.01	0\\
413.01	0\\
414.01	0\\
415.01	0\\
416.01	0\\
417.01	0\\
418.01	0\\
419.01	0\\
420.01	0\\
421.01	0\\
422.01	0\\
423.01	0\\
424.01	0\\
425.01	0\\
426.01	0\\
427.01	0\\
428.01	0\\
429.01	0\\
430.01	0\\
431.01	0\\
432.01	0\\
433.01	0\\
434.01	0\\
435.01	0\\
436.01	0\\
437.01	0\\
438.01	0\\
439.01	0\\
440.01	0\\
441.01	0\\
442.01	0\\
443.01	0\\
444.01	0\\
445.01	0\\
446.01	0\\
447.01	0\\
448.01	0\\
449.01	0\\
450.01	0\\
451.01	0\\
452.01	0\\
453.01	0\\
454.01	0\\
455.01	0\\
456.01	0\\
457.01	0\\
458.01	0\\
459.01	0\\
460.01	0\\
461.01	0\\
462.01	0\\
463.01	0\\
464.01	0\\
465.01	0\\
466.01	0\\
467.01	0\\
468.01	0\\
469.01	0\\
470.01	0\\
471.01	0\\
472.01	0\\
473.01	0\\
474.01	0\\
475.01	0\\
476.01	0\\
477.01	0\\
478.01	0\\
479.01	0\\
480.01	0\\
481.01	0\\
482.01	0\\
483.01	0\\
484.01	0\\
485.01	0\\
486.01	0\\
487.01	0\\
488.01	0\\
489.01	0\\
490.01	0\\
491.01	0\\
492.01	0\\
493.01	0\\
494.01	0\\
495.01	0\\
496.01	0\\
497.01	0\\
498.01	0\\
499.01	0\\
500.01	0\\
501.01	0\\
502.01	0\\
503.01	0\\
504.01	0\\
505.01	0\\
506.01	0\\
507.01	0\\
508.01	0\\
509.01	0\\
510.01	0\\
511.01	0\\
512.01	0\\
513.01	0\\
514.01	0\\
515.01	0\\
516.01	0\\
517.01	0\\
518.01	0\\
519.01	0\\
520.01	0\\
521.01	0\\
522.01	0\\
523.01	0\\
524.01	0\\
525.01	0\\
526.01	0\\
527.01	0\\
528.01	0\\
529.01	0\\
530.01	0\\
531.01	0\\
532.01	0\\
533.01	0\\
534.01	0\\
535.01	0\\
536.01	0\\
537.01	0\\
538.01	0\\
539.01	0\\
540.01	0\\
541.01	0\\
542.01	0\\
543.01	0\\
544.01	0\\
545.01	0\\
546.01	0\\
547.01	0\\
548.01	0\\
549.01	0\\
550.01	0\\
551.01	0\\
552.01	0\\
553.01	0\\
554.01	0\\
555.01	0\\
556.01	0\\
557.01	0\\
558.01	0\\
559.01	0\\
560.01	0\\
561.01	0\\
562.01	0\\
563.01	0\\
564.01	0\\
565.01	0\\
566.01	0\\
567.01	0.000102186357564371\\
568.01	0.000389004268560401\\
569.01	0.000693210272731084\\
570.01	0.00101718055313735\\
571.01	0.00136374365253529\\
572.01	0.0017362802601048\\
573.01	0.00213614104658714\\
574.01	0.00255403495792057\\
575.01	0.00298985795617562\\
576.01	0.00344508022503881\\
577.01	0.00392130871497483\\
578.01	0.00442029093212064\\
579.01	0.00494390684300543\\
580.01	0.00549412735162226\\
581.01	0.00607287639058549\\
582.01	0.00655902377829562\\
583.01	0.00679815156285074\\
584.01	0.00702626828205187\\
585.01	0.00724284999880616\\
586.01	0.00745955259200264\\
587.01	0.00767677270493367\\
588.01	0.00789364530608335\\
589.01	0.00810926959771742\\
590.01	0.00832353557737281\\
591.01	0.00853710703215436\\
592.01	0.00874855389543367\\
593.01	0.00895606871113526\\
594.01	0.00915750548245555\\
595.01	0.0093503362862595\\
596.01	0.0095316075842035\\
597.01	0.00969789970826579\\
598.01	0.00984529451390937\\
599.01	0.00995515823496999\\
599.02	0.00995597699786755\\
599.03	0.00995678973044666\\
599.04	0.00995759639776036\\
599.05	0.0099583969645674\\
599.06	0.00995919139532951\\
599.07	0.00995997965420866\\
599.08	0.00996076170506433\\
599.09	0.00996153751145075\\
599.1	0.00996230703661406\\
599.11	0.00996307024348954\\
599.12	0.00996382709469877\\
599.13	0.00996457755254676\\
599.14	0.00996532157901905\\
599.15	0.00996605913577888\\
599.16	0.00996679018416421\\
599.17	0.00996751468518479\\
599.18	0.00996823259951922\\
599.19	0.00996894388751196\\
599.2	0.00996964850917033\\
599.21	0.00997034642416144\\
599.22	0.00997103759180925\\
599.23	0.00997172197109139\\
599.24	0.00997239952063618\\
599.25	0.00997307019871944\\
599.26	0.00997373396326153\\
599.27	0.0099743907718255\\
599.28	0.00997504058161395\\
599.29	0.00997568334946589\\
599.3	0.00997631903185352\\
599.31	0.00997694758487909\\
599.32	0.00997756896427161\\
599.33	0.00997818312538369\\
599.34	0.00997879002318826\\
599.35	0.0099793896122753\\
599.36	0.00997998184684856\\
599.37	0.00998056668072227\\
599.38	0.00998114406731784\\
599.39	0.00998171395966052\\
599.4	0.00998227631037606\\
599.41	0.00998283107168736\\
599.42	0.00998337819541113\\
599.43	0.00998391763295447\\
599.44	0.00998444933531151\\
599.45	0.00998497325306004\\
599.46	0.00998548933306155\\
599.47	0.00998599752004539\\
599.48	0.00998649775826485\\
599.49	0.00998698999149323\\
599.5	0.00998747416301998\\
599.51	0.0099879502156468\\
599.52	0.00998841809168377\\
599.53	0.00998887773294543\\
599.54	0.00998932908074684\\
599.55	0.00998977207589977\\
599.56	0.00999020665870868\\
599.57	0.00999063276896691\\
599.58	0.00999105034595272\\
599.59	0.00999145932842542\\
599.6	0.00999185965462149\\
599.61	0.00999225126225067\\
599.62	0.00999263408849213\\
599.63	0.00999300806999059\\
599.64	0.0099933731428525\\
599.65	0.00999372924264221\\
599.66	0.00999407630437819\\
599.67	0.00999441426252925\\
599.68	0.00999474305101081\\
599.69	0.00999506260318117\\
599.7	0.00999537285183787\\
599.71	0.009995673729214\\
599.72	0.00999596516697468\\
599.73	0.00999624709621344\\
599.74	0.00999651944744879\\
599.75	0.00999678215062071\\
599.76	0.00999703513508736\\
599.77	0.0099972783296217\\
599.78	0.00999751166240827\\
599.79	0.00999773506104004\\
599.8	0.00999794845251533\\
599.81	0.00999815176323479\\
599.82	0.00999834491899857\\
599.83	0.00999852784500348\\
599.84	0.00999870046584035\\
599.85	0.00999886270549145\\
599.86	0.00999901448732809\\
599.87	0.00999915573410834\\
599.88	0.00999928636797488\\
599.89	0.00999940631045301\\
599.9	0.00999951548244887\\
599.91	0.00999961380424782\\
599.92	0.00999970119551297\\
599.93	0.00999977757528401\\
599.94	0.00999984286197615\\
599.95	0.00999989697337941\\
599.96	0.00999993982665806\\
599.97	0.0099999713383504\\
599.98	0.00999999142436879\\
599.99	0.01\\
600	0.01\\
};
\addplot [color=mycolor13,solid,forget plot]
  table[row sep=crcr]{%
0.01	0\\
1.01	0\\
2.01	0\\
3.01	0\\
4.01	0\\
5.01	0\\
6.01	0\\
7.01	0\\
8.01	0\\
9.01	0\\
10.01	0\\
11.01	0\\
12.01	0\\
13.01	0\\
14.01	0\\
15.01	0\\
16.01	0\\
17.01	0\\
18.01	0\\
19.01	0\\
20.01	0\\
21.01	0\\
22.01	0\\
23.01	0\\
24.01	0\\
25.01	0\\
26.01	0\\
27.01	0\\
28.01	0\\
29.01	0\\
30.01	0\\
31.01	0\\
32.01	0\\
33.01	0\\
34.01	0\\
35.01	0\\
36.01	0\\
37.01	0\\
38.01	0\\
39.01	0\\
40.01	0\\
41.01	0\\
42.01	0\\
43.01	0\\
44.01	0\\
45.01	0\\
46.01	0\\
47.01	0\\
48.01	0\\
49.01	0\\
50.01	0\\
51.01	0\\
52.01	0\\
53.01	0\\
54.01	0\\
55.01	0\\
56.01	0\\
57.01	0\\
58.01	0\\
59.01	0\\
60.01	0\\
61.01	0\\
62.01	0\\
63.01	0\\
64.01	0\\
65.01	0\\
66.01	0\\
67.01	0\\
68.01	0\\
69.01	0\\
70.01	0\\
71.01	0\\
72.01	0\\
73.01	0\\
74.01	0\\
75.01	0\\
76.01	0\\
77.01	0\\
78.01	0\\
79.01	0\\
80.01	0\\
81.01	0\\
82.01	0\\
83.01	0\\
84.01	0\\
85.01	0\\
86.01	0\\
87.01	0\\
88.01	0\\
89.01	0\\
90.01	0\\
91.01	0\\
92.01	0\\
93.01	0\\
94.01	0\\
95.01	0\\
96.01	0\\
97.01	0\\
98.01	0\\
99.01	0\\
100.01	0\\
101.01	0\\
102.01	0\\
103.01	0\\
104.01	0\\
105.01	0\\
106.01	0\\
107.01	0\\
108.01	0\\
109.01	0\\
110.01	0\\
111.01	0\\
112.01	0\\
113.01	0\\
114.01	0\\
115.01	0\\
116.01	0\\
117.01	0\\
118.01	0\\
119.01	0\\
120.01	0\\
121.01	0\\
122.01	0\\
123.01	0\\
124.01	0\\
125.01	0\\
126.01	0\\
127.01	0\\
128.01	0\\
129.01	0\\
130.01	0\\
131.01	0\\
132.01	0\\
133.01	0\\
134.01	0\\
135.01	0\\
136.01	0\\
137.01	0\\
138.01	0\\
139.01	0\\
140.01	0\\
141.01	0\\
142.01	0\\
143.01	0\\
144.01	0\\
145.01	0\\
146.01	0\\
147.01	0\\
148.01	0\\
149.01	0\\
150.01	0\\
151.01	0\\
152.01	0\\
153.01	0\\
154.01	0\\
155.01	0\\
156.01	0\\
157.01	0\\
158.01	0\\
159.01	0\\
160.01	0\\
161.01	0\\
162.01	0\\
163.01	0\\
164.01	0\\
165.01	0\\
166.01	0\\
167.01	0\\
168.01	0\\
169.01	0\\
170.01	0\\
171.01	0\\
172.01	0\\
173.01	0\\
174.01	0\\
175.01	0\\
176.01	0\\
177.01	0\\
178.01	0\\
179.01	0\\
180.01	0\\
181.01	0\\
182.01	0\\
183.01	0\\
184.01	0\\
185.01	0\\
186.01	0\\
187.01	0\\
188.01	0\\
189.01	0\\
190.01	0\\
191.01	0\\
192.01	0\\
193.01	0\\
194.01	0\\
195.01	0\\
196.01	0\\
197.01	0\\
198.01	0\\
199.01	0\\
200.01	0\\
201.01	0\\
202.01	0\\
203.01	0\\
204.01	0\\
205.01	0\\
206.01	0\\
207.01	0\\
208.01	0\\
209.01	0\\
210.01	0\\
211.01	0\\
212.01	0\\
213.01	0\\
214.01	0\\
215.01	0\\
216.01	0\\
217.01	0\\
218.01	0\\
219.01	0\\
220.01	0\\
221.01	0\\
222.01	0\\
223.01	0\\
224.01	0\\
225.01	0\\
226.01	0\\
227.01	0\\
228.01	0\\
229.01	0\\
230.01	0\\
231.01	0\\
232.01	0\\
233.01	0\\
234.01	0\\
235.01	0\\
236.01	0\\
237.01	0\\
238.01	0\\
239.01	0\\
240.01	0\\
241.01	0\\
242.01	0\\
243.01	0\\
244.01	0\\
245.01	0\\
246.01	0\\
247.01	0\\
248.01	0\\
249.01	0\\
250.01	0\\
251.01	0\\
252.01	0\\
253.01	0\\
254.01	0\\
255.01	0\\
256.01	0\\
257.01	0\\
258.01	0\\
259.01	0\\
260.01	0\\
261.01	0\\
262.01	0\\
263.01	0\\
264.01	0\\
265.01	0\\
266.01	0\\
267.01	0\\
268.01	0\\
269.01	0\\
270.01	0\\
271.01	0\\
272.01	0\\
273.01	0\\
274.01	0\\
275.01	0\\
276.01	0\\
277.01	0\\
278.01	0\\
279.01	0\\
280.01	0\\
281.01	0\\
282.01	0\\
283.01	0\\
284.01	0\\
285.01	0\\
286.01	0\\
287.01	0\\
288.01	0\\
289.01	0\\
290.01	0\\
291.01	0\\
292.01	0\\
293.01	0\\
294.01	0\\
295.01	0\\
296.01	0\\
297.01	0\\
298.01	0\\
299.01	0\\
300.01	0\\
301.01	0\\
302.01	0\\
303.01	0\\
304.01	0\\
305.01	0\\
306.01	0\\
307.01	0\\
308.01	0\\
309.01	0\\
310.01	0\\
311.01	0\\
312.01	0\\
313.01	0\\
314.01	0\\
315.01	0\\
316.01	0\\
317.01	0\\
318.01	0\\
319.01	0\\
320.01	0\\
321.01	0\\
322.01	0\\
323.01	0\\
324.01	0\\
325.01	0\\
326.01	0\\
327.01	0\\
328.01	0\\
329.01	0\\
330.01	0\\
331.01	0\\
332.01	0\\
333.01	0\\
334.01	0\\
335.01	0\\
336.01	0\\
337.01	0\\
338.01	0\\
339.01	0\\
340.01	0\\
341.01	0\\
342.01	0\\
343.01	0\\
344.01	0\\
345.01	0\\
346.01	0\\
347.01	0\\
348.01	0\\
349.01	0\\
350.01	0\\
351.01	0\\
352.01	0\\
353.01	0\\
354.01	0\\
355.01	0\\
356.01	0\\
357.01	0\\
358.01	0\\
359.01	0\\
360.01	0\\
361.01	0\\
362.01	0\\
363.01	0\\
364.01	0\\
365.01	0\\
366.01	0\\
367.01	0\\
368.01	0\\
369.01	0\\
370.01	0\\
371.01	0\\
372.01	0\\
373.01	0\\
374.01	0\\
375.01	0\\
376.01	0\\
377.01	0\\
378.01	0\\
379.01	0\\
380.01	0\\
381.01	0\\
382.01	0\\
383.01	0\\
384.01	0\\
385.01	0\\
386.01	0\\
387.01	0\\
388.01	0\\
389.01	0\\
390.01	0\\
391.01	0\\
392.01	0\\
393.01	0\\
394.01	0\\
395.01	0\\
396.01	0\\
397.01	0\\
398.01	0\\
399.01	0\\
400.01	0\\
401.01	0\\
402.01	0\\
403.01	0\\
404.01	0\\
405.01	0\\
406.01	0\\
407.01	0\\
408.01	0\\
409.01	0\\
410.01	0\\
411.01	0\\
412.01	0\\
413.01	0\\
414.01	0\\
415.01	0\\
416.01	0\\
417.01	0\\
418.01	0\\
419.01	0\\
420.01	0\\
421.01	0\\
422.01	0\\
423.01	0\\
424.01	0\\
425.01	0\\
426.01	0\\
427.01	0\\
428.01	0\\
429.01	0\\
430.01	0\\
431.01	0\\
432.01	0\\
433.01	0\\
434.01	0\\
435.01	0\\
436.01	0\\
437.01	0\\
438.01	0\\
439.01	0\\
440.01	0\\
441.01	0\\
442.01	0\\
443.01	0\\
444.01	0\\
445.01	0\\
446.01	0\\
447.01	0\\
448.01	0\\
449.01	0\\
450.01	0\\
451.01	0\\
452.01	0\\
453.01	0\\
454.01	0\\
455.01	0\\
456.01	0\\
457.01	0\\
458.01	0\\
459.01	0\\
460.01	0\\
461.01	0\\
462.01	0\\
463.01	0\\
464.01	0\\
465.01	0\\
466.01	0\\
467.01	0\\
468.01	0\\
469.01	0\\
470.01	0\\
471.01	0\\
472.01	0\\
473.01	0\\
474.01	0\\
475.01	0\\
476.01	0\\
477.01	0\\
478.01	0\\
479.01	0\\
480.01	0\\
481.01	0\\
482.01	0\\
483.01	0\\
484.01	0\\
485.01	0\\
486.01	0\\
487.01	0\\
488.01	0\\
489.01	0\\
490.01	0\\
491.01	0\\
492.01	0\\
493.01	0\\
494.01	0\\
495.01	0\\
496.01	0\\
497.01	0\\
498.01	0\\
499.01	0\\
500.01	0\\
501.01	0\\
502.01	0\\
503.01	0\\
504.01	0\\
505.01	0\\
506.01	0\\
507.01	0\\
508.01	0\\
509.01	0\\
510.01	0\\
511.01	0\\
512.01	0\\
513.01	0\\
514.01	0\\
515.01	0\\
516.01	0\\
517.01	0\\
518.01	0\\
519.01	0\\
520.01	0\\
521.01	0\\
522.01	0\\
523.01	0\\
524.01	0\\
525.01	0\\
526.01	0\\
527.01	0\\
528.01	0\\
529.01	0\\
530.01	0\\
531.01	0\\
532.01	0\\
533.01	0\\
534.01	0\\
535.01	0\\
536.01	0\\
537.01	0\\
538.01	0\\
539.01	0\\
540.01	0\\
541.01	0\\
542.01	0\\
543.01	0\\
544.01	0\\
545.01	0\\
546.01	0\\
547.01	0\\
548.01	0\\
549.01	0\\
550.01	0\\
551.01	0\\
552.01	0\\
553.01	0\\
554.01	0\\
555.01	0\\
556.01	0\\
557.01	0\\
558.01	0\\
559.01	0\\
560.01	0\\
561.01	0\\
562.01	0\\
563.01	0\\
564.01	0\\
565.01	0\\
566.01	0\\
567.01	0\\
568.01	0\\
569.01	0\\
570.01	0\\
571.01	0\\
572.01	0\\
573.01	0\\
574.01	0\\
575.01	0\\
576.01	0\\
577.01	0\\
578.01	0\\
579.01	0\\
580.01	0\\
581.01	0\\
582.01	0.000121361606057675\\
583.01	0.00050917585157843\\
584.01	0.000923898505224239\\
585.01	0.00136483889801735\\
586.01	0.00181970423639984\\
587.01	0.00228837891239878\\
588.01	0.00277241611548145\\
589.01	0.00327301277390237\\
590.01	0.00379000974525\\
591.01	0.00432249455715031\\
592.01	0.00487168325353377\\
593.01	0.0054388733982834\\
594.01	0.00602544747888225\\
595.01	0.00663295416224377\\
596.01	0.00726313661607713\\
597.01	0.00791796813678225\\
598.01	0.00859969802437893\\
599.01	0.00930685224832755\\
599.02	0.00931395529900888\\
599.03	0.00932105763771179\\
599.04	0.00932815923365039\\
599.05	0.00933526005568012\\
599.06	0.0093423600722933\\
599.07	0.00934945925161451\\
599.08	0.00935655756139602\\
599.09	0.00936365496901308\\
599.1	0.00937075144145913\\
599.11	0.009377846945341\\
599.12	0.00938494144687396\\
599.13	0.00939203491187677\\
599.14	0.00939912730576658\\
599.15	0.00940621859355381\\
599.16	0.00941330873983692\\
599.17	0.00942039770879713\\
599.18	0.00942748546419297\\
599.19	0.00943457196935488\\
599.2	0.00944165718717962\\
599.21	0.00944874108012463\\
599.22	0.00945582361020228\\
599.23	0.00946290473897406\\
599.24	0.00946998442754468\\
599.25	0.00947706263655601\\
599.26	0.00948413932618099\\
599.27	0.00949121445611743\\
599.28	0.00949828798558168\\
599.29	0.00950535987330221\\
599.3	0.0095124300775131\\
599.31	0.00951949855594741\\
599.32	0.00952656526583043\\
599.33	0.00953363016387283\\
599.34	0.00954069320626369\\
599.35	0.00954775434866342\\
599.36	0.00955481354619657\\
599.37	0.00956187075344444\\
599.38	0.0095689259244377\\
599.39	0.00957597901264877\\
599.4	0.00958302997098412\\
599.41	0.0095900787517764\\
599.42	0.00959712530677647\\
599.43	0.0096041695871453\\
599.44	0.00961121154344564\\
599.45	0.00961825112563368\\
599.46	0.0096252882830519\\
599.47	0.00963232296442116\\
599.48	0.00963935511783186\\
599.49	0.00964638469073488\\
599.5	0.00965341162993244\\
599.51	0.00966043588156874\\
599.52	0.00966745739112043\\
599.53	0.0096744761033869\\
599.54	0.00968149196248044\\
599.55	0.00968850491181611\\
599.56	0.00969551489410154\\
599.57	0.00970252185132643\\
599.58	0.00970952572475191\\
599.59	0.00971652645489968\\
599.6	0.0097235239815409\\
599.61	0.00973051824368496\\
599.62	0.00973750917956788\\
599.63	0.00974449672664063\\
599.64	0.00975148082155711\\
599.65	0.00975846140016193\\
599.66	0.00976543839747794\\
599.67	0.00977241174769349\\
599.68	0.00977938138414944\\
599.69	0.0097863472393259\\
599.7	0.00979330924482867\\
599.71	0.00980026733137542\\
599.72	0.00980722142878157\\
599.73	0.00981417146594583\\
599.74	0.00982111737083552\\
599.75	0.00982805907047141\\
599.76	0.00983499649091242\\
599.77	0.00984192955723982\\
599.78	0.00984885819354115\\
599.79	0.00985578232289379\\
599.8	0.0098627018673481\\
599.81	0.00986961674791019\\
599.82	0.00987652688452436\\
599.83	0.009883432196055\\
599.84	0.00989033260026818\\
599.85	0.00989722801381275\\
599.86	0.00990411835220096\\
599.87	0.00991100352978864\\
599.88	0.00991788345975494\\
599.89	0.00992475805408144\\
599.9	0.0099316272235309\\
599.91	0.00993849087762531\\
599.92	0.0099453489246235\\
599.93	0.00995220127149813\\
599.94	0.00995904782391208\\
599.95	0.00996588848619419\\
599.96	0.00997272316131441\\
599.97	0.00997955175085828\\
599.98	0.00998637415500062\\
599.99	0.00999319027247863\\
600	0.01\\
};
\addplot [color=mycolor14,solid,forget plot]
  table[row sep=crcr]{%
0.01	0.01\\
1.01	0.01\\
2.01	0.01\\
3.01	0.01\\
4.01	0.01\\
5.01	0.01\\
6.01	0.01\\
7.01	0.01\\
8.01	0.01\\
9.01	0.01\\
10.01	0.01\\
11.01	0.01\\
12.01	0.01\\
13.01	0.01\\
14.01	0.01\\
15.01	0.01\\
16.01	0.01\\
17.01	0.01\\
18.01	0.01\\
19.01	0.01\\
20.01	0.01\\
21.01	0.01\\
22.01	0.01\\
23.01	0.01\\
24.01	0.01\\
25.01	0.01\\
26.01	0.01\\
27.01	0.01\\
28.01	0.01\\
29.01	0.01\\
30.01	0.01\\
31.01	0.01\\
32.01	0.01\\
33.01	0.01\\
34.01	0.01\\
35.01	0.01\\
36.01	0.01\\
37.01	0.01\\
38.01	0.01\\
39.01	0.01\\
40.01	0.01\\
41.01	0.01\\
42.01	0.01\\
43.01	0.01\\
44.01	0.01\\
45.01	0.01\\
46.01	0.01\\
47.01	0.01\\
48.01	0.01\\
49.01	0.01\\
50.01	0.01\\
51.01	0.01\\
52.01	0.01\\
53.01	0.01\\
54.01	0.01\\
55.01	0.01\\
56.01	0.01\\
57.01	0.01\\
58.01	0.01\\
59.01	0.01\\
60.01	0.01\\
61.01	0.01\\
62.01	0.01\\
63.01	0.01\\
64.01	0.01\\
65.01	0.01\\
66.01	0.01\\
67.01	0.01\\
68.01	0.01\\
69.01	0.01\\
70.01	0.01\\
71.01	0.01\\
72.01	0.01\\
73.01	0.01\\
74.01	0.01\\
75.01	0.01\\
76.01	0.01\\
77.01	0.01\\
78.01	0.01\\
79.01	0.01\\
80.01	0.01\\
81.01	0.01\\
82.01	0.01\\
83.01	0.01\\
84.01	0.01\\
85.01	0.01\\
86.01	0.01\\
87.01	0.01\\
88.01	0.01\\
89.01	0.01\\
90.01	0.01\\
91.01	0.01\\
92.01	0.01\\
93.01	0.01\\
94.01	0.01\\
95.01	0.01\\
96.01	0.01\\
97.01	0.01\\
98.01	0.01\\
99.01	0.01\\
100.01	0.01\\
101.01	0.01\\
102.01	0.01\\
103.01	0.01\\
104.01	0.01\\
105.01	0.01\\
106.01	0.01\\
107.01	0.01\\
108.01	0.01\\
109.01	0.01\\
110.01	0.01\\
111.01	0.01\\
112.01	0.01\\
113.01	0.01\\
114.01	0.01\\
115.01	0.01\\
116.01	0.01\\
117.01	0.01\\
118.01	0.01\\
119.01	0.01\\
120.01	0.01\\
121.01	0.01\\
122.01	0.01\\
123.01	0.01\\
124.01	0.01\\
125.01	0.01\\
126.01	0.01\\
127.01	0.01\\
128.01	0.01\\
129.01	0.01\\
130.01	0.01\\
131.01	0.01\\
132.01	0.01\\
133.01	0.01\\
134.01	0.01\\
135.01	0.01\\
136.01	0.01\\
137.01	0.01\\
138.01	0.01\\
139.01	0.01\\
140.01	0.01\\
141.01	0.01\\
142.01	0.01\\
143.01	0.01\\
144.01	0.01\\
145.01	0.01\\
146.01	0.01\\
147.01	0.01\\
148.01	0.01\\
149.01	0.01\\
150.01	0.01\\
151.01	0.01\\
152.01	0.01\\
153.01	0.01\\
154.01	0.01\\
155.01	0.01\\
156.01	0.01\\
157.01	0.01\\
158.01	0.01\\
159.01	0.01\\
160.01	0.01\\
161.01	0.01\\
162.01	0.01\\
163.01	0.01\\
164.01	0.01\\
165.01	0.01\\
166.01	0.01\\
167.01	0.01\\
168.01	0.01\\
169.01	0.01\\
170.01	0.01\\
171.01	0.01\\
172.01	0.01\\
173.01	0.01\\
174.01	0.01\\
175.01	0.01\\
176.01	0.01\\
177.01	0.01\\
178.01	0.01\\
179.01	0.01\\
180.01	0.01\\
181.01	0.01\\
182.01	0.01\\
183.01	0.01\\
184.01	0.01\\
185.01	0.01\\
186.01	0.01\\
187.01	0.01\\
188.01	0.01\\
189.01	0.01\\
190.01	0.01\\
191.01	0.01\\
192.01	0.01\\
193.01	0.01\\
194.01	0.01\\
195.01	0.01\\
196.01	0.01\\
197.01	0.01\\
198.01	0.01\\
199.01	0.01\\
200.01	0.01\\
201.01	0.01\\
202.01	0.01\\
203.01	0.01\\
204.01	0.01\\
205.01	0.01\\
206.01	0.01\\
207.01	0.01\\
208.01	0.01\\
209.01	0.01\\
210.01	0.01\\
211.01	0.01\\
212.01	0.01\\
213.01	0.01\\
214.01	0.01\\
215.01	0.01\\
216.01	0.01\\
217.01	0.01\\
218.01	0.01\\
219.01	0.01\\
220.01	0.01\\
221.01	0.01\\
222.01	0.01\\
223.01	0.01\\
224.01	0.01\\
225.01	0.01\\
226.01	0.01\\
227.01	0.01\\
228.01	0.01\\
229.01	0.01\\
230.01	0.01\\
231.01	0.01\\
232.01	0.01\\
233.01	0.01\\
234.01	0.01\\
235.01	0.01\\
236.01	0.01\\
237.01	0.01\\
238.01	0.01\\
239.01	0.01\\
240.01	0.01\\
241.01	0.01\\
242.01	0.01\\
243.01	0.01\\
244.01	0.01\\
245.01	0.01\\
246.01	0.01\\
247.01	0.01\\
248.01	0.01\\
249.01	0.01\\
250.01	0.01\\
251.01	0.01\\
252.01	0.01\\
253.01	0.01\\
254.01	0.01\\
255.01	0.01\\
256.01	0.01\\
257.01	0.01\\
258.01	0.01\\
259.01	0.01\\
260.01	0.01\\
261.01	0.01\\
262.01	0.01\\
263.01	0.01\\
264.01	0.01\\
265.01	0.01\\
266.01	0.01\\
267.01	0.01\\
268.01	0.01\\
269.01	0.01\\
270.01	0.01\\
271.01	0.01\\
272.01	0.01\\
273.01	0.01\\
274.01	0.01\\
275.01	0.01\\
276.01	0.01\\
277.01	0.01\\
278.01	0.01\\
279.01	0.01\\
280.01	0.01\\
281.01	0.01\\
282.01	0.01\\
283.01	0.01\\
284.01	0.01\\
285.01	0.01\\
286.01	0.01\\
287.01	0.01\\
288.01	0.01\\
289.01	0.01\\
290.01	0.01\\
291.01	0.01\\
292.01	0.01\\
293.01	0.01\\
294.01	0.01\\
295.01	0.01\\
296.01	0.01\\
297.01	0.01\\
298.01	0.01\\
299.01	0.01\\
300.01	0.01\\
301.01	0.01\\
302.01	0.01\\
303.01	0.01\\
304.01	0.01\\
305.01	0.01\\
306.01	0.01\\
307.01	0.01\\
308.01	0.01\\
309.01	0.01\\
310.01	0.01\\
311.01	0.01\\
312.01	0.01\\
313.01	0.01\\
314.01	0.01\\
315.01	0.01\\
316.01	0.01\\
317.01	0.01\\
318.01	0.01\\
319.01	0.01\\
320.01	0.01\\
321.01	0.01\\
322.01	0.01\\
323.01	0.01\\
324.01	0.01\\
325.01	0.01\\
326.01	0.01\\
327.01	0.01\\
328.01	0.01\\
329.01	0.01\\
330.01	0.01\\
331.01	0.01\\
332.01	0.01\\
333.01	0.01\\
334.01	0.01\\
335.01	0.01\\
336.01	0.01\\
337.01	0.01\\
338.01	0.01\\
339.01	0.01\\
340.01	0.01\\
341.01	0.01\\
342.01	0.01\\
343.01	0.01\\
344.01	0.01\\
345.01	0.01\\
346.01	0.01\\
347.01	0.01\\
348.01	0.01\\
349.01	0.01\\
350.01	0.01\\
351.01	0.01\\
352.01	0.01\\
353.01	0.01\\
354.01	0.01\\
355.01	0.01\\
356.01	0.01\\
357.01	0.01\\
358.01	0.01\\
359.01	0.01\\
360.01	0.01\\
361.01	0.01\\
362.01	0.01\\
363.01	0.01\\
364.01	0.01\\
365.01	0.01\\
366.01	0.01\\
367.01	0.01\\
368.01	0.01\\
369.01	0.01\\
370.01	0.01\\
371.01	0.01\\
372.01	0.01\\
373.01	0.01\\
374.01	0.01\\
375.01	0.01\\
376.01	0.01\\
377.01	0.01\\
378.01	0.01\\
379.01	0.01\\
380.01	0.01\\
381.01	0.01\\
382.01	0.01\\
383.01	0.01\\
384.01	0.01\\
385.01	0.01\\
386.01	0.01\\
387.01	0.01\\
388.01	0.01\\
389.01	0.01\\
390.01	0.01\\
391.01	0.01\\
392.01	0.01\\
393.01	0.01\\
394.01	0.01\\
395.01	0.01\\
396.01	0.01\\
397.01	0.01\\
398.01	0.01\\
399.01	0.01\\
400.01	0.01\\
401.01	0.01\\
402.01	0.01\\
403.01	0.01\\
404.01	0.01\\
405.01	0.01\\
406.01	0.01\\
407.01	0.01\\
408.01	0.01\\
409.01	0.01\\
410.01	0.01\\
411.01	0.01\\
412.01	0.01\\
413.01	0.01\\
414.01	0.01\\
415.01	0.01\\
416.01	0.01\\
417.01	0.01\\
418.01	0.01\\
419.01	0.01\\
420.01	0.01\\
421.01	0.01\\
422.01	0.01\\
423.01	0.01\\
424.01	0.01\\
425.01	0.01\\
426.01	0.01\\
427.01	0.01\\
428.01	0.01\\
429.01	0.01\\
430.01	0.01\\
431.01	0.01\\
432.01	0.01\\
433.01	0.01\\
434.01	0.01\\
435.01	0.01\\
436.01	0.01\\
437.01	0.01\\
438.01	0.01\\
439.01	0.01\\
440.01	0.01\\
441.01	0.01\\
442.01	0.01\\
443.01	0.01\\
444.01	0.01\\
445.01	0.01\\
446.01	0.01\\
447.01	0.01\\
448.01	0.01\\
449.01	0.01\\
450.01	0.01\\
451.01	0.01\\
452.01	0.01\\
453.01	0.01\\
454.01	0.01\\
455.01	0.01\\
456.01	0.01\\
457.01	0.01\\
458.01	0.01\\
459.01	0.01\\
460.01	0.01\\
461.01	0.01\\
462.01	0.01\\
463.01	0.01\\
464.01	0.01\\
465.01	0.01\\
466.01	0.01\\
467.01	0.01\\
468.01	0.01\\
469.01	0.01\\
470.01	0.01\\
471.01	0.01\\
472.01	0.01\\
473.01	0.01\\
474.01	0.01\\
475.01	0.01\\
476.01	0.01\\
477.01	0.01\\
478.01	0.01\\
479.01	0.01\\
480.01	0.01\\
481.01	0.01\\
482.01	0.01\\
483.01	0.01\\
484.01	0.01\\
485.01	0.01\\
486.01	0.01\\
487.01	0.01\\
488.01	0.01\\
489.01	0.01\\
490.01	0.01\\
491.01	0.01\\
492.01	0.01\\
493.01	0.01\\
494.01	0.01\\
495.01	0.01\\
496.01	0.01\\
497.01	0.01\\
498.01	0.01\\
499.01	0.01\\
500.01	0.01\\
501.01	0.01\\
502.01	0.01\\
503.01	0.01\\
504.01	0.01\\
505.01	0.01\\
506.01	0.01\\
507.01	0.01\\
508.01	0.01\\
509.01	0.01\\
510.01	0.01\\
511.01	0.01\\
512.01	0.01\\
513.01	0.01\\
514.01	0.01\\
515.01	0.01\\
516.01	0.01\\
517.01	0.01\\
518.01	0.01\\
519.01	0.01\\
520.01	0.01\\
521.01	0.01\\
522.01	0.01\\
523.01	0.01\\
524.01	0.01\\
525.01	0.01\\
526.01	0.01\\
527.01	0.01\\
528.01	0.01\\
529.01	0.01\\
530.01	0.01\\
531.01	0.01\\
532.01	0.01\\
533.01	0.01\\
534.01	0.01\\
535.01	0.01\\
536.01	0.01\\
537.01	0.01\\
538.01	0.01\\
539.01	0.01\\
540.01	0.01\\
541.01	0.01\\
542.01	0.01\\
543.01	0.01\\
544.01	0.01\\
545.01	0.01\\
546.01	0.01\\
547.01	0.01\\
548.01	0.01\\
549.01	0.01\\
550.01	0.01\\
551.01	0.01\\
552.01	0.01\\
553.01	0.01\\
554.01	0.01\\
555.01	0.01\\
556.01	0.01\\
557.01	0.01\\
558.01	0.01\\
559.01	0.01\\
560.01	0.01\\
561.01	0.01\\
562.01	0.01\\
563.01	0.01\\
564.01	0.01\\
565.01	0.01\\
566.01	0.01\\
567.01	0.01\\
568.01	0.01\\
569.01	0.01\\
570.01	0.01\\
571.01	0.01\\
572.01	0.01\\
573.01	0.01\\
574.01	0.01\\
575.01	0.01\\
576.01	0.01\\
577.01	0.01\\
578.01	0.01\\
579.01	0.01\\
580.01	0.01\\
581.01	0.01\\
582.01	0.00992989647703613\\
583.01	0.00954177372982102\\
584.01	0.00912547726397663\\
585.01	0.00868439512189853\\
586.01	0.00822984681208113\\
587.01	0.00776098456227372\\
588.01	0.00727618774155096\\
589.01	0.00677424947721351\\
590.01	0.00625539153484212\\
591.01	0.00572055713570469\\
592.01	0.00516858725663456\\
593.01	0.00459821690406403\\
594.01	0.00400803771374159\\
595.01	0.00339646929396734\\
596.01	0.00276172792797194\\
597.01	0.00210178615661351\\
598.01	0.00141431937259587\\
599.01	0.000700643040783771\\
599.02	0.000693469184667535\\
599.03	0.00068629596350653\\
599.04	0.000679123407609256\\
599.05	0.000671951547635673\\
599.06	0.000664780414601591\\
599.07	0.000657610039883112\\
599.08	0.00065044045522116\\
599.09	0.00064327169272606\\
599.1	0.000636103784882198\\
599.11	0.000628936764552753\\
599.12	0.000621770664984485\\
599.13	0.000614605519812619\\
599.14	0.000607441363065791\\
599.15	0.000600278229171067\\
599.16	0.000593116152959046\\
599.17	0.000585955169669047\\
599.18	0.000578795314954366\\
599.19	0.000571636624887621\\
599.2	0.000564479135966182\\
599.21	0.000557322885117686\\
599.22	0.000550167909705638\\
599.23	0.000543014247535096\\
599.24	0.000535861936858464\\
599.25	0.000528711016381361\\
599.26	0.000521561525268591\\
599.27	0.000514413503150207\\
599.28	0.000507266990127684\\
599.29	0.000500122026780177\\
599.3	0.0004929786541709\\
599.31	0.000485836913853588\\
599.32	0.000478696847879087\\
599.33	0.000471558498802045\\
599.34	0.000464421909687714\\
599.35	0.00045728712411886\\
599.36	0.000450154186202811\\
599.37	0.000443023140578602\\
599.38	0.000435894032424248\\
599.39	0.000428766907464151\\
599.4	0.000421641811976625\\
599.41	0.000414518792801556\\
599.42	0.000407397897348187\\
599.43	0.000400279173603059\\
599.44	0.000393162670138064\\
599.45	0.000386048436118664\\
599.46	0.000378936521312195\\
599.47	0.000371826976094117\\
599.48	0.000364719851456648\\
599.49	0.000357615199017553\\
599.5	0.000350513071029101\\
599.51	0.000343413520387173\\
599.52	0.000336316600640549\\
599.53	0.000329222366000361\\
599.54	0.000322130871349731\\
599.55	0.000315042172253572\\
599.56	0.000307956324968602\\
599.57	0.000300873386453522\\
599.58	0.000293793414379399\\
599.59	0.000286716467140258\\
599.6	0.000279642603863864\\
599.61	0.000272571884422727\\
599.62	0.000265504369445314\\
599.63	0.000258440120327493\\
599.64	0.000251379199244197\\
599.65	0.000244321669161319\\
599.66	0.00023726759384787\\
599.67	0.000230217037888347\\
599.68	0.000223170066695397\\
599.69	0.000216126746522712\\
599.7	0.000209087144478199\\
599.71	0.00020205132853744\\
599.72	0.000195019367557427\\
599.73	0.000187991331290582\\
599.74	0.000180967290399096\\
599.75	0.000173947316469567\\
599.76	0.000166931482027961\\
599.77	0.000159919860554901\\
599.78	0.000152912526501302\\
599.79	0.000145909555304351\\
599.8	0.00013891102340384\\
599.81	0.000131917008258898\\
599.82	0.000124927588365074\\
599.83	0.000117942843271846\\
599.84	0.000110962853600519\\
599.85	0.000103987701062559\\
599.86	9.70174684783599e-05\\
599.87	9.00522397964624e-05\\
599.88	8.30921001132405e-05\\
599.89	7.61371356930683e-05\\
599.9	6.91874339889959e-05\\
599.91	6.22430836639334e-05\\
599.92	5.53041746123722e-05\\
599.93	4.83707979826668e-05\\
599.94	4.14430461998863e-05\\
599.95	3.45210129892633e-05\\
599.96	2.76047934002557e-05\\
599.97	2.06944838312597e-05\\
599.98	1.37901820549801e-05\\
599.99	6.891987244486e-06\\
600	0\\
};
\addplot [color=mycolor15,solid,forget plot]
  table[row sep=crcr]{%
0.01	0.01\\
1.01	0.01\\
2.01	0.01\\
3.01	0.01\\
4.01	0.01\\
5.01	0.01\\
6.01	0.01\\
7.01	0.01\\
8.01	0.01\\
9.01	0.01\\
10.01	0.01\\
11.01	0.01\\
12.01	0.01\\
13.01	0.01\\
14.01	0.01\\
15.01	0.01\\
16.01	0.01\\
17.01	0.01\\
18.01	0.01\\
19.01	0.01\\
20.01	0.01\\
21.01	0.01\\
22.01	0.01\\
23.01	0.01\\
24.01	0.01\\
25.01	0.01\\
26.01	0.01\\
27.01	0.01\\
28.01	0.01\\
29.01	0.01\\
30.01	0.01\\
31.01	0.01\\
32.01	0.01\\
33.01	0.01\\
34.01	0.01\\
35.01	0.01\\
36.01	0.01\\
37.01	0.01\\
38.01	0.01\\
39.01	0.01\\
40.01	0.01\\
41.01	0.01\\
42.01	0.01\\
43.01	0.01\\
44.01	0.01\\
45.01	0.01\\
46.01	0.01\\
47.01	0.01\\
48.01	0.01\\
49.01	0.01\\
50.01	0.01\\
51.01	0.01\\
52.01	0.01\\
53.01	0.01\\
54.01	0.01\\
55.01	0.01\\
56.01	0.01\\
57.01	0.01\\
58.01	0.01\\
59.01	0.01\\
60.01	0.01\\
61.01	0.01\\
62.01	0.01\\
63.01	0.01\\
64.01	0.01\\
65.01	0.01\\
66.01	0.01\\
67.01	0.01\\
68.01	0.01\\
69.01	0.01\\
70.01	0.01\\
71.01	0.01\\
72.01	0.01\\
73.01	0.01\\
74.01	0.01\\
75.01	0.01\\
76.01	0.01\\
77.01	0.01\\
78.01	0.01\\
79.01	0.01\\
80.01	0.01\\
81.01	0.01\\
82.01	0.01\\
83.01	0.01\\
84.01	0.01\\
85.01	0.01\\
86.01	0.01\\
87.01	0.01\\
88.01	0.01\\
89.01	0.01\\
90.01	0.01\\
91.01	0.01\\
92.01	0.01\\
93.01	0.01\\
94.01	0.01\\
95.01	0.01\\
96.01	0.01\\
97.01	0.01\\
98.01	0.01\\
99.01	0.01\\
100.01	0.01\\
101.01	0.01\\
102.01	0.01\\
103.01	0.01\\
104.01	0.01\\
105.01	0.01\\
106.01	0.01\\
107.01	0.01\\
108.01	0.01\\
109.01	0.01\\
110.01	0.01\\
111.01	0.01\\
112.01	0.01\\
113.01	0.01\\
114.01	0.01\\
115.01	0.01\\
116.01	0.01\\
117.01	0.01\\
118.01	0.01\\
119.01	0.01\\
120.01	0.01\\
121.01	0.01\\
122.01	0.01\\
123.01	0.01\\
124.01	0.01\\
125.01	0.01\\
126.01	0.01\\
127.01	0.01\\
128.01	0.01\\
129.01	0.01\\
130.01	0.01\\
131.01	0.01\\
132.01	0.01\\
133.01	0.01\\
134.01	0.01\\
135.01	0.01\\
136.01	0.01\\
137.01	0.01\\
138.01	0.01\\
139.01	0.01\\
140.01	0.01\\
141.01	0.01\\
142.01	0.01\\
143.01	0.01\\
144.01	0.01\\
145.01	0.01\\
146.01	0.01\\
147.01	0.01\\
148.01	0.01\\
149.01	0.01\\
150.01	0.01\\
151.01	0.01\\
152.01	0.01\\
153.01	0.01\\
154.01	0.01\\
155.01	0.01\\
156.01	0.01\\
157.01	0.01\\
158.01	0.01\\
159.01	0.01\\
160.01	0.01\\
161.01	0.01\\
162.01	0.01\\
163.01	0.01\\
164.01	0.01\\
165.01	0.01\\
166.01	0.01\\
167.01	0.01\\
168.01	0.01\\
169.01	0.01\\
170.01	0.01\\
171.01	0.01\\
172.01	0.01\\
173.01	0.01\\
174.01	0.01\\
175.01	0.01\\
176.01	0.01\\
177.01	0.01\\
178.01	0.01\\
179.01	0.01\\
180.01	0.01\\
181.01	0.01\\
182.01	0.01\\
183.01	0.01\\
184.01	0.01\\
185.01	0.01\\
186.01	0.01\\
187.01	0.01\\
188.01	0.01\\
189.01	0.01\\
190.01	0.01\\
191.01	0.01\\
192.01	0.01\\
193.01	0.01\\
194.01	0.01\\
195.01	0.01\\
196.01	0.01\\
197.01	0.01\\
198.01	0.01\\
199.01	0.01\\
200.01	0.01\\
201.01	0.01\\
202.01	0.01\\
203.01	0.01\\
204.01	0.01\\
205.01	0.01\\
206.01	0.01\\
207.01	0.01\\
208.01	0.01\\
209.01	0.01\\
210.01	0.01\\
211.01	0.01\\
212.01	0.01\\
213.01	0.01\\
214.01	0.01\\
215.01	0.01\\
216.01	0.01\\
217.01	0.01\\
218.01	0.01\\
219.01	0.01\\
220.01	0.01\\
221.01	0.01\\
222.01	0.01\\
223.01	0.01\\
224.01	0.01\\
225.01	0.01\\
226.01	0.01\\
227.01	0.01\\
228.01	0.01\\
229.01	0.01\\
230.01	0.01\\
231.01	0.01\\
232.01	0.01\\
233.01	0.01\\
234.01	0.01\\
235.01	0.01\\
236.01	0.01\\
237.01	0.01\\
238.01	0.01\\
239.01	0.01\\
240.01	0.01\\
241.01	0.01\\
242.01	0.01\\
243.01	0.01\\
244.01	0.01\\
245.01	0.01\\
246.01	0.01\\
247.01	0.01\\
248.01	0.01\\
249.01	0.01\\
250.01	0.01\\
251.01	0.01\\
252.01	0.01\\
253.01	0.01\\
254.01	0.01\\
255.01	0.01\\
256.01	0.01\\
257.01	0.01\\
258.01	0.01\\
259.01	0.01\\
260.01	0.01\\
261.01	0.01\\
262.01	0.01\\
263.01	0.01\\
264.01	0.01\\
265.01	0.01\\
266.01	0.01\\
267.01	0.01\\
268.01	0.01\\
269.01	0.01\\
270.01	0.01\\
271.01	0.01\\
272.01	0.01\\
273.01	0.01\\
274.01	0.01\\
275.01	0.01\\
276.01	0.01\\
277.01	0.01\\
278.01	0.01\\
279.01	0.01\\
280.01	0.01\\
281.01	0.01\\
282.01	0.01\\
283.01	0.01\\
284.01	0.01\\
285.01	0.01\\
286.01	0.01\\
287.01	0.01\\
288.01	0.01\\
289.01	0.01\\
290.01	0.01\\
291.01	0.01\\
292.01	0.01\\
293.01	0.01\\
294.01	0.01\\
295.01	0.01\\
296.01	0.01\\
297.01	0.01\\
298.01	0.01\\
299.01	0.01\\
300.01	0.01\\
301.01	0.01\\
302.01	0.01\\
303.01	0.01\\
304.01	0.01\\
305.01	0.01\\
306.01	0.01\\
307.01	0.01\\
308.01	0.01\\
309.01	0.01\\
310.01	0.01\\
311.01	0.01\\
312.01	0.01\\
313.01	0.01\\
314.01	0.01\\
315.01	0.01\\
316.01	0.01\\
317.01	0.01\\
318.01	0.01\\
319.01	0.01\\
320.01	0.01\\
321.01	0.01\\
322.01	0.01\\
323.01	0.01\\
324.01	0.01\\
325.01	0.01\\
326.01	0.01\\
327.01	0.01\\
328.01	0.01\\
329.01	0.01\\
330.01	0.01\\
331.01	0.01\\
332.01	0.01\\
333.01	0.01\\
334.01	0.01\\
335.01	0.01\\
336.01	0.01\\
337.01	0.01\\
338.01	0.01\\
339.01	0.01\\
340.01	0.01\\
341.01	0.01\\
342.01	0.01\\
343.01	0.01\\
344.01	0.01\\
345.01	0.01\\
346.01	0.01\\
347.01	0.01\\
348.01	0.01\\
349.01	0.01\\
350.01	0.01\\
351.01	0.01\\
352.01	0.01\\
353.01	0.01\\
354.01	0.01\\
355.01	0.01\\
356.01	0.01\\
357.01	0.01\\
358.01	0.01\\
359.01	0.01\\
360.01	0.01\\
361.01	0.01\\
362.01	0.01\\
363.01	0.01\\
364.01	0.01\\
365.01	0.01\\
366.01	0.01\\
367.01	0.01\\
368.01	0.01\\
369.01	0.01\\
370.01	0.01\\
371.01	0.01\\
372.01	0.01\\
373.01	0.01\\
374.01	0.01\\
375.01	0.01\\
376.01	0.01\\
377.01	0.01\\
378.01	0.01\\
379.01	0.01\\
380.01	0.01\\
381.01	0.01\\
382.01	0.01\\
383.01	0.01\\
384.01	0.01\\
385.01	0.01\\
386.01	0.01\\
387.01	0.01\\
388.01	0.01\\
389.01	0.01\\
390.01	0.01\\
391.01	0.01\\
392.01	0.01\\
393.01	0.01\\
394.01	0.01\\
395.01	0.01\\
396.01	0.01\\
397.01	0.01\\
398.01	0.01\\
399.01	0.01\\
400.01	0.01\\
401.01	0.01\\
402.01	0.01\\
403.01	0.01\\
404.01	0.01\\
405.01	0.01\\
406.01	0.01\\
407.01	0.01\\
408.01	0.01\\
409.01	0.01\\
410.01	0.01\\
411.01	0.01\\
412.01	0.01\\
413.01	0.01\\
414.01	0.01\\
415.01	0.01\\
416.01	0.01\\
417.01	0.01\\
418.01	0.01\\
419.01	0.01\\
420.01	0.01\\
421.01	0.01\\
422.01	0.01\\
423.01	0.01\\
424.01	0.01\\
425.01	0.01\\
426.01	0.01\\
427.01	0.01\\
428.01	0.01\\
429.01	0.01\\
430.01	0.01\\
431.01	0.01\\
432.01	0.01\\
433.01	0.01\\
434.01	0.01\\
435.01	0.01\\
436.01	0.01\\
437.01	0.01\\
438.01	0.01\\
439.01	0.01\\
440.01	0.01\\
441.01	0.01\\
442.01	0.01\\
443.01	0.01\\
444.01	0.01\\
445.01	0.01\\
446.01	0.01\\
447.01	0.01\\
448.01	0.01\\
449.01	0.01\\
450.01	0.01\\
451.01	0.01\\
452.01	0.01\\
453.01	0.01\\
454.01	0.01\\
455.01	0.01\\
456.01	0.01\\
457.01	0.01\\
458.01	0.01\\
459.01	0.01\\
460.01	0.01\\
461.01	0.01\\
462.01	0.01\\
463.01	0.01\\
464.01	0.01\\
465.01	0.01\\
466.01	0.01\\
467.01	0.01\\
468.01	0.01\\
469.01	0.01\\
470.01	0.01\\
471.01	0.01\\
472.01	0.01\\
473.01	0.01\\
474.01	0.01\\
475.01	0.01\\
476.01	0.01\\
477.01	0.01\\
478.01	0.01\\
479.01	0.01\\
480.01	0.01\\
481.01	0.01\\
482.01	0.01\\
483.01	0.01\\
484.01	0.01\\
485.01	0.01\\
486.01	0.01\\
487.01	0.01\\
488.01	0.01\\
489.01	0.01\\
490.01	0.01\\
491.01	0.01\\
492.01	0.01\\
493.01	0.01\\
494.01	0.01\\
495.01	0.01\\
496.01	0.01\\
497.01	0.01\\
498.01	0.01\\
499.01	0.01\\
500.01	0.01\\
501.01	0.01\\
502.01	0.01\\
503.01	0.01\\
504.01	0.01\\
505.01	0.01\\
506.01	0.01\\
507.01	0.01\\
508.01	0.01\\
509.01	0.01\\
510.01	0.01\\
511.01	0.01\\
512.01	0.01\\
513.01	0.01\\
514.01	0.01\\
515.01	0.01\\
516.01	0.01\\
517.01	0.01\\
518.01	0.01\\
519.01	0.01\\
520.01	0.01\\
521.01	0.01\\
522.01	0.01\\
523.01	0.01\\
524.01	0.01\\
525.01	0.01\\
526.01	0.01\\
527.01	0.01\\
528.01	0.01\\
529.01	0.01\\
530.01	0.01\\
531.01	0.01\\
532.01	0.01\\
533.01	0.01\\
534.01	0.01\\
535.01	0.01\\
536.01	0.01\\
537.01	0.01\\
538.01	0.01\\
539.01	0.01\\
540.01	0.01\\
541.01	0.01\\
542.01	0.01\\
543.01	0.01\\
544.01	0.01\\
545.01	0.01\\
546.01	0.01\\
547.01	0.01\\
548.01	0.01\\
549.01	0.01\\
550.01	0.01\\
551.01	0.01\\
552.01	0.01\\
553.01	0.01\\
554.01	0.01\\
555.01	0.01\\
556.01	0.01\\
557.01	0.01\\
558.01	0.01\\
559.01	0.01\\
560.01	0.01\\
561.01	0.01\\
562.01	0.01\\
563.01	0.01\\
564.01	0.01\\
565.01	0.01\\
566.01	0.01\\
567.01	0.01\\
568.01	0.00972744916310699\\
569.01	0.00942656800190368\\
570.01	0.00910533817373198\\
571.01	0.00876083290411074\\
572.01	0.00838953411780288\\
573.01	0.00799085123339294\\
574.01	0.00757422759961413\\
575.01	0.00713929514049568\\
576.01	0.00668456449300758\\
577.01	0.0062084103816373\\
578.01	0.00570906766945888\\
579.01	0.00518463758754371\\
580.01	0.00463312160520521\\
581.01	0.00405253293841158\\
582.01	0.00351244908859151\\
583.01	0.00326931276710382\\
584.01	0.00303822528933792\\
585.01	0.00281707885798722\\
586.01	0.00259498529778048\\
587.01	0.00237242508948119\\
588.01	0.00215029934370932\\
589.01	0.00192954919884637\\
590.01	0.00171025695550095\\
591.01	0.00149175973833009\\
592.01	0.00127554651852854\\
593.01	0.00106347889233958\\
594.01	0.000857764094569244\\
595.01	0.000660999733626294\\
596.01	0.000476218771598161\\
597.01	0.000306932213913688\\
598.01	0.000157169919638113\\
599.01	4.57135485354369e-05\\
599.02	4.48827722135618e-05\\
599.03	4.40580970898427e-05\\
599.04	4.32395580149871e-05\\
599.05	4.2427190131129e-05\\
599.06	4.16210288745004e-05\\
599.07	4.08211099781359e-05\\
599.08	4.00274694745805e-05\\
599.09	3.92401436986305e-05\\
599.1	3.84591692900969e-05\\
599.11	3.76845831965792e-05\\
599.12	3.69164226762652e-05\\
599.13	3.61547253007622e-05\\
599.14	3.53995289579228e-05\\
599.15	3.46508718547297e-05\\
599.16	3.39087925201615e-05\\
599.17	3.31733298081088e-05\\
599.18	3.24445229002955e-05\\
599.19	3.17224113092294e-05\\
599.2	3.100703488116e-05\\
599.21	3.02984337990763e-05\\
599.22	2.95966485857075e-05\\
599.23	2.89017201065538e-05\\
599.24	2.82136895729344e-05\\
599.25	2.75325985450488e-05\\
599.26	2.68584889344392e-05\\
599.27	2.61914030062933e-05\\
599.28	2.55313833825432e-05\\
599.29	2.48784730449945e-05\\
599.3	2.4232715338468e-05\\
599.31	2.35941539739638e-05\\
599.32	2.29628330318411e-05\\
599.33	2.23387969650098e-05\\
599.34	2.17220906021491e-05\\
599.35	2.11127591509385e-05\\
599.36	2.05108482012935e-05\\
599.37	1.99164037286426e-05\\
599.38	1.93294720971914e-05\\
599.39	1.87501000632195e-05\\
599.4	1.81783347783862e-05\\
599.41	1.76142237930405e-05\\
599.42	1.70578150595693e-05\\
599.43	1.65091569357192e-05\\
599.44	1.59682981879673e-05\\
599.45	1.54352879948774e-05\\
599.46	1.49101760638821e-05\\
599.47	1.43930175409875e-05\\
599.48	1.38838680450361e-05\\
599.49	1.33827836715731e-05\\
599.5	1.28898209967015e-05\\
599.51	1.24050370809564e-05\\
599.52	1.1928489473189e-05\\
599.53	1.14602362144486e-05\\
599.54	1.10003358418636e-05\\
599.55	1.05488473925511e-05\\
599.56	1.01058304074926e-05\\
599.57	9.67134493544235e-06\\
599.58	9.24545153681115e-06\\
599.59	8.82821128755755e-06\\
599.6	8.41968578306831e-06\\
599.61	8.01993714203558e-06\\
599.62	7.62902801032875e-06\\
599.63	7.24702156483861e-06\\
599.64	6.87398151732153e-06\\
599.65	6.50997211822796e-06\\
599.66	6.15505816049279e-06\\
599.67	5.80930498333791e-06\\
599.68	5.47277847600704e-06\\
599.69	5.14554508150751e-06\\
599.7	4.82767180030874e-06\\
599.71	4.51922619398859e-06\\
599.72	4.22027638887629e-06\\
599.73	3.93089107961382e-06\\
599.74	3.65113953270692e-06\\
599.75	3.38109158999275e-06\\
599.76	3.12081767206776e-06\\
599.77	2.87038878166342e-06\\
599.78	2.62987650693179e-06\\
599.79	2.39935302467388e-06\\
599.8	2.17889110350374e-06\\
599.81	1.96856410689117e-06\\
599.82	1.76844599615936e-06\\
599.83	1.57861133334887e-06\\
599.84	1.3991352839967e-06\\
599.85	1.23009361979905e-06\\
599.86	1.07156272114578e-06\\
599.87	9.23619579542082e-07\\
599.88	7.8634179987401e-07\\
599.89	6.59807602540474e-07\\
599.9	5.44095825409999e-07\\
599.91	4.39285925628308e-07\\
599.92	3.45457981228148e-07\\
599.93	2.62692692548291e-07\\
599.94	1.91071383456518e-07\\
599.95	1.30676002336669e-07\\
599.96	8.158912285193e-08\\
599.97	4.38939444496328e-08\\
599.98	1.76742926006473e-08\\
599.99	3.01461875948372e-09\\
600	0\\
};
\addplot [color=mycolor16,solid,forget plot]
  table[row sep=crcr]{%
0.01	0.01\\
1.01	0.01\\
2.01	0.01\\
3.01	0.01\\
4.01	0.01\\
5.01	0.01\\
6.01	0.01\\
7.01	0.01\\
8.01	0.01\\
9.01	0.01\\
10.01	0.01\\
11.01	0.01\\
12.01	0.01\\
13.01	0.01\\
14.01	0.01\\
15.01	0.01\\
16.01	0.01\\
17.01	0.01\\
18.01	0.01\\
19.01	0.01\\
20.01	0.01\\
21.01	0.01\\
22.01	0.01\\
23.01	0.01\\
24.01	0.01\\
25.01	0.01\\
26.01	0.01\\
27.01	0.01\\
28.01	0.01\\
29.01	0.01\\
30.01	0.01\\
31.01	0.01\\
32.01	0.01\\
33.01	0.01\\
34.01	0.01\\
35.01	0.01\\
36.01	0.01\\
37.01	0.01\\
38.01	0.01\\
39.01	0.01\\
40.01	0.01\\
41.01	0.01\\
42.01	0.01\\
43.01	0.01\\
44.01	0.01\\
45.01	0.01\\
46.01	0.01\\
47.01	0.01\\
48.01	0.01\\
49.01	0.01\\
50.01	0.01\\
51.01	0.01\\
52.01	0.01\\
53.01	0.01\\
54.01	0.01\\
55.01	0.01\\
56.01	0.01\\
57.01	0.01\\
58.01	0.01\\
59.01	0.01\\
60.01	0.01\\
61.01	0.01\\
62.01	0.01\\
63.01	0.01\\
64.01	0.01\\
65.01	0.01\\
66.01	0.01\\
67.01	0.01\\
68.01	0.01\\
69.01	0.01\\
70.01	0.01\\
71.01	0.01\\
72.01	0.01\\
73.01	0.01\\
74.01	0.01\\
75.01	0.01\\
76.01	0.01\\
77.01	0.01\\
78.01	0.01\\
79.01	0.01\\
80.01	0.01\\
81.01	0.01\\
82.01	0.01\\
83.01	0.01\\
84.01	0.01\\
85.01	0.01\\
86.01	0.01\\
87.01	0.01\\
88.01	0.01\\
89.01	0.01\\
90.01	0.01\\
91.01	0.01\\
92.01	0.01\\
93.01	0.01\\
94.01	0.01\\
95.01	0.01\\
96.01	0.01\\
97.01	0.01\\
98.01	0.01\\
99.01	0.01\\
100.01	0.01\\
101.01	0.01\\
102.01	0.01\\
103.01	0.01\\
104.01	0.01\\
105.01	0.01\\
106.01	0.01\\
107.01	0.01\\
108.01	0.01\\
109.01	0.01\\
110.01	0.01\\
111.01	0.01\\
112.01	0.01\\
113.01	0.01\\
114.01	0.01\\
115.01	0.01\\
116.01	0.01\\
117.01	0.01\\
118.01	0.01\\
119.01	0.01\\
120.01	0.01\\
121.01	0.01\\
122.01	0.01\\
123.01	0.01\\
124.01	0.01\\
125.01	0.01\\
126.01	0.01\\
127.01	0.01\\
128.01	0.01\\
129.01	0.01\\
130.01	0.01\\
131.01	0.01\\
132.01	0.01\\
133.01	0.01\\
134.01	0.01\\
135.01	0.01\\
136.01	0.01\\
137.01	0.01\\
138.01	0.01\\
139.01	0.01\\
140.01	0.01\\
141.01	0.01\\
142.01	0.01\\
143.01	0.01\\
144.01	0.01\\
145.01	0.01\\
146.01	0.01\\
147.01	0.01\\
148.01	0.01\\
149.01	0.01\\
150.01	0.01\\
151.01	0.01\\
152.01	0.01\\
153.01	0.01\\
154.01	0.01\\
155.01	0.01\\
156.01	0.01\\
157.01	0.01\\
158.01	0.01\\
159.01	0.01\\
160.01	0.01\\
161.01	0.01\\
162.01	0.01\\
163.01	0.01\\
164.01	0.01\\
165.01	0.01\\
166.01	0.01\\
167.01	0.01\\
168.01	0.01\\
169.01	0.01\\
170.01	0.01\\
171.01	0.01\\
172.01	0.01\\
173.01	0.01\\
174.01	0.01\\
175.01	0.01\\
176.01	0.01\\
177.01	0.01\\
178.01	0.01\\
179.01	0.01\\
180.01	0.01\\
181.01	0.01\\
182.01	0.01\\
183.01	0.01\\
184.01	0.01\\
185.01	0.01\\
186.01	0.01\\
187.01	0.01\\
188.01	0.01\\
189.01	0.01\\
190.01	0.01\\
191.01	0.01\\
192.01	0.01\\
193.01	0.01\\
194.01	0.01\\
195.01	0.01\\
196.01	0.01\\
197.01	0.01\\
198.01	0.01\\
199.01	0.01\\
200.01	0.01\\
201.01	0.01\\
202.01	0.01\\
203.01	0.01\\
204.01	0.01\\
205.01	0.01\\
206.01	0.01\\
207.01	0.01\\
208.01	0.01\\
209.01	0.01\\
210.01	0.01\\
211.01	0.01\\
212.01	0.01\\
213.01	0.01\\
214.01	0.01\\
215.01	0.01\\
216.01	0.01\\
217.01	0.01\\
218.01	0.01\\
219.01	0.01\\
220.01	0.01\\
221.01	0.01\\
222.01	0.01\\
223.01	0.01\\
224.01	0.01\\
225.01	0.01\\
226.01	0.01\\
227.01	0.01\\
228.01	0.01\\
229.01	0.01\\
230.01	0.01\\
231.01	0.01\\
232.01	0.01\\
233.01	0.01\\
234.01	0.01\\
235.01	0.01\\
236.01	0.01\\
237.01	0.01\\
238.01	0.01\\
239.01	0.01\\
240.01	0.01\\
241.01	0.01\\
242.01	0.01\\
243.01	0.01\\
244.01	0.01\\
245.01	0.01\\
246.01	0.01\\
247.01	0.01\\
248.01	0.01\\
249.01	0.01\\
250.01	0.01\\
251.01	0.01\\
252.01	0.01\\
253.01	0.01\\
254.01	0.01\\
255.01	0.01\\
256.01	0.01\\
257.01	0.01\\
258.01	0.01\\
259.01	0.01\\
260.01	0.01\\
261.01	0.01\\
262.01	0.01\\
263.01	0.01\\
264.01	0.01\\
265.01	0.01\\
266.01	0.01\\
267.01	0.01\\
268.01	0.01\\
269.01	0.01\\
270.01	0.01\\
271.01	0.01\\
272.01	0.01\\
273.01	0.01\\
274.01	0.01\\
275.01	0.01\\
276.01	0.01\\
277.01	0.01\\
278.01	0.01\\
279.01	0.01\\
280.01	0.01\\
281.01	0.01\\
282.01	0.01\\
283.01	0.01\\
284.01	0.01\\
285.01	0.01\\
286.01	0.01\\
287.01	0.01\\
288.01	0.01\\
289.01	0.01\\
290.01	0.01\\
291.01	0.01\\
292.01	0.01\\
293.01	0.01\\
294.01	0.01\\
295.01	0.01\\
296.01	0.01\\
297.01	0.01\\
298.01	0.01\\
299.01	0.01\\
300.01	0.01\\
301.01	0.01\\
302.01	0.01\\
303.01	0.01\\
304.01	0.01\\
305.01	0.01\\
306.01	0.01\\
307.01	0.01\\
308.01	0.01\\
309.01	0.01\\
310.01	0.01\\
311.01	0.01\\
312.01	0.01\\
313.01	0.01\\
314.01	0.01\\
315.01	0.01\\
316.01	0.01\\
317.01	0.01\\
318.01	0.01\\
319.01	0.01\\
320.01	0.01\\
321.01	0.01\\
322.01	0.01\\
323.01	0.01\\
324.01	0.01\\
325.01	0.01\\
326.01	0.01\\
327.01	0.01\\
328.01	0.01\\
329.01	0.01\\
330.01	0.01\\
331.01	0.01\\
332.01	0.01\\
333.01	0.01\\
334.01	0.01\\
335.01	0.01\\
336.01	0.01\\
337.01	0.01\\
338.01	0.01\\
339.01	0.01\\
340.01	0.01\\
341.01	0.01\\
342.01	0.01\\
343.01	0.01\\
344.01	0.01\\
345.01	0.01\\
346.01	0.01\\
347.01	0.01\\
348.01	0.01\\
349.01	0.01\\
350.01	0.01\\
351.01	0.01\\
352.01	0.01\\
353.01	0.01\\
354.01	0.01\\
355.01	0.01\\
356.01	0.01\\
357.01	0.01\\
358.01	0.01\\
359.01	0.01\\
360.01	0.01\\
361.01	0.01\\
362.01	0.01\\
363.01	0.01\\
364.01	0.01\\
365.01	0.01\\
366.01	0.01\\
367.01	0.01\\
368.01	0.01\\
369.01	0.01\\
370.01	0.01\\
371.01	0.01\\
372.01	0.01\\
373.01	0.01\\
374.01	0.01\\
375.01	0.01\\
376.01	0.01\\
377.01	0.01\\
378.01	0.01\\
379.01	0.01\\
380.01	0.01\\
381.01	0.01\\
382.01	0.01\\
383.01	0.01\\
384.01	0.01\\
385.01	0.01\\
386.01	0.01\\
387.01	0.01\\
388.01	0.01\\
389.01	0.01\\
390.01	0.01\\
391.01	0.01\\
392.01	0.01\\
393.01	0.01\\
394.01	0.01\\
395.01	0.01\\
396.01	0.01\\
397.01	0.01\\
398.01	0.01\\
399.01	0.01\\
400.01	0.01\\
401.01	0.01\\
402.01	0.01\\
403.01	0.01\\
404.01	0.01\\
405.01	0.01\\
406.01	0.01\\
407.01	0.01\\
408.01	0.01\\
409.01	0.01\\
410.01	0.01\\
411.01	0.01\\
412.01	0.01\\
413.01	0.01\\
414.01	0.01\\
415.01	0.01\\
416.01	0.01\\
417.01	0.01\\
418.01	0.01\\
419.01	0.01\\
420.01	0.01\\
421.01	0.01\\
422.01	0.01\\
423.01	0.01\\
424.01	0.01\\
425.01	0.01\\
426.01	0.01\\
427.01	0.01\\
428.01	0.01\\
429.01	0.01\\
430.01	0.01\\
431.01	0.01\\
432.01	0.01\\
433.01	0.01\\
434.01	0.01\\
435.01	0.01\\
436.01	0.01\\
437.01	0.01\\
438.01	0.01\\
439.01	0.01\\
440.01	0.01\\
441.01	0.01\\
442.01	0.01\\
443.01	0.01\\
444.01	0.01\\
445.01	0.01\\
446.01	0.01\\
447.01	0.01\\
448.01	0.01\\
449.01	0.01\\
450.01	0.01\\
451.01	0.01\\
452.01	0.01\\
453.01	0.01\\
454.01	0.01\\
455.01	0.01\\
456.01	0.01\\
457.01	0.01\\
458.01	0.01\\
459.01	0.01\\
460.01	0.01\\
461.01	0.01\\
462.01	0.01\\
463.01	0.01\\
464.01	0.01\\
465.01	0.01\\
466.01	0.01\\
467.01	0.01\\
468.01	0.01\\
469.01	0.01\\
470.01	0.01\\
471.01	0.01\\
472.01	0.01\\
473.01	0.01\\
474.01	0.01\\
475.01	0.01\\
476.01	0.01\\
477.01	0.01\\
478.01	0.01\\
479.01	0.01\\
480.01	0.01\\
481.01	0.01\\
482.01	0.01\\
483.01	0.01\\
484.01	0.01\\
485.01	0.01\\
486.01	0.01\\
487.01	0.01\\
488.01	0.01\\
489.01	0.01\\
490.01	0.01\\
491.01	0.01\\
492.01	0.01\\
493.01	0.01\\
494.01	0.01\\
495.01	0.01\\
496.01	0.01\\
497.01	0.01\\
498.01	0.01\\
499.01	0.01\\
500.01	0.01\\
501.01	0.01\\
502.01	0.01\\
503.01	0.01\\
504.01	0.01\\
505.01	0.01\\
506.01	0.01\\
507.01	0.01\\
508.01	0.01\\
509.01	0.01\\
510.01	0.01\\
511.01	0.01\\
512.01	0.01\\
513.01	0.01\\
514.01	0.01\\
515.01	0.01\\
516.01	0.01\\
517.01	0.01\\
518.01	0.01\\
519.01	0.01\\
520.01	0.01\\
521.01	0.01\\
522.01	0.01\\
523.01	0.01\\
524.01	0.01\\
525.01	0.01\\
526.01	0.01\\
527.01	0.01\\
528.01	0.01\\
529.01	0.01\\
530.01	0.01\\
531.01	0.01\\
532.01	0.01\\
533.01	0.01\\
534.01	0.01\\
535.01	0.01\\
536.01	0.01\\
537.01	0.01\\
538.01	0.01\\
539.01	0.01\\
540.01	0.01\\
541.01	0.01\\
542.01	0.01\\
543.01	0.01\\
544.01	0.01\\
545.01	0.01\\
546.01	0.01\\
547.01	0.01\\
548.01	0.01\\
549.01	0.01\\
550.01	0.00995817214987554\\
551.01	0.00976920720650903\\
552.01	0.00957068161494685\\
553.01	0.00936157648640411\\
554.01	0.00914070877388854\\
555.01	0.00890669871536501\\
556.01	0.00865793000953965\\
557.01	0.00839250036389308\\
558.01	0.00810815851763091\\
559.01	0.00780222450397486\\
560.01	0.00747301733874219\\
561.01	0.00712763658868991\\
562.01	0.00676663990071911\\
563.01	0.00638887747094233\\
564.01	0.00599300805243266\\
565.01	0.00557739361006792\\
566.01	0.00513978932111589\\
567.01	0.0046775026391738\\
568.01	0.00446437522935085\\
569.01	0.00425959342731657\\
570.01	0.00405456668876184\\
571.01	0.00385186976766412\\
572.01	0.00365507220738019\\
573.01	0.00346541136082133\\
574.01	0.00327522104056992\\
575.01	0.00308519632578407\\
576.01	0.00289654874182385\\
577.01	0.00271074026193477\\
578.01	0.00252951755813305\\
579.01	0.00235494432006656\\
580.01	0.00218942634918826\\
581.01	0.00203571962313973\\
582.01	0.00189569368076647\\
583.01	0.00176221935349623\\
584.01	0.00163117395006303\\
585.01	0.00150194649748184\\
586.01	0.00137451254708096\\
587.01	0.00124919829444105\\
588.01	0.001126282592439\\
589.01	0.00100599016945428\\
590.01	0.000888534916031865\\
591.01	0.00077433810196983\\
592.01	0.000663891166351075\\
593.01	0.000557617301030397\\
594.01	0.000455831854103557\\
595.01	0.000358696458436331\\
596.01	0.000266167733632852\\
597.01	0.000177943768206052\\
598.01	9.42984683149437e-05\\
599.01	2.97433029915143e-05\\
599.02	2.92255074340608e-05\\
599.03	2.87109129547599e-05\\
599.04	2.81995475120318e-05\\
599.05	2.76914393515584e-05\\
599.06	2.71866170091662e-05\\
599.07	2.66851093137094e-05\\
599.08	2.61869453900155e-05\\
599.09	2.56921546618414e-05\\
599.1	2.52007668548517e-05\\
599.11	2.47128119996477e-05\\
599.12	2.42283204348152e-05\\
599.13	2.37473228099933e-05\\
599.14	2.32698500889934e-05\\
599.15	2.2795933552925e-05\\
599.16	2.23256048033898e-05\\
599.17	2.18588957656557e-05\\
599.18	2.13958386919115e-05\\
599.19	2.09364661645194e-05\\
599.2	2.04808110993264e-05\\
599.21	2.00289067489832e-05\\
599.22	1.95807867063164e-05\\
599.23	1.91364849077228e-05\\
599.24	1.86960356366034e-05\\
599.25	1.82594735268233e-05\\
599.26	1.78268349628249e-05\\
599.27	1.73981584822864e-05\\
599.28	1.69734830200441e-05\\
599.29	1.65528479120682e-05\\
599.3	1.61362928994564e-05\\
599.31	1.57238581324892e-05\\
599.32	1.53155841747174e-05\\
599.33	1.49115120070802e-05\\
599.34	1.45116830320754e-05\\
599.35	1.41161390779781e-05\\
599.36	1.37249224030859e-05\\
599.37	1.33380757000104e-05\\
599.38	1.29556421000227e-05\\
599.39	1.2577665177432e-05\\
599.4	1.22041889540053e-05\\
599.41	1.18352579034451e-05\\
599.42	1.14709169558962e-05\\
599.43	1.11112115025128e-05\\
599.44	1.07561874000597e-05\\
599.45	1.04058909755692e-05\\
599.46	1.00603690309786e-05\\
599.47	9.7196688453368e-06\\
599.48	9.38383817956912e-06\\
599.49	9.05292528129154e-06\\
599.5	8.72697888967475e-06\\
599.51	8.40604824036036e-06\\
599.52	8.09018307042046e-06\\
599.53	7.77943362337274e-06\\
599.54	7.47385065425279e-06\\
599.55	7.17348543472461e-06\\
599.56	6.87838975826048e-06\\
599.57	6.58861594535556e-06\\
599.58	6.30421684882053e-06\\
599.59	6.02524585910542e-06\\
599.6	5.75175690969466e-06\\
599.61	5.4838044825558e-06\\
599.62	5.22144361363858e-06\\
599.63	4.96472989843823e-06\\
599.64	4.71371949762288e-06\\
599.65	4.46846914270785e-06\\
599.66	4.22903614179931e-06\\
599.67	3.9954783853987e-06\\
599.68	3.76785435226429e-06\\
599.69	3.54622311534575e-06\\
599.7	3.33064434777409e-06\\
599.71	3.12117832892804e-06\\
599.72	2.91788595054021e-06\\
599.73	2.72082872291787e-06\\
599.74	2.53006878117752e-06\\
599.75	2.34566889159918e-06\\
599.76	2.16769245801711e-06\\
599.77	1.99620352829381e-06\\
599.78	1.83126680087728e-06\\
599.79	1.67294763142416e-06\\
599.8	1.52131203949059e-06\\
599.81	1.37642671532712e-06\\
599.82	1.23835902672391e-06\\
599.83	1.10717702595832e-06\\
599.84	9.82949456815319e-07\\
599.85	8.65745761698122e-07\\
599.86	7.55636088823827e-07\\
599.87	6.52691299497105e-07\\
599.88	5.56982975498388e-07\\
599.89	4.6858342654145e-07\\
599.9	3.87565697841999e-07\\
599.91	3.14003577771282e-07\\
599.92	2.47971605622441e-07\\
599.93	1.89545079472275e-07\\
599.94	1.38800064147099e-07\\
599.95	9.58133993030769e-08\\
599.96	6.06627076158578e-08\\
599.97	3.34264030846937e-08\\
599.98	1.4183699454523e-08\\
599.99	3.01461875948372e-09\\
600	0\\
};
\addplot [color=mycolor17,solid,forget plot]
  table[row sep=crcr]{%
0.01	0.01\\
1.01	0.01\\
2.01	0.01\\
3.01	0.01\\
4.01	0.01\\
5.01	0.01\\
6.01	0.01\\
7.01	0.01\\
8.01	0.01\\
9.01	0.01\\
10.01	0.01\\
11.01	0.01\\
12.01	0.01\\
13.01	0.01\\
14.01	0.01\\
15.01	0.01\\
16.01	0.01\\
17.01	0.01\\
18.01	0.01\\
19.01	0.01\\
20.01	0.01\\
21.01	0.01\\
22.01	0.01\\
23.01	0.01\\
24.01	0.01\\
25.01	0.01\\
26.01	0.01\\
27.01	0.01\\
28.01	0.01\\
29.01	0.01\\
30.01	0.01\\
31.01	0.01\\
32.01	0.01\\
33.01	0.01\\
34.01	0.01\\
35.01	0.01\\
36.01	0.01\\
37.01	0.01\\
38.01	0.01\\
39.01	0.01\\
40.01	0.01\\
41.01	0.01\\
42.01	0.01\\
43.01	0.01\\
44.01	0.01\\
45.01	0.01\\
46.01	0.01\\
47.01	0.01\\
48.01	0.01\\
49.01	0.01\\
50.01	0.01\\
51.01	0.01\\
52.01	0.01\\
53.01	0.01\\
54.01	0.01\\
55.01	0.01\\
56.01	0.01\\
57.01	0.01\\
58.01	0.01\\
59.01	0.01\\
60.01	0.01\\
61.01	0.01\\
62.01	0.01\\
63.01	0.01\\
64.01	0.01\\
65.01	0.01\\
66.01	0.01\\
67.01	0.01\\
68.01	0.01\\
69.01	0.01\\
70.01	0.01\\
71.01	0.01\\
72.01	0.01\\
73.01	0.01\\
74.01	0.01\\
75.01	0.01\\
76.01	0.01\\
77.01	0.01\\
78.01	0.01\\
79.01	0.01\\
80.01	0.01\\
81.01	0.01\\
82.01	0.01\\
83.01	0.01\\
84.01	0.01\\
85.01	0.01\\
86.01	0.01\\
87.01	0.01\\
88.01	0.01\\
89.01	0.01\\
90.01	0.01\\
91.01	0.01\\
92.01	0.01\\
93.01	0.01\\
94.01	0.01\\
95.01	0.01\\
96.01	0.01\\
97.01	0.01\\
98.01	0.01\\
99.01	0.01\\
100.01	0.01\\
101.01	0.01\\
102.01	0.01\\
103.01	0.01\\
104.01	0.01\\
105.01	0.01\\
106.01	0.01\\
107.01	0.01\\
108.01	0.01\\
109.01	0.01\\
110.01	0.01\\
111.01	0.01\\
112.01	0.01\\
113.01	0.01\\
114.01	0.01\\
115.01	0.01\\
116.01	0.01\\
117.01	0.01\\
118.01	0.01\\
119.01	0.01\\
120.01	0.01\\
121.01	0.01\\
122.01	0.01\\
123.01	0.01\\
124.01	0.01\\
125.01	0.01\\
126.01	0.01\\
127.01	0.01\\
128.01	0.01\\
129.01	0.01\\
130.01	0.01\\
131.01	0.01\\
132.01	0.01\\
133.01	0.01\\
134.01	0.01\\
135.01	0.01\\
136.01	0.01\\
137.01	0.01\\
138.01	0.01\\
139.01	0.01\\
140.01	0.01\\
141.01	0.01\\
142.01	0.01\\
143.01	0.01\\
144.01	0.01\\
145.01	0.01\\
146.01	0.01\\
147.01	0.01\\
148.01	0.01\\
149.01	0.01\\
150.01	0.01\\
151.01	0.01\\
152.01	0.01\\
153.01	0.01\\
154.01	0.01\\
155.01	0.01\\
156.01	0.01\\
157.01	0.01\\
158.01	0.01\\
159.01	0.01\\
160.01	0.01\\
161.01	0.01\\
162.01	0.01\\
163.01	0.01\\
164.01	0.01\\
165.01	0.01\\
166.01	0.01\\
167.01	0.01\\
168.01	0.01\\
169.01	0.01\\
170.01	0.01\\
171.01	0.01\\
172.01	0.01\\
173.01	0.01\\
174.01	0.01\\
175.01	0.01\\
176.01	0.01\\
177.01	0.01\\
178.01	0.01\\
179.01	0.01\\
180.01	0.01\\
181.01	0.01\\
182.01	0.01\\
183.01	0.01\\
184.01	0.01\\
185.01	0.01\\
186.01	0.01\\
187.01	0.01\\
188.01	0.01\\
189.01	0.01\\
190.01	0.01\\
191.01	0.01\\
192.01	0.01\\
193.01	0.01\\
194.01	0.01\\
195.01	0.01\\
196.01	0.01\\
197.01	0.01\\
198.01	0.01\\
199.01	0.01\\
200.01	0.01\\
201.01	0.01\\
202.01	0.01\\
203.01	0.01\\
204.01	0.01\\
205.01	0.01\\
206.01	0.01\\
207.01	0.01\\
208.01	0.01\\
209.01	0.01\\
210.01	0.01\\
211.01	0.01\\
212.01	0.01\\
213.01	0.01\\
214.01	0.01\\
215.01	0.01\\
216.01	0.01\\
217.01	0.01\\
218.01	0.01\\
219.01	0.01\\
220.01	0.01\\
221.01	0.01\\
222.01	0.01\\
223.01	0.01\\
224.01	0.01\\
225.01	0.01\\
226.01	0.01\\
227.01	0.01\\
228.01	0.01\\
229.01	0.01\\
230.01	0.01\\
231.01	0.01\\
232.01	0.01\\
233.01	0.01\\
234.01	0.01\\
235.01	0.01\\
236.01	0.01\\
237.01	0.01\\
238.01	0.01\\
239.01	0.01\\
240.01	0.01\\
241.01	0.01\\
242.01	0.01\\
243.01	0.01\\
244.01	0.01\\
245.01	0.01\\
246.01	0.01\\
247.01	0.01\\
248.01	0.01\\
249.01	0.01\\
250.01	0.01\\
251.01	0.01\\
252.01	0.01\\
253.01	0.01\\
254.01	0.01\\
255.01	0.01\\
256.01	0.01\\
257.01	0.01\\
258.01	0.01\\
259.01	0.01\\
260.01	0.01\\
261.01	0.01\\
262.01	0.01\\
263.01	0.01\\
264.01	0.01\\
265.01	0.01\\
266.01	0.01\\
267.01	0.01\\
268.01	0.01\\
269.01	0.01\\
270.01	0.01\\
271.01	0.01\\
272.01	0.01\\
273.01	0.01\\
274.01	0.01\\
275.01	0.01\\
276.01	0.01\\
277.01	0.01\\
278.01	0.01\\
279.01	0.01\\
280.01	0.01\\
281.01	0.01\\
282.01	0.01\\
283.01	0.01\\
284.01	0.01\\
285.01	0.01\\
286.01	0.01\\
287.01	0.01\\
288.01	0.01\\
289.01	0.01\\
290.01	0.01\\
291.01	0.01\\
292.01	0.01\\
293.01	0.01\\
294.01	0.01\\
295.01	0.01\\
296.01	0.01\\
297.01	0.01\\
298.01	0.01\\
299.01	0.01\\
300.01	0.01\\
301.01	0.01\\
302.01	0.01\\
303.01	0.01\\
304.01	0.01\\
305.01	0.01\\
306.01	0.01\\
307.01	0.01\\
308.01	0.01\\
309.01	0.01\\
310.01	0.01\\
311.01	0.01\\
312.01	0.01\\
313.01	0.01\\
314.01	0.01\\
315.01	0.01\\
316.01	0.01\\
317.01	0.01\\
318.01	0.01\\
319.01	0.01\\
320.01	0.01\\
321.01	0.01\\
322.01	0.01\\
323.01	0.01\\
324.01	0.01\\
325.01	0.01\\
326.01	0.01\\
327.01	0.01\\
328.01	0.01\\
329.01	0.01\\
330.01	0.01\\
331.01	0.01\\
332.01	0.01\\
333.01	0.01\\
334.01	0.01\\
335.01	0.01\\
336.01	0.01\\
337.01	0.01\\
338.01	0.01\\
339.01	0.01\\
340.01	0.01\\
341.01	0.01\\
342.01	0.01\\
343.01	0.01\\
344.01	0.01\\
345.01	0.01\\
346.01	0.01\\
347.01	0.01\\
348.01	0.01\\
349.01	0.01\\
350.01	0.01\\
351.01	0.01\\
352.01	0.01\\
353.01	0.01\\
354.01	0.01\\
355.01	0.01\\
356.01	0.01\\
357.01	0.01\\
358.01	0.01\\
359.01	0.01\\
360.01	0.01\\
361.01	0.01\\
362.01	0.01\\
363.01	0.01\\
364.01	0.01\\
365.01	0.01\\
366.01	0.01\\
367.01	0.01\\
368.01	0.01\\
369.01	0.01\\
370.01	0.01\\
371.01	0.01\\
372.01	0.01\\
373.01	0.01\\
374.01	0.01\\
375.01	0.01\\
376.01	0.01\\
377.01	0.01\\
378.01	0.01\\
379.01	0.01\\
380.01	0.01\\
381.01	0.01\\
382.01	0.01\\
383.01	0.01\\
384.01	0.01\\
385.01	0.01\\
386.01	0.01\\
387.01	0.01\\
388.01	0.01\\
389.01	0.01\\
390.01	0.01\\
391.01	0.01\\
392.01	0.01\\
393.01	0.01\\
394.01	0.01\\
395.01	0.01\\
396.01	0.01\\
397.01	0.01\\
398.01	0.01\\
399.01	0.01\\
400.01	0.01\\
401.01	0.01\\
402.01	0.01\\
403.01	0.01\\
404.01	0.01\\
405.01	0.01\\
406.01	0.01\\
407.01	0.01\\
408.01	0.01\\
409.01	0.01\\
410.01	0.01\\
411.01	0.01\\
412.01	0.01\\
413.01	0.01\\
414.01	0.01\\
415.01	0.01\\
416.01	0.01\\
417.01	0.01\\
418.01	0.01\\
419.01	0.01\\
420.01	0.01\\
421.01	0.01\\
422.01	0.01\\
423.01	0.01\\
424.01	0.01\\
425.01	0.01\\
426.01	0.01\\
427.01	0.01\\
428.01	0.01\\
429.01	0.01\\
430.01	0.01\\
431.01	0.01\\
432.01	0.01\\
433.01	0.01\\
434.01	0.01\\
435.01	0.01\\
436.01	0.01\\
437.01	0.01\\
438.01	0.01\\
439.01	0.01\\
440.01	0.01\\
441.01	0.01\\
442.01	0.01\\
443.01	0.01\\
444.01	0.01\\
445.01	0.01\\
446.01	0.01\\
447.01	0.01\\
448.01	0.01\\
449.01	0.01\\
450.01	0.01\\
451.01	0.01\\
452.01	0.01\\
453.01	0.01\\
454.01	0.01\\
455.01	0.01\\
456.01	0.01\\
457.01	0.01\\
458.01	0.01\\
459.01	0.01\\
460.01	0.01\\
461.01	0.01\\
462.01	0.01\\
463.01	0.01\\
464.01	0.01\\
465.01	0.01\\
466.01	0.01\\
467.01	0.01\\
468.01	0.01\\
469.01	0.01\\
470.01	0.01\\
471.01	0.01\\
472.01	0.01\\
473.01	0.01\\
474.01	0.01\\
475.01	0.01\\
476.01	0.01\\
477.01	0.01\\
478.01	0.01\\
479.01	0.01\\
480.01	0.01\\
481.01	0.01\\
482.01	0.01\\
483.01	0.01\\
484.01	0.01\\
485.01	0.01\\
486.01	0.01\\
487.01	0.01\\
488.01	0.01\\
489.01	0.01\\
490.01	0.01\\
491.01	0.01\\
492.01	0.01\\
493.01	0.01\\
494.01	0.01\\
495.01	0.01\\
496.01	0.01\\
497.01	0.01\\
498.01	0.01\\
499.01	0.01\\
500.01	0.01\\
501.01	0.01\\
502.01	0.01\\
503.01	0.01\\
504.01	0.01\\
505.01	0.01\\
506.01	0.01\\
507.01	0.01\\
508.01	0.01\\
509.01	0.01\\
510.01	0.01\\
511.01	0.01\\
512.01	0.01\\
513.01	0.01\\
514.01	0.01\\
515.01	0.01\\
516.01	0.01\\
517.01	0.01\\
518.01	0.01\\
519.01	0.01\\
520.01	0.01\\
521.01	0.01\\
522.01	0.01\\
523.01	0.01\\
524.01	0.01\\
525.01	0.01\\
526.01	0.01\\
527.01	0.0099241290970491\\
528.01	0.0098175255165026\\
529.01	0.0097068930893769\\
530.01	0.00959196270396107\\
531.01	0.00947243515181103\\
532.01	0.00934797596358879\\
533.01	0.00921820807741072\\
534.01	0.00908269772191878\\
535.01	0.00894095105251903\\
536.01	0.00879240950820449\\
537.01	0.00863643967948476\\
538.01	0.00847232140769048\\
539.01	0.00829923383611116\\
540.01	0.00811623904417037\\
541.01	0.00792226282616251\\
542.01	0.00771606180810179\\
543.01	0.00749606450831156\\
544.01	0.00726030556795162\\
545.01	0.0070079519472063\\
546.01	0.00674388486960165\\
547.01	0.00646787322976985\\
548.01	0.00617876222036522\\
549.01	0.00587518142076962\\
550.01	0.00559746032953064\\
551.01	0.00545298715323179\\
552.01	0.00530500792545318\\
553.01	0.00515390011393403\\
554.01	0.00500019249536979\\
555.01	0.0048446042748905\\
556.01	0.00468808397530697\\
557.01	0.00453190880521706\\
558.01	0.00437781468838097\\
559.01	0.00422814498614577\\
560.01	0.00408450061202286\\
561.01	0.00393959397258884\\
562.01	0.0037924112109357\\
563.01	0.00364377063441716\\
564.01	0.00349484266244423\\
565.01	0.00334828400578513\\
566.01	0.0032068365243211\\
567.01	0.00307305339271171\\
568.01	0.00294563612557173\\
569.01	0.00282111130465503\\
570.01	0.00270007470687016\\
571.01	0.00258295637848159\\
572.01	0.00246985659109791\\
573.01	0.00236031264984525\\
574.01	0.00225264487338503\\
575.01	0.00214649048949318\\
576.01	0.00204212467871798\\
577.01	0.00193976321903908\\
578.01	0.00183953410827614\\
579.01	0.00174144548087779\\
580.01	0.00164535244959273\\
581.01	0.00155092827496037\\
582.01	0.00145768949973365\\
583.01	0.0013652994855999\\
584.01	0.00127372346543224\\
585.01	0.0011829917503887\\
586.01	0.00109316523826044\\
587.01	0.00100429568118757\\
588.01	0.00091642184966745\\
589.01	0.000829576198374293\\
590.01	0.000743791048770606\\
591.01	0.000659086383332572\\
592.01	0.000575443770475836\\
593.01	0.000492800020771381\\
594.01	0.000411047558901598\\
595.01	0.000330038786353392\\
596.01	0.000249595277727267\\
597.01	0.000169522162970183\\
598.01	9.18966740472912e-05\\
599.01	2.94706190942413e-05\\
599.02	2.89609613509639e-05\\
599.03	2.84543437073619e-05\\
599.04	2.79507956110585e-05\\
599.05	2.74503468013707e-05\\
599.06	2.69530273121939e-05\\
599.07	2.64588674749062e-05\\
599.08	2.59678979212998e-05\\
599.09	2.54801495865476e-05\\
599.1	2.49956537121988e-05\\
599.11	2.45144418491942e-05\\
599.12	2.40365458609258e-05\\
599.13	2.35619979263146e-05\\
599.14	2.30908305429295e-05\\
599.15	2.26230765301254e-05\\
599.16	2.21587690322159e-05\\
599.17	2.1697941521695e-05\\
599.18	2.12406278024529e-05\\
599.19	2.07868620130686e-05\\
599.2	2.03366786300958e-05\\
599.21	1.98901124714036e-05\\
599.22	1.9447198699547e-05\\
599.23	1.90079728251708e-05\\
599.24	1.85724707104321e-05\\
599.25	1.81407285724869e-05\\
599.26	1.77127848684563e-05\\
599.27	1.72886796869724e-05\\
599.28	1.68684535171004e-05\\
599.29	1.64521472522886e-05\\
599.3	1.60398021943637e-05\\
599.31	1.56314600575445e-05\\
599.32	1.52271629725205e-05\\
599.33	1.48269534905581e-05\\
599.34	1.44308745876499e-05\\
599.35	1.40389696686987e-05\\
599.36	1.36512825717559e-05\\
599.37	1.32678575722866e-05\\
599.38	1.28887393874896e-05\\
599.39	1.25139731806528e-05\\
599.4	1.21436045655531e-05\\
599.41	1.17776796109033e-05\\
599.42	1.14162448448338e-05\\
599.43	1.10593472594303e-05\\
599.44	1.07070343153062e-05\\
599.45	1.03593539462262e-05\\
599.46	1.00163545637792e-05\\
599.47	9.67808506207625e-06\\
599.48	9.34459482253812e-06\\
599.49	9.01593371867813e-06\\
599.5	8.69215212097318e-06\\
599.51	8.37330090176433e-06\\
599.52	8.05943144020597e-06\\
599.53	7.75059562726881e-06\\
599.54	7.44684587079478e-06\\
599.55	7.14823510058853e-06\\
599.56	6.85481677357644e-06\\
599.57	6.56664487900041e-06\\
599.58	6.28377394367753e-06\\
599.59	6.00625903730313e-06\\
599.6	5.73415577780415e-06\\
599.61	5.46752033675491e-06\\
599.62	5.20640944483286e-06\\
599.63	4.9508803973454e-06\\
599.64	4.70099105979137e-06\\
599.65	4.45679987349373e-06\\
599.66	4.21836586128252e-06\\
599.67	3.98574863322981e-06\\
599.68	3.75900839244374e-06\\
599.69	3.53820594092488e-06\\
599.7	3.32340268546956e-06\\
599.71	3.11466064364246e-06\\
599.72	2.9120424498031e-06\\
599.73	2.71561136118426e-06\\
599.74	2.52543126405373e-06\\
599.75	2.34156667990038e-06\\
599.76	2.16408277171551e-06\\
599.77	1.99304535031947e-06\\
599.78	1.82852088075058e-06\\
599.79	1.67057648871489e-06\\
599.8	1.51927996711911e-06\\
599.81	1.37469978262958e-06\\
599.82	1.23690508234062e-06\\
599.83	1.10596570046875e-06\\
599.84	9.81952165138994e-07\\
599.85	8.64935705228304e-07\\
599.86	7.54988257262862e-07\\
599.87	6.5218247242288e-07\\
599.88	5.56591723556085e-07\\
599.89	4.6829011231958e-07\\
599.9	3.87352476345984e-07\\
599.91	3.13854396508453e-07\\
599.92	2.47872204241217e-07\\
599.93	1.89482988934703e-07\\
599.94	1.38764605403519e-07\\
599.95	9.57956814255645e-08\\
599.96	6.06556253574669e-08\\
599.97	3.34246338211386e-08\\
599.98	1.4183699454523e-08\\
599.99	3.01461875948372e-09\\
600	0\\
};
\addplot [color=mycolor18,solid,forget plot]
  table[row sep=crcr]{%
0.01	0.01\\
1.01	0.01\\
2.01	0.01\\
3.01	0.01\\
4.01	0.01\\
5.01	0.01\\
6.01	0.01\\
7.01	0.01\\
8.01	0.01\\
9.01	0.01\\
10.01	0.01\\
11.01	0.01\\
12.01	0.01\\
13.01	0.01\\
14.01	0.01\\
15.01	0.01\\
16.01	0.01\\
17.01	0.01\\
18.01	0.01\\
19.01	0.01\\
20.01	0.01\\
21.01	0.01\\
22.01	0.01\\
23.01	0.01\\
24.01	0.01\\
25.01	0.01\\
26.01	0.01\\
27.01	0.01\\
28.01	0.01\\
29.01	0.01\\
30.01	0.01\\
31.01	0.01\\
32.01	0.01\\
33.01	0.01\\
34.01	0.01\\
35.01	0.01\\
36.01	0.01\\
37.01	0.01\\
38.01	0.01\\
39.01	0.01\\
40.01	0.01\\
41.01	0.01\\
42.01	0.01\\
43.01	0.01\\
44.01	0.01\\
45.01	0.01\\
46.01	0.01\\
47.01	0.01\\
48.01	0.01\\
49.01	0.01\\
50.01	0.01\\
51.01	0.01\\
52.01	0.01\\
53.01	0.01\\
54.01	0.01\\
55.01	0.01\\
56.01	0.01\\
57.01	0.01\\
58.01	0.01\\
59.01	0.01\\
60.01	0.01\\
61.01	0.01\\
62.01	0.01\\
63.01	0.01\\
64.01	0.01\\
65.01	0.01\\
66.01	0.01\\
67.01	0.01\\
68.01	0.01\\
69.01	0.01\\
70.01	0.01\\
71.01	0.01\\
72.01	0.01\\
73.01	0.01\\
74.01	0.01\\
75.01	0.01\\
76.01	0.01\\
77.01	0.01\\
78.01	0.01\\
79.01	0.01\\
80.01	0.01\\
81.01	0.01\\
82.01	0.01\\
83.01	0.01\\
84.01	0.01\\
85.01	0.01\\
86.01	0.01\\
87.01	0.01\\
88.01	0.01\\
89.01	0.01\\
90.01	0.01\\
91.01	0.01\\
92.01	0.01\\
93.01	0.01\\
94.01	0.01\\
95.01	0.01\\
96.01	0.01\\
97.01	0.01\\
98.01	0.01\\
99.01	0.01\\
100.01	0.01\\
101.01	0.01\\
102.01	0.01\\
103.01	0.01\\
104.01	0.01\\
105.01	0.01\\
106.01	0.01\\
107.01	0.01\\
108.01	0.01\\
109.01	0.01\\
110.01	0.01\\
111.01	0.01\\
112.01	0.01\\
113.01	0.01\\
114.01	0.01\\
115.01	0.01\\
116.01	0.01\\
117.01	0.01\\
118.01	0.01\\
119.01	0.01\\
120.01	0.01\\
121.01	0.01\\
122.01	0.01\\
123.01	0.01\\
124.01	0.01\\
125.01	0.01\\
126.01	0.01\\
127.01	0.01\\
128.01	0.01\\
129.01	0.01\\
130.01	0.01\\
131.01	0.01\\
132.01	0.01\\
133.01	0.01\\
134.01	0.01\\
135.01	0.01\\
136.01	0.01\\
137.01	0.01\\
138.01	0.01\\
139.01	0.01\\
140.01	0.01\\
141.01	0.01\\
142.01	0.01\\
143.01	0.01\\
144.01	0.01\\
145.01	0.01\\
146.01	0.01\\
147.01	0.01\\
148.01	0.01\\
149.01	0.01\\
150.01	0.01\\
151.01	0.01\\
152.01	0.01\\
153.01	0.01\\
154.01	0.01\\
155.01	0.01\\
156.01	0.01\\
157.01	0.01\\
158.01	0.01\\
159.01	0.01\\
160.01	0.01\\
161.01	0.01\\
162.01	0.01\\
163.01	0.01\\
164.01	0.01\\
165.01	0.01\\
166.01	0.01\\
167.01	0.01\\
168.01	0.01\\
169.01	0.01\\
170.01	0.01\\
171.01	0.01\\
172.01	0.01\\
173.01	0.01\\
174.01	0.01\\
175.01	0.01\\
176.01	0.01\\
177.01	0.01\\
178.01	0.01\\
179.01	0.01\\
180.01	0.01\\
181.01	0.01\\
182.01	0.01\\
183.01	0.01\\
184.01	0.01\\
185.01	0.01\\
186.01	0.01\\
187.01	0.01\\
188.01	0.01\\
189.01	0.01\\
190.01	0.01\\
191.01	0.01\\
192.01	0.01\\
193.01	0.01\\
194.01	0.01\\
195.01	0.01\\
196.01	0.01\\
197.01	0.01\\
198.01	0.01\\
199.01	0.01\\
200.01	0.01\\
201.01	0.01\\
202.01	0.01\\
203.01	0.01\\
204.01	0.01\\
205.01	0.01\\
206.01	0.01\\
207.01	0.01\\
208.01	0.01\\
209.01	0.01\\
210.01	0.01\\
211.01	0.01\\
212.01	0.01\\
213.01	0.01\\
214.01	0.01\\
215.01	0.01\\
216.01	0.01\\
217.01	0.01\\
218.01	0.01\\
219.01	0.01\\
220.01	0.01\\
221.01	0.01\\
222.01	0.01\\
223.01	0.01\\
224.01	0.01\\
225.01	0.01\\
226.01	0.01\\
227.01	0.01\\
228.01	0.01\\
229.01	0.01\\
230.01	0.01\\
231.01	0.01\\
232.01	0.01\\
233.01	0.01\\
234.01	0.01\\
235.01	0.01\\
236.01	0.01\\
237.01	0.01\\
238.01	0.01\\
239.01	0.01\\
240.01	0.01\\
241.01	0.01\\
242.01	0.01\\
243.01	0.01\\
244.01	0.01\\
245.01	0.01\\
246.01	0.01\\
247.01	0.01\\
248.01	0.01\\
249.01	0.01\\
250.01	0.01\\
251.01	0.01\\
252.01	0.01\\
253.01	0.01\\
254.01	0.01\\
255.01	0.01\\
256.01	0.01\\
257.01	0.01\\
258.01	0.01\\
259.01	0.01\\
260.01	0.01\\
261.01	0.01\\
262.01	0.01\\
263.01	0.01\\
264.01	0.01\\
265.01	0.01\\
266.01	0.01\\
267.01	0.01\\
268.01	0.01\\
269.01	0.01\\
270.01	0.01\\
271.01	0.01\\
272.01	0.01\\
273.01	0.01\\
274.01	0.01\\
275.01	0.01\\
276.01	0.01\\
277.01	0.01\\
278.01	0.01\\
279.01	0.01\\
280.01	0.01\\
281.01	0.01\\
282.01	0.01\\
283.01	0.01\\
284.01	0.01\\
285.01	0.01\\
286.01	0.01\\
287.01	0.01\\
288.01	0.01\\
289.01	0.01\\
290.01	0.01\\
291.01	0.01\\
292.01	0.01\\
293.01	0.01\\
294.01	0.01\\
295.01	0.01\\
296.01	0.01\\
297.01	0.01\\
298.01	0.01\\
299.01	0.01\\
300.01	0.01\\
301.01	0.01\\
302.01	0.01\\
303.01	0.01\\
304.01	0.01\\
305.01	0.01\\
306.01	0.01\\
307.01	0.01\\
308.01	0.01\\
309.01	0.01\\
310.01	0.01\\
311.01	0.01\\
312.01	0.01\\
313.01	0.01\\
314.01	0.01\\
315.01	0.01\\
316.01	0.01\\
317.01	0.01\\
318.01	0.01\\
319.01	0.01\\
320.01	0.01\\
321.01	0.01\\
322.01	0.01\\
323.01	0.01\\
324.01	0.01\\
325.01	0.01\\
326.01	0.01\\
327.01	0.01\\
328.01	0.01\\
329.01	0.01\\
330.01	0.01\\
331.01	0.01\\
332.01	0.01\\
333.01	0.01\\
334.01	0.01\\
335.01	0.01\\
336.01	0.01\\
337.01	0.01\\
338.01	0.01\\
339.01	0.01\\
340.01	0.01\\
341.01	0.01\\
342.01	0.01\\
343.01	0.01\\
344.01	0.01\\
345.01	0.01\\
346.01	0.01\\
347.01	0.01\\
348.01	0.01\\
349.01	0.01\\
350.01	0.01\\
351.01	0.01\\
352.01	0.01\\
353.01	0.01\\
354.01	0.01\\
355.01	0.01\\
356.01	0.01\\
357.01	0.01\\
358.01	0.01\\
359.01	0.01\\
360.01	0.01\\
361.01	0.01\\
362.01	0.01\\
363.01	0.01\\
364.01	0.01\\
365.01	0.01\\
366.01	0.01\\
367.01	0.01\\
368.01	0.01\\
369.01	0.01\\
370.01	0.01\\
371.01	0.01\\
372.01	0.01\\
373.01	0.01\\
374.01	0.01\\
375.01	0.01\\
376.01	0.01\\
377.01	0.01\\
378.01	0.01\\
379.01	0.01\\
380.01	0.01\\
381.01	0.01\\
382.01	0.01\\
383.01	0.01\\
384.01	0.01\\
385.01	0.01\\
386.01	0.01\\
387.01	0.01\\
388.01	0.01\\
389.01	0.01\\
390.01	0.01\\
391.01	0.01\\
392.01	0.01\\
393.01	0.01\\
394.01	0.01\\
395.01	0.01\\
396.01	0.01\\
397.01	0.01\\
398.01	0.01\\
399.01	0.01\\
400.01	0.01\\
401.01	0.01\\
402.01	0.01\\
403.01	0.01\\
404.01	0.01\\
405.01	0.01\\
406.01	0.01\\
407.01	0.01\\
408.01	0.01\\
409.01	0.01\\
410.01	0.01\\
411.01	0.01\\
412.01	0.01\\
413.01	0.01\\
414.01	0.01\\
415.01	0.01\\
416.01	0.01\\
417.01	0.01\\
418.01	0.01\\
419.01	0.01\\
420.01	0.01\\
421.01	0.01\\
422.01	0.01\\
423.01	0.01\\
424.01	0.01\\
425.01	0.01\\
426.01	0.01\\
427.01	0.01\\
428.01	0.01\\
429.01	0.01\\
430.01	0.01\\
431.01	0.01\\
432.01	0.01\\
433.01	0.01\\
434.01	0.01\\
435.01	0.01\\
436.01	0.01\\
437.01	0.01\\
438.01	0.01\\
439.01	0.01\\
440.01	0.01\\
441.01	0.01\\
442.01	0.01\\
443.01	0.01\\
444.01	0.01\\
445.01	0.01\\
446.01	0.01\\
447.01	0.01\\
448.01	0.01\\
449.01	0.01\\
450.01	0.01\\
451.01	0.01\\
452.01	0.01\\
453.01	0.01\\
454.01	0.01\\
455.01	0.01\\
456.01	0.01\\
457.01	0.01\\
458.01	0.01\\
459.01	0.01\\
460.01	0.01\\
461.01	0.01\\
462.01	0.01\\
463.01	0.01\\
464.01	0.01\\
465.01	0.01\\
466.01	0.01\\
467.01	0.01\\
468.01	0.01\\
469.01	0.01\\
470.01	0.01\\
471.01	0.01\\
472.01	0.01\\
473.01	0.01\\
474.01	0.01\\
475.01	0.01\\
476.01	0.01\\
477.01	0.01\\
478.01	0.01\\
479.01	0.01\\
480.01	0.01\\
481.01	0.01\\
482.01	0.00997587649458391\\
483.01	0.00994281436826373\\
484.01	0.00990865576868689\\
485.01	0.00987335571535595\\
486.01	0.00983686864441155\\
487.01	0.00979914916206342\\
488.01	0.00976015312757781\\
489.01	0.0097198391823294\\
490.01	0.0096781708806394\\
491.01	0.0096351196302989\\
492.01	0.00959066872002735\\
493.01	0.00954481854877986\\
494.01	0.00949756230674207\\
495.01	0.00944884687152364\\
496.01	0.00939861087884548\\
497.01	0.00934679062273646\\
498.01	0.00929332052445587\\
499.01	0.0092381338739154\\
500.01	0.00918116394599725\\
501.01	0.00912234562935636\\
502.01	0.00906161775233871\\
503.01	0.00899892635353021\\
504.01	0.00893422402364727\\
505.01	0.00886742921998204\\
506.01	0.00879843256617422\\
507.01	0.00872711390830633\\
508.01	0.00865334091708237\\
509.01	0.00857696723504866\\
510.01	0.00849783027010523\\
511.01	0.00841574855662225\\
512.01	0.00833051858576653\\
513.01	0.00824191098148276\\
514.01	0.00814966586632992\\
515.01	0.00805348721986368\\
516.01	0.00795303597859557\\
517.01	0.00784792155686998\\
518.01	0.00773769137706917\\
519.01	0.00762181787816733\\
520.01	0.00749968231209421\\
521.01	0.00737055443836958\\
522.01	0.00723356698045552\\
523.01	0.00708768334921067\\
524.01	0.00693165666092587\\
525.01	0.00676423745727685\\
526.01	0.00658790083961235\\
527.01	0.00647971681111041\\
528.01	0.00639593571072038\\
529.01	0.00630965465279673\\
530.01	0.00622083657043316\\
531.01	0.00612945588072651\\
532.01	0.00603550315539981\\
533.01	0.00593899257231927\\
534.01	0.00583997738401568\\
535.01	0.00573855520496804\\
536.01	0.00563487457077089\\
537.01	0.00552914787384191\\
538.01	0.00542166736493526\\
539.01	0.00531282489136661\\
540.01	0.00520313623617997\\
541.01	0.00509327112131359\\
542.01	0.00498410056830288\\
543.01	0.00487687893575851\\
544.01	0.00477334731136774\\
545.01	0.00467422638434592\\
546.01	0.0045743314764834\\
547.01	0.00447337191168162\\
548.01	0.00437198627604522\\
549.01	0.00427105996371826\\
550.01	0.00417166426615998\\
551.01	0.00407132678206081\\
552.01	0.00396880225878627\\
553.01	0.00386438111845274\\
554.01	0.00375849222447324\\
555.01	0.00365203488305964\\
556.01	0.0035466544291015\\
557.01	0.00344282389172102\\
558.01	0.00334084018973639\\
559.01	0.00324086386752923\\
560.01	0.00314283492651795\\
561.01	0.00304683398660745\\
562.01	0.00295326966246949\\
563.01	0.00286249569317403\\
564.01	0.00277470624413143\\
565.01	0.00268886161041842\\
566.01	0.00260428765967366\\
567.01	0.00252097709498086\\
568.01	0.00243880615864478\\
569.01	0.00235772436845609\\
570.01	0.00227765771045678\\
571.01	0.00219845346857509\\
572.01	0.00211989051998918\\
573.01	0.0020417157060724\\
574.01	0.00196376477038438\\
575.01	0.0018860263484256\\
576.01	0.00180848634643038\\
577.01	0.00173110876709557\\
578.01	0.00165383891558031\\
579.01	0.00157660995977223\\
580.01	0.00149935332576446\\
581.01	0.00142201270423878\\
582.01	0.00134455890324395\\
583.01	0.00126698852963796\\
584.01	0.00118930562924654\\
585.01	0.00111151429061311\\
586.01	0.00103361667856622\\
587.01	0.000955612541206028\\
588.01	0.000877500028226702\\
589.01	0.000799275845239126\\
590.01	0.000720934182363075\\
591.01	0.000642465092883062\\
592.01	0.000563855076711632\\
593.01	0.00048508951620752\\
594.01	0.000406155478895408\\
595.01	0.000327044352021353\\
596.01	0.000247753662207873\\
597.01	0.000168287229632226\\
598.01	9.17536481397484e-05\\
599.01	2.94669794711974e-05\\
599.02	2.89574678339153e-05\\
599.03	2.84509918799507e-05\\
599.04	2.79475811456025e-05\\
599.05	2.74472654580235e-05\\
599.06	2.69500749380813e-05\\
599.07	2.64560400032848e-05\\
599.08	2.59651913706981e-05\\
599.09	2.54775600599076e-05\\
599.1	2.49931773960189e-05\\
599.11	2.45120750126723e-05\\
599.12	2.40342848550935e-05\\
599.13	2.35598391831755e-05\\
599.14	2.30887705745895e-05\\
599.15	2.26211119279309e-05\\
599.16	2.21568964658882e-05\\
599.17	2.16961577384437e-05\\
599.18	2.12389296261259e-05\\
599.19	2.07852463432537e-05\\
599.2	2.03351424412582e-05\\
599.21	1.98886528120085e-05\\
599.22	1.94458126911764e-05\\
599.23	1.90066576616355e-05\\
599.24	1.85712236569005e-05\\
599.25	1.81395469645886e-05\\
599.26	1.77116661135811e-05\\
599.27	1.72876212607056e-05\\
599.28	1.68674529623658e-05\\
599.29	1.64512021784951e-05\\
599.3	1.60389102765428e-05\\
599.31	1.56306190355025e-05\\
599.32	1.52263706499779e-05\\
599.33	1.48262077342837e-05\\
599.34	1.44301733266006e-05\\
599.35	1.40383108931594e-05\\
599.36	1.36506643324697e-05\\
599.37	1.3267277979595e-05\\
599.38	1.28881966104562e-05\\
599.39	1.25134654462011e-05\\
599.4	1.21431301575905e-05\\
599.41	1.17772368694487e-05\\
599.42	1.14158321651448e-05\\
599.43	1.10589630911238e-05\\
599.44	1.07066771614876e-05\\
599.45	1.03590223626028e-05\\
599.46	1.00160471577775e-05\\
599.47	9.67780049196745e-06\\
599.48	9.3443317965361e-06\\
599.49	9.01569099405995e-06\\
599.5	8.6919285031805e-06\\
599.51	8.37309524350481e-06\\
599.52	8.05924264055125e-06\\
599.53	7.75042263075933e-06\\
599.54	7.44668766651697e-06\\
599.55	7.14809072127447e-06\\
599.56	6.85468529468269e-06\\
599.57	6.56652541779733e-06\\
599.58	6.28366565832289e-06\\
599.59	6.00616112590886e-06\\
599.6	5.73406747752039e-06\\
599.61	5.46744092282461e-06\\
599.62	5.20633822966195e-06\\
599.63	4.95081672955734e-06\\
599.64	4.70093432328351e-06\\
599.65	4.45674948649534e-06\\
599.66	4.21832127539025e-06\\
599.67	3.98570933245884e-06\\
599.68	3.75897389226325e-06\\
599.69	3.53817578729702e-06\\
599.7	3.32337645388148e-06\\
599.71	3.11463793812965e-06\\
599.72	2.9120229019762e-06\\
599.73	2.71559462925938e-06\\
599.74	2.52541703184977e-06\\
599.75	2.34155465586756e-06\\
599.76	2.16407268794142e-06\\
599.77	1.99303696153501e-06\\
599.78	1.82851396333256e-06\\
599.79	1.67057083969718e-06\\
599.8	1.51927540317613e-06\\
599.81	1.37469613908231e-06\\
599.82	1.23690221214869e-06\\
599.83	1.10596347322606e-06\\
599.84	9.81950466062351e-07\\
599.85	8.64934434142636e-07\\
599.86	7.54987327612408e-07\\
599.87	6.52181810235561e-07\\
599.88	5.56591266461653e-07\\
599.89	4.68289808524397e-07\\
599.9	3.87352283644227e-07\\
599.91	3.13854281279446e-07\\
599.92	2.47872140451966e-07\\
599.93	1.89482957158038e-07\\
599.94	1.38764591836246e-07\\
599.95	9.57956769204876e-08\\
599.96	6.06556244623496e-08\\
599.97	3.34246338194039e-08\\
599.98	1.4183699454523e-08\\
599.99	3.014618757749e-09\\
600	0\\
};
\addplot [color=red!25!mycolor17,solid,forget plot]
  table[row sep=crcr]{%
0.01	0.00894151321774357\\
1.01	0.00894151261464888\\
2.01	0.00894151199851174\\
3.01	0.00894151136904826\\
4.01	0.00894151072596831\\
5.01	0.00894151006897544\\
6.01	0.00894150939776663\\
7.01	0.0089415087120323\\
8.01	0.00894150801145603\\
9.01	0.00894150729571435\\
10.01	0.00894150656447688\\
11.01	0.00894150581740579\\
12.01	0.00894150505415598\\
13.01	0.00894150427437465\\
14.01	0.00894150347770131\\
15.01	0.00894150266376757\\
16.01	0.00894150183219687\\
17.01	0.00894150098260437\\
18.01	0.00894150011459686\\
19.01	0.00894149922777238\\
20.01	0.00894149832172012\\
21.01	0.00894149739602027\\
22.01	0.00894149645024375\\
23.01	0.00894149548395206\\
24.01	0.00894149449669706\\
25.01	0.00894149348802066\\
26.01	0.0089414924574547\\
27.01	0.00894149140452075\\
28.01	0.00894149032872973\\
29.01	0.00894148922958186\\
30.01	0.00894148810656636\\
31.01	0.00894148695916115\\
32.01	0.00894148578683254\\
33.01	0.00894148458903512\\
34.01	0.00894148336521152\\
35.01	0.00894148211479195\\
36.01	0.00894148083719403\\
37.01	0.00894147953182262\\
38.01	0.00894147819806935\\
39.01	0.00894147683531239\\
40.01	0.00894147544291619\\
41.01	0.00894147402023112\\
42.01	0.00894147256659323\\
43.01	0.0089414710813238\\
44.01	0.00894146956372919\\
45.01	0.00894146801310022\\
46.01	0.00894146642871228\\
47.01	0.00894146480982446\\
48.01	0.00894146315567963\\
49.01	0.00894146146550371\\
50.01	0.00894145973850553\\
51.01	0.00894145797387646\\
52.01	0.00894145617078981\\
53.01	0.00894145432840061\\
54.01	0.00894145244584514\\
55.01	0.0089414505222405\\
56.01	0.00894144855668418\\
57.01	0.00894144654825368\\
58.01	0.00894144449600596\\
59.01	0.00894144239897699\\
60.01	0.00894144025618145\\
61.01	0.00894143806661196\\
62.01	0.00894143582923881\\
63.01	0.00894143354300944\\
64.01	0.00894143120684771\\
65.01	0.00894142881965376\\
66.01	0.00894142638030302\\
67.01	0.00894142388764601\\
68.01	0.00894142134050764\\
69.01	0.00894141873768662\\
70.01	0.00894141607795485\\
71.01	0.00894141336005689\\
72.01	0.0089414105827093\\
73.01	0.00894140774460003\\
74.01	0.00894140484438768\\
75.01	0.00894140188070104\\
76.01	0.00894139885213815\\
77.01	0.00894139575726586\\
78.01	0.00894139259461894\\
79.01	0.00894138936269947\\
80.01	0.00894138605997596\\
81.01	0.00894138268488271\\
82.01	0.00894137923581905\\
83.01	0.0089413757111485\\
84.01	0.00894137210919787\\
85.01	0.00894136842825662\\
86.01	0.00894136466657578\\
87.01	0.00894136082236734\\
88.01	0.0089413568938032\\
89.01	0.00894135287901419\\
90.01	0.00894134877608929\\
91.01	0.00894134458307466\\
92.01	0.00894134029797261\\
93.01	0.0089413359187405\\
94.01	0.00894133144328999\\
95.01	0.00894132686948578\\
96.01	0.00894132219514458\\
97.01	0.00894131741803402\\
98.01	0.00894131253587157\\
99.01	0.00894130754632342\\
100.01	0.00894130244700323\\
101.01	0.00894129723547087\\
102.01	0.00894129190923148\\
103.01	0.00894128646573383\\
104.01	0.00894128090236932\\
105.01	0.00894127521647054\\
106.01	0.00894126940530998\\
107.01	0.0089412634660986\\
108.01	0.00894125739598443\\
109.01	0.0089412511920512\\
110.01	0.00894124485131678\\
111.01	0.00894123837073165\\
112.01	0.00894123174717752\\
113.01	0.00894122497746555\\
114.01	0.0089412180583348\\
115.01	0.0089412109864507\\
116.01	0.00894120375840312\\
117.01	0.00894119637070482\\
118.01	0.00894118881978959\\
119.01	0.00894118110201047\\
120.01	0.00894117321363788\\
121.01	0.00894116515085766\\
122.01	0.00894115690976925\\
123.01	0.00894114848638355\\
124.01	0.00894113987662105\\
125.01	0.0089411310763095\\
126.01	0.00894112208118201\\
127.01	0.00894111288687472\\
128.01	0.00894110348892462\\
129.01	0.0089410938827672\\
130.01	0.0089410840637342\\
131.01	0.00894107402705106\\
132.01	0.0089410637678346\\
133.01	0.0089410532810905\\
134.01	0.00894104256171054\\
135.01	0.00894103160447012\\
136.01	0.00894102040402564\\
137.01	0.00894100895491155\\
138.01	0.00894099725153761\\
139.01	0.00894098528818587\\
140.01	0.00894097305900795\\
141.01	0.0089409605580218\\
142.01	0.00894094777910862\\
143.01	0.00894093471600977\\
144.01	0.00894092136232341\\
145.01	0.00894090771150117\\
146.01	0.00894089375684481\\
147.01	0.00894087949150265\\
148.01	0.00894086490846605\\
149.01	0.00894085000056567\\
150.01	0.00894083476046775\\
151.01	0.00894081918067024\\
152.01	0.00894080325349894\\
153.01	0.00894078697110343\\
154.01	0.00894077032545292\\
155.01	0.00894075330833198\\
156.01	0.00894073591133631\\
157.01	0.0089407181258683\\
158.01	0.00894069994313242\\
159.01	0.00894068135413064\\
160.01	0.00894066234965762\\
161.01	0.00894064292029593\\
162.01	0.00894062305641088\\
163.01	0.00894060274814567\\
164.01	0.00894058198541587\\
165.01	0.00894056075790433\\
166.01	0.00894053905505555\\
167.01	0.00894051686606997\\
168.01	0.00894049417989846\\
169.01	0.00894047098523631\\
170.01	0.00894044727051697\\
171.01	0.00894042302390632\\
172.01	0.00894039823329595\\
173.01	0.00894037288629684\\
174.01	0.0089403469702327\\
175.01	0.00894032047213323\\
176.01	0.00894029337872692\\
177.01	0.00894026567643424\\
178.01	0.00894023735136012\\
179.01	0.00894020838928645\\
180.01	0.00894017877566452\\
181.01	0.00894014849560717\\
182.01	0.00894011753388064\\
183.01	0.00894008587489632\\
184.01	0.0089400535027025\\
185.01	0.00894002040097549\\
186.01	0.00893998655301091\\
187.01	0.00893995194171456\\
188.01	0.00893991654959314\\
189.01	0.00893988035874466\\
190.01	0.00893984335084885\\
191.01	0.00893980550715699\\
192.01	0.00893976680848176\\
193.01	0.00893972723518686\\
194.01	0.00893968676717593\\
195.01	0.0089396453838819\\
196.01	0.00893960306425557\\
197.01	0.00893955978675409\\
198.01	0.00893951552932906\\
199.01	0.00893947026941452\\
200.01	0.00893942398391432\\
201.01	0.00893937664918967\\
202.01	0.00893932824104591\\
203.01	0.00893927873471925\\
204.01	0.00893922810486304\\
205.01	0.00893917632553366\\
206.01	0.00893912337017641\\
207.01	0.00893906921161051\\
208.01	0.00893901382201416\\
209.01	0.00893895717290917\\
210.01	0.0089388992351449\\
211.01	0.00893883997888221\\
212.01	0.00893877937357691\\
213.01	0.00893871738796242\\
214.01	0.00893865399003264\\
215.01	0.00893858914702397\\
216.01	0.00893852282539686\\
217.01	0.00893845499081734\\
218.01	0.0089383856081374\\
219.01	0.00893831464137563\\
220.01	0.00893824205369682\\
221.01	0.00893816780739137\\
222.01	0.0089380918638539\\
223.01	0.00893801418356186\\
224.01	0.00893793472605301\\
225.01	0.00893785344990261\\
226.01	0.00893777031270021\\
227.01	0.00893768527102556\\
228.01	0.00893759828042407\\
229.01	0.00893750929538178\\
230.01	0.00893741826929937\\
231.01	0.00893732515446611\\
232.01	0.00893722990203235\\
233.01	0.00893713246198218\\
234.01	0.00893703278310474\\
235.01	0.00893693081296528\\
236.01	0.00893682649787527\\
237.01	0.00893671978286185\\
238.01	0.00893661061163641\\
239.01	0.00893649892656259\\
240.01	0.0089363846686233\\
241.01	0.00893626777738715\\
242.01	0.00893614819097369\\
243.01	0.00893602584601828\\
244.01	0.00893590067763559\\
245.01	0.00893577261938262\\
246.01	0.00893564160322045\\
247.01	0.00893550755947528\\
248.01	0.00893537041679852\\
249.01	0.0089352301021258\\
250.01	0.00893508654063487\\
251.01	0.00893493965570274\\
252.01	0.00893478936886143\\
253.01	0.00893463559975303\\
254.01	0.00893447826608312\\
255.01	0.00893431728357363\\
256.01	0.00893415256591417\\
257.01	0.00893398402471211\\
258.01	0.00893381156944177\\
259.01	0.00893363510739197\\
260.01	0.00893345454361264\\
261.01	0.00893326978085985\\
262.01	0.00893308071953978\\
263.01	0.00893288725765067\\
264.01	0.00893268929072435\\
265.01	0.00893248671176548\\
266.01	0.00893227941118956\\
267.01	0.0089320672767597\\
268.01	0.00893185019352134\\
269.01	0.00893162804373585\\
270.01	0.00893140070681221\\
271.01	0.008931168059237\\
272.01	0.00893092997450288\\
273.01	0.00893068632303518\\
274.01	0.00893043697211676\\
275.01	0.00893018178581092\\
276.01	0.00892992062488261\\
277.01	0.00892965334671749\\
278.01	0.00892937980523929\\
279.01	0.00892909985082486\\
280.01	0.00892881333021744\\
281.01	0.00892852008643756\\
282.01	0.00892821995869189\\
283.01	0.00892791278227995\\
284.01	0.0089275983884984\\
285.01	0.00892727660454309\\
286.01	0.0089269472534087\\
287.01	0.0089266101537859\\
288.01	0.00892626511995626\\
289.01	0.00892591196168419\\
290.01	0.00892555048410659\\
291.01	0.00892518048761965\\
292.01	0.00892480176776308\\
293.01	0.00892441411510132\\
294.01	0.00892401731510211\\
295.01	0.00892361114801182\\
296.01	0.00892319538872798\\
297.01	0.00892276980666865\\
298.01	0.00892233416563886\\
299.01	0.00892188822369321\\
300.01	0.00892143173299576\\
301.01	0.00892096443967636\\
302.01	0.00892048608368312\\
303.01	0.00891999639863201\\
304.01	0.00891949511165199\\
305.01	0.00891898194322718\\
306.01	0.00891845660703449\\
307.01	0.00891791880977794\\
308.01	0.00891736825101831\\
309.01	0.00891680462299919\\
310.01	0.00891622761046853\\
311.01	0.00891563689049579\\
312.01	0.00891503213228483\\
313.01	0.00891441299698199\\
314.01	0.00891377913747952\\
315.01	0.0089131301982144\\
316.01	0.00891246581496189\\
317.01	0.00891178561462424\\
318.01	0.00891108921501403\\
319.01	0.00891037622463211\\
320.01	0.00890964624244024\\
321.01	0.00890889885762785\\
322.01	0.00890813364937315\\
323.01	0.008907350186598\\
324.01	0.00890654802771699\\
325.01	0.00890572672037978\\
326.01	0.00890488580120724\\
327.01	0.00890402479552079\\
328.01	0.00890314321706468\\
329.01	0.00890224056772141\\
330.01	0.00890131633721959\\
331.01	0.00890037000283452\\
332.01	0.00889940102908055\\
333.01	0.00889840886739597\\
334.01	0.0088973929558192\\
335.01	0.00889635271865654\\
336.01	0.00889528756614137\\
337.01	0.00889419689408379\\
338.01	0.00889308008351116\\
339.01	0.00889193650029912\\
340.01	0.00889076549479213\\
341.01	0.00888956640141393\\
342.01	0.00888833853826738\\
343.01	0.00888708120672304\\
344.01	0.00888579369099642\\
345.01	0.00888447525771335\\
346.01	0.00888312515546309\\
347.01	0.00888174261433902\\
348.01	0.0088803268454657\\
349.01	0.00887887704051307\\
350.01	0.00887739237119596\\
351.01	0.00887587198875959\\
352.01	0.00887431502344956\\
353.01	0.0088727205839668\\
354.01	0.00887108775690598\\
355.01	0.0088694156061774\\
356.01	0.00886770317241179\\
357.01	0.00886594947234695\\
358.01	0.00886415349819599\\
359.01	0.00886231421699673\\
360.01	0.00886043056994076\\
361.01	0.00885850147168265\\
362.01	0.00885652580962747\\
363.01	0.00885450244319672\\
364.01	0.00885243020307152\\
365.01	0.00885030789041235\\
366.01	0.00884813427605492\\
367.01	0.00884590809968062\\
368.01	0.00884362806896159\\
369.01	0.00884129285867894\\
370.01	0.00883890110981364\\
371.01	0.008836451428609\\
372.01	0.00883394238560412\\
373.01	0.00883137251463696\\
374.01	0.00882874031181643\\
375.01	0.00882604423446269\\
376.01	0.00882328270001422\\
377.01	0.00882045408490118\\
378.01	0.00881755672338373\\
379.01	0.00881458890635412\\
380.01	0.00881154888010224\\
381.01	0.00880843484504269\\
382.01	0.00880524495440249\\
383.01	0.00880197731286887\\
384.01	0.00879862997519489\\
385.01	0.00879520094476274\\
386.01	0.0087916881721026\\
387.01	0.00878808955336624\\
388.01	0.00878440292875404\\
389.01	0.0087806260808934\\
390.01	0.00877675673316796\\
391.01	0.00877279254799538\\
392.01	0.00876873112505233\\
393.01	0.00876456999944523\\
394.01	0.00876030663982492\\
395.01	0.00875593844644316\\
396.01	0.00875146274914971\\
397.01	0.00874687680532766\\
398.01	0.00874217779776539\\
399.01	0.0087373628324628\\
400.01	0.00873242893637046\\
401.01	0.00872737305505869\\
402.01	0.00872219205031533\\
403.01	0.00871688269766945\\
404.01	0.00871144168383938\\
405.01	0.0087058656041022\\
406.01	0.0087001509595828\\
407.01	0.00869429415446006\\
408.01	0.00868829149308723\\
409.01	0.00868213917702424\\
410.01	0.00867583330197849\\
411.01	0.00866936985465104\\
412.01	0.00866274470948433\\
413.01	0.0086559536253075\\
414.01	0.00864899224187483\\
415.01	0.00864185607629205\\
416.01	0.00863454051932611\\
417.01	0.00862704083159274\\
418.01	0.00861935213961748\\
419.01	0.00861146943176589\\
420.01	0.00860338755403901\\
421.01	0.00859510120573041\\
422.01	0.00858660493493794\\
423.01	0.00857789313391867\\
424.01	0.0085689600342695\\
425.01	0.00855979970191195\\
426.01	0.00855040603185129\\
427.01	0.00854077274267304\\
428.01	0.00853089337072811\\
429.01	0.00852076126394522\\
430.01	0.00851036957519363\\
431.01	0.00849971125510028\\
432.01	0.0084887790442035\\
433.01	0.00847756546430114\\
434.01	0.00846606280882325\\
435.01	0.00845426313203126\\
436.01	0.00844215823681791\\
437.01	0.00842973966085852\\
438.01	0.00841699866085121\\
439.01	0.00840392619458725\\
440.01	0.00839051290063101\\
441.01	0.00837674907547543\\
442.01	0.00836262464820484\\
443.01	0.00834812915298423\\
444.01	0.00833325170015707\\
445.01	0.00831798094745767\\
446.01	0.00830230507491702\\
447.01	0.00828621177179482\\
448.01	0.00826968821763505\\
449.01	0.00825272104924269\\
450.01	0.00823529632331078\\
451.01	0.00821739947509464\\
452.01	0.00819901527257593\\
453.01	0.00818012776546474\\
454.01	0.00816072022827056\\
455.01	0.00814077509653561\\
456.01	0.00812027389515372\\
457.01	0.00809919715749363\\
458.01	0.00807752433379312\\
459.01	0.00805523368698338\\
460.01	0.00803230217372145\\
461.01	0.00800870530793988\\
462.01	0.00798441700363633\\
463.01	0.00795940939289735\\
464.01	0.00793365261423636\\
465.01	0.00790711456517616\\
466.01	0.0078797606115569\\
467.01	0.00785155324421342\\
468.01	0.00782245167133338\\
469.01	0.00779241133183461\\
470.01	0.00776138331130186\\
471.01	0.0077293136371557\\
472.01	0.00769614242347331\\
473.01	0.00766180282783474\\
474.01	0.00762621977218363\\
475.01	0.00758930836627548\\
476.01	0.00755097195491648\\
477.01	0.00751109968769789\\
478.01	0.00746956348057245\\
479.01	0.00742621419992225\\
480.01	0.00738087684906441\\
481.01	0.00733334447066516\\
482.01	0.00730772838409598\\
483.01	0.00728921539375883\\
484.01	0.00726986604526977\\
485.01	0.0072495895299529\\
486.01	0.00722827672678798\\
487.01	0.00720579559218599\\
488.01	0.00718198527132034\\
489.01	0.00715664855337426\\
490.01	0.00712954217595951\\
491.01	0.00710036432905067\\
492.01	0.00706873850339288\\
493.01	0.00703426993795644\\
494.01	0.00699822049880224\\
495.01	0.006961198131097\\
496.01	0.00692317937685264\\
497.01	0.00688413949676074\\
498.01	0.00684405178854448\\
499.01	0.00680288658077105\\
500.01	0.00676060976968188\\
501.01	0.00671718071535282\\
502.01	0.00667254924184988\\
503.01	0.00662665139217403\\
504.01	0.0065794087358317\\
505.01	0.00653076848116818\\
506.01	0.00648069800907385\\
507.01	0.00642916762080523\\
508.01	0.00637615143522\\
509.01	0.00632162873443743\\
510.01	0.00626558567286279\\
511.01	0.00620801745221732\\
512.01	0.00614893109665368\\
513.01	0.00608834900380557\\
514.01	0.00602631350328169\\
515.01	0.00596289272854014\\
516.01	0.00589818820786404\\
517.01	0.00583234471434273\\
518.01	0.00576556309567665\\
519.01	0.00569811707126496\\
520.01	0.00563037536487652\\
521.01	0.00556283014420392\\
522.01	0.00549613352115219\\
523.01	0.0054311450531315\\
524.01	0.00536899375642301\\
525.01	0.00531089743934325\\
526.01	0.00525433400642232\\
527.01	0.00519790383802797\\
528.01	0.00513996246353109\\
529.01	0.00508042133421524\\
530.01	0.00501928811296119\\
531.01	0.0049565761873247\\
532.01	0.00489230154984199\\
533.01	0.00482647312224399\\
534.01	0.00475908001949976\\
535.01	0.00469011082236774\\
536.01	0.00461955953879067\\
537.01	0.0045474267640274\\
538.01	0.00447372112767133\\
539.01	0.00439846114084963\\
540.01	0.00432167761323271\\
541.01	0.00424341689448369\\
542.01	0.00416374521879202\\
543.01	0.00408275070957415\\
544.01	0.00400054871057547\\
545.01	0.00391793968658502\\
546.01	0.00383545025872607\\
547.01	0.00375332823407465\\
548.01	0.00367184062787143\\
549.01	0.0035912528713037\\
550.01	0.00351180585965046\\
551.01	0.0034337427002255\\
552.01	0.00335737485324567\\
553.01	0.00328298039254646\\
554.01	0.00321074134821009\\
555.01	0.00314037875271919\\
556.01	0.00307077499893151\\
557.01	0.00300188263613312\\
558.01	0.00293373027196458\\
559.01	0.00286631379211223\\
560.01	0.00279960126959131\\
561.01	0.00273353630284404\\
562.01	0.0026680150261463\\
563.01	0.0026028766296019\\
564.01	0.00253791314161076\\
565.01	0.00247294065430676\\
566.01	0.00240790287881855\\
567.01	0.00234275431907017\\
568.01	0.00227743915861326\\
569.01	0.00221189383001435\\
570.01	0.00214604795923127\\
571.01	0.00207983098516851\\
572.01	0.00201318079937052\\
573.01	0.00194605206981758\\
574.01	0.0018784185498138\\
575.01	0.00181025995554062\\
576.01	0.00174155473393376\\
577.01	0.0016722815447536\\
578.01	0.00160242126650577\\
579.01	0.00153195886904355\\
580.01	0.00146088474238344\\
581.01	0.00138919499581868\\
582.01	0.0013168903329978\\
583.01	0.00124397413150743\\
584.01	0.00117045174146673\\
585.01	0.00109633076541331\\
586.01	0.00102162149322053\\
587.01	0.000946337383085976\\
588.01	0.000870495457990504\\
589.01	0.000794116593283583\\
590.01	0.00071722578280283\\
591.01	0.000639852503519138\\
592.01	0.00056203108043296\\
593.01	0.000483800710259085\\
594.01	0.000405204886350921\\
595.01	0.000326290009844911\\
596.01	0.000247102975742894\\
597.01	0.000167687548540458\\
598.01	9.17367679171607e-05\\
599.01	2.94669369274633e-05\\
599.02	2.89574274290674e-05\\
599.03	2.84509535274195e-05\\
599.04	2.7947544761462e-05\\
599.05	2.7447230960936e-05\\
599.06	2.69500422492523e-05\\
599.07	2.64560090463949e-05\\
599.08	2.5965162071857e-05\\
599.09	2.54775323476081e-05\\
599.1	2.49931512010769e-05\\
599.11	2.45120502681759e-05\\
599.12	2.40342614963567e-05\\
599.13	2.35598171476892e-05\\
599.14	2.30887498019697e-05\\
599.15	2.26210923598685e-05\\
599.16	2.21568780461034e-05\\
599.17	2.16961404126415e-05\\
599.18	2.12389133419366e-05\\
599.19	2.07852310502073e-05\\
599.2	2.03351280907219e-05\\
599.21	1.98886393571462e-05\\
599.22	1.94458000869063e-05\\
599.23	1.90066458645843e-05\\
599.24	1.85712126253585e-05\\
599.25	1.8139536658468e-05\\
599.26	1.77116564944301e-05\\
599.27	1.72876122915989e-05\\
599.28	1.68674446078751e-05\\
599.29	1.64511944046493e-05\\
599.3	1.60389030507795e-05\\
599.31	1.56306123266399e-05\\
599.32	1.52263644281631e-05\\
599.33	1.48262019709685e-05\\
599.34	1.44301679944961e-05\\
599.35	1.40383059661995e-05\\
599.36	1.36506597857768e-05\\
599.37	1.32672737894416e-05\\
599.38	1.28881927542353e-05\\
599.39	1.25134619023814e-05\\
599.4	1.21431269056952e-05\\
599.41	1.17772338900073e-05\\
599.42	1.14158294396705e-05\\
599.43	1.10589606020818e-05\\
599.44	1.07066748922489e-05\\
599.45	1.03590202974301e-05\\
599.46	1.00160452817816e-05\\
599.47	9.67779879108323e-06\\
599.48	9.34433025748943e-06\\
599.49	9.01568960434002e-06\\
599.5	8.69192725100679e-06\\
599.51	8.37309411780285e-06\\
599.52	8.0592416309283e-06\\
599.53	7.75042172746279e-06\\
599.54	7.4466868604222e-06\\
599.55	7.14809000385182e-06\\
599.56	6.85468465797499e-06\\
599.57	6.5665248543869e-06\\
599.58	6.283665161309e-06\\
599.59	6.00616068889559e-06\\
599.6	5.73406709457672e-06\\
599.61	5.46744058846881e-06\\
599.62	5.20633793883903e-06\\
599.63	4.95081647761304e-06\\
599.64	4.70093410595215e-06\\
599.65	4.45674929985991e-06\\
599.66	4.21832111588243e-06\\
599.67	3.98570919682602e-06\\
599.68	3.75897377755986e-06\\
599.69	3.5381756908516e-06\\
599.7	3.32337637328622e-06\\
599.71	3.11463787123004e-06\\
599.72	2.91202284684322e-06\\
599.73	2.71559458416872e-06\\
599.74	2.52541699527833e-06\\
599.75	2.34155462647441e-06\\
599.76	2.16407266454867e-06\\
599.77	1.99303694311918e-06\\
599.78	1.82851394900548e-06\\
599.79	1.67057082869557e-06\\
599.8	1.51927539484599e-06\\
599.81	1.37469613288067e-06\\
599.82	1.23690220761412e-06\\
599.83	1.10596346997692e-06\\
599.84	9.81950463786394e-07\\
599.85	8.64934432593528e-07\\
599.86	7.54987326588921e-07\\
599.87	6.5218180958504e-07\\
599.88	5.56591266064402e-07\\
599.89	4.68289808293679e-07\\
599.9	3.87352283521061e-07\\
599.91	3.13854281218731e-07\\
599.92	2.47872140425945e-07\\
599.93	1.89482957149364e-07\\
599.94	1.38764591834512e-07\\
599.95	9.57956769222224e-08\\
599.96	6.06556244606149e-08\\
599.97	3.34246338211386e-08\\
599.98	1.4183699454523e-08\\
599.99	3.01461875948372e-09\\
600	0\\
};
\addplot [color=mycolor19,solid,forget plot]
  table[row sep=crcr]{%
0.01	0.00739460781638744\\
1.01	0.00739460735721567\\
2.01	0.00739460688814162\\
3.01	0.00739460640895054\\
4.01	0.00739460591942291\\
5.01	0.00739460541933461\\
6.01	0.00739460490845621\\
7.01	0.00739460438655362\\
8.01	0.00739460385338746\\
9.01	0.00739460330871333\\
10.01	0.00739460275228116\\
11.01	0.00739460218383558\\
12.01	0.00739460160311581\\
13.01	0.00739460100985508\\
14.01	0.00739460040378089\\
15.01	0.00739459978461482\\
16.01	0.00739459915207223\\
17.01	0.00739459850586236\\
18.01	0.00739459784568802\\
19.01	0.00739459717124554\\
20.01	0.00739459648222471\\
21.01	0.00739459577830832\\
22.01	0.00739459505917233\\
23.01	0.00739459432448543\\
24.01	0.00739459357390921\\
25.01	0.00739459280709796\\
26.01	0.00739459202369806\\
27.01	0.00739459122334836\\
28.01	0.00739459040567979\\
29.01	0.00739458957031496\\
30.01	0.00739458871686853\\
31.01	0.00739458784494637\\
32.01	0.00739458695414586\\
33.01	0.0073945860440556\\
34.01	0.00739458511425492\\
35.01	0.00739458416431424\\
36.01	0.00739458319379423\\
37.01	0.00739458220224598\\
38.01	0.00739458118921078\\
39.01	0.0073945801542198\\
40.01	0.00739457909679383\\
41.01	0.0073945780164433\\
42.01	0.00739457691266755\\
43.01	0.00739457578495535\\
44.01	0.00739457463278377\\
45.01	0.00739457345561862\\
46.01	0.00739457225291381\\
47.01	0.00739457102411131\\
48.01	0.00739456976864065\\
49.01	0.00739456848591905\\
50.01	0.00739456717535065\\
51.01	0.00739456583632645\\
52.01	0.0073945644682242\\
53.01	0.0073945630704077\\
54.01	0.00739456164222689\\
55.01	0.00739456018301725\\
56.01	0.00739455869209964\\
57.01	0.00739455716877984\\
58.01	0.00739455561234832\\
59.01	0.00739455402208016\\
60.01	0.00739455239723384\\
61.01	0.00739455073705207\\
62.01	0.0073945490407604\\
63.01	0.00739454730756729\\
64.01	0.00739454553666394\\
65.01	0.00739454372722328\\
66.01	0.00739454187840033\\
67.01	0.0073945399893311\\
68.01	0.00739453805913265\\
69.01	0.00739453608690237\\
70.01	0.0073945340717177\\
71.01	0.00739453201263567\\
72.01	0.00739452990869233\\
73.01	0.00739452775890244\\
74.01	0.00739452556225888\\
75.01	0.00739452331773208\\
76.01	0.00739452102426993\\
77.01	0.00739451868079665\\
78.01	0.00739451628621268\\
79.01	0.00739451383939399\\
80.01	0.00739451133919172\\
81.01	0.00739450878443146\\
82.01	0.00739450617391255\\
83.01	0.00739450350640777\\
84.01	0.00739450078066259\\
85.01	0.00739449799539457\\
86.01	0.00739449514929286\\
87.01	0.00739449224101712\\
88.01	0.00739448926919739\\
89.01	0.00739448623243329\\
90.01	0.00739448312929309\\
91.01	0.00739447995831325\\
92.01	0.00739447671799745\\
93.01	0.00739447340681626\\
94.01	0.00739447002320606\\
95.01	0.00739446656556814\\
96.01	0.00739446303226845\\
97.01	0.00739445942163625\\
98.01	0.00739445573196368\\
99.01	0.00739445196150452\\
100.01	0.00739444810847374\\
101.01	0.00739444417104638\\
102.01	0.00739444014735652\\
103.01	0.00739443603549671\\
104.01	0.00739443183351683\\
105.01	0.00739442753942302\\
106.01	0.00739442315117697\\
107.01	0.00739441866669458\\
108.01	0.00739441408384518\\
109.01	0.00739440940045021\\
110.01	0.00739440461428254\\
111.01	0.00739439972306501\\
112.01	0.00739439472446939\\
113.01	0.00739438961611521\\
114.01	0.00739438439556867\\
115.01	0.00739437906034121\\
116.01	0.00739437360788864\\
117.01	0.00739436803560947\\
118.01	0.00739436234084363\\
119.01	0.00739435652087173\\
120.01	0.00739435057291278\\
121.01	0.00739434449412337\\
122.01	0.0073943382815963\\
123.01	0.00739433193235886\\
124.01	0.00739432544337121\\
125.01	0.00739431881152551\\
126.01	0.00739431203364364\\
127.01	0.0073943051064758\\
128.01	0.00739429802669893\\
129.01	0.00739429079091534\\
130.01	0.00739428339565034\\
131.01	0.00739427583735108\\
132.01	0.0073942681123843\\
133.01	0.00739426021703493\\
134.01	0.00739425214750385\\
135.01	0.00739424389990599\\
136.01	0.00739423547026865\\
137.01	0.00739422685452913\\
138.01	0.00739421804853298\\
139.01	0.00739420904803167\\
140.01	0.00739419984868031\\
141.01	0.00739419044603592\\
142.01	0.00739418083555458\\
143.01	0.00739417101258945\\
144.01	0.00739416097238848\\
145.01	0.00739415071009162\\
146.01	0.00739414022072869\\
147.01	0.00739412949921663\\
148.01	0.00739411854035687\\
149.01	0.00739410733883286\\
150.01	0.00739409588920699\\
151.01	0.00739408418591852\\
152.01	0.00739407222327968\\
153.01	0.00739405999547342\\
154.01	0.00739404749655035\\
155.01	0.00739403472042565\\
156.01	0.00739402166087577\\
157.01	0.00739400831153534\\
158.01	0.00739399466589381\\
159.01	0.0073939807172921\\
160.01	0.00739396645891918\\
161.01	0.00739395188380848\\
162.01	0.00739393698483459\\
163.01	0.00739392175470884\\
164.01	0.00739390618597618\\
165.01	0.00739389027101107\\
166.01	0.00739387400201339\\
167.01	0.00739385737100459\\
168.01	0.00739384036982332\\
169.01	0.0073938229901212\\
170.01	0.00739380522335864\\
171.01	0.00739378706080012\\
172.01	0.00739376849350976\\
173.01	0.00739374951234659\\
174.01	0.00739373010795965\\
175.01	0.0073937102707832\\
176.01	0.00739368999103181\\
177.01	0.00739366925869482\\
178.01	0.00739364806353133\\
179.01	0.00739362639506489\\
180.01	0.00739360424257758\\
181.01	0.00739358159510467\\
182.01	0.00739355844142866\\
183.01	0.0073935347700736\\
184.01	0.00739351056929848\\
185.01	0.00739348582709131\\
186.01	0.00739346053116302\\
187.01	0.00739343466894026\\
188.01	0.00739340822755921\\
189.01	0.00739338119385852\\
190.01	0.00739335355437231\\
191.01	0.0073933252953229\\
192.01	0.00739329640261368\\
193.01	0.00739326686182099\\
194.01	0.00739323665818721\\
195.01	0.00739320577661214\\
196.01	0.00739317420164503\\
197.01	0.00739314191747661\\
198.01	0.00739310890793014\\
199.01	0.00739307515645301\\
200.01	0.00739304064610744\\
201.01	0.00739300535956169\\
202.01	0.00739296927908054\\
203.01	0.00739293238651537\\
204.01	0.00739289466329505\\
205.01	0.00739285609041505\\
206.01	0.00739281664842748\\
207.01	0.0073927763174307\\
208.01	0.00739273507705801\\
209.01	0.00739269290646691\\
210.01	0.00739264978432763\\
211.01	0.00739260568881128\\
212.01	0.00739256059757806\\
213.01	0.00739251448776509\\
214.01	0.00739246733597363\\
215.01	0.0073924191182565\\
216.01	0.00739236981010455\\
217.01	0.00739231938643358\\
218.01	0.00739226782157008\\
219.01	0.00739221508923747\\
220.01	0.00739216116254139\\
221.01	0.00739210601395458\\
222.01	0.00739204961530218\\
223.01	0.00739199193774552\\
224.01	0.00739193295176645\\
225.01	0.00739187262715094\\
226.01	0.00739181093297207\\
227.01	0.00739174783757285\\
228.01	0.00739168330854878\\
229.01	0.0073916173127293\\
230.01	0.00739154981615984\\
231.01	0.00739148078408217\\
232.01	0.00739141018091542\\
233.01	0.0073913379702357\\
234.01	0.00739126411475604\\
235.01	0.00739118857630508\\
236.01	0.00739111131580584\\
237.01	0.00739103229325327\\
238.01	0.00739095146769215\\
239.01	0.00739086879719365\\
240.01	0.00739078423883171\\
241.01	0.0073906977486585\\
242.01	0.00739060928167972\\
243.01	0.00739051879182908\\
244.01	0.00739042623194195\\
245.01	0.00739033155372833\\
246.01	0.00739023470774575\\
247.01	0.0073901356433708\\
248.01	0.00739003430877038\\
249.01	0.00738993065087156\\
250.01	0.00738982461533181\\
251.01	0.00738971614650737\\
252.01	0.0073896051874218\\
253.01	0.00738949167973259\\
254.01	0.00738937556369788\\
255.01	0.00738925677814244\\
256.01	0.00738913526042168\\
257.01	0.00738901094638614\\
258.01	0.00738888377034369\\
259.01	0.00738875366502219\\
260.01	0.00738862056153028\\
261.01	0.00738848438931714\\
262.01	0.00738834507613185\\
263.01	0.00738820254798125\\
264.01	0.00738805672908689\\
265.01	0.00738790754184095\\
266.01	0.00738775490676085\\
267.01	0.00738759874244298\\
268.01	0.00738743896551507\\
269.01	0.00738727549058717\\
270.01	0.0073871082302019\\
271.01	0.00738693709478314\\
272.01	0.00738676199258338\\
273.01	0.00738658282962967\\
274.01	0.00738639950966869\\
275.01	0.00738621193410975\\
276.01	0.00738602000196675\\
277.01	0.00738582360979866\\
278.01	0.00738562265164842\\
279.01	0.00738541701898025\\
280.01	0.00738520660061536\\
281.01	0.00738499128266623\\
282.01	0.00738477094846914\\
283.01	0.00738454547851474\\
284.01	0.00738431475037708\\
285.01	0.00738407863864118\\
286.01	0.00738383701482791\\
287.01	0.00738358974731779\\
288.01	0.00738333670127216\\
289.01	0.00738307773855345\\
290.01	0.0073828127176421\\
291.01	0.00738254149355236\\
292.01	0.0073822639177452\\
293.01	0.00738197983803984\\
294.01	0.00738168909852228\\
295.01	0.007381391539452\\
296.01	0.00738108699716632\\
297.01	0.00738077530398194\\
298.01	0.0073804562880943\\
299.01	0.00738012977347433\\
300.01	0.00737979557976289\\
301.01	0.0073794535221616\\
302.01	0.00737910341132239\\
303.01	0.00737874505323258\\
304.01	0.00737837824909876\\
305.01	0.00737800279522633\\
306.01	0.00737761848289708\\
307.01	0.00737722509824278\\
308.01	0.00737682242211633\\
309.01	0.00737641022995919\\
310.01	0.00737598829166552\\
311.01	0.00737555637144354\\
312.01	0.00737511422767252\\
313.01	0.00737466161275675\\
314.01	0.00737419827297554\\
315.01	0.00737372394832979\\
316.01	0.0073732383723844\\
317.01	0.00737274127210699\\
318.01	0.00737223236770241\\
319.01	0.00737171137244328\\
320.01	0.00737117799249619\\
321.01	0.00737063192674344\\
322.01	0.00737007286660059\\
323.01	0.00736950049582947\\
324.01	0.00736891449034603\\
325.01	0.00736831451802444\\
326.01	0.0073677002384949\\
327.01	0.00736707130293777\\
328.01	0.00736642735387172\\
329.01	0.00736576802493683\\
330.01	0.00736509294067259\\
331.01	0.00736440171628967\\
332.01	0.00736369395743714\\
333.01	0.00736296925996266\\
334.01	0.0073622272096674\\
335.01	0.00736146738205507\\
336.01	0.00736068934207347\\
337.01	0.00735989264385179\\
338.01	0.00735907683042887\\
339.01	0.00735824143347624\\
340.01	0.00735738597301394\\
341.01	0.00735650995711878\\
342.01	0.00735561288162585\\
343.01	0.00735469422982191\\
344.01	0.00735375347213177\\
345.01	0.00735279006579585\\
346.01	0.00735180345454027\\
347.01	0.00735079306823823\\
348.01	0.00734975832256253\\
349.01	0.00734869861862977\\
350.01	0.00734761334263451\\
351.01	0.00734650186547459\\
352.01	0.00734536354236683\\
353.01	0.00734419771245164\\
354.01	0.00734300369838831\\
355.01	0.00734178080593891\\
356.01	0.00734052832354109\\
357.01	0.00733924552186971\\
358.01	0.00733793165338621\\
359.01	0.00733658595187573\\
360.01	0.00733520763197159\\
361.01	0.00733379588866597\\
362.01	0.00733234989680765\\
363.01	0.00733086881058492\\
364.01	0.0073293517629934\\
365.01	0.00732779786528957\\
366.01	0.00732620620642631\\
367.01	0.00732457585247379\\
368.01	0.0073229058460219\\
369.01	0.00732119520556453\\
370.01	0.00731944292486566\\
371.01	0.00731764797230505\\
372.01	0.00731580929020395\\
373.01	0.00731392579412926\\
374.01	0.00731199637217588\\
375.01	0.00731001988422509\\
376.01	0.00730799516117935\\
377.01	0.00730592100417133\\
378.01	0.00730379618374717\\
379.01	0.00730161943902166\\
380.01	0.0072993894768053\\
381.01	0.00729710497070073\\
382.01	0.00729476456016879\\
383.01	0.00729236684956112\\
384.01	0.00728991040711958\\
385.01	0.00728739376393997\\
386.01	0.0072848154128984\\
387.01	0.00728217380754028\\
388.01	0.00727946736092822\\
389.01	0.00727669444444872\\
390.01	0.00727385338657547\\
391.01	0.00727094247158749\\
392.01	0.00726795993824026\\
393.01	0.00726490397838857\\
394.01	0.00726177273555814\\
395.01	0.00725856430346603\\
396.01	0.00725527672448628\\
397.01	0.00725190798806082\\
398.01	0.00724845602905235\\
399.01	0.00724491872604027\\
400.01	0.00724129389955609\\
401.01	0.00723757931025992\\
402.01	0.00723377265705657\\
403.01	0.00722987157515321\\
404.01	0.00722587363405743\\
405.01	0.00722177633552025\\
406.01	0.00721757711142594\\
407.01	0.00721327332163232\\
408.01	0.00720886225176726\\
409.01	0.00720434111098601\\
410.01	0.00719970702969591\\
411.01	0.00719495705725344\\
412.01	0.00719008815963761\\
413.01	0.00718509721710061\\
414.01	0.00717998102179087\\
415.01	0.00717473627533758\\
416.01	0.00716935958637324\\
417.01	0.00716384746795955\\
418.01	0.00715819633486804\\
419.01	0.00715240250065579\\
420.01	0.00714646217447976\\
421.01	0.00714037145762384\\
422.01	0.00713412633977827\\
423.01	0.00712772269510845\\
424.01	0.00712115627811266\\
425.01	0.00711442271926145\\
426.01	0.00710751752041582\\
427.01	0.00710043605002663\\
428.01	0.00709317353812675\\
429.01	0.00708572507113625\\
430.01	0.00707808558651451\\
431.01	0.00707024986731098\\
432.01	0.00706221253668515\\
433.01	0.00705396805249208\\
434.01	0.00704551070205724\\
435.01	0.00703683459729696\\
436.01	0.00702793367037132\\
437.01	0.0070188016700894\\
438.01	0.00700943215930772\\
439.01	0.00699981851357164\\
440.01	0.0069899539212232\\
441.01	0.00697983138512249\\
442.01	0.00696944372596999\\
443.01	0.00695878358691883\\
444.01	0.00694784343866876\\
445.01	0.00693661558342861\\
446.01	0.00692509215389056\\
447.01	0.00691326509820568\\
448.01	0.00690112616826287\\
449.01	0.00688866691980275\\
450.01	0.00687587871469878\\
451.01	0.00686275272483023\\
452.01	0.00684927993790207\\
453.01	0.0068354511656438\\
454.01	0.0068212570549198\\
455.01	0.00680668810240477\\
456.01	0.00679173467363303\\
457.01	0.00677638702742366\\
458.01	0.00676063534693086\\
459.01	0.00674446977887691\\
460.01	0.00672788048292408\\
461.01	0.00671085769364399\\
462.01	0.00669339179818848\\
463.01	0.00667547343359382\\
464.01	0.00665709360870891\\
465.01	0.00663824385710418\\
466.01	0.00661891642907578\\
467.01	0.00659910453312369\\
468.01	0.00657880264020722\\
469.01	0.00655800686785727\\
470.01	0.00653671546609859\\
471.01	0.00651492943343259\\
472.01	0.00649265329926033\\
473.01	0.00646989611961106\\
474.01	0.0064466727465601\\
475.01	0.00642300544911735\\
476.01	0.00639892598551904\\
477.01	0.00637447825521281\\
478.01	0.00634972170595724\\
479.01	0.00632473573363495\\
480.01	0.00629962538319443\\
481.01	0.00627452875737553\\
482.01	0.00624939051251928\\
483.01	0.00622376064740071\\
484.01	0.00619765015738636\\
485.01	0.0061711067485383\\
486.01	0.00614419621701035\\
487.01	0.00611700764458573\\
488.01	0.00608966011335451\\
489.01	0.00606231140357749\\
490.01	0.00603516928644871\\
491.01	0.00600850621982586\\
492.01	0.00598267851658965\\
493.01	0.00595807385714244\\
494.01	0.00593342118060778\\
495.01	0.00590807476816829\\
496.01	0.00588201498763912\\
497.01	0.00585522118092143\\
498.01	0.00582767155076743\\
499.01	0.00579934311057551\\
500.01	0.00577021176562901\\
501.01	0.00574025263987014\\
502.01	0.00570944089116093\\
503.01	0.00567775295885528\\
504.01	0.00564516795050089\\
505.01	0.00561166826048236\\
506.01	0.00557723879844062\\
507.01	0.00554186725854646\\
508.01	0.0055055447350111\\
509.01	0.00546826638667768\\
510.01	0.00543003212621563\\
511.01	0.00539084729183767\\
512.01	0.00535072323266179\\
513.01	0.00530967769962473\\
514.01	0.00526773487704865\\
515.01	0.00522492480832201\\
516.01	0.00518128185258883\\
517.01	0.00513684164376462\\
518.01	0.0050916357889881\\
519.01	0.00504568026289477\\
520.01	0.00499895988309242\\
521.01	0.00495143501123638\\
522.01	0.00490303435970988\\
523.01	0.00485363928552798\\
524.01	0.00480306166516005\\
525.01	0.0047510147258005\\
526.01	0.004697213408747\\
527.01	0.00464152182251868\\
528.01	0.0045838662559028\\
529.01	0.00452422283921419\\
530.01	0.00446261138137171\\
531.01	0.00439911741448504\\
532.01	0.00433392363083164\\
533.01	0.00426778874389639\\
534.01	0.00420137881425066\\
535.01	0.00413485118336642\\
536.01	0.00406838381103923\\
537.01	0.00400217499259854\\
538.01	0.00393644160379946\\
539.01	0.0038714150870976\\
540.01	0.00380733405547405\\
541.01	0.00374443191699418\\
542.01	0.00368291726258625\\
543.01	0.00362294390816676\\
544.01	0.00356455659307963\\
545.01	0.00350696477086615\\
546.01	0.00344975170545223\\
547.01	0.00339298951158971\\
548.01	0.00333673215405864\\
549.01	0.00328100951661619\\
550.01	0.00322582173290015\\
551.01	0.00317113412249031\\
552.01	0.00311687052003248\\
553.01	0.00306290970465843\\
554.01	0.00300908991410397\\
555.01	0.00295523201111278\\
556.01	0.00290124563021432\\
557.01	0.002847098072094\\
558.01	0.00279274702632913\\
559.01	0.00273814046510795\\
560.01	0.00268321858787583\\
561.01	0.00262791619871552\\
562.01	0.00257216649574095\\
563.01	0.00251590691454799\\
564.01	0.00245908556494755\\
565.01	0.00240166502778602\\
566.01	0.00234361427178739\\
567.01	0.0022849016921835\\
568.01	0.0022254958877592\\
569.01	0.00216536696107141\\
570.01	0.0021044878379189\\
571.01	0.00204283533477184\\
572.01	0.00198039060692598\\
573.01	0.00191713873061543\\
574.01	0.0018530674700027\\
575.01	0.00178816633815506\\
576.01	0.00172242682014106\\
577.01	0.00165584284131613\\
578.01	0.001588411129313\\
579.01	0.00152013143583996\\
580.01	0.00145100662154961\\
581.01	0.00138104265774608\\
582.01	0.00131024865991341\\
583.01	0.0012386370874182\\
584.01	0.00116622409072038\\
585.01	0.00109302987562263\\
586.01	0.001019079034026\\
587.01	0.000944400819187679\\
588.01	0.000869029345008453\\
589.01	0.000793003687887234\\
590.01	0.000716367854795577\\
591.01	0.000639170551898716\\
592.01	0.000561464658762556\\
593.01	0.000483306311701305\\
594.01	0.000404753505687408\\
595.01	0.000325864119007519\\
596.01	0.000246693254416338\\
597.01	0.000167289772990669\\
598.01	9.17366978681194e-05\\
599.01	2.94669364335823e-05\\
599.02	2.89574269646022e-05\\
599.03	2.84509530909024e-05\\
599.04	2.79475443514862e-05\\
599.05	2.74472305761501e-05\\
599.06	2.69500418883586e-05\\
599.07	2.64560087081516e-05\\
599.08	2.59651617550757e-05\\
599.09	2.54775320511456e-05\\
599.1	2.49931509238403e-05\\
599.11	2.45120500091244e-05\\
599.12	2.40342612544929e-05\\
599.13	2.35598169220554e-05\\
599.14	2.30887495916553e-05\\
599.15	2.26210921640043e-05\\
599.16	2.21568778638551e-05\\
599.17	2.16961402432179e-05\\
599.18	2.12389131845833e-05\\
599.19	2.07852309042009e-05\\
599.2	2.03351279553787e-05\\
599.21	1.98886392318159e-05\\
599.22	1.94457999709644e-05\\
599.23	1.90066457574443e-05\\
599.24	1.85712125264602e-05\\
599.25	1.81395365672823e-05\\
599.26	1.77116564104504e-05\\
599.27	1.72876122143499e-05\\
599.28	1.68674445369076e-05\\
599.29	1.64511943395312e-05\\
599.3	1.60389029911102e-05\\
599.31	1.56306122720377e-05\\
599.32	1.52263643782673e-05\\
599.33	1.48262019254389e-05\\
599.34	1.44301679530119e-05\\
599.35	1.40383059284606e-05\\
599.36	1.36506597515022e-05\\
599.37	1.32672737583623e-05\\
599.38	1.28881927261015e-05\\
599.39	1.25134618769607e-05\\
599.4	1.21431268827639e-05\\
599.41	1.17772338693641e-05\\
599.42	1.14158294211263e-05\\
599.43	1.10589605854527e-05\\
599.44	1.0706674877372e-05\\
599.45	1.03590202841473e-05\\
599.46	1.00160452699508e-05\\
599.47	9.6777987805708e-06\\
599.48	9.34433024817223e-06\\
599.49	9.01568959610181e-06\\
599.5	8.6919272437435e-06\\
599.51	8.37309411141907e-06\\
599.52	8.05924162533209e-06\\
599.53	7.75042172257087e-06\\
599.54	7.44668685616172e-06\\
599.55	7.14809000015512e-06\\
599.56	6.85468465477443e-06\\
599.57	6.56652485162869e-06\\
599.58	6.28366515894284e-06\\
599.59	6.00616068687464e-06\\
599.6	5.73406709285587e-06\\
599.61	5.4674405870099e-06\\
599.62	5.20633793760911e-06\\
599.63	4.95081647658262e-06\\
599.64	4.70093410509e-06\\
599.65	4.45674929914694e-06\\
599.66	4.21832111529435e-06\\
599.67	3.9857091963455e-06\\
599.68	3.75897377716608e-06\\
599.69	3.53817569053415e-06\\
599.7	3.32337637303469e-06\\
599.71	3.11463787103054e-06\\
599.72	2.91202284668536e-06\\
599.73	2.71559458404555e-06\\
599.74	2.52541699518466e-06\\
599.75	2.34155462640155e-06\\
599.76	2.16407266449489e-06\\
599.77	1.99303694307755e-06\\
599.78	1.82851394897598e-06\\
599.79	1.67057082867302e-06\\
599.8	1.51927539483211e-06\\
599.81	1.37469613287027e-06\\
599.82	1.23690220760718e-06\\
599.83	1.10596346996998e-06\\
599.84	9.81950463782924e-07\\
599.85	8.64934432591793e-07\\
599.86	7.54987326588921e-07\\
599.87	6.5218180958504e-07\\
599.88	5.56591266064402e-07\\
599.89	4.68289808295413e-07\\
599.9	3.87352283521061e-07\\
599.91	3.13854281216996e-07\\
599.92	2.4787214042421e-07\\
599.93	1.8948295714763e-07\\
599.94	1.38764591834512e-07\\
599.95	9.57956769204876e-08\\
599.96	6.06556244606149e-08\\
599.97	3.34246338194039e-08\\
599.98	1.41836994527883e-08\\
599.99	3.01461875948372e-09\\
600	0\\
};
\addplot [color=red!50!mycolor17,solid,forget plot]
  table[row sep=crcr]{%
0.01	0.00656440582368625\\
1.01	0.00656440546540022\\
2.01	0.00656440509943144\\
3.01	0.00656440472561442\\
4.01	0.00656440434378001\\
5.01	0.00656440395375523\\
6.01	0.00656440355536398\\
7.01	0.00656440314842576\\
8.01	0.00656440273275647\\
9.01	0.00656440230816771\\
10.01	0.00656440187446719\\
11.01	0.00656440143145865\\
12.01	0.00656440097894109\\
13.01	0.00656440051670957\\
14.01	0.00656440004455453\\
15.01	0.00656439956226184\\
16.01	0.00656439906961291\\
17.01	0.0065643985663842\\
18.01	0.00656439805234737\\
19.01	0.00656439752726918\\
20.01	0.00656439699091115\\
21.01	0.00656439644302987\\
22.01	0.00656439588337666\\
23.01	0.00656439531169723\\
24.01	0.00656439472773186\\
25.01	0.00656439413121509\\
26.01	0.00656439352187586\\
27.01	0.00656439289943711\\
28.01	0.00656439226361582\\
29.01	0.00656439161412292\\
30.01	0.00656439095066264\\
31.01	0.00656439027293327\\
32.01	0.0065643895806263\\
33.01	0.00656438887342651\\
34.01	0.00656438815101185\\
35.01	0.00656438741305317\\
36.01	0.00656438665921446\\
37.01	0.00656438588915204\\
38.01	0.00656438510251502\\
39.01	0.00656438429894477\\
40.01	0.00656438347807471\\
41.01	0.00656438263953042\\
42.01	0.00656438178292943\\
43.01	0.00656438090788055\\
44.01	0.00656438001398459\\
45.01	0.0065643791008333\\
46.01	0.00656437816800967\\
47.01	0.00656437721508744\\
48.01	0.00656437624163129\\
49.01	0.00656437524719617\\
50.01	0.00656437423132769\\
51.01	0.0065643731935612\\
52.01	0.00656437213342191\\
53.01	0.00656437105042498\\
54.01	0.00656436994407461\\
55.01	0.00656436881386433\\
56.01	0.00656436765927691\\
57.01	0.00656436647978339\\
58.01	0.00656436527484341\\
59.01	0.00656436404390475\\
60.01	0.00656436278640348\\
61.01	0.0065643615017627\\
62.01	0.0065643601893935\\
63.01	0.0065643588486938\\
64.01	0.00656435747904851\\
65.01	0.00656435607982905\\
66.01	0.0065643546503928\\
67.01	0.00656435319008354\\
68.01	0.00656435169823045\\
69.01	0.00656435017414805\\
70.01	0.00656434861713593\\
71.01	0.00656434702647833\\
72.01	0.00656434540144379\\
73.01	0.00656434374128471\\
74.01	0.00656434204523739\\
75.01	0.00656434031252119\\
76.01	0.00656433854233837\\
77.01	0.00656433673387389\\
78.01	0.00656433488629476\\
79.01	0.00656433299874963\\
80.01	0.00656433107036859\\
81.01	0.00656432910026258\\
82.01	0.00656432708752323\\
83.01	0.00656432503122192\\
84.01	0.00656432293040998\\
85.01	0.00656432078411769\\
86.01	0.0065643185913541\\
87.01	0.00656431635110668\\
88.01	0.00656431406234046\\
89.01	0.00656431172399769\\
90.01	0.00656430933499771\\
91.01	0.0065643068942357\\
92.01	0.00656430440058294\\
93.01	0.00656430185288571\\
94.01	0.00656429924996476\\
95.01	0.00656429659061535\\
96.01	0.0065642938736059\\
97.01	0.00656429109767786\\
98.01	0.00656428826154488\\
99.01	0.00656428536389257\\
100.01	0.00656428240337719\\
101.01	0.00656427937862584\\
102.01	0.0065642762882352\\
103.01	0.00656427313077095\\
104.01	0.00656426990476732\\
105.01	0.00656426660872626\\
106.01	0.00656426324111655\\
107.01	0.00656425980037325\\
108.01	0.00656425628489711\\
109.01	0.00656425269305348\\
110.01	0.00656424902317158\\
111.01	0.00656424527354389\\
112.01	0.00656424144242516\\
113.01	0.00656423752803153\\
114.01	0.00656423352854006\\
115.01	0.00656422944208717\\
116.01	0.00656422526676833\\
117.01	0.00656422100063694\\
118.01	0.00656421664170345\\
119.01	0.00656421218793391\\
120.01	0.00656420763725004\\
121.01	0.00656420298752725\\
122.01	0.00656419823659398\\
123.01	0.00656419338223057\\
124.01	0.0065641884221685\\
125.01	0.00656418335408863\\
126.01	0.00656417817562074\\
127.01	0.0065641728843422\\
128.01	0.00656416747777677\\
129.01	0.00656416195339289\\
130.01	0.00656415630860342\\
131.01	0.00656415054076357\\
132.01	0.00656414464717013\\
133.01	0.00656413862505968\\
134.01	0.00656413247160771\\
135.01	0.00656412618392702\\
136.01	0.00656411975906615\\
137.01	0.00656411319400827\\
138.01	0.00656410648566962\\
139.01	0.00656409963089772\\
140.01	0.00656409262647038\\
141.01	0.00656408546909342\\
142.01	0.00656407815539972\\
143.01	0.00656407068194717\\
144.01	0.00656406304521711\\
145.01	0.00656405524161271\\
146.01	0.00656404726745693\\
147.01	0.00656403911899101\\
148.01	0.00656403079237245\\
149.01	0.00656402228367297\\
150.01	0.00656401358887737\\
151.01	0.00656400470388014\\
152.01	0.00656399562448489\\
153.01	0.00656398634640142\\
154.01	0.00656397686524381\\
155.01	0.00656396717652843\\
156.01	0.00656395727567134\\
157.01	0.00656394715798651\\
158.01	0.00656393681868305\\
159.01	0.00656392625286333\\
160.01	0.00656391545552006\\
161.01	0.0065639044215341\\
162.01	0.0065638931456716\\
163.01	0.00656388162258207\\
164.01	0.00656386984679521\\
165.01	0.00656385781271792\\
166.01	0.00656384551463225\\
167.01	0.00656383294669207\\
168.01	0.0065638201029203\\
169.01	0.00656380697720591\\
170.01	0.0065637935633009\\
171.01	0.00656377985481698\\
172.01	0.00656376584522262\\
173.01	0.00656375152784004\\
174.01	0.00656373689584109\\
175.01	0.00656372194224441\\
176.01	0.00656370665991189\\
177.01	0.00656369104154491\\
178.01	0.00656367507968084\\
179.01	0.00656365876668904\\
180.01	0.00656364209476747\\
181.01	0.00656362505593814\\
182.01	0.00656360764204395\\
183.01	0.00656358984474339\\
184.01	0.00656357165550749\\
185.01	0.00656355306561492\\
186.01	0.00656353406614752\\
187.01	0.00656351464798604\\
188.01	0.00656349480180533\\
189.01	0.00656347451806996\\
190.01	0.00656345378702875\\
191.01	0.00656343259871038\\
192.01	0.00656341094291802\\
193.01	0.00656338880922434\\
194.01	0.00656336618696599\\
195.01	0.00656334306523831\\
196.01	0.00656331943288979\\
197.01	0.00656329527851595\\
198.01	0.00656327059045398\\
199.01	0.00656324535677658\\
200.01	0.00656321956528616\\
201.01	0.00656319320350774\\
202.01	0.00656316625868347\\
203.01	0.00656313871776556\\
204.01	0.00656311056740936\\
205.01	0.00656308179396725\\
206.01	0.00656305238348084\\
207.01	0.00656302232167389\\
208.01	0.00656299159394528\\
209.01	0.00656296018536091\\
210.01	0.00656292808064631\\
211.01	0.00656289526417888\\
212.01	0.00656286171997907\\
213.01	0.0065628274317028\\
214.01	0.00656279238263234\\
215.01	0.0065627565556678\\
216.01	0.0065627199333185\\
217.01	0.00656268249769313\\
218.01	0.00656264423049121\\
219.01	0.00656260511299261\\
220.01	0.00656256512604795\\
221.01	0.00656252425006915\\
222.01	0.00656248246501819\\
223.01	0.00656243975039708\\
224.01	0.00656239608523671\\
225.01	0.00656235144808599\\
226.01	0.00656230581700023\\
227.01	0.00656225916952966\\
228.01	0.00656221148270753\\
229.01	0.00656216273303747\\
230.01	0.00656211289648114\\
231.01	0.00656206194844563\\
232.01	0.00656200986377009\\
233.01	0.00656195661671203\\
234.01	0.00656190218093363\\
235.01	0.00656184652948781\\
236.01	0.00656178963480315\\
237.01	0.00656173146866952\\
238.01	0.00656167200222242\\
239.01	0.00656161120592764\\
240.01	0.00656154904956486\\
241.01	0.00656148550221161\\
242.01	0.00656142053222612\\
243.01	0.00656135410723006\\
244.01	0.0065612861940909\\
245.01	0.00656121675890383\\
246.01	0.00656114576697296\\
247.01	0.00656107318279228\\
248.01	0.00656099897002596\\
249.01	0.00656092309148856\\
250.01	0.00656084550912443\\
251.01	0.00656076618398621\\
252.01	0.00656068507621333\\
253.01	0.00656060214501012\\
254.01	0.00656051734862315\\
255.01	0.00656043064431713\\
256.01	0.00656034198835127\\
257.01	0.00656025133595515\\
258.01	0.00656015864130328\\
259.01	0.0065600638574892\\
260.01	0.00655996693649861\\
261.01	0.00655986782918266\\
262.01	0.00655976648523\\
263.01	0.00655966285313819\\
264.01	0.00655955688018397\\
265.01	0.00655944851239334\\
266.01	0.00655933769451097\\
267.01	0.00655922436996773\\
268.01	0.00655910848084881\\
269.01	0.00655898996786009\\
270.01	0.00655886877029374\\
271.01	0.00655874482599301\\
272.01	0.00655861807131657\\
273.01	0.00655848844110089\\
274.01	0.00655835586862248\\
275.01	0.00655822028555866\\
276.01	0.00655808162194776\\
277.01	0.00655793980614784\\
278.01	0.00655779476479435\\
279.01	0.00655764642275691\\
280.01	0.00655749470309495\\
281.01	0.00655733952701165\\
282.01	0.00655718081380714\\
283.01	0.00655701848083072\\
284.01	0.00655685244343093\\
285.01	0.00655668261490464\\
286.01	0.00655650890644569\\
287.01	0.00655633122709056\\
288.01	0.00655614948366405\\
289.01	0.0065559635807221\\
290.01	0.00655577342049454\\
291.01	0.00655557890282469\\
292.01	0.00655537992510951\\
293.01	0.00655517638223547\\
294.01	0.00655496816651483\\
295.01	0.00655475516761933\\
296.01	0.00655453727251183\\
297.01	0.00655431436537697\\
298.01	0.0065540863275489\\
299.01	0.00655385303743793\\
300.01	0.00655361437045451\\
301.01	0.00655337019893227\\
302.01	0.00655312039204702\\
303.01	0.00655286481573576\\
304.01	0.00655260333261192\\
305.01	0.00655233580187901\\
306.01	0.00655206207924174\\
307.01	0.00655178201681469\\
308.01	0.00655149546302884\\
309.01	0.00655120226253515\\
310.01	0.00655090225610571\\
311.01	0.00655059528053253\\
312.01	0.0065502811685226\\
313.01	0.00654995974859157\\
314.01	0.00654963084495366\\
315.01	0.00654929427740841\\
316.01	0.0065489498612254\\
317.01	0.00654859740702488\\
318.01	0.00654823672065572\\
319.01	0.00654786760307014\\
320.01	0.00654748985019475\\
321.01	0.00654710325279886\\
322.01	0.0065467075963584\\
323.01	0.00654630266091722\\
324.01	0.00654588822094412\\
325.01	0.00654546404518628\\
326.01	0.00654502989651933\\
327.01	0.00654458553179271\\
328.01	0.00654413070167154\\
329.01	0.0065436651504752\\
330.01	0.00654318861601012\\
331.01	0.00654270082939996\\
332.01	0.0065422015149103\\
333.01	0.00654169038976984\\
334.01	0.00654116716398685\\
335.01	0.00654063154016085\\
336.01	0.00654008321329051\\
337.01	0.00653952187057521\\
338.01	0.00653894719121438\\
339.01	0.00653835884619936\\
340.01	0.00653775649810243\\
341.01	0.00653713980086034\\
342.01	0.00653650839955177\\
343.01	0.00653586193017142\\
344.01	0.0065352000193979\\
345.01	0.00653452228435681\\
346.01	0.00653382833237883\\
347.01	0.00653311776075192\\
348.01	0.00653239015646936\\
349.01	0.00653164509597112\\
350.01	0.00653088214488131\\
351.01	0.00653010085773923\\
352.01	0.00652930077772563\\
353.01	0.00652848143638347\\
354.01	0.00652764235333345\\
355.01	0.0065267830359844\\
356.01	0.00652590297923764\\
357.01	0.00652500166518708\\
358.01	0.00652407856281347\\
359.01	0.00652313312767335\\
360.01	0.00652216480158361\\
361.01	0.00652117301230029\\
362.01	0.00652015717319252\\
363.01	0.00651911668291189\\
364.01	0.00651805092505683\\
365.01	0.0065169592678316\\
366.01	0.00651584106370219\\
367.01	0.0065146956490462\\
368.01	0.00651352234379845\\
369.01	0.00651232045109326\\
370.01	0.00651108925690145\\
371.01	0.00650982802966368\\
372.01	0.00650853601991949\\
373.01	0.00650721245993259\\
374.01	0.00650585656331152\\
375.01	0.0065044675246271\\
376.01	0.00650304451902536\\
377.01	0.00650158670183648\\
378.01	0.00650009320817941\\
379.01	0.00649856315256223\\
380.01	0.00649699562847678\\
381.01	0.00649538970798896\\
382.01	0.0064937444413224\\
383.01	0.00649205885643525\\
384.01	0.00649033195858945\\
385.01	0.00648856272991013\\
386.01	0.00648675012893443\\
387.01	0.00648489309014677\\
388.01	0.00648299052349833\\
389.01	0.00648104131390808\\
390.01	0.00647904432073992\\
391.01	0.00647699837725221\\
392.01	0.00647490229001441\\
393.01	0.00647275483828189\\
394.01	0.00647055477332394\\
395.01	0.00646830081769172\\
396.01	0.00646599166441805\\
397.01	0.00646362597613286\\
398.01	0.00646120238408305\\
399.01	0.00645871948703488\\
400.01	0.00645617585004426\\
401.01	0.00645357000307023\\
402.01	0.00645090043941078\\
403.01	0.00644816561393521\\
404.01	0.00644536394109067\\
405.01	0.00644249379265812\\
406.01	0.00643955349523729\\
407.01	0.00643654132744707\\
408.01	0.00643345551683574\\
409.01	0.00643029423650974\\
410.01	0.00642705560150948\\
411.01	0.00642373766498846\\
412.01	0.00642033841428345\\
413.01	0.00641685576700698\\
414.01	0.00641328756733783\\
415.01	0.00640963158272969\\
416.01	0.00640588550129285\\
417.01	0.00640204693010357\\
418.01	0.00639811339462549\\
419.01	0.00639408233923406\\
420.01	0.00638995112838307\\
421.01	0.00638571704692442\\
422.01	0.00638137729839664\\
423.01	0.00637692900261399\\
424.01	0.00637236919322113\\
425.01	0.00636769481525123\\
426.01	0.00636290272270502\\
427.01	0.00635798967617175\\
428.01	0.00635295234051363\\
429.01	0.00634778728264398\\
430.01	0.00634249096942608\\
431.01	0.006337059765727\\
432.01	0.00633148993266037\\
433.01	0.00632577762605561\\
434.01	0.00631991889518877\\
435.01	0.00631390968181004\\
436.01	0.00630774581949997\\
437.01	0.00630142303337773\\
438.01	0.00629493694017861\\
439.01	0.00628828304870329\\
440.01	0.00628145676063154\\
441.01	0.00627445337167707\\
442.01	0.00626726807305398\\
443.01	0.00625989595323021\\
444.01	0.00625233199997218\\
445.01	0.0062445711027621\\
446.01	0.00623660805584039\\
447.01	0.00622843756253448\\
448.01	0.00622005424148167\\
449.01	0.00621145263441838\\
450.01	0.00620262721545311\\
451.01	0.00619357240196595\\
452.01	0.00618428256728609\\
453.01	0.00617475205529724\\
454.01	0.00616497519710713\\
455.01	0.00615494632990128\\
456.01	0.00614465981806608\\
457.01	0.00613411007662122\\
458.01	0.00612329159692982\\
459.01	0.00611219897456385\\
460.01	0.00610082693907493\\
461.01	0.00608917038525529\\
462.01	0.00607722440526771\\
463.01	0.00606498432075847\\
464.01	0.00605244571375353\\
465.01	0.00603960445476142\\
466.01	0.00602645672608229\\
467.01	0.00601299903786675\\
468.01	0.00599922823402013\\
469.01	0.00598514148467885\\
470.01	0.00597073626182786\\
471.01	0.00595601029487695\\
472.01	0.00594096150400683\\
473.01	0.0059255879113317\\
474.01	0.00590988753420292\\
475.01	0.00589385827255953\\
476.01	0.00587749783778725\\
477.01	0.00586080373908369\\
478.01	0.00584377299093302\\
479.01	0.00582640150876262\\
480.01	0.0058086831876211\\
481.01	0.00579060853200369\\
482.01	0.00577216432972964\\
483.01	0.00575334582220765\\
484.01	0.00573415424939939\\
485.01	0.00571459079020263\\
486.01	0.00569465572021814\\
487.01	0.00567434718468358\\
488.01	0.00565365936006162\\
489.01	0.00563257975556609\\
490.01	0.00561108531160021\\
491.01	0.00558913682306404\\
492.01	0.00556667103921158\\
493.01	0.00554358997176287\\
494.01	0.00551979880560138\\
495.01	0.00549527325324063\\
496.01	0.00546999979739121\\
497.01	0.00544396646201648\\
498.01	0.00541716238647811\\
499.01	0.00538957684242895\\
500.01	0.00536119737711359\\
501.01	0.00533200587711989\\
502.01	0.00530196903533774\\
503.01	0.00527104449029588\\
504.01	0.0052391864232501\\
505.01	0.00520634535208319\\
506.01	0.00517246793809063\\
507.01	0.00513749692894901\\
508.01	0.00510137127455286\\
509.01	0.00506402650617743\\
510.01	0.00502539551200544\\
511.01	0.00498540989747207\\
512.01	0.0049440021960149\\
513.01	0.0049011093030496\\
514.01	0.00485667765489779\\
515.01	0.0048106708810639\\
516.01	0.00476308094499622\\
517.01	0.00471394418628692\\
518.01	0.00466336422930108\\
519.01	0.00461186793412728\\
520.01	0.0045599046924575\\
521.01	0.00450754211297316\\
522.01	0.00445485964823526\\
523.01	0.00440195105852912\\
524.01	0.00434892780644088\\
525.01	0.00429592382681327\\
526.01	0.00424309974104571\\
527.01	0.00419063659086737\\
528.01	0.00413872705740767\\
529.01	0.0040875654696395\\
530.01	0.00403733280501714\\
531.01	0.00398817333943129\\
532.01	0.00394015889021307\\
533.01	0.0038927890665121\\
534.01	0.0038456196551268\\
535.01	0.00379871775713688\\
536.01	0.00375214226384924\\
537.01	0.00370593984677847\\
538.01	0.00366014038052297\\
539.01	0.00361475198846775\\
540.01	0.0035697561020783\\
541.01	0.00352510323923424\\
542.01	0.00348071068926054\\
543.01	0.00343646401783639\\
544.01	0.00339222542361496\\
545.01	0.00334788219760175\\
546.01	0.00330340572108235\\
547.01	0.00325877291055184\\
548.01	0.00321395096245914\\
549.01	0.00316889777513852\\
550.01	0.0031235631575099\\
551.01	0.00307789096322181\\
552.01	0.00303182227862506\\
553.01	0.00298529967226709\\
554.01	0.00293827203692452\\
555.01	0.00289069883880613\\
556.01	0.00284254826966974\\
557.01	0.00279378882254609\\
558.01	0.0027443879232907\\
559.01	0.00269431286133482\\
560.01	0.00264353176750194\\
561.01	0.0025920145436866\\
562.01	0.00253973362071961\\
563.01	0.00248666432825441\\
564.01	0.00243278467556005\\
565.01	0.00237807450916724\\
566.01	0.00232251470725555\\
567.01	0.00226608722974788\\
568.01	0.00220877542917444\\
569.01	0.00215056429547898\\
570.01	0.0020914406062683\\
571.01	0.00203139297475291\\
572.01	0.00197041181888403\\
573.01	0.00190848931254016\\
574.01	0.00184561940995461\\
575.01	0.00178179800211486\\
576.01	0.00171702313729112\\
577.01	0.00165129523694475\\
578.01	0.00158461730238309\\
579.01	0.00151699512093703\\
580.01	0.00144843748360971\\
581.01	0.00137895642524441\\
582.01	0.00130856749049479\\
583.01	0.00123729001347867\\
584.01	0.00116514738699767\\
585.01	0.00109216730167909\\
586.01	0.00101838193976332\\
587.01	0.000943828105516345\\
588.01	0.000868547269223433\\
589.01	0.000792585494808212\\
590.01	0.000715993212731169\\
591.01	0.000638824791183846\\
592.01	0.000561137850788497\\
593.01	0.000482992259190443\\
594.01	0.000404448728692637\\
595.01	0.000325566921518283\\
596.01	0.000246402942558918\\
597.01	0.000167006059591517\\
598.01	9.17366970263709e-05\\
599.01	2.94669364272089e-05\\
599.02	2.89574269586573e-05\\
599.03	2.84509530853617e-05\\
599.04	2.79475443463271e-05\\
599.05	2.74472305713484e-05\\
599.06	2.69500418838952e-05\\
599.07	2.64560087040022e-05\\
599.08	2.59651617512246e-05\\
599.09	2.54775320475721e-05\\
599.1	2.49931509205322e-05\\
599.11	2.45120500060592e-05\\
599.12	2.40342612516566e-05\\
599.13	2.35598169194343e-05\\
599.14	2.30887495892354e-05\\
599.15	2.262109216177e-05\\
599.16	2.21568778617977e-05\\
599.17	2.16961402413236e-05\\
599.18	2.12389131828417e-05\\
599.19	2.07852309026015e-05\\
599.2	2.03351279539129e-05\\
599.21	1.98886392304715e-05\\
599.22	1.94457999697362e-05\\
599.23	1.90066457563202e-05\\
599.24	1.85712125254332e-05\\
599.25	1.81395365663438e-05\\
599.26	1.77116564095969e-05\\
599.27	1.72876122135745e-05\\
599.28	1.68674445362033e-05\\
599.29	1.64511943388929e-05\\
599.3	1.60389029905342e-05\\
599.31	1.56306122715139e-05\\
599.32	1.52263643777972e-05\\
599.33	1.48262019250139e-05\\
599.34	1.4430167952632e-05\\
599.35	1.40383059281188e-05\\
599.36	1.36506597511951e-05\\
599.37	1.32672737580882e-05\\
599.38	1.28881927258552e-05\\
599.39	1.25134618767422e-05\\
599.4	1.21431268825713e-05\\
599.41	1.17772338691941e-05\\
599.42	1.14158294209736e-05\\
599.43	1.10589605853192e-05\\
599.44	1.0706674877254e-05\\
599.45	1.03590202840433e-05\\
599.46	1.00160452698606e-05\\
599.47	9.67779878049101e-06\\
599.48	9.34433024810284e-06\\
599.49	9.01568959604283e-06\\
599.5	8.69192724369319e-06\\
599.51	8.37309411137396e-06\\
599.52	8.05924162529219e-06\\
599.53	7.75042172253965e-06\\
599.54	7.4466868561357e-06\\
599.55	7.14809000013084e-06\\
599.56	6.85468465475708e-06\\
599.57	6.56652485161308e-06\\
599.58	6.2836651589307e-06\\
599.59	6.00616068686249e-06\\
599.6	5.73406709284546e-06\\
599.61	5.46744058700296e-06\\
599.62	5.20633793760217e-06\\
599.63	4.95081647657741e-06\\
599.64	4.70093410508653e-06\\
599.65	4.45674929914174e-06\\
599.66	4.21832111529089e-06\\
599.67	3.98570919634203e-06\\
599.68	3.75897377716608e-06\\
599.69	3.53817569053415e-06\\
599.7	3.32337637303469e-06\\
599.71	3.11463787102881e-06\\
599.72	2.91202284668363e-06\\
599.73	2.71559458404382e-06\\
599.74	2.52541699518292e-06\\
599.75	2.34155462640155e-06\\
599.76	2.16407266449663e-06\\
599.77	1.99303694307928e-06\\
599.78	1.82851394897598e-06\\
599.79	1.67057082867302e-06\\
599.8	1.51927539483211e-06\\
599.81	1.37469613287027e-06\\
599.82	1.23690220760544e-06\\
599.83	1.10596346997172e-06\\
599.84	9.81950463782924e-07\\
599.85	8.64934432591793e-07\\
599.86	7.54987326587186e-07\\
599.87	6.5218180958504e-07\\
599.88	5.56591266062667e-07\\
599.89	4.68289808295413e-07\\
599.9	3.87352283521061e-07\\
599.91	3.13854281218731e-07\\
599.92	2.47872140425945e-07\\
599.93	1.89482957149364e-07\\
599.94	1.38764591834512e-07\\
599.95	9.57956769204876e-08\\
599.96	6.06556244606149e-08\\
599.97	3.34246338211386e-08\\
599.98	1.4183699454523e-08\\
599.99	3.014618757749e-09\\
600	0\\
};
\addplot [color=red!40!mycolor19,solid,forget plot]
  table[row sep=crcr]{%
0.01	0.00627501982079631\\
1.01	0.00627501944692668\\
2.01	0.00627501906508025\\
3.01	0.00627501867508635\\
4.01	0.00627501827677046\\
5.01	0.00627501786995453\\
6.01	0.00627501745445643\\
7.01	0.00627501703009011\\
8.01	0.00627501659666559\\
9.01	0.00627501615398896\\
10.01	0.00627501570186209\\
11.01	0.00627501524008224\\
12.01	0.00627501476844279\\
13.01	0.00627501428673242\\
14.01	0.0062750137947353\\
15.01	0.00627501329223091\\
16.01	0.00627501277899408\\
17.01	0.00627501225479479\\
18.01	0.00627501171939803\\
19.01	0.00627501117256362\\
20.01	0.00627501061404649\\
21.01	0.00627501004359601\\
22.01	0.00627500946095607\\
23.01	0.00627500886586557\\
24.01	0.00627500825805735\\
25.01	0.00627500763725846\\
26.01	0.00627500700319044\\
27.01	0.00627500635556846\\
28.01	0.00627500569410162\\
29.01	0.00627500501849288\\
30.01	0.00627500432843883\\
31.01	0.00627500362362923\\
32.01	0.00627500290374755\\
33.01	0.00627500216847012\\
34.01	0.00627500141746654\\
35.01	0.00627500065039911\\
36.01	0.00627499986692271\\
37.01	0.00627499906668511\\
38.01	0.00627499824932634\\
39.01	0.00627499741447863\\
40.01	0.00627499656176637\\
41.01	0.00627499569080558\\
42.01	0.0062749948012044\\
43.01	0.00627499389256226\\
44.01	0.00627499296446983\\
45.01	0.00627499201650924\\
46.01	0.00627499104825335\\
47.01	0.00627499005926611\\
48.01	0.00627498904910168\\
49.01	0.00627498801730491\\
50.01	0.00627498696341052\\
51.01	0.00627498588694327\\
52.01	0.0062749847874179\\
53.01	0.00627498366433837\\
54.01	0.00627498251719804\\
55.01	0.00627498134547947\\
56.01	0.00627498014865353\\
57.01	0.00627497892618032\\
58.01	0.00627497767750798\\
59.01	0.00627497640207243\\
60.01	0.00627497509929776\\
61.01	0.00627497376859569\\
62.01	0.00627497240936482\\
63.01	0.00627497102099114\\
64.01	0.00627496960284693\\
65.01	0.00627496815429119\\
66.01	0.00627496667466925\\
67.01	0.00627496516331188\\
68.01	0.00627496361953534\\
69.01	0.00627496204264142\\
70.01	0.00627496043191656\\
71.01	0.00627495878663203\\
72.01	0.00627495710604279\\
73.01	0.00627495538938831\\
74.01	0.00627495363589117\\
75.01	0.00627495184475727\\
76.01	0.00627495001517523\\
77.01	0.00627494814631603\\
78.01	0.00627494623733275\\
79.01	0.00627494428736043\\
80.01	0.00627494229551492\\
81.01	0.00627494026089314\\
82.01	0.00627493818257228\\
83.01	0.00627493605960982\\
84.01	0.00627493389104248\\
85.01	0.0062749316758862\\
86.01	0.00627492941313573\\
87.01	0.00627492710176376\\
88.01	0.00627492474072093\\
89.01	0.00627492232893507\\
90.01	0.00627491986531064\\
91.01	0.00627491734872842\\
92.01	0.00627491477804482\\
93.01	0.00627491215209159\\
94.01	0.006274909469675\\
95.01	0.00627490672957539\\
96.01	0.00627490393054669\\
97.01	0.00627490107131582\\
98.01	0.00627489815058196\\
99.01	0.00627489516701594\\
100.01	0.00627489211926003\\
101.01	0.00627488900592669\\
102.01	0.00627488582559843\\
103.01	0.00627488257682694\\
104.01	0.00627487925813229\\
105.01	0.00627487586800252\\
106.01	0.00627487240489285\\
107.01	0.00627486886722501\\
108.01	0.00627486525338602\\
109.01	0.00627486156172819\\
110.01	0.00627485779056785\\
111.01	0.00627485393818469\\
112.01	0.00627485000282107\\
113.01	0.00627484598268125\\
114.01	0.00627484187593009\\
115.01	0.00627483768069272\\
116.01	0.00627483339505341\\
117.01	0.00627482901705461\\
118.01	0.00627482454469642\\
119.01	0.00627481997593549\\
120.01	0.00627481530868351\\
121.01	0.00627481054080715\\
122.01	0.00627480567012622\\
123.01	0.00627480069441342\\
124.01	0.00627479561139276\\
125.01	0.00627479041873867\\
126.01	0.00627478511407482\\
127.01	0.00627477969497318\\
128.01	0.00627477415895298\\
129.01	0.00627476850347911\\
130.01	0.00627476272596112\\
131.01	0.00627475682375241\\
132.01	0.00627475079414839\\
133.01	0.00627474463438554\\
134.01	0.0062747383416401\\
135.01	0.00627473191302679\\
136.01	0.00627472534559735\\
137.01	0.00627471863633906\\
138.01	0.00627471178217341\\
139.01	0.00627470477995492\\
140.01	0.00627469762646923\\
141.01	0.00627469031843194\\
142.01	0.00627468285248695\\
143.01	0.00627467522520483\\
144.01	0.00627466743308101\\
145.01	0.00627465947253462\\
146.01	0.0062746513399064\\
147.01	0.00627464303145715\\
148.01	0.00627463454336603\\
149.01	0.00627462587172868\\
150.01	0.00627461701255539\\
151.01	0.00627460796176902\\
152.01	0.00627459871520338\\
153.01	0.00627458926860131\\
154.01	0.00627457961761231\\
155.01	0.00627456975779074\\
156.01	0.00627455968459374\\
157.01	0.00627454939337881\\
158.01	0.00627453887940217\\
159.01	0.0062745281378158\\
160.01	0.00627451716366581\\
161.01	0.0062745059518896\\
162.01	0.00627449449731388\\
163.01	0.00627448279465192\\
164.01	0.00627447083850095\\
165.01	0.00627445862334025\\
166.01	0.00627444614352765\\
167.01	0.0062744333932974\\
168.01	0.00627442036675739\\
169.01	0.00627440705788619\\
170.01	0.00627439346053031\\
171.01	0.00627437956840123\\
172.01	0.00627436537507274\\
173.01	0.0062743508739768\\
174.01	0.00627433605840198\\
175.01	0.00627432092148916\\
176.01	0.00627430545622857\\
177.01	0.00627428965545641\\
178.01	0.00627427351185153\\
179.01	0.00627425701793191\\
180.01	0.00627424016605103\\
181.01	0.00627422294839444\\
182.01	0.00627420535697528\\
183.01	0.00627418738363186\\
184.01	0.00627416902002222\\
185.01	0.00627415025762123\\
186.01	0.00627413108771593\\
187.01	0.00627411150140158\\
188.01	0.00627409148957742\\
189.01	0.00627407104294207\\
190.01	0.00627405015198938\\
191.01	0.00627402880700365\\
192.01	0.00627400699805497\\
193.01	0.00627398471499471\\
194.01	0.00627396194745011\\
195.01	0.00627393868481967\\
196.01	0.00627391491626796\\
197.01	0.00627389063072046\\
198.01	0.00627386581685818\\
199.01	0.00627384046311181\\
200.01	0.00627381455765685\\
201.01	0.00627378808840736\\
202.01	0.00627376104301\\
203.01	0.0062737334088386\\
204.01	0.00627370517298764\\
205.01	0.00627367632226584\\
206.01	0.00627364684319003\\
207.01	0.00627361672197865\\
208.01	0.00627358594454488\\
209.01	0.00627355449648979\\
210.01	0.00627352236309501\\
211.01	0.00627348952931607\\
212.01	0.00627345597977495\\
213.01	0.00627342169875202\\
214.01	0.00627338667017911\\
215.01	0.00627335087763107\\
216.01	0.00627331430431797\\
217.01	0.00627327693307661\\
218.01	0.0062732387463622\\
219.01	0.00627319972623976\\
220.01	0.00627315985437529\\
221.01	0.00627311911202647\\
222.01	0.00627307748003373\\
223.01	0.00627303493881053\\
224.01	0.0062729914683338\\
225.01	0.00627294704813376\\
226.01	0.0062729016572838\\
227.01	0.00627285527439051\\
228.01	0.00627280787758198\\
229.01	0.00627275944449813\\
230.01	0.00627270995227875\\
231.01	0.00627265937755205\\
232.01	0.00627260769642319\\
233.01	0.00627255488446222\\
234.01	0.00627250091669176\\
235.01	0.00627244576757419\\
236.01	0.00627238941099914\\
237.01	0.00627233182026998\\
238.01	0.00627227296809058\\
239.01	0.00627221282655082\\
240.01	0.00627215136711321\\
241.01	0.00627208856059781\\
242.01	0.00627202437716745\\
243.01	0.00627195878631262\\
244.01	0.00627189175683558\\
245.01	0.00627182325683449\\
246.01	0.00627175325368689\\
247.01	0.00627168171403308\\
248.01	0.00627160860375854\\
249.01	0.0062715338879766\\
250.01	0.00627145753100993\\
251.01	0.00627137949637256\\
252.01	0.00627129974675016\\
253.01	0.00627121824398126\\
254.01	0.00627113494903623\\
255.01	0.00627104982199781\\
256.01	0.00627096282203976\\
257.01	0.00627087390740491\\
258.01	0.00627078303538332\\
259.01	0.00627069016228966\\
260.01	0.00627059524344006\\
261.01	0.00627049823312814\\
262.01	0.00627039908460019\\
263.01	0.00627029775003053\\
264.01	0.00627019418049573\\
265.01	0.00627008832594792\\
266.01	0.00626998013518788\\
267.01	0.0062698695558371\\
268.01	0.00626975653430933\\
269.01	0.00626964101578122\\
270.01	0.00626952294416235\\
271.01	0.00626940226206434\\
272.01	0.00626927891076863\\
273.01	0.00626915283019496\\
274.01	0.00626902395886698\\
275.01	0.00626889223387859\\
276.01	0.00626875759085857\\
277.01	0.00626861996393404\\
278.01	0.00626847928569379\\
279.01	0.00626833548714984\\
280.01	0.00626818849769801\\
281.01	0.00626803824507829\\
282.01	0.00626788465533277\\
283.01	0.00626772765276327\\
284.01	0.00626756715988795\\
285.01	0.00626740309739626\\
286.01	0.00626723538410234\\
287.01	0.00626706393689815\\
288.01	0.00626688867070387\\
289.01	0.00626670949841873\\
290.01	0.00626652633086838\\
291.01	0.00626633907675284\\
292.01	0.0062661476425907\\
293.01	0.00626595193266438\\
294.01	0.00626575184896115\\
295.01	0.00626554729111439\\
296.01	0.00626533815634235\\
297.01	0.00626512433938498\\
298.01	0.00626490573243966\\
299.01	0.00626468222509425\\
300.01	0.00626445370425851\\
301.01	0.00626422005409342\\
302.01	0.0062639811559389\\
303.01	0.00626373688823883\\
304.01	0.00626348712646358\\
305.01	0.00626323174303124\\
306.01	0.00626297060722554\\
307.01	0.00626270358511182\\
308.01	0.00626243053945027\\
309.01	0.00626215132960651\\
310.01	0.00626186581146005\\
311.01	0.0062615738373084\\
312.01	0.0062612752557708\\
313.01	0.00626096991168605\\
314.01	0.00626065764600972\\
315.01	0.00626033829570683\\
316.01	0.0062600116936413\\
317.01	0.00625967766846247\\
318.01	0.00625933604448791\\
319.01	0.00625898664158239\\
320.01	0.00625862927503338\\
321.01	0.00625826375542209\\
322.01	0.0062578898884914\\
323.01	0.00625750747500859\\
324.01	0.00625711631062481\\
325.01	0.00625671618572884\\
326.01	0.00625630688529746\\
327.01	0.00625588818874049\\
328.01	0.00625545986974115\\
329.01	0.00625502169609042\\
330.01	0.00625457342951782\\
331.01	0.00625411482551534\\
332.01	0.00625364563315641\\
333.01	0.00625316559490887\\
334.01	0.00625267444644229\\
335.01	0.00625217191642857\\
336.01	0.00625165772633658\\
337.01	0.0062511315902205\\
338.01	0.00625059321450026\\
339.01	0.00625004229773692\\
340.01	0.00624947853039881\\
341.01	0.00624890159462121\\
342.01	0.00624831116395897\\
343.01	0.00624770690313027\\
344.01	0.00624708846775251\\
345.01	0.00624645550407053\\
346.01	0.00624580764867563\\
347.01	0.00624514452821613\\
348.01	0.00624446575909892\\
349.01	0.00624377094718198\\
350.01	0.00624305968745715\\
351.01	0.00624233156372373\\
352.01	0.00624158614825195\\
353.01	0.00624082300143695\\
354.01	0.00624004167144166\\
355.01	0.00623924169383069\\
356.01	0.00623842259119347\\
357.01	0.00623758387275602\\
358.01	0.00623672503398323\\
359.01	0.00623584555617051\\
360.01	0.00623494490602452\\
361.01	0.006234022535234\\
362.01	0.00623307788003002\\
363.01	0.00623211036073653\\
364.01	0.00623111938131073\\
365.01	0.00623010432887544\\
366.01	0.00622906457324128\\
367.01	0.00622799946642182\\
368.01	0.00622690834214188\\
369.01	0.00622579051533872\\
370.01	0.0062246452816595\\
371.01	0.00622347191695422\\
372.01	0.00622226967676832\\
373.01	0.00622103779583443\\
374.01	0.00621977548756801\\
375.01	0.00621848194356779\\
376.01	0.00621715633312541\\
377.01	0.00621579780274746\\
378.01	0.00621440547569387\\
379.01	0.00621297845153867\\
380.01	0.00621151580575779\\
381.01	0.00621001658934978\\
382.01	0.00620847982849738\\
383.01	0.00620690452427863\\
384.01	0.00620528965243403\\
385.01	0.00620363416320286\\
386.01	0.00620193698123763\\
387.01	0.00620019700560957\\
388.01	0.00619841310991871\\
389.01	0.00619658414252233\\
390.01	0.00619470892689802\\
391.01	0.00619278626215702\\
392.01	0.00619081492372418\\
393.01	0.00618879366420226\\
394.01	0.00618672121443493\\
395.01	0.00618459628478509\\
396.01	0.00618241756663875\\
397.01	0.00618018373414341\\
398.01	0.00617789344617994\\
399.01	0.00617554534856301\\
400.01	0.00617313807644601\\
401.01	0.00617067025689468\\
402.01	0.00616814051156823\\
403.01	0.00616554745941866\\
404.01	0.00616288971928154\\
405.01	0.00616016591218665\\
406.01	0.00615737466316107\\
407.01	0.00615451460222969\\
408.01	0.00615158436424575\\
409.01	0.00614858258710004\\
410.01	0.00614550790777582\\
411.01	0.00614235895564246\\
412.01	0.00613913434234199\\
413.01	0.00613583264763858\\
414.01	0.00613245240073966\\
415.01	0.00612899205692615\\
416.01	0.00612544996998135\\
417.01	0.00612182436207277\\
418.01	0.00611811329471075\\
419.01	0.00611431464759695\\
420.01	0.00611042611998717\\
421.01	0.00610644528345852\\
422.01	0.00610236962267547\\
423.01	0.00609819653509054\\
424.01	0.00609392332698215\\
425.01	0.00608954720927813\\
426.01	0.00608506529315244\\
427.01	0.00608047458538285\\
428.01	0.00607577198345746\\
429.01	0.00607095427041315\\
430.01	0.00606601810939559\\
431.01	0.0060609600379238\\
432.01	0.0060557764618456\\
433.01	0.00605046364896941\\
434.01	0.00604501772236019\\
435.01	0.00603943465328612\\
436.01	0.00603371025380779\\
437.01	0.00602784016900473\\
438.01	0.00602181986883974\\
439.01	0.00601564463967026\\
440.01	0.00600930957542628\\
441.01	0.00600280956849132\\
442.01	0.00599613930034045\\
443.01	0.00598929323201499\\
444.01	0.0059822655945433\\
445.01	0.0059750503794517\\
446.01	0.00596764132954422\\
447.01	0.0059600319301649\\
448.01	0.00595221540118198\\
449.01	0.00594418468999795\\
450.01	0.00593593246598146\\
451.01	0.00592745111681121\\
452.01	0.00591873274732945\\
453.01	0.00590976918162688\\
454.01	0.00590055196922702\\
455.01	0.00589107239639559\\
456.01	0.00588132150378528\\
457.01	0.00587129011181471\\
458.01	0.00586096885537762\\
459.01	0.00585034822966616\\
460.01	0.00583941864904041\\
461.01	0.00582817052095897\\
462.01	0.00581659433692928\\
463.01	0.00580468078216617\\
464.01	0.00579242086502429\\
465.01	0.00577980606610628\\
466.01	0.0057668285049653\\
467.01	0.00575348111911687\\
468.01	0.0057397578450755\\
469.01	0.00572565378352102\\
470.01	0.00571116531930866\\
471.01	0.00569629015022579\\
472.01	0.00568102715383578\\
473.01	0.00566537598617683\\
474.01	0.00564933625488107\\
475.01	0.00563290599850248\\
476.01	0.00561607703700144\\
477.01	0.00559883378959261\\
478.01	0.00558115863084415\\
479.01	0.00556303221632345\\
480.01	0.00554443329402765\\
481.01	0.00552533853042991\\
482.01	0.00550572236828859\\
483.01	0.00548555671834599\\
484.01	0.00546481037634793\\
485.01	0.0054434485964233\\
486.01	0.00542143271600303\\
487.01	0.00539871981429891\\
488.01	0.00537526245314433\\
489.01	0.0053510085754977\\
490.01	0.00532590167516816\\
491.01	0.00529988140633078\\
492.01	0.00527288488008312\\
493.01	0.00524484900592452\\
494.01	0.00521571354443454\\
495.01	0.00518542144114009\\
496.01	0.00515392022106027\\
497.01	0.00512116654246566\\
498.01	0.00508713295858186\\
499.01	0.00505181766150161\\
500.01	0.00501525838772721\\
501.01	0.00497766164855981\\
502.01	0.00493943528264369\\
503.01	0.00490062690988339\\
504.01	0.00486126761608939\\
505.01	0.00482139625878809\\
506.01	0.00478106053220624\\
507.01	0.00474031806455034\\
508.01	0.00469923749003392\\
509.01	0.00465789940415897\\
510.01	0.00461639706089578\\
511.01	0.00457483659870346\\
512.01	0.00453333647999093\\
513.01	0.0044920256832665\\
514.01	0.00445103998170064\\
515.01	0.00441051535217822\\
516.01	0.00437057715191148\\
517.01	0.00433132312909362\\
518.01	0.00429279753583172\\
519.01	0.00425462094943984\\
520.01	0.00421646388263337\\
521.01	0.00417838022721257\\
522.01	0.004140422850745\\
523.01	0.00410264178097458\\
524.01	0.00406508188635623\\
525.01	0.00402777995207816\\
526.01	0.00399076106794841\\
527.01	0.00395403462590486\\
528.01	0.00391759050822663\\
529.01	0.00388139584061086\\
530.01	0.00384539310379418\\
531.01	0.00380950096592413\\
532.01	0.00377361999606866\\
533.01	0.00373766221781281\\
534.01	0.00370160676146456\\
535.01	0.00366544339044638\\
536.01	0.00362915467612006\\
537.01	0.00359271576305016\\
538.01	0.00355609454479962\\
539.01	0.00351925238193505\\
540.01	0.00348214548655068\\
541.01	0.00344472705759396\\
542.01	0.00340695015404107\\
543.01	0.0033687711001213\\
544.01	0.0033301528691409\\
545.01	0.0032910665487296\\
546.01	0.00325148580905002\\
547.01	0.00321138269631842\\
548.01	0.0031707279562313\\
549.01	0.00312949174391565\\
550.01	0.00308764439343728\\
551.01	0.00304515716922659\\
552.01	0.00300200289208221\\
553.01	0.00295815630627339\\
554.01	0.00291359405328688\\
555.01	0.00286829417430089\\
556.01	0.00282223533915401\\
557.01	0.0027753965662475\\
558.01	0.00272775741296249\\
559.01	0.00267929817756782\\
560.01	0.00263000004395317\\
561.01	0.00257984515678967\\
562.01	0.00252881662416956\\
563.01	0.00247689846218247\\
564.01	0.00242407552049739\\
565.01	0.00237033344952677\\
566.01	0.00231565875634426\\
567.01	0.00226003891137813\\
568.01	0.0022034624518856\\
569.01	0.0021459190761402\\
570.01	0.0020873997344778\\
571.01	0.00202789672639428\\
572.01	0.00196740381416456\\
573.01	0.00190591636143106\\
574.01	0.00184343149881933\\
575.01	0.00177994830972369\\
576.01	0.00171546802767963\\
577.01	0.00164999424345148\\
578.01	0.00158353312275398\\
579.01	0.00151609363483733\\
580.01	0.00144768779049829\\
581.01	0.0013783308857227\\
582.01	0.00130804174453238\\
583.01	0.00123684295248754\\
584.01	0.00116476107120689\\
585.01	0.00109182682306618\\
586.01	0.00101807523268008\\
587.01	0.000943545708142529\\
588.01	0.000868282040487625\\
589.01	0.000792332294341591\\
590.01	0.000715748556112592\\
591.01	0.000638586497983857\\
592.01	0.00056090470582257\\
593.01	0.00048276370607613\\
594.01	0.000404224610087212\\
595.01	0.000325347273344155\\
596.01	0.00024618784114796\\
597.01	0.000166807993572935\\
598.01	9.17366970068743e-05\\
599.01	2.94669364271152e-05\\
599.02	2.89574269585705e-05\\
599.03	2.84509530852819e-05\\
599.04	2.79475443462525e-05\\
599.05	2.7447230571279e-05\\
599.06	2.69500418838293e-05\\
599.07	2.64560087039449e-05\\
599.08	2.59651617511691e-05\\
599.09	2.54775320475235e-05\\
599.1	2.49931509204836e-05\\
599.11	2.45120500060158e-05\\
599.12	2.40342612516167e-05\\
599.13	2.35598169193996e-05\\
599.14	2.30887495892024e-05\\
599.15	2.26210921617405e-05\\
599.16	2.215687786177e-05\\
599.17	2.16961402412976e-05\\
599.18	2.12389131828191e-05\\
599.19	2.07852309025824e-05\\
599.2	2.03351279538938e-05\\
599.21	1.98886392304524e-05\\
599.22	1.94457999697188e-05\\
599.23	1.90066457563063e-05\\
599.24	1.85712125254194e-05\\
599.25	1.81395365663334e-05\\
599.26	1.77116564095865e-05\\
599.27	1.72876122135658e-05\\
599.28	1.68674445361946e-05\\
599.29	1.64511943388859e-05\\
599.3	1.60389029905273e-05\\
599.31	1.56306122715104e-05\\
599.32	1.52263643777902e-05\\
599.33	1.48262019250105e-05\\
599.34	1.44301679526268e-05\\
599.35	1.40383059281154e-05\\
599.36	1.36506597511899e-05\\
599.37	1.32672737580847e-05\\
599.38	1.28881927258535e-05\\
599.39	1.25134618767404e-05\\
599.4	1.21431268825696e-05\\
599.41	1.17772338691924e-05\\
599.42	1.14158294209736e-05\\
599.43	1.10589605853174e-05\\
599.44	1.07066748772523e-05\\
599.45	1.03590202840433e-05\\
599.46	1.00160452698589e-05\\
599.47	9.67779878049101e-06\\
599.48	9.34433024810284e-06\\
599.49	9.0156895960411e-06\\
599.5	8.69192724369319e-06\\
599.51	8.37309411137396e-06\\
599.52	8.05924162529392e-06\\
599.53	7.75042172253965e-06\\
599.54	7.44668685613396e-06\\
599.55	7.14809000013084e-06\\
599.56	6.85468465475535e-06\\
599.57	6.56652485161134e-06\\
599.58	6.28366515892896e-06\\
599.59	6.00616068686249e-06\\
599.6	5.73406709284546e-06\\
599.61	5.46744058700296e-06\\
599.62	5.2063379376039e-06\\
599.63	4.95081647657741e-06\\
599.64	4.70093410508653e-06\\
599.65	4.45674929914347e-06\\
599.66	4.21832111529262e-06\\
599.67	3.98570919634376e-06\\
599.68	3.75897377716608e-06\\
599.69	3.53817569053241e-06\\
599.7	3.32337637303295e-06\\
599.71	3.11463787102881e-06\\
599.72	2.91202284668363e-06\\
599.73	2.71559458404555e-06\\
599.74	2.52541699518466e-06\\
599.75	2.34155462640329e-06\\
599.76	2.16407266449489e-06\\
599.77	1.99303694307755e-06\\
599.78	1.82851394897772e-06\\
599.79	1.67057082867475e-06\\
599.8	1.51927539483385e-06\\
599.81	1.37469613287027e-06\\
599.82	1.23690220760718e-06\\
599.83	1.10596346997172e-06\\
599.84	9.81950463784659e-07\\
599.85	8.64934432591793e-07\\
599.86	7.54987326587186e-07\\
599.87	6.52181809583305e-07\\
599.88	5.56591266064402e-07\\
599.89	4.68289808295413e-07\\
599.9	3.87352283521061e-07\\
599.91	3.13854281216996e-07\\
599.92	2.4787214042421e-07\\
599.93	1.8948295714763e-07\\
599.94	1.38764591834512e-07\\
599.95	9.57956769204876e-08\\
599.96	6.06556244606149e-08\\
599.97	3.34246338194039e-08\\
599.98	1.4183699454523e-08\\
599.99	3.01461875948372e-09\\
600	0\\
};
\addplot [color=red!75!mycolor17,solid,forget plot]
  table[row sep=crcr]{%
0.01	0.00615124513121968\\
1.01	0.00615124466954397\\
2.01	0.00615124419804395\\
3.01	0.00615124371650986\\
4.01	0.00615124322472784\\
5.01	0.00615124272247878\\
6.01	0.00615124220953934\\
7.01	0.00615124168568137\\
8.01	0.00615124115067172\\
9.01	0.00615124060427232\\
10.01	0.00615124004623988\\
11.01	0.00615123947632624\\
12.01	0.00615123889427741\\
13.01	0.00615123829983411\\
14.01	0.00615123769273202\\
15.01	0.00615123707270048\\
16.01	0.00615123643946319\\
17.01	0.00615123579273817\\
18.01	0.00615123513223719\\
19.01	0.00615123445766603\\
20.01	0.00615123376872386\\
21.01	0.00615123306510371\\
22.01	0.00615123234649195\\
23.01	0.00615123161256789\\
24.01	0.00615123086300448\\
25.01	0.00615123009746729\\
26.01	0.00615122931561487\\
27.01	0.00615122851709837\\
28.01	0.00615122770156156\\
29.01	0.00615122686864046\\
30.01	0.00615122601796326\\
31.01	0.00615122514915043\\
32.01	0.00615122426181394\\
33.01	0.00615122335555755\\
34.01	0.0061512224299766\\
35.01	0.00615122148465763\\
36.01	0.00615122051917858\\
37.01	0.00615121953310805\\
38.01	0.00615121852600525\\
39.01	0.00615121749742009\\
40.01	0.00615121644689298\\
41.01	0.00615121537395443\\
42.01	0.00615121427812442\\
43.01	0.00615121315891327\\
44.01	0.00615121201582032\\
45.01	0.00615121084833429\\
46.01	0.00615120965593289\\
47.01	0.00615120843808264\\
48.01	0.00615120719423868\\
49.01	0.00615120592384418\\
50.01	0.00615120462633052\\
51.01	0.00615120330111693\\
52.01	0.00615120194761009\\
53.01	0.0061512005652038\\
54.01	0.00615119915327905\\
55.01	0.00615119771120327\\
56.01	0.00615119623833049\\
57.01	0.00615119473400065\\
58.01	0.00615119319753965\\
59.01	0.00615119162825901\\
60.01	0.00615119002545504\\
61.01	0.00615118838840926\\
62.01	0.00615118671638754\\
63.01	0.00615118500864007\\
64.01	0.00615118326440093\\
65.01	0.00615118148288763\\
66.01	0.00615117966330086\\
67.01	0.00615117780482403\\
68.01	0.00615117590662313\\
69.01	0.00615117396784602\\
70.01	0.00615117198762241\\
71.01	0.00615116996506288\\
72.01	0.00615116789925952\\
73.01	0.00615116578928429\\
74.01	0.00615116363418927\\
75.01	0.00615116143300638\\
76.01	0.00615115918474637\\
77.01	0.00615115688839902\\
78.01	0.00615115454293195\\
79.01	0.00615115214729076\\
80.01	0.00615114970039842\\
81.01	0.00615114720115439\\
82.01	0.00615114464843474\\
83.01	0.0061511420410911\\
84.01	0.00615113937795039\\
85.01	0.00615113665781445\\
86.01	0.00615113387945888\\
87.01	0.0061511310416332\\
88.01	0.00615112814305988\\
89.01	0.00615112518243382\\
90.01	0.00615112215842168\\
91.01	0.00615111906966155\\
92.01	0.00615111591476198\\
93.01	0.00615111269230171\\
94.01	0.00615110940082874\\
95.01	0.00615110603885955\\
96.01	0.00615110260487881\\
97.01	0.0061510990973384\\
98.01	0.00615109551465685\\
99.01	0.00615109185521853\\
100.01	0.00615108811737304\\
101.01	0.00615108429943421\\
102.01	0.0061510803996794\\
103.01	0.00615107641634907\\
104.01	0.0061510723476455\\
105.01	0.00615106819173216\\
106.01	0.00615106394673287\\
107.01	0.00615105961073085\\
108.01	0.00615105518176808\\
109.01	0.00615105065784422\\
110.01	0.00615104603691575\\
111.01	0.00615104131689492\\
112.01	0.00615103649564875\\
113.01	0.0061510315709984\\
114.01	0.00615102654071772\\
115.01	0.00615102140253287\\
116.01	0.00615101615412047\\
117.01	0.00615101079310716\\
118.01	0.00615100531706819\\
119.01	0.00615099972352604\\
120.01	0.0061509940099504\\
121.01	0.00615098817375558\\
122.01	0.00615098221230042\\
123.01	0.00615097612288638\\
124.01	0.00615096990275661\\
125.01	0.00615096354909479\\
126.01	0.00615095705902356\\
127.01	0.00615095042960346\\
128.01	0.00615094365783104\\
129.01	0.00615093674063835\\
130.01	0.00615092967489107\\
131.01	0.00615092245738647\\
132.01	0.00615091508485339\\
133.01	0.00615090755394924\\
134.01	0.00615089986125927\\
135.01	0.00615089200329484\\
136.01	0.00615088397649178\\
137.01	0.00615087577720894\\
138.01	0.00615086740172604\\
139.01	0.0061508588462425\\
140.01	0.00615085010687546\\
141.01	0.00615084117965773\\
142.01	0.00615083206053641\\
143.01	0.00615082274537083\\
144.01	0.00615081322993062\\
145.01	0.00615080350989381\\
146.01	0.00615079358084492\\
147.01	0.00615078343827253\\
148.01	0.00615077307756771\\
149.01	0.00615076249402159\\
150.01	0.00615075168282305\\
151.01	0.00615074063905718\\
152.01	0.00615072935770238\\
153.01	0.00615071783362799\\
154.01	0.00615070606159245\\
155.01	0.0061506940362403\\
156.01	0.00615068175210023\\
157.01	0.00615066920358219\\
158.01	0.00615065638497509\\
159.01	0.00615064329044373\\
160.01	0.00615062991402655\\
161.01	0.00615061624963279\\
162.01	0.00615060229103937\\
163.01	0.00615058803188819\\
164.01	0.00615057346568362\\
165.01	0.00615055858578872\\
166.01	0.00615054338542289\\
167.01	0.00615052785765803\\
168.01	0.00615051199541592\\
169.01	0.00615049579146478\\
170.01	0.00615047923841574\\
171.01	0.00615046232871981\\
172.01	0.00615044505466359\\
173.01	0.00615042740836686\\
174.01	0.006150409381778\\
175.01	0.00615039096667045\\
176.01	0.00615037215463921\\
177.01	0.00615035293709682\\
178.01	0.00615033330526905\\
179.01	0.00615031325019131\\
180.01	0.00615029276270415\\
181.01	0.00615027183344897\\
182.01	0.00615025045286421\\
183.01	0.00615022861118025\\
184.01	0.00615020629841525\\
185.01	0.00615018350437032\\
186.01	0.00615016021862495\\
187.01	0.0061501364305322\\
188.01	0.00615011212921345\\
189.01	0.00615008730355363\\
190.01	0.00615006194219591\\
191.01	0.00615003603353649\\
192.01	0.00615000956571926\\
193.01	0.00614998252662989\\
194.01	0.00614995490389073\\
195.01	0.00614992668485472\\
196.01	0.00614989785659962\\
197.01	0.00614986840592174\\
198.01	0.00614983831933027\\
199.01	0.0061498075830404\\
200.01	0.00614977618296703\\
201.01	0.00614974410471874\\
202.01	0.0061497113335904\\
203.01	0.00614967785455659\\
204.01	0.00614964365226432\\
205.01	0.0061496087110263\\
206.01	0.00614957301481319\\
207.01	0.00614953654724642\\
208.01	0.00614949929158999\\
209.01	0.00614946123074312\\
210.01	0.00614942234723224\\
211.01	0.00614938262320241\\
212.01	0.00614934204040888\\
213.01	0.00614930058020941\\
214.01	0.00614925822355434\\
215.01	0.00614921495097818\\
216.01	0.0061491707425906\\
217.01	0.00614912557806666\\
218.01	0.00614907943663756\\
219.01	0.00614903229708059\\
220.01	0.00614898413770928\\
221.01	0.00614893493636313\\
222.01	0.00614888467039695\\
223.01	0.0061488333166703\\
224.01	0.00614878085153668\\
225.01	0.00614872725083177\\
226.01	0.00614867248986285\\
227.01	0.00614861654339601\\
228.01	0.00614855938564508\\
229.01	0.00614850099025888\\
230.01	0.00614844133030864\\
231.01	0.00614838037827548\\
232.01	0.0061483181060369\\
233.01	0.00614825448485373\\
234.01	0.00614818948535598\\
235.01	0.00614812307752912\\
236.01	0.00614805523069937\\
237.01	0.00614798591351937\\
238.01	0.00614791509395259\\
239.01	0.00614784273925873\\
240.01	0.00614776881597704\\
241.01	0.00614769328991072\\
242.01	0.00614761612611038\\
243.01	0.00614753728885672\\
244.01	0.00614745674164379\\
245.01	0.00614737444716088\\
246.01	0.00614729036727486\\
247.01	0.00614720446301071\\
248.01	0.00614711669453367\\
249.01	0.00614702702112888\\
250.01	0.00614693540118201\\
251.01	0.00614684179215854\\
252.01	0.00614674615058323\\
253.01	0.00614664843201824\\
254.01	0.00614654859104191\\
255.01	0.00614644658122558\\
256.01	0.00614634235511138\\
257.01	0.00614623586418795\\
258.01	0.00614612705886717\\
259.01	0.00614601588845865\\
260.01	0.00614590230114513\\
261.01	0.00614578624395606\\
262.01	0.00614566766274146\\
263.01	0.00614554650214442\\
264.01	0.0061454227055733\\
265.01	0.00614529621517315\\
266.01	0.00614516697179641\\
267.01	0.00614503491497296\\
268.01	0.00614489998287925\\
269.01	0.0061447621123068\\
270.01	0.00614462123862972\\
271.01	0.00614447729577154\\
272.01	0.00614433021617141\\
273.01	0.00614417993074869\\
274.01	0.00614402636886761\\
275.01	0.00614386945830026\\
276.01	0.00614370912518887\\
277.01	0.00614354529400727\\
278.01	0.00614337788752106\\
279.01	0.00614320682674676\\
280.01	0.00614303203091033\\
281.01	0.00614285341740365\\
282.01	0.00614267090174094\\
283.01	0.00614248439751313\\
284.01	0.00614229381634143\\
285.01	0.00614209906782946\\
286.01	0.00614190005951409\\
287.01	0.00614169669681516\\
288.01	0.00614148888298399\\
289.01	0.00614127651904899\\
290.01	0.00614105950376205\\
291.01	0.00614083773354134\\
292.01	0.00614061110241417\\
293.01	0.00614037950195624\\
294.01	0.00614014282123195\\
295.01	0.00613990094672996\\
296.01	0.00613965376229902\\
297.01	0.00613940114908108\\
298.01	0.00613914298544244\\
299.01	0.00613887914690307\\
300.01	0.0061386095060638\\
301.01	0.00613833393253115\\
302.01	0.00613805229283985\\
303.01	0.00613776445037347\\
304.01	0.00613747026528237\\
305.01	0.00613716959439855\\
306.01	0.00613686229114891\\
307.01	0.00613654820546506\\
308.01	0.00613622718369033\\
309.01	0.00613589906848424\\
310.01	0.00613556369872345\\
311.01	0.00613522090939985\\
312.01	0.00613487053151501\\
313.01	0.00613451239197168\\
314.01	0.00613414631346101\\
315.01	0.00613377211434705\\
316.01	0.00613338960854596\\
317.01	0.00613299860540252\\
318.01	0.00613259890956122\\
319.01	0.00613219032083414\\
320.01	0.00613177263406312\\
321.01	0.00613134563897781\\
322.01	0.00613090912004864\\
323.01	0.00613046285633399\\
324.01	0.00613000662132245\\
325.01	0.00612954018276931\\
326.01	0.00612906330252674\\
327.01	0.00612857573636803\\
328.01	0.00612807723380498\\
329.01	0.00612756753789875\\
330.01	0.00612704638506286\\
331.01	0.00612651350485966\\
332.01	0.00612596861978793\\
333.01	0.0061254114450629\\
334.01	0.0061248416883865\\
335.01	0.0061242590497101\\
336.01	0.00612366322098599\\
337.01	0.00612305388590973\\
338.01	0.00612243071965169\\
339.01	0.00612179338857729\\
340.01	0.00612114154995517\\
341.01	0.00612047485165395\\
342.01	0.00611979293182497\\
343.01	0.00611909541857141\\
344.01	0.00611838192960334\\
345.01	0.00611765207187742\\
346.01	0.00611690544121983\\
347.01	0.00611614162193266\\
348.01	0.00611536018638253\\
349.01	0.00611456069456946\\
350.01	0.00611374269367594\\
351.01	0.00611290571759487\\
352.01	0.00611204928643453\\
353.01	0.00611117290600028\\
354.01	0.00611027606725035\\
355.01	0.00610935824572522\\
356.01	0.00610841890094807\\
357.01	0.00610745747579621\\
358.01	0.0061064733958397\\
359.01	0.00610546606864639\\
360.01	0.00610443488305109\\
361.01	0.00610337920838679\\
362.01	0.00610229839367529\\
363.01	0.00610119176677524\\
364.01	0.00610005863348544\\
365.01	0.00609889827659873\\
366.01	0.00609770995490624\\
367.01	0.00609649290214711\\
368.01	0.00609524632590198\\
369.01	0.00609396940642553\\
370.01	0.00609266129541588\\
371.01	0.0060913211147181\\
372.01	0.00608994795495598\\
373.01	0.00608854087409203\\
374.01	0.00608709889590906\\
375.01	0.0060856210084131\\
376.01	0.00608410616215264\\
377.01	0.00608255326845319\\
378.01	0.00608096119756439\\
379.01	0.0060793287767183\\
380.01	0.00607765478809893\\
381.01	0.00607593796672505\\
382.01	0.00607417699824758\\
383.01	0.00607237051666689\\
384.01	0.00607051710197895\\
385.01	0.00606861527775943\\
386.01	0.00606666350870381\\
387.01	0.00606466019814232\\
388.01	0.0060626036855588\\
389.01	0.00606049224414927\\
390.01	0.00605832407846529\\
391.01	0.00605609732220219\\
392.01	0.00605381003620425\\
393.01	0.00605146020677961\\
394.01	0.00604904574443693\\
395.01	0.00604656448318412\\
396.01	0.00604401418055822\\
397.01	0.00604139251859296\\
398.01	0.00603869710597121\\
399.01	0.00603592548165764\\
400.01	0.00603307512036123\\
401.01	0.00603014344023608\\
402.01	0.00602712781329085\\
403.01	0.00602402557904241\\
404.01	0.00602083406200576\\
405.01	0.00601755059365846\\
406.01	0.00601417253953499\\
407.01	0.00601069733207518\\
408.01	0.00600712250973919\\
409.01	0.00600344576266342\\
410.01	0.0059996649846932\\
411.01	0.00599577833089774\\
412.01	0.00599178427849172\\
413.01	0.00598768168726121\\
414.01	0.00598346985280637\\
415.01	0.0059791485417527\\
416.01	0.00597471799193696\\
417.01	0.00597017885158271\\
418.01	0.00596553201843817\\
419.01	0.00596077832103965\\
420.01	0.00595591762523831\\
421.01	0.0059509476872939\\
422.01	0.00594586574767648\\
423.01	0.00594066895397797\\
424.01	0.00593535435606941\\
425.01	0.00592991890088308\\
426.01	0.00592435942678017\\
427.01	0.00591867265746136\\
428.01	0.00591285519537157\\
429.01	0.00590690351454442\\
430.01	0.00590081395282439\\
431.01	0.005894582703397\\
432.01	0.00588820580554942\\
433.01	0.00588167913457138\\
434.01	0.00587499839069569\\
435.01	0.00586815908696614\\
436.01	0.00586115653590091\\
437.01	0.0058539858348075\\
438.01	0.00584664184958114\\
439.01	0.00583911919679941\\
440.01	0.00583141222389821\\
441.01	0.00582351498718718\\
442.01	0.00581542122743015\\
443.01	0.00580712434267734\\
444.01	0.00579861735800195\\
445.01	0.00578989289174168\\
446.01	0.00578094311780526\\
447.01	0.00577175972354865\\
448.01	0.00576233386267786\\
449.01	0.00575265610258605\\
450.01	0.00574271636548881\\
451.01	0.00573250386268671\\
452.01	0.00572200702127258\\
453.01	0.00571121340261374\\
454.01	0.00570010961200627\\
455.01	0.00568868119903684\\
456.01	0.00567691254843265\\
457.01	0.00566478676158142\\
458.01	0.00565228552953682\\
459.01	0.00563938899926999\\
460.01	0.00562607563633821\\
461.01	0.00561232208917969\\
462.01	0.0055981030631865\\
463.01	0.00558339121688668\\
464.01	0.0055681570984796\\
465.01	0.0055523691492855\\
466.01	0.00553599381229492\\
467.01	0.00551899580022513\\
468.01	0.00550133860003272\\
469.01	0.00548298532209655\\
470.01	0.00546390004556607\\
471.01	0.00544404987119809\\
472.01	0.00542340797560858\\
473.01	0.00540195807482421\\
474.01	0.00537970086212681\\
475.01	0.00535667491186894\\
476.01	0.00533310316495461\\
477.01	0.00530905067600642\\
478.01	0.00528451324571092\\
479.01	0.00525948785542397\\
480.01	0.00523397292823664\\
481.01	0.00520796863302586\\
482.01	0.00518147723529276\\
483.01	0.00515450349988047\\
484.01	0.0051270551583203\\
485.01	0.00509914344624079\\
486.01	0.00507078370929517\\
487.01	0.00504199607500807\\
488.01	0.00501280618139569\\
489.01	0.00498324594232345\\
490.01	0.00495335431279067\\
491.01	0.00492317799198879\\
492.01	0.00489277196421238\\
493.01	0.00486219972203605\\
494.01	0.00483153294461179\\
495.01	0.00480085041618139\\
496.01	0.00477023590868901\\
497.01	0.0047397744102359\\
498.01	0.00470954587410876\\
499.01	0.00467961535984617\\
500.01	0.00465001797757927\\
501.01	0.00462062509242448\\
502.01	0.00459109249044627\\
503.01	0.00456143234565688\\
504.01	0.0045316820202934\\
505.01	0.00450188043189105\\
506.01	0.00447206739146352\\
507.01	0.00444228271748359\\
508.01	0.00441256508991101\\
509.01	0.00438295061273709\\
510.01	0.00435347106457543\\
511.01	0.00432415183879997\\
512.01	0.00429500961361187\\
513.01	0.00426604985698718\\
514.01	0.00423726437445556\\
515.01	0.00420862926745604\\
516.01	0.00418010391291735\\
517.01	0.00415163193823281\\
518.01	0.00412314570316703\\
519.01	0.00409458785638967\\
520.01	0.00406595094842425\\
521.01	0.00403723620157672\\
522.01	0.00400844050139541\\
523.01	0.00397955589621132\\
524.01	0.00395056922292003\\
525.01	0.00392146193374226\\
526.01	0.00389221021685914\\
527.01	0.00386278551309164\\
528.01	0.0038331555110655\\
529.01	0.00380328567100357\\
530.01	0.00377314126510747\\
531.01	0.0037426897917219\\
532.01	0.00371190338476699\\
533.01	0.00368076009306139\\
534.01	0.00364924023746064\\
535.01	0.00361732258221762\\
536.01	0.00358498429767557\\
537.01	0.00355220137212305\\
538.01	0.00351894910426884\\
539.01	0.00348520264598917\\
540.01	0.00345093754616732\\
541.01	0.00341613022608219\\
542.01	0.0033807582992128\\
543.01	0.00334480064213531\\
544.01	0.00330823714412511\\
545.01	0.00327104815352956\\
546.01	0.00323321401108179\\
547.01	0.00319471507490733\\
548.01	0.0031555318906373\\
549.01	0.00311564533753994\\
550.01	0.00307503673297587\\
551.01	0.00303368788459745\\
552.01	0.00299158108659081\\
553.01	0.00294869906736923\\
554.01	0.00290502491097292\\
555.01	0.00286054198984103\\
556.01	0.00281523395154636\\
557.01	0.00276908476086833\\
558.01	0.0027220787504204\\
559.01	0.00267420066112334\\
560.01	0.00262543567394102\\
561.01	0.00257576943734479\\
562.01	0.00252518809674998\\
563.01	0.00247367833295417\\
564.01	0.00242122741540764\\
565.01	0.0023678232722642\\
566.01	0.00231345457350701\\
567.01	0.00225811082177995\\
568.01	0.00220178245017577\\
569.01	0.00214446092893834\\
570.01	0.00208613888320133\\
571.01	0.00202681022346972\\
572.01	0.00196647028977617\\
573.01	0.00190511600947689\\
574.01	0.00184274606784671\\
575.01	0.00177936109045576\\
576.01	0.00171496383673893\\
577.01	0.00164955940429165\\
578.01	0.00158315544293594\\
579.01	0.00151576237682872\\
580.01	0.00144739363196437\\
581.01	0.00137806586538633\\
582.01	0.0013077991912756\\
583.01	0.00123661739776533\\
584.01	0.0011645481466601\\
585.01	0.00109162314601654\\
586.01	0.00101787828266926\\
587.01	0.000943353698183423\\
588.01	0.000868093787227074\\
589.01	0.000792147091763009\\
590.01	0.000715566057480832\\
591.01	0.000638406610165536\\
592.01	0.000560727498800357\\
593.01	0.000482589338632919\\
594.01	0.000404053270619676\\
595.01	0.000325179132900562\\
596.01	0.000246023014372898\\
597.01	0.000166666784466021\\
598.01	9.17366970063904e-05\\
599.01	2.94669364271135e-05\\
599.02	2.89574269585705e-05\\
599.03	2.84509530852801e-05\\
599.04	2.79475443462508e-05\\
599.05	2.7447230571279e-05\\
599.06	2.69500418838293e-05\\
599.07	2.64560087039432e-05\\
599.08	2.59651617511691e-05\\
599.09	2.54775320475235e-05\\
599.1	2.49931509204854e-05\\
599.11	2.45120500060175e-05\\
599.12	2.40342612516185e-05\\
599.13	2.35598169193978e-05\\
599.14	2.30887495892024e-05\\
599.15	2.26210921617422e-05\\
599.16	2.21568778617717e-05\\
599.17	2.16961402412993e-05\\
599.18	2.12389131828191e-05\\
599.19	2.07852309025824e-05\\
599.2	2.03351279538938e-05\\
599.21	1.98886392304542e-05\\
599.22	1.94457999697188e-05\\
599.23	1.90066457563046e-05\\
599.24	1.85712125254211e-05\\
599.25	1.81395365663351e-05\\
599.26	1.77116564095865e-05\\
599.27	1.72876122135658e-05\\
599.28	1.68674445361946e-05\\
599.29	1.64511943388859e-05\\
599.3	1.60389029905273e-05\\
599.31	1.56306122715087e-05\\
599.32	1.5226364377792e-05\\
599.33	1.48262019250105e-05\\
599.34	1.44301679526268e-05\\
599.35	1.40383059281154e-05\\
599.36	1.36506597511916e-05\\
599.37	1.32672737580847e-05\\
599.38	1.28881927258552e-05\\
599.39	1.25134618767404e-05\\
599.4	1.21431268825696e-05\\
599.41	1.17772338691924e-05\\
599.42	1.14158294209719e-05\\
599.43	1.10589605853192e-05\\
599.44	1.07066748772523e-05\\
599.45	1.03590202840433e-05\\
599.46	1.00160452698606e-05\\
599.47	9.67779878049274e-06\\
599.48	9.34433024810284e-06\\
599.49	9.01568959604283e-06\\
599.5	8.69192724369319e-06\\
599.51	8.3730941113757e-06\\
599.52	8.05924162529219e-06\\
599.53	7.75042172253965e-06\\
599.54	7.4466868561357e-06\\
599.55	7.14809000013084e-06\\
599.56	6.85468465475535e-06\\
599.57	6.56652485161308e-06\\
599.58	6.2836651589307e-06\\
599.59	6.00616068686076e-06\\
599.6	5.73406709284546e-06\\
599.61	5.46744058700123e-06\\
599.62	5.2063379376039e-06\\
599.63	4.95081647657741e-06\\
599.64	4.70093410508653e-06\\
599.65	4.45674929914347e-06\\
599.66	4.21832111529089e-06\\
599.67	3.98570919634376e-06\\
599.68	3.75897377716435e-06\\
599.69	3.53817569053415e-06\\
599.7	3.32337637303295e-06\\
599.71	3.11463787103054e-06\\
599.72	2.91202284668536e-06\\
599.73	2.71559458404555e-06\\
599.74	2.52541699518292e-06\\
599.75	2.34155462640155e-06\\
599.76	2.16407266449489e-06\\
599.77	1.99303694307928e-06\\
599.78	1.82851394897598e-06\\
599.79	1.67057082867302e-06\\
599.8	1.51927539483211e-06\\
599.81	1.37469613287027e-06\\
599.82	1.23690220760718e-06\\
599.83	1.10596346997172e-06\\
599.84	9.81950463782924e-07\\
599.85	8.64934432591793e-07\\
599.86	7.54987326587186e-07\\
599.87	6.5218180958504e-07\\
599.88	5.56591266064402e-07\\
599.89	4.68289808295413e-07\\
599.9	3.87352283521061e-07\\
599.91	3.13854281218731e-07\\
599.92	2.47872140425945e-07\\
599.93	1.89482957149364e-07\\
599.94	1.38764591834512e-07\\
599.95	9.57956769222224e-08\\
599.96	6.06556244606149e-08\\
599.97	3.34246338211386e-08\\
599.98	1.4183699454523e-08\\
599.99	3.01461875948372e-09\\
600	0\\
};
\addplot [color=red!80!mycolor19,solid,forget plot]
  table[row sep=crcr]{%
0.01	0.0060323088673989\\
1.01	0.00603230822979061\\
2.01	0.0060323075786237\\
3.01	0.00603230691360877\\
4.01	0.00603230623445046\\
5.01	0.00603230554084724\\
6.01	0.00603230483249098\\
7.01	0.00603230410906711\\
8.01	0.00603230337025409\\
9.01	0.00603230261572357\\
10.01	0.00603230184514019\\
11.01	0.00603230105816139\\
12.01	0.00603230025443744\\
13.01	0.00603229943361105\\
14.01	0.00603229859531677\\
15.01	0.00603229773918224\\
16.01	0.00603229686482659\\
17.01	0.00603229597186072\\
18.01	0.00603229505988737\\
19.01	0.00603229412850074\\
20.01	0.00603229317728638\\
21.01	0.00603229220582086\\
22.01	0.00603229121367175\\
23.01	0.0060322902003973\\
24.01	0.0060322891655463\\
25.01	0.00603228810865806\\
26.01	0.0060322870292616\\
27.01	0.00603228592687629\\
28.01	0.006032284801011\\
29.01	0.00603228365116394\\
30.01	0.00603228247682287\\
31.01	0.0060322812774641\\
32.01	0.00603228005255319\\
33.01	0.00603227880154409\\
34.01	0.00603227752387895\\
35.01	0.00603227621898803\\
36.01	0.00603227488628917\\
37.01	0.00603227352518803\\
38.01	0.00603227213507735\\
39.01	0.00603227071533685\\
40.01	0.00603226926533295\\
41.01	0.00603226778441839\\
42.01	0.00603226627193235\\
43.01	0.00603226472719929\\
44.01	0.0060322631495296\\
45.01	0.0060322615382187\\
46.01	0.00603225989254695\\
47.01	0.00603225821177918\\
48.01	0.00603225649516439\\
49.01	0.00603225474193552\\
50.01	0.00603225295130922\\
51.01	0.00603225112248514\\
52.01	0.00603224925464555\\
53.01	0.00603224734695528\\
54.01	0.00603224539856145\\
55.01	0.00603224340859254\\
56.01	0.00603224137615865\\
57.01	0.00603223930035039\\
58.01	0.00603223718023908\\
59.01	0.0060322350148757\\
60.01	0.00603223280329144\\
61.01	0.00603223054449612\\
62.01	0.00603222823747868\\
63.01	0.00603222588120604\\
64.01	0.00603222347462281\\
65.01	0.00603222101665137\\
66.01	0.00603221850619037\\
67.01	0.00603221594211521\\
68.01	0.00603221332327691\\
69.01	0.00603221064850169\\
70.01	0.00603220791659062\\
71.01	0.00603220512631906\\
72.01	0.00603220227643587\\
73.01	0.00603219936566315\\
74.01	0.00603219639269535\\
75.01	0.0060321933561987\\
76.01	0.00603219025481103\\
77.01	0.00603218708714056\\
78.01	0.00603218385176591\\
79.01	0.00603218054723475\\
80.01	0.0060321771720635\\
81.01	0.00603217372473684\\
82.01	0.00603217020370665\\
83.01	0.00603216660739147\\
84.01	0.00603216293417598\\
85.01	0.00603215918240966\\
86.01	0.00603215535040686\\
87.01	0.00603215143644542\\
88.01	0.00603214743876594\\
89.01	0.00603214335557143\\
90.01	0.0060321391850259\\
91.01	0.00603213492525388\\
92.01	0.0060321305743395\\
93.01	0.00603212613032547\\
94.01	0.00603212159121241\\
95.01	0.00603211695495789\\
96.01	0.00603211221947524\\
97.01	0.00603210738263305\\
98.01	0.00603210244225382\\
99.01	0.00603209739611303\\
100.01	0.00603209224193806\\
101.01	0.00603208697740756\\
102.01	0.00603208160015007\\
103.01	0.00603207610774268\\
104.01	0.00603207049771058\\
105.01	0.00603206476752528\\
106.01	0.00603205891460374\\
107.01	0.00603205293630724\\
108.01	0.00603204682994021\\
109.01	0.00603204059274872\\
110.01	0.00603203422191947\\
111.01	0.00603202771457836\\
112.01	0.00603202106778945\\
113.01	0.00603201427855305\\
114.01	0.0060320073438051\\
115.01	0.00603200026041496\\
116.01	0.00603199302518452\\
117.01	0.00603198563484681\\
118.01	0.00603197808606388\\
119.01	0.0060319703754262\\
120.01	0.00603196249944992\\
121.01	0.00603195445457652\\
122.01	0.00603194623717027\\
123.01	0.00603193784351689\\
124.01	0.00603192926982196\\
125.01	0.00603192051220891\\
126.01	0.00603191156671731\\
127.01	0.00603190242930123\\
128.01	0.0060318930958273\\
129.01	0.0060318835620728\\
130.01	0.00603187382372352\\
131.01	0.00603186387637232\\
132.01	0.00603185371551617\\
133.01	0.00603184333655526\\
134.01	0.00603183273478989\\
135.01	0.00603182190541884\\
136.01	0.00603181084353705\\
137.01	0.00603179954413329\\
138.01	0.0060317880020879\\
139.01	0.00603177621217043\\
140.01	0.00603176416903718\\
141.01	0.00603175186722916\\
142.01	0.00603173930116886\\
143.01	0.00603172646515809\\
144.01	0.00603171335337539\\
145.01	0.00603169995987326\\
146.01	0.00603168627857525\\
147.01	0.00603167230327385\\
148.01	0.0060316580276267\\
149.01	0.0060316434451543\\
150.01	0.00603162854923662\\
151.01	0.00603161333311041\\
152.01	0.0060315977898659\\
153.01	0.00603158191244361\\
154.01	0.0060315656936313\\
155.01	0.0060315491260604\\
156.01	0.00603153220220251\\
157.01	0.00603151491436623\\
158.01	0.00603149725469341\\
159.01	0.00603147921515594\\
160.01	0.00603146078755138\\
161.01	0.0060314419634995\\
162.01	0.00603142273443861\\
163.01	0.0060314030916214\\
164.01	0.00603138302611078\\
165.01	0.00603136252877606\\
166.01	0.00603134159028829\\
167.01	0.00603132020111665\\
168.01	0.00603129835152348\\
169.01	0.0060312760315597\\
170.01	0.00603125323106085\\
171.01	0.00603122993964158\\
172.01	0.0060312061466916\\
173.01	0.0060311818413703\\
174.01	0.00603115701260169\\
175.01	0.00603113164907001\\
176.01	0.00603110573921339\\
177.01	0.00603107927121927\\
178.01	0.00603105223301893\\
179.01	0.00603102461228137\\
180.01	0.00603099639640822\\
181.01	0.00603096757252764\\
182.01	0.00603093812748813\\
183.01	0.00603090804785276\\
184.01	0.00603087731989301\\
185.01	0.00603084592958219\\
186.01	0.00603081386258891\\
187.01	0.00603078110427054\\
188.01	0.00603074763966666\\
189.01	0.00603071345349168\\
190.01	0.00603067853012803\\
191.01	0.00603064285361903\\
192.01	0.00603060640766096\\
193.01	0.00603056917559613\\
194.01	0.00603053114040492\\
195.01	0.00603049228469773\\
196.01	0.00603045259070696\\
197.01	0.00603041204027912\\
198.01	0.00603037061486567\\
199.01	0.00603032829551533\\
200.01	0.00603028506286447\\
201.01	0.00603024089712847\\
202.01	0.00603019577809261\\
203.01	0.00603014968510258\\
204.01	0.0060301025970547\\
205.01	0.00603005449238637\\
206.01	0.00603000534906598\\
207.01	0.00602995514458247\\
208.01	0.006029903855935\\
209.01	0.00602985145962232\\
210.01	0.00602979793163174\\
211.01	0.00602974324742779\\
212.01	0.00602968738194122\\
213.01	0.00602963030955659\\
214.01	0.00602957200410113\\
215.01	0.00602951243883215\\
216.01	0.00602945158642428\\
217.01	0.00602938941895736\\
218.01	0.00602932590790274\\
219.01	0.00602926102411058\\
220.01	0.00602919473779578\\
221.01	0.00602912701852399\\
222.01	0.0060290578351978\\
223.01	0.00602898715604198\\
224.01	0.00602891494858824\\
225.01	0.0060288411796609\\
226.01	0.00602876581536028\\
227.01	0.00602868882104774\\
228.01	0.00602861016132892\\
229.01	0.00602852980003709\\
230.01	0.00602844770021664\\
231.01	0.00602836382410504\\
232.01	0.00602827813311566\\
233.01	0.00602819058781932\\
234.01	0.00602810114792604\\
235.01	0.00602800977226561\\
236.01	0.00602791641876861\\
237.01	0.0060278210444464\\
238.01	0.00602772360537084\\
239.01	0.00602762405665334\\
240.01	0.00602752235242419\\
241.01	0.00602741844581051\\
242.01	0.00602731228891381\\
243.01	0.00602720383278795\\
244.01	0.00602709302741537\\
245.01	0.00602697982168388\\
246.01	0.00602686416336153\\
247.01	0.00602674599907317\\
248.01	0.00602662527427365\\
249.01	0.00602650193322271\\
250.01	0.00602637591895818\\
251.01	0.00602624717326863\\
252.01	0.00602611563666591\\
253.01	0.00602598124835641\\
254.01	0.00602584394621202\\
255.01	0.00602570366674065\\
256.01	0.00602556034505534\\
257.01	0.00602541391484351\\
258.01	0.00602526430833456\\
259.01	0.00602511145626771\\
260.01	0.00602495528785824\\
261.01	0.00602479573076368\\
262.01	0.00602463271104836\\
263.01	0.00602446615314793\\
264.01	0.0060242959798326\\
265.01	0.00602412211216966\\
266.01	0.00602394446948496\\
267.01	0.0060237629693236\\
268.01	0.00602357752740974\\
269.01	0.00602338805760507\\
270.01	0.00602319447186705\\
271.01	0.00602299668020533\\
272.01	0.00602279459063738\\
273.01	0.00602258810914331\\
274.01	0.00602237713961935\\
275.01	0.00602216158383037\\
276.01	0.00602194134136069\\
277.01	0.00602171630956459\\
278.01	0.00602148638351458\\
279.01	0.00602125145594939\\
280.01	0.00602101141721989\\
281.01	0.00602076615523391\\
282.01	0.0060205155554001\\
283.01	0.00602025950056934\\
284.01	0.00601999787097592\\
285.01	0.0060197305441762\\
286.01	0.00601945739498642\\
287.01	0.00601917829541845\\
288.01	0.00601889311461384\\
289.01	0.00601860171877663\\
290.01	0.00601830397110407\\
291.01	0.00601799973171564\\
292.01	0.00601768885757962\\
293.01	0.00601737120243923\\
294.01	0.00601704661673481\\
295.01	0.00601671494752553\\
296.01	0.00601637603840791\\
297.01	0.00601602972943313\\
298.01	0.00601567585702081\\
299.01	0.0060153142538717\\
300.01	0.00601494474887713\\
301.01	0.00601456716702615\\
302.01	0.00601418132931047\\
303.01	0.00601378705262579\\
304.01	0.00601338414967121\\
305.01	0.00601297242884532\\
306.01	0.00601255169413933\\
307.01	0.00601212174502735\\
308.01	0.00601168237635293\\
309.01	0.00601123337821282\\
310.01	0.00601077453583678\\
311.01	0.00601030562946407\\
312.01	0.00600982643421604\\
313.01	0.00600933671996531\\
314.01	0.00600883625119968\\
315.01	0.00600832478688312\\
316.01	0.00600780208031201\\
317.01	0.0060072678789662\\
318.01	0.00600672192435638\\
319.01	0.00600616395186524\\
320.01	0.00600559369058467\\
321.01	0.00600501086314687\\
322.01	0.00600441518554961\\
323.01	0.00600380636697669\\
324.01	0.00600318410961098\\
325.01	0.00600254810844194\\
326.01	0.00600189805106614\\
327.01	0.00600123361748045\\
328.01	0.0060005544798687\\
329.01	0.00599986030237971\\
330.01	0.00599915074089787\\
331.01	0.00599842544280482\\
332.01	0.00599768404673316\\
333.01	0.00599692618230945\\
334.01	0.00599615146988878\\
335.01	0.00599535952027785\\
336.01	0.00599454993444854\\
337.01	0.00599372230323855\\
338.01	0.00599287620704164\\
339.01	0.00599201121548363\\
340.01	0.00599112688708665\\
341.01	0.00599022276891755\\
342.01	0.00598929839622296\\
343.01	0.00598835329204795\\
344.01	0.00598738696683767\\
345.01	0.00598639891802182\\
346.01	0.00598538862958029\\
347.01	0.00598435557158891\\
348.01	0.00598329919974391\\
349.01	0.00598221895486372\\
350.01	0.00598111426236792\\
351.01	0.00597998453172951\\
352.01	0.00597882915590072\\
353.01	0.00597764751070914\\
354.01	0.00597643895422407\\
355.01	0.00597520282608819\\
356.01	0.00597393844681501\\
357.01	0.005972645117047\\
358.01	0.00597132211677355\\
359.01	0.0059699687045044\\
360.01	0.00596858411639517\\
361.01	0.00596716756532282\\
362.01	0.00596571823990379\\
363.01	0.00596423530345384\\
364.01	0.00596271789288124\\
365.01	0.00596116511751003\\
366.01	0.00595957605782627\\
367.01	0.00595794976414015\\
368.01	0.0059562852551558\\
369.01	0.00595458151644177\\
370.01	0.00595283749879069\\
371.01	0.00595105211645827\\
372.01	0.00594922424526912\\
373.01	0.00594735272057594\\
374.01	0.00594543633505635\\
375.01	0.00594347383632999\\
376.01	0.00594146392437586\\
377.01	0.00593940524872865\\
378.01	0.00593729640542735\\
379.01	0.00593513593368828\\
380.01	0.00593292231227016\\
381.01	0.00593065395549255\\
382.01	0.00592832920886761\\
383.01	0.00592594634429556\\
384.01	0.00592350355476975\\
385.01	0.00592099894852966\\
386.01	0.00591843054258989\\
387.01	0.00591579625556634\\
388.01	0.00591309389970917\\
389.01	0.00591032117203818\\
390.01	0.00590747564446821\\
391.01	0.00590455475279379\\
392.01	0.00590155578439159\\
393.01	0.00589847586448267\\
394.01	0.00589531194078513\\
395.01	0.00589206076637255\\
396.01	0.00588871888055167\\
397.01	0.00588528258756502\\
398.01	0.00588174793293934\\
399.01	0.00587811067732352\\
400.01	0.00587436626771316\\
401.01	0.0058705098060435\\
402.01	0.0058665360152756\\
403.01	0.00586243920331443\\
404.01	0.00585821322542409\\
405.01	0.00585385144627785\\
406.01	0.005849346703474\\
407.01	0.00584469127533662\\
408.01	0.00583987685722575\\
409.01	0.00583489455256717\\
410.01	0.00582973488760384\\
411.01	0.00582438786276795\\
412.01	0.00581884305901457\\
413.01	0.00581308982501403\\
414.01	0.00580711758158385\\
415.01	0.00580091629425014\\
416.01	0.00579447718488472\\
417.01	0.00578779378104929\\
418.01	0.0057808634398562\\
419.01	0.00577368953577726\\
420.01	0.00576631687256892\\
421.01	0.00575878074051462\\
422.01	0.00575107735371502\\
423.01	0.0057432028327119\\
424.01	0.00573515320250625\\
425.01	0.00572692439066269\\
426.01	0.00571851222553454\\
427.01	0.00570991243464606\\
428.01	0.00570112064327694\\
429.01	0.00569213237330246\\
430.01	0.00568294304235371\\
431.01	0.00567354796337234\\
432.01	0.00566394234465059\\
433.01	0.00565412129046134\\
434.01	0.00564407980240533\\
435.01	0.00563381278162197\\
436.01	0.00562331503203997\\
437.01	0.0056125812648738\\
438.01	0.00560160610460731\\
439.01	0.00559038409675234\\
440.01	0.00557890971771511\\
441.01	0.00556717738716608\\
442.01	0.00555518148337079\\
443.01	0.00554291636202407\\
444.01	0.00553037637921097\\
445.01	0.00551755591922929\\
446.01	0.00550444942811777\\
447.01	0.00549105145387235\\
448.01	0.00547735669447731\\
449.01	0.00546336005504276\\
450.01	0.00544905671551638\\
451.01	0.00543444221062791\\
452.01	0.00541951252391299\\
453.01	0.0054042641978536\\
454.01	0.00538869446233392\\
455.01	0.0053728013837293\\
456.01	0.00535658403697551\\
457.01	0.00534004270286315\\
458.01	0.00532317909247237\\
459.01	0.00530599660000874\\
460.01	0.00528850058414319\\
461.01	0.00527069867609095\\
462.01	0.00525260110973188\\
463.01	0.00523422106466096\\
464.01	0.00521557500650392\\
465.01	0.00519668299926648\\
466.01	0.00517756895069433\\
467.01	0.00515826073188677\\
468.01	0.00513879008439024\\
469.01	0.00511919218843788\\
470.01	0.00509950471039199\\
471.01	0.00507976606957098\\
472.01	0.00506001255591459\\
473.01	0.00504027377851333\\
474.01	0.00502056571458162\\
475.01	0.00500086854867461\\
476.01	0.00498097279191445\\
477.01	0.0049608181976079\\
478.01	0.00494041518376095\\
479.01	0.00491977568273822\\
480.01	0.00489891321125773\\
481.01	0.00487784292242705\\
482.01	0.00485658163246677\\
483.01	0.00483514781308942\\
484.01	0.0048135615384342\\
485.01	0.00479184437305596\\
486.01	0.00477001918516127\\
487.01	0.00474810986696565\\
488.01	0.00472614094195711\\
489.01	0.00470413703747452\\
490.01	0.00468212220105582\\
491.01	0.00466011904158858\\
492.01	0.00463814768309418\\
493.01	0.00461622453253979\\
494.01	0.00459436088709398\\
495.01	0.00457256144371503\\
496.01	0.00455082282733694\\
497.01	0.00452913234196017\\
498.01	0.00450746728654395\\
499.01	0.00448579538001983\\
500.01	0.00446407713748063\\
501.01	0.00444227403134053\\
502.01	0.00442037839858218\\
503.01	0.00439839898302056\\
504.01	0.00437634322091555\\
505.01	0.00435421663797677\\
506.01	0.00433202245018829\\
507.01	0.00430976117040764\\
508.01	0.00428743024502981\\
509.01	0.00426502375278557\\
510.01	0.00424253220619599\\
511.01	0.00421994250452162\\
512.01	0.00419723809369644\\
513.01	0.00417439939107087\\
514.01	0.00415140452641962\\
515.01	0.00412823042857954\\
516.01	0.00410485423826421\\
517.01	0.00408125493514216\\
518.01	0.00405741490545617\\
519.01	0.00403332066448421\\
520.01	0.00400896013218617\\
521.01	0.00398431983134321\\
522.01	0.0039593847362626\\
523.01	0.00393413843157042\\
524.01	0.00390856333801112\\
525.01	0.0038826410022877\\
526.01	0.0038563524398284\\
527.01	0.0038296785084463\\
528.01	0.00380260027802109\\
529.01	0.00377509934822052\\
530.01	0.00374715805511664\\
531.01	0.00371875950392218\\
532.01	0.00368988737917599\\
533.01	0.0036605255388462\\
534.01	0.00363065763087537\\
535.01	0.00360026706979577\\
536.01	0.00356933716041013\\
537.01	0.00353785121829806\\
538.01	0.00350579267201484\\
539.01	0.0034731451374028\\
540.01	0.00343989245617953\\
541.01	0.0034060186949134\\
542.01	0.003371508107079\\
543.01	0.0033363450700003\\
544.01	0.00330051401884971\\
545.01	0.00326399940755064\\
546.01	0.00322678571573412\\
547.01	0.00318885747810126\\
548.01	0.00315019930833668\\
549.01	0.00311079591342006\\
550.01	0.00307063209961312\\
551.01	0.00302969277287801\\
552.01	0.00298796293775365\\
553.01	0.0029454276994024\\
554.01	0.00290207227312257\\
555.01	0.00285788200362993\\
556.01	0.0028128423928549\\
557.01	0.00276693913212747\\
558.01	0.00272015813645232\\
559.01	0.00267248558161509\\
560.01	0.00262390794559945\\
561.01	0.00257441205578024\\
562.01	0.00252398514312515\\
563.01	0.00247261490420945\\
564.01	0.00242028957133667\\
565.01	0.00236699799067106\\
566.01	0.00231272970830979\\
567.01	0.00225747506467842\\
568.01	0.0022012252979938\\
569.01	0.00214397265754098\\
570.01	0.0020857105273739\\
571.01	0.00202643356088241\\
572.01	0.00196613782650036\\
573.01	0.00190482096468999\\
574.01	0.00184248235623363\\
575.01	0.00177912330175362\\
576.01	0.00171474721218609\\
577.01	0.00164935980960222\\
578.01	0.00158296933731052\\
579.01	0.001515586777603\\
580.01	0.00144722607478906\\
581.01	0.0013779043602682\\
582.01	0.00130764217525753\\
583.01	0.00123646368534738\\
584.01	0.00116439687921647\\
585.01	0.00109147374150863\\
586.01	0.00101773038694345\\
587.01	0.000943207139070279\\
588.01	0.000867948532496969\\
589.01	0.000792003211719553\\
590.01	0.000715423692577445\\
591.01	0.000638265943538507\\
592.01	0.000560588733085159\\
593.01	0.000482452675945771\\
594.01	0.00040391889420434\\
595.01	0.000325047188695244\\
596.01	0.000245893590649514\\
597.01	0.000166581383471795\\
598.01	9.17366970063782e-05\\
599.01	2.94669364271152e-05\\
599.02	2.89574269585705e-05\\
599.03	2.84509530852819e-05\\
599.04	2.79475443462525e-05\\
599.05	2.74472305712807e-05\\
599.06	2.6950041883831e-05\\
599.07	2.64560087039449e-05\\
599.08	2.59651617511691e-05\\
599.09	2.54775320475218e-05\\
599.1	2.49931509204854e-05\\
599.11	2.45120500060175e-05\\
599.12	2.40342612516167e-05\\
599.13	2.35598169193996e-05\\
599.14	2.30887495892024e-05\\
599.15	2.26210921617405e-05\\
599.16	2.215687786177e-05\\
599.17	2.16961402412993e-05\\
599.18	2.12389131828191e-05\\
599.19	2.07852309025824e-05\\
599.2	2.03351279538938e-05\\
599.21	1.98886392304542e-05\\
599.22	1.94457999697206e-05\\
599.23	1.90066457563063e-05\\
599.24	1.85712125254211e-05\\
599.25	1.81395365663334e-05\\
599.26	1.77116564095865e-05\\
599.27	1.72876122135658e-05\\
599.28	1.68674445361946e-05\\
599.29	1.64511943388877e-05\\
599.3	1.60389029905273e-05\\
599.31	1.56306122715087e-05\\
599.32	1.52263643777902e-05\\
599.33	1.48262019250105e-05\\
599.34	1.44301679526268e-05\\
599.35	1.40383059281154e-05\\
599.36	1.36506597511916e-05\\
599.37	1.32672737580865e-05\\
599.38	1.28881927258535e-05\\
599.39	1.25134618767404e-05\\
599.4	1.21431268825679e-05\\
599.41	1.17772338691924e-05\\
599.42	1.14158294209736e-05\\
599.43	1.10589605853174e-05\\
599.44	1.07066748772523e-05\\
599.45	1.03590202840433e-05\\
599.46	1.00160452698589e-05\\
599.47	9.67779878049101e-06\\
599.48	9.34433024810284e-06\\
599.49	9.0156895960411e-06\\
599.5	8.69192724369146e-06\\
599.51	8.37309411137396e-06\\
599.52	8.05924162529392e-06\\
599.53	7.75042172253965e-06\\
599.54	7.44668685613396e-06\\
599.55	7.14809000013084e-06\\
599.56	6.85468465475535e-06\\
599.57	6.56652485161308e-06\\
599.58	6.28366515892896e-06\\
599.59	6.00616068686249e-06\\
599.6	5.73406709284546e-06\\
599.61	5.46744058700296e-06\\
599.62	5.20633793760217e-06\\
599.63	4.95081647657741e-06\\
599.64	4.70093410508653e-06\\
599.65	4.45674929914347e-06\\
599.66	4.21832111529262e-06\\
599.67	3.98570919634203e-06\\
599.68	3.75897377716608e-06\\
599.69	3.53817569053415e-06\\
599.7	3.32337637303469e-06\\
599.71	3.11463787102881e-06\\
599.72	2.91202284668363e-06\\
599.73	2.71559458404382e-06\\
599.74	2.52541699518466e-06\\
599.75	2.34155462640329e-06\\
599.76	2.16407266449489e-06\\
599.77	1.99303694307928e-06\\
599.78	1.82851394897598e-06\\
599.79	1.67057082867302e-06\\
599.8	1.51927539483211e-06\\
599.81	1.37469613287027e-06\\
599.82	1.23690220760544e-06\\
599.83	1.10596346997172e-06\\
599.84	9.81950463782924e-07\\
599.85	8.64934432591793e-07\\
599.86	7.54987326588921e-07\\
599.87	6.5218180958504e-07\\
599.88	5.56591266064402e-07\\
599.89	4.68289808295413e-07\\
599.9	3.87352283521061e-07\\
599.91	3.13854281216996e-07\\
599.92	2.47872140425945e-07\\
599.93	1.8948295714763e-07\\
599.94	1.38764591834512e-07\\
599.95	9.57956769204876e-08\\
599.96	6.06556244606149e-08\\
599.97	3.34246338194039e-08\\
599.98	1.4183699454523e-08\\
599.99	3.014618757749e-09\\
600	0\\
};
\addplot [color=red,solid,forget plot]
  table[row sep=crcr]{%
0.01	0.00579827411309351\\
1.01	0.00579827314537753\\
2.01	0.00579827215707644\\
3.01	0.00579827114775116\\
4.01	0.00579827011695319\\
5.01	0.00579826906422434\\
6.01	0.00579826798909636\\
7.01	0.00579826689109139\\
8.01	0.00579826576972118\\
9.01	0.0057982646244869\\
10.01	0.00579826345487936\\
11.01	0.00579826226037807\\
12.01	0.00579826104045152\\
13.01	0.00579825979455707\\
14.01	0.00579825852214016\\
15.01	0.0057982572226343\\
16.01	0.00579825589546109\\
17.01	0.00579825454002963\\
18.01	0.00579825315573628\\
19.01	0.00579825174196466\\
20.01	0.00579825029808493\\
21.01	0.00579824882345383\\
22.01	0.0057982473174144\\
23.01	0.00579824577929556\\
24.01	0.00579824420841174\\
25.01	0.00579824260406261\\
26.01	0.0057982409655332\\
27.01	0.00579823929209277\\
28.01	0.00579823758299505\\
29.01	0.00579823583747786\\
30.01	0.00579823405476249\\
31.01	0.00579823223405377\\
32.01	0.00579823037453908\\
33.01	0.00579822847538871\\
34.01	0.00579822653575484\\
35.01	0.00579822455477176\\
36.01	0.00579822253155499\\
37.01	0.00579822046520106\\
38.01	0.0057982183547872\\
39.01	0.00579821619937075\\
40.01	0.00579821399798874\\
41.01	0.00579821174965755\\
42.01	0.00579820945337251\\
43.01	0.00579820710810723\\
44.01	0.00579820471281346\\
45.01	0.00579820226642009\\
46.01	0.00579819976783343\\
47.01	0.00579819721593598\\
48.01	0.00579819460958627\\
49.01	0.0057981919476184\\
50.01	0.00579818922884112\\
51.01	0.00579818645203786\\
52.01	0.00579818361596585\\
53.01	0.00579818071935561\\
54.01	0.00579817776091003\\
55.01	0.00579817473930455\\
56.01	0.00579817165318576\\
57.01	0.00579816850117141\\
58.01	0.0057981652818493\\
59.01	0.00579816199377695\\
60.01	0.00579815863548088\\
61.01	0.00579815520545563\\
62.01	0.00579815170216373\\
63.01	0.00579814812403414\\
64.01	0.00579814446946259\\
65.01	0.00579814073680959\\
66.01	0.00579813692440098\\
67.01	0.0057981330305261\\
68.01	0.00579812905343762\\
69.01	0.00579812499135079\\
70.01	0.00579812084244191\\
71.01	0.00579811660484832\\
72.01	0.00579811227666712\\
73.01	0.00579810785595438\\
74.01	0.00579810334072441\\
75.01	0.0057980987289487\\
76.01	0.00579809401855495\\
77.01	0.00579808920742611\\
78.01	0.0057980842933996\\
79.01	0.00579807927426606\\
80.01	0.0057980741477687\\
81.01	0.00579806891160204\\
82.01	0.00579806356341071\\
83.01	0.0057980581007887\\
84.01	0.00579805252127776\\
85.01	0.00579804682236706\\
86.01	0.00579804100149125\\
87.01	0.00579803505602974\\
88.01	0.00579802898330541\\
89.01	0.00579802278058322\\
90.01	0.00579801644506926\\
91.01	0.00579800997390934\\
92.01	0.00579800336418742\\
93.01	0.00579799661292453\\
94.01	0.00579798971707748\\
95.01	0.00579798267353733\\
96.01	0.00579797547912801\\
97.01	0.00579796813060459\\
98.01	0.00579796062465232\\
99.01	0.00579795295788471\\
100.01	0.00579794512684214\\
101.01	0.0057979371279901\\
102.01	0.00579792895771785\\
103.01	0.00579792061233645\\
104.01	0.00579791208807725\\
105.01	0.00579790338109002\\
106.01	0.00579789448744148\\
107.01	0.00579788540311315\\
108.01	0.0057978761239998\\
109.01	0.00579786664590704\\
110.01	0.00579785696455001\\
111.01	0.00579784707555114\\
112.01	0.00579783697443797\\
113.01	0.00579782665664153\\
114.01	0.00579781611749366\\
115.01	0.00579780535222541\\
116.01	0.00579779435596459\\
117.01	0.00579778312373305\\
118.01	0.00579777165044552\\
119.01	0.00579775993090621\\
120.01	0.00579774795980698\\
121.01	0.00579773573172455\\
122.01	0.00579772324111822\\
123.01	0.00579771048232739\\
124.01	0.00579769744956855\\
125.01	0.00579768413693305\\
126.01	0.00579767053838438\\
127.01	0.0057976566477548\\
128.01	0.00579764245874324\\
129.01	0.00579762796491163\\
130.01	0.00579761315968266\\
131.01	0.00579759803633612\\
132.01	0.00579758258800648\\
133.01	0.0057975668076788\\
134.01	0.00579755068818649\\
135.01	0.00579753422220721\\
136.01	0.00579751740225975\\
137.01	0.00579750022070049\\
138.01	0.00579748266972034\\
139.01	0.00579746474134026\\
140.01	0.00579744642740845\\
141.01	0.0057974277195958\\
142.01	0.00579740860939252\\
143.01	0.00579738908810407\\
144.01	0.00579736914684723\\
145.01	0.00579734877654559\\
146.01	0.00579732796792604\\
147.01	0.00579730671151351\\
148.01	0.00579728499762742\\
149.01	0.00579726281637671\\
150.01	0.00579724015765566\\
151.01	0.00579721701113865\\
152.01	0.00579719336627584\\
153.01	0.00579716921228806\\
154.01	0.00579714453816153\\
155.01	0.00579711933264326\\
156.01	0.00579709358423587\\
157.01	0.00579706728119147\\
158.01	0.00579704041150698\\
159.01	0.00579701296291797\\
160.01	0.0057969849228934\\
161.01	0.00579695627862977\\
162.01	0.00579692701704473\\
163.01	0.00579689712477128\\
164.01	0.00579686658815173\\
165.01	0.00579683539323069\\
166.01	0.00579680352574962\\
167.01	0.00579677097113907\\
168.01	0.00579673771451243\\
169.01	0.00579670374065922\\
170.01	0.00579666903403737\\
171.01	0.00579663357876651\\
172.01	0.00579659735862035\\
173.01	0.0057965603570189\\
174.01	0.00579652255702124\\
175.01	0.00579648394131712\\
176.01	0.00579644449221892\\
177.01	0.00579640419165386\\
178.01	0.00579636302115499\\
179.01	0.0057963209618527\\
180.01	0.00579627799446601\\
181.01	0.00579623409929379\\
182.01	0.00579618925620505\\
183.01	0.00579614344462948\\
184.01	0.00579609664354852\\
185.01	0.00579604883148476\\
186.01	0.00579599998649227\\
187.01	0.00579595008614644\\
188.01	0.0057958991075329\\
189.01	0.00579584702723753\\
190.01	0.00579579382133497\\
191.01	0.00579573946537734\\
192.01	0.00579568393438316\\
193.01	0.00579562720282547\\
194.01	0.00579556924461956\\
195.01	0.00579551003311128\\
196.01	0.005795449541064\\
197.01	0.00579538774064614\\
198.01	0.00579532460341794\\
199.01	0.0057952601003182\\
200.01	0.00579519420165056\\
201.01	0.00579512687706953\\
202.01	0.00579505809556621\\
203.01	0.0057949878254534\\
204.01	0.00579491603435147\\
205.01	0.00579484268917235\\
206.01	0.00579476775610444\\
207.01	0.00579469120059632\\
208.01	0.00579461298734054\\
209.01	0.00579453308025713\\
210.01	0.00579445144247664\\
211.01	0.00579436803632249\\
212.01	0.00579428282329334\\
213.01	0.00579419576404477\\
214.01	0.00579410681837108\\
215.01	0.00579401594518572\\
216.01	0.00579392310250235\\
217.01	0.00579382824741506\\
218.01	0.00579373133607749\\
219.01	0.00579363232368251\\
220.01	0.00579353116444109\\
221.01	0.00579342781156036\\
222.01	0.00579332221722175\\
223.01	0.00579321433255784\\
224.01	0.00579310410762988\\
225.01	0.00579299149140342\\
226.01	0.00579287643172464\\
227.01	0.00579275887529527\\
228.01	0.00579263876764722\\
229.01	0.00579251605311728\\
230.01	0.00579239067481969\\
231.01	0.00579226257462006\\
232.01	0.00579213169310728\\
233.01	0.00579199796956499\\
234.01	0.00579186134194323\\
235.01	0.00579172174682847\\
236.01	0.00579157911941347\\
237.01	0.00579143339346638\\
238.01	0.00579128450129919\\
239.01	0.00579113237373528\\
240.01	0.00579097694007667\\
241.01	0.00579081812806965\\
242.01	0.00579065586387095\\
243.01	0.00579049007201172\\
244.01	0.00579032067536187\\
245.01	0.00579014759509287\\
246.01	0.00578997075063997\\
247.01	0.00578979005966341\\
248.01	0.00578960543800922\\
249.01	0.00578941679966862\\
250.01	0.0057892240567364\\
251.01	0.00578902711936903\\
252.01	0.00578882589574102\\
253.01	0.00578862029200088\\
254.01	0.00578841021222596\\
255.01	0.00578819555837566\\
256.01	0.00578797623024446\\
257.01	0.0057877521254132\\
258.01	0.00578752313919994\\
259.01	0.00578728916460873\\
260.01	0.0057870500922779\\
261.01	0.00578680581042684\\
262.01	0.00578655620480192\\
263.01	0.00578630115862098\\
264.01	0.0057860405525161\\
265.01	0.0057857742644753\\
266.01	0.00578550216978347\\
267.01	0.00578522414096144\\
268.01	0.00578494004770291\\
269.01	0.00578464975681151\\
270.01	0.00578435313213467\\
271.01	0.00578405003449716\\
272.01	0.00578374032163303\\
273.01	0.00578342384811487\\
274.01	0.00578310046528246\\
275.01	0.00578277002116907\\
276.01	0.0057824323604267\\
277.01	0.00578208732424903\\
278.01	0.00578173475029253\\
279.01	0.00578137447259588\\
280.01	0.00578100632149719\\
281.01	0.00578063012354972\\
282.01	0.0057802457014351\\
283.01	0.00577985287387483\\
284.01	0.00577945145553875\\
285.01	0.00577904125695298\\
286.01	0.00577862208440381\\
287.01	0.00577819373984031\\
288.01	0.00577775602077444\\
289.01	0.00577730872017823\\
290.01	0.00577685162637893\\
291.01	0.00577638452295131\\
292.01	0.00577590718860751\\
293.01	0.00577541939708381\\
294.01	0.00577492091702463\\
295.01	0.00577441151186419\\
296.01	0.00577389093970455\\
297.01	0.00577335895319051\\
298.01	0.00577281529938182\\
299.01	0.00577225971962204\\
300.01	0.00577169194940389\\
301.01	0.00577111171823091\\
302.01	0.00577051874947646\\
303.01	0.00576991276023777\\
304.01	0.00576929346118716\\
305.01	0.00576866055641959\\
306.01	0.00576801374329498\\
307.01	0.00576735271227751\\
308.01	0.00576667714677085\\
309.01	0.00576598672294736\\
310.01	0.00576528110957491\\
311.01	0.0057645599678377\\
312.01	0.00576382295115243\\
313.01	0.00576306970497961\\
314.01	0.00576229986663068\\
315.01	0.00576151306506771\\
316.01	0.00576070892069997\\
317.01	0.00575988704517318\\
318.01	0.00575904704115399\\
319.01	0.0057581885021081\\
320.01	0.00575731101207152\\
321.01	0.00575641414541665\\
322.01	0.00575549746661117\\
323.01	0.00575456052996957\\
324.01	0.00575360287939838\\
325.01	0.005752624048134\\
326.01	0.00575162355847247\\
327.01	0.00575060092149164\\
328.01	0.00574955563676572\\
329.01	0.00574848719207092\\
330.01	0.00574739506308279\\
331.01	0.00574627871306462\\
332.01	0.00574513759254621\\
333.01	0.00574397113899348\\
334.01	0.00574277877646861\\
335.01	0.0057415599152792\\
336.01	0.00574031395161721\\
337.01	0.0057390402671877\\
338.01	0.00573773822882488\\
339.01	0.00573640718809818\\
340.01	0.00573504648090548\\
341.01	0.00573365542705417\\
342.01	0.00573223332982955\\
343.01	0.00573077947555017\\
344.01	0.00572929313311008\\
345.01	0.00572777355350623\\
346.01	0.00572621996935286\\
347.01	0.00572463159438094\\
348.01	0.00572300762292222\\
349.01	0.00572134722937925\\
350.01	0.00571964956767855\\
351.01	0.00571791377070902\\
352.01	0.00571613894974415\\
353.01	0.00571432419384813\\
354.01	0.00571246856926574\\
355.01	0.00571057111879649\\
356.01	0.00570863086115212\\
357.01	0.00570664679029909\\
358.01	0.00570461787478514\\
359.01	0.00570254305705154\\
360.01	0.00570042125273152\\
361.01	0.00569825134993498\\
362.01	0.0056960322085237\\
363.01	0.00569376265937483\\
364.01	0.00569144150363884\\
365.01	0.0056890675119921\\
366.01	0.00568663942388841\\
367.01	0.00568415594681422\\
368.01	0.00568161575555128\\
369.01	0.00567901749145439\\
370.01	0.00567635976175055\\
371.01	0.00567364113886814\\
372.01	0.0056708601598084\\
373.01	0.00566801532556912\\
374.01	0.00566510510063793\\
375.01	0.00566212791257145\\
376.01	0.00565908215168273\\
377.01	0.00565596617085954\\
378.01	0.00565277828554576\\
379.01	0.00564951677391856\\
380.01	0.00564617987730409\\
381.01	0.00564276580087828\\
382.01	0.00563927271471126\\
383.01	0.00563569875522094\\
384.01	0.00563204202711578\\
385.01	0.00562830060591658\\
386.01	0.00562447254116693\\
387.01	0.00562055586045687\\
388.01	0.00561654857440534\\
389.01	0.00561244868277319\\
390.01	0.00560825418190138\\
391.01	0.00560396307370325\\
392.01	0.00559957337646969\\
393.01	0.00559508313778593\\
394.01	0.00559049044989649\\
395.01	0.00558579346789798\\
396.01	0.00558099043117951\\
397.01	0.00557607968857266\\
398.01	0.00557105972770409\\
399.01	0.00556592920906667\\
400.01	0.00556068700532121\\
401.01	0.00555533224630952\\
402.01	0.00554986437016528\\
403.01	0.00554428318074\\
404.01	0.0055385889112695\\
405.01	0.00553278229373854\\
406.01	0.00552686463268604\\
407.01	0.00552083788111343\\
408.01	0.00551470471457791\\
409.01	0.0055084685972551\\
410.01	0.00550213383045587\\
411.01	0.00549570556939092\\
412.01	0.00548918978734055\\
413.01	0.0054825931570487\\
414.01	0.00547592280609357\\
415.01	0.00546918588474558\\
416.01	0.00546238885944615\\
417.01	0.00545553640981832\\
418.01	0.00544862975833197\\
419.01	0.00544166419427558\\
420.01	0.00543459249843498\\
421.01	0.00542737374615082\\
422.01	0.00542000527256111\\
423.01	0.00541248440613606\\
424.01	0.00540480847372568\\
425.01	0.00539697480624451\\
426.01	0.00538898074505829\\
427.01	0.0053808236491421\\
428.01	0.00537250090308864\\
429.01	0.00536400992604847\\
430.01	0.00535534818169118\\
431.01	0.00534651318928491\\
432.01	0.00533750253599418\\
433.01	0.00532831389050632\\
434.01	0.00531894501809819\\
435.01	0.00530939379726229\\
436.01	0.00529965823801376\\
437.01	0.00528973650200004\\
438.01	0.00527962692453595\\
439.01	0.00526932803867838\\
440.01	0.00525883860145041\\
441.01	0.00524815762230244\\
442.01	0.00523728439388126\\
443.01	0.00522621852513802\\
444.01	0.00521495997676421\\
445.01	0.00520350909887938\\
446.01	0.00519186667081365\\
447.01	0.00518003394271962\\
448.01	0.00516801267861244\\
449.01	0.00515580520026193\\
450.01	0.00514341443114205\\
451.01	0.00513084393937215\\
452.01	0.00511809797825092\\
453.01	0.00510518152257694\\
454.01	0.0050921002984596\\
455.01	0.00507886080374135\\
456.01	0.00506547031546919\\
457.01	0.00505193688006244\\
458.01	0.00503826928093613\\
459.01	0.00502447697736356\\
460.01	0.00501057000734561\\
461.01	0.0049965588462518\\
462.01	0.00498245421214625\\
463.01	0.00496826680818446\\
464.01	0.00495400699256695\\
465.01	0.00493968436772454\\
466.01	0.00492530728334791\\
467.01	0.00491088225358604\\
468.01	0.00489641329867669\\
469.01	0.00488190123757946\\
470.01	0.00486734298392155\\
471.01	0.0048527309371367\\
472.01	0.00483805262034577\\
473.01	0.0048232908051748\\
474.01	0.00480842449390443\\
475.01	0.00479343143458618\\
476.01	0.00477830033022567\\
477.01	0.00476303646284201\\
478.01	0.00474764684725635\\
479.01	0.00473213867838251\\
480.01	0.00471651923707135\\
481.01	0.00470079577686061\\
482.01	0.00468497539039543\\
483.01	0.00466906485480435\\
484.01	0.00465307045607682\\
485.01	0.00463699779360229\\
486.01	0.00462085156753712\\
487.01	0.00460463535367401\\
488.01	0.00458835137305685\\
489.01	0.00457200026678467\\
490.01	0.00455558089029488\\
491.01	0.0045390901458345\\
492.01	0.00452252287660589\\
493.01	0.0045058718507412\\
494.01	0.00448912786693964\\
495.01	0.00447228001483836\\
496.01	0.00445531611978078\\
497.01	0.00443822338997197\\
498.01	0.00442098925806168\\
499.01	0.00440360236002616\\
500.01	0.00438605350842562\\
501.01	0.00436833633741155\\
502.01	0.00435044623385012\\
503.01	0.00433237797366445\\
504.01	0.00431412521693101\\
505.01	0.00429568048089654\\
506.01	0.00427703514314936\\
507.01	0.00425817947909612\\
508.01	0.00423910273824293\\
509.01	0.00421979326213203\\
510.01	0.00420023864409566\\
511.01	0.00418042592702429\\
512.01	0.00416034182994515\\
513.01	0.00413997298735496\\
514.01	0.00411930617726896\\
515.01	0.00409832850577516\\
516.01	0.00407702750949197\\
517.01	0.00405539113648135\\
518.01	0.00403340757746375\\
519.01	0.00401106495778801\\
520.01	0.00398835105619131\\
521.01	0.00396525327661902\\
522.01	0.00394175872708765\\
523.01	0.00391785430582586\\
524.01	0.00389352678603324\\
525.01	0.00386876289331736\\
526.01	0.00384354936936821\\
527.01	0.00381787301565537\\
528.01	0.00379172071226016\\
529.01	0.00376507940970956\\
530.01	0.00373793609609847\\
531.01	0.00371027774778276\\
532.01	0.00368209127865747\\
533.01	0.00365336350813047\\
534.01	0.00362408116246383\\
535.01	0.00359423089568912\\
536.01	0.0035637993083127\\
537.01	0.00353277295895014\\
538.01	0.00350113836867015\\
539.01	0.00346888201873439\\
540.01	0.00343599034334621\\
541.01	0.00340244971988049\\
542.01	0.0033682464596358\\
543.01	0.00333336680210739\\
544.01	0.00329779691482852\\
545.01	0.00326152289880476\\
546.01	0.00322453079706372\\
547.01	0.00318680660347261\\
548.01	0.00314833627127922\\
549.01	0.00310910572210278\\
550.01	0.0030691008562936\\
551.01	0.00302830756557809\\
552.01	0.00298671174877585\\
553.01	0.00294429933111708\\
554.01	0.00290105628736226\\
555.01	0.00285696866863559\\
556.01	0.002812022632798\\
557.01	0.00276620447842027\\
558.01	0.00271950068276585\\
559.01	0.0026718979443291\\
560.01	0.0026233832304573\\
561.01	0.00257394383053661\\
562.01	0.00252356741517104\\
563.01	0.00247224210175046\\
564.01	0.00241995652680611\\
565.01	0.00236669992559503\\
566.01	0.00231246221942173\\
567.01	0.00225723411125625\\
568.01	0.00220100719020869\\
569.01	0.00214377404539634\\
570.01	0.00208552838970178\\
571.01	0.00202626519387332\\
572.01	0.00196598083135666\\
573.01	0.00190467323416349\\
574.01	0.00184234205996039\\
575.01	0.00177898887038548\\
576.01	0.00171461732035613\\
577.01	0.00164923335779841\\
578.01	0.00158284543280214\\
579.01	0.00151546471464598\\
580.01	0.00144710531442435\\
581.01	0.00137778451008954\\
582.01	0.00130752296955477\\
583.01	0.00123634496602094\\
584.01	0.00116427857781718\\
585.01	0.00109135586269028\\
586.01	0.00101761299352925\\
587.01	0.000943090338822979\\
588.01	0.000867832466552738\\
589.01	0.000791888044500708\\
590.01	0.000715309602850512\\
591.01	0.00063815311614427\\
592.01	0.000560477350747159\\
593.01	0.000482342910469856\\
594.01	0.000403810896310433\\
595.01	0.000324941075665367\\
596.01	0.00024578943091586\\
597.01	0.000166525623744458\\
598.01	9.17366970063782e-05\\
599.01	2.94669364271135e-05\\
599.02	2.89574269585688e-05\\
599.03	2.84509530852801e-05\\
599.04	2.79475443462508e-05\\
599.05	2.7447230571279e-05\\
599.06	2.69500418838293e-05\\
599.07	2.64560087039432e-05\\
599.08	2.59651617511691e-05\\
599.09	2.54775320475235e-05\\
599.1	2.49931509204836e-05\\
599.11	2.45120500060175e-05\\
599.12	2.40342612516185e-05\\
599.13	2.35598169193996e-05\\
599.14	2.30887495892024e-05\\
599.15	2.26210921617405e-05\\
599.16	2.215687786177e-05\\
599.17	2.16961402412976e-05\\
599.18	2.12389131828191e-05\\
599.19	2.07852309025824e-05\\
599.2	2.03351279538938e-05\\
599.21	1.98886392304542e-05\\
599.22	1.94457999697188e-05\\
599.23	1.90066457563063e-05\\
599.24	1.85712125254211e-05\\
599.25	1.81395365663334e-05\\
599.26	1.77116564095865e-05\\
599.27	1.72876122135658e-05\\
599.28	1.68674445361946e-05\\
599.29	1.64511943388859e-05\\
599.3	1.60389029905273e-05\\
599.31	1.56306122715087e-05\\
599.32	1.5226364377792e-05\\
599.33	1.48262019250105e-05\\
599.34	1.44301679526268e-05\\
599.35	1.40383059281154e-05\\
599.36	1.36506597511916e-05\\
599.37	1.32672737580847e-05\\
599.38	1.28881927258535e-05\\
599.39	1.25134618767404e-05\\
599.4	1.21431268825696e-05\\
599.41	1.17772338691924e-05\\
599.42	1.14158294209719e-05\\
599.43	1.10589605853174e-05\\
599.44	1.07066748772523e-05\\
599.45	1.03590202840433e-05\\
599.46	1.00160452698606e-05\\
599.47	9.67779878049101e-06\\
599.48	9.34433024810111e-06\\
599.49	9.01568959604283e-06\\
599.5	8.69192724369319e-06\\
599.51	8.37309411137396e-06\\
599.52	8.05924162529219e-06\\
599.53	7.75042172253791e-06\\
599.54	7.4466868561357e-06\\
599.55	7.14809000013084e-06\\
599.56	6.85468465475535e-06\\
599.57	6.56652485161134e-06\\
599.58	6.28366515892896e-06\\
599.59	6.00616068686249e-06\\
599.6	5.73406709284546e-06\\
599.61	5.46744058700296e-06\\
599.62	5.2063379376039e-06\\
599.63	4.95081647657741e-06\\
599.64	4.70093410508479e-06\\
599.65	4.45674929914174e-06\\
599.66	4.21832111529089e-06\\
599.67	3.98570919634376e-06\\
599.68	3.75897377716608e-06\\
599.69	3.53817569053241e-06\\
599.7	3.32337637303295e-06\\
599.71	3.11463787102881e-06\\
599.72	2.91202284668363e-06\\
599.73	2.71559458404555e-06\\
599.74	2.52541699518292e-06\\
599.75	2.34155462640155e-06\\
599.76	2.16407266449663e-06\\
599.77	1.99303694307755e-06\\
599.78	1.82851394897598e-06\\
599.79	1.67057082867475e-06\\
599.8	1.51927539483211e-06\\
599.81	1.37469613287027e-06\\
599.82	1.23690220760718e-06\\
599.83	1.10596346997172e-06\\
599.84	9.81950463784659e-07\\
599.85	8.64934432590059e-07\\
599.86	7.54987326587186e-07\\
599.87	6.52181809583305e-07\\
599.88	5.56591266064402e-07\\
599.89	4.68289808295413e-07\\
599.9	3.87352283521061e-07\\
599.91	3.13854281218731e-07\\
599.92	2.4787214042421e-07\\
599.93	1.8948295714763e-07\\
599.94	1.38764591834512e-07\\
599.95	9.57956769204876e-08\\
599.96	6.06556244606149e-08\\
599.97	3.34246338194039e-08\\
599.98	1.4183699454523e-08\\
599.99	3.01461875948372e-09\\
600	0\\
};
\addplot [color=mycolor20,solid,forget plot]
  table[row sep=crcr]{%
0.01	0.00553720637056298\\
1.01	0.00553720532562888\\
2.01	0.00553720425853025\\
3.01	0.00553720316879582\\
4.01	0.0055372020559441\\
5.01	0.00553720091948356\\
6.01	0.00553719975891235\\
7.01	0.00553719857371746\\
8.01	0.00553719736337522\\
9.01	0.00553719612735087\\
10.01	0.00553719486509796\\
11.01	0.00553719357605868\\
12.01	0.0055371922596631\\
13.01	0.00553719091532918\\
14.01	0.00553718954246257\\
15.01	0.00553718814045607\\
16.01	0.00553718670868974\\
17.01	0.00553718524653015\\
18.01	0.00553718375333059\\
19.01	0.00553718222843034\\
20.01	0.00553718067115473\\
21.01	0.00553717908081458\\
22.01	0.0055371774567061\\
23.01	0.00553717579811048\\
24.01	0.00553717410429338\\
25.01	0.005537172374505\\
26.01	0.00553717060797924\\
27.01	0.00553716880393391\\
28.01	0.00553716696157009\\
29.01	0.0055371650800717\\
30.01	0.00553716315860519\\
31.01	0.0055371611963193\\
32.01	0.00553715919234463\\
33.01	0.00553715714579308\\
34.01	0.0055371550557576\\
35.01	0.00553715292131169\\
36.01	0.00553715074150919\\
37.01	0.00553714851538364\\
38.01	0.00553714624194788\\
39.01	0.00553714392019385\\
40.01	0.00553714154909165\\
41.01	0.00553713912758965\\
42.01	0.00553713665461338\\
43.01	0.00553713412906581\\
44.01	0.00553713154982601\\
45.01	0.00553712891574957\\
46.01	0.00553712622566704\\
47.01	0.00553712347838437\\
48.01	0.00553712067268193\\
49.01	0.00553711780731382\\
50.01	0.00553711488100765\\
51.01	0.00553711189246362\\
52.01	0.00553710884035421\\
53.01	0.0055371057233236\\
54.01	0.00553710253998694\\
55.01	0.00553709928892959\\
56.01	0.00553709596870648\\
57.01	0.00553709257784203\\
58.01	0.00553708911482873\\
59.01	0.0055370855781272\\
60.01	0.00553708196616446\\
61.01	0.0055370782773344\\
62.01	0.00553707450999615\\
63.01	0.00553707066247405\\
64.01	0.00553706673305628\\
65.01	0.00553706271999447\\
66.01	0.00553705862150265\\
67.01	0.00553705443575672\\
68.01	0.00553705016089349\\
69.01	0.00553704579500971\\
70.01	0.00553704133616168\\
71.01	0.00553703678236353\\
72.01	0.0055370321315874\\
73.01	0.00553702738176148\\
74.01	0.0055370225307697\\
75.01	0.00553701757645073\\
76.01	0.00553701251659681\\
77.01	0.00553700734895287\\
78.01	0.00553700207121549\\
79.01	0.00553699668103181\\
80.01	0.0055369911759986\\
81.01	0.00553698555366081\\
82.01	0.00553697981151105\\
83.01	0.0055369739469879\\
84.01	0.00553696795747521\\
85.01	0.00553696184030048\\
86.01	0.00553695559273409\\
87.01	0.00553694921198756\\
88.01	0.00553694269521272\\
89.01	0.00553693603950039\\
90.01	0.0055369292418787\\
91.01	0.00553692229931204\\
92.01	0.00553691520869987\\
93.01	0.00553690796687468\\
94.01	0.00553690057060118\\
95.01	0.00553689301657463\\
96.01	0.00553688530141914\\
97.01	0.00553687742168652\\
98.01	0.00553686937385437\\
99.01	0.00553686115432469\\
100.01	0.00553685275942206\\
101.01	0.00553684418539229\\
102.01	0.00553683542840027\\
103.01	0.00553682648452875\\
104.01	0.00553681734977601\\
105.01	0.00553680802005464\\
106.01	0.00553679849118908\\
107.01	0.0055367887589142\\
108.01	0.00553677881887289\\
109.01	0.00553676866661483\\
110.01	0.00553675829759354\\
111.01	0.00553674770716508\\
112.01	0.00553673689058559\\
113.01	0.00553672584300898\\
114.01	0.00553671455948508\\
115.01	0.00553670303495715\\
116.01	0.00553669126425974\\
117.01	0.00553667924211642\\
118.01	0.00553666696313701\\
119.01	0.00553665442181534\\
120.01	0.00553664161252666\\
121.01	0.00553662852952538\\
122.01	0.00553661516694214\\
123.01	0.00553660151878125\\
124.01	0.00553658757891789\\
125.01	0.00553657334109543\\
126.01	0.00553655879892246\\
127.01	0.00553654394587013\\
128.01	0.00553652877526866\\
129.01	0.00553651328030499\\
130.01	0.00553649745401897\\
131.01	0.00553648128930096\\
132.01	0.00553646477888751\\
133.01	0.00553644791535928\\
134.01	0.00553643069113668\\
135.01	0.00553641309847707\\
136.01	0.00553639512947085\\
137.01	0.00553637677603802\\
138.01	0.00553635802992419\\
139.01	0.00553633888269745\\
140.01	0.0055363193257438\\
141.01	0.0055362993502634\\
142.01	0.0055362789472671\\
143.01	0.00553625810757184\\
144.01	0.0055362368217962\\
145.01	0.0055362150803568\\
146.01	0.00553619287346342\\
147.01	0.00553617019111457\\
148.01	0.00553614702309314\\
149.01	0.00553612335896175\\
150.01	0.00553609918805758\\
151.01	0.00553607449948804\\
152.01	0.00553604928212504\\
153.01	0.00553602352460071\\
154.01	0.00553599721530188\\
155.01	0.00553597034236451\\
156.01	0.00553594289366837\\
157.01	0.00553591485683214\\
158.01	0.00553588621920657\\
159.01	0.00553585696786985\\
160.01	0.00553582708962081\\
161.01	0.0055357965709733\\
162.01	0.00553576539814999\\
163.01	0.00553573355707596\\
164.01	0.00553570103337233\\
165.01	0.0055356678123497\\
166.01	0.00553563387900109\\
167.01	0.00553559921799565\\
168.01	0.00553556381367109\\
169.01	0.00553552765002675\\
170.01	0.0055354907107165\\
171.01	0.00553545297904048\\
172.01	0.00553541443793807\\
173.01	0.00553537506998019\\
174.01	0.00553533485736078\\
175.01	0.00553529378188889\\
176.01	0.00553525182498051\\
177.01	0.00553520896764966\\
178.01	0.00553516519049994\\
179.01	0.00553512047371552\\
180.01	0.00553507479705218\\
181.01	0.00553502813982761\\
182.01	0.00553498048091255\\
183.01	0.00553493179872054\\
184.01	0.00553488207119826\\
185.01	0.00553483127581537\\
186.01	0.00553477938955405\\
187.01	0.00553472638889845\\
188.01	0.00553467224982427\\
189.01	0.00553461694778693\\
190.01	0.00553456045771075\\
191.01	0.00553450275397771\\
192.01	0.00553444381041517\\
193.01	0.0055343836002841\\
194.01	0.00553432209626664\\
195.01	0.00553425927045353\\
196.01	0.00553419509433156\\
197.01	0.00553412953876997\\
198.01	0.0055340625740075\\
199.01	0.00553399416963834\\
200.01	0.00553392429459836\\
201.01	0.0055338529171507\\
202.01	0.00553378000487124\\
203.01	0.00553370552463382\\
204.01	0.00553362944259458\\
205.01	0.00553355172417663\\
206.01	0.00553347233405399\\
207.01	0.0055333912361354\\
208.01	0.00553330839354808\\
209.01	0.00553322376862004\\
210.01	0.00553313732286324\\
211.01	0.00553304901695543\\
212.01	0.00553295881072252\\
213.01	0.00553286666311976\\
214.01	0.00553277253221284\\
215.01	0.00553267637515866\\
216.01	0.00553257814818546\\
217.01	0.00553247780657256\\
218.01	0.00553237530463007\\
219.01	0.00553227059567756\\
220.01	0.00553216363202278\\
221.01	0.00553205436493915\\
222.01	0.00553194274464385\\
223.01	0.00553182872027455\\
224.01	0.00553171223986592\\
225.01	0.0055315932503259\\
226.01	0.00553147169741097\\
227.01	0.00553134752570107\\
228.01	0.00553122067857429\\
229.01	0.00553109109818053\\
230.01	0.00553095872541492\\
231.01	0.00553082349989038\\
232.01	0.00553068535990977\\
233.01	0.00553054424243772\\
234.01	0.00553040008307083\\
235.01	0.00553025281600856\\
236.01	0.00553010237402275\\
237.01	0.00552994868842619\\
238.01	0.00552979168904128\\
239.01	0.00552963130416764\\
240.01	0.00552946746054834\\
241.01	0.00552930008333689\\
242.01	0.00552912909606222\\
243.01	0.00552895442059329\\
244.01	0.00552877597710296\\
245.01	0.0055285936840316\\
246.01	0.00552840745804875\\
247.01	0.00552821721401476\\
248.01	0.00552802286494169\\
249.01	0.00552782432195278\\
250.01	0.00552762149424138\\
251.01	0.00552741428902916\\
252.01	0.00552720261152324\\
253.01	0.00552698636487199\\
254.01	0.00552676545012045\\
255.01	0.00552653976616447\\
256.01	0.00552630920970405\\
257.01	0.00552607367519561\\
258.01	0.00552583305480285\\
259.01	0.00552558723834706\\
260.01	0.0055253361132561\\
261.01	0.00552507956451263\\
262.01	0.00552481747460019\\
263.01	0.00552454972344926\\
264.01	0.00552427618838158\\
265.01	0.00552399674405377\\
266.01	0.00552371126239857\\
267.01	0.00552341961256598\\
268.01	0.00552312166086285\\
269.01	0.00552281727069066\\
270.01	0.00552250630248259\\
271.01	0.00552218861363884\\
272.01	0.00552186405846033\\
273.01	0.00552153248808176\\
274.01	0.00552119375040224\\
275.01	0.00552084769001537\\
276.01	0.00552049414813723\\
277.01	0.00552013296253245\\
278.01	0.00551976396743991\\
279.01	0.00551938699349545\\
280.01	0.00551900186765388\\
281.01	0.0055186084131092\\
282.01	0.00551820644921202\\
283.01	0.00551779579138705\\
284.01	0.00551737625104743\\
285.01	0.00551694763550737\\
286.01	0.00551650974789376\\
287.01	0.00551606238705486\\
288.01	0.00551560534746743\\
289.01	0.00551513841914275\\
290.01	0.00551466138752897\\
291.01	0.00551417403341258\\
292.01	0.00551367613281776\\
293.01	0.00551316745690295\\
294.01	0.00551264777185604\\
295.01	0.00551211683878689\\
296.01	0.00551157441361747\\
297.01	0.00551102024697017\\
298.01	0.00551045408405385\\
299.01	0.00550987566454698\\
300.01	0.00550928472247866\\
301.01	0.00550868098610768\\
302.01	0.00550806417779831\\
303.01	0.00550743401389427\\
304.01	0.00550679020459034\\
305.01	0.00550613245380003\\
306.01	0.00550546045902275\\
307.01	0.00550477391120687\\
308.01	0.00550407249461015\\
309.01	0.00550335588665917\\
310.01	0.00550262375780388\\
311.01	0.00550187577137102\\
312.01	0.00550111158341431\\
313.01	0.00550033084256196\\
314.01	0.00549953318986141\\
315.01	0.00549871825862176\\
316.01	0.00549788567425313\\
317.01	0.00549703505410395\\
318.01	0.00549616600729461\\
319.01	0.00549527813454928\\
320.01	0.0054943710280256\\
321.01	0.00549344427114077\\
322.01	0.00549249743839581\\
323.01	0.0054915300951978\\
324.01	0.00549054179767959\\
325.01	0.0054895320925169\\
326.01	0.00548850051674458\\
327.01	0.00548744659757026\\
328.01	0.00548636985218627\\
329.01	0.00548526978758043\\
330.01	0.00548414590034619\\
331.01	0.00548299767649076\\
332.01	0.00548182459124306\\
333.01	0.00548062610886197\\
334.01	0.00547940168244379\\
335.01	0.0054781507537308\\
336.01	0.00547687275292085\\
337.01	0.00547556709847771\\
338.01	0.00547423319694501\\
339.01	0.00547287044276185\\
340.01	0.00547147821808205\\
341.01	0.00547005589259885\\
342.01	0.00546860282337389\\
343.01	0.00546711835467275\\
344.01	0.00546560181780855\\
345.01	0.00546405253099437\\
346.01	0.00546246979920531\\
347.01	0.0054608529140528\\
348.01	0.0054592011536728\\
349.01	0.00545751378262847\\
350.01	0.00545579005183191\\
351.01	0.00545402919848476\\
352.01	0.00545223044604209\\
353.01	0.00545039300420147\\
354.01	0.00544851606892138\\
355.01	0.00544659882247069\\
356.01	0.00544464043351565\\
357.01	0.00544264005724628\\
358.01	0.00544059683554877\\
359.01	0.0054385098972271\\
360.01	0.00543637835828238\\
361.01	0.00543420132225272\\
362.01	0.00543197788062396\\
363.01	0.00542970711331608\\
364.01	0.00542738808925562\\
365.01	0.00542501986704131\\
366.01	0.00542260149571503\\
367.01	0.0054201320156464\\
368.01	0.00541761045954511\\
369.01	0.00541503585361179\\
370.01	0.00541240721884241\\
371.01	0.00540972357250119\\
372.01	0.00540698392977664\\
373.01	0.00540418730563962\\
374.01	0.00540133271692067\\
375.01	0.00539841918462585\\
376.01	0.00539544573651215\\
377.01	0.00539241140994294\\
378.01	0.00538931525504518\\
379.01	0.0053861563381919\\
380.01	0.00538293374582873\\
381.01	0.00537964658867044\\
382.01	0.00537629400628247\\
383.01	0.00537287517206913\\
384.01	0.00536938929867904\\
385.01	0.0053658356438392\\
386.01	0.00536221351661703\\
387.01	0.0053585222841025\\
388.01	0.00535476137848928\\
389.01	0.00535093030451537\\
390.01	0.00534702864720414\\
391.01	0.00534305607981669\\
392.01	0.00533901237189306\\
393.01	0.00533489739721644\\
394.01	0.00533071114147749\\
395.01	0.00532645370935609\\
396.01	0.00532212533064996\\
397.01	0.00531772636498756\\
398.01	0.00531325730454725\\
399.01	0.00530871877406405\\
400.01	0.00530411152725457\\
401.01	0.00529943643860297\\
402.01	0.00529469448925469\\
403.01	0.00528988674554758\\
404.01	0.00528501432848217\\
405.01	0.00528007837222782\\
406.01	0.00527507996957568\\
407.01	0.00527002010216799\\
408.01	0.00526489955338293\\
409.01	0.00525971880207216\\
410.01	0.00525447789606453\\
411.01	0.0052491763057043\\
412.01	0.00524381275999656\\
413.01	0.00523838507168268\\
414.01	0.00523288996340713\\
415.01	0.00522732291605919\\
416.01	0.00522167807372655\\
417.01	0.00521594825942429\\
418.01	0.00521012518459015\\
419.01	0.00520419997708913\\
420.01	0.00519816522500772\\
421.01	0.00519201847717648\\
422.01	0.0051857584488522\\
423.01	0.00517938390285342\\
424.01	0.00517289365542751\\
425.01	0.00516628658248699\\
426.01	0.00515956162622112\\
427.01	0.00515271780208505\\
428.01	0.00514575420616415\\
429.01	0.00513867002290691\\
430.01	0.00513146453321292\\
431.01	0.00512413712285335\\
432.01	0.00511668729119532\\
433.01	0.00510911466018588\\
434.01	0.00510141898354174\\
435.01	0.0050936001560743\\
436.01	0.00508565822306024\\
437.01	0.00507759338954814\\
438.01	0.00506940602946684\\
439.01	0.00506109669437276\\
440.01	0.00505266612164202\\
441.01	0.00504411524187867\\
442.01	0.00503544518526552\\
443.01	0.00502665728654644\\
444.01	0.00501775308827338\\
445.01	0.00500873434190417\\
446.01	0.00499960300627798\\
447.01	0.0049903612429404\\
448.01	0.00498101140773014\\
449.01	0.00497155603798732\\
450.01	0.00496199783469326\\
451.01	0.00495233963881487\\
452.01	0.00494258440110989\\
453.01	0.00493273514465536\\
454.01	0.00492279491941333\\
455.01	0.00491276674824503\\
456.01	0.00490265356395906\\
457.01	0.00489245813724484\\
458.01	0.00488218299572643\\
459.01	0.00487183033490665\\
460.01	0.00486140192248767\\
461.01	0.00485089899848782\\
462.01	0.00484032217475886\\
463.01	0.00482967133896136\\
464.01	0.00481894556979449\\
465.01	0.00480814307225671\\
466.01	0.00479726114385588\\
467.01	0.00478629618479182\\
468.01	0.00477524376685169\\
469.01	0.00476409877648323\\
470.01	0.00475285564626067\\
471.01	0.00474150868417128\\
472.01	0.00473005249939949\\
473.01	0.00471848250288801\\
474.01	0.00470679542534797\\
475.01	0.00469498973555899\\
476.01	0.00468306554036529\\
477.01	0.00467102333681361\\
478.01	0.00465886335633616\\
479.01	0.00464658549313939\\
480.01	0.00463418925027865\\
481.01	0.00462167368557904\\
482.01	0.00460903735892477\\
483.01	0.00459627828277447\\
484.01	0.00458339387811139\\
485.01	0.00457038093836366\\
486.01	0.00455723560412103\\
487.01	0.00454395335166238\\
488.01	0.00453052899834742\\
489.01	0.00451695672772696\\
490.01	0.00450323013668929\\
491.01	0.00448934230598205\\
492.01	0.00447528589388215\\
493.01	0.00446105325052688\\
494.01	0.00444663654737016\\
495.01	0.00443202791238837\\
496.01	0.00441721955719538\\
497.01	0.0044022038776158\\
498.01	0.00438697350549951\\
499.01	0.00437152128856624\\
500.01	0.00435584018016472\\
501.01	0.00433992303804354\\
502.01	0.00432376239648625\\
503.01	0.00430735037702502\\
504.01	0.00429067870800565\\
505.01	0.00427373876057116\\
506.01	0.00425652159158788\\
507.01	0.00423901799212023\\
508.01	0.00422121853940893\\
509.01	0.00420311364962767\\
510.01	0.00418469362806572\\
511.01	0.00416594871292255\\
512.01	0.00414686910874527\\
513.01	0.00412744500587848\\
514.01	0.00410766658334731\\
515.01	0.0040875239945441\\
516.01	0.00406700733806003\\
517.01	0.00404610661987399\\
518.01	0.00402481171727932\\
519.01	0.00400311235781672\\
520.01	0.00398099812237176\\
521.01	0.00395845846277685\\
522.01	0.00393548271928023\\
523.01	0.0039120601342729\\
524.01	0.00388817986146693\\
525.01	0.00386383097008843\\
526.01	0.00383900244409569\\
527.01	0.00381368317698033\\
528.01	0.00378786196329684\\
529.01	0.0037615274885997\\
530.01	0.00373466831978622\\
531.01	0.00370727289777221\\
532.01	0.00367932953377338\\
533.01	0.00365082640913506\\
534.01	0.00362175157696972\\
535.01	0.00359209296338246\\
536.01	0.00356183836755163\\
537.01	0.00353097546093829\\
538.01	0.00349949178610511\\
539.01	0.00346737475569416\\
540.01	0.00343461165211991\\
541.01	0.00340118962845988\\
542.01	0.00336709571087169\\
543.01	0.00333231680265383\\
544.01	0.00329683968985111\\
545.01	0.0032606510481884\\
546.01	0.00322373745118108\\
547.01	0.00318608537950776\\
548.01	0.00314768123191225\\
549.01	0.00310851133792719\\
550.01	0.00306856197269586\\
551.01	0.0030278193741351\\
552.01	0.00298626976264715\\
553.01	0.00294389936356977\\
554.01	0.0029006944325494\\
555.01	0.0028566412840523\\
556.01	0.00281172632328039\\
557.01	0.00276593608181439\\
558.01	0.00271925725734165\\
559.01	0.0026716767578422\\
560.01	0.00262318175062\\
561.01	0.00257375971658235\\
562.01	0.0025233985101925\\
563.01	0.00247208642554797\\
564.01	0.0024198122690649\\
565.01	0.00236656543928021\\
566.01	0.00231233601430398\\
567.01	0.00225711484746698\\
568.01	0.0022008936717135\\
569.01	0.00214366521328238\\
570.01	0.00208542331519992\\
571.01	0.0020261630710739\\
572.01	0.0019658809696171\\
573.01	0.00190457505024007\\
574.01	0.00184224506991961\\
575.01	0.00177889268136865\\
576.01	0.00171452162228293\\
577.01	0.00164913791511233\\
578.01	0.00158275007636962\\
579.01	0.00151536933392877\\
580.01	0.00144700985004043\\
581.01	0.0013776889468657\\
582.01	0.0013074273301515\\
583.01	0.00123624930517962\\
584.01	0.00116418297724289\\
585.01	0.00109126042654351\\
586.01	0.00101751784445324\\
587.01	0.000942995614387319\\
588.01	0.000867738315945504\\
589.01	0.000791794625255108\\
590.01	0.00071521707734773\\
591.01	0.000638061647592614\\
592.01	0.000560387098297354\\
593.01	0.000482254023083268\\
594.01	0.000403723504947367\\
595.01	0.000324855283297511\\
596.01	0.000245705299785164\\
597.01	0.000166480802755067\\
598.01	9.17366970063765e-05\\
599.01	2.94669364271135e-05\\
599.02	2.89574269585705e-05\\
599.03	2.84509530852801e-05\\
599.04	2.79475443462508e-05\\
599.05	2.7447230571279e-05\\
599.06	2.69500418838293e-05\\
599.07	2.64560087039432e-05\\
599.08	2.59651617511673e-05\\
599.09	2.54775320475218e-05\\
599.1	2.49931509204836e-05\\
599.11	2.45120500060158e-05\\
599.12	2.40342612516167e-05\\
599.13	2.35598169193978e-05\\
599.14	2.30887495892024e-05\\
599.15	2.26210921617405e-05\\
599.16	2.215687786177e-05\\
599.17	2.16961402412993e-05\\
599.18	2.12389131828191e-05\\
599.19	2.07852309025806e-05\\
599.2	2.03351279538921e-05\\
599.21	1.98886392304542e-05\\
599.22	1.94457999697188e-05\\
599.23	1.90066457563046e-05\\
599.24	1.85712125254211e-05\\
599.25	1.81395365663334e-05\\
599.26	1.77116564095865e-05\\
599.27	1.72876122135658e-05\\
599.28	1.68674445361946e-05\\
599.29	1.64511943388859e-05\\
599.3	1.60389029905273e-05\\
599.31	1.56306122715087e-05\\
599.32	1.52263643777902e-05\\
599.33	1.48262019250087e-05\\
599.34	1.44301679526268e-05\\
599.35	1.40383059281154e-05\\
599.36	1.36506597511916e-05\\
599.37	1.32672737580847e-05\\
599.38	1.28881927258535e-05\\
599.39	1.25134618767404e-05\\
599.4	1.21431268825696e-05\\
599.41	1.17772338691924e-05\\
599.42	1.14158294209719e-05\\
599.43	1.10589605853174e-05\\
599.44	1.0706674877254e-05\\
599.45	1.03590202840433e-05\\
599.46	1.00160452698589e-05\\
599.47	9.67779878049101e-06\\
599.48	9.34433024810284e-06\\
599.49	9.0156895960411e-06\\
599.5	8.69192724369146e-06\\
599.51	8.37309411137396e-06\\
599.52	8.05924162529392e-06\\
599.53	7.75042172253965e-06\\
599.54	7.44668685613396e-06\\
599.55	7.14809000013084e-06\\
599.56	6.85468465475708e-06\\
599.57	6.56652485161308e-06\\
599.58	6.2836651589307e-06\\
599.59	6.00616068686249e-06\\
599.6	5.7340670928472e-06\\
599.61	5.46744058700296e-06\\
599.62	5.20633793760217e-06\\
599.63	4.95081647657741e-06\\
599.64	4.70093410508653e-06\\
599.65	4.45674929914347e-06\\
599.66	4.21832111529089e-06\\
599.67	3.98570919634376e-06\\
599.68	3.75897377716608e-06\\
599.69	3.53817569053415e-06\\
599.7	3.32337637303469e-06\\
599.71	3.11463787102881e-06\\
599.72	2.91202284668536e-06\\
599.73	2.71559458404555e-06\\
599.74	2.52541699518466e-06\\
599.75	2.34155462640155e-06\\
599.76	2.16407266449489e-06\\
599.77	1.99303694307928e-06\\
599.78	1.82851394897598e-06\\
599.79	1.67057082867302e-06\\
599.8	1.51927539483211e-06\\
599.81	1.37469613287027e-06\\
599.82	1.23690220760718e-06\\
599.83	1.10596346997172e-06\\
599.84	9.81950463782924e-07\\
599.85	8.64934432591793e-07\\
599.86	7.54987326587186e-07\\
599.87	6.5218180958504e-07\\
599.88	5.56591266062667e-07\\
599.89	4.68289808295413e-07\\
599.9	3.87352283521061e-07\\
599.91	3.13854281216996e-07\\
599.92	2.47872140425945e-07\\
599.93	1.89482957149364e-07\\
599.94	1.38764591834512e-07\\
599.95	9.57956769222224e-08\\
599.96	6.06556244623496e-08\\
599.97	3.34246338211386e-08\\
599.98	1.4183699454523e-08\\
599.99	3.01461875948372e-09\\
600	0\\
};
\addplot [color=mycolor21,solid,forget plot]
  table[row sep=crcr]{%
0.01	0.00535026209346621\\
1.01	0.00535026104829641\\
2.01	0.00535025998105595\\
3.01	0.00535025889127814\\
4.01	0.0053502577784865\\
5.01	0.00535025664219418\\
6.01	0.00535025548190425\\
7.01	0.00535025429710914\\
8.01	0.00535025308729064\\
9.01	0.00535025185191931\\
10.01	0.00535025059045471\\
11.01	0.0053502493023449\\
12.01	0.00535024798702625\\
13.01	0.00535024664392316\\
14.01	0.00535024527244783\\
15.01	0.0053502438720001\\
16.01	0.00535024244196682\\
17.01	0.00535024098172227\\
18.01	0.005350239490627\\
19.01	0.00535023796802848\\
20.01	0.00535023641325996\\
21.01	0.0053502348256408\\
22.01	0.00535023320447572\\
23.01	0.00535023154905479\\
24.01	0.00535022985865317\\
25.01	0.00535022813253049\\
26.01	0.00535022636993082\\
27.01	0.00535022457008205\\
28.01	0.00535022273219562\\
29.01	0.00535022085546657\\
30.01	0.00535021893907261\\
31.01	0.00535021698217404\\
32.01	0.0053502149839134\\
33.01	0.00535021294341511\\
34.01	0.005350210859785\\
35.01	0.00535020873210986\\
36.01	0.00535020655945719\\
37.01	0.00535020434087453\\
38.01	0.0053502020753894\\
39.01	0.00535019976200858\\
40.01	0.00535019739971781\\
41.01	0.00535019498748146\\
42.01	0.00535019252424177\\
43.01	0.00535019000891849\\
44.01	0.00535018744040847\\
45.01	0.00535018481758538\\
46.01	0.0053501821392988\\
47.01	0.00535017940437382\\
48.01	0.00535017661161067\\
49.01	0.0053501737597842\\
50.01	0.00535017084764323\\
51.01	0.00535016787390984\\
52.01	0.00535016483727916\\
53.01	0.00535016173641865\\
54.01	0.00535015856996733\\
55.01	0.00535015533653546\\
56.01	0.00535015203470388\\
57.01	0.00535014866302299\\
58.01	0.00535014522001257\\
59.01	0.00535014170416103\\
60.01	0.0053501381139247\\
61.01	0.00535013444772711\\
62.01	0.00535013070395823\\
63.01	0.00535012688097388\\
64.01	0.00535012297709494\\
65.01	0.00535011899060661\\
66.01	0.00535011491975781\\
67.01	0.00535011076276013\\
68.01	0.00535010651778722\\
69.01	0.00535010218297388\\
70.01	0.00535009775641527\\
71.01	0.00535009323616632\\
72.01	0.00535008862024022\\
73.01	0.00535008390660834\\
74.01	0.0053500790931987\\
75.01	0.00535007417789551\\
76.01	0.00535006915853775\\
77.01	0.00535006403291869\\
78.01	0.00535005879878457\\
79.01	0.00535005345383386\\
80.01	0.00535004799571588\\
81.01	0.00535004242203036\\
82.01	0.00535003673032574\\
83.01	0.00535003091809843\\
84.01	0.00535002498279168\\
85.01	0.00535001892179418\\
86.01	0.0053500127324394\\
87.01	0.00535000641200381\\
88.01	0.00534999995770629\\
89.01	0.00534999336670627\\
90.01	0.00534998663610307\\
91.01	0.0053499797629341\\
92.01	0.0053499727441738\\
93.01	0.00534996557673259\\
94.01	0.00534995825745473\\
95.01	0.00534995078311778\\
96.01	0.00534994315043042\\
97.01	0.00534993535603153\\
98.01	0.00534992739648842\\
99.01	0.00534991926829532\\
100.01	0.00534991096787215\\
101.01	0.00534990249156239\\
102.01	0.00534989383563186\\
103.01	0.00534988499626685\\
104.01	0.00534987596957271\\
105.01	0.00534986675157164\\
106.01	0.00534985733820153\\
107.01	0.00534984772531359\\
108.01	0.00534983790867085\\
109.01	0.00534982788394607\\
110.01	0.00534981764672022\\
111.01	0.0053498071924799\\
112.01	0.00534979651661583\\
113.01	0.00534978561442046\\
114.01	0.00534977448108632\\
115.01	0.00534976311170351\\
116.01	0.00534975150125743\\
117.01	0.00534973964462699\\
118.01	0.00534972753658188\\
119.01	0.00534971517178064\\
120.01	0.00534970254476804\\
121.01	0.00534968964997274\\
122.01	0.00534967648170465\\
123.01	0.00534966303415268\\
124.01	0.00534964930138204\\
125.01	0.00534963527733151\\
126.01	0.00534962095581067\\
127.01	0.00534960633049756\\
128.01	0.00534959139493537\\
129.01	0.00534957614252992\\
130.01	0.00534956056654668\\
131.01	0.00534954466010754\\
132.01	0.00534952841618806\\
133.01	0.00534951182761399\\
134.01	0.00534949488705868\\
135.01	0.00534947758703889\\
136.01	0.00534945991991232\\
137.01	0.00534944187787386\\
138.01	0.00534942345295226\\
139.01	0.00534940463700621\\
140.01	0.00534938542172121\\
141.01	0.00534936579860556\\
142.01	0.00534934575898676\\
143.01	0.00534932529400738\\
144.01	0.00534930439462162\\
145.01	0.00534928305159063\\
146.01	0.00534926125547865\\
147.01	0.0053492389966496\\
148.01	0.00534921626526133\\
149.01	0.00534919305126237\\
150.01	0.00534916934438704\\
151.01	0.00534914513415085\\
152.01	0.00534912040984623\\
153.01	0.00534909516053715\\
154.01	0.00534906937505456\\
155.01	0.00534904304199172\\
156.01	0.00534901614969872\\
157.01	0.00534898868627732\\
158.01	0.0053489606395759\\
159.01	0.00534893199718371\\
160.01	0.00534890274642584\\
161.01	0.00534887287435704\\
162.01	0.00534884236775645\\
163.01	0.00534881121312145\\
164.01	0.0053487793966615\\
165.01	0.00534874690429258\\
166.01	0.0053487137216302\\
167.01	0.00534867983398361\\
168.01	0.0053486452263491\\
169.01	0.00534860988340292\\
170.01	0.00534857378949483\\
171.01	0.00534853692864163\\
172.01	0.00534849928451919\\
173.01	0.00534846084045555\\
174.01	0.0053484215794237\\
175.01	0.00534838148403353\\
176.01	0.00534834053652487\\
177.01	0.00534829871875869\\
178.01	0.00534825601220972\\
179.01	0.00534821239795825\\
180.01	0.00534816785668118\\
181.01	0.00534812236864375\\
182.01	0.00534807591369094\\
183.01	0.00534802847123837\\
184.01	0.00534798002026301\\
185.01	0.00534793053929424\\
186.01	0.00534788000640404\\
187.01	0.00534782839919714\\
188.01	0.00534777569480136\\
189.01	0.00534772186985746\\
190.01	0.00534766690050877\\
191.01	0.00534761076239052\\
192.01	0.00534755343061893\\
193.01	0.00534749487978045\\
194.01	0.0053474350839205\\
195.01	0.0053473740165318\\
196.01	0.00534731165054265\\
197.01	0.00534724795830508\\
198.01	0.00534718291158238\\
199.01	0.00534711648153705\\
200.01	0.00534704863871773\\
201.01	0.00534697935304602\\
202.01	0.00534690859380354\\
203.01	0.00534683632961815\\
204.01	0.00534676252845023\\
205.01	0.00534668715757838\\
206.01	0.00534661018358502\\
207.01	0.00534653157234164\\
208.01	0.00534645128899402\\
209.01	0.0053463692979464\\
210.01	0.00534628556284588\\
211.01	0.00534620004656689\\
212.01	0.00534611271119413\\
213.01	0.00534602351800639\\
214.01	0.00534593242745903\\
215.01	0.00534583939916739\\
216.01	0.00534574439188814\\
217.01	0.00534564736350164\\
218.01	0.00534554827099344\\
219.01	0.00534544707043504\\
220.01	0.00534534371696496\\
221.01	0.00534523816476892\\
222.01	0.0053451303670598\\
223.01	0.00534502027605716\\
224.01	0.0053449078429661\\
225.01	0.00534479301795617\\
226.01	0.00534467575013941\\
227.01	0.00534455598754815\\
228.01	0.0053444336771125\\
229.01	0.00534430876463645\\
230.01	0.00534418119477529\\
231.01	0.00534405091101063\\
232.01	0.00534391785562617\\
233.01	0.00534378196968224\\
234.01	0.00534364319299068\\
235.01	0.00534350146408818\\
236.01	0.00534335672020978\\
237.01	0.00534320889726148\\
238.01	0.00534305792979283\\
239.01	0.00534290375096808\\
240.01	0.00534274629253751\\
241.01	0.00534258548480774\\
242.01	0.00534242125661182\\
243.01	0.00534225353527838\\
244.01	0.00534208224660047\\
245.01	0.00534190731480294\\
246.01	0.00534172866251073\\
247.01	0.00534154621071516\\
248.01	0.00534135987873995\\
249.01	0.00534116958420659\\
250.01	0.00534097524299939\\
251.01	0.00534077676922931\\
252.01	0.00534057407519673\\
253.01	0.00534036707135504\\
254.01	0.00534015566627153\\
255.01	0.0053399397665895\\
256.01	0.00533971927698778\\
257.01	0.00533949410014039\\
258.01	0.00533926413667565\\
259.01	0.00533902928513409\\
260.01	0.00533878944192549\\
261.01	0.00533854450128538\\
262.01	0.00533829435523067\\
263.01	0.00533803889351393\\
264.01	0.0053377780035775\\
265.01	0.00533751157050692\\
266.01	0.00533723947698208\\
267.01	0.00533696160322902\\
268.01	0.00533667782697017\\
269.01	0.0053363880233736\\
270.01	0.00533609206500142\\
271.01	0.00533578982175755\\
272.01	0.00533548116083401\\
273.01	0.00533516594665643\\
274.01	0.00533484404082944\\
275.01	0.00533451530207933\\
276.01	0.00533417958619711\\
277.01	0.00533383674598056\\
278.01	0.00533348663117434\\
279.01	0.00533312908841\\
280.01	0.00533276396114464\\
281.01	0.00533239108959815\\
282.01	0.00533201031069036\\
283.01	0.00533162145797615\\
284.01	0.00533122436158038\\
285.01	0.00533081884813113\\
286.01	0.00533040474069229\\
287.01	0.00532998185869494\\
288.01	0.00532955001786797\\
289.01	0.00532910903016721\\
290.01	0.0053286587037039\\
291.01	0.00532819884267224\\
292.01	0.00532772924727555\\
293.01	0.00532724971365193\\
294.01	0.0053267600337985\\
295.01	0.00532625999549482\\
296.01	0.00532574938222581\\
297.01	0.00532522797310312\\
298.01	0.00532469554278615\\
299.01	0.00532415186140171\\
300.01	0.00532359669446365\\
301.01	0.00532302980279103\\
302.01	0.00532245094242565\\
303.01	0.00532185986454963\\
304.01	0.00532125631540115\\
305.01	0.00532064003619124\\
306.01	0.00532001076301838\\
307.01	0.00531936822678417\\
308.01	0.005318712153108\\
309.01	0.00531804226224141\\
310.01	0.00531735826898274\\
311.01	0.00531665988259147\\
312.01	0.00531594680670288\\
313.01	0.00531521873924296\\
314.01	0.00531447537234359\\
315.01	0.0053137163922588\\
316.01	0.00531294147928049\\
317.01	0.00531215030765688\\
318.01	0.00531134254551053\\
319.01	0.00531051785475856\\
320.01	0.00530967589103442\\
321.01	0.00530881630361108\\
322.01	0.00530793873532671\\
323.01	0.00530704282251277\\
324.01	0.00530612819492473\\
325.01	0.00530519447567572\\
326.01	0.00530424128117427\\
327.01	0.00530326822106537\\
328.01	0.00530227489817605\\
329.01	0.00530126090846623\\
330.01	0.00530022584098391\\
331.01	0.00529916927782694\\
332.01	0.00529809079411157\\
333.01	0.00529698995794719\\
334.01	0.00529586633041939\\
335.01	0.00529471946558195\\
336.01	0.00529354891045676\\
337.01	0.00529235420504582\\
338.01	0.00529113488235237\\
339.01	0.00528989046841502\\
340.01	0.00528862048235526\\
341.01	0.00528732443643789\\
342.01	0.00528600183614752\\
343.01	0.00528465218028151\\
344.01	0.00528327496106036\\
345.01	0.0052818696642579\\
346.01	0.00528043576935145\\
347.01	0.00527897274969461\\
348.01	0.0052774800727138\\
349.01	0.00527595720013088\\
350.01	0.0052744035882121\\
351.01	0.00527281868804799\\
352.01	0.00527120194586318\\
353.01	0.00526955280336065\\
354.01	0.00526787069810039\\
355.01	0.00526615506391757\\
356.01	0.00526440533137949\\
357.01	0.0052626209282854\\
358.01	0.00526080128021128\\
359.01	0.00525894581110196\\
360.01	0.00525705394391194\\
361.01	0.00525512510129937\\
362.01	0.00525315870637325\\
363.01	0.0052511541834974\\
364.01	0.00524911095915187\\
365.01	0.00524702846285501\\
366.01	0.00524490612814607\\
367.01	0.00524274339362976\\
368.01	0.00524053970408339\\
369.01	0.00523829451162582\\
370.01	0.00523600727694703\\
371.01	0.00523367747059659\\
372.01	0.00523130457432675\\
373.01	0.00522888808248597\\
374.01	0.00522642750345643\\
375.01	0.0052239223611262\\
376.01	0.00522137219638459\\
377.01	0.00521877656862848\\
378.01	0.00521613505726135\\
379.01	0.00521344726316258\\
380.01	0.00521071281010516\\
381.01	0.0052079313460878\\
382.01	0.00520510254454846\\
383.01	0.00520222610541506\\
384.01	0.00519930175594483\\
385.01	0.00519632925129438\\
386.01	0.00519330837475483\\
387.01	0.00519023893757661\\
388.01	0.00518712077829755\\
389.01	0.00518395376147891\\
390.01	0.00518073777574154\\
391.01	0.00517747273098405\\
392.01	0.00517415855465568\\
393.01	0.00517079518694496\\
394.01	0.00516738257474272\\
395.01	0.00516392066422929\\
396.01	0.00516040939194155\\
397.01	0.00515684867418499\\
398.01	0.00515323839467018\\
399.01	0.00514957839029082\\
400.01	0.00514586843500483\\
401.01	0.0051421082218543\\
402.01	0.00513829734325826\\
403.01	0.0051344352698442\\
404.01	0.00513052132825987\\
405.01	0.00512655467862695\\
406.01	0.00512253429258049\\
407.01	0.00511845893316631\\
408.01	0.00511432713827346\\
409.01	0.00511013720971888\\
410.01	0.00510588721058433\\
411.01	0.00510157497386776\\
412.01	0.00509719812589269\\
413.01	0.00509275412807227\\
414.01	0.00508824034035754\\
415.01	0.00508365410866688\\
416.01	0.00507899287629741\\
417.01	0.00507425431498538\\
418.01	0.00506943646375203\\
419.01	0.00506453785125295\\
420.01	0.00505955753787001\\
421.01	0.00505449487564457\\
422.01	0.0050493492743097\\
423.01	0.00504412018361756\\
424.01	0.00503880709561506\\
425.01	0.00503340954685702\\
426.01	0.00502792712052649\\
427.01	0.00502235944842922\\
428.01	0.00501670621282421\\
429.01	0.00501096714804755\\
430.01	0.00500514204188019\\
431.01	0.00499923073660854\\
432.01	0.00499323312971387\\
433.01	0.00498714917412894\\
434.01	0.00498097887798744\\
435.01	0.00497472230378842\\
436.01	0.00496837956689141\\
437.01	0.00496195083325152\\
438.01	0.00495543631629968\\
439.01	0.0049488362728684\\
440.01	0.00494215099806004\\
441.01	0.00493538081895403\\
442.01	0.00492852608705271\\
443.01	0.0049215871693661\\
444.01	0.00491456443804755\\
445.01	0.0049074582585057\\
446.01	0.00490026897593585\\
447.01	0.00489299690024187\\
448.01	0.00488564228935582\\
449.01	0.00487820533100495\\
450.01	0.00487068612303634\\
451.01	0.00486308465247448\\
452.01	0.00485540077357311\\
453.01	0.00484763418522\\
454.01	0.00483978440816255\\
455.01	0.00483185076265124\\
456.01	0.00482383234723436\\
457.01	0.00481572801957892\\
458.01	0.00480753638033996\\
459.01	0.00479925576123245\\
460.01	0.00479088421856735\\
461.01	0.00478241953357406\\
462.01	0.00477385922082033\\
463.01	0.0047652005459163\\
464.01	0.00475644055341732\\
465.01	0.00474757610536369\\
466.01	0.00473860393016235\\
467.01	0.00472952068047446\\
468.01	0.00472032299737828\\
469.01	0.0047110075763275\\
470.01	0.00470157122838082\\
471.01	0.00469201092801985\\
472.01	0.00468232383700174\\
473.01	0.0046725072928645\\
474.01	0.00466255875221294\\
475.01	0.00465247568499771\\
476.01	0.00464225543447259\\
477.01	0.00463189511510558\\
478.01	0.0046213915849398\\
479.01	0.00461074143567458\\
480.01	0.00459994098549038\\
481.01	0.00458898627489505\\
482.01	0.00457787306603472\\
483.01	0.00456659684584823\\
484.01	0.00455515283335924\\
485.01	0.00454353599126477\\
486.01	0.00453174104179636\\
487.01	0.00451976248659717\\
488.01	0.0045075946300765\\
489.01	0.00449523160536918\\
490.01	0.00448266740167535\\
491.01	0.00446989589138234\\
492.01	0.00445691085505024\\
493.01	0.00444370600210405\\
494.01	0.00443027498501948\\
495.01	0.00441661140500323\\
496.01	0.00440270880777203\\
497.01	0.00438856066912667\\
498.01	0.00437416037167617\\
499.01	0.00435950117623876\\
500.01	0.0043445761938544\\
501.01	0.00432937836627371\\
502.01	0.00431390046205232\\
503.01	0.00429813508615042\\
504.01	0.00428207469322905\\
505.01	0.00426571160085169\\
506.01	0.00424903800183912\\
507.01	0.0042320459751404\\
508.01	0.00421472749460594\\
509.01	0.00419707443513539\\
510.01	0.00417907857582396\\
511.01	0.0041607315999603\\
512.01	0.0041420250920226\\
513.01	0.00412295053216629\\
514.01	0.00410349928904926\\
515.01	0.00408366261213326\\
516.01	0.00406343162474818\\
517.01	0.00404279731908349\\
518.01	0.00402175055378084\\
519.01	0.00400028205391089\\
520.01	0.00397838241203012\\
521.01	0.00395604208873859\\
522.01	0.00393325141211583\\
523.01	0.00391000057605093\\
524.01	0.00388627963762727\\
525.01	0.00386207851380362\\
526.01	0.00383738697769382\\
527.01	0.00381219465478078\\
528.01	0.00378649101939319\\
529.01	0.00376026539170914\\
530.01	0.00373350693544559\\
531.01	0.00370620465624647\\
532.01	0.00367834740064358\\
533.01	0.00364992385538304\\
534.01	0.00362092254694498\\
535.01	0.00359133184124472\\
536.01	0.00356113994363432\\
537.01	0.00353033489935747\\
538.01	0.00349890459460845\\
539.01	0.00346683675832488\\
540.01	0.00343411896482426\\
541.01	0.00340073863736593\\
542.01	0.00336668305270029\\
543.01	0.0033319393466588\\
544.01	0.00329649452084877\\
545.01	0.00326033545053969\\
546.01	0.00322344889386798\\
547.01	0.00318582150251138\\
548.01	0.00314743983399592\\
549.01	0.00310829036580649\\
550.01	0.00306835951147481\\
551.01	0.0030276336388282\\
552.01	0.00298609909059895\\
553.01	0.00294374220760762\\
554.01	0.00290054935475943\\
555.01	0.0028565069501179\\
556.01	0.00281160149734305\\
557.01	0.0027658196218078\\
558.01	0.0027191481107284\\
559.01	0.00267157395766957\\
560.01	0.00262308441180904\\
561.01	0.00257366703237543\\
562.01	0.00252330974869932\\
563.01	0.00247200092634584\\
564.01	0.00241972943982337\\
565.01	0.00236648475238619\\
566.01	0.00231225700346725\\
567.01	0.00225703710429318\\
568.01	0.00220081684223704\\
569.01	0.00214358899446181\\
570.01	0.00208534745138918\\
571.01	0.00202608735049004\\
572.01	0.00196580522083402\\
573.01	0.00190449913874023\\
574.01	0.0018421688947402\\
575.01	0.00177881617187809\\
576.01	0.00171444473512429\\
577.01	0.00164906063134602\\
578.01	0.00158267239884402\\
579.01	0.00151529128489964\\
580.01	0.00144693146904958\\
581.01	0.00137761028887722\\
582.01	0.00130734846393035\\
583.01	0.00123617031188133\\
584.01	0.00116410394916785\\
585.01	0.00109118146599105\\
586.01	0.00101743906259415\\
587.01	0.000942917130054623\\
588.01	0.00086766025422433\\
589.01	0.000791717115732183\\
590.01	0.000715140251857945\\
591.01	0.000637985637273675\\
592.01	0.00056031202973124\\
593.01	0.000482180013262905\\
594.01	0.000403650654755804\\
595.01	0.00032478366912282\\
596.01	0.000245634962812777\\
597.01	0.000166444281554918\\
598.01	9.17366970063782e-05\\
599.01	2.94669364271135e-05\\
599.02	2.89574269585688e-05\\
599.03	2.84509530852819e-05\\
599.04	2.79475443462525e-05\\
599.05	2.7447230571279e-05\\
599.06	2.69500418838293e-05\\
599.07	2.64560087039432e-05\\
599.08	2.59651617511691e-05\\
599.09	2.54775320475235e-05\\
599.1	2.49931509204854e-05\\
599.11	2.45120500060175e-05\\
599.12	2.40342612516167e-05\\
599.13	2.35598169193996e-05\\
599.14	2.30887495892042e-05\\
599.15	2.26210921617422e-05\\
599.16	2.215687786177e-05\\
599.17	2.16961402412976e-05\\
599.18	2.12389131828191e-05\\
599.19	2.07852309025824e-05\\
599.2	2.03351279538938e-05\\
599.21	1.98886392304542e-05\\
599.22	1.94457999697188e-05\\
599.23	1.90066457563063e-05\\
599.24	1.85712125254194e-05\\
599.25	1.81395365663334e-05\\
599.26	1.77116564095883e-05\\
599.27	1.72876122135658e-05\\
599.28	1.68674445361946e-05\\
599.29	1.64511943388859e-05\\
599.3	1.60389029905273e-05\\
599.31	1.56306122715104e-05\\
599.32	1.5226364377792e-05\\
599.33	1.48262019250105e-05\\
599.34	1.44301679526285e-05\\
599.35	1.40383059281154e-05\\
599.36	1.36506597511916e-05\\
599.37	1.32672737580847e-05\\
599.38	1.28881927258552e-05\\
599.39	1.25134618767404e-05\\
599.4	1.21431268825696e-05\\
599.41	1.17772338691924e-05\\
599.42	1.14158294209736e-05\\
599.43	1.10589605853174e-05\\
599.44	1.07066748772523e-05\\
599.45	1.03590202840433e-05\\
599.46	1.00160452698606e-05\\
599.47	9.67779878049101e-06\\
599.48	9.34433024810284e-06\\
599.49	9.01568959604283e-06\\
599.5	8.69192724369319e-06\\
599.51	8.37309411137396e-06\\
599.52	8.05924162529219e-06\\
599.53	7.75042172253965e-06\\
599.54	7.4466868561357e-06\\
599.55	7.14809000013084e-06\\
599.56	6.85468465475535e-06\\
599.57	6.56652485161308e-06\\
599.58	6.28366515892896e-06\\
599.59	6.00616068686249e-06\\
599.6	5.7340670928472e-06\\
599.61	5.46744058700296e-06\\
599.62	5.20633793760217e-06\\
599.63	4.95081647657741e-06\\
599.64	4.70093410508653e-06\\
599.65	4.45674929914347e-06\\
599.66	4.21832111529262e-06\\
599.67	3.98570919634203e-06\\
599.68	3.75897377716435e-06\\
599.69	3.53817569053415e-06\\
599.7	3.32337637303295e-06\\
599.71	3.11463787103054e-06\\
599.72	2.91202284668363e-06\\
599.73	2.71559458404382e-06\\
599.74	2.52541699518292e-06\\
599.75	2.34155462640329e-06\\
599.76	2.16407266449489e-06\\
599.77	1.99303694307928e-06\\
599.78	1.82851394897598e-06\\
599.79	1.67057082867302e-06\\
599.8	1.51927539483211e-06\\
599.81	1.37469613287027e-06\\
599.82	1.23690220760718e-06\\
599.83	1.10596346997172e-06\\
599.84	9.81950463782924e-07\\
599.85	8.64934432591793e-07\\
599.86	7.54987326588921e-07\\
599.87	6.5218180958504e-07\\
599.88	5.56591266064402e-07\\
599.89	4.68289808295413e-07\\
599.9	3.87352283521061e-07\\
599.91	3.13854281218731e-07\\
599.92	2.4787214042421e-07\\
599.93	1.8948295714763e-07\\
599.94	1.38764591834512e-07\\
599.95	9.57956769204876e-08\\
599.96	6.06556244606149e-08\\
599.97	3.34246338194039e-08\\
599.98	1.41836994527883e-08\\
599.99	3.01461875948372e-09\\
600	0\\
};
\addplot [color=black!20!mycolor21,solid,forget plot]
  table[row sep=crcr]{%
0.01	0.00523138383269897\\
1.01	0.005231382805886\\
2.01	0.00523138175751092\\
3.01	0.0052313806871207\\
4.01	0.00523137959425308\\
5.01	0.00523137847843602\\
6.01	0.0052313773391874\\
7.01	0.0052313761760151\\
8.01	0.00523137498841659\\
9.01	0.00523137377587905\\
10.01	0.00523137253787854\\
11.01	0.00523137127388031\\
12.01	0.00523136998333838\\
13.01	0.00523136866569534\\
14.01	0.00523136732038186\\
15.01	0.00523136594681675\\
16.01	0.00523136454440681\\
17.01	0.00523136311254593\\
18.01	0.00523136165061591\\
19.01	0.00523136015798483\\
20.01	0.00523135863400801\\
21.01	0.0052313570780272\\
22.01	0.00523135548937016\\
23.01	0.00523135386735057\\
24.01	0.00523135221126765\\
25.01	0.00523135052040608\\
26.01	0.00523134879403519\\
27.01	0.00523134703140926\\
28.01	0.00523134523176695\\
29.01	0.00523134339433056\\
30.01	0.00523134151830637\\
31.01	0.00523133960288403\\
32.01	0.00523133764723592\\
33.01	0.00523133565051707\\
34.01	0.00523133361186495\\
35.01	0.00523133153039868\\
36.01	0.00523132940521904\\
37.01	0.00523132723540796\\
38.01	0.00523132502002802\\
39.01	0.005231322758122\\
40.01	0.00523132044871281\\
41.01	0.0052313180908025\\
42.01	0.00523131568337245\\
43.01	0.0052313132253825\\
44.01	0.00523131071577083\\
45.01	0.00523130815345305\\
46.01	0.00523130553732208\\
47.01	0.0052313028662477\\
48.01	0.00523130013907582\\
49.01	0.00523129735462829\\
50.01	0.00523129451170188\\
51.01	0.0052312916090685\\
52.01	0.00523128864547392\\
53.01	0.00523128561963773\\
54.01	0.00523128253025265\\
55.01	0.00523127937598381\\
56.01	0.00523127615546846\\
57.01	0.00523127286731504\\
58.01	0.0052312695101031\\
59.01	0.00523126608238191\\
60.01	0.00523126258267048\\
61.01	0.00523125900945685\\
62.01	0.00523125536119704\\
63.01	0.00523125163631499\\
64.01	0.00523124783320103\\
65.01	0.00523124395021218\\
66.01	0.00523123998567065\\
67.01	0.00523123593786356\\
68.01	0.00523123180504193\\
69.01	0.00523122758542023\\
70.01	0.0052312232771752\\
71.01	0.00523121887844556\\
72.01	0.00523121438733066\\
73.01	0.00523120980189001\\
74.01	0.00523120512014266\\
75.01	0.00523120034006559\\
76.01	0.00523119545959362\\
77.01	0.00523119047661807\\
78.01	0.00523118538898599\\
79.01	0.00523118019449925\\
80.01	0.00523117489091362\\
81.01	0.0052311694759376\\
82.01	0.00523116394723152\\
83.01	0.0052311583024069\\
84.01	0.00523115253902459\\
85.01	0.00523114665459489\\
86.01	0.00523114064657517\\
87.01	0.00523113451236992\\
88.01	0.00523112824932879\\
89.01	0.00523112185474595\\
90.01	0.00523111532585859\\
91.01	0.00523110865984609\\
92.01	0.00523110185382852\\
93.01	0.00523109490486543\\
94.01	0.00523108780995487\\
95.01	0.00523108056603126\\
96.01	0.00523107316996528\\
97.01	0.00523106561856186\\
98.01	0.0052310579085586\\
99.01	0.00523105003662475\\
100.01	0.00523104199935942\\
101.01	0.00523103379329062\\
102.01	0.00523102541487337\\
103.01	0.00523101686048825\\
104.01	0.00523100812643983\\
105.01	0.00523099920895525\\
106.01	0.00523099010418216\\
107.01	0.00523098080818766\\
108.01	0.00523097131695629\\
109.01	0.00523096162638815\\
110.01	0.00523095173229713\\
111.01	0.00523094163040974\\
112.01	0.00523093131636236\\
113.01	0.00523092078570006\\
114.01	0.0052309100338744\\
115.01	0.00523089905624127\\
116.01	0.00523088784805948\\
117.01	0.00523087640448797\\
118.01	0.00523086472058439\\
119.01	0.00523085279130236\\
120.01	0.00523084061149009\\
121.01	0.00523082817588724\\
122.01	0.00523081547912343\\
123.01	0.00523080251571542\\
124.01	0.00523078928006506\\
125.01	0.00523077576645655\\
126.01	0.00523076196905461\\
127.01	0.00523074788190097\\
128.01	0.00523073349891317\\
129.01	0.00523071881388052\\
130.01	0.00523070382046229\\
131.01	0.00523068851218465\\
132.01	0.00523067288243826\\
133.01	0.00523065692447512\\
134.01	0.0052306406314054\\
135.01	0.00523062399619537\\
136.01	0.00523060701166355\\
137.01	0.00523058967047768\\
138.01	0.00523057196515207\\
139.01	0.00523055388804402\\
140.01	0.00523053543135064\\
141.01	0.0052305165871056\\
142.01	0.00523049734717545\\
143.01	0.00523047770325624\\
144.01	0.00523045764687\\
145.01	0.00523043716936136\\
146.01	0.00523041626189342\\
147.01	0.00523039491544394\\
148.01	0.00523037312080204\\
149.01	0.00523035086856357\\
150.01	0.00523032814912757\\
151.01	0.00523030495269195\\
152.01	0.00523028126924912\\
153.01	0.00523025708858244\\
154.01	0.00523023240026072\\
155.01	0.00523020719363465\\
156.01	0.00523018145783191\\
157.01	0.00523015518175259\\
158.01	0.00523012835406437\\
159.01	0.00523010096319786\\
160.01	0.00523007299734137\\
161.01	0.00523004444443611\\
162.01	0.00523001529217109\\
163.01	0.00522998552797743\\
164.01	0.00522995513902352\\
165.01	0.00522992411220928\\
166.01	0.00522989243416079\\
167.01	0.00522986009122422\\
168.01	0.00522982706946012\\
169.01	0.00522979335463813\\
170.01	0.00522975893222984\\
171.01	0.00522972378740324\\
172.01	0.00522968790501672\\
173.01	0.00522965126961176\\
174.01	0.00522961386540708\\
175.01	0.00522957567629173\\
176.01	0.00522953668581796\\
177.01	0.00522949687719485\\
178.01	0.00522945623328061\\
179.01	0.0052294147365757\\
180.01	0.00522937236921498\\
181.01	0.00522932911296079\\
182.01	0.00522928494919445\\
183.01	0.00522923985890923\\
184.01	0.00522919382270152\\
185.01	0.00522914682076305\\
186.01	0.00522909883287243\\
187.01	0.00522904983838685\\
188.01	0.00522899981623294\\
189.01	0.00522894874489822\\
190.01	0.00522889660242194\\
191.01	0.00522884336638555\\
192.01	0.0052287890139039\\
193.01	0.00522873352161505\\
194.01	0.00522867686567052\\
195.01	0.00522861902172563\\
196.01	0.00522855996492882\\
197.01	0.00522849966991163\\
198.01	0.00522843811077804\\
199.01	0.00522837526109304\\
200.01	0.00522831109387224\\
201.01	0.00522824558157061\\
202.01	0.00522817869607035\\
203.01	0.0052281104086697\\
204.01	0.00522804069007087\\
205.01	0.00522796951036762\\
206.01	0.00522789683903318\\
207.01	0.00522782264490732\\
208.01	0.00522774689618307\\
209.01	0.00522766956039422\\
210.01	0.00522759060440158\\
211.01	0.00522750999437894\\
212.01	0.00522742769579952\\
213.01	0.00522734367342148\\
214.01	0.00522725789127358\\
215.01	0.00522717031263984\\
216.01	0.00522708090004511\\
217.01	0.00522698961523909\\
218.01	0.00522689641918104\\
219.01	0.00522680127202353\\
220.01	0.0052267041330961\\
221.01	0.005226604960889\\
222.01	0.00522650371303547\\
223.01	0.00522640034629519\\
224.01	0.00522629481653646\\
225.01	0.00522618707871809\\
226.01	0.00522607708687117\\
227.01	0.00522596479408068\\
228.01	0.00522585015246619\\
229.01	0.00522573311316276\\
230.01	0.00522561362630088\\
231.01	0.00522549164098697\\
232.01	0.00522536710528238\\
233.01	0.00522523996618295\\
234.01	0.0052251101695975\\
235.01	0.00522497766032645\\
236.01	0.00522484238203957\\
237.01	0.0052247042772541\\
238.01	0.00522456328731119\\
239.01	0.00522441935235325\\
240.01	0.00522427241130017\\
241.01	0.00522412240182509\\
242.01	0.00522396926033021\\
243.01	0.0052238129219215\\
244.01	0.00522365332038351\\
245.01	0.00522349038815377\\
246.01	0.00522332405629596\\
247.01	0.00522315425447371\\
248.01	0.00522298091092315\\
249.01	0.0052228039524253\\
250.01	0.00522262330427802\\
251.01	0.00522243889026721\\
252.01	0.00522225063263816\\
253.01	0.00522205845206537\\
254.01	0.00522186226762313\\
255.01	0.00522166199675438\\
256.01	0.00522145755524\\
257.01	0.00522124885716726\\
258.01	0.00522103581489799\\
259.01	0.00522081833903531\\
260.01	0.00522059633839147\\
261.01	0.00522036971995348\\
262.01	0.00522013838884953\\
263.01	0.00521990224831446\\
264.01	0.00521966119965464\\
265.01	0.00521941514221146\\
266.01	0.00521916397332687\\
267.01	0.00521890758830544\\
268.01	0.0052186458803775\\
269.01	0.00521837874066163\\
270.01	0.00521810605812694\\
271.01	0.00521782771955339\\
272.01	0.00521754360949365\\
273.01	0.00521725361023293\\
274.01	0.00521695760174888\\
275.01	0.00521665546167114\\
276.01	0.00521634706524061\\
277.01	0.00521603228526704\\
278.01	0.00521571099208779\\
279.01	0.00521538305352528\\
280.01	0.00521504833484423\\
281.01	0.00521470669870827\\
282.01	0.00521435800513644\\
283.01	0.00521400211145921\\
284.01	0.00521363887227421\\
285.01	0.00521326813940134\\
286.01	0.00521288976183818\\
287.01	0.00521250358571453\\
288.01	0.00521210945424713\\
289.01	0.0052117072076936\\
290.01	0.00521129668330694\\
291.01	0.00521087771528946\\
292.01	0.0052104501347468\\
293.01	0.00521001376964111\\
294.01	0.00520956844474569\\
295.01	0.0052091139815982\\
296.01	0.00520865019845492\\
297.01	0.00520817691024477\\
298.01	0.00520769392852357\\
299.01	0.0052072010614285\\
300.01	0.00520669811363309\\
301.01	0.00520618488630217\\
302.01	0.00520566117704823\\
303.01	0.00520512677988722\\
304.01	0.00520458148519613\\
305.01	0.0052040250796701\\
306.01	0.0052034573462818\\
307.01	0.00520287806424025\\
308.01	0.00520228700895233\\
309.01	0.00520168395198414\\
310.01	0.00520106866102452\\
311.01	0.00520044089984959\\
312.01	0.00519980042828956\\
313.01	0.00519914700219645\\
314.01	0.00519848037341427\\
315.01	0.00519780028975126\\
316.01	0.00519710649495468\\
317.01	0.00519639872868714\\
318.01	0.0051956767265069\\
319.01	0.00519494021985052\\
320.01	0.00519418893601805\\
321.01	0.00519342259816257\\
322.01	0.00519264092528269\\
323.01	0.00519184363221911\\
324.01	0.00519103042965536\\
325.01	0.00519020102412334\\
326.01	0.00518935511801243\\
327.01	0.00518849240958488\\
328.01	0.00518761259299581\\
329.01	0.00518671535831935\\
330.01	0.00518580039158017\\
331.01	0.00518486737479209\\
332.01	0.00518391598600277\\
333.01	0.00518294589934636\\
334.01	0.0051819567851026\\
335.01	0.00518094830976426\\
336.01	0.00517992013611334\\
337.01	0.00517887192330526\\
338.01	0.00517780332696261\\
339.01	0.00517671399927806\\
340.01	0.00517560358912744\\
341.01	0.00517447174219251\\
342.01	0.00517331810109557\\
343.01	0.00517214230554414\\
344.01	0.00517094399248761\\
345.01	0.00516972279628462\\
346.01	0.005168478348884\\
347.01	0.00516721028001698\\
348.01	0.00516591821740183\\
349.01	0.00516460178696177\\
350.01	0.0051632606130549\\
351.01	0.00516189431871695\\
352.01	0.00516050252591668\\
353.01	0.00515908485582376\\
354.01	0.00515764092908807\\
355.01	0.00515617036613068\\
356.01	0.00515467278744542\\
357.01	0.0051531478139106\\
358.01	0.00515159506710816\\
359.01	0.00515001416965157\\
360.01	0.00514840474551805\\
361.01	0.00514676642038459\\
362.01	0.00514509882196492\\
363.01	0.00514340158034452\\
364.01	0.00514167432831077\\
365.01	0.00513991670167382\\
366.01	0.00513812833957383\\
367.01	0.00513630888477042\\
368.01	0.00513445798390796\\
369.01	0.00513257528774949\\
370.01	0.00513066045137389\\
371.01	0.00512871313432686\\
372.01	0.00512673300071721\\
373.01	0.00512471971924839\\
374.01	0.00512267296317452\\
375.01	0.00512059241016839\\
376.01	0.00511847774208999\\
377.01	0.00511632864463985\\
378.01	0.00511414480688257\\
379.01	0.00511192592062577\\
380.01	0.00510967167963675\\
381.01	0.00510738177867964\\
382.01	0.00510505591235568\\
383.01	0.0051026937737284\\
384.01	0.00510029505271672\\
385.01	0.00509785943423936\\
386.01	0.00509538659609526\\
387.01	0.00509287620656777\\
388.01	0.00509032792174346\\
389.01	0.00508774138254064\\
390.01	0.00508511621144959\\
391.01	0.00508245200899375\\
392.01	0.00507974834993066\\
393.01	0.00507700477922407\\
394.01	0.00507422080783227\\
395.01	0.00507139590837533\\
396.01	0.00506852951076469\\
397.01	0.00506562099789917\\
398.01	0.0050626697015592\\
399.01	0.00505967489865909\\
400.01	0.00505663580804531\\
401.01	0.00505355158806207\\
402.01	0.00505042133513194\\
403.01	0.00504724408362658\\
404.01	0.0050440188073223\\
405.01	0.0050407444227402\\
406.01	0.00503741979466024\\
407.01	0.00503404374406445\\
408.01	0.0050306150586897\\
409.01	0.00502713250625813\\
410.01	0.00502359485027725\\
411.01	0.00502000086806385\\
412.01	0.00501634937032668\\
413.01	0.00501263922124967\\
414.01	0.00500886935755939\\
415.01	0.00500503880457567\\
416.01	0.00500114668680975\\
417.01	0.00499719223044989\\
418.01	0.00499317475531407\\
419.01	0.004989093654993\\
420.01	0.00498494836696045\\
421.01	0.00498073834559584\\
422.01	0.00497646305411321\\
423.01	0.00497212196381056\\
424.01	0.00496771455341656\\
425.01	0.00496324030825039\\
426.01	0.00495869871917461\\
427.01	0.00495408928132304\\
428.01	0.00494941149258343\\
429.01	0.00494466485181496\\
430.01	0.00493984885678136\\
431.01	0.00493496300177866\\
432.01	0.00493000677494162\\
433.01	0.00492497965520942\\
434.01	0.00491988110893609\\
435.01	0.004914710586133\\
436.01	0.00490946751633421\\
437.01	0.00490415130407918\\
438.01	0.0048987613240141\\
439.01	0.00489329691561693\\
440.01	0.00488775737756222\\
441.01	0.00488214196174756\\
442.01	0.00487644986701617\\
443.01	0.00487068023262009\\
444.01	0.00486483213148382\\
445.01	0.00485890456333915\\
446.01	0.00485289644782183\\
447.01	0.0048468066176342\\
448.01	0.00484063381189665\\
449.01	0.00483437666982691\\
450.01	0.00482803372490231\\
451.01	0.00482160339967746\\
452.01	0.00481508400143666\\
453.01	0.00480847371887101\\
454.01	0.00480177061996946\\
455.01	0.00479497265130526\\
456.01	0.00478807763887849\\
457.01	0.00478108329064709\\
458.01	0.0047739872008241\\
459.01	0.00476678685595524\\
460.01	0.00475947964270048\\
461.01	0.00475206285713371\\
462.01	0.00474453371524032\\
463.01	0.00473688936414579\\
464.01	0.00472912689344517\\
465.01	0.00472124334584061\\
466.01	0.00471323572615556\\
467.01	0.0047051010076903\\
468.01	0.00469683613486702\\
469.01	0.00468843802121081\\
470.01	0.00467990354198637\\
471.01	0.00467122952129828\\
472.01	0.00466241271420965\\
473.01	0.00465344978541999\\
474.01	0.00464433728719487\\
475.01	0.00463507164027726\\
476.01	0.00462564912182952\\
477.01	0.00461606586177708\\
478.01	0.00460631784347362\\
479.01	0.00459640090568819\\
480.01	0.00458631074558279\\
481.01	0.00457604292263263\\
482.01	0.0045655928633934\\
483.01	0.00455495586698143\\
484.01	0.00454412711108051\\
485.01	0.00453310165824177\\
486.01	0.0045218744621983\\
487.01	0.00451044037388256\\
488.01	0.00449879414680735\\
489.01	0.00448693044147396\\
490.01	0.00447484382848845\\
491.01	0.00446252879012912\\
492.01	0.00444997972019801\\
493.01	0.00443719092212448\\
494.01	0.00442415660545863\\
495.01	0.00441087088108571\\
496.01	0.00439732775569151\\
497.01	0.00438352112617594\\
498.01	0.00436944477479478\\
499.01	0.00435509236575418\\
500.01	0.00434045744372663\\
501.01	0.00432553343426664\\
502.01	0.00431031364545572\\
503.01	0.00429479126971478\\
504.01	0.00427895938512484\\
505.01	0.00426281095607702\\
506.01	0.0042463388331861\\
507.01	0.00422953575244153\\
508.01	0.0042123943336166\\
509.01	0.00419490707800405\\
510.01	0.00417706636559825\\
511.01	0.00415886445188262\\
512.01	0.00414029346441456\\
513.01	0.00412134539940507\\
514.01	0.00410201211847112\\
515.01	0.0040822853456925\\
516.01	0.00406215666502128\\
517.01	0.00404161751800122\\
518.01	0.00402065920166672\\
519.01	0.00399927286644217\\
520.01	0.00397744951389662\\
521.01	0.00395517999431117\\
522.01	0.00393245500410627\\
523.01	0.00390926508320447\\
524.01	0.00388560061240984\\
525.01	0.00386145181088166\\
526.01	0.00383680873377442\\
527.01	0.0038116612700997\\
528.01	0.00378599914084991\\
529.01	0.00375981189740243\\
530.01	0.00373308892021277\\
531.01	0.00370581941779547\\
532.01	0.00367799242599846\\
533.01	0.00364959680759006\\
534.01	0.00362062125220449\\
535.01	0.00359105427670131\\
536.01	0.0035608842260056\\
537.01	0.00353009927449551\\
538.01	0.00349868742799898\\
539.01	0.00346663652646789\\
540.01	0.00343393424739245\\
541.01	0.00340056811002963\\
542.01	0.00336652548052344\\
543.01	0.00333179357800268\\
544.01	0.00329635948175882\\
545.01	0.00326021013961405\\
546.01	0.00322333237760449\\
547.01	0.00318571291111079\\
548.01	0.0031473383575855\\
549.01	0.00310819525103227\\
550.01	0.00306827005840969\\
551.01	0.00302754919814918\\
552.01	0.00298601906098984\\
553.01	0.00294366603335671\\
554.01	0.0029004765235269\\
555.01	0.00285643699084808\\
556.01	0.0028115339783004\\
557.01	0.00276575414871214\\
558.01	0.00271908432496719\\
559.01	0.00267151153456668\\
560.01	0.00262302305893529\\
561.01	0.00257360648788791\\
562.01	0.00252324977970384\\
563.01	0.00247194132727781\\
564.01	0.00241967003084647\\
565.01	0.00236642537780996\\
566.01	0.00231219753018783\\
567.01	0.0022569774202631\\
568.01	0.00220075685497212\\
569.01	0.00214352862959637\\
570.01	0.00208528665129069\\
571.01	0.00202602607294628\\
572.01	0.00196574343782424\\
573.01	0.00190443683530311\\
574.01	0.00184210606794752\\
575.01	0.00177875282992286\\
576.01	0.00171438089652806\\
577.01	0.00164899632428734\\
578.01	0.00158260766060568\\
579.01	0.00151522616142658\\
580.01	0.00144686601460507\\
581.01	0.00137754456577855\\
582.01	0.00130728254233795\\
583.01	0.00123610426960943\\
584.01	0.00116403787147582\\
585.01	0.00109111544530627\\
586.01	0.00101737319810725\\
587.01	0.00094285152711592\\
588.01	0.000867595023456734\\
589.01	0.000791652371757953\\
590.01	0.000715076111516234\\
591.01	0.000637922217178835\\
592.01	0.000560249442989076\\
593.01	0.000482118365120262\\
594.01	0.00040359003690792\\
595.01	0.000324724152338418\\
596.01	0.000245576587458076\\
597.01	0.000166416924230011\\
598.01	9.17366970063765e-05\\
599.01	2.94669364271135e-05\\
599.02	2.89574269585705e-05\\
599.03	2.84509530852801e-05\\
599.04	2.79475443462508e-05\\
599.05	2.7447230571279e-05\\
599.06	2.69500418838293e-05\\
599.07	2.64560087039449e-05\\
599.08	2.59651617511691e-05\\
599.09	2.54775320475218e-05\\
599.1	2.49931509204836e-05\\
599.11	2.45120500060175e-05\\
599.12	2.40342612516185e-05\\
599.13	2.35598169193978e-05\\
599.14	2.30887495892007e-05\\
599.15	2.26210921617405e-05\\
599.16	2.215687786177e-05\\
599.17	2.16961402412993e-05\\
599.18	2.12389131828174e-05\\
599.19	2.07852309025806e-05\\
599.2	2.03351279538938e-05\\
599.21	1.98886392304524e-05\\
599.22	1.94457999697188e-05\\
599.23	1.90066457563046e-05\\
599.24	1.85712125254211e-05\\
599.25	1.81395365663334e-05\\
599.26	1.77116564095865e-05\\
599.27	1.72876122135658e-05\\
599.28	1.68674445361946e-05\\
599.29	1.64511943388859e-05\\
599.3	1.60389029905273e-05\\
599.31	1.56306122715087e-05\\
599.32	1.52263643777902e-05\\
599.33	1.48262019250105e-05\\
599.34	1.44301679526268e-05\\
599.35	1.40383059281154e-05\\
599.36	1.36506597511916e-05\\
599.37	1.32672737580847e-05\\
599.38	1.28881927258535e-05\\
599.39	1.25134618767404e-05\\
599.4	1.21431268825696e-05\\
599.41	1.17772338691906e-05\\
599.42	1.14158294209719e-05\\
599.43	1.10589605853174e-05\\
599.44	1.07066748772523e-05\\
599.45	1.03590202840433e-05\\
599.46	1.00160452698589e-05\\
599.47	9.67779878049101e-06\\
599.48	9.34433024810284e-06\\
599.49	9.0156895960411e-06\\
599.5	8.69192724369319e-06\\
599.51	8.37309411137396e-06\\
599.52	8.05924162529392e-06\\
599.53	7.75042172253965e-06\\
599.54	7.44668685613396e-06\\
599.55	7.14809000013084e-06\\
599.56	6.85468465475535e-06\\
599.57	6.56652485161134e-06\\
599.58	6.2836651589307e-06\\
599.59	6.00616068686249e-06\\
599.6	5.73406709284546e-06\\
599.61	5.46744058700123e-06\\
599.62	5.2063379376039e-06\\
599.63	4.95081647657741e-06\\
599.64	4.70093410508653e-06\\
599.65	4.45674929914174e-06\\
599.66	4.21832111529089e-06\\
599.67	3.98570919634376e-06\\
599.68	3.75897377716608e-06\\
599.69	3.53817569053241e-06\\
599.7	3.32337637303295e-06\\
599.71	3.11463787102881e-06\\
599.72	2.91202284668363e-06\\
599.73	2.71559458404555e-06\\
599.74	2.52541699518466e-06\\
599.75	2.34155462640155e-06\\
599.76	2.16407266449663e-06\\
599.77	1.99303694307755e-06\\
599.78	1.82851394897772e-06\\
599.79	1.67057082867475e-06\\
599.8	1.51927539483211e-06\\
599.81	1.37469613287027e-06\\
599.82	1.23690220760544e-06\\
599.83	1.10596346997172e-06\\
599.84	9.81950463784659e-07\\
599.85	8.64934432591793e-07\\
599.86	7.54987326587186e-07\\
599.87	6.52181809583305e-07\\
599.88	5.56591266064402e-07\\
599.89	4.68289808293679e-07\\
599.9	3.87352283521061e-07\\
599.91	3.13854281216996e-07\\
599.92	2.47872140425945e-07\\
599.93	1.89482957149364e-07\\
599.94	1.38764591834512e-07\\
599.95	9.57956769204876e-08\\
599.96	6.06556244606149e-08\\
599.97	3.34246338211386e-08\\
599.98	1.4183699454523e-08\\
599.99	3.014618757749e-09\\
600	0\\
};
\addplot [color=black!50!mycolor20,solid,forget plot]
  table[row sep=crcr]{%
0.01	0.00515885702990981\\
1.01	0.0051588560241\\
2.01	0.00515885499729988\\
3.01	0.00515885394907215\\
4.01	0.00515885287897005\\
5.01	0.00515885178653789\\
6.01	0.00515885067131022\\
7.01	0.00515884953281223\\
8.01	0.00515884837055896\\
9.01	0.00515884718405548\\
10.01	0.0051588459727966\\
11.01	0.00515884473626641\\
12.01	0.00515884347393856\\
13.01	0.00515884218527537\\
14.01	0.00515884086972846\\
15.01	0.00515883952673772\\
16.01	0.00515883815573141\\
17.01	0.0051588367561263\\
18.01	0.0051588353273264\\
19.01	0.00515883386872376\\
20.01	0.00515883237969774\\
21.01	0.00515883085961459\\
22.01	0.00515882930782761\\
23.01	0.00515882772367658\\
24.01	0.00515882610648755\\
25.01	0.00515882445557254\\
26.01	0.00515882277022942\\
27.01	0.00515882104974142\\
28.01	0.00515881929337658\\
29.01	0.00515881750038814\\
30.01	0.00515881567001362\\
31.01	0.00515881380147459\\
32.01	0.00515881189397682\\
33.01	0.00515880994670914\\
34.01	0.00515880795884389\\
35.01	0.00515880592953614\\
36.01	0.00515880385792326\\
37.01	0.0051588017431248\\
38.01	0.00515879958424202\\
39.01	0.00515879738035744\\
40.01	0.0051587951305346\\
41.01	0.00515879283381773\\
42.01	0.00515879048923105\\
43.01	0.00515878809577853\\
44.01	0.00515878565244348\\
45.01	0.00515878315818808\\
46.01	0.00515878061195318\\
47.01	0.00515877801265732\\
48.01	0.00515877535919682\\
49.01	0.00515877265044496\\
50.01	0.00515876988525172\\
51.01	0.00515876706244326\\
52.01	0.00515876418082143\\
53.01	0.00515876123916288\\
54.01	0.00515875823621926\\
55.01	0.00515875517071615\\
56.01	0.00515875204135281\\
57.01	0.00515874884680145\\
58.01	0.00515874558570655\\
59.01	0.00515874225668481\\
60.01	0.00515873885832414\\
61.01	0.00515873538918293\\
62.01	0.00515873184779\\
63.01	0.0051587282326435\\
64.01	0.00515872454221057\\
65.01	0.00515872077492647\\
66.01	0.00515871692919399\\
67.01	0.00515871300338292\\
68.01	0.00515870899582923\\
69.01	0.00515870490483436\\
70.01	0.00515870072866464\\
71.01	0.00515869646555036\\
72.01	0.0051586921136853\\
73.01	0.00515868767122566\\
74.01	0.0051586831362895\\
75.01	0.00515867850695594\\
76.01	0.00515867378126429\\
77.01	0.00515866895721325\\
78.01	0.00515866403276003\\
79.01	0.00515865900581964\\
80.01	0.00515865387426385\\
81.01	0.00515864863592042\\
82.01	0.00515864328857214\\
83.01	0.00515863782995577\\
84.01	0.0051586322577615\\
85.01	0.0051586265696312\\
86.01	0.00515862076315848\\
87.01	0.005158614835887\\
88.01	0.00515860878530952\\
89.01	0.00515860260886694\\
90.01	0.00515859630394725\\
91.01	0.00515858986788435\\
92.01	0.00515858329795716\\
93.01	0.00515857659138814\\
94.01	0.00515856974534228\\
95.01	0.00515856275692627\\
96.01	0.00515855562318678\\
97.01	0.00515854834110911\\
98.01	0.00515854090761676\\
99.01	0.00515853331956932\\
100.01	0.00515852557376181\\
101.01	0.00515851766692258\\
102.01	0.00515850959571263\\
103.01	0.00515850135672406\\
104.01	0.00515849294647825\\
105.01	0.005158484361425\\
106.01	0.00515847559794085\\
107.01	0.00515846665232715\\
108.01	0.00515845752080916\\
109.01	0.0051584481995342\\
110.01	0.00515843868457001\\
111.01	0.00515842897190299\\
112.01	0.00515841905743692\\
113.01	0.00515840893699111\\
114.01	0.00515839860629824\\
115.01	0.00515838806100324\\
116.01	0.00515837729666118\\
117.01	0.00515836630873546\\
118.01	0.0051583550925958\\
119.01	0.00515834364351641\\
120.01	0.00515833195667411\\
121.01	0.00515832002714636\\
122.01	0.00515830784990911\\
123.01	0.00515829541983468\\
124.01	0.00515828273168972\\
125.01	0.0051582697801332\\
126.01	0.00515825655971399\\
127.01	0.00515824306486858\\
128.01	0.00515822928991865\\
129.01	0.00515821522906947\\
130.01	0.00515820087640657\\
131.01	0.00515818622589406\\
132.01	0.00515817127137142\\
133.01	0.00515815600655153\\
134.01	0.00515814042501787\\
135.01	0.005158124520222\\
136.01	0.00515810828548049\\
137.01	0.0051580917139727\\
138.01	0.0051580747987376\\
139.01	0.00515805753267124\\
140.01	0.00515803990852314\\
141.01	0.00515802191889429\\
142.01	0.00515800355623319\\
143.01	0.00515798481283354\\
144.01	0.00515796568083053\\
145.01	0.00515794615219766\\
146.01	0.00515792621874401\\
147.01	0.00515790587211008\\
148.01	0.00515788510376496\\
149.01	0.00515786390500255\\
150.01	0.00515784226693803\\
151.01	0.00515782018050432\\
152.01	0.00515779763644852\\
153.01	0.00515777462532752\\
154.01	0.00515775113750518\\
155.01	0.00515772716314745\\
156.01	0.00515770269221862\\
157.01	0.00515767771447749\\
158.01	0.00515765221947307\\
159.01	0.00515762619654012\\
160.01	0.00515759963479528\\
161.01	0.00515757252313203\\
162.01	0.00515754485021663\\
163.01	0.00515751660448361\\
164.01	0.00515748777413071\\
165.01	0.00515745834711416\\
166.01	0.00515742831114403\\
167.01	0.00515739765367906\\
168.01	0.00515736636192184\\
169.01	0.00515733442281314\\
170.01	0.00515730182302728\\
171.01	0.00515726854896632\\
172.01	0.00515723458675454\\
173.01	0.00515719992223326\\
174.01	0.00515716454095461\\
175.01	0.00515712842817631\\
176.01	0.00515709156885521\\
177.01	0.00515705394764147\\
178.01	0.00515701554887255\\
179.01	0.0051569763565666\\
180.01	0.00515693635441651\\
181.01	0.00515689552578291\\
182.01	0.00515685385368791\\
183.01	0.0051568113208079\\
184.01	0.00515676790946716\\
185.01	0.00515672360163037\\
186.01	0.00515667837889592\\
187.01	0.00515663222248803\\
188.01	0.00515658511324966\\
189.01	0.00515653703163458\\
190.01	0.00515648795770014\\
191.01	0.0051564378710991\\
192.01	0.00515638675107132\\
193.01	0.005156334576436\\
194.01	0.00515628132558318\\
195.01	0.00515622697646506\\
196.01	0.00515617150658773\\
197.01	0.00515611489300189\\
198.01	0.00515605711229397\\
199.01	0.00515599814057718\\
200.01	0.00515593795348198\\
201.01	0.00515587652614653\\
202.01	0.00515581383320716\\
203.01	0.00515574984878829\\
204.01	0.00515568454649235\\
205.01	0.00515561789939006\\
206.01	0.00515554988000918\\
207.01	0.00515548046032457\\
208.01	0.00515540961174691\\
209.01	0.00515533730511193\\
210.01	0.00515526351066906\\
211.01	0.00515518819807012\\
212.01	0.00515511133635769\\
213.01	0.00515503289395306\\
214.01	0.00515495283864426\\
215.01	0.00515487113757402\\
216.01	0.00515478775722689\\
217.01	0.0051547026634168\\
218.01	0.00515461582127406\\
219.01	0.00515452719523198\\
220.01	0.00515443674901382\\
221.01	0.00515434444561874\\
222.01	0.00515425024730838\\
223.01	0.00515415411559234\\
224.01	0.00515405601121396\\
225.01	0.00515395589413587\\
226.01	0.00515385372352475\\
227.01	0.00515374945773651\\
228.01	0.00515364305430075\\
229.01	0.00515353446990535\\
230.01	0.00515342366038035\\
231.01	0.00515331058068166\\
232.01	0.00515319518487507\\
233.01	0.00515307742611925\\
234.01	0.00515295725664875\\
235.01	0.005152834627757\\
236.01	0.0051527094897785\\
237.01	0.0051525817920708\\
238.01	0.00515245148299724\\
239.01	0.00515231850990778\\
240.01	0.00515218281912065\\
241.01	0.00515204435590355\\
242.01	0.00515190306445377\\
243.01	0.00515175888787945\\
244.01	0.00515161176817957\\
245.01	0.00515146164622308\\
246.01	0.0051513084617296\\
247.01	0.00515115215324801\\
248.01	0.00515099265813542\\
249.01	0.00515082991253633\\
250.01	0.00515066385136038\\
251.01	0.00515049440826115\\
252.01	0.00515032151561338\\
253.01	0.00515014510449054\\
254.01	0.00514996510464259\\
255.01	0.00514978144447204\\
256.01	0.00514959405101135\\
257.01	0.00514940284989878\\
258.01	0.00514920776535424\\
259.01	0.00514900872015569\\
260.01	0.00514880563561422\\
261.01	0.0051485984315492\\
262.01	0.00514838702626342\\
263.01	0.00514817133651743\\
264.01	0.00514795127750393\\
265.01	0.00514772676282249\\
266.01	0.00514749770445233\\
267.01	0.00514726401272705\\
268.01	0.00514702559630736\\
269.01	0.00514678236215444\\
270.01	0.00514653421550262\\
271.01	0.00514628105983266\\
272.01	0.00514602279684371\\
273.01	0.00514575932642596\\
274.01	0.0051454905466326\\
275.01	0.005145216353652\\
276.01	0.0051449366417792\\
277.01	0.00514465130338831\\
278.01	0.00514436022890354\\
279.01	0.0051440633067712\\
280.01	0.00514376042343024\\
281.01	0.0051434514632844\\
282.01	0.0051431363086736\\
283.01	0.0051428148398448\\
284.01	0.00514248693492345\\
285.01	0.00514215246988519\\
286.01	0.00514181131852681\\
287.01	0.00514146335243798\\
288.01	0.00514110844097311\\
289.01	0.00514074645122274\\
290.01	0.00514037724798586\\
291.01	0.0051400006937419\\
292.01	0.00513961664862325\\
293.01	0.00513922497038804\\
294.01	0.00513882551439329\\
295.01	0.00513841813356863\\
296.01	0.00513800267838955\\
297.01	0.00513757899685242\\
298.01	0.00513714693444902\\
299.01	0.00513670633414256\\
300.01	0.00513625703634312\\
301.01	0.0051357988788851\\
302.01	0.00513533169700429\\
303.01	0.00513485532331667\\
304.01	0.00513436958779732\\
305.01	0.00513387431776059\\
306.01	0.00513336933784176\\
307.01	0.00513285446997861\\
308.01	0.00513232953339542\\
309.01	0.0051317943445872\\
310.01	0.00513124871730558\\
311.01	0.0051306924625467\\
312.01	0.00513012538853882\\
313.01	0.00512954730073386\\
314.01	0.00512895800179832\\
315.01	0.00512835729160743\\
316.01	0.00512774496724011\\
317.01	0.00512712082297672\\
318.01	0.00512648465029781\\
319.01	0.00512583623788549\\
320.01	0.00512517537162689\\
321.01	0.00512450183461948\\
322.01	0.0051238154071795\\
323.01	0.00512311586685151\\
324.01	0.00512240298842167\\
325.01	0.0051216765439323\\
326.01	0.00512093630270024\\
327.01	0.00512018203133715\\
328.01	0.00511941349377248\\
329.01	0.00511863045127929\\
330.01	0.00511783266250308\\
331.01	0.00511701988349325\\
332.01	0.00511619186773709\\
333.01	0.00511534836619702\\
334.01	0.00511448912735015\\
335.01	0.00511361389723113\\
336.01	0.0051127224194771\\
337.01	0.00511181443537549\\
338.01	0.005110889683914\\
339.01	0.0051099479018334\\
340.01	0.00510898882368156\\
341.01	0.00510801218187047\\
342.01	0.00510701770673342\\
343.01	0.00510600512658526\\
344.01	0.00510497416778141\\
345.01	0.00510392455477939\\
346.01	0.00510285601019871\\
347.01	0.00510176825488083\\
348.01	0.00510066100794834\\
349.01	0.00509953398686104\\
350.01	0.00509838690747049\\
351.01	0.00509721948406981\\
352.01	0.0050960314294396\\
353.01	0.00509482245488743\\
354.01	0.00509359227028119\\
355.01	0.00509234058407305\\
356.01	0.00509106710331452\\
357.01	0.00508977153365951\\
358.01	0.00508845357935487\\
359.01	0.00508711294321496\\
360.01	0.00508574932658015\\
361.01	0.00508436242925602\\
362.01	0.00508295194943009\\
363.01	0.0050815175835661\\
364.01	0.00508005902627021\\
365.01	0.0050785759701294\\
366.01	0.00507706810551666\\
367.01	0.00507553512036194\\
368.01	0.00507397669988443\\
369.01	0.00507239252628455\\
370.01	0.0050707822783914\\
371.01	0.00506914563126312\\
372.01	0.00506748225573749\\
373.01	0.00506579181792998\\
374.01	0.00506407397867616\\
375.01	0.00506232839291728\\
376.01	0.00506055470902655\\
377.01	0.0050587525680755\\
378.01	0.00505692160303954\\
379.01	0.00505506143794429\\
380.01	0.00505317168695244\\
381.01	0.00505125195339577\\
382.01	0.00504930182875586\\
383.01	0.00504732089159889\\
384.01	0.00504530870647419\\
385.01	0.00504326482278501\\
386.01	0.00504118877364544\\
387.01	0.00503908007473788\\
388.01	0.00503693822319072\\
389.01	0.00503476269649633\\
390.01	0.00503255295149568\\
391.01	0.00503030842345703\\
392.01	0.00502802852527992\\
393.01	0.0050257126468598\\
394.01	0.00502336015465031\\
395.01	0.00502097039146361\\
396.01	0.00501854267654705\\
397.01	0.00501607630598036\\
398.01	0.00501357055342879\\
399.01	0.00501102467128963\\
400.01	0.00500843789225892\\
401.01	0.00500580943133779\\
402.01	0.00500313848828397\\
403.01	0.00500042425049423\\
404.01	0.00499766589628516\\
405.01	0.00499486259850927\\
406.01	0.0049920135284137\\
407.01	0.00498911785961365\\
408.01	0.00498617477201683\\
409.01	0.00498318345549585\\
410.01	0.00498014311307841\\
411.01	0.0049770529633992\\
412.01	0.00497391224215602\\
413.01	0.00497072020233393\\
414.01	0.00496747611302017\\
415.01	0.00496417925673757\\
416.01	0.00496082892538658\\
417.01	0.00495742441509799\\
418.01	0.00495396502055228\\
419.01	0.00495045002956129\\
420.01	0.00494687871883495\\
421.01	0.0049432503514938\\
422.01	0.00493956417562139\\
423.01	0.0049358194228934\\
424.01	0.00493201530710189\\
425.01	0.00492815102256774\\
426.01	0.00492422574244058\\
427.01	0.00492023861688482\\
428.01	0.0049161887711512\\
429.01	0.00491207530353777\\
430.01	0.00490789728324046\\
431.01	0.00490365374809945\\
432.01	0.00489934370224745\\
433.01	0.00489496611366758\\
434.01	0.00489051991167313\\
435.01	0.00488600398431918\\
436.01	0.00488141717576553\\
437.01	0.00487675828360596\\
438.01	0.00487202605618598\\
439.01	0.00486721918993338\\
440.01	0.00486233632672787\\
441.01	0.0048573760513388\\
442.01	0.00485233688896245\\
443.01	0.00484721730289378\\
444.01	0.00484201569236629\\
445.01	0.00483673039059846\\
446.01	0.00483135966308048\\
447.01	0.00482590170613808\\
448.01	0.00482035464580481\\
449.01	0.00481471653703252\\
450.01	0.00480898536326192\\
451.01	0.00480315903636792\\
452.01	0.00479723539698434\\
453.01	0.00479121221520126\\
454.01	0.0047850871916114\\
455.01	0.0047788579586666\\
456.01	0.00477252208228742\\
457.01	0.00476607706364531\\
458.01	0.00475952034102221\\
459.01	0.00475284929162807\\
460.01	0.00474606123324042\\
461.01	0.0047391534255208\\
462.01	0.00473212307085284\\
463.01	0.00472496731455358\\
464.01	0.0047176832443234\\
465.01	0.00471026788883426\\
466.01	0.00470271821540123\\
467.01	0.00469503112674995\\
468.01	0.00468720345697479\\
469.01	0.0046792319668779\\
470.01	0.00467111333897339\\
471.01	0.00466284417252799\\
472.01	0.00465442097905768\\
473.01	0.00464584017869519\\
474.01	0.00463709809774316\\
475.01	0.0046281909675301\\
476.01	0.00461911492439009\\
477.01	0.00460986601030792\\
478.01	0.00460044017382117\\
479.01	0.00459083327102632\\
480.01	0.00458104106663141\\
481.01	0.00457105923499075\\
482.01	0.00456088336105611\\
483.01	0.00455050894117474\\
484.01	0.00453993138366703\\
485.01	0.0045291460091181\\
486.01	0.00451814805033159\\
487.01	0.00450693265190246\\
488.01	0.00449549486938637\\
489.01	0.00448382966806533\\
490.01	0.00447193192133378\\
491.01	0.00445979640875732\\
492.01	0.00444741781388005\\
493.01	0.0044347907218801\\
494.01	0.00442190961718438\\
495.01	0.00440876888115547\\
496.01	0.00439536278994963\\
497.01	0.00438168551261193\\
498.01	0.00436773110942803\\
499.01	0.00435349353049303\\
500.01	0.00433896661440256\\
501.01	0.00432414408693396\\
502.01	0.00430901955959005\\
503.01	0.00429358652792935\\
504.01	0.00427783836966785\\
505.01	0.00426176834256679\\
506.01	0.0042453695821297\\
507.01	0.00422863509914185\\
508.01	0.00421155777708945\\
509.01	0.0041941303695008\\
510.01	0.00417634549724843\\
511.01	0.00415819564585181\\
512.01	0.0041396731628067\\
513.01	0.00412077025496171\\
514.01	0.00410147898594817\\
515.01	0.00408179127365571\\
516.01	0.00406169888774293\\
517.01	0.00404119344716381\\
518.01	0.00402026641769837\\
519.01	0.00399890910948619\\
520.01	0.00397711267457466\\
521.01	0.00395486810450582\\
522.01	0.00393216622796758\\
523.01	0.00390899770853628\\
524.01	0.00388535304253463\\
525.01	0.00386122255702903\\
526.01	0.00383659640798494\\
527.01	0.00381146457860206\\
528.01	0.00378581687784591\\
529.01	0.00375964293919767\\
530.01	0.00373293221964467\\
531.01	0.00370567399893881\\
532.01	0.0036778573791551\\
533.01	0.00364947128458843\\
534.01	0.00362050446202891\\
535.01	0.00359094548146376\\
536.01	0.00356078273725293\\
537.01	0.0035300044498321\\
538.01	0.0034985986679996\\
539.01	0.00346655327184927\\
540.01	0.00343385597641933\\
541.01	0.00340049433613169\\
542.01	0.00336645575010434\\
543.01	0.00333172746843132\\
544.01	0.00329629659952941\\
545.01	0.0032601501186634\\
546.01	0.00322327487777092\\
547.01	0.00318565761672175\\
548.01	0.00314728497615331\\
549.01	0.00310814351204539\\
550.01	0.00306821971220547\\
551.01	0.00302750001485491\\
552.01	0.00298597082952428\\
553.01	0.00294361856048271\\
554.01	0.0029004296329475\\
555.01	0.00285639052234093\\
556.01	0.00281148778688329\\
557.01	0.00276570810383654\\
558.01	0.00271903830973548\\
559.01	0.0026714654449714\\
560.01	0.00262297680311917\\
561.01	0.00257355998542694\\
562.01	0.00252320296091246\\
563.01	0.00247189413254027\\
564.01	0.00241962240997658\\
565.01	0.00236637728944275\\
566.01	0.00231214894120763\\
567.01	0.0022569283052712\\
568.01	0.00220070719580021\\
569.01	0.00214347841486824\\
570.01	0.0020852358760371\\
571.01	0.00202597473827579\\
572.01	0.00196569155065329\\
573.01	0.0019043844081454\\
574.01	0.00184205311876436\\
575.01	0.00177869938203216\\
576.01	0.0017143269785692\\
577.01	0.00164894197023678\\
578.01	0.00158255290983592\\
579.01	0.00151517105879892\\
580.01	0.00144681061058329\\
581.01	0.00137748891654737\\
582.01	0.00130722670990687\\
583.01	0.00123604832187718\\
584.01	0.0011639818822259\\
585.01	0.00109105949409744\\
586.01	0.0010173173700143\\
587.01	0.000942795912266653\\
588.01	0.000867539716298419\\
589.01	0.000791597469971103\\
590.01	0.000715021714474689\\
591.01	0.000637868423831121\\
592.01	0.000560196349006645\\
593.01	0.000482066059121876\\
594.01	0.000403538595524574\\
595.01	0.000324673633827242\\
596.01	0.000245527023505882\\
597.01	0.000166394005355891\\
598.01	9.17366970063799e-05\\
599.01	2.94669364271135e-05\\
599.02	2.89574269585688e-05\\
599.03	2.84509530852819e-05\\
599.04	2.79475443462508e-05\\
599.05	2.7447230571279e-05\\
599.06	2.69500418838293e-05\\
599.07	2.64560087039432e-05\\
599.08	2.59651617511691e-05\\
599.09	2.54775320475218e-05\\
599.1	2.49931509204854e-05\\
599.11	2.45120500060158e-05\\
599.12	2.40342612516167e-05\\
599.13	2.35598169193996e-05\\
599.14	2.30887495892042e-05\\
599.15	2.26210921617405e-05\\
599.16	2.21568778617717e-05\\
599.17	2.16961402412976e-05\\
599.18	2.12389131828191e-05\\
599.19	2.07852309025806e-05\\
599.2	2.03351279538938e-05\\
599.21	1.98886392304542e-05\\
599.22	1.94457999697206e-05\\
599.23	1.90066457563046e-05\\
599.24	1.85712125254211e-05\\
599.25	1.81395365663334e-05\\
599.26	1.77116564095865e-05\\
599.27	1.72876122135641e-05\\
599.28	1.68674445361946e-05\\
599.29	1.64511943388859e-05\\
599.3	1.60389029905273e-05\\
599.31	1.56306122715087e-05\\
599.32	1.52263643777902e-05\\
599.33	1.48262019250087e-05\\
599.34	1.44301679526268e-05\\
599.35	1.40383059281154e-05\\
599.36	1.36506597511899e-05\\
599.37	1.32672737580847e-05\\
599.38	1.28881927258535e-05\\
599.39	1.25134618767387e-05\\
599.4	1.21431268825679e-05\\
599.41	1.17772338691924e-05\\
599.42	1.14158294209736e-05\\
599.43	1.10589605853174e-05\\
599.44	1.07066748772523e-05\\
599.45	1.03590202840433e-05\\
599.46	1.00160452698606e-05\\
599.47	9.67779878049101e-06\\
599.48	9.34433024810284e-06\\
599.49	9.0156895960411e-06\\
599.5	8.69192724369146e-06\\
599.51	8.37309411137396e-06\\
599.52	8.05924162529219e-06\\
599.53	7.75042172253791e-06\\
599.54	7.4466868561357e-06\\
599.55	7.14809000013084e-06\\
599.56	6.85468465475535e-06\\
599.57	6.56652485161308e-06\\
599.58	6.28366515892896e-06\\
599.59	6.00616068686249e-06\\
599.6	5.73406709284546e-06\\
599.61	5.46744058700296e-06\\
599.62	5.20633793760217e-06\\
599.63	4.95081647657568e-06\\
599.64	4.70093410508653e-06\\
599.65	4.45674929914347e-06\\
599.66	4.21832111529262e-06\\
599.67	3.98570919634203e-06\\
599.68	3.75897377716608e-06\\
599.69	3.53817569053415e-06\\
599.7	3.32337637303469e-06\\
599.71	3.11463787102881e-06\\
599.72	2.91202284668363e-06\\
599.73	2.71559458404555e-06\\
599.74	2.52541699518292e-06\\
599.75	2.34155462640155e-06\\
599.76	2.16407266449489e-06\\
599.77	1.99303694307928e-06\\
599.78	1.82851394897598e-06\\
599.79	1.67057082867302e-06\\
599.8	1.51927539483211e-06\\
599.81	1.37469613287027e-06\\
599.82	1.23690220760718e-06\\
599.83	1.10596346997172e-06\\
599.84	9.81950463782924e-07\\
599.85	8.64934432591793e-07\\
599.86	7.54987326587186e-07\\
599.87	6.5218180958504e-07\\
599.88	5.56591266064402e-07\\
599.89	4.68289808295413e-07\\
599.9	3.87352283521061e-07\\
599.91	3.13854281218731e-07\\
599.92	2.4787214042421e-07\\
599.93	1.8948295714763e-07\\
599.94	1.38764591834512e-07\\
599.95	9.57956769222224e-08\\
599.96	6.06556244606149e-08\\
599.97	3.34246338194039e-08\\
599.98	1.4183699454523e-08\\
599.99	3.01461875948372e-09\\
600	0\\
};
\addplot [color=black!60!mycolor21,solid,forget plot]
  table[row sep=crcr]{%
0.01	0.00511537456508619\\
1.01	0.00511537357866577\\
2.01	0.00511537257178985\\
3.01	0.00511537154403555\\
4.01	0.00511537049497141\\
5.01	0.00511536942415678\\
6.01	0.00511536833114231\\
7.01	0.00511536721546909\\
8.01	0.00511536607666884\\
9.01	0.00511536491426357\\
10.01	0.00511536372776555\\
11.01	0.00511536251667717\\
12.01	0.00511536128049021\\
13.01	0.0051153600186863\\
14.01	0.00511535873073629\\
15.01	0.00511535741610018\\
16.01	0.00511535607422688\\
17.01	0.00511535470455369\\
18.01	0.00511535330650676\\
19.01	0.00511535187950028\\
20.01	0.0051153504229363\\
21.01	0.00511534893620466\\
22.01	0.00511534741868266\\
23.01	0.00511534586973479\\
24.01	0.00511534428871243\\
25.01	0.0051153426749539\\
26.01	0.00511534102778349\\
27.01	0.00511533934651185\\
28.01	0.00511533763043549\\
29.01	0.00511533587883644\\
30.01	0.00511533409098209\\
31.01	0.00511533226612451\\
32.01	0.00511533040350066\\
33.01	0.00511532850233169\\
34.01	0.00511532656182295\\
35.01	0.0051153245811633\\
36.01	0.00511532255952498\\
37.01	0.00511532049606332\\
38.01	0.00511531838991627\\
39.01	0.00511531624020418\\
40.01	0.00511531404602932\\
41.01	0.00511531180647548\\
42.01	0.00511530952060763\\
43.01	0.00511530718747187\\
44.01	0.00511530480609428\\
45.01	0.00511530237548142\\
46.01	0.00511529989461918\\
47.01	0.00511529736247276\\
48.01	0.00511529477798627\\
49.01	0.00511529214008206\\
50.01	0.00511528944766045\\
51.01	0.00511528669959924\\
52.01	0.00511528389475322\\
53.01	0.00511528103195392\\
54.01	0.00511527811000864\\
55.01	0.00511527512770054\\
56.01	0.0051152720837875\\
57.01	0.00511526897700234\\
58.01	0.00511526580605207\\
59.01	0.0051152625696169\\
60.01	0.00511525926635029\\
61.01	0.00511525589487797\\
62.01	0.00511525245379763\\
63.01	0.00511524894167829\\
64.01	0.00511524535705981\\
65.01	0.00511524169845221\\
66.01	0.00511523796433488\\
67.01	0.00511523415315624\\
68.01	0.00511523026333304\\
69.01	0.0051152262932497\\
70.01	0.00511522224125771\\
71.01	0.0051152181056747\\
72.01	0.00511521388478418\\
73.01	0.0051152095768344\\
74.01	0.00511520518003802\\
75.01	0.00511520069257119\\
76.01	0.00511519611257275\\
77.01	0.00511519143814377\\
78.01	0.0051151866673465\\
79.01	0.00511518179820347\\
80.01	0.00511517682869724\\
81.01	0.00511517175676903\\
82.01	0.00511516658031813\\
83.01	0.005115161297201\\
84.01	0.00511515590523046\\
85.01	0.00511515040217494\\
86.01	0.00511514478575718\\
87.01	0.00511513905365357\\
88.01	0.00511513320349328\\
89.01	0.00511512723285709\\
90.01	0.00511512113927664\\
91.01	0.00511511492023344\\
92.01	0.00511510857315768\\
93.01	0.00511510209542713\\
94.01	0.00511509548436636\\
95.01	0.00511508873724558\\
96.01	0.00511508185127942\\
97.01	0.00511507482362601\\
98.01	0.00511506765138546\\
99.01	0.00511506033159934\\
100.01	0.00511505286124885\\
101.01	0.00511504523725388\\
102.01	0.00511503745647194\\
103.01	0.00511502951569646\\
104.01	0.00511502141165627\\
105.01	0.00511501314101339\\
106.01	0.0051150047003622\\
107.01	0.00511499608622811\\
108.01	0.00511498729506608\\
109.01	0.00511497832325891\\
110.01	0.00511496916711637\\
111.01	0.00511495982287342\\
112.01	0.00511495028668861\\
113.01	0.00511494055464267\\
114.01	0.00511493062273714\\
115.01	0.00511492048689247\\
116.01	0.00511491014294623\\
117.01	0.00511489958665213\\
118.01	0.00511488881367781\\
119.01	0.00511487781960355\\
120.01	0.00511486659991981\\
121.01	0.00511485515002609\\
122.01	0.00511484346522868\\
123.01	0.00511483154073941\\
124.01	0.00511481937167308\\
125.01	0.0051148069530458\\
126.01	0.00511479427977302\\
127.01	0.00511478134666765\\
128.01	0.00511476814843793\\
129.01	0.00511475467968496\\
130.01	0.00511474093490148\\
131.01	0.00511472690846857\\
132.01	0.00511471259465452\\
133.01	0.00511469798761181\\
134.01	0.00511468308137513\\
135.01	0.00511466786985886\\
136.01	0.005114652346855\\
137.01	0.00511463650603059\\
138.01	0.00511462034092494\\
139.01	0.00511460384494737\\
140.01	0.00511458701137482\\
141.01	0.0051145698333489\\
142.01	0.0051145523038734\\
143.01	0.00511453441581127\\
144.01	0.00511451616188224\\
145.01	0.00511449753465995\\
146.01	0.0051144785265685\\
147.01	0.00511445912988006\\
148.01	0.00511443933671178\\
149.01	0.00511441913902264\\
150.01	0.00511439852861036\\
151.01	0.00511437749710821\\
152.01	0.00511435603598174\\
153.01	0.00511433413652551\\
154.01	0.0051143117898597\\
155.01	0.00511428898692678\\
156.01	0.00511426571848785\\
157.01	0.00511424197511929\\
158.01	0.00511421774720892\\
159.01	0.00511419302495233\\
160.01	0.00511416779834923\\
161.01	0.00511414205719964\\
162.01	0.00511411579109974\\
163.01	0.00511408898943806\\
164.01	0.0051140616413913\\
165.01	0.00511403373592053\\
166.01	0.00511400526176639\\
167.01	0.00511397620744537\\
168.01	0.00511394656124498\\
169.01	0.00511391631121967\\
170.01	0.00511388544518601\\
171.01	0.00511385395071814\\
172.01	0.00511382181514319\\
173.01	0.00511378902553609\\
174.01	0.00511375556871525\\
175.01	0.00511372143123701\\
176.01	0.00511368659939112\\
177.01	0.00511365105919513\\
178.01	0.00511361479638916\\
179.01	0.00511357779643073\\
180.01	0.00511354004448925\\
181.01	0.00511350152544029\\
182.01	0.00511346222386012\\
183.01	0.00511342212402013\\
184.01	0.00511338120988018\\
185.01	0.00511333946508353\\
186.01	0.0051132968729502\\
187.01	0.00511325341647084\\
188.01	0.00511320907830058\\
189.01	0.00511316384075266\\
190.01	0.00511311768579164\\
191.01	0.00511307059502656\\
192.01	0.00511302254970484\\
193.01	0.00511297353070479\\
194.01	0.00511292351852883\\
195.01	0.00511287249329615\\
196.01	0.0051128204347357\\
197.01	0.00511276732217853\\
198.01	0.00511271313455037\\
199.01	0.00511265785036407\\
200.01	0.00511260144771163\\
201.01	0.00511254390425591\\
202.01	0.00511248519722357\\
203.01	0.00511242530339583\\
204.01	0.00511236419910063\\
205.01	0.00511230186020375\\
206.01	0.00511223826210062\\
207.01	0.00511217337970698\\
208.01	0.00511210718745071\\
209.01	0.00511203965926171\\
210.01	0.00511197076856337\\
211.01	0.00511190048826304\\
212.01	0.00511182879074205\\
213.01	0.00511175564784649\\
214.01	0.00511168103087718\\
215.01	0.00511160491057922\\
216.01	0.00511152725713227\\
217.01	0.00511144804013977\\
218.01	0.00511136722861875\\
219.01	0.00511128479098874\\
220.01	0.00511120069506087\\
221.01	0.00511111490802715\\
222.01	0.00511102739644873\\
223.01	0.00511093812624481\\
224.01	0.005110847062681\\
225.01	0.00511075417035707\\
226.01	0.00511065941319563\\
227.01	0.00511056275442924\\
228.01	0.00511046415658882\\
229.01	0.00511036358149029\\
230.01	0.00511026099022245\\
231.01	0.00511015634313397\\
232.01	0.00511004959981986\\
233.01	0.00510994071910844\\
234.01	0.00510982965904792\\
235.01	0.00510971637689247\\
236.01	0.00510960082908849\\
237.01	0.00510948297126058\\
238.01	0.0051093627581968\\
239.01	0.00510924014383445\\
240.01	0.00510911508124556\\
241.01	0.00510898752262192\\
242.01	0.00510885741925982\\
243.01	0.00510872472154461\\
244.01	0.00510858937893547\\
245.01	0.00510845133994987\\
246.01	0.00510831055214733\\
247.01	0.00510816696211345\\
248.01	0.00510802051544365\\
249.01	0.00510787115672668\\
250.01	0.00510771882952772\\
251.01	0.00510756347637195\\
252.01	0.00510740503872715\\
253.01	0.0051072434569865\\
254.01	0.00510707867045102\\
255.01	0.00510691061731235\\
256.01	0.00510673923463431\\
257.01	0.00510656445833546\\
258.01	0.00510638622317047\\
259.01	0.00510620446271231\\
260.01	0.0051060191093331\\
261.01	0.00510583009418576\\
262.01	0.0051056373471849\\
263.01	0.00510544079698826\\
264.01	0.00510524037097728\\
265.01	0.00510503599523753\\
266.01	0.00510482759453972\\
267.01	0.00510461509231971\\
268.01	0.00510439841065887\\
269.01	0.0051041774702647\\
270.01	0.00510395219044992\\
271.01	0.00510372248911315\\
272.01	0.00510348828271854\\
273.01	0.00510324948627538\\
274.01	0.00510300601331802\\
275.01	0.00510275777588507\\
276.01	0.00510250468449915\\
277.01	0.00510224664814656\\
278.01	0.00510198357425603\\
279.01	0.00510171536867864\\
280.01	0.00510144193566733\\
281.01	0.0051011631778558\\
282.01	0.0051008789962381\\
283.01	0.00510058929014804\\
284.01	0.00510029395723858\\
285.01	0.00509999289346137\\
286.01	0.0050996859930462\\
287.01	0.00509937314848069\\
288.01	0.00509905425048967\\
289.01	0.00509872918801566\\
290.01	0.00509839784819816\\
291.01	0.00509806011635408\\
292.01	0.0050977158759578\\
293.01	0.00509736500862192\\
294.01	0.00509700739407746\\
295.01	0.00509664291015461\\
296.01	0.00509627143276457\\
297.01	0.00509589283587995\\
298.01	0.00509550699151661\\
299.01	0.00509511376971576\\
300.01	0.00509471303852548\\
301.01	0.0050943046639842\\
302.01	0.00509388851010228\\
303.01	0.00509346443884578\\
304.01	0.00509303231011988\\
305.01	0.00509259198175275\\
306.01	0.00509214330947942\\
307.01	0.00509168614692717\\
308.01	0.00509122034559952\\
309.01	0.00509074575486255\\
310.01	0.00509026222193018\\
311.01	0.00508976959185067\\
312.01	0.00508926770749355\\
313.01	0.00508875640953611\\
314.01	0.00508823553645157\\
315.01	0.00508770492449654\\
316.01	0.00508716440769957\\
317.01	0.00508661381784954\\
318.01	0.00508605298448518\\
319.01	0.00508548173488376\\
320.01	0.00508489989405098\\
321.01	0.0050843072847109\\
322.01	0.0050837037272957\\
323.01	0.0050830890399361\\
324.01	0.00508246303845125\\
325.01	0.00508182553633909\\
326.01	0.00508117634476631\\
327.01	0.00508051527255817\\
328.01	0.00507984212618831\\
329.01	0.00507915670976748\\
330.01	0.00507845882503223\\
331.01	0.00507774827133237\\
332.01	0.00507702484561849\\
333.01	0.00507628834242714\\
334.01	0.00507553855386577\\
335.01	0.00507477526959526\\
336.01	0.00507399827681169\\
337.01	0.00507320736022492\\
338.01	0.0050724023020362\\
339.01	0.00507158288191181\\
340.01	0.00507074887695489\\
341.01	0.00506990006167322\\
342.01	0.00506903620794389\\
343.01	0.00506815708497276\\
344.01	0.00506726245925113\\
345.01	0.00506635209450567\\
346.01	0.00506542575164405\\
347.01	0.00506448318869422\\
348.01	0.00506352416073646\\
349.01	0.00506254841982928\\
350.01	0.00506155571492676\\
351.01	0.00506054579178812\\
352.01	0.00505951839287799\\
353.01	0.00505847325725697\\
354.01	0.00505741012046191\\
355.01	0.00505632871437514\\
356.01	0.00505522876708182\\
357.01	0.0050541100027152\\
358.01	0.00505297214128824\\
359.01	0.00505181489851223\\
360.01	0.00505063798560022\\
361.01	0.00504944110905546\\
362.01	0.00504822397044529\\
363.01	0.00504698626615752\\
364.01	0.00504572768714187\\
365.01	0.00504444791863355\\
366.01	0.00504314663986067\\
367.01	0.00504182352373438\\
368.01	0.00504047823652271\\
369.01	0.00503911043750747\\
370.01	0.00503771977862624\\
371.01	0.0050363059040985\\
372.01	0.00503486845003876\\
373.01	0.00503340704405683\\
374.01	0.00503192130484766\\
375.01	0.00503041084177334\\
376.01	0.00502887525443856\\
377.01	0.00502731413226412\\
378.01	0.00502572705406087\\
379.01	0.00502411358760904\\
380.01	0.0050224732892465\\
381.01	0.00502080570347132\\
382.01	0.00501911036256501\\
383.01	0.00501738678624098\\
384.01	0.00501563448132577\\
385.01	0.00501385294147864\\
386.01	0.00501204164695771\\
387.01	0.00501020006443945\\
388.01	0.00500832764689743\\
389.01	0.00500642383354889\\
390.01	0.00500448804987441\\
391.01	0.00500251970771589\\
392.01	0.00500051820545903\\
393.01	0.00499848292830042\\
394.01	0.00499641324860389\\
395.01	0.00499430852634055\\
396.01	0.0049921681096128\\
397.01	0.00498999133525027\\
398.01	0.00498777752947023\\
399.01	0.0049855260085841\\
400.01	0.00498323607973127\\
401.01	0.00498090704161521\\
402.01	0.00497853818521132\\
403.01	0.00497612879441471\\
404.01	0.00497367814658854\\
405.01	0.00497118551297381\\
406.01	0.00496865015892014\\
407.01	0.00496607134389796\\
408.01	0.00496344832125691\\
409.01	0.00496078033770405\\
410.01	0.00495806663248456\\
411.01	0.00495530643626621\\
412.01	0.00495249896974442\\
413.01	0.00494964344200882\\
414.01	0.0049467390487322\\
415.01	0.00494378497026461\\
416.01	0.004940780369726\\
417.01	0.00493772439119356\\
418.01	0.0049346161580621\\
419.01	0.00493145477161677\\
420.01	0.0049282393097965\\
421.01	0.00492496882606195\\
422.01	0.00492164234826773\\
423.01	0.00491825887750382\\
424.01	0.00491481738690305\\
425.01	0.00491131682041853\\
426.01	0.00490775609157437\\
427.01	0.0049041340821939\\
428.01	0.00490044964111023\\
429.01	0.00489670158286241\\
430.01	0.0048928886863865\\
431.01	0.00488900969370451\\
432.01	0.00488506330862008\\
433.01	0.00488104819542758\\
434.01	0.00487696297764154\\
435.01	0.00487280623675731\\
436.01	0.0048685765110472\\
437.01	0.00486427229440418\\
438.01	0.00485989203523984\\
439.01	0.00485543413544438\\
440.01	0.00485089694941698\\
441.01	0.00484627878317314\\
442.01	0.00484157789353523\\
443.01	0.00483679248740917\\
444.01	0.00483192072115264\\
445.01	0.00482696070003325\\
446.01	0.00482191047777615\\
447.01	0.0048167680561974\\
448.01	0.00481153138491524\\
449.01	0.00480619836112931\\
450.01	0.00480076682945332\\
451.01	0.00479523458178626\\
452.01	0.0047895993571997\\
453.01	0.00478385884181872\\
454.01	0.00477801066867047\\
455.01	0.00477205241747283\\
456.01	0.00476598161433339\\
457.01	0.00475979573133239\\
458.01	0.00475349218596161\\
459.01	0.00474706834039562\\
460.01	0.00474052150058208\\
461.01	0.00473384891513581\\
462.01	0.00472704777404177\\
463.01	0.00472011520717495\\
464.01	0.0047130482826625\\
465.01	0.00470584400512422\\
466.01	0.00469849931383928\\
467.01	0.00469101108089903\\
468.01	0.00468337610940962\\
469.01	0.00467559113180777\\
470.01	0.00466765280834874\\
471.01	0.00465955772580652\\
472.01	0.00465130239640618\\
473.01	0.00464288325697488\\
474.01	0.00463429666827228\\
475.01	0.00462553891443265\\
476.01	0.00461660620244293\\
477.01	0.00460749466159679\\
478.01	0.00459820034288695\\
479.01	0.00458871921831872\\
480.01	0.00457904718012815\\
481.01	0.00456918003989079\\
482.01	0.00455911352751145\\
483.01	0.00454884329008707\\
484.01	0.00453836489063851\\
485.01	0.00452767380671381\\
486.01	0.00451676542886776\\
487.01	0.00450563505902704\\
488.01	0.00449427790875835\\
489.01	0.00448268909745497\\
490.01	0.00447086365046334\\
491.01	0.00445879649716923\\
492.01	0.00444648246906463\\
493.01	0.00443391629780925\\
494.01	0.00442109261329795\\
495.01	0.00440800594173683\\
496.01	0.00439465070372407\\
497.01	0.00438102121232231\\
498.01	0.00436711167110619\\
499.01	0.00435291617216462\\
500.01	0.00433842869404037\\
501.01	0.00432364309959411\\
502.01	0.00430855313379167\\
503.01	0.00429315242141763\\
504.01	0.00427743446472644\\
505.01	0.0042613926410404\\
506.01	0.0042450202003061\\
507.01	0.00422831026262157\\
508.01	0.00421125581574242\\
509.01	0.00419384971257737\\
510.01	0.00417608466867955\\
511.01	0.00415795325973936\\
512.01	0.00413944791908216\\
513.01	0.00412056093517416\\
514.01	0.00410128444913751\\
515.01	0.00408161045227904\\
516.01	0.0040615307836341\\
517.01	0.00404103712753402\\
518.01	0.00402012101120428\\
519.01	0.00399877380240454\\
520.01	0.00397698670712463\\
521.01	0.00395475076734716\\
522.01	0.00393205685889359\\
523.01	0.00390889568936628\\
524.01	0.00388525779620339\\
525.01	0.00386113354486115\\
526.01	0.00383651312714225\\
527.01	0.00381138655968965\\
528.01	0.00378574368266509\\
529.01	0.00375957415863976\\
530.01	0.00373286747171988\\
531.01	0.00370561292693941\\
532.01	0.00367779964995124\\
533.01	0.0036494165870535\\
534.01	0.00362045250559075\\
535.01	0.0035908959947719\\
536.01	0.0035607354669546\\
537.01	0.00352995915944617\\
538.01	0.00349855513688045\\
539.01	0.00346651129423165\\
540.01	0.00343381536053716\\
541.01	0.0034004549034037\\
542.01	0.00336641733438228\\
543.01	0.00333168991530363\\
544.01	0.00329625976567544\\
545.01	0.00326011387125328\\
546.01	0.00322323909390545\\
547.01	0.00318562218290602\\
548.01	0.00314724978780293\\
549.01	0.00310810847301851\\
550.01	0.00306818473435897\\
551.01	0.00302746501762232\\
552.01	0.00298593573951309\\
553.01	0.00294358331108954\\
554.01	0.00290039416398957\\
555.01	0.00285635477970408\\
556.01	0.00281145172218551\\
557.01	0.00276567167410694\\
558.01	0.00271900147710948\\
559.01	0.00267142817640317\\
560.01	0.00262293907011199\\
561.01	0.00257352176378272\\
562.01	0.00252316423050231\\
563.01	0.00247185487709762\\
564.01	0.00241958261691408\\
565.01	0.00236633694969541\\
566.01	0.00231210804910259\\
567.01	0.00225688685842675\\
568.01	0.00220066519505408\\
569.01	0.00214343586423708\\
570.01	0.00208519278270673\\
571.01	0.0020259311126231\\
572.01	0.00196564740629848\\
573.01	0.00190433976203401\\
574.01	0.00184200799127701\\
575.01	0.00177865379711897\\
576.01	0.00171428096390598\\
577.01	0.00164889555739739\\
578.01	0.00158250613447483\\
579.01	0.00151512396083625\\
580.01	0.00144676323438244\\
581.01	0.00137744131107293\\
582.01	0.00130717892884751\\
583.01	0.00123600042371503\\
584.01	0.00116393393022847\\
585.01	0.00109101155620331\\
586.01	0.00101726951857635\\
587.01	0.000942748223607881\\
588.01	0.00086749227002462\\
589.01	0.000791550347971492\\
590.01	0.000714974999524649\\
591.01	0.000637822197691413\\
592.01	0.000560150689888671\\
593.01	0.000482021038358199\\
594.01	0.000403494273246938\\
595.01	0.000324630053410709\\
596.01	0.000245484204483332\\
597.01	0.00016637427046296\\
598.01	9.17366970063765e-05\\
599.01	2.94669364271135e-05\\
599.02	2.89574269585705e-05\\
599.03	2.84509530852819e-05\\
599.04	2.79475443462508e-05\\
599.05	2.7447230571279e-05\\
599.06	2.69500418838293e-05\\
599.07	2.64560087039432e-05\\
599.08	2.59651617511691e-05\\
599.09	2.54775320475235e-05\\
599.1	2.49931509204836e-05\\
599.11	2.45120500060158e-05\\
599.12	2.40342612516167e-05\\
599.13	2.35598169193978e-05\\
599.14	2.30887495892024e-05\\
599.15	2.26210921617405e-05\\
599.16	2.215687786177e-05\\
599.17	2.16961402412993e-05\\
599.18	2.12389131828191e-05\\
599.19	2.07852309025824e-05\\
599.2	2.03351279538938e-05\\
599.21	1.98886392304542e-05\\
599.22	1.94457999697206e-05\\
599.23	1.90066457563063e-05\\
599.24	1.85712125254211e-05\\
599.25	1.81395365663334e-05\\
599.26	1.77116564095865e-05\\
599.27	1.72876122135658e-05\\
599.28	1.68674445361963e-05\\
599.29	1.64511943388859e-05\\
599.3	1.60389029905273e-05\\
599.31	1.56306122715087e-05\\
599.32	1.5226364377792e-05\\
599.33	1.48262019250105e-05\\
599.34	1.44301679526268e-05\\
599.35	1.40383059281154e-05\\
599.36	1.36506597511916e-05\\
599.37	1.32672737580847e-05\\
599.38	1.28881927258535e-05\\
599.39	1.25134618767404e-05\\
599.4	1.21431268825696e-05\\
599.41	1.17772338691924e-05\\
599.42	1.14158294209719e-05\\
599.43	1.10589605853174e-05\\
599.44	1.0706674877254e-05\\
599.45	1.03590202840433e-05\\
599.46	1.00160452698589e-05\\
599.47	9.67779878049101e-06\\
599.48	9.34433024810284e-06\\
599.49	9.01568959604283e-06\\
599.5	8.69192724369319e-06\\
599.51	8.37309411137396e-06\\
599.52	8.05924162529392e-06\\
599.53	7.75042172253791e-06\\
599.54	7.44668685613396e-06\\
599.55	7.14809000013084e-06\\
599.56	6.85468465475708e-06\\
599.57	6.56652485161308e-06\\
599.58	6.28366515892896e-06\\
599.59	6.00616068686076e-06\\
599.6	5.73406709284546e-06\\
599.61	5.46744058700296e-06\\
599.62	5.2063379376039e-06\\
599.63	4.95081647657741e-06\\
599.64	4.70093410508653e-06\\
599.65	4.45674929914347e-06\\
599.66	4.21832111529089e-06\\
599.67	3.98570919634376e-06\\
599.68	3.75897377716608e-06\\
599.69	3.53817569053415e-06\\
599.7	3.32337637303295e-06\\
599.71	3.11463787102881e-06\\
599.72	2.91202284668536e-06\\
599.73	2.71559458404382e-06\\
599.74	2.52541699518466e-06\\
599.75	2.34155462640329e-06\\
599.76	2.16407266449489e-06\\
599.77	1.99303694307755e-06\\
599.78	1.82851394897598e-06\\
599.79	1.67057082867302e-06\\
599.8	1.51927539483211e-06\\
599.81	1.37469613287027e-06\\
599.82	1.23690220760718e-06\\
599.83	1.10596346997172e-06\\
599.84	9.81950463782924e-07\\
599.85	8.64934432591793e-07\\
599.86	7.54987326587186e-07\\
599.87	6.5218180958504e-07\\
599.88	5.56591266064402e-07\\
599.89	4.68289808295413e-07\\
599.9	3.87352283521061e-07\\
599.91	3.13854281216996e-07\\
599.92	2.47872140425945e-07\\
599.93	1.89482957149364e-07\\
599.94	1.38764591834512e-07\\
599.95	9.57956769204876e-08\\
599.96	6.06556244606149e-08\\
599.97	3.34246338211386e-08\\
599.98	1.4183699454523e-08\\
599.99	3.01461875948372e-09\\
600	0\\
};
\addplot [color=black!80!mycolor21,solid,forget plot]
  table[row sep=crcr]{%
0.01	0.00508968489854863\\
1.01	0.00508968392840915\\
2.01	0.00508968293827033\\
3.01	0.00508968192772168\\
4.01	0.00508968089634424\\
5.01	0.00508967984371108\\
6.01	0.00508967876938601\\
7.01	0.00508967767292402\\
8.01	0.0050896765538712\\
9.01	0.00508967541176465\\
10.01	0.00508967424613145\\
11.01	0.00508967305648953\\
12.01	0.00508967184234672\\
13.01	0.00508967060320077\\
14.01	0.00508966933853946\\
15.01	0.00508966804784016\\
16.01	0.00508966673056913\\
17.01	0.00508966538618245\\
18.01	0.00508966401412461\\
19.01	0.00508966261382906\\
20.01	0.00508966118471765\\
21.01	0.00508965972620041\\
22.01	0.00508965823767534\\
23.01	0.00508965671852852\\
24.01	0.00508965516813317\\
25.01	0.00508965358584988\\
26.01	0.00508965197102632\\
27.01	0.00508965032299697\\
28.01	0.00508964864108255\\
29.01	0.00508964692459043\\
30.01	0.00508964517281322\\
31.01	0.0050896433850298\\
32.01	0.00508964156050412\\
33.01	0.00508963969848547\\
34.01	0.00508963779820734\\
35.01	0.00508963585888803\\
36.01	0.00508963387973035\\
37.01	0.00508963185992034\\
38.01	0.00508962979862791\\
39.01	0.00508962769500595\\
40.01	0.00508962554819032\\
41.01	0.00508962335729921\\
42.01	0.00508962112143281\\
43.01	0.00508961883967355\\
44.01	0.00508961651108512\\
45.01	0.00508961413471206\\
46.01	0.00508961170957968\\
47.01	0.00508960923469374\\
48.01	0.0050896067090395\\
49.01	0.005089604131582\\
50.01	0.00508960150126519\\
51.01	0.00508959881701179\\
52.01	0.00508959607772295\\
53.01	0.00508959328227709\\
54.01	0.00508959042953037\\
55.01	0.00508958751831546\\
56.01	0.00508958454744193\\
57.01	0.0050895815156951\\
58.01	0.00508957842183554\\
59.01	0.00508957526459921\\
60.01	0.00508957204269618\\
61.01	0.00508956875481088\\
62.01	0.00508956539960106\\
63.01	0.00508956197569726\\
64.01	0.00508955848170247\\
65.01	0.00508955491619142\\
66.01	0.00508955127771036\\
67.01	0.00508954756477626\\
68.01	0.00508954377587622\\
69.01	0.00508953990946661\\
70.01	0.00508953596397271\\
71.01	0.00508953193778857\\
72.01	0.00508952782927572\\
73.01	0.00508952363676241\\
74.01	0.0050895193585434\\
75.01	0.0050895149928794\\
76.01	0.00508951053799624\\
77.01	0.00508950599208363\\
78.01	0.00508950135329541\\
79.01	0.00508949661974818\\
80.01	0.00508949178952017\\
81.01	0.00508948686065187\\
82.01	0.00508948183114403\\
83.01	0.00508947669895704\\
84.01	0.00508947146201124\\
85.01	0.00508946611818433\\
86.01	0.00508946066531193\\
87.01	0.00508945510118605\\
88.01	0.00508944942355465\\
89.01	0.00508944363012057\\
90.01	0.00508943771854086\\
91.01	0.00508943168642519\\
92.01	0.00508942553133553\\
93.01	0.00508941925078526\\
94.01	0.005089412842238\\
95.01	0.00508940630310644\\
96.01	0.00508939963075168\\
97.01	0.0050893928224821\\
98.01	0.00508938587555247\\
99.01	0.00508937878716225\\
100.01	0.00508937155445535\\
101.01	0.0050893641745188\\
102.01	0.00508935664438125\\
103.01	0.0050893489610122\\
104.01	0.00508934112132043\\
105.01	0.00508933312215349\\
106.01	0.00508932496029615\\
107.01	0.00508931663246878\\
108.01	0.00508930813532626\\
109.01	0.00508929946545756\\
110.01	0.00508929061938363\\
111.01	0.00508928159355538\\
112.01	0.00508927238435405\\
113.01	0.00508926298808852\\
114.01	0.00508925340099421\\
115.01	0.00508924361923217\\
116.01	0.00508923363888714\\
117.01	0.00508922345596582\\
118.01	0.0050892130663955\\
119.01	0.00508920246602324\\
120.01	0.00508919165061311\\
121.01	0.00508918061584572\\
122.01	0.00508916935731579\\
123.01	0.00508915787053103\\
124.01	0.00508914615090978\\
125.01	0.00508913419377994\\
126.01	0.00508912199437698\\
127.01	0.00508910954784219\\
128.01	0.0050890968492203\\
129.01	0.00508908389345868\\
130.01	0.00508907067540444\\
131.01	0.00508905718980313\\
132.01	0.00508904343129645\\
133.01	0.00508902939442083\\
134.01	0.00508901507360411\\
135.01	0.00508900046316471\\
136.01	0.00508898555730928\\
137.01	0.00508897035012952\\
138.01	0.00508895483560167\\
139.01	0.00508893900758307\\
140.01	0.00508892285981006\\
141.01	0.00508890638589551\\
142.01	0.00508888957932736\\
143.01	0.00508887243346507\\
144.01	0.00508885494153837\\
145.01	0.00508883709664292\\
146.01	0.00508881889173978\\
147.01	0.00508880031965169\\
148.01	0.00508878137306073\\
149.01	0.00508876204450545\\
150.01	0.00508874232637809\\
151.01	0.00508872221092231\\
152.01	0.00508870169022961\\
153.01	0.00508868075623708\\
154.01	0.00508865940072392\\
155.01	0.00508863761530899\\
156.01	0.00508861539144725\\
157.01	0.00508859272042685\\
158.01	0.00508856959336592\\
159.01	0.00508854600120961\\
160.01	0.00508852193472651\\
161.01	0.00508849738450504\\
162.01	0.00508847234095076\\
163.01	0.00508844679428235\\
164.01	0.00508842073452835\\
165.01	0.00508839415152334\\
166.01	0.00508836703490458\\
167.01	0.00508833937410815\\
168.01	0.00508831115836474\\
169.01	0.00508828237669681\\
170.01	0.00508825301791326\\
171.01	0.00508822307060691\\
172.01	0.00508819252314936\\
173.01	0.00508816136368744\\
174.01	0.00508812958013865\\
175.01	0.00508809716018666\\
176.01	0.00508806409127808\\
177.01	0.00508803036061665\\
178.01	0.00508799595515917\\
179.01	0.00508796086161161\\
180.01	0.00508792506642342\\
181.01	0.00508788855578339\\
182.01	0.00508785131561455\\
183.01	0.00508781333156922\\
184.01	0.00508777458902433\\
185.01	0.00508773507307595\\
186.01	0.00508769476853422\\
187.01	0.00508765365991789\\
188.01	0.00508761173144972\\
189.01	0.00508756896704989\\
190.01	0.00508752535033114\\
191.01	0.0050874808645932\\
192.01	0.00508743549281676\\
193.01	0.00508738921765766\\
194.01	0.00508734202144088\\
195.01	0.00508729388615485\\
196.01	0.00508724479344496\\
197.01	0.00508719472460746\\
198.01	0.00508714366058295\\
199.01	0.00508709158195026\\
200.01	0.00508703846891954\\
201.01	0.00508698430132592\\
202.01	0.0050869290586222\\
203.01	0.00508687271987262\\
204.01	0.00508681526374536\\
205.01	0.00508675666850569\\
206.01	0.00508669691200885\\
207.01	0.0050866359716921\\
208.01	0.00508657382456766\\
209.01	0.0050865104472155\\
210.01	0.0050864458157747\\
211.01	0.00508637990593653\\
212.01	0.00508631269293606\\
213.01	0.00508624415154391\\
214.01	0.0050861742560585\\
215.01	0.00508610298029722\\
216.01	0.00508603029758829\\
217.01	0.00508595618076212\\
218.01	0.00508588060214226\\
219.01	0.00508580353353721\\
220.01	0.00508572494623056\\
221.01	0.00508564481097246\\
222.01	0.00508556309797024\\
223.01	0.00508547977687867\\
224.01	0.00508539481679076\\
225.01	0.00508530818622798\\
226.01	0.00508521985313024\\
227.01	0.00508512978484633\\
228.01	0.00508503794812291\\
229.01	0.00508494430909511\\
230.01	0.00508484883327607\\
231.01	0.00508475148554574\\
232.01	0.00508465223014012\\
233.01	0.00508455103064164\\
234.01	0.00508444784996679\\
235.01	0.00508434265035488\\
236.01	0.00508423539335794\\
237.01	0.00508412603982814\\
238.01	0.00508401454990652\\
239.01	0.00508390088301113\\
240.01	0.00508378499782545\\
241.01	0.00508366685228511\\
242.01	0.00508354640356693\\
243.01	0.00508342360807613\\
244.01	0.00508329842143316\\
245.01	0.00508317079846144\\
246.01	0.00508304069317468\\
247.01	0.00508290805876296\\
248.01	0.00508277284758022\\
249.01	0.00508263501113062\\
250.01	0.00508249450005546\\
251.01	0.00508235126411846\\
252.01	0.00508220525219239\\
253.01	0.00508205641224551\\
254.01	0.00508190469132674\\
255.01	0.00508175003555108\\
256.01	0.00508159239008613\\
257.01	0.00508143169913625\\
258.01	0.00508126790592883\\
259.01	0.005081100952698\\
260.01	0.0050809307806709\\
261.01	0.0050807573300511\\
262.01	0.00508058054000401\\
263.01	0.00508040034864073\\
264.01	0.00508021669300259\\
265.01	0.00508002950904491\\
266.01	0.00507983873162133\\
267.01	0.00507964429446718\\
268.01	0.00507944613018312\\
269.01	0.00507924417021882\\
270.01	0.00507903834485659\\
271.01	0.00507882858319344\\
272.01	0.00507861481312517\\
273.01	0.00507839696132857\\
274.01	0.00507817495324472\\
275.01	0.00507794871306106\\
276.01	0.00507771816369434\\
277.01	0.00507748322677191\\
278.01	0.00507724382261517\\
279.01	0.0050769998702209\\
280.01	0.00507675128724319\\
281.01	0.00507649798997494\\
282.01	0.00507623989332975\\
283.01	0.0050759769108233\\
284.01	0.00507570895455487\\
285.01	0.00507543593518795\\
286.01	0.00507515776193144\\
287.01	0.00507487434252075\\
288.01	0.0050745855831988\\
289.01	0.00507429138869547\\
290.01	0.00507399166220907\\
291.01	0.00507368630538611\\
292.01	0.00507337521830149\\
293.01	0.0050730582994385\\
294.01	0.00507273544566812\\
295.01	0.00507240655222904\\
296.01	0.00507207151270666\\
297.01	0.00507173021901234\\
298.01	0.0050713825613626\\
299.01	0.00507102842825736\\
300.01	0.00507066770645871\\
301.01	0.00507030028096876\\
302.01	0.00506992603500801\\
303.01	0.00506954484999268\\
304.01	0.00506915660551168\\
305.01	0.00506876117930481\\
306.01	0.0050683584472374\\
307.01	0.00506794828327829\\
308.01	0.00506753055947467\\
309.01	0.00506710514592742\\
310.01	0.00506667191076681\\
311.01	0.00506623072012522\\
312.01	0.00506578143811225\\
313.01	0.00506532392678706\\
314.01	0.00506485804613066\\
315.01	0.00506438365401823\\
316.01	0.00506390060618953\\
317.01	0.00506340875621869\\
318.01	0.00506290795548339\\
319.01	0.0050623980531334\\
320.01	0.0050618788960569\\
321.01	0.0050613503288471\\
322.01	0.00506081219376555\\
323.01	0.00506026433070663\\
324.01	0.00505970657715841\\
325.01	0.00505913876816366\\
326.01	0.00505856073627759\\
327.01	0.00505797231152495\\
328.01	0.00505737332135447\\
329.01	0.00505676359059195\\
330.01	0.00505614294139056\\
331.01	0.00505551119317966\\
332.01	0.00505486816260919\\
333.01	0.00505421366349356\\
334.01	0.0050535475067512\\
335.01	0.00505286950034228\\
336.01	0.00505217944920215\\
337.01	0.00505147715517194\\
338.01	0.00505076241692552\\
339.01	0.00505003502989245\\
340.01	0.00504929478617756\\
341.01	0.00504854147447524\\
342.01	0.00504777487998067\\
343.01	0.00504699478429536\\
344.01	0.00504620096532795\\
345.01	0.00504539319719129\\
346.01	0.00504457125009211\\
347.01	0.00504373489021737\\
348.01	0.00504288387961356\\
349.01	0.00504201797606057\\
350.01	0.00504113693294076\\
351.01	0.00504024049909947\\
352.01	0.00503932841870174\\
353.01	0.00503840043108143\\
354.01	0.00503745627058352\\
355.01	0.00503649566640135\\
356.01	0.00503551834240592\\
357.01	0.00503452401696963\\
358.01	0.00503351240278326\\
359.01	0.00503248320666641\\
360.01	0.00503143612937219\\
361.01	0.00503037086538597\\
362.01	0.00502928710271854\\
363.01	0.00502818452269397\\
364.01	0.00502706279973244\\
365.01	0.00502592160112882\\
366.01	0.00502476058682842\\
367.01	0.00502357940919909\\
368.01	0.00502237771280032\\
369.01	0.00502115513415367\\
370.01	0.00501991130150968\\
371.01	0.00501864583461723\\
372.01	0.00501735834449525\\
373.01	0.00501604843320617\\
374.01	0.00501471569363434\\
375.01	0.00501335970927069\\
376.01	0.0050119800540043\\
377.01	0.0050105762919218\\
378.01	0.00500914797711773\\
379.01	0.00500769465351591\\
380.01	0.00500621585470338\\
381.01	0.00500471110377882\\
382.01	0.00500317991321475\\
383.01	0.00500162178473713\\
384.01	0.00500003620922055\\
385.01	0.00499842266659988\\
386.01	0.00499678062579989\\
387.01	0.004995109544679\\
388.01	0.00499340886998947\\
389.01	0.00499167803735109\\
390.01	0.00498991647123458\\
391.01	0.00498812358495372\\
392.01	0.00498629878066159\\
393.01	0.00498444144934696\\
394.01	0.00498255097082289\\
395.01	0.00498062671370551\\
396.01	0.00497866803537291\\
397.01	0.00497667428189747\\
398.01	0.00497464478794445\\
399.01	0.00497257887662799\\
400.01	0.00497047585931557\\
401.01	0.00496833503537484\\
402.01	0.00496615569185451\\
403.01	0.00496393710309253\\
404.01	0.00496167853024835\\
405.01	0.00495937922075627\\
406.01	0.00495703840769956\\
407.01	0.00495465530910754\\
408.01	0.00495222912718346\\
409.01	0.00494975904747006\\
410.01	0.00494724423796623\\
411.01	0.00494468384820806\\
412.01	0.00494207700833176\\
413.01	0.00493942282813428\\
414.01	0.0049367203961457\\
415.01	0.00493396877872884\\
416.01	0.00493116701921221\\
417.01	0.00492831413705652\\
418.01	0.00492540912705394\\
419.01	0.00492245095854786\\
420.01	0.00491943857466022\\
421.01	0.00491637089151961\\
422.01	0.00491324679748207\\
423.01	0.00491006515234761\\
424.01	0.00490682478657479\\
425.01	0.00490352450049337\\
426.01	0.00490016306351852\\
427.01	0.00489673921336825\\
428.01	0.00489325165528674\\
429.01	0.00488969906127556\\
430.01	0.00488608006933446\\
431.01	0.00488239328271397\\
432.01	0.00487863726918218\\
433.01	0.00487481056030704\\
434.01	0.00487091165075614\\
435.01	0.00486693899761418\\
436.01	0.00486289101971994\\
437.01	0.00485876609702326\\
438.01	0.00485456256996054\\
439.01	0.00485027873885053\\
440.01	0.00484591286330686\\
441.01	0.00484146316166711\\
442.01	0.00483692781043501\\
443.01	0.00483230494373362\\
444.01	0.00482759265276427\\
445.01	0.004822788985267\\
446.01	0.0048178919449794\\
447.01	0.00481289949108397\\
448.01	0.00480780953764226\\
449.01	0.00480261995300564\\
450.01	0.00479732855919909\\
451.01	0.00479193313126877\\
452.01	0.00478643139658867\\
453.01	0.00478082103412083\\
454.01	0.0047750996736238\\
455.01	0.00476926489480607\\
456.01	0.004763314226423\\
457.01	0.00475724514531626\\
458.01	0.00475105507539817\\
459.01	0.00474474138658562\\
460.01	0.00473830139368701\\
461.01	0.00473173235525485\\
462.01	0.00472503147240865\\
463.01	0.00471819588764355\\
464.01	0.00471122268363353\\
465.01	0.00470410888204121\\
466.01	0.00469685144234548\\
467.01	0.00468944726069261\\
468.01	0.00468189316877687\\
469.01	0.00467418593275002\\
470.01	0.00466632225215585\\
471.01	0.00465829875888062\\
472.01	0.00465011201610793\\
473.01	0.00464175851726228\\
474.01	0.0046332346849264\\
475.01	0.00462453686972014\\
476.01	0.00461566134912937\\
477.01	0.00460660432628032\\
478.01	0.00459736192865624\\
479.01	0.0045879302067535\\
480.01	0.00457830513267719\\
481.01	0.00456848259867723\\
482.01	0.0045584584156257\\
483.01	0.00454822831143751\\
484.01	0.00453778792943787\\
485.01	0.00452713282668092\\
486.01	0.00451625847222222\\
487.01	0.00450516024535199\\
488.01	0.00449383343379027\\
489.01	0.00448227323184985\\
490.01	0.0044704747385684\\
491.01	0.00445843295581342\\
492.01	0.00444614278635746\\
493.01	0.00443359903192642\\
494.01	0.00442079639121603\\
495.01	0.00440772945787658\\
496.01	0.00439439271846087\\
497.01	0.00438078055033382\\
498.01	0.00436688721954141\\
499.01	0.00435270687863867\\
500.01	0.00433823356447654\\
501.01	0.00432346119595061\\
502.01	0.00430838357171449\\
503.01	0.0042929943678623\\
504.01	0.00427728713558209\\
505.01	0.00426125529878641\\
506.01	0.00424489215172238\\
507.01	0.00422819085656428\\
508.01	0.00421114444099322\\
509.01	0.00419374579576618\\
510.01	0.0041759876722773\\
511.01	0.00415786268011639\\
512.01	0.00413936328462638\\
513.01	0.00412048180446544\\
514.01	0.00410121040917822\\
515.01	0.00408154111678125\\
516.01	0.00406146579137078\\
517.01	0.0040409761407585\\
518.01	0.00402006371414559\\
519.01	0.0039987198998435\\
520.01	0.00397693592305065\\
521.01	0.00395470284369803\\
522.01	0.00393201155437433\\
523.01	0.00390885277834494\\
524.01	0.00388521706767742\\
525.01	0.00386109480149263\\
526.01	0.00383647618435695\\
527.01	0.00381135124483494\\
528.01	0.00378570983422737\\
529.01	0.00375954162551551\\
530.01	0.00373283611253975\\
531.01	0.00370558260944227\\
532.01	0.0036777702504053\\
533.01	0.0036493879897215\\
534.01	0.00362042460223506\\
535.01	0.00359086868419784\\
536.01	0.00356070865458678\\
537.01	0.0035299327569359\\
538.01	0.00349852906174065\\
539.01	0.00346648546949845\\
540.01	0.00343378971445346\\
541.01	0.00340042936912447\\
542.01	0.00336639184969894\\
543.01	0.00333166442238612\\
544.01	0.00329623421082995\\
545.01	0.00326008820469374\\
546.01	0.00322321326953876\\
547.01	0.00318559615812999\\
548.01	0.00314722352331424\\
549.01	0.00310808193263235\\
550.01	0.00306815788483875\\
551.01	0.00302743782851966\\
552.01	0.00298590818301772\\
553.01	0.00294355536188978\\
554.01	0.00290036579914364\\
555.01	0.00285632597852142\\
556.01	0.00281142246612037\\
557.01	0.00276564194666362\\
558.01	0.00271897126376054\\
559.01	0.00267139746452063\\
560.01	0.00262290784891359\\
561.01	0.0025734900242925\\
562.01	0.00252313196552769\\
563.01	0.00247182208122277\\
564.01	0.00241954928651093\\
565.01	0.00236630308295206\\
566.01	0.00231207364607109\\
567.01	0.00225685192108934\\
568.01	0.00220062972740853\\
569.01	0.0021433998724014\\
570.01	0.00208515627504275\\
571.01	0.0020258940998783\\
572.01	0.00196560990176544\\
573.01	0.00190430178172618\\
574.01	0.00184196955411858\\
575.01	0.00177861492514706\\
576.01	0.00171424168248111\\
577.01	0.00164885589541889\\
578.01	0.00158246612459554\\
579.01	0.00151508363966999\\
580.01	0.00144672264269536\\
581.01	0.00137740049394915\\
582.01	0.0013071379358148\\
583.01	0.00123595930881451\\
584.01	0.00116389275200805\\
585.01	0.00109097037760846\\
586.01	0.00101722840670688\\
587.01	0.000942707249300885\\
588.01	0.000867451507214028\\
589.01	0.000791509872763252\\
590.01	0.000714934888910885\\
591.01	0.000637782527808906\\
592.01	0.000560111533702402\\
593.01	0.000481982462622189\\
594.01	0.000403456334556148\\
595.01	0.000324592793109807\\
596.01	0.000245447642136467\\
597.01	0.00016636800413272\\
598.01	9.17366970063782e-05\\
599.01	2.94669364271135e-05\\
599.02	2.89574269585688e-05\\
599.03	2.84509530852801e-05\\
599.04	2.79475443462508e-05\\
599.05	2.7447230571279e-05\\
599.06	2.69500418838293e-05\\
599.07	2.64560087039432e-05\\
599.08	2.59651617511691e-05\\
599.09	2.54775320475218e-05\\
599.1	2.49931509204854e-05\\
599.11	2.45120500060175e-05\\
599.12	2.40342612516185e-05\\
599.13	2.35598169193996e-05\\
599.14	2.30887495892024e-05\\
599.15	2.26210921617405e-05\\
599.16	2.215687786177e-05\\
599.17	2.16961402412976e-05\\
599.18	2.12389131828191e-05\\
599.19	2.07852309025824e-05\\
599.2	2.03351279538938e-05\\
599.21	1.98886392304542e-05\\
599.22	1.94457999697188e-05\\
599.23	1.90066457563046e-05\\
599.24	1.85712125254194e-05\\
599.25	1.81395365663351e-05\\
599.26	1.77116564095865e-05\\
599.27	1.72876122135658e-05\\
599.28	1.68674445361946e-05\\
599.29	1.64511943388859e-05\\
599.3	1.60389029905273e-05\\
599.31	1.56306122715087e-05\\
599.32	1.52263643777902e-05\\
599.33	1.48262019250105e-05\\
599.34	1.44301679526268e-05\\
599.35	1.40383059281154e-05\\
599.36	1.36506597511916e-05\\
599.37	1.32672737580847e-05\\
599.38	1.28881927258552e-05\\
599.39	1.25134618767404e-05\\
599.4	1.21431268825696e-05\\
599.41	1.17772338691924e-05\\
599.42	1.14158294209719e-05\\
599.43	1.10589605853174e-05\\
599.44	1.07066748772523e-05\\
599.45	1.03590202840433e-05\\
599.46	1.00160452698606e-05\\
599.47	9.67779878049101e-06\\
599.48	9.34433024810284e-06\\
599.49	9.01568959604283e-06\\
599.5	8.69192724369146e-06\\
599.51	8.3730941113757e-06\\
599.52	8.05924162529219e-06\\
599.53	7.75042172253965e-06\\
599.54	7.4466868561357e-06\\
599.55	7.14809000013257e-06\\
599.56	6.85468465475535e-06\\
599.57	6.56652485161134e-06\\
599.58	6.2836651589307e-06\\
599.59	6.00616068686249e-06\\
599.6	5.73406709284546e-06\\
599.61	5.46744058700296e-06\\
599.62	5.20633793760217e-06\\
599.63	4.95081647657741e-06\\
599.64	4.70093410508653e-06\\
599.65	4.45674929914347e-06\\
599.66	4.21832111529262e-06\\
599.67	3.98570919634376e-06\\
599.68	3.75897377716435e-06\\
599.69	3.53817569053241e-06\\
599.7	3.32337637303295e-06\\
599.71	3.11463787103054e-06\\
599.72	2.91202284668363e-06\\
599.73	2.71559458404555e-06\\
599.74	2.52541699518292e-06\\
599.75	2.34155462640155e-06\\
599.76	2.16407266449489e-06\\
599.77	1.99303694307928e-06\\
599.78	1.82851394897598e-06\\
599.79	1.67057082867475e-06\\
599.8	1.51927539483211e-06\\
599.81	1.37469613287027e-06\\
599.82	1.23690220760544e-06\\
599.83	1.10596346997172e-06\\
599.84	9.81950463784659e-07\\
599.85	8.64934432591793e-07\\
599.86	7.54987326588921e-07\\
599.87	6.52181809583305e-07\\
599.88	5.56591266062667e-07\\
599.89	4.68289808295413e-07\\
599.9	3.87352283521061e-07\\
599.91	3.13854281218731e-07\\
599.92	2.47872140425945e-07\\
599.93	1.8948295714763e-07\\
599.94	1.38764591834512e-07\\
599.95	9.57956769204876e-08\\
599.96	6.06556244606149e-08\\
599.97	3.34246338194039e-08\\
599.98	1.4183699454523e-08\\
599.99	3.014618757749e-09\\
600	0\\
};
\addplot [color=black,solid,forget plot]
  table[row sep=crcr]{%
0.01	0.00507509332097565\\
1.01	0.0050750923628945\\
2.01	0.00507509138515603\\
3.01	0.00507509038735957\\
4.01	0.00507508936909632\\
5.01	0.00507508832994861\\
6.01	0.00507508726949108\\
7.01	0.00507508618728923\\
8.01	0.00507508508290031\\
9.01	0.00507508395587187\\
10.01	0.00507508280574299\\
11.01	0.00507508163204302\\
12.01	0.00507508043429207\\
13.01	0.0050750792120006\\
14.01	0.00507507796466875\\
15.01	0.00507507669178709\\
16.01	0.00507507539283559\\
17.01	0.00507507406728384\\
18.01	0.00507507271459088\\
19.01	0.00507507133420444\\
20.01	0.00507506992556135\\
21.01	0.00507506848808715\\
22.01	0.00507506702119588\\
23.01	0.0050750655242895\\
24.01	0.00507506399675818\\
25.01	0.00507506243797951\\
26.01	0.00507506084731916\\
27.01	0.00507505922412902\\
28.01	0.0050750575677491\\
29.01	0.00507505587750485\\
30.01	0.00507505415270949\\
31.01	0.0050750523926618\\
32.01	0.0050750505966461\\
33.01	0.00507504876393276\\
34.01	0.00507504689377778\\
35.01	0.0050750449854215\\
36.01	0.00507504303808951\\
37.01	0.00507504105099175\\
38.01	0.00507503902332194\\
39.01	0.00507503695425819\\
40.01	0.00507503484296174\\
41.01	0.00507503268857721\\
42.01	0.00507503049023215\\
43.01	0.00507502824703633\\
44.01	0.00507502595808177\\
45.01	0.0050750236224424\\
46.01	0.00507502123917342\\
47.01	0.00507501880731144\\
48.01	0.00507501632587355\\
49.01	0.00507501379385709\\
50.01	0.0050750112102396\\
51.01	0.00507500857397792\\
52.01	0.00507500588400792\\
53.01	0.00507500313924481\\
54.01	0.00507500033858133\\
55.01	0.00507499748088886\\
56.01	0.00507499456501574\\
57.01	0.00507499158978734\\
58.01	0.00507498855400582\\
59.01	0.00507498545644934\\
60.01	0.00507498229587151\\
61.01	0.0050749790710015\\
62.01	0.0050749757805424\\
63.01	0.00507497242317229\\
64.01	0.00507496899754251\\
65.01	0.00507496550227766\\
66.01	0.00507496193597477\\
67.01	0.00507495829720301\\
68.01	0.00507495458450303\\
69.01	0.00507495079638682\\
70.01	0.00507494693133646\\
71.01	0.00507494298780371\\
72.01	0.00507493896420967\\
73.01	0.00507493485894434\\
74.01	0.00507493067036567\\
75.01	0.00507492639679894\\
76.01	0.0050749220365359\\
77.01	0.00507491758783472\\
78.01	0.00507491304891886\\
79.01	0.00507490841797665\\
80.01	0.00507490369316095\\
81.01	0.00507489887258731\\
82.01	0.00507489395433428\\
83.01	0.00507488893644271\\
84.01	0.00507488381691369\\
85.01	0.00507487859370976\\
86.01	0.00507487326475279\\
87.01	0.00507486782792364\\
88.01	0.00507486228106111\\
89.01	0.00507485662196151\\
90.01	0.00507485084837759\\
91.01	0.00507484495801778\\
92.01	0.00507483894854511\\
93.01	0.00507483281757652\\
94.01	0.00507482656268228\\
95.01	0.00507482018138451\\
96.01	0.00507481367115676\\
97.01	0.00507480702942243\\
98.01	0.00507480025355429\\
99.01	0.005074793340874\\
100.01	0.00507478628864997\\
101.01	0.00507477909409691\\
102.01	0.00507477175437477\\
103.01	0.00507476426658802\\
104.01	0.00507475662778401\\
105.01	0.00507474883495236\\
106.01	0.0050747408850232\\
107.01	0.00507473277486687\\
108.01	0.00507472450129263\\
109.01	0.00507471606104641\\
110.01	0.00507470745081088\\
111.01	0.00507469866720385\\
112.01	0.00507468970677664\\
113.01	0.00507468056601352\\
114.01	0.00507467124132977\\
115.01	0.00507466172907038\\
116.01	0.00507465202550914\\
117.01	0.0050746421268475\\
118.01	0.00507463202921244\\
119.01	0.00507462172865515\\
120.01	0.00507461122115039\\
121.01	0.00507460050259413\\
122.01	0.00507458956880223\\
123.01	0.00507457841550933\\
124.01	0.00507456703836704\\
125.01	0.00507455543294281\\
126.01	0.00507454359471697\\
127.01	0.00507453151908264\\
128.01	0.00507451920134384\\
129.01	0.00507450663671286\\
130.01	0.0050744938203094\\
131.01	0.00507448074715839\\
132.01	0.00507446741218821\\
133.01	0.0050744538102292\\
134.01	0.00507443993601179\\
135.01	0.00507442578416418\\
136.01	0.00507441134921064\\
137.01	0.00507439662557017\\
138.01	0.00507438160755303\\
139.01	0.00507436628936035\\
140.01	0.00507435066508125\\
141.01	0.00507433472869107\\
142.01	0.00507431847404841\\
143.01	0.00507430189489389\\
144.01	0.00507428498484734\\
145.01	0.00507426773740678\\
146.01	0.00507425014594418\\
147.01	0.00507423220370468\\
148.01	0.00507421390380356\\
149.01	0.00507419523922374\\
150.01	0.00507417620281398\\
151.01	0.00507415678728538\\
152.01	0.00507413698520982\\
153.01	0.00507411678901688\\
154.01	0.00507409619099173\\
155.01	0.005074075183271\\
156.01	0.00507405375784154\\
157.01	0.0050740319065375\\
158.01	0.00507400962103689\\
159.01	0.00507398689285877\\
160.01	0.00507396371336092\\
161.01	0.0050739400737363\\
162.01	0.00507391596501045\\
163.01	0.00507389137803769\\
164.01	0.00507386630349933\\
165.01	0.00507384073189902\\
166.01	0.00507381465356017\\
167.01	0.00507378805862289\\
168.01	0.00507376093704049\\
169.01	0.00507373327857531\\
170.01	0.00507370507279723\\
171.01	0.00507367630907735\\
172.01	0.0050736469765867\\
173.01	0.00507361706429175\\
174.01	0.00507358656095062\\
175.01	0.00507355545510958\\
176.01	0.00507352373509888\\
177.01	0.00507349138902891\\
178.01	0.00507345840478705\\
179.01	0.00507342477003229\\
180.01	0.00507339047219217\\
181.01	0.00507335549845812\\
182.01	0.00507331983578156\\
183.01	0.00507328347086898\\
184.01	0.00507324639017846\\
185.01	0.00507320857991437\\
186.01	0.00507317002602343\\
187.01	0.00507313071418995\\
188.01	0.00507309062983087\\
189.01	0.00507304975809168\\
190.01	0.00507300808384057\\
191.01	0.00507296559166482\\
192.01	0.00507292226586437\\
193.01	0.00507287809044797\\
194.01	0.00507283304912774\\
195.01	0.00507278712531304\\
196.01	0.00507274030210645\\
197.01	0.00507269256229743\\
198.01	0.00507264388835758\\
199.01	0.00507259426243429\\
200.01	0.00507254366634564\\
201.01	0.00507249208157421\\
202.01	0.00507243948926193\\
203.01	0.00507238587020346\\
204.01	0.00507233120484011\\
205.01	0.00507227547325413\\
206.01	0.00507221865516198\\
207.01	0.00507216072990919\\
208.01	0.00507210167646214\\
209.01	0.00507204147340324\\
210.01	0.00507198009892319\\
211.01	0.00507191753081424\\
212.01	0.00507185374646416\\
213.01	0.00507178872284889\\
214.01	0.00507172243652521\\
215.01	0.00507165486362391\\
216.01	0.00507158597984246\\
217.01	0.0050715157604375\\
218.01	0.00507144418021761\\
219.01	0.00507137121353524\\
220.01	0.0050712968342801\\
221.01	0.00507122101586983\\
222.01	0.00507114373124269\\
223.01	0.00507106495284985\\
224.01	0.005070984652647\\
225.01	0.00507090280208587\\
226.01	0.00507081937210558\\
227.01	0.00507073433312447\\
228.01	0.00507064765503149\\
229.01	0.00507055930717735\\
230.01	0.00507046925836458\\
231.01	0.00507037747684004\\
232.01	0.00507028393028489\\
233.01	0.00507018858580489\\
234.01	0.00507009140992138\\
235.01	0.00506999236856199\\
236.01	0.00506989142705026\\
237.01	0.00506978855009589\\
238.01	0.00506968370178452\\
239.01	0.00506957684556842\\
240.01	0.00506946794425484\\
241.01	0.00506935695999656\\
242.01	0.00506924385428079\\
243.01	0.00506912858791837\\
244.01	0.0050690111210332\\
245.01	0.00506889141305054\\
246.01	0.00506876942268595\\
247.01	0.00506864510793435\\
248.01	0.005068518426058\\
249.01	0.00506838933357479\\
250.01	0.00506825778624604\\
251.01	0.00506812373906494\\
252.01	0.00506798714624405\\
253.01	0.00506784796120308\\
254.01	0.00506770613655556\\
255.01	0.00506756162409706\\
256.01	0.00506741437479153\\
257.01	0.00506726433875824\\
258.01	0.00506711146525844\\
259.01	0.00506695570268196\\
260.01	0.00506679699853297\\
261.01	0.00506663529941704\\
262.01	0.00506647055102603\\
263.01	0.00506630269812392\\
264.01	0.00506613168453275\\
265.01	0.00506595745311698\\
266.01	0.00506577994576957\\
267.01	0.00506559910339558\\
268.01	0.00506541486589755\\
269.01	0.00506522717215955\\
270.01	0.00506503596003048\\
271.01	0.00506484116630971\\
272.01	0.00506464272672869\\
273.01	0.00506444057593556\\
274.01	0.00506423464747703\\
275.01	0.00506402487378258\\
276.01	0.00506381118614522\\
277.01	0.00506359351470526\\
278.01	0.00506337178843148\\
279.01	0.00506314593510227\\
280.01	0.00506291588128722\\
281.01	0.00506268155232887\\
282.01	0.00506244287232206\\
283.01	0.00506219976409476\\
284.01	0.00506195214918757\\
285.01	0.00506169994783382\\
286.01	0.00506144307893834\\
287.01	0.00506118146005622\\
288.01	0.0050609150073707\\
289.01	0.00506064363567161\\
290.01	0.00506036725833246\\
291.01	0.00506008578728751\\
292.01	0.00505979913300801\\
293.01	0.00505950720447802\\
294.01	0.00505920990917059\\
295.01	0.00505890715302186\\
296.01	0.00505859884040531\\
297.01	0.00505828487410596\\
298.01	0.00505796515529287\\
299.01	0.00505763958349196\\
300.01	0.00505730805655781\\
301.01	0.00505697047064422\\
302.01	0.00505662672017483\\
303.01	0.00505627669781245\\
304.01	0.00505592029442809\\
305.01	0.00505555739906764\\
306.01	0.00505518789892074\\
307.01	0.00505481167928527\\
308.01	0.00505442862353273\\
309.01	0.00505403861307289\\
310.01	0.00505364152731531\\
311.01	0.00505323724363369\\
312.01	0.00505282563732405\\
313.01	0.00505240658156526\\
314.01	0.00505197994737795\\
315.01	0.00505154560358038\\
316.01	0.0050511034167442\\
317.01	0.00505065325114879\\
318.01	0.00505019496873414\\
319.01	0.00504972842905117\\
320.01	0.00504925348921163\\
321.01	0.00504877000383523\\
322.01	0.00504827782499646\\
323.01	0.00504777680216768\\
324.01	0.00504726678216167\\
325.01	0.00504674760907182\\
326.01	0.00504621912421047\\
327.01	0.00504568116604355\\
328.01	0.00504513357012616\\
329.01	0.00504457616903229\\
330.01	0.00504400879228462\\
331.01	0.00504343126628027\\
332.01	0.00504284341421566\\
333.01	0.00504224505600726\\
334.01	0.00504163600820968\\
335.01	0.00504101608393246\\
336.01	0.00504038509275196\\
337.01	0.00503974284062201\\
338.01	0.00503908912978047\\
339.01	0.00503842375865283\\
340.01	0.00503774652175355\\
341.01	0.00503705720958282\\
342.01	0.00503635560852107\\
343.01	0.00503564150071973\\
344.01	0.00503491466398973\\
345.01	0.00503417487168396\\
346.01	0.00503342189258006\\
347.01	0.0050326554907563\\
348.01	0.00503187542546633\\
349.01	0.00503108145101045\\
350.01	0.00503027331660137\\
351.01	0.0050294507662298\\
352.01	0.00502861353852478\\
353.01	0.00502776136661102\\
354.01	0.00502689397796417\\
355.01	0.00502601109426194\\
356.01	0.00502511243123333\\
357.01	0.00502419769850433\\
358.01	0.00502326659944203\\
359.01	0.00502231883099567\\
360.01	0.00502135408353597\\
361.01	0.00502037204069237\\
362.01	0.00501937237918879\\
363.01	0.00501835476867738\\
364.01	0.00501731887157184\\
365.01	0.00501626434288031\\
366.01	0.00501519083003552\\
367.01	0.00501409797272657\\
368.01	0.00501298540273072\\
369.01	0.00501185274374292\\
370.01	0.00501069961120957\\
371.01	0.0050095256121601\\
372.01	0.00500833034503979\\
373.01	0.00500711339954497\\
374.01	0.00500587435645806\\
375.01	0.00500461278748491\\
376.01	0.00500332825509143\\
377.01	0.00500202031234377\\
378.01	0.00500068850274819\\
379.01	0.00499933236009091\\
380.01	0.00499795140827909\\
381.01	0.00499654516118146\\
382.01	0.00499511312246765\\
383.01	0.00499365478544587\\
384.01	0.00499216963289657\\
385.01	0.00499065713690462\\
386.01	0.00498911675868314\\
387.01	0.00498754794839435\\
388.01	0.00498595014495849\\
389.01	0.0049843227758546\\
390.01	0.00498266525690864\\
391.01	0.00498097699207024\\
392.01	0.0049792573731698\\
393.01	0.00497750577966067\\
394.01	0.00497572157834173\\
395.01	0.0049739041230573\\
396.01	0.00497205275437444\\
397.01	0.00497016679923682\\
398.01	0.00496824557059137\\
399.01	0.0049662883669906\\
400.01	0.00496429447216763\\
401.01	0.00496226315458532\\
402.01	0.00496019366695966\\
403.01	0.0049580852457607\\
404.01	0.0049559371106904\\
405.01	0.00495374846414219\\
406.01	0.00495151849064386\\
407.01	0.00494924635628814\\
408.01	0.00494693120815443\\
409.01	0.00494457217372436\\
410.01	0.00494216836029669\\
411.01	0.0049397188544016\\
412.01	0.00493722272122053\\
413.01	0.00493467900400838\\
414.01	0.00493208672352417\\
415.01	0.0049294448774652\\
416.01	0.00492675243990482\\
417.01	0.00492400836073521\\
418.01	0.00492121156510803\\
419.01	0.00491836095287491\\
420.01	0.00491545539802683\\
421.01	0.00491249374812972\\
422.01	0.00490947482375937\\
423.01	0.00490639741793495\\
424.01	0.00490326029554925\\
425.01	0.00490006219279999\\
426.01	0.00489680181662181\\
427.01	0.0048934778441154\\
428.01	0.00489008892197846\\
429.01	0.00488663366593589\\
430.01	0.00488311066016935\\
431.01	0.00487951845674879\\
432.01	0.0048758555750596\\
433.01	0.00487212050123231\\
434.01	0.00486831168756726\\
435.01	0.00486442755195876\\
436.01	0.00486046647731391\\
437.01	0.00485642681096679\\
438.01	0.00485230686408699\\
439.01	0.00484810491107931\\
440.01	0.00484381918897482\\
441.01	0.00483944789680955\\
442.01	0.00483498919499099\\
443.01	0.00483044120464899\\
444.01	0.0048258020069697\\
445.01	0.00482106964251116\\
446.01	0.00481624211049723\\
447.01	0.00481131736808979\\
448.01	0.00480629332963679\\
449.01	0.00480116786589522\\
450.01	0.00479593880322778\\
451.01	0.00479060392277257\\
452.01	0.00478516095958617\\
453.01	0.00477960760175957\\
454.01	0.00477394148950731\\
455.01	0.00476816021423241\\
456.01	0.0047622613175661\\
457.01	0.00475624229038595\\
458.01	0.00475010057181329\\
459.01	0.00474383354819188\\
460.01	0.00473743855205174\\
461.01	0.00473091286105579\\
462.01	0.0047242536969366\\
463.01	0.00471745822441844\\
464.01	0.00471052355013044\\
465.01	0.00470344672150507\\
466.01	0.0046962247256663\\
467.01	0.00468885448830111\\
468.01	0.0046813328725142\\
469.01	0.00467365667766195\\
470.01	0.00466582263816132\\
471.01	0.00465782742227181\\
472.01	0.00464966763084477\\
473.01	0.0046413397960396\\
474.01	0.00463284038000248\\
475.01	0.00462416577350806\\
476.01	0.00461531229456245\\
477.01	0.00460627618696779\\
478.01	0.00459705361884733\\
479.01	0.00458764068113383\\
480.01	0.00457803338601985\\
481.01	0.00456822766537132\\
482.01	0.00455821936910527\\
483.01	0.00454800426353266\\
484.01	0.00453757802966714\\
485.01	0.00452693626150026\\
486.01	0.00451607446424404\\
487.01	0.00450498805254091\\
488.01	0.00449367234864202\\
489.01	0.00448212258055315\\
490.01	0.00447033388014911\\
491.01	0.00445830128125419\\
492.01	0.00444601971769131\\
493.01	0.00443348402129692\\
494.01	0.00442068891990277\\
495.01	0.00440762903528331\\
496.01	0.00439429888107034\\
497.01	0.0043806928606338\\
498.01	0.0043668052649308\\
499.01	0.00435263027032252\\
500.01	0.0043381619363625\\
501.01	0.00432339420355518\\
502.01	0.00430832089108876\\
503.01	0.00429293569454225\\
504.01	0.00427723218357085\\
505.01	0.0042612037995689\\
506.01	0.00424484385331487\\
507.01	0.00422814552259942\\
508.01	0.0042111018498394\\
509.01	0.004193705739681\\
510.01	0.00417594995659516\\
511.01	0.00415782712246773\\
512.01	0.00413932971419067\\
513.01	0.00412045006125712\\
514.01	0.0041011803433656\\
515.01	0.00408151258804065\\
516.01	0.00406143866827475\\
517.01	0.00404095030019947\\
518.01	0.00402003904079373\\
519.01	0.00399869628563757\\
520.01	0.00397691326672204\\
521.01	0.0039546810503256\\
522.01	0.00393199053496824\\
523.01	0.00390883244945772\\
524.01	0.00388519735104193\\
525.01	0.00386107562368351\\
526.01	0.00383645747647488\\
527.01	0.00381133294221396\\
528.01	0.00378569187616113\\
529.01	0.00375952395500273\\
530.01	0.00373281867604811\\
531.01	0.00370556535668826\\
532.01	0.00367775313414919\\
533.01	0.00364937096557665\\
534.01	0.00362040762848954\\
535.01	0.00359085172164746\\
536.01	0.00356069166637926\\
537.01	0.00352991570842443\\
538.01	0.00349851192034694\\
539.01	0.0034664682045829\\
540.01	0.00343377229719395\\
541.01	0.00340041177240193\\
542.01	0.00336637404798946\\
543.01	0.00333164639165858\\
544.01	0.00329621592845059\\
545.01	0.00326006964933632\\
546.01	0.00322319442110026\\
547.01	0.00318557699765106\\
548.01	0.0031472040329063\\
549.01	0.00310806209541009\\
550.01	0.00306813768485848\\
551.01	0.00302741725072389\\
552.01	0.0029858872131864\\
553.01	0.00294353398659807\\
554.01	0.00290034400572704\\
555.01	0.00285630375504905\\
556.01	0.00281139980137633\\
557.01	0.00276561883013688\\
558.01	0.00271894768564463\\
559.01	0.00267137341572391\\
560.01	0.00262288332107934\\
561.01	0.00257346500983139\\
562.01	0.00252310645766198\\
563.01	0.00247179607404328\\
564.01	0.00241952277504812\\
565.01	0.00236627606326161\\
566.01	0.0023120461153342\\
567.01	0.0022568238777289\\
568.01	0.00220060117122217\\
569.01	0.00214337080471112\\
570.01	0.00208512669886227\\
571.01	0.00202586402009816\\
572.01	0.00196557932535577\\
573.01	0.00190427071795702\\
574.01	0.00184193801479703\\
575.01	0.00177858292487004\\
576.01	0.00171420923890147\\
577.01	0.00164882302952239\\
578.01	0.00158243286098413\\
579.01	0.00151505000684564\\
580.01	0.00144668867333727\\
581.01	0.00137736622517327\\
582.01	0.00130710340940336\\
583.01	0.00123592457139783\\
584.01	0.00116385785517657\\
585.01	0.00109093537792825\\
586.01	0.0010171933656031\\
587.01	0.0009426722327669\\
588.01	0.000867416585292753\\
589.01	0.000791475118736057\\
590.01	0.000714900378115771\\
591.01	0.000637748335993645\\
592.01	0.000560077734801639\\
593.01	0.000481949125828402\\
594.01	0.000403423520535174\\
595.01	0.000324560549193819\\
596.01	0.000245415996312425\\
597.01	0.000166366053398785\\
598.01	9.17366970063782e-05\\
599.01	2.94669364271135e-05\\
599.02	2.89574269585705e-05\\
599.03	2.84509530852801e-05\\
599.04	2.79475443462508e-05\\
599.05	2.7447230571279e-05\\
599.06	2.69500418838293e-05\\
599.07	2.64560087039432e-05\\
599.08	2.59651617511691e-05\\
599.09	2.54775320475235e-05\\
599.1	2.49931509204836e-05\\
599.11	2.45120500060175e-05\\
599.12	2.40342612516167e-05\\
599.13	2.35598169193978e-05\\
599.14	2.30887495892024e-05\\
599.15	2.26210921617405e-05\\
599.16	2.215687786177e-05\\
599.17	2.16961402412976e-05\\
599.18	2.12389131828191e-05\\
599.19	2.07852309025824e-05\\
599.2	2.03351279538938e-05\\
599.21	1.98886392304524e-05\\
599.22	1.94457999697188e-05\\
599.23	1.90066457563063e-05\\
599.24	1.85712125254211e-05\\
599.25	1.81395365663334e-05\\
599.26	1.77116564095865e-05\\
599.27	1.72876122135658e-05\\
599.28	1.68674445361946e-05\\
599.29	1.64511943388859e-05\\
599.3	1.60389029905273e-05\\
599.31	1.56306122715087e-05\\
599.32	1.5226364377792e-05\\
599.33	1.48262019250105e-05\\
599.34	1.44301679526268e-05\\
599.35	1.40383059281154e-05\\
599.36	1.36506597511916e-05\\
599.37	1.32672737580847e-05\\
599.38	1.28881927258535e-05\\
599.39	1.25134618767404e-05\\
599.4	1.21431268825696e-05\\
599.41	1.17772338691924e-05\\
599.42	1.14158294209736e-05\\
599.43	1.10589605853174e-05\\
599.44	1.07066748772523e-05\\
599.45	1.03590202840433e-05\\
599.46	1.00160452698589e-05\\
599.47	9.67779878049101e-06\\
599.48	9.34433024810111e-06\\
599.49	9.0156895960411e-06\\
599.5	8.69192724369319e-06\\
599.51	8.37309411137396e-06\\
599.52	8.05924162529392e-06\\
599.53	7.75042172253965e-06\\
599.54	7.44668685613396e-06\\
599.55	7.14809000013084e-06\\
599.56	6.85468465475535e-06\\
599.57	6.56652485161308e-06\\
599.58	6.28366515892896e-06\\
599.59	6.00616068686249e-06\\
599.6	5.73406709284546e-06\\
599.61	5.46744058700296e-06\\
599.62	5.20633793760217e-06\\
599.63	4.95081647657741e-06\\
599.64	4.70093410508653e-06\\
599.65	4.45674929914174e-06\\
599.66	4.21832111529089e-06\\
599.67	3.98570919634203e-06\\
599.68	3.75897377716608e-06\\
599.69	3.53817569053415e-06\\
599.7	3.32337637303469e-06\\
599.71	3.11463787102881e-06\\
599.72	2.91202284668363e-06\\
599.73	2.71559458404382e-06\\
599.74	2.52541699518466e-06\\
599.75	2.34155462640329e-06\\
599.76	2.16407266449663e-06\\
599.77	1.99303694307928e-06\\
599.78	1.82851394897598e-06\\
599.79	1.67057082867302e-06\\
599.8	1.51927539483385e-06\\
599.81	1.37469613287027e-06\\
599.82	1.23690220760718e-06\\
599.83	1.10596346997172e-06\\
599.84	9.81950463782924e-07\\
599.85	8.64934432591793e-07\\
599.86	7.54987326587186e-07\\
599.87	6.5218180958504e-07\\
599.88	5.56591266064402e-07\\
599.89	4.68289808295413e-07\\
599.9	3.87352283521061e-07\\
599.91	3.13854281216996e-07\\
599.92	2.4787214042421e-07\\
599.93	1.89482957149364e-07\\
599.94	1.38764591832777e-07\\
599.95	9.57956769222224e-08\\
599.96	6.06556244606149e-08\\
599.97	3.34246338211386e-08\\
599.98	1.4183699454523e-08\\
599.99	3.01461875948372e-09\\
600	0\\
};
\end{axis}
\end{tikzpicture}%

  \caption{Continuous Time}
\end{subfigure}%
\hfill%
\begin{subfigure}{.45\linewidth}
  \centering
  \setlength\figureheight{\linewidth} 
  \setlength\figurewidth{\linewidth}
  \tikzsetnextfilename{dm_dscr_z8}
  % This file was created by matlab2tikz.
%
%The latest updates can be retrieved from
%  http://www.mathworks.com/matlabcentral/fileexchange/22022-matlab2tikz-matlab2tikz
%where you can also make suggestions and rate matlab2tikz.
%
\definecolor{mycolor1}{rgb}{0.00000,1.00000,0.14286}%
\definecolor{mycolor2}{rgb}{0.00000,1.00000,0.28571}%
\definecolor{mycolor3}{rgb}{0.00000,1.00000,0.42857}%
\definecolor{mycolor4}{rgb}{0.00000,1.00000,0.57143}%
\definecolor{mycolor5}{rgb}{0.00000,1.00000,0.71429}%
\definecolor{mycolor6}{rgb}{0.00000,1.00000,0.85714}%
\definecolor{mycolor7}{rgb}{0.00000,1.00000,1.00000}%
\definecolor{mycolor8}{rgb}{0.00000,0.87500,1.00000}%
\definecolor{mycolor9}{rgb}{0.00000,0.62500,1.00000}%
\definecolor{mycolor10}{rgb}{0.12500,0.00000,1.00000}%
\definecolor{mycolor11}{rgb}{0.25000,0.00000,1.00000}%
\definecolor{mycolor12}{rgb}{0.37500,0.00000,1.00000}%
\definecolor{mycolor13}{rgb}{0.50000,0.00000,1.00000}%
\definecolor{mycolor14}{rgb}{0.62500,0.00000,1.00000}%
\definecolor{mycolor15}{rgb}{0.75000,0.00000,1.00000}%
\definecolor{mycolor16}{rgb}{0.87500,0.00000,1.00000}%
\definecolor{mycolor17}{rgb}{1.00000,0.00000,1.00000}%
\definecolor{mycolor18}{rgb}{1.00000,0.00000,0.87500}%
\definecolor{mycolor19}{rgb}{1.00000,0.00000,0.62500}%
\definecolor{mycolor20}{rgb}{0.85714,0.00000,0.00000}%
\definecolor{mycolor21}{rgb}{0.71429,0.00000,0.00000}%
%
\begin{tikzpicture}[trim axis left, trim axis right]

\begin{axis}[%
width=\figurewidth,
height=\figureheight,
at={(0\figurewidth,0\figureheight)},
scale only axis,
every outer x axis line/.append style={black},
every x tick label/.append style={font=\color{black}},
xmin=0,
xmax=600,
every outer y axis line/.append style={black},
every y tick label/.append style={font=\color{black}},
ymin=0,
ymax=0.014,
axis background/.style={fill=white},
axis x line*=bottom,
axis y line*=left,
yticklabel style={
        /pgf/number format/fixed,
        /pgf/number format/precision=3
},
scaled y ticks=false
]
\addplot [color=green,solid,forget plot]
  table[row sep=crcr]{%
1	0.00452315872884094\\
2	0.00452316046693674\\
3	0.00452316223750312\\
4	0.00452316404115143\\
5	0.00452316587850463\\
6	0.00452316775019734\\
7	0.00452316965687634\\
8	0.00452317159920071\\
9	0.00452317357784195\\
10	0.00452317559348431\\
11	0.00452317764682506\\
12	0.00452317973857474\\
13	0.00452318186945739\\
14	0.00452318404021077\\
15	0.00452318625158655\\
16	0.00452318850435089\\
17	0.0045231907992843\\
18	0.00452319313718228\\
19	0.00452319551885535\\
20	0.00452319794512945\\
21	0.00452320041684636\\
22	0.00452320293486368\\
23	0.00452320550005539\\
24	0.00452320811331204\\
25	0.00452321077554123\\
26	0.00452321348766779\\
27	0.00452321625063408\\
28	0.00452321906540037\\
29	0.00452322193294534\\
30	0.00452322485426628\\
31	0.00452322783037932\\
32	0.00452323086232009\\
33	0.00452323395114378\\
34	0.00452323709792584\\
35	0.00452324030376215\\
36	0.00452324356976944\\
37	0.00452324689708567\\
38	0.00452325028687062\\
39	0.00452325374030603\\
40	0.00452325725859632\\
41	0.00452326084296864\\
42	0.00452326449467375\\
43	0.00452326821498616\\
44	0.0045232720052047\\
45	0.00452327586665302\\
46	0.00452327980067996\\
47	0.00452328380866006\\
48	0.00452328789199416\\
49	0.00452329205210981\\
50	0.00452329629046181\\
51	0.0045233006085327\\
52	0.00452330500783329\\
53	0.00452330948990344\\
54	0.00452331405631215\\
55	0.00452331870865853\\
56	0.00452332344857221\\
57	0.00452332827771387\\
58	0.00452333319777617\\
59	0.0045233382104838\\
60	0.0045233433175946\\
61	0.00452334852089998\\
62	0.00452335382222545\\
63	0.0045233592234317\\
64	0.00452336472641475\\
65	0.00452337033310705\\
66	0.004523376045478\\
67	0.00452338186553446\\
68	0.00452338779532195\\
69	0.004523393836925\\
70	0.00452339999246799\\
71	0.0045234062641161\\
72	0.00452341265407587\\
73	0.0045234191645961\\
74	0.00452342579796865\\
75	0.00452343255652925\\
76	0.00452343944265848\\
77	0.00452344645878229\\
78	0.00452345360737332\\
79	0.00452346089095151\\
80	0.00452346831208512\\
81	0.00452347587339166\\
82	0.00452348357753866\\
83	0.00452349142724491\\
84	0.00452349942528142\\
85	0.00452350757447225\\
86	0.00452351587769559\\
87	0.0045235243378849\\
88	0.00452353295802987\\
89	0.00452354174117764\\
90	0.00452355069043371\\
91	0.00452355980896326\\
92	0.00452356909999225\\
93	0.00452357856680856\\
94	0.00452358821276313\\
95	0.00452359804127144\\
96	0.00452360805581433\\
97	0.00452361825993964\\
98	0.0045236286572635\\
99	0.00452363925147134\\
100	0.00452365004631944\\
101	0.00452366104563637\\
102	0.00452367225332424\\
103	0.00452368367336022\\
104	0.00452369530979804\\
105	0.00452370716676922\\
106	0.00452371924848483\\
107	0.00452373155923697\\
108	0.00452374410340033\\
109	0.0045237568854339\\
110	0.00452376990988233\\
111	0.00452378318137786\\
112	0.00452379670464186\\
113	0.00452381048448661\\
114	0.00452382452581721\\
115	0.00452383883363314\\
116	0.00452385341303017\\
117	0.00452386826920233\\
118	0.00452388340744382\\
119	0.00452389883315068\\
120	0.00452391455182317\\
121	0.00452393056906753\\
122	0.00452394689059812\\
123	0.00452396352223952\\
124	0.00452398046992876\\
125	0.00452399773971732\\
126	0.0045240153377735\\
127	0.00452403327038461\\
128	0.00452405154395938\\
129	0.00452407016503033\\
130	0.00452408914025606\\
131	0.00452410847642382\\
132	0.0045241281804519\\
133	0.00452414825939234\\
134	0.00452416872043348\\
135	0.00452418957090257\\
136	0.00452421081826844\\
137	0.00452423247014446\\
138	0.00452425453429121\\
139	0.00452427701861922\\
140	0.00452429993119225\\
141	0.0045243232802301\\
142	0.00452434707411149\\
143	0.00452437132137747\\
144	0.0045243960307344\\
145	0.00452442121105732\\
146	0.00452444687139296\\
147	0.00452447302096355\\
148	0.00452449966916991\\
149	0.00452452682559503\\
150	0.0045245545000078\\
151	0.00452458270236629\\
152	0.00452461144282191\\
153	0.00452464073172287\\
154	0.00452467057961811\\
155	0.00452470099726139\\
156	0.00452473199561524\\
157	0.00452476358585471\\
158	0.00452479577937223\\
159	0.00452482858778128\\
160	0.00452486202292111\\
161	0.00452489609686104\\
162	0.00452493082190476\\
163	0.00452496621059558\\
164	0.00452500227572051\\
165	0.00452503903031522\\
166	0.00452507648766919\\
167	0.00452511466133062\\
168	0.00452515356511137\\
169	0.00452519321309233\\
170	0.00452523361962866\\
171	0.00452527479935532\\
172	0.00452531676719259\\
173	0.00452535953835161\\
174	0.00452540312834027\\
175	0.00452544755296913\\
176	0.00452549282835722\\
177	0.00452553897093851\\
178	0.00452558599746792\\
179	0.00452563392502788\\
180	0.00452568277103469\\
181	0.00452573255324534\\
182	0.00452578328976438\\
183	0.00452583499905077\\
184	0.00452588769992498\\
185	0.00452594141157618\\
186	0.00452599615356996\\
187	0.00452605194585564\\
188	0.00452610880877411\\
189	0.00452616676306568\\
190	0.00452622582987839\\
191	0.004526286030776\\
192	0.00452634738774654\\
193	0.00452640992321107\\
194	0.00452647366003235\\
195	0.00452653862152379\\
196	0.00452660483145896\\
197	0.00452667231408077\\
198	0.00452674109411127\\
199	0.00452681119676146\\
200	0.00452688264774115\\
201	0.00452695547326951\\
202	0.00452702970008532\\
203	0.00452710535545801\\
204	0.00452718246719843\\
205	0.00452726106366999\\
206	0.00452734117380042\\
207	0.00452742282709338\\
208	0.00452750605364038\\
209	0.00452759088413332\\
210	0.00452767734987696\\
211	0.00452776548280188\\
212	0.00452785531547754\\
213	0.0045279468811259\\
214	0.00452804021363513\\
215	0.00452813534757391\\
216	0.00452823231820583\\
217	0.00452833116150417\\
218	0.00452843191416725\\
219	0.00452853461363391\\
220	0.00452863929809927\\
221	0.00452874600653135\\
222	0.00452885477868746\\
223	0.00452896565513163\\
224	0.00452907867725194\\
225	0.00452919388727849\\
226	0.00452931132830176\\
227	0.0045294310442915\\
228	0.00452955308011606\\
229	0.00452967748156208\\
230	0.00452980429535458\\
231	0.00452993356917786\\
232	0.00453006535169661\\
233	0.00453019969257749\\
234	0.00453033664251138\\
235	0.00453047625323617\\
236	0.00453061857755988\\
237	0.00453076366938419\\
238	0.00453091158372898\\
239	0.00453106237675683\\
240	0.00453121610579838\\
241	0.00453137282937816\\
242	0.00453153260724063\\
243	0.00453169550037713\\
244	0.00453186157105305\\
245	0.00453203088283551\\
246	0.00453220350062162\\
247	0.0045323794906666\\
248	0.00453255892061337\\
249	0.0045327418595212\\
250	0.00453292837789578\\
251	0.0045331185477188\\
252	0.00453331244247826\\
253	0.00453351013719849\\
254	0.00453371170847063\\
255	0.00453391723448326\\
256	0.00453412679505238\\
257	0.00453434047165204\\
258	0.00453455834744415\\
259	0.00453478050730869\\
260	0.00453500703787285\\
261	0.00453523802754061\\
262	0.00453547356652103\\
263	0.00453571374685653\\
264	0.00453595866245096\\
265	0.004536208409096\\
266	0.00453646308449827\\
267	0.00453672278830546\\
268	0.00453698762213224\\
269	0.00453725768958592\\
270	0.00453753309629288\\
271	0.00453781394992569\\
272	0.00453810036023167\\
273	0.00453839243906437\\
274	0.00453869030041973\\
275	0.00453899406047716\\
276	0.00453930383764456\\
277	0.00453961975259518\\
278	0.00453994192827682\\
279	0.00454027048994892\\
280	0.00454060556532613\\
281	0.00454094728470612\\
282	0.00454129578102249\\
283	0.00454165118989935\\
284	0.00454201364970598\\
285	0.00454238330161403\\
286	0.00454276028965466\\
287	0.00454314476077768\\
288	0.00454353686491168\\
289	0.00454393675502531\\
290	0.00454434458719027\\
291	0.00454476052064494\\
292	0.00454518471786017\\
293	0.00454561734460585\\
294	0.00454605857001922\\
295	0.00454650856667449\\
296	0.00454696751065401\\
297	0.004547435581621\\
298	0.00454791296289375\\
299	0.00454839984152137\\
300	0.00454889640836132\\
301	0.00454940285815846\\
302	0.0045499193896258\\
303	0.00455044620552731\\
304	0.00455098351276185\\
305	0.00455153152244971\\
306	0.00455209045002011\\
307	0.00455266051530151\\
308	0.00455324194261344\\
309	0.00455383496086004\\
310	0.00455443980362619\\
311	0.0045550567092753\\
312	0.00455568592104939\\
313	0.00455632768717136\\
314	0.00455698226094962\\
315	0.00455764990088462\\
316	0.00455833087077793\\
317	0.00455902543984399\\
318	0.00455973388282362\\
319	0.00456045648010068\\
320	0.00456119351782085\\
321	0.00456194528801333\\
322	0.00456271208871506\\
323	0.00456349422409742\\
324	0.00456429200459611\\
325	0.00456510574704381\\
326	0.0045659357748057\\
327	0.00456678241791769\\
328	0.0045676460132281\\
329	0.00456852690454239\\
330	0.00456942544277064\\
331	0.00457034198607908\\
332	0.00457127690004428\\
333	0.00457223055781079\\
334	0.00457320334025279\\
335	0.00457419563613845\\
336	0.00457520784229883\\
337	0.00457624036379983\\
338	0.0045772936141181\\
339	0.00457836801532107\\
340	0.00457946399825056\\
341	0.00458058200271057\\
342	0.00458172247765948\\
343	0.00458288588140608\\
344	0.00458407268181\\
345	0.00458528335648666\\
346	0.00458651839301686\\
347	0.00458777828916053\\
348	0.00458906355307579\\
349	0.00459037470354221\\
350	0.00459171227018967\\
351	0.00459307679373184\\
352	0.00459446882620498\\
353	0.00459588893121207\\
354	0.00459733768417267\\
355	0.00459881567257796\\
356	0.00460032349625215\\
357	0.0046018617676192\\
358	0.00460343111197661\\
359	0.0046050321677748\\
360	0.0046066655869037\\
361	0.0046083320349859\\
362	0.00461003219167716\\
363	0.00461176675097462\\
364	0.00461353642153235\\
365	0.00461534192698563\\
366	0.00461718400628351\\
367	0.00461906341403079\\
368	0.00462098092083953\\
369	0.00462293731369061\\
370	0.00462493339630607\\
371	0.00462696998953331\\
372	0.00462904793174128\\
373	0.00463116807922935\\
374	0.00463333130665076\\
375	0.00463553850745026\\
376	0.00463779059431784\\
377	0.00464008849965962\\
378	0.00464243317608621\\
379	0.00464482559692049\\
380	0.00464726675672593\\
381	0.00464975767185618\\
382	0.00465229938102637\\
383	0.0046548929459054\\
384	0.00465753945172617\\
385	0.00466024000791486\\
386	0.00466299574875455\\
387	0.00466580783409106\\
388	0.00466867745005077\\
389	0.00467160580976561\\
390	0.00467459415412751\\
391	0.00467764375257329\\
392	0.00468075590390244\\
393	0.00468393193712936\\
394	0.00468717321237244\\
395	0.004690481121782\\
396	0.00469385709050978\\
397	0.00469730257772228\\
398	0.0047008190776615\\
399	0.00470440812075528\\
400	0.00470807127478099\\
401	0.00471181014608572\\
402	0.00471562638086742\\
403	0.00471952166652048\\
404	0.00472349773304962\\
405	0.00472755635455839\\
406	0.00473169935081567\\
407	0.00473592858890621\\
408	0.00474024598497145\\
409	0.00474465350604619\\
410	0.00474915317199811\\
411	0.00475374705757748\\
412	0.00475843729458457\\
413	0.00476322607416346\\
414	0.00476811564923093\\
415	0.0047731083370499\\
416	0.00477820652195847\\
417	0.00478341265826482\\
418	0.00478872927332063\\
419	0.00479415897078559\\
420	0.00479970443409648\\
421	0.00480536843015597\\
422	0.00481115381325658\\
423	0.00481706352925816\\
424	0.00482310062003856\\
425	0.0048292682282373\\
426	0.00483556960231153\\
427	0.00484200810192902\\
428	0.00484858720372444\\
429	0.0048553105074458\\
430	0.00486218174252142\\
431	0.00486920477507797\\
432	0.00487638361544345\\
433	0.00488372242617159\\
434	0.00489122553062714\\
435	0.00489889742217275\\
436	0.00490674277400346\\
437	0.00491476644967619\\
438	0.00492297351438541\\
439	0.00493136924704052\\
440	0.00493995915320316\\
441	0.00494874897894669\\
442	0.00495774472570459\\
443	0.00496695266617787\\
444	0.00497637936137557\\
445	0.0049860316788681\\
446	0.00499591681233428\\
447	0.00500604230248929\\
448	0.00501641605948315\\
449	0.00502704638686232\\
450	0.00503794200718824\\
451	0.00504911208941001\\
452	0.00506056627808605\\
453	0.00507231472454848\\
454	0.00508436812010029\\
455	0.00509673773132658\\
456	0.00510943543759053\\
457	0.00512247377076895\\
458	0.00513586595725811\\
459	0.00514962596225105\\
460	0.00516376853624624\\
461	0.00517830926369315\\
462	0.00519326461361285\\
463	0.00520865199194336\\
464	0.00522448979525394\\
465	0.00524079746535624\\
466	0.00525759554424216\\
467	0.00527490572881899\\
468	0.00529275092545225\\
469	0.0053111553065269\\
470	0.00533014437869811\\
471	0.00534974509653556\\
472	0.00536998613150007\\
473	0.00539089864785317\\
474	0.00541251870713611\\
475	0.00543489489572283\\
476	0.00545808094742603\\
477	0.00548211710016086\\
478	0.00550704458460231\\
479	0.00553290603807106\\
480	0.0055597454522965\\
481	0.00558760791896004\\
482	0.00561653925504132\\
483	0.00564658548900369\\
484	0.0056777921705658\\
485	0.00571020345683139\\
486	0.00574386091531212\\
487	0.00577880196903098\\
488	0.00581505788958207\\
489	0.00585265121964843\\
490	0.00589159247564624\\
491	0.0059318759420463\\
492	0.00597347431896179\\
493	0.00601633191990093\\
494	0.00606035603019285\\
495	0.00610540591323029\\
496	0.00615127875127815\\
497	0.00619769141172501\\
498	0.00624425596333979\\
499	0.00629044409391648\\
500	0.00633552679506749\\
501	0.0063779860617938\\
502	0.00641495403470803\\
503	0.00644621125593945\\
504	0.00647225143934105\\
505	0.00649594148874345\\
506	0.00651842488242781\\
507	0.0065403977023012\\
508	0.00656240124457241\\
509	0.00658465185893241\\
510	0.00660724947477894\\
511	0.00663027228347454\\
512	0.00665376414919614\\
513	0.00667774682746756\\
514	0.00670223807200831\\
515	0.00672725333460772\\
516	0.00675280760071945\\
517	0.00677891777029354\\
518	0.00680560747851237\\
519	0.00683289727551202\\
520	0.00686080216120478\\
521	0.006889337902373\\
522	0.00691852119102528\\
523	0.00694836982578807\\
524	0.00697890293415211\\
525	0.00701014124534274\\
526	0.00704210742547359\\
527	0.00707482648705334\\
528	0.00710832627465372\\
529	0.00714263799293731\\
530	0.00717779661929973\\
531	0.00721329634049071\\
532	0.00724758846233135\\
533	0.00728053957675356\\
534	0.00731388166803929\\
535	0.00734770019297422\\
536	0.00738208290782488\\
537	0.00741706501967392\\
538	0.00745265251135492\\
539	0.00748884709593246\\
540	0.0075256492077746\\
541	0.00756305797465004\\
542	0.0076010711041356\\
543	0.00763968468798593\\
544	0.0076788929845296\\
545	0.00771868821588335\\
546	0.00775906047468624\\
547	0.00779999801066692\\
548	0.0078414886337878\\
549	0.00788351434448284\\
550	0.00792604782220941\\
551	0.00796905352480044\\
552	0.00801069465331129\\
553	0.00805192469198895\\
554	0.00809356139903176\\
555	0.0081356528678889\\
556	0.00817818447280016\\
557	0.00822113938111295\\
558	0.00826449920184153\\
559	0.00830824406656105\\
560	0.00835235276842425\\
561	0.00839680297132781\\
562	0.00844157148790236\\
563	0.00848663458413599\\
564	0.00853196814390219\\
565	0.00857754716278794\\
566	0.00862279530161032\\
567	0.00866833400681408\\
568	0.00871416234106978\\
569	0.00876025495322756\\
570	0.00880658414130209\\
571	0.00885311974160792\\
572	0.00889982902787483\\
573	0.00894667662535339\\
574	0.00899362444626615\\
575	0.00904063165462445\\
576	0.00908765467048829\\
577	0.00913464722628623\\
578	0.00918156049094262\\
579	0.00922834328141294\\
580	0.00927494238595959\\
581	0.00932130302929075\\
582	0.00936736951671607\\
583	0.00941308610287968\\
584	0.00945839814032784\\
585	0.00950325357339596\\
586	0.00954760485097041\\
587	0.00959141132963419\\
588	0.00963464220433353\\
589	0.00967727987662474\\
590	0.00971932328838559\\
591	0.0097606871825126\\
592	0.00980121135503816\\
593	0.00984070634831441\\
594	0.0098789201112722\\
595	0.00991544794299161\\
596	0.00994949078513196\\
597	0.00997920912842892\\
598	0.0100000808076597\\
599	0\\
600	0\\
};
\addplot [color=mycolor1,solid,forget plot]
  table[row sep=crcr]{%
1	0.00452207786763557\\
2	0.00452207970903561\\
3	0.00452208158462765\\
4	0.00452208349505152\\
5	0.0045220854409591\\
6	0.00452208742301457\\
7	0.00452208944189457\\
8	0.0045220914982886\\
9	0.00452209359289915\\
10	0.00452209572644192\\
11	0.00452209789964617\\
12	0.00452210011325487\\
13	0.00452210236802502\\
14	0.00452210466472795\\
15	0.00452210700414951\\
16	0.0045221093870905\\
17	0.00452211181436675\\
18	0.00452211428680951\\
19	0.00452211680526576\\
20	0.00452211937059846\\
21	0.00452212198368681\\
22	0.00452212464542671\\
23	0.00452212735673101\\
24	0.00452213011852987\\
25	0.00452213293177086\\
26	0.00452213579741959\\
27	0.00452213871645992\\
28	0.00452214168989434\\
29	0.00452214471874423\\
30	0.00452214780405033\\
31	0.0045221509468731\\
32	0.00452215414829315\\
33	0.00452215740941139\\
34	0.00452216073134963\\
35	0.00452216411525084\\
36	0.00452216756227983\\
37	0.00452217107362329\\
38	0.0045221746504905\\
39	0.00452217829411365\\
40	0.00452218200574808\\
41	0.00452218578667321\\
42	0.00452218963819251\\
43	0.00452219356163412\\
44	0.00452219755835155\\
45	0.00452220162972386\\
46	0.00452220577715619\\
47	0.00452221000208046\\
48	0.00452221430595568\\
49	0.00452221869026855\\
50	0.00452222315653389\\
51	0.00452222770629543\\
52	0.00452223234112614\\
53	0.0045222370626287\\
54	0.00452224187243635\\
55	0.00452224677221337\\
56	0.0045222517636555\\
57	0.00452225684849078\\
58	0.00452226202847993\\
59	0.00452226730541739\\
60	0.00452227268113135\\
61	0.00452227815748503\\
62	0.00452228373637687\\
63	0.00452228941974146\\
64	0.00452229520955011\\
65	0.00452230110781163\\
66	0.00452230711657294\\
67	0.00452231323792008\\
68	0.00452231947397857\\
69	0.00452232582691432\\
70	0.00452233229893468\\
71	0.00452233889228875\\
72	0.00452234560926842\\
73	0.00452235245220937\\
74	0.00452235942349166\\
75	0.00452236652554056\\
76	0.00452237376082745\\
77	0.00452238113187102\\
78	0.00452238864123761\\
79	0.00452239629154271\\
80	0.00452240408545143\\
81	0.00452241202567951\\
82	0.00452242011499482\\
83	0.00452242835621761\\
84	0.00452243675222199\\
85	0.00452244530593674\\
86	0.00452245402034666\\
87	0.00452246289849326\\
88	0.00452247194347612\\
89	0.00452248115845382\\
90	0.0045224905466452\\
91	0.00452250011133062\\
92	0.00452250985585271\\
93	0.00452251978361807\\
94	0.0045225298980982\\
95	0.0045225402028308\\
96	0.00452255070142118\\
97	0.00452256139754342\\
98	0.00452257229494148\\
99	0.00452258339743111\\
100	0.0045225947089007\\
101	0.00452260623331284\\
102	0.004522617974706\\
103	0.0045226299371955\\
104	0.00452264212497532\\
105	0.00452265454231963\\
106	0.00452266719358415\\
107	0.00452268008320796\\
108	0.00452269321571472\\
109	0.0045227065957146\\
110	0.00452272022790594\\
111	0.00452273411707682\\
112	0.00452274826810681\\
113	0.00452276268596887\\
114	0.00452277737573078\\
115	0.00452279234255741\\
116	0.00452280759171239\\
117	0.00452282312855991\\
118	0.00452283895856674\\
119	0.00452285508730439\\
120	0.00452287152045077\\
121	0.00452288826379246\\
122	0.00452290532322672\\
123	0.00452292270476384\\
124	0.00452294041452876\\
125	0.00452295845876401\\
126	0.00452297684383145\\
127	0.00452299557621464\\
128	0.00452301466252138\\
129	0.00452303410948595\\
130	0.00452305392397155\\
131	0.00452307411297281\\
132	0.00452309468361839\\
133	0.00452311564317338\\
134	0.004523136999042\\
135	0.00452315875877052\\
136	0.00452318093004958\\
137	0.00452320352071719\\
138	0.00452322653876152\\
139	0.00452324999232398\\
140	0.00452327388970195\\
141	0.00452329823935168\\
142	0.00452332304989161\\
143	0.0045233483301054\\
144	0.00452337408894503\\
145	0.00452340033553408\\
146	0.00452342707917124\\
147	0.00452345432933329\\
148	0.00452348209567875\\
149	0.00452351038805143\\
150	0.00452353921648377\\
151	0.0045235685912008\\
152	0.00452359852262356\\
153	0.00452362902137285\\
154	0.00452366009827336\\
155	0.00452369176435722\\
156	0.00452372403086827\\
157	0.00452375690926613\\
158	0.00452379041123002\\
159	0.00452382454866329\\
160	0.00452385933369759\\
161	0.00452389477869722\\
162	0.00452393089626379\\
163	0.00452396769924045\\
164	0.00452400520071673\\
165	0.00452404341403329\\
166	0.00452408235278657\\
167	0.00452412203083374\\
168	0.00452416246229764\\
169	0.00452420366157215\\
170	0.00452424564332674\\
171	0.0045242884225124\\
172	0.00452433201436654\\
173	0.0045243764344187\\
174	0.00452442169849599\\
175	0.00452446782272878\\
176	0.0045245148235564\\
177	0.00452456271773313\\
178	0.00452461152233399\\
179	0.00452466125476099\\
180	0.00452471193274926\\
181	0.00452476357437331\\
182	0.00452481619805351\\
183	0.00452486982256245\\
184	0.00452492446703195\\
185	0.00452498015095938\\
186	0.00452503689421492\\
187	0.00452509471704824\\
188	0.00452515364009591\\
189	0.00452521368438871\\
190	0.00452527487135878\\
191	0.00452533722284743\\
192	0.00452540076111275\\
193	0.00452546550883752\\
194	0.00452553148913714\\
195	0.00452559872556788\\
196	0.00452566724213488\\
197	0.00452573706330063\\
198	0.0045258082139944\\
199	0.00452588071962165\\
200	0.00452595460607363\\
201	0.00452602989973601\\
202	0.00452610662749867\\
203	0.00452618481676502\\
204	0.00452626449546195\\
205	0.00452634569204983\\
206	0.00452642843553249\\
207	0.00452651275546768\\
208	0.00452659868197782\\
209	0.00452668624576042\\
210	0.0045267754780994\\
211	0.00452686641087614\\
212	0.00452695907658104\\
213	0.00452705350832514\\
214	0.00452714973985222\\
215	0.00452724780555097\\
216	0.00452734774046733\\
217	0.00452744958031749\\
218	0.00452755336150077\\
219	0.00452765912111309\\
220	0.00452776689696069\\
221	0.00452787672757412\\
222	0.00452798865222266\\
223	0.00452810271092877\\
224	0.00452821894448373\\
225	0.00452833739446278\\
226	0.00452845810324135\\
227	0.00452858111401127\\
228	0.00452870647079789\\
229	0.00452883421847729\\
230	0.00452896440279457\\
231	0.00452909707038202\\
232	0.00452923226877861\\
233	0.00452937004644958\\
234	0.00452951045280718\\
235	0.00452965353823172\\
236	0.00452979935409377\\
237	0.00452994795277723\\
238	0.00453009938770284\\
239	0.00453025371335333\\
240	0.00453041098529911\\
241	0.00453057126022521\\
242	0.00453073459595957\\
243	0.00453090105150238\\
244	0.00453107068705706\\
245	0.00453124356406218\\
246	0.00453141974522535\\
247	0.00453159929455883\\
248	0.00453178227741608\\
249	0.00453196876053126\\
250	0.0045321588120593\\
251	0.00453235250161911\\
252	0.00453254990033783\\
253	0.00453275108089803\\
254	0.00453295611758626\\
255	0.004533165086344\\
256	0.00453337806482092\\
257	0.00453359513242978\\
258	0.00453381637040344\\
259	0.00453404186185353\\
260	0.00453427169183061\\
261	0.00453450594738515\\
262	0.0045347447176296\\
263	0.00453498809380035\\
264	0.00453523616931822\\
265	0.00453548903984839\\
266	0.00453574680335655\\
267	0.00453600956016101\\
268	0.0045362774129789\\
269	0.00453655046696416\\
270	0.00453682882973576\\
271	0.00453711261139279\\
272	0.00453740192451565\\
273	0.00453769688415137\\
274	0.00453799760778657\\
275	0.00453830421531886\\
276	0.004538616829053\\
277	0.00453893557375258\\
278	0.00453926057664216\\
279	0.0045395919662891\\
280	0.00453992987089679\\
281	0.00454027441941434\\
282	0.00454062574336298\\
283	0.00454098397688745\\
284	0.00454134925680933\\
285	0.00454172172268045\\
286	0.00454210151683848\\
287	0.00454248878446263\\
288	0.0045428836736313\\
289	0.00454328633538051\\
290	0.00454369692376366\\
291	0.0045441155959123\\
292	0.00454454251209863\\
293	0.00454497783579877\\
294	0.00454542173375777\\
295	0.00454587437605569\\
296	0.00454633593617506\\
297	0.00454680659107001\\
298	0.0045472865212365\\
299	0.00454777591078445\\
300	0.00454827494751082\\
301	0.00454878382297475\\
302	0.00454930273257408\\
303	0.00454983187562319\\
304	0.00455037145543324\\
305	0.0045509216793933\\
306	0.0045514827590537\\
307	0.00455205491021111\\
308	0.00455263835299518\\
309	0.00455323331195738\\
310	0.00455384001616163\\
311	0.00455445869927688\\
312	0.00455508959967174\\
313	0.00455573296051106\\
314	0.00455638902985485\\
315	0.00455705806075926\\
316	0.0045577403113798\\
317	0.0045584360450768\\
318	0.00455914553052346\\
319	0.00455986904181622\\
320	0.00456060685858734\\
321	0.00456135926612054\\
322	0.00456212655546889\\
323	0.00456290902357578\\
324	0.00456370697339837\\
325	0.00456452071403388\\
326	0.00456535056084919\\
327	0.00456619683561334\\
328	0.00456705986663289\\
329	0.00456793998889098\\
330	0.00456883754418938\\
331	0.00456975288129406\\
332	0.00457068635608438\\
333	0.00457163833170581\\
334	0.0045726091787264\\
335	0.00457359927529738\\
336	0.00457460900731717\\
337	0.00457563876860005\\
338	0.00457668896104881\\
339	0.00457775999483162\\
340	0.00457885228856365\\
341	0.00457996626949286\\
342	0.00458110237369096\\
343	0.00458226104624882\\
344	0.00458344274147711\\
345	0.00458464792311187\\
346	0.00458587706452501\\
347	0.00458713064894094\\
348	0.00458840916965753\\
349	0.00458971313027346\\
350	0.00459104304492074\\
351	0.0045923994385031\\
352	0.00459378284693983\\
353	0.00459519381741614\\
354	0.00459663290863856\\
355	0.00459810069109677\\
356	0.00459959774733094\\
357	0.00460112467220537\\
358	0.00460268207318746\\
359	0.00460427057063257\\
360	0.00460589079807496\\
361	0.00460754340252307\\
362	0.00460922904476137\\
363	0.0046109483996563\\
364	0.00461270215646754\\
365	0.00461449101916365\\
366	0.00461631570674194\\
367	0.00461817695355237\\
368	0.00462007550962498\\
369	0.00462201214100037\\
370	0.00462398763006374\\
371	0.00462600277588104\\
372	0.00462805839453806\\
373	0.00463015531948248\\
374	0.00463229440186787\\
375	0.00463447651090216\\
376	0.00463670253419952\\
377	0.00463897337813818\\
378	0.00464128996822604\\
379	0.00464365324947748\\
380	0.00464606418680574\\
381	0.00464852376543874\\
382	0.00465103299136724\\
383	0.00465359289183422\\
384	0.00465620451585897\\
385	0.00465886893473886\\
386	0.00466158724240846\\
387	0.00466436055593509\\
388	0.0046671900166725\\
389	0.00467007679140527\\
390	0.00467302207303266\\
391	0.00467602708127915\\
392	0.00467909306343182\\
393	0.00468222129510746\\
394	0.00468541308104989\\
395	0.00468866975596018\\
396	0.00469199268536117\\
397	0.00469538326649899\\
398	0.00469884292928305\\
399	0.00470237313726809\\
400	0.00470597538867996\\
401	0.00470965121748851\\
402	0.00471340219453075\\
403	0.00471722992868732\\
404	0.0047211360681166\\
405	0.00472512230154942\\
406	0.00472919035964957\\
407	0.00473334201644452\\
408	0.00473757909083108\\
409	0.00474190344816167\\
410	0.0047463170019176\\
411	0.00475082171547466\\
412	0.00475541960396921\\
413	0.00476011273627145\\
414	0.004764903237074\\
415	0.00476979328910541\\
416	0.00477478513547681\\
417	0.00477988108217257\\
418	0.00478508350069563\\
419	0.00479039483087892\\
420	0.00479581758387512\\
421	0.00480135434533621\\
422	0.0048070077787923\\
423	0.00481278062923831\\
424	0.00481867572694194\\
425	0.00482469599152052\\
426	0.0048308444363081\\
427	0.00483712417297317\\
428	0.00484353841640431\\
429	0.00485009048991847\\
430	0.00485678383081376\\
431	0.00486362199629287\\
432	0.00487060866978386\\
433	0.00487774766768602\\
434	0.00488504294657167\\
435	0.00489249861087657\\
436	0.0049001189211133\\
437	0.00490790830264392\\
438	0.00491587135505222\\
439	0.00492401286215595\\
440	0.00493233780270314\\
441	0.00494085136179931\\
442	0.00494955894311401\\
443	0.00495846618191822\\
444	0.00496757895900643\\
445	0.00497690341555866\\
446	0.00498644596900114\\
447	0.00499621332992477\\
448	0.00500621252012329\\
449	0.0050164508918155\\
450	0.00502693614811686\\
451	0.00503767636481767\\
452	0.0050486800135288\\
453	0.00505995598625684\\
454	0.00507151362146121\\
455	0.0050833627316404\\
456	0.0050955136324832\\
457	0.00510797717360826\\
458	0.00512076477089789\\
459	0.00513388844040813\\
460	0.00514736083380868\\
461	0.00516119527526913\\
462	0.00517540579966021\\
463	0.00519000719188261\\
464	0.00520501502706537\\
465	0.00522044571129635\\
466	0.00523631652246475\\
467	0.00525264565074371\\
468	0.00526945223830568\\
469	0.00528675641829778\\
470	0.00530457935456304\\
471	0.00532294328771034\\
472	0.00534187160358263\\
473	0.00536138896359307\\
474	0.00538152157685897\\
475	0.00540229770576962\\
476	0.00542375068111665\\
477	0.00544593431284566\\
478	0.0054688865391498\\
479	0.00549264278334905\\
480	0.00551723931321831\\
481	0.00554271303164222\\
482	0.00556910170302149\\
483	0.00559644372564849\\
484	0.00562477781141029\\
485	0.00565414256545888\\
486	0.00568457594788133\\
487	0.00571611459307592\\
488	0.00574879295741529\\
489	0.00578264226055108\\
490	0.0058176891797448\\
491	0.00585395424990462\\
492	0.00589144991451837\\
493	0.00593017816428257\\
494	0.00597012769051128\\
495	0.00601127046805938\\
496	0.00605355766419563\\
497	0.00609691473940739\\
498	0.00614123556207608\\
499	0.00618637536859615\\
500	0.00623214295550273\\
501	0.00627830856970893\\
502	0.00632461796596648\\
503	0.00637064668708425\\
504	0.00641567467857012\\
505	0.00645671676871352\\
506	0.00649279807232876\\
507	0.0065237014886583\\
508	0.00655016151929046\\
509	0.00657526352611622\\
510	0.00659937010473977\\
511	0.00662294536747645\\
512	0.0066466433476464\\
513	0.00667063634295364\\
514	0.00669501450747675\\
515	0.00671984429376435\\
516	0.00674517165239342\\
517	0.00677102675434827\\
518	0.00679743437258082\\
519	0.00682442093538079\\
520	0.00685200139201162\\
521	0.00688018979172476\\
522	0.00690900033568005\\
523	0.00693844765994201\\
524	0.00696854696431961\\
525	0.00699931416765851\\
526	0.00703076608002337\\
527	0.00706292056793817\\
528	0.0070957967454208\\
529	0.0071294151934158\\
530	0.007163798228666\\
531	0.00719898376652849\\
532	0.00723505305552559\\
533	0.00727207513737314\\
534	0.00730810867573445\\
535	0.00734290362955459\\
536	0.00737744449132286\\
537	0.00741243012050145\\
538	0.00744795226973005\\
539	0.00748407648654388\\
540	0.00752080901754714\\
541	0.00755815152183184\\
542	0.00759610281410559\\
543	0.00763466010033163\\
544	0.00767381896648215\\
545	0.0077135732249878\\
546	0.00775391485489104\\
547	0.00779483426923268\\
548	0.00783632159595532\\
549	0.00787836425375392\\
550	0.00792094055512449\\
551	0.00796402374945981\\
552	0.00800762177477663\\
553	0.00805044837838366\\
554	0.00809223040589093\\
555	0.00813437376803365\\
556	0.00817695262014052\\
557	0.00821996093904283\\
558	0.00826338056699623\\
559	0.00830719173003871\\
560	0.00835137316746913\\
561	0.0083959022891648\\
562	0.00844075539975523\\
563	0.00848590797942526\\
564	0.00853133496641271\\
565	0.0085770108849423\\
566	0.00862277750341835\\
567	0.0086683340068123\\
568	0.00871416234106959\\
569	0.00876025495322746\\
570	0.00880658414130204\\
571	0.00885311974160789\\
572	0.00889982902787481\\
573	0.00894667662535338\\
574	0.00899362444626613\\
575	0.00904063165462443\\
576	0.00908765467048829\\
577	0.00913464722628623\\
578	0.00918156049094263\\
579	0.00922834328141294\\
580	0.0092749423859596\\
581	0.00932130302929075\\
582	0.00936736951671608\\
583	0.00941308610287968\\
584	0.00945839814032785\\
585	0.00950325357339596\\
586	0.00954760485097041\\
587	0.00959141132963418\\
588	0.00963464220433353\\
589	0.00967727987662474\\
590	0.00971932328838558\\
591	0.0097606871825126\\
592	0.00980121135503816\\
593	0.00984070634831441\\
594	0.0098789201112722\\
595	0.00991544794299161\\
596	0.00994949078513196\\
597	0.00997920912842892\\
598	0.0100000808076597\\
599	0\\
600	0\\
};
\addplot [color=mycolor2,solid,forget plot]
  table[row sep=crcr]{%
1	0.00451871227606075\\
2	0.00451871437169823\\
3	0.00451871650576815\\
4	0.00451871867898057\\
5	0.00451872089205882\\
6	0.00451872314573974\\
7	0.00451872544077397\\
8	0.00451872777792605\\
9	0.00451873015797466\\
10	0.00451873258171324\\
11	0.00451873504994977\\
12	0.00451873756350744\\
13	0.00451874012322474\\
14	0.00451874272995574\\
15	0.00451874538457058\\
16	0.00451874808795546\\
17	0.00451875084101317\\
18	0.00451875364466338\\
19	0.00451875649984281\\
20	0.00451875940750582\\
21	0.00451876236862447\\
22	0.00451876538418902\\
23	0.00451876845520816\\
24	0.00451877158270938\\
25	0.00451877476773943\\
26	0.0045187780113646\\
27	0.00451878131467109\\
28	0.00451878467876536\\
29	0.00451878810477455\\
30	0.00451879159384689\\
31	0.00451879514715196\\
32	0.00451879876588131\\
33	0.00451880245124875\\
34	0.00451880620449078\\
35	0.004518810026867\\
36	0.00451881391966054\\
37	0.0045188178841786\\
38	0.0045188219217527\\
39	0.00451882603373931\\
40	0.00451883022152044\\
41	0.00451883448650373\\
42	0.00451883883012335\\
43	0.00451884325384019\\
44	0.00451884775914252\\
45	0.00451885234754646\\
46	0.00451885702059653\\
47	0.00451886177986622\\
48	0.00451886662695839\\
49	0.00451887156350594\\
50	0.00451887659117252\\
51	0.00451888171165278\\
52	0.00451888692667307\\
53	0.00451889223799228\\
54	0.00451889764740213\\
55	0.00451890315672795\\
56	0.00451890876782938\\
57	0.00451891448260083\\
58	0.00451892030297229\\
59	0.00451892623090981\\
60	0.00451893226841652\\
61	0.00451893841753291\\
62	0.00451894468033797\\
63	0.00451895105894953\\
64	0.00451895755552528\\
65	0.00451896417226332\\
66	0.00451897091140295\\
67	0.00451897777522571\\
68	0.00451898476605572\\
69	0.00451899188626101\\
70	0.00451899913825387\\
71	0.0045190065244919\\
72	0.004519014047479\\
73	0.00451902170976594\\
74	0.00451902951395145\\
75	0.00451903746268319\\
76	0.00451904555865848\\
77	0.0045190538046253\\
78	0.00451906220338336\\
79	0.00451907075778491\\
80	0.00451907947073585\\
81	0.00451908834519676\\
82	0.00451909738418377\\
83	0.00451910659076985\\
84	0.00451911596808564\\
85	0.00451912551932077\\
86	0.00451913524772471\\
87	0.00451914515660817\\
88	0.00451915524934406\\
89	0.0045191655293688\\
90	0.00451917600018353\\
91	0.00451918666535505\\
92	0.00451919752851753\\
93	0.00451920859337337\\
94	0.00451921986369475\\
95	0.00451923134332472\\
96	0.00451924303617883\\
97	0.00451925494624617\\
98	0.00451926707759104\\
99	0.00451927943435424\\
100	0.00451929202075457\\
101	0.00451930484109017\\
102	0.0045193178997402\\
103	0.00451933120116622\\
104	0.00451934474991389\\
105	0.00451935855061441\\
106	0.00451937260798628\\
107	0.00451938692683676\\
108	0.00451940151206385\\
109	0.00451941636865773\\
110	0.00451943150170245\\
111	0.004519446916378\\
112	0.00451946261796198\\
113	0.00451947861183122\\
114	0.00451949490346414\\
115	0.00451951149844222\\
116	0.00451952840245213\\
117	0.00451954562128766\\
118	0.00451956316085183\\
119	0.0045195810271588\\
120	0.00451959922633618\\
121	0.00451961776462707\\
122	0.00451963664839217\\
123	0.0045196558841119\\
124	0.00451967547838909\\
125	0.00451969543795066\\
126	0.00451971576965055\\
127	0.00451973648047193\\
128	0.00451975757752933\\
129	0.00451977906807145\\
130	0.00451980095948359\\
131	0.00451982325929025\\
132	0.00451984597515763\\
133	0.00451986911489648\\
134	0.00451989268646479\\
135	0.00451991669797028\\
136	0.00451994115767365\\
137	0.00451996607399107\\
138	0.00451999145549747\\
139	0.00452001731092922\\
140	0.00452004364918725\\
141	0.0045200704793401\\
142	0.00452009781062741\\
143	0.00452012565246256\\
144	0.00452015401443642\\
145	0.00452018290632045\\
146	0.00452021233807006\\
147	0.00452024231982816\\
148	0.00452027286192869\\
149	0.00452030397490012\\
150	0.00452033566946919\\
151	0.00452036795656438\\
152	0.00452040084732011\\
153	0.00452043435308025\\
154	0.00452046848540207\\
155	0.00452050325606043\\
156	0.00452053867705158\\
157	0.00452057476059748\\
158	0.00452061151914998\\
159	0.00452064896539512\\
160	0.00452068711225732\\
161	0.00452072597290406\\
162	0.00452076556075027\\
163	0.00452080588946287\\
164	0.00452084697296563\\
165	0.00452088882544375\\
166	0.00452093146134889\\
167	0.00452097489540375\\
168	0.00452101914260761\\
169	0.00452106421824077\\
170	0.00452111013787044\\
171	0.00452115691735537\\
172	0.00452120457285159\\
173	0.00452125312081774\\
174	0.00452130257802053\\
175	0.00452135296154039\\
176	0.00452140428877716\\
177	0.00452145657745593\\
178	0.00452150984563285\\
179	0.00452156411170107\\
180	0.0045216193943967\\
181	0.00452167571280506\\
182	0.00452173308636685\\
183	0.00452179153488429\\
184	0.00452185107852739\\
185	0.0045219117378409\\
186	0.00452197353375003\\
187	0.0045220364875675\\
188	0.00452210062099993\\
189	0.00452216595615454\\
190	0.00452223251554586\\
191	0.00452230032210258\\
192	0.00452236939917453\\
193	0.00452243977053937\\
194	0.00452251146041026\\
195	0.00452258449344212\\
196	0.00452265889473721\\
197	0.00452273468984554\\
198	0.004522811904759\\
199	0.00452289056592842\\
200	0.00452297070028416\\
201	0.00452305233524483\\
202	0.00452313549872512\\
203	0.00452322021914394\\
204	0.00452330652543267\\
205	0.00452339444704322\\
206	0.00452348401395655\\
207	0.00452357525669081\\
208	0.00452366820631\\
209	0.00452376289443239\\
210	0.00452385935323907\\
211	0.00452395761548251\\
212	0.00452405771449536\\
213	0.00452415968419912\\
214	0.00452426355911273\\
215	0.00452436937436143\\
216	0.00452447716568565\\
217	0.00452458696944978\\
218	0.00452469882265093\\
219	0.00452481276292773\\
220	0.0045249288285693\\
221	0.00452504705852401\\
222	0.00452516749240818\\
223	0.00452529017051515\\
224	0.00452541513382381\\
225	0.00452554242400735\\
226	0.00452567208344202\\
227	0.0045258041552159\\
228	0.0045259386831372\\
229	0.00452607571174306\\
230	0.00452621528630811\\
231	0.00452635745285289\\
232	0.0045265022581521\\
233	0.00452664974974353\\
234	0.00452679997593594\\
235	0.00452695298581789\\
236	0.00452710882926602\\
237	0.00452726755695365\\
238	0.00452742922035963\\
239	0.0045275938717771\\
240	0.00452776156432252\\
241	0.00452793235194537\\
242	0.00452810628943796\\
243	0.00452828343244594\\
244	0.00452846383747958\\
245	0.00452864756192603\\
246	0.00452883466406257\\
247	0.00452902520307134\\
248	0.00452921923905597\\
249	0.00452941683305987\\
250	0.00452961804708762\\
251	0.00452982294412909\\
252	0.00453003158818709\\
253	0.00453024404430957\\
254	0.00453046037862689\\
255	0.00453068065839513\\
256	0.00453090495204642\\
257	0.00453113332924775\\
258	0.00453136586096965\\
259	0.00453160261956582\\
260	0.00453184367886637\\
261	0.00453208911428605\\
262	0.00453233900294993\\
263	0.00453259342383914\\
264	0.00453285245795973\\
265	0.00453311618853711\\
266	0.0045333847012398\\
267	0.00453365808443593\\
268	0.004533936429486\\
269	0.00453421983107587\\
270	0.00453450838759292\\
271	0.00453480220154948\\
272	0.00453510138005671\\
273	0.00453540603535461\\
274	0.00453571628541195\\
275	0.00453603225464463\\
276	0.00453635407492285\\
277	0.0045366818874785\\
278	0.00453701584789232\\
279	0.00453735614202018\\
280	0.00453770297168611\\
281	0.0045380564961873\\
282	0.00453841684500994\\
283	0.00453878415014425\\
284	0.00453915854613261\\
285	0.00453954017011914\\
286	0.00453992916189902\\
287	0.00454032566397016\\
288	0.00454072982158479\\
289	0.00454114178280252\\
290	0.00454156169854417\\
291	0.00454198972264688\\
292	0.00454242601191999\\
293	0.00454287072620233\\
294	0.00454332402842006\\
295	0.00454378608464613\\
296	0.00454425706416068\\
297	0.00454473713951236\\
298	0.00454522648658109\\
299	0.00454572528464215\\
300	0.00454623371643095\\
301	0.00454675196820951\\
302	0.00454728022983401\\
303	0.00454781869482367\\
304	0.00454836756043091\\
305	0.00454892702771289\\
306	0.00454949730160458\\
307	0.00455007859099266\\
308	0.00455067110879168\\
309	0.00455127507202108\\
310	0.00455189070188381\\
311	0.00455251822384663\\
312	0.00455315786772211\\
313	0.00455380986775178\\
314	0.00455447446269133\\
315	0.00455515189589738\\
316	0.00455584241541609\\
317	0.00455654627407334\\
318	0.00455726372956697\\
319	0.00455799504456073\\
320	0.00455874048678061\\
321	0.0045595003291126\\
322	0.004560274849703\\
323	0.00456106433206073\\
324	0.00456186906516168\\
325	0.0045626893435561\\
326	0.00456352546747743\\
327	0.00456437774295423\\
328	0.00456524648192487\\
329	0.00456613200235439\\
330	0.00456703462835491\\
331	0.00456795469030839\\
332	0.00456889252499278\\
333	0.00456984847571166\\
334	0.00457082289242712\\
335	0.00457181613189606\\
336	0.00457282855781097\\
337	0.00457386054094396\\
338	0.0045749124592957\\
339	0.00457598469824853\\
340	0.00457707765072454\\
341	0.00457819171734856\\
342	0.0045793273066167\\
343	0.00458048483507056\\
344	0.00458166472747747\\
345	0.00458286741701738\\
346	0.00458409334547661\\
347	0.00458534296344845\\
348	0.00458661673054249\\
349	0.00458791511560153\\
350	0.00458923859692749\\
351	0.00459058766251702\\
352	0.0045919628103069\\
353	0.00459336454842965\\
354	0.00459479339548127\\
355	0.00459624988080001\\
356	0.00459773454475814\\
357	0.00459924793906671\\
358	0.00460079062709405\\
359	0.00460236318419905\\
360	0.00460396619807856\\
361	0.00460560026913146\\
362	0.00460726601083729\\
363	0.00460896405015152\\
364	0.00461069502791711\\
365	0.00461245959929127\\
366	0.00461425843418817\\
367	0.00461609221773589\\
368	0.00461796165074648\\
369	0.00461986745019802\\
370	0.00462181034972421\\
371	0.00462379110011039\\
372	0.00462581046979022\\
373	0.00462786924533773\\
374	0.00462996823194914\\
375	0.00463210825390427\\
376	0.0046342901549999\\
377	0.00463651479894229\\
378	0.00463878306968645\\
379	0.0046410958717084\\
380	0.00464345413019867\\
381	0.00464585879117171\\
382	0.00464831082150964\\
383	0.00465081120900912\\
384	0.00465336096259248\\
385	0.00465596111287649\\
386	0.00465861271248852\\
387	0.0046613168297418\\
388	0.00466407453808965\\
389	0.00466688692293326\\
390	0.00466975509344073\\
391	0.00467268018320013\\
392	0.00467566335089747\\
393	0.0046787057810201\\
394	0.0046818086845878\\
395	0.00468497329991092\\
396	0.00468820089337773\\
397	0.0046914927602709\\
398	0.00469485022561597\\
399	0.00469827464506302\\
400	0.00470176740580243\\
401	0.0047053299275175\\
402	0.00470896366337447\\
403	0.00471267010105296\\
404	0.0047164507638179\\
405	0.00472030721163566\\
406	0.00472424104233644\\
407	0.00472825389282557\\
408	0.00473234744034646\\
409	0.00473652340379741\\
410	0.00474078354510603\\
411	0.0047451296706654\\
412	0.00474956363283557\\
413	0.00475408733151605\\
414	0.00475870271579472\\
415	0.00476341178568049\\
416	0.00476821659392966\\
417	0.00477311924797598\\
418	0.00477812191197586\\
419	0.00478322680898394\\
420	0.00478843622328274\\
421	0.00479375250289279\\
422	0.00479917806228898\\
423	0.00480471538531341\\
424	0.00481036702817456\\
425	0.00481613562230699\\
426	0.0048220238778347\\
427	0.00482803458839506\\
428	0.00483417063595764\\
429	0.00484043499511429\\
430	0.00484683073764833\\
431	0.00485336103740203\\
432	0.00486002917546322\\
433	0.00486683854569523\\
434	0.00487379266063422\\
435	0.00488089515777991\\
436	0.00488814980630837\\
437	0.004895560514237\\
438	0.00490313133607284\\
439	0.00491086648097955\\
440	0.00491877032149859\\
441	0.00492684740286328\\
442	0.00493510245294659\\
443	0.004943540392886\\
444	0.00495216634843046\\
445	0.00496098566205728\\
446	0.00497000390590711\\
447	0.00497922689558483\\
448	0.00498866070486961\\
449	0.00499831168138\\
450	0.00500818646326957\\
451	0.00501829199708403\\
452	0.00502863555674415\\
453	0.00503922476364523\\
454	0.00505006760801625\\
455	0.00506117247159533\\
456	0.0050725481516778\\
457	0.00508420388658775\\
458	0.0050961493826191\\
459	0.0051083948424843\\
460	0.00512095099529726\\
461	0.00513382912810421\\
462	0.00514704111895647\\
463	0.00516059947149717\\
464	0.00517451735100407\\
465	0.00518880862179514\\
466	0.00520348788586309\\
467	0.0052185705225616\\
468	0.00523407272913598\\
469	0.00525001156189977\\
470	0.00526640497798165\\
471	0.00528327187800003\\
472	0.0053006321512189\\
473	0.00531850672786927\\
474	0.00533691765219881\\
475	0.00535588821953851\\
476	0.00537544318427549\\
477	0.00539560847498372\\
478	0.00541641436174454\\
479	0.00543790409154836\\
480	0.00546012348107261\\
481	0.0054831068312783\\
482	0.00550688990804856\\
483	0.0055315092460872\\
484	0.00555700258428041\\
485	0.00558340888960582\\
486	0.00561076822732668\\
487	0.00563912155208316\\
488	0.00566851042459942\\
489	0.00569897663524637\\
490	0.0057305617121639\\
491	0.00576330628566981\\
492	0.00579724927371497\\
493	0.00583242684427443\\
494	0.00586887109923844\\
495	0.0059066084098122\\
496	0.00594565731476752\\
497	0.00598602586954296\\
498	0.00602770830699975\\
499	0.0060706808417529\\
500	0.00611489639880138\\
501	0.0061602773956612\\
502	0.00620670579938571\\
503	0.0062540151150381\\
504	0.00630198388363382\\
505	0.00635038754980388\\
506	0.00639889021442986\\
507	0.0064469467189464\\
508	0.0064933046384227\\
509	0.00653475283368234\\
510	0.00657114786619823\\
511	0.00660260954618946\\
512	0.00663068986358215\\
513	0.00665749850387072\\
514	0.00668342678504002\\
515	0.00670891993780846\\
516	0.00673443880018458\\
517	0.00676029179365272\\
518	0.00678657072420855\\
519	0.00681334703880634\\
520	0.00684067124028119\\
521	0.00686857938846476\\
522	0.00689709178734309\\
523	0.00692622550812766\\
524	0.00695599714640317\\
525	0.00698642276054375\\
526	0.00701751826078969\\
527	0.0070493002254388\\
528	0.00708178609200776\\
529	0.00711499438550752\\
530	0.00714894500096339\\
531	0.00718365917320378\\
532	0.00721915874476496\\
533	0.00725546679015577\\
534	0.00729265742703601\\
535	0.00733082009894565\\
536	0.00736876606329719\\
537	0.00740560689150911\\
538	0.00744141267628468\\
539	0.00747763904614808\\
540	0.00751436412814335\\
541	0.00755167096489601\\
542	0.007589590593458\\
543	0.00762812584658022\\
544	0.0076672742280159\\
545	0.00770703145150301\\
546	0.00774739145822383\\
547	0.00778834669001185\\
548	0.00782988895145769\\
549	0.00787201070283719\\
550	0.00791469435950867\\
551	0.00795791814140556\\
552	0.008001654211164\\
553	0.00804589692883487\\
554	0.00809005106929018\\
555	0.00813262470712376\\
556	0.00817532413906588\\
557	0.00821841047972486\\
558	0.00826191490043305\\
559	0.00830581825873248\\
560	0.00835009906871955\\
561	0.0083947342775151\\
562	0.00843969944453349\\
563	0.00848496898822595\\
564	0.00853051649765713\\
565	0.00857631505397157\\
566	0.00862234018376705\\
567	0.0086683185450495\\
568	0.00871416234104972\\
569	0.00876025495322539\\
570	0.00880658414130106\\
571	0.00885311974160766\\
572	0.00889982902787472\\
573	0.00894667662535334\\
574	0.00899362444626613\\
575	0.00904063165462444\\
576	0.0090876546704883\\
577	0.00913464722628623\\
578	0.00918156049094263\\
579	0.00922834328141294\\
580	0.00927494238595959\\
581	0.00932130302929075\\
582	0.00936736951671607\\
583	0.00941308610287967\\
584	0.00945839814032784\\
585	0.00950325357339596\\
586	0.00954760485097041\\
587	0.00959141132963419\\
588	0.00963464220433353\\
589	0.00967727987662474\\
590	0.00971932328838558\\
591	0.0097606871825126\\
592	0.00980121135503816\\
593	0.00984070634831441\\
594	0.0098789201112722\\
595	0.00991544794299161\\
596	0.00994949078513196\\
597	0.00997920912842892\\
598	0.0100000808076597\\
599	0\\
600	0\\
};
\addplot [color=mycolor3,solid,forget plot]
  table[row sep=crcr]{%
1	0.00451009841467741\\
2	0.00451010094377074\\
3	0.00451010351851788\\
4	0.00451010613974862\\
5	0.00451010880830794\\
6	0.00451011152505622\\
7	0.0045101142908696\\
8	0.00451011710664034\\
9	0.00451011997327703\\
10	0.00451012289170479\\
11	0.00451012586286582\\
12	0.0045101288877195\\
13	0.00451013196724285\\
14	0.00451013510243078\\
15	0.00451013829429642\\
16	0.00451014154387144\\
17	0.00451014485220647\\
18	0.00451014822037139\\
19	0.00451015164945574\\
20	0.00451015514056905\\
21	0.00451015869484111\\
22	0.00451016231342251\\
23	0.00451016599748497\\
24	0.00451016974822167\\
25	0.00451017356684774\\
26	0.00451017745460055\\
27	0.00451018141274023\\
28	0.00451018544255005\\
29	0.00451018954533685\\
30	0.00451019372243137\\
31	0.00451019797518903\\
32	0.00451020230498993\\
33	0.00451020671323959\\
34	0.00451021120136939\\
35	0.00451021577083701\\
36	0.00451022042312691\\
37	0.0045102251597508\\
38	0.00451022998224831\\
39	0.00451023489218734\\
40	0.00451023989116448\\
41	0.00451024498080586\\
42	0.00451025016276744\\
43	0.00451025543873567\\
44	0.00451026081042807\\
45	0.00451026627959367\\
46	0.00451027184801385\\
47	0.00451027751750263\\
48	0.00451028328990747\\
49	0.00451028916710988\\
50	0.00451029515102595\\
51	0.00451030124360717\\
52	0.00451030744684094\\
53	0.00451031376275121\\
54	0.00451032019339926\\
55	0.0045103267408844\\
56	0.00451033340734449\\
57	0.00451034019495698\\
58	0.00451034710593926\\
59	0.00451035414254983\\
60	0.00451036130708864\\
61	0.00451036860189819\\
62	0.00451037602936399\\
63	0.00451038359191581\\
64	0.00451039129202804\\
65	0.00451039913222076\\
66	0.00451040711506073\\
67	0.00451041524316178\\
68	0.00451042351918625\\
69	0.00451043194584536\\
70	0.0045104405259006\\
71	0.00451044926216438\\
72	0.00451045815750101\\
73	0.0045104672148277\\
74	0.00451047643711562\\
75	0.00451048582739067\\
76	0.00451049538873472\\
77	0.00451050512428664\\
78	0.00451051503724313\\
79	0.0045105251308601\\
80	0.0045105354084536\\
81	0.00451054587340103\\
82	0.00451055652914205\\
83	0.00451056737918003\\
84	0.00451057842708306\\
85	0.00451058967648521\\
86	0.00451060113108781\\
87	0.00451061279466051\\
88	0.00451062467104277\\
89	0.00451063676414503\\
90	0.00451064907795005\\
91	0.0045106616165143\\
92	0.00451067438396925\\
93	0.0045106873845228\\
94	0.00451070062246071\\
95	0.00451071410214814\\
96	0.00451072782803084\\
97	0.00451074180463704\\
98	0.00451075603657866\\
99	0.0045107705285529\\
100	0.00451078528534404\\
101	0.00451080031182493\\
102	0.00451081561295842\\
103	0.0045108311937995\\
104	0.00451084705949653\\
105	0.00451086321529326\\
106	0.00451087966653053\\
107	0.00451089641864809\\
108	0.00451091347718625\\
109	0.00451093084778805\\
110	0.00451094853620102\\
111	0.00451096654827903\\
112	0.00451098488998432\\
113	0.00451100356738971\\
114	0.00451102258668017\\
115	0.00451104195415541\\
116	0.0045110616762317\\
117	0.0045110817594442\\
118	0.00451110221044893\\
119	0.00451112303602524\\
120	0.00451114424307801\\
121	0.00451116583863977\\
122	0.00451118782987345\\
123	0.00451121022407453\\
124	0.00451123302867348\\
125	0.00451125625123837\\
126	0.00451127989947738\\
127	0.00451130398124114\\
128	0.00451132850452584\\
129	0.00451135347747552\\
130	0.00451137890838494\\
131	0.00451140480570226\\
132	0.00451143117803204\\
133	0.00451145803413803\\
134	0.00451148538294595\\
135	0.00451151323354675\\
136	0.00451154159519941\\
137	0.00451157047733418\\
138	0.00451159988955562\\
139	0.00451162984164593\\
140	0.00451166034356796\\
141	0.00451169140546902\\
142	0.00451172303768364\\
143	0.00451175525073747\\
144	0.00451178805535055\\
145	0.00451182146244104\\
146	0.00451185548312867\\
147	0.00451189012873858\\
148	0.00451192541080508\\
149	0.0045119613410753\\
150	0.00451199793151337\\
151	0.00451203519430418\\
152	0.0045120731418574\\
153	0.00451211178681189\\
154	0.00451215114203954\\
155	0.00451219122064972\\
156	0.00451223203599368\\
157	0.00451227360166889\\
158	0.00451231593152342\\
159	0.00451235903966076\\
160	0.0045124029404445\\
161	0.00451244764850293\\
162	0.00451249317873396\\
163	0.00451253954631017\\
164	0.00451258676668367\\
165	0.0045126348555914\\
166	0.00451268382906015\\
167	0.00451273370341221\\
168	0.00451278449527046\\
169	0.00451283622156419\\
170	0.00451288889953456\\
171	0.00451294254674041\\
172	0.00451299718106416\\
173	0.00451305282071762\\
174	0.00451310948424834\\
175	0.00451316719054559\\
176	0.00451322595884697\\
177	0.00451328580874447\\
178	0.00451334676019147\\
179	0.00451340883350913\\
180	0.00451347204939356\\
181	0.00451353642892253\\
182	0.00451360199356266\\
183	0.00451366876517694\\
184	0.00451373676603195\\
185	0.00451380601880537\\
186	0.00451387654659394\\
187	0.00451394837292136\\
188	0.00451402152174623\\
189	0.00451409601747062\\
190	0.00451417188494869\\
191	0.00451424914949599\\
192	0.00451432783689985\\
193	0.0045144079734323\\
194	0.0045144895858667\\
195	0.00451457270150327\\
196	0.00451465734820478\\
197	0.00451474355443328\\
198	0.00451483134920902\\
199	0.00451492076162857\\
200	0.0045150118212891\\
201	0.00451510455831577\\
202	0.00451519900337015\\
203	0.00451529518765906\\
204	0.00451539314294316\\
205	0.00451549290154605\\
206	0.00451559449636294\\
207	0.00451569796086982\\
208	0.00451580332913238\\
209	0.00451591063581526\\
210	0.00451601991619093\\
211	0.0045161312061492\\
212	0.00451624454220613\\
213	0.00451635996151329\\
214	0.00451647750186724\\
215	0.00451659720171832\\
216	0.00451671910017993\\
217	0.0045168432370378\\
218	0.00451696965275866\\
219	0.00451709838849949\\
220	0.00451722948611608\\
221	0.0045173629881716\\
222	0.00451749893794523\\
223	0.00451763737944011\\
224	0.00451777835739124\\
225	0.00451792191727314\\
226	0.00451806810530696\\
227	0.00451821696846706\\
228	0.0045183685544874\\
229	0.00451852291186688\\
230	0.00451868008987433\\
231	0.00451884013855251\\
232	0.00451900310872137\\
233	0.00451916905197993\\
234	0.00451933802070746\\
235	0.00451951006806287\\
236	0.00451968524798292\\
237	0.00451986361517827\\
238	0.004520045225128\\
239	0.00452023013407156\\
240	0.00452041839899857\\
241	0.0045206100776353\\
242	0.00452080522842839\\
243	0.00452100391052457\\
244	0.00452120618374652\\
245	0.00452141210856396\\
246	0.0045216217460598\\
247	0.00452183515789019\\
248	0.00452205240623802\\
249	0.00452227355375925\\
250	0.00452249866352084\\
251	0.00452272779892923\\
252	0.00452296102364906\\
253	0.00452319840150951\\
254	0.00452343999639834\\
255	0.00452368587214139\\
256	0.00452393609236606\\
257	0.00452419072034723\\
258	0.00452444981883382\\
259	0.00452471344985377\\
260	0.00452498167449561\\
261	0.00452525455266472\\
262	0.00452553214281193\\
263	0.00452581450163227\\
264	0.0045261016837328\\
265	0.00452639374126684\\
266	0.00452669072353439\\
267	0.00452699267654762\\
268	0.0045272996425624\\
269	0.00452761165957714\\
270	0.00452792876080304\\
271	0.00452825097411083\\
272	0.00452857832146215\\
273	0.00452891081833463\\
274	0.0045292484731465\\
275	0.00452959128666663\\
276	0.00452993925132271\\
277	0.00453029235007477\\
278	0.00453065055371733\\
279	0.00453101381281063\\
280	0.00453138277408965\\
281	0.00453175865541518\\
282	0.00453214158584875\\
283	0.00453253169683085\\
284	0.00453292912222395\\
285	0.00453333399835564\\
286	0.00453374646406325\\
287	0.00453416666073869\\
288	0.00453459473237412\\
289	0.00453503082560858\\
290	0.00453547508977521\\
291	0.0045359276769494\\
292	0.00453638874199741\\
293	0.00453685844262605\\
294	0.00453733693943328\\
295	0.0045378243959593\\
296	0.00453832097873851\\
297	0.00453882685735259\\
298	0.00453934220448412\\
299	0.0045398671959708\\
300	0.00454040201086121\\
301	0.00454094683147089\\
302	0.00454150184343924\\
303	0.00454206723578759\\
304	0.00454264320097778\\
305	0.00454322993497191\\
306	0.00454382763729254\\
307	0.00454443651108432\\
308	0.00454505676317584\\
309	0.00454568860414269\\
310	0.00454633224837134\\
311	0.00454698791412379\\
312	0.00454765582360285\\
313	0.00454833620301881\\
314	0.00454902928265623\\
315	0.00454973529694189\\
316	0.00455045448451373\\
317	0.00455118708829001\\
318	0.00455193335553963\\
319	0.00455269353795307\\
320	0.00455346789171404\\
321	0.00455425667757185\\
322	0.00455506016091416\\
323	0.00455587861184099\\
324	0.00455671230523868\\
325	0.00455756152085474\\
326	0.00455842654337318\\
327	0.00455930766249049\\
328	0.00456020517299179\\
329	0.00456111937482776\\
330	0.00456205057319167\\
331	0.00456299907859687\\
332	0.00456396520695488\\
333	0.00456494927965354\\
334	0.00456595162363517\\
335	0.00456697257147563\\
336	0.00456801246146319\\
337	0.00456907163767762\\
338	0.00457015045006982\\
339	0.00457124925454149\\
340	0.00457236841302516\\
341	0.00457350829356467\\
342	0.00457466927039594\\
343	0.00457585172402875\\
344	0.00457705604132874\\
345	0.00457828261560088\\
346	0.00457953184667378\\
347	0.00458080414098624\\
348	0.00458209991167518\\
349	0.00458341957866655\\
350	0.0045847635687695\\
351	0.00458613231577429\\
352	0.00458752626055563\\
353	0.00458894585118182\\
354	0.00459039154303149\\
355	0.00459186379891944\\
356	0.00459336308923372\\
357	0.0045948898920856\\
358	0.00459644469347594\\
359	0.00459802798748025\\
360	0.00459964027645692\\
361	0.00460128207128185\\
362	0.00460295389161543\\
363	0.00460465626620702\\
364	0.00460638973324317\\
365	0.00460815484074823\\
366	0.00460995214704458\\
367	0.00461178222128362\\
368	0.00461364564405774\\
369	0.00461554300810616\\
370	0.00461747491912935\\
371	0.00461944199672686\\
372	0.00462144487547624\\
373	0.00462348420617282\\
374	0.00462556065724938\\
375	0.00462767491639905\\
376	0.00462982769242322\\
377	0.00463201971732722\\
378	0.00463425174868788\\
379	0.00463652457231326\\
380	0.00463883900521847\\
381	0.00464119589895196\\
382	0.00464359614335712\\
383	0.00464604067105701\\
384	0.00464853046368022\\
385	0.00465106656346177\\
386	0.00465365010319791\\
387	0.0046562824013039\\
388	0.00465896488769045\\
389	0.0046616987406995\\
390	0.00466448495858482\\
391	0.0046673245607607\\
392	0.00467021858847535\\
393	0.00467316810552371\\
394	0.00467617419900254\\
395	0.00467923798011746\\
396	0.00468236058503982\\
397	0.00468554317580312\\
398	0.00468878694123863\\
399	0.00469209309795728\\
400	0.00469546289137917\\
401	0.00469889759681303\\
402	0.00470239852058672\\
403	0.00470596700123041\\
404	0.00470960441071781\\
405	0.00471331215576373\\
406	0.00471709167917703\\
407	0.0047209444612786\\
408	0.00472487202138753\\
409	0.00472887591936719\\
410	0.00473295775722845\\
411	0.00473711918078489\\
412	0.00474136188135364\\
413	0.00474568759749272\\
414	0.00475009811676227\\
415	0.00475459527749204\\
416	0.00475918097053656\\
417	0.00476385714101253\\
418	0.00476862579004746\\
419	0.00477348897647613\\
420	0.00477844881842741\\
421	0.00478350749488801\\
422	0.00478866724739903\\
423	0.00479393038221615\\
424	0.00479929927328805\\
425	0.00480477636457613\\
426	0.00481036415760907\\
427	0.00481606519688541\\
428	0.00482188208777703\\
429	0.00482781751814373\\
430	0.00483387426227485\\
431	0.00484005518507004\\
432	0.00484636324647415\\
433	0.00485280150618256\\
434	0.00485937312863478\\
435	0.00486608138831628\\
436	0.00487292967538957\\
437	0.00487992150167683\\
438	0.00488706050701931\\
439	0.00489435046604027\\
440	0.00490179529534124\\
441	0.00490939906116454\\
442	0.00491716598755833\\
443	0.00492510046508518\\
444	0.00493320706012201\\
445	0.00494149052480638\\
446	0.00494995580769529\\
447	0.00495860806520727\\
448	0.00496745267389885\\
449	0.00497649524352265\\
450	0.00498574163058395\\
451	0.00499519795232604\\
452	0.00500487060388073\\
453	0.00501476627722972\\
454	0.00502489197929618\\
455	0.00503525505120246\\
456	0.00504586318875972\\
457	0.00505672446425562\\
458	0.00506784734960671\\
459	0.00507924074094122\\
460	0.00509091398467622\\
461	0.00510287690514977\\
462	0.00511513983386346\\
463	0.0051277136403825\\
464	0.00514060976492993\\
465	0.00515384025269642\\
466	0.00516741778986463\\
467	0.00518135574131796\\
468	0.00519566818997072\\
469	0.00521036997764839\\
470	0.00522547674750591\\
471	0.00524100498783343\\
472	0.00525697207709284\\
473	0.0052733963304204\\
474	0.00529029704833154\\
475	0.00530769456877052\\
476	0.0053256103281584\\
477	0.00534406698399128\\
478	0.00536308850691918\\
479	0.00538269994018463\\
480	0.00540292813163392\\
481	0.00542380541303251\\
482	0.00544538465379216\\
483	0.00546770383083483\\
484	0.00549079861274138\\
485	0.00551470623708547\\
486	0.00553946484680131\\
487	0.00556511414959188\\
488	0.0055916953501791\\
489	0.0056192510514426\\
490	0.00564782508412736\\
491	0.00567746226653577\\
492	0.00570820807635756\\
493	0.0057401082134299\\
494	0.00577320802697733\\
495	0.00580755177499453\\
496	0.00584318167632431\\
497	0.00588013670746371\\
498	0.00591845108578645\\
499	0.00595815236751033\\
500	0.00599925907007681\\
501	0.00604177772434366\\
502	0.00608569925935058\\
503	0.00613099444306557\\
504	0.00617760805787533\\
505	0.0062254493951574\\
506	0.00627438366965847\\
507	0.00632422376641152\\
508	0.00637473523813285\\
509	0.00642567095097926\\
510	0.00647661164865807\\
511	0.00652687139064926\\
512	0.00657423640575458\\
513	0.00661655140299186\\
514	0.00665371902148159\\
515	0.00668600310553593\\
516	0.0067159532734127\\
517	0.00674469677073458\\
518	0.00677265071039201\\
519	0.00680028665386025\\
520	0.00682785607782561\\
521	0.00685577755733429\\
522	0.0068841784578413\\
523	0.00691311383341591\\
524	0.00694262451131733\\
525	0.00697275170320295\\
526	0.00700352525222174\\
527	0.00703496351156598\\
528	0.00706708449889653\\
529	0.00709990592971264\\
530	0.0071334452888676\\
531	0.00716772053471979\\
532	0.00720275069189423\\
533	0.00723855615826742\\
534	0.00727515781455885\\
535	0.00731257698488151\\
536	0.00735086698766402\\
537	0.00739010957080671\\
538	0.00743004600073194\\
539	0.00746900737464546\\
540	0.00750673647633624\\
541	0.00754426309884338\\
542	0.00758223832146432\\
543	0.00762075423708593\\
544	0.00765988271016921\\
545	0.00769962684360228\\
546	0.00773998511969133\\
547	0.00778095188545817\\
548	0.00782252055311138\\
549	0.00786468598879321\\
550	0.00790743502023981\\
551	0.00795074979578723\\
552	0.00799460854250198\\
553	0.00803898414395021\\
554	0.00808385652088464\\
555	0.00812923065899918\\
556	0.00817327860425402\\
557	0.0082165586309912\\
558	0.00826016567862677\\
559	0.00830416817852981\\
560	0.00834855606995987\\
561	0.00839330614562769\\
562	0.00843839342276844\\
563	0.00848379145581574\\
564	0.00852947260656567\\
565	0.00857540837569983\\
566	0.00862156967192121\\
567	0.00866793238274147\\
568	0.00871414708042179\\
569	0.00876025495298665\\
570	0.00880658414128566\\
571	0.00885311974160069\\
572	0.00889982902787152\\
573	0.00894667662535195\\
574	0.00899362444626539\\
575	0.00904063165462433\\
576	0.00908765467048826\\
577	0.00913464722628621\\
578	0.00918156049094262\\
579	0.00922834328141294\\
580	0.0092749423859596\\
581	0.00932130302929076\\
582	0.00936736951671608\\
583	0.00941308610287968\\
584	0.00945839814032785\\
585	0.00950325357339596\\
586	0.00954760485097041\\
587	0.00959141132963419\\
588	0.00963464220433352\\
589	0.00967727987662474\\
590	0.00971932328838558\\
591	0.0097606871825126\\
592	0.00980121135503816\\
593	0.00984070634831441\\
594	0.0098789201112722\\
595	0.00991544794299161\\
596	0.00994949078513196\\
597	0.00997920912842892\\
598	0.0100000808076597\\
599	0\\
600	0\\
};
\addplot [color=mycolor4,solid,forget plot]
  table[row sep=crcr]{%
1	0.00449423493463222\\
2	0.0044942375116113\\
3	0.00449424013481967\\
4	0.00449424280509184\\
5	0.00449424552327744\\
6	0.00449424829024167\\
7	0.00449425110686551\\
8	0.00449425397404569\\
9	0.00449425689269547\\
10	0.00449425986374462\\
11	0.00449426288813989\\
12	0.00449426596684522\\
13	0.00449426910084208\\
14	0.00449427229112976\\
15	0.00449427553872578\\
16	0.00449427884466624\\
17	0.00449428221000596\\
18	0.0044942856358191\\
19	0.00449428912319931\\
20	0.00449429267326009\\
21	0.00449429628713538\\
22	0.0044942999659796\\
23	0.00449430371096836\\
24	0.00449430752329853\\
25	0.00449431140418892\\
26	0.00449431535488046\\
27	0.00449431937663676\\
28	0.00449432347074448\\
29	0.00449432763851366\\
30	0.00449433188127839\\
31	0.00449433620039683\\
32	0.0044943405972521\\
33	0.00449434507325255\\
34	0.00449434962983214\\
35	0.00449435426845108\\
36	0.00449435899059615\\
37	0.00449436379778132\\
38	0.0044943686915482\\
39	0.00449437367346648\\
40	0.00449437874513465\\
41	0.00449438390818025\\
42	0.00449438916426058\\
43	0.00449439451506331\\
44	0.0044943999623068\\
45	0.00449440550774096\\
46	0.00449441115314756\\
47	0.00449441690034101\\
48	0.00449442275116891\\
49	0.0044944287075126\\
50	0.00449443477128778\\
51	0.00449444094444526\\
52	0.00449444722897148\\
53	0.00449445362688923\\
54	0.00449446014025841\\
55	0.00449446677117652\\
56	0.00449447352177948\\
57	0.00449448039424233\\
58	0.00449448739077992\\
59	0.00449449451364756\\
60	0.00449450176514201\\
61	0.00449450914760201\\
62	0.00449451666340928\\
63	0.00449452431498901\\
64	0.0044945321048109\\
65	0.00449454003538998\\
66	0.00449454810928721\\
67	0.00449455632911073\\
68	0.00449456469751629\\
69	0.00449457321720858\\
70	0.00449458189094168\\
71	0.00449459072152037\\
72	0.00449459971180079\\
73	0.00449460886469152\\
74	0.00449461818315445\\
75	0.00449462767020601\\
76	0.0044946373289179\\
77	0.00449464716241824\\
78	0.00449465717389259\\
79	0.00449466736658513\\
80	0.00449467774379946\\
81	0.00449468830889998\\
82	0.00449469906531296\\
83	0.00449471001652756\\
84	0.00449472116609714\\
85	0.0044947325176404\\
86	0.00449474407484249\\
87	0.00449475584145647\\
88	0.00449476782130439\\
89	0.00449478001827853\\
90	0.00449479243634289\\
91	0.00449480507953436\\
92	0.00449481795196419\\
93	0.00449483105781932\\
94	0.0044948444013637\\
95	0.00449485798693985\\
96	0.00449487181897023\\
97	0.0044948859019587\\
98	0.00449490024049235\\
99	0.0044949148392425\\
100	0.00449492970296668\\
101	0.00449494483651001\\
102	0.00449496024480706\\
103	0.00449497593288333\\
104	0.00449499190585686\\
105	0.0044950081689402\\
106	0.00449502472744203\\
107	0.00449504158676875\\
108	0.00449505875242676\\
109	0.00449507623002388\\
110	0.00449509402527143\\
111	0.00449511214398615\\
112	0.00449513059209217\\
113	0.00449514937562279\\
114	0.00449516850072287\\
115	0.00449518797365063\\
116	0.00449520780077984\\
117	0.00449522798860189\\
118	0.00449524854372815\\
119	0.00449526947289197\\
120	0.00449529078295108\\
121	0.00449531248089002\\
122	0.00449533457382207\\
123	0.0044953570689921\\
124	0.00449537997377868\\
125	0.00449540329569676\\
126	0.00449542704240006\\
127	0.00449545122168376\\
128	0.00449547584148694\\
129	0.00449550090989545\\
130	0.00449552643514441\\
131	0.00449555242562119\\
132	0.00449557888986795\\
133	0.00449560583658473\\
134	0.00449563327463221\\
135	0.00449566121303478\\
136	0.00449568966098345\\
137	0.00449571862783907\\
138	0.00449574812313521\\
139	0.00449577815658144\\
140	0.00449580873806684\\
141	0.00449583987766272\\
142	0.00449587158562645\\
143	0.00449590387240465\\
144	0.00449593674863672\\
145	0.00449597022515827\\
146	0.00449600431300486\\
147	0.00449603902341557\\
148	0.00449607436783662\\
149	0.00449611035792529\\
150	0.00449614700555367\\
151	0.00449618432281266\\
152	0.00449622232201593\\
153	0.00449626101570376\\
154	0.00449630041664754\\
155	0.00449634053785365\\
156	0.00449638139256785\\
157	0.00449642299427956\\
158	0.00449646535672648\\
159	0.00449650849389877\\
160	0.00449655242004393\\
161	0.00449659714967121\\
162	0.00449664269755652\\
163	0.00449668907874721\\
164	0.00449673630856708\\
165	0.00449678440262114\\
166	0.00449683337680105\\
167	0.00449688324729004\\
168	0.00449693403056844\\
169	0.00449698574341888\\
170	0.00449703840293188\\
171	0.00449709202651152\\
172	0.00449714663188098\\
173	0.00449720223708861\\
174	0.00449725886051362\\
175	0.00449731652087233\\
176	0.00449737523722416\\
177	0.00449743502897812\\
178	0.00449749591589904\\
179	0.00449755791811421\\
180	0.00449762105611997\\
181	0.00449768535078856\\
182	0.00449775082337498\\
183	0.00449781749552394\\
184	0.00449788538927718\\
185	0.00449795452708047\\
186	0.00449802493179134\\
187	0.00449809662668611\\
188	0.0044981696354679\\
189	0.00449824398227419\\
190	0.00449831969168449\\
191	0.00449839678872839\\
192	0.00449847529889326\\
193	0.00449855524813228\\
194	0.00449863666287215\\
195	0.0044987195700201\\
196	0.00449880399697054\\
197	0.00449888997161056\\
198	0.00449897752232694\\
199	0.00449906667803031\\
200	0.00449915746816577\\
201	0.00449924992272337\\
202	0.00449934407224882\\
203	0.00449943994785433\\
204	0.00449953758122996\\
205	0.00449963700465506\\
206	0.00449973825101015\\
207	0.004499841353789\\
208	0.00449994634711121\\
209	0.00450005326573492\\
210	0.00450016214507018\\
211	0.0045002730211924\\
212	0.00450038593085647\\
213	0.00450050091151135\\
214	0.00450061800131465\\
215	0.00450073723914849\\
216	0.00450085866463531\\
217	0.00450098231815409\\
218	0.00450110824085814\\
219	0.00450123647469233\\
220	0.00450136706241179\\
221	0.00450150004760123\\
222	0.00450163547469482\\
223	0.00450177338899711\\
224	0.00450191383670501\\
225	0.00450205686493054\\
226	0.00450220252172479\\
227	0.00450235085610301\\
228	0.0045025019180712\\
229	0.00450265575865392\\
230	0.00450281242992355\\
231	0.00450297198503155\\
232	0.00450313447824137\\
233	0.00450329996496345\\
234	0.00450346850179227\\
235	0.00450364014654617\\
236	0.00450381495830945\\
237	0.00450399299747801\\
238	0.00450417432580749\\
239	0.0045043590064658\\
240	0.00450454710408909\\
241	0.0045047386848424\\
242	0.00450493381648505\\
243	0.00450513256844156\\
244	0.00450533501187844\\
245	0.00450554121978774\\
246	0.00450575126707775\\
247	0.00450596523067188\\
248	0.00450618318961665\\
249	0.00450640522519969\\
250	0.00450663142107836\\
251	0.00450686186342106\\
252	0.00450709664106149\\
253	0.00450733584566809\\
254	0.0045075795719296\\
255	0.00450782791775842\\
256	0.00450808098451368\\
257	0.00450833887724552\\
258	0.00450860170496262\\
259	0.00450886958092484\\
260	0.00450914262296295\\
261	0.00450942095382705\\
262	0.00450970470156564\\
263	0.00450999399993688\\
264	0.0045102889888528\\
265	0.00451058981485757\\
266	0.00451089663163914\\
267	0.00451120960057356\\
268	0.00451152889129869\\
269	0.00451185468231371\\
270	0.00451218716159669\\
271	0.00451252652723082\\
272	0.00451287298802491\\
273	0.00451322676410768\\
274	0.00451358808746551\\
275	0.00451395720237038\\
276	0.00451433436558568\\
277	0.00451471984604888\\
278	0.00451511392308451\\
279	0.0045155168798852\\
280	0.00451592827956583\\
281	0.00451634712919375\\
282	0.00451677356147549\\
283	0.00451720771141671\\
284	0.00451764971636046\\
285	0.00451809971602603\\
286	0.00451855785254834\\
287	0.00451902427051811\\
288	0.0045194991170225\\
289	0.00451998254168646\\
290	0.00452047469671473\\
291	0.00452097573693461\\
292	0.00452148581983912\\
293	0.00452200510563121\\
294	0.00452253375726843\\
295	0.00452307194050822\\
296	0.0045236198239542\\
297	0.00452417757910285\\
298	0.00452474538039131\\
299	0.00452532340524546\\
300	0.00452591183412912\\
301	0.00452651085059379\\
302	0.00452712064132927\\
303	0.00452774139621501\\
304	0.00452837330837215\\
305	0.00452901657421638\\
306	0.00452967139351182\\
307	0.00453033796942522\\
308	0.00453101650858133\\
309	0.00453170722111896\\
310	0.00453241032074778\\
311	0.00453312602480588\\
312	0.00453385455431816\\
313	0.00453459613405548\\
314	0.00453535099259463\\
315	0.00453611936237895\\
316	0.00453690147977952\\
317	0.00453769758515751\\
318	0.00453850792292673\\
319	0.004539332741617\\
320	0.00454017229393828\\
321	0.00454102683684493\\
322	0.00454189663160097\\
323	0.00454278194384522\\
324	0.00454368304365735\\
325	0.00454460020562361\\
326	0.0045455337089032\\
327	0.00454648383729433\\
328	0.00454745087930042\\
329	0.00454843512819572\\
330	0.00454943688209075\\
331	0.00455045644399712\\
332	0.00455149412189114\\
333	0.00455255022877655\\
334	0.00455362508274562\\
335	0.00455471900703852\\
336	0.00455583233010047\\
337	0.00455696538563604\\
338	0.00455811851266065\\
339	0.00455929205554782\\
340	0.00456048636407239\\
341	0.00456170179344842\\
342	0.00456293870436107\\
343	0.00456419746299149\\
344	0.00456547844103389\\
345	0.00456678201570295\\
346	0.00456810856973118\\
347	0.00456945849135335\\
348	0.00457083217427769\\
349	0.00457223001764085\\
350	0.00457365242594477\\
351	0.00457509980897295\\
352	0.00457657258168331\\
353	0.0045780711640745\\
354	0.00457959598102206\\
355	0.00458114746208063\\
356	0.00458272604124747\\
357	0.0045843321566828\\
358	0.00458596625038128\\
359	0.00458762876778821\\
360	0.00458932015735426\\
361	0.0045910408700209\\
362	0.00459279135862856\\
363	0.00459457207723827\\
364	0.00459638348035746\\
365	0.00459822602205886\\
366	0.00460010015498175\\
367	0.0046020063292031\\
368	0.00460394499096624\\
369	0.0046059165812542\\
370	0.00460792153419442\\
371	0.0046099602752823\\
372	0.0046120332194115\\
373	0.00461414076870063\\
374	0.0046162833101089\\
375	0.00461846121283676\\
376	0.00462067482551303\\
377	0.00462292447317906\\
378	0.00462521045408837\\
379	0.00462753303635533\\
380	0.0046298924544977\\
381	0.00463228890592646\\
382	0.00463472254741456\\
383	0.00463719349145573\\
384	0.00463970180197746\\
385	0.00464224748738354\\
386	0.00464483048404433\\
387	0.00464745060733401\\
388	0.00465011163097672\\
389	0.00465282112392569\\
390	0.00465557996561451\\
391	0.00465838904990187\\
392	0.0046612492853003\\
393	0.00466416159525333\\
394	0.00466712691843225\\
395	0.00467014620902821\\
396	0.00467322043730565\\
397	0.004676350590474\\
398	0.00467953767358557\\
399	0.00468278271044815\\
400	0.00468608674464115\\
401	0.0046894508406286\\
402	0.00469287608496788\\
403	0.00469636358758348\\
404	0.00469991448308326\\
405	0.00470352993223117\\
406	0.00470721112353897\\
407	0.00471095927481831\\
408	0.00471477563497575\\
409	0.00471866148616791\\
410	0.00472261814623317\\
411	0.00472664697143638\\
412	0.00473074935956745\\
413	0.0047349267534368\\
414	0.00473918064480881\\
415	0.00474351257879115\\
416	0.00474792415863194\\
417	0.00475241705074959\\
418	0.00475699298988722\\
419	0.00476165378647208\\
420	0.00476640133515224\\
421	0.00477123762342865\\
422	0.00477616474169527\\
423	0.00478118489691932\\
424	0.00478630043748749\\
425	0.00479151391609784\\
426	0.00479682828756586\\
427	0.00480224665873392\\
428	0.0048077717014042\\
429	0.0048134058479498\\
430	0.00481915161080988\\
431	0.00482501158653451\\
432	0.00483098846008237\\
433	0.00483708500937839\\
434	0.00484330411011993\\
435	0.00484964874084148\\
436	0.00485612198825435\\
437	0.00486272705286623\\
438	0.00486946725488301\\
439	0.00487634604039137\\
440	0.00488336698782245\\
441	0.00489053381469524\\
442	0.00489785038462899\\
443	0.00490532071461264\\
444	0.00491294898252247\\
445	0.00492073953489484\\
446	0.00492869689500556\\
447	0.00493682577142353\\
448	0.00494513106744138\\
449	0.00495361789206146\\
450	0.00496229157243045\\
451	0.00497115765924324\\
452	0.00498022189541562\\
453	0.00498949023472135\\
454	0.00499896890086613\\
455	0.00500866440345483\\
456	0.00501858355496273\\
457	0.00502873348878343\\
458	0.0050391216784205\\
459	0.00504975595789032\\
460	0.00506064454340514\\
461	0.00507179605641145\\
462	0.00508321954806219\\
463	0.00509492452520209\\
464	0.00510692097795116\\
465	0.00511921940897671\\
466	0.00513183086454611\\
467	0.00514476696743132\\
468	0.00515803995162999\\
469	0.00517166269858304\\
470	0.00518564877435193\\
471	0.00520001247033478\\
472	0.00521476884769909\\
473	0.00522993378197527\\
474	0.00524552400828319\\
475	0.00526155716821812\\
476	0.00527805185876928\\
477	0.00529502768237656\\
478	0.00531250530274405\\
479	0.00533050653714461\\
480	0.00534905449115622\\
481	0.00536817367281699\\
482	0.00538788937754491\\
483	0.00540822961449633\\
484	0.00542922958828796\\
485	0.00545094465505665\\
486	0.00547340954797532\\
487	0.00549666081081584\\
488	0.00552073654862956\\
489	0.00554567604275865\\
490	0.00557152030248971\\
491	0.00559831203024519\\
492	0.00562609555698195\\
493	0.00565491671668014\\
494	0.00568482266064022\\
495	0.00571586159606177\\
496	0.00574808243041682\\
497	0.00578153429842132\\
498	0.00581626594303576\\
499	0.00585232491525198\\
500	0.00588975654942133\\
501	0.00592860266082905\\
502	0.00596889989886678\\
503	0.00601067767630072\\
504	0.00605395557420223\\
505	0.00609874017324865\\
506	0.00614502108422947\\
507	0.00619276592548344\\
508	0.00624191346201506\\
509	0.00629236340636711\\
510	0.00634396887409176\\
511	0.00639652876133721\\
512	0.00644981299135517\\
513	0.00650352171461523\\
514	0.00655716683872166\\
515	0.00660995250271162\\
516	0.00665884847272803\\
517	0.0067025730922806\\
518	0.00674105919396559\\
519	0.00677464238152105\\
520	0.00680670886272085\\
521	0.00683765073781043\\
522	0.00686788391183168\\
523	0.00689790066434322\\
524	0.00692791783800038\\
525	0.00695809964758147\\
526	0.006988791538306\\
527	0.00702004862282146\\
528	0.00705191065062629\\
529	0.00708441945236982\\
530	0.00711761292912907\\
531	0.00715151659490811\\
532	0.00718615026195851\\
533	0.00722153381129203\\
534	0.00725768718797693\\
535	0.00729463064383161\\
536	0.00733238513454148\\
537	0.00737097201586305\\
538	0.00741042166806156\\
539	0.00745080892080739\\
540	0.00749222410093988\\
541	0.00753342790535465\\
542	0.00757357698590843\\
543	0.00761248676982125\\
544	0.00765181576701152\\
545	0.00769162635385801\\
546	0.00773200381947203\\
547	0.0077729962310868\\
548	0.0078146043505957\\
549	0.00785682649776452\\
550	0.00789965439511454\\
551	0.00794307340113038\\
552	0.00798706536205014\\
553	0.00803160812495948\\
554	0.00807667449279134\\
555	0.00812223043558096\\
556	0.00816827506779465\\
557	0.00821393678936324\\
558	0.0082580380895143\\
559	0.0083022106183908\\
560	0.00834671273049277\\
561	0.00839158339359311\\
562	0.00843679986462607\\
563	0.00848233518510498\\
564	0.00852816075852369\\
565	0.00857424671354175\\
566	0.00862056226333017\\
567	0.00866707593172449\\
568	0.00871376285763563\\
569	0.0087602376413274\\
570	0.008806584138207\\
571	0.00885311974148326\\
572	0.00889982902782198\\
573	0.00894667662532867\\
574	0.00899362444625509\\
575	0.0090406316546198\\
576	0.00908765467048649\\
577	0.00913464722628548\\
578	0.00918156049094254\\
579	0.00922834328141292\\
580	0.00927494238595959\\
581	0.00932130302929075\\
582	0.00936736951671608\\
583	0.00941308610287968\\
584	0.00945839814032785\\
585	0.00950325357339596\\
586	0.00954760485097041\\
587	0.00959141132963419\\
588	0.00963464220433353\\
589	0.00967727987662474\\
590	0.00971932328838559\\
591	0.0097606871825126\\
592	0.00980121135503816\\
593	0.00984070634831441\\
594	0.0098789201112722\\
595	0.00991544794299161\\
596	0.00994949078513196\\
597	0.00997920912842892\\
598	0.0100000808076597\\
599	0\\
600	0\\
};
\addplot [color=mycolor5,solid,forget plot]
  table[row sep=crcr]{%
1	0.00447611684544269\\
2	0.00447611975432778\\
3	0.004476122714739\\
4	0.00447612572759365\\
5	0.00447612879382535\\
6	0.00447613191438445\\
7	0.00447613509023832\\
8	0.00447613832237169\\
9	0.00447614161178692\\
10	0.00447614495950424\\
11	0.00447614836656219\\
12	0.00447615183401796\\
13	0.0044761553629476\\
14	0.0044761589544465\\
15	0.00447616260962961\\
16	0.00447616632963195\\
17	0.00447617011560884\\
18	0.00447617396873626\\
19	0.00447617789021134\\
20	0.0044761818812526\\
21	0.00447618594310047\\
22	0.0044761900770176\\
23	0.00447619428428926\\
24	0.00447619856622379\\
25	0.00447620292415304\\
26	0.00447620735943272\\
27	0.00447621187344283\\
28	0.0044762164675882\\
29	0.00447622114329883\\
30	0.00447622590203035\\
31	0.00447623074526464\\
32	0.00447623567451008\\
33	0.00447624069130213\\
34	0.0044762457972039\\
35	0.00447625099380653\\
36	0.00447625628272978\\
37	0.00447626166562246\\
38	0.004476267144163\\
39	0.00447627272006001\\
40	0.00447627839505287\\
41	0.00447628417091213\\
42	0.00447629004944023\\
43	0.00447629603247199\\
44	0.00447630212187526\\
45	0.00447630831955149\\
46	0.00447631462743623\\
47	0.00447632104749997\\
48	0.00447632758174864\\
49	0.00447633423222424\\
50	0.0044763410010055\\
51	0.00447634789020859\\
52	0.0044763549019879\\
53	0.00447636203853648\\
54	0.00447636930208688\\
55	0.00447637669491188\\
56	0.0044763842193253\\
57	0.00447639187768256\\
58	0.00447639967238155\\
59	0.00447640760586342\\
60	0.00447641568061329\\
61	0.00447642389916112\\
62	0.00447643226408248\\
63	0.00447644077799944\\
64	0.00447644944358135\\
65	0.00447645826354573\\
66	0.00447646724065916\\
67	0.00447647637773819\\
68	0.00447648567765016\\
69	0.00447649514331426\\
70	0.00447650477770241\\
71	0.00447651458384019\\
72	0.00447652456480787\\
73	0.00447653472374145\\
74	0.00447654506383355\\
75	0.00447655558833457\\
76	0.0044765663005537\\
77	0.00447657720385997\\
78	0.00447658830168344\\
79	0.00447659959751618\\
80	0.00447661109491352\\
81	0.00447662279749521\\
82	0.00447663470894648\\
83	0.00447664683301942\\
84	0.00447665917353406\\
85	0.0044766717343797\\
86	0.0044766845195162\\
87	0.00447669753297515\\
88	0.00447671077886135\\
89	0.00447672426135413\\
90	0.0044767379847086\\
91	0.00447675195325719\\
92	0.00447676617141101\\
93	0.00447678064366124\\
94	0.00447679537458084\\
95	0.00447681036882575\\
96	0.00447682563113668\\
97	0.00447684116634056\\
98	0.00447685697935212\\
99	0.00447687307517561\\
100	0.00447688945890633\\
101	0.00447690613573247\\
102	0.00447692311093672\\
103	0.00447694038989806\\
104	0.00447695797809361\\
105	0.00447697588110033\\
106	0.00447699410459692\\
107	0.00447701265436585\\
108	0.0044770315362951\\
109	0.00447705075638024\\
110	0.00447707032072641\\
111	0.00447709023555039\\
112	0.00447711050718259\\
113	0.00447713114206933\\
114	0.00447715214677491\\
115	0.00447717352798376\\
116	0.00447719529250283\\
117	0.00447721744726377\\
118	0.00447723999932537\\
119	0.00447726295587586\\
120	0.00447728632423534\\
121	0.00447731011185826\\
122	0.00447733432633605\\
123	0.00447735897539957\\
124	0.00447738406692172\\
125	0.00447740960892024\\
126	0.00447743560956029\\
127	0.00447746207715728\\
128	0.00447748902017979\\
129	0.00447751644725223\\
130	0.00447754436715799\\
131	0.00447757278884231\\
132	0.0044776017214154\\
133	0.00447763117415548\\
134	0.00447766115651197\\
135	0.00447769167810866\\
136	0.00447772274874714\\
137	0.00447775437840988\\
138	0.00447778657726387\\
139	0.00447781935566398\\
140	0.00447785272415639\\
141	0.00447788669348243\\
142	0.00447792127458189\\
143	0.00447795647859705\\
144	0.00447799231687621\\
145	0.00447802880097767\\
146	0.00447806594267358\\
147	0.0044781037539539\\
148	0.00447814224703048\\
149	0.00447818143434114\\
150	0.00447822132855375\\
151	0.00447826194257056\\
152	0.00447830328953255\\
153	0.00447834538282362\\
154	0.00447838823607513\\
155	0.00447843186317041\\
156	0.00447847627824922\\
157	0.00447852149571247\\
158	0.00447856753022692\\
159	0.00447861439672983\\
160	0.00447866211043392\\
161	0.00447871068683213\\
162	0.0044787601417027\\
163	0.00447881049111409\\
164	0.00447886175143014\\
165	0.00447891393931521\\
166	0.00447896707173942\\
167	0.00447902116598391\\
168	0.0044790762396463\\
169	0.00447913231064606\\
170	0.00447918939723015\\
171	0.00447924751797851\\
172	0.00447930669180993\\
173	0.00447936693798771\\
174	0.00447942827612567\\
175	0.00447949072619413\\
176	0.00447955430852613\\
177	0.00447961904382354\\
178	0.00447968495316366\\
179	0.00447975205800576\\
180	0.00447982038019771\\
181	0.00447988994198306\\
182	0.00447996076600804\\
183	0.00448003287532907\\
184	0.00448010629342027\\
185	0.00448018104418135\\
186	0.00448025715194572\\
187	0.00448033464148909\\
188	0.00448041353803804\\
189	0.00448049386727918\\
190	0.00448057565536867\\
191	0.00448065892894191\\
192	0.00448074371512369\\
193	0.00448083004153872\\
194	0.00448091793632218\\
195	0.00448100742813096\\
196	0.0044810985461549\\
197	0.0044811913201285\\
198	0.00448128578034253\\
199	0.00448138195765551\\
200	0.00448147988350512\\
201	0.00448157958992015\\
202	0.00448168110953266\\
203	0.00448178447559043\\
204	0.00448188972196978\\
205	0.00448199688318878\\
206	0.00448210599442063\\
207	0.0044822170915076\\
208	0.00448233021097527\\
209	0.00448244539004719\\
210	0.00448256266665979\\
211	0.00448268207947798\\
212	0.00448280366791103\\
213	0.00448292747212879\\
214	0.00448305353307862\\
215	0.00448318189250272\\
216	0.00448331259295575\\
217	0.00448344567782344\\
218	0.00448358119134112\\
219	0.00448371917861335\\
220	0.00448385968563385\\
221	0.00448400275930599\\
222	0.00448414844746402\\
223	0.00448429679889495\\
224	0.00448444786336075\\
225	0.00448460169162161\\
226	0.0044847583354595\\
227	0.00448491784770275\\
228	0.00448508028225083\\
229	0.00448524569410033\\
230	0.00448541413937123\\
231	0.00448558567533388\\
232	0.00448576036043674\\
233	0.00448593825433465\\
234	0.00448611941791771\\
235	0.00448630391334067\\
236	0.00448649180405276\\
237	0.00448668315482797\\
238	0.00448687803179556\\
239	0.00448707650247086\\
240	0.0044872786357859\\
241	0.0044874845021201\\
242	0.00448769417333041\\
243	0.00448790772278116\\
244	0.00448812522537248\\
245	0.00448834675756801\\
246	0.00448857239742051\\
247	0.00448880222459568\\
248	0.00448903632039302\\
249	0.00448927476776331\\
250	0.00448951765132232\\
251	0.00448976505735911\\
252	0.00449001707383878\\
253	0.00449027379039808\\
254	0.00449053529833253\\
255	0.00449080169057415\\
256	0.00449107306165765\\
257	0.00449134950767376\\
258	0.0044916311262074\\
259	0.00449191801625894\\
260	0.00449221027814579\\
261	0.00449250801338226\\
262	0.00449281132453496\\
263	0.00449312031505095\\
264	0.00449343508905576\\
265	0.0044937557511192\\
266	0.00449408240598558\\
267	0.00449441515826695\\
268	0.00449475411209723\\
269	0.00449509937074666\\
270	0.00449545103619653\\
271	0.00449580920867629\\
272	0.0044961739861663\\
273	0.00449654546387235\\
274	0.00449692373367918\\
275	0.00449730888359263\\
276	0.00449770099717835\\
277	0.004498100153003\\
278	0.00449850642410589\\
279	0.00449891987773123\\
280	0.00449934059954551\\
281	0.00449976871334684\\
282	0.00450020434492518\\
283	0.00450064762209174\\
284	0.00450109867470857\\
285	0.00450155763471871\\
286	0.00450202463617677\\
287	0.00450249981527977\\
288	0.00450298331039883\\
289	0.00450347526211091\\
290	0.00450397581323134\\
291	0.00450448510884673\\
292	0.00450500329634843\\
293	0.00450553052546662\\
294	0.00450606694830475\\
295	0.00450661271937486\\
296	0.00450716799563338\\
297	0.00450773293651758\\
298	0.00450830770398261\\
299	0.00450889246253951\\
300	0.00450948737929368\\
301	0.00451009262398429\\
302	0.00451070836902445\\
303	0.00451133478954237\\
304	0.00451197206342324\\
305	0.00451262037135227\\
306	0.00451327989685856\\
307	0.0045139508263603\\
308	0.00451463334921088\\
309	0.00451532765774638\\
310	0.00451603394733435\\
311	0.00451675241642394\\
312	0.0045174832665975\\
313	0.00451822670262389\\
314	0.00451898293251324\\
315	0.00451975216757383\\
316	0.00452053462247072\\
317	0.00452133051528654\\
318	0.00452214006758446\\
319	0.00452296350447377\\
320	0.00452380105467774\\
321	0.00452465295060458\\
322	0.00452551942842131\\
323	0.00452640072813092\\
324	0.00452729709365291\\
325	0.00452820877290797\\
326	0.00452913601790652\\
327	0.00453007908484172\\
328	0.00453103823418763\\
329	0.00453201373080226\\
330	0.00453300584403674\\
331	0.00453401484785032\\
332	0.00453504102093243\\
333	0.00453608464683206\\
334	0.00453714601409516\\
335	0.00453822541641085\\
336	0.00453932315276705\\
337	0.00454043952761685\\
338	0.00454157485105601\\
339	0.00454272943901333\\
340	0.00454390361345431\\
341	0.00454509770260014\\
342	0.00454631204116298\\
343	0.00454754697059925\\
344	0.00454880283938285\\
345	0.00455008000330001\\
346	0.00455137882576797\\
347	0.00455269967818017\\
348	0.00455404294028005\\
349	0.00455540900056651\\
350	0.00455679825673476\\
351	0.00455821111615508\\
352	0.00455964799639432\\
353	0.00456110932578362\\
354	0.00456259554403774\\
355	0.00456410710293037\\
356	0.00456564446703184\\
357	0.00456720811451467\\
358	0.00456879853803411\\
359	0.00457041624569074\\
360	0.00457206176208316\\
361	0.00457373562945894\\
362	0.00457543840897322\\
363	0.00457717068206428\\
364	0.00457893305195645\\
365	0.0045807261453007\\
366	0.00458255061396385\\
367	0.00458440713697706\\
368	0.00458629642265447\\
369	0.00458821921089194\\
370	0.0045901762756544\\
371	0.00459216842765954\\
372	0.00459419651726164\\
373	0.00459626143753648\\
374	0.00459836412756241\\
375	0.00460050557588601\\
376	0.00460268682415156\\
377	0.00460490897086097\\
378	0.00460717317521617\\
379	0.00460948066097518\\
380	0.00461183272022713\\
381	0.00461423071695319\\
382	0.0046166760901785\\
383	0.00461917035638782\\
384	0.00462171511052656\\
385	0.00462431202380871\\
386	0.00462696283278884\\
387	0.00462966930063988\\
388	0.0046324290892577\\
389	0.00463523615730953\\
390	0.00463809138296256\\
391	0.004640995657523\\
392	0.00464394988387168\\
393	0.00464695497503962\\
394	0.0046500118526693\\
395	0.00465312144460244\\
396	0.00465628467768117\\
397	0.0046595024758726\\
398	0.00466277576176307\\
399	0.00466610545512369\\
400	0.00466949247207683\\
401	0.00467293772497836\\
402	0.00467644212314377\\
403	0.00468000657459751\\
404	0.00468363198800422\\
405	0.00468731927282593\\
406	0.00469106934059947\\
407	0.00469488310559309\\
408	0.00469876147874895\\
409	0.00470270536537292\\
410	0.00470671566256336\\
411	0.00471079325638066\\
412	0.00471493901878245\\
413	0.0047191538043985\\
414	0.00472343844730269\\
415	0.00472779375805089\\
416	0.004732220521241\\
417	0.00473671949296509\\
418	0.00474129139272375\\
419	0.00474593686447294\\
420	0.00475065649639139\\
421	0.00475545081992455\\
422	0.00476032030463185\\
423	0.00476526535172703\\
424	0.00477028628204168\\
425	0.00477538330402379\\
426	0.00478055641412485\\
427	0.0047858146948815\\
428	0.0047911705580954\\
429	0.00479662606528385\\
430	0.00480218334248738\\
431	0.0048078445838655\\
432	0.00481361205549231\\
433	0.00481948809980964\\
434	0.00482547514086932\\
435	0.0048315756899261\\
436	0.00483779235143269\\
437	0.00484412782967704\\
438	0.00485058493614512\\
439	0.00485716659767541\\
440	0.00486387586540603\\
441	0.00487071592464134\\
442	0.00487769010583814\\
443	0.00488480189679685\\
444	0.00489205495613112\\
445	0.00489945312814394\\
446	0.00490700045932469\\
447	0.0049147012170273\\
448	0.00492255991216044\\
449	0.0049305813322908\\
450	0.00493877060788788\\
451	0.00494713339312314\\
452	0.00495567619066768\\
453	0.00496440503083706\\
454	0.00497332541514836\\
455	0.00498244309501798\\
456	0.00499176408687724\\
457	0.00500129468788659\\
458	0.00501104149252607\\
459	0.0050210114101395\\
460	0.00503121168346516\\
461	0.00504164990811298\\
462	0.00505233405302114\\
463	0.00506327248196118\\
464	0.00507447397609744\\
465	0.00508594775767296\\
466	0.00509770351500394\\
467	0.00510975142921399\\
468	0.00512210220353799\\
469	0.00513476709587684\\
470	0.00514775794984485\\
471	0.00516108718461144\\
472	0.00517476781097929\\
473	0.00518881351395189\\
474	0.00520323871490167\\
475	0.00521805861449824\\
476	0.00523328923748911\\
477	0.0052489474794104\\
478	0.00526505115546835\\
479	0.00528161905122463\\
480	0.00529867097509105\\
481	0.00531622781725356\\
482	0.00533431166979781\\
483	0.00535294595707221\\
484	0.00537215550865611\\
485	0.00539196607490652\\
486	0.00541240708418985\\
487	0.00543351778324593\\
488	0.00545534982680042\\
489	0.00547793860810592\\
490	0.00550132141132871\\
491	0.00552553711842989\\
492	0.00555062602073389\\
493	0.00557663030781944\\
494	0.00560359406592058\\
495	0.00563156325125534\\
496	0.00566058561091293\\
497	0.00569071055285296\\
498	0.00572198895115862\\
499	0.00575447287040434\\
500	0.00578821518875923\\
501	0.00582326909480821\\
502	0.00585968742701873\\
503	0.00589752181754076\\
504	0.00593682159388401\\
505	0.00597763237884641\\
506	0.0060199943178503\\
507	0.00606393984606206\\
508	0.00610949090172393\\
509	0.00615665551239479\\
510	0.00620542343797994\\
511	0.0062557606265721\\
512	0.00630760115894419\\
513	0.00636083816958934\\
514	0.00641531503036756\\
515	0.00647080979324049\\
516	0.00652709240422443\\
517	0.00658382920575679\\
518	0.00664047926688795\\
519	0.00669615993705724\\
520	0.00674721454700777\\
521	0.00679300276204671\\
522	0.00683346750418282\\
523	0.00686898993635551\\
524	0.00690332027777779\\
525	0.0069367898869669\\
526	0.00696958952394345\\
527	0.00700219544568973\\
528	0.00703483264984312\\
529	0.00706757580144111\\
530	0.00710071095286098\\
531	0.00713443674950731\\
532	0.00716879488284926\\
533	0.00720382464990422\\
534	0.00723957004356582\\
535	0.0072760672185886\\
536	0.0073133380867663\\
537	0.00735140403183977\\
538	0.00739028672141112\\
539	0.00743000712956586\\
540	0.0074705856411482\\
541	0.00751207498393718\\
542	0.00755455462753834\\
543	0.00759801868419028\\
544	0.00764058526073285\\
545	0.00768201238911643\\
546	0.00772274188174216\\
547	0.00776389966120609\\
548	0.00780555674788852\\
549	0.00784779558255256\\
550	0.0078906495142133\\
551	0.0079341122243053\\
552	0.00797817066749847\\
553	0.00802280696666616\\
554	0.00806799920954878\\
555	0.00811372102983461\\
556	0.00815993993413199\\
557	0.00820663589619449\\
558	0.00825380560204618\\
559	0.00829968509170375\\
560	0.00834448382791002\\
561	0.00838952352702058\\
562	0.00843487712750558\\
563	0.00848055736193632\\
564	0.0085265366415809\\
565	0.00857278410820387\\
566	0.0086192674609727\\
567	0.00866595340358717\\
568	0.00871280786305118\\
569	0.00875980412280098\\
570	0.00880656228901083\\
571	0.00885311969943392\\
572	0.00889982902690841\\
573	0.00894667662497712\\
574	0.00899362444608677\\
575	0.00904063165454186\\
576	0.00908765467045251\\
577	0.00913464722627185\\
578	0.00918156049093735\\
579	0.00922834328141122\\
580	0.00927494238595896\\
581	0.00932130302929071\\
582	0.00936736951671606\\
583	0.00941308610287967\\
584	0.00945839814032784\\
585	0.00950325357339596\\
586	0.00954760485097041\\
587	0.00959141132963419\\
588	0.00963464220433353\\
589	0.00967727987662474\\
590	0.00971932328838559\\
591	0.0097606871825126\\
592	0.00980121135503816\\
593	0.00984070634831441\\
594	0.0098789201112722\\
595	0.00991544794299161\\
596	0.00994949078513196\\
597	0.00997920912842892\\
598	0.0100000808076597\\
599	0\\
600	0\\
};
\addplot [color=mycolor6,solid,forget plot]
  table[row sep=crcr]{%
1	0.00443988434698203\\
2	0.00443988810065609\\
3	0.00443989191971127\\
4	0.00443989580528943\\
5	0.00443989975855245\\
6	0.00443990378068262\\
7	0.00443990787288285\\
8	0.00443991203637715\\
9	0.00443991627241093\\
10	0.00443992058225149\\
11	0.00443992496718828\\
12	0.0044399294285334\\
13	0.00443993396762186\\
14	0.0044399385858121\\
15	0.0044399432844864\\
16	0.00443994806505117\\
17	0.00443995292893754\\
18	0.00443995787760166\\
19	0.00443996291252527\\
20	0.00443996803521602\\
21	0.00443997324720802\\
22	0.00443997855006225\\
23	0.0044399839453671\\
24	0.00443998943473876\\
25	0.00443999501982173\\
26	0.00444000070228939\\
27	0.00444000648384447\\
28	0.00444001236621951\\
29	0.00444001835117744\\
30	0.00444002444051216\\
31	0.00444003063604895\\
32	0.00444003693964519\\
33	0.00444004335319077\\
34	0.00444004987860877\\
35	0.00444005651785596\\
36	0.00444006327292345\\
37	0.00444007014583734\\
38	0.00444007713865924\\
39	0.00444008425348698\\
40	0.00444009149245508\\
41	0.00444009885773562\\
42	0.00444010635153883\\
43	0.00444011397611372\\
44	0.00444012173374867\\
45	0.00444012962677235\\
46	0.0044401376575543\\
47	0.00444014582850562\\
48	0.0044401541420798\\
49	0.00444016260077338\\
50	0.00444017120712676\\
51	0.00444017996372497\\
52	0.00444018887319838\\
53	0.00444019793822371\\
54	0.00444020716152459\\
55	0.00444021654587255\\
56	0.00444022609408784\\
57	0.00444023580904025\\
58	0.00444024569365004\\
59	0.00444025575088877\\
60	0.00444026598378027\\
61	0.00444027639540157\\
62	0.00444028698888374\\
63	0.00444029776741294\\
64	0.00444030873423142\\
65	0.0044403198926384\\
66	0.00444033124599125\\
67	0.0044403427977063\\
68	0.00444035455126014\\
69	0.00444036651019041\\
70	0.00444037867809713\\
71	0.00444039105864369\\
72	0.00444040365555797\\
73	0.00444041647263353\\
74	0.00444042951373083\\
75	0.00444044278277819\\
76	0.00444045628377332\\
77	0.00444047002078431\\
78	0.00444048399795099\\
79	0.00444049821948622\\
80	0.00444051268967711\\
81	0.00444052741288647\\
82	0.00444054239355404\\
83	0.00444055763619792\\
84	0.00444057314541599\\
85	0.00444058892588734\\
86	0.00444060498237369\\
87	0.0044406213197209\\
88	0.00444063794286041\\
89	0.00444065485681087\\
90	0.00444067206667971\\
91	0.00444068957766468\\
92	0.00444070739505545\\
93	0.00444072552423538\\
94	0.00444074397068309\\
95	0.00444076273997429\\
96	0.00444078183778343\\
97	0.00444080126988558\\
98	0.00444082104215814\\
99	0.00444084116058286\\
100	0.00444086163124759\\
101	0.00444088246034824\\
102	0.00444090365419084\\
103	0.00444092521919338\\
104	0.00444094716188799\\
105	0.00444096948892307\\
106	0.00444099220706523\\
107	0.00444101532320159\\
108	0.00444103884434197\\
109	0.00444106277762114\\
110	0.00444108713030115\\
111	0.00444111190977358\\
112	0.00444113712356214\\
113	0.00444116277932484\\
114	0.00444118888485667\\
115	0.00444121544809212\\
116	0.00444124247710777\\
117	0.00444126998012494\\
118	0.00444129796551235\\
119	0.00444132644178891\\
120	0.00444135541762656\\
121	0.00444138490185321\\
122	0.00444141490345554\\
123	0.00444144543158201\\
124	0.00444147649554614\\
125	0.00444150810482941\\
126	0.00444154026908453\\
127	0.00444157299813877\\
128	0.00444160630199719\\
129	0.00444164019084616\\
130	0.00444167467505678\\
131	0.00444170976518845\\
132	0.00444174547199254\\
133	0.00444178180641608\\
134	0.00444181877960557\\
135	0.00444185640291085\\
136	0.00444189468788906\\
137	0.00444193364630882\\
138	0.00444197329015417\\
139	0.00444201363162899\\
140	0.00444205468316129\\
141	0.00444209645740751\\
142	0.00444213896725723\\
143	0.00444218222583761\\
144	0.00444222624651834\\
145	0.0044422710429162\\
146	0.00444231662890017\\
147	0.0044423630185964\\
148	0.00444241022639334\\
149	0.00444245826694698\\
150	0.00444250715518622\\
151	0.00444255690631831\\
152	0.00444260753583436\\
153	0.00444265905951504\\
154	0.0044427114934364\\
155	0.00444276485397558\\
156	0.00444281915781699\\
157	0.00444287442195816\\
158	0.00444293066371604\\
159	0.00444298790073329\\
160	0.0044430461509843\\
161	0.00444310543278208\\
162	0.0044431657647843\\
163	0.00444322716600013\\
164	0.00444328965579667\\
165	0.0044433532539057\\
166	0.00444341798043029\\
167	0.00444348385585156\\
168	0.00444355090103533\\
169	0.00444361913723894\\
170	0.00444368858611777\\
171	0.00444375926973208\\
172	0.00444383121055344\\
173	0.00444390443147135\\
174	0.00444397895579972\\
175	0.00444405480728312\\
176	0.00444413201010307\\
177	0.00444421058888409\\
178	0.00444429056869964\\
179	0.00444437197507792\\
180	0.00444445483400749\\
181	0.0044445391719428\\
182	0.00444462501580951\\
183	0.00444471239300965\\
184	0.00444480133142676\\
185	0.00444489185943108\\
186	0.00444498400588461\\
187	0.00444507780014642\\
188	0.00444517327207821\\
189	0.00444527045205028\\
190	0.00444536937094812\\
191	0.00444547006017983\\
192	0.00444557255168471\\
193	0.00444567687794316\\
194	0.00444578307198838\\
195	0.00444589116741974\\
196	0.00444600119841847\\
197	0.00444611319976587\\
198	0.00444622720686423\\
199	0.00444634325575604\\
200	0.00444646138313612\\
201	0.00444658162636341\\
202	0.00444670402347298\\
203	0.00444682861318829\\
204	0.00444695543493366\\
205	0.0044470845288467\\
206	0.00444721593579131\\
207	0.00444734969737062\\
208	0.00444748585594\\
209	0.00444762445462069\\
210	0.00444776553731306\\
211	0.0044479091487105\\
212	0.00444805533431314\\
213	0.00444820414044198\\
214	0.00444835561425298\\
215	0.00444850980375126\\
216	0.00444866675780565\\
217	0.00444882652616311\\
218	0.00444898915946344\\
219	0.00444915470925381\\
220	0.0044493232280035\\
221	0.0044494947691187\\
222	0.0044496693869572\\
223	0.00444984713684309\\
224	0.0044500280750815\\
225	0.0044502122589729\\
226	0.00445039974682779\\
227	0.00445059059798063\\
228	0.00445078487280411\\
229	0.00445098263272253\\
230	0.00445118394022547\\
231	0.0044513888588805\\
232	0.00445159745334574\\
233	0.0044518097893818\\
234	0.00445202593386317\\
235	0.00445224595478862\\
236	0.00445246992129131\\
237	0.00445269790364745\\
238	0.00445292997328459\\
239	0.00445316620278823\\
240	0.0044534066659077\\
241	0.00445365143756057\\
242	0.0044539005938358\\
243	0.00445415421199513\\
244	0.00445441237047334\\
245	0.00445467514887631\\
246	0.00445494262797784\\
247	0.0044552148897143\\
248	0.00445549201717753\\
249	0.00445577409460598\\
250	0.00445606120737381\\
251	0.00445635344197828\\
252	0.00445665088602554\\
253	0.00445695362821463\\
254	0.00445726175832059\\
255	0.00445757536717635\\
256	0.00445789454665423\\
257	0.00445821938964761\\
258	0.00445854999005319\\
259	0.00445888644275512\\
260	0.00445922884361175\\
261	0.00445957728944636\\
262	0.00445993187804314\\
263	0.00446029270815044\\
264	0.004460659879493\\
265	0.0044610334927951\\
266	0.00446141364981754\\
267	0.00446180045341031\\
268	0.00446219400758403\\
269	0.00446259441760274\\
270	0.00446300179010076\\
271	0.00446341623322544\\
272	0.00446383785680865\\
273	0.00446426677256683\\
274	0.00446470309433072\\
275	0.00446514693830279\\
276	0.00446559842333933\\
277	0.00446605767125373\\
278	0.00446652480713784\\
279	0.00446699995969591\\
280	0.00446748326076807\\
281	0.00446797484416975\\
282	0.00446847484571485\\
283	0.00446898340323895\\
284	0.00446950065662241\\
285	0.00447002674781371\\
286	0.00447056182085257\\
287	0.00447110602189325\\
288	0.00447165949922793\\
289	0.0044722224033098\\
290	0.00447279488677649\\
291	0.0044733771044732\\
292	0.00447396921347605\\
293	0.00447457137311504\\
294	0.00447518374499742\\
295	0.00447580649303063\\
296	0.0044764397834454\\
297	0.00447708378481865\\
298	0.00447773866809647\\
299	0.00447840460661679\\
300	0.00447908177613215\\
301	0.00447977035483232\\
302	0.00448047052336679\\
303	0.00448118246486716\\
304	0.00448190636496947\\
305	0.00448264241183628\\
306	0.00448339079617894\\
307	0.00448415171127922\\
308	0.00448492535301138\\
309	0.00448571191986387\\
310	0.0044865116129608\\
311	0.00448732463608351\\
312	0.00448815119569212\\
313	0.00448899150094657\\
314	0.00448984576372821\\
315	0.00449071419866074\\
316	0.00449159702313159\\
317	0.00449249445731291\\
318	0.00449340672418299\\
319	0.0044943340495473\\
320	0.0044952766620599\\
321	0.00449623479324491\\
322	0.00449720867751817\\
323	0.00449819855220906\\
324	0.00449920465758273\\
325	0.00450022723686248\\
326	0.00450126653625281\\
327	0.00450232280496273\\
328	0.00450339629522986\\
329	0.00450448726234507\\
330	0.00450559596467787\\
331	0.00450672266370287\\
332	0.00450786762402698\\
333	0.00450903111341803\\
334	0.00451021340283434\\
335	0.00451141476645583\\
336	0.00451263548171685\\
337	0.00451387582934027\\
338	0.00451513609337392\\
339	0.00451641656122851\\
340	0.00451771752371825\\
341	0.00451903927510343\\
342	0.00452038211313559\\
343	0.0045217463391056\\
344	0.004523132257894\\
345	0.00452454017802477\\
346	0.00452597041172187\\
347	0.00452742327496899\\
348	0.00452889908757225\\
349	0.00453039817322656\\
350	0.00453192085958441\\
351	0.00453346747832824\\
352	0.00453503836524521\\
353	0.00453663386030474\\
354	0.004538254307738\\
355	0.00453990005611897\\
356	0.00454157145844665\\
357	0.00454326887222697\\
358	0.0045449926595541\\
359	0.00454674318718944\\
360	0.00454852082663708\\
361	0.00455032595421382\\
362	0.00455215895111216\\
363	0.00455402020345356\\
364	0.00455591010233179\\
365	0.00455782904383779\\
366	0.00455977742907157\\
367	0.00456175566413174\\
368	0.00456376416008264\\
369	0.00456580333289546\\
370	0.00456787360336139\\
371	0.00456997539697468\\
372	0.00457210914378513\\
373	0.00457427527822062\\
374	0.00457647423888237\\
375	0.00457870646831864\\
376	0.00458097241278601\\
377	0.0045832725220125\\
378	0.00458560724898194\\
379	0.00458797704976702\\
380	0.00459038238344545\\
381	0.0045928237121427\\
382	0.00459530150124938\\
383	0.00459781621985977\\
384	0.00460036834145874\\
385	0.00460295834485448\\
386	0.0046055867154491\\
387	0.00460825394806984\\
388	0.00461096068773609\\
389	0.00461370786248071\\
390	0.00461649646736497\\
391	0.00461932756750267\\
392	0.00462220229993865\\
393	0.00462512187390994\\
394	0.00462808756894889\\
395	0.00463110073025394\\
396	0.00463416276087825\\
397	0.00463727511003961\\
398	0.00464043925712366\\
399	0.00464365669156101\\
400	0.00464692888928838\\
401	0.00465025728740081\\
402	0.00465364326015535\\
403	0.00465708810424914\\
404	0.00466059305644298\\
405	0.00466415935466607\\
406	0.00466778828544959\\
407	0.00467148118751305\\
408	0.0046752394557974\\
409	0.00467906454583477\\
410	0.00468295797846878\\
411	0.00468692134493405\\
412	0.0046909563122964\\
413	0.00469506462924373\\
414	0.00469924813220045\\
415	0.00470350875171316\\
416	0.00470784851902661\\
417	0.00471226957276226\\
418	0.00471677416576284\\
419	0.0047213646729592\\
420	0.00472604359814634\\
421	0.00473081357997302\\
422	0.00473567739667314\\
423	0.00474063796824552\\
424	0.00474569835258098\\
425	0.00475086172443716\\
426	0.00475613129875188\\
427	0.00476150100860485\\
428	0.00476696174085321\\
429	0.00477251514646555\\
430	0.00477816291302866\\
431	0.00478390676532251\\
432	0.00478974846443166\\
433	0.00479568979889695\\
434	0.00480173258390381\\
435	0.00480787866525026\\
436	0.00481412991719563\\
437	0.0048204882397668\\
438	0.00482695555562097\\
439	0.00483353380683453\\
440	0.00484022495192817\\
441	0.00484703095990086\\
442	0.00485395380320884\\
443	0.00486099545125395\\
444	0.00486815786339394\\
445	0.00487544298166432\\
446	0.00488285272342851\\
447	0.00489038897402912\\
448	0.00489805357877234\\
449	0.00490584833087968\\
450	0.00491377494304385\\
451	0.00492183496071388\\
452	0.00493003216004933\\
453	0.00493839580428843\\
454	0.00494693047413742\\
455	0.00495564094409584\\
456	0.0049645322073708\\
457	0.00497360949939487\\
458	0.00498287831645387\\
459	0.00499234443492129\\
460	0.00500201393247607\\
461	0.00501189321222362\\
462	0.00502198902795739\\
463	0.00503230851103793\\
464	0.00504285920001722\\
465	0.00505364907279017\\
466	0.00506468658177601\\
467	0.0050759806936686\\
468	0.00508754093911969\\
469	0.00509937749131456\\
470	0.00511150134088587\\
471	0.00512392480916206\\
472	0.00513666050747433\\
473	0.00514972061170796\\
474	0.00516311757134849\\
475	0.00517686450681094\\
476	0.00519097524949022\\
477	0.00520546438399243\\
478	0.00522034729241036\\
479	0.00523564020046909\\
480	0.00525136022574413\\
481	0.00526752542836824\\
482	0.00528415486317811\\
483	0.00530126863512505\\
484	0.00531888796705329\\
485	0.0053370353329557\\
486	0.00535573456835634\\
487	0.00537501076054266\\
488	0.0053948902036953\\
489	0.00541540337912831\\
490	0.00543659238149784\\
491	0.00545850651265662\\
492	0.00548118177392959\\
493	0.00550465614572163\\
494	0.00552896928868013\\
495	0.0055541624320719\\
496	0.00558027885924683\\
497	0.0056073639329982\\
498	0.00563546510812211\\
499	0.0056646318956748\\
500	0.00569491578478761\\
501	0.00572637010982894\\
502	0.00575904985295611\\
503	0.00579301136223277\\
504	0.00582831195856728\\
505	0.00586500940809809\\
506	0.00590316122678925\\
507	0.00594282377632008\\
508	0.00598405110070143\\
509	0.00602689343924358\\
510	0.00607139534233941\\
511	0.00611759329550034\\
512	0.0061655127590368\\
513	0.00621516445237077\\
514	0.00626653967464605\\
515	0.00631960498957062\\
516	0.00637429251727781\\
517	0.00643048735026673\\
518	0.00648801610107528\\
519	0.00654664820033151\\
520	0.00660616038086933\\
521	0.00666619595096453\\
522	0.00672617728114806\\
523	0.00678515195101265\\
524	0.00683910531560947\\
525	0.00688772972939196\\
526	0.00693095305790393\\
527	0.00696917291998219\\
528	0.00700618708052248\\
529	0.00704241005528292\\
530	0.00707806567073125\\
531	0.00711349520102825\\
532	0.00714895646519449\\
533	0.00718454591507818\\
534	0.00722032123488973\\
535	0.00725669086249029\\
536	0.00729371981307691\\
537	0.00733144877320762\\
538	0.00736991935045373\\
539	0.0074091751334022\\
540	0.00744925012719998\\
541	0.00749016740387005\\
542	0.00753194943633351\\
543	0.00757462036511259\\
544	0.00761825099356957\\
545	0.00766292592326124\\
546	0.00770799862809186\\
547	0.0077521313044696\\
548	0.00779506334357574\\
549	0.00783767294073671\\
550	0.00788071852488076\\
551	0.00792427223635543\\
552	0.0079684064144763\\
553	0.00801313566676493\\
554	0.00805844748222967\\
555	0.00810432169468469\\
556	0.00815073228634606\\
557	0.00819764750106807\\
558	0.00824502845287495\\
559	0.00829286730001371\\
560	0.00834063036016767\\
561	0.00838688363971901\\
562	0.00843252797332684\\
563	0.00847839154188996\\
564	0.00852453691216508\\
565	0.00857095865675348\\
566	0.00861762463421168\\
567	0.0086644998927601\\
568	0.00871154829412204\\
569	0.00875873284312044\\
570	0.0088060237500657\\
571	0.008853089225684\\
572	0.00889982844254686\\
573	0.00894667661741767\\
574	0.00899362444364643\\
575	0.0090406316533593\\
576	0.00908765466988921\\
577	0.00913464722601802\\
578	0.009181560490831\\
579	0.00922834328137045\\
580	0.00927494238594509\\
581	0.00932130302928644\\
582	0.009367369516715\\
583	0.00941308610287962\\
584	0.00945839814032784\\
585	0.00950325357339596\\
586	0.00954760485097041\\
587	0.00959141132963419\\
588	0.00963464220433353\\
589	0.00967727987662474\\
590	0.00971932328838559\\
591	0.0097606871825126\\
592	0.00980121135503816\\
593	0.00984070634831441\\
594	0.0098789201112722\\
595	0.00991544794299161\\
596	0.00994949078513196\\
597	0.00997920912842892\\
598	0.0100000808076597\\
599	0\\
600	0\\
};
\addplot [color=mycolor7,solid,forget plot]
  table[row sep=crcr]{%
1	0.00434651247971337\\
2	0.00434651807483875\\
3	0.00434652376590437\\
4	0.00434652955455554\\
5	0.0043465354424658\\
6	0.00434654143133733\\
7	0.00434654752290159\\
8	0.00434655371891963\\
9	0.00434656002118271\\
10	0.00434656643151284\\
11	0.00434657295176314\\
12	0.00434657958381858\\
13	0.00434658632959639\\
14	0.00434659319104669\\
15	0.00434660017015299\\
16	0.00434660726893279\\
17	0.00434661448943816\\
18	0.00434662183375635\\
19	0.0043466293040103\\
20	0.00434663690235946\\
21	0.00434664463100013\\
22	0.00434665249216633\\
23	0.00434666048813029\\
24	0.00434666862120322\\
25	0.00434667689373585\\
26	0.00434668530811926\\
27	0.00434669386678544\\
28	0.00434670257220797\\
29	0.00434671142690292\\
30	0.00434672043342936\\
31	0.00434672959439022\\
32	0.00434673891243299\\
33	0.00434674839025051\\
34	0.00434675803058171\\
35	0.00434676783621243\\
36	0.00434677780997621\\
37	0.0043467879547551\\
38	0.00434679827348043\\
39	0.00434680876913382\\
40	0.00434681944474788\\
41	0.00434683030340717\\
42	0.00434684134824894\\
43	0.0043468525824642\\
44	0.0043468640092986\\
45	0.00434687563205329\\
46	0.00434688745408597\\
47	0.00434689947881177\\
48	0.00434691170970424\\
49	0.00434692415029637\\
50	0.00434693680418164\\
51	0.00434694967501497\\
52	0.00434696276651383\\
53	0.00434697608245924\\
54	0.00434698962669694\\
55	0.00434700340313852\\
56	0.00434701741576235\\
57	0.00434703166861491\\
58	0.00434704616581192\\
59	0.00434706091153941\\
60	0.00434707591005509\\
61	0.00434709116568936\\
62	0.00434710668284681\\
63	0.00434712246600726\\
64	0.00434713851972711\\
65	0.00434715484864072\\
66	0.00434717145746163\\
67	0.00434718835098404\\
68	0.00434720553408402\\
69	0.00434722301172106\\
70	0.0043472407889394\\
71	0.00434725887086946\\
72	0.00434727726272942\\
73	0.00434729596982663\\
74	0.00434731499755911\\
75	0.0043473343514172\\
76	0.00434735403698499\\
77	0.00434737405994207\\
78	0.00434739442606508\\
79	0.00434741514122932\\
80	0.00434743621141058\\
81	0.00434745764268664\\
82	0.00434747944123926\\
83	0.00434750161335576\\
84	0.00434752416543092\\
85	0.00434754710396878\\
86	0.00434757043558451\\
87	0.0043475941670063\\
88	0.00434761830507734\\
89	0.00434764285675772\\
90	0.00434766782912651\\
91	0.00434769322938371\\
92	0.00434771906485238\\
93	0.00434774534298071\\
94	0.00434777207134421\\
95	0.00434779925764781\\
96	0.0043478269097282\\
97	0.00434785503555596\\
98	0.00434788364323801\\
99	0.00434791274101977\\
100	0.00434794233728766\\
101	0.00434797244057159\\
102	0.00434800305954726\\
103	0.00434803420303881\\
104	0.00434806588002127\\
105	0.00434809809962327\\
106	0.00434813087112966\\
107	0.00434816420398422\\
108	0.00434819810779235\\
109	0.00434823259232393\\
110	0.00434826766751625\\
111	0.00434830334347682\\
112	0.00434833963048625\\
113	0.00434837653900156\\
114	0.004348414079659\\
115	0.00434845226327734\\
116	0.00434849110086093\\
117	0.00434853060360309\\
118	0.00434857078288946\\
119	0.0043486116503013\\
120	0.00434865321761904\\
121	0.00434869549682575\\
122	0.00434873850011081\\
123	0.00434878223987357\\
124	0.00434882672872713\\
125	0.00434887197950216\\
126	0.00434891800525089\\
127	0.00434896481925105\\
128	0.00434901243501006\\
129	0.00434906086626916\\
130	0.00434911012700772\\
131	0.0043491602314477\\
132	0.00434921119405805\\
133	0.00434926302955933\\
134	0.00434931575292859\\
135	0.00434936937940399\\
136	0.00434942392448984\\
137	0.0043494794039618\\
138	0.00434953583387194\\
139	0.0043495932305541\\
140	0.00434965161062948\\
141	0.00434971099101228\\
142	0.00434977138891548\\
143	0.00434983282185682\\
144	0.00434989530766492\\
145	0.00434995886448566\\
146	0.00435002351078871\\
147	0.00435008926537415\\
148	0.00435015614737946\\
149	0.00435022417628672\\
150	0.00435029337192988\\
151	0.0043503637545024\\
152	0.00435043534456512\\
153	0.0043505081630544\\
154	0.00435058223129048\\
155	0.00435065757098619\\
156	0.00435073420425586\\
157	0.00435081215362476\\
158	0.00435089144203846\\
159	0.00435097209287301\\
160	0.00435105412994511\\
161	0.00435113757752277\\
162	0.00435122246033636\\
163	0.00435130880358993\\
164	0.00435139663297303\\
165	0.00435148597467282\\
166	0.00435157685538663\\
167	0.00435166930233494\\
168	0.00435176334327458\\
169	0.00435185900651246\\
170	0.00435195632091966\\
171	0.00435205531594563\\
172	0.00435215602163291\\
173	0.00435225846863204\\
174	0.00435236268821644\\
175	0.00435246871229779\\
176	0.00435257657344091\\
177	0.00435268630487905\\
178	0.00435279794052846\\
179	0.00435291151500279\\
180	0.0043530270636268\\
181	0.00435314462244911\\
182	0.00435326422825376\\
183	0.00435338591857034\\
184	0.00435350973168217\\
185	0.00435363570663192\\
186	0.00435376388322452\\
187	0.00435389430202661\\
188	0.00435402700436173\\
189	0.00435416203230097\\
190	0.00435429942864798\\
191	0.00435443923691831\\
192	0.00435458150131203\\
193	0.00435472626667982\\
194	0.00435487357848263\\
195	0.00435502348274559\\
196	0.00435517602600866\\
197	0.00435533125528079\\
198	0.00435548921802337\\
199	0.00435564996223779\\
200	0.00435581353674525\\
201	0.00435597999121139\\
202	0.00435614937615997\\
203	0.00435632174298671\\
204	0.00435649714397315\\
205	0.0043566756323009\\
206	0.00435685726206562\\
207	0.00435704208829158\\
208	0.00435723016694605\\
209	0.00435742155495391\\
210	0.00435761631021251\\
211	0.00435781449160627\\
212	0.00435801615902183\\
213	0.004358221373363\\
214	0.00435843019656594\\
215	0.0043586426916143\\
216	0.00435885892255465\\
217	0.00435907895451167\\
218	0.00435930285370362\\
219	0.00435953068745778\\
220	0.00435976252422609\\
221	0.00435999843360048\\
222	0.00436023848632848\\
223	0.00436048275432877\\
224	0.00436073131070671\\
225	0.00436098422976998\\
226	0.00436124158704396\\
227	0.00436150345928731\\
228	0.00436176992450741\\
229	0.00436204106197586\\
230	0.00436231695224373\\
231	0.0043625976771571\\
232	0.0043628833198723\\
233	0.00436317396487116\\
234	0.00436346969797637\\
235	0.00436377060636668\\
236	0.00436407677859219\\
237	0.00436438830458961\\
238	0.00436470527569768\\
239	0.00436502778467255\\
240	0.00436535592570346\\
241	0.00436568979442836\\
242	0.00436602948795006\\
243	0.00436637510485249\\
244	0.00436672674521755\\
245	0.00436708451064226\\
246	0.00436744850425658\\
247	0.00436781883074209\\
248	0.0043681955963513\\
249	0.00436857890892805\\
250	0.00436896887792922\\
251	0.0043693656144474\\
252	0.00436976923123555\\
253	0.00437017984273279\\
254	0.00437059756509263\\
255	0.00437102251621283\\
256	0.00437145481576781\\
257	0.0043718945852433\\
258	0.00437234194797376\\
259	0.00437279702918238\\
260	0.00437325995602403\\
261	0.0043737308576308\\
262	0.00437420986516062\\
263	0.00437469711184832\\
264	0.00437519273305914\\
265	0.00437569686634425\\
266	0.00437620965149783\\
267	0.00437673123061478\\
268	0.0043772617481485\\
269	0.00437780135096741\\
270	0.00437835018840901\\
271	0.00437890841233031\\
272	0.00437947617715241\\
273	0.00438005363989857\\
274	0.00438064096022314\\
275	0.00438123830043056\\
276	0.0043818458254833\\
277	0.00438246370299778\\
278	0.00438309210322861\\
279	0.00438373119904123\\
280	0.00438438116590097\\
281	0.00438504218190184\\
282	0.00438571442779544\\
283	0.00438639808701926\\
284	0.00438709334572505\\
285	0.00438780039280657\\
286	0.00438851941992721\\
287	0.00438925062154722\\
288	0.00438999419495006\\
289	0.00439075034026885\\
290	0.00439151926051175\\
291	0.00439230116158695\\
292	0.00439309625232684\\
293	0.00439390474451143\\
294	0.00439472685289098\\
295	0.00439556279520765\\
296	0.00439641279221586\\
297	0.00439727706770208\\
298	0.00439815584850282\\
299	0.00439904936452164\\
300	0.00439995784874471\\
301	0.00440088153725455\\
302	0.00440182066924226\\
303	0.0044027754870179\\
304	0.00440374623601882\\
305	0.00440473316481585\\
306	0.00440573652511716\\
307	0.00440675657176963\\
308	0.00440779356275753\\
309	0.00440884775919806\\
310	0.00440991942533418\\
311	0.00441100882852359\\
312	0.00441211623922427\\
313	0.00441324193097611\\
314	0.00441438618037822\\
315	0.00441554926706176\\
316	0.00441673147365777\\
317	0.00441793308575983\\
318	0.00441915439188096\\
319	0.00442039568340457\\
320	0.00442165725452871\\
321	0.00442293940220322\\
322	0.00442424242605964\\
323	0.00442556662833265\\
324	0.00442691231377286\\
325	0.00442827978955041\\
326	0.00442966936514809\\
327	0.00443108135224411\\
328	0.00443251606458275\\
329	0.00443397381783293\\
330	0.00443545492943302\\
331	0.00443695971842148\\
332	0.00443848850525196\\
333	0.0044400416115916\\
334	0.00444161936010186\\
335	0.00444322207419993\\
336	0.00444485007779979\\
337	0.00444650369503125\\
338	0.00444818324993537\\
339	0.00444988906613483\\
340	0.00445162146647689\\
341	0.00445338077264792\\
342	0.00445516730475664\\
343	0.00445698138088441\\
344	0.00445882331660034\\
345	0.00446069342443883\\
346	0.00446259201333687\\
347	0.00446451938802887\\
348	0.00446647584839628\\
349	0.00446846168876915\\
350	0.00447047719717716\\
351	0.00447252265454689\\
352	0.00447459833384303\\
353	0.00447670449915026\\
354	0.00447884140469353\\
355	0.00448100929379416\\
356	0.00448320839775927\\
357	0.00448543893470306\\
358	0.00448770110829806\\
359	0.00448999510645602\\
360	0.00449232109993792\\
361	0.0044946792408949\\
362	0.00449706966134178\\
363	0.00449949247156745\\
364	0.00450194775848845\\
365	0.00450443558395387\\
366	0.00450695598301332\\
367	0.00450950896216333\\
368	0.00451209449759191\\
369	0.00451471253344604\\
370	0.00451736298015452\\
371	0.00452004571284449\\
372	0.00452276056990067\\
373	0.0045255073517265\\
374	0.00452828581977889\\
375	0.00453109569596419\\
376	0.00453393666249946\\
377	0.00453680836236421\\
378	0.00453971040049124\\
379	0.00454264234587192\\
380	0.00454560373478265\\
381	0.00454859407537343\\
382	0.00455161285389888\\
383	0.00455465954291459\\
384	0.00455773361181121\\
385	0.00456083454011459\\
386	0.00456396183405238\\
387	0.00456711504693599\\
388	0.00457029379940714\\
389	0.00457349780034827\\
390	0.00457672688067931\\
391	0.00457998103358608\\
392	0.00458326046160581\\
393	0.00458656563070792\\
394	0.00458989733095649\\
395	0.00459325674239786\\
396	0.00459664550323233\\
397	0.00460006577507293\\
398	0.00460352029713849\\
399	0.00460701241654694\\
400	0.00461054607469983\\
401	0.00461412571844882\\
402	0.0046177560838761\\
403	0.00462144175842826\\
404	0.00462518636377057\\
405	0.00462899184650963\\
406	0.00463285911302995\\
407	0.00463678907706104\\
408	0.00464078265873753\\
409	0.00464484078349815\\
410	0.00464896438080981\\
411	0.00465315438270493\\
412	0.00465741172212179\\
413	0.00466173733104292\\
414	0.00466613213843183\\
415	0.00467059706797774\\
416	0.00467513303566899\\
417	0.00467974094723067\\
418	0.00468442169547462\\
419	0.00468917615760253\\
420	0.00469400519259262\\
421	0.00469890963880159\\
422	0.00470389031192407\\
423	0.00470894800344199\\
424	0.00471408347967937\\
425	0.0047192974818461\\
426	0.00472459072990329\\
427	0.0047299642343399\\
428	0.00473541941624471\\
429	0.00474095777893774\\
430	0.00474658091771805\\
431	0.00475229053063307\\
432	0.00475808843036846\\
433	0.00476397655756508\\
434	0.00476995699535457\\
435	0.00477603198497289\\
436	0.0047822039426173\\
437	0.00478847547748989\\
438	0.00479484941090554\\
439	0.00480132879623754\\
440	0.00480791693937004\\
441	0.00481461741927852\\
442	0.00482143410814186\\
443	0.00482837119013039\\
444	0.00483543317780448\\
445	0.00484262492476696\\
446	0.00484995163286941\\
447	0.00485741885182526\\
448	0.00486503246832815\\
449	0.00487279867991183\\
450	0.00488072394276536\\
451	0.00488881486105151\\
452	0.00489707538689978\\
453	0.00490548404132321\\
454	0.00491404485375322\\
455	0.00492276168372391\\
456	0.00493163820743163\\
457	0.00494067803111305\\
458	0.00494988485894009\\
459	0.00495926251721295\\
460	0.00496881494966901\\
461	0.00497854621360363\\
462	0.00498846051579542\\
463	0.00499856220289974\\
464	0.00500885576778474\\
465	0.0050193458562456\\
466	0.00503003727488824\\
467	0.00504093499902502\\
468	0.0050520441772743\\
469	0.00506337012180137\\
470	0.00507491824921319\\
471	0.00508669386210986\\
472	0.00509872880099538\\
473	0.0051110445927747\\
474	0.00512365181573613\\
475	0.00513656163478061\\
476	0.00514978583460693\\
477	0.00516333685398426\\
478	0.00517722782682597\\
479	0.00519147262857243\\
480	0.00520608592520309\\
481	0.00522108322762701\\
482	0.0052364809530022\\
483	0.00525229649540331\\
484	0.00526854831198945\\
485	0.00528525604166932\\
486	0.00530244071944048\\
487	0.00532012531372603\\
488	0.00533833440174359\\
489	0.00535709274577658\\
490	0.00537642549678822\\
491	0.0053963593062611\\
492	0.00541692526772349\\
493	0.00543816679012038\\
494	0.0054601328511869\\
495	0.00548286009345773\\
496	0.00550638725625778\\
497	0.00553075495764331\\
498	0.00555600539523579\\
499	0.00558218296489853\\
500	0.0056093343063611\\
501	0.00563750833293152\\
502	0.00566675616676641\\
503	0.00569713112249652\\
504	0.00572868871419175\\
505	0.0057614865281743\\
506	0.00579558402397112\\
507	0.00583104224483216\\
508	0.00586792341477941\\
509	0.00590629039435073\\
510	0.00594620595978366\\
511	0.00598773186204981\\
512	0.00603092761168722\\
513	0.00607584892414014\\
514	0.00612254574526682\\
515	0.00617105971326137\\
516	0.00622142104475782\\
517	0.0062736448941418\\
518	0.00632772683723461\\
519	0.00638363609952925\\
520	0.00644129768518299\\
521	0.00650058939073116\\
522	0.00656133712411865\\
523	0.00662330937190876\\
524	0.00668628930360181\\
525	0.00674991041805484\\
526	0.00681357797886698\\
527	0.00687629325884639\\
528	0.00693400058095519\\
529	0.00698635631962089\\
530	0.00703323780204888\\
531	0.00707498040099682\\
532	0.0071151436386198\\
533	0.00715440950427348\\
534	0.00719322716001215\\
535	0.00723172791743309\\
536	0.00727021881004343\\
537	0.00730883786633706\\
538	0.00734765867244396\\
539	0.00738683620058192\\
540	0.00742669148898072\\
541	0.00746726784514716\\
542	0.00750860710181578\\
543	0.00755075442426301\\
544	0.0075937523278058\\
545	0.00763763032101427\\
546	0.00768242812659327\\
547	0.00772822075283625\\
548	0.00777509749459468\\
549	0.00782196520821078\\
550	0.00786787090563359\\
551	0.00791254676205331\\
552	0.00795709422414918\\
553	0.00800206058870172\\
554	0.00804751255690859\\
555	0.00809351717962067\\
556	0.00814008242930643\\
557	0.00818718847282931\\
558	0.0082348069380387\\
559	0.0082829008319438\\
560	0.00833143697722997\\
561	0.00838040403829811\\
562	0.00842874919706433\\
563	0.00847554775795152\\
564	0.00852200777261543\\
565	0.00856864468872271\\
566	0.00861551385022715\\
567	0.00866260046940764\\
568	0.00870986819270545\\
569	0.00875727770806398\\
570	0.00880478853955548\\
571	0.00885236623413343\\
572	0.00889976980065599\\
573	0.00894666948230801\\
574	0.00899362437400096\\
575	0.00904063163668605\\
576	0.00908765466182056\\
577	0.00913464722207074\\
578	0.00918156048899183\\
579	0.00922834328057124\\
580	0.00927494238562664\\
581	0.00932130302917242\\
582	0.00936736951667931\\
583	0.00941308610287009\\
584	0.00945839814032588\\
585	0.00950325357339547\\
586	0.00954760485097041\\
587	0.00959141132963418\\
588	0.00963464220433352\\
589	0.00967727987662474\\
590	0.00971932328838558\\
591	0.0097606871825126\\
592	0.00980121135503816\\
593	0.00984070634831441\\
594	0.0098789201112722\\
595	0.00991544794299161\\
596	0.00994949078513196\\
597	0.00997920912842892\\
598	0.0100000808076597\\
599	0\\
600	0\\
};
\addplot [color=mycolor8,solid,forget plot]
  table[row sep=crcr]{%
1	0.00409471935275828\\
2	0.00409472878468677\\
3	0.00409473837666925\\
4	0.0040947481314172\\
5	0.0040947580516879\\
6	0.00409476814028525\\
7	0.00409477840006039\\
8	0.00409478883391274\\
9	0.00409479944479059\\
10	0.00409481023569205\\
11	0.00409482120966593\\
12	0.00409483236981234\\
13	0.00409484371928392\\
14	0.00409485526128638\\
15	0.0040948669990796\\
16	0.00409487893597846\\
17	0.00409489107535382\\
18	0.00409490342063338\\
19	0.00409491597530271\\
20	0.00409492874290612\\
21	0.00409494172704778\\
22	0.00409495493139259\\
23	0.00409496835966725\\
24	0.0040949820156614\\
25	0.00409499590322852\\
26	0.004095010026287\\
27	0.00409502438882139\\
28	0.00409503899488335\\
29	0.00409505384859282\\
30	0.00409506895413915\\
31	0.00409508431578235\\
32	0.00409509993785414\\
33	0.00409511582475924\\
34	0.00409513198097654\\
35	0.00409514841106037\\
36	0.0040951651196418\\
37	0.00409518211142973\\
38	0.00409519939121252\\
39	0.00409521696385896\\
40	0.0040952348343199\\
41	0.00409525300762944\\
42	0.00409527148890642\\
43	0.00409529028335573\\
44	0.00409530939626992\\
45	0.00409532883303046\\
46	0.0040953485991093\\
47	0.00409536870007043\\
48	0.00409538914157135\\
49	0.00409540992936467\\
50	0.00409543106929963\\
51	0.00409545256732381\\
52	0.00409547442948461\\
53	0.00409549666193114\\
54	0.00409551927091564\\
55	0.00409554226279537\\
56	0.00409556564403432\\
57	0.00409558942120499\\
58	0.0040956136009901\\
59	0.00409563819018461\\
60	0.00409566319569732\\
61	0.00409568862455305\\
62	0.00409571448389429\\
63	0.00409574078098332\\
64	0.00409576752320413\\
65	0.0040957947180645\\
66	0.00409582237319792\\
67	0.00409585049636576\\
68	0.00409587909545944\\
69	0.00409590817850237\\
70	0.00409593775365239\\
71	0.00409596782920384\\
72	0.00409599841358978\\
73	0.00409602951538436\\
74	0.00409606114330513\\
75	0.00409609330621538\\
76	0.00409612601312657\\
77	0.00409615927320071\\
78	0.00409619309575288\\
79	0.00409622749025378\\
80	0.00409626246633216\\
81	0.00409629803377758\\
82	0.00409633420254293\\
83	0.00409637098274717\\
84	0.00409640838467799\\
85	0.00409644641879464\\
86	0.00409648509573066\\
87	0.00409652442629685\\
88	0.00409656442148411\\
89	0.00409660509246628\\
90	0.00409664645060328\\
91	0.00409668850744401\\
92	0.00409673127472961\\
93	0.00409677476439644\\
94	0.00409681898857924\\
95	0.00409686395961453\\
96	0.00409690969004368\\
97	0.00409695619261642\\
98	0.00409700348029408\\
99	0.00409705156625309\\
100	0.00409710046388854\\
101	0.00409715018681751\\
102	0.00409720074888283\\
103	0.00409725216415675\\
104	0.00409730444694458\\
105	0.00409735761178838\\
106	0.00409741167347091\\
107	0.00409746664701944\\
108	0.00409752254770973\\
109	0.00409757939107002\\
110	0.00409763719288495\\
111	0.00409769596919991\\
112	0.00409775573632509\\
113	0.0040978165108397\\
114	0.00409787830959634\\
115	0.00409794114972538\\
116	0.00409800504863937\\
117	0.0040980700240376\\
118	0.00409813609391052\\
119	0.00409820327654466\\
120	0.00409827159052704\\
121	0.00409834105475023\\
122	0.00409841168841701\\
123	0.00409848351104543\\
124	0.00409855654247375\\
125	0.00409863080286566\\
126	0.0040987063127152\\
127	0.00409878309285225\\
128	0.00409886116444774\\
129	0.00409894054901912\\
130	0.00409902126843576\\
131	0.00409910334492466\\
132	0.00409918680107606\\
133	0.00409927165984925\\
134	0.00409935794457836\\
135	0.00409944567897838\\
136	0.00409953488715129\\
137	0.00409962559359207\\
138	0.00409971782319514\\
139	0.00409981160126066\\
140	0.0040999069535011\\
141	0.00410000390604781\\
142	0.00410010248545788\\
143	0.00410020271872099\\
144	0.00410030463326648\\
145	0.00410040825697055\\
146	0.00410051361816364\\
147	0.00410062074563799\\
148	0.00410072966865538\\
149	0.00410084041695503\\
150	0.00410095302076176\\
151	0.00410106751079444\\
152	0.00410118391827455\\
153	0.00410130227493521\\
154	0.00410142261303031\\
155	0.0041015449653441\\
156	0.00410166936520122\\
157	0.00410179584647685\\
158	0.00410192444360767\\
159	0.00410205519160304\\
160	0.00410218812605692\\
161	0.0041023232831602\\
162	0.00410246069971393\\
163	0.00410260041314319\\
164	0.00410274246151183\\
165	0.00410288688353832\\
166	0.00410303371861237\\
167	0.00410318300681308\\
168	0.00410333478892829\\
169	0.00410348910647554\\
170	0.00410364600172464\\
171	0.00410380551772252\\
172	0.00410396769831977\\
173	0.00410413258820016\\
174	0.00410430023291266\\
175	0.00410447067890666\\
176	0.00410464397357084\\
177	0.00410482016527579\\
178	0.00410499930342132\\
179	0.0041051814384885\\
180	0.00410536662209729\\
181	0.00410555490707039\\
182	0.00410574634750346\\
183	0.00410594099884308\\
184	0.00410613891797219\\
185	0.00410634016330431\\
186	0.00410654479488623\\
187	0.00410675287450986\\
188	0.00410696446583272\\
189	0.00410717963450638\\
190	0.00410739844831133\\
191	0.00410762097729525\\
192	0.0041078472939101\\
193	0.00410807747313952\\
194	0.00410831159260139\\
195	0.00410854973259449\\
196	0.00410879197601671\\
197	0.0041090384079717\\
198	0.00410928911458718\\
199	0.00410954417984716\\
200	0.00410980367953371\\
201	0.00411006769031332\\
202	0.00411033629015738\\
203	0.00411060955836327\\
204	0.00411088757557595\\
205	0.00411117042380969\\
206	0.0041114581864703\\
207	0.00411175094837737\\
208	0.00411204879578715\\
209	0.00411235181641552\\
210	0.0041126600994613\\
211	0.00411297373563009\\
212	0.00411329281715819\\
213	0.004113617437837\\
214	0.00411394769303769\\
215	0.00411428367973639\\
216	0.00411462549653938\\
217	0.0041149732437091\\
218	0.00411532702319008\\
219	0.00411568693863562\\
220	0.00411605309543458\\
221	0.00411642560073881\\
222	0.00411680456349074\\
223	0.00411719009445159\\
224	0.00411758230622993\\
225	0.00411798131331067\\
226	0.00411838723208453\\
227	0.00411880018087801\\
228	0.00411922027998387\\
229	0.00411964765169205\\
230	0.00412008242032115\\
231	0.00412052471225049\\
232	0.00412097465595284\\
233	0.00412143238202755\\
234	0.0041218980232345\\
235	0.00412237171452861\\
236	0.00412285359309504\\
237	0.00412334379838522\\
238	0.00412384247215345\\
239	0.00412434975849449\\
240	0.00412486580388179\\
241	0.00412539075720668\\
242	0.00412592476981848\\
243	0.0041264679955654\\
244	0.00412702059083641\\
245	0.00412758271460427\\
246	0.00412815452846947\\
247	0.00412873619670496\\
248	0.00412932788630241\\
249	0.00412992976701912\\
250	0.00413054201142622\\
251	0.00413116479495797\\
252	0.00413179829596173\\
253	0.00413244269574975\\
254	0.00413309817865102\\
255	0.00413376493206481\\
256	0.00413444314651472\\
257	0.00413513301570367\\
258	0.00413583473656964\\
259	0.00413654850934181\\
260	0.00413727453759731\\
261	0.00413801302831843\\
262	0.00413876419194971\\
263	0.00413952824245522\\
264	0.0041403053973756\\
265	0.00414109587788497\\
266	0.00414189990884696\\
267	0.00414271771887046\\
268	0.00414354954036471\\
269	0.00414439560959336\\
270	0.00414525616672802\\
271	0.00414613145590104\\
272	0.00414702172525806\\
273	0.00414792722700987\\
274	0.00414884821748507\\
275	0.00414978495718285\\
276	0.00415073771082714\\
277	0.00415170674742223\\
278	0.00415269234031066\\
279	0.00415369476723346\\
280	0.00415471431039246\\
281	0.00415575125651274\\
282	0.00415680589690584\\
283	0.00415787852753292\\
284	0.00415896944906856\\
285	0.00416007896696471\\
286	0.00416120739151483\\
287	0.00416235503791834\\
288	0.00416352222634519\\
289	0.00416470928200053\\
290	0.00416591653518933\\
291	0.00416714432138106\\
292	0.00416839298127427\\
293	0.00416966286086115\\
294	0.00417095431149141\\
295	0.00417226768993629\\
296	0.00417360335845207\\
297	0.00417496168484276\\
298	0.00417634304252241\\
299	0.00417774781057671\\
300	0.00417917637382346\\
301	0.00418062912287221\\
302	0.00418210645418283\\
303	0.00418360877012239\\
304	0.00418513647902089\\
305	0.00418668999522512\\
306	0.00418826973915053\\
307	0.00418987613733134\\
308	0.0041915096224677\\
309	0.00419317063347088\\
310	0.00419485961550495\\
311	0.00419657702002572\\
312	0.00419832330481578\\
313	0.00420009893401604\\
314	0.00420190437815243\\
315	0.00420374011415832\\
316	0.00420560662539173\\
317	0.00420750440164676\\
318	0.00420943393915873\\
319	0.0042113957406028\\
320	0.00421339031508502\\
321	0.00421541817812535\\
322	0.0042174798516317\\
323	0.0042195758638645\\
324	0.00422170674939061\\
325	0.00422387304902575\\
326	0.00422607530976445\\
327	0.00422831408469617\\
328	0.00423058993290668\\
329	0.00423290341936288\\
330	0.0042352551147802\\
331	0.00423764559547025\\
332	0.00424007544316754\\
333	0.00424254524483298\\
334	0.00424505559243219\\
335	0.00424760708268622\\
336	0.00425020031679224\\
337	0.00425283590011117\\
338	0.00425551444181943\\
339	0.00425823655452119\\
340	0.00426100285381778\\
341	0.00426381395782946\\
342	0.00426667048666579\\
343	0.00426957306183915\\
344	0.00427252230561587\\
345	0.00427551884029902\\
346	0.00427856328743619\\
347	0.00428165626694441\\
348	0.0042847983961442\\
349	0.00428799028869328\\
350	0.00429123255340986\\
351	0.00429452579297385\\
352	0.00429787060249336\\
353	0.00430126756792237\\
354	0.00430471726431331\\
355	0.00430822025388707\\
356	0.00431177708390056\\
357	0.00431538828428927\\
358	0.00431905436506038\\
359	0.00432277581340774\\
360	0.00432655309051805\\
361	0.00433038662803212\\
362	0.00433427682412248\\
363	0.00433822403914201\\
364	0.00434222859079419\\
365	0.00434629074876824\\
366	0.00435041072877683\\
367	0.00435458868592505\\
368	0.00435882470733183\\
369	0.00436311880391594\\
370	0.00436747090124764\\
371	0.00437188082935738\\
372	0.00437634831138134\\
373	0.00438087295091145\\
374	0.00438545421790718\\
375	0.00439009143301453\\
376	0.00439478375012934\\
377	0.00439953013703601\\
378	0.004404329353952\\
379	0.00440917992981549\\
380	0.00441408013617291\\
381	0.00441902795855937\\
382	0.00442402106532687\\
383	0.00442905677397115\\
384	0.0044341320151554\\
385	0.00443924329484421\\
386	0.00444438665526774\\
387	0.00444955763586556\\
388	0.00445475123608534\\
389	0.00445996188260971\\
390	0.00446518340389888\\
391	0.00447040901777451\\
392	0.00447563134154134\\
393	0.00448084243491355\\
394	0.00448603389119452\\
395	0.00449119699908397\\
396	0.00449632300769492\\
397	0.00450140353616743\\
398	0.00450643117766225\\
399	0.00451140037593866\\
400	0.00451630868902807\\
401	0.00452115859969589\\
402	0.00452596010458269\\
403	0.00453073439562598\\
404	0.00453551885313431\\
405	0.00454035689009744\\
406	0.00454526845684955\\
407	0.00455025419872524\\
408	0.00455531473320052\\
409	0.00456045064733117\\
410	0.00456566249511562\\
411	0.00457095079479666\\
412	0.0045763160261193\\
413	0.00458175862756687\\
414	0.00458727899360126\\
415	0.00459287747193892\\
416	0.00459855436089785\\
417	0.00460430990685025\\
418	0.00461014430179761\\
419	0.00461605768101712\\
420	0.00462205012076526\\
421	0.00462812163614724\\
422	0.00463427217918671\\
423	0.0046405016371167\\
424	0.00464680983092394\\
425	0.00465319651421383\\
426	0.00465966137245237\\
427	0.00466620401264077\\
428	0.0046728239497285\\
429	0.00467952060480954\\
430	0.00468629330461934\\
431	0.00469314128268854\\
432	0.00470006368258842\\
433	0.00470705956357643\\
434	0.00471412790894567\\
435	0.00472126763794894\\
436	0.00472847762240303\\
437	0.00473575670902751\\
438	0.00474310374874697\\
439	0.00475051763436941\\
440	0.00475799734823404\\
441	0.00476554202156696\\
442	0.00477315100736676\\
443	0.00478082396862773\\
444	0.0047885609835087\\
445	0.00479636266854668\\
446	0.0048042303200262\\
447	0.00481216607189223\\
448	0.00482017306576668\\
449	0.00482825562419828\\
450	0.00483641941172825\\
451	0.00484467155998184\\
452	0.00485302080688608\\
453	0.00486147823558437\\
454	0.00487005615936208\\
455	0.00487876739515582\\
456	0.00488762381471935\\
457	0.0048966336960555\\
458	0.00490580159971128\\
459	0.00491513187839367\\
460	0.00492462918575827\\
461	0.00493429850552375\\
462	0.00494414518123689\\
463	0.00495417494831305\\
464	0.00496439396764243\\
465	0.00497480886036978\\
466	0.0049854267431406\\
467	0.0049962552624931\\
468	0.00500730262539752\\
469	0.00501857761765141\\
470	0.00503008958358737\\
471	0.00504184827443425\\
472	0.00505383753227559\\
473	0.00506605325744882\\
474	0.00507850377275761\\
475	0.0050911979879413\\
476	0.00510414541993206\\
477	0.00511735618942451\\
478	0.00513084092552521\\
479	0.00514461079658594\\
480	0.0051586774843999\\
481	0.00517305312483077\\
482	0.00518775023819093\\
483	0.00520278163998963\\
484	0.00521816035067217\\
485	0.00523389952008031\\
486	0.00525001240412915\\
487	0.00526651239189596\\
488	0.0052834321027905\\
489	0.00530081710719221\\
490	0.00531868851700649\\
491	0.00533706882283471\\
492	0.0053559819906831\\
493	0.00537545321414362\\
494	0.00539550919538468\\
495	0.0054161806982298\\
496	0.00543750908917038\\
497	0.00545954765374349\\
498	0.00548233395052076\\
499	0.00550590796443481\\
500	0.00553031220596453\\
501	0.00555559141797218\\
502	0.00558179342641423\\
503	0.00560896712484049\\
504	0.00563716249952684\\
505	0.00566643200206524\\
506	0.00569683059922151\\
507	0.00572841577490032\\
508	0.00576124748146775\\
509	0.0057953880176723\\
510	0.00583090183187388\\
511	0.0058678552333366\\
512	0.00590631598795914\\
513	0.00594635276825647\\
514	0.00598803438977741\\
515	0.0060314287707132\\
516	0.00607660171035829\\
517	0.00612361535919635\\
518	0.00617252613545165\\
519	0.00622338212112285\\
520	0.00627622046018902\\
521	0.00633106335133495\\
522	0.00638791238341639\\
523	0.00644673152035281\\
524	0.00650744836701159\\
525	0.00656994687750357\\
526	0.00663405958604309\\
527	0.00669956301449569\\
528	0.0067662481716475\\
529	0.00683375718966924\\
530	0.00690150666611076\\
531	0.00696854218636775\\
532	0.00703099097322017\\
533	0.00708812750767527\\
534	0.00713975470076457\\
535	0.00718612445233945\\
536	0.0072299610167145\\
537	0.007272735809509\\
538	0.00731486961395786\\
539	0.00735675020593054\\
540	0.00739853082596234\\
541	0.0074404109759587\\
542	0.00748248026742258\\
543	0.00752480149480098\\
544	0.00756764572830246\\
545	0.00761122002134492\\
546	0.00765556875627886\\
547	0.00770073439979809\\
548	0.00774676363951815\\
549	0.00779372454967009\\
550	0.0078416987098562\\
551	0.00789078420299926\\
552	0.00793965564440568\\
553	0.00798755833548598\\
554	0.00803421360422451\\
555	0.00808071726034567\\
556	0.0081276029697395\\
557	0.00817492543105886\\
558	0.00822274721009335\\
559	0.00827107775662958\\
560	0.00831988857565171\\
561	0.00836914077122186\\
562	0.00841880697396669\\
563	0.00846886895839027\\
564	0.00851795908851319\\
565	0.00856547324180071\\
566	0.0086127065303714\\
567	0.00866005127945744\\
568	0.00870755946891178\\
569	0.00875521699430515\\
570	0.00880298409805163\\
571	0.00885081712473358\\
572	0.00889867606777689\\
573	0.00894645244108499\\
574	0.00899360235849341\\
575	0.00904063091028688\\
576	0.00908765454800972\\
577	0.00913464716853555\\
578	0.00918156046219706\\
579	0.00922834326765819\\
580	0.00927494237979902\\
581	0.00932130302675409\\
582	0.00936736951577526\\
583	0.00941308610257392\\
584	0.0094583981402442\\
585	0.00950325357337789\\
586	0.00954760485096767\\
587	0.00959141132963419\\
588	0.00963464220433353\\
589	0.00967727987662474\\
590	0.00971932328838559\\
591	0.0097606871825126\\
592	0.00980121135503816\\
593	0.00984070634831441\\
594	0.0098789201112722\\
595	0.00991544794299161\\
596	0.00994949078513196\\
597	0.00997920912842892\\
598	0.0100000808076597\\
599	0\\
600	0\\
};
\addplot [color=blue!25!mycolor7,solid,forget plot]
  table[row sep=crcr]{%
1	0.00349563927144954\\
2	0.00349565547998435\\
3	0.00349567196221764\\
4	0.00349568872275953\\
5	0.00349570576629746\\
6	0.00349572309759751\\
7	0.00349574072150579\\
8	0.00349575864294962\\
9	0.00349577686693903\\
10	0.0034957953985681\\
11	0.00349581424301621\\
12	0.00349583340554973\\
13	0.00349585289152322\\
14	0.00349587270638111\\
15	0.00349589285565901\\
16	0.00349591334498536\\
17	0.00349593418008288\\
18	0.00349595536677018\\
19	0.00349597691096339\\
20	0.00349599881867775\\
21	0.0034960210960292\\
22	0.00349604374923619\\
23	0.00349606678462125\\
24	0.00349609020861278\\
25	0.00349611402774678\\
26	0.00349613824866876\\
27	0.00349616287813527\\
28	0.0034961879230161\\
29	0.00349621339029588\\
30	0.00349623928707616\\
31	0.00349626562057723\\
32	0.00349629239814016\\
33	0.00349631962722875\\
34	0.00349634731543166\\
35	0.00349637547046435\\
36	0.00349640410017123\\
37	0.00349643321252792\\
38	0.00349646281564313\\
39	0.00349649291776114\\
40	0.00349652352726391\\
41	0.00349655465267344\\
42	0.00349658630265391\\
43	0.00349661848601429\\
44	0.00349665121171045\\
45	0.00349668448884779\\
46	0.00349671832668368\\
47	0.00349675273462988\\
48	0.00349678772225508\\
49	0.0034968232992876\\
50	0.00349685947561796\\
51	0.00349689626130152\\
52	0.00349693366656123\\
53	0.00349697170179036\\
54	0.00349701037755537\\
55	0.00349704970459867\\
56	0.00349708969384153\\
57	0.00349713035638701\\
58	0.0034971717035231\\
59	0.00349721374672547\\
60	0.00349725649766081\\
61	0.00349729996818975\\
62	0.0034973441703702\\
63	0.00349738911646052\\
64	0.0034974348189228\\
65	0.00349748129042613\\
66	0.00349752854385014\\
67	0.00349757659228825\\
68	0.00349762544905124\\
69	0.00349767512767098\\
70	0.00349772564190365\\
71	0.00349777700573373\\
72	0.00349782923337755\\
73	0.00349788233928716\\
74	0.00349793633815407\\
75	0.00349799124491316\\
76	0.0034980470747467\\
77	0.00349810384308833\\
78	0.0034981615656271\\
79	0.00349822025831163\\
80	0.00349827993735429\\
81	0.00349834061923561\\
82	0.00349840232070838\\
83	0.00349846505880218\\
84	0.00349852885082786\\
85	0.00349859371438204\\
86	0.0034986596673517\\
87	0.00349872672791888\\
88	0.0034987949145654\\
89	0.0034988642460777\\
90	0.00349893474155166\\
91	0.0034990064203977\\
92	0.00349907930234561\\
93	0.00349915340744982\\
94	0.00349922875609458\\
95	0.00349930536899912\\
96	0.00349938326722303\\
97	0.00349946247217166\\
98	0.00349954300560172\\
99	0.00349962488962672\\
100	0.00349970814672267\\
101	0.00349979279973383\\
102	0.00349987887187852\\
103	0.003499966386755\\
104	0.00350005536834753\\
105	0.00350014584103233\\
106	0.00350023782958374\\
107	0.00350033135918062\\
108	0.00350042645541246\\
109	0.00350052314428592\\
110	0.00350062145223133\\
111	0.00350072140610915\\
112	0.00350082303321681\\
113	0.00350092636129534\\
114	0.00350103141853637\\
115	0.00350113823358889\\
116	0.0035012468355664\\
117	0.00350135725405394\\
118	0.00350146951911545\\
119	0.00350158366130088\\
120	0.00350169971165374\\
121	0.00350181770171838\\
122	0.00350193766354775\\
123	0.00350205962971095\\
124	0.00350218363330097\\
125	0.00350230970794245\\
126	0.00350243788779974\\
127	0.00350256820758478\\
128	0.00350270070256509\\
129	0.00350283540857207\\
130	0.00350297236200913\\
131	0.00350311159986002\\
132	0.00350325315969716\\
133	0.00350339707969012\\
134	0.00350354339861404\\
135	0.0035036921558583\\
136	0.00350384339143496\\
137	0.00350399714598755\\
138	0.00350415346079976\\
139	0.00350431237780408\\
140	0.00350447393959065\\
141	0.00350463818941595\\
142	0.00350480517121164\\
143	0.00350497492959331\\
144	0.00350514750986919\\
145	0.00350532295804896\\
146	0.00350550132085233\\
147	0.00350568264571771\\
148	0.00350586698081067\\
149	0.00350605437503237\\
150	0.00350624487802787\\
151	0.00350643854019418\\
152	0.00350663541268818\\
153	0.00350683554743428\\
154	0.00350703899713187\\
155	0.00350724581526232\\
156	0.00350745605609576\\
157	0.00350766977469741\\
158	0.00350788702693329\\
159	0.0035081078694756\\
160	0.00350833235980722\\
161	0.00350856055622575\\
162	0.00350879251784652\\
163	0.00350902830460487\\
164	0.00350926797725726\\
165	0.00350951159738127\\
166	0.00350975922737441\\
167	0.00351001093045134\\
168	0.00351026677063961\\
169	0.00351052681277343\\
170	0.00351079112248589\\
171	0.00351105976619845\\
172	0.00351133281110865\\
173	0.00351161032517484\\
174	0.00351189237709844\\
175	0.00351217903630317\\
176	0.00351247037291119\\
177	0.00351276645771617\\
178	0.00351306736215318\\
179	0.00351337315826539\\
180	0.00351368391866816\\
181	0.00351399971651047\\
182	0.00351432062543523\\
183	0.00351464671953888\\
184	0.00351497807333264\\
185	0.00351531476170784\\
186	0.00351565685990902\\
187	0.00351600444352006\\
188	0.003516357588471\\
189	0.00351671637107748\\
190	0.00351708086813148\\
191	0.00351745115707903\\
192	0.00351782731635824\\
193	0.00351820942606943\\
194	0.00351859756941105\\
195	0.00351899183603133\\
196	0.0035193923304382\\
197	0.00351979919421016\\
198	0.00352021266665084\\
199	0.00352063325411669\\
200	0.0035210610951658\\
201	0.00352149631409005\\
202	0.00352193903728606\\
203	0.00352238939329015\\
204	0.00352284751281387\\
205	0.0035233135287801\\
206	0.00352378757635968\\
207	0.00352426979300869\\
208	0.00352476031850649\\
209	0.00352525929499407\\
210	0.00352576686701342\\
211	0.00352628318154713\\
212	0.00352680838805907\\
213	0.00352734263853542\\
214	0.00352788608752657\\
215	0.00352843889218958\\
216	0.00352900121233164\\
217	0.0035295732104538\\
218	0.00353015505179591\\
219	0.00353074690438202\\
220	0.00353134893906667\\
221	0.00353196132958199\\
222	0.00353258425258548\\
223	0.00353321788770879\\
224	0.00353386241760715\\
225	0.00353451802800986\\
226	0.00353518490777153\\
227	0.00353586324892411\\
228	0.00353655324673017\\
229	0.00353725509973659\\
230	0.00353796900982981\\
231	0.00353869518229152\\
232	0.00353943382585552\\
233	0.00354018515276569\\
234	0.00354094937883477\\
235	0.0035417267235043\\
236	0.00354251740990557\\
237	0.00354332166492161\\
238	0.00354413971925028\\
239	0.00354497180746848\\
240	0.00354581816809728\\
241	0.00354667904366845\\
242	0.00354755468079185\\
243	0.0035484453302241\\
244	0.00354935124693833\\
245	0.00355027269019499\\
246	0.00355120992361399\\
247	0.0035521632152478\\
248	0.00355313283765573\\
249	0.00355411906797952\\
250	0.00355512218801971\\
251	0.00355614248431342\\
252	0.00355718024821324\\
253	0.00355823577596696\\
254	0.00355930936879878\\
255	0.00356040133299125\\
256	0.00356151197996847\\
257	0.00356264162638036\\
258	0.00356379059418788\\
259	0.00356495921074953\\
260	0.00356614780890854\\
261	0.00356735672708154\\
262	0.00356858630934814\\
263	0.00356983690554148\\
264	0.00357110887134016\\
265	0.00357240256836121\\
266	0.00357371836425438\\
267	0.00357505663279772\\
268	0.00357641775399445\\
269	0.00357780211417136\\
270	0.00357921010607877\\
271	0.00358064212899207\\
272	0.0035820985888149\\
273	0.00358357989818425\\
274	0.0035850864765772\\
275	0.0035866187504197\\
276	0.00358817715319691\\
277	0.00358976212556588\\
278	0.00359137411546953\\
279	0.00359301357825256\\
280	0.0035946809767791\\
281	0.00359637678155212\\
282	0.00359810147083433\\
283	0.0035998555307713\\
284	0.00360163945551576\\
285	0.00360345374735437\\
286	0.00360529891683582\\
287	0.00360717548290107\\
288	0.00360908397301524\\
289	0.00361102492330163\\
290	0.0036129988786776\\
291	0.00361500639299238\\
292	0.00361704802916678\\
293	0.00361912435933504\\
294	0.00362123596498875\\
295	0.00362338343712262\\
296	0.00362556737638248\\
297	0.00362778839321547\\
298	0.00363004710802223\\
299	0.00363234415131138\\
300	0.0036346801638561\\
301	0.00363705579685316\\
302	0.00363947171208398\\
303	0.00364192858207814\\
304	0.00364442709027934\\
305	0.00364696793121358\\
306	0.00364955181065987\\
307	0.0036521794458234\\
308	0.00365485156551134\\
309	0.00365756891031102\\
310	0.00366033223277086\\
311	0.00366314229758401\\
312	0.0036659998817747\\
313	0.00366890577488722\\
314	0.00367186077917815\\
315	0.00367486570981106\\
316	0.00367792139505445\\
317	0.00368102867648278\\
318	0.00368418840918048\\
319	0.0036874014619493\\
320	0.00369066871751893\\
321	0.00369399107276098\\
322	0.00369736943890641\\
323	0.00370080474176657\\
324	0.00370429792195788\\
325	0.00370784993513027\\
326	0.00371146175219946\\
327	0.00371513435958317\\
328	0.00371886875944146\\
329	0.00372266596992104\\
330	0.003726527025404\\
331	0.00373045297676066\\
332	0.00373444489160691\\
333	0.00373850385456608\\
334	0.00374263096753489\\
335	0.00374682734995445\\
336	0.00375109413908509\\
337	0.00375543249028615\\
338	0.00375984357729957\\
339	0.00376432859253797\\
340	0.00376888874737615\\
341	0.00377352527244648\\
342	0.00377823941793688\\
343	0.00378303245389137\\
344	0.0037879056705122\\
345	0.00379286037846226\\
346	0.00379789790916681\\
347	0.00380301961511276\\
348	0.00380822687014362\\
349	0.00381352106974729\\
350	0.00381890363133408\\
351	0.00382437599450077\\
352	0.00382993962127663\\
353	0.00383559599634538\\
354	0.00384134662723656\\
355	0.00384719304447812\\
356	0.00385313680170011\\
357	0.00385917947567761\\
358	0.00386532266629801\\
359	0.00387156799643523\\
360	0.00387791711170917\\
361	0.00388437168010484\\
362	0.00389093339141917\\
363	0.00389760395649801\\
364	0.00390438510621653\\
365	0.00391127859014746\\
366	0.00391828617484823\\
367	0.00392540964168451\\
368	0.00393265078408875\\
369	0.00394001140412966\\
370	0.00394749330824154\\
371	0.00395509830192824\\
372	0.00396282818321325\\
373	0.00397068473455649\\
374	0.00397866971289117\\
375	0.00398678483735379\\
376	0.00399503177417705\\
377	0.0040034121180877\\
378	0.00401192736938939\\
379	0.00402057890570682\\
380	0.0040293679471097\\
381	0.00403829551300768\\
382	0.00404736236879247\\
383	0.00405656895967518\\
384	0.00406591532849624\\
385	0.00407540101344095\\
386	0.00408502492055559\\
387	0.00409478516476902\\
388	0.0041046788720431\\
389	0.00411470193527259\\
390	0.00412484872069535\\
391	0.00413511167136045\\
392	0.0041454807745694\\
393	0.00415594290794228\\
394	0.00416648100448392\\
395	0.0041770729654529\\
396	0.00418769021381985\\
397	0.00419829590891061\\
398	0.00420884281903749\\
399	0.0042192705393636\\
400	0.00422950181656293\\
401	0.00423943782636934\\
402	0.0042489521856744\\
403	0.00425788371573713\\
404	0.00426602917371545\\
405	0.0042736045653037\\
406	0.00428130134730261\\
407	0.00428912095057911\\
408	0.00429706477728599\\
409	0.00430513419565023\\
410	0.00431333053417412\\
411	0.0043216550751678\\
412	0.00433010904751633\\
413	0.00433869361856912\\
414	0.00434740988502378\\
415	0.00435625886267023\\
416	0.0043652414748839\\
417	0.00437435853986565\\
418	0.00438361075695377\\
419	0.00439299869317194\\
420	0.00440252276766291\\
421	0.00441218323219831\\
422	0.00442198014899021\\
423	0.00443191336573593\\
424	0.00444198248756836\\
425	0.00445218684554136\\
426	0.00446252546119181\\
427	0.0044729970068166\\
428	0.00448359976022577\\
429	0.00449433155293298\\
430	0.0045051897134399\\
431	0.00451617100583658\\
432	0.00452727156381899\\
433	0.00453848682438379\\
434	0.00454981146155924\\
435	0.0045612393093507\\
436	0.00457276327304812\\
437	0.0045843752349012\\
438	0.00459606595539919\\
439	0.00460782497223779\\
440	0.00461964050031618\\
441	0.00463149933806424\\
442	0.00464338678788077\\
443	0.00465528660169599\\
444	0.00466718096744677\\
445	0.00467905055891257\\
446	0.00469087468044236\\
447	0.00470263155058857\\
448	0.00471429878582557\\
449	0.00472585416914982\\
450	0.00473727682103648\\
451	0.00474854893592216\\
452	0.00475965830978701\\
453	0.00477060195962984\\
454	0.00478139135869376\\
455	0.00479205994723771\\
456	0.00480267377809916\\
457	0.00481334536376296\\
458	0.00482415164552343\\
459	0.00483510125511467\\
460	0.00484619399537625\\
461	0.00485742953046601\\
462	0.00486880740291954\\
463	0.00488032704387104\\
464	0.00489198778937994\\
465	0.00490378890447866\\
466	0.0049157296168588\\
467	0.00492780916242221\\
468	0.00494002684519871\\
469	0.00495238211442667\\
470	0.00496487466264251\\
471	0.00497750455511809\\
472	0.00499027322728633\\
473	0.005003183335035\\
474	0.00501623861267366\\
475	0.00502944412809434\\
476	0.00504280657258239\\
477	0.00505633458218204\\
478	0.0050700390839554\\
479	0.00508393364522376\\
480	0.00509803479709843\\
481	0.00511236228395829\\
482	0.00512693916203388\\
483	0.00514179162707186\\
484	0.00515694837326728\\
485	0.00517243912966033\\
486	0.00518829173213742\\
487	0.00520452791498555\\
488	0.00522114567267705\\
489	0.00523813258298007\\
490	0.00525550374885946\\
491	0.00527327535415479\\
492	0.00529146474369282\\
493	0.005310090529109\\
494	0.00532917272361812\\
495	0.0053487328337719\\
496	0.00536879364672981\\
497	0.00538937886351541\\
498	0.00541051464457971\\
499	0.00543223226322332\\
500	0.00545458659856475\\
501	0.00547761741063954\\
502	0.00550137021060624\\
503	0.00552592930741987\\
504	0.00555134068393207\\
505	0.00557765205289851\\
506	0.00560491245042303\\
507	0.00563317299283425\\
508	0.00566248720840161\\
509	0.00569291151305632\\
510	0.00572450529282427\\
511	0.00575733093224414\\
512	0.00579145384885994\\
513	0.0058269425244145\\
514	0.0058638688003334\\
515	0.00590230761898969\\
516	0.00594233422308555\\
517	0.00598402372049495\\
518	0.00602745171959321\\
519	0.00607269336021927\\
520	0.00611982201581636\\
521	0.00616890762974224\\
522	0.00622001464170937\\
523	0.00627320020832274\\
524	0.00632851071480576\\
525	0.00638597569546699\\
526	0.00644559491922356\\
527	0.00650734289043639\\
528	0.00657115894766724\\
529	0.00663694019598189\\
530	0.00670453404460053\\
531	0.00677373235014712\\
532	0.00684433767728545\\
533	0.00691602079828487\\
534	0.00698823853487104\\
535	0.00706009960762352\\
536	0.00712840852789063\\
537	0.00719151951789386\\
538	0.00724913361302738\\
539	0.00730133729111109\\
540	0.00734952052557149\\
541	0.00739643380499645\\
542	0.00744244729284901\\
543	0.00748802858840727\\
544	0.00753346656780794\\
545	0.00757890809413066\\
546	0.0076244867254931\\
547	0.00767028894680288\\
548	0.00771636958533522\\
549	0.00776310894134749\\
550	0.00781061456134046\\
551	0.00785893209079425\\
552	0.00790814535923132\\
553	0.00795835766102141\\
554	0.00800967648269705\\
555	0.00806077473448577\\
556	0.00811090637964637\\
557	0.00815979788770441\\
558	0.00820827888320951\\
559	0.00825708186260537\\
560	0.00830624124096328\\
561	0.00835581609740137\\
562	0.00840582623571061\\
563	0.00845623199085341\\
564	0.00850701217799831\\
565	0.00855814262702071\\
566	0.00860816055671462\\
567	0.008656578645634\\
568	0.00870454416643886\\
569	0.00875252871649243\\
570	0.00880058482301224\\
571	0.00884870958565739\\
572	0.00889686035518174\\
573	0.00894499072239392\\
574	0.00899306418267662\\
575	0.00904057390228334\\
576	0.00908764684168784\\
577	0.00913464636830236\\
578	0.00918156011513437\\
579	0.009228343091509\\
580	0.00927494229204303\\
581	0.00932130298557355\\
582	0.009367369497945\\
583	0.00941308609559833\\
584	0.0094583981378461\\
585	0.00950325357268192\\
586	0.00954760485080776\\
587	0.00959141132960813\\
588	0.00963464220433133\\
589	0.00967727987662474\\
590	0.00971932328838558\\
591	0.0097606871825126\\
592	0.00980121135503816\\
593	0.00984070634831441\\
594	0.00987892011127219\\
595	0.00991544794299161\\
596	0.00994949078513196\\
597	0.00997920912842892\\
598	0.0100000808076597\\
599	0\\
600	0\\
};
\addplot [color=mycolor9,solid,forget plot]
  table[row sep=crcr]{%
1	0.00316228443168399\\
2	0.00316229200294084\\
3	0.00316229970203089\\
4	0.00316230753110517\\
5	0.00316231549235073\\
6	0.00316232358799119\\
7	0.00316233182028728\\
8	0.00316234019153764\\
9	0.00316234870407927\\
10	0.00316235736028819\\
11	0.00316236616258024\\
12	0.00316237511341151\\
13	0.00316238421527917\\
14	0.0031623934707221\\
15	0.00316240288232152\\
16	0.00316241245270175\\
17	0.00316242218453095\\
18	0.00316243208052176\\
19	0.00316244214343205\\
20	0.00316245237606571\\
21	0.00316246278127343\\
22	0.00316247336195334\\
23	0.00316248412105193\\
24	0.00316249506156472\\
25	0.00316250618653721\\
26	0.00316251749906546\\
27	0.00316252900229724\\
28	0.00316254069943267\\
29	0.00316255259372505\\
30	0.00316256468848187\\
31	0.00316257698706557\\
32	0.0031625894928945\\
33	0.00316260220944391\\
34	0.00316261514024662\\
35	0.0031626282888943\\
36	0.00316264165903815\\
37	0.00316265525438996\\
38	0.00316266907872321\\
39	0.00316268313587398\\
40	0.00316269742974187\\
41	0.00316271196429115\\
42	0.00316272674355183\\
43	0.00316274177162062\\
44	0.00316275705266212\\
45	0.00316277259090989\\
46	0.00316278839066738\\
47	0.0031628044563094\\
48	0.00316282079228303\\
49	0.00316283740310887\\
50	0.00316285429338214\\
51	0.00316287146777396\\
52	0.00316288893103263\\
53	0.00316290668798469\\
54	0.00316292474353635\\
55	0.00316294310267462\\
56	0.00316296177046877\\
57	0.0031629807520715\\
58	0.00316300005272029\\
59	0.00316301967773889\\
60	0.0031630396325385\\
61	0.00316305992261928\\
62	0.00316308055357176\\
63	0.00316310153107822\\
64	0.00316312286091418\\
65	0.00316314454894993\\
66	0.00316316660115191\\
67	0.00316318902358435\\
68	0.00316321182241071\\
69	0.00316323500389523\\
70	0.00316325857440474\\
71	0.00316328254040994\\
72	0.00316330690848724\\
73	0.00316333168532038\\
74	0.00316335687770204\\
75	0.00316338249253562\\
76	0.00316340853683696\\
77	0.00316343501773597\\
78	0.00316346194247855\\
79	0.00316348931842827\\
80	0.00316351715306836\\
81	0.0031635454540032\\
82	0.00316357422896061\\
83	0.00316360348579345\\
84	0.00316363323248171\\
85	0.00316366347713427\\
86	0.00316369422799107\\
87	0.0031637254934249\\
88	0.00316375728194357\\
89	0.00316378960219192\\
90	0.00316382246295381\\
91	0.00316385587315428\\
92	0.00316388984186171\\
93	0.00316392437828981\\
94	0.00316395949179997\\
95	0.00316399519190328\\
96	0.00316403148826296\\
97	0.00316406839069641\\
98	0.00316410590917763\\
99	0.00316414405383934\\
100	0.00316418283497549\\
101	0.00316422226304351\\
102	0.00316426234866671\\
103	0.00316430310263666\\
104	0.00316434453591559\\
105	0.00316438665963895\\
106	0.00316442948511778\\
107	0.0031644730238412\\
108	0.00316451728747901\\
109	0.00316456228788414\\
110	0.00316460803709537\\
111	0.00316465454733971\\
112	0.00316470183103519\\
113	0.00316474990079352\\
114	0.00316479876942247\\
115	0.00316484844992894\\
116	0.0031648989555214\\
117	0.00316495029961274\\
118	0.00316500249582287\\
119	0.00316505555798166\\
120	0.00316510950013162\\
121	0.00316516433653079\\
122	0.00316522008165543\\
123	0.00316527675020296\\
124	0.0031653343570948\\
125	0.00316539291747926\\
126	0.00316545244673437\\
127	0.00316551296047087\\
128	0.00316557447453505\\
129	0.00316563700501179\\
130	0.00316570056822747\\
131	0.00316576518075287\\
132	0.00316583085940625\\
133	0.00316589762125633\\
134	0.00316596548362534\\
135	0.00316603446409201\\
136	0.00316610458049465\\
137	0.0031661758509342\\
138	0.00316624829377739\\
139	0.00316632192765992\\
140	0.00316639677148939\\
141	0.00316647284444871\\
142	0.00316655016599934\\
143	0.00316662875588438\\
144	0.00316670863413204\\
145	0.00316678982105902\\
146	0.00316687233727388\\
147	0.00316695620368061\\
148	0.00316704144148246\\
149	0.00316712807218526\\
150	0.00316721611760177\\
151	0.00316730559985536\\
152	0.00316739654138449\\
153	0.00316748896494697\\
154	0.00316758289362479\\
155	0.00316767835082899\\
156	0.00316777536030507\\
157	0.0031678739461387\\
158	0.00316797413276197\\
159	0.00316807594495995\\
160	0.00316817940787834\\
161	0.00316828454703137\\
162	0.00316839138831089\\
163	0.00316849995799623\\
164	0.00316861028276539\\
165	0.00316872238970732\\
166	0.00316883630633591\\
167	0.00316895206060557\\
168	0.00316906968092904\\
169	0.00316918919619718\\
170	0.00316931063580162\\
171	0.00316943402966036\\
172	0.00316955940824674\\
173	0.0031696868026224\\
174	0.00316981624447469\\
175	0.00316994776615891\\
176	0.00317008140074646\\
177	0.00317021718207924\\
178	0.003170355144831\\
179	0.00317049532457702\\
180	0.00317063775787219\\
181	0.00317078248233898\\
182	0.00317092953676565\\
183	0.00317107896121588\\
184	0.00317123079715056\\
185	0.00317138508756152\\
186	0.00317154187711821\\
187	0.00317170121232653\\
188	0.00317186314169884\\
189	0.00317202771593353\\
190	0.00317219498810152\\
191	0.00317236501383646\\
192	0.00317253785152635\\
193	0.00317271356250779\\
194	0.00317289221127217\\
195	0.0031730738656972\\
196	0.00317325859725482\\
197	0.00317344648070327\\
198	0.00317363759059267\\
199	0.00317383198255984\\
200	0.0031740297123048\\
201	0.00317423083645245\\
202	0.00317443541256746\\
203	0.00317464349916937\\
204	0.00317485515574768\\
205	0.00317507044277759\\
206	0.00317528942173578\\
207	0.00317551215511626\\
208	0.00317573870644687\\
209	0.00317596914030567\\
210	0.00317620352233772\\
211	0.00317644191927232\\
212	0.00317668439894007\\
213	0.0031769310302907\\
214	0.00317718188341081\\
215	0.00317743702954215\\
216	0.00317769654109991\\
217	0.00317796049169161\\
218	0.00317822895613614\\
219	0.00317850201048296\\
220	0.00317877973203187\\
221	0.00317906219935289\\
222	0.0031793494923066\\
223	0.00317964169206465\\
224	0.00317993888113084\\
225	0.00318024114336209\\
226	0.00318054856399025\\
227	0.00318086122964393\\
228	0.00318117922837065\\
229	0.00318150264965972\\
230	0.00318183158446477\\
231	0.00318216612522733\\
232	0.00318250636590043\\
233	0.00318285240197243\\
234	0.00318320433049157\\
235	0.00318356225009048\\
236	0.00318392626101144\\
237	0.00318429646513165\\
238	0.00318467296598903\\
239	0.00318505586880834\\
240	0.0031854452805278\\
241	0.00318584130982578\\
242	0.00318624406714805\\
243	0.00318665366473544\\
244	0.00318707021665162\\
245	0.00318749383881147\\
246	0.00318792464900968\\
247	0.00318836276694966\\
248	0.00318880831427293\\
249	0.00318926141458871\\
250	0.00318972219350399\\
251	0.0031901907786539\\
252	0.00319066729973239\\
253	0.00319115188852322\\
254	0.00319164467893146\\
255	0.00319214580701523\\
256	0.00319265541101787\\
257	0.00319317363140043\\
258	0.00319370061087445\\
259	0.00319423649443545\\
260	0.00319478142939652\\
261	0.00319533556542244\\
262	0.00319589905456423\\
263	0.00319647205129427\\
264	0.00319705471254176\\
265	0.00319764719772868\\
266	0.00319824966880645\\
267	0.00319886229029282\\
268	0.00319948522930965\\
269	0.00320011865562105\\
270	0.00320076274167214\\
271	0.00320141766262854\\
272	0.00320208359641615\\
273	0.00320276072376195\\
274	0.00320344922823516\\
275	0.00320414929628888\\
276	0.00320486111730288\\
277	0.00320558488362641\\
278	0.00320632079062198\\
279	0.00320706903670974\\
280	0.00320782982341259\\
281	0.00320860335540177\\
282	0.00320938984054336\\
283	0.00321018948994544\\
284	0.00321100251800603\\
285	0.00321182914246171\\
286	0.00321266958443709\\
287	0.00321352406849531\\
288	0.00321439282268908\\
289	0.00321527607861278\\
290	0.00321617407145557\\
291	0.0032170870400552\\
292	0.00321801522695329\\
293	0.0032189588784511\\
294	0.0032199182446668\\
295	0.0032208935795939\\
296	0.00322188514116064\\
297	0.00322289319129078\\
298	0.00322391799596588\\
299	0.00322495982528875\\
300	0.00322601895354852\\
301	0.0032270956592872\\
302	0.00322819022536797\\
303	0.00322930293904526\\
304	0.00323043409203645\\
305	0.00323158398059595\\
306	0.00323275290559103\\
307	0.00323394117257986\\
308	0.0032351490918924\\
309	0.00323637697871318\\
310	0.00323762515316721\\
311	0.00323889394040869\\
312	0.00324018367071274\\
313	0.00324149467957045\\
314	0.00324282730778755\\
315	0.00324418190158688\\
316	0.0032455588127148\\
317	0.00324695839855222\\
318	0.00324838102223021\\
319	0.00324982705275063\\
320	0.0032512968651125\\
321	0.00325279084044405\\
322	0.00325430936614137\\
323	0.00325585283601369\\
324	0.00325742165043639\\
325	0.00325901621651211\\
326	0.00326063694824035\\
327	0.00326228426669691\\
328	0.00326395860022338\\
329	0.00326566038462815\\
330	0.00326739006339922\\
331	0.00326914808793074\\
332	0.00327093491776367\\
333	0.00327275102084231\\
334	0.00327459687378825\\
335	0.0032764729621929\\
336	0.00327837978093072\\
337	0.0032803178344951\\
338	0.00328228763735894\\
339	0.00328428971436246\\
340	0.00328632460113084\\
341	0.00328839284452481\\
342	0.00329049500312757\\
343	0.00329263164777183\\
344	0.00329480336211095\\
345	0.00329701074323947\\
346	0.00329925440236782\\
347	0.00330153496555764\\
348	0.00330385307452439\\
349	0.00330620938751494\\
350	0.0033086045802693\\
351	0.00331103934707604\\
352	0.00331351440193286\\
353	0.00331603047982579\\
354	0.00331858833814152\\
355	0.00332118875822997\\
356	0.00332383254713728\\
357	0.00332652053953149\\
358	0.00332925359984762\\
359	0.00333203262468218\\
360	0.00333485854547329\\
361	0.00333773233150688\\
362	0.003340654993298\\
363	0.0033436275864031\\
364	0.00334665121573\\
365	0.00334972704042368\\
366	0.00335285627941992\\
367	0.003356040217776\\
368	0.00335928021390809\\
369	0.00336257770788967\\
370	0.00336593423099507\\
371	0.0033693514167085\\
372	0.00337283101346417\\
373	0.0033763748994354\\
374	0.0033799850997596\\
375	0.00338366380666557\\
376	0.00338741340307186\\
377	0.00339123649034924\\
378	0.00339513592109555\\
379	0.00339911483796374\\
380	0.00340317671982417\\
381	0.00340732543684008\\
382	0.00341156531640097\\
383	0.0034159012222927\\
384	0.00342033864994253\\
385	0.00342488384087433\\
386	0.00342954391895145\\
387	0.00343432704719503\\
388	0.00343924258888693\\
389	0.00344430120300294\\
390	0.0034495146187669\\
391	0.00345489768571548\\
392	0.00346046825912343\\
393	0.00346624773310527\\
394	0.00347226209368958\\
395	0.00347854365538854\\
396	0.00348513441637046\\
397	0.00349208288299282\\
398	0.00349944454005206\\
399	0.00350728729438178\\
400	0.00351569465965612\\
401	0.00352476978910882\\
402	0.00353464048990635\\
403	0.00354546509359081\\
404	0.00355743793467131\\
405	0.00357032627391072\\
406	0.00358342573374439\\
407	0.00359673971423877\\
408	0.00361027166355834\\
409	0.00362402507780284\\
410	0.00363800350082889\\
411	0.00365221052406864\\
412	0.00366664978634671\\
413	0.00368132497365631\\
414	0.00369623981875223\\
415	0.00371139810014581\\
416	0.00372680363940678\\
417	0.00374246029397659\\
418	0.0037583719384848\\
419	0.00377454241723286\\
420	0.00379097560198354\\
421	0.00380767539652213\\
422	0.0038246457305789\\
423	0.00384189055204371\\
424	0.00385941381717172\\
425	0.00387721947868306\\
426	0.00389531147250709\\
427	0.00391369370680589\\
428	0.00393237006597947\\
429	0.00395134439625981\\
430	0.00397062042918411\\
431	0.0039902016826495\\
432	0.00401009133098071\\
433	0.00403029195236743\\
434	0.00405080531606116\\
435	0.00407163238801865\\
436	0.00409277327635466\\
437	0.00411422701475109\\
438	0.00413599129269844\\
439	0.00415806211883574\\
440	0.00418043339863117\\
441	0.0042030963981149\\
442	0.0042260390693079\\
443	0.00424924520190304\\
444	0.00427269334975965\\
445	0.00429635546903201\\
446	0.00432019518610211\\
447	0.0043441655890934\\
448	0.00436820640477886\\
449	0.00439224038077202\\
450	0.0044161686379577\\
451	0.00443986468666294\\
452	0.00446316670968053\\
453	0.00448586761183722\\
454	0.00450770225711265\\
455	0.00452833145356001\\
456	0.00454732337643396\\
457	0.00456413816247552\\
458	0.00458087557685243\\
459	0.00459783464120776\\
460	0.00461501254915988\\
461	0.00463240535450826\\
462	0.00465000767998691\\
463	0.00466781304201693\\
464	0.00468581372057716\\
465	0.00470400061731553\\
466	0.0047223631021804\\
467	0.00474088884952668\\
468	0.00475956366572662\\
469	0.004778371311847\\
470	0.00479729332711133\\
471	0.00481630886151368\\
472	0.00483539450436513\\
473	0.00485452416713792\\
474	0.00487366905045886\\
475	0.00489279772441496\\
476	0.00491187638041358\\
477	0.00493086933698705\\
478	0.00494973991622638\\
479	0.00496845184242413\\
480	0.0049869713756926\\
481	0.00500527048746449\\
482	0.00502333150421946\\
483	0.00504115383351491\\
484	0.00505876366600892\\
485	0.00507622791296076\\
486	0.00509367369240875\\
487	0.00511128741718415\\
488	0.00512917468561282\\
489	0.00514734107376625\\
490	0.00516579290226959\\
491	0.00518453740921601\\
492	0.00520358295808275\\
493	0.00522293928533742\\
494	0.00524261779187922\\
495	0.00526263188392777\\
496	0.00528299738189291\\
497	0.00530373302362933\\
498	0.00532486101176553\\
499	0.00534640756522479\\
500	0.00536840230184885\\
501	0.00539087933810041\\
502	0.00541386790452218\\
503	0.00543736834907738\\
504	0.0054614439915598\\
505	0.00548617248850073\\
506	0.0055116125772874\\
507	0.00553782310361131\\
508	0.0055648569138631\\
509	0.00559275992944499\\
510	0.00562158019937333\\
511	0.00565136836734472\\
512	0.00568217708517777\\
513	0.00571406211100909\\
514	0.00574708197366385\\
515	0.00578131351406266\\
516	0.00581686953104215\\
517	0.00585382480276\\
518	0.00589225725566704\\
519	0.00593224754880653\\
520	0.00597387853742271\\
521	0.00601723455268032\\
522	0.00606240046760468\\
523	0.00610946054865342\\
524	0.00615849727031913\\
525	0.00620959043794676\\
526	0.00626281693151702\\
527	0.00631824788187477\\
528	0.00637594074863827\\
529	0.00643592729560997\\
530	0.00649822165307486\\
531	0.00656281489594014\\
532	0.0066296658447173\\
533	0.0066986942309058\\
534	0.00676977330252032\\
535	0.00684272394399031\\
536	0.00691736334893084\\
537	0.00699341814175984\\
538	0.00707042490542339\\
539	0.0071476188305839\\
540	0.00722294958776039\\
541	0.00729331746378857\\
542	0.00735830771343979\\
543	0.00741779756544083\\
544	0.00747195922604685\\
545	0.00752384580560199\\
546	0.00757452773738583\\
547	0.00762440303069397\\
548	0.00767396425646276\\
549	0.00772337919509641\\
550	0.00777282712471393\\
551	0.00782241109280941\\
552	0.00787221726405221\\
553	0.00792232832964961\\
554	0.00797317874546677\\
555	0.00802484701027154\\
556	0.0080774303620254\\
557	0.0081310393814554\\
558	0.00818463678864738\\
559	0.00823728050780312\\
560	0.00828871446636067\\
561	0.00833921187124901\\
562	0.00838994299338705\\
563	0.00844091791471574\\
564	0.00849218405124884\\
565	0.00854377391309125\\
566	0.00859567613829029\\
567	0.00864786633285968\\
568	0.00869902037501938\\
569	0.00874855302811077\\
570	0.008797220175021\\
571	0.00884578301511914\\
572	0.00889429852763621\\
573	0.00894278551034473\\
574	0.00899119860161022\\
575	0.00903949700000556\\
576	0.00908741881847461\\
577	0.00913460685472602\\
578	0.00918155405806483\\
579	0.009228340865056\\
580	0.00927494116860127\\
581	0.00932130240838532\\
582	0.00936736921603097\\
583	0.0094130859679884\\
584	0.00945839808545934\\
585	0.00950325355372117\\
586	0.00954760484499574\\
587	0.00959141132819297\\
588	0.0096346422040883\\
589	0.00967727987660274\\
590	0.00971932328838559\\
591	0.0097606871825126\\
592	0.00980121135503816\\
593	0.00984070634831442\\
594	0.0098789201112722\\
595	0.00991544794299161\\
596	0.00994949078513196\\
597	0.00997920912842892\\
598	0.0100000808076597\\
599	0\\
600	0\\
};
\addplot [color=blue!50!mycolor7,solid,forget plot]
  table[row sep=crcr]{%
1	0.00297192326543294\\
2	0.00297192849108771\\
3	0.00297193380486011\\
4	0.00297193920823002\\
5	0.00297194470270199\\
6	0.00297195028980567\\
7	0.0029719559710962\\
8	0.00297196174815465\\
9	0.00297196762258838\\
10	0.00297197359603158\\
11	0.00297197967014562\\
12	0.00297198584661952\\
13	0.00297199212717044\\
14	0.00297199851354404\\
15	0.00297200500751507\\
16	0.00297201161088781\\
17	0.0029720183254965\\
18	0.0029720251532059\\
19	0.0029720320959117\\
20	0.00297203915554108\\
21	0.00297204633405326\\
22	0.00297205363344002\\
23	0.00297206105572608\\
24	0.0029720686029698\\
25	0.00297207627726363\\
26	0.00297208408073486\\
27	0.00297209201554593\\
28	0.00297210008389506\\
29	0.00297210828801694\\
30	0.00297211663018328\\
31	0.00297212511270333\\
32	0.00297213373792456\\
33	0.00297214250823326\\
34	0.00297215142605521\\
35	0.00297216049385623\\
36	0.00297216971414283\\
37	0.00297217908946304\\
38	0.00297218862240686\\
39	0.00297219831560698\\
40	0.00297220817173956\\
41	0.00297221819352491\\
42	0.00297222838372811\\
43	0.00297223874515977\\
44	0.00297224928067679\\
45	0.00297225999318307\\
46	0.0029722708856303\\
47	0.00297228196101866\\
48	0.00297229322239763\\
49	0.00297230467286677\\
50	0.00297231631557649\\
51	0.00297232815372899\\
52	0.0029723401905788\\
53	0.00297235242943393\\
54	0.00297236487365653\\
55	0.00297237752666375\\
56	0.00297239039192869\\
57	0.00297240347298125\\
58	0.00297241677340894\\
59	0.00297243029685791\\
60	0.00297244404703381\\
61	0.00297245802770273\\
62	0.00297247224269217\\
63	0.00297248669589197\\
64	0.00297250139125534\\
65	0.00297251633279966\\
66	0.00297253152460777\\
67	0.00297254697082877\\
68	0.00297256267567915\\
69	0.0029725786434438\\
70	0.002972594878477\\
71	0.00297261138520368\\
72	0.00297262816812034\\
73	0.0029726452317962\\
74	0.00297266258087434\\
75	0.00297268022007288\\
76	0.00297269815418593\\
77	0.002972716388085\\
78	0.0029727349267201\\
79	0.0029727537751208\\
80	0.00297277293839767\\
81	0.00297279242174338\\
82	0.00297281223043392\\
83	0.00297283236982995\\
84	0.00297285284537795\\
85	0.00297287366261174\\
86	0.00297289482715353\\
87	0.00297291634471546\\
88	0.00297293822110087\\
89	0.00297296046220563\\
90	0.00297298307401954\\
91	0.00297300606262787\\
92	0.00297302943421253\\
93	0.00297305319505375\\
94	0.0029730773515314\\
95	0.00297310191012655\\
96	0.00297312687742294\\
97	0.00297315226010846\\
98	0.00297317806497672\\
99	0.00297320429892866\\
100	0.00297323096897412\\
101	0.00297325808223338\\
102	0.0029732856459389\\
103	0.00297331366743682\\
104	0.00297334215418881\\
105	0.00297337111377363\\
106	0.00297340055388894\\
107	0.00297343048235302\\
108	0.00297346090710656\\
109	0.00297349183621446\\
110	0.00297352327786756\\
111	0.00297355524038477\\
112	0.0029735877322147\\
113	0.00297362076193768\\
114	0.0029736543382678\\
115	0.00297368847005477\\
116	0.00297372316628598\\
117	0.00297375843608866\\
118	0.00297379428873192\\
119	0.0029738307336289\\
120	0.00297386778033898\\
121	0.00297390543857001\\
122	0.00297394371818062\\
123	0.00297398262918265\\
124	0.0029740221817433\\
125	0.0029740623861879\\
126	0.00297410325300227\\
127	0.00297414479283535\\
128	0.00297418701650188\\
129	0.00297422993498507\\
130	0.00297427355943963\\
131	0.00297431790119455\\
132	0.0029743629717561\\
133	0.00297440878281106\\
134	0.00297445534622989\\
135	0.0029745026740701\\
136	0.00297455077857986\\
137	0.00297459967220144\\
138	0.00297464936757518\\
139	0.00297469987754328\\
140	0.00297475121515414\\
141	0.00297480339366654\\
142	0.00297485642655414\\
143	0.00297491032751038\\
144	0.00297496511045341\\
145	0.00297502078953136\\
146	0.00297507737912785\\
147	0.00297513489386797\\
148	0.00297519334862422\\
149	0.00297525275852331\\
150	0.00297531313895285\\
151	0.00297537450556887\\
152	0.00297543687430332\\
153	0.00297550026137254\\
154	0.00297556468328577\\
155	0.00297563015685459\\
156	0.00297569669920258\\
157	0.00297576432777572\\
158	0.00297583306035351\\
159	0.00297590291506077\\
160	0.00297597391037986\\
161	0.00297604606516393\\
162	0.00297611939865087\\
163	0.00297619393047793\\
164	0.00297626968069726\\
165	0.00297634666979225\\
166	0.00297642491869472\\
167	0.00297650444880285\\
168	0.0029765852819999\\
169	0.0029766674406739\\
170	0.0029767509477377\\
171	0.00297683582665\\
172	0.0029769221014367\\
173	0.00297700979671255\\
174	0.00297709893770313\\
175	0.00297718955026662\\
176	0.00297728166091539\\
177	0.00297737529683672\\
178	0.00297747048591255\\
179	0.00297756725673731\\
180	0.00297766563863383\\
181	0.00297776566166611\\
182	0.00297786735664852\\
183	0.00297797075515031\\
184	0.00297807588949446\\
185	0.00297818279275005\\
186	0.00297829149871658\\
187	0.00297840204189932\\
188	0.00297851445747443\\
189	0.0029786287812428\\
190	0.00297874504957194\\
191	0.00297886329932526\\
192	0.00297898356777955\\
193	0.00297910589253177\\
194	0.00297923031139713\\
195	0.00297935686229904\\
196	0.00297948558314475\\
197	0.00297961651167625\\
198	0.00297974968537754\\
199	0.0029798851423208\\
200	0.00298002292121272\\
201	0.00298016306140481\\
202	0.002980305602904\\
203	0.0029804505863832\\
204	0.00298059805319241\\
205	0.00298074804536959\\
206	0.00298090060565201\\
207	0.00298105577748771\\
208	0.00298121360504698\\
209	0.00298137413323429\\
210	0.00298153740770037\\
211	0.00298170347485417\\
212	0.00298187238187557\\
213	0.0029820441767277\\
214	0.00298221890817005\\
215	0.00298239662577125\\
216	0.00298257737992246\\
217	0.00298276122185077\\
218	0.00298294820363292\\
219	0.00298313837820928\\
220	0.00298333179939782\\
221	0.00298352852190864\\
222	0.0029837286013585\\
223	0.00298393209428571\\
224	0.00298413905816518\\
225	0.00298434955142381\\
226	0.00298456363345598\\
227	0.00298478136463954\\
228	0.00298500280635175\\
229	0.00298522802098568\\
230	0.00298545707196699\\
231	0.00298569002377061\\
232	0.00298592694193797\\
233	0.0029861678930945\\
234	0.00298641294496723\\
235	0.00298666216640291\\
236	0.00298691562738625\\
237	0.0029871733990585\\
238	0.00298743555373633\\
239	0.00298770216493102\\
240	0.00298797330736793\\
241	0.00298824905700636\\
242	0.00298852949105966\\
243	0.00298881468801562\\
244	0.0029891047276573\\
245	0.00298939969108426\\
246	0.00298969966073378\\
247	0.00299000472040299\\
248	0.00299031495527087\\
249	0.00299063045192087\\
250	0.00299095129836397\\
251	0.00299127758406192\\
252	0.00299160939995109\\
253	0.00299194683846672\\
254	0.00299228999356744\\
255	0.00299263896076052\\
256	0.00299299383712729\\
257	0.00299335472134929\\
258	0.00299372171373491\\
259	0.00299409491624631\\
260	0.00299447443252718\\
261	0.00299486036793095\\
262	0.00299525282954938\\
263	0.00299565192624218\\
264	0.00299605776866688\\
265	0.00299647046930953\\
266	0.00299689014251589\\
267	0.00299731690452362\\
268	0.00299775087349492\\
269	0.00299819216954995\\
270	0.00299864091480115\\
271	0.00299909723338806\\
272	0.00299956125151336\\
273	0.00300003309747942\\
274	0.00300051290172588\\
275	0.00300100079686818\\
276	0.00300149691773686\\
277	0.00300200140141804\\
278	0.0030025143872948\\
279	0.00300303601708983\\
280	0.00300356643490879\\
281	0.00300410578728536\\
282	0.00300465422322711\\
283	0.00300521189426284\\
284	0.00300577895449122\\
285	0.00300635556063074\\
286	0.00300694187207132\\
287	0.00300753805092712\\
288	0.00300814426209132\\
289	0.00300876067329241\\
290	0.00300938745515202\\
291	0.00301002478124515\\
292	0.00301067282816177\\
293	0.00301133177557083\\
294	0.00301200180628622\\
295	0.00301268310633499\\
296	0.00301337586502825\\
297	0.00301408027503399\\
298	0.00301479653245312\\
299	0.00301552483689783\\
300	0.00301626539157315\\
301	0.00301701840336147\\
302	0.00301778408291026\\
303	0.00301856264472327\\
304	0.00301935430725549\\
305	0.00302015929301157\\
306	0.0030209778286485\\
307	0.00302181014508274\\
308	0.00302265647760128\\
309	0.00302351706597814\\
310	0.00302439215459537\\
311	0.00302528199256969\\
312	0.00302618683388472\\
313	0.00302710693752906\\
314	0.0030280425676409\\
315	0.00302899399365933\\
316	0.00302996149048252\\
317	0.00303094533863383\\
318	0.00303194582443583\\
319	0.00303296324019279\\
320	0.00303399788438222\\
321	0.00303505006185616\\
322	0.00303612008405267\\
323	0.00303720826921823\\
324	0.00303831494264185\\
325	0.00303944043690124\\
326	0.00304058509212267\\
327	0.00304174925625438\\
328	0.0030429332853553\\
329	0.00304413754389967\\
330	0.00304536240509889\\
331	0.00304660825124139\\
332	0.00304787547405226\\
333	0.00304916447507377\\
334	0.00305047566606819\\
335	0.00305180946944469\\
336	0.0030531663187121\\
337	0.00305454665895902\\
338	0.00305595094736394\\
339	0.00305737965373702\\
340	0.00305883326109636\\
341	0.00306031226628126\\
342	0.00306181718060535\\
343	0.00306334853055253\\
344	0.0030649068585198\\
345	0.00306649272360986\\
346	0.00306810670247837\\
347	0.00306974939024005\\
348	0.00307142140143847\\
349	0.00307312337108557\\
350	0.00307485595577593\\
351	0.00307661983488361\\
352	0.00307841571184806\\
353	0.00308024431555729\\
354	0.00308210640183747\\
355	0.00308400275505891\\
356	0.00308593418986885\\
357	0.00308790155306362\\
358	0.00308990572561349\\
359	0.0030919476248551\\
360	0.00309402820686782\\
361	0.00309614846905274\\
362	0.00309830945293402\\
363	0.00310051224720592\\
364	0.00310275799104982\\
365	0.00310504787774915\\
366	0.0031073831586329\\
367	0.00310976514738142\\
368	0.00311219522473144\\
369	0.00311467484362128\\
370	0.00311720553482062\\
371	0.00311978891309297\\
372	0.00312242668394282\\
373	0.00312512065100297\\
374	0.00312787272411934\\
375	0.00313068492819382\\
376	0.00313355941284423\\
377	0.00313649846293878\\
378	0.00313950451005479\\
379	0.00314258014490076\\
380	0.00314572813071897\\
381	0.00314895141765369\\
382	0.00315225315801962\\
383	0.00315563672232763\\
384	0.00315910571580269\\
385	0.00316266399494209\\
386	0.00316631568338992\\
387	0.00317006518615293\\
388	0.00317391720177608\\
389	0.00317787673782322\\
390	0.00318194916541734\\
391	0.00318614016871271\\
392	0.00319045571082272\\
393	0.00319490198744776\\
394	0.00319948534382541\\
395	0.00320421211262731\\
396	0.0032090882194404\\
397	0.0032141189441812\\
398	0.00321930867920897\\
399	0.00322466037059095\\
400	0.00323017473252839\\
401	0.00323584914734809\\
402	0.00324167612967501\\
403	0.00324764120046018\\
404	0.00325372006273676\\
405	0.00325989226713892\\
406	0.00326615919559881\\
407	0.0032725222585558\\
408	0.00327898289642443\\
409	0.00328554258162417\\
410	0.00329220282095133\\
411	0.00329896515837911\\
412	0.00330583117839172\\
413	0.00331280250998085\\
414	0.00331988083145314\\
415	0.00332706787622019\\
416	0.00333436543977539\\
417	0.00334177538819296\\
418	0.00334929966900734\\
419	0.00335694032674496\\
420	0.00336469951914406\\
421	0.00337257953309532\\
422	0.00338058280253955\\
423	0.00338871192823937\\
424	0.00339696969716105\\
425	0.00340535909282724\\
426	0.00341388326663196\\
427	0.00342254536842661\\
428	0.00343134789266632\\
429	0.00344029425853923\\
430	0.00344938958914137\\
431	0.00345864029843578\\
432	0.0034680554109027\\
433	0.00347765007899885\\
434	0.00348743807663984\\
435	0.00349742942308513\\
436	0.00350763547566956\\
437	0.00351806917208304\\
438	0.00352874532548726\\
439	0.00353968098715047\\
440	0.00355089590030034\\
441	0.00356241307142452\\
442	0.00357425946220679\\
443	0.00358646683223196\\
444	0.00359907278460739\\
445	0.00361212206819994\\
446	0.00362566820547359\\
447	0.00363977553478969\\
448	0.00365452178190924\\
449	0.00367000130911203\\
450	0.00368632923394757\\
451	0.00370364666515478\\
452	0.00372212737054539\\
453	0.00374198626058503\\
454	0.00376349009437049\\
455	0.00378697060501154\\
456	0.00381283910296181\\
457	0.00384159696106142\\
458	0.00387108876480651\\
459	0.00390101690640808\\
460	0.0039313962986117\\
461	0.00396224673830958\\
462	0.00399359360022317\\
463	0.00402544144866336\\
464	0.00405779400029026\\
465	0.00409065388320743\\
466	0.00412402233688062\\
467	0.00415789883720314\\
468	0.00419228062760907\\
469	0.00422716213246828\\
470	0.00426253422407991\\
471	0.00429838331160933\\
472	0.00433469016367517\\
473	0.00437142840292235\\
474	0.00440856262423862\\
475	0.00444604599926384\\
476	0.00448381721438347\\
477	0.00452179653481061\\
478	0.00455988070330554\\
479	0.00459793668918728\\
480	0.00463579349988218\\
481	0.00467323158753825\\
482	0.00470996921113637\\
483	0.00474564470575754\\
484	0.00477979385027262\\
485	0.00481182185046603\\
486	0.00484097243805054\\
487	0.00486701436438504\\
488	0.00489335538968276\\
489	0.00491998009974693\\
490	0.00494687037390386\\
491	0.00497400512234161\\
492	0.00500136003270348\\
493	0.00502890734383619\\
494	0.0050566156730444\\
495	0.00508444993491773\\
496	0.0051123714054439\\
497	0.00514033800587263\\
498	0.00516830491121116\\
499	0.0051962256267867\\
500	0.00522405378907773\\
501	0.00525174591771893\\
502	0.00527926584460572\\
503	0.00530659184906511\\
504	0.00533372336603672\\
505	0.00536069365428154\\
506	0.0053875904763741\\
507	0.00541458320534849\\
508	0.00544195809251283\\
509	0.00546992178969703\\
510	0.005498538335368\\
511	0.00552787455969804\\
512	0.0055579681207797\\
513	0.00558886125579721\\
514	0.0056206014619721\\
515	0.00565322638246048\\
516	0.00568674847140059\\
517	0.0057212308688625\\
518	0.00575674551790978\\
519	0.00579337306975074\\
520	0.00583120189214165\\
521	0.00587032800099864\\
522	0.0059108546186808\\
523	0.00595288891373624\\
524	0.00599653577426036\\
525	0.00604188913260583\\
526	0.00608903244819851\\
527	0.00613806556146913\\
528	0.00618912611787601\\
529	0.00624230508774993\\
530	0.00629769110517033\\
531	0.00635536736052469\\
532	0.00641539803371134\\
533	0.00647783464899278\\
534	0.00654271368216275\\
535	0.0066100513681072\\
536	0.00667983557178508\\
537	0.00675201885533084\\
538	0.00682651266126724\\
539	0.00690317866017905\\
540	0.00698184842468516\\
541	0.007062326382147\\
542	0.00714425974676821\\
543	0.00722705443205354\\
544	0.0073097348376849\\
545	0.00738867023630764\\
546	0.00746253417530004\\
547	0.00753099240649944\\
548	0.00759394049288154\\
549	0.00765186217486399\\
550	0.00770828056856722\\
551	0.00776348711156512\\
552	0.00781788255022238\\
553	0.00787193678617627\\
554	0.00792574450437309\\
555	0.00797954121087952\\
556	0.00803343939979156\\
557	0.00808752139615885\\
558	0.00814194817042168\\
559	0.00819715244792679\\
560	0.00825324282435806\\
561	0.00830974153159578\\
562	0.00836525676763802\\
563	0.0084195781121521\\
564	0.00847241130832356\\
565	0.00852512719578532\\
566	0.0085779445723264\\
567	0.00863087240662908\\
568	0.00868397018295661\\
569	0.0087372658152368\\
570	0.00878981800963339\\
571	0.00884072650297975\\
572	0.00889010449020096\\
573	0.00893923129005439\\
574	0.00898814985119668\\
575	0.0090368973890127\\
576	0.0090854595637907\\
577	0.00913379619373393\\
578	0.00918139648530322\\
579	0.00922829226536571\\
580	0.00927492667082063\\
581	0.00932129541419543\\
582	0.00936736553490938\\
583	0.00941308409583255\\
584	0.00945839719768034\\
585	0.00950325317005161\\
586	0.00954760469820633\\
587	0.00959141128043958\\
588	0.00963464219169453\\
589	0.00967727987432316\\
590	0.00971932328816298\\
591	0.0097606871825126\\
592	0.00980121135503816\\
593	0.00984070634831441\\
594	0.0098789201112722\\
595	0.00991544794299161\\
596	0.00994949078513196\\
597	0.00997920912842892\\
598	0.0100000808076597\\
599	0\\
600	0\\
};
\addplot [color=blue!40!mycolor9,solid,forget plot]
  table[row sep=crcr]{%
1	0.00253639596147216\\
2	0.00253640164164778\\
3	0.00253640741744393\\
4	0.00253641329046568\\
5	0.00253641926234494\\
6	0.00253642533474093\\
7	0.00253643150934064\\
8	0.00253643778785915\\
9	0.00253644417204027\\
10	0.002536450663657\\
11	0.00253645726451185\\
12	0.0025364639764376\\
13	0.00253647080129748\\
14	0.00253647774098598\\
15	0.00253648479742918\\
16	0.00253649197258531\\
17	0.00253649926844532\\
18	0.00253650668703341\\
19	0.00253651423040758\\
20	0.00253652190066021\\
21	0.00253652969991853\\
22	0.00253653763034527\\
23	0.00253654569413936\\
24	0.00253655389353639\\
25	0.0025365622308092\\
26	0.00253657070826856\\
27	0.0025365793282638\\
28	0.0025365880931834\\
29	0.00253659700545576\\
30	0.00253660606754963\\
31	0.00253661528197506\\
32	0.00253662465128386\\
33	0.00253663417807041\\
34	0.00253664386497228\\
35	0.00253665371467103\\
36	0.00253666372989298\\
37	0.0025366739134097\\
38	0.00253668426803905\\
39	0.00253669479664577\\
40	0.00253670550214233\\
41	0.00253671638748965\\
42	0.00253672745569804\\
43	0.00253673870982783\\
44	0.00253675015299036\\
45	0.00253676178834865\\
46	0.00253677361911846\\
47	0.00253678564856896\\
48	0.00253679788002373\\
49	0.00253681031686168\\
50	0.00253682296251791\\
51	0.00253683582048451\\
52	0.0025368488943118\\
53	0.002536862187609\\
54	0.00253687570404538\\
55	0.00253688944735116\\
56	0.00253690342131858\\
57	0.00253691762980282\\
58	0.0025369320767232\\
59	0.00253694676606402\\
60	0.00253696170187586\\
61	0.00253697688827652\\
62	0.00253699232945212\\
63	0.00253700802965832\\
64	0.00253702399322144\\
65	0.00253704022453959\\
66	0.0025370567280838\\
67	0.0025370735083994\\
68	0.00253709057010705\\
69	0.00253710791790409\\
70	0.00253712555656583\\
71	0.00253714349094664\\
72	0.00253716172598152\\
73	0.00253718026668726\\
74	0.00253719911816383\\
75	0.00253721828559579\\
76	0.0025372377742536\\
77	0.00253725758949513\\
78	0.00253727773676702\\
79	0.00253729822160626\\
80	0.00253731904964157\\
81	0.00253734022659491\\
82	0.00253736175828317\\
83	0.00253738365061957\\
84	0.00253740590961538\\
85	0.00253742854138151\\
86	0.0025374515521301\\
87	0.00253747494817631\\
88	0.00253749873593993\\
89	0.00253752292194723\\
90	0.00253754751283265\\
91	0.00253757251534061\\
92	0.00253759793632746\\
93	0.00253762378276318\\
94	0.00253765006173343\\
95	0.00253767678044137\\
96	0.0025377039462098\\
97	0.00253773156648305\\
98	0.00253775964882904\\
99	0.00253778820094148\\
100	0.00253781723064189\\
101	0.00253784674588182\\
102	0.00253787675474502\\
103	0.00253790726544987\\
104	0.00253793828635149\\
105	0.00253796982594422\\
106	0.00253800189286398\\
107	0.00253803449589078\\
108	0.0025380676439511\\
109	0.00253810134612066\\
110	0.0025381356116269\\
111	0.00253817044985164\\
112	0.00253820587033393\\
113	0.0025382418827727\\
114	0.00253827849702985\\
115	0.0025383157231329\\
116	0.00253835357127828\\
117	0.00253839205183419\\
118	0.0025384311753439\\
119	0.00253847095252884\\
120	0.00253851139429202\\
121	0.00253855251172144\\
122	0.00253859431609351\\
123	0.0025386368188766\\
124	0.0025386800317349\\
125	0.00253872396653199\\
126	0.00253876863533476\\
127	0.00253881405041753\\
128	0.00253886022426599\\
129	0.00253890716958165\\
130	0.00253895489928585\\
131	0.00253900342652449\\
132	0.00253905276467263\\
133	0.00253910292733914\\
134	0.00253915392837164\\
135	0.0025392057818615\\
136	0.00253925850214913\\
137	0.00253931210382931\\
138	0.00253936660175663\\
139	0.00253942201105135\\
140	0.00253947834710513\\
141	0.00253953562558726\\
142	0.00253959386245074\\
143	0.00253965307393895\\
144	0.00253971327659206\\
145	0.00253977448725413\\
146	0.00253983672308008\\
147	0.00253990000154291\\
148	0.00253996434044132\\
149	0.00254002975790738\\
150	0.00254009627241445\\
151	0.00254016390278529\\
152	0.00254023266820056\\
153	0.00254030258820717\\
154	0.00254037368272723\\
155	0.00254044597206686\\
156	0.00254051947692532\\
157	0.0025405942184045\\
158	0.00254067021801809\\
159	0.00254074749770118\\
160	0.00254082607981993\\
161	0.00254090598718129\\
162	0.00254098724304253\\
163	0.00254106987112111\\
164	0.00254115389560416\\
165	0.00254123934115806\\
166	0.00254132623293749\\
167	0.00254141459659483\\
168	0.00254150445828855\\
169	0.00254159584469172\\
170	0.00254168878299974\\
171	0.00254178330093764\\
172	0.00254187942676659\\
173	0.00254197718928996\\
174	0.00254207661785816\\
175	0.00254217774237286\\
176	0.00254228059329008\\
177	0.0025423852016223\\
178	0.00254249159893942\\
179	0.00254259981736869\\
180	0.00254270988959348\\
181	0.00254282184885109\\
182	0.00254293572892944\\
183	0.0025430515641633\\
184	0.00254316938942961\\
185	0.00254328924014285\\
186	0.00254341115225046\\
187	0.00254353516222887\\
188	0.00254366130708092\\
189	0.00254378962433511\\
190	0.00254392015204757\\
191	0.00254405292880769\\
192	0.00254418799374815\\
193	0.00254432538656004\\
194	0.00254446514751389\\
195	0.00254460731748657\\
196	0.00254475193799422\\
197	0.00254489905123227\\
198	0.00254504870012367\\
199	0.00254520092833324\\
200	0.00254535578028064\\
201	0.00254551330115373\\
202	0.00254567353692205\\
203	0.00254583653435064\\
204	0.00254600234101407\\
205	0.00254617100531076\\
206	0.00254634257647758\\
207	0.00254651710460463\\
208	0.00254669464065041\\
209	0.00254687523645725\\
210	0.00254705894476687\\
211	0.00254724581923657\\
212	0.00254743591445535\\
213	0.00254762928596062\\
214	0.00254782599025498\\
215	0.00254802608482361\\
216	0.0025482296281517\\
217	0.00254843667974238\\
218	0.00254864730013494\\
219	0.00254886155092341\\
220	0.00254907949477551\\
221	0.00254930119545188\\
222	0.00254952671782577\\
223	0.00254975612790314\\
224	0.00254998949284292\\
225	0.00255022688097804\\
226	0.0025504683618365\\
227	0.00255071400616294\\
228	0.00255096388594092\\
229	0.00255121807441522\\
230	0.00255147664611477\\
231	0.00255173967687609\\
232	0.0025520072438671\\
233	0.00255227942561137\\
234	0.00255255630201301\\
235	0.00255283795438185\\
236	0.00255312446545928\\
237	0.00255341591944451\\
238	0.00255371240202144\\
239	0.00255401400038605\\
240	0.0025543208032744\\
241	0.00255463290099099\\
242	0.00255495038543799\\
243	0.00255527335014495\\
244	0.00255560189029913\\
245	0.00255593610277633\\
246	0.00255627608617278\\
247	0.00255662194083725\\
248	0.00255697376890409\\
249	0.00255733167432705\\
250	0.00255769576291356\\
251	0.00255806614236005\\
252	0.00255844292228789\\
253	0.00255882621428021\\
254	0.00255921613191946\\
255	0.00255961279082587\\
256	0.00256001630869685\\
257	0.00256042680534712\\
258	0.0025608444027499\\
259	0.00256126922507912\\
260	0.00256170139875233\\
261	0.00256214105247489\\
262	0.00256258831728527\\
263	0.00256304332660093\\
264	0.00256350621626604\\
265	0.00256397712459965\\
266	0.0025644561924456\\
267	0.00256494356322318\\
268	0.00256543938297928\\
269	0.00256594380044197\\
270	0.00256645696707498\\
271	0.00256697903713412\\
272	0.00256751016772462\\
273	0.00256805051886018\\
274	0.00256860025352354\\
275	0.0025691595377286\\
276	0.00256972854058408\\
277	0.00257030743435891\\
278	0.00257089639454947\\
279	0.00257149559994836\\
280	0.00257210523271534\\
281	0.00257272547844991\\
282	0.00257335652626603\\
283	0.00257399856886889\\
284	0.00257465180263372\\
285	0.00257531642768697\\
286	0.00257599264798951\\
287	0.00257668067142252\\
288	0.00257738070987547\\
289	0.002578092979337\\
290	0.00257881769998821\\
291	0.00257955509629863\\
292	0.0025803053971252\\
293	0.00258106883581405\\
294	0.00258184565030525\\
295	0.00258263608324105\\
296	0.00258344038207675\\
297	0.00258425879919568\\
298	0.00258509159202711\\
299	0.00258593902316807\\
300	0.00258680136050881\\
301	0.00258767887736237\\
302	0.00258857185259791\\
303	0.00258948057077848\\
304	0.00259040532230295\\
305	0.00259134640355255\\
306	0.00259230411704221\\
307	0.0025932787715764\\
308	0.00259427068241031\\
309	0.00259528017141604\\
310	0.00259630756725441\\
311	0.002597353205552\\
312	0.00259841742908427\\
313	0.0025995005879645\\
314	0.00260060303983907\\
315	0.0026017251500888\\
316	0.00260286729203752\\
317	0.00260402984716693\\
318	0.00260521320533876\\
319	0.00260641776502429\\
320	0.00260764393354142\\
321	0.00260889212729953\\
322	0.00261016277205232\\
323	0.00261145630315913\\
324	0.00261277316585466\\
325	0.00261411381552761\\
326	0.00261547871800838\\
327	0.0026168683498661\\
328	0.00261828319871538\\
329	0.00261972376353269\\
330	0.00262119055498315\\
331	0.00262268409575739\\
332	0.00262420492091947\\
333	0.00262575357826517\\
334	0.00262733062869188\\
335	0.00262893664657938\\
336	0.00263057222018257\\
337	0.00263223795203581\\
338	0.0026339344593691\\
339	0.0026356623745365\\
340	0.00263742234545693\\
341	0.00263921503606698\\
342	0.00264104112678662\\
343	0.00264290131499716\\
344	0.00264479631553181\\
345	0.00264672686117889\\
346	0.00264869370319727\\
347	0.00265069761184437\\
348	0.00265273937691608\\
349	0.0026548198082987\\
350	0.00265693973653227\\
351	0.00265910001338502\\
352	0.00266130151243835\\
353	0.00266354512968164\\
354	0.00266583178411615\\
355	0.00266816241836693\\
356	0.0026705379993016\\
357	0.00267295951865475\\
358	0.00267542799365617\\
359	0.00267794446766103\\
360	0.00268051001077971\\
361	0.00268312572050456\\
362	0.00268579272233039\\
363	0.0026885121703649\\
364	0.00269128524792445\\
365	0.00269411316811011\\
366	0.00269699717435694\\
367	0.00269993854094965\\
368	0.0027029385734949\\
369	0.00270599860933981\\
370	0.00270912001792302\\
371	0.00271230420104327\\
372	0.00271555259302597\\
373	0.0027188666607654\\
374	0.00272224790361525\\
375	0.00272569785309454\\
376	0.00272921807236955\\
377	0.00273281015546447\\
378	0.00273647572614429\\
379	0.00274021643640217\\
380	0.00274403396447075\\
381	0.00274793001226238\\
382	0.00275190630212558\\
383	0.00275596457278575\\
384	0.00276010657431851\\
385	0.00276433406198306\\
386	0.00276864878873289\\
387	0.00277305249623351\\
388	0.00277754690425158\\
389	0.00278213369808515\\
390	0.00278681451131578\\
391	0.00279159090809728\\
392	0.00279646436373471\\
393	0.0028014362432775\\
394	0.00280650777849642\\
395	0.00281168004496556\\
396	0.0028169539500876\\
397	0.00282233022520607\\
398	0.00282780942040282\\
399	0.00283339191558903\\
400	0.00283907795869805\\
401	0.00284486774732679\\
402	0.00285076157877035\\
403	0.00285676010688751\\
404	0.00286286476146833\\
405	0.0028690777410587\\
406	0.00287540132671811\\
407	0.00288183788955991\\
408	0.0028883898902788\\
409	0.00289505987359192\\
410	0.00290185047089191\\
411	0.00290876440252378\\
412	0.00291580447962631\\
413	0.00292297360551568\\
414	0.0029302747766374\\
415	0.00293771108312262\\
416	0.00294528570878807\\
417	0.00295300192957681\\
418	0.00296086310730332\\
419	0.00296887267234843\\
420	0.00297703412422522\\
421	0.00298535110838597\\
422	0.00299382743950797\\
423	0.00300246711737256\\
424	0.00301127434427451\\
425	0.00302025354392576\\
426	0.00302940938297607\\
427	0.00303874680404258\\
428	0.00304827111838002\\
429	0.00305798802703667\\
430	0.00306790360342997\\
431	0.00307802431833261\\
432	0.00308835700932529\\
433	0.00309890857485213\\
434	0.00310968605719879\\
435	0.00312069691500516\\
436	0.00313194903110194\\
437	0.00314345070354735\\
438	0.00315521060342545\\
439	0.00316723765694964\\
440	0.00317954072972286\\
441	0.00319212836496672\\
442	0.0032050089429047\\
443	0.00321819075208515\\
444	0.00323168180287954\\
445	0.00324548956695319\\
446	0.00325962061679849\\
447	0.00327408013053603\\
448	0.00328887121522707\\
449	0.00330399398575248\\
450	0.00331944431418584\\
451	0.00333521213367938\\
452	0.00335127913578002\\
453	0.00336761562881565\\
454	0.00338417619765219\\
455	0.00340089354219989\\
456	0.00341766925817943\\
457	0.00343436953754599\\
458	0.00345090988110113\\
459	0.00346719757304012\\
460	0.00348312143658096\\
461	0.00349854249184381\\
462	0.00351332833008194\\
463	0.00352841623028243\\
464	0.00354382176044773\\
465	0.00355956250275068\\
466	0.00357565842101568\\
467	0.00359213229503034\\
468	0.00360901024685969\\
469	0.00362632238100128\\
470	0.00364410356576846\\
471	0.0036623943900681\\
472	0.00368124234026843\\
473	0.00370070325478469\\
474	0.00372084312763509\\
475	0.00374174035182288\\
476	0.0037634885201182\\
477	0.00378619993524313\\
478	0.0038100100276525\\
479	0.00383508292335543\\
480	0.00386161848171389\\
481	0.00388986122691616\\
482	0.003920111709042\\
483	0.00395274093417956\\
484	0.00398820821411585\\
485	0.00402708256651969\\
486	0.00407006462441754\\
487	0.00411729152427202\\
488	0.00416524753251665\\
489	0.00421392807437712\\
490	0.00426332466894195\\
491	0.00431342395254978\\
492	0.00436420646041381\\
493	0.00441564510446067\\
494	0.00446770326888301\\
495	0.00452033242394037\\
496	0.00457346913190461\\
497	0.0046270312852848\\
498	0.00468091337477193\\
499	0.00473498053074421\\
500	0.00478906101178071\\
501	0.00484293672558531\\
502	0.00489633122957595\\
503	0.00494889444927701\\
504	0.00500018067388641\\
505	0.00504959187065871\\
506	0.00509632230707249\\
507	0.00513933868248703\\
508	0.00517734008866799\\
509	0.00521518649483007\\
510	0.00525348577283978\\
511	0.00529220330473987\\
512	0.00533129910002751\\
513	0.00537072775154602\\
514	0.00541043858460858\\
515	0.00545037635967651\\
516	0.00549048179342777\\
517	0.00553069385821693\\
518	0.00557097950077138\\
519	0.00561135391525453\\
520	0.00565184172039206\\
521	0.00569247112426407\\
522	0.00573332640354678\\
523	0.0057745769136643\\
524	0.00581651792815651\\
525	0.00585957038292697\\
526	0.00590400955177054\\
527	0.00594989864740526\\
528	0.00599727004152914\\
529	0.00604620898691432\\
530	0.00609680337876937\\
531	0.00614914670090307\\
532	0.00620334057308976\\
533	0.00625949441629644\\
534	0.00631772389937083\\
535	0.00637814737263199\\
536	0.00644086813370763\\
537	0.00650598250331581\\
538	0.00657356748169236\\
539	0.00664372994771512\\
540	0.0067164929461154\\
541	0.00679184158282133\\
542	0.00686972598013654\\
543	0.00695005469541889\\
544	0.00703268989861237\\
545	0.00711750496221747\\
546	0.0072042757575853\\
547	0.0072925940425889\\
548	0.007381772208098\\
549	0.00747045918083721\\
550	0.00755460335954314\\
551	0.00763368590183355\\
552	0.00770742032988013\\
553	0.0077756227639839\\
554	0.00783907479661834\\
555	0.00790094230714676\\
556	0.00796148713830445\\
557	0.00802108062130685\\
558	0.00808012572541482\\
559	0.00813881115127087\\
560	0.00819737342835728\\
561	0.00825591836966895\\
562	0.00831457366311632\\
563	0.00837354999477016\\
564	0.00843330046964533\\
565	0.0084922227711323\\
566	0.00854989111099553\\
567	0.00860606389514142\\
568	0.00866092458253271\\
569	0.00871569108912042\\
570	0.00877035956264941\\
571	0.00882496530191853\\
572	0.00887928522500562\\
573	0.00893190365780291\\
574	0.00898266968503837\\
575	0.00903242284757866\\
576	0.009081753502002\\
577	0.00913070794903051\\
578	0.00917932274672422\\
579	0.00922746772004874\\
580	0.00927469143727902\\
581	0.00932119825023774\\
582	0.00936732242912149\\
583	0.00941306119544342\\
584	0.00945838508109932\\
585	0.0095032471389776\\
586	0.0095476019430922\\
587	0.0095914101600048\\
588	0.00963464180215895\\
589	0.00967727976556316\\
590	0.00971932326643662\\
591	0.00976068718020046\\
592	0.00980121135503816\\
593	0.00984070634831441\\
594	0.0098789201112722\\
595	0.00991544794299161\\
596	0.00994949078513196\\
597	0.00997920912842892\\
598	0.0100000808076597\\
599	0\\
600	0\\
};
\addplot [color=blue!75!mycolor7,solid,forget plot]
  table[row sep=crcr]{%
1	0.000302739994282598\\
2	0.000302755298175363\\
3	0.000302770859465515\\
4	0.000302786682478536\\
5	0.000302802771612746\\
6	0.000302819131340412\\
7	0.000302835766209014\\
8	0.000302852680842658\\
9	0.000302869879943232\\
10	0.000302887368291743\\
11	0.000302905150749726\\
12	0.000302923232260467\\
13	0.000302941617850634\\
14	0.000302960312631452\\
15	0.000302979321800244\\
16	0.000302998650641901\\
17	0.000303018304530334\\
18	0.000303038288929987\\
19	0.000303058609397385\\
20	0.00030307927158277\\
21	0.000303100281231542\\
22	0.000303121644186031\\
23	0.000303143366387033\\
24	0.000303165453875543\\
25	0.000303187912794512\\
26	0.000303210749390454\\
27	0.000303233970015301\\
28	0.000303257581128216\\
29	0.000303281589297375\\
30	0.000303306001201831\\
31	0.000303330823633419\\
32	0.000303356063498769\\
33	0.000303381727821081\\
34	0.000303407823742298\\
35	0.00030343435852506\\
36	0.000303461339554688\\
37	0.000303488774341525\\
38	0.000303516670522825\\
39	0.000303545035865099\\
40	0.000303573878266232\\
41	0.000303603205757823\\
42	0.000303633026507387\\
43	0.000303663348820819\\
44	0.000303694181144619\\
45	0.000303725532068492\\
46	0.000303757410327651\\
47	0.000303789824805433\\
48	0.000303822784535824\\
49	0.000303856298706026\\
50	0.000303890376659154\\
51	0.00030392502789695\\
52	0.000303960262082421\\
53	0.000303996089042767\\
54	0.000304032518772116\\
55	0.000304069561434508\\
56	0.000304107227366775\\
57	0.000304145527081554\\
58	0.000304184471270421\\
59	0.000304224070806923\\
60	0.000304264336749731\\
61	0.000304305280345971\\
62	0.000304346913034415\\
63	0.000304389246448883\\
64	0.000304432292421558\\
65	0.000304476062986634\\
66	0.000304520570383711\\
67	0.000304565827061396\\
68	0.000304611845681056\\
69	0.00030465863912049\\
70	0.000304706220477727\\
71	0.000304754603074987\\
72	0.000304803800462534\\
73	0.000304853826422759\\
74	0.00030490469497421\\
75	0.000304956420375838\\
76	0.000305009017131247\\
77	0.000305062499992962\\
78	0.0003051168839669\\
79	0.000305172184316864\\
80	0.000305228416569066\\
81	0.000305285596516918\\
82	0.000305343740225665\\
83	0.000305402864037326\\
84	0.000305462984575537\\
85	0.000305524118750678\\
86	0.000305586283765002\\
87	0.000305649497117785\\
88	0.000305713776610713\\
89	0.000305779140353304\\
90	0.000305845606768495\\
91	0.000305913194598173\\
92	0.000305981922908992\\
93	0.000306051811098289\\
94	0.000306122878899928\\
95	0.000306195146390578\\
96	0.000306268633995674\\
97	0.000306343362495964\\
98	0.000306419353033897\\
99	0.00030649662712013\\
100	0.000306575206640306\\
101	0.000306655113861894\\
102	0.000306736371441143\\
103	0.000306819002430128\\
104	0.000306903030284086\\
105	0.000306988478868732\\
106	0.000307075372467827\\
107	0.000307163735790794\\
108	0.000307253593980697\\
109	0.000307344972621927\\
110	0.000307437897748636\\
111	0.000307532395852832\\
112	0.000307628493892886\\
113	0.000307726219302178\\
114	0.000307825599997769\\
115	0.000307926664389475\\
116	0.000308029441388893\\
117	0.000308133960418679\\
118	0.000308240251422096\\
119	0.000308348344872658\\
120	0.000308458271783849\\
121	0.000308570063719231\\
122	0.000308683752802656\\
123	0.000308799371728588\\
124	0.000308916953772743\\
125	0.000309036532802854\\
126	0.000309158143289642\\
127	0.000309281820317972\\
128	0.000309407599598299\\
129	0.000309535517478099\\
130	0.00030966561095383\\
131	0.00030979791768277\\
132	0.000309932475995291\\
133	0.000310069324907164\\
134	0.000310208504132247\\
135	0.000310350054095319\\
136	0.000310494015944931\\
137	0.000310640431566784\\
138	0.000310789343597116\\
139	0.000310940795436256\\
140	0.000311094831262567\\
141	0.00031125149604636\\
142	0.000311410835564282\\
143	0.000311572896413493\\
144	0.000311737726026556\\
145	0.000311905372686038\\
146	0.000312075885539566\\
147	0.000312249314615009\\
148	0.000312425710835806\\
149	0.000312605126036437\\
150	0.000312787612978125\\
151	0.00031297322536468\\
152	0.000313162017858425\\
153	0.000313354046096433\\
154	0.00031354936670658\\
155	0.000313748037324175\\
156	0.00031395011660841\\
157	0.000314155664258882\\
158	0.000314364741032486\\
159	0.000314577408760316\\
160	0.000314793730364597\\
161	0.000315013769875746\\
162	0.000315237592449782\\
163	0.000315465264385431\\
164	0.000315696853141756\\
165	0.000315932427355669\\
166	0.000316172056859772\\
167	0.000316415812700104\\
168	0.000316663767154403\\
169	0.000316915993750346\\
170	0.000317172567284084\\
171	0.000317433563839116\\
172	0.000317699060805511\\
173	0.0003179691368993\\
174	0.000318243872182677\\
175	0.000318523348084365\\
176	0.000318807647420723\\
177	0.000319096854417525\\
178	0.000319391054732411\\
179	0.000319690335478203\\
180	0.000319994785247168\\
181	0.000320304494136168\\
182	0.000320619553773115\\
183	0.00032094005734438\\
184	0.000321266099623598\\
185	0.000321597777002187\\
186	0.000321935187518866\\
187	0.000322278430895144\\
188	0.000322627608567556\\
189	0.000322982823723317\\
190	0.000323344181336852\\
191	0.00032371178820715\\
192	0.000324085752995889\\
193	0.000324466186266264\\
194	0.000324853200521898\\
195	0.000325246910245845\\
196	0.000325647431939268\\
197	0.000326054884159735\\
198	0.000326469387558586\\
199	0.00032689106491899\\
200	0.000327320041194762\\
201	0.000327756443549825\\
202	0.000328200401398517\\
203	0.000328652046446661\\
204	0.00032911151273338\\
205	0.000329578936673707\\
206	0.000330054457101976\\
207	0.000330538215316278\\
208	0.000331030355123373\\
209	0.000331531022884799\\
210	0.000332040367563745\\
211	0.000332558540772829\\
212	0.000333085696822731\\
213	0.00033362199277197\\
214	0.000334167588477373\\
215	0.000334722646645708\\
216	0.00033528733288631\\
217	0.000335861815764586\\
218	0.000336446266856711\\
219	0.000337040860805315\\
220	0.000337645775376311\\
221	0.000338261191516651\\
222	0.000338887293413517\\
223	0.000339524268554393\\
224	0.000340172307788416\\
225	0.000340831605388913\\
226	0.000341502359117245\\
227	0.000342184770287815\\
228	0.000342879043834255\\
229	0.000343585388377193\\
230	0.000344304016293098\\
231	0.000345035143784588\\
232	0.000345778990952129\\
233	0.000346535781867141\\
234	0.000347305744646519\\
235	0.0003480891115287\\
236	0.000348886118951113\\
237	0.00034969700762937\\
238	0.000350522022637819\\
239	0.000351361413491837\\
240	0.000352215434231614\\
241	0.000353084343507873\\
242	0.000353968404669024\\
243	0.000354867885850151\\
244	0.000355783060063883\\
245	0.000356714205292957\\
246	0.000357661604584661\\
247	0.000358625546147134\\
248	0.000359606323447779\\
249	0.00036060423531335\\
250	0.000361619586032336\\
251	0.000362652685459215\\
252	0.00036370384912084\\
253	0.000364773398325016\\
254	0.000365861660271215\\
255	0.000366968968163518\\
256	0.000368095661325807\\
257	0.000369242085319333\\
258	0.000370408592062608\\
259	0.000371595539953593\\
260	0.000372803293994564\\
261	0.000374032225919337\\
262	0.000375282714323023\\
263	0.000376555144794507\\
264	0.000377849910051427\\
265	0.000379167410077922\\
266	0.000380508052265221\\
267	0.000381872251554916\\
268	0.000383260430585146\\
269	0.000384673019839667\\
270	0.000386110457799969\\
271	0.00038757319110028\\
272	0.000389061674685787\\
273	0.000390576371973988\\
274	0.000392117755019058\\
275	0.00039368630467967\\
276	0.000395282510790062\\
277	0.000396906872334452\\
278	0.000398559897624824\\
279	0.000400242104482305\\
280	0.000401954020422045\\
281	0.000403696182841659\\
282	0.000405469139213398\\
283	0.000407273447279991\\
284	0.000409109675254256\\
285	0.00041097840202263\\
286	0.000412880217352566\\
287	0.000414815722103829\\
288	0.000416785528443947\\
289	0.000418790260067521\\
290	0.000420830552419981\\
291	0.000422907052925273\\
292	0.000425020421217917\\
293	0.000427171329379411\\
294	0.000429360462178938\\
295	0.00043158851731845\\
296	0.000433856205682409\\
297	0.00043616425159192\\
298	0.000438513393063442\\
299	0.000440904382072244\\
300	0.00044333798482047\\
301	0.000445814982009945\\
302	0.000448336169119821\\
303	0.000450902356688971\\
304	0.00045351437060333\\
305	0.000456173052388134\\
306	0.000458879259505053\\
307	0.000461633865654408\\
308	0.000464437761082441\\
309	0.000467291852893507\\
310	0.000470197065367483\\
311	0.000473154340282278\\
312	0.000476164637241473\\
313	0.000479228934007206\\
314	0.000482348226838191\\
315	0.000485523530833032\\
316	0.000488755880278692\\
317	0.000492046329004314\\
318	0.000495395950740345\\
319	0.000498805839482868\\
320	0.000502277109863293\\
321	0.000505810897523383\\
322	0.000509408359495706\\
323	0.000513070674589215\\
324	0.000516799043780484\\
325	0.000520594690610192\\
326	0.000524458861585011\\
327	0.000528392826585036\\
328	0.000532397879276549\\
329	0.000536475337530421\\
330	0.000540626543846002\\
331	0.000544852865780636\\
332	0.000549155696384741\\
333	0.000553536454642801\\
334	0.000557996585920015\\
335	0.000562537562414985\\
336	0.000567160883618368\\
337	0.000571868076777683\\
338	0.000576660697368627\\
339	0.000581540329572756\\
340	0.000586508586761984\\
341	0.000591567111990128\\
342	0.000596717578491686\\
343	0.000601961690188222\\
344	0.000607301182202813\\
345	0.000612737821382818\\
346	0.000618273406831548\\
347	0.000623909770449231\\
348	0.000629648777484099\\
349	0.000635492327093763\\
350	0.00064144235291823\\
351	0.000647500823664655\\
352	0.000653669743705179\\
353	0.000659951153688474\\
354	0.000666347131166062\\
355	0.000672859791234539\\
356	0.000679491287194717\\
357	0.000686243811229083\\
358	0.000693119595098691\\
359	0.000700120910860963\\
360	0.000707250071609904\\
361	0.000714509432240095\\
362	0.000721901390236275\\
363	0.000729428386489971\\
364	0.000737092906145083\\
365	0.000744897479473957\\
366	0.000752844682786146\\
367	0.000760937139371359\\
368	0.000769177520478837\\
369	0.000777568546335077\\
370	0.000786112987201971\\
371	0.000794813664477665\\
372	0.00080367345184244\\
373	0.000812695276452394\\
374	0.000821882120183854\\
375	0.000831237020931818\\
376	0.000840763073966863\\
377	0.000850463433355251\\
378	0.000860341313448925\\
379	0.00087039999045302\\
380	0.000880642804081682\\
381	0.000891073159315225\\
382	0.000901694528276\\
383	0.000912510452245501\\
384	0.000923524543851675\\
385	0.000934740489463825\\
386	0.000946162051843226\\
387	0.000957793073108514\\
388	0.000969637478085907\\
389	0.000981699278124768\\
390	0.000993982575597846\\
391	0.00100649156913924\\
392	0.00101923055974167\\
393	0.00103220395786841\\
394	0.00104541629175878\\
395	0.0010588722171141\\
396	0.00107257652792512\\
397	0.00108653416852251\\
398	0.0011007502470664\\
399	0.00111523005016232\\
400	0.00112997905782196\\
401	0.00114500295712971\\
402	0.00116030765132674\\
403	0.00117589925760862\\
404	0.00119178407943591\\
405	0.00120796854801545\\
406	0.00122445906523196\\
407	0.00124126225393112\\
408	0.00125838546625417\\
409	0.00127583629813086\\
410	0.00129362259800167\\
411	0.00131175247520679\\
412	0.00133023430822517\\
413	0.00134907675376693\\
414	0.00136828875995944\\
415	0.00138787959260221\\
416	0.00140785889752151\\
417	0.00142823685487371\\
418	0.00144902455271073\\
419	0.00147023529251161\\
420	0.00149187921895475\\
421	0.00151396619965922\\
422	0.00153650632639625\\
423	0.0015595099148164\\
424	0.00158298750293422\\
425	0.00160694984817422\\
426	0.00163140792280975\\
427	0.00165637290741864\\
428	0.00168185617913752\\
429	0.00170786929741192\\
430	0.00173442398882915\\
431	0.0017615321282964\\
432	0.00178920571707137\\
433	0.00181745687084665\\
434	0.00184629779957813\\
435	0.00187574076391825\\
436	0.00190579800908742\\
437	0.00193648166432957\\
438	0.00196780363481082\\
439	0.00199977577575946\\
440	0.00203241540572296\\
441	0.00206575526652665\\
442	0.00209982417132157\\
443	0.00213464297298561\\
444	0.00217023364098462\\
445	0.00220661940529046\\
446	0.00224382493398252\\
447	0.00228187655486743\\
448	0.00232080253496434\\
449	0.00236063343643997\\
450	0.00240140257395713\\
451	0.00244314660699236\\
452	0.00248590631232981\\
453	0.00252972759821636\\
454	0.00257466284643414\\
455	0.00262077271350326\\
456	0.00266812865270664\\
457	0.0027168158767048\\
458	0.00276693368829818\\
459	0.00281859743436267\\
460	0.00287193917783625\\
461	0.00292676414338496\\
462	0.00294240337854788\\
463	0.00295841666957302\\
464	0.00297481870245109\\
465	0.0029916248220206\\
466	0.00300885120497219\\
467	0.00302651502893029\\
468	0.0030446345289735\\
469	0.00306322904835455\\
470	0.00308231908014449\\
471	0.00310192629727913\\
472	0.00312207355245652\\
473	0.00314278481432968\\
474	0.00316408504702386\\
475	0.00318600002940449\\
476	0.00320855606155137\\
477	0.00323177950499712\\
478	0.0032556960713458\\
479	0.00328032984485581\\
480	0.00330570192397114\\
481	0.00333182791229078\\
482	0.00335871295423447\\
483	0.00338635513003948\\
484	0.00341473919710743\\
485	0.00344382949236592\\
486	0.00347356394856738\\
487	0.00350387642864214\\
488	0.00353480764323836\\
489	0.00356637961672365\\
490	0.00359861635786098\\
491	0.00363154412771965\\
492	0.00366519174018748\\
493	0.00369959089548871\\
494	0.00373477654507782\\
495	0.00377078728313316\\
496	0.00380766575525671\\
497	0.00384545906863911\\
498	0.00388421918017913\\
499	0.00392400323196396\\
500	0.00396487380411488\\
501	0.00400689907951117\\
502	0.0040501529855268\\
503	0.00409471543509325\\
504	0.00414074037734099\\
505	0.00418951254045747\\
506	0.00424299189042817\\
507	0.00430213605657748\\
508	0.00436810608328926\\
509	0.0044358505684159\\
510	0.00450469277849023\\
511	0.00457459669947853\\
512	0.00464550917335289\\
513	0.00471735528806932\\
514	0.0047900322445073\\
515	0.00486340150823466\\
516	0.00493727886858318\\
517	0.00501142185703412\\
518	0.00508551225512467\\
519	0.00515913249674867\\
520	0.00523173759710111\\
521	0.00530262004564923\\
522	0.00537086305968205\\
523	0.00543528231093768\\
524	0.00549436055022657\\
525	0.00554768139256602\\
526	0.00560163332094288\\
527	0.00565618829004466\\
528	0.00571135522937516\\
529	0.00576720925055922\\
530	0.00582377256464386\\
531	0.00588105471869473\\
532	0.00593901732619033\\
533	0.00599763487375904\\
534	0.00605690683632071\\
535	0.00611687588535289\\
536	0.00617765578377269\\
537	0.00623947030199581\\
538	0.00630270723369619\\
539	0.0063678597314642\\
540	0.0064352631442971\\
541	0.00650497750623561\\
542	0.0065770598245537\\
543	0.00665156227790318\\
544	0.00672852947128862\\
545	0.0068079921199354\\
546	0.0068899619262005\\
547	0.00697442592241294\\
548	0.00706133840278816\\
549	0.00715062625320535\\
550	0.00724229015081122\\
551	0.00733611737618848\\
552	0.00743167017581119\\
553	0.00752820781927373\\
554	0.00762404907615123\\
555	0.00771546142067267\\
556	0.00780194620281726\\
557	0.00788310563601214\\
558	0.00795867414035873\\
559	0.00802894282748545\\
560	0.0080974851814359\\
561	0.00816449542624515\\
562	0.00823029209913526\\
563	0.00829522988454044\\
564	0.00835945147524811\\
565	0.00842331177448014\\
566	0.00848699209207288\\
567	0.0085506326298744\\
568	0.00861394395281648\\
569	0.00867586884864746\\
570	0.00873621921966545\\
571	0.00879472849927723\\
572	0.00885188900131368\\
573	0.00890865459807043\\
574	0.00896504023463299\\
575	0.00902007060520066\\
576	0.00907309170123428\\
577	0.00912399733381067\\
578	0.00917399234008678\\
579	0.00922330993051709\\
580	0.00927201307836196\\
581	0.0093199511204804\\
582	0.00936680707102158\\
583	0.00941279464299495\\
584	0.00945824439571526\\
585	0.00950316992917516\\
586	0.00954756153451497\\
587	0.00959139054066191\\
588	0.00963463325124416\\
589	0.00967727655368415\\
590	0.00971932228626027\\
591	0.00976068696475735\\
592	0.00980121132948587\\
593	0.00984070634831441\\
594	0.0098789201112722\\
595	0.00991544794299161\\
596	0.00994949078513196\\
597	0.00997920912842892\\
598	0.0100000808076597\\
599	0\\
600	0\\
};
\addplot [color=blue!80!mycolor9,solid,forget plot]
  table[row sep=crcr]{%
1	0.00015488211034242\\
2	0.000154883092248759\\
3	0.000154884090668618\\
4	0.000154885105879591\\
5	0.000154886138163946\\
6	0.000154887187808709\\
7	0.000154888255105741\\
8	0.000154889340351831\\
9	0.000154890443848748\\
10	0.000154891565903369\\
11	0.000154892706827736\\
12	0.000154893866939156\\
13	0.000154895046560281\\
14	0.000154896246019209\\
15	0.000154897465649573\\
16	0.000154898705790643\\
17	0.000154899966787396\\
18	0.000154901248990657\\
19	0.000154902552757154\\
20	0.000154903878449641\\
21	0.000154905226437011\\
22	0.000154906597094379\\
23	0.000154907990803196\\
24	0.00015490940795137\\
25	0.000154910848933348\\
26	0.000154912314150274\\
27	0.00015491380401005\\
28	0.000154915318927489\\
29	0.000154916859324418\\
30	0.000154918425629811\\
31	0.0001549200182799\\
32	0.000154921637718294\\
33	0.000154923284396135\\
34	0.000154924958772194\\
35	0.000154926661313015\\
36	0.000154928392493055\\
37	0.000154930152794818\\
38	0.00015493194270899\\
39	0.000154933762734581\\
40	0.000154935613379075\\
41	0.000154937495158559\\
42	0.000154939408597904\\
43	0.000154941354230874\\
44	0.000154943332600324\\
45	0.000154945344258329\\
46	0.000154947389766349\\
47	0.000154949469695409\\
48	0.00015495158462623\\
49	0.000154953735149439\\
50	0.000154955921865711\\
51	0.000154958145385961\\
52	0.000154960406331519\\
53	0.000154962705334299\\
54	0.000154965043037002\\
55	0.000154967420093298\\
56	0.000154969837168012\\
57	0.000154972294937329\\
58	0.000154974794088987\\
59	0.000154977335322483\\
60	0.000154979919349275\\
61	0.000154982546892998\\
62	0.000154985218689679\\
63	0.000154987935487942\\
64	0.000154990698049249\\
65	0.000154993507148116\\
66	0.000154996363572341\\
67	0.000154999268123251\\
68	0.000155002221615932\\
69	0.000155005224879467\\
70	0.000155008278757199\\
71	0.000155011384106974\\
72	0.000155014541801413\\
73	0.000155017752728146\\
74	0.00015502101779011\\
75	0.000155024337905808\\
76	0.00015502771400958\\
77	0.000155031147051908\\
78	0.000155034637999679\\
79	0.000155038187836492\\
80	0.000155041797562971\\
81	0.000155045468197029\\
82	0.000155049200774233\\
83	0.000155052996348072\\
84	0.000155056855990321\\
85	0.000155060780791334\\
86	0.0001550647718604\\
87	0.000155068830326089\\
88	0.00015507295733659\\
89	0.000155077154060058\\
90	0.000155081421685005\\
91	0.000155085761420623\\
92	0.000155090174497213\\
93	0.000155094662166523\\
94	0.000155099225702166\\
95	0.00015510386640001\\
96	0.000155108585578568\\
97	0.000155113384579435\\
98	0.000155118264767698\\
99	0.000155123227532354\\
100	0.000155128274286773\\
101	0.000155133406469113\\
102	0.000155138625542791\\
103	0.000155143932996949\\
104	0.000155149330346911\\
105	0.000155154819134666\\
106	0.000155160400929372\\
107	0.000155166077327831\\
108	0.000155171849955009\\
109	0.00015517772046455\\
110	0.000155183690539304\\
111	0.000155189761891854\\
112	0.000155195936265076\\
113	0.000155202215432683\\
114	0.000155208601199794\\
115	0.000155215095403502\\
116	0.000155221699913484\\
117	0.000155228416632558\\
118	0.000155235247497341\\
119	0.000155242194478808\\
120	0.000155249259582972\\
121	0.000155256444851501\\
122	0.00015526375236236\\
123	0.000155271184230484\\
124	0.00015527874260845\\
125	0.000155286429687168\\
126	0.000155294247696551\\
127	0.000155302198906244\\
128	0.000155310285626341\\
129	0.000155318510208102\\
130	0.000155326875044691\\
131	0.000155335382571954\\
132	0.00015534403526915\\
133	0.000155352835659748\\
134	0.000155361786312195\\
135	0.000155370889840738\\
136	0.000155380148906213\\
137	0.00015538956621686\\
138	0.000155399144529186\\
139	0.000155408886648768\\
140	0.000155418795431136\\
141	0.000155428873782622\\
142	0.00015543912466125\\
143	0.000155449551077607\\
144	0.000155460156095739\\
145	0.000155470942834089\\
146	0.000155481914466372\\
147	0.000155493074222546\\
148	0.000155504425389718\\
149	0.000155515971313119\\
150	0.000155527715397041\\
151	0.000155539661105841\\
152	0.00015555181196489\\
153	0.00015556417156158\\
154	0.000155576743546338\\
155	0.000155589531633616\\
156	0.000155602539602943\\
157	0.000155615771299945\\
158	0.000155629230637397\\
159	0.000155642921596293\\
160	0.000155656848226922\\
161	0.000155671014649952\\
162	0.000155685425057537\\
163	0.000155700083714449\\
164	0.000155714994959201\\
165	0.000155730163205235\\
166	0.000155745592942069\\
167	0.000155761288736533\\
168	0.000155777255233962\\
169	0.000155793497159499\\
170	0.000155810019319332\\
171	0.000155826826602045\\
172	0.000155843923979957\\
173	0.00015586131651052\\
174	0.000155879009337741\\
175	0.000155897007693678\\
176	0.000155915316899937\\
177	0.000155933942369263\\
178	0.00015595288960716\\
179	0.000155972164213587\\
180	0.000155991771884657\\
181	0.000156011718414489\\
182	0.000156032009697021\\
183	0.000156052651727942\\
184	0.000156073650606675\\
185	0.000156095012538398\\
186	0.000156116743836135\\
187	0.00015613885092289\\
188	0.000156161340333832\\
189	0.000156184218718512\\
190	0.000156207492843121\\
191	0.000156231169592749\\
192	0.000156255255973708\\
193	0.000156279759115809\\
194	0.000156304686274714\\
195	0.000156330044834234\\
196	0.000156355842308701\\
197	0.000156382086345331\\
198	0.00015640878472663\\
199	0.000156435945372878\\
200	0.000156463576344631\\
201	0.000156491685845267\\
202	0.0001565202822236\\
203	0.000156549373976547\\
204	0.000156578969751815\\
205	0.000156609078350671\\
206	0.00015663970873074\\
207	0.000156670870008891\\
208	0.000156702571464129\\
209	0.000156734822540589\\
210	0.000156767632850553\\
211	0.000156801012177554\\
212	0.000156834970479516\\
213	0.000156869517891961\\
214	0.000156904664731279\\
215	0.000156940421498074\\
216	0.000156976798880542\\
217	0.00015701380775795\\
218	0.000157051459204156\\
219	0.000157089764491205\\
220	0.000157128735093009\\
221	0.000157168382689058\\
222	0.000157208719168271\\
223	0.00015724975663283\\
224	0.000157291507402175\\
225	0.000157333984017022\\
226	0.000157377199243479\\
227	0.000157421166077235\\
228	0.000157465897747845\\
229	0.000157511407723067\\
230	0.000157557709713315\\
231	0.000157604817676187\\
232	0.00015765274582107\\
233	0.00015770150861385\\
234	0.000157751120781708\\
235	0.000157801597317998\\
236	0.000157852953487244\\
237	0.000157905204830202\\
238	0.00015795836716906\\
239	0.000158012456612691\\
240	0.000158067489562049\\
241	0.000158123482715648\\
242	0.000158180453075165\\
243	0.000158238417951116\\
244	0.000158297394968678\\
245	0.000158357402073622\\
246	0.000158418457538319\\
247	0.000158480579967926\\
248	0.00015854378830663\\
249	0.000158608101844062\\
250	0.000158673540221786\\
251	0.000158740123439963\\
252	0.00015880787186411\\
253	0.000158876806231978\\
254	0.000158946947660615\\
255	0.000159018317653499\\
256	0.000159090938107852\\
257	0.000159164831322071\\
258	0.000159240020003312\\
259	0.0001593165272752\\
260	0.00015939437668571\\
261	0.000159473592215169\\
262	0.000159554198284433\\
263	0.000159636219763197\\
264	0.000159719681978475\\
265	0.000159804610723217\\
266	0.000159891032265112\\
267	0.000159978973355529\\
268	0.00016006846123864\\
269	0.000160159523660706\\
270	0.000160252188879517\\
271	0.000160346485674009\\
272	0.000160442443354092\\
273	0.000160540091770583\\
274	0.000160639461325373\\
275	0.000160740582981764\\
276	0.00016084348827495\\
277	0.00016094820932275\\
278	0.000161054778836462\\
279	0.000161163230131935\\
280	0.000161273597140841\\
281	0.000161385914422109\\
282	0.000161500217173571\\
283	0.000161616541243816\\
284	0.000161734923144205\\
285	0.000161855400061107\\
286	0.000161978009868348\\
287	0.000162102791139821\\
288	0.000162229783162328\\
289	0.000162359025948629\\
290	0.000162490560250667\\
291	0.000162624427573018\\
292	0.000162760670186544\\
293	0.000162899331142243\\
294	0.000163040454285322\\
295	0.000163184084269452\\
296	0.000163330266571242\\
297	0.000163479047504936\\
298	0.000163630474237278\\
299	0.000163784594802615\\
300	0.000163941458118189\\
301	0.000164101113999645\\
302	0.000164263613176723\\
303	0.000164429007309166\\
304	0.000164597349002838\\
305	0.000164768691826019\\
306	0.000164943090325913\\
307	0.000165120600045348\\
308	0.000165301277539675\\
309	0.000165485180393856\\
310	0.00016567236723975\\
311	0.000165862897773576\\
312	0.000166056832773578\\
313	0.000166254234117864\\
314	0.000166455164802451\\
315	0.000166659688959454\\
316	0.000166867871875502\\
317	0.00016707978001031\\
318	0.000167295481015421\\
319	0.000167515043753159\\
320	0.00016773853831573\\
321	0.00016796603604454\\
322	0.000168197609549639\\
323	0.000168433332729428\\
324	0.000168673280790486\\
325	0.000168917530267617\\
326	0.000169166159044104\\
327	0.000169419246372175\\
328	0.000169676872893649\\
329	0.000169939120660854\\
330	0.00017020607315774\\
331	0.000170477815321322\\
332	0.000170754433563307\\
333	0.000171036015792087\\
334	0.000171322651435051\\
335	0.000171614431461185\\
336	0.000171911448404178\\
337	0.000172213796385812\\
338	0.000172521571139941\\
339	0.000172834870036911\\
340	0.000173153792108591\\
341	0.000173478438074022\\
342	0.000173808910365766\\
343	0.00017414531315704\\
344	0.000174487752389719\\
345	0.000174836335803289\\
346	0.000175191172964862\\
347	0.000175552375300369\\
348	0.000175920056127055\\
349	0.000176294330687406\\
350	0.000176675316184676\\
351	0.000177063131820124\\
352	0.000177457898832248\\
353	0.000177859740538043\\
354	0.000178268782376652\\
355	0.000178685151955503\\
356	0.000179108979099264\\
357	0.000179540395901786\\
358	0.000179979536781354\\
359	0.000180426538539535\\
360	0.000180881540423874\\
361	0.000181344684194807\\
362	0.000181816114197108\\
363	0.000182295977436206\\
364	0.00018278442365975\\
365	0.000183281605444806\\
366	0.000183787678291063\\
367	0.000184302800720449\\
368	0.000184827134383609\\
369	0.000185360844173591\\
370	0.000185904098347269\\
371	0.000186457068654812\\
372	0.000187019930477767\\
373	0.000187592862976061\\
374	0.000188176049244451\\
375	0.000188769676478764\\
376	0.000189373936152395\\
377	0.000189989024203414\\
378	0.000190615141232688\\
379	0.000191252492713373\\
380	0.00019190128921209\\
381	0.000192561746622143\\
382	0.000193234086408987\\
383	0.000193918535868284\\
384	0.000194615328396707\\
385	0.000195324703775734\\
386	0.000196046908468723\\
387	0.000196782195931467\\
388	0.000197530826936273\\
389	0.000198293069910181\\
390	0.000199069201286734\\
391	0.000199859505871521\\
392	0.000200664277221914\\
393	0.000201483818042263\\
394	0.000202318440598029\\
395	0.000203168467152662\\
396	0.000204034230436915\\
397	0.000204916074165757\\
398	0.000205814353621339\\
399	0.000206729436331614\\
400	0.000207661702895073\\
401	0.000208611548040533\\
402	0.000209579382061606\\
403	0.000210565632708723\\
404	0.000211570746819517\\
405	0.000212595186182724\\
406	0.000213639434908843\\
407	0.00021470401693947\\
408	0.000215789480275019\\
409	0.000216896398448818\\
410	0.000218025371968255\\
411	0.000219177029631961\\
412	0.000220352029559034\\
413	0.00022155105963449\\
414	0.000222774836879354\\
415	0.000224024105179248\\
416	0.000225299632027086\\
417	0.000226602212922861\\
418	0.000227932729517342\\
419	0.000229291979029647\\
420	0.000230680776974727\\
421	0.00023209997747756\\
422	0.000233550476975789\\
423	0.000235033218554093\\
424	0.000236549197053563\\
425	0.000238099465138812\\
426	0.000239685140561535\\
427	0.000241307414887714\\
428	0.000242967564074873\\
429	0.000244666961366925\\
430	0.000246407093026664\\
431	0.000248189577500712\\
432	0.000250016188763052\\
433	0.000251888883953665\\
434	0.000253809833511232\\
435	0.000255781444593046\\
436	0.000257806340671047\\
437	0.000259887159822472\\
438	0.000262025671287431\\
439	0.000264219391247132\\
440	0.000265653123763795\\
441	0.000266539703193121\\
442	0.000267445999822238\\
443	0.000268372837624298\\
444	0.000269321150024615\\
445	0.00027029200261589\\
446	0.000271286621087323\\
447	0.000272306425665031\\
448	0.000273353073710675\\
449	0.000274428512575686\\
450	0.000275535045388131\\
451	0.000276675413193355\\
452	0.000277852897824715\\
453	0.000279071451183423\\
454	0.000280335858971814\\
455	0.000281651954227921\\
456	0.000283026898240145\\
457	0.000284469580502665\\
458	0.000285991273022769\\
459	0.000287606940050815\\
460	0.000289338555855456\\
461	0.000291560589653606\\
462	0.00033475087741394\\
463	0.000378800548146425\\
464	0.000423731581377088\\
465	0.00046956925759918\\
466	0.000516343106275027\\
467	0.000564077957089587\\
468	0.000612799586412904\\
469	0.000662534713683252\\
470	0.000713310902393073\\
471	0.000765156032482341\\
472	0.000818099592129617\\
473	0.000872173562507793\\
474	0.000927411319321856\\
475	0.000983847844933004\\
476	0.00104152019214299\\
477	0.00110046853855466\\
478	0.00116073967158623\\
479	0.00122237382284638\\
480	0.00128541096596254\\
481	0.00134989317134273\\
482	0.00141586524065041\\
483	0.00148337512265894\\
484	0.00155247464605163\\
485	0.00162322065475195\\
486	0.00169567652205239\\
487	0.00176991282386335\\
488	0.00184600433055347\\
489	0.0019240317168221\\
490	0.00200408230754021\\
491	0.0020862509525294\\
492	0.00217064105824466\\
493	0.00225736581138706\\
494	0.0023465496383459\\
495	0.00243832995524582\\
496	0.00253285927584977\\
497	0.00263030775628589\\
498	0.0027308662582295\\
499	0.00283474998010372\\
500	0.00294220256296635\\
501	0.00305350010201554\\
502	0.00316895304009773\\
503	0.00328889884231375\\
504	0.00337618791790624\\
505	0.00341518401913672\\
506	0.00345539458659297\\
507	0.00349681743463734\\
508	0.00353938145182941\\
509	0.00358319395658858\\
510	0.00362835186531812\\
511	0.00367496700057595\\
512	0.00372316914945713\\
513	0.00377311007682587\\
514	0.00382496842778709\\
515	0.00387895565327868\\
516	0.00393532340530068\\
517	0.00399437278354749\\
518	0.00405646598404485\\
519	0.00412204106816463\\
520	0.0041916306164486\\
521	0.00426588516124152\\
522	0.00434560222725418\\
523	0.00443176040363315\\
524	0.00452555192015009\\
525	0.00462691381584357\\
526	0.00472887998864731\\
527	0.00483102097835813\\
528	0.00493277021635203\\
529	0.00503338018249328\\
530	0.0051320546255927\\
531	0.00522947284030173\\
532	0.00532726258769295\\
533	0.00542492194856567\\
534	0.00552178519292106\\
535	0.00561697078933847\\
536	0.00570930923656654\\
537	0.00579729764732285\\
538	0.00587907737522058\\
539	0.00595394909275127\\
540	0.00603026887772444\\
541	0.00610797313114292\\
542	0.006186988638929\\
543	0.00626723547553034\\
544	0.00634863243100694\\
545	0.0064311066315632\\
546	0.00651459620067426\\
547	0.00659908563166519\\
548	0.00668465351358367\\
549	0.00677151906431022\\
550	0.00686004127033893\\
551	0.00695075950672659\\
552	0.00704406272073439\\
553	0.00713993999621669\\
554	0.00723827766913395\\
555	0.00733897889017284\\
556	0.00744182721447793\\
557	0.0075464274848156\\
558	0.00765211352736142\\
559	0.00775763728897795\\
560	0.00785889117832517\\
561	0.0079554178650009\\
562	0.0080467589763176\\
563	0.00813252617384959\\
564	0.00821262848333457\\
565	0.00828954955575012\\
566	0.00836460143736337\\
567	0.00843801425665729\\
568	0.00851007021412093\\
569	0.00858085523599827\\
570	0.00865089280864312\\
571	0.00872045060593021\\
572	0.00878884033650076\\
573	0.00885493535220717\\
574	0.00891894260107574\\
575	0.00898068916277488\\
576	0.00904053984403706\\
577	0.00909951216131474\\
578	0.00915625737993752\\
579	0.00921049150682553\\
580	0.00926216809207642\\
581	0.00931255446014088\\
582	0.0093618754098504\\
583	0.00941001645246001\\
584	0.00945677027090458\\
585	0.00950230484347413\\
586	0.00954706952644597\\
587	0.00959111892966849\\
588	0.00963449203753374\\
589	0.00967720977486807\\
590	0.00971929472567353\\
591	0.00976067763631056\\
592	0.009801209018252\\
593	0.00984070603235368\\
594	0.0098789201112722\\
595	0.00991544794299161\\
596	0.00994949078513196\\
597	0.00997920912842892\\
598	0.0100000808076597\\
599	0\\
600	0\\
};
\addplot [color=blue,solid,forget plot]
  table[row sep=crcr]{%
1	4.4002968351485e-05\\
2	4.40030848319639e-05\\
3	4.40032032713052e-05\\
4	4.40033237024509e-05\\
5	4.40034461588924e-05\\
6	4.40035706746862e-05\\
7	4.40036972844694e-05\\
8	4.40038260234518e-05\\
9	4.40039569274455e-05\\
10	4.40040900328695e-05\\
11	4.40042253767505e-05\\
12	4.40043629967449e-05\\
13	4.40045029311362e-05\\
14	4.40046452188711e-05\\
15	4.40047898995341e-05\\
16	4.4004937013404e-05\\
17	4.40050866014108e-05\\
18	4.40052387051967e-05\\
19	4.40053933671016e-05\\
20	4.40055506301834e-05\\
21	4.40057105382221e-05\\
22	4.4005873135751e-05\\
23	4.40060384680379e-05\\
24	4.40062065811395e-05\\
25	4.40063775218742e-05\\
26	4.40065513378627e-05\\
27	4.40067280775274e-05\\
28	4.40069077901115e-05\\
29	4.40070905257031e-05\\
30	4.40072763352208e-05\\
31	4.40074652704675e-05\\
32	4.40076573841095e-05\\
33	4.40078527297137e-05\\
34	4.40080513617541e-05\\
35	4.40082533356276e-05\\
36	4.40084587076764e-05\\
37	4.40086675352024e-05\\
38	4.40088798764765e-05\\
39	4.40090957907601e-05\\
40	4.40093153383252e-05\\
41	4.40095385804658e-05\\
42	4.40097655795225e-05\\
43	4.40099963989008e-05\\
44	4.40102311030782e-05\\
45	4.40104697576375e-05\\
46	4.40107124292787e-05\\
47	4.4010959185832e-05\\
48	4.40112100962998e-05\\
49	4.40114652308501e-05\\
50	4.40117246608507e-05\\
51	4.40119884588924e-05\\
52	4.40122566988053e-05\\
53	4.40125294556806e-05\\
54	4.40128068059021e-05\\
55	4.40130888271466e-05\\
56	4.40133755984315e-05\\
57	4.40136672001302e-05\\
58	4.40139637139811e-05\\
59	4.40142652231388e-05\\
60	4.40145718121788e-05\\
61	4.40148835671232e-05\\
62	4.4015200575483e-05\\
63	4.40155229262573e-05\\
64	4.40158507099966e-05\\
65	4.40161840187911e-05\\
66	4.40165229463258e-05\\
67	4.40168675878921e-05\\
68	4.40172180404362e-05\\
69	4.40175744025606e-05\\
70	4.40179367745816e-05\\
71	4.40183052585403e-05\\
72	4.40186799582386e-05\\
73	4.40190609792839e-05\\
74	4.40194484290987e-05\\
75	4.40198424169697e-05\\
76	4.40202430540769e-05\\
77	4.40206504535169e-05\\
78	4.40210647303657e-05\\
79	4.40214860016852e-05\\
80	4.40219143865652e-05\\
81	4.40223500061715e-05\\
82	4.40227929837776e-05\\
83	4.40232434447935e-05\\
84	4.40237015168213e-05\\
85	4.40241673296849e-05\\
86	4.40246410154567e-05\\
87	4.40251227085345e-05\\
88	4.40256125456486e-05\\
89	4.40261106659238e-05\\
90	4.40266172109069e-05\\
91	4.40271323246318e-05\\
92	4.40276561536442e-05\\
93	4.40281888470696e-05\\
94	4.40287305566287e-05\\
95	4.40292814367152e-05\\
96	4.40298416444333e-05\\
97	4.40304113396352e-05\\
98	4.40309906849917e-05\\
99	4.40315798460349e-05\\
100	4.40321789912016e-05\\
101	4.40327882919027e-05\\
102	4.40334079225568e-05\\
103	4.40340380606719e-05\\
104	4.40346788868755e-05\\
105	4.40353305849892e-05\\
106	4.4035993342079e-05\\
107	4.40366673485142e-05\\
108	4.4037352798034e-05\\
109	4.40380498877957e-05\\
110	4.40387588184576e-05\\
111	4.40394797942236e-05\\
112	4.40402130229158e-05\\
113	4.40409587160398e-05\\
114	4.40417170888444e-05\\
115	4.40424883604106e-05\\
116	4.40432727536928e-05\\
117	4.4044070495606e-05\\
118	4.40448818171016e-05\\
119	4.4045706953216e-05\\
120	4.40465461431787e-05\\
121	4.4047399630454e-05\\
122	4.40482676628432e-05\\
123	4.40491504925418e-05\\
124	4.40500483762379e-05\\
125	4.40509615751716e-05\\
126	4.40518903552278e-05\\
127	4.40528349870264e-05\\
128	4.40537957459767e-05\\
129	4.40547729124036e-05\\
130	4.40557667715934e-05\\
131	4.40567776139107e-05\\
132	4.40578057348643e-05\\
133	4.40588514352186e-05\\
134	4.40599150210628e-05\\
135	4.40609968039222e-05\\
136	4.40620971008348e-05\\
137	4.40632162344596e-05\\
138	4.40643545331636e-05\\
139	4.4065512331128e-05\\
140	4.40666899684317e-05\\
141	4.40678877911606e-05\\
142	4.40691061515124e-05\\
143	4.40703454078883e-05\\
144	4.40716059250016e-05\\
145	4.40728880739918e-05\\
146	4.40741922325042e-05\\
147	4.40755187848228e-05\\
148	4.40768681219763e-05\\
149	4.40782406418253e-05\\
150	4.40796367492049e-05\\
151	4.40810568560139e-05\\
152	4.40825013813389e-05\\
153	4.40839707515709e-05\\
154	4.40854654005229e-05\\
155	4.40869857695358e-05\\
156	4.40885323076304e-05\\
157	4.40901054715937e-05\\
158	4.40917057261236e-05\\
159	4.40933335439607e-05\\
160	4.40949894060049e-05\\
161	4.40966738014478e-05\\
162	4.40983872279222e-05\\
163	4.41001301916176e-05\\
164	4.41019032074326e-05\\
165	4.41037067991253e-05\\
166	4.41055414994383e-05\\
167	4.41074078502611e-05\\
168	4.4109306402788e-05\\
169	4.41112377176637e-05\\
170	4.41132023651569e-05\\
171	4.41152009253177e-05\\
172	4.4117233988145e-05\\
173	4.41193021537768e-05\\
174	4.41214060326552e-05\\
175	4.41235462457261e-05\\
176	4.41257234246133e-05\\
177	4.41279382118317e-05\\
178	4.41301912609804e-05\\
179	4.41324832369571e-05\\
180	4.41348148161607e-05\\
181	4.41371866867239e-05\\
182	4.41395995487314e-05\\
183	4.41420541144464e-05\\
184	4.41445511085555e-05\\
185	4.41470912683947e-05\\
186	4.41496753442055e-05\\
187	4.41523040993763e-05\\
188	4.4154978310692e-05\\
189	4.4157698768594e-05\\
190	4.41604662774285e-05\\
191	4.41632816557103e-05\\
192	4.41661457363904e-05\\
193	4.41690593671019e-05\\
194	4.41720234104597e-05\\
195	4.41750387442926e-05\\
196	4.41781062619502e-05\\
197	4.41812268725741e-05\\
198	4.41844015013758e-05\\
199	4.41876310899414e-05\\
200	4.41909165965234e-05\\
201	4.41942589963445e-05\\
202	4.41976592819037e-05\\
203	4.42011184633005e-05\\
204	4.42046375685309e-05\\
205	4.420821764385e-05\\
206	4.42118597540743e-05\\
207	4.42155649829314e-05\\
208	4.4219334433409e-05\\
209	4.42231692281075e-05\\
210	4.42270705095851e-05\\
211	4.42310394407289e-05\\
212	4.42350772051494e-05\\
213	4.4239185007517e-05\\
214	4.4243364073984e-05\\
215	4.42476156525674e-05\\
216	4.42519410135438e-05\\
217	4.42563414498614e-05\\
218	4.42608182775526e-05\\
219	4.42653728361795e-05\\
220	4.42700064892367e-05\\
221	4.42747206246027e-05\\
222	4.42795166550016e-05\\
223	4.42843960184389e-05\\
224	4.4289360178684e-05\\
225	4.42944106257363e-05\\
226	4.42995488763069e-05\\
227	4.43047764743297e-05\\
228	4.43100949914319e-05\\
229	4.43155060274723e-05\\
230	4.43210112110567e-05\\
231	4.43266122000598e-05\\
232	4.43323106821681e-05\\
233	4.43381083754438e-05\\
234	4.43440070288772e-05\\
235	4.43500084229461e-05\\
236	4.43561143702319e-05\\
237	4.43623267159743e-05\\
238	4.43686473387042e-05\\
239	4.43750781508495e-05\\
240	4.43816210993603e-05\\
241	4.43882781663539e-05\\
242	4.43950513697655e-05\\
243	4.44019427640066e-05\\
244	4.44089544406401e-05\\
245	4.44160885290751e-05\\
246	4.44233471972579e-05\\
247	4.44307326524078e-05\\
248	4.4438247141703e-05\\
249	4.44458929530593e-05\\
250	4.445367241587e-05\\
251	4.44615879017487e-05\\
252	4.4469641825366e-05\\
253	4.44778366451764e-05\\
254	4.44861748642826e-05\\
255	4.44946590312308e-05\\
256	4.45032917408492e-05\\
257	4.45120756351111e-05\\
258	4.45210134040005e-05\\
259	4.45301077863811e-05\\
260	4.45393615709164e-05\\
261	4.45487775969723e-05\\
262	4.45583587555343e-05\\
263	4.45681079901821e-05\\
264	4.45780282980174e-05\\
265	4.45881227306739e-05\\
266	4.45983943952837e-05\\
267	4.46088464555019e-05\\
268	4.46194821325393e-05\\
269	4.46303047061891e-05\\
270	4.46413175159017e-05\\
271	4.46525239618515e-05\\
272	4.46639275060486e-05\\
273	4.46755316734377e-05\\
274	4.4687340053019e-05\\
275	4.46993562990273e-05\\
276	4.47115841320587e-05\\
277	4.47240273402758e-05\\
278	4.47366897806022e-05\\
279	4.47495753799409e-05\\
280	4.47626881364002e-05\\
281	4.47760321205624e-05\\
282	4.47896114767346e-05\\
283	4.48034304242579e-05\\
284	4.48174932587922e-05\\
285	4.48318043536452e-05\\
286	4.4846368161126e-05\\
287	4.48611892138861e-05\\
288	4.48762721262993e-05\\
289	4.48916215958507e-05\\
290	4.49072424045564e-05\\
291	4.4923139420376e-05\\
292	4.49393175986563e-05\\
293	4.49557819835998e-05\\
294	4.49725377097322e-05\\
295	4.49895900033875e-05\\
296	4.50069441842246e-05\\
297	4.5024605666745e-05\\
298	4.50425799618239e-05\\
299	4.50608726782711e-05\\
300	4.50794895243927e-05\\
301	4.50984363095716e-05\\
302	4.51177189458621e-05\\
303	4.51373434495972e-05\\
304	4.51573159430145e-05\\
305	4.51776426558854e-05\\
306	4.51983299271643e-05\\
307	4.52193842066511e-05\\
308	4.52408120566615e-05\\
309	4.5262620153708e-05\\
310	4.52848152902061e-05\\
311	4.53074043761662e-05\\
312	4.53303944409352e-05\\
313	4.5353792634902e-05\\
314	4.53776062312668e-05\\
315	4.54018426277839e-05\\
316	4.54265093485236e-05\\
317	4.54516140456654e-05\\
318	4.54771645012805e-05\\
319	4.55031686291604e-05\\
320	4.55296344766043e-05\\
321	4.55565702262981e-05\\
322	4.55839841981426e-05\\
323	4.56118848511417e-05\\
324	4.56402807852951e-05\\
325	4.56691807435217e-05\\
326	4.56985936136069e-05\\
327	4.57285284301631e-05\\
328	4.57589943766556e-05\\
329	4.57900007874387e-05\\
330	4.58215571498273e-05\\
331	4.5853673106249e-05\\
332	4.5886358456399e-05\\
333	4.59196231594949e-05\\
334	4.59534773365649e-05\\
335	4.5987931272831e-05\\
336	4.60229954201649e-05\\
337	4.60586803996068e-05\\
338	4.60949970040387e-05\\
339	4.61319562009179e-05\\
340	4.61695691351726e-05\\
341	4.62078471321956e-05\\
342	4.62468017010272e-05\\
343	4.62864445377018e-05\\
344	4.63267875287434e-05\\
345	4.636784275491e-05\\
346	4.64096224951479e-05\\
347	4.64521392307696e-05\\
348	4.64954056499421e-05\\
349	4.65394346524429e-05\\
350	4.65842393547247e-05\\
351	4.66298330953589e-05\\
352	4.6676229440844e-05\\
353	4.67234421918126e-05\\
354	4.67714853896862e-05\\
355	4.68203733238134e-05\\
356	4.68701205390992e-05\\
357	4.69207418442048e-05\\
358	4.69722523203164e-05\\
359	4.70246673305368e-05\\
360	4.70780025299507e-05\\
361	4.71322738763862e-05\\
362	4.71874976419489e-05\\
363	4.72436904252928e-05\\
364	4.73008691647738e-05\\
365	4.73590511524702e-05\\
366	4.74182540490824e-05\\
367	4.74784958998508e-05\\
368	4.75397951514163e-05\\
369	4.76021706698024e-05\\
370	4.76656417594064e-05\\
371	4.77302281831851e-05\\
372	4.77959501839953e-05\\
373	4.78628285071801e-05\\
374	4.79308844243845e-05\\
375	4.80001397587826e-05\\
376	4.80706169115897e-05\\
377	4.81423388901324e-05\\
378	4.82153293374013e-05\\
379	4.82896125632264e-05\\
380	4.83652135772021e-05\\
381	4.84421581235076e-05\\
382	4.85204727176926e-05\\
383	4.86001846857429e-05\\
384	4.86813222055992e-05\\
385	4.87639143513844e-05\\
386	4.88479911407526e-05\\
387	4.89335835857468e-05\\
388	4.90207237476218e-05\\
389	4.91094447962921e-05\\
390	4.91997810750532e-05\\
391	4.92917681715663e-05\\
392	4.93854429961038e-05\\
393	4.94808438684652e-05\\
394	4.95780106151693e-05\\
395	4.96769846789563e-05\\
396	4.97778092429405e-05\\
397	4.98805293723594e-05\\
398	4.998519217755e-05\\
399	5.00918470028816e-05\\
400	5.02005456472717e-05\\
401	5.03113426207483e-05\\
402	5.04242954323043e-05\\
403	5.05394648778098e-05\\
404	5.06569152803294e-05\\
405	5.07767150054672e-05\\
406	5.08989369098363e-05\\
407	5.10236580461519e-05\\
408	5.11509599655566e-05\\
409	5.12809290270339e-05\\
410	5.14136567064048e-05\\
411	5.15492398934133e-05\\
412	5.16877811610643e-05\\
413	5.18293889925369e-05\\
414	5.19741779742716e-05\\
415	5.21222690497928e-05\\
416	5.22737900990155e-05\\
417	5.24288769461028e-05\\
418	5.25876710152473e-05\\
419	5.27503242028632e-05\\
420	5.29170007747489e-05\\
421	5.30878788850655e-05\\
422	5.32631523226434e-05\\
423	5.34430325233854e-05\\
424	5.36277508951813e-05\\
425	5.38175615106991e-05\\
426	5.40127442341355e-05\\
427	5.42136083623032e-05\\
428	5.44204968787681e-05\\
429	5.46337914461977e-05\\
430	5.48539183112384e-05\\
431	5.50813554073704e-05\\
432	5.53166412245854e-05\\
433	5.55603868648996e-05\\
434	5.5813295385314e-05\\
435	5.60762013032522e-05\\
436	5.63501725665906e-05\\
437	5.66368177749057e-05\\
438	5.69392898770911e-05\\
439	5.72657023122009e-05\\
440	5.84355178911509e-05\\
441	6.0223903010301e-05\\
442	6.20558151990788e-05\\
443	6.39327454540697e-05\\
444	6.58562509288798e-05\\
445	6.78279556692026e-05\\
446	6.98495503091889e-05\\
447	7.19227903181847e-05\\
448	7.40494922533085e-05\\
449	7.62315272988923e-05\\
450	7.84708111468448e-05\\
451	8.07692889812236e-05\\
452	8.31289139700059e-05\\
453	8.55516172569433e-05\\
454	8.80392670610282e-05\\
455	9.05936141816103e-05\\
456	9.32162219280515e-05\\
457	9.5908380818092e-05\\
458	9.86710135888545e-05\\
459	0.000101504578744033\\
460	0.000104408928371354\\
461	0.000107382675909847\\
462	0.00011041970912035\\
463	0.000113519029717233\\
464	0.000116670870402944\\
465	0.000119568613798337\\
466	0.000122522814577921\\
467	0.000125535048306553\\
468	0.00012860698507874\\
469	0.000131740397768809\\
470	0.000134937148490566\\
471	0.000138199232513218\\
472	0.000141528810076279\\
473	0.00014492816804621\\
474	0.000148399720417713\\
475	0.000151946006980245\\
476	0.000155569712616721\\
477	0.000159273826554056\\
478	0.000163061231685159\\
479	0.000166934937060552\\
480	0.000170898167325885\\
481	0.000174954388577371\\
482	0.000179107332599719\\
483	0.000183361026063574\\
484	0.000187719826578781\\
485	0.000192188467528132\\
486	0.000196772103888549\\
487	0.000201476365968081\\
488	0.000206307446192123\\
489	0.000211272186500523\\
490	0.000216378182725448\\
491	0.000221633909690366\\
492	0.000227048871674095\\
493	0.000232633784034798\\
494	0.000238400793284522\\
495	0.000244363744934429\\
496	0.000250538511430334\\
497	0.000256943397632224\\
498	0.000263599652017432\\
499	0.000270532138774456\\
500	0.000277770303731369\\
501	0.00028534980931451\\
502	0.000293316003529031\\
503	0.000301733845745669\\
504	0.000347776846392177\\
505	0.0004439702370206\\
506	0.000539909968176231\\
507	0.000638237080438817\\
508	0.000739062812214991\\
509	0.000842505501988271\\
510	0.000948694356085766\\
511	0.00105777517345738\\
512	0.00116989640825859\\
513	0.00128520700968993\\
514	0.00140386690765618\\
515	0.00152604785740911\\
516	0.00165193427891756\\
517	0.00178172405020567\\
518	0.00191562918892527\\
519	0.00205387631525134\\
520	0.00219670672770122\\
521	0.00234437582965595\\
522	0.00249715154984676\\
523	0.00265531160203086\\
524	0.00281914281186404\\
525	0.00298898627183365\\
526	0.00316545530963642\\
527	0.00334928644278656\\
528	0.00354136646310608\\
529	0.00374278540652288\\
530	0.00389889375450711\\
531	0.00397436563078082\\
532	0.00405378043675502\\
533	0.00413774408457233\\
534	0.00422701083325615\\
535	0.0043225256783029\\
536	0.00442547990037413\\
537	0.00453740363811477\\
538	0.0046601761979177\\
539	0.00479419705691268\\
540	0.00493058648476403\\
541	0.00506907493707201\\
542	0.00520928547315331\\
543	0.00535070135442138\\
544	0.00549262361212353\\
545	0.00563411381878136\\
546	0.00577391780484862\\
547	0.0059103595524853\\
548	0.00604132830660727\\
549	0.00616427293355928\\
550	0.00627667503699958\\
551	0.00637997363732135\\
552	0.00648310598117967\\
553	0.00658842845119338\\
554	0.00669578903008696\\
555	0.00680500553483106\\
556	0.0069158695101526\\
557	0.0070281552748073\\
558	0.00714163986669235\\
559	0.00725613745998112\\
560	0.00737152104285232\\
561	0.00748780541696548\\
562	0.0076051600309856\\
563	0.00772360689230023\\
564	0.0078424386681726\\
565	0.00795837118433625\\
566	0.00806973324852884\\
567	0.00817594310481345\\
568	0.00827640839751729\\
569	0.00837077167292279\\
570	0.00845915442916129\\
571	0.00854517850817255\\
572	0.00862900269321693\\
573	0.00871087087988246\\
574	0.00879069410162503\\
575	0.00886883861298283\\
576	0.00894512777368276\\
577	0.00901844010182401\\
578	0.00908859230578235\\
579	0.00915558355192523\\
580	0.00921941725626107\\
581	0.00928029525006657\\
582	0.00933793711813075\\
583	0.00939225237544419\\
584	0.0094441076439401\\
585	0.00949409858814331\\
586	0.00954199399216512\\
587	0.00958798891943662\\
588	0.00963262455293709\\
589	0.00967615823465495\\
590	0.00971874630341068\\
591	0.00976042545830353\\
592	0.00980111218114172\\
593	0.00984067816888698\\
594	0.00987891554157904\\
595	0.00991544794299161\\
596	0.00994949078513196\\
597	0.00997920912842892\\
598	0.0100000808076597\\
599	0\\
600	0\\
};
\addplot [color=mycolor10,solid,forget plot]
  table[row sep=crcr]{%
1	2.92945535614953e-05\\
2	2.92945543816799e-05\\
3	2.92945552156552e-05\\
4	2.92945560636566e-05\\
5	2.9294556925916e-05\\
6	2.92945578026772e-05\\
7	2.92945586941875e-05\\
8	2.92945596006854e-05\\
9	2.92945605224286e-05\\
10	2.92945614596711e-05\\
11	2.92945624126754e-05\\
12	2.92945633817058e-05\\
13	2.92945643670353e-05\\
14	2.92945653689331e-05\\
15	2.92945663876806e-05\\
16	2.92945674235643e-05\\
17	2.9294568476862e-05\\
18	2.92945695478791e-05\\
19	2.92945706369088e-05\\
20	2.92945717442546e-05\\
21	2.92945728702216e-05\\
22	2.92945740151253e-05\\
23	2.92945751792899e-05\\
24	2.92945763630322e-05\\
25	2.92945775666849e-05\\
26	2.92945787905821e-05\\
27	2.92945800350718e-05\\
28	2.9294581300488e-05\\
29	2.9294582587194e-05\\
30	2.92945838955427e-05\\
31	2.92945852259024e-05\\
32	2.92945865786431e-05\\
33	2.92945879541366e-05\\
34	2.92945893527784e-05\\
35	2.9294590774947e-05\\
36	2.92945922210432e-05\\
37	2.92945936914727e-05\\
38	2.92945951866431e-05\\
39	2.92945967069721e-05\\
40	2.92945982528895e-05\\
41	2.92945998248197e-05\\
42	2.92946014232009e-05\\
43	2.92946030484869e-05\\
44	2.92946047011258e-05\\
45	2.92946063815851e-05\\
46	2.92946080903267e-05\\
47	2.92946098278349e-05\\
48	2.92946115945939e-05\\
49	2.92946133910965e-05\\
50	2.9294615217844e-05\\
51	2.9294617075348e-05\\
52	2.9294618964137e-05\\
53	2.92946208847293e-05\\
54	2.92946228376672e-05\\
55	2.92946248234964e-05\\
56	2.92946268427742e-05\\
57	2.92946288960687e-05\\
58	2.9294630983951e-05\\
59	2.92946331070127e-05\\
60	2.9294635265842e-05\\
61	2.9294637461053e-05\\
62	2.92946396932558e-05\\
63	2.92946419630781e-05\\
64	2.92946442711609e-05\\
65	2.92946466181523e-05\\
66	2.9294649004705e-05\\
67	2.92946514315032e-05\\
68	2.92946538992184e-05\\
69	2.9294656408548e-05\\
70	2.92946589602031e-05\\
71	2.92946615548964e-05\\
72	2.92946641933663e-05\\
73	2.92946668763492e-05\\
74	2.9294669604604e-05\\
75	2.92946723789014e-05\\
76	2.92946752000275e-05\\
77	2.92946780687733e-05\\
78	2.9294680985954e-05\\
79	2.92946839523913e-05\\
80	2.9294686968926e-05\\
81	2.92946900364104e-05\\
82	2.92946931557143e-05\\
83	2.92946963277175e-05\\
84	2.92946995533236e-05\\
85	2.92947028334431e-05\\
86	2.92947061690106e-05\\
87	2.92947095609688e-05\\
88	2.92947130102846e-05\\
89	2.92947165179385e-05\\
90	2.92947200849192e-05\\
91	2.92947237122569e-05\\
92	2.92947274009675e-05\\
93	2.92947311521135e-05\\
94	2.92947349667588e-05\\
95	2.92947388459895e-05\\
96	2.92947427909175e-05\\
97	2.92947468026663e-05\\
98	2.92947508823817e-05\\
99	2.92947550312283e-05\\
100	2.92947592503981e-05\\
101	2.92947635410947e-05\\
102	2.92947679045494e-05\\
103	2.92947723420084e-05\\
104	2.92947768547506e-05\\
105	2.9294781444065e-05\\
106	2.92947861112749e-05\\
107	2.92947908577204e-05\\
108	2.92947956847656e-05\\
109	2.92948005937967e-05\\
110	2.92948055862306e-05\\
111	2.92948106635099e-05\\
112	2.92948158270907e-05\\
113	2.92948210784736e-05\\
114	2.92948264191674e-05\\
115	2.92948318507188e-05\\
116	2.92948373746964e-05\\
117	2.92948429927029e-05\\
118	2.929484870636e-05\\
119	2.92948545173301e-05\\
120	2.92948604272927e-05\\
121	2.929486643797e-05\\
122	2.92948725511063e-05\\
123	2.92948787684748e-05\\
124	2.92948850918882e-05\\
125	2.92948915231809e-05\\
126	2.92948980642338e-05\\
127	2.92949047169516e-05\\
128	2.92949114832745e-05\\
129	2.9294918365177e-05\\
130	2.92949253646696e-05\\
131	2.9294932483793e-05\\
132	2.92949397246379e-05\\
133	2.92949470893202e-05\\
134	2.92949545799952e-05\\
135	2.92949621988573e-05\\
136	2.92949699481401e-05\\
137	2.92949778301149e-05\\
138	2.92949858471041e-05\\
139	2.92949940014503e-05\\
140	2.92950022955527e-05\\
141	2.92950107318547e-05\\
142	2.92950193128304e-05\\
143	2.92950280410049e-05\\
144	2.9295036918953e-05\\
145	2.92950459492818e-05\\
146	2.92950551346563e-05\\
147	2.92950644777757e-05\\
148	2.92950739814005e-05\\
149	2.92950836483323e-05\\
150	2.92950934814167e-05\\
151	2.92951034835611e-05\\
152	2.92951136577117e-05\\
153	2.92951240068729e-05\\
154	2.92951345340952e-05\\
155	2.92951452424887e-05\\
156	2.92951561352162e-05\\
157	2.9295167215492e-05\\
158	2.9295178486588e-05\\
159	2.92951899518292e-05\\
160	2.92952016146019e-05\\
161	2.92952134783521e-05\\
162	2.92952255465786e-05\\
163	2.92952378228451e-05\\
164	2.92952503107784e-05\\
165	2.92952630140613e-05\\
166	2.92952759364434e-05\\
167	2.92952890817391e-05\\
168	2.92953024538273e-05\\
169	2.9295316056652e-05\\
170	2.92953298942269e-05\\
171	2.92953439706323e-05\\
172	2.92953582900236e-05\\
173	2.92953728566173e-05\\
174	2.9295387674719e-05\\
175	2.9295402748685e-05\\
176	2.92954180829691e-05\\
177	2.92954336820898e-05\\
178	2.92954495506492e-05\\
179	2.92954656933225e-05\\
180	2.9295482114867e-05\\
181	2.92954988201319e-05\\
182	2.929551581404e-05\\
183	2.9295533101604e-05\\
184	2.92955506879223e-05\\
185	2.92955685781921e-05\\
186	2.9295586777682e-05\\
187	2.92956052917715e-05\\
188	2.92956241259273e-05\\
189	2.92956432857078e-05\\
190	2.92956627767775e-05\\
191	2.92956826048979e-05\\
192	2.92957027759292e-05\\
193	2.92957232958378e-05\\
194	2.92957441706936e-05\\
195	2.92957654066726e-05\\
196	2.9295787010063e-05\\
197	2.92958089872607e-05\\
198	2.9295831344781e-05\\
199	2.9295854089248e-05\\
200	2.9295877227402e-05\\
201	2.92959007661059e-05\\
202	2.92959247123439e-05\\
203	2.92959490732244e-05\\
204	2.9295973855984e-05\\
205	2.92959990679749e-05\\
206	2.92960247166944e-05\\
207	2.92960508097677e-05\\
208	2.92960773549547e-05\\
209	2.92961043601533e-05\\
210	2.92961318333946e-05\\
211	2.92961597828715e-05\\
212	2.92961882169066e-05\\
213	2.92962171439691e-05\\
214	2.92962465726867e-05\\
215	2.92962765118424e-05\\
216	2.92963069703691e-05\\
217	2.92963379573585e-05\\
218	2.92963694820691e-05\\
219	2.92964015539233e-05\\
220	2.92964341825105e-05\\
221	2.92964673775942e-05\\
222	2.92965011491029e-05\\
223	2.92965355071583e-05\\
224	2.92965704620454e-05\\
225	2.92966060242505e-05\\
226	2.92966422044355e-05\\
227	2.92966790134552e-05\\
228	2.92967164623618e-05\\
229	2.92967545624022e-05\\
230	2.92967933250312e-05\\
231	2.92968327619032e-05\\
232	2.92968728848803e-05\\
233	2.92969137060451e-05\\
234	2.92969552376875e-05\\
235	2.92969974923267e-05\\
236	2.92970404827063e-05\\
237	2.92970842217885e-05\\
238	2.92971287227827e-05\\
239	2.92971739991318e-05\\
240	2.9297220064514e-05\\
241	2.929726693286e-05\\
242	2.92973146183493e-05\\
243	2.92973631354224e-05\\
244	2.92974124987685e-05\\
245	2.92974627233568e-05\\
246	2.92975138244137e-05\\
247	2.9297565817447e-05\\
248	2.92976187182459e-05\\
249	2.92976725428792e-05\\
250	2.92977273077072e-05\\
251	2.92977830293921e-05\\
252	2.92978397248877e-05\\
253	2.92978974114648e-05\\
254	2.92979561066978e-05\\
255	2.92980158284834e-05\\
256	2.92980765950388e-05\\
257	2.92981384249153e-05\\
258	2.9298201336997e-05\\
259	2.92982653505103e-05\\
260	2.92983304850316e-05\\
261	2.92983967604899e-05\\
262	2.92984641971723e-05\\
263	2.92985328157362e-05\\
264	2.92986026372137e-05\\
265	2.92986736830176e-05\\
266	2.92987459749473e-05\\
267	2.9298819535193e-05\\
268	2.92988943863556e-05\\
269	2.92989705514334e-05\\
270	2.92990480538528e-05\\
271	2.9299126917456e-05\\
272	2.92992071665133e-05\\
273	2.92992888257451e-05\\
274	2.92993719203067e-05\\
275	2.9299456475817e-05\\
276	2.92995425183521e-05\\
277	2.92996300744551e-05\\
278	2.92997191711586e-05\\
279	2.9299809835969e-05\\
280	2.92999020969046e-05\\
281	2.92999959824658e-05\\
282	2.9300091521687e-05\\
283	2.93001887441141e-05\\
284	2.93002876798213e-05\\
285	2.93003883594272e-05\\
286	2.93004908140973e-05\\
287	2.93005950755515e-05\\
288	2.9300701176086e-05\\
289	2.93008091485611e-05\\
290	2.93009190264309e-05\\
291	2.9301030843746e-05\\
292	2.93011446351572e-05\\
293	2.93012604359397e-05\\
294	2.93013782819802e-05\\
295	2.93014982098157e-05\\
296	2.93016202566153e-05\\
297	2.9301744460215e-05\\
298	2.93018708591107e-05\\
299	2.93019994924751e-05\\
300	2.93021304001698e-05\\
301	2.93022636227504e-05\\
302	2.93023992014888e-05\\
303	2.9302537178364e-05\\
304	2.93026775960921e-05\\
305	2.93028204981324e-05\\
306	2.93029659286895e-05\\
307	2.93031139327333e-05\\
308	2.93032645560063e-05\\
309	2.93034178450352e-05\\
310	2.93035738471465e-05\\
311	2.93037326104699e-05\\
312	2.9303894183955e-05\\
313	2.93040586173785e-05\\
314	2.93042259613679e-05\\
315	2.93043962673878e-05\\
316	2.93045695877811e-05\\
317	2.93047459757638e-05\\
318	2.93049254854329e-05\\
319	2.93051081717897e-05\\
320	2.93052940907474e-05\\
321	2.93054832991433e-05\\
322	2.93056758547494e-05\\
323	2.93058718162801e-05\\
324	2.93060712434172e-05\\
325	2.93062741968122e-05\\
326	2.93064807380972e-05\\
327	2.93066909299121e-05\\
328	2.93069048358975e-05\\
329	2.93071225207239e-05\\
330	2.93073440500969e-05\\
331	2.93075694907774e-05\\
332	2.93077989105905e-05\\
333	2.93080323784385e-05\\
334	2.93082699643223e-05\\
335	2.93085117393557e-05\\
336	2.93087577757766e-05\\
337	2.93090081469735e-05\\
338	2.93092629274877e-05\\
339	2.93095221930489e-05\\
340	2.93097860205837e-05\\
341	2.9310054488238e-05\\
342	2.93103276754058e-05\\
343	2.93106056627359e-05\\
344	2.93108885321678e-05\\
345	2.93111763669594e-05\\
346	2.93114692517119e-05\\
347	2.93117672723854e-05\\
348	2.9312070516354e-05\\
349	2.93123790724184e-05\\
350	2.93126930308493e-05\\
351	2.93130124834283e-05\\
352	2.93133375234869e-05\\
353	2.9313668245942e-05\\
354	2.93140047473477e-05\\
355	2.93143471259616e-05\\
356	2.93146954817753e-05\\
357	2.93150499165739e-05\\
358	2.93154105340187e-05\\
359	2.93157774396921e-05\\
360	2.93161507411839e-05\\
361	2.93165305481502e-05\\
362	2.93169169724178e-05\\
363	2.93173101280456e-05\\
364	2.9317710131437e-05\\
365	2.93181171014337e-05\\
366	2.93185311594234e-05\\
367	2.93189524294295e-05\\
368	2.93193810382672e-05\\
369	2.93198171156363e-05\\
370	2.93202607942825e-05\\
371	2.93207122101126e-05\\
372	2.93211715023564e-05\\
373	2.93216388137363e-05\\
374	2.93221142906134e-05\\
375	2.93225980831699e-05\\
376	2.93230903456036e-05\\
377	2.93235912363104e-05\\
378	2.93241009181008e-05\\
379	2.93246195584184e-05\\
380	2.93251473295577e-05\\
381	2.93256844089236e-05\\
382	2.9326230979287e-05\\
383	2.93267872290512e-05\\
384	2.93273533525669e-05\\
385	2.93279295504187e-05\\
386	2.9328516029799e-05\\
387	2.93291130048485e-05\\
388	2.93297206970745e-05\\
389	2.93303393357681e-05\\
390	2.93309691585025e-05\\
391	2.9331610411655e-05\\
392	2.93322633510051e-05\\
393	2.93329282424103e-05\\
394	2.93336053625593e-05\\
395	2.93342949998903e-05\\
396	2.93349974556152e-05\\
397	2.9335713045006e-05\\
398	2.93364420989207e-05\\
399	2.93371849655834e-05\\
400	2.93379420124521e-05\\
401	2.93387136274519e-05\\
402	2.93395002186751e-05\\
403	2.93403022133355e-05\\
404	2.93411200645513e-05\\
405	2.93419542592261e-05\\
406	2.93428053119486e-05\\
407	2.934367376675e-05\\
408	2.93445601987013e-05\\
409	2.93454652152106e-05\\
410	2.93463894567766e-05\\
411	2.93473335971921e-05\\
412	2.93482983437994e-05\\
413	2.93492844400196e-05\\
414	2.93502926749115e-05\\
415	2.93513239030429e-05\\
416	2.93523790561021e-05\\
417	2.9353459078188e-05\\
418	2.93545649797045e-05\\
419	2.93556978540131e-05\\
420	2.93568588877659e-05\\
421	2.93580493728175e-05\\
422	2.93592707199875e-05\\
423	2.93605244750389e-05\\
424	2.93618123373119e-05\\
425	2.93631361816644e-05\\
426	2.93644980847501e-05\\
427	2.93659003576537e-05\\
428	2.9367345589429e-05\\
429	2.93688367128651e-05\\
430	2.93703771223244e-05\\
431	2.93719709233627e-05\\
432	2.9373623527674e-05\\
433	2.93753431573309e-05\\
434	2.9377144704496e-05\\
435	2.93790594587929e-05\\
436	2.93811583610192e-05\\
437	2.93836014576275e-05\\
438	2.93867139163353e-05\\
439	2.93909497531595e-05\\
440	2.93958007444269e-05\\
441	2.94007619228078e-05\\
442	2.94058366656329e-05\\
443	2.94110284738933e-05\\
444	2.94163409704679e-05\\
445	2.94217778959219e-05\\
446	2.94273431011709e-05\\
447	2.94330405361638e-05\\
448	2.94388742336733e-05\\
449	2.9444848287572e-05\\
450	2.94509668262902e-05\\
451	2.94572339860714e-05\\
452	2.94636538989708e-05\\
453	2.94702307346454e-05\\
454	2.94769688861485e-05\\
455	2.94838735010782e-05\\
456	2.94909517498971e-05\\
457	2.9498215470952e-05\\
458	2.95056859428349e-05\\
459	2.95134012685814e-05\\
460	2.95214296479719e-05\\
461	2.95299328546882e-05\\
462	2.95396193279775e-05\\
463	2.95528378468826e-05\\
464	2.95803420062255e-05\\
465	2.99290350430518e-05\\
466	3.02843769720776e-05\\
467	3.06465718787811e-05\\
468	3.10158363472616e-05\\
469	3.13923999078648e-05\\
470	3.17765065674537e-05\\
471	3.21684150181169e-05\\
472	3.25683971770029e-05\\
473	3.29767380548121e-05\\
474	3.33937356864948e-05\\
475	3.38197018418285e-05\\
476	3.42549637893222e-05\\
477	3.46998568036385e-05\\
478	3.51547348349051e-05\\
479	3.56199747338011e-05\\
480	3.60959783294e-05\\
481	3.65831746955473e-05\\
482	3.70820227779158e-05\\
483	3.75930144581007e-05\\
484	3.8116678086596e-05\\
485	3.86535823538297e-05\\
486	3.92043407065236e-05\\
487	3.97696167441084e-05\\
488	4.03501301223883e-05\\
489	4.09466633790861e-05\\
490	4.15600698645113e-05\\
491	4.2191283006067e-05\\
492	4.28413271980445e-05\\
493	4.35113307034588e-05\\
494	4.42025411217915e-05\\
495	4.49163443200449e-05\\
496	4.565428853314e-05\\
497	4.64181174155613e-05\\
498	4.72098213982531e-05\\
499	4.80317319258254e-05\\
500	4.88867247602302e-05\\
501	4.97787109656221e-05\\
502	5.07138903592371e-05\\
503	5.17039252238376e-05\\
504	5.2773590491166e-05\\
505	5.62881008045062e-05\\
506	6.25293421522672e-05\\
507	6.89163493773037e-05\\
508	7.54565141051495e-05\\
509	8.21578150615983e-05\\
510	8.90290817305844e-05\\
511	9.60796379660444e-05\\
512	0.000103319320965559\\
513	0.000110758911572879\\
514	0.000118410245030727\\
515	0.000126286335168709\\
516	0.000134401513115256\\
517	0.000142771580550614\\
518	0.000151413973435762\\
519	0.0001603479176125\\
520	0.000169594506304022\\
521	0.00017917644791002\\
522	0.000189116588322435\\
523	0.000199432019150991\\
524	0.000209855106043124\\
525	0.000219994825185259\\
526	0.000230587882499338\\
527	0.000241688397010983\\
528	0.000253368759290531\\
529	0.000265734529715713\\
530	0.000333939744381258\\
531	0.000492521453495952\\
532	0.000656581911048302\\
533	0.000826416986542003\\
534	0.00100232466729434\\
535	0.00118458860074955\\
536	0.0013734527705768\\
537	0.0015652182277861\\
538	0.00175869370656908\\
539	0.00195912638989968\\
540	0.00216717307919366\\
541	0.00238359445256181\\
542	0.00260927533971752\\
543	0.00284524670979816\\
544	0.00309272268363313\\
545	0.00335319055891825\\
546	0.00362847072779979\\
547	0.00392077891259866\\
548	0.00423261029600988\\
549	0.00455457722029184\\
550	0.004705151070944\\
551	0.0048711544097649\\
552	0.00504296396722858\\
553	0.0052177543791052\\
554	0.00539513889588837\\
555	0.00557459356943668\\
556	0.00575542697577849\\
557	0.00593667348212239\\
558	0.00611708049242699\\
559	0.00629519343678757\\
560	0.00646918013562882\\
561	0.00663623102128118\\
562	0.00679234126261342\\
563	0.00694119143088869\\
564	0.00708906026376377\\
565	0.00723399063414785\\
566	0.00737470726420145\\
567	0.00751706186543774\\
568	0.00766049339092212\\
569	0.00780402939508268\\
570	0.00794621332763545\\
571	0.00808257029606174\\
572	0.00821274391959336\\
573	0.00833671964364272\\
574	0.00845440234533997\\
575	0.00856536624062254\\
576	0.00866961822458976\\
577	0.00877046315297453\\
578	0.00886789571017702\\
579	0.00896196964244534\\
580	0.00905281931413838\\
581	0.00913922335315378\\
582	0.00922137192580619\\
583	0.00929923394238518\\
584	0.00937160294626864\\
585	0.00943851828322894\\
586	0.00950083114789612\\
587	0.00955851175084126\\
588	0.00961256890708658\\
589	0.00966312229721793\\
590	0.00971055142633372\\
591	0.00975566068548307\\
592	0.00979862589472037\\
593	0.00983957018512293\\
594	0.00987853445099475\\
595	0.00991537019330016\\
596	0.00994949078513196\\
597	0.00997920912842892\\
598	0.0100000808076597\\
599	0\\
600	0\\
};
\addplot [color=mycolor11,solid,forget plot]
  table[row sep=crcr]{%
1	2.91334472184831e-05\\
2	2.91334472520906e-05\\
3	2.91334472862643e-05\\
4	2.91334473210126e-05\\
5	2.9133447356344e-05\\
6	2.91334473922705e-05\\
7	2.91334474288006e-05\\
8	2.91334474659445e-05\\
9	2.9133447503716e-05\\
10	2.913344754212e-05\\
11	2.91334475811703e-05\\
12	2.91334476208752e-05\\
13	2.91334476612503e-05\\
14	2.91334477023058e-05\\
15	2.91334477440519e-05\\
16	2.91334477864952e-05\\
17	2.91334478296565e-05\\
18	2.91334478735424e-05\\
19	2.91334479181666e-05\\
20	2.91334479635411e-05\\
21	2.91334480096812e-05\\
22	2.91334480565919e-05\\
23	2.91334481042973e-05\\
24	2.91334481528023e-05\\
25	2.91334482021241e-05\\
26	2.91334482522746e-05\\
27	2.91334483032673e-05\\
28	2.91334483551195e-05\\
29	2.91334484078447e-05\\
30	2.91334484614582e-05\\
31	2.91334485159703e-05\\
32	2.91334485713981e-05\\
33	2.91334486277619e-05\\
34	2.91334486850756e-05\\
35	2.91334487433508e-05\\
36	2.91334488026065e-05\\
37	2.91334488628597e-05\\
38	2.91334489241275e-05\\
39	2.91334489864251e-05\\
40	2.91334490497713e-05\\
41	2.91334491141832e-05\\
42	2.91334491796796e-05\\
43	2.91334492462792e-05\\
44	2.91334493139991e-05\\
45	2.91334493828596e-05\\
46	2.91334494528797e-05\\
47	2.91334495240762e-05\\
48	2.91334495964731e-05\\
49	2.91334496700892e-05\\
50	2.91334497449449e-05\\
51	2.91334498210589e-05\\
52	2.91334498984604e-05\\
53	2.91334499771593e-05\\
54	2.91334500571849e-05\\
55	2.91334501385609e-05\\
56	2.91334502213044e-05\\
57	2.91334503054444e-05\\
58	2.91334503910047e-05\\
59	2.91334504780024e-05\\
60	2.91334505664666e-05\\
61	2.91334506564226e-05\\
62	2.91334507478928e-05\\
63	2.91334508409079e-05\\
64	2.91334509354882e-05\\
65	2.91334510316645e-05\\
66	2.91334511294623e-05\\
67	2.9133451228909e-05\\
68	2.91334513300352e-05\\
69	2.91334514328631e-05\\
70	2.91334515374285e-05\\
71	2.9133451643757e-05\\
72	2.9133451751881e-05\\
73	2.91334518618259e-05\\
74	2.91334519736295e-05\\
75	2.91334520873189e-05\\
76	2.91334522029265e-05\\
77	2.91334523204882e-05\\
78	2.91334524400346e-05\\
79	2.91334525616016e-05\\
80	2.91334526852198e-05\\
81	2.91334528109285e-05\\
82	2.91334529387583e-05\\
83	2.91334530687502e-05\\
84	2.91334532009382e-05\\
85	2.91334533353617e-05\\
86	2.9133453472058e-05\\
87	2.91334536110665e-05\\
88	2.91334537524229e-05\\
89	2.91334538961749e-05\\
90	2.91334540423568e-05\\
91	2.9133454191011e-05\\
92	2.91334543421871e-05\\
93	2.91334544959174e-05\\
94	2.91334546522532e-05\\
95	2.91334548112369e-05\\
96	2.91334549729113e-05\\
97	2.91334551373258e-05\\
98	2.91334553045281e-05\\
99	2.91334554745644e-05\\
100	2.91334556474822e-05\\
101	2.91334558233345e-05\\
102	2.91334560021673e-05\\
103	2.91334561840405e-05\\
104	2.91334563689929e-05\\
105	2.91334565570861e-05\\
106	2.91334567483748e-05\\
107	2.91334569429116e-05\\
108	2.91334571407529e-05\\
109	2.91334573419568e-05\\
110	2.91334575465758e-05\\
111	2.91334577546802e-05\\
112	2.9133457966321e-05\\
113	2.91334581815594e-05\\
114	2.91334584004622e-05\\
115	2.91334586230889e-05\\
116	2.91334588495078e-05\\
117	2.91334590797802e-05\\
118	2.9133459313976e-05\\
119	2.91334595521635e-05\\
120	2.91334597944092e-05\\
121	2.91334600407811e-05\\
122	2.91334602913596e-05\\
123	2.91334605462093e-05\\
124	2.91334608054088e-05\\
125	2.91334610690347e-05\\
126	2.91334613371587e-05\\
127	2.91334616098643e-05\\
128	2.91334618872248e-05\\
129	2.91334621693289e-05\\
130	2.91334624562534e-05\\
131	2.91334627480851e-05\\
132	2.91334630449094e-05\\
133	2.91334633468115e-05\\
134	2.913346365388e-05\\
135	2.91334639662036e-05\\
136	2.91334642838814e-05\\
137	2.91334646069983e-05\\
138	2.91334649356535e-05\\
139	2.91334652699388e-05\\
140	2.91334656099585e-05\\
141	2.91334659558096e-05\\
142	2.91334663075928e-05\\
143	2.91334666654154e-05\\
144	2.91334670293816e-05\\
145	2.91334673995952e-05\\
146	2.91334677761671e-05\\
147	2.91334681592134e-05\\
148	2.91334685488362e-05\\
149	2.91334689451601e-05\\
150	2.91334693482976e-05\\
151	2.91334697583697e-05\\
152	2.91334701754959e-05\\
153	2.9133470599797e-05\\
154	2.91334710314046e-05\\
155	2.91334714704379e-05\\
156	2.91334719170333e-05\\
157	2.91334723713222e-05\\
158	2.91334728334358e-05\\
159	2.91334733035104e-05\\
160	2.91334737816913e-05\\
161	2.91334742681094e-05\\
162	2.91334747629149e-05\\
163	2.91334752662493e-05\\
164	2.91334757782662e-05\\
165	2.91334762991137e-05\\
166	2.91334768289489e-05\\
167	2.91334773679235e-05\\
168	2.91334779162011e-05\\
169	2.91334784739351e-05\\
170	2.91334790412978e-05\\
171	2.91334796184582e-05\\
172	2.91334802055814e-05\\
173	2.91334808028397e-05\\
174	2.91334814104123e-05\\
175	2.9133482028478e-05\\
176	2.91334826572142e-05\\
177	2.91334832968154e-05\\
178	2.91334839474639e-05\\
179	2.91334846093525e-05\\
180	2.91334852826773e-05\\
181	2.91334859676393e-05\\
182	2.91334866644314e-05\\
183	2.913348737327e-05\\
184	2.9133488094358e-05\\
185	2.91334888279087e-05\\
186	2.91334895741402e-05\\
187	2.91334903332708e-05\\
188	2.91334911055273e-05\\
189	2.91334918911381e-05\\
190	2.91334926903336e-05\\
191	2.91334935033506e-05\\
192	2.91334943304313e-05\\
193	2.9133495171816e-05\\
194	2.91334960277573e-05\\
195	2.91334968985091e-05\\
196	2.91334977843254e-05\\
197	2.91334986854759e-05\\
198	2.91334996022142e-05\\
199	2.91335005348272e-05\\
200	2.91335014835857e-05\\
201	2.91335024487678e-05\\
202	2.91335034306616e-05\\
203	2.91335044295639e-05\\
204	2.91335054457661e-05\\
205	2.91335064795683e-05\\
206	2.91335075312845e-05\\
207	2.9133508601223e-05\\
208	2.91335096897011e-05\\
209	2.91335107970462e-05\\
210	2.91335119235891e-05\\
211	2.91335130696587e-05\\
212	2.91335142356048e-05\\
213	2.91335154217683e-05\\
214	2.91335166285073e-05\\
215	2.91335178561816e-05\\
216	2.91335191051542e-05\\
217	2.91335203758022e-05\\
218	2.91335216685022e-05\\
219	2.91335229836431e-05\\
220	2.91335243216222e-05\\
221	2.91335256828316e-05\\
222	2.91335270676823e-05\\
223	2.9133528476592e-05\\
224	2.91335299099766e-05\\
225	2.91335313682729e-05\\
226	2.91335328519156e-05\\
227	2.91335343613514e-05\\
228	2.91335358970355e-05\\
229	2.91335374594215e-05\\
230	2.91335390489869e-05\\
231	2.91335406662074e-05\\
232	2.91335423115654e-05\\
233	2.91335439855658e-05\\
234	2.91335456887082e-05\\
235	2.9133547421504e-05\\
236	2.91335491844784e-05\\
237	2.91335509781634e-05\\
238	2.9133552803103e-05\\
239	2.91335546598427e-05\\
240	2.91335565489505e-05\\
241	2.9133558470994e-05\\
242	2.91335604265582e-05\\
243	2.91335624162295e-05\\
244	2.91335644406167e-05\\
245	2.91335665003303e-05\\
246	2.91335685960029e-05\\
247	2.91335707282583e-05\\
248	2.91335728977551e-05\\
249	2.9133575105146e-05\\
250	2.91335773511046e-05\\
251	2.91335796363182e-05\\
252	2.91335819614755e-05\\
253	2.91335843272893e-05\\
254	2.91335867344808e-05\\
255	2.91335891837799e-05\\
256	2.91335916759419e-05\\
257	2.91335942117255e-05\\
258	2.91335967919033e-05\\
259	2.91335994172594e-05\\
260	2.9133602088609e-05\\
261	2.91336048067621e-05\\
262	2.91336075725506e-05\\
263	2.91336103868254e-05\\
264	2.91336132504459e-05\\
265	2.91336161642954e-05\\
266	2.91336191292639e-05\\
267	2.91336221462619e-05\\
268	2.91336252162205e-05\\
269	2.91336283400808e-05\\
270	2.91336315188045e-05\\
271	2.91336347533669e-05\\
272	2.9133638044769e-05\\
273	2.91336413940219e-05\\
274	2.91336448021588e-05\\
275	2.91336482702249e-05\\
276	2.91336517992998e-05\\
277	2.91336553904608e-05\\
278	2.91336590448266e-05\\
279	2.91336627635208e-05\\
280	2.91336665476945e-05\\
281	2.91336703985173e-05\\
282	2.91336743171742e-05\\
283	2.91336783048827e-05\\
284	2.91336823628756e-05\\
285	2.91336864924043e-05\\
286	2.9133690694753e-05\\
287	2.91336949712174e-05\\
288	2.91336993231241e-05\\
289	2.91337037518219e-05\\
290	2.91337082586765e-05\\
291	2.91337128450861e-05\\
292	2.91337175124763e-05\\
293	2.9133722262286e-05\\
294	2.91337270959936e-05\\
295	2.91337320150912e-05\\
296	2.91337370211012e-05\\
297	2.91337421155772e-05\\
298	2.9133747300098e-05\\
299	2.91337525762629e-05\\
300	2.91337579457107e-05\\
301	2.91337634101021e-05\\
302	2.91337689711233e-05\\
303	2.91337746304966e-05\\
304	2.91337803899784e-05\\
305	2.91337862513348e-05\\
306	2.91337922163906e-05\\
307	2.91337982869735e-05\\
308	2.9133804464966e-05\\
309	2.91338107522692e-05\\
310	2.91338171508202e-05\\
311	2.91338236625935e-05\\
312	2.91338302895839e-05\\
313	2.91338370338343e-05\\
314	2.91338438974199e-05\\
315	2.91338508824376e-05\\
316	2.91338579910345e-05\\
317	2.91338652253861e-05\\
318	2.91338725877057e-05\\
319	2.91338800802388e-05\\
320	2.9133887705279e-05\\
321	2.9133895465145e-05\\
322	2.91339033621984e-05\\
323	2.91339113988435e-05\\
324	2.91339195775152e-05\\
325	2.91339279006959e-05\\
326	2.91339363708991e-05\\
327	2.9133944990691e-05\\
328	2.91339537626666e-05\\
329	2.91339626894724e-05\\
330	2.91339717737905e-05\\
331	2.91339810183437e-05\\
332	2.91339904259082e-05\\
333	2.91339999992955e-05\\
334	2.91340097413601e-05\\
335	2.91340196550124e-05\\
336	2.91340297432007e-05\\
337	2.91340400089174e-05\\
338	2.91340504552095e-05\\
339	2.91340610851683e-05\\
340	2.91340719019347e-05\\
341	2.91340829086988e-05\\
342	2.91340941086987e-05\\
343	2.91341055052354e-05\\
344	2.91341171016458e-05\\
345	2.9134128901335e-05\\
346	2.91341409077557e-05\\
347	2.91341531244189e-05\\
348	2.91341655549002e-05\\
349	2.91341782028158e-05\\
350	2.91341910718644e-05\\
351	2.9134204165792e-05\\
352	2.91342174884146e-05\\
353	2.91342310436164e-05\\
354	2.91342448353482e-05\\
355	2.91342588676272e-05\\
356	2.91342731445455e-05\\
357	2.91342876702722e-05\\
358	2.91343024490582e-05\\
359	2.91343174852258e-05\\
360	2.91343327831878e-05\\
361	2.91343483474493e-05\\
362	2.913436418259e-05\\
363	2.91343802933024e-05\\
364	2.91343966843712e-05\\
365	2.91344133606796e-05\\
366	2.91344303272254e-05\\
367	2.91344475891188e-05\\
368	2.91344651515774e-05\\
369	2.91344830199588e-05\\
370	2.91345011997397e-05\\
371	2.913451969653e-05\\
372	2.91345385160844e-05\\
373	2.91345576643111e-05\\
374	2.91345771472651e-05\\
375	2.91345969711735e-05\\
376	2.91346171424253e-05\\
377	2.91346376675954e-05\\
378	2.91346585534482e-05\\
379	2.91346798069404e-05\\
380	2.91347014352422e-05\\
381	2.91347234457453e-05\\
382	2.91347458460562e-05\\
383	2.91347686440425e-05\\
384	2.91347918478171e-05\\
385	2.91348154657624e-05\\
386	2.91348395065508e-05\\
387	2.91348639791513e-05\\
388	2.91348888928534e-05\\
389	2.91349142572895e-05\\
390	2.91349400824483e-05\\
391	2.9134966378716e-05\\
392	2.91349931568841e-05\\
393	2.91350204282018e-05\\
394	2.91350482044073e-05\\
395	2.91350764977726e-05\\
396	2.91351053211633e-05\\
397	2.91351346880961e-05\\
398	2.9135164612811e-05\\
399	2.91351951103407e-05\\
400	2.91352261965314e-05\\
401	2.91352578880393e-05\\
402	2.91352902023712e-05\\
403	2.91353231581618e-05\\
404	2.91353567753325e-05\\
405	2.91353910749353e-05\\
406	2.91354260792246e-05\\
407	2.91354618117043e-05\\
408	2.91354982971726e-05\\
409	2.91355355617492e-05\\
410	2.91355736328995e-05\\
411	2.91356125394987e-05\\
412	2.91356523120719e-05\\
413	2.91356929832313e-05\\
414	2.9135734588264e-05\\
415	2.91357771650126e-05\\
416	2.91358207524772e-05\\
417	2.91358653925553e-05\\
418	2.91359111306641e-05\\
419	2.91359580161722e-05\\
420	2.91360061029402e-05\\
421	2.91360554498796e-05\\
422	2.91361061216834e-05\\
423	2.9136158189641e-05\\
424	2.91362117326519e-05\\
425	2.91362668385246e-05\\
426	2.91363236057962e-05\\
427	2.91363821466549e-05\\
428	2.91364425923217e-05\\
429	2.9136505104279e-05\\
430	2.91365698993266e-05\\
431	2.91366373069443e-05\\
432	2.9136707899165e-05\\
433	2.91367827730856e-05\\
434	2.91368641195576e-05\\
435	2.91369562113195e-05\\
436	2.91370666036468e-05\\
437	2.91372058495617e-05\\
438	2.91373803134786e-05\\
439	2.91375723970049e-05\\
440	2.91377688369308e-05\\
441	2.91379697660288e-05\\
442	2.91381753217225e-05\\
443	2.9138385645974e-05\\
444	2.91386008850432e-05\\
445	2.91388211891248e-05\\
446	2.91390467118279e-05\\
447	2.91392776094978e-05\\
448	2.91395140404172e-05\\
449	2.91397561641219e-05\\
450	2.91400041414482e-05\\
451	2.9140258136787e-05\\
452	2.91405183261365e-05\\
453	2.91407849187507e-05\\
454	2.9141058209356e-05\\
455	2.9141338697083e-05\\
456	2.9141627350729e-05\\
457	2.91419262117151e-05\\
458	2.91422398577735e-05\\
459	2.91425792894266e-05\\
460	2.9142972800045e-05\\
461	2.91434944505844e-05\\
462	2.91443157887165e-05\\
463	2.91457761461237e-05\\
464	2.91482686575031e-05\\
465	2.91508067663754e-05\\
466	2.9153391835831e-05\\
467	2.91560253011234e-05\\
468	2.91587086620527e-05\\
469	2.9161443507152e-05\\
470	2.91642315204465e-05\\
471	2.91670744554511e-05\\
472	2.91699741378013e-05\\
473	2.91729324882147e-05\\
474	2.91759515755792e-05\\
475	2.91790336454352e-05\\
476	2.91821809213613e-05\\
477	2.91853957310866e-05\\
478	2.91886805543982e-05\\
479	2.91920380384319e-05\\
480	2.91954710134097e-05\\
481	2.9198982511771e-05\\
482	2.92025757912387e-05\\
483	2.92062543601847e-05\\
484	2.92100220019461e-05\\
485	2.92138828039964e-05\\
486	2.92178411996482e-05\\
487	2.92219020110354e-05\\
488	2.92260704993232e-05\\
489	2.92303524241205e-05\\
490	2.92347541149657e-05\\
491	2.9239282560323e-05\\
492	2.92439455252145e-05\\
493	2.92487517231867e-05\\
494	2.92537111047442e-05\\
495	2.92588354168238e-05\\
496	2.92641394176664e-05\\
497	2.92696436914666e-05\\
498	2.92753813222656e-05\\
499	2.92814136014094e-05\\
500	2.92878657787034e-05\\
501	2.92950032124751e-05\\
502	2.93033737648888e-05\\
503	2.93140034679327e-05\\
504	2.93283909056348e-05\\
505	2.9345367043383e-05\\
506	2.93626498691991e-05\\
507	2.93802509151344e-05\\
508	2.93981847810798e-05\\
509	2.94164749245672e-05\\
510	2.94351441280634e-05\\
511	2.94542112261742e-05\\
512	2.94736968961389e-05\\
513	2.94936239273996e-05\\
514	2.951401764098e-05\\
515	2.95349067107322e-05\\
516	2.95563251451471e-05\\
517	2.95783177650537e-05\\
518	2.96009565176345e-05\\
519	2.96243911775211e-05\\
520	2.9649011512442e-05\\
521	2.96759779977796e-05\\
522	2.97089936828976e-05\\
523	2.97603259332652e-05\\
524	3.01285531565058e-05\\
525	3.12336256340762e-05\\
526	3.23730462087472e-05\\
527	3.35499626951945e-05\\
528	3.4768683559536e-05\\
529	3.60348198777282e-05\\
530	3.73506279021683e-05\\
531	3.87155128287655e-05\\
532	4.01347889526842e-05\\
533	4.16152715378144e-05\\
534	4.31670314793173e-05\\
535	4.48095944567131e-05\\
536	4.65884182561517e-05\\
537	5.238172228821e-05\\
538	6.35131371367965e-05\\
539	7.50081919967123e-05\\
540	8.68984978026443e-05\\
541	9.92200149655045e-05\\
542	0.000112013569866344\\
543	0.000125325496420289\\
544	0.000139209858182771\\
545	0.000153730567560828\\
546	0.000168965358143605\\
547	0.000185007807634933\\
548	0.000201987784704643\\
549	0.000232234449847391\\
550	0.00045732018375804\\
551	0.000691149294509253\\
552	0.000934473903749796\\
553	0.00118822270377315\\
554	0.00145344976012476\\
555	0.00173135515391889\\
556	0.00202179789943843\\
557	0.00232772136194283\\
558	0.0026512470911888\\
559	0.00299449635753595\\
560	0.00334972642406333\\
561	0.00372633025747027\\
562	0.00413198016992457\\
563	0.00456287316792553\\
564	0.00501287965838903\\
565	0.00548417949755252\\
566	0.00570681733731302\\
567	0.00592787416711455\\
568	0.00615074454069715\\
569	0.00637400287685103\\
570	0.00659595464972563\\
571	0.00681413234326367\\
572	0.00702500105593927\\
573	0.00722376998191054\\
574	0.00741972597966123\\
575	0.00761675200170958\\
576	0.00781249173981442\\
577	0.00800129511961784\\
578	0.00818080640978489\\
579	0.00834848236018121\\
580	0.0085016067060865\\
581	0.00864037295885419\\
582	0.00877109714693535\\
583	0.00889664174668058\\
584	0.00901861055231992\\
585	0.00913624217630486\\
586	0.00924802196241273\\
587	0.00935341826152304\\
588	0.00945071973419065\\
589	0.00953949586246901\\
590	0.00961959916802822\\
591	0.00969154209345007\\
592	0.00975628023629958\\
593	0.00981394990186687\\
594	0.00986501353431486\\
595	0.00990966266908178\\
596	0.0099479910848602\\
597	0.00997920912842892\\
598	0.0100000808076597\\
599	0\\
600	0\\
};
\addplot [color=mycolor12,solid,forget plot]
  table[row sep=crcr]{%
1	2.81337324669755e-05\\
2	2.81337324683635e-05\\
3	2.81337324697741e-05\\
4	2.81337324712078e-05\\
5	2.81337324726679e-05\\
6	2.81337324741477e-05\\
7	2.81337324756568e-05\\
8	2.81337324771919e-05\\
9	2.81337324787505e-05\\
10	2.81337324803363e-05\\
11	2.81337324819479e-05\\
12	2.81337324835865e-05\\
13	2.81337324852522e-05\\
14	2.81337324869471e-05\\
15	2.81337324886679e-05\\
16	2.81337324904209e-05\\
17	2.81337324922048e-05\\
18	2.81337324940158e-05\\
19	2.81337324958554e-05\\
20	2.81337324977294e-05\\
21	2.81337324996344e-05\\
22	2.81337325015715e-05\\
23	2.81337325035395e-05\\
24	2.81337325055414e-05\\
25	2.81337325075758e-05\\
26	2.81337325096475e-05\\
27	2.81337325117536e-05\\
28	2.81337325138934e-05\\
29	2.81337325160698e-05\\
30	2.81337325182811e-05\\
31	2.81337325205306e-05\\
32	2.81337325228186e-05\\
33	2.81337325251482e-05\\
34	2.81337325275132e-05\\
35	2.81337325299159e-05\\
36	2.81337325323632e-05\\
37	2.81337325348476e-05\\
38	2.81337325373782e-05\\
39	2.81337325399516e-05\\
40	2.81337325425655e-05\\
41	2.81337325452239e-05\\
42	2.81337325479285e-05\\
43	2.81337325506759e-05\\
44	2.81337325534723e-05\\
45	2.81337325563149e-05\\
46	2.81337325592036e-05\\
47	2.81337325621453e-05\\
48	2.81337325651308e-05\\
49	2.81337325681716e-05\\
50	2.81337325712614e-05\\
51	2.81337325744024e-05\\
52	2.81337325775963e-05\\
53	2.81337325808465e-05\\
54	2.81337325841497e-05\\
55	2.81337325875052e-05\\
56	2.81337325909244e-05\\
57	2.81337325943965e-05\\
58	2.81337325979277e-05\\
59	2.81337326015175e-05\\
60	2.81337326051716e-05\\
61	2.81337326088841e-05\\
62	2.81337326126597e-05\\
63	2.81337326164995e-05\\
64	2.81337326204029e-05\\
65	2.81337326243745e-05\\
66	2.81337326284109e-05\\
67	2.81337326325136e-05\\
68	2.81337326366896e-05\\
69	2.81337326409343e-05\\
70	2.81337326452517e-05\\
71	2.81337326496393e-05\\
72	2.81337326541019e-05\\
73	2.81337326586429e-05\\
74	2.81337326632592e-05\\
75	2.81337326679521e-05\\
76	2.81337326727233e-05\\
77	2.81337326775761e-05\\
78	2.81337326825112e-05\\
79	2.81337326875279e-05\\
80	2.81337326926336e-05\\
81	2.81337326978209e-05\\
82	2.81337327031006e-05\\
83	2.81337327084652e-05\\
84	2.81337327139234e-05\\
85	2.81337327194706e-05\\
86	2.81337327251151e-05\\
87	2.81337327308532e-05\\
88	2.81337327366869e-05\\
89	2.81337327426261e-05\\
90	2.81337327486591e-05\\
91	2.81337327547947e-05\\
92	2.81337327610361e-05\\
93	2.81337327673829e-05\\
94	2.81337327738355e-05\\
95	2.81337327803991e-05\\
96	2.81337327870752e-05\\
97	2.81337327938601e-05\\
98	2.81337328007666e-05\\
99	2.81337328077851e-05\\
100	2.81337328149236e-05\\
101	2.81337328221826e-05\\
102	2.81337328295677e-05\\
103	2.81337328370744e-05\\
104	2.81337328447107e-05\\
105	2.81337328524759e-05\\
106	2.81337328603746e-05\\
107	2.81337328684085e-05\\
108	2.81337328765746e-05\\
109	2.81337328848809e-05\\
110	2.81337328933303e-05\\
111	2.81337329019233e-05\\
112	2.81337329106605e-05\\
113	2.81337329195468e-05\\
114	2.81337329285846e-05\\
115	2.81337329377756e-05\\
116	2.81337329471255e-05\\
117	2.8133732956634e-05\\
118	2.81337329663026e-05\\
119	2.81337329761389e-05\\
120	2.81337329861413e-05\\
121	2.81337329963115e-05\\
122	2.81337330066626e-05\\
123	2.81337330171819e-05\\
124	2.81337330278843e-05\\
125	2.81337330387702e-05\\
126	2.81337330498426e-05\\
127	2.81337330611041e-05\\
128	2.81337330725582e-05\\
129	2.81337330842049e-05\\
130	2.81337330960532e-05\\
131	2.81337331081037e-05\\
132	2.81337331203613e-05\\
133	2.81337331328245e-05\\
134	2.81337331455085e-05\\
135	2.81337331584053e-05\\
136	2.81337331715251e-05\\
137	2.81337331848673e-05\\
138	2.81337331984375e-05\\
139	2.81337332122466e-05\\
140	2.81337332262842e-05\\
141	2.81337332405691e-05\\
142	2.81337332550938e-05\\
143	2.81337332698736e-05\\
144	2.81337332849035e-05\\
145	2.81337333001946e-05\\
146	2.81337333157442e-05\\
147	2.81337333315641e-05\\
148	2.81337333476548e-05\\
149	2.81337333640214e-05\\
150	2.81337333806696e-05\\
151	2.81337333976061e-05\\
152	2.81337334148326e-05\\
153	2.81337334323559e-05\\
154	2.81337334501821e-05\\
155	2.81337334683141e-05\\
156	2.81337334867598e-05\\
157	2.81337335055207e-05\\
158	2.8133733524606e-05\\
159	2.81337335440219e-05\\
160	2.813373356377e-05\\
161	2.81337335838599e-05\\
162	2.8133733604299e-05\\
163	2.81337336250854e-05\\
164	2.81337336462305e-05\\
165	2.8133733667744e-05\\
166	2.81337336896274e-05\\
167	2.81337337118909e-05\\
168	2.81337337345333e-05\\
169	2.81337337575694e-05\\
170	2.8133733781003e-05\\
171	2.81337338048421e-05\\
172	2.81337338290939e-05\\
173	2.81337338537602e-05\\
174	2.81337338788549e-05\\
175	2.81337339043839e-05\\
176	2.81337339303497e-05\\
177	2.8133733956769e-05\\
178	2.81337339836437e-05\\
179	2.8133734010982e-05\\
180	2.81337340387926e-05\\
181	2.8133734067083e-05\\
182	2.81337340958621e-05\\
183	2.81337341251397e-05\\
184	2.81337341549234e-05\\
185	2.81337341852219e-05\\
186	2.81337342160457e-05\\
187	2.81337342474005e-05\\
188	2.8133734279297e-05\\
189	2.81337343117442e-05\\
190	2.81337343447541e-05\\
191	2.81337343783371e-05\\
192	2.81337344125002e-05\\
193	2.81337344472528e-05\\
194	2.81337344826062e-05\\
195	2.81337345185728e-05\\
196	2.81337345551606e-05\\
197	2.8133734592383e-05\\
198	2.81337346302495e-05\\
199	2.8133734668771e-05\\
200	2.81337347079592e-05\\
201	2.81337347478259e-05\\
202	2.81337347883835e-05\\
203	2.81337348296434e-05\\
204	2.81337348716195e-05\\
205	2.81337349143226e-05\\
206	2.81337349577634e-05\\
207	2.81337350019581e-05\\
208	2.81337350469187e-05\\
209	2.81337350926603e-05\\
210	2.81337351391925e-05\\
211	2.81337351865339e-05\\
212	2.81337352346929e-05\\
213	2.8133735283688e-05\\
214	2.81337353335375e-05\\
215	2.81337353842462e-05\\
216	2.81337354358385e-05\\
217	2.81337354883256e-05\\
218	2.81337355417271e-05\\
219	2.81337355960493e-05\\
220	2.81337356513198e-05\\
221	2.81337357075464e-05\\
222	2.81337357647505e-05\\
223	2.81337358229525e-05\\
224	2.81337358821619e-05\\
225	2.81337359424029e-05\\
226	2.8133736003689e-05\\
227	2.81337360660451e-05\\
228	2.81337361294834e-05\\
229	2.81337361940214e-05\\
230	2.81337362596857e-05\\
231	2.81337363264936e-05\\
232	2.8133736394461e-05\\
233	2.81337364636171e-05\\
234	2.81337365339737e-05\\
235	2.81337366055545e-05\\
236	2.81337366783855e-05\\
237	2.81337367524816e-05\\
238	2.81337368278724e-05\\
239	2.81337369045758e-05\\
240	2.81337369826165e-05\\
241	2.81337370620194e-05\\
242	2.8133737142807e-05\\
243	2.81337372250029e-05\\
244	2.81337373086365e-05\\
245	2.81337373937267e-05\\
246	2.81337374803053e-05\\
247	2.81337375683947e-05\\
248	2.81337376580219e-05\\
249	2.81337377492134e-05\\
250	2.81337378420042e-05\\
251	2.81337379364133e-05\\
252	2.81337380324746e-05\\
253	2.81337381302156e-05\\
254	2.81337382296673e-05\\
255	2.81337383308561e-05\\
256	2.81337384338223e-05\\
257	2.8133738538584e-05\\
258	2.81337386451855e-05\\
259	2.81337387536525e-05\\
260	2.81337388640195e-05\\
261	2.81337389763213e-05\\
262	2.81337390905916e-05\\
263	2.81337392068667e-05\\
264	2.81337393251802e-05\\
265	2.81337394455704e-05\\
266	2.81337395680734e-05\\
267	2.81337396927264e-05\\
268	2.81337398195658e-05\\
269	2.81337399486358e-05\\
270	2.81337400799734e-05\\
271	2.81337402136168e-05\\
272	2.81337403496118e-05\\
273	2.81337404879955e-05\\
274	2.81337406288129e-05\\
275	2.8133740772108e-05\\
276	2.81337409179265e-05\\
277	2.81337410663047e-05\\
278	2.81337412173025e-05\\
279	2.81337413709549e-05\\
280	2.81337415273157e-05\\
281	2.81337416864284e-05\\
282	2.81337418483448e-05\\
283	2.81337420131174e-05\\
284	2.81337421807921e-05\\
285	2.81337423514255e-05\\
286	2.81337425250669e-05\\
287	2.81337427017727e-05\\
288	2.81337428815959e-05\\
289	2.81337430645888e-05\\
290	2.81337432508168e-05\\
291	2.81337434403305e-05\\
292	2.81337436331901e-05\\
293	2.81337438294605e-05\\
294	2.81337440291923e-05\\
295	2.81337442324553e-05\\
296	2.81337444393081e-05\\
297	2.81337446498206e-05\\
298	2.81337448640484e-05\\
299	2.81337450820653e-05\\
300	2.81337453039378e-05\\
301	2.81337455297346e-05\\
302	2.8133745759521e-05\\
303	2.81337459933731e-05\\
304	2.81337462313617e-05\\
305	2.81337464735578e-05\\
306	2.81337467200386e-05\\
307	2.81337469708812e-05\\
308	2.81337472261616e-05\\
309	2.81337474859554e-05\\
310	2.81337477503463e-05\\
311	2.81337480194149e-05\\
312	2.81337482932429e-05\\
313	2.81337485719119e-05\\
314	2.81337488555143e-05\\
315	2.81337491441317e-05\\
316	2.81337494378531e-05\\
317	2.81337497367669e-05\\
318	2.81337500409688e-05\\
319	2.81337503505455e-05\\
320	2.81337506655991e-05\\
321	2.81337509862137e-05\\
322	2.81337513124966e-05\\
323	2.81337516445435e-05\\
324	2.81337519824545e-05\\
325	2.81337523263334e-05\\
326	2.81337526762819e-05\\
327	2.81337530324066e-05\\
328	2.81337533948133e-05\\
329	2.81337537636112e-05\\
330	2.81337541389147e-05\\
331	2.81337545208279e-05\\
332	2.8133754909473e-05\\
333	2.81337553049598e-05\\
334	2.81337557074072e-05\\
335	2.81337561169384e-05\\
336	2.81337565336706e-05\\
337	2.81337569577322e-05\\
338	2.81337573892454e-05\\
339	2.813375782833e-05\\
340	2.8133758275131e-05\\
341	2.81337587297657e-05\\
342	2.81337591923758e-05\\
343	2.81337596630928e-05\\
344	2.81337601420543e-05\\
345	2.81337606294026e-05\\
346	2.81337611252803e-05\\
347	2.81337616298287e-05\\
348	2.81337621431987e-05\\
349	2.81337626655367e-05\\
350	2.81337631969941e-05\\
351	2.81337637377316e-05\\
352	2.8133764287901e-05\\
353	2.81337648476618e-05\\
354	2.81337654171806e-05\\
355	2.81337659966236e-05\\
356	2.81337665861546e-05\\
357	2.81337671859498e-05\\
358	2.81337677961862e-05\\
359	2.81337684170371e-05\\
360	2.81337690486934e-05\\
361	2.81337696913324e-05\\
362	2.81337703451543e-05\\
363	2.8133771010344e-05\\
364	2.81337716871016e-05\\
365	2.81337723756337e-05\\
366	2.81337730761476e-05\\
367	2.8133773788855e-05\\
368	2.81337745139698e-05\\
369	2.81337752517173e-05\\
370	2.8133776002329e-05\\
371	2.81337767660334e-05\\
372	2.8133777543075e-05\\
373	2.81337783336969e-05\\
374	2.8133779138155e-05\\
375	2.81337799567104e-05\\
376	2.81337807896256e-05\\
377	2.81337816371843e-05\\
378	2.81337824996615e-05\\
379	2.81337833773583e-05\\
380	2.81337842705663e-05\\
381	2.81337851796034e-05\\
382	2.81337861047881e-05\\
383	2.81337870464521e-05\\
384	2.81337880049332e-05\\
385	2.81337889805949e-05\\
386	2.8133789973795e-05\\
387	2.81337909849155e-05\\
388	2.81337920143539e-05\\
389	2.8133793062518e-05\\
390	2.81337941298265e-05\\
391	2.81337952167319e-05\\
392	2.8133796323689e-05\\
393	2.81337974511832e-05\\
394	2.81337985997164e-05\\
395	2.8133799769823e-05\\
396	2.81338009620535e-05\\
397	2.81338021770035e-05\\
398	2.81338034152865e-05\\
399	2.81338046775592e-05\\
400	2.81338059645036e-05\\
401	2.81338072768512e-05\\
402	2.81338086153701e-05\\
403	2.81338099808813e-05\\
404	2.8133811374255e-05\\
405	2.81338127964106e-05\\
406	2.81338142483146e-05\\
407	2.81338157309872e-05\\
408	2.81338172455053e-05\\
409	2.81338187929989e-05\\
410	2.81338203746592e-05\\
411	2.81338219917572e-05\\
412	2.81338236456489e-05\\
413	2.81338253377985e-05\\
414	2.81338270697715e-05\\
415	2.81338288431885e-05\\
416	2.81338306598057e-05\\
417	2.8133832521532e-05\\
418	2.8133834430437e-05\\
419	2.81338363887944e-05\\
420	2.81338383990824e-05\\
421	2.81338404640445e-05\\
422	2.81338425867032e-05\\
423	2.81338447704258e-05\\
424	2.81338470189991e-05\\
425	2.81338493367416e-05\\
426	2.81338517287282e-05\\
427	2.81338542012443e-05\\
428	2.81338567627118e-05\\
429	2.81338594256742e-05\\
430	2.81338622109332e-05\\
431	2.81338651557164e-05\\
432	2.81338683285516e-05\\
433	2.81338718526688e-05\\
434	2.81338759324782e-05\\
435	2.81338808550613e-05\\
436	2.8133886895082e-05\\
437	2.81338940379714e-05\\
438	2.81339017053768e-05\\
439	2.81339095467552e-05\\
440	2.81339175673867e-05\\
441	2.81339257727273e-05\\
442	2.81339341684084e-05\\
443	2.81339427602136e-05\\
444	2.813395155407e-05\\
445	2.81339605560315e-05\\
446	2.8133969772238e-05\\
447	2.81339792089267e-05\\
448	2.81339888724338e-05\\
449	2.81339987693175e-05\\
450	2.8134008906741e-05\\
451	2.81340192933972e-05\\
452	2.81340299418105e-05\\
453	2.81340408735861e-05\\
454	2.81340521313795e-05\\
455	2.81340638060062e-05\\
456	2.81340760982331e-05\\
457	2.81340894593793e-05\\
458	2.81341049043062e-05\\
459	2.81341246570371e-05\\
460	2.8134153248409e-05\\
461	2.81341985442049e-05\\
462	2.81342705358535e-05\\
463	2.81343714431625e-05\\
464	2.81344741978628e-05\\
465	2.81345788555313e-05\\
466	2.81346854745645e-05\\
467	2.81347941161404e-05\\
468	2.81349048449461e-05\\
469	2.8135017729203e-05\\
470	2.81351328400244e-05\\
471	2.81352502518505e-05\\
472	2.81353700436219e-05\\
473	2.81354923001765e-05\\
474	2.81356171116547e-05\\
475	2.81357445687438e-05\\
476	2.81358747668706e-05\\
477	2.81360078077962e-05\\
478	2.8136143800238e-05\\
479	2.81362828605612e-05\\
480	2.81364251135934e-05\\
481	2.81365706935816e-05\\
482	2.81367197452142e-05\\
483	2.81368724247119e-05\\
484	2.8137028901155e-05\\
485	2.81371893582139e-05\\
486	2.81373539960068e-05\\
487	2.81375230333143e-05\\
488	2.81376967102731e-05\\
489	2.81378752917995e-05\\
490	2.81380590722409e-05\\
491	2.81382483824275e-05\\
492	2.81384436017835e-05\\
493	2.81386451817024e-05\\
494	2.81388536944236e-05\\
495	2.81390699394162e-05\\
496	2.81392951760404e-05\\
497	2.81395316214934e-05\\
498	2.81397834665539e-05\\
499	2.81400587811053e-05\\
500	2.81403725769567e-05\\
501	2.81407501861896e-05\\
502	2.81412261158299e-05\\
503	2.81418240534903e-05\\
504	2.8142494056663e-05\\
505	2.81431760234654e-05\\
506	2.81438704515901e-05\\
507	2.81445780048806e-05\\
508	2.81452995937644e-05\\
509	2.81460360740744e-05\\
510	2.81467882039468e-05\\
511	2.81475568259223e-05\\
512	2.81483428950455e-05\\
513	2.8149147542357e-05\\
514	2.81499722334099e-05\\
515	2.81508191831984e-05\\
516	2.81516924607644e-05\\
517	2.8152600928986e-05\\
518	2.81535659695826e-05\\
519	2.81546412453434e-05\\
520	2.81559608159852e-05\\
521	2.81578454496116e-05\\
522	2.81609878942634e-05\\
523	2.81665275746811e-05\\
524	2.81745139544136e-05\\
525	2.81827731062642e-05\\
526	2.81913618656566e-05\\
527	2.82003572120429e-05\\
528	2.82098270310358e-05\\
529	2.82197309448046e-05\\
530	2.82299892179033e-05\\
531	2.8240682176392e-05\\
532	2.82519874106123e-05\\
533	2.82643194926879e-05\\
534	2.8278627569162e-05\\
535	2.82968125600345e-05\\
536	2.83217138938609e-05\\
537	2.83517246750475e-05\\
538	2.83824225838729e-05\\
539	2.84138633242904e-05\\
540	2.84461071893863e-05\\
541	2.8479214651361e-05\\
542	2.85132533630335e-05\\
543	2.85483492676952e-05\\
544	2.85847585649049e-05\\
545	2.86230065566317e-05\\
546	2.86640980500053e-05\\
547	2.87101522735654e-05\\
548	2.87650324029993e-05\\
549	2.88322145377716e-05\\
550	2.89007850823885e-05\\
551	2.89711728679132e-05\\
552	2.90443793562356e-05\\
553	2.91237095053045e-05\\
554	2.92217095466991e-05\\
555	2.93823413498258e-05\\
556	3.11949843758959e-05\\
557	3.33502711355773e-05\\
558	3.56515603484925e-05\\
559	3.81908908622266e-05\\
560	5.11968095143857e-05\\
561	6.84846993355973e-05\\
562	8.65990041172735e-05\\
563	0.000105648987965169\\
564	0.000125795849217875\\
565	0.00014730680110381\\
566	0.000436738661923103\\
567	0.000745306289239306\\
568	0.00107028389305651\\
569	0.00141388692712991\\
570	0.00177872151880483\\
571	0.00216787035026358\\
572	0.00258498679269163\\
573	0.00303438193863091\\
574	0.0035065421024845\\
575	0.00399750076460377\\
576	0.0045090003218873\\
577	0.00504326235531745\\
578	0.00560208264361704\\
579	0.00616546650138609\\
580	0.00666557068064624\\
581	0.00692509810864789\\
582	0.00717995926326238\\
583	0.00742260514911438\\
584	0.0076461016657815\\
585	0.00786669402013687\\
586	0.00808503029377876\\
587	0.00830010902163021\\
588	0.00851069790722017\\
589	0.00871579279726982\\
590	0.00891706003505728\\
591	0.00911135646133238\\
592	0.00929518010420687\\
593	0.00946575592890774\\
594	0.00961947444539451\\
595	0.00975255418641367\\
596	0.00986361120881298\\
597	0.0099487662812083\\
598	0.0100000808076597\\
599	0\\
600	0\\
};
\addplot [color=mycolor13,solid,forget plot]
  table[row sep=crcr]{%
1	0.000473175162553967\\
2	0.000473175162554024\\
3	0.000473175162554076\\
4	0.000473175162554132\\
5	0.000473175162554193\\
6	0.000473175162554252\\
7	0.000473175162554309\\
8	0.000473175162554373\\
9	0.000473175162554432\\
10	0.000473175162554499\\
11	0.000473175162554561\\
12	0.000473175162554627\\
13	0.000473175162554694\\
14	0.000473175162554759\\
15	0.000473175162554829\\
16	0.000473175162554897\\
17	0.000473175162554968\\
18	0.00047317516255504\\
19	0.000473175162555114\\
20	0.000473175162555188\\
21	0.000473175162555262\\
22	0.000473175162555341\\
23	0.000473175162555418\\
24	0.000473175162555499\\
25	0.000473175162555579\\
26	0.000473175162555658\\
27	0.000473175162555745\\
28	0.000473175162555828\\
29	0.000473175162555917\\
30	0.000473175162556005\\
31	0.000473175162556094\\
32	0.000473175162556184\\
33	0.000473175162556276\\
34	0.00047317516255637\\
35	0.000473175162556464\\
36	0.000473175162556559\\
37	0.000473175162556661\\
38	0.000473175162556758\\
39	0.000473175162556861\\
40	0.000473175162556963\\
41	0.000473175162557071\\
42	0.000473175162557178\\
43	0.000473175162557285\\
44	0.000473175162557397\\
45	0.000473175162557509\\
46	0.000473175162557622\\
47	0.000473175162557739\\
48	0.000473175162557858\\
49	0.000473175162557978\\
50	0.0004731751625581\\
51	0.000473175162558226\\
52	0.00047317516255835\\
53	0.00047317516255848\\
54	0.000473175162558612\\
55	0.000473175162558743\\
56	0.000473175162558877\\
57	0.000473175162559015\\
58	0.000473175162559155\\
59	0.000473175162559297\\
60	0.000473175162559441\\
61	0.000473175162559587\\
62	0.000473175162559737\\
63	0.000473175162559888\\
64	0.000473175162560042\\
65	0.000473175162560199\\
66	0.000473175162560358\\
67	0.00047317516256052\\
68	0.000473175162560687\\
69	0.000473175162560855\\
70	0.000473175162561022\\
71	0.000473175162561196\\
72	0.000473175162561373\\
73	0.000473175162561553\\
74	0.000473175162561736\\
75	0.000473175162561922\\
76	0.000473175162562109\\
77	0.000473175162562299\\
78	0.000473175162562493\\
79	0.000473175162562693\\
80	0.000473175162562893\\
81	0.000473175162563098\\
82	0.00047317516256331\\
83	0.000473175162563521\\
84	0.000473175162563736\\
85	0.000473175162563954\\
86	0.000473175162564173\\
87	0.000473175162564401\\
88	0.000473175162564631\\
89	0.000473175162564864\\
90	0.000473175162565103\\
91	0.000473175162565345\\
92	0.000473175162565589\\
93	0.000473175162565841\\
94	0.000473175162566093\\
95	0.000473175162566353\\
96	0.000473175162566617\\
97	0.000473175162566885\\
98	0.000473175162567156\\
99	0.000473175162567433\\
100	0.000473175162567713\\
101	0.000473175162567998\\
102	0.000473175162568288\\
103	0.000473175162568586\\
104	0.000473175162568884\\
105	0.00047317516256919\\
106	0.000473175162569502\\
107	0.000473175162569816\\
108	0.000473175162570139\\
109	0.000473175162570467\\
110	0.000473175162570799\\
111	0.000473175162571135\\
112	0.000473175162571481\\
113	0.000473175162571829\\
114	0.000473175162572185\\
115	0.000473175162572549\\
116	0.000473175162572913\\
117	0.000473175162573287\\
118	0.000473175162573665\\
119	0.000473175162574056\\
120	0.000473175162574446\\
121	0.00047317516257485\\
122	0.000473175162575254\\
123	0.000473175162575667\\
124	0.000473175162576088\\
125	0.000473175162576517\\
126	0.00047317516257695\\
127	0.000473175162577396\\
128	0.000473175162577844\\
129	0.000473175162578301\\
130	0.000473175162578767\\
131	0.000473175162579241\\
132	0.00047317516257972\\
133	0.000473175162580212\\
134	0.000473175162580709\\
135	0.000473175162581216\\
136	0.000473175162581732\\
137	0.000473175162582255\\
138	0.000473175162582787\\
139	0.00047317516258333\\
140	0.000473175162583881\\
141	0.000473175162584443\\
142	0.000473175162585011\\
143	0.000473175162585591\\
144	0.000473175162586182\\
145	0.000473175162586785\\
146	0.000473175162587393\\
147	0.000473175162588011\\
148	0.000473175162588645\\
149	0.000473175162589286\\
150	0.000473175162589939\\
151	0.000473175162590604\\
152	0.000473175162591281\\
153	0.000473175162591965\\
154	0.000473175162592667\\
155	0.000473175162593379\\
156	0.000473175162594103\\
157	0.000473175162594836\\
158	0.000473175162595584\\
159	0.000473175162596344\\
160	0.000473175162597121\\
161	0.000473175162597909\\
162	0.000473175162598709\\
163	0.000473175162599525\\
164	0.000473175162600351\\
165	0.000473175162601195\\
166	0.000473175162602055\\
167	0.000473175162602928\\
168	0.000473175162603813\\
169	0.000473175162604714\\
170	0.000473175162605637\\
171	0.000473175162606567\\
172	0.000473175162607518\\
173	0.000473175162608484\\
174	0.000473175162609468\\
175	0.000473175162610468\\
176	0.000473175162611483\\
177	0.000473175162612519\\
178	0.000473175162613571\\
179	0.000473175162614641\\
180	0.000473175162615732\\
181	0.000473175162616839\\
182	0.000473175162617967\\
183	0.000473175162619111\\
184	0.000473175162620275\\
185	0.000473175162621465\\
186	0.000473175162622671\\
187	0.000473175162623899\\
188	0.000473175162625145\\
189	0.000473175162626415\\
190	0.000473175162627707\\
191	0.000473175162629021\\
192	0.000473175162630358\\
193	0.000473175162631715\\
194	0.000473175162633099\\
195	0.000473175162634507\\
196	0.000473175162635938\\
197	0.000473175162637395\\
198	0.000473175162638872\\
199	0.000473175162640379\\
200	0.000473175162641913\\
201	0.000473175162643472\\
202	0.000473175162645059\\
203	0.000473175162646669\\
204	0.000473175162648313\\
205	0.000473175162649981\\
206	0.000473175162651678\\
207	0.000473175162653406\\
208	0.000473175162655161\\
209	0.000473175162656952\\
210	0.000473175162658768\\
211	0.00047317516266062\\
212	0.000473175162662502\\
213	0.000473175162664417\\
214	0.000473175162666363\\
215	0.000473175162668341\\
216	0.00047317516267036\\
217	0.000473175162672408\\
218	0.000473175162674495\\
219	0.000473175162676615\\
220	0.000473175162678773\\
221	0.000473175162680969\\
222	0.000473175162683203\\
223	0.000473175162685479\\
224	0.000473175162687788\\
225	0.000473175162690137\\
226	0.000473175162692531\\
227	0.000473175162694965\\
228	0.000473175162697441\\
229	0.000473175162699962\\
230	0.00047317516270252\\
231	0.000473175162705128\\
232	0.000473175162707781\\
233	0.00047317516271048\\
234	0.000473175162713224\\
235	0.000473175162716015\\
236	0.000473175162718856\\
237	0.000473175162721746\\
238	0.000473175162724689\\
239	0.000473175162727678\\
240	0.000473175162730722\\
241	0.000473175162733821\\
242	0.00047317516273697\\
243	0.000473175162740174\\
244	0.000473175162743434\\
245	0.00047317516274675\\
246	0.000473175162750123\\
247	0.000473175162753557\\
248	0.000473175162757051\\
249	0.000473175162760607\\
250	0.000473175162764218\\
251	0.000473175162767896\\
252	0.00047317516277164\\
253	0.000473175162775448\\
254	0.000473175162779324\\
255	0.000473175162783263\\
256	0.000473175162787273\\
257	0.000473175162791354\\
258	0.000473175162795504\\
259	0.000473175162799727\\
260	0.000473175162804025\\
261	0.000473175162808397\\
262	0.000473175162812848\\
263	0.000473175162817372\\
264	0.000473175162821976\\
265	0.000473175162826661\\
266	0.000473175162831428\\
267	0.000473175162836282\\
268	0.000473175162841216\\
269	0.000473175162846236\\
270	0.000473175162851346\\
271	0.000473175162856544\\
272	0.000473175162861833\\
273	0.000473175162867217\\
274	0.000473175162872693\\
275	0.000473175162878264\\
276	0.000473175162883936\\
277	0.000473175162889705\\
278	0.000473175162895571\\
279	0.000473175162901544\\
280	0.000473175162907623\\
281	0.000473175162913807\\
282	0.000473175162920097\\
283	0.000473175162926499\\
284	0.000473175162933016\\
285	0.000473175162939646\\
286	0.000473175162946388\\
287	0.000473175162953254\\
288	0.000473175162960235\\
289	0.000473175162967343\\
290	0.000473175162974574\\
291	0.00047317516298193\\
292	0.000473175162989419\\
293	0.000473175162997037\\
294	0.000473175163004791\\
295	0.000473175163012678\\
296	0.000473175163020703\\
297	0.000473175163028874\\
298	0.000473175163037184\\
299	0.000473175163045642\\
300	0.000473175163054246\\
301	0.000473175163063003\\
302	0.000473175163071915\\
303	0.000473175163080983\\
304	0.000473175163090209\\
305	0.000473175163099598\\
306	0.000473175163109153\\
307	0.000473175163118874\\
308	0.000473175163128767\\
309	0.000473175163138834\\
310	0.000473175163149079\\
311	0.000473175163159499\\
312	0.000473175163170107\\
313	0.000473175163180899\\
314	0.000473175163191879\\
315	0.000473175163203054\\
316	0.000473175163214424\\
317	0.000473175163225996\\
318	0.00047317516323777\\
319	0.000473175163249749\\
320	0.000473175163261941\\
321	0.000473175163274343\\
322	0.000473175163286964\\
323	0.000473175163299808\\
324	0.000473175163312874\\
325	0.000473175163326171\\
326	0.000473175163339701\\
327	0.000473175163353463\\
328	0.000473175163367471\\
329	0.000473175163381725\\
330	0.000473175163396226\\
331	0.000473175163410979\\
332	0.000473175163425988\\
333	0.000473175163441265\\
334	0.000473175163456805\\
335	0.000473175163472617\\
336	0.000473175163488704\\
337	0.000473175163505071\\
338	0.000473175163521726\\
339	0.000473175163538667\\
340	0.000473175163555905\\
341	0.000473175163573441\\
342	0.000473175163591281\\
343	0.000473175163609432\\
344	0.000473175163627897\\
345	0.000473175163646686\\
346	0.000473175163665796\\
347	0.000473175163685241\\
348	0.000473175163705021\\
349	0.00047317516372514\\
350	0.000473175163745611\\
351	0.000473175163766434\\
352	0.000473175163787617\\
353	0.000473175163809166\\
354	0.000473175163831086\\
355	0.000473175163853389\\
356	0.000473175163876072\\
357	0.000473175163899148\\
358	0.00047317516392262\\
359	0.0004731751639465\\
360	0.00047317516397079\\
361	0.000473175163995493\\
362	0.000473175164020628\\
363	0.000473175164046195\\
364	0.000473175164072203\\
365	0.000473175164098657\\
366	0.000473175164125571\\
367	0.000473175164152947\\
368	0.000473175164180796\\
369	0.000473175164209127\\
370	0.000473175164237944\\
371	0.000473175164267264\\
372	0.00047317516429709\\
373	0.000473175164327433\\
374	0.000473175164358301\\
375	0.00047317516438971\\
376	0.000473175164421665\\
377	0.000473175164454177\\
378	0.00047317516448726\\
379	0.000473175164520922\\
380	0.00047317516455517\\
381	0.000473175164590027\\
382	0.000473175164625496\\
383	0.000473175164661593\\
384	0.000473175164698331\\
385	0.000473175164735728\\
386	0.000473175164773787\\
387	0.000473175164812534\\
388	0.000473175164851977\\
389	0.000473175164892132\\
390	0.000473175164933022\\
391	0.000473175164974652\\
392	0.000473175165017051\\
393	0.000473175165060232\\
394	0.000473175165104212\\
395	0.000473175165149018\\
396	0.000473175165194667\\
397	0.00047317516524118\\
398	0.000473175165288586\\
399	0.000473175165336908\\
400	0.000473175165386172\\
401	0.000473175165436404\\
402	0.00047317516548764\\
403	0.000473175165539907\\
404	0.00047317516559324\\
405	0.000473175165647672\\
406	0.000473175165703247\\
407	0.000473175165759995\\
408	0.000473175165817966\\
409	0.000473175165877196\\
410	0.000473175165937734\\
411	0.000473175165999625\\
412	0.00047317516606293\\
413	0.000473175166127697\\
414	0.000473175166193986\\
415	0.000473175166261865\\
416	0.000473175166331389\\
417	0.000473175166402632\\
418	0.000473175166475673\\
419	0.000473175166550596\\
420	0.00047317516662747\\
421	0.00047317516670638\\
422	0.000473175166787368\\
423	0.000473175166870438\\
424	0.000473175166955503\\
425	0.000473175167042348\\
426	0.0004731751671306\\
427	0.000473175167219876\\
428	0.000473175167310199\\
429	0.000473175167402913\\
430	0.000473175167501735\\
431	0.000473175167612623\\
432	0.000473175167741854\\
433	0.000473175167898108\\
434	0.000473175168090725\\
435	0.000473175168324615\\
436	0.000473175168592866\\
437	0.000473175168876787\\
438	0.000473175169167039\\
439	0.000473175169463789\\
440	0.00047317516976716\\
441	0.000473175170077208\\
442	0.000473175170393811\\
443	0.000473175170716426\\
444	0.000473175171043648\\
445	0.000473175171372302\\
446	0.000473175171695894\\
447	0.000473175172002317\\
448	0.000473175172271706\\
449	0.000473175172477659\\
450	0.000473175172599086\\
451	0.000473175172648755\\
452	0.000473175172692034\\
453	0.00047317517274556\\
454	0.000473175172818368\\
455	0.000473175172929481\\
456	0.000473175173117959\\
457	0.000473175173460502\\
458	0.000473175174097164\\
459	0.000473175175253893\\
460	0.000473175177220325\\
461	0.000473175180203658\\
462	0.000473175184004874\\
463	0.000473175187875527\\
464	0.000473175191817725\\
465	0.000473175195833697\\
466	0.000473175199925758\\
467	0.000473175204096375\\
468	0.000473175208348131\\
469	0.000473175212683722\\
470	0.000473175217105985\\
471	0.000473175221617924\\
472	0.000473175226222766\\
473	0.000473175230923921\\
474	0.000473175235724834\\
475	0.000473175240629168\\
476	0.000473175245640815\\
477	0.000473175250763942\\
478	0.000473175256003015\\
479	0.000473175261362829\\
480	0.000473175266848549\\
481	0.000473175272465744\\
482	0.000473175278220452\\
483	0.000473175284119197\\
484	0.000473175290169105\\
485	0.000473175296377944\\
486	0.000473175302754253\\
487	0.000473175309307447\\
488	0.000473175316048008\\
489	0.000473175322987789\\
490	0.000473175330140504\\
491	0.000473175337522743\\
492	0.00047317534515592\\
493	0.000473175353070346\\
494	0.000473175361313399\\
495	0.000473175369965481\\
496	0.000473175379169413\\
497	0.00047317538917951\\
498	0.000473175400430638\\
499	0.000473175413601941\\
500	0.00047317542958557\\
501	0.000473175449168441\\
502	0.000473175472233765\\
503	0.000473175496963544\\
504	0.000473175522063292\\
505	0.000473175547622054\\
506	0.000473175573666092\\
507	0.000473175600228205\\
508	0.000473175627339787\\
509	0.000473175655030453\\
510	0.000473175683334556\\
511	0.000473175712294452\\
512	0.000473175741968268\\
513	0.000473175772448685\\
514	0.000473175803908078\\
515	0.000473175836705355\\
516	0.000473175871631882\\
517	0.000473175910454574\\
518	0.000473175957041446\\
519	0.000473176019460056\\
520	0.000473176113140806\\
521	0.000473176263169241\\
522	0.000473176496693719\\
523	0.000473176800581587\\
524	0.000473177115683316\\
525	0.000473177444155484\\
526	0.000473177788241468\\
527	0.000473178149092623\\
528	0.00047317852556872\\
529	0.000473178918025487\\
530	0.000473179333645463\\
531	0.000473179785852118\\
532	0.000473180300502788\\
533	0.00047318092238646\\
534	0.000473181712220031\\
535	0.000473182708947826\\
536	0.000473183825773694\\
537	0.000473184958722114\\
538	0.000473186119178134\\
539	0.000473187309526129\\
540	0.000473188532665951\\
541	0.000473189793084638\\
542	0.000473191099463133\\
543	0.000473192468412329\\
544	0.000473193930246653\\
545	0.000473195536382525\\
546	0.000473197367243832\\
547	0.000473199510717141\\
548	0.000473201958331766\\
549	0.000473204463124105\\
550	0.000473207119760541\\
551	0.000473210094446619\\
552	0.000473213797260883\\
553	0.000473219182365413\\
554	0.000473228128009639\\
555	0.000473242972942894\\
556	0.000473260476436618\\
557	0.000473281788286304\\
558	0.000473310383391123\\
559	0.000473350457424057\\
560	0.00047339436694403\\
561	0.000473440469139966\\
562	0.000473489873620997\\
563	0.000473545172561519\\
564	0.000473611055489094\\
565	0.00047369244246937\\
566	0.000473774598345136\\
567	0.00047385828569523\\
568	0.000473944105034825\\
569	0.000474033094445854\\
570	0.000474127191615011\\
571	0.000474229631036934\\
572	0.000474343905319057\\
573	0.000474462234903059\\
574	0.000474581258630803\\
575	0.000474708291910568\\
576	0.000474875563427192\\
577	0.000475156598555233\\
578	0.000475766978792443\\
579	0.000496370411251768\\
580	0.000600560749493593\\
581	0.000933347328653583\\
582	0.00128982762551076\\
583	0.00167386783687287\\
584	0.00209022542355256\\
585	0.00252405806352124\\
586	0.00297466555577394\\
587	0.0034428318942901\\
588	0.00392917113193939\\
589	0.00443391940083715\\
590	0.00495455263915653\\
591	0.00549180184738378\\
592	0.0060467654116253\\
593	0.00662076179573163\\
594	0.00721556153030671\\
595	0.00783363072746384\\
596	0.00847448664470217\\
597	0.00913730313064842\\
598	0.00981916109635161\\
599	0\\
600	0\\
};
\addplot [color=mycolor14,solid,forget plot]
  table[row sep=crcr]{%
1	0.00998511802871796\\
2	0.00998511802871785\\
3	0.00998511802871774\\
4	0.00998511802871762\\
5	0.00998511802871751\\
6	0.00998511802871739\\
7	0.00998511802871726\\
8	0.00998511802871714\\
9	0.00998511802871701\\
10	0.00998511802871689\\
11	0.00998511802871676\\
12	0.00998511802871662\\
13	0.00998511802871649\\
14	0.00998511802871635\\
15	0.00998511802871621\\
16	0.00998511802871607\\
17	0.00998511802871592\\
18	0.00998511802871577\\
19	0.00998511802871562\\
20	0.00998511802871547\\
21	0.00998511802871531\\
22	0.00998511802871515\\
23	0.00998511802871499\\
24	0.00998511802871483\\
25	0.00998511802871466\\
26	0.00998511802871449\\
27	0.00998511802871431\\
28	0.00998511802871414\\
29	0.00998511802871396\\
30	0.00998511802871377\\
31	0.00998511802871359\\
32	0.0099851180287134\\
33	0.0099851180287132\\
34	0.00998511802871301\\
35	0.00998511802871281\\
36	0.0099851180287126\\
37	0.0099851180287124\\
38	0.00998511802871218\\
39	0.00998511802871197\\
40	0.00998511802871175\\
41	0.00998511802871153\\
42	0.0099851180287113\\
43	0.00998511802871107\\
44	0.00998511802871084\\
45	0.0099851180287106\\
46	0.00998511802871035\\
47	0.00998511802871011\\
48	0.00998511802870986\\
49	0.0099851180287096\\
50	0.00998511802870934\\
51	0.00998511802870907\\
52	0.0099851180287088\\
53	0.00998511802870853\\
54	0.00998511802870825\\
55	0.00998511802870796\\
56	0.00998511802870767\\
57	0.00998511802870738\\
58	0.00998511802870707\\
59	0.00998511802870677\\
60	0.00998511802870646\\
61	0.00998511802870614\\
62	0.00998511802870582\\
63	0.00998511802870549\\
64	0.00998511802870515\\
65	0.00998511802870481\\
66	0.00998511802870447\\
67	0.00998511802870411\\
68	0.00998511802870376\\
69	0.00998511802870339\\
70	0.00998511802870302\\
71	0.00998511802870264\\
72	0.00998511802870226\\
73	0.00998511802870186\\
74	0.00998511802870146\\
75	0.00998511802870106\\
76	0.00998511802870064\\
77	0.00998511802870022\\
78	0.00998511802869979\\
79	0.00998511802869936\\
80	0.00998511802869891\\
81	0.00998511802869846\\
82	0.009985118028698\\
83	0.00998511802869754\\
84	0.00998511802869706\\
85	0.00998511802869657\\
86	0.00998511802869608\\
87	0.00998511802869558\\
88	0.00998511802869507\\
89	0.00998511802869455\\
90	0.00998511802869402\\
91	0.00998511802869348\\
92	0.00998511802869293\\
93	0.00998511802869237\\
94	0.0099851180286918\\
95	0.00998511802869122\\
96	0.00998511802869063\\
97	0.00998511802869003\\
98	0.00998511802868942\\
99	0.0099851180286888\\
100	0.00998511802868817\\
101	0.00998511802868753\\
102	0.00998511802868687\\
103	0.00998511802868621\\
104	0.00998511802868553\\
105	0.00998511802868484\\
106	0.00998511802868413\\
107	0.00998511802868342\\
108	0.00998511802868269\\
109	0.00998511802868195\\
110	0.00998511802868119\\
111	0.00998511802868043\\
112	0.00998511802867964\\
113	0.00998511802867885\\
114	0.00998511802867804\\
115	0.00998511802867722\\
116	0.00998511802867638\\
117	0.00998511802867552\\
118	0.00998511802867465\\
119	0.00998511802867377\\
120	0.00998511802867287\\
121	0.00998511802867195\\
122	0.00998511802867102\\
123	0.00998511802867007\\
124	0.0099851180286691\\
125	0.00998511802866812\\
126	0.00998511802866712\\
127	0.0099851180286661\\
128	0.00998511802866506\\
129	0.00998511802866401\\
130	0.00998511802866293\\
131	0.00998511802866184\\
132	0.00998511802866073\\
133	0.00998511802865959\\
134	0.00998511802865844\\
135	0.00998511802865727\\
136	0.00998511802865607\\
137	0.00998511802865485\\
138	0.00998511802865362\\
139	0.00998511802865236\\
140	0.00998511802865107\\
141	0.00998511802864977\\
142	0.00998511802864844\\
143	0.00998511802864709\\
144	0.00998511802864571\\
145	0.00998511802864431\\
146	0.00998511802864288\\
147	0.00998511802864143\\
148	0.00998511802863996\\
149	0.00998511802863845\\
150	0.00998511802863692\\
151	0.00998511802863536\\
152	0.00998511802863378\\
153	0.00998511802863216\\
154	0.00998511802863052\\
155	0.00998511802862885\\
156	0.00998511802862715\\
157	0.00998511802862541\\
158	0.00998511802862365\\
159	0.00998511802862186\\
160	0.00998511802862003\\
161	0.00998511802861817\\
162	0.00998511802861628\\
163	0.00998511802861435\\
164	0.00998511802861239\\
165	0.00998511802861039\\
166	0.00998511802860836\\
167	0.00998511802860629\\
168	0.00998511802860419\\
169	0.00998511802860204\\
170	0.00998511802859986\\
171	0.00998511802859764\\
172	0.00998511802859538\\
173	0.00998511802859308\\
174	0.00998511802859074\\
175	0.00998511802858836\\
176	0.00998511802858593\\
177	0.00998511802858346\\
178	0.00998511802858095\\
179	0.00998511802857839\\
180	0.00998511802857579\\
181	0.00998511802857314\\
182	0.00998511802857044\\
183	0.00998511802856769\\
184	0.0099851180285649\\
185	0.00998511802856205\\
186	0.00998511802855916\\
187	0.00998511802855621\\
188	0.00998511802855321\\
189	0.00998511802855015\\
190	0.00998511802854704\\
191	0.00998511802854388\\
192	0.00998511802854066\\
193	0.00998511802853738\\
194	0.00998511802853404\\
195	0.00998511802853064\\
196	0.00998511802852718\\
197	0.00998511802852366\\
198	0.00998511802852007\\
199	0.00998511802851643\\
200	0.00998511802851271\\
201	0.00998511802850893\\
202	0.00998511802850508\\
203	0.00998511802850117\\
204	0.00998511802849718\\
205	0.00998511802849312\\
206	0.00998511802848898\\
207	0.00998511802848478\\
208	0.0099851180284805\\
209	0.00998511802847614\\
210	0.0099851180284717\\
211	0.00998511802846718\\
212	0.00998511802846258\\
213	0.0099851180284579\\
214	0.00998511802845314\\
215	0.00998511802844829\\
216	0.00998511802844335\\
217	0.00998511802843832\\
218	0.0099851180284332\\
219	0.00998511802842799\\
220	0.00998511802842269\\
221	0.00998511802841729\\
222	0.00998511802841179\\
223	0.0099851180284062\\
224	0.0099851180284005\\
225	0.0099851180283947\\
226	0.00998511802838879\\
227	0.00998511802838278\\
228	0.00998511802837667\\
229	0.00998511802837043\\
230	0.00998511802836409\\
231	0.00998511802835764\\
232	0.00998511802835106\\
233	0.00998511802834437\\
234	0.00998511802833755\\
235	0.00998511802833062\\
236	0.00998511802832355\\
237	0.00998511802831637\\
238	0.00998511802830904\\
239	0.00998511802830159\\
240	0.009985118028294\\
241	0.00998511802828628\\
242	0.00998511802827841\\
243	0.0099851180282704\\
244	0.00998511802826225\\
245	0.00998511802825395\\
246	0.00998511802824549\\
247	0.00998511802823689\\
248	0.00998511802822813\\
249	0.0099851180282192\\
250	0.00998511802821012\\
251	0.00998511802820087\\
252	0.00998511802819145\\
253	0.00998511802818186\\
254	0.0099851180281721\\
255	0.00998511802816216\\
256	0.00998511802815203\\
257	0.00998511802814173\\
258	0.00998511802813123\\
259	0.00998511802812054\\
260	0.00998511802810966\\
261	0.00998511802809858\\
262	0.0099851180280873\\
263	0.0099851180280758\\
264	0.0099851180280641\\
265	0.00998511802805219\\
266	0.00998511802804006\\
267	0.0099851180280277\\
268	0.00998511802801512\\
269	0.00998511802800231\\
270	0.00998511802798926\\
271	0.00998511802797597\\
272	0.00998511802796244\\
273	0.00998511802794866\\
274	0.00998511802793463\\
275	0.00998511802792034\\
276	0.00998511802790578\\
277	0.00998511802789096\\
278	0.00998511802787587\\
279	0.0099851180278605\\
280	0.00998511802784484\\
281	0.0099851180278289\\
282	0.00998511802781266\\
283	0.00998511802779612\\
284	0.00998511802777928\\
285	0.00998511802776212\\
286	0.00998511802774465\\
287	0.00998511802772685\\
288	0.00998511802770873\\
289	0.00998511802769027\\
290	0.00998511802767147\\
291	0.00998511802765232\\
292	0.00998511802763282\\
293	0.00998511802761295\\
294	0.00998511802759272\\
295	0.00998511802757211\\
296	0.00998511802755112\\
297	0.00998511802752973\\
298	0.00998511802750795\\
299	0.00998511802748577\\
300	0.00998511802746317\\
301	0.00998511802744016\\
302	0.00998511802741671\\
303	0.00998511802739283\\
304	0.00998511802736851\\
305	0.00998511802734373\\
306	0.00998511802731849\\
307	0.00998511802729278\\
308	0.00998511802726659\\
309	0.00998511802723991\\
310	0.00998511802721273\\
311	0.00998511802718505\\
312	0.00998511802715685\\
313	0.00998511802712813\\
314	0.00998511802709887\\
315	0.00998511802706906\\
316	0.0099851180270387\\
317	0.00998511802700777\\
318	0.00998511802697626\\
319	0.00998511802694416\\
320	0.00998511802691146\\
321	0.00998511802687816\\
322	0.00998511802684423\\
323	0.00998511802680967\\
324	0.00998511802677446\\
325	0.00998511802673859\\
326	0.00998511802670206\\
327	0.00998511802666484\\
328	0.00998511802662693\\
329	0.00998511802658831\\
330	0.00998511802654897\\
331	0.00998511802650889\\
332	0.00998511802646807\\
333	0.00998511802642648\\
334	0.00998511802638412\\
335	0.00998511802634097\\
336	0.00998511802629701\\
337	0.00998511802625224\\
338	0.00998511802620663\\
339	0.00998511802616017\\
340	0.00998511802611284\\
341	0.00998511802606464\\
342	0.00998511802601553\\
343	0.00998511802596552\\
344	0.00998511802591457\\
345	0.00998511802586267\\
346	0.00998511802580981\\
347	0.00998511802575597\\
348	0.00998511802570112\\
349	0.00998511802564526\\
350	0.00998511802558836\\
351	0.0099851180255304\\
352	0.00998511802547137\\
353	0.00998511802541124\\
354	0.00998511802535\\
355	0.00998511802528762\\
356	0.00998511802522408\\
357	0.00998511802515937\\
358	0.00998511802509345\\
359	0.00998511802502632\\
360	0.00998511802495793\\
361	0.00998511802488829\\
362	0.00998511802481735\\
363	0.00998511802474509\\
364	0.0099851180246715\\
365	0.00998511802459654\\
366	0.00998511802452019\\
367	0.00998511802444242\\
368	0.00998511802436321\\
369	0.00998511802428253\\
370	0.00998511802420034\\
371	0.00998511802411663\\
372	0.00998511802403135\\
373	0.00998511802394449\\
374	0.009985118023856\\
375	0.00998511802376585\\
376	0.00998511802367402\\
377	0.00998511802358045\\
378	0.00998511802348513\\
379	0.009985118023388\\
380	0.00998511802328904\\
381	0.0099851180231882\\
382	0.00998511802308543\\
383	0.00998511802298071\\
384	0.00998511802287396\\
385	0.00998511802276516\\
386	0.00998511802265425\\
387	0.00998511802254117\\
388	0.00998511802242588\\
389	0.00998511802230831\\
390	0.00998511802218839\\
391	0.00998511802206607\\
392	0.00998511802194128\\
393	0.00998511802181394\\
394	0.00998511802168398\\
395	0.00998511802155131\\
396	0.00998511802141585\\
397	0.0099851180212775\\
398	0.00998511802113618\\
399	0.00998511802099178\\
400	0.00998511802084419\\
401	0.00998511802069331\\
402	0.00998511802053901\\
403	0.00998511802038118\\
404	0.00998511802021967\\
405	0.00998511802005435\\
406	0.00998511801988506\\
407	0.00998511801971165\\
408	0.00998511801953394\\
409	0.00998511801935175\\
410	0.00998511801916488\\
411	0.00998511801897311\\
412	0.00998511801877621\\
413	0.00998511801857392\\
414	0.00998511801836594\\
415	0.00998511801815196\\
416	0.00998511801793162\\
417	0.00998511801770451\\
418	0.00998511801747015\\
419	0.00998511801722797\\
420	0.00998511801697722\\
421	0.00998511801671685\\
422	0.00998511801644527\\
423	0.00998511801615986\\
424	0.00998511801585622\\
425	0.00998511801552696\\
426	0.00998511801516034\\
427	0.00998511801473954\\
428	0.00998511801424479\\
429	0.00998511801366148\\
430	0.00998511801299438\\
431	0.00998511801227561\\
432	0.00998511801153994\\
433	0.00998511801078679\\
434	0.00998511801001559\\
435	0.00998511800922569\\
436	0.00998511800841643\\
437	0.00998511800758706\\
438	0.00998511800673676\\
439	0.00998511800586447\\
440	0.00998511800496869\\
441	0.0099851180040469\\
442	0.0099851180030943\\
443	0.00998511800210096\\
444	0.00998511800104589\\
445	0.00998511799988498\\
446	0.00998511799852855\\
447	0.0099851179968042\\
448	0.0099851179944078\\
449	0.00998511799087192\\
450	0.00998511798564033\\
451	0.00998511797839867\\
452	0.00998511796963951\\
453	0.00998511796049541\\
454	0.00998511795117467\\
455	0.00998511794167471\\
456	0.00998511793199257\\
457	0.0099851179221248\\
458	0.00998511791206759\\
459	0.0099851179018172\\
460	0.00998511789137052\\
461	0.00998511788072526\\
462	0.00998511786987873\\
463	0.00998511785882602\\
464	0.00998511784756107\\
465	0.00998511783607754\\
466	0.00998511782436881\\
467	0.00998511781242792\\
468	0.00998511780024756\\
469	0.00998511778782005\\
470	0.00998511777513721\\
471	0.00998511776219027\\
472	0.00998511774896983\\
473	0.00998511773546623\\
474	0.00998511772166966\\
475	0.00998511770756967\\
476	0.00998511769315509\\
477	0.00998511767841397\\
478	0.00998511766333353\\
479	0.00998511764790004\\
480	0.00998511763209874\\
481	0.00998511761591373\\
482	0.00998511759932785\\
483	0.00998511758232254\\
484	0.00998511756487767\\
485	0.00998511754697134\\
486	0.00998511752857963\\
487	0.00998511750967639\\
488	0.00998511749023279\\
489	0.00998511747021697\\
490	0.00998511744959335\\
491	0.00998511742832166\\
492	0.00998511740635515\\
493	0.00998511738363715\\
494	0.00998511736009409\\
495	0.00998511733562108\\
496	0.00998511731005259\\
497	0.00998511728310503\\
498	0.00998511725427178\\
499	0.00998511722265289\\
500	0.00998511718673569\\
501	0.00998511714426407\\
502	0.00998511709260427\\
503	0.00998511703029534\\
504	0.00998511695971538\\
505	0.0099851168865339\\
506	0.00998511681192764\\
507	0.00998511673580025\\
508	0.00998511665805898\\
509	0.00998511657861976\\
510	0.00998511649738276\\
511	0.00998511641421903\\
512	0.00998511632893677\\
513	0.00998511624120089\\
514	0.00998511615034316\\
515	0.0099851160549237\\
516	0.00998511595175773\\
517	0.00998511583387876\\
518	0.00998511568666123\\
519	0.0099851154816301\\
520	0.00998511517025623\\
521	0.00998511468952417\\
522	0.00998511400757196\\
523	0.00998511319589745\\
524	0.0099851123591405\\
525	0.00998511149423317\\
526	0.00998511059677676\\
527	0.00998510966072279\\
528	0.00998510867893777\\
529	0.00998510764570494\\
530	0.00998510656027541\\
531	0.00998510542201478\\
532	0.00998510420842489\\
533	0.0099851028737463\\
534	0.00998510133580938\\
535	0.00998509946529904\\
536	0.00998509711340298\\
537	0.0099850942583463\\
538	0.00998509116098175\\
539	0.00998508798633979\\
540	0.00998508472752659\\
541	0.00998508137618612\\
542	0.0099850779213166\\
543	0.00998507434609298\\
544	0.00998507062019089\\
545	0.00998506668561362\\
546	0.009985062436933\\
547	0.00998505770153949\\
548	0.00998505224446593\\
549	0.00998504588696346\\
550	0.00998503876534755\\
551	0.00998503116579294\\
552	0.00998502254380382\\
553	0.00998501156688488\\
554	0.00998499534466784\\
555	0.00998496903017647\\
556	0.00998486242974353\\
557	0.00998474113946046\\
558	0.00998460827722289\\
559	0.00998445680868363\\
560	0.00998420241303315\\
561	0.00998335031741292\\
562	0.00998245911690205\\
563	0.00998152186243509\\
564	0.00998052746856814\\
565	0.00997945977598139\\
566	0.00997139215146719\\
567	0.009957536155044\\
568	0.0099438278548402\\
569	0.00993027725082325\\
570	0.00991690018786295\\
571	0.00990371980931498\\
572	0.00989077109398008\\
573	0.0098781215773626\\
574	0.00986627495676953\\
575	0.00985533030757299\\
576	0.00984527994672241\\
577	0.00983607263603388\\
578	0.0098275411057054\\
579	0.00980596155700599\\
580	0.00973876645665698\\
581	0.00936387568476288\\
582	0.00896433087946395\\
583	0.0085361859641418\\
584	0.00807736222344448\\
585	0.00760436507040351\\
586	0.00711641931186753\\
587	0.00661283978228301\\
588	0.00609312798611908\\
589	0.00555720333092511\\
590	0.00500796019506793\\
591	0.0044444351141619\\
592	0.00386554713627044\\
593	0.00326998733071649\\
594	0.00265597549594283\\
595	0.00202116980785495\\
596	0.0013661236347808\\
597	0.000691357200101974\\
598	8.08076597025309e-08\\
599	0\\
600	0\\
};
\addplot [color=mycolor15,solid,forget plot]
  table[row sep=crcr]{%
1	0.00997161957607644\\
2	0.0099716195760741\\
3	0.00997161957607171\\
4	0.00997161957606929\\
5	0.00997161957606682\\
6	0.0099716195760643\\
7	0.00997161957606175\\
8	0.00997161957605914\\
9	0.00997161957605649\\
10	0.00997161957605379\\
11	0.00997161957605105\\
12	0.00997161957604825\\
13	0.00997161957604541\\
14	0.00997161957604251\\
15	0.00997161957603956\\
16	0.00997161957603656\\
17	0.00997161957603351\\
18	0.0099716195760304\\
19	0.00997161957602724\\
20	0.00997161957602401\\
21	0.00997161957602073\\
22	0.0099716195760174\\
23	0.009971619576014\\
24	0.00997161957601054\\
25	0.00997161957600703\\
26	0.00997161957600344\\
27	0.0099716195759998\\
28	0.00997161957599609\\
29	0.00997161957599231\\
30	0.00997161957598846\\
31	0.00997161957598455\\
32	0.00997161957598057\\
33	0.00997161957597651\\
34	0.00997161957597238\\
35	0.00997161957596818\\
36	0.00997161957596391\\
37	0.00997161957595955\\
38	0.00997161957595512\\
39	0.00997161957595061\\
40	0.00997161957594603\\
41	0.00997161957594135\\
42	0.0099716195759366\\
43	0.00997161957593176\\
44	0.00997161957592683\\
45	0.00997161957592182\\
46	0.00997161957591671\\
47	0.00997161957591152\\
48	0.00997161957590623\\
49	0.00997161957590085\\
50	0.00997161957589537\\
51	0.00997161957588979\\
52	0.00997161957588411\\
53	0.00997161957587834\\
54	0.00997161957587246\\
55	0.00997161957586647\\
56	0.00997161957586038\\
57	0.00997161957585418\\
58	0.00997161957584787\\
59	0.00997161957584144\\
60	0.0099716195758349\\
61	0.00997161957582825\\
62	0.00997161957582147\\
63	0.00997161957581457\\
64	0.00997161957580755\\
65	0.00997161957580041\\
66	0.00997161957579314\\
67	0.00997161957578574\\
68	0.0099716195757782\\
69	0.00997161957577053\\
70	0.00997161957576273\\
71	0.00997161957575479\\
72	0.0099716195757467\\
73	0.00997161957573847\\
74	0.00997161957573009\\
75	0.00997161957572156\\
76	0.00997161957571288\\
77	0.00997161957570405\\
78	0.00997161957569506\\
79	0.0099716195756859\\
80	0.00997161957567659\\
81	0.00997161957566711\\
82	0.00997161957565746\\
83	0.00997161957564763\\
84	0.00997161957563763\\
85	0.00997161957562746\\
86	0.0099716195756171\\
87	0.00997161957560655\\
88	0.00997161957559582\\
89	0.0099716195755849\\
90	0.00997161957557378\\
91	0.00997161957556246\\
92	0.00997161957555094\\
93	0.00997161957553922\\
94	0.00997161957552729\\
95	0.00997161957551514\\
96	0.00997161957550278\\
97	0.00997161957549019\\
98	0.00997161957547739\\
99	0.00997161957546435\\
100	0.00997161957545108\\
101	0.00997161957543758\\
102	0.00997161957542383\\
103	0.00997161957540984\\
104	0.0099716195753956\\
105	0.0099716195753811\\
106	0.00997161957536634\\
107	0.00997161957535133\\
108	0.00997161957533604\\
109	0.00997161957532048\\
110	0.00997161957530465\\
111	0.00997161957528853\\
112	0.00997161957527212\\
113	0.00997161957525542\\
114	0.00997161957523843\\
115	0.00997161957522113\\
116	0.00997161957520352\\
117	0.0099716195751856\\
118	0.00997161957516735\\
119	0.00997161957514879\\
120	0.00997161957512989\\
121	0.00997161957511065\\
122	0.00997161957509107\\
123	0.00997161957507114\\
124	0.00997161957505086\\
125	0.00997161957503021\\
126	0.0099716195750092\\
127	0.00997161957498781\\
128	0.00997161957496604\\
129	0.00997161957494388\\
130	0.00997161957492132\\
131	0.00997161957489837\\
132	0.009971619574875\\
133	0.00997161957485121\\
134	0.00997161957482701\\
135	0.00997161957480237\\
136	0.00997161957477729\\
137	0.00997161957475176\\
138	0.00997161957472578\\
139	0.00997161957469933\\
140	0.00997161957467241\\
141	0.00997161957464501\\
142	0.00997161957461712\\
143	0.00997161957458874\\
144	0.00997161957455985\\
145	0.00997161957453044\\
146	0.00997161957450051\\
147	0.00997161957447004\\
148	0.00997161957443903\\
149	0.00997161957440746\\
150	0.00997161957437534\\
151	0.00997161957434264\\
152	0.00997161957430935\\
153	0.00997161957427547\\
154	0.00997161957424098\\
155	0.00997161957420589\\
156	0.00997161957417016\\
157	0.00997161957413379\\
158	0.00997161957409678\\
159	0.0099716195740591\\
160	0.00997161957402075\\
161	0.00997161957398172\\
162	0.00997161957394199\\
163	0.00997161957390155\\
164	0.00997161957386039\\
165	0.00997161957381849\\
166	0.00997161957377584\\
167	0.00997161957373243\\
168	0.00997161957368825\\
169	0.00997161957364328\\
170	0.0099716195735975\\
171	0.0099716195735509\\
172	0.00997161957350347\\
173	0.0099716195734552\\
174	0.00997161957340606\\
175	0.00997161957335604\\
176	0.00997161957330513\\
177	0.00997161957325331\\
178	0.00997161957320056\\
179	0.00997161957314686\\
180	0.00997161957309221\\
181	0.00997161957303658\\
182	0.00997161957297995\\
183	0.00997161957292231\\
184	0.00997161957286364\\
185	0.00997161957280392\\
186	0.00997161957274313\\
187	0.00997161957268125\\
188	0.00997161957261826\\
189	0.00997161957255414\\
190	0.00997161957248888\\
191	0.00997161957242245\\
192	0.00997161957235483\\
193	0.00997161957228599\\
194	0.00997161957221593\\
195	0.00997161957214461\\
196	0.009971619572072\\
197	0.0099716195719981\\
198	0.00997161957192287\\
199	0.0099716195718463\\
200	0.00997161957176835\\
201	0.009971619571689\\
202	0.00997161957160823\\
203	0.00997161957152601\\
204	0.00997161957144231\\
205	0.00997161957135711\\
206	0.00997161957127038\\
207	0.0099716195711821\\
208	0.00997161957109222\\
209	0.00997161957100074\\
210	0.00997161957090761\\
211	0.0099716195708128\\
212	0.00997161957071629\\
213	0.00997161957061805\\
214	0.00997161957051804\\
215	0.00997161957041623\\
216	0.00997161957031259\\
217	0.00997161957020709\\
218	0.00997161957009968\\
219	0.00997161956999034\\
220	0.00997161956987904\\
221	0.00997161956976572\\
222	0.00997161956965037\\
223	0.00997161956953294\\
224	0.00997161956941339\\
225	0.00997161956929168\\
226	0.00997161956916778\\
227	0.00997161956904165\\
228	0.00997161956891324\\
229	0.00997161956878251\\
230	0.00997161956864942\\
231	0.00997161956851392\\
232	0.00997161956837598\\
233	0.00997161956823555\\
234	0.00997161956809257\\
235	0.00997161956794702\\
236	0.00997161956779882\\
237	0.00997161956764795\\
238	0.00997161956749435\\
239	0.00997161956733796\\
240	0.00997161956717874\\
241	0.00997161956701664\\
242	0.0099716195668516\\
243	0.00997161956668356\\
244	0.00997161956651248\\
245	0.0099716195663383\\
246	0.00997161956616095\\
247	0.00997161956598039\\
248	0.00997161956579654\\
249	0.00997161956560935\\
250	0.00997161956541876\\
251	0.0099716195652247\\
252	0.00997161956502711\\
253	0.00997161956482593\\
254	0.00997161956462109\\
255	0.00997161956441251\\
256	0.00997161956420013\\
257	0.00997161956398388\\
258	0.00997161956376369\\
259	0.00997161956353948\\
260	0.00997161956331118\\
261	0.00997161956307871\\
262	0.00997161956284199\\
263	0.00997161956260094\\
264	0.00997161956235549\\
265	0.00997161956210555\\
266	0.00997161956185104\\
267	0.00997161956159186\\
268	0.00997161956132794\\
269	0.00997161956105918\\
270	0.0099716195607855\\
271	0.00997161956050679\\
272	0.00997161956022297\\
273	0.00997161955993394\\
274	0.0099716195596396\\
275	0.00997161955933985\\
276	0.00997161955903459\\
277	0.00997161955872372\\
278	0.00997161955840713\\
279	0.00997161955808471\\
280	0.00997161955775634\\
281	0.00997161955742193\\
282	0.00997161955708135\\
283	0.00997161955673449\\
284	0.00997161955638123\\
285	0.00997161955602145\\
286	0.00997161955565502\\
287	0.00997161955528182\\
288	0.00997161955490172\\
289	0.00997161955451459\\
290	0.0099716195541203\\
291	0.00997161955371871\\
292	0.00997161955330968\\
293	0.00997161955289307\\
294	0.00997161955246873\\
295	0.00997161955203653\\
296	0.00997161955159631\\
297	0.00997161955114792\\
298	0.0099716195506912\\
299	0.009971619550226\\
300	0.00997161954975216\\
301	0.0099716195492695\\
302	0.00997161954877787\\
303	0.00997161954827709\\
304	0.009971619547767\\
305	0.0099716195472474\\
306	0.00997161954671813\\
307	0.009971619546179\\
308	0.00997161954562983\\
309	0.00997161954507042\\
310	0.00997161954450058\\
311	0.00997161954392011\\
312	0.00997161954332881\\
313	0.00997161954272648\\
314	0.00997161954211291\\
315	0.00997161954148789\\
316	0.00997161954085119\\
317	0.00997161954020261\\
318	0.00997161953954192\\
319	0.00997161953886889\\
320	0.00997161953818328\\
321	0.00997161953748487\\
322	0.00997161953677341\\
323	0.00997161953604865\\
324	0.00997161953531036\\
325	0.00997161953455827\\
326	0.00997161953379213\\
327	0.00997161953301168\\
328	0.00997161953221665\\
329	0.00997161953140676\\
330	0.00997161953058175\\
331	0.00997161952974133\\
332	0.00997161952888521\\
333	0.00997161952801311\\
334	0.00997161952712473\\
335	0.00997161952621976\\
336	0.00997161952529791\\
337	0.00997161952435886\\
338	0.00997161952340229\\
339	0.00997161952242789\\
340	0.00997161952143532\\
341	0.00997161952042424\\
342	0.00997161951939433\\
343	0.00997161951834523\\
344	0.00997161951727659\\
345	0.00997161951618805\\
346	0.00997161951507926\\
347	0.00997161951394983\\
348	0.0099716195127994\\
349	0.00997161951162757\\
350	0.00997161951043396\\
351	0.00997161950921816\\
352	0.00997161950797978\\
353	0.00997161950671839\\
354	0.00997161950543359\\
355	0.00997161950412493\\
356	0.00997161950279198\\
357	0.0099716195014343\\
358	0.00997161950005143\\
359	0.0099716194986429\\
360	0.00997161949720825\\
361	0.00997161949574699\\
362	0.00997161949425864\\
363	0.00997161949274267\\
364	0.00997161949119859\\
365	0.00997161948962585\\
366	0.00997161948802394\\
367	0.00997161948639229\\
368	0.00997161948473033\\
369	0.0099716194830375\\
370	0.00997161948131319\\
371	0.00997161947955681\\
372	0.00997161947776771\\
373	0.00997161947594527\\
374	0.00997161947408882\\
375	0.00997161947219767\\
376	0.00997161947027114\\
377	0.00997161946830849\\
378	0.00997161946630897\\
379	0.00997161946427182\\
380	0.00997161946219624\\
381	0.00997161946008139\\
382	0.00997161945792643\\
383	0.00997161945573044\\
384	0.00997161945349252\\
385	0.00997161945121168\\
386	0.00997161944888691\\
387	0.00997161944651715\\
388	0.00997161944410129\\
389	0.00997161944163815\\
390	0.00997161943912649\\
391	0.00997161943656502\\
392	0.00997161943395236\\
393	0.00997161943128708\\
394	0.00997161942856766\\
395	0.00997161942579248\\
396	0.00997161942295982\\
397	0.00997161942006787\\
398	0.00997161941711471\\
399	0.00997161941409832\\
400	0.00997161941101658\\
401	0.00997161940786726\\
402	0.00997161940464805\\
403	0.00997161940135647\\
404	0.00997161939798994\\
405	0.00997161939454569\\
406	0.00997161939102071\\
407	0.00997161938741182\\
408	0.00997161938371568\\
409	0.00997161937992879\\
410	0.00997161937604739\\
411	0.00997161937206736\\
412	0.00997161936798425\\
413	0.0099716193637932\\
414	0.00997161935948889\\
415	0.00997161935506546\\
416	0.00997161935051644\\
417	0.00997161934583465\\
418	0.00997161934101207\\
419	0.00997161933603957\\
420	0.00997161933090659\\
421	0.00997161932560033\\
422	0.00997161932010432\\
423	0.00997161931439516\\
424	0.00997161930843594\\
425	0.00997161930216289\\
426	0.00997161929546067\\
427	0.00997161928812169\\
428	0.00997161927979449\\
429	0.00997161926995912\\
430	0.00997161925803873\\
431	0.00997161924380988\\
432	0.00997161922798527\\
433	0.0099716192117893\\
434	0.00997161919520971\\
435	0.00997161917823361\\
436	0.00997161916084743\\
437	0.00997161914303685\\
438	0.00997161912478671\\
439	0.00997161910608071\\
440	0.00997161908690088\\
441	0.009971619067226\\
442	0.00997161904702785\\
443	0.00997161902626157\\
444	0.00997161900484153\\
445	0.0099716189825827\\
446	0.00997161895906149\\
447	0.00997161893330036\\
448	0.00997161890309868\\
449	0.0099716188637618\\
450	0.00997161880616921\\
451	0.00997161871540505\\
452	0.00997161857526395\\
453	0.00997161838923931\\
454	0.00997161819246043\\
455	0.00997161799184835\\
456	0.00997161778735009\\
457	0.00997161757890393\\
458	0.00997161736643592\\
459	0.0099716171498591\\
460	0.00997161692908099\\
461	0.00997161670402402\\
462	0.00997161647464711\\
463	0.00997161624092234\\
464	0.00997161600275786\\
465	0.00997161576002339\\
466	0.00997161551258276\\
467	0.00997161526029385\\
468	0.00997161500300719\\
469	0.00997161474056532\\
470	0.00997161447280418\\
471	0.00997161419955083\\
472	0.00997161392061916\\
473	0.00997161363580753\\
474	0.00997161334490667\\
475	0.00997161304770836\\
476	0.00997161274399075\\
477	0.00997161243351716\\
478	0.0099716121160347\\
479	0.00997161179127267\\
480	0.00997161145894094\\
481	0.0099716111187279\\
482	0.00997161077029829\\
483	0.00997161041329071\\
484	0.00997161004731471\\
485	0.00997160967194749\\
486	0.00997160928672995\\
487	0.00997160889116217\\
488	0.00997160848469798\\
489	0.00997160806673853\\
490	0.00997160763662451\\
491	0.00997160719362658\\
492	0.00997160673693233\\
493	0.00997160626562629\\
494	0.00997160577865465\\
495	0.00997160527475635\\
496	0.00997160475231453\\
497	0.00997160420902367\\
498	0.00997160364114093\\
499	0.00997160304184654\\
500	0.00997160239785637\\
501	0.00997160168313578\\
502	0.00997160084956987\\
503	0.00997159982071879\\
504	0.00997159851454787\\
505	0.00997159694684271\\
506	0.00997159530434016\\
507	0.00997159362992084\\
508	0.00997159192143315\\
509	0.00997159017673463\\
510	0.00997158839407049\\
511	0.00997158657149256\\
512	0.00997158470680587\\
513	0.0099715827973843\\
514	0.00997158083970593\\
515	0.00997157882813671\\
516	0.00997157675165689\\
517	0.00997157458510792\\
518	0.00997157226624063\\
519	0.00997156963771637\\
520	0.00997156631032975\\
521	0.00997156138457766\\
522	0.0099715530944917\\
523	0.00997152698602179\\
524	0.00997147061397579\\
525	0.00997141254471425\\
526	0.00997135264997616\\
527	0.00997129076489653\\
528	0.00997122666723394\\
529	0.00997116006398247\\
530	0.00997109063563724\\
531	0.00997101822657742\\
532	0.00997094293052938\\
533	0.00997086429853465\\
534	0.00997078147445818\\
535	0.0099706928507203\\
536	0.00997059531120179\\
537	0.0099704833724489\\
538	0.00997011877376545\\
539	0.00996956491736756\\
540	0.00996899349270498\\
541	0.0099684029522671\\
542	0.00996779153206966\\
543	0.00996715721350507\\
544	0.00996649765959029\\
545	0.0099658100547804\\
546	0.00996509076836515\\
547	0.00996433481825704\\
548	0.00996353501546143\\
549	0.00996268052485097\\
550	0.00995982946921449\\
551	0.00994928346810332\\
552	0.00993880478490235\\
553	0.00992839502085653\\
554	0.00991804764840107\\
555	0.00990772740176818\\
556	0.00989599893488054\\
557	0.00988414242333609\\
558	0.00987228471722131\\
559	0.00986039963374106\\
560	0.00984692207798087\\
561	0.00982189670062208\\
562	0.00979625285445954\\
563	0.00977010122609225\\
564	0.00974349183507356\\
565	0.00971621872061662\\
566	0.00954423054780713\\
567	0.00923799364088698\\
568	0.00891475591122446\\
569	0.00857221754723436\\
570	0.00820767331375929\\
571	0.00781792641370085\\
572	0.00739919452067517\\
573	0.00694701835172799\\
574	0.00647404975102378\\
575	0.00598163137433479\\
576	0.00546809828280403\\
577	0.00493132419558792\\
578	0.0043693721128753\\
579	0.00379467588821947\\
580	0.00323888479067085\\
581	0.00297527177174917\\
582	0.00271736952479058\\
583	0.0024736950881471\\
584	0.00224822513408157\\
585	0.00202430503301372\\
586	0.00180307937177323\\
587	0.00158583765634778\\
588	0.00137394593328041\\
589	0.00116901387489065\\
590	0.000969713245411163\\
591	0.000778872544232193\\
592	0.000600233837811078\\
593	0.000437281366763225\\
594	0.000294147637454635\\
595	0.000175106077570684\\
596	8.23604936306264e-05\\
597	2.09371357655843e-05\\
598	8.08076597025309e-08\\
599	0\\
600	0\\
};
\addplot [color=mycolor16,solid,forget plot]
  table[row sep=crcr]{%
1	0.00997088323167744\\
2	0.00997088323162574\\
3	0.00997088323157311\\
4	0.00997088323151954\\
5	0.00997088323146502\\
6	0.00997088323140951\\
7	0.00997088323135302\\
8	0.00997088323129551\\
9	0.00997088323123698\\
10	0.0099708832311774\\
11	0.00997088323111675\\
12	0.00997088323105503\\
13	0.00997088323099219\\
14	0.00997088323092824\\
15	0.00997088323086314\\
16	0.00997088323079688\\
17	0.00997088323072943\\
18	0.00997088323066078\\
19	0.0099708832305909\\
20	0.00997088323051977\\
21	0.00997088323044737\\
22	0.00997088323037368\\
23	0.00997088323029867\\
24	0.00997088323022232\\
25	0.00997088323014461\\
26	0.0099708832300655\\
27	0.00997088322998499\\
28	0.00997088322990303\\
29	0.00997088322981961\\
30	0.0099708832297347\\
31	0.00997088322964827\\
32	0.0099708832295603\\
33	0.00997088322947076\\
34	0.00997088322937961\\
35	0.00997088322928684\\
36	0.00997088322919241\\
37	0.0099708832290963\\
38	0.00997088322899846\\
39	0.00997088322889888\\
40	0.00997088322879752\\
41	0.00997088322869435\\
42	0.00997088322858934\\
43	0.00997088322848245\\
44	0.00997088322837365\\
45	0.00997088322826291\\
46	0.00997088322815019\\
47	0.00997088322803546\\
48	0.00997088322791868\\
49	0.00997088322779981\\
50	0.00997088322767882\\
51	0.00997088322755567\\
52	0.00997088322743031\\
53	0.00997088322730273\\
54	0.00997088322717286\\
55	0.00997088322704067\\
56	0.00997088322690612\\
57	0.00997088322676918\\
58	0.00997088322662978\\
59	0.0099708832264879\\
60	0.00997088322634348\\
61	0.00997088322619648\\
62	0.00997088322604687\\
63	0.00997088322589458\\
64	0.00997088322573956\\
65	0.00997088322558179\\
66	0.00997088322542119\\
67	0.00997088322525773\\
68	0.00997088322509135\\
69	0.009970883224922\\
70	0.00997088322474963\\
71	0.00997088322457418\\
72	0.0099708832243956\\
73	0.00997088322421383\\
74	0.00997088322402882\\
75	0.0099708832238405\\
76	0.00997088322364883\\
77	0.00997088322345373\\
78	0.00997088322325515\\
79	0.00997088322305303\\
80	0.00997088322284729\\
81	0.00997088322263789\\
82	0.00997088322242475\\
83	0.00997088322220781\\
84	0.00997088322198699\\
85	0.00997088322176224\\
86	0.00997088322153347\\
87	0.00997088322130062\\
88	0.00997088322106361\\
89	0.00997088322082238\\
90	0.00997088322057684\\
91	0.00997088322032692\\
92	0.00997088322007254\\
93	0.00997088321981362\\
94	0.00997088321955008\\
95	0.00997088321928184\\
96	0.00997088321900881\\
97	0.00997088321873092\\
98	0.00997088321844806\\
99	0.00997088321816015\\
100	0.00997088321786711\\
101	0.00997088321756884\\
102	0.00997088321726525\\
103	0.00997088321695624\\
104	0.00997088321664172\\
105	0.00997088321632159\\
106	0.00997088321599575\\
107	0.00997088321566409\\
108	0.00997088321532652\\
109	0.00997088321498292\\
110	0.00997088321463319\\
111	0.00997088321427722\\
112	0.00997088321391491\\
113	0.00997088321354613\\
114	0.00997088321317076\\
115	0.00997088321278871\\
116	0.00997088321239984\\
117	0.00997088321200402\\
118	0.00997088321160115\\
119	0.00997088321119109\\
120	0.00997088321077372\\
121	0.00997088321034889\\
122	0.00997088320991649\\
123	0.00997088320947638\\
124	0.00997088320902841\\
125	0.00997088320857245\\
126	0.00997088320810836\\
127	0.00997088320763598\\
128	0.00997088320715518\\
129	0.0099708832066658\\
130	0.00997088320616768\\
131	0.00997088320566068\\
132	0.00997088320514464\\
133	0.00997088320461938\\
134	0.00997088320408476\\
135	0.00997088320354059\\
136	0.00997088320298672\\
137	0.00997088320242296\\
138	0.00997088320184914\\
139	0.00997088320126508\\
140	0.0099708832006706\\
141	0.00997088320006551\\
142	0.00997088319944962\\
143	0.00997088319882274\\
144	0.00997088319818468\\
145	0.00997088319753522\\
146	0.00997088319687417\\
147	0.00997088319620133\\
148	0.00997088319551647\\
149	0.00997088319481939\\
150	0.00997088319410987\\
151	0.00997088319338768\\
152	0.0099708831926526\\
153	0.00997088319190439\\
154	0.00997088319114283\\
155	0.00997088319036767\\
156	0.00997088318957867\\
157	0.00997088318877558\\
158	0.00997088318795815\\
159	0.00997088318712612\\
160	0.00997088318627923\\
161	0.00997088318541722\\
162	0.00997088318453982\\
163	0.00997088318364673\\
164	0.0099708831827377\\
165	0.00997088318181243\\
166	0.00997088318087063\\
167	0.009970883179912\\
168	0.00997088317893624\\
169	0.00997088317794305\\
170	0.00997088317693211\\
171	0.0099708831759031\\
172	0.0099708831748557\\
173	0.00997088317378957\\
174	0.0099708831727044\\
175	0.00997088317159981\\
176	0.00997088317047549\\
177	0.00997088316933105\\
178	0.00997088316816615\\
179	0.00997088316698042\\
180	0.00997088316577349\\
181	0.00997088316454496\\
182	0.00997088316329445\\
183	0.00997088316202157\\
184	0.00997088316072592\\
185	0.00997088315940708\\
186	0.00997088315806464\\
187	0.00997088315669816\\
188	0.00997088315530723\\
189	0.00997088315389139\\
190	0.0099708831524502\\
191	0.00997088315098321\\
192	0.00997088314948993\\
193	0.00997088314796991\\
194	0.00997088314642266\\
195	0.00997088314484768\\
196	0.00997088314324448\\
197	0.00997088314161254\\
198	0.00997088313995135\\
199	0.00997088313826038\\
200	0.00997088313653909\\
201	0.00997088313478692\\
202	0.00997088313300333\\
203	0.00997088313118774\\
204	0.00997088312933956\\
205	0.00997088312745823\\
206	0.00997088312554312\\
207	0.00997088312359363\\
208	0.00997088312160913\\
209	0.00997088311958899\\
210	0.00997088311753257\\
211	0.00997088311543919\\
212	0.0099708831133082\\
213	0.0099708831111389\\
214	0.00997088310893061\\
215	0.0099708831066826\\
216	0.00997088310439416\\
217	0.00997088310206455\\
218	0.00997088309969303\\
219	0.00997088309727882\\
220	0.00997088309482114\\
221	0.00997088309231922\\
222	0.00997088308977223\\
223	0.00997088308717935\\
224	0.00997088308453975\\
225	0.00997088308185258\\
226	0.00997088307911695\\
227	0.00997088307633199\\
228	0.00997088307349678\\
229	0.00997088307061043\\
230	0.00997088306767197\\
231	0.00997088306468046\\
232	0.00997088306163492\\
233	0.00997088305853436\\
234	0.00997088305537778\\
235	0.00997088305216413\\
236	0.00997088304889238\\
237	0.00997088304556145\\
238	0.00997088304217025\\
239	0.00997088303871768\\
240	0.0099708830352026\\
241	0.00997088303162385\\
242	0.00997088302798027\\
243	0.00997088302427066\\
244	0.00997088302049379\\
245	0.00997088301664842\\
246	0.00997088301273328\\
247	0.00997088300874709\\
248	0.00997088300468853\\
249	0.00997088300055625\\
250	0.0099708829963489\\
251	0.00997088299206507\\
252	0.00997088298770334\\
253	0.00997088298326227\\
254	0.00997088297874038\\
255	0.00997088297413617\\
256	0.00997088296944811\\
257	0.00997088296467463\\
258	0.00997088295981415\\
259	0.00997088295486503\\
260	0.00997088294982563\\
261	0.00997088294469426\\
262	0.0099708829394692\\
263	0.0099708829341487\\
264	0.00997088292873097\\
265	0.00997088292321419\\
266	0.00997088291759651\\
267	0.00997088291187604\\
268	0.00997088290605084\\
269	0.00997088290011895\\
270	0.00997088289407838\\
271	0.00997088288792707\\
272	0.00997088288166294\\
273	0.00997088287528387\\
274	0.00997088286878771\\
275	0.00997088286217224\\
276	0.00997088285543521\\
277	0.00997088284857433\\
278	0.00997088284158727\\
279	0.00997088283447165\\
280	0.00997088282722504\\
281	0.00997088281984496\\
282	0.00997088281232889\\
283	0.00997088280467426\\
284	0.00997088279687844\\
285	0.00997088278893878\\
286	0.00997088278085254\\
287	0.00997088277261694\\
288	0.00997088276422917\\
289	0.00997088275568633\\
290	0.00997088274698548\\
291	0.00997088273812363\\
292	0.00997088272909773\\
293	0.00997088271990466\\
294	0.00997088271054125\\
295	0.00997088270100426\\
296	0.00997088269129041\\
297	0.00997088268139634\\
298	0.00997088267131862\\
299	0.00997088266105377\\
300	0.00997088265059823\\
301	0.00997088263994839\\
302	0.00997088262910056\\
303	0.00997088261805098\\
304	0.00997088260679582\\
305	0.00997088259533118\\
306	0.00997088258365309\\
307	0.00997088257175749\\
308	0.00997088255964028\\
309	0.00997088254729723\\
310	0.00997088253472409\\
311	0.00997088252191648\\
312	0.00997088250886998\\
313	0.00997088249558006\\
314	0.00997088248204213\\
315	0.00997088246825149\\
316	0.00997088245420338\\
317	0.00997088243989294\\
318	0.00997088242531522\\
319	0.00997088241046518\\
320	0.00997088239533771\\
321	0.00997088237992758\\
322	0.00997088236422948\\
323	0.00997088234823802\\
324	0.00997088233194767\\
325	0.00997088231535286\\
326	0.00997088229844787\\
327	0.00997088228122692\\
328	0.00997088226368409\\
329	0.00997088224581339\\
330	0.0099708822276087\\
331	0.00997088220906381\\
332	0.0099708821901724\\
333	0.00997088217092802\\
334	0.00997088215132413\\
335	0.00997088213135407\\
336	0.00997088211101106\\
337	0.0099708820902882\\
338	0.00997088206917847\\
339	0.00997088204767474\\
340	0.00997088202576975\\
341	0.00997088200345609\\
342	0.00997088198072625\\
343	0.00997088195757258\\
344	0.00997088193398729\\
345	0.00997088190996246\\
346	0.00997088188549002\\
347	0.00997088186056175\\
348	0.00997088183516932\\
349	0.00997088180930421\\
350	0.00997088178295776\\
351	0.00997088175612117\\
352	0.00997088172878545\\
353	0.00997088170094148\\
354	0.00997088167257993\\
355	0.00997088164369133\\
356	0.00997088161426602\\
357	0.00997088158429415\\
358	0.00997088155376568\\
359	0.00997088152267038\\
360	0.0099708814909978\\
361	0.00997088145873731\\
362	0.00997088142587804\\
363	0.00997088139240887\\
364	0.0099708813583185\\
365	0.00997088132359532\\
366	0.00997088128822752\\
367	0.00997088125220297\\
368	0.00997088121550929\\
369	0.00997088117813379\\
370	0.00997088114006348\\
371	0.00997088110128504\\
372	0.00997088106178478\\
373	0.00997088102154868\\
374	0.00997088098056233\\
375	0.0099708809388109\\
376	0.00997088089627912\\
377	0.0099708808529513\\
378	0.00997088080881122\\
379	0.00997088076384217\\
380	0.00997088071802688\\
381	0.00997088067134747\\
382	0.00997088062378544\\
383	0.00997088057532162\\
384	0.00997088052593608\\
385	0.00997088047560811\\
386	0.00997088042431613\\
387	0.00997088037203762\\
388	0.00997088031874902\\
389	0.0099708802644256\\
390	0.00997088020904137\\
391	0.00997088015256893\\
392	0.0099708800949794\\
393	0.00997088003624245\\
394	0.00997087997632629\\
395	0.0099708799151974\\
396	0.00997087985282\\
397	0.00997087978915599\\
398	0.00997087972416513\\
399	0.00997087965780497\\
400	0.00997087959003072\\
401	0.00997087952079525\\
402	0.00997087945004903\\
403	0.00997087937774015\\
404	0.00997087930381426\\
405	0.00997087922821412\\
406	0.00997087915087875\\
407	0.00997087907174232\\
408	0.00997087899073419\\
409	0.0099708789077808\\
410	0.00997087882280568\\
411	0.00997087873572631\\
412	0.00997087864645341\\
413	0.00997087855489009\\
414	0.00997087846093081\\
415	0.00997087836446028\\
416	0.00997087826535206\\
417	0.00997087816346695\\
418	0.00997087805865112\\
419	0.0099708779507336\\
420	0.00997087783952304\\
421	0.00997087772480275\\
422	0.00997087760632211\\
423	0.00997087748377952\\
424	0.00997087735678526\\
425	0.00997087722477763\\
426	0.00997087708683273\\
427	0.00997087694124665\\
428	0.00997087678467685\\
429	0.00997087661058599\\
430	0.00997087640710714\\
431	0.00997087615635543\\
432	0.0099708758423248\\
433	0.00997087547616931\\
434	0.00997087510142908\\
435	0.00997087471782217\\
436	0.00997087432505228\\
437	0.00997087392280794\\
438	0.00997087351076154\\
439	0.0099708730885682\\
440	0.00997087265586409\\
441	0.00997087221226296\\
442	0.00997087175734777\\
443	0.00997087129064845\\
444	0.00997087081158073\\
445	0.00997087031927729\\
446	0.00997086981212308\\
447	0.00997086928649015\\
448	0.00997086873334978\\
449	0.009970868129446\\
450	0.00997086741534181\\
451	0.00997086644535829\\
452	0.00997086489407962\\
453	0.00997086218392809\\
454	0.00997084995943376\\
455	0.00997083558687619\\
456	0.00997082092872587\\
457	0.00997080598144195\\
458	0.00997079074116459\\
459	0.00997077520351485\\
460	0.00997075936338623\\
461	0.00997074321487689\\
462	0.0099707267516662\\
463	0.00997070996794885\\
464	0.00997069285835003\\
465	0.00997067541479627\\
466	0.0099706576273452\\
467	0.00997063948549053\\
468	0.00997062097813579\\
469	0.009970602093518\\
470	0.00997058281917374\\
471	0.0099705631420409\\
472	0.0099705430484314\\
473	0.00997052252394717\\
474	0.00997050155338471\\
475	0.00997048012095881\\
476	0.00997045821066169\\
477	0.00997043580545834\\
478	0.00997041288719576\\
479	0.00997038943650305\\
480	0.00997036543268011\\
481	0.00997034085357304\\
482	0.00997031567543413\\
483	0.00997028987276406\\
484	0.00997026341813321\\
485	0.0099702362819785\\
486	0.00997020843237057\\
487	0.00997017983474504\\
488	0.0099701504515919\\
489	0.00997012024209578\\
490	0.00997008916171321\\
491	0.00997005716168147\\
492	0.00997002418844186\\
493	0.00996999018296104\\
494	0.00996995507989501\\
495	0.0099699188065099\\
496	0.009969881281206\\
497	0.00996984241118437\\
498	0.00996980208812423\\
499	0.00996976017897087\\
500	0.00996971650451421\\
501	0.00996967078798026\\
502	0.00996962253368807\\
503	0.00996957076039186\\
504	0.00996951351513884\\
505	0.00996944750423008\\
506	0.00996919728888433\\
507	0.00996889505580105\\
508	0.0099685857996402\\
509	0.00996826915349537\\
510	0.00996794473301398\\
511	0.00996761215066056\\
512	0.00996727098548103\\
513	0.00996692078043861\\
514	0.00996656103749073\\
515	0.00996619121147095\\
516	0.00996581070127001\\
517	0.0099654188311917\\
518	0.00996501480127995\\
519	0.00996459753632495\\
520	0.00996416519652461\\
521	0.00996371353764319\\
522	0.00996323029610212\\
523	0.00996244014169892\\
524	0.00996106847533402\\
525	0.00995965184131424\\
526	0.00995818670183677\\
527	0.00995666890172792\\
528	0.00995509333603592\\
529	0.00995345327968193\\
530	0.00995173930274375\\
531	0.00994993854347067\\
532	0.00994198918067423\\
533	0.0099333698123967\\
534	0.00992466993527836\\
535	0.00991588581114185\\
536	0.00990700764409235\\
537	0.00989800040202258\\
538	0.00988423238652592\\
539	0.00986682912216097\\
540	0.0098490867747136\\
541	0.00983097299927687\\
542	0.00981245095406175\\
543	0.00979347863885308\\
544	0.00977400837567323\\
545	0.00975398593359851\\
546	0.00973334753349623\\
547	0.0097120141682128\\
548	0.009689887464935\\
549	0.00966683268985188\\
550	0.0096028502829079\\
551	0.00937383838563574\\
552	0.00913506924402005\\
553	0.00888568462608105\\
554	0.00862460118680686\\
555	0.00835057812394875\\
556	0.00806368187881237\\
557	0.00776086298557798\\
558	0.00744001059100445\\
559	0.00709886523164867\\
560	0.00673645084559179\\
561	0.00636019377928592\\
562	0.00595462410584711\\
563	0.00552385957210717\\
564	0.00507376490521309\\
565	0.00460223454999223\\
566	0.00425791895105631\\
567	0.00403202377162496\\
568	0.00380434847942717\\
569	0.00357645800057989\\
570	0.00335022643385689\\
571	0.00312853163689701\\
572	0.00291532356463257\\
573	0.00271595703756421\\
574	0.00251759306661718\\
575	0.00231909725360597\\
576	0.00212378268879351\\
577	0.00193655051480691\\
578	0.00176016692047821\\
579	0.00159741174489819\\
580	0.00145069524987436\\
581	0.0013122533768107\\
582	0.00118242050607405\\
583	0.00105767167839788\\
584	0.000936755859961898\\
585	0.000821067448764956\\
586	0.000711892271490388\\
587	0.000609730980753466\\
588	0.000516008457521624\\
589	0.000431104662045106\\
590	0.000354874504246777\\
591	0.00028727906894567\\
592	0.000227243916597471\\
593	0.000174151245484172\\
594	0.000127317437376423\\
595	8.62418326778849e-05\\
596	5.06250627329885e-05\\
597	2.09371357655843e-05\\
598	8.08076597025309e-08\\
599	0\\
600	0\\
};
\addplot [color=mycolor17,solid,forget plot]
  table[row sep=crcr]{%
1	0.0099697519045818\\
2	0.00996975190073246\\
3	0.0099697518968143\\
4	0.00996975189282606\\
5	0.00996975188876652\\
6	0.0099697518846344\\
7	0.00996975188042839\\
8	0.00996975187614718\\
9	0.00996975187178944\\
10	0.00996975186735378\\
11	0.00996975186283883\\
12	0.00996975185824316\\
13	0.00996975185356534\\
14	0.00996975184880389\\
15	0.00996975184395732\\
16	0.00996975183902413\\
17	0.00996975183400275\\
18	0.00996975182889162\\
19	0.00996975182368913\\
20	0.00996975181839366\\
21	0.00996975181300353\\
22	0.00996975180751708\\
23	0.00996975180193257\\
24	0.00996975179624825\\
25	0.00996975179046235\\
26	0.00996975178457304\\
27	0.0099697517785785\\
28	0.00996975177247683\\
29	0.00996975176626613\\
30	0.00996975175994444\\
31	0.0099697517535098\\
32	0.00996975174696019\\
33	0.00996975174029354\\
34	0.00996975173350779\\
35	0.00996975172660079\\
36	0.00996975171957039\\
37	0.00996975171241439\\
38	0.00996975170513054\\
39	0.00996975169771657\\
40	0.00996975169017015\\
41	0.00996975168248891\\
42	0.00996975167467047\\
43	0.00996975166671236\\
44	0.0099697516586121\\
45	0.00996975165036715\\
46	0.00996975164197493\\
47	0.00996975163343281\\
48	0.00996975162473812\\
49	0.00996975161588813\\
50	0.00996975160688009\\
51	0.00996975159771116\\
52	0.00996975158837848\\
53	0.00996975157887914\\
54	0.00996975156921014\\
55	0.00996975155936849\\
56	0.00996975154935108\\
57	0.00996975153915478\\
58	0.00996975152877642\\
59	0.00996975151821273\\
60	0.00996975150746042\\
61	0.00996975149651613\\
62	0.00996975148537642\\
63	0.00996975147403781\\
64	0.00996975146249676\\
65	0.00996975145074965\\
66	0.00996975143879282\\
67	0.00996975142662252\\
68	0.00996975141423494\\
69	0.00996975140162622\\
70	0.00996975138879241\\
71	0.00996975137572949\\
72	0.00996975136243338\\
73	0.00996975134889992\\
74	0.00996975133512488\\
75	0.00996975132110396\\
76	0.00996975130683276\\
77	0.00996975129230683\\
78	0.00996975127752162\\
79	0.00996975126247251\\
80	0.0099697512471548\\
81	0.00996975123156369\\
82	0.00996975121569432\\
83	0.00996975119954171\\
84	0.00996975118310083\\
85	0.00996975116636652\\
86	0.00996975114933357\\
87	0.00996975113199663\\
88	0.0099697511143503\\
89	0.00996975109638906\\
90	0.00996975107810729\\
91	0.00996975105949927\\
92	0.0099697510405592\\
93	0.00996975102128114\\
94	0.00996975100165907\\
95	0.00996975098168686\\
96	0.00996975096135827\\
97	0.00996975094066693\\
98	0.00996975091960638\\
99	0.00996975089817004\\
100	0.00996975087635121\\
101	0.00996975085414306\\
102	0.00996975083153865\\
103	0.00996975080853092\\
104	0.00996975078511266\\
105	0.00996975076127657\\
106	0.00996975073701518\\
107	0.00996975071232092\\
108	0.00996975068718606\\
109	0.00996975066160275\\
110	0.00996975063556297\\
111	0.0099697506090586\\
112	0.00996975058208135\\
113	0.00996975055462277\\
114	0.00996975052667429\\
115	0.00996975049822715\\
116	0.00996975046927248\\
117	0.0099697504398012\\
118	0.00996975040980411\\
119	0.00996975037927181\\
120	0.00996975034819477\\
121	0.00996975031656326\\
122	0.00996975028436738\\
123	0.00996975025159706\\
124	0.00996975021824206\\
125	0.00996975018429192\\
126	0.00996975014973604\\
127	0.0099697501145636\\
128	0.00996975007876359\\
129	0.0099697500423248\\
130	0.00996975000523584\\
131	0.00996974996748509\\
132	0.00996974992906073\\
133	0.00996974988995075\\
134	0.00996974985014287\\
135	0.00996974980962466\\
136	0.00996974976838341\\
137	0.00996974972640622\\
138	0.00996974968367992\\
139	0.00996974964019115\\
140	0.00996974959592627\\
141	0.00996974955087141\\
142	0.00996974950501246\\
143	0.00996974945833504\\
144	0.00996974941082453\\
145	0.00996974936246603\\
146	0.00996974931324437\\
147	0.00996974926314413\\
148	0.00996974921214958\\
149	0.00996974916024475\\
150	0.00996974910741334\\
151	0.00996974905363877\\
152	0.00996974899890417\\
153	0.00996974894319237\\
154	0.00996974888648587\\
155	0.00996974882876687\\
156	0.00996974877001724\\
157	0.00996974871021853\\
158	0.00996974864935195\\
159	0.00996974858739839\\
160	0.00996974852433836\\
161	0.00996974846015205\\
162	0.00996974839481927\\
163	0.00996974832831948\\
164	0.00996974826063176\\
165	0.00996974819173482\\
166	0.00996974812160697\\
167	0.00996974805022614\\
168	0.00996974797756986\\
169	0.00996974790361524\\
170	0.00996974782833898\\
171	0.00996974775171736\\
172	0.00996974767372624\\
173	0.00996974759434103\\
174	0.00996974751353669\\
175	0.00996974743128772\\
176	0.00996974734756818\\
177	0.00996974726235164\\
178	0.00996974717561119\\
179	0.00996974708731944\\
180	0.00996974699744848\\
181	0.00996974690596991\\
182	0.00996974681285481\\
183	0.00996974671807372\\
184	0.00996974662159667\\
185	0.0099697465233931\\
186	0.00996974642343193\\
187	0.00996974632168149\\
188	0.00996974621810954\\
189	0.00996974611268325\\
190	0.00996974600536918\\
191	0.00996974589613328\\
192	0.00996974578494088\\
193	0.00996974567175667\\
194	0.00996974555654469\\
195	0.00996974543926834\\
196	0.00996974531989031\\
197	0.00996974519837262\\
198	0.00996974507467661\\
199	0.00996974494876289\\
200	0.00996974482059133\\
201	0.00996974469012108\\
202	0.00996974455731054\\
203	0.00996974442211733\\
204	0.00996974428449828\\
205	0.00996974414440944\\
206	0.00996974400180604\\
207	0.00996974385664246\\
208	0.00996974370887227\\
209	0.00996974355844815\\
210	0.00996974340532192\\
211	0.00996974324944449\\
212	0.00996974309076586\\
213	0.00996974292923512\\
214	0.00996974276480038\\
215	0.00996974259740882\\
216	0.00996974242700659\\
217	0.00996974225353887\\
218	0.0099697420769498\\
219	0.00996974189718248\\
220	0.00996974171417895\\
221	0.00996974152788017\\
222	0.00996974133822597\\
223	0.00996974114515506\\
224	0.00996974094860503\\
225	0.00996974074851224\\
226	0.00996974054481191\\
227	0.00996974033743801\\
228	0.00996974012632328\\
229	0.00996973991139917\\
230	0.00996973969259585\\
231	0.0099697394698422\\
232	0.0099697392430657\\
233	0.00996973901219251\\
234	0.00996973877714735\\
235	0.00996973853785355\\
236	0.00996973829423296\\
237	0.00996973804620596\\
238	0.00996973779369141\\
239	0.00996973753660663\\
240	0.00996973727486737\\
241	0.00996973700838777\\
242	0.00996973673708034\\
243	0.00996973646085591\\
244	0.00996973617962363\\
245	0.00996973589329088\\
246	0.00996973560176329\\
247	0.00996973530494469\\
248	0.00996973500273705\\
249	0.00996973469504048\\
250	0.00996973438175314\\
251	0.00996973406277126\\
252	0.00996973373798909\\
253	0.00996973340729881\\
254	0.00996973307059056\\
255	0.00996973272775232\\
256	0.00996973237866995\\
257	0.0099697320232271\\
258	0.00996973166130517\\
259	0.00996973129278325\\
260	0.00996973091753812\\
261	0.00996973053544418\\
262	0.00996973014637337\\
263	0.00996972975019518\\
264	0.00996972934677655\\
265	0.00996972893598184\\
266	0.00996972851767281\\
267	0.0099697280917085\\
268	0.00996972765794523\\
269	0.00996972721623654\\
270	0.00996972676643309\\
271	0.00996972630838269\\
272	0.00996972584193014\\
273	0.00996972536691725\\
274	0.00996972488318275\\
275	0.00996972439056223\\
276	0.00996972388888807\\
277	0.00996972337798941\\
278	0.00996972285769206\\
279	0.00996972232781842\\
280	0.00996972178818746\\
281	0.00996972123861461\\
282	0.00996972067891173\\
283	0.009969720108887\\
284	0.00996971952834489\\
285	0.00996971893708604\\
286	0.00996971833490726\\
287	0.00996971772160139\\
288	0.00996971709695724\\
289	0.00996971646075956\\
290	0.00996971581278889\\
291	0.00996971515282155\\
292	0.00996971448062951\\
293	0.00996971379598037\\
294	0.00996971309863719\\
295	0.00996971238835851\\
296	0.0099697116648982\\
297	0.00996971092800538\\
298	0.00996971017742438\\
299	0.00996970941289461\\
300	0.00996970863415048\\
301	0.00996970784092136\\
302	0.0099697070329314\\
303	0.00996970620989955\\
304	0.00996970537153937\\
305	0.00996970451755901\\
306	0.0099697036476611\\
307	0.00996970276154264\\
308	0.00996970185889491\\
309	0.00996970093940339\\
310	0.00996970000274768\\
311	0.00996969904860135\\
312	0.0099696980766319\\
313	0.00996969708650065\\
314	0.0099696960778626\\
315	0.00996969505036639\\
316	0.00996969400365416\\
317	0.00996969293736147\\
318	0.00996969185111719\\
319	0.00996969074454339\\
320	0.00996968961725526\\
321	0.00996968846886098\\
322	0.0099696872989616\\
323	0.00996968610715099\\
324	0.00996968489301565\\
325	0.00996968365613468\\
326	0.0099696823960796\\
327	0.00996968111241426\\
328	0.00996967980469473\\
329	0.00996967847246915\\
330	0.00996967711527765\\
331	0.00996967573265217\\
332	0.00996967432411635\\
333	0.0099696728891854\\
334	0.00996967142736598\\
335	0.00996966993815599\\
336	0.00996966842104448\\
337	0.00996966687551148\\
338	0.00996966530102783\\
339	0.009969663697055\\
340	0.00996966206304497\\
341	0.00996966039844001\\
342	0.00996965870267248\\
343	0.00996965697516468\\
344	0.00996965521532864\\
345	0.0099696534225659\\
346	0.00996965159626729\\
347	0.00996964973581276\\
348	0.00996964784057105\\
349	0.00996964590989956\\
350	0.009969643943144\\
351	0.00996964193963817\\
352	0.00996963989870367\\
353	0.00996963781964958\\
354	0.00996963570177215\\
355	0.00996963354435446\\
356	0.00996963134666605\\
357	0.00996962910796253\\
358	0.00996962682748515\\
359	0.00996962450446035\\
360	0.00996962213809932\\
361	0.00996961972759738\\
362	0.00996961727213355\\
363	0.00996961477086983\\
364	0.00996961222295061\\
365	0.00996960962750197\\
366	0.00996960698363089\\
367	0.00996960429042446\\
368	0.00996960154694896\\
369	0.00996959875224894\\
370	0.00996959590534617\\
371	0.00996959300523851\\
372	0.00996959005089868\\
373	0.009969587041273\\
374	0.0099695839752799\\
375	0.0099695808518084\\
376	0.00996957766971643\\
377	0.00996957442782901\\
378	0.00996957112493625\\
379	0.00996956775979118\\
380	0.00996956433110742\\
381	0.00996956083755658\\
382	0.00996955727776543\\
383	0.00996955365031283\\
384	0.00996954995372628\\
385	0.00996954618647817\\
386	0.00996954234698156\\
387	0.00996953843358549\\
388	0.00996953444456968\\
389	0.00996953037813847\\
390	0.00996952623241385\\
391	0.00996952200542743\\
392	0.00996951769511141\\
393	0.00996951329928924\\
394	0.00996950881566731\\
395	0.009969504241828\\
396	0.00996949957521492\\
397	0.0099694948131044\\
398	0.00996948995260198\\
399	0.00996948499064973\\
400	0.00996947992401542\\
401	0.00996947474928193\\
402	0.00996946946283715\\
403	0.00996946406086517\\
404	0.0099694585393389\\
405	0.00996945289401371\\
406	0.00996944712041812\\
407	0.0099694412138342\\
408	0.00996943516926477\\
409	0.00996942898142314\\
410	0.00996942264479276\\
411	0.00996941615360393\\
412	0.00996940950164628\\
413	0.0099694026822156\\
414	0.0099693956880528\\
415	0.00996938851127348\\
416	0.00996938114328666\\
417	0.00996937357470046\\
418	0.00996936579521265\\
419	0.00996935779348277\\
420	0.00996934955698217\\
421	0.00996934107181592\\
422	0.00996933232250734\\
423	0.00996932329172564\\
424	0.0099693139599132\\
425	0.00996930430470253\\
426	0.00996929429983682\\
427	0.00996928391284405\\
428	0.00996927309953203\\
429	0.00996926179055022\\
430	0.00996924985937415\\
431	0.00996923705280914\\
432	0.00996922287533935\\
433	0.0099692022349139\\
434	0.00996913524090377\\
435	0.00996906662955365\\
436	0.00996899634635407\\
437	0.00996892433408144\\
438	0.00996885053266827\\
439	0.00996877487908856\\
440	0.00996869730725644\\
441	0.00996861774795321\\
442	0.00996853612879014\\
443	0.00996845237421559\\
444	0.00996836640556197\\
445	0.00996827814108012\\
446	0.00996818749574176\\
447	0.00996809438004087\\
448	0.00996799869520312\\
449	0.00996790031613258\\
450	0.00996779903285194\\
451	0.00996769435056924\\
452	0.00996758480255674\\
453	0.00996746556180121\\
454	0.00996716819519554\\
455	0.00996682851733857\\
456	0.00996648146788184\\
457	0.00996612692035036\\
458	0.00996576474950728\\
459	0.00996539483135169\\
460	0.00996501704267048\\
461	0.00996463126025148\\
462	0.00996423736223196\\
463	0.00996383524195827\\
464	0.00996342485689183\\
465	0.0099630062774873\\
466	0.00996257939085356\\
467	0.00996214395000095\\
468	0.00996169969369785\\
469	0.00996124634607922\\
470	0.00996078361489439\\
471	0.00996031118999826\\
472	0.0099598287463504\\
473	0.00995933594418152\\
474	0.00995883242797227\\
475	0.00995831782256985\\
476	0.0099577917352517\\
477	0.00995725376967477\\
478	0.00995670350391239\\
479	0.00995614048806059\\
480	0.00995556424159059\\
481	0.00995497425037914\\
482	0.00995436996336483\\
483	0.00995375078876872\\
484	0.0099531160898078\\
485	0.00995246517981726\\
486	0.00995179731667627\\
487	0.00995111169638735\\
488	0.00995040744559872\\
489	0.00994968361287447\\
490	0.00994893915849675\\
491	0.00994817294221437\\
492	0.00994738370892252\\
493	0.00994657007161606\\
494	0.00994573049152282\\
495	0.00994486325401003\\
496	0.00994396643861992\\
497	0.00994303788310322\\
498	0.0099420751374026\\
499	0.0099410753999758\\
500	0.00994003541600568\\
501	0.00993895127734314\\
502	0.00993781793806392\\
503	0.00993662785204834\\
504	0.00993536679887735\\
505	0.00993400027143389\\
506	0.00992880220654646\\
507	0.00991886310307535\\
508	0.00990874931381914\\
509	0.00989845337623101\\
510	0.00988796675750514\\
511	0.00987728087787082\\
512	0.00986638718327344\\
513	0.00985527643973289\\
514	0.00984393872426596\\
515	0.00983236330178472\\
516	0.00982053851508781\\
517	0.0098084517227916\\
518	0.00979608927005429\\
519	0.00978343657101979\\
520	0.00977047856672567\\
521	0.00975720146311516\\
522	0.00974359891240654\\
523	0.0097299489729012\\
524	0.00971653301037959\\
525	0.00970275241589315\\
526	0.00968857598405374\\
527	0.00967395985487108\\
528	0.00965885211634003\\
529	0.00964318799719018\\
530	0.00962687819315973\\
531	0.00960977755046326\\
532	0.00946496781662055\\
533	0.00930215422693526\\
534	0.00913313910001564\\
535	0.00895756757655554\\
536	0.00877510091477893\\
537	0.00858549004733624\\
538	0.00839346799398791\\
539	0.0081976047965105\\
540	0.00799399838265753\\
541	0.00778199712205672\\
542	0.0075607374126223\\
543	0.00732921195875231\\
544	0.00708624014032365\\
545	0.00683043431669419\\
546	0.006560140369778\\
547	0.00627333232033241\\
548	0.00596751074478589\\
549	0.00563978362762557\\
550	0.0053287263354748\\
551	0.0051620566362931\\
552	0.00498682220034265\\
553	0.00480834436971077\\
554	0.00462703817687728\\
555	0.00444347139523713\\
556	0.00425837905347628\\
557	0.00407284228376562\\
558	0.00388823702136661\\
559	0.0037061846201864\\
560	0.00352882374832847\\
561	0.0033589377063064\\
562	0.00320092250471758\\
563	0.00305050299774475\\
564	0.00290148240358262\\
565	0.00275602531448269\\
566	0.00261186017229875\\
567	0.00246626230945929\\
568	0.00232001517339028\\
569	0.00217417708329218\\
570	0.00203106134843138\\
571	0.00189386617328889\\
572	0.00176295494974905\\
573	0.00163817534342585\\
574	0.00152009271402506\\
575	0.00140922861412605\\
576	0.00130493267153831\\
577	0.00120470270368218\\
578	0.00110847664884727\\
579	0.00101595654786774\\
580	0.000926990517710547\\
581	0.000842661273370527\\
582	0.000762631469344003\\
583	0.000686774297128059\\
584	0.000615766704086889\\
585	0.000550119311053988\\
586	0.000489175988806172\\
587	0.000432959713352333\\
588	0.000380739850576222\\
589	0.000331803972468641\\
590	0.00028568406237948\\
591	0.000241695256631419\\
592	0.000199665955621898\\
593	0.00015946108286887\\
594	0.000121044515193095\\
595	8.4566189416569e-05\\
596	5.06250627329882e-05\\
597	2.09371357655837e-05\\
598	8.08076597025309e-08\\
599	0\\
600	0\\
};
\addplot [color=mycolor18,solid,forget plot]
  table[row sep=crcr]{%
1	0.00994530872608549\\
2	0.00994530863246163\\
3	0.00994530853716355\\
4	0.00994530844016134\\
5	0.0099453083414246\\
6	0.00994530824092232\\
7	0.00994530813862301\\
8	0.00994530803449456\\
9	0.00994530792850432\\
10	0.00994530782061906\\
11	0.00994530771080493\\
12	0.0099453075990275\\
13	0.00994530748525173\\
14	0.00994530736944193\\
15	0.00994530725156179\\
16	0.00994530713157436\\
17	0.00994530700944202\\
18	0.00994530688512649\\
19	0.00994530675858878\\
20	0.00994530662978925\\
21	0.00994530649868751\\
22	0.00994530636524248\\
23	0.00994530622941233\\
24	0.00994530609115449\\
25	0.00994530595042564\\
26	0.00994530580718168\\
27	0.00994530566137771\\
28	0.00994530551296807\\
29	0.00994530536190624\\
30	0.00994530520814489\\
31	0.00994530505163587\\
32	0.00994530489233012\\
33	0.00994530473017776\\
34	0.00994530456512799\\
35	0.00994530439712911\\
36	0.0099453042261285\\
37	0.0099453040520726\\
38	0.0099453038749069\\
39	0.00994530369457592\\
40	0.00994530351102319\\
41	0.00994530332419123\\
42	0.00994530313402154\\
43	0.00994530294045456\\
44	0.0099453027434297\\
45	0.00994530254288528\\
46	0.0099453023387585\\
47	0.00994530213098546\\
48	0.00994530191950113\\
49	0.00994530170423929\\
50	0.00994530148513257\\
51	0.0099453012621124\\
52	0.00994530103510896\\
53	0.0099453008040512\\
54	0.00994530056886682\\
55	0.00994530032948222\\
56	0.00994530008582247\\
57	0.00994529983781134\\
58	0.00994529958537121\\
59	0.0099452993284231\\
60	0.00994529906688662\\
61	0.00994529880067992\\
62	0.00994529852971972\\
63	0.00994529825392125\\
64	0.00994529797319823\\
65	0.00994529768746284\\
66	0.00994529739662569\\
67	0.0099452971005958\\
68	0.00994529679928058\\
69	0.00994529649258577\\
70	0.00994529618041545\\
71	0.00994529586267197\\
72	0.00994529553925595\\
73	0.00994529521006625\\
74	0.0099452948749999\\
75	0.00994529453395213\\
76	0.00994529418681627\\
77	0.00994529383348376\\
78	0.00994529347384412\\
79	0.00994529310778487\\
80	0.00994529273519155\\
81	0.00994529235594765\\
82	0.00994529196993458\\
83	0.00994529157703165\\
84	0.00994529117711599\\
85	0.00994529077006256\\
86	0.0099452903557441\\
87	0.00994528993403106\\
88	0.00994528950479157\\
89	0.00994528906789144\\
90	0.00994528862319407\\
91	0.00994528817056042\\
92	0.00994528770984898\\
93	0.0099452872409157\\
94	0.00994528676361397\\
95	0.00994528627779456\\
96	0.0099452857833056\\
97	0.00994528527999246\\
98	0.00994528476769781\\
99	0.00994528424626146\\
100	0.0099452837155204\\
101	0.00994528317530867\\
102	0.0099452826254574\\
103	0.00994528206579466\\
104	0.00994528149614547\\
105	0.00994528091633171\\
106	0.0099452803261721\\
107	0.00994527972548211\\
108	0.00994527911407392\\
109	0.00994527849175635\\
110	0.0099452778583348\\
111	0.00994527721361121\\
112	0.00994527655738395\\
113	0.00994527588944781\\
114	0.00994527520959392\\
115	0.00994527451760965\\
116	0.00994527381327858\\
117	0.00994527309638043\\
118	0.00994527236669097\\
119	0.00994527162398195\\
120	0.00994527086802106\\
121	0.00994527009857182\\
122	0.00994526931539352\\
123	0.00994526851824113\\
124	0.00994526770686526\\
125	0.00994526688101203\\
126	0.00994526604042303\\
127	0.00994526518483522\\
128	0.00994526431398083\\
129	0.00994526342758733\\
130	0.00994526252537728\\
131	0.00994526160706825\\
132	0.0099452606723728\\
133	0.00994525972099829\\
134	0.00994525875264685\\
135	0.00994525776701527\\
136	0.00994525676379489\\
137	0.00994525574267152\\
138	0.00994525470332534\\
139	0.00994525364543076\\
140	0.00994525256865638\\
141	0.00994525147266481\\
142	0.00994525035711263\\
143	0.00994524922165024\\
144	0.00994524806592175\\
145	0.00994524688956487\\
146	0.00994524569221081\\
147	0.00994524447348414\\
148	0.00994524323300266\\
149	0.00994524197037732\\
150	0.00994524068521204\\
151	0.00994523937710363\\
152	0.00994523804564161\\
153	0.00994523669040812\\
154	0.00994523531097779\\
155	0.00994523390691753\\
156	0.00994523247778649\\
157	0.00994523102313583\\
158	0.00994522954250864\\
159	0.00994522803543974\\
160	0.00994522650145554\\
161	0.00994522494007393\\
162	0.00994522335080406\\
163	0.00994522173314621\\
164	0.00994522008659163\\
165	0.00994521841062235\\
166	0.00994521670471104\\
167	0.00994521496832083\\
168	0.0099452132009051\\
169	0.00994521140190733\\
170	0.00994520957076094\\
171	0.00994520770688904\\
172	0.00994520580970428\\
173	0.00994520387860867\\
174	0.00994520191299334\\
175	0.00994519991223835\\
176	0.00994519787571252\\
177	0.00994519580277317\\
178	0.00994519369276593\\
179	0.00994519154502453\\
180	0.00994518935887056\\
181	0.00994518713361324\\
182	0.00994518486854921\\
183	0.00994518256296228\\
184	0.00994518021612318\\
185	0.00994517782728933\\
186	0.0099451753957046\\
187	0.00994517292059901\\
188	0.00994517040118852\\
189	0.00994516783667475\\
190	0.00994516522624466\\
191	0.00994516256907037\\
192	0.00994515986430879\\
193	0.00994515711110138\\
194	0.00994515430857384\\
195	0.0099451514558358\\
196	0.00994514855198055\\
197	0.00994514559608471\\
198	0.00994514258720789\\
199	0.0099451395243924\\
200	0.00994513640666291\\
201	0.00994513323302611\\
202	0.00994513000247034\\
203	0.0099451267139653\\
204	0.00994512336646163\\
205	0.0099451199588906\\
206	0.00994511649016369\\
207	0.00994511295917222\\
208	0.00994510936478699\\
209	0.00994510570585787\\
210	0.00994510198121336\\
211	0.00994509818966024\\
212	0.00994509432998311\\
213	0.00994509040094397\\
214	0.00994508640128176\\
215	0.00994508232971197\\
216	0.00994507818492613\\
217	0.00994507396559136\\
218	0.00994506967034989\\
219	0.00994506529781861\\
220	0.00994506084658853\\
221	0.00994505631522427\\
222	0.0099450517022636\\
223	0.00994504700621688\\
224	0.00994504222556649\\
225	0.00994503735876634\\
226	0.00994503240424128\\
227	0.00994502736038652\\
228	0.00994502222556708\\
229	0.00994501699811716\\
230	0.00994501167633954\\
231	0.00994500625850499\\
232	0.00994500074285158\\
233	0.00994499512758409\\
234	0.00994498941087331\\
235	0.0099449835908554\\
236	0.00994497766563117\\
237	0.00994497163326538\\
238	0.00994496549178608\\
239	0.00994495923918377\\
240	0.00994495287341078\\
241	0.0099449463923804\\
242	0.00994493979396616\\
243	0.00994493307600104\\
244	0.00994492623627661\\
245	0.00994491927254226\\
246	0.00994491218250431\\
247	0.00994490496382519\\
248	0.00994489761412252\\
249	0.00994489013096823\\
250	0.00994488251188764\\
251	0.00994487475435855\\
252	0.00994486685581024\\
253	0.00994485881362253\\
254	0.00994485062512477\\
255	0.00994484228759483\\
256	0.00994483379825808\\
257	0.00994482515428631\\
258	0.00994481635279665\\
259	0.00994480739085052\\
260	0.00994479826545245\\
261	0.00994478897354897\\
262	0.00994477951202745\\
263	0.00994476987771486\\
264	0.00994476006737664\\
265	0.00994475007771541\\
266	0.0099447399053697\\
267	0.0099447295469127\\
268	0.00994471899885095\\
269	0.00994470825762297\\
270	0.00994469731959795\\
271	0.00994468618107433\\
272	0.00994467483827838\\
273	0.00994466328736282\\
274	0.00994465152440527\\
275	0.00994463954540686\\
276	0.00994462734629062\\
277	0.00994461492289998\\
278	0.00994460227099722\\
279	0.0099445893862618\\
280	0.00994457626428883\\
281	0.0099445629005873\\
282	0.00994454929057851\\
283	0.00994453542959426\\
284	0.00994452131287519\\
285	0.00994450693556894\\
286	0.00994449229272843\\
287	0.00994447737930995\\
288	0.00994446219017139\\
289	0.00994444672007028\\
290	0.00994443096366195\\
291	0.00994441491549752\\
292	0.00994439857002202\\
293	0.00994438192157229\\
294	0.00994436496437508\\
295	0.00994434769254487\\
296	0.00994433010008191\\
297	0.00994431218087004\\
298	0.0099442939286746\\
299	0.00994427533714023\\
300	0.00994425639978874\\
301	0.00994423711001689\\
302	0.0099442174610941\\
303	0.00994419744616029\\
304	0.0099441770582235\\
305	0.00994415629015765\\
306	0.00994413513470017\\
307	0.00994411358444968\\
308	0.00994409163186359\\
309	0.00994406926925569\\
310	0.00994404648879376\\
311	0.0099440232824971\\
312	0.00994399964223405\\
313	0.00994397555971954\\
314	0.00994395102651253\\
315	0.00994392603401352\\
316	0.00994390057346192\\
317	0.00994387463593353\\
318	0.0099438482123379\\
319	0.00994382129341566\\
320	0.0099437938697359\\
321	0.00994376593169342\\
322	0.00994373746950604\\
323	0.00994370847321179\\
324	0.0099436789326661\\
325	0.00994364883753896\\
326	0.00994361817731201\\
327	0.00994358694127558\\
328	0.00994355511852568\\
329	0.00994352269796089\\
330	0.00994348966827926\\
331	0.00994345601797505\\
332	0.00994342173533543\\
333	0.00994338680843707\\
334	0.00994335122514262\\
335	0.00994331497309717\\
336	0.00994327803972444\\
337	0.00994324041222297\\
338	0.00994320207756214\\
339	0.00994316302247807\\
340	0.00994312323346933\\
341	0.00994308269679256\\
342	0.00994304139845792\\
343	0.00994299932422435\\
344	0.00994295645959474\\
345	0.00994291278981085\\
346	0.00994286829984814\\
347	0.00994282297441031\\
348	0.00994277679792376\\
349	0.00994272975453173\\
350	0.00994268182808826\\
351	0.00994263300215185\\
352	0.00994258325997883\\
353	0.00994253258451638\\
354	0.00994248095839516\\
355	0.00994242836392154\\
356	0.00994237478306923\\
357	0.00994232019747054\\
358	0.00994226458840696\\
359	0.00994220793679908\\
360	0.00994215022319597\\
361	0.0099420914277636\\
362	0.0099420315302726\\
363	0.00994197051008506\\
364	0.0099419083461403\\
365	0.00994184501693968\\
366	0.00994178050053014\\
367	0.00994171477448656\\
368	0.00994164781589266\\
369	0.0099415796013205\\
370	0.00994151010680826\\
371	0.00994143930783629\\
372	0.00994136717930126\\
373	0.00994129369548813\\
374	0.00994121883003993\\
375	0.00994114255592498\\
376	0.00994106484540139\\
377	0.00994098566997857\\
378	0.00994090500037551\\
379	0.00994082280647537\\
380	0.00994073905727608\\
381	0.00994065372083656\\
382	0.00994056676421792\\
383	0.00994047815341912\\
384	0.00994038785330629\\
385	0.00994029582753498\\
386	0.00994020203846408\\
387	0.0099401064470603\\
388	0.00994000901279129\\
389	0.00993990969350532\\
390	0.00993980844529442\\
391	0.00993970522233632\\
392	0.00993959997670972\\
393	0.00993949265817708\\
394	0.00993938321393834\\
395	0.00993927158839073\\
396	0.00993915772297817\\
397	0.00993904155596487\\
398	0.00993892302167932\\
399	0.00993880205036838\\
400	0.00993867856846095\\
401	0.00993855249832119\\
402	0.00993842375800448\\
403	0.00993829226102774\\
404	0.00993815791617112\\
405	0.00993802062733497\\
406	0.00993788029347234\\
407	0.00993773680857301\\
408	0.00993759006152621\\
409	0.00993743993545996\\
410	0.00993728630700324\\
411	0.0099371290478545\\
412	0.009936968025257\\
413	0.00993680309698175\\
414	0.00993663411020969\\
415	0.00993646090024259\\
416	0.00993628328901319\\
417	0.00993610108335826\\
418	0.00993591407301087\\
419	0.0099357220282588\\
420	0.0099355246972049\\
421	0.00993532180254993\\
422	0.00993511303779893\\
423	0.00993489806276213\\
424	0.00993467649816652\\
425	0.00993444791907428\\
426	0.00993421184647955\\
427	0.00993396773547921\\
428	0.00993371495529023\\
429	0.00993345274602725\\
430	0.00993318010200916\\
431	0.00993289541004126\\
432	0.00993259524622708\\
433	0.00993218797039568\\
434	0.00993090202583174\\
435	0.00992958478643436\\
436	0.00992823519380699\\
437	0.0099268521368044\\
438	0.00992543444893827\\
439	0.00992398090594877\\
440	0.00992249022366804\\
441	0.00992096105634524\\
442	0.00991939199566086\\
443	0.00991778157073748\\
444	0.00991612824956834\\
445	0.00991443044246893\\
446	0.00991268650850917\\
447	0.00991089476670881\\
448	0.00990905351606365\\
449	0.00990716107550552\\
450	0.00990521587789539\\
451	0.00990321672995807\\
452	0.00990116361941189\\
453	0.00989906039904507\\
454	0.00989708676040484\\
455	0.00989509865417535\\
456	0.00989305748934458\\
457	0.00989096163614686\\
458	0.00988880951929047\\
459	0.00988659966253827\\
460	0.00988433074067629\\
461	0.0098820016284305\\
462	0.00987961142015817\\
463	0.00987715938518256\\
464	0.00987464495652623\\
465	0.00987086353347779\\
466	0.00986635109645413\\
467	0.00986176052396705\\
468	0.00985709008348626\\
469	0.00985233797908782\\
470	0.00984750236458915\\
471	0.00984258125985193\\
472	0.00983757252158744\\
473	0.0098324739162371\\
474	0.0098272831187968\\
475	0.0098219976705889\\
476	0.00981661472105107\\
477	0.00981113146715365\\
478	0.00980554546545842\\
479	0.00979985413542198\\
480	0.00979405474235537\\
481	0.00978814437804885\\
482	0.0097821199377368\\
483	0.0097759780924385\\
484	0.00976971525541765\\
485	0.00976332754113389\\
486	0.00975681071456328\\
487	0.00975016012787242\\
488	0.00974337063931057\\
489	0.00973643650682528\\
490	0.00972935128498039\\
491	0.00972210785369496\\
492	0.009714698306149\\
493	0.00970711382319213\\
494	0.00969934452268357\\
495	0.00969137929782672\\
496	0.00968320557808637\\
497	0.00967480899524378\\
498	0.00966617302743403\\
499	0.00965727856292655\\
500	0.00964810336797047\\
501	0.00963862145067815\\
502	0.00962880235403383\\
503	0.00961861057918534\\
504	0.00960800598322956\\
505	0.0095969536006589\\
506	0.00958206644929212\\
507	0.00948999160190526\\
508	0.00939546331955479\\
509	0.00929836799636671\\
510	0.00919857319894122\\
511	0.00909593192290604\\
512	0.00899030201481941\\
513	0.00888153056390636\\
514	0.00876945416574308\\
515	0.00865389827400868\\
516	0.0085346751293968\\
517	0.00841158214722791\\
518	0.00828440077155095\\
519	0.00815289540195748\\
520	0.00801681250681786\\
521	0.0078758800354863\\
522	0.00772980690561726\\
523	0.00757827517193849\\
524	0.00742094180879576\\
525	0.00725746823434439\\
526	0.00708750417892308\\
527	0.00691061947897641\\
528	0.00672608878795486\\
529	0.00653306109930443\\
530	0.00633050943248134\\
531	0.00611721763458413\\
532	0.00602542620281169\\
533	0.00594176967878208\\
534	0.00585309165652184\\
535	0.00575854157516701\\
536	0.00565705236556634\\
537	0.00554727651271868\\
538	0.0054274059270895\\
539	0.00529537675222571\\
540	0.00515667036720481\\
541	0.00501558550724401\\
542	0.00487248204218308\\
543	0.00472785771404214\\
544	0.00458239017377988\\
545	0.00443699221575908\\
546	0.00429288543158003\\
547	0.00415170567236035\\
548	0.00401553206300564\\
549	0.00388694431160637\\
550	0.00376835702280204\\
551	0.00365892809920717\\
552	0.00355226912662941\\
553	0.00344334259773825\\
554	0.00333231711946476\\
555	0.00321939132440821\\
556	0.00310478872432758\\
557	0.00298874305514569\\
558	0.0028714797454717\\
559	0.00275318020120921\\
560	0.00263391143049639\\
561	0.00251360967999891\\
562	0.00239201742007009\\
563	0.00226936231983812\\
564	0.00214698967826542\\
565	0.00202902581711521\\
566	0.00191608259350492\\
567	0.00180882818786143\\
568	0.00170779715947139\\
569	0.00161325506472726\\
570	0.00152422353788928\\
571	0.00143814126276182\\
572	0.0013548747928504\\
573	0.00127413269596697\\
574	0.00119555682784887\\
575	0.00111868530807603\\
576	0.00104391196677983\\
577	0.000971929216366833\\
578	0.000902900964383295\\
579	0.000837024370391768\\
580	0.000774141917445512\\
581	0.000713541582665399\\
582	0.000656182834448444\\
583	0.000602149092776189\\
584	0.000551122945708443\\
585	0.000502035175715594\\
586	0.000454956872265477\\
587	0.000409666176427344\\
588	0.000365537640462562\\
589	0.000322413817603431\\
590	0.000280139864826597\\
591	0.000238748720197689\\
592	0.000198321849653529\\
593	0.000158985746646168\\
594	0.000120944048332155\\
595	8.45661894165683e-05\\
596	5.06250627329887e-05\\
597	2.09371357655843e-05\\
598	8.08076597025309e-08\\
599	0\\
600	0\\
};
\addplot [color=red!25!mycolor17,solid,forget plot]
  table[row sep=crcr]{%
1	0.00981893309360387\\
2	0.00981893209166415\\
3	0.00981893107180361\\
4	0.00981893003370215\\
5	0.009818928977034\\
6	0.00981892790146756\\
7	0.00981892680666533\\
8	0.00981892569228377\\
9	0.00981892455797325\\
10	0.00981892340337789\\
11	0.00981892222813544\\
12	0.00981892103187722\\
13	0.00981891981422799\\
14	0.00981891857480578\\
15	0.00981891731322183\\
16	0.00981891602908046\\
17	0.00981891472197893\\
18	0.00981891339150731\\
19	0.00981891203724837\\
20	0.00981891065877745\\
21	0.00981890925566231\\
22	0.00981890782746301\\
23	0.00981890637373177\\
24	0.00981890489401282\\
25	0.00981890338784228\\
26	0.009818901854748\\
27	0.00981890029424941\\
28	0.00981889870585737\\
29	0.00981889708907403\\
30	0.00981889544339267\\
31	0.00981889376829754\\
32	0.0098188920632637\\
33	0.00981889032775684\\
34	0.00981888856123314\\
35	0.00981888676313911\\
36	0.00981888493291136\\
37	0.00981888306997647\\
38	0.00981888117375084\\
39	0.00981887924364042\\
40	0.00981887727904061\\
41	0.00981887527933602\\
42	0.00981887324390029\\
43	0.00981887117209593\\
44	0.00981886906327406\\
45	0.00981886691677426\\
46	0.00981886473192434\\
47	0.00981886250804011\\
48	0.00981886024442524\\
49	0.00981885794037094\\
50	0.00981885559515583\\
51	0.00981885320804567\\
52	0.00981885077829313\\
53	0.00981884830513758\\
54	0.00981884578780482\\
55	0.00981884322550688\\
56	0.00981884061744174\\
57	0.0098188379627931\\
58	0.00981883526073011\\
59	0.00981883251040713\\
60	0.00981882971096344\\
61	0.00981882686152299\\
62	0.00981882396119415\\
63	0.00981882100906937\\
64	0.00981881800422495\\
65	0.00981881494572075\\
66	0.00981881183259985\\
67	0.00981880866388832\\
68	0.00981880543859483\\
69	0.00981880215571045\\
70	0.00981879881420823\\
71	0.00981879541304296\\
72	0.00981879195115077\\
73	0.00981878842744888\\
74	0.00981878484083519\\
75	0.00981878119018798\\
76	0.00981877747436554\\
77	0.00981877369220583\\
78	0.0098187698425261\\
79	0.00981876592412251\\
80	0.00981876193576981\\
81	0.00981875787622089\\
82	0.00981875374420643\\
83	0.0098187495384345\\
84	0.00981874525759013\\
85	0.00981874090033494\\
86	0.0098187364653067\\
87	0.00981873195111889\\
88	0.0098187273563603\\
89	0.00981872267959457\\
90	0.00981871791935971\\
91	0.00981871307416774\\
92	0.00981870814250409\\
93	0.00981870312282724\\
94	0.00981869801356818\\
95	0.00981869281312995\\
96	0.00981868751988709\\
97	0.0098186821321852\\
98	0.00981867664834036\\
99	0.00981867106663866\\
100	0.00981866538533559\\
101	0.00981865960265557\\
102	0.00981865371679136\\
103	0.00981864772590348\\
104	0.00981864162811966\\
105	0.00981863542153422\\
106	0.00981862910420752\\
107	0.00981862267416531\\
108	0.00981861612939811\\
109	0.00981860946786063\\
110	0.00981860268747106\\
111	0.00981859578611047\\
112	0.00981858876162212\\
113	0.00981858161181076\\
114	0.00981857433444199\\
115	0.00981856692724155\\
116	0.00981855938789454\\
117	0.00981855171404477\\
118	0.00981854390329399\\
119	0.00981853595320112\\
120	0.00981852786128149\\
121	0.00981851962500609\\
122	0.00981851124180071\\
123	0.00981850270904518\\
124	0.00981849402407253\\
125	0.00981848518416813\\
126	0.00981847618656885\\
127	0.0098184670284622\\
128	0.00981845770698542\\
129	0.00981844821922457\\
130	0.00981843856221362\\
131	0.00981842873293354\\
132	0.00981841872831128\\
133	0.00981840854521886\\
134	0.00981839818047233\\
135	0.00981838763083081\\
136	0.00981837689299542\\
137	0.00981836596360823\\
138	0.00981835483925125\\
139	0.00981834351644528\\
140	0.00981833199164884\\
141	0.00981832026125702\\
142	0.00981830832160038\\
143	0.00981829616894372\\
144	0.00981828379948494\\
145	0.00981827120935378\\
146	0.00981825839461064\\
147	0.00981824535124528\\
148	0.00981823207517557\\
149	0.00981821856224615\\
150	0.00981820480822713\\
151	0.00981819080881274\\
152	0.00981817655961992\\
153	0.00981816205618695\\
154	0.00981814729397198\\
155	0.00981813226835161\\
156	0.0098181169746194\\
157	0.00981810140798433\\
158	0.00981808556356928\\
159	0.00981806943640945\\
160	0.00981805302145078\\
161	0.0098180363135483\\
162	0.00981801930746446\\
163	0.00981800199786748\\
164	0.00981798437932957\\
165	0.00981796644632522\\
166	0.00981794819322938\\
167	0.00981792961431566\\
168	0.00981791070375446\\
169	0.00981789145561107\\
170	0.00981787186384373\\
171	0.00981785192230172\\
172	0.00981783162472328\\
173	0.00981781096473362\\
174	0.0098177899358428\\
175	0.00981776853144368\\
176	0.00981774674480965\\
177	0.00981772456909253\\
178	0.00981770199732026\\
179	0.00981767902239464\\
180	0.00981765563708897\\
181	0.00981763183404572\\
182	0.00981760760577405\\
183	0.00981758294464739\\
184	0.00981755784290089\\
185	0.00981753229262884\\
186	0.00981750628578212\\
187	0.00981747981416544\\
188	0.00981745286943469\\
189	0.00981742544309414\\
190	0.00981739752649362\\
191	0.00981736911082563\\
192	0.00981734018712241\\
193	0.00981731074625294\\
194	0.00981728077891988\\
195	0.00981725027565646\\
196	0.00981721922682333\\
197	0.00981718762260526\\
198	0.00981715545300789\\
199	0.00981712270785435\\
200	0.00981708937678182\\
201	0.00981705544923803\\
202	0.00981702091447768\\
203	0.00981698576155881\\
204	0.00981694997933909\\
205	0.00981691355647199\\
206	0.00981687648140299\\
207	0.00981683874236555\\
208	0.00981680032737717\\
209	0.00981676122423522\\
210	0.00981672142051281\\
211	0.0098166809035545\\
212	0.00981663966047193\\
213	0.00981659767813942\\
214	0.0098165549431894\\
215	0.00981651144200781\\
216	0.00981646716072939\\
217	0.00981642208523282\\
218	0.0098163762011359\\
219	0.00981632949379047\\
220	0.00981628194827731\\
221	0.00981623354940096\\
222	0.00981618428168439\\
223	0.00981613412936352\\
224	0.00981608307638177\\
225	0.00981603110638433\\
226	0.00981597820271245\\
227	0.00981592434839751\\
228	0.00981586952615505\\
229	0.0098158137183786\\
230	0.00981575690713348\\
231	0.00981569907415037\\
232	0.0098156402008188\\
233	0.00981558026818055\\
234	0.00981551925692278\\
235	0.00981545714737118\\
236	0.00981539391948286\\
237	0.00981532955283916\\
238	0.00981526402663826\\
239	0.0098151973196877\\
240	0.00981512941039666\\
241	0.00981506027676818\\
242	0.00981498989639113\\
243	0.0098149182464321\\
244	0.009814845303627\\
245	0.00981477104427264\\
246	0.00981469544421801\\
247	0.00981461847885545\\
248	0.00981454012311159\\
249	0.00981446035143814\\
250	0.00981437913780249\\
251	0.00981429645567807\\
252	0.00981421227803458\\
253	0.00981412657732793\\
254	0.00981403932549008\\
255	0.00981395049391858\\
256	0.00981386005346594\\
257	0.00981376797442877\\
258	0.00981367422653668\\
259	0.00981357877894102\\
260	0.00981348160020326\\
261	0.00981338265828331\\
262	0.0098132819205274\\
263	0.00981317935365591\\
264	0.00981307492375082\\
265	0.00981296859624293\\
266	0.00981286033589888\\
267	0.00981275010680784\\
268	0.00981263787236798\\
269	0.00981252359527264\\
270	0.00981240723749625\\
271	0.00981228876027994\\
272	0.0098121681241169\\
273	0.00981204528873745\\
274	0.00981192021309378\\
275	0.00981179285534442\\
276	0.00981166317283846\\
277	0.00981153112209935\\
278	0.00981139665880851\\
279	0.00981125973778855\\
280	0.00981112031298625\\
281	0.00981097833745513\\
282	0.00981083376333778\\
283	0.0098106865418478\\
284	0.00981053662325151\\
285	0.00981038395684921\\
286	0.00981022849095622\\
287	0.00981007017288351\\
288	0.009809908948918\\
289	0.00980974476430263\\
290	0.00980957756321593\\
291	0.00980940728875139\\
292	0.00980923388289641\\
293	0.00980905728651097\\
294	0.00980887743930594\\
295	0.00980869427982104\\
296	0.0098085077454025\\
297	0.0098083177721804\\
298	0.0098081242950456\\
299	0.00980792724762649\\
300	0.00980772656226531\\
301	0.00980752216999416\\
302	0.0098073140005108\\
303	0.00980710198215407\\
304	0.00980688604187898\\
305	0.00980666610523164\\
306	0.00980644209632384\\
307	0.00980621393780731\\
308	0.00980598155084782\\
309	0.00980574485509903\\
310	0.00980550376867597\\
311	0.00980525820812851\\
312	0.00980500808841439\\
313	0.00980475332287226\\
314	0.00980449382319433\\
315	0.00980422949939905\\
316	0.0098039602598034\\
317	0.00980368601099523\\
318	0.00980340665780533\\
319	0.00980312210327938\\
320	0.00980283224864979\\
321	0.00980253699330743\\
322	0.00980223623477314\\
323	0.00980192986866926\\
324	0.00980161778869088\\
325	0.00980129988657702\\
326	0.00980097605208163\\
327	0.00980064617294443\\
328	0.00980031013486147\\
329	0.00979996782145552\\
330	0.00979961911424608\\
331	0.00979926389261913\\
332	0.00979890203379635\\
333	0.009798533412804\\
334	0.00979815790244109\\
335	0.00979777537324694\\
336	0.00979738569346807\\
337	0.00979698872902409\\
338	0.00979658434347275\\
339	0.00979617239797392\\
340	0.00979575275125234\\
341	0.00979532525955932\\
342	0.00979488977663306\\
343	0.00979444615365778\\
344	0.0097939942392216\\
345	0.00979353387927335\\
346	0.00979306491707836\\
347	0.00979258719317333\\
348	0.00979210054532038\\
349	0.0097916048084604\\
350	0.00979109981466578\\
351	0.00979058539309274\\
352	0.00979006136993313\\
353	0.00978952756836588\\
354	0.00978898380850775\\
355	0.00978842990736347\\
356	0.00978786567877418\\
357	0.009787290933364\\
358	0.00978670547848445\\
359	0.00978610911815663\\
360	0.00978550165301074\\
361	0.00978488288022275\\
362	0.009784252593448\\
363	0.00978361058275118\\
364	0.00978295663453258\\
365	0.00978229053145008\\
366	0.00978161205233643\\
367	0.00978092097211152\\
368	0.00978021706168896\\
369	0.00977950008787657\\
370	0.00977876981327014\\
371	0.00977802599613982\\
372	0.00977726839030858\\
373	0.00977649674502179\\
374	0.00977571080480741\\
375	0.00977491030932561\\
376	0.0097740949932072\\
377	0.00977326458587948\\
378	0.00977241881137871\\
379	0.00977155738814757\\
380	0.00977068002881633\\
381	0.00976978643996602\\
382	0.00976887632187155\\
383	0.00976794936822248\\
384	0.00976700526581871\\
385	0.00976604369423765\\
386	0.00976506432546867\\
387	0.00976406682350957\\
388	0.00976305084391822\\
389	0.00976201603330999\\
390	0.00976096202878773\\
391	0.00975988845728312\\
392	0.00975879493477439\\
393	0.00975768106532158\\
394	0.0097565464398444\\
395	0.00975539063469898\\
396	0.0097542132111469\\
397	0.0097530137222884\\
398	0.00975179170696275\\
399	0.00975054666852783\\
400	0.00974927807772209\\
401	0.00974798538311508\\
402	0.00974666800982067\\
403	0.00974532535826987\\
404	0.0097439568031393\\
405	0.00974256169260215\\
406	0.00974113934816847\\
407	0.00973968906540693\\
408	0.00973821011514681\\
409	0.00973670174035603\\
410	0.0097351631233043\\
411	0.00973359338056676\\
412	0.00973199168448649\\
413	0.00973035727343333\\
414	0.00972868935935763\\
415	0.00972698712634983\\
416	0.00972524972834786\\
417	0.00972347628642404\\
418	0.00972166588554786\\
419	0.00971981757069356\\
420	0.00971793034212245\\
421	0.00971600314961931\\
422	0.00971403488541175\\
423	0.00971202437543407\\
424	0.00970997036854709\\
425	0.009707871523185\\
426	0.00970572639104961\\
427	0.00970353339817112\\
428	0.00970129082650495\\
429	0.00969899680969603\\
430	0.00969664939400621\\
431	0.00969424684852234\\
432	0.00969178888566503\\
433	0.00968937000593594\\
434	0.00968782120503358\\
435	0.00968624489285561\\
436	0.00968464067329799\\
437	0.00968300814730006\\
438	0.00968134691103479\\
439	0.00967965655328143\\
440	0.0096779366517134\\
441	0.00967618676775407\\
442	0.00967440643954922\\
443	0.00967259517247383\\
444	0.00967075242642675\\
445	0.00966887759897477\\
446	0.00966697000321139\\
447	0.00966502883908095\\
448	0.0096630531571047\\
449	0.00966104181438862\\
450	0.00965899342497405\\
451	0.00965690630754178\\
452	0.00965477841016621\\
453	0.00965260701596463\\
454	0.00965038426589639\\
455	0.00964810541040014\\
456	0.00964576593792202\\
457	0.00964336044025375\\
458	0.00964088238829493\\
459	0.00963832383884658\\
460	0.00963567503266231\\
461	0.00963292378458077\\
462	0.00963005438414037\\
463	0.00962704518044858\\
464	0.00962386147319565\\
465	0.00959594571665683\\
466	0.00955389578933913\\
467	0.00951093446060314\\
468	0.00946703587479382\\
469	0.00942217328036562\\
470	0.00937631935249833\\
471	0.00932944592545866\\
472	0.00928152202998148\\
473	0.00923251531405754\\
474	0.00918239182388736\\
475	0.00913111550865112\\
476	0.00907864709076097\\
477	0.00902494003618598\\
478	0.00896995740622145\\
479	0.00891366078861219\\
480	0.00885601018517335\\
481	0.00879696389044129\\
482	0.00873647833770005\\
483	0.00867450789779599\\
484	0.00861100460983407\\
485	0.00854591781389221\\
486	0.00847919364288878\\
487	0.00841077431084862\\
488	0.00834059710219882\\
489	0.00826859291199121\\
490	0.00819468426658173\\
491	0.00811878411067971\\
492	0.00804079804102239\\
493	0.0079606234633549\\
494	0.0078781487562435\\
495	0.00779325259146717\\
496	0.00770580334423899\\
497	0.00761565706157405\\
498	0.00752265475683277\\
499	0.00742662003796815\\
500	0.00732735627267455\\
501	0.00722464324062152\\
502	0.00711823332094557\\
503	0.00700784756310397\\
504	0.00689317279543677\\
505	0.00677385788265152\\
506	0.00665678290136369\\
507	0.00661621137311753\\
508	0.00657415257718716\\
509	0.00653062582915628\\
510	0.00648569939592293\\
511	0.0064394523023345\\
512	0.00639177025580273\\
513	0.00634251036815858\\
514	0.00629147939580797\\
515	0.00623847624139042\\
516	0.00618328415022093\\
517	0.00612564517139622\\
518	0.00606525068758113\\
519	0.00600172951567923\\
520	0.00593463281435459\\
521	0.0058634149276205\\
522	0.00578740929237026\\
523	0.00570579868968973\\
524	0.00561757957977018\\
525	0.00552152501469153\\
526	0.00541724252416291\\
527	0.0053121914098488\\
528	0.00520681182057042\\
529	0.0051016857967089\\
530	0.00499758054538814\\
531	0.00489550060878119\\
532	0.00479299450807677\\
533	0.00469024184420434\\
534	0.00458786551639743\\
535	0.00448669151964237\\
536	0.00438781420955728\\
537	0.00429267328776202\\
538	0.00420305832428333\\
539	0.00412124813666649\\
540	0.00404199444251681\\
541	0.00396120538761791\\
542	0.00387894996980512\\
543	0.00379530320397678\\
544	0.00371034037549661\\
545	0.00362412690987067\\
546	0.00353671986104819\\
547	0.00344812122380666\\
548	0.00335823668618554\\
549	0.00326682204498162\\
550	0.00317343475341059\\
551	0.00307747598607583\\
552	0.00297871476709441\\
553	0.00287728073394513\\
554	0.00277334222059271\\
555	0.00266714375230942\\
556	0.00255904646486252\\
557	0.00244959283909771\\
558	0.00233963885791423\\
559	0.00223206024903197\\
560	0.00212921839196439\\
561	0.00203169630193705\\
562	0.0019400073548151\\
563	0.00185447406932942\\
564	0.00177501519419827\\
565	0.00169801815902672\\
566	0.00162337211751993\\
567	0.0015508472684989\\
568	0.00148006776580853\\
569	0.00141053381409169\\
570	0.00134145520873824\\
571	0.00127257435582606\\
572	0.00120501215149874\\
573	0.00113944148947218\\
574	0.00107602645116795\\
575	0.00101502269159321\\
576	0.000955635319791849\\
577	0.000897124935990234\\
578	0.000840043707243829\\
579	0.000785472823091309\\
580	0.000733471579951849\\
581	0.000683494593268013\\
582	0.000634600854943696\\
583	0.000587054292825937\\
584	0.000540827063147928\\
585	0.000495625786305547\\
586	0.000451153242779368\\
587	0.000407354290724509\\
588	0.000364212872996445\\
589	0.000321706926583557\\
590	0.000279804231369382\\
591	0.000238614617512775\\
592	0.000198281300423209\\
593	0.00015897864980585\\
594	0.000120944048332155\\
595	8.45661894165685e-05\\
596	5.06250627329879e-05\\
597	2.09371357655841e-05\\
598	8.08076597025309e-08\\
599	0\\
600	0\\
};
\addplot [color=mycolor19,solid,forget plot]
  table[row sep=crcr]{%
1	0.00954105435199312\\
2	0.00954104506679872\\
3	0.00954103561549377\\
4	0.0095410259951106\\
5	0.00954101620262863\\
6	0.0095410062349734\\
7	0.00954099608901561\\
8	0.0095409857615701\\
9	0.00954097524939496\\
10	0.0095409645491904\\
11	0.00954095365759779\\
12	0.00954094257119861\\
13	0.00954093128651337\\
14	0.00954091980000052\\
15	0.0095409081080553\\
16	0.00954089620700875\\
17	0.0095408840931264\\
18	0.00954087176260723\\
19	0.00954085921158241\\
20	0.00954084643611415\\
21	0.00954083343219439\\
22	0.00954082019574366\\
23	0.0095408067226097\\
24	0.00954079300856623\\
25	0.0095407790493116\\
26	0.00954076484046747\\
27	0.00954075037757744\\
28	0.00954073565610562\\
29	0.0095407206714353\\
30	0.00954070541886741\\
31	0.00954068989361914\\
32	0.00954067409082238\\
33	0.00954065800552224\\
34	0.00954064163267549\\
35	0.00954062496714902\\
36	0.00954060800371816\\
37	0.00954059073706514\\
38	0.00954057316177738\\
39	0.0095405552723458\\
40	0.00954053706316314\\
41	0.00954051852852214\\
42	0.00954049966261384\\
43	0.00954048045952572\\
44	0.00954046091323987\\
45	0.00954044101763111\\
46	0.00954042076646508\\
47	0.00954040015339626\\
48	0.00954037917196606\\
49	0.00954035781560077\\
50	0.00954033607760948\\
51	0.00954031395118203\\
52	0.00954029142938687\\
53	0.00954026850516889\\
54	0.00954024517134727\\
55	0.00954022142061315\\
56	0.00954019724552743\\
57	0.00954017263851839\\
58	0.00954014759187937\\
59	0.00954012209776635\\
60	0.00954009614819546\\
61	0.00954006973504056\\
62	0.00954004285003067\\
63	0.00954001548474737\\
64	0.0095399876306222\\
65	0.00953995927893397\\
66	0.00953993042080604\\
67	0.00953990104720354\\
68	0.00953987114893055\\
69	0.00953984071662727\\
70	0.00953980974076702\\
71	0.00953977821165332\\
72	0.00953974611941685\\
73	0.00953971345401236\\
74	0.00953968020521552\\
75	0.00953964636261976\\
76	0.00953961191563297\\
77	0.00953957685347424\\
78	0.00953954116517045\\
79	0.00953950483955286\\
80	0.0095394678652536\\
81	0.00953943023070215\\
82	0.00953939192412167\\
83	0.00953935293352538\\
84	0.00953931324671274\\
85	0.0095392728512657\\
86	0.00953923173454475\\
87	0.00953918988368502\\
88	0.00953914728559221\\
89	0.00953910392693853\\
90	0.00953905979415848\\
91	0.00953901487344467\\
92	0.00953896915074341\\
93	0.00953892261175036\\
94	0.00953887524190606\\
95	0.00953882702639133\\
96	0.00953877795012263\\
97	0.00953872799774735\\
98	0.00953867715363897\\
99	0.00953862540189219\\
100	0.00953857272631792\\
101	0.00953851911043819\\
102	0.00953846453748101\\
103	0.00953840899037506\\
104	0.00953835245174436\\
105	0.0095382949039028\\
106	0.00953823632884861\\
107	0.00953817670825864\\
108	0.00953811602348267\\
109	0.00953805425553749\\
110	0.00953799138510096\\
111	0.00953792739250591\\
112	0.00953786225773397\\
113	0.00953779596040924\\
114	0.0095377284797919\\
115	0.00953765979477168\\
116	0.00953758988386112\\
117	0.00953751872518895\\
118	0.00953744629649304\\
119	0.00953737257511347\\
120	0.00953729753798533\\
121	0.00953722116163145\\
122	0.00953714342215502\\
123	0.00953706429523195\\
124	0.00953698375610328\\
125	0.00953690177956729\\
126	0.0095368183399715\\
127	0.00953673341120463\\
128	0.00953664696668826\\
129	0.00953655897936844\\
130	0.00953646942170708\\
131	0.00953637826567328\\
132	0.00953628548273438\\
133	0.00953619104384688\\
134	0.00953609491944729\\
135	0.00953599707944268\\
136	0.00953589749320111\\
137	0.00953579612954188\\
138	0.00953569295672564\\
139	0.00953558794244426\\
140	0.00953548105381048\\
141	0.00953537225734751\\
142	0.00953526151897828\\
143	0.00953514880401456\\
144	0.00953503407714595\\
145	0.00953491730242845\\
146	0.00953479844327312\\
147	0.00953467746243421\\
148	0.00953455432199735\\
149	0.0095344289833673\\
150	0.00953430140725564\\
151	0.0095341715536681\\
152	0.00953403938189171\\
153	0.00953390485048172\\
154	0.00953376791724827\\
155	0.00953362853924277\\
156	0.00953348667274407\\
157	0.00953334227324437\\
158	0.00953319529543481\\
159	0.00953304569319083\\
160	0.00953289341955725\\
161	0.00953273842673307\\
162	0.00953258066605603\\
163	0.0095324200879867\\
164	0.0095322566420925\\
165	0.00953209027703126\\
166	0.00953192094053452\\
167	0.00953174857939053\\
168	0.00953157313942687\\
169	0.00953139456549278\\
170	0.00953121280144113\\
171	0.0095310277901101\\
172	0.00953083947330439\\
173	0.00953064779177623\\
174	0.00953045268520587\\
175	0.00953025409218178\\
176	0.00953005195018049\\
177	0.00952984619554593\\
178	0.00952963676346849\\
179	0.00952942358796358\\
180	0.0095292066018499\\
181	0.00952898573672713\\
182	0.00952876092295328\\
183	0.00952853208962158\\
184	0.00952829916453695\\
185	0.00952806207419195\\
186	0.0095278207437423\\
187	0.00952757509698189\\
188	0.00952732505631731\\
189	0.00952707054274193\\
190	0.00952681147580941\\
191	0.00952654777360667\\
192	0.0095262793527264\\
193	0.00952600612823894\\
194	0.0095257280136637\\
195	0.00952544492093991\\
196	0.00952515676039685\\
197	0.00952486344072352\\
198	0.00952456486893753\\
199	0.00952426095035367\\
200	0.00952395158855151\\
201	0.00952363668534264\\
202	0.00952331614073704\\
203	0.00952298985290892\\
204	0.00952265771816174\\
205	0.00952231963089267\\
206	0.00952197548355618\\
207	0.00952162516662704\\
208	0.00952126856856244\\
209	0.00952090557576334\\
210	0.00952053607253523\\
211	0.0095201599410478\\
212	0.00951977706129402\\
213	0.0095193873110482\\
214	0.00951899056582333\\
215	0.00951858669882751\\
216	0.00951817558091937\\
217	0.00951775708056274\\
218	0.00951733106378027\\
219	0.00951689739410614\\
220	0.00951645593253767\\
221	0.00951600653748616\\
222	0.00951554906472646\\
223	0.00951508336734565\\
224	0.00951460929569052\\
225	0.00951412669731411\\
226	0.00951363541692092\\
227	0.00951313529631115\\
228	0.0095126261743237\\
229	0.00951210788677795\\
230	0.00951158026641431\\
231	0.0095110431428335\\
232	0.00951049634243459\\
233	0.00950993968835173\\
234	0.0095093730003894\\
235	0.00950879609495645\\
236	0.00950820878499864\\
237	0.00950761087992974\\
238	0.00950700218556117\\
239	0.0095063825040301\\
240	0.00950575163372607\\
241	0.00950510936921594\\
242	0.00950445550116726\\
243	0.00950378981627004\\
244	0.00950311209715671\\
245	0.00950242212232043\\
246	0.00950171966603164\\
247	0.00950100449825271\\
248	0.00950027638455081\\
249	0.00949953508600882\\
250	0.00949878035913436\\
251	0.00949801195576679\\
252	0.00949722962298219\\
253	0.00949643310299624\\
254	0.00949562213306496\\
255	0.00949479644538339\\
256	0.00949395576698182\\
257	0.00949309981961994\\
258	0.00949222831967861\\
259	0.00949134097804914\\
260	0.00949043750002032\\
261	0.00948951758516275\\
262	0.00948858092721084\\
263	0.00948762721394197\\
264	0.00948665612705315\\
265	0.00948566734203498\\
266	0.00948466052804264\\
267	0.00948363534776423\\
268	0.00948259145728615\\
269	0.00948152850595539\\
270	0.00948044613623896\\
271	0.00947934398358015\\
272	0.0094782216762516\\
273	0.00947707883520514\\
274	0.00947591507391841\\
275	0.00947472999823801\\
276	0.00947352320621932\\
277	0.00947229428796272\\
278	0.0094710428254463\\
279	0.00946976839235494\\
280	0.00946847055390564\\
281	0.00946714886666905\\
282	0.00946580287838723\\
283	0.00946443212778742\\
284	0.00946303614439192\\
285	0.00946161444832373\\
286	0.0094601665501082\\
287	0.00945869195047056\\
288	0.00945719014012897\\
289	0.00945566059958342\\
290	0.00945410279890026\\
291	0.00945251619749207\\
292	0.00945090024389332\\
293	0.00944925437553123\\
294	0.00944757801849214\\
295	0.00944587058728318\\
296	0.00944413148458918\\
297	0.00944236010102491\\
298	0.00944055581488251\\
299	0.00943871799187397\\
300	0.00943684598486894\\
301	0.00943493913362749\\
302	0.00943299676452811\\
303	0.0094310181902907\\
304	0.00942900270969468\\
305	0.00942694960729222\\
306	0.00942485815311652\\
307	0.00942272760238524\\
308	0.00942055719519903\\
309	0.00941834615623537\\
310	0.00941609369443741\\
311	0.00941379900269827\\
312	0.0094114612575406\\
313	0.00940907961879148\\
314	0.00940665322925292\\
315	0.00940418121436768\\
316	0.00940166268188094\\
317	0.0093990967214975\\
318	0.00939648240453481\\
319	0.00939381878357186\\
320	0.00939110489209395\\
321	0.00938833974413344\\
322	0.00938552233390653\\
323	0.00938265163544601\\
324	0.00937972660223006\\
325	0.00937674616680697\\
326	0.00937370924041564\\
327	0.00937061471260193\\
328	0.0093674614508302\\
329	0.00936424830009022\\
330	0.00936097408249859\\
331	0.00935763759689452\\
332	0.00935423761842906\\
333	0.00935077289814715\\
334	0.00934724216256156\\
335	0.00934364411321757\\
336	0.00933997742624734\\
337	0.00933624075191239\\
338	0.00933243271413295\\
339	0.00932855191000234\\
340	0.00932459690928483\\
341	0.00932056625389558\\
342	0.0093164584573609\\
343	0.0093122720042579\\
344	0.00930800534963229\\
345	0.0093036569183947\\
346	0.00929922510469773\\
347	0.00929470827129424\\
348	0.00929010474887816\\
349	0.00928541283540931\\
350	0.0092806307954246\\
351	0.00927575685933796\\
352	0.00927078922273194\\
353	0.00926572604564286\\
354	0.00926056545184171\\
355	0.0092553055281133\\
356	0.00924994432353229\\
357	0.0092444798487218\\
358	0.00923891007509284\\
359	0.00923323293406292\\
360	0.00922744631625256\\
361	0.00922154807065821\\
362	0.00921553600379962\\
363	0.00920940787884001\\
364	0.00920316141467679\\
365	0.00919679428500074\\
366	0.00919030411732093\\
367	0.00918368849195318\\
368	0.00917694494096856\\
369	0.00917007094709933\\
370	0.00916306394259861\\
371	0.00915592130805017\\
372	0.00914864037112456\\
373	0.00914121840527692\\
374	0.00913365262838229\\
375	0.00912594020130307\\
376	0.00911807822638347\\
377	0.00911006374586496\\
378	0.00910189374021642\\
379	0.00909356512637192\\
380	0.00908507475586858\\
381	0.00907641941287592\\
382	0.00906759581210695\\
383	0.00905860059660023\\
384	0.00904943033536034\\
385	0.00904008152084174\\
386	0.0090305505662588\\
387	0.00902083380270101\\
388	0.00901092747602878\\
389	0.00900082774352182\\
390	0.00899053067025269\\
391	0.00898003222517176\\
392	0.00896932827694486\\
393	0.00895841458975695\\
394	0.00894728681979088\\
395	0.00893594051445777\\
396	0.00892437112010351\\
397	0.00891257401331088\\
398	0.00890054464493216\\
399	0.0088882781859002\\
400	0.00887576936668032\\
401	0.00886301275015737\\
402	0.00885000272313758\\
403	0.00883673348728107\\
404	0.00882319904923152\\
405	0.00880939320927808\\
406	0.00879530954672234\\
407	0.00878094139713639\\
408	0.00876628180933352\\
409	0.00875132345258203\\
410	0.00873605840677239\\
411	0.00872047753112605\\
412	0.00870457137762418\\
413	0.00868833256005848\\
414	0.00867175348603844\\
415	0.00865482638721505\\
416	0.00863754333233094\\
417	0.00861989622849766\\
418	0.00860187682343641\\
419	0.00858347670890639\\
420	0.00856468732558878\\
421	0.00854549996961229\\
422	0.00852590580090784\\
423	0.00850589585381964\\
424	0.00848546105090948\\
425	0.00846459222119179\\
426	0.00844328012252689\\
427	0.00842151546943136\\
428	0.00839928896885391\\
429	0.00837659136898946\\
430	0.00835341352906584\\
431	0.00832974651068921\\
432	0.00830558161586922\\
433	0.00828090803381502\\
434	0.00825569546594994\\
435	0.0082299299745818\\
436	0.00820359712049388\\
437	0.00817668193108173\\
438	0.00814916886499131\\
439	0.00812104177272733\\
440	0.00809228385259608\\
441	0.0080628776012164\\
442	0.00803280475766741\\
443	0.0080020462401319\\
444	0.00797058207362718\\
445	0.00793839130707349\\
446	0.00790545191751089\\
447	0.0078717406986935\\
448	0.00783723313045762\\
449	0.00780190322389355\\
450	0.00776572333480546\\
451	0.00772866393397694\\
452	0.00769069331593108\\
453	0.00765177722616763\\
454	0.00761187858682169\\
455	0.00757095681878238\\
456	0.00752896705631929\\
457	0.00748585919816082\\
458	0.00744157667993156\\
459	0.00739605488538871\\
460	0.00734921910371628\\
461	0.00730098196358876\\
462	0.0072512404298274\\
463	0.00719987333402544\\
464	0.00714674623865303\\
465	0.00711760303370909\\
466	0.00710174172557186\\
467	0.00708548243406885\\
468	0.00706880815121044\\
469	0.00705170068183549\\
470	0.0070341405499504\\
471	0.00701610689486377\\
472	0.00699757729699631\\
473	0.00697852765849472\\
474	0.00695893229139063\\
475	0.00693876425878048\\
476	0.00691799586083601\\
477	0.00689659856404932\\
478	0.00687454242174914\\
479	0.00685179624360539\\
480	0.00682832789380348\\
481	0.00680410477086029\\
482	0.00677909454237817\\
483	0.00675326623561546\\
484	0.00672659182333767\\
485	0.00669904849969041\\
486	0.00667062192258152\\
487	0.00664131082227712\\
488	0.00661113354783845\\
489	0.00658013723843903\\
490	0.00654840785314776\\
491	0.00651604512363316\\
492	0.00648302957384543\\
493	0.00644933242630693\\
494	0.00641491642705494\\
495	0.00637972592307721\\
496	0.00634369599575878\\
497	0.006306781000763\\
498	0.00626892954268559\\
499	0.00623008362637209\\
500	0.00619017772034349\\
501	0.00614913773829663\\
502	0.00610687994228259\\
503	0.00606330974987213\\
504	0.00601832036462675\\
505	0.00597179117164384\\
506	0.00592339561337727\\
507	0.00587077049439408\\
508	0.00581308289609968\\
509	0.00574930639485627\\
510	0.00567869529099836\\
511	0.00560683371454598\\
512	0.00553375429773314\\
513	0.00545950829863216\\
514	0.00538417210214052\\
515	0.00530785303151261\\
516	0.00523069700889685\\
517	0.00515289920253759\\
518	0.0050747175010051\\
519	0.00499649061172258\\
520	0.00491866337599272\\
521	0.0048418179356782\\
522	0.0047667115036204\\
523	0.00469432612271735\\
524	0.00462593210387542\\
525	0.00456316190990716\\
526	0.00450693341295911\\
527	0.00444995527340339\\
528	0.00439223575135305\\
529	0.00433371622320349\\
530	0.00427436068933029\\
531	0.00421412867351201\\
532	0.00415304565754318\\
533	0.00409113239980163\\
534	0.00402838588213751\\
535	0.00396476188723626\\
536	0.00390014877352635\\
537	0.00383432965179881\\
538	0.00376693346722126\\
539	0.00369736709965045\\
540	0.00362531423041785\\
541	0.00355071083205017\\
542	0.00347349844520251\\
543	0.00339362692338361\\
544	0.00331105875876555\\
545	0.00322577618843704\\
546	0.00313785322812301\\
547	0.00304735757086433\\
548	0.0029543860589628\\
549	0.00285908640162155\\
550	0.0027616569639807\\
551	0.0026624716140596\\
552	0.00256216130895704\\
553	0.00246170784766462\\
554	0.00236479415202081\\
555	0.00227285197067198\\
556	0.00218638957627433\\
557	0.00210578892634816\\
558	0.00203145988233788\\
559	0.00196123802990767\\
560	0.001893203814526\\
561	0.00182711979869052\\
562	0.00176263153844591\\
563	0.00169927706082731\\
564	0.00163615673196857\\
565	0.00157305390098233\\
566	0.00150978604124525\\
567	0.00144615964986427\\
568	0.00138303411775455\\
569	0.00132135689023745\\
570	0.00126135463693187\\
571	0.00120333184970178\\
572	0.00114626240380204\\
573	0.00108959252485284\\
574	0.00103332374359285\\
575	0.000977489786969432\\
576	0.000923697480075172\\
577	0.000872102306767494\\
578	0.000822274105969262\\
579	0.000773153358595222\\
580	0.000724684439750609\\
581	0.000677213411419557\\
582	0.000630790157390296\\
583	0.000585008827883152\\
584	0.000539740734571962\\
585	0.000494991726043892\\
586	0.000450789565152486\\
587	0.000407159529458106\\
588	0.000364117535614033\\
589	0.000321665953029956\\
590	0.000279789571719997\\
591	0.000238610714896598\\
592	0.000198280712249114\\
593	0.00015897864980585\\
594	0.000120944048332156\\
595	8.45661894165682e-05\\
596	5.06250627329887e-05\\
597	2.09371357655838e-05\\
598	8.08076597025309e-08\\
599	0\\
600	0\\
};
\addplot [color=red!50!mycolor17,solid,forget plot]
  table[row sep=crcr]{%
1	0.00751331918908255\\
2	0.0075133156213837\\
3	0.00751331198975676\\
4	0.00751330829305758\\
5	0.00751330453012152\\
6	0.00751330069976314\\
7	0.00751329680077581\\
8	0.00751329283193138\\
9	0.00751328879197968\\
10	0.00751328467964827\\
11	0.00751328049364192\\
12	0.0075132762326423\\
13	0.00751327189530748\\
14	0.00751326748027161\\
15	0.00751326298614439\\
16	0.0075132584115107\\
17	0.00751325375493015\\
18	0.0075132490149366\\
19	0.00751324419003774\\
20	0.00751323927871459\\
21	0.00751323427942104\\
22	0.00751322919058334\\
23	0.00751322401059969\\
24	0.00751321873783963\\
25	0.0075132133706436\\
26	0.00751320790732242\\
27	0.0075132023461567\\
28	0.00751319668539638\\
29	0.00751319092326015\\
30	0.00751318505793488\\
31	0.00751317908757507\\
32	0.00751317301030226\\
33	0.00751316682420447\\
34	0.00751316052733558\\
35	0.00751315411771471\\
36	0.00751314759332565\\
37	0.00751314095211617\\
38	0.00751313419199741\\
39	0.0075131273108432\\
40	0.00751312030648952\\
41	0.00751311317673362\\
42	0.00751310591933346\\
43	0.007513098532007\\
44	0.00751309101243149\\
45	0.00751308335824269\\
46	0.00751307556703418\\
47	0.00751306763635665\\
48	0.00751305956371703\\
49	0.00751305134657779\\
50	0.0075130429823561\\
51	0.0075130344684231\\
52	0.00751302580210299\\
53	0.00751301698067228\\
54	0.00751300800135882\\
55	0.00751299886134109\\
56	0.00751298955774718\\
57	0.00751298008765401\\
58	0.0075129704480863\\
59	0.00751296063601578\\
60	0.00751295064836016\\
61	0.00751294048198212\\
62	0.00751293013368848\\
63	0.00751291960022906\\
64	0.00751290887829575\\
65	0.00751289796452146\\
66	0.00751288685547905\\
67	0.0075128755476803\\
68	0.00751286403757482\\
69	0.0075128523215489\\
70	0.0075128403959244\\
71	0.00751282825695768\\
72	0.00751281590083831\\
73	0.007512803323688\\
74	0.00751279052155928\\
75	0.00751277749043438\\
76	0.00751276422622394\\
77	0.00751275072476566\\
78	0.00751273698182318\\
79	0.00751272299308457\\
80	0.00751270875416111\\
81	0.00751269426058588\\
82	0.00751267950781242\\
83	0.00751266449121322\\
84	0.0075126492060784\\
85	0.00751263364761414\\
86	0.00751261781094126\\
87	0.00751260169109368\\
88	0.00751258528301688\\
89	0.00751256858156631\\
90	0.00751255158150583\\
91	0.00751253427750604\\
92	0.0075125166641427\\
93	0.00751249873589493\\
94	0.00751248048714359\\
95	0.00751246191216951\\
96	0.00751244300515169\\
97	0.00751242376016553\\
98	0.00751240417118097\\
99	0.00751238423206063\\
100	0.00751236393655789\\
101	0.00751234327831497\\
102	0.00751232225086095\\
103	0.00751230084760975\\
104	0.00751227906185811\\
105	0.00751225688678352\\
106	0.00751223431544205\\
107	0.00751221134076627\\
108	0.00751218795556299\\
109	0.00751216415251111\\
110	0.00751213992415927\\
111	0.0075121152629236\\
112	0.00751209016108533\\
113	0.00751206461078844\\
114	0.0075120386040372\\
115	0.00751201213269365\\
116	0.00751198518847521\\
117	0.00751195776295199\\
118	0.00751192984754424\\
119	0.00751190143351967\\
120	0.00751187251199077\\
121	0.00751184307391205\\
122	0.0075118131100772\\
123	0.00751178261111635\\
124	0.00751175156749306\\
125	0.00751171996950139\\
126	0.00751168780726298\\
127	0.0075116550707239\\
128	0.0075116217496516\\
129	0.00751158783363174\\
130	0.00751155331206493\\
131	0.00751151817416347\\
132	0.00751148240894806\\
133	0.00751144600524437\\
134	0.00751140895167961\\
135	0.00751137123667894\\
136	0.00751133284846203\\
137	0.0075112937750393\\
138	0.00751125400420829\\
139	0.00751121352354985\\
140	0.00751117232042434\\
141	0.00751113038196769\\
142	0.00751108769508747\\
143	0.00751104424645883\\
144	0.00751100002252035\\
145	0.00751095500946995\\
146	0.00751090919326054\\
147	0.00751086255959571\\
148	0.00751081509392535\\
149	0.00751076678144112\\
150	0.00751071760707191\\
151	0.00751066755547921\\
152	0.00751061661105232\\
153	0.00751056475790357\\
154	0.00751051197986342\\
155	0.00751045826047554\\
156	0.00751040358299157\\
157	0.00751034793036609\\
158	0.00751029128525135\\
159	0.0075102336299919\\
160	0.00751017494661916\\
161	0.00751011521684583\\
162	0.0075100544220603\\
163	0.00750999254332096\\
164	0.00750992956135026\\
165	0.00750986545652884\\
166	0.00750980020888944\\
167	0.00750973379811074\\
168	0.00750966620351119\\
169	0.00750959740404252\\
170	0.00750952737828326\\
171	0.00750945610443227\\
172	0.00750938356030185\\
173	0.00750930972331101\\
174	0.00750923457047843\\
175	0.00750915807841544\\
176	0.00750908022331878\\
177	0.00750900098096327\\
178	0.00750892032669432\\
179	0.00750883823542036\\
180	0.00750875468160504\\
181	0.00750866963925942\\
182	0.00750858308193392\\
183	0.00750849498271017\\
184	0.00750840531419268\\
185	0.00750831404850036\\
186	0.00750822115725796\\
187	0.00750812661158735\\
188	0.00750803038209849\\
189	0.00750793243888041\\
190	0.00750783275149198\\
191	0.00750773128895246\\
192	0.00750762801973195\\
193	0.00750752291174161\\
194	0.0075074159323238\\
195	0.00750730704824189\\
196	0.00750719622566998\\
197	0.00750708343018248\\
198	0.00750696862674344\\
199	0.00750685177969559\\
200	0.0075067328527494\\
201	0.00750661180897177\\
202	0.00750648861077457\\
203	0.00750636321990294\\
204	0.00750623559742347\\
205	0.00750610570371204\\
206	0.0075059734984415\\
207	0.00750583894056912\\
208	0.0075057019883238\\
209	0.00750556259919313\\
210	0.00750542072990997\\
211	0.00750527633643907\\
212	0.00750512937396325\\
213	0.00750497979686945\\
214	0.00750482755873441\\
215	0.00750467261231009\\
216	0.00750451490950896\\
217	0.00750435440138882\\
218	0.00750419103813745\\
219	0.00750402476905691\\
220	0.0075038555425477\\
221	0.00750368330609233\\
222	0.00750350800623889\\
223	0.00750332958858402\\
224	0.00750314799775581\\
225	0.00750296317739622\\
226	0.00750277507014313\\
227	0.00750258361761223\\
228	0.00750238876037835\\
229	0.00750219043795652\\
230	0.00750198858878264\\
231	0.00750178315019386\\
232	0.00750157405840841\\
233	0.00750136124850514\\
234	0.00750114465440265\\
235	0.00750092420883799\\
236	0.00750069984334485\\
237	0.00750047148823147\\
238	0.00750023907255793\\
239	0.00750000252411316\\
240	0.00749976176939129\\
241	0.00749951673356767\\
242	0.00749926734047426\\
243	0.00749901351257465\\
244	0.00749875517093839\\
245	0.00749849223521496\\
246	0.00749822462360702\\
247	0.00749795225284318\\
248	0.00749767503815017\\
249	0.00749739289322442\\
250	0.00749710573020301\\
251	0.00749681345963389\\
252	0.00749651599044562\\
253	0.00749621322991626\\
254	0.00749590508364171\\
255	0.00749559145550319\\
256	0.00749527224763411\\
257	0.00749494736038603\\
258	0.00749461669229398\\
259	0.00749428014004084\\
260	0.00749393759842095\\
261	0.00749358896030284\\
262	0.00749323411659098\\
263	0.00749287295618676\\
264	0.00749250536594836\\
265	0.00749213123064963\\
266	0.00749175043293808\\
267	0.00749136285329161\\
268	0.00749096836997425\\
269	0.00749056685899075\\
270	0.00749015819403988\\
271	0.00748974224646664\\
272	0.00748931888521302\\
273	0.00748888797676766\\
274	0.00748844938511376\\
275	0.00748800297167593\\
276	0.00748754859526533\\
277	0.00748708611202328\\
278	0.00748661537536327\\
279	0.00748613623591127\\
280	0.00748564854144434\\
281	0.00748515213682725\\
282	0.00748464686394742\\
283	0.00748413256164761\\
284	0.00748360906565678\\
285	0.00748307620851862\\
286	0.00748253381951788\\
287	0.00748198172460432\\
288	0.00748141974631422\\
289	0.00748084770368935\\
290	0.00748026541219312\\
291	0.00747967268362417\\
292	0.00747906932602679\\
293	0.00747845514359833\\
294	0.00747782993659363\\
295	0.00747719350122577\\
296	0.00747654562956371\\
297	0.00747588610942592\\
298	0.00747521472427055\\
299	0.00747453125308137\\
300	0.00747383547024977\\
301	0.00747312714545239\\
302	0.00747240604352419\\
303	0.00747167192432696\\
304	0.00747092454261289\\
305	0.00747016364788313\\
306	0.00746938898424108\\
307	0.00746860029024021\\
308	0.00746779729872633\\
309	0.00746697973667386\\
310	0.00746614732501635\\
311	0.00746529977847051\\
312	0.00746443680535408\\
313	0.00746355810739715\\
314	0.00746266337954666\\
315	0.00746175230976444\\
316	0.00746082457881805\\
317	0.00745987986006494\\
318	0.00745891781922972\\
319	0.00745793811417424\\
320	0.0074569403946613\\
321	0.00745592430211145\\
322	0.00745488946935351\\
323	0.00745383552036906\\
324	0.00745276207003129\\
325	0.00745166872383874\\
326	0.00745055507764469\\
327	0.00744942071738297\\
328	0.00744826521879105\\
329	0.00744708814713174\\
330	0.00744588905691439\\
331	0.00744466749161755\\
332	0.00744342298341423\\
333	0.0074421550529016\\
334	0.00744086320883715\\
335	0.00743954694788282\\
336	0.00743820575435934\\
337	0.00743683910001211\\
338	0.00743544644379027\\
339	0.00743402723163946\\
340	0.00743258089630789\\
341	0.00743110685716358\\
342	0.00742960452001935\\
343	0.00742807327696499\\
344	0.00742651250620663\\
345	0.00742492157187648\\
346	0.0074232998237597\\
347	0.00742164659698445\\
348	0.00741996121166483\\
349	0.00741824297248595\\
350	0.00741649116822055\\
351	0.0074147050711695\\
352	0.00741288393652355\\
353	0.00741102700164813\\
354	0.00740913348528247\\
355	0.0074072025866016\\
356	0.00740523348425667\\
357	0.00740322533576021\\
358	0.00740117727688395\\
359	0.00739908842104334\\
360	0.00739695785866856\\
361	0.00739478465656299\\
362	0.00739256785724916\\
363	0.00739030647830293\\
364	0.00738799951167641\\
365	0.00738564592301057\\
366	0.00738324465093836\\
367	0.00738079460637923\\
368	0.00737829467182664\\
369	0.00737574370062939\\
370	0.00737314051626897\\
371	0.00737048391163425\\
372	0.00736777264829584\\
373	0.00736500545578255\\
374	0.00736218103086266\\
375	0.00735929803683315\\
376	0.00735635510282097\\
377	0.0073533508231003\\
378	0.00735028375643146\\
379	0.00734715242542737\\
380	0.0073439553159552\\
381	0.00734069087658201\\
382	0.00733735751807535\\
383	0.00733395361297242\\
384	0.00733047749523386\\
385	0.00732692746000308\\
386	0.00732330176349646\\
387	0.00731959862305658\\
388	0.00731581621740913\\
389	0.00731195268717446\\
390	0.00730800613569904\\
391	0.00730397463028914\\
392	0.00729985620395103\\
393	0.00729564885776651\\
394	0.00729135056405371\\
395	0.00728695927045747\\
396	0.00728247290501362\\
397	0.00727788938188812\\
398	0.00727320660572712\\
399	0.00726842248720216\\
400	0.00726353496488103\\
401	0.00725854202302163\\
402	0.00725344171249736\\
403	0.00724823217492987\\
404	0.00724291166983082\\
405	0.00723747860410256\\
406	0.00723193156253973\\
407	0.00722626933693096\\
408	0.00722049094992341\\
409	0.00721459566802056\\
410	0.00720858299605636\\
411	0.00720245264596199\\
412	0.00719620441416646\\
413	0.0071898378927031\\
414	0.00718335206775525\\
415	0.00717674476220912\\
416	0.00717001326913652\\
417	0.0071631547859623\\
418	0.00715616640773693\\
419	0.00714904511890765\\
420	0.00714178778402385\\
421	0.00713439114080366\\
422	0.00712685179541471\\
423	0.00711916621227635\\
424	0.00711133068549084\\
425	0.00710334129221052\\
426	0.00709519387013695\\
427	0.00708688399455825\\
428	0.00707840695613547\\
429	0.00706975774125258\\
430	0.00706093101792183\\
431	0.00705192113375732\\
432	0.00704272214416629\\
433	0.00703332796193707\\
434	0.00702373294773449\\
435	0.00701393120207981\\
436	0.00700391655140177\\
437	0.0069936825344676\\
438	0.0069832223897311\\
439	0.00697252904434603\\
440	0.00696159510587803\\
441	0.00695041285813078\\
442	0.00693897426301689\\
443	0.00692727097109733\\
444	0.00691529434434937\\
445	0.00690303549599264\\
446	0.00689048535393341\\
447	0.00687763475674534\\
448	0.00686447459420903\\
449	0.0068509960078202\\
450	0.0068371906666808\\
451	0.0068230511124438\\
452	0.00680857120374722\\
453	0.00679374686260377\\
454	0.0067785772876793\\
455	0.00676306679408124\\
456	0.0067472266724489\\
457	0.00673107726701702\\
458	0.00671465139756564\\
459	0.00669799900776142\\
460	0.00668119349506899\\
461	0.00666434029973432\\
462	0.00664758812823895\\
463	0.00663113551009666\\
464	0.00661512930843446\\
465	0.00659910025852906\\
466	0.00658275089316905\\
467	0.00656606606563277\\
468	0.00654902866702232\\
469	0.00653161929988478\\
470	0.00651381596936665\\
471	0.00649559399653717\\
472	0.00647692687568139\\
473	0.00645777879148589\\
474	0.0064381044759488\\
475	0.00641784229739641\\
476	0.00639692258033515\\
477	0.00637527679537599\\
478	0.00635282356452617\\
479	0.00632946576441206\\
480	0.00630508689967567\\
481	0.0062795465466429\\
482	0.00625267460939819\\
483	0.00622426405575886\\
484	0.00619406170665633\\
485	0.00616175654183948\\
486	0.00612696488325146\\
487	0.00608921183738913\\
488	0.0060479089545663\\
489	0.00600233083482408\\
490	0.00595232342282874\\
491	0.00590148947569404\\
492	0.00584983403098362\\
493	0.00579736741938523\\
494	0.00574410690284839\\
495	0.00569007911631615\\
496	0.00563532253640301\\
497	0.00557988916743546\\
498	0.00552384799224969\\
499	0.005467289312921\\
500	0.00541033021765437\\
501	0.00535312147365448\\
502	0.0052958562282125\\
503	0.00523878100784451\\
504	0.00518220964821595\\
505	0.00512654099253959\\
506	0.0050722868923754\\
507	0.0050201594477792\\
508	0.00497107603847681\\
509	0.00492621083689633\\
510	0.00488651077423788\\
511	0.00484627150528883\\
512	0.00480552356951627\\
513	0.00476430382160936\\
514	0.00472265405186197\\
515	0.00468062062098477\\
516	0.00463825406006321\\
517	0.00459560960702857\\
518	0.00455273966843344\\
519	0.00450966338747461\\
520	0.00446633863580003\\
521	0.00442273329243951\\
522	0.00437879270968387\\
523	0.0043343963397969\\
524	0.00428932531676191\\
525	0.00424321732666755\\
526	0.00419558366898516\\
527	0.00414634016474393\\
528	0.00409539587910696\\
529	0.00404265584542624\\
530	0.0039880203821123\\
531	0.00393138511369978\\
532	0.00387263862740562\\
533	0.00381166298584016\\
534	0.00374833592451626\\
535	0.00368253669584913\\
536	0.00361419650138181\\
537	0.00354326709786294\\
538	0.00346968510843991\\
539	0.00339343017642894\\
540	0.00331452199307313\\
541	0.0032330184516686\\
542	0.00314902423490255\\
543	0.00306269953612847\\
544	0.00297426046031835\\
545	0.00288397108472537\\
546	0.00279215367453738\\
547	0.00269940852842206\\
548	0.00260671480571992\\
549	0.00251767802398831\\
550	0.00243374354967145\\
551	0.00235539842844913\\
552	0.00228321043198779\\
553	0.00221728694122106\\
554	0.00215441480369381\\
555	0.00209351059187338\\
556	0.00203426354693982\\
557	0.00197625458039651\\
558	0.00191861618170396\\
559	0.00186090343293077\\
560	0.00180297106052822\\
561	0.00174465815958719\\
562	0.00168579887758462\\
563	0.0016262284036316\\
564	0.00156617231964438\\
565	0.00150724890161879\\
566	0.00144968165214924\\
567	0.0013937628332597\\
568	0.00133873781490288\\
569	0.00128378748368581\\
570	0.00122891793735732\\
571	0.00117410293550197\\
572	0.00111934982997828\\
573	0.00106563016233201\\
574	0.00101384323193758\\
575	0.000964080244343637\\
576	0.000914782368524028\\
577	0.000865906095930388\\
578	0.000817416739590842\\
579	0.0007696950288388\\
580	0.000722860329501604\\
581	0.000676473369872012\\
582	0.000630449173627159\\
583	0.000584820757755489\\
584	0.00053963388785225\\
585	0.000494933672926577\\
586	0.00045076036357922\\
587	0.000407146253694499\\
588	0.00036411228580542\\
589	0.00032166424714621\\
590	0.000279789162700311\\
591	0.000238610660212778\\
592	0.000198280712249111\\
593	0.000158978649805849\\
594	0.000120944048332155\\
595	8.45661894165685e-05\\
596	5.06250627329882e-05\\
597	2.09371357655842e-05\\
598	8.08076597025309e-08\\
599	0\\
600	0\\
};
\addplot [color=red!40!mycolor19,solid,forget plot]
  table[row sep=crcr]{%
1	0.00714657264018434\\
2	0.00714656917865746\\
3	0.00714656565501015\\
4	0.00714656206812864\\
5	0.00714655841687927\\
6	0.00714655470010801\\
7	0.00714655091664021\\
8	0.00714654706528017\\
9	0.00714654314481079\\
10	0.00714653915399322\\
11	0.00714653509156638\\
12	0.00714653095624664\\
13	0.0071465267467274\\
14	0.00714652246167865\\
15	0.00714651809974661\\
16	0.00714651365955325\\
17	0.00714650913969589\\
18	0.00714650453874671\\
19	0.0071464998552524\\
20	0.00714649508773362\\
21	0.00714649023468453\\
22	0.00714648529457241\\
23	0.00714648026583704\\
24	0.00714647514689036\\
25	0.00714646993611585\\
26	0.00714646463186808\\
27	0.00714645923247221\\
28	0.00714645373622347\\
29	0.0071464481413865\\
30	0.00714644244619499\\
31	0.007146436648851\\
32	0.00714643074752446\\
33	0.00714642474035254\\
34	0.00714641862543914\\
35	0.00714641240085424\\
36	0.0071464060646333\\
37	0.00714639961477668\\
38	0.00714639304924896\\
39	0.00714638636597839\\
40	0.00714637956285609\\
41	0.00714637263773561\\
42	0.00714636558843204\\
43	0.00714635841272146\\
44	0.0071463511083402\\
45	0.00714634367298414\\
46	0.007146336104308\\
47	0.00714632839992457\\
48	0.00714632055740398\\
49	0.007146312574273\\
50	0.00714630444801419\\
51	0.00714629617606514\\
52	0.00714628775581766\\
53	0.00714627918461698\\
54	0.00714627045976095\\
55	0.00714626157849915\\
56	0.00714625253803201\\
57	0.00714624333551005\\
58	0.00714623396803286\\
59	0.00714622443264824\\
60	0.00714621472635131\\
61	0.00714620484608359\\
62	0.00714619478873197\\
63	0.00714618455112777\\
64	0.00714617413004582\\
65	0.00714616352220334\\
66	0.00714615272425904\\
67	0.00714614173281198\\
68	0.00714613054440053\\
69	0.00714611915550136\\
70	0.0071461075625283\\
71	0.00714609576183117\\
72	0.00714608374969477\\
73	0.0071460715223376\\
74	0.00714605907591077\\
75	0.00714604640649677\\
76	0.00714603351010825\\
77	0.0071460203826868\\
78	0.00714600702010168\\
79	0.00714599341814853\\
80	0.0071459795725481\\
81	0.00714596547894489\\
82	0.00714595113290578\\
83	0.00714593652991875\\
84	0.00714592166539138\\
85	0.00714590653464948\\
86	0.00714589113293561\\
87	0.00714587545540766\\
88	0.0071458594971373\\
89	0.00714584325310851\\
90	0.00714582671821597\\
91	0.00714580988726351\\
92	0.00714579275496249\\
93	0.00714577531593022\\
94	0.00714575756468826\\
95	0.0071457394956607\\
96	0.0071457211031725\\
97	0.00714570238144771\\
98	0.00714568332460771\\
99	0.00714566392666934\\
100	0.00714564418154312\\
101	0.00714562408303136\\
102	0.00714560362482624\\
103	0.00714558280050788\\
104	0.00714556160354241\\
105	0.00714554002727987\\
106	0.00714551806495224\\
107	0.00714549570967133\\
108	0.0071454729544267\\
109	0.00714544979208343\\
110	0.00714542621538006\\
111	0.00714540221692629\\
112	0.00714537778920068\\
113	0.0071453529245484\\
114	0.0071453276151789\\
115	0.00714530185316353\\
116	0.00714527563043308\\
117	0.00714524893877531\\
118	0.00714522176983246\\
119	0.00714519411509875\\
120	0.00714516596591772\\
121	0.00714513731347961\\
122	0.00714510814881871\\
123	0.00714507846281056\\
124	0.00714504824616924\\
125	0.00714501748944457\\
126	0.00714498618301913\\
127	0.00714495431710545\\
128	0.00714492188174297\\
129	0.00714488886679507\\
130	0.00714485526194597\\
131	0.00714482105669761\\
132	0.00714478624036647\\
133	0.00714475080208035\\
134	0.00714471473077508\\
135	0.0071446780151912\\
136	0.00714464064387051\\
137	0.00714460260515269\\
138	0.00714456388717172\\
139	0.00714452447785233\\
140	0.0071444843649064\\
141	0.0071444435358292\\
142	0.00714440197789573\\
143	0.00714435967815677\\
144	0.00714431662343513\\
145	0.00714427280032162\\
146	0.00714422819517108\\
147	0.00714418279409824\\
148	0.00714413658297368\\
149	0.00714408954741947\\
150	0.00714404167280502\\
151	0.00714399294424257\\
152	0.00714394334658293\\
153	0.00714389286441086\\
154	0.00714384148204049\\
155	0.00714378918351068\\
156	0.00714373595258033\\
157	0.00714368177272353\\
158	0.00714362662712465\\
159	0.00714357049867339\\
160	0.00714351336995973\\
161	0.00714345522326881\\
162	0.00714339604057568\\
163	0.00714333580353995\\
164	0.00714327449350047\\
165	0.00714321209146984\\
166	0.00714314857812884\\
167	0.00714308393382068\\
168	0.00714301813854538\\
169	0.00714295117195378\\
170	0.00714288301334178\\
171	0.00714281364164408\\
172	0.00714274303542822\\
173	0.00714267117288823\\
174	0.00714259803183838\\
175	0.00714252358970669\\
176	0.00714244782352838\\
177	0.00714237070993924\\
178	0.00714229222516887\\
179	0.00714221234503383\\
180	0.00714213104493067\\
181	0.0071420482998288\\
182	0.00714196408426332\\
183	0.00714187837232775\\
184	0.00714179113766653\\
185	0.00714170235346752\\
186	0.00714161199245437\\
187	0.00714152002687858\\
188	0.00714142642851182\\
189	0.00714133116863769\\
190	0.00714123421804369\\
191	0.00714113554701286\\
192	0.00714103512531542\\
193	0.00714093292220017\\
194	0.00714082890638584\\
195	0.00714072304605226\\
196	0.00714061530883143\\
197	0.00714050566179836\\
198	0.00714039407146192\\
199	0.00714028050375539\\
200	0.00714016492402699\\
201	0.00714004729703019\\
202	0.00713992758691388\\
203	0.00713980575721244\\
204	0.0071396817708356\\
205	0.00713955559005813\\
206	0.00713942717650946\\
207	0.00713929649116307\\
208	0.00713916349432572\\
209	0.00713902814562656\\
210	0.00713889040400605\\
211	0.00713875022770468\\
212	0.0071386075742516\\
213	0.00713846240045298\\
214	0.00713831466238026\\
215	0.00713816431535826\\
216	0.00713801131395299\\
217	0.00713785561195933\\
218	0.00713769716238868\\
219	0.00713753591745615\\
220	0.00713737182856772\\
221	0.00713720484630727\\
222	0.00713703492042322\\
223	0.00713686199981517\\
224	0.00713668603252024\\
225	0.00713650696569916\\
226	0.00713632474562234\\
227	0.0071361393176555\\
228	0.00713595062624524\\
229	0.0071357586149044\\
230	0.00713556322619708\\
231	0.00713536440172359\\
232	0.00713516208210508\\
233	0.00713495620696794\\
234	0.00713474671492803\\
235	0.0071345335435747\\
236	0.00713431662945441\\
237	0.00713409590805426\\
238	0.00713387131378533\\
239	0.0071336427799655\\
240	0.00713341023880234\\
241	0.00713317362137555\\
242	0.00713293285761916\\
243	0.0071326878763035\\
244	0.00713243860501696\\
245	0.00713218497014726\\
246	0.00713192689686272\\
247	0.00713166430909301\\
248	0.00713139712950981\\
249	0.00713112527950697\\
250	0.00713084867918049\\
251	0.00713056724730819\\
252	0.00713028090132904\\
253	0.00712998955732214\\
254	0.00712969312998538\\
255	0.0071293915326138\\
256	0.00712908467707759\\
257	0.00712877247379967\\
258	0.00712845483173296\\
259	0.00712813165833729\\
260	0.0071278028595559\\
261	0.00712746833979148\\
262	0.00712712800188196\\
263	0.00712678174707566\\
264	0.00712642947500619\\
265	0.00712607108366683\\
266	0.00712570646938435\\
267	0.00712533552679254\\
268	0.00712495814880507\\
269	0.00712457422658792\\
270	0.00712418364953122\\
271	0.00712378630522052\\
272	0.00712338207940754\\
273	0.00712297085598023\\
274	0.00712255251693224\\
275	0.00712212694233171\\
276	0.00712169401028928\\
277	0.0071212535969255\\
278	0.00712080557633741\\
279	0.00712034982056423\\
280	0.00711988619955227\\
281	0.00711941458111901\\
282	0.00711893483091603\\
283	0.00711844681239111\\
284	0.00711795038674917\\
285	0.00711744541291214\\
286	0.00711693174747759\\
287	0.00711640924467615\\
288	0.00711587775632744\\
289	0.00711533713179474\\
290	0.00711478721793801\\
291	0.00711422785906531\\
292	0.00711365889688249\\
293	0.00711308017044107\\
294	0.00711249151608388\\
295	0.00711189276738897\\
296	0.00711128375511087\\
297	0.00711066430711961\\
298	0.00711003424833701\\
299	0.00710939340067021\\
300	0.00710874158294212\\
301	0.00710807861081861\\
302	0.0071074042967322\\
303	0.00710671844980197\\
304	0.00710602087574934\\
305	0.00710531137680951\\
306	0.00710458975163814\\
307	0.00710385579521286\\
308	0.00710310929872931\\
309	0.00710235004949119\\
310	0.00710157783079375\\
311	0.00710079242180049\\
312	0.00709999359741213\\
313	0.00709918112812751\\
314	0.00709835477989571\\
315	0.00709751431395842\\
316	0.00709665948668224\\
317	0.00709579004937964\\
318	0.00709490574811789\\
319	0.00709400632351507\\
320	0.00709309151052191\\
321	0.00709216103818836\\
322	0.00709121462941418\\
323	0.00709025200068154\\
324	0.00708927286176902\\
325	0.00708827691544535\\
326	0.00708726385714153\\
327	0.00708623337460003\\
328	0.00708518514749971\\
329	0.00708411884705497\\
330	0.00708303413558839\\
331	0.00708193066607538\\
332	0.00708080808166053\\
333	0.00707966601514545\\
334	0.0070785040884486\\
335	0.00707732191203858\\
336	0.00707611908434388\\
337	0.00707489519114425\\
338	0.00707364980495158\\
339	0.00707238248439203\\
340	0.00707109277360407\\
341	0.00706978020166163\\
342	0.0070684442820041\\
343	0.00706708451180575\\
344	0.00706570037142761\\
345	0.00706429132490858\\
346	0.00706285682132353\\
347	0.00706139629495288\\
348	0.00705990916557185\\
349	0.00705839483885269\\
350	0.00705685270685883\\
351	0.00705528214859857\\
352	0.00705368253062184\\
353	0.00705205320775457\\
354	0.00705039352432943\\
355	0.00704870281612754\\
356	0.00704698040757701\\
357	0.00704522560213191\\
358	0.00704343768052313\\
359	0.00704161589954795\\
360	0.00703975949077013\\
361	0.00703786765912253\\
362	0.00703593958140317\\
363	0.00703397440465522\\
364	0.00703197124441994\\
365	0.00702992918285052\\
366	0.0070278472666739\\
367	0.00702572450498543\\
368	0.00702355986686047\\
369	0.00702135227876509\\
370	0.00701910062174535\\
371	0.00701680372837377\\
372	0.00701446037942807\\
373	0.0070120693002753\\
374	0.00700962915693134\\
375	0.00700713855176251\\
376	0.00700459601879298\\
377	0.00700200001857738\\
378	0.0069993489325944\\
379	0.00699664105711292\\
380	0.00699387459647759\\
381	0.00699104765575634\\
382	0.00698815823268833\\
383	0.00698520420886597\\
384	0.00698218334008164\\
385	0.00697909324576755\\
386	0.00697593139745581\\
387	0.00697269510618878\\
388	0.006969381508815\\
389	0.00696598755311835\\
390	0.00696250998174794\\
391	0.00695894531494774\\
392	0.00695528983213258\\
393	0.00695153955242708\\
394	0.0069476902143836\\
395	0.00694373725523702\\
396	0.00693967579025773\\
397	0.00693550059306791\\
398	0.00693120607827918\\
399	0.00692678628794537\\
400	0.00692223488422257\\
401	0.00691754515203319\\
402	0.00691271001675149\\
403	0.00690772208390629\\
404	0.00690257371061505\\
405	0.00689725712219679\\
406	0.00689176459255616\\
407	0.00688608871405062\\
408	0.00688022279250597\\
409	0.00687416141718248\\
410	0.00686790127599154\\
411	0.0068614423163579\\
412	0.00685478939649825\\
413	0.00684795461375176\\
414	0.00684096040235692\\
415	0.00683384148308431\\
416	0.00682660981061781\\
417	0.0068192631263709\\
418	0.00681179903824656\\
419	0.00680421500634271\\
420	0.0067965082937029\\
421	0.00678867585331783\\
422	0.00678071431369462\\
423	0.00677262014830582\\
424	0.0067643901459434\\
425	0.00675602142713426\\
426	0.00674751095422996\\
427	0.00673885551499242\\
428	0.00673005170384211\\
429	0.00672109590013635\\
430	0.00671198424270403\\
431	0.00670271259969299\\
432	0.00669327653186183\\
433	0.00668367124194257\\
434	0.00667389148944368\\
435	0.00666393151979407\\
436	0.00665378497985085\\
437	0.00664344481702159\\
438	0.00663290315843133\\
439	0.00662215116570827\\
440	0.0066111788598837\\
441	0.0065999749095405\\
442	0.00658852637362242\\
443	0.0065768183881377\\
444	0.00656483378324333\\
445	0.00655255261378508\\
446	0.00653995158230892\\
447	0.00652700332936857\\
448	0.00651367556391058\\
449	0.00649993001496426\\
450	0.00648572123375175\\
451	0.00647099546167405\\
452	0.00645568759692071\\
453	0.00643971453647389\\
454	0.00642297031736661\\
455	0.0064053203782991\\
456	0.00638661585051331\\
457	0.0063666733325996\\
458	0.00634526589107365\\
459	0.00632211168055256\\
460	0.00629685992708674\\
461	0.00626907512780289\\
462	0.00623822496005352\\
463	0.00620666097933113\\
464	0.00617455668529888\\
465	0.00614189884713309\\
466	0.00610868115772363\\
467	0.00607489816469105\\
468	0.00604054552593843\\
469	0.00600562032459422\\
470	0.00597012144867193\\
471	0.00593405005493817\\
472	0.00589741013366782\\
473	0.00586020951324984\\
474	0.00582246105559841\\
475	0.00578418440331258\\
476	0.00574540766878462\\
477	0.00570616913337013\\
478	0.0056665196574332\\
479	0.00562652601656258\\
480	0.00558627515479392\\
481	0.00554587961998691\\
482	0.00550548452728018\\
483	0.00546527650203704\\
484	0.00542549519063513\\
485	0.00538644809916497\\
486	0.00534852970999875\\
487	0.00531224592862163\\
488	0.00527824445523943\\
489	0.00524734870416829\\
490	0.00521981817435915\\
491	0.00519195639445821\\
492	0.00516378102273229\\
493	0.00513531276280783\\
494	0.00510657558401073\\
495	0.00507759686638866\\
496	0.00504840744928566\\
497	0.00501904159034034\\
498	0.00498953672929502\\
499	0.0049599329658641\\
500	0.00493027212386672\\
501	0.00490059622468179\\
502	0.00487094508999067\\
503	0.00484135271042037\\
504	0.00481184195663722\\
505	0.00478241696832337\\
506	0.00475305337953876\\
507	0.00472368184815616\\
508	0.00469416436904183\\
509	0.0046642624601736\\
510	0.00463363262209166\\
511	0.00460222942488623\\
512	0.00456996979571726\\
513	0.00453680937699965\\
514	0.00450269869313749\\
515	0.00446758235459592\\
516	0.00443139795638658\\
517	0.00439407347690287\\
518	0.00435552807572232\\
519	0.00431567218134258\\
520	0.0042744086345831\\
521	0.00423163184899988\\
522	0.00418722824832533\\
523	0.00414107872329734\\
524	0.00409306338657229\\
525	0.00404306998947941\\
526	0.00399104530187994\\
527	0.0039369274520662\\
528	0.00388062704587861\\
529	0.00382205742854237\\
530	0.00376113781934429\\
531	0.00369780174061995\\
532	0.0036320048997912\\
533	0.00356371943436877\\
534	0.00349293937377957\\
535	0.00341968836726662\\
536	0.00334399163295975\\
537	0.00326591109948276\\
538	0.00318560493753778\\
539	0.00310329276288081\\
540	0.00301924955388223\\
541	0.00293382333155161\\
542	0.00284757339143152\\
543	0.00276138886179295\\
544	0.00267821147993839\\
545	0.00260017302799815\\
546	0.00252792746801078\\
547	0.00246196393283174\\
548	0.00240220335436286\\
549	0.00234541275913034\\
550	0.00229046703658218\\
551	0.00223701840533003\\
552	0.00218432272635319\\
553	0.0021316553611119\\
554	0.00207876493238114\\
555	0.00202549869799243\\
556	0.00197171014068998\\
557	0.0019172681463836\\
558	0.00186209624768108\\
559	0.00180609021806162\\
560	0.00174914806723949\\
561	0.0016924567180754\\
562	0.00163686024127447\\
563	0.00158262548411114\\
564	0.00152972933497633\\
565	0.00147670430029631\\
566	0.00142354415148953\\
567	0.00137021221651862\\
568	0.00131671154650725\\
569	0.00126310473339564\\
570	0.00120942928907071\\
571	0.00115724186764634\\
572	0.00110694359459505\\
573	0.00105770968033556\\
574	0.00100874195106522\\
575	0.000959975015567494\\
576	0.000911451162123936\\
577	0.000863439751431702\\
578	0.000816206771206118\\
579	0.000769355734681053\\
580	0.000722745090630671\\
581	0.000676414840349812\\
582	0.00063041663004274\\
583	0.000584802944474557\\
584	0.00053962469606257\\
585	0.000494929310322039\\
586	0.000450758505923288\\
587	0.000407145570263204\\
588	0.000364112080950015\\
589	0.000321664202373709\\
590	0.000279789157284614\\
591	0.000238610660212779\\
592	0.000198280712249111\\
593	0.000158978649805848\\
594	0.000120944048332155\\
595	8.45661894165685e-05\\
596	5.06250627329882e-05\\
597	2.09371357655843e-05\\
598	8.08076597025309e-08\\
599	0\\
600	0\\
};
\addplot [color=red!75!mycolor17,solid,forget plot]
  table[row sep=crcr]{%
1	0.00697443521740056\\
2	0.00697443001047625\\
3	0.00697442471006976\\
4	0.00697441931450414\\
5	0.00697441382207237\\
6	0.00697440823103689\\
7	0.00697440253962896\\
8	0.00697439674604819\\
9	0.0069743908484619\\
10	0.00697438484500457\\
11	0.00697437873377733\\
12	0.00697437251284724\\
13	0.00697436618024668\\
14	0.00697435973397289\\
15	0.00697435317198721\\
16	0.00697434649221441\\
17	0.00697433969254216\\
18	0.00697433277082024\\
19	0.00697432572485995\\
20	0.00697431855243338\\
21	0.00697431125127274\\
22	0.00697430381906961\\
23	0.00697429625347424\\
24	0.00697428855209482\\
25	0.0069742807124967\\
26	0.00697427273220164\\
27	0.00697426460868702\\
28	0.00697425633938504\\
29	0.006974247921682\\
30	0.00697423935291735\\
31	0.0069742306303829\\
32	0.00697422175132206\\
33	0.00697421271292883\\
34	0.00697420351234702\\
35	0.00697419414666927\\
36	0.00697418461293626\\
37	0.00697417490813566\\
38	0.00697416502920126\\
39	0.00697415497301196\\
40	0.00697414473639084\\
41	0.00697413431610409\\
42	0.00697412370886009\\
43	0.00697411291130832\\
44	0.00697410192003827\\
45	0.0069740907315785\\
46	0.00697407934239543\\
47	0.00697406774889226\\
48	0.0069740559474079\\
49	0.00697404393421572\\
50	0.00697403170552249\\
51	0.00697401925746717\\
52	0.00697400658611959\\
53	0.00697399368747939\\
54	0.00697398055747464\\
55	0.00697396719196062\\
56	0.00697395358671851\\
57	0.00697393973745406\\
58	0.00697392563979629\\
59	0.00697391128929608\\
60	0.00697389668142476\\
61	0.00697388181157276\\
62	0.0069738666750481\\
63	0.00697385126707503\\
64	0.00697383558279236\\
65	0.00697381961725215\\
66	0.00697380336541796\\
67	0.00697378682216346\\
68	0.00697376998227072\\
69	0.00697375284042857\\
70	0.00697373539123106\\
71	0.00697371762917563\\
72	0.00697369954866146\\
73	0.00697368114398776\\
74	0.00697366240935194\\
75	0.00697364333884779\\
76	0.00697362392646368\\
77	0.00697360416608069\\
78	0.00697358405147064\\
79	0.00697356357629425\\
80	0.00697354273409911\\
81	0.0069735215183176\\
82	0.00697349992226501\\
83	0.0069734779391373\\
84	0.00697345556200908\\
85	0.00697343278383145\\
86	0.00697340959742978\\
87	0.00697338599550146\\
88	0.00697336197061373\\
89	0.00697333751520123\\
90	0.00697331262156375\\
91	0.00697328728186382\\
92	0.00697326148812429\\
93	0.00697323523222581\\
94	0.00697320850590436\\
95	0.00697318130074866\\
96	0.00697315360819757\\
97	0.00697312541953744\\
98	0.0069730967258994\\
99	0.00697306751825665\\
100	0.00697303778742164\\
101	0.00697300752404324\\
102	0.00697297671860382\\
103	0.00697294536141635\\
104	0.00697291344262135\\
105	0.00697288095218391\\
106	0.00697284787989053\\
107	0.00697281421534606\\
108	0.00697277994797036\\
109	0.00697274506699516\\
110	0.00697270956146066\\
111	0.0069726734202122\\
112	0.00697263663189682\\
113	0.00697259918495971\\
114	0.00697256106764076\\
115	0.00697252226797083\\
116	0.00697248277376813\\
117	0.00697244257263449\\
118	0.00697240165195154\\
119	0.00697235999887675\\
120	0.00697231760033966\\
121	0.00697227444303773\\
122	0.00697223051343234\\
123	0.00697218579774455\\
124	0.00697214028195105\\
125	0.00697209395177966\\
126	0.00697204679270516\\
127	0.00697199878994466\\
128	0.00697194992845329\\
129	0.00697190019291937\\
130	0.00697184956775999\\
131	0.00697179803711608\\
132	0.00697174558484764\\
133	0.00697169219452878\\
134	0.00697163784944282\\
135	0.00697158253257707\\
136	0.00697152622661777\\
137	0.0069714689139448\\
138	0.00697141057662628\\
139	0.00697135119641317\\
140	0.00697129075473378\\
141	0.00697122923268806\\
142	0.00697116661104191\\
143	0.00697110287022139\\
144	0.00697103799030681\\
145	0.00697097195102661\\
146	0.00697090473175136\\
147	0.00697083631148751\\
148	0.00697076666887101\\
149	0.00697069578216096\\
150	0.00697062362923296\\
151	0.00697055018757257\\
152	0.00697047543426849\\
153	0.00697039934600569\\
154	0.00697032189905842\\
155	0.00697024306928314\\
156	0.00697016283211122\\
157	0.00697008116254163\\
158	0.00696999803513354\\
159	0.00696991342399857\\
160	0.00696982730279327\\
161	0.00696973964471107\\
162	0.00696965042247446\\
163	0.00696955960832682\\
164	0.00696946717402418\\
165	0.00696937309082682\\
166	0.00696927732949077\\
167	0.00696917986025925\\
168	0.00696908065285367\\
169	0.00696897967646489\\
170	0.00696887689974398\\
171	0.0069687722907931\\
172	0.00696866581715602\\
173	0.00696855744580868\\
174	0.00696844714314939\\
175	0.00696833487498901\\
176	0.006968220606541\\
177	0.00696810430241118\\
178	0.0069679859265874\\
179	0.00696786544242908\\
180	0.0069677428126565\\
181	0.00696761799933993\\
182	0.00696749096388868\\
183	0.00696736166703985\\
184	0.00696723006884699\\
185	0.00696709612866853\\
186	0.00696695980515608\\
187	0.00696682105624244\\
188	0.00696667983912954\\
189	0.00696653611027615\\
190	0.00696638982538533\\
191	0.0069662409393918\\
192	0.00696608940644899\\
193	0.00696593517991599\\
194	0.00696577821234423\\
195	0.00696561845546401\\
196	0.00696545586017077\\
197	0.00696529037651113\\
198	0.00696512195366882\\
199	0.00696495053995026\\
200	0.00696477608277006\\
201	0.00696459852863611\\
202	0.00696441782313461\\
203	0.00696423391091491\\
204	0.00696404673567381\\
205	0.00696385624014006\\
206	0.00696366236605827\\
207	0.00696346505417275\\
208	0.00696326424421109\\
209	0.00696305987486742\\
210	0.00696285188378554\\
211	0.00696264020754168\\
212	0.0069624247816271\\
213	0.00696220554043039\\
214	0.00696198241721948\\
215	0.00696175534412348\\
216	0.00696152425211412\\
217	0.00696128907098708\\
218	0.00696104972934289\\
219	0.00696080615456768\\
220	0.00696055827281363\\
221	0.00696030600897903\\
222	0.0069600492866882\\
223	0.0069597880282711\\
224	0.00695952215474255\\
225	0.00695925158578129\\
226	0.00695897623970865\\
227	0.00695869603346696\\
228	0.00695841088259771\\
229	0.00695812070121927\\
230	0.0069578254020045\\
231	0.00695752489615792\\
232	0.00695721909339251\\
233	0.00695690790190647\\
234	0.00695659122835932\\
235	0.00695626897784792\\
236	0.0069559410538821\\
237	0.00695560735836001\\
238	0.00695526779154301\\
239	0.00695492225203047\\
240	0.00695457063673399\\
241	0.00695421284085145\\
242	0.00695384875784068\\
243	0.00695347827939285\\
244	0.00695310129540542\\
245	0.00695271769395484\\
246	0.00695232736126893\\
247	0.00695193018169879\\
248	0.00695152603769053\\
249	0.00695111480975653\\
250	0.0069506963764465\\
251	0.00695027061431801\\
252	0.00694983739790687\\
253	0.00694939659969698\\
254	0.00694894809009003\\
255	0.00694849173737464\\
256	0.00694802740769536\\
257	0.00694755496502116\\
258	0.00694707427111369\\
259	0.00694658518549513\\
260	0.00694608756541571\\
261	0.00694558126582085\\
262	0.00694506613931807\\
263	0.00694454203614343\\
264	0.00694400880412766\\
265	0.00694346628866201\\
266	0.00694291433266372\\
267	0.00694235277654112\\
268	0.00694178145815843\\
269	0.00694120021280022\\
270	0.0069406088731356\\
271	0.00694000726918194\\
272	0.00693939522826844\\
273	0.0069387725749992\\
274	0.00693813913121607\\
275	0.00693749471596127\\
276	0.00693683914543946\\
277	0.00693617223297977\\
278	0.00693549378899727\\
279	0.0069348036209544\\
280	0.00693410153332178\\
281	0.00693338732753913\\
282	0.00693266080197545\\
283	0.00693192175188935\\
284	0.00693116996938874\\
285	0.00693040524339043\\
286	0.00692962735957942\\
287	0.00692883610036785\\
288	0.00692803124485378\\
289	0.00692721256877962\\
290	0.00692637984449023\\
291	0.00692553284089095\\
292	0.00692467132340508\\
293	0.00692379505393127\\
294	0.00692290379080065\\
295	0.00692199728873348\\
296	0.00692107529879562\\
297	0.00692013756835471\\
298	0.0069191838410359\\
299	0.00691821385667738\\
300	0.0069172273512854\\
301	0.006916224056989\\
302	0.00691520370199416\\
303	0.00691416601053762\\
304	0.00691311070284016\\
305	0.00691203749505916\\
306	0.00691094609924062\\
307	0.00690983622327046\\
308	0.00690870757082499\\
309	0.00690755984132041\\
310	0.00690639272986126\\
311	0.00690520592718775\\
312	0.00690399911962161\\
313	0.0069027719890104\\
314	0.00690152421266996\\
315	0.00690025546332487\\
316	0.0068989654090463\\
317	0.00689765371318724\\
318	0.00689632003431416\\
319	0.00689496402613508\\
320	0.00689358533742299\\
321	0.00689218361193402\\
322	0.00689075848831922\\
323	0.00688930960002917\\
324	0.00688783657520971\\
325	0.00688633903658714\\
326	0.00688481660134102\\
327	0.00688326888096191\\
328	0.00688169548109105\\
329	0.00688009600133819\\
330	0.00687847003507264\\
331	0.00687681716918209\\
332	0.00687513698379149\\
333	0.00687342905193334\\
334	0.00687169293915756\\
335	0.00686992820306813\\
336	0.0068681343927699\\
337	0.00686631104820998\\
338	0.00686445769940255\\
339	0.00686257386554751\\
340	0.00686065905410599\\
341	0.00685871275999134\\
342	0.00685673446504807\\
343	0.00685472363699583\\
344	0.00685267972018958\\
345	0.00685060210801386\\
346	0.00684849014111145\\
347	0.00684634312846953\\
348	0.0068441603446841\\
349	0.00684194102711355\\
350	0.00683968437283323\\
351	0.00683738953509028\\
352	0.00683505561859495\\
353	0.00683268167289168\\
354	0.00683026668610741\\
355	0.00682780960132306\\
356	0.00682530947271196\\
357	0.00682276549800276\\
358	0.00682017686616917\\
359	0.00681754274509141\\
360	0.00681486228039612\\
361	0.00681213459418553\\
362	0.0068093587836425\\
363	0.006806533919496\\
364	0.0068036590443294\\
365	0.0068007331707113\\
366	0.00679775527912577\\
367	0.00679472431567512\\
368	0.00679163918952407\\
369	0.0067884987700497\\
370	0.00678530188365528\\
371	0.00678204731019959\\
372	0.00677873377898507\\
373	0.00677535996423866\\
374	0.00677192448000755\\
375	0.00676842587437858\\
376	0.00676486262291356\\
377	0.00676123312117293\\
378	0.00675753567617702\\
379	0.00675376849662485\\
380	0.00674992968165619\\
381	0.00674601720790075\\
382	0.00674202891450603\\
383	0.00673796248577412\\
384	0.00673381543095961\\
385	0.00672958506068687\\
386	0.00672526845932825\\
387	0.00672086245254055\\
388	0.00671636356897764\\
389	0.00671176799497491\\
390	0.00670707152072197\\
391	0.00670226947609188\\
392	0.00669735665385637\\
393	0.00669232721746447\\
394	0.00668717458986346\\
395	0.00668189131895802\\
396	0.00667646891418213\\
397	0.00667089764723925\\
398	0.00666516630826544\\
399	0.00665926190639548\\
400	0.0066531693003214\\
401	0.00664687074063363\\
402	0.0066403453009818\\
403	0.00663356816837005\\
404	0.0066265097543923\\
405	0.00661913457814618\\
406	0.00661139985717842\\
407	0.0066032537242131\\
408	0.00659463296376019\\
409	0.00658546013408615\\
410	0.00657563991066749\\
411	0.00656505447699311\\
412	0.00655355787425215\\
413	0.00654096968874532\\
414	0.00652707034884023\\
415	0.00651253800668734\\
416	0.00649776665598494\\
417	0.00648275307261863\\
418	0.00646749408441272\\
419	0.00645198673070179\\
420	0.00643622880474178\\
421	0.00642021853199766\\
422	0.00640395008869782\\
423	0.00638741319449528\\
424	0.00637058922939843\\
425	0.00635347363103091\\
426	0.00633606181146737\\
427	0.00631834916708622\\
428	0.00630033109070742\\
429	0.00628200298649446\\
430	0.00626336028819253\\
431	0.00624439848135732\\
432	0.00622511313032493\\
433	0.00620549991122188\\
434	0.00618555465485072\\
435	0.00616527340457467\\
436	0.00614465248260606\\
437	0.0061236885719669\\
438	0.00610237881792449\\
439	0.00608072095368021\\
440	0.00605871345632733\\
441	0.00603635574067196\\
442	0.00601364840052693\\
443	0.00599059350966812\\
444	0.00596719499795318\\
445	0.00594345912235241\\
446	0.00591939505810106\\
447	0.00589501564216136\\
448	0.00587033830998165\\
449	0.00584538627718953\\
450	0.00582019002928812\\
451	0.00579478918987554\\
452	0.00576923495068925\\
453	0.00574359335856268\\
454	0.00571794955022763\\
455	0.00569241318540358\\
456	0.00566712443978084\\
457	0.00564226246104975\\
458	0.0056180562603175\\
459	0.00559479867444079\\
460	0.00557286390687526\\
461	0.00555272801045121\\
462	0.00553498629122868\\
463	0.00551720217170495\\
464	0.00549917695364716\\
465	0.00548091559129267\\
466	0.00546242419120114\\
467	0.00544371014961826\\
468	0.00542478230167255\\
469	0.00540565108201909\\
470	0.00538632869631064\\
471	0.0053668293016931\\
472	0.00534716919278362\\
473	0.00532736697530311\\
474	0.00530744371587823\\
475	0.00528742303786596\\
476	0.0052673311522017\\
477	0.00524719680490251\\
478	0.00522705109420467\\
479	0.00520692708006242\\
480	0.00518685909330759\\
481	0.00516688161543541\\
482	0.00514702755245753\\
483	0.0051273256592821\\
484	0.00510779677689485\\
485	0.00508844841117742\\
486	0.00506926698681219\\
487	0.0050502068219656\\
488	0.00503117450541807\\
489	0.00501200737501212\\
490	0.00499249585682076\\
491	0.0049726313321013\\
492	0.00495240419721947\\
493	0.00493180366065873\\
494	0.0049108175429671\\
495	0.00488943200399819\\
496	0.00486763121782343\\
497	0.00484539700192792\\
498	0.00482270836230811\\
499	0.00479954091595925\\
500	0.00477586626949021\\
501	0.00475165146994277\\
502	0.00472685928849462\\
503	0.00470144780663686\\
504	0.0046753685380568\\
505	0.0046485628668029\\
506	0.00462093305330937\\
507	0.00459240321805095\\
508	0.00456290132003495\\
509	0.00453235845413376\\
510	0.00450071538220435\\
511	0.00446790941452859\\
512	0.0044338754515804\\
513	0.0043985447378437\\
514	0.00436184495043661\\
515	0.00432370608462245\\
516	0.00428411145589852\\
517	0.00424298401153894\\
518	0.00420024412816288\\
519	0.0041558101897725\\
520	0.00410959934070177\\
521	0.00406152847336208\\
522	0.00401151552825213\\
523	0.00395948108279468\\
524	0.00390534995414889\\
525	0.00384905322038492\\
526	0.00379048589670331\\
527	0.00372956493563506\\
528	0.00366626174964019\\
529	0.00360056683240661\\
530	0.00353249550467206\\
531	0.00346209544460541\\
532	0.00338945724273818\\
533	0.00331472351787948\\
534	0.00323809784148983\\
535	0.00315984050805826\\
536	0.00308025744414865\\
537	0.00299980871022\\
538	0.00291922833911188\\
539	0.00284020219425925\\
540	0.00276638572725419\\
541	0.00269849951041931\\
542	0.00263684499586817\\
543	0.00258138978366927\\
544	0.00252967922050828\\
545	0.00247978042691092\\
546	0.00243112704399096\\
547	0.00238300674475131\\
548	0.00233484331452528\\
549	0.00228638251316667\\
550	0.00223746986060232\\
551	0.00218795669036668\\
552	0.00213773350870465\\
553	0.00208674133385803\\
554	0.00203493627134918\\
555	0.00198227603480573\\
556	0.00192867604033419\\
557	0.0018740398220566\\
558	0.00182018557105755\\
559	0.00176745105389124\\
560	0.00171613247954439\\
561	0.00166521021022062\\
562	0.00161402343788605\\
563	0.0015625390490746\\
564	0.00151070458321067\\
565	0.0014585623292187\\
566	0.00140616386370591\\
567	0.00135357151224325\\
568	0.0013008466527655\\
569	0.0012498155052468\\
570	0.00120064029990444\\
571	0.00115186449819362\\
572	0.00110318472505531\\
573	0.00105458725117355\\
574	0.00100611405883502\\
575	0.000957852788140562\\
576	0.000910291753390808\\
577	0.000863157672941971\\
578	0.000816161492815582\\
579	0.000769336847989938\\
580	0.000722734999820213\\
581	0.000676409328092024\\
582	0.000630413733840029\\
583	0.000584801521858841\\
584	0.000539624056582973\\
585	0.000494929053821303\\
586	0.000450758417500461\\
587	0.000407145545582638\\
588	0.000364112075964822\\
589	0.000321664201822913\\
590	0.00027978915728462\\
591	0.000238610660212784\\
592	0.000198280712249117\\
593	0.000158978649805852\\
594	0.000120944048332157\\
595	8.4566189416569e-05\\
596	5.06250627329889e-05\\
597	2.09371357655836e-05\\
598	8.08076597025309e-08\\
599	0\\
600	0\\
};
\addplot [color=red!80!mycolor19,solid,forget plot]
  table[row sep=crcr]{%
1	0.00666972821198073\\
2	0.00666971752457823\\
3	0.00666970664534451\\
4	0.00666969557083904\\
5	0.0066696842975596\\
6	0.00666967282194128\\
7	0.00666966114035521\\
8	0.0066696492491076\\
9	0.00666963714443846\\
10	0.00666962482252039\\
11	0.00666961227945744\\
12	0.00666959951128392\\
13	0.00666958651396308\\
14	0.0066695732833858\\
15	0.00666955981536939\\
16	0.00666954610565626\\
17	0.00666953214991251\\
18	0.00666951794372665\\
19	0.00666950348260817\\
20	0.00666948876198612\\
21	0.0066694737772077\\
22	0.00666945852353674\\
23	0.00666944299615232\\
24	0.00666942719014711\\
25	0.00666941110052595\\
26	0.00666939472220418\\
27	0.00666937805000613\\
28	0.00666936107866343\\
29	0.00666934380281338\\
30	0.00666932621699726\\
31	0.00666930831565863\\
32	0.00666929009314151\\
33	0.00666927154368869\\
34	0.00666925266143987\\
35	0.00666923344042986\\
36	0.00666921387458666\\
37	0.00666919395772956\\
38	0.00666917368356723\\
39	0.00666915304569571\\
40	0.00666913203759641\\
41	0.00666911065263403\\
42	0.00666908888405454\\
43	0.00666906672498303\\
44	0.00666904416842152\\
45	0.00666902120724679\\
46	0.00666899783420813\\
47	0.0066689740419251\\
48	0.00666894982288515\\
49	0.00666892516944133\\
50	0.00666890007380982\\
51	0.00666887452806751\\
52	0.0066688485241496\\
53	0.00666882205384692\\
54	0.00666879510880342\\
55	0.00666876768051358\\
56	0.00666873976031976\\
57	0.00666871133940934\\
58	0.00666868240881214\\
59	0.00666865295939747\\
60	0.00666862298187134\\
61	0.00666859246677352\\
62	0.00666856140447456\\
63	0.00666852978517278\\
64	0.00666849759889122\\
65	0.00666846483547447\\
66	0.00666843148458556\\
67	0.00666839753570262\\
68	0.00666836297811567\\
69	0.00666832780092321\\
70	0.00666829199302883\\
71	0.00666825554313769\\
72	0.00666821843975312\\
73	0.00666818067117279\\
74	0.00666814222548525\\
75	0.0066681030905661\\
76	0.00666806325407426\\
77	0.00666802270344801\\
78	0.00666798142590112\\
79	0.00666793940841889\\
80	0.00666789663775392\\
81	0.00666785310042224\\
82	0.00666780878269877\\
83	0.00666776367061329\\
84	0.0066677177499459\\
85	0.00666767100622268\\
86	0.00666762342471105\\
87	0.00666757499041536\\
88	0.00666752568807196\\
89	0.00666747550214461\\
90	0.00666742441681959\\
91	0.00666737241600066\\
92	0.00666731948330418\\
93	0.00666726560205389\\
94	0.00666721075527572\\
95	0.00666715492569251\\
96	0.00666709809571865\\
97	0.00666704024745451\\
98	0.00666698136268091\\
99	0.00666692142285343\\
100	0.00666686040909663\\
101	0.00666679830219809\\
102	0.00666673508260257\\
103	0.00666667073040575\\
104	0.00666660522534812\\
105	0.0066665385468087\\
106	0.00666647067379847\\
107	0.00666640158495397\\
108	0.00666633125853061\\
109	0.00666625967239588\\
110	0.00666618680402243\\
111	0.00666611263048113\\
112	0.00666603712843389\\
113	0.00666596027412643\\
114	0.00666588204338081\\
115	0.00666580241158805\\
116	0.00666572135370032\\
117	0.00666563884422324\\
118	0.00666555485720799\\
119	0.0066654693662432\\
120	0.00666538234444671\\
121	0.00666529376445732\\
122	0.00666520359842615\\
123	0.00666511181800817\\
124	0.00666501839435328\\
125	0.00666492329809738\\
126	0.00666482649935327\\
127	0.00666472796770146\\
128	0.00666462767218056\\
129	0.0066645255812779\\
130	0.00666442166291958\\
131	0.00666431588446068\\
132	0.006664208212675\\
133	0.00666409861374497\\
134	0.00666398705325098\\
135	0.0066638734961609\\
136	0.00666375790681913\\
137	0.00666364024893562\\
138	0.00666352048557466\\
139	0.00666339857914343\\
140	0.00666327449138045\\
141	0.00666314818334371\\
142	0.00666301961539867\\
143	0.00666288874720607\\
144	0.00666275553770937\\
145	0.00666261994512222\\
146	0.00666248192691548\\
147	0.00666234143980424\\
148	0.0066621984397343\\
149	0.00666205288186877\\
150	0.00666190472057418\\
151	0.0066617539094065\\
152	0.00666160040109679\\
153	0.00666144414753671\\
154	0.00666128509976372\\
155	0.00666112320794606\\
156	0.00666095842136742\\
157	0.00666079068841139\\
158	0.00666061995654559\\
159	0.00666044617230563\\
160	0.00666026928127865\\
161	0.0066600892280867\\
162	0.00665990595636971\\
163	0.00665971940876835\\
164	0.00665952952690633\\
165	0.00665933625137271\\
166	0.00665913952170362\\
167	0.00665893927636378\\
168	0.00665873545272782\\
169	0.00665852798706106\\
170	0.00665831681450009\\
171	0.00665810186903307\\
172	0.00665788308347944\\
173	0.00665766038946962\\
174	0.00665743371742414\\
175	0.00665720299653249\\
176	0.00665696815473151\\
177	0.00665672911868356\\
178	0.00665648581375428\\
179	0.00665623816398981\\
180	0.00665598609209385\\
181	0.00665572951940424\\
182	0.00665546836586898\\
183	0.00665520255002218\\
184	0.00665493198895925\\
185	0.00665465659831192\\
186	0.00665437629222267\\
187	0.0066540909833189\\
188	0.00665380058268646\\
189	0.00665350499984282\\
190	0.00665320414270994\\
191	0.00665289791758632\\
192	0.00665258622911901\\
193	0.00665226898027479\\
194	0.00665194607231105\\
195	0.00665161740474612\\
196	0.00665128287532914\\
197	0.00665094238000936\\
198	0.00665059581290501\\
199	0.00665024306627157\\
200	0.00664988403046948\\
201	0.00664951859393149\\
202	0.00664914664312929\\
203	0.0066487680625395\\
204	0.00664838273460942\\
205	0.00664799053972189\\
206	0.00664759135615971\\
207	0.00664718506006942\\
208	0.00664677152542456\\
209	0.00664635062398818\\
210	0.00664592222527488\\
211	0.00664548619651211\\
212	0.00664504240260084\\
213	0.00664459070607565\\
214	0.00664413096706405\\
215	0.00664366304324527\\
216	0.00664318678980824\\
217	0.00664270205940901\\
218	0.00664220870212727\\
219	0.00664170656542238\\
220	0.00664119549408852\\
221	0.0066406753302092\\
222	0.00664014591311096\\
223	0.00663960707931637\\
224	0.00663905866249624\\
225	0.00663850049342098\\
226	0.00663793239991136\\
227	0.00663735420678821\\
228	0.00663676573582157\\
229	0.00663616680567883\\
230	0.00663555723187206\\
231	0.00663493682670458\\
232	0.00663430539921657\\
233	0.00663366275512987\\
234	0.00663300869679188\\
235	0.00663234302311854\\
236	0.00663166552953636\\
237	0.00663097600792367\\
238	0.00663027424655076\\
239	0.00662956003001915\\
240	0.00662883313919992\\
241	0.00662809335117101\\
242	0.00662734043915356\\
243	0.00662657417244724\\
244	0.00662579431636453\\
245	0.00662500063216402\\
246	0.00662419287698259\\
247	0.00662337080376664\\
248	0.00662253416120211\\
249	0.00662168269364355\\
250	0.00662081614104197\\
251	0.00661993423887163\\
252	0.0066190367180557\\
253	0.0066181233048907\\
254	0.00661719372096993\\
255	0.0066162476831056\\
256	0.00661528490324973\\
257	0.00661430508841406\\
258	0.00661330794058847\\
259	0.00661229315665839\\
260	0.00661126042832076\\
261	0.006610209441999\\
262	0.00660913987875634\\
263	0.00660805141420824\\
264	0.00660694371843321\\
265	0.00660581645588247\\
266	0.00660466928528825\\
267	0.00660350185957064\\
268	0.00660231382574331\\
269	0.00660110482481763\\
270	0.00659987449170545\\
271	0.00659862245512063\\
272	0.00659734833747898\\
273	0.0065960517547969\\
274	0.00659473231658857\\
275	0.0065933896257616\\
276	0.00659202327851136\\
277	0.00659063286421374\\
278	0.00658921796531662\\
279	0.00658777815722956\\
280	0.00658631300821239\\
281	0.00658482207926192\\
282	0.00658330492399752\\
283	0.00658176108854492\\
284	0.00658019011141862\\
285	0.00657859152340293\\
286	0.00657696484743122\\
287	0.00657530959846402\\
288	0.00657362528336532\\
289	0.00657191140077759\\
290	0.00657016744099529\\
291	0.00656839288583682\\
292	0.00656658720851515\\
293	0.00656474987350694\\
294	0.00656288033642023\\
295	0.00656097804386091\\
296	0.00655904243329753\\
297	0.00655707293292512\\
298	0.00655506896152745\\
299	0.00655302992833828\\
300	0.00655095523290118\\
301	0.00654884426492836\\
302	0.0065466964041584\\
303	0.00654451102021293\\
304	0.00654228747245242\\
305	0.00654002510983112\\
306	0.00653772327075134\\
307	0.006535381282917\\
308	0.00653299846318679\\
309	0.00653057411742701\\
310	0.00652810754036426\\
311	0.00652559801543821\\
312	0.00652304481465462\\
313	0.00652044719843905\\
314	0.00651780441549135\\
315	0.00651511570264146\\
316	0.00651238028470695\\
317	0.00650959737435263\\
318	0.00650676617195312\\
319	0.00650388586545891\\
320	0.00650095563026665\\
321	0.00649797462909506\\
322	0.00649494201186731\\
323	0.00649185691560162\\
324	0.00648871846431166\\
325	0.00648552576891933\\
326	0.00648227792718231\\
327	0.00647897402363986\\
328	0.00647561312958111\\
329	0.00647219430304098\\
330	0.00646871658883027\\
331	0.006465179018608\\
332	0.0064615806110065\\
333	0.0064579203718221\\
334	0.00645419729428919\\
335	0.00645041035946043\\
336	0.00644655853672792\\
337	0.0064426407845435\\
338	0.00643865605145956\\
339	0.00643460327779231\\
340	0.00643048139877115\\
341	0.00642628935182844\\
342	0.00642202609651451\\
343	0.00641769067474251\\
344	0.00641328240307497\\
345	0.00640880095069646\\
346	0.00640424527244084\\
347	0.00639961431543911\\
348	0.00639490701681601\\
349	0.00639012229936675\\
350	0.00638525906310528\\
351	0.00638031616739028\\
352	0.00637529238961384\\
353	0.00637018632215671\\
354	0.00636499610093067\\
355	0.00635971866436684\\
356	0.00635434768363513\\
357	0.00634888117272256\\
358	0.00634331745695464\\
359	0.00633765483284434\\
360	0.00633189156769182\\
361	0.00632602589920504\\
362	0.00632005603514739\\
363	0.00631398015302021\\
364	0.00630779639978955\\
365	0.00630150289166846\\
366	0.0062950977139685\\
367	0.00628857892103656\\
368	0.00628194453629712\\
369	0.0062751925524233\\
370	0.00626832093166558\\
371	0.0062613276063726\\
372	0.00625421047974569\\
373	0.00624696742687743\\
374	0.00623959629613504\\
375	0.00623209491096238\\
376	0.00622446107218952\\
377	0.00621669256095859\\
378	0.00620878714239739\\
379	0.00620074257020141\\
380	0.00619255659231993\\
381	0.0061842269579856\\
382	0.00617575142638087\\
383	0.00616712777730069\\
384	0.00615835382425461\\
385	0.00614942743055357\\
386	0.00614034652905609\\
387	0.00613110914640862\\
388	0.00612171343281731\\
389	0.00611215769864154\\
390	0.00610244045941964\\
391	0.00609256049133989\\
392	0.00608251689968037\\
393	0.00607230920338357\\
394	0.00606193743973921\\
395	0.00605140229414253\\
396	0.00604070526106553\\
397	0.00602984884357117\\
398	0.00601883679931619\\
399	0.00600767445700477\\
400	0.00599636911681533\\
401	0.00598493054757132\\
402	0.00597337161424599\\
403	0.00596170907212616\\
404	0.00594996457465894\\
405	0.0059381659560048\\
406	0.00592634886767835\\
407	0.00591455887274831\\
408	0.00590285413258532\\
409	0.00589130886169145\\
410	0.00588001777509125\\
411	0.00586910179785866\\
412	0.0058587152261411\\
413	0.00584905413856412\\
414	0.00584036411474068\\
415	0.00583194926436135\\
416	0.00582339773760308\\
417	0.00581470788444782\\
418	0.00580587806840159\\
419	0.00579690666528331\\
420	0.00578779204812145\\
421	0.00577853258267673\\
422	0.00576912675068263\\
423	0.00575957329364853\\
424	0.00574987147130422\\
425	0.00574002070496799\\
426	0.00573002059849492\\
427	0.00571987096163404\\
428	0.00570957183607207\\
429	0.00569912352447074\\
430	0.00568852662283488\\
431	0.00567778205658217\\
432	0.00566689112072931\\
433	0.00565585552464597\\
434	0.00564467744172826\\
435	0.00563335956413101\\
436	0.00562190516314289\\
437	0.00561031815553495\\
438	0.00559860317611937\\
439	0.00558676565660214\\
440	0.00557481191057514\\
441	0.00556274922413744\\
442	0.00555058595111456\\
443	0.00553833161109627\\
444	0.00552599698745284\\
445	0.00551359422099758\\
446	0.00550113689288508\\
447	0.00548864008745256\\
448	0.00547612042174682\\
449	0.00546359602307482\\
450	0.0054510864287009\\
451	0.00543861237243524\\
452	0.00542619540610162\\
453	0.00541385727800935\\
454	0.00540161896716163\\
455	0.00538949923483888\\
456	0.00537751254666102\\
457	0.00536566608822791\\
458	0.00535395551461079\\
459	0.00534235891825005\\
460	0.00533082828227063\\
461	0.00531927753688528\\
462	0.00530756731065481\\
463	0.00529568599891958\\
464	0.00528363280749588\\
465	0.00527140695465641\\
466	0.00525900764669507\\
467	0.00524643404459406\\
468	0.00523368522121634\\
469	0.00522076010713054\\
470	0.00520765742281147\\
471	0.00519437559460763\\
472	0.00518091265149298\\
473	0.00516726609976775\\
474	0.00515343277274591\\
475	0.00513940865317913\\
476	0.00512518866618245\\
477	0.00511076644067278\\
478	0.0050961340386193\\
479	0.00508128165445553\\
480	0.00506619729337088\\
481	0.00505086646952567\\
482	0.00503527190948447\\
483	0.00501939329339911\\
484	0.00500320711134743\\
485	0.00498668672962048\\
486	0.00496980281267815\\
487	0.00495252433352919\\
488	0.00493482057094311\\
489	0.00491666476926258\\
490	0.00489803768405018\\
491	0.00487891878736016\\
492	0.00485928615801974\\
493	0.00483911628937856\\
494	0.00481838259129712\\
495	0.00479705622994182\\
496	0.00477510616775985\\
497	0.00475249892767851\\
498	0.00472919929196462\\
499	0.00470517023887837\\
500	0.00468037049899494\\
501	0.00465474903653642\\
502	0.00462823043492447\\
503	0.00460076384820823\\
504	0.00457230303262053\\
505	0.00454284824132496\\
506	0.00451234731777022\\
507	0.00448074623469078\\
508	0.00444798897682091\\
509	0.00441401737437152\\
510	0.00437877067473109\\
511	0.00434218570442675\\
512	0.00430419687681335\\
513	0.00426473643300687\\
514	0.00422373499809923\\
515	0.00418111653559961\\
516	0.00413674413018385\\
517	0.00409053955993568\\
518	0.00404242533994876\\
519	0.00399232614603507\\
520	0.00394017062742537\\
521	0.00388589336553243\\
522	0.0038294374231203\\
523	0.00377075823624427\\
524	0.00370983393349168\\
525	0.00364667012592823\\
526	0.00358130207699017\\
527	0.00351380341503911\\
528	0.00344429536009399\\
529	0.00337295574855904\\
530	0.00330002036848931\\
531	0.00322578583316163\\
532	0.00315060254208725\\
533	0.00307503431258532\\
534	0.00300003941270825\\
535	0.0029291061898873\\
536	0.00286396354146385\\
537	0.00280497189454524\\
538	0.0027522244283108\\
539	0.0027047286641074\\
540	0.00265895622762533\\
541	0.00261425676109811\\
542	0.00257014214015897\\
543	0.00252605117422348\\
544	0.00248164053493229\\
545	0.00243675242361031\\
546	0.00239124966818091\\
547	0.00234503461969202\\
548	0.00229804948701554\\
549	0.00225025048581161\\
550	0.00220160078032583\\
551	0.00215207097100767\\
552	0.00210163079760497\\
553	0.00205020354468418\\
554	0.00199789065114578\\
555	0.00194648877624312\\
556	0.00189623454175593\\
557	0.00184743320422562\\
558	0.00179840226650776\\
559	0.00174900462157863\\
560	0.00169917589091746\\
561	0.00164890427183727\\
562	0.00159822238307425\\
563	0.00154717107678896\\
564	0.0014958031086108\\
565	0.00144417947473731\\
566	0.00139238321872931\\
567	0.00134225497904084\\
568	0.00129392869044432\\
569	0.00124564689870853\\
570	0.00119733364158835\\
571	0.00114900865986872\\
572	0.00110071416638326\\
573	0.00105249521053684\\
574	0.00100468219677716\\
575	0.000957445377304777\\
576	0.000910262582761565\\
577	0.000863151073240643\\
578	0.000816158324218955\\
579	0.000769335142801099\\
580	0.000722734095065969\\
581	0.000676408872065992\\
582	0.000630413520144035\\
583	0.000584801430598912\\
584	0.000539624021957423\\
585	0.000494929042580466\\
586	0.000450758414560671\\
587	0.000407145545029566\\
588	0.000364112075908287\\
589	0.000321664201822906\\
590	0.000279789157284612\\
591	0.000238610660212779\\
592	0.000198280712249112\\
593	0.000158978649805848\\
594	0.000120944048332154\\
595	8.45661894165686e-05\\
596	5.06250627329881e-05\\
597	2.09371357655843e-05\\
598	8.08076597025309e-08\\
599	0\\
600	0\\
};
\addplot [color=red,solid,forget plot]
  table[row sep=crcr]{%
1	0.00606327905803858\\
2	0.00606327282503031\\
3	0.00606326647983929\\
4	0.00606326002044673\\
5	0.00606325344479749\\
6	0.00606324675079944\\
7	0.00606323993632287\\
8	0.00606323299919967\\
9	0.00606322593722277\\
10	0.00606321874814536\\
11	0.00606321142968024\\
12	0.00606320397949901\\
13	0.00606319639523139\\
14	0.00606318867446449\\
15	0.00606318081474194\\
16	0.00606317281356325\\
17	0.00606316466838288\\
18	0.00606315637660951\\
19	0.0060631479356052\\
20	0.00606313934268454\\
21	0.00606313059511381\\
22	0.00606312169011013\\
23	0.00606311262484047\\
24	0.00606310339642094\\
25	0.00606309400191564\\
26	0.00606308443833598\\
27	0.00606307470263949\\
28	0.00606306479172901\\
29	0.00606305470245166\\
30	0.00606304443159784\\
31	0.00606303397590016\\
32	0.00606302333203247\\
33	0.00606301249660883\\
34	0.00606300146618231\\
35	0.006062990237244\\
36	0.00606297880622187\\
37	0.00606296716947963\\
38	0.00606295532331552\\
39	0.00606294326396124\\
40	0.00606293098758068\\
41	0.00606291849026869\\
42	0.00606290576804991\\
43	0.00606289281687743\\
44	0.00606287963263156\\
45	0.0060628662111185\\
46	0.00606285254806894\\
47	0.00606283863913685\\
48	0.00606282447989799\\
49	0.00606281006584855\\
50	0.00606279539240365\\
51	0.00606278045489598\\
52	0.00606276524857422\\
53	0.00606274976860165\\
54	0.00606273401005449\\
55	0.00606271796792042\\
56	0.00606270163709691\\
57	0.00606268501238973\\
58	0.00606266808851112\\
59	0.0060626508600783\\
60	0.0060626333216116\\
61	0.00606261546753281\\
62	0.00606259729216337\\
63	0.00606257878972259\\
64	0.00606255995432585\\
65	0.00606254077998262\\
66	0.00606252126059466\\
67	0.00606250138995403\\
68	0.00606248116174116\\
69	0.00606246056952277\\
70	0.00606243960674991\\
71	0.00606241826675587\\
72	0.006062396542754\\
73	0.00606237442783561\\
74	0.00606235191496778\\
75	0.00606232899699109\\
76	0.00606230566661739\\
77	0.00606228191642743\\
78	0.00606225773886856\\
79	0.00606223312625232\\
80	0.006062208070752\\
81	0.00606218256440008\\
82	0.0060621565990859\\
83	0.00606213016655282\\
84	0.00606210325839582\\
85	0.00606207586605873\\
86	0.00606204798083159\\
87	0.0060620195938477\\
88	0.0060619906960811\\
89	0.00606196127834344\\
90	0.00606193133128121\\
91	0.00606190084537276\\
92	0.00606186981092522\\
93	0.00606183821807152\\
94	0.00606180605676719\\
95	0.00606177331678714\\
96	0.00606173998772255\\
97	0.00606170605897746\\
98	0.00606167151976546\\
99	0.00606163635910625\\
100	0.00606160056582213\\
101	0.00606156412853465\\
102	0.00606152703566063\\
103	0.0060614892754089\\
104	0.00606145083577631\\
105	0.006061411704544\\
106	0.00606137186927353\\
107	0.00606133131730293\\
108	0.00606129003574277\\
109	0.00606124801147195\\
110	0.00606120523113364\\
111	0.00606116168113102\\
112	0.00606111734762299\\
113	0.00606107221651976\\
114	0.00606102627347844\\
115	0.0060609795038985\\
116	0.00606093189291711\\
117	0.0060608834254045\\
118	0.00606083408595909\\
119	0.00606078385890274\\
120	0.00606073272827568\\
121	0.00606068067783152\\
122	0.00606062769103217\\
123	0.0060605737510425\\
124	0.00606051884072513\\
125	0.00606046294263493\\
126	0.00606040603901365\\
127	0.0060603481117841\\
128	0.00606028914254469\\
129	0.00606022911256346\\
130	0.00606016800277218\\
131	0.00606010579376041\\
132	0.00606004246576937\\
133	0.00605997799868561\\
134	0.00605991237203482\\
135	0.00605984556497528\\
136	0.00605977755629133\\
137	0.00605970832438669\\
138	0.00605963784727761\\
139	0.00605956610258603\\
140	0.00605949306753253\\
141	0.00605941871892909\\
142	0.00605934303317189\\
143	0.00605926598623379\\
144	0.00605918755365682\\
145	0.00605910771054463\\
146	0.00605902643155444\\
147	0.00605894369088922\\
148	0.00605885946228958\\
149	0.00605877371902547\\
150	0.00605868643388793\\
151	0.00605859757918034\\
152	0.00605850712670997\\
153	0.00605841504777901\\
154	0.0060583213131756\\
155	0.0060582258931647\\
156	0.00605812875747879\\
157	0.00605802987530837\\
158	0.00605792921529236\\
159	0.00605782674550833\\
160	0.00605772243346237\\
161	0.00605761624607913\\
162	0.0060575081496914\\
163	0.00605739811002957\\
164	0.00605728609221107\\
165	0.00605717206072935\\
166	0.00605705597944289\\
167	0.00605693781156396\\
168	0.00605681751964714\\
169	0.0060566950655777\\
170	0.00605657041055966\\
171	0.00605644351510378\\
172	0.00605631433901546\\
173	0.00605618284138198\\
174	0.00605604898056002\\
175	0.00605591271416262\\
176	0.00605577399904616\\
177	0.00605563279129688\\
178	0.00605548904621744\\
179	0.00605534271831294\\
180	0.00605519376127698\\
181	0.00605504212797727\\
182	0.00605488777044124\\
183	0.00605473063984098\\
184	0.00605457068647849\\
185	0.0060544078597702\\
186	0.00605424210823147\\
187	0.00605407337946069\\
188	0.0060539016201233\\
189	0.00605372677593539\\
190	0.00605354879164693\\
191	0.00605336761102505\\
192	0.00605318317683669\\
193	0.00605299543083109\\
194	0.00605280431372206\\
195	0.00605260976516989\\
196	0.00605241172376291\\
197	0.00605221012699874\\
198	0.00605200491126539\\
199	0.0060517960118218\\
200	0.0060515833627783\\
201	0.00605136689707653\\
202	0.00605114654646913\\
203	0.00605092224149911\\
204	0.00605069391147889\\
205	0.00605046148446886\\
206	0.0060502248872558\\
207	0.00604998404533074\\
208	0.00604973888286658\\
209	0.00604948932269533\\
210	0.00604923528628491\\
211	0.00604897669371562\\
212	0.00604871346365631\\
213	0.00604844551333998\\
214	0.00604817275853918\\
215	0.00604789511354083\\
216	0.00604761249112078\\
217	0.00604732480251784\\
218	0.00604703195740759\\
219	0.00604673386387548\\
220	0.00604643042838979\\
221	0.00604612155577395\\
222	0.00604580714917856\\
223	0.00604548711005287\\
224	0.00604516133811584\\
225	0.00604482973132687\\
226	0.00604449218585578\\
227	0.00604414859605262\\
228	0.00604379885441692\\
229	0.00604344285156634\\
230	0.00604308047620496\\
231	0.00604271161509111\\
232	0.00604233615300456\\
233	0.00604195397271341\\
234	0.00604156495494022\\
235	0.00604116897832787\\
236	0.00604076591940484\\
237	0.00604035565254979\\
238	0.00603993804995586\\
239	0.00603951298159437\\
240	0.0060390803151778\\
241	0.00603863991612243\\
242	0.00603819164751041\\
243	0.00603773537005115\\
244	0.00603727094204229\\
245	0.00603679821932996\\
246	0.0060363170552686\\
247	0.00603582730068017\\
248	0.00603532880381267\\
249	0.00603482141029823\\
250	0.00603430496311047\\
251	0.00603377930252137\\
252	0.00603324426605742\\
253	0.00603269968845541\\
254	0.00603214540161721\\
255	0.00603158123456428\\
256	0.00603100701339141\\
257	0.00603042256121988\\
258	0.00602982769814995\\
259	0.00602922224121269\\
260	0.00602860600432137\\
261	0.00602797879822192\\
262	0.00602734043044299\\
263	0.00602669070524521\\
264	0.00602602942356993\\
265	0.0060253563829872\\
266	0.00602467137764317\\
267	0.00602397419820689\\
268	0.00602326463181621\\
269	0.00602254246202341\\
270	0.00602180746873986\\
271	0.0060210594281802\\
272	0.0060202981128058\\
273	0.00601952329126758\\
274	0.00601873472834826\\
275	0.00601793218490381\\
276	0.00601711541780445\\
277	0.00601628417987499\\
278	0.00601543821983428\\
279	0.00601457728223448\\
280	0.00601370110739935\\
281	0.00601280943136216\\
282	0.00601190198580288\\
283	0.0060109784979849\\
284	0.00601003869069112\\
285	0.00600908228215942\\
286	0.00600810898601791\\
287	0.00600711851121913\\
288	0.00600611056197431\\
289	0.00600508483768683\\
290	0.00600404103288531\\
291	0.00600297883715626\\
292	0.00600189793507647\\
293	0.0060007980061448\\
294	0.00599967872471395\\
295	0.00599853975992155\\
296	0.0059973807756215\\
297	0.00599620143031453\\
298	0.00599500137707915\\
299	0.00599378026350202\\
300	0.00599253773160868\\
301	0.00599127341779386\\
302	0.00598998695275223\\
303	0.00598867796140897\\
304	0.00598734606285064\\
305	0.00598599087025636\\
306	0.00598461199082915\\
307	0.00598320902572778\\
308	0.00598178156999902\\
309	0.00598032921251041\\
310	0.00597885153588374\\
311	0.005977348116429\\
312	0.00597581852407932\\
313	0.00597426232232657\\
314	0.00597267906815786\\
315	0.00597106831199303\\
316	0.00596942959762297\\
317	0.00596776246214913\\
318	0.00596606643592385\\
319	0.00596434104249168\\
320	0.00596258579853173\\
321	0.00596080021380062\\
322	0.00595898379107628\\
323	0.00595713602610194\\
324	0.00595525640753041\\
325	0.00595334441686779\\
326	0.00595139952841611\\
327	0.00594942120921433\\
328	0.00594740891897617\\
329	0.00594536211002369\\
330	0.00594328022721434\\
331	0.00594116270785922\\
332	0.00593900898162894\\
333	0.00593681847044298\\
334	0.00593459058833621\\
335	0.00593232474129501\\
336	0.00593002032705249\\
337	0.0059276767348286\\
338	0.00592529334499448\\
339	0.00592286952862953\\
340	0.00592040464691092\\
341	0.00591789805019575\\
342	0.00591534907640605\\
343	0.00591275704752625\\
344	0.00591012126046742\\
345	0.00590744097543618\\
346	0.00590471543358358\\
347	0.00590194385580295\\
348	0.00589912544156096\\
349	0.00589625936785703\\
350	0.00589334478848432\\
351	0.00589038083393682\\
352	0.00588736661273433\\
353	0.00588430121607885\\
354	0.00588118373089446\\
355	0.00587801327495084\\
356	0.00587478909096314\\
357	0.00587151043120173\\
358	0.00586817654951324\\
359	0.00586478670255204\\
360	0.00586134015114754\\
361	0.00585783616182132\\
362	0.00585427400847067\\
363	0.00585065297423628\\
364	0.00584697235357417\\
365	0.00584323145455429\\
366	0.00583942960141064\\
367	0.00583556613737083\\
368	0.00583164042779568\\
369	0.0058276518636641\\
370	0.005823599865441\\
371	0.00581948388737174\\
372	0.00581530342225071\\
373	0.00581105800671722\\
374	0.00580674722713838\\
375	0.00580237072614405\\
376	0.00579792820988772\\
377	0.00579341945611311\\
378	0.00578884432311665\\
379	0.00578420275970296\\
380	0.00577949481624141\\
381	0.00577472065694019\\
382	0.00576988057346412\\
383	0.00576497500003181\\
384	0.00576000453013423\\
385	0.00575496993502329\\
386	0.00574987218412\\
387	0.00574471246748822\\
388	0.00573949222050758\\
389	0.00573421315085504\\
390	0.00572887726786257\\
391	0.00572348691425219\\
392	0.00571804480014881\\
393	0.0057125540391237\\
394	0.00570701818580735\\
395	0.00570144127431131\\
396	0.00569582785628598\\
397	0.00569018303689547\\
398	0.00568451250632116\\
399	0.00567882256265524\\
400	0.00567312012059493\\
401	0.00566741269867856\\
402	0.00566170837455386\\
403	0.00565601569370654\\
404	0.00565034351153072\\
405	0.00564470074105716\\
406	0.00563909596831537\\
407	0.00563353688311485\\
408	0.0056280294533546\\
409	0.00562257674316511\\
410	0.00561717723472199\\
411	0.0056118224531518\\
412	0.00560649361144075\\
413	0.00560115692512977\\
414	0.00559575752949067\\
415	0.00559027328474995\\
416	0.00558470338171169\\
417	0.00557904704396517\\
418	0.00557330353265153\\
419	0.00556747215183497\\
420	0.00556155225500092\\
421	0.00555554325254316\\
422	0.00554944461636573\\
423	0.00554325588066259\\
424	0.00553697663466845\\
425	0.00553060652624587\\
426	0.00552414526538915\\
427	0.00551759262757341\\
428	0.0055109484568618\\
429	0.00550421266867575\\
430	0.00549738525211275\\
431	0.005490466271653\\
432	0.00548345586805899\\
433	0.0054763542582269\\
434	0.00546916173369887\\
435	0.00546187865749822\\
436	0.00545450545887131\\
437	0.00544704262543636\\
438	0.00543949069214535\\
439	0.00543185022635335\\
440	0.00542412180816669\\
441	0.00541630600510155\\
442	0.00540840333993562\\
443	0.00540041425047996\\
444	0.00539233903984555\\
445	0.00538417781564861\\
446	0.00537593041651388\\
447	0.0053675963242487\\
448	0.00535917456023116\\
449	0.00535066356495658\\
450	0.00534206105983422\\
451	0.00533336389267354\\
452	0.00532456787068219\\
453	0.00531566758839721\\
454	0.00530665626370843\\
455	0.00529752560367926\\
456	0.00528826573228859\\
457	0.00527886523037034\\
458	0.00526931136413408\\
459	0.00525959062182763\\
460	0.00524968975985358\\
461	0.00523959771663915\\
462	0.00522930889167615\\
463	0.00521881782911749\\
464	0.00520811873511819\\
465	0.00519720544743216\\
466	0.00518607140565486\\
467	0.0051747096579973\\
468	0.00516311283200748\\
469	0.0051512731022433\\
470	0.00513918215601689\\
471	0.0051268311573012\\
472	0.00511421070897352\\
473	0.0051013108136804\\
474	0.00508812083375645\\
475	0.00507462945079548\\
476	0.00506082462582009\\
477	0.00504669356149496\\
478	0.00503222266795953\\
479	0.00501739752866004\\
480	0.00500220282102467\\
481	0.00498662136535997\\
482	0.00497063443874574\\
483	0.00495422231351525\\
484	0.00493736429616554\\
485	0.0049200387793086\\
486	0.00490222329740413\\
487	0.0048838945684073\\
488	0.00486502848452711\\
489	0.00484559996011706\\
490	0.00482558257556014\\
491	0.00480494862373132\\
492	0.0047836693292968\\
493	0.00476176007246918\\
494	0.00473920168646042\\
495	0.00471596513454578\\
496	0.0046920191281953\\
497	0.00466732600038268\\
498	0.00464181864433913\\
499	0.00461545596194319\\
500	0.00458819573820931\\
501	0.00455999367910213\\
502	0.00453080411538099\\
503	0.00450057935505207\\
504	0.00446926265558328\\
505	0.00443674315717844\\
506	0.00440296011699823\\
507	0.00436785023178091\\
508	0.00433134775483851\\
509	0.00429338469667519\\
510	0.00425389114600167\\
511	0.00421279571052498\\
512	0.00417002610780525\\
513	0.00412550990689735\\
514	0.00407917531379895\\
515	0.00403095219799405\\
516	0.00398077828485345\\
517	0.00392859831444888\\
518	0.00387436698327408\\
519	0.003818052330662\\
520	0.00375964059924513\\
521	0.00369914807427347\\
522	0.00363662667824208\\
523	0.00357216670261621\\
524	0.00350589875478118\\
525	0.00343801206542371\\
526	0.0033687619954163\\
527	0.00329846349715342\\
528	0.00322752782255752\\
529	0.00315693561977478\\
530	0.00308771769538976\\
531	0.00302404388037806\\
532	0.00296635642735741\\
533	0.00291488621947875\\
534	0.00286945187213182\\
535	0.0028269308358288\\
536	0.00278559957777752\\
537	0.00274500529474445\\
538	0.00270461044732156\\
539	0.00266391377525266\\
540	0.00262276248512696\\
541	0.00258103182679686\\
542	0.00253862199275885\\
543	0.00249546998064093\\
544	0.00245153442464299\\
545	0.00240678026120397\\
546	0.00236117819802576\\
547	0.00231470362867669\\
548	0.0022673340259668\\
549	0.0022190462129072\\
550	0.00216976473247165\\
551	0.00211964270510427\\
552	0.00207044238962313\\
553	0.00202240905872459\\
554	0.00197565344549598\\
555	0.00192851318250082\\
556	0.00188093925523048\\
557	0.00183284421063473\\
558	0.00178424038990776\\
559	0.00173515199993384\\
560	0.0016856118630877\\
561	0.00163566034524177\\
562	0.00158534474355357\\
563	0.00153472014180543\\
564	0.00148382280604586\\
565	0.00143440624688174\\
566	0.00138673207142021\\
567	0.00133901646968657\\
568	0.00129117347449055\\
569	0.00124323567373714\\
570	0.00119524259563238\\
571	0.00114723506622825\\
572	0.00109925152891713\\
573	0.00105183107985831\\
574	0.00100464595484285\\
575	0.000957442170334152\\
576	0.00091026155066099\\
577	0.000863150546575395\\
578	0.000816158044783456\\
579	0.000769334999412795\\
580	0.000722734025690668\\
581	0.000676408840988291\\
582	0.000630413507501182\\
583	0.000584801426046155\\
584	0.000539624020560044\\
585	0.00049492904223658\\
586	0.000450758414500049\\
587	0.000407145545023687\\
588	0.000364112075908293\\
589	0.000321664201822908\\
590	0.000279789157284614\\
591	0.000238610660212779\\
592	0.000198280712249112\\
593	0.000158978649805849\\
594	0.000120944048332156\\
595	8.45661894165686e-05\\
596	5.0625062732988e-05\\
597	2.09371357655842e-05\\
598	8.08076597025309e-08\\
599	0\\
600	0\\
};
\addplot [color=mycolor20,solid,forget plot]
  table[row sep=crcr]{%
1	0.0058189943311047\\
2	0.00581899048861738\\
3	0.00581898657662179\\
4	0.00581898259385945\\
5	0.00581897853904904\\
6	0.00581897441088607\\
7	0.00581897020804234\\
8	0.0058189659291656\\
9	0.00581896157287904\\
10	0.00581895713778099\\
11	0.00581895262244425\\
12	0.00581894802541587\\
13	0.00581894334521644\\
14	0.00581893858033982\\
15	0.0058189337292525\\
16	0.00581892879039316\\
17	0.00581892376217224\\
18	0.0058189186429713\\
19	0.00581891343114257\\
20	0.00581890812500839\\
21	0.00581890272286067\\
22	0.00581889722296037\\
23	0.0058188916235369\\
24	0.00581888592278754\\
25	0.00581888011887689\\
26	0.00581887420993625\\
27	0.00581886819406307\\
28	0.00581886206932028\\
29	0.00581885583373564\\
30	0.0058188494853011\\
31	0.00581884302197231\\
32	0.00581883644166774\\
33	0.00581882974226811\\
34	0.00581882292161573\\
35	0.00581881597751374\\
36	0.00581880890772542\\
37	0.00581880170997348\\
38	0.00581879438193933\\
39	0.0058187869212623\\
40	0.00581877932553886\\
41	0.00581877159232193\\
42	0.00581876371911996\\
43	0.00581875570339625\\
44	0.00581874754256801\\
45	0.00581873923400561\\
46	0.00581873077503178\\
47	0.00581872216292051\\
48	0.00581871339489643\\
49	0.00581870446813374\\
50	0.00581869537975539\\
51	0.00581868612683212\\
52	0.0058186767063815\\
53	0.00581866711536691\\
54	0.0058186573506967\\
55	0.00581864740922304\\
56	0.00581863728774101\\
57	0.00581862698298743\\
58	0.00581861649163998\\
59	0.00581860581031593\\
60	0.00581859493557125\\
61	0.00581858386389928\\
62	0.00581857259172979\\
63	0.0058185611154277\\
64	0.00581854943129186\\
65	0.00581853753555401\\
66	0.00581852542437742\\
67	0.00581851309385572\\
68	0.00581850054001156\\
69	0.00581848775879548\\
70	0.00581847474608437\\
71	0.0058184614976803\\
72	0.00581844800930908\\
73	0.00581843427661892\\
74	0.00581842029517902\\
75	0.00581840606047805\\
76	0.00581839156792279\\
77	0.00581837681283659\\
78	0.00581836179045786\\
79	0.00581834649593852\\
80	0.00581833092434244\\
81	0.00581831507064385\\
82	0.00581829892972564\\
83	0.00581828249637776\\
84	0.00581826576529556\\
85	0.00581824873107802\\
86	0.00581823138822595\\
87	0.00581821373114036\\
88	0.00581819575412044\\
89	0.00581817745136188\\
90	0.00581815881695489\\
91	0.00581813984488234\\
92	0.00581812052901779\\
93	0.00581810086312344\\
94	0.00581808084084815\\
95	0.00581806045572551\\
96	0.00581803970117149\\
97	0.00581801857048245\\
98	0.005817997056833\\
99	0.00581797515327373\\
100	0.00581795285272891\\
101	0.00581793014799424\\
102	0.00581790703173459\\
103	0.00581788349648144\\
104	0.00581785953463063\\
105	0.00581783513843981\\
106	0.00581781030002592\\
107	0.00581778501136268\\
108	0.00581775926427791\\
109	0.00581773305045098\\
110	0.00581770636141002\\
111	0.00581767918852927\\
112	0.00581765152302614\\
113	0.00581762335595848\\
114	0.00581759467822167\\
115	0.00581756548054556\\
116	0.00581753575349166\\
117	0.00581750548744984\\
118	0.00581747467263543\\
119	0.00581744329908588\\
120	0.00581741135665765\\
121	0.00581737883502288\\
122	0.00581734572366601\\
123	0.00581731201188041\\
124	0.00581727768876489\\
125	0.00581724274322022\\
126	0.00581720716394537\\
127	0.00581717093943415\\
128	0.00581713405797115\\
129	0.00581709650762819\\
130	0.00581705827626026\\
131	0.00581701935150181\\
132	0.00581697972076256\\
133	0.00581693937122351\\
134	0.00581689828983281\\
135	0.00581685646330143\\
136	0.00581681387809896\\
137	0.00581677052044923\\
138	0.0058167263763258\\
139	0.00581668143144752\\
140	0.00581663567127376\\
141	0.00581658908099982\\
142	0.00581654164555217\\
143	0.0058164933495835\\
144	0.00581644417746784\\
145	0.0058163941132954\\
146	0.00581634314086757\\
147	0.00581629124369164\\
148	0.00581623840497542\\
149	0.0058161846076219\\
150	0.00581612983422373\\
151	0.00581607406705761\\
152	0.00581601728807853\\
153	0.005815959478914\\
154	0.00581590062085812\\
155	0.00581584069486558\\
156	0.00581577968154556\\
157	0.00581571756115536\\
158	0.00581565431359425\\
159	0.00581558991839684\\
160	0.00581552435472658\\
161	0.00581545760136911\\
162	0.0058153896367254\\
163	0.00581532043880485\\
164	0.00581524998521816\\
165	0.00581517825317028\\
166	0.00581510521945304\\
167	0.00581503086043775\\
168	0.00581495515206761\\
169	0.00581487806985002\\
170	0.00581479958884883\\
171	0.00581471968367632\\
172	0.00581463832848502\\
173	0.00581455549695965\\
174	0.00581447116230858\\
175	0.0058143852972554\\
176	0.00581429787403013\\
177	0.00581420886436058\\
178	0.00581411823946313\\
179	0.00581402597003379\\
180	0.0058139320262388\\
181	0.00581383637770522\\
182	0.00581373899351129\\
183	0.00581363984217669\\
184	0.00581353889165254\\
185	0.0058134361093113\\
186	0.00581333146193649\\
187	0.00581322491571224\\
188	0.00581311643621263\\
189	0.0058130059883908\\
190	0.0058128935365681\\
191	0.00581277904442274\\
192	0.00581266247497848\\
193	0.00581254379059309\\
194	0.00581242295294651\\
195	0.00581229992302894\\
196	0.00581217466112869\\
197	0.00581204712681978\\
198	0.00581191727894936\\
199	0.00581178507562501\\
200	0.0058116504742017\\
201	0.00581151343126855\\
202	0.00581137390263551\\
203	0.00581123184331967\\
204	0.00581108720753144\\
205	0.00581093994866043\\
206	0.00581079001926115\\
207	0.00581063737103848\\
208	0.00581048195483296\\
209	0.00581032372060566\\
210	0.00581016261742306\\
211	0.00580999859344151\\
212	0.00580983159589145\\
213	0.00580966157106156\\
214	0.0058094884642824\\
215	0.00580931221991001\\
216	0.00580913278130915\\
217	0.00580895009083635\\
218	0.00580876408982256\\
219	0.00580857471855574\\
220	0.00580838191626304\\
221	0.00580818562109268\\
222	0.00580798577009578\\
223	0.00580778229920759\\
224	0.00580757514322877\\
225	0.00580736423580614\\
226	0.00580714950941329\\
227	0.00580693089533093\\
228	0.00580670832362672\\
229	0.00580648172313517\\
230	0.00580625102143693\\
231	0.00580601614483805\\
232	0.00580577701834872\\
233	0.00580553356566181\\
234	0.00580528570913123\\
235	0.00580503336974976\\
236	0.00580477646712678\\
237	0.00580451491946559\\
238	0.00580424864354049\\
239	0.00580397755467338\\
240	0.00580370156671043\\
241	0.00580342059199805\\
242	0.00580313454135877\\
243	0.00580284332406672\\
244	0.00580254684782289\\
245	0.00580224501873001\\
246	0.00580193774126712\\
247	0.00580162491826385\\
248	0.00580130645087445\\
249	0.0058009822385514\\
250	0.0058006521790188\\
251	0.00580031616824536\\
252	0.00579997410041724\\
253	0.00579962586791034\\
254	0.00579927136126256\\
255	0.00579891046914552\\
256	0.00579854307833615\\
257	0.00579816907368785\\
258	0.00579778833810141\\
259	0.00579740075249563\\
260	0.00579700619577767\\
261	0.00579660454481293\\
262	0.00579619567439495\\
263	0.00579577945721478\\
264	0.00579535576383\\
265	0.00579492446263373\\
266	0.00579448541982312\\
267	0.00579403849936761\\
268	0.00579358356297703\\
269	0.00579312047006925\\
270	0.00579264907773764\\
271	0.00579216924071824\\
272	0.00579168081135672\\
273	0.00579118363957502\\
274	0.00579067757283759\\
275	0.00579016245611779\\
276	0.00578963813186343\\
277	0.00578910443996263\\
278	0.00578856121770904\\
279	0.00578800829976697\\
280	0.00578744551813627\\
281	0.00578687270211685\\
282	0.0057862896782732\\
283	0.00578569627039833\\
284	0.00578509229947774\\
285	0.0057844775836531\\
286	0.00578385193818555\\
287	0.00578321517541889\\
288	0.00578256710474258\\
289	0.00578190753255434\\
290	0.00578123626222265\\
291	0.00578055309404898\\
292	0.00577985782522971\\
293	0.00577915024981808\\
294	0.00577843015868544\\
295	0.00577769733948282\\
296	0.00577695157660183\\
297	0.00577619265113563\\
298	0.00577542034083951\\
299	0.00577463442009133\\
300	0.00577383465985178\\
301	0.00577302082762431\\
302	0.00577219268741509\\
303	0.00577134999969257\\
304	0.00577049252134708\\
305	0.00576962000565007\\
306	0.00576873220221342\\
307	0.00576782885694859\\
308	0.00576690971202568\\
309	0.00576597450583249\\
310	0.00576502297293362\\
311	0.00576405484402952\\
312	0.00576306984591583\\
313	0.00576206770144273\\
314	0.00576104812947463\\
315	0.00576001084485013\\
316	0.00575895555834248\\
317	0.00575788197662035\\
318	0.00575678980220947\\
319	0.00575567873345477\\
320	0.00575454846448344\\
321	0.00575339868516897\\
322	0.00575222908109635\\
323	0.00575103933352845\\
324	0.00574982911937387\\
325	0.0057485981111564\\
326	0.00574734597698629\\
327	0.0057460723805336\\
328	0.00574477698100365\\
329	0.00574345943311508\\
330	0.00574211938708086\\
331	0.00574075648859236\\
332	0.00573937037880739\\
333	0.0057379606943427\\
334	0.00573652706727171\\
335	0.00573506912512924\\
336	0.00573358649092473\\
337	0.00573207878316661\\
338	0.00573054561590189\\
339	0.0057289865987759\\
340	0.00572740133712038\\
341	0.00572578943208159\\
342	0.0057241504808119\\
343	0.0057224840767854\\
344	0.00572078981042825\\
345	0.00571906727018169\\
346	0.00571731604311944\\
347	0.0057155357156663\\
348	0.00571372587442581\\
349	0.00571188610712166\\
350	0.00571001600365173\\
351	0.00570811515724247\\
352	0.00570618316567106\\
353	0.00570421963247144\\
354	0.0057022241678897\\
355	0.00570019638887914\\
356	0.00569813591612427\\
357	0.00569604237440946\\
358	0.00569391539328897\\
359	0.00569175460780113\\
360	0.00568955965922774\\
361	0.0056873301959015\\
362	0.00568506587406302\\
363	0.00568276635876939\\
364	0.00568043132485621\\
365	0.0056780604579544\\
366	0.0056756534555632\\
367	0.00567321002818025\\
368	0.00567072990048924\\
369	0.00566821281260475\\
370	0.00566565852137346\\
371	0.00566306680172961\\
372	0.00566043744810143\\
373	0.00565777027586367\\
374	0.00565506512282926\\
375	0.00565232185077079\\
376	0.00564954034695943\\
377	0.00564672052570524\\
378	0.00564386232987799\\
379	0.00564096573238249\\
380	0.00563803073755523\\
381	0.00563505738244078\\
382	0.0056320457378995\\
383	0.00562899590947649\\
384	0.00562590803796374\\
385	0.00562278229955376\\
386	0.00561961890547257\\
387	0.00561641810095263\\
388	0.00561318016337811\\
389	0.00560990539940227\\
390	0.00560659414079835\\
391	0.00560324673876242\\
392	0.00559986355633739\\
393	0.00559644495857461\\
394	0.00559299129999278\\
395	0.00558950290883588\\
396	0.00558598006757688\\
397	0.00558242298906692\\
398	0.0055788317876936\\
399	0.0055752064449411\\
400	0.00557154676883735\\
401	0.00556785234694397\\
402	0.00556412249285875\\
403	0.00556035618675572\\
404	0.00555655201138245\\
405	0.0055527080863246\\
406	0.00554882200545003\\
407	0.00554489078555436\\
408	0.00554091083876069\\
409	0.00553687798780285\\
410	0.00553278755314146\\
411	0.00552863455684651\\
412	0.00552441411766449\\
413	0.00552012216811897\\
414	0.00551575668407216\\
415	0.00551131649874814\\
416	0.00550680042385179\\
417	0.00550220724822848\\
418	0.00549753573636986\\
419	0.00549278462675265\\
420	0.00548795262997548\\
421	0.00548303842664436\\
422	0.00547804066506772\\
423	0.00547295795884481\\
424	0.00546778888464244\\
425	0.00546253197979302\\
426	0.0054571857396711\\
427	0.00545174861479672\\
428	0.00544621900758791\\
429	0.00544059526833875\\
430	0.00543487569020763\\
431	0.00542905850356232\\
432	0.00542314186960916\\
433	0.00541712387323063\\
434	0.00541100251495298\\
435	0.00540477570196575\\
436	0.00539844123811781\\
437	0.00539199681282254\\
438	0.00538543998881774\\
439	0.00537876818874875\\
440	0.00537197868057494\\
441	0.0053650685618464\\
442	0.00535803474296162\\
443	0.00535087392960221\\
444	0.00534358260465212\\
445	0.00533615701004273\\
446	0.00532859312910794\\
447	0.0053208866701388\\
448	0.00531303305195366\\
449	0.00530502739391087\\
450	0.00529686453230709\\
451	0.00528853903392121\\
452	0.00528004520647956\\
453	0.00527137711762544\\
454	0.00526252862358042\\
455	0.00525349340803192\\
456	0.00524426503061253\\
457	0.00523483698231594\\
458	0.00522520274226246\\
459	0.00521535582592539\\
460	0.00520528980628244\\
461	0.00519499826267799\\
462	0.00518447454400959\\
463	0.00517371173398307\\
464	0.00516270263120784\\
465	0.00515143971860299\\
466	0.00513991506446306\\
467	0.0051281190755423\\
468	0.00511604156982477\\
469	0.00510367177878094\\
470	0.00509099831430835\\
471	0.00507800913626572\\
472	0.00506469152153891\\
473	0.00505103203520088\\
474	0.00503701650551431\\
475	0.00502263000528081\\
476	0.0050078568438303\\
477	0.00499268058064308\\
478	0.00497708409278853\\
479	0.00496104981779552\\
480	0.00494458940329023\\
481	0.00492770406071771\\
482	0.00491037752266448\\
483	0.00489259274397273\\
484	0.00487433185554927\\
485	0.00485557610943205\\
486	0.00483630581510927\\
487	0.00481650026529632\\
488	0.00479613761911881\\
489	0.00477519470434364\\
490	0.00475364681428678\\
491	0.00473146801694555\\
492	0.00470863109634954\\
493	0.00468505617964145\\
494	0.00466068205714473\\
495	0.0046354509297607\\
496	0.00460931890942681\\
497	0.00458223967253629\\
498	0.00455416526072493\\
499	0.00452504522101895\\
500	0.00449482653582271\\
501	0.00446345359508368\\
502	0.00443086813049848\\
503	0.00439700902290536\\
504	0.00436181226022157\\
505	0.00432521487795581\\
506	0.00428715248823557\\
507	0.00424755976829419\\
508	0.00420637111577386\\
509	0.00416352151206047\\
510	0.00411894764309728\\
511	0.0040725893774367\\
512	0.00402439184420049\\
513	0.00397430783873562\\
514	0.00392230085704828\\
515	0.0038683486727357\\
516	0.00381244726626583\\
517	0.00375461607817293\\
518	0.0036949020595949\\
519	0.0036333975298744\\
520	0.00357024590656978\\
521	0.00350565216217114\\
522	0.00343989017156044\\
523	0.00337364966034245\\
524	0.00330738993546743\\
525	0.003241822427488\\
526	0.0031784376872927\\
527	0.0031209041260667\\
528	0.00306946043912308\\
529	0.00302385980061399\\
530	0.00298377727836047\\
531	0.00294517454871405\\
532	0.00290752030953723\\
533	0.00287031947659085\\
534	0.00283301479825892\\
535	0.00279532016991253\\
536	0.00275710871567116\\
537	0.00271827405385313\\
538	0.0026787419386912\\
539	0.00263847265809618\\
540	0.00259743279835268\\
541	0.00255559439519641\\
542	0.00251293369164769\\
543	0.00246942852407803\\
544	0.00242505720159905\\
545	0.0023797986117253\\
546	0.00233363152927993\\
547	0.00228648906817017\\
548	0.00223833610645649\\
549	0.00219109649102826\\
550	0.00214501385724158\\
551	0.00210014128704073\\
552	0.00205485014034149\\
553	0.00200908772075898\\
554	0.00196276720468764\\
555	0.00191589603940005\\
556	0.00186848541154819\\
557	0.00182056079771141\\
558	0.00177215264727711\\
559	0.00172329667030389\\
560	0.00167403451171803\\
561	0.00162441520968452\\
562	0.00157448138289071\\
563	0.00152557528021982\\
564	0.0014784224607657\\
565	0.00143132660754472\\
566	0.00138403786757794\\
567	0.00133658557388737\\
568	0.00128900666033571\\
569	0.00124133897812713\\
570	0.0011936203318965\\
571	0.00114601541641846\\
572	0.00109894815101284\\
573	0.0010518280284347\\
574	0.00100464554763924\\
575	0.000957442005608793\\
576	0.000910261465246949\\
577	0.000863150502318578\\
578	0.00081615802285027\\
579	0.000769334989215063\\
580	0.00072273402131522\\
581	0.000676408839288589\\
582	0.000630413506918528\\
583	0.000584801425876618\\
584	0.000539624020520602\\
585	0.000494929042229906\\
586	0.000450758414499449\\
587	0.000407145545023685\\
588	0.000364112075908288\\
589	0.000321664201822908\\
590	0.000279789157284616\\
591	0.00023861066021278\\
592	0.000198280712249113\\
593	0.00015897864980585\\
594	0.000120944048332155\\
595	8.45661894165684e-05\\
596	5.06250627329885e-05\\
597	2.09371357655837e-05\\
598	8.08076597025309e-08\\
599	0\\
600	0\\
};
\addplot [color=mycolor21,solid,forget plot]
  table[row sep=crcr]{%
1	0.00572980858070865\\
2	0.00572980588990021\\
3	0.00572980315008438\\
4	0.00572980036036668\\
5	0.00572979751983632\\
6	0.00572979462756577\\
7	0.00572979168261063\\
8	0.00572978868400912\\
9	0.00572978563078193\\
10	0.00572978252193178\\
11	0.00572977935644314\\
12	0.00572977613328191\\
13	0.00572977285139503\\
14	0.00572976950971016\\
15	0.00572976610713536\\
16	0.00572976264255868\\
17	0.00572975911484778\\
18	0.00572975552284958\\
19	0.00572975186538997\\
20	0.00572974814127325\\
21	0.00572974434928183\\
22	0.00572974048817582\\
23	0.00572973655669266\\
24	0.00572973255354661\\
25	0.00572972847742839\\
26	0.00572972432700474\\
27	0.00572972010091796\\
28	0.00572971579778546\\
29	0.00572971141619932\\
30	0.00572970695472584\\
31	0.00572970241190501\\
32	0.0057296977862501\\
33	0.00572969307624711\\
34	0.00572968828035423\\
35	0.00572968339700146\\
36	0.00572967842459001\\
37	0.00572967336149171\\
38	0.0057296682060486\\
39	0.00572966295657228\\
40	0.0057296576113434\\
41	0.00572965216861109\\
42	0.00572964662659239\\
43	0.00572964098347156\\
44	0.00572963523739961\\
45	0.00572962938649362\\
46	0.00572962342883607\\
47	0.00572961736247435\\
48	0.00572961118541993\\
49	0.0057296048956479\\
50	0.0057295984910961\\
51	0.00572959196966453\\
52	0.00572958532921468\\
53	0.00572957856756878\\
54	0.00572957168250908\\
55	0.00572956467177711\\
56	0.00572955753307294\\
57	0.00572955026405439\\
58	0.00572954286233632\\
59	0.00572953532548976\\
60	0.00572952765104113\\
61	0.00572951983647145\\
62	0.00572951187921544\\
63	0.00572950377666068\\
64	0.00572949552614682\\
65	0.00572948712496463\\
66	0.00572947857035503\\
67	0.00572946985950834\\
68	0.00572946098956321\\
69	0.00572945195760569\\
70	0.00572944276066831\\
71	0.00572943339572902\\
72	0.00572942385971027\\
73	0.00572941414947792\\
74	0.00572940426184017\\
75	0.00572939419354663\\
76	0.00572938394128708\\
77	0.00572937350169045\\
78	0.00572936287132367\\
79	0.00572935204669055\\
80	0.00572934102423061\\
81	0.00572932980031781\\
82	0.00572931837125953\\
83	0.00572930673329514\\
84	0.00572929488259488\\
85	0.00572928281525844\\
86	0.00572927052731383\\
87	0.00572925801471591\\
88	0.00572924527334514\\
89	0.00572923229900613\\
90	0.00572921908742632\\
91	0.00572920563425446\\
92	0.00572919193505921\\
93	0.00572917798532768\\
94	0.00572916378046386\\
95	0.00572914931578709\\
96	0.00572913458653062\\
97	0.00572911958783983\\
98	0.0057291043147707\\
99	0.0057290887622882\\
100	0.00572907292526452\\
101	0.0057290567984774\\
102	0.00572904037660837\\
103	0.00572902365424101\\
104	0.00572900662585905\\
105	0.00572898928584461\\
106	0.00572897162847626\\
107	0.00572895364792716\\
108	0.00572893533826308\\
109	0.00572891669344039\\
110	0.00572889770730405\\
111	0.00572887837358558\\
112	0.00572885868590096\\
113	0.0057288386377484\\
114	0.00572881822250627\\
115	0.00572879743343088\\
116	0.00572877626365406\\
117	0.00572875470618113\\
118	0.00572873275388827\\
119	0.00572871039952033\\
120	0.00572868763568835\\
121	0.00572866445486702\\
122	0.00572864084939219\\
123	0.00572861681145837\\
124	0.00572859233311598\\
125	0.00572856740626882\\
126	0.00572854202267128\\
127	0.00572851617392549\\
128	0.00572848985147871\\
129	0.00572846304662026\\
130	0.00572843575047876\\
131	0.00572840795401903\\
132	0.00572837964803907\\
133	0.00572835082316702\\
134	0.00572832146985805\\
135	0.00572829157839101\\
136	0.00572826113886535\\
137	0.0057282301411976\\
138	0.00572819857511821\\
139	0.00572816643016788\\
140	0.00572813369569418\\
141	0.00572810036084794\\
142	0.00572806641457964\\
143	0.00572803184563565\\
144	0.00572799664255446\\
145	0.00572796079366283\\
146	0.00572792428707194\\
147	0.00572788711067325\\
148	0.00572784925213458\\
149	0.00572781069889593\\
150	0.00572777143816524\\
151	0.00572773145691407\\
152	0.00572769074187328\\
153	0.00572764927952856\\
154	0.00572760705611591\\
155	0.00572756405761695\\
156	0.0057275202697543\\
157	0.00572747567798676\\
158	0.00572743026750441\\
159	0.00572738402322368\\
160	0.00572733692978226\\
161	0.00572728897153397\\
162	0.00572724013254346\\
163	0.00572719039658093\\
164	0.0057271397471167\\
165	0.00572708816731559\\
166	0.00572703564003138\\
167	0.00572698214780098\\
168	0.00572692767283864\\
169	0.0057268721970299\\
170	0.00572681570192569\\
171	0.00572675816873603\\
172	0.00572669957832379\\
173	0.00572663991119822\\
174	0.00572657914750854\\
175	0.00572651726703724\\
176	0.00572645424919339\\
177	0.00572639007300561\\
178	0.00572632471711528\\
179	0.00572625815976923\\
180	0.00572619037881261\\
181	0.00572612135168142\\
182	0.00572605105539507\\
183	0.00572597946654861\\
184	0.00572590656130508\\
185	0.00572583231538744\\
186	0.00572575670407063\\
187	0.00572567970217327\\
188	0.00572560128404926\\
189	0.0057255214235794\\
190	0.0057254400941626\\
191	0.00572535726870714\\
192	0.00572527291962164\\
193	0.00572518701880602\\
194	0.00572509953764206\\
195	0.00572501044698407\\
196	0.00572491971714924\\
197	0.00572482731790783\\
198	0.0057247332184732\\
199	0.00572463738749171\\
200	0.00572453979303234\\
201	0.00572444040257632\\
202	0.00572433918300635\\
203	0.00572423610059577\\
204	0.00572413112099751\\
205	0.00572402420923289\\
206	0.00572391532968017\\
207	0.00572380444606294\\
208	0.00572369152143828\\
209	0.00572357651818473\\
210	0.00572345939799014\\
211	0.00572334012183916\\
212	0.00572321865000065\\
213	0.0057230949420148\\
214	0.00572296895668011\\
215	0.00572284065204006\\
216	0.00572270998536966\\
217	0.00572257691316169\\
218	0.00572244139111277\\
219	0.00572230337410918\\
220	0.00572216281621248\\
221	0.00572201967064488\\
222	0.00572187388977428\\
223	0.0057217254250993\\
224	0.00572157422723383\\
225	0.00572142024589144\\
226	0.00572126342986964\\
227	0.00572110372703367\\
228	0.00572094108430026\\
229	0.00572077544762098\\
230	0.00572060676196549\\
231	0.00572043497130432\\
232	0.00572026001859167\\
233	0.00572008184574766\\
234	0.00571990039364053\\
235	0.00571971560206847\\
236	0.00571952740974128\\
237	0.00571933575426161\\
238	0.00571914057210606\\
239	0.00571894179860604\\
240	0.00571873936792823\\
241	0.00571853321305482\\
242	0.00571832326576356\\
243	0.00571810945660734\\
244	0.00571789171489385\\
245	0.00571766996866448\\
246	0.00571744414467343\\
247	0.00571721416836621\\
248	0.00571697996385799\\
249	0.00571674145391172\\
250	0.00571649855991586\\
251	0.005716251201862\\
252	0.005715999298322\\
253	0.00571574276642509\\
254	0.00571548152183448\\
255	0.00571521547872393\\
256	0.00571494454975384\\
257	0.0057146686460472\\
258	0.00571438767716531\\
259	0.00571410155108312\\
260	0.00571381017416442\\
261	0.00571351345113674\\
262	0.00571321128506597\\
263	0.00571290357733082\\
264	0.00571259022759698\\
265	0.00571227113379102\\
266	0.00571194619207409\\
267	0.00571161529681544\\
268	0.00571127834056559\\
269	0.00571093521402938\\
270	0.00571058580603878\\
271	0.00571023000352545\\
272	0.00570986769149316\\
273	0.0057094987529899\\
274	0.00570912306907995\\
275	0.0057087405188155\\
276	0.0057083509792084\\
277	0.00570795432520142\\
278	0.00570755042963945\\
279	0.00570713916324061\\
280	0.00570672039456694\\
281	0.00570629398999513\\
282	0.00570585981368683\\
283	0.00570541772755903\\
284	0.00570496759125408\\
285	0.00570450926210949\\
286	0.0057040425951276\\
287	0.00570356744294506\\
288	0.00570308365580196\\
289	0.00570259108151089\\
290	0.00570208956542554\\
291	0.00570157895040927\\
292	0.00570105907680322\\
293	0.00570052978239412\\
294	0.00569999090238186\\
295	0.0056994422693466\\
296	0.00569888371321548\\
297	0.00569831506122902\\
298	0.00569773613790678\\
299	0.00569714676501285\\
300	0.00569654676152045\\
301	0.00569593594357608\\
302	0.00569531412446293\\
303	0.00569468111456369\\
304	0.00569403672132235\\
305	0.00569338074920533\\
306	0.00569271299966165\\
307	0.00569203327108205\\
308	0.00569134135875729\\
309	0.00569063705483507\\
310	0.00568992014827601\\
311	0.00568919042480851\\
312	0.00568844766688212\\
313	0.00568769165361995\\
314	0.00568692216076972\\
315	0.0056861389606537\\
316	0.00568534182211728\\
317	0.00568453051047669\\
318	0.00568370478746552\\
319	0.00568286441118059\\
320	0.00568200913602721\\
321	0.00568113871266388\\
322	0.00568025288794722\\
323	0.00567935140487701\\
324	0.00567843400254235\\
325	0.00567750041606899\\
326	0.00567655037656877\\
327	0.00567558361109176\\
328	0.00567459984258204\\
329	0.00567359878983792\\
330	0.00567258016747761\\
331	0.00567154368591135\\
332	0.00567048905132123\\
333	0.00566941596564959\\
334	0.00566832412659743\\
335	0.00566721322763351\\
336	0.00566608295801537\\
337	0.00566493300282299\\
338	0.00566376304300525\\
339	0.00566257275543965\\
340	0.00566136181300442\\
341	0.00566012988466183\\
342	0.00565887663555065\\
343	0.00565760172708427\\
344	0.00565630481704481\\
345	0.00565498555965529\\
346	0.00565364360564224\\
347	0.00565227860228427\\
348	0.00565089019344443\\
349	0.00564947801958723\\
350	0.00564804171777567\\
351	0.00564658092164472\\
352	0.00564509526134834\\
353	0.00564358436347828\\
354	0.00564204785095603\\
355	0.00564048534291579\\
356	0.00563889645469001\\
357	0.00563728079779628\\
358	0.00563563797991627\\
359	0.00563396760486469\\
360	0.00563226927254688\\
361	0.00563054257890199\\
362	0.0056287871158299\\
363	0.00562700247109842\\
364	0.00562518822822775\\
365	0.00562334396634844\\
366	0.00562146926002884\\
367	0.00561956367906719\\
368	0.00561762678824332\\
369	0.00561565814702416\\
370	0.00561365730921694\\
371	0.00561162382256267\\
372	0.00560955722826248\\
373	0.00560745706042848\\
374	0.00560532284544959\\
375	0.00560315410126286\\
376	0.00560095033651909\\
377	0.00559871104963185\\
378	0.00559643572769721\\
379	0.00559412384527191\\
380	0.00559177486299641\\
381	0.00558938822604944\\
382	0.00558696336242046\\
383	0.00558449968098667\\
384	0.00558199656938208\\
385	0.00557945339164709\\
386	0.00557686948564961\\
387	0.0055742441602709\\
388	0.00557157669235389\\
389	0.00556886632341682\\
390	0.00556611225614206\\
391	0.00556331365065909\\
392	0.00556046962065196\\
393	0.0055575792293357\\
394	0.00555464148536417\\
395	0.00555165533875192\\
396	0.00554861967691942\\
397	0.00554553332100103\\
398	0.00554239502259909\\
399	0.00553920346121166\\
400	0.00553595724246715\\
401	0.0055326548972854\\
402	0.00552929488275172\\
403	0.00552587558521238\\
404	0.00552239532614742\\
405	0.00551885237138094\\
406	0.00551524494411106\\
407	0.00551157124202752\\
408	0.00550782945835686\\
409	0.00550401780595425\\
410	0.00550013454248171\\
411	0.0054961779931372\\
412	0.00549214656440038\\
413	0.00548803873343755\\
414	0.00548385297464812\\
415	0.0054795877235188\\
416	0.00547524137412609\\
417	0.00547081227630575\\
418	0.00546629873245934\\
419	0.00546169899397192\\
420	0.00545701125722286\\
421	0.00545223365918544\\
422	0.00544736427262794\\
423	0.00544240110095367\\
424	0.00543734207273293\\
425	0.00543218503601366\\
426	0.0054269277524891\\
427	0.00542156789157018\\
428	0.00541610302508241\\
429	0.00541053063324547\\
430	0.00540484811549583\\
431	0.00539905278855239\\
432	0.0053931418846683\\
433	0.00538711255010305\\
434	0.00538096184384828\\
435	0.00537468673663552\\
436	0.00536828411025143\\
437	0.00536175075717169\\
438	0.00535508338050838\\
439	0.00534827859423989\\
440	0.00534133292365744\\
441	0.00533424280591705\\
442	0.00532700459052804\\
443	0.00531961453954857\\
444	0.00531206882721155\\
445	0.00530436353871283\\
446	0.00529649466797224\\
447	0.00528845811390706\\
448	0.0052802496708233\\
449	0.00527186498000972\\
450	0.00526329883177295\\
451	0.00525454543292015\\
452	0.00524559871886021\\
453	0.00523645233997658\\
454	0.00522709964683691\\
455	0.00521753367389918\\
456	0.00520774712137015\\
457	0.00519773233603844\\
458	0.00518748129105594\\
459	0.00517698556554422\\
460	0.00516623632534357\\
461	0.00515522430808488\\
462	0.00514393982759738\\
463	0.00513237280208273\\
464	0.00512051284339468\\
465	0.00510834953912344\\
466	0.0050959112396749\\
467	0.00508319080899618\\
468	0.00507017964850544\\
469	0.00505686879703203\\
470	0.00504324890299095\\
471	0.00502931018866832\\
472	0.00501504242834302\\
473	0.00500043491551162\\
474	0.00498547642149313\\
475	0.00497015515886423\\
476	0.00495445874798\\
477	0.00493837425526864\\
478	0.00492188808066493\\
479	0.00490498602191583\\
480	0.00488762167633039\\
481	0.00486975714924507\\
482	0.00485137058927254\\
483	0.00483243885327868\\
484	0.00481293742390933\\
485	0.00479284029011636\\
486	0.00477211974468235\\
487	0.00475074614835898\\
488	0.00472868819717229\\
489	0.00470591246305681\\
490	0.00468237981878312\\
491	0.00465803089770965\\
492	0.0046328120774756\\
493	0.00460667943089701\\
494	0.00457958768879581\\
495	0.00455148988903157\\
496	0.00452233691457148\\
497	0.00449207753025417\\
498	0.00446065843249402\\
499	0.00442802437117219\\
500	0.00439411833385243\\
501	0.00435888181187339\\
502	0.00432225517214836\\
503	0.00428417817715652\\
504	0.00424459071668929\\
505	0.00420343362284731\\
506	0.00416064997853145\\
507	0.00411618656330434\\
508	0.00406999566691615\\
509	0.00402203736931663\\
510	0.0039722823715641\\
511	0.00392071437273272\\
512	0.00386732727103458\\
513	0.00381213754938165\\
514	0.00375519160680992\\
515	0.00369657997734466\\
516	0.00363645886131694\\
517	0.00357537602172463\\
518	0.00351364615656154\\
519	0.00345164674216454\\
520	0.00338984365641533\\
521	0.00332895735969821\\
522	0.00327067693696174\\
523	0.003217961410773\\
524	0.00317098923397473\\
525	0.00312972553097366\\
526	0.0030932630615445\\
527	0.0030579938712638\\
528	0.00302347996153812\\
529	0.00298922036303293\\
530	0.00295467410305157\\
531	0.00291969843490789\\
532	0.00288417953508926\\
533	0.00284802943643086\\
534	0.00281119674925301\\
535	0.00277364795711038\\
536	0.00273535508556916\\
537	0.00269629493608243\\
538	0.00265644719581064\\
539	0.00261579181755458\\
540	0.00257430843506074\\
541	0.00253197591786793\\
542	0.00248877282878466\\
543	0.0024446770304164\\
544	0.00239963599060748\\
545	0.00235356205372431\\
546	0.00230807274884474\\
547	0.00226370411865981\\
548	0.00222069860746232\\
549	0.0021772558959252\\
550	0.00213331854929651\\
551	0.00208879951073871\\
552	0.00204370495426489\\
553	0.00199804505517157\\
554	0.00195183607002464\\
555	0.00190509398810132\\
556	0.00185784220060583\\
557	0.00181011150213989\\
558	0.00176193793786907\\
559	0.00171336418739766\\
560	0.00166444065104269\\
561	0.00161588556246529\\
562	0.00156903735578489\\
563	0.00152263100232567\\
564	0.00147598659776143\\
565	0.00142911890248718\\
566	0.0013820608930472\\
567	0.00133484662230631\\
568	0.00128751046845312\\
569	0.00124008656121583\\
570	0.00119285104188482\\
571	0.00114596638690145\\
572	0.00109894786732217\\
573	0.00105182797131446\\
574	0.00100464552156289\\
575	0.000957441992168387\\
576	0.000910261458474358\\
577	0.000863150499077139\\
578	0.000816158021400161\\
579	0.00076933498861807\\
580	0.000722734021093481\\
581	0.000676408839216057\\
582	0.000630413506898499\\
583	0.000584801425872104\\
584	0.000539624020519896\\
585	0.000494929042229852\\
586	0.000450758414499456\\
587	0.000407145545023688\\
588	0.000364112075908294\\
589	0.00032166420182291\\
590	0.000279789157284615\\
591	0.000238610660212781\\
592	0.000198280712249114\\
593	0.00015897864980585\\
594	0.000120944048332156\\
595	8.45661894165688e-05\\
596	5.06250627329888e-05\\
597	2.09371357655836e-05\\
598	8.08076597025309e-08\\
599	0\\
600	0\\
};
\addplot [color=black!20!mycolor21,solid,forget plot]
  table[row sep=crcr]{%
1	0.00569718343847395\\
2	0.00569718126928133\\
3	0.00569717906032088\\
4	0.0056971768108613\\
5	0.0056971745201578\\
6	0.00569717218745182\\
7	0.00569716981197079\\
8	0.00569716739292795\\
9	0.0056971649295219\\
10	0.00569716242093652\\
11	0.00569715986634063\\
12	0.00569715726488762\\
13	0.00569715461571529\\
14	0.00569715191794556\\
15	0.00569714917068407\\
16	0.00569714637301994\\
17	0.0056971435240255\\
18	0.0056971406227559\\
19	0.00569713766824881\\
20	0.00569713465952419\\
21	0.00569713159558383\\
22	0.0056971284754111\\
23	0.00569712529797054\\
24	0.00569712206220757\\
25	0.00569711876704813\\
26	0.00569711541139826\\
27	0.0056971119941438\\
28	0.00569710851414995\\
29	0.005697104970261\\
30	0.00569710136129979\\
31	0.00569709768606737\\
32	0.00569709394334267\\
33	0.00569709013188196\\
34	0.00569708625041859\\
35	0.00569708229766238\\
36	0.0056970782722993\\
37	0.00569707417299105\\
38	0.00569706999837446\\
39	0.00569706574706117\\
40	0.00569706141763718\\
41	0.00569705700866222\\
42	0.00569705251866934\\
43	0.00569704794616449\\
44	0.0056970432896259\\
45	0.00569703854750363\\
46	0.00569703371821904\\
47	0.00569702880016423\\
48	0.00569702379170148\\
49	0.00569701869116272\\
50	0.00569701349684902\\
51	0.00569700820702993\\
52	0.00569700281994292\\
53	0.00569699733379278\\
54	0.00569699174675103\\
55	0.00569698605695526\\
56	0.00569698026250856\\
57	0.00569697436147885\\
58	0.00569696835189815\\
59	0.00569696223176201\\
60	0.00569695599902877\\
61	0.00569694965161893\\
62	0.00569694318741438\\
63	0.00569693660425771\\
64	0.00569692989995143\\
65	0.00569692307225731\\
66	0.00569691611889558\\
67	0.00569690903754409\\
68	0.00569690182583756\\
69	0.00569689448136683\\
70	0.00569688700167802\\
71	0.0056968793842716\\
72	0.00569687162660165\\
73	0.00569686372607497\\
74	0.00569685568005011\\
75	0.00569684748583661\\
76	0.0056968391406939\\
77	0.00569683064183059\\
78	0.00569682198640335\\
79	0.00569681317151594\\
80	0.00569680419421833\\
81	0.00569679505150564\\
82	0.00569678574031703\\
83	0.00569677625753479\\
84	0.00569676659998317\\
85	0.00569675676442738\\
86	0.00569674674757241\\
87	0.00569673654606193\\
88	0.00569672615647716\\
89	0.00569671557533565\\
90	0.00569670479909013\\
91	0.0056966938241273\\
92	0.00569668264676654\\
93	0.00569667126325871\\
94	0.00569665966978485\\
95	0.00569664786245484\\
96	0.00569663583730606\\
97	0.00569662359030209\\
98	0.00569661111733131\\
99	0.00569659841420536\\
100	0.0056965854766579\\
101	0.00569657230034303\\
102	0.00569655888083381\\
103	0.00569654521362075\\
104	0.00569653129411023\\
105	0.005696517117623\\
106	0.00569650267939245\\
107	0.00569648797456305\\
108	0.0056964729981887\\
109	0.00569645774523092\\
110	0.00569644221055725\\
111	0.00569642638893935\\
112	0.00569641027505131\\
113	0.00569639386346771\\
114	0.00569637714866183\\
115	0.00569636012500371\\
116	0.00569634278675821\\
117	0.00569632512808301\\
118	0.00569630714302666\\
119	0.00569628882552643\\
120	0.00569627016940623\\
121	0.00569625116837454\\
122	0.00569623181602221\\
123	0.00569621210582019\\
124	0.00569619203111733\\
125	0.00569617158513805\\
126	0.00569615076097998\\
127	0.00569612955161158\\
128	0.00569610794986972\\
129	0.00569608594845721\\
130	0.00569606353994018\\
131	0.0056960407167456\\
132	0.00569601747115861\\
133	0.00569599379531984\\
134	0.00569596968122266\\
135	0.00569594512071045\\
136	0.00569592010547372\\
137	0.00569589462704725\\
138	0.00569586867680708\\
139	0.00569584224596762\\
140	0.00569581532557849\\
141	0.00569578790652145\\
142	0.0056957599795072\\
143	0.00569573153507217\\
144	0.00569570256357521\\
145	0.00569567305519422\\
146	0.00569564299992272\\
147	0.00569561238756635\\
148	0.00569558120773935\\
149	0.0056955494498609\\
150	0.00569551710315139\\
151	0.0056954841566287\\
152	0.00569545059910437\\
153	0.00569541641917963\\
154	0.00569538160524148\\
155	0.00569534614545854\\
156	0.00569531002777697\\
157	0.00569527323991622\\
158	0.00569523576936474\\
159	0.00569519760337553\\
160	0.00569515872896178\\
161	0.00569511913289225\\
162	0.00569507880168657\\
163	0.0056950377216106\\
164	0.00569499587867155\\
165	0.00569495325861309\\
166	0.00569490984691027\\
167	0.00569486562876449\\
168	0.00569482058909824\\
169	0.00569477471254987\\
170	0.00569472798346804\\
171	0.00569468038590632\\
172	0.0056946319036175\\
173	0.00569458252004794\\
174	0.00569453221833166\\
175	0.00569448098128439\\
176	0.00569442879139752\\
177	0.00569437563083196\\
178	0.00569432148141171\\
179	0.00569426632461761\\
180	0.00569421014158059\\
181	0.00569415291307519\\
182	0.00569409461951262\\
183	0.0056940352409339\\
184	0.00569397475700274\\
185	0.00569391314699838\\
186	0.0056938503898082\\
187	0.00569378646392024\\
188	0.00569372134741564\\
189	0.00569365501796078\\
190	0.00569358745279937\\
191	0.00569351862874437\\
192	0.00569344852216984\\
193	0.00569337710900242\\
194	0.00569330436471283\\
195	0.00569323026430715\\
196	0.00569315478231792\\
197	0.00569307789279513\\
198	0.00569299956929691\\
199	0.00569291978488015\\
200	0.0056928385120909\\
201	0.00569275572295461\\
202	0.00569267138896617\\
203	0.00569258548107973\\
204	0.00569249796969838\\
205	0.00569240882466356\\
206	0.00569231801524433\\
207	0.00569222551012637\\
208	0.00569213127740088\\
209	0.00569203528455321\\
210	0.0056919374984511\\
211	0.00569183788533303\\
212	0.00569173641079617\\
213	0.00569163303978391\\
214	0.00569152773657361\\
215	0.00569142046476361\\
216	0.00569131118726042\\
217	0.00569119986626537\\
218	0.00569108646326116\\
219	0.00569097093899809\\
220	0.00569085325348008\\
221	0.00569073336595034\\
222	0.00569061123487689\\
223	0.00569048681793773\\
224	0.00569036007200571\\
225	0.00569023095313321\\
226	0.00569009941653639\\
227	0.00568996541657927\\
228	0.0056898289067575\\
229	0.00568968983968178\\
230	0.00568954816706096\\
231	0.00568940383968486\\
232	0.00568925680740679\\
233	0.00568910701912575\\
234	0.0056889544227682\\
235	0.00568879896526974\\
236	0.00568864059255603\\
237	0.00568847924952387\\
238	0.00568831488002161\\
239	0.00568814742682924\\
240	0.00568797683163817\\
241	0.00568780303503076\\
242	0.00568762597645927\\
243	0.00568744559422459\\
244	0.00568726182545449\\
245	0.00568707460608165\\
246	0.00568688387082107\\
247	0.00568668955314736\\
248	0.00568649158527144\\
249	0.00568628989811687\\
250	0.00568608442129592\\
251	0.00568587508308502\\
252	0.00568566181040001\\
253	0.00568544452877082\\
254	0.00568522316231589\\
255	0.00568499763371597\\
256	0.00568476786418776\\
257	0.00568453377345691\\
258	0.00568429527973073\\
259	0.00568405229967045\\
260	0.005683804748363\\
261	0.00568355253929249\\
262	0.00568329558431115\\
263	0.00568303379360987\\
264	0.00568276707568848\\
265	0.00568249533732535\\
266	0.00568221848354685\\
267	0.00568193641759619\\
268	0.00568164904090193\\
269	0.00568135625304622\\
270	0.00568105795173245\\
271	0.00568075403275266\\
272	0.00568044438995444\\
273	0.00568012891520762\\
274	0.00567980749837057\\
275	0.00567948002725608\\
276	0.00567914638759691\\
277	0.00567880646301115\\
278	0.0056784601349672\\
279	0.00567810728274838\\
280	0.00567774778341744\\
281	0.00567738151178067\\
282	0.00567700834035196\\
283	0.00567662813931642\\
284	0.00567624077649389\\
285	0.00567584611730247\\
286	0.00567544402472168\\
287	0.00567503435925548\\
288	0.00567461697889537\\
289	0.00567419173908331\\
290	0.00567375849267459\\
291	0.00567331708990063\\
292	0.00567286737833181\\
293	0.00567240920284025\\
294	0.00567194240556276\\
295	0.0056714668258636\\
296	0.00567098230029739\\
297	0.00567048866257205\\
298	0.0056699857435118\\
299	0.00566947337102005\\
300	0.00566895137004241\\
301	0.00566841956252966\\
302	0.00566787776740054\\
303	0.00566732580050442\\
304	0.00566676347458381\\
305	0.00566619059923646\\
306	0.00566560698087707\\
307	0.00566501242269832\\
308	0.00566440672463118\\
309	0.00566378968330431\\
310	0.00566316109200209\\
311	0.00566252074062121\\
312	0.00566186841562555\\
313	0.00566120389999871\\
314	0.00566052697319397\\
315	0.00565983741108122\\
316	0.00565913498589039\\
317	0.00565841946615058\\
318	0.0056576906166246\\
319	0.00565694819823791\\
320	0.0056561919680016\\
321	0.00565542167892837\\
322	0.00565463707994107\\
323	0.00565383791577251\\
324	0.00565302392685632\\
325	0.0056521948492077\\
326	0.0056513504142936\\
327	0.00565049034889171\\
328	0.00564961437493762\\
329	0.00564872220936016\\
330	0.00564781356390471\\
331	0.00564688814494491\\
332	0.00564594565328337\\
333	0.00564498578394269\\
334	0.00564400822594817\\
335	0.0056430126621049\\
336	0.00564199876877183\\
337	0.00564096621563662\\
338	0.00563991466549547\\
339	0.00563884377404258\\
340	0.00563775318967487\\
341	0.00563664255331682\\
342	0.00563551149827092\\
343	0.00563435965009775\\
344	0.00563318662652773\\
345	0.00563199203740529\\
346	0.00563077548466425\\
347	0.00562953656232888\\
348	0.00562827485646794\\
349	0.0056269899449971\\
350	0.00562568139747634\\
351	0.00562434877490262\\
352	0.00562299162949859\\
353	0.00562160950449825\\
354	0.00562020193392963\\
355	0.00561876844239685\\
356	0.00561730854485759\\
357	0.00561582174639557\\
358	0.00561430754198841\\
359	0.00561276541627053\\
360	0.00561119484329123\\
361	0.0056095952862682\\
362	0.00560796619733573\\
363	0.00560630701728851\\
364	0.00560461717531995\\
365	0.0056028960887555\\
366	0.00560114316278022\\
367	0.00559935779016027\\
368	0.00559753935095776\\
369	0.00559568721223796\\
370	0.00559380072776782\\
371	0.00559187923770438\\
372	0.00558992206827112\\
373	0.00558792853142007\\
374	0.0055858979244768\\
375	0.00558382952976479\\
376	0.00558172261420483\\
377	0.00557957642888463\\
378	0.00557739020859249\\
379	0.00557516317130792\\
380	0.0055728945176411\\
381	0.00557058343021178\\
382	0.00556822907295696\\
383	0.00556583059035547\\
384	0.00556338710655636\\
385	0.00556089772439739\\
386	0.00555836152429889\\
387	0.00555577756301842\\
388	0.00555314487225254\\
389	0.00555046245707359\\
390	0.00554772929419316\\
391	0.00554494433004894\\
392	0.00554210647872061\\
393	0.00553921461969174\\
394	0.00553626759549063\\
395	0.00553326420926274\\
396	0.00553020322234368\\
397	0.00552708335189974\\
398	0.00552390326867801\\
399	0.00552066159517655\\
400	0.00551735690908407\\
401	0.00551398775464006\\
402	0.00551055264360586\\
403	0.00550705005641986\\
404	0.0055034784434781\\
405	0.00549983622644924\\
406	0.00549612179949791\\
407	0.00549233353025678\\
408	0.00548846976036909\\
409	0.00548452880543776\\
410	0.00548050895427997\\
411	0.0054764084674702\\
412	0.00547222557518504\\
413	0.00546795847462023\\
414	0.00546360533002913\\
415	0.00545916427439626\\
416	0.00545463341142711\\
417	0.00545001081784471\\
418	0.00544529454596211\\
419	0.00544048262647249\\
420	0.00543557307136288\\
421	0.00543056387681206\\
422	0.00542545302588308\\
423	0.00542023849078473\\
424	0.00541491823446975\\
425	0.00540949021139422\\
426	0.00540395236713207\\
427	0.00539830263456551\\
428	0.00539253890960939\\
429	0.00538665870230299\\
430	0.0053806589906111\\
431	0.00537453662056314\\
432	0.00536828829750209\\
433	0.00536191057680875\\
434	0.00535539985414691\\
435	0.00534875235528663\\
436	0.00534196412550602\\
437	0.00533503101875868\\
438	0.00532794868679913\\
439	0.00532071256853856\\
440	0.00531331788000771\\
441	0.0053057596054455\\
442	0.00529803249023325\\
443	0.00529013103671679\\
444	0.00528204950459116\\
445	0.00527378191909095\\
446	0.0052653220949441\\
447	0.00525666369996561\\
448	0.00524780043951273\\
449	0.00523874710533507\\
450	0.0052295097265494\\
451	0.00522008402379934\\
452	0.00521046560172387\\
453	0.00520064994707147\\
454	0.00519063242936198\\
455	0.00518040830726314\\
456	0.00516997270762846\\
457	0.00515932061653473\\
458	0.00514844686782151\\
459	0.00513734612837327\\
460	0.00512601287937211\\
461	0.00511444139328001\\
462	0.00510262570720484\\
463	0.00509055960021143\\
464	0.00507823659581142\\
465	0.00506565005223802\\
466	0.00505275168493743\\
467	0.00503952883615359\\
468	0.0050259693454732\\
469	0.00501206034803327\\
470	0.00499778822318492\\
471	0.00498313854034943\\
472	0.00496809600103626\\
473	0.00495264437734229\\
474	0.00493676644710445\\
475	0.00492044392413918\\
476	0.00490365737854072\\
477	0.00488638612266166\\
478	0.0048686080106196\\
479	0.00485029893580517\\
480	0.00483143471993527\\
481	0.00481199084152014\\
482	0.00479194141396989\\
483	0.0047712590106245\\
484	0.00474991445607194\\
485	0.00472787712301889\\
486	0.00470511446591804\\
487	0.00468158835288226\\
488	0.00465723956847642\\
489	0.00463201962023959\\
490	0.00460588570867792\\
491	0.00457879360055529\\
492	0.00455069748720035\\
493	0.00452154966276937\\
494	0.00449130056628367\\
495	0.00445989888669135\\
496	0.00442729174482272\\
497	0.00439342495537988\\
498	0.00435824339718851\\
499	0.00432169162255863\\
500	0.00428371465872289\\
501	0.00424425887742218\\
502	0.00420327309150399\\
503	0.0041607099218163\\
504	0.00411652742063411\\
505	0.00407068892326912\\
506	0.00402316124014736\\
507	0.0039739242721161\\
508	0.00392297554771347\\
509	0.00387033569849626\\
510	0.00381608883188232\\
511	0.00376061562342569\\
512	0.00370408950690089\\
513	0.00364675426522747\\
514	0.00358892504886515\\
515	0.00353098244951766\\
516	0.00347337301332899\\
517	0.00341643716177132\\
518	0.00336169708928125\\
519	0.00331247686871964\\
520	0.00326895186470896\\
521	0.00323100016946385\\
522	0.00319743055944266\\
523	0.00316493960918506\\
524	0.00313309722009586\\
525	0.00310140343419004\\
526	0.0030693856830213\\
527	0.00303692320869621\\
528	0.0030039140125688\\
529	0.00297028466841021\\
530	0.00293599961099384\\
531	0.00290102986355149\\
532	0.00286535158407958\\
533	0.00282894485519852\\
534	0.00279179125478551\\
535	0.00275387253062255\\
536	0.00271517010799752\\
537	0.0026756645130882\\
538	0.00263533482220334\\
539	0.00259415911555293\\
540	0.00255211462561724\\
541	0.0025091689220044\\
542	0.00246523866124301\\
543	0.0024213321608453\\
544	0.00237849945414524\\
545	0.00233702530019103\\
546	0.00229542339739121\\
547	0.00225332552808339\\
548	0.0022106323792316\\
549	0.00216734580381793\\
550	0.00212347159484954\\
551	0.00207902210358837\\
552	0.00203401237075111\\
553	0.00198846002048184\\
554	0.00194238440304103\\
555	0.00189580381914202\\
556	0.00184874813894257\\
557	0.00180125301740058\\
558	0.00175336078424276\\
559	0.00170510302000355\\
560	0.0016583782546357\\
561	0.00161270523959384\\
562	0.0015667741143525\\
563	0.00152056777557901\\
564	0.00147411490229661\\
565	0.00142744608095697\\
566	0.00138059251055549\\
567	0.00133358563653171\\
568	0.00128645670380241\\
569	0.00123955719506237\\
570	0.00119284725443033\\
571	0.0011459663569206\\
572	0.0010989478589404\\
573	0.00105182796727181\\
574	0.00100464551951379\\
575	0.000957441991165962\\
576	0.00091026145801078\\
577	0.000863150498877541\\
578	0.000816158021321173\\
579	0.000769334988589918\\
580	0.000722734021084486\\
581	0.000676408839213676\\
582	0.000630413506897997\\
583	0.00058480142587203\\
584	0.000539624020519884\\
585	0.000494929042229846\\
586	0.000450758414499446\\
587	0.000407145545023681\\
588	0.000364112075908286\\
589	0.000321664201822904\\
590	0.000279789157284613\\
591	0.000238610660212781\\
592	0.000198280712249113\\
593	0.000158978649805849\\
594	0.000120944048332156\\
595	8.45661894165686e-05\\
596	5.06250627329883e-05\\
597	2.09371357655842e-05\\
598	8.08076597025309e-08\\
599	0\\
600	0\\
};
\addplot [color=black!50!mycolor20,solid,forget plot]
  table[row sep=crcr]{%
1	0.00568444103234247\\
2	0.00568443904465649\\
3	0.00568443702037213\\
4	0.00568443495881312\\
5	0.00568443285929066\\
6	0.00568443072110318\\
7	0.00568442854353605\\
8	0.00568442632586139\\
9	0.00568442406733782\\
10	0.0056844217672102\\
11	0.00568441942470935\\
12	0.00568441703905182\\
13	0.00568441460943967\\
14	0.00568441213506002\\
15	0.00568440961508497\\
16	0.00568440704867125\\
17	0.00568440443495993\\
18	0.00568440177307605\\
19	0.00568439906212852\\
20	0.00568439630120954\\
21	0.00568439348939456\\
22	0.00568439062574178\\
23	0.00568438770929195\\
24	0.00568438473906796\\
25	0.00568438171407454\\
26	0.0056843786332979\\
27	0.00568437549570543\\
28	0.00568437230024531\\
29	0.00568436904584612\\
30	0.00568436573141657\\
31	0.00568436235584507\\
32	0.00568435891799926\\
33	0.00568435541672582\\
34	0.00568435185084991\\
35	0.00568434821917489\\
36	0.00568434452048178\\
37	0.0056843407535289\\
38	0.00568433691705157\\
39	0.00568433300976149\\
40	0.00568432903034631\\
41	0.00568432497746933\\
42	0.00568432084976898\\
43	0.00568431664585826\\
44	0.00568431236432436\\
45	0.00568430800372815\\
46	0.00568430356260371\\
47	0.00568429903945777\\
48	0.00568429443276931\\
49	0.00568428974098886\\
50	0.00568428496253812\\
51	0.00568428009580935\\
52	0.00568427513916489\\
53	0.00568427009093649\\
54	0.0056842649494248\\
55	0.0056842597128988\\
56	0.00568425437959518\\
57	0.00568424894771776\\
58	0.00568424341543676\\
59	0.00568423778088833\\
60	0.00568423204217389\\
61	0.00568422619735932\\
62	0.00568422024447445\\
63	0.00568421418151238\\
64	0.00568420800642866\\
65	0.00568420171714072\\
66	0.00568419531152706\\
67	0.00568418878742658\\
68	0.00568418214263785\\
69	0.00568417537491828\\
70	0.00568416848198336\\
71	0.00568416146150593\\
72	0.00568415431111533\\
73	0.00568414702839659\\
74	0.00568413961088956\\
75	0.00568413205608817\\
76	0.00568412436143946\\
77	0.00568411652434267\\
78	0.00568410854214853\\
79	0.00568410041215808\\
80	0.00568409213162198\\
81	0.00568408369773939\\
82	0.00568407510765705\\
83	0.00568406635846833\\
84	0.00568405744721218\\
85	0.00568404837087209\\
86	0.00568403912637512\\
87	0.0056840297105908\\
88	0.00568402012032991\\
89	0.00568401035234369\\
90	0.00568400040332235\\
91	0.00568399026989415\\
92	0.0056839799486242\\
93	0.00568396943601317\\
94	0.00568395872849616\\
95	0.0056839478224414\\
96	0.00568393671414908\\
97	0.00568392539984998\\
98	0.00568391387570415\\
99	0.00568390213779972\\
100	0.0056838901821513\\
101	0.00568387800469877\\
102	0.00568386560130577\\
103	0.00568385296775833\\
104	0.00568384009976335\\
105	0.00568382699294709\\
106	0.00568381364285375\\
107	0.00568380004494379\\
108	0.00568378619459231\\
109	0.00568377208708763\\
110	0.00568375771762943\\
111	0.00568374308132722\\
112	0.00568372817319852\\
113	0.00568371298816723\\
114	0.00568369752106174\\
115	0.00568368176661316\\
116	0.00568366571945346\\
117	0.00568364937411362\\
118	0.00568363272502172\\
119	0.00568361576650092\\
120	0.0056835984927675\\
121	0.00568358089792882\\
122	0.0056835629759813\\
123	0.00568354472080821\\
124	0.0056835261261776\\
125	0.00568350718574006\\
126	0.00568348789302649\\
127	0.00568346824144585\\
128	0.00568344822428281\\
129	0.00568342783469533\\
130	0.0056834070657124\\
131	0.00568338591023135\\
132	0.00568336436101556\\
133	0.00568334241069173\\
134	0.00568332005174738\\
135	0.00568329727652815\\
136	0.0056832740772351\\
137	0.00568325044592194\\
138	0.00568322637449219\\
139	0.00568320185469635\\
140	0.00568317687812887\\
141	0.00568315143622537\\
142	0.00568312552025939\\
143	0.00568309912133935\\
144	0.00568307223040546\\
145	0.00568304483822632\\
146	0.00568301693539584\\
147	0.00568298851232978\\
148	0.00568295955926229\\
149	0.00568293006624245\\
150	0.00568290002313079\\
151	0.00568286941959558\\
152	0.00568283824510917\\
153	0.00568280648894416\\
154	0.00568277414016965\\
155	0.0056827411876473\\
156	0.00568270762002725\\
157	0.00568267342574414\\
158	0.00568263859301285\\
159	0.00568260310982443\\
160	0.00568256696394159\\
161	0.0056825301428944\\
162	0.00568249263397576\\
163	0.00568245442423684\\
164	0.00568241550048239\\
165	0.00568237584926595\\
166	0.00568233545688505\\
167	0.00568229430937611\\
168	0.00568225239250957\\
169	0.00568220969178458\\
170	0.00568216619242386\\
171	0.00568212187936815\\
172	0.00568207673727097\\
173	0.00568203075049283\\
174	0.00568198390309564\\
175	0.00568193617883688\\
176	0.00568188756116361\\
177	0.00568183803320655\\
178	0.00568178757777381\\
179	0.0056817361773446\\
180	0.00568168381406292\\
181	0.00568163046973076\\
182	0.00568157612580173\\
183	0.00568152076337395\\
184	0.00568146436318336\\
185	0.00568140690559643\\
186	0.00568134837060298\\
187	0.00568128873780886\\
188	0.0056812279864283\\
189	0.00568116609527622\\
190	0.00568110304276053\\
191	0.0056810388068739\\
192	0.00568097336518573\\
193	0.00568090669483381\\
194	0.00568083877251577\\
195	0.00568076957448035\\
196	0.00568069907651864\\
197	0.00568062725395484\\
198	0.00568055408163733\\
199	0.00568047953392896\\
200	0.00568040358469757\\
201	0.00568032620730608\\
202	0.0056802473746026\\
203	0.00568016705891007\\
204	0.00568008523201585\\
205	0.00568000186516103\\
206	0.00567991692902969\\
207	0.00567983039373758\\
208	0.00567974222882093\\
209	0.00567965240322474\\
210	0.00567956088529112\\
211	0.00567946764274707\\
212	0.00567937264269216\\
213	0.00567927585158602\\
214	0.00567917723523539\\
215	0.00567907675878116\\
216	0.00567897438668473\\
217	0.00567887008271447\\
218	0.00567876380993172\\
219	0.00567865553067654\\
220	0.00567854520655303\\
221	0.00567843279841456\\
222	0.00567831826634846\\
223	0.0056782015696605\\
224	0.00567808266685904\\
225	0.00567796151563892\\
226	0.00567783807286461\\
227	0.00567771229455362\\
228	0.00567758413585906\\
229	0.00567745355105194\\
230	0.00567732049350322\\
231	0.00567718491566544\\
232	0.0056770467690537\\
233	0.00567690600422658\\
234	0.00567676257076642\\
235	0.0056766164172592\\
236	0.00567646749127418\\
237	0.00567631573934277\\
238	0.00567616110693711\\
239	0.00567600353844826\\
240	0.00567584297716376\\
241	0.00567567936524479\\
242	0.00567551264370272\\
243	0.0056753427523754\\
244	0.00567516962990253\\
245	0.00567499321370091\\
246	0.0056748134399388\\
247	0.00567463024350998\\
248	0.00567444355800704\\
249	0.00567425331569418\\
250	0.00567405944747938\\
251	0.00567386188288602\\
252	0.00567366055002384\\
253	0.00567345537555913\\
254	0.00567324628468448\\
255	0.00567303320108777\\
256	0.00567281604692042\\
257	0.0056725947427651\\
258	0.00567236920760255\\
259	0.00567213935877774\\
260	0.00567190511196546\\
261	0.0056716663811349\\
262	0.00567142307851362\\
263	0.00567117511455077\\
264	0.00567092239787941\\
265	0.00567066483527808\\
266	0.00567040233163161\\
267	0.00567013478989106\\
268	0.0056698621110328\\
269	0.00566958419401677\\
270	0.00566930093574391\\
271	0.00566901223101262\\
272	0.00566871797247453\\
273	0.0056684180505893\\
274	0.00566811235357841\\
275	0.00566780076737826\\
276	0.00566748317559234\\
277	0.00566715945944245\\
278	0.00566682949771909\\
279	0.00566649316673095\\
280	0.00566615034025352\\
281	0.005665800889477\\
282	0.00566544468295312\\
283	0.00566508158654144\\
284	0.00566471146335455\\
285	0.00566433417370281\\
286	0.00566394957503826\\
287	0.00566355752189777\\
288	0.00566315786584571\\
289	0.00566275045541605\\
290	0.0056623351360538\\
291	0.00566191175005636\\
292	0.00566148013651421\\
293	0.00566104013125161\\
294	0.00566059156676692\\
295	0.00566013427217326\\
296	0.00565966807313918\\
297	0.00565919279182944\\
298	0.00565870824684634\\
299	0.005658214253172\\
300	0.00565771062211094\\
301	0.00565719716123407\\
302	0.00565667367432387\\
303	0.005656139961321\\
304	0.00565559581827282\\
305	0.00565504103728382\\
306	0.00565447540646855\\
307	0.00565389870990709\\
308	0.0056533107276036\\
309	0.00565271123544789\\
310	0.0056521000051809\\
311	0.00565147680436381\\
312	0.00565084139635151\\
313	0.00565019354027044\\
314	0.00564953299100106\\
315	0.00564885949916517\\
316	0.0056481728111182\\
317	0.00564747266894625\\
318	0.00564675881046804\\
319	0.00564603096924131\\
320	0.00564528887457307\\
321	0.00564453225153317\\
322	0.00564376082097\\
323	0.00564297429952708\\
324	0.00564217239965872\\
325	0.00564135482964245\\
326	0.0056405212935854\\
327	0.00563967149142135\\
328	0.00563880511889436\\
329	0.00563792186752386\\
330	0.00563702142454547\\
331	0.00563610347282151\\
332	0.005635167690713\\
333	0.00563421375190499\\
334	0.00563324132517632\\
335	0.00563225007410346\\
336	0.00563123965668857\\
337	0.00563020972490113\\
338	0.00562915992412395\\
339	0.00562808989249515\\
340	0.00562699926014044\\
341	0.00562588764829433\\
342	0.00562475466831544\\
343	0.00562359992060705\\
344	0.00562242299346295\\
345	0.0056212234618562\\
346	0.00562000088618006\\
347	0.00561875481106686\\
348	0.00561748476637954\\
349	0.00561619027135431\\
350	0.00561487083429502\\
351	0.00561352595225677\\
352	0.0056121551107195\\
353	0.00561075778325032\\
354	0.00560933343115497\\
355	0.00560788150311803\\
356	0.00560640143483301\\
357	0.00560489264862845\\
358	0.0056033545530858\\
359	0.00560178654264964\\
360	0.00560018799723099\\
361	0.00559855828180517\\
362	0.00559689674600589\\
363	0.00559520272371706\\
364	0.00559347553266512\\
365	0.00559171447401408\\
366	0.00558991883196725\\
367	0.00558808787337906\\
368	0.00558622084738174\\
369	0.00558431698503228\\
370	0.00558237549898594\\
371	0.00558039558320338\\
372	0.00557837641269998\\
373	0.00557631714334662\\
374	0.00557421691173296\\
375	0.00557207483510496\\
376	0.00556989001139114\\
377	0.00556766151933183\\
378	0.00556538841872882\\
379	0.00556306975083356\\
380	0.00556070453889345\\
381	0.00555829178887707\\
382	0.00555583049040014\\
383	0.00555331961787386\\
384	0.00555075813189703\\
385	0.00554814498091099\\
386	0.00554547910313357\\
387	0.00554275942878227\\
388	0.00553998488258836\\
389	0.00553715438659163\\
390	0.00553426686318849\\
391	0.00553132123838425\\
392	0.00552831644517221\\
393	0.00552525142693057\\
394	0.0055221251406986\\
395	0.00551893656018128\\
396	0.00551568467835394\\
397	0.00551236850951086\\
398	0.00550898708969569\\
399	0.00550553946747901\\
400	0.00550202455094176\\
401	0.00549844087391128\\
402	0.00549478692530409\\
403	0.00549106114634814\\
404	0.00548726192750512\\
405	0.00548338760506759\\
406	0.00547943645741136\\
407	0.00547540670089354\\
408	0.0054712964854027\\
409	0.00546710388958458\\
410	0.00546282691576481\\
411	0.00545846348450469\\
412	0.00545401142841594\\
413	0.00544946848539916\\
414	0.00544483229221008\\
415	0.00544010037783863\\
416	0.00543527015669191\\
417	0.00543033892171178\\
418	0.00542530383764883\\
419	0.00542016193475662\\
420	0.00541491010327944\\
421	0.00540954508928007\\
422	0.00540406349256064\\
423	0.00539846176763133\\
424	0.00539273622972844\\
425	0.00538688307039046\\
426	0.00538089839743451\\
427	0.00537477834529986\\
428	0.00536852902987551\\
429	0.0053621609364736\\
430	0.00535567188712797\\
431	0.005349059679443\\
432	0.00534232208837259\\
433	0.00533545686765792\\
434	0.00532846175120648\\
435	0.00532133445737956\\
436	0.00531407269158525\\
437	0.00530667414879292\\
438	0.00529913651578357\\
439	0.00529145747287634\\
440	0.00528363469477699\\
441	0.00527566585007969\\
442	0.00526754859883314\\
443	0.0052592805875116\\
444	0.00525085944089303\\
445	0.00524228275129004\\
446	0.00523354806888744\\
447	0.005224652906959\\
448	0.00521559480355955\\
449	0.00520634896216264\\
450	0.00519689904231722\\
451	0.00518723893763778\\
452	0.0051773622311658\\
453	0.00516726217156772\\
454	0.00515693164707624\\
455	0.00514636315635683\\
456	0.00513554877772357\\
457	0.00512448013590437\\
458	0.00511314836626984\\
459	0.0051015440764003\\
460	0.00508965730458871\\
461	0.00507747747369602\\
462	0.00506499333461363\\
463	0.00505219287905965\\
464	0.00503906315270157\\
465	0.00502558973692776\\
466	0.00501175950554529\\
467	0.00499755869651216\\
468	0.00498297281078586\\
469	0.0049679865705743\\
470	0.00495258387614818\\
471	0.0049367477613386\\
472	0.00492046034791723\\
473	0.00490370279902835\\
474	0.00488645527171168\\
475	0.00486869686815325\\
476	0.00485040558484688\\
477	0.00483155826087716\\
478	0.00481213053143863\\
479	0.00479209679521992\\
480	0.00477143001664733\\
481	0.0047501014831728\\
482	0.00472808107660682\\
483	0.00470533681905085\\
484	0.00468183121988498\\
485	0.00465750570097574\\
486	0.0046323129687965\\
487	0.00460621131765535\\
488	0.00457915781331204\\
489	0.00455110799259425\\
490	0.00452201571945873\\
491	0.00449183329208078\\
492	0.00446051162388397\\
493	0.00442800068609629\\
494	0.00439424995728228\\
495	0.00435920892762554\\
496	0.00432282773709235\\
497	0.00428505796732398\\
498	0.00424585351644294\\
499	0.00420516802168202\\
500	0.00416295526388338\\
501	0.00411917570788622\\
502	0.00407379939478729\\
503	0.0040268097299452\\
504	0.00397828286017052\\
505	0.00392848429396668\\
506	0.00387749259989486\\
507	0.00382542261869115\\
508	0.00377243726927146\\
509	0.00371876528257629\\
510	0.00366467812185217\\
511	0.00361023745118332\\
512	0.00355582903224256\\
513	0.00350204198136932\\
514	0.00345000609062976\\
515	0.00340340346040714\\
516	0.00336242419048997\\
517	0.00332691505255038\\
518	0.00329580335689364\\
519	0.00326571975144159\\
520	0.00323623499177207\\
521	0.00320685788429823\\
522	0.00317715730421883\\
523	0.00314702056058131\\
524	0.00311635470536871\\
525	0.00308509641902598\\
526	0.00305321506056708\\
527	0.00302068530454675\\
528	0.00298748655944605\\
529	0.00295360150310148\\
530	0.00291901340561363\\
531	0.00288370566925851\\
532	0.00284766140198404\\
533	0.00281086298373782\\
534	0.00277329170757437\\
535	0.00273492742512477\\
536	0.00269574837248344\\
537	0.00265573192479514\\
538	0.00261485431622473\\
539	0.00257305185018472\\
540	0.00253054513102326\\
541	0.00248904819267114\\
542	0.00244882168813207\\
543	0.00240901116230925\\
544	0.0023687125789523\\
545	0.00232782977281964\\
546	0.00228634329516121\\
547	0.00224425424287554\\
548	0.00220157014813234\\
549	0.00215830087847079\\
550	0.00211445884076905\\
551	0.00207005904691561\\
552	0.00202511926249647\\
553	0.00197965983594131\\
554	0.00193370248963108\\
555	0.00188727179622585\\
556	0.00184040436086965\\
557	0.00179314440647957\\
558	0.00174637704470803\\
559	0.00170135274705942\\
560	0.00165619561418236\\
561	0.00161071974755171\\
562	0.00156494887908256\\
563	0.00151891030881841\\
564	0.00147263228813881\\
565	0.0014261435856498\\
566	0.00137947317581061\\
567	0.00133264988239314\\
568	0.00128606428881874\\
569	0.00123955692295332\\
570	0.0011928472508912\\
571	0.00114596635567785\\
572	0.0010989478583297\\
573	0.00105182796696899\\
574	0.00100464551937024\\
575	0.000957441991101777\\
576	0.000910261457984109\\
577	0.000863150498867171\\
578	0.000816158021317623\\
579	0.000769334988588847\\
580	0.000722734021084216\\
581	0.000676408839213622\\
582	0.000630413506897982\\
583	0.000584801425872013\\
584	0.000539624020519883\\
585	0.000494929042229848\\
586	0.000450758414499452\\
587	0.00040714554502369\\
588	0.000364112075908295\\
589	0.000321664201822912\\
590	0.000279789157284615\\
591	0.00023861066021278\\
592	0.000198280712249112\\
593	0.000158978649805849\\
594	0.000120944048332155\\
595	8.45661894165685e-05\\
596	5.06250627329881e-05\\
597	2.09371357655836e-05\\
598	8.08076597025309e-08\\
599	0\\
600	0\\
};
\addplot [color=black!60!mycolor21,solid,forget plot]
  table[row sep=crcr]{%
1	0.00567809048416249\\
2	0.00567808847818626\\
3	0.00567808643521751\\
4	0.00567808435457156\\
5	0.00567808223555101\\
6	0.00567808007744558\\
7	0.00567807787953175\\
8	0.0056780756410725\\
9	0.0056780733613172\\
10	0.00567807103950118\\
11	0.00567806867484559\\
12	0.00567806626655715\\
13	0.00567806381382766\\
14	0.00567806131583408\\
15	0.00567805877173799\\
16	0.00567805618068532\\
17	0.00567805354180621\\
18	0.00567805085421455\\
19	0.00567804811700768\\
20	0.00567804532926629\\
21	0.00567804249005388\\
22	0.00567803959841653\\
23	0.00567803665338259\\
24	0.00567803365396224\\
25	0.0056780305991473\\
26	0.00567802748791082\\
27	0.00567802431920666\\
28	0.00567802109196935\\
29	0.0056780178051134\\
30	0.00567801445753317\\
31	0.00567801104810247\\
32	0.00567800757567409\\
33	0.00567800403907943\\
34	0.00567800043712817\\
35	0.00567799676860772\\
36	0.00567799303228298\\
37	0.00567798922689576\\
38	0.00567798535116445\\
39	0.00567798140378357\\
40	0.00567797738342325\\
41	0.00567797328872888\\
42	0.00567796911832053\\
43	0.00567796487079252\\
44	0.00567796054471308\\
45	0.00567795613862359\\
46	0.00567795165103829\\
47	0.00567794708044365\\
48	0.00567794242529797\\
49	0.00567793768403077\\
50	0.00567793285504218\\
51	0.00567792793670253\\
52	0.00567792292735173\\
53	0.00567791782529865\\
54	0.00567791262882067\\
55	0.00567790733616293\\
56	0.00567790194553779\\
57	0.00567789645512424\\
58	0.00567789086306731\\
59	0.00567788516747735\\
60	0.00567787936642932\\
61	0.00567787345796232\\
62	0.00567786744007879\\
63	0.00567786131074374\\
64	0.00567785506788428\\
65	0.00567784870938864\\
66	0.00567784223310564\\
67	0.00567783563684392\\
68	0.00567782891837105\\
69	0.00567782207541286\\
70	0.00567781510565272\\
71	0.00567780800673055\\
72	0.00567780077624226\\
73	0.00567779341173862\\
74	0.0056777859107247\\
75	0.00567777827065875\\
76	0.00567777048895156\\
77	0.00567776256296542\\
78	0.00567775449001315\\
79	0.00567774626735738\\
80	0.0056777378922094\\
81	0.00567772936172824\\
82	0.00567772067301976\\
83	0.00567771182313566\\
84	0.00567770280907231\\
85	0.00567769362776987\\
86	0.00567768427611105\\
87	0.00567767475092016\\
88	0.0056776650489621\\
89	0.00567765516694092\\
90	0.00567764510149907\\
91	0.00567763484921584\\
92	0.00567762440660649\\
93	0.0056776137701209\\
94	0.00567760293614227\\
95	0.00567759190098605\\
96	0.00567758066089844\\
97	0.00567756921205521\\
98	0.0056775575505604\\
99	0.00567754567244477\\
100	0.00567753357366472\\
101	0.00567752125010056\\
102	0.00567750869755534\\
103	0.0056774959117532\\
104	0.00567748288833797\\
105	0.00567746962287164\\
106	0.0056774561108327\\
107	0.00567744234761478\\
108	0.00567742832852482\\
109	0.00567741404878153\\
110	0.00567739950351379\\
111	0.0056773846877587\\
112	0.00567736959646014\\
113	0.00567735422446682\\
114	0.00567733856653047\\
115	0.00567732261730413\\
116	0.00567730637134011\\
117	0.0056772898230882\\
118	0.00567727296689364\\
119	0.00567725579699519\\
120	0.0056772383075231\\
121	0.00567722049249705\\
122	0.00567720234582396\\
123	0.005677183861296\\
124	0.00567716503258824\\
125	0.00567714585325651\\
126	0.00567712631673523\\
127	0.00567710641633486\\
128	0.00567708614523971\\
129	0.00567706549650554\\
130	0.00567704446305695\\
131	0.00567702303768511\\
132	0.00567700121304504\\
133	0.00567697898165303\\
134	0.00567695633588411\\
135	0.00567693326796923\\
136	0.0056769097699925\\
137	0.0056768858338885\\
138	0.00567686145143931\\
139	0.00567683661427165\\
140	0.00567681131385399\\
141	0.00567678554149328\\
142	0.00567675928833205\\
143	0.00567673254534527\\
144	0.00567670530333702\\
145	0.00567667755293732\\
146	0.00567664928459876\\
147	0.00567662048859307\\
148	0.00567659115500764\\
149	0.0056765612737421\\
150	0.00567653083450452\\
151	0.005676499826808\\
152	0.00567646823996653\\
153	0.0056764360630916\\
154	0.00567640328508795\\
155	0.0056763698946498\\
156	0.00567633588025672\\
157	0.00567630123016948\\
158	0.00567626593242594\\
159	0.00567622997483655\\
160	0.00567619334498016\\
161	0.00567615603019945\\
162	0.00567611801759638\\
163	0.00567607929402758\\
164	0.00567603984609955\\
165	0.00567599966016382\\
166	0.00567595872231208\\
167	0.00567591701837112\\
168	0.0056758745338976\\
169	0.00567583125417303\\
170	0.00567578716419814\\
171	0.0056757422486877\\
172	0.00567569649206485\\
173	0.00567564987845532\\
174	0.00567560239168187\\
175	0.00567555401525824\\
176	0.00567550473238324\\
177	0.00567545452593444\\
178	0.00567540337846204\\
179	0.00567535127218244\\
180	0.00567529818897171\\
181	0.00567524411035906\\
182	0.00567518901751984\\
183	0.0056751328912688\\
184	0.00567507571205287\\
185	0.00567501745994408\\
186	0.00567495811463215\\
187	0.00567489765541692\\
188	0.00567483606120076\\
189	0.00567477331048082\\
190	0.00567470938134086\\
191	0.00567464425144328\\
192	0.0056745778980207\\
193	0.00567451029786751\\
194	0.0056744414273312\\
195	0.00567437126230355\\
196	0.00567429977821153\\
197	0.00567422695000824\\
198	0.00567415275216329\\
199	0.00567407715865331\\
200	0.00567400014295222\\
201	0.00567392167802108\\
202	0.00567384173629802\\
203	0.00567376028968755\\
204	0.00567367730955023\\
205	0.00567359276669154\\
206	0.00567350663135088\\
207	0.00567341887319018\\
208	0.00567332946128231\\
209	0.00567323836409926\\
210	0.00567314554950004\\
211	0.00567305098471828\\
212	0.00567295463634966\\
213	0.00567285647033903\\
214	0.00567275645196722\\
215	0.00567265454583749\\
216	0.00567255071586188\\
217	0.00567244492524723\\
218	0.0056723371364806\\
219	0.00567222731131477\\
220	0.00567211541075321\\
221	0.00567200139503466\\
222	0.0056718852236176\\
223	0.0056717668551641\\
224	0.00567164624752358\\
225	0.00567152335771585\\
226	0.00567139814191431\\
227	0.00567127055542813\\
228	0.00567114055268452\\
229	0.00567100808721042\\
230	0.00567087311161375\\
231	0.00567073557756419\\
232	0.00567059543577372\\
233	0.00567045263597654\\
234	0.00567030712690856\\
235	0.00567015885628637\\
236	0.00567000777078586\\
237	0.00566985381602016\\
238	0.00566969693651731\\
239	0.00566953707569702\\
240	0.00566937417584741\\
241	0.00566920817810054\\
242	0.00566903902240813\\
243	0.00566886664751596\\
244	0.00566869099093807\\
245	0.00566851198893027\\
246	0.00566832957646301\\
247	0.00566814368719355\\
248	0.00566795425343734\\
249	0.005667761206139\\
250	0.00566756447484228\\
251	0.00566736398765934\\
252	0.00566715967123946\\
253	0.00566695145073673\\
254	0.00566673924977695\\
255	0.00566652299042386\\
256	0.00566630259314436\\
257	0.00566607797677276\\
258	0.00566584905847437\\
259	0.00566561575370796\\
260	0.00566537797618719\\
261	0.00566513563784113\\
262	0.00566488864877381\\
263	0.00566463691722248\\
264	0.00566438034951494\\
265	0.00566411885002569\\
266	0.00566385232113074\\
267	0.00566358066316134\\
268	0.00566330377435651\\
269	0.005663021550814\\
270	0.00566273388644018\\
271	0.00566244067289831\\
272	0.00566214179955547\\
273	0.00566183715342793\\
274	0.00566152661912502\\
275	0.00566121007879132\\
276	0.00566088741204725\\
277	0.00566055849592793\\
278	0.00566022320482017\\
279	0.0056598814103977\\
280	0.0056595329815545\\
281	0.0056591777843361\\
282	0.00565881568186879\\
283	0.00565844653428678\\
284	0.0056580701986572\\
285	0.00565768652890283\\
286	0.00565729537572228\\
287	0.00565689658650815\\
288	0.0056564900052623\\
289	0.00565607547250887\\
290	0.00565565282520443\\
291	0.00565522189664553\\
292	0.0056547825163735\\
293	0.00565433451007637\\
294	0.00565387769948781\\
295	0.00565341190228322\\
296	0.00565293693197273\\
297	0.00565245259779122\\
298	0.00565195870458541\\
299	0.00565145505269756\\
300	0.00565094143784657\\
301	0.00565041765100585\\
302	0.00564988347827837\\
303	0.00564933870076909\\
304	0.00564878309445471\\
305	0.00564821643005098\\
306	0.0056476384728783\\
307	0.00564704898272545\\
308	0.00564644771371206\\
309	0.00564583441415089\\
310	0.00564520882640976\\
311	0.00564457068677489\\
312	0.00564391972531599\\
313	0.00564325566575467\\
314	0.00564257822533775\\
315	0.00564188711471682\\
316	0.00564118203783627\\
317	0.00564046269183229\\
318	0.00563972876694522\\
319	0.00563897994644876\\
320	0.00563821590659928\\
321	0.00563743631661002\\
322	0.00563664083865395\\
323	0.00563582912790186\\
324	0.0056350008326004\\
325	0.00563415559419769\\
326	0.00563329304752368\\
327	0.00563241282103339\\
328	0.00563151453712141\\
329	0.00563059781251795\\
330	0.00562966225877564\\
331	0.00562870748285701\\
332	0.00562773308783336\\
333	0.00562673867370369\\
334	0.00562572383834222\\
335	0.00562468817858035\\
336	0.00562363129142568\\
337	0.00562255277541634\\
338	0.00562145223210122\\
339	0.00562032926762856\\
340	0.00561918349441364\\
341	0.0056180145328434\\
342	0.0056168220129636\\
343	0.00561560557608983\\
344	0.00561436487629255\\
345	0.00561309958169459\\
346	0.00561180937513118\\
347	0.00561049395080012\\
348	0.00560915294926798\\
349	0.00560778586316021\\
350	0.00560639217433276\\
351	0.00560497135358197\\
352	0.00560352286034191\\
353	0.00560204614237058\\
354	0.00560054063542638\\
355	0.00559900576292892\\
356	0.00559744093556757\\
357	0.00559584555071847\\
358	0.00559421899197655\\
359	0.00559256062866768\\
360	0.005590869815326\\
361	0.00558914589113282\\
362	0.00558738817931278\\
363	0.00558559598648302\\
364	0.00558376860195012\\
365	0.00558190529694916\\
366	0.00558000532381854\\
367	0.00557806791510403\\
368	0.00557609228258414\\
369	0.0055740776162084\\
370	0.00557202308293975\\
371	0.00556992782549083\\
372	0.00556779096094347\\
373	0.00556561157923983\\
374	0.00556338874153293\\
375	0.00556112147838361\\
376	0.00555880878779034\\
377	0.00555644963303835\\
378	0.00555404294035469\\
379	0.0055515875963557\\
380	0.00554908244527632\\
381	0.00554652628597116\\
382	0.00554391786868255\\
383	0.00554125589157481\\
384	0.00553853899704262\\
385	0.0055357657678114\\
386	0.00553293472286253\\
387	0.0055300443132351\\
388	0.0055270929177823\\
389	0.00552407883899452\\
390	0.00552100029904639\\
391	0.00551785543628542\\
392	0.00551464230246388\\
393	0.00551135886114191\\
394	0.00550800298791872\\
395	0.00550457247366501\\
396	0.00550106503339204\\
397	0.0054974783281557\\
398	0.00549381002423004\\
399	0.0054900619116264\\
400	0.00548624290706115\\
401	0.00548235178080679\\
402	0.00547838729222659\\
403	0.00547434819084873\\
404	0.00547023321757309\\
405	0.00546604110598613\\
406	0.00546177058371816\\
407	0.00545742037371528\\
408	0.00545298919527317\\
409	0.00544847576500569\\
410	0.00544387879989337\\
411	0.00543919703292147\\
412	0.00543442922783031\\
413	0.00542957416440631\\
414	0.00542463064042559\\
415	0.00541959747692688\\
416	0.00541447352372659\\
417	0.00540925766394877\\
418	0.00540394881842779\\
419	0.00539854594953335\\
420	0.00539304806304162\\
421	0.00538745420769764\\
422	0.0053817634782857\\
423	0.00537597501827313\\
424	0.00537008803130529\\
425	0.00536410176360931\\
426	0.00535801550874562\\
427	0.00535182864296712\\
428	0.00534553015850132\\
429	0.00533910417020067\\
430	0.00533254773533086\\
431	0.00532585780089236\\
432	0.00531903119529265\\
433	0.00531206461918052\\
434	0.00530495463540126\\
435	0.00529769765781424\\
436	0.00529028993896503\\
437	0.00528272755651817\\
438	0.00527500639836222\\
439	0.0052671221463079\\
440	0.00525907025831589\\
441	0.00525084594920351\\
442	0.00524244416976144\\
443	0.00523385958405063\\
444	0.00522508654396468\\
445	0.00521611905762937\\
446	0.0052069507394992\\
447	0.00519757470048298\\
448	0.00518798323742698\\
449	0.0051781691748136\\
450	0.00516812557349384\\
451	0.00515784516556752\\
452	0.00514732033495743\\
453	0.00513654309696886\\
454	0.00512550507682469\\
455	0.00511419748721904\\
456	0.00510261110486613\\
457	0.00509073624605158\\
458	0.00507856274118473\\
459	0.00506607990831575\\
460	0.0050532765254909\\
461	0.00504014080165384\\
462	0.00502666034569408\\
463	0.00501282213425222\\
464	0.00499861248614533\\
465	0.00498401708559613\\
466	0.00496902086120803\\
467	0.00495360794110946\\
468	0.00493776160917825\\
469	0.00492146425996337\\
470	0.00490469735248744\\
471	0.0048874413631116\\
472	0.00486967573754343\\
473	0.00485137884177827\\
474	0.00483252791108766\\
475	0.00481309899538921\\
476	0.00479306688816326\\
477	0.00477240499050642\\
478	0.00475108511318944\\
479	0.00472907765574042\\
480	0.0047063512219255\\
481	0.00468286914969201\\
482	0.00465857457636931\\
483	0.00463342029941863\\
484	0.00460736573769077\\
485	0.0045803692656905\\
486	0.00455238797420791\\
487	0.00452337781077926\\
488	0.00449329378158706\\
489	0.00446209022754585\\
490	0.0044297211825423\\
491	0.00439614081044276\\
492	0.00436130378973641\\
493	0.00432516006707867\\
494	0.0042876590784513\\
495	0.00424875322804588\\
496	0.00420839976663155\\
497	0.0041665633379974\\
498	0.0041233495479658\\
499	0.00407892368507342\\
500	0.00403331532419861\\
501	0.00398657515722307\\
502	0.00393878055884966\\
503	0.0038900434019213\\
504	0.00384044169523914\\
505	0.0037899297860126\\
506	0.00373874864046165\\
507	0.00368721036981556\\
508	0.00363568060361862\\
509	0.0035846766944018\\
510	0.00353502545977107\\
511	0.00349026045693479\\
512	0.0034510383259538\\
513	0.00341727840339429\\
514	0.00338828080539846\\
515	0.0033603034033791\\
516	0.00333292637916966\\
517	0.0033056712146312\\
518	0.00327810821335681\\
519	0.00325013026032012\\
520	0.00322165046021432\\
521	0.00319261139597405\\
522	0.0031629851196311\\
523	0.0031327487188051\\
524	0.00310188361110334\\
525	0.00307037393404531\\
526	0.00303820426245424\\
527	0.00300535925395625\\
528	0.00297182326805869\\
529	0.00293758000443515\\
530	0.00290261228236942\\
531	0.00286690179634396\\
532	0.00283042880564908\\
533	0.00279317178368665\\
534	0.00275510776251015\\
535	0.00271621266535096\\
536	0.00267645224345698\\
537	0.00263574050227267\\
538	0.00259540589348668\\
539	0.00255623288231965\\
540	0.00251818655739537\\
541	0.00247967263649538\\
542	0.00244060753909096\\
543	0.00240093951291325\\
544	0.0023606660956766\\
545	0.00231979031813736\\
546	0.00227831783493422\\
547	0.00223625634576029\\
548	0.00219361564034022\\
549	0.00215040784764357\\
550	0.00210664770433439\\
551	0.00206235281730334\\
552	0.00201754381170918\\
553	0.00197224405722959\\
554	0.00192647818120919\\
555	0.00188027825721591\\
556	0.00183367925424745\\
557	0.00178845238246118\\
558	0.0017441151248563\\
559	0.00169943105413529\\
560	0.00165441047217305\\
561	0.00160907742738367\\
562	0.00156345722066659\\
563	0.00151757597629017\\
564	0.00147146027151485\\
565	0.00142513684443959\\
566	0.00137863232885305\\
567	0.00133234062049373\\
568	0.00128606426865085\\
569	0.0012395569225019\\
570	0.00119284725070826\\
571	0.00114596635558828\\
572	0.00109894785828638\\
573	0.00105182796694908\\
574	0.00100464551936143\\
575	0.000957441991098222\\
576	0.000910261457982793\\
577	0.000863150498866738\\
578	0.000816158021317497\\
579	0.000769334988588809\\
580	0.000722734021084213\\
581	0.000676408839213621\\
582	0.00063041350689799\\
583	0.00058480142587203\\
584	0.000539624020519884\\
585	0.000494929042229849\\
586	0.000450758414499452\\
587	0.000407145545023688\\
588	0.000364112075908291\\
589	0.000321664201822907\\
590	0.000279789157284617\\
591	0.000238610660212778\\
592	0.000198280712249112\\
593	0.000158978649805849\\
594	0.000120944048332156\\
595	8.45661894165684e-05\\
596	5.06250627329887e-05\\
597	2.09371357655837e-05\\
598	8.08076597025309e-08\\
599	0\\
600	0\\
};
\addplot [color=black!80!mycolor21,solid,forget plot]
  table[row sep=crcr]{%
1	0.0056726563473615\\
2	0.0056726541916737\\
3	0.00567265199625278\\
4	0.00567264976036399\\
5	0.00567264748325878\\
6	0.00567264516417476\\
7	0.00567264280233539\\
8	0.00567264039694974\\
9	0.00567263794721211\\
10	0.00567263545230193\\
11	0.00567263291138331\\
12	0.00567263032360487\\
13	0.00567262768809941\\
14	0.00567262500398364\\
15	0.00567262227035781\\
16	0.00567261948630549\\
17	0.00567261665089313\\
18	0.00567261376317001\\
19	0.0056726108221676\\
20	0.00567260782689944\\
21	0.00567260477636065\\
22	0.00567260166952771\\
23	0.00567259850535805\\
24	0.00567259528278981\\
25	0.00567259200074128\\
26	0.00567258865811068\\
27	0.00567258525377574\\
28	0.00567258178659324\\
29	0.00567257825539882\\
30	0.0056725746590064\\
31	0.00567257099620774\\
32	0.00567256726577234\\
33	0.00567256346644657\\
34	0.0056725595969536\\
35	0.00567255565599282\\
36	0.00567255164223939\\
37	0.00567254755434385\\
38	0.00567254339093157\\
39	0.00567253915060233\\
40	0.0056725348319299\\
41	0.00567253043346139\\
42	0.00567252595371698\\
43	0.00567252139118922\\
44	0.00567251674434254\\
45	0.00567251201161285\\
46	0.0056725071914069\\
47	0.00567250228210176\\
48	0.00567249728204424\\
49	0.00567249218955027\\
50	0.00567248700290462\\
51	0.00567248172035989\\
52	0.00567247634013613\\
53	0.0056724708604203\\
54	0.00567246527936552\\
55	0.00567245959509052\\
56	0.00567245380567888\\
57	0.00567244790917844\\
58	0.0056724419036007\\
59	0.00567243578692005\\
60	0.0056724295570731\\
61	0.00567242321195795\\
62	0.00567241674943351\\
63	0.00567241016731874\\
64	0.00567240346339191\\
65	0.00567239663538981\\
66	0.00567238968100698\\
67	0.00567238259789496\\
68	0.00567237538366147\\
69	0.00567236803586955\\
70	0.00567236055203672\\
71	0.00567235292963427\\
72	0.00567234516608605\\
73	0.00567233725876801\\
74	0.00567232920500696\\
75	0.00567232100207988\\
76	0.00567231264721276\\
77	0.00567230413757989\\
78	0.00567229547030271\\
79	0.00567228664244886\\
80	0.00567227765103116\\
81	0.00567226849300671\\
82	0.00567225916527569\\
83	0.0056722496646802\\
84	0.00567223998800343\\
85	0.00567223013196841\\
86	0.00567222009323692\\
87	0.00567220986840826\\
88	0.00567219945401817\\
89	0.00567218884653755\\
90	0.00567217804237132\\
91	0.0056721670378571\\
92	0.00567215582926401\\
93	0.00567214441279139\\
94	0.00567213278456741\\
95	0.00567212094064777\\
96	0.00567210887701432\\
97	0.00567209658957377\\
98	0.00567208407415609\\
99	0.00567207132651328\\
100	0.00567205834231768\\
101	0.00567204511716066\\
102	0.00567203164655094\\
103	0.00567201792591319\\
104	0.00567200395058628\\
105	0.00567198971582179\\
106	0.00567197521678236\\
107	0.00567196044853979\\
108	0.00567194540607374\\
109	0.00567193008426963\\
110	0.00567191447791703\\
111	0.00567189858170783\\
112	0.00567188239023431\\
113	0.00567186589798737\\
114	0.00567184909935462\\
115	0.00567183198861828\\
116	0.00567181455995341\\
117	0.00567179680742571\\
118	0.00567177872498955\\
119	0.00567176030648579\\
120	0.00567174154563972\\
121	0.00567172243605881\\
122	0.00567170297123054\\
123	0.00567168314452005\\
124	0.00567166294916791\\
125	0.00567164237828773\\
126	0.00567162142486369\\
127	0.00567160008174827\\
128	0.00567157834165957\\
129	0.00567155619717886\\
130	0.00567153364074794\\
131	0.00567151066466663\\
132	0.00567148726108983\\
133	0.0056714634220251\\
134	0.00567143913932954\\
135	0.00567141440470721\\
136	0.00567138920970606\\
137	0.00567136354571499\\
138	0.00567133740396096\\
139	0.00567131077550573\\
140	0.00567128365124281\\
141	0.0056712560218944\\
142	0.00567122787800794\\
143	0.00567119920995285\\
144	0.00567117000791715\\
145	0.00567114026190418\\
146	0.0056711099617287\\
147	0.00567107909701377\\
148	0.00567104765718677\\
149	0.00567101563147581\\
150	0.00567098300890594\\
151	0.00567094977829524\\
152	0.005670915928251\\
153	0.00567088144716547\\
154	0.00567084632321205\\
155	0.00567081054434096\\
156	0.00567077409827495\\
157	0.00567073697250507\\
158	0.00567069915428622\\
159	0.00567066063063264\\
160	0.00567062138831334\\
161	0.00567058141384729\\
162	0.00567054069349894\\
163	0.00567049921327297\\
164	0.00567045695890969\\
165	0.00567041391587978\\
166	0.00567037006937919\\
167	0.00567032540432386\\
168	0.00567027990534443\\
169	0.00567023355678068\\
170	0.00567018634267612\\
171	0.00567013824677215\\
172	0.00567008925250232\\
173	0.00567003934298654\\
174	0.00566998850102491\\
175	0.00566993670909169\\
176	0.00566988394932892\\
177	0.00566983020354021\\
178	0.0056697754531842\\
179	0.00566971967936781\\
180	0.00566966286283964\\
181	0.00566960498398291\\
182	0.00566954602280857\\
183	0.00566948595894806\\
184	0.00566942477164609\\
185	0.00566936243975301\\
186	0.00566929894171743\\
187	0.00566923425557824\\
188	0.00566916835895695\\
189	0.0056691012290493\\
190	0.0056690328426174\\
191	0.00566896317598113\\
192	0.00566889220500958\\
193	0.00566881990511236\\
194	0.00566874625123074\\
195	0.00566867121782855\\
196	0.00566859477888292\\
197	0.00566851690787473\\
198	0.00566843757777909\\
199	0.00566835676105558\\
200	0.005668274429638\\
201	0.00566819055492432\\
202	0.00566810510776611\\
203	0.00566801805845803\\
204	0.00566792937672686\\
205	0.00566783903172049\\
206	0.00566774699199642\\
207	0.00566765322551052\\
208	0.00566755769960502\\
209	0.00566746038099647\\
210	0.00566736123576365\\
211	0.00566726022933496\\
212	0.00566715732647555\\
213	0.00566705249127439\\
214	0.00566694568713103\\
215	0.00566683687674177\\
216	0.00566672602208604\\
217	0.00566661308441208\\
218	0.00566649802422252\\
219	0.0056663808012597\\
220	0.00566626137449055\\
221	0.00566613970209113\\
222	0.00566601574143113\\
223	0.00566588944905758\\
224	0.00566576078067866\\
225	0.00566562969114689\\
226	0.0056654961344421\\
227	0.00566536006365393\\
228	0.0056652214309641\\
229	0.00566508018762805\\
230	0.00566493628395654\\
231	0.00566478966929648\\
232	0.00566464029201169\\
233	0.00566448809946292\\
234	0.00566433303798769\\
235	0.00566417505287953\\
236	0.00566401408836683\\
237	0.00566385008759129\\
238	0.00566368299258569\\
239	0.00566351274425138\\
240	0.00566333928233519\\
241	0.00566316254540579\\
242	0.00566298247082947\\
243	0.00566279899474556\\
244	0.00566261205204114\\
245	0.00566242157632529\\
246	0.00566222749990253\\
247	0.00566202975374584\\
248	0.0056618282674691\\
249	0.00566162296929873\\
250	0.0056614137860448\\
251	0.00566120064307136\\
252	0.00566098346426612\\
253	0.00566076217200941\\
254	0.00566053668714245\\
255	0.00566030692893475\\
256	0.00566007281505087\\
257	0.00565983426151621\\
258	0.00565959118268199\\
259	0.00565934349118968\\
260	0.00565909109793399\\
261	0.00565883391202546\\
262	0.00565857184075185\\
263	0.00565830478953848\\
264	0.00565803266190785\\
265	0.00565775535943793\\
266	0.00565747278171955\\
267	0.00565718482631265\\
268	0.00565689138870126\\
269	0.00565659236224741\\
270	0.00565628763814372\\
271	0.00565597710536479\\
272	0.00565566065061714\\
273	0.00565533815828782\\
274	0.0056550095103916\\
275	0.00565467458651652\\
276	0.00565433326376807\\
277	0.00565398541671151\\
278	0.00565363091731262\\
279	0.00565326963487659\\
280	0.00565290143598511\\
281	0.00565252618443128\\
282	0.00565214374115274\\
283	0.00565175396416242\\
284	0.00565135670847695\\
285	0.00565095182604282\\
286	0.0056505391656598\\
287	0.00565011857290181\\
288	0.00564968989003479\\
289	0.00564925295593141\\
290	0.00564880760598278\\
291	0.00564835367200638\\
292	0.00564789098215056\\
293	0.00564741936079453\\
294	0.00564693862844479\\
295	0.00564644860162608\\
296	0.00564594909276809\\
297	0.00564543991008614\\
298	0.00564492085745638\\
299	0.00564439173428446\\
300	0.0056438523353672\\
301	0.0056433024507468\\
302	0.00564274186555663\\
303	0.00564217035985792\\
304	0.00564158770846669\\
305	0.00564099368076945\\
306	0.00564038804052692\\
307	0.00563977054566443\\
308	0.00563914094804761\\
309	0.00563849899324172\\
310	0.00563784442025323\\
311	0.00563717696125093\\
312	0.00563649634126553\\
313	0.00563580227786433\\
314	0.00563509448079877\\
315	0.00563437265162181\\
316	0.00563363648327187\\
317	0.00563288565961959\\
318	0.00563211985497355\\
319	0.0056313387335405\\
320	0.00563054194883551\\
321	0.00562972914303668\\
322	0.00562889994627923\\
323	0.00562805397588277\\
324	0.00562719083550601\\
325	0.00562631011422279\\
326	0.00562541138551255\\
327	0.00562449420615969\\
328	0.00562355811505626\\
329	0.00562260263190279\\
330	0.00562162725580444\\
331	0.00562063146376132\\
332	0.00561961470905441\\
333	0.00561857641953379\\
334	0.00561751599582163\\
335	0.00561643280944935\\
336	0.00561532620096076\\
337	0.00561419547802588\\
338	0.00561303991362954\\
339	0.0056118587444242\\
340	0.00561065116937118\\
341	0.00560941634884803\\
342	0.00560815340449562\\
343	0.00560686142029749\\
344	0.00560553944599839\\
345	0.00560418650597157\\
346	0.00560280162368316\\
347	0.00560138557586317\\
348	0.00559994199060956\\
349	0.00559847033895207\\
350	0.00559697008248249\\
351	0.00559544067319194\\
352	0.00559388155324721\\
353	0.00559229215469127\\
354	0.0055906718991996\\
355	0.00558902019873232\\
356	0.00558733646069755\\
357	0.00558562008347992\\
358	0.00558387045610992\\
359	0.00558208695831402\\
360	0.00558026896059007\\
361	0.00557841582431116\\
362	0.00557652690186134\\
363	0.00557460153680771\\
364	0.0055726390641128\\
365	0.00557063881039292\\
366	0.00556860009422757\\
367	0.00556652222652663\\
368	0.0055644045109616\\
369	0.00556224624446941\\
370	0.00556004671783661\\
371	0.00555780521637345\\
372	0.00555552102068824\\
373	0.00555319340757292\\
374	0.0055508216510114\\
375	0.00554840502332427\\
376	0.00554594279646288\\
377	0.00554343424346726\\
378	0.00554087864010198\\
379	0.00553827526668491\\
380	0.00553562341012239\\
381	0.00553292236616368\\
382	0.00553017144188423\\
383	0.00552736995840452\\
384	0.00552451725384453\\
385	0.0055216126865059\\
386	0.0055186556382616\\
387	0.00551564551811701\\
388	0.00551258176588414\\
389	0.00550946385588055\\
390	0.00550629130052811\\
391	0.00550306365367909\\
392	0.00549978051344604\\
393	0.00549644152427288\\
394	0.00549304637800769\\
395	0.00548959481397086\\
396	0.00548608661883855\\
397	0.00548252162953919\\
398	0.00547889974833801\\
399	0.00547521667394989\\
400	0.00547146051915688\\
401	0.00546762983118114\\
402	0.00546372312411694\\
403	0.00545973887738727\\
404	0.0054556755340311\\
405	0.00545153149879833\\
406	0.00544730513603014\\
407	0.00544299476730986\\
408	0.00543859866889288\\
409	0.00543411506896539\\
410	0.00542954214473845\\
411	0.00542487801858056\\
412	0.00542012075282477\\
413	0.00541526834495148\\
414	0.00541031872232915\\
415	0.00540526973629815\\
416	0.00540011915551897\\
417	0.00539486465858133\\
418	0.00538950382581998\\
419	0.00538403413029789\\
420	0.00537845292800877\\
421	0.00537275744742383\\
422	0.00536694477805421\\
423	0.00536101185751044\\
424	0.0053549554542474\\
425	0.00534877214097269\\
426	0.00534245823468136\\
427	0.00533600962441363\\
428	0.00532942229378362\\
429	0.00532269277555386\\
430	0.00531581747847759\\
431	0.00530879268081551\\
432	0.00530161452350418\\
433	0.00529427900297069\\
434	0.00528678196358629\\
435	0.00527911908976268\\
436	0.00527128589769575\\
437	0.00526327772676467\\
438	0.00525508973059878\\
439	0.00524671686782677\\
440	0.00523815389252323\\
441	0.00522939534435952\\
442	0.00522043553843984\\
443	0.00521126855474228\\
444	0.00520188822696695\\
445	0.00519228813048869\\
446	0.00518246156959665\\
447	0.00517240156821618\\
448	0.00516210088798269\\
449	0.0051515519836599\\
450	0.00514074696114472\\
451	0.00512967755664923\\
452	0.00511833511459614\\
453	0.00510671056416119\\
454	0.005094794394398\\
455	0.00508257662787727\\
456	0.00507004679277483\\
457	0.00505719389334406\\
458	0.00504400637871148\\
459	0.00503047210993742\\
460	0.00501657832529988\\
461	0.00500231160380322\\
462	0.00498765782703268\\
463	0.00497260213967534\\
464	0.00495712890879416\\
465	0.00494122167814308\\
466	0.00492486312459865\\
467	0.00490803501392739\\
468	0.00489071815608642\\
469	0.00487289236029096\\
470	0.00485453638978851\\
471	0.00483562791546907\\
472	0.00481614346554282\\
473	0.00479605836424475\\
474	0.0047753466415582\\
475	0.00475398086471228\\
476	0.00473193198692108\\
477	0.00470916922113812\\
478	0.00468565717276999\\
479	0.0046613430032187\\
480	0.00463617785589152\\
481	0.00461012307364819\\
482	0.00458313954172045\\
483	0.00455518754190648\\
484	0.00452622669940078\\
485	0.00449621615401403\\
486	0.00446511396787019\\
487	0.00443287084319127\\
488	0.00439943742381519\\
489	0.00436476519244143\\
490	0.0043288077222514\\
491	0.00429152248422299\\
492	0.00425307803979011\\
493	0.0042135402803181\\
494	0.00417291229962176\\
495	0.00413120908005444\\
496	0.00408846095841081\\
497	0.00404471867802105\\
498	0.00399992034517147\\
499	0.00395398073568304\\
500	0.00390700799853365\\
501	0.00385915879159777\\
502	0.00381064841806514\\
503	0.00376175499684683\\
504	0.00371282113824819\\
505	0.00366430898399663\\
506	0.00361695287096143\\
507	0.00357343411844418\\
508	0.00353534865560084\\
509	0.003502716927799\\
510	0.00347509849448955\\
511	0.00344896901998315\\
512	0.00342348499350885\\
513	0.00339818049604156\\
514	0.00337259461680241\\
515	0.00334662395062039\\
516	0.00332018477572799\\
517	0.00329322186395743\\
518	0.00326570931003315\\
519	0.00323762603084661\\
520	0.0032089550074286\\
521	0.00317968165844116\\
522	0.00314979181742309\\
523	0.00311927141060491\\
524	0.00308810612467769\\
525	0.00305628111969381\\
526	0.00302378086521883\\
527	0.00299058896066202\\
528	0.00295668793444659\\
529	0.00292205901373555\\
530	0.0028866818439151\\
531	0.00285053418336028\\
532	0.00281359265513241\\
533	0.00277583269080589\\
534	0.00273719698225847\\
535	0.00269791194539105\\
536	0.00265968349370554\\
537	0.00262277702853496\\
538	0.00258599870570809\\
539	0.00254871503697903\\
540	0.00251083467396983\\
541	0.00247235213481757\\
542	0.00243326608140515\\
543	0.00239357866527237\\
544	0.00235329365434179\\
545	0.0023124164573835\\
546	0.0022709542923999\\
547	0.00222891641628381\\
548	0.00218631439690431\\
549	0.00214316242247428\\
550	0.00209947764209676\\
551	0.00205528050502658\\
552	0.00201059499798038\\
553	0.0019654485360922\\
554	0.00191987076606147\\
555	0.00187427807567601\\
556	0.00183043641433233\\
557	0.00178659569603101\\
558	0.00174238353130683\\
559	0.0016978203498482\\
560	0.00165292849411383\\
561	0.00160773137094251\\
562	0.00156225314714374\\
563	0.00151651840132604\\
564	0.00147055184293855\\
565	0.00142437814984038\\
566	0.00137835963627298\\
567	0.00133234061892151\\
568	0.0012860642685907\\
569	0.0012395569224755\\
570	0.00119284725069563\\
571	0.0011459663555821\\
572	0.00109894785828362\\
573	0.00105182796694792\\
574	0.00100464551936097\\
575	0.000957441991098067\\
576	0.000910261457982733\\
577	0.000863150498866722\\
578	0.000816158021317482\\
579	0.000769334988588811\\
580	0.000722734021084211\\
581	0.000676408839213619\\
582	0.000630413506897986\\
583	0.000584801425872026\\
584	0.000539624020519893\\
585	0.000494929042229852\\
586	0.000450758414499455\\
587	0.000407145545023689\\
588	0.000364112075908295\\
589	0.000321664201822913\\
590	0.000279789157284617\\
591	0.00023861066021278\\
592	0.000198280712249114\\
593	0.000158978649805849\\
594	0.000120944048332155\\
595	8.45661894165681e-05\\
596	5.06250627329874e-05\\
597	2.09371357655839e-05\\
598	8.08076597025309e-08\\
599	0\\
600	0\\
};
\addplot [color=black,solid,forget plot]
  table[row sep=crcr]{%
1	0.00566514530114279\\
2	0.0056651429118458\\
3	0.00566514047857598\\
4	0.00566513800052144\\
5	0.00566513547685536\\
6	0.00566513290673557\\
7	0.00566513028930424\\
8	0.00566512762368769\\
9	0.00566512490899606\\
10	0.00566512214432295\\
11	0.00566511932874516\\
12	0.00566511646132242\\
13	0.00566511354109703\\
14	0.00566511056709351\\
15	0.00566510753831833\\
16	0.00566510445375951\\
17	0.00566510131238629\\
18	0.00566509811314881\\
19	0.00566509485497775\\
20	0.00566509153678395\\
21	0.00566508815745808\\
22	0.00566508471587021\\
23	0.00566508121086941\\
24	0.00566507764128344\\
25	0.00566507400591833\\
26	0.00566507030355794\\
27	0.00566506653296354\\
28	0.00566506269287343\\
29	0.00566505878200244\\
30	0.00566505479904164\\
31	0.0056650507426578\\
32	0.00566504661149283\\
33	0.0056650424041636\\
34	0.00566503811926119\\
35	0.00566503375535052\\
36	0.00566502931096985\\
37	0.00566502478463036\\
38	0.00566502017481558\\
39	0.00566501547998082\\
40	0.00566501069855273\\
41	0.00566500582892875\\
42	0.00566500086947651\\
43	0.00566499581853336\\
44	0.00566499067440573\\
45	0.00566498543536866\\
46	0.00566498009966501\\
47	0.00566497466550504\\
48	0.00566496913106575\\
49	0.00566496349449028\\
50	0.00566495775388721\\
51	0.00566495190733003\\
52	0.00566494595285645\\
53	0.00566493988846759\\
54	0.00566493371212739\\
55	0.005664927421762\\
56	0.00566492101525904\\
57	0.00566491449046677\\
58	0.00566490784519346\\
59	0.00566490107720661\\
60	0.00566489418423223\\
61	0.00566488716395397\\
62	0.0056648800140124\\
63	0.00566487273200424\\
64	0.0056648653154814\\
65	0.00566485776195041\\
66	0.00566485006887126\\
67	0.00566484223365663\\
68	0.00566483425367111\\
69	0.00566482612623027\\
70	0.00566481784859969\\
71	0.00566480941799389\\
72	0.00566480083157582\\
73	0.00566479208645537\\
74	0.00566478317968878\\
75	0.00566477410827727\\
76	0.00566476486916643\\
77	0.00566475545924478\\
78	0.00566474587534298\\
79	0.00566473611423273\\
80	0.0056647261726254\\
81	0.00566471604717118\\
82	0.00566470573445782\\
83	0.00566469523100958\\
84	0.00566468453328586\\
85	0.00566467363768014\\
86	0.00566466254051854\\
87	0.0056646512380589\\
88	0.00566463972648918\\
89	0.00566462800192629\\
90	0.00566461606041471\\
91	0.00566460389792527\\
92	0.00566459151035364\\
93	0.00566457889351881\\
94	0.00566456604316193\\
95	0.00566455295494477\\
96	0.00566453962444804\\
97	0.0056645260471701\\
98	0.00566451221852528\\
99	0.00566449813384239\\
100	0.00566448378836304\\
101	0.00566446917724013\\
102	0.00566445429553603\\
103	0.005664439138221\\
104	0.00566442370017141\\
105	0.00566440797616797\\
106	0.00566439196089395\\
107	0.00566437564893347\\
108	0.00566435903476941\\
109	0.00566434211278171\\
110	0.00566432487724527\\
111	0.00566430732232815\\
112	0.00566428944208952\\
113	0.00566427123047758\\
114	0.00566425268132741\\
115	0.00566423378835901\\
116	0.00566421454517496\\
117	0.00566419494525842\\
118	0.00566417498197069\\
119	0.00566415464854912\\
120	0.00566413393810459\\
121	0.00566411284361917\\
122	0.00566409135794391\\
123	0.00566406947379611\\
124	0.00566404718375698\\
125	0.00566402448026907\\
126	0.00566400135563352\\
127	0.00566397780200754\\
128	0.00566395381140172\\
129	0.00566392937567723\\
130	0.00566390448654299\\
131	0.00566387913555286\\
132	0.00566385331410271\\
133	0.00566382701342739\\
134	0.00566380022459783\\
135	0.00566377293851782\\
136	0.005663745145921\\
137	0.00566371683736761\\
138	0.0056636880032411\\
139	0.00566365863374509\\
140	0.00566362871889974\\
141	0.00566359824853836\\
142	0.00566356721230398\\
143	0.00566353559964567\\
144	0.00566350339981507\\
145	0.00566347060186233\\
146	0.0056634371946327\\
147	0.00566340316676241\\
148	0.00566336850667496\\
149	0.00566333320257688\\
150	0.0056632972424538\\
151	0.00566326061406625\\
152	0.00566322330494541\\
153	0.00566318530238891\\
154	0.00566314659345628\\
155	0.00566310716496441\\
156	0.00566306700348328\\
157	0.00566302609533105\\
158	0.00566298442656937\\
159	0.00566294198299865\\
160	0.00566289875015295\\
161	0.00566285471329524\\
162	0.00566280985741194\\
163	0.00566276416720805\\
164	0.00566271762710166\\
165	0.00566267022121861\\
166	0.0056626219333869\\
167	0.00566257274713119\\
168	0.00566252264566699\\
169	0.00566247161189495\\
170	0.00566241962839473\\
171	0.00566236667741915\\
172	0.00566231274088793\\
173	0.00566225780038144\\
174	0.00566220183713419\\
175	0.00566214483202855\\
176	0.00566208676558784\\
177	0.00566202761796977\\
178	0.00566196736895933\\
179	0.00566190599796197\\
180	0.00566184348399633\\
181	0.00566177980568696\\
182	0.00566171494125689\\
183	0.00566164886852002\\
184	0.00566158156487344\\
185	0.00566151300728951\\
186	0.00566144317230786\\
187	0.00566137203602732\\
188	0.00566129957409742\\
189	0.00566122576171006\\
190	0.00566115057359073\\
191	0.00566107398398981\\
192	0.00566099596667354\\
193	0.0056609164949149\\
194	0.00566083554148428\\
195	0.00566075307863993\\
196	0.00566066907811837\\
197	0.00566058351112452\\
198	0.0056604963483215\\
199	0.00566040755982058\\
200	0.00566031711517074\\
201	0.00566022498334777\\
202	0.00566013113274375\\
203	0.0056600355311559\\
204	0.00565993814577523\\
205	0.00565983894317521\\
206	0.0056597378893\\
207	0.00565963494945254\\
208	0.00565953008828242\\
209	0.0056594232697736\\
210	0.00565931445723167\\
211	0.00565920361327103\\
212	0.00565909069980196\\
213	0.00565897567801701\\
214	0.0056588585083776\\
215	0.00565873915060027\\
216	0.00565861756364221\\
217	0.00565849370568714\\
218	0.00565836753413066\\
219	0.00565823900556518\\
220	0.00565810807576468\\
221	0.00565797469966934\\
222	0.0056578388313696\\
223	0.00565770042409017\\
224	0.00565755943017341\\
225	0.00565741580106272\\
226	0.00565726948728548\\
227	0.00565712043843565\\
228	0.00565696860315595\\
229	0.00565681392912004\\
230	0.00565665636301381\\
231	0.00565649585051693\\
232	0.00565633233628342\\
233	0.00565616576392246\\
234	0.00565599607597833\\
235	0.00565582321391026\\
236	0.00565564711807182\\
237	0.00565546772768983\\
238	0.00565528498084298\\
239	0.00565509881444006\\
240	0.0056549091641976\\
241	0.00565471596461722\\
242	0.00565451914896254\\
243	0.00565431864923571\\
244	0.00565411439615327\\
245	0.00565390631912166\\
246	0.00565369434621249\\
247	0.00565347840413699\\
248	0.00565325841822012\\
249	0.00565303431237418\\
250	0.00565280600907201\\
251	0.00565257342931952\\
252	0.00565233649262779\\
253	0.00565209511698456\\
254	0.00565184921882539\\
255	0.00565159871300406\\
256	0.00565134351276229\\
257	0.00565108352969938\\
258	0.0056508186737409\\
259	0.00565054885310676\\
260	0.00565027397427885\\
261	0.00564999394196815\\
262	0.00564970865908098\\
263	0.00564941802668485\\
264	0.00564912194397359\\
265	0.00564882030823191\\
266	0.00564851301479925\\
267	0.00564819995703302\\
268	0.00564788102627128\\
269	0.00564755611179457\\
270	0.00564722510078736\\
271	0.00564688787829855\\
272	0.00564654432720164\\
273	0.00564619432815397\\
274	0.00564583775955547\\
275	0.00564547449750672\\
276	0.00564510441576643\\
277	0.00564472738570817\\
278	0.0056443432762768\\
279	0.00564395195394388\\
280	0.00564355328266304\\
281	0.0056431471238243\\
282	0.00564273333620817\\
283	0.00564231177593961\\
284	0.00564188229644087\\
285	0.00564144474838468\\
286	0.00564099897964686\\
287	0.00564054483525858\\
288	0.00564008215735895\\
289	0.00563961078514741\\
290	0.00563913055483612\\
291	0.005638641299603\\
292	0.00563814284954474\\
293	0.00563763503163103\\
294	0.00563711766965912\\
295	0.0056365905842098\\
296	0.00563605359260467\\
297	0.00563550650886536\\
298	0.00563494914367436\\
299	0.00563438130433877\\
300	0.00563380279475697\\
301	0.00563321341538856\\
302	0.00563261296322892\\
303	0.00563200123178824\\
304	0.00563137801107605\\
305	0.00563074308759293\\
306	0.00563009624432902\\
307	0.00562943726077165\\
308	0.00562876591292289\\
309	0.00562808197332833\\
310	0.00562738521111903\\
311	0.00562667539206863\\
312	0.00562595227866663\\
313	0.00562521563021188\\
314	0.00562446520292717\\
315	0.00562370075009862\\
316	0.00562292202224334\\
317	0.00562212876730795\\
318	0.00562132073090273\\
319	0.00562049765657527\\
320	0.00561965928612828\\
321	0.0056188053599871\\
322	0.00561793561762179\\
323	0.00561704979803023\\
324	0.00561614764028832\\
325	0.00561522888417362\\
326	0.00561429327086952\\
327	0.00561334054375629\\
328	0.00561237044929577\\
329	0.00561138273801525\\
330	0.00561037716559573\\
331	0.00560935349406716\\
332	0.00560831149311188\\
333	0.00560725094147317\\
334	0.00560617162845998\\
335	0.00560507335553368\\
336	0.00560395593795157\\
337	0.00560281920642894\\
338	0.00560166300876562\\
339	0.00560048721135973\\
340	0.00559929170050593\\
341	0.00559807638334732\\
342	0.00559684118834083\\
343	0.0055955860651451\\
344	0.00559431098405639\\
345	0.00559301593671818\\
346	0.00559170094544136\\
347	0.0055903642451844\\
348	0.00558900103640795\\
349	0.00558761079480055\\
350	0.00558619298603905\\
351	0.00558474706562684\\
352	0.00558327247873685\\
353	0.00558176866006991\\
354	0.00558023503374658\\
355	0.00557867101321475\\
356	0.00557707600083142\\
357	0.00557544938765034\\
358	0.00557379055322142\\
359	0.00557209886538367\\
360	0.00557037368005017\\
361	0.0055686143409844\\
362	0.00556682017956609\\
363	0.00556499051454463\\
364	0.00556312465177841\\
365	0.00556122188395712\\
366	0.00555928149030451\\
367	0.0055573027362577\\
368	0.00555528487311924\\
369	0.00555322713767746\\
370	0.00555112875178915\\
371	0.00554898892191865\\
372	0.00554680683862556\\
373	0.0055445816759929\\
374	0.00554231259098597\\
375	0.00553999872273046\\
376	0.00553763919169712\\
377	0.00553523309877796\\
378	0.00553277952423863\\
379	0.00553027752652712\\
380	0.00552772614091831\\
381	0.00552512437797179\\
382	0.00552247122177683\\
383	0.00551976562795693\\
384	0.0055170065214047\\
385	0.00551419279371535\\
386	0.00551132330028724\\
387	0.00550839685705738\\
388	0.00550541223684285\\
389	0.00550236816526288\\
390	0.00549926331622254\\
391	0.00549609630694718\\
392	0.00549286569255693\\
393	0.00548956996013117\\
394	0.00548620752202957\\
395	0.00548277670753128\\
396	0.0054792757493766\\
397	0.00547570275333859\\
398	0.00547205561052172\\
399	0.00546833214774797\\
400	0.00546453067591989\\
401	0.00546064946118188\\
402	0.00545668672318825\\
403	0.00545264063328102\\
404	0.00544850931257244\\
405	0.00544429082993078\\
406	0.00543998319986607\\
407	0.00543558438031584\\
408	0.00543109227032896\\
409	0.00542650470764308\\
410	0.00542181946614035\\
411	0.00541703425320038\\
412	0.00541214670700287\\
413	0.0054071543937554\\
414	0.00540205480485269\\
415	0.00539684535398294\\
416	0.00539152337420209\\
417	0.00538608611499885\\
418	0.00538053073937566\\
419	0.00537485432097505\\
420	0.00536905384127449\\
421	0.00536312618684996\\
422	0.00535706814667875\\
423	0.00535087640937629\\
424	0.0053445475602359\\
425	0.00533807807806117\\
426	0.00533146433382468\\
427	0.0053247026035342\\
428	0.00531778906287525\\
429	0.00531071975405167\\
430	0.00530349057913957\\
431	0.00529609729302756\\
432	0.00528853549591614\\
433	0.00528080062534305\\
434	0.00527288794770122\\
435	0.00526479254921233\\
436	0.00525650932631427\\
437	0.00524803297541887\\
438	0.0052393579819891\\
439	0.00523047860888262\\
440	0.00522138888389997\\
441	0.00521208258646953\\
442	0.00520255323339315\\
443	0.00519279406357184\\
444	0.00518279802164018\\
445	0.00517255774049004\\
446	0.00516206552277139\\
447	0.00515131332135599\\
448	0.00514029271671018\\
449	0.00512899489366288\\
450	0.00511741061770258\\
451	0.00510553020987678\\
452	0.005093343520253\\
453	0.00508083989991126\\
454	0.00506800817145046\\
455	0.00505483659800893\\
456	0.0050413128508221\\
457	0.0050274239753682\\
458	0.00501315635618966\\
459	0.00499849568052301\\
460	0.00498342690092761\\
461	0.0049679341971735\\
462	0.00495200093772706\\
463	0.00493560964123576\\
464	0.00491874193846943\\
465	0.00490137853557169\\
466	0.00488349917922388\\
467	0.00486508262444301\\
468	0.00484610660542807\\
469	0.00482654780868972\\
470	0.00480638184418371\\
471	0.0047855832007658\\
472	0.00476412514922022\\
473	0.0047419795065272\\
474	0.00471911609849913\\
475	0.00469550163461367\\
476	0.00467109313950746\\
477	0.00464583327006395\\
478	0.00461968494429691\\
479	0.00459261050554614\\
480	0.00456457155637613\\
481	0.00453552084182562\\
482	0.00450541053979845\\
483	0.00447419301978174\\
484	0.00444182196468356\\
485	0.00440827633474604\\
486	0.0043737678111907\\
487	0.00433828465314371\\
488	0.00430182109096222\\
489	0.00426437891046319\\
490	0.00422596967849117\\
491	0.00418661865770622\\
492	0.00414614925508811\\
493	0.00410451223513068\\
494	0.00406175251919963\\
495	0.00401793995061502\\
496	0.00397317490198698\\
497	0.00392759402932346\\
498	0.00388138686631831\\
499	0.0038348112092975\\
500	0.00378818505237093\\
501	0.00374188673431115\\
502	0.00369651175684453\\
503	0.00365374498033923\\
504	0.00361627117279784\\
505	0.00358417231330889\\
506	0.00355714814699657\\
507	0.00353260035140886\\
508	0.00350877072942706\\
509	0.00348521146747024\\
510	0.00346144902857779\\
511	0.00343733658539137\\
512	0.00341279094287211\\
513	0.00338775591009389\\
514	0.00336220692150631\\
515	0.00333612408026538\\
516	0.00330949143003606\\
517	0.00328229541491571\\
518	0.00325452289647019\\
519	0.00322616086140653\\
520	0.00319719612935532\\
521	0.0031676151180537\\
522	0.00313740372143682\\
523	0.0031065471811229\\
524	0.0030750299486129\\
525	0.0030428355329436\\
526	0.0030099463193908\\
527	0.00297634335401151\\
528	0.00294200608931303\\
529	0.00290691208330239\\
530	0.002871036616017\\
531	0.00283435195462797\\
532	0.00279678022206653\\
533	0.00275936145840275\\
534	0.00272311746092898\\
535	0.00268800067942794\\
536	0.00265242133634285\\
537	0.00261628720609073\\
538	0.00257955974435842\\
539	0.00254223372303644\\
540	0.00250430867385777\\
541	0.00246578542519781\\
542	0.00242666613763974\\
543	0.0023869543751822\\
544	0.00234665528040646\\
545	0.00230577578279145\\
546	0.00226432484051066\\
547	0.00222231372071924\\
548	0.00217975632582472\\
549	0.00213666957517705\\
550	0.0020930738501438\\
551	0.00204899349063787\\
552	0.00200445722507576\\
553	0.00195949792931471\\
554	0.00191512512573993\\
555	0.00187214468329469\\
556	0.00182878631204091\\
557	0.00178504432423926\\
558	0.00174093814984678\\
559	0.00169648845614493\\
560	0.00165171690631392\\
561	0.00160664588251358\\
562	0.001561298165923\\
563	0.00151569664284587\\
564	0.00146986415558105\\
565	0.00142409720804448\\
566	0.0013783596361442\\
567	0.00133234061891314\\
568	0.00128606426858687\\
569	0.00123955692247367\\
570	0.0011928472506948\\
571	0.00114596635558171\\
572	0.00109894785828343\\
573	0.00105182796694783\\
574	0.00100464551936094\\
575	0.000957441991098043\\
576	0.000910261457982726\\
577	0.000863150498866709\\
578	0.000816158021317482\\
579	0.000769334988588808\\
580	0.000722734021084206\\
581	0.000676408839213619\\
582	0.000630413506897983\\
583	0.000584801425872023\\
584	0.000539624020519883\\
585	0.000494929042229843\\
586	0.000450758414499446\\
587	0.000407145545023686\\
588	0.000364112075908292\\
589	0.000321664201822907\\
590	0.000279789157284612\\
591	0.000238610660212782\\
592	0.000198280712249115\\
593	0.000158978649805849\\
594	0.000120944048332156\\
595	8.45661894165696e-05\\
596	5.06250627329888e-05\\
597	2.09371357655836e-05\\
598	8.08076597025309e-08\\
599	0\\
600	0\\
};
\end{axis}
\end{tikzpicture}%
 
  \caption{Discrete Time}
\end{subfigure}\\
\vspace{1cm}
\begin{subfigure}{.45\linewidth}
  \centering
  \setlength\figureheight{\linewidth} 
  \setlength\figurewidth{\linewidth}
  \tikzsetnextfilename{dm_cts_nFPC_z8}
  % This file was created by matlab2tikz.
%
%The latest updates can be retrieved from
%  http://www.mathworks.com/matlabcentral/fileexchange/22022-matlab2tikz-matlab2tikz
%where you can also make suggestions and rate matlab2tikz.
%
\definecolor{mycolor1}{rgb}{0.00000,1.00000,0.14286}%
\definecolor{mycolor2}{rgb}{0.00000,1.00000,0.28571}%
\definecolor{mycolor3}{rgb}{0.00000,1.00000,0.42857}%
\definecolor{mycolor4}{rgb}{0.00000,1.00000,0.57143}%
\definecolor{mycolor5}{rgb}{0.00000,1.00000,0.71429}%
\definecolor{mycolor6}{rgb}{0.00000,1.00000,0.85714}%
\definecolor{mycolor7}{rgb}{0.00000,1.00000,1.00000}%
\definecolor{mycolor8}{rgb}{0.00000,0.87500,1.00000}%
\definecolor{mycolor9}{rgb}{0.00000,0.62500,1.00000}%
\definecolor{mycolor10}{rgb}{0.12500,0.00000,1.00000}%
\definecolor{mycolor11}{rgb}{0.25000,0.00000,1.00000}%
\definecolor{mycolor12}{rgb}{0.37500,0.00000,1.00000}%
\definecolor{mycolor13}{rgb}{0.50000,0.00000,1.00000}%
\definecolor{mycolor14}{rgb}{0.62500,0.00000,1.00000}%
\definecolor{mycolor15}{rgb}{0.75000,0.00000,1.00000}%
\definecolor{mycolor16}{rgb}{0.87500,0.00000,1.00000}%
\definecolor{mycolor17}{rgb}{1.00000,0.00000,1.00000}%
\definecolor{mycolor18}{rgb}{1.00000,0.00000,0.87500}%
\definecolor{mycolor19}{rgb}{1.00000,0.00000,0.62500}%
\definecolor{mycolor20}{rgb}{0.85714,0.00000,0.00000}%
\definecolor{mycolor21}{rgb}{0.71429,0.00000,0.00000}%
%
\begin{tikzpicture}

\begin{axis}[%
width=4.1in,
height=3.803in,
at={(0.809in,0.513in)},
scale only axis,
point meta min=0,
point meta max=1,
every outer x axis line/.append style={black},
every x tick label/.append style={font=\color{black}},
xmin=0,
xmax=600,
every outer y axis line/.append style={black},
every y tick label/.append style={font=\color{black}},
ymin=0,
ymax=0.012,
axis background/.style={fill=white},
axis x line*=bottom,
axis y line*=left,
colormap={mymap}{[1pt] rgb(0pt)=(0,1,0); rgb(7pt)=(0,1,1); rgb(15pt)=(0,0,1); rgb(23pt)=(1,0,1); rgb(31pt)=(1,0,0); rgb(38pt)=(0,0,0)},
colorbar,
colorbar style={separate axis lines,every outer x axis line/.append style={black},every x tick label/.append style={font=\color{black}},every outer y axis line/.append style={black},every y tick label/.append style={font=\color{black}},yticklabels={{-19},{-17},{-15},{-13},{-11},{-9},{-7},{-5},{-3},{-1},{1},{3},{5},{7},{9},{11},{13},{15},{17},{19}}}
]
\addplot [color=green,solid,forget plot]
  table[row sep=crcr]{%
0.01	0.0050253436469414\\
1.01	0.00502534430234539\\
2.01	0.00502534497054888\\
3.01	0.00502534565180165\\
4.01	0.0050253463463589\\
5.01	0.00502534705448057\\
6.01	0.00502534777643188\\
7.01	0.00502534851248307\\
8.01	0.00502534926290973\\
9.01	0.00502535002799284\\
10.01	0.00502535080801872\\
11.01	0.00502535160327991\\
12.01	0.00502535241407387\\
13.01	0.00502535324070441\\
14.01	0.00502535408348131\\
15.01	0.00502535494272008\\
16.01	0.00502535581874267\\
17.01	0.00502535671187737\\
18.01	0.00502535762245873\\
19.01	0.00502535855082814\\
20.01	0.00502535949733349\\
21.01	0.00502536046232946\\
22.01	0.00502536144617782\\
23.01	0.00502536244924762\\
24.01	0.00502536347191462\\
25.01	0.00502536451456237\\
26.01	0.00502536557758182\\
27.01	0.00502536666137174\\
28.01	0.00502536776633886\\
29.01	0.00502536889289751\\
30.01	0.00502537004147008\\
31.01	0.00502537121248772\\
32.01	0.00502537240638989\\
33.01	0.00502537362362432\\
34.01	0.00502537486464871\\
35.01	0.00502537612992798\\
36.01	0.00502537741993778\\
37.01	0.00502537873516229\\
38.01	0.00502538007609546\\
39.01	0.00502538144324083\\
40.01	0.00502538283711204\\
41.01	0.00502538425823262\\
42.01	0.00502538570713662\\
43.01	0.00502538718436846\\
44.01	0.00502538869048358\\
45.01	0.00502539022604787\\
46.01	0.00502539179163891\\
47.01	0.005025393387845\\
48.01	0.00502539501526664\\
49.01	0.00502539667451579\\
50.01	0.00502539836621682\\
51.01	0.00502540009100638\\
52.01	0.00502540184953321\\
53.01	0.00502540364245924\\
54.01	0.00502540547045979\\
55.01	0.00502540733422271\\
56.01	0.00502540923444972\\
57.01	0.00502541117185656\\
58.01	0.00502541314717336\\
59.01	0.00502541516114388\\
60.01	0.00502541721452765\\
61.01	0.00502541930809802\\
62.01	0.00502542144264437\\
63.01	0.00502542361897154\\
64.01	0.00502542583790042\\
65.01	0.00502542810026785\\
66.01	0.00502543040692743\\
67.01	0.00502543275874937\\
68.01	0.00502543515662152\\
69.01	0.00502543760144908\\
70.01	0.00502544009415481\\
71.01	0.00502544263568036\\
72.01	0.0050254452269857\\
73.01	0.00502544786904979\\
74.01	0.00502545056287121\\
75.01	0.00502545330946821\\
76.01	0.00502545610987915\\
77.01	0.00502545896516299\\
78.01	0.00502546187639957\\
79.01	0.00502546484469038\\
80.01	0.00502546787115871\\
81.01	0.00502547095695013\\
82.01	0.00502547410323306\\
83.01	0.00502547731119857\\
84.01	0.00502548058206204\\
85.01	0.00502548391706273\\
86.01	0.00502548731746463\\
87.01	0.0050254907845565\\
88.01	0.00502549431965296\\
89.01	0.00502549792409521\\
90.01	0.00502550159925003\\
91.01	0.00502550534651219\\
92.01	0.0050255091673041\\
93.01	0.00502551306307628\\
94.01	0.0050255170353086\\
95.01	0.00502552108550974\\
96.01	0.0050255252152184\\
97.01	0.00502552942600487\\
98.01	0.00502553371946982\\
99.01	0.00502553809724601\\
100.01	0.00502554256099869\\
101.01	0.00502554711242673\\
102.01	0.00502555175326247\\
103.01	0.00502555648527314\\
104.01	0.00502556131026061\\
105.01	0.00502556623006328\\
106.01	0.00502557124655656\\
107.01	0.00502557636165213\\
108.01	0.00502558157730099\\
109.01	0.00502558689549265\\
110.01	0.00502559231825647\\
111.01	0.00502559784766227\\
112.01	0.00502560348582148\\
113.01	0.00502560923488735\\
114.01	0.00502561509705665\\
115.01	0.00502562107456972\\
116.01	0.00502562716971211\\
117.01	0.00502563338481501\\
118.01	0.00502563972225592\\
119.01	0.0050256461844608\\
120.01	0.00502565277390346\\
121.01	0.00502565949310754\\
122.01	0.00502566634464752\\
123.01	0.00502567333114898\\
124.01	0.00502568045529087\\
125.01	0.00502568771980505\\
126.01	0.00502569512747891\\
127.01	0.00502570268115568\\
128.01	0.00502571038373564\\
129.01	0.00502571823817732\\
130.01	0.00502572624749856\\
131.01	0.00502573441477791\\
132.01	0.00502574274315543\\
133.01	0.00502575123583529\\
134.01	0.00502575989608575\\
135.01	0.00502576872724045\\
136.01	0.00502577773270074\\
137.01	0.00502578691593569\\
138.01	0.00502579628048477\\
139.01	0.00502580582995865\\
140.01	0.0050258155680408\\
141.01	0.0050258254984889\\
142.01	0.00502583562513637\\
143.01	0.00502584595189431\\
144.01	0.00502585648275183\\
145.01	0.00502586722177942\\
146.01	0.00502587817312949\\
147.01	0.00502588934103815\\
148.01	0.00502590072982748\\
149.01	0.00502591234390623\\
150.01	0.00502592418777278\\
151.01	0.00502593626601633\\
152.01	0.00502594858331887\\
153.01	0.00502596114445725\\
154.01	0.00502597395430432\\
155.01	0.00502598701783252\\
156.01	0.00502600034011449\\
157.01	0.00502601392632548\\
158.01	0.00502602778174588\\
159.01	0.00502604191176259\\
160.01	0.00502605632187225\\
161.01	0.00502607101768271\\
162.01	0.0050260860049155\\
163.01	0.00502610128940839\\
164.01	0.00502611687711763\\
165.01	0.00502613277412037\\
166.01	0.00502614898661696\\
167.01	0.00502616552093366\\
168.01	0.00502618238352512\\
169.01	0.00502619958097786\\
170.01	0.00502621712001197\\
171.01	0.00502623500748392\\
172.01	0.00502625325038977\\
173.01	0.00502627185586785\\
174.01	0.00502629083120179\\
175.01	0.0050263101838236\\
176.01	0.00502632992131696\\
177.01	0.00502635005141905\\
178.01	0.00502637058202549\\
179.01	0.00502639152119269\\
180.01	0.00502641287714084\\
181.01	0.00502643465825794\\
182.01	0.00502645687310335\\
183.01	0.00502647953041124\\
184.01	0.00502650263909296\\
185.01	0.00502652620824279\\
186.01	0.00502655024714027\\
187.01	0.00502657476525428\\
188.01	0.00502659977224754\\
189.01	0.00502662527797978\\
190.01	0.00502665129251281\\
191.01	0.00502667782611341\\
192.01	0.00502670488925874\\
193.01	0.00502673249264007\\
194.01	0.00502676064716748\\
195.01	0.00502678936397455\\
196.01	0.0050268186544222\\
197.01	0.00502684853010429\\
198.01	0.00502687900285209\\
199.01	0.00502691008473879\\
200.01	0.00502694178808552\\
201.01	0.00502697412546577\\
202.01	0.00502700710971057\\
203.01	0.00502704075391375\\
204.01	0.00502707507143804\\
205.01	0.00502711007592019\\
206.01	0.00502714578127678\\
207.01	0.00502718220170939\\
208.01	0.00502721935171177\\
209.01	0.00502725724607461\\
210.01	0.0050272958998929\\
211.01	0.00502733532857103\\
212.01	0.00502737554782998\\
213.01	0.00502741657371389\\
214.01	0.00502745842259641\\
215.01	0.00502750111118755\\
216.01	0.00502754465654134\\
217.01	0.0050275890760616\\
218.01	0.00502763438751083\\
219.01	0.00502768060901643\\
220.01	0.00502772775907867\\
221.01	0.00502777585657851\\
222.01	0.00502782492078555\\
223.01	0.00502787497136614\\
224.01	0.00502792602839108\\
225.01	0.00502797811234429\\
226.01	0.00502803124413168\\
227.01	0.00502808544508973\\
228.01	0.0050281407369944\\
229.01	0.00502819714206984\\
230.01	0.00502825468299856\\
231.01	0.00502831338293011\\
232.01	0.00502837326549152\\
233.01	0.0050284343547963\\
234.01	0.00502849667545505\\
235.01	0.00502856025258627\\
236.01	0.00502862511182556\\
237.01	0.00502869127933764\\
238.01	0.00502875878182693\\
239.01	0.00502882764654813\\
240.01	0.00502889790131839\\
241.01	0.00502896957452861\\
242.01	0.00502904269515575\\
243.01	0.0050291172927743\\
244.01	0.0050291933975688\\
245.01	0.00502927104034726\\
246.01	0.0050293502525528\\
247.01	0.00502943106627787\\
248.01	0.00502951351427761\\
249.01	0.00502959762998303\\
250.01	0.00502968344751517\\
251.01	0.00502977100169932\\
252.01	0.00502986032808012\\
253.01	0.00502995146293608\\
254.01	0.00503004444329483\\
255.01	0.00503013930694796\\
256.01	0.00503023609246817\\
257.01	0.00503033483922449\\
258.01	0.0050304355873991\\
259.01	0.00503053837800363\\
260.01	0.00503064325289665\\
261.01	0.00503075025480176\\
262.01	0.00503085942732462\\
263.01	0.00503097081497172\\
264.01	0.00503108446316855\\
265.01	0.00503120041827904\\
266.01	0.0050313187276243\\
267.01	0.00503143943950354\\
268.01	0.0050315626032135\\
269.01	0.00503168826906858\\
270.01	0.00503181648842269\\
271.01	0.00503194731369064\\
272.01	0.0050320807983697\\
273.01	0.00503221699706208\\
274.01	0.00503235596549806\\
275.01	0.00503249776055859\\
276.01	0.00503264244029958\\
277.01	0.00503279006397625\\
278.01	0.00503294069206728\\
279.01	0.0050330943863007\\
280.01	0.00503325120967907\\
281.01	0.0050334112265065\\
282.01	0.00503357450241435\\
283.01	0.00503374110439042\\
284.01	0.00503391110080476\\
285.01	0.00503408456144014\\
286.01	0.0050342615575199\\
287.01	0.00503444216173804\\
288.01	0.0050346264482895\\
289.01	0.00503481449290169\\
290.01	0.00503500637286478\\
291.01	0.00503520216706568\\
292.01	0.0050354019560189\\
293.01	0.00503560582190185\\
294.01	0.00503581384858895\\
295.01	0.00503602612168494\\
296.01	0.00503624272856248\\
297.01	0.00503646375839809\\
298.01	0.0050366893022086\\
299.01	0.00503691945289003\\
300.01	0.00503715430525531\\
301.01	0.00503739395607522\\
302.01	0.00503763850411706\\
303.01	0.00503788805018743\\
304.01	0.00503814269717304\\
305.01	0.0050384025500842\\
306.01	0.00503866771609884\\
307.01	0.00503893830460631\\
308.01	0.005039214427254\\
309.01	0.00503949619799313\\
310.01	0.00503978373312652\\
311.01	0.00504007715135721\\
312.01	0.00504037657383813\\
313.01	0.00504068212422235\\
314.01	0.0050409939287139\\
315.01	0.00504131211612241\\
316.01	0.00504163681791495\\
317.01	0.0050419681682714\\
318.01	0.00504230630414115\\
319.01	0.00504265136529913\\
320.01	0.00504300349440465\\
321.01	0.005043362837061\\
322.01	0.00504372954187635\\
323.01	0.00504410376052492\\
324.01	0.00504448564781135\\
325.01	0.00504487536173437\\
326.01	0.0050452730635529\\
327.01	0.00504567891785457\\
328.01	0.00504609309262244\\
329.01	0.00504651575930629\\
330.01	0.00504694709289338\\
331.01	0.00504738727198278\\
332.01	0.00504783647885829\\
333.01	0.00504829489956601\\
334.01	0.00504876272398996\\
335.01	0.00504924014593339\\
336.01	0.00504972736319814\\
337.01	0.00505022457766829\\
338.01	0.00505073199539322\\
339.01	0.00505124982667407\\
340.01	0.00505177828615133\\
341.01	0.00505231759289476\\
342.01	0.00505286797049365\\
343.01	0.0050534296471508\\
344.01	0.00505400285577759\\
345.01	0.00505458783409077\\
346.01	0.00505518482471143\\
347.01	0.00505579407526607\\
348.01	0.00505641583849006\\
349.01	0.00505705037233223\\
350.01	0.00505769794006256\\
351.01	0.00505835881038222\\
352.01	0.00505903325753451\\
353.01	0.00505972156141962\\
354.01	0.00506042400771045\\
355.01	0.00506114088797194\\
356.01	0.00506187249978191\\
357.01	0.00506261914685557\\
358.01	0.00506338113917071\\
359.01	0.00506415879309688\\
360.01	0.00506495243152727\\
361.01	0.00506576238401224\\
362.01	0.00506658898689749\\
363.01	0.00506743258346209\\
364.01	0.00506829352406188\\
365.01	0.00506917216627517\\
366.01	0.00507006887505074\\
367.01	0.00507098402286013\\
368.01	0.00507191798985146\\
369.01	0.00507287116400743\\
370.01	0.00507384394130747\\
371.01	0.00507483672589013\\
372.01	0.00507584993022273\\
373.01	0.00507688397527178\\
374.01	0.00507793929067735\\
375.01	0.00507901631493216\\
376.01	0.00508011549556292\\
377.01	0.00508123728931636\\
378.01	0.00508238216234905\\
379.01	0.00508355059042037\\
380.01	0.00508474305909051\\
381.01	0.00508596006392134\\
382.01	0.00508720211068256\\
383.01	0.00508846971556157\\
384.01	0.00508976340537737\\
385.01	0.00509108371779993\\
386.01	0.00509243120157252\\
387.01	0.00509380641673987\\
388.01	0.00509520993488072\\
389.01	0.0050966423393447\\
390.01	0.00509810422549539\\
391.01	0.00509959620095607\\
392.01	0.00510111888586348\\
393.01	0.00510267291312402\\
394.01	0.00510425892867744\\
395.01	0.00510587759176476\\
396.01	0.00510752957520158\\
397.01	0.00510921556565833\\
398.01	0.00511093626394493\\
399.01	0.00511269238530185\\
400.01	0.00511448465969741\\
401.01	0.00511631383213107\\
402.01	0.00511818066294276\\
403.01	0.0051200859281289\\
404.01	0.00512203041966497\\
405.01	0.00512401494583434\\
406.01	0.00512604033156455\\
407.01	0.00512810741876957\\
408.01	0.00513021706670055\\
409.01	0.00513237015230281\\
410.01	0.00513456757058009\\
411.01	0.00513681023496659\\
412.01	0.00513909907770765\\
413.01	0.00514143505024678\\
414.01	0.00514381912362155\\
415.01	0.0051462522888675\\
416.01	0.0051487355574302\\
417.01	0.00515126996158641\\
418.01	0.00515385655487299\\
419.01	0.00515649641252641\\
420.01	0.00515919063192849\\
421.01	0.005161940333065\\
422.01	0.00516474665899053\\
423.01	0.00516761077630448\\
424.01	0.00517053387563707\\
425.01	0.00517351717214504\\
426.01	0.00517656190601712\\
427.01	0.00517966934299052\\
428.01	0.00518284077487896\\
429.01	0.00518607752010913\\
430.01	0.00518938092427124\\
431.01	0.00519275236067859\\
432.01	0.00519619323094003\\
433.01	0.00519970496554423\\
434.01	0.00520328902445524\\
435.01	0.00520694689772078\\
436.01	0.00521068010609343\\
437.01	0.00521449020166515\\
438.01	0.00521837876851267\\
439.01	0.00522234742335928\\
440.01	0.00522639781624823\\
441.01	0.00523053163123074\\
442.01	0.00523475058706865\\
443.01	0.00523905643795085\\
444.01	0.00524345097422511\\
445.01	0.00524793602314479\\
446.01	0.00525251344963172\\
447.01	0.00525718515705385\\
448.01	0.00526195308801961\\
449.01	0.00526681922518889\\
450.01	0.0052717855921007\\
451.01	0.00527685425401767\\
452.01	0.00528202731878902\\
453.01	0.00528730693772994\\
454.01	0.00529269530652099\\
455.01	0.00529819466612468\\
456.01	0.00530380730372196\\
457.01	0.00530953555366792\\
458.01	0.00531538179846745\\
459.01	0.0053213484697711\\
460.01	0.00532743804939127\\
461.01	0.00533365307034027\\
462.01	0.00533999611788943\\
463.01	0.00534646983065033\\
464.01	0.00535307690167858\\
465.01	0.00535982007960048\\
466.01	0.0053667021697633\\
467.01	0.00537372603540943\\
468.01	0.00538089459887411\\
469.01	0.00538821084280963\\
470.01	0.00539567781143413\\
471.01	0.00540329861180545\\
472.01	0.00541107641512373\\
473.01	0.00541901445805865\\
474.01	0.00542711604410569\\
475.01	0.00543538454497058\\
476.01	0.00544382340198156\\
477.01	0.00545243612753163\\
478.01	0.00546122630655084\\
479.01	0.00547019759800861\\
480.01	0.00547935373644722\\
481.01	0.00548869853354755\\
482.01	0.00549823587972666\\
483.01	0.00550796974576846\\
484.01	0.00551790418448807\\
485.01	0.00552804333243061\\
486.01	0.00553839141160435\\
487.01	0.00554895273125022\\
488.01	0.00555973168964703\\
489.01	0.00557073277595338\\
490.01	0.00558196057208785\\
491.01	0.00559341975464639\\
492.01	0.00560511509685933\\
493.01	0.00561705147058765\\
494.01	0.00562923384835898\\
495.01	0.00564166730544473\\
496.01	0.00565435702197845\\
497.01	0.00566730828511622\\
498.01	0.00568052649123867\\
499.01	0.00569401714819741\\
500.01	0.00570778587760391\\
501.01	0.00572183841716205\\
502.01	0.0057361806230467\\
503.01	0.00575081847232464\\
504.01	0.0057657580654227\\
505.01	0.00578100562863958\\
506.01	0.00579656751670357\\
507.01	0.00581245021537605\\
508.01	0.00582866034409943\\
509.01	0.00584520465869111\\
510.01	0.00586209005408149\\
511.01	0.00587932356709678\\
512.01	0.00589691237928515\\
513.01	0.00591486381978583\\
514.01	0.00593318536823928\\
515.01	0.00595188465773874\\
516.01	0.00597096947781858\\
517.01	0.0059904477774809\\
518.01	0.00601032766825516\\
519.01	0.00603061742729019\\
520.01	0.0060513255004731\\
521.01	0.00607246050557363\\
522.01	0.0060940312354081\\
523.01	0.00611604666101788\\
524.01	0.00613851593485767\\
525.01	0.00616144839398609\\
526.01	0.00618485356325117\\
527.01	0.006208741158463\\
528.01	0.00623312108954262\\
529.01	0.00625800346363684\\
530.01	0.0062833985881868\\
531.01	0.00630931697393645\\
532.01	0.00633576933786364\\
533.01	0.00636276660601857\\
534.01	0.00639031991624708\\
535.01	0.00641844062077823\\
536.01	0.00644714028865036\\
537.01	0.00647643070794703\\
538.01	0.00650632388781151\\
539.01	0.00653683206020528\\
540.01	0.00656796768136911\\
541.01	0.00659974343294424\\
542.01	0.00663217222270325\\
543.01	0.00666526718483568\\
544.01	0.00669904167972642\\
545.01	0.00673350929315926\\
546.01	0.00676868383486742\\
547.01	0.00680457933634701\\
548.01	0.00684121004783805\\
549.01	0.00687859043436699\\
550.01	0.00691673517073471\\
551.01	0.0069556591353172\\
552.01	0.00699537740253582\\
553.01	0.00703590523383653\\
554.01	0.0070772580669986\\
555.01	0.00711945150357646\\
556.01	0.00716250129425623\\
557.01	0.00720642332188554\\
558.01	0.00725123358191065\\
559.01	0.00729694815992683\\
560.01	0.00734358320602013\\
561.01	0.00739115490554365\\
562.01	0.00743967944594207\\
563.01	0.00748917297919549\\
564.01	0.00753965157942145\\
565.01	0.00759113119512742\\
566.01	0.00764362759556816\\
567.01	0.00769715631061607\\
568.01	0.00775173256351049\\
569.01	0.0078073711958089\\
570.01	0.00786408658382248\\
571.01	0.00792189254578262\\
572.01	0.00798080223895675\\
573.01	0.00804082804591287\\
574.01	0.00810198144913145\\
575.01	0.00816427289318073\\
576.01	0.00822771163372017\\
577.01	0.00829230557268554\\
578.01	0.00835806107914579\\
579.01	0.00842498279553232\\
580.01	0.00849307342923551\\
581.01	0.00856233352997421\\
582.01	0.00863276125390358\\
583.01	0.00870435211617224\\
584.01	0.00877709873462935\\
585.01	0.00885099056867839\\
586.01	0.00892601365895706\\
587.01	0.00900215037570352\\
588.01	0.00907937918646837\\
589.01	0.0091576744574184\\
590.01	0.00923700630705446\\
591.01	0.00931734053698447\\
592.01	0.00939863867177418\\
593.01	0.00948085814924172\\
594.01	0.00956395271435918\\
595.01	0.00964787308480385\\
596.01	0.00973256797493054\\
597.01	0.0098179855884858\\
598.01	0.00990285876644176\\
599.01	0.00996919203046377\\
599.02	0.00996973314335038\\
599.03	0.0099702709571694\\
599.04	0.00997080543920336\\
599.05	0.0099713365564138\\
599.06	0.00997186427543808\\
599.07	0.0099723885625862\\
599.08	0.00997290938383756\\
599.09	0.00997342670483771\\
599.1	0.00997394049089505\\
599.11	0.00997445070697751\\
599.12	0.00997495731770918\\
599.13	0.00997546028736696\\
599.14	0.00997595957987707\\
599.15	0.00997645515881165\\
599.16	0.00997694698738526\\
599.17	0.00997743502845131\\
599.18	0.00997791924449856\\
599.19	0.00997839959764748\\
599.2	0.00997887604964662\\
599.21	0.00997934856186899\\
599.22	0.00997981709530829\\
599.23	0.00998028161057521\\
599.24	0.00998074206789365\\
599.25	0.0099811984270969\\
599.26	0.00998165064762378\\
599.27	0.00998209868851477\\
599.28	0.00998254250840806\\
599.29	0.00998298206553559\\
599.3	0.00998341731771905\\
599.31	0.00998384822236584\\
599.32	0.00998427473646497\\
599.33	0.00998469681658292\\
599.34	0.00998511441885952\\
599.35	0.0099855274990037\\
599.36	0.00998593601228926\\
599.37	0.00998633991355056\\
599.38	0.00998673915717821\\
599.39	0.00998713369711469\\
599.4	0.00998752348684992\\
599.41	0.00998790847811157\\
599.42	0.00998828861974491\\
599.43	0.00998866386008804\\
599.44	0.00998903414696681\\
599.45	0.00998939942768985\\
599.46	0.00998975964904344\\
599.47	0.00999011475728636\\
599.48	0.00999046469814472\\
599.49	0.00999080941680669\\
599.5	0.00999114885791725\\
599.51	0.00999148296557278\\
599.52	0.0099918116833157\\
599.53	0.00999213495412901\\
599.54	0.00999245272043077\\
599.55	0.00999276492406855\\
599.56	0.00999307150631382\\
599.57	0.00999337240785624\\
599.58	0.00999366756879799\\
599.59	0.00999395692864795\\
599.6	0.00999424042631586\\
599.61	0.00999451800010642\\
599.62	0.00999478958771336\\
599.63	0.0099950551262134\\
599.64	0.00999531455206019\\
599.65	0.00999556780107816\\
599.66	0.00999581480845634\\
599.67	0.00999605550874211\\
599.68	0.00999628983583487\\
599.69	0.00999651772297967\\
599.7	0.00999673910276076\\
599.71	0.00999695390709509\\
599.72	0.00999716206722577\\
599.73	0.00999736351371538\\
599.74	0.00999755817643933\\
599.75	0.00999774598457904\\
599.76	0.00999792686661517\\
599.77	0.00999810075032068\\
599.78	0.00999826756275386\\
599.79	0.00999842723025133\\
599.8	0.00999857967842092\\
599.81	0.0099987248321345\\
599.82	0.00999886261552071\\
599.83	0.00999899295195769\\
599.84	0.00999911576406567\\
599.85	0.00999923097369951\\
599.86	0.00999933850194117\\
599.87	0.00999943826909211\\
599.88	0.00999953019466558\\
599.89	0.00999961419737891\\
599.9	0.00999969019514566\\
599.91	0.00999975810506767\\
599.92	0.00999981784342713\\
599.93	0.00999986932567848\\
599.94	0.00999991246644028\\
599.95	0.00999994717948697\\
599.96	0.00999997337774056\\
599.97	0.00999999097326228\\
599.98	0.00999999987724406\\
599.99	0.01\\
600	0.01\\
};
\addplot [color=mycolor1,solid,forget plot]
  table[row sep=crcr]{%
0.01	0.00502505708478154\\
1.01	0.00502505776987136\\
2.01	0.00502505846833001\\
3.01	0.00502505918041851\\
4.01	0.00502505990640237\\
5.01	0.00502506064655234\\
6.01	0.00502506140114448\\
7.01	0.00502506217046036\\
8.01	0.00502506295478708\\
9.01	0.00502506375441727\\
10.01	0.0050250645696491\\
11.01	0.00502506540078659\\
12.01	0.00502506624814011\\
13.01	0.00502506711202561\\
14.01	0.00502506799276525\\
15.01	0.00502506889068779\\
16.01	0.00502506980612805\\
17.01	0.00502507073942768\\
18.01	0.00502507169093472\\
19.01	0.00502507266100416\\
20.01	0.00502507364999782\\
21.01	0.0050250746582847\\
22.01	0.00502507568624078\\
23.01	0.00502507673424957\\
24.01	0.0050250778027022\\
25.01	0.0050250788919973\\
26.01	0.00502508000254118\\
27.01	0.00502508113474812\\
28.01	0.00502508228904022\\
29.01	0.00502508346584857\\
30.01	0.00502508466561202\\
31.01	0.00502508588877826\\
32.01	0.00502508713580357\\
33.01	0.00502508840715338\\
34.01	0.00502508970330179\\
35.01	0.00502509102473275\\
36.01	0.00502509237193918\\
37.01	0.00502509374542376\\
38.01	0.0050250951456991\\
39.01	0.00502509657328767\\
40.01	0.00502509802872239\\
41.01	0.00502509951254626\\
42.01	0.00502510102531317\\
43.01	0.0050251025675877\\
44.01	0.00502510413994553\\
45.01	0.00502510574297341\\
46.01	0.00502510737726991\\
47.01	0.00502510904344494\\
48.01	0.00502511074212078\\
49.01	0.00502511247393153\\
50.01	0.00502511423952365\\
51.01	0.00502511603955635\\
52.01	0.00502511787470195\\
53.01	0.00502511974564554\\
54.01	0.00502512165308583\\
55.01	0.0050251235977353\\
56.01	0.00502512558032045\\
57.01	0.00502512760158177\\
58.01	0.00502512966227386\\
59.01	0.00502513176316688\\
60.01	0.00502513390504561\\
61.01	0.00502513608871036\\
62.01	0.00502513831497725\\
63.01	0.00502514058467798\\
64.01	0.00502514289866096\\
65.01	0.0050251452577908\\
66.01	0.00502514766294929\\
67.01	0.00502515011503596\\
68.01	0.00502515261496719\\
69.01	0.00502515516367757\\
70.01	0.00502515776212048\\
71.01	0.00502516041126738\\
72.01	0.0050251631121091\\
73.01	0.00502516586565591\\
74.01	0.00502516867293762\\
75.01	0.0050251715350047\\
76.01	0.00502517445292788\\
77.01	0.00502517742779912\\
78.01	0.00502518046073192\\
79.01	0.00502518355286122\\
80.01	0.00502518670534463\\
81.01	0.00502518991936224\\
82.01	0.00502519319611752\\
83.01	0.00502519653683742\\
84.01	0.0050251999427732\\
85.01	0.00502520341520069\\
86.01	0.00502520695542052\\
87.01	0.00502521056475919\\
88.01	0.00502521424456948\\
89.01	0.00502521799622993\\
90.01	0.00502522182114742\\
91.01	0.00502522572075539\\
92.01	0.00502522969651603\\
93.01	0.00502523374992067\\
94.01	0.00502523788248893\\
95.01	0.00502524209577135\\
96.01	0.00502524639134888\\
97.01	0.00502525077083307\\
98.01	0.00502525523586747\\
99.01	0.00502525978812848\\
100.01	0.00502526442932494\\
101.01	0.00502526916119955\\
102.01	0.00502527398552965\\
103.01	0.00502527890412721\\
104.01	0.0050252839188403\\
105.01	0.00502528903155341\\
106.01	0.00502529424418774\\
107.01	0.00502529955870326\\
108.01	0.00502530497709808\\
109.01	0.00502531050140967\\
110.01	0.00502531613371601\\
111.01	0.00502532187613616\\
112.01	0.00502532773083089\\
113.01	0.00502533370000388\\
114.01	0.00502533978590211\\
115.01	0.00502534599081714\\
116.01	0.00502535231708595\\
117.01	0.00502535876709181\\
118.01	0.00502536534326498\\
119.01	0.00502537204808393\\
120.01	0.005025378884076\\
121.01	0.00502538585381879\\
122.01	0.00502539295994089\\
123.01	0.00502540020512304\\
124.01	0.00502540759209896\\
125.01	0.0050254151236567\\
126.01	0.00502542280263979\\
127.01	0.0050254306319476\\
128.01	0.00502543861453753\\
129.01	0.00502544675342539\\
130.01	0.00502545505168694\\
131.01	0.00502546351245933\\
132.01	0.00502547213894153\\
133.01	0.00502548093439648\\
134.01	0.00502548990215143\\
135.01	0.00502549904560015\\
136.01	0.00502550836820367\\
137.01	0.0050255178734924\\
138.01	0.00502552756506625\\
139.01	0.00502553744659737\\
140.01	0.00502554752183065\\
141.01	0.00502555779458538\\
142.01	0.00502556826875752\\
143.01	0.00502557894831987\\
144.01	0.0050255898373251\\
145.01	0.00502560093990664\\
146.01	0.00502561226027967\\
147.01	0.00502562380274427\\
148.01	0.00502563557168551\\
149.01	0.00502564757157675\\
150.01	0.00502565980698014\\
151.01	0.00502567228254919\\
152.01	0.00502568500303027\\
153.01	0.00502569797326444\\
154.01	0.00502571119819006\\
155.01	0.00502572468284337\\
156.01	0.00502573843236212\\
157.01	0.0050257524519868\\
158.01	0.00502576674706218\\
159.01	0.00502578132304063\\
160.01	0.00502579618548352\\
161.01	0.00502581134006361\\
162.01	0.00502582679256727\\
163.01	0.00502584254889701\\
164.01	0.00502585861507368\\
165.01	0.00502587499723875\\
166.01	0.00502589170165724\\
167.01	0.00502590873471953\\
168.01	0.00502592610294486\\
169.01	0.00502594381298288\\
170.01	0.00502596187161705\\
171.01	0.00502598028576712\\
172.01	0.0050259990624921\\
173.01	0.00502601820899304\\
174.01	0.00502603773261556\\
175.01	0.00502605764085359\\
176.01	0.00502607794135135\\
177.01	0.00502609864190751\\
178.01	0.00502611975047776\\
179.01	0.00502614127517792\\
180.01	0.00502616322428776\\
181.01	0.00502618560625406\\
182.01	0.00502620842969345\\
183.01	0.00502623170339703\\
184.01	0.00502625543633339\\
185.01	0.00502627963765204\\
186.01	0.00502630431668719\\
187.01	0.00502632948296225\\
188.01	0.0050263551461926\\
189.01	0.00502638131629046\\
190.01	0.00502640800336845\\
191.01	0.00502643521774387\\
192.01	0.00502646296994291\\
193.01	0.00502649127070517\\
194.01	0.00502652013098762\\
195.01	0.00502654956196918\\
196.01	0.00502657957505573\\
197.01	0.00502661018188422\\
198.01	0.00502664139432787\\
199.01	0.00502667322450087\\
200.01	0.00502670568476324\\
201.01	0.00502673878772611\\
202.01	0.00502677254625666\\
203.01	0.00502680697348418\\
204.01	0.00502684208280435\\
205.01	0.00502687788788525\\
206.01	0.00502691440267317\\
207.01	0.00502695164139846\\
208.01	0.00502698961858103\\
209.01	0.0050270283490364\\
210.01	0.00502706784788198\\
211.01	0.00502710813054341\\
212.01	0.00502714921276085\\
213.01	0.00502719111059502\\
214.01	0.0050272338404346\\
215.01	0.00502727741900305\\
216.01	0.00502732186336444\\
217.01	0.005027367190932\\
218.01	0.00502741341947435\\
219.01	0.00502746056712336\\
220.01	0.00502750865238133\\
221.01	0.00502755769412885\\
222.01	0.00502760771163299\\
223.01	0.00502765872455443\\
224.01	0.0050277107529568\\
225.01	0.00502776381731429\\
226.01	0.00502781793852039\\
227.01	0.00502787313789665\\
228.01	0.00502792943720123\\
229.01	0.00502798685863878\\
230.01	0.0050280454248687\\
231.01	0.00502810515901558\\
232.01	0.00502816608467785\\
233.01	0.00502822822593832\\
234.01	0.00502829160737461\\
235.01	0.00502835625406842\\
236.01	0.00502842219161706\\
237.01	0.00502848944614304\\
238.01	0.00502855804430619\\
239.01	0.00502862801331419\\
240.01	0.00502869938093389\\
241.01	0.00502877217550329\\
242.01	0.00502884642594303\\
243.01	0.00502892216176901\\
244.01	0.00502899941310453\\
245.01	0.00502907821069254\\
246.01	0.0050291585859096\\
247.01	0.00502924057077767\\
248.01	0.00502932419797852\\
249.01	0.00502940950086742\\
250.01	0.0050294965134865\\
251.01	0.00502958527058009\\
252.01	0.00502967580760794\\
253.01	0.00502976816076126\\
254.01	0.00502986236697753\\
255.01	0.00502995846395607\\
256.01	0.00503005649017382\\
257.01	0.00503015648490135\\
258.01	0.00503025848821978\\
259.01	0.0050303625410372\\
260.01	0.00503046868510645\\
261.01	0.00503057696304171\\
262.01	0.00503068741833726\\
263.01	0.00503080009538562\\
264.01	0.00503091503949554\\
265.01	0.00503103229691196\\
266.01	0.00503115191483511\\
267.01	0.00503127394143976\\
268.01	0.00503139842589623\\
269.01	0.00503152541839064\\
270.01	0.00503165497014652\\
271.01	0.00503178713344521\\
272.01	0.00503192196164903\\
273.01	0.00503205950922278\\
274.01	0.00503219983175696\\
275.01	0.00503234298599126\\
276.01	0.00503248902983819\\
277.01	0.0050326380224073\\
278.01	0.00503279002402979\\
279.01	0.00503294509628419\\
280.01	0.00503310330202217\\
281.01	0.00503326470539428\\
282.01	0.00503342937187773\\
283.01	0.00503359736830303\\
284.01	0.00503376876288231\\
285.01	0.00503394362523771\\
286.01	0.00503412202643089\\
287.01	0.0050343040389923\\
288.01	0.00503448973695184\\
289.01	0.00503467919586953\\
290.01	0.00503487249286755\\
291.01	0.00503506970666168\\
292.01	0.00503527091759529\\
293.01	0.00503547620767214\\
294.01	0.00503568566059007\\
295.01	0.0050358993617779\\
296.01	0.00503611739842902\\
297.01	0.00503633985953877\\
298.01	0.0050365668359417\\
299.01	0.00503679842034899\\
300.01	0.00503703470738753\\
301.01	0.00503727579363877\\
302.01	0.00503752177767999\\
303.01	0.00503777276012465\\
304.01	0.00503802884366436\\
305.01	0.00503829013311213\\
306.01	0.00503855673544537\\
307.01	0.00503882875985134\\
308.01	0.00503910631777216\\
309.01	0.00503938952295147\\
310.01	0.00503967849148132\\
311.01	0.00503997334185133\\
312.01	0.00504027419499739\\
313.01	0.00504058117435266\\
314.01	0.00504089440589889\\
315.01	0.00504121401821822\\
316.01	0.0050415401425481\\
317.01	0.005041872912835\\
318.01	0.00504221246579045\\
319.01	0.00504255894094829\\
320.01	0.00504291248072317\\
321.01	0.00504327323046944\\
322.01	0.0050436413385419\\
323.01	0.00504401695635803\\
324.01	0.00504440023846103\\
325.01	0.00504479134258454\\
326.01	0.00504519042971829\\
327.01	0.00504559766417539\\
328.01	0.00504601321366138\\
329.01	0.00504643724934425\\
330.01	0.00504686994592582\\
331.01	0.00504731148171481\\
332.01	0.00504776203870178\\
333.01	0.00504822180263492\\
334.01	0.00504869096309848\\
335.01	0.00504916971359129\\
336.01	0.00504965825160864\\
337.01	0.00505015677872389\\
338.01	0.00505066550067459\\
339.01	0.00505118462744747\\
340.01	0.00505171437336652\\
341.01	0.00505225495718289\\
342.01	0.00505280660216778\\
343.01	0.00505336953620408\\
344.01	0.00505394399188328\\
345.01	0.00505453020660221\\
346.01	0.00505512842266326\\
347.01	0.00505573888737543\\
348.01	0.00505636185315794\\
349.01	0.00505699757764599\\
350.01	0.00505764632379935\\
351.01	0.00505830836001134\\
352.01	0.00505898396022197\\
353.01	0.00505967340403249\\
354.01	0.00506037697682216\\
355.01	0.0050610949698679\\
356.01	0.0050618276804657\\
357.01	0.00506257541205518\\
358.01	0.00506333847434595\\
359.01	0.00506411718344768\\
360.01	0.00506491186200134\\
361.01	0.00506572283931435\\
362.01	0.00506655045149716\\
363.01	0.00506739504160418\\
364.01	0.00506825695977667\\
365.01	0.00506913656338806\\
366.01	0.00507003421719339\\
367.01	0.00507095029348086\\
368.01	0.00507188517222661\\
369.01	0.0050728392412534\\
370.01	0.00507381289639131\\
371.01	0.00507480654164263\\
372.01	0.00507582058934995\\
373.01	0.00507685546036712\\
374.01	0.0050779115842346\\
375.01	0.00507898939935726\\
376.01	0.0050800893531875\\
377.01	0.00508121190241012\\
378.01	0.00508235751313217\\
379.01	0.00508352666107692\\
380.01	0.00508471983178049\\
381.01	0.00508593752079405\\
382.01	0.0050871802338892\\
383.01	0.00508844848726744\\
384.01	0.0050897428077748\\
385.01	0.00509106373312024\\
386.01	0.00509241181209835\\
387.01	0.00509378760481723\\
388.01	0.00509519168293052\\
389.01	0.00509662462987447\\
390.01	0.00509808704111012\\
391.01	0.00509957952436935\\
392.01	0.00510110269990748\\
393.01	0.00510265720076039\\
394.01	0.00510424367300651\\
395.01	0.00510586277603481\\
396.01	0.00510751518281828\\
397.01	0.0051092015801926\\
398.01	0.00511092266914112\\
399.01	0.00511267916508518\\
400.01	0.00511447179818102\\
401.01	0.00511630131362208\\
402.01	0.00511816847194851\\
403.01	0.00512007404936213\\
404.01	0.00512201883804843\\
405.01	0.00512400364650505\\
406.01	0.00512602929987734\\
407.01	0.00512809664030063\\
408.01	0.00513020652724942\\
409.01	0.00513235983789442\\
410.01	0.00513455746746602\\
411.01	0.00513680032962638\\
412.01	0.00513908935684861\\
413.01	0.00514142550080382\\
414.01	0.00514380973275642\\
415.01	0.00514624304396753\\
416.01	0.00514872644610682\\
417.01	0.00515126097167247\\
418.01	0.00515384767442111\\
419.01	0.00515648762980477\\
420.01	0.00515918193541862\\
421.01	0.00516193171145674\\
422.01	0.00516473810117871\\
423.01	0.00516760227138448\\
424.01	0.0051705254129001\\
425.01	0.00517350874107243\\
426.01	0.00517655349627588\\
427.01	0.00517966094442791\\
428.01	0.00518283237751574\\
429.01	0.00518606911413511\\
430.01	0.00518937250003826\\
431.01	0.00519274390869479\\
432.01	0.00519618474186347\\
433.01	0.00519969643017593\\
434.01	0.00520328043373308\\
435.01	0.00520693824271272\\
436.01	0.00521067137799075\\
437.01	0.00521448139177509\\
438.01	0.00521836986825194\\
439.01	0.00522233842424669\\
440.01	0.0052263887098974\\
441.01	0.0052305224093428\\
442.01	0.00523474124142474\\
443.01	0.00523904696040461\\
444.01	0.00524344135669504\\
445.01	0.00524792625760664\\
446.01	0.00525250352811009\\
447.01	0.00525717507161445\\
448.01	0.00526194283076127\\
449.01	0.00526680878823532\\
450.01	0.005271774967592\\
451.01	0.00527684343410191\\
452.01	0.00528201629561358\\
453.01	0.00528729570343331\\
454.01	0.00529268385322357\\
455.01	0.00529818298592039\\
456.01	0.00530379538866937\\
457.01	0.00530952339578142\\
458.01	0.00531536938970828\\
459.01	0.00532133580203822\\
460.01	0.0053274251145128\\
461.01	0.00533363986006455\\
462.01	0.00533998262387593\\
463.01	0.00534645604446064\\
464.01	0.00535306281476762\\
465.01	0.00535980568330775\\
466.01	0.00536668745530369\\
467.01	0.00537371099386401\\
468.01	0.00538087922118197\\
469.01	0.00538819511975855\\
470.01	0.00539566173365116\\
471.01	0.00540328216974893\\
472.01	0.00541105959907338\\
473.01	0.00541899725810693\\
474.01	0.00542709845014895\\
475.01	0.00543536654669927\\
476.01	0.00544380498887161\\
477.01	0.005452417288835\\
478.01	0.00546120703128584\\
479.01	0.00547017787495054\\
480.01	0.00547933355411916\\
481.01	0.00548867788021037\\
482.01	0.00549821474336881\\
483.01	0.00550794811409631\\
484.01	0.00551788204491581\\
485.01	0.00552802067206939\\
486.01	0.00553836821725223\\
487.01	0.00554892898938124\\
488.01	0.00555970738639985\\
489.01	0.00557070789712073\\
490.01	0.00558193510310437\\
491.01	0.00559339368057722\\
492.01	0.00560508840238808\\
493.01	0.00561702414000388\\
494.01	0.00562920586554576\\
495.01	0.00564163865386569\\
496.01	0.00565432768466423\\
497.01	0.00566727824465042\\
498.01	0.00568049572974454\\
499.01	0.00569398564732272\\
500.01	0.00570775361850585\\
501.01	0.00572180538049252\\
502.01	0.00573614678893519\\
503.01	0.00575078382036302\\
504.01	0.00576572257464746\\
505.01	0.00578096927751429\\
506.01	0.00579653028310122\\
507.01	0.00581241207655984\\
508.01	0.0058286212767039\\
509.01	0.00584516463870154\\
510.01	0.0058620490568133\\
511.01	0.00587928156717432\\
512.01	0.0058968693506196\\
513.01	0.00591481973555158\\
514.01	0.00593314020085063\\
515.01	0.00595183837882469\\
516.01	0.00597092205819774\\
517.01	0.00599039918713468\\
518.01	0.00601027787630053\\
519.01	0.00603056640195079\\
520.01	0.00605127320905013\\
521.01	0.00607240691441507\\
522.01	0.00609397630987676\\
523.01	0.00611599036545887\\
524.01	0.00613845823256408\\
525.01	0.00616138924716375\\
526.01	0.00618479293298224\\
527.01	0.00620867900466769\\
528.01	0.0062330573709399\\
529.01	0.00625793813770395\\
530.01	0.00628333161111733\\
531.01	0.00630924830059647\\
532.01	0.00633569892174679\\
533.01	0.00636269439919962\\
534.01	0.00639024586933368\\
535.01	0.00641836468286175\\
536.01	0.0064470624072543\\
537.01	0.00647635082897466\\
538.01	0.00650624195549183\\
539.01	0.00653674801703709\\
540.01	0.00656788146806421\\
541.01	0.00659965498836909\\
542.01	0.0066320814838192\\
543.01	0.00666517408663819\\
544.01	0.0066989461551837\\
545.01	0.00673341127314891\\
546.01	0.0067685832481139\\
547.01	0.0068044761093578\\
548.01	0.00684110410483998\\
549.01	0.00687848169724375\\
550.01	0.0069166235589634\\
551.01	0.00695554456590763\\
552.01	0.00699525978997213\\
553.01	0.00703578449001997\\
554.01	0.00707713410119494\\
555.01	0.00711932422236754\\
556.01	0.00716237060149713\\
557.01	0.00720628911866866\\
558.01	0.00725109576653823\\
559.01	0.00729680662789368\\
560.01	0.00734343785000768\\
561.01	0.00739100561542945\\
562.01	0.00743952610882453\\
563.01	0.00748901547943973\\
564.01	0.00753948979872679\\
565.01	0.00759096501262174\\
566.01	0.00764345688793244\\
567.01	0.00769698095224389\\
568.01	0.0077515524267073\\
569.01	0.00780718615103681\\
570.01	0.00786389649999601\\
571.01	0.00792169729062253\\
572.01	0.00798060167940823\\
573.01	0.00804062204863615\\
574.01	0.00810176988107205\\
575.01	0.00816405562222705\\
576.01	0.00822748852945653\\
577.01	0.00829207650724693\\
578.01	0.00835782592818142\\
579.01	0.0084247414392816\\
580.01	0.00849282575371788\\
581.01	0.0085620794282911\\
582.01	0.00863250062764312\\
583.01	0.00870408487690098\\
584.01	0.00877682480544311\\
585.01	0.00885070988576821\\
586.01	0.00892572617312741\\
587.01	0.00900185605375053\\
588.01	0.00907907801228958\\
589.01	0.00915736643267762\\
590.01	0.00923669145116018\\
591.01	0.00931701888605708\\
592.01	0.00939831027616845\\
593.01	0.00948052306904656\\
594.01	0.00956361101211311\\
595.01	0.00964752481442215\\
596.01	0.00973221316552893\\
597.01	0.00981762422138093\\
598.01	0.00990285624732339\\
599.01	0.00996919203046377\\
599.02	0.00996973314335038\\
599.03	0.0099702709571694\\
599.04	0.00997080543920336\\
599.05	0.0099713365564138\\
599.06	0.00997186427543808\\
599.07	0.0099723885625862\\
599.08	0.00997290938383756\\
599.09	0.00997342670483771\\
599.1	0.00997394049089505\\
599.11	0.00997445070697751\\
599.12	0.00997495731770918\\
599.13	0.00997546028736696\\
599.14	0.00997595957987707\\
599.15	0.00997645515881165\\
599.16	0.00997694698738526\\
599.17	0.00997743502845131\\
599.18	0.00997791924449856\\
599.19	0.00997839959764748\\
599.2	0.00997887604964662\\
599.21	0.00997934856186899\\
599.22	0.00997981709530829\\
599.23	0.00998028161057521\\
599.24	0.00998074206789365\\
599.25	0.0099811984270969\\
599.26	0.00998165064762378\\
599.27	0.00998209868851477\\
599.28	0.00998254250840806\\
599.29	0.00998298206553559\\
599.3	0.00998341731771905\\
599.31	0.00998384822236584\\
599.32	0.00998427473646497\\
599.33	0.00998469681658292\\
599.34	0.00998511441885952\\
599.35	0.0099855274990037\\
599.36	0.00998593601228926\\
599.37	0.00998633991355056\\
599.38	0.00998673915717821\\
599.39	0.00998713369711469\\
599.4	0.00998752348684992\\
599.41	0.00998790847811156\\
599.42	0.00998828861974491\\
599.43	0.00998866386008803\\
599.44	0.00998903414696681\\
599.45	0.00998939942768985\\
599.46	0.00998975964904344\\
599.47	0.00999011475728636\\
599.48	0.00999046469814471\\
599.49	0.00999080941680669\\
599.5	0.00999114885791725\\
599.51	0.00999148296557278\\
599.52	0.0099918116833157\\
599.53	0.00999213495412901\\
599.54	0.00999245272043077\\
599.55	0.00999276492406855\\
599.56	0.00999307150631381\\
599.57	0.00999337240785624\\
599.58	0.00999366756879799\\
599.59	0.00999395692864795\\
599.6	0.00999424042631585\\
599.61	0.00999451800010642\\
599.62	0.00999478958771336\\
599.63	0.0099950551262134\\
599.64	0.00999531455206019\\
599.65	0.00999556780107816\\
599.66	0.00999581480845634\\
599.67	0.00999605550874211\\
599.68	0.00999628983583487\\
599.69	0.00999651772297967\\
599.7	0.00999673910276076\\
599.71	0.00999695390709509\\
599.72	0.00999716206722577\\
599.73	0.00999736351371538\\
599.74	0.00999755817643933\\
599.75	0.00999774598457904\\
599.76	0.00999792686661517\\
599.77	0.00999810075032068\\
599.78	0.00999826756275386\\
599.79	0.00999842723025133\\
599.8	0.00999857967842092\\
599.81	0.0099987248321345\\
599.82	0.00999886261552071\\
599.83	0.00999899295195769\\
599.84	0.00999911576406567\\
599.85	0.00999923097369951\\
599.86	0.00999933850194117\\
599.87	0.00999943826909211\\
599.88	0.00999953019466558\\
599.89	0.00999961419737891\\
599.9	0.00999969019514566\\
599.91	0.00999975810506767\\
599.92	0.00999981784342713\\
599.93	0.00999986932567848\\
599.94	0.00999991246644028\\
599.95	0.00999994717948697\\
599.96	0.00999997337774056\\
599.97	0.00999999097326228\\
599.98	0.00999999987724406\\
599.99	0.01\\
600	0.01\\
};
\addplot [color=mycolor2,solid,forget plot]
  table[row sep=crcr]{%
0.01	0.00502433004032531\\
1.01	0.00502433078290415\\
2.01	0.00502433153996987\\
3.01	0.00502433231180427\\
4.01	0.00502433309869423\\
5.01	0.00502433390093257\\
6.01	0.00502433471881732\\
7.01	0.00502433555265265\\
8.01	0.00502433640274838\\
9.01	0.00502433726942045\\
10.01	0.00502433815299091\\
11.01	0.00502433905378807\\
12.01	0.00502433997214664\\
13.01	0.0050243409084075\\
14.01	0.00502434186291843\\
15.01	0.00502434283603398\\
16.01	0.00502434382811538\\
17.01	0.00502434483953081\\
18.01	0.00502434587065569\\
19.01	0.00502434692187287\\
20.01	0.00502434799357236\\
21.01	0.00502434908615175\\
22.01	0.00502435020001639\\
23.01	0.00502435133557945\\
24.01	0.00502435249326209\\
25.01	0.00502435367349358\\
26.01	0.00502435487671149\\
27.01	0.0050243561033619\\
28.01	0.00502435735389965\\
29.01	0.00502435862878807\\
30.01	0.00502435992849977\\
31.01	0.00502436125351623\\
32.01	0.00502436260432844\\
33.01	0.00502436398143678\\
34.01	0.00502436538535125\\
35.01	0.00502436681659189\\
36.01	0.00502436827568868\\
37.01	0.00502436976318171\\
38.01	0.00502437127962201\\
39.01	0.00502437282557061\\
40.01	0.00502437440159964\\
41.01	0.00502437600829278\\
42.01	0.00502437764624425\\
43.01	0.00502437931606002\\
44.01	0.00502438101835799\\
45.01	0.00502438275376783\\
46.01	0.00502438452293126\\
47.01	0.00502438632650271\\
48.01	0.00502438816514894\\
49.01	0.00502439003954984\\
50.01	0.0050243919503983\\
51.01	0.00502439389840066\\
52.01	0.00502439588427681\\
53.01	0.00502439790876083\\
54.01	0.00502439997260064\\
55.01	0.00502440207655885\\
56.01	0.00502440422141273\\
57.01	0.00502440640795451\\
58.01	0.00502440863699193\\
59.01	0.00502441090934809\\
60.01	0.00502441322586211\\
61.01	0.00502441558738942\\
62.01	0.0050244179948018\\
63.01	0.00502442044898818\\
64.01	0.00502442295085417\\
65.01	0.00502442550132326\\
66.01	0.00502442810133673\\
67.01	0.00502443075185414\\
68.01	0.00502443345385328\\
69.01	0.00502443620833122\\
70.01	0.00502443901630395\\
71.01	0.00502444187880739\\
72.01	0.00502444479689727\\
73.01	0.00502444777164984\\
74.01	0.00502445080416218\\
75.01	0.00502445389555228\\
76.01	0.00502445704696021\\
77.01	0.00502446025954768\\
78.01	0.00502446353449897\\
79.01	0.00502446687302142\\
80.01	0.00502447027634557\\
81.01	0.00502447374572551\\
82.01	0.00502447728244004\\
83.01	0.00502448088779259\\
84.01	0.00502448456311146\\
85.01	0.00502448830975082\\
86.01	0.0050244921290912\\
87.01	0.00502449602253954\\
88.01	0.00502449999152989\\
89.01	0.00502450403752445\\
90.01	0.00502450816201346\\
91.01	0.00502451236651598\\
92.01	0.00502451665258054\\
93.01	0.0050245210217854\\
94.01	0.00502452547573952\\
95.01	0.00502453001608305\\
96.01	0.00502453464448767\\
97.01	0.0050245393626575\\
98.01	0.00502454417232982\\
99.01	0.00502454907527516\\
100.01	0.00502455407329852\\
101.01	0.00502455916823986\\
102.01	0.00502456436197452\\
103.01	0.00502456965641446\\
104.01	0.00502457505350838\\
105.01	0.00502458055524285\\
106.01	0.00502458616364285\\
107.01	0.00502459188077253\\
108.01	0.00502459770873598\\
109.01	0.00502460364967827\\
110.01	0.00502460970578573\\
111.01	0.00502461587928722\\
112.01	0.00502462217245472\\
113.01	0.00502462858760408\\
114.01	0.00502463512709636\\
115.01	0.00502464179333817\\
116.01	0.00502464858878258\\
117.01	0.0050246555159305\\
118.01	0.00502466257733118\\
119.01	0.00502466977558323\\
120.01	0.00502467711333579\\
121.01	0.00502468459328946\\
122.01	0.00502469221819687\\
123.01	0.00502469999086443\\
124.01	0.0050247079141528\\
125.01	0.0050247159909782\\
126.01	0.00502472422431341\\
127.01	0.0050247326171889\\
128.01	0.0050247411726939\\
129.01	0.00502474989397785\\
130.01	0.00502475878425128\\
131.01	0.00502476784678701\\
132.01	0.00502477708492131\\
133.01	0.00502478650205564\\
134.01	0.00502479610165719\\
135.01	0.00502480588726081\\
136.01	0.0050248158624699\\
137.01	0.00502482603095807\\
138.01	0.00502483639647029\\
139.01	0.00502484696282447\\
140.01	0.00502485773391259\\
141.01	0.00502486871370277\\
142.01	0.00502487990624001\\
143.01	0.00502489131564829\\
144.01	0.00502490294613174\\
145.01	0.00502491480197648\\
146.01	0.00502492688755225\\
147.01	0.00502493920731371\\
148.01	0.00502495176580255\\
149.01	0.00502496456764895\\
150.01	0.00502497761757341\\
151.01	0.00502499092038851\\
152.01	0.00502500448100068\\
153.01	0.00502501830441233\\
154.01	0.00502503239572324\\
155.01	0.00502504676013321\\
156.01	0.0050250614029432\\
157.01	0.00502507632955788\\
158.01	0.00502509154548772\\
159.01	0.0050251070563508\\
160.01	0.00502512286787522\\
161.01	0.0050251389859009\\
162.01	0.00502515541638252\\
163.01	0.00502517216539074\\
164.01	0.00502518923911544\\
165.01	0.0050252066438678\\
166.01	0.00502522438608252\\
167.01	0.00502524247232077\\
168.01	0.00502526090927197\\
169.01	0.00502527970375694\\
170.01	0.00502529886273037\\
171.01	0.00502531839328352\\
172.01	0.00502533830264698\\
173.01	0.00502535859819302\\
174.01	0.00502537928743922\\
175.01	0.00502540037805077\\
176.01	0.00502542187784366\\
177.01	0.0050254437947877\\
178.01	0.00502546613700938\\
179.01	0.0050254889127955\\
180.01	0.00502551213059595\\
181.01	0.00502553579902699\\
182.01	0.00502555992687535\\
183.01	0.00502558452310024\\
184.01	0.00502560959683822\\
185.01	0.00502563515740609\\
186.01	0.00502566121430463\\
187.01	0.00502568777722227\\
188.01	0.00502571485603907\\
189.01	0.00502574246083047\\
190.01	0.00502577060187115\\
191.01	0.00502579928963928\\
192.01	0.00502582853482043\\
193.01	0.00502585834831201\\
194.01	0.00502588874122717\\
195.01	0.00502591972489998\\
196.01	0.00502595131088869\\
197.01	0.00502598351098166\\
198.01	0.00502601633720089\\
199.01	0.00502604980180751\\
200.01	0.00502608391730629\\
201.01	0.00502611869645081\\
202.01	0.0050261541522485\\
203.01	0.00502619029796553\\
204.01	0.0050262271471328\\
205.01	0.00502626471355053\\
206.01	0.00502630301129413\\
207.01	0.00502634205472018\\
208.01	0.00502638185847148\\
209.01	0.00502642243748359\\
210.01	0.00502646380699059\\
211.01	0.00502650598253092\\
212.01	0.00502654897995404\\
213.01	0.00502659281542676\\
214.01	0.00502663750543959\\
215.01	0.0050266830668135\\
216.01	0.00502672951670692\\
217.01	0.00502677687262224\\
218.01	0.00502682515241343\\
219.01	0.00502687437429299\\
220.01	0.00502692455683951\\
221.01	0.00502697571900508\\
222.01	0.00502702788012314\\
223.01	0.00502708105991628\\
224.01	0.00502713527850461\\
225.01	0.00502719055641328\\
226.01	0.00502724691458187\\
227.01	0.00502730437437221\\
228.01	0.00502736295757766\\
229.01	0.00502742268643143\\
230.01	0.00502748358361636\\
231.01	0.00502754567227415\\
232.01	0.00502760897601477\\
233.01	0.00502767351892637\\
234.01	0.00502773932558503\\
235.01	0.005027806421065\\
236.01	0.00502787483094894\\
237.01	0.00502794458133929\\
238.01	0.00502801569886797\\
239.01	0.00502808821070845\\
240.01	0.00502816214458639\\
241.01	0.0050282375287917\\
242.01	0.00502831439218987\\
243.01	0.00502839276423442\\
244.01	0.00502847267497887\\
245.01	0.00502855415508958\\
246.01	0.00502863723585828\\
247.01	0.00502872194921569\\
248.01	0.00502880832774445\\
249.01	0.00502889640469281\\
250.01	0.00502898621398901\\
251.01	0.00502907779025501\\
252.01	0.00502917116882139\\
253.01	0.00502926638574215\\
254.01	0.00502936347780983\\
255.01	0.00502946248257113\\
256.01	0.00502956343834272\\
257.01	0.00502966638422728\\
258.01	0.00502977136013015\\
259.01	0.00502987840677609\\
260.01	0.00502998756572667\\
261.01	0.00503009887939768\\
262.01	0.00503021239107712\\
263.01	0.00503032814494354\\
264.01	0.00503044618608481\\
265.01	0.00503056656051706\\
266.01	0.00503068931520437\\
267.01	0.00503081449807827\\
268.01	0.00503094215805883\\
269.01	0.00503107234507448\\
270.01	0.00503120511008398\\
271.01	0.00503134050509758\\
272.01	0.00503147858319909\\
273.01	0.00503161939856864\\
274.01	0.00503176300650546\\
275.01	0.00503190946345133\\
276.01	0.00503205882701463\\
277.01	0.00503221115599456\\
278.01	0.00503236651040657\\
279.01	0.0050325249515072\\
280.01	0.00503268654182049\\
281.01	0.00503285134516392\\
282.01	0.00503301942667632\\
283.01	0.00503319085284439\\
284.01	0.005033365691532\\
285.01	0.00503354401200766\\
286.01	0.00503372588497485\\
287.01	0.00503391138260131\\
288.01	0.00503410057854997\\
289.01	0.00503429354800942\\
290.01	0.00503449036772634\\
291.01	0.00503469111603755\\
292.01	0.00503489587290304\\
293.01	0.0050351047199394\\
294.01	0.00503531774045483\\
295.01	0.00503553501948349\\
296.01	0.00503575664382178\\
297.01	0.00503598270206433\\
298.01	0.0050362132846417\\
299.01	0.00503644848385791\\
300.01	0.00503668839392943\\
301.01	0.00503693311102424\\
302.01	0.00503718273330268\\
303.01	0.0050374373609581\\
304.01	0.00503769709625872\\
305.01	0.00503796204359043\\
306.01	0.00503823230950071\\
307.01	0.00503850800274236\\
308.01	0.00503878923431933\\
309.01	0.00503907611753321\\
310.01	0.00503936876802981\\
311.01	0.00503966730384769\\
312.01	0.00503997184546696\\
313.01	0.00504028251585979\\
314.01	0.00504059944054119\\
315.01	0.00504092274762133\\
316.01	0.0050412525678584\\
317.01	0.00504158903471354\\
318.01	0.00504193228440589\\
319.01	0.00504228245596902\\
320.01	0.00504263969130883\\
321.01	0.00504300413526243\\
322.01	0.00504337593565841\\
323.01	0.00504375524337804\\
324.01	0.00504414221241793\\
325.01	0.00504453699995414\\
326.01	0.00504493976640799\\
327.01	0.00504535067551151\\
328.01	0.00504576989437699\\
329.01	0.00504619759356566\\
330.01	0.00504663394715922\\
331.01	0.00504707913283256\\
332.01	0.00504753333192737\\
333.01	0.0050479967295282\\
334.01	0.00504846951454033\\
335.01	0.00504895187976848\\
336.01	0.00504944402199728\\
337.01	0.00504994614207464\\
338.01	0.00505045844499537\\
339.01	0.00505098113998795\\
340.01	0.00505151444060255\\
341.01	0.00505205856480088\\
342.01	0.00505261373504817\\
343.01	0.00505318017840738\\
344.01	0.00505375812663522\\
345.01	0.00505434781628047\\
346.01	0.00505494948878352\\
347.01	0.00505556339057943\\
348.01	0.0050561897732026\\
349.01	0.00505682889339344\\
350.01	0.00505748101320719\\
351.01	0.00505814640012613\\
352.01	0.0050588253271726\\
353.01	0.00505951807302557\\
354.01	0.00506022492213949\\
355.01	0.0050609461648646\\
356.01	0.00506168209757112\\
357.01	0.00506243302277486\\
358.01	0.0050631992492661\\
359.01	0.00506398109224075\\
360.01	0.00506477887343372\\
361.01	0.00506559292125611\\
362.01	0.00506642357093377\\
363.01	0.0050672711646495\\
364.01	0.00506813605168775\\
365.01	0.00506901858858232\\
366.01	0.00506991913926619\\
367.01	0.00507083807522581\\
368.01	0.00507177577565693\\
369.01	0.00507273262762459\\
370.01	0.0050737090262252\\
371.01	0.00507470537475273\\
372.01	0.00507572208486796\\
373.01	0.0050767595767709\\
374.01	0.00507781827937674\\
375.01	0.00507889863049542\\
376.01	0.00508000107701456\\
377.01	0.00508112607508641\\
378.01	0.00508227409031825\\
379.01	0.00508344559796643\\
380.01	0.00508464108313535\\
381.01	0.00508586104097879\\
382.01	0.00508710597690695\\
383.01	0.00508837640679624\\
384.01	0.0050896728572046\\
385.01	0.00509099586558949\\
386.01	0.00509234598053222\\
387.01	0.00509372376196553\\
388.01	0.00509512978140565\\
389.01	0.00509656462218988\\
390.01	0.0050980288797183\\
391.01	0.00509952316170056\\
392.01	0.00510104808840786\\
393.01	0.00510260429292941\\
394.01	0.00510419242143478\\
395.01	0.00510581313344144\\
396.01	0.00510746710208676\\
397.01	0.00510915501440734\\
398.01	0.00511087757162237\\
399.01	0.00511263548942352\\
400.01	0.00511442949827068\\
401.01	0.00511626034369345\\
402.01	0.0051181287865992\\
403.01	0.00512003560358712\\
404.01	0.00512198158726905\\
405.01	0.00512396754659643\\
406.01	0.00512599430719415\\
407.01	0.00512806271170159\\
408.01	0.00513017362012046\\
409.01	0.00513232791016949\\
410.01	0.00513452647764723\\
411.01	0.00513677023680156\\
412.01	0.00513906012070751\\
413.01	0.0051413970816525\\
414.01	0.00514378209152984\\
415.01	0.00514621614224034\\
416.01	0.00514870024610219\\
417.01	0.00515123543627008\\
418.01	0.00515382276716167\\
419.01	0.00515646331489446\\
420.01	0.00515915817773094\\
421.01	0.00516190847653376\\
422.01	0.00516471535522936\\
423.01	0.0051675799812823\\
424.01	0.00517050354617916\\
425.01	0.00517348726592291\\
426.01	0.00517653238153705\\
427.01	0.00517964015958057\\
428.01	0.00518281189267433\\
429.01	0.00518604890003725\\
430.01	0.00518935252803483\\
431.01	0.00519272415073881\\
432.01	0.00519616517049813\\
433.01	0.00519967701852253\\
434.01	0.00520326115547738\\
435.01	0.00520691907209199\\
436.01	0.00521065228978022\\
437.01	0.00521446236127313\\
438.01	0.00521835087126643\\
439.01	0.00522231943707985\\
440.01	0.0052263697093313\\
441.01	0.00523050337262421\\
442.01	0.00523472214624948\\
443.01	0.00523902778490222\\
444.01	0.00524342207941315\\
445.01	0.00524790685749485\\
446.01	0.00525248398450374\\
447.01	0.00525715536421802\\
448.01	0.00526192293963146\\
449.01	0.00526678869376357\\
450.01	0.00527175465048671\\
451.01	0.00527682287537118\\
452.01	0.00528199547654581\\
453.01	0.0052872746055784\\
454.01	0.00529266245837408\\
455.01	0.00529816127609135\\
456.01	0.00530377334607839\\
457.01	0.00530950100282836\\
458.01	0.00531534662895433\\
459.01	0.00532131265618531\\
460.01	0.00532740156638227\\
461.01	0.00533361589257574\\
462.01	0.0053399582200252\\
463.01	0.00534643118730034\\
464.01	0.0053530374873846\\
465.01	0.00535977986880204\\
466.01	0.00536666113676787\\
467.01	0.00537368415436255\\
468.01	0.00538085184373044\\
469.01	0.00538816718730339\\
470.01	0.00539563322904991\\
471.01	0.00540325307574993\\
472.01	0.00541102989829671\\
473.01	0.00541896693302511\\
474.01	0.00542706748306775\\
475.01	0.00543533491973951\\
476.01	0.00544377268394998\\
477.01	0.00545238428764663\\
478.01	0.00546117331528623\\
479.01	0.0054701434253375\\
480.01	0.0054792983518152\\
481.01	0.00548864190584508\\
482.01	0.00549817797726253\\
483.01	0.00550791053624219\\
484.01	0.00551784363496273\\
485.01	0.0055279814093054\\
486.01	0.00553832808058766\\
487.01	0.00554888795733173\\
488.01	0.0055596654370699\\
489.01	0.00557066500818651\\
490.01	0.00558189125179769\\
491.01	0.00559334884366847\\
492.01	0.00560504255616977\\
493.01	0.00561697726027399\\
494.01	0.00562915792759081\\
495.01	0.00564158963244404\\
496.01	0.00565427755398963\\
497.01	0.00566722697837492\\
498.01	0.00568044330094133\\
499.01	0.00569393202846911\\
500.01	0.00570769878146611\\
501.01	0.00572174929650032\\
502.01	0.00573608942857665\\
503.01	0.00575072515355798\\
504.01	0.00576566257063221\\
505.01	0.00578090790482362\\
506.01	0.00579646750954958\\
507.01	0.00581234786922323\\
508.01	0.00582855560190046\\
509.01	0.00584509746197324\\
510.01	0.00586198034290671\\
511.01	0.00587921128002072\\
512.01	0.00589679745331509\\
513.01	0.00591474619033792\\
514.01	0.00593306496909482\\
515.01	0.00595176142099829\\
516.01	0.0059708433338565\\
517.01	0.00599031865489785\\
518.01	0.00601019549382988\\
519.01	0.00603048212592962\\
520.01	0.00605118699516193\\
521.01	0.00607231871732236\\
522.01	0.0060938860831997\\
523.01	0.00611589806175381\\
524.01	0.00613836380330246\\
525.01	0.00616129264271094\\
526.01	0.00618469410257619\\
527.01	0.00620857789639841\\
528.01	0.006232953931729\\
529.01	0.00625783231328467\\
530.01	0.00628322334601478\\
531.01	0.00630913753810876\\
532.01	0.0063355856039277\\
533.01	0.00636257846684055\\
534.01	0.00639012726194826\\
535.01	0.00641824333867082\\
536.01	0.00644693826317301\\
537.01	0.00647622382060102\\
538.01	0.00650611201709725\\
539.01	0.00653661508155973\\
540.01	0.00656774546710449\\
541.01	0.00659951585218799\\
542.01	0.00663193914134093\\
543.01	0.00666502846545589\\
544.01	0.00669879718156972\\
545.01	0.00673325887207067\\
546.01	0.00676842734325422\\
547.01	0.00680431662314229\\
548.01	0.00684094095847129\\
549.01	0.00687831481074262\\
550.01	0.00691645285122013\\
551.01	0.00695536995474208\\
552.01	0.00699508119220391\\
553.01	0.00703560182155209\\
554.01	0.00707694727710953\\
555.01	0.00711913315703655\\
556.01	0.00716217520870922\\
557.01	0.00720608931177373\\
558.01	0.00725089145861135\\
559.01	0.0072965977319211\\
560.01	0.00734322427909608\\
561.01	0.00739078728304045\\
562.01	0.00743930292903837\\
563.01	0.00748878736724826\\
564.01	0.00753925667036025\\
565.01	0.00759072678591111\\
566.01	0.0076432134827101\\
567.01	0.00769673229078503\\
568.01	0.00775129843421511\\
569.01	0.00780692675617325\\
570.01	0.00786363163546109\\
571.01	0.0079214268937827\\
572.01	0.00798032569297568\\
573.01	0.00804034042139869\\
574.01	0.00810148256867251\\
575.01	0.00816376258798929\\
576.01	0.00822718974525323\\
577.01	0.00829177195440151\\
578.01	0.00835751559839206\\
579.01	0.00842442533555191\\
580.01	0.00849250389127248\\
581.01	0.00856175183544577\\
582.01	0.00863216734659208\\
583.01	0.00870374596436989\\
584.01	0.00877648033314344\\
585.01	0.00885035994056966\\
586.01	0.00892537085684264\\
587.01	0.0090014954823987\\
588.01	0.00907871231467107\\
589.01	0.0091569957480491\\
590.01	0.00923631592574849\\
591.01	0.00931663866808596\\
592.01	0.00939792550899438\\
593.01	0.00948013388190589\\
594.01	0.00956321750786664\\
595.01	0.00964712705354462\\
596.01	0.00973181114542127\\
597.01	0.00981721784988034\\
598.01	0.00990284350259786\\
599.01	0.00996919203046377\\
599.02	0.00996973314335038\\
599.03	0.0099702709571694\\
599.04	0.00997080543920336\\
599.05	0.0099713365564138\\
599.06	0.00997186427543808\\
599.07	0.0099723885625862\\
599.08	0.00997290938383756\\
599.09	0.00997342670483771\\
599.1	0.00997394049089505\\
599.11	0.00997445070697751\\
599.12	0.00997495731770918\\
599.13	0.00997546028736696\\
599.14	0.00997595957987707\\
599.15	0.00997645515881166\\
599.16	0.00997694698738526\\
599.17	0.00997743502845131\\
599.18	0.00997791924449856\\
599.19	0.00997839959764748\\
599.2	0.00997887604964662\\
599.21	0.00997934856186899\\
599.22	0.00997981709530829\\
599.23	0.00998028161057521\\
599.24	0.00998074206789365\\
599.25	0.0099811984270969\\
599.26	0.00998165064762378\\
599.27	0.00998209868851477\\
599.28	0.00998254250840806\\
599.29	0.00998298206553559\\
599.3	0.00998341731771905\\
599.31	0.00998384822236584\\
599.32	0.00998427473646497\\
599.33	0.00998469681658292\\
599.34	0.00998511441885952\\
599.35	0.0099855274990037\\
599.36	0.00998593601228926\\
599.37	0.00998633991355056\\
599.38	0.00998673915717821\\
599.39	0.00998713369711469\\
599.4	0.00998752348684992\\
599.41	0.00998790847811157\\
599.42	0.00998828861974491\\
599.43	0.00998866386008804\\
599.44	0.00998903414696681\\
599.45	0.00998939942768986\\
599.46	0.00998975964904344\\
599.47	0.00999011475728636\\
599.48	0.00999046469814472\\
599.49	0.00999080941680669\\
599.5	0.00999114885791725\\
599.51	0.00999148296557278\\
599.52	0.0099918116833157\\
599.53	0.00999213495412901\\
599.54	0.00999245272043077\\
599.55	0.00999276492406855\\
599.56	0.00999307150631381\\
599.57	0.00999337240785624\\
599.58	0.00999366756879799\\
599.59	0.00999395692864795\\
599.6	0.00999424042631586\\
599.61	0.00999451800010642\\
599.62	0.00999478958771336\\
599.63	0.0099950551262134\\
599.64	0.00999531455206019\\
599.65	0.00999556780107816\\
599.66	0.00999581480845634\\
599.67	0.00999605550874211\\
599.68	0.00999628983583487\\
599.69	0.00999651772297967\\
599.7	0.00999673910276076\\
599.71	0.00999695390709509\\
599.72	0.00999716206722577\\
599.73	0.00999736351371538\\
599.74	0.00999755817643933\\
599.75	0.00999774598457904\\
599.76	0.00999792686661517\\
599.77	0.00999810075032068\\
599.78	0.00999826756275386\\
599.79	0.00999842723025133\\
599.8	0.00999857967842092\\
599.81	0.0099987248321345\\
599.82	0.00999886261552071\\
599.83	0.00999899295195769\\
599.84	0.00999911576406567\\
599.85	0.00999923097369951\\
599.86	0.00999933850194117\\
599.87	0.00999943826909211\\
599.88	0.00999953019466558\\
599.89	0.00999961419737891\\
599.9	0.00999969019514566\\
599.91	0.00999975810506767\\
599.92	0.00999981784342713\\
599.93	0.00999986932567848\\
599.94	0.00999991246644028\\
599.95	0.00999994717948697\\
599.96	0.00999997337774056\\
599.97	0.00999999097326229\\
599.98	0.00999999987724406\\
599.99	0.01\\
600	0.01\\
};
\addplot [color=mycolor3,solid,forget plot]
  table[row sep=crcr]{%
0.01	0.00502269192807735\\
1.01	0.00502269275816619\\
2.01	0.00502269360447714\\
3.01	0.00502269446732508\\
4.01	0.00502269534703129\\
5.01	0.00502269624392311\\
6.01	0.00502269715833426\\
7.01	0.00502269809060478\\
8.01	0.00502269904108132\\
9.01	0.00502270001011726\\
10.01	0.00502270099807273\\
11.01	0.00502270200531477\\
12.01	0.00502270303221747\\
13.01	0.00502270407916229\\
14.01	0.00502270514653784\\
15.01	0.00502270623474015\\
16.01	0.00502270734417315\\
17.01	0.00502270847524855\\
18.01	0.00502270962838561\\
19.01	0.00502271080401182\\
20.01	0.00502271200256298\\
21.01	0.00502271322448329\\
22.01	0.00502271447022537\\
23.01	0.00502271574025047\\
24.01	0.00502271703502895\\
25.01	0.00502271835504012\\
26.01	0.00502271970077223\\
27.01	0.00502272107272335\\
28.01	0.00502272247140077\\
29.01	0.00502272389732167\\
30.01	0.0050227253510132\\
31.01	0.00502272683301236\\
32.01	0.00502272834386676\\
33.01	0.00502272988413437\\
34.01	0.00502273145438391\\
35.01	0.00502273305519495\\
36.01	0.00502273468715808\\
37.01	0.00502273635087555\\
38.01	0.00502273804696077\\
39.01	0.00502273977603926\\
40.01	0.00502274153874836\\
41.01	0.00502274333573718\\
42.01	0.00502274516766807\\
43.01	0.00502274703521559\\
44.01	0.00502274893906715\\
45.01	0.00502275087992363\\
46.01	0.00502275285849908\\
47.01	0.00502275487552123\\
48.01	0.00502275693173183\\
49.01	0.00502275902788675\\
50.01	0.00502276116475632\\
51.01	0.00502276334312574\\
52.01	0.00502276556379504\\
53.01	0.00502276782757959\\
54.01	0.00502277013531031\\
55.01	0.00502277248783412\\
56.01	0.00502277488601388\\
57.01	0.00502277733072924\\
58.01	0.00502277982287644\\
59.01	0.00502278236336892\\
60.01	0.00502278495313754\\
61.01	0.00502278759313086\\
62.01	0.00502279028431567\\
63.01	0.00502279302767693\\
64.01	0.00502279582421873\\
65.01	0.00502279867496436\\
66.01	0.0050228015809561\\
67.01	0.00502280454325642\\
68.01	0.00502280756294829\\
69.01	0.00502281064113506\\
70.01	0.00502281377894102\\
71.01	0.0050228169775119\\
72.01	0.00502282023801544\\
73.01	0.00502282356164153\\
74.01	0.00502282694960268\\
75.01	0.00502283040313457\\
76.01	0.00502283392349639\\
77.01	0.00502283751197109\\
78.01	0.00502284116986636\\
79.01	0.00502284489851453\\
80.01	0.00502284869927325\\
81.01	0.00502285257352628\\
82.01	0.00502285652268339\\
83.01	0.00502286054818123\\
84.01	0.00502286465148379\\
85.01	0.00502286883408294\\
86.01	0.00502287309749881\\
87.01	0.00502287744328052\\
88.01	0.00502288187300651\\
89.01	0.00502288638828523\\
90.01	0.00502289099075573\\
91.01	0.00502289568208819\\
92.01	0.00502290046398435\\
93.01	0.00502290533817826\\
94.01	0.00502291030643741\\
95.01	0.00502291537056186\\
96.01	0.00502292053238662\\
97.01	0.00502292579378125\\
98.01	0.00502293115665067\\
99.01	0.00502293662293596\\
100.01	0.00502294219461531\\
101.01	0.00502294787370401\\
102.01	0.0050229536622559\\
103.01	0.00502295956236332\\
104.01	0.00502296557615868\\
105.01	0.00502297170581455\\
106.01	0.00502297795354477\\
107.01	0.00502298432160513\\
108.01	0.00502299081229411\\
109.01	0.00502299742795358\\
110.01	0.00502300417096997\\
111.01	0.00502301104377507\\
112.01	0.00502301804884617\\
113.01	0.00502302518870795\\
114.01	0.00502303246593248\\
115.01	0.00502303988314082\\
116.01	0.00502304744300355\\
117.01	0.00502305514824167\\
118.01	0.00502306300162776\\
119.01	0.0050230710059868\\
120.01	0.00502307916419732\\
121.01	0.00502308747919222\\
122.01	0.00502309595395981\\
123.01	0.00502310459154491\\
124.01	0.00502311339505002\\
125.01	0.00502312236763623\\
126.01	0.00502313151252447\\
127.01	0.00502314083299653\\
128.01	0.00502315033239616\\
129.01	0.00502316001413047\\
130.01	0.00502316988167072\\
131.01	0.00502317993855393\\
132.01	0.00502319018838419\\
133.01	0.00502320063483327\\
134.01	0.00502321128164261\\
135.01	0.00502322213262422\\
136.01	0.00502323319166223\\
137.01	0.00502324446271411\\
138.01	0.00502325594981212\\
139.01	0.00502326765706478\\
140.01	0.0050232795886581\\
141.01	0.0050232917488572\\
142.01	0.00502330414200786\\
143.01	0.00502331677253788\\
144.01	0.00502332964495891\\
145.01	0.00502334276386766\\
146.01	0.00502335613394771\\
147.01	0.00502336975997114\\
148.01	0.0050233836468003\\
149.01	0.0050233977993893\\
150.01	0.00502341222278592\\
151.01	0.0050234269221333\\
152.01	0.00502344190267173\\
153.01	0.00502345716974069\\
154.01	0.00502347272878047\\
155.01	0.00502348858533407\\
156.01	0.00502350474504953\\
157.01	0.00502352121368163\\
158.01	0.00502353799709374\\
159.01	0.00502355510126019\\
160.01	0.00502357253226837\\
161.01	0.0050235902963208\\
162.01	0.00502360839973709\\
163.01	0.00502362684895667\\
164.01	0.00502364565054074\\
165.01	0.00502366481117458\\
166.01	0.00502368433766998\\
167.01	0.00502370423696777\\
168.01	0.00502372451614012\\
169.01	0.00502374518239317\\
170.01	0.0050237662430692\\
171.01	0.00502378770564997\\
172.01	0.00502380957775894\\
173.01	0.00502383186716404\\
174.01	0.00502385458178017\\
175.01	0.0050238777296725\\
176.01	0.00502390131905903\\
177.01	0.00502392535831374\\
178.01	0.00502394985596955\\
179.01	0.00502397482072109\\
180.01	0.0050240002614282\\
181.01	0.00502402618711899\\
182.01	0.00502405260699289\\
183.01	0.00502407953042421\\
184.01	0.00502410696696538\\
185.01	0.0050241349263505\\
186.01	0.00502416341849879\\
187.01	0.00502419245351826\\
188.01	0.00502422204170916\\
189.01	0.00502425219356792\\
190.01	0.00502428291979097\\
191.01	0.00502431423127846\\
192.01	0.00502434613913834\\
193.01	0.00502437865469038\\
194.01	0.00502441178947051\\
195.01	0.00502444555523426\\
196.01	0.00502447996396256\\
197.01	0.0050245150278644\\
198.01	0.00502455075938266\\
199.01	0.00502458717119787\\
200.01	0.00502462427623343\\
201.01	0.00502466208765993\\
202.01	0.00502470061890026\\
203.01	0.00502473988363467\\
204.01	0.00502477989580546\\
205.01	0.00502482066962262\\
206.01	0.00502486221956912\\
207.01	0.00502490456040573\\
208.01	0.00502494770717721\\
209.01	0.00502499167521754\\
210.01	0.00502503648015548\\
211.01	0.00502508213792142\\
212.01	0.00502512866475206\\
213.01	0.00502517607719771\\
214.01	0.0050252243921276\\
215.01	0.00502527362673716\\
216.01	0.00502532379855358\\
217.01	0.00502537492544335\\
218.01	0.00502542702561851\\
219.01	0.00502548011764381\\
220.01	0.00502553422044383\\
221.01	0.00502558935331021\\
222.01	0.00502564553590917\\
223.01	0.00502570278828886\\
224.01	0.00502576113088737\\
225.01	0.00502582058454044\\
226.01	0.00502588117048948\\
227.01	0.00502594291039012\\
228.01	0.0050260058263203\\
229.01	0.00502606994078937\\
230.01	0.00502613527674647\\
231.01	0.00502620185758955\\
232.01	0.00502626970717483\\
233.01	0.00502633884982615\\
234.01	0.00502640931034432\\
235.01	0.0050264811140172\\
236.01	0.00502655428662993\\
237.01	0.00502662885447454\\
238.01	0.00502670484436112\\
239.01	0.00502678228362818\\
240.01	0.00502686120015372\\
241.01	0.00502694162236637\\
242.01	0.005027023579257\\
243.01	0.0050271071003903\\
244.01	0.00502719221591665\\
245.01	0.00502727895658453\\
246.01	0.00502736735375309\\
247.01	0.00502745743940455\\
248.01	0.00502754924615793\\
249.01	0.00502764280728168\\
250.01	0.00502773815670789\\
251.01	0.00502783532904608\\
252.01	0.00502793435959722\\
253.01	0.00502803528436875\\
254.01	0.00502813814008939\\
255.01	0.00502824296422411\\
256.01	0.0050283497949901\\
257.01	0.00502845867137251\\
258.01	0.00502856963314093\\
259.01	0.00502868272086569\\
260.01	0.00502879797593533\\
261.01	0.00502891544057366\\
262.01	0.00502903515785794\\
263.01	0.0050291571717366\\
264.01	0.00502928152704825\\
265.01	0.00502940826954061\\
266.01	0.00502953744588969\\
267.01	0.00502966910371992\\
268.01	0.00502980329162449\\
269.01	0.00502994005918601\\
270.01	0.00503007945699756\\
271.01	0.00503022153668439\\
272.01	0.00503036635092638\\
273.01	0.00503051395348038\\
274.01	0.00503066439920342\\
275.01	0.00503081774407646\\
276.01	0.00503097404522812\\
277.01	0.00503113336095971\\
278.01	0.00503129575077038\\
279.01	0.0050314612753828\\
280.01	0.00503162999676922\\
281.01	0.00503180197817885\\
282.01	0.00503197728416456\\
283.01	0.00503215598061148\\
284.01	0.00503233813476518\\
285.01	0.00503252381526131\\
286.01	0.00503271309215458\\
287.01	0.00503290603695017\\
288.01	0.00503310272263373\\
289.01	0.00503330322370403\\
290.01	0.00503350761620454\\
291.01	0.00503371597775681\\
292.01	0.00503392838759395\\
293.01	0.00503414492659505\\
294.01	0.00503436567732025\\
295.01	0.00503459072404615\\
296.01	0.00503482015280256\\
297.01	0.00503505405140962\\
298.01	0.00503529250951544\\
299.01	0.00503553561863494\\
300.01	0.005035783472189\\
301.01	0.005036036165545\\
302.01	0.00503629379605716\\
303.01	0.00503655646310849\\
304.01	0.00503682426815327\\
305.01	0.00503709731476035\\
306.01	0.00503737570865702\\
307.01	0.00503765955777422\\
308.01	0.005037948972292\\
309.01	0.00503824406468613\\
310.01	0.00503854494977575\\
311.01	0.00503885174477148\\
312.01	0.00503916456932479\\
313.01	0.00503948354557794\\
314.01	0.00503980879821511\\
315.01	0.00504014045451451\\
316.01	0.0050404786444011\\
317.01	0.00504082350050034\\
318.01	0.00504117515819326\\
319.01	0.00504153375567223\\
320.01	0.00504189943399799\\
321.01	0.00504227233715752\\
322.01	0.00504265261212303\\
323.01	0.00504304040891196\\
324.01	0.00504343588064847\\
325.01	0.00504383918362587\\
326.01	0.00504425047736964\\
327.01	0.00504466992470332\\
328.01	0.00504509769181376\\
329.01	0.00504553394831879\\
330.01	0.00504597886733602\\
331.01	0.00504643262555302\\
332.01	0.00504689540329863\\
333.01	0.00504736738461671\\
334.01	0.00504784875733993\\
335.01	0.00504833971316666\\
336.01	0.0050488404477388\\
337.01	0.00504935116072139\\
338.01	0.00504987205588395\\
339.01	0.00505040334118401\\
340.01	0.00505094522885204\\
341.01	0.00505149793547943\\
342.01	0.00505206168210733\\
343.01	0.00505263669431827\\
344.01	0.00505322320233027\\
345.01	0.00505382144109255\\
346.01	0.00505443165038472\\
347.01	0.00505505407491719\\
348.01	0.0050556889644352\\
349.01	0.00505633657382473\\
350.01	0.0050569971632219\\
351.01	0.00505767099812442\\
352.01	0.00505835834950605\\
353.01	0.00505905949393413\\
354.01	0.00505977471368994\\
355.01	0.00506050429689232\\
356.01	0.00506124853762378\\
357.01	0.0050620077360596\\
358.01	0.00506278219860108\\
359.01	0.00506357223801002\\
360.01	0.00506437817354806\\
361.01	0.00506520033111743\\
362.01	0.0050660390434064\\
363.01	0.00506689465003635\\
364.01	0.00506776749771209\\
365.01	0.00506865794037575\\
366.01	0.00506956633936299\\
367.01	0.00507049306356233\\
368.01	0.00507143848957758\\
369.01	0.00507240300189217\\
370.01	0.00507338699303843\\
371.01	0.00507439086376758\\
372.01	0.00507541502322463\\
373.01	0.00507645988912479\\
374.01	0.00507752588793453\\
375.01	0.00507861345505498\\
376.01	0.0050797230350091\\
377.01	0.00508085508163242\\
378.01	0.00508201005826769\\
379.01	0.00508318843796281\\
380.01	0.00508439070367284\\
381.01	0.00508561734846642\\
382.01	0.00508686887573492\\
383.01	0.00508814579940786\\
384.01	0.00508944864417014\\
385.01	0.00509077794568545\\
386.01	0.00509213425082232\\
387.01	0.00509351811788605\\
388.01	0.00509493011685437\\
389.01	0.0050963708296172\\
390.01	0.00509784085022219\\
391.01	0.00509934078512408\\
392.01	0.00510087125343914\\
393.01	0.00510243288720465\\
394.01	0.00510402633164277\\
395.01	0.00510565224543046\\
396.01	0.0051073113009738\\
397.01	0.00510900418468772\\
398.01	0.00511073159728134\\
399.01	0.00511249425404832\\
400.01	0.00511429288516371\\
401.01	0.0051161282359854\\
402.01	0.00511800106736243\\
403.01	0.00511991215594831\\
404.01	0.00512186229452122\\
405.01	0.00512385229231005\\
406.01	0.00512588297532704\\
407.01	0.00512795518670606\\
408.01	0.00513006978704923\\
409.01	0.00513222765477883\\
410.01	0.00513442968649672\\
411.01	0.00513667679735155\\
412.01	0.00513896992141193\\
413.01	0.00514131001204859\\
414.01	0.00514369804232345\\
415.01	0.00514613500538642\\
416.01	0.005148621914881\\
417.01	0.00515115980535756\\
418.01	0.00515374973269582\\
419.01	0.00515639277453533\\
420.01	0.00515909003071588\\
421.01	0.00516184262372579\\
422.01	0.00516465169916107\\
423.01	0.00516751842619395\\
424.01	0.00517044399805046\\
425.01	0.00517342963249932\\
426.01	0.00517647657235074\\
427.01	0.00517958608596648\\
428.01	0.0051827594677801\\
429.01	0.00518599803882874\\
430.01	0.00518930314729744\\
431.01	0.00519267616907353\\
432.01	0.00519611850831445\\
433.01	0.00519963159802705\\
434.01	0.00520321690065996\\
435.01	0.00520687590870869\\
436.01	0.00521061014533324\\
437.01	0.00521442116498987\\
438.01	0.00521831055407579\\
439.01	0.00522227993158805\\
440.01	0.00522633094979617\\
441.01	0.00523046529492927\\
442.01	0.00523468468787772\\
443.01	0.00523899088490927\\
444.01	0.00524338567839996\\
445.01	0.00524787089758146\\
446.01	0.00525244840930239\\
447.01	0.00525712011880645\\
448.01	0.0052618879705259\\
449.01	0.00526675394889162\\
450.01	0.00527172007916012\\
451.01	0.00527678842825666\\
452.01	0.0052819611056364\\
453.01	0.00528724026416325\\
454.01	0.00529262810100554\\
455.01	0.00529812685855215\\
456.01	0.00530373882534568\\
457.01	0.00530946633703643\\
458.01	0.00531531177735521\\
459.01	0.00532127757910709\\
460.01	0.00532736622518554\\
461.01	0.00533358024960859\\
462.01	0.00533992223857581\\
463.01	0.00534639483154867\\
464.01	0.00535300072235285\\
465.01	0.00535974266030467\\
466.01	0.00536662345136033\\
467.01	0.00537364595929006\\
468.01	0.00538081310687699\\
469.01	0.00538812787714035\\
470.01	0.00539559331458557\\
471.01	0.00540321252647991\\
472.01	0.00541098868415421\\
473.01	0.00541892502433317\\
474.01	0.00542702485049187\\
475.01	0.00543529153424156\\
476.01	0.00544372851674385\\
477.01	0.00545233931015376\\
478.01	0.00546112749909385\\
479.01	0.00547009674215762\\
480.01	0.00547925077344463\\
481.01	0.0054885934041268\\
482.01	0.00549812852404753\\
483.01	0.00550786010335333\\
484.01	0.00551779219415929\\
485.01	0.00552792893224861\\
486.01	0.0055382745388064\\
487.01	0.00554883332219024\\
488.01	0.00555960967973582\\
489.01	0.00557060809959947\\
490.01	0.00558183316263904\\
491.01	0.00559328954433222\\
492.01	0.00560498201673348\\
493.01	0.00561691545047095\\
494.01	0.00562909481678298\\
495.01	0.00564152518959594\\
496.01	0.00565421174764243\\
497.01	0.0056671597766221\\
498.01	0.00568037467140432\\
499.01	0.00569386193827418\\
500.01	0.00570762719722163\\
501.01	0.00572167618427445\\
502.01	0.00573601475387593\\
503.01	0.00575064888130682\\
504.01	0.00576558466515187\\
505.01	0.00578082832981236\\
506.01	0.0057963862280633\\
507.01	0.00581226484365609\\
508.01	0.00582847079396645\\
509.01	0.00584501083268776\\
510.01	0.00586189185256834\\
511.01	0.00587912088819392\\
512.01	0.0058967051188127\\
513.01	0.00591465187120406\\
514.01	0.00593296862258812\\
515.01	0.00595166300357612\\
516.01	0.0059707428011586\\
517.01	0.00599021596173088\\
518.01	0.00601009059415211\\
519.01	0.00603037497283576\\
520.01	0.00605107754086892\\
521.01	0.0060722069131553\\
522.01	0.00609377187957812\\
523.01	0.00611578140817863\\
524.01	0.00613824464834352\\
525.01	0.00616117093399438\\
526.01	0.00618456978677384\\
527.01	0.00620845091921686\\
528.01	0.00623282423789978\\
529.01	0.00625769984655496\\
530.01	0.00628308804914007\\
531.01	0.00630899935284569\\
532.01	0.00633544447102788\\
533.01	0.00636243432604691\\
534.01	0.00638998005199212\\
535.01	0.00641809299727018\\
536.01	0.00644678472703378\\
537.01	0.00647606702541922\\
538.01	0.00650595189756438\\
539.01	0.00653645157136962\\
540.01	0.00656757849896372\\
541.01	0.00659934535782969\\
542.01	0.00663176505154073\\
543.01	0.00666485071005252\\
544.01	0.00669861568948972\\
545.01	0.00673307357135754\\
546.01	0.00676823816110224\\
547.01	0.00680412348593676\\
548.01	0.00684074379183394\\
549.01	0.00687811353958502\\
550.01	0.00691624739980339\\
551.01	0.0069551602467454\\
552.01	0.0069948671508024\\
553.01	0.00703538336950402\\
554.01	0.00707672433685472\\
555.01	0.00711890565080693\\
556.01	0.00716194305865242\\
557.01	0.00720585244009135\\
558.01	0.00725064978771305\\
559.01	0.00729635118459537\\
560.01	0.00734297277870056\\
561.01	0.00739053075371283\\
562.01	0.00743904129592875\\
563.01	0.00748852055677669\\
564.01	0.00753898461049995\\
565.01	0.00759044940649993\\
566.01	0.00764293071579204\\
567.01	0.00769644407098446\\
568.01	0.00775100469914526\\
569.01	0.00780662744688079\\
570.01	0.0078633266969083\\
571.01	0.00792111627536951\\
572.01	0.00798000934910111\\
573.01	0.00804001831206291\\
574.01	0.00810115466011737\\
575.01	0.0081634288533764\\
576.01	0.00822685016537466\\
577.01	0.00829142651841632\\
578.01	0.00835716430458038\\
579.01	0.00842406819207157\\
580.01	0.00849214091689975\\
581.01	0.0085613830602753\\
582.01	0.00863179281266063\\
583.01	0.00870336572616018\\
584.01	0.00877609445790773\\
585.01	0.00884996850839709\\
586.01	0.00892497396037102\\
587.01	0.00900109322604318\\
588.01	0.00907830481320621\\
589.01	0.00915658312433641\\
590.01	0.00923589830734327\\
591.01	0.00931621618238686\\
592.01	0.00939749827650748\\
593.01	0.00947970200708216\\
594.01	0.00956278106682799\\
595.01	0.0096466860778335\\
596.01	0.00973136560068165\\
597.01	0.00981676760809251\\
598.01	0.00990268918671883\\
599.01	0.00996919203046377\\
599.02	0.00996973314335038\\
599.03	0.0099702709571694\\
599.04	0.00997080543920336\\
599.05	0.0099713365564138\\
599.06	0.00997186427543808\\
599.07	0.0099723885625862\\
599.08	0.00997290938383756\\
599.09	0.00997342670483771\\
599.1	0.00997394049089505\\
599.11	0.00997445070697751\\
599.12	0.00997495731770918\\
599.13	0.00997546028736696\\
599.14	0.00997595957987707\\
599.15	0.00997645515881165\\
599.16	0.00997694698738526\\
599.17	0.00997743502845131\\
599.18	0.00997791924449856\\
599.19	0.00997839959764748\\
599.2	0.00997887604964662\\
599.21	0.00997934856186899\\
599.22	0.00997981709530829\\
599.23	0.00998028161057521\\
599.24	0.00998074206789365\\
599.25	0.0099811984270969\\
599.26	0.00998165064762378\\
599.27	0.00998209868851477\\
599.28	0.00998254250840806\\
599.29	0.00998298206553559\\
599.3	0.00998341731771905\\
599.31	0.00998384822236584\\
599.32	0.00998427473646497\\
599.33	0.00998469681658292\\
599.34	0.00998511441885952\\
599.35	0.0099855274990037\\
599.36	0.00998593601228926\\
599.37	0.00998633991355056\\
599.38	0.00998673915717821\\
599.39	0.00998713369711469\\
599.4	0.00998752348684992\\
599.41	0.00998790847811157\\
599.42	0.00998828861974491\\
599.43	0.00998866386008804\\
599.44	0.00998903414696681\\
599.45	0.00998939942768985\\
599.46	0.00998975964904344\\
599.47	0.00999011475728636\\
599.48	0.00999046469814472\\
599.49	0.00999080941680669\\
599.5	0.00999114885791725\\
599.51	0.00999148296557278\\
599.52	0.0099918116833157\\
599.53	0.00999213495412901\\
599.54	0.00999245272043077\\
599.55	0.00999276492406855\\
599.56	0.00999307150631382\\
599.57	0.00999337240785624\\
599.58	0.00999366756879799\\
599.59	0.00999395692864795\\
599.6	0.00999424042631585\\
599.61	0.00999451800010642\\
599.62	0.00999478958771336\\
599.63	0.0099950551262134\\
599.64	0.00999531455206019\\
599.65	0.00999556780107816\\
599.66	0.00999581480845634\\
599.67	0.00999605550874211\\
599.68	0.00999628983583487\\
599.69	0.00999651772297967\\
599.7	0.00999673910276076\\
599.71	0.00999695390709509\\
599.72	0.00999716206722577\\
599.73	0.00999736351371538\\
599.74	0.00999755817643933\\
599.75	0.00999774598457904\\
599.76	0.00999792686661517\\
599.77	0.00999810075032068\\
599.78	0.00999826756275386\\
599.79	0.00999842723025133\\
599.8	0.00999857967842092\\
599.81	0.0099987248321345\\
599.82	0.00999886261552071\\
599.83	0.00999899295195769\\
599.84	0.00999911576406567\\
599.85	0.00999923097369951\\
599.86	0.00999933850194117\\
599.87	0.00999943826909211\\
599.88	0.00999953019466558\\
599.89	0.00999961419737891\\
599.9	0.00999969019514566\\
599.91	0.00999975810506767\\
599.92	0.00999981784342713\\
599.93	0.00999986932567848\\
599.94	0.00999991246644028\\
599.95	0.00999994717948697\\
599.96	0.00999997337774056\\
599.97	0.00999999097326228\\
599.98	0.00999999987724406\\
599.99	0.01\\
600	0.01\\
};
\addplot [color=mycolor4,solid,forget plot]
  table[row sep=crcr]{%
0.01	0.0050190636862812\\
1.01	0.00501906463782538\\
2.01	0.00501906560805189\\
3.01	0.00501906659732479\\
4.01	0.00501906760601505\\
5.01	0.00501906863450116\\
6.01	0.00501906968316851\\
7.01	0.00501907075241023\\
8.01	0.00501907184262679\\
9.01	0.00501907295422666\\
10.01	0.00501907408762585\\
11.01	0.00501907524324856\\
12.01	0.00501907642152723\\
13.01	0.00501907762290241\\
14.01	0.00501907884782346\\
15.01	0.00501908009674813\\
16.01	0.00501908137014302\\
17.01	0.00501908266848361\\
18.01	0.00501908399225469\\
19.01	0.00501908534195042\\
20.01	0.0050190867180743\\
21.01	0.00501908812113945\\
22.01	0.00501908955166902\\
23.01	0.00501909101019615\\
24.01	0.0050190924972642\\
25.01	0.0050190940134268\\
26.01	0.00501909555924866\\
27.01	0.00501909713530482\\
28.01	0.00501909874218165\\
29.01	0.00501910038047652\\
30.01	0.00501910205079835\\
31.01	0.00501910375376788\\
32.01	0.00501910549001755\\
33.01	0.005019107260192\\
34.01	0.00501910906494818\\
35.01	0.00501911090495559\\
36.01	0.00501911278089671\\
37.01	0.00501911469346675\\
38.01	0.00501911664337441\\
39.01	0.0050191186313419\\
40.01	0.0050191206581052\\
41.01	0.00501912272441454\\
42.01	0.00501912483103419\\
43.01	0.0050191269787432\\
44.01	0.0050191291683354\\
45.01	0.00501913140061969\\
46.01	0.00501913367642064\\
47.01	0.00501913599657836\\
48.01	0.00501913836194895\\
49.01	0.005019140773405\\
50.01	0.00501914323183551\\
51.01	0.00501914573814656\\
52.01	0.00501914829326133\\
53.01	0.00501915089812062\\
54.01	0.00501915355368324\\
55.01	0.00501915626092608\\
56.01	0.00501915902084474\\
57.01	0.0050191618344535\\
58.01	0.00501916470278627\\
59.01	0.00501916762689629\\
60.01	0.00501917060785684\\
61.01	0.00501917364676173\\
62.01	0.00501917674472523\\
63.01	0.00501917990288314\\
64.01	0.00501918312239248\\
65.01	0.00501918640443227\\
66.01	0.00501918975020393\\
67.01	0.00501919316093166\\
68.01	0.00501919663786256\\
69.01	0.00501920018226763\\
70.01	0.0050192037954418\\
71.01	0.00501920747870464\\
72.01	0.00501921123340041\\
73.01	0.00501921506089889\\
74.01	0.00501921896259575\\
75.01	0.00501922293991293\\
76.01	0.00501922699429926\\
77.01	0.00501923112723107\\
78.01	0.00501923534021226\\
79.01	0.00501923963477527\\
80.01	0.00501924401248134\\
81.01	0.0050192484749212\\
82.01	0.00501925302371536\\
83.01	0.00501925766051513\\
84.01	0.00501926238700277\\
85.01	0.00501926720489194\\
86.01	0.00501927211592888\\
87.01	0.00501927712189236\\
88.01	0.0050192822245949\\
89.01	0.00501928742588281\\
90.01	0.00501929272763717\\
91.01	0.00501929813177436\\
92.01	0.00501930364024664\\
93.01	0.00501930925504299\\
94.01	0.00501931497818932\\
95.01	0.00501932081175031\\
96.01	0.00501932675782853\\
97.01	0.00501933281856613\\
98.01	0.00501933899614557\\
99.01	0.00501934529278994\\
100.01	0.0050193517107639\\
101.01	0.00501935825237451\\
102.01	0.00501936491997196\\
103.01	0.00501937171595036\\
104.01	0.00501937864274822\\
105.01	0.00501938570285006\\
106.01	0.0050193928987864\\
107.01	0.00501940023313502\\
108.01	0.00501940770852174\\
109.01	0.0050194153276215\\
110.01	0.00501942309315881\\
111.01	0.00501943100790882\\
112.01	0.00501943907469866\\
113.01	0.00501944729640778\\
114.01	0.00501945567596943\\
115.01	0.00501946421637105\\
116.01	0.0050194729206558\\
117.01	0.00501948179192331\\
118.01	0.00501949083333078\\
119.01	0.0050195000480939\\
120.01	0.00501950943948809\\
121.01	0.00501951901084931\\
122.01	0.00501952876557563\\
123.01	0.00501953870712786\\
124.01	0.00501954883903089\\
125.01	0.00501955916487502\\
126.01	0.00501956968831682\\
127.01	0.00501958041308039\\
128.01	0.00501959134295896\\
129.01	0.00501960248181554\\
130.01	0.00501961383358478\\
131.01	0.0050196254022737\\
132.01	0.00501963719196331\\
133.01	0.00501964920680996\\
134.01	0.00501966145104675\\
135.01	0.00501967392898464\\
136.01	0.00501968664501412\\
137.01	0.00501969960360645\\
138.01	0.00501971280931506\\
139.01	0.00501972626677753\\
140.01	0.00501973998071634\\
141.01	0.00501975395594097\\
142.01	0.00501976819734937\\
143.01	0.00501978270992932\\
144.01	0.00501979749876019\\
145.01	0.00501981256901469\\
146.01	0.00501982792596037\\
147.01	0.00501984357496139\\
148.01	0.00501985952148028\\
149.01	0.00501987577107953\\
150.01	0.0050198923294236\\
151.01	0.0050199092022809\\
152.01	0.00501992639552516\\
153.01	0.00501994391513751\\
154.01	0.00501996176720869\\
155.01	0.00501997995794085\\
156.01	0.00501999849364934\\
157.01	0.00502001738076462\\
158.01	0.00502003662583522\\
159.01	0.0050200562355288\\
160.01	0.00502007621663457\\
161.01	0.00502009657606595\\
162.01	0.00502011732086234\\
163.01	0.00502013845819144\\
164.01	0.00502015999535146\\
165.01	0.00502018193977367\\
166.01	0.00502020429902479\\
167.01	0.00502022708080903\\
168.01	0.00502025029297101\\
169.01	0.00502027394349808\\
170.01	0.00502029804052302\\
171.01	0.00502032259232618\\
172.01	0.00502034760733859\\
173.01	0.00502037309414435\\
174.01	0.00502039906148381\\
175.01	0.00502042551825597\\
176.01	0.0050204524735213\\
177.01	0.00502047993650487\\
178.01	0.00502050791659914\\
179.01	0.00502053642336707\\
180.01	0.0050205654665449\\
181.01	0.00502059505604584\\
182.01	0.00502062520196246\\
183.01	0.00502065591457078\\
184.01	0.00502068720433252\\
185.01	0.00502071908189955\\
186.01	0.00502075155811632\\
187.01	0.0050207846440241\\
188.01	0.00502081835086399\\
189.01	0.00502085269008092\\
190.01	0.00502088767332684\\
191.01	0.00502092331246481\\
192.01	0.00502095961957291\\
193.01	0.00502099660694766\\
194.01	0.00502103428710834\\
195.01	0.00502107267280101\\
196.01	0.00502111177700215\\
197.01	0.00502115161292357\\
198.01	0.00502119219401589\\
199.01	0.0050212335339737\\
200.01	0.00502127564673925\\
201.01	0.00502131854650727\\
202.01	0.00502136224772966\\
203.01	0.00502140676511986\\
204.01	0.005021452113658\\
205.01	0.00502149830859555\\
206.01	0.00502154536546001\\
207.01	0.00502159330006058\\
208.01	0.00502164212849283\\
209.01	0.00502169186714428\\
210.01	0.0050217425326993\\
211.01	0.0050217941421451\\
212.01	0.00502184671277701\\
213.01	0.00502190026220412\\
214.01	0.00502195480835549\\
215.01	0.00502201036948524\\
216.01	0.0050220669641797\\
217.01	0.0050221246113627\\
218.01	0.00502218333030202\\
219.01	0.00502224314061629\\
220.01	0.00502230406228095\\
221.01	0.00502236611563531\\
222.01	0.00502242932138887\\
223.01	0.00502249370062913\\
224.01	0.00502255927482793\\
225.01	0.00502262606584909\\
226.01	0.00502269409595579\\
227.01	0.00502276338781804\\
228.01	0.00502283396452048\\
229.01	0.00502290584957027\\
230.01	0.00502297906690523\\
231.01	0.005023053640902\\
232.01	0.00502312959638448\\
233.01	0.00502320695863241\\
234.01	0.00502328575339044\\
235.01	0.00502336600687672\\
236.01	0.00502344774579229\\
237.01	0.00502353099733051\\
238.01	0.00502361578918678\\
239.01	0.00502370214956807\\
240.01	0.00502379010720296\\
241.01	0.00502387969135258\\
242.01	0.00502397093182022\\
243.01	0.00502406385896251\\
244.01	0.00502415850370061\\
245.01	0.00502425489753115\\
246.01	0.00502435307253787\\
247.01	0.00502445306140335\\
248.01	0.00502455489742119\\
249.01	0.00502465861450822\\
250.01	0.0050247642472171\\
251.01	0.00502487183074932\\
252.01	0.00502498140096876\\
253.01	0.00502509299441485\\
254.01	0.00502520664831621\\
255.01	0.00502532240060563\\
256.01	0.0050254402899341\\
257.01	0.00502556035568567\\
258.01	0.005025682637993\\
259.01	0.00502580717775299\\
260.01	0.00502593401664239\\
261.01	0.00502606319713503\\
262.01	0.00502619476251753\\
263.01	0.00502632875690788\\
264.01	0.00502646522527174\\
265.01	0.00502660421344144\\
266.01	0.00502674576813428\\
267.01	0.00502688993697173\\
268.01	0.00502703676849824\\
269.01	0.00502718631220181\\
270.01	0.00502733861853444\\
271.01	0.00502749373893292\\
272.01	0.00502765172584038\\
273.01	0.00502781263272795\\
274.01	0.0050279765141182\\
275.01	0.00502814342560737\\
276.01	0.00502831342388961\\
277.01	0.00502848656678119\\
278.01	0.00502866291324529\\
279.01	0.00502884252341747\\
280.01	0.00502902545863208\\
281.01	0.00502921178144883\\
282.01	0.00502940155568045\\
283.01	0.00502959484642063\\
284.01	0.00502979172007285\\
285.01	0.00502999224438008\\
286.01	0.00503019648845489\\
287.01	0.00503040452281049\\
288.01	0.00503061641939237\\
289.01	0.00503083225161066\\
290.01	0.00503105209437366\\
291.01	0.00503127602412171\\
292.01	0.00503150411886187\\
293.01	0.00503173645820396\\
294.01	0.00503197312339637\\
295.01	0.00503221419736403\\
296.01	0.00503245976474617\\
297.01	0.0050327099119353\\
298.01	0.0050329647271169\\
299.01	0.00503322430031067\\
300.01	0.00503348872341158\\
301.01	0.00503375809023245\\
302.01	0.00503403249654735\\
303.01	0.00503431204013617\\
304.01	0.00503459682082916\\
305.01	0.00503488694055343\\
306.01	0.00503518250337986\\
307.01	0.00503548361557084\\
308.01	0.00503579038562912\\
309.01	0.0050361029243475\\
310.01	0.00503642134485913\\
311.01	0.00503674576268956\\
312.01	0.00503707629580859\\
313.01	0.00503741306468343\\
314.01	0.00503775619233326\\
315.01	0.00503810580438374\\
316.01	0.00503846202912313\\
317.01	0.00503882499755876\\
318.01	0.00503919484347427\\
319.01	0.00503957170348862\\
320.01	0.00503995571711445\\
321.01	0.00504034702681803\\
322.01	0.00504074577808022\\
323.01	0.00504115211945731\\
324.01	0.0050415662026435\\
325.01	0.00504198818253367\\
326.01	0.00504241821728675\\
327.01	0.00504285646839001\\
328.01	0.00504330310072384\\
329.01	0.00504375828262761\\
330.01	0.00504422218596549\\
331.01	0.00504469498619389\\
332.01	0.00504517686242904\\
333.01	0.00504566799751503\\
334.01	0.00504616857809335\\
335.01	0.00504667879467241\\
336.01	0.0050471988416986\\
337.01	0.00504772891762756\\
338.01	0.00504826922499717\\
339.01	0.00504881997050075\\
340.01	0.0050493813650619\\
341.01	0.00504995362391013\\
342.01	0.00505053696665841\\
343.01	0.00505113161738205\\
344.01	0.00505173780469856\\
345.01	0.00505235576185045\\
346.01	0.00505298572678886\\
347.01	0.00505362794226062\\
348.01	0.00505428265589696\\
349.01	0.00505495012030505\\
350.01	0.00505563059316302\\
351.01	0.00505632433731748\\
352.01	0.00505703162088522\\
353.01	0.00505775271735856\\
354.01	0.00505848790571424\\
355.01	0.00505923747052754\\
356.01	0.00506000170209087\\
357.01	0.0050607808965366\\
358.01	0.00506157535596578\\
359.01	0.005062385388582\\
360.01	0.00506321130883047\\
361.01	0.00506405343754281\\
362.01	0.0050649121020872\\
363.01	0.00506578763652464\\
364.01	0.00506668038176997\\
365.01	0.00506759068575829\\
366.01	0.00506851890361704\\
367.01	0.00506946539784152\\
368.01	0.00507043053847556\\
369.01	0.00507141470329618\\
370.01	0.00507241827800047\\
371.01	0.00507344165639692\\
372.01	0.00507448524059727\\
373.01	0.00507554944121311\\
374.01	0.00507663467755196\\
375.01	0.00507774137781693\\
376.01	0.00507886997930695\\
377.01	0.0050800209286208\\
378.01	0.00508119468186213\\
379.01	0.00508239170484881\\
380.01	0.00508361247332564\\
381.01	0.00508485747318098\\
382.01	0.00508612720066827\\
383.01	0.00508742216263094\\
384.01	0.00508874287673265\\
385.01	0.00509008987169214\\
386.01	0.00509146368752236\\
387.01	0.00509286487577354\\
388.01	0.00509429399978219\\
389.01	0.00509575163492443\\
390.01	0.00509723836887327\\
391.01	0.00509875480186192\\
392.01	0.00510030154695075\\
393.01	0.00510187923030017\\
394.01	0.00510348849144823\\
395.01	0.00510512998359208\\
396.01	0.00510680437387648\\
397.01	0.00510851234368537\\
398.01	0.00511025458893963\\
399.01	0.00511203182040038\\
400.01	0.00511384476397572\\
401.01	0.00511569416103477\\
402.01	0.00511758076872551\\
403.01	0.00511950536029861\\
404.01	0.00512146872543671\\
405.01	0.00512347167058867\\
406.01	0.00512551501930966\\
407.01	0.00512759961260707\\
408.01	0.0051297263092914\\
409.01	0.00513189598633347\\
410.01	0.00513410953922723\\
411.01	0.00513636788235847\\
412.01	0.00513867194937997\\
413.01	0.00514102269359204\\
414.01	0.00514342108833014\\
415.01	0.005145868127359\\
416.01	0.00514836482527282\\
417.01	0.00515091221790266\\
418.01	0.00515351136273091\\
419.01	0.00515616333931325\\
420.01	0.0051588692497074\\
421.01	0.00516163021891089\\
422.01	0.00516444739530594\\
423.01	0.00516732195111302\\
424.01	0.00517025508285339\\
425.01	0.00517324801182083\\
426.01	0.00517630198456262\\
427.01	0.00517941827337016\\
428.01	0.00518259817678116\\
429.01	0.00518584302009149\\
430.01	0.00518915415587828\\
431.01	0.00519253296453566\\
432.01	0.00519598085482181\\
433.01	0.00519949926441959\\
434.01	0.00520308966050941\\
435.01	0.0052067535403565\\
436.01	0.00521049243191227\\
437.01	0.0052143078944302\\
438.01	0.00521820151909683\\
439.01	0.00522217492967812\\
440.01	0.00522622978318196\\
441.01	0.00523036777053668\\
442.01	0.00523459061728625\\
443.01	0.00523890008430183\\
444.01	0.00524329796851115\\
445.01	0.00524778610364351\\
446.01	0.00525236636099316\\
447.01	0.00525704065019965\\
448.01	0.00526181092004457\\
449.01	0.00526667915926614\\
450.01	0.00527164739739069\\
451.01	0.00527671770558057\\
452.01	0.00528189219749942\\
453.01	0.00528717303019413\\
454.01	0.00529256240499335\\
455.01	0.00529806256842333\\
456.01	0.00530367581314065\\
457.01	0.00530940447888219\\
458.01	0.00531525095343356\\
459.01	0.0053212176736156\\
460.01	0.00532730712629066\\
461.01	0.00533352184938808\\
462.01	0.00533986443295212\\
463.01	0.00534633752021037\\
464.01	0.00535294380866658\\
465.01	0.00535968605121613\\
466.01	0.00536656705728716\\
467.01	0.00537358969400614\\
468.01	0.00538075688738979\\
469.01	0.00538807162356364\\
470.01	0.00539553695000666\\
471.01	0.00540315597682403\\
472.01	0.00541093187804739\\
473.01	0.00541886789296336\\
474.01	0.00542696732747124\\
475.01	0.00543523355546909\\
476.01	0.00544367002027044\\
477.01	0.00545228023605022\\
478.01	0.00546106778932162\\
479.01	0.00547003634044396\\
480.01	0.00547918962516099\\
481.01	0.00548853145617276\\
482.01	0.00549806572473786\\
483.01	0.00550779640231014\\
484.01	0.00551772754220767\\
485.01	0.00552786328131574\\
486.01	0.00553820784182571\\
487.01	0.00554876553300686\\
488.01	0.00555954075301628\\
489.01	0.00557053799074402\\
490.01	0.00558176182769582\\
491.01	0.00559321693991399\\
492.01	0.00560490809993744\\
493.01	0.00561684017880044\\
494.01	0.0056290181480723\\
495.01	0.0056414470819366\\
496.01	0.00565413215931353\\
497.01	0.00566707866602291\\
498.01	0.00568029199698998\\
499.01	0.0056937776584946\\
500.01	0.00570754127046308\\
501.01	0.00572158856880466\\
502.01	0.00573592540779119\\
503.01	0.00575055776248193\\
504.01	0.00576549173119376\\
505.01	0.00578073353801487\\
506.01	0.00579628953536469\\
507.01	0.00581216620659866\\
508.01	0.0058283701686583\\
509.01	0.00584490817476563\\
510.01	0.00586178711716329\\
511.01	0.00587901402989748\\
512.01	0.00589659609164545\\
513.01	0.00591454062858528\\
514.01	0.0059328551173065\\
515.01	0.00595154718776236\\
516.01	0.00597062462625876\\
517.01	0.00599009537848122\\
518.01	0.00600996755255487\\
519.01	0.00603024942213667\\
520.01	0.00605094942953484\\
521.01	0.00607207618885311\\
522.01	0.00609363848915531\\
523.01	0.00611564529764388\\
524.01	0.00613810576284839\\
525.01	0.00616102921781665\\
526.01	0.00618442518330081\\
527.01	0.00620830337092997\\
528.01	0.00623267368636038\\
529.01	0.00625754623239091\\
530.01	0.00628293131203315\\
531.01	0.00630883943152065\\
532.01	0.00633528130324207\\
533.01	0.00636226784858186\\
534.01	0.00638981020064584\\
535.01	0.00641791970685204\\
536.01	0.00644660793135991\\
537.01	0.00647588665731036\\
538.01	0.00650576788884577\\
539.01	0.0065362638528734\\
540.01	0.00656738700053343\\
541.01	0.00659915000832749\\
542.01	0.00663156577885837\\
543.01	0.00666464744112576\\
544.01	0.00669840835031567\\
545.01	0.00673286208701748\\
546.01	0.00676802245578906\\
547.01	0.00680390348298683\\
548.01	0.00684051941376644\\
549.01	0.00687788470814671\\
550.01	0.0069160140360215\\
551.01	0.0069549222709883\\
552.01	0.00699462448284956\\
553.01	0.00703513592862633\\
554.01	0.00707647204190588\\
555.01	0.00711864842032717\\
556.01	0.00716168081098622\\
557.01	0.00720558509352023\\
558.01	0.00725037726060528\\
559.01	0.00729607339557415\\
560.01	0.0073426896468327\\
561.01	0.00739024219871955\\
562.01	0.00743874723842159\\
563.01	0.00748822091851965\\
564.01	0.00753867931470128\\
565.01	0.00759013837813551\\
566.01	0.00764261388196354\\
567.01	0.0076961213613144\\
568.01	0.00775067604621158\\
569.01	0.00780629278669433\\
570.01	0.00786298596943516\\
571.01	0.00792076942509971\\
572.01	0.0079796563256667\\
573.01	0.00803965907090454\\
574.01	0.00810078916320144\\
575.01	0.00816305706995899\\
576.01	0.00822647207281021\\
577.01	0.00829104210300508\\
578.01	0.00835677356244437\\
579.01	0.00842367113004633\\
580.01	0.00849173755342209\\
581.01	0.00856097342624057\\
582.01	0.00863137695221383\\
583.01	0.00870294369737224\\
584.01	0.00877566633327528\\
585.01	0.00884953437508313\\
586.01	0.00892453392008026\\
587.01	0.00900064739439612\\
588.01	0.00907785331843724\\
589.01	0.00915612610509321\\
590.01	0.00923543590930589\\
591.01	0.00931574855334877\\
592.01	0.00939702555946881\\
593.01	0.00947922433078829\\
594.01	0.00956229853304221\\
595.01	0.00964619874445287\\
596.01	0.00973087345958245\\
597.01	0.0098162705563152\\
598.01	0.00990228776786639\\
599.01	0.00996919203046377\\
599.02	0.00996973314335038\\
599.03	0.0099702709571694\\
599.04	0.00997080543920336\\
599.05	0.0099713365564138\\
599.06	0.00997186427543808\\
599.07	0.0099723885625862\\
599.08	0.00997290938383756\\
599.09	0.00997342670483771\\
599.1	0.00997394049089505\\
599.11	0.00997445070697751\\
599.12	0.00997495731770918\\
599.13	0.00997546028736696\\
599.14	0.00997595957987707\\
599.15	0.00997645515881166\\
599.16	0.00997694698738526\\
599.17	0.00997743502845132\\
599.18	0.00997791924449856\\
599.19	0.00997839959764748\\
599.2	0.00997887604964662\\
599.21	0.00997934856186899\\
599.22	0.00997981709530829\\
599.23	0.00998028161057521\\
599.24	0.00998074206789365\\
599.25	0.0099811984270969\\
599.26	0.00998165064762378\\
599.27	0.00998209868851477\\
599.28	0.00998254250840806\\
599.29	0.00998298206553559\\
599.3	0.00998341731771905\\
599.31	0.00998384822236584\\
599.32	0.00998427473646497\\
599.33	0.00998469681658292\\
599.34	0.00998511441885952\\
599.35	0.0099855274990037\\
599.36	0.00998593601228926\\
599.37	0.00998633991355056\\
599.38	0.00998673915717821\\
599.39	0.00998713369711469\\
599.4	0.00998752348684992\\
599.41	0.00998790847811157\\
599.42	0.00998828861974491\\
599.43	0.00998866386008804\\
599.44	0.00998903414696681\\
599.45	0.00998939942768985\\
599.46	0.00998975964904344\\
599.47	0.00999011475728636\\
599.48	0.00999046469814472\\
599.49	0.00999080941680669\\
599.5	0.00999114885791725\\
599.51	0.00999148296557278\\
599.52	0.0099918116833157\\
599.53	0.00999213495412901\\
599.54	0.00999245272043077\\
599.55	0.00999276492406855\\
599.56	0.00999307150631381\\
599.57	0.00999337240785624\\
599.58	0.00999366756879799\\
599.59	0.00999395692864795\\
599.6	0.00999424042631586\\
599.61	0.00999451800010642\\
599.62	0.00999478958771336\\
599.63	0.0099950551262134\\
599.64	0.00999531455206019\\
599.65	0.00999556780107816\\
599.66	0.00999581480845634\\
599.67	0.00999605550874211\\
599.68	0.00999628983583487\\
599.69	0.00999651772297967\\
599.7	0.00999673910276075\\
599.71	0.00999695390709509\\
599.72	0.00999716206722577\\
599.73	0.00999736351371538\\
599.74	0.00999755817643933\\
599.75	0.00999774598457904\\
599.76	0.00999792686661517\\
599.77	0.00999810075032068\\
599.78	0.00999826756275386\\
599.79	0.00999842723025133\\
599.8	0.00999857967842092\\
599.81	0.0099987248321345\\
599.82	0.00999886261552071\\
599.83	0.00999899295195769\\
599.84	0.00999911576406567\\
599.85	0.00999923097369951\\
599.86	0.00999933850194117\\
599.87	0.00999943826909211\\
599.88	0.00999953019466558\\
599.89	0.00999961419737891\\
599.9	0.00999969019514566\\
599.91	0.00999975810506767\\
599.92	0.00999981784342713\\
599.93	0.00999986932567848\\
599.94	0.00999991246644028\\
599.95	0.00999994717948697\\
599.96	0.00999997337774056\\
599.97	0.00999999097326228\\
599.98	0.00999999987724406\\
599.99	0.01\\
600	0.01\\
};
\addplot [color=mycolor5,solid,forget plot]
  table[row sep=crcr]{%
0.01	0.00501105323809203\\
1.01	0.00501105434915347\\
2.01	0.00501105548219559\\
3.01	0.00501105663765027\\
4.01	0.00501105781595748\\
5.01	0.00501105901756574\\
6.01	0.00501106024293257\\
7.01	0.00501106149252414\\
8.01	0.00501106276681591\\
9.01	0.00501106406629223\\
10.01	0.00501106539144701\\
11.01	0.00501106674278391\\
12.01	0.00501106812081607\\
13.01	0.00501106952606685\\
14.01	0.00501107095906948\\
15.01	0.0050110724203678\\
16.01	0.00501107391051601\\
17.01	0.00501107543007902\\
18.01	0.00501107697963291\\
19.01	0.0050110785597646\\
20.01	0.00501108017107262\\
21.01	0.00501108181416682\\
22.01	0.00501108348966919\\
23.01	0.00501108519821353\\
24.01	0.00501108694044591\\
25.01	0.00501108871702494\\
26.01	0.00501109052862167\\
27.01	0.00501109237592055\\
28.01	0.00501109425961895\\
29.01	0.00501109618042773\\
30.01	0.00501109813907175\\
31.01	0.00501110013628946\\
32.01	0.00501110217283358\\
33.01	0.00501110424947158\\
34.01	0.00501110636698546\\
35.01	0.00501110852617254\\
36.01	0.00501111072784531\\
37.01	0.00501111297283197\\
38.01	0.00501111526197658\\
39.01	0.00501111759613956\\
40.01	0.00501111997619762\\
41.01	0.00501112240304449\\
42.01	0.00501112487759112\\
43.01	0.00501112740076579\\
44.01	0.00501112997351484\\
45.01	0.00501113259680265\\
46.01	0.00501113527161187\\
47.01	0.00501113799894434\\
48.01	0.00501114077982097\\
49.01	0.00501114361528207\\
50.01	0.00501114650638813\\
51.01	0.00501114945421972\\
52.01	0.00501115245987828\\
53.01	0.00501115552448617\\
54.01	0.00501115864918734\\
55.01	0.00501116183514756\\
56.01	0.00501116508355486\\
57.01	0.0050111683956198\\
58.01	0.00501117177257621\\
59.01	0.00501117521568152\\
60.01	0.00501117872621715\\
61.01	0.00501118230548872\\
62.01	0.00501118595482709\\
63.01	0.00501118967558804\\
64.01	0.00501119346915346\\
65.01	0.00501119733693153\\
66.01	0.00501120128035717\\
67.01	0.00501120530089255\\
68.01	0.00501120940002762\\
69.01	0.00501121357928062\\
70.01	0.00501121784019883\\
71.01	0.00501122218435855\\
72.01	0.00501122661336638\\
73.01	0.00501123112885897\\
74.01	0.0050112357325043\\
75.01	0.00501124042600191\\
76.01	0.00501124521108336\\
77.01	0.00501125008951314\\
78.01	0.00501125506308905\\
79.01	0.005011260133643\\
80.01	0.0050112653030413\\
81.01	0.00501127057318572\\
82.01	0.00501127594601383\\
83.01	0.00501128142349977\\
84.01	0.0050112870076547\\
85.01	0.00501129270052825\\
86.01	0.0050112985042082\\
87.01	0.00501130442082177\\
88.01	0.00501131045253633\\
89.01	0.00501131660155973\\
90.01	0.0050113228701416\\
91.01	0.00501132926057374\\
92.01	0.00501133577519107\\
93.01	0.00501134241637217\\
94.01	0.00501134918654049\\
95.01	0.00501135608816462\\
96.01	0.00501136312375958\\
97.01	0.00501137029588755\\
98.01	0.00501137760715856\\
99.01	0.00501138506023138\\
100.01	0.00501139265781471\\
101.01	0.00501140040266766\\
102.01	0.00501140829760099\\
103.01	0.00501141634547772\\
104.01	0.00501142454921466\\
105.01	0.00501143291178256\\
106.01	0.0050114414362077\\
107.01	0.00501145012557279\\
108.01	0.00501145898301772\\
109.01	0.00501146801174081\\
110.01	0.00501147721499982\\
111.01	0.00501148659611307\\
112.01	0.00501149615846029\\
113.01	0.00501150590548408\\
114.01	0.00501151584069057\\
115.01	0.00501152596765125\\
116.01	0.00501153629000348\\
117.01	0.00501154681145168\\
118.01	0.00501155753576905\\
119.01	0.00501156846679843\\
120.01	0.0050115796084534\\
121.01	0.00501159096471993\\
122.01	0.00501160253965742\\
123.01	0.00501161433740002\\
124.01	0.00501162636215804\\
125.01	0.00501163861821899\\
126.01	0.00501165110994933\\
127.01	0.00501166384179594\\
128.01	0.0050116768182871\\
129.01	0.00501169004403393\\
130.01	0.00501170352373236\\
131.01	0.0050117172621643\\
132.01	0.0050117312641991\\
133.01	0.00501174553479515\\
134.01	0.00501176007900125\\
135.01	0.00501177490195867\\
136.01	0.00501179000890223\\
137.01	0.00501180540516227\\
138.01	0.0050118210961662\\
139.01	0.00501183708743999\\
140.01	0.00501185338461066\\
141.01	0.00501186999340709\\
142.01	0.00501188691966192\\
143.01	0.00501190416931392\\
144.01	0.00501192174840956\\
145.01	0.00501193966310458\\
146.01	0.00501195791966616\\
147.01	0.00501197652447485\\
148.01	0.00501199548402642\\
149.01	0.00501201480493388\\
150.01	0.00501203449392944\\
151.01	0.00501205455786659\\
152.01	0.00501207500372232\\
153.01	0.00501209583859905\\
154.01	0.00501211706972662\\
155.01	0.00501213870446485\\
156.01	0.00501216075030556\\
157.01	0.00501218321487495\\
158.01	0.00501220610593539\\
159.01	0.0050122294313884\\
160.01	0.00501225319927683\\
161.01	0.00501227741778669\\
162.01	0.00501230209525023\\
163.01	0.00501232724014815\\
164.01	0.00501235286111248\\
165.01	0.00501237896692812\\
166.01	0.00501240556653661\\
167.01	0.00501243266903807\\
168.01	0.00501246028369394\\
169.01	0.00501248841992995\\
170.01	0.00501251708733844\\
171.01	0.00501254629568146\\
172.01	0.00501257605489341\\
173.01	0.00501260637508434\\
174.01	0.00501263726654223\\
175.01	0.00501266873973628\\
176.01	0.00501270080532015\\
177.01	0.00501273347413449\\
178.01	0.00501276675721027\\
179.01	0.00501280066577221\\
180.01	0.00501283521124173\\
181.01	0.00501287040523979\\
182.01	0.00501290625959114\\
183.01	0.00501294278632643\\
184.01	0.00501297999768707\\
185.01	0.00501301790612706\\
186.01	0.00501305652431797\\
187.01	0.00501309586515127\\
188.01	0.00501313594174276\\
189.01	0.00501317676743585\\
190.01	0.00501321835580532\\
191.01	0.00501326072066127\\
192.01	0.00501330387605255\\
193.01	0.00501334783627098\\
194.01	0.00501339261585526\\
195.01	0.00501343822959488\\
196.01	0.00501348469253411\\
197.01	0.00501353201997619\\
198.01	0.00501358022748774\\
199.01	0.00501362933090251\\
200.01	0.00501367934632607\\
201.01	0.00501373029014007\\
202.01	0.00501378217900687\\
203.01	0.00501383502987365\\
204.01	0.00501388885997742\\
205.01	0.00501394368684922\\
206.01	0.0050139995283193\\
207.01	0.00501405640252163\\
208.01	0.00501411432789881\\
209.01	0.00501417332320697\\
210.01	0.00501423340752125\\
211.01	0.00501429460023999\\
212.01	0.00501435692109077\\
213.01	0.00501442039013514\\
214.01	0.00501448502777423\\
215.01	0.00501455085475419\\
216.01	0.00501461789217139\\
217.01	0.00501468616147842\\
218.01	0.00501475568448942\\
219.01	0.00501482648338628\\
220.01	0.00501489858072407\\
221.01	0.00501497199943747\\
222.01	0.00501504676284646\\
223.01	0.00501512289466254\\
224.01	0.00501520041899522\\
225.01	0.00501527936035853\\
226.01	0.00501535974367673\\
227.01	0.00501544159429182\\
228.01	0.00501552493796999\\
229.01	0.00501560980090822\\
230.01	0.0050156962097412\\
231.01	0.00501578419154891\\
232.01	0.00501587377386335\\
233.01	0.00501596498467613\\
234.01	0.00501605785244565\\
235.01	0.00501615240610501\\
236.01	0.00501624867506979\\
237.01	0.00501634668924585\\
238.01	0.00501644647903703\\
239.01	0.00501654807535393\\
240.01	0.00501665150962216\\
241.01	0.00501675681379021\\
242.01	0.00501686402033923\\
243.01	0.00501697316229131\\
244.01	0.0050170842732184\\
245.01	0.00501719738725173\\
246.01	0.00501731253909156\\
247.01	0.0050174297640166\\
248.01	0.00501754909789333\\
249.01	0.00501767057718719\\
250.01	0.00501779423897193\\
251.01	0.00501792012094072\\
252.01	0.00501804826141644\\
253.01	0.00501817869936318\\
254.01	0.00501831147439737\\
255.01	0.00501844662679896\\
256.01	0.00501858419752384\\
257.01	0.00501872422821566\\
258.01	0.00501886676121815\\
259.01	0.00501901183958807\\
260.01	0.0050191595071083\\
261.01	0.00501930980830093\\
262.01	0.00501946278844137\\
263.01	0.00501961849357192\\
264.01	0.00501977697051723\\
265.01	0.00501993826689813\\
266.01	0.00502010243114755\\
267.01	0.00502026951252586\\
268.01	0.0050204395611374\\
269.01	0.0050206126279468\\
270.01	0.00502078876479636\\
271.01	0.00502096802442337\\
272.01	0.0050211504604781\\
273.01	0.00502133612754316\\
274.01	0.00502152508115168\\
275.01	0.0050217173778079\\
276.01	0.00502191307500735\\
277.01	0.00502211223125749\\
278.01	0.00502231490609989\\
279.01	0.00502252116013228\\
280.01	0.0050227310550317\\
281.01	0.00502294465357837\\
282.01	0.00502316201967989\\
283.01	0.00502338321839679\\
284.01	0.00502360831596875\\
285.01	0.0050238373798413\\
286.01	0.00502407047869415\\
287.01	0.00502430768246917\\
288.01	0.00502454906240107\\
289.01	0.0050247946910473\\
290.01	0.00502504464232019\\
291.01	0.00502529899151921\\
292.01	0.00502555781536547\\
293.01	0.00502582119203565\\
294.01	0.00502608920119921\\
295.01	0.00502636192405458\\
296.01	0.00502663944336823\\
297.01	0.00502692184351397\\
298.01	0.00502720921051425\\
299.01	0.00502750163208222\\
300.01	0.00502779919766547\\
301.01	0.00502810199849096\\
302.01	0.00502841012761156\\
303.01	0.00502872367995369\\
304.01	0.00502904275236694\\
305.01	0.00502936744367477\\
306.01	0.00502969785472668\\
307.01	0.00503003408845215\\
308.01	0.00503037624991633\\
309.01	0.00503072444637663\\
310.01	0.00503107878734196\\
311.01	0.00503143938463199\\
312.01	0.00503180635243938\\
313.01	0.00503217980739329\\
314.01	0.00503255986862387\\
315.01	0.00503294665782869\\
316.01	0.00503334029934097\\
317.01	0.00503374092019886\\
318.01	0.00503414865021638\\
319.01	0.0050345636220554\\
320.01	0.0050349859712993\\
321.01	0.0050354158365284\\
322.01	0.00503585335939509\\
323.01	0.00503629868470189\\
324.01	0.00503675196047861\\
325.01	0.00503721333806179\\
326.01	0.00503768297217426\\
327.01	0.00503816102100541\\
328.01	0.00503864764629162\\
329.01	0.00503914301339753\\
330.01	0.00503964729139655\\
331.01	0.00504016065315194\\
332.01	0.00504068327539701\\
333.01	0.00504121533881544\\
334.01	0.00504175702811992\\
335.01	0.00504230853213031\\
336.01	0.00504287004385055\\
337.01	0.00504344176054386\\
338.01	0.00504402388380591\\
339.01	0.00504461661963666\\
340.01	0.00504522017850959\\
341.01	0.00504583477543853\\
342.01	0.00504646063004185\\
343.01	0.00504709796660381\\
344.01	0.00504774701413349\\
345.01	0.00504840800641998\\
346.01	0.00504908118208582\\
347.01	0.00504976678463623\\
348.01	0.00505046506250637\\
349.01	0.0050511762691063\\
350.01	0.0050519006628631\\
351.01	0.00505263850726183\\
352.01	0.00505339007088509\\
353.01	0.00505415562745272\\
354.01	0.00505493545586137\\
355.01	0.00505572984022609\\
356.01	0.00505653906992413\\
357.01	0.0050573634396435\\
358.01	0.00505820324943602\\
359.01	0.00505905880477907\\
360.01	0.00505993041664424\\
361.01	0.00506081840157829\\
362.01	0.00506172308179569\\
363.01	0.00506264478528499\\
364.01	0.00506358384593185\\
365.01	0.00506454060365845\\
366.01	0.00506551540458144\\
367.01	0.00506650860118923\\
368.01	0.00506752055253759\\
369.01	0.00506855162446482\\
370.01	0.00506960218982375\\
371.01	0.0050706726287294\\
372.01	0.00507176332881949\\
373.01	0.00507287468552395\\
374.01	0.00507400710233857\\
375.01	0.00507516099110038\\
376.01	0.00507633677225852\\
377.01	0.00507753487513728\\
378.01	0.00507875573819004\\
379.01	0.0050799998092432\\
380.01	0.00508126754573331\\
381.01	0.00508255941494219\\
382.01	0.00508387589423588\\
383.01	0.00508521747130959\\
384.01	0.00508658464443858\\
385.01	0.00508797792273444\\
386.01	0.0050893978264083\\
387.01	0.00509084488704023\\
388.01	0.00509231964785468\\
389.01	0.00509382266400238\\
390.01	0.00509535450284851\\
391.01	0.00509691574426789\\
392.01	0.00509850698094573\\
393.01	0.00510012881868571\\
394.01	0.00510178187672403\\
395.01	0.0051034667880506\\
396.01	0.00510518419973573\\
397.01	0.00510693477326443\\
398.01	0.00510871918487622\\
399.01	0.00511053812591178\\
400.01	0.00511239230316565\\
401.01	0.00511428243924573\\
402.01	0.00511620927293786\\
403.01	0.0051181735595774\\
404.01	0.00512017607142572\\
405.01	0.00512221759805365\\
406.01	0.0051242989467291\\
407.01	0.00512642094281104\\
408.01	0.00512858443014799\\
409.01	0.00513079027148199\\
410.01	0.00513303934885734\\
411.01	0.00513533256403319\\
412.01	0.00513767083890113\\
413.01	0.00514005511590683\\
414.01	0.00514248635847511\\
415.01	0.00514496555143869\\
416.01	0.00514749370147027\\
417.01	0.00515007183751852\\
418.01	0.00515270101124604\\
419.01	0.00515538229747044\\
420.01	0.00515811679460831\\
421.01	0.00516090562512184\\
422.01	0.00516374993596685\\
423.01	0.00516665089904494\\
424.01	0.0051696097116572\\
425.01	0.00517262759696037\\
426.01	0.00517570580442675\\
427.01	0.00517884561030678\\
428.01	0.0051820483180947\\
429.01	0.00518531525899951\\
430.01	0.00518864779242004\\
431.01	0.00519204730642486\\
432.01	0.0051955152182399\\
433.01	0.00519905297474199\\
434.01	0.00520266205296244\\
435.01	0.00520634396059898\\
436.01	0.00521010023654003\\
437.01	0.005213932451401\\
438.01	0.00521784220807515\\
439.01	0.00522183114230099\\
440.01	0.00522590092324672\\
441.01	0.00523005325411513\\
442.01	0.00523428987276905\\
443.01	0.00523861255238114\\
444.01	0.00524302310210695\\
445.01	0.00524752336778546\\
446.01	0.00525211523266651\\
447.01	0.00525680061816582\\
448.01	0.00526158148465021\\
449.01	0.00526645983225105\\
450.01	0.00527143770170575\\
451.01	0.00527651717522777\\
452.01	0.00528170037740337\\
453.01	0.00528698947611138\\
454.01	0.00529238668346678\\
455.01	0.00529789425678259\\
456.01	0.00530351449954898\\
457.01	0.00530924976242536\\
458.01	0.00531510244424384\\
459.01	0.00532107499302052\\
460.01	0.00532716990697359\\
461.01	0.00533338973554799\\
462.01	0.00533973708044658\\
463.01	0.00534621459667049\\
464.01	0.00535282499357081\\
465.01	0.00535957103591755\\
466.01	0.00536645554498819\\
467.01	0.00537348139968213\\
468.01	0.00538065153766244\\
469.01	0.00538796895652568\\
470.01	0.0053954367150031\\
471.01	0.00540305793419209\\
472.01	0.00541083579882057\\
473.01	0.00541877355854389\\
474.01	0.00542687452927569\\
475.01	0.00543514209455356\\
476.01	0.0054435797069395\\
477.01	0.0054521908894558\\
478.01	0.00546097923705703\\
479.01	0.0054699484181375\\
480.01	0.0054791021760752\\
481.01	0.00548844433081055\\
482.01	0.00549797878046185\\
483.01	0.00550770950297491\\
484.01	0.00551764055780818\\
485.01	0.00552777608765277\\
486.01	0.00553812032018593\\
487.01	0.00554867756985955\\
488.01	0.00555945223972305\\
489.01	0.00557044882328042\\
490.01	0.00558167190638299\\
491.01	0.00559312616915737\\
492.01	0.00560481638797075\\
493.01	0.00561674743743441\\
494.01	0.00562892429244649\\
495.01	0.00564135203027566\\
496.01	0.00565403583268666\\
497.01	0.00566698098810914\\
498.01	0.00568019289385005\\
499.01	0.00569367705835026\\
500.01	0.00570743910348621\\
501.01	0.00572148476691542\\
502.01	0.00573581990446716\\
503.01	0.00575045049257814\\
504.01	0.005765382630772\\
505.01	0.0057806225441852\\
506.01	0.00579617658613588\\
507.01	0.0058120512407385\\
508.01	0.00582825312556255\\
509.01	0.00584478899433546\\
510.01	0.0058616657396896\\
511.01	0.00587889039595234\\
512.01	0.00589647014198016\\
513.01	0.00591441230403314\\
514.01	0.00593272435869262\\
515.01	0.00595141393581679\\
516.01	0.0059704888215366\\
517.01	0.00598995696128612\\
518.01	0.00600982646286827\\
519.01	0.00603010559955078\\
520.01	0.00605080281319069\\
521.01	0.00607192671738277\\
522.01	0.00609348610062707\\
523.01	0.00611548992951208\\
524.01	0.00613794735190604\\
525.01	0.00616086770015153\\
526.01	0.0061842604942537\\
527.01	0.00620813544505668\\
528.01	0.00623250245739489\\
529.01	0.0062573716332129\\
530.01	0.00628275327463575\\
531.01	0.00630865788698202\\
532.01	0.00633509618169938\\
533.01	0.00636207907920586\\
534.01	0.00638961771161965\\
535.01	0.00641772342535149\\
536.01	0.00644640778353617\\
537.01	0.00647568256827581\\
538.01	0.00650555978266212\\
539.01	0.00653605165254259\\
540.01	0.00656717062799204\\
541.01	0.00659892938444444\\
542.01	0.00663134082343626\\
543.01	0.00666441807290583\\
544.01	0.00669817448698827\\
545.01	0.00673262364523612\\
546.01	0.00676777935119077\\
547.01	0.00680365563021905\\
548.01	0.00684026672651947\\
549.01	0.00687762709919433\\
550.01	0.00691575141726969\\
551.01	0.00695465455353335\\
552.01	0.00699435157704655\\
553.01	0.00703485774416845\\
554.01	0.00707618848791713\\
555.01	0.00711835940546861\\
556.01	0.00716138624357784\\
557.01	0.00720528488168006\\
558.01	0.00725007131240749\\
559.01	0.00729576161922826\\
560.01	0.00734237195088536\\
561.01	0.00738991849228227\\
562.01	0.00743841743142593\\
563.01	0.00748788492200297\\
564.01	0.00753833704112559\\
565.01	0.00758978974174271\\
566.01	0.0076422587991696\\
567.01	0.0076957597511456\\
568.01	0.00775030783078647\\
569.01	0.00780591789175387\\
570.01	0.007862604324924\\
571.01	0.00792038096580129\\
572.01	0.00797926099189406\\
573.01	0.00803925680924829\\
574.01	0.00810037992733435\\
575.01	0.00816264082149692\\
576.01	0.00822604878222494\\
577.01	0.00829061175058319\\
578.01	0.00835633613928231\\
579.01	0.0084232266390671\\
580.01	0.00849128601039259\\
581.01	0.00856051486076035\\
582.01	0.00863091140863674\\
583.01	0.00870247123560845\\
584.01	0.0087751870294043\\
585.01	0.00884904832168768\\
586.01	0.00892404122618365\\
587.01	0.00900014818485135\\
588.01	0.00907734773257349\\
589.01	0.00915561429437061\\
590.01	0.00923491803366146\\
591.01	0.00931522477583024\\
592.01	0.00939649603864537\\
593.01	0.00947868921029156\\
594.01	0.00956175792742154\\
595.01	0.00964565272031552\\
596.01	0.00973032201072283\\
597.01	0.00981571357120332\\
598.01	0.00990176516613374\\
599.01	0.00996919203046377\\
599.02	0.00996973314335038\\
599.03	0.0099702709571694\\
599.04	0.00997080543920336\\
599.05	0.0099713365564138\\
599.06	0.00997186427543808\\
599.07	0.0099723885625862\\
599.08	0.00997290938383756\\
599.09	0.00997342670483771\\
599.1	0.00997394049089505\\
599.11	0.00997445070697751\\
599.12	0.00997495731770918\\
599.13	0.00997546028736696\\
599.14	0.00997595957987707\\
599.15	0.00997645515881165\\
599.16	0.00997694698738526\\
599.17	0.00997743502845131\\
599.18	0.00997791924449856\\
599.19	0.00997839959764748\\
599.2	0.00997887604964662\\
599.21	0.00997934856186899\\
599.22	0.00997981709530829\\
599.23	0.00998028161057521\\
599.24	0.00998074206789365\\
599.25	0.0099811984270969\\
599.26	0.00998165064762378\\
599.27	0.00998209868851477\\
599.28	0.00998254250840806\\
599.29	0.00998298206553559\\
599.3	0.00998341731771905\\
599.31	0.00998384822236584\\
599.32	0.00998427473646497\\
599.33	0.00998469681658292\\
599.34	0.00998511441885952\\
599.35	0.0099855274990037\\
599.36	0.00998593601228926\\
599.37	0.00998633991355056\\
599.38	0.00998673915717821\\
599.39	0.00998713369711469\\
599.4	0.00998752348684992\\
599.41	0.00998790847811157\\
599.42	0.00998828861974491\\
599.43	0.00998866386008803\\
599.44	0.00998903414696681\\
599.45	0.00998939942768986\\
599.46	0.00998975964904344\\
599.47	0.00999011475728636\\
599.48	0.00999046469814472\\
599.49	0.00999080941680669\\
599.5	0.00999114885791725\\
599.51	0.00999148296557278\\
599.52	0.0099918116833157\\
599.53	0.00999213495412901\\
599.54	0.00999245272043077\\
599.55	0.00999276492406855\\
599.56	0.00999307150631381\\
599.57	0.00999337240785624\\
599.58	0.00999366756879799\\
599.59	0.00999395692864795\\
599.6	0.00999424042631585\\
599.61	0.00999451800010642\\
599.62	0.00999478958771336\\
599.63	0.0099950551262134\\
599.64	0.00999531455206019\\
599.65	0.00999556780107816\\
599.66	0.00999581480845634\\
599.67	0.00999605550874211\\
599.68	0.00999628983583487\\
599.69	0.00999651772297967\\
599.7	0.00999673910276076\\
599.71	0.00999695390709509\\
599.72	0.00999716206722577\\
599.73	0.00999736351371538\\
599.74	0.00999755817643933\\
599.75	0.00999774598457904\\
599.76	0.00999792686661517\\
599.77	0.00999810075032068\\
599.78	0.00999826756275386\\
599.79	0.00999842723025133\\
599.8	0.00999857967842092\\
599.81	0.0099987248321345\\
599.82	0.00999886261552071\\
599.83	0.00999899295195769\\
599.84	0.00999911576406567\\
599.85	0.00999923097369951\\
599.86	0.00999933850194117\\
599.87	0.00999943826909211\\
599.88	0.00999953019466558\\
599.89	0.00999961419737891\\
599.9	0.00999969019514566\\
599.91	0.00999975810506767\\
599.92	0.00999981784342713\\
599.93	0.00999986932567848\\
599.94	0.00999991246644028\\
599.95	0.00999994717948697\\
599.96	0.00999997337774056\\
599.97	0.00999999097326228\\
599.98	0.00999999987724406\\
599.99	0.01\\
600	0.01\\
};
\addplot [color=mycolor6,solid,forget plot]
  table[row sep=crcr]{%
0.01	0.00499338655419056\\
1.01	0.00499338786632048\\
2.01	0.00499338920466236\\
3.01	0.00499339056973667\\
4.01	0.00499339196207427\\
5.01	0.00499339338221642\\
6.01	0.00499339483071506\\
7.01	0.00499339630813272\\
8.01	0.00499339781504335\\
9.01	0.0049933993520319\\
10.01	0.00499340091969511\\
11.01	0.00499340251864137\\
12.01	0.00499340414949088\\
13.01	0.00499340581287615\\
14.01	0.00499340750944205\\
15.01	0.00499340923984605\\
16.01	0.00499341100475871\\
17.01	0.00499341280486372\\
18.01	0.00499341464085801\\
19.01	0.00499341651345235\\
20.01	0.00499341842337139\\
21.01	0.00499342037135406\\
22.01	0.00499342235815352\\
23.01	0.00499342438453793\\
24.01	0.00499342645129039\\
25.01	0.00499342855920965\\
26.01	0.00499343070910956\\
27.01	0.00499343290182029\\
28.01	0.00499343513818806\\
29.01	0.00499343741907576\\
30.01	0.00499343974536304\\
31.01	0.00499344211794697\\
32.01	0.00499344453774193\\
33.01	0.00499344700568019\\
34.01	0.00499344952271234\\
35.01	0.00499345208980734\\
36.01	0.00499345470795319\\
37.01	0.00499345737815723\\
38.01	0.00499346010144638\\
39.01	0.00499346287886737\\
40.01	0.0049934657114879\\
41.01	0.00499346860039604\\
42.01	0.00499347154670089\\
43.01	0.00499347455153364\\
44.01	0.00499347761604714\\
45.01	0.00499348074141685\\
46.01	0.00499348392884111\\
47.01	0.0049934871795414\\
48.01	0.00499349049476293\\
49.01	0.00499349387577538\\
50.01	0.00499349732387294\\
51.01	0.00499350084037486\\
52.01	0.00499350442662611\\
53.01	0.00499350808399769\\
54.01	0.00499351181388723\\
55.01	0.00499351561771942\\
56.01	0.00499351949694649\\
57.01	0.00499352345304914\\
58.01	0.00499352748753636\\
59.01	0.00499353160194656\\
60.01	0.00499353579784786\\
61.01	0.00499354007683879\\
62.01	0.00499354444054856\\
63.01	0.0049935488906382\\
64.01	0.0049935534288007\\
65.01	0.00499355805676154\\
66.01	0.00499356277627991\\
67.01	0.0049935675891484\\
68.01	0.0049935724971947\\
69.01	0.00499357750228167\\
70.01	0.00499358260630776\\
71.01	0.00499358781120819\\
72.01	0.00499359311895536\\
73.01	0.00499359853155977\\
74.01	0.00499360405107056\\
75.01	0.00499360967957618\\
76.01	0.00499361541920532\\
77.01	0.00499362127212724\\
78.01	0.0049936272405534\\
79.01	0.00499363332673711\\
80.01	0.00499363953297549\\
81.01	0.00499364586160916\\
82.01	0.00499365231502408\\
83.01	0.00499365889565152\\
84.01	0.00499366560596964\\
85.01	0.0049936724485036\\
86.01	0.00499367942582712\\
87.01	0.00499368654056336\\
88.01	0.00499369379538488\\
89.01	0.00499370119301604\\
90.01	0.00499370873623257\\
91.01	0.00499371642786353\\
92.01	0.00499372427079173\\
93.01	0.00499373226795513\\
94.01	0.00499374042234736\\
95.01	0.00499374873701904\\
96.01	0.00499375721507912\\
97.01	0.00499376585969533\\
98.01	0.00499377467409542\\
99.01	0.004993783661569\\
100.01	0.00499379282546757\\
101.01	0.00499380216920652\\
102.01	0.00499381169626573\\
103.01	0.00499382141019117\\
104.01	0.00499383131459584\\
105.01	0.00499384141316096\\
106.01	0.0049938517096375\\
107.01	0.00499386220784727\\
108.01	0.0049938729116842\\
109.01	0.00499388382511565\\
110.01	0.00499389495218394\\
111.01	0.00499390629700729\\
112.01	0.0049939178637817\\
113.01	0.00499392965678191\\
114.01	0.00499394168036323\\
115.01	0.00499395393896263\\
116.01	0.00499396643710051\\
117.01	0.00499397917938196\\
118.01	0.00499399217049852\\
119.01	0.00499400541522942\\
120.01	0.00499401891844364\\
121.01	0.00499403268510105\\
122.01	0.0049940467202543\\
123.01	0.00499406102905034\\
124.01	0.00499407561673199\\
125.01	0.00499409048864025\\
126.01	0.00499410565021534\\
127.01	0.00499412110699898\\
128.01	0.00499413686463544\\
129.01	0.00499415292887462\\
130.01	0.00499416930557263\\
131.01	0.00499418600069457\\
132.01	0.0049942030203159\\
133.01	0.00499422037062471\\
134.01	0.00499423805792357\\
135.01	0.00499425608863156\\
136.01	0.00499427446928641\\
137.01	0.00499429320654646\\
138.01	0.00499431230719284\\
139.01	0.00499433177813158\\
140.01	0.00499435162639551\\
141.01	0.00499437185914724\\
142.01	0.00499439248368068\\
143.01	0.00499441350742372\\
144.01	0.00499443493794008\\
145.01	0.00499445678293226\\
146.01	0.0049944790502436\\
147.01	0.00499450174786076\\
148.01	0.00499452488391619\\
149.01	0.00499454846669074\\
150.01	0.00499457250461605\\
151.01	0.00499459700627705\\
152.01	0.00499462198041467\\
153.01	0.00499464743592884\\
154.01	0.00499467338188067\\
155.01	0.00499469982749551\\
156.01	0.00499472678216545\\
157.01	0.00499475425545247\\
158.01	0.00499478225709107\\
159.01	0.00499481079699134\\
160.01	0.00499483988524153\\
161.01	0.00499486953211163\\
162.01	0.0049948997480561\\
163.01	0.00499493054371672\\
164.01	0.00499496192992582\\
165.01	0.00499499391771016\\
166.01	0.00499502651829281\\
167.01	0.00499505974309748\\
168.01	0.00499509360375137\\
169.01	0.00499512811208846\\
170.01	0.00499516328015329\\
171.01	0.00499519912020387\\
172.01	0.00499523564471568\\
173.01	0.00499527286638476\\
174.01	0.00499531079813151\\
175.01	0.00499534945310437\\
176.01	0.00499538884468316\\
177.01	0.00499542898648321\\
178.01	0.00499546989235899\\
179.01	0.00499551157640772\\
180.01	0.00499555405297349\\
181.01	0.00499559733665106\\
182.01	0.00499564144228974\\
183.01	0.00499568638499785\\
184.01	0.00499573218014604\\
185.01	0.00499577884337205\\
186.01	0.00499582639058445\\
187.01	0.00499587483796736\\
188.01	0.00499592420198405\\
189.01	0.00499597449938159\\
190.01	0.00499602574719532\\
191.01	0.00499607796275328\\
192.01	0.00499613116368033\\
193.01	0.00499618536790311\\
194.01	0.00499624059365412\\
195.01	0.00499629685947682\\
196.01	0.00499635418423008\\
197.01	0.00499641258709284\\
198.01	0.00499647208756888\\
199.01	0.00499653270549189\\
200.01	0.00499659446103007\\
201.01	0.00499665737469117\\
202.01	0.00499672146732727\\
203.01	0.00499678676014038\\
204.01	0.00499685327468653\\
205.01	0.00499692103288189\\
206.01	0.0049969900570075\\
207.01	0.00499706036971419\\
208.01	0.00499713199402853\\
209.01	0.00499720495335762\\
210.01	0.00499727927149465\\
211.01	0.0049973549726242\\
212.01	0.00499743208132786\\
213.01	0.0049975106225897\\
214.01	0.00499759062180158\\
215.01	0.00499767210476922\\
216.01	0.00499775509771747\\
217.01	0.00499783962729614\\
218.01	0.00499792572058577\\
219.01	0.00499801340510302\\
220.01	0.00499810270880725\\
221.01	0.00499819366010555\\
222.01	0.00499828628785939\\
223.01	0.00499838062139017\\
224.01	0.004998476690485\\
225.01	0.00499857452540325\\
226.01	0.00499867415688249\\
227.01	0.00499877561614422\\
228.01	0.00499887893490024\\
229.01	0.00499898414535897\\
230.01	0.0049990912802317\\
231.01	0.00499920037273871\\
232.01	0.00499931145661536\\
233.01	0.00499942456611892\\
234.01	0.00499953973603448\\
235.01	0.00499965700168177\\
236.01	0.00499977639892117\\
237.01	0.00499989796416045\\
238.01	0.00500002173436129\\
239.01	0.00500014774704567\\
240.01	0.00500027604030258\\
241.01	0.00500040665279465\\
242.01	0.00500053962376464\\
243.01	0.00500067499304205\\
244.01	0.0050008128010505\\
245.01	0.00500095308881363\\
246.01	0.00500109589796245\\
247.01	0.00500124127074171\\
248.01	0.00500138925001742\\
249.01	0.0050015398792831\\
250.01	0.00500169320266722\\
251.01	0.00500184926494003\\
252.01	0.00500200811152057\\
253.01	0.00500216978848361\\
254.01	0.00500233434256725\\
255.01	0.00500250182118003\\
256.01	0.00500267227240769\\
257.01	0.00500284574502116\\
258.01	0.00500302228848373\\
259.01	0.00500320195295877\\
260.01	0.00500338478931665\\
261.01	0.00500357084914305\\
262.01	0.0050037601847466\\
263.01	0.00500395284916681\\
264.01	0.00500414889618174\\
265.01	0.00500434838031662\\
266.01	0.00500455135685159\\
267.01	0.00500475788183073\\
268.01	0.00500496801207019\\
269.01	0.00500518180516701\\
270.01	0.0050053993195083\\
271.01	0.0050056206142802\\
272.01	0.00500584574947749\\
273.01	0.00500607478591301\\
274.01	0.00500630778522779\\
275.01	0.00500654480990131\\
276.01	0.0050067859232618\\
277.01	0.00500703118949763\\
278.01	0.00500728067366785\\
279.01	0.00500753444171483\\
280.01	0.0050077925604757\\
281.01	0.0050080550976951\\
282.01	0.00500832212203871\\
283.01	0.00500859370310652\\
284.01	0.00500886991144747\\
285.01	0.00500915081857438\\
286.01	0.00500943649697945\\
287.01	0.00500972702015111\\
288.01	0.00501002246259069\\
289.01	0.00501032289983132\\
290.01	0.00501062840845641\\
291.01	0.00501093906612013\\
292.01	0.0050112549515685\\
293.01	0.00501157614466213\\
294.01	0.00501190272639887\\
295.01	0.00501223477894006\\
296.01	0.00501257238563586\\
297.01	0.00501291563105405\\
298.01	0.00501326460100887\\
299.01	0.00501361938259264\\
300.01	0.00501398006420875\\
301.01	0.00501434673560681\\
302.01	0.00501471948791993\\
303.01	0.00501509841370372\\
304.01	0.00501548360697839\\
305.01	0.00501587516327254\\
306.01	0.00501627317967014\\
307.01	0.00501667775486028\\
308.01	0.00501708898918908\\
309.01	0.00501750698471545\\
310.01	0.00501793184526991\\
311.01	0.00501836367651623\\
312.01	0.00501880258601761\\
313.01	0.00501924868330537\\
314.01	0.00501970207995231\\
315.01	0.00502016288964997\\
316.01	0.00502063122828927\\
317.01	0.00502110721404658\\
318.01	0.00502159096747391\\
319.01	0.00502208261159244\\
320.01	0.00502258227199301\\
321.01	0.00502309007693881\\
322.01	0.00502360615747541\\
323.01	0.00502413064754381\\
324.01	0.0050246636840998\\
325.01	0.00502520540723725\\
326.01	0.00502575596031725\\
327.01	0.00502631549010182\\
328.01	0.00502688414689204\\
329.01	0.00502746208467171\\
330.01	0.00502804946125478\\
331.01	0.00502864643843685\\
332.01	0.00502925318215132\\
333.01	0.00502986986262773\\
334.01	0.00503049665455403\\
335.01	0.00503113373724073\\
336.01	0.00503178129478642\\
337.01	0.00503243951624402\\
338.01	0.00503310859578748\\
339.01	0.0050337887328762\\
340.01	0.00503448013241763\\
341.01	0.00503518300492596\\
342.01	0.00503589756667509\\
343.01	0.00503662403984562\\
344.01	0.0050373626526614\\
345.01	0.00503811363951756\\
346.01	0.00503887724109417\\
347.01	0.0050396537044564\\
348.01	0.00504044328313653\\
349.01	0.00504124623719773\\
350.01	0.00504206283327525\\
351.01	0.00504289334459476\\
352.01	0.00504373805096422\\
353.01	0.00504459723873756\\
354.01	0.00504547120074937\\
355.01	0.00504636023621806\\
356.01	0.00504726465061738\\
357.01	0.00504818475551667\\
358.01	0.0050491208683893\\
359.01	0.00505007331239125\\
360.01	0.00505104241611436\\
361.01	0.00505202851331574\\
362.01	0.00505303194263147\\
363.01	0.00505405304728081\\
364.01	0.00505509217477058\\
365.01	0.00505614967661133\\
366.01	0.00505722590805743\\
367.01	0.00505832122788587\\
368.01	0.00505943599823176\\
369.01	0.00506057058449409\\
370.01	0.00506172535533017\\
371.01	0.0050629006827536\\
372.01	0.00506409694234744\\
373.01	0.00506531451359821\\
374.01	0.0050665537803536\\
375.01	0.00506781513138806\\
376.01	0.00506909896105726\\
377.01	0.00507040567000097\\
378.01	0.0050717356658435\\
379.01	0.00507308936382878\\
380.01	0.00507446718732461\\
381.01	0.0050758695681664\\
382.01	0.00507729694690821\\
383.01	0.00507874977306559\\
384.01	0.00508022850536756\\
385.01	0.00508173361201872\\
386.01	0.00508326557097012\\
387.01	0.00508482487020107\\
388.01	0.0050864120080109\\
389.01	0.0050880274933212\\
390.01	0.0050896718459903\\
391.01	0.00509134559713759\\
392.01	0.00509304928948121\\
393.01	0.00509478347768592\\
394.01	0.00509654872872523\\
395.01	0.00509834562225397\\
396.01	0.00510017475099574\\
397.01	0.00510203672114171\\
398.01	0.00510393215276361\\
399.01	0.00510586168023992\\
400.01	0.00510782595269503\\
401.01	0.00510982563445289\\
402.01	0.00511186140550353\\
403.01	0.00511393396198424\\
404.01	0.00511604401667314\\
405.01	0.00511819229949688\\
406.01	0.00512037955805184\\
407.01	0.00512260655813837\\
408.01	0.00512487408430706\\
409.01	0.00512718294041817\\
410.01	0.00512953395021211\\
411.01	0.00513192795789152\\
412.01	0.00513436582871407\\
413.01	0.0051368484495937\\
414.01	0.00513937672971189\\
415.01	0.00514195160113549\\
416.01	0.00514457401944182\\
417.01	0.00514724496434761\\
418.01	0.00514996544034281\\
419.01	0.00515273647732563\\
420.01	0.00515555913123816\\
421.01	0.00515843448470016\\
422.01	0.00516136364764033\\
423.01	0.00516434775792068\\
424.01	0.00516738798195487\\
425.01	0.00517048551531558\\
426.01	0.00517364158333061\\
427.01	0.00517685744166467\\
428.01	0.00518013437688505\\
429.01	0.00518347370700885\\
430.01	0.00518687678202942\\
431.01	0.00519034498442225\\
432.01	0.0051938797296254\\
433.01	0.00519748246649712\\
434.01	0.00520115467774707\\
435.01	0.00520489788034244\\
436.01	0.00520871362588934\\
437.01	0.00521260350098952\\
438.01	0.00521656912757588\\
439.01	0.00522061216322811\\
440.01	0.00522473430147351\\
441.01	0.0052289372720771\\
442.01	0.00523322284132761\\
443.01	0.00523759281232565\\
444.01	0.00524204902528289\\
445.01	0.00524659335784246\\
446.01	0.00525122772542844\\
447.01	0.00525595408163848\\
448.01	0.00526077441868902\\
449.01	0.00526569076792553\\
450.01	0.00527070520040915\\
451.01	0.00527581982759093\\
452.01	0.0052810368020802\\
453.01	0.00528635831851634\\
454.01	0.00529178661454525\\
455.01	0.0052973239718996\\
456.01	0.00530297271757832\\
457.01	0.00530873522511274\\
458.01	0.00531461391590231\\
459.01	0.00532061126059696\\
460.01	0.00532672978049861\\
461.01	0.00533297204894941\\
462.01	0.00533934069267547\\
463.01	0.00534583839305854\\
464.01	0.00535246788731461\\
465.01	0.00535923196957572\\
466.01	0.00536613349188794\\
467.01	0.00537317536516092\\
468.01	0.00538036056011246\\
469.01	0.00538769210822991\\
470.01	0.00539517310275549\\
471.01	0.00540280669969907\\
472.01	0.00541059611888273\\
473.01	0.00541854464502098\\
474.01	0.00542665562884202\\
475.01	0.00543493248825357\\
476.01	0.00544337870955725\\
477.01	0.00545199784871682\\
478.01	0.00546079353268274\\
479.01	0.00546976946077612\\
480.01	0.0054789294061358\\
481.01	0.00548827721722787\\
482.01	0.00549781681941935\\
483.01	0.00550755221661568\\
484.01	0.00551748749295894\\
485.01	0.00552762681458464\\
486.01	0.00553797443143256\\
487.01	0.00554853467910639\\
488.01	0.00555931198077608\\
489.01	0.0055703108491171\\
490.01	0.00558153588827973\\
491.01	0.00559299179588486\\
492.01	0.00560468336504057\\
493.01	0.00561661548637882\\
494.01	0.00562879315011237\\
495.01	0.00564122144811656\\
496.01	0.0056539055760409\\
497.01	0.0056668508354585\\
498.01	0.00568006263606116\\
499.01	0.0056935464979061\\
500.01	0.00570730805371605\\
501.01	0.00572135305123445\\
502.01	0.00573568735563307\\
503.01	0.00575031695197263\\
504.01	0.00576524794771362\\
505.01	0.00578048657527726\\
506.01	0.0057960391946542\\
507.01	0.00581191229605887\\
508.01	0.00582811250262841\\
509.01	0.00584464657316448\\
510.01	0.00586152140491583\\
511.01	0.00587874403640189\\
512.01	0.0058963216502738\\
513.01	0.00591426157621598\\
514.01	0.00593257129388382\\
515.01	0.00595125843587969\\
516.01	0.00597033079076532\\
517.01	0.00598979630611013\\
518.01	0.00600966309157334\\
519.01	0.00602993942201802\\
520.01	0.00605063374065308\\
521.01	0.00607175466219925\\
522.01	0.00609331097607468\\
523.01	0.00611531164959501\\
524.01	0.00613776583118063\\
525.01	0.00616068285356548\\
526.01	0.00618407223699981\\
527.01	0.00620794369243782\\
528.01	0.00623230712470173\\
529.01	0.00625717263560928\\
530.01	0.00628255052705605\\
531.01	0.00630845130403507\\
532.01	0.00633488567758089\\
533.01	0.00636186456761978\\
534.01	0.00638939910570525\\
535.01	0.0064175006376183\\
536.01	0.0064461807258063\\
537.01	0.00647545115163257\\
538.01	0.0065053239174051\\
539.01	0.00653581124814943\\
540.01	0.00656692559308616\\
541.01	0.00659867962676944\\
542.01	0.00663108624983591\\
543.01	0.00666415858931164\\
544.01	0.00669790999841302\\
545.01	0.00673235405577483\\
546.01	0.00676750456402943\\
547.01	0.00680337554765068\\
548.01	0.00683998124996985\\
549.01	0.00687733612925698\\
550.01	0.00691545485375123\\
551.01	0.0069543522955103\\
552.01	0.00699404352293434\\
553.01	0.00703454379180462\\
554.01	0.00707586853465878\\
555.01	0.00711803334830747\\
556.01	0.0071610539792728\\
557.01	0.00720494630691036\\
558.01	0.00724972632394758\\
559.01	0.00729541011414746\\
560.01	0.00734201382677425\\
561.01	0.00738955364750814\\
562.01	0.0074380457654204\\
563.01	0.00748750633558497\\
564.01	0.00753795143686202\\
565.01	0.00758939702435083\\
566.01	0.00764185887596373\\
567.01	0.00769535253253253\\
568.01	0.00774989323081213\\
569.01	0.00780549582870473\\
570.01	0.00786217472198644\\
571.01	0.007919943751782\\
572.01	0.00797881610200229\\
573.01	0.00803880418594278\\
574.01	0.0080999195212333\\
575.01	0.00816217259234991\\
576.01	0.00822557269994251\\
577.01	0.00829012779631636\\
578.01	0.00835584430654082\\
579.01	0.00842272693485917\\
580.01	0.00849077845636217\\
581.01	0.00855999949428942\\
582.01	0.00863038828386762\\
583.01	0.00870194042432736\\
584.01	0.0087746486217084\\
585.01	0.00884850242633413\\
586.01	0.00892348797048993\\
587.01	0.00899958771397793\\
588.01	0.00907678020797329\\
589.01	0.00915503989113028\\
590.01	0.0092343369363852\\
591.01	0.00931463717262456\\
592.01	0.00939590211264666\\
593.01	0.00947808912803386\\
594.01	0.00956115182316124\\
595.01	0.00964504067520493\\
596.01	0.00972970402544469\\
597.01	0.0098150895303299\\
598.01	0.00990114621169971\\
599.01	0.00996919203046377\\
599.02	0.00996973314335038\\
599.03	0.0099702709571694\\
599.04	0.00997080543920336\\
599.05	0.0099713365564138\\
599.06	0.00997186427543808\\
599.07	0.0099723885625862\\
599.08	0.00997290938383756\\
599.09	0.00997342670483771\\
599.1	0.00997394049089505\\
599.11	0.00997445070697751\\
599.12	0.00997495731770918\\
599.13	0.00997546028736696\\
599.14	0.00997595957987707\\
599.15	0.00997645515881166\\
599.16	0.00997694698738526\\
599.17	0.00997743502845131\\
599.18	0.00997791924449856\\
599.19	0.00997839959764748\\
599.2	0.00997887604964662\\
599.21	0.00997934856186899\\
599.22	0.00997981709530829\\
599.23	0.00998028161057521\\
599.24	0.00998074206789365\\
599.25	0.0099811984270969\\
599.26	0.00998165064762378\\
599.27	0.00998209868851477\\
599.28	0.00998254250840806\\
599.29	0.00998298206553559\\
599.3	0.00998341731771905\\
599.31	0.00998384822236584\\
599.32	0.00998427473646497\\
599.33	0.00998469681658292\\
599.34	0.00998511441885952\\
599.35	0.0099855274990037\\
599.36	0.00998593601228926\\
599.37	0.00998633991355056\\
599.38	0.00998673915717821\\
599.39	0.00998713369711469\\
599.4	0.00998752348684992\\
599.41	0.00998790847811156\\
599.42	0.00998828861974491\\
599.43	0.00998866386008804\\
599.44	0.00998903414696681\\
599.45	0.00998939942768985\\
599.46	0.00998975964904344\\
599.47	0.00999011475728636\\
599.48	0.00999046469814472\\
599.49	0.00999080941680669\\
599.5	0.00999114885791725\\
599.51	0.00999148296557278\\
599.52	0.0099918116833157\\
599.53	0.00999213495412901\\
599.54	0.00999245272043077\\
599.55	0.00999276492406855\\
599.56	0.00999307150631382\\
599.57	0.00999337240785624\\
599.58	0.00999366756879799\\
599.59	0.00999395692864795\\
599.6	0.00999424042631585\\
599.61	0.00999451800010642\\
599.62	0.00999478958771336\\
599.63	0.0099950551262134\\
599.64	0.00999531455206019\\
599.65	0.00999556780107816\\
599.66	0.00999581480845634\\
599.67	0.00999605550874211\\
599.68	0.00999628983583487\\
599.69	0.00999651772297967\\
599.7	0.00999673910276076\\
599.71	0.00999695390709509\\
599.72	0.00999716206722577\\
599.73	0.00999736351371538\\
599.74	0.00999755817643933\\
599.75	0.00999774598457904\\
599.76	0.00999792686661517\\
599.77	0.00999810075032068\\
599.78	0.00999826756275386\\
599.79	0.00999842723025133\\
599.8	0.00999857967842092\\
599.81	0.0099987248321345\\
599.82	0.00999886261552071\\
599.83	0.00999899295195769\\
599.84	0.00999911576406567\\
599.85	0.00999923097369951\\
599.86	0.00999933850194117\\
599.87	0.00999943826909211\\
599.88	0.00999953019466558\\
599.89	0.00999961419737891\\
599.9	0.00999969019514566\\
599.91	0.00999975810506767\\
599.92	0.00999981784342713\\
599.93	0.00999986932567848\\
599.94	0.00999991246644028\\
599.95	0.00999994717948697\\
599.96	0.00999997337774056\\
599.97	0.00999999097326228\\
599.98	0.00999999987724406\\
599.99	0.01\\
600	0.01\\
};
\addplot [color=mycolor7,solid,forget plot]
  table[row sep=crcr]{%
0.01	0.00495449873497758\\
1.01	0.00495450029197152\\
2.01	0.00495450188039694\\
3.01	0.00495450350088551\\
4.01	0.00495450515408258\\
5.01	0.00495450684064533\\
6.01	0.0049545085612448\\
7.01	0.00495451031656516\\
8.01	0.0049545121073042\\
9.01	0.00495451393417384\\
10.01	0.00495451579790009\\
11.01	0.00495451769922321\\
12.01	0.00495451963889873\\
13.01	0.00495452161769698\\
14.01	0.00495452363640381\\
15.01	0.00495452569582075\\
16.01	0.00495452779676507\\
17.01	0.00495452994007062\\
18.01	0.00495453212658796\\
19.01	0.00495453435718417\\
20.01	0.00495453663274401\\
21.01	0.00495453895416986\\
22.01	0.00495454132238195\\
23.01	0.00495454373831894\\
24.01	0.0049545462029381\\
25.01	0.00495454871721564\\
26.01	0.00495455128214774\\
27.01	0.00495455389874973\\
28.01	0.00495455656805779\\
29.01	0.00495455929112838\\
30.01	0.00495456206903892\\
31.01	0.00495456490288834\\
32.01	0.00495456779379762\\
33.01	0.00495457074290986\\
34.01	0.0049545737513909\\
35.01	0.00495457682042971\\
36.01	0.00495457995123902\\
37.01	0.00495458314505552\\
38.01	0.00495458640314056\\
39.01	0.0049545897267807\\
40.01	0.00495459311728746\\
41.01	0.00495459657599899\\
42.01	0.00495460010427994\\
43.01	0.0049546037035217\\
44.01	0.00495460737514361\\
45.01	0.00495461112059283\\
46.01	0.0049546149413455\\
47.01	0.00495461883890681\\
48.01	0.0049546228148121\\
49.01	0.0049546268706267\\
50.01	0.00495463100794719\\
51.01	0.00495463522840174\\
52.01	0.00495463953365069\\
53.01	0.00495464392538732\\
54.01	0.00495464840533831\\
55.01	0.00495465297526451\\
56.01	0.00495465763696179\\
57.01	0.00495466239226126\\
58.01	0.00495466724303043\\
59.01	0.00495467219117345\\
60.01	0.00495467723863223\\
61.01	0.00495468238738704\\
62.01	0.00495468763945725\\
63.01	0.00495469299690213\\
64.01	0.00495469846182128\\
65.01	0.00495470403635614\\
66.01	0.00495470972268981\\
67.01	0.00495471552304907\\
68.01	0.00495472143970416\\
69.01	0.00495472747497004\\
70.01	0.00495473363120746\\
71.01	0.00495473991082342\\
72.01	0.00495474631627246\\
73.01	0.00495475285005728\\
74.01	0.00495475951472986\\
75.01	0.00495476631289217\\
76.01	0.00495477324719747\\
77.01	0.00495478032035129\\
78.01	0.00495478753511203\\
79.01	0.0049547948942924\\
80.01	0.00495480240076\\
81.01	0.00495481005743911\\
82.01	0.00495481786731085\\
83.01	0.00495482583341533\\
84.01	0.00495483395885171\\
85.01	0.00495484224678019\\
86.01	0.0049548507004227\\
87.01	0.00495485932306419\\
88.01	0.00495486811805396\\
89.01	0.0049548770888066\\
90.01	0.00495488623880375\\
91.01	0.00495489557159466\\
92.01	0.00495490509079834\\
93.01	0.00495491480010417\\
94.01	0.00495492470327344\\
95.01	0.00495493480414112\\
96.01	0.00495494510661651\\
97.01	0.00495495561468547\\
98.01	0.00495496633241142\\
99.01	0.00495497726393669\\
100.01	0.00495498841348437\\
101.01	0.00495499978535961\\
102.01	0.00495501138395121\\
103.01	0.00495502321373342\\
104.01	0.0049550352792672\\
105.01	0.00495504758520215\\
106.01	0.00495506013627819\\
107.01	0.00495507293732686\\
108.01	0.00495508599327349\\
109.01	0.00495509930913893\\
110.01	0.00495511289004101\\
111.01	0.00495512674119682\\
112.01	0.00495514086792431\\
113.01	0.00495515527564409\\
114.01	0.0049551699698817\\
115.01	0.00495518495626897\\
116.01	0.00495520024054676\\
117.01	0.00495521582856674\\
118.01	0.004955231726293\\
119.01	0.00495524793980481\\
120.01	0.00495526447529829\\
121.01	0.00495528133908892\\
122.01	0.00495529853761351\\
123.01	0.00495531607743251\\
124.01	0.00495533396523258\\
125.01	0.00495535220782854\\
126.01	0.00495537081216579\\
127.01	0.00495538978532296\\
128.01	0.00495540913451452\\
129.01	0.00495542886709247\\
130.01	0.00495544899054995\\
131.01	0.0049554695125229\\
132.01	0.00495549044079345\\
133.01	0.00495551178329211\\
134.01	0.00495553354810058\\
135.01	0.00495555574345463\\
136.01	0.00495557837774689\\
137.01	0.00495560145952961\\
138.01	0.00495562499751775\\
139.01	0.00495564900059182\\
140.01	0.00495567347780106\\
141.01	0.00495569843836585\\
142.01	0.00495572389168203\\
143.01	0.00495574984732285\\
144.01	0.0049557763150429\\
145.01	0.00495580330478111\\
146.01	0.0049558308266643\\
147.01	0.00495585889101022\\
148.01	0.00495588750833134\\
149.01	0.00495591668933791\\
150.01	0.00495594644494192\\
151.01	0.00495597678626046\\
152.01	0.00495600772461968\\
153.01	0.00495603927155806\\
154.01	0.00495607143883059\\
155.01	0.00495610423841204\\
156.01	0.0049561376825016\\
157.01	0.00495617178352639\\
158.01	0.00495620655414555\\
159.01	0.00495624200725424\\
160.01	0.00495627815598809\\
161.01	0.00495631501372726\\
162.01	0.00495635259410016\\
163.01	0.00495639091098891\\
164.01	0.00495642997853274\\
165.01	0.00495646981113275\\
166.01	0.00495651042345693\\
167.01	0.00495655183044394\\
168.01	0.00495659404730854\\
169.01	0.00495663708954593\\
170.01	0.00495668097293626\\
171.01	0.00495672571355051\\
172.01	0.00495677132775403\\
173.01	0.00495681783221298\\
174.01	0.00495686524389857\\
175.01	0.00495691358009234\\
176.01	0.00495696285839146\\
177.01	0.00495701309671413\\
178.01	0.00495706431330482\\
179.01	0.00495711652673982\\
180.01	0.0049571697559327\\
181.01	0.00495722402013991\\
182.01	0.00495727933896652\\
183.01	0.00495733573237175\\
184.01	0.00495739322067508\\
185.01	0.00495745182456218\\
186.01	0.00495751156509033\\
187.01	0.00495757246369501\\
188.01	0.00495763454219597\\
189.01	0.00495769782280309\\
190.01	0.00495776232812312\\
191.01	0.00495782808116539\\
192.01	0.00495789510534889\\
193.01	0.00495796342450822\\
194.01	0.00495803306290048\\
195.01	0.00495810404521171\\
196.01	0.00495817639656367\\
197.01	0.00495825014252073\\
198.01	0.00495832530909653\\
199.01	0.00495840192276086\\
200.01	0.00495848001044674\\
201.01	0.00495855959955767\\
202.01	0.00495864071797426\\
203.01	0.00495872339406181\\
204.01	0.00495880765667776\\
205.01	0.00495889353517825\\
206.01	0.00495898105942588\\
207.01	0.00495907025979767\\
208.01	0.00495916116719171\\
209.01	0.00495925381303512\\
210.01	0.00495934822929152\\
211.01	0.00495944444846912\\
212.01	0.00495954250362785\\
213.01	0.00495964242838731\\
214.01	0.00495974425693477\\
215.01	0.00495984802403256\\
216.01	0.00495995376502656\\
217.01	0.00496006151585361\\
218.01	0.00496017131304975\\
219.01	0.00496028319375823\\
220.01	0.00496039719573714\\
221.01	0.00496051335736789\\
222.01	0.00496063171766306\\
223.01	0.00496075231627453\\
224.01	0.00496087519350162\\
225.01	0.00496100039029876\\
226.01	0.00496112794828414\\
227.01	0.0049612579097475\\
228.01	0.00496139031765823\\
229.01	0.00496152521567356\\
230.01	0.00496166264814635\\
231.01	0.00496180266013309\\
232.01	0.00496194529740224\\
233.01	0.00496209060644175\\
234.01	0.00496223863446724\\
235.01	0.00496238942942935\\
236.01	0.00496254304002211\\
237.01	0.0049626995156903\\
238.01	0.00496285890663714\\
239.01	0.00496302126383162\\
240.01	0.00496318663901635\\
241.01	0.0049633550847142\\
242.01	0.00496352665423608\\
243.01	0.00496370140168802\\
244.01	0.00496387938197736\\
245.01	0.00496406065082051\\
246.01	0.00496424526474883\\
247.01	0.00496443328111536\\
248.01	0.00496462475810122\\
249.01	0.00496481975472149\\
250.01	0.00496501833083085\\
251.01	0.00496522054712963\\
252.01	0.00496542646516901\\
253.01	0.00496563614735633\\
254.01	0.00496584965695947\\
255.01	0.00496606705811198\\
256.01	0.0049662884158171\\
257.01	0.004966513795952\\
258.01	0.00496674326527102\\
259.01	0.00496697689140936\\
260.01	0.0049672147428861\\
261.01	0.00496745688910649\\
262.01	0.00496770340036452\\
263.01	0.00496795434784431\\
264.01	0.00496820980362171\\
265.01	0.00496846984066524\\
266.01	0.00496873453283674\\
267.01	0.00496900395489093\\
268.01	0.00496927818247525\\
269.01	0.00496955729212913\\
270.01	0.0049698413612816\\
271.01	0.00497013046825032\\
272.01	0.0049704246922382\\
273.01	0.0049707241133307\\
274.01	0.00497102881249172\\
275.01	0.00497133887155958\\
276.01	0.00497165437324192\\
277.01	0.00497197540111015\\
278.01	0.00497230203959358\\
279.01	0.00497263437397219\\
280.01	0.0049729724903692\\
281.01	0.00497331647574369\\
282.01	0.00497366641788095\\
283.01	0.00497402240538389\\
284.01	0.00497438452766298\\
285.01	0.00497475287492517\\
286.01	0.00497512753816343\\
287.01	0.00497550860914453\\
288.01	0.00497589618039711\\
289.01	0.00497629034519845\\
290.01	0.00497669119756142\\
291.01	0.00497709883222086\\
292.01	0.004977513344619\\
293.01	0.00497793483089126\\
294.01	0.00497836338785153\\
295.01	0.00497879911297684\\
296.01	0.0049792421043927\\
297.01	0.00497969246085737\\
298.01	0.00498015028174784\\
299.01	0.00498061566704373\\
300.01	0.00498108871731409\\
301.01	0.00498156953370295\\
302.01	0.00498205821791588\\
303.01	0.00498255487220821\\
304.01	0.00498305959937317\\
305.01	0.00498357250273236\\
306.01	0.00498409368612773\\
307.01	0.00498462325391459\\
308.01	0.00498516131095768\\
309.01	0.00498570796262973\\
310.01	0.00498626331481203\\
311.01	0.0049868274738999\\
312.01	0.00498740054681021\\
313.01	0.00498798264099452\\
314.01	0.0049885738644562\\
315.01	0.00498917432577321\\
316.01	0.00498978413412694\\
317.01	0.00499040339933726\\
318.01	0.00499103223190508\\
319.01	0.00499167074306354\\
320.01	0.00499231904483693\\
321.01	0.00499297725011064\\
322.01	0.00499364547271063\\
323.01	0.00499432382749534\\
324.01	0.00499501243046012\\
325.01	0.00499571139885524\\
326.01	0.00499642085131839\\
327.01	0.00499714090802421\\
328.01	0.00499787169085036\\
329.01	0.00499861332356202\\
330.01	0.00499936593201675\\
331.01	0.00500012964438984\\
332.01	0.00500090459142272\\
333.01	0.0050016909066939\\
334.01	0.00500248872691636\\
335.01	0.00500329819225946\\
336.01	0.00500411944669937\\
337.01	0.00500495263839654\\
338.01	0.00500579792010264\\
339.01	0.00500665544959673\\
340.01	0.00500752539015163\\
341.01	0.00500840791102919\\
342.01	0.0050093031880058\\
343.01	0.00501021140392467\\
344.01	0.00501113274927688\\
345.01	0.00501206742280441\\
346.01	0.0050130156321261\\
347.01	0.0050139775943794\\
348.01	0.00501495353687549\\
349.01	0.00501594369775796\\
350.01	0.00501694832666061\\
351.01	0.00501796768535226\\
352.01	0.00501900204835869\\
353.01	0.00502005170354786\\
354.01	0.00502111695266431\\
355.01	0.00502219811179455\\
356.01	0.00502329551174445\\
357.01	0.00502440949830703\\
358.01	0.00502554043239732\\
359.01	0.00502668869002935\\
360.01	0.00502785466210898\\
361.01	0.00502903875401514\\
362.01	0.00503024138494439\\
363.01	0.00503146298699349\\
364.01	0.0050327040039593\\
365.01	0.00503396488984265\\
366.01	0.00503524610704879\\
367.01	0.00503654812429304\\
368.01	0.00503787141423349\\
369.01	0.00503921645087724\\
370.01	0.00504058370683111\\
371.01	0.0050419736505011\\
372.01	0.00504338674337992\\
373.01	0.00504482343760372\\
374.01	0.00504628417399551\\
375.01	0.00504776938084786\\
376.01	0.00504927947371088\\
377.01	0.00505081485643616\\
378.01	0.00505237592364963\\
379.01	0.00505396306465568\\
380.01	0.00505557666840886\\
381.01	0.00505721712776508\\
382.01	0.00505888484080708\\
383.01	0.00506058021101743\\
384.01	0.00506230364741797\\
385.01	0.00506405556471838\\
386.01	0.00506583638347773\\
387.01	0.00506764653027661\\
388.01	0.00506948643790339\\
389.01	0.00507135654555459\\
390.01	0.00507325729904906\\
391.01	0.00507518915106002\\
392.01	0.00507715256136289\\
393.01	0.00507914799710259\\
394.01	0.00508117593307975\\
395.01	0.0050832368520585\\
396.01	0.00508533124509594\\
397.01	0.00508745961189566\\
398.01	0.00508962246118645\\
399.01	0.00509182031112708\\
400.01	0.00509405368973993\\
401.01	0.00509632313537381\\
402.01	0.00509862919719781\\
403.01	0.00510097243572839\\
404.01	0.00510335342339101\\
405.01	0.00510577274511721\\
406.01	0.00510823099897995\\
407.01	0.00511072879686754\\
408.01	0.0051132667651991\\
409.01	0.00511584554568112\\
410.01	0.00511846579610811\\
411.01	0.0051211281912082\\
412.01	0.0051238334235334\\
413.01	0.00512658220439763\\
414.01	0.00512937526486082\\
415.01	0.00513221335676202\\
416.01	0.00513509725379858\\
417.01	0.00513802775265401\\
418.01	0.00514100567417107\\
419.01	0.00514403186457146\\
420.01	0.00514710719671833\\
421.01	0.00515023257141975\\
422.01	0.00515340891877078\\
423.01	0.00515663719952925\\
424.01	0.00515991840652104\\
425.01	0.00516325356606923\\
426.01	0.00516664373944022\\
427.01	0.00517009002429979\\
428.01	0.00517359355616892\\
429.01	0.00517715550987126\\
430.01	0.00518077710095871\\
431.01	0.00518445958710489\\
432.01	0.00518820426945177\\
433.01	0.00519201249389373\\
434.01	0.00519588565228472\\
435.01	0.00519982518355067\\
436.01	0.00520383257468861\\
437.01	0.00520790936163687\\
438.01	0.00521205712999526\\
439.01	0.00521627751557836\\
440.01	0.00522057220478593\\
441.01	0.00522494293477301\\
442.01	0.00522939149340879\\
443.01	0.00523391971901313\\
444.01	0.00523852949986739\\
445.01	0.00524322277349935\\
446.01	0.00524800152575124\\
447.01	0.00525286778964769\\
448.01	0.0052578236440903\\
449.01	0.005262871212418\\
450.01	0.00526801266088528\\
451.01	0.00527325019712209\\
452.01	0.00527858606865597\\
453.01	0.00528402256158897\\
454.01	0.0052895619995329\\
455.01	0.00529520674291668\\
456.01	0.00530095918878035\\
457.01	0.00530682177116795\\
458.01	0.00531279696221322\\
459.01	0.00531888727398484\\
460.01	0.0053250952611094\\
461.01	0.00533142352412674\\
462.01	0.00533787471344524\\
463.01	0.00534445153366397\\
464.01	0.0053511567479194\\
465.01	0.00535799318182135\\
466.01	0.00536496372649997\\
467.01	0.00537207134042046\\
468.01	0.00537931905026122\\
469.01	0.00538670995152291\\
470.01	0.00539424720906119\\
471.01	0.00540193405755518\\
472.01	0.00540977380191732\\
473.01	0.00541776981765332\\
474.01	0.00542592555118307\\
475.01	0.00543424452013756\\
476.01	0.00544273031364788\\
477.01	0.00545138659264675\\
478.01	0.00546021709020519\\
479.01	0.00546922561192885\\
480.01	0.00547841603644079\\
481.01	0.00548779231597983\\
482.01	0.0054973584771407\\
483.01	0.00550711862178339\\
484.01	0.00551707692813627\\
485.01	0.00552723765211148\\
486.01	0.00553760512884571\\
487.01	0.005548183774471\\
488.01	0.0055589780881089\\
489.01	0.00556999265406935\\
490.01	0.00558123214422312\\
491.01	0.00559270132050174\\
492.01	0.00560440503747062\\
493.01	0.00561634824491018\\
494.01	0.00562853599034233\\
495.01	0.00564097342144393\\
496.01	0.00565366578831375\\
497.01	0.00566661844558945\\
498.01	0.00567983685445803\\
499.01	0.00569332658464414\\
500.01	0.00570709331646155\\
501.01	0.00572114284296089\\
502.01	0.00573548107217994\\
503.01	0.00575011402949795\\
504.01	0.00576504786009478\\
505.01	0.0057802888315108\\
506.01	0.00579584333630428\\
507.01	0.00581171789479883\\
508.01	0.00582791915791098\\
509.01	0.0058444539100473\\
510.01	0.00586132907205845\\
511.01	0.00587855170423747\\
512.01	0.00589612900934987\\
513.01	0.0059140683356845\\
514.01	0.00593237718011849\\
515.01	0.00595106319119111\\
516.01	0.00597013417218746\\
517.01	0.0059895980842368\\
518.01	0.00600946304943295\\
519.01	0.00602973735398422\\
520.01	0.00605042945139851\\
521.01	0.00607154796570211\\
522.01	0.00609310169468649\\
523.01	0.00611509961317347\\
524.01	0.00613755087629289\\
525.01	0.00616046482276039\\
526.01	0.00618385097814808\\
527.01	0.00620771905813523\\
528.01	0.00623207897172903\\
529.01	0.00625694082444449\\
530.01	0.00628231492143\\
531.01	0.00630821177052611\\
532.01	0.00633464208524313\\
533.01	0.00636161678764127\\
534.01	0.00638914701109523\\
535.01	0.00641724410292135\\
536.01	0.00644591962684281\\
537.01	0.00647518536526556\\
538.01	0.00650505332133199\\
539.01	0.00653553572071758\\
540.01	0.00656664501313042\\
541.01	0.00659839387346958\\
542.01	0.00663079520259363\\
543.01	0.00666386212764396\\
544.01	0.00669760800186221\\
545.01	0.00673204640383408\\
546.01	0.00676719113608257\\
547.01	0.00680305622292752\\
548.01	0.00683965590751582\\
549.01	0.00687700464791808\\
550.01	0.00691511711217481\\
551.01	0.00695400817216172\\
552.01	0.00699369289613081\\
553.01	0.00703418653976667\\
554.01	0.00707550453558119\\
555.01	0.00711766248044977\\
556.01	0.00716067612107247\\
557.01	0.00720456133711941\\
558.01	0.00724933412179551\\
559.01	0.0072950105595322\\
560.01	0.00734160680048446\\
561.01	0.00738913903147977\\
562.01	0.00743762344303022\\
563.01	0.00748707619198446\\
564.01	0.00753751335935612\\
565.01	0.00758895090282426\\
566.01	0.00764140460335994\\
567.01	0.00769489000538857\\
568.01	0.00774942234985408\\
569.01	0.00780501649950793\\
570.01	0.00786168685570438\\
571.01	0.00791944726594716\\
572.01	0.00797831092140276\\
573.01	0.00803829024357548\\
574.01	0.00809939675933612\\
575.01	0.0081616409635105\\
576.01	0.00822503216828076\\
577.01	0.00828957833873377\\
578.01	0.00835528591402531\\
579.01	0.00842215961382854\\
580.01	0.00849020223002198\\
581.01	0.00855941440397067\\
582.01	0.00862979439029825\\
583.01	0.00870133780877493\\
584.01	0.00877403738691154\\
585.01	0.00884788269711461\\
586.01	0.0089228598939045\\
587.01	0.00899895145882848\\
588.01	0.00907613596344161\\
589.01	0.00915438786423933\\
590.01	0.00923367734790629\\
591.01	0.00931397025094757\\
592.01	0.00939522808500194\\
593.01	0.00947740820829374\\
594.01	0.00956046419524734\\
595.01	0.00964434647087689\\
596.01	0.00972900329493185\\
597.01	0.0098143822038764\\
598.01	0.00990043204779049\\
599.01	0.00996919203046377\\
599.02	0.00996973314335038\\
599.03	0.0099702709571694\\
599.04	0.00997080543920336\\
599.05	0.0099713365564138\\
599.06	0.00997186427543808\\
599.07	0.0099723885625862\\
599.08	0.00997290938383756\\
599.09	0.00997342670483771\\
599.1	0.00997394049089505\\
599.11	0.00997445070697751\\
599.12	0.00997495731770918\\
599.13	0.00997546028736696\\
599.14	0.00997595957987707\\
599.15	0.00997645515881165\\
599.16	0.00997694698738526\\
599.17	0.00997743502845131\\
599.18	0.00997791924449856\\
599.19	0.00997839959764748\\
599.2	0.00997887604964662\\
599.21	0.00997934856186899\\
599.22	0.00997981709530829\\
599.23	0.00998028161057521\\
599.24	0.00998074206789365\\
599.25	0.0099811984270969\\
599.26	0.00998165064762378\\
599.27	0.00998209868851477\\
599.28	0.00998254250840806\\
599.29	0.00998298206553559\\
599.3	0.00998341731771905\\
599.31	0.00998384822236584\\
599.32	0.00998427473646497\\
599.33	0.00998469681658292\\
599.34	0.00998511441885952\\
599.35	0.0099855274990037\\
599.36	0.00998593601228926\\
599.37	0.00998633991355056\\
599.38	0.00998673915717821\\
599.39	0.0099871336971147\\
599.4	0.00998752348684992\\
599.41	0.00998790847811157\\
599.42	0.00998828861974491\\
599.43	0.00998866386008804\\
599.44	0.00998903414696681\\
599.45	0.00998939942768985\\
599.46	0.00998975964904344\\
599.47	0.00999011475728636\\
599.48	0.00999046469814472\\
599.49	0.00999080941680669\\
599.5	0.00999114885791725\\
599.51	0.00999148296557278\\
599.52	0.0099918116833157\\
599.53	0.00999213495412901\\
599.54	0.00999245272043077\\
599.55	0.00999276492406855\\
599.56	0.00999307150631381\\
599.57	0.00999337240785624\\
599.58	0.00999366756879799\\
599.59	0.00999395692864795\\
599.6	0.00999424042631586\\
599.61	0.00999451800010642\\
599.62	0.00999478958771336\\
599.63	0.0099950551262134\\
599.64	0.00999531455206019\\
599.65	0.00999556780107816\\
599.66	0.00999581480845634\\
599.67	0.00999605550874211\\
599.68	0.00999628983583487\\
599.69	0.00999651772297967\\
599.7	0.00999673910276076\\
599.71	0.00999695390709509\\
599.72	0.00999716206722577\\
599.73	0.00999736351371538\\
599.74	0.00999755817643933\\
599.75	0.00999774598457904\\
599.76	0.00999792686661517\\
599.77	0.00999810075032067\\
599.78	0.00999826756275386\\
599.79	0.00999842723025133\\
599.8	0.00999857967842092\\
599.81	0.0099987248321345\\
599.82	0.00999886261552071\\
599.83	0.00999899295195769\\
599.84	0.00999911576406567\\
599.85	0.00999923097369951\\
599.86	0.00999933850194117\\
599.87	0.00999943826909211\\
599.88	0.00999953019466558\\
599.89	0.00999961419737891\\
599.9	0.00999969019514566\\
599.91	0.00999975810506767\\
599.92	0.00999981784342713\\
599.93	0.00999986932567848\\
599.94	0.00999991246644028\\
599.95	0.00999994717948697\\
599.96	0.00999997337774056\\
599.97	0.00999999097326228\\
599.98	0.00999999987724406\\
599.99	0.01\\
600	0.01\\
};
\addplot [color=mycolor8,solid,forget plot]
  table[row sep=crcr]{%
0.01	0.00486947107861205\\
1.01	0.00486947292236736\\
2.01	0.00486947480371144\\
3.01	0.00486947672340973\\
4.01	0.00486947868224241\\
5.01	0.00486948068100652\\
6.01	0.00486948272051449\\
7.01	0.00486948480159557\\
8.01	0.00486948692509585\\
9.01	0.00486948909187865\\
10.01	0.00486949130282463\\
11.01	0.00486949355883243\\
12.01	0.00486949586081895\\
13.01	0.0048694982097197\\
14.01	0.00486950060648905\\
15.01	0.0048695030521007\\
16.01	0.00486950554754846\\
17.01	0.00486950809384586\\
18.01	0.00486951069202711\\
19.01	0.00486951334314763\\
20.01	0.0048695160482839\\
21.01	0.00486951880853456\\
22.01	0.00486952162502026\\
23.01	0.00486952449888436\\
24.01	0.00486952743129361\\
25.01	0.00486953042343835\\
26.01	0.00486953347653281\\
27.01	0.0048695365918162\\
28.01	0.00486953977055238\\
29.01	0.00486954301403122\\
30.01	0.0048695463235686\\
31.01	0.00486954970050693\\
32.01	0.00486955314621586\\
33.01	0.00486955666209285\\
34.01	0.00486956024956363\\
35.01	0.00486956391008279\\
36.01	0.00486956764513434\\
37.01	0.00486957145623229\\
38.01	0.00486957534492136\\
39.01	0.00486957931277746\\
40.01	0.00486958336140846\\
41.01	0.00486958749245478\\
42.01	0.00486959170758994\\
43.01	0.00486959600852142\\
44.01	0.00486960039699094\\
45.01	0.00486960487477578\\
46.01	0.00486960944368901\\
47.01	0.00486961410558036\\
48.01	0.00486961886233696\\
49.01	0.00486962371588401\\
50.01	0.00486962866818569\\
51.01	0.00486963372124574\\
52.01	0.00486963887710859\\
53.01	0.00486964413785981\\
54.01	0.00486964950562714\\
55.01	0.0048696549825813\\
56.01	0.00486966057093674\\
57.01	0.00486966627295266\\
58.01	0.00486967209093384\\
59.01	0.00486967802723172\\
60.01	0.004869684084245\\
61.01	0.0048696902644208\\
62.01	0.00486969657025571\\
63.01	0.00486970300429641\\
64.01	0.00486970956914139\\
65.01	0.00486971626744086\\
66.01	0.00486972310189916\\
67.01	0.00486973007527472\\
68.01	0.00486973719038161\\
69.01	0.00486974445009067\\
70.01	0.00486975185733038\\
71.01	0.00486975941508849\\
72.01	0.00486976712641271\\
73.01	0.00486977499441213\\
74.01	0.00486978302225829\\
75.01	0.00486979121318698\\
76.01	0.00486979957049866\\
77.01	0.00486980809756034\\
78.01	0.00486981679780672\\
79.01	0.00486982567474153\\
80.01	0.00486983473193921\\
81.01	0.00486984397304571\\
82.01	0.00486985340178046\\
83.01	0.0048698630219374\\
84.01	0.00486987283738692\\
85.01	0.00486988285207705\\
86.01	0.00486989307003499\\
87.01	0.00486990349536882\\
88.01	0.0048699141322692\\
89.01	0.00486992498501079\\
90.01	0.00486993605795404\\
91.01	0.00486994735554673\\
92.01	0.00486995888232605\\
93.01	0.00486997064291982\\
94.01	0.0048699826420489\\
95.01	0.00486999488452863\\
96.01	0.00487000737527089\\
97.01	0.00487002011928564\\
98.01	0.00487003312168349\\
99.01	0.00487004638767716\\
100.01	0.00487005992258352\\
101.01	0.00487007373182632\\
102.01	0.00487008782093738\\
103.01	0.00487010219555918\\
104.01	0.00487011686144692\\
105.01	0.00487013182447101\\
106.01	0.00487014709061895\\
107.01	0.00487016266599795\\
108.01	0.00487017855683707\\
109.01	0.00487019476948977\\
110.01	0.00487021131043615\\
111.01	0.0048702281862858\\
112.01	0.00487024540377978\\
113.01	0.00487026296979401\\
114.01	0.0048702808913408\\
115.01	0.00487029917557284\\
116.01	0.00487031782978466\\
117.01	0.00487033686141632\\
118.01	0.00487035627805595\\
119.01	0.00487037608744272\\
120.01	0.00487039629746952\\
121.01	0.00487041691618612\\
122.01	0.00487043795180245\\
123.01	0.0048704594126915\\
124.01	0.00487048130739251\\
125.01	0.0048705036446141\\
126.01	0.00487052643323813\\
127.01	0.00487054968232194\\
128.01	0.0048705734011032\\
129.01	0.00487059759900224\\
130.01	0.0048706222856261\\
131.01	0.00487064747077201\\
132.01	0.00487067316443125\\
133.01	0.00487069937679273\\
134.01	0.00487072611824707\\
135.01	0.00487075339939004\\
136.01	0.00487078123102694\\
137.01	0.00487080962417641\\
138.01	0.0048708385900747\\
139.01	0.00487086814017983\\
140.01	0.00487089828617572\\
141.01	0.00487092903997673\\
142.01	0.00487096041373185\\
143.01	0.00487099241982938\\
144.01	0.00487102507090169\\
145.01	0.00487105837982942\\
146.01	0.00487109235974664\\
147.01	0.00487112702404564\\
148.01	0.00487116238638149\\
149.01	0.00487119846067759\\
150.01	0.0048712352611306\\
151.01	0.00487127280221532\\
152.01	0.00487131109869036\\
153.01	0.00487135016560362\\
154.01	0.00487139001829714\\
155.01	0.00487143067241363\\
156.01	0.00487147214390136\\
157.01	0.00487151444902012\\
158.01	0.0048715576043474\\
159.01	0.00487160162678421\\
160.01	0.00487164653356111\\
161.01	0.0048716923422442\\
162.01	0.00487173907074211\\
163.01	0.00487178673731159\\
164.01	0.00487183536056479\\
165.01	0.00487188495947542\\
166.01	0.0048719355533856\\
167.01	0.00487198716201284\\
168.01	0.00487203980545693\\
169.01	0.00487209350420692\\
170.01	0.004872148279149\\
171.01	0.00487220415157283\\
172.01	0.00487226114318008\\
173.01	0.00487231927609102\\
174.01	0.00487237857285301\\
175.01	0.00487243905644807\\
176.01	0.00487250075030086\\
177.01	0.0048725636782868\\
178.01	0.00487262786474045\\
179.01	0.00487269333446377\\
180.01	0.00487276011273462\\
181.01	0.00487282822531551\\
182.01	0.00487289769846232\\
183.01	0.00487296855893347\\
184.01	0.00487304083399875\\
185.01	0.00487311455144837\\
186.01	0.00487318973960289\\
187.01	0.00487326642732225\\
188.01	0.00487334464401561\\
189.01	0.00487342441965137\\
190.01	0.00487350578476671\\
191.01	0.00487358877047799\\
192.01	0.00487367340849093\\
193.01	0.00487375973111112\\
194.01	0.00487384777125442\\
195.01	0.00487393756245795\\
196.01	0.0048740291388905\\
197.01	0.00487412253536421\\
198.01	0.00487421778734529\\
199.01	0.00487431493096541\\
200.01	0.00487441400303351\\
201.01	0.00487451504104718\\
202.01	0.00487461808320468\\
203.01	0.00487472316841675\\
204.01	0.00487483033631851\\
205.01	0.00487493962728282\\
206.01	0.00487505108243155\\
207.01	0.00487516474364866\\
208.01	0.00487528065359315\\
209.01	0.00487539885571175\\
210.01	0.00487551939425231\\
211.01	0.00487564231427645\\
212.01	0.00487576766167358\\
213.01	0.00487589548317438\\
214.01	0.00487602582636409\\
215.01	0.00487615873969677\\
216.01	0.004876294272509\\
217.01	0.00487643247503433\\
218.01	0.00487657339841745\\
219.01	0.00487671709472827\\
220.01	0.00487686361697712\\
221.01	0.00487701301912899\\
222.01	0.00487716535611812\\
223.01	0.00487732068386365\\
224.01	0.00487747905928396\\
225.01	0.00487764054031238\\
226.01	0.00487780518591223\\
227.01	0.00487797305609189\\
228.01	0.00487814421192095\\
229.01	0.004878318715545\\
230.01	0.00487849663020176\\
231.01	0.00487867802023668\\
232.01	0.00487886295111854\\
233.01	0.0048790514894554\\
234.01	0.00487924370301043\\
235.01	0.00487943966071792\\
236.01	0.00487963943269878\\
237.01	0.00487984309027684\\
238.01	0.00488005070599444\\
239.01	0.00488026235362868\\
240.01	0.00488047810820687\\
241.01	0.00488069804602276\\
242.01	0.00488092224465167\\
243.01	0.00488115078296653\\
244.01	0.00488138374115347\\
245.01	0.00488162120072699\\
246.01	0.00488186324454526\\
247.01	0.00488210995682524\\
248.01	0.00488236142315752\\
249.01	0.00488261773052098\\
250.01	0.00488287896729749\\
251.01	0.00488314522328534\\
252.01	0.00488341658971392\\
253.01	0.00488369315925632\\
254.01	0.0048839750260436\\
255.01	0.00488426228567626\\
256.01	0.00488455503523741\\
257.01	0.0048848533733041\\
258.01	0.00488515739995884\\
259.01	0.00488546721679987\\
260.01	0.00488578292695181\\
261.01	0.0048861046350747\\
262.01	0.00488643244737243\\
263.01	0.00488676647160112\\
264.01	0.00488710681707634\\
265.01	0.00488745359467868\\
266.01	0.00488780691685956\\
267.01	0.00488816689764514\\
268.01	0.00488853365263966\\
269.01	0.00488890729902719\\
270.01	0.00488928795557286\\
271.01	0.00488967574262189\\
272.01	0.00489007078209797\\
273.01	0.0048904731974995\\
274.01	0.00489088311389492\\
275.01	0.00489130065791585\\
276.01	0.00489172595774905\\
277.01	0.00489215914312559\\
278.01	0.004892600345309\\
279.01	0.00489304969708074\\
280.01	0.0048935073327239\\
281.01	0.00489397338800394\\
282.01	0.00489444800014771\\
283.01	0.00489493130781972\\
284.01	0.0048954234510946\\
285.01	0.00489592457142875\\
286.01	0.00489643481162716\\
287.01	0.00489695431580782\\
288.01	0.00489748322936308\\
289.01	0.00489802169891686\\
290.01	0.00489856987227907\\
291.01	0.004899127898395\\
292.01	0.00489969592729207\\
293.01	0.00490027411002165\\
294.01	0.00490086259859672\\
295.01	0.00490146154592501\\
296.01	0.0049020711057375\\
297.01	0.00490269143251219\\
298.01	0.00490332268139221\\
299.01	0.00490396500809982\\
300.01	0.00490461856884361\\
301.01	0.00490528352022102\\
302.01	0.00490596001911514\\
303.01	0.00490664822258498\\
304.01	0.00490734828775043\\
305.01	0.00490806037167041\\
306.01	0.00490878463121465\\
307.01	0.00490952122292969\\
308.01	0.0049102703028981\\
309.01	0.00491103202659038\\
310.01	0.00491180654871179\\
311.01	0.00491259402304131\\
312.01	0.00491339460226485\\
313.01	0.00491420843780183\\
314.01	0.0049150356796258\\
315.01	0.00491587647607902\\
316.01	0.00491673097368153\\
317.01	0.00491759931693518\\
318.01	0.00491848164812264\\
319.01	0.00491937810710343\\
320.01	0.00492028883110579\\
321.01	0.00492121395451707\\
322.01	0.00492215360867281\\
323.01	0.00492310792164599\\
324.01	0.0049240770180385\\
325.01	0.00492506101877592\\
326.01	0.00492606004090838\\
327.01	0.00492707419741913\\
328.01	0.0049281035970444\\
329.01	0.00492914834410706\\
330.01	0.00493020853836761\\
331.01	0.00493128427489808\\
332.01	0.00493237564397986\\
333.01	0.00493348273103555\\
334.01	0.00493460561659563\\
335.01	0.00493574437631042\\
336.01	0.00493689908101126\\
337.01	0.00493806979683226\\
338.01	0.00493925658539787\\
339.01	0.00494045950408954\\
340.01	0.00494167860639888\\
341.01	0.00494291394238186\\
342.01	0.00494416555922468\\
343.01	0.00494543350193688\\
344.01	0.00494671781418379\\
345.01	0.00494801853927719\\
346.01	0.00494933572133761\\
347.01	0.00495066940664869\\
348.01	0.0049520196452166\\
349.01	0.0049533864925576\\
350.01	0.00495477001172624\\
351.01	0.00495617027560363\\
352.01	0.00495758736945942\\
353.01	0.0049590213937989\\
354.01	0.00496047246750404\\
355.01	0.00496194073127067\\
356.01	0.00496342635133638\\
357.01	0.00496492952348586\\
358.01	0.00496645047730471\\
359.01	0.00496798948064016\\
360.01	0.00496954684420485\\
361.01	0.00497112292623784\\
362.01	0.00497271813710551\\
363.01	0.00497433294369125\\
364.01	0.00497596787338372\\
365.01	0.00497762351742256\\
366.01	0.00497930053331316\\
367.01	0.00498099964596406\\
368.01	0.00498272164714229\\
369.01	0.00498446739279117\\
370.01	0.00498623779770917\\
371.01	0.0049880338270704\\
372.01	0.00498985648428585\\
373.01	0.00499170679479151\\
374.01	0.00499358578553845\\
375.01	0.0049954944603099\\
376.01	0.00499743377157359\\
377.01	0.00499940459050881\\
378.01	0.00500140767827624\\
379.01	0.00500344366373888\\
380.01	0.00500551304083137\\
381.01	0.00500761623451497\\
382.01	0.00500975366440684\\
383.01	0.00501192575025035\\
384.01	0.00501413291170537\\
385.01	0.00501637556813499\\
386.01	0.00501865413838863\\
387.01	0.00502096904058337\\
388.01	0.00502332069188344\\
389.01	0.0050257095082792\\
390.01	0.00502813590436763\\
391.01	0.00503060029313336\\
392.01	0.00503310308573499\\
393.01	0.00503564469129482\\
394.01	0.00503822551669646\\
395.01	0.00504084596639151\\
396.01	0.00504350644221681\\
397.01	0.00504620734322635\\
398.01	0.00504894906553885\\
399.01	0.0050517320022056\\
400.01	0.00505455654310108\\
401.01	0.00505742307484021\\
402.01	0.00506033198072661\\
403.01	0.00506328364073562\\
404.01	0.00506627843153779\\
405.01	0.00506931672656746\\
406.01	0.00507239889614258\\
407.01	0.00507552530764104\\
408.01	0.00507869632574179\\
409.01	0.00508191231273589\\
410.01	0.00508517362891761\\
411.01	0.00508848063306101\\
412.01	0.00509183368299319\\
413.01	0.00509523313627251\\
414.01	0.00509867935098175\\
415.01	0.0051021726866462\\
416.01	0.00510571350528854\\
417.01	0.00510930217263104\\
418.01	0.00511293905945774\\
419.01	0.00511662454314692\\
420.01	0.00512035900938866\\
421.01	0.00512414285409804\\
422.01	0.00512797648553649\\
423.01	0.00513186032665466\\
424.01	0.00513579481766586\\
425.01	0.00513978041886369\\
426.01	0.00514381761369105\\
427.01	0.00514790691206936\\
428.01	0.00515204885399412\\
429.01	0.0051562440133987\\
430.01	0.00516049300228848\\
431.01	0.0051647964751378\\
432.01	0.0051691551335438\\
433.01	0.0051735697311199\\
434.01	0.00517804107860816\\
435.01	0.00518257004917931\\
436.01	0.00518715758388222\\
437.01	0.00519180469719159\\
438.01	0.00519651248259237\\
439.01	0.005201282118126\\
440.01	0.00520611487180661\\
441.01	0.00521101210680318\\
442.01	0.00521597528626405\\
443.01	0.00522100597764519\\
444.01	0.00522610585638446\\
445.01	0.0052312767087502\\
446.01	0.00523652043367955\\
447.01	0.00524183904340754\\
448.01	0.0052472346626893\\
449.01	0.00525270952641858\\
450.01	0.00525826597545891\\
451.01	0.00526390645053876\\
452.01	0.00526963348410613\\
453.01	0.00527544969011183\\
454.01	0.00528135775179404\\
455.01	0.00528736040766942\\
456.01	0.00529346043611076\\
457.01	0.00529966063910327\\
458.01	0.00530596382602569\\
459.01	0.00531237279858311\\
460.01	0.00531889033831225\\
461.01	0.00532551919835066\\
462.01	0.00533226210133951\\
463.01	0.00533912174532289\\
464.01	0.00534610081914363\\
465.01	0.00535320202787512\\
466.01	0.00536042812681975\\
467.01	0.00536778195505186\\
468.01	0.00537526645014033\\
469.01	0.0053828846510761\\
470.01	0.00539063970006425\\
471.01	0.00539853484395138\\
472.01	0.00540657343524228\\
473.01	0.00541475893266269\\
474.01	0.0054230949012279\\
475.01	0.00543158501178372\\
476.01	0.00544023303999507\\
477.01	0.00544904286476906\\
478.01	0.00545801846611518\\
479.01	0.00546716392246428\\
480.01	0.00547648340748907\\
481.01	0.00548598118649742\\
482.01	0.00549566161249708\\
483.01	0.00550552912206398\\
484.01	0.00551558823117855\\
485.01	0.00552584353122865\\
486.01	0.00553629968540818\\
487.01	0.00554696142576086\\
488.01	0.00555783355113311\\
489.01	0.00556892092629246\\
490.01	0.0055802284824364\\
491.01	0.00559176121925617\\
492.01	0.00560352420861528\\
493.01	0.0056155225997567\\
494.01	0.00562776162575973\\
495.01	0.00564024661073782\\
496.01	0.00565298297702609\\
497.01	0.00566597625140927\\
498.01	0.00567923206939053\\
499.01	0.00569275617702374\\
500.01	0.00570655443140132\\
501.01	0.00572063280103759\\
502.01	0.00573499736640269\\
503.01	0.00574965432066073\\
504.01	0.00576460997066185\\
505.01	0.00577987073823816\\
506.01	0.00579544316184305\\
507.01	0.00581133389856794\\
508.01	0.00582754972655403\\
509.01	0.00584409754780098\\
510.01	0.00586098439135369\\
511.01	0.00587821741682283\\
512.01	0.00589580391817397\\
513.01	0.00591375132769374\\
514.01	0.0059320672200273\\
515.01	0.00595075931617198\\
516.01	0.00596983548732184\\
517.01	0.00598930375848478\\
518.01	0.00600917231184658\\
519.01	0.00602944948993105\\
520.01	0.00605014379868112\\
521.01	0.00607126391059673\\
522.01	0.00609281866798034\\
523.01	0.00611481708628279\\
524.01	0.00613726835753446\\
525.01	0.00616018185383922\\
526.01	0.00618356713090683\\
527.01	0.00620743393159559\\
528.01	0.00623179218943586\\
529.01	0.00625665203210274\\
530.01	0.00628202378480992\\
531.01	0.0063079179735996\\
532.01	0.00633434532850766\\
533.01	0.00636131678658915\\
534.01	0.00638884349479479\\
535.01	0.00641693681269148\\
536.01	0.00644560831501638\\
537.01	0.00647486979404623\\
538.01	0.0065047332617518\\
539.01	0.00653521095169769\\
540.01	0.00656631532064435\\
541.01	0.00659805904980486\\
542.01	0.00663045504570371\\
543.01	0.00666351644058184\\
544.01	0.00669725659228586\\
545.01	0.00673168908357341\\
546.01	0.00676682772076062\\
547.01	0.00680268653162876\\
548.01	0.00683927976249817\\
549.01	0.00687662187436601\\
550.01	0.00691472753799263\\
551.01	0.0069536116278066\\
552.01	0.00699328921448385\\
553.01	0.00703377555604178\\
554.01	0.00707508608726937\\
555.01	0.00711723640729895\\
556.01	0.00716024226510179\\
557.01	0.00720411954266781\\
558.01	0.00724888423560591\\
559.01	0.00729455243087272\\
560.01	0.00734114028130855\\
561.01	0.00738866397662728\\
562.01	0.00743713971047418\\
563.01	0.00748658364312575\\
564.01	0.00753701185936998\\
565.01	0.00758844032106287\\
566.01	0.00764088481381529\\
567.01	0.00769436088721993\\
568.01	0.00774888378798493\\
569.01	0.00780446838529657\\
570.01	0.00786112908769292\\
571.01	0.00791887975069271\\
572.01	0.00797773357439382\\
573.01	0.00803770299023592\\
574.01	0.00809879953611645\\
575.01	0.00816103371906509\\
576.01	0.00822441486472502\\
577.01	0.00828895095297273\\
578.01	0.00835464843913835\\
579.01	0.00842151206048843\\
580.01	0.00848954462791702\\
581.01	0.00855874680318726\\
582.01	0.0086291168626057\\
583.01	0.00870065044873541\\
584.01	0.00877334031271359\\
585.01	0.00884717605099796\\
586.01	0.00892214384200484\\
587.01	0.0089982261902219\\
588.01	0.00907540168810686\\
589.01	0.00915364480957812\\
590.01	0.00923292575336488\\
591.01	0.0093132103601611\\
592.01	0.00939446013472989\\
593.01	0.00947663241322402\\
594.01	0.00955968072750807\\
595.01	0.00964355543279665\\
596.01	0.00972820468321698\\
597.01	0.00981357586291021\\
598.01	0.00989961760918075\\
599.01	0.00996919203046377\\
599.02	0.00996973314335038\\
599.03	0.0099702709571694\\
599.04	0.00997080543920336\\
599.05	0.0099713365564138\\
599.06	0.00997186427543808\\
599.07	0.0099723885625862\\
599.08	0.00997290938383756\\
599.09	0.00997342670483771\\
599.1	0.00997394049089505\\
599.11	0.00997445070697751\\
599.12	0.00997495731770918\\
599.13	0.00997546028736696\\
599.14	0.00997595957987707\\
599.15	0.00997645515881166\\
599.16	0.00997694698738526\\
599.17	0.00997743502845131\\
599.18	0.00997791924449856\\
599.19	0.00997839959764748\\
599.2	0.00997887604964662\\
599.21	0.00997934856186899\\
599.22	0.00997981709530829\\
599.23	0.00998028161057521\\
599.24	0.00998074206789365\\
599.25	0.0099811984270969\\
599.26	0.00998165064762378\\
599.27	0.00998209868851477\\
599.28	0.00998254250840806\\
599.29	0.00998298206553559\\
599.3	0.00998341731771905\\
599.31	0.00998384822236584\\
599.32	0.00998427473646497\\
599.33	0.00998469681658292\\
599.34	0.00998511441885952\\
599.35	0.0099855274990037\\
599.36	0.00998593601228926\\
599.37	0.00998633991355056\\
599.38	0.00998673915717821\\
599.39	0.00998713369711469\\
599.4	0.00998752348684992\\
599.41	0.00998790847811157\\
599.42	0.00998828861974491\\
599.43	0.00998866386008804\\
599.44	0.00998903414696681\\
599.45	0.00998939942768985\\
599.46	0.00998975964904344\\
599.47	0.00999011475728636\\
599.48	0.00999046469814472\\
599.49	0.00999080941680669\\
599.5	0.00999114885791725\\
599.51	0.00999148296557278\\
599.52	0.0099918116833157\\
599.53	0.00999213495412901\\
599.54	0.00999245272043077\\
599.55	0.00999276492406855\\
599.56	0.00999307150631381\\
599.57	0.00999337240785624\\
599.58	0.00999366756879799\\
599.59	0.00999395692864795\\
599.6	0.00999424042631585\\
599.61	0.00999451800010642\\
599.62	0.00999478958771336\\
599.63	0.0099950551262134\\
599.64	0.00999531455206019\\
599.65	0.00999556780107816\\
599.66	0.00999581480845634\\
599.67	0.00999605550874211\\
599.68	0.00999628983583487\\
599.69	0.00999651772297967\\
599.7	0.00999673910276076\\
599.71	0.00999695390709509\\
599.72	0.00999716206722577\\
599.73	0.00999736351371538\\
599.74	0.00999755817643933\\
599.75	0.00999774598457904\\
599.76	0.00999792686661517\\
599.77	0.00999810075032068\\
599.78	0.00999826756275386\\
599.79	0.00999842723025133\\
599.8	0.00999857967842092\\
599.81	0.0099987248321345\\
599.82	0.00999886261552071\\
599.83	0.00999899295195769\\
599.84	0.00999911576406567\\
599.85	0.00999923097369951\\
599.86	0.00999933850194117\\
599.87	0.00999943826909211\\
599.88	0.00999953019466558\\
599.89	0.00999961419737891\\
599.9	0.00999969019514566\\
599.91	0.00999975810506767\\
599.92	0.00999981784342713\\
599.93	0.00999986932567848\\
599.94	0.00999991246644028\\
599.95	0.00999994717948697\\
599.96	0.00999997337774056\\
599.97	0.00999999097326228\\
599.98	0.00999999987724406\\
599.99	0.01\\
600	0.01\\
};
\addplot [color=blue!25!mycolor7,solid,forget plot]
  table[row sep=crcr]{%
0.01	0.0046877913708233\\
1.01	0.00468779351620759\\
2.01	0.00468779570568027\\
3.01	0.00468779794014776\\
4.01	0.00468780022053526\\
5.01	0.00468780254778671\\
6.01	0.00468780492286568\\
7.01	0.00468780734675557\\
8.01	0.00468780982045992\\
9.01	0.00468781234500285\\
10.01	0.00468781492142956\\
11.01	0.00468781755080697\\
12.01	0.00468782023422366\\
13.01	0.00468782297279062\\
14.01	0.00468782576764179\\
15.01	0.0046878286199345\\
16.01	0.0046878315308496\\
17.01	0.00468783450159244\\
18.01	0.00468783753339311\\
19.01	0.00468784062750699\\
20.01	0.00468784378521535\\
21.01	0.00468784700782547\\
22.01	0.00468785029667193\\
23.01	0.00468785365311665\\
24.01	0.00468785707854939\\
25.01	0.00468786057438855\\
26.01	0.00468786414208172\\
27.01	0.00468786778310613\\
28.01	0.0046878714989697\\
29.01	0.00468787529121094\\
30.01	0.0046878791614004\\
31.01	0.0046878831111405\\
32.01	0.00468788714206692\\
33.01	0.0046878912558488\\
34.01	0.00468789545418967\\
35.01	0.00468789973882809\\
36.01	0.0046879041115383\\
37.01	0.00468790857413097\\
38.01	0.004687913128454\\
39.01	0.00468791777639346\\
40.01	0.00468792251987409\\
41.01	0.00468792736085995\\
42.01	0.00468793230135572\\
43.01	0.00468793734340721\\
44.01	0.00468794248910237\\
45.01	0.00468794774057203\\
46.01	0.00468795309999054\\
47.01	0.00468795856957747\\
48.01	0.00468796415159733\\
49.01	0.0046879698483618\\
50.01	0.00468797566222959\\
51.01	0.00468798159560826\\
52.01	0.00468798765095454\\
53.01	0.00468799383077587\\
54.01	0.00468800013763086\\
55.01	0.00468800657413113\\
56.01	0.00468801314294176\\
57.01	0.00468801984678247\\
58.01	0.00468802668842902\\
59.01	0.00468803367071403\\
60.01	0.00468804079652857\\
61.01	0.00468804806882284\\
62.01	0.0046880554906078\\
63.01	0.00468806306495617\\
64.01	0.00468807079500363\\
65.01	0.00468807868395074\\
66.01	0.00468808673506337\\
67.01	0.00468809495167452\\
68.01	0.00468810333718582\\
69.01	0.00468811189506857\\
70.01	0.00468812062886539\\
71.01	0.00468812954219165\\
72.01	0.00468813863873707\\
73.01	0.00468814792226696\\
74.01	0.00468815739662418\\
75.01	0.00468816706573004\\
76.01	0.00468817693358677\\
77.01	0.00468818700427871\\
78.01	0.0046881972819738\\
79.01	0.00468820777092574\\
80.01	0.00468821847547531\\
81.01	0.00468822940005259\\
82.01	0.00468824054917864\\
83.01	0.00468825192746695\\
84.01	0.00468826353962619\\
85.01	0.00468827539046124\\
86.01	0.00468828748487558\\
87.01	0.00468829982787353\\
88.01	0.00468831242456196\\
89.01	0.00468832528015235\\
90.01	0.00468833839996327\\
91.01	0.00468835178942247\\
92.01	0.00468836545406862\\
93.01	0.00468837939955452\\
94.01	0.00468839363164836\\
95.01	0.00468840815623696\\
96.01	0.00468842297932769\\
97.01	0.00468843810705127\\
98.01	0.0046884535456638\\
99.01	0.0046884693015498\\
100.01	0.00468848538122497\\
101.01	0.0046885017913378\\
102.01	0.00468851853867376\\
103.01	0.00468853563015703\\
104.01	0.00468855307285371\\
105.01	0.00468857087397483\\
106.01	0.00468858904087895\\
107.01	0.00468860758107535\\
108.01	0.00468862650222744\\
109.01	0.00468864581215518\\
110.01	0.00468866551883915\\
111.01	0.00468868563042284\\
112.01	0.00468870615521692\\
113.01	0.00468872710170185\\
114.01	0.00468874847853218\\
115.01	0.00468877029453905\\
116.01	0.00468879255873471\\
117.01	0.0046888152803158\\
118.01	0.00468883846866718\\
119.01	0.00468886213336556\\
120.01	0.00468888628418372\\
121.01	0.0046889109310944\\
122.01	0.00468893608427406\\
123.01	0.00468896175410764\\
124.01	0.00468898795119226\\
125.01	0.00468901468634156\\
126.01	0.00468904197059049\\
127.01	0.00468906981519962\\
128.01	0.00468909823165932\\
129.01	0.00468912723169509\\
130.01	0.00468915682727183\\
131.01	0.00468918703059898\\
132.01	0.00468921785413555\\
133.01	0.00468924931059461\\
134.01	0.00468928141294914\\
135.01	0.00468931417443692\\
136.01	0.00468934760856594\\
137.01	0.00468938172912008\\
138.01	0.00468941655016439\\
139.01	0.00468945208605075\\
140.01	0.00468948835142417\\
141.01	0.00468952536122824\\
142.01	0.00468956313071139\\
143.01	0.00468960167543311\\
144.01	0.00468964101126965\\
145.01	0.00468968115442142\\
146.01	0.00468972212141856\\
147.01	0.00468976392912843\\
148.01	0.00468980659476177\\
149.01	0.00468985013587997\\
150.01	0.00468989457040206\\
151.01	0.004689939916612\\
152.01	0.00468998619316581\\
153.01	0.0046900334190991\\
154.01	0.00469008161383513\\
155.01	0.00469013079719205\\
156.01	0.00469018098939114\\
157.01	0.00469023221106503\\
158.01	0.00469028448326561\\
159.01	0.00469033782747277\\
160.01	0.00469039226560272\\
161.01	0.0046904478200171\\
162.01	0.00469050451353166\\
163.01	0.00469056236942537\\
164.01	0.00469062141144987\\
165.01	0.00469068166383869\\
166.01	0.00469074315131712\\
167.01	0.00469080589911186\\
168.01	0.00469086993296136\\
169.01	0.00469093527912584\\
170.01	0.00469100196439738\\
171.01	0.00469107001611126\\
172.01	0.00469113946215627\\
173.01	0.00469121033098589\\
174.01	0.00469128265162952\\
175.01	0.00469135645370398\\
176.01	0.00469143176742538\\
177.01	0.00469150862362071\\
178.01	0.00469158705374005\\
179.01	0.00469166708986911\\
180.01	0.0046917487647417\\
181.01	0.00469183211175282\\
182.01	0.00469191716497146\\
183.01	0.00469200395915406\\
184.01	0.0046920925297582\\
185.01	0.00469218291295631\\
186.01	0.00469227514564995\\
187.01	0.00469236926548416\\
188.01	0.00469246531086198\\
189.01	0.00469256332095967\\
190.01	0.0046926633357415\\
191.01	0.00469276539597568\\
192.01	0.00469286954325\\
193.01	0.00469297581998768\\
194.01	0.00469308426946434\\
195.01	0.00469319493582397\\
196.01	0.00469330786409664\\
197.01	0.00469342310021517\\
198.01	0.00469354069103299\\
199.01	0.00469366068434245\\
200.01	0.00469378312889261\\
201.01	0.00469390807440817\\
202.01	0.00469403557160822\\
203.01	0.00469416567222568\\
204.01	0.00469429842902691\\
205.01	0.00469443389583178\\
206.01	0.00469457212753394\\
207.01	0.00469471318012162\\
208.01	0.00469485711069827\\
209.01	0.00469500397750496\\
210.01	0.00469515383994136\\
211.01	0.00469530675858849\\
212.01	0.00469546279523128\\
213.01	0.00469562201288101\\
214.01	0.00469578447579952\\
215.01	0.00469595024952263\\
216.01	0.00469611940088441\\
217.01	0.00469629199804179\\
218.01	0.00469646811049951\\
219.01	0.00469664780913606\\
220.01	0.00469683116622897\\
221.01	0.00469701825548141\\
222.01	0.0046972091520493\\
223.01	0.00469740393256773\\
224.01	0.00469760267517942\\
225.01	0.00469780545956214\\
226.01	0.00469801236695772\\
227.01	0.00469822348020087\\
228.01	0.00469843888374873\\
229.01	0.00469865866371036\\
230.01	0.00469888290787765\\
231.01	0.00469911170575583\\
232.01	0.00469934514859474\\
233.01	0.00469958332942075\\
234.01	0.00469982634306883\\
235.01	0.00470007428621501\\
236.01	0.00470032725741021\\
237.01	0.004700585357113\\
238.01	0.00470084868772419\\
239.01	0.00470111735362118\\
240.01	0.00470139146119281\\
241.01	0.00470167111887497\\
242.01	0.00470195643718639\\
243.01	0.00470224752876482\\
244.01	0.00470254450840389\\
245.01	0.00470284749309043\\
246.01	0.00470315660204168\\
247.01	0.00470347195674359\\
248.01	0.00470379368098893\\
249.01	0.00470412190091624\\
250.01	0.00470445674504887\\
251.01	0.00470479834433482\\
252.01	0.00470514683218584\\
253.01	0.00470550234451828\\
254.01	0.00470586501979306\\
255.01	0.00470623499905659\\
256.01	0.00470661242598161\\
257.01	0.00470699744690858\\
258.01	0.00470739021088667\\
259.01	0.00470779086971571\\
260.01	0.00470819957798718\\
261.01	0.00470861649312641\\
262.01	0.00470904177543439\\
263.01	0.00470947558812912\\
264.01	0.00470991809738729\\
265.01	0.00471036947238596\\
266.01	0.0047108298853437\\
267.01	0.00471129951156185\\
268.01	0.0047117785294653\\
269.01	0.00471226712064268\\
270.01	0.00471276546988608\\
271.01	0.00471327376523074\\
272.01	0.00471379219799338\\
273.01	0.00471432096281013\\
274.01	0.00471486025767372\\
275.01	0.00471541028396903\\
276.01	0.00471597124650817\\
277.01	0.00471654335356393\\
278.01	0.00471712681690171\\
279.01	0.00471772185181048\\
280.01	0.004718328677131\\
281.01	0.00471894751528298\\
282.01	0.00471957859228921\\
283.01	0.00472022213779801\\
284.01	0.00472087838510299\\
285.01	0.00472154757115958\\
286.01	0.00472222993659834\\
287.01	0.00472292572573532\\
288.01	0.00472363518657853\\
289.01	0.00472435857082978\\
290.01	0.00472509613388258\\
291.01	0.00472584813481511\\
292.01	0.00472661483637805\\
293.01	0.00472739650497579\\
294.01	0.00472819341064325\\
295.01	0.00472900582701362\\
296.01	0.00472983403128104\\
297.01	0.00473067830415359\\
298.01	0.00473153892979923\\
299.01	0.00473241619578142\\
300.01	0.00473331039298522\\
301.01	0.00473422181553319\\
302.01	0.00473515076068848\\
303.01	0.00473609752874667\\
304.01	0.00473706242291323\\
305.01	0.00473804574916711\\
306.01	0.00473904781610854\\
307.01	0.00474006893478995\\
308.01	0.00474110941852864\\
309.01	0.00474216958270086\\
310.01	0.00474324974451443\\
311.01	0.00474435022275863\\
312.01	0.0047454713375312\\
313.01	0.00474661340993866\\
314.01	0.00474777676176948\\
315.01	0.00474896171513786\\
316.01	0.00475016859209591\\
317.01	0.00475139771421119\\
318.01	0.00475264940211023\\
319.01	0.00475392397498133\\
320.01	0.00475522175003794\\
321.01	0.00475654304193771\\
322.01	0.0047578881621554\\
323.01	0.0047592574183065\\
324.01	0.00476065111341886\\
325.01	0.00476206954514856\\
326.01	0.0047635130049386\\
327.01	0.00476498177711507\\
328.01	0.00476647613791956\\
329.01	0.00476799635447413\\
330.01	0.00476954268367623\\
331.01	0.00477111537101982\\
332.01	0.00477271464934223\\
333.01	0.0047743407374924\\
334.01	0.00477599383892136\\
335.01	0.0047776741401921\\
336.01	0.00477938180941025\\
337.01	0.00478111699457616\\
338.01	0.00478287982186117\\
339.01	0.00478467039381171\\
340.01	0.00478648878748918\\
341.01	0.00478833505255249\\
342.01	0.00479020920929768\\
343.01	0.00479211124666848\\
344.01	0.00479404112026045\\
345.01	0.00479599875034337\\
346.01	0.00479798401993539\\
347.01	0.00479999677296886\\
348.01	0.00480203681259828\\
349.01	0.0048041038997108\\
350.01	0.00480619775171291\\
351.01	0.00480831804168072\\
352.01	0.00481046439798116\\
353.01	0.00481263640448791\\
354.01	0.00481483360154139\\
355.01	0.00481705548782523\\
356.01	0.00481930152336384\\
357.01	0.00482157113387317\\
358.01	0.00482386371673542\\
359.01	0.00482617864890168\\
360.01	0.00482851529706731\\
361.01	0.0048308730304962\\
362.01	0.00483325123690904\\
363.01	0.0048356493418704\\
364.01	0.00483806683212031\\
365.01	0.00484050328328288\\
366.01	0.00484295839233395\\
367.01	0.00484543201510167\\
368.01	0.00484792420889242\\
369.01	0.00485043528003107\\
370.01	0.00485296583564418\\
371.01	0.00485551683832332\\
372.01	0.00485808966130107\\
373.01	0.00486068614033032\\
374.01	0.00486330861641787\\
375.01	0.00486595996071022\\
376.01	0.00486864356886456\\
377.01	0.0048713633067735\\
378.01	0.00487412338201897\\
379.01	0.00487692810520714\\
380.01	0.0048797810365406\\
381.01	0.00488268339855548\\
382.01	0.00488563585314032\\
383.01	0.00488863906102592\\
384.01	0.00489169368111608\\
385.01	0.00489480036977825\\
386.01	0.00489795978009008\\
387.01	0.00490117256104255\\
388.01	0.00490443935669549\\
389.01	0.00490776080528592\\
390.01	0.00491113753828591\\
391.01	0.00491457017940904\\
392.01	0.00491805934356283\\
393.01	0.00492160563574746\\
394.01	0.00492520964989636\\
395.01	0.00492887196766008\\
396.01	0.00493259315713027\\
397.01	0.00493637377150284\\
398.01	0.00494021434768074\\
399.01	0.00494411540481332\\
400.01	0.00494807744277466\\
401.01	0.00495210094057746\\
402.01	0.00495618635472721\\
403.01	0.00496033411751279\\
404.01	0.00496454463523903\\
405.01	0.0049688182864005\\
406.01	0.00497315541980127\\
407.01	0.0049775563526234\\
408.01	0.00498202136844939\\
409.01	0.00498655071524534\\
410.01	0.00499114460331061\\
411.01	0.00499580320320435\\
412.01	0.00500052664365926\\
413.01	0.0050053150094949\\
414.01	0.00501016833954575\\
415.01	0.00501508662462151\\
416.01	0.00502006980552037\\
417.01	0.00502511777111758\\
418.01	0.00503023035655757\\
419.01	0.00503540734158045\\
420.01	0.00504064844901714\\
421.01	0.00504595334349552\\
422.01	0.00505132163040094\\
423.01	0.00505675285514366\\
424.01	0.00506224650279165\\
425.01	0.00506780199813149\\
426.01	0.00507341870623117\\
427.01	0.00507909593358397\\
428.01	0.00508483292992171\\
429.01	0.0050906288907933\\
430.01	0.00509648296101652\\
431.01	0.00510239423911465\\
432.01	0.00510836178286296\\
433.01	0.0051143846160757\\
434.01	0.00512046173677136\\
435.01	0.00512659212686058\\
436.01	0.00513277476350303\\
437.01	0.00513900863227999\\
438.01	0.00514529274232582\\
439.01	0.00515162614354973\\
440.01	0.00515800794606375\\
441.01	0.00516443734190444\\
442.01	0.00517091362909716\\
443.01	0.00517743623805843\\
444.01	0.00518400476026008\\
445.01	0.00519061897898486\\
446.01	0.00519727890188222\\
447.01	0.00520398479488479\\
448.01	0.00521073721685766\\
449.01	0.00521753705412668\\
450.01	0.00522438555376617\\
451.01	0.00523128435420685\\
452.01	0.00523823551137073\\
453.01	0.00524524151814018\\
454.01	0.00525230531454857\\
455.01	0.00525943028566056\\
456.01	0.00526662024373692\\
457.01	0.00527387939102032\\
458.01	0.00528121225944443\\
459.01	0.00528862362391745\\
460.01	0.00529611838680955\\
461.01	0.00530370143321333\\
462.01	0.00531137745996266\\
463.01	0.0053191507869801\\
464.01	0.00532702516825855\\
465.01	0.0053350036328947\\
466.01	0.00534308841881588\\
467.01	0.00535128129746901\\
468.01	0.00535958409596056\\
469.01	0.00536799879274197\\
470.01	0.00537652753053824\\
471.01	0.00538517262929297\\
472.01	0.00539393659903768\\
473.01	0.00540282215246299\\
474.01	0.00541183221693191\\
475.01	0.00542096994562932\\
476.01	0.00543023872749784\\
477.01	0.00543964219556525\\
478.01	0.00544918423322939\\
479.01	0.00545886897804068\\
480.01	0.00546870082249888\\
481.01	0.00547868441138057\\
482.01	0.00548882463513328\\
483.01	0.00549912661892601\\
484.01	0.00550959570704213\\
485.01	0.00552023744245141\\
486.01	0.00553105754161531\\
487.01	0.00554206186487474\\
488.01	0.00555325638315463\\
489.01	0.00556464714220292\\
490.01	0.00557624022615136\\
491.01	0.00558804172283474\\
492.01	0.00560005769397902\\
493.01	0.00561229415398543\\
494.01	0.00562475706143512\\
495.01	0.00563745232737498\\
496.01	0.00565038584350466\\
497.01	0.00566356353096031\\
498.01	0.00567699140473343\\
499.01	0.00569067562622107\\
500.01	0.00570462251382754\\
501.01	0.0057188385381076\\
502.01	0.00573333031557464\\
503.01	0.00574810460175816\\
504.01	0.0057631682837062\\
505.01	0.00577852837218618\\
506.01	0.00579419199390473\\
507.01	0.00581016638412089\\
508.01	0.00582645888008326\\
509.01	0.00584307691575363\\
510.01	0.00586002801828694\\
511.01	0.00587731980670701\\
512.01	0.00589495999313232\\
513.01	0.00591295638675328\\
514.01	0.00593131690053231\\
515.01	0.00595004956028283\\
516.01	0.00596916251539123\\
517.01	0.0059886640500192\\
518.01	0.00600856259324166\\
519.01	0.00602886672640598\\
520.01	0.00604958518667911\\
521.01	0.00607072686832795\\
522.01	0.00609230082395634\\
523.01	0.00611431626621421\\
524.01	0.00613678257003705\\
525.01	0.00615970927544816\\
526.01	0.00618310609092311\\
527.01	0.00620698289728304\\
528.01	0.00623134975204244\\
529.01	0.00625621689409586\\
530.01	0.00628159474858923\\
531.01	0.00630749393178962\\
532.01	0.00633392525575795\\
533.01	0.00636089973264133\\
534.01	0.00638842857845291\\
535.01	0.00641652321629788\\
536.01	0.00644519527912794\\
537.01	0.00647445661221022\\
538.01	0.00650431927544885\\
539.01	0.00653479554555353\\
540.01	0.0065658979179944\\
541.01	0.00659763910867128\\
542.01	0.00663003205521878\\
543.01	0.00666308991785967\\
544.01	0.00669682607971868\\
545.01	0.00673125414650544\\
546.01	0.00676638794547803\\
547.01	0.00680224152360036\\
548.01	0.00683882914480995\\
549.01	0.00687616528631111\\
550.01	0.00691426463379902\\
551.01	0.00695314207550284\\
552.01	0.00699281269490666\\
553.01	0.00703329176198434\\
554.01	0.00707459472276593\\
555.01	0.00711673718703461\\
556.01	0.00715973491393524\\
557.01	0.00720360379525376\\
558.01	0.00724835983610392\\
559.01	0.00729401913273179\\
560.01	0.00734059784712131\\
561.01	0.00738811217805081\\
562.01	0.00743657832821639\\
563.01	0.00748601246700066\\
564.01	0.0075364306884236\\
565.01	0.00758784896377284\\
566.01	0.00764028308836689\\
567.01	0.00769374862186187\\
568.01	0.00774826082146843\\
569.01	0.00780383456740263\\
570.01	0.00786048427985252\\
571.01	0.00791822382670581\\
572.01	0.00797706642125281\\
573.01	0.00803702450905831\\
574.01	0.00809810964319064\\
575.01	0.00816033234701034\\
576.01	0.00822370196376449\\
577.01	0.0082882264923125\\
578.01	0.00835391240844111\\
579.01	0.00842076447142201\\
580.01	0.00848878551574813\\
581.01	0.008557976228378\\
582.01	0.00862833491235338\\
583.01	0.0086998572383756\\
584.01	0.00877253598687894\\
585.01	0.00884636078439139\\
586.01	0.00892131783960339\\
587.01	0.00899738968667349\\
588.01	0.00907455494601469\\
589.01	0.00915278811628361\\
590.01	0.0092320594157346\\
591.01	0.00931233469675008\\
592.01	0.00939357546453003\\
593.01	0.00947573903999945\\
594.01	0.00955877891846273\\
595.01	0.00964264538999877\\
596.01	0.00972728650580666\\
597.01	0.00981264949761991\\
598.01	0.00989868278608211\\
599.01	0.00996919203046377\\
599.02	0.00996973314335038\\
599.03	0.0099702709571694\\
599.04	0.00997080543920336\\
599.05	0.0099713365564138\\
599.06	0.00997186427543808\\
599.07	0.0099723885625862\\
599.08	0.00997290938383756\\
599.09	0.00997342670483771\\
599.1	0.00997394049089505\\
599.11	0.00997445070697751\\
599.12	0.00997495731770918\\
599.13	0.00997546028736696\\
599.14	0.00997595957987707\\
599.15	0.00997645515881165\\
599.16	0.00997694698738526\\
599.17	0.00997743502845131\\
599.18	0.00997791924449856\\
599.19	0.00997839959764748\\
599.2	0.00997887604964662\\
599.21	0.00997934856186899\\
599.22	0.00997981709530829\\
599.23	0.00998028161057521\\
599.24	0.00998074206789365\\
599.25	0.0099811984270969\\
599.26	0.00998165064762378\\
599.27	0.00998209868851477\\
599.28	0.00998254250840806\\
599.29	0.00998298206553559\\
599.3	0.00998341731771905\\
599.31	0.00998384822236584\\
599.32	0.00998427473646497\\
599.33	0.00998469681658292\\
599.34	0.00998511441885952\\
599.35	0.0099855274990037\\
599.36	0.00998593601228926\\
599.37	0.00998633991355056\\
599.38	0.00998673915717821\\
599.39	0.00998713369711469\\
599.4	0.00998752348684992\\
599.41	0.00998790847811157\\
599.42	0.00998828861974491\\
599.43	0.00998866386008803\\
599.44	0.00998903414696681\\
599.45	0.00998939942768986\\
599.46	0.00998975964904344\\
599.47	0.00999011475728636\\
599.48	0.00999046469814472\\
599.49	0.00999080941680669\\
599.5	0.00999114885791725\\
599.51	0.00999148296557278\\
599.52	0.0099918116833157\\
599.53	0.00999213495412901\\
599.54	0.00999245272043077\\
599.55	0.00999276492406855\\
599.56	0.00999307150631382\\
599.57	0.00999337240785624\\
599.58	0.00999366756879799\\
599.59	0.00999395692864795\\
599.6	0.00999424042631586\\
599.61	0.00999451800010642\\
599.62	0.00999478958771336\\
599.63	0.0099950551262134\\
599.64	0.00999531455206019\\
599.65	0.00999556780107816\\
599.66	0.00999581480845634\\
599.67	0.00999605550874211\\
599.68	0.00999628983583487\\
599.69	0.00999651772297967\\
599.7	0.00999673910276075\\
599.71	0.00999695390709509\\
599.72	0.00999716206722577\\
599.73	0.00999736351371538\\
599.74	0.00999755817643933\\
599.75	0.00999774598457904\\
599.76	0.00999792686661517\\
599.77	0.00999810075032068\\
599.78	0.00999826756275386\\
599.79	0.00999842723025133\\
599.8	0.00999857967842092\\
599.81	0.0099987248321345\\
599.82	0.00999886261552071\\
599.83	0.00999899295195769\\
599.84	0.00999911576406567\\
599.85	0.00999923097369951\\
599.86	0.00999933850194117\\
599.87	0.00999943826909211\\
599.88	0.00999953019466558\\
599.89	0.00999961419737891\\
599.9	0.00999969019514566\\
599.91	0.00999975810506767\\
599.92	0.00999981784342713\\
599.93	0.00999986932567848\\
599.94	0.00999991246644028\\
599.95	0.00999994717948697\\
599.96	0.00999997337774056\\
599.97	0.00999999097326228\\
599.98	0.00999999987724406\\
599.99	0.01\\
600	0.01\\
};
\addplot [color=mycolor9,solid,forget plot]
  table[row sep=crcr]{%
0.01	0.00433017084612224\\
1.01	0.00433017312500381\\
2.01	0.00433017545095497\\
3.01	0.00433017782494978\\
4.01	0.00433018024798222\\
5.01	0.00433018272106704\\
6.01	0.00433018524523977\\
7.01	0.00433018782155786\\
8.01	0.00433019045110017\\
9.01	0.00433019313496824\\
10.01	0.00433019587428649\\
11.01	0.00433019867020241\\
12.01	0.00433020152388747\\
13.01	0.00433020443653742\\
14.01	0.00433020740937283\\
15.01	0.00433021044363945\\
16.01	0.00433021354060889\\
17.01	0.00433021670157932\\
18.01	0.00433021992787565\\
19.01	0.00433022322085035\\
20.01	0.00433022658188384\\
21.01	0.0043302300123854\\
22.01	0.00433023351379334\\
23.01	0.00433023708757577\\
24.01	0.00433024073523149\\
25.01	0.00433024445829021\\
26.01	0.00433024825831346\\
27.01	0.00433025213689534\\
28.01	0.00433025609566255\\
29.01	0.00433026013627608\\
30.01	0.0043302642604309\\
31.01	0.00433026846985764\\
32.01	0.0043302727663225\\
33.01	0.00433027715162843\\
34.01	0.00433028162761594\\
35.01	0.00433028619616342\\
36.01	0.00433029085918867\\
37.01	0.00433029561864896\\
38.01	0.00433030047654247\\
39.01	0.00433030543490852\\
40.01	0.0043303104958291\\
41.01	0.00433031566142937\\
42.01	0.00433032093387836\\
43.01	0.00433032631539047\\
44.01	0.00433033180822573\\
45.01	0.00433033741469135\\
46.01	0.0043303431371427\\
47.01	0.00433034897798339\\
48.01	0.00433035493966745\\
49.01	0.00433036102469984\\
50.01	0.00433036723563757\\
51.01	0.00433037357509045\\
52.01	0.00433038004572285\\
53.01	0.00433038665025423\\
54.01	0.00433039339146085\\
55.01	0.00433040027217621\\
56.01	0.00433040729529295\\
57.01	0.004330414463764\\
58.01	0.00433042178060344\\
59.01	0.00433042924888788\\
60.01	0.00433043687175783\\
61.01	0.00433044465241937\\
62.01	0.00433045259414487\\
63.01	0.00433046070027501\\
64.01	0.00433046897421975\\
65.01	0.00433047741945979\\
66.01	0.00433048603954836\\
67.01	0.00433049483811269\\
68.01	0.00433050381885497\\
69.01	0.00433051298555459\\
70.01	0.0043305223420696\\
71.01	0.00433053189233826\\
72.01	0.00433054164038044\\
73.01	0.00433055159029981\\
74.01	0.00433056174628528\\
75.01	0.00433057211261298\\
76.01	0.00433058269364797\\
77.01	0.00433059349384581\\
78.01	0.00433060451775481\\
79.01	0.00433061577001809\\
80.01	0.00433062725537522\\
81.01	0.00433063897866427\\
82.01	0.00433065094482382\\
83.01	0.0043306631588957\\
84.01	0.00433067562602585\\
85.01	0.00433068835146781\\
86.01	0.00433070134058434\\
87.01	0.00433071459884948\\
88.01	0.00433072813185141\\
89.01	0.00433074194529469\\
90.01	0.00433075604500216\\
91.01	0.00433077043691803\\
92.01	0.00433078512711026\\
93.01	0.00433080012177299\\
94.01	0.00433081542722946\\
95.01	0.00433083104993409\\
96.01	0.00433084699647591\\
97.01	0.004330863273581\\
98.01	0.00433087988811552\\
99.01	0.00433089684708847\\
100.01	0.00433091415765466\\
101.01	0.00433093182711798\\
102.01	0.00433094986293427\\
103.01	0.00433096827271437\\
104.01	0.00433098706422815\\
105.01	0.00433100624540667\\
106.01	0.00433102582434618\\
107.01	0.00433104580931194\\
108.01	0.00433106620874071\\
109.01	0.00433108703124521\\
110.01	0.00433110828561756\\
111.01	0.0043311299808329\\
112.01	0.00433115212605309\\
113.01	0.00433117473063106\\
114.01	0.00433119780411428\\
115.01	0.00433122135624922\\
116.01	0.00433124539698553\\
117.01	0.00433126993647953\\
118.01	0.00433129498509968\\
119.01	0.0043313205534299\\
120.01	0.00433134665227493\\
121.01	0.00433137329266459\\
122.01	0.00433140048585827\\
123.01	0.00433142824335001\\
124.01	0.0043314565768733\\
125.01	0.00433148549840623\\
126.01	0.00433151502017609\\
127.01	0.00433154515466544\\
128.01	0.0043315759146164\\
129.01	0.00433160731303699\\
130.01	0.00433163936320614\\
131.01	0.00433167207867974\\
132.01	0.00433170547329559\\
133.01	0.00433173956118046\\
134.01	0.00433177435675522\\
135.01	0.00433180987474144\\
136.01	0.00433184613016742\\
137.01	0.0043318831383747\\
138.01	0.00433192091502473\\
139.01	0.00433195947610543\\
140.01	0.00433199883793768\\
141.01	0.00433203901718285\\
142.01	0.00433208003084949\\
143.01	0.00433212189630079\\
144.01	0.00433216463126229\\
145.01	0.00433220825382846\\
146.01	0.00433225278247137\\
147.01	0.00433229823604812\\
148.01	0.00433234463380919\\
149.01	0.00433239199540626\\
150.01	0.00433244034090077\\
151.01	0.00433248969077268\\
152.01	0.00433254006592888\\
153.01	0.00433259148771226\\
154.01	0.00433264397791094\\
155.01	0.00433269755876733\\
156.01	0.00433275225298779\\
157.01	0.00433280808375204\\
158.01	0.00433286507472368\\
159.01	0.0043329232500595\\
160.01	0.00433298263442053\\
161.01	0.00433304325298182\\
162.01	0.00433310513144383\\
163.01	0.00433316829604315\\
164.01	0.00433323277356362\\
165.01	0.00433329859134778\\
166.01	0.00433336577730876\\
167.01	0.00433343435994199\\
168.01	0.00433350436833752\\
169.01	0.00433357583219236\\
170.01	0.00433364878182305\\
171.01	0.0043337232481787\\
172.01	0.00433379926285424\\
173.01	0.00433387685810409\\
174.01	0.00433395606685556\\
175.01	0.00433403692272322\\
176.01	0.00433411946002318\\
177.01	0.00433420371378762\\
178.01	0.00433428971978003\\
179.01	0.00433437751451023\\
180.01	0.00433446713525018\\
181.01	0.00433455862004985\\
182.01	0.00433465200775332\\
183.01	0.00433474733801562\\
184.01	0.00433484465131949\\
185.01	0.00433494398899309\\
186.01	0.00433504539322707\\
187.01	0.00433514890709322\\
188.01	0.00433525457456245\\
189.01	0.00433536244052353\\
190.01	0.00433547255080315\\
191.01	0.00433558495218429\\
192.01	0.00433569969242707\\
193.01	0.00433581682028934\\
194.01	0.00433593638554661\\
195.01	0.00433605843901434\\
196.01	0.00433618303256922\\
197.01	0.00433631021917127\\
198.01	0.00433644005288711\\
199.01	0.00433657258891254\\
200.01	0.00433670788359651\\
201.01	0.00433684599446493\\
202.01	0.00433698698024577\\
203.01	0.00433713090089369\\
204.01	0.00433727781761672\\
205.01	0.00433742779290112\\
206.01	0.00433758089053933\\
207.01	0.00433773717565676\\
208.01	0.00433789671474018\\
209.01	0.0043380595756658\\
210.01	0.00433822582772853\\
211.01	0.00433839554167183\\
212.01	0.00433856878971793\\
213.01	0.00433874564559936\\
214.01	0.00433892618459006\\
215.01	0.00433911048353805\\
216.01	0.00433929862089806\\
217.01	0.00433949067676585\\
218.01	0.00433968673291245\\
219.01	0.00433988687281911\\
220.01	0.00434009118171333\\
221.01	0.00434029974660607\\
222.01	0.00434051265632852\\
223.01	0.00434073000157076\\
224.01	0.00434095187492078\\
225.01	0.0043411783709045\\
226.01	0.00434140958602644\\
227.01	0.00434164561881134\\
228.01	0.00434188656984646\\
229.01	0.0043421325418261\\
230.01	0.0043423836395948\\
231.01	0.00434263997019303\\
232.01	0.00434290164290378\\
233.01	0.00434316876929976\\
234.01	0.00434344146329135\\
235.01	0.0043437198411766\\
236.01	0.00434400402169097\\
237.01	0.00434429412605936\\
238.01	0.00434459027804859\\
239.01	0.00434489260402099\\
240.01	0.00434520123298968\\
241.01	0.0043455162966741\\
242.01	0.00434583792955812\\
243.01	0.00434616626894793\\
244.01	0.00434650145503243\\
245.01	0.00434684363094361\\
246.01	0.00434719294281992\\
247.01	0.00434754953986957\\
248.01	0.00434791357443598\\
249.01	0.00434828520206429\\
250.01	0.00434866458156952\\
251.01	0.00434905187510615\\
252.01	0.00434944724823953\\
253.01	0.00434985087001818\\
254.01	0.0043502629130481\\
255.01	0.00435068355356909\\
256.01	0.00435111297153221\\
257.01	0.00435155135067847\\
258.01	0.00435199887862077\\
259.01	0.00435245574692603\\
260.01	0.00435292215120055\\
261.01	0.00435339829117608\\
262.01	0.00435388437079812\\
263.01	0.00435438059831655\\
264.01	0.00435488718637828\\
265.01	0.00435540435212144\\
266.01	0.00435593231727177\\
267.01	0.00435647130824146\\
268.01	0.00435702155622998\\
269.01	0.00435758329732734\\
270.01	0.00435815677261919\\
271.01	0.0043587422282943\\
272.01	0.00435933991575502\\
273.01	0.00435995009172968\\
274.01	0.00436057301838702\\
275.01	0.0043612089634542\\
276.01	0.00436185820033605\\
277.01	0.00436252100823858\\
278.01	0.0043631976722935\\
279.01	0.00436388848368592\\
280.01	0.00436459373978564\\
281.01	0.00436531374427969\\
282.01	0.00436604880730937\\
283.01	0.00436679924560854\\
284.01	0.00436756538264553\\
285.01	0.00436834754876796\\
286.01	0.00436914608135021\\
287.01	0.0043699613249442\\
288.01	0.00437079363143217\\
289.01	0.00437164336018346\\
290.01	0.00437251087821332\\
291.01	0.0043733965603453\\
292.01	0.00437430078937547\\
293.01	0.00437522395624056\\
294.01	0.00437616646018743\\
295.01	0.00437712870894692\\
296.01	0.00437811111890782\\
297.01	0.00437911411529571\\
298.01	0.00438013813235176\\
299.01	0.00438118361351485\\
300.01	0.00438225101160518\\
301.01	0.00438334078900816\\
302.01	0.00438445341786145\\
303.01	0.00438558938024082\\
304.01	0.00438674916834735\\
305.01	0.0043879332846942\\
306.01	0.0043891422422918\\
307.01	0.00439037656483271\\
308.01	0.0043916367868736\\
309.01	0.00439292345401367\\
310.01	0.00439423712306931\\
311.01	0.00439557836224365\\
312.01	0.00439694775128867\\
313.01	0.00439834588165999\\
314.01	0.00439977335666125\\
315.01	0.00440123079157653\\
316.01	0.00440271881378908\\
317.01	0.00440423806288427\\
318.01	0.0044057891907312\\
319.01	0.00440737286154438\\
320.01	0.00440898975191703\\
321.01	0.00441064055082571\\
322.01	0.00441232595959958\\
323.01	0.00441404669184965\\
324.01	0.00441580347335209\\
325.01	0.00441759704187901\\
326.01	0.00441942814696891\\
327.01	0.00442129754962872\\
328.01	0.00442320602195728\\
329.01	0.00442515434667984\\
330.01	0.00442714331658105\\
331.01	0.00442917373382191\\
332.01	0.004431246409127\\
333.01	0.00443336216082205\\
334.01	0.00443552181370352\\
335.01	0.00443772619771644\\
336.01	0.004439976146417\\
337.01	0.00444227249518929\\
338.01	0.00444461607918624\\
339.01	0.00444700773095674\\
340.01	0.00444944827772067\\
341.01	0.00445193853824481\\
342.01	0.00445447931927011\\
343.01	0.00445707141143288\\
344.01	0.00445971558461861\\
345.01	0.00446241258267614\\
346.01	0.00446516311741744\\
347.01	0.00446796786181537\\
348.01	0.00447082744230989\\
349.01	0.00447374243011707\\
350.01	0.00447671333143547\\
351.01	0.00447974057643231\\
352.01	0.00448282450688689\\
353.01	0.00448596536236742\\
354.01	0.0044891632648139\\
355.01	0.00449241820140781\\
356.01	0.00449573000562052\\
357.01	0.00449909833635545\\
358.01	0.00450252265513718\\
359.01	0.00450600220135811\\
360.01	0.00450953596567819\\
361.01	0.00451312266179822\\
362.01	0.00451676069699428\\
363.01	0.00452044814204629\\
364.01	0.00452418270151704\\
365.01	0.00452796168578788\\
366.01	0.00453178198685936\\
367.01	0.00453564006073842\\
368.01	0.00453953192031815\\
369.01	0.00454345314410495\\
370.01	0.00454739890806999\\
371.01	0.0045513640504516\\
372.01	0.00455534318270358\\
373.01	0.00455933086423309\\
374.01	0.00456332186443281\\
375.01	0.00456731154322568\\
376.01	0.00457129639147949\\
377.01	0.00457527478596882\\
378.01	0.00457924803103825\\
379.01	0.00458322178203278\\
380.01	0.00458723576922909\\
381.01	0.00459132580643362\\
382.01	0.00459549317978798\\
383.01	0.00459973918071012\\
384.01	0.00460406511206603\\
385.01	0.00460847228753523\\
386.01	0.00461296203090729\\
387.01	0.00461753567530385\\
388.01	0.00462219456232112\\
389.01	0.00462694004108527\\
390.01	0.00463177346721592\\
391.01	0.00463669620168919\\
392.01	0.00464170960959405\\
393.01	0.00464681505877265\\
394.01	0.00465201391833729\\
395.01	0.00465730755705253\\
396.01	0.00466269734157579\\
397.01	0.00466818463454245\\
398.01	0.00467377079248614\\
399.01	0.00467945716358194\\
400.01	0.00468524508519763\\
401.01	0.00469113588124237\\
402.01	0.00469713085929439\\
403.01	0.00470323130749577\\
404.01	0.00470943849119407\\
405.01	0.00471575364931583\\
406.01	0.0047221779904517\\
407.01	0.00472871268863441\\
408.01	0.00473535887878805\\
409.01	0.00474211765182792\\
410.01	0.00474899004938528\\
411.01	0.00475597705813798\\
412.01	0.00476307960371755\\
413.01	0.00477029854417019\\
414.01	0.00477763466294448\\
415.01	0.0047850886613799\\
416.01	0.00479266115066852\\
417.01	0.00480035264326593\\
418.01	0.00480816354372278\\
419.01	0.00481609413891522\\
420.01	0.00482414458765092\\
421.01	0.00483231490963051\\
422.01	0.00484060497375127\\
423.01	0.00484901448574133\\
424.01	0.00485754297512131\\
425.01	0.00486618978150076\\
426.01	0.00487495404022541\\
427.01	0.00488383466740644\\
428.01	0.00489283034438197\\
429.01	0.00490193950168054\\
430.01	0.00491116030258371\\
431.01	0.00492049062641793\\
432.01	0.00492992805174296\\
433.01	0.00493946983965368\\
434.01	0.00494911291746512\\
435.01	0.00495885386312005\\
436.01	0.00496868889073661\\
437.01	0.00497861383780683\\
438.01	0.0049886241546663\\
439.01	0.00499871489698278\\
440.01	0.00500888072215998\\
441.01	0.00501911589071902\\
442.01	0.00502941427391444\\
443.01	0.00503976936905594\\
444.01	0.00505017432424469\\
445.01	0.00506062197449125\\
446.01	0.00507110489145547\\
447.01	0.00508161544932473\\
448.01	0.00509214590961422\\
449.01	0.00510268852790492\\
450.01	0.00511323568569115\\
451.01	0.00512378005054773\\
452.01	0.00513431476765311\\
453.01	0.00514483368523116\\
454.01	0.00515533161552699\\
455.01	0.00516580463131715\\
456.01	0.00517625039536382\\
457.01	0.00518666851626712\\
458.01	0.00519706091828341\\
459.01	0.005207432204131\\
460.01	0.00521778997756235\\
461.01	0.00522814507518973\\
462.01	0.00523851163286017\\
463.01	0.00524890687837509\\
464.01	0.00525935049808428\\
465.01	0.00526986336656013\\
466.01	0.00528046390900781\\
467.01	0.00529115632719554\\
468.01	0.00530193845264243\\
469.01	0.00531280807546486\\
470.01	0.00532376301529709\\
471.01	0.00533480115650591\\
472.01	0.00534592048846137\\
473.01	0.00535711915115739\\
474.01	0.00536839548638489\\
475.01	0.00537974809454407\\
476.01	0.00539117589704932\\
477.01	0.00540267820407779\\
478.01	0.00541425478715471\\
479.01	0.00542590595576906\\
480.01	0.00543763263680076\\
481.01	0.00544943645503797\\
482.01	0.00546131981246499\\
483.01	0.00547328596328735\\
484.01	0.0054853390808436\\
485.01	0.00549748431163628\\
486.01	0.00550972781073364\\
487.01	0.00552207675180393\\
488.01	0.00553453930414448\\
489.01	0.00554712456842048\\
490.01	0.00555984246268859\\
491.01	0.00557270355102107\\
492.01	0.00558571880923359\\
493.01	0.00559889932668814\\
494.01	0.00561225595107349\\
495.01	0.00562579889621909\\
496.01	0.00563953735363371\\
497.01	0.00565347917996437\\
498.01	0.00566763086460868\\
499.01	0.00568199845586227\\
500.01	0.00569658835651467\\
501.01	0.00571140738657194\\
502.01	0.00572646278248946\\
503.01	0.00574176219000061\\
504.01	0.0057573136497279\\
505.01	0.0057731255748838\\
506.01	0.00578920672058626\\
507.01	0.00580556614463656\\
508.01	0.0058222131600648\\
509.01	0.00583915728036193\\
510.01	0.00585640815910021\\
511.01	0.00587397552660074\\
512.01	0.00589186912741288\\
513.01	0.00591009866356706\\
514.01	0.00592867374970717\\
515.01	0.00594760388707689\\
516.01	0.00596689846351392\\
517.01	0.00598656678545\\
518.01	0.00600661814441272\\
519.01	0.00602706191228984\\
520.01	0.00604790762593617\\
521.01	0.00606916500494155\\
522.01	0.00609084394021554\\
523.01	0.00611295448121869\\
524.01	0.006135506823636\\
525.01	0.00615851129817362\\
526.01	0.00618197836122347\\
527.01	0.00620591858815658\\
528.01	0.00623034266995729\\
529.01	0.00625526141377142\\
530.01	0.00628068574768584\\
531.01	0.00630662672966783\\
532.01	0.00633309556005505\\
533.01	0.00636010359633393\\
534.01	0.00638766236823543\\
535.01	0.00641578359059092\\
536.01	0.00644447917130381\\
537.01	0.00647376121386944\\
538.01	0.00650364201786405\\
539.01	0.00653413407973057\\
540.01	0.00656525009409123\\
541.01	0.00659700295554078\\
542.01	0.00662940576081271\\
543.01	0.00666247181114391\\
544.01	0.00669621461459394\\
545.01	0.00673064788801325\\
546.01	0.00676578555831429\\
547.01	0.0068016417626947\\
548.01	0.00683823084750839\\
549.01	0.00687556736559398\\
550.01	0.00691366607204055\\
551.01	0.00695254191854659\\
552.01	0.00699221004650432\\
553.01	0.0070326857787142\\
554.01	0.00707398460951927\\
555.01	0.00711612219311656\\
556.01	0.00715911432978177\\
557.01	0.00720297694972537\\
558.01	0.00724772609428095\\
559.01	0.00729337789411719\\
560.01	0.00733994854415049\\
561.01	0.00738745427482568\\
562.01	0.00743591131940887\\
563.01	0.00748533587690427\\
564.01	0.00753574407015443\\
565.01	0.00758715189862288\\
566.01	0.00763957518530753\\
567.01	0.00769302951718983\\
568.01	0.00774753017858247\\
569.01	0.00780309207669819\\
570.01	0.00785972965872284\\
571.01	0.00791745681964229\\
572.01	0.00797628680004186\\
573.01	0.00803623207307779\\
574.01	0.0080973042198116\\
575.01	0.00815951379211108\\
576.01	0.00822287016236105\\
577.01	0.00828738135930615\\
578.01	0.00835305388947689\\
579.01	0.0084198925438449\\
580.01	0.00848790018963276\\
581.01	0.00855707754759355\\
582.01	0.00862742295560682\\
583.01	0.00869893212015025\\
584.01	0.00877159785815303\\
585.01	0.00884540983297857\\
586.01	0.00892035428990249\\
587.01	0.00899641379854713\\
588.01	0.00907356701243149\\
589.01	0.00915178845925224\\
590.01	0.00923104837992613\\
591.01	0.00931131264004169\\
592.01	0.00939254274449733\\
593.01	0.00947469599512985\\
594.01	0.00955772584254327\\
595.01	0.00964158249773262\\
596.01	0.00972621388721412\\
597.01	0.00981156705815533\\
598.01	0.00989759016861823\\
599.01	0.00996919203046376\\
599.02	0.00996973314335037\\
599.03	0.00997027095716939\\
599.04	0.00997080543920335\\
599.05	0.00997133655641379\\
599.06	0.00997186427543808\\
599.07	0.0099723885625862\\
599.08	0.00997290938383756\\
599.09	0.00997342670483771\\
599.1	0.00997394049089505\\
599.11	0.00997445070697751\\
599.12	0.00997495731770918\\
599.13	0.00997546028736696\\
599.14	0.00997595957987706\\
599.15	0.00997645515881165\\
599.16	0.00997694698738526\\
599.17	0.00997743502845131\\
599.18	0.00997791924449856\\
599.19	0.00997839959764748\\
599.2	0.00997887604964662\\
599.21	0.00997934856186899\\
599.22	0.00997981709530829\\
599.23	0.00998028161057521\\
599.24	0.00998074206789365\\
599.25	0.0099811984270969\\
599.26	0.00998165064762378\\
599.27	0.00998209868851477\\
599.28	0.00998254250840806\\
599.29	0.00998298206553559\\
599.3	0.00998341731771905\\
599.31	0.00998384822236584\\
599.32	0.00998427473646497\\
599.33	0.00998469681658292\\
599.34	0.00998511441885952\\
599.35	0.0099855274990037\\
599.36	0.00998593601228926\\
599.37	0.00998633991355056\\
599.38	0.00998673915717821\\
599.39	0.00998713369711469\\
599.4	0.00998752348684992\\
599.41	0.00998790847811157\\
599.42	0.00998828861974491\\
599.43	0.00998866386008804\\
599.44	0.00998903414696681\\
599.45	0.00998939942768985\\
599.46	0.00998975964904344\\
599.47	0.00999011475728636\\
599.48	0.00999046469814472\\
599.49	0.00999080941680669\\
599.5	0.00999114885791725\\
599.51	0.00999148296557278\\
599.52	0.0099918116833157\\
599.53	0.00999213495412901\\
599.54	0.00999245272043077\\
599.55	0.00999276492406855\\
599.56	0.00999307150631381\\
599.57	0.00999337240785624\\
599.58	0.00999366756879799\\
599.59	0.00999395692864795\\
599.6	0.00999424042631585\\
599.61	0.00999451800010642\\
599.62	0.00999478958771336\\
599.63	0.0099950551262134\\
599.64	0.00999531455206019\\
599.65	0.00999556780107816\\
599.66	0.00999581480845634\\
599.67	0.00999605550874211\\
599.68	0.00999628983583487\\
599.69	0.00999651772297967\\
599.7	0.00999673910276076\\
599.71	0.00999695390709509\\
599.72	0.00999716206722577\\
599.73	0.00999736351371538\\
599.74	0.00999755817643933\\
599.75	0.00999774598457904\\
599.76	0.00999792686661517\\
599.77	0.00999810075032068\\
599.78	0.00999826756275386\\
599.79	0.00999842723025133\\
599.8	0.00999857967842092\\
599.81	0.0099987248321345\\
599.82	0.00999886261552071\\
599.83	0.00999899295195769\\
599.84	0.00999911576406567\\
599.85	0.00999923097369951\\
599.86	0.00999933850194117\\
599.87	0.00999943826909211\\
599.88	0.00999953019466558\\
599.89	0.00999961419737891\\
599.9	0.00999969019514566\\
599.91	0.00999975810506767\\
599.92	0.00999981784342713\\
599.93	0.00999986932567848\\
599.94	0.00999991246644028\\
599.95	0.00999994717948697\\
599.96	0.00999997337774056\\
599.97	0.00999999097326228\\
599.98	0.00999999987724406\\
599.99	0.01\\
600	0.01\\
};
\addplot [color=blue!50!mycolor7,solid,forget plot]
  table[row sep=crcr]{%
0.01	0.00384759528872361\\
1.01	0.00384759650035796\\
2.01	0.00384759773699071\\
3.01	0.00384759899913818\\
4.01	0.0038476002873275\\
5.01	0.00384760160209639\\
6.01	0.00384760294399437\\
7.01	0.00384760431358136\\
8.01	0.00384760571142964\\
9.01	0.003847607138123\\
10.01	0.00384760859425731\\
11.01	0.00384761008044081\\
12.01	0.00384761159729444\\
13.01	0.00384761314545166\\
14.01	0.00384761472555935\\
15.01	0.00384761633827756\\
16.01	0.00384761798428016\\
17.01	0.00384761966425487\\
18.01	0.00384762137890358\\
19.01	0.00384762312894275\\
20.01	0.0038476249151037\\
21.01	0.00384762673813283\\
22.01	0.00384762859879202\\
23.01	0.00384763049785899\\
24.01	0.00384763243612715\\
25.01	0.00384763441440672\\
26.01	0.00384763643352468\\
27.01	0.00384763849432476\\
28.01	0.00384764059766865\\
29.01	0.00384764274443537\\
30.01	0.00384764493552242\\
31.01	0.00384764717184573\\
32.01	0.00384764945434041\\
33.01	0.00384765178396063\\
34.01	0.00384765416168033\\
35.01	0.00384765658849384\\
36.01	0.00384765906541576\\
37.01	0.00384766159348192\\
38.01	0.00384766417374949\\
39.01	0.00384766680729738\\
40.01	0.00384766949522697\\
41.01	0.00384767223866228\\
42.01	0.0038476750387509\\
43.01	0.00384767789666381\\
44.01	0.00384768081359639\\
45.01	0.00384768379076875\\
46.01	0.00384768682942599\\
47.01	0.00384768993083938\\
48.01	0.00384769309630635\\
49.01	0.00384769632715087\\
50.01	0.00384769962472466\\
51.01	0.00384770299040733\\
52.01	0.00384770642560689\\
53.01	0.00384770993176066\\
54.01	0.0038477135103356\\
55.01	0.00384771716282899\\
56.01	0.00384772089076922\\
57.01	0.00384772469571612\\
58.01	0.00384772857926173\\
59.01	0.00384773254303135\\
60.01	0.00384773658868366\\
61.01	0.0038477407179118\\
62.01	0.00384774493244371\\
63.01	0.00384774923404323\\
64.01	0.00384775362451083\\
65.01	0.00384775810568398\\
66.01	0.0038477626794382\\
67.01	0.00384776734768782\\
68.01	0.00384777211238706\\
69.01	0.00384777697553007\\
70.01	0.00384778193915269\\
71.01	0.00384778700533263\\
72.01	0.00384779217619059\\
73.01	0.0038477974538913\\
74.01	0.00384780284064408\\
75.01	0.00384780833870411\\
76.01	0.00384781395037312\\
77.01	0.00384781967800038\\
78.01	0.00384782552398412\\
79.01	0.00384783149077164\\
80.01	0.00384783758086119\\
81.01	0.00384784379680263\\
82.01	0.00384785014119883\\
83.01	0.00384785661670597\\
84.01	0.00384786322603557\\
85.01	0.00384786997195508\\
86.01	0.00384787685728921\\
87.01	0.00384788388492122\\
88.01	0.00384789105779376\\
89.01	0.00384789837891045\\
90.01	0.00384790585133714\\
91.01	0.00384791347820311\\
92.01	0.0038479212627022\\
93.01	0.00384792920809417\\
94.01	0.00384793731770647\\
95.01	0.00384794559493538\\
96.01	0.0038479540432473\\
97.01	0.00384796266618023\\
98.01	0.00384797146734549\\
99.01	0.00384798045042906\\
100.01	0.00384798961919297\\
101.01	0.00384799897747733\\
102.01	0.0038480085292014\\
103.01	0.00384801827836574\\
104.01	0.00384802822905329\\
105.01	0.00384803838543156\\
106.01	0.00384804875175446\\
107.01	0.00384805933236342\\
108.01	0.00384807013168999\\
109.01	0.0038480811542573\\
110.01	0.00384809240468159\\
111.01	0.00384810388767497\\
112.01	0.00384811560804697\\
113.01	0.00384812757070631\\
114.01	0.00384813978066323\\
115.01	0.00384815224303175\\
116.01	0.0038481649630312\\
117.01	0.00384817794598918\\
118.01	0.0038481911973432\\
119.01	0.00384820472264338\\
120.01	0.00384821852755422\\
121.01	0.00384823261785741\\
122.01	0.00384824699945434\\
123.01	0.00384826167836812\\
124.01	0.00384827666074634\\
125.01	0.00384829195286389\\
126.01	0.00384830756112499\\
127.01	0.00384832349206626\\
128.01	0.00384833975235986\\
129.01	0.00384835634881517\\
130.01	0.00384837328838253\\
131.01	0.00384839057815589\\
132.01	0.00384840822537588\\
133.01	0.00384842623743248\\
134.01	0.00384844462186852\\
135.01	0.00384846338638264\\
136.01	0.00384848253883255\\
137.01	0.00384850208723819\\
138.01	0.00384852203978511\\
139.01	0.00384854240482807\\
140.01	0.00384856319089436\\
141.01	0.00384858440668753\\
142.01	0.00384860606109089\\
143.01	0.0038486281631708\\
144.01	0.00384865072218134\\
145.01	0.00384867374756762\\
146.01	0.00384869724896983\\
147.01	0.00384872123622711\\
148.01	0.00384874571938182\\
149.01	0.00384877070868388\\
150.01	0.00384879621459475\\
151.01	0.00384882224779173\\
152.01	0.00384884881917291\\
153.01	0.0038488759398613\\
154.01	0.00384890362120956\\
155.01	0.00384893187480476\\
156.01	0.00384896071247302\\
157.01	0.00384899014628526\\
158.01	0.00384902018856088\\
159.01	0.00384905085187411\\
160.01	0.00384908214905842\\
161.01	0.00384911409321263\\
162.01	0.00384914669770575\\
163.01	0.00384917997618305\\
164.01	0.00384921394257109\\
165.01	0.00384924861108442\\
166.01	0.00384928399623073\\
167.01	0.00384932011281745\\
168.01	0.00384935697595744\\
169.01	0.00384939460107557\\
170.01	0.00384943300391552\\
171.01	0.00384947220054573\\
172.01	0.00384951220736654\\
173.01	0.00384955304111667\\
174.01	0.00384959471888087\\
175.01	0.00384963725809606\\
176.01	0.00384968067655957\\
177.01	0.0038497249924361\\
178.01	0.00384977022426548\\
179.01	0.00384981639097029\\
180.01	0.00384986351186402\\
181.01	0.0038499116066588\\
182.01	0.00384996069547404\\
183.01	0.0038500107988446\\
184.01	0.00385006193772938\\
185.01	0.00385011413352032\\
186.01	0.00385016740805139\\
187.01	0.00385022178360732\\
188.01	0.00385027728293341\\
189.01	0.00385033392924493\\
190.01	0.0038503917462368\\
191.01	0.00385045075809383\\
192.01	0.00385051098950047\\
193.01	0.00385057246565138\\
194.01	0.00385063521226227\\
195.01	0.00385069925558031\\
196.01	0.00385076462239561\\
197.01	0.00385083134005224\\
198.01	0.00385089943645991\\
199.01	0.00385096894010575\\
200.01	0.00385103988006619\\
201.01	0.00385111228601959\\
202.01	0.00385118618825844\\
203.01	0.00385126161770273\\
204.01	0.00385133860591234\\
205.01	0.00385141718510092\\
206.01	0.00385149738814972\\
207.01	0.00385157924862121\\
208.01	0.00385166280077365\\
209.01	0.00385174807957529\\
210.01	0.00385183512072013\\
211.01	0.00385192396064234\\
212.01	0.0038520146365324\\
213.01	0.00385210718635255\\
214.01	0.00385220164885365\\
215.01	0.00385229806359161\\
216.01	0.00385239647094437\\
217.01	0.00385249691212945\\
218.01	0.00385259942922155\\
219.01	0.00385270406517098\\
220.01	0.00385281086382266\\
221.01	0.00385291986993412\\
222.01	0.00385303112919588\\
223.01	0.00385314468825143\\
224.01	0.00385326059471697\\
225.01	0.00385337889720312\\
226.01	0.00385349964533548\\
227.01	0.0038536228897771\\
228.01	0.00385374868225075\\
229.01	0.0038538770755614\\
230.01	0.00385400812362009\\
231.01	0.0038541418814681\\
232.01	0.00385427840530114\\
233.01	0.00385441775249454\\
234.01	0.00385455998162938\\
235.01	0.00385470515251875\\
236.01	0.0038548533262347\\
237.01	0.00385500456513629\\
238.01	0.00385515893289778\\
239.01	0.00385531649453771\\
240.01	0.00385547731644899\\
241.01	0.0038556414664296\\
242.01	0.00385580901371367\\
243.01	0.00385598002900422\\
244.01	0.00385615458450575\\
245.01	0.00385633275395872\\
246.01	0.00385651461267368\\
247.01	0.00385670023756761\\
248.01	0.00385688970720062\\
249.01	0.00385708310181328\\
250.01	0.00385728050336587\\
251.01	0.00385748199557781\\
252.01	0.00385768766396863\\
253.01	0.00385789759590026\\
254.01	0.00385811188062017\\
255.01	0.003858330609306\\
256.01	0.0038585538751109\\
257.01	0.00385878177321131\\
258.01	0.00385901440085491\\
259.01	0.00385925185741097\\
260.01	0.00385949424442153\\
261.01	0.00385974166565466\\
262.01	0.00385999422715901\\
263.01	0.00386025203732057\\
264.01	0.00386051520691995\\
265.01	0.00386078384919294\\
266.01	0.00386105807989293\\
267.01	0.00386133801735403\\
268.01	0.00386162378255764\\
269.01	0.00386191549920036\\
270.01	0.00386221329376521\\
271.01	0.00386251729559428\\
272.01	0.00386282763696461\\
273.01	0.00386314445316586\\
274.01	0.00386346788258228\\
275.01	0.00386379806677595\\
276.01	0.0038641351505745\\
277.01	0.00386447928216084\\
278.01	0.00386483061316766\\
279.01	0.0038651892987745\\
280.01	0.00386555549780873\\
281.01	0.00386592937285116\\
282.01	0.0038663110903453\\
283.01	0.00386670082071099\\
284.01	0.00386709873846357\\
285.01	0.00386750502233688\\
286.01	0.00386791985541253\\
287.01	0.00386834342525398\\
288.01	0.00386877592404709\\
289.01	0.00386921754874664\\
290.01	0.00386966850122977\\
291.01	0.00387012898845597\\
292.01	0.0038705992226349\\
293.01	0.00387107942140262\\
294.01	0.00387156980800543\\
295.01	0.003872070611493\\
296.01	0.00387258206692204\\
297.01	0.00387310441556857\\
298.01	0.00387363790515272\\
299.01	0.0038741827900744\\
300.01	0.00387473933166152\\
301.01	0.00387530779843209\\
302.01	0.00387588846636999\\
303.01	0.00387648161921647\\
304.01	0.00387708754877832\\
305.01	0.00387770655525267\\
306.01	0.00387833894757257\\
307.01	0.00387898504377129\\
308.01	0.00387964517136917\\
309.01	0.00388031966778415\\
310.01	0.00388100888076767\\
311.01	0.003881713168868\\
312.01	0.00388243290192311\\
313.01	0.00388316846158542\\
314.01	0.00388392024188122\\
315.01	0.00388468864980734\\
316.01	0.0038854741059684\\
317.01	0.00388627704525726\\
318.01	0.0038870979175842\\
319.01	0.00388793718865625\\
320.01	0.00388879534081335\\
321.01	0.00388967287392496\\
322.01	0.00389057030635291\\
323.01	0.00389148817598656\\
324.01	0.00389242704135562\\
325.01	0.00389338748282962\\
326.01	0.00389437010390933\\
327.01	0.00389537553262136\\
328.01	0.0038964044230234\\
329.01	0.003897457456831\\
330.01	0.00389853534517806\\
331.01	0.00389963883052295\\
332.01	0.00390076868871406\\
333.01	0.00390192573123038\\
334.01	0.00390311080761456\\
335.01	0.00390432480811607\\
336.01	0.00390556866656577\\
337.01	0.00390684336350376\\
338.01	0.0039081499295861\\
339.01	0.00390948944929785\\
340.01	0.00391086306500041\\
341.01	0.00391227198134919\\
342.01	0.00391371747011414\\
343.01	0.00391520087544454\\
344.01	0.00391672361961845\\
345.01	0.00391828720932369\\
346.01	0.00391989324251817\\
347.01	0.00392154341592235\\
348.01	0.00392323953319734\\
349.01	0.00392498351386607\\
350.01	0.00392677740303493\\
351.01	0.00392862338197279\\
352.01	0.00393052377960143\\
353.01	0.00393248108494643\\
354.01	0.00393449796058669\\
355.01	0.00393657725712421\\
356.01	0.00393872202867239\\
357.01	0.00394093554932545\\
358.01	0.00394322133052201\\
359.01	0.00394558313914724\\
360.01	0.00394802501612099\\
361.01	0.00395055129509342\\
362.01	0.00395316662069238\\
363.01	0.00395587596553301\\
364.01	0.00395868464488944\\
365.01	0.00396159832751057\\
366.01	0.00396462304051402\\
367.01	0.00396776516556191\\
368.01	0.00397103142256452\\
369.01	0.00397442883589473\\
370.01	0.00397796467643419\\
371.01	0.00398164637059396\\
372.01	0.00398548136459353\\
373.01	0.00398947692853466\\
374.01	0.0039936398799023\\
375.01	0.00399797619969865\\
376.01	0.00400249050601793\\
377.01	0.00400718533887115\\
378.01	0.0040120601956959\\
379.01	0.00401711023818462\\
380.01	0.00402229707812446\\
381.01	0.00402758779734362\\
382.01	0.00403298441935207\\
383.01	0.00403848901199207\\
384.01	0.00404410368126233\\
385.01	0.00404983057187891\\
386.01	0.00405567186783042\\
387.01	0.0040616297929237\\
388.01	0.00406770661131813\\
389.01	0.00407390462804669\\
390.01	0.00408022618951885\\
391.01	0.00408667368400339\\
392.01	0.00409324954208565\\
393.01	0.00409995623709576\\
394.01	0.00410679628550188\\
395.01	0.0041137722472635\\
396.01	0.00412088672613654\\
397.01	0.00412814236992487\\
398.01	0.00413554187066803\\
399.01	0.00414308796475642\\
400.01	0.00415078343296359\\
401.01	0.00415863110038205\\
402.01	0.00416663383625075\\
403.01	0.00417479455365705\\
404.01	0.00418311620909673\\
405.01	0.00419160180187202\\
406.01	0.00420025437330558\\
407.01	0.0042090770057451\\
408.01	0.00421807282133005\\
409.01	0.00422724498048821\\
410.01	0.00423659668012834\\
411.01	0.00424613115148417\\
412.01	0.00425585165756716\\
413.01	0.00426576149017367\\
414.01	0.00427586396639013\\
415.01	0.00428616242452759\\
416.01	0.00429666021941308\\
417.01	0.00430736071695087\\
418.01	0.00431826728786092\\
419.01	0.00432938330048456\\
420.01	0.00434071211253735\\
421.01	0.00435225706167172\\
422.01	0.00436402145469421\\
423.01	0.00437600855526338\\
424.01	0.00438822156987256\\
425.01	0.00440066363189556\\
426.01	0.00441333778344674\\
427.01	0.00442624695477689\\
428.01	0.00443939394089102\\
429.01	0.00445278137503738\\
430.01	0.00446641169867595\\
431.01	0.00448028712748935\\
432.01	0.00449440961295119\\
433.01	0.00450878079891455\\
434.01	0.00452340197262956\\
435.01	0.00453827400954186\\
436.01	0.00455339731116888\\
437.01	0.00456877173529571\\
438.01	0.00458439651768355\\
439.01	0.00460027018444709\\
440.01	0.00461639045423427\\
441.01	0.00463275412934851\\
442.01	0.00464935697499815\\
443.01	0.00466619358595628\\
444.01	0.00468325724009715\\
445.01	0.00470053973856403\\
446.01	0.00471803123276715\\
447.01	0.00473572003906111\\
448.01	0.00475359244288599\\
449.01	0.0047716324954655\\
450.01	0.00478982180798224\\
451.01	0.00480813935064878\\
452.01	0.00482656126750973\\
453.01	0.00484506072241868\\
454.01	0.00486360779785242\\
455.01	0.00488216947654946\\
456.01	0.00490070974708109\\
457.01	0.00491918988926615\\
458.01	0.00493756901500489\\
459.01	0.00495580496616116\\
460.01	0.00497385570559329\\
461.01	0.00499168138295437\\
462.01	0.00500924731669823\\
463.01	0.00502652820708054\\
464.01	0.0050435138947469\\
465.01	0.00506021730225566\\
466.01	0.00507679117790721\\
467.01	0.00509351189056388\\
468.01	0.00511038239909231\\
469.01	0.00512739407767977\\
470.01	0.00514453748828496\\
471.01	0.00516180235492443\\
472.01	0.00517917754523689\\
473.01	0.00519665106255998\\
474.01	0.00521421005259325\\
475.01	0.00523184082933595\\
476.01	0.0052495289236258\\
477.01	0.00526725915335834\\
478.01	0.00528501571822812\\
479.01	0.00530278232776542\\
480.01	0.00532054236939011\\
481.01	0.00533827912325906\\
482.01	0.00535597603111575\\
483.01	0.00537361702653526\\
484.01	0.00539118693371128\\
485.01	0.00540867194101191\\
486.01	0.00542606015360278\\
487.01	0.00544334222601611\\
488.01	0.00546051206996545\\
489.01	0.00547756762406072\\
490.01	0.00549451165910218\\
491.01	0.00551135257363967\\
492.01	0.00552810510700925\\
493.01	0.0055447908580748\\
494.01	0.00556143844260698\\
495.01	0.005578083045178\\
496.01	0.00559476501949797\\
497.01	0.00561152704485662\\
498.01	0.00562840221187111\\
499.01	0.00564539569062671\\
500.01	0.0056625077707454\\
501.01	0.00567973968740582\\
502.01	0.00569709375686213\\
503.01	0.00571457351239865\\
504.01	0.00573218383614435\\
505.01	0.00574993108078917\\
506.01	0.00576782317370089\\
507.01	0.00578586969427461\\
508.01	0.00580408191360225\\
509.01	0.00582247278388372\\
510.01	0.00584105686366401\\
511.01	0.00585985016435045\\
512.01	0.00587886990413575\\
513.01	0.0058981341583032\\
514.01	0.00591766140122166\\
515.01	0.00593746994702705\\
516.01	0.00595757731575038\\
517.01	0.00597799958337351\\
518.01	0.00599875082345979\\
519.01	0.00601984291558474\\
520.01	0.00604128680950413\\
521.01	0.00606309389102688\\
522.01	0.00608527608585403\\
523.01	0.00610784581405038\\
524.01	0.00613081593090994\\
525.01	0.00615419965464391\\
526.01	0.00617801048227137\\
527.01	0.00620226209636413\\
528.01	0.00622696826687023\\
529.01	0.006252142754069\\
530.01	0.00627779922067693\\
531.01	0.0063039511629907\\
532.01	0.00633061187228706\\
533.01	0.00635779443775675\\
534.01	0.00638551179984718\\
535.01	0.00641377685612876\\
536.01	0.00644260260165693\\
537.01	0.00647200221739751\\
538.01	0.00650198906876377\\
539.01	0.00653257668666739\\
540.01	0.00656377875008843\\
541.01	0.00659560907141205\\
542.01	0.0066280815855853\\
543.01	0.00666121034404836\\
544.01	0.00669500951413639\\
545.01	0.00672949338414971\\
546.01	0.00676467637354784\\
547.01	0.00680057304676548\\
548.01	0.00683719812803159\\
549.01	0.00687456651353929\\
550.01	0.00691269327682495\\
551.01	0.00695159366529483\\
552.01	0.00699128309192888\\
553.01	0.00703177712610352\\
554.01	0.00707309148393236\\
555.01	0.00711524201785119\\
556.01	0.00715824470504815\\
557.01	0.00720211563422044\\
558.01	0.0072468709900294\\
559.01	0.0072925270345521\\
560.01	0.00733910008501349\\
561.01	0.00738660648715271\\
562.01	0.00743506258375307\\
563.01	0.00748448467812454\\
564.01	0.0075348889925328\\
565.01	0.00758629162138968\\
566.01	0.00763870847868745\\
567.01	0.00769215523903229\\
568.01	0.00774664727157911\\
569.01	0.00780219956613101\\
570.01	0.00785882665063752\\
571.01	0.00791654249931098\\
572.01	0.00797536043057803\\
573.01	0.00803529299409064\\
574.01	0.0080963518460355\\
575.01	0.00815854761199431\\
576.01	0.00822188973662608\\
577.01	0.00828638631949463\\
578.01	0.00835204393648558\\
579.01	0.00841886744645257\\
580.01	0.00848685978301172\\
581.01	0.00855602173179519\\
582.01	0.00862635169400304\\
583.01	0.00869784543780366\\
584.01	0.00877049584006834\\
585.01	0.00884429262215488\\
586.01	0.00891922208505276\\
587.01	0.00899526685127645\\
588.01	0.00907240562356496\\
589.01	0.00915061297387856\\
590.01	0.00922985918056657\\
591.01	0.00931011013716259\\
592.01	0.00939132736335054\\
593.01	0.00947346815761936\\
594.01	0.00955648594246635\\
595.01	0.00964033086731408\\
596.01	0.0097249507523202\\
597.01	0.00981029247891265\\
598.01	0.00989630396133466\\
599.01	0.00996919203046332\\
599.02	0.00996973314335002\\
599.03	0.00997027095716909\\
599.04	0.00997080543920308\\
599.05	0.00997133655641355\\
599.06	0.00997186427543785\\
599.07	0.00997238856258599\\
599.08	0.00997290938383737\\
599.09	0.00997342670483753\\
599.1	0.00997394049089489\\
599.11	0.00997445070697736\\
599.12	0.00997495731770904\\
599.13	0.00997546028736683\\
599.14	0.00997595957987695\\
599.15	0.00997645515881154\\
599.16	0.00997694698738516\\
599.17	0.00997743502845122\\
599.18	0.00997791924449848\\
599.19	0.0099783995976474\\
599.2	0.00997887604964655\\
599.21	0.00997934856186892\\
599.22	0.00997981709530823\\
599.23	0.00998028161057515\\
599.24	0.0099807420678936\\
599.25	0.00998119842709685\\
599.26	0.00998165064762374\\
599.27	0.00998209868851473\\
599.28	0.00998254250840802\\
599.29	0.00998298206553556\\
599.3	0.00998341731771902\\
599.31	0.00998384822236582\\
599.32	0.00998427473646495\\
599.33	0.0099846968165829\\
599.34	0.00998511441885951\\
599.35	0.00998552749900369\\
599.36	0.00998593601228924\\
599.37	0.00998633991355054\\
599.38	0.0099867391571782\\
599.39	0.00998713369711469\\
599.4	0.00998752348684992\\
599.41	0.00998790847811156\\
599.42	0.0099882886197449\\
599.43	0.00998866386008803\\
599.44	0.00998903414696681\\
599.45	0.00998939942768985\\
599.46	0.00998975964904344\\
599.47	0.00999011475728636\\
599.48	0.00999046469814471\\
599.49	0.00999080941680669\\
599.5	0.00999114885791725\\
599.51	0.00999148296557278\\
599.52	0.0099918116833157\\
599.53	0.00999213495412901\\
599.54	0.00999245272043077\\
599.55	0.00999276492406855\\
599.56	0.00999307150631381\\
599.57	0.00999337240785624\\
599.58	0.00999366756879799\\
599.59	0.00999395692864795\\
599.6	0.00999424042631585\\
599.61	0.00999451800010642\\
599.62	0.00999478958771336\\
599.63	0.0099950551262134\\
599.64	0.00999531455206019\\
599.65	0.00999556780107816\\
599.66	0.00999581480845634\\
599.67	0.00999605550874211\\
599.68	0.00999628983583487\\
599.69	0.00999651772297967\\
599.7	0.00999673910276076\\
599.71	0.00999695390709509\\
599.72	0.00999716206722577\\
599.73	0.00999736351371538\\
599.74	0.00999755817643933\\
599.75	0.00999774598457904\\
599.76	0.00999792686661517\\
599.77	0.00999810075032068\\
599.78	0.00999826756275386\\
599.79	0.00999842723025133\\
599.8	0.00999857967842092\\
599.81	0.0099987248321345\\
599.82	0.00999886261552071\\
599.83	0.00999899295195769\\
599.84	0.00999911576406567\\
599.85	0.00999923097369951\\
599.86	0.00999933850194117\\
599.87	0.00999943826909211\\
599.88	0.00999953019466558\\
599.89	0.00999961419737891\\
599.9	0.00999969019514566\\
599.91	0.00999975810506767\\
599.92	0.00999981784342713\\
599.93	0.00999986932567848\\
599.94	0.00999991246644028\\
599.95	0.00999994717948697\\
599.96	0.00999997337774056\\
599.97	0.00999999097326228\\
599.98	0.00999999987724406\\
599.99	0.01\\
600	0.01\\
};
\addplot [color=blue!40!mycolor9,solid,forget plot]
  table[row sep=crcr]{%
0.01	0.00367405832583603\\
1.01	0.00367405912102549\\
2.01	0.00367405993263262\\
3.01	0.00367406076099672\\
4.01	0.00367406160646402\\
5.01	0.00367406246938859\\
6.01	0.00367406335013067\\
7.01	0.00367406424905903\\
8.01	0.00367406516654958\\
9.01	0.00367406610298597\\
10.01	0.00367406705875989\\
11.01	0.00367406803427119\\
12.01	0.00367406902992784\\
13.01	0.00367407004614641\\
14.01	0.00367407108335191\\
15.01	0.00367407214197835\\
16.01	0.00367407322246871\\
17.01	0.00367407432527495\\
18.01	0.0036740754508585\\
19.01	0.00367407659969039\\
20.01	0.00367407777225132\\
21.01	0.00367407896903182\\
22.01	0.00367408019053285\\
23.01	0.00367408143726555\\
24.01	0.00367408270975178\\
25.01	0.00367408400852423\\
26.01	0.00367408533412621\\
27.01	0.00367408668711273\\
28.01	0.00367408806805\\
29.01	0.00367408947751607\\
30.01	0.00367409091610117\\
31.01	0.00367409238440748\\
32.01	0.00367409388304958\\
33.01	0.00367409541265495\\
34.01	0.00367409697386401\\
35.01	0.00367409856733053\\
36.01	0.0036741001937215\\
37.01	0.00367410185371803\\
38.01	0.00367410354801519\\
39.01	0.00367410527732256\\
40.01	0.00367410704236429\\
41.01	0.00367410884387957\\
42.01	0.00367411068262281\\
43.01	0.00367411255936412\\
44.01	0.00367411447488969\\
45.01	0.00367411643000158\\
46.01	0.00367411842551899\\
47.01	0.00367412046227759\\
48.01	0.00367412254113066\\
49.01	0.00367412466294911\\
50.01	0.00367412682862167\\
51.01	0.0036741290390556\\
52.01	0.0036741312951771\\
53.01	0.00367413359793124\\
54.01	0.00367413594828284\\
55.01	0.00367413834721683\\
56.01	0.00367414079573823\\
57.01	0.00367414329487306\\
58.01	0.00367414584566859\\
59.01	0.00367414844919367\\
60.01	0.00367415110653928\\
61.01	0.0036741538188189\\
62.01	0.00367415658716955\\
63.01	0.00367415941275123\\
64.01	0.00367416229674802\\
65.01	0.00367416524036898\\
66.01	0.0036741682448477\\
67.01	0.0036741713114434\\
68.01	0.00367417444144151\\
69.01	0.00367417763615423\\
70.01	0.00367418089692056\\
71.01	0.00367418422510739\\
72.01	0.00367418762210989\\
73.01	0.00367419108935223\\
74.01	0.00367419462828794\\
75.01	0.00367419824040066\\
76.01	0.0036742019272047\\
77.01	0.00367420569024587\\
78.01	0.00367420953110184\\
79.01	0.00367421345138316\\
80.01	0.00367421745273361\\
81.01	0.00367422153683078\\
82.01	0.00367422570538732\\
83.01	0.00367422996015113\\
84.01	0.00367423430290657\\
85.01	0.00367423873547459\\
86.01	0.00367424325971415\\
87.01	0.00367424787752242\\
88.01	0.00367425259083611\\
89.01	0.00367425740163188\\
90.01	0.00367426231192744\\
91.01	0.00367426732378213\\
92.01	0.00367427243929801\\
93.01	0.0036742776606208\\
94.01	0.00367428298994044\\
95.01	0.00367428842949226\\
96.01	0.00367429398155808\\
97.01	0.00367429964846672\\
98.01	0.00367430543259524\\
99.01	0.0036743113363702\\
100.01	0.00367431736226802\\
101.01	0.00367432351281664\\
102.01	0.0036743297905963\\
103.01	0.00367433619824085\\
104.01	0.00367434273843835\\
105.01	0.00367434941393292\\
106.01	0.00367435622752513\\
107.01	0.00367436318207376\\
108.01	0.00367437028049679\\
109.01	0.00367437752577268\\
110.01	0.00367438492094152\\
111.01	0.00367439246910635\\
112.01	0.00367440017343455\\
113.01	0.00367440803715885\\
114.01	0.00367441606357934\\
115.01	0.00367442425606429\\
116.01	0.00367443261805184\\
117.01	0.00367444115305116\\
118.01	0.00367444986464413\\
119.01	0.0036744587564871\\
120.01	0.00367446783231197\\
121.01	0.00367447709592808\\
122.01	0.00367448655122356\\
123.01	0.00367449620216714\\
124.01	0.00367450605281003\\
125.01	0.00367451610728699\\
126.01	0.00367452636981899\\
127.01	0.00367453684471392\\
128.01	0.0036745475363691\\
129.01	0.00367455844927339\\
130.01	0.00367456958800837\\
131.01	0.00367458095725051\\
132.01	0.00367459256177353\\
133.01	0.00367460440644998\\
134.01	0.00367461649625347\\
135.01	0.00367462883626063\\
136.01	0.00367464143165344\\
137.01	0.00367465428772142\\
138.01	0.0036746674098637\\
139.01	0.00367468080359164\\
140.01	0.00367469447453068\\
141.01	0.00367470842842302\\
142.01	0.00367472267113002\\
143.01	0.00367473720863488\\
144.01	0.00367475204704466\\
145.01	0.00367476719259334\\
146.01	0.00367478265164431\\
147.01	0.0036747984306931\\
148.01	0.00367481453637007\\
149.01	0.00367483097544331\\
150.01	0.00367484775482131\\
151.01	0.00367486488155649\\
152.01	0.00367488236284745\\
153.01	0.0036749002060424\\
154.01	0.0036749184186423\\
155.01	0.00367493700830412\\
156.01	0.00367495598284417\\
157.01	0.00367497535024085\\
158.01	0.00367499511863893\\
159.01	0.00367501529635234\\
160.01	0.00367503589186823\\
161.01	0.00367505691385027\\
162.01	0.00367507837114245\\
163.01	0.0036751002727728\\
164.01	0.00367512262795758\\
165.01	0.003675145446105\\
166.01	0.00367516873681918\\
167.01	0.0036751925099046\\
168.01	0.00367521677536986\\
169.01	0.00367524154343279\\
170.01	0.00367526682452371\\
171.01	0.00367529262929117\\
172.01	0.00367531896860572\\
173.01	0.00367534585356497\\
174.01	0.00367537329549816\\
175.01	0.00367540130597163\\
176.01	0.00367542989679323\\
177.01	0.00367545908001773\\
178.01	0.00367548886795219\\
179.01	0.00367551927316123\\
180.01	0.00367555030847222\\
181.01	0.00367558198698174\\
182.01	0.00367561432206047\\
183.01	0.00367564732735948\\
184.01	0.00367568101681631\\
185.01	0.00367571540466079\\
186.01	0.00367575050542167\\
187.01	0.00367578633393296\\
188.01	0.00367582290534023\\
189.01	0.00367586023510806\\
190.01	0.00367589833902585\\
191.01	0.00367593723321591\\
192.01	0.00367597693414015\\
193.01	0.00367601745860749\\
194.01	0.00367605882378159\\
195.01	0.00367610104718842\\
196.01	0.00367614414672417\\
197.01	0.00367618814066356\\
198.01	0.00367623304766794\\
199.01	0.00367627888679362\\
200.01	0.00367632567750083\\
201.01	0.00367637343966273\\
202.01	0.00367642219357428\\
203.01	0.0036764719599615\\
204.01	0.00367652275999129\\
205.01	0.00367657461528113\\
206.01	0.00367662754790893\\
207.01	0.00367668158042372\\
208.01	0.00367673673585571\\
209.01	0.00367679303772729\\
210.01	0.00367685051006431\\
211.01	0.00367690917740703\\
212.01	0.00367696906482211\\
213.01	0.00367703019791429\\
214.01	0.00367709260283866\\
215.01	0.00367715630631309\\
216.01	0.00367722133563136\\
217.01	0.00367728771867616\\
218.01	0.00367735548393265\\
219.01	0.00367742466050223\\
220.01	0.00367749527811696\\
221.01	0.00367756736715395\\
222.01	0.00367764095865095\\
223.01	0.00367771608432077\\
224.01	0.00367779277656807\\
225.01	0.00367787106850474\\
226.01	0.00367795099396734\\
227.01	0.00367803258753387\\
228.01	0.00367811588454109\\
229.01	0.00367820092110299\\
230.01	0.00367828773412955\\
231.01	0.00367837636134524\\
232.01	0.00367846684130912\\
233.01	0.00367855921343473\\
234.01	0.00367865351801115\\
235.01	0.00367874979622426\\
236.01	0.0036788480901784\\
237.01	0.00367894844291927\\
238.01	0.00367905089845701\\
239.01	0.0036791555017904\\
240.01	0.00367926229893116\\
241.01	0.0036793713369296\\
242.01	0.00367948266390042\\
243.01	0.00367959632904982\\
244.01	0.00367971238270315\\
245.01	0.00367983087633339\\
246.01	0.00367995186259069\\
247.01	0.00368007539533263\\
248.01	0.00368020152965538\\
249.01	0.00368033032192623\\
250.01	0.00368046182981627\\
251.01	0.00368059611233512\\
252.01	0.00368073322986634\\
253.01	0.00368087324420359\\
254.01	0.0036810162185886\\
255.01	0.00368116221774978\\
256.01	0.00368131130794246\\
257.01	0.00368146355699048\\
258.01	0.00368161903432902\\
259.01	0.00368177781104863\\
260.01	0.00368193995994104\\
261.01	0.00368210555554632\\
262.01	0.00368227467420189\\
263.01	0.00368244739409282\\
264.01	0.00368262379530425\\
265.01	0.00368280395987521\\
266.01	0.00368298797185448\\
267.01	0.00368317591735849\\
268.01	0.00368336788463104\\
269.01	0.00368356396410541\\
270.01	0.00368376424846817\\
271.01	0.00368396883272569\\
272.01	0.00368417781427288\\
273.01	0.00368439129296444\\
274.01	0.00368460937118845\\
275.01	0.00368483215394291\\
276.01	0.003685059748915\\
277.01	0.00368529226656313\\
278.01	0.00368552982020205\\
279.01	0.00368577252609094\\
280.01	0.00368602050352533\\
281.01	0.00368627387493163\\
282.01	0.00368653276596563\\
283.01	0.003686797305615\\
284.01	0.00368706762630486\\
285.01	0.00368734386400814\\
286.01	0.00368762615835948\\
287.01	0.00368791465277396\\
288.01	0.00368820949456992\\
289.01	0.00368851083509693\\
290.01	0.00368881882986823\\
291.01	0.00368913363869844\\
292.01	0.003689455425847\\
293.01	0.00368978436016636\\
294.01	0.00369012061525605\\
295.01	0.00369046436962391\\
296.01	0.00369081580685172\\
297.01	0.00369117511576859\\
298.01	0.00369154249063074\\
299.01	0.00369191813130812\\
300.01	0.00369230224347841\\
301.01	0.00369269503882901\\
302.01	0.00369309673526556\\
303.01	0.00369350755713052\\
304.01	0.00369392773542858\\
305.01	0.00369435750806205\\
306.01	0.00369479712007419\\
307.01	0.00369524682390303\\
308.01	0.00369570687964417\\
309.01	0.00369617755532457\\
310.01	0.00369665912718591\\
311.01	0.00369715187997963\\
312.01	0.00369765610727224\\
313.01	0.0036981721117632\\
314.01	0.00369870020561382\\
315.01	0.00369924071078917\\
316.01	0.003699793959412\\
317.01	0.00370036029413032\\
318.01	0.0037009400684974\\
319.01	0.00370153364736611\\
320.01	0.0037021414072972\\
321.01	0.00370276373698146\\
322.01	0.00370340103767629\\
323.01	0.00370405372365683\\
324.01	0.00370472222268241\\
325.01	0.00370540697647703\\
326.01	0.00370610844122576\\
327.01	0.00370682708808467\\
328.01	0.00370756340370593\\
329.01	0.00370831789077751\\
330.01	0.00370909106857538\\
331.01	0.00370988347352987\\
332.01	0.00371069565980338\\
333.01	0.00371152819987965\\
334.01	0.0037123816851611\\
335.01	0.00371325672657484\\
336.01	0.00371415395518282\\
337.01	0.00371507402279454\\
338.01	0.00371601760257771\\
339.01	0.00371698538966353\\
340.01	0.00371797810174047\\
341.01	0.0037189964796306\\
342.01	0.00372004128784083\\
343.01	0.00372111331508027\\
344.01	0.00372221337473191\\
345.01	0.00372334230526716\\
346.01	0.00372450097058653\\
347.01	0.0037256902602693\\
348.01	0.00372691108970988\\
349.01	0.00372816440011696\\
350.01	0.00372945115834382\\
351.01	0.00373077235651685\\
352.01	0.00373212901142059\\
353.01	0.00373352216359135\\
354.01	0.00373495287606578\\
355.01	0.0037364222327198\\
356.01	0.00373793133612502\\
357.01	0.00373948130483995\\
358.01	0.00374107327004145\\
359.01	0.00374270837139101\\
360.01	0.00374438775202041\\
361.01	0.00374611255250908\\
362.01	0.00374788390372034\\
363.01	0.00374970291835895\\
364.01	0.00375157068111615\\
365.01	0.00375348823728412\\
366.01	0.00375545657975034\\
367.01	0.00375747663434159\\
368.01	0.00375954924357592\\
369.01	0.0037616751490212\\
370.01	0.00376385497267603\\
371.01	0.003766089198096\\
372.01	0.00376837815244509\\
373.01	0.00377072199129273\\
374.01	0.00377312068888773\\
375.01	0.00377557403791379\\
376.01	0.00377808166449803\\
377.01	0.00378064306669181\\
378.01	0.00378325768800604\\
379.01	0.00378592504219442\\
380.01	0.00378864505725178\\
381.01	0.00379141857738417\\
382.01	0.00379424667125867\\
383.01	0.00379713043234045\\
384.01	0.00380007097979201\\
385.01	0.00380306945942864\\
386.01	0.00380612704473182\\
387.01	0.00380924493792692\\
388.01	0.00381242437112994\\
389.01	0.00381566660756871\\
390.01	0.00381897294288394\\
391.01	0.00382234470651839\\
392.01	0.0038257832631996\\
393.01	0.00382929001452471\\
394.01	0.00383286640065623\\
395.01	0.00383651390213738\\
396.01	0.00384023404183742\\
397.01	0.0038440283870385\\
398.01	0.00384789855167558\\
399.01	0.00385184619874275\\
400.01	0.00385587304288089\\
401.01	0.00385998085316258\\
402.01	0.00386417145609101\\
403.01	0.00386844673883276\\
404.01	0.00387280865270531\\
405.01	0.00387725921694236\\
406.01	0.00388180052276163\\
407.01	0.00388643473776389\\
408.01	0.00389116411069319\\
409.01	0.00389599097659257\\
410.01	0.00390091776239113\\
411.01	0.00390594699296457\\
412.01	0.00391108129771316\\
413.01	0.0039163234177077\\
414.01	0.00392167621345645\\
415.01	0.00392714267335492\\
416.01	0.00393272592288359\\
417.01	0.0039384292346282\\
418.01	0.00394425603920165\\
419.01	0.00395020993715844\\
420.01	0.00395629471199882\\
421.01	0.00396251434437153\\
422.01	0.00396887302759433\\
423.01	0.00397537518462468\\
424.01	0.00398202548662442\\
425.01	0.00398882887327795\\
426.01	0.00399579057503845\\
427.01	0.00400291613749372\\
428.01	0.00401021144805891\\
429.01	0.00401768276522525\\
430.01	0.00402533675060855\\
431.01	0.00403318050406516\\
432.01	0.00404122160215854\\
433.01	0.00404946814027847\\
434.01	0.00405792877873196\\
435.01	0.00406661279313371\\
436.01	0.00407553012943204\\
437.01	0.0040846914639002\\
438.01	0.0040941082684095\\
439.01	0.00410379288126253\\
440.01	0.00411375858380633\\
441.01	0.00412401968295002\\
442.01	0.00413459159956747\\
443.01	0.00414549096256252\\
444.01	0.00415673570807933\\
445.01	0.00416834518293846\\
446.01	0.00418034025082298\\
447.01	0.00419274339898484\\
448.01	0.00420557884222234\\
449.01	0.00421887261951838\\
450.01	0.00423265267690792\\
451.01	0.00424694892772715\\
452.01	0.00426179327819508\\
453.01	0.00427721960205298\\
454.01	0.00429326364241243\\
455.01	0.00430996281163371\\
456.01	0.00432735585042339\\
457.01	0.00434548229469846\\
458.01	0.00436438168218982\\
459.01	0.00438409240903823\\
460.01	0.00440465011819025\\
461.01	0.00442608546416438\\
462.01	0.00444842105000842\\
463.01	0.00447166726820816\\
464.01	0.00449581668842845\\
465.01	0.00452083653810631\\
466.01	0.00454655444585295\\
467.01	0.00457268258452102\\
468.01	0.00459920934856123\\
469.01	0.00462613306159242\\
470.01	0.00465345098640395\\
471.01	0.0046811591223869\\
472.01	0.0047092519385362\\
473.01	0.00473772202655891\\
474.01	0.00476655965321753\\
475.01	0.00479575222489502\\
476.01	0.00482528394100766\\
477.01	0.00485513585982052\\
478.01	0.00488528566086753\\
479.01	0.0049157072618335\\
480.01	0.00494637041136256\\
481.01	0.00497724026453776\\
482.01	0.0050082769508859\\
483.01	0.00503943514996553\\
484.01	0.00507066369672383\\
485.01	0.00510190524850224\\
486.01	0.0051330960586651\\
487.01	0.00516416591944975\\
488.01	0.00519503836025188\\
489.01	0.00522563121910394\\
490.01	0.00525585774707416\\
491.01	0.00528562846197784\\
492.01	0.00531485405192902\\
493.01	0.00534344970529017\\
494.01	0.00537134142467356\\
495.01	0.00539847497274243\\
496.01	0.00542482833968094\\
497.01	0.00545042992074787\\
498.01	0.00547574300765019\\
499.01	0.00550109153729324\\
500.01	0.00552645060857317\\
501.01	0.00555179473582924\\
502.01	0.00557709816769944\\
503.01	0.00560233528928952\\
504.01	0.00562748112029061\\
505.01	0.00565251192163366\\
506.01	0.00567740592120446\\
507.01	0.00570214416678325\\
508.01	0.00572671150974177\\
509.01	0.0057510977150463\\
510.01	0.00577529868053916\\
511.01	0.00579931772940393\\
512.01	0.00582316691157837\\
513.01	0.0058468682091317\\
514.01	0.00587045448251295\\
515.01	0.00589396991263982\\
516.01	0.00591746957905275\\
517.01	0.00594101765388125\\
518.01	0.00596468346858382\\
519.01	0.00598852627082647\\
520.01	0.00601256288541119\\
521.01	0.00603680034606036\\
522.01	0.00606124780133964\\
523.01	0.00608591668244098\\
524.01	0.00611082082606813\\
525.01	0.00613597653886037\\
526.01	0.00616140258431862\\
527.01	0.00618712007040617\\
528.01	0.00621315221465982\\
529.01	0.00623952396483905\\
530.01	0.00626626145807431\\
531.01	0.00629339131242648\\
532.01	0.00632093976509019\\
533.01	0.00634893170585903\\
534.01	0.00637738970975967\\
535.01	0.00640633326085757\\
536.01	0.00643577895491486\\
537.01	0.00646574287119041\\
538.01	0.00649624165212671\\
539.01	0.00652729244507207\\
540.01	0.0065589127943315\\
541.01	0.0065911205186732\\
542.01	0.00662393357819652\\
543.01	0.00665736993624814\\
544.01	0.00669144742622071\\
545.01	0.00672618363750042\\
546.01	0.00676159583667395\\
547.01	0.00679770094139254\\
548.01	0.00683451556229975\\
549.01	0.00687205612044749\\
550.01	0.00691033902574197\\
551.01	0.00694938081321948\\
552.01	0.00698919814651896\\
553.01	0.00702980778368401\\
554.01	0.0070712265449191\\
555.01	0.00711347128465088\\
556.01	0.00715655886917984\\
557.01	0.00720050616072122\\
558.01	0.0072453300078212\\
559.01	0.00729104724095515\\
560.01	0.00733767467060229\\
561.01	0.00738522908340952\\
562.01	0.00743372723060418\\
563.01	0.00748318580258283\\
564.01	0.00753362138874788\\
565.01	0.00758505042934002\\
566.01	0.00763748916253181\\
567.01	0.00769095356644288\\
568.01	0.00774545929526543\\
569.01	0.00780102160845901\\
570.01	0.0078576552917938\\
571.01	0.00791537456890428\\
572.01	0.00797419300200297\\
573.01	0.00803412338054319\\
574.01	0.00809517759694714\\
575.01	0.00815736650899904\\
576.01	0.00822069978879914\\
577.01	0.00828518575791731\\
578.01	0.00835083120818599\\
579.01	0.00841764120771106\\
580.01	0.00848561889195915\\
581.01	0.00855476524018159\\
582.01	0.00862507883799715\\
583.01	0.00869655562770745\\
584.01	0.00876918864890651\\
585.01	0.0088429677732069\\
586.01	0.00891787943850125\\
587.01	0.00899390639018665\\
588.01	0.00907102743936527\\
589.01	0.00914921725139712\\
590.01	0.00922844618250661\\
591.01	0.0093086801876622\\
592.01	0.00938988082996722\\
593.01	0.00947200543070414\\
594.01	0.00955500741044542\\
595.01	0.00963883688587809\\
596.01	0.00972344160493188\\
597.01	0.00980876832537324\\
598.01	0.00989476477037853\\
599.01	0.00996919203040513\\
599.02	0.00996973314331371\\
599.03	0.00997027095714612\\
599.04	0.00997080543918755\\
599.05	0.00997133655640176\\
599.06	0.00997186427542786\\
599.07	0.00997238856257705\\
599.08	0.00997290938382928\\
599.09	0.00997342670483015\\
599.1	0.00997394049088811\\
599.11	0.0099744507069711\\
599.12	0.00997495731770325\\
599.13	0.00997546028736146\\
599.14	0.00997595957987197\\
599.15	0.00997645515880693\\
599.16	0.00997694698738088\\
599.17	0.00997743502844726\\
599.18	0.00997791924449481\\
599.19	0.00997839959764401\\
599.2	0.00997887604964342\\
599.21	0.00997934856186603\\
599.22	0.00997981709530556\\
599.23	0.0099802816105727\\
599.24	0.00998074206789134\\
599.25	0.00998119842709478\\
599.26	0.00998165064762184\\
599.27	0.00998209868851299\\
599.28	0.00998254250840643\\
599.29	0.00998298206553411\\
599.3	0.0099834173177177\\
599.31	0.00998384822236462\\
599.32	0.00998427473646385\\
599.33	0.00998469681658191\\
599.34	0.00998511441885861\\
599.35	0.00998552749900288\\
599.36	0.00998593601228852\\
599.37	0.00998633991354989\\
599.38	0.00998673915717762\\
599.39	0.00998713369711416\\
599.4	0.00998752348684945\\
599.41	0.00998790847811114\\
599.42	0.00998828861974453\\
599.43	0.0099886638600877\\
599.44	0.00998903414696652\\
599.45	0.0099893994276896\\
599.46	0.00998975964904322\\
599.47	0.00999011475728616\\
599.48	0.00999046469814455\\
599.49	0.00999080941680655\\
599.5	0.00999114885791712\\
599.51	0.00999148296557267\\
599.52	0.00999181168331561\\
599.53	0.00999213495412893\\
599.54	0.0099924527204307\\
599.55	0.00999276492406849\\
599.56	0.00999307150631377\\
599.57	0.0099933724078562\\
599.58	0.00999366756879796\\
599.59	0.00999395692864792\\
599.6	0.00999424042631583\\
599.61	0.0099945180001064\\
599.62	0.00999478958771334\\
599.63	0.00999505512621339\\
599.64	0.00999531455206018\\
599.65	0.00999556780107815\\
599.66	0.00999581480845634\\
599.67	0.00999605550874211\\
599.68	0.00999628983583487\\
599.69	0.00999651772297967\\
599.7	0.00999673910276075\\
599.71	0.00999695390709509\\
599.72	0.00999716206722577\\
599.73	0.00999736351371538\\
599.74	0.00999755817643933\\
599.75	0.00999774598457904\\
599.76	0.00999792686661517\\
599.77	0.00999810075032068\\
599.78	0.00999826756275386\\
599.79	0.00999842723025133\\
599.8	0.00999857967842092\\
599.81	0.0099987248321345\\
599.82	0.00999886261552071\\
599.83	0.00999899295195769\\
599.84	0.00999911576406567\\
599.85	0.00999923097369951\\
599.86	0.00999933850194117\\
599.87	0.00999943826909211\\
599.88	0.00999953019466558\\
599.89	0.00999961419737891\\
599.9	0.00999969019514566\\
599.91	0.00999975810506767\\
599.92	0.00999981784342713\\
599.93	0.00999986932567848\\
599.94	0.00999991246644028\\
599.95	0.00999994717948697\\
599.96	0.00999997337774056\\
599.97	0.00999999097326228\\
599.98	0.00999999987724406\\
599.99	0.01\\
600	0.01\\
};
\addplot [color=blue!75!mycolor7,solid,forget plot]
  table[row sep=crcr]{%
0.01	0.00346027296821295\\
1.01	0.00346027362420681\\
2.01	0.00346027429381217\\
3.01	0.0034602749773122\\
4.01	0.00346027567499618\\
5.01	0.0034602763871589\\
6.01	0.00346027711410207\\
7.01	0.00346027785613308\\
8.01	0.00346027861356599\\
9.01	0.00346027938672151\\
10.01	0.00346028017592695\\
11.01	0.00346028098151639\\
12.01	0.00346028180383087\\
13.01	0.00346028264321883\\
14.01	0.00346028350003563\\
15.01	0.00346028437464441\\
16.01	0.00346028526741551\\
17.01	0.00346028617872732\\
18.01	0.00346028710896605\\
19.01	0.0034602880585259\\
20.01	0.00346028902780944\\
21.01	0.00346029001722768\\
22.01	0.00346029102720004\\
23.01	0.00346029205815479\\
24.01	0.00346029311052924\\
25.01	0.00346029418476959\\
26.01	0.00346029528133186\\
27.01	0.00346029640068117\\
28.01	0.00346029754329262\\
29.01	0.00346029870965113\\
30.01	0.00346029990025172\\
31.01	0.00346030111559985\\
32.01	0.00346030235621163\\
33.01	0.00346030362261389\\
34.01	0.00346030491534453\\
35.01	0.00346030623495253\\
36.01	0.00346030758199867\\
37.01	0.00346030895705529\\
38.01	0.00346031036070667\\
39.01	0.00346031179354944\\
40.01	0.00346031325619285\\
41.01	0.00346031474925859\\
42.01	0.00346031627338167\\
43.01	0.00346031782921026\\
44.01	0.00346031941740603\\
45.01	0.00346032103864478\\
46.01	0.00346032269361626\\
47.01	0.00346032438302477\\
48.01	0.00346032610758935\\
49.01	0.00346032786804411\\
50.01	0.0034603296651386\\
51.01	0.00346033149963806\\
52.01	0.00346033337232378\\
53.01	0.00346033528399348\\
54.01	0.00346033723546163\\
55.01	0.00346033922755966\\
56.01	0.00346034126113662\\
57.01	0.00346034333705923\\
58.01	0.00346034545621254\\
59.01	0.00346034761950002\\
60.01	0.00346034982784436\\
61.01	0.00346035208218756\\
62.01	0.0034603543834911\\
63.01	0.00346035673273698\\
64.01	0.00346035913092776\\
65.01	0.00346036157908705\\
66.01	0.00346036407826\\
67.01	0.00346036662951375\\
68.01	0.00346036923393772\\
69.01	0.00346037189264434\\
70.01	0.00346037460676944\\
71.01	0.00346037737747276\\
72.01	0.00346038020593843\\
73.01	0.00346038309337537\\
74.01	0.00346038604101807\\
75.01	0.00346038905012697\\
76.01	0.00346039212198898\\
77.01	0.00346039525791822\\
78.01	0.00346039845925639\\
79.01	0.00346040172737334\\
80.01	0.00346040506366795\\
81.01	0.00346040846956863\\
82.01	0.0034604119465335\\
83.01	0.00346041549605182\\
84.01	0.00346041911964391\\
85.01	0.00346042281886236\\
86.01	0.00346042659529237\\
87.01	0.00346043045055275\\
88.01	0.00346043438629623\\
89.01	0.00346043840421047\\
90.01	0.00346044250601864\\
91.01	0.00346044669348052\\
92.01	0.00346045096839281\\
93.01	0.00346045533259015\\
94.01	0.00346045978794601\\
95.01	0.00346046433637333\\
96.01	0.00346046897982541\\
97.01	0.00346047372029699\\
98.01	0.0034604785598249\\
99.01	0.0034604835004888\\
100.01	0.00346048854441262\\
101.01	0.003460493693765\\
102.01	0.00346049895076038\\
103.01	0.00346050431766014\\
104.01	0.00346050979677362\\
105.01	0.00346051539045875\\
106.01	0.00346052110112362\\
107.01	0.0034605269312271\\
108.01	0.00346053288328019\\
109.01	0.00346053895984705\\
110.01	0.00346054516354626\\
111.01	0.00346055149705174\\
112.01	0.00346055796309411\\
113.01	0.00346056456446209\\
114.01	0.00346057130400326\\
115.01	0.00346057818462554\\
116.01	0.00346058520929884\\
117.01	0.00346059238105594\\
118.01	0.00346059970299396\\
119.01	0.00346060717827577\\
120.01	0.00346061481013159\\
121.01	0.00346062260185992\\
122.01	0.0034606305568297\\
123.01	0.00346063867848137\\
124.01	0.0034606469703283\\
125.01	0.00346065543595889\\
126.01	0.00346066407903767\\
127.01	0.00346067290330737\\
128.01	0.00346068191259023\\
129.01	0.0034606911107898\\
130.01	0.00346070050189307\\
131.01	0.00346071008997172\\
132.01	0.00346071987918425\\
133.01	0.00346072987377806\\
134.01	0.00346074007809076\\
135.01	0.00346075049655294\\
136.01	0.00346076113368944\\
137.01	0.00346077199412187\\
138.01	0.00346078308257053\\
139.01	0.00346079440385637\\
140.01	0.00346080596290387\\
141.01	0.00346081776474244\\
142.01	0.00346082981450914\\
143.01	0.00346084211745108\\
144.01	0.00346085467892764\\
145.01	0.00346086750441305\\
146.01	0.00346088059949883\\
147.01	0.00346089396989629\\
148.01	0.00346090762143927\\
149.01	0.0034609215600868\\
150.01	0.00346093579192595\\
151.01	0.00346095032317421\\
152.01	0.00346096516018295\\
153.01	0.00346098030944\\
154.01	0.00346099577757264\\
155.01	0.00346101157135083\\
156.01	0.00346102769769017\\
157.01	0.00346104416365536\\
158.01	0.0034610609764632\\
159.01	0.00346107814348626\\
160.01	0.003461095672256\\
161.01	0.00346111357046652\\
162.01	0.0034611318459782\\
163.01	0.00346115050682108\\
164.01	0.00346116956119898\\
165.01	0.00346118901749312\\
166.01	0.00346120888426622\\
167.01	0.00346122917026641\\
168.01	0.0034612498844316\\
169.01	0.00346127103589311\\
170.01	0.00346129263398104\\
171.01	0.00346131468822759\\
172.01	0.00346133720837241\\
173.01	0.00346136020436684\\
174.01	0.00346138368637866\\
175.01	0.00346140766479729\\
176.01	0.00346143215023823\\
177.01	0.00346145715354887\\
178.01	0.00346148268581301\\
179.01	0.00346150875835664\\
180.01	0.00346153538275346\\
181.01	0.00346156257082997\\
182.01	0.00346159033467176\\
183.01	0.00346161868662928\\
184.01	0.00346164763932366\\
185.01	0.00346167720565309\\
186.01	0.00346170739879893\\
187.01	0.00346173823223225\\
188.01	0.00346176971972081\\
189.01	0.00346180187533525\\
190.01	0.0034618347134566\\
191.01	0.00346186824878295\\
192.01	0.00346190249633714\\
193.01	0.00346193747147391\\
194.01	0.00346197318988783\\
195.01	0.00346200966762107\\
196.01	0.00346204692107138\\
197.01	0.00346208496700048\\
198.01	0.00346212382254231\\
199.01	0.00346216350521213\\
200.01	0.00346220403291492\\
201.01	0.00346224542395469\\
202.01	0.00346228769704397\\
203.01	0.00346233087131317\\
204.01	0.00346237496632039\\
205.01	0.00346242000206184\\
206.01	0.00346246599898185\\
207.01	0.00346251297798326\\
208.01	0.00346256096043894\\
209.01	0.00346260996820243\\
210.01	0.00346266002361927\\
211.01	0.00346271114953928\\
212.01	0.00346276336932792\\
213.01	0.00346281670687917\\
214.01	0.00346287118662797\\
215.01	0.00346292683356303\\
216.01	0.00346298367324049\\
217.01	0.00346304173179739\\
218.01	0.00346310103596568\\
219.01	0.00346316161308691\\
220.01	0.0034632234911264\\
221.01	0.00346328669868928\\
222.01	0.00346335126503503\\
223.01	0.00346341722009433\\
224.01	0.00346348459448502\\
225.01	0.00346355341952927\\
226.01	0.00346362372727028\\
227.01	0.00346369555049068\\
228.01	0.0034637689227305\\
229.01	0.00346384387830587\\
230.01	0.00346392045232814\\
231.01	0.00346399868072374\\
232.01	0.00346407860025449\\
233.01	0.00346416024853842\\
234.01	0.00346424366407096\\
235.01	0.00346432888624702\\
236.01	0.00346441595538365\\
237.01	0.00346450491274311\\
238.01	0.00346459580055654\\
239.01	0.00346468866204866\\
240.01	0.00346478354146264\\
241.01	0.00346488048408608\\
242.01	0.0034649795362774\\
243.01	0.00346508074549308\\
244.01	0.00346518416031558\\
245.01	0.00346528983048211\\
246.01	0.00346539780691429\\
247.01	0.00346550814174827\\
248.01	0.00346562088836583\\
249.01	0.00346573610142654\\
250.01	0.00346585383690075\\
251.01	0.00346597415210312\\
252.01	0.0034660971057273\\
253.01	0.0034662227578818\\
254.01	0.00346635117012642\\
255.01	0.00346648240550983\\
256.01	0.00346661652860854\\
257.01	0.00346675360556607\\
258.01	0.00346689370413411\\
259.01	0.00346703689371411\\
260.01	0.00346718324540039\\
261.01	0.0034673328320244\\
262.01	0.00346748572819977\\
263.01	0.00346764201036888\\
264.01	0.00346780175685081\\
265.01	0.00346796504789028\\
266.01	0.00346813196570774\\
267.01	0.00346830259455144\\
268.01	0.00346847702075015\\
269.01	0.00346865533276736\\
270.01	0.00346883762125744\\
271.01	0.00346902397912266\\
272.01	0.00346921450157171\\
273.01	0.00346940928617991\\
274.01	0.00346960843295105\\
275.01	0.00346981204438045\\
276.01	0.00347002022551944\\
277.01	0.00347023308404214\\
278.01	0.00347045073031277\\
279.01	0.00347067327745584\\
280.01	0.00347090084142684\\
281.01	0.00347113354108534\\
282.01	0.0034713714982694\\
283.01	0.00347161483787171\\
284.01	0.00347186368791751\\
285.01	0.0034721181796441\\
286.01	0.00347237844758251\\
287.01	0.00347264462964009\\
288.01	0.00347291686718541\\
289.01	0.00347319530513482\\
290.01	0.00347348009204033\\
291.01	0.00347377138017986\\
292.01	0.00347406932564839\\
293.01	0.0034743740884513\\
294.01	0.00347468583259949\\
295.01	0.00347500472620522\\
296.01	0.00347533094158058\\
297.01	0.00347566465533702\\
298.01	0.0034760060484862\\
299.01	0.00347635530654243\\
300.01	0.00347671261962672\\
301.01	0.00347707818257161\\
302.01	0.00347745219502794\\
303.01	0.00347783486157195\\
304.01	0.00347822639181405\\
305.01	0.00347862700050824\\
306.01	0.00347903690766271\\
307.01	0.00347945633865043\\
308.01	0.00347988552432155\\
309.01	0.00348032470111508\\
310.01	0.00348077411117149\\
311.01	0.00348123400244509\\
312.01	0.00348170462881735\\
313.01	0.00348218625020862\\
314.01	0.00348267913269065\\
315.01	0.00348318354859799\\
316.01	0.00348369977663887\\
317.01	0.00348422810200456\\
318.01	0.00348476881647847\\
319.01	0.00348532221854259\\
320.01	0.00348588861348323\\
321.01	0.00348646831349399\\
322.01	0.00348706163777694\\
323.01	0.00348766891264135\\
324.01	0.00348829047159897\\
325.01	0.00348892665545671\\
326.01	0.00348957781240554\\
327.01	0.00349024429810557\\
328.01	0.00349092647576742\\
329.01	0.00349162471622844\\
330.01	0.00349233939802451\\
331.01	0.00349307090745658\\
332.01	0.00349381963865134\\
333.01	0.00349458599361593\\
334.01	0.00349537038228661\\
335.01	0.00349617322257001\\
336.01	0.00349699494037756\\
337.01	0.00349783596965168\\
338.01	0.00349869675238413\\
339.01	0.00349957773862557\\
340.01	0.00350047938648575\\
341.01	0.00350140216212408\\
342.01	0.00350234653972973\\
343.01	0.00350331300149104\\
344.01	0.00350430203755365\\
345.01	0.00350531414596637\\
346.01	0.003506349832615\\
347.01	0.00350740961114285\\
348.01	0.00350849400285846\\
349.01	0.0035096035366293\\
350.01	0.00351073874876223\\
351.01	0.00351190018287018\\
352.01	0.0035130883897261\\
353.01	0.00351430392710535\\
354.01	0.0035155473596173\\
355.01	0.00351681925852976\\
356.01	0.00351812020158951\\
357.01	0.0035194507728439\\
358.01	0.00352081156247081\\
359.01	0.00352220316662644\\
360.01	0.00352362618732319\\
361.01	0.00352508123235327\\
362.01	0.00352656891527919\\
363.01	0.0035280898555162\\
364.01	0.00352964467853894\\
365.01	0.00353123401625111\\
366.01	0.00353285850756602\\
367.01	0.00353451879925318\\
368.01	0.00353621554711559\\
369.01	0.00353794941757132\\
370.01	0.003539721089716\\
371.01	0.00354153125794724\\
372.01	0.00354338063522171\\
373.01	0.00354526995699678\\
374.01	0.0035471999858638\\
375.01	0.00354917151680359\\
376.01	0.0035511853828645\\
377.01	0.00355324246085807\\
378.01	0.0035553436763458\\
379.01	0.00355749000667126\\
380.01	0.00355968247959745\\
381.01	0.00356192216159083\\
382.01	0.00356421014844351\\
383.01	0.00356654756533187\\
384.01	0.0035689355677666\\
385.01	0.00357137534257635\\
386.01	0.00357386810892851\\
387.01	0.00357641511938682\\
388.01	0.00357901766100757\\
389.01	0.00358167705647555\\
390.01	0.00358439466528152\\
391.01	0.00358717188494148\\
392.01	0.00359001015226043\\
393.01	0.00359291094464055\\
394.01	0.00359587578143637\\
395.01	0.00359890622535709\\
396.01	0.00360200388391832\\
397.01	0.00360517041094378\\
398.01	0.00360840750811844\\
399.01	0.00361171692659451\\
400.01	0.00361510046865083\\
401.01	0.00361855998940695\\
402.01	0.00362209739859295\\
403.01	0.00362571466237528\\
404.01	0.00362941380523974\\
405.01	0.00363319691193145\\
406.01	0.0036370661294524\\
407.01	0.00364102366911618\\
408.01	0.00364507180865957\\
409.01	0.00364921289440978\\
410.01	0.00365344934350697\\
411.01	0.00365778364617858\\
412.01	0.00366221836806465\\
413.01	0.00366675615258946\\
414.01	0.00367139972337641\\
415.01	0.00367615188669986\\
416.01	0.00368101553396871\\
417.01	0.00368599364423231\\
418.01	0.00369108928670067\\
419.01	0.00369630562326653\\
420.01	0.00370164591101524\\
421.01	0.00370711350470643\\
422.01	0.00371271185920789\\
423.01	0.00371844453185686\\
424.01	0.00372431518472269\\
425.01	0.00373032758673646\\
426.01	0.00373648561564981\\
427.01	0.00374279325977601\\
428.01	0.00374925461946055\\
429.01	0.00375587390821697\\
430.01	0.00376265545345388\\
431.01	0.00376960369670651\\
432.01	0.00377672319327018\\
433.01	0.00378401861111711\\
434.01	0.00379149472895806\\
435.01	0.00379915643328761\\
436.01	0.00380700871422563\\
437.01	0.00381505665993977\\
438.01	0.00382330544939905\\
439.01	0.00383176034317328\\
440.01	0.00384042667195154\\
441.01	0.00384930982241018\\
442.01	0.00385841522001223\\
443.01	0.00386774830827336\\
444.01	0.00387731452398092\\
445.01	0.00388711926780829\\
446.01	0.00389716786973208\\
447.01	0.00390746554863938\\
448.01	0.00391801736552\\
449.01	0.00392882816968527\\
450.01	0.00393990253756635\\
451.01	0.00395124470384521\\
452.01	0.00396285848500437\\
453.01	0.0039747471959013\\
454.01	0.00398691356075675\\
455.01	0.00399935962109656\\
456.01	0.00401208664484539\\
457.01	0.0040250950431314\\
458.01	0.00403838430467857\\
459.01	0.00405195296229161\\
460.01	0.00406579861235158\\
461.01	0.00407991801718984\\
462.01	0.00409430733516124\\
463.01	0.00410896262159766\\
464.01	0.00412388169678813\\
465.01	0.0041390635464146\\
466.01	0.00415450916369327\\
467.01	0.00417022703877926\\
468.01	0.00418623020788076\\
469.01	0.00420253300612251\\
470.01	0.00421915334458593\\
471.01	0.00423611450437305\\
472.01	0.00425344692776813\\
473.01	0.00427119058098699\\
474.01	0.0042893980666307\\
475.01	0.00430813595353832\\
476.01	0.00432745418754948\\
477.01	0.00434738642238951\\
478.01	0.00436796854694536\\
479.01	0.00438923923852827\\
480.01	0.00441124012149316\\
481.01	0.00443401590478815\\
482.01	0.00445761448463721\\
483.01	0.00448208699358519\\
484.01	0.00450748777056427\\
485.01	0.00453387421800094\\
486.01	0.00456130650062177\\
487.01	0.00458984702570931\\
488.01	0.0046195596249929\\
489.01	0.00465050833221185\\
490.01	0.00468275559761284\\
491.01	0.00471635938155702\\
492.01	0.0047513690821035\\
493.01	0.00478782101251431\\
494.01	0.0048257319804129\\
495.01	0.00486509058339598\\
496.01	0.00490584537563689\\
497.01	0.00494788813973441\\
498.01	0.00499067706248237\\
499.01	0.00503382368061261\\
500.01	0.00507728600037356\\
501.01	0.00512101589736558\\
502.01	0.00516495853105634\\
503.01	0.00520905175454348\\
504.01	0.00525322552146334\\
505.01	0.00529740134029413\\
506.01	0.00534149182733723\\
507.01	0.00538540041562128\\
508.01	0.00542902130718709\\
509.01	0.005472239789007\\
510.01	0.00551493307597316\\
511.01	0.00555697190167869\\
512.01	0.00559822315355755\\
513.01	0.00563855394918638\\
514.01	0.0056778376830278\\
515.01	0.00571596274960215\\
516.01	0.00575284489717085\\
517.01	0.00578844451789892\\
518.01	0.00582279122548661\\
519.01	0.00585647215374869\\
520.01	0.00589007821401159\\
521.01	0.00592357682065511\\
522.01	0.00595693559919004\\
523.01	0.00599012481542081\\
524.01	0.00602311876244421\\
525.01	0.00605589728395156\\
526.01	0.00608844743812646\\
527.01	0.00612076532603694\\
528.01	0.00615285802945342\\
529.01	0.00618474555417383\\
530.01	0.00621646260793039\\
531.01	0.00624805994354959\\
532.01	0.00627960485971661\\
533.01	0.00631118026364552\\
534.01	0.00634288143230977\\
535.01	0.00637480875100246\\
536.01	0.00640702911071041\\
537.01	0.00643956106745507\\
538.01	0.00647242193593152\\
539.01	0.00650563211117126\\
540.01	0.00653921506614063\\
541.01	0.00657319725365189\\
542.01	0.00660760790790732\\
543.01	0.00664247870742235\\
544.01	0.00667784317785525\\
545.01	0.00671373582867647\\
546.01	0.0067501910560746\\
547.01	0.00678724184788092\\
548.01	0.00682491841768709\\
549.01	0.00686324700395446\\
550.01	0.00690224961801737\\
551.01	0.00694194683434955\\
552.01	0.00698235970957576\\
553.01	0.00702350973072825\\
554.01	0.00706541862943193\\
555.01	0.00710810817972993\\
556.01	0.00715159999085044\\
557.01	0.00719591531061176\\
558.01	0.00724107485952969\\
559.01	0.00728709871898849\\
560.01	0.00733400629729445\\
561.01	0.00738181639206545\\
562.01	0.00743054735256967\\
563.01	0.00748021729887154\\
564.01	0.00753084421884847\\
565.01	0.00758244591252368\\
566.01	0.00763503991220713\\
567.01	0.00768864340185235\\
568.01	0.00774327313825653\\
569.01	0.00779894537555497\\
570.01	0.00785567579287152\\
571.01	0.00791347942325609\\
572.01	0.00797237057980226\\
573.01	0.00803236277225495\\
574.01	0.00809346860516428\\
575.01	0.00815569964939458\\
576.01	0.00821906628915308\\
577.01	0.00828357755347728\\
578.01	0.0083492409343018\\
579.01	0.00841606219068275\\
580.01	0.00848404513879882\\
581.01	0.00855319142743901\\
582.01	0.00862350029902764\\
583.01	0.00869496833695104\\
584.01	0.00876758920113668\\
585.01	0.00884135335563897\\
586.01	0.00891624779447431\\
587.01	0.00899225577465261\\
588.01	0.00906935656751973\\
589.01	0.00914752524228344\\
590.01	0.00922673249973243\\
591.01	0.00930694457949549\\
592.01	0.00938812327098054\\
593.01	0.00947022606676473\\
594.01	0.00955320650816933\\
595.01	0.00963701478667503\\
596.01	0.00972159868250433\\
597.01	0.00980690494409953\\
598.01	0.00989288124056736\\
599.01	0.00996919201787552\\
599.02	0.00996973313515437\\
599.03	0.00997027095217347\\
599.04	0.00997080543638652\\
599.05	0.00997133655493895\\
599.06	0.00997186427466602\\
599.07	0.00997238856209093\\
599.08	0.00997290938342301\\
599.09	0.00997342670448527\\
599.1	0.00997394049058981\\
599.11	0.00997445070670796\\
599.12	0.00997495731746685\\
599.13	0.00997546028714593\\
599.14	0.00997595957967355\\
599.15	0.00997645515862339\\
599.16	0.00997694698721088\\
599.17	0.00997743502828965\\
599.18	0.0099779192443486\\
599.19	0.00997839959750833\\
599.2	0.0099788760495175\\
599.21	0.00997934856174921\\
599.22	0.00997981709519721\\
599.23	0.00998028161047228\\
599.24	0.00998074206779834\\
599.25	0.00998119842700871\\
599.26	0.00998165064754226\\
599.27	0.00998209868843949\\
599.28	0.00998254250833861\\
599.29	0.00998298206547159\\
599.3	0.00998341731766015\\
599.31	0.00998384822231171\\
599.32	0.00998427473641528\\
599.33	0.00998469681653738\\
599.34	0.00998511441881785\\
599.35	0.00998552749896564\\
599.36	0.00998593601225455\\
599.37	0.00998633991351897\\
599.38	0.00998673915714952\\
599.39	0.00998713369708868\\
599.4	0.0099875234868264\\
599.41	0.00998790847809034\\
599.42	0.0099882886197258\\
599.43	0.00998866386007088\\
599.44	0.00998903414695145\\
599.45	0.00998939942767614\\
599.46	0.00998975964903123\\
599.47	0.00999011475727552\\
599.48	0.00999046469813513\\
599.49	0.00999080941679823\\
599.5	0.00999114885790982\\
599.51	0.00999148296556627\\
599.52	0.00999181168331002\\
599.53	0.00999213495412407\\
599.54	0.0099924527204265\\
599.55	0.00999276492406487\\
599.56	0.00999307150631066\\
599.57	0.00999337240785355\\
599.58	0.00999366756879571\\
599.59	0.00999395692864602\\
599.6	0.00999424042631423\\
599.61	0.00999451800010506\\
599.62	0.00999478958771224\\
599.63	0.00999505512621248\\
599.64	0.00999531455205943\\
599.65	0.00999556780107754\\
599.66	0.00999581480845585\\
599.67	0.00999605550874172\\
599.68	0.00999628983583456\\
599.69	0.00999651772297942\\
599.7	0.00999673910276056\\
599.71	0.00999695390709495\\
599.72	0.00999716206722566\\
599.73	0.0099973635137153\\
599.74	0.00999755817643927\\
599.75	0.009997745984579\\
599.76	0.00999792686661514\\
599.77	0.00999810075032065\\
599.78	0.00999826756275384\\
599.79	0.00999842723025132\\
599.8	0.00999857967842091\\
599.81	0.00999872483213449\\
599.82	0.0099988626155207\\
599.83	0.00999899295195769\\
599.84	0.00999911576406567\\
599.85	0.00999923097369951\\
599.86	0.00999933850194117\\
599.87	0.00999943826909211\\
599.88	0.00999953019466558\\
599.89	0.00999961419737891\\
599.9	0.00999969019514566\\
599.91	0.00999975810506767\\
599.92	0.00999981784342713\\
599.93	0.00999986932567848\\
599.94	0.00999991246644028\\
599.95	0.00999994717948697\\
599.96	0.00999997337774056\\
599.97	0.00999999097326228\\
599.98	0.00999999987724406\\
599.99	0.01\\
600	0.01\\
};
\addplot [color=blue!80!mycolor9,solid,forget plot]
  table[row sep=crcr]{%
0.01	0.00266457163794791\\
1.01	0.0026645724869681\\
2.01	0.00266457335376111\\
3.01	0.00266457423870125\\
4.01	0.00266457514217061\\
5.01	0.00266457606455949\\
6.01	0.00266457700626637\\
7.01	0.0026645779676983\\
8.01	0.00266457894927092\\
9.01	0.0026645799514086\\
10.01	0.00266458097454493\\
11.01	0.00266458201912257\\
12.01	0.00266458308559365\\
13.01	0.00266458417441982\\
14.01	0.00266458528607275\\
15.01	0.00266458642103385\\
16.01	0.00266458757979494\\
17.01	0.0026645887628583\\
18.01	0.00266458997073694\\
19.01	0.0026645912039547\\
20.01	0.00266459246304654\\
21.01	0.0026645937485589\\
22.01	0.00266459506104984\\
23.01	0.00266459640108933\\
24.01	0.00266459776925942\\
25.01	0.00266459916615464\\
26.01	0.00266460059238206\\
27.01	0.00266460204856192\\
28.01	0.00266460353532749\\
29.01	0.00266460505332568\\
30.01	0.00266460660321709\\
31.01	0.00266460818567651\\
32.01	0.00266460980139311\\
33.01	0.00266461145107075\\
34.01	0.00266461313542835\\
35.01	0.00266461485520027\\
36.01	0.00266461661113639\\
37.01	0.00266461840400285\\
38.01	0.00266462023458196\\
39.01	0.0026646221036729\\
40.01	0.0026646240120918\\
41.01	0.00266462596067241\\
42.01	0.00266462795026632\\
43.01	0.00266462998174324\\
44.01	0.0026646320559917\\
45.01	0.00266463417391909\\
46.01	0.00266463633645225\\
47.01	0.00266463854453793\\
48.01	0.00266464079914327\\
49.01	0.00266464310125601\\
50.01	0.00266464545188521\\
51.01	0.00266464785206155\\
52.01	0.00266465030283769\\
53.01	0.00266465280528916\\
54.01	0.00266465536051439\\
55.01	0.0026646579696355\\
56.01	0.00266466063379865\\
57.01	0.00266466335417471\\
58.01	0.0026646661319597\\
59.01	0.00266466896837545\\
60.01	0.00266467186467\\
61.01	0.00266467482211827\\
62.01	0.00266467784202264\\
63.01	0.00266468092571354\\
64.01	0.00266468407455017\\
65.01	0.00266468728992068\\
66.01	0.0026646905732436\\
67.01	0.00266469392596776\\
68.01	0.00266469734957341\\
69.01	0.00266470084557262\\
70.01	0.00266470441551018\\
71.01	0.00266470806096418\\
72.01	0.0026647117835469\\
73.01	0.00266471558490531\\
74.01	0.00266471946672216\\
75.01	0.00266472343071654\\
76.01	0.00266472747864462\\
77.01	0.00266473161230064\\
78.01	0.00266473583351774\\
79.01	0.00266474014416871\\
80.01	0.00266474454616691\\
81.01	0.00266474904146714\\
82.01	0.00266475363206663\\
83.01	0.00266475832000586\\
84.01	0.00266476310736967\\
85.01	0.00266476799628808\\
86.01	0.00266477298893737\\
87.01	0.00266477808754098\\
88.01	0.00266478329437085\\
89.01	0.00266478861174824\\
90.01	0.00266479404204483\\
91.01	0.00266479958768383\\
92.01	0.00266480525114132\\
93.01	0.00266481103494726\\
94.01	0.00266481694168662\\
95.01	0.00266482297400069\\
96.01	0.00266482913458842\\
97.01	0.00266483542620755\\
98.01	0.00266484185167591\\
99.01	0.00266484841387294\\
100.01	0.00266485511574084\\
101.01	0.00266486196028608\\
102.01	0.00266486895058091\\
103.01	0.0026648760897645\\
104.01	0.00266488338104495\\
105.01	0.00266489082770029\\
106.01	0.00266489843308033\\
107.01	0.00266490620060843\\
108.01	0.0026649141337827\\
109.01	0.0026649222361779\\
110.01	0.00266493051144728\\
111.01	0.00266493896332422\\
112.01	0.0026649475956238\\
113.01	0.00266495641224506\\
114.01	0.00266496541717257\\
115.01	0.00266497461447854\\
116.01	0.00266498400832446\\
117.01	0.00266499360296354\\
118.01	0.00266500340274252\\
119.01	0.00266501341210373\\
120.01	0.00266502363558734\\
121.01	0.00266503407783369\\
122.01	0.00266504474358513\\
123.01	0.00266505563768876\\
124.01	0.0026650667650986\\
125.01	0.00266507813087803\\
126.01	0.00266508974020212\\
127.01	0.00266510159836031\\
128.01	0.00266511371075902\\
129.01	0.00266512608292429\\
130.01	0.00266513872050413\\
131.01	0.00266515162927195\\
132.01	0.00266516481512874\\
133.01	0.00266517828410628\\
134.01	0.00266519204237022\\
135.01	0.00266520609622276\\
136.01	0.00266522045210615\\
137.01	0.0026652351166055\\
138.01	0.00266525009645228\\
139.01	0.00266526539852762\\
140.01	0.00266528102986538\\
141.01	0.00266529699765627\\
142.01	0.00266531330925086\\
143.01	0.00266532997216343\\
144.01	0.00266534699407573\\
145.01	0.00266536438284079\\
146.01	0.00266538214648667\\
147.01	0.00266540029322075\\
148.01	0.00266541883143365\\
149.01	0.00266543776970332\\
150.01	0.00266545711679963\\
151.01	0.00266547688168856\\
152.01	0.00266549707353662\\
153.01	0.00266551770171561\\
154.01	0.00266553877580733\\
155.01	0.00266556030560825\\
156.01	0.00266558230113468\\
157.01	0.00266560477262746\\
158.01	0.00266562773055756\\
159.01	0.00266565118563111\\
160.01	0.00266567514879483\\
161.01	0.00266569963124173\\
162.01	0.00266572464441652\\
163.01	0.00266575020002176\\
164.01	0.00266577631002354\\
165.01	0.0026658029866576\\
166.01	0.00266583024243576\\
167.01	0.0026658580901521\\
168.01	0.00266588654288963\\
169.01	0.0026659156140269\\
170.01	0.00266594531724485\\
171.01	0.00266597566653391\\
172.01	0.00266600667620102\\
173.01	0.00266603836087716\\
174.01	0.00266607073552483\\
175.01	0.00266610381544564\\
176.01	0.00266613761628846\\
177.01	0.00266617215405712\\
178.01	0.002666207445119\\
179.01	0.00266624350621338\\
180.01	0.00266628035446018\\
181.01	0.00266631800736872\\
182.01	0.00266635648284699\\
183.01	0.00266639579921062\\
184.01	0.00266643597519291\\
185.01	0.00266647702995419\\
186.01	0.002666518983092\\
187.01	0.0026665618546513\\
188.01	0.00266660566513491\\
189.01	0.00266665043551419\\
190.01	0.00266669618724028\\
191.01	0.00266674294225503\\
192.01	0.00266679072300276\\
193.01	0.00266683955244196\\
194.01	0.00266688945405739\\
195.01	0.00266694045187248\\
196.01	0.00266699257046196\\
197.01	0.0026670458349648\\
198.01	0.00266710027109773\\
199.01	0.00266715590516838\\
200.01	0.00266721276408991\\
201.01	0.00266727087539453\\
202.01	0.00266733026724861\\
203.01	0.00266739096846752\\
204.01	0.00266745300853088\\
205.01	0.00266751641759831\\
206.01	0.00266758122652549\\
207.01	0.00266764746688065\\
208.01	0.00266771517096133\\
209.01	0.00266778437181175\\
210.01	0.00266785510324043\\
211.01	0.00266792739983816\\
212.01	0.00266800129699683\\
213.01	0.00266807683092807\\
214.01	0.00266815403868285\\
215.01	0.00266823295817135\\
216.01	0.00266831362818346\\
217.01	0.00266839608840921\\
218.01	0.00266848037946069\\
219.01	0.00266856654289356\\
220.01	0.00266865462122943\\
221.01	0.00266874465797903\\
222.01	0.00266883669766536\\
223.01	0.00266893078584811\\
224.01	0.00266902696914791\\
225.01	0.00266912529527159\\
226.01	0.0026692258130382\\
227.01	0.00266932857240515\\
228.01	0.00266943362449527\\
229.01	0.00266954102162476\\
230.01	0.00266965081733107\\
231.01	0.00266976306640226\\
232.01	0.00266987782490644\\
233.01	0.00266999515022235\\
234.01	0.00267011510107036\\
235.01	0.00267023773754418\\
236.01	0.00267036312114371\\
237.01	0.00267049131480807\\
238.01	0.00267062238294997\\
239.01	0.0026707563914905\\
240.01	0.00267089340789491\\
241.01	0.00267103350120903\\
242.01	0.00267117674209699\\
243.01	0.00267132320287911\\
244.01	0.0026714729575714\\
245.01	0.00267162608192548\\
246.01	0.00267178265346955\\
247.01	0.00267194275155039\\
248.01	0.00267210645737619\\
249.01	0.0026722738540605\\
250.01	0.002672445026667\\
251.01	0.00267262006225551\\
252.01	0.00267279904992885\\
253.01	0.00267298208088099\\
254.01	0.00267316924844579\\
255.01	0.00267336064814749\\
256.01	0.00267355637775195\\
257.01	0.00267375653731905\\
258.01	0.00267396122925628\\
259.01	0.00267417055837368\\
260.01	0.00267438463193963\\
261.01	0.00267460355973817\\
262.01	0.0026748274541278\\
263.01	0.00267505643010083\\
264.01	0.00267529060534465\\
265.01	0.00267553010030422\\
266.01	0.00267577503824559\\
267.01	0.00267602554532119\\
268.01	0.00267628175063627\\
269.01	0.00267654378631698\\
270.01	0.0026768117875793\\
271.01	0.00267708589280041\\
272.01	0.00267736624359056\\
273.01	0.00267765298486741\\
274.01	0.00267794626493075\\
275.01	0.00267824623553987\\
276.01	0.00267855305199202\\
277.01	0.00267886687320239\\
278.01	0.00267918786178601\\
279.01	0.00267951618414114\\
280.01	0.00267985201053431\\
281.01	0.00268019551518725\\
282.01	0.00268054687636552\\
283.01	0.00268090627646873\\
284.01	0.00268127390212285\\
285.01	0.0026816499442741\\
286.01	0.00268203459828478\\
287.01	0.00268242806403111\\
288.01	0.00268283054600282\\
289.01	0.00268324225340482\\
290.01	0.00268366340026081\\
291.01	0.00268409420551895\\
292.01	0.00268453489315959\\
293.01	0.00268498569230494\\
294.01	0.00268544683733134\\
295.01	0.00268591856798311\\
296.01	0.00268640112948912\\
297.01	0.00268689477268145\\
298.01	0.00268739975411637\\
299.01	0.00268791633619792\\
300.01	0.00268844478730357\\
301.01	0.00268898538191259\\
302.01	0.00268953840073737\\
303.01	0.00269010413085666\\
304.01	0.00269068286585223\\
305.01	0.00269127490594787\\
306.01	0.0026918805581516\\
307.01	0.00269250013640087\\
308.01	0.00269313396171067\\
309.01	0.00269378236232487\\
310.01	0.00269444567387148\\
311.01	0.00269512423952074\\
312.01	0.00269581841014684\\
313.01	0.00269652854449395\\
314.01	0.00269725500934594\\
315.01	0.00269799817969983\\
316.01	0.00269875843894426\\
317.01	0.00269953617904219\\
318.01	0.0027003318007179\\
319.01	0.00270114571364983\\
320.01	0.00270197833666783\\
321.01	0.00270283009795648\\
322.01	0.00270370143526415\\
323.01	0.00270459279611743\\
324.01	0.0027055046380432\\
325.01	0.0027064374287965\\
326.01	0.00270739164659582\\
327.01	0.00270836778036614\\
328.01	0.00270936632998942\\
329.01	0.00271038780656391\\
330.01	0.00271143273267192\\
331.01	0.00271250164265704\\
332.01	0.0027135950829113\\
333.01	0.002714713612172\\
334.01	0.00271585780182998\\
335.01	0.00271702823624914\\
336.01	0.00271822551309752\\
337.01	0.00271945024369141\\
338.01	0.00272070305335274\\
339.01	0.0027219845817795\\
340.01	0.00272329548343155\\
341.01	0.00272463642793109\\
342.01	0.00272600810047945\\
343.01	0.00272741120228998\\
344.01	0.00272884645103871\\
345.01	0.00273031458133332\\
346.01	0.00273181634520065\\
347.01	0.0027333525125948\\
348.01	0.00273492387192541\\
349.01	0.00273653123060816\\
350.01	0.00273817541563807\\
351.01	0.00273985727418651\\
352.01	0.00274157767422373\\
353.01	0.00274333750516745\\
354.01	0.00274513767855961\\
355.01	0.00274697912877253\\
356.01	0.00274886281374583\\
357.01	0.00275078971575658\\
358.01	0.00275276084222435\\
359.01	0.00275477722655294\\
360.01	0.00275683992901134\\
361.01	0.00275895003765677\\
362.01	0.00276110866930049\\
363.01	0.00276331697052061\\
364.01	0.00276557611872193\\
365.01	0.00276788732324527\\
366.01	0.00277025182652645\\
367.01	0.00277267090530398\\
368.01	0.00277514587187318\\
369.01	0.00277767807538037\\
370.01	0.00278026890314976\\
371.01	0.00278291978202763\\
372.01	0.00278563217972543\\
373.01	0.00278840760613281\\
374.01	0.00279124761456527\\
375.01	0.00279415380289599\\
376.01	0.00279712781451333\\
377.01	0.00280017133904137\\
378.01	0.00280328611321068\\
379.01	0.00280647392864265\\
380.01	0.00280973660924248\\
381.01	0.00281307600841563\\
382.01	0.00281649402727172\\
383.01	0.00281999261600058\\
384.01	0.00282357377511652\\
385.01	0.00282723955672774\\
386.01	0.00283099206583052\\
387.01	0.00283483346162763\\
388.01	0.00283876595887161\\
389.01	0.00284279182923128\\
390.01	0.00284691340268196\\
391.01	0.00285113306891834\\
392.01	0.00285545327878948\\
393.01	0.00285987654575489\\
394.01	0.00286440544736059\\
395.01	0.00286904262673481\\
396.01	0.00287379079410072\\
397.01	0.00287865272830584\\
398.01	0.00288363127836563\\
399.01	0.00288872936501952\\
400.01	0.00289394998229737\\
401.01	0.00289929619909338\\
402.01	0.00290477116074526\\
403.01	0.00291037809061451\\
404.01	0.00291612029166492\\
405.01	0.002922001148035\\
406.01	0.00292802412659954\\
407.01	0.00293419277851558\\
408.01	0.00294051074074617\\
409.01	0.00294698173755663\\
410.01	0.00295360958197426\\
411.01	0.00296039817720486\\
412.01	0.00296735151799532\\
413.01	0.00297447369193275\\
414.01	0.00298176888066713\\
415.01	0.00298924136104482\\
416.01	0.00299689550613684\\
417.01	0.00300473578614513\\
418.01	0.00301276676916589\\
419.01	0.00302099312178881\\
420.01	0.00302941960950472\\
421.01	0.00303805109689367\\
422.01	0.0030468925475587\\
423.01	0.00305594902376694\\
424.01	0.00306522568575372\\
425.01	0.00307472779063846\\
426.01	0.00308446069089321\\
427.01	0.00309442983229672\\
428.01	0.00310464075129446\\
429.01	0.00311509907167435\\
430.01	0.00312581050045253\\
431.01	0.00313678082284618\\
432.01	0.00314801589619097\\
433.01	0.00315952164263663\\
434.01	0.0031713040404252\\
435.01	0.00318336911352479\\
436.01	0.00319572291935036\\
437.01	0.00320837153425779\\
438.01	0.00322132103643843\\
439.01	0.00323457748577748\\
440.01	0.00324814690015546\\
441.01	0.00326203522757654\\
442.01	0.003276248313389\\
443.01	0.00329079186172066\\
444.01	0.00330567139007952\\
445.01	0.00332089217585739\\
446.01	0.00333645919321636\\
447.01	0.00335237703852002\\
448.01	0.00336864984208066\\
449.01	0.00338528116350889\\
450.01	0.00340227386735189\\
451.01	0.00341962997495885\\
452.01	0.0034373504875775\\
453.01	0.0034554351745154\\
454.01	0.00347388231872536\\
455.01	0.00349268841031794\\
456.01	0.00351184777615066\\
457.01	0.00353135213066103\\
458.01	0.0035511900293104\\
459.01	0.00357134620116009\\
460.01	0.00359180073090039\\
461.01	0.00361252805274208\\
462.01	0.00363349570902957\\
463.01	0.00365466282199656\\
464.01	0.00367597820322375\\
465.01	0.00369737782856295\\
466.01	0.00371878167771662\\
467.01	0.00374009055359666\\
468.01	0.00376133808360597\\
469.01	0.00378254449084704\\
470.01	0.00380359765481982\\
471.01	0.00382435754725771\\
472.01	0.00384465007892143\\
473.01	0.00386425891196347\\
474.01	0.00388291464302406\\
475.01	0.00390069266259455\\
476.01	0.00391881844986454\\
477.01	0.00393741767035533\\
478.01	0.00395650237704295\\
479.01	0.00397608452442232\\
480.01	0.00399617587633663\\
481.01	0.00401678789648693\\
482.01	0.00403793161924702\\
483.01	0.0040596174982915\\
484.01	0.00408185523052651\\
485.01	0.00410465355293726\\
486.01	0.00412802001034002\\
487.01	0.00415196069297844\\
488.01	0.00417647994903544\\
489.01	0.00420158020761564\\
490.01	0.00422726721482038\\
491.01	0.00425360121734022\\
492.01	0.00428062631288843\\
493.01	0.00430837070158088\\
494.01	0.00433684816062326\\
495.01	0.00436607012923727\\
496.01	0.00439604936910906\\
497.01	0.00442680147837339\\
498.01	0.00445834901184761\\
499.01	0.00449072614764581\\
500.01	0.00452397116594763\\
501.01	0.00455812478068281\\
502.01	0.00459323015236275\\
503.01	0.00462933283209169\\
504.01	0.00466648068637506\\
505.01	0.0047047241659741\\
506.01	0.0047441163056711\\
507.01	0.00478471255337544\\
508.01	0.00482657046799717\\
509.01	0.00486974924138452\\
510.01	0.00491430898617083\\
511.01	0.00496030971408751\\
512.01	0.00500780990689396\\
513.01	0.0050568645528759\\
514.01	0.00510752248267552\\
515.01	0.00515982277813339\\
516.01	0.00521378994643276\\
517.01	0.00526942495828889\\
518.01	0.00532661479683448\\
519.01	0.00538461916995028\\
520.01	0.00544272241137332\\
521.01	0.00550084980280282\\
522.01	0.00555891123659179\\
523.01	0.00561679111518276\\
524.01	0.00567435363040023\\
525.01	0.0057314487426998\\
526.01	0.0057879137030987\\
527.01	0.00584357440185067\\
528.01	0.00589824790083561\\
529.01	0.0059517466270772\\
530.01	0.00600388487332472\\
531.01	0.00605448868008126\\
532.01	0.00610341014093586\\
533.01	0.00615054761026743\\
534.01	0.00619587398233221\\
535.01	0.0062395504608343\\
536.01	0.00628272466409035\\
537.01	0.00632570687442321\\
538.01	0.0063684708154812\\
539.01	0.00641099749854468\\
540.01	0.00645327643854359\\
541.01	0.00649530646326558\\
542.01	0.0065370960188697\\
543.01	0.0065786662053737\\
544.01	0.00662005583566263\\
545.01	0.00666132366051709\\
546.01	0.00670254971458035\\
547.01	0.00674383537412062\\
548.01	0.00678530020261345\\
549.01	0.00682707436024067\\
550.01	0.00686925950890378\\
551.01	0.00691188719112618\\
552.01	0.00695498218311738\\
553.01	0.00699857348386164\\
554.01	0.00704269416314167\\
555.01	0.00708738096435513\\
556.01	0.00713267362631148\\
557.01	0.007178613887444\\
558.01	0.0072252441598414\\
559.01	0.00727260590629534\\
560.01	0.00732073783306641\\
561.01	0.00736967414027251\\
562.01	0.007419443185512\\
563.01	0.00747006821770754\\
564.01	0.00752157146830121\\
565.01	0.00757397540609622\\
566.01	0.00762730254286453\\
567.01	0.00768157517579493\\
568.01	0.00773681508916095\\
569.01	0.00779304325438374\\
570.01	0.0078502795596901\\
571.01	0.00790854260152617\\
572.01	0.007967849572551\\
573.01	0.0080282162773056\\
574.01	0.00808965727958438\\
575.01	0.00815218607059426\\
576.01	0.00821581501954256\\
577.01	0.00828055518630606\\
578.01	0.00834641611642781\\
579.01	0.00841340562488602\\
580.01	0.00848152957296922\\
581.01	0.00855079164303971\\
582.01	0.00862119311141858\\
583.01	0.00869273261663716\\
584.01	0.00876540591708722\\
585.01	0.00883920562907762\\
586.01	0.00891412093710792\\
587.01	0.00899013728670779\\
588.01	0.00906723608772262\\
589.01	0.00914539444740412\\
590.01	0.00922458495535043\\
591.01	0.00930477554784545\\
592.01	0.00938592948552506\\
593.01	0.00946800548660199\\
594.01	0.0095509580681256\\
595.01	0.00963473816035478\\
596.01	0.00971929407498429\\
597.01	0.00980457292772622\\
598.01	0.00989052264111993\\
599.01	0.0099691898561019\\
599.02	0.00996973151746593\\
599.03	0.00997026979807466\\
599.04	0.00997080466579245\\
599.05	0.00997133608806853\\
599.06	0.0099718640319399\\
599.07	0.00997238846402112\\
599.08	0.00997290935048918\\
599.09	0.00997342667879889\\
599.1	0.00997394047065652\\
599.11	0.00997445069119892\\
599.12	0.00997495730522922\\
599.13	0.0099754602772142\\
599.14	0.00997595957128175\\
599.15	0.00997645515121832\\
599.16	0.00997694698046649\\
599.17	0.00997743502212201\\
599.18	0.00997791923868709\\
599.19	0.00997839959229404\\
599.2	0.00997887604470164\\
599.21	0.00997934855729153\\
599.22	0.00997981709106457\\
599.23	0.00998028160663717\\
599.24	0.0099807420642377\\
599.25	0.00998119842370277\\
599.26	0.00998165064447353\\
599.27	0.00998209868559199\\
599.28	0.00998254250569747\\
599.29	0.00998298206302299\\
599.3	0.00998341731539121\\
599.31	0.00998384822021048\\
599.32	0.00998427473447066\\
599.33	0.00998469681473907\\
599.34	0.00998511441715626\\
599.35	0.00998552749743186\\
599.36	0.00998593601084028\\
599.37	0.00998633991221646\\
599.38	0.00998673915595153\\
599.39	0.00998713369598844\\
599.4	0.00998752348581752\\
599.41	0.00998790847716685\\
599.42	0.00998828861888209\\
599.43	0.0099886638593016\\
599.44	0.00998903414625157\\
599.45	0.00998939942704085\\
599.46	0.00998975964845599\\
599.47	0.00999011475675598\\
599.48	0.00999046469766715\\
599.49	0.0099908094163779\\
599.5	0.00999114885753339\\
599.51	0.00999148296523019\\
599.52	0.00999181168301094\\
599.53	0.00999213495385881\\
599.54	0.00999245272019206\\
599.55	0.00999276492385844\\
599.56	0.0099930715061296\\
599.57	0.00999337240769538\\
599.58	0.00999366756865814\\
599.59	0.00999395692852691\\
599.6	0.00999424042621159\\
599.61	0.00999451800001706\\
599.62	0.00999478958763719\\
599.63	0.00999505512614884\\
599.64	0.00999531455200579\\
599.65	0.00999556780103262\\
599.66	0.00999581480841847\\
599.67	0.00999605550871085\\
599.68	0.00999628983580927\\
599.69	0.00999651772295887\\
599.7	0.00999673910274401\\
599.71	0.00999695390708174\\
599.72	0.00999716206721524\\
599.73	0.00999736351370716\\
599.74	0.009997558176433\\
599.75	0.00999774598457423\\
599.76	0.00999792686661157\\
599.77	0.00999810075031802\\
599.78	0.00999826756275193\\
599.79	0.00999842723024996\\
599.8	0.00999857967841997\\
599.81	0.00999872483213385\\
599.82	0.00999886261552029\\
599.83	0.00999899295195742\\
599.84	0.00999911576406551\\
599.85	0.00999923097369942\\
599.86	0.00999933850194112\\
599.87	0.00999943826909208\\
599.88	0.00999953019466556\\
599.89	0.00999961419737891\\
599.9	0.00999969019514565\\
599.91	0.00999975810506767\\
599.92	0.00999981784342713\\
599.93	0.00999986932567848\\
599.94	0.00999991246644028\\
599.95	0.00999994717948697\\
599.96	0.00999997337774056\\
599.97	0.00999999097326229\\
599.98	0.00999999987724406\\
599.99	0.01\\
600	0.01\\
};
\addplot [color=blue,solid,forget plot]
  table[row sep=crcr]{%
0.01	0.00125393423089431\\
1.01	0.00125393505353844\\
2.01	0.00125393589347361\\
3.01	0.00125393675106585\\
4.01	0.00125393762668902\\
5.01	0.00125393852072497\\
6.01	0.00125393943356374\\
7.01	0.00125394036560372\\
8.01	0.00125394131725177\\
9.01	0.00125394228892352\\
10.01	0.00125394328104338\\
11.01	0.001253944294045\\
12.01	0.00125394532837118\\
13.01	0.00125394638447431\\
14.01	0.00125394746281636\\
15.01	0.00125394856386934\\
16.01	0.00125394968811527\\
17.01	0.0012539508360466\\
18.01	0.00125395200816618\\
19.01	0.00125395320498783\\
20.01	0.00125395442703635\\
21.01	0.00125395567484773\\
22.01	0.00125395694896955\\
23.01	0.00125395824996109\\
24.01	0.00125395957839373\\
25.01	0.00125396093485108\\
26.01	0.0012539623199293\\
27.01	0.00125396373423735\\
28.01	0.00125396517839733\\
29.01	0.00125396665304475\\
30.01	0.00125396815882872\\
31.01	0.00125396969641239\\
32.01	0.00125397126647316\\
33.01	0.00125397286970307\\
34.01	0.00125397450680905\\
35.01	0.00125397617851321\\
36.01	0.00125397788555339\\
37.01	0.00125397962868316\\
38.01	0.00125398140867252\\
39.01	0.00125398322630799\\
40.01	0.0012539850823931\\
41.01	0.00125398697774879\\
42.01	0.00125398891321365\\
43.01	0.00125399088964447\\
44.01	0.00125399290791648\\
45.01	0.00125399496892396\\
46.01	0.00125399707358046\\
47.01	0.00125399922281941\\
48.01	0.00125400141759426\\
49.01	0.0012540036588793\\
50.01	0.00125400594766984\\
51.01	0.00125400828498275\\
52.01	0.00125401067185706\\
53.01	0.00125401310935421\\
54.01	0.0012540155985587\\
55.01	0.00125401814057864\\
56.01	0.00125402073654616\\
57.01	0.00125402338761801\\
58.01	0.00125402609497604\\
59.01	0.00125402885982782\\
60.01	0.00125403168340715\\
61.01	0.00125403456697468\\
62.01	0.00125403751181852\\
63.01	0.00125404051925487\\
64.01	0.00125404359062846\\
65.01	0.0012540467273135\\
66.01	0.00125404993071397\\
67.01	0.00125405320226462\\
68.01	0.00125405654343137\\
69.01	0.00125405995571222\\
70.01	0.00125406344063787\\
71.01	0.00125406699977242\\
72.01	0.00125407063471412\\
73.01	0.00125407434709629\\
74.01	0.00125407813858777\\
75.01	0.00125408201089396\\
76.01	0.00125408596575772\\
77.01	0.00125409000495993\\
78.01	0.00125409413032049\\
79.01	0.00125409834369923\\
80.01	0.0012541026469967\\
81.01	0.00125410704215514\\
82.01	0.00125411153115942\\
83.01	0.00125411611603796\\
84.01	0.00125412079886368\\
85.01	0.00125412558175515\\
86.01	0.00125413046687735\\
87.01	0.00125413545644301\\
88.01	0.00125414055271341\\
89.01	0.0012541457579996\\
90.01	0.0012541510746636\\
91.01	0.00125415650511938\\
92.01	0.00125416205183409\\
93.01	0.00125416771732924\\
94.01	0.00125417350418203\\
95.01	0.00125417941502655\\
96.01	0.00125418545255487\\
97.01	0.00125419161951863\\
98.01	0.00125419791873025\\
99.01	0.00125420435306429\\
100.01	0.0012542109254589\\
101.01	0.0012542176389172\\
102.01	0.00125422449650878\\
103.01	0.00125423150137134\\
104.01	0.0012542386567118\\
105.01	0.00125424596580841\\
106.01	0.00125425343201209\\
107.01	0.00125426105874794\\
108.01	0.00125426884951726\\
109.01	0.00125427680789904\\
110.01	0.00125428493755174\\
111.01	0.00125429324221514\\
112.01	0.00125430172571228\\
113.01	0.00125431039195117\\
114.01	0.00125431924492683\\
115.01	0.00125432828872321\\
116.01	0.00125433752751532\\
117.01	0.0012543469655712\\
118.01	0.00125435660725406\\
119.01	0.0012543664570245\\
120.01	0.00125437651944268\\
121.01	0.00125438679917059\\
122.01	0.00125439730097445\\
123.01	0.00125440802972698\\
124.01	0.00125441899040989\\
125.01	0.00125443018811639\\
126.01	0.00125444162805375\\
127.01	0.00125445331554588\\
128.01	0.00125446525603598\\
129.01	0.00125447745508923\\
130.01	0.00125448991839586\\
131.01	0.00125450265177357\\
132.01	0.00125451566117089\\
133.01	0.00125452895266993\\
134.01	0.00125454253248947\\
135.01	0.00125455640698822\\
136.01	0.00125457058266792\\
137.01	0.0012545850661767\\
138.01	0.00125459986431244\\
139.01	0.00125461498402614\\
140.01	0.00125463043242562\\
141.01	0.00125464621677888\\
142.01	0.00125466234451806\\
143.01	0.00125467882324309\\
144.01	0.00125469566072556\\
145.01	0.00125471286491268\\
146.01	0.00125473044393147\\
147.01	0.00125474840609271\\
148.01	0.00125476675989535\\
149.01	0.00125478551403082\\
150.01	0.00125480467738739\\
151.01	0.00125482425905484\\
152.01	0.00125484426832918\\
153.01	0.00125486471471729\\
154.01	0.00125488560794186\\
155.01	0.00125490695794642\\
156.01	0.00125492877490042\\
157.01	0.00125495106920464\\
158.01	0.00125497385149625\\
159.01	0.0012549971326546\\
160.01	0.00125502092380669\\
161.01	0.00125504523633294\\
162.01	0.00125507008187316\\
163.01	0.00125509547233247\\
164.01	0.00125512141988763\\
165.01	0.00125514793699322\\
166.01	0.0012551750363882\\
167.01	0.00125520273110249\\
168.01	0.00125523103446372\\
169.01	0.00125525996010425\\
170.01	0.00125528952196834\\
171.01	0.00125531973431916\\
172.01	0.00125535061174647\\
173.01	0.00125538216917418\\
174.01	0.00125541442186815\\
175.01	0.00125544738544411\\
176.01	0.0012554810758759\\
177.01	0.00125551550950389\\
178.01	0.00125555070304341\\
179.01	0.00125558667359361\\
180.01	0.00125562343864633\\
181.01	0.00125566101609541\\
182.01	0.00125569942424602\\
183.01	0.00125573868182436\\
184.01	0.00125577880798727\\
185.01	0.00125581982233263\\
186.01	0.00125586174490941\\
187.01	0.00125590459622835\\
188.01	0.00125594839727284\\
189.01	0.0012559931695098\\
190.01	0.00125603893490105\\
191.01	0.001256085715915\\
192.01	0.00125613353553838\\
193.01	0.00125618241728846\\
194.01	0.00125623238522533\\
195.01	0.00125628346396483\\
196.01	0.00125633567869134\\
197.01	0.00125638905517123\\
198.01	0.00125644361976628\\
199.01	0.00125649939944798\\
200.01	0.00125655642181124\\
201.01	0.00125661471508962\\
202.01	0.00125667430816975\\
203.01	0.00125673523060681\\
204.01	0.0012567975126402\\
205.01	0.00125686118520943\\
206.01	0.00125692627997055\\
207.01	0.00125699282931286\\
208.01	0.00125706086637609\\
209.01	0.00125713042506786\\
210.01	0.00125720154008163\\
211.01	0.00125727424691511\\
212.01	0.00125734858188885\\
213.01	0.00125742458216568\\
214.01	0.00125750228577012\\
215.01	0.00125758173160866\\
216.01	0.00125766295949011\\
217.01	0.00125774601014684\\
218.01	0.00125783092525605\\
219.01	0.00125791774746189\\
220.01	0.00125800652039799\\
221.01	0.00125809728871031\\
222.01	0.0012581900980809\\
223.01	0.0012582849952516\\
224.01	0.001258382028049\\
225.01	0.00125848124540939\\
226.01	0.00125858269740458\\
227.01	0.00125868643526809\\
228.01	0.00125879251142234\\
229.01	0.00125890097950571\\
230.01	0.00125901189440101\\
231.01	0.00125912531226417\\
232.01	0.00125924129055357\\
233.01	0.00125935988806013\\
234.01	0.00125948116493796\\
235.01	0.00125960518273595\\
236.01	0.00125973200442973\\
237.01	0.00125986169445445\\
238.01	0.00125999431873861\\
239.01	0.00126012994473805\\
240.01	0.0012602686414711\\
241.01	0.0012604104795545\\
242.01	0.00126055553123994\\
243.01	0.00126070387045153\\
244.01	0.00126085557282399\\
245.01	0.00126101071574184\\
246.01	0.00126116937837926\\
247.01	0.00126133164174099\\
248.01	0.00126149758870402\\
249.01	0.00126166730406024\\
250.01	0.00126184087456004\\
251.01	0.00126201838895677\\
252.01	0.00126219993805231\\
253.01	0.00126238561474343\\
254.01	0.00126257551406953\\
255.01	0.001262769733261\\
256.01	0.00126296837178885\\
257.01	0.00126317153141541\\
258.01	0.00126337931624609\\
259.01	0.00126359183278239\\
260.01	0.00126380918997596\\
261.01	0.00126403149928375\\
262.01	0.00126425887472449\\
263.01	0.00126449143293652\\
264.01	0.00126472929323666\\
265.01	0.00126497257768039\\
266.01	0.00126522141112361\\
267.01	0.00126547592128547\\
268.01	0.0012657362388127\\
269.01	0.00126600249734532\\
270.01	0.00126627483358403\\
271.01	0.00126655338735864\\
272.01	0.00126683830169855\\
273.01	0.00126712972290432\\
274.01	0.00126742780062136\\
275.01	0.00126773268791475\\
276.01	0.00126804454134626\\
277.01	0.00126836352105265\\
278.01	0.00126868979082625\\
279.01	0.00126902351819685\\
280.01	0.00126936487451616\\
281.01	0.00126971403504353\\
282.01	0.00127007117903408\\
283.01	0.00127043648982901\\
284.01	0.00127081015494774\\
285.01	0.00127119236618251\\
286.01	0.00127158331969509\\
287.01	0.00127198321611599\\
288.01	0.00127239226064606\\
289.01	0.00127281066316061\\
290.01	0.00127323863831603\\
291.01	0.00127367640565934\\
292.01	0.00127412418974026\\
293.01	0.00127458222022641\\
294.01	0.00127505073202128\\
295.01	0.00127552996538557\\
296.01	0.0012760201660615\\
297.01	0.00127652158540051\\
298.01	0.00127703448049461\\
299.01	0.00127755911431093\\
300.01	0.00127809575583051\\
301.01	0.00127864468019063\\
302.01	0.00127920616883109\\
303.01	0.00127978050964508\\
304.01	0.00128036799713387\\
305.01	0.00128096893256652\\
306.01	0.00128158362414391\\
307.01	0.00128221238716784\\
308.01	0.00128285554421517\\
309.01	0.00128351342531733\\
310.01	0.00128418636814499\\
311.01	0.00128487471819885\\
312.01	0.0012855788290061\\
313.01	0.00128629906232317\\
314.01	0.00128703578834468\\
315.01	0.00128778938591937\\
316.01	0.00128856024277257\\
317.01	0.001289348755736\\
318.01	0.00129015533098513\\
319.01	0.00129098038428395\\
320.01	0.00129182434123811\\
321.01	0.00129268763755599\\
322.01	0.00129357071931841\\
323.01	0.00129447404325767\\
324.01	0.00129539807704496\\
325.01	0.00129634329958798\\
326.01	0.00129731020133809\\
327.01	0.00129829928460753\\
328.01	0.00129931106389727\\
329.01	0.00130034606623538\\
330.01	0.00130140483152689\\
331.01	0.00130248791291458\\
332.01	0.00130359587715185\\
333.01	0.00130472930498773\\
334.01	0.00130588879156403\\
335.01	0.0013070749468253\\
336.01	0.00130828839594193\\
337.01	0.00130952977974665\\
338.01	0.00131079975518428\\
339.01	0.00131209899577613\\
340.01	0.00131342819209829\\
341.01	0.00131478805227443\\
342.01	0.00131617930248375\\
343.01	0.00131760268748405\\
344.01	0.0013190589711502\\
345.01	0.00132054893702836\\
346.01	0.00132207338890637\\
347.01	0.00132363315140022\\
348.01	0.00132522907055724\\
349.01	0.00132686201447599\\
350.01	0.00132853287394308\\
351.01	0.00133024256308745\\
352.01	0.00133199202005168\\
353.01	0.00133378220768134\\
354.01	0.00133561411423148\\
355.01	0.00133748875409111\\
356.01	0.00133940716852558\\
357.01	0.00134137042643621\\
358.01	0.00134337962513773\\
359.01	0.00134543589115287\\
360.01	0.00134754038102385\\
361.01	0.0013496942821399\\
362.01	0.00135189881358083\\
363.01	0.00135415522697461\\
364.01	0.00135646480736899\\
365.01	0.00135882887411507\\
366.01	0.00136124878176148\\
367.01	0.00136372592095738\\
368.01	0.00136626171936242\\
369.01	0.00136885764256187\\
370.01	0.00137151519498482\\
371.01	0.00137423592082448\\
372.01	0.00137702140495972\\
373.01	0.00137987327387897\\
374.01	0.00138279319660863\\
375.01	0.00138578288565305\\
376.01	0.00138884409795629\\
377.01	0.0013919786359096\\
378.01	0.00139518834852271\\
379.01	0.00139847513283203\\
380.01	0.00140184093410652\\
381.01	0.00140528774874939\\
382.01	0.00140881762657582\\
383.01	0.00141243267238211\\
384.01	0.00141613504756924\\
385.01	0.00141992697182572\\
386.01	0.00142381072487216\\
387.01	0.00142778864827148\\
388.01	0.00143186314730736\\
389.01	0.00143603669293546\\
390.01	0.00144031182381085\\
391.01	0.00144469114839608\\
392.01	0.00144917734715422\\
393.01	0.00145377317483215\\
394.01	0.00145848146283897\\
395.01	0.0014633051217252\\
396.01	0.00146824714376911\\
397.01	0.00147331060567626\\
398.01	0.00147849867139957\\
399.01	0.00148381459508746\\
400.01	0.00148926172416791\\
401.01	0.00149484350257767\\
402.01	0.00150056347414533\\
403.01	0.00150642528613958\\
404.01	0.00151243269299284\\
405.01	0.00151858956021261\\
406.01	0.00152489986849356\\
407.01	0.0015313677180441\\
408.01	0.00153799733314351\\
409.01	0.00154479306694518\\
410.01	0.00155175940654517\\
411.01	0.00155890097833498\\
412.01	0.00156622255366081\\
413.01	0.00157372905481189\\
414.01	0.00158142556136492\\
415.01	0.00158931731691189\\
416.01	0.00159740973620281\\
417.01	0.00160570841273723\\
418.01	0.00161421912684314\\
419.01	0.00162294785428316\\
420.01	0.00163190077543637\\
421.01	0.00164108428510503\\
422.01	0.00165050500300411\\
423.01	0.00166016978499704\\
424.01	0.00167008573514798\\
425.01	0.00168026021867062\\
426.01	0.00169070087586256\\
427.01	0.00170141563712507\\
428.01	0.00171241273918239\\
429.01	0.00172370074262813\\
430.01	0.00173528855094479\\
431.01	0.00174718543116167\\
432.01	0.00175940103634059\\
433.01	0.00177194543010548\\
434.01	0.00178482911346454\\
435.01	0.00179806305421074\\
436.01	0.0018116587192307\\
437.01	0.00182562811010398\\
438.01	0.00183998380243701\\
439.01	0.00185473898944803\\
440.01	0.00186990753040684\\
441.01	0.00188550400463645\\
442.01	0.00190154377190793\\
443.01	0.00191804304020712\\
444.01	0.0019350189420306\\
445.01	0.0019524896205829\\
446.01	0.00197047432750603\\
447.01	0.00198899353408612\\
448.01	0.00200806905826303\\
449.01	0.0020277242102326\\
450.01	0.00204798395999738\\
451.01	0.0020688751309135\\
452.01	0.00209042662413107\\
453.01	0.00211266967986786\\
454.01	0.00213563818274333\\
455.01	0.00215936901998632\\
456.01	0.00218390250329652\\
457.01	0.00220928286757602\\
458.01	0.00223555886277866\\
459.01	0.00226278445889772\\
460.01	0.0022910196888251\\
461.01	0.00232033165971263\\
462.01	0.00235079577092562\\
463.01	0.00238249718617517\\
464.01	0.00241553261370436\\
465.01	0.00245001243512748\\
466.01	0.0024860629897989\\
467.01	0.00252269340217711\\
468.01	0.00253901728385071\\
469.01	0.00255639392450184\\
470.01	0.00257496450888221\\
471.01	0.00259489690471044\\
472.01	0.00261639053285141\\
473.01	0.00263968364758491\\
474.01	0.00266506226992916\\
475.01	0.00269245505943368\\
476.01	0.00272065093610824\\
477.01	0.00274955271826105\\
478.01	0.00277917729609491\\
479.01	0.00280954178593348\\
480.01	0.00284066351017588\\
481.01	0.00287255997420967\\
482.01	0.00290524884014917\\
483.01	0.00293874789733586\\
484.01	0.00297307502964344\\
485.01	0.00300824817977751\\
486.01	0.00304428531097289\\
487.01	0.00308120436689807\\
488.01	0.00311902323296728\\
489.01	0.00315775973843944\\
490.01	0.00319743226223462\\
491.01	0.00323806066098094\\
492.01	0.00327966220576215\\
493.01	0.00332225262627554\\
494.01	0.00336584605081814\\
495.01	0.00341045552030027\\
496.01	0.00345609286794123\\
497.01	0.00350276851295833\\
498.01	0.00355049120673921\\
499.01	0.00359926768736594\\
500.01	0.00364910239797952\\
501.01	0.00369999726787668\\
502.01	0.0037519516255841\\
503.01	0.00380496318885829\\
504.01	0.00385900734789926\\
505.01	0.00391403660138614\\
506.01	0.00396998585551344\\
507.01	0.0040267684552152\\
508.01	0.00408427124200197\\
509.01	0.00414234847335277\\
510.01	0.00420081432960689\\
511.01	0.00425943366533543\\
512.01	0.00431791057508681\\
513.01	0.00437587423268418\\
514.01	0.00443286131603846\\
515.01	0.00448829405724582\\
516.01	0.00454145169655468\\
517.01	0.00459201346058562\\
518.01	0.0046429781323353\\
519.01	0.00469489782545895\\
520.01	0.00474771890165974\\
521.01	0.00480136321045815\\
522.01	0.00485574293994415\\
523.01	0.00491145818565206\\
524.01	0.00496885945429269\\
525.01	0.00502801076693878\\
526.01	0.00508897183480263\\
527.01	0.00515179490913153\\
528.01	0.00521652055630504\\
529.01	0.00528317329708926\\
530.01	0.00535175362761153\\
531.01	0.00542221561762005\\
532.01	0.00549445951575484\\
533.01	0.00556831438034086\\
534.01	0.00564352082639238\\
535.01	0.00571974964221054\\
536.01	0.0057957442445397\\
537.01	0.00587105093686107\\
538.01	0.00594551194541186\\
539.01	0.00601897841781172\\
540.01	0.00609132568218902\\
541.01	0.00616247150904156\\
542.01	0.00623236290012066\\
543.01	0.0063008176648035\\
544.01	0.00636760942669748\\
545.01	0.00643254230701142\\
546.01	0.00649542408219803\\
547.01	0.00655610162007356\\
548.01	0.006614507589475\\
549.01	0.0066707307889387\\
550.01	0.00672605603223849\\
551.01	0.00678112490934816\\
552.01	0.00683590420357794\\
553.01	0.0068903704202434\\
554.01	0.00694451627906765\\
555.01	0.00699835431625151\\
556.01	0.00705192041560494\\
557.01	0.00710527686155195\\
558.01	0.00715851427507727\\
559.01	0.00721175147058358\\
560.01	0.00726513178502448\\
561.01	0.00731881411340836\\
562.01	0.00737295989880207\\
563.01	0.00742766298174074\\
564.01	0.00748295075754973\\
565.01	0.00753885104528757\\
566.01	0.0075953952343396\\
567.01	0.00765261854464693\\
568.01	0.00771055980364574\\
569.01	0.0077692603192376\\
570.01	0.00782876227912249\\
571.01	0.00788910669409165\\
572.01	0.0079503309275909\\
573.01	0.00801246605430344\\
574.01	0.00807553530800077\\
575.01	0.00813955656820071\\
576.01	0.00820454654838254\\
577.01	0.00827052111535247\\
578.01	0.00833749503822894\\
579.01	0.00840548173779629\\
580.01	0.00847449285543066\\
581.01	0.00854453779037137\\
582.01	0.00861562329403154\\
583.01	0.00868775317389058\\
584.01	0.00876092815854303\\
585.01	0.00883514596130418\\
586.01	0.00891040144580925\\
587.01	0.00898668645072877\\
588.01	0.00906398946084543\\
589.01	0.00914229527513433\\
590.01	0.0092215846004733\\
591.01	0.00930183366668292\\
592.01	0.00938301391778946\\
593.01	0.00946509184244681\\
594.01	0.00954802902026272\\
595.01	0.00963178247925906\\
596.01	0.00971630547999516\\
597.01	0.00980154886224196\\
598.01	0.00988746310858009\\
599.01	0.00996918176309228\\
599.02	0.00996972426151389\\
599.03	0.00997026325652431\\
599.04	0.00997079875308772\\
599.05	0.00997133075863836\\
599.06	0.00997185927537436\\
599.07	0.00997238430589235\\
599.08	0.00997290585509465\\
599.09	0.00997342387048931\\
599.1	0.00997393826406157\\
599.11	0.00997444900322816\\
599.12	0.00997495605368425\\
599.13	0.00997545938076358\\
599.14	0.00997595894942923\\
599.15	0.0099764547242638\\
599.16	0.00997694666945895\\
599.17	0.00997743474895895\\
599.18	0.00997791899742809\\
599.19	0.00997839937780886\\
599.2	0.00997887585262738\\
599.21	0.00997934838398374\\
599.22	0.00997981693354265\\
599.23	0.00998028146252\\
599.24	0.00998074193166859\\
599.25	0.00998119830126832\\
599.26	0.00998165053112998\\
599.27	0.0099820985805808\\
599.28	0.00998254240840773\\
599.29	0.00998298197288523\\
599.3	0.00998341723187397\\
599.31	0.00998384814281939\\
599.32	0.0099842746627474\\
599.33	0.00998469674826017\\
599.34	0.00998511435553172\\
599.35	0.00998552744030362\\
599.36	0.00998593595788055\\
599.37	0.00998633986312586\\
599.38	0.00998673911045709\\
599.39	0.00998713365384143\\
599.4	0.00998752344679118\\
599.41	0.00998790844105529\\
599.42	0.00998828858549677\\
599.43	0.0099886638284696\\
599.44	0.00998903411781367\\
599.45	0.00998939940084989\\
599.46	0.0099897596243751\\
599.47	0.00999011473465703\\
599.48	0.00999046467742929\\
599.49	0.00999080939788633\\
599.5	0.00999114884067832\\
599.51	0.00999148294990612\\
599.52	0.0099918116691162\\
599.53	0.00999213494129551\\
599.54	0.00999245270886605\\
599.55	0.0099927649136794\\
599.56	0.009993071497011\\
599.57	0.00999337239955457\\
599.58	0.00999366756141627\\
599.59	0.00999395692210898\\
599.6	0.00999424042054647\\
599.61	0.00999451799503739\\
599.62	0.00999478958327942\\
599.63	0.00999505512235319\\
599.64	0.00999531454871618\\
599.65	0.00999556779819663\\
599.66	0.00999581480598729\\
599.67	0.00999605550663917\\
599.68	0.00999628983405521\\
599.69	0.0099965177214839\\
599.7	0.00999673910151283\\
599.71	0.00999695390606217\\
599.72	0.00999716206637809\\
599.73	0.00999736351302614\\
599.74	0.00999755817588451\\
599.75	0.00999774598413729\\
599.76	0.00999792686626763\\
599.77	0.00999810075005082\\
599.78	0.00999826756254734\\
599.79	0.0099984272300958\\
599.8	0.00999857967830587\\
599.81	0.00999872483205108\\
599.82	0.00999886261546159\\
599.83	0.00999899295191686\\
599.84	0.0099991157640383\\
599.85	0.00999923097368178\\
599.86	0.00999933850193015\\
599.87	0.00999943826908557\\
599.88	0.00999953019466193\\
599.89	0.00999961419737702\\
599.9	0.00999969019514477\\
599.91	0.0099997581050673\\
599.92	0.009999817843427\\
599.93	0.00999986932567845\\
599.94	0.00999991246644027\\
599.95	0.00999994717948697\\
599.96	0.00999997337774056\\
599.97	0.00999999097326228\\
599.98	0.00999999987724406\\
599.99	0.01\\
600	0.01\\
};
\addplot [color=mycolor10,solid,forget plot]
  table[row sep=crcr]{%
0.01	0\\
1.01	0\\
2.01	0\\
3.01	0\\
4.01	0\\
5.01	0\\
6.01	0\\
7.01	0\\
8.01	0\\
9.01	0\\
10.01	0\\
11.01	0\\
12.01	0\\
13.01	0\\
14.01	0\\
15.01	0\\
16.01	0\\
17.01	0\\
18.01	0\\
19.01	0\\
20.01	0\\
21.01	0\\
22.01	0\\
23.01	0\\
24.01	0\\
25.01	0\\
26.01	0\\
27.01	0\\
28.01	0\\
29.01	0\\
30.01	0\\
31.01	0\\
32.01	0\\
33.01	0\\
34.01	0\\
35.01	0\\
36.01	0\\
37.01	0\\
38.01	0\\
39.01	0\\
40.01	0\\
41.01	0\\
42.01	0\\
43.01	0\\
44.01	0\\
45.01	0\\
46.01	0\\
47.01	0\\
48.01	0\\
49.01	0\\
50.01	0\\
51.01	0\\
52.01	0\\
53.01	0\\
54.01	0\\
55.01	0\\
56.01	0\\
57.01	0\\
58.01	0\\
59.01	0\\
60.01	0\\
61.01	0\\
62.01	0\\
63.01	0\\
64.01	0\\
65.01	0\\
66.01	0\\
67.01	0\\
68.01	0\\
69.01	0\\
70.01	0\\
71.01	0\\
72.01	0\\
73.01	0\\
74.01	0\\
75.01	0\\
76.01	0\\
77.01	0\\
78.01	0\\
79.01	0\\
80.01	0\\
81.01	0\\
82.01	0\\
83.01	0\\
84.01	0\\
85.01	0\\
86.01	0\\
87.01	0\\
88.01	0\\
89.01	0\\
90.01	0\\
91.01	0\\
92.01	0\\
93.01	0\\
94.01	0\\
95.01	0\\
96.01	0\\
97.01	0\\
98.01	0\\
99.01	0\\
100.01	0\\
101.01	0\\
102.01	0\\
103.01	0\\
104.01	0\\
105.01	0\\
106.01	0\\
107.01	0\\
108.01	0\\
109.01	0\\
110.01	0\\
111.01	0\\
112.01	0\\
113.01	0\\
114.01	0\\
115.01	0\\
116.01	0\\
117.01	0\\
118.01	0\\
119.01	0\\
120.01	0\\
121.01	0\\
122.01	0\\
123.01	0\\
124.01	0\\
125.01	0\\
126.01	0\\
127.01	0\\
128.01	0\\
129.01	0\\
130.01	0\\
131.01	0\\
132.01	0\\
133.01	0\\
134.01	0\\
135.01	0\\
136.01	0\\
137.01	0\\
138.01	0\\
139.01	0\\
140.01	0\\
141.01	0\\
142.01	0\\
143.01	0\\
144.01	0\\
145.01	0\\
146.01	0\\
147.01	0\\
148.01	0\\
149.01	0\\
150.01	0\\
151.01	0\\
152.01	0\\
153.01	0\\
154.01	0\\
155.01	0\\
156.01	0\\
157.01	0\\
158.01	0\\
159.01	0\\
160.01	0\\
161.01	0\\
162.01	0\\
163.01	0\\
164.01	0\\
165.01	0\\
166.01	0\\
167.01	0\\
168.01	0\\
169.01	0\\
170.01	0\\
171.01	0\\
172.01	0\\
173.01	0\\
174.01	0\\
175.01	0\\
176.01	0\\
177.01	0\\
178.01	0\\
179.01	0\\
180.01	0\\
181.01	0\\
182.01	0\\
183.01	0\\
184.01	0\\
185.01	0\\
186.01	0\\
187.01	0\\
188.01	0\\
189.01	0\\
190.01	0\\
191.01	0\\
192.01	0\\
193.01	0\\
194.01	0\\
195.01	0\\
196.01	0\\
197.01	0\\
198.01	0\\
199.01	0\\
200.01	0\\
201.01	0\\
202.01	0\\
203.01	0\\
204.01	0\\
205.01	0\\
206.01	0\\
207.01	0\\
208.01	0\\
209.01	0\\
210.01	0\\
211.01	0\\
212.01	0\\
213.01	0\\
214.01	0\\
215.01	0\\
216.01	0\\
217.01	0\\
218.01	0\\
219.01	0\\
220.01	0\\
221.01	0\\
222.01	0\\
223.01	0\\
224.01	0\\
225.01	0\\
226.01	0\\
227.01	0\\
228.01	0\\
229.01	0\\
230.01	0\\
231.01	0\\
232.01	0\\
233.01	0\\
234.01	0\\
235.01	0\\
236.01	0\\
237.01	0\\
238.01	0\\
239.01	0\\
240.01	0\\
241.01	0\\
242.01	0\\
243.01	0\\
244.01	0\\
245.01	0\\
246.01	0\\
247.01	0\\
248.01	0\\
249.01	0\\
250.01	0\\
251.01	0\\
252.01	0\\
253.01	0\\
254.01	0\\
255.01	0\\
256.01	0\\
257.01	0\\
258.01	0\\
259.01	0\\
260.01	0\\
261.01	0\\
262.01	0\\
263.01	0\\
264.01	0\\
265.01	0\\
266.01	0\\
267.01	0\\
268.01	0\\
269.01	0\\
270.01	0\\
271.01	0\\
272.01	0\\
273.01	0\\
274.01	0\\
275.01	0\\
276.01	0\\
277.01	0\\
278.01	0\\
279.01	0\\
280.01	0\\
281.01	0\\
282.01	0\\
283.01	0\\
284.01	0\\
285.01	0\\
286.01	0\\
287.01	0\\
288.01	0\\
289.01	0\\
290.01	0\\
291.01	0\\
292.01	0\\
293.01	0\\
294.01	0\\
295.01	0\\
296.01	0\\
297.01	0\\
298.01	0\\
299.01	0\\
300.01	0\\
301.01	0\\
302.01	0\\
303.01	0\\
304.01	0\\
305.01	0\\
306.01	0\\
307.01	0\\
308.01	0\\
309.01	0\\
310.01	0\\
311.01	0\\
312.01	0\\
313.01	0\\
314.01	0\\
315.01	0\\
316.01	0\\
317.01	0\\
318.01	0\\
319.01	0\\
320.01	0\\
321.01	0\\
322.01	0\\
323.01	0\\
324.01	0\\
325.01	0\\
326.01	0\\
327.01	0\\
328.01	0\\
329.01	0\\
330.01	0\\
331.01	0\\
332.01	0\\
333.01	0\\
334.01	0\\
335.01	0\\
336.01	0\\
337.01	0\\
338.01	0\\
339.01	0\\
340.01	0\\
341.01	0\\
342.01	0\\
343.01	0\\
344.01	0\\
345.01	0\\
346.01	0\\
347.01	0\\
348.01	0\\
349.01	0\\
350.01	0\\
351.01	0\\
352.01	0\\
353.01	0\\
354.01	0\\
355.01	0\\
356.01	0\\
357.01	0\\
358.01	0\\
359.01	0\\
360.01	0\\
361.01	0\\
362.01	0\\
363.01	0\\
364.01	0\\
365.01	0\\
366.01	0\\
367.01	0\\
368.01	0\\
369.01	0\\
370.01	0\\
371.01	0\\
372.01	0\\
373.01	0\\
374.01	0\\
375.01	0\\
376.01	0\\
377.01	0\\
378.01	0\\
379.01	0\\
380.01	0\\
381.01	0\\
382.01	0\\
383.01	0\\
384.01	0\\
385.01	0\\
386.01	0\\
387.01	0\\
388.01	0\\
389.01	0\\
390.01	0\\
391.01	0\\
392.01	0\\
393.01	0\\
394.01	0\\
395.01	0\\
396.01	0\\
397.01	0\\
398.01	0\\
399.01	0\\
400.01	0\\
401.01	0\\
402.01	0\\
403.01	0\\
404.01	0\\
405.01	0\\
406.01	0\\
407.01	0\\
408.01	0\\
409.01	0\\
410.01	0\\
411.01	0\\
412.01	0\\
413.01	0\\
414.01	0\\
415.01	0\\
416.01	0\\
417.01	0\\
418.01	0\\
419.01	0\\
420.01	0\\
421.01	0\\
422.01	0\\
423.01	0\\
424.01	0\\
425.01	0\\
426.01	0\\
427.01	0\\
428.01	0\\
429.01	0\\
430.01	0\\
431.01	0\\
432.01	0\\
433.01	0\\
434.01	0\\
435.01	0\\
436.01	0\\
437.01	0\\
438.01	0\\
439.01	0\\
440.01	0\\
441.01	0\\
442.01	0\\
443.01	0\\
444.01	0\\
445.01	0\\
446.01	0\\
447.01	0\\
448.01	0\\
449.01	0\\
450.01	0\\
451.01	0\\
452.01	0\\
453.01	0\\
454.01	0\\
455.01	0\\
456.01	0\\
457.01	0\\
458.01	0\\
459.01	0\\
460.01	0\\
461.01	0\\
462.01	0\\
463.01	0\\
464.01	0\\
465.01	0\\
466.01	0\\
467.01	1.1329020402559e-06\\
468.01	2.42926371570376e-05\\
469.01	4.81479683581976e-05\\
470.01	7.27151256976269e-05\\
471.01	9.80079158148517e-05\\
472.01	0.000124038584063818\\
473.01	0.000150816227182535\\
474.01	0.000178344407502577\\
475.01	0.000206620404597957\\
476.01	0.000235655901361494\\
477.01	0.000265476405587259\\
478.01	0.000296108967598012\\
479.01	0.000327582024237015\\
480.01	0.000359925502228145\\
481.01	0.000393170931133675\\
482.01	0.000427351566845549\\
483.01	0.000462502526611258\\
484.01	0.000498660936648215\\
485.01	0.000535866093441591\\
486.01	0.000574159639844557\\
487.01	0.000613585757155623\\
488.01	0.000654191375036341\\
489.01	0.000696026407974198\\
490.01	0.000739144056248947\\
491.01	0.000783601041649637\\
492.01	0.000829457753426376\\
493.01	0.000876778638348154\\
494.01	0.000925632540756657\\
495.01	0.000976093074296254\\
496.01	0.00102823898237879\\
497.01	0.00108215450488519\\
498.01	0.00113792974319596\\
499.01	0.00119566100888369\\
500.01	0.00125545111606378\\
501.01	0.00131740941963204\\
502.01	0.00138165037152721\\
503.01	0.00144829225281667\\
504.01	0.00151748425710271\\
505.01	0.00158940412526166\\
506.01	0.00166425130362097\\
507.01	0.00174225050833373\\
508.01	0.00182365606282622\\
509.01	0.00190875711176604\\
510.01	0.00199788390795892\\
511.01	0.00209141541598479\\
512.01	0.002189788535845\\
513.01	0.00229350932384645\\
514.01	0.00240316667304261\\
515.01	0.0025194489415711\\
516.01	0.00264316302433761\\
517.01	0.00277466880452097\\
518.01	0.00291105160323856\\
519.01	0.00305194372531061\\
520.01	0.00319760921929432\\
521.01	0.00334834332533889\\
522.01	0.00348683251894904\\
523.01	0.00356545994298809\\
524.01	0.00364603069415331\\
525.01	0.00372855121145225\\
526.01	0.0038130173250395\\
527.01	0.00389941178465692\\
528.01	0.00398770129590559\\
529.01	0.00407783303432947\\
530.01	0.00416973012800455\\
531.01	0.00426328637259697\\
532.01	0.00435836060467695\\
533.01	0.00445476958820691\\
534.01	0.00455228061471653\\
535.01	0.00465060558407056\\
536.01	0.00474938505180781\\
537.01	0.00484816680571053\\
538.01	0.00494636987829088\\
539.01	0.00504324604238324\\
540.01	0.00513783824991967\\
541.01	0.00522892631659753\\
542.01	0.00531858003379235\\
543.01	0.00541077469880275\\
544.01	0.00550551210428055\\
545.01	0.0056026673440873\\
546.01	0.00570201569717175\\
547.01	0.00580320740074056\\
548.01	0.00590572513662101\\
549.01	0.0060087956203401\\
550.01	0.00611035611335997\\
551.01	0.00621030214124193\\
552.01	0.00630902143547821\\
553.01	0.00640619490681774\\
554.01	0.00650148779051777\\
555.01	0.00659455811486975\\
556.01	0.00668506994227237\\
557.01	0.00677271322077196\\
558.01	0.00685723252977279\\
559.01	0.00693846697486137\\
560.01	0.00701641341732511\\
561.01	0.00709125918268363\\
562.01	0.0071634980266294\\
563.01	0.00723508800698468\\
564.01	0.00730639425146508\\
565.01	0.00737738964046476\\
566.01	0.00744807960110319\\
567.01	0.00751845268206095\\
568.01	0.00758851847485299\\
569.01	0.00765831772946058\\
570.01	0.00772792707527122\\
571.01	0.00779746294572383\\
572.01	0.00786708620901358\\
573.01	0.00793698753582011\\
574.01	0.00800732822400679\\
575.01	0.00807816822687876\\
576.01	0.00814952977669954\\
577.01	0.00822143569195578\\
578.01	0.00829390508513462\\
579.01	0.00836695756900242\\
580.01	0.00844061618955115\\
581.01	0.00851490610843392\\
582.01	0.008589852586896\\
583.01	0.00866547846795157\\
584.01	0.00874180134816968\\
585.01	0.00881883011789336\\
586.01	0.00889656636788261\\
587.01	0.00897501123941693\\
588.01	0.00905415976584804\\
589.01	0.00913400639453102\\
590.01	0.00921454649402135\\
591.01	0.00929577647899807\\
592.01	0.00937769332881967\\
593.01	0.00946029382921912\\
594.01	0.00954357359616978\\
595.01	0.00962752597645235\\
596.01	0.00971214096674654\\
597.01	0.00979740439062092\\
598.01	0.00988329769391883\\
599.01	0.00996873201918203\\
599.02	0.00996938150051711\\
599.03	0.00997000982619995\\
599.04	0.00997061680203572\\
599.05	0.00997120222994154\\
599.06	0.00997176894407421\\
599.07	0.00997231748063729\\
599.08	0.00997284766231445\\
599.09	0.00997337282663754\\
599.1	0.00997389327698541\\
599.11	0.00997440918165476\\
599.12	0.00997492099195222\\
599.13	0.00997542869053242\\
599.14	0.00997593226134337\\
599.15	0.00997643168971941\\
599.16	0.00997692696247924\\
599.17	0.00997741806787515\\
599.18	0.00997790492496225\\
599.19	0.00997838752592332\\
599.2	0.00997886586487288\\
599.21	0.00997933993670913\\
599.22	0.00997980973680635\\
599.23	0.00998027526269526\\
599.24	0.00998073651420421\\
599.25	0.00998119349148986\\
599.26	0.00998164618964857\\
599.27	0.00998209460591699\\
599.28	0.00998253874521366\\
599.29	0.00998297859807655\\
599.3	0.00998341412419777\\
599.31	0.0099838452819738\\
599.32	0.00998427202941182\\
599.33	0.00998469432412354\\
599.34	0.00998511212331871\\
599.35	0.00998552538379823\\
599.36	0.00998593406194686\\
599.37	0.00998633811372554\\
599.38	0.00998673749466321\\
599.39	0.00998713215984817\\
599.4	0.00998752206391895\\
599.41	0.00998790715979874\\
599.42	0.00998828739732171\\
599.43	0.00998866272571617\\
599.44	0.00998903309366109\\
599.45	0.00998939844927219\\
599.46	0.00998975874008727\\
599.47	0.00999011391305059\\
599.48	0.00999046391449633\\
599.49	0.00999080869013592\\
599.5	0.00999114818504226\\
599.51	0.0099914823436306\\
599.52	0.0099918111096529\\
599.53	0.00999213442619165\\
599.54	0.00999245223572696\\
599.55	0.00999276448016512\\
599.56	0.00999307110083833\\
599.57	0.00999337203849892\\
599.58	0.00999366723331349\\
599.59	0.00999395662485697\\
599.6	0.00999424015210655\\
599.61	0.00999451775343565\\
599.62	0.00999478936660834\\
599.63	0.00999505492877341\\
599.64	0.00999531437645813\\
599.65	0.00999556764556198\\
599.66	0.00999581467135026\\
599.67	0.0099960553884477\\
599.68	0.00999628973083186\\
599.69	0.00999651763182663\\
599.7	0.00999673902409553\\
599.71	0.00999695383963498\\
599.72	0.00999716200976746\\
599.73	0.00999736346513468\\
599.74	0.00999755813569058\\
599.75	0.00999774595069428\\
599.76	0.00999792683870299\\
599.77	0.00999810072756478\\
599.78	0.00999826754441137\\
599.79	0.00999842721565074\\
599.8	0.00999857966695976\\
599.81	0.00999872482327668\\
599.82	0.00999886260879365\\
599.83	0.00999899294694907\\
599.84	0.00999911576041993\\
599.85	0.00999923097111411\\
599.86	0.00999933850016262\\
599.87	0.00999943826791175\\
599.88	0.00999953019391525\\
599.89	0.00999961419692642\\
599.9	0.0099996901948902\\
599.91	0.00999975810493522\\
599.92	0.00999981784336591\\
599.93	0.00999986932565446\\
599.94	0.009999912466433\\
599.95	0.00999994717948561\\
599.96	0.00999997337774052\\
599.97	0.00999999097326228\\
599.98	0.00999999987724406\\
599.99	0.01\\
600	0.01\\
};
\addplot [color=mycolor11,solid,forget plot]
  table[row sep=crcr]{%
0.01	0\\
1.01	0\\
2.01	0\\
3.01	0\\
4.01	0\\
5.01	0\\
6.01	0\\
7.01	0\\
8.01	0\\
9.01	0\\
10.01	0\\
11.01	0\\
12.01	0\\
13.01	0\\
14.01	0\\
15.01	0\\
16.01	0\\
17.01	0\\
18.01	0\\
19.01	0\\
20.01	0\\
21.01	0\\
22.01	0\\
23.01	0\\
24.01	0\\
25.01	0\\
26.01	0\\
27.01	0\\
28.01	0\\
29.01	0\\
30.01	0\\
31.01	0\\
32.01	0\\
33.01	0\\
34.01	0\\
35.01	0\\
36.01	0\\
37.01	0\\
38.01	0\\
39.01	0\\
40.01	0\\
41.01	0\\
42.01	0\\
43.01	0\\
44.01	0\\
45.01	0\\
46.01	0\\
47.01	0\\
48.01	0\\
49.01	0\\
50.01	0\\
51.01	0\\
52.01	0\\
53.01	0\\
54.01	0\\
55.01	0\\
56.01	0\\
57.01	0\\
58.01	0\\
59.01	0\\
60.01	0\\
61.01	0\\
62.01	0\\
63.01	0\\
64.01	0\\
65.01	0\\
66.01	0\\
67.01	0\\
68.01	0\\
69.01	0\\
70.01	0\\
71.01	0\\
72.01	0\\
73.01	0\\
74.01	0\\
75.01	0\\
76.01	0\\
77.01	0\\
78.01	0\\
79.01	0\\
80.01	0\\
81.01	0\\
82.01	0\\
83.01	0\\
84.01	0\\
85.01	0\\
86.01	0\\
87.01	0\\
88.01	0\\
89.01	0\\
90.01	0\\
91.01	0\\
92.01	0\\
93.01	0\\
94.01	0\\
95.01	0\\
96.01	0\\
97.01	0\\
98.01	0\\
99.01	0\\
100.01	0\\
101.01	0\\
102.01	0\\
103.01	0\\
104.01	0\\
105.01	0\\
106.01	0\\
107.01	0\\
108.01	0\\
109.01	0\\
110.01	0\\
111.01	0\\
112.01	0\\
113.01	0\\
114.01	0\\
115.01	0\\
116.01	0\\
117.01	0\\
118.01	0\\
119.01	0\\
120.01	0\\
121.01	0\\
122.01	0\\
123.01	0\\
124.01	0\\
125.01	0\\
126.01	0\\
127.01	0\\
128.01	0\\
129.01	0\\
130.01	0\\
131.01	0\\
132.01	0\\
133.01	0\\
134.01	0\\
135.01	0\\
136.01	0\\
137.01	0\\
138.01	0\\
139.01	0\\
140.01	0\\
141.01	0\\
142.01	0\\
143.01	0\\
144.01	0\\
145.01	0\\
146.01	0\\
147.01	0\\
148.01	0\\
149.01	0\\
150.01	0\\
151.01	0\\
152.01	0\\
153.01	0\\
154.01	0\\
155.01	0\\
156.01	0\\
157.01	0\\
158.01	0\\
159.01	0\\
160.01	0\\
161.01	0\\
162.01	0\\
163.01	0\\
164.01	0\\
165.01	0\\
166.01	0\\
167.01	0\\
168.01	0\\
169.01	0\\
170.01	0\\
171.01	0\\
172.01	0\\
173.01	0\\
174.01	0\\
175.01	0\\
176.01	0\\
177.01	0\\
178.01	0\\
179.01	0\\
180.01	0\\
181.01	0\\
182.01	0\\
183.01	0\\
184.01	0\\
185.01	0\\
186.01	0\\
187.01	0\\
188.01	0\\
189.01	0\\
190.01	0\\
191.01	0\\
192.01	0\\
193.01	0\\
194.01	0\\
195.01	0\\
196.01	0\\
197.01	0\\
198.01	0\\
199.01	0\\
200.01	0\\
201.01	0\\
202.01	0\\
203.01	0\\
204.01	0\\
205.01	0\\
206.01	0\\
207.01	0\\
208.01	0\\
209.01	0\\
210.01	0\\
211.01	0\\
212.01	0\\
213.01	0\\
214.01	0\\
215.01	0\\
216.01	0\\
217.01	0\\
218.01	0\\
219.01	0\\
220.01	0\\
221.01	0\\
222.01	0\\
223.01	0\\
224.01	0\\
225.01	0\\
226.01	0\\
227.01	0\\
228.01	0\\
229.01	0\\
230.01	0\\
231.01	0\\
232.01	0\\
233.01	0\\
234.01	0\\
235.01	0\\
236.01	0\\
237.01	0\\
238.01	0\\
239.01	0\\
240.01	0\\
241.01	0\\
242.01	0\\
243.01	0\\
244.01	0\\
245.01	0\\
246.01	0\\
247.01	0\\
248.01	0\\
249.01	0\\
250.01	0\\
251.01	0\\
252.01	0\\
253.01	0\\
254.01	0\\
255.01	0\\
256.01	0\\
257.01	0\\
258.01	0\\
259.01	0\\
260.01	0\\
261.01	0\\
262.01	0\\
263.01	0\\
264.01	0\\
265.01	0\\
266.01	0\\
267.01	0\\
268.01	0\\
269.01	0\\
270.01	0\\
271.01	0\\
272.01	0\\
273.01	0\\
274.01	0\\
275.01	0\\
276.01	0\\
277.01	0\\
278.01	0\\
279.01	0\\
280.01	0\\
281.01	0\\
282.01	0\\
283.01	0\\
284.01	0\\
285.01	0\\
286.01	0\\
287.01	0\\
288.01	0\\
289.01	0\\
290.01	0\\
291.01	0\\
292.01	0\\
293.01	0\\
294.01	0\\
295.01	0\\
296.01	0\\
297.01	0\\
298.01	0\\
299.01	0\\
300.01	0\\
301.01	0\\
302.01	0\\
303.01	0\\
304.01	0\\
305.01	0\\
306.01	0\\
307.01	0\\
308.01	0\\
309.01	0\\
310.01	0\\
311.01	0\\
312.01	0\\
313.01	0\\
314.01	0\\
315.01	0\\
316.01	0\\
317.01	0\\
318.01	0\\
319.01	0\\
320.01	0\\
321.01	0\\
322.01	0\\
323.01	0\\
324.01	0\\
325.01	0\\
326.01	0\\
327.01	0\\
328.01	0\\
329.01	0\\
330.01	0\\
331.01	0\\
332.01	0\\
333.01	0\\
334.01	0\\
335.01	0\\
336.01	0\\
337.01	0\\
338.01	0\\
339.01	0\\
340.01	0\\
341.01	0\\
342.01	0\\
343.01	0\\
344.01	0\\
345.01	0\\
346.01	0\\
347.01	0\\
348.01	0\\
349.01	0\\
350.01	0\\
351.01	0\\
352.01	0\\
353.01	0\\
354.01	0\\
355.01	0\\
356.01	0\\
357.01	0\\
358.01	0\\
359.01	0\\
360.01	0\\
361.01	0\\
362.01	0\\
363.01	0\\
364.01	0\\
365.01	0\\
366.01	0\\
367.01	0\\
368.01	0\\
369.01	0\\
370.01	0\\
371.01	0\\
372.01	0\\
373.01	0\\
374.01	0\\
375.01	0\\
376.01	0\\
377.01	0\\
378.01	0\\
379.01	0\\
380.01	0\\
381.01	0\\
382.01	0\\
383.01	0\\
384.01	0\\
385.01	0\\
386.01	0\\
387.01	0\\
388.01	0\\
389.01	0\\
390.01	0\\
391.01	0\\
392.01	0\\
393.01	0\\
394.01	0\\
395.01	0\\
396.01	0\\
397.01	0\\
398.01	0\\
399.01	0\\
400.01	0\\
401.01	0\\
402.01	0\\
403.01	0\\
404.01	0\\
405.01	0\\
406.01	0\\
407.01	0\\
408.01	0\\
409.01	0\\
410.01	0\\
411.01	0\\
412.01	0\\
413.01	0\\
414.01	0\\
415.01	0\\
416.01	0\\
417.01	0\\
418.01	0\\
419.01	0\\
420.01	0\\
421.01	0\\
422.01	0\\
423.01	0\\
424.01	0\\
425.01	0\\
426.01	0\\
427.01	0\\
428.01	0\\
429.01	0\\
430.01	0\\
431.01	0\\
432.01	0\\
433.01	0\\
434.01	0\\
435.01	0\\
436.01	0\\
437.01	0\\
438.01	0\\
439.01	0\\
440.01	0\\
441.01	0\\
442.01	0\\
443.01	0\\
444.01	0\\
445.01	0\\
446.01	0\\
447.01	0\\
448.01	0\\
449.01	0\\
450.01	0\\
451.01	0\\
452.01	0\\
453.01	0\\
454.01	0\\
455.01	0\\
456.01	0\\
457.01	0\\
458.01	0\\
459.01	0\\
460.01	0\\
461.01	0\\
462.01	0\\
463.01	0\\
464.01	0\\
465.01	0\\
466.01	0\\
467.01	0\\
468.01	0\\
469.01	0\\
470.01	0\\
471.01	0\\
472.01	0\\
473.01	0\\
474.01	0\\
475.01	0\\
476.01	0\\
477.01	0\\
478.01	0\\
479.01	0\\
480.01	0\\
481.01	0\\
482.01	0\\
483.01	0\\
484.01	0\\
485.01	0\\
486.01	0\\
487.01	0\\
488.01	0\\
489.01	0\\
490.01	0\\
491.01	0\\
492.01	0\\
493.01	0\\
494.01	0\\
495.01	0\\
496.01	0\\
497.01	0\\
498.01	0\\
499.01	0\\
500.01	0\\
501.01	0\\
502.01	0\\
503.01	0\\
504.01	0\\
505.01	0\\
506.01	0\\
507.01	0\\
508.01	0\\
509.01	0\\
510.01	0\\
511.01	0\\
512.01	0\\
513.01	0\\
514.01	0\\
515.01	0\\
516.01	0\\
517.01	0\\
518.01	0\\
519.01	0\\
520.01	0\\
521.01	0\\
522.01	1.76072062530383e-05\\
523.01	0.0001000531427582\\
524.01	0.000185428064959862\\
525.01	0.000273915626818395\\
526.01	0.000365718517235863\\
527.01	0.000461061166109798\\
528.01	0.000560192925065827\\
529.01	0.000663391817041978\\
530.01	0.000770968941636186\\
531.01	0.000883273738164617\\
532.01	0.00100070026104646\\
533.01	0.00112369466667597\\
534.01	0.00125276429462983\\
535.01	0.00138848841820502\\
536.01	0.00153153040644095\\
537.01	0.00168265200432605\\
538.01	0.00184273081035254\\
539.01	0.00201279319736091\\
540.01	0.00219403790053492\\
541.01	0.00238785532069443\\
542.01	0.00259226846870504\\
543.01	0.00280354376610866\\
544.01	0.0030219682635569\\
545.01	0.00324793355738788\\
546.01	0.00348188162292224\\
547.01	0.00372430787917213\\
548.01	0.00397577250198962\\
549.01	0.0042369103544857\\
550.01	0.00449075671486131\\
551.01	0.00462618461453768\\
552.01	0.00476421212870169\\
553.01	0.00490458243399326\\
554.01	0.00504693095053299\\
555.01	0.00519075452171885\\
556.01	0.00533537205239804\\
557.01	0.00547987416083386\\
558.01	0.00562305849481418\\
559.01	0.0057633450823129\\
560.01	0.00589868845148751\\
561.01	0.00603278110898117\\
562.01	0.00616897852510308\\
563.01	0.00630509330493087\\
564.01	0.00644045192015123\\
565.01	0.00657462204057506\\
566.01	0.00670711983291309\\
567.01	0.00683740812363565\\
568.01	0.0069648990353388\\
569.01	0.00708895843143271\\
570.01	0.00720891484742703\\
571.01	0.00732402511855551\\
572.01	0.00743343423629905\\
573.01	0.00753834307265663\\
574.01	0.00764066192485596\\
575.01	0.00774196963664896\\
576.01	0.00784232616627326\\
577.01	0.00794188335541293\\
578.01	0.00804077814675698\\
579.01	0.00813893755468454\\
580.01	0.00823628502515087\\
581.01	0.00833276801027717\\
582.01	0.00842836104623377\\
583.01	0.0085230664726219\\
584.01	0.00861691980849836\\
585.01	0.00871008730799935\\
586.01	0.00880256534465678\\
587.01	0.00889458764294722\\
588.01	0.00898632359220033\\
589.01	0.00907770521085289\\
590.01	0.00916865124237691\\
591.01	0.00925908767877544\\
592.01	0.00934895807582376\\
593.01	0.00943823031869177\\
594.01	0.00952690140896054\\
595.01	0.00961500159478332\\
596.01	0.00970259742227558\\
597.01	0.00978979309540728\\
598.01	0.00987673031330844\\
599.01	0.00996359465763311\\
599.02	0.00996446178033267\\
599.03	0.00996532763967337\\
599.04	0.00996619223192198\\
599.05	0.00996705555325088\\
599.06	0.00996791457106968\\
599.07	0.00996876854874041\\
599.08	0.00996961745928199\\
599.09	0.00997044779379745\\
599.1	0.00997125907535553\\
599.11	0.00997205095980902\\
599.12	0.00997282281992336\\
599.13	0.00997357449570154\\
599.14	0.00997430582442724\\
599.15	0.00997501664056067\\
599.16	0.00997570677562952\\
599.17	0.00997637605811486\\
599.18	0.00997702431333591\\
599.19	0.00997765136331818\\
599.2	0.00997825702666225\\
599.21	0.00997884164734969\\
599.22	0.00997940571276279\\
599.23	0.00997994904050891\\
599.24	0.00998047144445088\\
599.25	0.00998097350251616\\
599.26	0.00998145771758323\\
599.27	0.00998192391942239\\
599.28	0.00998238161802606\\
599.29	0.00998283374884875\\
599.3	0.00998328059264519\\
599.31	0.009983722405159\\
599.32	0.00998415915282055\\
599.33	0.0099845908027081\\
599.34	0.00998501732260721\\
599.35	0.00998543868107325\\
599.36	0.00998585484749724\\
599.37	0.00998626579217519\\
599.38	0.00998667148638112\\
599.39	0.00998707190244393\\
599.4	0.00998746701382841\\
599.41	0.00998785679522082\\
599.42	0.0099882412571037\\
599.43	0.00998862039174518\\
599.44	0.00998899417396288\\
599.45	0.00998936258018044\\
599.46	0.00998972558854081\\
599.47	0.00999008317902579\\
599.48	0.00999043533358202\\
599.49	0.00999078203429291\\
599.5	0.00999112326444559\\
599.51	0.00999145900953319\\
599.52	0.00999178925246696\\
599.53	0.00999211397373816\\
599.54	0.00999243312917881\\
599.55	0.00999274666306492\\
599.56	0.00999305451751694\\
599.57	0.00999335663412312\\
599.58	0.00999365295393464\\
599.59	0.00999394341746056\\
599.6	0.00999422796466266\\
599.61	0.00999450653495009\\
599.62	0.00999477906702091\\
599.63	0.00999504549894827\\
599.64	0.0099953057682215\\
599.65	0.00999555981174296\\
599.66	0.0099958075658224\\
599.67	0.00999604896616963\\
599.68	0.00999628394788682\\
599.69	0.00999651244546047\\
599.7	0.00999673439275298\\
599.71	0.0099969497229938\\
599.72	0.00999715836877016\\
599.73	0.00999736026201729\\
599.74	0.00999755533403083\\
599.75	0.00999774351546318\\
599.76	0.00999792473631449\\
599.77	0.00999809892592316\\
599.78	0.00999826601295585\\
599.79	0.00999842592539837\\
599.8	0.00999857859054469\\
599.81	0.00999872393498536\\
599.82	0.00999886188459496\\
599.83	0.00999899236451894\\
599.84	0.0099991152991595\\
599.85	0.00999923061216063\\
599.86	0.00999933822639223\\
599.87	0.00999943806393315\\
599.88	0.00999953004605316\\
599.89	0.00999961409319372\\
599.9	0.00999969012494744\\
599.91	0.00999975806003624\\
599.92	0.0099998178162879\\
599.93	0.00999986931061113\\
599.94	0.00999991245896887\\
599.95	0.00999994717634967\\
599.96	0.00999997337673722\\
599.97	0.00999999097307755\\
599.98	0.00999999987724406\\
599.99	0.01\\
600	0.01\\
};
\addplot [color=mycolor12,solid,forget plot]
  table[row sep=crcr]{%
0.01	0\\
1.01	0\\
2.01	0\\
3.01	0\\
4.01	0\\
5.01	0\\
6.01	0\\
7.01	0\\
8.01	0\\
9.01	0\\
10.01	0\\
11.01	0\\
12.01	0\\
13.01	0\\
14.01	0\\
15.01	0\\
16.01	0\\
17.01	0\\
18.01	0\\
19.01	0\\
20.01	0\\
21.01	0\\
22.01	0\\
23.01	0\\
24.01	0\\
25.01	0\\
26.01	0\\
27.01	0\\
28.01	0\\
29.01	0\\
30.01	0\\
31.01	0\\
32.01	0\\
33.01	0\\
34.01	0\\
35.01	0\\
36.01	0\\
37.01	0\\
38.01	0\\
39.01	0\\
40.01	0\\
41.01	0\\
42.01	0\\
43.01	0\\
44.01	0\\
45.01	0\\
46.01	0\\
47.01	0\\
48.01	0\\
49.01	0\\
50.01	0\\
51.01	0\\
52.01	0\\
53.01	0\\
54.01	0\\
55.01	0\\
56.01	0\\
57.01	0\\
58.01	0\\
59.01	0\\
60.01	0\\
61.01	0\\
62.01	0\\
63.01	0\\
64.01	0\\
65.01	0\\
66.01	0\\
67.01	0\\
68.01	0\\
69.01	0\\
70.01	0\\
71.01	0\\
72.01	0\\
73.01	0\\
74.01	0\\
75.01	0\\
76.01	0\\
77.01	0\\
78.01	0\\
79.01	0\\
80.01	0\\
81.01	0\\
82.01	0\\
83.01	0\\
84.01	0\\
85.01	0\\
86.01	0\\
87.01	0\\
88.01	0\\
89.01	0\\
90.01	0\\
91.01	0\\
92.01	0\\
93.01	0\\
94.01	0\\
95.01	0\\
96.01	0\\
97.01	0\\
98.01	0\\
99.01	0\\
100.01	0\\
101.01	0\\
102.01	0\\
103.01	0\\
104.01	0\\
105.01	0\\
106.01	0\\
107.01	0\\
108.01	0\\
109.01	0\\
110.01	0\\
111.01	0\\
112.01	0\\
113.01	0\\
114.01	0\\
115.01	0\\
116.01	0\\
117.01	0\\
118.01	0\\
119.01	0\\
120.01	0\\
121.01	0\\
122.01	0\\
123.01	0\\
124.01	0\\
125.01	0\\
126.01	0\\
127.01	0\\
128.01	0\\
129.01	0\\
130.01	0\\
131.01	0\\
132.01	0\\
133.01	0\\
134.01	0\\
135.01	0\\
136.01	0\\
137.01	0\\
138.01	0\\
139.01	0\\
140.01	0\\
141.01	0\\
142.01	0\\
143.01	0\\
144.01	0\\
145.01	0\\
146.01	0\\
147.01	0\\
148.01	0\\
149.01	0\\
150.01	0\\
151.01	0\\
152.01	0\\
153.01	0\\
154.01	0\\
155.01	0\\
156.01	0\\
157.01	0\\
158.01	0\\
159.01	0\\
160.01	0\\
161.01	0\\
162.01	0\\
163.01	0\\
164.01	0\\
165.01	0\\
166.01	0\\
167.01	0\\
168.01	0\\
169.01	0\\
170.01	0\\
171.01	0\\
172.01	0\\
173.01	0\\
174.01	0\\
175.01	0\\
176.01	0\\
177.01	0\\
178.01	0\\
179.01	0\\
180.01	0\\
181.01	0\\
182.01	0\\
183.01	0\\
184.01	0\\
185.01	0\\
186.01	0\\
187.01	0\\
188.01	0\\
189.01	0\\
190.01	0\\
191.01	0\\
192.01	0\\
193.01	0\\
194.01	0\\
195.01	0\\
196.01	0\\
197.01	0\\
198.01	0\\
199.01	0\\
200.01	0\\
201.01	0\\
202.01	0\\
203.01	0\\
204.01	0\\
205.01	0\\
206.01	0\\
207.01	0\\
208.01	0\\
209.01	0\\
210.01	0\\
211.01	0\\
212.01	0\\
213.01	0\\
214.01	0\\
215.01	0\\
216.01	0\\
217.01	0\\
218.01	0\\
219.01	0\\
220.01	0\\
221.01	0\\
222.01	0\\
223.01	0\\
224.01	0\\
225.01	0\\
226.01	0\\
227.01	0\\
228.01	0\\
229.01	0\\
230.01	0\\
231.01	0\\
232.01	0\\
233.01	0\\
234.01	0\\
235.01	0\\
236.01	0\\
237.01	0\\
238.01	0\\
239.01	0\\
240.01	0\\
241.01	0\\
242.01	0\\
243.01	0\\
244.01	0\\
245.01	0\\
246.01	0\\
247.01	0\\
248.01	0\\
249.01	0\\
250.01	0\\
251.01	0\\
252.01	0\\
253.01	0\\
254.01	0\\
255.01	0\\
256.01	0\\
257.01	0\\
258.01	0\\
259.01	0\\
260.01	0\\
261.01	0\\
262.01	0\\
263.01	0\\
264.01	0\\
265.01	0\\
266.01	0\\
267.01	0\\
268.01	0\\
269.01	0\\
270.01	0\\
271.01	0\\
272.01	0\\
273.01	0\\
274.01	0\\
275.01	0\\
276.01	0\\
277.01	0\\
278.01	0\\
279.01	0\\
280.01	0\\
281.01	0\\
282.01	0\\
283.01	0\\
284.01	0\\
285.01	0\\
286.01	0\\
287.01	0\\
288.01	0\\
289.01	0\\
290.01	0\\
291.01	0\\
292.01	0\\
293.01	0\\
294.01	0\\
295.01	0\\
296.01	0\\
297.01	0\\
298.01	0\\
299.01	0\\
300.01	0\\
301.01	0\\
302.01	0\\
303.01	0\\
304.01	0\\
305.01	0\\
306.01	0\\
307.01	0\\
308.01	0\\
309.01	0\\
310.01	0\\
311.01	0\\
312.01	0\\
313.01	0\\
314.01	0\\
315.01	0\\
316.01	0\\
317.01	0\\
318.01	0\\
319.01	0\\
320.01	0\\
321.01	0\\
322.01	0\\
323.01	0\\
324.01	0\\
325.01	0\\
326.01	0\\
327.01	0\\
328.01	0\\
329.01	0\\
330.01	0\\
331.01	0\\
332.01	0\\
333.01	0\\
334.01	0\\
335.01	0\\
336.01	0\\
337.01	0\\
338.01	0\\
339.01	0\\
340.01	0\\
341.01	0\\
342.01	0\\
343.01	0\\
344.01	0\\
345.01	0\\
346.01	0\\
347.01	0\\
348.01	0\\
349.01	0\\
350.01	0\\
351.01	0\\
352.01	0\\
353.01	0\\
354.01	0\\
355.01	0\\
356.01	0\\
357.01	0\\
358.01	0\\
359.01	0\\
360.01	0\\
361.01	0\\
362.01	0\\
363.01	0\\
364.01	0\\
365.01	0\\
366.01	0\\
367.01	0\\
368.01	0\\
369.01	0\\
370.01	0\\
371.01	0\\
372.01	0\\
373.01	0\\
374.01	0\\
375.01	0\\
376.01	0\\
377.01	0\\
378.01	0\\
379.01	0\\
380.01	0\\
381.01	0\\
382.01	0\\
383.01	0\\
384.01	0\\
385.01	0\\
386.01	0\\
387.01	0\\
388.01	0\\
389.01	0\\
390.01	0\\
391.01	0\\
392.01	0\\
393.01	0\\
394.01	0\\
395.01	0\\
396.01	0\\
397.01	0\\
398.01	0\\
399.01	0\\
400.01	0\\
401.01	0\\
402.01	0\\
403.01	0\\
404.01	0\\
405.01	0\\
406.01	0\\
407.01	0\\
408.01	0\\
409.01	0\\
410.01	0\\
411.01	0\\
412.01	0\\
413.01	0\\
414.01	0\\
415.01	0\\
416.01	0\\
417.01	0\\
418.01	0\\
419.01	0\\
420.01	0\\
421.01	0\\
422.01	0\\
423.01	0\\
424.01	0\\
425.01	0\\
426.01	0\\
427.01	0\\
428.01	0\\
429.01	0\\
430.01	0\\
431.01	0\\
432.01	0\\
433.01	0\\
434.01	0\\
435.01	0\\
436.01	0\\
437.01	0\\
438.01	0\\
439.01	0\\
440.01	0\\
441.01	0\\
442.01	0\\
443.01	0\\
444.01	0\\
445.01	0\\
446.01	0\\
447.01	0\\
448.01	0\\
449.01	0\\
450.01	0\\
451.01	0\\
452.01	0\\
453.01	0\\
454.01	0\\
455.01	0\\
456.01	0\\
457.01	0\\
458.01	0\\
459.01	0\\
460.01	0\\
461.01	0\\
462.01	0\\
463.01	0\\
464.01	0\\
465.01	0\\
466.01	0\\
467.01	0\\
468.01	0\\
469.01	0\\
470.01	0\\
471.01	0\\
472.01	0\\
473.01	0\\
474.01	0\\
475.01	0\\
476.01	0\\
477.01	0\\
478.01	0\\
479.01	0\\
480.01	0\\
481.01	0\\
482.01	0\\
483.01	0\\
484.01	0\\
485.01	0\\
486.01	0\\
487.01	0\\
488.01	0\\
489.01	0\\
490.01	0\\
491.01	0\\
492.01	0\\
493.01	0\\
494.01	0\\
495.01	0\\
496.01	0\\
497.01	0\\
498.01	0\\
499.01	0\\
500.01	0\\
501.01	0\\
502.01	0\\
503.01	0\\
504.01	0\\
505.01	0\\
506.01	0\\
507.01	0\\
508.01	0\\
509.01	0\\
510.01	0\\
511.01	0\\
512.01	0\\
513.01	0\\
514.01	0\\
515.01	0\\
516.01	0\\
517.01	0\\
518.01	0\\
519.01	0\\
520.01	0\\
521.01	0\\
522.01	0\\
523.01	0\\
524.01	0\\
525.01	0\\
526.01	0\\
527.01	0\\
528.01	0\\
529.01	0\\
530.01	0\\
531.01	0\\
532.01	0\\
533.01	0\\
534.01	0\\
535.01	0\\
536.01	0\\
537.01	0\\
538.01	0\\
539.01	0\\
540.01	0\\
541.01	0\\
542.01	0\\
543.01	0\\
544.01	0\\
545.01	0\\
546.01	0\\
547.01	0\\
548.01	0\\
549.01	0\\
550.01	1.76577693769484e-05\\
551.01	0.000163601594765743\\
552.01	0.000316552654936606\\
553.01	0.000477202157931674\\
554.01	0.000646343853199616\\
555.01	0.000824893090787756\\
556.01	0.00101390996811625\\
557.01	0.0012146275223706\\
558.01	0.00142848601941258\\
559.01	0.00165717309688124\\
560.01	0.00190263038585375\\
561.01	0.00216086094684066\\
562.01	0.00242801592023746\\
563.01	0.00270459001499955\\
564.01	0.00299112571883046\\
565.01	0.00328819303298395\\
566.01	0.00359638629634986\\
567.01	0.00391632018100313\\
568.01	0.00424862352259442\\
569.01	0.0045939301115959\\
570.01	0.00495286578552835\\
571.01	0.00532608000731284\\
572.01	0.00568646434170741\\
573.01	0.00588160416759255\\
574.01	0.00607494172253639\\
575.01	0.00626304604111949\\
576.01	0.00644372525962685\\
577.01	0.0066149467502195\\
578.01	0.00678349529972556\\
579.01	0.00695169599783461\\
580.01	0.00711915538111959\\
581.01	0.00728547230152028\\
582.01	0.00745025912556778\\
583.01	0.00761317424151823\\
584.01	0.00777397103082797\\
585.01	0.00793312457297383\\
586.01	0.00809207708383814\\
587.01	0.00825067850764279\\
588.01	0.0084088581555231\\
589.01	0.00856615818895188\\
590.01	0.00872204584445263\\
591.01	0.00887592796155797\\
592.01	0.00902715589244679\\
593.01	0.00917502985911509\\
594.01	0.00931880522180199\\
595.01	0.0094577020919197\\
596.01	0.00959091850522286\\
597.01	0.00971763773133404\\
598.01	0.00983693415442488\\
599.01	0.00994672806842535\\
599.02	0.00994775855856473\\
599.03	0.00994878730830944\\
599.04	0.00994981430099912\\
599.05	0.0099508395197916\\
599.06	0.00995186294794709\\
599.07	0.00995288456862204\\
599.08	0.00995390436481037\\
599.09	0.00995492232059677\\
599.1	0.00995593841996281\\
599.11	0.00995695264678405\\
599.12	0.00995796498486604\\
599.13	0.00995897541791202\\
599.14	0.00995998392953529\\
599.15	0.0099609905032724\\
599.16	0.00996199512259775\\
599.17	0.00996299777093928\\
599.18	0.00996399843169559\\
599.19	0.00996499708825443\\
599.2	0.00996599372401278\\
599.21	0.00996698779596926\\
599.22	0.00996797861736903\\
599.23	0.00996896616765423\\
599.24	0.00996995042631169\\
599.25	0.00997093060729154\\
599.26	0.00997190400413337\\
599.27	0.00997287058004875\\
599.28	0.00997382061001036\\
599.29	0.00997475095702614\\
599.3	0.00997566116920662\\
599.31	0.00997655082052863\\
599.32	0.0099774197729459\\
599.33	0.00997826788498202\\
599.34	0.00997909501386427\\
599.35	0.00997990101551993\\
599.36	0.00998068574457414\\
599.37	0.00998144905434973\\
599.38	0.00998219079686904\\
599.39	0.00998291082285798\\
599.4	0.0099836089817526\\
599.41	0.00998428512170819\\
599.42	0.00998493905288865\\
599.43	0.00998557060322691\\
599.44	0.00998617961841143\\
599.45	0.00998676594296101\\
599.46	0.0099873294202526\\
599.47	0.00998786989255357\\
599.48	0.00998838720105844\\
599.49	0.00998888190498132\\
599.5	0.00998935417965473\\
599.51	0.00998980387068191\\
599.52	0.0099902324022735\\
599.53	0.00999064107301789\\
599.54	0.00999103756789193\\
599.55	0.0099914252526318\\
599.56	0.0099918045563071\\
599.57	0.00999217544917753\\
599.58	0.00999253790543692\\
599.59	0.00999289190353504\\
599.6	0.00999323742652058\\
599.61	0.00999357446240663\\
599.62	0.00999390303937479\\
599.63	0.00999422316746411\\
599.64	0.00999453485133593\\
599.65	0.00999483808574109\\
599.66	0.00999513286499583\\
599.67	0.00999541918934847\\
599.68	0.00999569706543861\\
599.69	0.00999596650678687\\
599.7	0.009996227534317\\
599.71	0.00999648017691256\\
599.72	0.00999672447201024\\
599.73	0.0099969604662323\\
599.74	0.00999718812015448\\
599.75	0.00999740737458996\\
599.76	0.00999761817122877\\
599.77	0.00999782045273265\\
599.78	0.00999801416283588\\
599.79	0.00999819924009764\\
599.8	0.00999837562306064\\
599.81	0.00999854325056222\\
599.82	0.00999870206199949\\
599.83	0.00999885199763102\\
599.84	0.00999899299866258\\
599.85	0.00999912500733794\\
599.86	0.009999247967035\\
599.87	0.0099993618223675\\
599.88	0.00999946651929288\\
599.89	0.0099995620052265\\
599.9	0.00999964822916273\\
599.91	0.00999972514180333\\
599.92	0.00999979269569369\\
599.93	0.00999985084536732\\
599.94	0.00999989954749936\\
599.95	0.00999993876106953\\
599.96	0.00999996844753539\\
599.97	0.00999998857101644\\
599.98	0.00999999909849005\\
599.99	0.01\\
600	0.01\\
};
\addplot [color=mycolor13,solid,forget plot]
  table[row sep=crcr]{%
0.01	0\\
1.01	0\\
2.01	0\\
3.01	0\\
4.01	0\\
5.01	0\\
6.01	0\\
7.01	0\\
8.01	0\\
9.01	0\\
10.01	0\\
11.01	0\\
12.01	0\\
13.01	0\\
14.01	0\\
15.01	0\\
16.01	0\\
17.01	0\\
18.01	0\\
19.01	0\\
20.01	0\\
21.01	0\\
22.01	0\\
23.01	0\\
24.01	0\\
25.01	0\\
26.01	0\\
27.01	0\\
28.01	0\\
29.01	0\\
30.01	0\\
31.01	0\\
32.01	0\\
33.01	0\\
34.01	0\\
35.01	0\\
36.01	0\\
37.01	0\\
38.01	0\\
39.01	0\\
40.01	0\\
41.01	0\\
42.01	0\\
43.01	0\\
44.01	0\\
45.01	0\\
46.01	0\\
47.01	0\\
48.01	0\\
49.01	0\\
50.01	0\\
51.01	0\\
52.01	0\\
53.01	0\\
54.01	0\\
55.01	0\\
56.01	0\\
57.01	0\\
58.01	0\\
59.01	0\\
60.01	0\\
61.01	0\\
62.01	0\\
63.01	0\\
64.01	0\\
65.01	0\\
66.01	0\\
67.01	0\\
68.01	0\\
69.01	0\\
70.01	0\\
71.01	0\\
72.01	0\\
73.01	0\\
74.01	0\\
75.01	0\\
76.01	0\\
77.01	0\\
78.01	0\\
79.01	0\\
80.01	0\\
81.01	0\\
82.01	0\\
83.01	0\\
84.01	0\\
85.01	0\\
86.01	0\\
87.01	0\\
88.01	0\\
89.01	0\\
90.01	0\\
91.01	0\\
92.01	0\\
93.01	0\\
94.01	0\\
95.01	0\\
96.01	0\\
97.01	0\\
98.01	0\\
99.01	0\\
100.01	0\\
101.01	0\\
102.01	0\\
103.01	0\\
104.01	0\\
105.01	0\\
106.01	0\\
107.01	0\\
108.01	0\\
109.01	0\\
110.01	0\\
111.01	0\\
112.01	0\\
113.01	0\\
114.01	0\\
115.01	0\\
116.01	0\\
117.01	0\\
118.01	0\\
119.01	0\\
120.01	0\\
121.01	0\\
122.01	0\\
123.01	0\\
124.01	0\\
125.01	0\\
126.01	0\\
127.01	0\\
128.01	0\\
129.01	0\\
130.01	0\\
131.01	0\\
132.01	0\\
133.01	0\\
134.01	0\\
135.01	0\\
136.01	0\\
137.01	0\\
138.01	0\\
139.01	0\\
140.01	0\\
141.01	0\\
142.01	0\\
143.01	0\\
144.01	0\\
145.01	0\\
146.01	0\\
147.01	0\\
148.01	0\\
149.01	0\\
150.01	0\\
151.01	0\\
152.01	0\\
153.01	0\\
154.01	0\\
155.01	0\\
156.01	0\\
157.01	0\\
158.01	0\\
159.01	0\\
160.01	0\\
161.01	0\\
162.01	0\\
163.01	0\\
164.01	0\\
165.01	0\\
166.01	0\\
167.01	0\\
168.01	0\\
169.01	0\\
170.01	0\\
171.01	0\\
172.01	0\\
173.01	0\\
174.01	0\\
175.01	0\\
176.01	0\\
177.01	0\\
178.01	0\\
179.01	0\\
180.01	0\\
181.01	0\\
182.01	0\\
183.01	0\\
184.01	0\\
185.01	0\\
186.01	0\\
187.01	0\\
188.01	0\\
189.01	0\\
190.01	0\\
191.01	0\\
192.01	0\\
193.01	0\\
194.01	0\\
195.01	0\\
196.01	0\\
197.01	0\\
198.01	0\\
199.01	0\\
200.01	0\\
201.01	0\\
202.01	0\\
203.01	0\\
204.01	0\\
205.01	0\\
206.01	0\\
207.01	0\\
208.01	0\\
209.01	0\\
210.01	0\\
211.01	0\\
212.01	0\\
213.01	0\\
214.01	0\\
215.01	0\\
216.01	0\\
217.01	0\\
218.01	0\\
219.01	0\\
220.01	0\\
221.01	0\\
222.01	0\\
223.01	0\\
224.01	0\\
225.01	0\\
226.01	0\\
227.01	0\\
228.01	0\\
229.01	0\\
230.01	0\\
231.01	0\\
232.01	0\\
233.01	0\\
234.01	0\\
235.01	0\\
236.01	0\\
237.01	0\\
238.01	0\\
239.01	0\\
240.01	0\\
241.01	0\\
242.01	0\\
243.01	0\\
244.01	0\\
245.01	0\\
246.01	0\\
247.01	0\\
248.01	0\\
249.01	0\\
250.01	0\\
251.01	0\\
252.01	0\\
253.01	0\\
254.01	0\\
255.01	0\\
256.01	0\\
257.01	0\\
258.01	0\\
259.01	0\\
260.01	0\\
261.01	0\\
262.01	0\\
263.01	0\\
264.01	0\\
265.01	0\\
266.01	0\\
267.01	0\\
268.01	0\\
269.01	0\\
270.01	0\\
271.01	0\\
272.01	0\\
273.01	0\\
274.01	0\\
275.01	0\\
276.01	0\\
277.01	0\\
278.01	0\\
279.01	0\\
280.01	0\\
281.01	0\\
282.01	0\\
283.01	0\\
284.01	0\\
285.01	0\\
286.01	0\\
287.01	0\\
288.01	0\\
289.01	0\\
290.01	0\\
291.01	0\\
292.01	0\\
293.01	0\\
294.01	0\\
295.01	0\\
296.01	0\\
297.01	0\\
298.01	0\\
299.01	0\\
300.01	0\\
301.01	0\\
302.01	0\\
303.01	0\\
304.01	0\\
305.01	0\\
306.01	0\\
307.01	0\\
308.01	0\\
309.01	0\\
310.01	0\\
311.01	0\\
312.01	0\\
313.01	0\\
314.01	0\\
315.01	0\\
316.01	0\\
317.01	0\\
318.01	0\\
319.01	0\\
320.01	0\\
321.01	0\\
322.01	0\\
323.01	0\\
324.01	0\\
325.01	0\\
326.01	0\\
327.01	0\\
328.01	0\\
329.01	0\\
330.01	0\\
331.01	0\\
332.01	0\\
333.01	0\\
334.01	0\\
335.01	0\\
336.01	0\\
337.01	0\\
338.01	0\\
339.01	0\\
340.01	0\\
341.01	0\\
342.01	0\\
343.01	0\\
344.01	0\\
345.01	0\\
346.01	0\\
347.01	0\\
348.01	0\\
349.01	0\\
350.01	0\\
351.01	0\\
352.01	0\\
353.01	0\\
354.01	0\\
355.01	0\\
356.01	0\\
357.01	0\\
358.01	0\\
359.01	0\\
360.01	0\\
361.01	0\\
362.01	0\\
363.01	0\\
364.01	0\\
365.01	0\\
366.01	0\\
367.01	0\\
368.01	0\\
369.01	0\\
370.01	0\\
371.01	0\\
372.01	0\\
373.01	0\\
374.01	0\\
375.01	0\\
376.01	0\\
377.01	0\\
378.01	0\\
379.01	0\\
380.01	0\\
381.01	0\\
382.01	0\\
383.01	0\\
384.01	0\\
385.01	0\\
386.01	0\\
387.01	0\\
388.01	0\\
389.01	0\\
390.01	0\\
391.01	0\\
392.01	0\\
393.01	0\\
394.01	0\\
395.01	0\\
396.01	0\\
397.01	0\\
398.01	0\\
399.01	0\\
400.01	0\\
401.01	0\\
402.01	0\\
403.01	0\\
404.01	0\\
405.01	0\\
406.01	0\\
407.01	0\\
408.01	0\\
409.01	0\\
410.01	0\\
411.01	0\\
412.01	0\\
413.01	0\\
414.01	0\\
415.01	0\\
416.01	0\\
417.01	0\\
418.01	0\\
419.01	0\\
420.01	0\\
421.01	0\\
422.01	0\\
423.01	0\\
424.01	0\\
425.01	0\\
426.01	0\\
427.01	0\\
428.01	0\\
429.01	0\\
430.01	0\\
431.01	0\\
432.01	0\\
433.01	0\\
434.01	0\\
435.01	0\\
436.01	0\\
437.01	0\\
438.01	0\\
439.01	0\\
440.01	0\\
441.01	0\\
442.01	0\\
443.01	0\\
444.01	0\\
445.01	0\\
446.01	0\\
447.01	0\\
448.01	0\\
449.01	0\\
450.01	0\\
451.01	0\\
452.01	0\\
453.01	0\\
454.01	0\\
455.01	0\\
456.01	0\\
457.01	0\\
458.01	0\\
459.01	0\\
460.01	0\\
461.01	0\\
462.01	0\\
463.01	0\\
464.01	0\\
465.01	0\\
466.01	0\\
467.01	0\\
468.01	0\\
469.01	0\\
470.01	0\\
471.01	0\\
472.01	0\\
473.01	0\\
474.01	0\\
475.01	0\\
476.01	0\\
477.01	0\\
478.01	0\\
479.01	0\\
480.01	0\\
481.01	0\\
482.01	0\\
483.01	0\\
484.01	0\\
485.01	0\\
486.01	0\\
487.01	0\\
488.01	0\\
489.01	0\\
490.01	0\\
491.01	0\\
492.01	0\\
493.01	0\\
494.01	0\\
495.01	0\\
496.01	0\\
497.01	0\\
498.01	0\\
499.01	0\\
500.01	0\\
501.01	0\\
502.01	0\\
503.01	0\\
504.01	0\\
505.01	0\\
506.01	0\\
507.01	0\\
508.01	0\\
509.01	0\\
510.01	0\\
511.01	0\\
512.01	0\\
513.01	0\\
514.01	0\\
515.01	0\\
516.01	0\\
517.01	0\\
518.01	0\\
519.01	0\\
520.01	0\\
521.01	0\\
522.01	0\\
523.01	0\\
524.01	0\\
525.01	0\\
526.01	0\\
527.01	0\\
528.01	0\\
529.01	0\\
530.01	0\\
531.01	0\\
532.01	0\\
533.01	0\\
534.01	0\\
535.01	0\\
536.01	0\\
537.01	0\\
538.01	0\\
539.01	0\\
540.01	0\\
541.01	0\\
542.01	0\\
543.01	0\\
544.01	0\\
545.01	0\\
546.01	0\\
547.01	0\\
548.01	0\\
549.01	0\\
550.01	0\\
551.01	0\\
552.01	0\\
553.01	0\\
554.01	0\\
555.01	0\\
556.01	0\\
557.01	0\\
558.01	0\\
559.01	0\\
560.01	0\\
561.01	0\\
562.01	0\\
563.01	0\\
564.01	0\\
565.01	0\\
566.01	0\\
567.01	0\\
568.01	0\\
569.01	0\\
570.01	0\\
571.01	0\\
572.01	2.7816904777541e-05\\
573.01	0.000234469092053179\\
574.01	0.000454399221483506\\
575.01	0.000689399833623931\\
576.01	0.0009415921916204\\
577.01	0.00121282930990188\\
578.01	0.00149614830095804\\
579.01	0.00178923528638872\\
580.01	0.00209247197993115\\
581.01	0.00240619622796399\\
582.01	0.00273067721320507\\
583.01	0.00306608229716627\\
584.01	0.00341243284835259\\
585.01	0.00376898156286783\\
586.01	0.00413388772680092\\
587.01	0.00450662171783373\\
588.01	0.00488671348865318\\
589.01	0.00527426155529603\\
590.01	0.0056693963527141\\
591.01	0.00607223737509095\\
592.01	0.00648288259282408\\
593.01	0.00690139908379193\\
594.01	0.00732781618824651\\
595.01	0.00776211870274235\\
596.01	0.00820423876877381\\
597.01	0.00865403604580581\\
598.01	0.0091111651447944\\
599.01	0.00957383789362019\\
599.02	0.0095784699445533\\
599.03	0.00958310152356088\\
599.04	0.009587732601833\\
599.05	0.00959236314982192\\
599.06	0.00959699313722105\\
599.07	0.00960162253294322\\
599.08	0.00960625130509822\\
599.09	0.00961087942096949\\
599.1	0.00961550684699015\\
599.11	0.00962013354871818\\
599.12	0.00962475949081084\\
599.13	0.00962938463699814\\
599.14	0.00963400895005545\\
599.15	0.00963863239177514\\
599.16	0.00964325492293724\\
599.17	0.00964787650327905\\
599.18	0.00965249709146373\\
599.19	0.00965711664504773\\
599.2	0.00966173512044705\\
599.21	0.00966635247295013\\
599.22	0.00967096865669379\\
599.23	0.0096755836245642\\
599.24	0.00968019732815788\\
599.25	0.00968480971781115\\
599.26	0.00968942074273276\\
599.27	0.00969403035071384\\
599.28	0.00969863848897433\\
599.29	0.00970324510350724\\
599.3	0.00970785013878026\\
599.31	0.00971245353768087\\
599.32	0.0097170552414366\\
599.33	0.00972165518956062\\
599.34	0.00972625331979421\\
599.35	0.00973084956804691\\
599.36	0.00973544386833419\\
599.37	0.00974003615271259\\
599.38	0.00974462635121213\\
599.39	0.00974921439176584\\
599.4	0.00975380020013637\\
599.41	0.00975838369983937\\
599.42	0.00976296481206596\\
599.43	0.00976754345559703\\
599.44	0.00977211954671424\\
599.45	0.00977669299910894\\
599.46	0.00978126372378694\\
599.47	0.00978583162896891\\
599.48	0.00979039661998624\\
599.49	0.00979495788351871\\
599.5	0.00979951498770784\\
599.51	0.00980406782641168\\
599.52	0.00980861471572478\\
599.53	0.00981315408788168\\
599.54	0.00981767801023239\\
599.55	0.00982218287897037\\
599.56	0.00982666802718657\\
599.57	0.00983113324532572\\
599.58	0.00983557831854249\\
599.59	0.00984000302649006\\
599.6	0.00984440714309827\\
599.61	0.00984879043634082\\
599.62	0.00985315263350748\\
599.63	0.00985749347945079\\
599.64	0.00986181272388278\\
599.65	0.00986611010955735\\
599.66	0.00987038537198554\\
599.67	0.00987463823912518\\
599.68	0.00987886843105376\\
599.69	0.00988307565962328\\
599.7	0.00988725962809584\\
599.71	0.00989142003075855\\
599.72	0.00989555655251621\\
599.73	0.00989966886846011\\
599.74	0.00990375664357885\\
599.75	0.00990781953233169\\
599.76	0.00991185717815917\\
599.77	0.00991586921296373\\
599.78	0.00991985525655799\\
599.79	0.00992381491609016\\
599.8	0.00992774778542166\\
599.81	0.00993165344446322\\
599.82	0.00993553145846567\\
599.83	0.00993938137726134\\
599.84	0.00994320273445221\\
599.85	0.00994699504653957\\
599.86	0.00995075781198986\\
599.87	0.00995449051023017\\
599.88	0.00995819260056652\\
599.89	0.00996186352101685\\
599.9	0.00996550268704979\\
599.91	0.00996910949021901\\
599.92	0.00997268329668161\\
599.93	0.00997622344558731\\
599.94	0.00997972924732342\\
599.95	0.00998319998159831\\
599.96	0.00998663489534342\\
599.97	0.0099900332004109\\
599.98	0.00999339407104016\\
599.99	0.00999671664106234\\
600	0.01\\
};
\addplot [color=mycolor14,solid,forget plot]
  table[row sep=crcr]{%
0.01	0.01\\
1.01	0.01\\
2.01	0.01\\
3.01	0.01\\
4.01	0.01\\
5.01	0.01\\
6.01	0.01\\
7.01	0.01\\
8.01	0.01\\
9.01	0.01\\
10.01	0.01\\
11.01	0.01\\
12.01	0.01\\
13.01	0.01\\
14.01	0.01\\
15.01	0.01\\
16.01	0.01\\
17.01	0.01\\
18.01	0.01\\
19.01	0.01\\
20.01	0.01\\
21.01	0.01\\
22.01	0.01\\
23.01	0.01\\
24.01	0.01\\
25.01	0.01\\
26.01	0.01\\
27.01	0.01\\
28.01	0.01\\
29.01	0.01\\
30.01	0.01\\
31.01	0.01\\
32.01	0.01\\
33.01	0.01\\
34.01	0.01\\
35.01	0.01\\
36.01	0.01\\
37.01	0.01\\
38.01	0.01\\
39.01	0.01\\
40.01	0.01\\
41.01	0.01\\
42.01	0.01\\
43.01	0.01\\
44.01	0.01\\
45.01	0.01\\
46.01	0.01\\
47.01	0.01\\
48.01	0.01\\
49.01	0.01\\
50.01	0.01\\
51.01	0.01\\
52.01	0.01\\
53.01	0.01\\
54.01	0.01\\
55.01	0.01\\
56.01	0.01\\
57.01	0.01\\
58.01	0.01\\
59.01	0.01\\
60.01	0.01\\
61.01	0.01\\
62.01	0.01\\
63.01	0.01\\
64.01	0.01\\
65.01	0.01\\
66.01	0.01\\
67.01	0.01\\
68.01	0.01\\
69.01	0.01\\
70.01	0.01\\
71.01	0.01\\
72.01	0.01\\
73.01	0.01\\
74.01	0.01\\
75.01	0.01\\
76.01	0.01\\
77.01	0.01\\
78.01	0.01\\
79.01	0.01\\
80.01	0.01\\
81.01	0.01\\
82.01	0.01\\
83.01	0.01\\
84.01	0.01\\
85.01	0.01\\
86.01	0.01\\
87.01	0.01\\
88.01	0.01\\
89.01	0.01\\
90.01	0.01\\
91.01	0.01\\
92.01	0.01\\
93.01	0.01\\
94.01	0.01\\
95.01	0.01\\
96.01	0.01\\
97.01	0.01\\
98.01	0.01\\
99.01	0.01\\
100.01	0.01\\
101.01	0.01\\
102.01	0.01\\
103.01	0.01\\
104.01	0.01\\
105.01	0.01\\
106.01	0.01\\
107.01	0.01\\
108.01	0.01\\
109.01	0.01\\
110.01	0.01\\
111.01	0.01\\
112.01	0.01\\
113.01	0.01\\
114.01	0.01\\
115.01	0.01\\
116.01	0.01\\
117.01	0.01\\
118.01	0.01\\
119.01	0.01\\
120.01	0.01\\
121.01	0.01\\
122.01	0.01\\
123.01	0.01\\
124.01	0.01\\
125.01	0.01\\
126.01	0.01\\
127.01	0.01\\
128.01	0.01\\
129.01	0.01\\
130.01	0.01\\
131.01	0.01\\
132.01	0.01\\
133.01	0.01\\
134.01	0.01\\
135.01	0.01\\
136.01	0.01\\
137.01	0.01\\
138.01	0.01\\
139.01	0.01\\
140.01	0.01\\
141.01	0.01\\
142.01	0.01\\
143.01	0.01\\
144.01	0.01\\
145.01	0.01\\
146.01	0.01\\
147.01	0.01\\
148.01	0.01\\
149.01	0.01\\
150.01	0.01\\
151.01	0.01\\
152.01	0.01\\
153.01	0.01\\
154.01	0.01\\
155.01	0.01\\
156.01	0.01\\
157.01	0.01\\
158.01	0.01\\
159.01	0.01\\
160.01	0.01\\
161.01	0.01\\
162.01	0.01\\
163.01	0.01\\
164.01	0.01\\
165.01	0.01\\
166.01	0.01\\
167.01	0.01\\
168.01	0.01\\
169.01	0.01\\
170.01	0.01\\
171.01	0.01\\
172.01	0.01\\
173.01	0.01\\
174.01	0.01\\
175.01	0.01\\
176.01	0.01\\
177.01	0.01\\
178.01	0.01\\
179.01	0.01\\
180.01	0.01\\
181.01	0.01\\
182.01	0.01\\
183.01	0.01\\
184.01	0.01\\
185.01	0.01\\
186.01	0.01\\
187.01	0.01\\
188.01	0.01\\
189.01	0.01\\
190.01	0.01\\
191.01	0.01\\
192.01	0.01\\
193.01	0.01\\
194.01	0.01\\
195.01	0.01\\
196.01	0.01\\
197.01	0.01\\
198.01	0.01\\
199.01	0.01\\
200.01	0.01\\
201.01	0.01\\
202.01	0.01\\
203.01	0.01\\
204.01	0.01\\
205.01	0.01\\
206.01	0.01\\
207.01	0.01\\
208.01	0.01\\
209.01	0.01\\
210.01	0.01\\
211.01	0.01\\
212.01	0.01\\
213.01	0.01\\
214.01	0.01\\
215.01	0.01\\
216.01	0.01\\
217.01	0.01\\
218.01	0.01\\
219.01	0.01\\
220.01	0.01\\
221.01	0.01\\
222.01	0.01\\
223.01	0.01\\
224.01	0.01\\
225.01	0.01\\
226.01	0.01\\
227.01	0.01\\
228.01	0.01\\
229.01	0.01\\
230.01	0.01\\
231.01	0.01\\
232.01	0.01\\
233.01	0.01\\
234.01	0.01\\
235.01	0.01\\
236.01	0.01\\
237.01	0.01\\
238.01	0.01\\
239.01	0.01\\
240.01	0.01\\
241.01	0.01\\
242.01	0.01\\
243.01	0.01\\
244.01	0.01\\
245.01	0.01\\
246.01	0.01\\
247.01	0.01\\
248.01	0.01\\
249.01	0.01\\
250.01	0.01\\
251.01	0.01\\
252.01	0.01\\
253.01	0.01\\
254.01	0.01\\
255.01	0.01\\
256.01	0.01\\
257.01	0.01\\
258.01	0.01\\
259.01	0.01\\
260.01	0.01\\
261.01	0.01\\
262.01	0.01\\
263.01	0.01\\
264.01	0.01\\
265.01	0.01\\
266.01	0.01\\
267.01	0.01\\
268.01	0.01\\
269.01	0.01\\
270.01	0.01\\
271.01	0.01\\
272.01	0.01\\
273.01	0.01\\
274.01	0.01\\
275.01	0.01\\
276.01	0.01\\
277.01	0.01\\
278.01	0.01\\
279.01	0.01\\
280.01	0.01\\
281.01	0.01\\
282.01	0.01\\
283.01	0.01\\
284.01	0.01\\
285.01	0.01\\
286.01	0.01\\
287.01	0.01\\
288.01	0.01\\
289.01	0.01\\
290.01	0.01\\
291.01	0.01\\
292.01	0.01\\
293.01	0.01\\
294.01	0.01\\
295.01	0.01\\
296.01	0.01\\
297.01	0.01\\
298.01	0.01\\
299.01	0.01\\
300.01	0.01\\
301.01	0.01\\
302.01	0.01\\
303.01	0.01\\
304.01	0.01\\
305.01	0.01\\
306.01	0.01\\
307.01	0.01\\
308.01	0.01\\
309.01	0.01\\
310.01	0.01\\
311.01	0.01\\
312.01	0.01\\
313.01	0.01\\
314.01	0.01\\
315.01	0.01\\
316.01	0.01\\
317.01	0.01\\
318.01	0.01\\
319.01	0.01\\
320.01	0.01\\
321.01	0.01\\
322.01	0.01\\
323.01	0.01\\
324.01	0.01\\
325.01	0.01\\
326.01	0.01\\
327.01	0.01\\
328.01	0.01\\
329.01	0.01\\
330.01	0.01\\
331.01	0.01\\
332.01	0.01\\
333.01	0.01\\
334.01	0.01\\
335.01	0.01\\
336.01	0.01\\
337.01	0.01\\
338.01	0.01\\
339.01	0.01\\
340.01	0.01\\
341.01	0.01\\
342.01	0.01\\
343.01	0.01\\
344.01	0.01\\
345.01	0.01\\
346.01	0.01\\
347.01	0.01\\
348.01	0.01\\
349.01	0.01\\
350.01	0.01\\
351.01	0.01\\
352.01	0.01\\
353.01	0.01\\
354.01	0.01\\
355.01	0.01\\
356.01	0.01\\
357.01	0.01\\
358.01	0.01\\
359.01	0.01\\
360.01	0.01\\
361.01	0.01\\
362.01	0.01\\
363.01	0.01\\
364.01	0.01\\
365.01	0.01\\
366.01	0.01\\
367.01	0.01\\
368.01	0.01\\
369.01	0.01\\
370.01	0.01\\
371.01	0.01\\
372.01	0.01\\
373.01	0.01\\
374.01	0.01\\
375.01	0.01\\
376.01	0.01\\
377.01	0.01\\
378.01	0.01\\
379.01	0.01\\
380.01	0.01\\
381.01	0.01\\
382.01	0.01\\
383.01	0.01\\
384.01	0.01\\
385.01	0.01\\
386.01	0.01\\
387.01	0.01\\
388.01	0.01\\
389.01	0.01\\
390.01	0.01\\
391.01	0.01\\
392.01	0.01\\
393.01	0.01\\
394.01	0.01\\
395.01	0.01\\
396.01	0.01\\
397.01	0.01\\
398.01	0.01\\
399.01	0.01\\
400.01	0.01\\
401.01	0.01\\
402.01	0.01\\
403.01	0.01\\
404.01	0.01\\
405.01	0.01\\
406.01	0.01\\
407.01	0.01\\
408.01	0.01\\
409.01	0.01\\
410.01	0.01\\
411.01	0.01\\
412.01	0.01\\
413.01	0.01\\
414.01	0.01\\
415.01	0.01\\
416.01	0.01\\
417.01	0.01\\
418.01	0.01\\
419.01	0.01\\
420.01	0.01\\
421.01	0.01\\
422.01	0.01\\
423.01	0.01\\
424.01	0.01\\
425.01	0.01\\
426.01	0.01\\
427.01	0.01\\
428.01	0.01\\
429.01	0.01\\
430.01	0.01\\
431.01	0.01\\
432.01	0.01\\
433.01	0.01\\
434.01	0.01\\
435.01	0.01\\
436.01	0.01\\
437.01	0.01\\
438.01	0.01\\
439.01	0.01\\
440.01	0.01\\
441.01	0.01\\
442.01	0.01\\
443.01	0.01\\
444.01	0.01\\
445.01	0.01\\
446.01	0.01\\
447.01	0.01\\
448.01	0.01\\
449.01	0.01\\
450.01	0.01\\
451.01	0.01\\
452.01	0.01\\
453.01	0.01\\
454.01	0.01\\
455.01	0.01\\
456.01	0.01\\
457.01	0.01\\
458.01	0.01\\
459.01	0.01\\
460.01	0.01\\
461.01	0.01\\
462.01	0.01\\
463.01	0.01\\
464.01	0.01\\
465.01	0.01\\
466.01	0.01\\
467.01	0.01\\
468.01	0.01\\
469.01	0.01\\
470.01	0.01\\
471.01	0.01\\
472.01	0.01\\
473.01	0.01\\
474.01	0.01\\
475.01	0.01\\
476.01	0.01\\
477.01	0.01\\
478.01	0.01\\
479.01	0.01\\
480.01	0.01\\
481.01	0.01\\
482.01	0.01\\
483.01	0.01\\
484.01	0.01\\
485.01	0.01\\
486.01	0.01\\
487.01	0.01\\
488.01	0.01\\
489.01	0.01\\
490.01	0.01\\
491.01	0.01\\
492.01	0.01\\
493.01	0.01\\
494.01	0.01\\
495.01	0.01\\
496.01	0.01\\
497.01	0.01\\
498.01	0.01\\
499.01	0.01\\
500.01	0.01\\
501.01	0.01\\
502.01	0.01\\
503.01	0.01\\
504.01	0.01\\
505.01	0.01\\
506.01	0.01\\
507.01	0.01\\
508.01	0.01\\
509.01	0.01\\
510.01	0.01\\
511.01	0.01\\
512.01	0.01\\
513.01	0.01\\
514.01	0.01\\
515.01	0.01\\
516.01	0.01\\
517.01	0.01\\
518.01	0.01\\
519.01	0.01\\
520.01	0.01\\
521.01	0.01\\
522.01	0.01\\
523.01	0.01\\
524.01	0.01\\
525.01	0.01\\
526.01	0.01\\
527.01	0.01\\
528.01	0.01\\
529.01	0.01\\
530.01	0.01\\
531.01	0.01\\
532.01	0.01\\
533.01	0.01\\
534.01	0.01\\
535.01	0.01\\
536.01	0.01\\
537.01	0.01\\
538.01	0.01\\
539.01	0.01\\
540.01	0.01\\
541.01	0.01\\
542.01	0.01\\
543.01	0.01\\
544.01	0.01\\
545.01	0.01\\
546.01	0.01\\
547.01	0.01\\
548.01	0.01\\
549.01	0.01\\
550.01	0.01\\
551.01	0.01\\
552.01	0.01\\
553.01	0.01\\
554.01	0.01\\
555.01	0.01\\
556.01	0.01\\
557.01	0.01\\
558.01	0.01\\
559.01	0.01\\
560.01	0.01\\
561.01	0.01\\
562.01	0.01\\
563.01	0.01\\
564.01	0.01\\
565.01	0.01\\
566.01	0.01\\
567.01	0.01\\
568.01	0.01\\
569.01	0.01\\
570.01	0.01\\
571.01	0.01\\
572.01	0.01\\
573.01	0.00980894049358069\\
574.01	0.00959285206314057\\
575.01	0.00936137092261357\\
576.01	0.00911232543663533\\
577.01	0.00884392085869094\\
578.01	0.00856331316716572\\
579.01	0.00827261631885943\\
580.01	0.00797142148655063\\
581.01	0.00765935959990316\\
582.01	0.00733612494795841\\
583.01	0.00700150675312477\\
584.01	0.00665543123614698\\
585.01	0.00629825648550703\\
586.01	0.00593194486745375\\
587.01	0.0055575347412609\\
588.01	0.00517546743094326\\
589.01	0.00478557032423368\\
590.01	0.00438767896675604\\
591.01	0.00398164747908303\\
592.01	0.00356735318276342\\
593.01	0.00314470158774184\\
594.01	0.00271363155295263\\
595.01	0.00227412039702548\\
596.01	0.00182618962595358\\
597.01	0.00136992176993503\\
598.01	0.000905592002350849\\
599.01	0.00043492981576735\\
599.02	0.000430213974982772\\
599.03	0.000425498536119657\\
599.04	0.000420783528401717\\
599.05	0.000416068981802759\\
599.06	0.000411354927068031\\
599.07	0.000406641395736292\\
599.08	0.000401928420162703\\
599.09	0.000397216033542409\\
599.1	0.000392504269934873\\
599.11	0.000387793164289036\\
599.12	0.000383082752469323\\
599.13	0.000378373071282544\\
599.14	0.000373664158505704\\
599.15	0.000368956052914794\\
599.16	0.000364248794314553\\
599.17	0.000359542423569297\\
599.18	0.000354836982634823\\
599.19	0.000350132514591449\\
599.2	0.000345429063678236\\
599.21	0.000340726675289876\\
599.22	0.000336025395992838\\
599.23	0.000331325273624394\\
599.24	0.000326626357332171\\
599.25	0.000321928697615173\\
599.26	0.000317232346366351\\
599.27	0.000312537356328739\\
599.28	0.000307843780926497\\
599.29	0.000303151675113411\\
599.3	0.000298461095422742\\
599.31	0.000293772099994554\\
599.32	0.000289084748632946\\
599.33	0.000284399102886653\\
599.34	0.000279715226107495\\
599.35	0.000275033183508976\\
599.36	0.000270353042232248\\
599.37	0.000265674871413527\\
599.38	0.000260998742252962\\
599.39	0.00025632472808642\\
599.4	0.000251652904460315\\
599.41	0.000246983349209667\\
599.42	0.000242316142539528\\
599.43	0.000237651367110008\\
599.44	0.000232989108125029\\
599.45	0.000228329453425088\\
599.46	0.000223672493584178\\
599.47	0.000219018322011167\\
599.48	0.000214367035055838\\
599.49	0.000209719387470494\\
599.5	0.000205075830177353\\
599.51	0.00020043647066856\\
599.52	0.000195801419884652\\
599.53	0.000191173145721217\\
599.54	0.000186563422008259\\
599.55	0.000181972450344473\\
599.56	0.000177400620217833\\
599.57	0.00017284847721002\\
599.58	0.000168316234915521\\
599.59	0.000163804156364234\\
599.6	0.00015931246726116\\
599.61	0.000154841397385808\\
599.62	0.000150391182695489\\
599.63	0.000145962065581585\\
599.64	0.000141554295139023\\
599.65	0.000137168127449856\\
599.66	0.000132803825881878\\
599.67	0.000128461661373839\\
599.68	0.000124141912752868\\
599.69	0.000119844867080403\\
599.7	0.00011557082001749\\
599.71	0.00011132007621082\\
599.72	0.000107092949700736\\
599.73	0.000102889764201437\\
599.74	9.87108534851212e-05\\
599.75	9.45565618428821e-05\\
599.76	9.04272445685847e-05\\
599.77	8.6323268469345e-05\\
599.78	8.22450124151018e-05\\
599.79	7.81928679230454e-05\\
599.8	7.41672397797984e-05\\
599.81	7.01685467046336e-05\\
599.82	6.6197222057314e-05\\
599.83	6.22537145945345e-05\\
599.84	5.83384892794431e-05\\
599.85	5.44520281491948e-05\\
599.86	5.05948312460661e-05\\
599.87	4.67674176184013e-05\\
599.88	4.29703263983248e-05\\
599.89	3.92041179641221e-05\\
599.9	3.54693751961673e-05\\
599.91	3.17667048364539e-05\\
599.92	2.80967389631615e-05\\
599.93	2.44601365932736e-05\\
599.94	2.08575854280871e-05\\
599.95	1.72898037587021e-05\\
599.96	1.37575425510246e-05\\
599.97	1.02615877329605e-05\\
599.98	6.80276270998044e-06\\
599.99	3.38193113953895e-06\\
600	0\\
};
\addplot [color=mycolor15,solid,forget plot]
  table[row sep=crcr]{%
0.01	0.01\\
1.01	0.01\\
2.01	0.01\\
3.01	0.01\\
4.01	0.01\\
5.01	0.01\\
6.01	0.01\\
7.01	0.01\\
8.01	0.01\\
9.01	0.01\\
10.01	0.01\\
11.01	0.01\\
12.01	0.01\\
13.01	0.01\\
14.01	0.01\\
15.01	0.01\\
16.01	0.01\\
17.01	0.01\\
18.01	0.01\\
19.01	0.01\\
20.01	0.01\\
21.01	0.01\\
22.01	0.01\\
23.01	0.01\\
24.01	0.01\\
25.01	0.01\\
26.01	0.01\\
27.01	0.01\\
28.01	0.01\\
29.01	0.01\\
30.01	0.01\\
31.01	0.01\\
32.01	0.01\\
33.01	0.01\\
34.01	0.01\\
35.01	0.01\\
36.01	0.01\\
37.01	0.01\\
38.01	0.01\\
39.01	0.01\\
40.01	0.01\\
41.01	0.01\\
42.01	0.01\\
43.01	0.01\\
44.01	0.01\\
45.01	0.01\\
46.01	0.01\\
47.01	0.01\\
48.01	0.01\\
49.01	0.01\\
50.01	0.01\\
51.01	0.01\\
52.01	0.01\\
53.01	0.01\\
54.01	0.01\\
55.01	0.01\\
56.01	0.01\\
57.01	0.01\\
58.01	0.01\\
59.01	0.01\\
60.01	0.01\\
61.01	0.01\\
62.01	0.01\\
63.01	0.01\\
64.01	0.01\\
65.01	0.01\\
66.01	0.01\\
67.01	0.01\\
68.01	0.01\\
69.01	0.01\\
70.01	0.01\\
71.01	0.01\\
72.01	0.01\\
73.01	0.01\\
74.01	0.01\\
75.01	0.01\\
76.01	0.01\\
77.01	0.01\\
78.01	0.01\\
79.01	0.01\\
80.01	0.01\\
81.01	0.01\\
82.01	0.01\\
83.01	0.01\\
84.01	0.01\\
85.01	0.01\\
86.01	0.01\\
87.01	0.01\\
88.01	0.01\\
89.01	0.01\\
90.01	0.01\\
91.01	0.01\\
92.01	0.01\\
93.01	0.01\\
94.01	0.01\\
95.01	0.01\\
96.01	0.01\\
97.01	0.01\\
98.01	0.01\\
99.01	0.01\\
100.01	0.01\\
101.01	0.01\\
102.01	0.01\\
103.01	0.01\\
104.01	0.01\\
105.01	0.01\\
106.01	0.01\\
107.01	0.01\\
108.01	0.01\\
109.01	0.01\\
110.01	0.01\\
111.01	0.01\\
112.01	0.01\\
113.01	0.01\\
114.01	0.01\\
115.01	0.01\\
116.01	0.01\\
117.01	0.01\\
118.01	0.01\\
119.01	0.01\\
120.01	0.01\\
121.01	0.01\\
122.01	0.01\\
123.01	0.01\\
124.01	0.01\\
125.01	0.01\\
126.01	0.01\\
127.01	0.01\\
128.01	0.01\\
129.01	0.01\\
130.01	0.01\\
131.01	0.01\\
132.01	0.01\\
133.01	0.01\\
134.01	0.01\\
135.01	0.01\\
136.01	0.01\\
137.01	0.01\\
138.01	0.01\\
139.01	0.01\\
140.01	0.01\\
141.01	0.01\\
142.01	0.01\\
143.01	0.01\\
144.01	0.01\\
145.01	0.01\\
146.01	0.01\\
147.01	0.01\\
148.01	0.01\\
149.01	0.01\\
150.01	0.01\\
151.01	0.01\\
152.01	0.01\\
153.01	0.01\\
154.01	0.01\\
155.01	0.01\\
156.01	0.01\\
157.01	0.01\\
158.01	0.01\\
159.01	0.01\\
160.01	0.01\\
161.01	0.01\\
162.01	0.01\\
163.01	0.01\\
164.01	0.01\\
165.01	0.01\\
166.01	0.01\\
167.01	0.01\\
168.01	0.01\\
169.01	0.01\\
170.01	0.01\\
171.01	0.01\\
172.01	0.01\\
173.01	0.01\\
174.01	0.01\\
175.01	0.01\\
176.01	0.01\\
177.01	0.01\\
178.01	0.01\\
179.01	0.01\\
180.01	0.01\\
181.01	0.01\\
182.01	0.01\\
183.01	0.01\\
184.01	0.01\\
185.01	0.01\\
186.01	0.01\\
187.01	0.01\\
188.01	0.01\\
189.01	0.01\\
190.01	0.01\\
191.01	0.01\\
192.01	0.01\\
193.01	0.01\\
194.01	0.01\\
195.01	0.01\\
196.01	0.01\\
197.01	0.01\\
198.01	0.01\\
199.01	0.01\\
200.01	0.01\\
201.01	0.01\\
202.01	0.01\\
203.01	0.01\\
204.01	0.01\\
205.01	0.01\\
206.01	0.01\\
207.01	0.01\\
208.01	0.01\\
209.01	0.01\\
210.01	0.01\\
211.01	0.01\\
212.01	0.01\\
213.01	0.01\\
214.01	0.01\\
215.01	0.01\\
216.01	0.01\\
217.01	0.01\\
218.01	0.01\\
219.01	0.01\\
220.01	0.01\\
221.01	0.01\\
222.01	0.01\\
223.01	0.01\\
224.01	0.01\\
225.01	0.01\\
226.01	0.01\\
227.01	0.01\\
228.01	0.01\\
229.01	0.01\\
230.01	0.01\\
231.01	0.01\\
232.01	0.01\\
233.01	0.01\\
234.01	0.01\\
235.01	0.01\\
236.01	0.01\\
237.01	0.01\\
238.01	0.01\\
239.01	0.01\\
240.01	0.01\\
241.01	0.01\\
242.01	0.01\\
243.01	0.01\\
244.01	0.01\\
245.01	0.01\\
246.01	0.01\\
247.01	0.01\\
248.01	0.01\\
249.01	0.01\\
250.01	0.01\\
251.01	0.01\\
252.01	0.01\\
253.01	0.01\\
254.01	0.01\\
255.01	0.01\\
256.01	0.01\\
257.01	0.01\\
258.01	0.01\\
259.01	0.01\\
260.01	0.01\\
261.01	0.01\\
262.01	0.01\\
263.01	0.01\\
264.01	0.01\\
265.01	0.01\\
266.01	0.01\\
267.01	0.01\\
268.01	0.01\\
269.01	0.01\\
270.01	0.01\\
271.01	0.01\\
272.01	0.01\\
273.01	0.01\\
274.01	0.01\\
275.01	0.01\\
276.01	0.01\\
277.01	0.01\\
278.01	0.01\\
279.01	0.01\\
280.01	0.01\\
281.01	0.01\\
282.01	0.01\\
283.01	0.01\\
284.01	0.01\\
285.01	0.01\\
286.01	0.01\\
287.01	0.01\\
288.01	0.01\\
289.01	0.01\\
290.01	0.01\\
291.01	0.01\\
292.01	0.01\\
293.01	0.01\\
294.01	0.01\\
295.01	0.01\\
296.01	0.01\\
297.01	0.01\\
298.01	0.01\\
299.01	0.01\\
300.01	0.01\\
301.01	0.01\\
302.01	0.01\\
303.01	0.01\\
304.01	0.01\\
305.01	0.01\\
306.01	0.01\\
307.01	0.01\\
308.01	0.01\\
309.01	0.01\\
310.01	0.01\\
311.01	0.01\\
312.01	0.01\\
313.01	0.01\\
314.01	0.01\\
315.01	0.01\\
316.01	0.01\\
317.01	0.01\\
318.01	0.01\\
319.01	0.01\\
320.01	0.01\\
321.01	0.01\\
322.01	0.01\\
323.01	0.01\\
324.01	0.01\\
325.01	0.01\\
326.01	0.01\\
327.01	0.01\\
328.01	0.01\\
329.01	0.01\\
330.01	0.01\\
331.01	0.01\\
332.01	0.01\\
333.01	0.01\\
334.01	0.01\\
335.01	0.01\\
336.01	0.01\\
337.01	0.01\\
338.01	0.01\\
339.01	0.01\\
340.01	0.01\\
341.01	0.01\\
342.01	0.01\\
343.01	0.01\\
344.01	0.01\\
345.01	0.01\\
346.01	0.01\\
347.01	0.01\\
348.01	0.01\\
349.01	0.01\\
350.01	0.01\\
351.01	0.01\\
352.01	0.01\\
353.01	0.01\\
354.01	0.01\\
355.01	0.01\\
356.01	0.01\\
357.01	0.01\\
358.01	0.01\\
359.01	0.01\\
360.01	0.01\\
361.01	0.01\\
362.01	0.01\\
363.01	0.01\\
364.01	0.01\\
365.01	0.01\\
366.01	0.01\\
367.01	0.01\\
368.01	0.01\\
369.01	0.01\\
370.01	0.01\\
371.01	0.01\\
372.01	0.01\\
373.01	0.01\\
374.01	0.01\\
375.01	0.01\\
376.01	0.01\\
377.01	0.01\\
378.01	0.01\\
379.01	0.01\\
380.01	0.01\\
381.01	0.01\\
382.01	0.01\\
383.01	0.01\\
384.01	0.01\\
385.01	0.01\\
386.01	0.01\\
387.01	0.01\\
388.01	0.01\\
389.01	0.01\\
390.01	0.01\\
391.01	0.01\\
392.01	0.01\\
393.01	0.01\\
394.01	0.01\\
395.01	0.01\\
396.01	0.01\\
397.01	0.01\\
398.01	0.01\\
399.01	0.01\\
400.01	0.01\\
401.01	0.01\\
402.01	0.01\\
403.01	0.01\\
404.01	0.01\\
405.01	0.01\\
406.01	0.01\\
407.01	0.01\\
408.01	0.01\\
409.01	0.01\\
410.01	0.01\\
411.01	0.01\\
412.01	0.01\\
413.01	0.01\\
414.01	0.01\\
415.01	0.01\\
416.01	0.01\\
417.01	0.01\\
418.01	0.01\\
419.01	0.01\\
420.01	0.01\\
421.01	0.01\\
422.01	0.01\\
423.01	0.01\\
424.01	0.01\\
425.01	0.01\\
426.01	0.01\\
427.01	0.01\\
428.01	0.01\\
429.01	0.01\\
430.01	0.01\\
431.01	0.01\\
432.01	0.01\\
433.01	0.01\\
434.01	0.01\\
435.01	0.01\\
436.01	0.01\\
437.01	0.01\\
438.01	0.01\\
439.01	0.01\\
440.01	0.01\\
441.01	0.01\\
442.01	0.01\\
443.01	0.01\\
444.01	0.01\\
445.01	0.01\\
446.01	0.01\\
447.01	0.01\\
448.01	0.01\\
449.01	0.01\\
450.01	0.01\\
451.01	0.01\\
452.01	0.01\\
453.01	0.01\\
454.01	0.01\\
455.01	0.01\\
456.01	0.01\\
457.01	0.01\\
458.01	0.01\\
459.01	0.01\\
460.01	0.01\\
461.01	0.01\\
462.01	0.01\\
463.01	0.01\\
464.01	0.01\\
465.01	0.01\\
466.01	0.01\\
467.01	0.01\\
468.01	0.01\\
469.01	0.01\\
470.01	0.01\\
471.01	0.01\\
472.01	0.01\\
473.01	0.01\\
474.01	0.01\\
475.01	0.01\\
476.01	0.01\\
477.01	0.01\\
478.01	0.01\\
479.01	0.01\\
480.01	0.01\\
481.01	0.01\\
482.01	0.01\\
483.01	0.01\\
484.01	0.01\\
485.01	0.01\\
486.01	0.01\\
487.01	0.01\\
488.01	0.01\\
489.01	0.01\\
490.01	0.01\\
491.01	0.01\\
492.01	0.01\\
493.01	0.01\\
494.01	0.01\\
495.01	0.01\\
496.01	0.01\\
497.01	0.01\\
498.01	0.01\\
499.01	0.01\\
500.01	0.01\\
501.01	0.01\\
502.01	0.01\\
503.01	0.01\\
504.01	0.01\\
505.01	0.01\\
506.01	0.01\\
507.01	0.01\\
508.01	0.01\\
509.01	0.01\\
510.01	0.01\\
511.01	0.01\\
512.01	0.01\\
513.01	0.01\\
514.01	0.01\\
515.01	0.01\\
516.01	0.01\\
517.01	0.01\\
518.01	0.01\\
519.01	0.01\\
520.01	0.01\\
521.01	0.01\\
522.01	0.01\\
523.01	0.01\\
524.01	0.01\\
525.01	0.01\\
526.01	0.01\\
527.01	0.01\\
528.01	0.01\\
529.01	0.01\\
530.01	0.01\\
531.01	0.01\\
532.01	0.01\\
533.01	0.01\\
534.01	0.01\\
535.01	0.01\\
536.01	0.01\\
537.01	0.01\\
538.01	0.01\\
539.01	0.01\\
540.01	0.01\\
541.01	0.01\\
542.01	0.01\\
543.01	0.01\\
544.01	0.01\\
545.01	0.01\\
546.01	0.01\\
547.01	0.01\\
548.01	0.01\\
549.01	0.01\\
550.01	0.01\\
551.01	0.00990204942161797\\
552.01	0.00975551902864949\\
553.01	0.00960130073362049\\
554.01	0.0094386003805723\\
555.01	0.009266503152353\\
556.01	0.0090839507765319\\
557.01	0.00888971377982333\\
558.01	0.008682357705687\\
559.01	0.00846020294590994\\
560.01	0.00822128900639638\\
561.01	0.00796766659822521\\
562.01	0.0077048159941505\\
563.01	0.00743232512672041\\
564.01	0.0071496454824594\\
565.01	0.0068561899957042\\
566.01	0.0065513453559236\\
567.01	0.00623447560958786\\
568.01	0.00590492764055903\\
569.01	0.00556203943909303\\
570.01	0.00520515136201691\\
571.01	0.0048335653439925\\
572.01	0.00444648371089207\\
573.01	0.0042357046706426\\
574.01	0.00403720111108808\\
575.01	0.00384342391724351\\
576.01	0.00365706307610982\\
577.01	0.00348009927715622\\
578.01	0.00330555158339844\\
579.01	0.00313128876087222\\
580.01	0.00295772959046831\\
581.01	0.00278530784327323\\
582.01	0.0026144529462283\\
583.01	0.00244555985981434\\
584.01	0.00227894406209801\\
585.01	0.00211453903896056\\
586.01	0.00195076928643856\\
587.01	0.00178744637180153\\
588.01	0.00162464815568858\\
589.01	0.00146288912824371\\
590.01	0.00130274409803109\\
591.01	0.00114484388588358\\
592.01	0.000989873810182567\\
593.01	0.00083856959646709\\
594.01	0.00069170989414428\\
595.01	0.000550104571421454\\
596.01	0.000414578932401176\\
597.01	0.000285963596768515\\
598.01	0.000165188062198949\\
599.01	5.4345015548133e-05\\
599.02	5.33064624635531e-05\\
599.03	5.22697048745222e-05\\
599.04	5.12347597904136e-05\\
599.05	5.0201644409836e-05\\
599.06	4.91703761153605e-05\\
599.07	4.8140972467612e-05\\
599.08	4.71134500464859e-05\\
599.09	4.60878252401993e-05\\
599.1	4.50641145578815e-05\\
599.11	4.40423346413817e-05\\
599.12	4.30225022459152e-05\\
599.13	4.20046341915222e-05\\
599.14	4.0988747398104e-05\\
599.15	3.99748588722423e-05\\
599.16	3.89629856928746e-05\\
599.17	3.79531449942027e-05\\
599.18	3.69453539466515e-05\\
599.19	3.59396297421356e-05\\
599.2	3.49359895741794e-05\\
599.21	3.39348997364097e-05\\
599.22	3.29370757084346e-05\\
599.23	3.19425380165637e-05\\
599.24	3.09513071439539e-05\\
599.25	2.99634035031259e-05\\
599.26	2.89788474063021e-05\\
599.27	2.80044066980276e-05\\
599.28	2.70493587152636e-05\\
599.29	2.61138364157729e-05\\
599.3	2.51979740088139e-05\\
599.31	2.43022091112599e-05\\
599.32	2.34269645671102e-05\\
599.33	2.25723778285291e-05\\
599.34	2.17385877173907e-05\\
599.35	2.09257721965651e-05\\
599.36	2.01340862337751e-05\\
599.37	1.9363671100418e-05\\
599.38	1.86146691965421e-05\\
599.39	1.78872240385961e-05\\
599.4	1.71814802439572e-05\\
599.41	1.64975835122289e-05\\
599.42	1.58356806037336e-05\\
599.43	1.5195919314372e-05\\
599.44	1.457844844663e-05\\
599.45	1.39834177763486e-05\\
599.46	1.34109780150146e-05\\
599.47	1.28612807671867e-05\\
599.48	1.23344784827405e-05\\
599.49	1.18300658477049e-05\\
599.5	1.13478434777131e-05\\
599.51	1.08879592333778e-05\\
599.52	1.04505664172659e-05\\
599.53	1.00334667578663e-05\\
599.54	9.62511880027439e-06\\
599.55	9.2255549021085e-06\\
599.56	8.83462029344952e-06\\
599.57	8.45200412851778e-06\\
599.58	8.07772879304891e-06\\
599.59	7.71176793122087e-06\\
599.6	7.35413349658119e-06\\
599.61	7.00483423372172e-06\\
599.62	6.66387342551991e-06\\
599.63	6.33124847425164e-06\\
599.64	6.00695045552616e-06\\
599.65	5.69096364331188e-06\\
599.66	5.38326500425235e-06\\
599.67	5.08384138973167e-06\\
599.68	4.79267948952887e-06\\
599.69	4.50975886944101e-06\\
599.7	4.23505144031729e-06\\
599.71	3.96852089248369e-06\\
599.72	3.71012224967097e-06\\
599.73	3.45988538647847e-06\\
599.74	3.21786860717557e-06\\
599.75	2.98412945139333e-06\\
599.76	2.75872706875617e-06\\
599.77	2.54172382291654e-06\\
599.78	2.33318159911268e-06\\
599.79	2.13316174173424e-06\\
599.8	1.94172498835417e-06\\
599.81	1.75893139995643e-06\\
599.82	1.58484028716689e-06\\
599.83	1.41951013224277e-06\\
599.84	1.2629985065276e-06\\
599.85	1.11536198312695e-06\\
599.86	9.76656044501464e-07\\
599.87	8.46934984602421e-07\\
599.88	7.26251805284461e-07\\
599.89	6.14658106537491e-07\\
599.9	5.12203970199493e-07\\
599.91	4.18937836669728e-07\\
599.92	3.34906374173036e-07\\
599.93	2.60154340070429e-07\\
599.94	1.94724433657395e-07\\
599.95	1.38657139853171e-07\\
599.96	9.19905631651535e-08\\
599.97	5.47602512050716e-08\\
599.98	2.6999006997111e-08\\
599.99	8.73668930083393e-09\\
600	0\\
};
\addplot [color=mycolor16,solid,forget plot]
  table[row sep=crcr]{%
0.01	0.01\\
1.01	0.01\\
2.01	0.01\\
3.01	0.01\\
4.01	0.01\\
5.01	0.01\\
6.01	0.01\\
7.01	0.01\\
8.01	0.01\\
9.01	0.01\\
10.01	0.01\\
11.01	0.01\\
12.01	0.01\\
13.01	0.01\\
14.01	0.01\\
15.01	0.01\\
16.01	0.01\\
17.01	0.01\\
18.01	0.01\\
19.01	0.01\\
20.01	0.01\\
21.01	0.01\\
22.01	0.01\\
23.01	0.01\\
24.01	0.01\\
25.01	0.01\\
26.01	0.01\\
27.01	0.01\\
28.01	0.01\\
29.01	0.01\\
30.01	0.01\\
31.01	0.01\\
32.01	0.01\\
33.01	0.01\\
34.01	0.01\\
35.01	0.01\\
36.01	0.01\\
37.01	0.01\\
38.01	0.01\\
39.01	0.01\\
40.01	0.01\\
41.01	0.01\\
42.01	0.01\\
43.01	0.01\\
44.01	0.01\\
45.01	0.01\\
46.01	0.01\\
47.01	0.01\\
48.01	0.01\\
49.01	0.01\\
50.01	0.01\\
51.01	0.01\\
52.01	0.01\\
53.01	0.01\\
54.01	0.01\\
55.01	0.01\\
56.01	0.01\\
57.01	0.01\\
58.01	0.01\\
59.01	0.01\\
60.01	0.01\\
61.01	0.01\\
62.01	0.01\\
63.01	0.01\\
64.01	0.01\\
65.01	0.01\\
66.01	0.01\\
67.01	0.01\\
68.01	0.01\\
69.01	0.01\\
70.01	0.01\\
71.01	0.01\\
72.01	0.01\\
73.01	0.01\\
74.01	0.01\\
75.01	0.01\\
76.01	0.01\\
77.01	0.01\\
78.01	0.01\\
79.01	0.01\\
80.01	0.01\\
81.01	0.01\\
82.01	0.01\\
83.01	0.01\\
84.01	0.01\\
85.01	0.01\\
86.01	0.01\\
87.01	0.01\\
88.01	0.01\\
89.01	0.01\\
90.01	0.01\\
91.01	0.01\\
92.01	0.01\\
93.01	0.01\\
94.01	0.01\\
95.01	0.01\\
96.01	0.01\\
97.01	0.01\\
98.01	0.01\\
99.01	0.01\\
100.01	0.01\\
101.01	0.01\\
102.01	0.01\\
103.01	0.01\\
104.01	0.01\\
105.01	0.01\\
106.01	0.01\\
107.01	0.01\\
108.01	0.01\\
109.01	0.01\\
110.01	0.01\\
111.01	0.01\\
112.01	0.01\\
113.01	0.01\\
114.01	0.01\\
115.01	0.01\\
116.01	0.01\\
117.01	0.01\\
118.01	0.01\\
119.01	0.01\\
120.01	0.01\\
121.01	0.01\\
122.01	0.01\\
123.01	0.01\\
124.01	0.01\\
125.01	0.01\\
126.01	0.01\\
127.01	0.01\\
128.01	0.01\\
129.01	0.01\\
130.01	0.01\\
131.01	0.01\\
132.01	0.01\\
133.01	0.01\\
134.01	0.01\\
135.01	0.01\\
136.01	0.01\\
137.01	0.01\\
138.01	0.01\\
139.01	0.01\\
140.01	0.01\\
141.01	0.01\\
142.01	0.01\\
143.01	0.01\\
144.01	0.01\\
145.01	0.01\\
146.01	0.01\\
147.01	0.01\\
148.01	0.01\\
149.01	0.01\\
150.01	0.01\\
151.01	0.01\\
152.01	0.01\\
153.01	0.01\\
154.01	0.01\\
155.01	0.01\\
156.01	0.01\\
157.01	0.01\\
158.01	0.01\\
159.01	0.01\\
160.01	0.01\\
161.01	0.01\\
162.01	0.01\\
163.01	0.01\\
164.01	0.01\\
165.01	0.01\\
166.01	0.01\\
167.01	0.01\\
168.01	0.01\\
169.01	0.01\\
170.01	0.01\\
171.01	0.01\\
172.01	0.01\\
173.01	0.01\\
174.01	0.01\\
175.01	0.01\\
176.01	0.01\\
177.01	0.01\\
178.01	0.01\\
179.01	0.01\\
180.01	0.01\\
181.01	0.01\\
182.01	0.01\\
183.01	0.01\\
184.01	0.01\\
185.01	0.01\\
186.01	0.01\\
187.01	0.01\\
188.01	0.01\\
189.01	0.01\\
190.01	0.01\\
191.01	0.01\\
192.01	0.01\\
193.01	0.01\\
194.01	0.01\\
195.01	0.01\\
196.01	0.01\\
197.01	0.01\\
198.01	0.01\\
199.01	0.01\\
200.01	0.01\\
201.01	0.01\\
202.01	0.01\\
203.01	0.01\\
204.01	0.01\\
205.01	0.01\\
206.01	0.01\\
207.01	0.01\\
208.01	0.01\\
209.01	0.01\\
210.01	0.01\\
211.01	0.01\\
212.01	0.01\\
213.01	0.01\\
214.01	0.01\\
215.01	0.01\\
216.01	0.01\\
217.01	0.01\\
218.01	0.01\\
219.01	0.01\\
220.01	0.01\\
221.01	0.01\\
222.01	0.01\\
223.01	0.01\\
224.01	0.01\\
225.01	0.01\\
226.01	0.01\\
227.01	0.01\\
228.01	0.01\\
229.01	0.01\\
230.01	0.01\\
231.01	0.01\\
232.01	0.01\\
233.01	0.01\\
234.01	0.01\\
235.01	0.01\\
236.01	0.01\\
237.01	0.01\\
238.01	0.01\\
239.01	0.01\\
240.01	0.01\\
241.01	0.01\\
242.01	0.01\\
243.01	0.01\\
244.01	0.01\\
245.01	0.01\\
246.01	0.01\\
247.01	0.01\\
248.01	0.01\\
249.01	0.01\\
250.01	0.01\\
251.01	0.01\\
252.01	0.01\\
253.01	0.01\\
254.01	0.01\\
255.01	0.01\\
256.01	0.01\\
257.01	0.01\\
258.01	0.01\\
259.01	0.01\\
260.01	0.01\\
261.01	0.01\\
262.01	0.01\\
263.01	0.01\\
264.01	0.01\\
265.01	0.01\\
266.01	0.01\\
267.01	0.01\\
268.01	0.01\\
269.01	0.01\\
270.01	0.01\\
271.01	0.01\\
272.01	0.01\\
273.01	0.01\\
274.01	0.01\\
275.01	0.01\\
276.01	0.01\\
277.01	0.01\\
278.01	0.01\\
279.01	0.01\\
280.01	0.01\\
281.01	0.01\\
282.01	0.01\\
283.01	0.01\\
284.01	0.01\\
285.01	0.01\\
286.01	0.01\\
287.01	0.01\\
288.01	0.01\\
289.01	0.01\\
290.01	0.01\\
291.01	0.01\\
292.01	0.01\\
293.01	0.01\\
294.01	0.01\\
295.01	0.01\\
296.01	0.01\\
297.01	0.01\\
298.01	0.01\\
299.01	0.01\\
300.01	0.01\\
301.01	0.01\\
302.01	0.01\\
303.01	0.01\\
304.01	0.01\\
305.01	0.01\\
306.01	0.01\\
307.01	0.01\\
308.01	0.01\\
309.01	0.01\\
310.01	0.01\\
311.01	0.01\\
312.01	0.01\\
313.01	0.01\\
314.01	0.01\\
315.01	0.01\\
316.01	0.01\\
317.01	0.01\\
318.01	0.01\\
319.01	0.01\\
320.01	0.01\\
321.01	0.01\\
322.01	0.01\\
323.01	0.01\\
324.01	0.01\\
325.01	0.01\\
326.01	0.01\\
327.01	0.01\\
328.01	0.01\\
329.01	0.01\\
330.01	0.01\\
331.01	0.01\\
332.01	0.01\\
333.01	0.01\\
334.01	0.01\\
335.01	0.01\\
336.01	0.01\\
337.01	0.01\\
338.01	0.01\\
339.01	0.01\\
340.01	0.01\\
341.01	0.01\\
342.01	0.01\\
343.01	0.01\\
344.01	0.01\\
345.01	0.01\\
346.01	0.01\\
347.01	0.01\\
348.01	0.01\\
349.01	0.01\\
350.01	0.01\\
351.01	0.01\\
352.01	0.01\\
353.01	0.01\\
354.01	0.01\\
355.01	0.01\\
356.01	0.01\\
357.01	0.01\\
358.01	0.01\\
359.01	0.01\\
360.01	0.01\\
361.01	0.01\\
362.01	0.01\\
363.01	0.01\\
364.01	0.01\\
365.01	0.01\\
366.01	0.01\\
367.01	0.01\\
368.01	0.01\\
369.01	0.01\\
370.01	0.01\\
371.01	0.01\\
372.01	0.01\\
373.01	0.01\\
374.01	0.01\\
375.01	0.01\\
376.01	0.01\\
377.01	0.01\\
378.01	0.01\\
379.01	0.01\\
380.01	0.01\\
381.01	0.01\\
382.01	0.01\\
383.01	0.01\\
384.01	0.01\\
385.01	0.01\\
386.01	0.01\\
387.01	0.01\\
388.01	0.01\\
389.01	0.01\\
390.01	0.01\\
391.01	0.01\\
392.01	0.01\\
393.01	0.01\\
394.01	0.01\\
395.01	0.01\\
396.01	0.01\\
397.01	0.01\\
398.01	0.01\\
399.01	0.01\\
400.01	0.01\\
401.01	0.01\\
402.01	0.01\\
403.01	0.01\\
404.01	0.01\\
405.01	0.01\\
406.01	0.01\\
407.01	0.01\\
408.01	0.01\\
409.01	0.01\\
410.01	0.01\\
411.01	0.01\\
412.01	0.01\\
413.01	0.01\\
414.01	0.01\\
415.01	0.01\\
416.01	0.01\\
417.01	0.01\\
418.01	0.01\\
419.01	0.01\\
420.01	0.01\\
421.01	0.01\\
422.01	0.01\\
423.01	0.01\\
424.01	0.01\\
425.01	0.01\\
426.01	0.01\\
427.01	0.01\\
428.01	0.01\\
429.01	0.01\\
430.01	0.01\\
431.01	0.01\\
432.01	0.01\\
433.01	0.01\\
434.01	0.01\\
435.01	0.01\\
436.01	0.01\\
437.01	0.01\\
438.01	0.01\\
439.01	0.01\\
440.01	0.01\\
441.01	0.01\\
442.01	0.01\\
443.01	0.01\\
444.01	0.01\\
445.01	0.01\\
446.01	0.01\\
447.01	0.01\\
448.01	0.01\\
449.01	0.01\\
450.01	0.01\\
451.01	0.01\\
452.01	0.01\\
453.01	0.01\\
454.01	0.01\\
455.01	0.01\\
456.01	0.01\\
457.01	0.01\\
458.01	0.01\\
459.01	0.01\\
460.01	0.01\\
461.01	0.01\\
462.01	0.01\\
463.01	0.01\\
464.01	0.01\\
465.01	0.01\\
466.01	0.01\\
467.01	0.01\\
468.01	0.01\\
469.01	0.01\\
470.01	0.01\\
471.01	0.01\\
472.01	0.01\\
473.01	0.01\\
474.01	0.01\\
475.01	0.01\\
476.01	0.01\\
477.01	0.01\\
478.01	0.01\\
479.01	0.01\\
480.01	0.01\\
481.01	0.01\\
482.01	0.01\\
483.01	0.01\\
484.01	0.01\\
485.01	0.01\\
486.01	0.01\\
487.01	0.01\\
488.01	0.01\\
489.01	0.01\\
490.01	0.01\\
491.01	0.01\\
492.01	0.01\\
493.01	0.01\\
494.01	0.01\\
495.01	0.01\\
496.01	0.01\\
497.01	0.01\\
498.01	0.01\\
499.01	0.01\\
500.01	0.01\\
501.01	0.01\\
502.01	0.01\\
503.01	0.01\\
504.01	0.01\\
505.01	0.01\\
506.01	0.01\\
507.01	0.01\\
508.01	0.01\\
509.01	0.01\\
510.01	0.01\\
511.01	0.01\\
512.01	0.01\\
513.01	0.01\\
514.01	0.01\\
515.01	0.01\\
516.01	0.01\\
517.01	0.01\\
518.01	0.01\\
519.01	0.01\\
520.01	0.01\\
521.01	0.01\\
522.01	0.01\\
523.01	0.00992575245227274\\
524.01	0.00984652916315449\\
525.01	0.00976427383091153\\
526.01	0.00967878858387452\\
527.01	0.00958985481377644\\
528.01	0.00949723016296407\\
529.01	0.00940064498293353\\
530.01	0.00929979815948588\\
531.01	0.00919435215483429\\
532.01	0.0090839270942965\\
533.01	0.00896809366624845\\
534.01	0.00884636485433692\\
535.01	0.00871818833204663\\
536.01	0.00858293155538475\\
537.01	0.00843986048094458\\
538.01	0.00828812853295346\\
539.01	0.00812675686309659\\
540.01	0.00795460963928865\\
541.01	0.00777036785781529\\
542.01	0.00757342059949487\\
543.01	0.00736856329333439\\
544.01	0.00715647415153917\\
545.01	0.00693675136318322\\
546.01	0.00670895106381417\\
547.01	0.00647257909973894\\
548.01	0.00622708057036442\\
549.01	0.00597182819023422\\
550.01	0.00570611914029568\\
551.01	0.00552764479099582\\
552.01	0.00538785028227558\\
553.01	0.00524547405651935\\
554.01	0.00510085683299053\\
555.01	0.0049544724765762\\
556.01	0.00480696566205194\\
557.01	0.00465919998370261\\
558.01	0.00451231962311478\\
559.01	0.00436782949647694\\
560.01	0.00422770128462472\\
561.01	0.00409012220592474\\
562.01	0.0039497661840341\\
563.01	0.00380802511746821\\
564.01	0.00366686467040957\\
565.01	0.00352675473052659\\
566.01	0.00338818459516706\\
567.01	0.00325170044485083\\
568.01	0.00311790483298779\\
569.01	0.00298745287269209\\
570.01	0.00286104575543987\\
571.01	0.00273947549593472\\
572.01	0.00262369203361567\\
573.01	0.00251316420804345\\
574.01	0.00240612417636698\\
575.01	0.00230064181470498\\
576.01	0.00219622952468568\\
577.01	0.00209274219424811\\
578.01	0.00199006113334727\\
579.01	0.00188828358359086\\
580.01	0.00178750904497525\\
581.01	0.00168781344225515\\
582.01	0.00158924687363906\\
583.01	0.0014918307579822\\
584.01	0.00139556043494016\\
585.01	0.00130039603260936\\
586.01	0.00120615033170115\\
587.01	0.00111236845038909\\
588.01	0.00101905521743938\\
589.01	0.000926321771227706\\
590.01	0.000834264424821173\\
591.01	0.000742959206678368\\
592.01	0.000652455943327473\\
593.01	0.000562772011437012\\
594.01	0.000473886068201661\\
595.01	0.00038573221032095\\
596.01	0.000298195161474923\\
597.01	0.000211107141103397\\
598.01	0.000124246194188974\\
599.01	3.7327034981641e-05\\
599.02	3.64581259681349e-05\\
599.03	3.55904023650409e-05\\
599.04	3.47238671974722e-05\\
599.05	3.38585235776824e-05\\
599.06	3.29943747087724e-05\\
599.07	3.21314238885661e-05\\
599.08	3.12828146845964e-05\\
599.09	3.04526072935907e-05\\
599.1	2.96411999832092e-05\\
599.11	2.8848744456229e-05\\
599.12	2.80755012468988e-05\\
599.13	2.73221646003644e-05\\
599.14	2.65888933353459e-05\\
599.15	2.58758490175037e-05\\
599.16	2.51831960653721e-05\\
599.17	2.45111189409661e-05\\
599.18	2.38598295667809e-05\\
599.19	2.3229501472391e-05\\
599.2	2.26203115193009e-05\\
599.21	2.20319886654764e-05\\
599.22	2.14640126570079e-05\\
599.23	2.09165616798686e-05\\
599.24	2.03898175673723e-05\\
599.25	1.98839659503354e-05\\
599.26	1.93991964147905e-05\\
599.27	1.89289341987753e-05\\
599.28	1.8464068823297e-05\\
599.29	1.80046351070922e-05\\
599.3	1.75506675047163e-05\\
599.31	1.71018958724945e-05\\
599.32	1.66580635847045e-05\\
599.33	1.62192026103292e-05\\
599.34	1.5785344236359e-05\\
599.35	1.53564802825266e-05\\
599.36	1.49326256067388e-05\\
599.37	1.4513809013501e-05\\
599.38	1.41000583480724e-05\\
599.39	1.36914004212514e-05\\
599.4	1.32878626875211e-05\\
599.41	1.28894738098337e-05\\
599.42	1.24962611940758e-05\\
599.43	1.21082508957385e-05\\
599.44	1.17254675215759e-05\\
599.45	1.13479341259243e-05\\
599.46	1.09756721014351e-05\\
599.47	1.06087010638255e-05\\
599.48	1.02470387303398e-05\\
599.49	9.8907026131103e-06\\
599.5	9.53970915088496e-06\\
599.51	9.19407265268304e-06\\
599.52	8.85380515453817e-06\\
599.53	8.51892385799261e-06\\
599.54	8.18948273669656e-06\\
599.55	7.86553617346253e-06\\
599.56	7.54713965199667e-06\\
599.57	7.23435039426401e-06\\
599.58	6.92722614961672e-06\\
599.59	6.6258253988271e-06\\
599.6	6.33020718107669e-06\\
599.61	6.04043109214518e-06\\
599.62	5.75655729072831e-06\\
599.63	5.47864650500898e-06\\
599.64	5.2067600394886e-06\\
599.65	4.94095978209944e-06\\
599.66	4.68130821164141e-06\\
599.67	4.42786840111967e-06\\
599.68	4.18070402385777e-06\\
599.69	3.93987936155027e-06\\
599.7	3.70545931271775e-06\\
599.71	3.477509401573e-06\\
599.72	3.25609578728098e-06\\
599.73	3.04128525271952e-06\\
599.74	2.83314520516977e-06\\
599.75	2.63174368469489e-06\\
599.76	2.43714937234532e-06\\
599.77	2.24943159845246e-06\\
599.78	2.06866035230673e-06\\
599.79	1.89490629239931e-06\\
599.8	1.72824075721396e-06\\
599.81	1.56873577666743e-06\\
599.82	1.41646408419877e-06\\
599.83	1.27149912958896e-06\\
599.84	1.13391509257328e-06\\
599.85	1.00378689725684e-06\\
599.86	8.81190227465176e-07\\
599.87	7.66201543043674e-07\\
599.88	6.58898097204846e-07\\
599.89	5.59357955015258e-07\\
599.9	4.67660013056884e-07\\
599.91	3.8388402042247e-07\\
599.92	3.08110601103875e-07\\
599.93	2.40421277865333e-07\\
599.94	1.80898497769907e-07\\
599.95	1.29625659426799e-07\\
599.96	8.66871421312254e-08\\
599.97	5.21683370079129e-08\\
599.98	2.61556803542173e-08\\
599.99	8.73668930083393e-09\\
600	0\\
};
\addplot [color=mycolor17,solid,forget plot]
  table[row sep=crcr]{%
0.01	0.01\\
1.01	0.01\\
2.01	0.01\\
3.01	0.01\\
4.01	0.01\\
5.01	0.01\\
6.01	0.01\\
7.01	0.01\\
8.01	0.01\\
9.01	0.01\\
10.01	0.01\\
11.01	0.01\\
12.01	0.01\\
13.01	0.01\\
14.01	0.01\\
15.01	0.01\\
16.01	0.01\\
17.01	0.01\\
18.01	0.01\\
19.01	0.01\\
20.01	0.01\\
21.01	0.01\\
22.01	0.01\\
23.01	0.01\\
24.01	0.01\\
25.01	0.01\\
26.01	0.01\\
27.01	0.01\\
28.01	0.01\\
29.01	0.01\\
30.01	0.01\\
31.01	0.01\\
32.01	0.01\\
33.01	0.01\\
34.01	0.01\\
35.01	0.01\\
36.01	0.01\\
37.01	0.01\\
38.01	0.01\\
39.01	0.01\\
40.01	0.01\\
41.01	0.01\\
42.01	0.01\\
43.01	0.01\\
44.01	0.01\\
45.01	0.01\\
46.01	0.01\\
47.01	0.01\\
48.01	0.01\\
49.01	0.01\\
50.01	0.01\\
51.01	0.01\\
52.01	0.01\\
53.01	0.01\\
54.01	0.01\\
55.01	0.01\\
56.01	0.01\\
57.01	0.01\\
58.01	0.01\\
59.01	0.01\\
60.01	0.01\\
61.01	0.01\\
62.01	0.01\\
63.01	0.01\\
64.01	0.01\\
65.01	0.01\\
66.01	0.01\\
67.01	0.01\\
68.01	0.01\\
69.01	0.01\\
70.01	0.01\\
71.01	0.01\\
72.01	0.01\\
73.01	0.01\\
74.01	0.01\\
75.01	0.01\\
76.01	0.01\\
77.01	0.01\\
78.01	0.01\\
79.01	0.01\\
80.01	0.01\\
81.01	0.01\\
82.01	0.01\\
83.01	0.01\\
84.01	0.01\\
85.01	0.01\\
86.01	0.01\\
87.01	0.01\\
88.01	0.01\\
89.01	0.01\\
90.01	0.01\\
91.01	0.01\\
92.01	0.01\\
93.01	0.01\\
94.01	0.01\\
95.01	0.01\\
96.01	0.01\\
97.01	0.01\\
98.01	0.01\\
99.01	0.01\\
100.01	0.01\\
101.01	0.01\\
102.01	0.01\\
103.01	0.01\\
104.01	0.01\\
105.01	0.01\\
106.01	0.01\\
107.01	0.01\\
108.01	0.01\\
109.01	0.01\\
110.01	0.01\\
111.01	0.01\\
112.01	0.01\\
113.01	0.01\\
114.01	0.01\\
115.01	0.01\\
116.01	0.01\\
117.01	0.01\\
118.01	0.01\\
119.01	0.01\\
120.01	0.01\\
121.01	0.01\\
122.01	0.01\\
123.01	0.01\\
124.01	0.01\\
125.01	0.01\\
126.01	0.01\\
127.01	0.01\\
128.01	0.01\\
129.01	0.01\\
130.01	0.01\\
131.01	0.01\\
132.01	0.01\\
133.01	0.01\\
134.01	0.01\\
135.01	0.01\\
136.01	0.01\\
137.01	0.01\\
138.01	0.01\\
139.01	0.01\\
140.01	0.01\\
141.01	0.01\\
142.01	0.01\\
143.01	0.01\\
144.01	0.01\\
145.01	0.01\\
146.01	0.01\\
147.01	0.01\\
148.01	0.01\\
149.01	0.01\\
150.01	0.01\\
151.01	0.01\\
152.01	0.01\\
153.01	0.01\\
154.01	0.01\\
155.01	0.01\\
156.01	0.01\\
157.01	0.01\\
158.01	0.01\\
159.01	0.01\\
160.01	0.01\\
161.01	0.01\\
162.01	0.01\\
163.01	0.01\\
164.01	0.01\\
165.01	0.01\\
166.01	0.01\\
167.01	0.01\\
168.01	0.01\\
169.01	0.01\\
170.01	0.01\\
171.01	0.01\\
172.01	0.01\\
173.01	0.01\\
174.01	0.01\\
175.01	0.01\\
176.01	0.01\\
177.01	0.01\\
178.01	0.01\\
179.01	0.01\\
180.01	0.01\\
181.01	0.01\\
182.01	0.01\\
183.01	0.01\\
184.01	0.01\\
185.01	0.01\\
186.01	0.01\\
187.01	0.01\\
188.01	0.01\\
189.01	0.01\\
190.01	0.01\\
191.01	0.01\\
192.01	0.01\\
193.01	0.01\\
194.01	0.01\\
195.01	0.01\\
196.01	0.01\\
197.01	0.01\\
198.01	0.01\\
199.01	0.01\\
200.01	0.01\\
201.01	0.01\\
202.01	0.01\\
203.01	0.01\\
204.01	0.01\\
205.01	0.01\\
206.01	0.01\\
207.01	0.01\\
208.01	0.01\\
209.01	0.01\\
210.01	0.01\\
211.01	0.01\\
212.01	0.01\\
213.01	0.01\\
214.01	0.01\\
215.01	0.01\\
216.01	0.01\\
217.01	0.01\\
218.01	0.01\\
219.01	0.01\\
220.01	0.01\\
221.01	0.01\\
222.01	0.01\\
223.01	0.01\\
224.01	0.01\\
225.01	0.01\\
226.01	0.01\\
227.01	0.01\\
228.01	0.01\\
229.01	0.01\\
230.01	0.01\\
231.01	0.01\\
232.01	0.01\\
233.01	0.01\\
234.01	0.01\\
235.01	0.01\\
236.01	0.01\\
237.01	0.01\\
238.01	0.01\\
239.01	0.01\\
240.01	0.01\\
241.01	0.01\\
242.01	0.01\\
243.01	0.01\\
244.01	0.01\\
245.01	0.01\\
246.01	0.01\\
247.01	0.01\\
248.01	0.01\\
249.01	0.01\\
250.01	0.01\\
251.01	0.01\\
252.01	0.01\\
253.01	0.01\\
254.01	0.01\\
255.01	0.01\\
256.01	0.01\\
257.01	0.01\\
258.01	0.01\\
259.01	0.01\\
260.01	0.01\\
261.01	0.01\\
262.01	0.01\\
263.01	0.01\\
264.01	0.01\\
265.01	0.01\\
266.01	0.01\\
267.01	0.01\\
268.01	0.01\\
269.01	0.01\\
270.01	0.01\\
271.01	0.01\\
272.01	0.01\\
273.01	0.01\\
274.01	0.01\\
275.01	0.01\\
276.01	0.01\\
277.01	0.01\\
278.01	0.01\\
279.01	0.01\\
280.01	0.01\\
281.01	0.01\\
282.01	0.01\\
283.01	0.01\\
284.01	0.01\\
285.01	0.01\\
286.01	0.01\\
287.01	0.01\\
288.01	0.01\\
289.01	0.01\\
290.01	0.01\\
291.01	0.01\\
292.01	0.01\\
293.01	0.01\\
294.01	0.01\\
295.01	0.01\\
296.01	0.01\\
297.01	0.01\\
298.01	0.01\\
299.01	0.01\\
300.01	0.01\\
301.01	0.01\\
302.01	0.01\\
303.01	0.01\\
304.01	0.01\\
305.01	0.01\\
306.01	0.01\\
307.01	0.01\\
308.01	0.01\\
309.01	0.01\\
310.01	0.01\\
311.01	0.01\\
312.01	0.01\\
313.01	0.01\\
314.01	0.01\\
315.01	0.01\\
316.01	0.01\\
317.01	0.01\\
318.01	0.01\\
319.01	0.01\\
320.01	0.01\\
321.01	0.01\\
322.01	0.01\\
323.01	0.01\\
324.01	0.01\\
325.01	0.01\\
326.01	0.01\\
327.01	0.01\\
328.01	0.01\\
329.01	0.01\\
330.01	0.01\\
331.01	0.01\\
332.01	0.01\\
333.01	0.01\\
334.01	0.01\\
335.01	0.01\\
336.01	0.01\\
337.01	0.01\\
338.01	0.01\\
339.01	0.01\\
340.01	0.01\\
341.01	0.01\\
342.01	0.01\\
343.01	0.01\\
344.01	0.01\\
345.01	0.01\\
346.01	0.01\\
347.01	0.01\\
348.01	0.01\\
349.01	0.01\\
350.01	0.01\\
351.01	0.01\\
352.01	0.01\\
353.01	0.01\\
354.01	0.01\\
355.01	0.01\\
356.01	0.01\\
357.01	0.01\\
358.01	0.01\\
359.01	0.01\\
360.01	0.01\\
361.01	0.01\\
362.01	0.01\\
363.01	0.01\\
364.01	0.01\\
365.01	0.01\\
366.01	0.01\\
367.01	0.01\\
368.01	0.01\\
369.01	0.01\\
370.01	0.01\\
371.01	0.01\\
372.01	0.01\\
373.01	0.01\\
374.01	0.01\\
375.01	0.01\\
376.01	0.01\\
377.01	0.01\\
378.01	0.01\\
379.01	0.01\\
380.01	0.01\\
381.01	0.01\\
382.01	0.01\\
383.01	0.01\\
384.01	0.01\\
385.01	0.01\\
386.01	0.01\\
387.01	0.01\\
388.01	0.01\\
389.01	0.01\\
390.01	0.01\\
391.01	0.01\\
392.01	0.01\\
393.01	0.01\\
394.01	0.01\\
395.01	0.01\\
396.01	0.01\\
397.01	0.01\\
398.01	0.01\\
399.01	0.01\\
400.01	0.01\\
401.01	0.01\\
402.01	0.01\\
403.01	0.01\\
404.01	0.01\\
405.01	0.01\\
406.01	0.01\\
407.01	0.01\\
408.01	0.01\\
409.01	0.01\\
410.01	0.01\\
411.01	0.01\\
412.01	0.01\\
413.01	0.01\\
414.01	0.01\\
415.01	0.01\\
416.01	0.01\\
417.01	0.01\\
418.01	0.01\\
419.01	0.01\\
420.01	0.01\\
421.01	0.01\\
422.01	0.01\\
423.01	0.01\\
424.01	0.01\\
425.01	0.01\\
426.01	0.01\\
427.01	0.01\\
428.01	0.01\\
429.01	0.01\\
430.01	0.01\\
431.01	0.01\\
432.01	0.01\\
433.01	0.01\\
434.01	0.01\\
435.01	0.01\\
436.01	0.01\\
437.01	0.01\\
438.01	0.01\\
439.01	0.01\\
440.01	0.01\\
441.01	0.01\\
442.01	0.01\\
443.01	0.01\\
444.01	0.01\\
445.01	0.01\\
446.01	0.01\\
447.01	0.01\\
448.01	0.01\\
449.01	0.01\\
450.01	0.01\\
451.01	0.01\\
452.01	0.01\\
453.01	0.01\\
454.01	0.01\\
455.01	0.01\\
456.01	0.01\\
457.01	0.01\\
458.01	0.01\\
459.01	0.01\\
460.01	0.01\\
461.01	0.00999078095185203\\
462.01	0.00997419417939734\\
463.01	0.009957082611733\\
464.01	0.00993942808881208\\
465.01	0.00992121208632641\\
466.01	0.00990241583879067\\
467.01	0.00988302051352144\\
468.01	0.00986300745183245\\
469.01	0.00984235849873459\\
470.01	0.00982105644896362\\
471.01	0.0097990856459315\\
472.01	0.00977643278174111\\
473.01	0.00975308797251662\\
474.01	0.00972904630606034\\
475.01	0.00970430778655313\\
476.01	0.00967885773352897\\
477.01	0.00965267082451434\\
478.01	0.00962572037708996\\
479.01	0.00959797835263745\\
480.01	0.00956941526577682\\
481.01	0.00954000008832665\\
482.01	0.00950970014810938\\
483.01	0.00947848102321919\\
484.01	0.00944630643266237\\
485.01	0.00941313812349246\\
486.01	0.00937893574377376\\
487.01	0.00934365667464908\\
488.01	0.00930725604099166\\
489.01	0.00926968669288044\\
490.01	0.00923089904542123\\
491.01	0.00919084104334817\\
492.01	0.0091494581801977\\
493.01	0.00910669359017154\\
494.01	0.00906248826125746\\
495.01	0.00901678156923852\\
496.01	0.0089695132699148\\
497.01	0.00892062368141041\\
498.01	0.0088700333650678\\
499.01	0.00881764971181154\\
500.01	0.00876337149402897\\
501.01	0.0087070876953619\\
502.01	0.0086486761348062\\
503.01	0.00858800184576233\\
504.01	0.00852491515234896\\
505.01	0.00845924938584853\\
506.01	0.00839081825084369\\
507.01	0.00831941260721302\\
508.01	0.00824479661368092\\
509.01	0.00816670313252492\\
510.01	0.00808482820202401\\
511.01	0.00799882436587168\\
512.01	0.00790829259769038\\
513.01	0.00781277249428077\\
514.01	0.00771173033149313\\
515.01	0.00760454449104528\\
516.01	0.00749048781583137\\
517.01	0.00736870869453522\\
518.01	0.00723963818854821\\
519.01	0.00710597811992642\\
520.01	0.00696760229751159\\
521.01	0.00682422000099556\\
522.01	0.00667551254793456\\
523.01	0.00659587815347017\\
524.01	0.00651633098827587\\
525.01	0.00643473006207726\\
526.01	0.00635106660776621\\
527.01	0.00626534366701773\\
528.01	0.00617757882137513\\
529.01	0.00608780722930537\\
530.01	0.00599608567697558\\
531.01	0.00590249750060823\\
532.01	0.00580715853831556\\
533.01	0.00571022394240866\\
534.01	0.0056118929793264\\
535.01	0.00551242338902114\\
536.01	0.00541215254983005\\
537.01	0.00531151088970588\\
538.01	0.00521102795614884\\
539.01	0.00511137278512313\\
540.01	0.00501339837207893\\
541.01	0.00491819638302972\\
542.01	0.00482625485291065\\
543.01	0.0047326019711171\\
544.01	0.00463623227679512\\
545.01	0.00453723161982669\\
546.01	0.00443576164286419\\
547.01	0.00433208802327658\\
548.01	0.00422661498624084\\
549.01	0.00411993151685858\\
550.01	0.00401290011585353\\
551.01	0.00390721634275333\\
552.01	0.0038026967159869\\
553.01	0.00369958105289584\\
554.01	0.00359821419173106\\
555.01	0.00349895359268961\\
556.01	0.00340215802666711\\
557.01	0.00330817029183982\\
558.01	0.00321729180630797\\
559.01	0.00312974662100248\\
560.01	0.00304562319959933\\
561.01	0.0029648310383562\\
562.01	0.00288723971163634\\
563.01	0.00281168088604776\\
564.01	0.00273659112181641\\
565.01	0.00266192996599381\\
566.01	0.00258771427401347\\
567.01	0.00251394505248555\\
568.01	0.00244063027530275\\
569.01	0.00236774713459566\\
570.01	0.00229523575951096\\
571.01	0.00222299396948937\\
572.01	0.00215087054873885\\
573.01	0.00207867101495317\\
574.01	0.00200620911203144\\
575.01	0.00193337366729925\\
576.01	0.00186014255824003\\
577.01	0.00178649907178294\\
578.01	0.00171242998631992\\
579.01	0.0016379206162627\\
580.01	0.0015629515820556\\
581.01	0.00148749997079724\\
582.01	0.00141154117821388\\
583.01	0.00133505131726857\\
584.01	0.00125800997726646\\
585.01	0.00118040287913029\\
586.01	0.00110222304799643\\
587.01	0.00102347123973567\\
588.01	0.00094415206962192\\
589.01	0.000864266394493165\\
590.01	0.00078381051162838\\
591.01	0.00070277603982059\\
592.01	0.000621149948535656\\
593.01	0.000538914720361251\\
594.01	0.000456048589420978\\
595.01	0.000372525732669495\\
596.01	0.000288316195459062\\
597.01	0.00020338520208954\\
598.01	0.000117691340829394\\
599.01	3.2249382697221e-05\\
599.02	3.15980747054457e-05\\
599.03	3.09676859235018e-05\\
599.04	3.03584576151362e-05\\
599.05	2.97705837207663e-05\\
599.06	2.92042621379707e-05\\
599.07	2.86596948884162e-05\\
599.08	2.8123912998973e-05\\
599.09	2.75930227461897e-05\\
599.1	2.70667931478105e-05\\
599.11	2.65452441665006e-05\\
599.12	2.60282881461519e-05\\
599.13	2.55154035494164e-05\\
599.14	2.50066056569926e-05\\
599.15	2.45019084058242e-05\\
599.16	2.40013242945435e-05\\
599.17	2.35048471433803e-05\\
599.18	2.30124445947163e-05\\
599.19	2.25241241168985e-05\\
599.2	2.20398912888203e-05\\
599.21	2.15597507792053e-05\\
599.22	2.10837069131182e-05\\
599.23	2.0611761905133e-05\\
599.24	2.01439157217088e-05\\
599.25	1.96801659357645e-05\\
599.26	1.92205075730031e-05\\
599.27	1.87649527270909e-05\\
599.28	1.83135392298733e-05\\
599.29	1.78663051708071e-05\\
599.3	1.74232888957779e-05\\
599.31	1.6984530027039e-05\\
599.32	1.65500694928791e-05\\
599.33	1.61199486044834e-05\\
599.34	1.56942090620352e-05\\
599.35	1.52728931138019e-05\\
599.36	1.48560434799095e-05\\
599.37	1.44437033046136e-05\\
599.38	1.40359161648531e-05\\
599.39	1.36327268429586e-05\\
599.4	1.32341836384959e-05\\
599.41	1.28403353542913e-05\\
599.42	1.24512313076607e-05\\
599.43	1.2066921342313e-05\\
599.44	1.16874558409707e-05\\
599.45	1.13128857387526e-05\\
599.46	1.09432625373523e-05\\
599.47	1.05786383200971e-05\\
599.48	1.02190657679229e-05\\
599.49	9.86459817164075e-06\\
599.5	9.51528944691422e-06\\
599.51	9.17119415258169e-06\\
599.52	8.83236751010076e-06\\
599.53	8.4988654011544e-06\\
599.54	8.17074426504201e-06\\
599.55	7.84806110491927e-06\\
599.56	7.53087349160408e-06\\
599.57	7.21923956508233e-06\\
599.58	6.91321804026715e-06\\
599.59	6.6128682119828e-06\\
599.6	6.31824996075517e-06\\
599.61	6.02942375867339e-06\\
599.62	5.74645067535212e-06\\
599.63	5.46939238393711e-06\\
599.64	5.19831116718369e-06\\
599.65	4.93326992361681e-06\\
599.66	4.67433217375343e-06\\
599.67	4.42156206638573e-06\\
599.68	4.17502438496142e-06\\
599.69	3.93478455401261e-06\\
599.7	3.70090864566797e-06\\
599.71	3.47346338623079e-06\\
599.72	3.2525161628473e-06\\
599.73	3.03813503022173e-06\\
599.74	2.83038871743899e-06\\
599.75	2.62934663483247e-06\\
599.76	2.43507888095067e-06\\
599.77	2.2476562495672e-06\\
599.78	2.06715023678798e-06\\
599.79	1.89363304820347e-06\\
599.8	1.72717760612943e-06\\
599.81	1.56785755689968e-06\\
599.82	1.41574727823349e-06\\
599.83	1.27092188665495e-06\\
599.84	1.1334572449697e-06\\
599.85	1.00342996981438e-06\\
599.86	8.80917439232201e-07\\
599.87	7.65997800316123e-07\\
599.88	6.58749976879813e-07\\
599.89	5.59253677163279e-07\\
599.9	4.67589401585353e-07\\
599.91	3.83838450489227e-07\\
599.92	3.08082931920958e-07\\
599.93	2.40405769407967e-07\\
599.94	1.8089070970978e-07\\
599.95	1.2962233057745e-07\\
599.96	8.66860484539933e-08\\
599.97	5.21681261331924e-08\\
599.98	2.61556803542173e-08\\
599.99	8.73668930083393e-09\\
600	0\\
};
\addplot [color=mycolor18,solid,forget plot]
  table[row sep=crcr]{%
0.01	0.0085282693244888\\
1.01	0.00852826887287535\\
2.01	0.00852826841151541\\
3.01	0.00852826794019706\\
4.01	0.00852826745870373\\
5.01	0.00852826696681405\\
6.01	0.00852826646430188\\
7.01	0.00852826595093601\\
8.01	0.00852826542648029\\
9.01	0.00852826489069324\\
10.01	0.0085282643433281\\
11.01	0.00852826378413266\\
12.01	0.00852826321284936\\
13.01	0.0085282626292147\\
14.01	0.00852826203295951\\
15.01	0.00852826142380871\\
16.01	0.00852826080148116\\
17.01	0.00852826016568943\\
18.01	0.00852825951613987\\
19.01	0.00852825885253234\\
20.01	0.00852825817456001\\
21.01	0.0085282574819093\\
22.01	0.00852825677425974\\
23.01	0.00852825605128382\\
24.01	0.00852825531264676\\
25.01	0.00852825455800638\\
26.01	0.00852825378701293\\
27.01	0.00852825299930898\\
28.01	0.00852825219452916\\
29.01	0.00852825137229997\\
30.01	0.0085282505322398\\
31.01	0.00852824967395849\\
32.01	0.00852824879705723\\
33.01	0.00852824790112842\\
34.01	0.00852824698575541\\
35.01	0.00852824605051238\\
36.01	0.008528245094964\\
37.01	0.00852824411866531\\
38.01	0.00852824312116146\\
39.01	0.00852824210198756\\
40.01	0.00852824106066839\\
41.01	0.00852823999671815\\
42.01	0.00852823890964029\\
43.01	0.00852823779892717\\
44.01	0.00852823666405996\\
45.01	0.00852823550450822\\
46.01	0.00852823431972977\\
47.01	0.00852823310917022\\
48.01	0.00852823187226311\\
49.01	0.00852823060842911\\
50.01	0.00852822931707615\\
51.01	0.00852822799759891\\
52.01	0.00852822664937859\\
53.01	0.00852822527178259\\
54.01	0.00852822386416421\\
55.01	0.00852822242586225\\
56.01	0.00852822095620088\\
57.01	0.00852821945448912\\
58.01	0.00852821792002046\\
59.01	0.00852821635207279\\
60.01	0.00852821474990769\\
61.01	0.0085282131127703\\
62.01	0.00852821143988891\\
63.01	0.00852820973047442\\
64.01	0.0085282079837202\\
65.01	0.0085282061988014\\
66.01	0.00852820437487486\\
67.01	0.0085282025110784\\
68.01	0.00852820060653054\\
69.01	0.00852819866033\\
70.01	0.0085281966715553\\
71.01	0.00852819463926426\\
72.01	0.00852819256249349\\
73.01	0.00852819044025805\\
74.01	0.00852818827155076\\
75.01	0.00852818605534186\\
76.01	0.0085281837905783\\
77.01	0.00852818147618344\\
78.01	0.00852817911105634\\
79.01	0.0085281766940713\\
80.01	0.00852817422407719\\
81.01	0.00852817169989689\\
82.01	0.0085281691203268\\
83.01	0.00852816648413615\\
84.01	0.00852816379006628\\
85.01	0.00852816103683019\\
86.01	0.00852815822311177\\
87.01	0.00852815534756507\\
88.01	0.00852815240881381\\
89.01	0.00852814940545044\\
90.01	0.00852814633603559\\
91.01	0.00852814319909731\\
92.01	0.00852813999313021\\
93.01	0.00852813671659478\\
94.01	0.00852813336791666\\
95.01	0.00852812994548565\\
96.01	0.00852812644765504\\
97.01	0.00852812287274061\\
98.01	0.00852811921901989\\
99.01	0.00852811548473132\\
100.01	0.00852811166807319\\
101.01	0.00852810776720269\\
102.01	0.00852810378023512\\
103.01	0.00852809970524282\\
104.01	0.00852809554025416\\
105.01	0.00852809128325249\\
106.01	0.0085280869321752\\
107.01	0.00852808248491249\\
108.01	0.00852807793930634\\
109.01	0.00852807329314944\\
110.01	0.00852806854418401\\
111.01	0.0085280636901005\\
112.01	0.00852805872853654\\
113.01	0.00852805365707559\\
114.01	0.0085280484732458\\
115.01	0.0085280431745185\\
116.01	0.00852803775830707\\
117.01	0.00852803222196547\\
118.01	0.0085280265627868\\
119.01	0.008528020778002\\
120.01	0.00852801486477833\\
121.01	0.00852800882021782\\
122.01	0.00852800264135578\\
123.01	0.00852799632515918\\
124.01	0.00852798986852503\\
125.01	0.00852798326827892\\
126.01	0.00852797652117303\\
127.01	0.0085279696238846\\
128.01	0.00852796257301403\\
129.01	0.00852795536508316\\
130.01	0.00852794799653337\\
131.01	0.00852794046372366\\
132.01	0.0085279327629287\\
133.01	0.00852792489033672\\
134.01	0.0085279168420477\\
135.01	0.00852790861407104\\
136.01	0.00852790020232355\\
137.01	0.00852789160262709\\
138.01	0.00852788281070649\\
139.01	0.0085278738221871\\
140.01	0.00852786463259244\\
141.01	0.00852785523734174\\
142.01	0.00852784563174758\\
143.01	0.00852783581101312\\
144.01	0.00852782577022966\\
145.01	0.00852781550437395\\
146.01	0.00852780500830533\\
147.01	0.00852779427676297\\
148.01	0.00852778330436295\\
149.01	0.00852777208559542\\
150.01	0.00852776061482139\\
151.01	0.00852774888626977\\
152.01	0.00852773689403411\\
153.01	0.00852772463206939\\
154.01	0.00852771209418864\\
155.01	0.00852769927405947\\
156.01	0.00852768616520067\\
157.01	0.0085276727609785\\
158.01	0.00852765905460308\\
159.01	0.00852764503912454\\
160.01	0.0085276307074292\\
161.01	0.00852761605223562\\
162.01	0.00852760106609042\\
163.01	0.00852758574136424\\
164.01	0.00852757007024735\\
165.01	0.00852755404474531\\
166.01	0.00852753765667452\\
167.01	0.00852752089765755\\
168.01	0.00852750375911848\\
169.01	0.00852748623227795\\
170.01	0.00852746830814835\\
171.01	0.00852744997752857\\
172.01	0.008527431230999\\
173.01	0.00852741205891598\\
174.01	0.00852739245140631\\
175.01	0.00852737239836189\\
176.01	0.00852735188943368\\
177.01	0.00852733091402596\\
178.01	0.00852730946129019\\
179.01	0.0085272875201189\\
180.01	0.00852726507913922\\
181.01	0.00852724212670645\\
182.01	0.00852721865089733\\
183.01	0.00852719463950314\\
184.01	0.00852717008002282\\
185.01	0.00852714495965563\\
186.01	0.00852711926529373\\
187.01	0.00852709298351479\\
188.01	0.00852706610057399\\
189.01	0.00852703860239624\\
190.01	0.00852701047456802\\
191.01	0.00852698170232879\\
192.01	0.00852695227056261\\
193.01	0.00852692216378937\\
194.01	0.00852689136615564\\
195.01	0.00852685986142535\\
196.01	0.00852682763297052\\
197.01	0.00852679466376136\\
198.01	0.0085267609363564\\
199.01	0.00852672643289213\\
200.01	0.00852669113507269\\
201.01	0.00852665502415893\\
202.01	0.00852661808095746\\
203.01	0.00852658028580938\\
204.01	0.00852654161857869\\
205.01	0.00852650205864015\\
206.01	0.0085264615848675\\
207.01	0.00852642017562057\\
208.01	0.00852637780873278\\
209.01	0.00852633446149765\\
210.01	0.00852629011065562\\
211.01	0.00852624473238001\\
212.01	0.00852619830226295\\
213.01	0.00852615079530072\\
214.01	0.00852610218587904\\
215.01	0.0085260524477576\\
216.01	0.00852600155405443\\
217.01	0.00852594947722981\\
218.01	0.0085258961890697\\
219.01	0.00852584166066901\\
220.01	0.00852578586241413\\
221.01	0.00852572876396513\\
222.01	0.00852567033423764\\
223.01	0.00852561054138394\\
224.01	0.00852554935277405\\
225.01	0.00852548673497577\\
226.01	0.00852542265373473\\
227.01	0.00852535707395371\\
228.01	0.00852528995967124\\
229.01	0.00852522127404003\\
230.01	0.0085251509793045\\
231.01	0.00852507903677823\\
232.01	0.0085250054068201\\
233.01	0.00852493004881053\\
234.01	0.00852485292112682\\
235.01	0.00852477398111765\\
236.01	0.00852469318507734\\
237.01	0.00852461048821923\\
238.01	0.00852452584464841\\
239.01	0.00852443920733379\\
240.01	0.00852435052807936\\
241.01	0.00852425975749496\\
242.01	0.00852416684496602\\
243.01	0.00852407173862272\\
244.01	0.00852397438530831\\
245.01	0.0085238747305466\\
246.01	0.00852377271850874\\
247.01	0.00852366829197907\\
248.01	0.00852356139232001\\
249.01	0.0085234519594363\\
250.01	0.00852333993173823\\
251.01	0.00852322524610382\\
252.01	0.00852310783784022\\
253.01	0.00852298764064399\\
254.01	0.00852286458656052\\
255.01	0.00852273860594251\\
256.01	0.00852260962740696\\
257.01	0.00852247757779165\\
258.01	0.00852234238211008\\
259.01	0.00852220396350558\\
260.01	0.00852206224320412\\
261.01	0.00852191714046591\\
262.01	0.00852176857253592\\
263.01	0.00852161645459301\\
264.01	0.00852146069969794\\
265.01	0.00852130121873993\\
266.01	0.00852113792038201\\
267.01	0.00852097071100482\\
268.01	0.0085207994946494\\
269.01	0.00852062417295801\\
270.01	0.00852044464511388\\
271.01	0.00852026080777938\\
272.01	0.00852007255503248\\
273.01	0.00851987977830176\\
274.01	0.00851968236629997\\
275.01	0.00851948020495553\\
276.01	0.00851927317734264\\
277.01	0.00851906116360957\\
278.01	0.00851884404090515\\
279.01	0.00851862168330348\\
280.01	0.0085183939617267\\
281.01	0.00851816074386581\\
282.01	0.00851792189409971\\
283.01	0.00851767727341198\\
284.01	0.00851742673930586\\
285.01	0.00851717014571683\\
286.01	0.00851690734292317\\
287.01	0.00851663817745436\\
288.01	0.00851636249199709\\
289.01	0.00851608012529895\\
290.01	0.00851579091206974\\
291.01	0.00851549468288027\\
292.01	0.00851519126405875\\
293.01	0.00851488047758453\\
294.01	0.00851456214097902\\
295.01	0.00851423606719427\\
296.01	0.00851390206449821\\
297.01	0.00851355993635757\\
298.01	0.00851320948131761\\
299.01	0.00851285049287855\\
300.01	0.00851248275936947\\
301.01	0.00851210606381833\\
302.01	0.00851172018381939\\
303.01	0.00851132489139673\\
304.01	0.00851091995286449\\
305.01	0.00851050512868361\\
306.01	0.00851008017331476\\
307.01	0.00850964483506748\\
308.01	0.00850919885594564\\
309.01	0.00850874197148854\\
310.01	0.00850827391060823\\
311.01	0.00850779439542229\\
312.01	0.00850730314108252\\
313.01	0.0085067998555988\\
314.01	0.00850628423965839\\
315.01	0.00850575598644074\\
316.01	0.00850521478142667\\
317.01	0.00850466030220307\\
318.01	0.00850409221826213\\
319.01	0.00850351019079479\\
320.01	0.00850291387247883\\
321.01	0.00850230290726128\\
322.01	0.00850167693013432\\
323.01	0.0085010355669054\\
324.01	0.00850037843396062\\
325.01	0.00849970513802145\\
326.01	0.0084990152758946\\
327.01	0.00849830843421463\\
328.01	0.00849758418917936\\
329.01	0.00849684210627718\\
330.01	0.00849608174000655\\
331.01	0.00849530263358736\\
332.01	0.00849450431866355\\
333.01	0.00849368631499666\\
334.01	0.0084928481301504\\
335.01	0.00849198925916552\\
336.01	0.00849110918422448\\
337.01	0.00849020737430603\\
338.01	0.00848928328482888\\
339.01	0.00848833635728427\\
340.01	0.00848736601885668\\
341.01	0.00848637168203296\\
342.01	0.00848535274419822\\
343.01	0.00848430858721917\\
344.01	0.00848323857701382\\
345.01	0.00848214206310694\\
346.01	0.00848101837817107\\
347.01	0.00847986683755223\\
348.01	0.00847868673877964\\
349.01	0.00847747736105926\\
350.01	0.00847623796475034\\
351.01	0.00847496779082389\\
352.01	0.00847366606030291\\
353.01	0.00847233197368387\\
354.01	0.00847096471033751\\
355.01	0.00846956342789008\\
356.01	0.00846812726158228\\
357.01	0.00846665532360672\\
358.01	0.00846514670242193\\
359.01	0.00846360046204281\\
360.01	0.00846201564130635\\
361.01	0.00846039125311161\\
362.01	0.00845872628363337\\
363.01	0.00845701969150814\\
364.01	0.00845527040699201\\
365.01	0.00845347733108871\\
366.01	0.00845163933464759\\
367.01	0.00844975525743003\\
368.01	0.00844782390714294\\
369.01	0.00844584405843904\\
370.01	0.00844381445188201\\
371.01	0.00844173379287559\\
372.01	0.00843960075055563\\
373.01	0.00843741395664327\\
374.01	0.0084351720042585\\
375.01	0.00843287344669222\\
376.01	0.00843051679613575\\
377.01	0.00842810052236599\\
378.01	0.00842562305138452\\
379.01	0.00842308276400934\\
380.01	0.00842047799441703\\
381.01	0.00841780702863358\\
382.01	0.00841506810297179\\
383.01	0.00841225940241283\\
384.01	0.00840937905892985\\
385.01	0.00840642514975087\\
386.01	0.00840339569555807\\
387.01	0.00840028865862046\\
388.01	0.00839710194085653\\
389.01	0.00839383338182355\\
390.01	0.00839048075662889\\
391.01	0.00838704177375985\\
392.01	0.00838351407282648\\
393.01	0.00837989522221236\\
394.01	0.00837618271662836\\
395.01	0.00837237397456203\\
396.01	0.00836846633561692\\
397.01	0.0083644570577336\\
398.01	0.00836034331428519\\
399.01	0.00835612219103809\\
400.01	0.00835179068296904\\
401.01	0.00834734569092825\\
402.01	0.00834278401813774\\
403.01	0.0083381023665139\\
404.01	0.00833329733280185\\
405.01	0.00832836540450924\\
406.01	0.00832330295562662\\
407.01	0.00831810624212161\\
408.01	0.00831277139719408\\
409.01	0.00830729442628117\\
410.01	0.0083016712018026\\
411.01	0.00829589745764005\\
412.01	0.00828996878335004\\
413.01	0.00828388061811719\\
414.01	0.00827762824445846\\
415.01	0.00827120678153902\\
416.01	0.00826461117320351\\
417.01	0.0082578361846318\\
418.01	0.00825087643389198\\
419.01	0.00824372635047618\\
420.01	0.00823638016567071\\
421.01	0.00822883193369906\\
422.01	0.00822107557696015\\
423.01	0.00821310472571571\\
424.01	0.00820491269494183\\
425.01	0.00819649253760645\\
426.01	0.00818783703140281\\
427.01	0.00817893866291245\\
428.01	0.00816978961040823\\
429.01	0.00816038172513833\\
430.01	0.00815070651090978\\
431.01	0.00814075510176216\\
432.01	0.00813051823749057\\
433.01	0.00811998623674044\\
434.01	0.00810914896735115\\
435.01	0.00809799581357453\\
436.01	0.00808651563973082\\
437.01	0.00807469674979188\\
438.01	0.00806252684229236\\
439.01	0.00804999295986385\\
440.01	0.00803708143255964\\
441.01	0.00802377781398531\\
442.01	0.00801006680906401\\
443.01	0.00799593219204271\\
444.01	0.00798135671307224\\
445.01	0.00796632199136345\\
446.01	0.00795080839251616\\
447.01	0.00793479488712468\\
448.01	0.00791825888715257\\
449.01	0.00790117605582211\\
450.01	0.00788352008580184\\
451.01	0.00786526243873771\\
452.01	0.00784637203025313\\
453.01	0.00782681487280826\\
454.01	0.00780655371765629\\
455.01	0.00778554744100704\\
456.01	0.00776375037486257\\
457.01	0.00774111152438106\\
458.01	0.00771757362418356\\
459.01	0.007693072062799\\
460.01	0.00766753430450629\\
461.01	0.00765013319258523\\
462.01	0.00763908808592548\\
463.01	0.00762750964586218\\
464.01	0.00761534584623699\\
465.01	0.00760253654626921\\
466.01	0.00758901191196083\\
467.01	0.00757469048882546\\
468.01	0.0075594768419738\\
469.01	0.00754325865786523\\
470.01	0.00752590317422022\\
471.01	0.0075072527753302\\
472.01	0.0074871195413191\\
473.01	0.00746527851243589\\
474.01	0.00744145998050491\\
475.01	0.00741583934100435\\
476.01	0.00738948680289242\\
477.01	0.00736244307460696\\
478.01	0.00733469087315198\\
479.01	0.00730621265144998\\
480.01	0.0072769906088408\\
481.01	0.0072470067004919\\
482.01	0.00721624264423079\\
483.01	0.00718467992263655\\
484.01	0.00715229977706487\\
485.01	0.00711908318482394\\
486.01	0.00708501073440597\\
487.01	0.00705006155323819\\
488.01	0.00701421368066979\\
489.01	0.00697744713463668\\
490.01	0.00693974183472198\\
491.01	0.00690107722893671\\
492.01	0.00686143209391965\\
493.01	0.00682078424647268\\
494.01	0.00677911014682784\\
495.01	0.0067363844159549\\
496.01	0.00669257897406453\\
497.01	0.00664766209734195\\
498.01	0.00660161409848995\\
499.01	0.006554423046006\\
500.01	0.00650608067788641\\
501.01	0.00645658341378783\\
502.01	0.0064059336528664\\
503.01	0.00635414145503738\\
504.01	0.00630122645225831\\
505.01	0.00624721976066043\\
506.01	0.00619216778140031\\
507.01	0.00613613602116493\\
508.01	0.00607921259138708\\
509.01	0.00602151344421225\\
510.01	0.00596318885914\\
511.01	0.00590443146170561\\
512.01	0.0058454861538424\\
513.01	0.00578666243426488\\
514.01	0.00572834971386359\\
515.01	0.00567103640614396\\
516.01	0.00561533396415977\\
517.01	0.0055620088737059\\
518.01	0.00551059015308814\\
519.01	0.00545824927896969\\
520.01	0.0054049202581716\\
521.01	0.00535069205361565\\
522.01	0.00529567320000898\\
523.01	0.00523949329040571\\
524.01	0.00518161246188112\\
525.01	0.00512196036795267\\
526.01	0.00506046576669617\\
527.01	0.00499706214664321\\
528.01	0.00493169862185053\\
529.01	0.0048643375711972\\
530.01	0.0047949600813584\\
531.01	0.00472357368012475\\
532.01	0.00465022245915389\\
533.01	0.00457499852317799\\
534.01	0.00449799425098407\\
535.01	0.00441925191459439\\
536.01	0.00433906268995177\\
537.01	0.00425833640930586\\
538.01	0.00417838431595103\\
539.01	0.00409939265699864\\
540.01	0.00402143966493794\\
541.01	0.00394459238813641\\
542.01	0.00386891012544723\\
543.01	0.00379451669365658\\
544.01	0.00372165678674702\\
545.01	0.00365058060888114\\
546.01	0.00358150836090378\\
547.01	0.00351463948028555\\
548.01	0.00345014330083666\\
549.01	0.00338809864123182\\
550.01	0.00332842287084302\\
551.01	0.00326978705576702\\
552.01	0.0032115050237175\\
553.01	0.0031536159779063\\
554.01	0.00309614412897075\\
555.01	0.0030390947138821\\
556.01	0.00298245005115764\\
557.01	0.00292616583595391\\
558.01	0.00287016822054579\\
559.01	0.00281435251935063\\
560.01	0.00275858482690329\\
561.01	0.00270270824872089\\
562.01	0.00264655033766167\\
563.01	0.00258995010948459\\
564.01	0.00253284519234311\\
565.01	0.00247521006594279\\
566.01	0.00241701529277463\\
567.01	0.00235822741438245\\
568.01	0.00229880868214501\\
569.01	0.00223871784692923\\
570.01	0.00217791174383267\\
571.01	0.00211634742053402\\
572.01	0.0020539848162932\\
573.01	0.00199078985346574\\
574.01	0.00192673690423219\\
575.01	0.00186180846663631\\
576.01	0.00179598983847555\\
577.01	0.00172926700343405\\
578.01	0.00166162680595677\\
579.01	0.00159305719081834\\
580.01	0.00152354764606489\\
581.01	0.00145308967998356\\
582.01	0.0013816772350871\\
583.01	0.00130930697684359\\
584.01	0.00123597839125735\\
585.01	0.00116169364078332\\
586.01	0.00108645721274621\\
587.01	0.00101027566830084\\
588.01	0.000933157587584601\\
589.01	0.000855113653909664\\
590.01	0.000776156728517496\\
591.01	0.000696301791282395\\
592.01	0.000615565671439002\\
593.01	0.000533966471584386\\
594.01	0.000451522561386629\\
595.01	0.000368250987049191\\
596.01	0.000284165109520231\\
597.01	0.000199271250183436\\
598.01	0.000113564089302298\\
599.01	3.18261402467001e-05\\
599.02	3.12794051966058e-05\\
599.03	3.07361362371396e-05\\
599.04	3.01962768802927e-05\\
599.05	2.96598193405082e-05\\
599.06	2.91267532562291e-05\\
599.07	2.85970655242371e-05\\
599.08	2.80707764810702e-05\\
599.09	2.75479164762726e-05\\
599.1	2.70285167089327e-05\\
599.11	2.65126085881601e-05\\
599.12	2.60002240642203e-05\\
599.13	2.54913970200877e-05\\
599.14	2.49861616816692e-05\\
599.15	2.44845526258131e-05\\
599.16	2.39866047888303e-05\\
599.17	2.34923535378934e-05\\
599.18	2.30018347767025e-05\\
599.19	2.251508481242e-05\\
599.2	2.20321403675509e-05\\
599.21	2.15530385901663e-05\\
599.22	2.10778170631979e-05\\
599.23	2.06065138181385e-05\\
599.24	2.01391673495636e-05\\
599.25	1.96758166306349e-05\\
599.26	1.92165011295622e-05\\
599.27	1.87612607734713e-05\\
599.28	1.8310135886844e-05\\
599.29	1.78631671960919e-05\\
599.3	1.74203958342063e-05\\
599.31	1.69818633422789e-05\\
599.32	1.65476116708516e-05\\
599.33	1.61176831841251e-05\\
599.34	1.56921206641847e-05\\
599.35	1.52709673147209e-05\\
599.36	1.485426676507e-05\\
599.37	1.44420630744727e-05\\
599.38	1.40344007363812e-05\\
599.39	1.36313255098709e-05\\
599.4	1.3232886592919e-05\\
599.41	1.28391336731322e-05\\
599.42	1.24501169325771e-05\\
599.43	1.20658870526348e-05\\
599.44	1.16864952188697e-05\\
599.45	1.13119931259391e-05\\
599.46	1.09424329824918e-05\\
599.47	1.05778675161157e-05\\
599.48	1.02183499782645e-05\\
599.49	9.86393414922036e-06\\
599.5	9.51467434305575e-06\\
599.51	9.1706254125995e-06\\
599.52	8.83184275438965e-06\\
599.53	8.4983823136868e-06\\
599.54	8.17030058985514e-06\\
599.55	7.84765464179223e-06\\
599.56	7.53050209342451e-06\\
599.57	7.21890113926371e-06\\
599.58	6.91291055002208e-06\\
599.59	6.61258967827975e-06\\
599.6	6.31799846421283e-06\\
599.61	6.02919744138383e-06\\
599.62	5.74624774257564e-06\\
599.63	5.46921110568602e-06\\
599.64	5.19814987969512e-06\\
599.65	4.9331270306676e-06\\
599.66	4.67420614782936e-06\\
599.67	4.42145144970672e-06\\
599.68	4.17492779030207e-06\\
599.69	3.93470066536311e-06\\
599.7	3.70083621868862e-06\\
599.71	3.47340124850354e-06\\
599.72	3.25246321389479e-06\\
599.73	3.03809024131303e-06\\
599.74	2.83035113113513e-06\\
599.75	2.62931536429249e-06\\
599.76	2.43505310895328e-06\\
599.77	2.24763522728433e-06\\
599.78	2.06713328227207e-06\\
599.79	1.89361954460419e-06\\
599.8	1.72716699962938e-06\\
599.81	1.56784935437075e-06\\
599.82	1.41574104461396e-06\\
599.83	1.27091724206295e-06\\
599.84	1.13345386157197e-06\\
599.85	1.00342756843147e-06\\
599.86	8.80915785733682e-07\\
599.87	7.65996701814625e-07\\
599.88	6.58749277755027e-07\\
599.89	5.59253254961076e-07\\
599.9	4.67589162821483e-07\\
599.91	3.83838326433947e-07\\
599.92	3.08082874409671e-07\\
599.93	2.40405746740335e-07\\
599.94	1.80890702781294e-07\\
599.95	1.2962232925906e-07\\
599.96	8.66860484019516e-08\\
599.97	5.21681261349272e-08\\
599.98	2.61556803542173e-08\\
599.99	8.73668930083393e-09\\
600	0\\
};
\addplot [color=red!25!mycolor17,solid,forget plot]
  table[row sep=crcr]{%
0.01	0.00734126507339155\\
1.01	0.00734126455459817\\
2.01	0.00734126402465449\\
3.01	0.00734126348331942\\
4.01	0.00734126293034655\\
5.01	0.00734126236548402\\
6.01	0.0073412617884747\\
7.01	0.00734126119905562\\
8.01	0.00734126059695818\\
9.01	0.00734125998190793\\
10.01	0.00734125935362445\\
11.01	0.00734125871182122\\
12.01	0.00734125805620525\\
13.01	0.00734125738647739\\
14.01	0.00734125670233179\\
15.01	0.00734125600345608\\
16.01	0.00734125528953093\\
17.01	0.00734125456023013\\
18.01	0.00734125381522025\\
19.01	0.00734125305416054\\
20.01	0.00734125227670295\\
21.01	0.00734125148249162\\
22.01	0.00734125067116316\\
23.01	0.00734124984234594\\
24.01	0.00734124899566034\\
25.01	0.00734124813071843\\
26.01	0.00734124724712374\\
27.01	0.00734124634447111\\
28.01	0.0073412454223465\\
29.01	0.00734124448032689\\
30.01	0.00734124351797983\\
31.01	0.00734124253486344\\
32.01	0.00734124153052618\\
33.01	0.00734124050450658\\
34.01	0.00734123945633304\\
35.01	0.00734123838552357\\
36.01	0.00734123729158562\\
37.01	0.0073412361740158\\
38.01	0.00734123503229976\\
39.01	0.00734123386591169\\
40.01	0.0073412326743143\\
41.01	0.00734123145695845\\
42.01	0.0073412302132829\\
43.01	0.00734122894271422\\
44.01	0.00734122764466611\\
45.01	0.0073412263185395\\
46.01	0.0073412249637221\\
47.01	0.00734122357958816\\
48.01	0.00734122216549811\\
49.01	0.00734122072079827\\
50.01	0.00734121924482052\\
51.01	0.00734121773688207\\
52.01	0.00734121619628499\\
53.01	0.00734121462231603\\
54.01	0.00734121301424613\\
55.01	0.00734121137133014\\
56.01	0.00734120969280643\\
57.01	0.00734120797789652\\
58.01	0.00734120622580483\\
59.01	0.00734120443571808\\
60.01	0.00734120260680509\\
61.01	0.00734120073821619\\
62.01	0.00734119882908289\\
63.01	0.00734119687851762\\
64.01	0.00734119488561302\\
65.01	0.00734119284944171\\
66.01	0.0073411907690556\\
67.01	0.00734118864348575\\
68.01	0.00734118647174166\\
69.01	0.00734118425281084\\
70.01	0.00734118198565839\\
71.01	0.00734117966922635\\
72.01	0.00734117730243338\\
73.01	0.00734117488417396\\
74.01	0.0073411724133181\\
75.01	0.00734116988871071\\
76.01	0.00734116730917097\\
77.01	0.00734116467349179\\
78.01	0.00734116198043919\\
79.01	0.00734115922875178\\
80.01	0.00734115641713994\\
81.01	0.00734115354428552\\
82.01	0.00734115060884088\\
83.01	0.00734114760942827\\
84.01	0.00734114454463934\\
85.01	0.00734114141303422\\
86.01	0.00734113821314096\\
87.01	0.00734113494345461\\
88.01	0.00734113160243682\\
89.01	0.00734112818851467\\
90.01	0.00734112470008018\\
91.01	0.00734112113548928\\
92.01	0.00734111749306128\\
93.01	0.00734111377107779\\
94.01	0.00734110996778191\\
95.01	0.00734110608137747\\
96.01	0.00734110211002788\\
97.01	0.00734109805185562\\
98.01	0.00734109390494087\\
99.01	0.00734108966732074\\
100.01	0.00734108533698838\\
101.01	0.00734108091189177\\
102.01	0.00734107638993288\\
103.01	0.00734107176896641\\
104.01	0.00734106704679891\\
105.01	0.00734106222118752\\
106.01	0.00734105728983888\\
107.01	0.00734105225040815\\
108.01	0.00734104710049755\\
109.01	0.00734104183765527\\
110.01	0.0073410364593742\\
111.01	0.00734103096309071\\
112.01	0.00734102534618341\\
113.01	0.00734101960597159\\
114.01	0.00734101373971391\\
115.01	0.00734100774460734\\
116.01	0.00734100161778521\\
117.01	0.00734099535631623\\
118.01	0.00734098895720262\\
119.01	0.00734098241737884\\
120.01	0.00734097573370982\\
121.01	0.00734096890298946\\
122.01	0.00734096192193895\\
123.01	0.0073409547872052\\
124.01	0.00734094749535894\\
125.01	0.00734094004289301\\
126.01	0.00734093242622056\\
127.01	0.00734092464167328\\
128.01	0.0073409166854994\\
129.01	0.00734090855386181\\
130.01	0.00734090024283592\\
131.01	0.00734089174840787\\
132.01	0.00734088306647234\\
133.01	0.00734087419283038\\
134.01	0.00734086512318723\\
135.01	0.00734085585315009\\
136.01	0.00734084637822589\\
137.01	0.00734083669381879\\
138.01	0.00734082679522806\\
139.01	0.00734081667764527\\
140.01	0.00734080633615198\\
141.01	0.00734079576571723\\
142.01	0.00734078496119466\\
143.01	0.00734077391732002\\
144.01	0.00734076262870825\\
145.01	0.00734075108985067\\
146.01	0.00734073929511214\\
147.01	0.00734072723872807\\
148.01	0.00734071491480133\\
149.01	0.0073407023172991\\
150.01	0.00734068944004967\\
151.01	0.00734067627673933\\
152.01	0.00734066282090877\\
153.01	0.00734064906594976\\
154.01	0.00734063500510159\\
155.01	0.00734062063144758\\
156.01	0.00734060593791108\\
157.01	0.00734059091725212\\
158.01	0.00734057556206308\\
159.01	0.00734055986476504\\
160.01	0.0073405438176036\\
161.01	0.00734052741264453\\
162.01	0.00734051064176986\\
163.01	0.00734049349667319\\
164.01	0.00734047596885525\\
165.01	0.00734045804961941\\
166.01	0.00734043973006696\\
167.01	0.00734042100109207\\
168.01	0.00734040185337714\\
169.01	0.00734038227738751\\
170.01	0.00734036226336636\\
171.01	0.00734034180132939\\
172.01	0.00734032088105916\\
173.01	0.00734029949209986\\
174.01	0.00734027762375134\\
175.01	0.00734025526506323\\
176.01	0.00734023240482905\\
177.01	0.00734020903157987\\
178.01	0.00734018513357814\\
179.01	0.00734016069881111\\
180.01	0.00734013571498427\\
181.01	0.00734011016951444\\
182.01	0.00734008404952291\\
183.01	0.00734005734182818\\
184.01	0.00734003003293865\\
185.01	0.00734000210904514\\
186.01	0.00733997355601325\\
187.01	0.00733994435937539\\
188.01	0.00733991450432257\\
189.01	0.00733988397569632\\
190.01	0.00733985275797997\\
191.01	0.00733982083529023\\
192.01	0.00733978819136796\\
193.01	0.00733975480956899\\
194.01	0.00733972067285493\\
195.01	0.00733968576378361\\
196.01	0.00733965006449877\\
197.01	0.00733961355672036\\
198.01	0.00733957622173397\\
199.01	0.0073395380403803\\
200.01	0.00733949899304402\\
201.01	0.00733945905964283\\
202.01	0.00733941821961589\\
203.01	0.00733937645191195\\
204.01	0.00733933373497738\\
205.01	0.00733929004674399\\
206.01	0.00733924536461594\\
207.01	0.00733919966545706\\
208.01	0.00733915292557724\\
209.01	0.00733910512071911\\
210.01	0.00733905622604368\\
211.01	0.00733900621611606\\
212.01	0.00733895506489082\\
213.01	0.0073389027456968\\
214.01	0.00733884923122136\\
215.01	0.00733879449349471\\
216.01	0.00733873850387362\\
217.01	0.00733868123302454\\
218.01	0.00733862265090651\\
219.01	0.00733856272675349\\
220.01	0.00733850142905631\\
221.01	0.00733843872554425\\
222.01	0.00733837458316583\\
223.01	0.00733830896806951\\
224.01	0.00733824184558348\\
225.01	0.00733817318019561\\
226.01	0.0073381029355318\\
227.01	0.00733803107433481\\
228.01	0.00733795755844195\\
229.01	0.00733788234876259\\
230.01	0.00733780540525474\\
231.01	0.00733772668690099\\
232.01	0.00733764615168425\\
233.01	0.00733756375656269\\
234.01	0.00733747945744349\\
235.01	0.00733739320915682\\
236.01	0.00733730496542871\\
237.01	0.00733721467885293\\
238.01	0.00733712230086257\\
239.01	0.00733702778170085\\
240.01	0.00733693107039089\\
241.01	0.00733683211470503\\
242.01	0.00733673086113305\\
243.01	0.00733662725484983\\
244.01	0.00733652123968203\\
245.01	0.00733641275807399\\
246.01	0.00733630175105267\\
247.01	0.00733618815819157\\
248.01	0.00733607191757416\\
249.01	0.00733595296575578\\
250.01	0.00733583123772501\\
251.01	0.00733570666686371\\
252.01	0.00733557918490631\\
253.01	0.00733544872189793\\
254.01	0.00733531520615136\\
255.01	0.00733517856420289\\
256.01	0.00733503872076728\\
257.01	0.00733489559869112\\
258.01	0.00733474911890551\\
259.01	0.00733459920037698\\
260.01	0.00733444576005756\\
261.01	0.00733428871283346\\
262.01	0.00733412797147213\\
263.01	0.00733396344656839\\
264.01	0.00733379504648899\\
265.01	0.00733362267731562\\
266.01	0.00733344624278665\\
267.01	0.0073332656442373\\
268.01	0.00733308078053815\\
269.01	0.00733289154803233\\
270.01	0.0073326978404708\\
271.01	0.00733249954894618\\
272.01	0.00733229656182496\\
273.01	0.00733208876467764\\
274.01	0.00733187604020731\\
275.01	0.00733165826817652\\
276.01	0.00733143532533209\\
277.01	0.007331207085328\\
278.01	0.00733097341864639\\
279.01	0.00733073419251666\\
280.01	0.00733048927083215\\
281.01	0.00733023851406531\\
282.01	0.00732998177918005\\
283.01	0.00732971891954244\\
284.01	0.00732944978482864\\
285.01	0.00732917422093099\\
286.01	0.00732889206986159\\
287.01	0.00732860316965309\\
288.01	0.00732830735425762\\
289.01	0.00732800445344265\\
290.01	0.00732769429268461\\
291.01	0.00732737669305965\\
292.01	0.00732705147113174\\
293.01	0.00732671843883795\\
294.01	0.0073263774033711\\
295.01	0.0073260281670591\\
296.01	0.00732567052724178\\
297.01	0.00732530427614428\\
298.01	0.00732492920074749\\
299.01	0.0073245450826555\\
300.01	0.00732415169795939\\
301.01	0.00732374881709819\\
302.01	0.00732333620471588\\
303.01	0.00732291361951545\\
304.01	0.00732248081410905\\
305.01	0.00732203753486475\\
306.01	0.00732158352174942\\
307.01	0.00732111850816789\\
308.01	0.00732064222079828\\
309.01	0.00732015437942341\\
310.01	0.00731965469675795\\
311.01	0.00731914287827173\\
312.01	0.00731861862200856\\
313.01	0.007318081618401\\
314.01	0.00731753155008062\\
315.01	0.00731696809168352\\
316.01	0.00731639090965177\\
317.01	0.00731579966202974\\
318.01	0.00731519399825555\\
319.01	0.00731457355894801\\
320.01	0.00731393797568816\\
321.01	0.00731328687079602\\
322.01	0.00731261985710168\\
323.01	0.00731193653771122\\
324.01	0.00731123650576725\\
325.01	0.00731051934420369\\
326.01	0.00730978462549459\\
327.01	0.00730903191139745\\
328.01	0.00730826075269011\\
329.01	0.00730747068890161\\
330.01	0.00730666124803663\\
331.01	0.00730583194629368\\
332.01	0.00730498228777586\\
333.01	0.00730411176419571\\
334.01	0.00730321985457197\\
335.01	0.0073023060249201\\
336.01	0.00730136972793427\\
337.01	0.00730041040266253\\
338.01	0.00729942747417348\\
339.01	0.00729842035321495\\
340.01	0.00729738843586434\\
341.01	0.00729633110316978\\
342.01	0.00729524772078298\\
343.01	0.00729413763858231\\
344.01	0.00729300019028617\\
345.01	0.00729183469305672\\
346.01	0.00729064044709288\\
347.01	0.00728941673521282\\
348.01	0.00728816282242548\\
349.01	0.00728687795549005\\
350.01	0.00728556136246381\\
351.01	0.00728421225223741\\
352.01	0.00728282981405698\\
353.01	0.00728141321703262\\
354.01	0.00727996160963335\\
355.01	0.00727847411916644\\
356.01	0.00727694985124217\\
357.01	0.00727538788922212\\
358.01	0.00727378729365114\\
359.01	0.00727214710167144\\
360.01	0.00727046632641884\\
361.01	0.00726874395639984\\
362.01	0.00726697895484907\\
363.01	0.00726517025906573\\
364.01	0.00726331677972892\\
365.01	0.00726141740019008\\
366.01	0.00725947097574201\\
367.01	0.00725747633286347\\
368.01	0.00725543226843849\\
369.01	0.0072533375489489\\
370.01	0.00725119090963921\\
371.01	0.00724899105365316\\
372.01	0.00724673665113992\\
373.01	0.00724442633833004\\
374.01	0.00724205871657875\\
375.01	0.0072396323513764\\
376.01	0.00723714577132455\\
377.01	0.00723459746707628\\
378.01	0.0072319858902405\\
379.01	0.00722930945224778\\
380.01	0.00722656652317783\\
381.01	0.00722375543054717\\
382.01	0.00722087445805516\\
383.01	0.00721792184428871\\
384.01	0.00721489578138365\\
385.01	0.00721179441364176\\
386.01	0.00720861583610336\\
387.01	0.00720535809307382\\
388.01	0.00720201917660352\\
389.01	0.00719859702492024\\
390.01	0.00719508952081353\\
391.01	0.00719149448997062\\
392.01	0.00718780969926303\\
393.01	0.00718403285498466\\
394.01	0.00718016160104026\\
395.01	0.00717619351708595\\
396.01	0.00717212611662225\\
397.01	0.00716795684504191\\
398.01	0.00716368307763514\\
399.01	0.0071593021175567\\
400.01	0.00715481119376046\\
401.01	0.00715020745890973\\
402.01	0.00714548798727503\\
403.01	0.00714064977263219\\
404.01	0.00713568972618098\\
405.01	0.00713060467450701\\
406.01	0.00712539135761644\\
407.01	0.00712004642707825\\
408.01	0.00711456644431503\\
409.01	0.0071089478790868\\
410.01	0.00710318710821348\\
411.01	0.00709728041457593\\
412.01	0.00709122398641762\\
413.01	0.00708501391693231\\
414.01	0.0070786462040412\\
415.01	0.00707211675003738\\
416.01	0.00706542136092544\\
417.01	0.00705855574717348\\
418.01	0.00705151552320142\\
419.01	0.00704429620649164\\
420.01	0.00703689321959463\\
421.01	0.00702930188928812\\
422.01	0.00702151739972556\\
423.01	0.00701353482583495\\
424.01	0.00700534917614831\\
425.01	0.0069969553614563\\
426.01	0.00698834819560703\\
427.01	0.00697952239786452\\
428.01	0.00697047259597068\\
429.01	0.00696119333002084\\
430.01	0.00695167905728106\\
431.01	0.00694192415809635\\
432.01	0.00693192294306247\\
433.01	0.00692166966166322\\
434.01	0.00691115851260798\\
435.01	0.00690038365614526\\
436.01	0.00688933922867446\\
437.01	0.0068780193600339\\
438.01	0.00686641819391086\\
439.01	0.00685452991189808\\
440.01	0.00684234876181789\\
441.01	0.00682986909104756\\
442.01	0.00681708538572061\\
443.01	0.00680399231684287\\
444.01	0.00679058479456501\\
445.01	0.0067768580321002\\
446.01	0.00676280762107438\\
447.01	0.00674842962046204\\
448.01	0.00673372066171481\\
449.01	0.00671867807323846\\
450.01	0.00670330002799456\\
451.01	0.00668758571715203\\
452.01	0.00667153548351915\\
453.01	0.00665515057052235\\
454.01	0.00663843559358089\\
455.01	0.00662139865022594\\
456.01	0.00660405128096307\\
457.01	0.00658640926690019\\
458.01	0.00656849362613499\\
459.01	0.00655033190451136\\
460.01	0.00653195996748447\\
461.01	0.00651338992196601\\
462.01	0.00649451268265961\\
463.01	0.00647533204807501\\
464.01	0.00645587440892889\\
465.01	0.00643617364606535\\
466.01	0.00641627280342872\\
467.01	0.00639622615410085\\
468.01	0.0063761017581098\\
469.01	0.00635598463514874\\
470.01	0.00633598069084061\\
471.01	0.00631622165165349\\
472.01	0.00629687134529433\\
473.01	0.00627813341218427\\
474.01	0.0062602617677022\\
475.01	0.00624307687454638\\
476.01	0.00622549082442006\\
477.01	0.00620743337667525\\
478.01	0.00618889187776189\\
479.01	0.00616985371567746\\
480.01	0.00615030642372044\\
481.01	0.00613023781053281\\
482.01	0.00610963612259572\\
483.01	0.00608849024679421\\
484.01	0.00606678996198266\\
485.01	0.00604452623973263\\
486.01	0.00602169128206165\\
487.01	0.00599826555399007\\
488.01	0.00597418206558548\\
489.01	0.00594941305773239\\
490.01	0.00592394139900088\\
491.01	0.00589775033755464\\
492.01	0.00587082362165473\\
493.01	0.00584314561672445\\
494.01	0.00581470141027783\\
495.01	0.00578547689168251\\
496.01	0.00575545871032362\\
497.01	0.00572463408788436\\
498.01	0.00569299073822983\\
499.01	0.00566051586224345\\
500.01	0.00562719492563089\\
501.01	0.00559300990457205\\
502.01	0.0055579343979256\\
503.01	0.00552192924888774\\
504.01	0.00548494980648591\\
505.01	0.00544693740411984\\
506.01	0.0054078213643012\\
507.01	0.00536754783778978\\
508.01	0.005326061155258\\
509.01	0.00528330235990979\\
510.01	0.00523920912376\\
511.01	0.00519371569541818\\
512.01	0.00514675288793071\\
513.01	0.00509824811625097\\
514.01	0.00504812549505705\\
515.01	0.00499630601086775\\
516.01	0.00494270780002012\\
517.01	0.00488724652526266\\
518.01	0.00482984735845951\\
519.01	0.00477055288280926\\
520.01	0.00470952603271803\\
521.01	0.0046472138863196\\
522.01	0.00458470131205031\\
523.01	0.00452219569378065\\
524.01	0.00445980107877253\\
525.01	0.00439765771346164\\
526.01	0.00433592197309556\\
527.01	0.00427476476752623\\
528.01	0.00421437048237607\\
529.01	0.00415493489511919\\
530.01	0.00409666162626346\\
531.01	0.00403975646328572\\
532.01	0.00398441874170262\\
533.01	0.00393082869873044\\
534.01	0.00387912966662672\\
535.01	0.00382940493481397\\
536.01	0.00378164249179029\\
537.01	0.00373524000500358\\
538.01	0.00368912833644244\\
539.01	0.00364330189677589\\
540.01	0.00359778390829597\\
541.01	0.00355259025757311\\
542.01	0.00350772801655309\\
543.01	0.00346319391049507\\
544.01	0.00341896868234572\\
545.01	0.00337501297581907\\
546.01	0.00333126511313421\\
547.01	0.00328763977111955\\
548.01	0.00324402773980095\\
549.01	0.00320029915127878\\
550.01	0.00315631167800961\\
551.01	0.00311195285554549\\
552.01	0.0030671882490935\\
553.01	0.00302199127603092\\
554.01	0.00297633053814613\\
555.01	0.00293017008017734\\
556.01	0.00288346993778933\\
557.01	0.00283618702641363\\
558.01	0.00278827640582232\\
559.01	0.0027396929194969\\
560.01	0.00269039314200024\\
561.01	0.00264033745611435\\
562.01	0.00258949198921666\\
563.01	0.00253782990454083\\
564.01	0.0024853288845688\\
565.01	0.00243196691153569\\
566.01	0.00237772185534557\\
567.01	0.00232257175221729\\
568.01	0.00226649513730902\\
569.01	0.00220947140606029\\
570.01	0.00215148115625655\\
571.01	0.00209250647178937\\
572.01	0.00203253110496798\\
573.01	0.00197154051305623\\
574.01	0.00190952172343432\\
575.01	0.00184646307785447\\
576.01	0.00178235412443806\\
577.01	0.0017171857660873\\
578.01	0.00165095045995029\\
579.01	0.00158364242165109\\
580.01	0.00151525782603943\\
581.01	0.00144579499572334\\
582.01	0.00137525457316008\\
583.01	0.00130363967477813\\
584.01	0.00123095602868033\\
585.01	0.00115721210051214\\
586.01	0.00108241921223445\\
587.01	0.00100659164275123\\
588.01	0.000929746671129862\\
589.01	0.000851904520258075\\
590.01	0.000773088158107163\\
591.01	0.000693322906705438\\
592.01	0.000612635797368357\\
593.01	0.000531054596121653\\
594.01	0.000448606405509172\\
595.01	0.00036531572727916\\
596.01	0.000281201843626103\\
597.01	0.000196275341074087\\
598.01	0.000110533558374905\\
599.01	3.18230760274988e-05\\
599.02	3.12771225888381e-05\\
599.03	3.07343852024352e-05\\
599.04	3.01948954524109e-05\\
599.05	2.96586853198184e-05\\
599.06	2.91257871969947e-05\\
599.07	2.85962339028577e-05\\
599.08	2.8070058607518e-05\\
599.09	2.75472948133501e-05\\
599.1	2.70279763573446e-05\\
599.11	2.65121374149781e-05\\
599.12	2.59998125032215e-05\\
599.13	2.54910364795111e-05\\
599.14	2.49858445451751e-05\\
599.15	2.44842722489057e-05\\
599.16	2.39863554902346e-05\\
599.17	2.34921305228321e-05\\
599.18	2.30016339574767e-05\\
599.19	2.25149027655146e-05\\
599.2	2.20319742823254e-05\\
599.21	2.15528862108072e-05\\
599.22	2.10776766248668e-05\\
599.23	2.06063839729136e-05\\
599.24	2.01390470813568e-05\\
599.25	1.96757051580971e-05\\
599.26	1.92163977960017e-05\\
599.27	1.87611649765159e-05\\
599.28	1.83100470734845e-05\\
599.29	1.78630848569894e-05\\
599.3	1.74203194972418e-05\\
599.31	1.69817925685135e-05\\
599.32	1.65475460531112e-05\\
599.33	1.61176223453848e-05\\
599.34	1.56920642557928e-05\\
599.35	1.52709150149869e-05\\
599.36	1.48542182779494e-05\\
599.37	1.44420181281777e-05\\
599.38	1.40343590818905e-05\\
599.39	1.36312869210512e-05\\
599.4	1.32328508630885e-05\\
599.41	1.2839100613395e-05\\
599.42	1.24500863701339e-05\\
599.43	1.20658588290742e-05\\
599.44	1.16864691884749e-05\\
599.45	1.13119691540261e-05\\
599.46	1.09424109438309e-05\\
599.47	1.05778472934363e-05\\
599.48	1.02183314609188e-05\\
599.49	9.86391723201314e-06\\
599.5	9.51465892530708e-06\\
599.51	9.17061139746149e-06\\
599.52	8.83183004850643e-06\\
599.53	8.49837082718408e-06\\
599.54	8.17029023633675e-06\\
599.55	7.84764533835573e-06\\
599.56	7.53049376068203e-06\\
599.57	7.21889370135405e-06\\
599.58	6.91290393461941e-06\\
599.59	6.61258381660576e-06\\
599.6	6.31799329102282e-06\\
599.61	6.02919289495116e-06\\
599.62	5.74624376466562e-06\\
599.63	5.46920764152997e-06\\
599.64	5.19814687793482e-06\\
599.65	4.93312444331713e-06\\
599.66	4.67420393020224e-06\\
599.67	4.42144956034653e-06\\
599.68	4.17492619090958e-06\\
599.69	3.93469932070262e-06\\
599.7	3.70083509648733e-06\\
599.71	3.47340031935442e-06\\
599.72	3.25246245113688e-06\\
599.73	3.03808962091867e-06\\
599.74	2.8303506315764e-06\\
599.75	2.62931496640938e-06\\
599.76	2.43505279581835e-06\\
599.77	2.24763498406222e-06\\
599.78	2.06713309606858e-06\\
599.79	1.89361940432578e-06\\
599.8	1.72716689582353e-06\\
599.81	1.56784927908028e-06\\
599.82	1.41574099122785e-06\\
599.83	1.27091720517752e-06\\
599.84	1.13345383683135e-06\\
599.85	1.00342755240089e-06\\
599.86	8.80915775762492e-07\\
599.87	7.65996695906157e-07\\
599.88	6.58749274453849e-07\\
599.89	5.59253253248904e-07\\
599.9	4.67589162013102e-07\\
599.91	3.8383832610088e-07\\
599.92	3.0808287429171e-07\\
599.93	2.4040574671258e-07\\
599.94	1.80890702777825e-07\\
599.95	1.2962232925906e-07\\
599.96	8.66860484019516e-08\\
599.97	5.21681261331924e-08\\
599.98	2.61556803542173e-08\\
599.99	8.73668929909921e-09\\
600	0\\
};
\addplot [color=mycolor19,solid,forget plot]
  table[row sep=crcr]{%
0.01	0.00662800561349471\\
1.01	0.00662800516698634\\
2.01	0.00662800471096339\\
3.01	0.00662800424522229\\
4.01	0.00662800376955526\\
5.01	0.00662800328375014\\
6.01	0.00662800278758982\\
7.01	0.00662800228085294\\
8.01	0.00662800176331315\\
9.01	0.00662800123473928\\
10.01	0.00662800069489526\\
11.01	0.00662800014353982\\
12.01	0.00662799958042655\\
13.01	0.00662799900530394\\
14.01	0.00662799841791487\\
15.01	0.00662799781799666\\
16.01	0.00662799720528108\\
17.01	0.00662799657949403\\
18.01	0.00662799594035567\\
19.01	0.00662799528758004\\
20.01	0.006627994620875\\
21.01	0.0066279939399423\\
22.01	0.00662799324447699\\
23.01	0.00662799253416775\\
24.01	0.00662799180869655\\
25.01	0.00662799106773842\\
26.01	0.00662799031096156\\
27.01	0.00662798953802681\\
28.01	0.00662798874858787\\
29.01	0.00662798794229095\\
30.01	0.00662798711877457\\
31.01	0.0066279862776695\\
32.01	0.00662798541859861\\
33.01	0.00662798454117659\\
34.01	0.00662798364500984\\
35.01	0.00662798272969625\\
36.01	0.00662798179482509\\
37.01	0.00662798083997678\\
38.01	0.00662797986472263\\
39.01	0.00662797886862471\\
40.01	0.00662797785123572\\
41.01	0.00662797681209865\\
42.01	0.00662797575074674\\
43.01	0.00662797466670292\\
44.01	0.0066279735594801\\
45.01	0.00662797242858051\\
46.01	0.00662797127349571\\
47.01	0.0066279700937063\\
48.01	0.0066279688886815\\
49.01	0.00662796765787931\\
50.01	0.00662796640074588\\
51.01	0.0066279651167154\\
52.01	0.00662796380520997\\
53.01	0.00662796246563897\\
54.01	0.00662796109739918\\
55.01	0.00662795969987448\\
56.01	0.00662795827243515\\
57.01	0.00662795681443804\\
58.01	0.00662795532522601\\
59.01	0.0066279538041277\\
60.01	0.00662795225045732\\
61.01	0.00662795066351422\\
62.01	0.00662794904258259\\
63.01	0.00662794738693112\\
64.01	0.00662794569581267\\
65.01	0.00662794396846396\\
66.01	0.00662794220410519\\
67.01	0.00662794040193965\\
68.01	0.00662793856115338\\
69.01	0.00662793668091491\\
70.01	0.0066279347603746\\
71.01	0.00662793279866449\\
72.01	0.00662793079489781\\
73.01	0.0066279287481686\\
74.01	0.00662792665755128\\
75.01	0.00662792452210008\\
76.01	0.00662792234084891\\
77.01	0.00662792011281058\\
78.01	0.00662791783697666\\
79.01	0.00662791551231669\\
80.01	0.00662791313777803\\
81.01	0.00662791071228504\\
82.01	0.00662790823473874\\
83.01	0.00662790570401648\\
84.01	0.00662790311897112\\
85.01	0.00662790047843062\\
86.01	0.0066278977811975\\
87.01	0.00662789502604836\\
88.01	0.00662789221173306\\
89.01	0.00662788933697446\\
90.01	0.00662788640046762\\
91.01	0.00662788340087927\\
92.01	0.00662788033684711\\
93.01	0.00662787720697921\\
94.01	0.00662787400985347\\
95.01	0.00662787074401677\\
96.01	0.0066278674079845\\
97.01	0.0066278640002396\\
98.01	0.00662786051923218\\
99.01	0.00662785696337853\\
100.01	0.00662785333106043\\
101.01	0.0066278496206245\\
102.01	0.00662784583038131\\
103.01	0.0066278419586047\\
104.01	0.00662783800353083\\
105.01	0.00662783396335757\\
106.01	0.00662782983624339\\
107.01	0.00662782562030659\\
108.01	0.0066278213136246\\
109.01	0.00662781691423285\\
110.01	0.00662781242012386\\
111.01	0.00662780782924653\\
112.01	0.00662780313950477\\
113.01	0.00662779834875697\\
114.01	0.00662779345481474\\
115.01	0.00662778845544178\\
116.01	0.0066277833483531\\
117.01	0.00662777813121369\\
118.01	0.00662777280163762\\
119.01	0.00662776735718675\\
120.01	0.00662776179536972\\
121.01	0.00662775611364071\\
122.01	0.00662775030939814\\
123.01	0.00662774437998358\\
124.01	0.00662773832268047\\
125.01	0.00662773213471277\\
126.01	0.00662772581324374\\
127.01	0.00662771935537439\\
128.01	0.00662771275814245\\
129.01	0.00662770601852059\\
130.01	0.00662769913341532\\
131.01	0.00662769209966521\\
132.01	0.00662768491403954\\
133.01	0.00662767757323686\\
134.01	0.0066276700738832\\
135.01	0.00662766241253068\\
136.01	0.00662765458565565\\
137.01	0.00662764658965726\\
138.01	0.00662763842085535\\
139.01	0.0066276300754892\\
140.01	0.00662762154971541\\
141.01	0.00662761283960593\\
142.01	0.00662760394114659\\
143.01	0.00662759485023485\\
144.01	0.00662758556267798\\
145.01	0.00662757607419107\\
146.01	0.00662756638039477\\
147.01	0.0066275564768134\\
148.01	0.00662754635887267\\
149.01	0.00662753602189763\\
150.01	0.00662752546111022\\
151.01	0.00662751467162706\\
152.01	0.00662750364845716\\
153.01	0.00662749238649931\\
154.01	0.00662748088053993\\
155.01	0.00662746912525013\\
156.01	0.00662745711518362\\
157.01	0.00662744484477359\\
158.01	0.00662743230833038\\
159.01	0.00662741950003849\\
160.01	0.0066274064139538\\
161.01	0.00662739304400081\\
162.01	0.00662737938396948\\
163.01	0.00662736542751227\\
164.01	0.00662735116814124\\
165.01	0.00662733659922458\\
166.01	0.0066273217139835\\
167.01	0.00662730650548892\\
168.01	0.00662729096665803\\
169.01	0.00662727509025096\\
170.01	0.00662725886886697\\
171.01	0.00662724229494104\\
172.01	0.00662722536074013\\
173.01	0.00662720805835907\\
174.01	0.00662719037971722\\
175.01	0.0066271723165538\\
176.01	0.00662715386042436\\
177.01	0.00662713500269641\\
178.01	0.00662711573454498\\
179.01	0.00662709604694843\\
180.01	0.00662707593068403\\
181.01	0.00662705537632315\\
182.01	0.00662703437422661\\
183.01	0.00662701291454014\\
184.01	0.00662699098718905\\
185.01	0.00662696858187351\\
186.01	0.00662694568806321\\
187.01	0.00662692229499199\\
188.01	0.00662689839165287\\
189.01	0.006626873966792\\
190.01	0.0066268490089033\\
191.01	0.00662682350622247\\
192.01	0.0066267974467212\\
193.01	0.00662677081810105\\
194.01	0.00662674360778704\\
195.01	0.00662671580292151\\
196.01	0.00662668739035743\\
197.01	0.00662665835665177\\
198.01	0.00662662868805853\\
199.01	0.00662659837052181\\
200.01	0.00662656738966869\\
201.01	0.0066265357308016\\
202.01	0.00662650337889101\\
203.01	0.00662647031856767\\
204.01	0.00662643653411447\\
205.01	0.00662640200945852\\
206.01	0.00662636672816279\\
207.01	0.00662633067341761\\
208.01	0.00662629382803171\\
209.01	0.0066262561744237\\
210.01	0.00662621769461239\\
211.01	0.00662617837020773\\
212.01	0.00662613818240104\\
213.01	0.00662609711195519\\
214.01	0.00662605513919448\\
215.01	0.00662601224399421\\
216.01	0.00662596840576998\\
217.01	0.00662592360346703\\
218.01	0.0066258778155487\\
219.01	0.00662583101998533\\
220.01	0.00662578319424236\\
221.01	0.00662573431526805\\
222.01	0.00662568435948147\\
223.01	0.00662563330275989\\
224.01	0.00662558112042532\\
225.01	0.0066255277872315\\
226.01	0.00662547327735037\\
227.01	0.00662541756435768\\
228.01	0.00662536062121904\\
229.01	0.0066253024202747\\
230.01	0.00662524293322482\\
231.01	0.00662518213111368\\
232.01	0.00662511998431401\\
233.01	0.00662505646251033\\
234.01	0.00662499153468248\\
235.01	0.00662492516908842\\
236.01	0.00662485733324617\\
237.01	0.00662478799391618\\
238.01	0.00662471711708254\\
239.01	0.00662464466793375\\
240.01	0.00662457061084337\\
241.01	0.00662449490934975\\
242.01	0.00662441752613552\\
243.01	0.00662433842300622\\
244.01	0.0066242575608687\\
245.01	0.00662417489970877\\
246.01	0.00662409039856812\\
247.01	0.00662400401552091\\
248.01	0.00662391570764941\\
249.01	0.00662382543101951\\
250.01	0.00662373314065458\\
251.01	0.00662363879051004\\
252.01	0.00662354233344579\\
253.01	0.00662344372119895\\
254.01	0.00662334290435512\\
255.01	0.00662323983231958\\
256.01	0.00662313445328695\\
257.01	0.00662302671421067\\
258.01	0.00662291656077115\\
259.01	0.00662280393734327\\
260.01	0.00662268878696307\\
261.01	0.00662257105129317\\
262.01	0.00662245067058778\\
263.01	0.00662232758365616\\
264.01	0.00662220172782539\\
265.01	0.00662207303890209\\
266.01	0.00662194145113291\\
267.01	0.00662180689716418\\
268.01	0.00662166930800013\\
269.01	0.00662152861296002\\
270.01	0.00662138473963445\\
271.01	0.00662123761383974\\
272.01	0.00662108715957155\\
273.01	0.00662093329895719\\
274.01	0.00662077595220619\\
275.01	0.00662061503755995\\
276.01	0.00662045047123949\\
277.01	0.00662028216739223\\
278.01	0.00662011003803687\\
279.01	0.0066199339930068\\
280.01	0.00661975393989208\\
281.01	0.00661956978397959\\
282.01	0.00661938142819155\\
283.01	0.00661918877302219\\
284.01	0.00661899171647292\\
285.01	0.00661879015398522\\
286.01	0.00661858397837213\\
287.01	0.00661837307974748\\
288.01	0.00661815734545293\\
289.01	0.00661793665998358\\
290.01	0.00661771090491081\\
291.01	0.0066174799588034\\
292.01	0.00661724369714626\\
293.01	0.00661700199225676\\
294.01	0.00661675471319886\\
295.01	0.00661650172569491\\
296.01	0.0066162428920347\\
297.01	0.00661597807098249\\
298.01	0.00661570711768067\\
299.01	0.00661542988355159\\
300.01	0.00661514621619602\\
301.01	0.00661485595928926\\
302.01	0.00661455895247416\\
303.01	0.00661425503125132\\
304.01	0.00661394402686626\\
305.01	0.00661362576619343\\
306.01	0.0066133000716175\\
307.01	0.00661296676091094\\
308.01	0.00661262564710884\\
309.01	0.00661227653837986\\
310.01	0.00661191923789445\\
311.01	0.00661155354368929\\
312.01	0.00661117924852798\\
313.01	0.0066107961397587\\
314.01	0.00661040399916789\\
315.01	0.00661000260283024\\
316.01	0.00660959172095541\\
317.01	0.0066091711177301\\
318.01	0.0066087405511571\\
319.01	0.00660829977289013\\
320.01	0.00660784852806491\\
321.01	0.00660738655512567\\
322.01	0.00660691358564859\\
323.01	0.00660642934416056\\
324.01	0.00660593354795405\\
325.01	0.00660542590689797\\
326.01	0.00660490612324442\\
327.01	0.00660437389143127\\
328.01	0.00660382889788074\\
329.01	0.00660327082079387\\
330.01	0.00660269932994098\\
331.01	0.00660211408644781\\
332.01	0.00660151474257809\\
333.01	0.00660090094151163\\
334.01	0.0066002723171192\\
335.01	0.00659962849373248\\
336.01	0.00659896908591152\\
337.01	0.00659829369820749\\
338.01	0.00659760192492218\\
339.01	0.00659689334986413\\
340.01	0.00659616754610136\\
341.01	0.00659542407571067\\
342.01	0.00659466248952397\\
343.01	0.00659388232687187\\
344.01	0.00659308311532434\\
345.01	0.00659226437042884\\
346.01	0.00659142559544619\\
347.01	0.00659056628108438\\
348.01	0.00658968590523029\\
349.01	0.00658878393268019\\
350.01	0.0065878598148684\\
351.01	0.00658691298959533\\
352.01	0.00658594288075464\\
353.01	0.00658494889805978\\
354.01	0.00658393043677062\\
355.01	0.00658288687742027\\
356.01	0.00658181758554252\\
357.01	0.00658072191140011\\
358.01	0.00657959918971456\\
359.01	0.00657844873939757\\
360.01	0.00657726986328474\\
361.01	0.00657606184787175\\
362.01	0.00657482396305372\\
363.01	0.00657355546186785\\
364.01	0.00657225558024\\
365.01	0.006570923536736\\
366.01	0.00656955853231751\\
367.01	0.00656815975010318\\
368.01	0.00656672635513631\\
369.01	0.00656525749415812\\
370.01	0.00656375229538862\\
371.01	0.00656220986831459\\
372.01	0.00656062930348597\\
373.01	0.00655900967232032\\
374.01	0.00655735002691722\\
375.01	0.00655564939988173\\
376.01	0.00655390680415868\\
377.01	0.00655212123287765\\
378.01	0.00655029165920917\\
379.01	0.00654841703623332\\
380.01	0.00654649629681985\\
381.01	0.00654452835352096\\
382.01	0.00654251209847646\\
383.01	0.00654044640333092\\
384.01	0.00653833011916143\\
385.01	0.00653616207641647\\
386.01	0.0065339410848613\\
387.01	0.00653166593352889\\
388.01	0.0065293353906704\\
389.01	0.0065269482037002\\
390.01	0.00652450309912636\\
391.01	0.00652199878245545\\
392.01	0.0065194339380574\\
393.01	0.00651680722897081\\
394.01	0.00651411729662454\\
395.01	0.00651136276044442\\
396.01	0.00650854221730586\\
397.01	0.00650565424078453\\
398.01	0.00650269738014472\\
399.01	0.00649967015899487\\
400.01	0.00649657107352375\\
401.01	0.00649339859021859\\
402.01	0.00649015114294966\\
403.01	0.00648682712929694\\
404.01	0.00648342490598466\\
405.01	0.00647994278329268\\
406.01	0.00647637901833192\\
407.01	0.00647273180711472\\
408.01	0.00646899927543674\\
409.01	0.00646517946873548\\
410.01	0.00646127034133285\\
411.01	0.00645726974585314\\
412.01	0.00645317542419395\\
413.01	0.00644898500231479\\
414.01	0.00644469599438852\\
415.01	0.00644030582556887\\
416.01	0.00643581184422117\\
417.01	0.00643121132031199\\
418.01	0.00642650144298454\\
419.01	0.00642167931811403\\
420.01	0.00641674196582409\\
421.01	0.00641168631758996\\
422.01	0.0064065092131186\\
423.01	0.00640120740019603\\
424.01	0.00639577753187652\\
425.01	0.00639021616335015\\
426.01	0.00638451974956823\\
427.01	0.00637868464293872\\
428.01	0.00637270709110753\\
429.01	0.00636658323485393\\
430.01	0.00636030910613026\\
431.01	0.00635388062628287\\
432.01	0.00634729360449606\\
433.01	0.00634054373650708\\
434.01	0.00633362660364876\\
435.01	0.00632653767228254\\
436.01	0.00631927229369635\\
437.01	0.00631182570455096\\
438.01	0.00630419302797106\\
439.01	0.00629636927538985\\
440.01	0.00628834934927172\\
441.01	0.00628012804685388\\
442.01	0.00627170006506416\\
443.01	0.00626306000679248\\
444.01	0.00625420238871374\\
445.01	0.00624512165087826\\
446.01	0.00623581216830759\\
447.01	0.00622626826485076\\
448.01	0.00621648422956873\\
449.01	0.00620645433592308\\
450.01	0.00619617286403988\\
451.01	0.00618563412629655\\
452.01	0.00617483249659878\\
453.01	0.00616376244593114\\
454.01	0.00615241856387239\\
455.01	0.00614079555763997\\
456.01	0.00612888826101543\\
457.01	0.00611669163481851\\
458.01	0.0061042007499225\\
459.01	0.00609141074431236\\
460.01	0.00607831673480827\\
461.01	0.00606491372901676\\
462.01	0.00605119794332509\\
463.01	0.00603716701194948\\
464.01	0.00602281903691963\\
465.01	0.00600815244782415\\
466.01	0.00599316579941203\\
467.01	0.00597785743865201\\
468.01	0.00596222498751151\\
469.01	0.00594626455940978\\
470.01	0.00592996931183786\\
471.01	0.00591332644919716\\
472.01	0.0058963183014063\\
473.01	0.00587891840890463\\
474.01	0.00586108519619017\\
475.01	0.00584275945284479\\
476.01	0.0058239000907944\\
477.01	0.00580447904511014\\
478.01	0.00578446622209283\\
479.01	0.00576382917878065\\
480.01	0.00574253294140469\\
481.01	0.00572053981903716\\
482.01	0.00569780921596359\\
483.01	0.00567429744785275\\
484.01	0.00564995756883459\\
485.01	0.00562473921933519\\
486.01	0.00559858850938103\\
487.01	0.0055714480080374\\
488.01	0.00554325732314974\\
489.01	0.00551395347978441\\
490.01	0.00548347094837184\\
491.01	0.00545174230923784\\
492.01	0.00541869929752596\\
493.01	0.00538427432238069\\
494.01	0.00534840261886492\\
495.01	0.0053110252402656\\
496.01	0.00527209316299654\\
497.01	0.00523157286128748\\
498.01	0.00518945381409247\\
499.01	0.00514575855936362\\
500.01	0.00510055609813132\\
501.01	0.00505398507554129\\
502.01	0.00500669829198533\\
503.01	0.00495908387073525\\
504.01	0.00491119851801213\\
505.01	0.00486310693772908\\
506.01	0.00481488274983092\\
507.01	0.00476660888573337\\
508.01	0.0047183779636502\\
509.01	0.00467029270123617\\
510.01	0.00462246611531557\\
511.01	0.00457502139463206\\
512.01	0.00452809129427338\\
513.01	0.00448181684757209\\
514.01	0.00443634512148415\\
515.01	0.00439182564908723\\
516.01	0.0043484050477132\\
517.01	0.00430621914513456\\
518.01	0.00426538166528717\\
519.01	0.00422596707374778\\
520.01	0.00418798502878103\\
521.01	0.00415117282828407\\
522.01	0.004114607957536\\
523.01	0.00407821214291625\\
524.01	0.00404202278944255\\
525.01	0.00400607380377639\\
526.01	0.00397039386513776\\
527.01	0.00393500465022193\\
528.01	0.00389991884824442\\
529.01	0.00386513800886344\\
530.01	0.0038306503075186\\
531.01	0.00379642837598923\\
532.01	0.00376242743600688\\
533.01	0.00372858410164934\\
534.01	0.00369481639498225\\
535.01	0.00366102575010299\\
536.01	0.00362710213479891\\
537.01	0.00359294119869851\\
538.01	0.00355850046695407\\
539.01	0.00352376222467012\\
540.01	0.00348870536509431\\
541.01	0.00345330520045406\\
542.01	0.00341753352031443\\
543.01	0.00338135877420527\\
544.01	0.00334474647734787\\
545.01	0.00330765993147755\\
546.01	0.00327006123083538\\
547.01	0.00323191254095152\\
548.01	0.00319317761547406\\
549.01	0.0031538234145313\\
550.01	0.00311382155102045\\
551.01	0.0030731486693184\\
552.01	0.0030317832424425\\
553.01	0.00298970328206819\\
554.01	0.00294688640917473\\
555.01	0.0029033100905392\\
556.01	0.00285895189237939\\
557.01	0.00281378973570623\\
558.01	0.00276780213238578\\
559.01	0.00272096837574328\\
560.01	0.00267326865627233\\
561.01	0.00262468407428047\\
562.01	0.00257519653052143\\
563.01	0.00252478849624054\\
564.01	0.00247344276509564\\
565.01	0.00242114241434334\\
566.01	0.00236787089249708\\
567.01	0.0023136121155352\\
568.01	0.00225835056070582\\
569.01	0.00220207135471073\\
570.01	0.00214476035493563\\
571.01	0.00208640422450415\\
572.01	0.00202699050448087\\
573.01	0.00196650768972543\\
574.01	0.00190494531817138\\
575.01	0.00184229408482995\\
576.01	0.00177854598418229\\
577.01	0.00171369446888605\\
578.01	0.0016477346162388\\
579.01	0.0015806633005132\\
580.01	0.00151247936952008\\
581.01	0.00144318382328261\\
582.01	0.00137277999184947\\
583.01	0.00130127370788135\\
584.01	0.0012286734675554\\
585.01	0.00115499057031396\\
586.01	0.00108023922386147\\
587.01	0.00100443659586242\\
588.01	0.000927602789133309\\
589.01	0.000849760712020215\\
590.01	0.000770935808457237\\
591.01	0.00069115560281989\\
592.01	0.000610449002857873\\
593.01	0.000528845289120537\\
594.01	0.000446372700577078\\
595.01	0.000363056502572266\\
596.01	0.000278916393581779\\
597.01	0.000193963069867737\\
598.01	0.000108193720153206\\
599.01	3.18230447902152e-05\\
599.02	3.12770961984968e-05\\
599.03	3.07343624924919e-05\\
599.04	3.0194875613938e-05\\
599.05	2.96586678183493e-05\\
599.06	2.91257716778671e-05\\
599.07	2.85962200841493e-05\\
599.08	2.80700462514365e-05\\
599.09	2.75472837197206e-05\\
599.1	2.70279663579388e-05\\
599.11	2.65121283671912e-05\\
599.12	2.59998042840028e-05\\
599.13	2.54910289836183e-05\\
599.14	2.49858376833341e-05\\
599.15	2.44842659458625e-05\\
599.16	2.39863496827274e-05\\
599.17	2.34921251576915e-05\\
599.18	2.30016289902261e-05\\
599.19	2.2514898159005e-05\\
599.2	2.20319700054462e-05\\
599.21	2.15528822372704e-05\\
599.22	2.10776729321089e-05\\
599.23	2.06063805411449e-05\\
599.24	2.01390438927945e-05\\
599.25	1.96757021964072e-05\\
599.26	1.92163950460209e-05\\
599.27	1.87611624241565e-05\\
599.28	1.83100447056338e-05\\
599.29	1.78630826614314e-05\\
599.3	1.74203174626037e-05\\
599.31	1.69817906841944e-05\\
599.32	1.65475443092387e-05\\
599.33	1.61176207327651e-05\\
599.34	1.56920627658545e-05\\
599.35	1.52709136397415e-05\\
599.36	1.48542170099464e-05\\
599.37	1.44420169604503e-05\\
599.38	1.40343580079232e-05\\
599.39	1.36312859347863e-05\\
599.4	1.32328499588292e-05\\
599.41	1.28390997857844e-05\\
599.42	1.24500856141119e-05\\
599.43	1.20658581398547e-05\\
599.44	1.16864685615215e-05\\
599.45	1.13119685850316e-05\\
599.46	1.09424104286961e-05\\
599.47	1.05778468282615e-05\\
599.48	1.02183310419918e-05\\
599.49	9.86391685580193e-06\\
599.5	9.5146585884498e-06\\
599.51	9.17061109676973e-06\\
599.52	8.83182978096524e-06\\
599.53	8.49837058993982e-06\\
599.54	8.17029002670236e-06\\
599.55	7.84764515380544e-06\\
599.56	7.53049359884274e-06\\
599.57	7.21889356001225e-06\\
599.58	6.91290381171078e-06\\
599.59	6.61258371020476e-06\\
599.6	6.31799319935483e-06\\
599.61	6.02919281637339e-06\\
599.62	5.74624369766713e-06\\
599.63	5.46920758472819e-06\\
599.64	5.1981468300686e-06\\
599.65	4.93312440323461e-06\\
599.66	4.67420389686606e-06\\
599.67	4.42144953282167e-06\\
599.68	4.1749261683599e-06\\
599.69	3.93469930238047e-06\\
599.7	3.70083508173524e-06\\
599.71	3.47340030758779e-06\\
599.72	3.25246244185264e-06\\
599.73	3.03808961367273e-06\\
599.74	2.83035062599232e-06\\
599.75	2.62931496216104e-06\\
599.76	2.43505279263513e-06\\
599.77	2.24763498171514e-06\\
599.78	2.06713309437029e-06\\
599.79	1.89361940312015e-06\\
599.8	1.72716689498566e-06\\
599.81	1.56784927851129e-06\\
599.82	1.41574099085662e-06\\
599.83	1.27091720493987e-06\\
599.84	1.13345383668736e-06\\
599.85	1.00342755231589e-06\\
599.86	8.8091577571392e-07\\
599.87	7.65996695880136e-07\\
599.88	6.58749274441706e-07\\
599.89	5.59253253243699e-07\\
599.9	4.67589162013102e-07\\
599.91	3.8383832609741e-07\\
599.92	3.0808287429171e-07\\
599.93	2.40405746710845e-07\\
599.94	1.8089070277609e-07\\
599.95	1.2962232925906e-07\\
599.96	8.66860484002169e-08\\
599.97	5.21681261331924e-08\\
599.98	2.61556803542173e-08\\
599.99	8.73668930083393e-09\\
600	0\\
};
\addplot [color=red!50!mycolor17,solid,forget plot]
  table[row sep=crcr]{%
0.01	0.00643330023985397\\
1.01	0.00643329966361264\\
2.01	0.00643329907515522\\
3.01	0.00643329847422214\\
4.01	0.00643329786054826\\
5.01	0.00643329723386302\\
6.01	0.00643329659388985\\
7.01	0.00643329594034644\\
8.01	0.00643329527294438\\
9.01	0.00643329459138911\\
10.01	0.00643329389537988\\
11.01	0.00643329318460946\\
12.01	0.0064332924587642\\
13.01	0.00643329171752365\\
14.01	0.00643329096056051\\
15.01	0.0064332901875405\\
16.01	0.00643328939812239\\
17.01	0.00643328859195762\\
18.01	0.00643328776869002\\
19.01	0.00643328692795602\\
20.01	0.00643328606938434\\
21.01	0.00643328519259551\\
22.01	0.00643328429720226\\
23.01	0.00643328338280896\\
24.01	0.00643328244901161\\
25.01	0.0064332814953974\\
26.01	0.0064332805215449\\
27.01	0.0064332795270238\\
28.01	0.00643327851139429\\
29.01	0.00643327747420746\\
30.01	0.00643327641500483\\
31.01	0.006433275333318\\
32.01	0.00643327422866862\\
33.01	0.00643327310056819\\
34.01	0.00643327194851781\\
35.01	0.00643327077200789\\
36.01	0.0064332695705179\\
37.01	0.0064332683435164\\
38.01	0.00643326709046047\\
39.01	0.00643326581079568\\
40.01	0.00643326450395578\\
41.01	0.00643326316936238\\
42.01	0.00643326180642484\\
43.01	0.00643326041453989\\
44.01	0.00643325899309134\\
45.01	0.00643325754144988\\
46.01	0.00643325605897279\\
47.01	0.00643325454500375\\
48.01	0.00643325299887223\\
49.01	0.00643325141989364\\
50.01	0.00643324980736839\\
51.01	0.00643324816058252\\
52.01	0.00643324647880629\\
53.01	0.00643324476129471\\
54.01	0.00643324300728684\\
55.01	0.00643324121600541\\
56.01	0.00643323938665682\\
57.01	0.00643323751843024\\
58.01	0.00643323561049773\\
59.01	0.00643323366201344\\
60.01	0.00643323167211386\\
61.01	0.00643322963991663\\
62.01	0.00643322756452081\\
63.01	0.00643322544500617\\
64.01	0.00643322328043285\\
65.01	0.00643322106984085\\
66.01	0.00643321881224981\\
67.01	0.0064332165066584\\
68.01	0.00643321415204387\\
69.01	0.00643321174736173\\
70.01	0.00643320929154495\\
71.01	0.00643320678350395\\
72.01	0.0064332042221259\\
73.01	0.00643320160627412\\
74.01	0.00643319893478762\\
75.01	0.00643319620648064\\
76.01	0.00643319342014218\\
77.01	0.00643319057453537\\
78.01	0.00643318766839686\\
79.01	0.00643318470043639\\
80.01	0.00643318166933613\\
81.01	0.00643317857375005\\
82.01	0.00643317541230344\\
83.01	0.00643317218359219\\
84.01	0.00643316888618216\\
85.01	0.00643316551860874\\
86.01	0.00643316207937583\\
87.01	0.00643315856695557\\
88.01	0.0064331549797873\\
89.01	0.00643315131627714\\
90.01	0.00643314757479708\\
91.01	0.0064331437536844\\
92.01	0.00643313985124065\\
93.01	0.00643313586573155\\
94.01	0.00643313179538536\\
95.01	0.00643312763839263\\
96.01	0.00643312339290536\\
97.01	0.0064331190570357\\
98.01	0.0064331146288558\\
99.01	0.00643311010639661\\
100.01	0.00643310548764689\\
101.01	0.00643310077055248\\
102.01	0.00643309595301526\\
103.01	0.00643309103289239\\
104.01	0.00643308600799515\\
105.01	0.00643308087608805\\
106.01	0.00643307563488783\\
107.01	0.00643307028206251\\
108.01	0.0064330648152302\\
109.01	0.00643305923195813\\
110.01	0.00643305352976162\\
111.01	0.00643304770610254\\
112.01	0.00643304175838908\\
113.01	0.00643303568397349\\
114.01	0.00643302948015176\\
115.01	0.00643302314416188\\
116.01	0.00643301667318305\\
117.01	0.0064330100643339\\
118.01	0.0064330033146717\\
119.01	0.00643299642119072\\
120.01	0.00643298938082084\\
121.01	0.00643298219042644\\
122.01	0.00643297484680501\\
123.01	0.00643296734668524\\
124.01	0.00643295968672614\\
125.01	0.00643295186351517\\
126.01	0.00643294387356687\\
127.01	0.00643293571332134\\
128.01	0.00643292737914249\\
129.01	0.00643291886731646\\
130.01	0.00643291017404996\\
131.01	0.00643290129546863\\
132.01	0.00643289222761531\\
133.01	0.00643288296644819\\
134.01	0.0064328735078389\\
135.01	0.00643286384757091\\
136.01	0.00643285398133756\\
137.01	0.00643284390474001\\
138.01	0.00643283361328532\\
139.01	0.00643282310238452\\
140.01	0.00643281236735039\\
141.01	0.00643280140339574\\
142.01	0.00643279020563064\\
143.01	0.00643277876906069\\
144.01	0.0064327670885847\\
145.01	0.00643275515899224\\
146.01	0.0064327429749616\\
147.01	0.0064327305310571\\
148.01	0.0064327178217268\\
149.01	0.00643270484129986\\
150.01	0.0064326915839843\\
151.01	0.00643267804386418\\
152.01	0.00643266421489673\\
153.01	0.00643265009091029\\
154.01	0.00643263566560078\\
155.01	0.00643262093252944\\
156.01	0.0064326058851197\\
157.01	0.00643259051665394\\
158.01	0.0064325748202712\\
159.01	0.00643255878896333\\
160.01	0.00643254241557224\\
161.01	0.00643252569278664\\
162.01	0.00643250861313882\\
163.01	0.00643249116900114\\
164.01	0.00643247335258249\\
165.01	0.00643245515592519\\
166.01	0.00643243657090099\\
167.01	0.00643241758920761\\
168.01	0.00643239820236508\\
169.01	0.00643237840171157\\
170.01	0.00643235817839993\\
171.01	0.00643233752339339\\
172.01	0.00643231642746167\\
173.01	0.00643229488117671\\
174.01	0.0064322728749082\\
175.01	0.00643225039881996\\
176.01	0.00643222744286454\\
177.01	0.00643220399677939\\
178.01	0.00643218005008191\\
179.01	0.00643215559206491\\
180.01	0.00643213061179134\\
181.01	0.00643210509808991\\
182.01	0.00643207903954972\\
183.01	0.00643205242451492\\
184.01	0.00643202524108001\\
185.01	0.00643199747708377\\
186.01	0.00643196912010412\\
187.01	0.0064319401574526\\
188.01	0.00643191057616808\\
189.01	0.00643188036301151\\
190.01	0.00643184950445939\\
191.01	0.00643181798669802\\
192.01	0.00643178579561694\\
193.01	0.00643175291680245\\
194.01	0.00643171933553146\\
195.01	0.00643168503676419\\
196.01	0.00643165000513781\\
197.01	0.00643161422495921\\
198.01	0.00643157768019776\\
199.01	0.00643154035447815\\
200.01	0.00643150223107265\\
201.01	0.00643146329289373\\
202.01	0.00643142352248596\\
203.01	0.00643138290201833\\
204.01	0.00643134141327574\\
205.01	0.00643129903765055\\
206.01	0.00643125575613437\\
207.01	0.00643121154930904\\
208.01	0.00643116639733792\\
209.01	0.00643112027995636\\
210.01	0.00643107317646256\\
211.01	0.006431025065708\\
212.01	0.00643097592608755\\
213.01	0.00643092573552946\\
214.01	0.0064308744714851\\
215.01	0.00643082211091856\\
216.01	0.00643076863029572\\
217.01	0.00643071400557331\\
218.01	0.00643065821218798\\
219.01	0.00643060122504442\\
220.01	0.00643054301850361\\
221.01	0.00643048356637099\\
222.01	0.0064304228418841\\
223.01	0.00643036081769961\\
224.01	0.0064302974658809\\
225.01	0.00643023275788465\\
226.01	0.00643016666454723\\
227.01	0.00643009915607106\\
228.01	0.00643003020200987\\
229.01	0.00642995977125506\\
230.01	0.00642988783202002\\
231.01	0.00642981435182537\\
232.01	0.00642973929748326\\
233.01	0.00642966263508117\\
234.01	0.00642958432996606\\
235.01	0.00642950434672698\\
236.01	0.00642942264917838\\
237.01	0.00642933920034242\\
238.01	0.00642925396243062\\
239.01	0.00642916689682585\\
240.01	0.00642907796406301\\
241.01	0.00642898712380994\\
242.01	0.00642889433484729\\
243.01	0.0064287995550484\\
244.01	0.00642870274135785\\
245.01	0.00642860384977059\\
246.01	0.0064285028353096\\
247.01	0.00642839965200366\\
248.01	0.00642829425286382\\
249.01	0.00642818658985994\\
250.01	0.00642807661389634\\
251.01	0.00642796427478643\\
252.01	0.00642784952122762\\
253.01	0.00642773230077468\\
254.01	0.00642761255981288\\
255.01	0.0064274902435299\\
256.01	0.00642736529588781\\
257.01	0.00642723765959359\\
258.01	0.00642710727606899\\
259.01	0.00642697408541999\\
260.01	0.00642683802640458\\
261.01	0.0064266990364007\\
262.01	0.0064265570513725\\
263.01	0.00642641200583604\\
264.01	0.0064262638328237\\
265.01	0.0064261124638481\\
266.01	0.00642595782886436\\
267.01	0.00642579985623197\\
268.01	0.00642563847267495\\
269.01	0.00642547360324142\\
270.01	0.00642530517126144\\
271.01	0.00642513309830385\\
272.01	0.0064249573041322\\
273.01	0.00642477770665875\\
274.01	0.00642459422189782\\
275.01	0.0064244067639166\\
276.01	0.00642421524478602\\
277.01	0.0064240195745288\\
278.01	0.00642381966106671\\
279.01	0.00642361541016589\\
280.01	0.00642340672538038\\
281.01	0.00642319350799416\\
282.01	0.00642297565696161\\
283.01	0.0064227530688452\\
284.01	0.00642252563775196\\
285.01	0.00642229325526805\\
286.01	0.00642205581039044\\
287.01	0.00642181318945708\\
288.01	0.0064215652760747\\
289.01	0.00642131195104404\\
290.01	0.00642105309228254\\
291.01	0.00642078857474497\\
292.01	0.00642051827034081\\
293.01	0.00642024204784927\\
294.01	0.00641995977283155\\
295.01	0.00641967130753948\\
296.01	0.00641937651082184\\
297.01	0.00641907523802669\\
298.01	0.00641876734090096\\
299.01	0.00641845266748652\\
300.01	0.00641813106201194\\
301.01	0.00641780236478143\\
302.01	0.00641746641205873\\
303.01	0.00641712303594808\\
304.01	0.00641677206427006\\
305.01	0.00641641332043338\\
306.01	0.00641604662330142\\
307.01	0.00641567178705515\\
308.01	0.00641528862104976\\
309.01	0.00641489692966672\\
310.01	0.00641449651216027\\
311.01	0.00641408716249786\\
312.01	0.00641366866919556\\
313.01	0.00641324081514603\\
314.01	0.00641280337744097\\
315.01	0.00641235612718674\\
316.01	0.00641189882931247\\
317.01	0.00641143124237128\\
318.01	0.00641095311833422\\
319.01	0.00641046420237525\\
320.01	0.00640996423264893\\
321.01	0.0064094529400591\\
322.01	0.00640893004801868\\
323.01	0.00640839527219962\\
324.01	0.00640784832027409\\
325.01	0.00640728889164443\\
326.01	0.00640671667716335\\
327.01	0.00640613135884255\\
328.01	0.00640553260954994\\
329.01	0.00640492009269518\\
330.01	0.00640429346190234\\
331.01	0.00640365236066948\\
332.01	0.00640299642201571\\
333.01	0.00640232526811258\\
334.01	0.00640163850990212\\
335.01	0.00640093574669889\\
336.01	0.00640021656577682\\
337.01	0.0063994805419384\\
338.01	0.00639872723706807\\
339.01	0.00639795619966709\\
340.01	0.00639716696436969\\
341.01	0.00639635905144049\\
342.01	0.00639553196625191\\
343.01	0.00639468519874068\\
344.01	0.00639381822284263\\
345.01	0.00639293049590642\\
346.01	0.00639202145808291\\
347.01	0.00639109053169153\\
348.01	0.00639013712056225\\
349.01	0.00638916060935153\\
350.01	0.00638816036283289\\
351.01	0.0063871357251601\\
352.01	0.00638608601910334\\
353.01	0.00638501054525607\\
354.01	0.00638390858121396\\
355.01	0.0063827793807233\\
356.01	0.00638162217279975\\
357.01	0.00638043616081613\\
358.01	0.00637922052155926\\
359.01	0.00637797440425547\\
360.01	0.00637669692956446\\
361.01	0.00637538718854211\\
362.01	0.00637404424157169\\
363.01	0.00637266711726575\\
364.01	0.00637125481133701\\
365.01	0.00636980628544193\\
366.01	0.00636832046599739\\
367.01	0.00636679624297341\\
368.01	0.00636523246866418\\
369.01	0.00636362795644195\\
370.01	0.00636198147949785\\
371.01	0.00636029176957551\\
372.01	0.00635855751570492\\
373.01	0.00635677736294391\\
374.01	0.00635494991113921\\
375.01	0.00635307371371823\\
376.01	0.00635114727652669\\
377.01	0.00634916905673055\\
378.01	0.00634713746180239\\
379.01	0.0063450508486184\\
380.01	0.0063429075226946\\
381.01	0.00634070573759887\\
382.01	0.00633844369457892\\
383.01	0.00633611954245547\\
384.01	0.00633373137783847\\
385.01	0.00633127724573051\\
386.01	0.00632875514059788\\
387.01	0.00632616300799689\\
388.01	0.00632349874686038\\
389.01	0.00632076021256291\\
390.01	0.00631794522090061\\
391.01	0.0063150515531395\\
392.01	0.00631207696230492\\
393.01	0.00630901918090426\\
394.01	0.00630587593029035\\
395.01	0.00630264493189209\\
396.01	0.00629932392054452\\
397.01	0.00629591066015217\\
398.01	0.00629240296190524\\
399.01	0.0062887987052299\\
400.01	0.00628509586158552\\
401.01	0.0062812925210984\\
402.01	0.0062773869218394\\
403.01	0.00627337748125873\\
404.01	0.0062692628288728\\
405.01	0.00626504183867491\\
406.01	0.00626071365886577\\
407.01	0.00625627773524493\\
408.01	0.00625173382286064\\
409.01	0.0062470819780779\\
410.01	0.00624232251986272\\
411.01	0.00623745594444802\\
412.01	0.00623248277120422\\
413.01	0.00622740328765818\\
414.01	0.00622221673713735\\
415.01	0.00621692070537143\\
416.01	0.00621151246139671\\
417.01	0.00620598916715711\\
418.01	0.0062003478704892\\
419.01	0.00619458549751251\\
420.01	0.0061886988443706\\
421.01	0.00618268456826306\\
422.01	0.00617653917770893\\
423.01	0.00617025902193658\\
424.01	0.0061638402793036\\
425.01	0.00615727894469201\\
426.01	0.00615057081576944\\
427.01	0.00614371147799952\\
428.01	0.00613669628828092\\
429.01	0.00612952035707672\\
430.01	0.00612217852888811\\
431.01	0.00611466536090582\\
432.01	0.00610697509966166\\
433.01	0.00609910165548131\\
434.01	0.00609103857452474\\
435.01	0.00608277900817753\\
436.01	0.00607431567953827\\
437.01	0.00606564084672496\\
438.01	0.00605674626270089\\
439.01	0.00604762313130064\\
440.01	0.00603826205911502\\
441.01	0.00602865300287525\\
442.01	0.00601878521196404\\
443.01	0.00600864716567108\\
444.01	0.00599822650481147\\
445.01	0.00598750995734001\\
446.01	0.00597648325762419\\
447.01	0.00596513105909743\\
448.01	0.00595343684010651\\
449.01	0.00594138280290273\\
450.01	0.005928949765932\\
451.01	0.00591611704986275\\
452.01	0.0059028623581921\\
453.01	0.00588916165380416\\
454.01	0.00587498903373414\\
455.01	0.00586031660584085\\
456.01	0.00584511437208053\\
457.01	0.00582935012488772\\
458.01	0.00581298936587319\\
459.01	0.00579599525945634\\
460.01	0.00577832863859256\\
461.01	0.00575994808581527\\
462.01	0.00574081011055646\\
463.01	0.00572086944900231\\
464.01	0.00570007956420959\\
465.01	0.00567839341078409\\
466.01	0.00565576454955953\\
467.01	0.00563214873078398\\
468.01	0.00560750610165938\\
469.01	0.0055818042429304\\
470.01	0.00555502230411261\\
471.01	0.0055271565999938\\
472.01	0.00549822810063586\\
473.01	0.00546829475838497\\
474.01	0.00543770048875961\\
475.01	0.0054066488083951\\
476.01	0.00537514743510029\\
477.01	0.00534320637975664\\
478.01	0.00531083814295213\\
479.01	0.00527805803414741\\
480.01	0.00524488452308555\\
481.01	0.00521133962399908\\
482.01	0.00517744931233549\\
483.01	0.00514324397220679\\
484.01	0.00510875887058511\\
485.01	0.00507403465119368\\
486.01	0.00503911783676843\\
487.01	0.0050040613225054\\
488.01	0.00496892483227736\\
489.01	0.00493377529014779\\
490.01	0.00489868707071135\\
491.01	0.00486374205840157\\
492.01	0.00482902941053039\\
493.01	0.00479464488819632\\
494.01	0.00476068957112154\\
495.01	0.00472726770782663\\
496.01	0.00469448336651602\\
497.01	0.00466243544789561\\
498.01	0.00463121047938355\\
499.01	0.00460087238312644\\
500.01	0.00457144816890996\\
501.01	0.00454290277306364\\
502.01	0.00451468377798736\\
503.01	0.00448648666934954\\
504.01	0.00445834215488247\\
505.01	0.00443028109400982\\
506.01	0.00440233396611083\\
507.01	0.00437453021896771\\
508.01	0.00434689749021075\\
509.01	0.00431946068556158\\
510.01	0.00429224090559129\\
511.01	0.00426525422123244\\
512.01	0.00423851031014622\\
513.01	0.00421201098427128\\
514.01	0.0041857486660573\\
515.01	0.00415970491050224\\
516.01	0.00413384912698003\\
517.01	0.00410813773590309\\
518.01	0.00408251411026784\\
519.01	0.00405690981719585\\
520.01	0.00403124793559125\\
521.01	0.00400545145824772\\
522.01	0.00397948220161627\\
523.01	0.00395333152360722\\
524.01	0.00392698916657118\\
525.01	0.00390044210047198\\
526.01	0.00387367439700243\\
527.01	0.00384666718283707\\
528.01	0.00381939869609618\\
529.01	0.00379184447350555\\
530.01	0.00376397769607122\\
531.01	0.003735769718089\\
532.01	0.00370719079578369\\
533.01	0.00367821101462431\\
534.01	0.0036488013837574\\
535.01	0.00361893501550884\\
536.01	0.00358858822822191\\
537.01	0.00355774121106488\\
538.01	0.00352637650554267\\
539.01	0.00349447638476056\\
540.01	0.00346202253866048\\
541.01	0.00342899620714643\\
542.01	0.00339537833345592\\
543.01	0.00336114973292802\\
544.01	0.00332629126957986\\
545.01	0.0032907840270283\\
546.01	0.00325460945664299\\
547.01	0.00321774948331401\\
548.01	0.00318018654712834\\
549.01	0.00314190356106398\\
550.01	0.00310288377422265\\
551.01	0.00306311055643913\\
552.01	0.00302256723038261\\
553.01	0.00298123706917909\\
554.01	0.00293910334223729\\
555.01	0.00289614935795801\\
556.01	0.00285235850019123\\
557.01	0.00280771425681943\\
558.01	0.00276220023958686\\
559.01	0.00271580019548417\\
560.01	0.00266849801167274\\
561.01	0.00262027771800452\\
562.01	0.00257112349337822\\
563.01	0.00252101968387742\\
564.01	0.00246995083963625\\
565.01	0.00241790176743256\\
566.01	0.00236485758999453\\
567.01	0.00231080380985966\\
568.01	0.00225572637833465\\
569.01	0.00219961177044417\\
570.01	0.0021424470669859\\
571.01	0.00208422004494936\\
572.01	0.00202491927756344\\
573.01	0.00196453424505074\\
574.01	0.00190305545671095\\
575.01	0.00184047458421142\\
576.01	0.00177678460515771\\
577.01	0.00171197995585397\\
578.01	0.00164605669237123\\
579.01	0.0015790126587281\\
580.01	0.00151084766032477\\
581.01	0.00144156363987375\\
582.01	0.0013711648518907\\
583.01	0.00129965803028412\\
584.01	0.00122705254163708\\
585.01	0.0011533605143295\\
586.01	0.00107859693060129\\
587.01	0.00100277966487854\\
588.01	0.000925929446941498\\
589.01	0.000848069722467409\\
590.01	0.00076922637580672\\
591.01	0.000689427270172824\\
592.01	0.000608701548246961\\
593.01	0.000527078620894876\\
594.01	0.000444586752481062\\
595.01	0.000361251127183708\\
596.01	0.000277091250558454\\
597.01	0.000192117502885737\\
598.01	0.00010632661372072\\
599.01	3.1823044268001e-05\\
599.02	3.12770957282913e-05\\
599.03	3.07343620667162e-05\\
599.04	3.01948752267685e-05\\
599.05	2.96586674650972e-05\\
599.06	2.91257713545909e-05\\
599.07	2.85962197874873e-05\\
599.08	2.80700459785194e-05\\
599.09	2.75472834680903e-05\\
599.1	2.70279661254754e-05\\
599.11	2.65121281520647e-05\\
599.12	2.59998040846362e-05\\
599.13	2.54910287986413e-05\\
599.14	2.49858375115618e-05\\
599.15	2.44842657862541e-05\\
599.16	2.39863495343721e-05\\
599.17	2.34921250197827e-05\\
599.18	2.30016288620422e-05\\
599.19	2.25148980398954e-05\\
599.2	2.2031969894816e-05\\
599.21	2.15528821345731e-05\\
599.22	2.10776728368396e-05\\
599.23	2.06063804528353e-05\\
599.24	2.01390438110006e-05\\
599.25	1.96757021207125e-05\\
599.26	1.92163949760387e-05\\
599.27	1.8761162359519e-05\\
599.28	1.8310044645994e-05\\
599.29	1.78630826064737e-05\\
599.3	1.7420317412014e-05\\
599.31	1.69817906376935e-05\\
599.32	1.65475442665524e-05\\
599.33	1.61176206936401e-05\\
599.34	1.5692062730048e-05\\
599.35	1.52709136070281e-05\\
599.36	1.48542169801109e-05\\
599.37	1.44420169332915e-05\\
599.38	1.40343579832485e-05\\
599.39	1.36312859124171e-05\\
599.4	1.32328499385954e-05\\
599.41	1.2839099767523e-05\\
599.42	1.24500855976737e-05\\
599.43	1.20658581250939e-05\\
599.44	1.16864685483029e-05\\
599.45	1.13119685732268e-05\\
599.46	1.09424104181854e-05\\
599.47	1.05778468189321e-05\\
599.48	1.02183310337345e-05\\
599.49	9.86391684851956e-06\\
599.5	9.51465858204867e-06\\
599.51	9.17061109116483e-06\\
599.52	8.83182977607332e-06\\
599.53	8.49837058568975e-06\\
599.54	8.17029002302475e-06\\
599.55	7.84764515063437e-06\\
599.56	7.53049359612443e-06\\
599.57	7.21889355768945e-06\\
599.58	6.91290380973666e-06\\
599.59	6.61258370854116e-06\\
599.6	6.31799319795665e-06\\
599.61	6.02919281520418e-06\\
599.62	5.74624369669915e-06\\
599.63	5.46920758392848e-06\\
599.64	5.19814682941461e-06\\
599.65	4.93312440270205e-06\\
599.66	4.67420389643758e-06\\
599.67	4.42144953247993e-06\\
599.68	4.17492616808929e-06\\
599.69	3.93469930216883e-06\\
599.7	3.70083508157044e-06\\
599.71	3.47340030746289e-06\\
599.72	3.25246244175723e-06\\
599.73	3.03808961360334e-06\\
599.74	2.83035062593855e-06\\
599.75	2.62931496212288e-06\\
599.76	2.43505279260738e-06\\
599.77	2.24763498169432e-06\\
599.78	2.06713309435641e-06\\
599.79	1.89361940310974e-06\\
599.8	1.72716689497872e-06\\
599.81	1.56784927850956e-06\\
599.82	1.41574099085315e-06\\
599.83	1.27091720493813e-06\\
599.84	1.13345383668563e-06\\
599.85	1.00342755231589e-06\\
599.86	8.8091577571392e-07\\
599.87	7.65996695880136e-07\\
599.88	6.58749274441706e-07\\
599.89	5.59253253243699e-07\\
599.9	4.67589162011367e-07\\
599.91	3.83838326099145e-07\\
599.92	3.08082874293444e-07\\
599.93	2.40405746710845e-07\\
599.94	1.80890702777825e-07\\
599.95	1.2962232925906e-07\\
599.96	8.66860484019516e-08\\
599.97	5.21681261349272e-08\\
599.98	2.61556803542173e-08\\
599.99	8.73668930083393e-09\\
600	0\\
};
\addplot [color=red!40!mycolor19,solid,forget plot]
  table[row sep=crcr]{%
0.01	0.0062874344136244\\
1.01	0.00628743349800273\\
2.01	0.00628743256299474\\
3.01	0.00628743160818932\\
4.01	0.00628743063316646\\
5.01	0.00628742963749692\\
6.01	0.00628742862074276\\
7.01	0.0062874275824567\\
8.01	0.00628742652218161\\
9.01	0.00628742543945105\\
10.01	0.00628742433378832\\
11.01	0.0062874232047069\\
12.01	0.00628742205170972\\
13.01	0.00628742087428894\\
14.01	0.00628741967192625\\
15.01	0.00628741844409226\\
16.01	0.00628741719024629\\
17.01	0.00628741590983598\\
18.01	0.00628741460229756\\
19.01	0.00628741326705499\\
20.01	0.00628741190351998\\
21.01	0.00628741051109212\\
22.01	0.00628740908915761\\
23.01	0.00628740763709008\\
24.01	0.00628740615424945\\
25.01	0.00628740463998247\\
26.01	0.0062874030936216\\
27.01	0.00628740151448521\\
28.01	0.00628739990187734\\
29.01	0.00628739825508692\\
30.01	0.0062873965733878\\
31.01	0.00628739485603868\\
32.01	0.00628739310228218\\
33.01	0.00628739131134478\\
34.01	0.00628738948243671\\
35.01	0.00628738761475131\\
36.01	0.0062873857074647\\
37.01	0.00628738375973543\\
38.01	0.00628738177070424\\
39.01	0.00628737973949339\\
40.01	0.00628737766520673\\
41.01	0.00628737554692879\\
42.01	0.00628737338372471\\
43.01	0.0062873711746397\\
44.01	0.00628736891869874\\
45.01	0.00628736661490585\\
46.01	0.00628736426224395\\
47.01	0.00628736185967426\\
48.01	0.00628735940613593\\
49.01	0.0062873569005455\\
50.01	0.00628735434179671\\
51.01	0.006287351728759\\
52.01	0.00628734906027848\\
53.01	0.00628734633517628\\
54.01	0.00628734355224853\\
55.01	0.0062873407102658\\
56.01	0.00628733780797224\\
57.01	0.00628733484408533\\
58.01	0.00628733181729537\\
59.01	0.00628732872626476\\
60.01	0.00628732556962706\\
61.01	0.0062873223459872\\
62.01	0.00628731905392003\\
63.01	0.00628731569197016\\
64.01	0.00628731225865114\\
65.01	0.00628730875244501\\
66.01	0.00628730517180127\\
67.01	0.0062873015151364\\
68.01	0.00628729778083321\\
69.01	0.00628729396723982\\
70.01	0.00628729007266971\\
71.01	0.00628728609539976\\
72.01	0.00628728203367046\\
73.01	0.00628727788568475\\
74.01	0.00628727364960721\\
75.01	0.00628726932356354\\
76.01	0.00628726490563911\\
77.01	0.00628726039387883\\
78.01	0.00628725578628547\\
79.01	0.00628725108081969\\
80.01	0.00628724627539847\\
81.01	0.00628724136789457\\
82.01	0.00628723635613503\\
83.01	0.00628723123790091\\
84.01	0.00628722601092583\\
85.01	0.00628722067289502\\
86.01	0.00628721522144452\\
87.01	0.00628720965415978\\
88.01	0.0062872039685749\\
89.01	0.00628719816217146\\
90.01	0.00628719223237714\\
91.01	0.00628718617656494\\
92.01	0.00628717999205191\\
93.01	0.00628717367609755\\
94.01	0.00628716722590323\\
95.01	0.00628716063861062\\
96.01	0.00628715391130024\\
97.01	0.0062871470409906\\
98.01	0.00628714002463641\\
99.01	0.00628713285912764\\
100.01	0.0062871255412879\\
101.01	0.00628711806787282\\
102.01	0.00628711043556921\\
103.01	0.00628710264099285\\
104.01	0.00628709468068761\\
105.01	0.00628708655112356\\
106.01	0.0062870782486955\\
107.01	0.00628706976972144\\
108.01	0.00628706111044066\\
109.01	0.00628705226701221\\
110.01	0.00628704323551351\\
111.01	0.00628703401193834\\
112.01	0.00628702459219453\\
113.01	0.00628701497210318\\
114.01	0.00628700514739588\\
115.01	0.00628699511371355\\
116.01	0.00628698486660368\\
117.01	0.00628697440151922\\
118.01	0.00628696371381592\\
119.01	0.00628695279875048\\
120.01	0.00628694165147855\\
121.01	0.00628693026705234\\
122.01	0.00628691864041858\\
123.01	0.00628690676641635\\
124.01	0.00628689463977442\\
125.01	0.00628688225510939\\
126.01	0.00628686960692276\\
127.01	0.00628685668959899\\
128.01	0.00628684349740249\\
129.01	0.00628683002447583\\
130.01	0.00628681626483625\\
131.01	0.00628680221237351\\
132.01	0.00628678786084703\\
133.01	0.00628677320388323\\
134.01	0.00628675823497253\\
135.01	0.00628674294746665\\
136.01	0.00628672733457538\\
137.01	0.00628671138936365\\
138.01	0.00628669510474871\\
139.01	0.00628667847349666\\
140.01	0.00628666148821946\\
141.01	0.00628664414137117\\
142.01	0.00628662642524527\\
143.01	0.00628660833197088\\
144.01	0.0062865898535092\\
145.01	0.00628657098165012\\
146.01	0.00628655170800836\\
147.01	0.00628653202401986\\
148.01	0.00628651192093804\\
149.01	0.00628649138982988\\
150.01	0.00628647042157179\\
151.01	0.00628644900684564\\
152.01	0.0062864271361349\\
153.01	0.00628640479971971\\
154.01	0.00628638198767357\\
155.01	0.00628635868985792\\
156.01	0.00628633489591809\\
157.01	0.00628631059527905\\
158.01	0.00628628577713991\\
159.01	0.00628626043047003\\
160.01	0.00628623454400331\\
161.01	0.00628620810623378\\
162.01	0.00628618110540997\\
163.01	0.00628615352953008\\
164.01	0.00628612536633663\\
165.01	0.00628609660331089\\
166.01	0.00628606722766725\\
167.01	0.00628603722634769\\
168.01	0.00628600658601601\\
169.01	0.00628597529305184\\
170.01	0.00628594333354455\\
171.01	0.00628591069328737\\
172.01	0.0062858773577703\\
173.01	0.00628584331217474\\
174.01	0.00628580854136631\\
175.01	0.00628577302988811\\
176.01	0.00628573676195391\\
177.01	0.00628569972144138\\
178.01	0.00628566189188479\\
179.01	0.00628562325646768\\
180.01	0.00628558379801538\\
181.01	0.00628554349898735\\
182.01	0.00628550234146987\\
183.01	0.00628546030716749\\
184.01	0.006285417377395\\
185.01	0.00628537353306957\\
186.01	0.00628532875470208\\
187.01	0.00628528302238823\\
188.01	0.00628523631580016\\
189.01	0.00628518861417705\\
190.01	0.00628513989631629\\
191.01	0.00628509014056373\\
192.01	0.00628503932480434\\
193.01	0.00628498742645229\\
194.01	0.00628493442244111\\
195.01	0.0062848802892134\\
196.01	0.00628482500271029\\
197.01	0.00628476853836071\\
198.01	0.00628471087107083\\
199.01	0.00628465197521284\\
200.01	0.00628459182461342\\
201.01	0.0062845303925423\\
202.01	0.00628446765170066\\
203.01	0.00628440357420835\\
204.01	0.00628433813159232\\
205.01	0.00628427129477378\\
206.01	0.00628420303405488\\
207.01	0.00628413331910632\\
208.01	0.00628406211895315\\
209.01	0.0062839894019618\\
210.01	0.00628391513582533\\
211.01	0.00628383928754987\\
212.01	0.00628376182343942\\
213.01	0.00628368270908119\\
214.01	0.0062836019093302\\
215.01	0.00628351938829409\\
216.01	0.00628343510931667\\
217.01	0.00628334903496229\\
218.01	0.0062832611269986\\
219.01	0.00628317134637997\\
220.01	0.00628307965323034\\
221.01	0.0062829860068251\\
222.01	0.00628289036557343\\
223.01	0.00628279268699974\\
224.01	0.00628269292772447\\
225.01	0.0062825910434453\\
226.01	0.00628248698891729\\
227.01	0.0062823807179326\\
228.01	0.00628227218330053\\
229.01	0.00628216133682583\\
230.01	0.00628204812928764\\
231.01	0.00628193251041765\\
232.01	0.00628181442887756\\
233.01	0.00628169383223631\\
234.01	0.00628157066694637\\
235.01	0.00628144487832066\\
236.01	0.0062813164105073\\
237.01	0.00628118520646496\\
238.01	0.00628105120793778\\
239.01	0.00628091435542867\\
240.01	0.00628077458817327\\
241.01	0.00628063184411234\\
242.01	0.00628048605986412\\
243.01	0.00628033717069563\\
244.01	0.00628018511049424\\
245.01	0.0062800298117374\\
246.01	0.00627987120546236\\
247.01	0.00627970922123501\\
248.01	0.00627954378711856\\
249.01	0.00627937482964052\\
250.01	0.00627920227375994\\
251.01	0.00627902604283324\\
252.01	0.00627884605857934\\
253.01	0.00627866224104423\\
254.01	0.00627847450856465\\
255.01	0.00627828277773102\\
256.01	0.00627808696334902\\
257.01	0.00627788697840107\\
258.01	0.00627768273400627\\
259.01	0.0062774741393794\\
260.01	0.00627726110178995\\
261.01	0.00627704352651842\\
262.01	0.00627682131681328\\
263.01	0.00627659437384595\\
264.01	0.00627636259666503\\
265.01	0.00627612588214916\\
266.01	0.00627588412495935\\
267.01	0.00627563721748919\\
268.01	0.00627538504981479\\
269.01	0.00627512750964293\\
270.01	0.006274864482258\\
271.01	0.00627459585046796\\
272.01	0.00627432149454862\\
273.01	0.00627404129218614\\
274.01	0.00627375511841894\\
275.01	0.0062734628455778\\
276.01	0.00627316434322387\\
277.01	0.00627285947808595\\
278.01	0.00627254811399527\\
279.01	0.00627223011181933\\
280.01	0.0062719053293935\\
281.01	0.00627157362145101\\
282.01	0.00627123483955036\\
283.01	0.00627088883200205\\
284.01	0.00627053544379253\\
285.01	0.00627017451650581\\
286.01	0.00626980588824328\\
287.01	0.0062694293935412\\
288.01	0.00626904486328561\\
289.01	0.00626865212462511\\
290.01	0.00626825100088108\\
291.01	0.00626784131145485\\
292.01	0.00626742287173261\\
293.01	0.0062669954929873\\
294.01	0.0062665589822768\\
295.01	0.00626611314234064\\
296.01	0.00626565777149172\\
297.01	0.00626519266350579\\
298.01	0.00626471760750708\\
299.01	0.00626423238784969\\
300.01	0.00626373678399616\\
301.01	0.00626323057039097\\
302.01	0.0062627135163308\\
303.01	0.00626218538582935\\
304.01	0.00626164593747834\\
305.01	0.00626109492430317\\
306.01	0.00626053209361404\\
307.01	0.00625995718685075\\
308.01	0.00625936993942274\\
309.01	0.00625877008054312\\
310.01	0.00625815733305593\\
311.01	0.00625753141325752\\
312.01	0.00625689203071042\\
313.01	0.00625623888805078\\
314.01	0.00625557168078717\\
315.01	0.00625489009709179\\
316.01	0.00625419381758304\\
317.01	0.00625348251509964\\
318.01	0.00625275585446366\\
319.01	0.00625201349223544\\
320.01	0.00625125507645604\\
321.01	0.00625048024637951\\
322.01	0.00624968863219228\\
323.01	0.00624887985472068\\
324.01	0.00624805352512373\\
325.01	0.0062472092445725\\
326.01	0.00624634660391358\\
327.01	0.00624546518331611\\
328.01	0.00624456455190193\\
329.01	0.00624364426735621\\
330.01	0.0062427038755192\\
331.01	0.00624174290995664\\
332.01	0.00624076089150639\\
333.01	0.00623975732780229\\
334.01	0.00623873171277178\\
335.01	0.00623768352610487\\
336.01	0.00623661223269421\\
337.01	0.0062355172820435\\
338.01	0.0062343981076407\\
339.01	0.006233254126295\\
340.01	0.00623208473743404\\
341.01	0.00623088932235893\\
342.01	0.00622966724345284\\
343.01	0.00622841784334071\\
344.01	0.00622714044399542\\
345.01	0.00622583434578652\\
346.01	0.0062244988264667\\
347.01	0.00622313314009197\\
348.01	0.00622173651586795\\
349.01	0.0062203081569188\\
350.01	0.00621884723897052\\
351.01	0.00621735290894193\\
352.01	0.00621582428343583\\
353.01	0.00621426044712168\\
354.01	0.00621266045099962\\
355.01	0.00621102331053706\\
356.01	0.00620934800366509\\
357.01	0.00620763346862314\\
358.01	0.00620587860163797\\
359.01	0.00620408225442216\\
360.01	0.006202243231476\\
361.01	0.00620036028717386\\
362.01	0.00619843212261682\\
363.01	0.00619645738222726\\
364.01	0.006194434650064\\
365.01	0.00619236244583001\\
366.01	0.00619023922054431\\
367.01	0.0061880633518454\\
368.01	0.00618583313889235\\
369.01	0.00618354679682352\\
370.01	0.00618120245073172\\
371.01	0.00617879812910809\\
372.01	0.00617633175670501\\
373.01	0.00617380114676127\\
374.01	0.00617120399252893\\
375.01	0.00616853785803526\\
376.01	0.00616580016800745\\
377.01	0.0061629881968826\\
378.01	0.00616009905681787\\
379.01	0.00615712968461129\\
380.01	0.00615407682743772\\
381.01	0.00615093702729765\\
382.01	0.00614770660407561\\
383.01	0.00614438163709984\\
384.01	0.00614095794509658\\
385.01	0.00613743106443638\\
386.01	0.0061337962255768\\
387.01	0.00613004832762434\\
388.01	0.00612618191096159\\
389.01	0.00612219112792623\\
390.01	0.00611806971158247\\
391.01	0.00611381094270715\\
392.01	0.00610940761522144\\
393.01	0.00610485200045085\\
394.01	0.00610013581080529\\
395.01	0.00609525016374377\\
396.01	0.00609018554726357\\
397.01	0.00608493178864853\\
398.01	0.00607947802886572\\
399.01	0.00607381270586999\\
400.01	0.00606792355121726\\
401.01	0.00606179760589656\\
402.01	0.00605542126325994\\
403.01	0.00604878034952735\\
404.01	0.00604186025572631\\
405.01	0.00603464613937136\\
406.01	0.00602712321997983\\
407.01	0.00601927720010481\\
408.01	0.00601109485345431\\
409.01	0.00600256483459055\\
410.01	0.00599367878155951\\
411.01	0.00598443280481278\\
412.01	0.00597482948450378\\
413.01	0.00596488193307401\\
414.01	0.0059546831236043\\
415.01	0.00594427709668789\\
416.01	0.00593365990536041\\
417.01	0.00592282755153628\\
418.01	0.00591177598822489\\
419.01	0.00590050112219411\\
420.01	0.00588899881712395\\
421.01	0.00587726489722895\\
422.01	0.00586529515155599\\
423.01	0.00585308533949892\\
424.01	0.00584063119717485\\
425.01	0.00582792844452006\\
426.01	0.00581497279344729\\
427.01	0.00580175995723419\\
428.01	0.00578828566130119\\
429.01	0.00577454565556098\\
430.01	0.00576053572854219\\
431.01	0.0057462517235175\\
432.01	0.0057316895568933\\
433.01	0.00571684523915225\\
434.01	0.00570171489867311\\
435.01	0.00568629480879477\\
436.01	0.00567058141853382\\
437.01	0.00565457138741205\\
438.01	0.00563826162490599\\
439.01	0.00562164933508589\\
440.01	0.00560473206707327\\
441.01	0.00558750777201248\\
442.01	0.00556997486731927\\
443.01	0.00555213230904023\\
444.01	0.00553397967322594\\
445.01	0.00551551724728434\\
446.01	0.00549674613234149\\
447.01	0.00547766835767428\\
448.01	0.00545828700829854\\
449.01	0.00543860636677846\\
450.01	0.00541863207024607\\
451.01	0.00539837128347489\\
452.01	0.00537783288860757\\
453.01	0.00535702769195026\\
454.01	0.00533596864670675\\
455.01	0.00531467108993984\\
456.01	0.0052931529927509\\
457.01	0.0052714352185777\\
458.01	0.00524954178253281\\
459.01	0.00522750010109665\\
460.01	0.0052053412154063\\
461.01	0.00518309996732541\\
462.01	0.00516081511544892\\
463.01	0.00513852934235167\\
464.01	0.00511628908477032\\
465.01	0.00509414411971495\\
466.01	0.00507214680710524\\
467.01	0.00505035085397365\\
468.01	0.0050288094198075\\
469.01	0.00500757232323613\\
470.01	0.00498668204924411\\
471.01	0.00496616831028183\\
472.01	0.00494604015159451\\
473.01	0.00492627232442627\\
474.01	0.00490655094030001\\
475.01	0.00488669277603261\\
476.01	0.0048667116707059\\
477.01	0.00484662252716021\\
478.01	0.00482644130395952\\
479.01	0.00480618498903938\\
480.01	0.00478587155017494\\
481.01	0.00476551985709734\\
482.01	0.00474514956932396\\
483.01	0.00472478098298111\\
484.01	0.00470443482912806\\
485.01	0.00468413201539438\\
486.01	0.00466389330222162\\
487.01	0.00464373890478968\\
488.01	0.00462368801200742\\
489.01	0.0046037582152793\\
490.01	0.00458396484218238\\
491.01	0.00456432019387871\\
492.01	0.0045448326919775\\
493.01	0.00452550595109121\\
494.01	0.00450633780899151\\
495.01	0.00448731936920279\\
496.01	0.00446843414388941\\
497.01	0.00444965743157864\\
498.01	0.00443095612914953\\
499.01	0.00441228926890717\\
500.01	0.0043936096982971\\
501.01	0.00437486750573409\\
502.01	0.00435602565748428\\
503.01	0.0043370789330409\\
504.01	0.00431802629035681\\
505.01	0.00429886530309774\\
506.01	0.00427959199346607\\
507.01	0.00426020067367799\\
508.01	0.0042406838042687\\
509.01	0.00422103187942214\\
510.01	0.00420123335174307\\
511.01	0.00418127461102447\\
512.01	0.00416114003344601\\
513.01	0.00414081211889399\\
514.01	0.00412027173411878\\
515.01	0.00409949847735075\\
516.01	0.00407847117443859\\
517.01	0.00405716850560663\\
518.01	0.00403556974269061\\
519.01	0.00401365554521901\\
520.01	0.00399140871398024\\
521.01	0.00396881471001509\\
522.01	0.00394586099374016\\
523.01	0.00392253494191472\\
524.01	0.0038988233106298\\
525.01	0.00387471229518079\\
526.01	0.00385018761236385\\
527.01	0.00382523459925919\\
528.01	0.00379983832618786\\
529.01	0.00377398371974654\\
530.01	0.00374765568967166\\
531.01	0.00372083925084878\\
532.01	0.00369351962926699\\
533.01	0.00366568233851864\\
534.01	0.00363731321224822\\
535.01	0.00360839837891926\\
536.01	0.00357892417022308\\
537.01	0.00354887696689357\\
538.01	0.00351824303743295\\
539.01	0.00348700849082071\\
540.01	0.00345515929959017\\
541.01	0.00342268132576639\\
542.01	0.0033895603433927\\
543.01	0.00335578205619657\\
544.01	0.00332133210906197\\
545.01	0.00328619609232783\\
546.01	0.00325035953859222\\
547.01	0.00321380791264442\\
548.01	0.00317652659640092\\
549.01	0.00313850087220274\\
550.01	0.00309971590924588\\
551.01	0.00306015675865827\\
552.01	0.00301980836014317\\
553.01	0.00297865555498463\\
554.01	0.00293668310007627\\
555.01	0.00289387568239544\\
556.01	0.00285021793428497\\
557.01	0.00280569445007147\\
558.01	0.00276028980470851\\
559.01	0.00271398857526332\\
560.01	0.0026667753661384\\
561.01	0.0026186348388955\\
562.01	0.0025695517473923\\
563.01	0.00251951097863612\\
564.01	0.00246849759935086\\
565.01	0.00241649690806831\\
566.01	0.0023634944929314\\
567.01	0.0023094762957353\\
568.01	0.00225442868280603\\
569.01	0.00219833852332488\\
570.01	0.00214119327569023\\
571.01	0.00208298108246332\\
572.01	0.00202369087436417\\
573.01	0.00196331248367459\\
574.01	0.00190183676726109\\
575.01	0.0018392557392658\\
576.01	0.0017755627133218\\
577.01	0.00171075245390312\\
578.01	0.00164482133605403\\
579.01	0.00157776751222645\\
580.01	0.00150959108427172\\
581.01	0.00144029427774478\\
582.01	0.00136988161453246\\
583.01	0.001298360078346\\
584.01	0.00122573926573864\\
585.01	0.00115203151291554\\
586.01	0.0010772519855635\\
587.01	0.00100141871507007\\
588.01	0.000924552559616688\\
589.01	0.000846677062455354\\
590.01	0.000767818171894426\\
591.01	0.000688003777723883\\
592.01	0.000607263006503161\\
593.01	0.000525625202691551\\
594.01	0.000443118503243908\\
595.01	0.000359767889058513\\
596.01	0.000275592566341293\\
597.01	0.000190602493045661\\
598.01	0.000104793818189718\\
599.01	3.18230442566507e-05\\
599.02	3.12770957178118e-05\\
599.03	3.07343620570243e-05\\
599.04	3.01948752177914e-05\\
599.05	2.96586674567757e-05\\
599.06	2.9125771346868e-05\\
599.07	2.8596219780316e-05\\
599.08	2.80700459718581e-05\\
599.09	2.75472834618973e-05\\
599.1	2.70279661197196e-05\\
599.11	2.65121281467183e-05\\
599.12	2.59998040796645e-05\\
599.13	2.54910287940252e-05\\
599.14	2.49858375072735e-05\\
599.15	2.44842657822781e-05\\
599.16	2.39863495306858e-05\\
599.17	2.34921250163653e-05\\
599.18	2.30016288588781e-05\\
599.19	2.2514898036969e-05\\
599.2	2.20319698921133e-05\\
599.21	2.15528821320803e-05\\
599.22	2.10776728345428e-05\\
599.23	2.06063804507207e-05\\
599.24	2.01390438090542e-05\\
599.25	1.96757021189257e-05\\
599.26	1.92163949743994e-05\\
599.27	1.87611623580202e-05\\
599.28	1.83100446446254e-05\\
599.29	1.78630826052212e-05\\
599.3	1.7420317410876e-05\\
599.31	1.69817906366561e-05\\
599.32	1.65475442656122e-05\\
599.33	1.61176206927866e-05\\
599.34	1.56920627292778e-05\\
599.35	1.52709136063342e-05\\
599.36	1.48542169794846e-05\\
599.37	1.44420169327295e-05\\
599.38	1.40343579827472e-05\\
599.39	1.36312859119678e-05\\
599.4	1.32328499381947e-05\\
599.41	1.28390997671674e-05\\
599.42	1.24500855973562e-05\\
599.43	1.20658581248146e-05\\
599.44	1.16864685480583e-05\\
599.45	1.13119685730117e-05\\
599.46	1.09424104179964e-05\\
599.47	1.05778468187656e-05\\
599.48	1.02183310335905e-05\\
599.49	9.86391684839466e-06\\
599.5	9.51465858194285e-06\\
599.51	9.17061109107463e-06\\
599.52	8.83182977599699e-06\\
599.53	8.49837058562383e-06\\
599.54	8.1702900229675e-06\\
599.55	7.84764515058753e-06\\
599.56	7.53049359608279e-06\\
599.57	7.21889355765649e-06\\
599.58	6.91290380971064e-06\\
599.59	6.61258370851861e-06\\
599.6	6.31799319793756e-06\\
599.61	6.02919281519031e-06\\
599.62	5.74624369668701e-06\\
599.63	5.46920758391981e-06\\
599.64	5.19814682940593e-06\\
599.65	4.93312440269685e-06\\
599.66	4.67420389643411e-06\\
599.67	4.42144953247646e-06\\
599.68	4.17492616808582e-06\\
599.69	3.9346993021671e-06\\
599.7	3.70083508156871e-06\\
599.71	3.47340030746289e-06\\
599.72	3.25246244175549e-06\\
599.73	3.03808961359987e-06\\
599.74	2.83035062593855e-06\\
599.75	2.62931496212288e-06\\
599.76	2.43505279260738e-06\\
599.77	2.24763498169606e-06\\
599.78	2.06713309435641e-06\\
599.79	1.89361940310974e-06\\
599.8	1.72716689498045e-06\\
599.81	1.56784927850782e-06\\
599.82	1.41574099085315e-06\\
599.83	1.27091720493813e-06\\
599.84	1.13345383668736e-06\\
599.85	1.00342755231415e-06\\
599.86	8.8091577571392e-07\\
599.87	7.65996695881871e-07\\
599.88	6.58749274441706e-07\\
599.89	5.59253253241965e-07\\
599.9	4.67589162013102e-07\\
599.91	3.83838326099145e-07\\
599.92	3.0808287429171e-07\\
599.93	2.4040574671258e-07\\
599.94	1.80890702777825e-07\\
599.95	1.29622329257326e-07\\
599.96	8.66860484019516e-08\\
599.97	5.21681261331924e-08\\
599.98	2.61556803542173e-08\\
599.99	8.73668930083393e-09\\
600	0\\
};
\addplot [color=red!75!mycolor17,solid,forget plot]
  table[row sep=crcr]{%
0.01	0.00588381158434445\\
1.01	0.00588380981397887\\
2.01	0.00588380800610954\\
3.01	0.00588380615994001\\
4.01	0.0058838042746568\\
5.01	0.00588380234942946\\
6.01	0.0058838003834096\\
7.01	0.00588379837573055\\
8.01	0.00588379632550778\\
9.01	0.00588379423183732\\
10.01	0.00588379209379631\\
11.01	0.00588378991044209\\
12.01	0.00588378768081189\\
13.01	0.00588378540392255\\
14.01	0.00588378307877\\
15.01	0.00588378070432864\\
16.01	0.00588377827955094\\
17.01	0.00588377580336728\\
18.01	0.00588377327468511\\
19.01	0.0058837706923886\\
20.01	0.00588376805533812\\
21.01	0.00588376536236967\\
22.01	0.00588376261229484\\
23.01	0.00588375980389944\\
24.01	0.00588375693594367\\
25.01	0.00588375400716098\\
26.01	0.00588375101625819\\
27.01	0.0058837479619143\\
28.01	0.00588374484278038\\
29.01	0.00588374165747836\\
30.01	0.00588373840460112\\
31.01	0.00588373508271131\\
32.01	0.00588373169034102\\
33.01	0.00588372822599129\\
34.01	0.0058837246881306\\
35.01	0.00588372107519516\\
36.01	0.00588371738558779\\
37.01	0.00588371361767722\\
38.01	0.00588370976979719\\
39.01	0.00588370584024625\\
40.01	0.00588370182728623\\
41.01	0.00588369772914211\\
42.01	0.00588369354400089\\
43.01	0.00588368927001094\\
44.01	0.00588368490528103\\
45.01	0.00588368044787969\\
46.01	0.00588367589583411\\
47.01	0.00588367124712947\\
48.01	0.00588366649970772\\
49.01	0.00588366165146701\\
50.01	0.00588365670026041\\
51.01	0.00588365164389562\\
52.01	0.00588364648013299\\
53.01	0.00588364120668539\\
54.01	0.00588363582121674\\
55.01	0.00588363032134091\\
56.01	0.00588362470462086\\
57.01	0.00588361896856785\\
58.01	0.00588361311063934\\
59.01	0.00588360712823887\\
60.01	0.00588360101871454\\
61.01	0.00588359477935745\\
62.01	0.00588358840740108\\
63.01	0.00588358190001976\\
64.01	0.00588357525432742\\
65.01	0.00588356846737608\\
66.01	0.00588356153615494\\
67.01	0.00588355445758881\\
68.01	0.00588354722853688\\
69.01	0.00588353984579109\\
70.01	0.00588353230607453\\
71.01	0.0058835246060407\\
72.01	0.00588351674227109\\
73.01	0.00588350871127429\\
74.01	0.0058835005094842\\
75.01	0.00588349213325842\\
76.01	0.00588348357887669\\
77.01	0.00588347484253906\\
78.01	0.00588346592036446\\
79.01	0.00588345680838855\\
80.01	0.00588344750256232\\
81.01	0.00588343799874999\\
82.01	0.00588342829272758\\
83.01	0.00588341838018034\\
84.01	0.00588340825670129\\
85.01	0.00588339791778932\\
86.01	0.00588338735884676\\
87.01	0.00588337657517755\\
88.01	0.00588336556198514\\
89.01	0.00588335431437017\\
90.01	0.00588334282732862\\
91.01	0.00588333109574909\\
92.01	0.00588331911441085\\
93.01	0.00588330687798141\\
94.01	0.00588329438101373\\
95.01	0.00588328161794454\\
96.01	0.00588326858309107\\
97.01	0.00588325527064932\\
98.01	0.00588324167469059\\
99.01	0.00588322778915878\\
100.01	0.00588321360786858\\
101.01	0.00588319912450226\\
102.01	0.00588318433260617\\
103.01	0.00588316922558886\\
104.01	0.00588315379671724\\
105.01	0.00588313803911415\\
106.01	0.00588312194575496\\
107.01	0.00588310550946438\\
108.01	0.00588308872291343\\
109.01	0.00588307157861613\\
110.01	0.00588305406892553\\
111.01	0.00588303618603111\\
112.01	0.00588301792195478\\
113.01	0.00588299926854728\\
114.01	0.00588298021748495\\
115.01	0.00588296076026503\\
116.01	0.005882940888203\\
117.01	0.00588292059242756\\
118.01	0.00588289986387734\\
119.01	0.00588287869329652\\
120.01	0.00588285707123074\\
121.01	0.00588283498802288\\
122.01	0.00588281243380844\\
123.01	0.00588278939851127\\
124.01	0.00588276587183914\\
125.01	0.00588274184327876\\
126.01	0.0058827173020915\\
127.01	0.00588269223730786\\
128.01	0.00588266663772314\\
129.01	0.00588264049189196\\
130.01	0.00588261378812332\\
131.01	0.0058825865144752\\
132.01	0.00588255865874927\\
133.01	0.00588253020848514\\
134.01	0.00588250115095503\\
135.01	0.00588247147315758\\
136.01	0.00588244116181235\\
137.01	0.00588241020335382\\
138.01	0.00588237858392497\\
139.01	0.00588234628937106\\
140.01	0.00588231330523328\\
141.01	0.00588227961674249\\
142.01	0.00588224520881197\\
143.01	0.00588221006603104\\
144.01	0.00588217417265786\\
145.01	0.00588213751261204\\
146.01	0.00588210006946808\\
147.01	0.0058820618264473\\
148.01	0.00588202276641042\\
149.01	0.00588198287184971\\
150.01	0.00588194212488134\\
151.01	0.00588190050723683\\
152.01	0.0058818580002552\\
153.01	0.00588181458487425\\
154.01	0.00588177024162174\\
155.01	0.00588172495060694\\
156.01	0.00588167869151168\\
157.01	0.00588163144358043\\
158.01	0.00588158318561174\\
159.01	0.00588153389594783\\
160.01	0.00588148355246559\\
161.01	0.00588143213256567\\
162.01	0.00588137961316285\\
163.01	0.00588132597067528\\
164.01	0.00588127118101404\\
165.01	0.0058812152195715\\
166.01	0.0058811580612111\\
167.01	0.00588109968025551\\
168.01	0.00588104005047472\\
169.01	0.00588097914507433\\
170.01	0.00588091693668332\\
171.01	0.0058808533973417\\
172.01	0.00588078849848801\\
173.01	0.00588072221094589\\
174.01	0.00588065450491102\\
175.01	0.0058805853499375\\
176.01	0.00588051471492441\\
177.01	0.00588044256810102\\
178.01	0.00588036887701293\\
179.01	0.00588029360850673\\
180.01	0.00588021672871559\\
181.01	0.00588013820304342\\
182.01	0.00588005799614911\\
183.01	0.0058799760719306\\
184.01	0.00587989239350861\\
185.01	0.00587980692320948\\
186.01	0.00587971962254822\\
187.01	0.00587963045221116\\
188.01	0.00587953937203801\\
189.01	0.00587944634100323\\
190.01	0.00587935131719767\\
191.01	0.00587925425780959\\
192.01	0.0058791551191048\\
193.01	0.00587905385640724\\
194.01	0.00587895042407826\\
195.01	0.00587884477549585\\
196.01	0.00587873686303343\\
197.01	0.00587862663803878\\
198.01	0.00587851405081081\\
199.01	0.00587839905057767\\
200.01	0.00587828158547348\\
201.01	0.00587816160251482\\
202.01	0.0058780390475759\\
203.01	0.00587791386536484\\
204.01	0.00587778599939807\\
205.01	0.0058776553919744\\
206.01	0.00587752198414932\\
207.01	0.00587738571570717\\
208.01	0.0058772465251347\\
209.01	0.00587710434959241\\
210.01	0.00587695912488636\\
211.01	0.0058768107854383\\
212.01	0.00587665926425642\\
213.01	0.00587650449290441\\
214.01	0.00587634640147051\\
215.01	0.00587618491853521\\
216.01	0.00587601997113923\\
217.01	0.00587585148474994\\
218.01	0.00587567938322747\\
219.01	0.00587550358878984\\
220.01	0.00587532402197739\\
221.01	0.0058751406016171\\
222.01	0.00587495324478494\\
223.01	0.00587476186676826\\
224.01	0.00587456638102727\\
225.01	0.00587436669915539\\
226.01	0.00587416273083883\\
227.01	0.0058739543838156\\
228.01	0.00587374156383323\\
229.01	0.00587352417460554\\
230.01	0.00587330211776942\\
231.01	0.00587307529283898\\
232.01	0.00587284359716003\\
233.01	0.00587260692586363\\
234.01	0.00587236517181778\\
235.01	0.00587211822557813\\
236.01	0.00587186597533873\\
237.01	0.00587160830688041\\
238.01	0.00587134510351866\\
239.01	0.00587107624605066\\
240.01	0.00587080161270023\\
241.01	0.00587052107906241\\
242.01	0.00587023451804657\\
243.01	0.00586994179981858\\
244.01	0.00586964279174073\\
245.01	0.00586933735831147\\
246.01	0.00586902536110367\\
247.01	0.00586870665870073\\
248.01	0.00586838110663213\\
249.01	0.00586804855730707\\
250.01	0.00586770885994708\\
251.01	0.00586736186051688\\
252.01	0.00586700740165385\\
253.01	0.00586664532259585\\
254.01	0.00586627545910731\\
255.01	0.00586589764340432\\
256.01	0.00586551170407741\\
257.01	0.00586511746601293\\
258.01	0.00586471475031261\\
259.01	0.00586430337421192\\
260.01	0.00586388315099517\\
261.01	0.00586345388991064\\
262.01	0.00586301539608245\\
263.01	0.00586256747042118\\
264.01	0.00586210990953192\\
265.01	0.00586164250562143\\
266.01	0.00586116504640177\\
267.01	0.00586067731499283\\
268.01	0.00586017908982239\\
269.01	0.00585967014452416\\
270.01	0.00585915024783305\\
271.01	0.00585861916347889\\
272.01	0.00585807665007678\\
273.01	0.0058575224610167\\
274.01	0.00585695634434821\\
275.01	0.00585637804266499\\
276.01	0.00585578729298533\\
277.01	0.00585518382663051\\
278.01	0.00585456736910046\\
279.01	0.0058539376399463\\
280.01	0.00585329435264057\\
281.01	0.00585263721444388\\
282.01	0.00585196592626974\\
283.01	0.00585128018254474\\
284.01	0.00585057967106704\\
285.01	0.00584986407286065\\
286.01	0.00584913306202712\\
287.01	0.00584838630559376\\
288.01	0.00584762346335831\\
289.01	0.0058468441877298\\
290.01	0.00584604812356651\\
291.01	0.0058452349080099\\
292.01	0.00584440417031455\\
293.01	0.00584355553167466\\
294.01	0.00584268860504691\\
295.01	0.00584180299496801\\
296.01	0.00584089829736983\\
297.01	0.00583997409938889\\
298.01	0.00583902997917247\\
299.01	0.00583806550567998\\
300.01	0.00583708023847964\\
301.01	0.00583607372754137\\
302.01	0.0058350455130235\\
303.01	0.00583399512505619\\
304.01	0.00583292208351904\\
305.01	0.00583182589781418\\
306.01	0.00583070606663385\\
307.01	0.00582956207772283\\
308.01	0.00582839340763591\\
309.01	0.00582719952148923\\
310.01	0.00582597987270716\\
311.01	0.00582473390276236\\
312.01	0.00582346104091066\\
313.01	0.00582216070392023\\
314.01	0.00582083229579483\\
315.01	0.00581947520749061\\
316.01	0.00581808881662733\\
317.01	0.00581667248719307\\
318.01	0.00581522556924281\\
319.01	0.00581374739859083\\
320.01	0.00581223729649667\\
321.01	0.0058106945693446\\
322.01	0.00580911850831705\\
323.01	0.00580750838906135\\
324.01	0.00580586347135024\\
325.01	0.00580418299873598\\
326.01	0.00580246619819846\\
327.01	0.00580071227978684\\
328.01	0.00579892043625547\\
329.01	0.00579708984269393\\
330.01	0.00579521965615169\\
331.01	0.0057933090152569\\
332.01	0.00579135703983091\\
333.01	0.00578936283049848\\
334.01	0.00578732546829255\\
335.01	0.00578524401425749\\
336.01	0.00578311750904827\\
337.01	0.00578094497252816\\
338.01	0.00577872540336539\\
339.01	0.00577645777862977\\
340.01	0.00577414105339103\\
341.01	0.00577177416031952\\
342.01	0.00576935600929145\\
343.01	0.0057668854870002\\
344.01	0.00576436145657629\\
345.01	0.0057617827572184\\
346.01	0.00575914820383806\\
347.01	0.00575645658672056\\
348.01	0.00575370667120748\\
349.01	0.00575089719740307\\
350.01	0.0057480268799102\\
351.01	0.00574509440760137\\
352.01	0.00574209844342953\\
353.01	0.00573903762428736\\
354.01	0.00573591056092131\\
355.01	0.00573271583790994\\
356.01	0.0057294520137157\\
357.01	0.00572611762082254\\
358.01	0.00572271116597093\\
359.01	0.00571923113050552\\
360.01	0.00571567597085059\\
361.01	0.00571204411913296\\
362.01	0.00570833398397189\\
363.01	0.00570454395145984\\
364.01	0.00570067238636121\\
365.01	0.00569671763355712\\
366.01	0.0056926780197718\\
367.01	0.00568855185561727\\
368.01	0.00568433743799914\\
369.01	0.00568003305293321\\
370.01	0.00567563697882439\\
371.01	0.00567114749027247\\
372.01	0.00566656286246969\\
373.01	0.0056618813762707\\
374.01	0.0056571013240185\\
375.01	0.00565222101622497\\
376.01	0.0056472387892136\\
377.01	0.00564215301384461\\
378.01	0.00563696210545766\\
379.01	0.00563166453517971\\
380.01	0.00562625884276317\\
381.01	0.00562074365113432\\
382.01	0.0056151176828499\\
383.01	0.00560937977867574\\
384.01	0.00560352891851695\\
385.01	0.00559756424494493\\
386.01	0.00559148508957717\\
387.01	0.00558529100257136\\
388.01	0.00557898178549608\\
389.01	0.00557255752782516\\
390.01	0.00556601864727755\\
391.01	0.00555936593417019\\
392.01	0.00555260059987158\\
393.01	0.00554572432931722\\
394.01	0.0055387393373643\\
395.01	0.00553164842850102\\
396.01	0.00552445505905949\\
397.01	0.00551716340056851\\
398.01	0.00550977840219078\\
399.01	0.00550230584923837\\
400.01	0.00549475241348927\\
401.01	0.00548712568931336\\
402.01	0.0054794342073349\\
403.01	0.00547168741431669\\
404.01	0.00546389560392106\\
405.01	0.00545606977766861\\
406.01	0.00544822140836945\\
407.01	0.00544036206901984\\
408.01	0.00543250287793826\\
409.01	0.00542465369485019\\
410.01	0.00541682198154169\\
411.01	0.0054090112130263\\
412.01	0.00540121868888936\\
413.01	0.0053934311467888\\
414.01	0.00538555388893473\\
415.01	0.00537753794871848\\
416.01	0.00536938187750997\\
417.01	0.00536108428583244\\
418.01	0.00535264385052405\\
419.01	0.00534405932245016\\
420.01	0.00533532953480055\\
421.01	0.005326453412012\\
422.01	0.0053174299793551\\
423.01	0.00530825837321148\\
424.01	0.00529893785206253\\
425.01	0.00528946780823533\\
426.01	0.00527984778044697\\
427.01	0.0052700774671813\\
428.01	0.00526015674092809\\
429.01	0.00525008566331355\\
430.01	0.00523986450114147\\
431.01	0.00522949374336115\\
432.01	0.00521897411896488\\
433.01	0.00520830661580828\\
434.01	0.00519749250032951\\
435.01	0.00518653333812438\\
436.01	0.00517543101530906\\
437.01	0.00516418776057607\\
438.01	0.00515280616780604\\
439.01	0.00514128921905997\\
440.01	0.00512964030771935\\
441.01	0.00511786326147795\\
442.01	0.00510596236481325\\
443.01	0.005093942380472\\
444.01	0.00508180856939705\\
445.01	0.00506956670839642\\
446.01	0.00505722310470398\\
447.01	0.00504478460641066\\
448.01	0.00503225860754379\\
449.01	0.00501965304634434\\
450.01	0.00500697639503538\\
451.01	0.00499423763908425\\
452.01	0.00498144624364622\\
453.01	0.00496861210453071\\
454.01	0.00495574548067791\\
455.01	0.00494285690482716\\
456.01	0.00492995706867163\\
457.01	0.00491705667849878\\
458.01	0.00490416627718298\\
459.01	0.00489129602842898\\
460.01	0.00487845545954164\\
461.01	0.00486565315984728\\
462.01	0.00485289643286296\\
463.01	0.00484019090226389\\
464.01	0.00482754007636114\\
465.01	0.00481494488149143\\
466.01	0.00480240318289198\\
467.01	0.00478990932375952\\
468.01	0.00477745373048349\\
469.01	0.0047650226563763\\
470.01	0.00475259816988718\\
471.01	0.00474015853571868\\
472.01	0.00472767920367289\\
473.01	0.00471513474777203\\
474.01	0.00470250725175043\\
475.01	0.00468979689859931\\
476.01	0.00467700686081363\\
477.01	0.00466414014221827\\
478.01	0.00465119951291316\\
479.01	0.00463818743693199\\
480.01	0.00462510599264845\\
481.01	0.00461195678621757\\
482.01	0.00459874085866007\\
483.01	0.00458545858761119\\
484.01	0.00457210958527451\\
485.01	0.00455869259475305\\
486.01	0.00454520538770111\\
487.01	0.00453164466714042\\
488.01	0.00451800598032893\\
489.01	0.00450428364772914\\
490.01	0.00449047071535862\\
491.01	0.00447655893905665\\
492.01	0.00446253881033878\\
493.01	0.00444839963432259\\
494.01	0.00443412967037021\\
495.01	0.00441971634512523\\
496.01	0.00440514654481864\\
497.01	0.00439040698802298\\
498.01	0.00437548466997626\\
499.01	0.00436036735305318\\
500.01	0.00434504405182897\\
501.01	0.00432950542122603\\
502.01	0.00431374376253249\\
503.01	0.00429775181588751\\
504.01	0.00428152185964032\\
505.01	0.00426504566107352\\
506.01	0.00424831449026248\\
507.01	0.00423131914387567\\
508.01	0.00421404997969988\\
509.01	0.00419649696233378\\
510.01	0.00417864971998277\\
511.01	0.00416049761158845\\
512.01	0.00414202980261649\\
513.01	0.00412323534667832\\
514.01	0.00410410326881245\\
515.01	0.00408462264473276\\
516.01	0.00406478266880003\\
517.01	0.00404457270213406\\
518.01	0.00402398229155767\\
519.01	0.00400300115064997\\
520.01	0.00398161909713912\\
521.01	0.00395982594793969\\
522.01	0.00393761139778053\\
523.01	0.0039149649645009\\
524.01	0.00389187599995229\\
525.01	0.00386833370872558\\
526.01	0.00384432716607575\\
527.01	0.00381984533406512\\
528.01	0.00379487707489989\\
529.01	0.00376941116044047\\
530.01	0.00374343627697373\\
531.01	0.00371694102457945\\
532.01	0.00368991391083733\\
533.01	0.00366234333922554\\
534.01	0.00363421759335883\\
535.01	0.00360552481914565\\
536.01	0.00357625300787387\\
537.01	0.00354638998385786\\
538.01	0.0035159233995706\\
539.01	0.00348484073635985\\
540.01	0.00345312930589523\\
541.01	0.00342077625093994\\
542.01	0.0033877685454502\\
543.01	0.00335409299413567\\
544.01	0.00331973623170642\\
545.01	0.00328468472213207\\
546.01	0.00324892475832306\\
547.01	0.0032124424627048\\
548.01	0.00317522378916534\\
549.01	0.00313725452679205\\
550.01	0.00309852030566009\\
551.01	0.00305900660468921\\
552.01	0.00301869876133101\\
553.01	0.0029775819828656\\
554.01	0.00293564135940069\\
555.01	0.00289286187882892\\
556.01	0.00284922844403485\\
557.01	0.00280472589266526\\
558.01	0.00275933901978964\\
559.01	0.00271305260378651\\
560.01	0.00266585143579169\\
561.01	0.00261772035304431\\
562.01	0.00256864427646527\\
563.01	0.00251860825281521\\
564.01	0.00246759750180826\\
565.01	0.00241559746860962\\
566.01	0.00236259388219452\\
567.01	0.00230857282007239\\
568.01	0.00225352077989341\\
569.01	0.00219742475845609\\
570.01	0.00214027233862706\\
571.01	0.00208205178465884\\
572.01	0.00202275214634875\\
573.01	0.00196236337241117\\
574.01	0.00190087643333459\\
575.01	0.00183828345384417\\
576.01	0.00177457785488343\\
577.01	0.00170975450473966\\
578.01	0.00164380987854659\\
579.01	0.00157674222487883\\
580.01	0.00150855173747032\\
581.01	0.00143924072920082\\
582.01	0.00136881380434555\\
583.01	0.00129727802360513\\
584.01	0.00122464305453996\\
585.01	0.00115092129761855\\
586.01	0.00107612797501737\\
587.01	0.00100028116541664\\
588.01	0.000923401763108929\\
589.01	0.000845513333520894\\
590.01	0.000766641829415248\\
591.01	0.000686815122188048\\
592.01	0.000606062290302432\\
593.01	0.000524412591375313\\
594.01	0.000441894024979966\\
595.01	0.000358531368867966\\
596.01	0.000274343540840818\\
597.01	0.0001893401004031\\
598.01	0.000103516656727439\\
599.01	3.18230442563801e-05\\
599.02	3.1277095717562e-05\\
599.03	3.07343620567901e-05\\
599.04	3.01948752175762e-05\\
599.05	2.96586674565728e-05\\
599.06	2.91257713466823e-05\\
599.07	2.85962197801425e-05\\
599.08	2.8070045971695e-05\\
599.09	2.75472834617499e-05\\
599.1	2.70279661195826e-05\\
599.11	2.65121281465882e-05\\
599.12	2.59998040795482e-05\\
599.13	2.54910287939159e-05\\
599.14	2.49858375071747e-05\\
599.15	2.44842657821844e-05\\
599.16	2.39863495305973e-05\\
599.17	2.34921250162872e-05\\
599.18	2.30016288588052e-05\\
599.19	2.25148980369048e-05\\
599.2	2.20319698920526e-05\\
599.21	2.1552882132023e-05\\
599.22	2.10776728344908e-05\\
599.23	2.06063804506721e-05\\
599.24	2.01390438090126e-05\\
599.25	1.96757021188858e-05\\
599.26	1.92163949743647e-05\\
599.27	1.87611623579872e-05\\
599.28	1.83100446445959e-05\\
599.29	1.78630826051952e-05\\
599.3	1.74203174108534e-05\\
599.31	1.69817906366353e-05\\
599.32	1.65475442655931e-05\\
599.33	1.61176206927693e-05\\
599.34	1.56920627292639e-05\\
599.35	1.52709136063203e-05\\
599.36	1.48542169794742e-05\\
599.37	1.44420169327208e-05\\
599.38	1.40343579827368e-05\\
599.39	1.36312859119591e-05\\
599.4	1.32328499381877e-05\\
599.41	1.28390997671604e-05\\
599.42	1.24500855973528e-05\\
599.43	1.20658581248094e-05\\
599.44	1.16864685480531e-05\\
599.45	1.13119685730065e-05\\
599.46	1.09424104179946e-05\\
599.47	1.05778468187639e-05\\
599.48	1.02183310335888e-05\\
599.49	9.86391684839466e-06\\
599.5	9.51465858194112e-06\\
599.51	9.17061109107116e-06\\
599.52	8.83182977599525e-06\\
599.53	8.4983705856221e-06\\
599.54	8.1702900229675e-06\\
599.55	7.84764515058753e-06\\
599.56	7.53049359608453e-06\\
599.57	7.21889355765649e-06\\
599.58	6.91290380971064e-06\\
599.59	6.61258370851688e-06\\
599.6	6.31799319793756e-06\\
599.61	6.02919281519031e-06\\
599.62	5.74624369668701e-06\\
599.63	5.46920758391981e-06\\
599.64	5.19814682940767e-06\\
599.65	4.93312440269685e-06\\
599.66	4.67420389643411e-06\\
599.67	4.42144953247646e-06\\
599.68	4.17492616808582e-06\\
599.69	3.93469930216536e-06\\
599.7	3.70083508157044e-06\\
599.71	3.47340030746116e-06\\
599.72	3.25246244175549e-06\\
599.73	3.03808961360161e-06\\
599.74	2.83035062593855e-06\\
599.75	2.62931496212288e-06\\
599.76	2.43505279260738e-06\\
599.77	2.24763498169606e-06\\
599.78	2.06713309435641e-06\\
599.79	1.89361940311147e-06\\
599.8	1.72716689497872e-06\\
599.81	1.56784927850956e-06\\
599.82	1.41574099085315e-06\\
599.83	1.27091720493987e-06\\
599.84	1.13345383668563e-06\\
599.85	1.00342755231589e-06\\
599.86	8.8091577571392e-07\\
599.87	7.65996695880136e-07\\
599.88	6.5874927444344e-07\\
599.89	5.59253253243699e-07\\
599.9	4.67589162011367e-07\\
599.91	3.83838326099145e-07\\
599.92	3.0808287429171e-07\\
599.93	2.40405746710845e-07\\
599.94	1.8089070277609e-07\\
599.95	1.2962232925906e-07\\
599.96	8.66860484002169e-08\\
599.97	5.21681261331924e-08\\
599.98	2.61556803542173e-08\\
599.99	8.73668930083393e-09\\
600	0\\
};
\addplot [color=red!80!mycolor19,solid,forget plot]
  table[row sep=crcr]{%
0.01	0.00547444890358919\\
1.01	0.00547444712667968\\
2.01	0.00547444531233606\\
3.01	0.00547444345976878\\
4.01	0.0054744415681716\\
5.01	0.00547443963672114\\
6.01	0.00547443766457662\\
7.01	0.00547443565087979\\
8.01	0.00547443359475392\\
9.01	0.005474431495304\\
10.01	0.00547442935161593\\
11.01	0.00547442716275643\\
12.01	0.00547442492777244\\
13.01	0.005474422645691\\
14.01	0.00547442031551823\\
15.01	0.00547441793623944\\
16.01	0.00547441550681854\\
17.01	0.00547441302619754\\
18.01	0.00547441049329609\\
19.01	0.00547440790701098\\
20.01	0.00547440526621585\\
21.01	0.00547440256976035\\
22.01	0.00547439981646993\\
23.01	0.00547439700514498\\
24.01	0.00547439413456091\\
25.01	0.00547439120346722\\
26.01	0.0054743882105868\\
27.01	0.00547438515461574\\
28.01	0.00547438203422226\\
29.01	0.00547437884804688\\
30.01	0.00547437559470116\\
31.01	0.00547437227276731\\
32.01	0.00547436888079773\\
33.01	0.0054743654173139\\
34.01	0.00547436188080657\\
35.01	0.00547435826973421\\
36.01	0.00547435458252294\\
37.01	0.0054743508175653\\
38.01	0.00547434697322043\\
39.01	0.00547434304781212\\
40.01	0.00547433903962925\\
41.01	0.00547433494692446\\
42.01	0.00547433076791327\\
43.01	0.00547432650077363\\
44.01	0.00547432214364497\\
45.01	0.00547431769462733\\
46.01	0.00547431315178085\\
47.01	0.00547430851312424\\
48.01	0.00547430377663492\\
49.01	0.00547429894024707\\
50.01	0.00547429400185142\\
51.01	0.00547428895929409\\
52.01	0.00547428381037574\\
53.01	0.00547427855285056\\
54.01	0.00547427318442522\\
55.01	0.00547426770275792\\
56.01	0.00547426210545742\\
57.01	0.00547425639008182\\
58.01	0.00547425055413781\\
59.01	0.00547424459507934\\
60.01	0.00547423851030633\\
61.01	0.0054742322971641\\
62.01	0.0054742259529418\\
63.01	0.00547421947487096\\
64.01	0.00547421286012485\\
65.01	0.00547420610581714\\
66.01	0.00547419920900045\\
67.01	0.00547419216666483\\
68.01	0.0054741849757369\\
69.01	0.00547417763307836\\
70.01	0.00547417013548458\\
71.01	0.00547416247968332\\
72.01	0.00547415466233293\\
73.01	0.00547414668002145\\
74.01	0.00547413852926446\\
75.01	0.00547413020650412\\
76.01	0.00547412170810745\\
77.01	0.00547411303036458\\
78.01	0.0054741041694873\\
79.01	0.00547409512160737\\
80.01	0.00547408588277475\\
81.01	0.00547407644895593\\
82.01	0.00547406681603214\\
83.01	0.00547405697979773\\
84.01	0.00547404693595819\\
85.01	0.00547403668012787\\
86.01	0.00547402620782881\\
87.01	0.00547401551448836\\
88.01	0.00547400459543732\\
89.01	0.00547399344590764\\
90.01	0.00547398206103065\\
91.01	0.00547397043583482\\
92.01	0.00547395856524361\\
93.01	0.00547394644407307\\
94.01	0.00547393406702997\\
95.01	0.00547392142870883\\
96.01	0.00547390852359046\\
97.01	0.00547389534603869\\
98.01	0.00547388189029847\\
99.01	0.00547386815049334\\
100.01	0.00547385412062224\\
101.01	0.0054738397945576\\
102.01	0.0054738251660425\\
103.01	0.0054738102286877\\
104.01	0.00547379497596908\\
105.01	0.00547377940122467\\
106.01	0.00547376349765189\\
107.01	0.00547374725830422\\
108.01	0.00547373067608866\\
109.01	0.00547371374376237\\
110.01	0.00547369645392947\\
111.01	0.00547367879903787\\
112.01	0.00547366077137625\\
113.01	0.00547364236307018\\
114.01	0.00547362356607897\\
115.01	0.00547360437219226\\
116.01	0.00547358477302621\\
117.01	0.00547356476002007\\
118.01	0.00547354432443229\\
119.01	0.00547352345733684\\
120.01	0.00547350214961929\\
121.01	0.00547348039197256\\
122.01	0.00547345817489328\\
123.01	0.00547343548867756\\
124.01	0.00547341232341654\\
125.01	0.00547338866899223\\
126.01	0.00547336451507322\\
127.01	0.00547333985110985\\
128.01	0.0054733146663301\\
129.01	0.00547328894973431\\
130.01	0.00547326269009112\\
131.01	0.00547323587593198\\
132.01	0.00547320849554641\\
133.01	0.00547318053697676\\
134.01	0.00547315198801326\\
135.01	0.00547312283618873\\
136.01	0.0054730930687728\\
137.01	0.00547306267276675\\
138.01	0.00547303163489744\\
139.01	0.00547299994161216\\
140.01	0.00547296757907228\\
141.01	0.00547293453314739\\
142.01	0.00547290078940917\\
143.01	0.00547286633312511\\
144.01	0.00547283114925213\\
145.01	0.00547279522243025\\
146.01	0.00547275853697545\\
147.01	0.00547272107687336\\
148.01	0.00547268282577222\\
149.01	0.00547264376697546\\
150.01	0.00547260388343509\\
151.01	0.00547256315774372\\
152.01	0.00547252157212722\\
153.01	0.00547247910843723\\
154.01	0.00547243574814315\\
155.01	0.00547239147232418\\
156.01	0.00547234626166093\\
157.01	0.00547230009642765\\
158.01	0.00547225295648282\\
159.01	0.00547220482126146\\
160.01	0.00547215566976527\\
161.01	0.00547210548055443\\
162.01	0.00547205423173795\\
163.01	0.00547200190096412\\
164.01	0.005471948465411\\
165.01	0.00547189390177681\\
166.01	0.00547183818626952\\
167.01	0.00547178129459688\\
168.01	0.0054717232019556\\
169.01	0.00547166388302144\\
170.01	0.0054716033119376\\
171.01	0.00547154146230374\\
172.01	0.00547147830716507\\
173.01	0.00547141381900007\\
174.01	0.00547134796970932\\
175.01	0.00547128073060318\\
176.01	0.00547121207238956\\
177.01	0.00547114196516117\\
178.01	0.00547107037838286\\
179.01	0.00547099728087881\\
180.01	0.0054709226408186\\
181.01	0.00547084642570411\\
182.01	0.00547076860235531\\
183.01	0.00547068913689645\\
184.01	0.00547060799474112\\
185.01	0.00547052514057796\\
186.01	0.00547044053835526\\
187.01	0.00547035415126584\\
188.01	0.00547026594173106\\
189.01	0.00547017587138508\\
190.01	0.00547008390105858\\
191.01	0.00546998999076199\\
192.01	0.00546989409966854\\
193.01	0.00546979618609676\\
194.01	0.00546969620749292\\
195.01	0.00546959412041315\\
196.01	0.00546948988050503\\
197.01	0.0054693834424882\\
198.01	0.00546927476013626\\
199.01	0.00546916378625628\\
200.01	0.00546905047266936\\
201.01	0.00546893477018997\\
202.01	0.00546881662860582\\
203.01	0.00546869599665605\\
204.01	0.0054685728220099\\
205.01	0.0054684470512447\\
206.01	0.00546831862982342\\
207.01	0.00546818750207195\\
208.01	0.00546805361115537\\
209.01	0.00546791689905432\\
210.01	0.00546777730654086\\
211.01	0.00546763477315352\\
212.01	0.00546748923717205\\
213.01	0.00546734063559167\\
214.01	0.0054671889040968\\
215.01	0.00546703397703456\\
216.01	0.00546687578738701\\
217.01	0.0054667142667433\\
218.01	0.00546654934527175\\
219.01	0.00546638095169039\\
220.01	0.00546620901323807\\
221.01	0.00546603345564368\\
222.01	0.00546585420309565\\
223.01	0.00546567117821107\\
224.01	0.0054654843020036\\
225.01	0.00546529349385084\\
226.01	0.00546509867146126\\
227.01	0.00546489975084056\\
228.01	0.0054646966462573\\
229.01	0.00546448927020777\\
230.01	0.00546427753338016\\
231.01	0.0054640613446185\\
232.01	0.00546384061088514\\
233.01	0.00546361523722327\\
234.01	0.00546338512671819\\
235.01	0.00546315018045823\\
236.01	0.0054629102974949\\
237.01	0.00546266537480206\\
238.01	0.00546241530723413\\
239.01	0.00546215998748435\\
240.01	0.00546189930604151\\
241.01	0.00546163315114636\\
242.01	0.00546136140874677\\
243.01	0.00546108396245219\\
244.01	0.0054608006934879\\
245.01	0.00546051148064738\\
246.01	0.00546021620024458\\
247.01	0.00545991472606548\\
248.01	0.00545960692931768\\
249.01	0.0054592926785807\\
250.01	0.00545897183975395\\
251.01	0.00545864427600442\\
252.01	0.00545830984771375\\
253.01	0.00545796841242399\\
254.01	0.00545761982478237\\
255.01	0.00545726393648532\\
256.01	0.00545690059622154\\
257.01	0.00545652964961366\\
258.01	0.00545615093916022\\
259.01	0.00545576430417472\\
260.01	0.00545536958072562\\
261.01	0.00545496660157401\\
262.01	0.00545455519611091\\
263.01	0.00545413519029307\\
264.01	0.00545370640657917\\
265.01	0.00545326866386236\\
266.01	0.00545282177740459\\
267.01	0.0054523655587686\\
268.01	0.00545189981574863\\
269.01	0.00545142435230043\\
270.01	0.00545093896847087\\
271.01	0.00545044346032559\\
272.01	0.00544993761987627\\
273.01	0.00544942123500623\\
274.01	0.00544889408939605\\
275.01	0.00544835596244741\\
276.01	0.00544780662920622\\
277.01	0.00544724586028455\\
278.01	0.00544667342178206\\
279.01	0.00544608907520625\\
280.01	0.00544549257739157\\
281.01	0.00544488368041826\\
282.01	0.00544426213152959\\
283.01	0.00544362767304888\\
284.01	0.00544298004229537\\
285.01	0.00544231897149949\\
286.01	0.00544164418771732\\
287.01	0.0054409554127445\\
288.01	0.00544025236302916\\
289.01	0.00543953474958493\\
290.01	0.00543880227790276\\
291.01	0.00543805464786266\\
292.01	0.00543729155364504\\
293.01	0.00543651268364177\\
294.01	0.00543571772036664\\
295.01	0.00543490634036652\\
296.01	0.00543407821413152\\
297.01	0.00543323300600598\\
298.01	0.00543237037409957\\
299.01	0.00543148997019855\\
300.01	0.00543059143967702\\
301.01	0.00542967442140992\\
302.01	0.00542873854768631\\
303.01	0.00542778344412324\\
304.01	0.00542680872958117\\
305.01	0.00542581401608091\\
306.01	0.00542479890872185\\
307.01	0.0054237630056023\\
308.01	0.00542270589774171\\
309.01	0.00542162716900558\\
310.01	0.00542052639603238\\
311.01	0.00541940314816471\\
312.01	0.00541825698738258\\
313.01	0.00541708746824154\\
314.01	0.00541589413781383\\
315.01	0.00541467653563516\\
316.01	0.00541343419365607\\
317.01	0.00541216663619915\\
318.01	0.0054108733799222\\
319.01	0.00540955393378804\\
320.01	0.0054082077990423\\
321.01	0.0054068344691983\\
322.01	0.00540543343003169\\
323.01	0.00540400415958425\\
324.01	0.0054025461281779\\
325.01	0.0054010587984407\\
326.01	0.00539954162534438\\
327.01	0.00539799405625615\\
328.01	0.00539641553100435\\
329.01	0.00539480548195996\\
330.01	0.00539316333413541\\
331.01	0.00539148850530168\\
332.01	0.00538978040612542\\
333.01	0.0053880384403271\\
334.01	0.00538626200486351\\
335.01	0.00538445049013432\\
336.01	0.00538260328021608\\
337.01	0.00538071975312591\\
338.01	0.0053787992811161\\
339.01	0.00537684123100312\\
340.01	0.00537484496453274\\
341.01	0.00537280983878454\\
342.01	0.00537073520661919\\
343.01	0.00536862041717076\\
344.01	0.0053664648163883\\
345.01	0.00536426774762926\\
346.01	0.00536202855230991\\
347.01	0.00535974657061605\\
348.01	0.00535742114227824\\
349.01	0.00535505160741688\\
350.01	0.0053526373074612\\
351.01	0.00535017758614789\\
352.01	0.0053476717906051\\
353.01	0.00534511927252624\\
354.01	0.00534251938944185\\
355.01	0.00533987150609316\\
356.01	0.00533717499591671\\
357.01	0.0053344292426446\\
358.01	0.00533163364202873\\
359.01	0.00532878760369663\\
360.01	0.00532589055314491\\
361.01	0.00532294193387944\\
362.01	0.00531994120970843\\
363.01	0.00531688786719797\\
364.01	0.00531378141829418\\
365.01	0.00531062140312217\\
366.01	0.0053074073929661\\
367.01	0.00530413899343624\\
368.01	0.00530081584782815\\
369.01	0.00529743764067618\\
370.01	0.00529400410150313\\
371.01	0.00529051500876392\\
372.01	0.0052869701939804\\
373.01	0.00528336954605725\\
374.01	0.0052797130157684\\
375.01	0.00527600062039373\\
376.01	0.00527223244848314\\
377.01	0.00526840866471069\\
378.01	0.00526452951477912\\
379.01	0.00526059533031518\\
380.01	0.00525660653368822\\
381.01	0.00525256364266209\\
382.01	0.00524846727477379\\
383.01	0.00524431815130667\\
384.01	0.00524011710069707\\
385.01	0.00523586506118336\\
386.01	0.00523156308246611\\
387.01	0.00522721232610426\\
388.01	0.00522281406432562\\
389.01	0.00521836967687051\\
390.01	0.00521388064542686\\
391.01	0.00520934854514482\\
392.01	0.00520477503264535\\
393.01	0.00520016182985849\\
394.01	0.0051955107029452\\
395.01	0.0051908234354836\\
396.01	0.00518610179502812\\
397.01	0.00518134749210581\\
398.01	0.00517656213069732\\
399.01	0.00517174714929188\\
400.01	0.00516690375172526\\
401.01	0.00516203282725931\\
402.01	0.00515713485978329\\
403.01	0.00515220982670024\\
404.01	0.00514725708910238\\
405.01	0.00514227527638556\\
406.01	0.00513726217069819\\
407.01	0.00513221459982413\\
408.01	0.00512712835162468\\
409.01	0.00512199812947355\\
410.01	0.00511681757686981\\
411.01	0.00511157941144495\\
412.01	0.00510627572504719\\
413.01	0.00510089853303709\\
414.01	0.00509544215081351\\
415.01	0.00508990559050473\\
416.01	0.00508428859435006\\
417.01	0.00507859095793022\\
418.01	0.0050728125332189\\
419.01	0.00506695323167737\\
420.01	0.00506101302737739\\
421.01	0.00505499196013406\\
422.01	0.00504889013862831\\
423.01	0.00504270774349266\\
424.01	0.00503644503033259\\
425.01	0.00503010233265198\\
426.01	0.00502368006464182\\
427.01	0.0050171787237906\\
428.01	0.00501059889326493\\
429.01	0.00500394124400285\\
430.01	0.00499720653645617\\
431.01	0.00499039562190678\\
432.01	0.00498350944327549\\
433.01	0.00497654903533095\\
434.01	0.00496951552419494\\
435.01	0.00496241012603001\\
436.01	0.00495523414478367\\
437.01	0.00494798896884916\\
438.01	0.00494067606649338\\
439.01	0.0049332969798862\\
440.01	0.00492585331755582\\
441.01	0.00491834674508316\\
442.01	0.00491077897383633\\
443.01	0.00490315174754159\\
444.01	0.00489546682648113\\
445.01	0.00488772596910899\\
446.01	0.00487993091088409\\
447.01	0.00487208334013407\\
448.01	0.00486418487078903\\
449.01	0.00485623701186795\\
450.01	0.00484824113365184\\
451.01	0.00484019843056258\\
452.01	0.00483210988086774\\
453.01	0.00482397620346859\\
454.01	0.0048157978122043\\
455.01	0.00480757476831897\\
456.01	0.00479930673200534\\
457.01	0.00479099291426222\\
458.01	0.00478263203068424\\
459.01	0.00477422225924417\\
460.01	0.00476576120462847\\
461.01	0.00475724587222647\\
462.01	0.00474867265544567\\
463.01	0.00474003734060925\\
464.01	0.0047313351341798\\
465.01	0.00472256071731719\\
466.01	0.00471370833270468\\
467.01	0.00470477190790311\\
468.01	0.00469574521786123\\
469.01	0.00468662208610703\\
470.01	0.00467739661883009\\
471.01	0.00466806345757149\\
472.01	0.00465861802321931\\
473.01	0.00464905670308286\\
474.01	0.00463937683481917\\
475.01	0.00462957605184824\\
476.01	0.00461965177964405\\
477.01	0.00460960118805203\\
478.01	0.00459942117493768\\
479.01	0.00458910835096083\\
480.01	0.00457865902595065\\
481.01	0.00456806919740823\\
482.01	0.0045573345417218\\
483.01	0.00454645040872085\\
484.01	0.0045354118202245\\
485.01	0.00452421347324612\\
486.01	0.00451284974848832\\
487.01	0.00450131472469421\\
488.01	0.00448960219928953\\
489.01	0.00447770571555365\\
490.01	0.00446561859626462\\
491.01	0.00445333398337029\\
492.01	0.00444084488272215\\
493.01	0.00442814421226662\\
494.01	0.00441522485132207\\
495.01	0.00440207968771001\\
496.01	0.00438870165860932\\
497.01	0.00437508378018656\\
498.01	0.00436121916053661\\
499.01	0.00434710099056351\\
500.01	0.00433272250870228\\
501.01	0.00431807693864201\\
502.01	0.00430315740715636\\
503.01	0.00428795688086955\\
504.01	0.00427246815788195\\
505.01	0.00425668387727377\\
506.01	0.00424059652999392\\
507.01	0.00422419847041546\\
508.01	0.0042074819281634\\
509.01	0.00419043901973585\\
510.01	0.00417306175936144\\
511.01	0.00415534206847596\\
512.01	0.00413727178316525\\
513.01	0.00411884265894367\\
514.01	0.00410004637230826\\
515.01	0.00408087451867664\\
516.01	0.00406131860657965\\
517.01	0.00404137004835897\\
518.01	0.00402102014811284\\
519.01	0.00400026008820573\\
520.01	0.00397908091623087\\
521.01	0.00395747353471852\\
522.01	0.00393542869566511\\
523.01	0.00391293699939012\\
524.01	0.00388998889433544\\
525.01	0.00386657467628649\\
526.01	0.00384268448686651\\
527.01	0.00381830831127323\\
528.01	0.00379343597527271\\
529.01	0.00376805714151781\\
530.01	0.00374216130531905\\
531.01	0.0037157377900536\\
532.01	0.00368877574245122\\
533.01	0.00366126412803118\\
534.01	0.00363319172696691\\
535.01	0.00360454713061691\\
536.01	0.00357531873886363\\
537.01	0.00354549475825077\\
538.01	0.00351506320072506\\
539.01	0.0034840118827168\\
540.01	0.00345232842446228\\
541.01	0.00342000024963096\\
542.01	0.00338701458535218\\
543.01	0.00335335846274843\\
544.01	0.00331901871809061\\
545.01	0.00328398199469261\\
546.01	0.00324823474566422\\
547.01	0.00321176323763672\\
548.01	0.00317455355556667\\
549.01	0.00313659160871879\\
550.01	0.00309786313792477\\
551.01	0.00305835372422298\\
552.01	0.00301804879900678\\
553.01	0.00297693365584456\\
554.01	0.00293499346416344\\
555.01	0.00289221328500843\\
556.01	0.00284857808910646\\
557.01	0.00280407277748279\\
558.01	0.00275868220489704\\
559.01	0.00271239120638774\\
560.01	0.00266518462723621\\
561.01	0.00261704735668648\\
562.01	0.0025679643657854\\
563.01	0.00251792074973548\\
564.01	0.00246690177518275\\
565.01	0.00241489293289015\\
566.01	0.00236187999627078\\
567.01	0.00230784908627568\\
568.01	0.00225278674314507\\
569.01	0.00219668000553777\\
570.01	0.00213951649754955\\
571.01	0.00208128452410928\\
572.01	0.00202197317520257\\
573.01	0.00196157243930301\\
574.01	0.00190007332628658\\
575.01	0.00183746799995258\\
576.01	0.0017737499200606\\
577.01	0.00170891399350014\\
578.01	0.00164295673381588\\
579.01	0.00157587642778816\\
580.01	0.00150767330708272\\
581.01	0.00143834972208804\\
582.01	0.0013679103139034\\
583.01	0.00129636217895157\\
584.01	0.0012237150187842\\
585.01	0.00114998126521834\\
586.01	0.00107517616785102\\
587.01	0.000999317827082614\\
588.01	0.000922427150825461\\
589.01	0.000844527706822454\\
590.01	0.000765645434626115\\
591.01	0.000685808171383964\\
592.01	0.000605044933136151\\
593.01	0.000523384877725838\\
594.01	0.000440855855864791\\
595.01	0.00035748243240945\\
596.01	0.00027328322926679\\
597.01	0.000188267403047507\\
598.01	0.000102430022728519\\
599.01	3.18230442563731e-05\\
599.02	3.12770957175551e-05\\
599.03	3.07343620567866e-05\\
599.04	3.0194875217571e-05\\
599.05	2.96586674565693e-05\\
599.06	2.91257713466754e-05\\
599.07	2.85962197801391e-05\\
599.08	2.80700459716916e-05\\
599.09	2.75472834617447e-05\\
599.1	2.70279661195773e-05\\
599.11	2.65121281465865e-05\\
599.12	2.5999804079543e-05\\
599.13	2.54910287939124e-05\\
599.14	2.49858375071695e-05\\
599.15	2.4484265782181e-05\\
599.16	2.39863495305973e-05\\
599.17	2.34921250162837e-05\\
599.18	2.30016288588035e-05\\
599.19	2.25148980369013e-05\\
599.2	2.20319698920508e-05\\
599.21	2.1552882132023e-05\\
599.22	2.10776728344908e-05\\
599.23	2.06063804506721e-05\\
599.24	2.01390438090109e-05\\
599.25	1.96757021188858e-05\\
599.26	1.92163949743647e-05\\
599.27	1.87611623579855e-05\\
599.28	1.83100446445959e-05\\
599.29	1.78630826051952e-05\\
599.3	1.74203174108517e-05\\
599.31	1.69817906366353e-05\\
599.32	1.65475442655931e-05\\
599.33	1.61176206927693e-05\\
599.34	1.56920627292622e-05\\
599.35	1.52709136063186e-05\\
599.36	1.48542169794725e-05\\
599.37	1.4442016932719e-05\\
599.38	1.40343579827368e-05\\
599.39	1.36312859119591e-05\\
599.4	1.32328499381877e-05\\
599.41	1.28390997671621e-05\\
599.42	1.2450085597351e-05\\
599.43	1.20658581248094e-05\\
599.44	1.16864685480531e-05\\
599.45	1.13119685730082e-05\\
599.46	1.09424104179929e-05\\
599.47	1.05778468187639e-05\\
599.48	1.02183310335905e-05\\
599.49	9.86391684839293e-06\\
599.5	9.51465858193938e-06\\
599.51	9.17061109107289e-06\\
599.52	8.83182977599352e-06\\
599.53	8.4983705856221e-06\\
599.54	8.1702900229675e-06\\
599.55	7.84764515058753e-06\\
599.56	7.53049359608279e-06\\
599.57	7.21889355765649e-06\\
599.58	6.91290380971064e-06\\
599.59	6.61258370851688e-06\\
599.6	6.31799319793756e-06\\
599.61	6.02919281519031e-06\\
599.62	5.74624369668701e-06\\
599.63	5.46920758391981e-06\\
599.64	5.19814682940593e-06\\
599.65	4.93312440269685e-06\\
599.66	4.67420389643237e-06\\
599.67	4.42144953247646e-06\\
599.68	4.17492616808755e-06\\
599.69	3.9346993021671e-06\\
599.7	3.70083508156871e-06\\
599.71	3.47340030746116e-06\\
599.72	3.25246244175723e-06\\
599.73	3.03808961360161e-06\\
599.74	2.83035062593855e-06\\
599.75	2.62931496212288e-06\\
599.76	2.43505279260738e-06\\
599.77	2.24763498169606e-06\\
599.78	2.06713309435641e-06\\
599.79	1.89361940310974e-06\\
599.8	1.72716689498045e-06\\
599.81	1.56784927850782e-06\\
599.82	1.41574099085488e-06\\
599.83	1.27091720493813e-06\\
599.84	1.13345383668563e-06\\
599.85	1.00342755231589e-06\\
599.86	8.8091577571392e-07\\
599.87	7.65996695880136e-07\\
599.88	6.58749274441706e-07\\
599.89	5.59253253243699e-07\\
599.9	4.67589162013102e-07\\
599.91	3.83838326099145e-07\\
599.92	3.0808287429171e-07\\
599.93	2.40405746710845e-07\\
599.94	1.80890702777825e-07\\
599.95	1.2962232925906e-07\\
599.96	8.66860484019516e-08\\
599.97	5.21681261331924e-08\\
599.98	2.61556803542173e-08\\
599.99	8.73668930083393e-09\\
600	0\\
};
\addplot [color=red,solid,forget plot]
  table[row sep=crcr]{%
0.01	0.00524634702484204\\
1.01	0.00524634538167428\\
2.01	0.00524634370419065\\
3.01	0.00524634199167481\\
4.01	0.00524634024339566\\
5.01	0.00524633845860666\\
6.01	0.00524633663654615\\
7.01	0.00524633477643592\\
8.01	0.00524633287748189\\
9.01	0.0052463309388734\\
10.01	0.00524632895978309\\
11.01	0.00524632693936595\\
12.01	0.00524632487675974\\
13.01	0.00524632277108397\\
14.01	0.0052463206214401\\
15.01	0.00524631842691058\\
16.01	0.00524631618655892\\
17.01	0.00524631389942906\\
18.01	0.00524631156454481\\
19.01	0.00524630918091006\\
20.01	0.00524630674750763\\
21.01	0.00524630426329937\\
22.01	0.00524630172722527\\
23.01	0.00524629913820358\\
24.01	0.0052462964951298\\
25.01	0.00524629379687638\\
26.01	0.00524629104229229\\
27.01	0.00524628823020258\\
28.01	0.00524628535940793\\
29.01	0.00524628242868409\\
30.01	0.00524627943678111\\
31.01	0.00524627638242321\\
32.01	0.00524627326430783\\
33.01	0.00524627008110561\\
34.01	0.00524626683145922\\
35.01	0.00524626351398313\\
36.01	0.00524626012726317\\
37.01	0.00524625666985588\\
38.01	0.00524625314028707\\
39.01	0.00524624953705275\\
40.01	0.00524624585861699\\
41.01	0.00524624210341223\\
42.01	0.0052462382698383\\
43.01	0.0052462343562615\\
44.01	0.00524623036101445\\
45.01	0.00524622628239498\\
46.01	0.00524622211866539\\
47.01	0.00524621786805216\\
48.01	0.0052462135287446\\
49.01	0.0052462090988945\\
50.01	0.00524620457661531\\
51.01	0.00524619995998124\\
52.01	0.00524619524702633\\
53.01	0.00524619043574392\\
54.01	0.00524618552408546\\
55.01	0.00524618050996003\\
56.01	0.00524617539123337\\
57.01	0.00524617016572659\\
58.01	0.0052461648312158\\
59.01	0.00524615938543075\\
60.01	0.00524615382605415\\
61.01	0.00524614815072038\\
62.01	0.00524614235701494\\
63.01	0.0052461364424732\\
64.01	0.00524613040457932\\
65.01	0.0052461242407651\\
66.01	0.00524611794840926\\
67.01	0.00524611152483606\\
68.01	0.00524610496731396\\
69.01	0.00524609827305526\\
70.01	0.00524609143921402\\
71.01	0.0052460844628854\\
72.01	0.00524607734110442\\
73.01	0.00524607007084434\\
74.01	0.00524606264901603\\
75.01	0.00524605507246604\\
76.01	0.00524604733797553\\
77.01	0.00524603944225902\\
78.01	0.00524603138196313\\
79.01	0.00524602315366473\\
80.01	0.0052460147538698\\
81.01	0.00524600617901195\\
82.01	0.00524599742545109\\
83.01	0.00524598848947141\\
84.01	0.00524597936728023\\
85.01	0.00524597005500673\\
86.01	0.00524596054869953\\
87.01	0.00524595084432566\\
88.01	0.00524594093776854\\
89.01	0.00524593082482648\\
90.01	0.00524592050121083\\
91.01	0.0052459099625443\\
92.01	0.00524589920435879\\
93.01	0.00524588822209397\\
94.01	0.00524587701109538\\
95.01	0.00524586556661231\\
96.01	0.00524585388379547\\
97.01	0.00524584195769568\\
98.01	0.00524582978326137\\
99.01	0.00524581735533652\\
100.01	0.00524580466865875\\
101.01	0.00524579171785691\\
102.01	0.00524577849744897\\
103.01	0.00524576500183954\\
104.01	0.00524575122531796\\
105.01	0.00524573716205573\\
106.01	0.0052457228061036\\
107.01	0.00524570815139027\\
108.01	0.00524569319171874\\
109.01	0.00524567792076409\\
110.01	0.00524566233207136\\
111.01	0.00524564641905242\\
112.01	0.00524563017498314\\
113.01	0.00524561359300103\\
114.01	0.00524559666610192\\
115.01	0.00524557938713776\\
116.01	0.00524556174881299\\
117.01	0.00524554374368194\\
118.01	0.00524552536414564\\
119.01	0.00524550660244865\\
120.01	0.00524548745067598\\
121.01	0.00524546790074993\\
122.01	0.00524544794442652\\
123.01	0.00524542757329218\\
124.01	0.0052454067787606\\
125.01	0.00524538555206882\\
126.01	0.00524536388427362\\
127.01	0.0052453417662487\\
128.01	0.00524531918867973\\
129.01	0.00524529614206134\\
130.01	0.00524527261669292\\
131.01	0.00524524860267488\\
132.01	0.0052452240899045\\
133.01	0.00524519906807203\\
134.01	0.00524517352665601\\
135.01	0.0052451474549193\\
136.01	0.00524512084190489\\
137.01	0.00524509367643101\\
138.01	0.00524506594708726\\
139.01	0.00524503764222927\\
140.01	0.00524500874997448\\
141.01	0.00524497925819697\\
142.01	0.00524494915452292\\
143.01	0.00524491842632542\\
144.01	0.00524488706071916\\
145.01	0.00524485504455592\\
146.01	0.00524482236441842\\
147.01	0.00524478900661545\\
148.01	0.00524475495717632\\
149.01	0.00524472020184529\\
150.01	0.00524468472607508\\
151.01	0.00524464851502255\\
152.01	0.00524461155354148\\
153.01	0.0052445738261769\\
154.01	0.00524453531715894\\
155.01	0.00524449601039674\\
156.01	0.00524445588947163\\
157.01	0.0052444149376306\\
158.01	0.00524437313778031\\
159.01	0.00524433047247939\\
160.01	0.00524428692393208\\
161.01	0.00524424247398101\\
162.01	0.00524419710409988\\
163.01	0.00524415079538634\\
164.01	0.00524410352855446\\
165.01	0.00524405528392705\\
166.01	0.00524400604142793\\
167.01	0.00524395578057405\\
168.01	0.0052439044804675\\
169.01	0.00524385211978692\\
170.01	0.00524379867677972\\
171.01	0.00524374412925308\\
172.01	0.00524368845456543\\
173.01	0.00524363162961788\\
174.01	0.00524357363084466\\
175.01	0.00524351443420435\\
176.01	0.00524345401517028\\
177.01	0.00524339234872135\\
178.01	0.00524332940933197\\
179.01	0.00524326517096204\\
180.01	0.00524319960704773\\
181.01	0.00524313269049009\\
182.01	0.00524306439364537\\
183.01	0.00524299468831421\\
184.01	0.00524292354573063\\
185.01	0.00524285093655134\\
186.01	0.00524277683084407\\
187.01	0.0052427011980764\\
188.01	0.00524262400710397\\
189.01	0.00524254522615907\\
190.01	0.00524246482283786\\
191.01	0.00524238276408858\\
192.01	0.00524229901619896\\
193.01	0.00524221354478338\\
194.01	0.00524212631476993\\
195.01	0.00524203729038728\\
196.01	0.00524194643515123\\
197.01	0.00524185371185099\\
198.01	0.00524175908253533\\
199.01	0.00524166250849836\\
200.01	0.00524156395026513\\
201.01	0.00524146336757735\\
202.01	0.00524136071937776\\
203.01	0.00524125596379558\\
204.01	0.00524114905813088\\
205.01	0.00524103995883895\\
206.01	0.00524092862151417\\
207.01	0.00524081500087397\\
208.01	0.00524069905074223\\
209.01	0.00524058072403255\\
210.01	0.00524045997273106\\
211.01	0.00524033674787933\\
212.01	0.00524021099955651\\
213.01	0.0052400826768615\\
214.01	0.00523995172789479\\
215.01	0.00523981809973951\\
216.01	0.00523968173844319\\
217.01	0.00523954258899855\\
218.01	0.00523940059532382\\
219.01	0.00523925570024301\\
220.01	0.00523910784546591\\
221.01	0.00523895697156793\\
222.01	0.00523880301796943\\
223.01	0.00523864592291425\\
224.01	0.00523848562344873\\
225.01	0.00523832205540018\\
226.01	0.00523815515335461\\
227.01	0.00523798485063467\\
228.01	0.00523781107927687\\
229.01	0.00523763377000877\\
230.01	0.00523745285222539\\
231.01	0.00523726825396603\\
232.01	0.00523707990188999\\
233.01	0.00523688772125235\\
234.01	0.0052366916358793\\
235.01	0.00523649156814334\\
236.01	0.00523628743893765\\
237.01	0.00523607916765078\\
238.01	0.00523586667214054\\
239.01	0.00523564986870783\\
240.01	0.00523542867206966\\
241.01	0.00523520299533229\\
242.01	0.00523497274996426\\
243.01	0.00523473784576858\\
244.01	0.00523449819085457\\
245.01	0.00523425369160987\\
246.01	0.00523400425267175\\
247.01	0.00523374977689813\\
248.01	0.00523349016533854\\
249.01	0.00523322531720449\\
250.01	0.00523295512983983\\
251.01	0.00523267949869064\\
252.01	0.00523239831727478\\
253.01	0.0052321114771511\\
254.01	0.0052318188678893\\
255.01	0.00523152037703769\\
256.01	0.00523121589009264\\
257.01	0.00523090529046706\\
258.01	0.00523058845945782\\
259.01	0.00523026527621435\\
260.01	0.00522993561770634\\
261.01	0.00522959935869157\\
262.01	0.00522925637168265\\
263.01	0.00522890652691559\\
264.01	0.00522854969231577\\
265.01	0.00522818573346614\\
266.01	0.00522781451357373\\
267.01	0.0052274358934371\\
268.01	0.00522704973141289\\
269.01	0.00522665588338346\\
270.01	0.00522625420272365\\
271.01	0.00522584454026808\\
272.01	0.0052254267442784\\
273.01	0.00522500066041105\\
274.01	0.0052245661316846\\
275.01	0.00522412299844759\\
276.01	0.00522367109834689\\
277.01	0.00522321026629624\\
278.01	0.00522274033444507\\
279.01	0.00522226113214746\\
280.01	0.00522177248593254\\
281.01	0.00522127421947445\\
282.01	0.00522076615356294\\
283.01	0.00522024810607567\\
284.01	0.00521971989194984\\
285.01	0.00521918132315574\\
286.01	0.00521863220867095\\
287.01	0.0052180723544547\\
288.01	0.00521750156342459\\
289.01	0.00521691963543351\\
290.01	0.00521632636724819\\
291.01	0.00521572155252897\\
292.01	0.00521510498181107\\
293.01	0.00521447644248759\\
294.01	0.00521383571879398\\
295.01	0.00521318259179452\\
296.01	0.00521251683937033\\
297.01	0.00521183823621045\\
298.01	0.00521114655380373\\
299.01	0.00521044156043415\\
300.01	0.00520972302117921\\
301.01	0.00520899069790897\\
302.01	0.00520824434929016\\
303.01	0.00520748373079238\\
304.01	0.00520670859469767\\
305.01	0.00520591869011355\\
306.01	0.00520511376299\\
307.01	0.00520429355614022\\
308.01	0.00520345780926567\\
309.01	0.00520260625898526\\
310.01	0.00520173863886974\\
311.01	0.00520085467948036\\
312.01	0.00519995410841374\\
313.01	0.00519903665035063\\
314.01	0.00519810202711284\\
315.01	0.00519714995772436\\
316.01	0.00519618015847998\\
317.01	0.00519519234302001\\
318.01	0.00519418622241354\\
319.01	0.00519316150524767\\
320.01	0.00519211789772595\\
321.01	0.00519105510377478\\
322.01	0.00518997282515848\\
323.01	0.00518887076160365\\
324.01	0.00518774861093347\\
325.01	0.00518660606921215\\
326.01	0.00518544283089948\\
327.01	0.0051842585890171\\
328.01	0.00518305303532577\\
329.01	0.0051818258605159\\
330.01	0.00518057675440898\\
331.01	0.00517930540617347\\
332.01	0.00517801150455399\\
333.01	0.00517669473811432\\
334.01	0.00517535479549591\\
335.01	0.00517399136569088\\
336.01	0.00517260413833114\\
337.01	0.00517119280399358\\
338.01	0.00516975705452242\\
339.01	0.00516829658336771\\
340.01	0.00516681108594328\\
341.01	0.00516530026000083\\
342.01	0.00516376380602336\\
343.01	0.00516220142763736\\
344.01	0.00516061283204347\\
345.01	0.0051589977304664\\
346.01	0.00515735583862371\\
347.01	0.00515568687721378\\
348.01	0.00515399057242243\\
349.01	0.00515226665644796\\
350.01	0.00515051486804406\\
351.01	0.00514873495308003\\
352.01	0.00514692666511627\\
353.01	0.00514508976599623\\
354.01	0.0051432240264493\\
355.01	0.00514132922670752\\
356.01	0.00513940515712955\\
357.01	0.00513745161883129\\
358.01	0.00513546842431997\\
359.01	0.00513345539812616\\
360.01	0.00513141237743133\\
361.01	0.00512933921268395\\
362.01	0.00512723576819906\\
363.01	0.00512510192273381\\
364.01	0.00512293757003164\\
365.01	0.00512074261932499\\
366.01	0.00511851699578688\\
367.01	0.00511626064092056\\
368.01	0.00511397351287178\\
369.01	0.00511165558665115\\
370.01	0.00510930685424922\\
371.01	0.00510692732462547\\
372.01	0.00510451702355187\\
373.01	0.00510207599328724\\
374.01	0.00509960429205898\\
375.01	0.00509710199332412\\
376.01	0.00509456918477933\\
377.01	0.00509200596709101\\
378.01	0.00508941245230719\\
379.01	0.00508678876191842\\
380.01	0.00508413502452625\\
381.01	0.00508145137308095\\
382.01	0.00507873794164737\\
383.01	0.00507599486165573\\
384.01	0.00507322225759954\\
385.01	0.00507042024214\\
386.01	0.00506758891058257\\
387.01	0.00506472833469952\\
388.01	0.00506183855587652\\
389.01	0.00505891957757931\\
390.01	0.00505597135715017\\
391.01	0.00505299379696817\\
392.01	0.00504998673503601\\
393.01	0.00504694993509147\\
394.01	0.00504388307639118\\
395.01	0.00504078574336494\\
396.01	0.00503765741541047\\
397.01	0.00503449745717789\\
398.01	0.00503130510978448\\
399.01	0.00502807948350922\\
400.01	0.00502481955263314\\
401.01	0.00502152415321797\\
402.01	0.00501819198474338\\
403.01	0.00501482161663915\\
404.01	0.0050114115008428\\
405.01	0.00500795999154164\\
406.01	0.00500446537320313\\
407.01	0.00500092589777272\\
408.01	0.00499733983145012\\
409.01	0.00499370551062219\\
410.01	0.00499002140512592\\
411.01	0.00498628618482141\\
412.01	0.00498249878208409\\
413.01	0.00497865843779457\\
414.01	0.00497476469135668\\
415.01	0.00497081721153147\\
416.01	0.00496681568223637\\
417.01	0.00496275979613698\\
418.01	0.00495864925421746\\
419.01	0.00495448376522575\\
420.01	0.00495026304497878\\
421.01	0.00494598681551536\\
422.01	0.00494165480407954\\
423.01	0.00493726674192088\\
424.01	0.00493282236289352\\
425.01	0.00492832140183725\\
426.01	0.00492376359272367\\
427.01	0.00491914866654613\\
428.01	0.0049144763489378\\
429.01	0.00490974635749719\\
430.01	0.0049049583988014\\
431.01	0.00490011216509085\\
432.01	0.00489520733060504\\
433.01	0.00489024354755414\\
434.01	0.00488522044171163\\
435.01	0.00488013760761407\\
436.01	0.00487499460336023\\
437.01	0.00486979094500281\\
438.01	0.00486452610053278\\
439.01	0.00485919948346191\\
440.01	0.00485381044601682\\
441.01	0.00484835827196602\\
442.01	0.00484284216911223\\
443.01	0.00483726126149552\\
444.01	0.00483161458136402\\
445.01	0.00482590106098931\\
446.01	0.00482011952441736\\
447.01	0.0048142686792673\\
448.01	0.00480834710871583\\
449.01	0.00480235326382089\\
450.01	0.00479628545637138\\
451.01	0.00479014185246842\\
452.01	0.00478392046707218\\
453.01	0.00477761915976888\\
454.01	0.00477123563203365\\
455.01	0.00476476742627694\\
456.01	0.00475821192696775\\
457.01	0.00475156636411687\\
458.01	0.00474482781937998\\
459.01	0.00473799323499074\\
460.01	0.00473105942565689\\
461.01	0.00472402309344056\\
462.01	0.00471688084548702\\
463.01	0.00470962921426139\\
464.01	0.0047022646796878\\
465.01	0.00469478369226967\\
466.01	0.00468718269589198\\
467.01	0.00467945814860222\\
468.01	0.00467160653924801\\
469.01	0.00466362439749715\\
470.01	0.00465550829456611\\
471.01	0.00464725483211195\\
472.01	0.00463886061745229\\
473.01	0.00463032222497244\\
474.01	0.00462163614761239\\
475.01	0.00461279875818079\\
476.01	0.00460380629909003\\
477.01	0.00459465488160419\\
478.01	0.00458534048603782\\
479.01	0.00457585896272411\\
480.01	0.00456620603379292\\
481.01	0.00455637729577941\\
482.01	0.00454636822306456\\
483.01	0.00453617417212127\\
484.01	0.00452579038650785\\
485.01	0.00451521200251422\\
486.01	0.00450443405532398\\
487.01	0.0044934514855084\\
488.01	0.0044822591456219\\
489.01	0.00447085180662029\\
490.01	0.00445922416378062\\
491.01	0.00444737084176718\\
492.01	0.00443528639847168\\
493.01	0.00442296532726427\\
494.01	0.0044104020573332\\
495.01	0.00439759095188005\\
496.01	0.00438452630407865\\
497.01	0.00437120233091795\\
498.01	0.00435761316531334\\
499.01	0.00434375284719876\\
500.01	0.00432961531465032\\
501.01	0.00431519439636444\\
502.01	0.00430048380688169\\
503.01	0.00428547714512883\\
504.01	0.00427016789463693\\
505.01	0.00425454942377817\\
506.01	0.00423861498570822\\
507.01	0.00422235771793727\\
508.01	0.00420577064146486\\
509.01	0.00418884665942247\\
510.01	0.00417157855518518\\
511.01	0.00415395898993742\\
512.01	0.00413598049970358\\
513.01	0.00411763549188932\\
514.01	0.00409891624141305\\
515.01	0.00407981488654231\\
516.01	0.00406032342458033\\
517.01	0.00404043370756525\\
518.01	0.00402013743814512\\
519.01	0.00399942616576509\\
520.01	0.00397829128324327\\
521.01	0.00395672402371646\\
522.01	0.00393471545782475\\
523.01	0.00391225649092619\\
524.01	0.00388933786022006\\
525.01	0.00386595013177418\\
526.01	0.0038420836974821\\
527.01	0.003817728771985\\
528.01	0.00379287538959846\\
529.01	0.00376751340128798\\
530.01	0.00374163247173836\\
531.01	0.00371522207656307\\
532.01	0.00368827149969528\\
533.01	0.00366076983099453\\
534.01	0.00363270596409685\\
535.01	0.00360406859452609\\
536.01	0.00357484621807641\\
537.01	0.00354502712947658\\
538.01	0.00351459942135183\\
539.01	0.00348355098351638\\
540.01	0.00345186950264478\\
541.01	0.00341954246237786\\
542.01	0.00338655714392597\\
543.01	0.00335290062723492\\
544.01	0.00331855979278612\\
545.01	0.00328352132410759\\
546.01	0.00324777171107945\\
547.01	0.00321129725412398\\
548.01	0.00317408406938027\\
549.01	0.0031361180949736\\
550.01	0.00309738509850105\\
551.01	0.00305787068587207\\
552.01	0.00301756031165513\\
553.01	0.00297643929110013\\
554.01	0.00293449281402208\\
555.01	0.00289170596074975\\
556.01	0.00284806372036218\\
557.01	0.00280355101145738\\
558.01	0.00275815270571891\\
559.01	0.0027118536545707\\
560.01	0.00266463871923601\\
561.01	0.00261649280454222\\
562.01	0.00256740089684071\\
563.01	0.00251734810643931\\
564.01	0.00246631971497154\\
565.01	0.00241430122815276\\
566.01	0.00236127843439859\\
567.01	0.00230723746979893\\
568.01	0.00225216488995792\\
569.01	0.00219604774921472\\
570.01	0.00213887368775576\\
571.01	0.00208063102710875\\
572.01	0.00202130887446654\\
573.01	0.00196089723622006\\
574.01	0.00189938714097235\\
575.01	0.00183677077215304\\
576.01	0.00177304161013619\\
577.01	0.00170819458346989\\
578.01	0.00164222622842918\\
579.01	0.00157513485557815\\
580.01	0.0015069207213362\\
581.01	0.00143758620164401\\
582.01	0.0013671359636618\\
583.01	0.00129557712993561\\
584.01	0.0012229194275526\\
585.01	0.00114917531236232\\
586.01	0.00107436005523633\\
587.01	0.000998491773401186\\
588.01	0.000921591384903622\\
589.01	0.000843682457983908\\
590.01	0.00076479091922294\\
591.01	0.00068494457437703\\
592.01	0.000604172383317383\\
593.01	0.000522503414813216\\
594.01	0.000439965387250307\\
595.01	0.000356582676778022\\
596.01	0.000272373643602759\\
597.01	0.000187347088672941\\
598.01	0.000101497604931054\\
599.01	3.18230442563749e-05\\
599.02	3.12770957175551e-05\\
599.03	3.07343620567849e-05\\
599.04	3.0194875217571e-05\\
599.05	2.96586674565693e-05\\
599.06	2.91257713466771e-05\\
599.07	2.85962197801391e-05\\
599.08	2.80700459716916e-05\\
599.09	2.75472834617447e-05\\
599.1	2.70279661195791e-05\\
599.11	2.65121281465865e-05\\
599.12	2.59998040795448e-05\\
599.13	2.54910287939142e-05\\
599.14	2.49858375071712e-05\\
599.15	2.44842657821827e-05\\
599.16	2.39863495305973e-05\\
599.17	2.34921250162855e-05\\
599.18	2.30016288588035e-05\\
599.19	2.25148980369013e-05\\
599.2	2.20319698920526e-05\\
599.21	2.1552882132023e-05\\
599.22	2.10776728344891e-05\\
599.23	2.06063804506738e-05\\
599.24	2.01390438090109e-05\\
599.25	1.96757021188876e-05\\
599.26	1.92163949743647e-05\\
599.27	1.87611623579872e-05\\
599.28	1.83100446445959e-05\\
599.29	1.78630826051952e-05\\
599.3	1.74203174108517e-05\\
599.31	1.69817906366335e-05\\
599.32	1.65475442655914e-05\\
599.33	1.61176206927693e-05\\
599.34	1.56920627292639e-05\\
599.35	1.52709136063203e-05\\
599.36	1.48542169794742e-05\\
599.37	1.4442016932719e-05\\
599.38	1.40343579827385e-05\\
599.39	1.36312859119591e-05\\
599.4	1.32328499381877e-05\\
599.41	1.28390997671604e-05\\
599.42	1.24500855973528e-05\\
599.43	1.20658581248094e-05\\
599.44	1.16864685480531e-05\\
599.45	1.13119685730065e-05\\
599.46	1.09424104179929e-05\\
599.47	1.05778468187639e-05\\
599.48	1.02183310335888e-05\\
599.49	9.86391684839293e-06\\
599.5	9.51465858194112e-06\\
599.51	9.17061109107116e-06\\
599.52	8.83182977599525e-06\\
599.53	8.4983705856221e-06\\
599.54	8.1702900229675e-06\\
599.55	7.84764515058579e-06\\
599.56	7.53049359608453e-06\\
599.57	7.21889355765649e-06\\
599.58	6.91290380970891e-06\\
599.59	6.61258370851861e-06\\
599.6	6.31799319793756e-06\\
599.61	6.02919281519031e-06\\
599.62	5.74624369668701e-06\\
599.63	5.46920758391807e-06\\
599.64	5.19814682940593e-06\\
599.65	4.93312440269685e-06\\
599.66	4.67420389643411e-06\\
599.67	4.42144953247646e-06\\
599.68	4.17492616808582e-06\\
599.69	3.93469930216536e-06\\
599.7	3.70083508156871e-06\\
599.71	3.47340030746116e-06\\
599.72	3.25246244175549e-06\\
599.73	3.03808961360161e-06\\
599.74	2.83035062593855e-06\\
599.75	2.62931496212288e-06\\
599.76	2.43505279260911e-06\\
599.77	2.24763498169606e-06\\
599.78	2.06713309435641e-06\\
599.79	1.89361940310974e-06\\
599.8	1.72716689497872e-06\\
599.81	1.56784927850956e-06\\
599.82	1.41574099085315e-06\\
599.83	1.27091720493813e-06\\
599.84	1.13345383668736e-06\\
599.85	1.00342755231589e-06\\
599.86	8.8091577571392e-07\\
599.87	7.65996695880136e-07\\
599.88	6.58749274441706e-07\\
599.89	5.59253253243699e-07\\
599.9	4.67589162011367e-07\\
599.91	3.83838326099145e-07\\
599.92	3.08082874293444e-07\\
599.93	2.4040574671258e-07\\
599.94	1.80890702777825e-07\\
599.95	1.2962232925906e-07\\
599.96	8.66860484019516e-08\\
599.97	5.21681261349272e-08\\
599.98	2.61556803542173e-08\\
599.99	8.73668929909921e-09\\
600	0\\
};
\addplot [color=mycolor20,solid,forget plot]
  table[row sep=crcr]{%
0.01	0.00513241483564837\\
1.01	0.00513241334424749\\
2.01	0.00513241182204282\\
3.01	0.00513241026839975\\
4.01	0.00513240868267067\\
5.01	0.00513240706419493\\
6.01	0.00513240541229786\\
7.01	0.00513240372629167\\
8.01	0.00513240200547385\\
9.01	0.00513240024912786\\
10.01	0.00513239845652213\\
11.01	0.0051323966269105\\
12.01	0.00513239475953114\\
13.01	0.00513239285360692\\
14.01	0.00513239090834441\\
15.01	0.00513238892293442\\
16.01	0.00513238689655086\\
17.01	0.00513238482835078\\
18.01	0.00513238271747402\\
19.01	0.00513238056304285\\
20.01	0.00513237836416136\\
21.01	0.00513237611991541\\
22.01	0.0051323738293723\\
23.01	0.00513237149158016\\
24.01	0.00513236910556749\\
25.01	0.00513236667034322\\
26.01	0.00513236418489582\\
27.01	0.00513236164819297\\
28.01	0.0051323590591816\\
29.01	0.00513235641678669\\
30.01	0.00513235371991137\\
31.01	0.00513235096743661\\
32.01	0.0051323481582204\\
33.01	0.00513234529109728\\
34.01	0.00513234236487794\\
35.01	0.0051323393783491\\
36.01	0.00513233633027242\\
37.01	0.00513233321938416\\
38.01	0.00513233004439521\\
39.01	0.00513232680398992\\
40.01	0.00513232349682574\\
41.01	0.00513232012153271\\
42.01	0.00513231667671307\\
43.01	0.00513231316094059\\
44.01	0.00513230957275992\\
45.01	0.00513230591068593\\
46.01	0.00513230217320327\\
47.01	0.00513229835876573\\
48.01	0.00513229446579563\\
49.01	0.00513229049268308\\
50.01	0.00513228643778559\\
51.01	0.00513228229942684\\
52.01	0.00513227807589697\\
53.01	0.00513227376545062\\
54.01	0.00513226936630757\\
55.01	0.00513226487665105\\
56.01	0.00513226029462723\\
57.01	0.0051322556183449\\
58.01	0.00513225084587409\\
59.01	0.00513224597524579\\
60.01	0.00513224100445096\\
61.01	0.00513223593143984\\
62.01	0.0051322307541208\\
63.01	0.00513222547035982\\
64.01	0.00513222007797958\\
65.01	0.00513221457475855\\
66.01	0.00513220895843004\\
67.01	0.00513220322668146\\
68.01	0.00513219737715356\\
69.01	0.0051321914074386\\
70.01	0.00513218531508069\\
71.01	0.00513217909757361\\
72.01	0.00513217275236081\\
73.01	0.00513216627683364\\
74.01	0.00513215966833091\\
75.01	0.00513215292413726\\
76.01	0.00513214604148243\\
77.01	0.00513213901754015\\
78.01	0.00513213184942692\\
79.01	0.00513212453420093\\
80.01	0.00513211706886089\\
81.01	0.0051321094503449\\
82.01	0.00513210167552917\\
83.01	0.0051320937412265\\
84.01	0.00513208564418569\\
85.01	0.00513207738108971\\
86.01	0.00513206894855464\\
87.01	0.00513206034312811\\
88.01	0.00513205156128827\\
89.01	0.00513204259944226\\
90.01	0.00513203345392471\\
91.01	0.00513202412099615\\
92.01	0.00513201459684214\\
93.01	0.00513200487757133\\
94.01	0.0051319949592137\\
95.01	0.00513198483771948\\
96.01	0.00513197450895748\\
97.01	0.0051319639687133\\
98.01	0.00513195321268775\\
99.01	0.00513194223649546\\
100.01	0.00513193103566274\\
101.01	0.00513191960562597\\
102.01	0.00513190794172987\\
103.01	0.00513189603922598\\
104.01	0.00513188389327018\\
105.01	0.00513187149892124\\
106.01	0.00513185885113905\\
107.01	0.00513184594478222\\
108.01	0.00513183277460645\\
109.01	0.00513181933526249\\
110.01	0.00513180562129378\\
111.01	0.00513179162713463\\
112.01	0.00513177734710791\\
113.01	0.00513176277542302\\
114.01	0.00513174790617369\\
115.01	0.00513173273333546\\
116.01	0.00513171725076347\\
117.01	0.00513170145219042\\
118.01	0.00513168533122378\\
119.01	0.00513166888134339\\
120.01	0.00513165209589928\\
121.01	0.00513163496810875\\
122.01	0.00513161749105399\\
123.01	0.00513159965767963\\
124.01	0.00513158146078958\\
125.01	0.00513156289304461\\
126.01	0.00513154394695976\\
127.01	0.00513152461490098\\
128.01	0.00513150488908276\\
129.01	0.00513148476156499\\
130.01	0.00513146422424991\\
131.01	0.00513144326887912\\
132.01	0.00513142188703037\\
133.01	0.0051314000701145\\
134.01	0.00513137780937233\\
135.01	0.00513135509587135\\
136.01	0.00513133192050217\\
137.01	0.00513130827397524\\
138.01	0.00513128414681747\\
139.01	0.00513125952936845\\
140.01	0.0051312344117773\\
141.01	0.00513120878399871\\
142.01	0.00513118263578901\\
143.01	0.00513115595670288\\
144.01	0.00513112873608927\\
145.01	0.00513110096308722\\
146.01	0.00513107262662244\\
147.01	0.00513104371540271\\
148.01	0.00513101421791389\\
149.01	0.00513098412241576\\
150.01	0.00513095341693789\\
151.01	0.00513092208927484\\
152.01	0.0051308901269819\\
153.01	0.00513085751737094\\
154.01	0.00513082424750541\\
155.01	0.00513079030419562\\
156.01	0.00513075567399406\\
157.01	0.00513072034319089\\
158.01	0.00513068429780831\\
159.01	0.00513064752359607\\
160.01	0.00513061000602636\\
161.01	0.0051305717302882\\
162.01	0.00513053268128248\\
163.01	0.00513049284361671\\
164.01	0.00513045220159896\\
165.01	0.00513041073923299\\
166.01	0.00513036844021219\\
167.01	0.00513032528791378\\
168.01	0.00513028126539313\\
169.01	0.00513023635537786\\
170.01	0.00513019054026152\\
171.01	0.00513014380209775\\
172.01	0.00513009612259363\\
173.01	0.00513004748310344\\
174.01	0.00512999786462264\\
175.01	0.00512994724778056\\
176.01	0.00512989561283425\\
177.01	0.00512984293966133\\
178.01	0.00512978920775333\\
179.01	0.00512973439620871\\
180.01	0.00512967848372522\\
181.01	0.00512962144859345\\
182.01	0.00512956326868863\\
183.01	0.00512950392146356\\
184.01	0.00512944338394127\\
185.01	0.00512938163270644\\
186.01	0.00512931864389846\\
187.01	0.00512925439320302\\
188.01	0.00512918885584404\\
189.01	0.0051291220065753\\
190.01	0.00512905381967257\\
191.01	0.00512898426892433\\
192.01	0.00512891332762384\\
193.01	0.00512884096856025\\
194.01	0.00512876716400947\\
195.01	0.00512869188572549\\
196.01	0.00512861510493096\\
197.01	0.00512853679230802\\
198.01	0.00512845691798884\\
199.01	0.00512837545154612\\
200.01	0.00512829236198347\\
201.01	0.00512820761772528\\
202.01	0.00512812118660717\\
203.01	0.00512803303586553\\
204.01	0.00512794313212738\\
205.01	0.00512785144140001\\
206.01	0.00512775792906014\\
207.01	0.00512766255984398\\
208.01	0.00512756529783547\\
209.01	0.005127466106456\\
210.01	0.00512736494845286\\
211.01	0.00512726178588829\\
212.01	0.00512715658012785\\
213.01	0.00512704929182884\\
214.01	0.00512693988092886\\
215.01	0.00512682830663378\\
216.01	0.00512671452740574\\
217.01	0.00512659850095092\\
218.01	0.0051264801842074\\
219.01	0.00512635953333281\\
220.01	0.00512623650369184\\
221.01	0.00512611104984295\\
222.01	0.00512598312552607\\
223.01	0.00512585268364942\\
224.01	0.00512571967627659\\
225.01	0.00512558405461285\\
226.01	0.00512544576899208\\
227.01	0.00512530476886323\\
228.01	0.00512516100277649\\
229.01	0.00512501441836955\\
230.01	0.00512486496235384\\
231.01	0.0051247125804999\\
232.01	0.00512455721762403\\
233.01	0.00512439881757345\\
234.01	0.00512423732321212\\
235.01	0.00512407267640616\\
236.01	0.00512390481800906\\
237.01	0.00512373368784713\\
238.01	0.00512355922470507\\
239.01	0.00512338136630993\\
240.01	0.00512320004931743\\
241.01	0.00512301520929611\\
242.01	0.00512282678071231\\
243.01	0.00512263469691465\\
244.01	0.00512243889011949\\
245.01	0.00512223929139481\\
246.01	0.00512203583064516\\
247.01	0.00512182843659607\\
248.01	0.00512161703677903\\
249.01	0.00512140155751532\\
250.01	0.00512118192390098\\
251.01	0.00512095805979098\\
252.01	0.00512072988778389\\
253.01	0.00512049732920648\\
254.01	0.00512026030409788\\
255.01	0.00512001873119458\\
256.01	0.00511977252791487\\
257.01	0.00511952161034348\\
258.01	0.00511926589321649\\
259.01	0.00511900528990631\\
260.01	0.00511873971240635\\
261.01	0.00511846907131616\\
262.01	0.00511819327582711\\
263.01	0.00511791223370725\\
264.01	0.00511762585128748\\
265.01	0.00511733403344705\\
266.01	0.00511703668359962\\
267.01	0.00511673370367922\\
268.01	0.00511642499412766\\
269.01	0.00511611045388039\\
270.01	0.0051157899803543\\
271.01	0.00511546346943497\\
272.01	0.00511513081546445\\
273.01	0.00511479191122941\\
274.01	0.00511444664795006\\
275.01	0.00511409491526913\\
276.01	0.0051137366012415\\
277.01	0.00511337159232413\\
278.01	0.00511299977336665\\
279.01	0.00511262102760302\\
280.01	0.00511223523664279\\
281.01	0.00511184228046374\\
282.01	0.00511144203740521\\
283.01	0.00511103438416184\\
284.01	0.00511061919577816\\
285.01	0.00511019634564426\\
286.01	0.0051097657054915\\
287.01	0.0051093271453905\\
288.01	0.00510888053374866\\
289.01	0.00510842573730956\\
290.01	0.00510796262115269\\
291.01	0.0051074910486952\\
292.01	0.00510701088169404\\
293.01	0.00510652198024906\\
294.01	0.0051060242028085\\
295.01	0.00510551740617429\\
296.01	0.00510500144550995\\
297.01	0.00510447617434908\\
298.01	0.0051039414446056\\
299.01	0.00510339710658558\\
300.01	0.00510284300900028\\
301.01	0.00510227899898119\\
302.01	0.0051017049220967\\
303.01	0.00510112062237015\\
304.01	0.00510052594230007\\
305.01	0.00509992072288226\\
306.01	0.00509930480363314\\
307.01	0.00509867802261618\\
308.01	0.005098040216469\\
309.01	0.00509739122043389\\
310.01	0.00509673086838949\\
311.01	0.00509605899288485\\
312.01	0.00509537542517597\\
313.01	0.0050946799952644\\
314.01	0.00509397253193796\\
315.01	0.00509325286281404\\
316.01	0.00509252081438535\\
317.01	0.0050917762120676\\
318.01	0.00509101888024977\\
319.01	0.00509024864234706\\
320.01	0.00508946532085563\\
321.01	0.00508866873741051\\
322.01	0.00508785871284507\\
323.01	0.00508703506725312\\
324.01	0.00508619762005377\\
325.01	0.00508534619005734\\
326.01	0.00508448059553462\\
327.01	0.00508360065428733\\
328.01	0.00508270618372094\\
329.01	0.0050817970009187\\
330.01	0.00508087292271784\\
331.01	0.00507993376578699\\
332.01	0.00507897934670416\\
333.01	0.00507800948203693\\
334.01	0.00507702398842144\\
335.01	0.00507602268264328\\
336.01	0.00507500538171723\\
337.01	0.00507397190296693\\
338.01	0.00507292206410296\\
339.01	0.00507185568330056\\
340.01	0.00507077257927324\\
341.01	0.00506967257134569\\
342.01	0.00506855547952166\\
343.01	0.00506742112454792\\
344.01	0.00506626932797324\\
345.01	0.00506509991220067\\
346.01	0.00506391270053324\\
347.01	0.00506270751721066\\
348.01	0.0050614841874369\\
349.01	0.00506024253739589\\
350.01	0.00505898239425578\\
351.01	0.00505770358615814\\
352.01	0.00505640594219222\\
353.01	0.00505508929235078\\
354.01	0.00505375346746671\\
355.01	0.00505239829912762\\
356.01	0.00505102361956614\\
357.01	0.00504962926152432\\
358.01	0.00504821505808848\\
359.01	0.00504678084249199\\
360.01	0.00504532644788424\\
361.01	0.00504385170706125\\
362.01	0.00504235645215611\\
363.01	0.00504084051428484\\
364.01	0.00503930372314545\\
365.01	0.00503774590656626\\
366.01	0.00503616689000066\\
367.01	0.00503456649596352\\
368.01	0.00503294454340806\\
369.01	0.0050313008470378\\
370.01	0.00502963521655258\\
371.01	0.00502794745582441\\
372.01	0.00502623736200119\\
373.01	0.00502450472453784\\
374.01	0.00502274932415067\\
375.01	0.00502097093169802\\
376.01	0.00501916930698559\\
377.01	0.00501734419749892\\
378.01	0.00501549533706614\\
379.01	0.00501362244445593\\
380.01	0.00501172522191803\\
381.01	0.0050098033536758\\
382.01	0.00500785650438091\\
383.01	0.00500588431754847\\
384.01	0.00500388641398753\\
385.01	0.00500186239025184\\
386.01	0.00499981181713583\\
387.01	0.00499773423824771\\
388.01	0.00499562916869507\\
389.01	0.00499349609392439\\
390.01	0.00499133446875998\\
391.01	0.00498914371669356\\
392.01	0.00498692322948159\\
393.01	0.00498467236710779\\
394.01	0.0049823904581755\\
395.01	0.00498007680079112\\
396.01	0.00497773066400296\\
397.01	0.00497535128984978\\
398.01	0.00497293789606803\\
399.01	0.00497048967948952\\
400.01	0.00496800582013984\\
401.01	0.00496548548601524\\
402.01	0.00496292783847709\\
403.01	0.00496033203814928\\
404.01	0.00495769725113726\\
405.01	0.00495502265531409\\
406.01	0.00495230744632645\\
407.01	0.00494955084288241\\
408.01	0.00494675209079259\\
409.01	0.00494391046515843\\
410.01	0.00494102527007969\\
411.01	0.00493809583530152\\
412.01	0.0049351215094292\\
413.01	0.00493210164978179\\
414.01	0.00492903560998622\\
415.01	0.00492592273033172\\
416.01	0.00492276233481905\\
417.01	0.00491955373006865\\
418.01	0.00491629620420586\\
419.01	0.00491298902567613\\
420.01	0.00490963144199091\\
421.01	0.00490622267840105\\
422.01	0.00490276193649748\\
423.01	0.00489924839273843\\
424.01	0.00489568119690254\\
425.01	0.00489205947046875\\
426.01	0.0048883823049225\\
427.01	0.00488464875999325\\
428.01	0.00488085786182232\\
429.01	0.00487700860106707\\
430.01	0.00487309993094574\\
431.01	0.00486913076522753\\
432.01	0.00486509997617781\\
433.01	0.00486100639246517\\
434.01	0.00485684879704068\\
435.01	0.00485262592500365\\
436.01	0.00484833646146494\\
437.01	0.00484397903942548\\
438.01	0.00483955223768783\\
439.01	0.00483505457881895\\
440.01	0.00483048452718812\\
441.01	0.00482584048710082\\
442.01	0.00482112080105842\\
443.01	0.00481632374816491\\
444.01	0.0048114475427145\\
445.01	0.00480649033298484\\
446.01	0.00480145020026725\\
447.01	0.00479632515816302\\
448.01	0.00479111315217012\\
449.01	0.00478581205958852\\
450.01	0.00478041968976229\\
451.01	0.00477493378467433\\
452.01	0.00476935201990197\\
453.01	0.00476367200593159\\
454.01	0.00475789128981889\\
455.01	0.00475200735717024\\
456.01	0.00474601763439902\\
457.01	0.004739919491201\\
458.01	0.00473371024316399\\
459.01	0.0047273871544136\\
460.01	0.00472094744016993\\
461.01	0.00471438826907376\\
462.01	0.00470770676512423\\
463.01	0.00470090000905348\\
464.01	0.00469396503897045\\
465.01	0.00468689885010838\\
466.01	0.00467969839354542\\
467.01	0.00467236057381463\\
468.01	0.00466488224539955\\
469.01	0.00465726020821576\\
470.01	0.00464949120231721\\
471.01	0.00464157190222172\\
472.01	0.00463349891140976\\
473.01	0.00462526875768144\\
474.01	0.00461687789006565\\
475.01	0.00460832267757614\\
476.01	0.00459959940900615\\
477.01	0.00459070429294342\\
478.01	0.00458163345786526\\
479.01	0.00457238295229263\\
480.01	0.00456294874497674\\
481.01	0.00455332672509053\\
482.01	0.00454351270239028\\
483.01	0.00453350240731102\\
484.01	0.00452329149095487\\
485.01	0.0045128755249317\\
486.01	0.00450225000100838\\
487.01	0.00449141033052669\\
488.01	0.00448035184355369\\
489.01	0.00446906978773375\\
490.01	0.00445755932682209\\
491.01	0.00444581553889245\\
492.01	0.00443383341422656\\
493.01	0.00442160785291254\\
494.01	0.00440913366219645\\
495.01	0.00439640555365504\\
496.01	0.00438341814026985\\
497.01	0.00437016593349746\\
498.01	0.00435664334043017\\
499.01	0.00434284466113117\\
500.01	0.00432876408619353\\
501.01	0.00431439569452768\\
502.01	0.00429973345131089\\
503.01	0.00428477120597454\\
504.01	0.00426950269010674\\
505.01	0.00425392151523104\\
506.01	0.0042380211704564\\
507.01	0.00422179502000055\\
508.01	0.00420523630059349\\
509.01	0.00418833811877135\\
510.01	0.00417109344807316\\
511.01	0.00415349512615779\\
512.01	0.00413553585186063\\
513.01	0.00411720818221027\\
514.01	0.00409850452942619\\
515.01	0.00407941715791657\\
516.01	0.00405993818129196\\
517.01	0.00404005955940527\\
518.01	0.00401977309542275\\
519.01	0.0039990704329228\\
520.01	0.00397794305301479\\
521.01	0.00395638227146918\\
522.01	0.00393437923584688\\
523.01	0.00391192492263153\\
524.01	0.00388901013437162\\
525.01	0.00386562549684443\\
526.01	0.00384176145625882\\
527.01	0.00381740827651028\\
528.01	0.00379255603650427\\
529.01	0.00376719462756544\\
530.01	0.00374131375094899\\
531.01	0.00371490291547166\\
532.01	0.0036879514352818\\
533.01	0.00366044842778692\\
534.01	0.00363238281176114\\
535.01	0.00360374330565611\\
536.01	0.0035745184261425\\
537.01	0.00354469648691371\\
538.01	0.00351426559778857\\
539.01	0.00348321366415254\\
540.01	0.0034515283867838\\
541.01	0.00341919726211485\\
542.01	0.00338620758298481\\
543.01	0.00335254643994394\\
544.01	0.00331820072317836\\
545.01	0.00328315712513108\\
546.01	0.00324740214390159\\
547.01	0.0032109220875175\\
548.01	0.00317370307917983\\
549.01	0.00313573106359633\\
550.01	0.00309699181452752\\
551.01	0.00305747094368483\\
552.01	0.00301715391113351\\
553.01	0.00297602603736896\\
554.01	0.00293407251725179\\
555.01	0.00289127843600545\\
556.01	0.00284762878749973\\
557.01	0.0028031084950645\\
558.01	0.00275770243510144\\
559.01	0.00271139546378433\\
560.01	0.00266417244716499\\
561.01	0.0026160182950273\\
562.01	0.00256691799886021\\
563.01	0.00251685667434632\\
564.01	0.00246581960879106\\
565.01	0.00241379231394375\\
566.01	0.00236076058468414\\
567.01	0.00230671056407004\\
568.01	0.00225162881525418\\
569.01	0.00219550240078614\\
570.01	0.00213831896980923\\
571.01	0.00208006685364094\\
572.01	0.0020207351701836\\
573.01	0.00196031393754173\\
574.01	0.00189879419711535\\
575.01	0.0018361681462836\\
576.01	0.00177242928057545\\
577.01	0.00170757254492862\\
578.01	0.00164159449323789\\
579.01	0.00157449345486604\\
580.01	0.00150626970609533\\
581.01	0.00143692564359493\\
582.01	0.00136646595580921\\
583.01	0.00129489778666962\\
584.01	0.00122223088410795\\
585.01	0.00114847772339413\\
586.01	0.00107365359220303\\
587.01	0.000997776620360649\\
588.01	0.000920867732220132\\
589.01	0.000842950493310577\\
590.01	0.000764050814954753\\
591.01	0.000684196470559372\\
592.01	0.000603416364730805\\
593.01	0.000521739480625437\\
594.01	0.000439193411214457\\
595.01	0.000355802355441642\\
596.01	0.00027158442935076\\
597.01	0.000186548103627336\\
598.01	0.000100687530733697\\
599.01	3.18230442563749e-05\\
599.02	3.12770957175551e-05\\
599.03	3.07343620567866e-05\\
599.04	3.0194875217571e-05\\
599.05	2.96586674565693e-05\\
599.06	2.91257713466771e-05\\
599.07	2.85962197801391e-05\\
599.08	2.80700459716933e-05\\
599.09	2.75472834617464e-05\\
599.1	2.70279661195791e-05\\
599.11	2.65121281465865e-05\\
599.12	2.59998040795448e-05\\
599.13	2.54910287939124e-05\\
599.14	2.49858375071712e-05\\
599.15	2.44842657821827e-05\\
599.16	2.39863495305956e-05\\
599.17	2.34921250162837e-05\\
599.18	2.30016288588052e-05\\
599.19	2.25148980369013e-05\\
599.2	2.20319698920508e-05\\
599.21	2.1552882132023e-05\\
599.22	2.10776728344908e-05\\
599.23	2.06063804506721e-05\\
599.24	2.01390438090126e-05\\
599.25	1.96757021188858e-05\\
599.26	1.9216394974363e-05\\
599.27	1.87611623579872e-05\\
599.28	1.83100446445959e-05\\
599.29	1.78630826051952e-05\\
599.3	1.74203174108517e-05\\
599.31	1.69817906366353e-05\\
599.32	1.65475442655931e-05\\
599.33	1.61176206927693e-05\\
599.34	1.56920627292622e-05\\
599.35	1.52709136063203e-05\\
599.36	1.48542169794725e-05\\
599.37	1.44420169327208e-05\\
599.38	1.40343579827368e-05\\
599.39	1.36312859119591e-05\\
599.4	1.32328499381877e-05\\
599.41	1.28390997671604e-05\\
599.42	1.2450085597351e-05\\
599.43	1.20658581248111e-05\\
599.44	1.16864685480531e-05\\
599.45	1.13119685730082e-05\\
599.46	1.09424104179929e-05\\
599.47	1.05778468187639e-05\\
599.48	1.02183310335888e-05\\
599.49	9.86391684839293e-06\\
599.5	9.51465858193938e-06\\
599.51	9.17061109107289e-06\\
599.52	8.83182977599525e-06\\
599.53	8.4983705856221e-06\\
599.54	8.1702900229675e-06\\
599.55	7.84764515058753e-06\\
599.56	7.53049359608279e-06\\
599.57	7.21889355765649e-06\\
599.58	6.91290380971064e-06\\
599.59	6.61258370851688e-06\\
599.6	6.31799319793756e-06\\
599.61	6.02919281519031e-06\\
599.62	5.74624369668528e-06\\
599.63	5.46920758391981e-06\\
599.64	5.19814682940593e-06\\
599.65	4.93312440269685e-06\\
599.66	4.67420389643237e-06\\
599.67	4.42144953247819e-06\\
599.68	4.17492616808582e-06\\
599.69	3.9346993021671e-06\\
599.7	3.70083508156871e-06\\
599.71	3.47340030746289e-06\\
599.72	3.25246244175549e-06\\
599.73	3.03808961359987e-06\\
599.74	2.83035062593855e-06\\
599.75	2.62931496212288e-06\\
599.76	2.43505279260738e-06\\
599.77	2.24763498169606e-06\\
599.78	2.06713309435641e-06\\
599.79	1.89361940311147e-06\\
599.8	1.72716689497872e-06\\
599.81	1.56784927850956e-06\\
599.82	1.41574099085315e-06\\
599.83	1.27091720493813e-06\\
599.84	1.13345383668563e-06\\
599.85	1.00342755231415e-06\\
599.86	8.8091577571392e-07\\
599.87	7.65996695880136e-07\\
599.88	6.58749274441706e-07\\
599.89	5.59253253241965e-07\\
599.9	4.67589162013102e-07\\
599.91	3.83838326099145e-07\\
599.92	3.0808287429171e-07\\
599.93	2.40405746710845e-07\\
599.94	1.8089070277609e-07\\
599.95	1.29622329257326e-07\\
599.96	8.66860484002169e-08\\
599.97	5.21681261331924e-08\\
599.98	2.61556803542173e-08\\
599.99	8.73668930083393e-09\\
600	0\\
};
\addplot [color=mycolor21,solid,forget plot]
  table[row sep=crcr]{%
0.01	0.00507708887705905\\
1.01	0.00507708753621324\\
2.01	0.0050770861680176\\
3.01	0.00507708477191713\\
4.01	0.0050770833473457\\
5.01	0.00507708189372535\\
6.01	0.00507708041046706\\
7.01	0.00507707889696935\\
8.01	0.00507707735261923\\
9.01	0.00507707577679103\\
10.01	0.00507707416884656\\
11.01	0.00507707252813467\\
12.01	0.00507707085399132\\
13.01	0.00507706914573891\\
14.01	0.0050770674026864\\
15.01	0.00507706562412882\\
16.01	0.0050770638093468\\
17.01	0.00507706195760675\\
18.01	0.00507706006816042\\
19.01	0.00507705814024422\\
20.01	0.00507705617307949\\
21.01	0.00507705416587195\\
22.01	0.00507705211781113\\
23.01	0.00507705002807054\\
24.01	0.0050770478958069\\
25.01	0.00507704572016018\\
26.01	0.00507704350025313\\
27.01	0.00507704123519085\\
28.01	0.00507703892406036\\
29.01	0.00507703656593057\\
30.01	0.00507703415985188\\
31.01	0.0050770317048553\\
32.01	0.00507702919995268\\
33.01	0.00507702664413585\\
34.01	0.00507702403637687\\
35.01	0.00507702137562703\\
36.01	0.00507701866081639\\
37.01	0.00507701589085413\\
38.01	0.0050770130646273\\
39.01	0.0050770101810007\\
40.01	0.00507700723881643\\
41.01	0.00507700423689375\\
42.01	0.00507700117402806\\
43.01	0.00507699804899074\\
44.01	0.00507699486052866\\
45.01	0.00507699160736389\\
46.01	0.0050769882881928\\
47.01	0.00507698490168587\\
48.01	0.00507698144648692\\
49.01	0.00507697792121297\\
50.01	0.00507697432445339\\
51.01	0.00507697065476941\\
52.01	0.00507696691069357\\
53.01	0.00507696309072958\\
54.01	0.00507695919335097\\
55.01	0.00507695521700095\\
56.01	0.0050769511600919\\
57.01	0.00507694702100466\\
58.01	0.00507694279808788\\
59.01	0.00507693848965749\\
60.01	0.00507693409399573\\
61.01	0.00507692960935092\\
62.01	0.00507692503393644\\
63.01	0.00507692036593068\\
64.01	0.00507691560347542\\
65.01	0.00507691074467593\\
66.01	0.00507690578759975\\
67.01	0.0050769007302762\\
68.01	0.00507689557069561\\
69.01	0.00507689030680855\\
70.01	0.00507688493652499\\
71.01	0.00507687945771347\\
72.01	0.00507687386820064\\
73.01	0.00507686816577011\\
74.01	0.0050768623481615\\
75.01	0.00507685641306991\\
76.01	0.00507685035814514\\
77.01	0.00507684418099038\\
78.01	0.00507683787916161\\
79.01	0.00507683145016669\\
80.01	0.00507682489146434\\
81.01	0.0050768182004631\\
82.01	0.00507681137452044\\
83.01	0.00507680441094229\\
84.01	0.00507679730698098\\
85.01	0.00507679005983503\\
86.01	0.00507678266664797\\
87.01	0.00507677512450732\\
88.01	0.00507676743044301\\
89.01	0.00507675958142686\\
90.01	0.00507675157437126\\
91.01	0.00507674340612815\\
92.01	0.00507673507348772\\
93.01	0.00507672657317712\\
94.01	0.00507671790185935\\
95.01	0.00507670905613247\\
96.01	0.00507670003252765\\
97.01	0.00507669082750827\\
98.01	0.00507668143746858\\
99.01	0.00507667185873229\\
100.01	0.0050766620875516\\
101.01	0.00507665212010514\\
102.01	0.00507664195249747\\
103.01	0.00507663158075676\\
104.01	0.00507662100083391\\
105.01	0.0050766102086011\\
106.01	0.00507659919985008\\
107.01	0.00507658797029066\\
108.01	0.00507657651554915\\
109.01	0.00507656483116712\\
110.01	0.0050765529125992\\
111.01	0.00507654075521193\\
112.01	0.00507652835428203\\
113.01	0.00507651570499456\\
114.01	0.00507650280244127\\
115.01	0.00507648964161869\\
116.01	0.00507647621742685\\
117.01	0.00507646252466667\\
118.01	0.00507644855803879\\
119.01	0.00507643431214153\\
120.01	0.00507641978146884\\
121.01	0.00507640496040837\\
122.01	0.00507638984323938\\
123.01	0.00507637442413115\\
124.01	0.0050763586971405\\
125.01	0.00507634265621009\\
126.01	0.00507632629516564\\
127.01	0.0050763096077144\\
128.01	0.00507629258744297\\
129.01	0.00507627522781448\\
130.01	0.00507625752216674\\
131.01	0.00507623946371005\\
132.01	0.00507622104552452\\
133.01	0.00507620226055775\\
134.01	0.00507618310162244\\
135.01	0.00507616356139414\\
136.01	0.00507614363240815\\
137.01	0.00507612330705765\\
138.01	0.00507610257759048\\
139.01	0.005076081436107\\
140.01	0.00507605987455689\\
141.01	0.00507603788473684\\
142.01	0.00507601545828773\\
143.01	0.00507599258669128\\
144.01	0.00507596926126792\\
145.01	0.00507594547317327\\
146.01	0.00507592121339545\\
147.01	0.00507589647275183\\
148.01	0.00507587124188621\\
149.01	0.00507584551126559\\
150.01	0.00507581927117713\\
151.01	0.00507579251172424\\
152.01	0.00507576522282425\\
153.01	0.0050757373942047\\
154.01	0.00507570901539941\\
155.01	0.00507568007574582\\
156.01	0.00507565056438109\\
157.01	0.00507562047023848\\
158.01	0.00507558978204405\\
159.01	0.00507555848831241\\
160.01	0.00507552657734379\\
161.01	0.00507549403721958\\
162.01	0.00507546085579868\\
163.01	0.0050754270207135\\
164.01	0.00507539251936635\\
165.01	0.00507535733892502\\
166.01	0.00507532146631874\\
167.01	0.00507528488823392\\
168.01	0.00507524759111015\\
169.01	0.00507520956113589\\
170.01	0.00507517078424397\\
171.01	0.00507513124610713\\
172.01	0.00507509093213366\\
173.01	0.00507504982746286\\
174.01	0.00507500791695994\\
175.01	0.0050749651852123\\
176.01	0.00507492161652357\\
177.01	0.00507487719490985\\
178.01	0.005074831904094\\
179.01	0.0050747857275009\\
180.01	0.00507473864825254\\
181.01	0.00507469064916273\\
182.01	0.00507464171273208\\
183.01	0.00507459182114255\\
184.01	0.00507454095625181\\
185.01	0.00507448909958861\\
186.01	0.0050744362323465\\
187.01	0.00507438233537851\\
188.01	0.0050743273891917\\
189.01	0.005074271373941\\
190.01	0.00507421426942365\\
191.01	0.00507415605507348\\
192.01	0.0050740967099546\\
193.01	0.00507403621275545\\
194.01	0.00507397454178257\\
195.01	0.0050739116749546\\
196.01	0.00507384758979588\\
197.01	0.00507378226342988\\
198.01	0.00507371567257321\\
199.01	0.00507364779352859\\
200.01	0.00507357860217837\\
201.01	0.00507350807397796\\
202.01	0.00507343618394885\\
203.01	0.00507336290667201\\
204.01	0.00507328821628047\\
205.01	0.00507321208645276\\
206.01	0.00507313449040569\\
207.01	0.00507305540088659\\
208.01	0.00507297479016699\\
209.01	0.00507289263003433\\
210.01	0.00507280889178514\\
211.01	0.00507272354621716\\
212.01	0.0050726365636218\\
213.01	0.00507254791377652\\
214.01	0.00507245756593692\\
215.01	0.00507236548882902\\
216.01	0.00507227165064126\\
217.01	0.00507217601901641\\
218.01	0.00507207856104334\\
219.01	0.00507197924324943\\
220.01	0.00507187803159139\\
221.01	0.00507177489144776\\
222.01	0.0050716697876099\\
223.01	0.00507156268427412\\
224.01	0.00507145354503242\\
225.01	0.00507134233286413\\
226.01	0.00507122901012767\\
227.01	0.00507111353855088\\
228.01	0.00507099587922289\\
229.01	0.00507087599258482\\
230.01	0.00507075383842098\\
231.01	0.00507062937585008\\
232.01	0.00507050256331542\\
233.01	0.00507037335857621\\
234.01	0.00507024171869829\\
235.01	0.00507010760004478\\
236.01	0.00506997095826677\\
237.01	0.00506983174829409\\
238.01	0.00506968992432524\\
239.01	0.00506954543981895\\
240.01	0.00506939824748368\\
241.01	0.00506924829926839\\
242.01	0.00506909554635304\\
243.01	0.00506893993913896\\
244.01	0.00506878142723886\\
245.01	0.00506861995946761\\
246.01	0.00506845548383187\\
247.01	0.0050682879475211\\
248.01	0.00506811729689692\\
249.01	0.00506794347748404\\
250.01	0.00506776643395975\\
251.01	0.005067586110145\\
252.01	0.00506740244899392\\
253.01	0.00506721539258424\\
254.01	0.00506702488210727\\
255.01	0.00506683085785847\\
256.01	0.00506663325922733\\
257.01	0.00506643202468773\\
258.01	0.0050662270917882\\
259.01	0.00506601839714222\\
260.01	0.00506580587641847\\
261.01	0.00506558946433108\\
262.01	0.00506536909463067\\
263.01	0.00506514470009385\\
264.01	0.00506491621251443\\
265.01	0.00506468356269365\\
266.01	0.005064446680431\\
267.01	0.0050642054945148\\
268.01	0.00506395993271301\\
269.01	0.00506370992176399\\
270.01	0.00506345538736767\\
271.01	0.00506319625417625\\
272.01	0.00506293244578591\\
273.01	0.00506266388472717\\
274.01	0.00506239049245704\\
275.01	0.00506211218935009\\
276.01	0.00506182889469009\\
277.01	0.00506154052666153\\
278.01	0.0050612470023417\\
279.01	0.00506094823769231\\
280.01	0.00506064414755202\\
281.01	0.00506033464562809\\
282.01	0.00506001964448908\\
283.01	0.00505969905555732\\
284.01	0.00505937278910131\\
285.01	0.00505904075422845\\
286.01	0.00505870285887829\\
287.01	0.00505835900981521\\
288.01	0.00505800911262169\\
289.01	0.00505765307169174\\
290.01	0.00505729079022411\\
291.01	0.00505692217021622\\
292.01	0.00505654711245727\\
293.01	0.00505616551652256\\
294.01	0.00505577728076712\\
295.01	0.00505538230231984\\
296.01	0.00505498047707758\\
297.01	0.00505457169969901\\
298.01	0.0050541558635993\\
299.01	0.00505373286094394\\
300.01	0.00505330258264308\\
301.01	0.00505286491834604\\
302.01	0.00505241975643513\\
303.01	0.00505196698401995\\
304.01	0.00505150648693165\\
305.01	0.00505103814971642\\
306.01	0.00505056185562958\\
307.01	0.00505007748662879\\
308.01	0.00504958492336787\\
309.01	0.00504908404518902\\
310.01	0.00504857473011588\\
311.01	0.00504805685484564\\
312.01	0.00504753029474055\\
313.01	0.00504699492381921\\
314.01	0.0050464506147466\\
315.01	0.00504589723882434\\
316.01	0.00504533466597911\\
317.01	0.00504476276475078\\
318.01	0.00504418140227924\\
319.01	0.00504359044429042\\
320.01	0.00504298975508064\\
321.01	0.00504237919749954\\
322.01	0.00504175863293179\\
323.01	0.0050411279212771\\
324.01	0.00504048692092766\\
325.01	0.00503983548874477\\
326.01	0.00503917348003241\\
327.01	0.00503850074850792\\
328.01	0.00503781714627182\\
329.01	0.00503712252377266\\
330.01	0.00503641672977052\\
331.01	0.00503569961129585\\
332.01	0.00503497101360607\\
333.01	0.00503423078013689\\
334.01	0.00503347875245049\\
335.01	0.00503271477017883\\
336.01	0.00503193867096227\\
337.01	0.00503115029038325\\
338.01	0.00503034946189406\\
339.01	0.00502953601673909\\
340.01	0.00502870978387105\\
341.01	0.00502787058985985\\
342.01	0.00502701825879568\\
343.01	0.00502615261218312\\
344.01	0.00502527346882931\\
345.01	0.00502438064472175\\
346.01	0.00502347395289909\\
347.01	0.00502255320331151\\
348.01	0.005021618202672\\
349.01	0.00502066875429794\\
350.01	0.00501970465794117\\
351.01	0.0050187257096079\\
352.01	0.00501773170136676\\
353.01	0.00501672242114493\\
354.01	0.00501569765251243\\
355.01	0.00501465717445341\\
356.01	0.00501360076112476\\
357.01	0.00501252818160172\\
358.01	0.00501143919960925\\
359.01	0.00501033357324104\\
360.01	0.00500921105466373\\
361.01	0.00500807138980852\\
362.01	0.00500691431804838\\
363.01	0.00500573957186251\\
364.01	0.00500454687648826\\
365.01	0.0050033359495601\\
366.01	0.0050021065007374\\
367.01	0.00500085823132129\\
368.01	0.00499959083386242\\
369.01	0.00499830399176034\\
370.01	0.00499699737885614\\
371.01	0.00499567065902101\\
372.01	0.00499432348574205\\
373.01	0.00499295550170825\\
374.01	0.00499156633840018\\
375.01	0.00499015561568501\\
376.01	0.00498872294142227\\
377.01	0.00498726791108275\\
378.01	0.00498579010738665\\
379.01	0.00498428909996395\\
380.01	0.00498276444504269\\
381.01	0.00498121568517083\\
382.01	0.00497964234897718\\
383.01	0.00497804395097536\\
384.01	0.00497641999142021\\
385.01	0.00497476995621788\\
386.01	0.00497309331689844\\
387.01	0.00497138953065381\\
388.01	0.00496965804044674\\
389.01	0.00496789827519218\\
390.01	0.00496610965001477\\
391.01	0.00496429156658303\\
392.01	0.00496244341351421\\
393.01	0.00496056456685244\\
394.01	0.00495865439060511\\
395.01	0.00495671223733239\\
396.01	0.00495473744877186\\
397.01	0.00495272935648013\\
398.01	0.00495068728246698\\
399.01	0.00494861053979386\\
400.01	0.00494649843310342\\
401.01	0.00494435025904248\\
402.01	0.00494216530653852\\
403.01	0.00493994285688795\\
404.01	0.0049376821836139\\
405.01	0.00493538255205873\\
406.01	0.00493304321868069\\
407.01	0.0049306634300419\\
408.01	0.00492824242149082\\
409.01	0.00492577941557181\\
410.01	0.00492327362022339\\
411.01	0.00492072422686674\\
412.01	0.00491813040851883\\
413.01	0.00491549131808978\\
414.01	0.00491280608702397\\
415.01	0.00491007382433465\\
416.01	0.00490729361582734\\
417.01	0.00490446452333173\\
418.01	0.00490158558391598\\
419.01	0.00489865580908632\\
420.01	0.00489567418397143\\
421.01	0.00489263966649393\\
422.01	0.00488955118653039\\
423.01	0.00488640764506158\\
424.01	0.00488320791331547\\
425.01	0.00487995083190449\\
426.01	0.00487663520996096\\
427.01	0.00487325982427065\\
428.01	0.00486982341841193\\
429.01	0.00486632470189805\\
430.01	0.00486276234933108\\
431.01	0.00485913499956767\\
432.01	0.00485544125490201\\
433.01	0.00485167968026843\\
434.01	0.00484784880246914\\
435.01	0.00484394710942952\\
436.01	0.00483997304948575\\
437.01	0.00483592503070799\\
438.01	0.00483180142026234\\
439.01	0.0048276005438161\\
440.01	0.00482332068498685\\
441.01	0.00481896008484052\\
442.01	0.00481451694143632\\
443.01	0.00480998940942305\\
444.01	0.0048053755996833\\
445.01	0.00480067357902606\\
446.01	0.00479588136992385\\
447.01	0.00479099695029043\\
448.01	0.00478601825329393\\
449.01	0.00478094316719583\\
450.01	0.00477576953520854\\
451.01	0.00477049515535841\\
452.01	0.00476511778034158\\
453.01	0.0047596351173582\\
454.01	0.00475404482790781\\
455.01	0.00474834452752741\\
456.01	0.00474253178545619\\
457.01	0.00473660412420473\\
458.01	0.00473055901901403\\
459.01	0.00472439389718501\\
460.01	0.00471810613726825\\
461.01	0.0047116930681009\\
462.01	0.00470515196768929\\
463.01	0.00469848006193911\\
464.01	0.00469167452324305\\
465.01	0.00468473246894722\\
466.01	0.00467765095972418\\
467.01	0.00467042699789038\\
468.01	0.00466305752571401\\
469.01	0.00465553942376162\\
470.01	0.00464786950933107\\
471.01	0.00464004453501251\\
472.01	0.0046320611874001\\
473.01	0.00462391608595133\\
474.01	0.00461560578196254\\
475.01	0.00460712675759189\\
476.01	0.0045984754248689\\
477.01	0.00458964812466511\\
478.01	0.00458064112561716\\
479.01	0.00457145062299611\\
480.01	0.00456207273751556\\
481.01	0.00455250351407303\\
482.01	0.00454273892041895\\
483.01	0.00453277484574797\\
484.01	0.00452260709921037\\
485.01	0.00451223140833936\\
486.01	0.00450164341739547\\
487.01	0.00449083868562842\\
488.01	0.00447981268545985\\
489.01	0.00446856080059189\\
490.01	0.00445707832404862\\
491.01	0.00444536045615936\\
492.01	0.00443340230249307\\
493.01	0.00442119887175426\\
494.01	0.00440874507365103\\
495.01	0.00439603571674383\\
496.01	0.0043830655062815\\
497.01	0.00436982904202923\\
498.01	0.00435632081608763\\
499.01	0.00434253521069777\\
500.01	0.0043284664960249\\
501.01	0.00431410882791098\\
502.01	0.00429945624558325\\
503.01	0.00428450266931315\\
504.01	0.0042692418980233\\
505.01	0.00425366760684348\\
506.01	0.00423777334461912\\
507.01	0.004221552531376\\
508.01	0.00420499845574363\\
509.01	0.00418810427234193\\
510.01	0.00417086299913536\\
511.01	0.00415326751475782\\
512.01	0.0041353105558127\\
513.01	0.00411698471414996\\
514.01	0.00409828243412553\\
515.01	0.00407919600984361\\
516.01	0.00405971758238504\\
517.01	0.00403983913702328\\
518.01	0.00401955250043088\\
519.01	0.0039988493378783\\
520.01	0.00397772115042922\\
521.01	0.00395615927213539\\
522.01	0.00393415486723955\\
523.01	0.00391169892739118\\
524.01	0.00388878226888378\\
525.01	0.00386539552992501\\
526.01	0.00384152916794622\\
527.01	0.00381717345696531\\
528.01	0.00379231848501426\\
529.01	0.00376695415164407\\
530.01	0.00374107016552353\\
531.01	0.00371465604214729\\
532.01	0.00368770110167236\\
533.01	0.00366019446690445\\
534.01	0.00363212506145671\\
535.01	0.0036034816081067\\
536.01	0.00357425262738278\\
537.01	0.00354442643640999\\
538.01	0.00351399114805384\\
539.01	0.0034829346704015\\
540.01	0.00345124470662508\\
541.01	0.00341890875527855\\
542.01	0.00338591411108143\\
543.01	0.00335224786625271\\
544.01	0.00331789691246243\\
545.01	0.00328284794347674\\
546.01	0.00324708745858068\\
547.01	0.00321060176687146\\
548.01	0.00317337699252519\\
549.01	0.00313539908115085\\
550.01	0.00309665380735802\\
551.01	0.00305712678367617\\
552.01	0.00301680347097966\\
553.01	0.00297566919058672\\
554.01	0.00293370913821819\\
555.01	0.00289090840001998\\
556.01	0.00284725197087291\\
557.01	0.00280272477523499\\
558.01	0.00275731169078333\\
559.01	0.00271099757514768\\
560.01	0.0026637672960521\\
561.01	0.00261560576520825\\
562.01	0.00256649797633006\\
563.01	0.00251642904766814\\
564.01	0.00246538426948843\\
565.01	0.00241334915694547\\
566.01	0.0023603095088256\\
567.01	0.00230625147265386\\
568.01	0.00225116161667357\\
569.01	0.00219502700921337\\
570.01	0.00213783530595022\\
571.01	0.00207957484555631\\
572.01	0.00202023475417473\\
573.01	0.00195980505909759\\
574.01	0.00189827681191261\\
575.01	0.00183564222122825\\
576.01	0.00177189479486902\\
577.01	0.00170702949113364\\
578.01	0.00164104287830919\\
579.01	0.00157393330110091\\
580.01	0.00150570105194178\\
581.01	0.00143634854423568\\
582.01	0.0013658804834154\\
583.01	0.00129430403018606\\
584.01	0.00122162894839109\\
585.01	0.00114786772747427\\
586.01	0.00107303566637796\\
587.01	0.000997150901747018\\
588.01	0.000920234358289583\\
589.01	0.00084230959280966\\
590.01	0.000763402495449801\\
591.01	0.000683540801648747\\
592.01	0.000602753355717998\\
593.01	0.000521069051135661\\
594.01	0.000438515352848756\\
595.01	0.000355116282076462\\
596.01	0.000270889713084794\\
597.01	0.000185843792620275\\
598.01	9.99722432115345e-05\\
599.01	3.18230442563731e-05\\
599.02	3.12770957175551e-05\\
599.03	3.07343620567866e-05\\
599.04	3.0194875217571e-05\\
599.05	2.96586674565693e-05\\
599.06	2.91257713466771e-05\\
599.07	2.85962197801391e-05\\
599.08	2.80700459716916e-05\\
599.09	2.75472834617447e-05\\
599.1	2.70279661195773e-05\\
599.11	2.65121281465865e-05\\
599.12	2.59998040795448e-05\\
599.13	2.54910287939142e-05\\
599.14	2.49858375071729e-05\\
599.15	2.44842657821827e-05\\
599.16	2.39863495305973e-05\\
599.17	2.34921250162855e-05\\
599.18	2.30016288588035e-05\\
599.19	2.2514898036903e-05\\
599.2	2.20319698920508e-05\\
599.21	2.1552882132023e-05\\
599.22	2.10776728344908e-05\\
599.23	2.06063804506721e-05\\
599.24	2.01390438090109e-05\\
599.25	1.96757021188858e-05\\
599.26	1.92163949743647e-05\\
599.27	1.87611623579855e-05\\
599.28	1.83100446445959e-05\\
599.29	1.78630826051952e-05\\
599.3	1.74203174108517e-05\\
599.31	1.69817906366353e-05\\
599.32	1.65475442655931e-05\\
599.33	1.61176206927693e-05\\
599.34	1.56920627292639e-05\\
599.35	1.52709136063203e-05\\
599.36	1.48542169794742e-05\\
599.37	1.4442016932719e-05\\
599.38	1.40343579827368e-05\\
599.39	1.36312859119591e-05\\
599.4	1.32328499381877e-05\\
599.41	1.28390997671604e-05\\
599.42	1.2450085597351e-05\\
599.43	1.20658581248094e-05\\
599.44	1.16864685480531e-05\\
599.45	1.13119685730065e-05\\
599.46	1.09424104179929e-05\\
599.47	1.05778468187639e-05\\
599.48	1.02183310335888e-05\\
599.49	9.86391684839293e-06\\
599.5	9.51465858194112e-06\\
599.51	9.17061109107116e-06\\
599.52	8.83182977599352e-06\\
599.53	8.49837058562383e-06\\
599.54	8.17029002296576e-06\\
599.55	7.84764515058753e-06\\
599.56	7.53049359608453e-06\\
599.57	7.21889355765649e-06\\
599.58	6.91290380971064e-06\\
599.59	6.61258370851688e-06\\
599.6	6.31799319793756e-06\\
599.61	6.02919281518857e-06\\
599.62	5.74624369668701e-06\\
599.63	5.46920758391981e-06\\
599.64	5.19814682940767e-06\\
599.65	4.93312440269685e-06\\
599.66	4.67420389643411e-06\\
599.67	4.42144953247646e-06\\
599.68	4.17492616808755e-06\\
599.69	3.9346993021671e-06\\
599.7	3.70083508157044e-06\\
599.71	3.47340030746116e-06\\
599.72	3.25246244175723e-06\\
599.73	3.03808961360161e-06\\
599.74	2.83035062593855e-06\\
599.75	2.62931496212288e-06\\
599.76	2.43505279260738e-06\\
599.77	2.24763498169432e-06\\
599.78	2.06713309435815e-06\\
599.79	1.89361940310974e-06\\
599.8	1.72716689498045e-06\\
599.81	1.56784927850782e-06\\
599.82	1.41574099085315e-06\\
599.83	1.27091720493987e-06\\
599.84	1.13345383668563e-06\\
599.85	1.00342755231589e-06\\
599.86	8.8091577571392e-07\\
599.87	7.65996695880136e-07\\
599.88	6.58749274441706e-07\\
599.89	5.59253253243699e-07\\
599.9	4.67589162011367e-07\\
599.91	3.83838326099145e-07\\
599.92	3.0808287429171e-07\\
599.93	2.4040574671258e-07\\
599.94	1.80890702777825e-07\\
599.95	1.2962232925906e-07\\
599.96	8.66860484019516e-08\\
599.97	5.21681261331924e-08\\
599.98	2.61556803542173e-08\\
599.99	8.73668930083393e-09\\
600	0\\
};
\addplot [color=black!20!mycolor21,solid,forget plot]
  table[row sep=crcr]{%
0.01	0.00505041622436337\\
1.01	0.00505041502687961\\
2.01	0.00505041380528635\\
3.01	0.00505041255910172\\
4.01	0.00505041128783399\\
5.01	0.00505040999098198\\
6.01	0.00505040866803445\\
7.01	0.0050504073184702\\
8.01	0.00505040594175756\\
9.01	0.0050504045373543\\
10.01	0.00505040310470765\\
11.01	0.00505040164325379\\
12.01	0.00505040015241783\\
13.01	0.00505039863161348\\
14.01	0.00505039708024294\\
15.01	0.00505039549769652\\
16.01	0.00505039388335266\\
17.01	0.00505039223657732\\
18.01	0.00505039055672402\\
19.01	0.00505038884313369\\
20.01	0.00505038709513408\\
21.01	0.00505038531203984\\
22.01	0.00505038349315189\\
23.01	0.00505038163775768\\
24.01	0.00505037974513043\\
25.01	0.00505037781452916\\
26.01	0.00505037584519808\\
27.01	0.00505037383636694\\
28.01	0.00505037178725006\\
29.01	0.0050503696970464\\
30.01	0.00505036756493908\\
31.01	0.00505036539009538\\
32.01	0.00505036317166596\\
33.01	0.00505036090878514\\
34.01	0.00505035860056998\\
35.01	0.00505035624612024\\
36.01	0.00505035384451818\\
37.01	0.00505035139482785\\
38.01	0.0050503488960952\\
39.01	0.00505034634734725\\
40.01	0.00505034374759241\\
41.01	0.00505034109581914\\
42.01	0.00505033839099619\\
43.01	0.0050503356320726\\
44.01	0.00505033281797631\\
45.01	0.00505032994761456\\
46.01	0.00505032701987323\\
47.01	0.0050503240336164\\
48.01	0.00505032098768594\\
49.01	0.00505031788090126\\
50.01	0.00505031471205845\\
51.01	0.00505031147993047\\
52.01	0.00505030818326621\\
53.01	0.00505030482078999\\
54.01	0.0050503013912013\\
55.01	0.00505029789317477\\
56.01	0.00505029432535876\\
57.01	0.00505029068637558\\
58.01	0.00505028697482046\\
59.01	0.00505028318926152\\
60.01	0.00505027932823905\\
61.01	0.005050275390265\\
62.01	0.00505027137382239\\
63.01	0.00505026727736469\\
64.01	0.00505026309931556\\
65.01	0.00505025883806804\\
66.01	0.00505025449198377\\
67.01	0.00505025005939313\\
68.01	0.00505024553859374\\
69.01	0.0050502409278504\\
70.01	0.00505023622539427\\
71.01	0.00505023142942263\\
72.01	0.00505022653809739\\
73.01	0.00505022154954516\\
74.01	0.00505021646185647\\
75.01	0.00505021127308494\\
76.01	0.00505020598124666\\
77.01	0.00505020058431944\\
78.01	0.00505019508024193\\
79.01	0.00505018946691318\\
80.01	0.00505018374219198\\
81.01	0.00505017790389557\\
82.01	0.00505017194979935\\
83.01	0.00505016587763594\\
84.01	0.00505015968509456\\
85.01	0.00505015336981967\\
86.01	0.00505014692941063\\
87.01	0.00505014036142088\\
88.01	0.00505013366335681\\
89.01	0.00505012683267706\\
90.01	0.00505011986679147\\
91.01	0.00505011276306055\\
92.01	0.00505010551879388\\
93.01	0.00505009813124976\\
94.01	0.00505009059763433\\
95.01	0.0050500829150999\\
96.01	0.00505007508074471\\
97.01	0.00505006709161143\\
98.01	0.00505005894468656\\
99.01	0.00505005063689899\\
100.01	0.00505004216511883\\
101.01	0.00505003352615715\\
102.01	0.00505002471676396\\
103.01	0.00505001573362758\\
104.01	0.00505000657337335\\
105.01	0.00504999723256242\\
106.01	0.00504998770769082\\
107.01	0.00504997799518791\\
108.01	0.00504996809141548\\
109.01	0.00504995799266634\\
110.01	0.00504994769516324\\
111.01	0.00504993719505728\\
112.01	0.00504992648842675\\
113.01	0.00504991557127588\\
114.01	0.00504990443953338\\
115.01	0.00504989308905139\\
116.01	0.00504988151560328\\
117.01	0.0050498697148834\\
118.01	0.00504985768250443\\
119.01	0.00504984541399691\\
120.01	0.00504983290480681\\
121.01	0.00504982015029495\\
122.01	0.00504980714573495\\
123.01	0.00504979388631146\\
124.01	0.00504978036711917\\
125.01	0.0050497665831604\\
126.01	0.0050497525293443\\
127.01	0.00504973820048466\\
128.01	0.00504972359129796\\
129.01	0.00504970869640226\\
130.01	0.00504969351031514\\
131.01	0.00504967802745166\\
132.01	0.00504966224212308\\
133.01	0.00504964614853445\\
134.01	0.00504962974078308\\
135.01	0.00504961301285629\\
136.01	0.0050495959586299\\
137.01	0.00504957857186603\\
138.01	0.00504956084621088\\
139.01	0.00504954277519308\\
140.01	0.00504952435222127\\
141.01	0.00504950557058214\\
142.01	0.00504948642343829\\
143.01	0.00504946690382603\\
144.01	0.00504944700465291\\
145.01	0.00504942671869607\\
146.01	0.00504940603859908\\
147.01	0.00504938495687064\\
148.01	0.00504936346588136\\
149.01	0.00504934155786169\\
150.01	0.00504931922489933\\
151.01	0.00504929645893711\\
152.01	0.00504927325177016\\
153.01	0.00504924959504327\\
154.01	0.00504922548024874\\
155.01	0.00504920089872304\\
156.01	0.00504917584164474\\
157.01	0.00504915030003159\\
158.01	0.00504912426473755\\
159.01	0.00504909772645034\\
160.01	0.0050490706756881\\
161.01	0.00504904310279709\\
162.01	0.00504901499794829\\
163.01	0.00504898635113455\\
164.01	0.00504895715216749\\
165.01	0.00504892739067457\\
166.01	0.0050488970560957\\
167.01	0.00504886613768058\\
168.01	0.00504883462448481\\
169.01	0.00504880250536705\\
170.01	0.00504876976898552\\
171.01	0.0050487364037947\\
172.01	0.00504870239804215\\
173.01	0.00504866773976427\\
174.01	0.00504863241678364\\
175.01	0.00504859641670479\\
176.01	0.00504855972691134\\
177.01	0.00504852233456132\\
178.01	0.00504848422658404\\
179.01	0.00504844538967644\\
180.01	0.00504840581029883\\
181.01	0.00504836547467099\\
182.01	0.00504832436876866\\
183.01	0.00504828247831899\\
184.01	0.00504823978879698\\
185.01	0.0050481962854207\\
186.01	0.00504815195314776\\
187.01	0.00504810677667052\\
188.01	0.00504806074041221\\
189.01	0.00504801382852224\\
190.01	0.00504796602487202\\
191.01	0.00504791731305022\\
192.01	0.00504786767635818\\
193.01	0.00504781709780566\\
194.01	0.00504776556010587\\
195.01	0.00504771304567071\\
196.01	0.00504765953660593\\
197.01	0.00504760501470664\\
198.01	0.00504754946145155\\
199.01	0.005047492857999\\
200.01	0.0050474351851811\\
201.01	0.00504737642349863\\
202.01	0.00504731655311634\\
203.01	0.00504725555385704\\
204.01	0.00504719340519661\\
205.01	0.00504713008625858\\
206.01	0.00504706557580845\\
207.01	0.00504699985224826\\
208.01	0.00504693289361077\\
209.01	0.00504686467755405\\
210.01	0.00504679518135547\\
211.01	0.00504672438190567\\
212.01	0.00504665225570302\\
213.01	0.00504657877884728\\
214.01	0.00504650392703367\\
215.01	0.00504642767554666\\
216.01	0.00504634999925358\\
217.01	0.00504627087259842\\
218.01	0.00504619026959572\\
219.01	0.00504610816382335\\
220.01	0.00504602452841662\\
221.01	0.00504593933606141\\
222.01	0.00504585255898741\\
223.01	0.00504576416896103\\
224.01	0.00504567413727911\\
225.01	0.00504558243476158\\
226.01	0.00504548903174409\\
227.01	0.00504539389807151\\
228.01	0.00504529700309017\\
229.01	0.00504519831564083\\
230.01	0.00504509780405106\\
231.01	0.00504499543612776\\
232.01	0.00504489117914962\\
233.01	0.00504478499985944\\
234.01	0.0050446768644564\\
235.01	0.00504456673858798\\
236.01	0.00504445458734198\\
237.01	0.00504434037523866\\
238.01	0.00504422406622267\\
239.01	0.00504410562365426\\
240.01	0.0050439850103012\\
241.01	0.00504386218833065\\
242.01	0.00504373711929989\\
243.01	0.00504360976414827\\
244.01	0.00504348008318793\\
245.01	0.00504334803609521\\
246.01	0.00504321358190158\\
247.01	0.00504307667898427\\
248.01	0.00504293728505743\\
249.01	0.00504279535716248\\
250.01	0.00504265085165886\\
251.01	0.00504250372421412\\
252.01	0.00504235392979442\\
253.01	0.00504220142265456\\
254.01	0.00504204615632818\\
255.01	0.00504188808361737\\
256.01	0.00504172715658271\\
257.01	0.00504156332653268\\
258.01	0.00504139654401304\\
259.01	0.00504122675879636\\
260.01	0.00504105391987104\\
261.01	0.00504087797543049\\
262.01	0.00504069887286166\\
263.01	0.00504051655873399\\
264.01	0.00504033097878772\\
265.01	0.00504014207792223\\
266.01	0.00503994980018425\\
267.01	0.00503975408875577\\
268.01	0.00503955488594175\\
269.01	0.00503935213315789\\
270.01	0.00503914577091756\\
271.01	0.00503893573881922\\
272.01	0.00503872197553281\\
273.01	0.00503850441878714\\
274.01	0.00503828300535608\\
275.01	0.00503805767104399\\
276.01	0.0050378283506724\\
277.01	0.00503759497806512\\
278.01	0.0050373574860338\\
279.01	0.00503711580636279\\
280.01	0.0050368698697938\\
281.01	0.00503661960601027\\
282.01	0.00503636494362115\\
283.01	0.00503610581014482\\
284.01	0.00503584213199201\\
285.01	0.00503557383444909\\
286.01	0.00503530084165991\\
287.01	0.00503502307660814\\
288.01	0.00503474046109883\\
289.01	0.00503445291573879\\
290.01	0.00503416035991835\\
291.01	0.00503386271178994\\
292.01	0.00503355988824878\\
293.01	0.00503325180491076\\
294.01	0.00503293837609126\\
295.01	0.00503261951478264\\
296.01	0.00503229513263092\\
297.01	0.00503196513991238\\
298.01	0.0050316294455087\\
299.01	0.0050312879568819\\
300.01	0.00503094058004804\\
301.01	0.00503058721955057\\
302.01	0.00503022777843219\\
303.01	0.00502986215820591\\
304.01	0.00502949025882556\\
305.01	0.00502911197865478\\
306.01	0.00502872721443513\\
307.01	0.00502833586125309\\
308.01	0.00502793781250557\\
309.01	0.00502753295986485\\
310.01	0.00502712119324115\\
311.01	0.00502670240074511\\
312.01	0.0050262764686474\\
313.01	0.00502584328133831\\
314.01	0.00502540272128466\\
315.01	0.00502495466898543\\
316.01	0.00502449900292578\\
317.01	0.00502403559952918\\
318.01	0.00502356433310781\\
319.01	0.00502308507581059\\
320.01	0.00502259769756933\\
321.01	0.00502210206604349\\
322.01	0.00502159804656149\\
323.01	0.00502108550206058\\
324.01	0.00502056429302427\\
325.01	0.00502003427741649\\
326.01	0.00501949531061407\\
327.01	0.00501894724533653\\
328.01	0.00501838993157195\\
329.01	0.0050178232165013\\
330.01	0.005017246944419\\
331.01	0.00501666095665098\\
332.01	0.00501606509146876\\
333.01	0.00501545918400077\\
334.01	0.00501484306614038\\
335.01	0.00501421656644987\\
336.01	0.00501357951006183\\
337.01	0.00501293171857526\\
338.01	0.00501227300994974\\
339.01	0.00501160319839449\\
340.01	0.00501092209425414\\
341.01	0.00501022950389015\\
342.01	0.00500952522955797\\
343.01	0.00500880906928109\\
344.01	0.00500808081671953\\
345.01	0.00500734026103516\\
346.01	0.00500658718675241\\
347.01	0.00500582137361472\\
348.01	0.00500504259643691\\
349.01	0.00500425062495316\\
350.01	0.00500344522366115\\
351.01	0.00500262615166124\\
352.01	0.00500179316249306\\
353.01	0.00500094600396683\\
354.01	0.0050000844179919\\
355.01	0.00499920814040097\\
356.01	0.00499831690077129\\
357.01	0.00499741042224258\\
358.01	0.00499648842133201\\
359.01	0.0049955506077466\\
360.01	0.00499459668419335\\
361.01	0.00499362634618713\\
362.01	0.00499263928185728\\
363.01	0.00499163517175377\\
364.01	0.00499061368865077\\
365.01	0.00498957449735275\\
366.01	0.00498851725449865\\
367.01	0.00498744160836883\\
368.01	0.00498634719869235\\
369.01	0.00498523365645737\\
370.01	0.00498410060372413\\
371.01	0.00498294765344128\\
372.01	0.004981774409267\\
373.01	0.00498058046539411\\
374.01	0.00497936540638164\\
375.01	0.00497812880699148\\
376.01	0.00497687023203234\\
377.01	0.00497558923621056\\
378.01	0.00497428536398801\\
379.01	0.00497295814944752\\
380.01	0.00497160711616688\\
381.01	0.00497023177709904\\
382.01	0.00496883163446029\\
383.01	0.00496740617962532\\
384.01	0.00496595489302728\\
385.01	0.00496447724406343\\
386.01	0.00496297269100318\\
387.01	0.00496144068089814\\
388.01	0.00495988064949051\\
389.01	0.0049582920211187\\
390.01	0.00495667420861591\\
391.01	0.00495502661319822\\
392.01	0.00495334862434032\\
393.01	0.00495163961963088\\
394.01	0.0049498989646079\\
395.01	0.00494812601256582\\
396.01	0.00494632010433079\\
397.01	0.00494448056800078\\
398.01	0.00494260671864545\\
399.01	0.00494069785796225\\
400.01	0.00493875327388629\\
401.01	0.00493677224015327\\
402.01	0.00493475401581408\\
403.01	0.00493269784470363\\
404.01	0.00493060295486799\\
405.01	0.00492846855795484\\
406.01	0.00492629384857654\\
407.01	0.00492407800365421\\
408.01	0.00492182018175633\\
409.01	0.00491951952244241\\
410.01	0.00491717514562545\\
411.01	0.00491478615096051\\
412.01	0.0049123516172678\\
413.01	0.00490987060198812\\
414.01	0.0049073421406633\\
415.01	0.00490476524642877\\
416.01	0.00490213890950422\\
417.01	0.00489946209668025\\
418.01	0.00489673375080099\\
419.01	0.00489395279024194\\
420.01	0.00489111810838552\\
421.01	0.00488822857309285\\
422.01	0.0048852830261742\\
423.01	0.00488228028285686\\
424.01	0.00487921913125238\\
425.01	0.00487609833182316\\
426.01	0.0048729166168476\\
427.01	0.00486967268988787\\
428.01	0.00486636522525517\\
429.01	0.00486299286747915\\
430.01	0.00485955423077612\\
431.01	0.00485604789852002\\
432.01	0.00485247242271422\\
433.01	0.00484882632346511\\
434.01	0.00484510808845668\\
435.01	0.0048413161724264\\
436.01	0.00483744899664086\\
437.01	0.00483350494837175\\
438.01	0.00482948238037028\\
439.01	0.00482537961033915\\
440.01	0.00482119492040131\\
441.01	0.00481692655656285\\
442.01	0.00481257272816965\\
443.01	0.00480813160735409\\
444.01	0.00480360132847085\\
445.01	0.00479897998751858\\
446.01	0.00479426564154536\\
447.01	0.00478945630803472\\
448.01	0.00478454996426917\\
449.01	0.00477954454666905\\
450.01	0.00477443795010257\\
451.01	0.00476922802716536\\
452.01	0.00476391258742675\\
453.01	0.0047584893966391\\
454.01	0.00475295617590996\\
455.01	0.0047473106008352\\
456.01	0.00474155030059124\\
457.01	0.00473567285698744\\
458.01	0.00472967580347889\\
459.01	0.00472355662414201\\
460.01	0.00471731275261322\\
461.01	0.00471094157099698\\
462.01	0.00470444040874411\\
463.01	0.00469780654150765\\
464.01	0.00469103718997992\\
465.01	0.00468412951871555\\
466.01	0.00467708063494494\\
467.01	0.00466988758738365\\
468.01	0.00466254736503701\\
469.01	0.00465505689600287\\
470.01	0.0046474130462699\\
471.01	0.00463961261850789\\
472.01	0.00463165235084363\\
473.01	0.00462352891561738\\
474.01	0.00461523891810933\\
475.01	0.00460677889523471\\
476.01	0.00459814531420051\\
477.01	0.0045893345711252\\
478.01	0.00458034298961881\\
479.01	0.00457116681932295\\
480.01	0.00456180223441107\\
481.01	0.0045522453320476\\
482.01	0.00454249213080608\\
483.01	0.00453253856904761\\
484.01	0.00452238050325833\\
485.01	0.00451201370634854\\
486.01	0.00450143386591313\\
487.01	0.00449063658245445\\
488.01	0.0044796173675692\\
489.01	0.00446837164210053\\
490.01	0.00445689473425607\\
491.01	0.00444518187769261\\
492.01	0.00443322820956902\\
493.01	0.00442102876856632\\
494.01	0.00440857849287668\\
495.01	0.00439587221815883\\
496.01	0.00438290467546104\\
497.01	0.00436967048910916\\
498.01	0.00435616417455901\\
499.01	0.0043423801362124\\
500.01	0.00432831266519476\\
501.01	0.0043139559370941\\
502.01	0.00429930400966176\\
503.01	0.00428435082047391\\
504.01	0.0042690901845561\\
505.01	0.00425351579197043\\
506.01	0.00423762120536703\\
507.01	0.00422139985750059\\
508.01	0.00420484504871337\\
509.01	0.00418794994438539\\
510.01	0.00417070757235356\\
511.01	0.00415311082030088\\
512.01	0.00413515243311747\\
513.01	0.00411682501023545\\
514.01	0.00409812100293835\\
515.01	0.00407903271164887\\
516.01	0.00405955228319665\\
517.01	0.00403967170806943\\
518.01	0.00401938281765047\\
519.01	0.00399867728144732\\
520.01	0.00397754660431566\\
521.01	0.00395598212368432\\
522.01	0.00393397500678653\\
523.01	0.00391151624790538\\
524.01	0.0038885966656406\\
525.01	0.00386520690020487\\
526.01	0.0038413374107603\\
527.01	0.00381697847280496\\
528.01	0.0037921201756219\\
529.01	0.00376675241980425\\
530.01	0.00374086491487127\\
531.01	0.00371444717699319\\
532.01	0.00368748852684237\\
533.01	0.00365997808759294\\
534.01	0.00363190478309231\\
535.01	0.00360325733623105\\
536.01	0.00357402426753979\\
537.01	0.0035441938940475\\
538.01	0.00351375432843491\\
539.01	0.00348269347852603\\
540.01	0.00345099904716186\\
541.01	0.00341865853250594\\
542.01	0.00338565922883782\\
543.01	0.00335198822789567\\
544.01	0.00331763242083754\\
545.01	0.00328257850089516\\
546.01	0.00324681296680706\\
547.01	0.00321032212712155\\
548.01	0.00317309210547439\\
549.01	0.00313510884695503\\
550.01	0.00309635812568573\\
551.01	0.00305682555375531\\
552.01	0.00301649659165863\\
553.01	0.00297535656041113\\
554.01	0.0029333906555258\\
555.01	0.00289058396305458\\
556.01	0.00284692147791951\\
557.01	0.00280238812477814\\
558.01	0.00275696878169113\\
559.01	0.00271064830688349\\
560.01	0.00266341156891711\\
561.01	0.00261524348061712\\
562.01	0.00256612903712336\\
563.01	0.00251605335846367\\
564.01	0.00246500173707516\\
565.01	0.00241295969072369\\
566.01	0.00235991302129535\\
567.01	0.0023058478799553\\
568.01	0.00225075083918186\\
569.01	0.00219460897218979\\
570.01	0.0021374099402512\\
571.01	0.00207914208840051\\
572.01	0.00201979454996673\\
573.01	0.00195935736030474\\
574.01	0.00189782157998909\\
575.01	0.00183517942757606\\
576.01	0.00177142442182105\\
577.01	0.00170655153293824\\
578.01	0.00164055734208577\\
579.01	0.0015734402077281\\
580.01	0.00150520043682321\\
581.01	0.00143584045787427\\
582.01	0.00136536499170297\\
583.01	0.00129378121428826\\
584.01	0.00122109890407295\\
585.01	0.00114733056366669\\
586.01	0.0010724915027304\\
587.01	0.000996599864842329\\
588.01	0.000919676576107574\\
589.01	0.000841745186917533\\
590.01	0.000762831570259378\\
591.01	0.00068296342990758\\
592.01	0.00060216955918722\\
593.01	0.000520478775138188\\
594.01	0.000437918433036224\\
595.01	0.000354512401346254\\
596.01	0.000270278346057853\\
597.01	0.000185224134443608\\
598.01	9.93507029036552e-05\\
599.01	3.18230442563749e-05\\
599.02	3.12770957175551e-05\\
599.03	3.07343620567849e-05\\
599.04	3.01948752175693e-05\\
599.05	2.96586674565693e-05\\
599.06	2.91257713466771e-05\\
599.07	2.85962197801373e-05\\
599.08	2.80700459716933e-05\\
599.09	2.75472834617447e-05\\
599.1	2.70279661195791e-05\\
599.11	2.65121281465847e-05\\
599.12	2.59998040795448e-05\\
599.13	2.54910287939124e-05\\
599.14	2.49858375071695e-05\\
599.15	2.4484265782181e-05\\
599.16	2.39863495305973e-05\\
599.17	2.34921250162837e-05\\
599.18	2.30016288588035e-05\\
599.19	2.25148980369013e-05\\
599.2	2.20319698920508e-05\\
599.21	2.1552882132023e-05\\
599.22	2.10776728344891e-05\\
599.23	2.06063804506721e-05\\
599.24	2.01390438090109e-05\\
599.25	1.96757021188876e-05\\
599.26	1.92163949743647e-05\\
599.27	1.87611623579872e-05\\
599.28	1.83100446445941e-05\\
599.29	1.78630826051934e-05\\
599.3	1.74203174108517e-05\\
599.31	1.69817906366353e-05\\
599.32	1.65475442655914e-05\\
599.33	1.61176206927693e-05\\
599.34	1.56920627292622e-05\\
599.35	1.52709136063186e-05\\
599.36	1.48542169794725e-05\\
599.37	1.4442016932719e-05\\
599.38	1.40343579827368e-05\\
599.39	1.36312859119591e-05\\
599.4	1.32328499381877e-05\\
599.41	1.28390997671621e-05\\
599.42	1.2450085597351e-05\\
599.43	1.20658581248094e-05\\
599.44	1.16864685480531e-05\\
599.45	1.13119685730082e-05\\
599.46	1.09424104179929e-05\\
599.47	1.05778468187639e-05\\
599.48	1.02183310335888e-05\\
599.49	9.86391684839466e-06\\
599.5	9.51465858194112e-06\\
599.51	9.17061109107116e-06\\
599.52	8.83182977599352e-06\\
599.53	8.4983705856221e-06\\
599.54	8.1702900229675e-06\\
599.55	7.84764515058579e-06\\
599.56	7.53049359608279e-06\\
599.57	7.21889355765649e-06\\
599.58	6.91290380971064e-06\\
599.59	6.61258370851861e-06\\
599.6	6.31799319793756e-06\\
599.61	6.02919281519031e-06\\
599.62	5.74624369668701e-06\\
599.63	5.46920758391981e-06\\
599.64	5.19814682940593e-06\\
599.65	4.93312440269685e-06\\
599.66	4.67420389643411e-06\\
599.67	4.42144953247646e-06\\
599.68	4.17492616808582e-06\\
599.69	3.93469930216536e-06\\
599.7	3.70083508156871e-06\\
599.71	3.47340030746116e-06\\
599.72	3.25246244175549e-06\\
599.73	3.03808961360161e-06\\
599.74	2.83035062593855e-06\\
599.75	2.62931496212288e-06\\
599.76	2.43505279260738e-06\\
599.77	2.24763498169606e-06\\
599.78	2.06713309435641e-06\\
599.79	1.89361940310974e-06\\
599.8	1.72716689497872e-06\\
599.81	1.56784927850956e-06\\
599.82	1.41574099085488e-06\\
599.83	1.27091720493813e-06\\
599.84	1.13345383668563e-06\\
599.85	1.00342755231589e-06\\
599.86	8.8091577571392e-07\\
599.87	7.65996695881871e-07\\
599.88	6.58749274441706e-07\\
599.89	5.59253253243699e-07\\
599.9	4.67589162013102e-07\\
599.91	3.83838326099145e-07\\
599.92	3.08082874293444e-07\\
599.93	2.40405746710845e-07\\
599.94	1.8089070277609e-07\\
599.95	1.2962232925906e-07\\
599.96	8.66860484019516e-08\\
599.97	5.21681261349272e-08\\
599.98	2.61556803542173e-08\\
599.99	8.73668930083393e-09\\
600	0\\
};
\addplot [color=black!50!mycolor20,solid,forget plot]
  table[row sep=crcr]{%
0.01	0.00503757637221215\\
1.01	0.00503757530562437\\
2.01	0.00503757421782917\\
3.01	0.00503757310840824\\
4.01	0.00503757197693535\\
5.01	0.00503757082297613\\
6.01	0.00503756964608718\\
7.01	0.00503756844581672\\
8.01	0.00503756722170427\\
9.01	0.00503756597328036\\
10.01	0.00503756470006631\\
11.01	0.00503756340157425\\
12.01	0.00503756207730662\\
13.01	0.0050375607267563\\
14.01	0.00503755934940636\\
15.01	0.00503755794472965\\
16.01	0.0050375565121889\\
17.01	0.00503755505123641\\
18.01	0.00503755356131391\\
19.01	0.00503755204185207\\
20.01	0.0050375504922708\\
21.01	0.00503754891197831\\
22.01	0.00503754730037193\\
23.01	0.00503754565683676\\
24.01	0.00503754398074654\\
25.01	0.00503754227146226\\
26.01	0.00503754052833307\\
27.01	0.00503753875069508\\
28.01	0.00503753693787184\\
29.01	0.00503753508917366\\
30.01	0.0050375332038976\\
31.01	0.00503753128132708\\
32.01	0.00503752932073171\\
33.01	0.00503752732136696\\
34.01	0.00503752528247392\\
35.01	0.00503752320327906\\
36.01	0.00503752108299393\\
37.01	0.00503751892081476\\
38.01	0.00503751671592218\\
39.01	0.0050375144674814\\
40.01	0.00503751217464121\\
41.01	0.00503750983653403\\
42.01	0.00503750745227585\\
43.01	0.00503750502096533\\
44.01	0.00503750254168386\\
45.01	0.00503750001349514\\
46.01	0.00503749743544505\\
47.01	0.00503749480656079\\
48.01	0.00503749212585129\\
49.01	0.00503748939230592\\
50.01	0.00503748660489518\\
51.01	0.00503748376256947\\
52.01	0.00503748086425903\\
53.01	0.00503747790887379\\
54.01	0.00503747489530284\\
55.01	0.00503747182241355\\
56.01	0.00503746868905205\\
57.01	0.00503746549404199\\
58.01	0.00503746223618486\\
59.01	0.00503745891425895\\
60.01	0.00503745552701947\\
61.01	0.0050374520731974\\
62.01	0.00503744855150008\\
63.01	0.00503744496060934\\
64.01	0.00503744129918257\\
65.01	0.00503743756585114\\
66.01	0.00503743375922066\\
67.01	0.00503742987786966\\
68.01	0.00503742592034982\\
69.01	0.00503742188518542\\
70.01	0.00503741777087245\\
71.01	0.00503741357587815\\
72.01	0.00503740929864081\\
73.01	0.00503740493756893\\
74.01	0.00503740049104078\\
75.01	0.00503739595740386\\
76.01	0.00503739133497405\\
77.01	0.00503738662203546\\
78.01	0.00503738181683963\\
79.01	0.00503737691760501\\
80.01	0.00503737192251585\\
81.01	0.0050373668297227\\
82.01	0.00503736163734053\\
83.01	0.00503735634344888\\
84.01	0.00503735094609075\\
85.01	0.00503734544327243\\
86.01	0.00503733983296231\\
87.01	0.00503733411309024\\
88.01	0.00503732828154737\\
89.01	0.00503732233618485\\
90.01	0.00503731627481322\\
91.01	0.00503731009520176\\
92.01	0.00503730379507781\\
93.01	0.00503729737212582\\
94.01	0.00503729082398685\\
95.01	0.00503728414825724\\
96.01	0.00503727734248828\\
97.01	0.00503727040418533\\
98.01	0.00503726333080657\\
99.01	0.00503725611976281\\
100.01	0.00503724876841621\\
101.01	0.00503724127407932\\
102.01	0.00503723363401428\\
103.01	0.00503722584543207\\
104.01	0.00503721790549134\\
105.01	0.00503720981129765\\
106.01	0.00503720155990228\\
107.01	0.00503719314830158\\
108.01	0.0050371845734357\\
109.01	0.00503717583218735\\
110.01	0.00503716692138161\\
111.01	0.00503715783778399\\
112.01	0.00503714857809979\\
113.01	0.0050371391389728\\
114.01	0.00503712951698459\\
115.01	0.00503711970865275\\
116.01	0.0050371097104304\\
117.01	0.00503709951870469\\
118.01	0.00503708912979552\\
119.01	0.00503707853995445\\
120.01	0.00503706774536366\\
121.01	0.00503705674213456\\
122.01	0.00503704552630625\\
123.01	0.00503703409384467\\
124.01	0.00503702244064106\\
125.01	0.0050370105625105\\
126.01	0.00503699845519107\\
127.01	0.00503698611434173\\
128.01	0.00503697353554149\\
129.01	0.00503696071428767\\
130.01	0.00503694764599466\\
131.01	0.0050369343259925\\
132.01	0.00503692074952483\\
133.01	0.00503690691174793\\
134.01	0.00503689280772931\\
135.01	0.00503687843244535\\
136.01	0.00503686378078044\\
137.01	0.00503684884752489\\
138.01	0.00503683362737365\\
139.01	0.00503681811492427\\
140.01	0.0050368023046754\\
141.01	0.00503678619102482\\
142.01	0.0050367697682679\\
143.01	0.00503675303059582\\
144.01	0.00503673597209362\\
145.01	0.00503671858673818\\
146.01	0.0050367008683967\\
147.01	0.00503668281082442\\
148.01	0.0050366644076628\\
149.01	0.00503664565243758\\
150.01	0.00503662653855665\\
151.01	0.00503660705930822\\
152.01	0.0050365872078584\\
153.01	0.00503656697724912\\
154.01	0.00503654636039616\\
155.01	0.00503652535008687\\
156.01	0.00503650393897796\\
157.01	0.00503648211959321\\
158.01	0.00503645988432092\\
159.01	0.00503643722541213\\
160.01	0.00503641413497769\\
161.01	0.00503639060498585\\
162.01	0.00503636662726021\\
163.01	0.00503634219347696\\
164.01	0.00503631729516229\\
165.01	0.00503629192368973\\
166.01	0.00503626607027794\\
167.01	0.00503623972598729\\
168.01	0.00503621288171818\\
169.01	0.00503618552820714\\
170.01	0.00503615765602495\\
171.01	0.00503612925557314\\
172.01	0.00503610031708122\\
173.01	0.00503607083060432\\
174.01	0.00503604078601939\\
175.01	0.00503601017302279\\
176.01	0.00503597898112645\\
177.01	0.00503594719965542\\
178.01	0.00503591481774434\\
179.01	0.00503588182433441\\
180.01	0.00503584820816981\\
181.01	0.00503581395779453\\
182.01	0.00503577906154872\\
183.01	0.0050357435075658\\
184.01	0.00503570728376812\\
185.01	0.00503567037786405\\
186.01	0.00503563277734397\\
187.01	0.00503559446947682\\
188.01	0.00503555544130603\\
189.01	0.00503551567964594\\
190.01	0.0050354751710778\\
191.01	0.00503543390194573\\
192.01	0.00503539185835323\\
193.01	0.0050353490261584\\
194.01	0.00503530539097027\\
195.01	0.00503526093814411\\
196.01	0.00503521565277792\\
197.01	0.00503516951970722\\
198.01	0.00503512252350134\\
199.01	0.00503507464845814\\
200.01	0.00503502587860027\\
201.01	0.00503497619767009\\
202.01	0.00503492558912484\\
203.01	0.00503487403613193\\
204.01	0.00503482152156468\\
205.01	0.00503476802799605\\
206.01	0.00503471353769469\\
207.01	0.00503465803261952\\
208.01	0.00503460149441431\\
209.01	0.00503454390440248\\
210.01	0.00503448524358185\\
211.01	0.00503442549261899\\
212.01	0.00503436463184354\\
213.01	0.00503430264124295\\
214.01	0.00503423950045613\\
215.01	0.00503417518876822\\
216.01	0.00503410968510407\\
217.01	0.00503404296802266\\
218.01	0.00503397501571035\\
219.01	0.00503390580597508\\
220.01	0.00503383531623984\\
221.01	0.00503376352353603\\
222.01	0.00503369040449694\\
223.01	0.00503361593535149\\
224.01	0.00503354009191626\\
225.01	0.00503346284958959\\
226.01	0.00503338418334417\\
227.01	0.00503330406771952\\
228.01	0.00503322247681503\\
229.01	0.00503313938428232\\
230.01	0.00503305476331757\\
231.01	0.00503296858665383\\
232.01	0.00503288082655331\\
233.01	0.00503279145479899\\
234.01	0.00503270044268673\\
235.01	0.00503260776101671\\
236.01	0.00503251338008532\\
237.01	0.00503241726967597\\
238.01	0.00503231939905068\\
239.01	0.00503221973694097\\
240.01	0.00503211825153886\\
241.01	0.00503201491048714\\
242.01	0.00503190968087044\\
243.01	0.00503180252920504\\
244.01	0.00503169342142931\\
245.01	0.00503158232289353\\
246.01	0.00503146919834986\\
247.01	0.00503135401194118\\
248.01	0.0050312367271912\\
249.01	0.00503111730699308\\
250.01	0.00503099571359866\\
251.01	0.00503087190860672\\
252.01	0.00503074585295193\\
253.01	0.00503061750689229\\
254.01	0.00503048682999773\\
255.01	0.00503035378113773\\
256.01	0.00503021831846828\\
257.01	0.00503008039941939\\
258.01	0.00502993998068197\\
259.01	0.00502979701819428\\
260.01	0.0050296514671284\\
261.01	0.00502950328187615\\
262.01	0.00502935241603455\\
263.01	0.00502919882239165\\
264.01	0.00502904245291109\\
265.01	0.00502888325871732\\
266.01	0.00502872119007941\\
267.01	0.00502855619639554\\
268.01	0.00502838822617612\\
269.01	0.00502821722702749\\
270.01	0.00502804314563448\\
271.01	0.00502786592774291\\
272.01	0.00502768551814172\\
273.01	0.00502750186064476\\
274.01	0.00502731489807084\\
275.01	0.00502712457222613\\
276.01	0.00502693082388276\\
277.01	0.00502673359275983\\
278.01	0.0050265328175017\\
279.01	0.00502632843565699\\
280.01	0.0050261203836568\\
281.01	0.00502590859679258\\
282.01	0.00502569300919249\\
283.01	0.00502547355379831\\
284.01	0.00502525016234141\\
285.01	0.00502502276531742\\
286.01	0.00502479129196121\\
287.01	0.00502455567022076\\
288.01	0.00502431582673\\
289.01	0.00502407168678189\\
290.01	0.00502382317429955\\
291.01	0.00502357021180749\\
292.01	0.00502331272040208\\
293.01	0.0050230506197208\\
294.01	0.0050227838279105\\
295.01	0.00502251226159539\\
296.01	0.00502223583584404\\
297.01	0.00502195446413479\\
298.01	0.00502166805832143\\
299.01	0.0050213765285966\\
300.01	0.00502107978345517\\
301.01	0.00502077772965604\\
302.01	0.0050204702721828\\
303.01	0.0050201573142045\\
304.01	0.00501983875703335\\
305.01	0.00501951450008285\\
306.01	0.00501918444082401\\
307.01	0.00501884847474066\\
308.01	0.005018506495283\\
309.01	0.00501815839382062\\
310.01	0.00501780405959352\\
311.01	0.00501744337966192\\
312.01	0.00501707623885547\\
313.01	0.00501670251971957\\
314.01	0.00501632210246132\\
315.01	0.00501593486489365\\
316.01	0.00501554068237779\\
317.01	0.00501513942776458\\
318.01	0.00501473097133332\\
319.01	0.00501431518072992\\
320.01	0.00501389192090298\\
321.01	0.00501346105403746\\
322.01	0.00501302243948797\\
323.01	0.00501257593370905\\
324.01	0.00501212139018412\\
325.01	0.00501165865935276\\
326.01	0.00501118758853553\\
327.01	0.00501070802185731\\
328.01	0.00501021980016853\\
329.01	0.00500972276096439\\
330.01	0.0050092167383022\\
331.01	0.00500870156271605\\
332.01	0.00500817706113033\\
333.01	0.00500764305677077\\
334.01	0.00500709936907302\\
335.01	0.0050065458135899\\
336.01	0.00500598220189525\\
337.01	0.00500540834148718\\
338.01	0.0050048240356881\\
339.01	0.00500422908354254\\
340.01	0.00500362327971341\\
341.01	0.00500300641437589\\
342.01	0.0050023782731091\\
343.01	0.00500173863678528\\
344.01	0.00500108728145748\\
345.01	0.00500042397824491\\
346.01	0.00499974849321595\\
347.01	0.00499906058726931\\
348.01	0.00499836001601358\\
349.01	0.0049976465296438\\
350.01	0.00499691987281695\\
351.01	0.00499617978452547\\
352.01	0.00499542599796851\\
353.01	0.00499465824042179\\
354.01	0.00499387623310557\\
355.01	0.00499307969105064\\
356.01	0.00499226832296384\\
357.01	0.0049914418310903\\
358.01	0.00499059991107576\\
359.01	0.00498974225182652\\
360.01	0.00498886853536857\\
361.01	0.00498797843670493\\
362.01	0.00498707162367234\\
363.01	0.00498614775679577\\
364.01	0.00498520648914337\\
365.01	0.00498424746617824\\
366.01	0.00498327032561077\\
367.01	0.00498227469724855\\
368.01	0.00498126020284642\\
369.01	0.00498022645595387\\
370.01	0.00497917306176225\\
371.01	0.00497809961694975\\
372.01	0.00497700570952573\\
373.01	0.00497589091867262\\
374.01	0.00497475481458582\\
375.01	0.0049735969583125\\
376.01	0.00497241690158675\\
377.01	0.00497121418666233\\
378.01	0.00496998834614267\\
379.01	0.00496873890280737\\
380.01	0.00496746536943322\\
381.01	0.00496616724861297\\
382.01	0.00496484403256673\\
383.01	0.00496349520294922\\
384.01	0.0049621202306498\\
385.01	0.00496071857558527\\
386.01	0.00495928968648604\\
387.01	0.00495783300067191\\
388.01	0.00495634794382075\\
389.01	0.00495483392972596\\
390.01	0.00495329036004352\\
391.01	0.00495171662402862\\
392.01	0.00495011209825996\\
393.01	0.00494847614635249\\
394.01	0.00494680811865763\\
395.01	0.0049451073519513\\
396.01	0.00494337316910967\\
397.01	0.00494160487877243\\
398.01	0.00493980177499517\\
399.01	0.00493796313689028\\
400.01	0.00493608822825808\\
401.01	0.00493417629720918\\
402.01	0.00493222657577901\\
403.01	0.00493023827953653\\
404.01	0.00492821060718771\\
405.01	0.00492614274017615\\
406.01	0.00492403384228107\\
407.01	0.00492188305921548\\
408.01	0.00491968951822292\\
409.01	0.0049174523276755\\
410.01	0.00491517057667094\\
411.01	0.00491284333463032\\
412.01	0.00491046965089229\\
413.01	0.00490804855430651\\
414.01	0.0049055790528206\\
415.01	0.00490306013306335\\
416.01	0.00490049075992149\\
417.01	0.00489786987611105\\
418.01	0.00489519640174224\\
419.01	0.00489246923387906\\
420.01	0.0048896872460923\\
421.01	0.00488684928800671\\
422.01	0.00488395418484163\\
423.01	0.00488100073694514\\
424.01	0.00487798771932154\\
425.01	0.0048749138811516\\
426.01	0.00487177794530663\\
427.01	0.00486857860785368\\
428.01	0.00486531453755403\\
429.01	0.00486198437535299\\
430.01	0.00485858673386123\\
431.01	0.00485512019682707\\
432.01	0.00485158331859944\\
433.01	0.00484797462358041\\
434.01	0.00484429260566794\\
435.01	0.00484053572768602\\
436.01	0.00483670242080498\\
437.01	0.00483279108394687\\
438.01	0.00482880008317991\\
439.01	0.00482472775109747\\
440.01	0.00482057238618295\\
441.01	0.00481633225215964\\
442.01	0.00481200557732326\\
443.01	0.0048075905538585\\
444.01	0.00480308533713759\\
445.01	0.00479848804500103\\
446.01	0.00479379675701847\\
447.01	0.00478900951373037\\
448.01	0.00478412431586967\\
449.01	0.00477913912356268\\
450.01	0.00477405185550922\\
451.01	0.00476886038814144\\
452.01	0.00476356255476095\\
453.01	0.0047581561446548\\
454.01	0.00475263890219039\\
455.01	0.00474700852588799\\
456.01	0.00474126266747335\\
457.01	0.00473539893090912\\
458.01	0.00472941487140539\\
459.01	0.00472330799441088\\
460.01	0.00471707575458347\\
461.01	0.00471071555474243\\
462.01	0.00470422474480017\\
463.01	0.00469760062067582\\
464.01	0.00469084042318818\\
465.01	0.00468394133692999\\
466.01	0.00467690048912133\\
467.01	0.0046697149484417\\
468.01	0.00466238172384067\\
469.01	0.00465489776332429\\
470.01	0.00464725995271764\\
471.01	0.00463946511440073\\
472.01	0.00463151000601805\\
473.01	0.00462339131915957\\
474.01	0.00461510567801282\\
475.01	0.00460664963798493\\
476.01	0.00459801968429604\\
477.01	0.00458921223054078\\
478.01	0.00458022361722044\\
479.01	0.00457105011024327\\
480.01	0.00456168789939413\\
481.01	0.00455213309677256\\
482.01	0.00454238173519966\\
483.01	0.00453242976659276\\
484.01	0.00452227306030811\\
485.01	0.0045119074014521\\
486.01	0.00450132848915897\\
487.01	0.004490531934837\\
488.01	0.00447951326038137\\
489.01	0.00446826789635372\\
490.01	0.00445679118012794\\
491.01	0.00444507835400288\\
492.01	0.00443312456327982\\
493.01	0.00442092485430535\\
494.01	0.00440847417247915\\
495.01	0.00439576736022571\\
496.01	0.00438279915493058\\
497.01	0.0043695641868392\\
498.01	0.00435605697691972\\
499.01	0.00434227193468841\\
500.01	0.00432820335599812\\
501.01	0.00431384542079002\\
502.01	0.0042991921908077\\
503.01	0.00428423760727525\\
504.01	0.00426897548853848\\
505.01	0.00425339952767048\\
506.01	0.0042375032900411\\
507.01	0.00422128021085203\\
508.01	0.00420472359263741\\
509.01	0.00418782660273114\\
510.01	0.00417058227070207\\
511.01	0.00415298348575844\\
512.01	0.00413502299412217\\
513.01	0.00411669339637638\\
514.01	0.00409798714478656\\
515.01	0.0040788965405988\\
516.01	0.00405941373131771\\
517.01	0.00403953070796672\\
518.01	0.00401923930233447\\
519.01	0.00399853118421253\\
520.01	0.00397739785862717\\
521.01	0.00395583066307222\\
522.01	0.00393382076474854\\
523.01	0.00391135915781609\\
524.01	0.00388843666066737\\
525.01	0.00386504391323036\\
526.01	0.00384117137431015\\
527.01	0.00381680931898115\\
528.01	0.00379194783604039\\
529.01	0.00376657682553818\\
530.01	0.00374068599639765\\
531.01	0.0037142648641438\\
532.01	0.00368730274875859\\
533.01	0.00365978877268431\\
534.01	0.00363171185899915\\
535.01	0.00360306072979041\\
536.01	0.00357382390475556\\
537.01	0.00354398970006315\\
538.01	0.00351354622751146\\
539.01	0.0034824813940233\\
540.01	0.00345078290152427\\
541.01	0.00341843824725186\\
542.01	0.00338543472455447\\
543.01	0.00335175942423902\\
544.01	0.00331739923653709\\
545.01	0.00328234085376592\\
546.01	0.00324657077376695\\
547.01	0.00321007530421679\\
548.01	0.0031728405679133\\
549.01	0.00313485250914984\\
550.01	0.00309609690130523\\
551.01	0.00305655935578795\\
552.01	0.0030162253324876\\
553.01	0.00297508015190375\\
554.01	0.00293310900913688\\
555.01	0.00289029698994688\\
556.01	0.00284662908910202\\
557.01	0.00280209023126455\\
558.01	0.00275666529468013\\
559.01	0.00271033913796359\\
560.01	0.00266309663029763\\
561.01	0.00261492268538809\\
562.01	0.00256580229954686\\
563.01	0.0025157205942994\\
564.01	0.00246466286394252\\
565.01	0.00241261462850278\\
566.01	0.00235956169257031\\
567.01	0.00230549021050174\\
568.01	0.00225038675850025\\
569.01	0.00219423841408641\\
570.01	0.00213703284346752\\
571.01	0.00207875839729022\\
572.01	0.00201940421521853\\
573.01	0.00195896033970692\\
574.01	0.00189741783922914\\
575.01	0.00183476894106481\\
576.01	0.00177100717352635\\
577.01	0.00170612751720696\\
578.01	0.00164012656442551\\
579.01	0.0015730026855086\\
580.01	0.00150475619984723\\
581.01	0.00143538954874788\\
582.01	0.00136490746591783\\
583.01	0.00129331713990011\\
584.01	0.0012206283608285\\
585.01	0.00114685364138917\\
586.01	0.00107200829872149\\
587.01	0.000996110479992625\\
588.01	0.000919181109325396\\
589.01	0.000841243727382117\\
590.01	0.000762324186875217\\
591.01	0.000682450157175529\\
592.01	0.000601650378505039\\
593.01	0.000519953590291149\\
594.01	0.00043738703832388\\
595.01	0.00035397444039822\\
596.01	0.000269733258902364\\
597.01	0.000184671089782441\\
598.01	9.88225819852275e-05\\
599.01	3.18230442563731e-05\\
599.02	3.12770957175551e-05\\
599.03	3.07343620567849e-05\\
599.04	3.0194875217571e-05\\
599.05	2.96586674565693e-05\\
599.06	2.91257713466771e-05\\
599.07	2.85962197801391e-05\\
599.08	2.80700459716916e-05\\
599.09	2.75472834617447e-05\\
599.1	2.70279661195773e-05\\
599.11	2.65121281465865e-05\\
599.12	2.5999804079543e-05\\
599.13	2.54910287939142e-05\\
599.14	2.49858375071712e-05\\
599.15	2.44842657821827e-05\\
599.16	2.39863495305956e-05\\
599.17	2.34921250162855e-05\\
599.18	2.30016288588035e-05\\
599.19	2.25148980369013e-05\\
599.2	2.20319698920508e-05\\
599.21	2.15528821320213e-05\\
599.22	2.10776728344908e-05\\
599.23	2.06063804506721e-05\\
599.24	2.01390438090109e-05\\
599.25	1.96757021188858e-05\\
599.26	1.9216394974363e-05\\
599.27	1.87611623579855e-05\\
599.28	1.83100446445959e-05\\
599.29	1.78630826051952e-05\\
599.3	1.74203174108517e-05\\
599.31	1.69817906366353e-05\\
599.32	1.65475442655931e-05\\
599.33	1.61176206927693e-05\\
599.34	1.56920627292622e-05\\
599.35	1.52709136063203e-05\\
599.36	1.48542169794742e-05\\
599.37	1.4442016932719e-05\\
599.38	1.40343579827368e-05\\
599.39	1.36312859119591e-05\\
599.4	1.32328499381877e-05\\
599.41	1.28390997671604e-05\\
599.42	1.2450085597351e-05\\
599.43	1.20658581248094e-05\\
599.44	1.16864685480531e-05\\
599.45	1.13119685730065e-05\\
599.46	1.09424104179929e-05\\
599.47	1.05778468187639e-05\\
599.48	1.02183310335888e-05\\
599.49	9.86391684839293e-06\\
599.5	9.51465858193938e-06\\
599.51	9.17061109107289e-06\\
599.52	8.83182977599525e-06\\
599.53	8.4983705856221e-06\\
599.54	8.1702900229675e-06\\
599.55	7.84764515058753e-06\\
599.56	7.53049359608453e-06\\
599.57	7.21889355765649e-06\\
599.58	6.91290380970891e-06\\
599.59	6.61258370851688e-06\\
599.6	6.31799319793756e-06\\
599.61	6.02919281519031e-06\\
599.62	5.74624369668701e-06\\
599.63	5.46920758391807e-06\\
599.64	5.19814682940593e-06\\
599.65	4.93312440269685e-06\\
599.66	4.67420389643237e-06\\
599.67	4.42144953247646e-06\\
599.68	4.17492616808582e-06\\
599.69	3.9346993021671e-06\\
599.7	3.70083508156871e-06\\
599.71	3.47340030746289e-06\\
599.72	3.25246244175723e-06\\
599.73	3.03808961360161e-06\\
599.74	2.83035062593855e-06\\
599.75	2.62931496212288e-06\\
599.76	2.43505279260738e-06\\
599.77	2.24763498169606e-06\\
599.78	2.06713309435641e-06\\
599.79	1.89361940311147e-06\\
599.8	1.72716689498045e-06\\
599.81	1.56784927850782e-06\\
599.82	1.41574099085315e-06\\
599.83	1.27091720493813e-06\\
599.84	1.13345383668736e-06\\
599.85	1.00342755231415e-06\\
599.86	8.8091577571392e-07\\
599.87	7.65996695880136e-07\\
599.88	6.58749274441706e-07\\
599.89	5.59253253243699e-07\\
599.9	4.67589162011367e-07\\
599.91	3.83838326099145e-07\\
599.92	3.0808287429171e-07\\
599.93	2.40405746710845e-07\\
599.94	1.80890702777825e-07\\
599.95	1.2962232925906e-07\\
599.96	8.66860484002169e-08\\
599.97	5.21681261331924e-08\\
599.98	2.61556803542173e-08\\
599.99	8.73668930083393e-09\\
600	0\\
};
\addplot [color=black!60!mycolor21,solid,forget plot]
  table[row sep=crcr]{%
0.01	0.00503139859049536\\
1.01	0.00503139763734625\\
2.01	0.00503139666545196\\
3.01	0.00503139567444704\\
4.01	0.00503139466395896\\
5.01	0.00503139363360786\\
6.01	0.00503139258300674\\
7.01	0.00503139151176115\\
8.01	0.00503139041946888\\
9.01	0.00503138930572007\\
10.01	0.00503138817009698\\
11.01	0.00503138701217382\\
12.01	0.00503138583151661\\
13.01	0.005031384627683\\
14.01	0.00503138340022211\\
15.01	0.00503138214867432\\
16.01	0.00503138087257144\\
17.01	0.00503137957143602\\
18.01	0.00503137824478144\\
19.01	0.00503137689211191\\
20.01	0.00503137551292208\\
21.01	0.00503137410669669\\
22.01	0.00503137267291078\\
23.01	0.00503137121102927\\
24.01	0.00503136972050663\\
25.01	0.00503136820078729\\
26.01	0.0050313666513046\\
27.01	0.00503136507148146\\
28.01	0.00503136346072934\\
29.01	0.00503136181844857\\
30.01	0.00503136014402813\\
31.01	0.00503135843684492\\
32.01	0.00503135669626435\\
33.01	0.00503135492163941\\
34.01	0.0050313531123108\\
35.01	0.00503135126760659\\
36.01	0.0050313493868421\\
37.01	0.00503134746931938\\
38.01	0.00503134551432744\\
39.01	0.00503134352114138\\
40.01	0.00503134148902256\\
41.01	0.00503133941721832\\
42.01	0.00503133730496162\\
43.01	0.00503133515147055\\
44.01	0.00503133295594866\\
45.01	0.00503133071758422\\
46.01	0.00503132843554987\\
47.01	0.00503132610900244\\
48.01	0.00503132373708287\\
49.01	0.00503132131891577\\
50.01	0.00503131885360891\\
51.01	0.00503131634025312\\
52.01	0.00503131377792208\\
53.01	0.00503131116567176\\
54.01	0.0050313085025401\\
55.01	0.00503130578754686\\
56.01	0.00503130301969303\\
57.01	0.00503130019796089\\
58.01	0.00503129732131309\\
59.01	0.00503129438869273\\
60.01	0.0050312913990228\\
61.01	0.00503128835120597\\
62.01	0.00503128524412393\\
63.01	0.00503128207663747\\
64.01	0.0050312788475854\\
65.01	0.00503127555578462\\
66.01	0.00503127220002975\\
67.01	0.00503126877909265\\
68.01	0.0050312652917218\\
69.01	0.00503126173664195\\
70.01	0.00503125811255386\\
71.01	0.00503125441813366\\
72.01	0.0050312506520323\\
73.01	0.00503124681287535\\
74.01	0.00503124289926258\\
75.01	0.00503123890976706\\
76.01	0.0050312348429351\\
77.01	0.00503123069728553\\
78.01	0.00503122647130935\\
79.01	0.00503122216346869\\
80.01	0.00503121777219725\\
81.01	0.00503121329589863\\
82.01	0.00503120873294689\\
83.01	0.005031204081685\\
84.01	0.00503119934042499\\
85.01	0.00503119450744717\\
86.01	0.00503118958099935\\
87.01	0.0050311845592965\\
88.01	0.00503117944051983\\
89.01	0.00503117422281647\\
90.01	0.00503116890429895\\
91.01	0.00503116348304399\\
92.01	0.00503115795709251\\
93.01	0.00503115232444854\\
94.01	0.00503114658307843\\
95.01	0.00503114073091068\\
96.01	0.00503113476583492\\
97.01	0.0050311286857011\\
98.01	0.00503112248831905\\
99.01	0.00503111617145736\\
100.01	0.005031109732843\\
101.01	0.00503110317016036\\
102.01	0.00503109648105029\\
103.01	0.00503108966310976\\
104.01	0.00503108271389102\\
105.01	0.00503107563090009\\
106.01	0.00503106841159678\\
107.01	0.00503106105339315\\
108.01	0.005031053553653\\
109.01	0.00503104590969137\\
110.01	0.00503103811877243\\
111.01	0.00503103017810982\\
112.01	0.00503102208486511\\
113.01	0.00503101383614706\\
114.01	0.00503100542901042\\
115.01	0.00503099686045517\\
116.01	0.00503098812742551\\
117.01	0.00503097922680829\\
118.01	0.00503097015543318\\
119.01	0.00503096091007028\\
120.01	0.00503095148743003\\
121.01	0.00503094188416136\\
122.01	0.0050309320968513\\
123.01	0.00503092212202324\\
124.01	0.0050309119561359\\
125.01	0.00503090159558271\\
126.01	0.00503089103668979\\
127.01	0.00503088027571513\\
128.01	0.00503086930884764\\
129.01	0.00503085813220537\\
130.01	0.00503084674183433\\
131.01	0.00503083513370743\\
132.01	0.00503082330372289\\
133.01	0.00503081124770333\\
134.01	0.00503079896139353\\
135.01	0.00503078644045992\\
136.01	0.00503077368048869\\
137.01	0.00503076067698438\\
138.01	0.00503074742536853\\
139.01	0.00503073392097806\\
140.01	0.00503072015906363\\
141.01	0.00503070613478845\\
142.01	0.00503069184322624\\
143.01	0.00503067727935992\\
144.01	0.00503066243807978\\
145.01	0.00503064731418194\\
146.01	0.00503063190236643\\
147.01	0.00503061619723582\\
148.01	0.00503060019329304\\
149.01	0.00503058388493984\\
150.01	0.00503056726647493\\
151.01	0.00503055033209187\\
152.01	0.00503053307587752\\
153.01	0.00503051549180999\\
154.01	0.00503049757375628\\
155.01	0.00503047931547104\\
156.01	0.00503046071059376\\
157.01	0.00503044175264716\\
158.01	0.00503042243503509\\
159.01	0.00503040275103985\\
160.01	0.00503038269382059\\
161.01	0.00503036225641073\\
162.01	0.00503034143171598\\
163.01	0.00503032021251167\\
164.01	0.0050302985914408\\
165.01	0.00503027656101107\\
166.01	0.00503025411359278\\
167.01	0.00503023124141671\\
168.01	0.00503020793657069\\
169.01	0.00503018419099782\\
170.01	0.00503015999649337\\
171.01	0.00503013534470243\\
172.01	0.00503011022711691\\
173.01	0.00503008463507281\\
174.01	0.00503005855974766\\
175.01	0.00503003199215706\\
176.01	0.00503000492315242\\
177.01	0.00502997734341756\\
178.01	0.00502994924346589\\
179.01	0.00502992061363698\\
180.01	0.00502989144409368\\
181.01	0.0050298617248189\\
182.01	0.00502983144561235\\
183.01	0.00502980059608679\\
184.01	0.00502976916566513\\
185.01	0.00502973714357697\\
186.01	0.00502970451885467\\
187.01	0.00502967128032999\\
188.01	0.0050296374166304\\
189.01	0.00502960291617576\\
190.01	0.0050295677671735\\
191.01	0.00502953195761594\\
192.01	0.00502949547527527\\
193.01	0.00502945830770057\\
194.01	0.00502942044221267\\
195.01	0.00502938186590086\\
196.01	0.00502934256561803\\
197.01	0.00502930252797667\\
198.01	0.00502926173934435\\
199.01	0.0050292201858392\\
200.01	0.00502917785332523\\
201.01	0.00502913472740798\\
202.01	0.00502909079342936\\
203.01	0.00502904603646299\\
204.01	0.00502900044130903\\
205.01	0.0050289539924897\\
206.01	0.00502890667424345\\
207.01	0.00502885847051987\\
208.01	0.0050288093649748\\
209.01	0.00502875934096441\\
210.01	0.00502870838153951\\
211.01	0.00502865646944035\\
212.01	0.00502860358709064\\
213.01	0.00502854971659139\\
214.01	0.00502849483971531\\
215.01	0.0050284389379003\\
216.01	0.00502838199224336\\
217.01	0.00502832398349412\\
218.01	0.00502826489204852\\
219.01	0.00502820469794217\\
220.01	0.00502814338084295\\
221.01	0.00502808092004487\\
222.01	0.00502801729446075\\
223.01	0.0050279524826145\\
224.01	0.00502788646263481\\
225.01	0.00502781921224657\\
226.01	0.00502775070876387\\
227.01	0.00502768092908203\\
228.01	0.00502760984966948\\
229.01	0.00502753744655966\\
230.01	0.0050274636953428\\
231.01	0.00502738857115743\\
232.01	0.00502731204868156\\
233.01	0.005027234102124\\
234.01	0.00502715470521516\\
235.01	0.00502707383119783\\
236.01	0.00502699145281805\\
237.01	0.00502690754231499\\
238.01	0.00502682207141134\\
239.01	0.00502673501130347\\
240.01	0.00502664633265087\\
241.01	0.00502655600556584\\
242.01	0.0050264639996026\\
243.01	0.0050263702837467\\
244.01	0.00502627482640345\\
245.01	0.00502617759538634\\
246.01	0.00502607855790602\\
247.01	0.00502597768055802\\
248.01	0.00502587492931026\\
249.01	0.00502577026949106\\
250.01	0.00502566366577619\\
251.01	0.00502555508217566\\
252.01	0.00502544448202057\\
253.01	0.00502533182794971\\
254.01	0.0050252170818954\\
255.01	0.00502510020506895\\
256.01	0.00502498115794663\\
257.01	0.0050248599002545\\
258.01	0.00502473639095347\\
259.01	0.00502461058822324\\
260.01	0.00502448244944653\\
261.01	0.00502435193119314\\
262.01	0.00502421898920295\\
263.01	0.00502408357836889\\
264.01	0.00502394565271949\\
265.01	0.00502380516540097\\
266.01	0.0050236620686594\\
267.01	0.00502351631382134\\
268.01	0.00502336785127532\\
269.01	0.00502321663045207\\
270.01	0.00502306259980404\\
271.01	0.00502290570678556\\
272.01	0.00502274589783178\\
273.01	0.0050225831183366\\
274.01	0.00502241731263146\\
275.01	0.00502224842396228\\
276.01	0.00502207639446676\\
277.01	0.00502190116515079\\
278.01	0.0050217226758641\\
279.01	0.00502154086527589\\
280.01	0.00502135567084918\\
281.01	0.00502116702881495\\
282.01	0.00502097487414639\\
283.01	0.00502077914053067\\
284.01	0.00502057976034186\\
285.01	0.00502037666461219\\
286.01	0.00502016978300333\\
287.01	0.00501995904377588\\
288.01	0.00501974437375977\\
289.01	0.00501952569832223\\
290.01	0.00501930294133615\\
291.01	0.00501907602514742\\
292.01	0.00501884487054113\\
293.01	0.00501860939670745\\
294.01	0.00501836952120628\\
295.01	0.00501812515993157\\
296.01	0.00501787622707447\\
297.01	0.00501762263508545\\
298.01	0.00501736429463557\\
299.01	0.00501710111457737\\
300.01	0.00501683300190418\\
301.01	0.00501655986170874\\
302.01	0.00501628159714126\\
303.01	0.00501599810936545\\
304.01	0.00501570929751445\\
305.01	0.00501541505864538\\
306.01	0.00501511528769338\\
307.01	0.00501480987742318\\
308.01	0.00501449871838138\\
309.01	0.00501418169884616\\
310.01	0.00501385870477673\\
311.01	0.0050135296197612\\
312.01	0.0050131943249632\\
313.01	0.00501285269906782\\
314.01	0.00501250461822576\\
315.01	0.00501214995599656\\
316.01	0.00501178858329041\\
317.01	0.00501142036830881\\
318.01	0.00501104517648384\\
319.01	0.00501066287041591\\
320.01	0.00501027330981074\\
321.01	0.00500987635141417\\
322.01	0.00500947184894646\\
323.01	0.00500905965303438\\
324.01	0.00500863961114235\\
325.01	0.00500821156750197\\
326.01	0.0050077753630402\\
327.01	0.00500733083530581\\
328.01	0.0050068778183947\\
329.01	0.00500641614287336\\
330.01	0.00500594563570093\\
331.01	0.00500546612014977\\
332.01	0.00500497741572432\\
333.01	0.00500447933807829\\
334.01	0.00500397169893088\\
335.01	0.00500345430598014\\
336.01	0.00500292696281612\\
337.01	0.00500238946883131\\
338.01	0.00500184161912962\\
339.01	0.0050012832044345\\
340.01	0.00500071401099379\\
341.01	0.00500013382048451\\
342.01	0.00499954240991468\\
343.01	0.00499893955152418\\
344.01	0.00499832501268307\\
345.01	0.0049976985557892\\
346.01	0.00499705993816283\\
347.01	0.0049964089119405\\
348.01	0.0049957452239657\\
349.01	0.00499506861567949\\
350.01	0.00499437882300773\\
351.01	0.0049936755762471\\
352.01	0.00499295859994831\\
353.01	0.00499222761279911\\
354.01	0.00499148232750315\\
355.01	0.00499072245065837\\
356.01	0.00498994768263262\\
357.01	0.00498915771743696\\
358.01	0.0049883522425973\\
359.01	0.00498753093902332\\
360.01	0.00498669348087552\\
361.01	0.00498583953542931\\
362.01	0.00498496876293744\\
363.01	0.00498408081648879\\
364.01	0.00498317534186573\\
365.01	0.00498225197739787\\
366.01	0.00498131035381339\\
367.01	0.0049803500940877\\
368.01	0.00497937081328793\\
369.01	0.00497837211841585\\
370.01	0.0049773536082462\\
371.01	0.00497631487316221\\
372.01	0.0049752554949872\\
373.01	0.0049741750468126\\
374.01	0.00497307309282207\\
375.01	0.00497194918811139\\
376.01	0.00497080287850422\\
377.01	0.00496963370036354\\
378.01	0.00496844118039817\\
379.01	0.00496722483546439\\
380.01	0.00496598417236356\\
381.01	0.00496471868763301\\
382.01	0.00496342786733346\\
383.01	0.00496211118682948\\
384.01	0.00496076811056514\\
385.01	0.00495939809183382\\
386.01	0.0049580005725418\\
387.01	0.00495657498296681\\
388.01	0.0049551207415093\\
389.01	0.00495363725443867\\
390.01	0.0049521239156333\\
391.01	0.00495058010631375\\
392.01	0.0049490051947711\\
393.01	0.00494739853608841\\
394.01	0.00494575947185656\\
395.01	0.00494408732988409\\
396.01	0.00494238142390204\\
397.01	0.00494064105326258\\
398.01	0.0049388655026322\\
399.01	0.00493705404168015\\
400.01	0.00493520592476195\\
401.01	0.00493332039059669\\
402.01	0.00493139666194075\\
403.01	0.00492943394525598\\
404.01	0.00492743143037305\\
405.01	0.00492538829015008\\
406.01	0.00492330368012597\\
407.01	0.00492117673816835\\
408.01	0.00491900658411652\\
409.01	0.00491679231941834\\
410.01	0.00491453302676086\\
411.01	0.00491222776969481\\
412.01	0.00490987559225262\\
413.01	0.0049074755185588\\
414.01	0.00490502655243384\\
415.01	0.00490252767699035\\
416.01	0.00489997785422148\\
417.01	0.00489737602458166\\
418.01	0.00489472110655943\\
419.01	0.00489201199624185\\
420.01	0.00488924756687061\\
421.01	0.00488642666838923\\
422.01	0.00488354812698167\\
423.01	0.00488061074460138\\
424.01	0.00487761329849112\\
425.01	0.0048745545406932\\
426.01	0.00487143319754917\\
427.01	0.00486824796918974\\
428.01	0.00486499752901395\\
429.01	0.00486168052315687\\
430.01	0.00485829556994715\\
431.01	0.00485484125935185\\
432.01	0.00485131615240987\\
433.01	0.00484771878065343\\
434.01	0.00484404764551618\\
435.01	0.00484030121772918\\
436.01	0.00483647793670301\\
437.01	0.00483257620989627\\
438.01	0.00482859441217073\\
439.01	0.00482453088513136\\
440.01	0.00482038393645248\\
441.01	0.00481615183918814\\
442.01	0.00481183283106798\\
443.01	0.00480742511377696\\
444.01	0.00480292685222\\
445.01	0.00479833617376934\\
446.01	0.00479365116749608\\
447.01	0.00478886988338537\\
448.01	0.00478399033153439\\
449.01	0.00477901048133291\\
450.01	0.00477392826062676\\
451.01	0.00476874155486356\\
452.01	0.00476344820622057\\
453.01	0.0047580460127148\\
454.01	0.00475253272729457\\
455.01	0.00474690605691219\\
456.01	0.00474116366157828\\
457.01	0.00473530315339724\\
458.01	0.00472932209558291\\
459.01	0.00472321800145414\\
460.01	0.0047169883334113\\
461.01	0.00471063050189128\\
462.01	0.00470414186430254\\
463.01	0.0046975197239378\\
464.01	0.00469076132886535\\
465.01	0.00468386387079798\\
466.01	0.00467682448393921\\
467.01	0.00466964024380527\\
468.01	0.00466230816602322\\
469.01	0.00465482520510529\\
470.01	0.00464718825319733\\
471.01	0.00463939413880171\\
472.01	0.00463143962547456\\
473.01	0.00462332141049567\\
474.01	0.00461503612351281\\
475.01	0.0046065803251571\\
476.01	0.00459795050563165\\
477.01	0.00458914308327175\\
478.01	0.00458015440307578\\
479.01	0.00457098073520809\\
480.01	0.00456161827347155\\
481.01	0.00455206313375069\\
482.01	0.00454231135242416\\
483.01	0.00453235888474604\\
484.01	0.00452220160319721\\
485.01	0.00451183529580319\\
486.01	0.00450125566442095\\
487.01	0.00449045832299321\\
488.01	0.00447943879576871\\
489.01	0.00446819251548963\\
490.01	0.00445671482154523\\
491.01	0.00444500095808993\\
492.01	0.00443304607212781\\
493.01	0.00442084521156108\\
494.01	0.00440839332320334\\
495.01	0.00439568525075659\\
496.01	0.00438271573275181\\
497.01	0.0043694794004541\\
498.01	0.00435597077572967\\
499.01	0.0043421842688769\\
500.01	0.00432811417642031\\
501.01	0.0043137546788676\\
502.01	0.00429909983842902\\
503.01	0.0042841435967016\\
504.01	0.00426887977231503\\
505.01	0.00425330205854258\\
506.01	0.00423740402087563\\
507.01	0.00422117909456325\\
508.01	0.00420462058211687\\
509.01	0.00418772165078221\\
510.01	0.00417047532997789\\
511.01	0.00415287450870299\\
512.01	0.00413491193291534\\
513.01	0.00411658020288142\\
514.01	0.00409787177050073\\
515.01	0.00407877893660669\\
516.01	0.00405929384824654\\
517.01	0.00403940849594464\\
518.01	0.00401911471095102\\
519.01	0.00399840416248059\\
520.01	0.00397726835494781\\
521.01	0.00395569862520106\\
522.01	0.00393368613976365\\
523.01	0.00391122189208767\\
524.01	0.00388829669982948\\
525.01	0.00386490120215339\\
526.01	0.00384102585707547\\
527.01	0.00381666093885656\\
528.01	0.00379179653545806\\
529.01	0.00376642254607224\\
530.01	0.00374052867874502\\
531.01	0.003714104448105\\
532.01	0.00368713917322073\\
533.01	0.00365962197560618\\
534.01	0.00363154177739699\\
535.01	0.00360288729972611\\
536.01	0.00357364706132669\\
537.01	0.00354380937739529\\
538.01	0.00351336235875247\\
539.01	0.00348229391134084\\
540.01	0.00345059173610505\\
541.01	0.00341824332930561\\
542.01	0.00338523598332041\\
543.01	0.00335155678799634\\
544.01	0.00331719263262028\\
545.01	0.00328213020858442\\
546.01	0.00324635601283099\\
547.01	0.00320985635216906\\
548.01	0.00317261734856691\\
549.01	0.0031346249455342\\
550.01	0.00309586491572126\\
551.01	0.00305632286987125\\
552.01	0.00301598426728306\\
553.01	0.00297483442795171\\
554.01	0.00293285854657185\\
555.01	0.00289004170861091\\
556.01	0.00284636890867354\\
557.01	0.00280182507140488\\
558.01	0.00275639507519943\\
559.01	0.00271006377900744\\
560.01	0.0026628160525571\\
561.01	0.00261463681033538\\
562.01	0.00256551104969777\\
563.01	0.00251542389350631\\
564.01	0.00246436063771941\\
565.01	0.00241230680438499\\
566.01	0.00235924820051124\\
567.01	0.00230517098330829\\
568.01	0.00225006173230929\\
569.01	0.00219390752888354\\
570.01	0.00213669604364802\\
571.01	0.00207841563226292\\
572.01	0.00201905544004984\\
573.01	0.00195860551580188\\
574.01	0.00189705693504289\\
575.01	0.0018344019328357\\
576.01	0.00177063404601639\\
577.01	0.00170574826442997\\
578.01	0.0016397411903371\\
579.01	0.00157261120462172\\
580.01	0.00150435863772479\\
581.01	0.00143498594231021\\
582.01	0.00136449786348202\\
583.01	0.00129290160084457\\
584.01	0.00122020695474439\\
585.01	0.00114642644654167\\
586.01	0.00107157539959435\\
587.01	0.000995671963626618\\
588.01	0.000918737060083706\\
589.01	0.000840794219677079\\
590.01	0.000761869275267243\\
591.01	0.000681989863100538\\
592.01	0.000601184672692693\\
593.01	0.000519482369694076\\
594.01	0.000436910096074035\\
595.01	0.000353491426927423\\
596.01	0.000269243631889871\\
597.01	0.000184174050000235\\
598.01	9.85575539648349e-05\\
599.01	3.18230442563731e-05\\
599.02	3.12770957175568e-05\\
599.03	3.07343620567866e-05\\
599.04	3.0194875217571e-05\\
599.05	2.96586674565693e-05\\
599.06	2.91257713466771e-05\\
599.07	2.85962197801391e-05\\
599.08	2.80700459716933e-05\\
599.09	2.75472834617464e-05\\
599.1	2.70279661195791e-05\\
599.11	2.65121281465865e-05\\
599.12	2.59998040795448e-05\\
599.13	2.54910287939124e-05\\
599.14	2.49858375071712e-05\\
599.15	2.44842657821827e-05\\
599.16	2.39863495305973e-05\\
599.17	2.34921250162837e-05\\
599.18	2.30016288588035e-05\\
599.19	2.25148980369013e-05\\
599.2	2.20319698920526e-05\\
599.21	2.15528821320247e-05\\
599.22	2.10776728344908e-05\\
599.23	2.06063804506721e-05\\
599.24	2.01390438090126e-05\\
599.25	1.96757021188858e-05\\
599.26	1.92163949743647e-05\\
599.27	1.87611623579872e-05\\
599.28	1.83100446445959e-05\\
599.29	1.78630826051952e-05\\
599.3	1.74203174108517e-05\\
599.31	1.69817906366353e-05\\
599.32	1.65475442655931e-05\\
599.33	1.61176206927693e-05\\
599.34	1.56920627292639e-05\\
599.35	1.52709136063203e-05\\
599.36	1.48542169794725e-05\\
599.37	1.44420169327208e-05\\
599.38	1.40343579827368e-05\\
599.39	1.36312859119591e-05\\
599.4	1.32328499381877e-05\\
599.41	1.28390997671604e-05\\
599.42	1.24500855973528e-05\\
599.43	1.20658581248111e-05\\
599.44	1.16864685480531e-05\\
599.45	1.13119685730082e-05\\
599.46	1.09424104179929e-05\\
599.47	1.05778468187639e-05\\
599.48	1.02183310335888e-05\\
599.49	9.86391684839293e-06\\
599.5	9.51465858194112e-06\\
599.51	9.17061109107116e-06\\
599.52	8.83182977599525e-06\\
599.53	8.4983705856221e-06\\
599.54	8.1702900229675e-06\\
599.55	7.84764515058753e-06\\
599.56	7.53049359608279e-06\\
599.57	7.21889355765649e-06\\
599.58	6.91290380971064e-06\\
599.59	6.61258370851688e-06\\
599.6	6.31799319793583e-06\\
599.61	6.02919281519031e-06\\
599.62	5.74624369668701e-06\\
599.63	5.46920758391981e-06\\
599.64	5.19814682940593e-06\\
599.65	4.93312440269511e-06\\
599.66	4.67420389643411e-06\\
599.67	4.42144953247646e-06\\
599.68	4.17492616808755e-06\\
599.69	3.9346993021671e-06\\
599.7	3.70083508156871e-06\\
599.71	3.47340030746116e-06\\
599.72	3.25246244175549e-06\\
599.73	3.03808961359987e-06\\
599.74	2.83035062593855e-06\\
599.75	2.62931496212288e-06\\
599.76	2.43505279260738e-06\\
599.77	2.24763498169606e-06\\
599.78	2.06713309435641e-06\\
599.79	1.89361940310974e-06\\
599.8	1.72716689497872e-06\\
599.81	1.56784927850956e-06\\
599.82	1.41574099085315e-06\\
599.83	1.27091720493987e-06\\
599.84	1.13345383668563e-06\\
599.85	1.00342755231589e-06\\
599.86	8.80915775712185e-07\\
599.87	7.65996695880136e-07\\
599.88	6.5874927444344e-07\\
599.89	5.59253253243699e-07\\
599.9	4.67589162013102e-07\\
599.91	3.8383832609741e-07\\
599.92	3.0808287429171e-07\\
599.93	2.4040574671258e-07\\
599.94	1.80890702777825e-07\\
599.95	1.2962232925906e-07\\
599.96	8.66860484019516e-08\\
599.97	5.21681261331924e-08\\
599.98	2.61556803542173e-08\\
599.99	8.73668929909921e-09\\
600	0\\
};
\addplot [color=black!80!mycolor21,solid,forget plot]
  table[row sep=crcr]{%
0.01	0.00502843436055783\\
1.01	0.00502843349929418\\
2.01	0.00502843262123861\\
3.01	0.0050284317260663\\
4.01	0.00502843081344614\\
5.01	0.00502842988304058\\
6.01	0.00502842893450587\\
7.01	0.00502842796749136\\
8.01	0.00502842698163999\\
9.01	0.00502842597658766\\
10.01	0.00502842495196327\\
11.01	0.00502842390738872\\
12.01	0.0050284228424786\\
13.01	0.00502842175684009\\
14.01	0.00502842065007276\\
15.01	0.00502841952176892\\
16.01	0.0050284183715127\\
17.01	0.00502841719888041\\
18.01	0.00502841600344017\\
19.01	0.00502841478475189\\
20.01	0.00502841354236696\\
21.01	0.00502841227582853\\
22.01	0.00502841098467075\\
23.01	0.00502840966841888\\
24.01	0.00502840832658917\\
25.01	0.00502840695868844\\
26.01	0.00502840556421434\\
27.01	0.00502840414265467\\
28.01	0.00502840269348778\\
29.01	0.00502840121618183\\
30.01	0.0050283997101948\\
31.01	0.00502839817497453\\
32.01	0.00502839660995804\\
33.01	0.00502839501457191\\
34.01	0.00502839338823163\\
35.01	0.00502839173034153\\
36.01	0.00502839004029431\\
37.01	0.00502838831747177\\
38.01	0.00502838656124318\\
39.01	0.00502838477096642\\
40.01	0.00502838294598677\\
41.01	0.00502838108563712\\
42.01	0.00502837918923754\\
43.01	0.00502837725609535\\
44.01	0.0050283752855046\\
45.01	0.00502837327674568\\
46.01	0.00502837122908562\\
47.01	0.00502836914177741\\
48.01	0.00502836701405978\\
49.01	0.00502836484515692\\
50.01	0.00502836263427834\\
51.01	0.00502836038061843\\
52.01	0.00502835808335623\\
53.01	0.00502835574165529\\
54.01	0.00502835335466321\\
55.01	0.00502835092151147\\
56.01	0.00502834844131502\\
57.01	0.00502834591317168\\
58.01	0.00502834333616256\\
59.01	0.00502834070935102\\
60.01	0.00502833803178272\\
61.01	0.00502833530248539\\
62.01	0.00502833252046801\\
63.01	0.0050283296847209\\
64.01	0.0050283267942151\\
65.01	0.00502832384790237\\
66.01	0.0050283208447142\\
67.01	0.00502831778356208\\
68.01	0.00502831466333673\\
69.01	0.00502831148290791\\
70.01	0.00502830824112375\\
71.01	0.00502830493681093\\
72.01	0.00502830156877361\\
73.01	0.0050282981357933\\
74.01	0.00502829463662843\\
75.01	0.00502829107001377\\
76.01	0.00502828743466048\\
77.01	0.00502828372925505\\
78.01	0.00502827995245877\\
79.01	0.0050282761029083\\
80.01	0.00502827217921379\\
81.01	0.00502826817995968\\
82.01	0.00502826410370298\\
83.01	0.00502825994897383\\
84.01	0.0050282557142745\\
85.01	0.00502825139807869\\
86.01	0.00502824699883142\\
87.01	0.0050282425149483\\
88.01	0.00502823794481481\\
89.01	0.00502823328678589\\
90.01	0.00502822853918555\\
91.01	0.00502822370030597\\
92.01	0.00502821876840701\\
93.01	0.00502821374171555\\
94.01	0.00502820861842523\\
95.01	0.00502820339669533\\
96.01	0.00502819807465029\\
97.01	0.00502819265037929\\
98.01	0.00502818712193531\\
99.01	0.00502818148733451\\
100.01	0.00502817574455551\\
101.01	0.00502816989153885\\
102.01	0.00502816392618621\\
103.01	0.00502815784635948\\
104.01	0.00502815164988006\\
105.01	0.00502814533452847\\
106.01	0.00502813889804326\\
107.01	0.00502813233812017\\
108.01	0.00502812565241156\\
109.01	0.0050281188385253\\
110.01	0.00502811189402423\\
111.01	0.00502810481642512\\
112.01	0.00502809760319793\\
113.01	0.00502809025176476\\
114.01	0.00502808275949907\\
115.01	0.00502807512372475\\
116.01	0.00502806734171527\\
117.01	0.00502805941069259\\
118.01	0.00502805132782616\\
119.01	0.0050280430902323\\
120.01	0.00502803469497239\\
121.01	0.00502802613905277\\
122.01	0.00502801741942322\\
123.01	0.00502800853297598\\
124.01	0.00502799947654461\\
125.01	0.00502799024690289\\
126.01	0.00502798084076383\\
127.01	0.00502797125477838\\
128.01	0.00502796148553406\\
129.01	0.00502795152955438\\
130.01	0.00502794138329717\\
131.01	0.00502793104315328\\
132.01	0.00502792050544566\\
133.01	0.00502790976642754\\
134.01	0.00502789882228181\\
135.01	0.00502788766911928\\
136.01	0.00502787630297728\\
137.01	0.00502786471981833\\
138.01	0.00502785291552887\\
139.01	0.00502784088591756\\
140.01	0.00502782862671417\\
141.01	0.00502781613356767\\
142.01	0.00502780340204514\\
143.01	0.00502779042762982\\
144.01	0.00502777720571961\\
145.01	0.0050277637316257\\
146.01	0.00502775000057084\\
147.01	0.00502773600768727\\
148.01	0.00502772174801552\\
149.01	0.00502770721650234\\
150.01	0.00502769240799919\\
151.01	0.00502767731726028\\
152.01	0.00502766193894071\\
153.01	0.00502764626759434\\
154.01	0.00502763029767243\\
155.01	0.00502761402352129\\
156.01	0.00502759743938044\\
157.01	0.00502758053938022\\
158.01	0.00502756331754031\\
159.01	0.00502754576776728\\
160.01	0.00502752788385231\\
161.01	0.00502750965946929\\
162.01	0.00502749108817226\\
163.01	0.00502747216339341\\
164.01	0.00502745287844059\\
165.01	0.00502743322649504\\
166.01	0.00502741320060887\\
167.01	0.00502739279370249\\
168.01	0.00502737199856242\\
169.01	0.00502735080783846\\
170.01	0.00502732921404082\\
171.01	0.00502730720953808\\
172.01	0.00502728478655421\\
173.01	0.00502726193716557\\
174.01	0.00502723865329805\\
175.01	0.00502721492672447\\
176.01	0.00502719074906172\\
177.01	0.00502716611176715\\
178.01	0.00502714100613615\\
179.01	0.00502711542329865\\
180.01	0.00502708935421598\\
181.01	0.00502706278967793\\
182.01	0.00502703572029884\\
183.01	0.00502700813651456\\
184.01	0.00502698002857915\\
185.01	0.00502695138656097\\
186.01	0.00502692220033937\\
187.01	0.00502689245960104\\
188.01	0.00502686215383585\\
189.01	0.0050268312723334\\
190.01	0.00502679980417929\\
191.01	0.00502676773825054\\
192.01	0.00502673506321213\\
193.01	0.00502670176751231\\
194.01	0.00502666783937896\\
195.01	0.00502663326681498\\
196.01	0.00502659803759368\\
197.01	0.00502656213925446\\
198.01	0.00502652555909849\\
199.01	0.00502648828418351\\
200.01	0.00502645030131966\\
201.01	0.0050264115970639\\
202.01	0.00502637215771571\\
203.01	0.00502633196931158\\
204.01	0.0050262910176199\\
205.01	0.0050262492881358\\
206.01	0.00502620676607569\\
207.01	0.00502616343637166\\
208.01	0.0050261192836659\\
209.01	0.00502607429230486\\
210.01	0.00502602844633365\\
211.01	0.00502598172948986\\
212.01	0.0050259341251974\\
213.01	0.00502588561656033\\
214.01	0.00502583618635638\\
215.01	0.00502578581703054\\
216.01	0.00502573449068861\\
217.01	0.00502568218908995\\
218.01	0.00502562889364079\\
219.01	0.00502557458538716\\
220.01	0.00502551924500772\\
221.01	0.0050254628528063\\
222.01	0.0050254053887043\\
223.01	0.00502534683223293\\
224.01	0.00502528716252563\\
225.01	0.00502522635830983\\
226.01	0.00502516439789862\\
227.01	0.00502510125918259\\
228.01	0.00502503691962126\\
229.01	0.00502497135623418\\
230.01	0.00502490454559219\\
231.01	0.00502483646380783\\
232.01	0.00502476708652681\\
233.01	0.00502469638891764\\
234.01	0.00502462434566235\\
235.01	0.00502455093094678\\
236.01	0.00502447611844968\\
237.01	0.0050243998813329\\
238.01	0.005024322192231\\
239.01	0.00502424302323956\\
240.01	0.00502416234590481\\
241.01	0.00502408013121223\\
242.01	0.0050239963495747\\
243.01	0.00502391097082081\\
244.01	0.00502382396418274\\
245.01	0.00502373529828444\\
246.01	0.00502364494112798\\
247.01	0.00502355286008181\\
248.01	0.00502345902186676\\
249.01	0.005023363392543\\
250.01	0.00502326593749602\\
251.01	0.00502316662142288\\
252.01	0.00502306540831744\\
253.01	0.00502296226145618\\
254.01	0.00502285714338281\\
255.01	0.00502275001589302\\
256.01	0.00502264084001892\\
257.01	0.00502252957601304\\
258.01	0.00502241618333144\\
259.01	0.00502230062061778\\
260.01	0.00502218284568591\\
261.01	0.00502206281550182\\
262.01	0.00502194048616658\\
263.01	0.00502181581289769\\
264.01	0.00502168875001045\\
265.01	0.00502155925089892\\
266.01	0.005021427268016\\
267.01	0.00502129275285441\\
268.01	0.00502115565592534\\
269.01	0.00502101592673802\\
270.01	0.00502087351377883\\
271.01	0.00502072836448901\\
272.01	0.00502058042524221\\
273.01	0.00502042964132244\\
274.01	0.00502027595690047\\
275.01	0.00502011931500993\\
276.01	0.00501995965752344\\
277.01	0.00501979692512705\\
278.01	0.00501963105729606\\
279.01	0.00501946199226796\\
280.01	0.00501928966701663\\
281.01	0.00501911401722497\\
282.01	0.00501893497725712\\
283.01	0.00501875248013079\\
284.01	0.00501856645748749\\
285.01	0.00501837683956397\\
286.01	0.00501818355516119\\
287.01	0.00501798653161407\\
288.01	0.00501778569475948\\
289.01	0.00501758096890471\\
290.01	0.00501737227679403\\
291.01	0.00501715953957528\\
292.01	0.00501694267676547\\
293.01	0.00501672160621599\\
294.01	0.00501649624407649\\
295.01	0.00501626650475819\\
296.01	0.00501603230089676\\
297.01	0.00501579354331405\\
298.01	0.00501555014097906\\
299.01	0.005015302000968\\
300.01	0.00501504902842402\\
301.01	0.00501479112651473\\
302.01	0.00501452819639081\\
303.01	0.00501426013714194\\
304.01	0.00501398684575311\\
305.01	0.00501370821705909\\
306.01	0.00501342414369801\\
307.01	0.00501313451606501\\
308.01	0.00501283922226349\\
309.01	0.0050125381480563\\
310.01	0.00501223117681559\\
311.01	0.00501191818947187\\
312.01	0.00501159906446138\\
313.01	0.00501127367767312\\
314.01	0.00501094190239442\\
315.01	0.00501060360925566\\
316.01	0.00501025866617318\\
317.01	0.00500990693829178\\
318.01	0.00500954828792586\\
319.01	0.00500918257449919\\
320.01	0.00500880965448356\\
321.01	0.00500842938133637\\
322.01	0.00500804160543641\\
323.01	0.00500764617401919\\
324.01	0.00500724293111037\\
325.01	0.00500683171745786\\
326.01	0.005006412370463\\
327.01	0.00500598472411009\\
328.01	0.00500554860889441\\
329.01	0.00500510385174934\\
330.01	0.005004650275971\\
331.01	0.00500418770114295\\
332.01	0.00500371594305808\\
333.01	0.00500323481363959\\
334.01	0.00500274412086025\\
335.01	0.00500224366866026\\
336.01	0.0050017332568633\\
337.01	0.0050012126810911\\
338.01	0.0050006817326763\\
339.01	0.0050001401985734\\
340.01	0.00499958786126887\\
341.01	0.00499902449868812\\
342.01	0.00499844988410169\\
343.01	0.00499786378602948\\
344.01	0.00499726596814258\\
345.01	0.00499665618916393\\
346.01	0.00499603420276663\\
347.01	0.00499539975746997\\
348.01	0.00499475259653424\\
349.01	0.0049940924578527\\
350.01	0.00499341907384166\\
351.01	0.00499273217132882\\
352.01	0.00499203147143882\\
353.01	0.00499131668947634\\
354.01	0.00499058753480788\\
355.01	0.00498984371074027\\
356.01	0.0049890849143966\\
357.01	0.00498831083659057\\
358.01	0.00498752116169746\\
359.01	0.00498671556752294\\
360.01	0.00498589372516844\\
361.01	0.00498505529889489\\
362.01	0.00498419994598239\\
363.01	0.00498332731658825\\
364.01	0.00498243705360065\\
365.01	0.00498152879249033\\
366.01	0.00498060216115879\\
367.01	0.00497965677978277\\
368.01	0.00497869226065643\\
369.01	0.00497770820802893\\
370.01	0.00497670421793944\\
371.01	0.00497567987804857\\
372.01	0.00497463476746508\\
373.01	0.00497356845657062\\
374.01	0.0049724805068389\\
375.01	0.00497137047065233\\
376.01	0.00497023789111349\\
377.01	0.00496908230185381\\
378.01	0.00496790322683697\\
379.01	0.00496670018015875\\
380.01	0.00496547266584231\\
381.01	0.00496422017762947\\
382.01	0.00496294219876715\\
383.01	0.00496163820178908\\
384.01	0.00496030764829392\\
385.01	0.00495894998871806\\
386.01	0.00495756466210336\\
387.01	0.00495615109586128\\
388.01	0.00495470870553127\\
389.01	0.00495323689453462\\
390.01	0.0049517350539238\\
391.01	0.00495020256212605\\
392.01	0.00494863878468264\\
393.01	0.00494704307398262\\
394.01	0.00494541476899205\\
395.01	0.00494375319497745\\
396.01	0.00494205766322433\\
397.01	0.00494032747074996\\
398.01	0.00493856190001135\\
399.01	0.00493676021860735\\
400.01	0.00493492167897478\\
401.01	0.00493304551807963\\
402.01	0.00493113095710157\\
403.01	0.00492917720111273\\
404.01	0.00492718343875042\\
405.01	0.00492514884188331\\
406.01	0.00492307256527112\\
407.01	0.00492095374621756\\
408.01	0.00491879150421652\\
409.01	0.004916584940591\\
410.01	0.00491433313812519\\
411.01	0.00491203516068845\\
412.01	0.0049096900528525\\
413.01	0.00490729683949968\\
414.01	0.00490485452542453\\
415.01	0.0049023620949264\\
416.01	0.00489981851139374\\
417.01	0.00489722271688028\\
418.01	0.00489457363167176\\
419.01	0.00489187015384463\\
420.01	0.00488911115881449\\
421.01	0.00488629549887635\\
422.01	0.00488342200273419\\
423.01	0.00488048947502156\\
424.01	0.00487749669581191\\
425.01	0.00487444242011829\\
426.01	0.00487132537738289\\
427.01	0.00486814427095616\\
428.01	0.00486489777756404\\
429.01	0.00486158454676542\\
430.01	0.00485820320039651\\
431.01	0.0048547523320048\\
432.01	0.0048512305062708\\
433.01	0.00484763625841695\\
434.01	0.00484396809360503\\
435.01	0.00484022448632022\\
436.01	0.00483640387974204\\
437.01	0.00483250468510353\\
438.01	0.00482852528103479\\
439.01	0.00482446401289458\\
440.01	0.00482031919208718\\
441.01	0.00481608909536478\\
442.01	0.0048117719641162\\
443.01	0.00480736600364003\\
444.01	0.00480286938240266\\
445.01	0.0047982802312817\\
446.01	0.00479359664279346\\
447.01	0.00478881667030427\\
448.01	0.00478393832722535\\
449.01	0.00477895958619237\\
450.01	0.00477387837822654\\
451.01	0.00476869259188027\\
452.01	0.00476340007236439\\
453.01	0.00475799862065733\\
454.01	0.00475248599259701\\
455.01	0.00474685989795386\\
456.01	0.00474111799948513\\
457.01	0.00473525791196947\\
458.01	0.00472927720122272\\
459.01	0.00472317338309335\\
460.01	0.00471694392243755\\
461.01	0.00471058623207308\\
462.01	0.00470409767171277\\
463.01	0.00469747554687555\\
464.01	0.00469071710777627\\
465.01	0.00468381954819199\\
466.01	0.00467678000430614\\
467.01	0.00466959555352891\\
468.01	0.00466226321329406\\
469.01	0.00465477993983148\\
470.01	0.00464714262691423\\
471.01	0.00463934810458196\\
472.01	0.00463139313783721\\
473.01	0.00462327442531657\\
474.01	0.00461498859793493\\
475.01	0.00460653221750294\\
476.01	0.00459790177531674\\
477.01	0.00458909369072005\\
478.01	0.0045801043096378\\
479.01	0.00457092990308097\\
480.01	0.00456156666562179\\
481.01	0.00455201071383896\\
482.01	0.00454225808473339\\
483.01	0.00453230473411217\\
484.01	0.00452214653494158\\
485.01	0.00451177927566853\\
486.01	0.00450119865850974\\
487.01	0.00449040029770726\\
488.01	0.00447937971775215\\
489.01	0.00446813235157375\\
490.01	0.00445665353869464\\
491.01	0.00444493852335173\\
492.01	0.00443298245258193\\
493.01	0.00442078037427268\\
494.01	0.00440832723517685\\
495.01	0.00439561787889161\\
496.01	0.00438264704380088\\
497.01	0.00436940936098138\\
498.01	0.00435589935207233\\
499.01	0.00434211142710735\\
500.01	0.00432803988231031\\
501.01	0.00431367889785326\\
502.01	0.00429902253557849\\
503.01	0.00428406473668267\\
504.01	0.00426879931936507\\
505.01	0.00425321997643889\\
506.01	0.00423732027290725\\
507.01	0.00422109364350371\\
508.01	0.00420453339019825\\
509.01	0.00418763267966946\\
510.01	0.00417038454074419\\
511.01	0.0041527818618063\\
512.01	0.00413481738817535\\
513.01	0.0041164837194568\\
514.01	0.00409777330686782\\
515.01	0.00407867845053805\\
516.01	0.00405919129679085\\
517.01	0.00403930383540636\\
518.01	0.00401900789687119\\
519.01	0.00399829514961824\\
520.01	0.00397715709726165\\
521.01	0.00395558507583142\\
522.01	0.00393357025101552\\
523.01	0.00391110361541448\\
524.01	0.00388817598581671\\
525.01	0.0038647780005037\\
526.01	0.00384090011659382\\
527.01	0.00381653260743666\\
528.01	0.00379166556006887\\
529.01	0.00376628887274622\\
530.01	0.00374039225256658\\
531.01	0.00371396521320085\\
532.01	0.00368699707275148\\
533.01	0.00365947695175801\\
534.01	0.00363139377137604\\
535.01	0.00360273625175423\\
536.01	0.00357349291063878\\
537.01	0.00354365206223945\\
538.01	0.00351320181639204\\
539.01	0.00348213007805842\\
540.01	0.00345042454721088\\
541.01	0.00341807271914801\\
542.01	0.00338506188529999\\
543.01	0.00335137913458498\\
544.01	0.00331701135538371\\
545.01	0.00328194523821044\\
546.01	0.00324616727916235\\
547.01	0.00320966378424351\\
548.01	0.00317242087466379\\
549.01	0.00313442449322877\\
550.01	0.00309566041194589\\
551.01	0.00305611424098788\\
552.01	0.00301577143916516\\
553.01	0.0029746173260766\\
554.01	0.00293263709612681\\
555.01	0.00288981583461184\\
556.01	0.0028461385360993\\
557.01	0.00280159012534766\\
558.01	0.00275615548103306\\
559.01	0.00270981946257547\\
560.01	0.00266256694038211\\
561.01	0.0026143828298511\\
562.01	0.00256525212950647\\
563.01	0.00251515996366286\\
564.01	0.00246409163004402\\
565.01	0.00241203265280651\\
566.01	0.00235896884144259\\
567.01	0.00230488635605555\\
568.01	0.00224977177951561\\
569.01	0.00219361219700808\\
570.01	0.00213639528348043\\
571.01	0.00207810939947119\\
572.01	0.00201874369576008\\
573.01	0.00195828822720497\\
574.01	0.00189673407602165\\
575.01	0.00183407348460219\\
576.01	0.0017702999977462\\
577.01	0.00170540861387421\\
578.01	0.00163939594438615\\
579.01	0.00157226037978655\\
580.01	0.00150400226049134\\
581.01	0.00143462404930831\\
582.01	0.00136413050139218\\
583.01	0.00129252882594572\\
584.01	0.00121982883197683\\
585.01	0.00114604304792577\\
586.01	0.00107118680180153\\
587.01	0.000995278244445114\\
588.01	0.000918338293453008\\
589.01	0.0008403904688796\\
590.01	0.000761460583759315\\
591.01	0.000681576242330336\\
592.01	0.00060076608608719\\
593.01	0.000519058711787984\\
594.01	0.000436481165494431\\
595.01	0.000353056891624152\\
596.01	0.000268802984598084\\
597.01	0.000183726551418536\\
598.01	9.84661228899224e-05\\
599.01	3.18230442563749e-05\\
599.02	3.12770957175551e-05\\
599.03	3.07343620567866e-05\\
599.04	3.0194875217571e-05\\
599.05	2.9658667456571e-05\\
599.06	2.91257713466771e-05\\
599.07	2.85962197801391e-05\\
599.08	2.80700459716916e-05\\
599.09	2.75472834617447e-05\\
599.1	2.70279661195791e-05\\
599.11	2.65121281465865e-05\\
599.12	2.59998040795448e-05\\
599.13	2.54910287939142e-05\\
599.14	2.49858375071712e-05\\
599.15	2.4484265782181e-05\\
599.16	2.39863495305973e-05\\
599.17	2.34921250162855e-05\\
599.18	2.30016288588052e-05\\
599.19	2.25148980369013e-05\\
599.2	2.20319698920508e-05\\
599.21	2.1552882132023e-05\\
599.22	2.10776728344891e-05\\
599.23	2.06063804506721e-05\\
599.24	2.01390438090109e-05\\
599.25	1.96757021188858e-05\\
599.26	1.92163949743647e-05\\
599.27	1.87611623579872e-05\\
599.28	1.83100446445959e-05\\
599.29	1.78630826051952e-05\\
599.3	1.74203174108517e-05\\
599.31	1.69817906366335e-05\\
599.32	1.65475442655914e-05\\
599.33	1.61176206927693e-05\\
599.34	1.56920627292622e-05\\
599.35	1.52709136063186e-05\\
599.36	1.48542169794742e-05\\
599.37	1.4442016932719e-05\\
599.38	1.40343579827385e-05\\
599.39	1.36312859119608e-05\\
599.4	1.3232849938186e-05\\
599.41	1.28390997671621e-05\\
599.42	1.2450085597351e-05\\
599.43	1.20658581248094e-05\\
599.44	1.16864685480531e-05\\
599.45	1.13119685730065e-05\\
599.46	1.09424104179946e-05\\
599.47	1.05778468187639e-05\\
599.48	1.02183310335905e-05\\
599.49	9.86391684839293e-06\\
599.5	9.51465858193938e-06\\
599.51	9.17061109107289e-06\\
599.52	8.83182977599352e-06\\
599.53	8.4983705856221e-06\\
599.54	8.1702900229675e-06\\
599.55	7.84764515058579e-06\\
599.56	7.53049359608279e-06\\
599.57	7.21889355765649e-06\\
599.58	6.91290380971064e-06\\
599.59	6.61258370851688e-06\\
599.6	6.31799319793756e-06\\
599.61	6.02919281519031e-06\\
599.62	5.74624369668528e-06\\
599.63	5.46920758391981e-06\\
599.64	5.19814682940767e-06\\
599.65	4.93312440269685e-06\\
599.66	4.67420389643237e-06\\
599.67	4.42144953247646e-06\\
599.68	4.17492616808582e-06\\
599.69	3.93469930216536e-06\\
599.7	3.70083508156871e-06\\
599.71	3.47340030746116e-06\\
599.72	3.25246244175549e-06\\
599.73	3.03808961360161e-06\\
599.74	2.83035062593855e-06\\
599.75	2.62931496212288e-06\\
599.76	2.43505279260911e-06\\
599.77	2.24763498169606e-06\\
599.78	2.06713309435641e-06\\
599.79	1.89361940310974e-06\\
599.8	1.72716689498045e-06\\
599.81	1.56784927850782e-06\\
599.82	1.41574099085315e-06\\
599.83	1.27091720493813e-06\\
599.84	1.13345383668563e-06\\
599.85	1.00342755231589e-06\\
599.86	8.8091577571392e-07\\
599.87	7.65996695880136e-07\\
599.88	6.58749274441706e-07\\
599.89	5.59253253241965e-07\\
599.9	4.67589162011367e-07\\
599.91	3.83838326099145e-07\\
599.92	3.0808287429171e-07\\
599.93	2.40405746710845e-07\\
599.94	1.8089070277609e-07\\
599.95	1.29622329257326e-07\\
599.96	8.66860484019516e-08\\
599.97	5.21681261349272e-08\\
599.98	2.61556803542173e-08\\
599.99	8.73668930083393e-09\\
600	0\\
};
\addplot [color=black,solid,forget plot]
  table[row sep=crcr]{%
0.01	0.00502703576604918\\
1.01	0.00502703497172564\\
2.01	0.00502703416200814\\
3.01	0.00502703333659998\\
4.01	0.0050270324951992\\
5.01	0.00502703163749766\\
6.01	0.00502703076318172\\
7.01	0.00502702987193134\\
8.01	0.0050270289634203\\
9.01	0.00502702803731684\\
10.01	0.00502702709328209\\
11.01	0.00502702613097098\\
12.01	0.00502702515003172\\
13.01	0.00502702415010615\\
14.01	0.00502702313082892\\
15.01	0.00502702209182774\\
16.01	0.00502702103272314\\
17.01	0.00502701995312872\\
18.01	0.00502701885265022\\
19.01	0.00502701773088608\\
20.01	0.00502701658742719\\
21.01	0.00502701542185629\\
22.01	0.00502701423374847\\
23.01	0.00502701302267043\\
24.01	0.00502701178818072\\
25.01	0.00502701052982919\\
26.01	0.00502700924715755\\
27.01	0.00502700793969848\\
28.01	0.00502700660697566\\
29.01	0.00502700524850406\\
30.01	0.00502700386378847\\
31.01	0.00502700245232527\\
32.01	0.00502700101360067\\
33.01	0.00502699954709104\\
34.01	0.0050269980522628\\
35.01	0.00502699652857234\\
36.01	0.00502699497546558\\
37.01	0.00502699339237751\\
38.01	0.0050269917787328\\
39.01	0.00502699013394496\\
40.01	0.00502698845741587\\
41.01	0.00502698674853659\\
42.01	0.00502698500668635\\
43.01	0.00502698323123265\\
44.01	0.0050269814215304\\
45.01	0.0050269795769228\\
46.01	0.00502697769673993\\
47.01	0.00502697578029949\\
48.01	0.00502697382690603\\
49.01	0.00502697183585049\\
50.01	0.00502696980641097\\
51.01	0.00502696773785102\\
52.01	0.00502696562942034\\
53.01	0.00502696348035482\\
54.01	0.00502696128987477\\
55.01	0.00502695905718655\\
56.01	0.00502695678148061\\
57.01	0.00502695446193257\\
58.01	0.0050269520977019\\
59.01	0.00502694968793225\\
60.01	0.00502694723175042\\
61.01	0.00502694472826718\\
62.01	0.00502694217657556\\
63.01	0.0050269395757521\\
64.01	0.00502693692485474\\
65.01	0.00502693422292425\\
66.01	0.00502693146898289\\
67.01	0.00502692866203387\\
68.01	0.00502692580106153\\
69.01	0.0050269228850307\\
70.01	0.00502691991288679\\
71.01	0.00502691688355467\\
72.01	0.00502691379593868\\
73.01	0.00502691064892259\\
74.01	0.00502690744136828\\
75.01	0.00502690417211634\\
76.01	0.00502690083998477\\
77.01	0.00502689744376923\\
78.01	0.00502689398224234\\
79.01	0.00502689045415315\\
80.01	0.00502688685822671\\
81.01	0.00502688319316384\\
82.01	0.00502687945764056\\
83.01	0.00502687565030747\\
84.01	0.00502687176978926\\
85.01	0.00502686781468428\\
86.01	0.00502686378356412\\
87.01	0.00502685967497303\\
88.01	0.00502685548742783\\
89.01	0.00502685121941631\\
90.01	0.00502684686939733\\
91.01	0.00502684243580049\\
92.01	0.00502683791702561\\
93.01	0.00502683331144142\\
94.01	0.00502682861738518\\
95.01	0.00502682383316288\\
96.01	0.00502681895704779\\
97.01	0.00502681398728019\\
98.01	0.00502680892206636\\
99.01	0.00502680375957853\\
100.01	0.00502679849795363\\
101.01	0.00502679313529323\\
102.01	0.0050267876696622\\
103.01	0.00502678209908856\\
104.01	0.00502677642156221\\
105.01	0.00502677063503451\\
106.01	0.00502676473741767\\
107.01	0.00502675872658393\\
108.01	0.00502675260036436\\
109.01	0.00502674635654883\\
110.01	0.00502673999288452\\
111.01	0.0050267335070753\\
112.01	0.0050267268967811\\
113.01	0.00502672015961691\\
114.01	0.00502671329315179\\
115.01	0.00502670629490811\\
116.01	0.00502669916236056\\
117.01	0.00502669189293583\\
118.01	0.00502668448401019\\
119.01	0.00502667693291004\\
120.01	0.00502666923691057\\
121.01	0.00502666139323414\\
122.01	0.00502665339904985\\
123.01	0.00502664525147199\\
124.01	0.00502663694755967\\
125.01	0.00502662848431528\\
126.01	0.00502661985868368\\
127.01	0.00502661106755034\\
128.01	0.00502660210774116\\
129.01	0.00502659297602076\\
130.01	0.00502658366909147\\
131.01	0.00502657418359178\\
132.01	0.00502656451609557\\
133.01	0.00502655466311061\\
134.01	0.00502654462107741\\
135.01	0.00502653438636722\\
136.01	0.00502652395528167\\
137.01	0.00502651332405101\\
138.01	0.00502650248883249\\
139.01	0.00502649144570913\\
140.01	0.00502648019068817\\
141.01	0.00502646871969995\\
142.01	0.00502645702859572\\
143.01	0.00502644511314656\\
144.01	0.00502643296904195\\
145.01	0.00502642059188779\\
146.01	0.00502640797720459\\
147.01	0.00502639512042618\\
148.01	0.00502638201689851\\
149.01	0.00502636866187672\\
150.01	0.00502635505052398\\
151.01	0.00502634117790974\\
152.01	0.00502632703900747\\
153.01	0.00502631262869365\\
154.01	0.00502629794174505\\
155.01	0.00502628297283685\\
156.01	0.00502626771654075\\
157.01	0.00502625216732305\\
158.01	0.00502623631954242\\
159.01	0.00502622016744737\\
160.01	0.00502620370517524\\
161.01	0.00502618692674866\\
162.01	0.00502616982607423\\
163.01	0.00502615239693952\\
164.01	0.005026134633011\\
165.01	0.00502611652783171\\
166.01	0.00502609807481878\\
167.01	0.00502607926726106\\
168.01	0.00502606009831574\\
169.01	0.00502604056100712\\
170.01	0.00502602064822275\\
171.01	0.00502600035271122\\
172.01	0.00502597966707928\\
173.01	0.00502595858378888\\
174.01	0.00502593709515498\\
175.01	0.0050259151933413\\
176.01	0.00502589287035859\\
177.01	0.00502587011806067\\
178.01	0.00502584692814166\\
179.01	0.00502582329213277\\
180.01	0.00502579920139933\\
181.01	0.00502577464713643\\
182.01	0.0050257496203664\\
183.01	0.0050257241119352\\
184.01	0.00502569811250916\\
185.01	0.00502567161257017\\
186.01	0.00502564460241322\\
187.01	0.00502561707214224\\
188.01	0.00502558901166555\\
189.01	0.00502556041069319\\
190.01	0.00502553125873202\\
191.01	0.00502550154508167\\
192.01	0.00502547125883069\\
193.01	0.00502544038885155\\
194.01	0.00502540892379753\\
195.01	0.00502537685209701\\
196.01	0.00502534416194964\\
197.01	0.00502531084132103\\
198.01	0.005025276877939\\
199.01	0.00502524225928753\\
200.01	0.00502520697260286\\
201.01	0.00502517100486792\\
202.01	0.00502513434280697\\
203.01	0.00502509697288092\\
204.01	0.00502505888128121\\
205.01	0.00502502005392489\\
206.01	0.00502498047644921\\
207.01	0.00502494013420463\\
208.01	0.00502489901225038\\
209.01	0.00502485709534807\\
210.01	0.00502481436795496\\
211.01	0.00502477081421839\\
212.01	0.00502472641796924\\
213.01	0.00502468116271534\\
214.01	0.00502463503163499\\
215.01	0.00502458800756995\\
216.01	0.00502454007301841\\
217.01	0.00502449121012867\\
218.01	0.00502444140069091\\
219.01	0.00502439062613034\\
220.01	0.0050243388674997\\
221.01	0.00502428610547143\\
222.01	0.00502423232032982\\
223.01	0.00502417749196284\\
224.01	0.00502412159985483\\
225.01	0.005024064623076\\
226.01	0.00502400654027658\\
227.01	0.00502394732967589\\
228.01	0.00502388696905482\\
229.01	0.0050238254357458\\
230.01	0.00502376270662396\\
231.01	0.00502369875809768\\
232.01	0.00502363356609869\\
233.01	0.00502356710607221\\
234.01	0.00502349935296707\\
235.01	0.00502343028122427\\
236.01	0.0050233598647678\\
237.01	0.00502328807699373\\
238.01	0.00502321489075774\\
239.01	0.00502314027836531\\
240.01	0.00502306421156023\\
241.01	0.00502298666151143\\
242.01	0.00502290759880255\\
243.01	0.00502282699341911\\
244.01	0.00502274481473582\\
245.01	0.00502266103150399\\
246.01	0.00502257561183809\\
247.01	0.005022488523203\\
248.01	0.00502239973240027\\
249.01	0.00502230920555365\\
250.01	0.00502221690809563\\
251.01	0.00502212280475224\\
252.01	0.00502202685952898\\
253.01	0.00502192903569476\\
254.01	0.0050218292957669\\
255.01	0.00502172760149603\\
256.01	0.00502162391384869\\
257.01	0.00502151819299161\\
258.01	0.0050214103982748\\
259.01	0.00502130048821416\\
260.01	0.00502118842047428\\
261.01	0.00502107415184989\\
262.01	0.00502095763824812\\
263.01	0.00502083883466974\\
264.01	0.00502071769518958\\
265.01	0.00502059417293812\\
266.01	0.00502046822008034\\
267.01	0.00502033978779554\\
268.01	0.00502020882625775\\
269.01	0.00502007528461342\\
270.01	0.00501993911095965\\
271.01	0.0050198002523227\\
272.01	0.00501965865463546\\
273.01	0.0050195142627137\\
274.01	0.00501936702023292\\
275.01	0.00501921686970466\\
276.01	0.00501906375245166\\
277.01	0.00501890760858225\\
278.01	0.00501874837696522\\
279.01	0.00501858599520404\\
280.01	0.00501842039960989\\
281.01	0.00501825152517415\\
282.01	0.00501807930554093\\
283.01	0.00501790367297823\\
284.01	0.00501772455834978\\
285.01	0.0050175418910846\\
286.01	0.00501735559914695\\
287.01	0.00501716560900614\\
288.01	0.00501697184560448\\
289.01	0.00501677423232486\\
290.01	0.0050165726909586\\
291.01	0.00501636714167152\\
292.01	0.00501615750297024\\
293.01	0.00501594369166666\\
294.01	0.00501572562284251\\
295.01	0.00501550320981382\\
296.01	0.00501527636409247\\
297.01	0.00501504499534952\\
298.01	0.00501480901137572\\
299.01	0.0050145683180425\\
300.01	0.00501432281926165\\
301.01	0.00501407241694434\\
302.01	0.00501381701095937\\
303.01	0.00501355649908949\\
304.01	0.00501329077698911\\
305.01	0.00501301973813854\\
306.01	0.00501274327379982\\
307.01	0.00501246127296886\\
308.01	0.00501217362232905\\
309.01	0.00501188020620328\\
310.01	0.00501158090650419\\
311.01	0.00501127560268378\\
312.01	0.00501096417168221\\
313.01	0.00501064648787578\\
314.01	0.00501032242302336\\
315.01	0.00500999184621224\\
316.01	0.00500965462380213\\
317.01	0.00500931061936889\\
318.01	0.00500895969364617\\
319.01	0.00500860170446727\\
320.01	0.00500823650670423\\
321.01	0.00500786395220709\\
322.01	0.00500748388974067\\
323.01	0.0050070961649207\\
324.01	0.00500670062014944\\
325.01	0.00500629709454859\\
326.01	0.00500588542389132\\
327.01	0.00500546544053379\\
328.01	0.00500503697334403\\
329.01	0.00500459984762995\\
330.01	0.0050041538850663\\
331.01	0.00500369890361948\\
332.01	0.00500323471747099\\
333.01	0.00500276113694053\\
334.01	0.0050022779684046\\
335.01	0.00500178501421676\\
336.01	0.00500128207262463\\
337.01	0.00500076893768534\\
338.01	0.00500024539917948\\
339.01	0.00499971124252294\\
340.01	0.00499916624867749\\
341.01	0.004998610194059\\
342.01	0.00499804285044488\\
343.01	0.00499746398487761\\
344.01	0.00499687335956869\\
345.01	0.00499627073179867\\
346.01	0.00499565585381676\\
347.01	0.00499502847273668\\
348.01	0.0049943883304321\\
349.01	0.00499373516342843\\
350.01	0.00499306870279381\\
351.01	0.00499238867402622\\
352.01	0.00499169479694037\\
353.01	0.00499098678554984\\
354.01	0.00499026434794896\\
355.01	0.00498952718619072\\
356.01	0.00498877499616313\\
357.01	0.00498800746746252\\
358.01	0.004987224283264\\
359.01	0.00498642512019002\\
360.01	0.00498560964817559\\
361.01	0.00498477753033095\\
362.01	0.004983928422801\\
363.01	0.00498306197462264\\
364.01	0.00498217782757774\\
365.01	0.00498127561604488\\
366.01	0.00498035496684642\\
367.01	0.00497941549909322\\
368.01	0.00497845682402568\\
369.01	0.00497747854485262\\
370.01	0.0049764802565847\\
371.01	0.00497546154586643\\
372.01	0.00497442199080368\\
373.01	0.00497336116078744\\
374.01	0.00497227861631527\\
375.01	0.00497117390880699\\
376.01	0.00497004658041888\\
377.01	0.00496889616385164\\
378.01	0.00496772218215657\\
379.01	0.00496652414853671\\
380.01	0.0049653015661442\\
381.01	0.00496405392787294\\
382.01	0.00496278071614793\\
383.01	0.00496148140270978\\
384.01	0.00496015544839467\\
385.01	0.0049588023029107\\
386.01	0.00495742140460907\\
387.01	0.00495601218025018\\
388.01	0.00495457404476648\\
389.01	0.00495310640101909\\
390.01	0.0049516086395499\\
391.01	0.00495008013832966\\
392.01	0.00494852026249938\\
393.01	0.00494692836410812\\
394.01	0.00494530378184428\\
395.01	0.00494364584076202\\
396.01	0.00494195385200213\\
397.01	0.00494022711250749\\
398.01	0.00493846490473222\\
399.01	0.00493666649634533\\
400.01	0.00493483113992809\\
401.01	0.00493295807266601\\
402.01	0.00493104651603314\\
403.01	0.00492909567547131\\
404.01	0.00492710474006213\\
405.01	0.00492507288219215\\
406.01	0.0049229992572118\\
407.01	0.00492088300308632\\
408.01	0.0049187232400408\\
409.01	0.00491651907019639\\
410.01	0.0049142695772005\\
411.01	0.00491197382584832\\
412.01	0.00490963086169665\\
413.01	0.00490723971067068\\
414.01	0.00490479937866128\\
415.01	0.00490230885111476\\
416.01	0.0048997670926143\\
417.01	0.00489717304645162\\
418.01	0.00489452563419164\\
419.01	0.00489182375522598\\
420.01	0.00488906628631888\\
421.01	0.00488625208114297\\
422.01	0.00488337996980528\\
423.01	0.00488044875836417\\
424.01	0.00487745722833499\\
425.01	0.00487440413618674\\
426.01	0.00487128821282793\\
427.01	0.00486810816308112\\
428.01	0.00486486266514787\\
429.01	0.00486155037006109\\
430.01	0.00485816990112722\\
431.01	0.00485471985335676\\
432.01	0.00485119879288194\\
433.01	0.00484760525636404\\
434.01	0.00484393775038724\\
435.01	0.00484019475084038\\
436.01	0.00483637470228587\\
437.01	0.00483247601731603\\
438.01	0.00482849707589545\\
439.01	0.00482443622469048\\
440.01	0.00482029177638453\\
441.01	0.00481606200897971\\
442.01	0.00481174516508364\\
443.01	0.00480733945118229\\
444.01	0.00480284303689729\\
445.01	0.00479825405422835\\
446.01	0.00479357059678031\\
447.01	0.00478879071897395\\
448.01	0.00478391243524159\\
449.01	0.00477893371920487\\
450.01	0.0047738525028374\\
451.01	0.00476866667560869\\
452.01	0.0047633740836119\\
453.01	0.00475797252867356\\
454.01	0.00475245976744409\\
455.01	0.00474683351047193\\
456.01	0.00474109142125671\\
457.01	0.00473523111528438\\
458.01	0.00472925015904235\\
459.01	0.00472314606901494\\
460.01	0.00471691631065769\\
461.01	0.00471055829735238\\
462.01	0.00470406938933889\\
463.01	0.00469744689262706\\
464.01	0.00469068805788581\\
465.01	0.00468379007931023\\
466.01	0.00467675009346508\\
467.01	0.00466956517810593\\
468.01	0.00466223235097598\\
469.01	0.00465474856857865\\
470.01	0.00464711072492618\\
471.01	0.00463931565026219\\
472.01	0.0046313601097596\\
473.01	0.00462324080219216\\
474.01	0.00461495435857925\\
475.01	0.00460649734080428\\
476.01	0.00459786624020519\\
477.01	0.00458905747613741\\
478.01	0.00458006739450801\\
479.01	0.0045708922662815\\
480.01	0.00456152828595533\\
481.01	0.00455197157000686\\
482.01	0.00454221815530865\\
483.01	0.00453226399751377\\
484.01	0.00452210496940937\\
485.01	0.00451173685923857\\
486.01	0.00450115536898976\\
487.01	0.00449035611265384\\
488.01	0.00447933461444723\\
489.01	0.00446808630700198\\
490.01	0.00445660652952146\\
491.01	0.00444489052590145\\
492.01	0.00443293344281625\\
493.01	0.00442073032776978\\
494.01	0.0044082761271104\\
495.01	0.00439556568401039\\
496.01	0.00438259373640875\\
497.01	0.00436935491491723\\
498.01	0.00435584374068995\\
499.01	0.00434205462325613\\
500.01	0.00432798185831573\\
501.01	0.00431361962549771\\
502.01	0.00429896198608176\\
503.01	0.0042840028806827\\
504.01	0.00426873612689873\\
505.01	0.00425315541692276\\
506.01	0.00423725431511873\\
507.01	0.0042210262555614\\
508.01	0.00420446453954272\\
509.01	0.00418756233304362\\
510.01	0.00417031266417407\\
511.01	0.0041527084205811\\
512.01	0.00413474234682745\\
513.01	0.00411640704174263\\
514.01	0.00409769495574692\\
515.01	0.00407859838815333\\
516.01	0.00405910948444813\\
517.01	0.00403922023355392\\
518.01	0.00401892246507926\\
519.01	0.00399820784655814\\
520.01	0.00397706788068469\\
521.01	0.00395549390254862\\
522.01	0.00393347707687628\\
523.01	0.00391100839528585\\
524.01	0.00388807867356233\\
525.01	0.00386467854896303\\
526.01	0.00384079847756167\\
527.01	0.00381642873164236\\
528.01	0.00379155939715641\\
529.01	0.00376618037125456\\
530.01	0.00374028135991029\\
531.01	0.00371385187565177\\
532.01	0.00368688123541993\\
533.01	0.00365935855857643\\
534.01	0.00363127276508319\\
535.01	0.00360261257388014\\
536.01	0.00357336650149188\\
537.01	0.00354352286089486\\
538.01	0.00351306976068174\\
539.01	0.00348199510456496\\
540.01	0.00345028659126204\\
541.01	0.00341793171481548\\
542.01	0.00338491776540139\\
543.01	0.0033512318306892\\
544.01	0.00331686079782177\\
545.01	0.00328179135609021\\
546.01	0.00324601000039032\\
547.01	0.00320950303555117\\
548.01	0.00317225658164214\\
549.01	0.00313425658037082\\
550.01	0.00309548880269848\\
551.01	0.0030559388578128\\
552.01	0.0030155922036114\\
553.01	0.00297443415886602\\
554.01	0.0029324499172527\\
555.01	0.00288962456345301\\
556.01	0.00284594309155133\\
557.01	0.00280139042597268\\
558.01	0.00275595144523016\\
559.01	0.00270961100877377\\
560.01	0.0026623539872585\\
561.01	0.00261416529657479\\
562.01	0.0025650299360136\\
563.01	0.00251493303096204\\
564.01	0.00246385988055685\\
565.01	0.00241179601074453\\
566.01	0.00235872723322304\\
567.01	0.0023046397107587\\
568.01	0.00224952002938472\\
569.01	0.0021933552779938\\
570.01	0.00213613313583009\\
571.01	0.00207784196836236\\
572.01	0.00201847093197674\\
573.01	0.00195801008785242\\
574.01	0.00189645052527396\\
575.01	0.00183378449447307\\
576.01	0.00177000554886907\\
577.01	0.00170510869627301\\
578.01	0.00163909055821012\\
579.01	0.00157194953597514\\
580.01	0.00150368598132186\\
581.01	0.00143430236876668\\
582.01	0.00136380346528834\\
583.01	0.00129219649167196\\
584.01	0.00121949126778043\\
585.01	0.00114570033153021\\
586.01	0.00107083901816671\\
587.01	0.000994925482401572\\
588.01	0.000917980640875388\\
589.01	0.000840028005979048\\
590.01	0.000761093373966642\\
591.01	0.000681204320110056\\
592.01	0.000600389440856771\\
593.01	0.000518677266913736\\
594.01	0.000436094751083286\\
595.01	0.000352665209519579\\
596.01	0.000268405563604855\\
597.01	0.000183322690309106\\
598.01	9.83859705868326e-05\\
599.01	3.18230442563731e-05\\
599.02	3.12770957175551e-05\\
599.03	3.07343620567849e-05\\
599.04	3.0194875217571e-05\\
599.05	2.96586674565693e-05\\
599.06	2.91257713466771e-05\\
599.07	2.85962197801391e-05\\
599.08	2.80700459716933e-05\\
599.09	2.75472834617447e-05\\
599.1	2.70279661195773e-05\\
599.11	2.65121281465865e-05\\
599.12	2.59998040795448e-05\\
599.13	2.54910287939124e-05\\
599.14	2.49858375071712e-05\\
599.15	2.44842657821827e-05\\
599.16	2.39863495305956e-05\\
599.17	2.34921250162837e-05\\
599.18	2.30016288588035e-05\\
599.19	2.2514898036903e-05\\
599.2	2.20319698920508e-05\\
599.21	2.1552882132023e-05\\
599.22	2.10776728344908e-05\\
599.23	2.06063804506721e-05\\
599.24	2.01390438090109e-05\\
599.25	1.96757021188876e-05\\
599.26	1.9216394974363e-05\\
599.27	1.87611623579855e-05\\
599.28	1.83100446445959e-05\\
599.29	1.78630826051952e-05\\
599.3	1.74203174108517e-05\\
599.31	1.69817906366335e-05\\
599.32	1.65475442655914e-05\\
599.33	1.61176206927693e-05\\
599.34	1.56920627292622e-05\\
599.35	1.52709136063186e-05\\
599.36	1.48542169794725e-05\\
599.37	1.4442016932719e-05\\
599.38	1.40343579827368e-05\\
599.39	1.36312859119591e-05\\
599.4	1.3232849938186e-05\\
599.41	1.28390997671604e-05\\
599.42	1.24500855973528e-05\\
599.43	1.20658581248094e-05\\
599.44	1.16864685480531e-05\\
599.45	1.13119685730082e-05\\
599.46	1.09424104179929e-05\\
599.47	1.05778468187639e-05\\
599.48	1.02183310335888e-05\\
599.49	9.86391684839293e-06\\
599.5	9.51465858194112e-06\\
599.51	9.17061109107116e-06\\
599.52	8.83182977599525e-06\\
599.53	8.4983705856221e-06\\
599.54	8.1702900229675e-06\\
599.55	7.84764515058753e-06\\
599.56	7.53049359608453e-06\\
599.57	7.21889355765649e-06\\
599.58	6.91290380970891e-06\\
599.59	6.61258370851688e-06\\
599.6	6.31799319793756e-06\\
599.61	6.02919281519031e-06\\
599.62	5.74624369668701e-06\\
599.63	5.46920758391981e-06\\
599.64	5.19814682940593e-06\\
599.65	4.93312440269685e-06\\
599.66	4.67420389643411e-06\\
599.67	4.42144953247646e-06\\
599.68	4.17492616808582e-06\\
599.69	3.9346993021671e-06\\
599.7	3.70083508157044e-06\\
599.71	3.47340030746116e-06\\
599.72	3.25246244175723e-06\\
599.73	3.03808961360161e-06\\
599.74	2.83035062593855e-06\\
599.75	2.62931496212288e-06\\
599.76	2.43505279260738e-06\\
599.77	2.24763498169432e-06\\
599.78	2.06713309435641e-06\\
599.79	1.89361940311147e-06\\
599.8	1.72716689497872e-06\\
599.81	1.56784927850956e-06\\
599.82	1.41574099085488e-06\\
599.83	1.27091720493813e-06\\
599.84	1.13345383668736e-06\\
599.85	1.00342755231589e-06\\
599.86	8.8091577571392e-07\\
599.87	7.65996695880136e-07\\
599.88	6.58749274441706e-07\\
599.89	5.59253253243699e-07\\
599.9	4.67589162013102e-07\\
599.91	3.83838326099145e-07\\
599.92	3.08082874293444e-07\\
599.93	2.40405746710845e-07\\
599.94	1.80890702777825e-07\\
599.95	1.2962232925906e-07\\
599.96	8.66860484019516e-08\\
599.97	5.21681261331924e-08\\
599.98	2.61556803542173e-08\\
599.99	8.73668930083393e-09\\
600	0\\
};
\end{axis}
\end{tikzpicture}% 
  \caption{Continuous Time w/ nFPC}
\end{subfigure}%
\hfill%
\begin{subfigure}{.45\linewidth}
  \centering
  \setlength\figureheight{\linewidth} 
  \setlength\figurewidth{\linewidth}
  \tikzsetnextfilename{dm_dscr_nFPC_z8}
  % This file was created by matlab2tikz.
%
%The latest updates can be retrieved from
%  http://www.mathworks.com/matlabcentral/fileexchange/22022-matlab2tikz-matlab2tikz
%where you can also make suggestions and rate matlab2tikz.
%
\definecolor{mycolor1}{rgb}{0.00000,1.00000,0.14286}%
\definecolor{mycolor2}{rgb}{0.00000,1.00000,0.28571}%
\definecolor{mycolor3}{rgb}{0.00000,1.00000,0.42857}%
\definecolor{mycolor4}{rgb}{0.00000,1.00000,0.57143}%
\definecolor{mycolor5}{rgb}{0.00000,1.00000,0.71429}%
\definecolor{mycolor6}{rgb}{0.00000,1.00000,0.85714}%
\definecolor{mycolor7}{rgb}{0.00000,1.00000,1.00000}%
\definecolor{mycolor8}{rgb}{0.00000,0.87500,1.00000}%
\definecolor{mycolor9}{rgb}{0.00000,0.62500,1.00000}%
\definecolor{mycolor10}{rgb}{0.12500,0.00000,1.00000}%
\definecolor{mycolor11}{rgb}{0.25000,0.00000,1.00000}%
\definecolor{mycolor12}{rgb}{0.37500,0.00000,1.00000}%
\definecolor{mycolor13}{rgb}{0.50000,0.00000,1.00000}%
\definecolor{mycolor14}{rgb}{0.62500,0.00000,1.00000}%
\definecolor{mycolor15}{rgb}{0.75000,0.00000,1.00000}%
\definecolor{mycolor16}{rgb}{0.87500,0.00000,1.00000}%
\definecolor{mycolor17}{rgb}{1.00000,0.00000,1.00000}%
\definecolor{mycolor18}{rgb}{1.00000,0.00000,0.87500}%
\definecolor{mycolor19}{rgb}{1.00000,0.00000,0.62500}%
\definecolor{mycolor20}{rgb}{0.85714,0.00000,0.00000}%
\definecolor{mycolor21}{rgb}{0.71429,0.00000,0.00000}%
%
\begin{tikzpicture}

\begin{axis}[%
width=4.1in,
height=3.803in,
at={(0.809in,0.513in)},
scale only axis,
point meta min=0,
point meta max=1,
every outer x axis line/.append style={black},
every x tick label/.append style={font=\color{black}},
xmin=0,
xmax=600,
every outer y axis line/.append style={black},
every y tick label/.append style={font=\color{black}},
ymin=0,
ymax=0.012,
axis background/.style={fill=white},
axis x line*=bottom,
axis y line*=left,
colormap={mymap}{[1pt] rgb(0pt)=(0,1,0); rgb(7pt)=(0,1,1); rgb(15pt)=(0,0,1); rgb(23pt)=(1,0,1); rgb(31pt)=(1,0,0); rgb(38pt)=(0,0,0)},
colorbar,
colorbar style={separate axis lines,every outer x axis line/.append style={black},every x tick label/.append style={font=\color{black}},every outer y axis line/.append style={black},every y tick label/.append style={font=\color{black}},yticklabels={{-19},{-17},{-15},{-13},{-11},{-9},{-7},{-5},{-3},{-1},{1},{3},{5},{7},{9},{11},{13},{15},{17},{19}}}
]
\addplot [color=green,solid,forget plot]
  table[row sep=crcr]{%
1	0.00409784475759984\\
2	0.00409792030352774\\
3	0.00409799759515541\\
4	0.00409807667318205\\
5	0.00409815757926939\\
6	0.00409824035606549\\
7	0.00409832504722819\\
8	0.00409841169745005\\
9	0.00409850035248332\\
10	0.00409859105916587\\
11	0.00409868386544764\\
12	0.00409877882041777\\
13	0.00409887597433279\\
14	0.0040989753786449\\
15	0.0040990770860314\\
16	0.00409918115042499\\
17	0.00409928762704437\\
18	0.00409939657242646\\
19	0.00409950804445848\\
20	0.00409962210241159\\
21	0.00409973880697533\\
22	0.00409985822029268\\
23	0.00409998040599627\\
24	0.00410010542924576\\
25	0.00410023335676568\\
26	0.00410036425688493\\
27	0.00410049819957701\\
28	0.00410063525650136\\
29	0.00410077550104588\\
30	0.00410091900837079\\
31	0.00410106585545345\\
32	0.00410121612113476\\
33	0.00410136988616654\\
34	0.0041015272332605\\
35	0.00410168824713853\\
36	0.00410185301458445\\
37	0.00410202162449739\\
38	0.00410219416794652\\
39	0.00410237073822733\\
40	0.00410255143092025\\
41	0.00410273634395007\\
42	0.00410292557764804\\
43	0.00410311923481519\\
44	0.00410331742078817\\
45	0.00410352024350664\\
46	0.00410372781358327\\
47	0.00410394024437562\\
48	0.0041041576520604\\
49	0.00410438015571046\\
50	0.00410460787737368\\
51	0.00410484094215535\\
52	0.00410507947830251\\
53	0.00410532361729158\\
54	0.00410557349391918\\
55	0.00410582924639567\\
56	0.00410609101644245\\
57	0.00410635894939252\\
58	0.00410663319429459\\
59	0.00410691390402163\\
60	0.00410720123538273\\
61	0.00410749534923982\\
62	0.0041077964106286\\
63	0.00410810458888434\\
64	0.0041084200577726\\
65	0.00410874299562529\\
66	0.00410907358548255\\
67	0.00410941201524031\\
68	0.00410975847780406\\
69	0.0041101131712496\\
70	0.00411047629899072\\
71	0.00411084806995413\\
72	0.00411122869876268\\
73	0.00411161840592728\\
74	0.00411201741804687\\
75	0.00411242596801909\\
76	0.00411284429526073\\
77	0.00411327264593887\\
78	0.00411371127321411\\
79	0.00411416043749602\\
80	0.00411462040671191\\
81	0.00411509145658996\\
82	0.00411557387095742\\
83	0.00411606794205526\\
84	0.00411657397087042\\
85	0.00411709226748659\\
86	0.00411762315145538\\
87	0.00411816695218918\\
88	0.00411872400937669\\
89	0.00411929467342452\\
90	0.00411987930592472\\
91	0.00412047828015157\\
92	0.00412109198158957\\
93	0.00412172080849527\\
94	0.00412236517249433\\
95	0.0041230254992186\\
96	0.00412370222898441\\
97	0.00412439581751649\\
98	0.0041251067367204\\
99	0.00412583547550783\\
100	0.00412658254067848\\
101	0.00412734845786299\\
102	0.00412813377253208\\
103	0.00412893905107647\\
104	0.00412976488196353\\
105	0.00413061187697616\\
106	0.00413148067254064\\
107	0.00413237193114969\\
108	0.00413328634288785\\
109	0.00413422462706693\\
110	0.00413518753397913\\
111	0.00413617584677592\\
112	0.00413719038348127\\
113	0.00413823199914742\\
114	0.00413930158816231\\
115	0.00414040008671661\\
116	0.0041415284754392\\
117	0.00414268778220749\\
118	0.00414387908514009\\
119	0.00414510351577602\\
120	0.004146362262443\\
121	0.00414765657381456\\
122	0.00414898776265122\\
123	0.00415035720971535\\
124	0.00415176636784119\\
125	0.00415321676613401\\
126	0.00415471001425678\\
127	0.00415624780674915\\
128	0.00415783192730104\\
129	0.00415946425287718\\
130	0.00416114675755497\\
131	0.00416288151589485\\
132	0.00416467070560643\\
133	0.00416651660920656\\
134	0.0041684216142744\\
135	0.00417038821180032\\
136	0.00417241899198365\\
137	0.00417451663665705\\
138	0.00417668390729333\\
139	0.00417892362726813\\
140	0.00418123865669847\\
141	0.00418363185772974\\
142	0.00418610604758351\\
143	0.00418866393596887\\
144	0.00419130804257029\\
145	0.00419404058920214\\
146	0.00419686335980469\\
147	0.00419977751967251\\
148	0.00420278338304558\\
149	0.00420588011530566\\
150	0.0042090653522319\\
151	0.00421233471331403\\
152	0.00421568117537134\\
153	0.00421906581522593\\
154	0.00422248597450611\\
155	0.00422594201091482\\
156	0.00422943428503874\\
157	0.00423296316033096\\
158	0.00423652900308989\\
159	0.00424013218243538\\
160	0.00424377307028067\\
161	0.00424745204130091\\
162	0.00425116947289712\\
163	0.00425492574515596\\
164	0.00425872124080423\\
165	0.00426255634515837\\
166	0.0042664314460684\\
167	0.00427034693385525\\
168	0.00427430320124195\\
169	0.00427830064327726\\
170	0.00428233965725188\\
171	0.00428642064260612\\
172	0.00429054400082835\\
173	0.00429471013534397\\
174	0.00429891945139335\\
175	0.00430317235589869\\
176	0.00430746925731783\\
177	0.00431181056548482\\
178	0.00431619669143569\\
179	0.00432062804721811\\
180	0.00432510504568366\\
181	0.00432962810026109\\
182	0.00433419762470916\\
183	0.00433881403284646\\
184	0.00434347773825736\\
185	0.00434818915397096\\
186	0.00435294869211118\\
187	0.00435775676351291\\
188	0.00436261377730807\\
189	0.00436752014046877\\
190	0.00437247625731131\\
191	0.00437748252895453\\
192	0.00438253935272789\\
193	0.00438764712152498\\
194	0.00439280622309771\\
195	0.00439801703928436\\
196	0.00440327994516592\\
197	0.00440859530814329\\
198	0.00441396348692798\\
199	0.00441938483043717\\
200	0.00442485967658419\\
201	0.00443038835095385\\
202	0.00443597116535115\\
203	0.00444160841621043\\
204	0.00444730038285186\\
205	0.00445304732556862\\
206	0.00445884948352894\\
207	0.00446470707247404\\
208	0.00447062028219071\\
209	0.00447658927373774\\
210	0.0044826141763994\\
211	0.00448869508434091\\
212	0.00449483205293488\\
213	0.0045010250947277\\
214	0.00450727417500996\\
215	0.00451357920695347\\
216	0.00451994004627456\\
217	0.00452635648537871\\
218	0.00453282824694102\\
219	0.00453935497687283\\
220	0.00454593623662294\\
221	0.00455257149476086\\
222	0.00455926011778804\\
223	0.00456600136012463\\
224	0.00457279435322196\\
225	0.00457963809375516\\
226	0.00458653143086177\\
227	0.00459347305240365\\
228	0.00460046147025117\\
229	0.00460749500461732\\
230	0.00461457176750347\\
231	0.00462168964535071\\
232	0.00462884628093494\\
233	0.0046360390539445\\
234	0.00464326505527298\\
235	0.0046505210069576\\
236	0.00465780349876262\\
237	0.00466510876667863\\
238	0.00467243266248855\\
239	0.00467977062581579\\
240	0.00468711765648356\\
241	0.00469446828812083\\
242	0.00470181656430447\\
243	0.00470915601898387\\
244	0.00471647966353329\\
245	0.00472377998355318\\
246	0.00473104894955389\\
247	0.00473827804697045\\
248	0.0047454583326852\\
249	0.00475258052754034\\
250	0.00475963515751682\\
251	0.00476661276115319\\
252	0.00477350419045295\\
253	0.00478030099699438\\
254	0.00478699576098433\\
255	0.00479358310053279\\
256	0.00480006068343935\\
257	0.00480643056821392\\
258	0.00481270097541078\\
259	0.00481890168130836\\
260	0.00482513036488285\\
261	0.00483138572166826\\
262	0.00483766632757198\\
263	0.00484397062201996\\
264	0.00485029688657508\\
265	0.00485664321544166\\
266	0.00486300746246877\\
267	0.00486938705377548\\
268	0.0048757800140581\\
269	0.00488218440010652\\
270	0.00488859816855402\\
271	0.00489501917381586\\
272	0.00490144516664088\\
273	0.00490787379348034\\
274	0.00491430259692962\\
275	0.00492072901756244\\
276	0.00492715039755593\\
277	0.00493356398660151\\
278	0.00493996695070905\\
279	0.00494635638464548\\
280	0.00495272932888653\\
281	0.00495908279218016\\
282	0.00496541378042456\\
283	0.00497171933271478\\
284	0.00497799656666794\\
285	0.00498424273248062\\
286	0.00499045527173479\\
287	0.00499663188759957\\
288	0.00500276885455785\\
289	0.00500886179727862\\
290	0.00501490622294663\\
291	0.0050208975384141\\
292	0.00502683107286921\\
293	0.00503270210805545\\
294	0.00503850591519979\\
295	0.00504423779490658\\
296	0.00504989312489541\\
297	0.00505546741658601\\
298	0.00506095638156396\\
299	0.00506635600894974\\
300	0.00507166265461885\\
301	0.00507687314304434\\
302	0.00508198488219067\\
303	0.00508699599120746\\
304	0.00509190543908644\\
305	0.00509671318596088\\
306	0.00510142031598759\\
307	0.00510602917595886\\
308	0.00511054350191797\\
309	0.00511496851579151\\
310	0.00511931096968429\\
311	0.00512357910642704\\
312	0.00512778249307162\\
313	0.00513193166910957\\
314	0.00513603753568379\\
315	0.00514011038915024\\
316	0.00514415339170176\\
317	0.00514816558872805\\
318	0.00515214555600522\\
319	0.00515609192551052\\
320	0.00516000339726006\\
321	0.0051638787522915\\
322	0.00516771686684419\\
323	0.00517151672770696\\
324	0.00517527744867228\\
325	0.00517899828803098\\
326	0.00518267866697795\\
327	0.00518631818872863\\
328	0.00518991665801778\\
329	0.00519347410061639\\
330	0.00519699078257065\\
331	0.00520046722872537\\
332	0.00520390423989118\\
333	0.00520730290787344\\
334	0.00521066462749302\\
335	0.0052139911047977\\
336	0.00521728436050494\\
337	0.00522054672698069\\
338	0.00522378083724281\\
339	0.00522698960436026\\
340	0.00523017618951947\\
341	0.00523334395679822\\
342	0.00523649641238523\\
343	0.00523963712698979\\
344	0.00524276964205906\\
345	0.00524589736136628\\
346	0.00524902343209221\\
347	0.00525215062342403\\
348	0.00525528121650131\\
349	0.0052584169280298\\
350	0.00526155892451648\\
351	0.00526470821934065\\
352	0.0052678659099201\\
353	0.00527103317704476\\
354	0.0052742112832648\\
355	0.00527740157022393\\
356	0.00528060545483306\\
357	0.00528382442418496\\
358	0.00528706002911852\\
359	0.0052903138763541\\
360	0.00529358761915325\\
361	0.00529688294650302\\
362	0.0053002015708826\\
363	0.00530354521475302\\
364	0.00530691559602828\\
365	0.00531031441291814\\
366	0.00531374332865523\\
367	0.00531720395678034\\
368	0.00532069784783065\\
369	0.00532422647844399\\
370	0.00532779124403392\\
371	0.00533139345626187\\
372	0.00533503434648635\\
373	0.00533871507609726\\
374	0.00534243675401099\\
375	0.00534620046040236\\
376	0.00535000727230451\\
377	0.00535385827275344\\
378	0.0053577545480272\\
379	0.00536169718486027\\
380	0.00536568726769117\\
381	0.00536972587601087\\
382	0.00537381408188833\\
383	0.00537795294775574\\
384	0.00538214352454116\\
385	0.00538638685023603\\
386	0.00539068394898272\\
387	0.00539503583075542\\
388	0.00539944349168897\\
389	0.00540390791508142\\
390	0.00540843007305335\\
391	0.0054130109287946\\
392	0.00541765143926053\\
393	0.00542235255810581\\
394	0.00542711523856707\\
395	0.00543194043594388\\
396	0.00543682910931025\\
397	0.00544178222215611\\
398	0.00544680074245792\\
399	0.00545188564276676\\
400	0.00545703790040008\\
401	0.00546225849773876\\
402	0.00546754842262698\\
403	0.00547290866886857\\
404	0.00547834023680947\\
405	0.00548384413399068\\
406	0.00548942137585202\\
407	0.00549507298646269\\
408	0.00550079999925233\\
409	0.00550660345771517\\
410	0.00551248441606048\\
411	0.00551844393979057\\
412	0.00552448310619344\\
413	0.00553060300475403\\
414	0.00553680473750273\\
415	0.00554308941933771\\
416	0.00554945817835165\\
417	0.00555591215616768\\
418	0.00556245250828261\\
419	0.00556908040441308\\
420	0.00557579702884114\\
421	0.0055826035807562\\
422	0.00558950127459017\\
423	0.00559649134034323\\
424	0.00560357502389917\\
425	0.00561075358732925\\
426	0.00561802830918566\\
427	0.00562540048478577\\
428	0.00563287142648992\\
429	0.00564044246397542\\
430	0.00564811494450953\\
431	0.00565589023322228\\
432	0.00566376971337998\\
433	0.00567175478665855\\
434	0.00567984687341632\\
435	0.00568804741296627\\
436	0.00569635786384754\\
437	0.00570477970409607\\
438	0.00571331443151478\\
439	0.00572196356394295\\
440	0.00573072863952553\\
441	0.00573961121698208\\
442	0.00574861287587567\\
443	0.00575773521688192\\
444	0.00576697986205737\\
445	0.00577634845510774\\
446	0.00578584266165521\\
447	0.00579546416950444\\
448	0.00580521468890684\\
449	0.00581509595282272\\
450	0.00582510971718042\\
451	0.00583525776113227\\
452	0.00584554188730608\\
453	0.0058559639220517\\
454	0.00586652571568156\\
455	0.00587722914270388\\
456	0.00588807610204783\\
457	0.00589906851727852\\
458	0.00591020833680067\\
459	0.00592149753404935\\
460	0.00593293810766529\\
461	0.00594453208165292\\
462	0.00595628150551875\\
463	0.00596818845438693\\
464	0.00598025502908955\\
465	0.00599248335622785\\
466	0.00600487558820092\\
467	0.00601743390319778\\
468	0.00603016050514805\\
469	0.00604305762362689\\
470	0.00605612751370817\\
471	0.00606937245576052\\
472	0.00608279475517958\\
473	0.00609639674204979\\
474	0.00611018077072839\\
475	0.00612414921934346\\
476	0.00613830448919851\\
477	0.00615264900407409\\
478	0.00616718520941869\\
479	0.00618191557141915\\
480	0.00619684257594279\\
481	0.00621196872734266\\
482	0.00622729654711955\\
483	0.00624282857243494\\
484	0.00625856735447267\\
485	0.00627451545665015\\
486	0.00629067545268511\\
487	0.00630704992453122\\
488	0.00632364146020511\\
489	0.00634045265154001\\
490	0.00635748609191801\\
491	0.00637474437405498\\
492	0.00639223008794037\\
493	0.00640994581907178\\
494	0.0064278941471719\\
495	0.00644607764563728\\
496	0.00646449888204806\\
497	0.00648316042016958\\
498	0.00650206482400709\\
499	0.00652121466464175\\
500	0.00654061253078801\\
501	0.00656026104428372\\
502	0.00658016288206822\\
503	0.00660032080664221\\
504	0.00662073770756038\\
505	0.00664141665721423\\
506	0.00666236098506164\\
507	0.00668357437559948\\
508	0.00670506099682256\\
509	0.00672682566774802\\
510	0.00674886776502838\\
511	0.00677119585749369\\
512	0.00679382297813684\\
513	0.00681676657528694\\
514	0.00684004385018383\\
515	0.00686366344368807\\
516	0.00688763434733743\\
517	0.00691196385407253\\
518	0.00693665747087436\\
519	0.00696171964759408\\
520	0.00698715326698022\\
521	0.00701295919276243\\
522	0.00703913570363854\\
523	0.00706567763267601\\
524	0.00709257527891002\\
525	0.00711981302177701\\
526	0.0071473288879066\\
527	0.00717508630917037\\
528	0.00720315062282531\\
529	0.00723170825117882\\
530	0.00726080771265837\\
531	0.00729050452473134\\
532	0.00732083845906349\\
533	0.00735185058042227\\
534	0.00738358424139605\\
535	0.00741608727970591\\
536	0.0074494127711794\\
537	0.00748362000688747\\
538	0.0075187754227737\\
539	0.00755495371725533\\
540	0.00759223914403396\\
541	0.00763072064444217\\
542	0.00767050041992187\\
543	0.00771053336520066\\
544	0.00774814489342967\\
545	0.00778501922575503\\
546	0.00782234883417097\\
547	0.00786028395490776\\
548	0.00789883697694226\\
549	0.00793799584592659\\
550	0.00797774149567381\\
551	0.00801804850376936\\
552	0.00805888399757698\\
553	0.00810020597452262\\
554	0.00814196096787807\\
555	0.00818408040193254\\
556	0.00822647299553078\\
557	0.00826814713066404\\
558	0.00830933123458027\\
559	0.00835081763203065\\
560	0.00839260575580856\\
561	0.00843466894641207\\
562	0.00847698026063976\\
563	0.00851951301986419\\
564	0.00856218669075257\\
565	0.00860460466750246\\
566	0.00864738981450226\\
567	0.00869053926768215\\
568	0.00873403631435603\\
569	0.00877786229262354\\
570	0.00882199645485595\\
571	0.0088664158231443\\
572	0.00891109504152541\\
573	0.00895600622686392\\
574	0.0090011188209025\\
575	0.00904639944683076\\
576	0.00909181177477994\\
577	0.00913731640198181\\
578	0.00918287075500257\\
579	0.00922842902355833\\
580	0.0092739421380386\\
581	0.00931935780612553\\
582	0.00936462062793334\\
583	0.00940967231402828\\
584	0.00945445203657523\\
585	0.0094988969504773\\
586	0.00954294292772307\\
587	0.00958652555102227\\
588	0.00962958140299686\\
589	0.00967204963943729\\
590	0.0097138736812812\\
591	0.0097550024176802\\
592	0.00979538909575469\\
593	0.00983498278816331\\
594	0.00987369853931868\\
595	0.00991118387968948\\
596	0.00994658651044256\\
597	0.00997788999445116\\
598	0.010000292044645\\
599	0\\
600	0\\
};
\addplot [color=mycolor1,solid,forget plot]
  table[row sep=crcr]{%
1	0.00409772721480342\\
2	0.00409779894368203\\
3	0.00409787229004294\\
4	0.00409794729009925\\
5	0.00409802398086176\\
6	0.00409810240015601\\
7	0.00409818258663963\\
8	0.00409826457981992\\
9	0.00409834842007161\\
10	0.00409843414865529\\
11	0.00409852180773607\\
12	0.00409861144040237\\
13	0.00409870309068543\\
14	0.00409879680357872\\
15	0.00409889262505827\\
16	0.00409899060210289\\
17	0.00409909078271514\\
18	0.00409919321594221\\
19	0.00409929795189766\\
20	0.00409940504178319\\
21	0.00409951453791119\\
22	0.00409962649372711\\
23	0.00409974096383279\\
24	0.00409985800400979\\
25	0.00409997767124344\\
26	0.00410010002374688\\
27	0.00410022512098599\\
28	0.00410035302370438\\
29	0.00410048379394882\\
30	0.00410061749509548\\
31	0.0041007541918758\\
32	0.00410089395040352\\
33	0.00410103683820191\\
34	0.00410118292423112\\
35	0.0041013322789164\\
36	0.00410148497417633\\
37	0.00410164108345169\\
38	0.00410180068173495\\
39	0.00410196384559975\\
40	0.00410213065323089\\
41	0.00410230118445509\\
42	0.00410247552077149\\
43	0.00410265374538353\\
44	0.00410283594323015\\
45	0.00410302220101832\\
46	0.0041032126072553\\
47	0.00410340725228164\\
48	0.00410360622830462\\
49	0.00410380962943171\\
50	0.00410401755170488\\
51	0.00410423009313479\\
52	0.00410444735373589\\
53	0.00410466943556137\\
54	0.00410489644273873\\
55	0.00410512848150574\\
56	0.00410536566024641\\
57	0.00410560808952759\\
58	0.00410585588213612\\
59	0.00410610915311544\\
60	0.00410636801980324\\
61	0.00410663260186913\\
62	0.00410690302135278\\
63	0.00410717940270171\\
64	0.00410746187281014\\
65	0.00410775056105755\\
66	0.00410804559934759\\
67	0.00410834712214699\\
68	0.0041086552665253\\
69	0.00410897017219387\\
70	0.0041092919815457\\
71	0.00410962083969529\\
72	0.00410995689451837\\
73	0.00411030029669195\\
74	0.00411065119973452\\
75	0.00411100976004612\\
76	0.00411137613694869\\
77	0.00411175049272651\\
78	0.00411213299266626\\
79	0.00411252380509742\\
80	0.00411292310143299\\
81	0.00411333105620942\\
82	0.00411374784712704\\
83	0.00411417365509031\\
84	0.00411460866424798\\
85	0.00411505306203289\\
86	0.00411550703920199\\
87	0.00411597078987582\\
88	0.00411644451157802\\
89	0.00411692840527417\\
90	0.00411742267541044\\
91	0.00411792752995182\\
92	0.00411844318041973\\
93	0.00411896984192869\\
94	0.00411950773322249\\
95	0.0041200570767091\\
96	0.00412061809849419\\
97	0.00412119102841346\\
98	0.00412177610006287\\
99	0.00412237355082669\\
100	0.00412298362190276\\
101	0.0041236065583249\\
102	0.00412424260898072\\
103	0.00412489202662549\\
104	0.00412555506788985\\
105	0.00412623199328115\\
106	0.0041269230671765\\
107	0.00412762855780644\\
108	0.00412834873722706\\
109	0.00412908388127877\\
110	0.0041298342695286\\
111	0.00413060018519369\\
112	0.00413138191504216\\
113	0.00413217974926719\\
114	0.00413299398132975\\
115	0.00413382490776419\\
116	0.00413467282794\\
117	0.004135538043773\\
118	0.00413642085937625\\
119	0.00413732158064065\\
120	0.00413824051473449\\
121	0.00413917796950705\\
122	0.00414013425278189\\
123	0.00414110967152146\\
124	0.00414210453084355\\
125	0.00414311913286581\\
126	0.00414415377535373\\
127	0.00414520875014204\\
128	0.00414628434129883\\
129	0.0041473808229959\\
130	0.00414849845704804\\
131	0.00414963749007916\\
132	0.00415079815027464\\
133	0.00415198064367354\\
134	0.00415318514996139\\
135	0.00415441181772467\\
136	0.00415566075913799\\
137	0.00415693204407067\\
138	0.00415822569362178\\
139	0.00415954167312933\\
140	0.00416087988475102\\
141	0.00416224015978999\\
142	0.0041636222510447\\
143	0.00416502582561039\\
144	0.00416645045876467\\
145	0.00416789562984994\\
146	0.00416936072144634\\
147	0.00417084502364335\\
148	0.00417234774590553\\
149	0.00417386803993707\\
150	0.00417540503813451\\
151	0.0041769579138338\\
152	0.00417852597327299\\
153	0.00418010928885501\\
154	0.00418170799276501\\
155	0.00418332221827263\\
156	0.00418495209976837\\
157	0.0041865977728027\\
158	0.00418825937412943\\
159	0.00418993704175247\\
160	0.00419163091497726\\
161	0.00419334113446629\\
162	0.00419506784230025\\
163	0.00419681118204391\\
164	0.00419857129881815\\
165	0.00420034833937799\\
166	0.00420214245219747\\
167	0.00420395378756188\\
168	0.00420578249766757\\
169	0.00420762873673033\\
170	0.00420949266110283\\
171	0.00421137442940206\\
172	0.00421327420264693\\
173	0.00421519214440761\\
174	0.00421712842096694\\
175	0.00421908320149493\\
176	0.00422105665823793\\
177	0.00422304896672268\\
178	0.00422506030597716\\
179	0.00422709085876897\\
180	0.00422914081186331\\
181	0.00423121035630107\\
182	0.00423329968769973\\
183	0.00423540900657803\\
184	0.00423753851870656\\
185	0.00423968843548626\\
186	0.00424185897435704\\
187	0.00424405035923917\\
188	0.00424626282100948\\
189	0.00424849659801558\\
190	0.00425075193663102\\
191	0.00425302909185485\\
192	0.00425532832795867\\
193	0.00425764991918607\\
194	0.00425999415050744\\
195	0.00426236131843595\\
196	0.00426475173190895\\
197	0.00426716571324103\\
198	0.00426960359915399\\
199	0.00427206574189107\\
200	0.00427455251042226\\
201	0.00427706429174896\\
202	0.00427960149231631\\
203	0.00428216453954314\\
204	0.00428475388347959\\
205	0.00428736999860489\\
206	0.00429001338577685\\
207	0.00429268457434783\\
208	0.0042953841244628\\
209	0.00429811262955571\\
210	0.00430087071906375\\
211	0.00430365906137983\\
212	0.00430647836706625\\
213	0.0043093293923543\\
214	0.00431221294295856\\
215	0.0043151298782353\\
216	0.00431808111571905\\
217	0.00432106763607472\\
218	0.00432409048850475\\
219	0.00432715079665611\\
220	0.00433024976507524\\
221	0.0043333886862628\\
222	0.00433656894838491\\
223	0.00433979204370076\\
224	0.00434305957777054\\
225	0.00434637327951035\\
226	0.00434973501216295\\
227	0.00435314678525194\\
228	0.00435661076758406\\
229	0.00436012930135168\\
230	0.00436370491735915\\
231	0.00436734035132117\\
232	0.00437103856100468\\
233	0.00437480274356612\\
234	0.00437863635279915\\
235	0.00438254314485878\\
236	0.00438652718250342\\
237	0.00439059285877664\\
238	0.00439474492172882\\
239	0.00439898849948371\\
240	0.00440332912502909\\
241	0.00440777275985302\\
242	0.0044123258151979\\
243	0.00441699516925015\\
244	0.00442178817798137\\
245	0.00442671267656985\\
246	0.00443177696730284\\
247	0.00443698978851966\\
248	0.0044423602574157\\
249	0.00444789777725692\\
250	0.00445361189653253\\
251	0.00445951210304723\\
252	0.00446560752534391\\
253	0.00447190650305558\\
254	0.00447841604892367\\
255	0.00448514108665174\\
256	0.00449208342506626\\
257	0.00449924038604222\\
258	0.00450660297522872\\
259	0.00451414025698812\\
260	0.0045217525252719\\
261	0.00452944028791647\\
262	0.00453720402905694\\
263	0.00454504420549145\\
264	0.00455296124256425\\
265	0.00456095552955954\\
266	0.00456902741892628\\
267	0.00457717730623837\\
268	0.00458540556282764\\
269	0.00459371251732887\\
270	0.00460209845009204\\
271	0.00461056358702463\\
272	0.00461910809282056\\
273	0.00462773206353394\\
274	0.00463643551845733\\
275	0.00464521839126953\\
276	0.00465408052042462\\
277	0.00466302163876009\\
278	0.00467204136227265\\
279	0.00468113917773462\\
280	0.0046903144267936\\
281	0.00469956626631459\\
282	0.00470889371913389\\
283	0.00471829568279564\\
284	0.00472777082691935\\
285	0.00473731755503774\\
286	0.00474693394218696\\
287	0.00475661756059487\\
288	0.00476636552940042\\
289	0.00477617451557519\\
290	0.00478604067846271\\
291	0.0047959596220509\\
292	0.0048059263359506\\
293	0.00481593480188697\\
294	0.00482597820079109\\
295	0.0048360488286401\\
296	0.00484613800659609\\
297	0.00485623598621501\\
298	0.00486633185105257\\
299	0.00487641341673388\\
300	0.00488646713248571\\
301	0.00489647798800039\\
302	0.00490642942862883\\
303	0.00491630326797561\\
304	0.00492607946572709\\
305	0.00493573738047742\\
306	0.00494525536311096\\
307	0.00495461088938635\\
308	0.00496378103654142\\
309	0.00497274319842527\\
310	0.00498147612422024\\
311	0.0049899613955635\\
312	0.00499818550441449\\
313	0.00500614280671684\\
314	0.0050138383289735\\
315	0.00502129149402076\\
316	0.00502870260229165\\
317	0.00503610288391933\\
318	0.00504348774286446\\
319	0.00505085233759865\\
320	0.0050581915765222\\
321	0.00506550011410386\\
322	0.00507277234802151\\
323	0.00508000242071015\\
324	0.00508718422583651\\
325	0.00509431141980478\\
326	0.00510137743966654\\
327	0.00510837552893968\\
328	0.00511529877420358\\
329	0.00512214015415874\\
330	0.00512889259727722\\
331	0.00513554905037802\\
332	0.00514210256268315\\
333	0.00514854638708865\\
334	0.00515487409757838\\
335	0.00516107971488429\\
336	0.00516715784227585\\
337	0.00517310383639541\\
338	0.00517891399802226\\
339	0.00518458578150427\\
340	0.0051901180217842\\
341	0.00519551118469681\\
342	0.00520076765133354\\
343	0.00520589199788326\\
344	0.0052108912081264\\
345	0.00521577481157448\\
346	0.0052205548933601\\
347	0.00522524589882257\\
348	0.00522986412393199\\
349	0.0052344267388042\\
350	0.00523894945645597\\
351	0.00524343605940803\\
352	0.00524788560074577\\
353	0.00525229734303664\\
354	0.00525667078600199\\
355	0.00526100569488241\\
356	0.0052653021289482\\
357	0.00526956046970549\\
358	0.00527378144828387\\
359	0.00527796617157201\\
360	0.00528211614594839\\
361	0.00528623329710285\\
362	0.00529031998433898\\
363	0.00529437900711757\\
364	0.00529841360059086\\
365	0.00530242741703889\\
366	0.00530642449123779\\
367	0.00531040918682558\\
368	0.00531438612091678\\
369	0.00531836006481072\\
370	0.00532233581986472\\
371	0.00532631806978906\\
372	0.00533031121401844\\
373	0.00533431919234488\\
374	0.00533834531948073\\
375	0.00534239216076209\\
376	0.0053464615391883\\
377	0.00535055505978985\\
378	0.0053546744226983\\
379	0.00535882141702043\\
380	0.00536299791282706\\
381	0.00536720585112944\\
382	0.005371447231776\\
383	0.00537572409929912\\
384	0.00538003852682219\\
385	0.00538439259824991\\
386	0.00538878838911245\\
387	0.00539322794662035\\
388	0.0053977132697017\\
389	0.00540224629003136\\
390	0.00540682885531752\\
391	0.00541146271635344\\
392	0.00541614951952548\\
393	0.00542089080652173\\
394	0.00542568802278514\\
395	0.00543054253561985\\
396	0.00543545566152567\\
397	0.00544042869990902\\
398	0.00544546294960872\\
399	0.00545055970798066\\
400	0.00545572026769364\\
401	0.00546094591375229\\
402	0.00546623792085765\\
403	0.00547159755122089\\
404	0.00547702605294676\\
405	0.00548252465909634\\
406	0.00548809458752136\\
407	0.00549373704153303\\
408	0.00549945321142147\\
409	0.00550524427678036\\
410	0.00551111140950813\\
411	0.00551705577726307\\
412	0.00552307854704392\\
413	0.0055291808884754\\
414	0.00553536397632439\\
415	0.00554162899181428\\
416	0.00554797712303811\\
417	0.00555440956521041\\
418	0.00556092752104443\\
419	0.00556753220125155\\
420	0.00557422482515346\\
421	0.00558100662139226\\
422	0.00558787882871715\\
423	0.00559484269682172\\
424	0.00560189948720002\\
425	0.0056090504739877\\
426	0.00561629694475481\\
427	0.00562364020122091\\
428	0.00563108155987295\\
429	0.0056386223524815\\
430	0.00564626392653087\\
431	0.00565400764560148\\
432	0.00566185488974582\\
433	0.00566980705587261\\
434	0.00567786555813514\\
435	0.00568603182831905\\
436	0.00569430731622558\\
437	0.00570269349004649\\
438	0.00571119183672758\\
439	0.00571980386231872\\
440	0.00572853109230956\\
441	0.00573737507195137\\
442	0.00574633736656675\\
443	0.00575541956184987\\
444	0.00576462326416059\\
445	0.00577395010081486\\
446	0.00578340172037297\\
447	0.00579297979292529\\
448	0.00580268601037476\\
449	0.00581252208671591\\
450	0.00582248975830964\\
451	0.00583259078415376\\
452	0.00584282694614889\\
453	0.00585320004935955\\
454	0.00586371192227051\\
455	0.00587436441703796\\
456	0.00588515940973543\\
457	0.00589609880059389\\
458	0.0059071845142356\\
459	0.0059184184999007\\
460	0.00592980273166581\\
461	0.00594133920865384\\
462	0.00595302995523355\\
463	0.00596487702120814\\
464	0.00597688248199155\\
465	0.00598904843877122\\
466	0.00600137701865596\\
467	0.00601387037480744\\
468	0.00602653068655392\\
469	0.00603936015948441\\
470	0.00605236102552181\\
471	0.00606553554297312\\
472	0.00607888599655534\\
473	0.00609241469739513\\
474	0.00610612398300072\\
475	0.00612001621720504\\
476	0.00613409379007817\\
477	0.00614835911780911\\
478	0.00616281464255603\\
479	0.00617746283226529\\
480	0.00619230618046065\\
481	0.00620734720600465\\
482	0.00622258845283574\\
483	0.00623803248968672\\
484	0.00625368190979207\\
485	0.00626953933059434\\
486	0.00628560739346397\\
487	0.00630188876345045\\
488	0.00631838612908839\\
489	0.00633510220228862\\
490	0.00635203971835239\\
491	0.00636920143615627\\
492	0.00638659013856709\\
493	0.00640420863316077\\
494	0.00642205975333513\\
495	0.00644014635992797\\
496	0.00645847134347519\\
497	0.006477037627273\\
498	0.00649584817144181\\
499	0.0065149059782292\\
500	0.00653421409883596\\
501	0.00655377564210219\\
502	0.00657359378545241\\
503	0.00659367178856756\\
504	0.00661401301032988\\
505	0.00663462092967143\\
506	0.00665549917104909\\
507	0.00667665153536294\\
508	0.0066980820372294\\
509	0.00671979494960482\\
510	0.00674179493719375\\
511	0.0067640868956916\\
512	0.0067866759769133\\
513	0.00680956400369565\\
514	0.00683275491124096\\
515	0.00685626196973593\\
516	0.0068800986408156\\
517	0.00690428490325568\\
518	0.00692883527291861\\
519	0.00695375870382165\\
520	0.00697906425632247\\
521	0.00700475949871382\\
522	0.00703085045342747\\
523	0.00705734188569942\\
524	0.00708423681480593\\
525	0.00711153593740922\\
526	0.00713923746350955\\
527	0.00716733497321819\\
528	0.00719581077674005\\
529	0.00722455855703865\\
530	0.00725363372862162\\
531	0.00728313100693132\\
532	0.00731320423426021\\
533	0.00734390257698169\\
534	0.00737527918989368\\
535	0.00740737901823451\\
536	0.00744024906925833\\
537	0.00747393941753061\\
538	0.00750850490350621\\
539	0.00754400577180493\\
540	0.00758050847644705\\
541	0.00761808214679199\\
542	0.00765680105430943\\
543	0.00769676637028498\\
544	0.00773813039112718\\
545	0.00777868539256093\\
546	0.0078168880070114\\
547	0.00785492834535674\\
548	0.00789347294193416\\
549	0.00793261873474102\\
550	0.0079723685954058\\
551	0.00801270533222008\\
552	0.00805360340878692\\
553	0.00809502949182715\\
554	0.00813694106531913\\
555	0.00817928405928756\\
556	0.0082219892856467\\
557	0.00826498033629833\\
558	0.00830749515365131\\
559	0.00834919568350826\\
560	0.00839115862383602\\
561	0.00843340529491727\\
562	0.00847591105247397\\
563	0.00851864731722591\\
564	0.00856158658383183\\
565	0.0086045920915919\\
566	0.00864738978534756\\
567	0.00869053926739589\\
568	0.00873403631432268\\
569	0.00877786229261397\\
570	0.00882199645485205\\
571	0.00886641582314234\\
572	0.00891109504152458\\
573	0.00895600622686346\\
574	0.00900111882090225\\
575	0.00904639944683062\\
576	0.00909181177477987\\
577	0.00913731640198179\\
578	0.00918287075500256\\
579	0.00922842902355833\\
580	0.00927394213803861\\
581	0.00931935780612554\\
582	0.00936462062793335\\
583	0.00940967231402828\\
584	0.00945445203657523\\
585	0.0094988969504773\\
586	0.00954294292772307\\
587	0.00958652555102228\\
588	0.00962958140299686\\
589	0.0096720496394373\\
590	0.0097138736812812\\
591	0.0097550024176802\\
592	0.00979538909575469\\
593	0.00983498278816331\\
594	0.00987369853931868\\
595	0.00991118387968948\\
596	0.00994658651044256\\
597	0.00997788999445116\\
598	0.010000292044645\\
599	0\\
600	0\\
};
\addplot [color=mycolor2,solid,forget plot]
  table[row sep=crcr]{%
1	0.00409747161189642\\
2	0.00409753539813578\\
3	0.00409760055013061\\
4	0.00409766709594947\\
5	0.00409773506418996\\
6	0.00409780448398646\\
7	0.00409787538501809\\
8	0.00409794779751668\\
9	0.00409802175227469\\
10	0.00409809728065326\\
11	0.00409817441458998\\
12	0.00409825318660705\\
13	0.00409833362981889\\
14	0.00409841577794048\\
15	0.00409849966529485\\
16	0.00409858532682111\\
17	0.00409867279808217\\
18	0.00409876211527228\\
19	0.00409885331522501\\
20	0.00409894643542064\\
21	0.00409904151399345\\
22	0.0040991385897394\\
23	0.00409923770212313\\
24	0.00409933889128525\\
25	0.00409944219804905\\
26	0.00409954766392779\\
27	0.00409965533113078\\
28	0.00409976524257016\\
29	0.00409987744186714\\
30	0.00409999197335779\\
31	0.00410010888209922\\
32	0.00410022821387476\\
33	0.0041003500151994\\
34	0.00410047433332463\\
35	0.00410060121624315\\
36	0.00410073071269329\\
37	0.00410086287216279\\
38	0.00410099774489229\\
39	0.00410113538187862\\
40	0.00410127583487759\\
41	0.00410141915640617\\
42	0.00410156539974421\\
43	0.00410171461893575\\
44	0.00410186686878994\\
45	0.00410202220488092\\
46	0.00410218068354748\\
47	0.0041023423618923\\
48	0.00410250729777974\\
49	0.00410267554983383\\
50	0.00410284717743505\\
51	0.0041030222407164\\
52	0.00410320080055873\\
53	0.00410338291858535\\
54	0.00410356865715563\\
55	0.00410375807935773\\
56	0.00410395124900057\\
57	0.00410414823060455\\
58	0.00410434908939137\\
59	0.00410455389127309\\
60	0.00410476270283974\\
61	0.00410497559134594\\
62	0.00410519262469638\\
63	0.0041054138714302\\
64	0.00410563940070404\\
65	0.00410586928227383\\
66	0.00410610358647515\\
67	0.00410634238420284\\
68	0.00410658574688835\\
69	0.00410683374647658\\
70	0.0041070864554006\\
71	0.00410734394655529\\
72	0.00410760629326943\\
73	0.00410787356927631\\
74	0.00410814584868262\\
75	0.00410842320593601\\
76	0.00410870571579089\\
77	0.00410899345327286\\
78	0.00410928649364127\\
79	0.00410958491235057\\
80	0.00410988878500935\\
81	0.0041101981873385\\
82	0.00411051319512743\\
83	0.00411083388418873\\
84	0.00411116033031089\\
85	0.00411149260921013\\
86	0.00411183079648004\\
87	0.00411217496753997\\
88	0.00411252519758179\\
89	0.00411288156151534\\
90	0.00411324413391216\\
91	0.00411361298894819\\
92	0.00411398820034483\\
93	0.0041143698413086\\
94	0.00411475798447002\\
95	0.00411515270182075\\
96	0.00411555406465005\\
97	0.00411596214347976\\
98	0.00411637700799863\\
99	0.00411679872699537\\
100	0.00411722736829093\\
101	0.00411766299867002\\
102	0.0041181056838117\\
103	0.00411855548821906\\
104	0.00411901247514837\\
105	0.00411947670653741\\
106	0.0041199482429331\\
107	0.00412042714341815\\
108	0.0041209134655373\\
109	0.0041214072652224\\
110	0.00412190859671682\\
111	0.00412241751249883\\
112	0.00412293406320339\\
113	0.00412345829754346\\
114	0.00412399026222934\\
115	0.0041245300018866\\
116	0.00412507755897211\\
117	0.00412563297368823\\
118	0.00412619628389468\\
119	0.00412676752501859\\
120	0.00412734672996138\\
121	0.00412793392900395\\
122	0.00412852914970932\\
123	0.00412913241682325\\
124	0.00412974375217359\\
125	0.00413036317456882\\
126	0.00413099069969742\\
127	0.00413162634002976\\
128	0.00413227010472435\\
129	0.00413292199954316\\
130	0.00413358202677864\\
131	0.00413425018519948\\
132	0.00413492647002104\\
133	0.00413561087291054\\
134	0.00413630338203718\\
135	0.00413700398218096\\
136	0.00413771265491638\\
137	0.00413842937889044\\
138	0.00413915413021722\\
139	0.00413988688301409\\
140	0.00414062761010833\\
141	0.00414137628394292\\
142	0.00414213287771258\\
143	0.00414289736675549\\
144	0.00414366973022058\\
145	0.00414444995301399\\
146	0.0041452380280014\\
147	0.00414603395840073\\
148	0.00414683776023207\\
149	0.00414764946459766\\
150	0.00414846911946697\\
151	0.00414929679070874\\
152	0.00415013256280046\\
153	0.00415097652319167\\
154	0.00415182876115608\\
155	0.00415268936787204\\
156	0.00415355843650752\\
157	0.00415443606230929\\
158	0.00415532234269612\\
159	0.00415621737735695\\
160	0.00415712126835361\\
161	0.00415803412022869\\
162	0.00415895604011845\\
163	0.00415988713787103\\
164	0.00416082752617065\\
165	0.00416177732066735\\
166	0.00416273664011272\\
167	0.00416370560650229\\
168	0.00416468434522427\\
169	0.00416567298521502\\
170	0.00416667165912175\\
171	0.00416768050347225\\
172	0.00416869965885266\\
173	0.0041697292700925\\
174	0.00417076948645847\\
175	0.00417182046185595\\
176	0.00417288235503981\\
177	0.00417395532983361\\
178	0.00417503955535845\\
179	0.00417613520627111\\
180	0.00417724246301193\\
181	0.0041783615120632\\
182	0.00417949254621711\\
183	0.0041806357648553\\
184	0.00418179137423853\\
185	0.00418295958780796\\
186	0.00418414062649785\\
187	0.00418533471905979\\
188	0.00418654210239906\\
189	0.00418776302192296\\
190	0.00418899773190199\\
191	0.00419024649584326\\
192	0.00419150958687672\\
193	0.00419278728815461\\
194	0.00419407989326373\\
195	0.00419538770665124\\
196	0.00419671104406349\\
197	0.00419805023299822\\
198	0.00419940561317019\\
199	0.00420077753698996\\
200	0.00420216637005581\\
201	0.00420357249165847\\
202	0.00420499629529858\\
203	0.00420643818921634\\
204	0.0042078985969331\\
205	0.00420937795780344\\
206	0.00421087672757834\\
207	0.00421239537897701\\
208	0.00421393440226728\\
209	0.00421549430585238\\
210	0.0042170756168629\\
211	0.00421867888175132\\
212	0.00422030466688705\\
213	0.00422195355914864\\
214	0.0042236261665097\\
215	0.00422532311861412\\
216	0.00422704506733573\\
217	0.00422879268731582\\
218	0.00423056667647189\\
219	0.00423236775646867\\
220	0.00423419667314141\\
221	0.004236054196859\\
222	0.00423794112281342\\
223	0.0042398582712181\\
224	0.0042418064873953\\
225	0.00424378664172982\\
226	0.0042457996294605\\
227	0.00424784637027782\\
228	0.00424992780768809\\
229	0.00425204490809677\\
230	0.00425419865955228\\
231	0.00425639007007598\\
232	0.0042586201655153\\
233	0.00426088998707641\\
234	0.00426320058987421\\
235	0.00426555303748082\\
236	0.00426794839720525\\
237	0.00427038773444267\\
238	0.00427287210592968\\
239	0.0042754025517495\\
240	0.00427798008592344\\
241	0.00428060568541552\\
242	0.00428328027738017\\
243	0.00428600472449318\\
244	0.00428877980823606\\
245	0.0042916062100587\\
246	0.0042944844904334\\
247	0.00429741506595932\\
248	0.00430039818488352\\
249	0.00430343390169871\\
250	0.0043065220518256\\
251	0.00430966222770246\\
252	0.00431285375894432\\
253	0.00431609570450518\\
254	0.00431938685736227\\
255	0.00432272577187187\\
256	0.00432611082522832\\
257	0.0043295403286064\\
258	0.00433301270856113\\
259	0.0043365270143948\\
260	0.0043400842428449\\
261	0.00434368544098068\\
262	0.00434733170972222\\
263	0.00435102420769006\\
264	0.00435476415560266\\
265	0.00435855284221954\\
266	0.00436239163542775\\
267	0.00436628198165968\\
268	0.00437022541053829\\
269	0.00437422354125197\\
270	0.00437827808949937\\
271	0.0043823908750584\\
272	0.00438656383003684\\
273	0.00439079900786693\\
274	0.00439509859310615\\
275	0.00439946491211647\\
276	0.00440390044467257\\
277	0.00440840783653867\\
278	0.00441298991295949\\
279	0.00441764969279\\
280	0.00442239040296108\\
281	0.00442721550334059\\
282	0.00443212870843422\\
283	0.0044371340008761\\
284	0.00444223565292052\\
285	0.00444743824742085\\
286	0.00445274669560388\\
287	0.00445816627621191\\
288	0.00446370267254473\\
289	0.00446936200821571\\
290	0.00447515088662875\\
291	0.00448107642829197\\
292	0.00448714628446144\\
293	0.00449336870794162\\
294	0.00449975259858886\\
295	0.00450630754965302\\
296	0.00451304389387529\\
297	0.00451997274778971\\
298	0.004527106052044\\
299	0.00453445660470557\\
300	0.00454203808334817\\
301	0.00454986505006974\\
302	0.00455795293147298\\
303	0.00456631796775024\\
304	0.00457497720938437\\
305	0.00458394828581726\\
306	0.00459324913323066\\
307	0.00460289761933221\\
308	0.00461291100682783\\
309	0.00462330520184831\\
310	0.00463409371679639\\
311	0.00464528625435607\\
312	0.00465688678276178\\
313	0.00466889085781066\\
314	0.00468128210320214\\
315	0.00469402777432611\\
316	0.00470691739611307\\
317	0.00471991242935172\\
318	0.00473301002317151\\
319	0.00474620688294381\\
320	0.00475949917767092\\
321	0.00477288264150297\\
322	0.0047863524755296\\
323	0.00479990320100424\\
324	0.00481352857623578\\
325	0.00482722150450891\\
326	0.00484097393982608\\
327	0.0048547768328975\\
328	0.00486861970165124\\
329	0.00488249054713525\\
330	0.00489637589406776\\
331	0.00491026063622077\\
332	0.00492412785784585\\
333	0.00493795855675552\\
334	0.00495173168165891\\
335	0.00496542528834626\\
336	0.00497901542468126\\
337	0.00499247611671266\\
338	0.00500577942968237\\
339	0.00501889565646387\\
340	0.00503179379203509\\
341	0.00504444043675032\\
342	0.005056798614081\\
343	0.00506883063675311\\
344	0.00508049923158478\\
345	0.0050917691756241\\
346	0.00510260962124097\\
347	0.00511299734576308\\
348	0.00512292124026069\\
349	0.00513238847118869\\
350	0.00514145356279036\\
351	0.00515044292812221\\
352	0.00515934757340538\\
353	0.0051681582470874\\
354	0.00517686548853945\\
355	0.00518545969417062\\
356	0.00519393120286888\\
357	0.00520227039350545\\
358	0.00521046779163548\\
359	0.00521851417751173\\
360	0.00522640072029858\\
361	0.00523411914484358\\
362	0.00524166192587091\\
363	0.00524902252478635\\
364	0.00525619569649764\\
365	0.00526317784285559\\
366	0.00526996735421263\\
367	0.00527656497134352\\
368	0.00528297415389167\\
369	0.00528920143293488\\
370	0.0052952567130429\\
371	0.00530115347201255\\
372	0.00530690878823694\\
373	0.00531254308939831\\
374	0.00531807947203472\\
375	0.00532354238354427\\
376	0.00532895418314075\\
377	0.00533432142320478\\
378	0.0053396445335948\\
379	0.00534492439119021\\
380	0.00535016235681699\\
381	0.00535536030967686\\
382	0.0053605206769865\\
383	0.0053656464553484\\
384	0.00537074122179324\\
385	0.00537580913203551\\
386	0.00538085490254376\\
387	0.00538588377271432\\
388	0.00539090144328955\\
389	0.00539591398731883\\
390	0.00540092773045593\\
391	0.00540594909860276\\
392	0.00541098443331722\\
393	0.00541603977941393\\
394	0.00542112065567026\\
395	0.00542623182967747\\
396	0.00543137713273228\\
397	0.00543655937249408\\
398	0.00544178091217529\\
399	0.0054470441437416\\
400	0.00545235153771615\\
401	0.0054577056261782\\
402	0.00546310898322712\\
403	0.00546856420313171\\
404	0.0054740738765625\\
405	0.00547964056552351\\
406	0.0054852667778617\\
407	0.0054909549425352\\
408	0.00549670738714891\\
409	0.00550252631958955\\
410	0.00550841381585341\\
411	0.00551437181627652\\
412	0.00552040213222794\\
413	0.00552650646469684\\
414	0.00553268643478951\\
415	0.00553894362349785\\
416	0.00554527959897722\\
417	0.00555169592110127\\
418	0.00555819413806102\\
419	0.00556477578347433\\
420	0.00557144237415245\\
421	0.00557819540866512\\
422	0.00558503636682932\\
423	0.00559196671021332\\
424	0.00559898788369828\\
425	0.00560610131806688\\
426	0.00561330843349454\\
427	0.00562061064370411\\
428	0.0056280093604178\\
429	0.00563550599761747\\
430	0.00564310197503964\\
431	0.00565079872034391\\
432	0.00565859766997983\\
433	0.00566650026965315\\
434	0.00567450797493938\\
435	0.00568262225203333\\
436	0.00569084457861544\\
437	0.00569917644480915\\
438	0.00570761935419601\\
439	0.00571617482485011\\
440	0.00572484439035047\\
441	0.00573362960073054\\
442	0.0057425320233305\\
443	0.00575155324352999\\
444	0.00576069486535867\\
445	0.00576995851200706\\
446	0.00577934582628691\\
447	0.00578885847108561\\
448	0.00579849812982527\\
449	0.00580826650692118\\
450	0.00581816532823506\\
451	0.00582819634151819\\
452	0.00583836131684128\\
453	0.00584866204700868\\
454	0.00585910034795612\\
455	0.00586967805913298\\
456	0.00588039704387136\\
457	0.00589125918974586\\
458	0.0059022664089277\\
459	0.00591342063853709\\
460	0.00592472384099479\\
461	0.00593617800437307\\
462	0.00594778514274589\\
463	0.00595954729653819\\
464	0.00597146653287447\\
465	0.00598354494592692\\
466	0.0059957846572637\\
467	0.00600818781619825\\
468	0.00602075660013988\\
469	0.00603349321494683\\
470	0.00604639989528218\\
471	0.00605947890497346\\
472	0.00607273253737647\\
473	0.00608616311574416\\
474	0.00609977299360119\\
475	0.00611356455512523\\
476	0.00612754021553649\\
477	0.00614170242149662\\
478	0.00615605365151884\\
479	0.00617059641639156\\
480	0.00618533325961788\\
481	0.00620026675787385\\
482	0.00621539952148958\\
483	0.00623073419495676\\
484	0.00624627345746848\\
485	0.00626202002349724\\
486	0.00627797664341869\\
487	0.0062941461041901\\
488	0.0063105312300946\\
489	0.00632713488356401\\
490	0.00634395996609632\\
491	0.00636100941928663\\
492	0.00637828622599498\\
493	0.00639579341167863\\
494	0.00641353404592295\\
495	0.00643151124421204\\
496	0.00644972816998979\\
497	0.00646818803707225\\
498	0.00648689411248712\\
499	0.0065058497198329\\
500	0.00652505824327141\\
501	0.006544523132295\\
502	0.006564247907443\\
503	0.00658423616718477\\
504	0.00660449159624057\\
505	0.0066250179756797\\
506	0.0066458191952222\\
507	0.00666689926828206\\
508	0.00668826235043057\\
509	0.00670991276214244\\
510	0.00673185501589693\\
511	0.00675409385627773\\
512	0.00677663430915819\\
513	0.00679948178453945\\
514	0.00682264204388124\\
515	0.00684612107772275\\
516	0.00686992525165649\\
517	0.00689405601162315\\
518	0.00691852182093339\\
519	0.00694333742202316\\
520	0.00696851780686047\\
521	0.006994085086136\\
522	0.00702005457319023\\
523	0.00704643675525956\\
524	0.00707324204297694\\
525	0.0071004793565474\\
526	0.00712815559940787\\
527	0.00715627592819553\\
528	0.00718484326852532\\
529	0.00721385844412268\\
530	0.0072433155463926\\
531	0.00727317878533223\\
532	0.00730338078918359\\
533	0.0073339813864525\\
534	0.00736509082672553\\
535	0.00739685283269898\\
536	0.00742932163048537\\
537	0.00746255622633911\\
538	0.00749660705933109\\
539	0.00753152688765596\\
540	0.00756737104112477\\
541	0.00760419757173407\\
542	0.00764206275967574\\
543	0.0076810318607664\\
544	0.00772118083695201\\
545	0.00776262635070966\\
546	0.00780552562472109\\
547	0.00784719944470293\\
548	0.00788651753070341\\
549	0.00792583956011556\\
550	0.00796564674856095\\
551	0.00800603755448718\\
552	0.00804701590222305\\
553	0.00808856038361576\\
554	0.00813063840062174\\
555	0.00817320762573787\\
556	0.00821621465504425\\
557	0.00825959134358754\\
558	0.00830325988408064\\
559	0.00834680859180681\\
560	0.00838920854817912\\
561	0.00843169352148518\\
562	0.00847443197159942\\
563	0.00851741118575558\\
564	0.00856060116250899\\
565	0.0086039734318009\\
566	0.00864737435478978\\
567	0.00869053904167139\\
568	0.00873403631210821\\
569	0.00877786229236826\\
570	0.00882199645478541\\
571	0.00886641582311596\\
572	0.00891109504151141\\
573	0.00895600622685637\\
574	0.00900111882089835\\
575	0.00904639944682846\\
576	0.0090918117747785\\
577	0.00913731640198132\\
578	0.00918287075500235\\
579	0.00922842902355825\\
580	0.00927394213803858\\
581	0.00931935780612553\\
582	0.00936462062793333\\
583	0.00940967231402827\\
584	0.00945445203657523\\
585	0.00949889695047729\\
586	0.00954294292772306\\
587	0.00958652555102227\\
588	0.00962958140299686\\
589	0.00967204963943729\\
590	0.0097138736812812\\
591	0.00975500241768019\\
592	0.00979538909575469\\
593	0.00983498278816331\\
594	0.00987369853931868\\
595	0.00991118387968948\\
596	0.00994658651044256\\
597	0.00997788999445116\\
598	0.010000292044645\\
599	0\\
600	0\\
};
\addplot [color=mycolor3,solid,forget plot]
  table[row sep=crcr]{%
1	0.00409709505721719\\
2	0.00409714790196123\\
3	0.00409720179506604\\
4	0.00409725675550612\\
5	0.00409731280253287\\
6	0.00409736995567597\\
7	0.00409742823474466\\
8	0.00409748765982912\\
9	0.0040975482513015\\
10	0.0040976100298169\\
11	0.00409767301631428\\
12	0.00409773723201712\\
13	0.00409780269843404\\
14	0.00409786943735913\\
15	0.00409793747087224\\
16	0.00409800682133909\\
17	0.00409807751141115\\
18	0.0040981495640253\\
19	0.00409822300240346\\
20	0.00409829785005182\\
21	0.00409837413075997\\
22	0.00409845186859983\\
23	0.00409853108792437\\
24	0.0040986118133659\\
25	0.00409869406983447\\
26	0.00409877788251555\\
27	0.00409886327686802\\
28	0.00409895027862135\\
29	0.00409903891377286\\
30	0.0040991292085846\\
31	0.00409922118957973\\
32	0.00409931488353903\\
33	0.00409941031749665\\
34	0.00409950751873585\\
35	0.00409960651478426\\
36	0.00409970733340878\\
37	0.00409981000261029\\
38	0.00409991455061785\\
39	0.00410002100588264\\
40	0.00410012939707149\\
41	0.00410023975305994\\
42	0.00410035210292515\\
43	0.00410046647593821\\
44	0.00410058290155609\\
45	0.00410070140941337\\
46	0.00410082202931329\\
47	0.00410094479121861\\
48	0.00410106972524196\\
49	0.00410119686163573\\
50	0.00410132623078173\\
51	0.00410145786318031\\
52	0.00410159178943899\\
53	0.00410172804026081\\
54	0.00410186664643225\\
55	0.00410200763881073\\
56	0.00410215104831164\\
57	0.00410229690589511\\
58	0.00410244524255241\\
59	0.00410259608929171\\
60	0.0041027494771238\\
61	0.00410290543704742\\
62	0.00410306400003411\\
63	0.00410322519701274\\
64	0.0041033890588541\\
65	0.00410355561635478\\
66	0.00410372490022111\\
67	0.00410389694105278\\
68	0.00410407176932631\\
69	0.0041042494153785\\
70	0.00410442990938947\\
71	0.00410461328136599\\
72	0.00410479956112467\\
73	0.00410498877827502\\
74	0.00410518096220291\\
75	0.00410537614205391\\
76	0.00410557434671693\\
77	0.00410577560480814\\
78	0.00410597994465524\\
79	0.00410618739428196\\
80	0.00410639798139328\\
81	0.00410661173336094\\
82	0.00410682867720988\\
83	0.00410704883960502\\
84	0.00410727224683913\\
85	0.00410749892482147\\
86	0.00410772889906749\\
87	0.00410796219468941\\
88	0.00410819883638821\\
89	0.00410843884844677\\
90	0.0041086822547244\\
91	0.00410892907865275\\
92	0.00410917934323363\\
93	0.00410943307103807\\
94	0.00410969028420758\\
95	0.00410995100445693\\
96	0.00411021525307929\\
97	0.00411048305095331\\
98	0.00411075441855213\\
99	0.00411102937595528\\
100	0.00411130794286244\\
101	0.00411159013860977\\
102	0.00411187598218897\\
103	0.00411216549226877\\
104	0.00411245868721916\\
105	0.00411275558513829\\
106	0.00411305620388219\\
107	0.00411336056109718\\
108	0.00411366867425543\\
109	0.00411398056069289\\
110	0.00411429623765032\\
111	0.00411461572231712\\
112	0.00411493903187795\\
113	0.00411526618356198\\
114	0.00411559719469509\\
115	0.00411593208275438\\
116	0.00411627086542568\\
117	0.00411661356066322\\
118	0.00411696018675189\\
119	0.00411731076237185\\
120	0.00411766530666538\\
121	0.00411802383930604\\
122	0.00411838638056982\\
123	0.00411875295140849\\
124	0.00411912357352497\\
125	0.00411949826945069\\
126	0.00411987706262516\\
127	0.00412025997747728\\
128	0.00412064703950929\\
129	0.00412103827538233\\
130	0.0041214337130048\\
131	0.00412183338162291\\
132	0.0041222373119138\\
133	0.0041226455360812\\
134	0.00412305808795372\\
135	0.00412347500308553\\
136	0.00412389631885935\\
137	0.00412432207459104\\
138	0.00412475231163499\\
139	0.004125187073489\\
140	0.00412562640589676\\
141	0.00412607035694534\\
142	0.00412651897715428\\
143	0.00412697231955158\\
144	0.00412743043973162\\
145	0.00412789339588805\\
146	0.00412836124881516\\
147	0.0041288340618709\\
148	0.00412931190089873\\
149	0.00412979483411735\\
150	0.00413028293201533\\
151	0.00413077626729512\\
152	0.0041312749144199\\
153	0.0041317789496272\\
154	0.00413228845098237\\
155	0.00413280349843347\\
156	0.00413332417386732\\
157	0.0041338505611666\\
158	0.00413438274626842\\
159	0.00413492081722397\\
160	0.00413546486425925\\
161	0.00413601497983702\\
162	0.00413657125871991\\
163	0.00413713379803445\\
164	0.00413770269733617\\
165	0.00413827805867575\\
166	0.00413885998666593\\
167	0.00413944858854946\\
168	0.00414004397426772\\
169	0.00414064625652988\\
170	0.00414125555088306\\
171	0.00414187197578273\\
172	0.00414249565266374\\
173	0.00414312670601149\\
174	0.00414376526343341\\
175	0.00414441145573073\\
176	0.0041450654169698\\
177	0.00414572728455374\\
178	0.00414639719929339\\
179	0.00414707530547805\\
180	0.00414776175094538\\
181	0.00414845668715073\\
182	0.00414916026923534\\
183	0.0041498726560934\\
184	0.00415059401043789\\
185	0.00415132449886459\\
186	0.00415206429191454\\
187	0.00415281356413448\\
188	0.00415357249413494\\
189	0.0041543412646461\\
190	0.0041551200625707\\
191	0.00415590907903425\\
192	0.00415670850943206\\
193	0.00415751855347266\\
194	0.0041583394152178\\
195	0.00415917130311833\\
196	0.00416001443004602\\
197	0.00416086901332105\\
198	0.00416173527473464\\
199	0.00416261344056694\\
200	0.00416350374159974\\
201	0.00416440641312374\\
202	0.00416532169494035\\
203	0.00416624983135754\\
204	0.00416719107117978\\
205	0.00416814566769192\\
206	0.00416911387863618\\
207	0.00417009596618309\\
208	0.00417109219689547\\
209	0.00417210284168541\\
210	0.00417312817576454\\
211	0.00417416847858684\\
212	0.00417522403378432\\
213	0.0041762951290953\\
214	0.00417738205628507\\
215	0.00417848511105917\\
216	0.00417960459296875\\
217	0.0041807408053083\\
218	0.00418189405500533\\
219	0.00418306465250219\\
220	0.00418425291162984\\
221	0.00418545914947342\\
222	0.0041866836862296\\
223	0.00418792684505579\\
224	0.00418918895191122\\
225	0.00419047033538989\\
226	0.00419177132654538\\
227	0.00419309225870856\\
228	0.00419443346729708\\
229	0.00419579528961866\\
230	0.00419717806466829\\
231	0.00419858213292799\\
232	0.00420000783619921\\
233	0.00420145551752562\\
234	0.00420292552087986\\
235	0.00420441819099777\\
236	0.00420593387324132\\
237	0.00420747291350122\\
238	0.00420903565815524\\
239	0.00421062245410225\\
240	0.00421223364889825\\
241	0.00421386959102594\\
242	0.00421553063033765\\
243	0.00421721711871981\\
244	0.00421892941103672\\
245	0.00422066786642197\\
246	0.00422243284999639\\
247	0.00422422473510007\\
248	0.00422604390613041\\
249	0.0042278907620704\\
250	0.00422976572077321\\
251	0.00423166922412071\\
252	0.00423360174433245\\
253	0.00423556379097977\\
254	0.00423755591868842\\
255	0.00423957873513579\\
256	0.00424163290862752\\
257	0.00424371917420263\\
258	0.00424583833787788\\
259	0.00424799127431861\\
260	0.00425017888885837\\
261	0.00425240211861172\\
262	0.00425466193363533\\
263	0.00425695933817204\\
264	0.0042592953720731\\
265	0.00426167111248506\\
266	0.00426408767458076\\
267	0.00426654621277113\\
268	0.00426904792205369\\
269	0.0042715940393867\\
270	0.00427418584508419\\
271	0.00427682466422699\\
272	0.00427951186808263\\
273	0.00428224887552587\\
274	0.00428503715444896\\
275	0.00428787822314751\\
276	0.0042907736516625\\
277	0.00429372506305063\\
278	0.0042967341345539\\
279	0.00429980259872177\\
280	0.00430293224504238\\
281	0.00430612492086036\\
282	0.00430938253134836\\
283	0.00431270703968716\\
284	0.00431610046668569\\
285	0.00431956488984633\\
286	0.00432310244353248\\
287	0.00432671531807215\\
288	0.00433040575779906\\
289	0.00433417605820851\\
290	0.004338028561485\\
291	0.00434196564953624\\
292	0.00434598973981118\\
293	0.00435010327728252\\
294	0.00435430872458364\\
295	0.0043586085500224\\
296	0.00436300521317298\\
297	0.00436750114773264\\
298	0.00437209874132715\\
299	0.00437680031196629\\
300	0.00438160808091853\\
301	0.00438652414202639\\
302	0.00439155042855168\\
303	0.00439668868214047\\
304	0.00440194040551478\\
305	0.00440730681827399\\
306	0.00441278881462466\\
307	0.00441838692466119\\
308	0.00442410128403887\\
309	0.00442993161913903\\
310	0.00443587725775034\\
311	0.00444193717853756\\
312	0.00444811011538356\\
313	0.00445439475816206\\
314	0.00446079010033279\\
315	0.0044672959489268\\
316	0.00447391631848287\\
317	0.00448065625056135\\
318	0.00448752117149148\\
319	0.00449451692340825\\
320	0.00450164981628869\\
321	0.00450892666435019\\
322	0.0045163548226895\\
323	0.00452394223476899\\
324	0.00453169748434551\\
325	0.0045396298526327\\
326	0.0045477493794233\\
327	0.00455606689500467\\
328	0.0045645940897534\\
329	0.00457334359700911\\
330	0.00458232905924628\\
331	0.0045915651923621\\
332	0.00460106784652621\\
333	0.00461085409850468\\
334	0.00462094238577422\\
335	0.00463135242098504\\
336	0.00464210518561074\\
337	0.00465322288161534\\
338	0.00466472882292711\\
339	0.00467664723130864\\
340	0.00468900278185558\\
341	0.00470182005770181\\
342	0.0047151231142723\\
343	0.00472893456204501\\
344	0.00474327429492926\\
345	0.00475815774915058\\
346	0.00477359354148361\\
347	0.00478958028699191\\
348	0.00480610232394138\\
349	0.0048231239137538\\
350	0.00484056158046822\\
351	0.00485806520159102\\
352	0.00487562323831201\\
353	0.00489322270900446\\
354	0.0049108491103253\\
355	0.00492848572021073\\
356	0.00494611359557722\\
357	0.00496371126707528\\
358	0.00498125594075324\\
359	0.0049987233684562\\
360	0.00501608709894146\\
361	0.00503331849331765\\
362	0.00505038698214896\\
363	0.0050672585025896\\
364	0.00508389293770087\\
365	0.00510024698592675\\
366	0.00511627433362389\\
367	0.00513192602333867\\
368	0.00514715109622411\\
369	0.00516189761431189\\
370	0.00517611420720388\\
371	0.00518975233876229\\
372	0.00520276938265742\\
373	0.00521513302496895\\
374	0.00522682733997493\\
375	0.00523786108158864\\
376	0.00524831508627264\\
377	0.00525861736023971\\
378	0.00526875578457428\\
379	0.00527871832643526\\
380	0.00528849317371806\\
381	0.00529806890299139\\
382	0.00530743470690838\\
383	0.00531658071304657\\
384	0.00532549831821052\\
385	0.00533418054159597\\
386	0.00534262242059119\\
387	0.0053508214454658\\
388	0.00535877802430025\\
389	0.00536649596191278\\
390	0.00537398293026189\\
391	0.00538125089273225\\
392	0.00538831642273599\\
393	0.00539520083064398\\
394	0.00540192997276148\\
395	0.00540853355941611\\
396	0.00541504371526052\\
397	0.00542149244662168\\
398	0.00542789359962197\\
399	0.00543425068402017\\
400	0.00544056657450049\\
401	0.00544684484705187\\
402	0.00545308979767841\\
403	0.00545930644642226\\
404	0.00546550052258368\\
405	0.00547167842660904\\
406	0.0054778471637682\\
407	0.00548401424463447\\
408	0.00549018754787465\\
409	0.0054963751420991\\
410	0.00550258506634743\\
411	0.0055088250732971\\
412	0.00551510234635268\\
413	0.00552142321306698\\
414	0.00552779289425235\\
415	0.00553421535301886\\
416	0.00554069376033234\\
417	0.00554723114868104\\
418	0.00555383060849965\\
419	0.00556049526235636\\
420	0.0055672282365609\\
421	0.00557403263085659\\
422	0.00558091148717831\\
423	0.00558786775882935\\
424	0.00559490428182437\\
425	0.00560202375054027\\
426	0.00560922870016655\\
427	0.00561652149865555\\
428	0.00562390435080193\\
429	0.00563137931650108\\
430	0.00563894834380563\\
431	0.00564661331460629\\
432	0.00565437608506419\\
433	0.00566223849751214\\
434	0.00567020237707114\\
435	0.00567826952907412\\
436	0.00568644173746758\\
437	0.00569472076433976\\
438	0.0057031083506825\\
439	0.00571160621843063\\
440	0.00572021607373206\\
441	0.00572893961128431\\
442	0.00573777851943147\\
443	0.00574673448555957\\
444	0.00575580920118335\\
445	0.00576500436602482\\
446	0.00577432169042485\\
447	0.00578376289640306\\
448	0.00579332971842165\\
449	0.00580302390431187\\
450	0.00581284721633995\\
451	0.0058228014323806\\
452	0.00583288834715807\\
453	0.00584310977350766\\
454	0.0058534675436082\\
455	0.005863963510137\\
456	0.00587459954730718\\
457	0.00588537755176319\\
458	0.00589629944333533\\
459	0.00590736716568488\\
460	0.00591858268690471\\
461	0.00592994800012101\\
462	0.00594146512410113\\
463	0.00595313610386228\\
464	0.00596496301127626\\
465	0.00597694794566633\\
466	0.00598909303439411\\
467	0.00600140043343626\\
468	0.00601387232795212\\
469	0.00602651093284681\\
470	0.00603931849333446\\
471	0.00605229728550793\\
472	0.00606544961692059\\
473	0.0060787778271834\\
474	0.0060922842885795\\
475	0.00610597140669768\\
476	0.00611984162108701\\
477	0.00613389740593545\\
478	0.00614814127077531\\
479	0.00616257576121957\\
480	0.00617720345973286\\
481	0.006192026986442\\
482	0.00620704899999094\\
483	0.00622227219844596\\
484	0.00623769932025699\\
485	0.00625333314528213\\
486	0.00626917649588283\\
487	0.00628523223809936\\
488	0.00630150328291617\\
489	0.00631799258762984\\
490	0.00633470315733296\\
491	0.00635163804653024\\
492	0.00636880036090543\\
493	0.00638619325926091\\
494	0.00640381995565534\\
495	0.00642168372176941\\
496	0.00643978788953463\\
497	0.00645813585406661\\
498	0.00647673107695201\\
499	0.00649557708994683\\
500	0.00651467749915488\\
501	0.00653403598976808\\
502	0.00655365633146543\\
503	0.00657354238458662\\
504	0.00659369810721729\\
505	0.00661412756335069\\
506	0.00663483493232087\\
507	0.00665582451974051\\
508	0.00667710077022156\\
509	0.0066986682822085\\
510	0.00672053182533012\\
511	0.00674269636055962\\
512	0.00676516706373502\\
513	0.00678794935267998\\
514	0.00681104892293287\\
515	0.0068344717947585\\
516	0.00685822436532986\\
517	0.00688231353410591\\
518	0.00690674657611931\\
519	0.0069315311009031\\
520	0.00695667520078294\\
521	0.00698218171358456\\
522	0.00700806184724271\\
523	0.00703433235251492\\
524	0.0070610103146217\\
525	0.0070881189831278\\
526	0.00711567716246021\\
527	0.00714369741365174\\
528	0.00717219191737849\\
529	0.00720117163858217\\
530	0.00723064483866799\\
531	0.00726061777558896\\
532	0.00729109401196795\\
533	0.00732206999733117\\
534	0.00735350410895862\\
535	0.00738534794863182\\
536	0.00741766494491498\\
537	0.00745056597663264\\
538	0.00748420149227407\\
539	0.0075186282540824\\
540	0.00755391134287273\\
541	0.00759010429212257\\
542	0.00762725429437914\\
543	0.00766541139405159\\
544	0.00770463275832746\\
545	0.00774498450574268\\
546	0.00778654350476932\\
547	0.00782943284053057\\
548	0.007873812470974\\
549	0.00791685142016626\\
550	0.00795760885345326\\
551	0.00799826335611184\\
552	0.008039343319122\\
553	0.00808097441322759\\
554	0.00812317053076772\\
555	0.00816590702824844\\
556	0.00820914385485902\\
557	0.00825282989559372\\
558	0.00829690046744554\\
559	0.00834127797993642\\
560	0.00838587427252687\\
561	0.00842944747538469\\
562	0.00847248750936672\\
563	0.00851573631428289\\
564	0.00855920311781243\\
565	0.00860285881927415\\
566	0.00864667312030307\\
567	0.00869051061764692\\
568	0.00873403472335589\\
569	0.00877786227543177\\
570	0.00882199645300022\\
571	0.00886641582265344\\
572	0.00891109504133455\\
573	0.00895600622677124\\
574	0.00900111882085289\\
575	0.00904639944680352\\
576	0.00909181177476488\\
577	0.00913731640197366\\
578	0.00918287075499839\\
579	0.00922842902355639\\
580	0.00927394213803805\\
581	0.00931935780612538\\
582	0.00936462062793332\\
583	0.00940967231402828\\
584	0.00945445203657523\\
585	0.0094988969504773\\
586	0.00954294292772307\\
587	0.00958652555102228\\
588	0.00962958140299686\\
589	0.0096720496394373\\
590	0.0097138736812812\\
591	0.0097550024176802\\
592	0.00979538909575469\\
593	0.00983498278816331\\
594	0.00987369853931868\\
595	0.00991118387968948\\
596	0.00994658651044256\\
597	0.00997788999445116\\
598	0.010000292044645\\
599	0\\
600	0\\
};
\addplot [color=mycolor4,solid,forget plot]
  table[row sep=crcr]{%
1	0.00409664868257122\\
2	0.00409668962381598\\
3	0.00409673130488141\\
4	0.00409677373721799\\
5	0.00409681693239205\\
6	0.00409686090208509\\
7	0.00409690565809265\\
8	0.00409695121232338\\
9	0.00409699757679793\\
10	0.00409704476364767\\
11	0.00409709278511336\\
12	0.00409714165354387\\
13	0.00409719138139446\\
14	0.00409724198122542\\
15	0.00409729346570032\\
16	0.00409734584758411\\
17	0.00409739913974129\\
18	0.00409745335513396\\
19	0.0040975085068197\\
20	0.00409756460794935\\
21	0.0040976216717648\\
22	0.00409767971159646\\
23	0.00409773874086082\\
24	0.00409779877305787\\
25	0.00409785982176835\\
26	0.00409792190065095\\
27	0.00409798502343938\\
28	0.0040980492039394\\
29	0.0040981144560256\\
30	0.00409818079363827\\
31	0.00409824823078012\\
32	0.00409831678151281\\
33	0.00409838645995345\\
34	0.00409845728027101\\
35	0.00409852925668279\\
36	0.00409860240345046\\
37	0.00409867673487639\\
38	0.00409875226529966\\
39	0.0040988290090922\\
40	0.00409890698065467\\
41	0.00409898619441248\\
42	0.00409906666481163\\
43	0.00409914840631443\\
44	0.00409923143339559\\
45	0.00409931576053771\\
46	0.00409940140222727\\
47	0.00409948837295028\\
48	0.00409957668718817\\
49	0.00409966635941348\\
50	0.00409975740408583\\
51	0.00409984983564766\\
52	0.00409994366852032\\
53	0.00410003891709999\\
54	0.0041001355957539\\
55	0.0041002337188164\\
56	0.00410033330058548\\
57	0.00410043435531914\\
58	0.0041005368972321\\
59	0.00410064094049262\\
60	0.00410074649921956\\
61	0.00410085358747956\\
62	0.00410096221928472\\
63	0.00410107240859023\\
64	0.00410118416929251\\
65	0.00410129751522758\\
66	0.00410141246016985\\
67	0.00410152901783121\\
68	0.00410164720186052\\
69	0.00410176702584359\\
70	0.00410188850330351\\
71	0.00410201164770159\\
72	0.0041021364724387\\
73	0.00410226299085719\\
74	0.00410239121624346\\
75	0.00410252116183092\\
76	0.00410265284080381\\
77	0.00410278626630155\\
78	0.00410292145142377\\
79	0.00410305840923608\\
80	0.00410319715277667\\
81	0.00410333769506352\\
82	0.00410348004910242\\
83	0.00410362422789598\\
84	0.00410377024445325\\
85	0.00410391811180042\\
86	0.00410406784299215\\
87	0.00410421945112408\\
88	0.00410437294934605\\
89	0.00410452835087618\\
90	0.0041046856690162\\
91	0.00410484491716742\\
92	0.00410500610884766\\
93	0.00410516925770937\\
94	0.00410533437755831\\
95	0.00410550148237347\\
96	0.00410567058632778\\
97	0.00410584170380952\\
98	0.00410601484944506\\
99	0.0041061900381219\\
100	0.00410636728501282\\
101	0.00410654660560078\\
102	0.00410672801570444\\
103	0.00410691153150435\\
104	0.00410709716956986\\
105	0.00410728494688636\\
106	0.00410747488088329\\
107	0.00410766698946234\\
108	0.00410786129102603\\
109	0.00410805780450654\\
110	0.00410825654939485\\
111	0.00410845754576972\\
112	0.00410866081432693\\
113	0.00410886637640819\\
114	0.00410907425402994\\
115	0.00410928446991207\\
116	0.0041094970475059\\
117	0.00410971201102203\\
118	0.00410992938545747\\
119	0.0041101491966223\\
120	0.00411037147116538\\
121	0.00411059623659954\\
122	0.00411082352132581\\
123	0.00411105335465676\\
124	0.00411128576683904\\
125	0.00411152078907476\\
126	0.00411175845354187\\
127	0.00411199879341387\\
128	0.00411224184287793\\
129	0.00411248763715254\\
130	0.00411273621250376\\
131	0.00411298760626071\\
132	0.00411324185682997\\
133	0.00411349900370904\\
134	0.00411375908749892\\
135	0.00411402214991591\\
136	0.00411428823380235\\
137	0.00411455738313678\\
138	0.00411482964304308\\
139	0.00411510505979911\\
140	0.00411538368084452\\
141	0.00411566555478786\\
142	0.00411595073141336\\
143	0.00411623926168708\\
144	0.00411653119776312\\
145	0.00411682659298973\\
146	0.00411712550191647\\
147	0.00411742798030309\\
148	0.00411773408513234\\
149	0.00411804387462939\\
150	0.0041183574082853\\
151	0.00411867474685224\\
152	0.00411899595236577\\
153	0.00411932108816943\\
154	0.00411965021893952\\
155	0.00411998341070969\\
156	0.00412032073089582\\
157	0.00412066224832075\\
158	0.004121008033239\\
159	0.00412135815736145\\
160	0.0041217126938801\\
161	0.00412207171749242\\
162	0.00412243530442588\\
163	0.00412280353246209\\
164	0.004123176480961\\
165	0.00412355423088474\\
166	0.0041239368648212\\
167	0.00412432446700746\\
168	0.00412471712335291\\
169	0.00412511492146215\\
170	0.00412551795065742\\
171	0.00412592630200085\\
172	0.00412634006831619\\
173	0.0041267593442104\\
174	0.00412718422609469\\
175	0.00412761481220526\\
176	0.00412805120262364\\
177	0.00412849349929665\\
178	0.00412894180605595\\
179	0.00412939622863741\\
180	0.00412985687469984\\
181	0.00413032385384373\\
182	0.00413079727762944\\
183	0.00413127725959528\\
184	0.00413176391527536\\
185	0.00413225736221742\\
186	0.00413275772000041\\
187	0.00413326511025216\\
188	0.00413377965666715\\
189	0.00413430148502457\\
190	0.00413483072320659\\
191	0.00413536750121703\\
192	0.00413591195120078\\
193	0.00413646420746377\\
194	0.00413702440649402\\
195	0.00413759268698367\\
196	0.00413816918985233\\
197	0.00413875405827195\\
198	0.00413934743769338\\
199	0.00413994947587498\\
200	0.00414056032291342\\
201	0.00414118013127704\\
202	0.00414180905584204\\
203	0.00414244725393175\\
204	0.00414309488535929\\
205	0.00414375211247407\\
206	0.00414441910021256\\
207	0.00414509601615324\\
208	0.00414578303057657\\
209	0.0041464803165302\\
210	0.0041471880498995\\
211	0.0041479064094845\\
212	0.00414863557708264\\
213	0.00414937573757863\\
214	0.00415012707904113\\
215	0.00415088979282709\\
216	0.00415166407369351\\
217	0.00415245011991762\\
218	0.00415324813342542\\
219	0.00415405831992878\\
220	0.00415488088907169\\
221	0.00415571605458558\\
222	0.00415656403445428\\
223	0.00415742505108853\\
224	0.00415829933151019\\
225	0.00415918710754658\\
226	0.0041600886160349\\
227	0.00416100409903648\\
228	0.0041619338040616\\
229	0.00416287798430444\\
230	0.00416383689888962\\
231	0.00416481081313237\\
232	0.00416579999881265\\
233	0.00416680473444185\\
234	0.00416782530555101\\
235	0.00416886200499009\\
236	0.00416991513323826\\
237	0.00417098499872553\\
238	0.0041720719181655\\
239	0.00417317621689987\\
240	0.00417429822925369\\
241	0.00417543829890231\\
242	0.00417659677924825\\
243	0.00417777403380749\\
244	0.00417897043660341\\
245	0.00418018637256503\\
246	0.00418142223792556\\
247	0.00418267844061465\\
248	0.00418395540063496\\
249	0.0041852535504115\\
250	0.00418657333510657\\
251	0.00418791521289753\\
252	0.0041892796551678\\
253	0.00419066714660251\\
254	0.00419207818516755\\
255	0.00419351328196016\\
256	0.00419497296095737\\
257	0.00419645775879288\\
258	0.0041979682245596\\
259	0.00419950491873774\\
260	0.00420106841334051\\
261	0.004202659292063\\
262	0.00420427815043769\\
263	0.00420592559600409\\
264	0.00420760224848415\\
265	0.00420930873987836\\
266	0.00421104571460549\\
267	0.00421281382965351\\
268	0.0042146137547326\\
269	0.00421644617243015\\
270	0.00421831177836837\\
271	0.00422021128136477\\
272	0.0042221454035958\\
273	0.00422411488076409\\
274	0.00422612046226925\\
275	0.00422816291138236\\
276	0.00423024300542409\\
277	0.00423236153594855\\
278	0.00423451930894415\\
279	0.00423671714508118\\
280	0.00423895587990949\\
281	0.00424123636400916\\
282	0.00424355946318336\\
283	0.00424592605864222\\
284	0.00424833704720788\\
285	0.0042507933416628\\
286	0.00425329587106279\\
287	0.00425584558103343\\
288	0.00425844343407693\\
289	0.00426109040985437\\
290	0.00426378750546487\\
291	0.00426653573608827\\
292	0.00426933613552122\\
293	0.00427218975681514\\
294	0.00427509767304613\\
295	0.00427806097825504\\
296	0.00428108078860607\\
297	0.00428415824382355\\
298	0.00428729450898293\\
299	0.00429049077675482\\
300	0.00429374827024759\\
301	0.00429706824668784\\
302	0.00430045200220975\\
303	0.00430390087654047\\
304	0.00430741625931877\\
305	0.00431099959792094\\
306	0.00431465240683752\\
307	0.00431837627874854\\
308	0.0043221728973601\\
309	0.00432604405188818\\
310	0.00432999165274966\\
311	0.00433401774763531\\
312	0.00433812453795846\\
313	0.00434231439422532\\
314	0.00434658986551328\\
315	0.00435095368955819\\
316	0.00435540871945394\\
317	0.00435995791231748\\
318	0.00436460433143802\\
319	0.00436935114925647\\
320	0.00437420164829873\\
321	0.00437915922141272\\
322	0.00438422737192456\\
323	0.00438940971298996\\
324	0.00439470996591586\\
325	0.00440013195682813\\
326	0.00440567960975303\\
327	0.00441135694093633\\
328	0.00441716805169503\\
329	0.00442311711736991\\
330	0.00442920837349606\\
331	0.00443544609938092\\
332	0.00444183460149077\\
333	0.00444837819425593\\
334	0.00445508116082882\\
335	0.00446194771879704\\
336	0.00446898198085834\\
337	0.00447618791006925\\
338	0.00448356926800846\\
339	0.0044911295503311\\
340	0.00449887193144283\\
341	0.00450679923471691\\
342	0.00451491389170535\\
343	0.00452321791763498\\
344	0.00453171291502352\\
345	0.00454040012236478\\
346	0.00454928053170496\\
347	0.0045583551079548\\
348	0.00456762515584229\\
349	0.0045770929316558\\
350	0.00458676283704058\\
351	0.00459664640722626\\
352	0.00460675613519573\\
353	0.00461710555285139\\
354	0.00462770926379756\\
355	0.00463858304664908\\
356	0.00464974395497892\\
357	0.00466121052009476\\
358	0.0046730027990592\\
359	0.00468514236738451\\
360	0.00469765236112168\\
361	0.00471055747325224\\
362	0.00472388374758736\\
363	0.00473765833306152\\
364	0.00475190959924341\\
365	0.00476666685219704\\
366	0.00478195989826624\\
367	0.00479781840288387\\
368	0.00481427097438884\\
369	0.00483134387867026\\
370	0.00484905924751179\\
371	0.00486743258499649\\
372	0.00488646990496595\\
373	0.00490616315581023\\
374	0.00492648442273783\\
375	0.00494737823307882\\
376	0.00496871592348503\\
377	0.00499002714010541\\
378	0.00501128758581625\\
379	0.00503247096782956\\
380	0.0050535488223626\\
381	0.00507449094119751\\
382	0.00509526083085377\\
383	0.0051158164758645\\
384	0.00513611189221272\\
385	0.00515609703013776\\
386	0.0051757177862393\\
387	0.00519491617700118\\
388	0.00521363074694414\\
389	0.00523179732370554\\
390	0.00524935012226\\
391	0.00526622344716711\\
392	0.00528235423550805\\
393	0.00529768567804759\\
394	0.00531217237850484\\
395	0.00532578763345195\\
396	0.0053385332321474\\
397	0.00535045281232267\\
398	0.00536206547182922\\
399	0.00537344406308905\\
400	0.00538457529139072\\
401	0.00539544694147929\\
402	0.00540604824494743\\
403	0.0054163703009139\\
404	0.00542640655005739\\
405	0.0054361532980586\\
406	0.00544561028227294\\
407	0.00545478126853225\\
408	0.0054636746509502\\
409	0.00547230401387973\\
410	0.0054806885824226\\
411	0.00548885346400712\\
412	0.00549682954230167\\
413	0.00550465281782058\\
414	0.00551236291050277\\
415	0.00552000033082267\\
416	0.00552758999632444\\
417	0.00553514042149178\\
418	0.00554265655480703\\
419	0.00555014421767857\\
420	0.00555761009506642\\
421	0.00556506169942287\\
422	0.00557250730179434\\
423	0.00557995582371182\\
424	0.00558741668426262\\
425	0.00559489959813635\\
426	0.00560241432305374\\
427	0.00560997035970928\\
428	0.00561757661503771\\
429	0.00562524105176726\\
430	0.00563297036571746\\
431	0.00564076975981897\\
432	0.00564864322931405\\
433	0.00565659441994265\\
434	0.0056646270121865\\
435	0.00567274468719499\\
436	0.00568095109050692\\
437	0.00568924979480068\\
438	0.00569764426333909\\
439	0.00570613781625333\\
440	0.00571473360229106\\
441	0.00572343457906517\\
442	0.00573224350506446\\
443	0.00574116294654159\\
444	0.00575019530158278\\
445	0.00575934284175239\\
446	0.0057686077680368\\
447	0.00577799225250322\\
448	0.00578749844787233\\
449	0.00579712848460786\\
450	0.00580688446918496\\
451	0.00581676848371286\\
452	0.00582678258703619\\
453	0.00583692881735942\\
454	0.00584720919632644\\
455	0.00585762573434246\\
456	0.00586818043675091\\
457	0.0058788753102902\\
458	0.00588971236908199\\
459	0.00590069363930562\\
460	0.00591182116179189\\
461	0.00592309699327209\\
462	0.00593452320750874\\
463	0.00594610189660377\\
464	0.00595783517244888\\
465	0.00596972516827276\\
466	0.00598177404023063\\
467	0.00599398396897853\\
468	0.00600635716117462\\
469	0.00601889585085943\\
470	0.00603160230068508\\
471	0.00604447880299224\\
472	0.00605752768077079\\
473	0.00607075128858001\\
474	0.00608415201348621\\
475	0.00609773227602922\\
476	0.0061114945312139\\
477	0.00612544126952421\\
478	0.00613957501795931\\
479	0.00615389834109338\\
480	0.00616841384216404\\
481	0.00618312416419666\\
482	0.00619803199117528\\
483	0.00621314004927233\\
484	0.00622845110815102\\
485	0.00624396798235272\\
486	0.00625969353278066\\
487	0.00627563066829013\\
488	0.00629178234739776\\
489	0.00630815158012391\\
490	0.00632474142998397\\
491	0.00634155501614687\\
492	0.00635859551578096\\
493	0.00637586616661024\\
494	0.0063933702697064\\
495	0.00641111119254501\\
496	0.00642909237235806\\
497	0.0064473173198182\\
498	0.00646578962309558\\
499	0.00648451295233245\\
500	0.00650349106458752\\
501	0.00652272780930867\\
502	0.00654222713440037\\
503	0.00656199309296133\\
504	0.00658202985077848\\
505	0.00660234169467454\\
506	0.00662293304182132\\
507	0.00664380845014584\\
508	0.00666497262997551\\
509	0.00668643045708999\\
510	0.00670818698737221\\
511	0.00673024747328626\\
512	0.00675261738244383\\
513	0.00677530241856389\\
514	0.00679830854505638\\
515	0.00682164201148414\\
516	0.00684530938344261\\
517	0.0068693175757188\\
518	0.00689367389591976\\
519	0.00691838609654099\\
520	0.00694346243095817\\
521	0.00696891178565863\\
522	0.00699474352995971\\
523	0.00702096749344242\\
524	0.00704759410736137\\
525	0.00707462948156095\\
526	0.00710208597353999\\
527	0.00712998298566222\\
528	0.00715834050109882\\
529	0.00718718223334858\\
530	0.00721653322220688\\
531	0.00724640874563171\\
532	0.00727682321657789\\
533	0.00730779033496002\\
534	0.00733932054955617\\
535	0.00737142094251038\\
536	0.00740409155219721\\
537	0.00743729626273568\\
538	0.00747098540990441\\
539	0.00750522364908087\\
540	0.00754010540771196\\
541	0.00757580349715638\\
542	0.00761236890146435\\
543	0.00764985959963282\\
544	0.00768832507234716\\
545	0.00772781694935591\\
546	0.00776839367582466\\
547	0.00781012138020359\\
548	0.00785307687559534\\
549	0.00789738450562302\\
550	0.0079432034083031\\
551	0.00798788543240736\\
552	0.00803044603523655\\
553	0.00807251022837668\\
554	0.00811489868865496\\
555	0.00815779179765849\\
556	0.0082012227307077\\
557	0.00824515999795583\\
558	0.00828955575295806\\
559	0.00833434738316874\\
560	0.00837945454762876\\
561	0.00842479121386117\\
562	0.0084696778803419\\
563	0.00851341295005585\\
564	0.00855721779447572\\
565	0.00860119556072858\\
566	0.00864533797554444\\
567	0.00868961368428026\\
568	0.00873393789191277\\
569	0.00877785590949144\\
570	0.00882199632513923\\
571	0.0088664158097615\\
572	0.00891109503814012\\
573	0.00895600622559429\\
574	0.00900111882029819\\
575	0.00904639944650971\\
576	0.00909181177460439\\
577	0.00913731640188469\\
578	0.00918287075494969\\
579	0.00922842902353094\\
580	0.00927394213802573\\
581	0.00931935780612039\\
582	0.00936462062793178\\
583	0.00940967231402812\\
584	0.00945445203657523\\
585	0.00949889695047729\\
586	0.00954294292772307\\
587	0.00958652555102227\\
588	0.00962958140299686\\
589	0.00967204963943729\\
590	0.0097138736812812\\
591	0.0097550024176802\\
592	0.00979538909575469\\
593	0.00983498278816331\\
594	0.00987369853931868\\
595	0.00991118387968948\\
596	0.00994658651044256\\
597	0.00997788999445116\\
598	0.010000292044645\\
599	0\\
600	0\\
};
\addplot [color=mycolor5,solid,forget plot]
  table[row sep=crcr]{%
1	0.00409619160294684\\
2	0.00409622158226429\\
3	0.00409625205221089\\
4	0.00409628301926326\\
5	0.00409631448994264\\
6	0.00409634647081406\\
7	0.00409637896848576\\
8	0.00409641198960823\\
9	0.0040964455408736\\
10	0.0040964796290148\\
11	0.00409651426080469\\
12	0.00409654944305536\\
13	0.00409658518261726\\
14	0.0040966214863783\\
15	0.00409665836126297\\
16	0.00409669581423155\\
17	0.00409673385227931\\
18	0.00409677248243541\\
19	0.00409681171176222\\
20	0.00409685154735435\\
21	0.00409689199633775\\
22	0.00409693306586891\\
23	0.00409697476313393\\
24	0.00409701709534774\\
25	0.00409706006975311\\
26	0.00409710369362\\
27	0.00409714797424461\\
28	0.00409719291894871\\
29	0.00409723853507883\\
30	0.00409728483000557\\
31	0.00409733181112282\\
32	0.00409737948584722\\
33	0.00409742786161753\\
34	0.00409747694589413\\
35	0.00409752674615837\\
36	0.00409757726991232\\
37	0.00409762852467825\\
38	0.00409768051799848\\
39	0.00409773325743501\\
40	0.00409778675056946\\
41	0.00409784100500304\\
42	0.00409789602835649\\
43	0.0040979518282704\\
44	0.00409800841240532\\
45	0.00409806578844213\\
46	0.00409812396408264\\
47	0.00409818294705012\\
48	0.00409824274509012\\
49	0.00409830336597133\\
50	0.00409836481748665\\
51	0.00409842710745441\\
52	0.00409849024371973\\
53	0.00409855423415614\\
54	0.00409861908666724\\
55	0.00409868480918863\\
56	0.00409875140969016\\
57	0.00409881889617818\\
58	0.00409888727669801\\
59	0.00409895655933686\\
60	0.0040990267522268\\
61	0.00409909786354797\\
62	0.00409916990153199\\
63	0.00409924287446588\\
64	0.00409931679069591\\
65	0.00409939165863193\\
66	0.00409946748675191\\
67	0.00409954428360668\\
68	0.00409962205782512\\
69	0.00409970081811947\\
70	0.00409978057329107\\
71	0.00409986133223633\\
72	0.00409994310395293\\
73	0.00410002589754655\\
74	0.00410010972223754\\
75	0.00410019458736832\\
76	0.00410028050241081\\
77	0.00410036747697415\\
78	0.00410045552081292\\
79	0.00410054464383554\\
80	0.00410063485611294\\
81	0.00410072616788758\\
82	0.00410081858958281\\
83	0.00410091213181236\\
84	0.00410100680539028\\
85	0.00410110262134104\\
86	0.00410119959090989\\
87	0.00410129772557349\\
88	0.0041013970370508\\
89	0.00410149753731418\\
90	0.00410159923860063\\
91	0.00410170215342322\\
92	0.00410180629458288\\
93	0.00410191167518001\\
94	0.0041020183086265\\
95	0.00410212620865782\\
96	0.00410223538934498\\
97	0.00410234586510692\\
98	0.00410245765072262\\
99	0.00410257076134334\\
100	0.00410268521250494\\
101	0.00410280102014015\\
102	0.00410291820059057\\
103	0.00410303677061892\\
104	0.004103156747421\\
105	0.00410327814863759\\
106	0.00410340099236612\\
107	0.00410352529717227\\
108	0.00410365108210135\\
109	0.00410377836668947\\
110	0.00410390717097453\\
111	0.00410403751550685\\
112	0.00410416942135973\\
113	0.00410430291013961\\
114	0.00410443800399612\\
115	0.00410457472563163\\
116	0.00410471309831071\\
117	0.00410485314586941\\
118	0.00410499489272402\\
119	0.00410513836387979\\
120	0.00410528358493932\\
121	0.00410543058211083\\
122	0.00410557938221616\\
123	0.00410573001269859\\
124	0.00410588250163056\\
125	0.0041060368777213\\
126	0.00410619317032435\\
127	0.00410635140944498\\
128	0.00410651162574777\\
129	0.00410667385056405\\
130	0.00410683811589949\\
131	0.00410700445444173\\
132	0.00410717289956831\\
133	0.00410734348535462\\
134	0.00410751624658221\\
135	0.00410769121874715\\
136	0.00410786843806889\\
137	0.00410804794149926\\
138	0.00410822976673195\\
139	0.0041084139522122\\
140	0.00410860053714694\\
141	0.00410878956151536\\
142	0.00410898106607968\\
143	0.00410917509239668\\
144	0.00410937168282914\\
145	0.00410957088055834\\
146	0.00410977272959654\\
147	0.00410997727480027\\
148	0.00411018456188428\\
149	0.00411039463743527\\
150	0.0041106075489235\\
151	0.0041108233447165\\
152	0.00411104207409314\\
153	0.00411126378725806\\
154	0.00411148853535573\\
155	0.00411171637048515\\
156	0.00411194734571431\\
157	0.00411218151509491\\
158	0.00411241893367731\\
159	0.00411265965752546\\
160	0.00411290374373211\\
161	0.00411315125043418\\
162	0.00411340223682819\\
163	0.00411365676318623\\
164	0.00411391489087168\\
165	0.00411417668235555\\
166	0.00411444220123294\\
167	0.00411471151223977\\
168	0.00411498468126977\\
169	0.00411526177539199\\
170	0.0041155428628684\\
171	0.00411582801317217\\
172	0.00411611729700612\\
173	0.00411641078632185\\
174	0.0041167085543392\\
175	0.00411701067556636\\
176	0.00411731722582052\\
177	0.00411762828224914\\
178	0.00411794392335213\\
179	0.00411826422900421\\
180	0.00411858928047872\\
181	0.00411891916047184\\
182	0.00411925395312786\\
183	0.00411959374406541\\
184	0.00411993862040482\\
185	0.00412028867079633\\
186	0.00412064398544972\\
187	0.00412100465616507\\
188	0.00412137077636482\\
189	0.00412174244112731\\
190	0.00412211974722161\\
191	0.00412250279314403\\
192	0.00412289167915623\\
193	0.00412328650732495\\
194	0.00412368738156349\\
195	0.00412409440767505\\
196	0.00412450769339802\\
197	0.00412492734845314\\
198	0.00412535348459278\\
199	0.00412578621565219\\
200	0.00412622565760295\\
201	0.00412667192860869\\
202	0.00412712514908281\\
203	0.00412758544174877\\
204	0.00412805293170244\\
205	0.00412852774647686\\
206	0.00412901001610912\\
207	0.00412949987320979\\
208	0.00412999745303449\\
209	0.00413050289355773\\
210	0.00413101633554895\\
211	0.00413153792265059\\
212	0.00413206780145841\\
213	0.00413260612160365\\
214	0.00413315303583702\\
215	0.00413370870011443\\
216	0.00413427327368446\\
217	0.00413484691917709\\
218	0.00413542980269359\\
219	0.00413602209389782\\
220	0.004136623966108\\
221	0.00413723559638941\\
222	0.0041378571656474\\
223	0.00413848885872059\\
224	0.00413913086447404\\
225	0.00413978337589208\\
226	0.00414044659017064\\
227	0.00414112070880878\\
228	0.0041418059376992\\
229	0.00414250248721764\\
230	0.00414321057231111\\
231	0.00414393041258449\\
232	0.00414466223238378\\
233	0.00414540626087849\\
234	0.00414616273214168\\
235	0.00414693188522777\\
236	0.00414771396424824\\
237	0.00414850921844467\\
238	0.00414931790225951\\
239	0.00415014027540443\\
240	0.00415097660292645\\
241	0.00415182715527154\\
242	0.00415269220834638\\
243	0.0041535720435779\\
244	0.00415446694797091\\
245	0.00415537721416414\\
246	0.00415630314048469\\
247	0.00415724503100112\\
248	0.00415820319557572\\
249	0.00415917794991717\\
250	0.00416016961563463\\
251	0.00416117852029278\\
252	0.00416220499746934\\
253	0.0041632493868176\\
254	0.0041643120341376\\
255	0.00416539329146222\\
256	0.00416649351716763\\
257	0.00416761307609297\\
258	0.00416875233960254\\
259	0.00416991168572639\\
260	0.00417109149930953\\
261	0.00417229217217076\\
262	0.00417351410327181\\
263	0.00417475769889585\\
264	0.00417602337283\\
265	0.00417731154656293\\
266	0.00417862264949541\\
267	0.00417995711916321\\
268	0.00418131540147287\\
269	0.00418269795095136\\
270	0.00418410523100984\\
271	0.00418553771422237\\
272	0.00418699588261993\\
273	0.00418848022800048\\
274	0.00418999125225548\\
275	0.00419152946771326\\
276	0.00419309539750048\\
277	0.00419468957592257\\
278	0.0041963125488651\\
279	0.00419796487420833\\
280	0.00419964712225698\\
281	0.00420135987619224\\
282	0.00420310373254267\\
283	0.00420487930167866\\
284	0.00420668720833811\\
285	0.00420852809216799\\
286	0.00421040260828383\\
287	0.00421231142784932\\
288	0.00421425523867425\\
289	0.00421623474583657\\
290	0.00421825067235086\\
291	0.00422030375984715\\
292	0.00422239476927642\\
293	0.00422452448164367\\
294	0.00422669369876994\\
295	0.00422890324408457\\
296	0.0042311539634484\\
297	0.00423344672600954\\
298	0.00423578242509367\\
299	0.00423816197913298\\
300	0.00424058633263962\\
301	0.00424305645721312\\
302	0.00424557335247821\\
303	0.00424813804707692\\
304	0.00425075159967099\\
305	0.00425341509992398\\
306	0.00425612966943876\\
307	0.00425889646261729\\
308	0.00426171666739985\\
309	0.0042645915058305\\
310	0.00426752223441344\\
311	0.00427051014433791\\
312	0.00427355656159489\\
313	0.00427666284695063\\
314	0.00427983039637195\\
315	0.00428306063945929\\
316	0.00428635503992115\\
317	0.00428971509653306\\
318	0.0042931423442123\\
319	0.00429663835501139\\
320	0.00430020473916595\\
321	0.00430384314625455\\
322	0.00430755526643084\\
323	0.0043113428317262\\
324	0.0043152076173955\\
325	0.0043191514432586\\
326	0.00432317617543228\\
327	0.00432728372825828\\
328	0.00433147606632453\\
329	0.00433575520676757\\
330	0.00434012322199784\\
331	0.00434458224309119\\
332	0.00434913446355973\\
333	0.0043537821425117\\
334	0.00435852760938808\\
335	0.00436337326953708\\
336	0.00436832161071977\\
337	0.00437337521052818\\
338	0.00437853674471211\\
339	0.00438380899845113\\
340	0.00438919488129389\\
341	0.00439469744265264\\
342	0.00440031988971178\\
343	0.00440606560777859\\
344	0.00441193818286184\\
345	0.00441794142585865\\
346	0.00442407939712489\\
347	0.00443035642955957\\
348	0.0044367771487174\\
349	0.00444334648174883\\
350	0.00445006965806816\\
351	0.00445695208320719\\
352	0.00446399933593276\\
353	0.00447121716027577\\
354	0.00447861146097566\\
355	0.00448618829689002\\
356	0.00449395387742492\\
357	0.00450191454159263\\
358	0.00451007673028543\\
359	0.00451844695920616\\
360	0.00452703178309679\\
361	0.00453583774289785\\
362	0.00454487131592899\\
363	0.00455413889370712\\
364	0.00456364672977185\\
365	0.00457340088478789\\
366	0.00458340717185425\\
367	0.00459367110664716\\
368	0.00460419786932714\\
369	0.00461499228825761\\
370	0.00462605886356467\\
371	0.00463740188191018\\
372	0.00464902557505931\\
373	0.00466093443897203\\
374	0.00467313375313041\\
375	0.00468563037627639\\
376	0.00469843443609457\\
377	0.00471156587778469\\
378	0.00472504627848721\\
379	0.00473889890666229\\
380	0.00475314875338926\\
381	0.00476782220581069\\
382	0.0047829472136154\\
383	0.00479855342091387\\
384	0.00481467205339902\\
385	0.00483133570831505\\
386	0.00484857801096762\\
387	0.00486643308849613\\
388	0.004884934788385\\
389	0.0049041154916689\\
390	0.0049240048555595\\
391	0.00494462772721307\\
392	0.00496600122010846\\
393	0.00498813086022231\\
394	0.00501100534714339\\
395	0.00503459170974232\\
396	0.00505882507912041\\
397	0.00508359712951041\\
398	0.00510833073767857\\
399	0.00513289684048401\\
400	0.005157250316384\\
401	0.00518134153147043\\
402	0.005205116122047\\
403	0.00522851485600113\\
404	0.00525147361912384\\
405	0.00527392360208589\\
406	0.00529579166536875\\
407	0.00531700101044047\\
408	0.00533747237227751\\
409	0.00535712580524692\\
410	0.00537588365823375\\
411	0.00539367467105489\\
412	0.00541043974944722\\
413	0.00542614015003393\\
414	0.00544076868404582\\
415	0.00545436489319491\\
416	0.00546738929631827\\
417	0.00548011849802279\\
418	0.00549253944262353\\
419	0.00550464114701441\\
420	0.00551641524832511\\
421	0.00552785661896403\\
422	0.00553896404323548\\
423	0.00554974094209379\\
424	0.00556019610678231\\
425	0.00557034439458198\\
426	0.00558020731548305\\
427	0.00558981340369295\\
428	0.00559919822137929\\
429	0.00560840377312078\\
430	0.00561747701797148\\
431	0.00562646704137285\\
432	0.00563541158760834\\
433	0.00564432490145999\\
434	0.00565321393937883\\
435	0.00566208666998028\\
436	0.00567095202740168\\
437	0.00567981982455819\\
438	0.00568870061929596\\
439	0.00569760552686288\\
440	0.00570654597380089\\
441	0.00571553339180765\\
442	0.00572457885615251\\
443	0.00573369268309565\\
444	0.00574288401618527\\
445	0.00575216045473461\\
446	0.00576152781243046\\
447	0.00577099067349807\\
448	0.00578055331278669\\
449	0.00579022000283565\\
450	0.00579999497122763\\
451	0.00580988235693061\\
452	0.00581988616766082\\
453	0.0058300102408789\\
454	0.00584025821160249\\
455	0.00585063349069623\\
456	0.00586113925753365\\
457	0.00587177847068472\\
458	0.00588255389919582\\
459	0.00589346817452325\\
460	0.00590452385842724\\
461	0.00591572348402295\\
462	0.00592706956128874\\
463	0.00593856457489556\\
464	0.00595021098360879\\
465	0.00596201122142354\\
466	0.00597396770050443\\
467	0.00598608281587155\\
468	0.00599835895160588\\
469	0.00601079848814156\\
470	0.00602340380998847\\
471	0.00603617731301882\\
472	0.00604912141032395\\
473	0.00606223853571863\\
474	0.00607553114562047\\
475	0.00608900172073521\\
476	0.00610265276796723\\
477	0.00611648682251159\\
478	0.00613050645007231\\
479	0.00614471424914408\\
480	0.00615911285329214\\
481	0.00617370493337021\\
482	0.00618849319963286\\
483	0.00620348040372628\\
484	0.00621866934058261\\
485	0.00623406285029019\\
486	0.00624966382004183\\
487	0.00626547518621064\\
488	0.00628149993655961\\
489	0.0062977411125919\\
490	0.00631420181205383\\
491	0.00633088519160514\\
492	0.00634779446967697\\
493	0.00636493292954258\\
494	0.00638230392263143\\
495	0.0063999108721217\\
496	0.00641775727685009\\
497	0.00643584671557906\\
498	0.00645418285166201\\
499	0.00647276943814956\\
500	0.00649161032338459\\
501	0.0065107094571397\\
502	0.00653007089735626\\
503	0.00654969881755083\\
504	0.0065695975149618\\
505	0.00658977141951615\\
506	0.00661022510370496\\
507	0.00663096329346413\\
508	0.00665199088016667\\
509	0.0066733129338431\\
510	0.00669493471775744\\
511	0.00671686170447795\\
512	0.00673909959359459\\
513	0.00676165433124745\\
514	0.00678453213164819\\
515	0.00680773950079382\\
516	0.00683128326258385\\
517	0.00685517058756828\\
518	0.0068794090243893\\
519	0.00690400653407585\\
520	0.00692897152752586\\
521	0.00695431290571268\\
522	0.00698004010997143\\
523	0.00700616317875087\\
524	0.00703269280624844\\
525	0.00705964046356102\\
526	0.00708701828899484\\
527	0.00711483901941279\\
528	0.00714311610684903\\
529	0.00717186079668715\\
530	0.00720108516360538\\
531	0.00723081189144472\\
532	0.0072610644887466\\
533	0.00729186641857461\\
534	0.00732325074755106\\
535	0.00735523634621418\\
536	0.0073878402515663\\
537	0.00742107811239321\\
538	0.00745496230375221\\
539	0.0074894968293927\\
540	0.00752466168945731\\
541	0.00756038385752855\\
542	0.00759672502313432\\
543	0.00763374757337947\\
544	0.00767162181685216\\
545	0.00771041237843783\\
546	0.00775017685925997\\
547	0.00779097741675524\\
548	0.0078328729291118\\
549	0.00787593015589296\\
550	0.00792022484569437\\
551	0.00796587725701316\\
552	0.00801304218089454\\
553	0.00805960843508264\\
554	0.00810430047028959\\
555	0.00814779918778151\\
556	0.00819148112952877\\
557	0.00823562173607292\\
558	0.00828025250104444\\
559	0.00832535069346646\\
560	0.00837086503244462\\
561	0.00841672430456974\\
562	0.00846284579668691\\
563	0.00850912918133008\\
564	0.00855432580474309\\
565	0.00859878079862082\\
566	0.00864330641528182\\
567	0.00868795759871341\\
568	0.00873271555494383\\
569	0.00877755218112524\\
570	0.00882198003310067\\
571	0.00886641486864619\\
572	0.00891109494574295\\
573	0.00895600620365328\\
574	0.00900111881251587\\
575	0.00904639944292441\\
576	0.00909181177272146\\
577	0.0091373164008612\\
578	0.00918287075438226\\
579	0.00922842902321877\\
580	0.00927394213786092\\
581	0.00931935780604082\\
582	0.00936462062789891\\
583	0.00940967231401762\\
584	0.00945445203657308\\
585	0.0094988969504773\\
586	0.00954294292772307\\
587	0.00958652555102228\\
588	0.00962958140299686\\
589	0.00967204963943729\\
590	0.0097138736812812\\
591	0.00975500241768019\\
592	0.00979538909575469\\
593	0.00983498278816331\\
594	0.00987369853931868\\
595	0.00991118387968948\\
596	0.00994658651044256\\
597	0.00997788999445116\\
598	0.010000292044645\\
599	0\\
600	0\\
};
\addplot [color=mycolor6,solid,forget plot]
  table[row sep=crcr]{%
1	0.0040957575257232\\
2	0.00409577859341713\\
3	0.00409579997628435\\
4	0.004095821678038\\
5	0.00409584370241615\\
6	0.00409586605318159\\
7	0.00409588873412207\\
8	0.00409591174905029\\
9	0.00409593510180391\\
10	0.00409595879624581\\
11	0.00409598283626408\\
12	0.00409600722577218\\
13	0.00409603196870914\\
14	0.00409605706903975\\
15	0.00409608253075475\\
16	0.00409610835787113\\
17	0.00409613455443227\\
18	0.00409616112450847\\
19	0.00409618807219714\\
20	0.00409621540162322\\
21	0.00409624311693958\\
22	0.00409627122232759\\
23	0.00409629972199747\\
24	0.00409632862018893\\
25	0.00409635792117177\\
26	0.00409638762924648\\
27	0.00409641774874495\\
28	0.00409644828403125\\
29	0.00409647923950236\\
30	0.00409651061958913\\
31	0.00409654242875714\\
32	0.00409657467150769\\
33	0.00409660735237884\\
34	0.00409664047594657\\
35	0.00409667404682594\\
36	0.00409670806967228\\
37	0.00409674254918267\\
38	0.00409677749009724\\
39	0.00409681289720065\\
40	0.00409684877532375\\
41	0.00409688512934512\\
42	0.00409692196419291\\
43	0.00409695928484662\\
44	0.00409699709633897\\
45	0.00409703540375805\\
46	0.00409707421224927\\
47	0.00409711352701767\\
48	0.00409715335333018\\
49	0.00409719369651803\\
50	0.00409723456197922\\
51	0.00409727595518121\\
52	0.00409731788166357\\
53	0.00409736034704081\\
54	0.00409740335700531\\
55	0.00409744691733042\\
56	0.00409749103387354\\
57	0.00409753571257934\\
58	0.00409758095948331\\
59	0.00409762678071508\\
60	0.00409767318250206\\
61	0.00409772017117324\\
62	0.00409776775316286\\
63	0.00409781593501451\\
64	0.00409786472338505\\
65	0.00409791412504892\\
66	0.00409796414690219\\
67	0.00409801479596717\\
68	0.00409806607939678\\
69	0.00409811800447917\\
70	0.00409817057864237\\
71	0.00409822380945914\\
72	0.0040982777046519\\
73	0.00409833227209759\\
74	0.00409838751983293\\
75	0.00409844345605942\\
76	0.00409850008914866\\
77	0.00409855742764775\\
78	0.00409861548028459\\
79	0.00409867425597347\\
80	0.00409873376382058\\
81	0.00409879401312954\\
82	0.00409885501340726\\
83	0.00409891677436947\\
84	0.00409897930594667\\
85	0.00409904261828975\\
86	0.00409910672177596\\
87	0.0040991716270148\\
88	0.00409923734485389\\
89	0.00409930388638487\\
90	0.00409937126294934\\
91	0.00409943948614486\\
92	0.00409950856783085\\
93	0.00409957852013454\\
94	0.00409964935545692\\
95	0.00409972108647858\\
96	0.00409979372616584\\
97	0.00409986728777638\\
98	0.00409994178486525\\
99	0.00410001723129069\\
100	0.00410009364121998\\
101	0.0041001710291351\\
102	0.00410024940983868\\
103	0.00410032879845951\\
104	0.00410040921045839\\
105	0.00410049066163375\\
106	0.00410057316812724\\
107	0.0041006567464294\\
108	0.00410074141338527\\
109	0.00410082718619986\\
110	0.00410091408244383\\
111	0.00410100212005902\\
112	0.00410109131736399\\
113	0.00410118169305965\\
114	0.00410127326623474\\
115	0.0041013660563716\\
116	0.00410146008335181\\
117	0.00410155536746177\\
118	0.00410165192939864\\
119	0.00410174979027601\\
120	0.00410184897163003\\
121	0.00410194949542532\\
122	0.00410205138406099\\
123	0.00410215466037715\\
124	0.00410225934766104\\
125	0.00410236546965366\\
126	0.00410247305055639\\
127	0.00410258211503788\\
128	0.00410269268824098\\
129	0.00410280479578989\\
130	0.00410291846379755\\
131	0.00410303371887327\\
132	0.0041031505881303\\
133	0.00410326909919388\\
134	0.00410338928020944\\
135	0.00410351115985074\\
136	0.00410363476732871\\
137	0.00410376013240006\\
138	0.00410388728537628\\
139	0.00410401625713288\\
140	0.00410414707911871\\
141	0.0041042797833657\\
142	0.00410441440249856\\
143	0.00410455096974481\\
144	0.00410468951894524\\
145	0.00410483008456399\\
146	0.00410497270169956\\
147	0.00410511740609558\\
148	0.00410526423415193\\
149	0.00410541322293573\\
150	0.00410556441019279\\
151	0.00410571783435943\\
152	0.00410587353457443\\
153	0.0041060315506911\\
154	0.00410619192328988\\
155	0.00410635469369109\\
156	0.0041065199039677\\
157	0.00410668759695891\\
158	0.00410685781628351\\
159	0.004107030606354\\
160	0.00410720601239067\\
161	0.00410738408043628\\
162	0.00410756485737098\\
163	0.00410774839092757\\
164	0.00410793472970733\\
165	0.00410812392319595\\
166	0.00410831602178023\\
167	0.00410851107676491\\
168	0.00410870914039013\\
169	0.00410891026584936\\
170	0.00410911450730776\\
171	0.00410932191992095\\
172	0.00410953255985457\\
173	0.00410974648430407\\
174	0.00410996375151537\\
175	0.00411018442080578\\
176	0.0041104085525858\\
177	0.0041106362083814\\
178	0.00411086745085676\\
179	0.00411110234383806\\
180	0.00411134095233764\\
181	0.00411158334257883\\
182	0.00411182958202163\\
183	0.004112079739389\\
184	0.00411233388469389\\
185	0.004112592089267\\
186	0.00411285442578532\\
187	0.00411312096830136\\
188	0.00411339179227324\\
189	0.00411366697459539\\
190	0.00411394659363018\\
191	0.00411423072924033\\
192	0.0041145194628219\\
193	0.00411481287733823\\
194	0.00411511105735464\\
195	0.00411541408907388\\
196	0.00411572206037222\\
197	0.0041160350608364\\
198	0.00411635318180128\\
199	0.00411667651638819\\
200	0.004117005159544\\
201	0.00411733920808075\\
202	0.00411767876071612\\
203	0.00411802391811416\\
204	0.00411837478292706\\
205	0.00411873145983705\\
206	0.00411909405559909\\
207	0.00411946267908387\\
208	0.00411983744132134\\
209	0.00412021845554453\\
210	0.00412060583723384\\
211	0.00412099970416156\\
212	0.00412140017643675\\
213	0.00412180737655021\\
214	0.00412222142941977\\
215	0.00412264246243567\\
216	0.00412307060550606\\
217	0.00412350599110271\\
218	0.0041239487543066\\
219	0.00412439903285371\\
220	0.00412485696718078\\
221	0.00412532270047123\\
222	0.00412579637870083\\
223	0.00412627815068381\\
224	0.00412676816811873\\
225	0.00412726658563466\\
226	0.00412777356083737\\
227	0.00412828925435592\\
228	0.0041288138298894\\
229	0.00412934745425412\\
230	0.00412989029743132\\
231	0.00413044253261524\\
232	0.0041310043362622\\
233	0.00413157588814017\\
234	0.00413215737137989\\
235	0.00413274897252666\\
236	0.00413335088159374\\
237	0.00413396329211722\\
238	0.00413458640121268\\
239	0.00413522040963379\\
240	0.00413586552183297\\
241	0.00413652194602454\\
242	0.00413718989425032\\
243	0.00413786958244821\\
244	0.00413856123052352\\
245	0.00413926506242377\\
246	0.00413998130621683\\
247	0.00414071019417249\\
248	0.00414145196284843\\
249	0.00414220685317966\\
250	0.00414297511057278\\
251	0.00414375698500402\\
252	0.00414455273112235\\
253	0.00414536260835737\\
254	0.00414618688103258\\
255	0.00414702581848413\\
256	0.00414787969518288\\
257	0.00414874879085554\\
258	0.00414963339061607\\
259	0.00415053378510179\\
260	0.00415145027061411\\
261	0.00415238314926421\\
262	0.00415333272912317\\
263	0.00415429932437663\\
264	0.00415528325548441\\
265	0.00415628484934529\\
266	0.00415730443946647\\
267	0.00415834236613792\\
268	0.00415939897661162\\
269	0.00416047462528526\\
270	0.00416156967389093\\
271	0.00416268449168826\\
272	0.00416381945566199\\
273	0.00416497495072398\\
274	0.00416615136991966\\
275	0.00416734911463855\\
276	0.00416856859482894\\
277	0.00416981022921671\\
278	0.00417107444552737\\
279	0.00417236168071149\\
280	0.00417367238117411\\
281	0.00417500700300752\\
282	0.00417636601222807\\
283	0.00417774988501727\\
284	0.00417915910796586\\
285	0.00418059417832096\\
286	0.00418205560423672\\
287	0.00418354390502853\\
288	0.00418505961143148\\
289	0.00418660326586487\\
290	0.00418817542270023\\
291	0.00418977664853424\\
292	0.00419140752246745\\
293	0.00419306863638921\\
294	0.00419476059526954\\
295	0.00419648401745842\\
296	0.00419823953499375\\
297	0.00420002779391894\\
298	0.0042018494546114\\
299	0.0042037051921232\\
300	0.00420559569653311\\
301	0.00420752167330453\\
302	0.00420948384366157\\
303	0.00421148294498171\\
304	0.00421351973120494\\
305	0.00421559497326092\\
306	0.00421770945951509\\
307	0.0042198639962358\\
308	0.00422205940808414\\
309	0.0042242965386328\\
310	0.00422657625092935\\
311	0.00422889942811147\\
312	0.00423126697407714\\
313	0.00423367981422548\\
314	0.00423613889609186\\
315	0.00423864519013073\\
316	0.00424119969056351\\
317	0.00424380341627011\\
318	0.00424645741171143\\
319	0.0042491627478955\\
320	0.00425192052339256\\
321	0.00425473186539667\\
322	0.00425759793083365\\
323	0.0042605199075153\\
324	0.00426349901534109\\
325	0.00426653650757716\\
326	0.00426963367219471\\
327	0.0042727918332629\\
328	0.0042760123524123\\
329	0.00427929663038039\\
330	0.00428264610864861\\
331	0.00428606227112797\\
332	0.0042895466458308\\
333	0.00429310080669641\\
334	0.00429672637548867\\
335	0.00430042502375289\\
336	0.00430419847480939\\
337	0.00430804850578537\\
338	0.00431197694982626\\
339	0.0043159856984577\\
340	0.00432007670380599\\
341	0.0043242519807938\\
342	0.00432851360926847\\
343	0.00433286373601202\\
344	0.00433730457657591\\
345	0.00434183841689476\\
346	0.00434646761468339\\
347	0.00435119460075521\\
348	0.00435602188021876\\
349	0.00436095203435385\\
350	0.00436598771960515\\
351	0.00437113167052693\\
352	0.00437638670281178\\
353	0.00438175571708934\\
354	0.00438724170317217\\
355	0.00439284774500271\\
356	0.00439857702484328\\
357	0.00440443282788267\\
358	0.00441041854789761\\
359	0.00441653769327272\\
360	0.00442279389322759\\
361	0.0044291909064023\\
362	0.00443573263321883\\
363	0.00444242312760875\\
364	0.0044492666106289\\
365	0.0044562674862465\\
366	0.00446343035955969\\
367	0.00447076005766673\\
368	0.0044782616533322\\
369	0.0044859404917794\\
370	0.00449380222149397\\
371	0.00450185282282893\\
372	0.00451009863813425\\
373	0.00451854639854589\\
374	0.00452720324248943\\
375	0.00453607672148651\\
376	0.00454517480924487\\
377	0.00455450571499868\\
378	0.00456407786561144\\
379	0.00457389987876267\\
380	0.00458398051178957\\
381	0.00459432865007819\\
382	0.00460495328678225\\
383	0.00461586348087892\\
384	0.00462706830927145\\
385	0.0046385768137451\\
386	0.00465039794434209\\
387	0.0046625405016459\\
388	0.00467501308273088\\
389	0.00468782406537107\\
390	0.0047009815918066\\
391	0.00471449359032724\\
392	0.00472836787429916\\
393	0.00474261236282055\\
394	0.00475723558076776\\
395	0.00477224709037745\\
396	0.00478765835628073\\
397	0.00480348381186787\\
398	0.00481974914418876\\
399	0.00483648354993576\\
400	0.00485371815958732\\
401	0.00487148598747839\\
402	0.00488982180268676\\
403	0.00490876188850308\\
404	0.00492834364062747\\
405	0.00494860488779465\\
406	0.00496958322414589\\
407	0.00499131486987985\\
408	0.00501383256332424\\
409	0.00503716515917807\\
410	0.00506133397257322\\
411	0.00508634837559588\\
412	0.00511220029646462\\
413	0.00513884973395452\\
414	0.00516621948017444\\
415	0.00519418140474314\\
416	0.00522219113231119\\
417	0.00524989647293191\\
418	0.00527723538116368\\
419	0.00530414018582434\\
420	0.00533053757369377\\
421	0.00535634855725208\\
422	0.00538148875596453\\
423	0.00540586895196347\\
424	0.00542939656222645\\
425	0.00545197769857029\\
426	0.00547352034188589\\
427	0.00549393886548258\\
428	0.00551316036094833\\
429	0.00553113346425168\\
430	0.00554784052366552\\
431	0.00556331422444391\\
432	0.00557791451485326\\
433	0.00559216759613327\\
434	0.00560606131561664\\
435	0.00561958666055824\\
436	0.00563273850432294\\
437	0.0056455164115504\\
438	0.00565792548096981\\
439	0.00566997719468297\\
440	0.00568169021734538\\
441	0.00569309105650054\\
442	0.00570421445207726\\
443	0.00571510330308959\\
444	0.00572580785695178\\
445	0.00573638377343506\\
446	0.00574688852537332\\
447	0.00575736051133426\\
448	0.00576781473905054\\
449	0.00577826051033063\\
450	0.00578870823467299\\
451	0.00579916932351866\\
452	0.00580965602840855\\
453	0.00582018121534943\\
454	0.00583075806991991\\
455	0.00584139973202969\\
456	0.00585211886682199\\
457	0.00586292719038153\\
458	0.00587383498817624\\
459	0.0058848506935462\\
460	0.00589598063712908\\
461	0.00590722996532073\\
462	0.00591860360210382\\
463	0.00593010643016358\\
464	0.00594174324038562\\
465	0.0059535186821428\\
466	0.0059654372173562\\
467	0.00597750308200778\\
468	0.00598972025937734\\
469	0.00600209246959655\\
470	0.00601462317989138\\
471	0.00602731563870397\\
472	0.00604017293410104\\
473	0.00605319807152112\\
474	0.00606639402431047\\
475	0.00607976374298815\\
476	0.00609331015374547\\
477	0.00610703615890761\\
478	0.00612094463951541\\
479	0.00613503846005787\\
480	0.00614932047521305\\
481	0.00616379353823352\\
482	0.00617846051035483\\
483	0.00619332427033949\\
484	0.00620838772305117\\
485	0.00622365380589142\\
486	0.00623912549261596\\
487	0.00625480579621778\\
488	0.00627069777218987\\
489	0.00628680452212941\\
490	0.00630312919763236\\
491	0.00631967500441896\\
492	0.00633644520662642\\
493	0.00635344313121247\\
494	0.00637067217243157\\
495	0.00638813579637969\\
496	0.00640583754565251\\
497	0.00642378104422246\\
498	0.00644197000268395\\
499	0.00646040822396084\\
500	0.00647909960951374\\
501	0.006498048166091\\
502	0.00651725801307465\\
503	0.00653673339048063\\
504	0.00655647866768176\\
505	0.0065764983529322\\
506	0.00659679710378154\\
507	0.00661737973847604\\
508	0.00663825124845157\\
509	0.00665941681202778\\
510	0.00668088180941628\\
511	0.00670265183916187\\
512	0.00672473273614287\\
513	0.00674713059126367\\
514	0.0067698517729772\\
515	0.00679290295077906\\
516	0.00681629112081456\\
517	0.00684002363373771\\
518	0.00686410822495721\\
519	0.00688855304739165\\
520	0.00691336670683068\\
521	0.00693855829996578\\
522	0.00696413745491618\\
523	0.00699011437411914\\
524	0.00701649987951138\\
525	0.00704330545914029\\
526	0.00707054332066159\\
527	0.00709822644926038\\
528	0.00712636866359132\\
529	0.0071549847035114\\
530	0.00718409021224089\\
531	0.00721370155228759\\
532	0.00724383586893235\\
533	0.00727451118922645\\
534	0.00730573830354627\\
535	0.00733754289382939\\
536	0.00736995197449274\\
537	0.00740299261041146\\
538	0.00743669497365421\\
539	0.00747108869579545\\
540	0.00750619254437731\\
541	0.00754202239752112\\
542	0.00757858696594791\\
543	0.00761588370558736\\
544	0.00765381442335629\\
545	0.00769240422359834\\
546	0.00773170952485417\\
547	0.00777184860363677\\
548	0.0078129512996561\\
549	0.0078550743470806\\
550	0.00789829083241618\\
551	0.00794266950789453\\
552	0.00798828443682109\\
553	0.00803524805689781\\
554	0.00808370675822525\\
555	0.00813245686176951\\
556	0.00817966759564629\\
557	0.00822469162758563\\
558	0.00826987878715762\\
559	0.00831532757736804\\
560	0.00836118747187362\\
561	0.00840745876100177\\
562	0.00845408897122118\\
563	0.00850099755665664\\
564	0.00854809439039289\\
565	0.00859487531085683\\
566	0.00864051776728843\\
567	0.00868567121298489\\
568	0.00873085196361323\\
569	0.00877609289333616\\
570	0.00882137592721109\\
571	0.00886636048870486\\
572	0.00891108851230718\\
573	0.00895600554864665\\
574	0.00900111866282342\\
575	0.00904639939181835\\
576	0.00909181174974404\\
577	0.00913731638890149\\
578	0.00918287074791749\\
579	0.00922842901963624\\
580	0.00927394213588079\\
581	0.00931935780498493\\
582	0.00936462062738148\\
583	0.00940967231379966\\
584	0.00945445203650304\\
585	0.00949889695046246\\
586	0.00954294292772186\\
587	0.00958652555102227\\
588	0.00962958140299686\\
589	0.00967204963943729\\
590	0.0097138736812812\\
591	0.00975500241768019\\
592	0.00979538909575469\\
593	0.00983498278816331\\
594	0.00987369853931868\\
595	0.00991118387968948\\
596	0.00994658651044256\\
597	0.00997788999445116\\
598	0.010000292044645\\
599	0\\
600	0\\
};
\addplot [color=mycolor7,solid,forget plot]
  table[row sep=crcr]{%
1	0.00409529888158573\\
2	0.00409531335916626\\
3	0.00409532803996346\\
4	0.00409534292632739\\
5	0.00409535802063159\\
6	0.00409537332527361\\
7	0.00409538884267553\\
8	0.0040954045752844\\
9	0.00409542052557287\\
10	0.00409543669603968\\
11	0.00409545308921039\\
12	0.00409546970763792\\
13	0.00409548655390317\\
14	0.0040955036306158\\
15	0.00409552094041491\\
16	0.00409553848596972\\
17	0.00409555626998042\\
18	0.00409557429517898\\
19	0.00409559256432993\\
20	0.00409561108023117\\
21	0.00409562984571509\\
22	0.00409564886364924\\
23	0.00409566813693751\\
24	0.00409568766852107\\
25	0.00409570746137935\\
26	0.00409572751853122\\
27	0.00409574784303607\\
28	0.00409576843799497\\
29	0.00409578930655189\\
30	0.00409581045189494\\
31	0.00409583187725764\\
32	0.00409585358592025\\
33	0.00409587558121114\\
34	0.00409589786650818\\
35	0.0040959204452403\\
36	0.00409594332088887\\
37	0.00409596649698929\\
38	0.00409598997713262\\
39	0.00409601376496716\\
40	0.00409603786420016\\
41	0.00409606227859958\\
42	0.00409608701199582\\
43	0.00409611206828361\\
44	0.00409613745142382\\
45	0.00409616316544542\\
46	0.00409618921444747\\
47	0.00409621560260106\\
48	0.00409624233415152\\
49	0.00409626941342034\\
50	0.00409629684480756\\
51	0.00409632463279382\\
52	0.00409635278194263\\
53	0.00409638129690279\\
54	0.00409641018241064\\
55	0.00409643944329256\\
56	0.00409646908446722\\
57	0.00409649911094831\\
58	0.004096529527847\\
59	0.00409656034037446\\
60	0.00409659155384456\\
61	0.00409662317367653\\
62	0.00409665520539764\\
63	0.00409668765464605\\
64	0.0040967205271735\\
65	0.00409675382884819\\
66	0.0040967875656577\\
67	0.00409682174371186\\
68	0.00409685636924571\\
69	0.00409689144862251\\
70	0.00409692698833673\\
71	0.00409696299501717\\
72	0.00409699947543001\\
73	0.0040970364364819\\
74	0.00409707388522319\\
75	0.00409711182885107\\
76	0.00409715027471284\\
77	0.00409718923030907\\
78	0.00409722870329695\\
79	0.00409726870149347\\
80	0.00409730923287888\\
81	0.00409735030559997\\
82	0.00409739192797341\\
83	0.00409743410848921\\
84	0.00409747685581403\\
85	0.00409752017879478\\
86	0.00409756408646196\\
87	0.00409760858803316\\
88	0.00409765369291657\\
89	0.0040976994107146\\
90	0.00409774575122724\\
91	0.00409779272445585\\
92	0.00409784034060656\\
93	0.00409788861009412\\
94	0.00409793754354529\\
95	0.00409798715180282\\
96	0.00409803744592888\\
97	0.00409808843720899\\
98	0.00409814013715571\\
99	0.00409819255751257\\
100	0.00409824571025767\\
101	0.00409829960760789\\
102	0.00409835426202258\\
103	0.00409840968620775\\
104	0.00409846589311991\\
105	0.0040985228959703\\
106	0.00409858070822895\\
107	0.00409863934362897\\
108	0.00409869881617072\\
109	0.00409875914012631\\
110	0.00409882033004383\\
111	0.004098882400752\\
112	0.00409894536736461\\
113	0.00409900924528532\\
114	0.00409907405021224\\
115	0.00409913979814298\\
116	0.00409920650537936\\
117	0.00409927418853265\\
118	0.00409934286452856\\
119	0.00409941255061264\\
120	0.00409948326435549\\
121	0.00409955502365831\\
122	0.00409962784675861\\
123	0.00409970175223569\\
124	0.00409977675901671\\
125	0.00409985288638252\\
126	0.00409993015397387\\
127	0.00410000858179754\\
128	0.00410008819023275\\
129	0.00410016900003767\\
130	0.00410025103235615\\
131	0.0041003343087243\\
132	0.00410041885107761\\
133	0.00410050468175801\\
134	0.00410059182352093\\
135	0.00410068029954293\\
136	0.00410077013342902\\
137	0.00410086134922046\\
138	0.00410095397140261\\
139	0.00410104802491292\\
140	0.00410114353514911\\
141	0.00410124052797757\\
142	0.00410133902974192\\
143	0.00410143906727174\\
144	0.00410154066789134\\
145	0.00410164385942924\\
146	0.00410174867022732\\
147	0.00410185512915045\\
148	0.00410196326559641\\
149	0.00410207310950645\\
150	0.0041021846913758\\
151	0.00410229804226414\\
152	0.00410241319380625\\
153	0.00410253017822295\\
154	0.00410264902833234\\
155	0.00410276977756104\\
156	0.00410289245995604\\
157	0.00410301711019652\\
158	0.00410314376360608\\
159	0.00410327245616516\\
160	0.00410340322452384\\
161	0.00410353610601481\\
162	0.00410367113866676\\
163	0.00410380836121798\\
164	0.00410394781313014\\
165	0.00410408953460283\\
166	0.00410423356658778\\
167	0.00410437995080393\\
168	0.00410452872975262\\
169	0.00410467994673294\\
170	0.00410483364585779\\
171	0.00410498987207004\\
172	0.00410514867115905\\
173	0.0041053100897776\\
174	0.00410547417545914\\
175	0.00410564097663547\\
176	0.00410581054265464\\
177	0.00410598292379933\\
178	0.00410615817130577\\
179	0.00410633633738256\\
180	0.0041065174752303\\
181	0.0041067016390614\\
182	0.00410688888412046\\
183	0.0041070792667047\\
184	0.00410727284418501\\
185	0.0041074696750275\\
186	0.00410766981881499\\
187	0.00410787333626945\\
188	0.00410808028927429\\
189	0.00410829074089746\\
190	0.00410850475541469\\
191	0.0041087223983331\\
192	0.00410894373641531\\
193	0.00410916883770393\\
194	0.00410939777154627\\
195	0.00410963060861962\\
196	0.00410986742095669\\
197	0.00411010828197159\\
198	0.00411035326648619\\
199	0.00411060245075664\\
200	0.00411085591250051\\
201	0.00411111373092417\\
202	0.0041113759867505\\
203	0.00411164276224718\\
204	0.00411191414125503\\
205	0.004112190209217\\
206	0.00411247105320746\\
207	0.00411275676196188\\
208	0.00411304742590691\\
209	0.00411334313719079\\
210	0.00411364398971436\\
211	0.00411395007916245\\
212	0.00411426150303559\\
213	0.00411457836068234\\
214	0.00411490075333212\\
215	0.00411522878412847\\
216	0.00411556255816302\\
217	0.00411590218250975\\
218	0.0041162477662602\\
219	0.00411659942055925\\
220	0.00411695725864131\\
221	0.00411732139586763\\
222	0.00411769194976415\\
223	0.00411806904006028\\
224	0.00411845278872851\\
225	0.00411884332002493\\
226	0.00411924076053081\\
227	0.00411964523919514\\
228	0.00412005688737832\\
229	0.00412047583889703\\
230	0.00412090223007023\\
231	0.00412133619976646\\
232	0.00412177788945259\\
233	0.00412222744324388\\
234	0.00412268500795542\\
235	0.0041231507331553\\
236	0.00412362477121914\\
237	0.00412410727738639\\
238	0.00412459840981831\\
239	0.00412509832965749\\
240	0.00412560720108942\\
241	0.00412612519140568\\
242	0.004126652471069\\
243	0.00412718921378027\\
244	0.00412773559654744\\
245	0.00412829179975632\\
246	0.00412885800724327\\
247	0.0041294344063702\\
248	0.00413002118810105\\
249	0.00413061854708067\\
250	0.00413122668171561\\
251	0.00413184579425673\\
252	0.00413247609088421\\
253	0.00413311778179431\\
254	0.00413377108128832\\
255	0.00413443620786323\\
256	0.00413511338430401\\
257	0.00413580283777837\\
258	0.00413650479993368\\
259	0.00413721950699552\\
260	0.00413794719986878\\
261	0.00413868812424015\\
262	0.0041394425306831\\
263	0.00414021067476449\\
264	0.00414099281715355\\
265	0.00414178922373251\\
266	0.00414260016570967\\
267	0.00414342591973415\\
268	0.00414426676801282\\
269	0.00414512299842949\\
270	0.00414599490466596\\
271	0.00414688278632531\\
272	0.00414778694905744\\
273	0.00414870770468688\\
274	0.00414964537134285\\
275	0.00415060027359177\\
276	0.00415157274257249\\
277	0.00415256311613383\\
278	0.00415357173897492\\
279	0.00415459896278864\\
280	0.0041556451464081\\
281	0.00415671065595639\\
282	0.00415779586500009\\
283	0.00415890115470628\\
284	0.0041600269140035\\
285	0.00416117353974716\\
286	0.00416234143688919\\
287	0.00416353101865285\\
288	0.00416474270671286\\
289	0.00416597693138086\\
290	0.00416723413179686\\
291	0.00416851475612708\\
292	0.00416981926176857\\
293	0.00417114811556082\\
294	0.00417250179400487\\
295	0.00417388078349076\\
296	0.00417528558053294\\
297	0.0041767166920149\\
298	0.00417817463544284\\
299	0.00417965993920883\\
300	0.00418117314286336\\
301	0.00418271479739847\\
302	0.00418428546554185\\
303	0.00418588572206159\\
304	0.00418751615408255\\
305	0.00418917736141384\\
306	0.00419086995688843\\
307	0.00419259456671431\\
308	0.00419435183083869\\
309	0.00419614240332592\\
310	0.00419796695274931\\
311	0.00419982616259705\\
312	0.0042017207316897\\
313	0.00420365137459837\\
314	0.00420561882208432\\
315	0.00420762382155358\\
316	0.0042096671375247\\
317	0.00421174955210941\\
318	0.00421387186550712\\
319	0.00421603489651504\\
320	0.00421823948305327\\
321	0.00422048648270602\\
322	0.00422277677327956\\
323	0.00422511125337804\\
324	0.00422749084300021\\
325	0.00422991648415631\\
326	0.00423238914150635\\
327	0.00423490980302193\\
328	0.00423747948067404\\
329	0.00424009921114819\\
330	0.00424277005658351\\
331	0.00424549310533493\\
332	0.00424826947277318\\
333	0.00425110030211756\\
334	0.00425398676530208\\
335	0.0042569300638763\\
336	0.00425993142994667\\
337	0.00426299212717308\\
338	0.00426611345181838\\
339	0.00426929673383262\\
340	0.00427254333798882\\
341	0.00427585466507403\\
342	0.00427923215314009\\
343	0.00428267727882064\\
344	0.00428619155872267\\
345	0.00428977655090712\\
346	0.00429343385647985\\
347	0.00429716512129608\\
348	0.00430097203779398\\
349	0.00430485634670329\\
350	0.0043088198390626\\
351	0.00431286435833666\\
352	0.00431699180268599\\
353	0.00432120412736427\\
354	0.00432550334724409\\
355	0.00432989153936209\\
356	0.00433437084559166\\
357	0.00433894347548184\\
358	0.00434361170919735\\
359	0.00434837790057111\\
360	0.0043532444804495\\
361	0.00435821396036348\\
362	0.00436328893615132\\
363	0.00436847209172436\\
364	0.00437376620296244\\
365	0.00437917414171798\\
366	0.00438469887990066\\
367	0.00439034349361296\\
368	0.00439611116732586\\
369	0.00440200519807622\\
370	0.00440802899928185\\
371	0.00441418610453587\\
372	0.00442048017121563\\
373	0.00442691498401626\\
374	0.00443349445901006\\
375	0.00444022264981431\\
376	0.00444710374764763\\
377	0.00445414208774742\\
378	0.00446134215594629\\
379	0.00446870859513666\\
380	0.00447624621648438\\
381	0.00448396001123058\\
382	0.00449185516245293\\
383	0.00449993705836801\\
384	0.00450821130742561\\
385	0.00451668375545995\\
386	0.00452536050518122\\
387	0.00453424793849747\\
388	0.00454335274328614\\
389	0.00455268194107426\\
390	0.00456224291751835\\
391	0.00457204345661088\\
392	0.00458209177891305\\
393	0.00459239658516182\\
394	0.00460296707186981\\
395	0.00461381294564437\\
396	0.00462494440577857\\
397	0.00463637213268345\\
398	0.00464810711780087\\
399	0.00466016060587126\\
400	0.00467254406173833\\
401	0.00468526913030364\\
402	0.00469834758954639\\
403	0.00471179129669854\\
404	0.00472561212869725\\
405	0.004739821942918\\
406	0.0047544325287009\\
407	0.00476945555926563\\
408	0.00478490272671644\\
409	0.0048007856878958\\
410	0.00481711609581658\\
411	0.00483390569989827\\
412	0.00485116620336472\\
413	0.0048689103631968\\
414	0.00488715290178423\\
415	0.0049059118337603\\
416	0.00492521605889702\\
417	0.00494510181564877\\
418	0.0049656072975093\\
419	0.0049867723737793\\
420	0.00500863798590982\\
421	0.00503124585634181\\
422	0.0050546384992814\\
423	0.00507885959850083\\
424	0.00510395023558282\\
425	0.00512994484037258\\
426	0.00515686433680294\\
427	0.00518471516750045\\
428	0.00521348289821372\\
429	0.00524312368771077\\
430	0.00527355295684454\\
431	0.0053046303856445\\
432	0.00533589051850966\\
433	0.00536671291745671\\
434	0.00539701960726992\\
435	0.00542672580803933\\
436	0.00545573994399862\\
437	0.00548396439442113\\
438	0.00551129656975341\\
439	0.00553763057692947\\
440	0.00556285975532503\\
441	0.00558688044713349\\
442	0.00560959744130947\\
443	0.00563093167375537\\
444	0.00565083095726243\\
445	0.00566928477518812\\
446	0.00568634437505773\\
447	0.00570257952405059\\
448	0.00571841830284484\\
449	0.00573385115307405\\
450	0.00574887307694424\\
451	0.00576348456606036\\
452	0.0057776925779305\\
453	0.00579151151564601\\
454	0.00580496413961544\\
455	0.00581808230232539\\
456	0.00583090734542928\\
457	0.00584348992964762\\
458	0.00585588896168109\\
459	0.00586816914156737\\
460	0.00588039646581119\\
461	0.00589260790387998\\
462	0.00590481832838029\\
463	0.00591703939841473\\
464	0.0059292839423036\\
465	0.0059415657811149\\
466	0.00595389947914521\\
467	0.00596630001433183\\
468	0.00597878236640431\\
469	0.00599136102925191\\
470	0.00600404946788492\\
471	0.00601685956198455\\
472	0.00602980111100503\\
473	0.00604288152490197\\
474	0.00605610677287527\\
475	0.00606948250346558\\
476	0.00608301430127271\\
477	0.00609670762906814\\
478	0.00611056777217856\\
479	0.00612459978900502\\
480	0.00613880847232134\\
481	0.00615319832658378\\
482	0.00616777356663\\
483	0.0061825381424517\\
484	0.00619749579257747\\
485	0.00621265012407515\\
486	0.00622800469780967\\
487	0.00624356306178102\\
488	0.00625932874999612\\
489	0.00627530528369641\\
490	0.00629149617516352\\
491	0.00630790493418527\\
492	0.00632453507706829\\
493	0.00634139013782821\\
494	0.00635847368088586\\
495	0.00637578931427811\\
496	0.00639334070211891\\
497	0.00641113157493753\\
498	0.00642916573711289\\
499	0.00644744707338765\\
500	0.00646597955615283\\
501	0.00648476725348922\\
502	0.00650381433793958\\
503	0.00652312509597778\\
504	0.00654270393814206\\
505	0.00656255540981227\\
506	0.00658268420263934\\
507	0.00660309516668271\\
508	0.00662379332337688\\
509	0.00664478387952437\\
510	0.00666607224253939\\
511	0.00668766403709905\\
512	0.00670956512331254\\
513	0.00673178161652677\\
514	0.00675431990889423\\
515	0.00677718669283627\\
516	0.00680038898653935\\
517	0.00682393416162442\\
518	0.00684782997312543\\
519	0.0068720845919022\\
520	0.00689670663959\\
521	0.00692170522615219\\
522	0.00694708999005904\\
523	0.00697287114105492\\
524	0.00699905950538988\\
525	0.00702566657328029\\
526	0.00705270454805575\\
527	0.00708018639631153\\
528	0.00710812589829871\\
529	0.00713653769710684\\
530	0.00716543734818967\\
531	0.00719484137021395\\
532	0.007224767287208\\
533	0.00725523365595185\\
534	0.00728626016991656\\
535	0.00731786730671243\\
536	0.0073500762855338\\
537	0.00738290903487387\\
538	0.00741638488680999\\
539	0.00745052410967822\\
540	0.00748535622462389\\
541	0.00752091025166277\\
542	0.00755721342335093\\
543	0.00759429613161743\\
544	0.00763218088623283\\
545	0.00767086853324643\\
546	0.00771034715895772\\
547	0.00775055806174717\\
548	0.00779145697827102\\
549	0.00783309498128678\\
550	0.00787553646602525\\
551	0.00791895405846008\\
552	0.00796342795168289\\
553	0.00800902558206601\\
554	0.00805583223394632\\
555	0.00810394296610672\\
556	0.00815348917904342\\
557	0.00820452788108392\\
558	0.00825427676315796\\
559	0.00830217862287868\\
560	0.00834893731572075\\
561	0.00839579436605613\\
562	0.00844291676212203\\
563	0.00849036691366464\\
564	0.00853811179111934\\
565	0.00858607445145949\\
566	0.00863415405798469\\
567	0.00868157869679225\\
568	0.00872782524480766\\
569	0.00877366426417583\\
570	0.00881945432030142\\
571	0.00886522995351837\\
572	0.00891084129040892\\
573	0.00895597597182961\\
574	0.00900111411185122\\
575	0.00904639837989421\\
576	0.00909181141669918\\
577	0.00913731624296646\\
578	0.00918287067265086\\
579	0.00922842897919994\\
580	0.00927394211349327\\
581	0.00931935779255734\\
582	0.0093646206206887\\
583	0.00940967231047064\\
584	0.00945445203507202\\
585	0.00949889694999284\\
586	0.00954294292762137\\
587	0.00958652555101391\\
588	0.00962958140299686\\
589	0.00967204963943729\\
590	0.0097138736812812\\
591	0.00975500241768019\\
592	0.00979538909575469\\
593	0.00983498278816331\\
594	0.00987369853931868\\
595	0.00991118387968948\\
596	0.00994658651044256\\
597	0.00997788999445116\\
598	0.010000292044645\\
599	0\\
600	0\\
};
\addplot [color=mycolor8,solid,forget plot]
  table[row sep=crcr]{%
1	0.0040944850682663\\
2	0.00409449503326179\\
3	0.00409450513478612\\
4	0.00409451537454594\\
5	0.0040945257542724\\
6	0.00409453627572179\\
7	0.00409454694067606\\
8	0.00409455775094352\\
9	0.00409456870835943\\
10	0.00409457981478679\\
11	0.00409459107211679\\
12	0.00409460248226963\\
13	0.00409461404719534\\
14	0.0040946257688743\\
15	0.00409463764931816\\
16	0.00409464969057058\\
17	0.00409466189470794\\
18	0.00409467426384026\\
19	0.00409468680011199\\
20	0.00409469950570288\\
21	0.00409471238282884\\
22	0.00409472543374286\\
23	0.00409473866073591\\
24	0.00409475206613794\\
25	0.00409476565231875\\
26	0.00409477942168906\\
27	0.00409479337670149\\
28	0.00409480751985164\\
29	0.00409482185367903\\
30	0.00409483638076831\\
31	0.00409485110375029\\
32	0.00409486602530311\\
33	0.00409488114815337\\
34	0.00409489647507731\\
35	0.00409491200890193\\
36	0.00409492775250638\\
37	0.00409494370882305\\
38	0.00409495988083891\\
39	0.00409497627159686\\
40	0.0040949928841969\\
41	0.00409500972179762\\
42	0.00409502678761747\\
43	0.00409504408493619\\
44	0.00409506161709623\\
45	0.0040950793875042\\
46	0.00409509739963229\\
47	0.0040951156570198\\
48	0.00409513416327465\\
49	0.00409515292207488\\
50	0.00409517193717029\\
51	0.00409519121238387\\
52	0.00409521075161362\\
53	0.00409523055883402\\
54	0.00409525063809774\\
55	0.00409527099353731\\
56	0.00409529162936692\\
57	0.00409531254988402\\
58	0.00409533375947105\\
59	0.00409535526259738\\
60	0.00409537706382103\\
61	0.0040953991677904\\
62	0.00409542157924627\\
63	0.00409544430302357\\
64	0.00409546734405334\\
65	0.00409549070736458\\
66	0.0040955143980863\\
67	0.00409553842144936\\
68	0.00409556278278859\\
69	0.00409558748754472\\
70	0.0040956125412665\\
71	0.00409563794961268\\
72	0.00409566371835417\\
73	0.00409568985337613\\
74	0.0040957163606802\\
75	0.00409574324638658\\
76	0.00409577051673624\\
77	0.0040957981780932\\
78	0.00409582623694681\\
79	0.00409585469991399\\
80	0.00409588357374152\\
81	0.0040959128653085\\
82	0.00409594258162867\\
83	0.00409597272985288\\
84	0.00409600331727147\\
85	0.00409603435131681\\
86	0.00409606583956581\\
87	0.00409609778974255\\
88	0.00409613020972078\\
89	0.00409616310752666\\
90	0.00409619649134142\\
91	0.0040962303695041\\
92	0.00409626475051434\\
93	0.00409629964303513\\
94	0.00409633505589592\\
95	0.00409637099809522\\
96	0.00409640747880389\\
97	0.004096444507368\\
98	0.00409648209331201\\
99	0.00409652024634192\\
100	0.00409655897634848\\
101	0.00409659829341052\\
102	0.00409663820779827\\
103	0.00409667872997674\\
104	0.00409671987060928\\
105	0.00409676164056121\\
106	0.00409680405090336\\
107	0.00409684711291586\\
108	0.00409689083809193\\
109	0.00409693523814174\\
110	0.00409698032499651\\
111	0.00409702611081243\\
112	0.00409707260797493\\
113	0.00409711982910283\\
114	0.0040971677870529\\
115	0.00409721649492399\\
116	0.00409726596606184\\
117	0.00409731621406359\\
118	0.00409736725278263\\
119	0.00409741909633328\\
120	0.004097471759096\\
121	0.00409752525572223\\
122	0.00409757960113968\\
123	0.00409763481055758\\
124	0.00409769089947206\\
125	0.0040977478836717\\
126	0.00409780577924305\\
127	0.0040978646025764\\
128	0.00409792437037164\\
129	0.00409798509964415\\
130	0.00409804680773077\\
131	0.00409810951229613\\
132	0.00409817323133875\\
133	0.00409823798319739\\
134	0.0040983037865577\\
135	0.00409837066045853\\
136	0.00409843862429871\\
137	0.00409850769784378\\
138	0.00409857790123282\\
139	0.00409864925498518\\
140	0.00409872178000752\\
141	0.00409879549760041\\
142	0.00409887042946505\\
143	0.00409894659770956\\
144	0.00409902402485492\\
145	0.00409910273383964\\
146	0.00409918274802353\\
147	0.00409926409118955\\
148	0.00409934678754345\\
149	0.00409943086171322\\
150	0.00409951633875851\\
151	0.00409960324420857\\
152	0.00409969160407124\\
153	0.00409978144484217\\
154	0.00409987279351427\\
155	0.00409996567758716\\
156	0.00410006012507712\\
157	0.00410015616452687\\
158	0.00410025382501584\\
159	0.00410035313617052\\
160	0.00410045412817502\\
161	0.00410055683178177\\
162	0.00410066127832264\\
163	0.00410076749971998\\
164	0.00410087552849824\\
165	0.0041009853977953\\
166	0.00410109714137453\\
167	0.00410121079363689\\
168	0.00410132638963302\\
169	0.00410144396507606\\
170	0.00410156355635425\\
171	0.00410168520054384\\
172	0.00410180893542254\\
173	0.00410193479948294\\
174	0.00410206283194616\\
175	0.00410219307277598\\
176	0.00410232556269296\\
177	0.00410246034318903\\
178	0.00410259745654214\\
179	0.00410273694583135\\
180	0.00410287885495212\\
181	0.00410302322863191\\
182	0.00410317011244582\\
183	0.00410331955283289\\
184	0.00410347159711248\\
185	0.00410362629350077\\
186	0.00410378369112799\\
187	0.00410394384005544\\
188	0.00410410679129315\\
189	0.00410427259681782\\
190	0.00410444130959084\\
191	0.00410461298357691\\
192	0.00410478767376283\\
193	0.00410496543617664\\
194	0.00410514632790705\\
195	0.00410533040712331\\
196	0.00410551773309538\\
197	0.00410570836621455\\
198	0.00410590236801416\\
199	0.00410609980119111\\
200	0.00410630072962732\\
201	0.00410650521841196\\
202	0.0041067133338639\\
203	0.00410692514355454\\
204	0.00410714071633124\\
205	0.00410736012234113\\
206	0.00410758343305527\\
207	0.00410781072129342\\
208	0.00410804206124933\\
209	0.00410827752851641\\
210	0.00410851720011405\\
211	0.00410876115451436\\
212	0.00410900947166952\\
213	0.00410926223303989\\
214	0.00410951952162238\\
215	0.00410978142197975\\
216	0.00411004802027031\\
217	0.00411031940427848\\
218	0.00411059566344596\\
219	0.00411087688890343\\
220	0.0041111631735034\\
221	0.00411145461185331\\
222	0.00411175130034976\\
223	0.0041120533372133\\
224	0.00411236082252425\\
225	0.00411267385825921\\
226	0.00411299254832841\\
227	0.00411331699861403\\
228	0.00411364731700936\\
229	0.00411398361345889\\
230	0.00411432599999937\\
231	0.00411467459080176\\
232	0.00411502950221417\\
233	0.00411539085280594\\
234	0.00411575876341246\\
235	0.00411613335718126\\
236	0.00411651475961895\\
237	0.00411690309863952\\
238	0.00411729850461322\\
239	0.00411770111041709\\
240	0.00411811105148613\\
241	0.00411852846586585\\
242	0.00411895349426588\\
243	0.00411938628011473\\
244	0.00411982696961559\\
245	0.00412027571180354\\
246	0.00412073265860372\\
247	0.00412119796489075\\
248	0.00412167178854948\\
249	0.00412215429053686\\
250	0.004122645634945\\
251	0.00412314598906578\\
252	0.00412365552345631\\
253	0.00412417441200613\\
254	0.00412470283200538\\
255	0.00412524096421461\\
256	0.00412578899293563\\
257	0.0041263471060841\\
258	0.00412691549526344\\
259	0.00412749435584006\\
260	0.00412808388702039\\
261	0.00412868429192916\\
262	0.00412929577768935\\
263	0.00412991855550383\\
264	0.0041305528407385\\
265	0.00413119885300734\\
266	0.00413185681625898\\
267	0.00413252695886524\\
268	0.00413320951371146\\
269	0.0041339047182888\\
270	0.00413461281478836\\
271	0.00413533405019761\\
272	0.00413606867639873\\
273	0.0041368169502692\\
274	0.00413757913378469\\
275	0.00413835549412417\\
276	0.00413914630377745\\
277	0.00413995184065516\\
278	0.00414077238820099\\
279	0.00414160823550564\\
280	0.00414245967742329\\
281	0.00414332701469304\\
282	0.00414421055406293\\
283	0.00414511060841741\\
284	0.00414602749690825\\
285	0.00414696154508837\\
286	0.00414791308504986\\
287	0.00414888245556494\\
288	0.00414987000223126\\
289	0.00415087607762071\\
290	0.00415190104143217\\
291	0.0041529452606486\\
292	0.00415400910969802\\
293	0.00415509297061887\\
294	0.00415619723322993\\
295	0.00415732229530426\\
296	0.00415846856274837\\
297	0.00415963644978543\\
298	0.00416082637914369\\
299	0.00416203878224955\\
300	0.00416327409942557\\
301	0.00416453278009381\\
302	0.00416581528298421\\
303	0.00416712207634835\\
304	0.00416845363817869\\
305	0.00416981045643338\\
306	0.00417119302926692\\
307	0.00417260186526686\\
308	0.0041740374836967\\
309	0.00417550041474544\\
310	0.00417699119978404\\
311	0.00417851039162818\\
312	0.00418005855480817\\
313	0.00418163626584737\\
314	0.00418324411354955\\
315	0.00418488269929651\\
316	0.00418655263735967\\
317	0.0041882545552216\\
318	0.00418998909390109\\
319	0.00419175690828564\\
320	0.00419355866747476\\
321	0.00419539505513425\\
322	0.00419726676986236\\
323	0.00419917452556858\\
324	0.00420111905186532\\
325	0.00420310109447362\\
326	0.0042051214156433\\
327	0.0042071807945886\\
328	0.00420928002793997\\
329	0.00421141993021198\\
330	0.00421360133428882\\
331	0.00421582509192851\\
332	0.00421809207428652\\
333	0.0042204031724594\\
334	0.0042227592980493\\
335	0.00422516138375087\\
336	0.00422761038396227\\
337	0.00423010727542065\\
338	0.00423265305786158\\
339	0.00423524875470434\\
340	0.00423789541376437\\
341	0.00424059410799402\\
342	0.00424334593625302\\
343	0.00424615202410991\\
344	0.00424901352467674\\
345	0.00425193161947885\\
346	0.00425490751936006\\
347	0.00425794246542194\\
348	0.0042610377299823\\
349	0.00426419461758929\\
350	0.00426741446607708\\
351	0.00427069864766853\\
352	0.00427404857012499\\
353	0.00427746567794334\\
354	0.00428095145359514\\
355	0.00428450741882037\\
356	0.00428813513598078\\
357	0.00429183620947113\\
358	0.00429561228719527\\
359	0.00429946506212598\\
360	0.00430339627395013\\
361	0.00430740771077397\\
362	0.00431150121091045\\
363	0.00431567866475392\\
364	0.00431994201674703\\
365	0.00432429326744639\\
366	0.00432873447569465\\
367	0.00433326776090921\\
368	0.00433789530549518\\
369	0.00434261935737215\\
370	0.00434744223265894\\
371	0.00435236631852469\\
372	0.00435739407623897\\
373	0.00436252804447881\\
374	0.00436777084290759\\
375	0.00437312517544296\\
376	0.00437859383406985\\
377	0.00438417970281689\\
378	0.00438988576193537\\
379	0.00439571509262823\\
380	0.00440167088194655\\
381	0.00440775642781222\\
382	0.00441397514429105\\
383	0.00442033056711756\\
384	0.00442682635946839\\
385	0.00443346631798028\\
386	0.00444025437902488\\
387	0.00444719462527896\\
388	0.00445429129230295\\
389	0.0044615487752745\\
390	0.00446897163596021\\
391	0.00447656460997009\\
392	0.00448433261412566\\
393	0.00449228075180321\\
394	0.00450041431890918\\
395	0.0045087388093455\\
396	0.0045172599237471\\
397	0.00452598357799719\\
398	0.00453491591445068\\
399	0.00454406331532495\\
400	0.0045534324175252\\
401	0.00456303012911759\\
402	0.00457286364769764\\
403	0.00458294048112205\\
404	0.00459326847221748\\
405	0.00460385582532282\\
406	0.00461471113660434\\
407	0.00462584343692504\\
408	0.00463726221812296\\
409	0.00464897746180932\\
410	0.00466099966391952\\
411	0.00467333983998608\\
412	0.00468600960271405\\
413	0.00469902118314442\\
414	0.00471238743222883\\
415	0.00472612181762712\\
416	0.00474023824665073\\
417	0.00475475092648928\\
418	0.00476967431722962\\
419	0.00478502307559469\\
420	0.00480081204463977\\
421	0.0048170562845384\\
422	0.00483377108440233\\
423	0.00485097166490057\\
424	0.00486867295982357\\
425	0.00488688942746511\\
426	0.00490563551255536\\
427	0.00492492582990952\\
428	0.00494477550822037\\
429	0.00496520076900319\\
430	0.00498621983945126\\
431	0.00500785429668989\\
432	0.00503013499635789\\
433	0.00505310424657145\\
434	0.00507680680091776\\
435	0.0051012920114171\\
436	0.00512661182744845\\
437	0.00515281394145678\\
438	0.00517994117337858\\
439	0.00520803153272416\\
440	0.00523711506502273\\
441	0.00526720957702407\\
442	0.00529831488517551\\
443	0.00533040512244193\\
444	0.00536341849519898\\
445	0.00539724369300036\\
446	0.00543170189834466\\
447	0.00546610176344987\\
448	0.00549990610319642\\
449	0.00553301837589037\\
450	0.00556533487705187\\
451	0.00559674536991488\\
452	0.0056271342825192\\
453	0.00565638269071956\\
454	0.00568437138183165\\
455	0.00571098540078678\\
456	0.00573612057040539\\
457	0.00575969257333471\\
458	0.00578164973338271\\
459	0.00580199069287926\\
460	0.00582078861599557\\
461	0.0058388742305532\\
462	0.00585652106058518\\
463	0.00587372288773997\\
464	0.00589047957807101\\
465	0.00590679819241133\\
466	0.00592269410962667\\
467	0.00593819208772081\\
468	0.00595332714098071\\
469	0.00596814504838323\\
470	0.00598270223198533\\
471	0.00599706463399752\\
472	0.00601130506551497\\
473	0.00602549828895608\\
474	0.00603968883362203\\
475	0.00605389505768314\\
476	0.00606813068370529\\
477	0.00608241069479026\\
478	0.00609675109911606\\
479	0.00611116860606077\\
480	0.00612568020806276\\
481	0.00614030266958773\\
482	0.00615505193696179\\
483	0.00616994250272472\\
484	0.00618498678894013\\
485	0.0062001946602503\\
486	0.00621557354234828\\
487	0.00623112997508471\\
488	0.00624687045373967\\
489	0.006262801363535\\
490	0.00627892891613285\\
491	0.00629525909250054\\
492	0.00631179759747023\\
493	0.00632854983207803\\
494	0.00634552089004974\\
495	0.00636271558416575\\
496	0.00638013850598619\\
497	0.00639779411750585\\
498	0.00641568685551596\\
499	0.00643382117631433\\
500	0.00645220155775853\\
501	0.00647083250442214\\
502	0.00648971855613801\\
503	0.00650886430004107\\
504	0.0065282743859806\\
505	0.00654795354486264\\
506	0.00656790660911695\\
507	0.00658813853410985\\
508	0.00660865441902504\\
509	0.00662945952567692\\
510	0.00665055929520593\\
511	0.00667195936502683\\
512	0.00669366558756179\\
513	0.00671568405080497\\
514	0.00673802110075266\\
515	0.00676068336572467\\
516	0.00678367778260441\\
517	0.00680701162504017\\
518	0.00683069253368399\\
519	0.00685472854859688\\
520	0.00687912814401729\\
521	0.00690390026575907\\
522	0.00692905437143166\\
523	0.00695460047353777\\
524	0.00698054918542238\\
525	0.00700691176997008\\
526	0.00703370019084742\\
527	0.00706092716595188\\
528	0.00708860622254237\\
529	0.00711675175328499\\
530	0.00714537907203752\\
531	0.00717450446771347\\
532	0.00720414525424005\\
533	0.00723431981403947\\
534	0.00726504763054851\\
535	0.00729634931502445\\
536	0.00732824661246005\\
537	0.00736076237665303\\
538	0.0073939205406289\\
539	0.00742774592141068\\
540	0.00746226382161746\\
541	0.007497499781614\\
542	0.00753347926021568\\
543	0.00757022206150404\\
544	0.00760775241840647\\
545	0.00764609839092532\\
546	0.00768528247896232\\
547	0.00772530972612136\\
548	0.00776618996457804\\
549	0.00780792246057549\\
550	0.00785049159263132\\
551	0.00789380041828882\\
552	0.00793786967130199\\
553	0.00798275837360501\\
554	0.00802854609916383\\
555	0.00807542371984085\\
556	0.00812345795096185\\
557	0.00817273046972689\\
558	0.00822337605108118\\
559	0.0082755419680626\\
560	0.00832801029844381\\
561	0.00837909318940157\\
562	0.00842813555899623\\
563	0.00847661969636791\\
564	0.00852511945157862\\
565	0.00857380264738345\\
566	0.00862269013955515\\
567	0.00867172225479524\\
568	0.00872080160460923\\
569	0.00876904162934258\\
570	0.00881606796714063\\
571	0.00886258479454568\\
572	0.00890894303025856\\
573	0.00895519463224402\\
574	0.00900097900874467\\
575	0.00904636842012797\\
576	0.00909180468302204\\
577	0.00913731409301758\\
578	0.00918286975492669\\
579	0.0092284285101118\\
580	0.00927394186312772\\
581	0.00931935765412942\\
582	0.00936462054353618\\
583	0.00940967226850926\\
584	0.00945445201388313\\
585	0.00949889694069117\\
586	0.00954294292450232\\
587	0.00958652555033089\\
588	0.00962958140293929\\
589	0.0096720496394373\\
590	0.0097138736812812\\
591	0.0097550024176802\\
592	0.00979538909575469\\
593	0.00983498278816331\\
594	0.00987369853931868\\
595	0.00991118387968948\\
596	0.00994658651044256\\
597	0.00997788999445116\\
598	0.010000292044645\\
599	0\\
600	0\\
};
\addplot [color=blue!25!mycolor7,solid,forget plot]
  table[row sep=crcr]{%
1	0.00409179167663971\\
2	0.00409179884488707\\
3	0.00409180611331869\\
4	0.00409181348334759\\
5	0.00409182095641101\\
6	0.00409182853397098\\
7	0.00409183621751489\\
8	0.00409184400855593\\
9	0.00409185190863384\\
10	0.00409185991931529\\
11	0.00409186804219466\\
12	0.00409187627889456\\
13	0.00409188463106638\\
14	0.00409189310039107\\
15	0.00409190168857962\\
16	0.00409191039737383\\
17	0.004091919228547\\
18	0.00409192818390445\\
19	0.00409193726528435\\
20	0.0040919464745584\\
21	0.00409195581363252\\
22	0.00409196528444763\\
23	0.00409197488898028\\
24	0.00409198462924354\\
25	0.00409199450728775\\
26	0.00409200452520124\\
27	0.0040920146851111\\
28	0.00409202498918411\\
29	0.00409203543962756\\
30	0.00409204603868992\\
31	0.00409205678866186\\
32	0.00409206769187707\\
33	0.0040920787507131\\
34	0.00409208996759235\\
35	0.00409210134498289\\
36	0.00409211288539939\\
37	0.00409212459140411\\
38	0.00409213646560782\\
39	0.00409214851067072\\
40	0.00409216072930356\\
41	0.00409217312426847\\
42	0.00409218569838013\\
43	0.00409219845450664\\
44	0.00409221139557072\\
45	0.0040922245245506\\
46	0.00409223784448126\\
47	0.00409225135845542\\
48	0.00409226506962462\\
49	0.00409227898120043\\
50	0.00409229309645551\\
51	0.00409230741872489\\
52	0.00409232195140688\\
53	0.00409233669796452\\
54	0.00409235166192667\\
55	0.00409236684688922\\
56	0.00409238225651635\\
57	0.00409239789454178\\
58	0.00409241376477004\\
59	0.00409242987107773\\
60	0.00409244621741486\\
61	0.00409246280780622\\
62	0.00409247964635263\\
63	0.00409249673723235\\
64	0.00409251408470245\\
65	0.00409253169310027\\
66	0.00409254956684473\\
67	0.00409256771043788\\
68	0.00409258612846634\\
69	0.00409260482560274\\
70	0.00409262380660726\\
71	0.00409264307632917\\
72	0.00409266263970835\\
73	0.00409268250177698\\
74	0.00409270266766095\\
75	0.0040927231425816\\
76	0.00409274393185739\\
77	0.00409276504090553\\
78	0.00409278647524373\\
79	0.00409280824049184\\
80	0.00409283034237376\\
81	0.00409285278671906\\
82	0.00409287557946492\\
83	0.00409289872665787\\
84	0.00409292223445579\\
85	0.00409294610912974\\
86	0.004092970357066\\
87	0.00409299498476787\\
88	0.00409301999885785\\
89	0.0040930454060796\\
90	0.00409307121330013\\
91	0.00409309742751178\\
92	0.0040931240558345\\
93	0.00409315110551802\\
94	0.00409317858394403\\
95	0.00409320649862865\\
96	0.00409323485722457\\
97	0.00409326366752356\\
98	0.00409329293745885\\
99	0.00409332267510763\\
100	0.00409335288869359\\
101	0.00409338358658947\\
102	0.00409341477731978\\
103	0.00409344646956346\\
104	0.00409347867215667\\
105	0.00409351139409552\\
106	0.00409354464453912\\
107	0.00409357843281241\\
108	0.00409361276840935\\
109	0.00409364766099587\\
110	0.00409368312041317\\
111	0.00409371915668098\\
112	0.00409375578000092\\
113	0.00409379300076002\\
114	0.00409383082953415\\
115	0.00409386927709188\\
116	0.00409390835439814\\
117	0.00409394807261812\\
118	0.00409398844312139\\
119	0.00409402947748602\\
120	0.00409407118750284\\
121	0.00409411358518003\\
122	0.00409415668274764\\
123	0.00409420049266249\\
124	0.00409424502761317\\
125	0.0040942903005252\\
126	0.00409433632456659\\
127	0.00409438311315353\\
128	0.00409443067995644\\
129	0.00409447903890625\\
130	0.00409452820420132\\
131	0.00409457819031449\\
132	0.00409462901200102\\
133	0.00409468068430702\\
134	0.00409473322257879\\
135	0.00409478664247332\\
136	0.00409484095997023\\
137	0.00409489619138556\\
138	0.00409495235338789\\
139	0.00409500946301823\\
140	0.00409506753771362\\
141	0.00409512659533744\\
142	0.00409518665421704\\
143	0.00409524773319237\\
144	0.00409530985167809\\
145	0.0040953730297432\\
146	0.00409543728820971\\
147	0.00409550264876417\\
148	0.00409556913404742\\
149	0.00409563676758475\\
150	0.00409570557304781\\
151	0.00409577557197914\\
152	0.00409584678632554\\
153	0.00409591923844574\\
154	0.00409599295111822\\
155	0.00409606794754915\\
156	0.00409614425138048\\
157	0.00409622188669833\\
158	0.00409630087804125\\
159	0.0040963812504089\\
160	0.00409646302927072\\
161	0.00409654624057492\\
162	0.00409663091075744\\
163	0.00409671706675124\\
164	0.00409680473599569\\
165	0.00409689394644617\\
166	0.00409698472658383\\
167	0.00409707710542542\\
168	0.00409717111253366\\
169	0.00409726677802718\\
170	0.00409736413259134\\
171	0.0040974632074888\\
172	0.00409756403457039\\
173	0.00409766664628621\\
174	0.00409777107569697\\
175	0.0040978773564854\\
176	0.00409798552296808\\
177	0.00409809561010729\\
178	0.00409820765352321\\
179	0.00409832168950623\\
180	0.00409843775502965\\
181	0.00409855588776244\\
182	0.00409867612608244\\
183	0.00409879850908959\\
184	0.00409892307661945\\
185	0.00409904986925723\\
186	0.00409917892835166\\
187	0.00409931029602955\\
188	0.0040994440152103\\
189	0.00409958012962086\\
190	0.00409971868381087\\
191	0.00409985972316823\\
192	0.00410000329393484\\
193	0.00410014944322269\\
194	0.00410029821903024\\
195	0.00410044967025921\\
196	0.00410060384673164\\
197	0.00410076079920708\\
198	0.00410092057940066\\
199	0.00410108324000084\\
200	0.00410124883468806\\
201	0.00410141741815344\\
202	0.00410158904611804\\
203	0.00410176377535236\\
204	0.00410194166369637\\
205	0.00410212277007982\\
206	0.00410230715454314\\
207	0.00410249487825852\\
208	0.00410268600355158\\
209	0.00410288059392352\\
210	0.0041030787140736\\
211	0.00410328042992225\\
212	0.00410348580863442\\
213	0.00410369491864363\\
214	0.00410390782967637\\
215	0.00410412461277718\\
216	0.00410434534033399\\
217	0.00410457008610425\\
218	0.00410479892524139\\
219	0.004105031934322\\
220	0.00410526919137341\\
221	0.004105510775902\\
222	0.00410575676892189\\
223	0.00410600725298448\\
224	0.00410626231220832\\
225	0.00410652203230979\\
226	0.00410678650063431\\
227	0.00410705580618822\\
228	0.00410733003967132\\
229	0.00410760929350995\\
230	0.00410789366189082\\
231	0.00410818324079571\\
232	0.00410847812803645\\
233	0.00410877842329085\\
234	0.00410908422813944\\
235	0.00410939564610263\\
236	0.00410971278267897\\
237	0.00411003574538369\\
238	0.00411036464378859\\
239	0.00411069958956213\\
240	0.00411104069651084\\
241	0.00411138808062097\\
242	0.00411174186010148\\
243	0.00411210215542755\\
244	0.00411246908938508\\
245	0.00411284278711603\\
246	0.00411322337616464\\
247	0.0041136109865246\\
248	0.00411400575068721\\
249	0.00411440780369023\\
250	0.0041148172831682\\
251	0.00411523432940332\\
252	0.00411565908537769\\
253	0.00411609169682644\\
254	0.00411653231229203\\
255	0.00411698108317972\\
256	0.00411743816381432\\
257	0.00411790371149797\\
258	0.00411837788656926\\
259	0.00411886085246391\\
260	0.0041193527757763\\
261	0.00411985382632294\\
262	0.0041203641772071\\
263	0.00412088400488505\\
264	0.00412141348923386\\
265	0.00412195281362084\\
266	0.00412250216497463\\
267	0.00412306173385831\\
268	0.00412363171454399\\
269	0.00412421230508999\\
270	0.00412480370741964\\
271	0.00412540612740305\\
272	0.00412601977494126\\
273	0.00412664486405411\\
274	0.00412728161297198\\
275	0.00412793024423284\\
276	0.00412859098478615\\
277	0.0041292640661058\\
278	0.00412994972431493\\
279	0.00413064820032278\\
280	0.00413135973994347\\
281	0.00413208459388185\\
282	0.00413282301783096\\
283	0.00413357527257914\\
284	0.00413434162411919\\
285	0.00413512234376086\\
286	0.00413591770824562\\
287	0.00413672799986431\\
288	0.00413755350657742\\
289	0.00413839452213839\\
290	0.00413925134621974\\
291	0.00414012428454199\\
292	0.00414101364900559\\
293	0.00414191975782595\\
294	0.00414284293567111\\
295	0.00414378351380293\\
296	0.00414474183022088\\
297	0.00414571822980925\\
298	0.00414671306448721\\
299	0.00414772669336217\\
300	0.0041487594828863\\
301	0.00414981180701613\\
302	0.00415088404737535\\
303	0.00415197659342082\\
304	0.00415308984261149\\
305	0.00415422420058033\\
306	0.00415538008130855\\
307	0.00415655790730176\\
308	0.0041577581097668\\
309	0.00415898112878767\\
310	0.00416022741349748\\
311	0.00416149742224266\\
312	0.00416279162273294\\
313	0.00416411049216903\\
314	0.00416545451733653\\
315	0.00416682419465783\\
316	0.00416822003021616\\
317	0.00416964253990388\\
318	0.004171092250146\\
319	0.0041725696983368\\
320	0.00417407543309548\\
321	0.00417561001452936\\
322	0.00417717401450484\\
323	0.0041787680169263\\
324	0.00418039261802365\\
325	0.00418204842664817\\
326	0.00418373606457772\\
327	0.00418545616683093\\
328	0.00418720938199121\\
329	0.00418899637254059\\
330	0.00419081781520407\\
331	0.00419267440130464\\
332	0.0041945668371294\\
333	0.00419649584430726\\
334	0.00419846216019875\\
335	0.00420046653829827\\
336	0.00420250974864935\\
337	0.00420459257827343\\
338	0.00420671583161246\\
339	0.00420888033098623\\
340	0.00421108691706498\\
341	0.0042133364493579\\
342	0.0042156298067184\\
343	0.00421796788786684\\
344	0.00422035161193176\\
345	0.00422278191901008\\
346	0.00422525977074702\\
347	0.00422778615093572\\
348	0.00423036206614031\\
349	0.00423298854634276\\
350	0.00423566664561411\\
351	0.00423839744281184\\
352	0.00424118204230375\\
353	0.00424402157471958\\
354	0.00424691719773253\\
355	0.00424987009687203\\
356	0.00425288148636925\\
357	0.00425595261003717\\
358	0.00425908474218753\\
359	0.0042622791885842\\
360	0.004265537287431\\
361	0.00426886041040458\\
362	0.00427224996373771\\
363	0.00427570738934931\\
364	0.00427923416602433\\
365	0.00428283181064584\\
366	0.00428650187948255\\
367	0.00429024596953427\\
368	0.00429406571993667\\
369	0.00429796281343218\\
370	0.00430193897790976\\
371	0.00430599598801906\\
372	0.0043101356668646\\
373	0.00431435988777598\\
374	0.00431867057611981\\
375	0.0043230697112264\\
376	0.00432755932840648\\
377	0.00433214152107077\\
378	0.00433681844298267\\
379	0.00434159231061924\\
380	0.00434646540565764\\
381	0.00435144007758906\\
382	0.00435651874643855\\
383	0.00436170390561014\\
384	0.00436699812487807\\
385	0.00437240405353485\\
386	0.00437792442370775\\
387	0.00438356205383751\\
388	0.00438931985234851\\
389	0.00439520082153542\\
390	0.00440120806168268\\
391	0.00440734477539559\\
392	0.00441361427201974\\
393	0.00442001997239376\\
394	0.00442656541389017\\
395	0.00443325425600352\\
396	0.00444009028617354\\
397	0.00444707742590148\\
398	0.0044542197372455\\
399	0.00446152142965565\\
400	0.00446898686715969\\
401	0.00447662057591346\\
402	0.00448442725215078\\
403	0.00449241177061241\\
404	0.00450057919329857\\
405	0.00450893477873799\\
406	0.00451748399207861\\
407	0.00452623251381179\\
408	0.00453518624866284\\
409	0.00454435133425066\\
410	0.00455373414953379\\
411	0.00456334132928628\\
412	0.00457317977488908\\
413	0.00458325666578606\\
414	0.00459357947365316\\
415	0.00460415597434657\\
416	0.00461499426294936\\
417	0.00462610277355903\\
418	0.00463749030177741\\
419	0.00464916603393684\\
420	0.00466113958139602\\
421	0.00467342101085839\\
422	0.0046860208531977\\
423	0.00469895012246663\\
424	0.00471222034629779\\
425	0.00472584364112071\\
426	0.00473983275497101\\
427	0.00475420110916848\\
428	0.00476896283486819\\
429	0.00478413279786332\\
430	0.00479972660194332\\
431	0.0048157605760717\\
432	0.00483225164319992\\
433	0.00484921720001487\\
434	0.00486667521852141\\
435	0.00488464392529949\\
436	0.0049031412572039\\
437	0.00492218498644744\\
438	0.00494179291144521\\
439	0.00496198283679481\\
440	0.00498277260286462\\
441	0.00500418019598847\\
442	0.00502622398219888\\
443	0.00504892311777252\\
444	0.00507229816531374\\
445	0.00509637169284442\\
446	0.00512117354920067\\
447	0.00514674900909566\\
448	0.00517315033241187\\
449	0.00520042753936902\\
450	0.00522863063663961\\
451	0.00525780818888461\\
452	0.00528800531821086\\
453	0.00531926094482009\\
454	0.00535160402406627\\
455	0.00538504845803528\\
456	0.00541958625039071\\
457	0.00545517834741904\\
458	0.00549174252478596\\
459	0.00552913740957662\\
460	0.00556714135810725\\
461	0.005604787298736\\
462	0.0056416798000527\\
463	0.00567770459455718\\
464	0.00571274007328253\\
465	0.00574665857487867\\
466	0.00577932847560934\\
467	0.0058106174084599\\
468	0.00584039718465546\\
469	0.0058685510370182\\
470	0.00589498376403355\\
471	0.00591963580200271\\
472	0.00594250260714197\\
473	0.00596366113022463\\
474	0.00598397845686839\\
475	0.0060038223638901\\
476	0.00602318874475772\\
477	0.00604208086658422\\
478	0.00606051065279006\\
479	0.00607849994106377\\
480	0.00609608160931803\\
481	0.00611330040582075\\
482	0.00613021324010889\\
483	0.00614688858232525\\
484	0.00616340446907811\\
485	0.00617984441050601\\
486	0.00619628223156762\\
487	0.00621274872484579\\
488	0.00622925904839289\\
489	0.00624582983842187\\
490	0.006262478959408\\
491	0.00627922515411315\\
492	0.00629608758567791\\
493	0.00631308527169584\\
494	0.00633023642363993\\
495	0.00634755772678505\\
496	0.00636506362978485\\
497	0.00638276576432228\\
498	0.00640067291806857\\
499	0.00641879282478134\\
500	0.00643713319965338\\
501	0.00645570166760095\\
502	0.00647450569462391\\
503	0.00649355252754852\\
504	0.00651284914856596\\
505	0.00653240225181333\\
506	0.00655221824944383\\
507	0.00657230331361228\\
508	0.00659266345765685\\
509	0.00661330465311988\\
510	0.00663423294085632\\
511	0.00665545447750711\\
512	0.00667697554746031\\
513	0.00669880257938798\\
514	0.00672094216772658\\
515	0.00674340109923942\\
516	0.00676618638448121\\
517	0.00678930529357986\\
518	0.00681276539528721\\
519	0.00683657459779465\\
520	0.00686074118949649\\
521	0.00688527387795266\\
522	0.00691018182879633\\
523	0.00693547470735929\\
524	0.00696116272365154\\
525	0.0069872566806296\\
526	0.00701376802560303\\
527	0.00704070890452965\\
528	0.00706809221883977\\
529	0.00709593168430487\\
530	0.00712424189132716\\
531	0.00715303836586031\\
532	0.00718233762994827\\
533	0.00721215726049923\\
534	0.00724251594426931\\
535	0.00727343352600266\\
536	0.00730493104605926\\
537	0.00733703076285003\\
538	0.00736975615367599\\
539	0.00740313188943341\\
540	0.00743718377564411\\
541	0.00747193864038859\\
542	0.00750742415076012\\
543	0.00754366859468992\\
544	0.00758070031442702\\
545	0.00761854702852742\\
546	0.00765723521324297\\
547	0.00769678967550589\\
548	0.00773722468672789\\
549	0.00777854347761477\\
550	0.00782075245913859\\
551	0.00786385640142318\\
552	0.00790785477206057\\
553	0.0079527453250826\\
554	0.00799851308237961\\
555	0.00804505598680104\\
556	0.00809242117741512\\
557	0.00814067103610812\\
558	0.00818989521339174\\
559	0.00824028601165865\\
560	0.00829193340389175\\
561	0.00834496567632439\\
562	0.00839955265381671\\
563	0.00845374671421613\\
564	0.00850651646442516\\
565	0.00855718422615864\\
566	0.00860737522808454\\
567	0.00865747862960277\\
568	0.0087075839716849\\
569	0.00875773555704555\\
570	0.00880783885424744\\
571	0.00885708033400435\\
572	0.0089050682940259\\
573	0.00895224789310436\\
574	0.00899912084127209\\
575	0.00904569281043743\\
576	0.00909165223986215\\
577	0.00913727099022529\\
578	0.00918285607127283\\
579	0.00922842280592135\\
580	0.00927393897059496\\
581	0.00931935612056587\\
582	0.00936461969699102\\
583	0.00940967179489983\\
584	0.00945445175374296\\
585	0.00949889680731318\\
586	0.00954294286465648\\
587	0.00958652552981098\\
588	0.00962958139833628\\
589	0.00967204963904421\\
590	0.0097138736812812\\
591	0.00975500241768019\\
592	0.00979538909575469\\
593	0.00983498278816331\\
594	0.00987369853931868\\
595	0.00991118387968948\\
596	0.00994658651044256\\
597	0.00997788999445116\\
598	0.010000292044645\\
599	0\\
600	0\\
};
\addplot [color=mycolor9,solid,forget plot]
  table[row sep=crcr]{%
1	0.00408032568010838\\
2	0.0040803317214586\\
3	0.00408033785212888\\
4	0.00408034407351985\\
5	0.00408035038705741\\
6	0.00408035679419308\\
7	0.00408036329640468\\
8	0.00408036989519683\\
9	0.00408037659210129\\
10	0.00408038338867769\\
11	0.00408039028651398\\
12	0.00408039728722694\\
13	0.00408040439246285\\
14	0.00408041160389798\\
15	0.00408041892323913\\
16	0.00408042635222434\\
17	0.00408043389262328\\
18	0.00408044154623809\\
19	0.00408044931490382\\
20	0.00408045720048911\\
21	0.00408046520489688\\
22	0.00408047333006484\\
23	0.00408048157796627\\
24	0.00408048995061064\\
25	0.00408049845004424\\
26	0.00408050707835092\\
27	0.00408051583765277\\
28	0.00408052473011081\\
29	0.00408053375792565\\
30	0.00408054292333841\\
31	0.00408055222863123\\
32	0.00408056167612811\\
33	0.00408057126819572\\
34	0.00408058100724404\\
35	0.00408059089572735\\
36	0.00408060093614478\\
37	0.00408061113104128\\
38	0.0040806214830084\\
39	0.00408063199468506\\
40	0.00408064266875844\\
41	0.00408065350796486\\
42	0.00408066451509057\\
43	0.00408067569297266\\
44	0.00408068704449997\\
45	0.00408069857261397\\
46	0.00408071028030962\\
47	0.00408072217063644\\
48	0.00408073424669928\\
49	0.0040807465116594\\
50	0.00408075896873532\\
51	0.00408077162120389\\
52	0.00408078447240126\\
53	0.00408079752572385\\
54	0.00408081078462939\\
55	0.00408082425263794\\
56	0.00408083793333293\\
57	0.00408085183036224\\
58	0.00408086594743926\\
59	0.00408088028834398\\
60	0.00408089485692403\\
61	0.00408090965709584\\
62	0.00408092469284576\\
63	0.00408093996823122\\
64	0.00408095548738186\\
65	0.00408097125450065\\
66	0.00408098727386516\\
67	0.00408100354982865\\
68	0.00408102008682132\\
69	0.00408103688935157\\
70	0.00408105396200713\\
71	0.00408107130945636\\
72	0.00408108893644952\\
73	0.00408110684781996\\
74	0.00408112504848546\\
75	0.00408114354344953\\
76	0.0040811623378026\\
77	0.00408118143672343\\
78	0.0040812008454803\\
79	0.0040812205694325\\
80	0.00408124061403147\\
81	0.00408126098482225\\
82	0.00408128168744471\\
83	0.00408130272763498\\
84	0.00408132411122675\\
85	0.00408134584415252\\
86	0.00408136793244503\\
87	0.00408139038223856\\
88	0.00408141319977018\\
89	0.00408143639138112\\
90	0.00408145996351798\\
91	0.00408148392273409\\
92	0.00408150827569066\\
93	0.00408153302915806\\
94	0.00408155819001696\\
95	0.00408158376525952\\
96	0.00408160976199053\\
97	0.00408163618742834\\
98	0.00408166304890601\\
99	0.00408169035387211\\
100	0.00408171810989168\\
101	0.00408174632464702\\
102	0.00408177500593827\\
103	0.0040818041616841\\
104	0.00408183379992208\\
105	0.00408186392880907\\
106	0.00408189455662128\\
107	0.00408192569175447\\
108	0.00408195734272342\\
109	0.00408198951816175\\
110	0.00408202222682114\\
111	0.00408205547757034\\
112	0.00408208927939387\\
113	0.00408212364139036\\
114	0.00408215857277041\\
115	0.00408219408285411\\
116	0.00408223018106779\\
117	0.00408226687694039\\
118	0.0040823041800988\\
119	0.00408234210026281\\
120	0.00408238064723854\\
121	0.00408241983091128\\
122	0.00408245966123659\\
123	0.00408250014823023\\
124	0.00408254130195599\\
125	0.00408258313251164\\
126	0.00408262565001214\\
127	0.00408266886457015\\
128	0.00408271278627252\\
129	0.0040827574251529\\
130	0.00408280279115896\\
131	0.00408284889411325\\
132	0.00408289574366644\\
133	0.00408294334924135\\
134	0.00408299171996512\\
135	0.00408304086458768\\
136	0.00408309079138236\\
137	0.00408314150802506\\
138	0.00408319302144679\\
139	0.00408324533765281\\
140	0.00408329846150164\\
141	0.00408335239643531\\
142	0.00408340714415597\\
143	0.00408346270425434\\
144	0.00408351907383572\\
145	0.00408357624731534\\
146	0.00408363421695109\\
147	0.00408369297595483\\
148	0.00408375253033698\\
149	0.00408381294198024\\
150	0.00408387442250217\\
151	0.00408393699150774\\
152	0.0040840006689656\\
153	0.00408406547521487\\
154	0.004084131430972\\
155	0.0040841985573378\\
156	0.00408426687580454\\
157	0.00408433640826325\\
158	0.00408440717701115\\
159	0.0040844792047591\\
160	0.00408455251463939\\
161	0.00408462713021347\\
162	0.00408470307548001\\
163	0.00408478037488287\\
164	0.00408485905331948\\
165	0.00408493913614922\\
166	0.00408502064920193\\
167	0.00408510361878671\\
168	0.00408518807170073\\
169	0.00408527403523841\\
170	0.00408536153720045\\
171	0.00408545060590334\\
172	0.00408554127018892\\
173	0.00408563355943405\\
174	0.00408572750356054\\
175	0.00408582313304534\\
176	0.0040859204789307\\
177	0.00408601957283474\\
178	0.00408612044696201\\
179	0.00408622313411457\\
180	0.0040863276677028\\
181	0.00408643408175689\\
182	0.00408654241093819\\
183	0.00408665269055101\\
184	0.00408676495655458\\
185	0.00408687924557497\\
186	0.00408699559491781\\
187	0.00408711404258058\\
188	0.00408723462726558\\
189	0.00408735738839309\\
190	0.00408748236611461\\
191	0.00408760960132647\\
192	0.00408773913568356\\
193	0.00408787101161365\\
194	0.00408800527233164\\
195	0.00408814196185414\\
196	0.00408828112501439\\
197	0.00408842280747765\\
198	0.00408856705575633\\
199	0.00408871391722615\\
200	0.0040888634401419\\
201	0.00408901567365398\\
202	0.00408917066782506\\
203	0.00408932847364704\\
204	0.00408948914305838\\
205	0.00408965272896175\\
206	0.00408981928524199\\
207	0.00408998886678448\\
208	0.0040901615294938\\
209	0.00409033733031263\\
210	0.00409051632724121\\
211	0.00409069857935697\\
212	0.00409088414683478\\
213	0.00409107309096724\\
214	0.00409126547418565\\
215	0.00409146136008117\\
216	0.00409166081342647\\
217	0.0040918639001977\\
218	0.00409207068759704\\
219	0.0040922812440754\\
220	0.00409249563935562\\
221	0.00409271394445633\\
222	0.00409293623171591\\
223	0.00409316257481705\\
224	0.00409339304881169\\
225	0.00409362773014645\\
226	0.00409386669668852\\
227	0.0040941100277519\\
228	0.0040943578041243\\
229	0.00409461010809421\\
230	0.00409486702347883\\
231	0.00409512863565205\\
232	0.00409539503157325\\
233	0.00409566629981644\\
234	0.00409594253059986\\
235	0.00409622381581618\\
236	0.00409651024906302\\
237	0.00409680192567429\\
238	0.00409709894275164\\
239	0.00409740139919666\\
240	0.00409770939574349\\
241	0.00409802303499203\\
242	0.00409834242144152\\
243	0.00409866766152481\\
244	0.00409899886364285\\
245	0.00409933613819994\\
246	0.00409967959763939\\
247	0.00410002935647955\\
248	0.00410038553135043\\
249	0.00410074824103093\\
250	0.00410111760648614\\
251	0.00410149375090549\\
252	0.00410187679974104\\
253	0.00410226688074624\\
254	0.00410266412401525\\
255	0.00410306866202235\\
256	0.00410348062966169\\
257	0.00410390016428735\\
258	0.00410432740575361\\
259	0.0041047624964549\\
260	0.00410520558136627\\
261	0.00410565680808287\\
262	0.00410611632685941\\
263	0.0041065842906483\\
264	0.00410706085513649\\
265	0.00410754617878015\\
266	0.00410804042283614\\
267	0.00410854375138903\\
268	0.00410905633137184\\
269	0.00410957833257757\\
270	0.00411010992765883\\
271	0.00411065129211007\\
272	0.00411120260422655\\
273	0.00411176404503058\\
274	0.00411233579815451\\
275	0.00411291804966501\\
276	0.00411351098781273\\
277	0.0041141148026981\\
278	0.00411472968589211\\
279	0.00411535583028356\\
280	0.00411599343159979\\
281	0.00411664269946865\\
282	0.0041173038480168\\
283	0.00411797709534109\\
284	0.00411866266359079\\
285	0.00411936077905226\\
286	0.0041200716722368\\
287	0.00412079557797152\\
288	0.00412153273549352\\
289	0.0041222833885479\\
290	0.00412304778548947\\
291	0.00412382617938882\\
292	0.00412461882814302\\
293	0.00412542599459136\\
294	0.004126247946637\\
295	0.00412708495737457\\
296	0.00412793730522538\\
297	0.0041288052740808\\
298	0.00412968915345553\\
299	0.00413058923865261\\
300	0.0041315058309427\\
301	0.00413243923776124\\
302	0.00413338977292813\\
303	0.00413435775689608\\
304	0.00413534351703653\\
305	0.00413634738797513\\
306	0.00413736971199214\\
307	0.00413841083951022\\
308	0.00413947112969897\\
309	0.00414055095123541\\
310	0.00414165068327278\\
311	0.00414277071668319\\
312	0.00414391145564883\\
313	0.00414507331965988\\
314	0.0041462567458689\\
315	0.00414746219133093\\
316	0.00414869013318215\\
317	0.00414994105951313\\
318	0.00415121542405593\\
319	0.00415251367288903\\
320	0.00415383626087322\\
321	0.00415518365184186\\
322	0.00415655631879631\\
323	0.00415795474410643\\
324	0.00415937941971667\\
325	0.00416083084735821\\
326	0.00416230953876658\\
327	0.0041638160159058\\
328	0.0041653508111988\\
329	0.00416691446776466\\
330	0.00416850753966274\\
331	0.00417013059214414\\
332	0.00417178420191086\\
333	0.00417346895738277\\
334	0.00417518545897301\\
335	0.00417693431937199\\
336	0.0041787161638408\\
337	0.00418053163051362\\
338	0.00418238137071067\\
339	0.0041842660492613\\
340	0.00418618634483813\\
341	0.00418814295030285\\
342	0.00419013657306392\\
343	0.00419216793544708\\
344	0.00419423777507914\\
345	0.00419634684528539\\
346	0.00419849591550199\\
347	0.00420068577170344\\
348	0.0042029172168468\\
349	0.00420519107133266\\
350	0.00420750817348505\\
351	0.00420986938005095\\
352	0.00421227556672175\\
353	0.00421472762867913\\
354	0.00421722648116924\\
355	0.00421977306011006\\
356	0.00422236832273927\\
357	0.00422501324831298\\
358	0.00422770883886822\\
359	0.00423045612005951\\
360	0.00423325614202024\\
361	0.00423610997994662\\
362	0.00423901873440536\\
363	0.00424198353210752\\
364	0.00424500552671295\\
365	0.00424808589966584\\
366	0.00425122586106436\\
367	0.00425442665056831\\
368	0.00425768953834222\\
369	0.00426101582603336\\
370	0.0042644068477877\\
371	0.00426786397130313\\
372	0.004271388598917\\
373	0.00427498216872125\\
374	0.00427864615570287\\
375	0.00428238207289374\\
376	0.00428619147251039\\
377	0.00429007594705438\\
378	0.00429403713033155\\
379	0.00429807669835758\\
380	0.00430219637020581\\
381	0.00430639790936694\\
382	0.00431068312732309\\
383	0.00431505388634855\\
384	0.00431951210153059\\
385	0.00432405974289698\\
386	0.0043286988376544\\
387	0.00433343147254496\\
388	0.00433825979632698\\
389	0.00434318602238383\\
390	0.00434821243146024\\
391	0.00435334137452092\\
392	0.00435857527575316\\
393	0.00436391663571122\\
394	0.00436936803461629\\
395	0.00437493213578097\\
396	0.00438061168915676\\
397	0.0043864095350035\\
398	0.00439232860766343\\
399	0.00439837193942105\\
400	0.00440454266442583\\
401	0.00441084402264989\\
402	0.0044172793638488\\
403	0.00442385215147291\\
404	0.00443056596649915\\
405	0.00443742451113101\\
406	0.00444443161216516\\
407	0.00445159122412042\\
408	0.00445890743212007\\
409	0.00446638445468958\\
410	0.00447402664703967\\
411	0.00448183850433959\\
412	0.004489824665813\\
413	0.00449798992021728\\
414	0.00450633921284684\\
415	0.00451487766131778\\
416	0.00452361057509217\\
417	0.00453254346895203\\
418	0.00454168207405567\\
419	0.00455103234957194\\
420	0.00456060049389352\\
421	0.00457039295449526\\
422	0.00458041643914397\\
423	0.00459067792893476\\
424	0.00460118469484066\\
425	0.00461194431180844\\
426	0.00462296467320176\\
427	0.00463425400557573\\
428	0.00464582088389187\\
429	0.00465767424757893\\
430	0.00466982341915739\\
431	0.00468227812477799\\
432	0.0046950485232572\\
433	0.00470814523954862\\
434	0.00472157936550173\\
435	0.00473536245780382\\
436	0.00474950659521184\\
437	0.0047640244387787\\
438	0.00477892927470935\\
439	0.00479423506031151\\
440	0.00480995647212761\\
441	0.00482610895435044\\
442	0.00484270876391218\\
443	0.00485977300663505\\
444	0.00487731967260649\\
445	0.00489536790457525\\
446	0.00491393775780624\\
447	0.00493304950806499\\
448	0.00495272349413835\\
449	0.00497298037922986\\
450	0.00499384109334846\\
451	0.00501532678326504\\
452	0.00503745878020474\\
453	0.00506025859515393\\
454	0.0050837479230485\\
455	0.00510794836847699\\
456	0.0051328848498848\\
457	0.00515858542537479\\
458	0.00518507829680298\\
459	0.0052123916363027\\
460	0.00524055774871919\\
461	0.00526962519665851\\
462	0.0052996483229499\\
463	0.00533068136551119\\
464	0.00536277687710101\\
465	0.00539598350622364\\
466	0.00543034292173908\\
467	0.00546588564894863\\
468	0.00550262551694845\\
469	0.00554055226288177\\
470	0.00557962158440338\\
471	0.00561974192124584\\
472	0.00566075695097854\\
473	0.00570242247316536\\
474	0.00574371723299744\\
475	0.00578413666240843\\
476	0.0058235514348476\\
477	0.00586182452730329\\
478	0.00589881302973534\\
479	0.00593437122793735\\
480	0.00596835508548402\\
481	0.0060006288329282\\
482	0.00603107436035096\\
483	0.00605960441782868\\
484	0.00608618090950803\\
485	0.0061108399790797\\
486	0.00613394827269195\\
487	0.0061565572231749\\
488	0.00617866106963987\\
489	0.00620026206261332\\
490	0.00622137190575537\\
491	0.00624201318766271\\
492	0.00626222068822813\\
493	0.00628204238213333\\
494	0.00630153987527973\\
495	0.00632078789161669\\
496	0.00633987226080195\\
497	0.00635888563866321\\
498	0.00637791389053429\\
499	0.00639699321982514\\
500	0.00641614092023951\\
501	0.00643537595941995\\
502	0.00645471868676816\\
503	0.00647419042522651\\
504	0.00649381293908639\\
505	0.00651360777983297\\
506	0.00653359552902565\\
507	0.00655379498480331\\
508	0.00657422238156807\\
509	0.00659489079711179\\
510	0.00661581066893441\\
511	0.00663699164239693\\
512	0.00665844341738469\\
513	0.00668017567478128\\
514	0.00670219800864469\\
515	0.00672451987082082\\
516	0.00674715053602441\\
517	0.00677009909628874\\
518	0.0067933744936005\\
519	0.00681698559766519\\
520	0.00684094133084513\\
521	0.00686525083245867\\
522	0.00688992357169755\\
523	0.006914969393402\\
524	0.00694039855405188\\
525	0.00696622176460758\\
526	0.00699245024049316\\
527	0.00701909575863972\\
528	0.00704617072100075\\
529	0.00707368822332517\\
530	0.00710166212726932\\
531	0.00713010713325534\\
532	0.00715903885104884\\
533	0.00718847386631718\\
534	0.00721842980511202\\
535	0.00724892539692609\\
536	0.00727998053398537\\
537	0.00731161632367419\\
538	0.00734385513004147\\
539	0.00737672059906469\\
540	0.00741023766083625\\
541	0.00744443250022344\\
542	0.00747933248540618\\
543	0.00751496604033652\\
544	0.00755136244984742\\
545	0.00758855157424061\\
546	0.00762656343832719\\
547	0.00766542764552213\\
548	0.00770517274565856\\
549	0.00774582545318689\\
550	0.00778739842297575\\
551	0.00782990245782332\\
552	0.00787334632801186\\
553	0.00791773109971343\\
554	0.00796305422576511\\
555	0.00800931980616587\\
556	0.00805652647189953\\
557	0.0081046653681536\\
558	0.00815370949778185\\
559	0.00820357552247344\\
560	0.00825431025459932\\
561	0.00830597583091507\\
562	0.00835865057764933\\
563	0.00841255528549486\\
564	0.00846782554576636\\
565	0.00852461550995907\\
566	0.00858084970818208\\
567	0.00863568580905793\\
568	0.00868850380505972\\
569	0.00874038008329295\\
570	0.00879204607992877\\
571	0.00884343635142392\\
572	0.00889467572782135\\
573	0.00894516958355532\\
574	0.00899434902706773\\
575	0.00904222600225824\\
576	0.00908965295883001\\
577	0.00913649196733126\\
578	0.00918262170103313\\
579	0.00922833831212848\\
580	0.00927390407257964\\
581	0.00931933850652799\\
582	0.00936461041460092\\
583	0.00940966667844402\\
584	0.00945444888056296\\
585	0.00949889521326668\\
586	0.00954294203465611\\
587	0.00958652514879544\\
588	0.00962958126464585\\
589	0.00967204960829027\\
590	0.00971387367861981\\
591	0.0097550024176802\\
592	0.00979538909575469\\
593	0.00983498278816331\\
594	0.00987369853931868\\
595	0.00991118387968948\\
596	0.00994658651044256\\
597	0.00997788999445116\\
598	0.010000292044645\\
599	0\\
600	0\\
};
\addplot [color=blue!50!mycolor7,solid,forget plot]
  table[row sep=crcr]{%
1	0.00402964426831132\\
2	0.00402965034379505\\
3	0.00402965651562647\\
4	0.00402966278541615\\
5	0.0040296691548033\\
6	0.00402967562545648\\
7	0.00402968219907398\\
8	0.0040296888773844\\
9	0.00402969566214725\\
10	0.00402970255515347\\
11	0.00402970955822588\\
12	0.00402971667321998\\
13	0.00402972390202436\\
14	0.00402973124656131\\
15	0.00402973870878748\\
16	0.00402974629069442\\
17	0.00402975399430933\\
18	0.00402976182169553\\
19	0.00402976977495321\\
20	0.00402977785622004\\
21	0.00402978606767185\\
22	0.00402979441152328\\
23	0.00402980289002851\\
24	0.00402981150548188\\
25	0.00402982026021867\\
26	0.00402982915661576\\
27	0.00402983819709248\\
28	0.00402984738411119\\
29	0.00402985672017811\\
30	0.00402986620784412\\
31	0.00402987584970546\\
32	0.00402988564840465\\
33	0.0040298956066311\\
34	0.00402990572712215\\
35	0.0040299160126637\\
36	0.00402992646609121\\
37	0.00402993709029046\\
38	0.00402994788819844\\
39	0.00402995886280432\\
40	0.00402997001715026\\
41	0.0040299813543323\\
42	0.00402999287750143\\
43	0.00403000458986439\\
44	0.00403001649468467\\
45	0.00403002859528357\\
46	0.00403004089504111\\
47	0.00403005339739706\\
48	0.00403006610585196\\
49	0.00403007902396819\\
50	0.00403009215537106\\
51	0.00403010550374975\\
52	0.00403011907285858\\
53	0.00403013286651801\\
54	0.00403014688861583\\
55	0.00403016114310827\\
56	0.00403017563402121\\
57	0.00403019036545137\\
58	0.00403020534156747\\
59	0.00403022056661152\\
60	0.00403023604490003\\
61	0.00403025178082535\\
62	0.00403026777885687\\
63	0.00403028404354242\\
64	0.00403030057950953\\
65	0.00403031739146684\\
66	0.0040303344842055\\
67	0.00403035186260053\\
68	0.00403036953161226\\
69	0.00403038749628784\\
70	0.00403040576176262\\
71	0.00403042433326177\\
72	0.00403044321610175\\
73	0.00403046241569182\\
74	0.00403048193753568\\
75	0.0040305017872331\\
76	0.00403052197048147\\
77	0.0040305424930776\\
78	0.00403056336091931\\
79	0.00403058458000714\\
80	0.00403060615644616\\
81	0.00403062809644772\\
82	0.00403065040633129\\
83	0.0040306730925263\\
84	0.00403069616157396\\
85	0.00403071962012924\\
86	0.00403074347496271\\
87	0.00403076773296263\\
88	0.00403079240113689\\
89	0.00403081748661507\\
90	0.00403084299665048\\
91	0.00403086893862231\\
92	0.00403089532003773\\
93	0.00403092214853418\\
94	0.00403094943188149\\
95	0.00403097717798414\\
96	0.00403100539488353\\
97	0.00403103409076045\\
98	0.00403106327393735\\
99	0.00403109295288078\\
100	0.00403112313620386\\
101	0.00403115383266873\\
102	0.00403118505118914\\
103	0.00403121680083304\\
104	0.00403124909082514\\
105	0.00403128193054971\\
106	0.00403131532955316\\
107	0.00403134929754687\\
108	0.00403138384440994\\
109	0.00403141898019218\\
110	0.00403145471511682\\
111	0.00403149105958357\\
112	0.0040315280241716\\
113	0.00403156561964252\\
114	0.00403160385694349\\
115	0.0040316427472103\\
116	0.00403168230177054\\
117	0.00403172253214682\\
118	0.00403176345005995\\
119	0.00403180506743214\\
120	0.00403184739639037\\
121	0.00403189044926957\\
122	0.00403193423861598\\
123	0.00403197877719031\\
124	0.00403202407797098\\
125	0.00403207015415745\\
126	0.00403211701917317\\
127	0.00403216468666858\\
128	0.00403221317052421\\
129	0.00403226248485323\\
130	0.00403231264400393\\
131	0.00403236366256223\\
132	0.00403241555535348\\
133	0.00403246833744424\\
134	0.00403252202414366\\
135	0.00403257663100445\\
136	0.00403263217382357\\
137	0.00403268866864269\\
138	0.00403274613174887\\
139	0.00403280457967544\\
140	0.00403286402920493\\
141	0.00403292449737621\\
142	0.00403298600150241\\
143	0.00403304855921572\\
144	0.00403311218858156\\
145	0.00403317690839188\\
146	0.00403324273891722\\
147	0.00403330970375947\\
148	0.0040333778337388\\
149	0.004033447168939\\
150	0.00403351773105495\\
151	0.00403358954217357\\
152	0.00403366262478084\\
153	0.00403373700176913\\
154	0.0040338126964444\\
155	0.00403388973253367\\
156	0.0040339681341927\\
157	0.00403404792601353\\
158	0.00403412913303239\\
159	0.00403421178073775\\
160	0.00403429589507838\\
161	0.0040343815024716\\
162	0.00403446862981177\\
163	0.0040345573044788\\
164	0.00403464755434689\\
165	0.00403473940779341\\
166	0.0040348328937079\\
167	0.0040349280415013\\
168	0.00403502488111525\\
169	0.00403512344303169\\
170	0.00403522375828242\\
171	0.0040353258584591\\
172	0.0040354297757232\\
173	0.00403553554281623\\
174	0.00403564319307008\\
175	0.00403575276041768\\
176	0.00403586427940372\\
177	0.00403597778519553\\
178	0.00403609331359445\\
179	0.00403621090104681\\
180	0.00403633058465587\\
181	0.00403645240219323\\
182	0.00403657639211106\\
183	0.00403670259355399\\
184	0.0040368310463717\\
185	0.00403696179113145\\
186	0.00403709486913082\\
187	0.00403723032241092\\
188	0.00403736819376953\\
189	0.00403750852677466\\
190	0.00403765136577822\\
191	0.00403779675593019\\
192	0.00403794474319268\\
193	0.00403809537435453\\
194	0.00403824869704588\\
195	0.00403840475975344\\
196	0.00403856361183554\\
197	0.00403872530353779\\
198	0.00403888988600878\\
199	0.00403905741131628\\
200	0.00403922793246357\\
201	0.00403940150340611\\
202	0.00403957817906847\\
203	0.00403975801536169\\
204	0.00403994106920072\\
205	0.00404012739852243\\
206	0.00404031706230372\\
207	0.00404051012058006\\
208	0.00404070663446437\\
209	0.00404090666616626\\
210	0.00404111027901159\\
211	0.00404131753746227\\
212	0.00404152850713655\\
213	0.00404174325482977\\
214	0.00404196184853525\\
215	0.00404218435746575\\
216	0.00404241085207537\\
217	0.00404264140408154\\
218	0.0040428760864879\\
219	0.00404311497360726\\
220	0.00404335814108511\\
221	0.00404360566592364\\
222	0.00404385762650622\\
223	0.00404411410262221\\
224	0.00404437517549263\\
225	0.00404464092779601\\
226	0.00404491144369495\\
227	0.00404518680886314\\
228	0.00404546711051317\\
229	0.00404575243742481\\
230	0.00404604287997398\\
231	0.00404633853016232\\
232	0.00404663948164759\\
233	0.00404694582977477\\
234	0.00404725767160795\\
235	0.00404757510596315\\
236	0.00404789823344184\\
237	0.00404822715646565\\
238	0.00404856197931195\\
239	0.00404890280815061\\
240	0.00404924975108192\\
241	0.00404960291817575\\
242	0.0040499624215123\\
243	0.00405032837522407\\
244	0.00405070089553996\\
245	0.00405108010083074\\
246	0.00405146611165699\\
247	0.00405185905081918\\
248	0.00405225904341034\\
249	0.00405266621687171\\
250	0.00405308070105178\\
251	0.00405350262826892\\
252	0.00405393213337873\\
253	0.00405436935384623\\
254	0.004054814429824\\
255	0.00405526750423721\\
256	0.00405572872287694\\
257	0.00405619823450304\\
258	0.00405667619095848\\
259	0.00405716274729766\\
260	0.00405765806193113\\
261	0.00405816229679059\\
262	0.00405867561751827\\
263	0.00405919819368625\\
264	0.00405973019905237\\
265	0.00406027181186149\\
266	0.0040608232152023\\
267	0.00406138459743367\\
268	0.00406195615269679\\
269	0.0040625380815348\\
270	0.00406313059164622\\
271	0.00406373389880394\\
272	0.00406434822797799\\
273	0.00406497381469557\\
274	0.00406561090663889\\
275	0.00406625976532493\\
276	0.00406692066708692\\
277	0.0040675939000439\\
278	0.00406827974262886\\
279	0.00406897835323118\\
280	0.00406968914749888\\
281	0.00407041227445104\\
282	0.00407114793169407\\
283	0.0040718963188144\\
284	0.00407265763728611\\
285	0.0040734320903632\\
286	0.00407421988295385\\
287	0.00407502122147461\\
288	0.00407583631368072\\
289	0.00407666536846943\\
290	0.00407750859565094\\
291	0.00407836620568241\\
292	0.00407923840935802\\
293	0.00408012541744706\\
294	0.00408102744027068\\
295	0.00408194468720526\\
296	0.00408287736609797\\
297	0.00408382568257669\\
298	0.0040847898392321\\
299	0.00408577003464479\\
300	0.00408676646222342\\
301	0.00408777930881153\\
302	0.00408880875300992\\
303	0.00408985496314912\\
304	0.0040909180948289\\
305	0.00409199828792148\\
306	0.00409309566291238\\
307	0.00409421031642586\\
308	0.00409534231576585\\
309	0.00409649169232576\\
310	0.00409765843387609\\
311	0.00409884247631765\\
312	0.00410004369736373\\
313	0.00410126192051685\\
314	0.00410249695646168\\
315	0.0041037487710781\\
316	0.00410501809269451\\
317	0.00410630872083941\\
318	0.00410762228403794\\
319	0.00410895917827685\\
320	0.00411031980594528\\
321	0.00411170457592415\\
322	0.00411311390367631\\
323	0.0041145482113375\\
324	0.00411600792780806\\
325	0.00411749348884517\\
326	0.00411900533715594\\
327	0.00412054392249108\\
328	0.00412210970173916\\
329	0.00412370313902142\\
330	0.00412532470578721\\
331	0.00412697488090991\\
332	0.00412865415078315\\
333	0.00413036300941769\\
334	0.00413210195853834\\
335	0.00413387150768126\\
336	0.00413567217429118\\
337	0.00413750448381872\\
338	0.00413936896981707\\
339	0.00414126617403825\\
340	0.00414319664652806\\
341	0.004145160945719\\
342	0.00414715963852069\\
343	0.0041491933004056\\
344	0.00415126251548892\\
345	0.0041533678765995\\
346	0.00415550998533789\\
347	0.00415768945211674\\
348	0.00415990689617547\\
349	0.00416216294555975\\
350	0.00416445823705093\\
351	0.00416679341602682\\
352	0.00416916913622703\\
353	0.00417158605938683\\
354	0.00417404485469147\\
355	0.00417654619798789\\
356	0.00417909077067757\\
357	0.00418167925822084\\
358	0.00418431234829119\\
359	0.00418699072918654\\
360	0.00418971509198084\\
361	0.00419248615450633\\
362	0.00419530468493859\\
363	0.00419817146485468\\
364	0.00420108728974278\\
365	0.00420405296957705\\
366	0.00420706932946481\\
367	0.00421013721038834\\
368	0.00421325747015759\\
369	0.00421643098468567\\
370	0.00421965864940848\\
371	0.00422294138104568\\
372	0.00422628011982449\\
373	0.00422967583232177\\
374	0.00423312951512707\\
375	0.00423664219958411\\
376	0.00424021495790487\\
377	0.00424384891089465\\
378	0.00424754523712741\\
379	0.00425130518182945\\
380	0.00425513005820619\\
381	0.00425902121396172\\
382	0.0042629798657054\\
383	0.0042670071798537\\
384	0.00427110434913889\\
385	0.00427527259430024\\
386	0.00427951316601193\\
387	0.00428382734708664\\
388	0.00428821645499868\\
389	0.00429268184477737\\
390	0.00429722491232846\\
391	0.00430184709825241\\
392	0.00430654989223478\\
393	0.00431133483809401\\
394	0.00431620353957894\\
395	0.00432115766701866\\
396	0.00432619896493554\\
397	0.00433132926073578\\
398	0.00433655047459088\\
399	0.00434186463061658\\
400	0.00434727386943476\\
401	0.00435278046216777\\
402	0.00435838682584821\\
403	0.00436409554012939\\
404	0.00436990936502242\\
405	0.0043758312591468\\
406	0.00438186439766819\\
407	0.00438801218860048\\
408	0.00439427828525927\\
409	0.00440066659110231\\
410	0.0044071812515267\\
411	0.00441382662958582\\
412	0.00442060725501341\\
413	0.00442752773369903\\
414	0.00443459259727504\\
415	0.00444180604812211\\
416	0.00444917192835531\\
417	0.00445669404834207\\
418	0.00446437634750615\\
419	0.00447222290003713\\
420	0.00448023792066377\\
421	0.00448842577068774\\
422	0.00449679096427237\\
423	0.00450533817498008\\
424	0.00451407224216589\\
425	0.00452299817835337\\
426	0.00453212118052739\\
427	0.0045414466429674\\
428	0.00455098016755669\\
429	0.00456072757490198\\
430	0.00457069491622738\\
431	0.00458088848628549\\
432	0.00459131483644244\\
433	0.00460198078562155\\
434	0.00461289343244344\\
435	0.00462406017275493\\
436	0.00463548871752189\\
437	0.00464718710969808\\
438	0.00465916374184674\\
439	0.00467142737449136\\
440	0.00468398715518755\\
441	0.00469685263836591\\
442	0.00471003380649355\\
443	0.00472354109554875\\
444	0.00473738543486821\\
445	0.00475157824689911\\
446	0.00476613143619902\\
447	0.00478105743071025\\
448	0.00479636924656813\\
449	0.00481208052910316\\
450	0.00482820559756616\\
451	0.00484475949353333\\
452	0.00486175803227568\\
453	0.00487921785586886\\
454	0.00489715649787599\\
455	0.00491559266873435\\
456	0.00493454612292579\\
457	0.0049540373357115\\
458	0.00497408748771998\\
459	0.00499471871224096\\
460	0.00501595407012546\\
461	0.00503781727833482\\
462	0.00506033258423081\\
463	0.00508352468522577\\
464	0.00510741842122384\\
465	0.00513204033536679\\
466	0.00515741948366903\\
467	0.00518358660865367\\
468	0.00521056816764443\\
469	0.00523839073483765\\
470	0.00526708158535866\\
471	0.00529666956048661\\
472	0.00532718623681849\\
473	0.00535866769970322\\
474	0.00539116649826347\\
475	0.00542474181628269\\
476	0.00545945169037871\\
477	0.00549535093760036\\
478	0.00553248836020859\\
479	0.00557090284134059\\
480	0.00561061805473222\\
481	0.00565163534862653\\
482	0.00569392423879731\\
483	0.00573740976427773\\
484	0.00578195573058062\\
485	0.00582734256557108\\
486	0.00587302073186874\\
487	0.00591777929369159\\
488	0.00596147832584702\\
489	0.00600396924585436\\
490	0.00604509677476079\\
491	0.00608470203551141\\
492	0.00612262728640956\\
493	0.00615872287684319\\
494	0.0061928572469614\\
495	0.00622493098413339\\
496	0.00625489643300038\\
497	0.0062827844896127\\
498	0.0063089053613691\\
499	0.00633450440142987\\
500	0.00635957631827112\\
501	0.00638412492653279\\
502	0.00640816477749661\\
503	0.00643172275774873\\
504	0.00645483951784793\\
505	0.0064775705149099\\
506	0.00649998635097495\\
507	0.00652217194101431\\
508	0.00654422384304134\\
509	0.0065662448150298\\
510	0.00658831692692719\\
511	0.00661047543979157\\
512	0.00663274081961346\\
513	0.00665513543500258\\
514	0.00667768320198779\\
515	0.0067004090917962\\
516	0.00672333849461296\\
517	0.00674649644599174\\
518	0.00676990674623638\\
519	0.00679359104100118\\
520	0.00681756798909042\\
521	0.00684185273053656\\
522	0.00686645862083711\\
523	0.00689139898441964\\
524	0.00691668742247002\\
525	0.00694233775045284\\
526	0.0069683639461627\\
527	0.00699478011683277\\
528	0.00702160049519244\\
529	0.00704883947499101\\
530	0.0070765116956076\\
531	0.007104632181729\\
532	0.00713321653579037\\
533	0.00716228113862316\\
534	0.00719184326231583\\
535	0.00722192112766865\\
536	0.00725253396467986\\
537	0.00728370207453166\\
538	0.00731544689055641\\
539	0.00734779103435149\\
540	0.00738075836152634\\
541	0.00741437398947174\\
542	0.00744866429705612\\
543	0.0074836568833871\\
544	0.00751938046982297\\
545	0.00755586473097468\\
546	0.0075931400381968\\
547	0.00763123709119732\\
548	0.00767018640119527\\
549	0.00771001757935446\\
550	0.00775075870022047\\
551	0.00779242478641457\\
552	0.00783502989694319\\
553	0.00787858722644379\\
554	0.00792310886900782\\
555	0.00796860537723829\\
556	0.00801508542259554\\
557	0.00806255540390485\\
558	0.00811101877401344\\
559	0.00816046466610628\\
560	0.00821089105072984\\
561	0.00826228888233602\\
562	0.00831463976241251\\
563	0.00836783969989453\\
564	0.00842192152680974\\
565	0.0084769504865854\\
566	0.00853302421756685\\
567	0.0085903217031176\\
568	0.00864905932111027\\
569	0.00870766413426631\\
570	0.00876495225119506\\
571	0.00882046728958851\\
572	0.00887398586474071\\
573	0.00892717754532439\\
574	0.00897988555572072\\
575	0.00903197458790182\\
576	0.00908265151027188\\
577	0.00913183644158302\\
578	0.00917988739287592\\
579	0.00922711699783626\\
580	0.00927343806398176\\
581	0.00931913102044634\\
582	0.0093645049608711\\
583	0.00940961130043772\\
584	0.0094544183622864\\
585	0.00949887800402646\\
586	0.00954293238643295\\
587	0.00958652004546675\\
588	0.00962957886512476\\
589	0.00967204874598804\\
590	0.00971387347495566\\
591	0.00975500239981678\\
592	0.00979538909575469\\
593	0.00983498278816331\\
594	0.00987369853931868\\
595	0.00991118387968948\\
596	0.00994658651044256\\
597	0.00997788999445116\\
598	0.010000292044645\\
599	0\\
600	0\\
};
\addplot [color=blue!40!mycolor9,solid,forget plot]
  table[row sep=crcr]{%
1	0.00381944488791505\\
2	0.00381945382964767\\
3	0.00381946292077991\\
4	0.00381947216387073\\
5	0.00381948156152383\\
6	0.00381949111638838\\
7	0.00381950083115981\\
8	0.00381951070858063\\
9	0.00381952075144128\\
10	0.00381953096258094\\
11	0.00381954134488834\\
12	0.00381955190130264\\
13	0.00381956263481437\\
14	0.00381957354846626\\
15	0.00381958464535422\\
16	0.00381959592862812\\
17	0.00381960740149288\\
18	0.00381961906720937\\
19	0.00381963092909535\\
20	0.00381964299052653\\
21	0.00381965525493748\\
22	0.00381966772582277\\
23	0.00381968040673791\\
24	0.00381969330130045\\
25	0.0038197064131911\\
26	0.00381971974615478\\
27	0.00381973330400171\\
28	0.0038197470906086\\
29	0.0038197611099198\\
30	0.00381977536594841\\
31	0.0038197898627776\\
32	0.00381980460456168\\
33	0.00381981959552742\\
34	0.00381983483997533\\
35	0.00381985034228096\\
36	0.00381986610689601\\
37	0.00381988213834992\\
38	0.00381989844125108\\
39	0.00381991502028821\\
40	0.00381993188023176\\
41	0.0038199490259354\\
42	0.00381996646233736\\
43	0.00381998419446203\\
44	0.00382000222742135\\
45	0.00382002056641638\\
46	0.0038200392167389\\
47	0.00382005818377291\\
48	0.00382007747299627\\
49	0.0038200970899824\\
50	0.00382011704040181\\
51	0.00382013733002396\\
52	0.00382015796471891\\
53	0.00382017895045909\\
54	0.00382020029332103\\
55	0.00382022199948732\\
56	0.00382024407524837\\
57	0.00382026652700434\\
58	0.00382028936126701\\
59	0.00382031258466178\\
60	0.00382033620392975\\
61	0.00382036022592958\\
62	0.00382038465763972\\
63	0.00382040950616044\\
64	0.00382043477871595\\
65	0.00382046048265671\\
66	0.0038204866254615\\
67	0.00382051321473978\\
68	0.00382054025823403\\
69	0.00382056776382201\\
70	0.00382059573951924\\
71	0.00382062419348138\\
72	0.00382065313400674\\
73	0.00382068256953883\\
74	0.00382071250866895\\
75	0.00382074296013872\\
76	0.00382077393284292\\
77	0.00382080543583205\\
78	0.00382083747831521\\
79	0.00382087006966293\\
80	0.00382090321940996\\
81	0.00382093693725835\\
82	0.00382097123308031\\
83	0.00382100611692133\\
84	0.00382104159900323\\
85	0.00382107768972738\\
86	0.00382111439967794\\
87	0.00382115173962502\\
88	0.00382118972052815\\
89	0.00382122835353958\\
90	0.00382126765000783\\
91	0.00382130762148123\\
92	0.00382134827971144\\
93	0.00382138963665715\\
94	0.0038214317044878\\
95	0.00382147449558743\\
96	0.00382151802255859\\
97	0.00382156229822609\\
98	0.00382160733564117\\
99	0.00382165314808561\\
100	0.00382169974907573\\
101	0.00382174715236684\\
102	0.00382179537195737\\
103	0.00382184442209338\\
104	0.00382189431727298\\
105	0.00382194507225083\\
106	0.00382199670204285\\
107	0.00382204922193089\\
108	0.00382210264746757\\
109	0.00382215699448107\\
110	0.00382221227908013\\
111	0.00382226851765912\\
112	0.00382232572690309\\
113	0.00382238392379307\\
114	0.00382244312561133\\
115	0.00382250334994673\\
116	0.00382256461470026\\
117	0.00382262693809057\\
118	0.00382269033865958\\
119	0.00382275483527822\\
120	0.00382282044715228\\
121	0.00382288719382837\\
122	0.00382295509519975\\
123	0.00382302417151253\\
124	0.00382309444337185\\
125	0.00382316593174796\\
126	0.0038232386579827\\
127	0.00382331264379583\\
128	0.00382338791129142\\
129	0.00382346448296455\\
130	0.0038235423817078\\
131	0.00382362163081799\\
132	0.00382370225400282\\
133	0.00382378427538783\\
134	0.00382386771952312\\
135	0.00382395261139026\\
136	0.00382403897640941\\
137	0.00382412684044632\\
138	0.00382421622981948\\
139	0.00382430717130773\\
140	0.00382439969215842\\
141	0.00382449382009734\\
142	0.00382458958334283\\
143	0.00382468701063115\\
144	0.00382478613126695\\
145	0.00382488697523013\\
146	0.00382498957338099\\
147	0.00382509395772975\\
148	0.00382520016121877\\
149	0.00382530821570657\\
150	0.00382541815360922\\
151	0.00382553000791004\\
152	0.00382564381216948\\
153	0.00382575960053517\\
154	0.00382587740775211\\
155	0.003825997269173\\
156	0.00382611922076889\\
157	0.00382624329913982\\
158	0.00382636954152591\\
159	0.00382649798581826\\
160	0.00382662867057043\\
161	0.00382676163500982\\
162	0.00382689691904938\\
163	0.00382703456329954\\
164	0.00382717460908022\\
165	0.00382731709843319\\
166	0.0038274620741345\\
167	0.00382760957970729\\
168	0.00382775965943458\\
169	0.00382791235837252\\
170	0.00382806772236377\\
171	0.00382822579805092\\
172	0.00382838663289046\\
173	0.00382855027516681\\
174	0.00382871677400655\\
175	0.00382888617939297\\
176	0.0038290585421808\\
177	0.00382923391411128\\
178	0.00382941234782737\\
179	0.00382959389688925\\
180	0.00382977861579013\\
181	0.00382996655997224\\
182	0.00383015778584317\\
183	0.00383035235079235\\
184	0.00383055031320795\\
185	0.00383075173249401\\
186	0.00383095666908772\\
187	0.00383116518447723\\
188	0.00383137734121947\\
189	0.00383159320295851\\
190	0.00383181283444411\\
191	0.00383203630155047\\
192	0.00383226367129549\\
193	0.00383249501186013\\
194	0.00383273039260829\\
195	0.00383296988410682\\
196	0.00383321355814593\\
197	0.00383346148775993\\
198	0.00383371374724835\\
199	0.00383397041219723\\
200	0.00383423155950099\\
201	0.00383449726738433\\
202	0.00383476761542474\\
203	0.00383504268457528\\
204	0.00383532255718768\\
205	0.00383560731703575\\
206	0.00383589704933934\\
207	0.00383619184078845\\
208	0.00383649177956781\\
209	0.00383679695538181\\
210	0.00383710745947986\\
211	0.00383742338468209\\
212	0.00383774482540548\\
213	0.00383807187769018\\
214	0.00383840463922657\\
215	0.00383874320938237\\
216	0.00383908768923041\\
217	0.00383943818157668\\
218	0.00383979479098878\\
219	0.00384015762382479\\
220	0.00384052678826269\\
221	0.00384090239432995\\
222	0.00384128455393374\\
223	0.00384167338089152\\
224	0.00384206899096197\\
225	0.00384247150187654\\
226	0.00384288103337115\\
227	0.00384329770721868\\
228	0.00384372164726155\\
229	0.00384415297944502\\
230	0.00384459183185069\\
231	0.00384503833473076\\
232	0.00384549262054244\\
233	0.00384595482398295\\
234	0.003846425082025\\
235	0.0038469035339526\\
236	0.00384739032139756\\
237	0.00384788558837617\\
238	0.00384838948132651\\
239	0.00384890214914628\\
240	0.00384942374323097\\
241	0.0038499544175126\\
242	0.00385049432849878\\
243	0.00385104363531254\\
244	0.00385160249973235\\
245	0.00385217108623283\\
246	0.00385274956202587\\
247	0.00385333809710236\\
248	0.00385393686427437\\
249	0.00385454603921786\\
250	0.00385516580051607\\
251	0.00385579632970351\\
252	0.00385643781131035\\
253	0.00385709043290793\\
254	0.00385775438515451\\
255	0.00385842986184218\\
256	0.00385911705994454\\
257	0.00385981617966518\\
258	0.00386052742448745\\
259	0.00386125100122527\\
260	0.00386198712007523\\
261	0.00386273599467023\\
262	0.00386349784213462\\
263	0.00386427288314106\\
264	0.00386506134196932\\
265	0.00386586344656686\\
266	0.00386667942861144\\
267	0.00386750952357506\\
268	0.00386835397078955\\
269	0.0038692130135119\\
270	0.00387008689898778\\
271	0.00387097587850831\\
272	0.00387188020744911\\
273	0.00387280014526215\\
274	0.00387373595533601\\
275	0.00387468790447146\\
276	0.0038756562611982\\
277	0.00387664129061343\\
278	0.00387764323968271\\
279	0.00387866230829795\\
280	0.0038796987741781\\
281	0.00388075292388058\\
282	0.00388182504825039\\
283	0.00388291544246592\\
284	0.00388402440608404\\
285	0.00388515224308433\\
286	0.0038862992619126\\
287	0.0038874657755231\\
288	0.0038886521014196\\
289	0.00388985856169444\\
290	0.00389108548306613\\
291	0.00389233319691414\\
292	0.00389360203931105\\
293	0.00389489235105145\\
294	0.00389620447767661\\
295	0.00389753876949465\\
296	0.0038988955815951\\
297	0.00390027527385714\\
298	0.0039016782109505\\
299	0.00390310476232804\\
300	0.00390455530220914\\
301	0.00390603020955305\\
302	0.00390752986802195\\
303	0.00390905466593376\\
304	0.00391060499620635\\
305	0.00391218125629692\\
306	0.0039137838481452\\
307	0.00391541317813866\\
308	0.00391706965714266\\
309	0.0039187537006957\\
310	0.00392046572962547\\
311	0.00392220617174694\\
312	0.00392397546638137\\
313	0.00392577407619996\\
314	0.00392760251734056\\
315	0.00392946142885158\\
316	0.00393135167357656\\
317	0.00393327387039218\\
318	0.00393522852295292\\
319	0.00393721614132281\\
320	0.00393923724199673\\
321	0.00394129234791969\\
322	0.00394338198850431\\
323	0.00394550669964648\\
324	0.00394766702373917\\
325	0.00394986350968491\\
326	0.00395209671290709\\
327	0.00395436719536063\\
328	0.00395667552554228\\
329	0.00395902227850162\\
330	0.00396140803585341\\
331	0.00396383338579263\\
332	0.0039662989231134\\
333	0.00396880524923413\\
334	0.00397135297223093\\
335	0.00397394270688251\\
336	0.00397657507473022\\
337	0.00397925070415858\\
338	0.00398197023050168\\
339	0.00398473429618398\\
340	0.00398754355090492\\
341	0.00399039865187992\\
342	0.00399330026415337\\
343	0.00399624906100389\\
344	0.00399924572446666\\
345	0.00400229094600521\\
346	0.00400538542737336\\
347	0.00400852988171844\\
348	0.0040117250349921\\
349	0.00401497162775132\\
350	0.00401827041745474\\
351	0.00402162218138518\\
352	0.00402502772035511\\
353	0.00402848786336176\\
354	0.00403200347329119\\
355	0.00403557545339288\\
356	0.00403920475267724\\
357	0.00404289236201626\\
358	0.00404663926532477\\
359	0.00405044617696116\\
360	0.00405431209755343\\
361	0.00405823454676541\\
362	0.00406221390273528\\
363	0.00406625050416226\\
364	0.00407034464456845\\
365	0.00407449656600271\\
366	0.00407870645169283\\
367	0.00408297441481805\\
368	0.00408730047692172\\
369	0.00409168456088001\\
370	0.00409612647725224\\
371	0.00410062590479673\\
372	0.00410518236700039\\
373	0.00410979520407372\\
374	0.00411446354037165\\
375	0.00411918624930451\\
376	0.00412396192469406\\
377	0.00412878888933131\\
378	0.00413366534096513\\
379	0.00413858996594277\\
380	0.00414356417617088\\
381	0.00414860064463897\\
382	0.00415370519126221\\
383	0.00415887782354044\\
384	0.00416411846490067\\
385	0.00416942694647098\\
386	0.00417480299821334\\
387	0.00418024623940122\\
388	0.00418575616844292\\
389	0.00419133215207308\\
390	0.00419697341396355\\
391	0.00420267902284577\\
392	0.00420844788029291\\
393	0.00421427870838504\\
394	0.00422017003758266\\
395	0.00422612019527058\\
396	0.00423212729561477\\
397	0.00423818923161663\\
398	0.00424430367056677\\
399	0.00425046805452244\\
400	0.00425667960798663\\
401	0.00426293535569741\\
402	0.00426923215439836\\
403	0.00427556674372872\\
404	0.00428193582305656\\
405	0.00428833616336053\\
406	0.00429476476652256\\
407	0.00430121908978586\\
408	0.00430769736567081\\
409	0.00431419908071125\\
410	0.00432072567623176\\
411	0.00432728118476319\\
412	0.00433387326282209\\
413	0.00434051456486937\\
414	0.00434722456676876\\
415	0.00435403197492565\\
416	0.00436096030699787\\
417	0.0043680178596261\\
418	0.00437520729992339\\
419	0.00438253139700803\\
420	0.00438999303583359\\
421	0.00439759523439973\\
422	0.0044053411642879\\
423	0.00441323417190158\\
424	0.00442127778875179\\
425	0.00442947568622172\\
426	0.00443783142315854\\
427	0.00444634839223422\\
428	0.00445503009046111\\
429	0.00446388012457871\\
430	0.00447290221682272\\
431	0.00448210021101293\\
432	0.0044914780788647\\
433	0.00450103992690116\\
434	0.00451079000428179\\
435	0.00452073271115161\\
436	0.00453087260748145\\
437	0.00454121442262168\\
438	0.00455176306564887\\
439	0.00456252363660099\\
440	0.00457350143872782\\
441	0.00458470199196287\\
442	0.00459613104799112\\
443	0.00460779460714568\\
444	0.00461969893332353\\
445	0.00463185057046742\\
446	0.00464425636526101\\
447	0.00465692349307205\\
448	0.00466985948416286\\
449	0.00468307225273333\\
450	0.00469657012804901\\
451	0.00471036188452164\\
452	0.0047244567769889\\
453	0.00473886458248146\\
454	0.00475359565575727\\
455	0.0047686609505671\\
456	0.00478407203480342\\
457	0.00479984113684248\\
458	0.00481598120812612\\
459	0.00483250597144564\\
460	0.00484943004417186\\
461	0.00486676903423521\\
462	0.00488453962375556\\
463	0.00490275965605901\\
464	0.00492144834571443\\
465	0.00494062633330873\\
466	0.0049603155092367\\
467	0.00498053854270406\\
468	0.00500131922287815\\
469	0.005022682272008\\
470	0.00504465297027654\\
471	0.00506725646762278\\
472	0.00509051803742607\\
473	0.00511446380997644\\
474	0.00513912196386034\\
475	0.00516452336427147\\
476	0.00519070172399126\\
477	0.00521768674120825\\
478	0.00524550790757517\\
479	0.00527419459831302\\
480	0.00530377649605227\\
481	0.00533428372592657\\
482	0.00536574715474855\\
483	0.00539819893647581\\
484	0.00543167340888997\\
485	0.00546620850085091\\
486	0.00550185097014559\\
487	0.00553866543163327\\
488	0.00557671540533525\\
489	0.00561606113875089\\
490	0.00565675655137328\\
491	0.00569884505698706\\
492	0.00574235390404411\\
493	0.00578728657121308\\
494	0.00583361260813407\\
495	0.00588125412523459\\
496	0.00593006789083969\\
497	0.00597982167539138\\
498	0.00603000261139891\\
499	0.00607919001400096\\
500	0.0061272311955095\\
501	0.00617396403990961\\
502	0.00621921947190392\\
503	0.00626282523322078\\
504	0.00630461149639069\\
505	0.00634441913381876\\
506	0.00638211144982614\\
507	0.00641759075241061\\
508	0.00645082139107356\\
509	0.00648186133141168\\
510	0.00651137009306334\\
511	0.00654034185867726\\
512	0.00656877382473748\\
513	0.00659667381570486\\
514	0.00662406212808724\\
515	0.00665097331469975\\
516	0.00667745772009153\\
517	0.00670358249291407\\
518	0.00672943165361346\\
519	0.00675510460981876\\
520	0.00678071227161439\\
521	0.00680636956803016\\
522	0.0068321418354832\\
523	0.00685805883381921\\
524	0.00688414616269528\\
525	0.00691043166666499\\
526	0.0069369449982943\\
527	0.00696371702157117\\
528	0.006990779052795\\
529	0.00701816195599068\\
530	0.00704589514320308\\
531	0.00707400558272338\\
532	0.007102517000605\\
533	0.00713145022215655\\
534	0.00716082564304089\\
535	0.00719066452125459\\
536	0.00722098892233578\\
537	0.00725182166615787\\
538	0.00728318628224142\\
539	0.00731510698194115\\
540	0.00734760865663381\\
541	0.00738071691038189\\
542	0.00741445813238253\\
543	0.0074488596067575\\
544	0.0074839496420057\\
545	0.00751975758741009\\
546	0.00755631369103536\\
547	0.00759364885472858\\
548	0.00763179428830673\\
549	0.00767078103297187\\
550	0.00771063930619072\\
551	0.00775139804237961\\
552	0.00779307363174633\\
553	0.00783568124693249\\
554	0.00787923540543176\\
555	0.00792374989807655\\
556	0.00796923763533189\\
557	0.00801571046336702\\
558	0.00806317895231008\\
559	0.00811165226865771\\
560	0.00816113770274232\\
561	0.00821164035063354\\
562	0.00826316276313242\\
563	0.00831570535723852\\
564	0.0083692583124851\\
565	0.00842380278768651\\
566	0.00847932001064811\\
567	0.00853574542202786\\
568	0.00859302797587814\\
569	0.00865122838399295\\
570	0.00871044661767393\\
571	0.00877081641557857\\
572	0.00883206177374144\\
573	0.00889210788623348\\
574	0.00895060193483743\\
575	0.0090069743480842\\
576	0.00906164998966982\\
577	0.00911569254926328\\
578	0.00916848694084165\\
579	0.00921948532352991\\
580	0.00926878734421818\\
581	0.00931668587461442\\
582	0.00936337646786582\\
583	0.00940899867258259\\
584	0.00945409509992045\\
585	0.00949869896494848\\
586	0.00954283083080037\\
587	0.00958646243837779\\
588	0.00962954789259008\\
589	0.0096720338085118\\
590	0.00971386797226699\\
591	0.00975500106378656\\
592	0.00979538897701793\\
593	0.00983498278816331\\
594	0.00987369853931868\\
595	0.00991118387968948\\
596	0.00994658651044256\\
597	0.00997788999445116\\
598	0.010000292044645\\
599	0\\
600	0\\
};
\addplot [color=blue!75!mycolor7,solid,forget plot]
  table[row sep=crcr]{%
1	0.00303384705345048\\
2	0.003033865512802\\
3	0.00303388428662202\\
4	0.00303390338031515\\
5	0.00303392279937922\\
6	0.00303394254940679\\
7	0.00303396263608701\\
8	0.00303398306520715\\
9	0.00303400384265426\\
10	0.0030340249744169\\
11	0.00303404646658698\\
12	0.00303406832536142\\
13	0.00303409055704408\\
14	0.00303411316804746\\
15	0.00303413616489464\\
16	0.0030341595542212\\
17	0.00303418334277715\\
18	0.00303420753742885\\
19	0.00303423214516106\\
20	0.00303425717307896\\
21	0.00303428262841024\\
22	0.0030343085185072\\
23	0.00303433485084882\\
24	0.00303436163304314\\
25	0.00303438887282922\\
26	0.00303441657807957\\
27	0.00303444475680241\\
28	0.00303447341714398\\
29	0.00303450256739096\\
30	0.00303453221597281\\
31	0.00303456237146433\\
32	0.00303459304258804\\
33	0.00303462423821688\\
34	0.00303465596737664\\
35	0.00303468823924865\\
36	0.00303472106317255\\
37	0.00303475444864891\\
38	0.00303478840534193\\
39	0.0030348229430825\\
40	0.00303485807187084\\
41	0.00303489380187951\\
42	0.0030349301434564\\
43	0.00303496710712772\\
44	0.00303500470360105\\
45	0.00303504294376846\\
46	0.0030350818387098\\
47	0.00303512139969574\\
48	0.00303516163819128\\
49	0.00303520256585894\\
50	0.00303524419456222\\
51	0.00303528653636906\\
52	0.00303532960355534\\
53	0.00303537340860847\\
54	0.00303541796423115\\
55	0.00303546328334487\\
56	0.00303550937909384\\
57	0.00303555626484877\\
58	0.00303560395421081\\
59	0.00303565246101549\\
60	0.00303570179933677\\
61	0.00303575198349118\\
62	0.00303580302804193\\
63	0.00303585494780319\\
64	0.00303590775784455\\
65	0.00303596147349518\\
66	0.00303601611034853\\
67	0.00303607168426674\\
68	0.00303612821138534\\
69	0.00303618570811794\\
70	0.00303624419116103\\
71	0.00303630367749889\\
72	0.00303636418440846\\
73	0.00303642572946448\\
74	0.00303648833054459\\
75	0.00303655200583452\\
76	0.00303661677383344\\
77	0.00303668265335929\\
78	0.00303674966355437\\
79	0.00303681782389086\\
80	0.0030368871541765\\
81	0.00303695767456038\\
82	0.00303702940553884\\
83	0.00303710236796138\\
84	0.00303717658303682\\
85	0.0030372520723394\\
86	0.00303732885781506\\
87	0.00303740696178794\\
88	0.00303748640696674\\
89	0.00303756721645135\\
90	0.00303764941373972\\
91	0.00303773302273442\\
92	0.00303781806774981\\
93	0.00303790457351899\\
94	0.00303799256520105\\
95	0.00303808206838826\\
96	0.00303817310911356\\
97	0.00303826571385812\\
98	0.00303835990955895\\
99	0.00303845572361669\\
100	0.00303855318390362\\
101	0.0030386523187716\\
102	0.00303875315706035\\
103	0.00303885572810569\\
104	0.00303896006174804\\
105	0.00303906618834104\\
106	0.00303917413876022\\
107	0.00303928394441195\\
108	0.0030393956372424\\
109	0.00303950924974678\\
110	0.00303962481497866\\
111	0.00303974236655932\\
112	0.00303986193868759\\
113	0.00303998356614951\\
114	0.0030401072843283\\
115	0.00304023312921447\\
116	0.00304036113741615\\
117	0.00304049134616944\\
118	0.00304062379334912\\
119	0.00304075851747948\\
120	0.00304089555774519\\
121	0.00304103495400245\\
122	0.00304117674679044\\
123	0.00304132097734277\\
124	0.00304146768759921\\
125	0.00304161692021761\\
126	0.00304176871858596\\
127	0.0030419231268348\\
128	0.00304208018984966\\
129	0.00304223995328376\\
130	0.00304240246357105\\
131	0.00304256776793924\\
132	0.00304273591442327\\
133	0.0030429069518788\\
134	0.00304308092999609\\
135	0.00304325789931412\\
136	0.00304343791123465\\
137	0.00304362101803702\\
138	0.00304380727289281\\
139	0.00304399672988108\\
140	0.00304418944400408\\
141	0.00304438547120398\\
142	0.00304458486838166\\
143	0.00304478769341881\\
144	0.00304499400520721\\
145	0.00304520386368493\\
146	0.00304541732986664\\
147	0.00304563446581277\\
148	0.00304585533449473\\
149	0.00304607999995983\\
150	0.00304630852734943\\
151	0.00304654098291754\\
152	0.00304677743404971\\
153	0.00304701794928214\\
154	0.00304726259832124\\
155	0.00304751145206357\\
156	0.00304776458261577\\
157	0.00304802206331531\\
158	0.00304828396875115\\
159	0.00304855037478505\\
160	0.00304882135857308\\
161	0.00304909699858753\\
162	0.00304937737463915\\
163	0.00304966256789988\\
164	0.0030499526609258\\
165	0.00305024773768053\\
166	0.00305054788355909\\
167	0.00305085318541202\\
168	0.00305116373157007\\
169	0.00305147961186905\\
170	0.00305180091767536\\
171	0.00305212774191184\\
172	0.00305246017908392\\
173	0.00305279832530643\\
174	0.00305314227833069\\
175	0.00305349213757211\\
176	0.00305384800413824\\
177	0.00305420998085731\\
178	0.00305457817230707\\
179	0.00305495268484445\\
180	0.00305533362663533\\
181	0.00305572110768509\\
182	0.00305611523986942\\
183	0.00305651613696585\\
184	0.00305692391468572\\
185	0.00305733869070659\\
186	0.0030577605847053\\
187	0.00305818971839144\\
188	0.00305862621554155\\
189	0.00305907020203372\\
190	0.00305952180588283\\
191	0.00305998115727626\\
192	0.00306044838861037\\
193	0.00306092363452737\\
194	0.00306140703195296\\
195	0.00306189872013442\\
196	0.00306239884067943\\
197	0.00306290753759554\\
198	0.00306342495733007\\
199	0.00306395124881094\\
200	0.00306448656348788\\
201	0.00306503105537458\\
202	0.00306558488109133\\
203	0.00306614819990824\\
204	0.00306672117378955\\
205	0.00306730396743808\\
206	0.00306789674834098\\
207	0.00306849968681576\\
208	0.00306911295605723\\
209	0.00306973673218526\\
210	0.00307037119429307\\
211	0.00307101652449648\\
212	0.00307167290798388\\
213	0.0030723405330669\\
214	0.00307301959123206\\
215	0.00307371027719301\\
216	0.00307441278894365\\
217	0.00307512732781234\\
218	0.00307585409851649\\
219	0.00307659330921831\\
220	0.00307734517158135\\
221	0.00307810990082795\\
222	0.00307888771579747\\
223	0.00307967883900548\\
224	0.00308048349670392\\
225	0.00308130191894195\\
226	0.00308213433962811\\
227	0.00308298099659295\\
228	0.00308384213165301\\
229	0.0030847179906756\\
230	0.00308560882364454\\
231	0.00308651488472681\\
232	0.00308743643234043\\
233	0.00308837372922319\\
234	0.0030893270425023\\
235	0.00309029664376532\\
236	0.00309128280913192\\
237	0.00309228581932678\\
238	0.0030933059597536\\
239	0.00309434352057004\\
240	0.00309539879676386\\
241	0.00309647208823016\\
242	0.00309756369984966\\
243	0.00309867394156809\\
244	0.00309980312847685\\
245	0.00310095158089455\\
246	0.00310211962444996\\
247	0.00310330759016596\\
248	0.00310451581454468\\
249	0.00310574463965382\\
250	0.00310699441321429\\
251	0.00310826548868878\\
252	0.00310955822537173\\
253	0.00311087298848048\\
254	0.00311221014924765\\
255	0.00311357008501479\\
256	0.00311495317932709\\
257	0.00311635982202967\\
258	0.00311779040936487\\
259	0.00311924534407096\\
260	0.00312072503548212\\
261	0.0031222298996297\\
262	0.00312376035934481\\
263	0.00312531684436224\\
264	0.00312689979142575\\
265	0.00312850964439462\\
266	0.00313014685435141\\
267	0.00313181187971142\\
268	0.00313350518633285\\
269	0.00313522724762836\\
270	0.00313697854467666\\
271	0.00313875956633266\\
272	0.00314057080933009\\
273	0.00314241277836093\\
274	0.00314428598608756\\
275	0.00314619095297797\\
276	0.00314812820673057\\
277	0.00315009828105087\\
278	0.00315210171510535\\
279	0.00315413906353755\\
280	0.00315621089020438\\
281	0.00315831776787394\\
282	0.00316046027834203\\
283	0.00316263901254875\\
284	0.003164854570695\\
285	0.00316710756235895\\
286	0.00316939860661227\\
287	0.00317172833213598\\
288	0.00317409737733571\\
289	0.00317650639045645\\
290	0.00317895602969642\\
291	0.00318144696331995\\
292	0.00318397986976924\\
293	0.00318655543777469\\
294	0.0031891743664637\\
295	0.00319183736546763\\
296	0.00319454515502686\\
297	0.00319729846609331\\
298	0.00320009804043076\\
299	0.00320294463071217\\
300	0.00320583900061387\\
301	0.00320878192490665\\
302	0.0032117741895431\\
303	0.0032148165917414\\
304	0.00321790994006526\\
305	0.00322105505450091\\
306	0.00322425276653286\\
307	0.00322750391922457\\
308	0.0032308093673196\\
309	0.00323416997740608\\
310	0.00323758662825071\\
311	0.00324106021155859\\
312	0.00324459163371413\\
313	0.00324818181944748\\
314	0.00325183171778768\\
315	0.00325554230340428\\
316	0.00325931453812834\\
317	0.00326314938512363\\
318	0.00326704781990338\\
319	0.00327101083037703\\
320	0.00327503941689374\\
321	0.00327913459228231\\
322	0.00328329738188779\\
323	0.0032875288236051\\
324	0.00329182996790991\\
325	0.00329620187788686\\
326	0.0033006456292558\\
327	0.00330516231039655\\
328	0.00330975302237266\\
329	0.00331441887895505\\
330	0.00331916100664649\\
331	0.00332398054470761\\
332	0.00332887864518655\\
333	0.00333385647295279\\
334	0.00333891520573777\\
335	0.00334405603418376\\
336	0.00334928016190405\\
337	0.00335458880555669\\
338	0.00335998319493566\\
339	0.00336546457308269\\
340	0.00337103419642487\\
341	0.00337669333494218\\
342	0.0033824432723713\\
343	0.0033882853064516\\
344	0.00339422074922057\\
345	0.00340025092736631\\
346	0.00340637718264471\\
347	0.00341260087237047\\
348	0.00341892336998915\\
349	0.00342534606573737\\
350	0.00343187036739461\\
351	0.00343849770112212\\
352	0.00344522951236527\\
353	0.00345206726674163\\
354	0.00345901245067994\\
355	0.00346606657109623\\
356	0.00347323115190361\\
357	0.00348050772061386\\
358	0.00348789776607916\\
359	0.00349540263597865\\
360	0.00350302364655946\\
361	0.00351076243536027\\
362	0.00351862066977116\\
363	0.00352660004706458\\
364	0.00353470229387294\\
365	0.00354292916578727\\
366	0.00355128244940566\\
367	0.00355976397110082\\
368	0.00356837560093155\\
369	0.00357711908392077\\
370	0.00358599606644035\\
371	0.00359500817362262\\
372	0.00360415701181463\\
373	0.00361344416473827\\
374	0.00362287119070602\\
375	0.00363243962329641\\
376	0.0036421509819296\\
377	0.00365200680929468\\
378	0.00366200877770539\\
379	0.00367215894944063\\
380	0.00368246019276606\\
381	0.00369291448887022\\
382	0.00370352316447297\\
383	0.00371428743342848\\
384	0.0037252083752065\\
385	0.00373628690975085\\
386	0.00374752376805794\\
387	0.00375891945768785\\
388	0.00377047422226314\\
389	0.00378218799381759\\
390	0.00379406033662271\\
391	0.00380609038083258\\
392	0.00381827674393536\\
393	0.00383061743756721\\
394	0.00384310975671144\\
395	0.00385575014764735\\
396	0.00386853405019943\\
397	0.00388145570882703\\
398	0.00389450794583856\\
399	0.00390768188844423\\
400	0.00392096663938711\\
401	0.0039343488783687\\
402	0.00394781237813493\\
403	0.00396133741420826\\
404	0.00397490003859048\\
405	0.00398847116685206\\
406	0.00400201535859076\\
407	0.00401548887315432\\
408	0.00402883494268167\\
409	0.00404197666253574\\
410	0.00405483387971556\\
411	0.00406730920686715\\
412	0.00407928411461078\\
413	0.00409061413591738\\
414	0.00410112328338029\\
415	0.00411059866972351\\
416	0.00411976746527252\\
417	0.00412906916830723\\
418	0.00413850301769023\\
419	0.00414806780402478\\
420	0.0041577618081022\\
421	0.00416758279534187\\
422	0.00417752822733678\\
423	0.00418759622241075\\
424	0.004197789117264\\
425	0.0042081269899841\\
426	0.00421863205451258\\
427	0.00422930607143691\\
428	0.00424015075867423\\
429	0.004251167785805\\
430	0.00426235876789\\
431	0.00427372525871681\\
432	0.00428526874345597\\
433	0.00429699063068474\\
434	0.00430889224367596\\
435	0.00432097481087908\\
436	0.00433323945553472\\
437	0.00434568718434684\\
438	0.0043583188751327\\
439	0.00437113526337202\\
440	0.00438413692758075\\
441	0.00439732427343946\\
442	0.00441069751655755\\
443	0.00442425666353447\\
444	0.00443800149163091\\
445	0.00445193152736049\\
446	0.00446604602374304\\
447	0.00448034393626135\\
448	0.00449482389843095\\
449	0.00450948420924725\\
450	0.00452432288789258\\
451	0.00453933797452697\\
452	0.00455452740746064\\
453	0.00456988901507084\\
454	0.00458542062548439\\
455	0.00460112023002508\\
456	0.00461698622005927\\
457	0.0046330177100885\\
458	0.00464921493486472\\
459	0.00466557939521225\\
460	0.00468210867217519\\
461	0.00469879965952423\\
462	0.00471565019985468\\
463	0.00473265971222406\\
464	0.00474983003071213\\
465	0.00476716651370559\\
466	0.00478467953169976\\
467	0.00480238653623335\\
468	0.00482031477230763\\
469	0.00483850488659161\\
470	0.00485701572111196\\
471	0.004875929161333\\
472	0.00489529644610638\\
473	0.00491513298305584\\
474	0.00493545509421521\\
475	0.00495627994266942\\
476	0.00497762505705277\\
477	0.00499950880073145\\
478	0.0050219504705099\\
479	0.0050449704075158\\
480	0.00506859007336904\\
481	0.00509283211145518\\
482	0.0051177203005794\\
483	0.00514327950696849\\
484	0.00516953856622172\\
485	0.00519652882533619\\
486	0.00522428493328236\\
487	0.00525283976962182\\
488	0.00528222471205487\\
489	0.00531247156248563\\
490	0.00534361250795315\\
491	0.00537568038786467\\
492	0.00540870878097448\\
493	0.00544273219379265\\
494	0.00547778641436578\\
495	0.00551390909868753\\
496	0.0055511406830368\\
497	0.00558952584142584\\
498	0.00562911789773015\\
499	0.00566999128690472\\
500	0.00571221932201546\\
501	0.0057558715749304\\
502	0.00580101005582034\\
503	0.00584768388787208\\
504	0.00589592198047116\\
505	0.00594572303841862\\
506	0.00599704395404753\\
507	0.00604978581485652\\
508	0.00610377318904819\\
509	0.00615872670430141\\
510	0.00621377305779257\\
511	0.00626773342543394\\
512	0.00632044037470168\\
513	0.00637171706810509\\
514	0.00642138012600038\\
515	0.00646924415730192\\
516	0.00651512881061301\\
517	0.00655886878984116\\
518	0.00660032857703474\\
519	0.00663942309655355\\
520	0.00667614607675877\\
521	0.00671060900481495\\
522	0.00674415544607651\\
523	0.00677716675605977\\
524	0.00680964580317178\\
525	0.00684160813228221\\
526	0.00687308399823199\\
527	0.00690412025177207\\
528	0.00693478183651414\\
529	0.00696515253551898\\
530	0.00699533442800454\\
531	0.00702544528207324\\
532	0.00705561271722277\\
533	0.00708594791661066\\
534	0.00711650232276352\\
535	0.00714730727861487\\
536	0.00717839739551637\\
537	0.00720981020187562\\
538	0.00724158560986581\\
539	0.00727376518449118\\
540	0.00730639121578814\\
541	0.00733950562402815\\
542	0.00737314876893419\\
543	0.00740735831282291\\
544	0.00744216840464061\\
545	0.00747761164752185\\
546	0.00751372125221221\\
547	0.00755053132123895\\
548	0.00758807649293622\\
549	0.00762639148986282\\
550	0.0076655105538372\\
551	0.00770546673078169\\
552	0.00774629139584217\\
553	0.00778800482332902\\
554	0.00783062355809369\\
555	0.00787416316338176\\
556	0.00791863849542332\\
557	0.00796406368646087\\
558	0.00801045200277901\\
559	0.00805781568956767\\
560	0.00810616580577066\\
561	0.00815551204477823\\
562	0.00820586253849474\\
563	0.00825722363207021\\
564	0.00830959973629561\\
565	0.00836299312329599\\
566	0.00841740357329269\\
567	0.00847282849714414\\
568	0.0085292617624085\\
569	0.00858669084759215\\
570	0.00864508965557358\\
571	0.00870442086865101\\
572	0.00876459910238825\\
573	0.00882562379752669\\
574	0.00888756930913724\\
575	0.00895054169680455\\
576	0.00901356110634767\\
577	0.00907514900313189\\
578	0.00913502099859305\\
579	0.00919263117907495\\
580	0.0092479758283961\\
581	0.00930184486036383\\
582	0.00935368351991403\\
583	0.00940333326402388\\
584	0.00945082356531764\\
585	0.00949690757289053\\
586	0.00954180842209612\\
587	0.00958587549421746\\
588	0.00962920975611778\\
589	0.0096718487149668\\
590	0.00971377621437142\\
591	0.00975496638267559\\
592	0.00979538031248056\\
593	0.00983498200924467\\
594	0.00987369853931868\\
595	0.00991118387968948\\
596	0.00994658651044256\\
597	0.00997788999445116\\
598	0.010000292044645\\
599	0\\
600	0\\
};
\addplot [color=blue!80!mycolor9,solid,forget plot]
  table[row sep=crcr]{%
1	0.00247811649021119\\
2	0.00247812555694198\\
3	0.00247813477847231\\
4	0.00247814415746359\\
5	0.00247815369662314\\
6	0.00247816339870495\\
7	0.00247817326651042\\
8	0.00247818330288924\\
9	0.00247819351074017\\
10	0.00247820389301195\\
11	0.00247821445270408\\
12	0.00247822519286778\\
13	0.00247823611660675\\
14	0.00247824722707825\\
15	0.00247825852749384\\
16	0.00247827002112047\\
17	0.00247828171128131\\
18	0.00247829360135681\\
19	0.00247830569478558\\
20	0.00247831799506553\\
21	0.00247833050575474\\
22	0.00247834323047263\\
23	0.00247835617290096\\
24	0.00247836933678482\\
25	0.00247838272593385\\
26	0.00247839634422328\\
27	0.00247841019559507\\
28	0.00247842428405909\\
29	0.00247843861369422\\
30	0.00247845318864961\\
31	0.00247846801314581\\
32	0.00247848309147611\\
33	0.00247849842800767\\
34	0.00247851402718287\\
35	0.00247852989352059\\
36	0.00247854603161746\\
37	0.00247856244614932\\
38	0.0024785791418725\\
39	0.00247859612362519\\
40	0.00247861339632894\\
41	0.00247863096498999\\
42	0.00247864883470076\\
43	0.00247866701064138\\
44	0.00247868549808116\\
45	0.00247870430238013\\
46	0.00247872342899056\\
47	0.00247874288345863\\
48	0.00247876267142599\\
49	0.00247878279863142\\
50	0.00247880327091246\\
51	0.0024788240942072\\
52	0.0024788452745559\\
53	0.00247886681810285\\
54	0.00247888873109806\\
55	0.00247891101989917\\
56	0.00247893369097324\\
57	0.00247895675089867\\
58	0.00247898020636708\\
59	0.00247900406418525\\
60	0.00247902833127715\\
61	0.00247905301468592\\
62	0.00247907812157592\\
63	0.00247910365923484\\
64	0.00247912963507571\\
65	0.00247915605663926\\
66	0.00247918293159587\\
67	0.00247921026774801\\
68	0.00247923807303234\\
69	0.00247926635552212\\
70	0.00247929512342953\\
71	0.00247932438510803\\
72	0.00247935414905477\\
73	0.0024793844239131\\
74	0.00247941521847502\\
75	0.00247944654168382\\
76	0.00247947840263653\\
77	0.00247951081058671\\
78	0.00247954377494698\\
79	0.00247957730529186\\
80	0.00247961141136047\\
81	0.00247964610305934\\
82	0.00247968139046533\\
83	0.00247971728382847\\
84	0.00247975379357496\\
85	0.00247979093031016\\
86	0.00247982870482167\\
87	0.00247986712808238\\
88	0.00247990621125367\\
89	0.00247994596568866\\
90	0.00247998640293537\\
91	0.00248002753474013\\
92	0.0024800693730509\\
93	0.0024801119300207\\
94	0.00248015521801113\\
95	0.00248019924959592\\
96	0.00248024403756445\\
97	0.00248028959492547\\
98	0.00248033593491084\\
99	0.00248038307097927\\
100	0.00248043101682014\\
101	0.00248047978635751\\
102	0.00248052939375398\\
103	0.00248057985341478\\
104	0.0024806311799919\\
105	0.00248068338838818\\
106	0.00248073649376166\\
107	0.00248079051152982\\
108	0.00248084545737399\\
109	0.00248090134724376\\
110	0.00248095819736159\\
111	0.00248101602422742\\
112	0.00248107484462324\\
113	0.00248113467561801\\
114	0.00248119553457237\\
115	0.0024812574391437\\
116	0.00248132040729099\\
117	0.00248138445728006\\
118	0.00248144960768869\\
119	0.00248151587741186\\
120	0.0024815832856672\\
121	0.00248165185200038\\
122	0.00248172159629065\\
123	0.00248179253875655\\
124	0.00248186469996156\\
125	0.00248193810082007\\
126	0.00248201276260318\\
127	0.00248208870694486\\
128	0.00248216595584803\\
129	0.00248224453169084\\
130	0.00248232445723302\\
131	0.00248240575562236\\
132	0.0024824884504013\\
133	0.00248257256551368\\
134	0.0024826581253114\\
135	0.00248274515456154\\
136	0.00248283367845326\\
137	0.00248292372260504\\
138	0.00248301531307206\\
139	0.00248310847635362\\
140	0.00248320323940094\\
141	0.00248329962962534\\
142	0.00248339767490694\\
143	0.0024834974036043\\
144	0.00248359884456432\\
145	0.00248370202713047\\
146	0.0024838069811441\\
147	0.00248391373694049\\
148	0.00248402232537194\\
149	0.00248413277781647\\
150	0.00248424512618674\\
151	0.00248435940293909\\
152	0.00248447564108274\\
153	0.00248459387418913\\
154	0.00248471413640142\\
155	0.00248483646244414\\
156	0.00248496088763306\\
157	0.00248508744788513\\
158	0.00248521617972863\\
159	0.00248534712031352\\
160	0.00248548030742194\\
161	0.00248561577947882\\
162	0.0024857535755628\\
163	0.00248589373541721\\
164	0.00248603629946128\\
165	0.00248618130880154\\
166	0.0024863288052434\\
167	0.00248647883130295\\
168	0.00248663143021888\\
169	0.00248678664596469\\
170	0.00248694452326102\\
171	0.00248710510758828\\
172	0.00248726844519937\\
173	0.00248743458313271\\
174	0.00248760356922545\\
175	0.00248777545212685\\
176	0.00248795028131197\\
177	0.0024881281070955\\
178	0.00248830898064589\\
179	0.00248849295399968\\
180	0.00248868008007601\\
181	0.00248887041269149\\
182	0.00248906400657519\\
183	0.002489260917384\\
184	0.00248946120171807\\
185	0.00248966491713671\\
186	0.00248987212217436\\
187	0.00249008287635696\\
188	0.0024902972402185\\
189	0.0024905152753179\\
190	0.00249073704425609\\
191	0.00249096261069343\\
192	0.00249119203936747\\
193	0.00249142539611082\\
194	0.00249166274786942\\
195	0.00249190416272117\\
196	0.00249214970989475\\
197	0.00249239945978876\\
198	0.00249265348399125\\
199	0.00249291185529943\\
200	0.0024931746477399\\
201	0.00249344193658893\\
202	0.0024937137983933\\
203	0.00249399031099142\\
204	0.00249427155353465\\
205	0.00249455760650921\\
206	0.00249484855175818\\
207	0.002495144472504\\
208	0.00249544545337136\\
209	0.0024957515804103\\
210	0.00249606294111987\\
211	0.00249637962447197\\
212	0.00249670172093573\\
213	0.00249702932250214\\
214	0.00249736252270925\\
215	0.0024977014166675\\
216	0.00249804610108577\\
217	0.00249839667429747\\
218	0.00249875323628743\\
219	0.00249911588871892\\
220	0.00249948473496118\\
221	0.00249985988011741\\
222	0.00250024143105315\\
223	0.00250062949642516\\
224	0.00250102418671064\\
225	0.002501425614237\\
226	0.00250183389321195\\
227	0.00250224913975434\\
228	0.00250267147192506\\
229	0.00250310100975868\\
230	0.00250353787529553\\
231	0.0025039821926142\\
232	0.00250443408786451\\
233	0.00250489368930104\\
234	0.0025053611273171\\
235	0.0025058365344792\\
236	0.00250632004556201\\
237	0.00250681179758382\\
238	0.00250731192984247\\
239	0.00250782058395191\\
240	0.00250833790387907\\
241	0.00250886403598137\\
242	0.00250939912904469\\
243	0.00250994333432182\\
244	0.00251049680557143\\
245	0.00251105969909756\\
246	0.00251163217378957\\
247	0.00251221439116256\\
248	0.00251280651539834\\
249	0.00251340871338679\\
250	0.00251402115476774\\
251	0.0025146440119733\\
252	0.00251527746027064\\
253	0.00251592167780518\\
254	0.00251657684564421\\
255	0.00251724314782092\\
256	0.00251792077137885\\
257	0.00251860990641661\\
258	0.00251931074613306\\
259	0.00252002348687272\\
260	0.00252074832817157\\
261	0.00252148547280303\\
262	0.00252223512682423\\
263	0.00252299749962257\\
264	0.00252377280396222\\
265	0.00252456125603103\\
266	0.00252536307548745\\
267	0.00252617848550731\\
268	0.00252700771283079\\
269	0.00252785098780889\\
270	0.00252870854444927\\
271	0.00252958062045988\\
272	0.00253046745728757\\
273	0.00253136930014451\\
274	0.00253228639800854\\
275	0.00253321900358004\\
276	0.00253416737321523\\
277	0.00253513176705866\\
278	0.00253611244993631\\
279	0.00253710969076723\\
280	0.00253812376256445\\
281	0.00253915494247581\\
282	0.00254020351182379\\
283	0.00254126975614418\\
284	0.00254235396522361\\
285	0.00254345643313563\\
286	0.0025445774582754\\
287	0.00254571734339269\\
288	0.00254687639562325\\
289	0.00254805492651808\\
290	0.00254925325207081\\
291	0.00255047169274272\\
292	0.00255171057348544\\
293	0.00255297022376099\\
294	0.00255425097755892\\
295	0.00255555317341061\\
296	0.00255687715440021\\
297	0.00255822326817209\\
298	0.00255959186693452\\
299	0.0025609833074594\\
300	0.00256239795107767\\
301	0.00256383616367012\\
302	0.00256529831565325\\
303	0.00256678478196009\\
304	0.00256829594201556\\
305	0.00256983217970675\\
306	0.00257139388334885\\
307	0.00257298144564973\\
308	0.00257459526368051\\
309	0.00257623573886992\\
310	0.00257790327705853\\
311	0.00257959828867543\\
312	0.00258132118909213\\
313	0.00258307239899605\\
314	0.00258485234379022\\
315	0.00258666145005346\\
316	0.00258850014697706\\
317	0.0025903688672863\\
318	0.00259226804708924\\
319	0.0025941981257115\\
320	0.00259615954551571\\
321	0.00259815275170518\\
322	0.0026001781921106\\
323	0.00260223631695857\\
324	0.00260432757862129\\
325	0.00260645243134581\\
326	0.00260861133096183\\
327	0.00261080473456659\\
328	0.00261303310018548\\
329	0.00261529688640687\\
330	0.00261759655198955\\
331	0.00261993255544103\\
332	0.00262230535456489\\
333	0.00262471540597562\\
334	0.00262716316457854\\
335	0.00262964908301356\\
336	0.00263217361106048\\
337	0.00263473719500448\\
338	0.00263734027696049\\
339	0.00263998329415559\\
340	0.00264266667816952\\
341	0.00264539085413462\\
342	0.0026481562398981\\
343	0.00265096324515207\\
344	0.00265381227054041\\
345	0.00265670370675562\\
346	0.00265963793364637\\
347	0.00266261531936403\\
348	0.00266563621958899\\
349	0.00266870097689276\\
350	0.00267180992031164\\
351	0.00267496336523225\\
352	0.00267816161371457\\
353	0.00268140495539548\\
354	0.00268469366908969\\
355	0.00268802802506198\\
356	0.00269140828756429\\
357	0.00269483471685336\\
358	0.00269830757285886\\
359	0.00270182714513143\\
360	0.00270539380062159\\
361	0.00270900798273177\\
362	0.00271267024811181\\
363	0.00271638131646151\\
364	0.00272014213838986\\
365	0.00272395399022021\\
366	0.00272781861511963\\
367	0.00273173846273063\\
368	0.00273571718681733\\
369	0.00273976151590172\\
370	0.00274387685410735\\
371	0.0027480652175805\\
372	0.0027523284555114\\
373	0.00275666849663476\\
374	0.00276108735077459\\
375	0.00276558710992141\\
376	0.0027701699490338\\
377	0.00277483812421509\\
378	0.00277959394645428\\
379	0.00278443959488847\\
380	0.00278937597696776\\
381	0.00279440505860262\\
382	0.00279952987475586\\
383	0.00280475368908226\\
384	0.00281008002083447\\
385	0.0028155126757165\\
386	0.0028210557813721\\
387	0.00282671382833309\\
388	0.00283249171741898\\
389	0.00283839481478436\\
390	0.00284442901606047\\
391	0.00285060082134511\\
392	0.00285691742317362\\
393	0.00286338681006984\\
394	0.00287001788885343\\
395	0.00287682062959411\\
396	0.00288380623799019\\
397	0.00289098736105367\\
398	0.00289837833336593\\
399	0.00290599547291058\\
400	0.00291385743772171\\
401	0.00292198565753793\\
402	0.00293040485881307\\
403	0.00293914370801658\\
404	0.00294823561058515\\
405	0.00295771973216965\\
406	0.0029676423972256\\
407	0.00297805938066788\\
408	0.00298904182625361\\
409	0.00300066528391573\\
410	0.00301301628128928\\
411	0.00302619752181333\\
412	0.00304033140484477\\
413	0.00305556435770621\\
414	0.00307207207453714\\
415	0.00309006482955999\\
416	0.00310880661883055\\
417	0.00312785605903395\\
418	0.00314721714365967\\
419	0.00316689384652706\\
420	0.00318689014002555\\
421	0.00320721005795063\\
422	0.00322785787452478\\
423	0.00324883855072928\\
424	0.0032701585118829\\
425	0.003291823707487\\
426	0.0033138379259991\\
427	0.0033362048377587\\
428	0.00335892797542417\\
429	0.00338201071198679\\
430	0.00340545623601502\\
431	0.00342926752372865\\
432	0.00345344730742967\\
433	0.00347799803973121\\
434	0.00350292185292983\\
435	0.00352822051275047\\
436	0.00355389536554866\\
437	0.00357994727788222\\
438	0.00360637656715544\\
439	0.00363318292178348\\
440	0.0036603653090112\\
441	0.00368792186812286\\
442	0.00371584978624311\\
443	0.0037441451531795\\
444	0.00377280279012463\\
445	0.00380181604259669\\
446	0.00383117651323895\\
447	0.00386087365251626\\
448	0.00389089388583082\\
449	0.00392121690199829\\
450	0.00395179716637846\\
451	0.00398259838267485\\
452	0.00401358468378152\\
453	0.00404471265061179\\
454	0.00407592977923796\\
455	0.00410717277754148\\
456	0.00413836605779342\\
457	0.00416942179077619\\
458	0.00420024634456966\\
459	0.00423081211061245\\
460	0.00426140020028307\\
461	0.00429200101853043\\
462	0.00432253003295564\\
463	0.0043528838050062\\
464	0.00438293598073556\\
465	0.0044125325848898\\
466	0.0044414859553579\\
467	0.00446956698539481\\
468	0.00449649542059098\\
469	0.00452192790322189\\
470	0.00454544332458644\\
471	0.00456660753965114\\
472	0.00458814729311082\\
473	0.0046100682754802\\
474	0.00463237617641393\\
475	0.00465507664007141\\
476	0.00467817529335925\\
477	0.00470167773876455\\
478	0.00472558954445719\\
479	0.00474991623144585\\
480	0.00477466327102612\\
481	0.00479983614302007\\
482	0.00482544066736576\\
483	0.00485148331035823\\
484	0.00487797072227519\\
485	0.00490490972166369\\
486	0.00493230688573686\\
487	0.00496016866117437\\
488	0.00498850152122487\\
489	0.00501731171296575\\
490	0.00504660464559965\\
491	0.00507638424155821\\
492	0.0051066539743321\\
493	0.00513741754812156\\
494	0.00516867868920914\\
495	0.00520044483843773\\
496	0.00523272670992935\\
497	0.00526553942337987\\
498	0.00529890398854494\\
499	0.00533285017825539\\
500	0.00536740912600212\\
501	0.00540262192716388\\
502	0.00543854471967436\\
503	0.0054752551006337\\
504	0.00551286051987373\\
505	0.0055515092381846\\
506	0.00559130719092906\\
507	0.00563230368435302\\
508	0.00567455068067331\\
509	0.00571810435251307\\
510	0.00576302553152001\\
511	0.00580939070107368\\
512	0.00585727370026442\\
513	0.00590674250978723\\
514	0.005957854838526\\
515	0.00601065214427572\\
516	0.00606515167623523\\
517	0.00612133597063666\\
518	0.00617913868276773\\
519	0.00623842574325406\\
520	0.00629897040266387\\
521	0.00636042026026284\\
522	0.00642121846438805\\
523	0.00648079970694343\\
524	0.00653898423739954\\
525	0.00659558459818578\\
526	0.00665040955394938\\
527	0.00670326996925213\\
528	0.00675398737755338\\
529	0.00680240615739019\\
530	0.00684841085110631\\
531	0.00689195017212662\\
532	0.00693307314023976\\
533	0.00697237816780817\\
534	0.00701119533494809\\
535	0.00704952235122946\\
536	0.00708736978307209\\
537	0.00712476342145339\\
538	0.0071617466085178\\
539	0.00719838229666012\\
540	0.00723475449016275\\
541	0.00727096855127332\\
542	0.00730714997322877\\
543	0.00734344027894289\\
544	0.00737998840528387\\
545	0.00741688811617026\\
546	0.00745418951685885\\
547	0.00749193730271656\\
548	0.00753017973388053\\
549	0.00756896791481386\\
550	0.00760835477395233\\
551	0.00764839371723731\\
552	0.00768913694195016\\
553	0.00773063378523078\\
554	0.00777292424702521\\
555	0.00781603521840035\\
556	0.00785998739163325\\
557	0.00790479871128239\\
558	0.00795048664294056\\
559	0.00799706792415065\\
560	0.00804455829354392\\
561	0.00809297220498276\\
562	0.00814232253144814\\
563	0.00819262028262012\\
564	0.00824387435428926\\
565	0.00829609133116553\\
566	0.00834927536349354\\
567	0.00840342812126299\\
568	0.00845854884012727\\
569	0.00851463422662447\\
570	0.00857167829860371\\
571	0.00862967217526905\\
572	0.00868860424489086\\
573	0.00874845809866899\\
574	0.00880921036058856\\
575	0.00887082894270814\\
576	0.00893328022478691\\
577	0.00899649525040652\\
578	0.00906043870669162\\
579	0.00912517915203537\\
580	0.00919005656715616\\
581	0.00925329200342372\\
582	0.00931456036497874\\
583	0.00937352397491844\\
584	0.00942923087906576\\
585	0.00948220461794547\\
586	0.00953271440881462\\
587	0.00958044208054697\\
588	0.00962596473294314\\
589	0.00966992069233143\\
590	0.00971269563307853\\
591	0.00975441316854107\\
592	0.00979516557260156\\
593	0.00983492675369396\\
594	0.00987369348216993\\
595	0.00991118387968948\\
596	0.00994658651044256\\
597	0.00997788999445116\\
598	0.010000292044645\\
599	0\\
600	0\\
};
\addplot [color=blue,solid,forget plot]
  table[row sep=crcr]{%
1	0.000300501426608031\\
2	0.000300519631885623\\
3	0.000300538148804544\\
4	0.000300556982713254\\
5	0.000300576139051885\\
6	0.000300595623353805\\
7	0.000300615441247206\\
8	0.000300635598456722\\
9	0.000300656100805079\\
10	0.000300676954214768\\
11	0.000300698164709764\\
12	0.000300719738417195\\
13	0.0003007416815692\\
14	0.000300764000504607\\
15	0.000300786701670854\\
16	0.00030080979162578\\
17	0.000300833277039507\\
18	0.000300857164696399\\
19	0.000300881461496977\\
20	0.000300906174459875\\
21	0.000300931310723911\\
22	0.000300956877550082\\
23	0.00030098288232366\\
24	0.000301009332556328\\
25	0.000301036235888336\\
26	0.000301063600090647\\
27	0.000301091433067201\\
28	0.000301119742857146\\
29	0.000301148537637201\\
30	0.000301177825723917\\
31	0.000301207615576129\\
32	0.000301237915797319\\
33	0.000301268735138103\\
34	0.000301300082498759\\
35	0.000301331966931708\\
36	0.000301364397644179\\
37	0.000301397384000774\\
38	0.000301430935526176\\
39	0.000301465061907863\\
40	0.000301499772998864\\
41	0.00030153507882061\\
42	0.000301570989565755\\
43	0.000301607515601091\\
44	0.00030164466747051\\
45	0.000301682455898007\\
46	0.000301720891790759\\
47	0.000301759986242223\\
48	0.00030179975053527\\
49	0.000301840196145408\\
50	0.000301881334744117\\
51	0.000301923178202066\\
52	0.000301965738592561\\
53	0.000302009028194973\\
54	0.000302053059498194\\
55	0.000302097845204237\\
56	0.000302143398231815\\
57	0.000302189731720019\\
58	0.000302236859032043\\
59	0.000302284793759027\\
60	0.000302333549723835\\
61	0.000302383140985043\\
62	0.000302433581840894\\
63	0.000302484886833373\\
64	0.000302537070752326\\
65	0.000302590148639623\\
66	0.000302644135793482\\
67	0.00030269904777273\\
68	0.000302754900401292\\
69	0.000302811709772578\\
70	0.000302869492254094\\
71	0.000302928264492068\\
72	0.000302988043416155\\
73	0.000303048846244216\\
74	0.000303110690487172\\
75	0.000303173593953979\\
76	0.00030323757475667\\
77	0.00030330265131539\\
78	0.000303368842363687\\
79	0.000303436166953741\\
80	0.000303504644461784\\
81	0.000303574294593534\\
82	0.000303645137389719\\
83	0.000303717193231815\\
84	0.000303790482847713\\
85	0.000303865027317568\\
86	0.00030394084807977\\
87	0.000304017966936938\\
88	0.000304096406062103\\
89	0.000304176188004911\\
90	0.000304257335697963\\
91	0.000304339872463302\\
92	0.000304423822018948\\
93	0.000304509208485584\\
94	0.000304596056393289\\
95	0.000304684390688489\\
96	0.000304774236740935\\
97	0.000304865620350862\\
98	0.000304958567756184\\
99	0.000305053105639937\\
100	0.000305149261137713\\
101	0.000305247061845276\\
102	0.000305346535826389\\
103	0.000305447711620587\\
104	0.000305550618251278\\
105	0.000305655285233842\\
106	0.000305761742583948\\
107	0.00030587002082592\\
108	0.000305980151001394\\
109	0.000306092164677971\\
110	0.000306206093958081\\
111	0.000306321971488003\\
112	0.000306439830467069\\
113	0.000306559704656874\\
114	0.000306681628390895\\
115	0.000306805636583991\\
116	0.000306931764742322\\
117	0.000307060048973214\\
118	0.000307190525995378\\
119	0.000307323233149159\\
120	0.000307458208407031\\
121	0.000307595490384293\\
122	0.000307735118349858\\
123	0.000307877132237315\\
124	0.0003080215726561\\
125	0.000308168480902933\\
126	0.000308317898973392\\
127	0.000308469869573712\\
128	0.000308624436132747\\
129	0.000308781642814205\\
130	0.000308941534529002\\
131	0.000309104156947894\\
132	0.00030926955651428\\
133	0.00030943778045723\\
134	0.000309608876804769\\
135	0.000309782894397335\\
136	0.000309959882901501\\
137	0.000310139892823947\\
138	0.00031032297552562\\
139	0.000310509183236287\\
140	0.000310698569069238\\
141	0.000310891187036353\\
142	0.000311087092063406\\
143	0.000311286340005505\\
144	0.000311488987662527\\
145	0.000311695092794237\\
146	0.000311904714135457\\
147	0.000312117911414112\\
148	0.000312334745368019\\
149	0.000312555277761922\\
150	0.000312779571404839\\
151	0.000313007690167653\\
152	0.000313239699001119\\
153	0.000313475663954089\\
154	0.000313715652191992\\
155	0.000313959732015797\\
156	0.000314207972881154\\
157	0.000314460445417914\\
158	0.00031471722145005\\
159	0.000314978374015754\\
160	0.000315243977388053\\
161	0.000315514107095696\\
162	0.000315788839944392\\
163	0.000316068254038434\\
164	0.000316352428802739\\
165	0.000316641445005159\\
166	0.000316935384779305\\
167	0.000317234331647649\\
168	0.000317538370545133\\
169	0.000317847587843084\\
170	0.000318162071373661\\
171	0.000318481910454595\\
172	0.000318807195914479\\
173	0.000319138020118415\\
174	0.000319474476994141\\
175	0.000319816662058643\\
176	0.000320164672445176\\
177	0.000320518606930806\\
178	0.000320878565964381\\
179	0.000321244651695082\\
180	0.00032161696800138\\
181	0.000321995620520563\\
182	0.000322380716678798\\
183	0.000322772365721634\\
184	0.000323170678745169\\
185	0.000323575768727642\\
186	0.000323987750561753\\
187	0.000324406741087356\\
188	0.000324832859124922\\
189	0.000325266225509471\\
190	0.000325706963125168\\
191	0.000326155196940573\\
192	0.000326611054044423\\
193	0.000327074663682172\\
194	0.000327546157293139\\
195	0.000328025668548302\\
196	0.00032851333338885\\
197	0.000329009290065364\\
198	0.000329513679177751\\
199	0.000330026643715947\\
200	0.000330548329101246\\
201	0.00033107888322853\\
202	0.000331618456509183\\
203	0.000332167201914825\\
204	0.00033272527502188\\
205	0.00033329283405693\\
206	0.000333870039942939\\
207	0.000334457056346362\\
208	0.000335054049725074\\
209	0.00033566118937726\\
210	0.000336278647491166\\
211	0.000336906599195865\\
212	0.000337545222612936\\
213	0.00033819469890916\\
214	0.000338855212350104\\
215	0.000339526950354936\\
216	0.000340210103552097\\
217	0.000340904865836186\\
218	0.00034161143442579\\
219	0.000342330009922658\\
220	0.000343060796371776\\
221	0.000343804001322807\\
222	0.000344559835892605\\
223	0.000345328514829062\\
224	0.000346110256576102\\
225	0.000346905283340077\\
226	0.000347713821157424\\
227	0.000348536099963604\\
228	0.000349372353663567\\
229	0.000350222820203491\\
230	0.000351087741644082\\
231	0.000351967364235264\\
232	0.000352861938492422\\
233	0.000353771719274236\\
234	0.00035469696586207\\
235	0.000355637942040964\\
236	0.000356594916182395\\
237	0.000357568161328625\\
238	0.000358557955278963\\
239	0.000359564580677665\\
240	0.000360588325103814\\
241	0.00036162948116306\\
242	0.000362688346581259\\
243	0.000363765224300177\\
244	0.00036486042257567\\
245	0.000365974255075711\\
246	0.000367107040985112\\
247	0.000368259105107945\\
248	0.000369430777974456\\
249	0.000370622395950006\\
250	0.00037183430134628\\
251	0.000373066842535078\\
252	0.000374320374064625\\
253	0.000375595256778484\\
254	0.000376891857937245\\
255	0.000378210551342952\\
256	0.000379551717466371\\
257	0.000380915743577281\\
258	0.000382303023877824\\
259	0.000383713959638895\\
260	0.000385148959339871\\
261	0.000386608438811652\\
262	0.000388092821383163\\
263	0.000389602538031377\\
264	0.000391138027535063\\
265	0.0003926997366323\\
266	0.000394288120181847\\
267	0.000395903641328527\\
268	0.000397546771672606\\
269	0.000399217991443161\\
270	0.000400917789675084\\
271	0.00040264666438957\\
272	0.000404405122777409\\
273	0.000406193681385395\\
274	0.000408012866308006\\
275	0.000409863213390112\\
276	0.000411745268452753\\
277	0.000413659587586473\\
278	0.000415606737301452\\
279	0.000417587294749511\\
280	0.000419601847956675\\
281	0.000421650996062257\\
282	0.000423735349564715\\
283	0.000425855530574405\\
284	0.000428012173073514\\
285	0.000430205923183494\\
286	0.000432437439440065\\
287	0.000434707393076316\\
288	0.000437016468313859\\
289	0.000439365362662696\\
290	0.000441754787229672\\
291	0.00044418546703613\\
292	0.000446658141344895\\
293	0.000449173563996945\\
294	0.000451732503758152\\
295	0.000454335744676249\\
296	0.000456984086448427\\
297	0.000459678344800051\\
298	0.000462419351874562\\
299	0.000465207956635017\\
300	0.000468045025277779\\
301	0.000470931441658408\\
302	0.000473868107730537\\
303	0.000476855943997822\\
304	0.000479895889980021\\
305	0.000482988904693828\\
306	0.000486135967150064\\
307	0.000489338076869641\\
308	0.000492596254421141\\
309	0.00049591154198332\\
310	0.000499285003931393\\
311	0.000502717727435768\\
312	0.000506210823043187\\
313	0.000509765425183453\\
314	0.000513382692465421\\
315	0.000517063808503682\\
316	0.000520809982684537\\
317	0.000524622450834215\\
318	0.000528502475904416\\
319	0.000532451348675109\\
320	0.000536470388474993\\
321	0.000540560943919258\\
322	0.000544724393664747\\
323	0.000548962147182398\\
324	0.000553275645546648\\
325	0.000557666362241569\\
326	0.000562135803983405\\
327	0.000566685511558842\\
328	0.000571317060678727\\
329	0.000576032062846095\\
330	0.000580832166237965\\
331	0.000585719056599692\\
332	0.000590694458150562\\
333	0.000595760134499168\\
334	0.000600917889566737\\
335	0.000606169568516076\\
336	0.000611517058683465\\
337	0.000616962290510151\\
338	0.000622507238469203\\
339	0.000628153921982653\\
340	0.000633904406322138\\
341	0.000639760803484944\\
342	0.0006457252730346\\
343	0.000651800022892481\\
344	0.000657987310062489\\
345	0.000664289441266136\\
346	0.000670708773457938\\
347	0.000677247714182177\\
348	0.000683908721719769\\
349	0.000690694304957674\\
350	0.000697607022890975\\
351	0.000704649483637008\\
352	0.000711824342799276\\
353	0.00071913430096381\\
354	0.000726582100049746\\
355	0.000734170518191728\\
356	0.000741902362825378\\
357	0.000749780461643887\\
358	0.000757807652416314\\
359	0.00076598676939214\\
360	0.00077432061904643\\
361	0.000782811951648188\\
362	0.000791463422611622\\
363	0.00080027753796957\\
364	0.000809256571552262\\
365	0.000818402422494648\\
366	0.000827716324432034\\
367	0.000837198121641135\\
368	0.000846843886573104\\
369	0.00085621759755801\\
370	0.000865719835157762\\
371	0.000875384922099738\\
372	0.000885215919596194\\
373	0.000895215956711118\\
374	0.000905388232153993\\
375	0.000915736015490102\\
376	0.000926262646707834\\
377	0.000936971533698378\\
378	0.000947866152129432\\
379	0.000958949985040097\\
380	0.000970226771537842\\
381	0.000981700472945479\\
382	0.000993375175281812\\
383	0.00100525509678743\\
384	0.00101734459611838\\
385	0.00102964818126044\\
386	0.00104217051921655\\
387	0.00105491644651524\\
388	0.00106789098058222\\
389	0.00108109933200473\\
390	0.00109454691770115\\
391	0.00110823937498109\\
392	0.00112218257644393\\
393	0.00113638264561016\\
394	0.0011508459731072\\
395	0.00116557923313257\\
396	0.00118058939978738\\
397	0.00119588376270498\\
398	0.00121146994119191\\
399	0.00122735589585838\\
400	0.00124354993647432\\
401	0.00126006072461482\\
402	0.00127689726966284\\
403	0.00129406891689551\\
404	0.00131158532557222\\
405	0.00132945642701809\\
406	0.00134769232504865\\
407	0.00136630334920238\\
408	0.00138529863967356\\
409	0.0014046865214009\\
410	0.00142447447639833\\
411	0.00144466876411519\\
412	0.00146527380807004\\
413	0.00148629236136712\\
414	0.00150772242408549\\
415	0.00152955174845827\\
416	0.00155178053429491\\
417	0.00157441738488319\\
418	0.00159747117000727\\
419	0.0016209510473805\\
420	0.00164486649233153\\
421	0.00166922733669089\\
422	0.00169404381114076\\
423	0.00171932658913223\\
424	0.00174508660911482\\
425	0.00177133497272887\\
426	0.00179808317403233\\
427	0.00182534312284982\\
428	0.00185312717023215\\
429	0.00188144813627173\\
430	0.00191031934055164\\
431	0.00193975463554364\\
432	0.00196976844331324\\
433	0.00200037579593989\\
434	0.00203159238011775\\
435	0.00206343458646723\\
436	0.00209591956415803\\
437	0.00212906528151442\\
438	0.00216289059332055\\
439	0.00219741531550776\\
440	0.00223266030762541\\
441	0.00226864756255277\\
442	0.00230540030025111\\
443	0.00234294305536536\\
444	0.00238130173134972\\
445	0.00242050355843802\\
446	0.00246057688303989\\
447	0.00250155151506867\\
448	0.00254345254983823\\
449	0.0025863163109077\\
450	0.00263018519724896\\
451	0.00267510706601041\\
452	0.00272113538969605\\
453	0.00276833032719836\\
454	0.00281676004265518\\
455	0.00286650228071771\\
456	0.00291764598612343\\
457	0.00297029183945389\\
458	0.00302454769034921\\
459	0.00305055853264484\\
460	0.0030717054879797\\
461	0.00309373699915778\\
462	0.00311676704613136\\
463	0.00314094401885549\\
464	0.00316642522675929\\
465	0.00319338423266275\\
466	0.00322202684986058\\
467	0.00325259823185108\\
468	0.00328539262995512\\
469	0.00332076001564318\\
470	0.00335911688018843\\
471	0.00340087897834244\\
472	0.00344329284598584\\
473	0.00348636279786332\\
474	0.00353009234460264\\
475	0.00357448410288194\\
476	0.00361953969633514\\
477	0.00366525962840545\\
478	0.00371164300374218\\
479	0.00375868663529404\\
480	0.00380638141616894\\
481	0.00385469345016474\\
482	0.00390358045788171\\
483	0.00395302257859817\\
484	0.00400299667579476\\
485	0.00405347626436221\\
486	0.0041044320174268\\
487	0.00415583392408783\\
488	0.00420765852502683\\
489	0.00425991707445931\\
490	0.00431270042363151\\
491	0.00436601211529089\\
492	0.0044197932248762\\
493	0.00447397060726542\\
494	0.00452845408490524\\
495	0.00458313250642518\\
496	0.00463786904438652\\
497	0.00469249538226218\\
498	0.00474680451832123\\
499	0.00480054107464478\\
500	0.00485339074819056\\
501	0.00490496638135537\\
502	0.00495479087986692\\
503	0.00500227689936128\\
504	0.00504670127209525\\
505	0.00508717390834856\\
506	0.00512773714340334\\
507	0.00516850937848077\\
508	0.00520938472009274\\
509	0.00525054066415425\\
510	0.00529261159274347\\
511	0.00533562078544122\\
512	0.00537959235534939\\
513	0.00542455163858445\\
514	0.00547052605096188\\
515	0.00551754726203399\\
516	0.00556563622915616\\
517	0.00561481300511334\\
518	0.00566509525535443\\
519	0.00571650047162964\\
520	0.00576904879683756\\
521	0.00582276570887622\\
522	0.00587769973423755\\
523	0.00593390604542822\\
524	0.00599143941593479\\
525	0.00605035287192922\\
526	0.00611069585636212\\
527	0.00617251175254849\\
528	0.00623583452787127\\
529	0.00630068445836571\\
530	0.00636706283199385\\
531	0.00643494485661551\\
532	0.00650411822733752\\
533	0.00657373800204402\\
534	0.0066422614192598\\
535	0.00670949828702219\\
536	0.0067752479697538\\
537	0.00683930310358746\\
538	0.00690145550474471\\
539	0.00696150512625364\\
540	0.00701927321286701\\
541	0.00707462004096888\\
542	0.00712745146192194\\
543	0.0071777532813973\\
544	0.00722564490142717\\
545	0.00727260397683621\\
546	0.00731915191368715\\
547	0.00736529988918563\\
548	0.00741107522451367\\
549	0.007456523672545\\
550	0.00750171124054802\\
551	0.00754672506984011\\
552	0.00759167265990017\\
553	0.00763667928142201\\
554	0.00768188692804579\\
555	0.00772745204322683\\
556	0.00777352896986929\\
557	0.00782017594083521\\
558	0.00786741630291251\\
559	0.00791527588252588\\
560	0.00796378264304856\\
561	0.00801296618049495\\
562	0.00806285723861679\\
563	0.00811348655083227\\
564	0.00816488335329511\\
565	0.0082170736010332\\
566	0.00827007798995417\\
567	0.00832390999756632\\
568	0.00837857422847543\\
569	0.00843407061551391\\
570	0.0084903967595156\\
571	0.00854754764423447\\
572	0.00860551529537557\\
573	0.00866428858236032\\
574	0.00872385303281798\\
575	0.00878419052251876\\
576	0.00884527895369516\\
577	0.00890709236790587\\
578	0.00896959986638077\\
579	0.00903276347919527\\
580	0.00909654468053119\\
581	0.00916091558216332\\
582	0.00922584619394916\\
583	0.00929124381478611\\
584	0.00935712289323052\\
585	0.00942180913543368\\
586	0.00948442867380405\\
587	0.00954462375269818\\
588	0.00960141065634341\\
589	0.00965385408934247\\
590	0.00970260651805489\\
591	0.00974846666295701\\
592	0.00979195816415004\\
593	0.00983364285089848\\
594	0.00987334736606283\\
595	0.0099111519251561\\
596	0.00994658651044256\\
597	0.00997788999445116\\
598	0.010000292044645\\
599	0\\
600	0\\
};
\addplot [color=mycolor10,solid,forget plot]
  table[row sep=crcr]{%
1	4.64198342570775e-05\\
2	4.64200387921881e-05\\
3	4.64202468284378e-05\\
4	4.64204584259058e-05\\
5	4.64206736456894e-05\\
6	4.64208925499346e-05\\
7	4.6421115201856e-05\\
8	4.64213416657481e-05\\
9	4.64215720070079e-05\\
10	4.64218062921487e-05\\
11	4.6422044588823e-05\\
12	4.64222869658486e-05\\
13	4.64225334932136e-05\\
14	4.64227842421049e-05\\
15	4.64230392849374e-05\\
16	4.64232986953489e-05\\
17	4.64235625482508e-05\\
18	4.6423830919835e-05\\
19	4.64241038875909e-05\\
20	4.6424381530333e-05\\
21	4.64246639282335e-05\\
22	4.64249511628312e-05\\
23	4.64252433170587e-05\\
24	4.64255404752666e-05\\
25	4.64258427232428e-05\\
26	4.64261501482564e-05\\
27	4.64264628390569e-05\\
28	4.6426780885913e-05\\
29	4.64271043806428e-05\\
30	4.64274334166229e-05\\
31	4.64277680888393e-05\\
32	4.6428108493887e-05\\
33	4.64284547300268e-05\\
34	4.64288068971803e-05\\
35	4.64291650969916e-05\\
36	4.6429529432838e-05\\
37	4.64299000098569e-05\\
38	4.64302769349861e-05\\
39	4.64306603169893e-05\\
40	4.64310502664832e-05\\
41	4.64314468959798e-05\\
42	4.64318503199062e-05\\
43	4.64322606546443e-05\\
44	4.64326780185624e-05\\
45	4.64331025320574e-05\\
46	4.64335343175637e-05\\
47	4.64339734996173e-05\\
48	4.64344202048737e-05\\
49	4.6434874562151e-05\\
50	4.6435336702462e-05\\
51	4.64358067590574e-05\\
52	4.64362848674592e-05\\
53	4.6436771165494e-05\\
54	4.64372657933456e-05\\
55	4.64377688935835e-05\\
56	4.64382806112103e-05\\
57	4.64388010936927e-05\\
58	4.6439330491014e-05\\
59	4.64398689557109e-05\\
60	4.64404166429274e-05\\
61	4.6440973710437e-05\\
62	4.64415403187029e-05\\
63	4.64421166309248e-05\\
64	4.64427028130801e-05\\
65	4.64432990339679e-05\\
66	4.64439054652635e-05\\
67	4.64445222815589e-05\\
68	4.64451496604215e-05\\
69	4.64457877824375e-05\\
70	4.64464368312647e-05\\
71	4.64470969936825e-05\\
72	4.64477684596473e-05\\
73	4.64484514223477e-05\\
74	4.64491460782499e-05\\
75	4.64498526271697e-05\\
76	4.64505712723039e-05\\
77	4.64513022203158e-05\\
78	4.64520456813705e-05\\
79	4.64528018692056e-05\\
80	4.64535710011883e-05\\
81	4.64543532983768e-05\\
82	4.64551489855851e-05\\
83	4.645595829144e-05\\
84	4.64567814484457e-05\\
85	4.64576186930545e-05\\
86	4.64584702657316e-05\\
87	4.64593364110224e-05\\
88	4.64602173776148e-05\\
89	4.64611134184232e-05\\
90	4.64620247906381e-05\\
91	4.64629517558166e-05\\
92	4.64638945799521e-05\\
93	4.64648535335428e-05\\
94	4.64658288916688e-05\\
95	4.646682093407e-05\\
96	4.64678299452291e-05\\
97	4.64688562144476e-05\\
98	4.64699000359229e-05\\
99	4.64709617088292e-05\\
100	4.64720415374174e-05\\
101	4.64731398310753e-05\\
102	4.64742569044417e-05\\
103	4.64753930774618e-05\\
104	4.64765486755127e-05\\
105	4.64777240294628e-05\\
106	4.6478919475784e-05\\
107	4.64801353566316e-05\\
108	4.64813720199543e-05\\
109	4.64826298195808e-05\\
110	4.6483909115327e-05\\
111	4.64852102730836e-05\\
112	4.64865336649245e-05\\
113	4.6487879669222e-05\\
114	4.64892486707316e-05\\
115	4.6490641060719e-05\\
116	4.64920572370491e-05\\
117	4.64934976043213e-05\\
118	4.64949625739567e-05\\
119	4.64964525643282e-05\\
120	4.64979680008789e-05\\
121	4.64995093162281e-05\\
122	4.65010769503138e-05\\
123	4.65026713504791e-05\\
124	4.65042929716377e-05\\
125	4.650594227637e-05\\
126	4.65076197350724e-05\\
127	4.65093258260765e-05\\
128	4.65110610357785e-05\\
129	4.65128258587917e-05\\
130	4.65146207980738e-05\\
131	4.65164463650569e-05\\
132	4.6518303079815e-05\\
133	4.65201914711785e-05\\
134	4.65221120769123e-05\\
135	4.65240654438355e-05\\
136	4.65260521279929e-05\\
137	4.65280726947906e-05\\
138	4.65301277191657e-05\\
139	4.65322177857247e-05\\
140	4.65343434889061e-05\\
141	4.65365054331653e-05\\
142	4.65387042331824e-05\\
143	4.65409405141296e-05\\
144	4.65432149117916e-05\\
145	4.65455280724838e-05\\
146	4.65478806534146e-05\\
147	4.65502733228561e-05\\
148	4.65527067603531e-05\\
149	4.65551816568849e-05\\
150	4.65576987150701e-05\\
151	4.65602586493646e-05\\
152	4.65628621862613e-05\\
153	4.65655100644745e-05\\
154	4.65682030351674e-05\\
155	4.65709418621456e-05\\
156	4.65737273220795e-05\\
157	4.65765602047107e-05\\
158	4.65794413130808e-05\\
159	4.65823714637411e-05\\
160	4.65853514869981e-05\\
161	4.65883822271284e-05\\
162	4.65914645426221e-05\\
163	4.6594599306412e-05\\
164	4.65977874061337e-05\\
165	4.66010297443526e-05\\
166	4.66043272388252e-05\\
167	4.6607680822761e-05\\
168	4.66110914450668e-05\\
169	4.66145600706136e-05\\
170	4.66180876805248e-05\\
171	4.66216752724192e-05\\
172	4.66253238607178e-05\\
173	4.66290344769032e-05\\
174	4.66328081698201e-05\\
175	4.6636646005968e-05\\
176	4.66405490697888e-05\\
177	4.66445184639809e-05\\
178	4.66485553098001e-05\\
179	4.66526607473745e-05\\
180	4.66568359360315e-05\\
181	4.66610820545924e-05\\
182	4.66654003017481e-05\\
183	4.66697918963588e-05\\
184	4.66742580778049e-05\\
185	4.6678800106339e-05\\
186	4.6683419263433e-05\\
187	4.6688116852151e-05\\
188	4.66928941974884e-05\\
189	4.66977526467724e-05\\
190	4.67026935700293e-05\\
191	4.67077183603696e-05\\
192	4.67128284343681e-05\\
193	4.67180252324946e-05\\
194	4.67233102194869e-05\\
195	4.67286848847864e-05\\
196	4.67341507429446e-05\\
197	4.67397093340665e-05\\
198	4.67453622242354e-05\\
199	4.67511110059563e-05\\
200	4.67569572986157e-05\\
201	4.67629027489341e-05\\
202	4.6768949031438e-05\\
203	4.67750978489395e-05\\
204	4.67813509330188e-05\\
205	4.67877100445156e-05\\
206	4.67941769740326e-05\\
207	4.68007535424626e-05\\
208	4.68074416014848e-05\\
209	4.68142430341166e-05\\
210	4.68211597552507e-05\\
211	4.68281937122044e-05\\
212	4.68353468852803e-05\\
213	4.68426212883389e-05\\
214	4.68500189693852e-05\\
215	4.68575420111601e-05\\
216	4.68651925317421e-05\\
217	4.68729726851704e-05\\
218	4.68808846620662e-05\\
219	4.68889306902817e-05\\
220	4.6897113035541e-05\\
221	4.69054340021147e-05\\
222	4.69138959334946e-05\\
223	4.69225012130759e-05\\
224	4.69312522648865e-05\\
225	4.69401515542656e-05\\
226	4.69492015886304e-05\\
227	4.69584049182035e-05\\
228	4.69677641367905e-05\\
229	4.6977281882533e-05\\
230	4.69869608387094e-05\\
231	4.69968037345613e-05\\
232	4.70068133460858e-05\\
233	4.70169924968945e-05\\
234	4.70273440590636e-05\\
235	4.70378709540219e-05\\
236	4.70485761534177e-05\\
237	4.70594626800511e-05\\
238	4.70705336088008e-05\\
239	4.70817920675649e-05\\
240	4.70932412382305e-05\\
241	4.71048843576777e-05\\
242	4.71167247187618e-05\\
243	4.71287656713784e-05\\
244	4.71410106234908e-05\\
245	4.71534630422283e-05\\
246	4.71661264549561e-05\\
247	4.71790044504387e-05\\
248	4.71921006799539e-05\\
249	4.7205418858496e-05\\
250	4.72189627659455e-05\\
251	4.72327362483355e-05\\
252	4.72467432190677e-05\\
253	4.72609876602362e-05\\
254	4.7275473623911e-05\\
255	4.72902052335034e-05\\
256	4.73051866851324e-05\\
257	4.73204222490495e-05\\
258	4.73359162710681e-05\\
259	4.73516731740516e-05\\
260	4.73676974594389e-05\\
261	4.73839937087783e-05\\
262	4.74005665853531e-05\\
263	4.74174208357881e-05\\
264	4.7434561291748e-05\\
265	4.74519928716659e-05\\
266	4.74697205825169e-05\\
267	4.74877495216719e-05\\
268	4.75060848788384e-05\\
269	4.7524731938104e-05\\
270	4.75436960801877e-05\\
271	4.75629827848413e-05\\
272	4.75825976331039e-05\\
273	4.76025463083986e-05\\
274	4.76228345948215e-05\\
275	4.76434683760235e-05\\
276	4.76644536527703e-05\\
277	4.76857965380295e-05\\
278	4.77075032589524e-05\\
279	4.77295801595217e-05\\
280	4.77520337032116e-05\\
281	4.77748704757659e-05\\
282	4.77980971880835e-05\\
283	4.78217206792177e-05\\
284	4.78457479194409e-05\\
285	4.78701860134646e-05\\
286	4.78950422037711e-05\\
287	4.79203238740843e-05\\
288	4.79460385529402e-05\\
289	4.7972193917466e-05\\
290	4.79987977972544e-05\\
291	4.80258581784642e-05\\
292	4.80533832080408e-05\\
293	4.80813811981757e-05\\
294	4.81098606309409e-05\\
295	4.81388301631565e-05\\
296	4.81682986314961e-05\\
297	4.81982750578213e-05\\
298	4.82287686548244e-05\\
299	4.82597888319346e-05\\
300	4.82913452015431e-05\\
301	4.8323447585574e-05\\
302	4.83561060223944e-05\\
303	4.83893307740658e-05\\
304	4.8423132333964e-05\\
305	4.84575214346171e-05\\
306	4.84925090557011e-05\\
307	4.85281064320679e-05\\
308	4.85643250620445e-05\\
309	4.86011767177443e-05\\
310	4.86386734621415e-05\\
311	4.8676827678597e-05\\
312	4.87156520978664e-05\\
313	4.87551597564142e-05\\
314	4.8795364022418e-05\\
315	4.88362786247455e-05\\
316	4.88779176705513e-05\\
317	4.89202956642902e-05\\
318	4.89634275283708e-05\\
319	4.9007328625485e-05\\
320	4.90520147829779e-05\\
321	4.90975023193e-05\\
322	4.91438080728845e-05\\
323	4.91909494336969e-05\\
324	4.92389443776877e-05\\
325	4.92878115045817e-05\\
326	4.93375700792812e-05\\
327	4.93882400773528e-05\\
328	4.94398422350384e-05\\
329	4.94923981043484e-05\\
330	4.95459301137594e-05\\
331	4.96004616352609e-05\\
332	4.96560170583851e-05\\
333	4.97126218722017e-05\\
334	4.97703027560492e-05\\
335	4.98290876801908e-05\\
336	4.9889006017507e-05\\
337	4.99500886676085e-05\\
338	5.00123681948524e-05\\
339	5.00758789819625e-05\\
340	5.01406574012291e-05\\
341	5.02067420053495e-05\\
342	5.02741737404943e-05\\
343	5.034299618427e-05\\
344	5.04132558117271e-05\\
345	5.04850022930345e-05\\
346	5.05582888267819e-05\\
347	5.06331725135548e-05\\
348	5.07097147751425e-05\\
349	5.07879818256077e-05\\
350	5.08680452019124e-05\\
351	5.09499823634595e-05\\
352	5.10338773721552e-05\\
353	5.11198216630597e-05\\
354	5.1207914899369e-05\\
355	5.12982658542689e-05\\
356	5.13909932478607e-05\\
357	5.14862270201235e-05\\
358	5.15841102216357e-05\\
359	5.16848002389749e-05\\
360	5.17884705448819e-05\\
361	5.18953126650908e-05\\
362	5.20055383462529e-05\\
363	5.21193817847083e-05\\
364	5.22371013522779e-05\\
365	5.23589791632466e-05\\
366	5.24853177180695e-05\\
367	5.26164927355185e-05\\
368	5.2754106380934e-05\\
369	5.33348951666007e-05\\
370	5.3957204054167e-05\\
371	5.45896062072596e-05\\
372	5.52322612354649e-05\\
373	5.588533110313e-05\\
374	5.65489803489557e-05\\
375	5.72233768696586e-05\\
376	5.79086938147883e-05\\
377	5.86051095135397e-05\\
378	5.93127818377497e-05\\
379	6.00318790577998e-05\\
380	6.07625838000523e-05\\
381	6.15050821419064e-05\\
382	6.22595638379744e-05\\
383	6.30262225781473e-05\\
384	6.38052562815622e-05\\
385	6.4596867430693e-05\\
386	6.54012634503663e-05\\
387	6.62186571366786e-05\\
388	6.70492671414194e-05\\
389	6.78933185177992e-05\\
390	6.87510433338879e-05\\
391	6.96226813603081e-05\\
392	7.05084808391372e-05\\
393	7.14086993412281e-05\\
394	7.23236047190819e-05\\
395	7.32534761625014e-05\\
396	7.41986053638208e-05\\
397	7.51592977986433e-05\\
398	7.6135874125322e-05\\
399	7.71286716975121e-05\\
400	7.81380461538272e-05\\
401	7.91643729526588e-05\\
402	8.02080484589262e-05\\
403	8.12694897303985e-05\\
404	8.23491329950335e-05\\
405	8.3447444861849e-05\\
406	8.45650264339686e-05\\
407	8.57024394423497e-05\\
408	8.68602485026623e-05\\
409	8.80390433137111e-05\\
410	8.92394189516993e-05\\
411	9.04619516920421e-05\\
412	9.17075342590605e-05\\
413	9.29772832542274e-05\\
414	9.42715832514439e-05\\
415	9.5590917679188e-05\\
416	9.69359307017726e-05\\
417	9.83072918494114e-05\\
418	9.9705696704008e-05\\
419	0.000101131867124936\\
420	0.000102586552091485\\
421	0.000104070534828886\\
422	0.000105584656761338\\
423	0.000107129805491077\\
424	0.000108706888637646\\
425	0.000110316857657739\\
426	0.000111960711090713\\
427	0.000113639498139863\\
428	0.00011535432263472\\
429	0.000117106347427119\\
430	0.000118896799282379\\
431	0.000120726974337093\\
432	0.000122598244207171\\
433	0.000124512062844152\\
434	0.000126469974254772\\
435	0.000128473621219041\\
436	0.000130524755165883\\
437	0.000132625247394223\\
438	0.000134777101861739\\
439	0.000136982469809312\\
440	0.000139243666565943\\
441	0.000141563191083511\\
442	0.000143943749479445\\
443	0.000146388286722446\\
444	0.000148900041242269\\
445	0.00015148266725791\\
446	0.000154140434601715\\
447	0.000156876604403537\\
448	0.000159695618446798\\
449	0.000162602583136702\\
450	0.000165603269980838\\
451	0.000168704141583991\\
452	0.000171912461067029\\
453	0.000175236415477924\\
454	0.00017868524931195\\
455	0.000182269412078902\\
456	0.000186000824948488\\
457	0.000189893986741423\\
458	0.000193970883317604\\
459	0.000228095183085844\\
460	0.00026877496219362\\
461	0.000310340054739103\\
462	0.000352813735782594\\
463	0.000396216382073708\\
464	0.000440565637866915\\
465	0.000485877110274124\\
466	0.000532160774069255\\
467	0.000579418257293104\\
468	0.00062738419818233\\
469	0.00067622482309534\\
470	0.000726048322986633\\
471	0.000776839706170607\\
472	0.000828636037132636\\
473	0.00088147678700264\\
474	0.000935403965952333\\
475	0.000990462253691594\\
476	0.00104669910155119\\
477	0.00110416455205536\\
478	0.00116291308479596\\
479	0.00122300222748416\\
480	0.00128449561765596\\
481	0.00134744304775154\\
482	0.00141189995900876\\
483	0.00147792462610287\\
484	0.00154557809802488\\
485	0.00161492391398022\\
486	0.0016860270809891\\
487	0.00175895033040454\\
488	0.00183373919105755\\
489	0.00190806619835934\\
490	0.0019832894053654\\
491	0.00206057427645102\\
492	0.00214005154122694\\
493	0.00222186862771036\\
494	0.00230619258930748\\
495	0.00239321367185647\\
496	0.00248314968594651\\
497	0.00257625145724361\\
498	0.00267280965494731\\
499	0.00277316374541858\\
500	0.0028777121918144\\
501	0.00298693285794693\\
502	0.0031013825506269\\
503	0.00322166732056289\\
504	0.00334849896939119\\
505	0.00348271740882836\\
506	0.00362027029291665\\
507	0.00376105869656856\\
508	0.00388611688136062\\
509	0.0039572418151582\\
510	0.00402944410584187\\
511	0.00410270379462366\\
512	0.00417699316230488\\
513	0.00425227577011228\\
514	0.00432850697474614\\
515	0.00440563912522553\\
516	0.00448365029461639\\
517	0.0045626587564041\\
518	0.00464264237219054\\
519	0.00472351612283466\\
520	0.00480517248048174\\
521	0.00488747531542428\\
522	0.00497025317401618\\
523	0.00505329154666099\\
524	0.00513632313230388\\
525	0.00521901734880742\\
526	0.00530096376322706\\
527	0.00538166427589422\\
528	0.00546050799752127\\
529	0.00553673733990381\\
530	0.00560940948104321\\
531	0.00567736265017387\\
532	0.00574657938646294\\
533	0.00581728614637982\\
534	0.0058895628459596\\
535	0.0059634355660333\\
536	0.00603886694478586\\
537	0.00611577996857281\\
538	0.00619404396421993\\
539	0.006273456223647\\
540	0.0063537180019648\\
541	0.00643455409956856\\
542	0.00651658735225979\\
543	0.00660009219092987\\
544	0.00668464219506546\\
545	0.00676848387754154\\
546	0.00685086218497004\\
547	0.00693155004757897\\
548	0.00701032202772476\\
549	0.00708696541304999\\
550	0.00716129639164414\\
551	0.00723318316095625\\
552	0.00730257843968756\\
553	0.00736952647902213\\
554	0.00743402578181928\\
555	0.00749607700169981\\
556	0.00755586493724629\\
557	0.00761541016178557\\
558	0.00767469271572267\\
559	0.00773370315312266\\
560	0.00779244441339949\\
561	0.00785091703370655\\
562	0.00790914232129952\\
563	0.00796716616600809\\
564	0.00802506207440758\\
565	0.00808293312451273\\
566	0.00814091167253474\\
567	0.00819915577571846\\
568	0.00825784157242284\\
569	0.00831701598922605\\
570	0.0083766845918232\\
571	0.00843685157104802\\
572	0.00849751790386216\\
573	0.00855867710387677\\
574	0.00862031664618866\\
575	0.00868242246674999\\
576	0.0087449779716787\\
577	0.00880796255215469\\
578	0.0088713498714046\\
579	0.00893510609465251\\
580	0.00899918825703781\\
581	0.00906354428152085\\
582	0.00912811788932691\\
583	0.00919285217075625\\
584	0.00925768822972085\\
585	0.00932258151766213\\
586	0.00938749662428032\\
587	0.00945240446296449\\
588	0.00951729025111872\\
589	0.0095821249003219\\
590	0.00964561720054655\\
591	0.0097066206152574\\
592	0.00976430172597694\\
593	0.00981781039518062\\
594	0.00986668609051667\\
595	0.00990919367976287\\
596	0.009946404199063\\
597	0.00997788999445116\\
598	0.010000292044645\\
599	0\\
600	0\\
};
\addplot [color=mycolor11,solid,forget plot]
  table[row sep=crcr]{%
1	2.87074056288534e-05\\
2	2.87074219770342e-05\\
3	2.8707438605019e-05\\
4	2.87074555176098e-05\\
5	2.87074727196844e-05\\
6	2.87074902162143e-05\\
7	2.87075080122498e-05\\
8	2.87075261129244e-05\\
9	2.87075445234721e-05\\
10	2.87075632491999e-05\\
11	2.87075822955249e-05\\
12	2.87076016679376e-05\\
13	2.87076213720374e-05\\
14	2.87076414135115e-05\\
15	2.87076617981448e-05\\
16	2.87076825318254e-05\\
17	2.87077036205368e-05\\
18	2.87077250703705e-05\\
19	2.87077468875226e-05\\
20	2.8707769078286e-05\\
21	2.87077916490701e-05\\
22	2.87078146063905e-05\\
23	2.87078379568757e-05\\
24	2.87078617072681e-05\\
25	2.87078858644229e-05\\
26	2.87079104353101e-05\\
27	2.87079354270323e-05\\
28	2.87079608467935e-05\\
29	2.8707986701937e-05\\
30	2.87080129999283e-05\\
31	2.87080397483458e-05\\
32	2.87080669549237e-05\\
33	2.87080946275071e-05\\
34	2.87081227740844e-05\\
35	2.87081514027762e-05\\
36	2.87081805218467e-05\\
37	2.87082101397009e-05\\
38	2.87082402648758e-05\\
39	2.87082709060712e-05\\
40	2.87083020721246e-05\\
41	2.87083337720337e-05\\
42	2.87083660149329e-05\\
43	2.87083988101306e-05\\
44	2.87084321670868e-05\\
45	2.87084660954185e-05\\
46	2.87085006049105e-05\\
47	2.87085357055212e-05\\
48	2.87085714073691e-05\\
49	2.87086077207499e-05\\
50	2.87086446561366e-05\\
51	2.87086822241715e-05\\
52	2.87087204356876e-05\\
53	2.87087593017029e-05\\
54	2.87087988334197e-05\\
55	2.87088390422278e-05\\
56	2.87088799397196e-05\\
57	2.87089215376791e-05\\
58	2.87089638480933e-05\\
59	2.87090068831497e-05\\
60	2.87090506552488e-05\\
61	2.87090951769986e-05\\
62	2.87091404612286e-05\\
63	2.87091865209722e-05\\
64	2.87092333694985e-05\\
65	2.87092810202971e-05\\
66	2.87093294870827e-05\\
67	2.87093787838186e-05\\
68	2.87094289246864e-05\\
69	2.87094799241193e-05\\
70	2.87095317967987e-05\\
71	2.87095845576489e-05\\
72	2.87096382218437e-05\\
73	2.87096928048299e-05\\
74	2.87097483223065e-05\\
75	2.87098047902417e-05\\
76	2.8709862224873e-05\\
77	2.87099206427135e-05\\
78	2.87099800605614e-05\\
79	2.87100404954949e-05\\
80	2.87101019648874e-05\\
81	2.87101644864034e-05\\
82	2.87102280780061e-05\\
83	2.8710292757966e-05\\
84	2.87103585448645e-05\\
85	2.87104254575978e-05\\
86	2.87104935153779e-05\\
87	2.87105627377484e-05\\
88	2.87106331445835e-05\\
89	2.8710704756096e-05\\
90	2.87107775928283e-05\\
91	2.87108516756889e-05\\
92	2.87109270259289e-05\\
93	2.87110036651669e-05\\
94	2.87110816153827e-05\\
95	2.87111608989232e-05\\
96	2.87112415385331e-05\\
97	2.87113235573141e-05\\
98	2.87114069787841e-05\\
99	2.87114918268461e-05\\
100	2.87115781258095e-05\\
101	2.87116659003968e-05\\
102	2.87117551757434e-05\\
103	2.87118459774197e-05\\
104	2.87119383314177e-05\\
105	2.87120322641772e-05\\
106	2.871212780258e-05\\
107	2.87122249739635e-05\\
108	2.87123238061285e-05\\
109	2.87124243273415e-05\\
110	2.8712526566356e-05\\
111	2.87126305524002e-05\\
112	2.87127363151966e-05\\
113	2.8712843884981e-05\\
114	2.87129532924846e-05\\
115	2.87130645689699e-05\\
116	2.87131777462208e-05\\
117	2.87132928565585e-05\\
118	2.87134099328484e-05\\
119	2.87135290085129e-05\\
120	2.8713650117539e-05\\
121	2.87137732944825e-05\\
122	2.87138985744863e-05\\
123	2.87140259932864e-05\\
124	2.87141555872151e-05\\
125	2.87142873932249e-05\\
126	2.87144214488846e-05\\
127	2.87145577924034e-05\\
128	2.87146964626287e-05\\
129	2.87148374990611e-05\\
130	2.87149809418793e-05\\
131	2.87151268319188e-05\\
132	2.87152752107225e-05\\
133	2.87154261205211e-05\\
134	2.87155796042571e-05\\
135	2.87157357055965e-05\\
136	2.87158944689389e-05\\
137	2.87160559394325e-05\\
138	2.87162201629782e-05\\
139	2.87163871862413e-05\\
140	2.87165570566855e-05\\
141	2.87167298225847e-05\\
142	2.87169055330413e-05\\
143	2.87170842379845e-05\\
144	2.87172659881683e-05\\
145	2.87174508351859e-05\\
146	2.87176388315083e-05\\
147	2.87178300304862e-05\\
148	2.87180244863631e-05\\
149	2.87182222542978e-05\\
150	2.87184233903808e-05\\
151	2.87186279516347e-05\\
152	2.87188359960527e-05\\
153	2.8719047582593e-05\\
154	2.87192627712097e-05\\
155	2.87194816228568e-05\\
156	2.87197041995304e-05\\
157	2.87199305642401e-05\\
158	2.87201607810818e-05\\
159	2.87203949152054e-05\\
160	2.87206330328721e-05\\
161	2.87208752014378e-05\\
162	2.87211214894095e-05\\
163	2.87213719664297e-05\\
164	2.87216267033141e-05\\
165	2.87218857720663e-05\\
166	2.87221492458993e-05\\
167	2.87224171992527e-05\\
168	2.87226897078212e-05\\
169	2.87229668485622e-05\\
170	2.87232486997237e-05\\
171	2.87235353408754e-05\\
172	2.87238268529116e-05\\
173	2.87241233180891e-05\\
174	2.87244248200478e-05\\
175	2.87247314438186e-05\\
176	2.87250432758698e-05\\
177	2.87253604041133e-05\\
178	2.8725682917941e-05\\
179	2.8726010908238e-05\\
180	2.8726344467414e-05\\
181	2.87266836894331e-05\\
182	2.8727028669831e-05\\
183	2.87273795057451e-05\\
184	2.8727736295944e-05\\
185	2.87280991408507e-05\\
186	2.87284681425684e-05\\
187	2.87288434049218e-05\\
188	2.87292250334677e-05\\
189	2.87296131355383e-05\\
190	2.87300078202593e-05\\
191	2.87304091985894e-05\\
192	2.87308173833562e-05\\
193	2.87312324892674e-05\\
194	2.87316546329607e-05\\
195	2.87320839330218e-05\\
196	2.87325205100417e-05\\
197	2.87329644866219e-05\\
198	2.87334159874207e-05\\
199	2.87338751391964e-05\\
200	2.87343420708235e-05\\
201	2.87348169133468e-05\\
202	2.87352998000084e-05\\
203	2.87357908662871e-05\\
204	2.87362902499362e-05\\
205	2.87367980910208e-05\\
206	2.87373145319596e-05\\
207	2.87378397175658e-05\\
208	2.87383737950801e-05\\
209	2.87389169142191e-05\\
210	2.87394692272215e-05\\
211	2.87400308888777e-05\\
212	2.87406020565832e-05\\
213	2.87411828903803e-05\\
214	2.87417735530064e-05\\
215	2.87423742099366e-05\\
216	2.87429850294291e-05\\
217	2.87436061825764e-05\\
218	2.87442378433557e-05\\
219	2.87448801886679e-05\\
220	2.87455333984064e-05\\
221	2.87461976554881e-05\\
222	2.87468731459194e-05\\
223	2.87475600588475e-05\\
224	2.8748258586608e-05\\
225	2.87489689247892e-05\\
226	2.87496912722746e-05\\
227	2.8750425831325e-05\\
228	2.87511728076093e-05\\
229	2.87519324102845e-05\\
230	2.87527048520416e-05\\
231	2.8753490349177e-05\\
232	2.87542891216552e-05\\
233	2.87551013931665e-05\\
234	2.87559273912032e-05\\
235	2.87567673471142e-05\\
236	2.87576214961827e-05\\
237	2.87584900776872e-05\\
238	2.87593733349814e-05\\
239	2.87602715155676e-05\\
240	2.87611848711529e-05\\
241	2.87621136577445e-05\\
242	2.87630581357161e-05\\
243	2.87640185698889e-05\\
244	2.8764995229616e-05\\
245	2.87659883888463e-05\\
246	2.87669983262339e-05\\
247	2.87680253252047e-05\\
248	2.87690696740387e-05\\
249	2.87701316659861e-05\\
250	2.87712115993204e-05\\
251	2.87723097774585e-05\\
252	2.87734265090484e-05\\
253	2.87745621080573e-05\\
254	2.87757168938728e-05\\
255	2.87768911914122e-05\\
256	2.87780853312216e-05\\
257	2.8779299649569e-05\\
258	2.87805344885712e-05\\
259	2.87817901962847e-05\\
260	2.87830671268389e-05\\
261	2.87843656405262e-05\\
262	2.87856861039463e-05\\
263	2.87870288901108e-05\\
264	2.87883943785704e-05\\
265	2.87897829555578e-05\\
266	2.87911950141216e-05\\
267	2.87926309542748e-05\\
268	2.87940911831882e-05\\
269	2.87955761153534e-05\\
270	2.87970861727768e-05\\
271	2.8798621785011e-05\\
272	2.88001833889952e-05\\
273	2.88017714286501e-05\\
274	2.8803386355146e-05\\
275	2.88050286289825e-05\\
276	2.88066987190239e-05\\
277	2.88083971026034e-05\\
278	2.88101242656961e-05\\
279	2.88118807031026e-05\\
280	2.88136669186484e-05\\
281	2.88154834253669e-05\\
282	2.88173307457072e-05\\
283	2.88192094117412e-05\\
284	2.88211199653794e-05\\
285	2.8823062958591e-05\\
286	2.88250389536283e-05\\
287	2.8827048523281e-05\\
288	2.88290922510929e-05\\
289	2.88311707316444e-05\\
290	2.88332845708011e-05\\
291	2.88354343859876e-05\\
292	2.88376208064754e-05\\
293	2.88398444736696e-05\\
294	2.88421060414301e-05\\
295	2.88444061763612e-05\\
296	2.88467455581792e-05\\
297	2.88491248800287e-05\\
298	2.88515448488514e-05\\
299	2.88540061857641e-05\\
300	2.88565096264286e-05\\
301	2.88590559214596e-05\\
302	2.88616458368236e-05\\
303	2.88642801541959e-05\\
304	2.88669596713435e-05\\
305	2.88696852024063e-05\\
306	2.88724575782161e-05\\
307	2.88752776467553e-05\\
308	2.88781462742516e-05\\
309	2.88810643474518e-05\\
310	2.88840327767239e-05\\
311	2.88870524963344e-05\\
312	2.88901244576064e-05\\
313	2.88932496319469e-05\\
314	2.88964290135874e-05\\
315	2.88996636204524e-05\\
316	2.89029544951364e-05\\
317	2.89063027058867e-05\\
318	2.89097093477215e-05\\
319	2.8913175543578e-05\\
320	2.89167024455904e-05\\
321	2.89202912364288e-05\\
322	2.89239431307811e-05\\
323	2.89276593769228e-05\\
324	2.89314412584406e-05\\
325	2.89352900960943e-05\\
326	2.8939207249852e-05\\
327	2.89431941210691e-05\\
328	2.89472521549184e-05\\
329	2.89513828430158e-05\\
330	2.89555877262789e-05\\
331	2.89598683980968e-05\\
332	2.89642265077754e-05\\
333	2.89686637643585e-05\\
334	2.89731819407897e-05\\
335	2.89777828785489e-05\\
336	2.89824684927481e-05\\
337	2.89872407777614e-05\\
338	2.89921018134589e-05\\
339	2.8997053772138e-05\\
340	2.90020989261909e-05\\
341	2.9007239656636e-05\\
342	2.90124784626291e-05\\
343	2.90178179720329e-05\\
344	2.90232609532281e-05\\
345	2.90288103283296e-05\\
346	2.90344691880262e-05\\
347	2.90402408082604e-05\\
348	2.90461286691986e-05\\
349	2.90521364769399e-05\\
350	2.90582681884599e-05\\
351	2.90645280401054e-05\\
352	2.90709205783075e-05\\
353	2.90774506878393e-05\\
354	2.90841236102916e-05\\
355	2.90909449613259e-05\\
356	2.90979208100995e-05\\
357	2.91050578025798e-05\\
358	2.91123631980276e-05\\
359	2.91198450062145e-05\\
360	2.91275122628458e-05\\
361	2.91353756940605e-05\\
362	2.9143449519973e-05\\
363	2.91517566766702e-05\\
364	2.91603444727928e-05\\
365	2.91693323614963e-05\\
366	2.91790572625571e-05\\
367	2.91904883455599e-05\\
368	2.92059904748458e-05\\
369	2.92220118418757e-05\\
370	2.92382851160593e-05\\
371	2.92548141350258e-05\\
372	2.92716028900053e-05\\
373	2.92886556623828e-05\\
374	2.93059772348849e-05\\
375	2.93235729474886e-05\\
376	2.93414476655059e-05\\
377	2.93596027275639e-05\\
378	2.93780407253037e-05\\
379	2.93967659119906e-05\\
380	2.94157826108006e-05\\
381	2.943509521892e-05\\
382	2.94547082122986e-05\\
383	2.94746261510439e-05\\
384	2.94948536856192e-05\\
385	2.95153955639028e-05\\
386	2.95362566391953e-05\\
387	2.95574418793126e-05\\
388	2.95789563768745e-05\\
389	2.96008053608892e-05\\
390	2.96229942097429e-05\\
391	2.96455284656924e-05\\
392	2.96684138507856e-05\\
393	2.96916562839949e-05\\
394	2.97152618986561e-05\\
395	2.97392370579873e-05\\
396	2.97635883633745e-05\\
397	2.97883226435484e-05\\
398	2.98134469000132e-05\\
399	2.983896816252e-05\\
400	2.98648931854672e-05\\
401	2.98912279412757e-05\\
402	2.9917977129054e-05\\
403	2.99451449001304e-05\\
404	2.99727400478062e-05\\
405	3.00007863959452e-05\\
406	3.00292971289041e-05\\
407	3.00582798077221e-05\\
408	3.00877378599853e-05\\
409	3.01176675906268e-05\\
410	3.01480642028151e-05\\
411	3.01789779503984e-05\\
412	3.02104853764506e-05\\
413	3.02425629355838e-05\\
414	3.02752030331295e-05\\
415	3.03084177303424e-05\\
416	3.03422192657478e-05\\
417	3.03766199085939e-05\\
418	3.04116318696233e-05\\
419	3.04472676443895e-05\\
420	3.04835414641953e-05\\
421	3.0520471431376e-05\\
422	3.05580764393456e-05\\
423	3.05963730460801e-05\\
424	3.06353785031662e-05\\
425	3.0675110802949e-05\\
426	3.07155887302112e-05\\
427	3.0756831918944e-05\\
428	3.07988609148731e-05\\
429	3.08416972445709e-05\\
430	3.08853634919592e-05\\
431	3.09298833833985e-05\\
432	3.09752818826228e-05\\
433	3.1021585297479e-05\\
434	3.1068821401146e-05\\
435	3.11170195730472e-05\\
436	3.11662109697854e-05\\
437	3.12164287501284e-05\\
438	3.12677084112617e-05\\
439	3.1320088374597e-05\\
440	3.13736111458136e-05\\
441	3.14283257581362e-05\\
442	3.14842927968935e-05\\
443	3.15415932268637e-05\\
444	3.16003361893756e-05\\
445	3.16606287197654e-05\\
446	3.17223983966092e-05\\
447	3.17856802878807e-05\\
448	3.18505393092603e-05\\
449	3.19170570309623e-05\\
450	3.19853231436586e-05\\
451	3.20554371384342e-05\\
452	3.21275132520946e-05\\
453	3.22016981889936e-05\\
454	3.22782337643279e-05\\
455	3.23576621816601e-05\\
456	3.24414281777169e-05\\
457	3.2533347786978e-05\\
458	3.26421118733469e-05\\
459	3.27552039992171e-05\\
460	3.28698685584011e-05\\
461	3.29864062298302e-05\\
462	3.31050456097429e-05\\
463	3.32259930698725e-05\\
464	3.33497356061983e-05\\
465	3.34764470236495e-05\\
466	3.36068489184736e-05\\
467	3.37469988655559e-05\\
468	3.41714855216607e-05\\
469	3.47180892944729e-05\\
470	3.52757507581536e-05\\
471	3.58451339388148e-05\\
472	3.64269757434914e-05\\
473	3.70220954982187e-05\\
474	3.76314063928785e-05\\
475	3.82559312065372e-05\\
476	3.88968456767497e-05\\
477	3.95557883791894e-05\\
478	4.02206634430905e-05\\
479	4.09064513907733e-05\\
480	4.16150064117664e-05\\
481	4.23479713284909e-05\\
482	4.31071548316835e-05\\
483	4.38945480553534e-05\\
484	4.47123580236007e-05\\
485	4.55631273707271e-05\\
486	4.6450251439557e-05\\
487	4.73802971189456e-05\\
488	4.83744473987448e-05\\
489	5.18255247195229e-05\\
490	5.64432043508488e-05\\
491	6.11757869814177e-05\\
492	6.60287180328892e-05\\
493	7.10079443662274e-05\\
494	7.61199865734989e-05\\
495	8.13720243861874e-05\\
496	8.67719896607228e-05\\
497	9.23286159209255e-05\\
498	9.80514008825647e-05\\
499	0.000103950459488896\\
500	0.000110039573939583\\
501	0.000116338196138091\\
502	0.000122866088819676\\
503	0.000129603148917134\\
504	0.000136554914834019\\
505	0.000143792738973412\\
506	0.000151348309712374\\
507	0.000159250000101562\\
508	0.000186465740107389\\
509	0.000270913190465583\\
510	0.000357339833777734\\
511	0.000445850217445032\\
512	0.000536559442871119\\
513	0.000629595116395348\\
514	0.000725100692200193\\
515	0.000823235332269393\\
516	0.00092418800105013\\
517	0.00102815991107104\\
518	0.00113536914372073\\
519	0.00124605849399606\\
520	0.00136049122063145\\
521	0.00147897623912208\\
522	0.0016018593706772\\
523	0.00172952379390597\\
524	0.00186241913462318\\
525	0.00200099235793762\\
526	0.00214576766890253\\
527	0.00229735402594387\\
528	0.00245567253882486\\
529	0.00262203513585597\\
530	0.00279776832073401\\
531	0.0029840588322844\\
532	0.00317503465125662\\
533	0.00336753682105394\\
534	0.00356080563091648\\
535	0.00375884896364931\\
536	0.00396189643029828\\
537	0.00417020157699044\\
538	0.00438404649089458\\
539	0.00460374472348798\\
540	0.00482962825192461\\
541	0.00498742931533325\\
542	0.0051011079388929\\
543	0.00521618695727248\\
544	0.00533272468242733\\
545	0.00545049487325927\\
546	0.00556921337530202\\
547	0.00568850362809646\\
548	0.00580786947567521\\
549	0.00592666237003143\\
550	0.00604404005502914\\
551	0.00615891412564674\\
552	0.00626988461770694\\
553	0.00637687442962497\\
554	0.00648639688360578\\
555	0.00659810957107587\\
556	0.00671146170833344\\
557	0.00682409800646433\\
558	0.00693572396512232\\
559	0.00704601191043983\\
560	0.00715459986066793\\
561	0.00726109172587377\\
562	0.00736505911124072\\
563	0.00746604653227905\\
564	0.00756358087956845\\
565	0.00765718855990147\\
566	0.0077464333920976\\
567	0.00783089357803292\\
568	0.00791059914192856\\
569	0.00798962772058366\\
570	0.00806803480922161\\
571	0.00814585569413195\\
572	0.00822322735200265\\
573	0.00830037097742555\\
574	0.00837736890349071\\
575	0.00845418668081767\\
576	0.00853079440325115\\
577	0.00860717799344965\\
578	0.00868333907131728\\
579	0.00875929228624666\\
580	0.00883505938019689\\
581	0.00891063794895334\\
582	0.00898592552194447\\
583	0.00906078197573007\\
584	0.00913506393062985\\
585	0.00920862685199412\\
586	0.00928132795734763\\
587	0.00935303006810276\\
588	0.00942360690540113\\
589	0.00949294844638006\\
590	0.00956097836809741\\
591	0.00962766009652905\\
592	0.0096930024817262\\
593	0.00975706769624642\\
594	0.00981997518922992\\
595	0.00988192166735141\\
596	0.00993785230806569\\
597	0.00997762486405646\\
598	0.010000292044645\\
599	0\\
600	0\\
};
\addplot [color=mycolor12,solid,forget plot]
  table[row sep=crcr]{%
1	4.34936132690398e-07\\
2	4.34936809193483e-07\\
3	4.34937497278928e-07\\
4	4.34938197143251e-07\\
5	4.34938908987252e-07\\
6	4.34939633019321e-07\\
7	4.34940369445737e-07\\
8	4.34941118477938e-07\\
9	4.34941880336154e-07\\
10	4.34942655238677e-07\\
11	4.34943443407577e-07\\
12	4.3494424507319e-07\\
13	4.34945060464087e-07\\
14	4.34945889817803e-07\\
15	4.34946733373395e-07\\
16	4.34947591374897e-07\\
17	4.34948464068387e-07\\
18	4.34949351704745e-07\\
19	4.34950254543989e-07\\
20	4.34951172843841e-07\\
21	4.34952106870986e-07\\
22	4.34953056895178e-07\\
23	4.34954023186657e-07\\
24	4.34955006031013e-07\\
25	4.34956005704797e-07\\
26	4.34957022502341e-07\\
27	4.34958056714123e-07\\
28	4.34959108639579e-07\\
29	4.34960178582937e-07\\
30	4.34961266850286e-07\\
31	4.34962373756493e-07\\
32	4.34963499624852e-07\\
33	4.3496464477342e-07\\
34	4.34965809537505e-07\\
35	4.34966994252714e-07\\
36	4.34968199259269e-07\\
37	4.34969424904786e-07\\
38	4.34970671543397e-07\\
39	4.34971939534548e-07\\
40	4.34973229244549e-07\\
41	4.34974541045021e-07\\
42	4.34975875314098e-07\\
43	4.34977232435398e-07\\
44	4.34978612800955e-07\\
45	4.34980016810016e-07\\
46	4.34981444864525e-07\\
47	4.34982897379353e-07\\
48	4.34984374772243e-07\\
49	4.34985877466414e-07\\
50	4.34987405898337e-07\\
51	4.34988960505118e-07\\
52	4.34990541739016e-07\\
53	4.34992150052223e-07\\
54	4.34993785907593e-07\\
55	4.34995449777764e-07\\
56	4.34997142144305e-07\\
57	4.3499886349079e-07\\
58	4.35000614313864e-07\\
59	4.35002395115809e-07\\
60	4.35004206414401e-07\\
61	4.35006048726993e-07\\
62	4.35007922585395e-07\\
63	4.35009828529289e-07\\
64	4.35011767105724e-07\\
65	4.35013738873596e-07\\
66	4.35015744402102e-07\\
67	4.35017784262257e-07\\
68	4.35019859047303e-07\\
69	4.35021969349018e-07\\
70	4.35024115777591e-07\\
71	4.35026298947283e-07\\
72	4.35028519489035e-07\\
73	4.35030778037682e-07\\
74	4.35033075244735e-07\\
75	4.35035411769056e-07\\
76	4.35037788284799e-07\\
77	4.35040205472766e-07\\
78	4.35042664024561e-07\\
79	4.35045164650362e-07\\
80	4.35047708065609e-07\\
81	4.35050295002757e-07\\
82	4.35052926203153e-07\\
83	4.35055602421873e-07\\
84	4.35058324427544e-07\\
85	4.3506109300079e-07\\
86	4.3506390893648e-07\\
87	4.35066773040953e-07\\
88	4.35069686137898e-07\\
89	4.3507264906006e-07\\
90	4.35075662658212e-07\\
91	4.35078727795986e-07\\
92	4.35081845350373e-07\\
93	4.35085016219171e-07\\
94	4.35088241305231e-07\\
95	4.35091521535492e-07\\
96	4.35094857850591e-07\\
97	4.35098251201583e-07\\
98	4.3510170256187e-07\\
99	4.35105212923911e-07\\
100	4.35108783283132e-07\\
101	4.351124146682e-07\\
102	4.35116108116108e-07\\
103	4.35119864682721e-07\\
104	4.3512368544278e-07\\
105	4.35127571486957e-07\\
106	4.35131523925825e-07\\
107	4.35135543891247e-07\\
108	4.35139632531871e-07\\
109	4.35143791013824e-07\\
110	4.35148020525372e-07\\
111	4.3515232227381e-07\\
112	4.35156697488917e-07\\
113	4.35161147417243e-07\\
114	4.35165673328162e-07\\
115	4.35170276514907e-07\\
116	4.35174958289548e-07\\
117	4.35179719990259e-07\\
118	4.35184562968507e-07\\
119	4.35189488609126e-07\\
120	4.35194498313168e-07\\
121	4.35199593511938e-07\\
122	4.35204775653304e-07\\
123	4.35210046214503e-07\\
124	4.3521540669677e-07\\
125	4.35220858624137e-07\\
126	4.35226403551018e-07\\
127	4.35232043052886e-07\\
128	4.35237778735943e-07\\
129	4.35243612230372e-07\\
130	4.35249545194144e-07\\
131	4.35255579316124e-07\\
132	4.35261716310876e-07\\
133	4.3526795792039e-07\\
134	4.35274305919615e-07\\
135	4.35280762109189e-07\\
136	4.35287328329096e-07\\
137	4.35294006434969e-07\\
138	4.35300798329048e-07\\
139	4.35307705933702e-07\\
140	4.35314731213234e-07\\
141	4.35321876155703e-07\\
142	4.35329142791769e-07\\
143	4.35336533180348e-07\\
144	4.35344049414302e-07\\
145	4.35351693624241e-07\\
146	4.35359467977315e-07\\
147	4.3536737467287e-07\\
148	4.35375415953705e-07\\
149	4.3538359409498e-07\\
150	4.35391911409577e-07\\
151	4.35400370252256e-07\\
152	4.35408973016518e-07\\
153	4.35417722137904e-07\\
154	4.35426620087401e-07\\
155	4.35435669381134e-07\\
156	4.35444872576374e-07\\
157	4.35454232275347e-07\\
158	4.35463751121059e-07\\
159	4.35473431803189e-07\\
160	4.35483277054617e-07\\
161	4.35493289654351e-07\\
162	4.35503472429955e-07\\
163	4.35513828253031e-07\\
164	4.35524360045623e-07\\
165	4.35535070777784e-07\\
166	4.35545963471205e-07\\
167	4.35557041193836e-07\\
168	4.35568307070617e-07\\
169	4.35579764274289e-07\\
170	4.35591416033009e-07\\
171	4.35603265628087e-07\\
172	4.35615316395709e-07\\
173	4.3562757173091e-07\\
174	4.35640035078223e-07\\
175	4.35652709949485e-07\\
176	4.35665599906192e-07\\
177	4.35678708575397e-07\\
178	4.35692039642616e-07\\
179	4.35705596855099e-07\\
180	4.35719384024077e-07\\
181	4.35733405023703e-07\\
182	4.35747663792794e-07\\
183	4.35762164332894e-07\\
184	4.35776910721599e-07\\
185	4.35791907093346e-07\\
186	4.35807157659635e-07\\
187	4.35822666701067e-07\\
188	4.35838438566988e-07\\
189	4.35854477678761e-07\\
190	4.35870788540484e-07\\
191	4.35887375721681e-07\\
192	4.3590424387251e-07\\
193	4.35921397718915e-07\\
194	4.35938842069352e-07\\
195	4.35956581809419e-07\\
196	4.35974621911367e-07\\
197	4.35992967422306e-07\\
198	4.36011623482897e-07\\
199	4.3603059531262e-07\\
200	4.36049888221872e-07\\
201	4.36069507612482e-07\\
202	4.36089458969045e-07\\
203	4.36109747875333e-07\\
204	4.36130380006508e-07\\
205	4.36151361129285e-07\\
206	4.36172697114527e-07\\
207	4.36194393921511e-07\\
208	4.3621645761762e-07\\
209	4.36238894371234e-07\\
210	4.36261710447751e-07\\
211	4.36284912224091e-07\\
212	4.36308506181932e-07\\
213	4.36332498913064e-07\\
214	4.36356897116083e-07\\
215	4.3638170760641e-07\\
216	4.36406937311076e-07\\
217	4.36432593274428e-07\\
218	4.36458682660864e-07\\
219	4.36485212751887e-07\\
220	4.36512190952827e-07\\
221	4.36539624796284e-07\\
222	4.36567521936743e-07\\
223	4.36595890161637e-07\\
224	4.36624737388367e-07\\
225	4.36654071665517e-07\\
226	4.36683901182853e-07\\
227	4.36714234260227e-07\\
228	4.36745079367812e-07\\
229	4.36776445105821e-07\\
230	4.36808340234951e-07\\
231	4.36840773651435e-07\\
232	4.36873754409861e-07\\
233	4.36907291716056e-07\\
234	4.36941394925503e-07\\
235	4.36976073562528e-07\\
236	4.37011337307279e-07\\
237	4.37047196003859e-07\\
238	4.37083659664784e-07\\
239	4.37120738474413e-07\\
240	4.37158442787372e-07\\
241	4.37196783133892e-07\\
242	4.37235770229807e-07\\
243	4.37275414967875e-07\\
244	4.37315728427107e-07\\
245	4.37356721880501e-07\\
246	4.37398406788451e-07\\
247	4.37440794810819e-07\\
248	4.374838978038e-07\\
249	4.37527727835106e-07\\
250	4.37572297171646e-07\\
251	4.37617618291956e-07\\
252	4.37663703894814e-07\\
253	4.37710566894014e-07\\
254	4.37758220430087e-07\\
255	4.37806677865781e-07\\
256	4.37855952795697e-07\\
257	4.37906059056981e-07\\
258	4.37957010719085e-07\\
259	4.38008822104305e-07\\
260	4.38061507775458e-07\\
261	4.38115082560407e-07\\
262	4.38169561534547e-07\\
263	4.38224960050163e-07\\
264	4.38281293717537e-07\\
265	4.38338578430848e-07\\
266	4.38396830362579e-07\\
267	4.3845606596573e-07\\
268	4.3851630199262e-07\\
269	4.38577555491015e-07\\
270	4.38639843808431e-07\\
271	4.38703184608483e-07\\
272	4.38767595867724e-07\\
273	4.38833095886327e-07\\
274	4.38899703295951e-07\\
275	4.38967437064215e-07\\
276	4.39036316501856e-07\\
277	4.39106361278283e-07\\
278	4.39177591412809e-07\\
279	4.39250027300402e-07\\
280	4.39323689707259e-07\\
281	4.39398599786122e-07\\
282	4.39474779085917e-07\\
283	4.39552249555797e-07\\
284	4.39631033554414e-07\\
285	4.39711153868686e-07\\
286	4.39792633715257e-07\\
287	4.39875496753221e-07\\
288	4.39959767098527e-07\\
289	4.40045469331684e-07\\
290	4.40132628516862e-07\\
291	4.40221270199159e-07\\
292	4.40311420438062e-07\\
293	4.40403105805777e-07\\
294	4.40496353412509e-07\\
295	4.40591190912235e-07\\
296	4.40687646524333e-07\\
297	4.40785749051628e-07\\
298	4.40885527893176e-07\\
299	4.40987013069395e-07\\
300	4.4109023523483e-07\\
301	4.41195225703087e-07\\
302	4.41302016470131e-07\\
303	4.41410640237651e-07\\
304	4.41521130432157e-07\\
305	4.41633521237861e-07\\
306	4.41747847627492e-07\\
307	4.41864145380769e-07\\
308	4.41982451133916e-07\\
309	4.42102802396652e-07\\
310	4.4222523760222e-07\\
311	4.42349796141637e-07\\
312	4.42476518405553e-07\\
313	4.42605445829455e-07\\
314	4.42736620952122e-07\\
315	4.42870087460403e-07\\
316	4.43005890250698e-07\\
317	4.4314407549477e-07\\
318	4.432846907065e-07\\
319	4.43427784814887e-07\\
320	4.43573408251904e-07\\
321	4.43721613033377e-07\\
322	4.43872452867464e-07\\
323	4.4402598324785e-07\\
324	4.44182261581356e-07\\
325	4.44341347315447e-07\\
326	4.44503302074274e-07\\
327	4.44668189813898e-07\\
328	4.44836077005044e-07\\
329	4.45007032804416e-07\\
330	4.45181129277089e-07\\
331	4.45358441622988e-07\\
332	4.45539048424822e-07\\
333	4.45723031947661e-07\\
334	4.45910478435786e-07\\
335	4.46101478465686e-07\\
336	4.46296127338126e-07\\
337	4.46494525500603e-07\\
338	4.46696779035211e-07\\
339	4.46903000178788e-07\\
340	4.47113307935286e-07\\
341	4.47327828717876e-07\\
342	4.47546697106324e-07\\
343	4.47770056669999e-07\\
344	4.47998060887322e-07\\
345	4.48230874191427e-07\\
346	4.4846867312327e-07\\
347	4.48711647631946e-07\\
348	4.48960002536266e-07\\
349	4.49213959164141e-07\\
350	4.49473757180344e-07\\
351	4.49739656669636e-07\\
352	4.50011940460597e-07\\
353	4.50290916782777e-07\\
354	4.50576922303904e-07\\
355	4.50870325745966e-07\\
356	4.51171532379942e-07\\
357	4.51480990167528e-07\\
358	4.51799199628439e-07\\
359	4.5212673234318e-07\\
360	4.52464270487952e-07\\
361	4.52812697191808e-07\\
362	4.53173306591444e-07\\
363	4.53548280021433e-07\\
364	4.53941692239194e-07\\
365	4.54361331329531e-07\\
366	4.54820730812272e-07\\
367	4.55334917625691e-07\\
368	4.5586024961437e-07\\
369	4.56393569206599e-07\\
370	4.56934984276364e-07\\
371	4.57484603740934e-07\\
372	4.58042537619154e-07\\
373	4.58608896809749e-07\\
374	4.59183792054823e-07\\
375	4.59767331818493e-07\\
376	4.60359621601097e-07\\
377	4.60960772392783e-07\\
378	4.61570899231697e-07\\
379	4.62190117959462e-07\\
380	4.62818545260268e-07\\
381	4.63456298693708e-07\\
382	4.6410349676993e-07\\
383	4.64760259023138e-07\\
384	4.65426706120389e-07\\
385	4.66102959973883e-07\\
386	4.66789143907746e-07\\
387	4.67485382833585e-07\\
388	4.68191803483914e-07\\
389	4.6890853465892e-07\\
390	4.69635707550573e-07\\
391	4.70373456090364e-07\\
392	4.71121917374378e-07\\
393	4.71881232125928e-07\\
394	4.72651545220905e-07\\
395	4.73433006233936e-07\\
396	4.74225769959324e-07\\
397	4.75029996811312e-07\\
398	4.75845852856791e-07\\
399	4.76673509100651e-07\\
400	4.77513139254203e-07\\
401	4.78364914254846e-07\\
402	4.79228988678671e-07\\
403	4.80105463284348e-07\\
404	4.80994279056869e-07\\
405	4.81894963224822e-07\\
406	4.82806276198916e-07\\
407	4.83725750066658e-07\\
408	4.84649734057819e-07\\
409	4.85575883494642e-07\\
410	4.86509813026681e-07\\
411	4.87459809787173e-07\\
412	4.88426119947507e-07\\
413	4.89409045121629e-07\\
414	4.90408939179918e-07\\
415	4.91426167774683e-07\\
416	4.92461108933913e-07\\
417	4.93514153760769e-07\\
418	4.9458570726104e-07\\
419	4.95676189302587e-07\\
420	4.96786035413769e-07\\
421	4.97915696913407e-07\\
422	4.99065641521776e-07\\
423	5.00236354831127e-07\\
424	5.01428341285755e-07\\
425	5.02642125220279e-07\\
426	5.03878251956168e-07\\
427	5.05137288982068e-07\\
428	5.06419827203559e-07\\
429	5.07726482271197e-07\\
430	5.09057896007294e-07\\
431	5.10414737898169e-07\\
432	5.1179770672642e-07\\
433	5.13207532280221e-07\\
434	5.14644977210104e-07\\
435	5.16110839052974e-07\\
436	5.17605952488326e-07\\
437	5.19131191943217e-07\\
438	5.20687474859379e-07\\
439	5.22275765951502e-07\\
440	5.23897082869642e-07\\
441	5.25552502653173e-07\\
442	5.27243164114881e-07\\
443	5.2897025031219e-07\\
444	5.30734924846659e-07\\
445	5.32538268289753e-07\\
446	5.34381629665944e-07\\
447	5.36266440973359e-07\\
448	5.38194216928926e-07\\
449	5.4016657023308e-07\\
450	5.42185250479107e-07\\
451	5.44252239932341e-07\\
452	5.46369993147793e-07\\
453	5.48542023433573e-07\\
454	5.50774254206475e-07\\
455	5.53077726880314e-07\\
456	5.55472044605509e-07\\
457	5.57978676489192e-07\\
458	5.60537195289438e-07\\
459	5.6313925249674e-07\\
460	5.65794875088331e-07\\
461	5.68525814775929e-07\\
462	5.71376082141709e-07\\
463	5.74413697553009e-07\\
464	5.77673303769232e-07\\
465	5.81332842023018e-07\\
466	5.86155485931426e-07\\
467	5.93913121381586e-07\\
468	6.02796678308234e-07\\
469	6.11827278019675e-07\\
470	6.21011842535799e-07\\
471	6.30356316779779e-07\\
472	6.39863208224527e-07\\
473	6.49526178899239e-07\\
474	6.59319845506786e-07\\
475	6.69186724437135e-07\\
476	6.79045612325569e-07\\
477	6.88913043163912e-07\\
478	7.1191710836285e-07\\
479	7.36251307929028e-07\\
480	7.61283855274156e-07\\
481	7.8706759957787e-07\\
482	8.13671492430865e-07\\
483	8.41200630273418e-07\\
484	8.69849572101272e-07\\
485	9.00045155930217e-07\\
486	9.32825888050371e-07\\
487	9.70793912198952e-07\\
488	1.0197698550852e-06\\
489	1.07349631091925e-06\\
490	1.12819026207342e-06\\
491	1.1838815485207e-06\\
492	1.24059958651208e-06\\
493	1.29836989074042e-06\\
494	1.35720566489286e-06\\
495	1.41708990526517e-06\\
496	1.47794345427834e-06\\
497	1.53959193930114e-06\\
498	1.60183097725021e-06\\
499	1.66489561723599e-06\\
500	1.72946768205313e-06\\
501	1.79565925023842e-06\\
502	1.86364139327674e-06\\
503	1.97547071890742e-06\\
504	2.14963390653748e-06\\
505	2.33186913378658e-06\\
506	2.52587543034996e-06\\
507	2.74397810153313e-06\\
508	2.99255659583446e-06\\
509	3.24410594676526e-06\\
510	3.49866347970021e-06\\
511	3.7562276102568e-06\\
512	4.01675334510043e-06\\
513	4.28015214097894e-06\\
514	4.54550038981188e-06\\
515	4.81392577946725e-06\\
516	5.08616511233571e-06\\
517	5.36234433405714e-06\\
518	5.64253156912134e-06\\
519	5.92666385147519e-06\\
520	6.22197699225318e-06\\
521	6.52325092110364e-06\\
522	6.82768405190497e-06\\
523	7.14319826271199e-06\\
524	7.46858967594716e-06\\
525	7.80362340238037e-06\\
526	8.15494582598335e-06\\
527	8.54647610794914e-06\\
528	9.7814929141284e-06\\
529	1.14106440019494e-05\\
530	1.31283062583692e-05\\
531	1.49323076594289e-05\\
532	1.68675940995809e-05\\
533	2.20733874857024e-05\\
534	3.1423030944064e-05\\
535	4.11197137462416e-05\\
536	5.11971048988361e-05\\
537	6.16939429296373e-05\\
538	7.26557429743505e-05\\
539	8.4140481352317e-05\\
540	9.62439842851792e-05\\
541	0.000183017843515156\\
542	0.000320058431690405\\
543	0.000461979071212767\\
544	0.000609226141479381\\
545	0.000762310686206207\\
546	0.000921818775126619\\
547	0.00108842497921463\\
548	0.00126290890867717\\
549	0.00144617528986034\\
550	0.00163927812649895\\
551	0.00184344890560487\\
552	0.00206012562878466\\
553	0.00228932252299241\\
554	0.00252454541366836\\
555	0.00276610985074724\\
556	0.00301435835283926\\
557	0.00326969255277124\\
558	0.00353252020540563\\
559	0.00380323021433977\\
560	0.00408230479019987\\
561	0.00437023316844891\\
562	0.00466746777496262\\
563	0.00497427187802432\\
564	0.00529105479441378\\
565	0.00561646159330649\\
566	0.00594152902958734\\
567	0.00626327298138607\\
568	0.00643204595372145\\
569	0.00659656549703352\\
570	0.00675547050848819\\
571	0.00690723144372143\\
572	0.00704974781070761\\
573	0.00718309251742785\\
574	0.00731539000987618\\
575	0.00744700693395213\\
576	0.0075776877763659\\
577	0.00770719439458229\\
578	0.0078353307681382\\
579	0.00796197427776425\\
580	0.00808701048954674\\
581	0.00821087605745987\\
582	0.00833514879710221\\
583	0.00845957027804329\\
584	0.00858384582922777\\
585	0.00870763757218698\\
586	0.00883055685842396\\
587	0.00895211666486942\\
588	0.0090717816925381\\
589	0.0091889696268204\\
590	0.00930305497846415\\
591	0.00941337639759054\\
592	0.00951924639550553\\
593	0.00961996226627061\\
594	0.00971482541990144\\
595	0.00980316865229207\\
596	0.00988444283510058\\
597	0.00995832188366689\\
598	0.010000292044645\\
599	0\\
600	0\\
};
\addplot [color=mycolor13,solid,forget plot]
  table[row sep=crcr]{%
1	0.000229137842526684\\
2	0.000229137842526684\\
3	0.000229137842526684\\
4	0.000229137842526684\\
5	0.000229137842526684\\
6	0.000229137842526684\\
7	0.000229137842526684\\
8	0.000229137842526684\\
9	0.000229137842526684\\
10	0.000229137842526684\\
11	0.000229137842526684\\
12	0.000229137842526684\\
13	0.000229137842526684\\
14	0.000229137842526684\\
15	0.000229137842526684\\
16	0.000229137842526684\\
17	0.000229137842526684\\
18	0.000229137842526684\\
19	0.000229137842526684\\
20	0.000229137842526684\\
21	0.000229137842526684\\
22	0.000229137842526684\\
23	0.000229137842526684\\
24	0.000229137842526684\\
25	0.000229137842526684\\
26	0.000229137842526684\\
27	0.000229137842526684\\
28	0.000229137842526684\\
29	0.000229137842526684\\
30	0.000229137842526684\\
31	0.000229137842526684\\
32	0.000229137842526684\\
33	0.000229137842526684\\
34	0.000229137842526684\\
35	0.000229137842526684\\
36	0.000229137842526684\\
37	0.000229137842526684\\
38	0.000229137842526684\\
39	0.000229137842526684\\
40	0.000229137842526684\\
41	0.000229137842526684\\
42	0.000229137842526684\\
43	0.000229137842526684\\
44	0.000229137842526684\\
45	0.000229137842526684\\
46	0.000229137842526684\\
47	0.000229137842526684\\
48	0.000229137842526684\\
49	0.000229137842526684\\
50	0.000229137842526684\\
51	0.000229137842526684\\
52	0.000229137842526684\\
53	0.000229137842526684\\
54	0.000229137842526684\\
55	0.000229137842526684\\
56	0.000229137842526684\\
57	0.000229137842526684\\
58	0.000229137842526684\\
59	0.000229137842526684\\
60	0.000229137842526684\\
61	0.000229137842526684\\
62	0.000229137842526684\\
63	0.000229137842526684\\
64	0.000229137842526684\\
65	0.000229137842526684\\
66	0.000229137842526684\\
67	0.000229137842526684\\
68	0.000229137842526684\\
69	0.000229137842526684\\
70	0.000229137842526684\\
71	0.000229137842526684\\
72	0.000229137842526684\\
73	0.000229137842526684\\
74	0.000229137842526684\\
75	0.000229137842526684\\
76	0.000229137842526684\\
77	0.000229137842526684\\
78	0.000229137842526684\\
79	0.000229137842526684\\
80	0.000229137842526684\\
81	0.000229137842526684\\
82	0.000229137842526684\\
83	0.000229137842526684\\
84	0.000229137842526684\\
85	0.000229137842526684\\
86	0.000229137842526684\\
87	0.000229137842526684\\
88	0.000229137842526684\\
89	0.000229137842526684\\
90	0.000229137842526684\\
91	0.000229137842526684\\
92	0.000229137842526684\\
93	0.000229137842526684\\
94	0.000229137842526684\\
95	0.000229137842526684\\
96	0.000229137842526684\\
97	0.000229137842526684\\
98	0.000229137842526684\\
99	0.000229137842526684\\
100	0.000229137842526684\\
101	0.000229137842526684\\
102	0.000229137842526684\\
103	0.000229137842526684\\
104	0.000229137842526684\\
105	0.000229137842526684\\
106	0.000229137842526684\\
107	0.000229137842526684\\
108	0.000229137842526684\\
109	0.000229137842526684\\
110	0.000229137842526684\\
111	0.000229137842526684\\
112	0.000229137842526684\\
113	0.000229137842526684\\
114	0.000229137842526684\\
115	0.000229137842526684\\
116	0.000229137842526684\\
117	0.000229137842526684\\
118	0.000229137842526684\\
119	0.000229137842526684\\
120	0.000229137842526684\\
121	0.000229137842526684\\
122	0.000229137842526684\\
123	0.000229137842526684\\
124	0.000229137842526684\\
125	0.000229137842526684\\
126	0.000229137842526684\\
127	0.000229137842526684\\
128	0.000229137842526684\\
129	0.000229137842526684\\
130	0.000229137842526684\\
131	0.000229137842526684\\
132	0.000229137842526684\\
133	0.000229137842526684\\
134	0.000229137842526684\\
135	0.000229137842526684\\
136	0.000229137842526684\\
137	0.000229137842526684\\
138	0.000229137842526684\\
139	0.000229137842526684\\
140	0.000229137842526684\\
141	0.000229137842526684\\
142	0.000229137842526684\\
143	0.000229137842526684\\
144	0.000229137842526684\\
145	0.000229137842526684\\
146	0.000229137842526684\\
147	0.000229137842526684\\
148	0.000229137842526684\\
149	0.000229137842526684\\
150	0.000229137842526684\\
151	0.000229137842526684\\
152	0.000229137842526684\\
153	0.000229137842526684\\
154	0.000229137842526684\\
155	0.000229137842526684\\
156	0.000229137842526684\\
157	0.000229137842526684\\
158	0.000229137842526684\\
159	0.000229137842526684\\
160	0.000229137842526684\\
161	0.000229137842526684\\
162	0.000229137842526684\\
163	0.000229137842526684\\
164	0.000229137842526684\\
165	0.000229137842526684\\
166	0.000229137842526684\\
167	0.000229137842526684\\
168	0.000229137842526684\\
169	0.000229137842526684\\
170	0.000229137842526684\\
171	0.000229137842526684\\
172	0.000229137842526684\\
173	0.000229137842526684\\
174	0.000229137842526684\\
175	0.000229137842526684\\
176	0.000229137842526684\\
177	0.000229137842526684\\
178	0.000229137842526684\\
179	0.000229137842526684\\
180	0.000229137842526684\\
181	0.000229137842526684\\
182	0.000229137842526684\\
183	0.000229137842526684\\
184	0.000229137842526684\\
185	0.000229137842526684\\
186	0.000229137842526684\\
187	0.000229137842526684\\
188	0.000229137842526684\\
189	0.000229137842526684\\
190	0.000229137842526684\\
191	0.000229137842526684\\
192	0.000229137842526684\\
193	0.000229137842526684\\
194	0.000229137842526684\\
195	0.000229137842526684\\
196	0.000229137842526684\\
197	0.000229137842526684\\
198	0.000229137842526684\\
199	0.000229137842526684\\
200	0.000229137842526684\\
201	0.000229137842526684\\
202	0.000229137842526684\\
203	0.000229137842526684\\
204	0.000229137842526684\\
205	0.000229137842526684\\
206	0.000229137842526684\\
207	0.000229137842526684\\
208	0.000229137842526684\\
209	0.000229137842526684\\
210	0.000229137842526684\\
211	0.000229137842526684\\
212	0.000229137842526684\\
213	0.000229137842526684\\
214	0.000229137842526684\\
215	0.000229137842526684\\
216	0.000229137842526684\\
217	0.000229137842526684\\
218	0.000229137842526684\\
219	0.000229137842526684\\
220	0.000229137842526684\\
221	0.000229137842526684\\
222	0.000229137842526684\\
223	0.000229137842526684\\
224	0.000229137842526684\\
225	0.000229137842526684\\
226	0.000229137842526684\\
227	0.000229137842526684\\
228	0.000229137842526684\\
229	0.000229137842526684\\
230	0.000229137842526684\\
231	0.000229137842526684\\
232	0.000229137842526684\\
233	0.000229137842526684\\
234	0.000229137842526684\\
235	0.000229137842526684\\
236	0.000229137842526684\\
237	0.000229137842526684\\
238	0.000229137842526684\\
239	0.000229137842526684\\
240	0.000229137842526684\\
241	0.000229137842526684\\
242	0.000229137842526684\\
243	0.000229137842526684\\
244	0.000229137842526684\\
245	0.000229137842526684\\
246	0.000229137842526684\\
247	0.000229137842526684\\
248	0.000229137842526684\\
249	0.000229137842526684\\
250	0.000229137842526684\\
251	0.000229137842526684\\
252	0.000229137842526684\\
253	0.000229137842526684\\
254	0.000229137842526684\\
255	0.000229137842526684\\
256	0.000229137842526684\\
257	0.000229137842526684\\
258	0.000229137842526684\\
259	0.000229137842526684\\
260	0.000229137842526684\\
261	0.000229137842526684\\
262	0.000229137842526684\\
263	0.000229137842526684\\
264	0.000229137842526684\\
265	0.000229137842526684\\
266	0.000229137842526684\\
267	0.000229137842526684\\
268	0.000229137842526684\\
269	0.000229137842526684\\
270	0.000229137842526684\\
271	0.000229137842526684\\
272	0.000229137842526684\\
273	0.000229137842526684\\
274	0.000229137842526684\\
275	0.000229137842526684\\
276	0.000229137842526684\\
277	0.000229137842526684\\
278	0.000229137842526684\\
279	0.000229137842526684\\
280	0.000229137842526684\\
281	0.000229137842526684\\
282	0.000229137842526684\\
283	0.000229137842526684\\
284	0.000229137842526684\\
285	0.000229137842526684\\
286	0.000229137842526684\\
287	0.000229137842526684\\
288	0.000229137842526684\\
289	0.000229137842526684\\
290	0.000229137842526684\\
291	0.000229137842526684\\
292	0.000229137842526684\\
293	0.000229137842526684\\
294	0.000229137842526684\\
295	0.000229137842526684\\
296	0.000229137842526684\\
297	0.000229137842526684\\
298	0.000229137842526684\\
299	0.000229137842526684\\
300	0.000229137842526684\\
301	0.000229137842526684\\
302	0.000229137842526684\\
303	0.000229137842526684\\
304	0.000229137842526684\\
305	0.000229137842526684\\
306	0.000229137842526684\\
307	0.000229137842526684\\
308	0.000229137842526684\\
309	0.000229137842526684\\
310	0.000229137842526684\\
311	0.000229137842526684\\
312	0.000229137842526684\\
313	0.000229137842526684\\
314	0.000229137842526684\\
315	0.000229137842526684\\
316	0.000229137842526684\\
317	0.000229137842526684\\
318	0.000229137842526684\\
319	0.000229137842526684\\
320	0.000229137842526684\\
321	0.000229137842526684\\
322	0.000229137842526684\\
323	0.000229137842526684\\
324	0.000229137842526684\\
325	0.000229137842526684\\
326	0.000229137842526684\\
327	0.000229137842526684\\
328	0.000229137842526684\\
329	0.000229137842526684\\
330	0.000229137842526684\\
331	0.000229137842526684\\
332	0.000229137842526684\\
333	0.000229137842526684\\
334	0.000229137842526684\\
335	0.000229137842526684\\
336	0.000229137842526684\\
337	0.000229137842526684\\
338	0.000229137842526684\\
339	0.000229137842526684\\
340	0.000229137842526684\\
341	0.000229137842526684\\
342	0.000229137842526684\\
343	0.000229137842526684\\
344	0.000229137842526684\\
345	0.000229137842526684\\
346	0.000229137842526684\\
347	0.000229137842526684\\
348	0.000229137842526684\\
349	0.000229137842526684\\
350	0.000229137842526684\\
351	0.000229137842526684\\
352	0.000229137842526684\\
353	0.000229137842526684\\
354	0.000229137842526684\\
355	0.000229137842526684\\
356	0.000229137842526684\\
357	0.000229137842526684\\
358	0.000229137842526684\\
359	0.000229137842526684\\
360	0.000229137842526684\\
361	0.000229137842526684\\
362	0.000229137842526684\\
363	0.000229137842526684\\
364	0.000229137842526684\\
365	0.000229137842526684\\
366	0.000229137842526684\\
367	0.000229137842526684\\
368	0.000229137842526684\\
369	0.000229137842526684\\
370	0.000229137842526684\\
371	0.000229137842526684\\
372	0.000229137842526684\\
373	0.000229137842526684\\
374	0.000229137842526684\\
375	0.000229137842526684\\
376	0.000229137842526684\\
377	0.000229137842526684\\
378	0.000229137842526684\\
379	0.000229137842526684\\
380	0.000229137842526684\\
381	0.000229137842526684\\
382	0.000229137842526684\\
383	0.000229137842526684\\
384	0.000229137842526684\\
385	0.000229137842526684\\
386	0.000229137842526684\\
387	0.000229137842526684\\
388	0.000229137842526684\\
389	0.000229137842526684\\
390	0.000229137842526684\\
391	0.000229137842526684\\
392	0.000229137842526684\\
393	0.000229137842526684\\
394	0.000229137842526684\\
395	0.000229137842526684\\
396	0.000229137842526684\\
397	0.000229137842526684\\
398	0.000229137842526684\\
399	0.000229137842526684\\
400	0.000229137842526684\\
401	0.000229137842526684\\
402	0.000229137842526684\\
403	0.000229137842526684\\
404	0.000229137842526684\\
405	0.000229137842526684\\
406	0.000229137842526684\\
407	0.000229137842526684\\
408	0.000229137842526684\\
409	0.000229137842526684\\
410	0.000229137842526684\\
411	0.000229137842526684\\
412	0.000229137842526684\\
413	0.000229137842526684\\
414	0.000229137842526684\\
415	0.000229137842526684\\
416	0.000229137842526684\\
417	0.000229137842526684\\
418	0.000229137842526684\\
419	0.000229137842526684\\
420	0.000229137842526684\\
421	0.000229137842526684\\
422	0.000229137842526684\\
423	0.000229137842526684\\
424	0.000229137842526684\\
425	0.000229137842526684\\
426	0.000229137842526684\\
427	0.000229137842526684\\
428	0.000229137842526684\\
429	0.000229137842526684\\
430	0.000229137842526684\\
431	0.000229137842526684\\
432	0.000229137842526684\\
433	0.000229137842526684\\
434	0.000229137842526684\\
435	0.000229137842526684\\
436	0.000229137842526684\\
437	0.000229137842526684\\
438	0.000229137842526684\\
439	0.000229137842526684\\
440	0.000229137842526684\\
441	0.000229137842526684\\
442	0.000229137842526684\\
443	0.000229137842526684\\
444	0.000229137842526684\\
445	0.000229137842526684\\
446	0.000229137842526684\\
447	0.000229137842526684\\
448	0.000229137842526684\\
449	0.000229137842526684\\
450	0.000229137842526684\\
451	0.000229137842526684\\
452	0.000229137842526684\\
453	0.000229137842526684\\
454	0.000229137842526684\\
455	0.000229137842526684\\
456	0.000229137842526684\\
457	0.000229137842526684\\
458	0.000229137842526684\\
459	0.000229137842526684\\
460	0.000229137842526684\\
461	0.000229137842526684\\
462	0.000229137842526684\\
463	0.000229137842526684\\
464	0.000229137842526684\\
465	0.000229137842526684\\
466	0.000229137842526684\\
467	0.000229137842526684\\
468	0.000229137842526684\\
469	0.000229137842526684\\
470	0.000229137842526684\\
471	0.000229137842526684\\
472	0.000229137842526684\\
473	0.000229137842526684\\
474	0.000229137842526684\\
475	0.000229137842526684\\
476	0.000229137842526684\\
477	0.000229137842526684\\
478	0.000229137842526684\\
479	0.000229137842526684\\
480	0.000229137842526684\\
481	0.000229137842526684\\
482	0.000229137842526684\\
483	0.000229137842526684\\
484	0.000229137842526684\\
485	0.000229137842526684\\
486	0.000229137842526684\\
487	0.000229137842526684\\
488	0.000229137842526684\\
489	0.000229137842526684\\
490	0.000229137842526684\\
491	0.000229137842526684\\
492	0.000229137842526684\\
493	0.000229137842526684\\
494	0.000229137842526684\\
495	0.000229137842526684\\
496	0.000229137842526684\\
497	0.000229137842526684\\
498	0.000229137842526684\\
499	0.000229137842526684\\
500	0.000229137842526684\\
501	0.000229137842526684\\
502	0.000229137842526684\\
503	0.000229137842526684\\
504	0.000229137842526684\\
505	0.000229137842526684\\
506	0.000229137842526684\\
507	0.000229137842526684\\
508	0.000229137842526684\\
509	0.000229137842526684\\
510	0.000229137842526684\\
511	0.000229137842526684\\
512	0.000229137842526684\\
513	0.000229137842526684\\
514	0.000229137842526684\\
515	0.000229137842526684\\
516	0.000229137842526684\\
517	0.000229137842526684\\
518	0.000229137842526684\\
519	0.000229137842526684\\
520	0.000229137842526684\\
521	0.000229137842526684\\
522	0.000229137842526684\\
523	0.000229137842526684\\
524	0.000229137842526684\\
525	0.000229137842526684\\
526	0.000229137842526684\\
527	0.000229137842526684\\
528	0.000229137842526684\\
529	0.000229137842526684\\
530	0.000229137842526684\\
531	0.000229137842526684\\
532	0.000229137842526684\\
533	0.000229137842526684\\
534	0.000229137842526684\\
535	0.000229137842526684\\
536	0.000229137842526684\\
537	0.000229137842526684\\
538	0.000229137842526684\\
539	0.000229137842526684\\
540	0.000229137842526684\\
541	0.000229137842526684\\
542	0.000229137842526684\\
543	0.000229137842526684\\
544	0.000229137842526684\\
545	0.000229137842526684\\
546	0.000229137842526684\\
547	0.000229137842526684\\
548	0.000229137842526684\\
549	0.000229137842526684\\
550	0.000229137842526684\\
551	0.000229137842526684\\
552	0.000229137842526684\\
553	0.000229137842526684\\
554	0.000229137842526684\\
555	0.000229137842526684\\
556	0.000229137842526684\\
557	0.000229137842526684\\
558	0.000229137842526684\\
559	0.000229137842526684\\
560	0.000229137842526684\\
561	0.000229137842526684\\
562	0.000229137842526684\\
563	0.000229137842526684\\
564	0.000229137842526684\\
565	0.00023087401359618\\
566	0.000242990420949804\\
567	0.000269136136587107\\
568	0.000448812459300998\\
569	0.000637589915763916\\
570	0.000838128118940092\\
571	0.00105199896671285\\
572	0.00128105703203953\\
573	0.00152512637960484\\
574	0.00177692804255711\\
575	0.00203669514046483\\
576	0.00230473255962503\\
577	0.00258130913841595\\
578	0.00286642117413111\\
579	0.00316021118843213\\
580	0.00346264275586949\\
581	0.00377236863713507\\
582	0.00408776308700265\\
583	0.00440895297433541\\
584	0.00473600934600418\\
585	0.0050690485378831\\
586	0.00540818140277331\\
587	0.00575350248078163\\
588	0.00610506048892851\\
589	0.00646281347291639\\
590	0.00682657112268328\\
591	0.00719605229431493\\
592	0.0075711377940388\\
593	0.00795166435354403\\
594	0.00833736409264152\\
595	0.00872777975004381\\
596	0.00912205985501356\\
597	0.00951842928902401\\
598	0.00991325377008988\\
599	0\\
600	0\\
};
\addplot [color=mycolor14,solid,forget plot]
  table[row sep=crcr]{%
1	0.0100002859420401\\
2	0.0100002859420007\\
3	0.0100002859419605\\
4	0.0100002859419196\\
5	0.0100002859418779\\
6	0.0100002859418355\\
7	0.0100002859417923\\
8	0.0100002859417484\\
9	0.0100002859417036\\
10	0.010000285941658\\
11	0.0100002859416117\\
12	0.0100002859415644\\
13	0.0100002859415163\\
14	0.0100002859414674\\
15	0.0100002859414176\\
16	0.0100002859413668\\
17	0.0100002859413152\\
18	0.0100002859412626\\
19	0.010000285941209\\
20	0.0100002859411545\\
21	0.010000285941099\\
22	0.0100002859410425\\
23	0.010000285940985\\
24	0.0100002859409265\\
25	0.0100002859408668\\
26	0.0100002859408061\\
27	0.0100002859407443\\
28	0.0100002859406814\\
29	0.0100002859406174\\
30	0.0100002859405522\\
31	0.0100002859404858\\
32	0.0100002859404182\\
33	0.0100002859403494\\
34	0.0100002859402793\\
35	0.010000285940208\\
36	0.0100002859401354\\
37	0.0100002859400615\\
38	0.0100002859399862\\
39	0.0100002859399096\\
40	0.0100002859398316\\
41	0.0100002859397522\\
42	0.0100002859396714\\
43	0.0100002859395891\\
44	0.0100002859395053\\
45	0.01000028593942\\
46	0.0100002859393331\\
47	0.0100002859392447\\
48	0.0100002859391547\\
49	0.010000285939063\\
50	0.0100002859389697\\
51	0.0100002859388748\\
52	0.0100002859387781\\
53	0.0100002859386796\\
54	0.0100002859385794\\
55	0.0100002859384774\\
56	0.0100002859383735\\
57	0.0100002859382677\\
58	0.0100002859381601\\
59	0.0100002859380505\\
60	0.0100002859379389\\
61	0.0100002859378253\\
62	0.0100002859377097\\
63	0.0100002859375919\\
64	0.0100002859374721\\
65	0.0100002859373501\\
66	0.0100002859372258\\
67	0.0100002859370994\\
68	0.0100002859369706\\
69	0.0100002859368396\\
70	0.0100002859367061\\
71	0.0100002859365703\\
72	0.010000285936432\\
73	0.0100002859362912\\
74	0.0100002859361479\\
75	0.010000285936002\\
76	0.0100002859358534\\
77	0.0100002859357022\\
78	0.0100002859355483\\
79	0.0100002859353916\\
80	0.010000285935232\\
81	0.0100002859350696\\
82	0.0100002859349042\\
83	0.0100002859347359\\
84	0.0100002859345645\\
85	0.0100002859343901\\
86	0.0100002859342125\\
87	0.0100002859340317\\
88	0.0100002859338476\\
89	0.0100002859336602\\
90	0.0100002859334695\\
91	0.0100002859332753\\
92	0.0100002859330776\\
93	0.0100002859328764\\
94	0.0100002859326715\\
95	0.0100002859324629\\
96	0.0100002859322506\\
97	0.0100002859320344\\
98	0.0100002859318144\\
99	0.0100002859315904\\
100	0.0100002859313624\\
101	0.0100002859311302\\
102	0.0100002859308939\\
103	0.0100002859306533\\
104	0.0100002859304084\\
105	0.0100002859301591\\
106	0.0100002859299053\\
107	0.010000285929647\\
108	0.010000285929384\\
109	0.0100002859291162\\
110	0.0100002859288436\\
111	0.0100002859285662\\
112	0.0100002859282837\\
113	0.0100002859279962\\
114	0.0100002859277035\\
115	0.0100002859274055\\
116	0.0100002859271022\\
117	0.0100002859267934\\
118	0.010000285926479\\
119	0.0100002859261591\\
120	0.0100002859258333\\
121	0.0100002859255017\\
122	0.0100002859251642\\
123	0.0100002859248205\\
124	0.0100002859244707\\
125	0.0100002859241147\\
126	0.0100002859237522\\
127	0.0100002859233832\\
128	0.0100002859230075\\
129	0.0100002859226252\\
130	0.0100002859222359\\
131	0.0100002859218397\\
132	0.0100002859214364\\
133	0.0100002859210258\\
134	0.0100002859206078\\
135	0.0100002859201823\\
136	0.0100002859197492\\
137	0.0100002859193083\\
138	0.0100002859188596\\
139	0.0100002859184027\\
140	0.0100002859179377\\
141	0.0100002859174643\\
142	0.0100002859169824\\
143	0.0100002859164918\\
144	0.0100002859159925\\
145	0.0100002859154842\\
146	0.0100002859149668\\
147	0.0100002859144401\\
148	0.010000285913904\\
149	0.0100002859133583\\
150	0.0100002859128027\\
151	0.0100002859122372\\
152	0.0100002859116616\\
153	0.0100002859110757\\
154	0.0100002859104792\\
155	0.0100002859098721\\
156	0.010000285909254\\
157	0.0100002859086249\\
158	0.0100002859079846\\
159	0.0100002859073327\\
160	0.0100002859066692\\
161	0.0100002859059938\\
162	0.0100002859053063\\
163	0.0100002859046065\\
164	0.0100002859038942\\
165	0.0100002859031691\\
166	0.010000285902431\\
167	0.0100002859016797\\
168	0.0100002859009149\\
169	0.0100002859001365\\
170	0.0100002858993441\\
171	0.0100002858985376\\
172	0.0100002858977166\\
173	0.0100002858968809\\
174	0.0100002858960303\\
175	0.0100002858951645\\
176	0.0100002858942831\\
177	0.010000285893386\\
178	0.0100002858924729\\
179	0.0100002858915434\\
180	0.0100002858905973\\
181	0.0100002858896343\\
182	0.0100002858886541\\
183	0.0100002858876564\\
184	0.0100002858866408\\
185	0.010000285885607\\
186	0.0100002858845548\\
187	0.0100002858834838\\
188	0.0100002858823936\\
189	0.010000285881284\\
190	0.0100002858801545\\
191	0.0100002858790049\\
192	0.0100002858778347\\
193	0.0100002858766436\\
194	0.0100002858754313\\
195	0.0100002858741973\\
196	0.0100002858729412\\
197	0.0100002858716627\\
198	0.0100002858703614\\
199	0.0100002858690369\\
200	0.0100002858676887\\
201	0.0100002858663165\\
202	0.0100002858649197\\
203	0.010000285863498\\
204	0.010000285862051\\
205	0.0100002858605781\\
206	0.010000285859079\\
207	0.010000285857553\\
208	0.0100002858559999\\
209	0.0100002858544191\\
210	0.01000028585281\\
211	0.0100002858511723\\
212	0.0100002858495053\\
213	0.0100002858478086\\
214	0.0100002858460817\\
215	0.0100002858443239\\
216	0.0100002858425348\\
217	0.0100002858407138\\
218	0.0100002858388603\\
219	0.0100002858369738\\
220	0.0100002858350536\\
221	0.0100002858330993\\
222	0.01000028583111\\
223	0.0100002858290853\\
224	0.0100002858270246\\
225	0.0100002858249271\\
226	0.0100002858227922\\
227	0.0100002858206192\\
228	0.0100002858184075\\
229	0.0100002858161565\\
230	0.0100002858138653\\
231	0.0100002858115332\\
232	0.0100002858091597\\
233	0.0100002858067438\\
234	0.0100002858042849\\
235	0.0100002858017821\\
236	0.0100002857992348\\
237	0.0100002857966421\\
238	0.0100002857940032\\
239	0.0100002857913173\\
240	0.0100002857885836\\
241	0.0100002857858011\\
242	0.0100002857829691\\
243	0.0100002857800867\\
244	0.0100002857771528\\
245	0.0100002857741668\\
246	0.0100002857711275\\
247	0.0100002857680341\\
248	0.0100002857648856\\
249	0.010000285761681\\
250	0.0100002857584194\\
251	0.0100002857550996\\
252	0.0100002857517207\\
253	0.0100002857482816\\
254	0.0100002857447813\\
255	0.0100002857412186\\
256	0.0100002857375925\\
257	0.0100002857339017\\
258	0.0100002857301452\\
259	0.0100002857263218\\
260	0.0100002857224303\\
261	0.0100002857184694\\
262	0.010000285714438\\
263	0.0100002857103348\\
264	0.0100002857061585\\
265	0.0100002857019077\\
266	0.0100002856975812\\
267	0.0100002856931777\\
268	0.0100002856886956\\
269	0.0100002856841337\\
270	0.0100002856794905\\
271	0.0100002856747646\\
272	0.0100002856699544\\
273	0.0100002856650585\\
274	0.0100002856600753\\
275	0.0100002856550033\\
276	0.0100002856498409\\
277	0.0100002856445865\\
278	0.0100002856392384\\
279	0.0100002856337949\\
280	0.0100002856282544\\
281	0.0100002856226151\\
282	0.0100002856168752\\
283	0.0100002856110329\\
284	0.0100002856050865\\
285	0.010000285599034\\
286	0.0100002855928735\\
287	0.0100002855866031\\
288	0.0100002855802208\\
289	0.0100002855737247\\
290	0.0100002855671127\\
291	0.0100002855603826\\
292	0.0100002855535325\\
293	0.01000028554656\\
294	0.0100002855394632\\
295	0.0100002855322396\\
296	0.0100002855248871\\
297	0.0100002855174033\\
298	0.0100002855097859\\
299	0.0100002855020324\\
300	0.0100002854941405\\
301	0.0100002854861076\\
302	0.0100002854779313\\
303	0.0100002854696089\\
304	0.0100002854611379\\
305	0.0100002854525155\\
306	0.0100002854437391\\
307	0.0100002854348059\\
308	0.0100002854257131\\
309	0.0100002854164578\\
310	0.0100002854070372\\
311	0.0100002853974482\\
312	0.0100002853876879\\
313	0.0100002853777532\\
314	0.0100002853676411\\
315	0.0100002853573482\\
316	0.0100002853468715\\
317	0.0100002853362076\\
318	0.0100002853253532\\
319	0.010000285314305\\
320	0.0100002853030595\\
321	0.0100002852916132\\
322	0.0100002852799626\\
323	0.010000285268104\\
324	0.0100002852560339\\
325	0.0100002852437485\\
326	0.0100002852312441\\
327	0.0100002852185168\\
328	0.0100002852055627\\
329	0.0100002851923779\\
330	0.0100002851789585\\
331	0.0100002851653003\\
332	0.010000285151399\\
333	0.0100002851372501\\
334	0.0100002851228489\\
335	0.0100002851081895\\
336	0.0100002850932653\\
337	0.0100002850780708\\
338	0.010000285062604\\
339	0.0100002850468651\\
340	0.0100002850308494\\
341	0.0100002850145525\\
342	0.0100002849979691\\
343	0.0100002849810934\\
344	0.0100002849639176\\
345	0.0100002849464311\\
346	0.01000028492862\\
347	0.0100002849104738\\
348	0.0100002848920035\\
349	0.0100002848732209\\
350	0.0100002848541209\\
351	0.010000284834698\\
352	0.0100002848149469\\
353	0.0100002847948621\\
354	0.010000284774438\\
355	0.010000284753669\\
356	0.0100002847325493\\
357	0.0100002847110732\\
358	0.0100002846892346\\
359	0.0100002846670276\\
360	0.0100002846444462\\
361	0.010000284621484\\
362	0.0100002845981349\\
363	0.0100002845743925\\
364	0.0100002845502503\\
365	0.0100002845257017\\
366	0.0100002845007402\\
367	0.010000284475359\\
368	0.0100002844495512\\
369	0.01000028442331\\
370	0.0100002843966282\\
371	0.0100002843694987\\
372	0.0100002843419144\\
373	0.0100002843138678\\
374	0.0100002842853515\\
375	0.0100002842563579\\
376	0.0100002842268794\\
377	0.0100002841969083\\
378	0.0100002841664373\\
379	0.0100002841354591\\
380	0.0100002841039676\\
381	0.010000284071959\\
382	0.0100002840394324\\
383	0.010000284006391\\
384	0.0100002839728374\\
385	0.0100002839387595\\
386	0.0100002839041161\\
387	0.0100002838688902\\
388	0.0100002838330719\\
389	0.0100002837966509\\
390	0.0100002837596164\\
391	0.0100002837219574\\
392	0.0100002836836623\\
393	0.0100002836447191\\
394	0.0100002836051154\\
395	0.0100002835648387\\
396	0.010000283523877\\
397	0.0100002834822202\\
398	0.0100002834398633\\
399	0.0100002833968121\\
400	0.0100002833530918\\
401	0.0100002833087522\\
402	0.0100002832638447\\
403	0.0100002832183229\\
404	0.0100002831719307\\
405	0.0100002831246517\\
406	0.0100002830764681\\
407	0.0100002830273611\\
408	0.0100002829773097\\
409	0.010000282926292\\
410	0.0100002828742905\\
411	0.0100002828213006\\
412	0.0100002827672993\\
413	0.0100002827122627\\
414	0.0100002826561662\\
415	0.0100002825989842\\
416	0.0100002825406901\\
417	0.0100002824812564\\
418	0.0100002824206546\\
419	0.0100002823588548\\
420	0.010000282295826\\
421	0.010000282231536\\
422	0.0100002821659511\\
423	0.0100002820990362\\
424	0.0100002820307548\\
425	0.0100002819610685\\
426	0.0100002818899375\\
427	0.0100002818173197\\
428	0.0100002817431712\\
429	0.0100002816674453\\
430	0.0100002815900927\\
431	0.0100002815110593\\
432	0.0100002814302846\\
433	0.010000281347695\\
434	0.010000281263193\\
435	0.0100002811766399\\
436	0.0100002810878466\\
437	0.0100002809966367\\
438	0.0100002809030812\\
439	0.0100002808073389\\
440	0.0100002807093254\\
441	0.0100002806089503\\
442	0.0100002805061168\\
443	0.0100002804007212\\
444	0.0100002802926518\\
445	0.0100002801817893\\
446	0.0100002800680067\\
447	0.0100002799511711\\
448	0.0100002798311475\\
449	0.0100002797078077\\
450	0.0100002795810433\\
451	0.0100002794507622\\
452	0.0100002793167834\\
453	0.0100002791783705\\
454	0.0100002790329246\\
455	0.0100002788742218\\
456	0.0100002786912104\\
457	0.0100002784662658\\
458	0.0100002781385529\\
459	0.0100002777823409\\
460	0.0100002774191207\\
461	0.0100002770480306\\
462	0.0100002766683527\\
463	0.0100002762820748\\
464	0.0100002758890429\\
465	0.0100002754890043\\
466	0.0100002750816683\\
467	0.0100002746666901\\
468	0.0100002742436629\\
469	0.010000273812189\\
470	0.010000273372242\\
471	0.0100002729235318\\
472	0.0100002724657608\\
473	0.010000271998872\\
474	0.0100002715233001\\
475	0.0100002710401318\\
476	0.0100002705501455\\
477	0.0100002700492408\\
478	0.010000269532826\\
479	0.0100002689996552\\
480	0.0100002684479949\\
481	0.0100002678749656\\
482	0.0100002672747749\\
483	0.0100002666339708\\
484	0.0100002659191922\\
485	0.0100002650527385\\
486	0.0100002639818543\\
487	0.0100002628908232\\
488	0.0100002617790337\\
489	0.0100002606458778\\
490	0.0100002594907488\\
491	0.0100002583130897\\
492	0.0100002571124782\\
493	0.0100002556337237\\
494	0.0100002527280105\\
495	0.0100002497661292\\
496	0.0100002467509689\\
497	0.010000243677361\\
498	0.010000240524656\\
499	0.0100002372882573\\
500	0.0100002339625459\\
501	0.010000230539997\\
502	0.0100002270092879\\
503	0.0100002233519699\\
504	0.0100002195409957\\
505	0.0100002155711457\\
506	0.0100002114097909\\
507	0.0100002069276187\\
508	0.0100002017910705\\
509	0.0100001961855836\\
510	0.0100001905050262\\
511	0.010000184748613\\
512	0.0100001789157995\\
513	0.0100001730075422\\
514	0.0100001670608041\\
515	0.0100001610419521\\
516	0.0100001549299995\\
517	0.0100001487215882\\
518	0.0100001424139541\\
519	0.010000136005646\\
520	0.0100001294962251\\
521	0.0100001226945644\\
522	0.0100001156869292\\
523	0.010000108464679\\
524	0.0100001010177974\\
525	0.0100000931838069\\
526	0.010000084356173\\
527	0.0100000538977315\\
528	0.0100000193361654\\
529	0.00999998321785323\\
530	0.0099999452882706\\
531	0.00999990517493529\\
532	0.00999986185080286\\
533	0.00999973171419417\\
534	0.00999952840856016\\
535	0.00999931776418426\\
536	0.00999909913812807\\
537	0.00999887174402923\\
538	0.00999863468328567\\
539	0.00999838690736834\\
540	0.00999812704560558\\
541	0.00999779878085236\\
542	0.0099948011198057\\
543	0.009991775940215\\
544	0.00998872111588451\\
545	0.00998563430540921\\
546	0.00998251297075415\\
547	0.00997935439821524\\
548	0.00997615571946956\\
549	0.00997291395177169\\
550	0.00996962606791562\\
551	0.00996628911718749\\
552	0.00996290045236572\\
553	0.00995945824145994\\
554	0.00995607539716139\\
555	0.00995278083794119\\
556	0.00994957833519301\\
557	0.00994647191919197\\
558	0.0099434660064334\\
559	0.00994056475167603\\
560	0.00993777262166188\\
561	0.00993509473746594\\
562	0.00993253721631877\\
563	0.00993010821294715\\
564	0.00992781407422411\\
565	0.00992488752606354\\
566	0.00991049821720781\\
567	0.00989541754955392\\
568	0.00973893862780728\\
569	0.00954294358828082\\
570	0.00933480713277825\\
571	0.00911292764504251\\
572	0.00887542241305692\\
573	0.0086216917751517\\
574	0.008360185452806\\
575	0.00809066765029874\\
576	0.00781282768321296\\
577	0.00752645339503333\\
578	0.00723148144731463\\
579	0.00692771451397655\\
580	0.00661569752735087\\
581	0.00629589973465196\\
582	0.00597019483658525\\
583	0.00563888259460737\\
584	0.00530187182734894\\
585	0.00495904947440365\\
586	0.0046102952828595\\
587	0.00425551508307342\\
588	0.00389467562961874\\
589	0.00352786639485424\\
590	0.00315529010407331\\
591	0.00277712112335691\\
592	0.00239346664055391\\
593	0.00200447501126164\\
594	0.00161039185165808\\
595	0.00121163414212813\\
596	0.000808977057617982\\
597	0.000404030214303389\\
598	2.9204464504877e-07\\
599	0\\
600	0\\
};
\addplot [color=mycolor15,solid,forget plot]
  table[row sep=crcr]{%
1	0.0099993294716719\\
2	0.00999932946961846\\
3	0.00999932946752786\\
4	0.00999932946539942\\
5	0.00999932946323247\\
6	0.0099993294610263\\
7	0.00999932945878021\\
8	0.00999932945649348\\
9	0.00999932945416537\\
10	0.00999932945179514\\
11	0.00999932944938202\\
12	0.00999932944692525\\
13	0.00999932944442402\\
14	0.00999932944187754\\
15	0.009999329439285\\
16	0.00999932943664556\\
17	0.00999932943395837\\
18	0.00999932943122257\\
19	0.00999932942843729\\
20	0.00999932942560163\\
21	0.00999932942271469\\
22	0.00999932941977553\\
23	0.00999932941678321\\
24	0.00999932941373678\\
25	0.00999932941063526\\
26	0.00999932940747765\\
27	0.00999932940426295\\
28	0.00999932940099012\\
29	0.00999932939765812\\
30	0.00999932939426587\\
31	0.00999932939081229\\
32	0.00999932938729628\\
33	0.00999932938371671\\
34	0.00999932938007243\\
35	0.00999932937636227\\
36	0.00999932937258505\\
37	0.00999932936873956\\
38	0.00999932936482457\\
39	0.00999932936083882\\
40	0.00999932935678105\\
41	0.00999932935264994\\
42	0.00999932934844418\\
43	0.00999932934416243\\
44	0.00999932933980331\\
45	0.00999932933536544\\
46	0.00999932933084738\\
47	0.0099993293262477\\
48	0.00999932932156493\\
49	0.00999932931679757\\
50	0.0099993293119441\\
51	0.00999932930700296\\
52	0.00999932930197258\\
53	0.00999932929685135\\
54	0.00999932929163763\\
55	0.00999932928632976\\
56	0.00999932928092604\\
57	0.00999932927542475\\
58	0.00999932926982413\\
59	0.0099993292641224\\
60	0.00999932925831773\\
61	0.00999932925240827\\
62	0.00999932924639214\\
63	0.00999932924026742\\
64	0.00999932923403215\\
65	0.00999932922768434\\
66	0.00999932922122197\\
67	0.00999932921464298\\
68	0.00999932920794527\\
69	0.00999932920112671\\
70	0.00999932919418512\\
71	0.00999932918711829\\
72	0.00999932917992397\\
73	0.00999932917259987\\
74	0.00999932916514366\\
75	0.00999932915755295\\
76	0.00999932914982534\\
77	0.00999932914195835\\
78	0.0099993291339495\\
79	0.00999932912579623\\
80	0.00999932911749593\\
81	0.00999932910904599\\
82	0.00999932910044369\\
83	0.00999932909168632\\
84	0.00999932908277107\\
85	0.00999932907369513\\
86	0.00999932906445559\\
87	0.00999932905504954\\
88	0.00999932904547396\\
89	0.00999932903572584\\
90	0.00999932902580205\\
91	0.00999932901569946\\
92	0.00999932900541485\\
93	0.00999932899494496\\
94	0.00999932898428646\\
95	0.00999932897343597\\
96	0.00999932896239005\\
97	0.00999932895114518\\
98	0.00999932893969781\\
99	0.00999932892804429\\
100	0.00999932891618094\\
101	0.00999932890410399\\
102	0.00999932889180961\\
103	0.0099993288792939\\
104	0.0099993288665529\\
105	0.00999932885358256\\
106	0.00999932884037878\\
107	0.00999932882693737\\
108	0.00999932881325408\\
109	0.00999932879932457\\
110	0.00999932878514444\\
111	0.00999932877070918\\
112	0.00999932875601424\\
113	0.00999932874105496\\
114	0.0099993287258266\\
115	0.00999932871032436\\
116	0.00999932869454332\\
117	0.0099993286784785\\
118	0.00999932866212481\\
119	0.00999932864547709\\
120	0.00999932862853007\\
121	0.0099993286112784\\
122	0.00999932859371662\\
123	0.00999932857583919\\
124	0.00999932855764046\\
125	0.00999932853911468\\
126	0.00999932852025601\\
127	0.00999932850105849\\
128	0.00999932848151605\\
129	0.00999932846162254\\
130	0.00999932844137168\\
131	0.00999932842075708\\
132	0.00999932839977223\\
133	0.00999932837841052\\
134	0.00999932835666521\\
135	0.00999932833452945\\
136	0.00999932831199627\\
137	0.00999932828905855\\
138	0.00999932826570908\\
139	0.0099993282419405\\
140	0.00999932821774533\\
141	0.00999932819311594\\
142	0.00999932816804458\\
143	0.00999932814252336\\
144	0.00999932811654425\\
145	0.00999932809009907\\
146	0.00999932806317951\\
147	0.00999932803577709\\
148	0.00999932800788321\\
149	0.00999932797948908\\
150	0.00999932795058578\\
151	0.00999932792116424\\
152	0.00999932789121519\\
153	0.00999932786072924\\
154	0.0099993278296968\\
155	0.00999932779810814\\
156	0.00999932776595332\\
157	0.00999932773322226\\
158	0.00999932769990468\\
159	0.00999932766599012\\
160	0.00999932763146793\\
161	0.0099993275963273\\
162	0.0099993275605572\\
163	0.0099993275241464\\
164	0.00999932748708349\\
165	0.00999932744935685\\
166	0.00999932741095465\\
167	0.00999932737186486\\
168	0.00999932733207523\\
169	0.00999932729157329\\
170	0.00999932725034635\\
171	0.0099993272083815\\
172	0.0099993271656656\\
173	0.00999932712218528\\
174	0.00999932707792692\\
175	0.00999932703287667\\
176	0.00999932698702043\\
177	0.00999932694034386\\
178	0.00999932689283234\\
179	0.00999932684447103\\
180	0.00999932679524479\\
181	0.00999932674513823\\
182	0.00999932669413568\\
183	0.00999932664222121\\
184	0.00999932658937857\\
185	0.00999932653559127\\
186	0.00999932648084248\\
187	0.00999932642511511\\
188	0.00999932636839175\\
189	0.00999932631065468\\
190	0.00999932625188587\\
191	0.00999932619206697\\
192	0.00999932613117929\\
193	0.00999932606920384\\
194	0.00999932600612126\\
195	0.00999932594191187\\
196	0.00999932587655564\\
197	0.00999932581003216\\
198	0.00999932574232068\\
199	0.00999932567340008\\
200	0.00999932560324887\\
201	0.00999932553184515\\
202	0.00999932545916667\\
203	0.00999932538519075\\
204	0.00999932530989434\\
205	0.00999932523325396\\
206	0.00999932515524571\\
207	0.00999932507584528\\
208	0.00999932499502793\\
209	0.00999932491276845\\
210	0.00999932482904122\\
211	0.00999932474382014\\
212	0.00999932465707866\\
213	0.00999932456878976\\
214	0.00999932447892592\\
215	0.00999932438745914\\
216	0.00999932429436094\\
217	0.00999932419960231\\
218	0.00999932410315373\\
219	0.00999932400498516\\
220	0.00999932390506602\\
221	0.00999932380336519\\
222	0.00999932369985099\\
223	0.00999932359449118\\
224	0.00999932348725295\\
225	0.00999932337810289\\
226	0.00999932326700701\\
227	0.00999932315393072\\
228	0.00999932303883878\\
229	0.00999932292169536\\
230	0.00999932280246397\\
231	0.00999932268110747\\
232	0.00999932255758806\\
233	0.00999932243186726\\
234	0.00999932230390592\\
235	0.00999932217366417\\
236	0.00999932204110144\\
237	0.00999932190617643\\
238	0.00999932176884709\\
239	0.00999932162907065\\
240	0.00999932148680354\\
241	0.00999932134200143\\
242	0.00999932119461919\\
243	0.00999932104461088\\
244	0.00999932089192975\\
245	0.0099993207365282\\
246	0.00999932057835777\\
247	0.00999932041736915\\
248	0.00999932025351213\\
249	0.00999932008673561\\
250	0.00999931991698756\\
251	0.00999931974421503\\
252	0.00999931956836411\\
253	0.00999931938937993\\
254	0.00999931920720662\\
255	0.00999931902178732\\
256	0.00999931883306414\\
257	0.00999931864097814\\
258	0.00999931844546934\\
259	0.00999931824647667\\
260	0.00999931804393796\\
261	0.00999931783778991\\
262	0.00999931762796809\\
263	0.00999931741440691\\
264	0.0099993171970396\\
265	0.00999931697579818\\
266	0.00999931675061343\\
267	0.0099993165214149\\
268	0.00999931628813086\\
269	0.00999931605068828\\
270	0.0099993158090128\\
271	0.00999931556302874\\
272	0.00999931531265904\\
273	0.00999931505782528\\
274	0.0099993147984476\\
275	0.00999931453444469\\
276	0.00999931426573378\\
277	0.00999931399223057\\
278	0.00999931371384927\\
279	0.00999931343050251\\
280	0.00999931314210137\\
281	0.00999931284855528\\
282	0.00999931254977206\\
283	0.00999931224565783\\
284	0.00999931193611706\\
285	0.00999931162105244\\
286	0.00999931130036492\\
287	0.00999931097395366\\
288	0.00999931064171599\\
289	0.00999931030354737\\
290	0.00999930995934138\\
291	0.00999930960898969\\
292	0.00999930925238198\\
293	0.00999930888940597\\
294	0.00999930851994732\\
295	0.00999930814388966\\
296	0.00999930776111449\\
297	0.0099993073715012\\
298	0.009999306974927\\
299	0.00999930657126689\\
300	0.00999930616039365\\
301	0.00999930574217775\\
302	0.00999930531648737\\
303	0.00999930488318832\\
304	0.00999930444214405\\
305	0.00999930399321555\\
306	0.0099993035362614\\
307	0.00999930307113768\\
308	0.00999930259769793\\
309	0.00999930211579314\\
310	0.00999930162527162\\
311	0.00999930112597899\\
312	0.00999930061775813\\
313	0.00999930010044936\\
314	0.00999929957389025\\
315	0.00999929903791562\\
316	0.0099992984923575\\
317	0.00999929793704509\\
318	0.00999929737180476\\
319	0.00999929679646003\\
320	0.00999929621083155\\
321	0.00999929561473706\\
322	0.00999929500799143\\
323	0.00999929439040663\\
324	0.00999929376179172\\
325	0.00999929312195288\\
326	0.00999929247069341\\
327	0.00999929180781369\\
328	0.00999929113311117\\
329	0.00999929044638015\\
330	0.00999928974741139\\
331	0.00999928903599092\\
332	0.00999928831189758\\
333	0.00999928757489793\\
334	0.0099992868247374\\
335	0.00999928606112985\\
336	0.00999928528375961\\
337	0.00999928449233488\\
338	0.00999928368671541\\
339	0.00999928286686936\\
340	0.00999928203259543\\
341	0.00999928118365192\\
342	0.00999928031977998\\
343	0.00999927944068345\\
344	0.00999927854599073\\
345	0.00999927763520075\\
346	0.00999927670767158\\
347	0.00999927576287146\\
348	0.00999927480114783\\
349	0.00999927382301776\\
350	0.00999927282832036\\
351	0.00999927181677662\\
352	0.00999927078810297\\
353	0.00999926974201125\\
354	0.00999926867820868\\
355	0.0099992675963982\\
356	0.00999926649627893\\
357	0.00999926537754606\\
358	0.0099992642398888\\
359	0.00999926308299162\\
360	0.0099992619065342\\
361	0.0099992607101914\\
362	0.00999925949363315\\
363	0.00999925825652447\\
364	0.00999925699852534\\
365	0.00999925571929069\\
366	0.0099992544184703\\
367	0.00999925309570874\\
368	0.00999925175064532\\
369	0.009999250382914\\
370	0.00999924899214341\\
371	0.00999924757795682\\
372	0.00999924613997222\\
373	0.00999924467780235\\
374	0.0099992431910543\\
375	0.00999924167932856\\
376	0.00999924014221759\\
377	0.00999923857930842\\
378	0.00999923699019723\\
379	0.00999923537449653\\
380	0.00999923373186045\\
381	0.0099992320620289\\
382	0.0099992303648864\\
383	0.00999922864048714\\
384	0.00999922688889\\
385	0.0099992251095598\\
386	0.00999922330070009\\
387	0.00999922146127299\\
388	0.00999921959069207\\
389	0.00999921768840462\\
390	0.0099992157538371\\
391	0.00999921378639325\\
392	0.00999921178545247\\
393	0.00999920975036931\\
394	0.00999920768047584\\
395	0.00999920557509121\\
396	0.00999920343354766\\
397	0.00999920125525213\\
398	0.00999919903981771\\
399	0.00999919678731162\\
400	0.00999919449862671\\
401	0.00999919217573563\\
402	0.0099991898208095\\
403	0.00999918743212194\\
404	0.00999918499899439\\
405	0.00999918251891641\\
406	0.00999917999097951\\
407	0.00999917741414851\\
408	0.00999917478725977\\
409	0.00999917210902027\\
410	0.00999916937824074\\
411	0.00999916659449198\\
412	0.00999916375757198\\
413	0.00999916086623067\\
414	0.00999915791917385\\
415	0.00999915491506151\\
416	0.00999915185250656\\
417	0.00999914873007448\\
418	0.00999914554628377\\
419	0.00999914229960513\\
420	0.00999913898845258\\
421	0.00999913561115313\\
422	0.00999913216589674\\
423	0.00999912865076033\\
424	0.0099991250638269\\
425	0.00999912140309305\\
426	0.00999911766646265\\
427	0.00999911385173922\\
428	0.00999910995661611\\
429	0.00999910597866151\\
430	0.0099991019152918\\
431	0.00999909776371751\\
432	0.00999909352082707\\
433	0.00999908918293894\\
434	0.00999908474531591\\
435	0.00999908020140347\\
436	0.00999907554235725\\
437	0.00999907075949376\\
438	0.00999906585399951\\
439	0.00999906083218615\\
440	0.0099990556912567\\
441	0.00999905042646166\\
442	0.00999904503269721\\
443	0.00999903950445559\\
444	0.00999903383576376\\
445	0.00999902802012308\\
446	0.00999902205051696\\
447	0.00999901591968957\\
448	0.00999900962088153\\
449	0.00999900314709705\\
450	0.00999899649175576\\
451	0.00999898964872395\\
452	0.00999898260774513\\
453	0.00999897533405877\\
454	0.00999896770985755\\
455	0.00999895944454393\\
456	0.00999894998388565\\
457	0.00999893838990642\\
458	0.00999892236539694\\
459	0.00999889969768127\\
460	0.00999887523716839\\
461	0.00999885026190805\\
462	0.00999882473405357\\
463	0.00999879872460709\\
464	0.00999877223404172\\
465	0.0099987452434011\\
466	0.00999871773053609\\
467	0.00999868966956539\\
468	0.00999866102980771\\
469	0.00999863177775147\\
470	0.00999860189300072\\
471	0.00999857136223286\\
472	0.00999854015882764\\
473	0.00999850826004149\\
474	0.00999847566151176\\
475	0.00999844238425681\\
476	0.00999840843607569\\
477	0.00999837362278948\\
478	0.00999833769663121\\
479	0.0099983005414136\\
480	0.00999826205301255\\
481	0.00999822207094425\\
482	0.00999818030273864\\
483	0.00999813612354068\\
484	0.00999808804373537\\
485	0.00999803258721717\\
486	0.00999795214372719\\
487	0.00999781736381448\\
488	0.00999767961429178\\
489	0.00999753875507356\\
490	0.00999739466506454\\
491	0.00999724721376824\\
492	0.00999709624852324\\
493	0.00999693121355686\\
494	0.00999670530755466\\
495	0.00999647416658922\\
496	0.00999623775903836\\
497	0.00999599571349533\\
498	0.00999574697161254\\
499	0.0099954910077634\\
500	0.00999522733925143\\
501	0.00999495537165017\\
502	0.00999467431269238\\
503	0.00999438302117948\\
504	0.00999407991692828\\
505	0.0099937641613075\\
506	0.00999343394490521\\
507	0.00999308334337721\\
508	0.00999269762074172\\
509	0.0099914554365218\\
510	0.00998940124696233\\
511	0.00998732863778075\\
512	0.0099852370976057\\
513	0.00998312610223551\\
514	0.00998099650252101\\
515	0.00997884616187541\\
516	0.00997667304539808\\
517	0.00997447597048032\\
518	0.00997225383250046\\
519	0.00997000550783338\\
520	0.0099677298408233\\
521	0.00996541777538252\\
522	0.00996307154385438\\
523	0.00996068939531412\\
524	0.00995826957185022\\
525	0.00995580426317674\\
526	0.00995326696843847\\
527	0.00994980420231118\\
528	0.00994613560965982\\
529	0.00994236367068748\\
530	0.0099384752067908\\
531	0.0099344521415761\\
532	0.00993027162010349\\
533	0.00992252256538219\\
534	0.00991175654377404\\
535	0.00990070756465675\\
536	0.00988934850347928\\
537	0.00987764589693117\\
538	0.00986556114281452\\
539	0.00985304879076982\\
540	0.00984004906369896\\
541	0.00982416100283728\\
542	0.00969282387754647\\
543	0.00955665231647043\\
544	0.00941520952431028\\
545	0.00926799773951631\\
546	0.00911444673581855\\
547	0.00895390159476861\\
548	0.00878560714903328\\
549	0.00860868904960451\\
550	0.00842213090002012\\
551	0.00822474711779364\\
552	0.00801515303577693\\
553	0.00779174238648091\\
554	0.00756053130714489\\
555	0.0073227869414988\\
556	0.00707816704069707\\
557	0.00682631110018469\\
558	0.00656685138566235\\
559	0.00629933175157336\\
560	0.00602325745472997\\
561	0.00573812592479245\\
562	0.00544347453026945\\
563	0.00513904679861026\\
564	0.00482435547151584\\
565	0.00449969466636478\\
566	0.00417568949117384\\
567	0.00384096269838464\\
568	0.00363912200043192\\
569	0.00346881456343741\\
570	0.00330378970436501\\
571	0.00314579069162\\
572	0.0029969195629914\\
573	0.00285788076580331\\
574	0.0027194058618991\\
575	0.00258161828951404\\
576	0.00244481403871134\\
577	0.00230928115503963\\
578	0.00217527788184951\\
579	0.00204300485446921\\
580	0.00191264896058479\\
581	0.00178427582037178\\
582	0.00165614535808679\\
583	0.00152811363328425\\
584	0.00140051135640073\\
585	0.00127372219294322\\
586	0.00114818338399011\\
587	0.00102441882820826\\
588	0.000903028512395\\
589	0.000784662617985752\\
590	0.000670016384725282\\
591	0.00055982374888513\\
592	0.000454847943832255\\
593	0.000355865647533659\\
594	0.000263643164224137\\
595	0.000178902643795078\\
596	0.000102272945261959\\
597	3.41569359503316e-05\\
598	2.9204464504877e-07\\
599	0\\
600	0\\
};
\addplot [color=mycolor16,solid,forget plot]
  table[row sep=crcr]{%
1	0.00996936184596852\\
2	0.009969361812907\\
3	0.00996936177924712\\
4	0.00996936174497806\\
5	0.00996936171008882\\
6	0.0099693616745682\\
7	0.00996936163840478\\
8	0.00996936160158697\\
9	0.00996936156410294\\
10	0.00996936152594065\\
11	0.00996936148708786\\
12	0.0099693614475321\\
13	0.00996936140726067\\
14	0.00996936136626065\\
15	0.00996936132451889\\
16	0.00996936128202199\\
17	0.00996936123875631\\
18	0.00996936119470799\\
19	0.00996936114986289\\
20	0.00996936110420663\\
21	0.00996936105772457\\
22	0.0099693610104018\\
23	0.00996936096222315\\
24	0.00996936091317318\\
25	0.00996936086323616\\
26	0.00996936081239608\\
27	0.00996936076063665\\
28	0.00996936070794129\\
29	0.00996936065429311\\
30	0.00996936059967491\\
31	0.00996936054406921\\
32	0.00996936048745819\\
33	0.00996936042982372\\
34	0.00996936037114734\\
35	0.00996936031141026\\
36	0.00996936025059335\\
37	0.00996936018867715\\
38	0.00996936012564183\\
39	0.00996936006146721\\
40	0.00996935999613276\\
41	0.00996935992961757\\
42	0.00996935986190036\\
43	0.00996935979295946\\
44	0.00996935972277281\\
45	0.00996935965131798\\
46	0.0099693595785721\\
47	0.00996935950451191\\
48	0.00996935942911374\\
49	0.00996935935235347\\
50	0.00996935927420657\\
51	0.00996935919464807\\
52	0.00996935911365253\\
53	0.00996935903119407\\
54	0.00996935894724636\\
55	0.00996935886178256\\
56	0.00996935877477539\\
57	0.00996935868619705\\
58	0.00996935859601926\\
59	0.00996935850421322\\
60	0.00996935841074963\\
61	0.00996935831559864\\
62	0.0099693582187299\\
63	0.00996935812011248\\
64	0.00996935801971492\\
65	0.00996935791750519\\
66	0.00996935781345069\\
67	0.00996935770751822\\
68	0.009969357599674\\
69	0.00996935748988366\\
70	0.00996935737811217\\
71	0.00996935726432392\\
72	0.00996935714848263\\
73	0.00996935703055139\\
74	0.00996935691049262\\
75	0.00996935678826808\\
76	0.00996935666383882\\
77	0.00996935653716523\\
78	0.00996935640820695\\
79	0.00996935627692293\\
80	0.00996935614327138\\
81	0.00996935600720974\\
82	0.00996935586869473\\
83	0.00996935572768225\\
84	0.00996935558412745\\
85	0.00996935543798466\\
86	0.00996935528920739\\
87	0.00996935513774834\\
88	0.00996935498355933\\
89	0.00996935482659136\\
90	0.00996935466679452\\
91	0.00996935450411804\\
92	0.0099693543385102\\
93	0.00996935416991841\\
94	0.00996935399828909\\
95	0.00996935382356775\\
96	0.00996935364569888\\
97	0.00996935346462602\\
98	0.00996935328029167\\
99	0.00996935309263733\\
100	0.00996935290160345\\
101	0.00996935270712939\\
102	0.00996935250915347\\
103	0.00996935230761288\\
104	0.0099693521024437\\
105	0.00996935189358088\\
106	0.00996935168095817\\
107	0.0099693514645082\\
108	0.00996935124416234\\
109	0.00996935101985078\\
110	0.00996935079150243\\
111	0.00996935055904496\\
112	0.00996935032240473\\
113	0.00996935008150682\\
114	0.00996934983627493\\
115	0.00996934958663143\\
116	0.0099693493324973\\
117	0.00996934907379211\\
118	0.009969348810434\\
119	0.00996934854233965\\
120	0.00996934826942425\\
121	0.00996934799160149\\
122	0.00996934770878352\\
123	0.00996934742088092\\
124	0.00996934712780268\\
125	0.00996934682945618\\
126	0.00996934652574712\\
127	0.00996934621657956\\
128	0.00996934590185584\\
129	0.00996934558147654\\
130	0.00996934525534049\\
131	0.00996934492334473\\
132	0.00996934458538443\\
133	0.00996934424135295\\
134	0.00996934389114169\\
135	0.00996934353464017\\
136	0.00996934317173592\\
137	0.00996934280231446\\
138	0.0099693424262593\\
139	0.00996934204345186\\
140	0.00996934165377145\\
141	0.00996934125709524\\
142	0.00996934085329822\\
143	0.00996934044225315\\
144	0.00996934002383052\\
145	0.00996933959789853\\
146	0.00996933916432303\\
147	0.00996933872296749\\
148	0.00996933827369294\\
149	0.00996933781635796\\
150	0.00996933735081861\\
151	0.00996933687692837\\
152	0.00996933639453814\\
153	0.00996933590349615\\
154	0.00996933540364796\\
155	0.00996933489483636\\
156	0.00996933437690135\\
157	0.0099693338496801\\
158	0.00996933331300687\\
159	0.00996933276671299\\
160	0.00996933221062677\\
161	0.00996933164457348\\
162	0.00996933106837531\\
163	0.00996933048185124\\
164	0.00996932988481708\\
165	0.00996932927708533\\
166	0.00996932865846519\\
167	0.00996932802876245\\
168	0.00996932738777944\\
169	0.009969326735315\\
170	0.00996932607116438\\
171	0.0099693253951192\\
172	0.00996932470696737\\
173	0.00996932400649304\\
174	0.0099693232934765\\
175	0.00996932256769416\\
176	0.00996932182891845\\
177	0.00996932107691774\\
178	0.0099693203114563\\
179	0.00996931953229421\\
180	0.00996931873918727\\
181	0.00996931793188694\\
182	0.00996931711014028\\
183	0.00996931627368983\\
184	0.00996931542227356\\
185	0.00996931455562478\\
186	0.00996931367347207\\
187	0.00996931277553915\\
188	0.00996931186154485\\
189	0.009969310931203\\
190	0.00996930998422232\\
191	0.00996930902030636\\
192	0.0099693080391534\\
193	0.00996930704045633\\
194	0.00996930602390258\\
195	0.00996930498917402\\
196	0.00996930393594686\\
197	0.00996930286389154\\
198	0.00996930177267263\\
199	0.0099693006619487\\
200	0.00996929953137229\\
201	0.00996929838058969\\
202	0.00996929720924093\\
203	0.0099692960169596\\
204	0.00996929480337277\\
205	0.00996929356810084\\
206	0.00996929231075747\\
207	0.00996929103094939\\
208	0.00996928972827635\\
209	0.00996928840233092\\
210	0.00996928705269843\\
211	0.00996928567895679\\
212	0.00996928428067636\\
213	0.00996928285741984\\
214	0.00996928140874213\\
215	0.00996927993419013\\
216	0.00996927843330267\\
217	0.00996927690561034\\
218	0.00996927535063532\\
219	0.00996927376789122\\
220	0.00996927215688299\\
221	0.00996927051710667\\
222	0.0099692688480493\\
223	0.00996926714918873\\
224	0.00996926541999343\\
225	0.00996926365992238\\
226	0.00996926186842481\\
227	0.0099692600449401\\
228	0.00996925818889757\\
229	0.00996925629971628\\
230	0.00996925437680486\\
231	0.00996925241956132\\
232	0.00996925042737284\\
233	0.00996924839961561\\
234	0.00996924633565455\\
235	0.00996924423484321\\
236	0.00996924209652345\\
237	0.00996923992002532\\
238	0.00996923770466676\\
239	0.00996923544975345\\
240	0.00996923315457854\\
241	0.00996923081842241\\
242	0.00996922844055249\\
243	0.00996922602022294\\
244	0.00996922355667447\\
245	0.00996922104913407\\
246	0.00996921849681475\\
247	0.00996921589891527\\
248	0.00996921325461991\\
249	0.00996921056309817\\
250	0.0099692078235045\\
251	0.00996920503497803\\
252	0.00996920219664228\\
253	0.00996919930760488\\
254	0.00996919636695724\\
255	0.00996919337377428\\
256	0.00996919032711411\\
257	0.00996918722601773\\
258	0.00996918406950867\\
259	0.00996918085659272\\
260	0.00996917758625754\\
261	0.00996917425747237\\
262	0.00996917086918766\\
263	0.0099691674203347\\
264	0.0099691639098253\\
265	0.00996916033655136\\
266	0.00996915669938455\\
267	0.00996915299717585\\
268	0.00996914922875514\\
269	0.00996914539293081\\
270	0.00996914148848932\\
271	0.00996913751419497\\
272	0.00996913346878982\\
273	0.00996912935099353\\
274	0.00996912515950223\\
275	0.00996912089298756\\
276	0.00996911655009667\\
277	0.00996911212945185\\
278	0.0099691076296501\\
279	0.00996910304926263\\
280	0.00996909838683436\\
281	0.00996909364088347\\
282	0.00996908880990086\\
283	0.00996908389234966\\
284	0.00996907888666469\\
285	0.00996907379125197\\
286	0.00996906860448814\\
287	0.00996906332471995\\
288	0.00996905795026368\\
289	0.00996905247940459\\
290	0.00996904691039637\\
291	0.00996904124146049\\
292	0.00996903547078571\\
293	0.00996902959652741\\
294	0.00996902361680701\\
295	0.00996901752971138\\
296	0.0099690113332922\\
297	0.00996900502556532\\
298	0.00996899860451019\\
299	0.00996899206806916\\
300	0.00996898541414689\\
301	0.0099689786406097\\
302	0.00996897174528493\\
303	0.00996896472596036\\
304	0.00996895758038362\\
305	0.00996895030626168\\
306	0.00996894290126038\\
307	0.00996893536300399\\
308	0.00996892768907455\\
309	0.00996891987701054\\
310	0.00996891192430447\\
311	0.0099689038284003\\
312	0.0099688955866946\\
313	0.00996888719654264\\
314	0.00996887865525406\\
315	0.00996886996009152\\
316	0.00996886110827023\\
317	0.00996885209695749\\
318	0.00996884292327232\\
319	0.00996883358428505\\
320	0.00996882407701709\\
321	0.00996881439844069\\
322	0.00996880454547877\\
323	0.00996879451500491\\
324	0.0099687843038434\\
325	0.00996877390876939\\
326	0.00996876332650918\\
327	0.00996875255374065\\
328	0.00996874158709383\\
329	0.00996873042315152\\
330	0.00996871905844984\\
331	0.00996870748947838\\
332	0.00996869571267886\\
333	0.00996868372444004\\
334	0.00996867152108336\\
335	0.00996865909882866\\
336	0.00996864645372656\\
337	0.00996863358157793\\
338	0.00996862047806023\\
339	0.0099686071397144\\
340	0.00996859356415082\\
341	0.00996857974775333\\
342	0.00996856568664523\\
343	0.00996855137690936\\
344	0.0099685368145441\\
345	0.00996852199534319\\
346	0.00996850691460096\\
347	0.00996849156656939\\
348	0.00996847594453828\\
349	0.00996846004729916\\
350	0.00996844387572912\\
351	0.00996842742591287\\
352	0.00996841069300453\\
353	0.00996839367206885\\
354	0.00996837635808168\\
355	0.00996835874594157\\
356	0.00996834083049468\\
357	0.00996832260653619\\
358	0.00996830406872731\\
359	0.00996828521163068\\
360	0.00996826602971775\\
361	0.00996824651736727\\
362	0.00996822666886374\\
363	0.00996820647839577\\
364	0.0099681859400543\\
365	0.00996816504783079\\
366	0.00996814379561519\\
367	0.00996812217719412\\
368	0.00996810018624911\\
369	0.00996807781635557\\
370	0.00996805506098307\\
371	0.00996803191349778\\
372	0.00996800836716694\\
373	0.00996798441516115\\
374	0.00996796005054027\\
375	0.00996793526619519\\
376	0.00996791005473633\\
377	0.00996788440846369\\
378	0.00996785831972465\\
379	0.00996783178076111\\
380	0.00996780478369372\\
381	0.00996777732058907\\
382	0.0099677493836164\\
383	0.00996772096535449\\
384	0.00996769205922427\\
385	0.00996766265946431\\
386	0.00996763275824557\\
387	0.0099676023368896\\
388	0.00996757138155168\\
389	0.00996753988143256\\
390	0.00996750782569727\\
391	0.00996747520304785\\
392	0.00996744200167978\\
393	0.00996740820923558\\
394	0.00996737381275691\\
395	0.00996733879863829\\
396	0.00996730315259064\\
397	0.00996726685963096\\
398	0.00996722990415092\\
399	0.00996719227016634\\
400	0.00996715394200545\\
401	0.00996711490592282\\
402	0.00996707515313806\\
403	0.00996703468212015\\
404	0.00996699348243015\\
405	0.00996695146777249\\
406	0.00996690860947114\\
407	0.00996686488848824\\
408	0.0099668202829035\\
409	0.00996677476655935\\
410	0.00996672830918364\\
411	0.00996668088824669\\
412	0.00996663252183162\\
413	0.00996658319375107\\
414	0.0099665328798882\\
415	0.00996648155525311\\
416	0.00996642919398741\\
417	0.00996637576941329\\
418	0.00996632125413397\\
419	0.00996626562011802\\
420	0.00996620883848122\\
421	0.00996615087834981\\
422	0.00996609170467166\\
423	0.00996603127879021\\
424	0.00996596956358955\\
425	0.00996590652082387\\
426	0.00996584211038325\\
427	0.00996577629015359\\
428	0.00996570901586141\\
429	0.00996564024089977\\
430	0.00996556991612889\\
431	0.00996549798963701\\
432	0.00996542440642666\\
433	0.00996534910793691\\
434	0.00996527203116697\\
435	0.00996519310679865\\
436	0.00996511225496241\\
437	0.00996502937706096\\
438	0.00996494435245766\\
439	0.0099648571154515\\
440	0.00996476764919769\\
441	0.0099646758685781\\
442	0.00996458166758121\\
443	0.00996448493022034\\
444	0.00996438552820102\\
445	0.00996428331792217\\
446	0.00996417813876847\\
447	0.00996406982018708\\
448	0.00996395820526764\\
449	0.00996384310879453\\
450	0.00996372431229352\\
451	0.00996360156980491\\
452	0.00996347459852649\\
453	0.00996334303737268\\
454	0.00996320627394929\\
455	0.00996306276019777\\
456	0.0099629082738575\\
457	0.00996272160641869\\
458	0.00996183313692551\\
459	0.00996070904071886\\
460	0.00995951123324781\\
461	0.00995829345035272\\
462	0.00995705525002235\\
463	0.00995579604567714\\
464	0.00995451605156727\\
465	0.00995321492125422\\
466	0.00995189217067572\\
467	0.00995054729742794\\
468	0.00994917977205121\\
469	0.00994778901926564\\
470	0.0099463744863286\\
471	0.00994493653174981\\
472	0.00994347522703532\\
473	0.00994198965175815\\
474	0.00994047858563699\\
475	0.00993894072108857\\
476	0.00993737472282705\\
477	0.00993577929934507\\
478	0.00993415176582555\\
479	0.00993248904722541\\
480	0.00993078956095074\\
481	0.00992905159421222\\
482	0.00992727297102582\\
483	0.00992545098097248\\
484	0.00992358152455413\\
485	0.00992165486828391\\
486	0.00991903208414031\\
487	0.00991417478746761\\
488	0.00990921257372146\\
489	0.00990413980369272\\
490	0.00989895150439706\\
491	0.00989364231171935\\
492	0.00988820558811218\\
493	0.0098826437066687\\
494	0.00987700135821326\\
495	0.00987121387196477\\
496	0.00986527323551472\\
497	0.00985917101724499\\
498	0.00985289670902821\\
499	0.00984643355755357\\
500	0.00983976795828853\\
501	0.00983288524363733\\
502	0.00982576869275228\\
503	0.00981839896031053\\
504	0.00981075281302382\\
505	0.00980279932324318\\
506	0.00979451444131653\\
507	0.00978591827079131\\
508	0.00977698445580002\\
509	0.00973195803472746\\
510	0.00965105588747158\\
511	0.00956811607322863\\
512	0.0094830290924446\\
513	0.00939567396480019\\
514	0.00930591449011373\\
515	0.00921360188318442\\
516	0.00911855396417855\\
517	0.00902057789311013\\
518	0.00891946808231931\\
519	0.00881499712801225\\
520	0.00870691226604886\\
521	0.00859494030878527\\
522	0.00847876237623706\\
523	0.0083580246132408\\
524	0.00823233644397344\\
525	0.0081012817910472\\
526	0.00796438806893423\\
527	0.0078219666668102\\
528	0.00767259631689416\\
529	0.00751537094821622\\
530	0.00734937101066814\\
531	0.00717352887574939\\
532	0.00698838745773369\\
533	0.00680199976078023\\
534	0.00661369802893454\\
535	0.00642052120659462\\
536	0.00622223797007795\\
537	0.00601859692696998\\
538	0.00580931975662197\\
539	0.00559409486758147\\
540	0.00537257221695576\\
541	0.0051467254174597\\
542	0.00503134269814583\\
543	0.00491423520334139\\
544	0.00479541795300435\\
545	0.00467512378405048\\
546	0.00455364026061907\\
547	0.00443132099421412\\
548	0.00430862748349586\\
549	0.00418616509956596\\
550	0.00406472203417226\\
551	0.00394531883637027\\
552	0.00382927076407412\\
553	0.00371826159400786\\
554	0.00360615050474313\\
555	0.00349161200839405\\
556	0.00337507055368945\\
557	0.00325756846643711\\
558	0.00314106624760566\\
559	0.00302590863410745\\
560	0.00291248032569543\\
561	0.00280120628054289\\
562	0.00269255205987361\\
563	0.00258702102201481\\
564	0.00248514913993936\\
565	0.00238748317032211\\
566	0.00229440259723362\\
567	0.00220660880533274\\
568	0.0021222859857676\\
569	0.0020401299752841\\
570	0.00195865682970082\\
571	0.00187781943912992\\
572	0.00179745585614607\\
573	0.00171731732385599\\
574	0.00163747296345394\\
575	0.00155798189225075\\
576	0.00147888770982828\\
577	0.00140021641993085\\
578	0.00132197584547392\\
579	0.00124415743122899\\
580	0.00116673974565477\\
581	0.00108970039256183\\
582	0.00101315452783778\\
583	0.0009372510557956\\
584	0.000862139893158467\\
585	0.000787968575672158\\
586	0.000714878431535376\\
587	0.000642998962491715\\
588	0.000572441638192295\\
589	0.000503293309197684\\
590	0.000435609150453159\\
591	0.000369387784493793\\
592	0.00030456560169281\\
593	0.000241014280104653\\
594	0.000178532861776521\\
595	0.000116902108299214\\
596	6.17886640995154e-05\\
597	2.28062284332058e-05\\
598	2.9204464504877e-07\\
599	0\\
600	0\\
};
\addplot [color=mycolor17,solid,forget plot]
  table[row sep=crcr]{%
1	0.00993093460416607\\
2	0.00993093371141091\\
3	0.00993093280249765\\
4	0.00993093187713434\\
5	0.0099309309350238\\
6	0.00993092997586345\\
7	0.0099309289993453\\
8	0.0099309280051558\\
9	0.00993092699297573\\
10	0.00993092596248016\\
11	0.00993092491333828\\
12	0.00993092384521334\\
13	0.00993092275776252\\
14	0.00993092165063682\\
15	0.00993092052348097\\
16	0.0099309193759333\\
17	0.00993091820762563\\
18	0.00993091701818316\\
19	0.00993091580722433\\
20	0.00993091457436074\\
21	0.00993091331919698\\
22	0.00993091204133054\\
23	0.00993091074035165\\
24	0.00993090941584318\\
25	0.00993090806738052\\
26	0.00993090669453139\\
27	0.00993090529685576\\
28	0.00993090387390566\\
29	0.00993090242522511\\
30	0.00993090095034989\\
31	0.00993089944880746\\
32	0.00993089792011679\\
33	0.00993089636378817\\
34	0.00993089477932314\\
35	0.00993089316621425\\
36	0.00993089152394494\\
37	0.00993088985198937\\
38	0.00993088814981226\\
39	0.00993088641686871\\
40	0.00993088465260403\\
41	0.00993088285645359\\
42	0.00993088102784259\\
43	0.00993087916618593\\
44	0.00993087727088801\\
45	0.00993087534134252\\
46	0.00993087337693229\\
47	0.00993087137702905\\
48	0.00993086934099328\\
49	0.00993086726817395\\
50	0.0099308651579084\\
51	0.00993086300952203\\
52	0.00993086082232817\\
53	0.00993085859562783\\
54	0.00993085632870949\\
55	0.00993085402084884\\
56	0.00993085167130862\\
57	0.00993084927933834\\
58	0.00993084684417404\\
59	0.00993084436503808\\
60	0.00993084184113889\\
61	0.0099308392716707\\
62	0.0099308366558133\\
63	0.0099308339927318\\
64	0.00993083128157635\\
65	0.00993082852148184\\
66	0.00993082571156773\\
67	0.00993082285093764\\
68	0.00993081993867919\\
69	0.00993081697386364\\
70	0.00993081395554562\\
71	0.00993081088276285\\
72	0.0099308077545358\\
73	0.00993080456986743\\
74	0.00993080132774285\\
75	0.00993079802712901\\
76	0.00993079466697437\\
77	0.00993079124620859\\
78	0.00993078776374216\\
79	0.0099307842184661\\
80	0.00993078060925159\\
81	0.00993077693494961\\
82	0.00993077319439063\\
83	0.00993076938638418\\
84	0.0099307655097185\\
85	0.00993076156316018\\
86	0.00993075754545378\\
87	0.0099307534553214\\
88	0.00993074929146231\\
89	0.00993074505255253\\
90	0.00993074073724446\\
91	0.00993073634416636\\
92	0.00993073187192206\\
93	0.0099307273190904\\
94	0.00993072268422486\\
95	0.00993071796585309\\
96	0.00993071316247644\\
97	0.00993070827256952\\
98	0.00993070329457969\\
99	0.00993069822692661\\
100	0.00993069306800171\\
101	0.00993068781616775\\
102	0.00993068246975825\\
103	0.00993067702707699\\
104	0.0099306714863975\\
105	0.0099306658459625\\
106	0.00993066010398338\\
107	0.00993065425863961\\
108	0.00993064830807821\\
109	0.00993064225041314\\
110	0.00993063608372475\\
111	0.00993062980605913\\
112	0.00993062341542758\\
113	0.00993061690980591\\
114	0.00993061028713389\\
115	0.00993060354531454\\
116	0.00993059668221352\\
117	0.00993058969565846\\
118	0.00993058258343829\\
119	0.00993057534330254\\
120	0.00993056797296064\\
121	0.00993056047008123\\
122	0.00993055283229143\\
123	0.00993054505717609\\
124	0.00993053714227705\\
125	0.0099305290850924\\
126	0.00993052088307564\\
127	0.00993051253363497\\
128	0.00993050403413244\\
129	0.00993049538188314\\
130	0.00993048657415436\\
131	0.00993047760816477\\
132	0.00993046848108353\\
133	0.00993045919002943\\
134	0.00993044973206998\\
135	0.00993044010422053\\
136	0.00993043030344332\\
137	0.00993042032664654\\
138	0.00993041017068338\\
139	0.00993039983235107\\
140	0.00993038930838985\\
141	0.00993037859548201\\
142	0.00993036769025081\\
143	0.00993035658925947\\
144	0.00993034528901009\\
145	0.00993033378594258\\
146	0.00993032207643354\\
147	0.00993031015679516\\
148	0.00993029802327405\\
149	0.0099302856720501\\
150	0.00993027309923528\\
151	0.00993026030087247\\
152	0.00993024727293417\\
153	0.00993023401132132\\
154	0.00993022051186199\\
155	0.00993020677031011\\
156	0.00993019278234413\\
157	0.0099301785435657\\
158	0.0099301640494983\\
159	0.00993014929558586\\
160	0.00993013427719133\\
161	0.00993011898959527\\
162	0.00993010342799436\\
163	0.00993008758749992\\
164	0.0099300714631364\\
165	0.00993005504983984\\
166	0.00993003834245628\\
167	0.00993002133574018\\
168	0.00993000402435278\\
169	0.00992998640286044\\
170	0.00992996846573298\\
171	0.00992995020734192\\
172	0.00992993162195879\\
173	0.00992991270375326\\
174	0.00992989344679144\\
175	0.00992987384503395\\
176	0.00992985389233406\\
177	0.00992983358243582\\
178	0.00992981290897206\\
179	0.00992979186546243\\
180	0.00992977044531139\\
181	0.00992974864180616\\
182	0.0099297264481146\\
183	0.0099297038572831\\
184	0.00992968086223442\\
185	0.00992965745576546\\
186	0.00992963363054504\\
187	0.00992960937911158\\
188	0.0099295846938708\\
189	0.00992955956709332\\
190	0.00992953399091226\\
191	0.00992950795732077\\
192	0.00992948145816955\\
193	0.00992945448516424\\
194	0.00992942702986291\\
195	0.00992939908367331\\
196	0.00992937063785027\\
197	0.00992934168349288\\
198	0.00992931221154176\\
199	0.00992928221277616\\
200	0.00992925167781109\\
201	0.00992922059709437\\
202	0.00992918896090359\\
203	0.00992915675934309\\
204	0.00992912398234081\\
205	0.00992909061964512\\
206	0.00992905666082161\\
207	0.00992902209524973\\
208	0.00992898691211951\\
209	0.00992895110042806\\
210	0.00992891464897615\\
211	0.00992887754636463\\
212	0.00992883978099082\\
213	0.00992880134104482\\
214	0.00992876221450577\\
215	0.00992872238913799\\
216	0.00992868185248715\\
217	0.00992864059187624\\
218	0.00992859859440157\\
219	0.00992855584692861\\
220	0.00992851233608782\\
221	0.00992846804827039\\
222	0.00992842296962383\\
223	0.00992837708604758\\
224	0.00992833038318847\\
225	0.0099282828464361\\
226	0.00992823446091814\\
227	0.00992818521149559\\
228	0.00992813508275784\\
229	0.00992808405901774\\
230	0.00992803212430653\\
231	0.00992797926236865\\
232	0.00992792545665653\\
233	0.00992787069032519\\
234	0.0099278149462268\\
235	0.0099277582069051\\
236	0.00992770045458973\\
237	0.00992764167119046\\
238	0.00992758183829127\\
239	0.00992752093714438\\
240	0.0099274589486641\\
241	0.00992739585342058\\
242	0.00992733163163352\\
243	0.00992726626316561\\
244	0.00992719972751594\\
245	0.00992713200381332\\
246	0.00992706307080935\\
247	0.00992699290687146\\
248	0.00992692148997578\\
249	0.00992684879769986\\
250	0.00992677480721526\\
251	0.00992669949528001\\
252	0.0099266228382309\\
253	0.00992654481197564\\
254	0.00992646539198487\\
255	0.00992638455328397\\
256	0.00992630227044479\\
257	0.00992621851757716\\
258	0.00992613326832022\\
259	0.00992604649583368\\
260	0.00992595817278878\\
261	0.00992586827135918\\
262	0.0099257767632116\\
263	0.00992568361949633\\
264	0.00992558881083752\\
265	0.0099254923073233\\
266	0.0099253940784957\\
267	0.0099252940933403\\
268	0.00992519232027576\\
269	0.00992508872714296\\
270	0.00992498328119394\\
271	0.00992487594908067\\
272	0.00992476669684414\\
273	0.00992465548990483\\
274	0.00992454229305393\\
275	0.00992442707043759\\
276	0.00992430978553932\\
277	0.00992419040117137\\
278	0.0099240688794629\\
279	0.00992394518184714\\
280	0.00992381926904825\\
281	0.00992369110106802\\
282	0.00992356063717229\\
283	0.00992342783587714\\
284	0.00992329265493487\\
285	0.00992315505131967\\
286	0.00992301498121313\\
287	0.00992287239998948\\
288	0.00992272726220057\\
289	0.00992257952156063\\
290	0.00992242913093084\\
291	0.00992227604230362\\
292	0.00992212020678668\\
293	0.00992196157458697\\
294	0.00992180009499428\\
295	0.00992163571636469\\
296	0.00992146838610395\\
297	0.00992129805065047\\
298	0.00992112465545834\\
299	0.00992094814498012\\
300	0.00992076846264954\\
301	0.00992058555086411\\
302	0.00992039935096766\\
303	0.00992020980323288\\
304	0.00992001684684393\\
305	0.00991982041987912\\
306	0.00991962045929402\\
307	0.00991941690090489\\
308	0.00991920967937278\\
309	0.00991899872818766\\
310	0.00991878397965054\\
311	0.00991856536484889\\
312	0.00991834281362408\\
313	0.0099181162545567\\
314	0.00991788561501206\\
315	0.00991765082109772\\
316	0.00991741179764503\\
317	0.0099171684681984\\
318	0.00991692075500623\\
319	0.00991666857901346\\
320	0.00991641185985619\\
321	0.0099161505158586\\
322	0.00991588446403243\\
323	0.00991561362007958\\
324	0.00991533789839802\\
325	0.00991505721209155\\
326	0.009914771472984\\
327	0.0099144805916381\\
328	0.00991418447737987\\
329	0.00991388303832891\\
330	0.00991357618143524\\
331	0.00991326381252325\\
332	0.00991294583634334\\
333	0.00991262215663096\\
334	0.00991229267617106\\
335	0.00991195729685783\\
336	0.00991161591971723\\
337	0.00991126844480942\\
338	0.00991091477091414\\
339	0.00991055479563033\\
340	0.00991018842127026\\
341	0.00990981555678587\\
342	0.00990943610069714\\
343	0.00990904994924596\\
344	0.00990865699838769\\
345	0.00990825714382186\\
346	0.00990785028089197\\
347	0.00990743630395627\\
348	0.00990701510394328\\
349	0.00990658656299084\\
350	0.00990615059228333\\
351	0.00990570710632888\\
352	0.00990525597935747\\
353	0.00990479707769154\\
354	0.00990433026527094\\
355	0.00990385540359984\\
356	0.00990337235174144\\
357	0.00990288096645583\\
358	0.0099023811023399\\
359	0.00990187261105522\\
360	0.00990135534156717\\
361	0.00990082914015778\\
362	0.00990029385036582\\
363	0.00989974931292212\\
364	0.00989919536567946\\
365	0.00989863184353588\\
366	0.00989805857835061\\
367	0.00989747539885124\\
368	0.00989688213053113\\
369	0.00989627859553574\\
370	0.00989566461253761\\
371	0.0098950399966002\\
372	0.00989440455903375\\
373	0.00989375810724723\\
374	0.00989310044459479\\
375	0.0098924313701728\\
376	0.00989175067839197\\
377	0.00989105815796495\\
378	0.00989035359062928\\
379	0.00988963675395991\\
380	0.00988890741967059\\
381	0.00988816535283248\\
382	0.00988741031139915\\
383	0.00988664204585051\\
384	0.00988586029923622\\
385	0.00988506480802466\\
386	0.00988425530285592\\
387	0.00988343149816776\\
388	0.00988259302418796\\
389	0.00988173954720821\\
390	0.00988087074932679\\
391	0.00987998630199731\\
392	0.00987908586177459\\
393	0.00987816906908797\\
394	0.0098772355469642\\
395	0.0098762848997191\\
396	0.00987531671164377\\
397	0.00987433054571894\\
398	0.00987332594240293\\
399	0.00987230241856791\\
400	0.0098712594667651\\
401	0.00987019655538514\\
402	0.0098691131315321\\
403	0.00986800863167006\\
404	0.0098668825074114\\
405	0.00986573420438174\\
406	0.00986456268705081\\
407	0.00986336742889049\\
408	0.00986214796778733\\
409	0.00986090383596427\\
410	0.00985963455651705\\
411	0.00985833962749916\\
412	0.00985701855127223\\
413	0.00985567125008841\\
414	0.00985429741320922\\
415	0.00985289638097873\\
416	0.0098514674683948\\
417	0.00985000996371651\\
418	0.00984852312713555\\
419	0.00984700618967824\\
420	0.00984545835245872\\
421	0.00984387878557286\\
422	0.00984226662243417\\
423	0.00984062093994073\\
424	0.00983894075034871\\
425	0.00983722504644191\\
426	0.00983547277651676\\
427	0.00983368283629492\\
428	0.00983185406482189\\
429	0.0098299852399367\\
430	0.00982807507325984\\
431	0.00982612220464185\\
432	0.00982412519601059\\
433	0.00982208252454507\\
434	0.00981999257505293\\
435	0.00981785363117089\\
436	0.00981566386388784\\
437	0.00981342131164205\\
438	0.00981112383340522\\
439	0.00980876901406392\\
440	0.00980635468891059\\
441	0.00980387887236951\\
442	0.00980133881930461\\
443	0.00979873145935177\\
444	0.00979605343638203\\
445	0.0097933010641487\\
446	0.00979047026815278\\
447	0.00978755650962274\\
448	0.00978455472915414\\
449	0.00978145956802953\\
450	0.00977826493401206\\
451	0.00977496387266974\\
452	0.0097715485274978\\
453	0.00976801002983891\\
454	0.00976433841792072\\
455	0.00976052260274695\\
456	0.00975654908188087\\
457	0.00975187337510829\\
458	0.00971722508156654\\
459	0.00968202931715126\\
460	0.00964609828519901\\
461	0.00960935424366499\\
462	0.00957177471230639\\
463	0.00953333699360045\\
464	0.00949401797492805\\
465	0.00945379720402719\\
466	0.00941264977884392\\
467	0.0093705515138918\\
468	0.00932747989843\\
469	0.00928341523661596\\
470	0.00923834220907166\\
471	0.00919225152495966\\
472	0.00914514529375721\\
473	0.00909702209437552\\
474	0.00904784884878742\\
475	0.00899758273021712\\
476	0.00894617829931817\\
477	0.00889358731911783\\
478	0.0088397592487368\\
479	0.00878463786755986\\
480	0.00872817097275927\\
481	0.00867031628596089\\
482	0.0086110286693143\\
483	0.00855026144771242\\
484	0.00848797038646977\\
485	0.0084241215869838\\
486	0.00835933739897901\\
487	0.00829516852057658\\
488	0.00822936249567639\\
489	0.0081618265298163\\
490	0.00809245455317616\\
491	0.00802114020293189\\
492	0.00794776691462567\\
493	0.00787220486253284\\
494	0.00779430110599939\\
495	0.00771388558172433\\
496	0.00763076347031366\\
497	0.00754470957221899\\
498	0.00745546162674872\\
499	0.00736270831820382\\
500	0.00726607771570519\\
501	0.00716519142095032\\
502	0.00705961055299592\\
503	0.00694882431976277\\
504	0.00683223712188952\\
505	0.00670915556106427\\
506	0.00657876271387543\\
507	0.0064443686339057\\
508	0.00630665936193442\\
509	0.0062021269716531\\
510	0.00613081217928649\\
511	0.00605834931285408\\
512	0.0059847585457202\\
513	0.00591006788827003\\
514	0.00583431268416269\\
515	0.00575753028797165\\
516	0.00567973480983292\\
517	0.00560081025772098\\
518	0.00552075419464518\\
519	0.0054396348244347\\
520	0.00535754050306231\\
521	0.00527458362688283\\
522	0.00519090733902166\\
523	0.00510669232669511\\
524	0.00502216522608326\\
525	0.0049376084664734\\
526	0.00485337485200936\\
527	0.00476988823244741\\
528	0.0046876916705995\\
529	0.00460746011895404\\
530	0.00453002778502993\\
531	0.00445642443236886\\
532	0.00438606543318127\\
533	0.00431411991300258\\
534	0.0042405441121292\\
535	0.00416536101625191\\
536	0.00408859738768097\\
537	0.00401031416093105\\
538	0.00393061848213843\\
539	0.00384967953737994\\
540	0.00376774944707301\\
541	0.00368515450614076\\
542	0.00360059345075749\\
543	0.00351426419829046\\
544	0.00342647862209044\\
545	0.00333769703700908\\
546	0.00324979024912032\\
547	0.00316349059199877\\
548	0.00307903376118264\\
549	0.00299664663417148\\
550	0.00291653241851946\\
551	0.00283884930907833\\
552	0.00276368032716527\\
553	0.0026909913567225\\
554	0.00262089551155703\\
555	0.00255348050617069\\
556	0.00248871242420521\\
557	0.00242593103386591\\
558	0.00236349281322508\\
559	0.00230141454744795\\
560	0.00223970032677368\\
561	0.00217835547665538\\
562	0.00211736633243854\\
563	0.00205669315196962\\
564	0.00199626632061458\\
565	0.00193598367096274\\
566	0.00187571269089403\\
567	0.0018152830983602\\
568	0.00175454130039174\\
569	0.00169338982056584\\
570	0.00163182870759884\\
571	0.0015698585537607\\
572	0.00150748492972648\\
573	0.0014447228627418\\
574	0.00138158723222657\\
575	0.00131809357034907\\
576	0.00125425925685432\\
577	0.00119010506498894\\
578	0.00112565699842713\\
579	0.00106094824693509\\
580	0.000996021033245017\\
581	0.000930927885436012\\
582	0.000865723636385916\\
583	0.000800463003868688\\
584	0.000735200717684098\\
585	0.000669991619104265\\
586	0.000604865722874402\\
587	0.000539844839955322\\
588	0.000474935038415231\\
589	0.000410184885592955\\
590	0.000347278553854741\\
591	0.000287502621334158\\
592	0.000231419174092323\\
593	0.000179467143924426\\
594	0.000132079920923583\\
595	9.10093979249611e-05\\
596	5.42660945238511e-05\\
597	2.28062284332058e-05\\
598	2.9204464504877e-07\\
599	0\\
600	0\\
};
\addplot [color=mycolor18,solid,forget plot]
  table[row sep=crcr]{%
1	0.00935388246113559\\
2	0.00935387188018864\\
3	0.00935386110766325\\
4	0.00935385014009582\\
5	0.0093538389739603\\
6	0.00935382760566703\\
7	0.00935381603156166\\
8	0.00935380424792389\\
9	0.00935379225096639\\
10	0.00935378003683355\\
11	0.0093537676016002\\
12	0.00935375494127047\\
13	0.00935374205177643\\
14	0.00935372892897686\\
15	0.00935371556865589\\
16	0.00935370196652169\\
17	0.00935368811820508\\
18	0.00935367401925817\\
19	0.00935365966515292\\
20	0.00935364505127973\\
21	0.00935363017294598\\
22	0.00935361502537449\\
23	0.00935359960370209\\
24	0.00935358390297799\\
25	0.00935356791816226\\
26	0.00935355164412422\\
27	0.00935353507564083\\
28	0.009353518207395\\
29	0.00935350103397394\\
30	0.00935348354986742\\
31	0.00935346574946604\\
32	0.00935344762705944\\
33	0.00935342917683452\\
34	0.00935341039287356\\
35	0.00935339126915239\\
36	0.00935337179953844\\
37	0.00935335197778883\\
38	0.00935333179754839\\
39	0.00935331125234764\\
40	0.00935329033560077\\
41	0.00935326904060352\\
42	0.00935324736053109\\
43	0.00935322528843596\\
44	0.00935320281724575\\
45	0.00935317993976088\\
46	0.00935315664865243\\
47	0.00935313293645972\\
48	0.00935310879558799\\
49	0.00935308421830602\\
50	0.00935305919674368\\
51	0.00935303372288943\\
52	0.00935300778858781\\
53	0.00935298138553686\\
54	0.00935295450528552\\
55	0.00935292713923094\\
56	0.00935289927861579\\
57	0.00935287091452547\\
58	0.00935284203788535\\
59	0.00935281263945787\\
60	0.00935278270983964\\
61	0.00935275223945848\\
62	0.00935272121857043\\
63	0.00935268963725665\\
64	0.00935265748542032\\
65	0.00935262475278343\\
66	0.00935259142888359\\
67	0.00935255750307072\\
68	0.00935252296450371\\
69	0.00935248780214698\\
70	0.00935245200476705\\
71	0.00935241556092898\\
72	0.00935237845899281\\
73	0.00935234068710986\\
74	0.00935230223321906\\
75	0.0093522630850431\\
76	0.00935222323008464\\
77	0.00935218265562232\\
78	0.00935214134870684\\
79	0.00935209929615682\\
80	0.00935205648455474\\
81	0.00935201290024265\\
82	0.00935196852931797\\
83	0.00935192335762906\\
84	0.00935187737077081\\
85	0.00935183055408016\\
86	0.00935178289263141\\
87	0.00935173437123165\\
88	0.00935168497441593\\
89	0.00935163468644244\\
90	0.00935158349128757\\
91	0.00935153137264089\\
92	0.00935147831390006\\
93	0.00935142429816562\\
94	0.00935136930823569\\
95	0.00935131332660059\\
96	0.00935125633543739\\
97	0.00935119831660427\\
98	0.00935113925163492\\
99	0.0093510791217327\\
100	0.00935101790776482\\
101	0.00935095559025627\\
102	0.00935089214938385\\
103	0.00935082756496984\\
104	0.0093507618164758\\
105	0.00935069488299613\\
106	0.00935062674325145\\
107	0.00935055737558208\\
108	0.00935048675794118\\
109	0.0093504148678879\\
110	0.00935034168258037\\
111	0.00935026717876854\\
112	0.00935019133278695\\
113	0.00935011412054733\\
114	0.00935003551753107\\
115	0.0093499554987816\\
116	0.00934987403889654\\
117	0.00934979111201982\\
118	0.00934970669183359\\
119	0.00934962075155001\\
120	0.0093495332639029\\
121	0.00934944420113922\\
122	0.00934935353501042\\
123	0.00934926123676361\\
124	0.00934916727713261\\
125	0.00934907162632883\\
126	0.00934897425403195\\
127	0.00934887512938047\\
128	0.00934877422096211\\
129	0.00934867149680399\\
130	0.00934856692436265\\
131	0.00934846047051393\\
132	0.00934835210154262\\
133	0.00934824178313197\\
134	0.00934812948035296\\
135	0.00934801515765343\\
136	0.00934789877884699\\
137	0.00934778030710172\\
138	0.00934765970492871\\
139	0.00934753693417037\\
140	0.00934741195598852\\
141	0.00934728473085229\\
142	0.00934715521852576\\
143	0.00934702337805546\\
144	0.00934688916775761\\
145	0.00934675254520502\\
146	0.00934661346721399\\
147	0.00934647188983075\\
148	0.00934632776831779\\
149	0.00934618105713989\\
150	0.00934603170994993\\
151	0.00934587967957439\\
152	0.00934572491799871\\
153	0.00934556737635221\\
154	0.00934540700489291\\
155	0.00934524375299197\\
156	0.0093450775691179\\
157	0.00934490840082048\\
158	0.00934473619471431\\
159	0.00934456089646227\\
160	0.00934438245075841\\
161	0.00934420080131076\\
162	0.00934401589082371\\
163	0.00934382766098008\\
164	0.00934363605242294\\
165	0.00934344100473701\\
166	0.00934324245642976\\
167	0.00934304034491221\\
168	0.0093428346064793\\
169	0.00934262517628997\\
170	0.0093424119883469\\
171	0.00934219497547575\\
172	0.00934197406930419\\
173	0.00934174920024043\\
174	0.00934152029745143\\
175	0.00934128728884067\\
176	0.00934105010102554\\
177	0.0093408086593143\\
178	0.00934056288768263\\
179	0.00934031270874978\\
180	0.0093400580437542\\
181	0.00933979881252881\\
182	0.0093395349334758\\
183	0.00933926632354093\\
184	0.00933899289818741\\
185	0.00933871457136921\\
186	0.00933843125550403\\
187	0.00933814286144563\\
188	0.0093378492984557\\
189	0.00933755047417525\\
190	0.00933724629459542\\
191	0.00933693666402776\\
192	0.00933662148507399\\
193	0.00933630065859514\\
194	0.0093359740836802\\
195	0.00933564165761412\\
196	0.00933530327584523\\
197	0.00933495883195209\\
198	0.00933460821760966\\
199	0.00933425132255488\\
200	0.00933388803455162\\
201	0.00933351823935491\\
202	0.00933314182067458\\
203	0.00933275866013817\\
204	0.00933236863725313\\
205	0.00933197162936836\\
206	0.00933156751163497\\
207	0.00933115615696634\\
208	0.00933073743599738\\
209	0.00933031121704311\\
210	0.00932987736605633\\
211	0.00932943574658456\\
212	0.00932898621972619\\
213	0.0093285286440857\\
214	0.00932806287572813\\
215	0.00932758876813258\\
216	0.00932710617214492\\
217	0.00932661493592952\\
218	0.00932611490492008\\
219	0.00932560592176952\\
220	0.0093250878262989\\
221	0.00932456045544534\\
222	0.00932402364320896\\
223	0.0093234772205988\\
224	0.00932292101557764\\
225	0.00932235485300581\\
226	0.00932177855458387\\
227	0.0093211919387942\\
228	0.00932059482084144\\
229	0.00931998701259174\\
230	0.00931936832251089\\
231	0.00931873855560115\\
232	0.00931809751333696\\
233	0.00931744499359923\\
234	0.00931678079060848\\
235	0.00931610469485662\\
236	0.00931541649303729\\
237	0.00931471596797501\\
238	0.00931400289855272\\
239	0.00931327705963808\\
240	0.00931253822200812\\
241	0.00931178615227254\\
242	0.00931102061279544\\
243	0.00931024136161539\\
244	0.00930944815236406\\
245	0.00930864073418312\\
246	0.00930781885163949\\
247	0.0093069822446389\\
248	0.0093061306483377\\
249	0.0093052637930529\\
250	0.00930438140417041\\
251	0.0093034832020513\\
252	0.00930256890193639\\
253	0.00930163821384867\\
254	0.00930069084249383\\
255	0.0092997264871588\\
256	0.00929874484160814\\
257	0.00929774559397829\\
258	0.00929672842666973\\
259	0.00929569301623684\\
260	0.00929463903327553\\
261	0.00929356614230848\\
262	0.00929247400166804\\
263	0.00929136226337673\\
264	0.00929023057302517\\
265	0.00928907856964746\\
266	0.00928790588559415\\
267	0.00928671214640243\\
268	0.00928549697066392\\
269	0.00928425996989008\\
270	0.00928300074837557\\
271	0.00928171890306\\
272	0.00928041402338833\\
273	0.00927908569116927\\
274	0.00927773348042881\\
275	0.00927635695725176\\
276	0.0092749556795893\\
277	0.00927352919708048\\
278	0.00927207705092298\\
279	0.00927059877371125\\
280	0.00926909388926583\\
281	0.00926756191245897\\
282	0.00926600234903667\\
283	0.00926441469543684\\
284	0.00926279843860374\\
285	0.00926115305579856\\
286	0.00925947801440592\\
287	0.00925777277173657\\
288	0.00925603677482595\\
289	0.00925426946022866\\
290	0.0092524702538089\\
291	0.00925063857052676\\
292	0.00924877381422034\\
293	0.00924687537738379\\
294	0.00924494264094119\\
295	0.00924297497401639\\
296	0.00924097173369885\\
297	0.00923893226480545\\
298	0.00923685589963861\\
299	0.00923474195774062\\
300	0.00923258974564453\\
301	0.00923039855662175\\
302	0.00922816767042654\\
303	0.00922589635303781\\
304	0.0092235838563983\\
305	0.00922122941815141\\
306	0.00921883226137546\\
307	0.00921639159431492\\
308	0.00921390661010708\\
309	0.00921137648650256\\
310	0.0092088003855786\\
311	0.00920617745344888\\
312	0.00920350681998642\\
313	0.00920078759859547\\
314	0.00919801888610708\\
315	0.00919519976283647\\
316	0.00919232929209615\\
317	0.00918940651995461\\
318	0.00918643047506029\\
319	0.00918340016848936\\
320	0.00918031459362137\\
321	0.00917717272604791\\
322	0.00917397352351965\\
323	0.00917071592593803\\
324	0.00916739885539842\\
325	0.00916402121629215\\
326	0.00916058189547558\\
327	0.00915707976251482\\
328	0.00915351367001522\\
329	0.00914988245404487\\
330	0.00914618493466163\\
331	0.00914241991655318\\
332	0.00913858618980006\\
333	0.00913468253077488\\
334	0.00913070770319948\\
335	0.0091266604594075\\
336	0.00912253954191899\\
337	0.00911834368555332\\
338	0.00911407162045904\\
339	0.00910972207626186\\
340	0.00910529378589186\\
341	0.00910078549893887\\
342	0.00909619595261919\\
343	0.00909152377963481\\
344	0.00908676758765\\
345	0.00908192597161524\\
346	0.00907699751383327\\
347	0.00907198078462663\\
348	0.00906687434498613\\
349	0.00906167675267459\\
350	0.00905638655439896\\
351	0.00905100241924003\\
352	0.00904552286378997\\
353	0.00903994608758106\\
354	0.00903427025068874\\
355	0.00902849347266619\\
356	0.0090226138315067\\
357	0.00901662936258298\\
358	0.00901053805740051\\
359	0.00900433786148541\\
360	0.00899802666873709\\
361	0.00899160232238428\\
362	0.00898506261350247\\
363	0.0089784052788706\\
364	0.00897162799864498\\
365	0.008964728393827\\
366	0.00895770402349778\\
367	0.00895055238178909\\
368	0.00894327089455504\\
369	0.00893585691570293\\
370	0.00892830772313262\\
371	0.00892062051422059\\
372	0.00891279240076288\\
373	0.00890482040325693\\
374	0.00889670144435709\\
375	0.00888843234130537\\
376	0.00888000979720011\\
377	0.00887143039126845\\
378	0.00886269056886668\\
379	0.00885378663281679\\
380	0.00884471474983117\\
381	0.0088354709255763\\
382	0.00882605098805673\\
383	0.00881645057115677\\
384	0.00880666509483924\\
385	0.00879668973995517\\
386	0.00878651941543478\\
387	0.00877614872036761\\
388	0.00876557191870218\\
389	0.00875478279931898\\
390	0.00874377548020158\\
391	0.00873254406419085\\
392	0.00872108240461992\\
393	0.00870938406743709\\
394	0.00869744231546698\\
395	0.00868525009351515\\
396	0.00867280001506848\\
397	0.00866008435157889\\
398	0.0086470950255808\\
399	0.00863382360912698\\
400	0.00862026132908542\\
401	0.00860639908035345\\
402	0.00859222744626044\\
403	0.00857773672126252\\
404	0.00856291692610347\\
405	0.00854775783126497\\
406	0.00853224903854124\\
407	0.00851637874887452\\
408	0.00850014017392674\\
409	0.00848352729068936\\
410	0.00846653532930649\\
411	0.00844916134249677\\
412	0.00843140494628865\\
413	0.0084132692340222\\
414	0.00839476293802955\\
415	0.00837589427803379\\
416	0.00835665512812732\\
417	0.00833703712372604\\
418	0.00831703164985451\\
419	0.00829662982675444\\
420	0.00827582249080892\\
421	0.00825460016794972\\
422	0.0082329530375656\\
423	0.00821087088992473\\
424	0.00818834308085885\\
425	0.0081653586062401\\
426	0.00814190635765207\\
427	0.00811797489406353\\
428	0.00809355238571992\\
429	0.00806862659615527\\
430	0.00804318486362214\\
431	0.00801721408211817\\
432	0.00799070068231922\\
433	0.00796363061299112\\
434	0.0079359893240296\\
435	0.00790776175364678\\
436	0.00787893232556888\\
437	0.00784948497019549\\
438	0.00781940320103334\\
439	0.00778867029193811\\
440	0.00775726930641245\\
441	0.00772518524823265\\
442	0.0076924019746264\\
443	0.00765889704039598\\
444	0.00762464640987071\\
445	0.00758962423797925\\
446	0.0075538026267833\\
447	0.00751715137017973\\
448	0.00747963769494051\\
449	0.00744122593783304\\
450	0.0074018779348395\\
451	0.00736154953242777\\
452	0.00732018995426554\\
453	0.00727774139820778\\
454	0.00723413764358201\\
455	0.0071893023992448\\
456	0.00714314881497628\\
457	0.00709611890625548\\
458	0.00707814715998119\\
459	0.00705950064542732\\
460	0.007040111390837\\
461	0.00701990586931755\\
462	0.006998799285829\\
463	0.00697669296911973\\
464	0.0069534716361377\\
465	0.00692900000368419\\
466	0.00690311877158677\\
467	0.00687563960169342\\
468	0.00684633893625431\\
469	0.00681495038705406\\
470	0.00678115540694164\\
471	0.00674457210883074\\
472	0.00670474309112749\\
473	0.00666369476105502\\
474	0.00662197457892596\\
475	0.00657957827687717\\
476	0.00653650234801006\\
477	0.00649274409124263\\
478	0.00644830169130089\\
479	0.00640317456720802\\
480	0.00635736400836668\\
481	0.0063108771065416\\
482	0.00626375221462998\\
483	0.00621601161985001\\
484	0.00616766531824011\\
485	0.00611872205674312\\
486	0.00606917904683886\\
487	0.00601901680727206\\
488	0.0059682650676501\\
489	0.00591694912971216\\
490	0.00586505640756299\\
491	0.00581251825469725\\
492	0.00575937889418669\\
493	0.00570569420173042\\
494	0.00565153429849766\\
495	0.00559698631085357\\
496	0.00554215845861785\\
497	0.00548718488524454\\
498	0.00543223164892994\\
499	0.00537750445415099\\
500	0.0053232586911779\\
501	0.00526980852182763\\
502	0.00521754039160833\\
503	0.00516692999283385\\
504	0.00511856338139452\\
505	0.00507316311871932\\
506	0.0050316199649441\\
507	0.0049905546194071\\
508	0.00494928986390594\\
509	0.00490747668529065\\
510	0.00486469649171915\\
511	0.00482092246210445\\
512	0.00477612639420774\\
513	0.00473027858039279\\
514	0.00468334734606081\\
515	0.00463529756612277\\
516	0.00458609874967004\\
517	0.00453572889551623\\
518	0.0044841677197608\\
519	0.00443139563389439\\
520	0.00437739300423209\\
521	0.00432213901410709\\
522	0.00426560983196854\\
523	0.00420777542104398\\
524	0.00414859175016062\\
525	0.00408799773907985\\
526	0.00402593207117221\\
527	0.00396233451736068\\
528	0.00389714705872905\\
529	0.00383031591407272\\
530	0.00376179331226578\\
531	0.00369153993728676\\
532	0.00361959527846768\\
533	0.0035462583284732\\
534	0.00347208389533472\\
535	0.00339908980553376\\
536	0.00332748166785388\\
537	0.00325747574193593\\
538	0.00318929462662516\\
539	0.00312316045408918\\
540	0.00305928458020588\\
541	0.00299785326245797\\
542	0.00293903789899305\\
543	0.00288294885860575\\
544	0.00282960002502755\\
545	0.00277886000119549\\
546	0.00272909952961711\\
547	0.00267981308881545\\
548	0.0026309828134207\\
549	0.00258257253401098\\
550	0.00253452574506337\\
551	0.00248676442562614\\
552	0.00243918939882931\\
553	0.00239168320911\\
554	0.00234410597758729\\
555	0.00229629841183345\\
556	0.0022480992879957\\
557	0.0021993849561593\\
558	0.00215013390444389\\
559	0.0021003217604376\\
560	0.0020499215771393\\
561	0.00199890396187714\\
562	0.00194723786732274\\
563	0.00189489179927033\\
564	0.00184183539852713\\
565	0.00178804137997534\\
566	0.00173348768596366\\
567	0.00167815996825155\\
568	0.00162205285376998\\
569	0.00156516806544259\\
570	0.00150750949614769\\
571	0.00144908358804972\\
572	0.00138989961244892\\
573	0.00132996979794916\\
574	0.00126930973751084\\
575	0.00120793876053359\\
576	0.00114588022454223\\
577	0.00108316193301972\\
578	0.00101981685596092\\
579	0.000955885432697557\\
580	0.000891416745494425\\
581	0.000826449428908736\\
582	0.00076103325331982\\
583	0.000695249245061119\\
584	0.000629112420938282\\
585	0.000564944790052522\\
586	0.000503134290179678\\
587	0.000444772794001453\\
588	0.000390227629686837\\
589	0.000340264159084332\\
590	0.000293674396315564\\
591	0.00024937252729716\\
592	0.000206953440128856\\
593	0.000166169107244796\\
594	0.000127115113967414\\
595	8.96111722547531e-05\\
596	5.42660945238511e-05\\
597	2.28062284332059e-05\\
598	2.9204464504877e-07\\
599	0\\
600	0\\
};
\addplot [color=red!25!mycolor17,solid,forget plot]
  table[row sep=crcr]{%
1	0.00752975719577589\\
2	0.00752975205343657\\
3	0.00752974681777696\\
4	0.00752974148710355\\
5	0.00752973605969214\\
6	0.00752973053378722\\
7	0.00752972490760141\\
8	0.00752971917931491\\
9	0.00752971334707491\\
10	0.00752970740899495\\
11	0.00752970136315436\\
12	0.00752969520759761\\
13	0.00752968894033371\\
14	0.00752968255933553\\
15	0.00752967606253913\\
16	0.00752966944784316\\
17	0.00752966271310816\\
18	0.00752965585615583\\
19	0.00752964887476838\\
20	0.00752964176668774\\
21	0.00752963452961493\\
22	0.00752962716120926\\
23	0.00752961965908754\\
24	0.00752961202082343\\
25	0.00752960424394652\\
26	0.00752959632594162\\
27	0.00752958826424795\\
28	0.00752958005625826\\
29	0.00752957169931803\\
30	0.00752956319072461\\
31	0.00752955452772636\\
32	0.00752954570752172\\
33	0.00752953672725836\\
34	0.00752952758403223\\
35	0.00752951827488665\\
36	0.00752950879681133\\
37	0.00752949914674141\\
38	0.0075294893215565\\
39	0.00752947931807964\\
40	0.00752946913307629\\
41	0.00752945876325331\\
42	0.00752944820525786\\
43	0.00752943745567639\\
44	0.00752942651103344\\
45	0.00752941536779064\\
46	0.00752940402234548\\
47	0.0075293924710302\\
48	0.00752938071011062\\
49	0.0075293687357849\\
50	0.00752935654418234\\
51	0.00752934413136215\\
52	0.00752933149331217\\
53	0.0075293186259476\\
54	0.00752930552510964\\
55	0.00752929218656425\\
56	0.00752927860600069\\
57	0.00752926477903021\\
58	0.00752925070118459\\
59	0.00752923636791476\\
60	0.0075292217745893\\
61	0.007529206916493\\
62	0.0075291917888253\\
63	0.00752917638669876\\
64	0.00752916070513756\\
65	0.00752914473907581\\
66	0.00752912848335602\\
67	0.00752911193272739\\
68	0.00752909508184412\\
69	0.00752907792526378\\
70	0.00752906045744548\\
71	0.00752904267274817\\
72	0.00752902456542876\\
73	0.00752900612964038\\
74	0.00752898735943041\\
75	0.00752896824873868\\
76	0.00752894879139544\\
77	0.00752892898111949\\
78	0.00752890881151607\\
79	0.00752888827607488\\
80	0.00752886736816801\\
81	0.00752884608104782\\
82	0.00752882440784475\\
83	0.00752880234156517\\
84	0.00752877987508915\\
85	0.00752875700116815\\
86	0.00752873371242279\\
87	0.0075287100013404\\
88	0.00752868586027273\\
89	0.00752866128143343\\
90	0.0075286362568956\\
91	0.0075286107785893\\
92	0.00752858483829895\\
93	0.00752855842766072\\
94	0.00752853153815989\\
95	0.00752850416112809\\
96	0.00752847628774064\\
97	0.00752844790901365\\
98	0.00752841901580124\\
99	0.00752838959879258\\
100	0.00752835964850895\\
101	0.00752832915530077\\
102	0.00752829810934449\\
103	0.00752826650063949\\
104	0.00752823431900494\\
105	0.00752820155407651\\
106	0.00752816819530313\\
107	0.00752813423194364\\
108	0.00752809965306338\\
109	0.00752806444753073\\
110	0.00752802860401361\\
111	0.00752799211097584\\
112	0.00752795495667353\\
113	0.00752791712915137\\
114	0.00752787861623882\\
115	0.00752783940554629\\
116	0.00752779948446119\\
117	0.00752775884014398\\
118	0.00752771745952413\\
119	0.00752767532929593\\
120	0.00752763243591434\\
121	0.0075275887655907\\
122	0.00752754430428839\\
123	0.00752749903771837\\
124	0.00752745295133474\\
125	0.00752740603033007\\
126	0.00752735825963083\\
127	0.00752730962389257\\
128	0.00752726010749509\\
129	0.00752720969453758\\
130	0.00752715836883358\\
131	0.00752710611390588\\
132	0.00752705291298136\\
133	0.0075269987489857\\
134	0.00752694360453804\\
135	0.00752688746194548\\
136	0.00752683030319757\\
137	0.00752677210996064\\
138	0.00752671286357203\\
139	0.00752665254503426\\
140	0.00752659113500905\\
141	0.0075265286138113\\
142	0.00752646496140288\\
143	0.00752640015738638\\
144	0.0075263341809987\\
145	0.0075262670111046\\
146	0.00752619862619003\\
147	0.00752612900435545\\
148	0.00752605812330895\\
149	0.00752598596035932\\
150	0.00752591249240895\\
151	0.0075258376959466\\
152	0.00752576154704007\\
153	0.00752568402132878\\
154	0.00752560509401615\\
155	0.00752552473986181\\
156	0.00752544293317384\\
157	0.00752535964780071\\
158	0.00752527485712319\\
159	0.00752518853404599\\
160	0.00752510065098941\\
161	0.00752501117988075\\
162	0.00752492009214558\\
163	0.00752482735869887\\
164	0.00752473294993598\\
165	0.00752463683572344\\
166	0.00752453898538968\\
167	0.00752443936771548\\
168	0.0075243379509243\\
169	0.00752423470267244\\
170	0.00752412959003905\\
171	0.00752402257951598\\
172	0.00752391363699737\\
173	0.00752380272776916\\
174	0.00752368981649834\\
175	0.0075235748672221\\
176	0.00752345784333668\\
177	0.00752333870758614\\
178	0.00752321742205086\\
179	0.00752309394813589\\
180	0.00752296824655904\\
181	0.00752284027733886\\
182	0.0075227099997823\\
183	0.00752257737247225\\
184	0.00752244235325482\\
185	0.0075223048992264\\
186	0.00752216496672052\\
187	0.0075220225112945\\
188	0.00752187748771579\\
189	0.00752172984994817\\
190	0.00752157955113768\\
191	0.00752142654359825\\
192	0.00752127077879721\\
193	0.00752111220734044\\
194	0.0075209507789573\\
195	0.00752078644248533\\
196	0.00752061914585465\\
197	0.00752044883607212\\
198	0.00752027545920519\\
199	0.00752009896036555\\
200	0.00751991928369243\\
201	0.00751973637233562\\
202	0.00751955016843826\\
203	0.00751936061311927\\
204	0.00751916764645548\\
205	0.00751897120746353\\
206	0.00751877123408139\\
207	0.00751856766314955\\
208	0.007518360430392\\
209	0.00751814947039674\\
210	0.00751793471659604\\
211	0.0075177161012464\\
212	0.00751749355540804\\
213	0.00751726700892417\\
214	0.00751703639039983\\
215	0.00751680162718042\\
216	0.00751656264532977\\
217	0.00751631936960797\\
218	0.00751607172344869\\
219	0.00751581962893617\\
220	0.00751556300678189\\
221	0.00751530177630066\\
222	0.00751503585538647\\
223	0.00751476516048784\\
224	0.00751448960658274\\
225	0.00751420910715318\\
226	0.00751392357415917\\
227	0.00751363291801246\\
228	0.00751333704754961\\
229	0.00751303587000479\\
230	0.007512729290982\\
231	0.00751241721442677\\
232	0.00751209954259747\\
233	0.00751177617603611\\
234	0.0075114470135386\\
235	0.00751111195212445\\
236	0.00751077088700609\\
237	0.00751042371155747\\
238	0.00751007031728229\\
239	0.00750971059378157\\
240	0.00750934442872073\\
241	0.00750897170779599\\
242	0.0075085923147003\\
243	0.00750820613108868\\
244	0.00750781303654287\\
245	0.00750741290853539\\
246	0.0075070056223931\\
247	0.00750659105125999\\
248	0.00750616906605933\\
249	0.0075057395354553\\
250	0.00750530232581381\\
251	0.0075048573011628\\
252	0.00750440432315164\\
253	0.00750394325101007\\
254	0.00750347394150628\\
255	0.00750299624890437\\
256	0.00750251002492099\\
257	0.0075020151186813\\
258	0.00750151137667419\\
259	0.00750099864270668\\
260	0.00750047675785759\\
261	0.00749994556043041\\
262	0.00749940488590532\\
263	0.00749885456689047\\
264	0.00749829443307233\\
265	0.00749772431116531\\
266	0.00749714402486035\\
267	0.00749655339477289\\
268	0.00749595223838985\\
269	0.00749534037001628\\
270	0.00749471760072198\\
271	0.00749408373829001\\
272	0.00749343858717124\\
273	0.0074927819484534\\
274	0.00749211361985863\\
275	0.00749143339577178\\
276	0.00749074106721274\\
277	0.00749003642144281\\
278	0.00748931924166022\\
279	0.00748858930723845\\
280	0.00748784639371893\\
281	0.00748709027274426\\
282	0.00748632071199015\\
283	0.00748553747509582\\
284	0.00748474032159292\\
285	0.00748392900683273\\
286	0.00748310328191177\\
287	0.0074822628935955\\
288	0.00748140758424024\\
289	0.00748053709171292\\
290	0.00747965114930884\\
291	0.007478749485667\\
292	0.00747783182468312\\
293	0.00747689788541996\\
294	0.00747594738201499\\
295	0.0074749800235849\\
296	0.00747399551412713\\
297	0.0074729935524178\\
298	0.00747197383190605\\
299	0.00747093604060444\\
300	0.00746987986097498\\
301	0.00746880496981065\\
302	0.00746771103811184\\
303	0.00746659773095754\\
304	0.0074654647073704\\
305	0.00746431162017531\\
306	0.00746313811585005\\
307	0.00746194383436596\\
308	0.00746072840901369\\
309	0.0074594914662036\\
310	0.00745823262521787\\
311	0.00745695149787069\\
312	0.00745564768800999\\
313	0.00745432079083502\\
314	0.00745297039232933\\
315	0.0074515960700461\\
316	0.00745019739601793\\
317	0.00744877393411322\\
318	0.00744732523925988\\
319	0.00744585085715755\\
320	0.00744435032397979\\
321	0.00744282316606823\\
322	0.00744126889962242\\
323	0.00743968703038992\\
324	0.00743807705336373\\
325	0.00743643845249687\\
326	0.00743477070044707\\
327	0.00743307325836974\\
328	0.00743134557578306\\
329	0.007429587090537\\
330	0.00742779722892859\\
331	0.00742597540601873\\
332	0.00742412102622348\\
333	0.00742223348427516\\
334	0.00742031216667975\\
335	0.00741835645384029\\
336	0.00741636572308849\\
337	0.00741433935301452\\
338	0.00741227672988319\\
339	0.00741017725826789\\
340	0.00740804038339811\\
341	0.00740586565793154\\
342	0.007403653038201\\
343	0.00740140271340758\\
344	0.00739911405810567\\
345	0.00739678643373589\\
346	0.00739441918769577\\
347	0.00739201165231525\\
348	0.00738956314374368\\
349	0.00738707296073435\\
350	0.00738454038327247\\
351	0.00738196466808564\\
352	0.00737934504954275\\
353	0.00737668074415684\\
354	0.0073739709498462\\
355	0.00737121484528177\\
356	0.00736841158958132\\
357	0.00736556032300077\\
358	0.00736266016990002\\
359	0.00735971024460689\\
360	0.00735670964922029\\
361	0.00735365742967529\\
362	0.00735055259845563\\
363	0.00734739413805547\\
364	0.00734418099895487\\
365	0.00734091209734935\\
366	0.0073375863125907\\
367	0.00733420248428568\\
368	0.00733075940898739\\
369	0.00732725583639804\\
370	0.00732369046498075\\
371	0.00732006193684797\\
372	0.00731636883174691\\
373	0.00731260965987817\\
374	0.00730878285311568\\
375	0.00730488675383421\\
376	0.00730091959977377\\
377	0.00729687950186456\\
378	0.00729276441000695\\
379	0.00728857206466934\\
380	0.00728429996771987\\
381	0.00727994554468052\\
382	0.00727550599739177\\
383	0.00727097823098026\\
384	0.00726635879220191\\
385	0.00726164378880268\\
386	0.00725682877613685\\
387	0.00725190857554885\\
388	0.00724687691834787\\
389	0.00724172555728899\\
390	0.00723644137563214\\
391	0.00723101554785491\\
392	0.00722543988330375\\
393	0.00721970529499155\\
394	0.00721380166796882\\
395	0.00720771770449372\\
396	0.00720144074128882\\
397	0.00719495653310726\\
398	0.00718824899550521\\
399	0.00718129989806737\\
400	0.00717408849727336\\
401	0.00716659109563358\\
402	0.00715878051056588\\
403	0.0071506254326964\\
404	0.00714208964907079\\
405	0.00713313110317246\\
406	0.00712370076604672\\
407	0.00711374132651605\\
408	0.00710318559509768\\
409	0.0070919547364818\\
410	0.00707995392375266\\
411	0.00706706986772846\\
412	0.00705316692827304\\
413	0.00703808231943389\\
414	0.00702167943378768\\
415	0.00700498846860179\\
416	0.0069880052658891\\
417	0.00697072568106828\\
418	0.00695314558986649\\
419	0.00693526089418565\\
420	0.00691706752383685\\
421	0.00689856142557651\\
422	0.00687973851619553\\
423	0.0068605945394869\\
424	0.00684112469188092\\
425	0.00682132287849355\\
426	0.00680118231290836\\
427	0.00678069868600467\\
428	0.00675986817327228\\
429	0.0067386870870183\\
430	0.00671715189842607\\
431	0.00669525926230584\\
432	0.00667300604491383\\
433	0.00665038935527493\\
434	0.00662740658050536\\
435	0.00660405542568729\\
436	0.00658033395887662\\
437	0.00655624066174944\\
438	0.0065317744859354\\
439	0.00650693491354474\\
440	0.00648172201726202\\
441	0.00645613645862427\\
442	0.00643017960258967\\
443	0.00640385381117351\\
444	0.00637716262060013\\
445	0.00635011095562376\\
446	0.00632270539431041\\
447	0.00629495451382842\\
448	0.00626686941319138\\
449	0.00623846479105154\\
450	0.00620976236471191\\
451	0.00618080754270315\\
452	0.0061516398098084\\
453	0.00612229921115789\\
454	0.00609283475388691\\
455	0.00606330621852963\\
456	0.00603378593943971\\
457	0.00600435309360236\\
458	0.00597472622139851\\
459	0.005944936642734\\
460	0.00591497869761773\\
461	0.00588488049490957\\
462	0.00585470720823154\\
463	0.00582453831237067\\
464	0.00579447066612896\\
465	0.00576462228241092\\
466	0.0057351371544715\\
467	0.00570619111318079\\
468	0.0056779990950215\\
469	0.00565082417820276\\
470	0.00562498881388385\\
471	0.00560088871720802\\
472	0.00557900960011686\\
473	0.00555729142567498\\
474	0.00553516701095325\\
475	0.00551262943556047\\
476	0.00548967173836112\\
477	0.00546628689209586\\
478	0.00544246779344238\\
479	0.00541820726369776\\
480	0.00539349805074807\\
481	0.00536833279070289\\
482	0.00534270362687262\\
483	0.00531660233968035\\
484	0.00529002037339372\\
485	0.00526294907004596\\
486	0.00523538001615325\\
487	0.00520730590458239\\
488	0.00517871950737512\\
489	0.00514961353753942\\
490	0.00511998114587108\\
491	0.00508981678447065\\
492	0.00505911483619851\\
493	0.00502786944273571\\
494	0.00499607435605367\\
495	0.00496372307028727\\
496	0.00493080396722983\\
497	0.00489730203712122\\
498	0.00486319740704639\\
499	0.00482846311294119\\
500	0.00479306325290543\\
501	0.00475696059332692\\
502	0.00472010551273883\\
503	0.00468243136231268\\
504	0.0046438481817106\\
505	0.00460423429291239\\
506	0.0045634250990333\\
507	0.00452134962894445\\
508	0.00447795628043695\\
509	0.00443319610689204\\
510	0.00438702368694223\\
511	0.00433939085082558\\
512	0.00429024426436863\\
513	0.00423951819484293\\
514	0.00418714193967163\\
515	0.00413305427633992\\
516	0.00407720885976529\\
517	0.00401958143675479\\
518	0.00396017969591317\\
519	0.0038990565281705\\
520	0.00383632709739516\\
521	0.00377219079310824\\
522	0.00370696148339691\\
523	0.00364150103709476\\
524	0.00357730663765718\\
525	0.003514572456412\\
526	0.003453502088954\\
527	0.00339430491184796\\
528	0.00333719045874468\\
529	0.0032823599463531\\
530	0.00322999434560389\\
531	0.00318023771765325\\
532	0.00313317193087718\\
533	0.00308877696155659\\
534	0.00304677637041605\\
535	0.00300533620988421\\
536	0.00296445707726527\\
537	0.00292412483708432\\
538	0.00288430801836305\\
539	0.00284495531333978\\
540	0.002805993459618\\
541	0.00276732591444782\\
542	0.00272883235327698\\
543	0.00269037068913318\\
544	0.00265178253040087\\
545	0.00261290383274353\\
546	0.00257366270674616\\
547	0.00253401370536589\\
548	0.00249390732631952\\
549	0.00245329074706882\\
550	0.00241210889436304\\
551	0.00237030589512094\\
552	0.00232782673648705\\
553	0.00228461998623048\\
554	0.00224065046301698\\
555	0.0021958884073572\\
556	0.00215031064061035\\
557	0.00210389946353566\\
558	0.00205663759187896\\
559	0.00200850842755855\\
560	0.00195949635800379\\
561	0.00190958708395927\\
562	0.00185876795497196\\
563	0.00180702829309604\\
564	0.00175435968293312\\
565	0.00170075620342841\\
566	0.00164621457913793\\
567	0.0015907342281127\\
568	0.00153431721326273\\
569	0.00147696824692171\\
570	0.00141869494997049\\
571	0.00135950808434287\\
572	0.00129942189673999\\
573	0.00123845564645323\\
574	0.00117663625677103\\
575	0.00111400030455689\\
576	0.00105059644939085\\
577	0.000986530157595525\\
578	0.000921762352719769\\
579	0.000856219106191558\\
580	0.000791664481391484\\
581	0.000729004586915961\\
582	0.000668600899830575\\
583	0.000611598515788096\\
584	0.000558751325735728\\
585	0.000508153571256534\\
586	0.000460174865235934\\
587	0.000414142193949086\\
588	0.000370019893892401\\
589	0.000327223396935425\\
590	0.000285420585329276\\
591	0.000244524545387522\\
592	0.000204480456411576\\
593	0.000165271055315995\\
594	0.000126870159937007\\
595	8.96111722547517e-05\\
596	5.42660945238508e-05\\
597	2.28062284332056e-05\\
598	2.9204464504877e-07\\
599	0\\
600	0\\
};
\addplot [color=mycolor19,solid,forget plot]
  table[row sep=crcr]{%
1	0.00695390139793097\\
2	0.00695389015754209\\
3	0.00695387871314316\\
4	0.00695386706103132\\
5	0.00695385519743642\\
6	0.00695384311851986\\
7	0.00695383082037335\\
8	0.00695381829901761\\
9	0.00695380555040107\\
10	0.00695379257039861\\
11	0.0069537793548102\\
12	0.00695376589935953\\
13	0.00695375219969261\\
14	0.00695373825137641\\
15	0.00695372404989741\\
16	0.0069537095906601\\
17	0.00695369486898552\\
18	0.00695367988010977\\
19	0.00695366461918243\\
20	0.00695364908126501\\
21	0.00695363326132935\\
22	0.00695361715425597\\
23	0.00695360075483249\\
24	0.00695358405775181\\
25	0.00695356705761055\\
26	0.0069535497489072\\
27	0.0069535321260403\\
28	0.00695351418330684\\
29	0.00695349591490013\\
30	0.00695347731490815\\
31	0.00695345837731154\\
32	0.00695343909598167\\
33	0.00695341946467864\\
34	0.0069533994770493\\
35	0.00695337912662519\\
36	0.00695335840682044\\
37	0.00695333731092963\\
38	0.00695331583212566\\
39	0.0069532939634575\\
40	0.00695327169784803\\
41	0.00695324902809163\\
42	0.00695322594685197\\
43	0.00695320244665956\\
44	0.00695317851990943\\
45	0.00695315415885856\\
46	0.0069531293556235\\
47	0.00695310410217771\\
48	0.0069530783903491\\
49	0.00695305221181729\\
50	0.006953025558111\\
51	0.00695299842060523\\
52	0.0069529707905186\\
53	0.00695294265891042\\
54	0.00695291401667788\\
55	0.00695288485455306\\
56	0.00695285516309997\\
57	0.00695282493271153\\
58	0.00695279415360646\\
59	0.00695276281582613\\
60	0.00695273090923136\\
61	0.00695269842349913\\
62	0.00695266534811933\\
63	0.00695263167239128\\
64	0.0069525973854204\\
65	0.00695256247611462\\
66	0.00695252693318083\\
67	0.00695249074512126\\
68	0.00695245390022984\\
69	0.00695241638658834\\
70	0.0069523781920626\\
71	0.00695233930429862\\
72	0.00695229971071861\\
73	0.00695225939851692\\
74	0.00695221835465596\\
75	0.00695217656586201\\
76	0.00695213401862091\\
77	0.00695209069917379\\
78	0.00695204659351266\\
79	0.00695200168737586\\
80	0.00695195596624356\\
81	0.00695190941533302\\
82	0.00695186201959392\\
83	0.00695181376370349\\
84	0.0069517646320617\\
85	0.00695171460878611\\
86	0.0069516636777069\\
87	0.00695161182236169\\
88	0.00695155902599015\\
89	0.00695150527152884\\
90	0.00695145054160554\\
91	0.00695139481853385\\
92	0.00695133808430745\\
93	0.00695128032059439\\
94	0.00695122150873117\\
95	0.00695116162971682\\
96	0.00695110066420685\\
97	0.00695103859250703\\
98	0.00695097539456708\\
99	0.00695091104997433\\
100	0.0069508455379472\\
101	0.00695077883732847\\
102	0.00695071092657865\\
103	0.00695064178376904\\
104	0.00695057138657477\\
105	0.00695049971226762\\
106	0.00695042673770889\\
107	0.00695035243934188\\
108	0.00695027679318451\\
109	0.0069501997748216\\
110	0.00695012135939712\\
111	0.0069500415216063\\
112	0.00694996023568754\\
113	0.00694987747541421\\
114	0.00694979321408632\\
115	0.00694970742452203\\
116	0.006949620079049\\
117	0.00694953114949558\\
118	0.00694944060718188\\
119	0.00694934842291061\\
120	0.00694925456695786\\
121	0.00694915900906359\\
122	0.00694906171842212\\
123	0.00694896266367224\\
124	0.00694886181288735\\
125	0.00694875913356523\\
126	0.00694865459261785\\
127	0.00694854815636073\\
128	0.00694843979050242\\
129	0.00694832946013351\\
130	0.00694821712971561\\
131	0.00694810276307007\\
132	0.00694798632336657\\
133	0.00694786777311141\\
134	0.00694774707413566\\
135	0.00694762418758307\\
136	0.00694749907389779\\
137	0.00694737169281185\\
138	0.00694724200333245\\
139	0.00694710996372896\\
140	0.00694697553151976\\
141	0.00694683866345879\\
142	0.00694669931552195\\
143	0.00694655744289313\\
144	0.00694641299995012\\
145	0.00694626594025021\\
146	0.00694611621651549\\
147	0.00694596378061796\\
148	0.00694580858356443\\
149	0.00694565057548093\\
150	0.00694548970559717\\
151	0.00694532592223043\\
152	0.00694515917276932\\
153	0.00694498940365725\\
154	0.00694481656037552\\
155	0.00694464058742624\\
156	0.00694446142831484\\
157	0.00694427902553231\\
158	0.00694409332053714\\
159	0.00694390425373695\\
160	0.00694371176446977\\
161	0.006943515790985\\
162	0.006943316270424\\
163	0.00694311313880048\\
164	0.00694290633098026\\
165	0.00694269578066109\\
166	0.00694248142035172\\
167	0.00694226318135076\\
168	0.00694204099372528\\
169	0.00694181478628887\\
170	0.00694158448657936\\
171	0.00694135002083621\\
172	0.00694111131397747\\
173	0.00694086828957625\\
174	0.00694062086983692\\
175	0.00694036897557088\\
176	0.00694011252617177\\
177	0.00693985143959037\\
178	0.00693958563230906\\
179	0.00693931501931578\\
180	0.00693903951407757\\
181	0.00693875902851364\\
182	0.006938473472968\\
183	0.00693818275618153\\
184	0.0069378867852637\\
185	0.00693758546566369\\
186	0.00693727870114104\\
187	0.00693696639373578\\
188	0.00693664844373811\\
189	0.00693632474965749\\
190	0.00693599520819113\\
191	0.00693565971419217\\
192	0.00693531816063699\\
193	0.00693497043859225\\
194	0.00693461643718118\\
195	0.00693425604354933\\
196	0.00693388914282978\\
197	0.0069335156181077\\
198	0.00693313535038432\\
199	0.00693274821854031\\
200	0.00693235409929843\\
201	0.00693195286718566\\
202	0.00693154439449466\\
203	0.00693112855124446\\
204	0.00693070520514063\\
205	0.00693027422153472\\
206	0.00692983546338289\\
207	0.00692938879120407\\
208	0.00692893406303712\\
209	0.00692847113439755\\
210	0.00692799985823328\\
211	0.00692752008487976\\
212	0.00692703166201432\\
213	0.00692653443460973\\
214	0.00692602824488693\\
215	0.00692551293226704\\
216	0.00692498833332255\\
217	0.00692445428172762\\
218	0.00692391060820754\\
219	0.00692335714048752\\
220	0.0069227937032403\\
221	0.00692222011803323\\
222	0.00692163620327414\\
223	0.00692104177415652\\
224	0.00692043664260375\\
225	0.00691982061721227\\
226	0.00691919350319398\\
227	0.00691855510231751\\
228	0.00691790521284864\\
229	0.00691724362948962\\
230	0.00691657014331765\\
231	0.00691588454172209\\
232	0.00691518660834084\\
233	0.00691447612299557\\
234	0.00691375286162589\\
235	0.00691301659622249\\
236	0.00691226709475902\\
237	0.00691150412112304\\
238	0.00691072743504577\\
239	0.00690993679203053\\
240	0.0069091319432803\\
241	0.00690831263562384\\
242	0.00690747861144072\\
243	0.00690662960858509\\
244	0.00690576536030822\\
245	0.0069048855951798\\
246	0.00690399003700792\\
247	0.00690307840475774\\
248	0.00690215041246895\\
249	0.00690120576917171\\
250	0.00690024417880139\\
251	0.00689926534011188\\
252	0.00689826894658748\\
253	0.00689725468635337\\
254	0.00689622224208467\\
255	0.00689517129091393\\
256	0.00689410150433732\\
257	0.00689301254811908\\
258	0.00689190408219455\\
259	0.00689077576057163\\
260	0.00688962723123057\\
261	0.00688845813602224\\
262	0.00688726811056464\\
263	0.00688605678413775\\
264	0.00688482377957676\\
265	0.00688356871316349\\
266	0.00688229119451611\\
267	0.00688099082647716\\
268	0.00687966720499996\\
269	0.00687831991903377\\
270	0.00687694855040877\\
271	0.00687555267372398\\
272	0.00687413185624691\\
273	0.00687268565784921\\
274	0.00687121363104095\\
275	0.00686971532124631\\
276	0.00686819026753304\\
277	0.00686663800347691\\
278	0.00686505805409404\\
279	0.00686344993264916\\
280	0.0068618131435009\\
281	0.00686014718256823\\
282	0.00685845153720273\\
283	0.00685672568606032\\
284	0.00685496909897197\\
285	0.00685318123681425\\
286	0.00685136155137918\\
287	0.00684950948524381\\
288	0.00684762447163972\\
289	0.00684570593432244\\
290	0.00684375328744113\\
291	0.0068417659354086\\
292	0.00683974327277196\\
293	0.00683768468408408\\
294	0.00683558954377621\\
295	0.00683345721603199\\
296	0.00683128705466313\\
297	0.00682907840298717\\
298	0.00682683059370771\\
299	0.00682454294879728\\
300	0.00682221477938361\\
301	0.00681984538563942\\
302	0.00681743405667633\\
303	0.00681498007044334\\
304	0.00681248269363003\\
305	0.00680994118157487\\
306	0.00680735477817774\\
307	0.00680472271581441\\
308	0.00680204421524527\\
309	0.00679931848549755\\
310	0.00679654472366677\\
311	0.00679372211450116\\
312	0.00679084982945135\\
313	0.00678792702453238\\
314	0.00678495283606589\\
315	0.00678192637496652\\
316	0.00677884673122257\\
317	0.00677571301116319\\
318	0.00677252431404008\\
319	0.00676927972679045\\
320	0.00676597832402508\\
321	0.00676261916802415\\
322	0.00675920130874011\\
323	0.0067557237838069\\
324	0.00675218561855415\\
325	0.00674858582602453\\
326	0.00674492340699199\\
327	0.00674119734997746\\
328	0.00673740663125787\\
329	0.00673355021486284\\
330	0.00672962705255205\\
331	0.00672563608376393\\
332	0.00672157623552386\\
333	0.0067174464222968\\
334	0.00671324554576512\\
335	0.0067089724945067\\
336	0.00670462614354091\\
337	0.00670020535369889\\
338	0.00669570897075152\\
339	0.00669113582416517\\
340	0.006686484725114\\
341	0.00668175446227064\\
342	0.00667694378810846\\
343	0.0066720514101265\\
344	0.00666707601442083\\
345	0.0066620162650849\\
346	0.0066568708036065\\
347	0.00665163824824929\\
348	0.00664631719336097\\
349	0.00664090620841075\\
350	0.0066354038361421\\
351	0.00662980858811405\\
352	0.00662411894676957\\
353	0.00661833337075079\\
354	0.00661245029465419\\
355	0.00660646812903499\\
356	0.00660038526113502\\
357	0.00659420005779262\\
358	0.00658791087504414\\
359	0.0065815160877256\\
360	0.00657501416938438\\
361	0.00656840376917955\\
362	0.00656168322022079\\
363	0.00655485078322227\\
364	0.00654790469801547\\
365	0.00654084318433572\\
366	0.00653366444282823\\
367	0.00652636665631342\\
368	0.00651894799135598\\
369	0.00651140660018733\\
370	0.00650374062303331\\
371	0.00649594819089508\\
372	0.00648802742880782\\
373	0.00647997645952718\\
374	0.00647179340737348\\
375	0.00646347640134645\\
376	0.00645502357491394\\
377	0.00644643305523614\\
378	0.00643770292244578\\
379	0.0064288310908781\\
380	0.00641981501860316\\
381	0.00641065128683896\\
382	0.0064013378902591\\
383	0.00639187317032329\\
384	0.00638225559557552\\
385	0.0063724837922119\\
386	0.00636255658396266\\
387	0.0063524730422203\\
388	0.00634223255301183\\
389	0.00633183492038799\\
390	0.00632128057070477\\
391	0.00631057048172071\\
392	0.00629970623276913\\
393	0.00628869011538802\\
394	0.00627752526468924\\
395	0.00626621581572812\\
396	0.00625476709009516\\
397	0.00624318581914798\\
398	0.0062314804117944\\
399	0.00621966127660698\\
400	0.00620774121040289\\
401	0.00619573586840969\\
402	0.00618366433499069\\
403	0.00617154981904448\\
404	0.00615942050553408\\
405	0.00614731060645685\\
406	0.00613526167730345\\
407	0.00612332430667249\\
408	0.00611156033221292\\
409	0.0061000470048815\\
410	0.00608888767423589\\
411	0.00607819528708678\\
412	0.0060681075480346\\
413	0.00605879216119026\\
414	0.00605039235996826\\
415	0.00604185743285749\\
416	0.00603318689318356\\
417	0.00602438049187169\\
418	0.00601543826557051\\
419	0.00600636059380947\\
420	0.00599714826356753\\
421	0.00598780253140995\\
422	0.00597832514686596\\
423	0.00596871821590453\\
424	0.00595898350381705\\
425	0.0059491197902136\\
426	0.00593911287090259\\
427	0.00592893537021737\\
428	0.00591858525772538\\
429	0.00590806053171042\\
430	0.00589735922382137\\
431	0.00588647940413584\\
432	0.0058754191866799\\
433	0.00586417673545369\\
434	0.00585275027102884\\
435	0.00584113807777788\\
436	0.00582933851177035\\
437	0.00581735000938853\\
438	0.00580517109684997\\
439	0.00579280040080217\\
440	0.00578023666036211\\
441	0.00576747874230785\\
442	0.00575452565753874\\
443	0.0057413765754582\\
444	0.00572803083883235\\
445	0.00571448797890645\\
446	0.00570074773024775\\
447	0.00568681004396134\\
448	0.00567267509619579\\
449	0.00565834328446983\\
450	0.00564381517755388\\
451	0.00562909122899558\\
452	0.0056141718403278\\
453	0.00559905734025508\\
454	0.00558374783688781\\
455	0.00556824300784175\\
456	0.0055525418165271\\
457	0.00553664223866819\\
458	0.0055205460104428\\
459	0.0055042548798025\\
460	0.00548777105199359\\
461	0.00547109670729628\\
462	0.00545423336834274\\
463	0.00543718144625986\\
464	0.00541993961752241\\
465	0.00540250397815816\\
466	0.00538486693980148\\
467	0.00536701575891807\\
468	0.00534893050941499\\
469	0.00533058140256037\\
470	0.00531192525664044\\
471	0.00529290082854101\\
472	0.00527342265064941\\
473	0.00525345928920426\\
474	0.00523299400938363\\
475	0.00521200905849416\\
476	0.00519048576086233\\
477	0.00516840498273882\\
478	0.00514574679734546\\
479	0.00512249027630091\\
480	0.00509861340375058\\
481	0.00507409300201966\\
482	0.00504890468123225\\
483	0.00502302294423405\\
484	0.00499642067975817\\
485	0.00496906674974231\\
486	0.0049409272542987\\
487	0.00491196446639418\\
488	0.00488214330691491\\
489	0.00485142886448882\\
490	0.0048197850881542\\
491	0.00478717504516566\\
492	0.00475356100831737\\
493	0.00471890451952105\\
494	0.0046831664209815\\
495	0.00464630683692947\\
496	0.00460828516353634\\
497	0.0045690598648893\\
498	0.00452858806221636\\
499	0.00448682481535209\\
500	0.00444372186236416\\
501	0.00439922444155628\\
502	0.00435326183881953\\
503	0.0043057663241142\\
504	0.00425668069219755\\
505	0.00420596463645533\\
506	0.00415360351552566\\
507	0.00409961616185581\\
508	0.00404406752357827\\
509	0.00398708599350592\\
510	0.00392888654377625\\
511	0.00386979945639612\\
512	0.00381098264892001\\
513	0.00375343889528498\\
514	0.00369735913363088\\
515	0.00364294323764303\\
516	0.00359039584672001\\
517	0.00353991996381999\\
518	0.00349170750589745\\
519	0.00344592571380514\\
520	0.00340269812280261\\
521	0.00336207834154239\\
522	0.00332401368087333\\
523	0.00328787877285572\\
524	0.00325235286091564\\
525	0.00321743390702476\\
526	0.00318310583236878\\
527	0.00314933622795304\\
528	0.00311607420835653\\
529	0.00308324867862273\\
530	0.00305076739670388\\
531	0.00301851739566748\\
532	0.00298636765065977\\
533	0.0029541753336562\\
534	0.00292180350962232\\
535	0.00288922044366352\\
536	0.00285639051087333\\
537	0.00282327447655823\\
538	0.00278982996972125\\
539	0.00275601217596438\\
540	0.00272177476417683\\
541	0.002687071039504\\
542	0.00265185528873788\\
543	0.00261608420867369\\
544	0.00257971821736867\\
545	0.0025427223019672\\
546	0.00250506177064863\\
547	0.00246670110567965\\
548	0.00242760442670507\\
549	0.0023877360820317\\
550	0.00234706140132897\\
551	0.00230554737593107\\
552	0.00226317048307561\\
553	0.00221991333580777\\
554	0.00217575964277463\\
555	0.00213069413955516\\
556	0.00208470244627019\\
557	0.0020377709651064\\
558	0.00198988703819046\\
559	0.00194103911958854\\
560	0.00189121696355546\\
561	0.00184041183165241\\
562	0.00178861672312973\\
563	0.00173582662502646\\
564	0.00168203875846467\\
565	0.0016272528277323\\
566	0.00157147127416107\\
567	0.00151469961622109\\
568	0.00145694766974197\\
569	0.00139823255111959\\
570	0.00133858101940291\\
571	0.00127804978197904\\
572	0.00121673136249583\\
573	0.00115457311231883\\
574	0.00109149386651943\\
575	0.00102737860674039\\
576	0.000964660699241159\\
577	0.000903620067808924\\
578	0.000844693973416802\\
579	0.00078896063058065\\
580	0.000735683997274819\\
581	0.000684393506326076\\
582	0.000635262737748697\\
583	0.000587719065319528\\
584	0.000541596458167286\\
585	0.000496834403922398\\
586	0.00045297776209589\\
587	0.0004098793501683\\
588	0.000367452930850881\\
589	0.000325673544872563\\
590	0.00028455397376655\\
591	0.000244105317769071\\
592	0.000204333810489543\\
593	0.000165233217676712\\
594	0.000126870159937007\\
595	8.96111722547522e-05\\
596	5.4266094523851e-05\\
597	2.28062284332059e-05\\
598	2.9204464504877e-07\\
599	0\\
600	0\\
};
\addplot [color=red!50!mycolor17,solid,forget plot]
  table[row sep=crcr]{%
1	0.00628094651839509\\
2	0.00628094102369023\\
3	0.00628093542892778\\
4	0.00628092973228313\\
5	0.00628092393189816\\
6	0.00628091802588075\\
7	0.00628091201230395\\
8	0.00628090588920548\\
9	0.00628089965458708\\
10	0.0062808933064137\\
11	0.00628088684261299\\
12	0.00628088026107447\\
13	0.00628087355964898\\
14	0.00628086673614781\\
15	0.00628085978834202\\
16	0.00628085271396175\\
17	0.00628084551069534\\
18	0.00628083817618871\\
19	0.00628083070804443\\
20	0.00628082310382097\\
21	0.00628081536103192\\
22	0.00628080747714516\\
23	0.00628079944958188\\
24	0.00628079127571589\\
25	0.00628078295287261\\
26	0.00628077447832827\\
27	0.00628076584930893\\
28	0.00628075706298955\\
29	0.00628074811649311\\
30	0.00628073900688952\\
31	0.00628072973119478\\
32	0.00628072028636988\\
33	0.00628071066931981\\
34	0.00628070087689251\\
35	0.00628069090587783\\
36	0.00628068075300646\\
37	0.00628067041494876\\
38	0.00628065988831376\\
39	0.00628064916964789\\
40	0.00628063825543392\\
41	0.00628062714208971\\
42	0.00628061582596708\\
43	0.0062806043033505\\
44	0.00628059257045588\\
45	0.00628058062342928\\
46	0.00628056845834568\\
47	0.00628055607120759\\
48	0.00628054345794367\\
49	0.00628053061440746\\
50	0.00628051753637593\\
51	0.00628050421954806\\
52	0.00628049065954342\\
53	0.00628047685190062\\
54	0.00628046279207592\\
55	0.00628044847544159\\
56	0.00628043389728444\\
57	0.00628041905280421\\
58	0.00628040393711189\\
59	0.00628038854522819\\
60	0.00628037287208172\\
61	0.00628035691250748\\
62	0.00628034066124487\\
63	0.00628032411293615\\
64	0.00628030726212452\\
65	0.00628029010325225\\
66	0.00628027263065894\\
67	0.0062802548385795\\
68	0.00628023672114227\\
69	0.00628021827236698\\
70	0.0062801994861628\\
71	0.00628018035632629\\
72	0.00628016087653928\\
73	0.00628014104036676\\
74	0.00628012084125468\\
75	0.00628010027252782\\
76	0.0062800793273875\\
77	0.00628005799890928\\
78	0.00628003628004066\\
79	0.0062800141635987\\
80	0.0062799916422676\\
81	0.00627996870859623\\
82	0.00627994535499566\\
83	0.00627992157373659\\
84	0.0062798973569467\\
85	0.00627987269660813\\
86	0.00627984758455471\\
87	0.00627982201246914\\
88	0.00627979597188038\\
89	0.00627976945416064\\
90	0.00627974245052258\\
91	0.00627971495201631\\
92	0.0062796869495264\\
93	0.00627965843376886\\
94	0.00627962939528797\\
95	0.00627959982445314\\
96	0.00627956971145567\\
97	0.00627953904630546\\
98	0.0062795078188277\\
99	0.00627947601865933\\
100	0.00627944363524576\\
101	0.00627941065783719\\
102	0.00627937707548512\\
103	0.00627934287703853\\
104	0.00627930805114029\\
105	0.00627927258622331\\
106	0.00627923647050663\\
107	0.00627919969199157\\
108	0.00627916223845761\\
109	0.00627912409745839\\
110	0.00627908525631748\\
111	0.00627904570212421\\
112	0.00627900542172933\\
113	0.00627896440174055\\
114	0.00627892262851818\\
115	0.00627888008817048\\
116	0.00627883676654913\\
117	0.00627879264924439\\
118	0.00627874772158035\\
119	0.00627870196861005\\
120	0.00627865537511043\\
121	0.00627860792557735\\
122	0.00627855960422035\\
123	0.0062785103949574\\
124	0.00627846028140953\\
125	0.00627840924689548\\
126	0.00627835727442593\\
127	0.00627830434669812\\
128	0.00627825044608983\\
129	0.0062781955546537\\
130	0.00627813965411117\\
131	0.00627808272584647\\
132	0.00627802475090034\\
133	0.00627796570996379\\
134	0.00627790558337171\\
135	0.00627784435109629\\
136	0.00627778199274037\\
137	0.0062777184875307\\
138	0.00627765381431106\\
139	0.00627758795153517\\
140	0.00627752087725963\\
141	0.00627745256913659\\
142	0.00627738300440642\\
143	0.00627731215989005\\
144	0.00627724001198138\\
145	0.00627716653663946\\
146	0.00627709170938051\\
147	0.00627701550526985\\
148	0.00627693789891362\\
149	0.00627685886445043\\
150	0.00627677837554275\\
151	0.00627669640536831\\
152	0.0062766129266111\\
153	0.00627652791145255\\
154	0.00627644133156215\\
155	0.00627635315808825\\
156	0.00627626336164851\\
157	0.00627617191232016\\
158	0.00627607877963023\\
159	0.00627598393254549\\
160	0.00627588733946222\\
161	0.00627578896819584\\
162	0.00627568878597031\\
163	0.00627558675940736\\
164	0.0062754828545156\\
165	0.00627537703667921\\
166	0.00627526927064674\\
167	0.00627515952051947\\
168	0.00627504774973969\\
169	0.00627493392107862\\
170	0.00627481799662442\\
171	0.00627469993776958\\
172	0.00627457970519841\\
173	0.00627445725887422\\
174	0.00627433255802615\\
175	0.00627420556113595\\
176	0.00627407622592435\\
177	0.0062739445093374\\
178	0.00627381036753238\\
179	0.0062736737558635\\
180	0.00627353462886746\\
181	0.00627339294024866\\
182	0.00627324864286411\\
183	0.0062731016887082\\
184	0.00627295202889711\\
185	0.00627279961365297\\
186	0.00627264439228783\\
187	0.00627248631318715\\
188	0.00627232532379321\\
189	0.00627216137058811\\
190	0.00627199439907659\\
191	0.00627182435376837\\
192	0.00627165117816039\\
193	0.00627147481471861\\
194	0.0062712952048595\\
195	0.00627111228893133\\
196	0.00627092600619503\\
197	0.00627073629480474\\
198	0.00627054309178807\\
199	0.00627034633302594\\
200	0.00627014595323224\\
201	0.00626994188593301\\
202	0.00626973406344522\\
203	0.00626952241685541\\
204	0.00626930687599775\\
205	0.00626908736943188\\
206	0.00626886382442027\\
207	0.00626863616690529\\
208	0.00626840432148594\\
209	0.00626816821139397\\
210	0.00626792775846993\\
211	0.00626768288313855\\
212	0.00626743350438385\\
213	0.00626717953972388\\
214	0.00626692090518494\\
215	0.00626665751527545\\
216	0.00626638928295935\\
217	0.00626611611962919\\
218	0.00626583793507867\\
219	0.00626555463747476\\
220	0.00626526613332943\\
221	0.00626497232747089\\
222	0.00626467312301442\\
223	0.00626436842133268\\
224	0.00626405812202566\\
225	0.00626374212288999\\
226	0.00626342031988799\\
227	0.00626309260711611\\
228	0.00626275887677293\\
229	0.00626241901912664\\
230	0.00626207292248208\\
231	0.00626172047314722\\
232	0.00626136155539926\\
233	0.00626099605145005\\
234	0.00626062384141118\\
235	0.0062602448032584\\
236	0.00625985881279563\\
237	0.00625946574361841\\
238	0.00625906546707674\\
239	0.00625865785223759\\
240	0.00625824276584659\\
241	0.00625782007228937\\
242	0.00625738963355231\\
243	0.00625695130918266\\
244	0.00625650495624811\\
245	0.00625605042929581\\
246	0.00625558758031067\\
247	0.00625511625867318\\
248	0.0062546363111165\\
249	0.00625414758168291\\
250	0.00625364991167956\\
251	0.00625314313963339\\
252	0.00625262710124551\\
253	0.00625210162934443\\
254	0.0062515665538387\\
255	0.0062510217016683\\
256	0.00625046689675522\\
257	0.00624990195995269\\
258	0.00624932670899319\\
259	0.00624874095843501\\
260	0.0062481445196072\\
261	0.00624753720055269\\
262	0.00624691880596951\\
263	0.00624628913714975\\
264	0.006245647991916\\
265	0.00624499516455529\\
266	0.00624433044575023\\
267	0.00624365362250734\\
268	0.00624296447808276\\
269	0.00624226279190632\\
270	0.00624154833950568\\
271	0.00624082089243628\\
272	0.00624008021823186\\
273	0.00623932608041917\\
274	0.00623855823872793\\
275	0.0062377764498981\\
276	0.00623698047030328\\
277	0.0062361700637432\\
278	0.0062353450193773\\
279	0.00623450512588287\\
280	0.00623365013075074\\
281	0.00623277977116969\\
282	0.00623189378012479\\
283	0.00623099188634747\\
284	0.00623007381426606\\
285	0.00622913928395706\\
286	0.00622818801109727\\
287	0.00622721970691689\\
288	0.00622623407815404\\
289	0.00622523082701093\\
290	0.00622420965111174\\
291	0.00622317024346303\\
292	0.00622211229241692\\
293	0.00622103548163765\\
294	0.0062199394900721\\
295	0.00621882399192518\\
296	0.00621768865664089\\
297	0.00621653314889005\\
298	0.00621535712856596\\
299	0.0062141602507891\\
300	0.00621294216592259\\
301	0.00621170251959943\\
302	0.00621044095276355\\
303	0.00620915710172548\\
304	0.006207850598234\\
305	0.00620652106956347\\
306	0.00620516813861468\\
307	0.00620379142402274\\
308	0.00620239054025357\\
309	0.0062009650976434\\
310	0.00619951470226192\\
311	0.00619803895528635\\
312	0.00619653745105236\\
313	0.00619500977157134\\
314	0.00619345547183289\\
315	0.00619187404272906\\
316	0.00619026483260075\\
317	0.00618862701493381\\
318	0.00618696005667385\\
319	0.0061852634724058\\
320	0.00618353677000882\\
321	0.00618177945060518\\
322	0.00617999100850778\\
323	0.00617817093116535\\
324	0.00617631869910501\\
325	0.00617443378587163\\
326	0.00617251565796289\\
327	0.00617056377475961\\
328	0.00616857758845009\\
329	0.0061665565439479\\
330	0.00616450007880163\\
331	0.00616240762309565\\
332	0.00616027859934056\\
333	0.00615811242235246\\
334	0.0061559084991195\\
335	0.00615366622865523\\
336	0.00615138500183909\\
337	0.00614906420124558\\
338	0.00614670320096608\\
339	0.0061443013664249\\
340	0.00614185805416666\\
341	0.00613937261154164\\
342	0.00613684437664284\\
343	0.00613427267874974\\
344	0.00613165683802625\\
345	0.00612899616513374\\
346	0.00612628996073316\\
347	0.00612353751484298\\
348	0.00612073810601039\\
349	0.00611789100024825\\
350	0.0061149954497102\\
351	0.00611205069110537\\
352	0.00610905594355766\\
353	0.00610601040567376\\
354	0.0061029132518177\\
355	0.00609976362747239\\
356	0.00609656064378319\\
357	0.00609330337230913\\
358	0.00608999084475669\\
359	0.00608662207816113\\
360	0.00608319621868035\\
361	0.00607971329950062\\
362	0.00607617618023874\\
363	0.00607258442884062\\
364	0.00606893764994095\\
365	0.0060652354907542\\
366	0.00606147764795087\\
367	0.00605766387572508\\
368	0.00605379399530891\\
369	0.00604986790625016\\
370	0.00604588559984586\\
371	0.00604184717521095\\
372	0.00603775285855215\\
373	0.00603360302627036\\
374	0.00602939823240385\\
375	0.0060251392402592\\
376	0.0060208270556653\\
377	0.00601646295147004\\
378	0.00601204844818144\\
379	0.00600758513639789\\
380	0.00600307396102361\\
381	0.00599851261440092\\
382	0.00599388544244718\\
383	0.00598919227205787\\
384	0.00598443314402723\\
385	0.0059796081968888\\
386	0.00597471767623049\\
387	0.00596976194436436\\
388	0.00596474149005861\\
389	0.00595965693759411\\
390	0.0059545090529708\\
391	0.00594929875289887\\
392	0.00594402711429341\\
393	0.00593869538304248\\
394	0.00593330498149172\\
395	0.00592785751390258\\
396	0.00592235476888149\\
397	0.00591679871743948\\
398	0.00591119150489946\\
399	0.00590553543428755\\
400	0.0058998329380851\\
401	0.00589408653422213\\
402	0.00588829876088577\\
403	0.00588247208299798\\
404	0.00587660876093922\\
405	0.00587071066903032\\
406	0.00586477904703945\\
407	0.00585881416217398\\
408	0.00585281485363119\\
409	0.00584677790125402\\
410	0.0058406970847465\\
411	0.00583456223492706\\
412	0.0058283578137602\\
413	0.00582206104411124\\
414	0.00581564132950739\\
415	0.00580909611989684\\
416	0.00580242278769309\\
417	0.00579561862140608\\
418	0.00578868081793819\\
419	0.00578160647315369\\
420	0.00577439257025361\\
421	0.00576703596549714\\
422	0.00575953337116851\\
423	0.00575188133728517\\
424	0.00574407623924873\\
425	0.00573611429910769\\
426	0.00572799174826807\\
427	0.00571970501971221\\
428	0.0057112504374799\\
429	0.00570262421128045\\
430	0.00569382243073522\\
431	0.00568484105921746\\
432	0.005675675927249\\
433	0.00566632272540006\\
434	0.00565677699666341\\
435	0.00564703412839972\\
436	0.00563708934390052\\
437	0.00562693769319163\\
438	0.0056165740425122\\
439	0.00560599306326688\\
440	0.00559518922013013\\
441	0.00558415675807559\\
442	0.00557288968817381\\
443	0.00556138177202994\\
444	0.00554962650456028\\
445	0.00553761709472205\\
446	0.00552534644538259\\
447	0.00551280713336373\\
448	0.00549999138768484\\
449	0.0054868910588211\\
450	0.00547349758831802\\
451	0.00545980198065547\\
452	0.0054457947712163\\
453	0.00543146599114788\\
454	0.00541680513034979\\
455	0.00540180109703208\\
456	0.00538644216815536\\
457	0.00537071598176243\\
458	0.00535460946385435\\
459	0.00533810875409222\\
460	0.00532119911108202\\
461	0.00530386483378619\\
462	0.00528608918842895\\
463	0.00526785433637298\\
464	0.00524914126685343\\
465	0.00522992969894428\\
466	0.00521019785529346\\
467	0.00518992263385337\\
468	0.0051690801278424\\
469	0.00514764554730165\\
470	0.00512559328073365\\
471	0.00510289727727141\\
472	0.00507953175229053\\
473	0.005055469974674\\
474	0.00503068410002096\\
475	0.00500514388755872\\
476	0.00497881545852904\\
477	0.00495166159305947\\
478	0.00492364845257451\\
479	0.00489474256434685\\
480	0.00486491000694632\\
481	0.00483411653850486\\
482	0.00480232771466033\\
483	0.004769508966389\\
484	0.00473562561016074\\
485	0.00470064279906671\\
486	0.00466452525591064\\
487	0.00462723672678512\\
488	0.00458873886863503\\
489	0.00454898918719202\\
490	0.00450793510723609\\
491	0.00446551549812167\\
492	0.00442167582343966\\
493	0.0043763724822605\\
494	0.00432957844475902\\
495	0.00428129072677211\\
496	0.00423154041810466\\
497	0.0041804059763298\\
498	0.00412803078363405\\
499	0.0040746463120854\\
500	0.00402060285304561\\
501	0.00396743933357704\\
502	0.00391556399806644\\
503	0.00386515616431572\\
504	0.00381640428098737\\
505	0.00376949956777484\\
506	0.00372462926229314\\
507	0.00368196700638546\\
508	0.0036416591034706\\
509	0.00360380526237516\\
510	0.00356843183757421\\
511	0.00353545553400809\\
512	0.00350392253949669\\
513	0.00347300447686475\\
514	0.00344269685930694\\
515	0.00341298179059693\\
516	0.00338382592998904\\
517	0.00335517866997427\\
518	0.00332697080439069\\
519	0.00329911409782002\\
520	0.00327150234383665\\
521	0.0032440147440854\\
522	0.00321652281051031\\
523	0.00318892662393476\\
524	0.00316120027083114\\
525	0.00313331491948637\\
526	0.0031052391674743\\
527	0.00307693956129006\\
528	0.00304838130493532\\
529	0.00301952916079463\\
530	0.00299034852332022\\
531	0.00296080660805138\\
532	0.00293087363594622\\
533	0.00290052379454109\\
534	0.00286973531170303\\
535	0.00283848542659518\\
536	0.00280675046465217\\
537	0.00277450591734095\\
538	0.00274172652008376\\
539	0.00270838631990362\\
540	0.00267445872297293\\
541	0.00263991651194983\\
542	0.00260473182461684\\
543	0.00256887609111131\\
544	0.00253231993928337\\
545	0.00249503310058525\\
546	0.00245698458609395\\
547	0.00241814303104803\\
548	0.00237847717142211\\
549	0.002337956497622\\
550	0.00229655183905641\\
551	0.00225424597099964\\
552	0.00221102370153545\\
553	0.00216687040382646\\
554	0.00212177211836048\\
555	0.00207571568159297\\
556	0.002028688891195\\
557	0.00198068071385247\\
558	0.00193168153175534\\
559	0.00188168342228085\\
560	0.00183068044001795\\
561	0.00177866892288609\\
562	0.00172564781376172\\
563	0.00167161948272084\\
564	0.00161659226410366\\
565	0.00156058298271428\\
566	0.00150362990328009\\
567	0.00144583137164707\\
568	0.00138714982852563\\
569	0.00132751977693301\\
570	0.00126683277350473\\
571	0.00120492049445878\\
572	0.00114382947878174\\
573	0.00108426163518732\\
574	0.00102666342567682\\
575	0.000971846090115743\\
576	0.000918265947001419\\
577	0.000866422173777079\\
578	0.000816462067759592\\
579	0.000767915527104241\\
580	0.000720194215446356\\
581	0.000673645505827697\\
582	0.000628150962911244\\
583	0.000583436223402909\\
584	0.000539226991113701\\
585	0.000495475690639145\\
586	0.000452196148255144\\
587	0.000409418505768188\\
588	0.000367181587116058\\
589	0.000325525791425008\\
590	0.000284484611477045\\
591	0.000244081548911079\\
592	0.000204328087251261\\
593	0.000165233217676711\\
594	0.000126870159937007\\
595	8.96111722547522e-05\\
596	5.42660945238508e-05\\
597	2.28062284332056e-05\\
598	2.9204464504877e-07\\
599	0\\
600	0\\
};
\addplot [color=red!40!mycolor19,solid,forget plot]
  table[row sep=crcr]{%
1	0.00609949007723919\\
2	0.00609948663958327\\
3	0.00609948313907455\\
4	0.00609947957456534\\
5	0.00609947594488679\\
6	0.00609947224884837\\
7	0.00609946848523761\\
8	0.00609946465281958\\
9	0.00609946075033646\\
10	0.00609945677650724\\
11	0.00609945273002708\\
12	0.00609944860956702\\
13	0.00609944441377342\\
14	0.00609944014126752\\
15	0.00609943579064496\\
16	0.00609943136047536\\
17	0.00609942684930172\\
18	0.00609942225563992\\
19	0.00609941757797826\\
20	0.00609941281477694\\
21	0.00609940796446747\\
22	0.00609940302545211\\
23	0.00609939799610336\\
24	0.00609939287476341\\
25	0.00609938765974349\\
26	0.00609938234932327\\
27	0.00609937694175034\\
28	0.00609937143523951\\
29	0.00609936582797223\\
30	0.00609936011809597\\
31	0.00609935430372347\\
32	0.00609934838293216\\
33	0.00609934235376347\\
34	0.0060993362142221\\
35	0.00609932996227535\\
36	0.00609932359585234\\
37	0.00609931711284339\\
38	0.00609931051109918\\
39	0.00609930378842994\\
40	0.00609929694260481\\
41	0.00609928997135099\\
42	0.00609928287235284\\
43	0.00609927564325116\\
44	0.00609926828164235\\
45	0.00609926078507749\\
46	0.00609925315106148\\
47	0.00609924537705219\\
48	0.0060992374604595\\
49	0.00609922939864443\\
50	0.0060992211889181\\
51	0.0060992128285409\\
52	0.00609920431472136\\
53	0.00609919564461529\\
54	0.00609918681532465\\
55	0.00609917782389656\\
56	0.00609916866732222\\
57	0.00609915934253584\\
58	0.00609914984641353\\
59	0.00609914017577214\\
60	0.00609913032736824\\
61	0.00609912029789677\\
62	0.00609911008399001\\
63	0.00609909968221624\\
64	0.00609908908907858\\
65	0.0060990783010137\\
66	0.00609906731439054\\
67	0.00609905612550898\\
68	0.00609904473059852\\
69	0.00609903312581693\\
70	0.00609902130724886\\
71	0.00609900927090432\\
72	0.00609899701271745\\
73	0.00609898452854482\\
74	0.00609897181416409\\
75	0.00609895886527241\\
76	0.00609894567748484\\
77	0.00609893224633285\\
78	0.00609891856726264\\
79	0.00609890463563345\\
80	0.00609889044671596\\
81	0.00609887599569055\\
82	0.00609886127764551\\
83	0.00609884628757535\\
84	0.00609883102037887\\
85	0.00609881547085742\\
86	0.00609879963371292\\
87	0.00609878350354604\\
88	0.00609876707485416\\
89	0.00609875034202935\\
90	0.00609873329935646\\
91	0.00609871594101096\\
92	0.00609869826105684\\
93	0.00609868025344447\\
94	0.00609866191200839\\
95	0.00609864323046511\\
96	0.00609862420241078\\
97	0.00609860482131893\\
98	0.00609858508053802\\
99	0.00609856497328922\\
100	0.00609854449266368\\
101	0.00609852363162027\\
102	0.00609850238298286\\
103	0.00609848073943784\\
104	0.00609845869353143\\
105	0.00609843623766696\\
106	0.00609841336410217\\
107	0.00609839006494641\\
108	0.0060983663321577\\
109	0.00609834215753996\\
110	0.00609831753273993\\
111	0.00609829244924423\\
112	0.00609826689837624\\
113	0.00609824087129303\\
114	0.00609821435898207\\
115	0.0060981873522581\\
116	0.00609815984175969\\
117	0.00609813181794595\\
118	0.00609810327109312\\
119	0.00609807419129092\\
120	0.00609804456843922\\
121	0.00609801439224414\\
122	0.00609798365221457\\
123	0.00609795233765821\\
124	0.00609792043767789\\
125	0.00609788794116754\\
126	0.00609785483680825\\
127	0.00609782111306418\\
128	0.00609778675817847\\
129	0.00609775176016894\\
130	0.00609771610682384\\
131	0.00609767978569751\\
132	0.00609764278410584\\
133	0.00609760508912176\\
134	0.00609756668757063\\
135	0.00609752756602546\\
136	0.00609748771080222\\
137	0.00609744710795483\\
138	0.00609740574327021\\
139	0.00609736360226324\\
140	0.00609732067017151\\
141	0.00609727693195015\\
142	0.00609723237226632\\
143	0.00609718697549387\\
144	0.00609714072570772\\
145	0.00609709360667812\\
146	0.00609704560186496\\
147	0.00609699669441184\\
148	0.00609694686714002\\
149	0.00609689610254236\\
150	0.0060968443827771\\
151	0.00609679168966141\\
152	0.00609673800466514\\
153	0.00609668330890394\\
154	0.00609662758313276\\
155	0.00609657080773896\\
156	0.00609651296273535\\
157	0.00609645402775305\\
158	0.00609639398203434\\
159	0.00609633280442522\\
160	0.00609627047336795\\
161	0.00609620696689341\\
162	0.0060961422626133\\
163	0.00609607633771221\\
164	0.00609600916893961\\
165	0.0060959407326015\\
166	0.00609587100455211\\
167	0.00609579996018535\\
168	0.00609572757442607\\
169	0.0060956538217213\\
170	0.00609557867603111\\
171	0.00609550211081947\\
172	0.00609542409904487\\
173	0.00609534461315076\\
174	0.00609526362505591\\
175	0.00609518110614439\\
176	0.00609509702725557\\
177	0.00609501135867384\\
178	0.00609492407011807\\
179	0.00609483513073107\\
180	0.00609474450906861\\
181	0.00609465217308845\\
182	0.00609455809013899\\
183	0.00609446222694786\\
184	0.00609436454961022\\
185	0.00609426502357682\\
186	0.00609416361364184\\
187	0.00609406028393064\\
188	0.00609395499788708\\
189	0.0060938477182607\\
190	0.00609373840709372\\
191	0.00609362702570769\\
192	0.00609351353468993\\
193	0.00609339789387978\\
194	0.00609328006235448\\
195	0.00609315999841487\\
196	0.00609303765957081\\
197	0.00609291300252628\\
198	0.0060927859831643\\
199	0.00609265655653147\\
200	0.00609252467682226\\
201	0.00609239029736297\\
202	0.00609225337059552\\
203	0.00609211384806071\\
204	0.00609197168038146\\
205	0.00609182681724539\\
206	0.00609167920738737\\
207	0.00609152879857161\\
208	0.00609137553757333\\
209	0.00609121937016029\\
210	0.00609106024107379\\
211	0.00609089809400936\\
212	0.00609073287159712\\
213	0.0060905645153817\\
214	0.00609039296580175\\
215	0.00609021816216924\\
216	0.00609004004264809\\
217	0.00608985854423248\\
218	0.00608967360272485\\
219	0.00608948515271332\\
220	0.0060892931275486\\
221	0.00608909745932065\\
222	0.0060888980788347\\
223	0.00608869491558677\\
224	0.00608848789773869\\
225	0.00608827695209266\\
226	0.0060880620040652\\
227	0.00608784297766047\\
228	0.00608761979544311\\
229	0.00608739237851043\\
230	0.00608716064646387\\
231	0.0060869245173799\\
232	0.00608668390778015\\
233	0.00608643873260086\\
234	0.00608618890516139\\
235	0.00608593433713219\\
236	0.00608567493850159\\
237	0.00608541061754193\\
238	0.00608514128077454\\
239	0.00608486683293366\\
240	0.00608458717692941\\
241	0.00608430221380937\\
242	0.00608401184271884\\
243	0.0060837159608598\\
244	0.00608341446344822\\
245	0.00608310724366962\\
246	0.00608279419263279\\
247	0.00608247519932137\\
248	0.00608215015054305\\
249	0.00608181893087611\\
250	0.00608148142261296\\
251	0.00608113750570028\\
252	0.0060807870576751\\
253	0.00608042995359647\\
254	0.00608006606597174\\
255	0.00607969526467671\\
256	0.00607931741686841\\
257	0.00607893238688942\\
258	0.00607854003616171\\
259	0.00607814022306827\\
260	0.00607773280281994\\
261	0.00607731762730415\\
262	0.00607689454491185\\
263	0.00607646340033768\\
264	0.00607602403434751\\
265	0.00607557628350515\\
266	0.00607511997984932\\
267	0.00607465495050848\\
268	0.0060741810172383\\
269	0.00607369799586306\\
270	0.00607320569559741\\
271	0.00607270391822066\\
272	0.00607219245707472\\
273	0.00607167109586992\\
274	0.00607113960735399\\
275	0.00607059775219016\\
276	0.00607004527956652\\
277	0.00606948193612833\\
278	0.0060689075145937\\
279	0.00606832212225923\\
280	0.00606772596503167\\
281	0.00606711885765475\\
282	0.00606650061256708\\
283	0.00606587103995243\\
284	0.00606522994780081\\
285	0.00606457714198132\\
286	0.00606391242632902\\
287	0.00606323560274758\\
288	0.00606254647133033\\
289	0.00606184483050241\\
290	0.00606113047718779\\
291	0.00606040320700509\\
292	0.00605966281449736\\
293	0.00605890909340178\\
294	0.00605814183696667\\
295	0.0060573608383249\\
296	0.00605656589093458\\
297	0.0060557567891005\\
298	0.00605493332859318\\
299	0.00605409530738581\\
300	0.00605324252653497\\
301	0.0060523747912365\\
302	0.00605149191209669\\
303	0.0060505937066678\\
304	0.00604968000131025\\
305	0.00604875063345861\\
306	0.006047805454388\\
307	0.00604684433259849\\
308	0.00604586715795675\\
309	0.00604487384674168\\
310	0.00604386434769657\\
311	0.0060428386489718\\
312	0.00604179678508451\\
313	0.00604073884058972\\
314	0.00603966493946368\\
315	0.00603857518426247\\
316	0.00603746942339006\\
317	0.00603634639281748\\
318	0.00603520227378819\\
319	0.00603403668044389\\
320	0.00603284922021106\\
321	0.00603163949369277\\
322	0.00603040709455885\\
323	0.00602915160943468\\
324	0.00602787261778861\\
325	0.00602656969181786\\
326	0.00602524239633337\\
327	0.00602389028864327\\
328	0.00602251291843567\\
329	0.00602110982766015\\
330	0.00601968055040881\\
331	0.00601822461279687\\
332	0.00601674153284308\\
333	0.00601523082035019\\
334	0.00601369197678595\\
335	0.00601212449516497\\
336	0.006010527859932\\
337	0.00600890154684694\\
338	0.00600724502287205\\
339	0.00600555774606172\\
340	0.00600383916545539\\
341	0.00600208872097596\\
342	0.00600030584332892\\
343	0.0059984899538931\\
344	0.00599664046461557\\
345	0.00599475677791284\\
346	0.00599283828658081\\
347	0.00599088437371715\\
348	0.00598889441266128\\
349	0.00598686776695855\\
350	0.00598480379035758\\
351	0.00598270182685029\\
352	0.00598056121077298\\
353	0.00597838126699589\\
354	0.00597616131123422\\
355	0.00597390065052333\\
356	0.00597159858391212\\
357	0.00596925440342943\\
358	0.00596686739533017\\
359	0.00596443684136008\\
360	0.00596196201846727\\
361	0.00595944218832653\\
362	0.00595687655876742\\
363	0.00595426432285699\\
364	0.00595160465835615\\
365	0.00594889672710658\\
366	0.00594613967433101\\
367	0.00594333262782399\\
368	0.00594047469700265\\
369	0.00593756497177677\\
370	0.00593460252118352\\
371	0.00593158639171427\\
372	0.00592851560523585\\
373	0.00592538915637774\\
374	0.00592220600921807\\
375	0.00591896509306117\\
376	0.00591566529708073\\
377	0.00591230546367939\\
378	0.00590888438083942\\
379	0.00590540077532086\\
380	0.00590185331409043\\
381	0.0058982406413478\\
382	0.0058945615607665\\
383	0.00589081484917345\\
384	0.00588699925315538\\
385	0.00588311348689868\\
386	0.00587915622972221\\
387	0.00587512612325945\\
388	0.00587102176824432\\
389	0.00586684172085428\\
390	0.00586258448862514\\
391	0.00585824852581624\\
392	0.00585383222813431\\
393	0.0058493339267348\\
394	0.00584475188141529\\
395	0.00584008427291369\\
396	0.00583532919422693\\
397	0.00583048464087574\\
398	0.00582554850005957\\
399	0.00582051853867892\\
400	0.00581539239025816\\
401	0.00581016754088863\\
402	0.00580484131444731\\
403	0.00579941085753478\\
404	0.00579387312485534\\
405	0.0057882248661641\\
406	0.00578246261647293\\
407	0.00577658269202535\\
408	0.00577058119559946\\
409	0.0057644540365423\\
410	0.00575819697446965\\
411	0.0057518056932042\\
412	0.00574527592104245\\
413	0.00573860362103671\\
414	0.00573178525222442\\
415	0.00572481715987743\\
416	0.00571769557054584\\
417	0.00571041658686079\\
418	0.00570297618209032\\
419	0.00569537019445206\\
420	0.00568759432119414\\
421	0.00567964411248218\\
422	0.00567151496514972\\
423	0.0056632021163124\\
424	0.00565470063701057\\
425	0.00564600542546769\\
426	0.0056371111976312\\
427	0.00562801247359913\\
428	0.00561870356763885\\
429	0.00560917857772503\\
430	0.00559943137445791\\
431	0.00558945558930404\\
432	0.0055792446020997\\
433	0.00556879152770817\\
434	0.00555808920160435\\
435	0.00554713016404182\\
436	0.00553590664433304\\
437	0.00552441054698277\\
438	0.00551263343706816\\
439	0.00550056651711529\\
440	0.00548820060611104\\
441	0.00547552611821937\\
442	0.0054625330403664\\
443	0.00544921090865481\\
444	0.00543554878338468\\
445	0.00542153522134404\\
446	0.00540715823967644\\
447	0.00539240528487763\\
448	0.00537726322626625\\
449	0.00536171836902219\\
450	0.00534575638316665\\
451	0.00532936225790829\\
452	0.00531252026494597\\
453	0.00529521391674586\\
454	0.00527742591865204\\
455	0.00525913811865879\\
456	0.00524033143738105\\
457	0.00522098557186356\\
458	0.00520107913309478\\
459	0.00518059002488525\\
460	0.00515949530239541\\
461	0.00513777091349513\\
462	0.00511539163234859\\
463	0.00509233104181511\\
464	0.00506856155377677\\
465	0.00504405459075588\\
466	0.00501878070847056\\
467	0.00499270703806631\\
468	0.00496579830873763\\
469	0.00493802272386192\\
470	0.00490934976632691\\
471	0.00487974891411962\\
472	0.00484918962606995\\
473	0.0048176412409891\\
474	0.00478507269140209\\
475	0.00475145198857918\\
476	0.00471674534696897\\
477	0.00468091556270191\\
478	0.00464391755227902\\
479	0.00460569392592616\\
480	0.00456619201277347\\
481	0.00452536715465093\\
482	0.00448318728872223\\
483	0.00443963912314147\\
484	0.00439473634050376\\
485	0.00434853028216572\\
486	0.00430112419214191\\
487	0.00425269202561763\\
488	0.00420350323911374\\
489	0.00415437121824339\\
490	0.00410637070479002\\
491	0.00405966503355721\\
492	0.00401442487833185\\
493	0.00397082558574463\\
494	0.00392904397100763\\
495	0.00388925106606612\\
496	0.00385160133021408\\
497	0.00381621843275766\\
498	0.00378317542947194\\
499	0.00375246742237028\\
500	0.00372397408550459\\
501	0.00369632046858735\\
502	0.00366926832203856\\
503	0.00364280960574355\\
504	0.00361692350771421\\
505	0.00359157468192518\\
506	0.00356671175220235\\
507	0.00354226635502932\\
508	0.00351815313048391\\
509	0.00349427124709755\\
510	0.0034705082972028\\
511	0.00344674770856446\\
512	0.00342292178581892\\
513	0.00339900788620017\\
514	0.00337498108408554\\
515	0.00335081455714488\\
516	0.00332648013014735\\
517	0.00330194898684415\\
518	0.00327719254604736\\
519	0.00325218347362656\\
520	0.00322689676323357\\
521	0.00320131075848223\\
522	0.00317540789701119\\
523	0.00314917367059463\\
524	0.00312259320271672\\
525	0.00309565133254662\\
526	0.00306833269407408\\
527	0.00304062178171857\\
528	0.00301250299167785\\
529	0.0029839606266123\\
530	0.00295497885052935\\
531	0.00292554158190216\\
532	0.00289563231772567\\
533	0.00286523389183085\\
534	0.00283432820493941\\
535	0.00280289616912907\\
536	0.00277091764986146\\
537	0.00273837140739797\\
538	0.00270523504060233\\
539	0.00267148493771352\\
540	0.00263709624074912\\
541	0.00260204283283811\\
542	0.00256629736102423\\
543	0.0025298313108894\\
544	0.00249261515356304\\
545	0.00245461858992128\\
546	0.00241581091084629\\
547	0.00237616150442382\\
548	0.00233564055112879\\
549	0.00229421974472738\\
550	0.00225188413966668\\
551	0.00220861946225582\\
552	0.00216441225788334\\
553	0.00211925011293704\\
554	0.00207312192622846\\
555	0.00202601822576494\\
556	0.00197793150135645\\
557	0.00192885657953965\\
558	0.00187879103415134\\
559	0.00182773633444033\\
560	0.00177570121612501\\
561	0.00172270419382382\\
562	0.00166883078065186\\
563	0.00161408765059616\\
564	0.00155842221278116\\
565	0.00150176416055747\\
566	0.00144398692427377\\
567	0.00138484423759802\\
568	0.00132527813048257\\
569	0.00126704480548041\\
570	0.00121055312007744\\
571	0.00115653787994659\\
572	0.00110330586081578\\
573	0.00105101228541044\\
574	0.00100035963156781\\
575	0.000951144518913378\\
576	0.000902948852799658\\
577	0.000855365674177795\\
578	0.000808355188236387\\
579	0.000762216035417117\\
580	0.00071690457967\\
581	0.00067198153227095\\
582	0.000627351837523727\\
583	0.000583007319138113\\
584	0.000538986303744318\\
585	0.000495338379819224\\
586	0.000452116120303638\\
587	0.000409372268862694\\
588	0.000367156964357134\\
589	0.000325514548405105\\
590	0.000284480845955471\\
591	0.000244080684870838\\
592	0.000204328087251261\\
593	0.000165233217676712\\
594	0.000126870159937008\\
595	8.96111722547531e-05\\
596	5.42660945238511e-05\\
597	2.28062284332057e-05\\
598	2.9204464504877e-07\\
599	0\\
600	0\\
};
\addplot [color=red!75!mycolor17,solid,forget plot]
  table[row sep=crcr]{%
1	0.0060542548985145\\
2	0.00605425199370972\\
3	0.00605424903553986\\
4	0.00605424602303839\\
5	0.00605424295522111\\
6	0.006054239831086\\
7	0.00605423664961272\\
8	0.00605423340976236\\
9	0.00605423011047707\\
10	0.00605422675067964\\
11	0.00605422332927326\\
12	0.00605421984514103\\
13	0.00605421629714559\\
14	0.00605421268412883\\
15	0.00605420900491139\\
16	0.00605420525829226\\
17	0.00605420144304841\\
18	0.00605419755793437\\
19	0.00605419360168178\\
20	0.00605418957299889\\
21	0.00605418547057024\\
22	0.00605418129305609\\
23	0.00605417703909203\\
24	0.00605417270728841\\
25	0.00605416829622995\\
26	0.00605416380447523\\
27	0.00605415923055607\\
28	0.00605415457297718\\
29	0.0060541498302155\\
30	0.00605414500071974\\
31	0.00605414008290975\\
32	0.00605413507517608\\
33	0.00605412997587927\\
34	0.00605412478334934\\
35	0.00605411949588519\\
36	0.00605411411175401\\
37	0.00605410862919055\\
38	0.0060541030463966\\
39	0.00605409736154033\\
40	0.00605409157275556\\
41	0.00605408567814111\\
42	0.00605407967576014\\
43	0.00605407356363946\\
44	0.00605406733976871\\
45	0.00605406100209975\\
46	0.00605405454854584\\
47	0.00605404797698082\\
48	0.0060540412852385\\
49	0.00605403447111171\\
50	0.00605402753235154\\
51	0.0060540204666665\\
52	0.00605401327172166\\
53	0.00605400594513783\\
54	0.00605399848449061\\
55	0.00605399088730956\\
56	0.00605398315107722\\
57	0.0060539752732282\\
58	0.00605396725114822\\
59	0.00605395908217315\\
60	0.00605395076358791\\
61	0.00605394229262556\\
62	0.00605393366646624\\
63	0.00605392488223606\\
64	0.00605391593700604\\
65	0.00605390682779104\\
66	0.00605389755154856\\
67	0.00605388810517763\\
68	0.00605387848551762\\
69	0.00605386868934704\\
70	0.00605385871338232\\
71	0.00605384855427655\\
72	0.00605383820861819\\
73	0.00605382767292976\\
74	0.00605381694366659\\
75	0.00605380601721534\\
76	0.00605379488989271\\
77	0.00605378355794394\\
78	0.00605377201754145\\
79	0.00605376026478337\\
80	0.00605374829569191\\
81	0.00605373610621198\\
82	0.00605372369220956\\
83	0.00605371104947008\\
84	0.0060536981736968\\
85	0.00605368506050917\\
86	0.00605367170544116\\
87	0.00605365810393945\\
88	0.00605364425136173\\
89	0.00605363014297491\\
90	0.00605361577395319\\
91	0.00605360113937629\\
92	0.00605358623422746\\
93	0.00605357105339158\\
94	0.00605355559165315\\
95	0.00605353984369429\\
96	0.00605352380409261\\
97	0.00605350746731915\\
98	0.00605349082773621\\
99	0.00605347387959511\\
100	0.00605345661703403\\
101	0.00605343903407568\\
102	0.00605342112462494\\
103	0.0060534028824666\\
104	0.00605338430126275\\
105	0.00605336537455043\\
106	0.00605334609573917\\
107	0.00605332645810821\\
108	0.00605330645480415\\
109	0.00605328607883798\\
110	0.00605326532308261\\
111	0.00605324418026994\\
112	0.00605322264298808\\
113	0.00605320070367846\\
114	0.00605317835463289\\
115	0.00605315558799049\\
116	0.00605313239573474\\
117	0.00605310876969023\\
118	0.00605308470151957\\
119	0.00605306018272017\\
120	0.00605303520462078\\
121	0.00605300975837828\\
122	0.00605298383497411\\
123	0.00605295742521092\\
124	0.00605293051970883\\
125	0.00605290310890187\\
126	0.00605287518303431\\
127	0.00605284673215675\\
128	0.00605281774612234\\
129	0.00605278821458291\\
130	0.00605275812698484\\
131	0.00605272747256499\\
132	0.00605269624034661\\
133	0.00605266441913503\\
134	0.00605263199751339\\
135	0.00605259896383815\\
136	0.00605256530623462\\
137	0.00605253101259243\\
138	0.00605249607056078\\
139	0.00605246046754374\\
140	0.00605242419069537\\
141	0.00605238722691476\\
142	0.006052349562841\\
143	0.00605231118484809\\
144	0.00605227207903958\\
145	0.00605223223124333\\
146	0.00605219162700609\\
147	0.00605215025158783\\
148	0.00605210808995623\\
149	0.00605206512678079\\
150	0.00605202134642706\\
151	0.00605197673295059\\
152	0.00605193127009082\\
153	0.0060518849412649\\
154	0.00605183772956136\\
155	0.00605178961773358\\
156	0.00605174058819324\\
157	0.00605169062300365\\
158	0.00605163970387276\\
159	0.00605158781214639\\
160	0.00605153492880093\\
161	0.00605148103443623\\
162	0.00605142610926808\\
163	0.00605137013312077\\
164	0.00605131308541933\\
165	0.00605125494518182\\
166	0.00605119569101119\\
167	0.00605113530108726\\
168	0.00605107375315838\\
169	0.00605101102453295\\
170	0.00605094709207084\\
171	0.00605088193217454\\
172	0.00605081552078026\\
173	0.00605074783334872\\
174	0.00605067884485589\\
175	0.00605060852978345\\
176	0.00605053686210912\\
177	0.00605046381529679\\
178	0.00605038936228641\\
179	0.00605031347548376\\
180	0.00605023612674995\\
181	0.00605015728739076\\
182	0.00605007692814574\\
183	0.00604999501917712\\
184	0.0060499115300585\\
185	0.00604982642976324\\
186	0.00604973968665278\\
187	0.00604965126846458\\
188	0.00604956114229992\\
189	0.00604946927461135\\
190	0.00604937563119001\\
191	0.00604928017715268\\
192	0.00604918287692849\\
193	0.00604908369424541\\
194	0.0060489825921166\\
195	0.00604887953282627\\
196	0.00604877447791537\\
197	0.00604866738816707\\
198	0.0060485582235918\\
199	0.00604844694341212\\
200	0.00604833350604725\\
201	0.00604821786909728\\
202	0.00604809998932707\\
203	0.00604797982264992\\
204	0.00604785732411068\\
205	0.00604773244786887\\
206	0.0060476051471812\\
207	0.00604747537438382\\
208	0.00604734308087434\\
209	0.00604720821709327\\
210	0.00604707073250529\\
211	0.00604693057558015\\
212	0.006046787693773\\
213	0.00604664203350458\\
214	0.00604649354014091\\
215	0.0060463421579725\\
216	0.00604618783019338\\
217	0.00604603049887952\\
218	0.00604587010496691\\
219	0.00604570658822923\\
220	0.00604553988725512\\
221	0.00604536993942488\\
222	0.00604519668088687\\
223	0.0060450200465334\\
224	0.00604483996997613\\
225	0.00604465638352118\\
226	0.00604446921814338\\
227	0.00604427840346052\\
228	0.00604408386770679\\
229	0.00604388553770576\\
230	0.00604368333884301\\
231	0.00604347719503816\\
232	0.00604326702871631\\
233	0.00604305276077907\\
234	0.00604283431057514\\
235	0.00604261159587007\\
236	0.0060423845328158\\
237	0.00604215303591947\\
238	0.00604191701801181\\
239	0.00604167639021485\\
240	0.00604143106190916\\
241	0.00604118094070057\\
242	0.00604092593238629\\
243	0.00604066594092043\\
244	0.00604040086837903\\
245	0.00604013061492462\\
246	0.00603985507877001\\
247	0.00603957415614173\\
248	0.00603928774124293\\
249	0.00603899572621561\\
250	0.00603869800110246\\
251	0.00603839445380817\\
252	0.00603808497006022\\
253	0.00603776943336931\\
254	0.00603744772498924\\
255	0.00603711972387649\\
256	0.00603678530664946\\
257	0.00603644434754738\\
258	0.00603609671838921\\
259	0.00603574228853227\\
260	0.00603538092483121\\
261	0.00603501249159726\\
262	0.00603463685055806\\
263	0.00603425386081876\\
264	0.00603386337882462\\
265	0.00603346525832618\\
266	0.00603305935034801\\
267	0.00603264550316234\\
268	0.00603222356226987\\
269	0.00603179337039027\\
270	0.00603135476746624\\
271	0.0060309075906859\\
272	0.00603045167453047\\
273	0.00602998685085563\\
274	0.00602951294901667\\
275	0.00602902979604564\\
276	0.0060285372168711\\
277	0.00602803503448028\\
278	0.00602752306947831\\
279	0.00602700113593192\\
280	0.00602646903952961\\
281	0.0060259265821394\\
282	0.00602537356172687\\
283	0.00602480977227086\\
284	0.00602423500367617\\
285	0.00602364904168356\\
286	0.00602305166777651\\
287	0.00602244265908472\\
288	0.00602182178828396\\
289	0.00602118882349233\\
290	0.00602054352816206\\
291	0.00601988566096721\\
292	0.00601921497568633\\
293	0.00601853122107978\\
294	0.00601783414076147\\
295	0.00601712347306377\\
296	0.00601639895089535\\
297	0.00601566030159045\\
298	0.0060149072467486\\
299	0.00601413950206287\\
300	0.00601335677713433\\
301	0.00601255877527002\\
302	0.00601174519325995\\
303	0.00601091572112823\\
304	0.00601007004185066\\
305	0.00600920783102938\\
306	0.00600832875651105\\
307	0.00600743247793097\\
308	0.00600651864615927\\
309	0.00600558690261738\\
310	0.00600463687842416\\
311	0.0060036681933215\\
312	0.00600268045432796\\
313	0.00600167325409898\\
314	0.00600064616910949\\
315	0.00599959875827926\\
316	0.00599853056443055\\
317	0.00599744112762684\\
318	0.00599633002043226\\
319	0.00599519680657051\\
320	0.0059940410407279\\
321	0.005992862268351\\
322	0.00599166002543922\\
323	0.00599043383833153\\
324	0.00598918322348752\\
325	0.00598790768726258\\
326	0.00598660672567669\\
327	0.005985279824177\\
328	0.00598392645739348\\
329	0.00598254608888788\\
330	0.00598113817089524\\
331	0.0059797021440581\\
332	0.00597823743715262\\
333	0.00597674346680675\\
334	0.00597521963720975\\
335	0.00597366533981282\\
336	0.0059720799530204\\
337	0.00597046284187164\\
338	0.00596881335771161\\
339	0.00596713083785189\\
340	0.00596541460521975\\
341	0.00596366396799539\\
342	0.00596187821923701\\
343	0.00596005663649259\\
344	0.00595819848139854\\
345	0.00595630299926424\\
346	0.00595436941864167\\
347	0.00595239695087973\\
348	0.00595038478966199\\
349	0.00594833211052729\\
350	0.00594623807037185\\
351	0.00594410180693178\\
352	0.0059419224382442\\
353	0.00593969906208525\\
354	0.00593743075538152\\
355	0.00593511657359166\\
356	0.00593275555005242\\
357	0.00593034669528196\\
358	0.0059278889962312\\
359	0.0059253814154743\\
360	0.00592282289034598\\
361	0.00592021233215843\\
362	0.00591754862594789\\
363	0.00591483062968437\\
364	0.00591205717345138\\
365	0.00590922705859407\\
366	0.0059063390568347\\
367	0.00590339190935374\\
368	0.00590038432583566\\
369	0.00589731498347839\\
370	0.00589418252596617\\
371	0.0058909855624063\\
372	0.0058877226662318\\
373	0.0058843923740743\\
374	0.00588099318461349\\
375	0.0058775235574158\\
376	0.00587398191177944\\
377	0.00587036662561085\\
378	0.00586667603436179\\
379	0.00586290843004274\\
380	0.00585906206024255\\
381	0.00585513512668948\\
382	0.00585112578103346\\
383	0.00584703212292182\\
384	0.0058428521980019\\
385	0.00583858399582633\\
386	0.00583422544765775\\
387	0.00582977442416987\\
388	0.00582522873304119\\
389	0.00582058611644157\\
390	0.00581584424840936\\
391	0.00581100073211716\\
392	0.0058060530970288\\
393	0.00580099879595051\\
394	0.00579583520198189\\
395	0.00579055960537552\\
396	0.00578516921031676\\
397	0.00577966113164156\\
398	0.00577403239152266\\
399	0.0057682799161579\\
400	0.00576240053247438\\
401	0.00575639096487995\\
402	0.00575024783207291\\
403	0.00574396764398991\\
404	0.00573754679892206\\
405	0.00573098158081809\\
406	0.00572426815680069\\
407	0.00571740257477954\\
408	0.0057103807610439\\
409	0.00570319851758423\\
410	0.00569585151864958\\
411	0.00568833530593727\\
412	0.00568064528179326\\
413	0.00567277669913962\\
414	0.00566472464014436\\
415	0.00565648400799689\\
416	0.00564804951792851\\
417	0.00563941568793487\\
418	0.00563057682899589\\
419	0.0056215270346997\\
420	0.00561226017019854\\
421	0.00560276986037648\\
422	0.00559304947721987\\
423	0.00558309212656227\\
424	0.00557289063323799\\
425	0.00556243752687089\\
426	0.00555172502862437\\
427	0.00554074503770892\\
428	0.00552948911145671\\
429	0.00551794844589325\\
430	0.00550611385673614\\
431	0.00549397575973935\\
432	0.00548152415026784\\
433	0.00546874858210162\\
434	0.00545563814522781\\
435	0.00544218144097815\\
436	0.00542836654737445\\
437	0.00541418098913318\\
438	0.00539961173926262\\
439	0.00538464523869321\\
440	0.0053692673161546\\
441	0.00535346313975109\\
442	0.0053372171746004\\
443	0.00532051313740289\\
444	0.00530333394848065\\
445	0.00528566168259585\\
446	0.0052674775156372\\
447	0.00524876156771987\\
448	0.00522949276320377\\
449	0.00520964896449091\\
450	0.00518920757373021\\
451	0.00516814511954083\\
452	0.00514643720112049\\
453	0.00512405864723791\\
454	0.00510098378576996\\
455	0.00507718651561285\\
456	0.00505264043758917\\
457	0.00502731926123773\\
458	0.00500119338249417\\
459	0.00497423113914844\\
460	0.00494640367555496\\
461	0.00491768390259255\\
462	0.00488804416224272\\
463	0.00485745557444009\\
464	0.00482588695608326\\
465	0.00479330279722531\\
466	0.0047596576280196\\
467	0.00472489779996188\\
468	0.00468897354242522\\
469	0.0046518417927839\\
470	0.00461347006042861\\
471	0.00457384167076456\\
472	0.00453296284521914\\
473	0.0044908721285255\\
474	0.0044476528958311\\
475	0.00440344963230884\\
476	0.00435848950460793\\
477	0.00431330090536759\\
478	0.00426911901941018\\
479	0.00422609131441622\\
480	0.0041843734761192\\
481	0.00414412701115376\\
482	0.00410551496252874\\
483	0.00406869694891489\\
484	0.00403382195664205\\
485	0.00400101794806777\\
486	0.0039703749638837\\
487	0.00394192340717972\\
488	0.00391560420247279\\
489	0.00389078961723187\\
490	0.00386655392115217\\
491	0.00384289551839856\\
492	0.00381980222562067\\
493	0.00379724964136457\\
494	0.00377519960045295\\
495	0.00375359895964029\\
496	0.00373237900858465\\
497	0.0037114559002514\\
498	0.00369073269417876\\
499	0.00367010384999713\\
500	0.00364946334530463\\
501	0.00362877905153483\\
502	0.00360803028005762\\
503	0.00358719451586058\\
504	0.00356624782116794\\
505	0.00354516538132623\\
506	0.00352392219824332\\
507	0.00350249392209815\\
508	0.00348085778748331\\
509	0.00345899358143589\\
510	0.00343688451126113\\
511	0.00341451775218712\\
512	0.0033918824440211\\
513	0.0033689676683681\\
514	0.00334576254116211\\
515	0.00332225629935928\\
516	0.00329843837371602\\
517	0.00327429843792487\\
518	0.0032498264231834\\
519	0.00322501248710322\\
520	0.00319984692763712\\
521	0.00317432003781731\\
522	0.00314842190777655\\
523	0.00312214225160906\\
524	0.00309547037273244\\
525	0.00306839512133135\\
526	0.00304090484325028\\
527	0.00301298732007266\\
528	0.00298462970066008\\
529	0.00295581842514291\\
530	0.00292653914324031\\
531	0.00289677662978668\\
532	0.00286651470130723\\
533	0.00283573613813255\\
534	0.00280442261571909\\
535	0.00277255463848415\\
536	0.00274011147920116\\
537	0.00270707112800494\\
538	0.0026734102563127\\
539	0.00263910420253288\\
540	0.00260412698838064\\
541	0.00256845137703628\\
542	0.00253204898738828\\
543	0.00249489048234822\\
544	0.00245694585391358\\
545	0.00241818483357086\\
546	0.00237857746439831\\
547	0.00233809488657614\\
548	0.00229671045645981\\
549	0.00225441111083414\\
550	0.00221118503939151\\
551	0.00216702202657525\\
552	0.00212191382248228\\
553	0.0020758545648019\\
554	0.00202884126536382\\
555	0.00198087510715297\\
556	0.00193196503358525\\
557	0.00188213043144942\\
558	0.00183148136505412\\
559	0.00177997739204913\\
560	0.00172755807796478\\
561	0.00167414569304286\\
562	0.00161956559511978\\
563	0.00156366756377305\\
564	0.00150631068618296\\
565	0.00144923707526241\\
566	0.00139360112641226\\
567	0.00133997155536227\\
568	0.00128796297954385\\
569	0.00123625110336431\\
570	0.0011849967424863\\
571	0.00113500063362126\\
572	0.00108630229033245\\
573	0.00103861044929871\\
574	0.000991302384123051\\
575	0.000944334643980418\\
576	0.00089771161392153\\
577	0.000851824028850601\\
578	0.000806587975280099\\
579	0.000761530778505787\\
580	0.000716607563069663\\
581	0.000671842104011154\\
582	0.000627277405667915\\
583	0.000582965786062491\\
584	0.000538962689738965\\
585	0.000495324737689842\\
586	0.000452108368453091\\
587	0.000409368226051223\\
588	0.000367155166688129\\
589	0.000325513960353391\\
590	0.000284480716949123\\
591	0.000244080684870838\\
592	0.000204328087251262\\
593	0.000165233217676712\\
594	0.000126870159937009\\
595	8.96111722547531e-05\\
596	5.42660945238512e-05\\
597	2.28062284332062e-05\\
598	2.9204464504877e-07\\
599	0\\
600	0\\
};
\addplot [color=red!80!mycolor19,solid,forget plot]
  table[row sep=crcr]{%
1	0.00604379698173766\\
2	0.00604379373848888\\
3	0.0060437904349912\\
4	0.00604378707016116\\
5	0.00604378364289623\\
6	0.00604378015207447\\
7	0.0060437765965541\\
8	0.00604377297517326\\
9	0.0060437692867496\\
10	0.00604376553007986\\
11	0.00604376170393957\\
12	0.00604375780708261\\
13	0.0060437538382409\\
14	0.00604374979612386\\
15	0.00604374567941812\\
16	0.00604374148678706\\
17	0.00604373721687039\\
18	0.00604373286828375\\
19	0.00604372843961818\\
20	0.00604372392943977\\
21	0.00604371933628912\\
22	0.00604371465868094\\
23	0.00604370989510351\\
24	0.0060437050440182\\
25	0.00604370010385904\\
26	0.0060436950730321\\
27	0.00604368994991509\\
28	0.00604368473285671\\
29	0.0060436794201762\\
30	0.00604367401016273\\
31	0.00604366850107489\\
32	0.00604366289114001\\
33	0.00604365717855372\\
34	0.00604365136147922\\
35	0.00604364543804676\\
36	0.00604363940635292\\
37	0.00604363326446007\\
38	0.00604362701039563\\
39	0.00604362064215146\\
40	0.00604361415768315\\
41	0.00604360755490934\\
42	0.00604360083171102\\
43	0.00604359398593074\\
44	0.00604358701537194\\
45	0.00604357991779815\\
46	0.00604357269093224\\
47	0.00604356533245563\\
48	0.00604355784000749\\
49	0.00604355021118384\\
50	0.00604354244353681\\
51	0.00604353453457374\\
52	0.00604352648175627\\
53	0.00604351828249949\\
54	0.00604350993417095\\
55	0.00604350143408981\\
56	0.00604349277952583\\
57	0.00604348396769837\\
58	0.00604347499577549\\
59	0.00604346586087274\\
60	0.00604345656005232\\
61	0.0060434470903219\\
62	0.00604343744863352\\
63	0.00604342763188252\\
64	0.00604341763690638\\
65	0.00604340746048358\\
66	0.00604339709933237\\
67	0.00604338655010954\\
68	0.00604337580940924\\
69	0.00604336487376163\\
70	0.0060433537396317\\
71	0.00604334240341784\\
72	0.00604333086145045\\
73	0.00604331910999072\\
74	0.00604330714522906\\
75	0.00604329496328367\\
76	0.00604328256019913\\
77	0.00604326993194488\\
78	0.00604325707441361\\
79	0.00604324398341973\\
80	0.00604323065469773\\
81	0.00604321708390063\\
82	0.00604320326659809\\
83	0.00604318919827488\\
84	0.00604317487432911\\
85	0.0060431602900703\\
86	0.00604314544071769\\
87	0.00604313032139822\\
88	0.00604311492714479\\
89	0.00604309925289413\\
90	0.00604308329348493\\
91	0.00604306704365572\\
92	0.00604305049804286\\
93	0.00604303365117828\\
94	0.00604301649748742\\
95	0.00604299903128696\\
96	0.00604298124678256\\
97	0.0060429631380665\\
98	0.00604294469911541\\
99	0.00604292592378771\\
100	0.00604290680582129\\
101	0.00604288733883078\\
102	0.00604286751630521\\
103	0.00604284733160519\\
104	0.00604282677796039\\
105	0.00604280584846656\\
106	0.00604278453608295\\
107	0.00604276283362935\\
108	0.00604274073378318\\
109	0.0060427182290765\\
110	0.00604269531189297\\
111	0.00604267197446476\\
112	0.00604264820886932\\
113	0.00604262400702621\\
114	0.0060425993606937\\
115	0.00604257426146556\\
116	0.00604254870076742\\
117	0.00604252266985342\\
118	0.00604249615980252\\
119	0.00604246916151483\\
120	0.00604244166570795\\
121	0.00604241366291311\\
122	0.00604238514347133\\
123	0.00604235609752938\\
124	0.00604232651503577\\
125	0.00604229638573664\\
126	0.00604226569917148\\
127	0.00604223444466896\\
128	0.00604220261134241\\
129	0.0060421701880854\\
130	0.00604213716356718\\
131	0.00604210352622806\\
132	0.00604206926427466\\
133	0.00604203436567501\\
134	0.00604199881815365\\
135	0.0060419626091866\\
136	0.00604192572599635\\
137	0.00604188815554639\\
138	0.00604184988453604\\
139	0.00604181089939501\\
140	0.00604177118627776\\
141	0.00604173073105795\\
142	0.0060416895193226\\
143	0.00604164753636629\\
144	0.00604160476718508\\
145	0.0060415611964705\\
146	0.0060415168086032\\
147	0.00604147158764674\\
148	0.00604142551734103\\
149	0.00604137858109577\\
150	0.00604133076198368\\
151	0.00604128204273377\\
152	0.0060412324057242\\
153	0.00604118183297524\\
154	0.00604113030614201\\
155	0.00604107780650709\\
156	0.00604102431497294\\
157	0.00604096981205427\\
158	0.00604091427787014\\
159	0.00604085769213601\\
160	0.00604080003415563\\
161	0.00604074128281264\\
162	0.00604068141656227\\
163	0.00604062041342257\\
164	0.00604055825096566\\
165	0.00604049490630886\\
166	0.0060404303561054\\
167	0.00604036457653526\\
168	0.00604029754329556\\
169	0.00604022923159099\\
170	0.00604015961612381\\
171	0.00604008867108396\\
172	0.0060400163701386\\
173	0.00603994268642189\\
174	0.00603986759252411\\
175	0.00603979106048093\\
176	0.00603971306176223\\
177	0.00603963356726089\\
178	0.00603955254728118\\
179	0.00603946997152703\\
180	0.00603938580909002\\
181	0.00603930002843719\\
182	0.00603921259739859\\
183	0.00603912348315443\\
184	0.00603903265222221\\
185	0.00603894007044345\\
186	0.0060388457029702\\
187	0.0060387495142512\\
188	0.00603865146801789\\
189	0.00603855152727004\\
190	0.00603844965426113\\
191	0.00603834581048335\\
192	0.00603823995665241\\
193	0.00603813205269206\\
194	0.00603802205771807\\
195	0.00603790993002215\\
196	0.00603779562705542\\
197	0.00603767910541146\\
198	0.00603756032080915\\
199	0.00603743922807512\\
200	0.00603731578112576\\
201	0.00603718993294888\\
202	0.00603706163558518\\
203	0.00603693084010897\\
204	0.00603679749660884\\
205	0.00603666155416769\\
206	0.00603652296084248\\
207	0.00603638166364351\\
208	0.00603623760851322\\
209	0.00603609074030466\\
210	0.00603594100275937\\
211	0.00603578833848494\\
212	0.00603563268893194\\
213	0.00603547399437057\\
214	0.00603531219386657\\
215	0.00603514722525689\\
216	0.00603497902512458\\
217	0.00603480752877349\\
218	0.006034632670202\\
219	0.00603445438207663\\
220	0.00603427259570483\\
221	0.00603408724100724\\
222	0.00603389824648951\\
223	0.00603370553921332\\
224	0.00603350904476698\\
225	0.0060333086872352\\
226	0.00603310438916856\\
227	0.00603289607155193\\
228	0.00603268365377257\\
229	0.00603246705358736\\
230	0.00603224618708944\\
231	0.00603202096867407\\
232	0.00603179131100386\\
233	0.00603155712497328\\
234	0.00603131831967223\\
235	0.00603107480234912\\
236	0.00603082647837303\\
237	0.00603057325119501\\
238	0.00603031502230869\\
239	0.00603005169120995\\
240	0.00602978315535595\\
241	0.00602950931012283\\
242	0.00602923004876309\\
243	0.00602894526236156\\
244	0.00602865483979086\\
245	0.00602835866766542\\
246	0.00602805663029497\\
247	0.00602774860963676\\
248	0.00602743448524672\\
249	0.00602711413422966\\
250	0.00602678743118839\\
251	0.00602645424817156\\
252	0.00602611445462051\\
253	0.00602576791731476\\
254	0.00602541450031635\\
255	0.00602505406491296\\
256	0.00602468646955961\\
257	0.00602431156981905\\
258	0.00602392921830078\\
259	0.0060235392645986\\
260	0.00602314155522674\\
261	0.00602273593355424\\
262	0.0060223222397381\\
263	0.00602190031065444\\
264	0.00602146997982814\\
265	0.00602103107736065\\
266	0.00602058342985596\\
267	0.00602012686034462\\
268	0.0060196611882056\\
269	0.00601918622908599\\
270	0.00601870179481818\\
271	0.00601820769333418\\
272	0.0060177037285767\\
273	0.00601718970040628\\
274	0.00601666540450337\\
275	0.00601613063226448\\
276	0.00601558517069031\\
277	0.00601502880226538\\
278	0.00601446130483493\\
279	0.00601388245153339\\
280	0.00601329201074085\\
281	0.00601268974597786\\
282	0.00601207541579783\\
283	0.00601144877367671\\
284	0.00601080956790019\\
285	0.0060101575414481\\
286	0.00600949243187622\\
287	0.00600881397119506\\
288	0.006008121885746\\
289	0.00600741589607442\\
290	0.00600669571679988\\
291	0.00600596105648334\\
292	0.00600521161749125\\
293	0.00600444709585661\\
294	0.00600366718113684\\
295	0.00600287155626848\\
296	0.00600205989741867\\
297	0.00600123187383337\\
298	0.00600038714768226\\
299	0.00599952537390029\\
300	0.00599864620002601\\
301	0.00599774926603647\\
302	0.0059968342041787\\
303	0.00599590063879805\\
304	0.00599494818616345\\
305	0.00599397645428955\\
306	0.00599298504275645\\
307	0.00599197354252766\\
308	0.00599094153576698\\
309	0.00598988859565647\\
310	0.0059888142862175\\
311	0.00598771816213875\\
312	0.00598659976861655\\
313	0.005985458641214\\
314	0.00598429430574684\\
315	0.00598310627819779\\
316	0.00598189406463208\\
317	0.00598065716095028\\
318	0.00597939505211107\\
319	0.00597810721187351\\
320	0.00597679310253201\\
321	0.0059754521746443\\
322	0.00597408386675148\\
323	0.00597268760509062\\
324	0.0059712628032988\\
325	0.00596980886210919\\
326	0.00596832516903809\\
327	0.00596681109806301\\
328	0.00596526600929152\\
329	0.00596368924862003\\
330	0.00596208014738282\\
331	0.00596043802199029\\
332	0.00595876217355639\\
333	0.00595705188751468\\
334	0.0059553064332226\\
335	0.00595352506355342\\
336	0.00595170701447526\\
337	0.00594985150461702\\
338	0.00594795773482025\\
339	0.00594602488767673\\
340	0.00594405212705106\\
341	0.00594203859758758\\
342	0.00593998342420066\\
343	0.00593788571154831\\
344	0.00593574454348784\\
345	0.00593355898251308\\
346	0.00593132806917237\\
347	0.00592905082146658\\
348	0.00592672623422611\\
349	0.00592435327846604\\
350	0.0059219309007185\\
351	0.005919458022341\\
352	0.00591693353879988\\
353	0.00591435631892751\\
354	0.00591172520415271\\
355	0.00590903900770255\\
356	0.00590629651377466\\
357	0.00590349647667853\\
358	0.00590063761994472\\
359	0.00589771863540054\\
360	0.00589473818221175\\
361	0.00589169488589031\\
362	0.00588858733725277\\
363	0.00588541409133603\\
364	0.00588217366626758\\
365	0.00587886454208901\\
366	0.00587548515953178\\
367	0.00587203391874112\\
368	0.00586850917794364\\
369	0.00586490925205651\\
370	0.00586123241123398\\
371	0.00585747687934691\\
372	0.00585364083239288\\
373	0.00584972239683445\\
374	0.00584571964786044\\
375	0.00584163060756524\\
376	0.00583745324304039\\
377	0.00583318546437371\\
378	0.00582882512254872\\
379	0.00582437000723769\\
380	0.00581981784447777\\
381	0.00581516629421113\\
382	0.00581041294774655\\
383	0.0058055553250961\\
384	0.00580059087218987\\
385	0.0057955169579579\\
386	0.00579033087126014\\
387	0.00578502981765638\\
388	0.00577961091601375\\
389	0.00577407119492913\\
390	0.00576840758897508\\
391	0.00576261693475519\\
392	0.00575669596675555\\
393	0.00575064131297993\\
394	0.00574444949035536\\
395	0.00573811689989189\\
396	0.00573163982157519\\
397	0.0057250144089629\\
398	0.00571823668345897\\
399	0.00571130252832844\\
400	0.00570420768255797\\
401	0.00569694773434462\\
402	0.00568951811419927\\
403	0.00568191408721808\\
404	0.00567413074496403\\
405	0.00566616299687093\\
406	0.00565800556104346\\
407	0.00564965295490358\\
408	0.00564109948494235\\
409	0.00563233923572894\\
410	0.00562336605821633\\
411	0.00561417355720179\\
412	0.00560475507766134\\
413	0.00559510369100685\\
414	0.005585212183069\\
415	0.0055750730417266\\
416	0.00556467844256432\\
417	0.00555402023031086\\
418	0.00554308990211027\\
419	0.00553187859084382\\
420	0.00552037704796187\\
421	0.00550857562560085\\
422	0.00549646425690355\\
423	0.00548403243420698\\
424	0.00547126918874049\\
425	0.00545816305475796\\
426	0.00544470204193996\\
427	0.00543087362543778\\
428	0.00541666475211721\\
429	0.00540206177521736\\
430	0.00538705039846046\\
431	0.00537161563139494\\
432	0.00535574174438478\\
433	0.00533941222309604\\
434	0.00532260972527357\\
435	0.00530531604468355\\
436	0.0052875120838388\\
437	0.00526917772433\\
438	0.00525029172219893\\
439	0.00523083201498075\\
440	0.00521077665230321\\
441	0.00519010343163706\\
442	0.00516878988349136\\
443	0.00514681334450342\\
444	0.00512415101997684\\
445	0.00510078002476416\\
446	0.00507667740522985\\
447	0.00505182028960638\\
448	0.00502618457628976\\
449	0.00499974289798137\\
450	0.00497246427710687\\
451	0.00494432117897287\\
452	0.00491528257570121\\
453	0.00488530972768461\\
454	0.00485435170143604\\
455	0.00482235931248683\\
456	0.00478928719534402\\
457	0.00475509663674163\\
458	0.00471975949187953\\
459	0.00468326338633841\\
460	0.00464561844076083\\
461	0.00460686637231825\\
462	0.00456709256807714\\
463	0.00452644208904394\\
464	0.00448514072074695\\
465	0.00444389923794743\\
466	0.00440357687039676\\
467	0.00436430832544291\\
468	0.00432623602995577\\
469	0.00428950815751473\\
470	0.00425427567283625\\
471	0.00422068738185323\\
472	0.00418888285783103\\
473	0.00415898311876477\\
474	0.00413107750704409\\
475	0.00410520609121975\\
476	0.00408133284316304\\
477	0.00405911043870122\\
478	0.00403743665823675\\
479	0.0040163131829262\\
480	0.00399573266567172\\
481	0.00397567727603209\\
482	0.00395611732407331\\
483	0.00393701007397081\\
484	0.00391829896134864\\
485	0.00389991353540765\\
486	0.00388177063464587\\
487	0.00386377740380242\\
488	0.00384583705763168\\
489	0.00382788266290682\\
490	0.00380989663293803\\
491	0.00379185963450906\\
492	0.00377375088435137\\
493	0.00375554856466038\\
494	0.00373723036853259\\
495	0.00371877417539081\\
496	0.00370015884067846\\
497	0.00368136505947589\\
498	0.00366237622430955\\
499	0.00364317913778042\\
500	0.00362376435192982\\
501	0.0036041230010163\\
502	0.00358424630052364\\
503	0.00356412563595359\\
504	0.00354375264480726\\
505	0.00352311928443311\\
506	0.00350221787714692\\
507	0.00348104112326624\\
508	0.00345958207301436\\
509	0.00343783405051128\\
510	0.0034157905286329\\
511	0.00339344496435606\\
512	0.0033707907054899\\
513	0.0033478209780482\\
514	0.00332452886784109\\
515	0.00330090729577894\\
516	0.00327694898667039\\
517	0.00325264643170242\\
518	0.00322799184531334\\
519	0.00320297711777891\\
520	0.00317759376545378\\
521	0.00315183288108833\\
522	0.00312568508669098\\
523	0.00309914048826814\\
524	0.00307218862662699\\
525	0.00304481842431517\\
526	0.00301701812888583\\
527	0.00298877525281087\\
528	0.00296007651051954\\
529	0.00293090775321448\\
530	0.00290125390231508\\
531	0.00287109888260652\\
532	0.00284042555644818\\
533	0.00280921566075377\\
534	0.00277744974899561\\
535	0.002745107141692\\
536	0.00271216588991355\\
537	0.00267860275770603\\
538	0.00264439323105173\\
539	0.00260951156316653\\
540	0.00257393086867269\\
541	0.00253762328263359\\
542	0.00250056020476996\\
543	0.00246271265464424\\
544	0.00242405177078421\\
545	0.00238454949588862\\
546	0.00234417950976365\\
547	0.00230291826941214\\
548	0.00226075729913756\\
549	0.00221769056880834\\
550	0.00217371502577192\\
551	0.00212883183532615\\
552	0.00208304983956979\\
553	0.00203638997391316\\
554	0.00198896231752856\\
555	0.00194072070159704\\
556	0.00189160323636402\\
557	0.00184153124543727\\
558	0.0017902977245096\\
559	0.00173782298923212\\
560	0.00168399686198448\\
561	0.0016286614295943\\
562	0.00157372532456192\\
563	0.00152029620686089\\
564	0.00146903631266427\\
565	0.0014186652881098\\
566	0.00136848386464934\\
567	0.00131849638653166\\
568	0.00126901022566881\\
569	0.00122069256463649\\
570	0.0011736084075615\\
571	0.00112676679780215\\
572	0.00108013609015035\\
573	0.00103366236371688\\
574	0.000987377889550623\\
575	0.000941691633156011\\
576	0.000896566979252233\\
577	0.000851509644641342\\
578	0.000806470720006763\\
579	0.0007614808141167\\
580	0.000716583918777049\\
581	0.000671829456087438\\
582	0.000627270354366442\\
583	0.000582961784628753\\
584	0.000538960396115172\\
585	0.000495323454366403\\
586	0.000452107712405483\\
587	0.000409367941602314\\
588	0.000367155075774456\\
589	0.0003255139412327\\
590	0.000284480716949123\\
591	0.000244080684870837\\
592	0.00020432808725126\\
593	0.00016523321767671\\
594	0.000126870159937007\\
595	8.96111722547515e-05\\
596	5.42660945238509e-05\\
597	2.28062284332055e-05\\
598	2.9204464504877e-07\\
599	0\\
600	0\\
};
\addplot [color=red,solid,forget plot]
  table[row sep=crcr]{%
1	0.00604132329944266\\
2	0.00604131925264746\\
3	0.00604131512902119\\
4	0.00604131092717107\\
5	0.00604130664568032\\
6	0.00604130228310772\\
7	0.00604129783798725\\
8	0.00604129330882765\\
9	0.00604128869411197\\
10	0.00604128399229723\\
11	0.00604127920181386\\
12	0.00604127432106537\\
13	0.0060412693484278\\
14	0.00604126428224932\\
15	0.00604125912084976\\
16	0.00604125386252011\\
17	0.00604124850552209\\
18	0.0060412430480875\\
19	0.00604123748841791\\
20	0.00604123182468409\\
21	0.00604122605502535\\
22	0.0060412201775492\\
23	0.0060412141903307\\
24	0.00604120809141197\\
25	0.00604120187880152\\
26	0.00604119555047378\\
27	0.00604118910436851\\
28	0.00604118253839014\\
29	0.0060411758504072\\
30	0.00604116903825171\\
31	0.00604116209971853\\
32	0.00604115503256479\\
33	0.00604114783450912\\
34	0.00604114050323102\\
35	0.00604113303637021\\
36	0.00604112543152595\\
37	0.00604111768625627\\
38	0.0060411097980773\\
39	0.00604110176446248\\
40	0.00604109358284187\\
41	0.00604108525060133\\
42	0.00604107676508178\\
43	0.00604106812357833\\
44	0.00604105932333964\\
45	0.00604105036156682\\
46	0.00604104123541281\\
47	0.00604103194198142\\
48	0.00604102247832642\\
49	0.00604101284145067\\
50	0.00604100302830524\\
51	0.0060409930357884\\
52	0.00604098286074466\\
53	0.00604097249996382\\
54	0.00604096195017998\\
55	0.0060409512080705\\
56	0.00604094027025489\\
57	0.00604092913329385\\
58	0.00604091779368803\\
59	0.00604090624787712\\
60	0.00604089449223851\\
61	0.00604088252308624\\
62	0.00604087033666972\\
63	0.00604085792917265\\
64	0.00604084529671161\\
65	0.00604083243533483\\
66	0.00604081934102101\\
67	0.00604080600967785\\
68	0.00604079243714074\\
69	0.00604077861917138\\
70	0.00604076455145631\\
71	0.00604075022960549\\
72	0.00604073564915083\\
73	0.00604072080554458\\
74	0.00604070569415793\\
75	0.00604069031027922\\
76	0.00604067464911248\\
77	0.00604065870577569\\
78	0.00604064247529903\\
79	0.0060406259526232\\
80	0.0060406091325977\\
81	0.00604059200997878\\
82	0.00604057457942783\\
83	0.00604055683550932\\
84	0.00604053877268885\\
85	0.00604052038533122\\
86	0.00604050166769827\\
87	0.00604048261394694\\
88	0.00604046321812696\\
89	0.00604044347417879\\
90	0.0060404233759313\\
91	0.00604040291709949\\
92	0.00604038209128214\\
93	0.00604036089195942\\
94	0.00604033931249039\\
95	0.00604031734611052\\
96	0.00604029498592913\\
97	0.00604027222492669\\
98	0.00604024905595212\\
99	0.00604022547172015\\
100	0.0060402014648083\\
101	0.00604017702765421\\
102	0.0060401521525525\\
103	0.00604012683165182\\
104	0.00604010105695174\\
105	0.00604007482029962\\
106	0.00604004811338731\\
107	0.0060400209277479\\
108	0.00603999325475223\\
109	0.00603996508560554\\
110	0.00603993641134386\\
111	0.0060399072228304\\
112	0.00603987751075174\\
113	0.00603984726561425\\
114	0.00603981647773998\\
115	0.00603978513726282\\
116	0.00603975323412439\\
117	0.00603972075806994\\
118	0.00603968769864406\\
119	0.00603965404518634\\
120	0.00603961978682697\\
121	0.00603958491248219\\
122	0.00603954941084964\\
123	0.00603951327040361\\
124	0.00603947647939026\\
125	0.00603943902582262\\
126	0.00603940089747556\\
127	0.0060393620818806\\
128	0.00603932256632063\\
129	0.00603928233782457\\
130	0.00603924138316183\\
131	0.00603919968883666\\
132	0.00603915724108239\\
133	0.00603911402585568\\
134	0.00603907002883038\\
135	0.00603902523539151\\
136	0.00603897963062892\\
137	0.00603893319933103\\
138	0.00603888592597824\\
139	0.00603883779473625\\
140	0.0060387887894494\\
141	0.00603873889363357\\
142	0.00603868809046935\\
143	0.00603863636279456\\
144	0.0060385836930971\\
145	0.00603853006350736\\
146	0.00603847545579058\\
147	0.00603841985133902\\
148	0.00603836323116402\\
149	0.00603830557588792\\
150	0.00603824686573561\\
151	0.00603818708052626\\
152	0.00603812619966465\\
153	0.0060380642021324\\
154	0.00603800106647893\\
155	0.00603793677081245\\
156	0.00603787129279056\\
157	0.00603780460961084\\
158	0.00603773669800116\\
159	0.00603766753420982\\
160	0.00603759709399551\\
161	0.00603752535261709\\
162	0.00603745228482317\\
163	0.00603737786484147\\
164	0.00603730206636806\\
165	0.00603722486255624\\
166	0.00603714622600539\\
167	0.00603706612874949\\
168	0.00603698454224547\\
169	0.0060369014373613\\
170	0.00603681678436404\\
171	0.00603673055290728\\
172	0.0060366427120188\\
173	0.00603655323008767\\
174	0.00603646207485128\\
175	0.00603636921338198\\
176	0.00603627461207375\\
177	0.00603617823662822\\
178	0.00603608005204077\\
179	0.00603598002258623\\
180	0.00603587811180431\\
181	0.00603577428248475\\
182	0.00603566849665226\\
183	0.00603556071555106\\
184	0.00603545089962927\\
185	0.00603533900852289\\
186	0.00603522500103951\\
187	0.00603510883514169\\
188	0.00603499046793007\\
189	0.00603486985562615\\
190	0.00603474695355463\\
191	0.00603462171612562\\
192	0.00603449409681624\\
193	0.00603436404815203\\
194	0.00603423152168799\\
195	0.00603409646798919\\
196	0.00603395883661098\\
197	0.00603381857607889\\
198	0.00603367563386799\\
199	0.00603352995638198\\
200	0.00603338148893178\\
201	0.00603323017571368\\
202	0.006033075959787\\
203	0.00603291878305142\\
204	0.00603275858622366\\
205	0.00603259530881392\\
206	0.00603242888910147\\
207	0.00603225926411005\\
208	0.00603208636958272\\
209	0.00603191013995585\\
210	0.006031730508333\\
211	0.00603154740645792\\
212	0.00603136076468708\\
213	0.00603117051196166\\
214	0.00603097657577886\\
215	0.00603077888216255\\
216	0.0060305773556334\\
217	0.00603037191917825\\
218	0.00603016249421888\\
219	0.00602994900058007\\
220	0.0060297313564569\\
221	0.0060295094783815\\
222	0.00602928328118882\\
223	0.00602905267798188\\
224	0.00602881758009614\\
225	0.00602857789706305\\
226	0.00602833353657291\\
227	0.00602808440443683\\
228	0.00602783040454781\\
229	0.00602757143884117\\
230	0.0060273074072538\\
231	0.00602703820768281\\
232	0.00602676373594302\\
233	0.00602648388572377\\
234	0.00602619854854449\\
235	0.00602590761370964\\
236	0.00602561096826228\\
237	0.00602530849693701\\
238	0.00602500008211152\\
239	0.00602468560375743\\
240	0.00602436493938971\\
241	0.00602403796401531\\
242	0.0060237045500804\\
243	0.0060233645674167\\
244	0.0060230178831864\\
245	0.00602266436182607\\
246	0.00602230386498922\\
247	0.00602193625148769\\
248	0.00602156137723161\\
249	0.00602117909516823\\
250	0.00602078925521923\\
251	0.00602039170421677\\
252	0.00601998628583804\\
253	0.00601957284053832\\
254	0.00601915120548276\\
255	0.00601872121447631\\
256	0.00601828269789236\\
257	0.00601783548259966\\
258	0.00601737939188752\\
259	0.0060169142453895\\
260	0.00601643985900513\\
261	0.00601595604482015\\
262	0.00601546261102462\\
263	0.00601495936182939\\
264	0.00601444609738052\\
265	0.00601392261367178\\
266	0.00601338870245519\\
267	0.00601284415114934\\
268	0.00601228874274578\\
269	0.00601172225571316\\
270	0.0060111444638991\\
271	0.00601055513642994\\
272	0.00600995403760808\\
273	0.00600934092680692\\
274	0.00600871555836368\\
275	0.00600807768146947\\
276	0.00600742704005722\\
277	0.00600676337268714\\
278	0.0060060864124301\\
279	0.00600539588674823\\
280	0.00600469151737163\\
281	0.00600397302017226\\
282	0.00600324010503467\\
283	0.00600249247572393\\
284	0.00600172982975035\\
285	0.00600095185823096\\
286	0.00600015824574771\\
287	0.00599934867020256\\
288	0.00599852280266881\\
289	0.00599768030723909\\
290	0.00599682084086989\\
291	0.00599594405322201\\
292	0.00599504958649758\\
293	0.00599413707527296\\
294	0.00599320614632769\\
295	0.00599225641846931\\
296	0.0059912875023539\\
297	0.00599029900030216\\
298	0.00598929050611112\\
299	0.0059882616048609\\
300	0.0059872118727169\\
301	0.00598614087672656\\
302	0.00598504817461133\\
303	0.00598393331455281\\
304	0.00598279583497354\\
305	0.00598163526431188\\
306	0.00598045112079092\\
307	0.00597924291218078\\
308	0.00597801013555466\\
309	0.00597675227703763\\
310	0.00597546881154838\\
311	0.00597415920253324\\
312	0.00597282290169219\\
313	0.00597145934869616\\
314	0.00597006797089525\\
315	0.00596864818301673\\
316	0.00596719938685158\\
317	0.00596572097092812\\
318	0.00596421231018518\\
319	0.00596267276563555\\
320	0.00596110168401903\\
321	0.00595949839744503\\
322	0.00595786222302412\\
323	0.00595619246248798\\
324	0.00595448840179789\\
325	0.00595274931074066\\
326	0.00595097444251194\\
327	0.00594916303328601\\
328	0.00594731430177211\\
329	0.00594542744875613\\
330	0.00594350165662732\\
331	0.00594153608888943\\
332	0.00593952988965533\\
333	0.00593748218312489\\
334	0.00593539207304499\\
335	0.00593325864215104\\
336	0.00593108095158899\\
337	0.00592885804031686\\
338	0.0059265889244846\\
339	0.00592427259679165\\
340	0.00592190802582186\\
341	0.00591949415535425\\
342	0.00591702990364704\\
343	0.00591451416269395\\
344	0.00591194579745183\\
345	0.00590932364503768\\
346	0.00590664651389399\\
347	0.00590391318292038\\
348	0.00590112240057143\\
349	0.00589827288391688\\
350	0.005895363317663\\
351	0.00589239235313299\\
352	0.00588935860720373\\
353	0.00588626066119526\\
354	0.00588309705971205\\
355	0.00587986630943471\\
356	0.00587656687785905\\
357	0.00587319719197867\\
358	0.00586975563690877\\
359	0.0058662405544477\\
360	0.00586265024157304\\
361	0.00585898294886799\\
362	0.00585523687887468\\
363	0.00585141018436922\\
364	0.00584750096655261\\
365	0.00584350727315025\\
366	0.00583942709641713\\
367	0.00583525837107253\\
368	0.0058309989721399\\
369	0.0058266467126732\\
370	0.00582219934137166\\
371	0.00581765454007395\\
372	0.00581300992110441\\
373	0.00580826302448079\\
374	0.00580341131499454\\
375	0.00579845217914324\\
376	0.00579338292190432\\
377	0.00578820076334026\\
378	0.00578290283502511\\
379	0.00577748617628099\\
380	0.0057719477302107\\
381	0.0057662843395138\\
382	0.00576049274206254\\
383	0.00575456956620739\\
384	0.00574851132579433\\
385	0.00574231441499663\\
386	0.00573597510292741\\
387	0.00572948952787593\\
388	0.00572285369113555\\
389	0.00571606345047797\\
390	0.00570911451306406\\
391	0.00570200242799517\\
392	0.00569472257847336\\
393	0.0056872701735383\\
394	0.0056796402393677\\
395	0.00567182761012981\\
396	0.00566382691837387\\
397	0.00565563258491992\\
398	0.00564723880811962\\
399	0.00563863955222837\\
400	0.00562982853558676\\
401	0.00562079922056375\\
402	0.00561154480310268\\
403	0.0056020582030006\\
404	0.00559233204912305\\
405	0.00558235866513765\\
406	0.00557213005533737\\
407	0.00556163788835732\\
408	0.00555087348540174\\
409	0.00553982780368971\\
410	0.00552849141492641\\
411	0.00551685448156283\\
412	0.00550490672915224\\
413	0.00549263740829848\\
414	0.00548003525961385\\
415	0.00546708849749552\\
416	0.00545378479780484\\
417	0.0054401112754309\\
418	0.00542605441187016\\
419	0.0054116000085708\\
420	0.00539673314876577\\
421	0.00538143816631\\
422	0.00536569862809995\\
423	0.00534949732924722\\
424	0.00533281630784088\\
425	0.00531563700179651\\
426	0.00529794031994624\\
427	0.00527970666589112\\
428	0.00526091616858311\\
429	0.00524154930988612\\
430	0.00522158670897148\\
431	0.00520100901224537\\
432	0.00517979689563761\\
433	0.00515793102197964\\
434	0.00513539190146522\\
435	0.00511215960693481\\
436	0.00508821329996635\\
437	0.0050635306819359\\
438	0.00503808544981885\\
439	0.00501184283113749\\
440	0.00498475355696809\\
441	0.00495677028297857\\
442	0.00492784673408646\\
443	0.004897939422489\\
444	0.00486701000135425\\
445	0.00483502847171133\\
446	0.00480197750681774\\
447	0.00476785824140474\\
448	0.00473269801185335\\
449	0.00469656058548712\\
450	0.0046595596934835\\
451	0.00462187671951774\\
452	0.00458393393414347\\
453	0.00454674949141144\\
454	0.00451044051279913\\
455	0.00447513188695105\\
456	0.00444095508964119\\
457	0.00440804619863455\\
458	0.00437654283428951\\
459	0.00434657956098983\\
460	0.00431828136589076\\
461	0.00429175330593373\\
462	0.00426706771200028\\
463	0.00424424587992064\\
464	0.00422323385217898\\
465	0.0042034722512688\\
466	0.00418422708684037\\
467	0.00416550069813697\\
468	0.00414728742576752\\
469	0.00412957230571683\\
470	0.00411232980882289\\
471	0.00409552275599316\\
472	0.00407910158538559\\
473	0.00406300421735797\\
474	0.00404715689755966\\
475	0.00403147654769942\\
476	0.00401587544089208\\
477	0.00400027999628674\\
478	0.0039846749808583\\
479	0.00396904354869625\\
480	0.00395336747204813\\
481	0.0039376274716649\\
482	0.00392180365609093\\
483	0.00390587607449125\\
484	0.00388982537664086\\
485	0.00387363355521733\\
486	0.00385728471522756\\
487	0.00384076577288425\\
488	0.00382406692090728\\
489	0.00380718053226114\\
490	0.00379009904886561\\
491	0.0037728150640437\\
492	0.00375532140232142\\
493	0.00373761119101988\\
494	0.00371967791683657\\
495	0.00370151545967474\\
496	0.00368311809563851\\
497	0.00366448046189818\\
498	0.00364559747892885\\
499	0.0036264642317458\\
500	0.00360707582316718\\
501	0.00358742735026712\\
502	0.00356751390046808\\
503	0.00354733054335028\\
504	0.00352687231790604\\
505	0.00350613421524147\\
506	0.00348511115709701\\
507	0.00346379797101353\\
508	0.00344218936347826\\
509	0.00342027989286092\\
510	0.00339806394424488\\
511	0.00337553570809208\\
512	0.00335268916009062\\
513	0.00332951803879312\\
514	0.00330601582100793\\
515	0.00328217569492879\\
516	0.00325799053100626\\
517	0.00323345285057013\\
518	0.00320855479219949\\
519	0.00318328807580211\\
520	0.00315764396430225\\
521	0.00313161322275183\\
522	0.00310518607459123\\
523	0.00307835215483463\\
524	0.00305110046025353\\
525	0.00302341929672944\\
526	0.00299529622407598\\
527	0.00296671799879961\\
528	0.00293767051549207\\
529	0.00290813874783711\\
530	0.0028781066905927\\
531	0.0028475573044006\\
532	0.00281647246590783\\
533	0.00278483292649969\\
534	0.00275261828398187\\
535	0.0027198069728484\\
536	0.0026863762804127\\
537	0.00265230239814621\\
538	0.00261756052016224\\
539	0.00258212500404021\\
540	0.00254596961341247\\
541	0.00250906786742201\\
542	0.00247139352909162\\
543	0.00243292127202608\\
544	0.00239362755840972\\
545	0.00235349175704467\\
546	0.00231249727013801\\
547	0.00227064453612754\\
548	0.00222794134987418\\
549	0.00218440577001577\\
550	0.00214013908144317\\
551	0.00209511622696201\\
552	0.00204927801102705\\
553	0.00200254729293833\\
554	0.00195472088222525\\
555	0.00190573084441342\\
556	0.00185549613625779\\
557	0.00180390471392892\\
558	0.00175081620496203\\
559	0.00169770725359811\\
560	0.00164609814802067\\
561	0.0015966641520743\\
562	0.00154794430499767\\
563	0.00149930786152006\\
564	0.00145063724791586\\
565	0.00140192736334399\\
566	0.0013538859561472\\
567	0.00130694443089763\\
568	0.00126075754464001\\
569	0.00121470715974243\\
570	0.00116868405784354\\
571	0.00112272218384932\\
572	0.00107685136522041\\
573	0.00103136023930306\\
574	0.000986412336410125\\
575	0.000941498720255945\\
576	0.000896516120383694\\
577	0.000851490573360361\\
578	0.000806462517147885\\
579	0.000761476872772062\\
580	0.000716581800501493\\
581	0.00067182827412899\\
582	0.00062726968466177\\
583	0.000582961403638049\\
584	0.000538960186060454\\
585	0.000495323349035172\\
586	0.000452107667820891\\
587	0.000409367927672478\\
588	0.000367155072958957\\
589	0.000325513941232698\\
590	0.000284480716949121\\
591	0.000244080684870835\\
592	0.00020432808725126\\
593	0.000165233217676711\\
594	0.000126870159937007\\
595	8.9611172254752e-05\\
596	5.42660945238507e-05\\
597	2.28062284332057e-05\\
598	2.9204464504877e-07\\
599	0\\
600	0\\
};
\addplot [color=mycolor20,solid,forget plot]
  table[row sep=crcr]{%
1	0.00604070286126672\\
2	0.00604069768145596\\
3	0.00604069240011982\\
4	0.00604068701536704\\
5	0.00604068152527344\\
6	0.00604067592788133\\
7	0.00604067022119914\\
8	0.00604066440320073\\
9	0.00604065847182493\\
10	0.006040652424975\\
11	0.00604064626051793\\
12	0.00604063997628406\\
13	0.00604063357006636\\
14	0.00604062703961985\\
15	0.00604062038266108\\
16	0.00604061359686745\\
17	0.00604060667987656\\
18	0.00604059962928574\\
19	0.00604059244265129\\
20	0.00604058511748774\\
21	0.00604057765126749\\
22	0.00604057004141987\\
23	0.00604056228533052\\
24	0.0060405543803408\\
25	0.00604054632374706\\
26	0.00604053811279983\\
27	0.00604052974470326\\
28	0.00604052121661428\\
29	0.00604051252564191\\
30	0.0060405036688465\\
31	0.0060404946432389\\
32	0.00604048544577976\\
33	0.00604047607337873\\
34	0.00604046652289367\\
35	0.00604045679112977\\
36	0.00604044687483879\\
37	0.00604043677071818\\
38	0.00604042647541023\\
39	0.00604041598550124\\
40	0.00604040529752056\\
41	0.00604039440793977\\
42	0.0060403833131717\\
43	0.00604037200956956\\
44	0.00604036049342594\\
45	0.00604034876097188\\
46	0.0060403368083759\\
47	0.00604032463174296\\
48	0.00604031222711347\\
49	0.00604029959046228\\
50	0.00604028671769763\\
51	0.00604027360465999\\
52	0.00604026024712113\\
53	0.00604024664078291\\
54	0.00604023278127614\\
55	0.00604021866415943\\
56	0.00604020428491812\\
57	0.00604018963896296\\
58	0.00604017472162895\\
59	0.00604015952817414\\
60	0.00604014405377826\\
61	0.00604012829354153\\
62	0.00604011224248334\\
63	0.00604009589554082\\
64	0.00604007924756761\\
65	0.00604006229333234\\
66	0.00604004502751723\\
67	0.00604002744471677\\
68	0.00604000953943605\\
69	0.00603999130608937\\
70	0.00603997273899869\\
71	0.00603995383239199\\
72	0.00603993458040168\\
73	0.00603991497706302\\
74	0.00603989501631235\\
75	0.00603987469198551\\
76	0.00603985399781589\\
77	0.00603983292743276\\
78	0.00603981147435948\\
79	0.00603978963201152\\
80	0.00603976739369456\\
81	0.00603974475260268\\
82	0.00603972170181615\\
83	0.00603969823429948\\
84	0.00603967434289937\\
85	0.00603965002034245\\
86	0.00603962525923319\\
87	0.00603960005205163\\
88	0.00603957439115098\\
89	0.00603954826875544\\
90	0.00603952167695767\\
91	0.00603949460771633\\
92	0.00603946705285358\\
93	0.00603943900405253\\
94	0.0060394104528545\\
95	0.00603938139065637\\
96	0.00603935180870775\\
97	0.00603932169810817\\
98	0.00603929104980418\\
99	0.00603925985458628\\
100	0.00603922810308589\\
101	0.0060391957857722\\
102	0.00603916289294902\\
103	0.0060391294147513\\
104	0.00603909534114195\\
105	0.00603906066190825\\
106	0.00603902536665826\\
107	0.00603898944481726\\
108	0.00603895288562394\\
109	0.00603891567812664\\
110	0.00603887781117927\\
111	0.00603883927343742\\
112	0.00603880005335422\\
113	0.00603876013917593\\
114	0.00603871951893777\\
115	0.00603867818045937\\
116	0.00603863611134024\\
117	0.00603859329895501\\
118	0.00603854973044864\\
119	0.00603850539273156\\
120	0.00603846027247447\\
121	0.0060384143561033\\
122	0.00603836762979378\\
123	0.00603832007946602\\
124	0.00603827169077896\\
125	0.00603822244912454\\
126	0.00603817233962188\\
127	0.00603812134711125\\
128	0.00603806945614792\\
129	0.0060380166509958\\
130	0.00603796291562085\\
131	0.00603790823368464\\
132	0.00603785258853736\\
133	0.00603779596321088\\
134	0.0060377383404116\\
135	0.00603767970251324\\
136	0.00603762003154909\\
137	0.00603755930920449\\
138	0.0060374975168089\\
139	0.0060374346353279\\
140	0.0060373706453548\\
141	0.00603730552710236\\
142	0.00603723926039393\\
143	0.00603717182465477\\
144	0.00603710319890287\\
145	0.00603703336173978\\
146	0.00603696229134103\\
147	0.00603688996544644\\
148	0.00603681636135023\\
149	0.00603674145589082\\
150	0.00603666522544051\\
151	0.00603658764589482\\
152	0.0060365086926617\\
153	0.00603642834065032\\
154	0.00603634656425989\\
155	0.00603626333736804\\
156	0.00603617863331896\\
157	0.00603609242491136\\
158	0.00603600468438619\\
159	0.00603591538341399\\
160	0.0060358244930821\\
161	0.00603573198388151\\
162	0.00603563782569354\\
163	0.00603554198777613\\
164	0.00603544443874995\\
165	0.00603534514658427\\
166	0.00603524407858244\\
167	0.00603514120136712\\
168	0.00603503648086523\\
169	0.0060349298822928\\
170	0.00603482137013902\\
171	0.0060347109081507\\
172	0.00603459845931581\\
173	0.00603448398584715\\
174	0.00603436744916537\\
175	0.00603424880988208\\
176	0.00603412802778226\\
177	0.00603400506180656\\
178	0.00603387987003342\\
179	0.00603375240966048\\
180	0.00603362263698609\\
181	0.00603349050739019\\
182	0.00603335597531511\\
183	0.00603321899424582\\
184	0.00603307951668995\\
185	0.00603293749415748\\
186	0.00603279287714004\\
187	0.00603264561508996\\
188	0.00603249565639885\\
189	0.00603234294837584\\
190	0.00603218743722551\\
191	0.00603202906802543\\
192	0.00603186778470337\\
193	0.00603170353001399\\
194	0.00603153624551522\\
195	0.00603136587154434\\
196	0.0060311923471935\\
197	0.0060310156102849\\
198	0.0060308355973455\\
199	0.00603065224358139\\
200	0.00603046548285162\\
201	0.00603027524764155\\
202	0.00603008146903589\\
203	0.00602988407669103\\
204	0.00602968299880715\\
205	0.00602947816209949\\
206	0.00602926949176939\\
207	0.00602905691147463\\
208	0.00602884034329925\\
209	0.00602861970772277\\
210	0.00602839492358883\\
211	0.0060281659080733\\
212	0.00602793257665158\\
213	0.0060276948430654\\
214	0.00602745261928877\\
215	0.00602720581549342\\
216	0.00602695434001352\\
217	0.00602669809930934\\
218	0.00602643699793054\\
219	0.00602617093847832\\
220	0.00602589982156705\\
221	0.00602562354578477\\
222	0.00602534200765297\\
223	0.00602505510158548\\
224	0.00602476271984619\\
225	0.00602446475250616\\
226	0.00602416108739942\\
227	0.00602385161007798\\
228	0.00602353620376556\\
229	0.00602321474931045\\
230	0.00602288712513714\\
231	0.00602255320719681\\
232	0.00602221286891671\\
233	0.00602186598114831\\
234	0.00602151241211429\\
235	0.00602115202735406\\
236	0.00602078468966836\\
237	0.0060204102590622\\
238	0.00602002859268669\\
239	0.00601963954477947\\
240	0.00601924296660363\\
241	0.0060188387063855\\
242	0.00601842660925068\\
243	0.00601800651715886\\
244	0.00601757826883696\\
245	0.00601714169971099\\
246	0.0060166966418361\\
247	0.00601624292382536\\
248	0.0060157803707767\\
249	0.00601530880419837\\
250	0.00601482804193277\\
251	0.00601433789807843\\
252	0.00601383818291052\\
253	0.00601332870279949\\
254	0.00601280926012784\\
255	0.00601227965320519\\
256	0.00601173967618163\\
257	0.00601118911895872\\
258	0.00601062776709899\\
259	0.00601005540173316\\
260	0.00600947179946547\\
261	0.00600887673227664\\
262	0.00600826996742509\\
263	0.00600765126734565\\
264	0.0060070203895459\\
265	0.00600637708650021\\
266	0.00600572110554137\\
267	0.0060050521887495\\
268	0.00600437007283856\\
269	0.00600367448903991\\
270	0.00600296516298329\\
271	0.00600224181457464\\
272	0.00600150415787112\\
273	0.00600075190095296\\
274	0.00599998474579196\\
275	0.0059992023881168\\
276	0.0059984045172748\\
277	0.00599759081609012\\
278	0.00599676096071829\\
279	0.00599591462049686\\
280	0.005995051457792\\
281	0.00599417112784122\\
282	0.00599327327859168\\
283	0.0059923575505342\\
284	0.0059914235765328\\
285	0.00599047098164961\\
286	0.00598949938296499\\
287	0.00598850838939272\\
288	0.00598749760149035\\
289	0.00598646661126413\\
290	0.0059854150019687\\
291	0.00598434234790135\\
292	0.0059832482141907\\
293	0.00598213215657961\\
294	0.00598099372120207\\
295	0.00597983244435416\\
296	0.00597864785225872\\
297	0.00597743946082362\\
298	0.0059762067753936\\
299	0.00597494929049533\\
300	0.00597366648957539\\
301	0.00597235784473117\\
302	0.00597102281643418\\
303	0.00596966085324572\\
304	0.00596827139152416\\
305	0.00596685385512431\\
306	0.00596540765508857\\
307	0.00596393218932975\\
308	0.00596242684230389\\
309	0.00596089098467301\\
310	0.005959323972958\\
311	0.00595772514918041\\
312	0.00595609384049316\\
313	0.00595442935879954\\
314	0.00595273100035977\\
315	0.00595099804538495\\
316	0.00594922975761662\\
317	0.00594742538389277\\
318	0.00594558415369862\\
319	0.00594370527870151\\
320	0.00594178795226922\\
321	0.00593983134897069\\
322	0.00593783462405842\\
323	0.00593579691293132\\
324	0.0059337173305774\\
325	0.00593159497099554\\
326	0.00592942890659603\\
327	0.0059272181875775\\
328	0.0059249618412794\\
329	0.00592265887150968\\
330	0.00592030825784598\\
331	0.00591790895490918\\
332	0.00591545989160667\\
333	0.00591295997034463\\
334	0.00591040806620996\\
335	0.00590780302611918\\
336	0.00590514366793275\\
337	0.00590242877953272\\
338	0.00589965711786155\\
339	0.00589682740791893\\
340	0.00589393834171451\\
341	0.00589098857718418\\
342	0.00588797673707183\\
343	0.00588490140775219\\
344	0.0058817611380014\\
345	0.00587855443771407\\
346	0.00587527977656284\\
347	0.00587193558259683\\
348	0.00586852024077776\\
349	0.00586503209146127\\
350	0.00586146942880643\\
351	0.00585783049911073\\
352	0.00585411349906987\\
353	0.00585031657395378\\
354	0.00584643781567608\\
355	0.00584247526076415\\
356	0.00583842688824244\\
357	0.0058342906174133\\
358	0.00583006430552663\\
359	0.00582574574533156\\
360	0.00582133266250288\\
361	0.00581682271293385\\
362	0.0058122134798853\\
363	0.00580750247098027\\
364	0.00580268711502714\\
365	0.00579776475864244\\
366	0.00579273266261862\\
367	0.00578758799798392\\
368	0.00578232784208436\\
369	0.00577694917450169\\
370	0.00577144887262362\\
371	0.0057658237069727\\
372	0.00576007033631017\\
373	0.00575418530222471\\
374	0.00574816502335601\\
375	0.00574200578947321\\
376	0.00573570375528036\\
377	0.00572925493391668\\
378	0.00572265519015235\\
379	0.00571590023328521\\
380	0.00570898560975258\\
381	0.00570190669547667\\
382	0.00569465868796211\\
383	0.00568723659813321\\
384	0.00567963524178665\\
385	0.00567184923042796\\
386	0.00566387296297438\\
387	0.00565570061850434\\
388	0.00564732614852558\\
389	0.00563874326842201\\
390	0.00562994544885467\\
391	0.00562092590397597\\
392	0.00561167757684297\\
393	0.0056021931235798\\
394	0.00559246489594869\\
395	0.00558248492210143\\
396	0.00557224488527129\\
397	0.00556173610014804\\
398	0.00555094948656369\\
399	0.00553987553928657\\
400	0.00552850428898409\\
401	0.00551682525811876\\
402	0.00550482743946284\\
403	0.00549249927104814\\
404	0.00547982863506249\\
405	0.00546680280346584\\
406	0.00545340840135631\\
407	0.00543963139660914\\
408	0.00542545709129351\\
409	0.00541087024193215\\
410	0.00539585526736261\\
411	0.00538039629792995\\
412	0.00536447723384199\\
413	0.00534808180737729\\
414	0.00533119353783731\\
415	0.0053137956986983\\
416	0.00529587151388881\\
417	0.00527740443553554\\
418	0.00525837840941861\\
419	0.00523877743823224\\
420	0.00521858532440277\\
421	0.00519778529766863\\
422	0.00517635939137337\\
423	0.00515428747691237\\
424	0.00513154562729427\\
425	0.00510810224031989\\
426	0.00508391495677255\\
427	0.00505893916759164\\
428	0.00503312793904506\\
429	0.00500643350322129\\
430	0.00497881596042503\\
431	0.00495024230059195\\
432	0.00492068858544584\\
433	0.00489014442261511\\
434	0.00485861893677142\\
435	0.0048261487135882\\
436	0.00479280832293565\\
437	0.00475872421739817\\
438	0.00472409309455758\\
439	0.0046897229291746\\
440	0.00465601899576579\\
441	0.00462308586759372\\
442	0.00459103525379076\\
443	0.00455998509454473\\
444	0.00453005799718654\\
445	0.00450137864226732\\
446	0.00447406996077481\\
447	0.00444824758907592\\
448	0.00442401202173317\\
449	0.00440143778640491\\
450	0.00438055781551818\\
451	0.00436134261793029\\
452	0.00434351468438177\\
453	0.00432616324410694\\
454	0.00430929422423594\\
455	0.00429290781882491\\
456	0.00427699674321518\\
457	0.00426154508181158\\
458	0.00424652720145676\\
459	0.00423190685177035\\
460	0.00421763662973471\\
461	0.004203658089645\\
462	0.0041899028287776\\
463	0.00417629507945139\\
464	0.0041627565142112\\
465	0.004149235382823\\
466	0.00413571810586105\\
467	0.00412218965680862\\
468	0.00410863376027093\\
469	0.00409503317653824\\
470	0.00408137008153313\\
471	0.0040676265459852\\
472	0.00405378510905811\\
473	0.00403982942760372\\
474	0.00402574495899102\\
475	0.00401151960005543\\
476	0.00399714414979814\\
477	0.0039826119151884\\
478	0.00396791625508381\\
479	0.00395305065346735\\
480	0.00393800879180704\\
481	0.00392278461619124\\
482	0.00390737239387512\\
483	0.00389176675294497\\
484	0.00387596269825045\\
485	0.00385995559697531\\
486	0.00384374112887292\\
487	0.00382731520011974\\
488	0.00381067382738024\\
489	0.00379381305678002\\
490	0.00377672896545118\\
491	0.00375941765990687\\
492	0.00374187527095354\\
493	0.0037240979450396\\
494	0.00370608183220922\\
495	0.00368782307117653\\
496	0.00366931777244603\\
497	0.00365056200083808\\
498	0.00363155175914802\\
499	0.0036122829748172\\
500	0.00359275149114712\\
501	0.00357295305835337\\
502	0.00355288332358055\\
503	0.00353253781991185\\
504	0.00351191195442204\\
505	0.00349100099532988\\
506	0.00346980005829799\\
507	0.00344830409190142\\
508	0.00342650786223379\\
509	0.00340440593654245\\
510	0.00338199266568192\\
511	0.00335926216506952\\
512	0.00333620829390472\\
513	0.00331282463253544\\
514	0.00328910445785155\\
515	0.00326504071658356\\
516	0.00324062599638607\\
517	0.00321585249459072\\
518	0.00319071198452735\\
519	0.00316519577933604\\
520	0.00313929469323235\\
521	0.00311299900024939\\
522	0.00308629839056859\\
523	0.0030591819246678\\
524	0.00303163798565948\\
525	0.00300365423037989\\
526	0.00297521754003508\\
527	0.00294631397152773\\
528	0.00291692871099567\\
529	0.00288704603161653\\
530	0.00285664925839858\\
531	0.00282572074352561\\
532	0.00279424185689702\\
533	0.00276219299786336\\
534	0.00272955363586645\\
535	0.00269630238983699\\
536	0.00266241715889706\\
537	0.00262787532050961\\
538	0.00259265401715977\\
539	0.00255673055832753\\
540	0.00252008296455489\\
541	0.00248269066657626\\
542	0.00244453538964067\\
543	0.00240560232750984\\
544	0.00236588246753321\\
545	0.00232537716519942\\
546	0.00228415675127574\\
547	0.00224224875684687\\
548	0.00219959959602875\\
549	0.00215614081365568\\
550	0.0021116893143768\\
551	0.00206616683318269\\
552	0.00201950322840626\\
553	0.00197160885480873\\
554	0.00192240234639964\\
555	0.00187175949147671\\
556	0.00182016778867611\\
557	0.0017700079133193\\
558	0.00172189222910075\\
559	0.00167483274808183\\
560	0.00162776118290781\\
561	0.00158051982821157\\
562	0.0015330935121052\\
563	0.00148554422170036\\
564	0.00143866178250926\\
565	0.00139285578388573\\
566	0.0013474996131113\\
567	0.00130214936842766\\
568	0.00125674675529522\\
569	0.00121131505068596\\
570	0.00116589341138373\\
571	0.00112058409493486\\
572	0.00107579239048085\\
573	0.00103115420667899\\
574	0.000986382803600775\\
575	0.00094149076835946\\
576	0.000896513086086129\\
577	0.000851489245902153\\
578	0.000806461868510235\\
579	0.000761476522194667\\
580	0.000716581604601778\\
581	0.000671828163310488\\
582	0.00062726962209076\\
583	0.000582961369628161\\
584	0.000538960169322208\\
585	0.000495323342111254\\
586	0.000452107665705174\\
587	0.000409367927260633\\
588	0.000367155072958956\\
589	0.000325513941232701\\
590	0.000284480716949122\\
591	0.000244080684870839\\
592	0.000204328087251261\\
593	0.000165233217676712\\
594	0.000126870159937008\\
595	8.9611172254752e-05\\
596	5.42660945238509e-05\\
597	2.28062284332057e-05\\
598	2.9204464504877e-07\\
599	0\\
600	0\\
};
\addplot [color=mycolor21,solid,forget plot]
  table[row sep=crcr]{%
1	0.0060405115386396\\
2	0.00604050497580479\\
3	0.00604049827929662\\
4	0.0060404914465168\\
5	0.00604048447481996\\
6	0.0060404773615129\\
7	0.00604047010385364\\
8	0.00604046269905084\\
9	0.00604045514426286\\
10	0.00604044743659698\\
11	0.0060404395731087\\
12	0.00604043155080067\\
13	0.00604042336662207\\
14	0.00604041501746772\\
15	0.00604040650017708\\
16	0.00604039781153354\\
17	0.00604038894826346\\
18	0.00604037990703524\\
19	0.00604037068445846\\
20	0.00604036127708301\\
21	0.00604035168139799\\
22	0.00604034189383093\\
23	0.00604033191074684\\
24	0.00604032172844708\\
25	0.00604031134316853\\
26	0.00604030075108255\\
27	0.00604028994829395\\
28	0.00604027893083998\\
29	0.00604026769468939\\
30	0.00604025623574118\\
31	0.0060402445498238\\
32	0.0060402326326939\\
33	0.00604022048003529\\
34	0.00604020808745783\\
35	0.00604019545049642\\
36	0.00604018256460975\\
37	0.0060401694251792\\
38	0.00604015602750773\\
39	0.00604014236681863\\
40	0.0060401284382545\\
41	0.00604011423687585\\
42	0.00604009975766\\
43	0.00604008499549986\\
44	0.00604006994520267\\
45	0.00604005460148872\\
46	0.00604003895899013\\
47	0.00604002301224949\\
48	0.00604000675571866\\
49	0.00603999018375738\\
50	0.00603997329063187\\
51	0.00603995607051361\\
52	0.00603993851747789\\
53	0.00603992062550243\\
54	0.00603990238846597\\
55	0.00603988380014686\\
56	0.00603986485422158\\
57	0.00603984554426332\\
58	0.00603982586374054\\
59	0.00603980580601526\\
60	0.00603978536434184\\
61	0.00603976453186516\\
62	0.00603974330161922\\
63	0.00603972166652551\\
64	0.00603969961939137\\
65	0.00603967715290844\\
66	0.00603965425965089\\
67	0.00603963093207381\\
68	0.00603960716251152\\
69	0.00603958294317581\\
70	0.00603955826615418\\
71	0.00603953312340808\\
72	0.0060395075067711\\
73	0.00603948140794716\\
74	0.00603945481850856\\
75	0.00603942772989412\\
76	0.00603940013340739\\
77	0.00603937202021446\\
78	0.00603934338134211\\
79	0.00603931420767584\\
80	0.00603928448995767\\
81	0.00603925421878412\\
82	0.0060392233846041\\
83	0.00603919197771672\\
84	0.00603915998826905\\
85	0.00603912740625393\\
86	0.00603909422150771\\
87	0.00603906042370773\\
88	0.00603902600237024\\
89	0.00603899094684771\\
90	0.00603895524632647\\
91	0.00603891888982421\\
92	0.00603888186618738\\
93	0.00603884416408854\\
94	0.00603880577202371\\
95	0.00603876667830962\\
96	0.00603872687108093\\
97	0.00603868633828738\\
98	0.00603864506769076\\
99	0.00603860304686205\\
100	0.00603856026317835\\
101	0.0060385167038197\\
102	0.00603847235576585\\
103	0.00603842720579312\\
104	0.00603838124047081\\
105	0.00603833444615807\\
106	0.00603828680900013\\
107	0.00603823831492472\\
108	0.00603818894963851\\
109	0.00603813869862311\\
110	0.0060380875471314\\
111	0.00603803548018332\\
112	0.00603798248256189\\
113	0.00603792853880897\\
114	0.00603787363322096\\
115	0.0060378177498443\\
116	0.00603776087247091\\
117	0.00603770298463356\\
118	0.00603764406960102\\
119	0.00603758411037308\\
120	0.0060375230896755\\
121	0.00603746098995471\\
122	0.00603739779337252\\
123	0.00603733348180048\\
124	0.00603726803681418\\
125	0.00603720143968747\\
126	0.00603713367138634\\
127	0.00603706471256278\\
128	0.00603699454354828\\
129	0.00603692314434729\\
130	0.00603685049463051\\
131	0.0060367765737278\\
132	0.00603670136062105\\
133	0.00603662483393677\\
134	0.0060365469719385\\
135	0.00603646775251884\\
136	0.00603638715319152\\
137	0.006036305151083\\
138	0.00603622172292386\\
139	0.00603613684504009\\
140	0.00603605049334386\\
141	0.00603596264332425\\
142	0.00603587327003759\\
143	0.00603578234809756\\
144	0.00603568985166492\\
145	0.00603559575443699\\
146	0.00603550002963689\\
147	0.00603540265000233\\
148	0.00603530358777422\\
149	0.00603520281468479\\
150	0.00603510030194546\\
151	0.00603499602023439\\
152	0.00603488993968359\\
153	0.00603478202986581\\
154	0.00603467225978081\\
155	0.00603456059784149\\
156	0.00603444701185953\\
157	0.00603433146903064\\
158	0.0060342139359194\\
159	0.00603409437844369\\
160	0.00603397276185872\\
161	0.00603384905074067\\
162	0.00603372320896976\\
163	0.0060335951997131\\
164	0.00603346498540687\\
165	0.00603333252773816\\
166	0.00603319778762635\\
167	0.00603306072520401\\
168	0.00603292129979739\\
169	0.00603277946990622\\
170	0.00603263519318343\\
171	0.00603248842641388\\
172	0.00603233912549306\\
173	0.00603218724540499\\
174	0.0060320327401998\\
175	0.00603187556297068\\
176	0.00603171566583043\\
177	0.00603155299988751\\
178	0.0060313875152214\\
179	0.00603121916085778\\
180	0.00603104788474286\\
181	0.00603087363371748\\
182	0.00603069635349043\\
183	0.00603051598861155\\
184	0.00603033248244412\\
185	0.00603014577713679\\
186	0.00602995581359512\\
187	0.0060297625314524\\
188	0.00602956586904013\\
189	0.00602936576335804\\
190	0.00602916215004345\\
191	0.00602895496334029\\
192	0.00602874413606756\\
193	0.00602852959958735\\
194	0.00602831128377235\\
195	0.00602808911697287\\
196	0.00602786302598349\\
197	0.00602763293600906\\
198	0.0060273987706305\\
199	0.00602716045176988\\
200	0.00602691789965512\\
201	0.00602667103278441\\
202	0.0060264197678899\\
203	0.00602616401990116\\
204	0.00602590370190799\\
205	0.006025638725123\\
206	0.00602536899884355\\
207	0.00602509443041335\\
208	0.00602481492518334\\
209	0.00602453038647252\\
210	0.00602424071552795\\
211	0.00602394581148433\\
212	0.00602364557132305\\
213	0.00602333988983085\\
214	0.00602302865955784\\
215	0.00602271177077484\\
216	0.00602238911143023\\
217	0.00602206056710623\\
218	0.00602172602097441\\
219	0.00602138535375063\\
220	0.00602103844364905\\
221	0.00602068516633558\\
222	0.00602032539488046\\
223	0.00601995899970986\\
224	0.00601958584855677\\
225	0.00601920580641077\\
226	0.00601881873546681\\
227	0.00601842449507293\\
228	0.00601802294167679\\
229	0.00601761392877109\\
230	0.00601719730683757\\
231	0.00601677292328973\\
232	0.00601634062241428\\
233	0.0060159002453106\\
234	0.00601545162982939\\
235	0.00601499461050892\\
236	0.00601452901851008\\
237	0.00601405468154942\\
238	0.00601357142383042\\
239	0.00601307906597272\\
240	0.00601257742493947\\
241	0.00601206631396247\\
242	0.00601154554246535\\
243	0.00601101491598445\\
244	0.00601047423608748\\
245	0.00600992330028988\\
246	0.00600936190196889\\
247	0.00600878983027507\\
248	0.00600820687004161\\
249	0.00600761280169106\\
250	0.00600700740113942\\
251	0.00600639043969812\\
252	0.00600576168397293\\
253	0.00600512089576077\\
254	0.00600446783194364\\
255	0.00600380224438007\\
256	0.00600312387979381\\
257	0.00600243247966014\\
258	0.00600172778008923\\
259	0.00600100951170685\\
260	0.00600027739953232\\
261	0.00599953116285375\\
262	0.00599877051510059\\
263	0.00599799516371429\\
264	0.00599720481001554\\
265	0.00599639914906835\\
266	0.0059955778695411\\
267	0.0059947406535649\\
268	0.00599388717658837\\
269	0.00599301710722961\\
270	0.00599213010712454\\
271	0.00599122583077206\\
272	0.00599030392537543\\
273	0.00598936403067997\\
274	0.00598840577880696\\
275	0.00598742879408339\\
276	0.00598643269286741\\
277	0.00598541708336929\\
278	0.00598438156546767\\
279	0.00598332573052086\\
280	0.0059822491611728\\
281	0.00598115143115341\\
282	0.00598003210507315\\
283	0.00597889073821138\\
284	0.00597772687629799\\
285	0.00597654005528827\\
286	0.00597532980113045\\
287	0.00597409562952539\\
288	0.00597283704567818\\
289	0.00597155354404147\\
290	0.00597024460804948\\
291	0.00596890970984297\\
292	0.00596754830998413\\
293	0.00596615985716149\\
294	0.00596474378788404\\
295	0.00596329952616421\\
296	0.0059618264831893\\
297	0.00596032405698086\\
298	0.00595879163204187\\
299	0.00595722857899214\\
300	0.00595563425419103\\
301	0.00595400799934622\\
302	0.00595234914110841\\
303	0.00595065699065172\\
304	0.00594893084323893\\
305	0.00594716997776979\\
306	0.00594537365631309\\
307	0.0059435411236291\\
308	0.00594167160668233\\
309	0.00593976431413166\\
310	0.00593781843580287\\
311	0.00593583314214417\\
312	0.0059338075836631\\
313	0.00593174089034451\\
314	0.00592963217105018\\
315	0.00592748051289843\\
316	0.00592528498062269\\
317	0.00592304461590789\\
318	0.00592075843670291\\
319	0.00591842543650829\\
320	0.00591604458363759\\
321	0.0059136148204504\\
322	0.00591113506255556\\
323	0.00590860419798202\\
324	0.00590602108631484\\
325	0.0059033845577936\\
326	0.00590069341237401\\
327	0.0058979464187612\\
328	0.00589514231339425\\
329	0.0058922797993854\\
330	0.00588935754541443\\
331	0.0058863741845763\\
332	0.00588332831317656\\
333	0.00588021848945845\\
334	0.00587704323226054\\
335	0.0058738010196226\\
336	0.00587049028732595\\
337	0.00586710942736092\\
338	0.00586365678631363\\
339	0.00586013066366021\\
340	0.00585652930994459\\
341	0.0058528509248092\\
342	0.00584909365496575\\
343	0.0058452555921781\\
344	0.00584133477099838\\
345	0.00583732916634428\\
346	0.00583323669094712\\
347	0.00582905519265595\\
348	0.00582478245157597\\
349	0.00582041617703999\\
350	0.00581595400455628\\
351	0.00581139349257955\\
352	0.00580673211910165\\
353	0.00580196727811239\\
354	0.00579709627591544\\
355	0.00579211632706821\\
356	0.00578702455004549\\
357	0.00578181796285351\\
358	0.00577649347849695\\
359	0.00577104790028927\\
360	0.00576547791702843\\
361	0.00575978009807038\\
362	0.00575395088834157\\
363	0.00574798660334017\\
364	0.00574188342418147\\
365	0.00573563739272935\\
366	0.00572924440677059\\
367	0.00572270021484352\\
368	0.00571600040952584\\
369	0.005709140423154\\
370	0.0057021155240592\\
371	0.00569492080962159\\
372	0.0056875511982807\\
373	0.0056800014216376\\
374	0.00567226601295422\\
375	0.00566433929319607\\
376	0.00565621535729714\\
377	0.00564788805911819\\
378	0.00563935099452979\\
379	0.0056305974823533\\
380	0.00562162054289488\\
381	0.00561241287384588\\
382	0.00560296682340784\\
383	0.00559327436064272\\
384	0.00558332704312636\\
385	0.00557311598119243\\
386	0.00556263179377011\\
387	0.0055518645746237\\
388	0.0055408038787054\\
389	0.00552943871860183\\
390	0.00551775757784875\\
391	0.00550574847396325\\
392	0.00549339911978506\\
393	0.00548069695745734\\
394	0.00546762919783081\\
395	0.00545418286695475\\
396	0.00544034485903358\\
397	0.0054261019943971\\
398	0.00541144107993989\\
399	0.00539634896805736\\
400	0.00538081260476591\\
401	0.00536481898707886\\
402	0.00534835499273021\\
403	0.00533140743911178\\
404	0.00531396296337945\\
405	0.00529600816592463\\
406	0.00527752883261459\\
407	0.00525850891416183\\
408	0.00523892916154191\\
409	0.00521876400267892\\
410	0.00519797662168044\\
411	0.00517652869294469\\
412	0.00515438080539511\\
413	0.00513149313071235\\
414	0.0051078266056657\\
415	0.00508334309874557\\
416	0.00505800565036955\\
417	0.00503178177338743\\
418	0.00500464863693634\\
419	0.00497660150946273\\
420	0.00494765572762768\\
421	0.00491785471270213\\
422	0.00488728058215687\\
423	0.00485606816725119\\
424	0.00482455303040914\\
425	0.0047935031221998\\
426	0.00476300169002057\\
427	0.00473313885047253\\
428	0.00470401144068946\\
429	0.00467572247426099\\
430	0.00464837987732808\\
431	0.00462209462361826\\
432	0.00459697793237357\\
433	0.00457313715956891\\
434	0.00455066990763092\\
435	0.00452965565608938\\
436	0.00451014421183016\\
437	0.00449213987423766\\
438	0.00447557991696671\\
439	0.00445976160220178\\
440	0.00444438329017133\\
441	0.00442945124921817\\
442	0.00441496622615378\\
443	0.00440092238274243\\
444	0.00438730644245399\\
445	0.00437409738269767\\
446	0.00436126490607438\\
447	0.00434876889534301\\
448	0.00433655936263782\\
449	0.0043245771894617\\
450	0.00431275609427381\\
451	0.00430102640784474\\
452	0.00428932972198311\\
453	0.0042776548476847\\
454	0.00426598925603829\\
455	0.00425431918821024\\
456	0.00424262985238187\\
457	0.0042309056985038\\
458	0.00421913077775146\\
459	0.0042072891883445\\
460	0.00419536560038398\\
461	0.00418334583722357\\
462	0.00417121746949017\\
463	0.00415897034334175\\
464	0.0041465969142752\\
465	0.00413409127308022\\
466	0.00412144755919448\\
467	0.00410866002540348\\
468	0.00409572310204968\\
469	0.00408263145707985\\
470	0.0040693800473816\\
471	0.00405596415608319\\
472	0.00404237940998745\\
473	0.00402862177137887\\
474	0.00401468749958905\\
475	0.00400057308067138\\
476	0.00398627512949324\\
477	0.00397179029839382\\
478	0.00395711528042726\\
479	0.00394224681010434\\
480	0.00392718166136174\\
481	0.00391191664262769\\
482	0.00389644858905306\\
483	0.00388077435224436\\
484	0.00386489078816142\\
485	0.00384879474420444\\
486	0.00383248304685515\\
487	0.00381595249145306\\
488	0.0037991998355973\\
489	0.00378222179513981\\
490	0.00376501503962558\\
491	0.00374757618723257\\
492	0.00372990179928235\\
493	0.00371198837440514\\
494	0.00369383234244635\\
495	0.00367543005819135\\
496	0.00365677779495725\\
497	0.00363787173804987\\
498	0.00361870797801236\\
499	0.00359928250350329\\
500	0.00357959119355817\\
501	0.00355962980915254\\
502	0.00353939398401156\\
503	0.00351887921460329\\
504	0.00349808084924341\\
505	0.00347699407622836\\
506	0.00345561391090399\\
507	0.00343393518156551\\
508	0.00341195251407696\\
509	0.00338966031509233\\
510	0.00336705275376157\\
511	0.00334412374181056\\
512	0.00332086691189175\\
513	0.00329727559410764\\
514	0.00327334279062074\\
515	0.00324906114828117\\
516	0.00322442292923013\\
517	0.00319941997947536\\
518	0.00317404369548875\\
519	0.00314828498894659\\
520	0.00312213424983111\\
521	0.00309558130823694\\
522	0.00306861539539113\\
523	0.00304122510460897\\
524	0.00301339835318134\\
525	0.00298512234654203\\
526	0.00295638354651234\\
527	0.00292716764599154\\
528	0.00289745955318733\\
529	0.0028672433893974\\
530	0.0028365025055124\\
531	0.00280521952386659\\
532	0.00277337641388418\\
533	0.00274095461225293\\
534	0.00270793520146081\\
535	0.00267429916488165\\
536	0.0026400277413916\\
537	0.00260510289781278\\
538	0.0025695079186138\\
539	0.00253322814582862\\
540	0.00249625223342463\\
541	0.00245857508153023\\
542	0.00242020649288695\\
543	0.00238123570251701\\
544	0.00234160205815798\\
545	0.00230124504563887\\
546	0.00226002773265821\\
547	0.00221783170899319\\
548	0.00217459767527134\\
549	0.0021302495305581\\
550	0.00208471834938659\\
551	0.00203792665729891\\
552	0.00198977932691693\\
553	0.00194012959396351\\
554	0.00189105467690076\\
555	0.0018437709329238\\
556	0.00179836906095719\\
557	0.00175287820462187\\
558	0.00170717331469996\\
559	0.00166116212177324\\
560	0.00161488078334582\\
561	0.00156841503513133\\
562	0.0015224984313825\\
563	0.00147763480850627\\
564	0.00143310019390703\\
565	0.00138847730609411\\
566	0.00134373361196189\\
567	0.00129889564446826\\
568	0.00125400069337501\\
569	0.00120908421228601\\
570	0.00116444936949479\\
571	0.0011201870485504\\
572	0.00107576112986418\\
573	0.00103114982609361\\
574	0.000986381581866229\\
575	0.000941490291903521\\
576	0.000896512873534784\\
577	0.000851489140303056\\
578	0.000806461811104773\\
579	0.000761476490067704\\
580	0.000716581586462604\\
581	0.000671828153146611\\
582	0.000627269616642477\\
583	0.000582961366995044\\
584	0.000538960168256757\\
585	0.00049532334179279\\
586	0.00045210766564535\\
587	0.000409367927260643\\
588	0.000367155072958964\\
589	0.000325513941232703\\
590	0.000284480716949125\\
591	0.000244080684870838\\
592	0.000204328087251261\\
593	0.000165233217676711\\
594	0.000126870159937007\\
595	8.96111722547522e-05\\
596	5.42660945238511e-05\\
597	2.2806228433206e-05\\
598	2.9204464504877e-07\\
599	0\\
600	0\\
};
\addplot [color=black!20!mycolor21,solid,forget plot]
  table[row sep=crcr]{%
1	0.00604041973083569\\
2	0.00604041163451223\\
3	0.00604040336646474\\
4	0.00604039492318405\\
5	0.0060403863010932\\
6	0.00604037749654635\\
7	0.00604036850582758\\
8	0.00604035932514956\\
9	0.00604034995065246\\
10	0.00604034037840258\\
11	0.00604033060439107\\
12	0.00604032062453276\\
13	0.00604031043466472\\
14	0.00604030003054496\\
15	0.00604028940785117\\
16	0.0060402785621792\\
17	0.00604026748904179\\
18	0.00604025618386714\\
19	0.00604024464199752\\
20	0.0060402328586877\\
21	0.00604022082910366\\
22	0.00604020854832101\\
23	0.0060401960113235\\
24	0.00604018321300155\\
25	0.00604017014815062\\
26	0.00604015681146975\\
27	0.00604014319755995\\
28	0.00604012930092258\\
29	0.00604011511595774\\
30	0.0060401006369627\\
31	0.0060400858581301\\
32	0.00604007077354647\\
33	0.00604005537719037\\
34	0.00604003966293073\\
35	0.00604002362452516\\
36	0.00604000725561808\\
37	0.0060399905497391\\
38	0.00603997350030111\\
39	0.00603995610059853\\
40	0.00603993834380532\\
41	0.00603992022297334\\
42	0.00603990173103038\\
43	0.00603988286077825\\
44	0.00603986360489084\\
45	0.00603984395591222\\
46	0.0060398239062547\\
47	0.00603980344819681\\
48	0.00603978257388132\\
49	0.0060397612753132\\
50	0.00603973954435765\\
51	0.00603971737273794\\
52	0.00603969475203342\\
53	0.00603967167367738\\
54	0.00603964812895493\\
55	0.00603962410900086\\
56	0.00603959960479752\\
57	0.00603957460717259\\
58	0.00603954910679688\\
59	0.00603952309418217\\
60	0.00603949655967891\\
61	0.00603946949347404\\
62	0.0060394418855886\\
63	0.00603941372587553\\
64	0.00603938500401733\\
65	0.00603935570952367\\
66	0.00603932583172915\\
67	0.00603929535979078\\
68	0.00603926428268575\\
69	0.00603923258920885\\
70	0.00603920026797016\\
71	0.00603916730739258\\
72	0.00603913369570929\\
73	0.00603909942096132\\
74	0.00603906447099502\\
75	0.0060390288334596\\
76	0.00603899249580445\\
77	0.00603895544527667\\
78	0.00603891766891847\\
79	0.00603887915356454\\
80	0.00603883988583942\\
81	0.0060387998521549\\
82	0.0060387590387073\\
83	0.0060387174314748\\
84	0.00603867501621471\\
85	0.00603863177846084\\
86	0.00603858770352055\\
87	0.00603854277647222\\
88	0.00603849698216225\\
89	0.00603845030520236\\
90	0.00603840272996667\\
91	0.00603835424058895\\
92	0.00603830482095957\\
93	0.00603825445472271\\
94	0.00603820312527338\\
95	0.00603815081575444\\
96	0.00603809750905357\\
97	0.00603804318780031\\
98	0.00603798783436296\\
99	0.0060379314308455\\
100	0.00603787395908446\\
101	0.00603781540064582\\
102	0.00603775573682165\\
103	0.00603769494862711\\
104	0.00603763301679705\\
105	0.00603756992178262\\
106	0.00603750564374808\\
107	0.00603744016256737\\
108	0.0060373734578205\\
109	0.00603730550879026\\
110	0.00603723629445852\\
111	0.00603716579350268\\
112	0.00603709398429197\\
113	0.00603702084488379\\
114	0.00603694635301978\\
115	0.00603687048612213\\
116	0.00603679322128945\\
117	0.00603671453529285\\
118	0.00603663440457184\\
119	0.00603655280523012\\
120	0.00603646971303127\\
121	0.00603638510339442\\
122	0.00603629895138968\\
123	0.00603621123173367\\
124	0.00603612191878477\\
125	0.00603603098653822\\
126	0.00603593840862124\\
127	0.0060358441582879\\
128	0.0060357482084139\\
129	0.00603565053149122\\
130	0.00603555109962257\\
131	0.00603544988451557\\
132	0.00603534685747704\\
133	0.00603524198940678\\
134	0.00603513525079143\\
135	0.00603502661169791\\
136	0.00603491604176677\\
137	0.00603480351020528\\
138	0.00603468898578028\\
139	0.00603457243681071\\
140	0.00603445383116012\\
141	0.00603433313622849\\
142	0.0060342103189442\\
143	0.00603408534575543\\
144	0.00603395818262127\\
145	0.00603382879500259\\
146	0.00603369714785249\\
147	0.00603356320560641\\
148	0.0060334269321718\\
149	0.0060332882909176\\
150	0.00603314724466295\\
151	0.00603300375566584\\
152	0.00603285778561102\\
153	0.0060327092955976\\
154	0.00603255824612614\\
155	0.00603240459708521\\
156	0.00603224830773736\\
157	0.00603208933670466\\
158	0.00603192764195358\\
159	0.00603176318077928\\
160	0.00603159590978937\\
161	0.0060314257848869\\
162	0.00603125276125283\\
163	0.00603107679332769\\
164	0.00603089783479258\\
165	0.00603071583854953\\
166	0.00603053075670093\\
167	0.00603034254052844\\
168	0.00603015114047072\\
169	0.0060299565061008\\
170	0.00602975858610219\\
171	0.00602955732824444\\
172	0.00602935267935756\\
173	0.00602914458530571\\
174	0.00602893299095985\\
175	0.0060287178401695\\
176	0.00602849907573345\\
177	0.00602827663936958\\
178	0.00602805047168352\\
179	0.00602782051213645\\
180	0.00602758669901166\\
181	0.00602734896938018\\
182	0.00602710725906535\\
183	0.00602686150260615\\
184	0.00602661163321953\\
185	0.00602635758276164\\
186	0.00602609928168786\\
187	0.00602583665901176\\
188	0.00602556964226287\\
189	0.0060252981574434\\
190	0.00602502212898375\\
191	0.00602474147969693\\
192	0.00602445613073175\\
193	0.00602416600152512\\
194	0.00602387100975294\\
195	0.00602357107128012\\
196	0.00602326610010947\\
197	0.00602295600832944\\
198	0.00602264070606097\\
199	0.0060223201014031\\
200	0.00602199410037796\\
201	0.00602166260687438\\
202	0.00602132552259091\\
203	0.00602098274697774\\
204	0.00602063417717792\\
205	0.00602027970796776\\
206	0.00601991923169633\\
207	0.00601955263822441\\
208	0.00601917981486271\\
209	0.00601880064630949\\
210	0.00601841501458756\\
211	0.00601802279898083\\
212	0.00601762387597035\\
213	0.00601721811916998\\
214	0.00601680539926162\\
215	0.00601638558393026\\
216	0.0060159585377985\\
217	0.00601552412236119\\
218	0.00601508219591953\\
219	0.00601463261351521\\
220	0.00601417522686442\\
221	0.00601370988429164\\
222	0.00601323643066339\\
223	0.0060127547073219\\
224	0.00601226455201865\\
225	0.00601176579884772\\
226	0.00601125827817919\\
227	0.00601074181659219\\
228	0.00601021623680778\\
229	0.00600968135762151\\
230	0.00600913699383562\\
231	0.00600858295619075\\
232	0.00600801905129705\\
233	0.00600744508156469\\
234	0.0060068608451333\\
235	0.0060062661358006\\
236	0.00600566074294979\\
237	0.00600504445147547\\
238	0.00600441704170816\\
239	0.00600377828933684\\
240	0.00600312796532969\\
241	0.00600246583585235\\
242	0.00600179166218383\\
243	0.00600110520062961\\
244	0.00600040620243174\\
245	0.00599969441367572\\
246	0.00599896957519369\\
247	0.00599823142246401\\
248	0.00599747968550672\\
249	0.00599671408877461\\
250	0.00599593435104\\
251	0.00599514018527647\\
252	0.00599433129853582\\
253	0.00599350739181975\\
254	0.0059926681599462\\
255	0.00599181329141022\\
256	0.00599094246823893\\
257	0.00599005536584096\\
258	0.00598915165284961\\
259	0.00598823099095999\\
260	0.00598729303475947\\
261	0.00598633743155098\\
262	0.0059853638211689\\
263	0.0059843718357903\\
264	0.0059833610997498\\
265	0.00598233122934154\\
266	0.00598128183261407\\
267	0.0059802125091595\\
268	0.00597912284989707\\
269	0.00597801243685141\\
270	0.00597688084292488\\
271	0.00597572763166461\\
272	0.00597455235702391\\
273	0.0059733545631182\\
274	0.00597213378397527\\
275	0.00597088954328013\\
276	0.00596962135411411\\
277	0.00596832871868851\\
278	0.00596701112807231\\
279	0.00596566806191407\\
280	0.0059642989881579\\
281	0.00596290336275316\\
282	0.00596148062935757\\
283	0.00596003021903368\\
284	0.0059585515499382\\
285	0.00595704402700358\\
286	0.00595550704161166\\
287	0.00595393997125856\\
288	0.00595234217921039\\
289	0.00595071301414884\\
290	0.00594905180980607\\
291	0.00594735788458802\\
292	0.00594563054118526\\
293	0.00594386906617028\\
294	0.00594207272958046\\
295	0.00594024078448527\\
296	0.00593837246653645\\
297	0.00593646699349897\\
298	0.00593452356476108\\
299	0.00593254136082331\\
300	0.00593051954277351\\
301	0.00592845725174483\\
302	0.00592635360834158\\
303	0.00592420771203587\\
304	0.00592201864053301\\
305	0.0059197854490976\\
306	0.00591750716982406\\
307	0.00591518281083395\\
308	0.00591281135547901\\
309	0.00591039176156725\\
310	0.00590792296046258\\
311	0.00590540385611892\\
312	0.00590283332406944\\
313	0.00590021021036055\\
314	0.00589753333042427\\
315	0.00589480146791016\\
316	0.00589201337346598\\
317	0.0058891677634614\\
318	0.00588626331865267\\
319	0.00588329868278457\\
320	0.00588027246112666\\
321	0.0058771832189402\\
322	0.00587402947987144\\
323	0.0058708097242668\\
324	0.00586752238740279\\
325	0.00586416585761891\\
326	0.0058607384743371\\
327	0.00585723852597493\\
328	0.00585366424790664\\
329	0.00585001382025991\\
330	0.00584628536560832\\
331	0.00584247694659282\\
332	0.00583858656350006\\
333	0.00583461215180657\\
334	0.00583055157955604\\
335	0.00582640264457449\\
336	0.00582216307181674\\
337	0.0058178305107879\\
338	0.00581340253306314\\
339	0.0058088766299555\\
340	0.00580425021035672\\
341	0.00579952059865069\\
342	0.00579468503224592\\
343	0.00578974065955761\\
344	0.00578468453962854\\
345	0.00577951363976895\\
346	0.00577422483167486\\
347	0.00576881488723711\\
348	0.00576328047393433\\
349	0.00575761814956387\\
350	0.00575182435598938\\
351	0.00574589541383171\\
352	0.00573982751635916\\
353	0.00573361672235197\\
354	0.00572725894847346\\
355	0.00572074996134833\\
356	0.00571408536631671\\
357	0.00570726059326061\\
358	0.00570027088200122\\
359	0.00569311126585978\\
360	0.00568577655283809\\
361	0.00567826130420807\\
362	0.00567055981040699\\
363	0.00566266606433347\\
364	0.00565457373248421\\
365	0.00564627612493743\\
366	0.00563776616607646\\
367	0.00562903636909487\\
368	0.00562007881702173\\
369	0.00561088513657466\\
370	0.00560144652750161\\
371	0.00559175389375945\\
372	0.00558179789037141\\
373	0.00557156895084197\\
374	0.00556105733721312\\
375	0.00555025316917609\\
376	0.00553914644108116\\
377	0.00552772706280232\\
378	0.00551598490973305\\
379	0.00550390987707951\\
380	0.00549149193592538\\
381	0.00547872118685052\\
382	0.00546558790448464\\
383	0.0054520825630505\\
384	0.00543819582848082\\
385	0.00542391849660757\\
386	0.00540924134281841\\
387	0.00539415472574238\\
388	0.00537864806235072\\
389	0.00536270919272062\\
390	0.00534632333267171\\
391	0.00532947120568322\\
392	0.00531212333194729\\
393	0.00529424825184008\\
394	0.00527581284121551\\
395	0.00525678251559689\\
396	0.00523712156378698\\
397	0.00521679365642495\\
398	0.00519576258982919\\
399	0.00517399334701273\\
400	0.00515145359893942\\
401	0.00512811592022666\\
402	0.00510396008989104\\
403	0.00507897457342055\\
404	0.00505316281341298\\
405	0.00502654890918629\\
406	0.00499919242962357\\
407	0.00497119736069615\\
408	0.00494277929217888\\
409	0.0049146626275627\\
410	0.00488691175444974\\
411	0.00485959695242314\\
412	0.00483279448306534\\
413	0.00480658649718681\\
414	0.00478106067117138\\
415	0.00475630951499183\\
416	0.00473242920785781\\
417	0.0047095176753174\\
418	0.00468767160117613\\
419	0.00466698195385279\\
420	0.00464752785040454\\
421	0.00462936801802073\\
422	0.00461252901153922\\
423	0.00459698912504476\\
424	0.00458252040870373\\
425	0.00456843631081279\\
426	0.00455474838043038\\
427	0.00454146452366907\\
428	0.00452858822431064\\
429	0.00451611773544137\\
430	0.00450404487672025\\
431	0.00449235441662874\\
432	0.00448102327740238\\
433	0.00447001987603998\\
434	0.00445930405550695\\
435	0.0044488278667083\\
436	0.00443853618507812\\
437	0.0044283692003319\\
438	0.00441826682854435\\
439	0.00440820415116507\\
440	0.00439817136788632\\
441	0.00438815749405422\\
442	0.0043781504687486\\
443	0.00436813731957653\\
444	0.00435810438547989\\
445	0.00434803759671732\\
446	0.00433792285124789\\
447	0.00432774646157792\\
448	0.00431749565613897\\
449	0.0043071591014906\\
450	0.00429672738417001\\
451	0.00428619335106061\\
452	0.00427555179762931\\
453	0.00426479752868549\\
454	0.00425392541315694\\
455	0.00424293044073412\\
456	0.00423180777714498\\
457	0.00422055281468181\\
458	0.00420916121386276\\
459	0.00419762893149913\\
460	0.00418595223013638\\
461	0.00417412766416368\\
462	0.00416215203923854\\
463	0.0041500223447995\\
464	0.00413773566535698\\
465	0.00412528912332364\\
466	0.00411267988230371\\
467	0.00409990514827473\\
468	0.00408696216842916\\
469	0.00407384822755778\\
470	0.00406056064202319\\
471	0.00404709675159173\\
472	0.00403345390966303\\
473	0.00401962947274033\\
474	0.00400562079028181\\
475	0.00399142519628373\\
476	0.00397704000392844\\
477	0.00396246250337754\\
478	0.0039476899592021\\
479	0.00393271960750239\\
480	0.00391754865278553\\
481	0.00390217426468383\\
482	0.00388659357460285\\
483	0.00387080367238742\\
484	0.00385480160307617\\
485	0.00383858436378452\\
486	0.00382214890070395\\
487	0.00380549210614036\\
488	0.00378861081544281\\
489	0.00377150180369735\\
490	0.00375416178217399\\
491	0.00373658739451004\\
492	0.00371877521260873\\
493	0.00370072173222481\\
494	0.00368242336820189\\
495	0.00366387644931887\\
496	0.00364507721269443\\
497	0.0036260217976926\\
498	0.00360670623926831\\
499	0.0035871264606905\\
500	0.00356727826558349\\
501	0.00354715732922336\\
502	0.00352675918902258\\
503	0.00350607923413049\\
504	0.00348511269407485\\
505	0.00346385462636618\\
506	0.00344229990298504\\
507	0.00342044319567329\\
508	0.00339827895995153\\
509	0.00337580141779204\\
510	0.00335300453888381\\
511	0.00332988202044153\\
512	0.00330642726552966\\
513	0.00328263335990133\\
514	0.00325849304739117\\
515	0.00323399870395426\\
516	0.0032091423105132\\
517	0.00318391542486947\\
518	0.00315830915305462\\
519	0.00313231412065579\\
520	0.00310592044485016\\
521	0.00307911770814284\\
522	0.00305189493513307\\
523	0.00302424057405301\\
524	0.00299614248535714\\
525	0.002967587940314\\
526	0.0029385636334032\\
527	0.00290905571339415\\
528	0.00287904983933047\\
529	0.00284853126932706\\
530	0.0028174849921919\\
531	0.00278589591474815\\
532	0.00275374912178569\\
533	0.00272103023008121\\
534	0.00268772585133614\\
535	0.00265382415755588\\
536	0.00261931558188282\\
537	0.00258419417888923\\
538	0.00254846100558979\\
539	0.00251216052959282\\
540	0.00247532628282182\\
541	0.00243788997122817\\
542	0.00239976415995061\\
543	0.00236075191454028\\
544	0.00232080359268986\\
545	0.00227985780608986\\
546	0.00223785281570349\\
547	0.00219472782639527\\
548	0.00215041177863719\\
549	0.00210482397663955\\
550	0.00205785625152206\\
551	0.00200963534708887\\
552	0.00196294539021133\\
553	0.00191846763346604\\
554	0.00187458162861202\\
555	0.00183052002201858\\
556	0.00178605773398743\\
557	0.00174121417877548\\
558	0.00169604534204883\\
559	0.00165063743477819\\
560	0.00160552143434389\\
561	0.00156143381812764\\
562	0.00151771985037248\\
563	0.00147385304555012\\
564	0.00142980369459937\\
565	0.00138559882693199\\
566	0.00134127297237882\\
567	0.00129686098742586\\
568	0.00125239367639895\\
569	0.00120835592750733\\
570	0.00116438756191665\\
571	0.00112018262134879\\
572	0.00107576048997661\\
573	0.0010311496404989\\
574	0.00098638150776886\\
575	0.000941490258173559\\
576	0.000896512856511153\\
577	0.000851489130996976\\
578	0.000806461805889532\\
579	0.000761476487129886\\
580	0.000716581584829257\\
581	0.000671828152282903\\
582	0.00062726961623216\\
583	0.000582961366832659\\
584	0.000538960168209238\\
585	0.000495323341784078\\
586	0.000452107665645342\\
587	0.000409367927260636\\
588	0.000367155072958956\\
589	0.0003255139412327\\
590	0.000284480716949123\\
591	0.000244080684870839\\
592	0.000204328087251262\\
593	0.000165233217676712\\
594	0.000126870159937008\\
595	8.96111722547525e-05\\
596	5.4266094523851e-05\\
597	2.28062284332056e-05\\
598	2.9204464504877e-07\\
599	0\\
600	0\\
};
\addplot [color=black!50!mycolor20,solid,forget plot]
  table[row sep=crcr]{%
1	0.00604035455130753\\
2	0.00604034491250777\\
3	0.00604033506129769\\
4	0.00604032499311465\\
5	0.00604031470330184\\
6	0.00604030418710641\\
7	0.00604029343967774\\
8	0.00604028245606537\\
9	0.00604027123121725\\
10	0.00604025975997777\\
11	0.00604024803708565\\
12	0.00604023605717213\\
13	0.00604022381475875\\
14	0.00604021130425541\\
15	0.00604019851995815\\
16	0.00604018545604706\\
17	0.00604017210658414\\
18	0.006040158465511\\
19	0.00604014452664662\\
20	0.00604013028368517\\
21	0.00604011573019356\\
22	0.00604010085960915\\
23	0.00604008566523733\\
24	0.00604007014024908\\
25	0.00604005427767858\\
26	0.00604003807042058\\
27	0.00604002151122794\\
28	0.00604000459270902\\
29	0.00603998730732511\\
30	0.00603996964738761\\
31	0.00603995160505559\\
32	0.00603993317233276\\
33	0.00603991434106494\\
34	0.00603989510293712\\
35	0.00603987544947057\\
36	0.00603985537202005\\
37	0.0060398348617708\\
38	0.00603981390973551\\
39	0.00603979250675143\\
40	0.00603977064347726\\
41	0.00603974831039\\
42	0.00603972549778182\\
43	0.0060397021957569\\
44	0.0060396783942282\\
45	0.00603965408291424\\
46	0.00603962925133564\\
47	0.00603960388881196\\
48	0.00603957798445819\\
49	0.00603955152718133\\
50	0.00603952450567691\\
51	0.00603949690842561\\
52	0.00603946872368945\\
53	0.00603943993950841\\
54	0.0060394105436967\\
55	0.00603938052383908\\
56	0.00603934986728715\\
57	0.00603931856115561\\
58	0.00603928659231845\\
59	0.00603925394740509\\
60	0.00603922061279656\\
61	0.00603918657462144\\
62	0.00603915181875213\\
63	0.00603911633080062\\
64	0.00603908009611462\\
65	0.00603904309977344\\
66	0.00603900532658388\\
67	0.00603896676107607\\
68	0.00603892738749933\\
69	0.00603888718981799\\
70	0.00603884615170701\\
71	0.00603880425654783\\
72	0.00603876148742409\\
73	0.00603871782711716\\
74	0.00603867325810185\\
75	0.00603862776254197\\
76	0.00603858132228596\\
77	0.00603853391886239\\
78	0.00603848553347546\\
79	0.00603843614700055\\
80	0.00603838573997964\\
81	0.00603833429261675\\
82	0.00603828178477338\\
83	0.00603822819596396\\
84	0.00603817350535117\\
85	0.00603811769174128\\
86	0.00603806073357962\\
87	0.00603800260894578\\
88	0.00603794329554906\\
89	0.00603788277072365\\
90	0.00603782101142409\\
91	0.00603775799422036\\
92	0.00603769369529337\\
93	0.00603762809043015\\
94	0.0060375611550191\\
95	0.00603749286404532\\
96	0.00603742319208591\\
97	0.00603735211330515\\
98	0.0060372796014499\\
99	0.00603720562984483\\
100	0.00603713017138775\\
101	0.00603705319854488\\
102	0.00603697468334623\\
103	0.00603689459738088\\
104	0.00603681291179239\\
105	0.00603672959727415\\
106	0.00603664462406469\\
107	0.0060365579619432\\
108	0.00603646958022498\\
109	0.00603637944775674\\
110	0.00603628753291225\\
111	0.0060361938035878\\
112	0.0060360982271977\\
113	0.0060360007706699\\
114	0.00603590140044156\\
115	0.00603580008245474\\
116	0.00603569678215202\\
117	0.00603559146447224\\
118	0.00603548409384623\\
119	0.00603537463419263\\
120	0.00603526304891366\\
121	0.00603514930089098\\
122	0.00603503335248161\\
123	0.0060349151655138\\
124	0.00603479470128311\\
125	0.00603467192054829\\
126	0.00603454678352742\\
127	0.00603441924989387\\
128	0.00603428927877249\\
129	0.00603415682873569\\
130	0.00603402185779955\\
131	0.0060338843234201\\
132	0.00603374418248944\\
133	0.00603360139133191\\
134	0.00603345590570039\\
135	0.00603330768077241\\
136	0.00603315667114645\\
137	0.00603300283083806\\
138	0.00603284611327618\\
139	0.00603268647129906\\
140	0.00603252385715058\\
141	0.00603235822247626\\
142	0.00603218951831917\\
143	0.00603201769511605\\
144	0.00603184270269301\\
145	0.0060316644902614\\
146	0.00603148300641334\\
147	0.00603129819911738\\
148	0.00603111001571373\\
149	0.00603091840290947\\
150	0.00603072330677369\\
151	0.00603052467273213\\
152	0.00603032244556186\\
153	0.00603011656938548\\
154	0.00602990698766532\\
155	0.00602969364319684\\
156	0.00602947647810232\\
157	0.0060292554338236\\
158	0.00602903045111478\\
159	0.00602880147003427\\
160	0.0060285684299365\\
161	0.00602833126946292\\
162	0.00602808992653266\\
163	0.00602784433833239\\
164	0.00602759444130564\\
165	0.00602734017114131\\
166	0.00602708146276157\\
167	0.00602681825030869\\
168	0.00602655046713143\\
169	0.00602627804576997\\
170	0.00602600091794035\\
171	0.00602571901451754\\
172	0.00602543226551752\\
173	0.00602514060007818\\
174	0.006024843946439\\
175	0.00602454223191927\\
176	0.00602423538289514\\
177	0.00602392332477499\\
178	0.00602360598197339\\
179	0.00602328327788333\\
180	0.00602295513484693\\
181	0.00602262147412411\\
182	0.00602228221585955\\
183	0.00602193727904763\\
184	0.00602158658149546\\
185	0.00602123003978353\\
186	0.00602086756922437\\
187	0.00602049908381879\\
188	0.00602012449620984\\
189	0.006019743717634\\
190	0.00601935665787035\\
191	0.00601896322518649\\
192	0.00601856332628224\\
193	0.00601815686623016\\
194	0.00601774374841347\\
195	0.00601732387446088\\
196	0.00601689714417841\\
197	0.00601646345547823\\
198	0.00601602270430429\\
199	0.00601557478455476\\
200	0.00601511958800134\\
201	0.00601465700420523\\
202	0.00601418692042981\\
203	0.00601370922155019\\
204	0.00601322378995926\\
205	0.00601273050547047\\
206	0.0060122292452175\\
207	0.00601171988355051\\
208	0.00601120229192933\\
209	0.0060106763388133\\
210	0.00601014188954833\\
211	0.00600959880625064\\
212	0.00600904694768804\\
213	0.00600848616915798\\
214	0.00600791632236334\\
215	0.00600733725528554\\
216	0.00600674881205547\\
217	0.00600615083282223\\
218	0.00600554315362001\\
219	0.00600492560623318\\
220	0.00600429801805994\\
221	0.00600366021197484\\
222	0.00600301200619012\\
223	0.00600235321411653\\
224	0.00600168364422354\\
225	0.0060010030998996\\
226	0.00600031137931222\\
227	0.0059996082752687\\
228	0.0059988935750774\\
229	0.00599816706040992\\
230	0.0059974285071646\\
231	0.0059966776853314\\
232	0.00599591435885837\\
233	0.00599513828552009\\
234	0.00599434921678806\\
235	0.00599354689770343\\
236	0.00599273106675183\\
237	0.00599190145574067\\
238	0.00599105778967874\\
239	0.00599019978665826\\
240	0.00598932715773907\\
241	0.00598843960683508\\
242	0.00598753683060262\\
243	0.00598661851833054\\
244	0.00598568435183174\\
245	0.00598473400533584\\
246	0.00598376714538252\\
247	0.00598278343071531\\
248	0.00598178251217499\\
249	0.00598076403259272\\
250	0.00597972762668163\\
251	0.00597867292092701\\
252	0.00597759953347416\\
253	0.00597650707401343\\
254	0.00597539514366176\\
255	0.00597426333484019\\
256	0.00597311123114652\\
257	0.005971938407222\\
258	0.00597074442861153\\
259	0.00596952885161539\\
260	0.00596829122313056\\
261	0.00596703108047722\\
262	0.00596574795120211\\
263	0.00596444135284604\\
264	0.00596311079268821\\
265	0.00596175576758368\\
266	0.00596037576370285\\
267	0.00595897025622848\\
268	0.00595753870903681\\
269	0.00595608057436188\\
270	0.00595459529244277\\
271	0.00595308229115259\\
272	0.00595154098560904\\
273	0.00594997077776536\\
274	0.00594837105598171\\
275	0.00594674119457572\\
276	0.00594508055335192\\
277	0.00594338847710932\\
278	0.00594166429512658\\
279	0.00593990732062399\\
280	0.00593811685020162\\
281	0.00593629216325293\\
282	0.00593443252135306\\
283	0.00593253716762066\\
284	0.00593060532605261\\
285	0.00592863620083044\\
286	0.00592662897559697\\
287	0.00592458281270248\\
288	0.00592249685241824\\
289	0.00592037021211663\\
290	0.00591820198541585\\
291	0.00591599124128818\\
292	0.00591373702313046\\
293	0.00591143834779619\\
294	0.00590909420458879\\
295	0.00590670355421709\\
296	0.00590426532771432\\
297	0.00590177842532272\\
298	0.00589924171534324\\
299	0.00589665403294819\\
300	0.00589401417896816\\
301	0.00589132091877082\\
302	0.00588857298127079\\
303	0.00588576905794149\\
304	0.00588290780192623\\
305	0.0058799878273027\\
306	0.00587700770851719\\
307	0.00587396597990175\\
308	0.00587086113495447\\
309	0.00586769162621109\\
310	0.00586445586609552\\
311	0.00586115222654217\\
312	0.00585777903781928\\
313	0.0058543345873161\\
314	0.00585081711820609\\
315	0.00584722482787601\\
316	0.00584355586639754\\
317	0.00583980833493617\\
318	0.00583598028405877\\
319	0.00583206971192252\\
320	0.00582807456232503\\
321	0.0058239927225891\\
322	0.00581982202124891\\
323	0.00581556022549524\\
324	0.00581120503832473\\
325	0.0058067540953129\\
326	0.00580220496087143\\
327	0.00579755512370845\\
328	0.00579280199116947\\
329	0.00578794288452105\\
330	0.00578297503252372\\
331	0.00577789556368191\\
332	0.00577270149724791\\
333	0.00576738973298516\\
334	0.00576195703974413\\
335	0.00575640004094692\\
336	0.00575071519598767\\
337	0.00574489878061114\\
338	0.0057389468653358\\
339	0.00573285529206363\\
340	0.0057266196498375\\
341	0.00572023525148013\\
342	0.00571369711339678\\
343	0.00570699993620161\\
344	0.00570013810097834\\
345	0.00569310572673465\\
346	0.00568589673648015\\
347	0.0056785048609832\\
348	0.00567092364497663\\
349	0.00566314645599178\\
350	0.00565516649457982\\
351	0.00564697679917844\\
352	0.0056385702728284\\
353	0.00562993971370758\\
354	0.00562107784745955\\
355	0.00561197736864539\\
356	0.00560263100200876\\
357	0.00559303154919701\\
358	0.00558317192088877\\
359	0.00557304518645487\\
360	0.00556264462551128\\
361	0.00555196377299146\\
362	0.00554099644978443\\
363	0.00552973676717171\\
364	0.0055181790880087\\
365	0.00550631792034881\\
366	0.00549414770945814\\
367	0.00548166248157145\\
368	0.00546885527818469\\
369	0.00545571730669379\\
370	0.00544223620615092\\
371	0.00542839179521135\\
372	0.00541416081700759\\
373	0.00539951832034958\\
374	0.00538443766169766\\
375	0.00536889081424874\\
376	0.00535284846039011\\
377	0.00533627990800812\\
378	0.00531915330324797\\
379	0.00530143601499595\\
380	0.00528309520580291\\
381	0.005264098651944\\
382	0.00524441589473051\\
383	0.00522401983064083\\
384	0.00520288888166286\\
385	0.00518100993417347\\
386	0.00515838231517751\\
387	0.00513502332663846\\
388	0.0051109739445012\\
389	0.00508630711750994\\
390	0.00506114078397471\\
391	0.00503610410007305\\
392	0.00501125655682597\\
393	0.00498664792446473\\
394	0.00496233283192958\\
395	0.00493837095736671\\
396	0.00491482711891741\\
397	0.00489177121564937\\
398	0.00486927795068279\\
399	0.00484742624496875\\
400	0.00482629821873276\\
401	0.0048059775741019\\
402	0.00478654718902244\\
403	0.00476808569265558\\
404	0.00475066251264518\\
405	0.00473433095851641\\
406	0.00471911843610295\\
407	0.00470501322286141\\
408	0.00469189080838035\\
409	0.00467909168690972\\
410	0.00466662958433403\\
411	0.0046545159743818\\
412	0.00464275949432454\\
413	0.00463136529122577\\
414	0.00462033432580973\\
415	0.00460966263109742\\
416	0.00459934058267878\\
417	0.00458935216047305\\
418	0.00457967434280781\\
419	0.00457027675662039\\
420	0.00456112164386289\\
421	0.00455216435069586\\
422	0.00454335500797968\\
423	0.00453464140084274\\
424	0.00452598108331289\\
425	0.00451736707410352\\
426	0.00450879128376258\\
427	0.0045002445499172\\
428	0.00449171671181201\\
429	0.00448319673197135\\
430	0.00447467287969468\\
431	0.0044661329667574\\
432	0.00445756464614725\\
433	0.00444895577249972\\
434	0.00444029480843122\\
435	0.004431571252739\\
436	0.00442277607768038\\
437	0.00441390208889142\\
438	0.00440494409640422\\
439	0.00439589753527377\\
440	0.00438675784669873\\
441	0.00437752052518183\\
442	0.0043681811668333\\
443	0.00435873551675346\\
444	0.00434917951305033\\
445	0.0043395093245137\\
446	0.00432972137728854\\
447	0.00431981236625983\\
448	0.00430977924699373\\
449	0.00429961920502578\\
450	0.00428932960160009\\
451	0.00427890789942998\\
452	0.00426835159385037\\
453	0.00425765821680932\\
454	0.00424682533924926\\
455	0.00423585057161851\\
456	0.00422473156234556\\
457	0.00421346599421916\\
458	0.00420205157877209\\
459	0.00419048604896569\\
460	0.00417876715071148\\
461	0.00416689263402309\\
462	0.00415486024482622\\
463	0.00414266771859025\\
464	0.00413031277684553\\
465	0.00411779312564403\\
466	0.00410510645374804\\
467	0.0040922504305949\\
468	0.00407922270410051\\
469	0.00406602089837641\\
470	0.00405264261144239\\
471	0.00403908541301646\\
472	0.00402534684245244\\
473	0.00401142440687092\\
474	0.00399731557948974\\
475	0.00398301779810677\\
476	0.0039685284636279\\
477	0.00395384493851744\\
478	0.0039389645451738\\
479	0.00392388456423045\\
480	0.00390860223278089\\
481	0.00389311474252117\\
482	0.00387741923780042\\
483	0.00386151281356359\\
484	0.00384539251316722\\
485	0.00382905532604283\\
486	0.00381249818518186\\
487	0.00379571796441353\\
488	0.00377871147545074\\
489	0.00376147546468013\\
490	0.00374400660966948\\
491	0.00372630151536363\\
492	0.00370835670993593\\
493	0.0036901686402604\\
494	0.00367173366696607\\
495	0.00365304805903214\\
496	0.00363410798788008\\
497	0.00361490952091579\\
498	0.00359544861447277\\
499	0.00357572110610464\\
500	0.00355572270617314\\
501	0.0035354489886765\\
502	0.00351489538126131\\
503	0.00349405715436267\\
504	0.00347292940941807\\
505	0.00345150706610567\\
506	0.00342978484856397\\
507	0.00340775727056105\\
508	0.00338541861959726\\
509	0.00336276293994635\\
510	0.00333978401467161\\
511	0.00331647534669245\\
512	0.00329283013903121\\
513	0.00326884127444044\\
514	0.0032445012947022\\
515	0.00321980238000906\\
516	0.00319473632898957\\
517	0.00316929454013484\\
518	0.00314346799563162\\
519	0.00311724724892223\\
520	0.00309062241771182\\
521	0.00306358318464741\\
522	0.00303611880853064\\
523	0.00300821814972765\\
524	0.0029798697144472\\
525	0.00295106172382384\\
526	0.00292178221532365\\
527	0.00289201918595378\\
528	0.00286176078938083\\
529	0.00283099560279367\\
530	0.00279971298377082\\
531	0.00276790353115984\\
532	0.00273555964225991\\
533	0.00270267619243534\\
534	0.00266925192065413\\
535	0.00263529296920442\\
536	0.00260086606279785\\
537	0.00256597321413507\\
538	0.00253054433721722\\
539	0.00249444974095169\\
540	0.00245754414628867\\
541	0.00241976540072678\\
542	0.00238103965629058\\
543	0.0023413125141704\\
544	0.00230052829688574\\
545	0.00225862371766199\\
546	0.00221552624303288\\
547	0.00217115084894059\\
548	0.00212537826171806\\
549	0.00207910769893968\\
550	0.00203457920162987\\
551	0.00199229323501405\\
552	0.00194984963616151\\
553	0.00190703680818334\\
554	0.00186376888205866\\
555	0.00182006478878576\\
556	0.00177598546288734\\
557	0.00173161014088666\\
558	0.00168713765704981\\
559	0.00164367024345205\\
560	0.00160078372768793\\
561	0.00155771264955085\\
562	0.00151441038560274\\
563	0.00147090213189777\\
564	0.00142722036978999\\
565	0.00138339816260417\\
566	0.00133946689808092\\
567	0.00129553652878565\\
568	0.00125204062884596\\
569	0.00120834736434655\\
570	0.00116438695026368\\
571	0.00112018252877128\\
572	0.00107576046202357\\
573	0.00103114962906728\\
574	0.000986381502459647\\
575	0.000941490255454449\\
576	0.000896512855016981\\
577	0.000851489130158736\\
578	0.00080646180541855\\
579	0.000761476486870332\\
580	0.000716581584693804\\
581	0.000671828152219676\\
582	0.000627269616207832\\
583	0.000582961366825552\\
584	0.000538960168207982\\
585	0.000495323341784074\\
586	0.000452107665645344\\
587	0.000409367927260635\\
588	0.000367155072958959\\
589	0.0003255139412327\\
590	0.00028448071694912\\
591	0.000244080684870837\\
592	0.000204328087251259\\
593	0.00016523321767671\\
594	0.000126870159937007\\
595	8.96111722547517e-05\\
596	5.42660945238506e-05\\
597	2.28062284332055e-05\\
598	2.9204464504877e-07\\
599	0\\
600	0\\
};
\addplot [color=black!60!mycolor21,solid,forget plot]
  table[row sep=crcr]{%
1	0.00604030315615854\\
2	0.00604029213280765\\
3	0.00604028085850546\\
4	0.00604026932762131\\
5	0.00604025753440102\\
6	0.00604024547296424\\
7	0.00604023313730183\\
8	0.0060402205212733\\
9	0.00604020761860375\\
10	0.00604019442288126\\
11	0.00604018092755395\\
12	0.00604016712592705\\
13	0.00604015301115993\\
14	0.00604013857626299\\
15	0.00604012381409466\\
16	0.00604010871735813\\
17	0.0060400932785982\\
18	0.00604007749019795\\
19	0.00604006134437541\\
20	0.00604004483318011\\
21	0.00604002794848966\\
22	0.00604001068200617\\
23	0.00603999302525264\\
24	0.00603997496956928\\
25	0.00603995650610979\\
26	0.00603993762583747\\
27	0.00603991831952148\\
28	0.00603989857773269\\
29	0.00603987839083977\\
30	0.00603985774900511\\
31	0.00603983664218049\\
32	0.00603981506010304\\
33	0.00603979299229069\\
34	0.00603977042803793\\
35	0.00603974735641119\\
36	0.00603972376624438\\
37	0.00603969964613416\\
38	0.00603967498443524\\
39	0.0060396497692556\\
40	0.00603962398845151\\
41	0.00603959762962259\\
42	0.00603957068010675\\
43	0.00603954312697506\\
44	0.00603951495702644\\
45	0.00603948615678234\\
46	0.0060394567124814\\
47	0.0060394266100738\\
48	0.00603939583521576\\
49	0.00603936437326383\\
50	0.0060393322092691\\
51	0.00603929932797119\\
52	0.0060392657137925\\
53	0.00603923135083194\\
54	0.0060391962228589\\
55	0.00603916031330686\\
56	0.00603912360526713\\
57	0.0060390860814823\\
58	0.00603904772433975\\
59	0.00603900851586495\\
60	0.00603896843771466\\
61	0.00603892747117013\\
62	0.00603888559713016\\
63	0.00603884279610392\\
64	0.00603879904820394\\
65	0.00603875433313866\\
66	0.00603870863020521\\
67	0.00603866191828188\\
68	0.00603861417582054\\
69	0.00603856538083893\\
70	0.00603851551091298\\
71	0.00603846454316885\\
72	0.00603841245427489\\
73	0.0060383592204337\\
74	0.00603830481737387\\
75	0.00603824922034165\\
76	0.00603819240409257\\
77	0.00603813434288305\\
78	0.00603807501046174\\
79	0.00603801438006077\\
80	0.00603795242438711\\
81	0.00603788911561362\\
82	0.00603782442537004\\
83	0.00603775832473398\\
84	0.00603769078422181\\
85	0.0060376217737793\\
86	0.00603755126277238\\
87	0.00603747921997769\\
88	0.00603740561357308\\
89	0.00603733041112803\\
90	0.00603725357959399\\
91	0.00603717508529463\\
92	0.00603709489391608\\
93	0.00603701297049695\\
94	0.00603692927941852\\
95	0.00603684378439462\\
96	0.00603675644846161\\
97	0.00603666723396823\\
98	0.00603657610256547\\
99	0.00603648301519629\\
100	0.00603638793208539\\
101	0.00603629081272888\\
102	0.00603619161588401\\
103	0.00603609029955871\\
104	0.00603598682100124\\
105	0.00603588113668981\\
106	0.00603577320232221\\
107	0.00603566297280529\\
108	0.0060355504022446\\
109	0.00603543544393402\\
110	0.00603531805034534\\
111	0.00603519817311788\\
112	0.00603507576304822\\
113	0.00603495077007992\\
114	0.00603482314329318\\
115	0.00603469283089476\\
116	0.00603455978020791\\
117	0.00603442393766219\\
118	0.00603428524878369\\
119	0.00603414365818501\\
120	0.00603399910955568\\
121	0.0060338515456524\\
122	0.00603370090828966\\
123	0.00603354713833032\\
124	0.00603339017567638\\
125	0.00603322995925995\\
126	0.00603306642703442\\
127	0.00603289951596574\\
128	0.00603272916202395\\
129	0.00603255530017487\\
130	0.0060323778643721\\
131	0.00603219678754922\\
132	0.00603201200161214\\
133	0.00603182343743191\\
134	0.0060316310248376\\
135	0.0060314346926097\\
136	0.00603123436847357\\
137	0.0060310299790934\\
138	0.00603082145006639\\
139	0.0060306087059174\\
140	0.00603039167009382\\
141	0.00603017026496083\\
142	0.00602994441179719\\
143	0.00602971403079119\\
144	0.00602947904103725\\
145	0.00602923936053267\\
146	0.00602899490617513\\
147	0.00602874559376031\\
148	0.00602849133798023\\
149	0.0060282320524218\\
150	0.00602796764956598\\
151	0.00602769804078748\\
152	0.00602742313635467\\
153	0.00602714284543029\\
154	0.00602685707607232\\
155	0.00602656573523567\\
156	0.00602626872877395\\
157	0.00602596596144213\\
158	0.00602565733689928\\
159	0.00602534275771209\\
160	0.00602502212535866\\
161	0.00602469534023269\\
162	0.00602436230164816\\
163	0.00602402290784449\\
164	0.00602367705599188\\
165	0.00602332464219712\\
166	0.00602296556150962\\
167	0.00602259970792783\\
168	0.00602222697440578\\
169	0.00602184725285984\\
170	0.00602146043417563\\
171	0.00602106640821494\\
172	0.00602066506382275\\
173	0.00602025628883411\\
174	0.00601983997008088\\
175	0.00601941599339837\\
176	0.00601898424363137\\
177	0.00601854460464016\\
178	0.0060180969593057\\
179	0.00601764118953426\\
180	0.00601717717626128\\
181	0.00601670479945442\\
182	0.00601622393811539\\
183	0.00601573447028078\\
184	0.00601523627302126\\
185	0.0060147292224395\\
186	0.00601421319366627\\
187	0.00601368806085453\\
188	0.00601315369717146\\
189	0.00601260997478819\\
190	0.00601205676486678\\
191	0.00601149393754458\\
192	0.00601092136191525\\
193	0.00601033890600672\\
194	0.00600974643675521\\
195	0.00600914381997563\\
196	0.00600853092032748\\
197	0.00600790760127641\\
198	0.00600727372505088\\
199	0.00600662915259358\\
200	0.00600597374350736\\
201	0.00600530735599531\\
202	0.00600462984679469\\
203	0.00600394107110405\\
204	0.00600324088250367\\
205	0.00600252913286865\\
206	0.00600180567227416\\
207	0.00600107034889287\\
208	0.00600032300888391\\
209	0.00599956349627308\\
210	0.00599879165282408\\
211	0.00599800731790024\\
212	0.00599721032831651\\
213	0.00599640051818154\\
214	0.00599557771872936\\
215	0.00599474175814054\\
216	0.00599389246135252\\
217	0.00599302964985902\\
218	0.00599215314149819\\
219	0.00599126275022954\\
220	0.00599035828589953\\
221	0.00598943955399564\\
222	0.00598850635538901\\
223	0.00598755848606573\\
224	0.00598659573684676\\
225	0.00598561789309678\\
226	0.00598462473442191\\
227	0.00598361603435678\\
228	0.0059825915600411\\
229	0.00598155107188605\\
230	0.00598049432323093\\
231	0.00597942105999027\\
232	0.00597833102029219\\
233	0.00597722393410799\\
234	0.00597609952287374\\
235	0.00597495749910423\\
236	0.00597379756599957\\
237	0.005972619417045\\
238	0.00597142273560416\\
239	0.00597020719450606\\
240	0.00596897245562604\\
241	0.00596771816946078\\
242	0.00596644397469735\\
243	0.00596514949777626\\
244	0.00596383435244846\\
245	0.00596249813932576\\
246	0.00596114044542479\\
247	0.00595976084370408\\
248	0.00595835889259425\\
249	0.00595693413552139\\
250	0.00595548610042442\\
251	0.00595401429926767\\
252	0.00595251822755056\\
253	0.00595099736381837\\
254	0.00594945116917886\\
255	0.00594787908683215\\
256	0.00594628054162402\\
257	0.00594465493963533\\
258	0.00594300166782471\\
259	0.00594132009374463\\
260	0.00593960956535445\\
261	0.00593786941095441\\
262	0.005936098939254\\
263	0.00593429743954531\\
264	0.00593246418184334\\
265	0.00593059841689974\\
266	0.00592869937769454\\
267	0.00592676628006091\\
268	0.00592479832244917\\
269	0.00592279468568871\\
270	0.00592075453274704\\
271	0.00591867700848528\\
272	0.00591656123940989\\
273	0.00591440633341983\\
274	0.00591221137954805\\
275	0.00590997544769704\\
276	0.00590769758836703\\
277	0.00590537683237556\\
278	0.00590301219056673\\
279	0.00590060265350832\\
280	0.00589814719117372\\
281	0.00589564475260517\\
282	0.00589309426555345\\
283	0.00589049463608816\\
284	0.00588784474817003\\
285	0.00588514346317461\\
286	0.00588238961935384\\
287	0.00587958203121711\\
288	0.00587671948880924\\
289	0.00587380075685598\\
290	0.00587082457374034\\
291	0.00586778965026332\\
292	0.0058646946681321\\
293	0.00586153827810502\\
294	0.00585831909770823\\
295	0.00585503570842117\\
296	0.0058516866522094\\
297	0.00584827042726107\\
298	0.005844785482755\\
299	0.00584123021243587\\
300	0.00583760294666414\\
301	0.00583390194251364\\
302	0.00583012537284562\\
303	0.00582627131476884\\
304	0.00582233773560924\\
305	0.00581832247727876\\
306	0.00581422323984336\\
307	0.00581003756541961\\
308	0.00580576282366509\\
309	0.00580139619656889\\
310	0.00579693467624333\\
311	0.00579237510283586\\
312	0.00578771420092257\\
313	0.00578294857752554\\
314	0.00577807472093154\\
315	0.00577308899970213\\
316	0.00576798766021564\\
317	0.00576276682629832\\
318	0.00575742250012165\\
319	0.00575195056427633\\
320	0.0057463467852586\\
321	0.00574060681863975\\
322	0.00573472621622314\\
323	0.00572870043553005\\
324	0.00572252485198907\\
325	0.00571619477423348\\
326	0.0057097054628966\\
327	0.00570305215303784\\
328	0.00569623007880404\\
329	0.00568923449356538\\
330	0.00568206071621986\\
331	0.00567470417398735\\
332	0.00566716044728615\\
333	0.0056594253182353\\
334	0.00565149482309718\\
335	0.00564336531262988\\
336	0.00563503349789744\\
337	0.00562649645778915\\
338	0.00561775163546331\\
339	0.00560879679679947\\
340	0.00559962992549786\\
341	0.00559024902400292\\
342	0.00558065177867284\\
343	0.00557083503636195\\
344	0.00556079387954658\\
345	0.00555051953043704\\
346	0.00553999868927008\\
347	0.00552921699659402\\
348	0.00551815895815368\\
349	0.00550680787097888\\
350	0.00549514575425154\\
351	0.00548315328290356\\
352	0.00547080958467503\\
353	0.00545809225929875\\
354	0.00544497741749707\\
355	0.00543143971394483\\
356	0.00541745244928231\\
357	0.00540298796483707\\
358	0.00538801795719036\\
359	0.00537251368367673\\
360	0.00535644652525741\\
361	0.00533978882818451\\
362	0.00532251507603197\\
363	0.00530460349848151\\
364	0.00528603825579737\\
365	0.00526681238042775\\
366	0.00524693171268722\\
367	0.00522642014082995\\
368	0.00520532655642379\\
369	0.00518373409893584\\
370	0.00516206586647816\\
371	0.00514044460870371\\
372	0.00511889942685035\\
373	0.005097461684701\\
374	0.00507616541703685\\
375	0.0050550469394263\\
376	0.00503414834652415\\
377	0.00501351824451031\\
378	0.00499320950024386\\
379	0.0049732792138121\\
380	0.00495378852032667\\
381	0.00493480214886549\\
382	0.00491638764213609\\
383	0.00489861409876216\\
384	0.0048815502818452\\
385	0.00486526185955928\\
386	0.00484980747629532\\
387	0.0048352332527258\\
388	0.00482156524752625\\
389	0.0048087991710341\\
390	0.00479688634022394\\
391	0.00478524062734394\\
392	0.004773860368139\\
393	0.00476275899406963\\
394	0.00475194851101444\\
395	0.00474143907290504\\
396	0.00473123849955345\\
397	0.00472135174083783\\
398	0.00471178029418218\\
399	0.0047025215893457\\
400	0.0046935683648113\\
401	0.00468490807480182\\
402	0.0046765223856452\\
403	0.00466838684684614\\
404	0.00466047086612644\\
405	0.00465273816968375\\
406	0.00464514801062233\\
407	0.00463765747607518\\
408	0.00463022821741895\\
409	0.00462285557451509\\
410	0.00461553393591348\\
411	0.00460825674039417\\
412	0.00460101649287061\\
413	0.00459380480350022\\
414	0.00458661236496382\\
415	0.00457942916295125\\
416	0.00457224459312569\\
417	0.00456504808471705\\
418	0.00455782910114952\\
419	0.00455057744375803\\
420	0.00454328362249896\\
421	0.00453593924048389\\
422	0.00452853733886829\\
423	0.00452107263056984\\
424	0.00451354120769885\\
425	0.00450593911397278\\
426	0.00449826238226361\\
427	0.00449050707496815\\
428	0.0044826693259891\\
429	0.00447474538262651\\
430	0.00446673164499422\\
431	0.00445862470053646\\
432	0.00445042135051302\\
433	0.00444211862500887\\
434	0.00443371378317185\\
435	0.00442520429581949\\
436	0.00441658780808193\\
437	0.00440786208314939\\
438	0.00439902493336187\\
439	0.00439007420042411\\
440	0.00438100775895551\\
441	0.00437182351869122\\
442	0.00436251942513567\\
443	0.00435309345852253\\
444	0.00434354363100991\\
445	0.00433386798214939\\
446	0.00432406457285262\\
447	0.00431413147827598\\
448	0.00430406678026078\\
449	0.00429386856017406\\
450	0.00428353489313293\\
451	0.00427306384456706\\
452	0.0042624534691652\\
453	0.00425170180958279\\
454	0.00424080689494397\\
455	0.00422976673918413\\
456	0.00421857933929183\\
457	0.00420724267351705\\
458	0.00419575469961823\\
459	0.00418411335321687\\
460	0.00417231654631692\\
461	0.00416036216602138\\
462	0.00414824807344308\\
463	0.00413597210275987\\
464	0.00412353206031631\\
465	0.0041109257237\\
466	0.0040981508407988\\
467	0.00408520512884452\\
468	0.00407208627344699\\
469	0.00405879192761907\\
470	0.00404531971079127\\
471	0.00403166720781047\\
472	0.00401783196791326\\
473	0.00400381150366288\\
474	0.00398960328983551\\
475	0.0039752047622424\\
476	0.0039606133164768\\
477	0.00394582630657787\\
478	0.00393084104360233\\
479	0.00391565479409372\\
480	0.00390026477843745\\
481	0.00388466816908875\\
482	0.00386886208865883\\
483	0.00385284360784337\\
484	0.00383660974317631\\
485	0.00382015745459029\\
486	0.00380348364276359\\
487	0.00378658514623244\\
488	0.00376945873824518\\
489	0.0037521011233333\\
490	0.00373450893357197\\
491	0.00371667872450067\\
492	0.00369860697067291\\
493	0.0036802900608012\\
494	0.00366172429246222\\
495	0.00364290586632503\\
496	0.00362383087986406\\
497	0.00360449532051765\\
498	0.003584895058252\\
499	0.00356502583749236\\
500	0.003544883268384\\
501	0.0035244628173497\\
502	0.00350375979691629\\
503	0.00348276935479163\\
504	0.00346148646218569\\
505	0.0034399059013872\\
506	0.00341802225263116\\
507	0.00339582988032334\\
508	0.00337332291873124\\
509	0.0033504952573046\\
510	0.00332734052585997\\
511	0.00330385207995603\\
512	0.00328002298690368\\
513	0.00325584601300512\\
514	0.00323131361280798\\
515	0.00320641792140324\\
516	0.0031811507511027\\
517	0.00315550359421973\\
518	0.00312946763416657\\
519	0.00310303376769541\\
520	0.00307619264188358\\
521	0.00304893471043217\\
522	0.00302125031506175\\
523	0.00299312979930897\\
524	0.00296456366392906\\
525	0.00293554277548647\\
526	0.00290605864302008\\
527	0.00287610378238502\\
528	0.00284567218440825\\
529	0.00281475988321566\\
530	0.00278336563667194\\
531	0.00275149226689524\\
532	0.00271915008403821\\
533	0.00268641628363034\\
534	0.00265328435369643\\
535	0.00261968582869031\\
536	0.0025854688063535\\
537	0.00255052199012215\\
538	0.00251478365610303\\
539	0.00247819002717404\\
540	0.00244068417819549\\
541	0.00240220348343551\\
542	0.00236268161584073\\
543	0.00232205422193516\\
544	0.00228024789926472\\
545	0.00223717673401289\\
546	0.00219271866604341\\
547	0.00214821380497191\\
548	0.00210568170443776\\
549	0.00206477854581583\\
550	0.00202367834717305\\
551	0.00198207658025245\\
552	0.00193997595629076\\
553	0.00189740567257551\\
554	0.00185441906889823\\
555	0.00181108860982139\\
556	0.00176750323728597\\
557	0.00172451269574327\\
558	0.00168246236750703\\
559	0.00164022170205425\\
560	0.00159770829159438\\
561	0.00155494262970847\\
562	0.00151195463762704\\
563	0.00146877499999463\\
564	0.00142543401352537\\
565	0.00138195982240711\\
566	0.00133854778582265\\
567	0.00129544180006035\\
568	0.00125203948509083\\
569	0.00120834728092063\\
570	0.00116438693696481\\
571	0.0011201825245913\\
572	0.0010757604602725\\
573	0.00103114962823804\\
574	0.00098638150202912\\
575	0.000941490255217017\\
576	0.000896512854883693\\
577	0.000851489130084093\\
578	0.000806461805377778\\
579	0.00076147648684911\\
580	0.000716581584684075\\
581	0.000671828152215967\\
582	0.000627269616206792\\
583	0.000582961366825369\\
584	0.00053896016820798\\
585	0.00049532334178407\\
586	0.000452107665645336\\
587	0.000409367927260632\\
588	0.000367155072958954\\
589	0.000325513941232697\\
590	0.000284480716949123\\
591	0.000244080684870839\\
592	0.000204328087251263\\
593	0.000165233217676714\\
594	0.000126870159937009\\
595	8.96111722547525e-05\\
596	5.42660945238512e-05\\
597	2.28062284332059e-05\\
598	2.9204464504877e-07\\
599	0\\
600	0\\
};
\addplot [color=black!80!mycolor21,solid,forget plot]
  table[row sep=crcr]{%
1	0.00604026546700612\\
2	0.0060402533658538\\
3	0.00604024098238466\\
4	0.00604022831004924\\
5	0.00604021534214751\\
6	0.00604020207182548\\
7	0.00604018849207171\\
8	0.00604017459571358\\
9	0.00604016037541399\\
10	0.00604014582366729\\
11	0.0060401309327957\\
12	0.00604011569494533\\
13	0.00604010010208226\\
14	0.00604008414598843\\
15	0.00604006781825751\\
16	0.00604005111029071\\
17	0.00604003401329235\\
18	0.00604001651826554\\
19	0.0060399986160076\\
20	0.00603998029710551\\
21	0.00603996155193108\\
22	0.00603994237063621\\
23	0.00603992274314792\\
24	0.00603990265916339\\
25	0.00603988210814473\\
26	0.00603986107931386\\
27	0.00603983956164693\\
28	0.00603981754386914\\
29	0.0060397950144488\\
30	0.00603977196159192\\
31	0.00603974837323622\\
32	0.00603972423704519\\
33	0.006039699540402\\
34	0.0060396742704033\\
35	0.00603964841385282\\
36	0.00603962195725503\\
37	0.0060395948868083\\
38	0.00603956718839841\\
39	0.00603953884759133\\
40	0.00603950984962644\\
41	0.00603948017940923\\
42	0.00603944982150397\\
43	0.00603941876012617\\
44	0.00603938697913498\\
45	0.0060393544620254\\
46	0.00603932119192022\\
47	0.00603928715156196\\
48	0.00603925232330455\\
49	0.00603921668910482\\
50	0.00603918023051389\\
51	0.00603914292866831\\
52	0.00603910476428106\\
53	0.0060390657176325\\
54	0.0060390257685607\\
55	0.00603898489645214\\
56	0.00603894308023186\\
57	0.00603890029835354\\
58	0.00603885652878928\\
59	0.00603881174901931\\
60	0.00603876593602144\\
61	0.00603871906626033\\
62	0.00603867111567635\\
63	0.00603862205967457\\
64	0.00603857187311328\\
65	0.00603852053029232\\
66	0.00603846800494131\\
67	0.00603841427020753\\
68	0.00603835929864351\\
69	0.00603830306219463\\
70	0.00603824553218625\\
71	0.00603818667931065\\
72	0.00603812647361384\\
73	0.00603806488448191\\
74	0.00603800188062737\\
75	0.00603793743007505\\
76	0.00603787150014784\\
77	0.0060378040574521\\
78	0.00603773506786285\\
79	0.00603766449650872\\
80	0.00603759230775658\\
81	0.00603751846519583\\
82	0.00603744293162264\\
83	0.00603736566902364\\
84	0.00603728663855948\\
85	0.00603720580054809\\
86	0.00603712311444767\\
87	0.00603703853883924\\
88	0.00603695203140918\\
89	0.00603686354893113\\
90	0.00603677304724792\\
91	0.00603668048125302\\
92	0.00603658580487165\\
93	0.00603648897104176\\
94	0.00603638993169448\\
95	0.00603628863773451\\
96	0.0060361850390201\\
97	0.00603607908434256\\
98	0.0060359707214058\\
99	0.00603585989680519\\
100	0.00603574655600643\\
101	0.00603563064332395\\
102	0.00603551210189894\\
103	0.00603539087367727\\
104	0.00603526689938696\\
105	0.0060351401185154\\
106	0.00603501046928619\\
107	0.00603487788863583\\
108	0.00603474231218996\\
109	0.00603460367423942\\
110	0.00603446190771583\\
111	0.00603431694416723\\
112	0.0060341687137331\\
113	0.00603401714511918\\
114	0.00603386216557229\\
115	0.00603370370085445\\
116	0.00603354167521696\\
117	0.00603337601137441\\
118	0.00603320663047808\\
119	0.00603303345208935\\
120	0.0060328563941528\\
121	0.00603267537296921\\
122	0.00603249030316817\\
123	0.00603230109768058\\
124	0.00603210766771103\\
125	0.00603190992271004\\
126	0.00603170777034592\\
127	0.00603150111647674\\
128	0.00603128986512214\\
129	0.00603107391843494\\
130	0.00603085317667271\\
131	0.00603062753816934\\
132	0.00603039689930652\\
133	0.00603016115448515\\
134	0.00602992019609692\\
135	0.00602967391449568\\
136	0.00602942219796904\\
137	0.00602916493271007\\
138	0.006028902002789\\
139	0.00602863329012514\\
140	0.00602835867445888\\
141	0.00602807803332407\\
142	0.00602779124202049\\
143	0.0060274981735866\\
144	0.00602719869877273\\
145	0.00602689268601452\\
146	0.00602658000140678\\
147	0.00602626050867775\\
148	0.00602593406916396\\
149	0.00602560054178549\\
150	0.00602525978302194\\
151	0.00602491164688882\\
152	0.00602455598491491\\
153	0.00602419264611996\\
154	0.00602382147699359\\
155	0.00602344232147472\\
156	0.00602305502093198\\
157	0.00602265941414526\\
158	0.00602225533728789\\
159	0.00602184262391019\\
160	0.00602142110492413\\
161	0.00602099060858907\\
162	0.0060205509604988\\
163	0.00602010198356985\\
164	0.00601964349803142\\
165	0.00601917532141646\\
166	0.00601869726855445\\
167	0.00601820915156574\\
168	0.00601771077985748\\
169	0.00601720196012123\\
170	0.00601668249633258\\
171	0.00601615218975228\\
172	0.00601561083892954\\
173	0.00601505823970703\\
174	0.00601449418522819\\
175	0.00601391846594625\\
176	0.00601333086963577\\
177	0.00601273118140585\\
178	0.00601211918371597\\
179	0.00601149465639383\\
180	0.00601085737665546\\
181	0.00601020711912762\\
182	0.00600954365587254\\
183	0.00600886675641485\\
184	0.00600817618777094\\
185	0.00600747171448041\\
186	0.00600675309863981\\
187	0.00600602009993884\\
188	0.00600527247569823\\
189	0.0060045099809102\\
190	0.00600373236828059\\
191	0.00600293938827294\\
192	0.00600213078915456\\
193	0.00600130631704403\\
194	0.00600046571596032\\
195	0.00599960872787332\\
196	0.00599873509275525\\
197	0.00599784454863329\\
198	0.00599693683164267\\
199	0.00599601167608023\\
200	0.00599506881445815\\
201	0.00599410797755745\\
202	0.0059931288944809\\
203	0.00599213129270514\\
204	0.00599111489813133\\
205	0.00599007943513419\\
206	0.00598902462660888\\
207	0.00598795019401522\\
208	0.00598685585741859\\
209	0.0059857413355276\\
210	0.00598460634572723\\
211	0.00598345060410724\\
212	0.00598227382548555\\
213	0.00598107572342554\\
214	0.00597985601024703\\
215	0.00597861439703013\\
216	0.00597735059361167\\
217	0.00597606430857348\\
218	0.00597475524922178\\
219	0.00597342312155775\\
220	0.00597206763023817\\
221	0.00597068847852598\\
222	0.00596928536823059\\
223	0.00596785799963718\\
224	0.00596640607142497\\
225	0.00596492928057418\\
226	0.00596342732226176\\
227	0.00596189988974522\\
228	0.00596034667423549\\
229	0.00595876736475797\\
230	0.00595716164800227\\
231	0.00595552920816083\\
232	0.00595386972675607\\
233	0.00595218288245648\\
234	0.00595046835088112\\
235	0.00594872580439209\\
236	0.00594695491187441\\
237	0.00594515533850142\\
238	0.00594332674548392\\
239	0.00594146878979969\\
240	0.00593958112389889\\
241	0.00593766339537894\\
242	0.00593571524662091\\
243	0.00593373631437569\\
244	0.00593172622928573\\
245	0.00592968461532358\\
246	0.00592761108912331\\
247	0.00592550525917501\\
248	0.00592336672484548\\
249	0.0059211950751797\\
250	0.00591898988742878\\
251	0.0059167507252386\\
252	0.0059144771364235\\
253	0.00591216865023706\\
254	0.00590982477404137\\
255	0.00590744498926841\\
256	0.00590502874656275\\
257	0.00590257545999976\\
258	0.00590008450029143\\
259	0.00589755518693252\\
260	0.00589498677931358\\
261	0.00589237846695294\\
262	0.0058897293591971\\
263	0.00588703847501438\\
264	0.00588430473372601\\
265	0.00588152694691003\\
266	0.00587870381006663\\
267	0.00587583392198569\\
268	0.00587291582604199\\
269	0.00586994800842341\\
270	0.00586692889637673\\
271	0.00586385685648655\\
272	0.00586073019300609\\
273	0.0058575471462592\\
274	0.00585430589114777\\
275	0.00585100453578624\\
276	0.00584764112030902\\
277	0.00584421361589246\\
278	0.00584071992404575\\
279	0.005837157876233\\
280	0.00583352523390158\\
281	0.005829819689003\\
282	0.00582603886510844\\
283	0.0058221803192368\\
284	0.00581824154453196\\
285	0.00581421997394661\\
286	0.00581011298511246\\
287	0.00580591790660013\\
288	0.00580163202579745\\
289	0.00579725259865931\\
290	0.00579277686160511\\
291	0.00578820204585885\\
292	0.00578352539453689\\
293	0.00577874418278647\\
294	0.00577385574125485\\
295	0.00576885748311349\\
296	0.00576374693476257\\
297	0.00575852177017284\\
298	0.00575317984855776\\
299	0.00574771925464148\\
300	0.00574213834001177\\
301	0.00573643576217835\\
302	0.00573061051281166\\
303	0.00572466194125788\\
304	0.00571858977587763\\
305	0.00571239410950164\\
306	0.00570607534071958\\
307	0.00569963405213641\\
308	0.0056930707978446\\
309	0.00568638576265987\\
310	0.00567957813374577\\
311	0.00567264463828431\\
312	0.0056655794880873\\
313	0.00565837646431389\\
314	0.00565102888368784\\
315	0.00564352956717776\\
316	0.00563587080655857\\
317	0.00562804430077823\\
318	0.00562004110318492\\
319	0.005611851568863\\
320	0.00560346529909061\\
321	0.00559487108331325\\
322	0.00558605683930868\\
323	0.00557700955256866\\
324	0.00556771521640067\\
325	0.00555815877491492\\
326	0.00554832407201765\\
327	0.00553819381099456\\
328	0.00552774953117986\\
329	0.0055169716024414\\
330	0.00550583909829185\\
331	0.00549432998080299\\
332	0.00548242125738638\\
333	0.00547008920475404\\
334	0.0054573097112811\\
335	0.00544405879011807\\
336	0.00543031344118811\\
337	0.0054160527628837\\
338	0.00540125912742442\\
339	0.00538592000330503\\
340	0.00537003044851019\\
341	0.00535359644526481\\
342	0.00533663934030161\\
343	0.00531920174680257\\
344	0.00530144252339929\\
345	0.00528361270366819\\
346	0.00526572555361539\\
347	0.00524779599459207\\
348	0.00522984077687911\\
349	0.00521187866831396\\
350	0.00519393066326458\\
351	0.00517602024434208\\
352	0.00515817393827313\\
353	0.00514041964728142\\
354	0.00512278743406033\\
355	0.00510530997362699\\
356	0.00508802247112745\\
357	0.00507096222996936\\
358	0.00505417115077324\\
359	0.00503769643319219\\
360	0.00502158814721375\\
361	0.00500589866402464\\
362	0.00499068173694374\\
363	0.00497599110596318\\
364	0.0049618784526552\\
365	0.0049483904821479\\
366	0.00493556483546533\\
367	0.00492342444071178\\
368	0.00491196978455394\\
369	0.00490116839863309\\
370	0.0048906324354223\\
371	0.00488027870923257\\
372	0.00487012001402301\\
373	0.00486016882262852\\
374	0.00485043709822925\\
375	0.00484093607625654\\
376	0.00483167592810388\\
377	0.0048226653696756\\
378	0.00481391127770395\\
379	0.00480541826302987\\
380	0.0047971882060768\\
381	0.00478921976522046\\
382	0.00478150787669776\\
383	0.0047740432764957\\
384	0.00476681208989165\\
385	0.00475979555689918\\
386	0.00475296999263243\\
387	0.00474630712384782\\
388	0.00473977499796648\\
389	0.00473333973708328\\
390	0.00472696851258153\\
391	0.00472065829735722\\
392	0.00471440615456371\\
393	0.0047082083800895\\
394	0.00470206048044941\\
395	0.00469595716595798\\
396	0.00468989236351199\\
397	0.00468385925379096\\
398	0.00467785033801642\\
399	0.00467185753945659\\
400	0.00466587234443052\\
401	0.00465988598636044\\
402	0.00465388967405033\\
403	0.00464787486131385\\
404	0.00464183354823129\\
405	0.00463575859381497\\
406	0.00462964400391982\\
407	0.00462348513651544\\
408	0.00461727863610413\\
409	0.00461102107911178\\
410	0.00460470887362543\\
411	0.00459833843936909\\
412	0.00459190635956627\\
413	0.00458540926575913\\
414	0.00457884387749456\\
415	0.00457220703776355\\
416	0.0045654957449966\\
417	0.00455870716969015\\
418	0.00455183867148872\\
419	0.00454488780577619\\
420	0.00453785231597264\\
421	0.00453073010934476\\
422	0.00452351921611247\\
423	0.00451621773483804\\
424	0.0045088237842544\\
425	0.00450133550763517\\
426	0.00449375107627498\\
427	0.00448606869187626\\
428	0.00447828658765964\\
429	0.00447040302805675\\
430	0.00446241630691595\\
431	0.0044543247442354\\
432	0.00444612668155861\\
433	0.00443782047631773\\
434	0.00442940449557253\\
435	0.00442087710976413\\
436	0.0044122366872682\\
437	0.00440348159058331\\
438	0.00439461017483153\\
439	0.00438562078690467\\
440	0.00437651176441105\\
441	0.00436728143444992\\
442	0.0043579281122518\\
443	0.00434845009973193\\
444	0.00433884568401337\\
445	0.00432911313598069\\
446	0.00431925070892445\\
447	0.00430925663732789\\
448	0.00429912913582875\\
449	0.00428886639836054\\
450	0.00427846659744047\\
451	0.00426792788352919\\
452	0.00425724838437782\\
453	0.00424642620436951\\
454	0.00423545942386197\\
455	0.00422434609853724\\
456	0.00421308425876252\\
457	0.00420167190896466\\
458	0.00419010702701767\\
459	0.00417838756364069\\
460	0.00416651144180048\\
461	0.0041544765561112\\
462	0.00414228077222317\\
463	0.00412992192619294\\
464	0.00411739782383058\\
465	0.00410470624002143\\
466	0.00409184491801969\\
467	0.00407881156871013\\
468	0.00406560386983372\\
469	0.00405221946517284\\
470	0.00403865596368979\\
471	0.00402491093861322\\
472	0.00401098192646624\\
473	0.00399686642602873\\
474	0.00398256189722682\\
475	0.00396806575994199\\
476	0.00395337539273091\\
477	0.00393848813144712\\
478	0.00392340126775386\\
479	0.0039081120475174\\
480	0.00389261766906804\\
481	0.00387691528131579\\
482	0.00386100198170571\\
483	0.00384487481399766\\
484	0.00382853076585254\\
485	0.00381196676620695\\
486	0.00379517968241609\\
487	0.00377816631714323\\
488	0.00376092340497299\\
489	0.0037434476087237\\
490	0.00372573551543333\\
491	0.00370778363199189\\
492	0.00368958838039254\\
493	0.00367114609257297\\
494	0.00365245300481891\\
495	0.00363350525170209\\
496	0.00361429885952708\\
497	0.00359482973926421\\
498	0.00357509367895112\\
499	0.00355508633555229\\
500	0.00353480322627609\\
501	0.00351423971936296\\
502	0.00349339102437709\\
503	0.00347225218205858\\
504	0.00345081805382653\\
505	0.00342908331106559\\
506	0.00340704242438401\\
507	0.00338468965310275\\
508	0.00336201903532531\\
509	0.00333902437905489\\
510	0.00331569925497228\\
511	0.00329203699167503\\
512	0.00326803067441413\\
513	0.00324367314866261\\
514	0.00321895703022545\\
515	0.00319387472407181\\
516	0.00316841845466331\\
517	0.0031425803112958\\
518	0.00311635231290443\\
519	0.00308972649795068\\
520	0.00306269504647383\\
521	0.003035250443222\\
522	0.00300738569307181\\
523	0.00297909460281731\\
524	0.00295037214700139\\
525	0.00292121493997158\\
526	0.00289162182508419\\
527	0.00286159456307669\\
528	0.00283113903050781\\
529	0.00280026820797992\\
530	0.00276905767127757\\
531	0.00273751343323209\\
532	0.00270557045782752\\
533	0.00267307211406851\\
534	0.00263992242027967\\
535	0.00260606339991704\\
536	0.00257144120591396\\
537	0.00253600462960201\\
538	0.00249969764946177\\
539	0.0024624594910934\\
540	0.0024242224843366\\
541	0.00238491137467294\\
542	0.00234444454378673\\
543	0.00230273639459776\\
544	0.00225966796547375\\
545	0.00221668865662068\\
546	0.00217582521296366\\
547	0.0021362607008898\\
548	0.00209639682394114\\
549	0.00205600872796146\\
550	0.00201508651193405\\
551	0.00197366306424266\\
552	0.00193178370444651\\
553	0.00188950895448212\\
554	0.00184691508122861\\
555	0.0018042815234374\\
556	0.00176267919822897\\
557	0.00172129441691307\\
558	0.00167960212397915\\
559	0.0016376158968852\\
560	0.00159536277689962\\
561	0.00155287132132885\\
562	0.00151016995834333\\
563	0.0014672865789184\\
564	0.00142424694170975\\
565	0.00138130050605624\\
566	0.00133853402642386\\
567	0.00129544164993273\\
568	0.00125203947380946\\
569	0.00120834727902135\\
570	0.00116438693634367\\
571	0.00112018252432505\\
572	0.00107576046014404\\
573	0.00103114962817055\\
574	0.000986381501991801\\
575	0.000941490255195813\\
576	0.000896512854871852\\
577	0.000851489130077657\\
578	0.000806461805374505\\
579	0.000761476486847638\\
580	0.000716581584683521\\
581	0.000671828152215824\\
582	0.000627269616206769\\
583	0.00058296136682538\\
584	0.000538960168207986\\
585	0.000495323341784082\\
586	0.00045210766564535\\
587	0.000409367927260641\\
588	0.000367155072958961\\
589	0.000325513941232703\\
590	0.000284480716949125\\
591	0.000244080684870838\\
592	0.00020432808725126\\
593	0.000165233217676711\\
594	0.000126870159937008\\
595	8.9611172254752e-05\\
596	5.42660945238509e-05\\
597	2.28062284332057e-05\\
598	2.9204464504877e-07\\
599	0\\
600	0\\
};
\addplot [color=black,solid,forget plot]
  table[row sep=crcr]{%
1	0.00604024220115463\\
2	0.00604022941156559\\
3	0.00604021631894476\\
4	0.00604020291611037\\
5	0.00604018919571038\\
6	0.00604017515021865\\
7	0.0060401607719305\\
8	0.00604014605295886\\
9	0.00604013098522959\\
10	0.00604011556047734\\
11	0.00604009977024089\\
12	0.00604008360585844\\
13	0.00604006705846304\\
14	0.00604005011897763\\
15	0.00604003277811003\\
16	0.00604001502634798\\
17	0.0060399968539538\\
18	0.00603997825095909\\
19	0.00603995920715936\\
20	0.00603993971210836\\
21	0.00603991975511225\\
22	0.00603989932522407\\
23	0.00603987841123736\\
24	0.00603985700168037\\
25	0.00603983508480957\\
26	0.00603981264860322\\
27	0.00603978968075492\\
28	0.00603976616866678\\
29	0.0060397420994427\\
30	0.00603971745988097\\
31	0.00603969223646742\\
32	0.00603966641536785\\
33	0.00603963998242046\\
34	0.00603961292312817\\
35	0.00603958522265073\\
36	0.00603955686579649\\
37	0.00603952783701423\\
38	0.00603949812038457\\
39	0.00603946769961138\\
40	0.00603943655801282\\
41	0.00603940467851219\\
42	0.00603937204362877\\
43	0.00603933863546809\\
44	0.00603930443571237\\
45	0.00603926942561029\\
46	0.00603923358596707\\
47	0.00603919689713374\\
48	0.00603915933899659\\
49	0.00603912089096622\\
50	0.00603908153196619\\
51	0.00603904124042176\\
52	0.00603899999424794\\
53	0.00603895777083757\\
54	0.00603891454704906\\
55	0.00603887029919381\\
56	0.00603882500302325\\
57	0.00603877863371572\\
58	0.00603873116586302\\
59	0.00603868257345652\\
60	0.00603863282987321\\
61	0.00603858190786107\\
62	0.00603852977952445\\
63	0.00603847641630884\\
64	0.00603842178898546\\
65	0.00603836586763539\\
66	0.0060383086216335\\
67	0.00603825001963159\\
68	0.0060381900295418\\
69	0.00603812861851904\\
70	0.00603806575294322\\
71	0.00603800139840123\\
72	0.00603793551966834\\
73	0.00603786808068912\\
74	0.00603779904455802\\
75	0.00603772837349957\\
76	0.00603765602884802\\
77	0.00603758197102642\\
78	0.00603750615952539\\
79	0.00603742855288145\\
80	0.00603734910865462\\
81	0.00603726778340568\\
82	0.00603718453267291\\
83	0.00603709931094821\\
84	0.00603701207165275\\
85	0.00603692276711206\\
86	0.00603683134853047\\
87	0.00603673776596517\\
88	0.00603664196829936\\
89	0.0060365439032152\\
90	0.00603644351716582\\
91	0.0060363407553467\\
92	0.00603623556166671\\
93	0.0060361278787182\\
94	0.00603601764774663\\
95	0.00603590480861926\\
96	0.00603578929979333\\
97	0.00603567105828357\\
98	0.00603555001962885\\
99	0.00603542611785818\\
100	0.00603529928545591\\
101	0.00603516945332611\\
102	0.00603503655075626\\
103	0.00603490050538015\\
104	0.00603476124313985\\
105	0.00603461868824691\\
106	0.00603447276314273\\
107	0.00603432338845804\\
108	0.00603417048297137\\
109	0.00603401396356695\\
110	0.00603385374519139\\
111	0.00603368974080943\\
112	0.00603352186135891\\
113	0.0060333500157048\\
114	0.00603317411059178\\
115	0.00603299405059647\\
116	0.00603280973807822\\
117	0.00603262107312888\\
118	0.00603242795352169\\
119	0.00603223027465891\\
120	0.00603202792951842\\
121	0.00603182080859922\\
122	0.00603160879986567\\
123	0.00603139178869086\\
124	0.00603116965779838\\
125	0.00603094228720326\\
126	0.00603070955415151\\
127	0.00603047133305845\\
128	0.0060302274954458\\
129	0.00602997790987749\\
130	0.00602972244189418\\
131	0.00602946095394636\\
132	0.00602919330532638\\
133	0.00602891935209879\\
134	0.00602863894702952\\
135	0.00602835193951357\\
136	0.00602805817550143\\
137	0.00602775749742385\\
138	0.00602744974411521\\
139	0.00602713475073561\\
140	0.00602681234869119\\
141	0.00602648236555314\\
142	0.0060261446249751\\
143	0.00602579894660905\\
144	0.00602544514601946\\
145	0.00602508303459615\\
146	0.00602471241946532\\
147	0.00602433310339914\\
148	0.00602394488472337\\
149	0.00602354755722375\\
150	0.00602314091005041\\
151	0.00602272472762075\\
152	0.00602229878952048\\
153	0.00602186287040315\\
154	0.00602141673988773\\
155	0.00602096016245453\\
156	0.00602049289733946\\
157	0.00602001469842612\\
158	0.00601952531413678\\
159	0.0060190244873208\\
160	0.00601851195514175\\
161	0.00601798744896242\\
162	0.00601745069422821\\
163	0.00601690141034843\\
164	0.00601633931057578\\
165	0.00601576410188413\\
166	0.00601517548484403\\
167	0.00601457315349658\\
168	0.00601395679522537\\
169	0.00601332609062629\\
170	0.00601268071337537\\
171	0.00601202033009492\\
172	0.00601134460021717\\
173	0.0060106531758464\\
174	0.00600994570161868\\
175	0.00600922181455949\\
176	0.00600848114393944\\
177	0.00600772331112782\\
178	0.00600694792944373\\
179	0.00600615460400528\\
180	0.00600534293157628\\
181	0.00600451250041084\\
182	0.00600366289009542\\
183	0.00600279367138868\\
184	0.00600190440605857\\
185	0.00600099464671707\\
186	0.00600006393665222\\
187	0.00599911180965739\\
188	0.0059981377898579\\
189	0.00599714139153454\\
190	0.00599612211894405\\
191	0.00599507946613675\\
192	0.0059940129167705\\
193	0.00599292194392151\\
194	0.00599180600989156\\
195	0.00599066456601142\\
196	0.00598949705244036\\
197	0.00598830289796187\\
198	0.00598708151977458\\
199	0.00598583232327963\\
200	0.00598455470186254\\
201	0.00598324803667099\\
202	0.00598191169638715\\
203	0.00598054503699518\\
204	0.00597914740154348\\
205	0.00597771811990127\\
206	0.00597625650851\\
207	0.00597476187012911\\
208	0.0059732334935763\\
209	0.00597167065346244\\
210	0.00597007260992117\\
211	0.00596843860833389\\
212	0.00596676787905015\\
213	0.00596505963710449\\
214	0.00596331308193016\\
215	0.00596152739707178\\
216	0.00595970174989727\\
217	0.00595783529131195\\
218	0.00595592715547681\\
219	0.0059539764595338\\
220	0.00595198230334206\\
221	0.00594994376922937\\
222	0.0059478599217644\\
223	0.00594572980755597\\
224	0.0059435524550875\\
225	0.00594132687459575\\
226	0.00593905205800509\\
227	0.00593672697893076\\
228	0.00593435059276621\\
229	0.00593192183687351\\
230	0.00592943963089844\\
231	0.00592690287723542\\
232	0.00592431046167212\\
233	0.00592166125424854\\
234	0.00591895411037039\\
235	0.00591618787222353\\
236	0.00591336137054312\\
237	0.00591047342679857\\
238	0.00590752285586539\\
239	0.00590450846926347\\
240	0.00590142907905286\\
241	0.0058982835024888\\
242	0.00589507056754937\\
243	0.00589178911946169\\
244	0.00588843802836297\\
245	0.00588501619824405\\
246	0.00588152257732885\\
247	0.00587795617004913\\
248	0.00587431605076752\\
249	0.00587060137939107\\
250	0.00586681141898741\\
251	0.00586294555546768\\
252	0.00585900331931978\\
253	0.00585498440925583\\
254	0.00585088871745866\\
255	0.00584671635585565\\
256	0.00584246768248407\\
257	0.00583814332650342\\
258	0.00583374420970669\\
259	0.00582927156141854\\
260	0.0058247269223566\\
261	0.00582011213126507\\
262	0.00581542928577073\\
263	0.0058106806658248\\
264	0.00580586860408352\\
265	0.00580099528198818\\
266	0.00579606241162627\\
267	0.00579107060308566\\
268	0.00578601800834475\\
269	0.00578090264365087\\
270	0.00577572237782153\\
271	0.00577047491949016\\
272	0.00576515780320051\\
273	0.00575976837424582\\
274	0.00575430377214284\\
275	0.00574876091262352\\
276	0.00574313646802086\\
277	0.00573742684592163\\
278	0.00573162816595557\\
279	0.00572573623459187\\
280	0.00571974651781746\\
281	0.00571365411158153\\
282	0.00570745370991107\\
283	0.00570113957063665\\
284	0.00569470547871148\\
285	0.00568814470717387\\
286	0.00568144997589694\\
287	0.00567461340840066\\
288	0.00566762648717802\\
289	0.0056604800082285\\
290	0.00565316403581248\\
291	0.00564566785886966\\
292	0.00563797995111085\\
293	0.00563008793754007\\
294	0.005621978571147\\
295	0.00561363772479723\\
296	0.00560505040503344\\
297	0.00559620079670585\\
298	0.0055870723502374\\
299	0.00557764792713239\\
300	0.00556791002437553\\
301	0.00555784110485093\\
302	0.00554742406491736\\
303	0.00553664279888841\\
304	0.00552548302651021\\
305	0.00551393362608928\\
306	0.00550198829571939\\
307	0.00548964775149012\\
308	0.00547692265350177\\
309	0.00546383750077499\\
310	0.00545050411544554\\
311	0.00543706551203978\\
312	0.00542352426312607\\
313	0.00540988338268015\\
314	0.00539614638453567\\
315	0.00538231736054493\\
316	0.00536840113025545\\
317	0.00535440347191429\\
318	0.00534033101754344\\
319	0.00532619131263288\\
320	0.00531199291781422\\
321	0.00529774552004493\\
322	0.00528346005357307\\
323	0.00526914883070616\\
324	0.00525482568205746\\
325	0.0052405061054705\\
326	0.00522620742230711\\
327	0.00521194893974284\\
328	0.00519775212120448\\
329	0.00518364079415995\\
330	0.00516964162363341\\
331	0.00515578212083989\\
332	0.00514209199396442\\
333	0.00512860302430427\\
334	0.00511534880080009\\
335	0.00510236423489391\\
336	0.00508968465564633\\
337	0.00507734632258802\\
338	0.00506538684370882\\
339	0.00505384123288094\\
340	0.00504273935097975\\
341	0.00503210235176745\\
342	0.0050219378023539\\
343	0.00501223303726919\\
344	0.00500285468968795\\
345	0.00499356955559161\\
346	0.00498438739563515\\
347	0.00497531818451975\\
348	0.00496637204863509\\
349	0.00495755918753014\\
350	0.00494888977653276\\
351	0.00494037384742799\\
352	0.00493202114184904\\
353	0.00492384098152797\\
354	0.00491584209842258\\
355	0.00490803242980753\\
356	0.00490041888681241\\
357	0.00489300709639044\\
358	0.0048858010512431\\
359	0.00487880271345722\\
360	0.00487201164452336\\
361	0.00486542462797597\\
362	0.00485903530661304\\
363	0.00485283386790705\\
364	0.00484680682774371\\
365	0.00484093698516495\\
366	0.00483520365176605\\
367	0.00482958330142587\\
368	0.00482405084203799\\
369	0.00481858178443982\\
370	0.00481317112843914\\
371	0.00480781825833003\\
372	0.00480252205246886\\
373	0.00479728084542065\\
374	0.0047920923938142\\
375	0.00478695384774833\\
376	0.00478186173200704\\
377	0.00477681194038512\\
378	0.00477179974536357\\
379	0.00476681982690542\\
380	0.00476186632440995\\
381	0.00475693291591714\\
382	0.00475201292834093\\
383	0.00474709948159236\\
384	0.00474218566766209\\
385	0.00473726476257569\\
386	0.00473233046398241\\
387	0.00472737713907369\\
388	0.00472240005545754\\
389	0.00471739555086495\\
390	0.00471236108295356\\
391	0.00470729403270115\\
392	0.00470219168751057\\
393	0.00469705126007499\\
394	0.00469186991009814\\
395	0.0046866447686414\\
396	0.00468137296467694\\
397	0.00467605165319088\\
398	0.0046706780439002\\
399	0.00466524942932727\\
400	0.00465976321063176\\
401	0.00465421691925515\\
402	0.00464860823213952\\
403	0.00464293497810728\\
404	0.00463719513305938\\
405	0.00463138680214551\\
406	0.00462550818845853\\
407	0.00461955755337387\\
408	0.00461353313115817\\
409	0.00460743324585354\\
410	0.00460125630724931\\
411	0.00459500075370312\\
412	0.00458866504993294\\
413	0.00458224768832692\\
414	0.00457574718895802\\
415	0.00456916209832784\\
416	0.00456249098687029\\
417	0.00455573244556927\\
418	0.00454888508158583\\
419	0.00454194751318431\\
420	0.00453491836442718\\
421	0.00452779626024947\\
422	0.00452057982260584\\
423	0.00451326766834301\\
424	0.00450585840873635\\
425	0.00449835064883985\\
426	0.0044907429866612\\
427	0.00448303401218258\\
428	0.00447522230625659\\
429	0.00446730643941554\\
430	0.00445928497063901\\
431	0.00445115644612937\\
432	0.00444291939814595\\
433	0.00443457234394397\\
434	0.00442611378485224\\
435	0.00441754220550516\\
436	0.0044088560732158\\
437	0.00440005383744301\\
438	0.00439113392927468\\
439	0.00438209476090384\\
440	0.00437293472510363\\
441	0.00436365219470771\\
442	0.00435424552210145\\
443	0.00434471303872821\\
444	0.00433505305461357\\
445	0.00432526385790813\\
446	0.00431534371444779\\
447	0.00430529086732777\\
448	0.00429510353648548\\
449	0.00428477991828609\\
450	0.00427431818510563\\
451	0.00426371648490848\\
452	0.00425297294081893\\
453	0.00424208565068619\\
454	0.00423105268664214\\
455	0.00421987209465014\\
456	0.00420854189404392\\
457	0.00419706007705395\\
458	0.0041854246083197\\
459	0.00417363342438504\\
460	0.00416168443317458\\
461	0.00414957551344839\\
462	0.00413730451423243\\
463	0.00412486925422227\\
464	0.00411226752115713\\
465	0.00409949707116105\\
466	0.00408655562804788\\
467	0.00407344088258621\\
468	0.00406015049171967\\
469	0.00404668207773832\\
470	0.00403303322739574\\
471	0.00401920149096636\\
472	0.00400518438123652\\
473	0.00399097937242283\\
474	0.00397658389901009\\
475	0.00396199535450056\\
476	0.00394721109006594\\
477	0.00393222841309141\\
478	0.00391704458560201\\
479	0.00390165682255867\\
480	0.00388606229001174\\
481	0.00387025810309765\\
482	0.00385424132386436\\
483	0.00383800895890885\\
484	0.00382155795681001\\
485	0.00380488520533814\\
486	0.00378798752842205\\
487	0.00377086168285347\\
488	0.00375350435470764\\
489	0.00373591215545927\\
490	0.00371808161777201\\
491	0.00370000919094156\\
492	0.0036816912359725\\
493	0.00366312402027269\\
494	0.00364430371195221\\
495	0.00362522637371953\\
496	0.00360588795637567\\
497	0.00358628429191823\\
498	0.00356641108628137\\
499	0.0035462639117581\\
500	0.00352583819917636\\
501	0.00350512922993411\\
502	0.00348413212804223\\
503	0.00346284185237889\\
504	0.00344125318943081\\
505	0.00341936074688669\\
506	0.00339715894856318\\
507	0.00337464203128774\\
508	0.0033518040445461\\
509	0.00332863885393061\\
510	0.0033051401497136\\
511	0.00328130146222999\\
512	0.00325711618620332\\
513	0.00323257761671049\\
514	0.00320767900018014\\
515	0.00318241360468962\\
516	0.00315677481490683\\
517	0.00313075625836541\\
518	0.00310435197142525\\
519	0.00307755661533165\\
520	0.00305036575533494\\
521	0.00302277621898349\\
522	0.00299478655359284\\
523	0.00296639760769014\\
524	0.00293761326714388\\
525	0.00290844138313012\\
526	0.00287889567964912\\
527	0.00284903708034976\\
528	0.00281890494915272\\
529	0.00278843855151698\\
530	0.00275749196392408\\
531	0.00272596507989511\\
532	0.00269380524096619\\
533	0.00266096644932591\\
534	0.00262740435415402\\
535	0.00259307084256343\\
536	0.00255791304693705\\
537	0.00252187174740753\\
538	0.00248488045298894\\
539	0.00244686410993284\\
540	0.0024077377673102\\
541	0.00236740477819526\\
542	0.00232575612961217\\
543	0.00228403044480058\\
544	0.00224447891824068\\
545	0.00220617374851022\\
546	0.00216751041189044\\
547	0.00212829967778029\\
548	0.00208853590537212\\
549	0.00204824471904266\\
550	0.00200746452793853\\
551	0.00196624548713097\\
552	0.0019246531104606\\
553	0.00188276877233496\\
554	0.00184135597079297\\
555	0.00180079542484969\\
556	0.00175995385435818\\
557	0.00171878217399611\\
558	0.00167730425837276\\
559	0.0016355463416697\\
560	0.00159353515574625\\
561	0.00155129727986493\\
562	0.00150885875639163\\
563	0.0014662435534718\\
564	0.00142371991366068\\
565	0.00138129869913588\\
566	0.00133853400693808\\
567	0.00129544164841737\\
568	0.00125203947353977\\
569	0.0012083472789297\\
570	0.00116438693630348\\
571	0.0011201825243051\\
572	0.00107576046013345\\
573	0.00103114962816466\\
574	0.000986381501988496\\
575	0.000941490255193976\\
576	0.000896512854870878\\
577	0.000851489130077175\\
578	0.000806461805374287\\
579	0.000761476486847549\\
580	0.000716581584683501\\
581	0.000671828152215815\\
582	0.000627269616206765\\
583	0.00058296136682537\\
584	0.000538960168207977\\
585	0.00049532334178407\\
586	0.000452107665645345\\
587	0.000409367927260634\\
588	0.000367155072958957\\
589	0.0003255139412327\\
590	0.000284480716949121\\
591	0.000244080684870836\\
592	0.00020432808725126\\
593	0.00016523321767671\\
594	0.000126870159937007\\
595	8.96111722547526e-05\\
596	5.4266094523851e-05\\
597	2.2806228433206e-05\\
598	2.9204464504877e-07\\
599	0\\
600	0\\
};
\end{axis}
\end{tikzpicture}% 
  \caption{Discrete Time w/ nFPC}
\end{subfigure}\\

\leavevmode\smash{\makebox[0pt]{\hspace{-7em}% HORIZONTAL POSITION           
  \rotatebox[origin=l]{90}{\hspace{20em}% VERTICAL POSITION
    Depth $\delta^-$}%
}}\hspace{0pt plus 1filll}\null

Time (s)

\vspace{1cm}
\begin{subfigure}{\linewidth}
  \centering
  \tikzsetnextfilename{deltalegend}
  \documentclass{article}
\usepackage{pgfplots}
\usetikzlibrary{backgrounds}
\pgfplotsset{compat=newest}  
\newlength\figureheight 
\newlength\figurewidth 

\begin{document}
%
%\begin{figure}
%  \centering
%  \setlength\figureheight{\linewidth} 
%  \setlength\figurewidth{\linewidth}
%  \input{/home/anton/Documents/masc/ml/thesis/tikz/ORCL_comp4stoch.tikz}
%  \caption{Backtest strategy comparison}
%  \label{fig:insample}
%\end{figure}
\definecolor{mycolor1}{rgb}{1.00000,0.00000,1.00000}%
\begin{tikzpicture}[framed]
    \begingroup
    % inits/clears the lists (which might be populated from previous
    % axes):
    \csname pgfplots@init@cleared@structures\endcsname
    \pgfplotsset{legend style={at={(0,1)},anchor=north west},legend columns=-1,legend style={draw=none,column sep=1ex},legend entries={$q=-4$,$q=-3$,$q=-2$,$q=-1$}}%
    
    \csname pgfplots@addlegendimage\endcsname{thick,green,dashed,sharp plot}
    \csname pgfplots@addlegendimage\endcsname{thick,mycolor1,dashed,sharp plot}
    \csname pgfplots@addlegendimage\endcsname{thick,red,dashed,sharp plot}
    \csname pgfplots@addlegendimage\endcsname{thick,blue,dashed,sharp plot}

    % draws the legend:
    \csname pgfplots@createlegend\endcsname
    \endgroup

    \begingroup
    % inits/clears the lists (which might be populated from previous
    % axes):
    \csname pgfplots@init@cleared@structures\endcsname
    \pgfplotsset{legend style={at={(3.45,0.5)},anchor=north west},legend columns=-1,legend style={draw=none,column sep=1ex},legend entries={$q=0$}}%

    \csname pgfplots@addlegendimage\endcsname{thick,black,sharp plot}

    % draws the legend:
    \csname pgfplots@createlegend\endcsname
    \endgroup

    \begingroup
    % inits/clears the lists (which might be populated from previous
    % axes):
    \csname pgfplots@init@cleared@structures\endcsname
    \pgfplotsset{legend style={at={(0,0)},anchor=north west},legend columns=-1,legend style={draw=none,column sep=1ex},legend entries={$q=+4$,$q=+3$,$q=+2$,$q=+1$}}%
    
    \csname pgfplots@addlegendimage\endcsname{thick,green,sharp plot}
    \csname pgfplots@addlegendimage\endcsname{thick,mycolor1,sharp plot}
    \csname pgfplots@addlegendimage\endcsname{thick,red,sharp plot}
    \csname pgfplots@addlegendimage\endcsname{thick,blue,sharp plot}

    % draws the legend:
    \csname pgfplots@createlegend\endcsname
    \endgroup
\end{tikzpicture}

\end{document} 
\end{subfigure}%
  \caption{Optimal sell depths $\delta^-$ for Markov state $Z=(\rho = 0, \Delta S = 0)$, implying neutral imbalance and no previous price change. We expect no change in midprice.}
  \label{fig:comp_dm_z8}
\end{figure}

\begin{figure}
\centering
\begin{subfigure}{.45\linewidth}
  \centering
  \setlength\figureheight{\linewidth} 
  \setlength\figurewidth{\linewidth}
  \tikzsetnextfilename{dm_cts_z15}
  % This file was created by matlab2tikz.
%
%The latest updates can be retrieved from
%  http://www.mathworks.com/matlabcentral/fileexchange/22022-matlab2tikz-matlab2tikz
%where you can also make suggestions and rate matlab2tikz.
%
\definecolor{mycolor1}{rgb}{1.00000,0.00000,1.00000}%
%
\begin{tikzpicture}

\begin{axis}[%
width=4.564in,
height=3.803in,
at={(1.067in,0.513in)},
scale only axis,
every outer x axis line/.append style={black},
every x tick label/.append style={font=\color{black}},
xmin=0,
xmax=100,
xlabel={Time},
every outer y axis line/.append style={black},
every y tick label/.append style={font=\color{black}},
ymin=0,
ymax=0.01,
ylabel={Depth $\delta$},
axis background/.style={fill=white},
title={Z=15},
axis x line*=bottom,
axis y line*=left,
legend style={legend cell align=left,align=left,draw=black}
]
\addplot [color=green,dashed,forget plot]
  table[row sep=crcr]{%
0.01	0\\
0.02	0\\
0.03	0\\
0.04	0\\
0.05	0\\
0.06	0\\
0.07	0\\
0.08	0\\
0.09	0\\
0.1	0\\
0.11	0\\
0.12	0\\
0.13	0\\
0.14	0\\
0.15	0\\
0.16	0\\
0.17	0\\
0.18	0\\
0.19	0\\
0.2	0\\
0.21	0\\
0.22	0\\
0.23	0\\
0.24	0\\
0.25	0\\
0.26	0\\
0.27	0\\
0.28	0\\
0.29	0\\
0.3	0\\
0.31	0\\
0.32	0\\
0.33	0\\
0.34	0\\
0.35	0\\
0.36	0\\
0.37	0\\
0.38	0\\
0.39	0\\
0.4	0\\
0.41	0\\
0.42	0\\
0.43	0\\
0.44	0\\
0.45	0\\
0.46	0\\
0.47	0\\
0.48	0\\
0.49	0\\
0.5	0\\
0.51	0\\
0.52	0\\
0.53	0\\
0.54	0\\
0.55	0\\
0.56	0\\
0.57	0\\
0.58	0\\
0.59	0\\
0.6	0\\
0.61	0\\
0.62	0\\
0.63	0\\
0.64	0\\
0.65	0\\
0.66	0\\
0.67	0\\
0.68	0\\
0.69	0\\
0.7	0\\
0.71	0\\
0.72	0\\
0.73	0\\
0.74	0\\
0.75	0\\
0.76	0\\
0.77	0\\
0.78	0\\
0.79	0\\
0.8	0\\
0.81	0\\
0.82	0\\
0.83	0\\
0.84	0\\
0.85	0\\
0.86	0\\
0.87	0\\
0.88	0\\
0.89	0\\
0.9	0\\
0.91	0\\
0.92	0\\
0.93	0\\
0.94	0\\
0.95	0\\
0.96	0\\
0.97	0\\
0.98	0\\
0.99	0\\
1	0\\
1.01	0\\
1.02	0\\
1.03	0\\
1.04	0\\
1.05	0\\
1.06	0\\
1.07	0\\
1.08	0\\
1.09	0\\
1.1	0\\
1.11	0\\
1.12	0\\
1.13	0\\
1.14	0\\
1.15	0\\
1.16	0\\
1.17	0\\
1.18	0\\
1.19	0\\
1.2	0\\
1.21	0\\
1.22	0\\
1.23	0\\
1.24	0\\
1.25	0\\
1.26	0\\
1.27	0\\
1.28	0\\
1.29	0\\
1.3	0\\
1.31	0\\
1.32	0\\
1.33	0\\
1.34	0\\
1.35	0\\
1.36	0\\
1.37	0\\
1.38	0\\
1.39	0\\
1.4	0\\
1.41	0\\
1.42	0\\
1.43	0\\
1.44	0\\
1.45	0\\
1.46	0\\
1.47	0\\
1.48	0\\
1.49	0\\
1.5	0\\
1.51	0\\
1.52	0\\
1.53	0\\
1.54	0\\
1.55	0\\
1.56	0\\
1.57	0\\
1.58	0\\
1.59	0\\
1.6	0\\
1.61	0\\
1.62	0\\
1.63	0\\
1.64	0\\
1.65	0\\
1.66	0\\
1.67	0\\
1.68	0\\
1.69	0\\
1.7	0\\
1.71	0\\
1.72	0\\
1.73	0\\
1.74	0\\
1.75	0\\
1.76	0\\
1.77	0\\
1.78	0\\
1.79	0\\
1.8	0\\
1.81	0\\
1.82	0\\
1.83	0\\
1.84	0\\
1.85	0\\
1.86	0\\
1.87	0\\
1.88	0\\
1.89	0\\
1.9	0\\
1.91	0\\
1.92	0\\
1.93	0\\
1.94	0\\
1.95	0\\
1.96	0\\
1.97	0\\
1.98	0\\
1.99	0\\
2	0\\
2.01	0\\
2.02	0\\
2.03	0\\
2.04	0\\
2.05	0\\
2.06	0\\
2.07	0\\
2.08	0\\
2.09	0\\
2.1	0\\
2.11	0\\
2.12	0\\
2.13	0\\
2.14	0\\
2.15	0\\
2.16	0\\
2.17	0\\
2.18	0\\
2.19	0\\
2.2	0\\
2.21	0\\
2.22	0\\
2.23	0\\
2.24	0\\
2.25	0\\
2.26	0\\
2.27	0\\
2.28	0\\
2.29	0\\
2.3	0\\
2.31	0\\
2.32	0\\
2.33	0\\
2.34	0\\
2.35	0\\
2.36	0\\
2.37	0\\
2.38	0\\
2.39	0\\
2.4	0\\
2.41	0\\
2.42	0\\
2.43	0\\
2.44	0\\
2.45	0\\
2.46	0\\
2.47	0\\
2.48	0\\
2.49	0\\
2.5	0\\
2.51	0\\
2.52	0\\
2.53	0\\
2.54	0\\
2.55	0\\
2.56	0\\
2.57	0\\
2.58	0\\
2.59	0\\
2.6	0\\
2.61	0\\
2.62	0\\
2.63	0\\
2.64	0\\
2.65	0\\
2.66	0\\
2.67	0\\
2.68	0\\
2.69	0\\
2.7	0\\
2.71	0\\
2.72	0\\
2.73	0\\
2.74	0\\
2.75	0\\
2.76	0\\
2.77	0\\
2.78	0\\
2.79	0\\
2.8	0\\
2.81	0\\
2.82	0\\
2.83	0\\
2.84	0\\
2.85	0\\
2.86	0\\
2.87	0\\
2.88	0\\
2.89	0\\
2.9	0\\
2.91	0\\
2.92	0\\
2.93	0\\
2.94	0\\
2.95	0\\
2.96	0\\
2.97	0\\
2.98	0\\
2.99	0\\
3	0\\
3.01	0\\
3.02	0\\
3.03	0\\
3.04	0\\
3.05	0\\
3.06	0\\
3.07	0\\
3.08	0\\
3.09	0\\
3.1	0\\
3.11	0\\
3.12	0\\
3.13	0\\
3.14	0\\
3.15	0\\
3.16	0\\
3.17	0\\
3.18	0\\
3.19	0\\
3.2	0\\
3.21	0\\
3.22	0\\
3.23	0\\
3.24	0\\
3.25	0\\
3.26	0\\
3.27	0\\
3.28	0\\
3.29	0\\
3.3	0\\
3.31	0\\
3.32	0\\
3.33	0\\
3.34	0\\
3.35	0\\
3.36	0\\
3.37	0\\
3.38	0\\
3.39	0\\
3.4	0\\
3.41	0\\
3.42	0\\
3.43	0\\
3.44	0\\
3.45	0\\
3.46	0\\
3.47	0\\
3.48	0\\
3.49	0\\
3.5	0\\
3.51	0\\
3.52	0\\
3.53	0\\
3.54	0\\
3.55	0\\
3.56	0\\
3.57	0\\
3.58	0\\
3.59	0\\
3.6	0\\
3.61	0\\
3.62	0\\
3.63	0\\
3.64	0\\
3.65	0\\
3.66	0\\
3.67	0\\
3.68	0\\
3.69	0\\
3.7	0\\
3.71	0\\
3.72	0\\
3.73	0\\
3.74	0\\
3.75	0\\
3.76	0\\
3.77	0\\
3.78	0\\
3.79	0\\
3.8	0\\
3.81	0\\
3.82	0\\
3.83	0\\
3.84	0\\
3.85	0\\
3.86	0\\
3.87	0\\
3.88	0\\
3.89	0\\
3.9	0\\
3.91	0\\
3.92	0\\
3.93	0\\
3.94	0\\
3.95	0\\
3.96	0\\
3.97	0\\
3.98	0\\
3.99	0\\
4	0\\
4.01	0\\
4.02	0\\
4.03	0\\
4.04	0\\
4.05	0\\
4.06	0\\
4.07	0\\
4.08	0\\
4.09	0\\
4.1	0\\
4.11	0\\
4.12	0\\
4.13	0\\
4.14	0\\
4.15	0\\
4.16	0\\
4.17	0\\
4.18	0\\
4.19	0\\
4.2	0\\
4.21	0\\
4.22	0\\
4.23	0\\
4.24	0\\
4.25	0\\
4.26	0\\
4.27	0\\
4.28	0\\
4.29	0\\
4.3	0\\
4.31	0\\
4.32	0\\
4.33	0\\
4.34	0\\
4.35	0\\
4.36	0\\
4.37	0\\
4.38	0\\
4.39	0\\
4.4	0\\
4.41	0\\
4.42	0\\
4.43	0\\
4.44	0\\
4.45	0\\
4.46	0\\
4.47	0\\
4.48	0\\
4.49	0\\
4.5	0\\
4.51	0\\
4.52	0\\
4.53	0\\
4.54	0\\
4.55	0\\
4.56	0\\
4.57	0\\
4.58	0\\
4.59	0\\
4.6	0\\
4.61	0\\
4.62	0\\
4.63	0\\
4.64	0\\
4.65	0\\
4.66	0\\
4.67	0\\
4.68	0\\
4.69	0\\
4.7	0\\
4.71	0\\
4.72	0\\
4.73	0\\
4.74	0\\
4.75	0\\
4.76	0\\
4.77	0\\
4.78	0\\
4.79	0\\
4.8	0\\
4.81	0\\
4.82	0\\
4.83	0\\
4.84	0\\
4.85	0\\
4.86	0\\
4.87	0\\
4.88	0\\
4.89	0\\
4.9	0\\
4.91	0\\
4.92	0\\
4.93	0\\
4.94	0\\
4.95	0\\
4.96	0\\
4.97	0\\
4.98	0\\
4.99	0\\
5	0\\
5.01	0\\
5.02	0\\
5.03	0\\
5.04	0\\
5.05	0\\
5.06	0\\
5.07	0\\
5.08	0\\
5.09	0\\
5.1	0\\
5.11	0\\
5.12	0\\
5.13	0\\
5.14	0\\
5.15	0\\
5.16	0\\
5.17	0\\
5.18	0\\
5.19	0\\
5.2	0\\
5.21	0\\
5.22	0\\
5.23	0\\
5.24	0\\
5.25	0\\
5.26	0\\
5.27	0\\
5.28	0\\
5.29	0\\
5.3	0\\
5.31	0\\
5.32	0\\
5.33	0\\
5.34	0\\
5.35	0\\
5.36	0\\
5.37	0\\
5.38	0\\
5.39	0\\
5.4	0\\
5.41	0\\
5.42	0\\
5.43	0\\
5.44	0\\
5.45	0\\
5.46	0\\
5.47	0\\
5.48	0\\
5.49	0\\
5.5	0\\
5.51	0\\
5.52	0\\
5.53	0\\
5.54	0\\
5.55	0\\
5.56	0\\
5.57	0\\
5.58	0\\
5.59	0\\
5.6	0\\
5.61	0\\
5.62	0\\
5.63	0\\
5.64	0\\
5.65	0\\
5.66	0\\
5.67	0\\
5.68	0\\
5.69	0\\
5.7	0\\
5.71	0\\
5.72	0\\
5.73	0\\
5.74	0\\
5.75	0\\
5.76	0\\
5.77	0\\
5.78	0\\
5.79	0\\
5.8	0\\
5.81	0\\
5.82	0\\
5.83	0\\
5.84	0\\
5.85	0\\
5.86	0\\
5.87	0\\
5.88	0\\
5.89	0\\
5.9	0\\
5.91	0\\
5.92	0\\
5.93	0\\
5.94	0\\
5.95	0\\
5.96	0\\
5.97	0\\
5.98	0\\
5.99	0\\
6	0\\
6.01	0\\
6.02	0\\
6.03	0\\
6.04	0\\
6.05	0\\
6.06	0\\
6.07	0\\
6.08	0\\
6.09	0\\
6.1	0\\
6.11	0\\
6.12	0\\
6.13	0\\
6.14	0\\
6.15	0\\
6.16	0\\
6.17	0\\
6.18	0\\
6.19	0\\
6.2	0\\
6.21	0\\
6.22	0\\
6.23	0\\
6.24	0\\
6.25	0\\
6.26	0\\
6.27	0\\
6.28	0\\
6.29	0\\
6.3	0\\
6.31	0\\
6.32	0\\
6.33	0\\
6.34	0\\
6.35	0\\
6.36	0\\
6.37	0\\
6.38	0\\
6.39	0\\
6.4	0\\
6.41	0\\
6.42	0\\
6.43	0\\
6.44	0\\
6.45	0\\
6.46	0\\
6.47	0\\
6.48	0\\
6.49	0\\
6.5	0\\
6.51	0\\
6.52	0\\
6.53	0\\
6.54	0\\
6.55	0\\
6.56	0\\
6.57	0\\
6.58	0\\
6.59	0\\
6.6	0\\
6.61	0\\
6.62	0\\
6.63	0\\
6.64	0\\
6.65	0\\
6.66	0\\
6.67	0\\
6.68	0\\
6.69	0\\
6.7	0\\
6.71	0\\
6.72	0\\
6.73	0\\
6.74	0\\
6.75	0\\
6.76	0\\
6.77	0\\
6.78	0\\
6.79	0\\
6.8	0\\
6.81	0\\
6.82	0\\
6.83	0\\
6.84	0\\
6.85	0\\
6.86	0\\
6.87	0\\
6.88	0\\
6.89	0\\
6.9	0\\
6.91	0\\
6.92	0\\
6.93	0\\
6.94	0\\
6.95	0\\
6.96	0\\
6.97	0\\
6.98	0\\
6.99	0\\
7	0\\
7.01	0\\
7.02	0\\
7.03	0\\
7.04	0\\
7.05	0\\
7.06	0\\
7.07	0\\
7.08	0\\
7.09	0\\
7.1	0\\
7.11	0\\
7.12	0\\
7.13	0\\
7.14	0\\
7.15	0\\
7.16	0\\
7.17	0\\
7.18	0\\
7.19	0\\
7.2	0\\
7.21	0\\
7.22	0\\
7.23	0\\
7.24	0\\
7.25	0\\
7.26	0\\
7.27	0\\
7.28	0\\
7.29	0\\
7.3	0\\
7.31	0\\
7.32	0\\
7.33	0\\
7.34	0\\
7.35	0\\
7.36	0\\
7.37	0\\
7.38	0\\
7.39	0\\
7.4	0\\
7.41	0\\
7.42	0\\
7.43	0\\
7.44	0\\
7.45	0\\
7.46	0\\
7.47	0\\
7.48	0\\
7.49	0\\
7.5	0\\
7.51	0\\
7.52	0\\
7.53	0\\
7.54	0\\
7.55	0\\
7.56	0\\
7.57	0\\
7.58	0\\
7.59	0\\
7.6	0\\
7.61	0\\
7.62	0\\
7.63	0\\
7.64	0\\
7.65	0\\
7.66	0\\
7.67	0\\
7.68	0\\
7.69	0\\
7.7	0\\
7.71	0\\
7.72	0\\
7.73	0\\
7.74	0\\
7.75	0\\
7.76	0\\
7.77	0\\
7.78	0\\
7.79	0\\
7.8	0\\
7.81	0\\
7.82	0\\
7.83	0\\
7.84	0\\
7.85	0\\
7.86	0\\
7.87	0\\
7.88	0\\
7.89	0\\
7.9	0\\
7.91	0\\
7.92	0\\
7.93	0\\
7.94	0\\
7.95	0\\
7.96	0\\
7.97	0\\
7.98	0\\
7.99	0\\
8	0\\
8.01	0\\
8.02	0\\
8.03	0\\
8.04	0\\
8.05	0\\
8.06	0\\
8.07	0\\
8.08	0\\
8.09	0\\
8.1	0\\
8.11	0\\
8.12	0\\
8.13	0\\
8.14	0\\
8.15	0\\
8.16	0\\
8.17	0\\
8.18	0\\
8.19	0\\
8.2	0\\
8.21	0\\
8.22	0\\
8.23	0\\
8.24	0\\
8.25	0\\
8.26	0\\
8.27	0\\
8.28	0\\
8.29	0\\
8.3	0\\
8.31	0\\
8.32	0\\
8.33	0\\
8.34	0\\
8.35	0\\
8.36	0\\
8.37	0\\
8.38	0\\
8.39	0\\
8.4	0\\
8.41	0\\
8.42	0\\
8.43	0\\
8.44	0\\
8.45	0\\
8.46	0\\
8.47	0\\
8.48	0\\
8.49	0\\
8.5	0\\
8.51	0\\
8.52	0\\
8.53	0\\
8.54	0\\
8.55	0\\
8.56	0\\
8.57	0\\
8.58	0\\
8.59	0\\
8.6	0\\
8.61	0\\
8.62	0\\
8.63	0\\
8.64	0\\
8.65	0\\
8.66	0\\
8.67	0\\
8.68	0\\
8.69	0\\
8.7	0\\
8.71	0\\
8.72	0\\
8.73	0\\
8.74	0\\
8.75	0\\
8.76	0\\
8.77	0\\
8.78	0\\
8.79	0\\
8.8	0\\
8.81	0\\
8.82	0\\
8.83	0\\
8.84	0\\
8.85	0\\
8.86	0\\
8.87	0\\
8.88	0\\
8.89	0\\
8.9	0\\
8.91	0\\
8.92	0\\
8.93	0\\
8.94	0\\
8.95	0\\
8.96	0\\
8.97	0\\
8.98	0\\
8.99	0\\
9	0\\
9.01	0\\
9.02	0\\
9.03	0\\
9.04	0\\
9.05	0\\
9.06	0\\
9.07	0\\
9.08	0\\
9.09	0\\
9.1	0\\
9.11	0\\
9.12	0\\
9.13	0\\
9.14	0\\
9.15	0\\
9.16	0\\
9.17	0\\
9.18	0\\
9.19	0\\
9.2	0\\
9.21	0\\
9.22	0\\
9.23	0\\
9.24	0\\
9.25	0\\
9.26	0\\
9.27	0\\
9.28	0\\
9.29	0\\
9.3	0\\
9.31	0\\
9.32	0\\
9.33	0\\
9.34	0\\
9.35	0\\
9.36	0\\
9.37	0\\
9.38	0\\
9.39	0\\
9.4	0\\
9.41	0\\
9.42	0\\
9.43	0\\
9.44	0\\
9.45	0\\
9.46	0\\
9.47	0\\
9.48	0\\
9.49	0\\
9.5	0\\
9.51	0\\
9.52	0\\
9.53	0\\
9.54	0\\
9.55	0\\
9.56	0\\
9.57	0\\
9.58	0\\
9.59	0\\
9.6	0\\
9.61	0\\
9.62	0\\
9.63	0\\
9.64	0\\
9.65	0\\
9.66	0\\
9.67	0\\
9.68	0\\
9.69	0\\
9.7	0\\
9.71	0\\
9.72	0\\
9.73	0\\
9.74	0\\
9.75	0\\
9.76	0\\
9.77	0\\
9.78	0\\
9.79	0\\
9.8	0\\
9.81	0\\
9.82	0\\
9.83	0\\
9.84	0\\
9.85	0\\
9.86	0\\
9.87	0\\
9.88	0\\
9.89	0\\
9.9	0\\
9.91	0\\
9.92	0\\
9.93	0\\
9.94	0\\
9.95	0\\
9.96	0\\
9.97	0\\
9.98	0\\
9.99	0\\
10	0\\
10.01	0\\
10.02	0\\
10.03	0\\
10.04	0\\
10.05	0\\
10.06	0\\
10.07	0\\
10.08	0\\
10.09	0\\
10.1	0\\
10.11	0\\
10.12	0\\
10.13	0\\
10.14	0\\
10.15	0\\
10.16	0\\
10.17	0\\
10.18	0\\
10.19	0\\
10.2	0\\
10.21	0\\
10.22	0\\
10.23	0\\
10.24	0\\
10.25	0\\
10.26	0\\
10.27	0\\
10.28	0\\
10.29	0\\
10.3	0\\
10.31	0\\
10.32	0\\
10.33	0\\
10.34	0\\
10.35	0\\
10.36	0\\
10.37	0\\
10.38	0\\
10.39	0\\
10.4	0\\
10.41	0\\
10.42	0\\
10.43	0\\
10.44	0\\
10.45	0\\
10.46	0\\
10.47	0\\
10.48	0\\
10.49	0\\
10.5	0\\
10.51	0\\
10.52	0\\
10.53	0\\
10.54	0\\
10.55	0\\
10.56	0\\
10.57	0\\
10.58	0\\
10.59	0\\
10.6	0\\
10.61	0\\
10.62	0\\
10.63	0\\
10.64	0\\
10.65	0\\
10.66	0\\
10.67	0\\
10.68	0\\
10.69	0\\
10.7	0\\
10.71	0\\
10.72	0\\
10.73	0\\
10.74	0\\
10.75	0\\
10.76	0\\
10.77	0\\
10.78	0\\
10.79	0\\
10.8	0\\
10.81	0\\
10.82	0\\
10.83	0\\
10.84	0\\
10.85	0\\
10.86	0\\
10.87	0\\
10.88	0\\
10.89	0\\
10.9	0\\
10.91	0\\
10.92	0\\
10.93	0\\
10.94	0\\
10.95	0\\
10.96	0\\
10.97	0\\
10.98	0\\
10.99	0\\
11	0\\
11.01	0\\
11.02	0\\
11.03	0\\
11.04	0\\
11.05	0\\
11.06	0\\
11.07	0\\
11.08	0\\
11.09	0\\
11.1	0\\
11.11	0\\
11.12	0\\
11.13	0\\
11.14	0\\
11.15	0\\
11.16	0\\
11.17	0\\
11.18	0\\
11.19	0\\
11.2	0\\
11.21	0\\
11.22	0\\
11.23	0\\
11.24	0\\
11.25	0\\
11.26	0\\
11.27	0\\
11.28	0\\
11.29	0\\
11.3	0\\
11.31	0\\
11.32	0\\
11.33	0\\
11.34	0\\
11.35	0\\
11.36	0\\
11.37	0\\
11.38	0\\
11.39	0\\
11.4	0\\
11.41	0\\
11.42	0\\
11.43	0\\
11.44	0\\
11.45	0\\
11.46	0\\
11.47	0\\
11.48	0\\
11.49	0\\
11.5	0\\
11.51	0\\
11.52	0\\
11.53	0\\
11.54	0\\
11.55	0\\
11.56	0\\
11.57	0\\
11.58	0\\
11.59	0\\
11.6	0\\
11.61	0\\
11.62	0\\
11.63	0\\
11.64	0\\
11.65	0\\
11.66	0\\
11.67	0\\
11.68	0\\
11.69	0\\
11.7	0\\
11.71	0\\
11.72	0\\
11.73	0\\
11.74	0\\
11.75	0\\
11.76	0\\
11.77	0\\
11.78	0\\
11.79	0\\
11.8	0\\
11.81	0\\
11.82	0\\
11.83	0\\
11.84	0\\
11.85	0\\
11.86	0\\
11.87	0\\
11.88	0\\
11.89	0\\
11.9	0\\
11.91	0\\
11.92	0\\
11.93	0\\
11.94	0\\
11.95	0\\
11.96	0\\
11.97	0\\
11.98	0\\
11.99	0\\
12	0\\
12.01	0\\
12.02	0\\
12.03	0\\
12.04	0\\
12.05	0\\
12.06	0\\
12.07	0\\
12.08	0\\
12.09	0\\
12.1	0\\
12.11	0\\
12.12	0\\
12.13	0\\
12.14	0\\
12.15	0\\
12.16	0\\
12.17	0\\
12.18	0\\
12.19	0\\
12.2	0\\
12.21	0\\
12.22	0\\
12.23	0\\
12.24	0\\
12.25	0\\
12.26	0\\
12.27	0\\
12.28	0\\
12.29	0\\
12.3	0\\
12.31	0\\
12.32	0\\
12.33	0\\
12.34	0\\
12.35	0\\
12.36	0\\
12.37	0\\
12.38	0\\
12.39	0\\
12.4	0\\
12.41	0\\
12.42	0\\
12.43	0\\
12.44	0\\
12.45	0\\
12.46	0\\
12.47	0\\
12.48	0\\
12.49	0\\
12.5	0\\
12.51	0\\
12.52	0\\
12.53	0\\
12.54	0\\
12.55	0\\
12.56	0\\
12.57	0\\
12.58	0\\
12.59	0\\
12.6	0\\
12.61	0\\
12.62	0\\
12.63	0\\
12.64	0\\
12.65	0\\
12.66	0\\
12.67	0\\
12.68	0\\
12.69	0\\
12.7	0\\
12.71	0\\
12.72	0\\
12.73	0\\
12.74	0\\
12.75	0\\
12.76	0\\
12.77	0\\
12.78	0\\
12.79	0\\
12.8	0\\
12.81	0\\
12.82	0\\
12.83	0\\
12.84	0\\
12.85	0\\
12.86	0\\
12.87	0\\
12.88	0\\
12.89	0\\
12.9	0\\
12.91	0\\
12.92	0\\
12.93	0\\
12.94	0\\
12.95	0\\
12.96	0\\
12.97	0\\
12.98	0\\
12.99	0\\
13	0\\
13.01	0\\
13.02	0\\
13.03	0\\
13.04	0\\
13.05	0\\
13.06	0\\
13.07	0\\
13.08	0\\
13.09	0\\
13.1	0\\
13.11	0\\
13.12	0\\
13.13	0\\
13.14	0\\
13.15	0\\
13.16	0\\
13.17	0\\
13.18	0\\
13.19	0\\
13.2	0\\
13.21	0\\
13.22	0\\
13.23	0\\
13.24	0\\
13.25	0\\
13.26	0\\
13.27	0\\
13.28	0\\
13.29	0\\
13.3	0\\
13.31	0\\
13.32	0\\
13.33	0\\
13.34	0\\
13.35	0\\
13.36	0\\
13.37	0\\
13.38	0\\
13.39	0\\
13.4	0\\
13.41	0\\
13.42	0\\
13.43	0\\
13.44	0\\
13.45	0\\
13.46	0\\
13.47	0\\
13.48	0\\
13.49	0\\
13.5	0\\
13.51	0\\
13.52	0\\
13.53	0\\
13.54	0\\
13.55	0\\
13.56	0\\
13.57	0\\
13.58	0\\
13.59	0\\
13.6	0\\
13.61	0\\
13.62	0\\
13.63	0\\
13.64	0\\
13.65	0\\
13.66	0\\
13.67	0\\
13.68	0\\
13.69	0\\
13.7	0\\
13.71	0\\
13.72	0\\
13.73	0\\
13.74	0\\
13.75	0\\
13.76	0\\
13.77	0\\
13.78	0\\
13.79	0\\
13.8	0\\
13.81	0\\
13.82	0\\
13.83	0\\
13.84	0\\
13.85	0\\
13.86	0\\
13.87	0\\
13.88	0\\
13.89	0\\
13.9	0\\
13.91	0\\
13.92	0\\
13.93	0\\
13.94	0\\
13.95	0\\
13.96	0\\
13.97	0\\
13.98	0\\
13.99	0\\
14	0\\
14.01	0\\
14.02	0\\
14.03	0\\
14.04	0\\
14.05	0\\
14.06	0\\
14.07	0\\
14.08	0\\
14.09	0\\
14.1	0\\
14.11	0\\
14.12	0\\
14.13	0\\
14.14	0\\
14.15	0\\
14.16	0\\
14.17	0\\
14.18	0\\
14.19	0\\
14.2	0\\
14.21	0\\
14.22	0\\
14.23	0\\
14.24	0\\
14.25	0\\
14.26	0\\
14.27	0\\
14.28	0\\
14.29	0\\
14.3	0\\
14.31	0\\
14.32	0\\
14.33	0\\
14.34	0\\
14.35	0\\
14.36	0\\
14.37	0\\
14.38	0\\
14.39	0\\
14.4	0\\
14.41	0\\
14.42	0\\
14.43	0\\
14.44	0\\
14.45	0\\
14.46	0\\
14.47	0\\
14.48	0\\
14.49	0\\
14.5	0\\
14.51	0\\
14.52	0\\
14.53	0\\
14.54	0\\
14.55	0\\
14.56	0\\
14.57	0\\
14.58	0\\
14.59	0\\
14.6	0\\
14.61	0\\
14.62	0\\
14.63	0\\
14.64	0\\
14.65	0\\
14.66	0\\
14.67	0\\
14.68	0\\
14.69	0\\
14.7	0\\
14.71	0\\
14.72	0\\
14.73	0\\
14.74	0\\
14.75	0\\
14.76	0\\
14.77	0\\
14.78	0\\
14.79	0\\
14.8	0\\
14.81	0\\
14.82	0\\
14.83	0\\
14.84	0\\
14.85	0\\
14.86	0\\
14.87	0\\
14.88	0\\
14.89	0\\
14.9	0\\
14.91	0\\
14.92	0\\
14.93	0\\
14.94	0\\
14.95	0\\
14.96	0\\
14.97	0\\
14.98	0\\
14.99	0\\
15	0\\
15.01	0\\
15.02	0\\
15.03	0\\
15.04	0\\
15.05	0\\
15.06	0\\
15.07	0\\
15.08	0\\
15.09	0\\
15.1	0\\
15.11	0\\
15.12	0\\
15.13	0\\
15.14	0\\
15.15	0\\
15.16	0\\
15.17	0\\
15.18	0\\
15.19	0\\
15.2	0\\
15.21	0\\
15.22	0\\
15.23	0\\
15.24	0\\
15.25	0\\
15.26	0\\
15.27	0\\
15.28	0\\
15.29	0\\
15.3	0\\
15.31	0\\
15.32	0\\
15.33	0\\
15.34	0\\
15.35	0\\
15.36	0\\
15.37	0\\
15.38	0\\
15.39	0\\
15.4	0\\
15.41	0\\
15.42	0\\
15.43	0\\
15.44	0\\
15.45	0\\
15.46	0\\
15.47	0\\
15.48	0\\
15.49	0\\
15.5	0\\
15.51	0\\
15.52	0\\
15.53	0\\
15.54	0\\
15.55	0\\
15.56	0\\
15.57	0\\
15.58	0\\
15.59	0\\
15.6	0\\
15.61	0\\
15.62	0\\
15.63	0\\
15.64	0\\
15.65	0\\
15.66	0\\
15.67	0\\
15.68	0\\
15.69	0\\
15.7	0\\
15.71	0\\
15.72	0\\
15.73	0\\
15.74	0\\
15.75	0\\
15.76	0\\
15.77	0\\
15.78	0\\
15.79	0\\
15.8	0\\
15.81	0\\
15.82	0\\
15.83	0\\
15.84	0\\
15.85	0\\
15.86	0\\
15.87	0\\
15.88	0\\
15.89	0\\
15.9	0\\
15.91	0\\
15.92	0\\
15.93	0\\
15.94	0\\
15.95	0\\
15.96	0\\
15.97	0\\
15.98	0\\
15.99	0\\
16	0\\
16.01	0\\
16.02	0\\
16.03	0\\
16.04	0\\
16.05	0\\
16.06	0\\
16.07	0\\
16.08	0\\
16.09	0\\
16.1	0\\
16.11	0\\
16.12	0\\
16.13	0\\
16.14	0\\
16.15	0\\
16.16	0\\
16.17	0\\
16.18	0\\
16.19	0\\
16.2	0\\
16.21	0\\
16.22	0\\
16.23	0\\
16.24	0\\
16.25	0\\
16.26	0\\
16.27	0\\
16.28	0\\
16.29	0\\
16.3	0\\
16.31	0\\
16.32	0\\
16.33	0\\
16.34	0\\
16.35	0\\
16.36	0\\
16.37	0\\
16.38	0\\
16.39	0\\
16.4	0\\
16.41	0\\
16.42	0\\
16.43	0\\
16.44	0\\
16.45	0\\
16.46	0\\
16.47	0\\
16.48	0\\
16.49	0\\
16.5	0\\
16.51	0\\
16.52	0\\
16.53	0\\
16.54	0\\
16.55	0\\
16.56	0\\
16.57	0\\
16.58	0\\
16.59	0\\
16.6	0\\
16.61	0\\
16.62	0\\
16.63	0\\
16.64	0\\
16.65	0\\
16.66	0\\
16.67	0\\
16.68	0\\
16.69	0\\
16.7	0\\
16.71	0\\
16.72	0\\
16.73	0\\
16.74	0\\
16.75	0\\
16.76	0\\
16.77	0\\
16.78	0\\
16.79	0\\
16.8	0\\
16.81	0\\
16.82	0\\
16.83	0\\
16.84	0\\
16.85	0\\
16.86	0\\
16.87	0\\
16.88	0\\
16.89	0\\
16.9	0\\
16.91	0\\
16.92	0\\
16.93	0\\
16.94	0\\
16.95	0\\
16.96	0\\
16.97	0\\
16.98	0\\
16.99	0\\
17	0\\
17.01	0\\
17.02	0\\
17.03	0\\
17.04	0\\
17.05	0\\
17.06	0\\
17.07	0\\
17.08	0\\
17.09	0\\
17.1	0\\
17.11	0\\
17.12	0\\
17.13	0\\
17.14	0\\
17.15	0\\
17.16	0\\
17.17	0\\
17.18	0\\
17.19	0\\
17.2	0\\
17.21	0\\
17.22	0\\
17.23	0\\
17.24	0\\
17.25	0\\
17.26	0\\
17.27	0\\
17.28	0\\
17.29	0\\
17.3	0\\
17.31	0\\
17.32	0\\
17.33	0\\
17.34	0\\
17.35	0\\
17.36	0\\
17.37	0\\
17.38	0\\
17.39	0\\
17.4	0\\
17.41	0\\
17.42	0\\
17.43	0\\
17.44	0\\
17.45	0\\
17.46	0\\
17.47	0\\
17.48	0\\
17.49	0\\
17.5	0\\
17.51	0\\
17.52	0\\
17.53	0\\
17.54	0\\
17.55	0\\
17.56	0\\
17.57	0\\
17.58	0\\
17.59	0\\
17.6	0\\
17.61	0\\
17.62	0\\
17.63	0\\
17.64	0\\
17.65	0\\
17.66	0\\
17.67	0\\
17.68	0\\
17.69	0\\
17.7	0\\
17.71	0\\
17.72	0\\
17.73	0\\
17.74	0\\
17.75	0\\
17.76	0\\
17.77	0\\
17.78	0\\
17.79	0\\
17.8	0\\
17.81	0\\
17.82	0\\
17.83	0\\
17.84	0\\
17.85	0\\
17.86	0\\
17.87	0\\
17.88	0\\
17.89	0\\
17.9	0\\
17.91	0\\
17.92	0\\
17.93	0\\
17.94	0\\
17.95	0\\
17.96	0\\
17.97	0\\
17.98	0\\
17.99	0\\
18	0\\
18.01	0\\
18.02	0\\
18.03	0\\
18.04	0\\
18.05	0\\
18.06	0\\
18.07	0\\
18.08	0\\
18.09	0\\
18.1	0\\
18.11	0\\
18.12	0\\
18.13	0\\
18.14	0\\
18.15	0\\
18.16	0\\
18.17	0\\
18.18	0\\
18.19	0\\
18.2	0\\
18.21	0\\
18.22	0\\
18.23	0\\
18.24	0\\
18.25	0\\
18.26	0\\
18.27	0\\
18.28	0\\
18.29	0\\
18.3	0\\
18.31	0\\
18.32	0\\
18.33	0\\
18.34	0\\
18.35	0\\
18.36	0\\
18.37	0\\
18.38	0\\
18.39	0\\
18.4	0\\
18.41	0\\
18.42	0\\
18.43	0\\
18.44	0\\
18.45	0\\
18.46	0\\
18.47	0\\
18.48	0\\
18.49	0\\
18.5	0\\
18.51	0\\
18.52	0\\
18.53	0\\
18.54	0\\
18.55	0\\
18.56	0\\
18.57	0\\
18.58	0\\
18.59	0\\
18.6	0\\
18.61	0\\
18.62	0\\
18.63	0\\
18.64	0\\
18.65	0\\
18.66	0\\
18.67	0\\
18.68	0\\
18.69	0\\
18.7	0\\
18.71	0\\
18.72	0\\
18.73	0\\
18.74	0\\
18.75	0\\
18.76	0\\
18.77	0\\
18.78	0\\
18.79	0\\
18.8	0\\
18.81	0\\
18.82	0\\
18.83	0\\
18.84	0\\
18.85	0\\
18.86	0\\
18.87	0\\
18.88	0\\
18.89	0\\
18.9	0\\
18.91	0\\
18.92	0\\
18.93	0\\
18.94	0\\
18.95	0\\
18.96	0\\
18.97	0\\
18.98	0\\
18.99	0\\
19	0\\
19.01	0\\
19.02	0\\
19.03	0\\
19.04	0\\
19.05	0\\
19.06	0\\
19.07	0\\
19.08	0\\
19.09	0\\
19.1	0\\
19.11	0\\
19.12	0\\
19.13	0\\
19.14	0\\
19.15	0\\
19.16	0\\
19.17	0\\
19.18	0\\
19.19	0\\
19.2	0\\
19.21	0\\
19.22	0\\
19.23	0\\
19.24	0\\
19.25	0\\
19.26	0\\
19.27	0\\
19.28	0\\
19.29	0\\
19.3	0\\
19.31	0\\
19.32	0\\
19.33	0\\
19.34	0\\
19.35	0\\
19.36	0\\
19.37	0\\
19.38	0\\
19.39	0\\
19.4	0\\
19.41	0\\
19.42	0\\
19.43	0\\
19.44	0\\
19.45	0\\
19.46	0\\
19.47	0\\
19.48	0\\
19.49	0\\
19.5	0\\
19.51	0\\
19.52	0\\
19.53	0\\
19.54	0\\
19.55	0\\
19.56	0\\
19.57	0\\
19.58	0\\
19.59	0\\
19.6	0\\
19.61	0\\
19.62	0\\
19.63	0\\
19.64	0\\
19.65	0\\
19.66	0\\
19.67	0\\
19.68	0\\
19.69	0\\
19.7	0\\
19.71	0\\
19.72	0\\
19.73	0\\
19.74	0\\
19.75	0\\
19.76	0\\
19.77	0\\
19.78	0\\
19.79	0\\
19.8	0\\
19.81	0\\
19.82	0\\
19.83	0\\
19.84	0\\
19.85	0\\
19.86	0\\
19.87	0\\
19.88	0\\
19.89	0\\
19.9	0\\
19.91	0\\
19.92	0\\
19.93	0\\
19.94	0\\
19.95	0\\
19.96	0\\
19.97	0\\
19.98	0\\
19.99	0\\
20	0\\
20.01	0\\
20.02	0\\
20.03	0\\
20.04	0\\
20.05	0\\
20.06	0\\
20.07	0\\
20.08	0\\
20.09	0\\
20.1	0\\
20.11	0\\
20.12	0\\
20.13	0\\
20.14	0\\
20.15	0\\
20.16	0\\
20.17	0\\
20.18	0\\
20.19	0\\
20.2	0\\
20.21	0\\
20.22	0\\
20.23	0\\
20.24	0\\
20.25	0\\
20.26	0\\
20.27	0\\
20.28	0\\
20.29	0\\
20.3	0\\
20.31	0\\
20.32	0\\
20.33	0\\
20.34	0\\
20.35	0\\
20.36	0\\
20.37	0\\
20.38	0\\
20.39	0\\
20.4	0\\
20.41	0\\
20.42	0\\
20.43	0\\
20.44	0\\
20.45	0\\
20.46	0\\
20.47	0\\
20.48	0\\
20.49	0\\
20.5	0\\
20.51	0\\
20.52	0\\
20.53	0\\
20.54	0\\
20.55	0\\
20.56	0\\
20.57	0\\
20.58	0\\
20.59	0\\
20.6	0\\
20.61	0\\
20.62	0\\
20.63	0\\
20.64	0\\
20.65	0\\
20.66	0\\
20.67	0\\
20.68	0\\
20.69	0\\
20.7	0\\
20.71	0\\
20.72	0\\
20.73	0\\
20.74	0\\
20.75	0\\
20.76	0\\
20.77	0\\
20.78	0\\
20.79	0\\
20.8	0\\
20.81	0\\
20.82	0\\
20.83	0\\
20.84	0\\
20.85	0\\
20.86	0\\
20.87	0\\
20.88	0\\
20.89	0\\
20.9	0\\
20.91	0\\
20.92	0\\
20.93	0\\
20.94	0\\
20.95	0\\
20.96	0\\
20.97	0\\
20.98	0\\
20.99	0\\
21	0\\
21.01	0\\
21.02	0\\
21.03	0\\
21.04	0\\
21.05	0\\
21.06	0\\
21.07	0\\
21.08	0\\
21.09	0\\
21.1	0\\
21.11	0\\
21.12	0\\
21.13	0\\
21.14	0\\
21.15	0\\
21.16	0\\
21.17	0\\
21.18	0\\
21.19	0\\
21.2	0\\
21.21	0\\
21.22	0\\
21.23	0\\
21.24	0\\
21.25	0\\
21.26	0\\
21.27	0\\
21.28	0\\
21.29	0\\
21.3	0\\
21.31	0\\
21.32	0\\
21.33	0\\
21.34	0\\
21.35	0\\
21.36	0\\
21.37	0\\
21.38	0\\
21.39	0\\
21.4	0\\
21.41	0\\
21.42	0\\
21.43	0\\
21.44	0\\
21.45	0\\
21.46	0\\
21.47	0\\
21.48	0\\
21.49	0\\
21.5	0\\
21.51	0\\
21.52	0\\
21.53	0\\
21.54	0\\
21.55	0\\
21.56	0\\
21.57	0\\
21.58	0\\
21.59	0\\
21.6	0\\
21.61	0\\
21.62	0\\
21.63	0\\
21.64	0\\
21.65	0\\
21.66	0\\
21.67	0\\
21.68	0\\
21.69	0\\
21.7	0\\
21.71	0\\
21.72	0\\
21.73	0\\
21.74	0\\
21.75	0\\
21.76	0\\
21.77	0\\
21.78	0\\
21.79	0\\
21.8	0\\
21.81	0\\
21.82	0\\
21.83	0\\
21.84	0\\
21.85	0\\
21.86	0\\
21.87	0\\
21.88	0\\
21.89	0\\
21.9	0\\
21.91	0\\
21.92	0\\
21.93	0\\
21.94	0\\
21.95	0\\
21.96	0\\
21.97	0\\
21.98	0\\
21.99	0\\
22	0\\
22.01	0\\
22.02	0\\
22.03	0\\
22.04	0\\
22.05	0\\
22.06	0\\
22.07	0\\
22.08	0\\
22.09	0\\
22.1	0\\
22.11	0\\
22.12	0\\
22.13	0\\
22.14	0\\
22.15	0\\
22.16	0\\
22.17	0\\
22.18	0\\
22.19	0\\
22.2	0\\
22.21	0\\
22.22	0\\
22.23	0\\
22.24	0\\
22.25	0\\
22.26	0\\
22.27	0\\
22.28	0\\
22.29	0\\
22.3	0\\
22.31	0\\
22.32	0\\
22.33	0\\
22.34	0\\
22.35	0\\
22.36	0\\
22.37	0\\
22.38	0\\
22.39	0\\
22.4	0\\
22.41	0\\
22.42	0\\
22.43	0\\
22.44	0\\
22.45	0\\
22.46	0\\
22.47	0\\
22.48	0\\
22.49	0\\
22.5	0\\
22.51	0\\
22.52	0\\
22.53	0\\
22.54	0\\
22.55	0\\
22.56	0\\
22.57	0\\
22.58	0\\
22.59	0\\
22.6	0\\
22.61	0\\
22.62	0\\
22.63	0\\
22.64	0\\
22.65	0\\
22.66	0\\
22.67	0\\
22.68	0\\
22.69	0\\
22.7	0\\
22.71	0\\
22.72	0\\
22.73	0\\
22.74	0\\
22.75	0\\
22.76	0\\
22.77	0\\
22.78	0\\
22.79	0\\
22.8	0\\
22.81	0\\
22.82	0\\
22.83	0\\
22.84	0\\
22.85	0\\
22.86	0\\
22.87	0\\
22.88	0\\
22.89	0\\
22.9	0\\
22.91	0\\
22.92	0\\
22.93	0\\
22.94	0\\
22.95	0\\
22.96	0\\
22.97	0\\
22.98	0\\
22.99	0\\
23	0\\
23.01	0\\
23.02	0\\
23.03	0\\
23.04	0\\
23.05	0\\
23.06	0\\
23.07	0\\
23.08	0\\
23.09	0\\
23.1	0\\
23.11	0\\
23.12	0\\
23.13	0\\
23.14	0\\
23.15	0\\
23.16	0\\
23.17	0\\
23.18	0\\
23.19	0\\
23.2	0\\
23.21	0\\
23.22	0\\
23.23	0\\
23.24	0\\
23.25	0\\
23.26	0\\
23.27	0\\
23.28	0\\
23.29	0\\
23.3	0\\
23.31	0\\
23.32	0\\
23.33	0\\
23.34	0\\
23.35	0\\
23.36	0\\
23.37	0\\
23.38	0\\
23.39	0\\
23.4	0\\
23.41	0\\
23.42	0\\
23.43	0\\
23.44	0\\
23.45	0\\
23.46	0\\
23.47	0\\
23.48	0\\
23.49	0\\
23.5	0\\
23.51	0\\
23.52	0\\
23.53	0\\
23.54	0\\
23.55	0\\
23.56	0\\
23.57	0\\
23.58	0\\
23.59	0\\
23.6	0\\
23.61	0\\
23.62	0\\
23.63	0\\
23.64	0\\
23.65	0\\
23.66	0\\
23.67	0\\
23.68	0\\
23.69	0\\
23.7	0\\
23.71	0\\
23.72	0\\
23.73	0\\
23.74	0\\
23.75	0\\
23.76	0\\
23.77	0\\
23.78	0\\
23.79	0\\
23.8	0\\
23.81	0\\
23.82	0\\
23.83	0\\
23.84	0\\
23.85	0\\
23.86	0\\
23.87	0\\
23.88	0\\
23.89	0\\
23.9	0\\
23.91	0\\
23.92	0\\
23.93	0\\
23.94	0\\
23.95	0\\
23.96	0\\
23.97	0\\
23.98	0\\
23.99	0\\
24	0\\
24.01	0\\
24.02	0\\
24.03	0\\
24.04	0\\
24.05	0\\
24.06	0\\
24.07	0\\
24.08	0\\
24.09	0\\
24.1	0\\
24.11	0\\
24.12	0\\
24.13	0\\
24.14	0\\
24.15	0\\
24.16	0\\
24.17	0\\
24.18	0\\
24.19	0\\
24.2	0\\
24.21	0\\
24.22	0\\
24.23	0\\
24.24	0\\
24.25	0\\
24.26	0\\
24.27	0\\
24.28	0\\
24.29	0\\
24.3	0\\
24.31	0\\
24.32	0\\
24.33	0\\
24.34	0\\
24.35	0\\
24.36	0\\
24.37	0\\
24.38	0\\
24.39	0\\
24.4	0\\
24.41	0\\
24.42	0\\
24.43	0\\
24.44	0\\
24.45	0\\
24.46	0\\
24.47	0\\
24.48	0\\
24.49	0\\
24.5	0\\
24.51	0\\
24.52	0\\
24.53	0\\
24.54	0\\
24.55	0\\
24.56	0\\
24.57	0\\
24.58	0\\
24.59	0\\
24.6	0\\
24.61	0\\
24.62	0\\
24.63	0\\
24.64	0\\
24.65	0\\
24.66	0\\
24.67	0\\
24.68	0\\
24.69	0\\
24.7	0\\
24.71	0\\
24.72	0\\
24.73	0\\
24.74	0\\
24.75	0\\
24.76	0\\
24.77	0\\
24.78	0\\
24.79	0\\
24.8	0\\
24.81	0\\
24.82	0\\
24.83	0\\
24.84	0\\
24.85	0\\
24.86	0\\
24.87	0\\
24.88	0\\
24.89	0\\
24.9	0\\
24.91	0\\
24.92	0\\
24.93	0\\
24.94	0\\
24.95	0\\
24.96	0\\
24.97	0\\
24.98	0\\
24.99	0\\
25	0\\
25.01	0\\
25.02	0\\
25.03	0\\
25.04	0\\
25.05	0\\
25.06	0\\
25.07	0\\
25.08	0\\
25.09	0\\
25.1	0\\
25.11	0\\
25.12	0\\
25.13	0\\
25.14	0\\
25.15	0\\
25.16	0\\
25.17	0\\
25.18	0\\
25.19	0\\
25.2	0\\
25.21	0\\
25.22	0\\
25.23	0\\
25.24	0\\
25.25	0\\
25.26	0\\
25.27	0\\
25.28	0\\
25.29	0\\
25.3	0\\
25.31	0\\
25.32	0\\
25.33	0\\
25.34	0\\
25.35	0\\
25.36	0\\
25.37	0\\
25.38	0\\
25.39	0\\
25.4	0\\
25.41	0\\
25.42	0\\
25.43	0\\
25.44	0\\
25.45	0\\
25.46	0\\
25.47	0\\
25.48	0\\
25.49	0\\
25.5	0\\
25.51	0\\
25.52	0\\
25.53	0\\
25.54	0\\
25.55	0\\
25.56	0\\
25.57	0\\
25.58	0\\
25.59	0\\
25.6	0\\
25.61	0\\
25.62	0\\
25.63	0\\
25.64	0\\
25.65	0\\
25.66	0\\
25.67	0\\
25.68	0\\
25.69	0\\
25.7	0\\
25.71	0\\
25.72	0\\
25.73	0\\
25.74	0\\
25.75	0\\
25.76	0\\
25.77	0\\
25.78	0\\
25.79	0\\
25.8	0\\
25.81	0\\
25.82	0\\
25.83	0\\
25.84	0\\
25.85	0\\
25.86	0\\
25.87	0\\
25.88	0\\
25.89	0\\
25.9	0\\
25.91	0\\
25.92	0\\
25.93	0\\
25.94	0\\
25.95	0\\
25.96	0\\
25.97	0\\
25.98	0\\
25.99	0\\
26	0\\
26.01	0\\
26.02	0\\
26.03	0\\
26.04	0\\
26.05	0\\
26.06	0\\
26.07	0\\
26.08	0\\
26.09	0\\
26.1	0\\
26.11	0\\
26.12	0\\
26.13	0\\
26.14	0\\
26.15	0\\
26.16	0\\
26.17	0\\
26.18	0\\
26.19	0\\
26.2	0\\
26.21	0\\
26.22	0\\
26.23	0\\
26.24	0\\
26.25	0\\
26.26	0\\
26.27	0\\
26.28	0\\
26.29	0\\
26.3	0\\
26.31	0\\
26.32	0\\
26.33	0\\
26.34	0\\
26.35	0\\
26.36	0\\
26.37	0\\
26.38	0\\
26.39	0\\
26.4	0\\
26.41	0\\
26.42	0\\
26.43	0\\
26.44	0\\
26.45	0\\
26.46	0\\
26.47	0\\
26.48	0\\
26.49	0\\
26.5	0\\
26.51	0\\
26.52	0\\
26.53	0\\
26.54	0\\
26.55	0\\
26.56	0\\
26.57	0\\
26.58	0\\
26.59	0\\
26.6	0\\
26.61	0\\
26.62	0\\
26.63	0\\
26.64	0\\
26.65	0\\
26.66	0\\
26.67	0\\
26.68	0\\
26.69	0\\
26.7	0\\
26.71	0\\
26.72	0\\
26.73	0\\
26.74	0\\
26.75	0\\
26.76	0\\
26.77	0\\
26.78	0\\
26.79	0\\
26.8	0\\
26.81	0\\
26.82	0\\
26.83	0\\
26.84	0\\
26.85	0\\
26.86	0\\
26.87	0\\
26.88	0\\
26.89	0\\
26.9	0\\
26.91	0\\
26.92	0\\
26.93	0\\
26.94	0\\
26.95	0\\
26.96	0\\
26.97	0\\
26.98	0\\
26.99	0\\
27	0\\
27.01	0\\
27.02	0\\
27.03	0\\
27.04	0\\
27.05	0\\
27.06	0\\
27.07	0\\
27.08	0\\
27.09	0\\
27.1	0\\
27.11	0\\
27.12	0\\
27.13	0\\
27.14	0\\
27.15	0\\
27.16	0\\
27.17	0\\
27.18	0\\
27.19	0\\
27.2	0\\
27.21	0\\
27.22	0\\
27.23	0\\
27.24	0\\
27.25	0\\
27.26	0\\
27.27	0\\
27.28	0\\
27.29	0\\
27.3	0\\
27.31	0\\
27.32	0\\
27.33	0\\
27.34	0\\
27.35	0\\
27.36	0\\
27.37	0\\
27.38	0\\
27.39	0\\
27.4	0\\
27.41	0\\
27.42	0\\
27.43	0\\
27.44	0\\
27.45	0\\
27.46	0\\
27.47	0\\
27.48	0\\
27.49	0\\
27.5	0\\
27.51	0\\
27.52	0\\
27.53	0\\
27.54	0\\
27.55	0\\
27.56	0\\
27.57	0\\
27.58	0\\
27.59	0\\
27.6	0\\
27.61	0\\
27.62	0\\
27.63	0\\
27.64	0\\
27.65	0\\
27.66	0\\
27.67	0\\
27.68	0\\
27.69	0\\
27.7	0\\
27.71	0\\
27.72	0\\
27.73	0\\
27.74	0\\
27.75	0\\
27.76	0\\
27.77	0\\
27.78	0\\
27.79	0\\
27.8	0\\
27.81	0\\
27.82	0\\
27.83	0\\
27.84	0\\
27.85	0\\
27.86	0\\
27.87	0\\
27.88	0\\
27.89	0\\
27.9	0\\
27.91	0\\
27.92	0\\
27.93	0\\
27.94	0\\
27.95	0\\
27.96	0\\
27.97	0\\
27.98	0\\
27.99	0\\
28	0\\
28.01	0\\
28.02	0\\
28.03	0\\
28.04	0\\
28.05	0\\
28.06	0\\
28.07	0\\
28.08	0\\
28.09	0\\
28.1	0\\
28.11	0\\
28.12	0\\
28.13	0\\
28.14	0\\
28.15	0\\
28.16	0\\
28.17	0\\
28.18	0\\
28.19	0\\
28.2	0\\
28.21	0\\
28.22	0\\
28.23	0\\
28.24	0\\
28.25	0\\
28.26	0\\
28.27	0\\
28.28	0\\
28.29	0\\
28.3	0\\
28.31	0\\
28.32	0\\
28.33	0\\
28.34	0\\
28.35	0\\
28.36	0\\
28.37	0\\
28.38	0\\
28.39	0\\
28.4	0\\
28.41	0\\
28.42	0\\
28.43	0\\
28.44	0\\
28.45	0\\
28.46	0\\
28.47	0\\
28.48	0\\
28.49	0\\
28.5	0\\
28.51	0\\
28.52	0\\
28.53	0\\
28.54	0\\
28.55	0\\
28.56	0\\
28.57	0\\
28.58	0\\
28.59	0\\
28.6	0\\
28.61	0\\
28.62	0\\
28.63	0\\
28.64	0\\
28.65	0\\
28.66	0\\
28.67	0\\
28.68	0\\
28.69	0\\
28.7	0\\
28.71	0\\
28.72	0\\
28.73	0\\
28.74	0\\
28.75	0\\
28.76	0\\
28.77	0\\
28.78	0\\
28.79	0\\
28.8	0\\
28.81	0\\
28.82	0\\
28.83	0\\
28.84	0\\
28.85	0\\
28.86	0\\
28.87	0\\
28.88	0\\
28.89	0\\
28.9	0\\
28.91	0\\
28.92	0\\
28.93	0\\
28.94	0\\
28.95	0\\
28.96	0\\
28.97	0\\
28.98	0\\
28.99	0\\
29	0\\
29.01	0\\
29.02	0\\
29.03	0\\
29.04	0\\
29.05	0\\
29.06	0\\
29.07	0\\
29.08	0\\
29.09	0\\
29.1	0\\
29.11	0\\
29.12	0\\
29.13	0\\
29.14	0\\
29.15	0\\
29.16	0\\
29.17	0\\
29.18	0\\
29.19	0\\
29.2	0\\
29.21	0\\
29.22	0\\
29.23	0\\
29.24	0\\
29.25	0\\
29.26	0\\
29.27	0\\
29.28	0\\
29.29	0\\
29.3	0\\
29.31	0\\
29.32	0\\
29.33	0\\
29.34	0\\
29.35	0\\
29.36	0\\
29.37	0\\
29.38	0\\
29.39	0\\
29.4	0\\
29.41	0\\
29.42	0\\
29.43	0\\
29.44	0\\
29.45	0\\
29.46	0\\
29.47	0\\
29.48	0\\
29.49	0\\
29.5	0\\
29.51	0\\
29.52	0\\
29.53	0\\
29.54	0\\
29.55	0\\
29.56	0\\
29.57	0\\
29.58	0\\
29.59	0\\
29.6	0\\
29.61	0\\
29.62	0\\
29.63	0\\
29.64	0\\
29.65	0\\
29.66	0\\
29.67	0\\
29.68	0\\
29.69	0\\
29.7	0\\
29.71	0\\
29.72	0\\
29.73	0\\
29.74	0\\
29.75	0\\
29.76	0\\
29.77	0\\
29.78	0\\
29.79	0\\
29.8	0\\
29.81	0\\
29.82	0\\
29.83	0\\
29.84	0\\
29.85	0\\
29.86	0\\
29.87	0\\
29.88	0\\
29.89	0\\
29.9	0\\
29.91	0\\
29.92	0\\
29.93	0\\
29.94	0\\
29.95	0\\
29.96	0\\
29.97	0\\
29.98	0\\
29.99	0\\
30	0\\
30.01	0\\
30.02	0\\
30.03	0\\
30.04	0\\
30.05	0\\
30.06	0\\
30.07	0\\
30.08	0\\
30.09	0\\
30.1	0\\
30.11	0\\
30.12	0\\
30.13	0\\
30.14	0\\
30.15	0\\
30.16	0\\
30.17	0\\
30.18	0\\
30.19	0\\
30.2	0\\
30.21	0\\
30.22	0\\
30.23	0\\
30.24	0\\
30.25	0\\
30.26	0\\
30.27	0\\
30.28	0\\
30.29	0\\
30.3	0\\
30.31	0\\
30.32	0\\
30.33	0\\
30.34	0\\
30.35	0\\
30.36	0\\
30.37	0\\
30.38	0\\
30.39	0\\
30.4	0\\
30.41	0\\
30.42	0\\
30.43	0\\
30.44	0\\
30.45	0\\
30.46	0\\
30.47	0\\
30.48	0\\
30.49	0\\
30.5	0\\
30.51	0\\
30.52	0\\
30.53	0\\
30.54	0\\
30.55	0\\
30.56	0\\
30.57	0\\
30.58	0\\
30.59	0\\
30.6	0\\
30.61	0\\
30.62	0\\
30.63	0\\
30.64	0\\
30.65	0\\
30.66	0\\
30.67	0\\
30.68	0\\
30.69	0\\
30.7	0\\
30.71	0\\
30.72	0\\
30.73	0\\
30.74	0\\
30.75	0\\
30.76	0\\
30.77	0\\
30.78	0\\
30.79	0\\
30.8	0\\
30.81	0\\
30.82	0\\
30.83	0\\
30.84	0\\
30.85	0\\
30.86	0\\
30.87	0\\
30.88	0\\
30.89	0\\
30.9	0\\
30.91	0\\
30.92	0\\
30.93	0\\
30.94	0\\
30.95	0\\
30.96	0\\
30.97	0\\
30.98	0\\
30.99	0\\
31	0\\
31.01	0\\
31.02	0\\
31.03	0\\
31.04	0\\
31.05	0\\
31.06	0\\
31.07	0\\
31.08	0\\
31.09	0\\
31.1	0\\
31.11	0\\
31.12	0\\
31.13	0\\
31.14	0\\
31.15	0\\
31.16	0\\
31.17	0\\
31.18	0\\
31.19	0\\
31.2	0\\
31.21	0\\
31.22	0\\
31.23	0\\
31.24	0\\
31.25	0\\
31.26	0\\
31.27	0\\
31.28	0\\
31.29	0\\
31.3	0\\
31.31	0\\
31.32	0\\
31.33	0\\
31.34	0\\
31.35	0\\
31.36	0\\
31.37	0\\
31.38	0\\
31.39	0\\
31.4	0\\
31.41	0\\
31.42	0\\
31.43	0\\
31.44	0\\
31.45	0\\
31.46	0\\
31.47	0\\
31.48	0\\
31.49	0\\
31.5	0\\
31.51	0\\
31.52	0\\
31.53	0\\
31.54	0\\
31.55	0\\
31.56	0\\
31.57	0\\
31.58	0\\
31.59	0\\
31.6	0\\
31.61	0\\
31.62	0\\
31.63	0\\
31.64	0\\
31.65	0\\
31.66	0\\
31.67	0\\
31.68	0\\
31.69	0\\
31.7	0\\
31.71	0\\
31.72	0\\
31.73	0\\
31.74	0\\
31.75	0\\
31.76	0\\
31.77	0\\
31.78	0\\
31.79	0\\
31.8	0\\
31.81	0\\
31.82	0\\
31.83	0\\
31.84	0\\
31.85	0\\
31.86	0\\
31.87	0\\
31.88	0\\
31.89	0\\
31.9	0\\
31.91	0\\
31.92	0\\
31.93	0\\
31.94	0\\
31.95	0\\
31.96	0\\
31.97	0\\
31.98	0\\
31.99	0\\
32	0\\
32.01	0\\
32.02	0\\
32.03	0\\
32.04	0\\
32.05	0\\
32.06	0\\
32.07	0\\
32.08	0\\
32.09	0\\
32.1	0\\
32.11	0\\
32.12	0\\
32.13	0\\
32.14	0\\
32.15	0\\
32.16	0\\
32.17	0\\
32.18	0\\
32.19	0\\
32.2	0\\
32.21	0\\
32.22	0\\
32.23	0\\
32.24	0\\
32.25	0\\
32.26	0\\
32.27	0\\
32.28	0\\
32.29	0\\
32.3	0\\
32.31	0\\
32.32	0\\
32.33	0\\
32.34	0\\
32.35	0\\
32.36	0\\
32.37	0\\
32.38	0\\
32.39	0\\
32.4	0\\
32.41	0\\
32.42	0\\
32.43	0\\
32.44	0\\
32.45	0\\
32.46	0\\
32.47	0\\
32.48	0\\
32.49	0\\
32.5	0\\
32.51	0\\
32.52	0\\
32.53	0\\
32.54	0\\
32.55	0\\
32.56	0\\
32.57	0\\
32.58	0\\
32.59	0\\
32.6	0\\
32.61	0\\
32.62	0\\
32.63	0\\
32.64	0\\
32.65	0\\
32.66	0\\
32.67	0\\
32.68	0\\
32.69	0\\
32.7	0\\
32.71	0\\
32.72	0\\
32.73	0\\
32.74	0\\
32.75	0\\
32.76	0\\
32.77	0\\
32.78	0\\
32.79	0\\
32.8	0\\
32.81	0\\
32.82	0\\
32.83	0\\
32.84	0\\
32.85	0\\
32.86	0\\
32.87	0\\
32.88	0\\
32.89	0\\
32.9	0\\
32.91	0\\
32.92	0\\
32.93	0\\
32.94	0\\
32.95	0\\
32.96	0\\
32.97	0\\
32.98	0\\
32.99	0\\
33	0\\
33.01	0\\
33.02	0\\
33.03	0\\
33.04	0\\
33.05	0\\
33.06	0\\
33.07	0\\
33.08	0\\
33.09	0\\
33.1	0\\
33.11	0\\
33.12	0\\
33.13	0\\
33.14	0\\
33.15	0\\
33.16	0\\
33.17	0\\
33.18	0\\
33.19	0\\
33.2	0\\
33.21	0\\
33.22	0\\
33.23	0\\
33.24	0\\
33.25	0\\
33.26	0\\
33.27	0\\
33.28	0\\
33.29	0\\
33.3	0\\
33.31	0\\
33.32	0\\
33.33	0\\
33.34	0\\
33.35	0\\
33.36	0\\
33.37	0\\
33.38	0\\
33.39	0\\
33.4	0\\
33.41	0\\
33.42	0\\
33.43	0\\
33.44	0\\
33.45	0\\
33.46	0\\
33.47	0\\
33.48	0\\
33.49	0\\
33.5	0\\
33.51	0\\
33.52	0\\
33.53	0\\
33.54	0\\
33.55	0\\
33.56	0\\
33.57	0\\
33.58	0\\
33.59	0\\
33.6	0\\
33.61	0\\
33.62	0\\
33.63	0\\
33.64	0\\
33.65	0\\
33.66	0\\
33.67	0\\
33.68	0\\
33.69	0\\
33.7	0\\
33.71	0\\
33.72	0\\
33.73	0\\
33.74	0\\
33.75	0\\
33.76	0\\
33.77	0\\
33.78	0\\
33.79	0\\
33.8	0\\
33.81	0\\
33.82	0\\
33.83	0\\
33.84	0\\
33.85	0\\
33.86	0\\
33.87	0\\
33.88	0\\
33.89	0\\
33.9	0\\
33.91	0\\
33.92	0\\
33.93	0\\
33.94	0\\
33.95	0\\
33.96	0\\
33.97	0\\
33.98	0\\
33.99	0\\
34	0\\
34.01	0\\
34.02	0\\
34.03	0\\
34.04	0\\
34.05	0\\
34.06	0\\
34.07	0\\
34.08	0\\
34.09	0\\
34.1	0\\
34.11	0\\
34.12	0\\
34.13	0\\
34.14	0\\
34.15	0\\
34.16	0\\
34.17	0\\
34.18	0\\
34.19	0\\
34.2	0\\
34.21	0\\
34.22	0\\
34.23	0\\
34.24	0\\
34.25	0\\
34.26	0\\
34.27	0\\
34.28	0\\
34.29	0\\
34.3	0\\
34.31	0\\
34.32	0\\
34.33	0\\
34.34	0\\
34.35	0\\
34.36	0\\
34.37	0\\
34.38	0\\
34.39	0\\
34.4	0\\
34.41	0\\
34.42	0\\
34.43	0\\
34.44	0\\
34.45	0\\
34.46	0\\
34.47	0\\
34.48	0\\
34.49	0\\
34.5	0\\
34.51	0\\
34.52	0\\
34.53	0\\
34.54	0\\
34.55	0\\
34.56	0\\
34.57	0\\
34.58	0\\
34.59	0\\
34.6	0\\
34.61	0\\
34.62	0\\
34.63	0\\
34.64	0\\
34.65	0\\
34.66	0\\
34.67	0\\
34.68	0\\
34.69	0\\
34.7	0\\
34.71	0\\
34.72	0\\
34.73	0\\
34.74	0\\
34.75	0\\
34.76	0\\
34.77	0\\
34.78	0\\
34.79	0\\
34.8	0\\
34.81	0\\
34.82	0\\
34.83	0\\
34.84	0\\
34.85	0\\
34.86	0\\
34.87	0\\
34.88	0\\
34.89	0\\
34.9	0\\
34.91	0\\
34.92	0\\
34.93	0\\
34.94	0\\
34.95	0\\
34.96	0\\
34.97	0\\
34.98	0\\
34.99	0\\
35	0\\
35.01	0\\
35.02	0\\
35.03	0\\
35.04	0\\
35.05	0\\
35.06	0\\
35.07	0\\
35.08	0\\
35.09	0\\
35.1	0\\
35.11	0\\
35.12	0\\
35.13	0\\
35.14	0\\
35.15	0\\
35.16	0\\
35.17	0\\
35.18	0\\
35.19	0\\
35.2	0\\
35.21	0\\
35.22	0\\
35.23	0\\
35.24	0\\
35.25	0\\
35.26	0\\
35.27	0\\
35.28	0\\
35.29	0\\
35.3	0\\
35.31	0\\
35.32	0\\
35.33	0\\
35.34	0\\
35.35	0\\
35.36	0\\
35.37	0\\
35.38	0\\
35.39	0\\
35.4	0\\
35.41	0\\
35.42	0\\
35.43	0\\
35.44	0\\
35.45	0\\
35.46	0\\
35.47	0\\
35.48	0\\
35.49	0\\
35.5	0\\
35.51	0\\
35.52	0\\
35.53	0\\
35.54	0\\
35.55	0\\
35.56	0\\
35.57	0\\
35.58	0\\
35.59	0\\
35.6	0\\
35.61	0\\
35.62	0\\
35.63	0\\
35.64	0\\
35.65	0\\
35.66	0\\
35.67	0\\
35.68	0\\
35.69	0\\
35.7	0\\
35.71	0\\
35.72	0\\
35.73	0\\
35.74	0\\
35.75	0\\
35.76	0\\
35.77	0\\
35.78	0\\
35.79	0\\
35.8	0\\
35.81	0\\
35.82	0\\
35.83	0\\
35.84	0\\
35.85	0\\
35.86	0\\
35.87	0\\
35.88	0\\
35.89	0\\
35.9	0\\
35.91	0\\
35.92	0\\
35.93	0\\
35.94	0\\
35.95	0\\
35.96	0\\
35.97	0\\
35.98	0\\
35.99	0\\
36	0\\
36.01	0\\
36.02	0\\
36.03	0\\
36.04	0\\
36.05	0\\
36.06	0\\
36.07	0\\
36.08	0\\
36.09	0\\
36.1	0\\
36.11	0\\
36.12	0\\
36.13	0\\
36.14	0\\
36.15	0\\
36.16	0\\
36.17	0\\
36.18	0\\
36.19	0\\
36.2	0\\
36.21	0\\
36.22	0\\
36.23	0\\
36.24	0\\
36.25	0\\
36.26	0\\
36.27	0\\
36.28	0\\
36.29	0\\
36.3	0\\
36.31	0\\
36.32	0\\
36.33	0\\
36.34	0\\
36.35	0\\
36.36	0\\
36.37	0\\
36.38	0\\
36.39	0\\
36.4	0\\
36.41	0\\
36.42	0\\
36.43	0\\
36.44	0\\
36.45	0\\
36.46	0\\
36.47	0\\
36.48	0\\
36.49	0\\
36.5	0\\
36.51	0\\
36.52	0\\
36.53	0\\
36.54	0\\
36.55	0\\
36.56	0\\
36.57	0\\
36.58	0\\
36.59	0\\
36.6	0\\
36.61	0\\
36.62	0\\
36.63	0\\
36.64	0\\
36.65	0\\
36.66	0\\
36.67	0\\
36.68	0\\
36.69	0\\
36.7	0\\
36.71	0\\
36.72	0\\
36.73	0\\
36.74	0\\
36.75	0\\
36.76	0\\
36.77	0\\
36.78	0\\
36.79	0\\
36.8	0\\
36.81	0\\
36.82	0\\
36.83	0\\
36.84	0\\
36.85	0\\
36.86	0\\
36.87	0\\
36.88	0\\
36.89	0\\
36.9	0\\
36.91	0\\
36.92	0\\
36.93	0\\
36.94	0\\
36.95	0\\
36.96	0\\
36.97	0\\
36.98	0\\
36.99	0\\
37	0\\
37.01	0\\
37.02	0\\
37.03	0\\
37.04	0\\
37.05	0\\
37.06	0\\
37.07	0\\
37.08	0\\
37.09	0\\
37.1	0\\
37.11	0\\
37.12	0\\
37.13	0\\
37.14	0\\
37.15	0\\
37.16	0\\
37.17	0\\
37.18	0\\
37.19	0\\
37.2	0\\
37.21	0\\
37.22	0\\
37.23	0\\
37.24	0\\
37.25	0\\
37.26	0\\
37.27	0\\
37.28	0\\
37.29	0\\
37.3	0\\
37.31	0\\
37.32	0\\
37.33	0\\
37.34	0\\
37.35	0\\
37.36	0\\
37.37	0\\
37.38	0\\
37.39	0\\
37.4	0\\
37.41	0\\
37.42	0\\
37.43	0\\
37.44	0\\
37.45	0\\
37.46	0\\
37.47	0\\
37.48	0\\
37.49	0\\
37.5	0\\
37.51	0\\
37.52	0\\
37.53	0\\
37.54	0\\
37.55	0\\
37.56	0\\
37.57	0\\
37.58	0\\
37.59	0\\
37.6	0\\
37.61	0\\
37.62	0\\
37.63	0\\
37.64	0\\
37.65	0\\
37.66	0\\
37.67	0\\
37.68	0\\
37.69	0\\
37.7	0\\
37.71	0\\
37.72	0\\
37.73	0\\
37.74	0\\
37.75	0\\
37.76	0\\
37.77	0\\
37.78	0\\
37.79	0\\
37.8	0\\
37.81	0\\
37.82	0\\
37.83	0\\
37.84	0\\
37.85	0\\
37.86	0\\
37.87	0\\
37.88	0\\
37.89	0\\
37.9	0\\
37.91	0\\
37.92	0\\
37.93	0\\
37.94	0\\
37.95	0\\
37.96	0\\
37.97	0\\
37.98	0\\
37.99	0\\
38	0\\
38.01	0\\
38.02	0\\
38.03	0\\
38.04	0\\
38.05	0\\
38.06	0\\
38.07	0\\
38.08	0\\
38.09	0\\
38.1	0\\
38.11	0\\
38.12	0\\
38.13	0\\
38.14	0\\
38.15	0\\
38.16	0\\
38.17	0\\
38.18	0\\
38.19	0\\
38.2	0\\
38.21	0\\
38.22	0\\
38.23	0\\
38.24	0\\
38.25	0\\
38.26	0\\
38.27	0\\
38.28	0\\
38.29	0\\
38.3	0\\
38.31	0\\
38.32	0\\
38.33	0\\
38.34	0\\
38.35	0\\
38.36	0\\
38.37	0\\
38.38	0\\
38.39	0\\
38.4	0\\
38.41	0\\
38.42	0\\
38.43	0\\
38.44	0\\
38.45	0\\
38.46	0\\
38.47	0\\
38.48	0\\
38.49	0\\
38.5	0\\
38.51	0\\
38.52	0\\
38.53	0\\
38.54	0\\
38.55	0\\
38.56	0\\
38.57	0\\
38.58	0\\
38.59	0\\
38.6	0\\
38.61	0\\
38.62	0\\
38.63	0\\
38.64	0\\
38.65	0\\
38.66	0\\
38.67	0\\
38.68	0\\
38.69	0\\
38.7	0\\
38.71	0\\
38.72	0\\
38.73	0\\
38.74	0\\
38.75	0\\
38.76	0\\
38.77	0\\
38.78	0\\
38.79	0\\
38.8	0\\
38.81	0\\
38.82	0\\
38.83	0\\
38.84	0\\
38.85	0\\
38.86	0\\
38.87	0\\
38.88	0\\
38.89	0\\
38.9	0\\
38.91	0\\
38.92	0\\
38.93	0\\
38.94	0\\
38.95	0\\
38.96	0\\
38.97	0\\
38.98	0\\
38.99	0\\
39	0\\
39.01	0\\
39.02	0\\
39.03	0\\
39.04	0\\
39.05	0\\
39.06	0\\
39.07	0\\
39.08	0\\
39.09	0\\
39.1	0\\
39.11	0\\
39.12	0\\
39.13	0\\
39.14	0\\
39.15	0\\
39.16	0\\
39.17	0\\
39.18	0\\
39.19	0\\
39.2	0\\
39.21	0\\
39.22	0\\
39.23	0\\
39.24	0\\
39.25	0\\
39.26	0\\
39.27	0\\
39.28	0\\
39.29	0\\
39.3	0\\
39.31	0\\
39.32	0\\
39.33	0\\
39.34	0\\
39.35	0\\
39.36	0\\
39.37	0\\
39.38	0\\
39.39	0\\
39.4	0\\
39.41	0\\
39.42	0\\
39.43	0\\
39.44	0\\
39.45	0\\
39.46	0\\
39.47	0\\
39.48	0\\
39.49	0\\
39.5	0\\
39.51	0\\
39.52	0\\
39.53	0\\
39.54	0\\
39.55	0\\
39.56	0\\
39.57	0\\
39.58	0\\
39.59	0\\
39.6	0\\
39.61	0\\
39.62	0\\
39.63	0\\
39.64	0\\
39.65	0\\
39.66	0\\
39.67	0\\
39.68	0\\
39.69	0\\
39.7	0\\
39.71	0\\
39.72	0\\
39.73	0\\
39.74	0\\
39.75	0\\
39.76	0\\
39.77	0\\
39.78	0\\
39.79	0\\
39.8	0\\
39.81	0\\
39.82	0\\
39.83	0\\
39.84	0\\
39.85	0\\
39.86	0\\
39.87	0\\
39.88	0\\
39.89	0\\
39.9	0\\
39.91	0\\
39.92	0\\
39.93	0\\
39.94	0\\
39.95	0\\
39.96	0\\
39.97	0\\
39.98	0\\
39.99	0\\
40	0\\
40.01	0\\
};
\addplot [color=green,dashed,forget plot]
  table[row sep=crcr]{%
40.01	0\\
40.02	0\\
40.03	0\\
40.04	0\\
40.05	0\\
40.06	0\\
40.07	0\\
40.08	0\\
40.09	0\\
40.1	0\\
40.11	0\\
40.12	0\\
40.13	0\\
40.14	0\\
40.15	0\\
40.16	0\\
40.17	0\\
40.18	0\\
40.19	0\\
40.2	0\\
40.21	0\\
40.22	0\\
40.23	0\\
40.24	0\\
40.25	0\\
40.26	0\\
40.27	0\\
40.28	0\\
40.29	0\\
40.3	0\\
40.31	0\\
40.32	0\\
40.33	0\\
40.34	0\\
40.35	0\\
40.36	0\\
40.37	0\\
40.38	0\\
40.39	0\\
40.4	0\\
40.41	0\\
40.42	0\\
40.43	0\\
40.44	0\\
40.45	0\\
40.46	0\\
40.47	0\\
40.48	0\\
40.49	0\\
40.5	0\\
40.51	0\\
40.52	0\\
40.53	0\\
40.54	0\\
40.55	0\\
40.56	0\\
40.57	0\\
40.58	0\\
40.59	0\\
40.6	0\\
40.61	0\\
40.62	0\\
40.63	0\\
40.64	0\\
40.65	0\\
40.66	0\\
40.67	0\\
40.68	0\\
40.69	0\\
40.7	0\\
40.71	0\\
40.72	0\\
40.73	0\\
40.74	0\\
40.75	0\\
40.76	0\\
40.77	0\\
40.78	0\\
40.79	0\\
40.8	0\\
40.81	0\\
40.82	0\\
40.83	0\\
40.84	0\\
40.85	0\\
40.86	0\\
40.87	0\\
40.88	0\\
40.89	0\\
40.9	0\\
40.91	0\\
40.92	0\\
40.93	0\\
40.94	0\\
40.95	0\\
40.96	0\\
40.97	0\\
40.98	0\\
40.99	0\\
41	0\\
41.01	0\\
41.02	0\\
41.03	0\\
41.04	0\\
41.05	0\\
41.06	0\\
41.07	0\\
41.08	0\\
41.09	0\\
41.1	0\\
41.11	0\\
41.12	0\\
41.13	0\\
41.14	0\\
41.15	0\\
41.16	0\\
41.17	0\\
41.18	0\\
41.19	0\\
41.2	0\\
41.21	0\\
41.22	0\\
41.23	0\\
41.24	0\\
41.25	0\\
41.26	0\\
41.27	0\\
41.28	0\\
41.29	0\\
41.3	0\\
41.31	0\\
41.32	0\\
41.33	0\\
41.34	0\\
41.35	0\\
41.36	0\\
41.37	0\\
41.38	0\\
41.39	0\\
41.4	0\\
41.41	0\\
41.42	0\\
41.43	0\\
41.44	0\\
41.45	0\\
41.46	0\\
41.47	0\\
41.48	0\\
41.49	0\\
41.5	0\\
41.51	0\\
41.52	0\\
41.53	0\\
41.54	0\\
41.55	0\\
41.56	0\\
41.57	0\\
41.58	0\\
41.59	0\\
41.6	0\\
41.61	0\\
41.62	0\\
41.63	0\\
41.64	0\\
41.65	0\\
41.66	0\\
41.67	0\\
41.68	0\\
41.69	0\\
41.7	0\\
41.71	0\\
41.72	0\\
41.73	0\\
41.74	0\\
41.75	0\\
41.76	0\\
41.77	0\\
41.78	0\\
41.79	0\\
41.8	0\\
41.81	0\\
41.82	0\\
41.83	0\\
41.84	0\\
41.85	0\\
41.86	0\\
41.87	0\\
41.88	0\\
41.89	0\\
41.9	0\\
41.91	0\\
41.92	0\\
41.93	0\\
41.94	0\\
41.95	0\\
41.96	0\\
41.97	0\\
41.98	0\\
41.99	0\\
42	0\\
42.01	0\\
42.02	0\\
42.03	0\\
42.04	0\\
42.05	0\\
42.06	0\\
42.07	0\\
42.08	0\\
42.09	0\\
42.1	0\\
42.11	0\\
42.12	0\\
42.13	0\\
42.14	0\\
42.15	0\\
42.16	0\\
42.17	0\\
42.18	0\\
42.19	0\\
42.2	0\\
42.21	0\\
42.22	0\\
42.23	0\\
42.24	0\\
42.25	0\\
42.26	0\\
42.27	0\\
42.28	0\\
42.29	0\\
42.3	0\\
42.31	0\\
42.32	0\\
42.33	0\\
42.34	0\\
42.35	0\\
42.36	0\\
42.37	0\\
42.38	0\\
42.39	0\\
42.4	0\\
42.41	0\\
42.42	0\\
42.43	0\\
42.44	0\\
42.45	0\\
42.46	0\\
42.47	0\\
42.48	0\\
42.49	0\\
42.5	0\\
42.51	0\\
42.52	0\\
42.53	0\\
42.54	0\\
42.55	0\\
42.56	0\\
42.57	0\\
42.58	0\\
42.59	0\\
42.6	0\\
42.61	0\\
42.62	0\\
42.63	0\\
42.64	0\\
42.65	0\\
42.66	0\\
42.67	0\\
42.68	0\\
42.69	0\\
42.7	0\\
42.71	0\\
42.72	0\\
42.73	0\\
42.74	0\\
42.75	0\\
42.76	0\\
42.77	0\\
42.78	0\\
42.79	0\\
42.8	0\\
42.81	0\\
42.82	0\\
42.83	0\\
42.84	0\\
42.85	0\\
42.86	0\\
42.87	0\\
42.88	0\\
42.89	0\\
42.9	0\\
42.91	0\\
42.92	0\\
42.93	0\\
42.94	0\\
42.95	0\\
42.96	0\\
42.97	0\\
42.98	0\\
42.99	0\\
43	0\\
43.01	0\\
43.02	0\\
43.03	0\\
43.04	0\\
43.05	0\\
43.06	0\\
43.07	0\\
43.08	0\\
43.09	0\\
43.1	0\\
43.11	0\\
43.12	0\\
43.13	0\\
43.14	0\\
43.15	0\\
43.16	0\\
43.17	0\\
43.18	0\\
43.19	0\\
43.2	0\\
43.21	0\\
43.22	0\\
43.23	0\\
43.24	0\\
43.25	0\\
43.26	0\\
43.27	0\\
43.28	0\\
43.29	0\\
43.3	0\\
43.31	0\\
43.32	0\\
43.33	0\\
43.34	0\\
43.35	0\\
43.36	0\\
43.37	0\\
43.38	0\\
43.39	0\\
43.4	0\\
43.41	0\\
43.42	0\\
43.43	0\\
43.44	0\\
43.45	0\\
43.46	0\\
43.47	0\\
43.48	0\\
43.49	0\\
43.5	0\\
43.51	0\\
43.52	0\\
43.53	0\\
43.54	0\\
43.55	0\\
43.56	0\\
43.57	0\\
43.58	0\\
43.59	0\\
43.6	0\\
43.61	0\\
43.62	0\\
43.63	0\\
43.64	0\\
43.65	0\\
43.66	0\\
43.67	0\\
43.68	0\\
43.69	0\\
43.7	0\\
43.71	0\\
43.72	0\\
43.73	0\\
43.74	0\\
43.75	0\\
43.76	0\\
43.77	0\\
43.78	0\\
43.79	0\\
43.8	0\\
43.81	0\\
43.82	0\\
43.83	0\\
43.84	0\\
43.85	0\\
43.86	0\\
43.87	0\\
43.88	0\\
43.89	0\\
43.9	0\\
43.91	0\\
43.92	0\\
43.93	0\\
43.94	0\\
43.95	0\\
43.96	0\\
43.97	0\\
43.98	0\\
43.99	0\\
44	0\\
44.01	0\\
44.02	0\\
44.03	0\\
44.04	0\\
44.05	0\\
44.06	0\\
44.07	0\\
44.08	0\\
44.09	0\\
44.1	0\\
44.11	0\\
44.12	0\\
44.13	0\\
44.14	0\\
44.15	0\\
44.16	0\\
44.17	0\\
44.18	0\\
44.19	0\\
44.2	0\\
44.21	0\\
44.22	0\\
44.23	0\\
44.24	0\\
44.25	0\\
44.26	0\\
44.27	0\\
44.28	0\\
44.29	0\\
44.3	0\\
44.31	0\\
44.32	0\\
44.33	0\\
44.34	0\\
44.35	0\\
44.36	0\\
44.37	0\\
44.38	0\\
44.39	0\\
44.4	0\\
44.41	0\\
44.42	0\\
44.43	0\\
44.44	0\\
44.45	0\\
44.46	0\\
44.47	0\\
44.48	0\\
44.49	0\\
44.5	0\\
44.51	0\\
44.52	0\\
44.53	0\\
44.54	0\\
44.55	0\\
44.56	0\\
44.57	0\\
44.58	0\\
44.59	0\\
44.6	0\\
44.61	0\\
44.62	0\\
44.63	0\\
44.64	0\\
44.65	0\\
44.66	0\\
44.67	0\\
44.68	0\\
44.69	0\\
44.7	0\\
44.71	0\\
44.72	0\\
44.73	0\\
44.74	0\\
44.75	0\\
44.76	0\\
44.77	0\\
44.78	0\\
44.79	0\\
44.8	0\\
44.81	0\\
44.82	0\\
44.83	0\\
44.84	0\\
44.85	0\\
44.86	0\\
44.87	0\\
44.88	0\\
44.89	0\\
44.9	0\\
44.91	0\\
44.92	0\\
44.93	0\\
44.94	0\\
44.95	0\\
44.96	0\\
44.97	0\\
44.98	0\\
44.99	0\\
45	0\\
45.01	0\\
45.02	0\\
45.03	0\\
45.04	0\\
45.05	0\\
45.06	0\\
45.07	0\\
45.08	0\\
45.09	0\\
45.1	0\\
45.11	0\\
45.12	0\\
45.13	0\\
45.14	0\\
45.15	0\\
45.16	0\\
45.17	0\\
45.18	0\\
45.19	0\\
45.2	0\\
45.21	0\\
45.22	0\\
45.23	0\\
45.24	0\\
45.25	0\\
45.26	0\\
45.27	0\\
45.28	0\\
45.29	0\\
45.3	0\\
45.31	0\\
45.32	0\\
45.33	0\\
45.34	0\\
45.35	0\\
45.36	0\\
45.37	0\\
45.38	0\\
45.39	0\\
45.4	0\\
45.41	0\\
45.42	0\\
45.43	0\\
45.44	0\\
45.45	0\\
45.46	0\\
45.47	0\\
45.48	0\\
45.49	0\\
45.5	0\\
45.51	0\\
45.52	0\\
45.53	0\\
45.54	0\\
45.55	0\\
45.56	0\\
45.57	0\\
45.58	0\\
45.59	0\\
45.6	0\\
45.61	0\\
45.62	0\\
45.63	0\\
45.64	0\\
45.65	0\\
45.66	0\\
45.67	0\\
45.68	0\\
45.69	0\\
45.7	0\\
45.71	0\\
45.72	0\\
45.73	0\\
45.74	0\\
45.75	0\\
45.76	0\\
45.77	0\\
45.78	0\\
45.79	0\\
45.8	0\\
45.81	0\\
45.82	0\\
45.83	0\\
45.84	0\\
45.85	0\\
45.86	0\\
45.87	0\\
45.88	0\\
45.89	0\\
45.9	0\\
45.91	0\\
45.92	0\\
45.93	0\\
45.94	0\\
45.95	0\\
45.96	0\\
45.97	0\\
45.98	0\\
45.99	0\\
46	0\\
46.01	0\\
46.02	0\\
46.03	0\\
46.04	0\\
46.05	0\\
46.06	0\\
46.07	0\\
46.08	0\\
46.09	0\\
46.1	0\\
46.11	0\\
46.12	0\\
46.13	0\\
46.14	0\\
46.15	0\\
46.16	0\\
46.17	0\\
46.18	0\\
46.19	0\\
46.2	0\\
46.21	0\\
46.22	0\\
46.23	0\\
46.24	0\\
46.25	0\\
46.26	0\\
46.27	0\\
46.28	0\\
46.29	0\\
46.3	0\\
46.31	0\\
46.32	0\\
46.33	0\\
46.34	0\\
46.35	0\\
46.36	0\\
46.37	0\\
46.38	0\\
46.39	0\\
46.4	0\\
46.41	0\\
46.42	0\\
46.43	0\\
46.44	0\\
46.45	0\\
46.46	0\\
46.47	0\\
46.48	0\\
46.49	0\\
46.5	0\\
46.51	0\\
46.52	0\\
46.53	0\\
46.54	0\\
46.55	0\\
46.56	0\\
46.57	0\\
46.58	0\\
46.59	0\\
46.6	0\\
46.61	0\\
46.62	0\\
46.63	0\\
46.64	0\\
46.65	0\\
46.66	0\\
46.67	0\\
46.68	0\\
46.69	0\\
46.7	0\\
46.71	0\\
46.72	0\\
46.73	0\\
46.74	0\\
46.75	0\\
46.76	0\\
46.77	0\\
46.78	0\\
46.79	0\\
46.8	0\\
46.81	0\\
46.82	0\\
46.83	0\\
46.84	0\\
46.85	0\\
46.86	0\\
46.87	0\\
46.88	0\\
46.89	0\\
46.9	0\\
46.91	0\\
46.92	0\\
46.93	0\\
46.94	0\\
46.95	0\\
46.96	0\\
46.97	0\\
46.98	0\\
46.99	0\\
47	0\\
47.01	0\\
47.02	0\\
47.03	0\\
47.04	0\\
47.05	0\\
47.06	0\\
47.07	0\\
47.08	0\\
47.09	0\\
47.1	0\\
47.11	0\\
47.12	0\\
47.13	0\\
47.14	0\\
47.15	0\\
47.16	0\\
47.17	0\\
47.18	0\\
47.19	0\\
47.2	0\\
47.21	0\\
47.22	0\\
47.23	0\\
47.24	0\\
47.25	0\\
47.26	0\\
47.27	0\\
47.28	0\\
47.29	0\\
47.3	0\\
47.31	0\\
47.32	0\\
47.33	0\\
47.34	0\\
47.35	0\\
47.36	0\\
47.37	0\\
47.38	0\\
47.39	0\\
47.4	0\\
47.41	0\\
47.42	0\\
47.43	0\\
47.44	0\\
47.45	0\\
47.46	0\\
47.47	0\\
47.48	0\\
47.49	0\\
47.5	0\\
47.51	0\\
47.52	0\\
47.53	0\\
47.54	0\\
47.55	0\\
47.56	0\\
47.57	0\\
47.58	0\\
47.59	0\\
47.6	0\\
47.61	0\\
47.62	0\\
47.63	0\\
47.64	0\\
47.65	0\\
47.66	0\\
47.67	0\\
47.68	0\\
47.69	0\\
47.7	0\\
47.71	0\\
47.72	0\\
47.73	0\\
47.74	0\\
47.75	0\\
47.76	0\\
47.77	0\\
47.78	0\\
47.79	0\\
47.8	0\\
47.81	0\\
47.82	0\\
47.83	0\\
47.84	0\\
47.85	0\\
47.86	0\\
47.87	0\\
47.88	0\\
47.89	0\\
47.9	0\\
47.91	0\\
47.92	0\\
47.93	0\\
47.94	0\\
47.95	0\\
47.96	0\\
47.97	0\\
47.98	0\\
47.99	0\\
48	0\\
48.01	0\\
48.02	0\\
48.03	0\\
48.04	0\\
48.05	0\\
48.06	0\\
48.07	0\\
48.08	0\\
48.09	0\\
48.1	0\\
48.11	0\\
48.12	0\\
48.13	0\\
48.14	0\\
48.15	0\\
48.16	0\\
48.17	0\\
48.18	0\\
48.19	0\\
48.2	0\\
48.21	0\\
48.22	0\\
48.23	0\\
48.24	0\\
48.25	0\\
48.26	0\\
48.27	0\\
48.28	0\\
48.29	0\\
48.3	0\\
48.31	0\\
48.32	0\\
48.33	0\\
48.34	0\\
48.35	0\\
48.36	0\\
48.37	0\\
48.38	0\\
48.39	0\\
48.4	0\\
48.41	0\\
48.42	0\\
48.43	0\\
48.44	0\\
48.45	0\\
48.46	0\\
48.47	0\\
48.48	0\\
48.49	0\\
48.5	0\\
48.51	0\\
48.52	0\\
48.53	0\\
48.54	0\\
48.55	0\\
48.56	0\\
48.57	0\\
48.58	0\\
48.59	0\\
48.6	0\\
48.61	0\\
48.62	0\\
48.63	0\\
48.64	0\\
48.65	0\\
48.66	0\\
48.67	0\\
48.68	0\\
48.69	0\\
48.7	0\\
48.71	0\\
48.72	0\\
48.73	0\\
48.74	0\\
48.75	0\\
48.76	0\\
48.77	0\\
48.78	0\\
48.79	0\\
48.8	0\\
48.81	0\\
48.82	0\\
48.83	0\\
48.84	0\\
48.85	0\\
48.86	0\\
48.87	0\\
48.88	0\\
48.89	0\\
48.9	0\\
48.91	0\\
48.92	0\\
48.93	0\\
48.94	0\\
48.95	0\\
48.96	0\\
48.97	0\\
48.98	0\\
48.99	0\\
49	0\\
49.01	0\\
49.02	0\\
49.03	0\\
49.04	0\\
49.05	0\\
49.06	0\\
49.07	0\\
49.08	0\\
49.09	0\\
49.1	0\\
49.11	0\\
49.12	0\\
49.13	0\\
49.14	0\\
49.15	0\\
49.16	0\\
49.17	0\\
49.18	0\\
49.19	0\\
49.2	0\\
49.21	0\\
49.22	0\\
49.23	0\\
49.24	0\\
49.25	0\\
49.26	0\\
49.27	0\\
49.28	0\\
49.29	0\\
49.3	0\\
49.31	0\\
49.32	0\\
49.33	0\\
49.34	0\\
49.35	0\\
49.36	0\\
49.37	0\\
49.38	0\\
49.39	0\\
49.4	0\\
49.41	0\\
49.42	0\\
49.43	0\\
49.44	0\\
49.45	0\\
49.46	0\\
49.47	0\\
49.48	0\\
49.49	0\\
49.5	0\\
49.51	0\\
49.52	0\\
49.53	0\\
49.54	0\\
49.55	0\\
49.56	0\\
49.57	0\\
49.58	0\\
49.59	0\\
49.6	0\\
49.61	0\\
49.62	0\\
49.63	0\\
49.64	0\\
49.65	0\\
49.66	0\\
49.67	0\\
49.68	0\\
49.69	0\\
49.7	0\\
49.71	0\\
49.72	0\\
49.73	0\\
49.74	0\\
49.75	0\\
49.76	0\\
49.77	0\\
49.78	0\\
49.79	0\\
49.8	0\\
49.81	0\\
49.82	0\\
49.83	0\\
49.84	0\\
49.85	0\\
49.86	0\\
49.87	0\\
49.88	0\\
49.89	0\\
49.9	0\\
49.91	0\\
49.92	0\\
49.93	0\\
49.94	0\\
49.95	0\\
49.96	0\\
49.97	0\\
49.98	0\\
49.99	0\\
50	0\\
50.01	0\\
50.02	0\\
50.03	0\\
50.04	0\\
50.05	0\\
50.06	0\\
50.07	0\\
50.08	0\\
50.09	0\\
50.1	0\\
50.11	0\\
50.12	0\\
50.13	0\\
50.14	0\\
50.15	0\\
50.16	0\\
50.17	0\\
50.18	0\\
50.19	0\\
50.2	0\\
50.21	0\\
50.22	0\\
50.23	0\\
50.24	0\\
50.25	0\\
50.26	0\\
50.27	0\\
50.28	0\\
50.29	0\\
50.3	0\\
50.31	0\\
50.32	0\\
50.33	0\\
50.34	0\\
50.35	0\\
50.36	0\\
50.37	0\\
50.38	0\\
50.39	0\\
50.4	0\\
50.41	0\\
50.42	0\\
50.43	0\\
50.44	0\\
50.45	0\\
50.46	0\\
50.47	0\\
50.48	0\\
50.49	0\\
50.5	0\\
50.51	0\\
50.52	0\\
50.53	0\\
50.54	0\\
50.55	0\\
50.56	0\\
50.57	0\\
50.58	0\\
50.59	0\\
50.6	0\\
50.61	0\\
50.62	0\\
50.63	0\\
50.64	0\\
50.65	0\\
50.66	0\\
50.67	0\\
50.68	0\\
50.69	0\\
50.7	0\\
50.71	0\\
50.72	0\\
50.73	0\\
50.74	0\\
50.75	0\\
50.76	0\\
50.77	0\\
50.78	0\\
50.79	0\\
50.8	0\\
50.81	0\\
50.82	0\\
50.83	0\\
50.84	0\\
50.85	0\\
50.86	0\\
50.87	0\\
50.88	0\\
50.89	0\\
50.9	0\\
50.91	0\\
50.92	0\\
50.93	0\\
50.94	0\\
50.95	0\\
50.96	0\\
50.97	0\\
50.98	0\\
50.99	0\\
51	0\\
51.01	0\\
51.02	0\\
51.03	0\\
51.04	0\\
51.05	0\\
51.06	0\\
51.07	0\\
51.08	0\\
51.09	0\\
51.1	0\\
51.11	0\\
51.12	0\\
51.13	0\\
51.14	0\\
51.15	0\\
51.16	0\\
51.17	0\\
51.18	0\\
51.19	0\\
51.2	0\\
51.21	0\\
51.22	0\\
51.23	0\\
51.24	0\\
51.25	0\\
51.26	0\\
51.27	0\\
51.28	0\\
51.29	0\\
51.3	0\\
51.31	0\\
51.32	0\\
51.33	0\\
51.34	0\\
51.35	0\\
51.36	0\\
51.37	0\\
51.38	0\\
51.39	0\\
51.4	0\\
51.41	0\\
51.42	0\\
51.43	0\\
51.44	0\\
51.45	0\\
51.46	0\\
51.47	0\\
51.48	0\\
51.49	0\\
51.5	0\\
51.51	0\\
51.52	0\\
51.53	0\\
51.54	0\\
51.55	0\\
51.56	0\\
51.57	0\\
51.58	0\\
51.59	0\\
51.6	0\\
51.61	0\\
51.62	0\\
51.63	0\\
51.64	0\\
51.65	0\\
51.66	0\\
51.67	0\\
51.68	0\\
51.69	0\\
51.7	0\\
51.71	0\\
51.72	0\\
51.73	0\\
51.74	0\\
51.75	0\\
51.76	0\\
51.77	0\\
51.78	0\\
51.79	0\\
51.8	0\\
51.81	0\\
51.82	0\\
51.83	0\\
51.84	0\\
51.85	0\\
51.86	0\\
51.87	0\\
51.88	0\\
51.89	0\\
51.9	0\\
51.91	0\\
51.92	0\\
51.93	0\\
51.94	0\\
51.95	0\\
51.96	0\\
51.97	0\\
51.98	0\\
51.99	0\\
52	0\\
52.01	0\\
52.02	0\\
52.03	0\\
52.04	0\\
52.05	0\\
52.06	0\\
52.07	0\\
52.08	0\\
52.09	0\\
52.1	0\\
52.11	0\\
52.12	0\\
52.13	0\\
52.14	0\\
52.15	0\\
52.16	0\\
52.17	0\\
52.18	0\\
52.19	0\\
52.2	0\\
52.21	0\\
52.22	0\\
52.23	0\\
52.24	0\\
52.25	0\\
52.26	0\\
52.27	0\\
52.28	0\\
52.29	0\\
52.3	0\\
52.31	0\\
52.32	0\\
52.33	0\\
52.34	0\\
52.35	0\\
52.36	0\\
52.37	0\\
52.38	0\\
52.39	0\\
52.4	0\\
52.41	0\\
52.42	0\\
52.43	0\\
52.44	0\\
52.45	0\\
52.46	0\\
52.47	0\\
52.48	0\\
52.49	0\\
52.5	0\\
52.51	0\\
52.52	0\\
52.53	0\\
52.54	0\\
52.55	0\\
52.56	0\\
52.57	0\\
52.58	0\\
52.59	0\\
52.6	0\\
52.61	0\\
52.62	0\\
52.63	0\\
52.64	0\\
52.65	0\\
52.66	0\\
52.67	0\\
52.68	0\\
52.69	0\\
52.7	0\\
52.71	0\\
52.72	0\\
52.73	0\\
52.74	0\\
52.75	0\\
52.76	0\\
52.77	0\\
52.78	0\\
52.79	0\\
52.8	0\\
52.81	0\\
52.82	0\\
52.83	0\\
52.84	0\\
52.85	0\\
52.86	0\\
52.87	0\\
52.88	0\\
52.89	0\\
52.9	0\\
52.91	0\\
52.92	0\\
52.93	0\\
52.94	0\\
52.95	0\\
52.96	0\\
52.97	0\\
52.98	0\\
52.99	0\\
53	0\\
53.01	0\\
53.02	0\\
53.03	0\\
53.04	0\\
53.05	0\\
53.06	0\\
53.07	0\\
53.08	0\\
53.09	0\\
53.1	0\\
53.11	0\\
53.12	0\\
53.13	0\\
53.14	0\\
53.15	0\\
53.16	0\\
53.17	0\\
53.18	0\\
53.19	0\\
53.2	0\\
53.21	0\\
53.22	0\\
53.23	0\\
53.24	0\\
53.25	0\\
53.26	0\\
53.27	0\\
53.28	0\\
53.29	0\\
53.3	0\\
53.31	0\\
53.32	0\\
53.33	0\\
53.34	0\\
53.35	0\\
53.36	0\\
53.37	0\\
53.38	0\\
53.39	0\\
53.4	0\\
53.41	0\\
53.42	0\\
53.43	0\\
53.44	0\\
53.45	0\\
53.46	0\\
53.47	0\\
53.48	0\\
53.49	0\\
53.5	0\\
53.51	0\\
53.52	0\\
53.53	0\\
53.54	0\\
53.55	0\\
53.56	0\\
53.57	0\\
53.58	0\\
53.59	0\\
53.6	0\\
53.61	0\\
53.62	0\\
53.63	0\\
53.64	0\\
53.65	0\\
53.66	0\\
53.67	0\\
53.68	0\\
53.69	0\\
53.7	0\\
53.71	0\\
53.72	0\\
53.73	0\\
53.74	0\\
53.75	0\\
53.76	0\\
53.77	0\\
53.78	0\\
53.79	0\\
53.8	0\\
53.81	0\\
53.82	0\\
53.83	0\\
53.84	0\\
53.85	0\\
53.86	0\\
53.87	0\\
53.88	0\\
53.89	0\\
53.9	0\\
53.91	0\\
53.92	0\\
53.93	0\\
53.94	0\\
53.95	0\\
53.96	0\\
53.97	0\\
53.98	0\\
53.99	0\\
54	0\\
54.01	0\\
54.02	0\\
54.03	0\\
54.04	0\\
54.05	0\\
54.06	0\\
54.07	0\\
54.08	0\\
54.09	0\\
54.1	0\\
54.11	0\\
54.12	0\\
54.13	0\\
54.14	0\\
54.15	0\\
54.16	0\\
54.17	0\\
54.18	0\\
54.19	0\\
54.2	0\\
54.21	0\\
54.22	0\\
54.23	0\\
54.24	0\\
54.25	0\\
54.26	0\\
54.27	0\\
54.28	0\\
54.29	0\\
54.3	0\\
54.31	0\\
54.32	0\\
54.33	0\\
54.34	0\\
54.35	0\\
54.36	0\\
54.37	0\\
54.38	0\\
54.39	0\\
54.4	0\\
54.41	0\\
54.42	0\\
54.43	0\\
54.44	0\\
54.45	0\\
54.46	0\\
54.47	0\\
54.48	0\\
54.49	0\\
54.5	0\\
54.51	0\\
54.52	0\\
54.53	0\\
54.54	0\\
54.55	0\\
54.56	0\\
54.57	0\\
54.58	0\\
54.59	0\\
54.6	0\\
54.61	0\\
54.62	0\\
54.63	0\\
54.64	0\\
54.65	0\\
54.66	0\\
54.67	0\\
54.68	0\\
54.69	0\\
54.7	0\\
54.71	0\\
54.72	0\\
54.73	0\\
54.74	0\\
54.75	0\\
54.76	0\\
54.77	0\\
54.78	0\\
54.79	0\\
54.8	0\\
54.81	0\\
54.82	0\\
54.83	0\\
54.84	0\\
54.85	0\\
54.86	0\\
54.87	0\\
54.88	0\\
54.89	0\\
54.9	0\\
54.91	0\\
54.92	0\\
54.93	0\\
54.94	0\\
54.95	0\\
54.96	0\\
54.97	0\\
54.98	0\\
54.99	0\\
55	0\\
55.01	0\\
55.02	0\\
55.03	0\\
55.04	0\\
55.05	0\\
55.06	0\\
55.07	0\\
55.08	0\\
55.09	0\\
55.1	0\\
55.11	0\\
55.12	0\\
55.13	0\\
55.14	0\\
55.15	0\\
55.16	0\\
55.17	0\\
55.18	0\\
55.19	0\\
55.2	0\\
55.21	0\\
55.22	0\\
55.23	0\\
55.24	0\\
55.25	0\\
55.26	0\\
55.27	0\\
55.28	0\\
55.29	0\\
55.3	0\\
55.31	0\\
55.32	0\\
55.33	0\\
55.34	0\\
55.35	0\\
55.36	0\\
55.37	0\\
55.38	0\\
55.39	0\\
55.4	0\\
55.41	0\\
55.42	0\\
55.43	0\\
55.44	0\\
55.45	0\\
55.46	0\\
55.47	0\\
55.48	0\\
55.49	0\\
55.5	0\\
55.51	0\\
55.52	0\\
55.53	0\\
55.54	0\\
55.55	0\\
55.56	0\\
55.57	0\\
55.58	0\\
55.59	0\\
55.6	0\\
55.61	0\\
55.62	0\\
55.63	0\\
55.64	0\\
55.65	0\\
55.66	0\\
55.67	0\\
55.68	0\\
55.69	0\\
55.7	0\\
55.71	0\\
55.72	0\\
55.73	0\\
55.74	0\\
55.75	0\\
55.76	0\\
55.77	0\\
55.78	0\\
55.79	0\\
55.8	0\\
55.81	0\\
55.82	0\\
55.83	0\\
55.84	0\\
55.85	0\\
55.86	0\\
55.87	0\\
55.88	0\\
55.89	0\\
55.9	0\\
55.91	0\\
55.92	0\\
55.93	0\\
55.94	0\\
55.95	0\\
55.96	0\\
55.97	0\\
55.98	0\\
55.99	0\\
56	0\\
56.01	0\\
56.02	0\\
56.03	0\\
56.04	0\\
56.05	0\\
56.06	0\\
56.07	0\\
56.08	0\\
56.09	0\\
56.1	0\\
56.11	0\\
56.12	0\\
56.13	0\\
56.14	0\\
56.15	0\\
56.16	0\\
56.17	0\\
56.18	0\\
56.19	0\\
56.2	0\\
56.21	0\\
56.22	0\\
56.23	0\\
56.24	0\\
56.25	0\\
56.26	0\\
56.27	0\\
56.28	0\\
56.29	0\\
56.3	0\\
56.31	0\\
56.32	0\\
56.33	0\\
56.34	0\\
56.35	0\\
56.36	0\\
56.37	0\\
56.38	0\\
56.39	0\\
56.4	0\\
56.41	0\\
56.42	0\\
56.43	0\\
56.44	0\\
56.45	0\\
56.46	0\\
56.47	0\\
56.48	0\\
56.49	0\\
56.5	0\\
56.51	0\\
56.52	0\\
56.53	0\\
56.54	0\\
56.55	0\\
56.56	0\\
56.57	0\\
56.58	0\\
56.59	0\\
56.6	0\\
56.61	0\\
56.62	0\\
56.63	0\\
56.64	0\\
56.65	0\\
56.66	0\\
56.67	0\\
56.68	0\\
56.69	0\\
56.7	0\\
56.71	0\\
56.72	0\\
56.73	0\\
56.74	0\\
56.75	0\\
56.76	0\\
56.77	0\\
56.78	0\\
56.79	0\\
56.8	0\\
56.81	0\\
56.82	0\\
56.83	0\\
56.84	0\\
56.85	0\\
56.86	0\\
56.87	0\\
56.88	0\\
56.89	0\\
56.9	0\\
56.91	0\\
56.92	0\\
56.93	0\\
56.94	0\\
56.95	0\\
56.96	0\\
56.97	0\\
56.98	0\\
56.99	0\\
57	0\\
57.01	0\\
57.02	0\\
57.03	0\\
57.04	0\\
57.05	0\\
57.06	0\\
57.07	0\\
57.08	0\\
57.09	0\\
57.1	0\\
57.11	0\\
57.12	0\\
57.13	0\\
57.14	0\\
57.15	0\\
57.16	0\\
57.17	0\\
57.18	0\\
57.19	0\\
57.2	0\\
57.21	0\\
57.22	0\\
57.23	0\\
57.24	0\\
57.25	0\\
57.26	0\\
57.27	0\\
57.28	0\\
57.29	0\\
57.3	0\\
57.31	0\\
57.32	0\\
57.33	0\\
57.34	0\\
57.35	0\\
57.36	0\\
57.37	0\\
57.38	0\\
57.39	0\\
57.4	0\\
57.41	0\\
57.42	0\\
57.43	0\\
57.44	0\\
57.45	0\\
57.46	0\\
57.47	0\\
57.48	0\\
57.49	0\\
57.5	0\\
57.51	0\\
57.52	0\\
57.53	0\\
57.54	0\\
57.55	0\\
57.56	0\\
57.57	0\\
57.58	0\\
57.59	0\\
57.6	0\\
57.61	0\\
57.62	0\\
57.63	0\\
57.64	0\\
57.65	0\\
57.66	0\\
57.67	0\\
57.68	0\\
57.69	0\\
57.7	0\\
57.71	0\\
57.72	0\\
57.73	0\\
57.74	0\\
57.75	0\\
57.76	0\\
57.77	0\\
57.78	0\\
57.79	0\\
57.8	0\\
57.81	0\\
57.82	0\\
57.83	0\\
57.84	0\\
57.85	0\\
57.86	0\\
57.87	0\\
57.88	0\\
57.89	0\\
57.9	0\\
57.91	0\\
57.92	0\\
57.93	0\\
57.94	0\\
57.95	0\\
57.96	0\\
57.97	0\\
57.98	0\\
57.99	0\\
58	0\\
58.01	0\\
58.02	0\\
58.03	0\\
58.04	0\\
58.05	0\\
58.06	0\\
58.07	0\\
58.08	0\\
58.09	0\\
58.1	0\\
58.11	0\\
58.12	0\\
58.13	0\\
58.14	0\\
58.15	0\\
58.16	0\\
58.17	0\\
58.18	0\\
58.19	0\\
58.2	0\\
58.21	0\\
58.22	0\\
58.23	0\\
58.24	0\\
58.25	0\\
58.26	0\\
58.27	0\\
58.28	0\\
58.29	0\\
58.3	0\\
58.31	0\\
58.32	0\\
58.33	0\\
58.34	0\\
58.35	0\\
58.36	0\\
58.37	0\\
58.38	0\\
58.39	0\\
58.4	0\\
58.41	0\\
58.42	0\\
58.43	0\\
58.44	0\\
58.45	0\\
58.46	0\\
58.47	0\\
58.48	0\\
58.49	0\\
58.5	0\\
58.51	0\\
58.52	0\\
58.53	0\\
58.54	0\\
58.55	0\\
58.56	0\\
58.57	0\\
58.58	0\\
58.59	0\\
58.6	0\\
58.61	0\\
58.62	0\\
58.63	0\\
58.64	0\\
58.65	0\\
58.66	0\\
58.67	0\\
58.68	0\\
58.69	0\\
58.7	0\\
58.71	0\\
58.72	0\\
58.73	0\\
58.74	0\\
58.75	0\\
58.76	0\\
58.77	0\\
58.78	0\\
58.79	0\\
58.8	0\\
58.81	0\\
58.82	0\\
58.83	0\\
58.84	0\\
58.85	0\\
58.86	0\\
58.87	0\\
58.88	0\\
58.89	0\\
58.9	0\\
58.91	0\\
58.92	0\\
58.93	0\\
58.94	0\\
58.95	0\\
58.96	0\\
58.97	0\\
58.98	0\\
58.99	0\\
59	0\\
59.01	0\\
59.02	0\\
59.03	0\\
59.04	0\\
59.05	0\\
59.06	0\\
59.07	0\\
59.08	0\\
59.09	0\\
59.1	0\\
59.11	0\\
59.12	0\\
59.13	0\\
59.14	0\\
59.15	0\\
59.16	0\\
59.17	0\\
59.18	0\\
59.19	0\\
59.2	0\\
59.21	0\\
59.22	0\\
59.23	0\\
59.24	0\\
59.25	0\\
59.26	0\\
59.27	0\\
59.28	0\\
59.29	0\\
59.3	0\\
59.31	0\\
59.32	0\\
59.33	0\\
59.34	0\\
59.35	0\\
59.36	0\\
59.37	0\\
59.38	0\\
59.39	0\\
59.4	0\\
59.41	0\\
59.42	0\\
59.43	0\\
59.44	0\\
59.45	0\\
59.46	0\\
59.47	0\\
59.48	0\\
59.49	0\\
59.5	0\\
59.51	0\\
59.52	0\\
59.53	0\\
59.54	0\\
59.55	0\\
59.56	0\\
59.57	0\\
59.58	0\\
59.59	0\\
59.6	0\\
59.61	0\\
59.62	0\\
59.63	0\\
59.64	0\\
59.65	0\\
59.66	0\\
59.67	0\\
59.68	0\\
59.69	0\\
59.7	0\\
59.71	0\\
59.72	0\\
59.73	0\\
59.74	0\\
59.75	0\\
59.76	0\\
59.77	0\\
59.78	0\\
59.79	0\\
59.8	0\\
59.81	0\\
59.82	0\\
59.83	0\\
59.84	0\\
59.85	0\\
59.86	0\\
59.87	0\\
59.88	0\\
59.89	0\\
59.9	0\\
59.91	0\\
59.92	0\\
59.93	0\\
59.94	0\\
59.95	0\\
59.96	0\\
59.97	0\\
59.98	0\\
59.99	0\\
60	0\\
60.01	0\\
60.02	0\\
60.03	0\\
60.04	0\\
60.05	0\\
60.06	0\\
60.07	0\\
60.08	0\\
60.09	0\\
60.1	0\\
60.11	0\\
60.12	0\\
60.13	0\\
60.14	0\\
60.15	0\\
60.16	0\\
60.17	0\\
60.18	0\\
60.19	0\\
60.2	0\\
60.21	0\\
60.22	0\\
60.23	0\\
60.24	0\\
60.25	0\\
60.26	0\\
60.27	0\\
60.28	0\\
60.29	0\\
60.3	0\\
60.31	0\\
60.32	0\\
60.33	0\\
60.34	0\\
60.35	0\\
60.36	0\\
60.37	0\\
60.38	0\\
60.39	0\\
60.4	0\\
60.41	0\\
60.42	0\\
60.43	0\\
60.44	0\\
60.45	0\\
60.46	0\\
60.47	0\\
60.48	0\\
60.49	0\\
60.5	0\\
60.51	0\\
60.52	0\\
60.53	0\\
60.54	0\\
60.55	0\\
60.56	0\\
60.57	0\\
60.58	0\\
60.59	0\\
60.6	0\\
60.61	0\\
60.62	0\\
60.63	0\\
60.64	0\\
60.65	0\\
60.66	0\\
60.67	0\\
60.68	0\\
60.69	0\\
60.7	0\\
60.71	0\\
60.72	0\\
60.73	0\\
60.74	0\\
60.75	0\\
60.76	0\\
60.77	0\\
60.78	0\\
60.79	0\\
60.8	0\\
60.81	0\\
60.82	0\\
60.83	0\\
60.84	0\\
60.85	0\\
60.86	0\\
60.87	0\\
60.88	0\\
60.89	0\\
60.9	0\\
60.91	0\\
60.92	0\\
60.93	0\\
60.94	0\\
60.95	0\\
60.96	0\\
60.97	0\\
60.98	0\\
60.99	0\\
61	0\\
61.01	0\\
61.02	0\\
61.03	0\\
61.04	0\\
61.05	0\\
61.06	0\\
61.07	0\\
61.08	0\\
61.09	0\\
61.1	0\\
61.11	0\\
61.12	0\\
61.13	0\\
61.14	0\\
61.15	0\\
61.16	0\\
61.17	0\\
61.18	0\\
61.19	0\\
61.2	0\\
61.21	0\\
61.22	0\\
61.23	0\\
61.24	0\\
61.25	0\\
61.26	0\\
61.27	0\\
61.28	0\\
61.29	0\\
61.3	0\\
61.31	0\\
61.32	0\\
61.33	0\\
61.34	0\\
61.35	0\\
61.36	0\\
61.37	0\\
61.38	0\\
61.39	0\\
61.4	0\\
61.41	0\\
61.42	0\\
61.43	0\\
61.44	0\\
61.45	0\\
61.46	0\\
61.47	0\\
61.48	0\\
61.49	0\\
61.5	0\\
61.51	0\\
61.52	0\\
61.53	0\\
61.54	0\\
61.55	0\\
61.56	0\\
61.57	0\\
61.58	0\\
61.59	0\\
61.6	0\\
61.61	0\\
61.62	0\\
61.63	0\\
61.64	0\\
61.65	0\\
61.66	0\\
61.67	0\\
61.68	0\\
61.69	0\\
61.7	0\\
61.71	0\\
61.72	0\\
61.73	0\\
61.74	0\\
61.75	0\\
61.76	0\\
61.77	0\\
61.78	0\\
61.79	0\\
61.8	0\\
61.81	0\\
61.82	0\\
61.83	0\\
61.84	0\\
61.85	0\\
61.86	0\\
61.87	0\\
61.88	0\\
61.89	0\\
61.9	0\\
61.91	0\\
61.92	0\\
61.93	0\\
61.94	0\\
61.95	0\\
61.96	0\\
61.97	0\\
61.98	0\\
61.99	0\\
62	0\\
62.01	0\\
62.02	0\\
62.03	0\\
62.04	0\\
62.05	0\\
62.06	0\\
62.07	0\\
62.08	0\\
62.09	0\\
62.1	0\\
62.11	0\\
62.12	0\\
62.13	0\\
62.14	0\\
62.15	0\\
62.16	0\\
62.17	0\\
62.18	0\\
62.19	0\\
62.2	0\\
62.21	0\\
62.22	0\\
62.23	0\\
62.24	0\\
62.25	0\\
62.26	0\\
62.27	0\\
62.28	0\\
62.29	0\\
62.3	0\\
62.31	0\\
62.32	0\\
62.33	0\\
62.34	0\\
62.35	0\\
62.36	0\\
62.37	0\\
62.38	0\\
62.39	0\\
62.4	0\\
62.41	0\\
62.42	0\\
62.43	0\\
62.44	0\\
62.45	0\\
62.46	0\\
62.47	0\\
62.48	0\\
62.49	0\\
62.5	0\\
62.51	0\\
62.52	0\\
62.53	0\\
62.54	0\\
62.55	0\\
62.56	0\\
62.57	0\\
62.58	0\\
62.59	0\\
62.6	0\\
62.61	0\\
62.62	0\\
62.63	0\\
62.64	0\\
62.65	0\\
62.66	0\\
62.67	0\\
62.68	0\\
62.69	0\\
62.7	0\\
62.71	0\\
62.72	0\\
62.73	0\\
62.74	0\\
62.75	0\\
62.76	0\\
62.77	0\\
62.78	0\\
62.79	0\\
62.8	0\\
62.81	0\\
62.82	0\\
62.83	0\\
62.84	0\\
62.85	0\\
62.86	0\\
62.87	0\\
62.88	0\\
62.89	0\\
62.9	0\\
62.91	0\\
62.92	0\\
62.93	0\\
62.94	0\\
62.95	0\\
62.96	0\\
62.97	0\\
62.98	0\\
62.99	0\\
63	0\\
63.01	0\\
63.02	0\\
63.03	0\\
63.04	0\\
63.05	0\\
63.06	0\\
63.07	0\\
63.08	0\\
63.09	0\\
63.1	0\\
63.11	0\\
63.12	0\\
63.13	0\\
63.14	0\\
63.15	0\\
63.16	0\\
63.17	0\\
63.18	0\\
63.19	0\\
63.2	0\\
63.21	0\\
63.22	0\\
63.23	0\\
63.24	0\\
63.25	0\\
63.26	0\\
63.27	0\\
63.28	0\\
63.29	0\\
63.3	0\\
63.31	0\\
63.32	0\\
63.33	0\\
63.34	0\\
63.35	0\\
63.36	0\\
63.37	0\\
63.38	0\\
63.39	0\\
63.4	0\\
63.41	0\\
63.42	0\\
63.43	0\\
63.44	0\\
63.45	0\\
63.46	0\\
63.47	0\\
63.48	0\\
63.49	0\\
63.5	0\\
63.51	0\\
63.52	0\\
63.53	0\\
63.54	0\\
63.55	0\\
63.56	0\\
63.57	0\\
63.58	0\\
63.59	0\\
63.6	0\\
63.61	0\\
63.62	0\\
63.63	0\\
63.64	0\\
63.65	0\\
63.66	0\\
63.67	0\\
63.68	0\\
63.69	0\\
63.7	0\\
63.71	0\\
63.72	0\\
63.73	0\\
63.74	0\\
63.75	0\\
63.76	0\\
63.77	0\\
63.78	0\\
63.79	0\\
63.8	0\\
63.81	0\\
63.82	0\\
63.83	0\\
63.84	0\\
63.85	0\\
63.86	0\\
63.87	0\\
63.88	0\\
63.89	0\\
63.9	0\\
63.91	0\\
63.92	0\\
63.93	0\\
63.94	0\\
63.95	0\\
63.96	0\\
63.97	0\\
63.98	0\\
63.99	0\\
64	0\\
64.01	0\\
64.02	0\\
64.03	0\\
64.04	0\\
64.05	0\\
64.06	0\\
64.07	0\\
64.08	0\\
64.09	0\\
64.1	0\\
64.11	0\\
64.12	0\\
64.13	0\\
64.14	0\\
64.15	0\\
64.16	0\\
64.17	0\\
64.18	0\\
64.19	0\\
64.2	0\\
64.21	0\\
64.22	0\\
64.23	0\\
64.24	0\\
64.25	0\\
64.26	0\\
64.27	0\\
64.28	0\\
64.29	0\\
64.3	0\\
64.31	0\\
64.32	0\\
64.33	0\\
64.34	0\\
64.35	0\\
64.36	0\\
64.37	0\\
64.38	0\\
64.39	0\\
64.4	0\\
64.41	0\\
64.42	0\\
64.43	0\\
64.44	0\\
64.45	0\\
64.46	0\\
64.47	0\\
64.48	0\\
64.49	0\\
64.5	0\\
64.51	0\\
64.52	0\\
64.53	0\\
64.54	0\\
64.55	0\\
64.56	0\\
64.57	0\\
64.58	0\\
64.59	0\\
64.6	0\\
64.61	0\\
64.62	0\\
64.63	0\\
64.64	0\\
64.65	0\\
64.66	0\\
64.67	0\\
64.68	0\\
64.69	0\\
64.7	0\\
64.71	0\\
64.72	0\\
64.73	0\\
64.74	0\\
64.75	0\\
64.76	0\\
64.77	0\\
64.78	0\\
64.79	0\\
64.8	0\\
64.81	0\\
64.82	0\\
64.83	0\\
64.84	0\\
64.85	0\\
64.86	0\\
64.87	0\\
64.88	0\\
64.89	0\\
64.9	0\\
64.91	0\\
64.92	0\\
64.93	0\\
64.94	0\\
64.95	0\\
64.96	0\\
64.97	0\\
64.98	0\\
64.99	0\\
65	0\\
65.01	0\\
65.02	0\\
65.03	0\\
65.04	0\\
65.05	0\\
65.06	0\\
65.07	0\\
65.08	0\\
65.09	0\\
65.1	0\\
65.11	0\\
65.12	0\\
65.13	0\\
65.14	0\\
65.15	0\\
65.16	0\\
65.17	0\\
65.18	0\\
65.19	0\\
65.2	0\\
65.21	0\\
65.22	0\\
65.23	0\\
65.24	0\\
65.25	0\\
65.26	0\\
65.27	0\\
65.28	0\\
65.29	0\\
65.3	0\\
65.31	0\\
65.32	0\\
65.33	0\\
65.34	0\\
65.35	0\\
65.36	0\\
65.37	0\\
65.38	0\\
65.39	0\\
65.4	0\\
65.41	0\\
65.42	0\\
65.43	0\\
65.44	0\\
65.45	0\\
65.46	0\\
65.47	0\\
65.48	0\\
65.49	0\\
65.5	0\\
65.51	0\\
65.52	0\\
65.53	0\\
65.54	0\\
65.55	0\\
65.56	0\\
65.57	0\\
65.58	0\\
65.59	0\\
65.6	0\\
65.61	0\\
65.62	0\\
65.63	0\\
65.64	0\\
65.65	0\\
65.66	0\\
65.67	0\\
65.68	0\\
65.69	0\\
65.7	0\\
65.71	0\\
65.72	0\\
65.73	0\\
65.74	0\\
65.75	0\\
65.76	0\\
65.77	0\\
65.78	0\\
65.79	0\\
65.8	0\\
65.81	0\\
65.82	0\\
65.83	0\\
65.84	0\\
65.85	0\\
65.86	0\\
65.87	0\\
65.88	0\\
65.89	0\\
65.9	0\\
65.91	0\\
65.92	0\\
65.93	0\\
65.94	0\\
65.95	0\\
65.96	0\\
65.97	0\\
65.98	0\\
65.99	0\\
66	0\\
66.01	0\\
66.02	0\\
66.03	0\\
66.04	0\\
66.05	0\\
66.06	0\\
66.07	0\\
66.08	0\\
66.09	0\\
66.1	0\\
66.11	0\\
66.12	0\\
66.13	0\\
66.14	0\\
66.15	0\\
66.16	0\\
66.17	0\\
66.18	0\\
66.19	0\\
66.2	0\\
66.21	0\\
66.22	0\\
66.23	0\\
66.24	0\\
66.25	0\\
66.26	0\\
66.27	0\\
66.28	0\\
66.29	0\\
66.3	0\\
66.31	0\\
66.32	0\\
66.33	0\\
66.34	0\\
66.35	0\\
66.36	0\\
66.37	0\\
66.38	0\\
66.39	0\\
66.4	0\\
66.41	0\\
66.42	0\\
66.43	0\\
66.44	0\\
66.45	0\\
66.46	0\\
66.47	0\\
66.48	0\\
66.49	0\\
66.5	0\\
66.51	0\\
66.52	0\\
66.53	0\\
66.54	0\\
66.55	0\\
66.56	0\\
66.57	0\\
66.58	0\\
66.59	0\\
66.6	0\\
66.61	0\\
66.62	0\\
66.63	0\\
66.64	0\\
66.65	0\\
66.66	0\\
66.67	0\\
66.68	0\\
66.69	0\\
66.7	0\\
66.71	0\\
66.72	0\\
66.73	0\\
66.74	0\\
66.75	0\\
66.76	0\\
66.77	0\\
66.78	0\\
66.79	0\\
66.8	0\\
66.81	0\\
66.82	0\\
66.83	0\\
66.84	0\\
66.85	0\\
66.86	0\\
66.87	0\\
66.88	0\\
66.89	0\\
66.9	0\\
66.91	0\\
66.92	0\\
66.93	0\\
66.94	0\\
66.95	0\\
66.96	0\\
66.97	0\\
66.98	0\\
66.99	0\\
67	0\\
67.01	0\\
67.02	0\\
67.03	0\\
67.04	0\\
67.05	0\\
67.06	0\\
67.07	0\\
67.08	0\\
67.09	0\\
67.1	0\\
67.11	0\\
67.12	0\\
67.13	0\\
67.14	0\\
67.15	0\\
67.16	0\\
67.17	0\\
67.18	0\\
67.19	0\\
67.2	0\\
67.21	0\\
67.22	0\\
67.23	0\\
67.24	0\\
67.25	0\\
67.26	0\\
67.27	0\\
67.28	0\\
67.29	0\\
67.3	0\\
67.31	0\\
67.32	0\\
67.33	0\\
67.34	0\\
67.35	0\\
67.36	0\\
67.37	0\\
67.38	0\\
67.39	0\\
67.4	0\\
67.41	0\\
67.42	0\\
67.43	0\\
67.44	0\\
67.45	0\\
67.46	0\\
67.47	0\\
67.48	0\\
67.49	0\\
67.5	0\\
67.51	0\\
67.52	0\\
67.53	0\\
67.54	0\\
67.55	0\\
67.56	0\\
67.57	0\\
67.58	0\\
67.59	0\\
67.6	0\\
67.61	0\\
67.62	0\\
67.63	0\\
67.64	0\\
67.65	0\\
67.66	0\\
67.67	0\\
67.68	0\\
67.69	0\\
67.7	0\\
67.71	0\\
67.72	0\\
67.73	0\\
67.74	0\\
67.75	0\\
67.76	0\\
67.77	0\\
67.78	0\\
67.79	0\\
67.8	0\\
67.81	0\\
67.82	0\\
67.83	0\\
67.84	0\\
67.85	0\\
67.86	0\\
67.87	0\\
67.88	0\\
67.89	0\\
67.9	0\\
67.91	0\\
67.92	0\\
67.93	0\\
67.94	0\\
67.95	0\\
67.96	0\\
67.97	0\\
67.98	0\\
67.99	0\\
68	0\\
68.01	0\\
68.02	0\\
68.03	0\\
68.04	0\\
68.05	0\\
68.06	0\\
68.07	0\\
68.08	0\\
68.09	0\\
68.1	0\\
68.11	0\\
68.12	0\\
68.13	0\\
68.14	0\\
68.15	0\\
68.16	0\\
68.17	0\\
68.18	0\\
68.19	0\\
68.2	0\\
68.21	0\\
68.22	0\\
68.23	0\\
68.24	0\\
68.25	0\\
68.26	0\\
68.27	0\\
68.28	0\\
68.29	0\\
68.3	0\\
68.31	0\\
68.32	0\\
68.33	0\\
68.34	0\\
68.35	0\\
68.36	0\\
68.37	0\\
68.38	0\\
68.39	0\\
68.4	0\\
68.41	0\\
68.42	0\\
68.43	0\\
68.44	0\\
68.45	0\\
68.46	0\\
68.47	0\\
68.48	0\\
68.49	0\\
68.5	0\\
68.51	0\\
68.52	0\\
68.53	0\\
68.54	0\\
68.55	0\\
68.56	0\\
68.57	0\\
68.58	0\\
68.59	0\\
68.6	0\\
68.61	0\\
68.62	0\\
68.63	0\\
68.64	0\\
68.65	0\\
68.66	0\\
68.67	0\\
68.68	0\\
68.69	0\\
68.7	0\\
68.71	0\\
68.72	0\\
68.73	0\\
68.74	0\\
68.75	0\\
68.76	0\\
68.77	0\\
68.78	0\\
68.79	0\\
68.8	0\\
68.81	0\\
68.82	0\\
68.83	0\\
68.84	0\\
68.85	0\\
68.86	0\\
68.87	0\\
68.88	0\\
68.89	0\\
68.9	0\\
68.91	0\\
68.92	0\\
68.93	0\\
68.94	0\\
68.95	0\\
68.96	0\\
68.97	0\\
68.98	0\\
68.99	0\\
69	0\\
69.01	0\\
69.02	0\\
69.03	0\\
69.04	0\\
69.05	0\\
69.06	0\\
69.07	0\\
69.08	0\\
69.09	0\\
69.1	0\\
69.11	0\\
69.12	0\\
69.13	0\\
69.14	0\\
69.15	0\\
69.16	0\\
69.17	0\\
69.18	0\\
69.19	0\\
69.2	0\\
69.21	0\\
69.22	0\\
69.23	0\\
69.24	0\\
69.25	0\\
69.26	0\\
69.27	0\\
69.28	0\\
69.29	0\\
69.3	0\\
69.31	0\\
69.32	0\\
69.33	0\\
69.34	0\\
69.35	0\\
69.36	0\\
69.37	0\\
69.38	0\\
69.39	0\\
69.4	0\\
69.41	0\\
69.42	0\\
69.43	0\\
69.44	0\\
69.45	0\\
69.46	0\\
69.47	0\\
69.48	0\\
69.49	0\\
69.5	0\\
69.51	0\\
69.52	0\\
69.53	0\\
69.54	0\\
69.55	0\\
69.56	0\\
69.57	0\\
69.58	0\\
69.59	0\\
69.6	0\\
69.61	0\\
69.62	0\\
69.63	0\\
69.64	0\\
69.65	0\\
69.66	0\\
69.67	0\\
69.68	0\\
69.69	0\\
69.7	0\\
69.71	0\\
69.72	0\\
69.73	0\\
69.74	0\\
69.75	0\\
69.76	0\\
69.77	0\\
69.78	0\\
69.79	0\\
69.8	0\\
69.81	0\\
69.82	0\\
69.83	0\\
69.84	0\\
69.85	0\\
69.86	0\\
69.87	0\\
69.88	0\\
69.89	0\\
69.9	0\\
69.91	0\\
69.92	0\\
69.93	0\\
69.94	0\\
69.95	0\\
69.96	0\\
69.97	0\\
69.98	0\\
69.99	0\\
70	0\\
70.01	0\\
70.02	0\\
70.03	0\\
70.04	0\\
70.05	0\\
70.06	0\\
70.07	0\\
70.08	0\\
70.09	0\\
70.1	0\\
70.11	0\\
70.12	0\\
70.13	0\\
70.14	0\\
70.15	0\\
70.16	0\\
70.17	0\\
70.18	0\\
70.19	0\\
70.2	0\\
70.21	0\\
70.22	0\\
70.23	0\\
70.24	0\\
70.25	0\\
70.26	0\\
70.27	0\\
70.28	0\\
70.29	0\\
70.3	0\\
70.31	0\\
70.32	0\\
70.33	0\\
70.34	0\\
70.35	0\\
70.36	0\\
70.37	0\\
70.38	0\\
70.39	0\\
70.4	0\\
70.41	0\\
70.42	0\\
70.43	0\\
70.44	0\\
70.45	0\\
70.46	0\\
70.47	0\\
70.48	0\\
70.49	0\\
70.5	0\\
70.51	0\\
70.52	0\\
70.53	0\\
70.54	0\\
70.55	0\\
70.56	0\\
70.57	0\\
70.58	0\\
70.59	0\\
70.6	0\\
70.61	0\\
70.62	0\\
70.63	0\\
70.64	0\\
70.65	0\\
70.66	0\\
70.67	0\\
70.68	0\\
70.69	0\\
70.7	0\\
70.71	0\\
70.72	0\\
70.73	0\\
70.74	0\\
70.75	0\\
70.76	0\\
70.77	0\\
70.78	0\\
70.79	0\\
70.8	0\\
70.81	0\\
70.82	0\\
70.83	0\\
70.84	0\\
70.85	0\\
70.86	0\\
70.87	0\\
70.88	0\\
70.89	0\\
70.9	0\\
70.91	0\\
70.92	0\\
70.93	0\\
70.94	0\\
70.95	0\\
70.96	0\\
70.97	0\\
70.98	0\\
70.99	0\\
71	0\\
71.01	0\\
71.02	0\\
71.03	0\\
71.04	0\\
71.05	0\\
71.06	0\\
71.07	0\\
71.08	0\\
71.09	0\\
71.1	0\\
71.11	0\\
71.12	0\\
71.13	0\\
71.14	0\\
71.15	0\\
71.16	0\\
71.17	0\\
71.18	0\\
71.19	0\\
71.2	0\\
71.21	0\\
71.22	0\\
71.23	0\\
71.24	0\\
71.25	0\\
71.26	0\\
71.27	0\\
71.28	0\\
71.29	0\\
71.3	0\\
71.31	0\\
71.32	0\\
71.33	0\\
71.34	0\\
71.35	0\\
71.36	0\\
71.37	0\\
71.38	0\\
71.39	0\\
71.4	0\\
71.41	0\\
71.42	0\\
71.43	0\\
71.44	0\\
71.45	0\\
71.46	0\\
71.47	0\\
71.48	0\\
71.49	0\\
71.5	0\\
71.51	0\\
71.52	0\\
71.53	0\\
71.54	0\\
71.55	0\\
71.56	0\\
71.57	0\\
71.58	0\\
71.59	0\\
71.6	0\\
71.61	0\\
71.62	0\\
71.63	0\\
71.64	0\\
71.65	0\\
71.66	0\\
71.67	0\\
71.68	0\\
71.69	0\\
71.7	0\\
71.71	0\\
71.72	0\\
71.73	0\\
71.74	0\\
71.75	0\\
71.76	0\\
71.77	0\\
71.78	0\\
71.79	0\\
71.8	0\\
71.81	0\\
71.82	0\\
71.83	0\\
71.84	0\\
71.85	0\\
71.86	0\\
71.87	0\\
71.88	0\\
71.89	0\\
71.9	0\\
71.91	0\\
71.92	0\\
71.93	0\\
71.94	0\\
71.95	0\\
71.96	0\\
71.97	0\\
71.98	0\\
71.99	0\\
72	0\\
72.01	0\\
72.02	0\\
72.03	0\\
72.04	0\\
72.05	0\\
72.06	0\\
72.07	0\\
72.08	0\\
72.09	0\\
72.1	0\\
72.11	0\\
72.12	0\\
72.13	0\\
72.14	0\\
72.15	0\\
72.16	0\\
72.17	0\\
72.18	0\\
72.19	0\\
72.2	0\\
72.21	0\\
72.22	0\\
72.23	0\\
72.24	0\\
72.25	0\\
72.26	0\\
72.27	0\\
72.28	0\\
72.29	0\\
72.3	0\\
72.31	0\\
72.32	0\\
72.33	0\\
72.34	0\\
72.35	0\\
72.36	0\\
72.37	0\\
72.38	0\\
72.39	0\\
72.4	0\\
72.41	0\\
72.42	0\\
72.43	0\\
72.44	0\\
72.45	0\\
72.46	0\\
72.47	0\\
72.48	0\\
72.49	0\\
72.5	0\\
72.51	0\\
72.52	0\\
72.53	0\\
72.54	0\\
72.55	0\\
72.56	0\\
72.57	0\\
72.58	0\\
72.59	0\\
72.6	0\\
72.61	0\\
72.62	0\\
72.63	0\\
72.64	0\\
72.65	0\\
72.66	0\\
72.67	0\\
72.68	0\\
72.69	0\\
72.7	0\\
72.71	0\\
72.72	0\\
72.73	0\\
72.74	0\\
72.75	0\\
72.76	0\\
72.77	0\\
72.78	0\\
72.79	0\\
72.8	0\\
72.81	0\\
72.82	0\\
72.83	0\\
72.84	0\\
72.85	0\\
72.86	0\\
72.87	0\\
72.88	0\\
72.89	0\\
72.9	0\\
72.91	0\\
72.92	0\\
72.93	0\\
72.94	0\\
72.95	0\\
72.96	0\\
72.97	0\\
72.98	0\\
72.99	0\\
73	0\\
73.01	0\\
73.02	0\\
73.03	0\\
73.04	0\\
73.05	0\\
73.06	0\\
73.07	0\\
73.08	0\\
73.09	0\\
73.1	0\\
73.11	0\\
73.12	0\\
73.13	0\\
73.14	0\\
73.15	0\\
73.16	0\\
73.17	0\\
73.18	0\\
73.19	0\\
73.2	0\\
73.21	0\\
73.22	0\\
73.23	0\\
73.24	0\\
73.25	0\\
73.26	0\\
73.27	0\\
73.28	0\\
73.29	0\\
73.3	0\\
73.31	0\\
73.32	0\\
73.33	0\\
73.34	0\\
73.35	0\\
73.36	0\\
73.37	0\\
73.38	0\\
73.39	0\\
73.4	0\\
73.41	0\\
73.42	0\\
73.43	0\\
73.44	0\\
73.45	0\\
73.46	0\\
73.47	0\\
73.48	0\\
73.49	0\\
73.5	0\\
73.51	0\\
73.52	0\\
73.53	0\\
73.54	0\\
73.55	0\\
73.56	0\\
73.57	0\\
73.58	0\\
73.59	0\\
73.6	0\\
73.61	0\\
73.62	0\\
73.63	0\\
73.64	0\\
73.65	0\\
73.66	0\\
73.67	0\\
73.68	0\\
73.69	0\\
73.7	0\\
73.71	0\\
73.72	0\\
73.73	0\\
73.74	0\\
73.75	0\\
73.76	0\\
73.77	0\\
73.78	0\\
73.79	0\\
73.8	0\\
73.81	0\\
73.82	0\\
73.83	0\\
73.84	0\\
73.85	0\\
73.86	0\\
73.87	0\\
73.88	0\\
73.89	0\\
73.9	0\\
73.91	0\\
73.92	0\\
73.93	0\\
73.94	0\\
73.95	0\\
73.96	0\\
73.97	0\\
73.98	0\\
73.99	0\\
74	0\\
74.01	0\\
74.02	0\\
74.03	0\\
74.04	0\\
74.05	0\\
74.06	0\\
74.07	0\\
74.08	0\\
74.09	0\\
74.1	0\\
74.11	0\\
74.12	0\\
74.13	0\\
74.14	0\\
74.15	0\\
74.16	0\\
74.17	0\\
74.18	0\\
74.19	0\\
74.2	0\\
74.21	0\\
74.22	0\\
74.23	0\\
74.24	0\\
74.25	0\\
74.26	0\\
74.27	0\\
74.28	0\\
74.29	0\\
74.3	0\\
74.31	0\\
74.32	0\\
74.33	0\\
74.34	0\\
74.35	0\\
74.36	0\\
74.37	0\\
74.38	0\\
74.39	0\\
74.4	0\\
74.41	0\\
74.42	0\\
74.43	0\\
74.44	0\\
74.45	0\\
74.46	0\\
74.47	0\\
74.48	0\\
74.49	0\\
74.5	0\\
74.51	0\\
74.52	0\\
74.53	0\\
74.54	0\\
74.55	0\\
74.56	0\\
74.57	0\\
74.58	0\\
74.59	0\\
74.6	0\\
74.61	0\\
74.62	0\\
74.63	0\\
74.64	0\\
74.65	0\\
74.66	0\\
74.67	0\\
74.68	0\\
74.69	0\\
74.7	0\\
74.71	0\\
74.72	0\\
74.73	0\\
74.74	0\\
74.75	0\\
74.76	0\\
74.77	0\\
74.78	0\\
74.79	0\\
74.8	0\\
74.81	0\\
74.82	0\\
74.83	0\\
74.84	0\\
74.85	0\\
74.86	0\\
74.87	0\\
74.88	0\\
74.89	0\\
74.9	0\\
74.91	0\\
74.92	0\\
74.93	0\\
74.94	0\\
74.95	0\\
74.96	0\\
74.97	0\\
74.98	0\\
74.99	0\\
75	0\\
75.01	0\\
75.02	0\\
75.03	0\\
75.04	0\\
75.05	0\\
75.06	0\\
75.07	0\\
75.08	0\\
75.09	0\\
75.1	0\\
75.11	0\\
75.12	0\\
75.13	0\\
75.14	0\\
75.15	0\\
75.16	0\\
75.17	0\\
75.18	0\\
75.19	0\\
75.2	0\\
75.21	0\\
75.22	0\\
75.23	0\\
75.24	0\\
75.25	0\\
75.26	0\\
75.27	0\\
75.28	0\\
75.29	0\\
75.3	0\\
75.31	0\\
75.32	0\\
75.33	0\\
75.34	0\\
75.35	0\\
75.36	0\\
75.37	0\\
75.38	0\\
75.39	0\\
75.4	0\\
75.41	0\\
75.42	0\\
75.43	0\\
75.44	0\\
75.45	0\\
75.46	0\\
75.47	0\\
75.48	0\\
75.49	0\\
75.5	0\\
75.51	0\\
75.52	0\\
75.53	0\\
75.54	0\\
75.55	0\\
75.56	0\\
75.57	0\\
75.58	0\\
75.59	0\\
75.6	0\\
75.61	0\\
75.62	0\\
75.63	0\\
75.64	0\\
75.65	0\\
75.66	0\\
75.67	0\\
75.68	0\\
75.69	0\\
75.7	0\\
75.71	0\\
75.72	0\\
75.73	0\\
75.74	0\\
75.75	0\\
75.76	0\\
75.77	0\\
75.78	0\\
75.79	0\\
75.8	0\\
75.81	0\\
75.82	0\\
75.83	0\\
75.84	0\\
75.85	0\\
75.86	0\\
75.87	0\\
75.88	0\\
75.89	0\\
75.9	0\\
75.91	0\\
75.92	0\\
75.93	0\\
75.94	0\\
75.95	0\\
75.96	0\\
75.97	0\\
75.98	0\\
75.99	0\\
76	0\\
76.01	0\\
76.02	0\\
76.03	0\\
76.04	0\\
76.05	0\\
76.06	0\\
76.07	0\\
76.08	0\\
76.09	0\\
76.1	0\\
76.11	0\\
76.12	0\\
76.13	0\\
76.14	0\\
76.15	0\\
76.16	0\\
76.17	0\\
76.18	0\\
76.19	0\\
76.2	0\\
76.21	0\\
76.22	0\\
76.23	0\\
76.24	0\\
76.25	0\\
76.26	0\\
76.27	0\\
76.28	0\\
76.29	0\\
76.3	0\\
76.31	0\\
76.32	0\\
76.33	0\\
76.34	0\\
76.35	0\\
76.36	0\\
76.37	0\\
76.38	0\\
76.39	0\\
76.4	0\\
76.41	0\\
76.42	0\\
76.43	0\\
76.44	0\\
76.45	0\\
76.46	0\\
76.47	0\\
76.48	0\\
76.49	0\\
76.5	0\\
76.51	0\\
76.52	0\\
76.53	0\\
76.54	0\\
76.55	0\\
76.56	0\\
76.57	0\\
76.58	0\\
76.59	0\\
76.6	0\\
76.61	0\\
76.62	0\\
76.63	0\\
76.64	0\\
76.65	0\\
76.66	0\\
76.67	0\\
76.68	0\\
76.69	0\\
76.7	0\\
76.71	0\\
76.72	0\\
76.73	0\\
76.74	0\\
76.75	0\\
76.76	0\\
76.77	0\\
76.78	0\\
76.79	0\\
76.8	0\\
76.81	0\\
76.82	0\\
76.83	0\\
76.84	0\\
76.85	0\\
76.86	0\\
76.87	0\\
76.88	0\\
76.89	0\\
76.9	0\\
76.91	0\\
76.92	0\\
76.93	0\\
76.94	0\\
76.95	0\\
76.96	0\\
76.97	0\\
76.98	0\\
76.99	0\\
77	0\\
77.01	0\\
77.02	0\\
77.03	0\\
77.04	0\\
77.05	0\\
77.06	0\\
77.07	0\\
77.08	0\\
77.09	0\\
77.1	0\\
77.11	0\\
77.12	0\\
77.13	0\\
77.14	0\\
77.15	0\\
77.16	0\\
77.17	0\\
77.18	0\\
77.19	0\\
77.2	0\\
77.21	0\\
77.22	0\\
77.23	0\\
77.24	0\\
77.25	0\\
77.26	0\\
77.27	0\\
77.28	0\\
77.29	0\\
77.3	0\\
77.31	0\\
77.32	0\\
77.33	0\\
77.34	0\\
77.35	0\\
77.36	0\\
77.37	0\\
77.38	0\\
77.39	0\\
77.4	0\\
77.41	0\\
77.42	0\\
77.43	0\\
77.44	0\\
77.45	0\\
77.46	0\\
77.47	0\\
77.48	0\\
77.49	0\\
77.5	0\\
77.51	0\\
77.52	0\\
77.53	0\\
77.54	0\\
77.55	0\\
77.56	0\\
77.57	0\\
77.58	0\\
77.59	0\\
77.6	0\\
77.61	0\\
77.62	0\\
77.63	0\\
77.64	0\\
77.65	0\\
77.66	0\\
77.67	0\\
77.68	0\\
77.69	0\\
77.7	0\\
77.71	0\\
77.72	0\\
77.73	0\\
77.74	0\\
77.75	0\\
77.76	0\\
77.77	0\\
77.78	0\\
77.79	0\\
77.8	0\\
77.81	0\\
77.82	0\\
77.83	0\\
77.84	0\\
77.85	0\\
77.86	0\\
77.87	0\\
77.88	0\\
77.89	0\\
77.9	0\\
77.91	0\\
77.92	0\\
77.93	0\\
77.94	0\\
77.95	0\\
77.96	0\\
77.97	0\\
77.98	0\\
77.99	0\\
78	0\\
78.01	0\\
78.02	0\\
78.03	0\\
78.04	0\\
78.05	0\\
78.06	0\\
78.07	0\\
78.08	0\\
78.09	0\\
78.1	0\\
78.11	0\\
78.12	0\\
78.13	0\\
78.14	0\\
78.15	0\\
78.16	0\\
78.17	0\\
78.18	0\\
78.19	0\\
78.2	0\\
78.21	0\\
78.22	0\\
78.23	0\\
78.24	0\\
78.25	0\\
78.26	0\\
78.27	0\\
78.28	0\\
78.29	0\\
78.3	0\\
78.31	0\\
78.32	0\\
78.33	0\\
78.34	0\\
78.35	0\\
78.36	0\\
78.37	0\\
78.38	0\\
78.39	0\\
78.4	0\\
78.41	0\\
78.42	0\\
78.43	0\\
78.44	0\\
78.45	0\\
78.46	0\\
78.47	0\\
78.48	0\\
78.49	0\\
78.5	0\\
78.51	0\\
78.52	0\\
78.53	0\\
78.54	0\\
78.55	0\\
78.56	0\\
78.57	0\\
78.58	0\\
78.59	0\\
78.6	0\\
78.61	0\\
78.62	0\\
78.63	0\\
78.64	0\\
78.65	0\\
78.66	0\\
78.67	0\\
78.68	0\\
78.69	0\\
78.7	0\\
78.71	0\\
78.72	0\\
78.73	0\\
78.74	0\\
78.75	0\\
78.76	0\\
78.77	0\\
78.78	0\\
78.79	0\\
78.8	0\\
78.81	0\\
78.82	0\\
78.83	0\\
78.84	0\\
78.85	0\\
78.86	0\\
78.87	0\\
78.88	0\\
78.89	0\\
78.9	0\\
78.91	0\\
78.92	0\\
78.93	0\\
78.94	0\\
78.95	0\\
78.96	0\\
78.97	0\\
78.98	0\\
78.99	0\\
79	0\\
79.01	0\\
79.02	0\\
79.03	0\\
79.04	0\\
79.05	0\\
79.06	0\\
79.07	0\\
79.08	0\\
79.09	0\\
79.1	0\\
79.11	0\\
79.12	0\\
79.13	0\\
79.14	0\\
79.15	0\\
79.16	0\\
79.17	0\\
79.18	0\\
79.19	0\\
79.2	0\\
79.21	0\\
79.22	0\\
79.23	0\\
79.24	0\\
79.25	0\\
79.26	0\\
79.27	0\\
79.28	0\\
79.29	0\\
79.3	0\\
79.31	0\\
79.32	0\\
79.33	0\\
79.34	0\\
79.35	0\\
79.36	0\\
79.37	0\\
79.38	0\\
79.39	0\\
79.4	0\\
79.41	0\\
79.42	0\\
79.43	0\\
79.44	0\\
79.45	0\\
79.46	0\\
79.47	0\\
79.48	0\\
79.49	0\\
79.5	0\\
79.51	0\\
79.52	0\\
79.53	0\\
79.54	0\\
79.55	0\\
79.56	0\\
79.57	0\\
79.58	0\\
79.59	0\\
79.6	0\\
79.61	0\\
79.62	0\\
79.63	0\\
79.64	0\\
79.65	0\\
79.66	0\\
79.67	0\\
79.68	0\\
79.69	0\\
79.7	0\\
79.71	0\\
79.72	0\\
79.73	0\\
79.74	0\\
79.75	0\\
79.76	0\\
79.77	0\\
79.78	0\\
79.79	0\\
79.8	0\\
79.81	0\\
79.82	0\\
79.83	0\\
79.84	0\\
79.85	0\\
79.86	0\\
79.87	0\\
79.88	0\\
79.89	0\\
79.9	0\\
79.91	0\\
79.92	0\\
79.93	0\\
79.94	0\\
79.95	0\\
79.96	0\\
79.97	0\\
79.98	0\\
79.99	0\\
80	0\\
80.01	0\\
};
\addplot [color=green,dashed]
  table[row sep=crcr]{%
80.01	0\\
80.02	0\\
80.03	0\\
80.04	0\\
80.05	0\\
80.06	0\\
80.07	0\\
80.08	0\\
80.09	0\\
80.1	0\\
80.11	0\\
80.12	0\\
80.13	0\\
80.14	0\\
80.15	0\\
80.16	0\\
80.17	0\\
80.18	0\\
80.19	0\\
80.2	0\\
80.21	0\\
80.22	0\\
80.23	0\\
80.24	0\\
80.25	0\\
80.26	0\\
80.27	0\\
80.28	0\\
80.29	0\\
80.3	0\\
80.31	0\\
80.32	0\\
80.33	0\\
80.34	0\\
80.35	0\\
80.36	0\\
80.37	0\\
80.38	0\\
80.39	0\\
80.4	0\\
80.41	0\\
80.42	0\\
80.43	0\\
80.44	0\\
80.45	0\\
80.46	0\\
80.47	0\\
80.48	0\\
80.49	0\\
80.5	0\\
80.51	0\\
80.52	0\\
80.53	0\\
80.54	0\\
80.55	0\\
80.56	0\\
80.57	0\\
80.58	0\\
80.59	0\\
80.6	0\\
80.61	0\\
80.62	0\\
80.63	0\\
80.64	0\\
80.65	0\\
80.66	0\\
80.67	0\\
80.68	0\\
80.69	0\\
80.7	0\\
80.71	0\\
80.72	0\\
80.73	0\\
80.74	0\\
80.75	0\\
80.76	0\\
80.77	0\\
80.78	0\\
80.79	0\\
80.8	0\\
80.81	0\\
80.82	0\\
80.83	0\\
80.84	0\\
80.85	0\\
80.86	0\\
80.87	0\\
80.88	0\\
80.89	0\\
80.9	0\\
80.91	0\\
80.92	0\\
80.93	0\\
80.94	0\\
80.95	0\\
80.96	0\\
80.97	0\\
80.98	0\\
80.99	0\\
81	0\\
81.01	0\\
81.02	0\\
81.03	0\\
81.04	0\\
81.05	0\\
81.06	0\\
81.07	0\\
81.08	0\\
81.09	0\\
81.1	0\\
81.11	0\\
81.12	0\\
81.13	0\\
81.14	0\\
81.15	0\\
81.16	0\\
81.17	0\\
81.18	0\\
81.19	0\\
81.2	0\\
81.21	0\\
81.22	0\\
81.23	0\\
81.24	0\\
81.25	0\\
81.26	0\\
81.27	0\\
81.28	0\\
81.29	0\\
81.3	0\\
81.31	0\\
81.32	0\\
81.33	0\\
81.34	0\\
81.35	0\\
81.36	0\\
81.37	0\\
81.38	0\\
81.39	0\\
81.4	0\\
81.41	0\\
81.42	0\\
81.43	0\\
81.44	0\\
81.45	0\\
81.46	0\\
81.47	0\\
81.48	0\\
81.49	0\\
81.5	0\\
81.51	0\\
81.52	0\\
81.53	0\\
81.54	0\\
81.55	0\\
81.56	0\\
81.57	0\\
81.58	0\\
81.59	0\\
81.6	0\\
81.61	0\\
81.62	0\\
81.63	0\\
81.64	0\\
81.65	0\\
81.66	0\\
81.67	0\\
81.68	0\\
81.69	0\\
81.7	0\\
81.71	0\\
81.72	0\\
81.73	0\\
81.74	0\\
81.75	0\\
81.76	0\\
81.77	0\\
81.78	0\\
81.79	0\\
81.8	0\\
81.81	0\\
81.82	0\\
81.83	0\\
81.84	0\\
81.85	0\\
81.86	0\\
81.87	0\\
81.88	0\\
81.89	0\\
81.9	0\\
81.91	0\\
81.92	0\\
81.93	0\\
81.94	0\\
81.95	0\\
81.96	0\\
81.97	0\\
81.98	0\\
81.99	0\\
82	0\\
82.01	0\\
82.02	0\\
82.03	0\\
82.04	0\\
82.05	0\\
82.06	0\\
82.07	0\\
82.08	0\\
82.09	0\\
82.1	0\\
82.11	0\\
82.12	0\\
82.13	0\\
82.14	0\\
82.15	0\\
82.16	0\\
82.17	0\\
82.18	0\\
82.19	0\\
82.2	0\\
82.21	0\\
82.22	0\\
82.23	0\\
82.24	0\\
82.25	0\\
82.26	0\\
82.27	0\\
82.28	0\\
82.29	0\\
82.3	0\\
82.31	0\\
82.32	0\\
82.33	0\\
82.34	0\\
82.35	0\\
82.36	0\\
82.37	0\\
82.38	0\\
82.39	0\\
82.4	0\\
82.41	0\\
82.42	0\\
82.43	0\\
82.44	0\\
82.45	0\\
82.46	0\\
82.47	0\\
82.48	0\\
82.49	0\\
82.5	0\\
82.51	0\\
82.52	0\\
82.53	0\\
82.54	0\\
82.55	0\\
82.56	0\\
82.57	0\\
82.58	0\\
82.59	0\\
82.6	0\\
82.61	0\\
82.62	0\\
82.63	0\\
82.64	0\\
82.65	0\\
82.66	0\\
82.67	0\\
82.68	0\\
82.69	0\\
82.7	0\\
82.71	0\\
82.72	0\\
82.73	0\\
82.74	0\\
82.75	0\\
82.76	0\\
82.77	0\\
82.78	0\\
82.79	0\\
82.8	0\\
82.81	0\\
82.82	0\\
82.83	0\\
82.84	0\\
82.85	0\\
82.86	0\\
82.87	0\\
82.88	0\\
82.89	0\\
82.9	0\\
82.91	0\\
82.92	0\\
82.93	0\\
82.94	0\\
82.95	0\\
82.96	0\\
82.97	0\\
82.98	0\\
82.99	0\\
83	0\\
83.01	0\\
83.02	0\\
83.03	0\\
83.04	0\\
83.05	0\\
83.06	0\\
83.07	0\\
83.08	0\\
83.09	0\\
83.1	0\\
83.11	0\\
83.12	0\\
83.13	0\\
83.14	0\\
83.15	0\\
83.16	0\\
83.17	0\\
83.18	0\\
83.19	0\\
83.2	0\\
83.21	0\\
83.22	0\\
83.23	0\\
83.24	0\\
83.25	0\\
83.26	0\\
83.27	0\\
83.28	0\\
83.29	0\\
83.3	0\\
83.31	0\\
83.32	0\\
83.33	0\\
83.34	0\\
83.35	0\\
83.36	0\\
83.37	0\\
83.38	0\\
83.39	0\\
83.4	0\\
83.41	0\\
83.42	0\\
83.43	0\\
83.44	0\\
83.45	0\\
83.46	0\\
83.47	0\\
83.48	0\\
83.49	0\\
83.5	0\\
83.51	0\\
83.52	0\\
83.53	0\\
83.54	0\\
83.55	0\\
83.56	0\\
83.57	0\\
83.58	0\\
83.59	0\\
83.6	0\\
83.61	0\\
83.62	0\\
83.63	0\\
83.64	0\\
83.65	0\\
83.66	0\\
83.67	0\\
83.68	0\\
83.69	0\\
83.7	0\\
83.71	0\\
83.72	0\\
83.73	0\\
83.74	0\\
83.75	0\\
83.76	0\\
83.77	0\\
83.78	0\\
83.79	0\\
83.8	0\\
83.81	0\\
83.82	0\\
83.83	0\\
83.84	0\\
83.85	0\\
83.86	0\\
83.87	0\\
83.88	0\\
83.89	0\\
83.9	0\\
83.91	0\\
83.92	0\\
83.93	0\\
83.94	0\\
83.95	0\\
83.96	0\\
83.97	0\\
83.98	0\\
83.99	0\\
84	0\\
84.01	0\\
84.02	0\\
84.03	0\\
84.04	0\\
84.05	0\\
84.06	0\\
84.07	0\\
84.08	0\\
84.09	0\\
84.1	0\\
84.11	0\\
84.12	0\\
84.13	0\\
84.14	0\\
84.15	0\\
84.16	0\\
84.17	0\\
84.18	0\\
84.19	0\\
84.2	0\\
84.21	0\\
84.22	0\\
84.23	0\\
84.24	0\\
84.25	0\\
84.26	0\\
84.27	0\\
84.28	0\\
84.29	0\\
84.3	0\\
84.31	0\\
84.32	0\\
84.33	0\\
84.34	0\\
84.35	0\\
84.36	0\\
84.37	0\\
84.38	0\\
84.39	0\\
84.4	0\\
84.41	0\\
84.42	0\\
84.43	0\\
84.44	0\\
84.45	0\\
84.46	0\\
84.47	0\\
84.48	0\\
84.49	0\\
84.5	0\\
84.51	0\\
84.52	0\\
84.53	0\\
84.54	0\\
84.55	0\\
84.56	0\\
84.57	0\\
84.58	0\\
84.59	0\\
84.6	0\\
84.61	0\\
84.62	0\\
84.63	0\\
84.64	0\\
84.65	0\\
84.66	0\\
84.67	0\\
84.68	0\\
84.69	0\\
84.7	0\\
84.71	0\\
84.72	0\\
84.73	0\\
84.74	0\\
84.75	0\\
84.76	0\\
84.77	0\\
84.78	0\\
84.79	0\\
84.8	0\\
84.81	0\\
84.82	0\\
84.83	0\\
84.84	0\\
84.85	0\\
84.86	0\\
84.87	0\\
84.88	0\\
84.89	0\\
84.9	0\\
84.91	0\\
84.92	0\\
84.93	0\\
84.94	0\\
84.95	0\\
84.96	0\\
84.97	0\\
84.98	0\\
84.99	0\\
85	0\\
85.01	0\\
85.02	0\\
85.03	0\\
85.04	0\\
85.05	0\\
85.06	0\\
85.07	0\\
85.08	0\\
85.09	0\\
85.1	0\\
85.11	0\\
85.12	0\\
85.13	0\\
85.14	0\\
85.15	0\\
85.16	0\\
85.17	0\\
85.18	0\\
85.19	0\\
85.2	0\\
85.21	0\\
85.22	0\\
85.23	0\\
85.24	0\\
85.25	0\\
85.26	0\\
85.27	0\\
85.28	0\\
85.29	0\\
85.3	0\\
85.31	0\\
85.32	0\\
85.33	0\\
85.34	0\\
85.35	0\\
85.36	0\\
85.37	0\\
85.38	0\\
85.39	0\\
85.4	0\\
85.41	0\\
85.42	0\\
85.43	0\\
85.44	0\\
85.45	0\\
85.46	0\\
85.47	0\\
85.48	0\\
85.49	0\\
85.5	0\\
85.51	0\\
85.52	0\\
85.53	0\\
85.54	0\\
85.55	0\\
85.56	0\\
85.57	0\\
85.58	0\\
85.59	0\\
85.6	0\\
85.61	0\\
85.62	0\\
85.63	0\\
85.64	0\\
85.65	0\\
85.66	0\\
85.67	0\\
85.68	0\\
85.69	0\\
85.7	0\\
85.71	0\\
85.72	0\\
85.73	0\\
85.74	0\\
85.75	0\\
85.76	0\\
85.77	0\\
85.78	0\\
85.79	0\\
85.8	0\\
85.81	0\\
85.82	0\\
85.83	0\\
85.84	0\\
85.85	0\\
85.86	0\\
85.87	0\\
85.88	0\\
85.89	0\\
85.9	0\\
85.91	0\\
85.92	0\\
85.93	0\\
85.94	0\\
85.95	0\\
85.96	0\\
85.97	0\\
85.98	0\\
85.99	0\\
86	0\\
86.01	0\\
86.02	0\\
86.03	0\\
86.04	0\\
86.05	0\\
86.06	0\\
86.07	0\\
86.08	0\\
86.09	0\\
86.1	0\\
86.11	0\\
86.12	0\\
86.13	0\\
86.14	0\\
86.15	0\\
86.16	0\\
86.17	0\\
86.18	0\\
86.19	0\\
86.2	0\\
86.21	0\\
86.22	0\\
86.23	0\\
86.24	0\\
86.25	0\\
86.26	0\\
86.27	0\\
86.28	0\\
86.29	0\\
86.3	0\\
86.31	0\\
86.32	0\\
86.33	0\\
86.34	0\\
86.35	0\\
86.36	0\\
86.37	0\\
86.38	0\\
86.39	0\\
86.4	0\\
86.41	0\\
86.42	0\\
86.43	0\\
86.44	0\\
86.45	0\\
86.46	0\\
86.47	0\\
86.48	0\\
86.49	0\\
86.5	0\\
86.51	0\\
86.52	0\\
86.53	0\\
86.54	0\\
86.55	0\\
86.56	0\\
86.57	0\\
86.58	0\\
86.59	0\\
86.6	0\\
86.61	0\\
86.62	0\\
86.63	0\\
86.64	0\\
86.65	0\\
86.66	0\\
86.67	0\\
86.68	0\\
86.69	0\\
86.7	0\\
86.71	0\\
86.72	0\\
86.73	0\\
86.74	0\\
86.75	0\\
86.76	0\\
86.77	0\\
86.78	0\\
86.79	0\\
86.8	0\\
86.81	0\\
86.82	0\\
86.83	0\\
86.84	0\\
86.85	0\\
86.86	0\\
86.87	0\\
86.88	0\\
86.89	0\\
86.9	0\\
86.91	0\\
86.92	0\\
86.93	0\\
86.94	0\\
86.95	0\\
86.96	0\\
86.97	0\\
86.98	0\\
86.99	0\\
87	0\\
87.01	0\\
87.02	0\\
87.03	0\\
87.04	0\\
87.05	0\\
87.06	0\\
87.07	0\\
87.08	0\\
87.09	0\\
87.1	0\\
87.11	0\\
87.12	0\\
87.13	0\\
87.14	0\\
87.15	0\\
87.16	0\\
87.17	0\\
87.18	0\\
87.19	0\\
87.2	0\\
87.21	0\\
87.22	0\\
87.23	0\\
87.24	0\\
87.25	0\\
87.26	0\\
87.27	0\\
87.28	0\\
87.29	0\\
87.3	0\\
87.31	0\\
87.32	0\\
87.33	0\\
87.34	0\\
87.35	0\\
87.36	0\\
87.37	0\\
87.38	0\\
87.39	0\\
87.4	0\\
87.41	0\\
87.42	0\\
87.43	0\\
87.44	0\\
87.45	0\\
87.46	0\\
87.47	0\\
87.48	0\\
87.49	0\\
87.5	0\\
87.51	0\\
87.52	0\\
87.53	0\\
87.54	0\\
87.55	0\\
87.56	0\\
87.57	0\\
87.58	0\\
87.59	0\\
87.6	0\\
87.61	0\\
87.62	0\\
87.63	0\\
87.64	0\\
87.65	0\\
87.66	0\\
87.67	0\\
87.68	0\\
87.69	0\\
87.7	0\\
87.71	0\\
87.72	0\\
87.73	0\\
87.74	0\\
87.75	0\\
87.76	0\\
87.77	0\\
87.78	0\\
87.79	0\\
87.8	0\\
87.81	0\\
87.82	0\\
87.83	0\\
87.84	0\\
87.85	0\\
87.86	0\\
87.87	0\\
87.88	0\\
87.89	0\\
87.9	0\\
87.91	0\\
87.92	0\\
87.93	0\\
87.94	0\\
87.95	0\\
87.96	0\\
87.97	0\\
87.98	0\\
87.99	0\\
88	0\\
88.01	0\\
88.02	0\\
88.03	0\\
88.04	0\\
88.05	0\\
88.06	0\\
88.07	0\\
88.08	0\\
88.09	0\\
88.1	0\\
88.11	0\\
88.12	0\\
88.13	0\\
88.14	0\\
88.15	0\\
88.16	0\\
88.17	0\\
88.18	0\\
88.19	0\\
88.2	0\\
88.21	0\\
88.22	0\\
88.23	0\\
88.24	0\\
88.25	0\\
88.26	0\\
88.27	0\\
88.28	0\\
88.29	0\\
88.3	0\\
88.31	0\\
88.32	0\\
88.33	0\\
88.34	0\\
88.35	0\\
88.36	0\\
88.37	0\\
88.38	0\\
88.39	0\\
88.4	0\\
88.41	0\\
88.42	0\\
88.43	0\\
88.44	0\\
88.45	0\\
88.46	0\\
88.47	0\\
88.48	0\\
88.49	0\\
88.5	0\\
88.51	0\\
88.52	0\\
88.53	0\\
88.54	0\\
88.55	0\\
88.56	0\\
88.57	0\\
88.58	0\\
88.59	0\\
88.6	0\\
88.61	0\\
88.62	0\\
88.63	0\\
88.64	0\\
88.65	0\\
88.66	0\\
88.67	0\\
88.68	0\\
88.69	0\\
88.7	0\\
88.71	0\\
88.72	0\\
88.73	0\\
88.74	0\\
88.75	0\\
88.76	0\\
88.77	0\\
88.78	0\\
88.79	0\\
88.8	0\\
88.81	0\\
88.82	0\\
88.83	0\\
88.84	0\\
88.85	0\\
88.86	0\\
88.87	0\\
88.88	0\\
88.89	0\\
88.9	0\\
88.91	0\\
88.92	0\\
88.93	0\\
88.94	0\\
88.95	0\\
88.96	0\\
88.97	0\\
88.98	0\\
88.99	0\\
89	0\\
89.01	0\\
89.02	0\\
89.03	0\\
89.04	0\\
89.05	0\\
89.06	0\\
89.07	0\\
89.08	0\\
89.09	0\\
89.1	0\\
89.11	0\\
89.12	0\\
89.13	0\\
89.14	0\\
89.15	0\\
89.16	0\\
89.17	0\\
89.18	0\\
89.19	0\\
89.2	0\\
89.21	0\\
89.22	0\\
89.23	0\\
89.24	0\\
89.25	0\\
89.26	0\\
89.27	0\\
89.28	0\\
89.29	0\\
89.3	0\\
89.31	0\\
89.32	0\\
89.33	0\\
89.34	0\\
89.35	0\\
89.36	0\\
89.37	0\\
89.38	0\\
89.39	0\\
89.4	0\\
89.41	0\\
89.42	0\\
89.43	0\\
89.44	0\\
89.45	0\\
89.46	0\\
89.47	0\\
89.48	0\\
89.49	0\\
89.5	0\\
89.51	0\\
89.52	0\\
89.53	0\\
89.54	0\\
89.55	0\\
89.56	0\\
89.57	0\\
89.58	0\\
89.59	0\\
89.6	0\\
89.61	0\\
89.62	0\\
89.63	0\\
89.64	0\\
89.65	0\\
89.66	0\\
89.67	0\\
89.68	0\\
89.69	0\\
89.7	0\\
89.71	0\\
89.72	0\\
89.73	0\\
89.74	0\\
89.75	0\\
89.76	0\\
89.77	0\\
89.78	0\\
89.79	0\\
89.8	0\\
89.81	0\\
89.82	0\\
89.83	0\\
89.84	0\\
89.85	0\\
89.86	0\\
89.87	0\\
89.88	0\\
89.89	0\\
89.9	0\\
89.91	0\\
89.92	0\\
89.93	0\\
89.94	0\\
89.95	0\\
89.96	0\\
89.97	0\\
89.98	0\\
89.99	0\\
90	0\\
90.01	0\\
90.02	0\\
90.03	0\\
90.04	0\\
90.05	0\\
90.06	0\\
90.07	0\\
90.08	0\\
90.09	0\\
90.1	0\\
90.11	0\\
90.12	0\\
90.13	0\\
90.14	0\\
90.15	0\\
90.16	0\\
90.17	0\\
90.18	0\\
90.19	0\\
90.2	0\\
90.21	0\\
90.22	0\\
90.23	0\\
90.24	0\\
90.25	0\\
90.26	0\\
90.27	0\\
90.28	0\\
90.29	0\\
90.3	0\\
90.31	0\\
90.32	0\\
90.33	0\\
90.34	0\\
90.35	0\\
90.36	0\\
90.37	0\\
90.38	0\\
90.39	0\\
90.4	0\\
90.41	0\\
90.42	0\\
90.43	0\\
90.44	0\\
90.45	0\\
90.46	0\\
90.47	0\\
90.48	0\\
90.49	0\\
90.5	0\\
90.51	0\\
90.52	0\\
90.53	0\\
90.54	0\\
90.55	0\\
90.56	0\\
90.57	0\\
90.58	0\\
90.59	0\\
90.6	0\\
90.61	0\\
90.62	0\\
90.63	0\\
90.64	0\\
90.65	0\\
90.66	0\\
90.67	0\\
90.68	0\\
90.69	0\\
90.7	0\\
90.71	0\\
90.72	0\\
90.73	0\\
90.74	0\\
90.75	0\\
90.76	0\\
90.77	0\\
90.78	0\\
90.79	0\\
90.8	0\\
90.81	0\\
90.82	0\\
90.83	0\\
90.84	0\\
90.85	0\\
90.86	0\\
90.87	0\\
90.88	0\\
90.89	0\\
90.9	0\\
90.91	0\\
90.92	0\\
90.93	0\\
90.94	0\\
90.95	0\\
90.96	0\\
90.97	0\\
90.98	0\\
90.99	0\\
91	0\\
91.01	0\\
91.02	0\\
91.03	0\\
91.04	0\\
91.05	0\\
91.06	0\\
91.07	0\\
91.08	0\\
91.09	0\\
91.1	0\\
91.11	0\\
91.12	0\\
91.13	0\\
91.14	0\\
91.15	0\\
91.16	0\\
91.17	0\\
91.18	0\\
91.19	0\\
91.2	0\\
91.21	0\\
91.22	0\\
91.23	0\\
91.24	0\\
91.25	0\\
91.26	0\\
91.27	0\\
91.28	0\\
91.29	0\\
91.3	0\\
91.31	0\\
91.32	0\\
91.33	0\\
91.34	0\\
91.35	0\\
91.36	0\\
91.37	0\\
91.38	0\\
91.39	0\\
91.4	0\\
91.41	0\\
91.42	0\\
91.43	0\\
91.44	0\\
91.45	0\\
91.46	0\\
91.47	0\\
91.48	0\\
91.49	0\\
91.5	0\\
91.51	0\\
91.52	0\\
91.53	0\\
91.54	0\\
91.55	0\\
91.56	0\\
91.57	0\\
91.58	0\\
91.59	0\\
91.6	0\\
91.61	0\\
91.62	0\\
91.63	0\\
91.64	0\\
91.65	0\\
91.66	0\\
91.67	0\\
91.68	0\\
91.69	0\\
91.7	0\\
91.71	0\\
91.72	0\\
91.73	0\\
91.74	0\\
91.75	0\\
91.76	0\\
91.77	0\\
91.78	0\\
91.79	0\\
91.8	0\\
91.81	0\\
91.82	0\\
91.83	0\\
91.84	0\\
91.85	0\\
91.86	0\\
91.87	0\\
91.88	0\\
91.89	0\\
91.9	0\\
91.91	0\\
91.92	0\\
91.93	0\\
91.94	0\\
91.95	0\\
91.96	0\\
91.97	0\\
91.98	0\\
91.99	0\\
92	0\\
92.01	0\\
92.02	0\\
92.03	0\\
92.04	0\\
92.05	0\\
92.06	0\\
92.07	0\\
92.08	0\\
92.09	0\\
92.1	0\\
92.11	0\\
92.12	0\\
92.13	0\\
92.14	0\\
92.15	0\\
92.16	0\\
92.17	0\\
92.18	0\\
92.19	0\\
92.2	0\\
92.21	0\\
92.22	0\\
92.23	0\\
92.24	0\\
92.25	0\\
92.26	0\\
92.27	0\\
92.28	0\\
92.29	0\\
92.3	0\\
92.31	0\\
92.32	0\\
92.33	0\\
92.34	0\\
92.35	0\\
92.36	0\\
92.37	0\\
92.38	0\\
92.39	0\\
92.4	0\\
92.41	0\\
92.42	0\\
92.43	0\\
92.44	0\\
92.45	0\\
92.46	0\\
92.47	0\\
92.48	0\\
92.49	0\\
92.5	0\\
92.51	0\\
92.52	0\\
92.53	0\\
92.54	0\\
92.55	0\\
92.56	0\\
92.57	0\\
92.58	0\\
92.59	0\\
92.6	0\\
92.61	0\\
92.62	0\\
92.63	0\\
92.64	0\\
92.65	0\\
92.66	0\\
92.67	0\\
92.68	0\\
92.69	0\\
92.7	0\\
92.71	0\\
92.72	0\\
92.73	0\\
92.74	0\\
92.75	0\\
92.76	0\\
92.77	0\\
92.78	0\\
92.79	0\\
92.8	0\\
92.81	0\\
92.82	0\\
92.83	0\\
92.84	0\\
92.85	0\\
92.86	0\\
92.87	0\\
92.88	0\\
92.89	0\\
92.9	0\\
92.91	0\\
92.92	0\\
92.93	0\\
92.94	0\\
92.95	0\\
92.96	0\\
92.97	0\\
92.98	0\\
92.99	0\\
93	0\\
93.01	0\\
93.02	0\\
93.03	0\\
93.04	0\\
93.05	0\\
93.06	0\\
93.07	0\\
93.08	0\\
93.09	0\\
93.1	0\\
93.11	0\\
93.12	0\\
93.13	0\\
93.14	0\\
93.15	0\\
93.16	0\\
93.17	0\\
93.18	0\\
93.19	0\\
93.2	0\\
93.21	0\\
93.22	0\\
93.23	0\\
93.24	0\\
93.25	0\\
93.26	0\\
93.27	0\\
93.28	0\\
93.29	0\\
93.3	0\\
93.31	0\\
93.32	0\\
93.33	0\\
93.34	0\\
93.35	0\\
93.36	0\\
93.37	0\\
93.38	0\\
93.39	0\\
93.4	0\\
93.41	0\\
93.42	0\\
93.43	0\\
93.44	0\\
93.45	0\\
93.46	0\\
93.47	0\\
93.48	0\\
93.49	0\\
93.5	0\\
93.51	0\\
93.52	0\\
93.53	0\\
93.54	0\\
93.55	0\\
93.56	0\\
93.57	0\\
93.58	0\\
93.59	0\\
93.6	0\\
93.61	0\\
93.62	0\\
93.63	0\\
93.64	0\\
93.65	0\\
93.66	0\\
93.67	0\\
93.68	0\\
93.69	0\\
93.7	0\\
93.71	0\\
93.72	0\\
93.73	0\\
93.74	0\\
93.75	0\\
93.76	0\\
93.77	0\\
93.78	0\\
93.79	0\\
93.8	0\\
93.81	0\\
93.82	0\\
93.83	0\\
93.84	0\\
93.85	0\\
93.86	0\\
93.87	0\\
93.88	0\\
93.89	0\\
93.9	0\\
93.91	0\\
93.92	0\\
93.93	0\\
93.94	0\\
93.95	0\\
93.96	0\\
93.97	0\\
93.98	0\\
93.99	0\\
94	0\\
94.01	0\\
94.02	0\\
94.03	0\\
94.04	0\\
94.05	0\\
94.06	0\\
94.07	0\\
94.08	0\\
94.09	0\\
94.1	0\\
94.11	0\\
94.12	0\\
94.13	0\\
94.14	0\\
94.15	0\\
94.16	0\\
94.17	0\\
94.18	0\\
94.19	0\\
94.2	0\\
94.21	0\\
94.22	0\\
94.23	0\\
94.24	0\\
94.25	0\\
94.26	0\\
94.27	0\\
94.28	0\\
94.29	0\\
94.3	0\\
94.31	0\\
94.32	0\\
94.33	0\\
94.34	0\\
94.35	0\\
94.36	0\\
94.37	0\\
94.38	0\\
94.39	0\\
94.4	0\\
94.41	0\\
94.42	0\\
94.43	0\\
94.44	0\\
94.45	0\\
94.46	0\\
94.47	0\\
94.48	0\\
94.49	0\\
94.5	0\\
94.51	0\\
94.52	0\\
94.53	0\\
94.54	0\\
94.55	0\\
94.56	0\\
94.57	0\\
94.58	0\\
94.59	0\\
94.6	0\\
94.61	0\\
94.62	0\\
94.63	0\\
94.64	0\\
94.65	0\\
94.66	0\\
94.67	0\\
94.68	0\\
94.69	0\\
94.7	0\\
94.71	0\\
94.72	0\\
94.73	0\\
94.74	0\\
94.75	0\\
94.76	0\\
94.77	0\\
94.78	0\\
94.79	0\\
94.8	0\\
94.81	0\\
94.82	0\\
94.83	0\\
94.84	0\\
94.85	0\\
94.86	0\\
94.87	0\\
94.88	0\\
94.89	0\\
94.9	0\\
94.91	0\\
94.92	0\\
94.93	0\\
94.94	0\\
94.95	0\\
94.96	0\\
94.97	0\\
94.98	0\\
94.99	0\\
95	0\\
95.01	0\\
95.02	0\\
95.03	0\\
95.04	0\\
95.05	0\\
95.06	0\\
95.07	0\\
95.08	0\\
95.09	0\\
95.1	0\\
95.11	0\\
95.12	0\\
95.13	0\\
95.14	0\\
95.15	0\\
95.16	0\\
95.17	0\\
95.18	0\\
95.19	0\\
95.2	0\\
95.21	0\\
95.22	0\\
95.23	0\\
95.24	0\\
95.25	0\\
95.26	0\\
95.27	0\\
95.28	0\\
95.29	0\\
95.3	0\\
95.31	0\\
95.32	0\\
95.33	0\\
95.34	0\\
95.35	0\\
95.36	0\\
95.37	0\\
95.38	0\\
95.39	0\\
95.4	0\\
95.41	0\\
95.42	0\\
95.43	0\\
95.44	0\\
95.45	0\\
95.46	0\\
95.47	0\\
95.48	0\\
95.49	0\\
95.5	0\\
95.51	0\\
95.52	0\\
95.53	0\\
95.54	0\\
95.55	0\\
95.56	0\\
95.57	0\\
95.58	0\\
95.59	0\\
95.6	0\\
95.61	0\\
95.62	0\\
95.63	0\\
95.64	0\\
95.65	0\\
95.66	0\\
95.67	0\\
95.68	0\\
95.69	0\\
95.7	0\\
95.71	0\\
95.72	0\\
95.73	0\\
95.74	0\\
95.75	0\\
95.76	0\\
95.77	0\\
95.78	0\\
95.79	0\\
95.8	0\\
95.81	0\\
95.82	0\\
95.83	0\\
95.84	0\\
95.85	0\\
95.86	0\\
95.87	0\\
95.88	0\\
95.89	0\\
95.9	0\\
95.91	0\\
95.92	0\\
95.93	0\\
95.94	0\\
95.95	0\\
95.96	0\\
95.97	0\\
95.98	0\\
95.99	0\\
96	0\\
96.01	0\\
96.02	0\\
96.03	0\\
96.04	0\\
96.05	0\\
96.06	0\\
96.07	0\\
96.08	0\\
96.09	0\\
96.1	0\\
96.11	0\\
96.12	0\\
96.13	0\\
96.14	0\\
96.15	0\\
96.16	0\\
96.17	0\\
96.18	0\\
96.19	0\\
96.2	0\\
96.21	0\\
96.22	0\\
96.23	0\\
96.24	0\\
96.25	0\\
96.26	0\\
96.27	0\\
96.28	0\\
96.29	0\\
96.3	0\\
96.31	0\\
96.32	0\\
96.33	0\\
96.34	0\\
96.35	0\\
96.36	0\\
96.37	0\\
96.38	0\\
96.39	0\\
96.4	0\\
96.41	0\\
96.42	0\\
96.43	0\\
96.44	0\\
96.45	0\\
96.46	0\\
96.47	0\\
96.48	0\\
96.49	0\\
96.5	0\\
96.51	0\\
96.52	0\\
96.53	0\\
96.54	0\\
96.55	0\\
96.56	0\\
96.57	0\\
96.58	0\\
96.59	0\\
96.6	0\\
96.61	0\\
96.62	0\\
96.63	0\\
96.64	0\\
96.65	0\\
96.66	0\\
96.67	0\\
96.68	0\\
96.69	0\\
96.7	0\\
96.71	0\\
96.72	0\\
96.73	0\\
96.74	0\\
96.75	0\\
96.76	0\\
96.77	0\\
96.78	0\\
96.79	0\\
96.8	0\\
96.81	0\\
96.82	0\\
96.83	0\\
96.84	0\\
96.85	0\\
96.86	0\\
96.87	0\\
96.88	0\\
96.89	0\\
96.9	0\\
96.91	0\\
96.92	0\\
96.93	0\\
96.94	0\\
96.95	0\\
96.96	0\\
96.97	0\\
96.98	0\\
96.99	0\\
97	0\\
97.01	0\\
97.02	0\\
97.03	0\\
97.04	0\\
97.05	0\\
97.06	0\\
97.07	0\\
97.08	0\\
97.09	0\\
97.1	0\\
97.11	0\\
97.12	0\\
97.13	0\\
97.14	0\\
97.15	0\\
97.16	0\\
97.17	0\\
97.18	0\\
97.19	0\\
97.2	0\\
97.21	0\\
97.22	0\\
97.23	0\\
97.24	0\\
97.25	0\\
97.26	0\\
97.27	0\\
97.28	0\\
97.29	0\\
97.3	0\\
97.31	0\\
97.32	0\\
97.33	0\\
97.34	0\\
97.35	0\\
97.36	0\\
97.37	0\\
97.38	0\\
97.39	0\\
97.4	0\\
97.41	0\\
97.42	0\\
97.43	0\\
97.44	0\\
97.45	0\\
97.46	0\\
97.47	0\\
97.48	0\\
97.49	0\\
97.5	0\\
97.51	0\\
97.52	0\\
97.53	0\\
97.54	0\\
97.55	0\\
97.56	0\\
97.57	0\\
97.58	0\\
97.59	0\\
97.6	0\\
97.61	0\\
97.62	0\\
97.63	0\\
97.64	0\\
97.65	0\\
97.66	0\\
97.67	0\\
97.68	0\\
97.69	0\\
97.7	0\\
97.71	0\\
97.72	0\\
97.73	0\\
97.74	0\\
97.75	0\\
97.76	0\\
97.77	0\\
97.78	0\\
97.79	0\\
97.8	0\\
97.81	0\\
97.82	0\\
97.83	0\\
97.84	0\\
97.85	0\\
97.86	0\\
97.87	0\\
97.88	0\\
97.89	0\\
97.9	0\\
97.91	0\\
97.92	0\\
97.93	0\\
97.94	0\\
97.95	0\\
97.96	0\\
97.97	0\\
97.98	0\\
97.99	0\\
98	0\\
98.01	0\\
98.02	0\\
98.03	0\\
98.04	0\\
98.05	0\\
98.06	0\\
98.07	0\\
98.08	0\\
98.09	0\\
98.1	0\\
98.11	0\\
98.12	0\\
98.13	0\\
98.14	0\\
98.15	0\\
98.16	0\\
98.17	0\\
98.18	0\\
98.19	0\\
98.2	0\\
98.21	0\\
98.22	0\\
98.23	0\\
98.24	0\\
98.25	0\\
98.26	0\\
98.27	0\\
98.28	0\\
98.29	0\\
98.3	0\\
98.31	0\\
98.32	0\\
98.33	0\\
98.34	0\\
98.35	0\\
98.36	0\\
98.37	0\\
98.38	0\\
98.39	0\\
98.4	0\\
98.41	0\\
98.42	0\\
98.43	0\\
98.44	0\\
98.45	0\\
98.46	0\\
98.47	0\\
98.48	0\\
98.49	0\\
98.5	0\\
98.51	0\\
98.52	0\\
98.53	0\\
98.54	0\\
98.55	0\\
98.56	0\\
98.57	0\\
98.58	0\\
98.59	0\\
98.6	0\\
98.61	0\\
98.62	0\\
98.63	0\\
98.64	0\\
98.65	0\\
98.66	0\\
98.67	0\\
98.68	0\\
98.69	0\\
98.7	0\\
98.71	0\\
98.72	6.00597942677708e-05\\
98.73	0.00029476566325861\\
98.74	0.000531361831572298\\
98.75	0.000769878026065637\\
98.76	0.00101034490737443\\
98.77	0.00125279410301949\\
98.78	0.0014972581613695\\
98.79	0.00174377059821627\\
98.8	0.00199236602442002\\
98.81	0.00224301754052945\\
98.82	0.00249575392569249\\
98.83	0.00259299473069209\\
98.84	0.00263207067604759\\
98.85	0.00267125466607201\\
98.86	0.00271053515451244\\
98.87	0.00274989991922517\\
98.88	0.00278933603815415\\
98.89	0.0028288298557744\\
98.9	0.00286836694777702\\
98.91	0.00290795367839173\\
98.92	0.00294761706011325\\
98.93	0.00298734244848259\\
98.94	0.00302711435158348\\
98.95	0.00306691638794276\\
98.96	0.00310682647912407\\
98.97	0.00314709012468934\\
98.98	0.00318771077238027\\
98.99	0.00322869193263868\\
99	0.00327003718149948\\
99.01	0.00331175016347651\\
99.02	0.00335383459458577\\
99.03	0.00339629425946782\\
99.04	0.00343913299375206\\
99.05	0.00348235448270438\\
99.06	0.00352596247377274\\
99.07	0.00356996078207136\\
99.08	0.0036143532932759\\
99.09	0.003659143966672\\
99.1	0.00370433683836476\\
99.11	0.00374993602465781\\
99.12	0.00379594572561081\\
99.13	0.00384237022878453\\
99.14	0.00388921386374265\\
99.15	0.00393648045349505\\
99.16	0.00398417385589169\\
99.17	0.00403229796388142\\
99.18	0.00408085670576604\\
99.19	0.00412985404544918\\
99.2	0.00417929398267937\\
99.21	0.00422918055328676\\
99.22	0.00427951782957623\\
99.23	0.00433030992114294\\
99.24	0.00438156097518348\\
99.25	0.00443327517680532\\
99.26	0.00448545674933452\\
99.27	0.00453810995462083\\
99.28	0.00459123909334014\\
99.29	0.00464484850529341\\
99.3	0.00469894256970183\\
99.31	0.00475352570549745\\
99.32	0.00480860237160862\\
99.33	0.00486417706723978\\
99.34	0.00492025433214455\\
99.35	0.00497683874689156\\
99.36	0.00503393493312218\\
99.37	0.00509154755379893\\
99.38	0.00514968131344361\\
99.39	0.0052083409583641\\
99.4	0.00526753127686883\\
99.41	0.00532725709946772\\
99.42	0.00538752330062317\\
99.43	0.00544833479915421\\
99.44	0.0055096965586415\\
99.45	0.00557161358783625\\
99.46	0.00563409094107324\\
99.47	0.00569713371868781\\
99.48	0.0057607470674359\\
99.49	0.00582493618091646\\
99.5	0.0058897062999978\\
99.51	0.00595506271324757\\
99.52	0.00602101075736671\\
99.53	0.00608755581762727\\
99.54	0.00615470332831403\\
99.55	0.00622245877317017\\
99.56	0.00629082768584681\\
99.57	0.00635981565035658\\
99.58	0.00642942830153121\\
99.59	0.00649967132548313\\
99.6	0.00657055046007118\\
99.61	0.00664207149537041\\
99.62	0.00671424027414604\\
99.63	0.00678706269233151\\
99.64	0.00686054469951087\\
99.65	0.0069346922994052\\
99.66	0.00700951155036349\\
99.67	0.00708500856585769\\
99.68	0.00716118951498209\\
99.69	0.00723806060688493\\
99.7	0.00731562810494498\\
99.71	0.00739389832935691\\
99.72	0.007472877657644\\
99.73	0.00755257252517513\\
99.74	0.0076329894256864\\
99.75	0.00771413491180722\\
99.76	0.00779601559559102\\
99.77	0.00787863814905056\\
99.78	0.00796200930469796\\
99.79	0.00804613585608947\\
99.8	0.00813102465837497\\
99.81	0.00821668262885248\\
99.82	0.00830311674752747\\
99.83	0.00839033405767729\\
99.84	0.00847834166642063\\
99.85	0.00856714674529227\\
99.86	0.00865675653082306\\
99.87	0.00874717832512533\\
99.88	0.00883841949648396\\
99.89	0.00893048747995297\\
99.9	0.00902338977795818\\
99.91	0.00911713396090572\\
99.92	0.0092117276677969\\
99.93	0.00930717860684955\\
99.94	0.009403494556126\\
99.95	0.00950068336416817\\
99.96	0.00959875295063989\\
99.97	0.009697711306977\\
99.98	0.00979756649704546\\
99.99	0.00989832665780802\\
100	0.01\\
};
\addlegendentry{$q=-4$};

\addplot [color=mycolor1,dashed,forget plot]
  table[row sep=crcr]{%
0.01	0\\
0.02	0\\
0.03	0\\
0.04	0\\
0.05	0\\
0.06	0\\
0.07	0\\
0.08	0\\
0.09	0\\
0.1	0\\
0.11	0\\
0.12	0\\
0.13	0\\
0.14	0\\
0.15	0\\
0.16	0\\
0.17	0\\
0.18	0\\
0.19	0\\
0.2	0\\
0.21	0\\
0.22	0\\
0.23	0\\
0.24	0\\
0.25	0\\
0.26	0\\
0.27	0\\
0.28	0\\
0.29	0\\
0.3	0\\
0.31	0\\
0.32	0\\
0.33	0\\
0.34	0\\
0.35	0\\
0.36	0\\
0.37	0\\
0.38	0\\
0.39	0\\
0.4	0\\
0.41	0\\
0.42	0\\
0.43	0\\
0.44	0\\
0.45	0\\
0.46	0\\
0.47	0\\
0.48	0\\
0.49	0\\
0.5	0\\
0.51	0\\
0.52	0\\
0.53	0\\
0.54	0\\
0.55	0\\
0.56	0\\
0.57	0\\
0.58	0\\
0.59	0\\
0.6	0\\
0.61	0\\
0.62	0\\
0.63	0\\
0.64	0\\
0.65	0\\
0.66	0\\
0.67	0\\
0.68	0\\
0.69	0\\
0.7	0\\
0.71	0\\
0.72	0\\
0.73	0\\
0.74	0\\
0.75	0\\
0.76	0\\
0.77	0\\
0.78	0\\
0.79	0\\
0.8	0\\
0.81	0\\
0.82	0\\
0.83	0\\
0.84	0\\
0.85	0\\
0.86	0\\
0.87	0\\
0.88	0\\
0.89	0\\
0.9	0\\
0.91	0\\
0.92	0\\
0.93	0\\
0.94	0\\
0.95	0\\
0.96	0\\
0.97	0\\
0.98	0\\
0.99	0\\
1	0\\
1.01	0\\
1.02	0\\
1.03	0\\
1.04	0\\
1.05	0\\
1.06	0\\
1.07	0\\
1.08	0\\
1.09	0\\
1.1	0\\
1.11	0\\
1.12	0\\
1.13	0\\
1.14	0\\
1.15	0\\
1.16	0\\
1.17	0\\
1.18	0\\
1.19	0\\
1.2	0\\
1.21	0\\
1.22	0\\
1.23	0\\
1.24	0\\
1.25	0\\
1.26	0\\
1.27	0\\
1.28	0\\
1.29	0\\
1.3	0\\
1.31	0\\
1.32	0\\
1.33	0\\
1.34	0\\
1.35	0\\
1.36	0\\
1.37	0\\
1.38	0\\
1.39	0\\
1.4	0\\
1.41	0\\
1.42	0\\
1.43	0\\
1.44	0\\
1.45	0\\
1.46	0\\
1.47	0\\
1.48	0\\
1.49	0\\
1.5	0\\
1.51	0\\
1.52	0\\
1.53	0\\
1.54	0\\
1.55	0\\
1.56	0\\
1.57	0\\
1.58	0\\
1.59	0\\
1.6	0\\
1.61	0\\
1.62	0\\
1.63	0\\
1.64	0\\
1.65	0\\
1.66	0\\
1.67	0\\
1.68	0\\
1.69	0\\
1.7	0\\
1.71	0\\
1.72	0\\
1.73	0\\
1.74	0\\
1.75	0\\
1.76	0\\
1.77	0\\
1.78	0\\
1.79	0\\
1.8	0\\
1.81	0\\
1.82	0\\
1.83	0\\
1.84	0\\
1.85	0\\
1.86	0\\
1.87	0\\
1.88	0\\
1.89	0\\
1.9	0\\
1.91	0\\
1.92	0\\
1.93	0\\
1.94	0\\
1.95	0\\
1.96	0\\
1.97	0\\
1.98	0\\
1.99	0\\
2	0\\
2.01	0\\
2.02	0\\
2.03	0\\
2.04	0\\
2.05	0\\
2.06	0\\
2.07	0\\
2.08	0\\
2.09	0\\
2.1	0\\
2.11	0\\
2.12	0\\
2.13	0\\
2.14	0\\
2.15	0\\
2.16	0\\
2.17	0\\
2.18	0\\
2.19	0\\
2.2	0\\
2.21	0\\
2.22	0\\
2.23	0\\
2.24	0\\
2.25	0\\
2.26	0\\
2.27	0\\
2.28	0\\
2.29	0\\
2.3	0\\
2.31	0\\
2.32	0\\
2.33	0\\
2.34	0\\
2.35	0\\
2.36	0\\
2.37	0\\
2.38	0\\
2.39	0\\
2.4	0\\
2.41	0\\
2.42	0\\
2.43	0\\
2.44	0\\
2.45	0\\
2.46	0\\
2.47	0\\
2.48	0\\
2.49	0\\
2.5	0\\
2.51	0\\
2.52	0\\
2.53	0\\
2.54	0\\
2.55	0\\
2.56	0\\
2.57	0\\
2.58	0\\
2.59	0\\
2.6	0\\
2.61	0\\
2.62	0\\
2.63	0\\
2.64	0\\
2.65	0\\
2.66	0\\
2.67	0\\
2.68	0\\
2.69	0\\
2.7	0\\
2.71	0\\
2.72	0\\
2.73	0\\
2.74	0\\
2.75	0\\
2.76	0\\
2.77	0\\
2.78	0\\
2.79	0\\
2.8	0\\
2.81	0\\
2.82	0\\
2.83	0\\
2.84	0\\
2.85	0\\
2.86	0\\
2.87	0\\
2.88	0\\
2.89	0\\
2.9	0\\
2.91	0\\
2.92	0\\
2.93	0\\
2.94	0\\
2.95	0\\
2.96	0\\
2.97	0\\
2.98	0\\
2.99	0\\
3	0\\
3.01	0\\
3.02	0\\
3.03	0\\
3.04	0\\
3.05	0\\
3.06	0\\
3.07	0\\
3.08	0\\
3.09	0\\
3.1	0\\
3.11	0\\
3.12	0\\
3.13	0\\
3.14	0\\
3.15	0\\
3.16	0\\
3.17	0\\
3.18	0\\
3.19	0\\
3.2	0\\
3.21	0\\
3.22	0\\
3.23	0\\
3.24	0\\
3.25	0\\
3.26	0\\
3.27	0\\
3.28	0\\
3.29	0\\
3.3	0\\
3.31	0\\
3.32	0\\
3.33	0\\
3.34	0\\
3.35	0\\
3.36	0\\
3.37	0\\
3.38	0\\
3.39	0\\
3.4	0\\
3.41	0\\
3.42	0\\
3.43	0\\
3.44	0\\
3.45	0\\
3.46	0\\
3.47	0\\
3.48	0\\
3.49	0\\
3.5	0\\
3.51	0\\
3.52	0\\
3.53	0\\
3.54	0\\
3.55	0\\
3.56	0\\
3.57	0\\
3.58	0\\
3.59	0\\
3.6	0\\
3.61	0\\
3.62	0\\
3.63	0\\
3.64	0\\
3.65	0\\
3.66	0\\
3.67	0\\
3.68	0\\
3.69	0\\
3.7	0\\
3.71	0\\
3.72	0\\
3.73	0\\
3.74	0\\
3.75	0\\
3.76	0\\
3.77	0\\
3.78	0\\
3.79	0\\
3.8	0\\
3.81	0\\
3.82	0\\
3.83	0\\
3.84	0\\
3.85	0\\
3.86	0\\
3.87	0\\
3.88	0\\
3.89	0\\
3.9	0\\
3.91	0\\
3.92	0\\
3.93	0\\
3.94	0\\
3.95	0\\
3.96	0\\
3.97	0\\
3.98	0\\
3.99	0\\
4	0\\
4.01	0\\
4.02	0\\
4.03	0\\
4.04	0\\
4.05	0\\
4.06	0\\
4.07	0\\
4.08	0\\
4.09	0\\
4.1	0\\
4.11	0\\
4.12	0\\
4.13	0\\
4.14	0\\
4.15	0\\
4.16	0\\
4.17	0\\
4.18	0\\
4.19	0\\
4.2	0\\
4.21	0\\
4.22	0\\
4.23	0\\
4.24	0\\
4.25	0\\
4.26	0\\
4.27	0\\
4.28	0\\
4.29	0\\
4.3	0\\
4.31	0\\
4.32	0\\
4.33	0\\
4.34	0\\
4.35	0\\
4.36	0\\
4.37	0\\
4.38	0\\
4.39	0\\
4.4	0\\
4.41	0\\
4.42	0\\
4.43	0\\
4.44	0\\
4.45	0\\
4.46	0\\
4.47	0\\
4.48	0\\
4.49	0\\
4.5	0\\
4.51	0\\
4.52	0\\
4.53	0\\
4.54	0\\
4.55	0\\
4.56	0\\
4.57	0\\
4.58	0\\
4.59	0\\
4.6	0\\
4.61	0\\
4.62	0\\
4.63	0\\
4.64	0\\
4.65	0\\
4.66	0\\
4.67	0\\
4.68	0\\
4.69	0\\
4.7	0\\
4.71	0\\
4.72	0\\
4.73	0\\
4.74	0\\
4.75	0\\
4.76	0\\
4.77	0\\
4.78	0\\
4.79	0\\
4.8	0\\
4.81	0\\
4.82	0\\
4.83	0\\
4.84	0\\
4.85	0\\
4.86	0\\
4.87	0\\
4.88	0\\
4.89	0\\
4.9	0\\
4.91	0\\
4.92	0\\
4.93	0\\
4.94	0\\
4.95	0\\
4.96	0\\
4.97	0\\
4.98	0\\
4.99	0\\
5	0\\
5.01	0\\
5.02	0\\
5.03	0\\
5.04	0\\
5.05	0\\
5.06	0\\
5.07	0\\
5.08	0\\
5.09	0\\
5.1	0\\
5.11	0\\
5.12	0\\
5.13	0\\
5.14	0\\
5.15	0\\
5.16	0\\
5.17	0\\
5.18	0\\
5.19	0\\
5.2	0\\
5.21	0\\
5.22	0\\
5.23	0\\
5.24	0\\
5.25	0\\
5.26	0\\
5.27	0\\
5.28	0\\
5.29	0\\
5.3	0\\
5.31	0\\
5.32	0\\
5.33	0\\
5.34	0\\
5.35	0\\
5.36	0\\
5.37	0\\
5.38	0\\
5.39	0\\
5.4	0\\
5.41	0\\
5.42	0\\
5.43	0\\
5.44	0\\
5.45	0\\
5.46	0\\
5.47	0\\
5.48	0\\
5.49	0\\
5.5	0\\
5.51	0\\
5.52	0\\
5.53	0\\
5.54	0\\
5.55	0\\
5.56	0\\
5.57	0\\
5.58	0\\
5.59	0\\
5.6	0\\
5.61	0\\
5.62	0\\
5.63	0\\
5.64	0\\
5.65	0\\
5.66	0\\
5.67	0\\
5.68	0\\
5.69	0\\
5.7	0\\
5.71	0\\
5.72	0\\
5.73	0\\
5.74	0\\
5.75	0\\
5.76	0\\
5.77	0\\
5.78	0\\
5.79	0\\
5.8	0\\
5.81	0\\
5.82	0\\
5.83	0\\
5.84	0\\
5.85	0\\
5.86	0\\
5.87	0\\
5.88	0\\
5.89	0\\
5.9	0\\
5.91	0\\
5.92	0\\
5.93	0\\
5.94	0\\
5.95	0\\
5.96	0\\
5.97	0\\
5.98	0\\
5.99	0\\
6	0\\
6.01	0\\
6.02	0\\
6.03	0\\
6.04	0\\
6.05	0\\
6.06	0\\
6.07	0\\
6.08	0\\
6.09	0\\
6.1	0\\
6.11	0\\
6.12	0\\
6.13	0\\
6.14	0\\
6.15	0\\
6.16	0\\
6.17	0\\
6.18	0\\
6.19	0\\
6.2	0\\
6.21	0\\
6.22	0\\
6.23	0\\
6.24	0\\
6.25	0\\
6.26	0\\
6.27	0\\
6.28	0\\
6.29	0\\
6.3	0\\
6.31	0\\
6.32	0\\
6.33	0\\
6.34	0\\
6.35	0\\
6.36	0\\
6.37	0\\
6.38	0\\
6.39	0\\
6.4	0\\
6.41	0\\
6.42	0\\
6.43	0\\
6.44	0\\
6.45	0\\
6.46	0\\
6.47	0\\
6.48	0\\
6.49	0\\
6.5	0\\
6.51	0\\
6.52	0\\
6.53	0\\
6.54	0\\
6.55	0\\
6.56	0\\
6.57	0\\
6.58	0\\
6.59	0\\
6.6	0\\
6.61	0\\
6.62	0\\
6.63	0\\
6.64	0\\
6.65	0\\
6.66	0\\
6.67	0\\
6.68	0\\
6.69	0\\
6.7	0\\
6.71	0\\
6.72	0\\
6.73	0\\
6.74	0\\
6.75	0\\
6.76	0\\
6.77	0\\
6.78	0\\
6.79	0\\
6.8	0\\
6.81	0\\
6.82	0\\
6.83	0\\
6.84	0\\
6.85	0\\
6.86	0\\
6.87	0\\
6.88	0\\
6.89	0\\
6.9	0\\
6.91	0\\
6.92	0\\
6.93	0\\
6.94	0\\
6.95	0\\
6.96	0\\
6.97	0\\
6.98	0\\
6.99	0\\
7	0\\
7.01	0\\
7.02	0\\
7.03	0\\
7.04	0\\
7.05	0\\
7.06	0\\
7.07	0\\
7.08	0\\
7.09	0\\
7.1	0\\
7.11	0\\
7.12	0\\
7.13	0\\
7.14	0\\
7.15	0\\
7.16	0\\
7.17	0\\
7.18	0\\
7.19	0\\
7.2	0\\
7.21	0\\
7.22	0\\
7.23	0\\
7.24	0\\
7.25	0\\
7.26	0\\
7.27	0\\
7.28	0\\
7.29	0\\
7.3	0\\
7.31	0\\
7.32	0\\
7.33	0\\
7.34	0\\
7.35	0\\
7.36	0\\
7.37	0\\
7.38	0\\
7.39	0\\
7.4	0\\
7.41	0\\
7.42	0\\
7.43	0\\
7.44	0\\
7.45	0\\
7.46	0\\
7.47	0\\
7.48	0\\
7.49	0\\
7.5	0\\
7.51	0\\
7.52	0\\
7.53	0\\
7.54	0\\
7.55	0\\
7.56	0\\
7.57	0\\
7.58	0\\
7.59	0\\
7.6	0\\
7.61	0\\
7.62	0\\
7.63	0\\
7.64	0\\
7.65	0\\
7.66	0\\
7.67	0\\
7.68	0\\
7.69	0\\
7.7	0\\
7.71	0\\
7.72	0\\
7.73	0\\
7.74	0\\
7.75	0\\
7.76	0\\
7.77	0\\
7.78	0\\
7.79	0\\
7.8	0\\
7.81	0\\
7.82	0\\
7.83	0\\
7.84	0\\
7.85	0\\
7.86	0\\
7.87	0\\
7.88	0\\
7.89	0\\
7.9	0\\
7.91	0\\
7.92	0\\
7.93	0\\
7.94	0\\
7.95	0\\
7.96	0\\
7.97	0\\
7.98	0\\
7.99	0\\
8	0\\
8.01	0\\
8.02	0\\
8.03	0\\
8.04	0\\
8.05	0\\
8.06	0\\
8.07	0\\
8.08	0\\
8.09	0\\
8.1	0\\
8.11	0\\
8.12	0\\
8.13	0\\
8.14	0\\
8.15	0\\
8.16	0\\
8.17	0\\
8.18	0\\
8.19	0\\
8.2	0\\
8.21	0\\
8.22	0\\
8.23	0\\
8.24	0\\
8.25	0\\
8.26	0\\
8.27	0\\
8.28	0\\
8.29	0\\
8.3	0\\
8.31	0\\
8.32	0\\
8.33	0\\
8.34	0\\
8.35	0\\
8.36	0\\
8.37	0\\
8.38	0\\
8.39	0\\
8.4	0\\
8.41	0\\
8.42	0\\
8.43	0\\
8.44	0\\
8.45	0\\
8.46	0\\
8.47	0\\
8.48	0\\
8.49	0\\
8.5	0\\
8.51	0\\
8.52	0\\
8.53	0\\
8.54	0\\
8.55	0\\
8.56	0\\
8.57	0\\
8.58	0\\
8.59	0\\
8.6	0\\
8.61	0\\
8.62	0\\
8.63	0\\
8.64	0\\
8.65	0\\
8.66	0\\
8.67	0\\
8.68	0\\
8.69	0\\
8.7	0\\
8.71	0\\
8.72	0\\
8.73	0\\
8.74	0\\
8.75	0\\
8.76	0\\
8.77	0\\
8.78	0\\
8.79	0\\
8.8	0\\
8.81	0\\
8.82	0\\
8.83	0\\
8.84	0\\
8.85	0\\
8.86	0\\
8.87	0\\
8.88	0\\
8.89	0\\
8.9	0\\
8.91	0\\
8.92	0\\
8.93	0\\
8.94	0\\
8.95	0\\
8.96	0\\
8.97	0\\
8.98	0\\
8.99	0\\
9	0\\
9.01	0\\
9.02	0\\
9.03	0\\
9.04	0\\
9.05	0\\
9.06	0\\
9.07	0\\
9.08	0\\
9.09	0\\
9.1	0\\
9.11	0\\
9.12	0\\
9.13	0\\
9.14	0\\
9.15	0\\
9.16	0\\
9.17	0\\
9.18	0\\
9.19	0\\
9.2	0\\
9.21	0\\
9.22	0\\
9.23	0\\
9.24	0\\
9.25	0\\
9.26	0\\
9.27	0\\
9.28	0\\
9.29	0\\
9.3	0\\
9.31	0\\
9.32	0\\
9.33	0\\
9.34	0\\
9.35	0\\
9.36	0\\
9.37	0\\
9.38	0\\
9.39	0\\
9.4	0\\
9.41	0\\
9.42	0\\
9.43	0\\
9.44	0\\
9.45	0\\
9.46	0\\
9.47	0\\
9.48	0\\
9.49	0\\
9.5	0\\
9.51	0\\
9.52	0\\
9.53	0\\
9.54	0\\
9.55	0\\
9.56	0\\
9.57	0\\
9.58	0\\
9.59	0\\
9.6	0\\
9.61	0\\
9.62	0\\
9.63	0\\
9.64	0\\
9.65	0\\
9.66	0\\
9.67	0\\
9.68	0\\
9.69	0\\
9.7	0\\
9.71	0\\
9.72	0\\
9.73	0\\
9.74	0\\
9.75	0\\
9.76	0\\
9.77	0\\
9.78	0\\
9.79	0\\
9.8	0\\
9.81	0\\
9.82	0\\
9.83	0\\
9.84	0\\
9.85	0\\
9.86	0\\
9.87	0\\
9.88	0\\
9.89	0\\
9.9	0\\
9.91	0\\
9.92	0\\
9.93	0\\
9.94	0\\
9.95	0\\
9.96	0\\
9.97	0\\
9.98	0\\
9.99	0\\
10	0\\
10.01	0\\
10.02	0\\
10.03	0\\
10.04	0\\
10.05	0\\
10.06	0\\
10.07	0\\
10.08	0\\
10.09	0\\
10.1	0\\
10.11	0\\
10.12	0\\
10.13	0\\
10.14	0\\
10.15	0\\
10.16	0\\
10.17	0\\
10.18	0\\
10.19	0\\
10.2	0\\
10.21	0\\
10.22	0\\
10.23	0\\
10.24	0\\
10.25	0\\
10.26	0\\
10.27	0\\
10.28	0\\
10.29	0\\
10.3	0\\
10.31	0\\
10.32	0\\
10.33	0\\
10.34	0\\
10.35	0\\
10.36	0\\
10.37	0\\
10.38	0\\
10.39	0\\
10.4	0\\
10.41	0\\
10.42	0\\
10.43	0\\
10.44	0\\
10.45	0\\
10.46	0\\
10.47	0\\
10.48	0\\
10.49	0\\
10.5	0\\
10.51	0\\
10.52	0\\
10.53	0\\
10.54	0\\
10.55	0\\
10.56	0\\
10.57	0\\
10.58	0\\
10.59	0\\
10.6	0\\
10.61	0\\
10.62	0\\
10.63	0\\
10.64	0\\
10.65	0\\
10.66	0\\
10.67	0\\
10.68	0\\
10.69	0\\
10.7	0\\
10.71	0\\
10.72	0\\
10.73	0\\
10.74	0\\
10.75	0\\
10.76	0\\
10.77	0\\
10.78	0\\
10.79	0\\
10.8	0\\
10.81	0\\
10.82	0\\
10.83	0\\
10.84	0\\
10.85	0\\
10.86	0\\
10.87	0\\
10.88	0\\
10.89	0\\
10.9	0\\
10.91	0\\
10.92	0\\
10.93	0\\
10.94	0\\
10.95	0\\
10.96	0\\
10.97	0\\
10.98	0\\
10.99	0\\
11	0\\
11.01	0\\
11.02	0\\
11.03	0\\
11.04	0\\
11.05	0\\
11.06	0\\
11.07	0\\
11.08	0\\
11.09	0\\
11.1	0\\
11.11	0\\
11.12	0\\
11.13	0\\
11.14	0\\
11.15	0\\
11.16	0\\
11.17	0\\
11.18	0\\
11.19	0\\
11.2	0\\
11.21	0\\
11.22	0\\
11.23	0\\
11.24	0\\
11.25	0\\
11.26	0\\
11.27	0\\
11.28	0\\
11.29	0\\
11.3	0\\
11.31	0\\
11.32	0\\
11.33	0\\
11.34	0\\
11.35	0\\
11.36	0\\
11.37	0\\
11.38	0\\
11.39	0\\
11.4	0\\
11.41	0\\
11.42	0\\
11.43	0\\
11.44	0\\
11.45	0\\
11.46	0\\
11.47	0\\
11.48	0\\
11.49	0\\
11.5	0\\
11.51	0\\
11.52	0\\
11.53	0\\
11.54	0\\
11.55	0\\
11.56	0\\
11.57	0\\
11.58	0\\
11.59	0\\
11.6	0\\
11.61	0\\
11.62	0\\
11.63	0\\
11.64	0\\
11.65	0\\
11.66	0\\
11.67	0\\
11.68	0\\
11.69	0\\
11.7	0\\
11.71	0\\
11.72	0\\
11.73	0\\
11.74	0\\
11.75	0\\
11.76	0\\
11.77	0\\
11.78	0\\
11.79	0\\
11.8	0\\
11.81	0\\
11.82	0\\
11.83	0\\
11.84	0\\
11.85	0\\
11.86	0\\
11.87	0\\
11.88	0\\
11.89	0\\
11.9	0\\
11.91	0\\
11.92	0\\
11.93	0\\
11.94	0\\
11.95	0\\
11.96	0\\
11.97	0\\
11.98	0\\
11.99	0\\
12	0\\
12.01	0\\
12.02	0\\
12.03	0\\
12.04	0\\
12.05	0\\
12.06	0\\
12.07	0\\
12.08	0\\
12.09	0\\
12.1	0\\
12.11	0\\
12.12	0\\
12.13	0\\
12.14	0\\
12.15	0\\
12.16	0\\
12.17	0\\
12.18	0\\
12.19	0\\
12.2	0\\
12.21	0\\
12.22	0\\
12.23	0\\
12.24	0\\
12.25	0\\
12.26	0\\
12.27	0\\
12.28	0\\
12.29	0\\
12.3	0\\
12.31	0\\
12.32	0\\
12.33	0\\
12.34	0\\
12.35	0\\
12.36	0\\
12.37	0\\
12.38	0\\
12.39	0\\
12.4	0\\
12.41	0\\
12.42	0\\
12.43	0\\
12.44	0\\
12.45	0\\
12.46	0\\
12.47	0\\
12.48	0\\
12.49	0\\
12.5	0\\
12.51	0\\
12.52	0\\
12.53	0\\
12.54	0\\
12.55	0\\
12.56	0\\
12.57	0\\
12.58	0\\
12.59	0\\
12.6	0\\
12.61	0\\
12.62	0\\
12.63	0\\
12.64	0\\
12.65	0\\
12.66	0\\
12.67	0\\
12.68	0\\
12.69	0\\
12.7	0\\
12.71	0\\
12.72	0\\
12.73	0\\
12.74	0\\
12.75	0\\
12.76	0\\
12.77	0\\
12.78	0\\
12.79	0\\
12.8	0\\
12.81	0\\
12.82	0\\
12.83	0\\
12.84	0\\
12.85	0\\
12.86	0\\
12.87	0\\
12.88	0\\
12.89	0\\
12.9	0\\
12.91	0\\
12.92	0\\
12.93	0\\
12.94	0\\
12.95	0\\
12.96	0\\
12.97	0\\
12.98	0\\
12.99	0\\
13	0\\
13.01	0\\
13.02	0\\
13.03	0\\
13.04	0\\
13.05	0\\
13.06	0\\
13.07	0\\
13.08	0\\
13.09	0\\
13.1	0\\
13.11	0\\
13.12	0\\
13.13	0\\
13.14	0\\
13.15	0\\
13.16	0\\
13.17	0\\
13.18	0\\
13.19	0\\
13.2	0\\
13.21	0\\
13.22	0\\
13.23	0\\
13.24	0\\
13.25	0\\
13.26	0\\
13.27	0\\
13.28	0\\
13.29	0\\
13.3	0\\
13.31	0\\
13.32	0\\
13.33	0\\
13.34	0\\
13.35	0\\
13.36	0\\
13.37	0\\
13.38	0\\
13.39	0\\
13.4	0\\
13.41	0\\
13.42	0\\
13.43	0\\
13.44	0\\
13.45	0\\
13.46	0\\
13.47	0\\
13.48	0\\
13.49	0\\
13.5	0\\
13.51	0\\
13.52	0\\
13.53	0\\
13.54	0\\
13.55	0\\
13.56	0\\
13.57	0\\
13.58	0\\
13.59	0\\
13.6	0\\
13.61	0\\
13.62	0\\
13.63	0\\
13.64	0\\
13.65	0\\
13.66	0\\
13.67	0\\
13.68	0\\
13.69	0\\
13.7	0\\
13.71	0\\
13.72	0\\
13.73	0\\
13.74	0\\
13.75	0\\
13.76	0\\
13.77	0\\
13.78	0\\
13.79	0\\
13.8	0\\
13.81	0\\
13.82	0\\
13.83	0\\
13.84	0\\
13.85	0\\
13.86	0\\
13.87	0\\
13.88	0\\
13.89	0\\
13.9	0\\
13.91	0\\
13.92	0\\
13.93	0\\
13.94	0\\
13.95	0\\
13.96	0\\
13.97	0\\
13.98	0\\
13.99	0\\
14	0\\
14.01	0\\
14.02	0\\
14.03	0\\
14.04	0\\
14.05	0\\
14.06	0\\
14.07	0\\
14.08	0\\
14.09	0\\
14.1	0\\
14.11	0\\
14.12	0\\
14.13	0\\
14.14	0\\
14.15	0\\
14.16	0\\
14.17	0\\
14.18	0\\
14.19	0\\
14.2	0\\
14.21	0\\
14.22	0\\
14.23	0\\
14.24	0\\
14.25	0\\
14.26	0\\
14.27	0\\
14.28	0\\
14.29	0\\
14.3	0\\
14.31	0\\
14.32	0\\
14.33	0\\
14.34	0\\
14.35	0\\
14.36	0\\
14.37	0\\
14.38	0\\
14.39	0\\
14.4	0\\
14.41	0\\
14.42	0\\
14.43	0\\
14.44	0\\
14.45	0\\
14.46	0\\
14.47	0\\
14.48	0\\
14.49	0\\
14.5	0\\
14.51	0\\
14.52	0\\
14.53	0\\
14.54	0\\
14.55	0\\
14.56	0\\
14.57	0\\
14.58	0\\
14.59	0\\
14.6	0\\
14.61	0\\
14.62	0\\
14.63	0\\
14.64	0\\
14.65	0\\
14.66	0\\
14.67	0\\
14.68	0\\
14.69	0\\
14.7	0\\
14.71	0\\
14.72	0\\
14.73	0\\
14.74	0\\
14.75	0\\
14.76	0\\
14.77	0\\
14.78	0\\
14.79	0\\
14.8	0\\
14.81	0\\
14.82	0\\
14.83	0\\
14.84	0\\
14.85	0\\
14.86	0\\
14.87	0\\
14.88	0\\
14.89	0\\
14.9	0\\
14.91	0\\
14.92	0\\
14.93	0\\
14.94	0\\
14.95	0\\
14.96	0\\
14.97	0\\
14.98	0\\
14.99	0\\
15	0\\
15.01	0\\
15.02	0\\
15.03	0\\
15.04	0\\
15.05	0\\
15.06	0\\
15.07	0\\
15.08	0\\
15.09	0\\
15.1	0\\
15.11	0\\
15.12	0\\
15.13	0\\
15.14	0\\
15.15	0\\
15.16	0\\
15.17	0\\
15.18	0\\
15.19	0\\
15.2	0\\
15.21	0\\
15.22	0\\
15.23	0\\
15.24	0\\
15.25	0\\
15.26	0\\
15.27	0\\
15.28	0\\
15.29	0\\
15.3	0\\
15.31	0\\
15.32	0\\
15.33	0\\
15.34	0\\
15.35	0\\
15.36	0\\
15.37	0\\
15.38	0\\
15.39	0\\
15.4	0\\
15.41	0\\
15.42	0\\
15.43	0\\
15.44	0\\
15.45	0\\
15.46	0\\
15.47	0\\
15.48	0\\
15.49	0\\
15.5	0\\
15.51	0\\
15.52	0\\
15.53	0\\
15.54	0\\
15.55	0\\
15.56	0\\
15.57	0\\
15.58	0\\
15.59	0\\
15.6	0\\
15.61	0\\
15.62	0\\
15.63	0\\
15.64	0\\
15.65	0\\
15.66	0\\
15.67	0\\
15.68	0\\
15.69	0\\
15.7	0\\
15.71	0\\
15.72	0\\
15.73	0\\
15.74	0\\
15.75	0\\
15.76	0\\
15.77	0\\
15.78	0\\
15.79	0\\
15.8	0\\
15.81	0\\
15.82	0\\
15.83	0\\
15.84	0\\
15.85	0\\
15.86	0\\
15.87	0\\
15.88	0\\
15.89	0\\
15.9	0\\
15.91	0\\
15.92	0\\
15.93	0\\
15.94	0\\
15.95	0\\
15.96	0\\
15.97	0\\
15.98	0\\
15.99	0\\
16	0\\
16.01	0\\
16.02	0\\
16.03	0\\
16.04	0\\
16.05	0\\
16.06	0\\
16.07	0\\
16.08	0\\
16.09	0\\
16.1	0\\
16.11	0\\
16.12	0\\
16.13	0\\
16.14	0\\
16.15	0\\
16.16	0\\
16.17	0\\
16.18	0\\
16.19	0\\
16.2	0\\
16.21	0\\
16.22	0\\
16.23	0\\
16.24	0\\
16.25	0\\
16.26	0\\
16.27	0\\
16.28	0\\
16.29	0\\
16.3	0\\
16.31	0\\
16.32	0\\
16.33	0\\
16.34	0\\
16.35	0\\
16.36	0\\
16.37	0\\
16.38	0\\
16.39	0\\
16.4	0\\
16.41	0\\
16.42	0\\
16.43	0\\
16.44	0\\
16.45	0\\
16.46	0\\
16.47	0\\
16.48	0\\
16.49	0\\
16.5	0\\
16.51	0\\
16.52	0\\
16.53	0\\
16.54	0\\
16.55	0\\
16.56	0\\
16.57	0\\
16.58	0\\
16.59	0\\
16.6	0\\
16.61	0\\
16.62	0\\
16.63	0\\
16.64	0\\
16.65	0\\
16.66	0\\
16.67	0\\
16.68	0\\
16.69	0\\
16.7	0\\
16.71	0\\
16.72	0\\
16.73	0\\
16.74	0\\
16.75	0\\
16.76	0\\
16.77	0\\
16.78	0\\
16.79	0\\
16.8	0\\
16.81	0\\
16.82	0\\
16.83	0\\
16.84	0\\
16.85	0\\
16.86	0\\
16.87	0\\
16.88	0\\
16.89	0\\
16.9	0\\
16.91	0\\
16.92	0\\
16.93	0\\
16.94	0\\
16.95	0\\
16.96	0\\
16.97	0\\
16.98	0\\
16.99	0\\
17	0\\
17.01	0\\
17.02	0\\
17.03	0\\
17.04	0\\
17.05	0\\
17.06	0\\
17.07	0\\
17.08	0\\
17.09	0\\
17.1	0\\
17.11	0\\
17.12	0\\
17.13	0\\
17.14	0\\
17.15	0\\
17.16	0\\
17.17	0\\
17.18	0\\
17.19	0\\
17.2	0\\
17.21	0\\
17.22	0\\
17.23	0\\
17.24	0\\
17.25	0\\
17.26	0\\
17.27	0\\
17.28	0\\
17.29	0\\
17.3	0\\
17.31	0\\
17.32	0\\
17.33	0\\
17.34	0\\
17.35	0\\
17.36	0\\
17.37	0\\
17.38	0\\
17.39	0\\
17.4	0\\
17.41	0\\
17.42	0\\
17.43	0\\
17.44	0\\
17.45	0\\
17.46	0\\
17.47	0\\
17.48	0\\
17.49	0\\
17.5	0\\
17.51	0\\
17.52	0\\
17.53	0\\
17.54	0\\
17.55	0\\
17.56	0\\
17.57	0\\
17.58	0\\
17.59	0\\
17.6	0\\
17.61	0\\
17.62	0\\
17.63	0\\
17.64	0\\
17.65	0\\
17.66	0\\
17.67	0\\
17.68	0\\
17.69	0\\
17.7	0\\
17.71	0\\
17.72	0\\
17.73	0\\
17.74	0\\
17.75	0\\
17.76	0\\
17.77	0\\
17.78	0\\
17.79	0\\
17.8	0\\
17.81	0\\
17.82	0\\
17.83	0\\
17.84	0\\
17.85	0\\
17.86	0\\
17.87	0\\
17.88	0\\
17.89	0\\
17.9	0\\
17.91	0\\
17.92	0\\
17.93	0\\
17.94	0\\
17.95	0\\
17.96	0\\
17.97	0\\
17.98	0\\
17.99	0\\
18	0\\
18.01	0\\
18.02	0\\
18.03	0\\
18.04	0\\
18.05	0\\
18.06	0\\
18.07	0\\
18.08	0\\
18.09	0\\
18.1	0\\
18.11	0\\
18.12	0\\
18.13	0\\
18.14	0\\
18.15	0\\
18.16	0\\
18.17	0\\
18.18	0\\
18.19	0\\
18.2	0\\
18.21	0\\
18.22	0\\
18.23	0\\
18.24	0\\
18.25	0\\
18.26	0\\
18.27	0\\
18.28	0\\
18.29	0\\
18.3	0\\
18.31	0\\
18.32	0\\
18.33	0\\
18.34	0\\
18.35	0\\
18.36	0\\
18.37	0\\
18.38	0\\
18.39	0\\
18.4	0\\
18.41	0\\
18.42	0\\
18.43	0\\
18.44	0\\
18.45	0\\
18.46	0\\
18.47	0\\
18.48	0\\
18.49	0\\
18.5	0\\
18.51	0\\
18.52	0\\
18.53	0\\
18.54	0\\
18.55	0\\
18.56	0\\
18.57	0\\
18.58	0\\
18.59	0\\
18.6	0\\
18.61	0\\
18.62	0\\
18.63	0\\
18.64	0\\
18.65	0\\
18.66	0\\
18.67	0\\
18.68	0\\
18.69	0\\
18.7	0\\
18.71	0\\
18.72	0\\
18.73	0\\
18.74	0\\
18.75	0\\
18.76	0\\
18.77	0\\
18.78	0\\
18.79	0\\
18.8	0\\
18.81	0\\
18.82	0\\
18.83	0\\
18.84	0\\
18.85	0\\
18.86	0\\
18.87	0\\
18.88	0\\
18.89	0\\
18.9	0\\
18.91	0\\
18.92	0\\
18.93	0\\
18.94	0\\
18.95	0\\
18.96	0\\
18.97	0\\
18.98	0\\
18.99	0\\
19	0\\
19.01	0\\
19.02	0\\
19.03	0\\
19.04	0\\
19.05	0\\
19.06	0\\
19.07	0\\
19.08	0\\
19.09	0\\
19.1	0\\
19.11	0\\
19.12	0\\
19.13	0\\
19.14	0\\
19.15	0\\
19.16	0\\
19.17	0\\
19.18	0\\
19.19	0\\
19.2	0\\
19.21	0\\
19.22	0\\
19.23	0\\
19.24	0\\
19.25	0\\
19.26	0\\
19.27	0\\
19.28	0\\
19.29	0\\
19.3	0\\
19.31	0\\
19.32	0\\
19.33	0\\
19.34	0\\
19.35	0\\
19.36	0\\
19.37	0\\
19.38	0\\
19.39	0\\
19.4	0\\
19.41	0\\
19.42	0\\
19.43	0\\
19.44	0\\
19.45	0\\
19.46	0\\
19.47	0\\
19.48	0\\
19.49	0\\
19.5	0\\
19.51	0\\
19.52	0\\
19.53	0\\
19.54	0\\
19.55	0\\
19.56	0\\
19.57	0\\
19.58	0\\
19.59	0\\
19.6	0\\
19.61	0\\
19.62	0\\
19.63	0\\
19.64	0\\
19.65	0\\
19.66	0\\
19.67	0\\
19.68	0\\
19.69	0\\
19.7	0\\
19.71	0\\
19.72	0\\
19.73	0\\
19.74	0\\
19.75	0\\
19.76	0\\
19.77	0\\
19.78	0\\
19.79	0\\
19.8	0\\
19.81	0\\
19.82	0\\
19.83	0\\
19.84	0\\
19.85	0\\
19.86	0\\
19.87	0\\
19.88	0\\
19.89	0\\
19.9	0\\
19.91	0\\
19.92	0\\
19.93	0\\
19.94	0\\
19.95	0\\
19.96	0\\
19.97	0\\
19.98	0\\
19.99	0\\
20	0\\
20.01	0\\
20.02	0\\
20.03	0\\
20.04	0\\
20.05	0\\
20.06	0\\
20.07	0\\
20.08	0\\
20.09	0\\
20.1	0\\
20.11	0\\
20.12	0\\
20.13	0\\
20.14	0\\
20.15	0\\
20.16	0\\
20.17	0\\
20.18	0\\
20.19	0\\
20.2	0\\
20.21	0\\
20.22	0\\
20.23	0\\
20.24	0\\
20.25	0\\
20.26	0\\
20.27	0\\
20.28	0\\
20.29	0\\
20.3	0\\
20.31	0\\
20.32	0\\
20.33	0\\
20.34	0\\
20.35	0\\
20.36	0\\
20.37	0\\
20.38	0\\
20.39	0\\
20.4	0\\
20.41	0\\
20.42	0\\
20.43	0\\
20.44	0\\
20.45	0\\
20.46	0\\
20.47	0\\
20.48	0\\
20.49	0\\
20.5	0\\
20.51	0\\
20.52	0\\
20.53	0\\
20.54	0\\
20.55	0\\
20.56	0\\
20.57	0\\
20.58	0\\
20.59	0\\
20.6	0\\
20.61	0\\
20.62	0\\
20.63	0\\
20.64	0\\
20.65	0\\
20.66	0\\
20.67	0\\
20.68	0\\
20.69	0\\
20.7	0\\
20.71	0\\
20.72	0\\
20.73	0\\
20.74	0\\
20.75	0\\
20.76	0\\
20.77	0\\
20.78	0\\
20.79	0\\
20.8	0\\
20.81	0\\
20.82	0\\
20.83	0\\
20.84	0\\
20.85	0\\
20.86	0\\
20.87	0\\
20.88	0\\
20.89	0\\
20.9	0\\
20.91	0\\
20.92	0\\
20.93	0\\
20.94	0\\
20.95	0\\
20.96	0\\
20.97	0\\
20.98	0\\
20.99	0\\
21	0\\
21.01	0\\
21.02	0\\
21.03	0\\
21.04	0\\
21.05	0\\
21.06	0\\
21.07	0\\
21.08	0\\
21.09	0\\
21.1	0\\
21.11	0\\
21.12	0\\
21.13	0\\
21.14	0\\
21.15	0\\
21.16	0\\
21.17	0\\
21.18	0\\
21.19	0\\
21.2	0\\
21.21	0\\
21.22	0\\
21.23	0\\
21.24	0\\
21.25	0\\
21.26	0\\
21.27	0\\
21.28	0\\
21.29	0\\
21.3	0\\
21.31	0\\
21.32	0\\
21.33	0\\
21.34	0\\
21.35	0\\
21.36	0\\
21.37	0\\
21.38	0\\
21.39	0\\
21.4	0\\
21.41	0\\
21.42	0\\
21.43	0\\
21.44	0\\
21.45	0\\
21.46	0\\
21.47	0\\
21.48	0\\
21.49	0\\
21.5	0\\
21.51	0\\
21.52	0\\
21.53	0\\
21.54	0\\
21.55	0\\
21.56	0\\
21.57	0\\
21.58	0\\
21.59	0\\
21.6	0\\
21.61	0\\
21.62	0\\
21.63	0\\
21.64	0\\
21.65	0\\
21.66	0\\
21.67	0\\
21.68	0\\
21.69	0\\
21.7	0\\
21.71	0\\
21.72	0\\
21.73	0\\
21.74	0\\
21.75	0\\
21.76	0\\
21.77	0\\
21.78	0\\
21.79	0\\
21.8	0\\
21.81	0\\
21.82	0\\
21.83	0\\
21.84	0\\
21.85	0\\
21.86	0\\
21.87	0\\
21.88	0\\
21.89	0\\
21.9	0\\
21.91	0\\
21.92	0\\
21.93	0\\
21.94	0\\
21.95	0\\
21.96	0\\
21.97	0\\
21.98	0\\
21.99	0\\
22	0\\
22.01	0\\
22.02	0\\
22.03	0\\
22.04	0\\
22.05	0\\
22.06	0\\
22.07	0\\
22.08	0\\
22.09	0\\
22.1	0\\
22.11	0\\
22.12	0\\
22.13	0\\
22.14	0\\
22.15	0\\
22.16	0\\
22.17	0\\
22.18	0\\
22.19	0\\
22.2	0\\
22.21	0\\
22.22	0\\
22.23	0\\
22.24	0\\
22.25	0\\
22.26	0\\
22.27	0\\
22.28	0\\
22.29	0\\
22.3	0\\
22.31	0\\
22.32	0\\
22.33	0\\
22.34	0\\
22.35	0\\
22.36	0\\
22.37	0\\
22.38	0\\
22.39	0\\
22.4	0\\
22.41	0\\
22.42	0\\
22.43	0\\
22.44	0\\
22.45	0\\
22.46	0\\
22.47	0\\
22.48	0\\
22.49	0\\
22.5	0\\
22.51	0\\
22.52	0\\
22.53	0\\
22.54	0\\
22.55	0\\
22.56	0\\
22.57	0\\
22.58	0\\
22.59	0\\
22.6	0\\
22.61	0\\
22.62	0\\
22.63	0\\
22.64	0\\
22.65	0\\
22.66	0\\
22.67	0\\
22.68	0\\
22.69	0\\
22.7	0\\
22.71	0\\
22.72	0\\
22.73	0\\
22.74	0\\
22.75	0\\
22.76	0\\
22.77	0\\
22.78	0\\
22.79	0\\
22.8	0\\
22.81	0\\
22.82	0\\
22.83	0\\
22.84	0\\
22.85	0\\
22.86	0\\
22.87	0\\
22.88	0\\
22.89	0\\
22.9	0\\
22.91	0\\
22.92	0\\
22.93	0\\
22.94	0\\
22.95	0\\
22.96	0\\
22.97	0\\
22.98	0\\
22.99	0\\
23	0\\
23.01	0\\
23.02	0\\
23.03	0\\
23.04	0\\
23.05	0\\
23.06	0\\
23.07	0\\
23.08	0\\
23.09	0\\
23.1	0\\
23.11	0\\
23.12	0\\
23.13	0\\
23.14	0\\
23.15	0\\
23.16	0\\
23.17	0\\
23.18	0\\
23.19	0\\
23.2	0\\
23.21	0\\
23.22	0\\
23.23	0\\
23.24	0\\
23.25	0\\
23.26	0\\
23.27	0\\
23.28	0\\
23.29	0\\
23.3	0\\
23.31	0\\
23.32	0\\
23.33	0\\
23.34	0\\
23.35	0\\
23.36	0\\
23.37	0\\
23.38	0\\
23.39	0\\
23.4	0\\
23.41	0\\
23.42	0\\
23.43	0\\
23.44	0\\
23.45	0\\
23.46	0\\
23.47	0\\
23.48	0\\
23.49	0\\
23.5	0\\
23.51	0\\
23.52	0\\
23.53	0\\
23.54	0\\
23.55	0\\
23.56	0\\
23.57	0\\
23.58	0\\
23.59	0\\
23.6	0\\
23.61	0\\
23.62	0\\
23.63	0\\
23.64	0\\
23.65	0\\
23.66	0\\
23.67	0\\
23.68	0\\
23.69	0\\
23.7	0\\
23.71	0\\
23.72	0\\
23.73	0\\
23.74	0\\
23.75	0\\
23.76	0\\
23.77	0\\
23.78	0\\
23.79	0\\
23.8	0\\
23.81	0\\
23.82	0\\
23.83	0\\
23.84	0\\
23.85	0\\
23.86	0\\
23.87	0\\
23.88	0\\
23.89	0\\
23.9	0\\
23.91	0\\
23.92	0\\
23.93	0\\
23.94	0\\
23.95	0\\
23.96	0\\
23.97	0\\
23.98	0\\
23.99	0\\
24	0\\
24.01	0\\
24.02	0\\
24.03	0\\
24.04	0\\
24.05	0\\
24.06	0\\
24.07	0\\
24.08	0\\
24.09	0\\
24.1	0\\
24.11	0\\
24.12	0\\
24.13	0\\
24.14	0\\
24.15	0\\
24.16	0\\
24.17	0\\
24.18	0\\
24.19	0\\
24.2	0\\
24.21	0\\
24.22	0\\
24.23	0\\
24.24	0\\
24.25	0\\
24.26	0\\
24.27	0\\
24.28	0\\
24.29	0\\
24.3	0\\
24.31	0\\
24.32	0\\
24.33	0\\
24.34	0\\
24.35	0\\
24.36	0\\
24.37	0\\
24.38	0\\
24.39	0\\
24.4	0\\
24.41	0\\
24.42	0\\
24.43	0\\
24.44	0\\
24.45	0\\
24.46	0\\
24.47	0\\
24.48	0\\
24.49	0\\
24.5	0\\
24.51	0\\
24.52	0\\
24.53	0\\
24.54	0\\
24.55	0\\
24.56	0\\
24.57	0\\
24.58	0\\
24.59	0\\
24.6	0\\
24.61	0\\
24.62	0\\
24.63	0\\
24.64	0\\
24.65	0\\
24.66	0\\
24.67	0\\
24.68	0\\
24.69	0\\
24.7	0\\
24.71	0\\
24.72	0\\
24.73	0\\
24.74	0\\
24.75	0\\
24.76	0\\
24.77	0\\
24.78	0\\
24.79	0\\
24.8	0\\
24.81	0\\
24.82	0\\
24.83	0\\
24.84	0\\
24.85	0\\
24.86	0\\
24.87	0\\
24.88	0\\
24.89	0\\
24.9	0\\
24.91	0\\
24.92	0\\
24.93	0\\
24.94	0\\
24.95	0\\
24.96	0\\
24.97	0\\
24.98	0\\
24.99	0\\
25	0\\
25.01	0\\
25.02	0\\
25.03	0\\
25.04	0\\
25.05	0\\
25.06	0\\
25.07	0\\
25.08	0\\
25.09	0\\
25.1	0\\
25.11	0\\
25.12	0\\
25.13	0\\
25.14	0\\
25.15	0\\
25.16	0\\
25.17	0\\
25.18	0\\
25.19	0\\
25.2	0\\
25.21	0\\
25.22	0\\
25.23	0\\
25.24	0\\
25.25	0\\
25.26	0\\
25.27	0\\
25.28	0\\
25.29	0\\
25.3	0\\
25.31	0\\
25.32	0\\
25.33	0\\
25.34	0\\
25.35	0\\
25.36	0\\
25.37	0\\
25.38	0\\
25.39	0\\
25.4	0\\
25.41	0\\
25.42	0\\
25.43	0\\
25.44	0\\
25.45	0\\
25.46	0\\
25.47	0\\
25.48	0\\
25.49	0\\
25.5	0\\
25.51	0\\
25.52	0\\
25.53	0\\
25.54	0\\
25.55	0\\
25.56	0\\
25.57	0\\
25.58	0\\
25.59	0\\
25.6	0\\
25.61	0\\
25.62	0\\
25.63	0\\
25.64	0\\
25.65	0\\
25.66	0\\
25.67	0\\
25.68	0\\
25.69	0\\
25.7	0\\
25.71	0\\
25.72	0\\
25.73	0\\
25.74	0\\
25.75	0\\
25.76	0\\
25.77	0\\
25.78	0\\
25.79	0\\
25.8	0\\
25.81	0\\
25.82	0\\
25.83	0\\
25.84	0\\
25.85	0\\
25.86	0\\
25.87	0\\
25.88	0\\
25.89	0\\
25.9	0\\
25.91	0\\
25.92	0\\
25.93	0\\
25.94	0\\
25.95	0\\
25.96	0\\
25.97	0\\
25.98	0\\
25.99	0\\
26	0\\
26.01	0\\
26.02	0\\
26.03	0\\
26.04	0\\
26.05	0\\
26.06	0\\
26.07	0\\
26.08	0\\
26.09	0\\
26.1	0\\
26.11	0\\
26.12	0\\
26.13	0\\
26.14	0\\
26.15	0\\
26.16	0\\
26.17	0\\
26.18	0\\
26.19	0\\
26.2	0\\
26.21	0\\
26.22	0\\
26.23	0\\
26.24	0\\
26.25	0\\
26.26	0\\
26.27	0\\
26.28	0\\
26.29	0\\
26.3	0\\
26.31	0\\
26.32	0\\
26.33	0\\
26.34	0\\
26.35	0\\
26.36	0\\
26.37	0\\
26.38	0\\
26.39	0\\
26.4	0\\
26.41	0\\
26.42	0\\
26.43	0\\
26.44	0\\
26.45	0\\
26.46	0\\
26.47	0\\
26.48	0\\
26.49	0\\
26.5	0\\
26.51	0\\
26.52	0\\
26.53	0\\
26.54	0\\
26.55	0\\
26.56	0\\
26.57	0\\
26.58	0\\
26.59	0\\
26.6	0\\
26.61	0\\
26.62	0\\
26.63	0\\
26.64	0\\
26.65	0\\
26.66	0\\
26.67	0\\
26.68	0\\
26.69	0\\
26.7	0\\
26.71	0\\
26.72	0\\
26.73	0\\
26.74	0\\
26.75	0\\
26.76	0\\
26.77	0\\
26.78	0\\
26.79	0\\
26.8	0\\
26.81	0\\
26.82	0\\
26.83	0\\
26.84	0\\
26.85	0\\
26.86	0\\
26.87	0\\
26.88	0\\
26.89	0\\
26.9	0\\
26.91	0\\
26.92	0\\
26.93	0\\
26.94	0\\
26.95	0\\
26.96	0\\
26.97	0\\
26.98	0\\
26.99	0\\
27	0\\
27.01	0\\
27.02	0\\
27.03	0\\
27.04	0\\
27.05	0\\
27.06	0\\
27.07	0\\
27.08	0\\
27.09	0\\
27.1	0\\
27.11	0\\
27.12	0\\
27.13	0\\
27.14	0\\
27.15	0\\
27.16	0\\
27.17	0\\
27.18	0\\
27.19	0\\
27.2	0\\
27.21	0\\
27.22	0\\
27.23	0\\
27.24	0\\
27.25	0\\
27.26	0\\
27.27	0\\
27.28	0\\
27.29	0\\
27.3	0\\
27.31	0\\
27.32	0\\
27.33	0\\
27.34	0\\
27.35	0\\
27.36	0\\
27.37	0\\
27.38	0\\
27.39	0\\
27.4	0\\
27.41	0\\
27.42	0\\
27.43	0\\
27.44	0\\
27.45	0\\
27.46	0\\
27.47	0\\
27.48	0\\
27.49	0\\
27.5	0\\
27.51	0\\
27.52	0\\
27.53	0\\
27.54	0\\
27.55	0\\
27.56	0\\
27.57	0\\
27.58	0\\
27.59	0\\
27.6	0\\
27.61	0\\
27.62	0\\
27.63	0\\
27.64	0\\
27.65	0\\
27.66	0\\
27.67	0\\
27.68	0\\
27.69	0\\
27.7	0\\
27.71	0\\
27.72	0\\
27.73	0\\
27.74	0\\
27.75	0\\
27.76	0\\
27.77	0\\
27.78	0\\
27.79	0\\
27.8	0\\
27.81	0\\
27.82	0\\
27.83	0\\
27.84	0\\
27.85	0\\
27.86	0\\
27.87	0\\
27.88	0\\
27.89	0\\
27.9	0\\
27.91	0\\
27.92	0\\
27.93	0\\
27.94	0\\
27.95	0\\
27.96	0\\
27.97	0\\
27.98	0\\
27.99	0\\
28	0\\
28.01	0\\
28.02	0\\
28.03	0\\
28.04	0\\
28.05	0\\
28.06	0\\
28.07	0\\
28.08	0\\
28.09	0\\
28.1	0\\
28.11	0\\
28.12	0\\
28.13	0\\
28.14	0\\
28.15	0\\
28.16	0\\
28.17	0\\
28.18	0\\
28.19	0\\
28.2	0\\
28.21	0\\
28.22	0\\
28.23	0\\
28.24	0\\
28.25	0\\
28.26	0\\
28.27	0\\
28.28	0\\
28.29	0\\
28.3	0\\
28.31	0\\
28.32	0\\
28.33	0\\
28.34	0\\
28.35	0\\
28.36	0\\
28.37	0\\
28.38	0\\
28.39	0\\
28.4	0\\
28.41	0\\
28.42	0\\
28.43	0\\
28.44	0\\
28.45	0\\
28.46	0\\
28.47	0\\
28.48	0\\
28.49	0\\
28.5	0\\
28.51	0\\
28.52	0\\
28.53	0\\
28.54	0\\
28.55	0\\
28.56	0\\
28.57	0\\
28.58	0\\
28.59	0\\
28.6	0\\
28.61	0\\
28.62	0\\
28.63	0\\
28.64	0\\
28.65	0\\
28.66	0\\
28.67	0\\
28.68	0\\
28.69	0\\
28.7	0\\
28.71	0\\
28.72	0\\
28.73	0\\
28.74	0\\
28.75	0\\
28.76	0\\
28.77	0\\
28.78	0\\
28.79	0\\
28.8	0\\
28.81	0\\
28.82	0\\
28.83	0\\
28.84	0\\
28.85	0\\
28.86	0\\
28.87	0\\
28.88	0\\
28.89	0\\
28.9	0\\
28.91	0\\
28.92	0\\
28.93	0\\
28.94	0\\
28.95	0\\
28.96	0\\
28.97	0\\
28.98	0\\
28.99	0\\
29	0\\
29.01	0\\
29.02	0\\
29.03	0\\
29.04	0\\
29.05	0\\
29.06	0\\
29.07	0\\
29.08	0\\
29.09	0\\
29.1	0\\
29.11	0\\
29.12	0\\
29.13	0\\
29.14	0\\
29.15	0\\
29.16	0\\
29.17	0\\
29.18	0\\
29.19	0\\
29.2	0\\
29.21	0\\
29.22	0\\
29.23	0\\
29.24	0\\
29.25	0\\
29.26	0\\
29.27	0\\
29.28	0\\
29.29	0\\
29.3	0\\
29.31	0\\
29.32	0\\
29.33	0\\
29.34	0\\
29.35	0\\
29.36	0\\
29.37	0\\
29.38	0\\
29.39	0\\
29.4	0\\
29.41	0\\
29.42	0\\
29.43	0\\
29.44	0\\
29.45	0\\
29.46	0\\
29.47	0\\
29.48	0\\
29.49	0\\
29.5	0\\
29.51	0\\
29.52	0\\
29.53	0\\
29.54	0\\
29.55	0\\
29.56	0\\
29.57	0\\
29.58	0\\
29.59	0\\
29.6	0\\
29.61	0\\
29.62	0\\
29.63	0\\
29.64	0\\
29.65	0\\
29.66	0\\
29.67	0\\
29.68	0\\
29.69	0\\
29.7	0\\
29.71	0\\
29.72	0\\
29.73	0\\
29.74	0\\
29.75	0\\
29.76	0\\
29.77	0\\
29.78	0\\
29.79	0\\
29.8	0\\
29.81	0\\
29.82	0\\
29.83	0\\
29.84	0\\
29.85	0\\
29.86	0\\
29.87	0\\
29.88	0\\
29.89	0\\
29.9	0\\
29.91	0\\
29.92	0\\
29.93	0\\
29.94	0\\
29.95	0\\
29.96	0\\
29.97	0\\
29.98	0\\
29.99	0\\
30	0\\
30.01	0\\
30.02	0\\
30.03	0\\
30.04	0\\
30.05	0\\
30.06	0\\
30.07	0\\
30.08	0\\
30.09	0\\
30.1	0\\
30.11	0\\
30.12	0\\
30.13	0\\
30.14	0\\
30.15	0\\
30.16	0\\
30.17	0\\
30.18	0\\
30.19	0\\
30.2	0\\
30.21	0\\
30.22	0\\
30.23	0\\
30.24	0\\
30.25	0\\
30.26	0\\
30.27	0\\
30.28	0\\
30.29	0\\
30.3	0\\
30.31	0\\
30.32	0\\
30.33	0\\
30.34	0\\
30.35	0\\
30.36	0\\
30.37	0\\
30.38	0\\
30.39	0\\
30.4	0\\
30.41	0\\
30.42	0\\
30.43	0\\
30.44	0\\
30.45	0\\
30.46	0\\
30.47	0\\
30.48	0\\
30.49	0\\
30.5	0\\
30.51	0\\
30.52	0\\
30.53	0\\
30.54	0\\
30.55	0\\
30.56	0\\
30.57	0\\
30.58	0\\
30.59	0\\
30.6	0\\
30.61	0\\
30.62	0\\
30.63	0\\
30.64	0\\
30.65	0\\
30.66	0\\
30.67	0\\
30.68	0\\
30.69	0\\
30.7	0\\
30.71	0\\
30.72	0\\
30.73	0\\
30.74	0\\
30.75	0\\
30.76	0\\
30.77	0\\
30.78	0\\
30.79	0\\
30.8	0\\
30.81	0\\
30.82	0\\
30.83	0\\
30.84	0\\
30.85	0\\
30.86	0\\
30.87	0\\
30.88	0\\
30.89	0\\
30.9	0\\
30.91	0\\
30.92	0\\
30.93	0\\
30.94	0\\
30.95	0\\
30.96	0\\
30.97	0\\
30.98	0\\
30.99	0\\
31	0\\
31.01	0\\
31.02	0\\
31.03	0\\
31.04	0\\
31.05	0\\
31.06	0\\
31.07	0\\
31.08	0\\
31.09	0\\
31.1	0\\
31.11	0\\
31.12	0\\
31.13	0\\
31.14	0\\
31.15	0\\
31.16	0\\
31.17	0\\
31.18	0\\
31.19	0\\
31.2	0\\
31.21	0\\
31.22	0\\
31.23	0\\
31.24	0\\
31.25	0\\
31.26	0\\
31.27	0\\
31.28	0\\
31.29	0\\
31.3	0\\
31.31	0\\
31.32	0\\
31.33	0\\
31.34	0\\
31.35	0\\
31.36	0\\
31.37	0\\
31.38	0\\
31.39	0\\
31.4	0\\
31.41	0\\
31.42	0\\
31.43	0\\
31.44	0\\
31.45	0\\
31.46	0\\
31.47	0\\
31.48	0\\
31.49	0\\
31.5	0\\
31.51	0\\
31.52	0\\
31.53	0\\
31.54	0\\
31.55	0\\
31.56	0\\
31.57	0\\
31.58	0\\
31.59	0\\
31.6	0\\
31.61	0\\
31.62	0\\
31.63	0\\
31.64	0\\
31.65	0\\
31.66	0\\
31.67	0\\
31.68	0\\
31.69	0\\
31.7	0\\
31.71	0\\
31.72	0\\
31.73	0\\
31.74	0\\
31.75	0\\
31.76	0\\
31.77	0\\
31.78	0\\
31.79	0\\
31.8	0\\
31.81	0\\
31.82	0\\
31.83	0\\
31.84	0\\
31.85	0\\
31.86	0\\
31.87	0\\
31.88	0\\
31.89	0\\
31.9	0\\
31.91	0\\
31.92	0\\
31.93	0\\
31.94	0\\
31.95	0\\
31.96	0\\
31.97	0\\
31.98	0\\
31.99	0\\
32	0\\
32.01	0\\
32.02	0\\
32.03	0\\
32.04	0\\
32.05	0\\
32.06	0\\
32.07	0\\
32.08	0\\
32.09	0\\
32.1	0\\
32.11	0\\
32.12	0\\
32.13	0\\
32.14	0\\
32.15	0\\
32.16	0\\
32.17	0\\
32.18	0\\
32.19	0\\
32.2	0\\
32.21	0\\
32.22	0\\
32.23	0\\
32.24	0\\
32.25	0\\
32.26	0\\
32.27	0\\
32.28	0\\
32.29	0\\
32.3	0\\
32.31	0\\
32.32	0\\
32.33	0\\
32.34	0\\
32.35	0\\
32.36	0\\
32.37	0\\
32.38	0\\
32.39	0\\
32.4	0\\
32.41	0\\
32.42	0\\
32.43	0\\
32.44	0\\
32.45	0\\
32.46	0\\
32.47	0\\
32.48	0\\
32.49	0\\
32.5	0\\
32.51	0\\
32.52	0\\
32.53	0\\
32.54	0\\
32.55	0\\
32.56	0\\
32.57	0\\
32.58	0\\
32.59	0\\
32.6	0\\
32.61	0\\
32.62	0\\
32.63	0\\
32.64	0\\
32.65	0\\
32.66	0\\
32.67	0\\
32.68	0\\
32.69	0\\
32.7	0\\
32.71	0\\
32.72	0\\
32.73	0\\
32.74	0\\
32.75	0\\
32.76	0\\
32.77	0\\
32.78	0\\
32.79	0\\
32.8	0\\
32.81	0\\
32.82	0\\
32.83	0\\
32.84	0\\
32.85	0\\
32.86	0\\
32.87	0\\
32.88	0\\
32.89	0\\
32.9	0\\
32.91	0\\
32.92	0\\
32.93	0\\
32.94	0\\
32.95	0\\
32.96	0\\
32.97	0\\
32.98	0\\
32.99	0\\
33	0\\
33.01	0\\
33.02	0\\
33.03	0\\
33.04	0\\
33.05	0\\
33.06	0\\
33.07	0\\
33.08	0\\
33.09	0\\
33.1	0\\
33.11	0\\
33.12	0\\
33.13	0\\
33.14	0\\
33.15	0\\
33.16	0\\
33.17	0\\
33.18	0\\
33.19	0\\
33.2	0\\
33.21	0\\
33.22	0\\
33.23	0\\
33.24	0\\
33.25	0\\
33.26	0\\
33.27	0\\
33.28	0\\
33.29	0\\
33.3	0\\
33.31	0\\
33.32	0\\
33.33	0\\
33.34	0\\
33.35	0\\
33.36	0\\
33.37	0\\
33.38	0\\
33.39	0\\
33.4	0\\
33.41	0\\
33.42	0\\
33.43	0\\
33.44	0\\
33.45	0\\
33.46	0\\
33.47	0\\
33.48	0\\
33.49	0\\
33.5	0\\
33.51	0\\
33.52	0\\
33.53	0\\
33.54	0\\
33.55	0\\
33.56	0\\
33.57	0\\
33.58	0\\
33.59	0\\
33.6	0\\
33.61	0\\
33.62	0\\
33.63	0\\
33.64	0\\
33.65	0\\
33.66	0\\
33.67	0\\
33.68	0\\
33.69	0\\
33.7	0\\
33.71	0\\
33.72	0\\
33.73	0\\
33.74	0\\
33.75	0\\
33.76	0\\
33.77	0\\
33.78	0\\
33.79	0\\
33.8	0\\
33.81	0\\
33.82	0\\
33.83	0\\
33.84	0\\
33.85	0\\
33.86	0\\
33.87	0\\
33.88	0\\
33.89	0\\
33.9	0\\
33.91	0\\
33.92	0\\
33.93	0\\
33.94	0\\
33.95	0\\
33.96	0\\
33.97	0\\
33.98	0\\
33.99	0\\
34	0\\
34.01	0\\
34.02	0\\
34.03	0\\
34.04	0\\
34.05	0\\
34.06	0\\
34.07	0\\
34.08	0\\
34.09	0\\
34.1	0\\
34.11	0\\
34.12	0\\
34.13	0\\
34.14	0\\
34.15	0\\
34.16	0\\
34.17	0\\
34.18	0\\
34.19	0\\
34.2	0\\
34.21	0\\
34.22	0\\
34.23	0\\
34.24	0\\
34.25	0\\
34.26	0\\
34.27	0\\
34.28	0\\
34.29	0\\
34.3	0\\
34.31	0\\
34.32	0\\
34.33	0\\
34.34	0\\
34.35	0\\
34.36	0\\
34.37	0\\
34.38	0\\
34.39	0\\
34.4	0\\
34.41	0\\
34.42	0\\
34.43	0\\
34.44	0\\
34.45	0\\
34.46	0\\
34.47	0\\
34.48	0\\
34.49	0\\
34.5	0\\
34.51	0\\
34.52	0\\
34.53	0\\
34.54	0\\
34.55	0\\
34.56	0\\
34.57	0\\
34.58	0\\
34.59	0\\
34.6	0\\
34.61	0\\
34.62	0\\
34.63	0\\
34.64	0\\
34.65	0\\
34.66	0\\
34.67	0\\
34.68	0\\
34.69	0\\
34.7	0\\
34.71	0\\
34.72	0\\
34.73	0\\
34.74	0\\
34.75	0\\
34.76	0\\
34.77	0\\
34.78	0\\
34.79	0\\
34.8	0\\
34.81	0\\
34.82	0\\
34.83	0\\
34.84	0\\
34.85	0\\
34.86	0\\
34.87	0\\
34.88	0\\
34.89	0\\
34.9	0\\
34.91	0\\
34.92	0\\
34.93	0\\
34.94	0\\
34.95	0\\
34.96	0\\
34.97	0\\
34.98	0\\
34.99	0\\
35	0\\
35.01	0\\
35.02	0\\
35.03	0\\
35.04	0\\
35.05	0\\
35.06	0\\
35.07	0\\
35.08	0\\
35.09	0\\
35.1	0\\
35.11	0\\
35.12	0\\
35.13	0\\
35.14	0\\
35.15	0\\
35.16	0\\
35.17	0\\
35.18	0\\
35.19	0\\
35.2	0\\
35.21	0\\
35.22	0\\
35.23	0\\
35.24	0\\
35.25	0\\
35.26	0\\
35.27	0\\
35.28	0\\
35.29	0\\
35.3	0\\
35.31	0\\
35.32	0\\
35.33	0\\
35.34	0\\
35.35	0\\
35.36	0\\
35.37	0\\
35.38	0\\
35.39	0\\
35.4	0\\
35.41	0\\
35.42	0\\
35.43	0\\
35.44	0\\
35.45	0\\
35.46	0\\
35.47	0\\
35.48	0\\
35.49	0\\
35.5	0\\
35.51	0\\
35.52	0\\
35.53	0\\
35.54	0\\
35.55	0\\
35.56	0\\
35.57	0\\
35.58	0\\
35.59	0\\
35.6	0\\
35.61	0\\
35.62	0\\
35.63	0\\
35.64	0\\
35.65	0\\
35.66	0\\
35.67	0\\
35.68	0\\
35.69	0\\
35.7	0\\
35.71	0\\
35.72	0\\
35.73	0\\
35.74	0\\
35.75	0\\
35.76	0\\
35.77	0\\
35.78	0\\
35.79	0\\
35.8	0\\
35.81	0\\
35.82	0\\
35.83	0\\
35.84	0\\
35.85	0\\
35.86	0\\
35.87	0\\
35.88	0\\
35.89	0\\
35.9	0\\
35.91	0\\
35.92	0\\
35.93	0\\
35.94	0\\
35.95	0\\
35.96	0\\
35.97	0\\
35.98	0\\
35.99	0\\
36	0\\
36.01	0\\
36.02	0\\
36.03	0\\
36.04	0\\
36.05	0\\
36.06	0\\
36.07	0\\
36.08	0\\
36.09	0\\
36.1	0\\
36.11	0\\
36.12	0\\
36.13	0\\
36.14	0\\
36.15	0\\
36.16	0\\
36.17	0\\
36.18	0\\
36.19	0\\
36.2	0\\
36.21	0\\
36.22	0\\
36.23	0\\
36.24	0\\
36.25	0\\
36.26	0\\
36.27	0\\
36.28	0\\
36.29	0\\
36.3	0\\
36.31	0\\
36.32	0\\
36.33	0\\
36.34	0\\
36.35	0\\
36.36	0\\
36.37	0\\
36.38	0\\
36.39	0\\
36.4	0\\
36.41	0\\
36.42	0\\
36.43	0\\
36.44	0\\
36.45	0\\
36.46	0\\
36.47	0\\
36.48	0\\
36.49	0\\
36.5	0\\
36.51	0\\
36.52	0\\
36.53	0\\
36.54	0\\
36.55	0\\
36.56	0\\
36.57	0\\
36.58	0\\
36.59	0\\
36.6	0\\
36.61	0\\
36.62	0\\
36.63	0\\
36.64	0\\
36.65	0\\
36.66	0\\
36.67	0\\
36.68	0\\
36.69	0\\
36.7	0\\
36.71	0\\
36.72	0\\
36.73	0\\
36.74	0\\
36.75	0\\
36.76	0\\
36.77	0\\
36.78	0\\
36.79	0\\
36.8	0\\
36.81	0\\
36.82	0\\
36.83	0\\
36.84	0\\
36.85	0\\
36.86	0\\
36.87	0\\
36.88	0\\
36.89	0\\
36.9	0\\
36.91	0\\
36.92	0\\
36.93	0\\
36.94	0\\
36.95	0\\
36.96	0\\
36.97	0\\
36.98	0\\
36.99	0\\
37	0\\
37.01	0\\
37.02	0\\
37.03	0\\
37.04	0\\
37.05	0\\
37.06	0\\
37.07	0\\
37.08	0\\
37.09	0\\
37.1	0\\
37.11	0\\
37.12	0\\
37.13	0\\
37.14	0\\
37.15	0\\
37.16	0\\
37.17	0\\
37.18	0\\
37.19	0\\
37.2	0\\
37.21	0\\
37.22	0\\
37.23	0\\
37.24	0\\
37.25	0\\
37.26	0\\
37.27	0\\
37.28	0\\
37.29	0\\
37.3	0\\
37.31	0\\
37.32	0\\
37.33	0\\
37.34	0\\
37.35	0\\
37.36	0\\
37.37	0\\
37.38	0\\
37.39	0\\
37.4	0\\
37.41	0\\
37.42	0\\
37.43	0\\
37.44	0\\
37.45	0\\
37.46	0\\
37.47	0\\
37.48	0\\
37.49	0\\
37.5	0\\
37.51	0\\
37.52	0\\
37.53	0\\
37.54	0\\
37.55	0\\
37.56	0\\
37.57	0\\
37.58	0\\
37.59	0\\
37.6	0\\
37.61	0\\
37.62	0\\
37.63	0\\
37.64	0\\
37.65	0\\
37.66	0\\
37.67	0\\
37.68	0\\
37.69	0\\
37.7	0\\
37.71	0\\
37.72	0\\
37.73	0\\
37.74	0\\
37.75	0\\
37.76	0\\
37.77	0\\
37.78	0\\
37.79	0\\
37.8	0\\
37.81	0\\
37.82	0\\
37.83	0\\
37.84	0\\
37.85	0\\
37.86	0\\
37.87	0\\
37.88	0\\
37.89	0\\
37.9	0\\
37.91	0\\
37.92	0\\
37.93	0\\
37.94	0\\
37.95	0\\
37.96	0\\
37.97	0\\
37.98	0\\
37.99	0\\
38	0\\
38.01	0\\
38.02	0\\
38.03	0\\
38.04	0\\
38.05	0\\
38.06	0\\
38.07	0\\
38.08	0\\
38.09	0\\
38.1	0\\
38.11	0\\
38.12	0\\
38.13	0\\
38.14	0\\
38.15	0\\
38.16	0\\
38.17	0\\
38.18	0\\
38.19	0\\
38.2	0\\
38.21	0\\
38.22	0\\
38.23	0\\
38.24	0\\
38.25	0\\
38.26	0\\
38.27	0\\
38.28	0\\
38.29	0\\
38.3	0\\
38.31	0\\
38.32	0\\
38.33	0\\
38.34	0\\
38.35	0\\
38.36	0\\
38.37	0\\
38.38	0\\
38.39	0\\
38.4	0\\
38.41	0\\
38.42	0\\
38.43	0\\
38.44	0\\
38.45	0\\
38.46	0\\
38.47	0\\
38.48	0\\
38.49	0\\
38.5	0\\
38.51	0\\
38.52	0\\
38.53	0\\
38.54	0\\
38.55	0\\
38.56	0\\
38.57	0\\
38.58	0\\
38.59	0\\
38.6	0\\
38.61	0\\
38.62	0\\
38.63	0\\
38.64	0\\
38.65	0\\
38.66	0\\
38.67	0\\
38.68	0\\
38.69	0\\
38.7	0\\
38.71	0\\
38.72	0\\
38.73	0\\
38.74	0\\
38.75	0\\
38.76	0\\
38.77	0\\
38.78	0\\
38.79	0\\
38.8	0\\
38.81	0\\
38.82	0\\
38.83	0\\
38.84	0\\
38.85	0\\
38.86	0\\
38.87	0\\
38.88	0\\
38.89	0\\
38.9	0\\
38.91	0\\
38.92	0\\
38.93	0\\
38.94	0\\
38.95	0\\
38.96	0\\
38.97	0\\
38.98	0\\
38.99	0\\
39	0\\
39.01	0\\
39.02	0\\
39.03	0\\
39.04	0\\
39.05	0\\
39.06	0\\
39.07	0\\
39.08	0\\
39.09	0\\
39.1	0\\
39.11	0\\
39.12	0\\
39.13	0\\
39.14	0\\
39.15	0\\
39.16	0\\
39.17	0\\
39.18	0\\
39.19	0\\
39.2	0\\
39.21	0\\
39.22	0\\
39.23	0\\
39.24	0\\
39.25	0\\
39.26	0\\
39.27	0\\
39.28	0\\
39.29	0\\
39.3	0\\
39.31	0\\
39.32	0\\
39.33	0\\
39.34	0\\
39.35	0\\
39.36	0\\
39.37	0\\
39.38	0\\
39.39	0\\
39.4	0\\
39.41	0\\
39.42	0\\
39.43	0\\
39.44	0\\
39.45	0\\
39.46	0\\
39.47	0\\
39.48	0\\
39.49	0\\
39.5	0\\
39.51	0\\
39.52	0\\
39.53	0\\
39.54	0\\
39.55	0\\
39.56	0\\
39.57	0\\
39.58	0\\
39.59	0\\
39.6	0\\
39.61	0\\
39.62	0\\
39.63	0\\
39.64	0\\
39.65	0\\
39.66	0\\
39.67	0\\
39.68	0\\
39.69	0\\
39.7	0\\
39.71	0\\
39.72	0\\
39.73	0\\
39.74	0\\
39.75	0\\
39.76	0\\
39.77	0\\
39.78	0\\
39.79	0\\
39.8	0\\
39.81	0\\
39.82	0\\
39.83	0\\
39.84	0\\
39.85	0\\
39.86	0\\
39.87	0\\
39.88	0\\
39.89	0\\
39.9	0\\
39.91	0\\
39.92	0\\
39.93	0\\
39.94	0\\
39.95	0\\
39.96	0\\
39.97	0\\
39.98	0\\
39.99	0\\
40	0\\
40.01	0\\
};
\addplot [color=mycolor1,dashed,forget plot]
  table[row sep=crcr]{%
40.01	0\\
40.02	0\\
40.03	0\\
40.04	0\\
40.05	0\\
40.06	0\\
40.07	0\\
40.08	0\\
40.09	0\\
40.1	0\\
40.11	0\\
40.12	0\\
40.13	0\\
40.14	0\\
40.15	0\\
40.16	0\\
40.17	0\\
40.18	0\\
40.19	0\\
40.2	0\\
40.21	0\\
40.22	0\\
40.23	0\\
40.24	0\\
40.25	0\\
40.26	0\\
40.27	0\\
40.28	0\\
40.29	0\\
40.3	0\\
40.31	0\\
40.32	0\\
40.33	0\\
40.34	0\\
40.35	0\\
40.36	0\\
40.37	0\\
40.38	0\\
40.39	0\\
40.4	0\\
40.41	0\\
40.42	0\\
40.43	0\\
40.44	0\\
40.45	0\\
40.46	0\\
40.47	0\\
40.48	0\\
40.49	0\\
40.5	0\\
40.51	0\\
40.52	0\\
40.53	0\\
40.54	0\\
40.55	0\\
40.56	0\\
40.57	0\\
40.58	0\\
40.59	0\\
40.6	0\\
40.61	0\\
40.62	0\\
40.63	0\\
40.64	0\\
40.65	0\\
40.66	0\\
40.67	0\\
40.68	0\\
40.69	0\\
40.7	0\\
40.71	0\\
40.72	0\\
40.73	0\\
40.74	0\\
40.75	0\\
40.76	0\\
40.77	0\\
40.78	0\\
40.79	0\\
40.8	0\\
40.81	0\\
40.82	0\\
40.83	0\\
40.84	0\\
40.85	0\\
40.86	0\\
40.87	0\\
40.88	0\\
40.89	0\\
40.9	0\\
40.91	0\\
40.92	0\\
40.93	0\\
40.94	0\\
40.95	0\\
40.96	0\\
40.97	0\\
40.98	0\\
40.99	0\\
41	0\\
41.01	0\\
41.02	0\\
41.03	0\\
41.04	0\\
41.05	0\\
41.06	0\\
41.07	0\\
41.08	0\\
41.09	0\\
41.1	0\\
41.11	0\\
41.12	0\\
41.13	0\\
41.14	0\\
41.15	0\\
41.16	0\\
41.17	0\\
41.18	0\\
41.19	0\\
41.2	0\\
41.21	0\\
41.22	0\\
41.23	0\\
41.24	0\\
41.25	0\\
41.26	0\\
41.27	0\\
41.28	0\\
41.29	0\\
41.3	0\\
41.31	0\\
41.32	0\\
41.33	0\\
41.34	0\\
41.35	0\\
41.36	0\\
41.37	0\\
41.38	0\\
41.39	0\\
41.4	0\\
41.41	0\\
41.42	0\\
41.43	0\\
41.44	0\\
41.45	0\\
41.46	0\\
41.47	0\\
41.48	0\\
41.49	0\\
41.5	0\\
41.51	0\\
41.52	0\\
41.53	0\\
41.54	0\\
41.55	0\\
41.56	0\\
41.57	0\\
41.58	0\\
41.59	0\\
41.6	0\\
41.61	0\\
41.62	0\\
41.63	0\\
41.64	0\\
41.65	0\\
41.66	0\\
41.67	0\\
41.68	0\\
41.69	0\\
41.7	0\\
41.71	0\\
41.72	0\\
41.73	0\\
41.74	0\\
41.75	0\\
41.76	0\\
41.77	0\\
41.78	0\\
41.79	0\\
41.8	0\\
41.81	0\\
41.82	0\\
41.83	0\\
41.84	0\\
41.85	0\\
41.86	0\\
41.87	0\\
41.88	0\\
41.89	0\\
41.9	0\\
41.91	0\\
41.92	0\\
41.93	0\\
41.94	0\\
41.95	0\\
41.96	0\\
41.97	0\\
41.98	0\\
41.99	0\\
42	0\\
42.01	0\\
42.02	0\\
42.03	0\\
42.04	0\\
42.05	0\\
42.06	0\\
42.07	0\\
42.08	0\\
42.09	0\\
42.1	0\\
42.11	0\\
42.12	0\\
42.13	0\\
42.14	0\\
42.15	0\\
42.16	0\\
42.17	0\\
42.18	0\\
42.19	0\\
42.2	0\\
42.21	0\\
42.22	0\\
42.23	0\\
42.24	0\\
42.25	0\\
42.26	0\\
42.27	0\\
42.28	0\\
42.29	0\\
42.3	0\\
42.31	0\\
42.32	0\\
42.33	0\\
42.34	0\\
42.35	0\\
42.36	0\\
42.37	0\\
42.38	0\\
42.39	0\\
42.4	0\\
42.41	0\\
42.42	0\\
42.43	0\\
42.44	0\\
42.45	0\\
42.46	0\\
42.47	0\\
42.48	0\\
42.49	0\\
42.5	0\\
42.51	0\\
42.52	0\\
42.53	0\\
42.54	0\\
42.55	0\\
42.56	0\\
42.57	0\\
42.58	0\\
42.59	0\\
42.6	0\\
42.61	0\\
42.62	0\\
42.63	0\\
42.64	0\\
42.65	0\\
42.66	0\\
42.67	0\\
42.68	0\\
42.69	0\\
42.7	0\\
42.71	0\\
42.72	0\\
42.73	0\\
42.74	0\\
42.75	0\\
42.76	0\\
42.77	0\\
42.78	0\\
42.79	0\\
42.8	0\\
42.81	0\\
42.82	0\\
42.83	0\\
42.84	0\\
42.85	0\\
42.86	0\\
42.87	0\\
42.88	0\\
42.89	0\\
42.9	0\\
42.91	0\\
42.92	0\\
42.93	0\\
42.94	0\\
42.95	0\\
42.96	0\\
42.97	0\\
42.98	0\\
42.99	0\\
43	0\\
43.01	0\\
43.02	0\\
43.03	0\\
43.04	0\\
43.05	0\\
43.06	0\\
43.07	0\\
43.08	0\\
43.09	0\\
43.1	0\\
43.11	0\\
43.12	0\\
43.13	0\\
43.14	0\\
43.15	0\\
43.16	0\\
43.17	0\\
43.18	0\\
43.19	0\\
43.2	0\\
43.21	0\\
43.22	0\\
43.23	0\\
43.24	0\\
43.25	0\\
43.26	0\\
43.27	0\\
43.28	0\\
43.29	0\\
43.3	0\\
43.31	0\\
43.32	0\\
43.33	0\\
43.34	0\\
43.35	0\\
43.36	0\\
43.37	0\\
43.38	0\\
43.39	0\\
43.4	0\\
43.41	0\\
43.42	0\\
43.43	0\\
43.44	0\\
43.45	0\\
43.46	0\\
43.47	0\\
43.48	0\\
43.49	0\\
43.5	0\\
43.51	0\\
43.52	0\\
43.53	0\\
43.54	0\\
43.55	0\\
43.56	0\\
43.57	0\\
43.58	0\\
43.59	0\\
43.6	0\\
43.61	0\\
43.62	0\\
43.63	0\\
43.64	0\\
43.65	0\\
43.66	0\\
43.67	0\\
43.68	0\\
43.69	0\\
43.7	0\\
43.71	0\\
43.72	0\\
43.73	0\\
43.74	0\\
43.75	0\\
43.76	0\\
43.77	0\\
43.78	0\\
43.79	0\\
43.8	0\\
43.81	0\\
43.82	0\\
43.83	0\\
43.84	0\\
43.85	0\\
43.86	0\\
43.87	0\\
43.88	0\\
43.89	0\\
43.9	0\\
43.91	0\\
43.92	0\\
43.93	0\\
43.94	0\\
43.95	0\\
43.96	0\\
43.97	0\\
43.98	0\\
43.99	0\\
44	0\\
44.01	0\\
44.02	0\\
44.03	0\\
44.04	0\\
44.05	0\\
44.06	0\\
44.07	0\\
44.08	0\\
44.09	0\\
44.1	0\\
44.11	0\\
44.12	0\\
44.13	0\\
44.14	0\\
44.15	0\\
44.16	0\\
44.17	0\\
44.18	0\\
44.19	0\\
44.2	0\\
44.21	0\\
44.22	0\\
44.23	0\\
44.24	0\\
44.25	0\\
44.26	0\\
44.27	0\\
44.28	0\\
44.29	0\\
44.3	0\\
44.31	0\\
44.32	0\\
44.33	0\\
44.34	0\\
44.35	0\\
44.36	0\\
44.37	0\\
44.38	0\\
44.39	0\\
44.4	0\\
44.41	0\\
44.42	0\\
44.43	0\\
44.44	0\\
44.45	0\\
44.46	0\\
44.47	0\\
44.48	0\\
44.49	0\\
44.5	0\\
44.51	0\\
44.52	0\\
44.53	0\\
44.54	0\\
44.55	0\\
44.56	0\\
44.57	0\\
44.58	0\\
44.59	0\\
44.6	0\\
44.61	0\\
44.62	0\\
44.63	0\\
44.64	0\\
44.65	0\\
44.66	0\\
44.67	0\\
44.68	0\\
44.69	0\\
44.7	0\\
44.71	0\\
44.72	0\\
44.73	0\\
44.74	0\\
44.75	0\\
44.76	0\\
44.77	0\\
44.78	0\\
44.79	0\\
44.8	0\\
44.81	0\\
44.82	0\\
44.83	0\\
44.84	0\\
44.85	0\\
44.86	0\\
44.87	0\\
44.88	0\\
44.89	0\\
44.9	0\\
44.91	0\\
44.92	0\\
44.93	0\\
44.94	0\\
44.95	0\\
44.96	0\\
44.97	0\\
44.98	0\\
44.99	0\\
45	0\\
45.01	0\\
45.02	0\\
45.03	0\\
45.04	0\\
45.05	0\\
45.06	0\\
45.07	0\\
45.08	0\\
45.09	0\\
45.1	0\\
45.11	0\\
45.12	0\\
45.13	0\\
45.14	0\\
45.15	0\\
45.16	0\\
45.17	0\\
45.18	0\\
45.19	0\\
45.2	0\\
45.21	0\\
45.22	0\\
45.23	0\\
45.24	0\\
45.25	0\\
45.26	0\\
45.27	0\\
45.28	0\\
45.29	0\\
45.3	0\\
45.31	0\\
45.32	0\\
45.33	0\\
45.34	0\\
45.35	0\\
45.36	0\\
45.37	0\\
45.38	0\\
45.39	0\\
45.4	0\\
45.41	0\\
45.42	0\\
45.43	0\\
45.44	0\\
45.45	0\\
45.46	0\\
45.47	0\\
45.48	0\\
45.49	0\\
45.5	0\\
45.51	0\\
45.52	0\\
45.53	0\\
45.54	0\\
45.55	0\\
45.56	0\\
45.57	0\\
45.58	0\\
45.59	0\\
45.6	0\\
45.61	0\\
45.62	0\\
45.63	0\\
45.64	0\\
45.65	0\\
45.66	0\\
45.67	0\\
45.68	0\\
45.69	0\\
45.7	0\\
45.71	0\\
45.72	0\\
45.73	0\\
45.74	0\\
45.75	0\\
45.76	0\\
45.77	0\\
45.78	0\\
45.79	0\\
45.8	0\\
45.81	0\\
45.82	0\\
45.83	0\\
45.84	0\\
45.85	0\\
45.86	0\\
45.87	0\\
45.88	0\\
45.89	0\\
45.9	0\\
45.91	0\\
45.92	0\\
45.93	0\\
45.94	0\\
45.95	0\\
45.96	0\\
45.97	0\\
45.98	0\\
45.99	0\\
46	0\\
46.01	0\\
46.02	0\\
46.03	0\\
46.04	0\\
46.05	0\\
46.06	0\\
46.07	0\\
46.08	0\\
46.09	0\\
46.1	0\\
46.11	0\\
46.12	0\\
46.13	0\\
46.14	0\\
46.15	0\\
46.16	0\\
46.17	0\\
46.18	0\\
46.19	0\\
46.2	0\\
46.21	0\\
46.22	0\\
46.23	0\\
46.24	0\\
46.25	0\\
46.26	0\\
46.27	0\\
46.28	0\\
46.29	0\\
46.3	0\\
46.31	0\\
46.32	0\\
46.33	0\\
46.34	0\\
46.35	0\\
46.36	0\\
46.37	0\\
46.38	0\\
46.39	0\\
46.4	0\\
46.41	0\\
46.42	0\\
46.43	0\\
46.44	0\\
46.45	0\\
46.46	0\\
46.47	0\\
46.48	0\\
46.49	0\\
46.5	0\\
46.51	0\\
46.52	0\\
46.53	0\\
46.54	0\\
46.55	0\\
46.56	0\\
46.57	0\\
46.58	0\\
46.59	0\\
46.6	0\\
46.61	0\\
46.62	0\\
46.63	0\\
46.64	0\\
46.65	0\\
46.66	0\\
46.67	0\\
46.68	0\\
46.69	0\\
46.7	0\\
46.71	0\\
46.72	0\\
46.73	0\\
46.74	0\\
46.75	0\\
46.76	0\\
46.77	0\\
46.78	0\\
46.79	0\\
46.8	0\\
46.81	0\\
46.82	0\\
46.83	0\\
46.84	0\\
46.85	0\\
46.86	0\\
46.87	0\\
46.88	0\\
46.89	0\\
46.9	0\\
46.91	0\\
46.92	0\\
46.93	0\\
46.94	0\\
46.95	0\\
46.96	0\\
46.97	0\\
46.98	0\\
46.99	0\\
47	0\\
47.01	0\\
47.02	0\\
47.03	0\\
47.04	0\\
47.05	0\\
47.06	0\\
47.07	0\\
47.08	0\\
47.09	0\\
47.1	0\\
47.11	0\\
47.12	0\\
47.13	0\\
47.14	0\\
47.15	0\\
47.16	0\\
47.17	0\\
47.18	0\\
47.19	0\\
47.2	0\\
47.21	0\\
47.22	0\\
47.23	0\\
47.24	0\\
47.25	0\\
47.26	0\\
47.27	0\\
47.28	0\\
47.29	0\\
47.3	0\\
47.31	0\\
47.32	0\\
47.33	0\\
47.34	0\\
47.35	0\\
47.36	0\\
47.37	0\\
47.38	0\\
47.39	0\\
47.4	0\\
47.41	0\\
47.42	0\\
47.43	0\\
47.44	0\\
47.45	0\\
47.46	0\\
47.47	0\\
47.48	0\\
47.49	0\\
47.5	0\\
47.51	0\\
47.52	0\\
47.53	0\\
47.54	0\\
47.55	0\\
47.56	0\\
47.57	0\\
47.58	0\\
47.59	0\\
47.6	0\\
47.61	0\\
47.62	0\\
47.63	0\\
47.64	0\\
47.65	0\\
47.66	0\\
47.67	0\\
47.68	0\\
47.69	0\\
47.7	0\\
47.71	0\\
47.72	0\\
47.73	0\\
47.74	0\\
47.75	0\\
47.76	0\\
47.77	0\\
47.78	0\\
47.79	0\\
47.8	0\\
47.81	0\\
47.82	0\\
47.83	0\\
47.84	0\\
47.85	0\\
47.86	0\\
47.87	0\\
47.88	0\\
47.89	0\\
47.9	0\\
47.91	0\\
47.92	0\\
47.93	0\\
47.94	0\\
47.95	0\\
47.96	0\\
47.97	0\\
47.98	0\\
47.99	0\\
48	0\\
48.01	0\\
48.02	0\\
48.03	0\\
48.04	0\\
48.05	0\\
48.06	0\\
48.07	0\\
48.08	0\\
48.09	0\\
48.1	0\\
48.11	0\\
48.12	0\\
48.13	0\\
48.14	0\\
48.15	0\\
48.16	0\\
48.17	0\\
48.18	0\\
48.19	0\\
48.2	0\\
48.21	0\\
48.22	0\\
48.23	0\\
48.24	0\\
48.25	0\\
48.26	0\\
48.27	0\\
48.28	0\\
48.29	0\\
48.3	0\\
48.31	0\\
48.32	0\\
48.33	0\\
48.34	0\\
48.35	0\\
48.36	0\\
48.37	0\\
48.38	0\\
48.39	0\\
48.4	0\\
48.41	0\\
48.42	0\\
48.43	0\\
48.44	0\\
48.45	0\\
48.46	0\\
48.47	0\\
48.48	0\\
48.49	0\\
48.5	0\\
48.51	0\\
48.52	0\\
48.53	0\\
48.54	0\\
48.55	0\\
48.56	0\\
48.57	0\\
48.58	0\\
48.59	0\\
48.6	0\\
48.61	0\\
48.62	0\\
48.63	0\\
48.64	0\\
48.65	0\\
48.66	0\\
48.67	0\\
48.68	0\\
48.69	0\\
48.7	0\\
48.71	0\\
48.72	0\\
48.73	0\\
48.74	0\\
48.75	0\\
48.76	0\\
48.77	0\\
48.78	0\\
48.79	0\\
48.8	0\\
48.81	0\\
48.82	0\\
48.83	0\\
48.84	0\\
48.85	0\\
48.86	0\\
48.87	0\\
48.88	0\\
48.89	0\\
48.9	0\\
48.91	0\\
48.92	0\\
48.93	0\\
48.94	0\\
48.95	0\\
48.96	0\\
48.97	0\\
48.98	0\\
48.99	0\\
49	0\\
49.01	0\\
49.02	0\\
49.03	0\\
49.04	0\\
49.05	0\\
49.06	0\\
49.07	0\\
49.08	0\\
49.09	0\\
49.1	0\\
49.11	0\\
49.12	0\\
49.13	0\\
49.14	0\\
49.15	0\\
49.16	0\\
49.17	0\\
49.18	0\\
49.19	0\\
49.2	0\\
49.21	0\\
49.22	0\\
49.23	0\\
49.24	0\\
49.25	0\\
49.26	0\\
49.27	0\\
49.28	0\\
49.29	0\\
49.3	0\\
49.31	0\\
49.32	0\\
49.33	0\\
49.34	0\\
49.35	0\\
49.36	0\\
49.37	0\\
49.38	0\\
49.39	0\\
49.4	0\\
49.41	0\\
49.42	0\\
49.43	0\\
49.44	0\\
49.45	0\\
49.46	0\\
49.47	0\\
49.48	0\\
49.49	0\\
49.5	0\\
49.51	0\\
49.52	0\\
49.53	0\\
49.54	0\\
49.55	0\\
49.56	0\\
49.57	0\\
49.58	0\\
49.59	0\\
49.6	0\\
49.61	0\\
49.62	0\\
49.63	0\\
49.64	0\\
49.65	0\\
49.66	0\\
49.67	0\\
49.68	0\\
49.69	0\\
49.7	0\\
49.71	0\\
49.72	0\\
49.73	0\\
49.74	0\\
49.75	0\\
49.76	0\\
49.77	0\\
49.78	0\\
49.79	0\\
49.8	0\\
49.81	0\\
49.82	0\\
49.83	0\\
49.84	0\\
49.85	0\\
49.86	0\\
49.87	0\\
49.88	0\\
49.89	0\\
49.9	0\\
49.91	0\\
49.92	0\\
49.93	0\\
49.94	0\\
49.95	0\\
49.96	0\\
49.97	0\\
49.98	0\\
49.99	0\\
50	0\\
50.01	0\\
50.02	0\\
50.03	0\\
50.04	0\\
50.05	0\\
50.06	0\\
50.07	0\\
50.08	0\\
50.09	0\\
50.1	0\\
50.11	0\\
50.12	0\\
50.13	0\\
50.14	0\\
50.15	0\\
50.16	0\\
50.17	0\\
50.18	0\\
50.19	0\\
50.2	0\\
50.21	0\\
50.22	0\\
50.23	0\\
50.24	0\\
50.25	0\\
50.26	0\\
50.27	0\\
50.28	0\\
50.29	0\\
50.3	0\\
50.31	0\\
50.32	0\\
50.33	0\\
50.34	0\\
50.35	0\\
50.36	0\\
50.37	0\\
50.38	0\\
50.39	0\\
50.4	0\\
50.41	0\\
50.42	0\\
50.43	0\\
50.44	0\\
50.45	0\\
50.46	0\\
50.47	0\\
50.48	0\\
50.49	0\\
50.5	0\\
50.51	0\\
50.52	0\\
50.53	0\\
50.54	0\\
50.55	0\\
50.56	0\\
50.57	0\\
50.58	0\\
50.59	0\\
50.6	0\\
50.61	0\\
50.62	0\\
50.63	0\\
50.64	0\\
50.65	0\\
50.66	0\\
50.67	0\\
50.68	0\\
50.69	0\\
50.7	0\\
50.71	0\\
50.72	0\\
50.73	0\\
50.74	0\\
50.75	0\\
50.76	0\\
50.77	0\\
50.78	0\\
50.79	0\\
50.8	0\\
50.81	0\\
50.82	0\\
50.83	0\\
50.84	0\\
50.85	0\\
50.86	0\\
50.87	0\\
50.88	0\\
50.89	0\\
50.9	0\\
50.91	0\\
50.92	0\\
50.93	0\\
50.94	0\\
50.95	0\\
50.96	0\\
50.97	0\\
50.98	0\\
50.99	0\\
51	0\\
51.01	0\\
51.02	0\\
51.03	0\\
51.04	0\\
51.05	0\\
51.06	0\\
51.07	0\\
51.08	0\\
51.09	0\\
51.1	0\\
51.11	0\\
51.12	0\\
51.13	0\\
51.14	0\\
51.15	0\\
51.16	0\\
51.17	0\\
51.18	0\\
51.19	0\\
51.2	0\\
51.21	0\\
51.22	0\\
51.23	0\\
51.24	0\\
51.25	0\\
51.26	0\\
51.27	0\\
51.28	0\\
51.29	0\\
51.3	0\\
51.31	0\\
51.32	0\\
51.33	0\\
51.34	0\\
51.35	0\\
51.36	0\\
51.37	0\\
51.38	0\\
51.39	0\\
51.4	0\\
51.41	0\\
51.42	0\\
51.43	0\\
51.44	0\\
51.45	0\\
51.46	0\\
51.47	0\\
51.48	0\\
51.49	0\\
51.5	0\\
51.51	0\\
51.52	0\\
51.53	0\\
51.54	0\\
51.55	0\\
51.56	0\\
51.57	0\\
51.58	0\\
51.59	0\\
51.6	0\\
51.61	0\\
51.62	0\\
51.63	0\\
51.64	0\\
51.65	0\\
51.66	0\\
51.67	0\\
51.68	0\\
51.69	0\\
51.7	0\\
51.71	0\\
51.72	0\\
51.73	0\\
51.74	0\\
51.75	0\\
51.76	0\\
51.77	0\\
51.78	0\\
51.79	0\\
51.8	0\\
51.81	0\\
51.82	0\\
51.83	0\\
51.84	0\\
51.85	0\\
51.86	0\\
51.87	0\\
51.88	0\\
51.89	0\\
51.9	0\\
51.91	0\\
51.92	0\\
51.93	0\\
51.94	0\\
51.95	0\\
51.96	0\\
51.97	0\\
51.98	0\\
51.99	0\\
52	0\\
52.01	0\\
52.02	0\\
52.03	0\\
52.04	0\\
52.05	0\\
52.06	0\\
52.07	0\\
52.08	0\\
52.09	0\\
52.1	0\\
52.11	0\\
52.12	0\\
52.13	0\\
52.14	0\\
52.15	0\\
52.16	0\\
52.17	0\\
52.18	0\\
52.19	0\\
52.2	0\\
52.21	0\\
52.22	0\\
52.23	0\\
52.24	0\\
52.25	0\\
52.26	0\\
52.27	0\\
52.28	0\\
52.29	0\\
52.3	0\\
52.31	0\\
52.32	0\\
52.33	0\\
52.34	0\\
52.35	0\\
52.36	0\\
52.37	0\\
52.38	0\\
52.39	0\\
52.4	0\\
52.41	0\\
52.42	0\\
52.43	0\\
52.44	0\\
52.45	0\\
52.46	0\\
52.47	0\\
52.48	0\\
52.49	0\\
52.5	0\\
52.51	0\\
52.52	0\\
52.53	0\\
52.54	0\\
52.55	0\\
52.56	0\\
52.57	0\\
52.58	0\\
52.59	0\\
52.6	0\\
52.61	0\\
52.62	0\\
52.63	0\\
52.64	0\\
52.65	0\\
52.66	0\\
52.67	0\\
52.68	0\\
52.69	0\\
52.7	0\\
52.71	0\\
52.72	0\\
52.73	0\\
52.74	0\\
52.75	0\\
52.76	0\\
52.77	0\\
52.78	0\\
52.79	0\\
52.8	0\\
52.81	0\\
52.82	0\\
52.83	0\\
52.84	0\\
52.85	0\\
52.86	0\\
52.87	0\\
52.88	0\\
52.89	0\\
52.9	0\\
52.91	0\\
52.92	0\\
52.93	0\\
52.94	0\\
52.95	0\\
52.96	0\\
52.97	0\\
52.98	0\\
52.99	0\\
53	0\\
53.01	0\\
53.02	0\\
53.03	0\\
53.04	0\\
53.05	0\\
53.06	0\\
53.07	0\\
53.08	0\\
53.09	0\\
53.1	0\\
53.11	0\\
53.12	0\\
53.13	0\\
53.14	0\\
53.15	0\\
53.16	0\\
53.17	0\\
53.18	0\\
53.19	0\\
53.2	0\\
53.21	0\\
53.22	0\\
53.23	0\\
53.24	0\\
53.25	0\\
53.26	0\\
53.27	0\\
53.28	0\\
53.29	0\\
53.3	0\\
53.31	0\\
53.32	0\\
53.33	0\\
53.34	0\\
53.35	0\\
53.36	0\\
53.37	0\\
53.38	0\\
53.39	0\\
53.4	0\\
53.41	0\\
53.42	0\\
53.43	0\\
53.44	0\\
53.45	0\\
53.46	0\\
53.47	0\\
53.48	0\\
53.49	0\\
53.5	0\\
53.51	0\\
53.52	0\\
53.53	0\\
53.54	0\\
53.55	0\\
53.56	0\\
53.57	0\\
53.58	0\\
53.59	0\\
53.6	0\\
53.61	0\\
53.62	0\\
53.63	0\\
53.64	0\\
53.65	0\\
53.66	0\\
53.67	0\\
53.68	0\\
53.69	0\\
53.7	0\\
53.71	0\\
53.72	0\\
53.73	0\\
53.74	0\\
53.75	0\\
53.76	0\\
53.77	0\\
53.78	0\\
53.79	0\\
53.8	0\\
53.81	0\\
53.82	0\\
53.83	0\\
53.84	0\\
53.85	0\\
53.86	0\\
53.87	0\\
53.88	0\\
53.89	0\\
53.9	0\\
53.91	0\\
53.92	0\\
53.93	0\\
53.94	0\\
53.95	0\\
53.96	0\\
53.97	0\\
53.98	0\\
53.99	0\\
54	0\\
54.01	0\\
54.02	0\\
54.03	0\\
54.04	0\\
54.05	0\\
54.06	0\\
54.07	0\\
54.08	0\\
54.09	0\\
54.1	0\\
54.11	0\\
54.12	0\\
54.13	0\\
54.14	0\\
54.15	0\\
54.16	0\\
54.17	0\\
54.18	0\\
54.19	0\\
54.2	0\\
54.21	0\\
54.22	0\\
54.23	0\\
54.24	0\\
54.25	0\\
54.26	0\\
54.27	0\\
54.28	0\\
54.29	0\\
54.3	0\\
54.31	0\\
54.32	0\\
54.33	0\\
54.34	0\\
54.35	0\\
54.36	0\\
54.37	0\\
54.38	0\\
54.39	0\\
54.4	0\\
54.41	0\\
54.42	0\\
54.43	0\\
54.44	0\\
54.45	0\\
54.46	0\\
54.47	0\\
54.48	0\\
54.49	0\\
54.5	0\\
54.51	0\\
54.52	0\\
54.53	0\\
54.54	0\\
54.55	0\\
54.56	0\\
54.57	0\\
54.58	0\\
54.59	0\\
54.6	0\\
54.61	0\\
54.62	0\\
54.63	0\\
54.64	0\\
54.65	0\\
54.66	0\\
54.67	0\\
54.68	0\\
54.69	0\\
54.7	0\\
54.71	0\\
54.72	0\\
54.73	0\\
54.74	0\\
54.75	0\\
54.76	0\\
54.77	0\\
54.78	0\\
54.79	0\\
54.8	0\\
54.81	0\\
54.82	0\\
54.83	0\\
54.84	0\\
54.85	0\\
54.86	0\\
54.87	0\\
54.88	0\\
54.89	0\\
54.9	0\\
54.91	0\\
54.92	0\\
54.93	0\\
54.94	0\\
54.95	0\\
54.96	0\\
54.97	0\\
54.98	0\\
54.99	0\\
55	0\\
55.01	0\\
55.02	0\\
55.03	0\\
55.04	0\\
55.05	0\\
55.06	0\\
55.07	0\\
55.08	0\\
55.09	0\\
55.1	0\\
55.11	0\\
55.12	0\\
55.13	0\\
55.14	0\\
55.15	0\\
55.16	0\\
55.17	0\\
55.18	0\\
55.19	0\\
55.2	0\\
55.21	0\\
55.22	0\\
55.23	0\\
55.24	0\\
55.25	0\\
55.26	0\\
55.27	0\\
55.28	0\\
55.29	0\\
55.3	0\\
55.31	0\\
55.32	0\\
55.33	0\\
55.34	0\\
55.35	0\\
55.36	0\\
55.37	0\\
55.38	0\\
55.39	0\\
55.4	0\\
55.41	0\\
55.42	0\\
55.43	0\\
55.44	0\\
55.45	0\\
55.46	0\\
55.47	0\\
55.48	0\\
55.49	0\\
55.5	0\\
55.51	0\\
55.52	0\\
55.53	0\\
55.54	0\\
55.55	0\\
55.56	0\\
55.57	0\\
55.58	0\\
55.59	0\\
55.6	0\\
55.61	0\\
55.62	0\\
55.63	0\\
55.64	0\\
55.65	0\\
55.66	0\\
55.67	0\\
55.68	0\\
55.69	0\\
55.7	0\\
55.71	0\\
55.72	0\\
55.73	0\\
55.74	0\\
55.75	0\\
55.76	0\\
55.77	0\\
55.78	0\\
55.79	0\\
55.8	0\\
55.81	0\\
55.82	0\\
55.83	0\\
55.84	0\\
55.85	0\\
55.86	0\\
55.87	0\\
55.88	0\\
55.89	0\\
55.9	0\\
55.91	0\\
55.92	0\\
55.93	0\\
55.94	0\\
55.95	0\\
55.96	0\\
55.97	0\\
55.98	0\\
55.99	0\\
56	0\\
56.01	0\\
56.02	0\\
56.03	0\\
56.04	0\\
56.05	0\\
56.06	0\\
56.07	0\\
56.08	0\\
56.09	0\\
56.1	0\\
56.11	0\\
56.12	0\\
56.13	0\\
56.14	0\\
56.15	0\\
56.16	0\\
56.17	0\\
56.18	0\\
56.19	0\\
56.2	0\\
56.21	0\\
56.22	0\\
56.23	0\\
56.24	0\\
56.25	0\\
56.26	0\\
56.27	0\\
56.28	0\\
56.29	0\\
56.3	0\\
56.31	0\\
56.32	0\\
56.33	0\\
56.34	0\\
56.35	0\\
56.36	0\\
56.37	0\\
56.38	0\\
56.39	0\\
56.4	0\\
56.41	0\\
56.42	0\\
56.43	0\\
56.44	0\\
56.45	0\\
56.46	0\\
56.47	0\\
56.48	0\\
56.49	0\\
56.5	0\\
56.51	0\\
56.52	0\\
56.53	0\\
56.54	0\\
56.55	0\\
56.56	0\\
56.57	0\\
56.58	0\\
56.59	0\\
56.6	0\\
56.61	0\\
56.62	0\\
56.63	0\\
56.64	0\\
56.65	0\\
56.66	0\\
56.67	0\\
56.68	0\\
56.69	0\\
56.7	0\\
56.71	0\\
56.72	0\\
56.73	0\\
56.74	0\\
56.75	0\\
56.76	0\\
56.77	0\\
56.78	0\\
56.79	0\\
56.8	0\\
56.81	0\\
56.82	0\\
56.83	0\\
56.84	0\\
56.85	0\\
56.86	0\\
56.87	0\\
56.88	0\\
56.89	0\\
56.9	0\\
56.91	0\\
56.92	0\\
56.93	0\\
56.94	0\\
56.95	0\\
56.96	0\\
56.97	0\\
56.98	0\\
56.99	0\\
57	0\\
57.01	0\\
57.02	0\\
57.03	0\\
57.04	0\\
57.05	0\\
57.06	0\\
57.07	0\\
57.08	0\\
57.09	0\\
57.1	0\\
57.11	0\\
57.12	0\\
57.13	0\\
57.14	0\\
57.15	0\\
57.16	0\\
57.17	0\\
57.18	0\\
57.19	0\\
57.2	0\\
57.21	0\\
57.22	0\\
57.23	0\\
57.24	0\\
57.25	0\\
57.26	0\\
57.27	0\\
57.28	0\\
57.29	0\\
57.3	0\\
57.31	0\\
57.32	0\\
57.33	0\\
57.34	0\\
57.35	0\\
57.36	0\\
57.37	0\\
57.38	0\\
57.39	0\\
57.4	0\\
57.41	0\\
57.42	0\\
57.43	0\\
57.44	0\\
57.45	0\\
57.46	0\\
57.47	0\\
57.48	0\\
57.49	0\\
57.5	0\\
57.51	0\\
57.52	0\\
57.53	0\\
57.54	0\\
57.55	0\\
57.56	0\\
57.57	0\\
57.58	0\\
57.59	0\\
57.6	0\\
57.61	0\\
57.62	0\\
57.63	0\\
57.64	0\\
57.65	0\\
57.66	0\\
57.67	0\\
57.68	0\\
57.69	0\\
57.7	0\\
57.71	0\\
57.72	0\\
57.73	0\\
57.74	0\\
57.75	0\\
57.76	0\\
57.77	0\\
57.78	0\\
57.79	0\\
57.8	0\\
57.81	0\\
57.82	0\\
57.83	0\\
57.84	0\\
57.85	0\\
57.86	0\\
57.87	0\\
57.88	0\\
57.89	0\\
57.9	0\\
57.91	0\\
57.92	0\\
57.93	0\\
57.94	0\\
57.95	0\\
57.96	0\\
57.97	0\\
57.98	0\\
57.99	0\\
58	0\\
58.01	0\\
58.02	0\\
58.03	0\\
58.04	0\\
58.05	0\\
58.06	0\\
58.07	0\\
58.08	0\\
58.09	0\\
58.1	0\\
58.11	0\\
58.12	0\\
58.13	0\\
58.14	0\\
58.15	0\\
58.16	0\\
58.17	0\\
58.18	0\\
58.19	0\\
58.2	0\\
58.21	0\\
58.22	0\\
58.23	0\\
58.24	0\\
58.25	0\\
58.26	0\\
58.27	0\\
58.28	0\\
58.29	0\\
58.3	0\\
58.31	0\\
58.32	0\\
58.33	0\\
58.34	0\\
58.35	0\\
58.36	0\\
58.37	0\\
58.38	0\\
58.39	0\\
58.4	0\\
58.41	0\\
58.42	0\\
58.43	0\\
58.44	0\\
58.45	0\\
58.46	0\\
58.47	0\\
58.48	0\\
58.49	0\\
58.5	0\\
58.51	0\\
58.52	0\\
58.53	0\\
58.54	0\\
58.55	0\\
58.56	0\\
58.57	0\\
58.58	0\\
58.59	0\\
58.6	0\\
58.61	0\\
58.62	0\\
58.63	0\\
58.64	0\\
58.65	0\\
58.66	0\\
58.67	0\\
58.68	0\\
58.69	0\\
58.7	0\\
58.71	0\\
58.72	0\\
58.73	0\\
58.74	0\\
58.75	0\\
58.76	0\\
58.77	0\\
58.78	0\\
58.79	0\\
58.8	0\\
58.81	0\\
58.82	0\\
58.83	0\\
58.84	0\\
58.85	0\\
58.86	0\\
58.87	0\\
58.88	0\\
58.89	0\\
58.9	0\\
58.91	0\\
58.92	0\\
58.93	0\\
58.94	0\\
58.95	0\\
58.96	0\\
58.97	0\\
58.98	0\\
58.99	0\\
59	0\\
59.01	0\\
59.02	0\\
59.03	0\\
59.04	0\\
59.05	0\\
59.06	0\\
59.07	0\\
59.08	0\\
59.09	0\\
59.1	0\\
59.11	0\\
59.12	0\\
59.13	0\\
59.14	0\\
59.15	0\\
59.16	0\\
59.17	0\\
59.18	0\\
59.19	0\\
59.2	0\\
59.21	0\\
59.22	0\\
59.23	0\\
59.24	0\\
59.25	0\\
59.26	0\\
59.27	0\\
59.28	0\\
59.29	0\\
59.3	0\\
59.31	0\\
59.32	0\\
59.33	0\\
59.34	0\\
59.35	0\\
59.36	0\\
59.37	0\\
59.38	0\\
59.39	0\\
59.4	0\\
59.41	0\\
59.42	0\\
59.43	0\\
59.44	0\\
59.45	0\\
59.46	0\\
59.47	0\\
59.48	0\\
59.49	0\\
59.5	0\\
59.51	0\\
59.52	0\\
59.53	0\\
59.54	0\\
59.55	0\\
59.56	0\\
59.57	0\\
59.58	0\\
59.59	0\\
59.6	0\\
59.61	0\\
59.62	0\\
59.63	0\\
59.64	0\\
59.65	0\\
59.66	0\\
59.67	0\\
59.68	0\\
59.69	0\\
59.7	0\\
59.71	0\\
59.72	0\\
59.73	0\\
59.74	0\\
59.75	0\\
59.76	0\\
59.77	0\\
59.78	0\\
59.79	0\\
59.8	0\\
59.81	0\\
59.82	0\\
59.83	0\\
59.84	0\\
59.85	0\\
59.86	0\\
59.87	0\\
59.88	0\\
59.89	0\\
59.9	0\\
59.91	0\\
59.92	0\\
59.93	0\\
59.94	0\\
59.95	0\\
59.96	0\\
59.97	0\\
59.98	0\\
59.99	0\\
60	0\\
60.01	0\\
60.02	0\\
60.03	0\\
60.04	0\\
60.05	0\\
60.06	0\\
60.07	0\\
60.08	0\\
60.09	0\\
60.1	0\\
60.11	0\\
60.12	0\\
60.13	0\\
60.14	0\\
60.15	0\\
60.16	0\\
60.17	0\\
60.18	0\\
60.19	0\\
60.2	0\\
60.21	0\\
60.22	0\\
60.23	0\\
60.24	0\\
60.25	0\\
60.26	0\\
60.27	0\\
60.28	0\\
60.29	0\\
60.3	0\\
60.31	0\\
60.32	0\\
60.33	0\\
60.34	0\\
60.35	0\\
60.36	0\\
60.37	0\\
60.38	0\\
60.39	0\\
60.4	0\\
60.41	0\\
60.42	0\\
60.43	0\\
60.44	0\\
60.45	0\\
60.46	0\\
60.47	0\\
60.48	0\\
60.49	0\\
60.5	0\\
60.51	0\\
60.52	0\\
60.53	0\\
60.54	0\\
60.55	0\\
60.56	0\\
60.57	0\\
60.58	0\\
60.59	0\\
60.6	0\\
60.61	0\\
60.62	0\\
60.63	0\\
60.64	0\\
60.65	0\\
60.66	0\\
60.67	0\\
60.68	0\\
60.69	0\\
60.7	0\\
60.71	0\\
60.72	0\\
60.73	0\\
60.74	0\\
60.75	0\\
60.76	0\\
60.77	0\\
60.78	0\\
60.79	0\\
60.8	0\\
60.81	0\\
60.82	0\\
60.83	0\\
60.84	0\\
60.85	0\\
60.86	0\\
60.87	0\\
60.88	0\\
60.89	0\\
60.9	0\\
60.91	0\\
60.92	0\\
60.93	0\\
60.94	0\\
60.95	0\\
60.96	0\\
60.97	0\\
60.98	0\\
60.99	0\\
61	0\\
61.01	0\\
61.02	0\\
61.03	0\\
61.04	0\\
61.05	0\\
61.06	0\\
61.07	0\\
61.08	0\\
61.09	0\\
61.1	0\\
61.11	0\\
61.12	0\\
61.13	0\\
61.14	0\\
61.15	0\\
61.16	0\\
61.17	0\\
61.18	0\\
61.19	0\\
61.2	0\\
61.21	0\\
61.22	0\\
61.23	0\\
61.24	0\\
61.25	0\\
61.26	0\\
61.27	0\\
61.28	0\\
61.29	0\\
61.3	0\\
61.31	0\\
61.32	0\\
61.33	0\\
61.34	0\\
61.35	0\\
61.36	0\\
61.37	0\\
61.38	0\\
61.39	0\\
61.4	0\\
61.41	0\\
61.42	0\\
61.43	0\\
61.44	0\\
61.45	0\\
61.46	0\\
61.47	0\\
61.48	0\\
61.49	0\\
61.5	0\\
61.51	0\\
61.52	0\\
61.53	0\\
61.54	0\\
61.55	0\\
61.56	0\\
61.57	0\\
61.58	0\\
61.59	0\\
61.6	0\\
61.61	0\\
61.62	0\\
61.63	0\\
61.64	0\\
61.65	0\\
61.66	0\\
61.67	0\\
61.68	0\\
61.69	0\\
61.7	0\\
61.71	0\\
61.72	0\\
61.73	0\\
61.74	0\\
61.75	0\\
61.76	0\\
61.77	0\\
61.78	0\\
61.79	0\\
61.8	0\\
61.81	0\\
61.82	0\\
61.83	0\\
61.84	0\\
61.85	0\\
61.86	0\\
61.87	0\\
61.88	0\\
61.89	0\\
61.9	0\\
61.91	0\\
61.92	0\\
61.93	0\\
61.94	0\\
61.95	0\\
61.96	0\\
61.97	0\\
61.98	0\\
61.99	0\\
62	0\\
62.01	0\\
62.02	0\\
62.03	0\\
62.04	0\\
62.05	0\\
62.06	0\\
62.07	0\\
62.08	0\\
62.09	0\\
62.1	0\\
62.11	0\\
62.12	0\\
62.13	0\\
62.14	0\\
62.15	0\\
62.16	0\\
62.17	0\\
62.18	0\\
62.19	0\\
62.2	0\\
62.21	0\\
62.22	0\\
62.23	0\\
62.24	0\\
62.25	0\\
62.26	0\\
62.27	0\\
62.28	0\\
62.29	0\\
62.3	0\\
62.31	0\\
62.32	0\\
62.33	0\\
62.34	0\\
62.35	0\\
62.36	0\\
62.37	0\\
62.38	0\\
62.39	0\\
62.4	0\\
62.41	0\\
62.42	0\\
62.43	0\\
62.44	0\\
62.45	0\\
62.46	0\\
62.47	0\\
62.48	0\\
62.49	0\\
62.5	0\\
62.51	0\\
62.52	0\\
62.53	0\\
62.54	0\\
62.55	0\\
62.56	0\\
62.57	0\\
62.58	0\\
62.59	0\\
62.6	0\\
62.61	0\\
62.62	0\\
62.63	0\\
62.64	0\\
62.65	0\\
62.66	0\\
62.67	0\\
62.68	0\\
62.69	0\\
62.7	0\\
62.71	0\\
62.72	0\\
62.73	0\\
62.74	0\\
62.75	0\\
62.76	0\\
62.77	0\\
62.78	0\\
62.79	0\\
62.8	0\\
62.81	0\\
62.82	0\\
62.83	0\\
62.84	0\\
62.85	0\\
62.86	0\\
62.87	0\\
62.88	0\\
62.89	0\\
62.9	0\\
62.91	0\\
62.92	0\\
62.93	0\\
62.94	0\\
62.95	0\\
62.96	0\\
62.97	0\\
62.98	0\\
62.99	0\\
63	0\\
63.01	0\\
63.02	0\\
63.03	0\\
63.04	0\\
63.05	0\\
63.06	0\\
63.07	0\\
63.08	0\\
63.09	0\\
63.1	0\\
63.11	0\\
63.12	0\\
63.13	0\\
63.14	0\\
63.15	0\\
63.16	0\\
63.17	0\\
63.18	0\\
63.19	0\\
63.2	0\\
63.21	0\\
63.22	0\\
63.23	0\\
63.24	0\\
63.25	0\\
63.26	0\\
63.27	0\\
63.28	0\\
63.29	0\\
63.3	0\\
63.31	0\\
63.32	0\\
63.33	0\\
63.34	0\\
63.35	0\\
63.36	0\\
63.37	0\\
63.38	0\\
63.39	0\\
63.4	0\\
63.41	0\\
63.42	0\\
63.43	0\\
63.44	0\\
63.45	0\\
63.46	0\\
63.47	0\\
63.48	0\\
63.49	0\\
63.5	0\\
63.51	0\\
63.52	0\\
63.53	0\\
63.54	0\\
63.55	0\\
63.56	0\\
63.57	0\\
63.58	0\\
63.59	0\\
63.6	0\\
63.61	0\\
63.62	0\\
63.63	0\\
63.64	0\\
63.65	0\\
63.66	0\\
63.67	0\\
63.68	0\\
63.69	0\\
63.7	0\\
63.71	0\\
63.72	0\\
63.73	0\\
63.74	0\\
63.75	0\\
63.76	0\\
63.77	0\\
63.78	0\\
63.79	0\\
63.8	0\\
63.81	0\\
63.82	0\\
63.83	0\\
63.84	0\\
63.85	0\\
63.86	0\\
63.87	0\\
63.88	0\\
63.89	0\\
63.9	0\\
63.91	0\\
63.92	0\\
63.93	0\\
63.94	0\\
63.95	0\\
63.96	0\\
63.97	0\\
63.98	0\\
63.99	0\\
64	0\\
64.01	0\\
64.02	0\\
64.03	0\\
64.04	0\\
64.05	0\\
64.06	0\\
64.07	0\\
64.08	0\\
64.09	0\\
64.1	0\\
64.11	0\\
64.12	0\\
64.13	0\\
64.14	0\\
64.15	0\\
64.16	0\\
64.17	0\\
64.18	0\\
64.19	0\\
64.2	0\\
64.21	0\\
64.22	0\\
64.23	0\\
64.24	0\\
64.25	0\\
64.26	0\\
64.27	0\\
64.28	0\\
64.29	0\\
64.3	0\\
64.31	0\\
64.32	0\\
64.33	0\\
64.34	0\\
64.35	0\\
64.36	0\\
64.37	0\\
64.38	0\\
64.39	0\\
64.4	0\\
64.41	0\\
64.42	0\\
64.43	0\\
64.44	0\\
64.45	0\\
64.46	0\\
64.47	0\\
64.48	0\\
64.49	0\\
64.5	0\\
64.51	0\\
64.52	0\\
64.53	0\\
64.54	0\\
64.55	0\\
64.56	0\\
64.57	0\\
64.58	0\\
64.59	0\\
64.6	0\\
64.61	0\\
64.62	0\\
64.63	0\\
64.64	0\\
64.65	0\\
64.66	0\\
64.67	0\\
64.68	0\\
64.69	0\\
64.7	0\\
64.71	0\\
64.72	0\\
64.73	0\\
64.74	0\\
64.75	0\\
64.76	0\\
64.77	0\\
64.78	0\\
64.79	0\\
64.8	0\\
64.81	0\\
64.82	0\\
64.83	0\\
64.84	0\\
64.85	0\\
64.86	0\\
64.87	0\\
64.88	0\\
64.89	0\\
64.9	0\\
64.91	0\\
64.92	0\\
64.93	0\\
64.94	0\\
64.95	0\\
64.96	0\\
64.97	0\\
64.98	0\\
64.99	0\\
65	0\\
65.01	0\\
65.02	0\\
65.03	0\\
65.04	0\\
65.05	0\\
65.06	0\\
65.07	0\\
65.08	0\\
65.09	0\\
65.1	0\\
65.11	0\\
65.12	0\\
65.13	0\\
65.14	0\\
65.15	0\\
65.16	0\\
65.17	0\\
65.18	0\\
65.19	0\\
65.2	0\\
65.21	0\\
65.22	0\\
65.23	0\\
65.24	0\\
65.25	0\\
65.26	0\\
65.27	0\\
65.28	0\\
65.29	0\\
65.3	0\\
65.31	0\\
65.32	0\\
65.33	0\\
65.34	0\\
65.35	0\\
65.36	0\\
65.37	0\\
65.38	0\\
65.39	0\\
65.4	0\\
65.41	0\\
65.42	0\\
65.43	0\\
65.44	0\\
65.45	0\\
65.46	0\\
65.47	0\\
65.48	0\\
65.49	0\\
65.5	0\\
65.51	0\\
65.52	0\\
65.53	0\\
65.54	0\\
65.55	0\\
65.56	0\\
65.57	0\\
65.58	0\\
65.59	0\\
65.6	0\\
65.61	0\\
65.62	0\\
65.63	0\\
65.64	0\\
65.65	0\\
65.66	0\\
65.67	0\\
65.68	0\\
65.69	0\\
65.7	0\\
65.71	0\\
65.72	0\\
65.73	0\\
65.74	0\\
65.75	0\\
65.76	0\\
65.77	0\\
65.78	0\\
65.79	0\\
65.8	0\\
65.81	0\\
65.82	0\\
65.83	0\\
65.84	0\\
65.85	0\\
65.86	0\\
65.87	0\\
65.88	0\\
65.89	0\\
65.9	0\\
65.91	0\\
65.92	0\\
65.93	0\\
65.94	0\\
65.95	0\\
65.96	0\\
65.97	0\\
65.98	0\\
65.99	0\\
66	0\\
66.01	0\\
66.02	0\\
66.03	0\\
66.04	0\\
66.05	0\\
66.06	0\\
66.07	0\\
66.08	0\\
66.09	0\\
66.1	0\\
66.11	0\\
66.12	0\\
66.13	0\\
66.14	0\\
66.15	0\\
66.16	0\\
66.17	0\\
66.18	0\\
66.19	0\\
66.2	0\\
66.21	0\\
66.22	0\\
66.23	0\\
66.24	0\\
66.25	0\\
66.26	0\\
66.27	0\\
66.28	0\\
66.29	0\\
66.3	0\\
66.31	0\\
66.32	0\\
66.33	0\\
66.34	0\\
66.35	0\\
66.36	0\\
66.37	0\\
66.38	0\\
66.39	0\\
66.4	0\\
66.41	0\\
66.42	0\\
66.43	0\\
66.44	0\\
66.45	0\\
66.46	0\\
66.47	0\\
66.48	0\\
66.49	0\\
66.5	0\\
66.51	0\\
66.52	0\\
66.53	0\\
66.54	0\\
66.55	0\\
66.56	0\\
66.57	0\\
66.58	0\\
66.59	0\\
66.6	0\\
66.61	0\\
66.62	0\\
66.63	0\\
66.64	0\\
66.65	0\\
66.66	0\\
66.67	0\\
66.68	0\\
66.69	0\\
66.7	0\\
66.71	0\\
66.72	0\\
66.73	0\\
66.74	0\\
66.75	0\\
66.76	0\\
66.77	0\\
66.78	0\\
66.79	0\\
66.8	0\\
66.81	0\\
66.82	0\\
66.83	0\\
66.84	0\\
66.85	0\\
66.86	0\\
66.87	0\\
66.88	0\\
66.89	0\\
66.9	0\\
66.91	0\\
66.92	0\\
66.93	0\\
66.94	0\\
66.95	0\\
66.96	0\\
66.97	0\\
66.98	0\\
66.99	0\\
67	0\\
67.01	0\\
67.02	0\\
67.03	0\\
67.04	0\\
67.05	0\\
67.06	0\\
67.07	0\\
67.08	0\\
67.09	0\\
67.1	0\\
67.11	0\\
67.12	0\\
67.13	0\\
67.14	0\\
67.15	0\\
67.16	0\\
67.17	0\\
67.18	0\\
67.19	0\\
67.2	0\\
67.21	0\\
67.22	0\\
67.23	0\\
67.24	0\\
67.25	0\\
67.26	0\\
67.27	0\\
67.28	0\\
67.29	0\\
67.3	0\\
67.31	0\\
67.32	0\\
67.33	0\\
67.34	0\\
67.35	0\\
67.36	0\\
67.37	0\\
67.38	0\\
67.39	0\\
67.4	0\\
67.41	0\\
67.42	0\\
67.43	0\\
67.44	0\\
67.45	0\\
67.46	0\\
67.47	0\\
67.48	0\\
67.49	0\\
67.5	0\\
67.51	0\\
67.52	0\\
67.53	0\\
67.54	0\\
67.55	0\\
67.56	0\\
67.57	0\\
67.58	0\\
67.59	0\\
67.6	0\\
67.61	0\\
67.62	0\\
67.63	0\\
67.64	0\\
67.65	0\\
67.66	0\\
67.67	0\\
67.68	0\\
67.69	0\\
67.7	0\\
67.71	0\\
67.72	0\\
67.73	0\\
67.74	0\\
67.75	0\\
67.76	0\\
67.77	0\\
67.78	0\\
67.79	0\\
67.8	0\\
67.81	0\\
67.82	0\\
67.83	0\\
67.84	0\\
67.85	0\\
67.86	0\\
67.87	0\\
67.88	0\\
67.89	0\\
67.9	0\\
67.91	0\\
67.92	0\\
67.93	0\\
67.94	0\\
67.95	0\\
67.96	0\\
67.97	0\\
67.98	0\\
67.99	0\\
68	0\\
68.01	0\\
68.02	0\\
68.03	0\\
68.04	0\\
68.05	0\\
68.06	0\\
68.07	0\\
68.08	0\\
68.09	0\\
68.1	0\\
68.11	0\\
68.12	0\\
68.13	0\\
68.14	0\\
68.15	0\\
68.16	0\\
68.17	0\\
68.18	0\\
68.19	0\\
68.2	0\\
68.21	0\\
68.22	0\\
68.23	0\\
68.24	0\\
68.25	0\\
68.26	0\\
68.27	0\\
68.28	0\\
68.29	0\\
68.3	0\\
68.31	0\\
68.32	0\\
68.33	0\\
68.34	0\\
68.35	0\\
68.36	0\\
68.37	0\\
68.38	0\\
68.39	0\\
68.4	0\\
68.41	0\\
68.42	0\\
68.43	0\\
68.44	0\\
68.45	0\\
68.46	0\\
68.47	0\\
68.48	0\\
68.49	0\\
68.5	0\\
68.51	0\\
68.52	0\\
68.53	0\\
68.54	0\\
68.55	0\\
68.56	0\\
68.57	0\\
68.58	0\\
68.59	0\\
68.6	0\\
68.61	0\\
68.62	0\\
68.63	0\\
68.64	0\\
68.65	0\\
68.66	0\\
68.67	0\\
68.68	0\\
68.69	0\\
68.7	0\\
68.71	0\\
68.72	0\\
68.73	0\\
68.74	0\\
68.75	0\\
68.76	0\\
68.77	0\\
68.78	0\\
68.79	0\\
68.8	0\\
68.81	0\\
68.82	0\\
68.83	0\\
68.84	0\\
68.85	0\\
68.86	0\\
68.87	0\\
68.88	0\\
68.89	0\\
68.9	0\\
68.91	0\\
68.92	0\\
68.93	0\\
68.94	0\\
68.95	0\\
68.96	0\\
68.97	0\\
68.98	0\\
68.99	0\\
69	0\\
69.01	0\\
69.02	0\\
69.03	0\\
69.04	0\\
69.05	0\\
69.06	0\\
69.07	0\\
69.08	0\\
69.09	0\\
69.1	0\\
69.11	0\\
69.12	0\\
69.13	0\\
69.14	0\\
69.15	0\\
69.16	0\\
69.17	0\\
69.18	0\\
69.19	0\\
69.2	0\\
69.21	0\\
69.22	0\\
69.23	0\\
69.24	0\\
69.25	0\\
69.26	0\\
69.27	0\\
69.28	0\\
69.29	0\\
69.3	0\\
69.31	0\\
69.32	0\\
69.33	0\\
69.34	0\\
69.35	0\\
69.36	0\\
69.37	0\\
69.38	0\\
69.39	0\\
69.4	0\\
69.41	0\\
69.42	0\\
69.43	0\\
69.44	0\\
69.45	0\\
69.46	0\\
69.47	0\\
69.48	0\\
69.49	0\\
69.5	0\\
69.51	0\\
69.52	0\\
69.53	0\\
69.54	0\\
69.55	0\\
69.56	0\\
69.57	0\\
69.58	0\\
69.59	0\\
69.6	0\\
69.61	0\\
69.62	0\\
69.63	0\\
69.64	0\\
69.65	0\\
69.66	0\\
69.67	0\\
69.68	0\\
69.69	0\\
69.7	0\\
69.71	0\\
69.72	0\\
69.73	0\\
69.74	0\\
69.75	0\\
69.76	0\\
69.77	0\\
69.78	0\\
69.79	0\\
69.8	0\\
69.81	0\\
69.82	0\\
69.83	0\\
69.84	0\\
69.85	0\\
69.86	0\\
69.87	0\\
69.88	0\\
69.89	0\\
69.9	0\\
69.91	0\\
69.92	0\\
69.93	0\\
69.94	0\\
69.95	0\\
69.96	0\\
69.97	0\\
69.98	0\\
69.99	0\\
70	0\\
70.01	0\\
70.02	0\\
70.03	0\\
70.04	0\\
70.05	0\\
70.06	0\\
70.07	0\\
70.08	0\\
70.09	0\\
70.1	0\\
70.11	0\\
70.12	0\\
70.13	0\\
70.14	0\\
70.15	0\\
70.16	0\\
70.17	0\\
70.18	0\\
70.19	0\\
70.2	0\\
70.21	0\\
70.22	0\\
70.23	0\\
70.24	0\\
70.25	0\\
70.26	0\\
70.27	0\\
70.28	0\\
70.29	0\\
70.3	0\\
70.31	0\\
70.32	0\\
70.33	0\\
70.34	0\\
70.35	0\\
70.36	0\\
70.37	0\\
70.38	0\\
70.39	0\\
70.4	0\\
70.41	0\\
70.42	0\\
70.43	0\\
70.44	0\\
70.45	0\\
70.46	0\\
70.47	0\\
70.48	0\\
70.49	0\\
70.5	0\\
70.51	0\\
70.52	0\\
70.53	0\\
70.54	0\\
70.55	0\\
70.56	0\\
70.57	0\\
70.58	0\\
70.59	0\\
70.6	0\\
70.61	0\\
70.62	0\\
70.63	0\\
70.64	0\\
70.65	0\\
70.66	0\\
70.67	0\\
70.68	0\\
70.69	0\\
70.7	0\\
70.71	0\\
70.72	0\\
70.73	0\\
70.74	0\\
70.75	0\\
70.76	0\\
70.77	0\\
70.78	0\\
70.79	0\\
70.8	0\\
70.81	0\\
70.82	0\\
70.83	0\\
70.84	0\\
70.85	0\\
70.86	0\\
70.87	0\\
70.88	0\\
70.89	0\\
70.9	0\\
70.91	0\\
70.92	0\\
70.93	0\\
70.94	0\\
70.95	0\\
70.96	0\\
70.97	0\\
70.98	0\\
70.99	0\\
71	0\\
71.01	0\\
71.02	0\\
71.03	0\\
71.04	0\\
71.05	0\\
71.06	0\\
71.07	0\\
71.08	0\\
71.09	0\\
71.1	0\\
71.11	0\\
71.12	0\\
71.13	0\\
71.14	0\\
71.15	0\\
71.16	0\\
71.17	0\\
71.18	0\\
71.19	0\\
71.2	0\\
71.21	0\\
71.22	0\\
71.23	0\\
71.24	0\\
71.25	0\\
71.26	0\\
71.27	0\\
71.28	0\\
71.29	0\\
71.3	0\\
71.31	0\\
71.32	0\\
71.33	0\\
71.34	0\\
71.35	0\\
71.36	0\\
71.37	0\\
71.38	0\\
71.39	0\\
71.4	0\\
71.41	0\\
71.42	0\\
71.43	0\\
71.44	0\\
71.45	0\\
71.46	0\\
71.47	0\\
71.48	0\\
71.49	0\\
71.5	0\\
71.51	0\\
71.52	0\\
71.53	0\\
71.54	0\\
71.55	0\\
71.56	0\\
71.57	0\\
71.58	0\\
71.59	0\\
71.6	0\\
71.61	0\\
71.62	0\\
71.63	0\\
71.64	0\\
71.65	0\\
71.66	0\\
71.67	0\\
71.68	0\\
71.69	0\\
71.7	0\\
71.71	0\\
71.72	0\\
71.73	0\\
71.74	0\\
71.75	0\\
71.76	0\\
71.77	0\\
71.78	0\\
71.79	0\\
71.8	0\\
71.81	0\\
71.82	0\\
71.83	0\\
71.84	0\\
71.85	0\\
71.86	0\\
71.87	0\\
71.88	0\\
71.89	0\\
71.9	0\\
71.91	0\\
71.92	0\\
71.93	0\\
71.94	0\\
71.95	0\\
71.96	0\\
71.97	0\\
71.98	0\\
71.99	0\\
72	0\\
72.01	0\\
72.02	0\\
72.03	0\\
72.04	0\\
72.05	0\\
72.06	0\\
72.07	0\\
72.08	0\\
72.09	0\\
72.1	0\\
72.11	0\\
72.12	0\\
72.13	0\\
72.14	0\\
72.15	0\\
72.16	0\\
72.17	0\\
72.18	0\\
72.19	0\\
72.2	0\\
72.21	0\\
72.22	0\\
72.23	0\\
72.24	0\\
72.25	0\\
72.26	0\\
72.27	0\\
72.28	0\\
72.29	0\\
72.3	0\\
72.31	0\\
72.32	0\\
72.33	0\\
72.34	0\\
72.35	0\\
72.36	0\\
72.37	0\\
72.38	0\\
72.39	0\\
72.4	0\\
72.41	0\\
72.42	0\\
72.43	0\\
72.44	0\\
72.45	0\\
72.46	0\\
72.47	0\\
72.48	0\\
72.49	0\\
72.5	0\\
72.51	0\\
72.52	0\\
72.53	0\\
72.54	0\\
72.55	0\\
72.56	0\\
72.57	0\\
72.58	0\\
72.59	0\\
72.6	0\\
72.61	0\\
72.62	0\\
72.63	0\\
72.64	0\\
72.65	0\\
72.66	0\\
72.67	0\\
72.68	0\\
72.69	0\\
72.7	0\\
72.71	0\\
72.72	0\\
72.73	0\\
72.74	0\\
72.75	0\\
72.76	0\\
72.77	0\\
72.78	0\\
72.79	0\\
72.8	0\\
72.81	0\\
72.82	0\\
72.83	0\\
72.84	0\\
72.85	0\\
72.86	0\\
72.87	0\\
72.88	0\\
72.89	0\\
72.9	0\\
72.91	0\\
72.92	0\\
72.93	0\\
72.94	0\\
72.95	0\\
72.96	0\\
72.97	0\\
72.98	0\\
72.99	0\\
73	0\\
73.01	0\\
73.02	0\\
73.03	0\\
73.04	0\\
73.05	0\\
73.06	0\\
73.07	0\\
73.08	0\\
73.09	0\\
73.1	0\\
73.11	0\\
73.12	0\\
73.13	0\\
73.14	0\\
73.15	0\\
73.16	0\\
73.17	0\\
73.18	0\\
73.19	0\\
73.2	0\\
73.21	0\\
73.22	0\\
73.23	0\\
73.24	0\\
73.25	0\\
73.26	0\\
73.27	0\\
73.28	0\\
73.29	0\\
73.3	0\\
73.31	0\\
73.32	0\\
73.33	0\\
73.34	0\\
73.35	0\\
73.36	0\\
73.37	0\\
73.38	0\\
73.39	0\\
73.4	0\\
73.41	0\\
73.42	0\\
73.43	0\\
73.44	0\\
73.45	0\\
73.46	0\\
73.47	0\\
73.48	0\\
73.49	0\\
73.5	0\\
73.51	0\\
73.52	0\\
73.53	0\\
73.54	0\\
73.55	0\\
73.56	0\\
73.57	0\\
73.58	0\\
73.59	0\\
73.6	0\\
73.61	0\\
73.62	0\\
73.63	0\\
73.64	0\\
73.65	0\\
73.66	0\\
73.67	0\\
73.68	0\\
73.69	0\\
73.7	0\\
73.71	0\\
73.72	0\\
73.73	0\\
73.74	0\\
73.75	0\\
73.76	0\\
73.77	0\\
73.78	0\\
73.79	0\\
73.8	0\\
73.81	0\\
73.82	0\\
73.83	0\\
73.84	0\\
73.85	0\\
73.86	0\\
73.87	0\\
73.88	0\\
73.89	0\\
73.9	0\\
73.91	0\\
73.92	0\\
73.93	0\\
73.94	0\\
73.95	0\\
73.96	0\\
73.97	0\\
73.98	0\\
73.99	0\\
74	0\\
74.01	0\\
74.02	0\\
74.03	0\\
74.04	0\\
74.05	0\\
74.06	0\\
74.07	0\\
74.08	0\\
74.09	0\\
74.1	0\\
74.11	0\\
74.12	0\\
74.13	0\\
74.14	0\\
74.15	0\\
74.16	0\\
74.17	0\\
74.18	0\\
74.19	0\\
74.2	0\\
74.21	0\\
74.22	0\\
74.23	0\\
74.24	0\\
74.25	0\\
74.26	0\\
74.27	0\\
74.28	0\\
74.29	0\\
74.3	0\\
74.31	0\\
74.32	0\\
74.33	0\\
74.34	0\\
74.35	0\\
74.36	0\\
74.37	0\\
74.38	0\\
74.39	0\\
74.4	0\\
74.41	0\\
74.42	0\\
74.43	0\\
74.44	0\\
74.45	0\\
74.46	0\\
74.47	0\\
74.48	0\\
74.49	0\\
74.5	0\\
74.51	0\\
74.52	0\\
74.53	0\\
74.54	0\\
74.55	0\\
74.56	0\\
74.57	0\\
74.58	0\\
74.59	0\\
74.6	0\\
74.61	0\\
74.62	0\\
74.63	0\\
74.64	0\\
74.65	0\\
74.66	0\\
74.67	0\\
74.68	0\\
74.69	0\\
74.7	0\\
74.71	0\\
74.72	0\\
74.73	0\\
74.74	0\\
74.75	0\\
74.76	0\\
74.77	0\\
74.78	0\\
74.79	0\\
74.8	0\\
74.81	0\\
74.82	0\\
74.83	0\\
74.84	0\\
74.85	0\\
74.86	0\\
74.87	0\\
74.88	0\\
74.89	0\\
74.9	0\\
74.91	0\\
74.92	0\\
74.93	0\\
74.94	0\\
74.95	0\\
74.96	0\\
74.97	0\\
74.98	0\\
74.99	0\\
75	0\\
75.01	0\\
75.02	0\\
75.03	0\\
75.04	0\\
75.05	0\\
75.06	0\\
75.07	0\\
75.08	0\\
75.09	0\\
75.1	0\\
75.11	0\\
75.12	0\\
75.13	0\\
75.14	0\\
75.15	0\\
75.16	0\\
75.17	0\\
75.18	0\\
75.19	0\\
75.2	0\\
75.21	0\\
75.22	0\\
75.23	0\\
75.24	0\\
75.25	0\\
75.26	0\\
75.27	0\\
75.28	0\\
75.29	0\\
75.3	0\\
75.31	0\\
75.32	0\\
75.33	0\\
75.34	0\\
75.35	0\\
75.36	0\\
75.37	0\\
75.38	0\\
75.39	0\\
75.4	0\\
75.41	0\\
75.42	0\\
75.43	0\\
75.44	0\\
75.45	0\\
75.46	0\\
75.47	0\\
75.48	0\\
75.49	0\\
75.5	0\\
75.51	0\\
75.52	0\\
75.53	0\\
75.54	0\\
75.55	0\\
75.56	0\\
75.57	0\\
75.58	0\\
75.59	0\\
75.6	0\\
75.61	0\\
75.62	0\\
75.63	0\\
75.64	0\\
75.65	0\\
75.66	0\\
75.67	0\\
75.68	0\\
75.69	0\\
75.7	0\\
75.71	0\\
75.72	0\\
75.73	0\\
75.74	0\\
75.75	0\\
75.76	0\\
75.77	0\\
75.78	0\\
75.79	0\\
75.8	0\\
75.81	0\\
75.82	0\\
75.83	0\\
75.84	0\\
75.85	0\\
75.86	0\\
75.87	0\\
75.88	0\\
75.89	0\\
75.9	0\\
75.91	0\\
75.92	0\\
75.93	0\\
75.94	0\\
75.95	0\\
75.96	0\\
75.97	0\\
75.98	0\\
75.99	0\\
76	0\\
76.01	0\\
76.02	0\\
76.03	0\\
76.04	0\\
76.05	0\\
76.06	0\\
76.07	0\\
76.08	0\\
76.09	0\\
76.1	0\\
76.11	0\\
76.12	0\\
76.13	0\\
76.14	0\\
76.15	0\\
76.16	0\\
76.17	0\\
76.18	0\\
76.19	0\\
76.2	0\\
76.21	0\\
76.22	0\\
76.23	0\\
76.24	0\\
76.25	0\\
76.26	0\\
76.27	0\\
76.28	0\\
76.29	0\\
76.3	0\\
76.31	0\\
76.32	0\\
76.33	0\\
76.34	0\\
76.35	0\\
76.36	0\\
76.37	0\\
76.38	0\\
76.39	0\\
76.4	0\\
76.41	0\\
76.42	0\\
76.43	0\\
76.44	0\\
76.45	0\\
76.46	0\\
76.47	0\\
76.48	0\\
76.49	0\\
76.5	0\\
76.51	0\\
76.52	0\\
76.53	0\\
76.54	0\\
76.55	0\\
76.56	0\\
76.57	0\\
76.58	0\\
76.59	0\\
76.6	0\\
76.61	0\\
76.62	0\\
76.63	0\\
76.64	0\\
76.65	0\\
76.66	0\\
76.67	0\\
76.68	0\\
76.69	0\\
76.7	0\\
76.71	0\\
76.72	0\\
76.73	0\\
76.74	0\\
76.75	0\\
76.76	0\\
76.77	0\\
76.78	0\\
76.79	0\\
76.8	0\\
76.81	0\\
76.82	0\\
76.83	0\\
76.84	0\\
76.85	0\\
76.86	0\\
76.87	0\\
76.88	0\\
76.89	0\\
76.9	0\\
76.91	0\\
76.92	0\\
76.93	0\\
76.94	0\\
76.95	0\\
76.96	0\\
76.97	0\\
76.98	0\\
76.99	0\\
77	0\\
77.01	0\\
77.02	0\\
77.03	0\\
77.04	0\\
77.05	0\\
77.06	0\\
77.07	0\\
77.08	0\\
77.09	0\\
77.1	0\\
77.11	0\\
77.12	0\\
77.13	0\\
77.14	0\\
77.15	0\\
77.16	0\\
77.17	0\\
77.18	0\\
77.19	0\\
77.2	0\\
77.21	0\\
77.22	0\\
77.23	0\\
77.24	0\\
77.25	0\\
77.26	0\\
77.27	0\\
77.28	0\\
77.29	0\\
77.3	0\\
77.31	0\\
77.32	0\\
77.33	0\\
77.34	0\\
77.35	0\\
77.36	0\\
77.37	0\\
77.38	0\\
77.39	0\\
77.4	0\\
77.41	0\\
77.42	0\\
77.43	0\\
77.44	0\\
77.45	0\\
77.46	0\\
77.47	0\\
77.48	0\\
77.49	0\\
77.5	0\\
77.51	0\\
77.52	0\\
77.53	0\\
77.54	0\\
77.55	0\\
77.56	0\\
77.57	0\\
77.58	0\\
77.59	0\\
77.6	0\\
77.61	0\\
77.62	0\\
77.63	0\\
77.64	0\\
77.65	0\\
77.66	0\\
77.67	0\\
77.68	0\\
77.69	0\\
77.7	0\\
77.71	0\\
77.72	0\\
77.73	0\\
77.74	0\\
77.75	0\\
77.76	0\\
77.77	0\\
77.78	0\\
77.79	0\\
77.8	0\\
77.81	0\\
77.82	0\\
77.83	0\\
77.84	0\\
77.85	0\\
77.86	0\\
77.87	0\\
77.88	0\\
77.89	0\\
77.9	0\\
77.91	0\\
77.92	0\\
77.93	0\\
77.94	0\\
77.95	0\\
77.96	0\\
77.97	0\\
77.98	0\\
77.99	0\\
78	0\\
78.01	0\\
78.02	0\\
78.03	0\\
78.04	0\\
78.05	0\\
78.06	0\\
78.07	0\\
78.08	0\\
78.09	0\\
78.1	0\\
78.11	0\\
78.12	0\\
78.13	0\\
78.14	0\\
78.15	0\\
78.16	0\\
78.17	0\\
78.18	0\\
78.19	0\\
78.2	0\\
78.21	0\\
78.22	0\\
78.23	0\\
78.24	0\\
78.25	0\\
78.26	0\\
78.27	0\\
78.28	0\\
78.29	0\\
78.3	0\\
78.31	0\\
78.32	0\\
78.33	0\\
78.34	0\\
78.35	0\\
78.36	0\\
78.37	0\\
78.38	0\\
78.39	0\\
78.4	0\\
78.41	0\\
78.42	0\\
78.43	0\\
78.44	0\\
78.45	0\\
78.46	0\\
78.47	0\\
78.48	0\\
78.49	0\\
78.5	0\\
78.51	0\\
78.52	0\\
78.53	0\\
78.54	0\\
78.55	0\\
78.56	0\\
78.57	0\\
78.58	0\\
78.59	0\\
78.6	0\\
78.61	0\\
78.62	0\\
78.63	0\\
78.64	0\\
78.65	0\\
78.66	0\\
78.67	0\\
78.68	0\\
78.69	0\\
78.7	0\\
78.71	0\\
78.72	0\\
78.73	0\\
78.74	0\\
78.75	0\\
78.76	0\\
78.77	0\\
78.78	0\\
78.79	0\\
78.8	0\\
78.81	0\\
78.82	0\\
78.83	0\\
78.84	0\\
78.85	0\\
78.86	0\\
78.87	0\\
78.88	0\\
78.89	0\\
78.9	0\\
78.91	0\\
78.92	0\\
78.93	0\\
78.94	0\\
78.95	0\\
78.96	0\\
78.97	0\\
78.98	0\\
78.99	0\\
79	0\\
79.01	0\\
79.02	0\\
79.03	0\\
79.04	0\\
79.05	0\\
79.06	0\\
79.07	0\\
79.08	0\\
79.09	0\\
79.1	0\\
79.11	0\\
79.12	0\\
79.13	0\\
79.14	0\\
79.15	0\\
79.16	0\\
79.17	0\\
79.18	0\\
79.19	0\\
79.2	0\\
79.21	0\\
79.22	0\\
79.23	0\\
79.24	0\\
79.25	0\\
79.26	0\\
79.27	0\\
79.28	0\\
79.29	0\\
79.3	0\\
79.31	0\\
79.32	0\\
79.33	0\\
79.34	0\\
79.35	0\\
79.36	0\\
79.37	0\\
79.38	0\\
79.39	0\\
79.4	0\\
79.41	0\\
79.42	0\\
79.43	0\\
79.44	0\\
79.45	0\\
79.46	0\\
79.47	0\\
79.48	0\\
79.49	0\\
79.5	0\\
79.51	0\\
79.52	0\\
79.53	0\\
79.54	0\\
79.55	0\\
79.56	0\\
79.57	0\\
79.58	0\\
79.59	0\\
79.6	0\\
79.61	0\\
79.62	0\\
79.63	0\\
79.64	0\\
79.65	0\\
79.66	0\\
79.67	0\\
79.68	0\\
79.69	0\\
79.7	0\\
79.71	0\\
79.72	0\\
79.73	0\\
79.74	0\\
79.75	0\\
79.76	0\\
79.77	0\\
79.78	0\\
79.79	0\\
79.8	0\\
79.81	0\\
79.82	0\\
79.83	0\\
79.84	0\\
79.85	0\\
79.86	0\\
79.87	0\\
79.88	0\\
79.89	0\\
79.9	0\\
79.91	0\\
79.92	0\\
79.93	0\\
79.94	0\\
79.95	0\\
79.96	0\\
79.97	0\\
79.98	0\\
79.99	0\\
80	0\\
80.01	0\\
};
\addplot [color=mycolor1,dashed]
  table[row sep=crcr]{%
80.01	0\\
80.02	0\\
80.03	0\\
80.04	0\\
80.05	0\\
80.06	0\\
80.07	0\\
80.08	0\\
80.09	0\\
80.1	0\\
80.11	0\\
80.12	0\\
80.13	0\\
80.14	0\\
80.15	0\\
80.16	0\\
80.17	0\\
80.18	0\\
80.19	0\\
80.2	0\\
80.21	0\\
80.22	0\\
80.23	0\\
80.24	0\\
80.25	0\\
80.26	0\\
80.27	0\\
80.28	0\\
80.29	0\\
80.3	0\\
80.31	0\\
80.32	0\\
80.33	0\\
80.34	0\\
80.35	0\\
80.36	0\\
80.37	0\\
80.38	0\\
80.39	0\\
80.4	0\\
80.41	0\\
80.42	0\\
80.43	0\\
80.44	0\\
80.45	0\\
80.46	0\\
80.47	0\\
80.48	0\\
80.49	0\\
80.5	0\\
80.51	0\\
80.52	0\\
80.53	0\\
80.54	0\\
80.55	0\\
80.56	0\\
80.57	0\\
80.58	0\\
80.59	0\\
80.6	0\\
80.61	0\\
80.62	0\\
80.63	0\\
80.64	0\\
80.65	0\\
80.66	0\\
80.67	0\\
80.68	0\\
80.69	0\\
80.7	0\\
80.71	0\\
80.72	0\\
80.73	0\\
80.74	0\\
80.75	0\\
80.76	0\\
80.77	0\\
80.78	0\\
80.79	0\\
80.8	0\\
80.81	0\\
80.82	0\\
80.83	0\\
80.84	0\\
80.85	0\\
80.86	0\\
80.87	0\\
80.88	0\\
80.89	0\\
80.9	0\\
80.91	0\\
80.92	0\\
80.93	0\\
80.94	0\\
80.95	0\\
80.96	0\\
80.97	0\\
80.98	0\\
80.99	0\\
81	0\\
81.01	0\\
81.02	0\\
81.03	0\\
81.04	0\\
81.05	0\\
81.06	0\\
81.07	0\\
81.08	0\\
81.09	0\\
81.1	0\\
81.11	0\\
81.12	0\\
81.13	0\\
81.14	0\\
81.15	0\\
81.16	0\\
81.17	0\\
81.18	0\\
81.19	0\\
81.2	0\\
81.21	0\\
81.22	0\\
81.23	0\\
81.24	0\\
81.25	0\\
81.26	0\\
81.27	0\\
81.28	0\\
81.29	0\\
81.3	0\\
81.31	0\\
81.32	0\\
81.33	0\\
81.34	0\\
81.35	0\\
81.36	0\\
81.37	0\\
81.38	0\\
81.39	0\\
81.4	0\\
81.41	0\\
81.42	0\\
81.43	0\\
81.44	0\\
81.45	0\\
81.46	0\\
81.47	0\\
81.48	0\\
81.49	0\\
81.5	0\\
81.51	0\\
81.52	0\\
81.53	0\\
81.54	0\\
81.55	0\\
81.56	0\\
81.57	0\\
81.58	0\\
81.59	0\\
81.6	0\\
81.61	0\\
81.62	0\\
81.63	0\\
81.64	0\\
81.65	0\\
81.66	0\\
81.67	0\\
81.68	0\\
81.69	0\\
81.7	0\\
81.71	0\\
81.72	0\\
81.73	0\\
81.74	0\\
81.75	0\\
81.76	0\\
81.77	0\\
81.78	0\\
81.79	0\\
81.8	0\\
81.81	0\\
81.82	0\\
81.83	0\\
81.84	0\\
81.85	0\\
81.86	0\\
81.87	0\\
81.88	0\\
81.89	0\\
81.9	0\\
81.91	0\\
81.92	0\\
81.93	0\\
81.94	0\\
81.95	0\\
81.96	0\\
81.97	0\\
81.98	0\\
81.99	0\\
82	0\\
82.01	0\\
82.02	0\\
82.03	0\\
82.04	0\\
82.05	0\\
82.06	0\\
82.07	0\\
82.08	0\\
82.09	0\\
82.1	0\\
82.11	0\\
82.12	0\\
82.13	0\\
82.14	0\\
82.15	0\\
82.16	0\\
82.17	0\\
82.18	0\\
82.19	0\\
82.2	0\\
82.21	0\\
82.22	0\\
82.23	0\\
82.24	0\\
82.25	0\\
82.26	0\\
82.27	0\\
82.28	0\\
82.29	0\\
82.3	0\\
82.31	0\\
82.32	0\\
82.33	0\\
82.34	0\\
82.35	0\\
82.36	0\\
82.37	0\\
82.38	0\\
82.39	0\\
82.4	0\\
82.41	0\\
82.42	0\\
82.43	0\\
82.44	0\\
82.45	0\\
82.46	0\\
82.47	0\\
82.48	0\\
82.49	0\\
82.5	0\\
82.51	0\\
82.52	0\\
82.53	0\\
82.54	0\\
82.55	0\\
82.56	0\\
82.57	0\\
82.58	0\\
82.59	0\\
82.6	0\\
82.61	0\\
82.62	0\\
82.63	0\\
82.64	0\\
82.65	0\\
82.66	0\\
82.67	0\\
82.68	0\\
82.69	0\\
82.7	0\\
82.71	0\\
82.72	0\\
82.73	0\\
82.74	0\\
82.75	0\\
82.76	0\\
82.77	0\\
82.78	0\\
82.79	0\\
82.8	0\\
82.81	0\\
82.82	0\\
82.83	0\\
82.84	0\\
82.85	0\\
82.86	0\\
82.87	0\\
82.88	0\\
82.89	0\\
82.9	0\\
82.91	0\\
82.92	0\\
82.93	0\\
82.94	0\\
82.95	0\\
82.96	0\\
82.97	0\\
82.98	0\\
82.99	0\\
83	0\\
83.01	0\\
83.02	0\\
83.03	0\\
83.04	0\\
83.05	0\\
83.06	0\\
83.07	0\\
83.08	0\\
83.09	0\\
83.1	0\\
83.11	0\\
83.12	0\\
83.13	0\\
83.14	0\\
83.15	0\\
83.16	0\\
83.17	0\\
83.18	0\\
83.19	0\\
83.2	0\\
83.21	0\\
83.22	0\\
83.23	0\\
83.24	0\\
83.25	0\\
83.26	0\\
83.27	0\\
83.28	0\\
83.29	0\\
83.3	0\\
83.31	0\\
83.32	0\\
83.33	0\\
83.34	0\\
83.35	0\\
83.36	0\\
83.37	0\\
83.38	0\\
83.39	0\\
83.4	0\\
83.41	0\\
83.42	0\\
83.43	0\\
83.44	0\\
83.45	0\\
83.46	0\\
83.47	0\\
83.48	0\\
83.49	0\\
83.5	0\\
83.51	0\\
83.52	0\\
83.53	0\\
83.54	0\\
83.55	0\\
83.56	0\\
83.57	0\\
83.58	0\\
83.59	0\\
83.6	0\\
83.61	0\\
83.62	0\\
83.63	0\\
83.64	0\\
83.65	0\\
83.66	0\\
83.67	0\\
83.68	0\\
83.69	0\\
83.7	0\\
83.71	0\\
83.72	0\\
83.73	0\\
83.74	0\\
83.75	0\\
83.76	0\\
83.77	0\\
83.78	0\\
83.79	0\\
83.8	0\\
83.81	0\\
83.82	0\\
83.83	0\\
83.84	0\\
83.85	0\\
83.86	0\\
83.87	0\\
83.88	0\\
83.89	0\\
83.9	0\\
83.91	0\\
83.92	0\\
83.93	0\\
83.94	0\\
83.95	0\\
83.96	0\\
83.97	0\\
83.98	0\\
83.99	0\\
84	0\\
84.01	0\\
84.02	0\\
84.03	0\\
84.04	0\\
84.05	0\\
84.06	0\\
84.07	0\\
84.08	0\\
84.09	0\\
84.1	0\\
84.11	0\\
84.12	0\\
84.13	0\\
84.14	0\\
84.15	0\\
84.16	0\\
84.17	0\\
84.18	0\\
84.19	0\\
84.2	0\\
84.21	0\\
84.22	0\\
84.23	0\\
84.24	0\\
84.25	0\\
84.26	0\\
84.27	0\\
84.28	0\\
84.29	0\\
84.3	0\\
84.31	0\\
84.32	0\\
84.33	0\\
84.34	0\\
84.35	0\\
84.36	0\\
84.37	0\\
84.38	0\\
84.39	0\\
84.4	0\\
84.41	0\\
84.42	0\\
84.43	0\\
84.44	0\\
84.45	0\\
84.46	0\\
84.47	0\\
84.48	0\\
84.49	0\\
84.5	0\\
84.51	0\\
84.52	0\\
84.53	0\\
84.54	0\\
84.55	0\\
84.56	0\\
84.57	0\\
84.58	0\\
84.59	0\\
84.6	0\\
84.61	0\\
84.62	0\\
84.63	0\\
84.64	0\\
84.65	0\\
84.66	0\\
84.67	0\\
84.68	0\\
84.69	0\\
84.7	0\\
84.71	0\\
84.72	0\\
84.73	0\\
84.74	0\\
84.75	0\\
84.76	0\\
84.77	0\\
84.78	0\\
84.79	0\\
84.8	0\\
84.81	0\\
84.82	0\\
84.83	0\\
84.84	0\\
84.85	0\\
84.86	0\\
84.87	0\\
84.88	0\\
84.89	0\\
84.9	0\\
84.91	0\\
84.92	0\\
84.93	0\\
84.94	0\\
84.95	0\\
84.96	0\\
84.97	0\\
84.98	0\\
84.99	0\\
85	0\\
85.01	0\\
85.02	0\\
85.03	0\\
85.04	0\\
85.05	0\\
85.06	0\\
85.07	0\\
85.08	0\\
85.09	0\\
85.1	0\\
85.11	0\\
85.12	0\\
85.13	0\\
85.14	0\\
85.15	0\\
85.16	0\\
85.17	0\\
85.18	0\\
85.19	0\\
85.2	0\\
85.21	0\\
85.22	0\\
85.23	0\\
85.24	0\\
85.25	0\\
85.26	0\\
85.27	0\\
85.28	0\\
85.29	0\\
85.3	0\\
85.31	0\\
85.32	0\\
85.33	0\\
85.34	0\\
85.35	0\\
85.36	0\\
85.37	0\\
85.38	0\\
85.39	0\\
85.4	0\\
85.41	0\\
85.42	0\\
85.43	0\\
85.44	0\\
85.45	0\\
85.46	0\\
85.47	0\\
85.48	0\\
85.49	0\\
85.5	0\\
85.51	0\\
85.52	0\\
85.53	0\\
85.54	0\\
85.55	0\\
85.56	0\\
85.57	0\\
85.58	0\\
85.59	0\\
85.6	0\\
85.61	0\\
85.62	0\\
85.63	0\\
85.64	0\\
85.65	0\\
85.66	0\\
85.67	0\\
85.68	0\\
85.69	0\\
85.7	0\\
85.71	0\\
85.72	0\\
85.73	0\\
85.74	0\\
85.75	0\\
85.76	0\\
85.77	0\\
85.78	0\\
85.79	0\\
85.8	0\\
85.81	0\\
85.82	0\\
85.83	0\\
85.84	0\\
85.85	0\\
85.86	0\\
85.87	0\\
85.88	0\\
85.89	0\\
85.9	0\\
85.91	0\\
85.92	0\\
85.93	0\\
85.94	0\\
85.95	0\\
85.96	0\\
85.97	0\\
85.98	0\\
85.99	0\\
86	0\\
86.01	0\\
86.02	0\\
86.03	0\\
86.04	0\\
86.05	0\\
86.06	0\\
86.07	0\\
86.08	0\\
86.09	0\\
86.1	0\\
86.11	0\\
86.12	0\\
86.13	0\\
86.14	0\\
86.15	0\\
86.16	0\\
86.17	0\\
86.18	0\\
86.19	0\\
86.2	0\\
86.21	0\\
86.22	0\\
86.23	0\\
86.24	0\\
86.25	0\\
86.26	0\\
86.27	0\\
86.28	0\\
86.29	0\\
86.3	0\\
86.31	0\\
86.32	0\\
86.33	0\\
86.34	0\\
86.35	0\\
86.36	0\\
86.37	0\\
86.38	0\\
86.39	0\\
86.4	0\\
86.41	0\\
86.42	0\\
86.43	0\\
86.44	0\\
86.45	0\\
86.46	0\\
86.47	0\\
86.48	0\\
86.49	0\\
86.5	0\\
86.51	0\\
86.52	0\\
86.53	0\\
86.54	0\\
86.55	0\\
86.56	0\\
86.57	0\\
86.58	0\\
86.59	0\\
86.6	0\\
86.61	0\\
86.62	0\\
86.63	0\\
86.64	0\\
86.65	0\\
86.66	0\\
86.67	0\\
86.68	0\\
86.69	0\\
86.7	0\\
86.71	0\\
86.72	0\\
86.73	0\\
86.74	0\\
86.75	0\\
86.76	0\\
86.77	0\\
86.78	0\\
86.79	0\\
86.8	0\\
86.81	0\\
86.82	0\\
86.83	0\\
86.84	0\\
86.85	0\\
86.86	0\\
86.87	0\\
86.88	0\\
86.89	0\\
86.9	0\\
86.91	0\\
86.92	0\\
86.93	0\\
86.94	0\\
86.95	0\\
86.96	0\\
86.97	0\\
86.98	0\\
86.99	0\\
87	0\\
87.01	0\\
87.02	0\\
87.03	0\\
87.04	0\\
87.05	0\\
87.06	0\\
87.07	0\\
87.08	0\\
87.09	0\\
87.1	0\\
87.11	0\\
87.12	0\\
87.13	0\\
87.14	0\\
87.15	0\\
87.16	0\\
87.17	0\\
87.18	0\\
87.19	0\\
87.2	0\\
87.21	0\\
87.22	0\\
87.23	0\\
87.24	0\\
87.25	0\\
87.26	0\\
87.27	0\\
87.28	0\\
87.29	0\\
87.3	0\\
87.31	0\\
87.32	0\\
87.33	0\\
87.34	0\\
87.35	0\\
87.36	0\\
87.37	0\\
87.38	0\\
87.39	0\\
87.4	0\\
87.41	0\\
87.42	0\\
87.43	0\\
87.44	0\\
87.45	0\\
87.46	0\\
87.47	0\\
87.48	0\\
87.49	0\\
87.5	0\\
87.51	0\\
87.52	0\\
87.53	0\\
87.54	0\\
87.55	0\\
87.56	0\\
87.57	0\\
87.58	0\\
87.59	0\\
87.6	0\\
87.61	0\\
87.62	0\\
87.63	0\\
87.64	0\\
87.65	0\\
87.66	0\\
87.67	0\\
87.68	0\\
87.69	0\\
87.7	0\\
87.71	0\\
87.72	0\\
87.73	0\\
87.74	0\\
87.75	0\\
87.76	0\\
87.77	0\\
87.78	0\\
87.79	0\\
87.8	0\\
87.81	0\\
87.82	0\\
87.83	0\\
87.84	0\\
87.85	0\\
87.86	0\\
87.87	0\\
87.88	0\\
87.89	0\\
87.9	0\\
87.91	0\\
87.92	0\\
87.93	0\\
87.94	0\\
87.95	0\\
87.96	0\\
87.97	0\\
87.98	0\\
87.99	0\\
88	0\\
88.01	0\\
88.02	0\\
88.03	0\\
88.04	0\\
88.05	0\\
88.06	0\\
88.07	0\\
88.08	0\\
88.09	0\\
88.1	0\\
88.11	0\\
88.12	0\\
88.13	0\\
88.14	0\\
88.15	0\\
88.16	0\\
88.17	0\\
88.18	0\\
88.19	0\\
88.2	0\\
88.21	0\\
88.22	0\\
88.23	0\\
88.24	0\\
88.25	0\\
88.26	0\\
88.27	0\\
88.28	0\\
88.29	0\\
88.3	0\\
88.31	0\\
88.32	0\\
88.33	0\\
88.34	0\\
88.35	0\\
88.36	0\\
88.37	0\\
88.38	0\\
88.39	0\\
88.4	0\\
88.41	0\\
88.42	0\\
88.43	0\\
88.44	0\\
88.45	0\\
88.46	0\\
88.47	0\\
88.48	0\\
88.49	0\\
88.5	0\\
88.51	0\\
88.52	0\\
88.53	0\\
88.54	0\\
88.55	0\\
88.56	0\\
88.57	0\\
88.58	0\\
88.59	0\\
88.6	0\\
88.61	0\\
88.62	0\\
88.63	0\\
88.64	0\\
88.65	0\\
88.66	0\\
88.67	0\\
88.68	0\\
88.69	0\\
88.7	0\\
88.71	0\\
88.72	0\\
88.73	0\\
88.74	0\\
88.75	0\\
88.76	0\\
88.77	0\\
88.78	0\\
88.79	0\\
88.8	0\\
88.81	0\\
88.82	0\\
88.83	0\\
88.84	0\\
88.85	0\\
88.86	0\\
88.87	0\\
88.88	0\\
88.89	0\\
88.9	0\\
88.91	0\\
88.92	0\\
88.93	0\\
88.94	0\\
88.95	0\\
88.96	0\\
88.97	0\\
88.98	0\\
88.99	0\\
89	0\\
89.01	0\\
89.02	0\\
89.03	0\\
89.04	0\\
89.05	0\\
89.06	0\\
89.07	0\\
89.08	0\\
89.09	0\\
89.1	0\\
89.11	0\\
89.12	0\\
89.13	0\\
89.14	0\\
89.15	0\\
89.16	0\\
89.17	0\\
89.18	0\\
89.19	0\\
89.2	0\\
89.21	0\\
89.22	0\\
89.23	0\\
89.24	0\\
89.25	0\\
89.26	0\\
89.27	0\\
89.28	0\\
89.29	0\\
89.3	0\\
89.31	0\\
89.32	0\\
89.33	0\\
89.34	0\\
89.35	0\\
89.36	0\\
89.37	0\\
89.38	0\\
89.39	0\\
89.4	0\\
89.41	0\\
89.42	0\\
89.43	0\\
89.44	0\\
89.45	0\\
89.46	0\\
89.47	0\\
89.48	0\\
89.49	0\\
89.5	0\\
89.51	0\\
89.52	0\\
89.53	0\\
89.54	0\\
89.55	0\\
89.56	0\\
89.57	0\\
89.58	0\\
89.59	0\\
89.6	0\\
89.61	0\\
89.62	0\\
89.63	0\\
89.64	0\\
89.65	0\\
89.66	0\\
89.67	0\\
89.68	0\\
89.69	0\\
89.7	0\\
89.71	0\\
89.72	0\\
89.73	0\\
89.74	0\\
89.75	0\\
89.76	0\\
89.77	0\\
89.78	0\\
89.79	0\\
89.8	0\\
89.81	0\\
89.82	0\\
89.83	0\\
89.84	0\\
89.85	0\\
89.86	0\\
89.87	0\\
89.88	0\\
89.89	0\\
89.9	0\\
89.91	0\\
89.92	0\\
89.93	0\\
89.94	0\\
89.95	0\\
89.96	0\\
89.97	0\\
89.98	0\\
89.99	0\\
90	0\\
90.01	0\\
90.02	0\\
90.03	0\\
90.04	0\\
90.05	0\\
90.06	0\\
90.07	0\\
90.08	0\\
90.09	0\\
90.1	0\\
90.11	0\\
90.12	0\\
90.13	0\\
90.14	0\\
90.15	0\\
90.16	0\\
90.17	0\\
90.18	0\\
90.19	0\\
90.2	0\\
90.21	0\\
90.22	0\\
90.23	0\\
90.24	0\\
90.25	0\\
90.26	0\\
90.27	0\\
90.28	0\\
90.29	0\\
90.3	0\\
90.31	0\\
90.32	0\\
90.33	0\\
90.34	0\\
90.35	0\\
90.36	0\\
90.37	0\\
90.38	0\\
90.39	0\\
90.4	0\\
90.41	0\\
90.42	0\\
90.43	0\\
90.44	0\\
90.45	0\\
90.46	0\\
90.47	0\\
90.48	0\\
90.49	0\\
90.5	0\\
90.51	0\\
90.52	0\\
90.53	0\\
90.54	0\\
90.55	0\\
90.56	0\\
90.57	0\\
90.58	0\\
90.59	0\\
90.6	0\\
90.61	0\\
90.62	0\\
90.63	0\\
90.64	0\\
90.65	0\\
90.66	0\\
90.67	0\\
90.68	0\\
90.69	0\\
90.7	0\\
90.71	0\\
90.72	0\\
90.73	0\\
90.74	0\\
90.75	0\\
90.76	0\\
90.77	0\\
90.78	0\\
90.79	0\\
90.8	0\\
90.81	0\\
90.82	0\\
90.83	0\\
90.84	0\\
90.85	0\\
90.86	0\\
90.87	0\\
90.88	0\\
90.89	0\\
90.9	0\\
90.91	0\\
90.92	0\\
90.93	0\\
90.94	0\\
90.95	0\\
90.96	0\\
90.97	0\\
90.98	0\\
90.99	0\\
91	0\\
91.01	0\\
91.02	0\\
91.03	0\\
91.04	0\\
91.05	0\\
91.06	0\\
91.07	0\\
91.08	0\\
91.09	0\\
91.1	0\\
91.11	0\\
91.12	0\\
91.13	0\\
91.14	0\\
91.15	0\\
91.16	0\\
91.17	0\\
91.18	0\\
91.19	0\\
91.2	0\\
91.21	0\\
91.22	0\\
91.23	0\\
91.24	0\\
91.25	0\\
91.26	0\\
91.27	0\\
91.28	0\\
91.29	0\\
91.3	0\\
91.31	0\\
91.32	0\\
91.33	0\\
91.34	0\\
91.35	0\\
91.36	0\\
91.37	0\\
91.38	0\\
91.39	0\\
91.4	0\\
91.41	0\\
91.42	0\\
91.43	0\\
91.44	0\\
91.45	0\\
91.46	0\\
91.47	0\\
91.48	0\\
91.49	0\\
91.5	0\\
91.51	0\\
91.52	0\\
91.53	0\\
91.54	0\\
91.55	0\\
91.56	0\\
91.57	0\\
91.58	0\\
91.59	0\\
91.6	0\\
91.61	0\\
91.62	0\\
91.63	0\\
91.64	0\\
91.65	0\\
91.66	0\\
91.67	0\\
91.68	0\\
91.69	0\\
91.7	0\\
91.71	0\\
91.72	0\\
91.73	0\\
91.74	0\\
91.75	0\\
91.76	0\\
91.77	0\\
91.78	0\\
91.79	0\\
91.8	0\\
91.81	0\\
91.82	0\\
91.83	0\\
91.84	0\\
91.85	0\\
91.86	0\\
91.87	0\\
91.88	0\\
91.89	0\\
91.9	0\\
91.91	0\\
91.92	0\\
91.93	0\\
91.94	0\\
91.95	0\\
91.96	0\\
91.97	0\\
91.98	0\\
91.99	0\\
92	0\\
92.01	0\\
92.02	0\\
92.03	0\\
92.04	0\\
92.05	0\\
92.06	0\\
92.07	0\\
92.08	0\\
92.09	0\\
92.1	0\\
92.11	0\\
92.12	0\\
92.13	0\\
92.14	0\\
92.15	0\\
92.16	0\\
92.17	0\\
92.18	0\\
92.19	0\\
92.2	0\\
92.21	0\\
92.22	0\\
92.23	0\\
92.24	0\\
92.25	0\\
92.26	0\\
92.27	0\\
92.28	0\\
92.29	0\\
92.3	0\\
92.31	0\\
92.32	0\\
92.33	0\\
92.34	0\\
92.35	0\\
92.36	0\\
92.37	0\\
92.38	0\\
92.39	0\\
92.4	0\\
92.41	0\\
92.42	0\\
92.43	0\\
92.44	0\\
92.45	0\\
92.46	0\\
92.47	0\\
92.48	0\\
92.49	0\\
92.5	0\\
92.51	0\\
92.52	0\\
92.53	0\\
92.54	0\\
92.55	0\\
92.56	0\\
92.57	0\\
92.58	0\\
92.59	0\\
92.6	0\\
92.61	0\\
92.62	0\\
92.63	0\\
92.64	0\\
92.65	0\\
92.66	0\\
92.67	0\\
92.68	0\\
92.69	0\\
92.7	0\\
92.71	0\\
92.72	0\\
92.73	0\\
92.74	0\\
92.75	0\\
92.76	0\\
92.77	0\\
92.78	0\\
92.79	0\\
92.8	0\\
92.81	0\\
92.82	0\\
92.83	0\\
92.84	0\\
92.85	0\\
92.86	0\\
92.87	0\\
92.88	0\\
92.89	0\\
92.9	0\\
92.91	0\\
92.92	0\\
92.93	0\\
92.94	0\\
92.95	0\\
92.96	0\\
92.97	0\\
92.98	0\\
92.99	0\\
93	0\\
93.01	0\\
93.02	0\\
93.03	0\\
93.04	0\\
93.05	0\\
93.06	0\\
93.07	0\\
93.08	0\\
93.09	0\\
93.1	0\\
93.11	0\\
93.12	0\\
93.13	0\\
93.14	0\\
93.15	0\\
93.16	0\\
93.17	0\\
93.18	0\\
93.19	0\\
93.2	0\\
93.21	0\\
93.22	0\\
93.23	0\\
93.24	0\\
93.25	0\\
93.26	0\\
93.27	0\\
93.28	0\\
93.29	0\\
93.3	0\\
93.31	0\\
93.32	0\\
93.33	0\\
93.34	0\\
93.35	0\\
93.36	0\\
93.37	0\\
93.38	0\\
93.39	0\\
93.4	0\\
93.41	0\\
93.42	0\\
93.43	0\\
93.44	0\\
93.45	0\\
93.46	0\\
93.47	0\\
93.48	0\\
93.49	0\\
93.5	0\\
93.51	0\\
93.52	0\\
93.53	0\\
93.54	0\\
93.55	0\\
93.56	0\\
93.57	0\\
93.58	0\\
93.59	0\\
93.6	0\\
93.61	0\\
93.62	0\\
93.63	0\\
93.64	0\\
93.65	0\\
93.66	0\\
93.67	0\\
93.68	0\\
93.69	0\\
93.7	0\\
93.71	0\\
93.72	0\\
93.73	0\\
93.74	0\\
93.75	0\\
93.76	0\\
93.77	0\\
93.78	0\\
93.79	0\\
93.8	0\\
93.81	0\\
93.82	0\\
93.83	0\\
93.84	0\\
93.85	0\\
93.86	0\\
93.87	0\\
93.88	0\\
93.89	0\\
93.9	0\\
93.91	0\\
93.92	0\\
93.93	0\\
93.94	0\\
93.95	0\\
93.96	0\\
93.97	0\\
93.98	0\\
93.99	0\\
94	0\\
94.01	0\\
94.02	0\\
94.03	0\\
94.04	0\\
94.05	0\\
94.06	0\\
94.07	0\\
94.08	0\\
94.09	0\\
94.1	0\\
94.11	0\\
94.12	0\\
94.13	0\\
94.14	0\\
94.15	0\\
94.16	0\\
94.17	0\\
94.18	0\\
94.19	0\\
94.2	0\\
94.21	0\\
94.22	0\\
94.23	0\\
94.24	0\\
94.25	0\\
94.26	0\\
94.27	0\\
94.28	0\\
94.29	0\\
94.3	0\\
94.31	0\\
94.32	0\\
94.33	0\\
94.34	0\\
94.35	0\\
94.36	0\\
94.37	0\\
94.38	0\\
94.39	0\\
94.4	0\\
94.41	0\\
94.42	0\\
94.43	0\\
94.44	0\\
94.45	0\\
94.46	0\\
94.47	0\\
94.48	0\\
94.49	0\\
94.5	0\\
94.51	0\\
94.52	0\\
94.53	0\\
94.54	0\\
94.55	0\\
94.56	0\\
94.57	0\\
94.58	0\\
94.59	0\\
94.6	0\\
94.61	0\\
94.62	0\\
94.63	0\\
94.64	0\\
94.65	0\\
94.66	0\\
94.67	0\\
94.68	0\\
94.69	0\\
94.7	0\\
94.71	0\\
94.72	0\\
94.73	0\\
94.74	0\\
94.75	0\\
94.76	0\\
94.77	0\\
94.78	0\\
94.79	0\\
94.8	0\\
94.81	0\\
94.82	0\\
94.83	0\\
94.84	0\\
94.85	0\\
94.86	0\\
94.87	0\\
94.88	0\\
94.89	0\\
94.9	0\\
94.91	0\\
94.92	0\\
94.93	0\\
94.94	0\\
94.95	0\\
94.96	0\\
94.97	0\\
94.98	0\\
94.99	0\\
95	0\\
95.01	0\\
95.02	0\\
95.03	0\\
95.04	0\\
95.05	0\\
95.06	0\\
95.07	0\\
95.08	0\\
95.09	0\\
95.1	0\\
95.11	0\\
95.12	0\\
95.13	0\\
95.14	0\\
95.15	0\\
95.16	0\\
95.17	0\\
95.18	0\\
95.19	0\\
95.2	0\\
95.21	0\\
95.22	0\\
95.23	0\\
95.24	0\\
95.25	0\\
95.26	0\\
95.27	0\\
95.28	0\\
95.29	0\\
95.3	0\\
95.31	0\\
95.32	0\\
95.33	0\\
95.34	0\\
95.35	0\\
95.36	0\\
95.37	0\\
95.38	0\\
95.39	0\\
95.4	0\\
95.41	0\\
95.42	0\\
95.43	0\\
95.44	0\\
95.45	0\\
95.46	0\\
95.47	0\\
95.48	0\\
95.49	0\\
95.5	0\\
95.51	0\\
95.52	0\\
95.53	0\\
95.54	0\\
95.55	0\\
95.56	0\\
95.57	0\\
95.58	0\\
95.59	0\\
95.6	0\\
95.61	0\\
95.62	0\\
95.63	0\\
95.64	0\\
95.65	0\\
95.66	0\\
95.67	0\\
95.68	0\\
95.69	0\\
95.7	0\\
95.71	0\\
95.72	0\\
95.73	0\\
95.74	0\\
95.75	0\\
95.76	0\\
95.77	0\\
95.78	0\\
95.79	0\\
95.8	0\\
95.81	0\\
95.82	0\\
95.83	0\\
95.84	0\\
95.85	0\\
95.86	0\\
95.87	0\\
95.88	0\\
95.89	0\\
95.9	0\\
95.91	0\\
95.92	0\\
95.93	0\\
95.94	0\\
95.95	0\\
95.96	0\\
95.97	0\\
95.98	0\\
95.99	0\\
96	0\\
96.01	0\\
96.02	0\\
96.03	0\\
96.04	0\\
96.05	0\\
96.06	0\\
96.07	0\\
96.08	0\\
96.09	0\\
96.1	0\\
96.11	0\\
96.12	0\\
96.13	0\\
96.14	0\\
96.15	0\\
96.16	0\\
96.17	0\\
96.18	0\\
96.19	0\\
96.2	0\\
96.21	0\\
96.22	0\\
96.23	0\\
96.24	0\\
96.25	0\\
96.26	0\\
96.27	0\\
96.28	0\\
96.29	0\\
96.3	0\\
96.31	0\\
96.32	0\\
96.33	0\\
96.34	0\\
96.35	0\\
96.36	0\\
96.37	0\\
96.38	0\\
96.39	0\\
96.4	0\\
96.41	0\\
96.42	0\\
96.43	0\\
96.44	0\\
96.45	0\\
96.46	0\\
96.47	0\\
96.48	0\\
96.49	0\\
96.5	0\\
96.51	0\\
96.52	0\\
96.53	0\\
96.54	0\\
96.55	0\\
96.56	0\\
96.57	0\\
96.58	0\\
96.59	0\\
96.6	0\\
96.61	0\\
96.62	0\\
96.63	0\\
96.64	0\\
96.65	0\\
96.66	0\\
96.67	0\\
96.68	0\\
96.69	0\\
96.7	0\\
96.71	0\\
96.72	0\\
96.73	0\\
96.74	0\\
96.75	0\\
96.76	0\\
96.77	0\\
96.78	0\\
96.79	0\\
96.8	0\\
96.81	0\\
96.82	0\\
96.83	0\\
96.84	0\\
96.85	0\\
96.86	0\\
96.87	0\\
96.88	0\\
96.89	0\\
96.9	0\\
96.91	0\\
96.92	0\\
96.93	0\\
96.94	0\\
96.95	0\\
96.96	0\\
96.97	0\\
96.98	0\\
96.99	0\\
97	0\\
97.01	0\\
97.02	0\\
97.03	0\\
97.04	0\\
97.05	0\\
97.06	0\\
97.07	0\\
97.08	0\\
97.09	0\\
97.1	0\\
97.11	0\\
97.12	0\\
97.13	0\\
97.14	0\\
97.15	0\\
97.16	0\\
97.17	0\\
97.18	0\\
97.19	0\\
97.2	0\\
97.21	0\\
97.22	0\\
97.23	0\\
97.24	0\\
97.25	0\\
97.26	0\\
97.27	0\\
97.28	0\\
97.29	0\\
97.3	0\\
97.31	0\\
97.32	0\\
97.33	0\\
97.34	0\\
97.35	0\\
97.36	0\\
97.37	0\\
97.38	0\\
97.39	0\\
97.4	0\\
97.41	0\\
97.42	0\\
97.43	0\\
97.44	0\\
97.45	0\\
97.46	0\\
97.47	0\\
97.48	0\\
97.49	0\\
97.5	0\\
97.51	0\\
97.52	0\\
97.53	0\\
97.54	0\\
97.55	0\\
97.56	0\\
97.57	0\\
97.58	0\\
97.59	0\\
97.6	0\\
97.61	0\\
97.62	0\\
97.63	0\\
97.64	0\\
97.65	0\\
97.66	0\\
97.67	0\\
97.68	0\\
97.69	0\\
97.7	0\\
97.71	0\\
97.72	0\\
97.73	0\\
97.74	0\\
97.75	0\\
97.76	0\\
97.77	0\\
97.78	0\\
97.79	0\\
97.8	0\\
97.81	0\\
97.82	0\\
97.83	0\\
97.84	0\\
97.85	0\\
97.86	0\\
97.87	0\\
97.88	0\\
97.89	0\\
97.9	0\\
97.91	0\\
97.92	0\\
97.93	0\\
97.94	0\\
97.95	0\\
97.96	0\\
97.97	0\\
97.98	0\\
97.99	0\\
98	0\\
98.01	0\\
98.02	0\\
98.03	0\\
98.04	0\\
98.05	0\\
98.06	0\\
98.07	0\\
98.08	0\\
98.09	0\\
98.1	0\\
98.11	0\\
98.12	0\\
98.13	0\\
98.14	0\\
98.15	0\\
98.16	0\\
98.17	0\\
98.18	0\\
98.19	0\\
98.2	0\\
98.21	0\\
98.22	0\\
98.23	0\\
98.24	0\\
98.25	0\\
98.26	0\\
98.27	0\\
98.28	0\\
98.29	0\\
98.3	0\\
98.31	0\\
98.32	0\\
98.33	0\\
98.34	0\\
98.35	0\\
98.36	0\\
98.37	0\\
98.38	0\\
98.39	0\\
98.4	0\\
98.41	0\\
98.42	0\\
98.43	0\\
98.44	0\\
98.45	0\\
98.46	0\\
98.47	0\\
98.48	0\\
98.49	0\\
98.5	0\\
98.51	0\\
98.52	0\\
98.53	0\\
98.54	0\\
98.55	0\\
98.56	0\\
98.57	0\\
98.58	0\\
98.59	0\\
98.6	0\\
98.61	0\\
98.62	0\\
98.63	0\\
98.64	0\\
98.65	0\\
98.66	0\\
98.67	0\\
98.68	0\\
98.69	0\\
98.7	0\\
98.71	0\\
98.72	0\\
98.73	0\\
98.74	0\\
98.75	0\\
98.76	0\\
98.77	0\\
98.78	0\\
98.79	0\\
98.8	0\\
98.81	0\\
98.82	0\\
98.83	0.000157347822180701\\
98.84	0.000374631038096693\\
98.85	0.000593590675311409\\
98.86	0.000814252312230923\\
98.87	0.00103664230225356\\
98.88	0.00126078779650802\\
98.89	0.0014867167826122\\
98.9	0.00171445811745203\\
98.91	0.00194402004433675\\
98.92	0.0021753904182945\\
98.93	0.00240859886120944\\
98.94	0.00264367595120245\\
98.95	0.00288065326346108\\
98.96	0.00306604494409387\\
98.97	0.00311012117618544\\
98.98	0.00315440844710029\\
98.99	0.00319889925383041\\
99	0.0032435855790672\\
99.01	0.00328845890704418\\
99.02	0.00333351020433462\\
99.03	0.00337872989622107\\
99.04	0.00342410781940597\\
99.05	0.00346969067490262\\
99.06	0.00351546976428445\\
99.07	0.00356143509901703\\
99.08	0.00360757607931048\\
99.09	0.00365388146525549\\
99.1	0.00370033934650487\\
99.11	0.00374693711037871\\
99.12	0.00379366140830112\\
99.13	0.00384049812047099\\
99.14	0.0038874580034141\\
99.15	0.00393483565423353\\
99.16	0.00398263494425564\\
99.17	0.00403085980130233\\
99.18	0.0040795142118173\\
99.19	0.00412860222311627\\
99.2	0.00417812794569628\\
99.21	0.00422809555566403\\
99.22	0.00427850924718199\\
99.23	0.00432937310681449\\
99.24	0.0043806912758079\\
99.25	0.00443246795197619\\
99.26	0.00448470739168515\\
99.27	0.00453741391194087\\
99.28	0.00459059189258777\\
99.29	0.0046442457786222\\
99.3	0.00469838008262783\\
99.31	0.00475299938733924\\
99.32	0.00480810834834083\\
99.33	0.00486371169690803\\
99.34	0.00491981424299882\\
99.35	0.0049764208784027\\
99.36	0.00503353658005642\\
99.37	0.00509116641355064\\
99.38	0.00514931553692431\\
99.39	0.00520798920455344\\
99.4	0.00526719277124689\\
99.41	0.00532693169656051\\
99.42	0.00538721084840616\\
99.43	0.00544803513735539\\
99.44	0.00550940951836646\\
99.45	0.00557133899114194\\
99.46	0.0056338286004846\\
99.47	0.00569688343665094\\
99.48	0.0057605086360643\\
99.49	0.00582470938215867\\
99.5	0.00588949090583306\\
99.51	0.00595485848591075\\
99.52	0.00602081744960348\\
99.53	0.00608737317298039\\
99.54	0.00615453108144196\\
99.55	0.00622229665019873\\
99.56	0.00629067540475496\\
99.57	0.00635967292139696\\
99.58	0.00642929482768636\\
99.59	0.0064995468029579\\
99.6	0.00657043457882205\\
99.61	0.00664196393967204\\
99.62	0.0067141407231954\\
99.63	0.00678697082088982\\
99.64	0.00686046017858321\\
99.65	0.00693461479695774\\
99.66	0.0070094407320778\\
99.67	0.00708494409592144\\
99.68	0.00716113105691529\\
99.69	0.00723800782441484\\
99.7	0.00731558066288144\\
99.71	0.0073938558945165\\
99.72	0.0074728398998023\\
99.73	0.0075525391180442\\
99.74	0.00763296004791393\\
99.75	0.0077141092479932\\
99.76	0.00779599333731718\\
99.77	0.00787861899591701\\
99.78	0.00796199296536071\\
99.79	0.00804612204929144\\
99.8	0.00813101311396239\\
99.81	0.00821667308876694\\
99.82	0.00830310896676325\\
99.83	0.00839032780519167\\
99.84	0.00847833672598379\\
99.85	0.00856714291626138\\
99.86	0.00865675362882351\\
99.87	0.00874717618262004\\
99.88	0.00883841796320915\\
99.89	0.00893048642319673\\
99.9	0.00902338908265498\\
99.91	0.00911713352951744\\
99.92	0.00921172741994723\\
99.93	0.00930717847867511\\
99.94	0.00940349449930366\\
99.95	0.00950068334457319\\
99.96	0.00959875294658504\\
99.97	0.009697711306977\\
99.98	0.00979756649704546\\
99.99	0.00989832665780802\\
100	0.01\\
};
\addlegendentry{$q=-3$};

\addplot [color=red,dashed,forget plot]
  table[row sep=crcr]{%
0.01	0\\
0.02	0\\
0.03	0\\
0.04	0\\
0.05	0\\
0.06	0\\
0.07	0\\
0.08	0\\
0.09	0\\
0.1	0\\
0.11	0\\
0.12	0\\
0.13	0\\
0.14	0\\
0.15	0\\
0.16	0\\
0.17	0\\
0.18	0\\
0.19	0\\
0.2	0\\
0.21	0\\
0.22	0\\
0.23	0\\
0.24	0\\
0.25	0\\
0.26	0\\
0.27	0\\
0.28	0\\
0.29	0\\
0.3	0\\
0.31	0\\
0.32	0\\
0.33	0\\
0.34	0\\
0.35	0\\
0.36	0\\
0.37	0\\
0.38	0\\
0.39	0\\
0.4	0\\
0.41	0\\
0.42	0\\
0.43	0\\
0.44	0\\
0.45	0\\
0.46	0\\
0.47	0\\
0.48	0\\
0.49	0\\
0.5	0\\
0.51	0\\
0.52	0\\
0.53	0\\
0.54	0\\
0.55	0\\
0.56	0\\
0.57	0\\
0.58	0\\
0.59	0\\
0.6	0\\
0.61	0\\
0.62	0\\
0.63	0\\
0.64	0\\
0.65	0\\
0.66	0\\
0.67	0\\
0.68	0\\
0.69	0\\
0.7	0\\
0.71	0\\
0.72	0\\
0.73	0\\
0.74	0\\
0.75	0\\
0.76	0\\
0.77	0\\
0.78	0\\
0.79	0\\
0.8	0\\
0.81	0\\
0.82	0\\
0.83	0\\
0.84	0\\
0.85	0\\
0.86	0\\
0.87	0\\
0.88	0\\
0.89	0\\
0.9	0\\
0.91	0\\
0.92	0\\
0.93	0\\
0.94	0\\
0.95	0\\
0.96	0\\
0.97	0\\
0.98	0\\
0.99	0\\
1	0\\
1.01	0\\
1.02	0\\
1.03	0\\
1.04	0\\
1.05	0\\
1.06	0\\
1.07	0\\
1.08	0\\
1.09	0\\
1.1	0\\
1.11	0\\
1.12	0\\
1.13	0\\
1.14	0\\
1.15	0\\
1.16	0\\
1.17	0\\
1.18	0\\
1.19	0\\
1.2	0\\
1.21	0\\
1.22	0\\
1.23	0\\
1.24	0\\
1.25	0\\
1.26	0\\
1.27	0\\
1.28	0\\
1.29	0\\
1.3	0\\
1.31	0\\
1.32	0\\
1.33	0\\
1.34	0\\
1.35	0\\
1.36	0\\
1.37	0\\
1.38	0\\
1.39	0\\
1.4	0\\
1.41	0\\
1.42	0\\
1.43	0\\
1.44	0\\
1.45	0\\
1.46	0\\
1.47	0\\
1.48	0\\
1.49	0\\
1.5	0\\
1.51	0\\
1.52	0\\
1.53	0\\
1.54	0\\
1.55	0\\
1.56	0\\
1.57	0\\
1.58	0\\
1.59	0\\
1.6	0\\
1.61	0\\
1.62	0\\
1.63	0\\
1.64	0\\
1.65	0\\
1.66	0\\
1.67	0\\
1.68	0\\
1.69	0\\
1.7	0\\
1.71	0\\
1.72	0\\
1.73	0\\
1.74	0\\
1.75	0\\
1.76	0\\
1.77	0\\
1.78	0\\
1.79	0\\
1.8	0\\
1.81	0\\
1.82	0\\
1.83	0\\
1.84	0\\
1.85	0\\
1.86	0\\
1.87	0\\
1.88	0\\
1.89	0\\
1.9	0\\
1.91	0\\
1.92	0\\
1.93	0\\
1.94	0\\
1.95	0\\
1.96	0\\
1.97	0\\
1.98	0\\
1.99	0\\
2	0\\
2.01	0\\
2.02	0\\
2.03	0\\
2.04	0\\
2.05	0\\
2.06	0\\
2.07	0\\
2.08	0\\
2.09	0\\
2.1	0\\
2.11	0\\
2.12	0\\
2.13	0\\
2.14	0\\
2.15	0\\
2.16	0\\
2.17	0\\
2.18	0\\
2.19	0\\
2.2	0\\
2.21	0\\
2.22	0\\
2.23	0\\
2.24	0\\
2.25	0\\
2.26	0\\
2.27	0\\
2.28	0\\
2.29	0\\
2.3	0\\
2.31	0\\
2.32	0\\
2.33	0\\
2.34	0\\
2.35	0\\
2.36	0\\
2.37	0\\
2.38	0\\
2.39	0\\
2.4	0\\
2.41	0\\
2.42	0\\
2.43	0\\
2.44	0\\
2.45	0\\
2.46	0\\
2.47	0\\
2.48	0\\
2.49	0\\
2.5	0\\
2.51	0\\
2.52	0\\
2.53	0\\
2.54	0\\
2.55	0\\
2.56	0\\
2.57	0\\
2.58	0\\
2.59	0\\
2.6	0\\
2.61	0\\
2.62	0\\
2.63	0\\
2.64	0\\
2.65	0\\
2.66	0\\
2.67	0\\
2.68	0\\
2.69	0\\
2.7	0\\
2.71	0\\
2.72	0\\
2.73	0\\
2.74	0\\
2.75	0\\
2.76	0\\
2.77	0\\
2.78	0\\
2.79	0\\
2.8	0\\
2.81	0\\
2.82	0\\
2.83	0\\
2.84	0\\
2.85	0\\
2.86	0\\
2.87	0\\
2.88	0\\
2.89	0\\
2.9	0\\
2.91	0\\
2.92	0\\
2.93	0\\
2.94	0\\
2.95	0\\
2.96	0\\
2.97	0\\
2.98	0\\
2.99	0\\
3	0\\
3.01	0\\
3.02	0\\
3.03	0\\
3.04	0\\
3.05	0\\
3.06	0\\
3.07	0\\
3.08	0\\
3.09	0\\
3.1	0\\
3.11	0\\
3.12	0\\
3.13	0\\
3.14	0\\
3.15	0\\
3.16	0\\
3.17	0\\
3.18	0\\
3.19	0\\
3.2	0\\
3.21	0\\
3.22	0\\
3.23	0\\
3.24	0\\
3.25	0\\
3.26	0\\
3.27	0\\
3.28	0\\
3.29	0\\
3.3	0\\
3.31	0\\
3.32	0\\
3.33	0\\
3.34	0\\
3.35	0\\
3.36	0\\
3.37	0\\
3.38	0\\
3.39	0\\
3.4	0\\
3.41	0\\
3.42	0\\
3.43	0\\
3.44	0\\
3.45	0\\
3.46	0\\
3.47	0\\
3.48	0\\
3.49	0\\
3.5	0\\
3.51	0\\
3.52	0\\
3.53	0\\
3.54	0\\
3.55	0\\
3.56	0\\
3.57	0\\
3.58	0\\
3.59	0\\
3.6	0\\
3.61	0\\
3.62	0\\
3.63	0\\
3.64	0\\
3.65	0\\
3.66	0\\
3.67	0\\
3.68	0\\
3.69	0\\
3.7	0\\
3.71	0\\
3.72	0\\
3.73	0\\
3.74	0\\
3.75	0\\
3.76	0\\
3.77	0\\
3.78	0\\
3.79	0\\
3.8	0\\
3.81	0\\
3.82	0\\
3.83	0\\
3.84	0\\
3.85	0\\
3.86	0\\
3.87	0\\
3.88	0\\
3.89	0\\
3.9	0\\
3.91	0\\
3.92	0\\
3.93	0\\
3.94	0\\
3.95	0\\
3.96	0\\
3.97	0\\
3.98	0\\
3.99	0\\
4	0\\
4.01	0\\
4.02	0\\
4.03	0\\
4.04	0\\
4.05	0\\
4.06	0\\
4.07	0\\
4.08	0\\
4.09	0\\
4.1	0\\
4.11	0\\
4.12	0\\
4.13	0\\
4.14	0\\
4.15	0\\
4.16	0\\
4.17	0\\
4.18	0\\
4.19	0\\
4.2	0\\
4.21	0\\
4.22	0\\
4.23	0\\
4.24	0\\
4.25	0\\
4.26	0\\
4.27	0\\
4.28	0\\
4.29	0\\
4.3	0\\
4.31	0\\
4.32	0\\
4.33	0\\
4.34	0\\
4.35	0\\
4.36	0\\
4.37	0\\
4.38	0\\
4.39	0\\
4.4	0\\
4.41	0\\
4.42	0\\
4.43	0\\
4.44	0\\
4.45	0\\
4.46	0\\
4.47	0\\
4.48	0\\
4.49	0\\
4.5	0\\
4.51	0\\
4.52	0\\
4.53	0\\
4.54	0\\
4.55	0\\
4.56	0\\
4.57	0\\
4.58	0\\
4.59	0\\
4.6	0\\
4.61	0\\
4.62	0\\
4.63	0\\
4.64	0\\
4.65	0\\
4.66	0\\
4.67	0\\
4.68	0\\
4.69	0\\
4.7	0\\
4.71	0\\
4.72	0\\
4.73	0\\
4.74	0\\
4.75	0\\
4.76	0\\
4.77	0\\
4.78	0\\
4.79	0\\
4.8	0\\
4.81	0\\
4.82	0\\
4.83	0\\
4.84	0\\
4.85	0\\
4.86	0\\
4.87	0\\
4.88	0\\
4.89	0\\
4.9	0\\
4.91	0\\
4.92	0\\
4.93	0\\
4.94	0\\
4.95	0\\
4.96	0\\
4.97	0\\
4.98	0\\
4.99	0\\
5	0\\
5.01	0\\
5.02	0\\
5.03	0\\
5.04	0\\
5.05	0\\
5.06	0\\
5.07	0\\
5.08	0\\
5.09	0\\
5.1	0\\
5.11	0\\
5.12	0\\
5.13	0\\
5.14	0\\
5.15	0\\
5.16	0\\
5.17	0\\
5.18	0\\
5.19	0\\
5.2	0\\
5.21	0\\
5.22	0\\
5.23	0\\
5.24	0\\
5.25	0\\
5.26	0\\
5.27	0\\
5.28	0\\
5.29	0\\
5.3	0\\
5.31	0\\
5.32	0\\
5.33	0\\
5.34	0\\
5.35	0\\
5.36	0\\
5.37	0\\
5.38	0\\
5.39	0\\
5.4	0\\
5.41	0\\
5.42	0\\
5.43	0\\
5.44	0\\
5.45	0\\
5.46	0\\
5.47	0\\
5.48	0\\
5.49	0\\
5.5	0\\
5.51	0\\
5.52	0\\
5.53	0\\
5.54	0\\
5.55	0\\
5.56	0\\
5.57	0\\
5.58	0\\
5.59	0\\
5.6	0\\
5.61	0\\
5.62	0\\
5.63	0\\
5.64	0\\
5.65	0\\
5.66	0\\
5.67	0\\
5.68	0\\
5.69	0\\
5.7	0\\
5.71	0\\
5.72	0\\
5.73	0\\
5.74	0\\
5.75	0\\
5.76	0\\
5.77	0\\
5.78	0\\
5.79	0\\
5.8	0\\
5.81	0\\
5.82	0\\
5.83	0\\
5.84	0\\
5.85	0\\
5.86	0\\
5.87	0\\
5.88	0\\
5.89	0\\
5.9	0\\
5.91	0\\
5.92	0\\
5.93	0\\
5.94	0\\
5.95	0\\
5.96	0\\
5.97	0\\
5.98	0\\
5.99	0\\
6	0\\
6.01	0\\
6.02	0\\
6.03	0\\
6.04	0\\
6.05	0\\
6.06	0\\
6.07	0\\
6.08	0\\
6.09	0\\
6.1	0\\
6.11	0\\
6.12	0\\
6.13	0\\
6.14	0\\
6.15	0\\
6.16	0\\
6.17	0\\
6.18	0\\
6.19	0\\
6.2	0\\
6.21	0\\
6.22	0\\
6.23	0\\
6.24	0\\
6.25	0\\
6.26	0\\
6.27	0\\
6.28	0\\
6.29	0\\
6.3	0\\
6.31	0\\
6.32	0\\
6.33	0\\
6.34	0\\
6.35	0\\
6.36	0\\
6.37	0\\
6.38	0\\
6.39	0\\
6.4	0\\
6.41	0\\
6.42	0\\
6.43	0\\
6.44	0\\
6.45	0\\
6.46	0\\
6.47	0\\
6.48	0\\
6.49	0\\
6.5	0\\
6.51	0\\
6.52	0\\
6.53	0\\
6.54	0\\
6.55	0\\
6.56	0\\
6.57	0\\
6.58	0\\
6.59	0\\
6.6	0\\
6.61	0\\
6.62	0\\
6.63	0\\
6.64	0\\
6.65	0\\
6.66	0\\
6.67	0\\
6.68	0\\
6.69	0\\
6.7	0\\
6.71	0\\
6.72	0\\
6.73	0\\
6.74	0\\
6.75	0\\
6.76	0\\
6.77	0\\
6.78	0\\
6.79	0\\
6.8	0\\
6.81	0\\
6.82	0\\
6.83	0\\
6.84	0\\
6.85	0\\
6.86	0\\
6.87	0\\
6.88	0\\
6.89	0\\
6.9	0\\
6.91	0\\
6.92	0\\
6.93	0\\
6.94	0\\
6.95	0\\
6.96	0\\
6.97	0\\
6.98	0\\
6.99	0\\
7	0\\
7.01	0\\
7.02	0\\
7.03	0\\
7.04	0\\
7.05	0\\
7.06	0\\
7.07	0\\
7.08	0\\
7.09	0\\
7.1	0\\
7.11	0\\
7.12	0\\
7.13	0\\
7.14	0\\
7.15	0\\
7.16	0\\
7.17	0\\
7.18	0\\
7.19	0\\
7.2	0\\
7.21	0\\
7.22	0\\
7.23	0\\
7.24	0\\
7.25	0\\
7.26	0\\
7.27	0\\
7.28	0\\
7.29	0\\
7.3	0\\
7.31	0\\
7.32	0\\
7.33	0\\
7.34	0\\
7.35	0\\
7.36	0\\
7.37	0\\
7.38	0\\
7.39	0\\
7.4	0\\
7.41	0\\
7.42	0\\
7.43	0\\
7.44	0\\
7.45	0\\
7.46	0\\
7.47	0\\
7.48	0\\
7.49	0\\
7.5	0\\
7.51	0\\
7.52	0\\
7.53	0\\
7.54	0\\
7.55	0\\
7.56	0\\
7.57	0\\
7.58	0\\
7.59	0\\
7.6	0\\
7.61	0\\
7.62	0\\
7.63	0\\
7.64	0\\
7.65	0\\
7.66	0\\
7.67	0\\
7.68	0\\
7.69	0\\
7.7	0\\
7.71	0\\
7.72	0\\
7.73	0\\
7.74	0\\
7.75	0\\
7.76	0\\
7.77	0\\
7.78	0\\
7.79	0\\
7.8	0\\
7.81	0\\
7.82	0\\
7.83	0\\
7.84	0\\
7.85	0\\
7.86	0\\
7.87	0\\
7.88	0\\
7.89	0\\
7.9	0\\
7.91	0\\
7.92	0\\
7.93	0\\
7.94	0\\
7.95	0\\
7.96	0\\
7.97	0\\
7.98	0\\
7.99	0\\
8	0\\
8.01	0\\
8.02	0\\
8.03	0\\
8.04	0\\
8.05	0\\
8.06	0\\
8.07	0\\
8.08	0\\
8.09	0\\
8.1	0\\
8.11	0\\
8.12	0\\
8.13	0\\
8.14	0\\
8.15	0\\
8.16	0\\
8.17	0\\
8.18	0\\
8.19	0\\
8.2	0\\
8.21	0\\
8.22	0\\
8.23	0\\
8.24	0\\
8.25	0\\
8.26	0\\
8.27	0\\
8.28	0\\
8.29	0\\
8.3	0\\
8.31	0\\
8.32	0\\
8.33	0\\
8.34	0\\
8.35	0\\
8.36	0\\
8.37	0\\
8.38	0\\
8.39	0\\
8.4	0\\
8.41	0\\
8.42	0\\
8.43	0\\
8.44	0\\
8.45	0\\
8.46	0\\
8.47	0\\
8.48	0\\
8.49	0\\
8.5	0\\
8.51	0\\
8.52	0\\
8.53	0\\
8.54	0\\
8.55	0\\
8.56	0\\
8.57	0\\
8.58	0\\
8.59	0\\
8.6	0\\
8.61	0\\
8.62	0\\
8.63	0\\
8.64	0\\
8.65	0\\
8.66	0\\
8.67	0\\
8.68	0\\
8.69	0\\
8.7	0\\
8.71	0\\
8.72	0\\
8.73	0\\
8.74	0\\
8.75	0\\
8.76	0\\
8.77	0\\
8.78	0\\
8.79	0\\
8.8	0\\
8.81	0\\
8.82	0\\
8.83	0\\
8.84	0\\
8.85	0\\
8.86	0\\
8.87	0\\
8.88	0\\
8.89	0\\
8.9	0\\
8.91	0\\
8.92	0\\
8.93	0\\
8.94	0\\
8.95	0\\
8.96	0\\
8.97	0\\
8.98	0\\
8.99	0\\
9	0\\
9.01	0\\
9.02	0\\
9.03	0\\
9.04	0\\
9.05	0\\
9.06	0\\
9.07	0\\
9.08	0\\
9.09	0\\
9.1	0\\
9.11	0\\
9.12	0\\
9.13	0\\
9.14	0\\
9.15	0\\
9.16	0\\
9.17	0\\
9.18	0\\
9.19	0\\
9.2	0\\
9.21	0\\
9.22	0\\
9.23	0\\
9.24	0\\
9.25	0\\
9.26	0\\
9.27	0\\
9.28	0\\
9.29	0\\
9.3	0\\
9.31	0\\
9.32	0\\
9.33	0\\
9.34	0\\
9.35	0\\
9.36	0\\
9.37	0\\
9.38	0\\
9.39	0\\
9.4	0\\
9.41	0\\
9.42	0\\
9.43	0\\
9.44	0\\
9.45	0\\
9.46	0\\
9.47	0\\
9.48	0\\
9.49	0\\
9.5	0\\
9.51	0\\
9.52	0\\
9.53	0\\
9.54	0\\
9.55	0\\
9.56	0\\
9.57	0\\
9.58	0\\
9.59	0\\
9.6	0\\
9.61	0\\
9.62	0\\
9.63	0\\
9.64	0\\
9.65	0\\
9.66	0\\
9.67	0\\
9.68	0\\
9.69	0\\
9.7	0\\
9.71	0\\
9.72	0\\
9.73	0\\
9.74	0\\
9.75	0\\
9.76	0\\
9.77	0\\
9.78	0\\
9.79	0\\
9.8	0\\
9.81	0\\
9.82	0\\
9.83	0\\
9.84	0\\
9.85	0\\
9.86	0\\
9.87	0\\
9.88	0\\
9.89	0\\
9.9	0\\
9.91	0\\
9.92	0\\
9.93	0\\
9.94	0\\
9.95	0\\
9.96	0\\
9.97	0\\
9.98	0\\
9.99	0\\
10	0\\
10.01	0\\
10.02	0\\
10.03	0\\
10.04	0\\
10.05	0\\
10.06	0\\
10.07	0\\
10.08	0\\
10.09	0\\
10.1	0\\
10.11	0\\
10.12	0\\
10.13	0\\
10.14	0\\
10.15	0\\
10.16	0\\
10.17	0\\
10.18	0\\
10.19	0\\
10.2	0\\
10.21	0\\
10.22	0\\
10.23	0\\
10.24	0\\
10.25	0\\
10.26	0\\
10.27	0\\
10.28	0\\
10.29	0\\
10.3	0\\
10.31	0\\
10.32	0\\
10.33	0\\
10.34	0\\
10.35	0\\
10.36	0\\
10.37	0\\
10.38	0\\
10.39	0\\
10.4	0\\
10.41	0\\
10.42	0\\
10.43	0\\
10.44	0\\
10.45	0\\
10.46	0\\
10.47	0\\
10.48	0\\
10.49	0\\
10.5	0\\
10.51	0\\
10.52	0\\
10.53	0\\
10.54	0\\
10.55	0\\
10.56	0\\
10.57	0\\
10.58	0\\
10.59	0\\
10.6	0\\
10.61	0\\
10.62	0\\
10.63	0\\
10.64	0\\
10.65	0\\
10.66	0\\
10.67	0\\
10.68	0\\
10.69	0\\
10.7	0\\
10.71	0\\
10.72	0\\
10.73	0\\
10.74	0\\
10.75	0\\
10.76	0\\
10.77	0\\
10.78	0\\
10.79	0\\
10.8	0\\
10.81	0\\
10.82	0\\
10.83	0\\
10.84	0\\
10.85	0\\
10.86	0\\
10.87	0\\
10.88	0\\
10.89	0\\
10.9	0\\
10.91	0\\
10.92	0\\
10.93	0\\
10.94	0\\
10.95	0\\
10.96	0\\
10.97	0\\
10.98	0\\
10.99	0\\
11	0\\
11.01	0\\
11.02	0\\
11.03	0\\
11.04	0\\
11.05	0\\
11.06	0\\
11.07	0\\
11.08	0\\
11.09	0\\
11.1	0\\
11.11	0\\
11.12	0\\
11.13	0\\
11.14	0\\
11.15	0\\
11.16	0\\
11.17	0\\
11.18	0\\
11.19	0\\
11.2	0\\
11.21	0\\
11.22	0\\
11.23	0\\
11.24	0\\
11.25	0\\
11.26	0\\
11.27	0\\
11.28	0\\
11.29	0\\
11.3	0\\
11.31	0\\
11.32	0\\
11.33	0\\
11.34	0\\
11.35	0\\
11.36	0\\
11.37	0\\
11.38	0\\
11.39	0\\
11.4	0\\
11.41	0\\
11.42	0\\
11.43	0\\
11.44	0\\
11.45	0\\
11.46	0\\
11.47	0\\
11.48	0\\
11.49	0\\
11.5	0\\
11.51	0\\
11.52	0\\
11.53	0\\
11.54	0\\
11.55	0\\
11.56	0\\
11.57	0\\
11.58	0\\
11.59	0\\
11.6	0\\
11.61	0\\
11.62	0\\
11.63	0\\
11.64	0\\
11.65	0\\
11.66	0\\
11.67	0\\
11.68	0\\
11.69	0\\
11.7	0\\
11.71	0\\
11.72	0\\
11.73	0\\
11.74	0\\
11.75	0\\
11.76	0\\
11.77	0\\
11.78	0\\
11.79	0\\
11.8	0\\
11.81	0\\
11.82	0\\
11.83	0\\
11.84	0\\
11.85	0\\
11.86	0\\
11.87	0\\
11.88	0\\
11.89	0\\
11.9	0\\
11.91	0\\
11.92	0\\
11.93	0\\
11.94	0\\
11.95	0\\
11.96	0\\
11.97	0\\
11.98	0\\
11.99	0\\
12	0\\
12.01	0\\
12.02	0\\
12.03	0\\
12.04	0\\
12.05	0\\
12.06	0\\
12.07	0\\
12.08	0\\
12.09	0\\
12.1	0\\
12.11	0\\
12.12	0\\
12.13	0\\
12.14	0\\
12.15	0\\
12.16	0\\
12.17	0\\
12.18	0\\
12.19	0\\
12.2	0\\
12.21	0\\
12.22	0\\
12.23	0\\
12.24	0\\
12.25	0\\
12.26	0\\
12.27	0\\
12.28	0\\
12.29	0\\
12.3	0\\
12.31	0\\
12.32	0\\
12.33	0\\
12.34	0\\
12.35	0\\
12.36	0\\
12.37	0\\
12.38	0\\
12.39	0\\
12.4	0\\
12.41	0\\
12.42	0\\
12.43	0\\
12.44	0\\
12.45	0\\
12.46	0\\
12.47	0\\
12.48	0\\
12.49	0\\
12.5	0\\
12.51	0\\
12.52	0\\
12.53	0\\
12.54	0\\
12.55	0\\
12.56	0\\
12.57	0\\
12.58	0\\
12.59	0\\
12.6	0\\
12.61	0\\
12.62	0\\
12.63	0\\
12.64	0\\
12.65	0\\
12.66	0\\
12.67	0\\
12.68	0\\
12.69	0\\
12.7	0\\
12.71	0\\
12.72	0\\
12.73	0\\
12.74	0\\
12.75	0\\
12.76	0\\
12.77	0\\
12.78	0\\
12.79	0\\
12.8	0\\
12.81	0\\
12.82	0\\
12.83	0\\
12.84	0\\
12.85	0\\
12.86	0\\
12.87	0\\
12.88	0\\
12.89	0\\
12.9	0\\
12.91	0\\
12.92	0\\
12.93	0\\
12.94	0\\
12.95	0\\
12.96	0\\
12.97	0\\
12.98	0\\
12.99	0\\
13	0\\
13.01	0\\
13.02	0\\
13.03	0\\
13.04	0\\
13.05	0\\
13.06	0\\
13.07	0\\
13.08	0\\
13.09	0\\
13.1	0\\
13.11	0\\
13.12	0\\
13.13	0\\
13.14	0\\
13.15	0\\
13.16	0\\
13.17	0\\
13.18	0\\
13.19	0\\
13.2	0\\
13.21	0\\
13.22	0\\
13.23	0\\
13.24	0\\
13.25	0\\
13.26	0\\
13.27	0\\
13.28	0\\
13.29	0\\
13.3	0\\
13.31	0\\
13.32	0\\
13.33	0\\
13.34	0\\
13.35	0\\
13.36	0\\
13.37	0\\
13.38	0\\
13.39	0\\
13.4	0\\
13.41	0\\
13.42	0\\
13.43	0\\
13.44	0\\
13.45	0\\
13.46	0\\
13.47	0\\
13.48	0\\
13.49	0\\
13.5	0\\
13.51	0\\
13.52	0\\
13.53	0\\
13.54	0\\
13.55	0\\
13.56	0\\
13.57	0\\
13.58	0\\
13.59	0\\
13.6	0\\
13.61	0\\
13.62	0\\
13.63	0\\
13.64	0\\
13.65	0\\
13.66	0\\
13.67	0\\
13.68	0\\
13.69	0\\
13.7	0\\
13.71	0\\
13.72	0\\
13.73	0\\
13.74	0\\
13.75	0\\
13.76	0\\
13.77	0\\
13.78	0\\
13.79	0\\
13.8	0\\
13.81	0\\
13.82	0\\
13.83	0\\
13.84	0\\
13.85	0\\
13.86	0\\
13.87	0\\
13.88	0\\
13.89	0\\
13.9	0\\
13.91	0\\
13.92	0\\
13.93	0\\
13.94	0\\
13.95	0\\
13.96	0\\
13.97	0\\
13.98	0\\
13.99	0\\
14	0\\
14.01	0\\
14.02	0\\
14.03	0\\
14.04	0\\
14.05	0\\
14.06	0\\
14.07	0\\
14.08	0\\
14.09	0\\
14.1	0\\
14.11	0\\
14.12	0\\
14.13	0\\
14.14	0\\
14.15	0\\
14.16	0\\
14.17	0\\
14.18	0\\
14.19	0\\
14.2	0\\
14.21	0\\
14.22	0\\
14.23	0\\
14.24	0\\
14.25	0\\
14.26	0\\
14.27	0\\
14.28	0\\
14.29	0\\
14.3	0\\
14.31	0\\
14.32	0\\
14.33	0\\
14.34	0\\
14.35	0\\
14.36	0\\
14.37	0\\
14.38	0\\
14.39	0\\
14.4	0\\
14.41	0\\
14.42	0\\
14.43	0\\
14.44	0\\
14.45	0\\
14.46	0\\
14.47	0\\
14.48	0\\
14.49	0\\
14.5	0\\
14.51	0\\
14.52	0\\
14.53	0\\
14.54	0\\
14.55	0\\
14.56	0\\
14.57	0\\
14.58	0\\
14.59	0\\
14.6	0\\
14.61	0\\
14.62	0\\
14.63	0\\
14.64	0\\
14.65	0\\
14.66	0\\
14.67	0\\
14.68	0\\
14.69	0\\
14.7	0\\
14.71	0\\
14.72	0\\
14.73	0\\
14.74	0\\
14.75	0\\
14.76	0\\
14.77	0\\
14.78	0\\
14.79	0\\
14.8	0\\
14.81	0\\
14.82	0\\
14.83	0\\
14.84	0\\
14.85	0\\
14.86	0\\
14.87	0\\
14.88	0\\
14.89	0\\
14.9	0\\
14.91	0\\
14.92	0\\
14.93	0\\
14.94	0\\
14.95	0\\
14.96	0\\
14.97	0\\
14.98	0\\
14.99	0\\
15	0\\
15.01	0\\
15.02	0\\
15.03	0\\
15.04	0\\
15.05	0\\
15.06	0\\
15.07	0\\
15.08	0\\
15.09	0\\
15.1	0\\
15.11	0\\
15.12	0\\
15.13	0\\
15.14	0\\
15.15	0\\
15.16	0\\
15.17	0\\
15.18	0\\
15.19	0\\
15.2	0\\
15.21	0\\
15.22	0\\
15.23	0\\
15.24	0\\
15.25	0\\
15.26	0\\
15.27	0\\
15.28	0\\
15.29	0\\
15.3	0\\
15.31	0\\
15.32	0\\
15.33	0\\
15.34	0\\
15.35	0\\
15.36	0\\
15.37	0\\
15.38	0\\
15.39	0\\
15.4	0\\
15.41	0\\
15.42	0\\
15.43	0\\
15.44	0\\
15.45	0\\
15.46	0\\
15.47	0\\
15.48	0\\
15.49	0\\
15.5	0\\
15.51	0\\
15.52	0\\
15.53	0\\
15.54	0\\
15.55	0\\
15.56	0\\
15.57	0\\
15.58	0\\
15.59	0\\
15.6	0\\
15.61	0\\
15.62	0\\
15.63	0\\
15.64	0\\
15.65	0\\
15.66	0\\
15.67	0\\
15.68	0\\
15.69	0\\
15.7	0\\
15.71	0\\
15.72	0\\
15.73	0\\
15.74	0\\
15.75	0\\
15.76	0\\
15.77	0\\
15.78	0\\
15.79	0\\
15.8	0\\
15.81	0\\
15.82	0\\
15.83	0\\
15.84	0\\
15.85	0\\
15.86	0\\
15.87	0\\
15.88	0\\
15.89	0\\
15.9	0\\
15.91	0\\
15.92	0\\
15.93	0\\
15.94	0\\
15.95	0\\
15.96	0\\
15.97	0\\
15.98	0\\
15.99	0\\
16	0\\
16.01	0\\
16.02	0\\
16.03	0\\
16.04	0\\
16.05	0\\
16.06	0\\
16.07	0\\
16.08	0\\
16.09	0\\
16.1	0\\
16.11	0\\
16.12	0\\
16.13	0\\
16.14	0\\
16.15	0\\
16.16	0\\
16.17	0\\
16.18	0\\
16.19	0\\
16.2	0\\
16.21	0\\
16.22	0\\
16.23	0\\
16.24	0\\
16.25	0\\
16.26	0\\
16.27	0\\
16.28	0\\
16.29	0\\
16.3	0\\
16.31	0\\
16.32	0\\
16.33	0\\
16.34	0\\
16.35	0\\
16.36	0\\
16.37	0\\
16.38	0\\
16.39	0\\
16.4	0\\
16.41	0\\
16.42	0\\
16.43	0\\
16.44	0\\
16.45	0\\
16.46	0\\
16.47	0\\
16.48	0\\
16.49	0\\
16.5	0\\
16.51	0\\
16.52	0\\
16.53	0\\
16.54	0\\
16.55	0\\
16.56	0\\
16.57	0\\
16.58	0\\
16.59	0\\
16.6	0\\
16.61	0\\
16.62	0\\
16.63	0\\
16.64	0\\
16.65	0\\
16.66	0\\
16.67	0\\
16.68	0\\
16.69	0\\
16.7	0\\
16.71	0\\
16.72	0\\
16.73	0\\
16.74	0\\
16.75	0\\
16.76	0\\
16.77	0\\
16.78	0\\
16.79	0\\
16.8	0\\
16.81	0\\
16.82	0\\
16.83	0\\
16.84	0\\
16.85	0\\
16.86	0\\
16.87	0\\
16.88	0\\
16.89	0\\
16.9	0\\
16.91	0\\
16.92	0\\
16.93	0\\
16.94	0\\
16.95	0\\
16.96	0\\
16.97	0\\
16.98	0\\
16.99	0\\
17	0\\
17.01	0\\
17.02	0\\
17.03	0\\
17.04	0\\
17.05	0\\
17.06	0\\
17.07	0\\
17.08	0\\
17.09	0\\
17.1	0\\
17.11	0\\
17.12	0\\
17.13	0\\
17.14	0\\
17.15	0\\
17.16	0\\
17.17	0\\
17.18	0\\
17.19	0\\
17.2	0\\
17.21	0\\
17.22	0\\
17.23	0\\
17.24	0\\
17.25	0\\
17.26	0\\
17.27	0\\
17.28	0\\
17.29	0\\
17.3	0\\
17.31	0\\
17.32	0\\
17.33	0\\
17.34	0\\
17.35	0\\
17.36	0\\
17.37	0\\
17.38	0\\
17.39	0\\
17.4	0\\
17.41	0\\
17.42	0\\
17.43	0\\
17.44	0\\
17.45	0\\
17.46	0\\
17.47	0\\
17.48	0\\
17.49	0\\
17.5	0\\
17.51	0\\
17.52	0\\
17.53	0\\
17.54	0\\
17.55	0\\
17.56	0\\
17.57	0\\
17.58	0\\
17.59	0\\
17.6	0\\
17.61	0\\
17.62	0\\
17.63	0\\
17.64	0\\
17.65	0\\
17.66	0\\
17.67	0\\
17.68	0\\
17.69	0\\
17.7	0\\
17.71	0\\
17.72	0\\
17.73	0\\
17.74	0\\
17.75	0\\
17.76	0\\
17.77	0\\
17.78	0\\
17.79	0\\
17.8	0\\
17.81	0\\
17.82	0\\
17.83	0\\
17.84	0\\
17.85	0\\
17.86	0\\
17.87	0\\
17.88	0\\
17.89	0\\
17.9	0\\
17.91	0\\
17.92	0\\
17.93	0\\
17.94	0\\
17.95	0\\
17.96	0\\
17.97	0\\
17.98	0\\
17.99	0\\
18	0\\
18.01	0\\
18.02	0\\
18.03	0\\
18.04	0\\
18.05	0\\
18.06	0\\
18.07	0\\
18.08	0\\
18.09	0\\
18.1	0\\
18.11	0\\
18.12	0\\
18.13	0\\
18.14	0\\
18.15	0\\
18.16	0\\
18.17	0\\
18.18	0\\
18.19	0\\
18.2	0\\
18.21	0\\
18.22	0\\
18.23	0\\
18.24	0\\
18.25	0\\
18.26	0\\
18.27	0\\
18.28	0\\
18.29	0\\
18.3	0\\
18.31	0\\
18.32	0\\
18.33	0\\
18.34	0\\
18.35	0\\
18.36	0\\
18.37	0\\
18.38	0\\
18.39	0\\
18.4	0\\
18.41	0\\
18.42	0\\
18.43	0\\
18.44	0\\
18.45	0\\
18.46	0\\
18.47	0\\
18.48	0\\
18.49	0\\
18.5	0\\
18.51	0\\
18.52	0\\
18.53	0\\
18.54	0\\
18.55	0\\
18.56	0\\
18.57	0\\
18.58	0\\
18.59	0\\
18.6	0\\
18.61	0\\
18.62	0\\
18.63	0\\
18.64	0\\
18.65	0\\
18.66	0\\
18.67	0\\
18.68	0\\
18.69	0\\
18.7	0\\
18.71	0\\
18.72	0\\
18.73	0\\
18.74	0\\
18.75	0\\
18.76	0\\
18.77	0\\
18.78	0\\
18.79	0\\
18.8	0\\
18.81	0\\
18.82	0\\
18.83	0\\
18.84	0\\
18.85	0\\
18.86	0\\
18.87	0\\
18.88	0\\
18.89	0\\
18.9	0\\
18.91	0\\
18.92	0\\
18.93	0\\
18.94	0\\
18.95	0\\
18.96	0\\
18.97	0\\
18.98	0\\
18.99	0\\
19	0\\
19.01	0\\
19.02	0\\
19.03	0\\
19.04	0\\
19.05	0\\
19.06	0\\
19.07	0\\
19.08	0\\
19.09	0\\
19.1	0\\
19.11	0\\
19.12	0\\
19.13	0\\
19.14	0\\
19.15	0\\
19.16	0\\
19.17	0\\
19.18	0\\
19.19	0\\
19.2	0\\
19.21	0\\
19.22	0\\
19.23	0\\
19.24	0\\
19.25	0\\
19.26	0\\
19.27	0\\
19.28	0\\
19.29	0\\
19.3	0\\
19.31	0\\
19.32	0\\
19.33	0\\
19.34	0\\
19.35	0\\
19.36	0\\
19.37	0\\
19.38	0\\
19.39	0\\
19.4	0\\
19.41	0\\
19.42	0\\
19.43	0\\
19.44	0\\
19.45	0\\
19.46	0\\
19.47	0\\
19.48	0\\
19.49	0\\
19.5	0\\
19.51	0\\
19.52	0\\
19.53	0\\
19.54	0\\
19.55	0\\
19.56	0\\
19.57	0\\
19.58	0\\
19.59	0\\
19.6	0\\
19.61	0\\
19.62	0\\
19.63	0\\
19.64	0\\
19.65	0\\
19.66	0\\
19.67	0\\
19.68	0\\
19.69	0\\
19.7	0\\
19.71	0\\
19.72	0\\
19.73	0\\
19.74	0\\
19.75	0\\
19.76	0\\
19.77	0\\
19.78	0\\
19.79	0\\
19.8	0\\
19.81	0\\
19.82	0\\
19.83	0\\
19.84	0\\
19.85	0\\
19.86	0\\
19.87	0\\
19.88	0\\
19.89	0\\
19.9	0\\
19.91	0\\
19.92	0\\
19.93	0\\
19.94	0\\
19.95	0\\
19.96	0\\
19.97	0\\
19.98	0\\
19.99	0\\
20	0\\
20.01	0\\
20.02	0\\
20.03	0\\
20.04	0\\
20.05	0\\
20.06	0\\
20.07	0\\
20.08	0\\
20.09	0\\
20.1	0\\
20.11	0\\
20.12	0\\
20.13	0\\
20.14	0\\
20.15	0\\
20.16	0\\
20.17	0\\
20.18	0\\
20.19	0\\
20.2	0\\
20.21	0\\
20.22	0\\
20.23	0\\
20.24	0\\
20.25	0\\
20.26	0\\
20.27	0\\
20.28	0\\
20.29	0\\
20.3	0\\
20.31	0\\
20.32	0\\
20.33	0\\
20.34	0\\
20.35	0\\
20.36	0\\
20.37	0\\
20.38	0\\
20.39	0\\
20.4	0\\
20.41	0\\
20.42	0\\
20.43	0\\
20.44	0\\
20.45	0\\
20.46	0\\
20.47	0\\
20.48	0\\
20.49	0\\
20.5	0\\
20.51	0\\
20.52	0\\
20.53	0\\
20.54	0\\
20.55	0\\
20.56	0\\
20.57	0\\
20.58	0\\
20.59	0\\
20.6	0\\
20.61	0\\
20.62	0\\
20.63	0\\
20.64	0\\
20.65	0\\
20.66	0\\
20.67	0\\
20.68	0\\
20.69	0\\
20.7	0\\
20.71	0\\
20.72	0\\
20.73	0\\
20.74	0\\
20.75	0\\
20.76	0\\
20.77	0\\
20.78	0\\
20.79	0\\
20.8	0\\
20.81	0\\
20.82	0\\
20.83	0\\
20.84	0\\
20.85	0\\
20.86	0\\
20.87	0\\
20.88	0\\
20.89	0\\
20.9	0\\
20.91	0\\
20.92	0\\
20.93	0\\
20.94	0\\
20.95	0\\
20.96	0\\
20.97	0\\
20.98	0\\
20.99	0\\
21	0\\
21.01	0\\
21.02	0\\
21.03	0\\
21.04	0\\
21.05	0\\
21.06	0\\
21.07	0\\
21.08	0\\
21.09	0\\
21.1	0\\
21.11	0\\
21.12	0\\
21.13	0\\
21.14	0\\
21.15	0\\
21.16	0\\
21.17	0\\
21.18	0\\
21.19	0\\
21.2	0\\
21.21	0\\
21.22	0\\
21.23	0\\
21.24	0\\
21.25	0\\
21.26	0\\
21.27	0\\
21.28	0\\
21.29	0\\
21.3	0\\
21.31	0\\
21.32	0\\
21.33	0\\
21.34	0\\
21.35	0\\
21.36	0\\
21.37	0\\
21.38	0\\
21.39	0\\
21.4	0\\
21.41	0\\
21.42	0\\
21.43	0\\
21.44	0\\
21.45	0\\
21.46	0\\
21.47	0\\
21.48	0\\
21.49	0\\
21.5	0\\
21.51	0\\
21.52	0\\
21.53	0\\
21.54	0\\
21.55	0\\
21.56	0\\
21.57	0\\
21.58	0\\
21.59	0\\
21.6	0\\
21.61	0\\
21.62	0\\
21.63	0\\
21.64	0\\
21.65	0\\
21.66	0\\
21.67	0\\
21.68	0\\
21.69	0\\
21.7	0\\
21.71	0\\
21.72	0\\
21.73	0\\
21.74	0\\
21.75	0\\
21.76	0\\
21.77	0\\
21.78	0\\
21.79	0\\
21.8	0\\
21.81	0\\
21.82	0\\
21.83	0\\
21.84	0\\
21.85	0\\
21.86	0\\
21.87	0\\
21.88	0\\
21.89	0\\
21.9	0\\
21.91	0\\
21.92	0\\
21.93	0\\
21.94	0\\
21.95	0\\
21.96	0\\
21.97	0\\
21.98	0\\
21.99	0\\
22	0\\
22.01	0\\
22.02	0\\
22.03	0\\
22.04	0\\
22.05	0\\
22.06	0\\
22.07	0\\
22.08	0\\
22.09	0\\
22.1	0\\
22.11	0\\
22.12	0\\
22.13	0\\
22.14	0\\
22.15	0\\
22.16	0\\
22.17	0\\
22.18	0\\
22.19	0\\
22.2	0\\
22.21	0\\
22.22	0\\
22.23	0\\
22.24	0\\
22.25	0\\
22.26	0\\
22.27	0\\
22.28	0\\
22.29	0\\
22.3	0\\
22.31	0\\
22.32	0\\
22.33	0\\
22.34	0\\
22.35	0\\
22.36	0\\
22.37	0\\
22.38	0\\
22.39	0\\
22.4	0\\
22.41	0\\
22.42	0\\
22.43	0\\
22.44	0\\
22.45	0\\
22.46	0\\
22.47	0\\
22.48	0\\
22.49	0\\
22.5	0\\
22.51	0\\
22.52	0\\
22.53	0\\
22.54	0\\
22.55	0\\
22.56	0\\
22.57	0\\
22.58	0\\
22.59	0\\
22.6	0\\
22.61	0\\
22.62	0\\
22.63	0\\
22.64	0\\
22.65	0\\
22.66	0\\
22.67	0\\
22.68	0\\
22.69	0\\
22.7	0\\
22.71	0\\
22.72	0\\
22.73	0\\
22.74	0\\
22.75	0\\
22.76	0\\
22.77	0\\
22.78	0\\
22.79	0\\
22.8	0\\
22.81	0\\
22.82	0\\
22.83	0\\
22.84	0\\
22.85	0\\
22.86	0\\
22.87	0\\
22.88	0\\
22.89	0\\
22.9	0\\
22.91	0\\
22.92	0\\
22.93	0\\
22.94	0\\
22.95	0\\
22.96	0\\
22.97	0\\
22.98	0\\
22.99	0\\
23	0\\
23.01	0\\
23.02	0\\
23.03	0\\
23.04	0\\
23.05	0\\
23.06	0\\
23.07	0\\
23.08	0\\
23.09	0\\
23.1	0\\
23.11	0\\
23.12	0\\
23.13	0\\
23.14	0\\
23.15	0\\
23.16	0\\
23.17	0\\
23.18	0\\
23.19	0\\
23.2	0\\
23.21	0\\
23.22	0\\
23.23	0\\
23.24	0\\
23.25	0\\
23.26	0\\
23.27	0\\
23.28	0\\
23.29	0\\
23.3	0\\
23.31	0\\
23.32	0\\
23.33	0\\
23.34	0\\
23.35	0\\
23.36	0\\
23.37	0\\
23.38	0\\
23.39	0\\
23.4	0\\
23.41	0\\
23.42	0\\
23.43	0\\
23.44	0\\
23.45	0\\
23.46	0\\
23.47	0\\
23.48	0\\
23.49	0\\
23.5	0\\
23.51	0\\
23.52	0\\
23.53	0\\
23.54	0\\
23.55	0\\
23.56	0\\
23.57	0\\
23.58	0\\
23.59	0\\
23.6	0\\
23.61	0\\
23.62	0\\
23.63	0\\
23.64	0\\
23.65	0\\
23.66	0\\
23.67	0\\
23.68	0\\
23.69	0\\
23.7	0\\
23.71	0\\
23.72	0\\
23.73	0\\
23.74	0\\
23.75	0\\
23.76	0\\
23.77	0\\
23.78	0\\
23.79	0\\
23.8	0\\
23.81	0\\
23.82	0\\
23.83	0\\
23.84	0\\
23.85	0\\
23.86	0\\
23.87	0\\
23.88	0\\
23.89	0\\
23.9	0\\
23.91	0\\
23.92	0\\
23.93	0\\
23.94	0\\
23.95	0\\
23.96	0\\
23.97	0\\
23.98	0\\
23.99	0\\
24	0\\
24.01	0\\
24.02	0\\
24.03	0\\
24.04	0\\
24.05	0\\
24.06	0\\
24.07	0\\
24.08	0\\
24.09	0\\
24.1	0\\
24.11	0\\
24.12	0\\
24.13	0\\
24.14	0\\
24.15	0\\
24.16	0\\
24.17	0\\
24.18	0\\
24.19	0\\
24.2	0\\
24.21	0\\
24.22	0\\
24.23	0\\
24.24	0\\
24.25	0\\
24.26	0\\
24.27	0\\
24.28	0\\
24.29	0\\
24.3	0\\
24.31	0\\
24.32	0\\
24.33	0\\
24.34	0\\
24.35	0\\
24.36	0\\
24.37	0\\
24.38	0\\
24.39	0\\
24.4	0\\
24.41	0\\
24.42	0\\
24.43	0\\
24.44	0\\
24.45	0\\
24.46	0\\
24.47	0\\
24.48	0\\
24.49	0\\
24.5	0\\
24.51	0\\
24.52	0\\
24.53	0\\
24.54	0\\
24.55	0\\
24.56	0\\
24.57	0\\
24.58	0\\
24.59	0\\
24.6	0\\
24.61	0\\
24.62	0\\
24.63	0\\
24.64	0\\
24.65	0\\
24.66	0\\
24.67	0\\
24.68	0\\
24.69	0\\
24.7	0\\
24.71	0\\
24.72	0\\
24.73	0\\
24.74	0\\
24.75	0\\
24.76	0\\
24.77	0\\
24.78	0\\
24.79	0\\
24.8	0\\
24.81	0\\
24.82	0\\
24.83	0\\
24.84	0\\
24.85	0\\
24.86	0\\
24.87	0\\
24.88	0\\
24.89	0\\
24.9	0\\
24.91	0\\
24.92	0\\
24.93	0\\
24.94	0\\
24.95	0\\
24.96	0\\
24.97	0\\
24.98	0\\
24.99	0\\
25	0\\
25.01	0\\
25.02	0\\
25.03	0\\
25.04	0\\
25.05	0\\
25.06	0\\
25.07	0\\
25.08	0\\
25.09	0\\
25.1	0\\
25.11	0\\
25.12	0\\
25.13	0\\
25.14	0\\
25.15	0\\
25.16	0\\
25.17	0\\
25.18	0\\
25.19	0\\
25.2	0\\
25.21	0\\
25.22	0\\
25.23	0\\
25.24	0\\
25.25	0\\
25.26	0\\
25.27	0\\
25.28	0\\
25.29	0\\
25.3	0\\
25.31	0\\
25.32	0\\
25.33	0\\
25.34	0\\
25.35	0\\
25.36	0\\
25.37	0\\
25.38	0\\
25.39	0\\
25.4	0\\
25.41	0\\
25.42	0\\
25.43	0\\
25.44	0\\
25.45	0\\
25.46	0\\
25.47	0\\
25.48	0\\
25.49	0\\
25.5	0\\
25.51	0\\
25.52	0\\
25.53	0\\
25.54	0\\
25.55	0\\
25.56	0\\
25.57	0\\
25.58	0\\
25.59	0\\
25.6	0\\
25.61	0\\
25.62	0\\
25.63	0\\
25.64	0\\
25.65	0\\
25.66	0\\
25.67	0\\
25.68	0\\
25.69	0\\
25.7	0\\
25.71	0\\
25.72	0\\
25.73	0\\
25.74	0\\
25.75	0\\
25.76	0\\
25.77	0\\
25.78	0\\
25.79	0\\
25.8	0\\
25.81	0\\
25.82	0\\
25.83	0\\
25.84	0\\
25.85	0\\
25.86	0\\
25.87	0\\
25.88	0\\
25.89	0\\
25.9	0\\
25.91	0\\
25.92	0\\
25.93	0\\
25.94	0\\
25.95	0\\
25.96	0\\
25.97	0\\
25.98	0\\
25.99	0\\
26	0\\
26.01	0\\
26.02	0\\
26.03	0\\
26.04	0\\
26.05	0\\
26.06	0\\
26.07	0\\
26.08	0\\
26.09	0\\
26.1	0\\
26.11	0\\
26.12	0\\
26.13	0\\
26.14	0\\
26.15	0\\
26.16	0\\
26.17	0\\
26.18	0\\
26.19	0\\
26.2	0\\
26.21	0\\
26.22	0\\
26.23	0\\
26.24	0\\
26.25	0\\
26.26	0\\
26.27	0\\
26.28	0\\
26.29	0\\
26.3	0\\
26.31	0\\
26.32	0\\
26.33	0\\
26.34	0\\
26.35	0\\
26.36	0\\
26.37	0\\
26.38	0\\
26.39	0\\
26.4	0\\
26.41	0\\
26.42	0\\
26.43	0\\
26.44	0\\
26.45	0\\
26.46	0\\
26.47	0\\
26.48	0\\
26.49	0\\
26.5	0\\
26.51	0\\
26.52	0\\
26.53	0\\
26.54	0\\
26.55	0\\
26.56	0\\
26.57	0\\
26.58	0\\
26.59	0\\
26.6	0\\
26.61	0\\
26.62	0\\
26.63	0\\
26.64	0\\
26.65	0\\
26.66	0\\
26.67	0\\
26.68	0\\
26.69	0\\
26.7	0\\
26.71	0\\
26.72	0\\
26.73	0\\
26.74	0\\
26.75	0\\
26.76	0\\
26.77	0\\
26.78	0\\
26.79	0\\
26.8	0\\
26.81	0\\
26.82	0\\
26.83	0\\
26.84	0\\
26.85	0\\
26.86	0\\
26.87	0\\
26.88	0\\
26.89	0\\
26.9	0\\
26.91	0\\
26.92	0\\
26.93	0\\
26.94	0\\
26.95	0\\
26.96	0\\
26.97	0\\
26.98	0\\
26.99	0\\
27	0\\
27.01	0\\
27.02	0\\
27.03	0\\
27.04	0\\
27.05	0\\
27.06	0\\
27.07	0\\
27.08	0\\
27.09	0\\
27.1	0\\
27.11	0\\
27.12	0\\
27.13	0\\
27.14	0\\
27.15	0\\
27.16	0\\
27.17	0\\
27.18	0\\
27.19	0\\
27.2	0\\
27.21	0\\
27.22	0\\
27.23	0\\
27.24	0\\
27.25	0\\
27.26	0\\
27.27	0\\
27.28	0\\
27.29	0\\
27.3	0\\
27.31	0\\
27.32	0\\
27.33	0\\
27.34	0\\
27.35	0\\
27.36	0\\
27.37	0\\
27.38	0\\
27.39	0\\
27.4	0\\
27.41	0\\
27.42	0\\
27.43	0\\
27.44	0\\
27.45	0\\
27.46	0\\
27.47	0\\
27.48	0\\
27.49	0\\
27.5	0\\
27.51	0\\
27.52	0\\
27.53	0\\
27.54	0\\
27.55	0\\
27.56	0\\
27.57	0\\
27.58	0\\
27.59	0\\
27.6	0\\
27.61	0\\
27.62	0\\
27.63	0\\
27.64	0\\
27.65	0\\
27.66	0\\
27.67	0\\
27.68	0\\
27.69	0\\
27.7	0\\
27.71	0\\
27.72	0\\
27.73	0\\
27.74	0\\
27.75	0\\
27.76	0\\
27.77	0\\
27.78	0\\
27.79	0\\
27.8	0\\
27.81	0\\
27.82	0\\
27.83	0\\
27.84	0\\
27.85	0\\
27.86	0\\
27.87	0\\
27.88	0\\
27.89	0\\
27.9	0\\
27.91	0\\
27.92	0\\
27.93	0\\
27.94	0\\
27.95	0\\
27.96	0\\
27.97	0\\
27.98	0\\
27.99	0\\
28	0\\
28.01	0\\
28.02	0\\
28.03	0\\
28.04	0\\
28.05	0\\
28.06	0\\
28.07	0\\
28.08	0\\
28.09	0\\
28.1	0\\
28.11	0\\
28.12	0\\
28.13	0\\
28.14	0\\
28.15	0\\
28.16	0\\
28.17	0\\
28.18	0\\
28.19	0\\
28.2	0\\
28.21	0\\
28.22	0\\
28.23	0\\
28.24	0\\
28.25	0\\
28.26	0\\
28.27	0\\
28.28	0\\
28.29	0\\
28.3	0\\
28.31	0\\
28.32	0\\
28.33	0\\
28.34	0\\
28.35	0\\
28.36	0\\
28.37	0\\
28.38	0\\
28.39	0\\
28.4	0\\
28.41	0\\
28.42	0\\
28.43	0\\
28.44	0\\
28.45	0\\
28.46	0\\
28.47	0\\
28.48	0\\
28.49	0\\
28.5	0\\
28.51	0\\
28.52	0\\
28.53	0\\
28.54	0\\
28.55	0\\
28.56	0\\
28.57	0\\
28.58	0\\
28.59	0\\
28.6	0\\
28.61	0\\
28.62	0\\
28.63	0\\
28.64	0\\
28.65	0\\
28.66	0\\
28.67	0\\
28.68	0\\
28.69	0\\
28.7	0\\
28.71	0\\
28.72	0\\
28.73	0\\
28.74	0\\
28.75	0\\
28.76	0\\
28.77	0\\
28.78	0\\
28.79	0\\
28.8	0\\
28.81	0\\
28.82	0\\
28.83	0\\
28.84	0\\
28.85	0\\
28.86	0\\
28.87	0\\
28.88	0\\
28.89	0\\
28.9	0\\
28.91	0\\
28.92	0\\
28.93	0\\
28.94	0\\
28.95	0\\
28.96	0\\
28.97	0\\
28.98	0\\
28.99	0\\
29	0\\
29.01	0\\
29.02	0\\
29.03	0\\
29.04	0\\
29.05	0\\
29.06	0\\
29.07	0\\
29.08	0\\
29.09	0\\
29.1	0\\
29.11	0\\
29.12	0\\
29.13	0\\
29.14	0\\
29.15	0\\
29.16	0\\
29.17	0\\
29.18	0\\
29.19	0\\
29.2	0\\
29.21	0\\
29.22	0\\
29.23	0\\
29.24	0\\
29.25	0\\
29.26	0\\
29.27	0\\
29.28	0\\
29.29	0\\
29.3	0\\
29.31	0\\
29.32	0\\
29.33	0\\
29.34	0\\
29.35	0\\
29.36	0\\
29.37	0\\
29.38	0\\
29.39	0\\
29.4	0\\
29.41	0\\
29.42	0\\
29.43	0\\
29.44	0\\
29.45	0\\
29.46	0\\
29.47	0\\
29.48	0\\
29.49	0\\
29.5	0\\
29.51	0\\
29.52	0\\
29.53	0\\
29.54	0\\
29.55	0\\
29.56	0\\
29.57	0\\
29.58	0\\
29.59	0\\
29.6	0\\
29.61	0\\
29.62	0\\
29.63	0\\
29.64	0\\
29.65	0\\
29.66	0\\
29.67	0\\
29.68	0\\
29.69	0\\
29.7	0\\
29.71	0\\
29.72	0\\
29.73	0\\
29.74	0\\
29.75	0\\
29.76	0\\
29.77	0\\
29.78	0\\
29.79	0\\
29.8	0\\
29.81	0\\
29.82	0\\
29.83	0\\
29.84	0\\
29.85	0\\
29.86	0\\
29.87	0\\
29.88	0\\
29.89	0\\
29.9	0\\
29.91	0\\
29.92	0\\
29.93	0\\
29.94	0\\
29.95	0\\
29.96	0\\
29.97	0\\
29.98	0\\
29.99	0\\
30	0\\
30.01	0\\
30.02	0\\
30.03	0\\
30.04	0\\
30.05	0\\
30.06	0\\
30.07	0\\
30.08	0\\
30.09	0\\
30.1	0\\
30.11	0\\
30.12	0\\
30.13	0\\
30.14	0\\
30.15	0\\
30.16	0\\
30.17	0\\
30.18	0\\
30.19	0\\
30.2	0\\
30.21	0\\
30.22	0\\
30.23	0\\
30.24	0\\
30.25	0\\
30.26	0\\
30.27	0\\
30.28	0\\
30.29	0\\
30.3	0\\
30.31	0\\
30.32	0\\
30.33	0\\
30.34	0\\
30.35	0\\
30.36	0\\
30.37	0\\
30.38	0\\
30.39	0\\
30.4	0\\
30.41	0\\
30.42	0\\
30.43	0\\
30.44	0\\
30.45	0\\
30.46	0\\
30.47	0\\
30.48	0\\
30.49	0\\
30.5	0\\
30.51	0\\
30.52	0\\
30.53	0\\
30.54	0\\
30.55	0\\
30.56	0\\
30.57	0\\
30.58	0\\
30.59	0\\
30.6	0\\
30.61	0\\
30.62	0\\
30.63	0\\
30.64	0\\
30.65	0\\
30.66	0\\
30.67	0\\
30.68	0\\
30.69	0\\
30.7	0\\
30.71	0\\
30.72	0\\
30.73	0\\
30.74	0\\
30.75	0\\
30.76	0\\
30.77	0\\
30.78	0\\
30.79	0\\
30.8	0\\
30.81	0\\
30.82	0\\
30.83	0\\
30.84	0\\
30.85	0\\
30.86	0\\
30.87	0\\
30.88	0\\
30.89	0\\
30.9	0\\
30.91	0\\
30.92	0\\
30.93	0\\
30.94	0\\
30.95	0\\
30.96	0\\
30.97	0\\
30.98	0\\
30.99	0\\
31	0\\
31.01	0\\
31.02	0\\
31.03	0\\
31.04	0\\
31.05	0\\
31.06	0\\
31.07	0\\
31.08	0\\
31.09	0\\
31.1	0\\
31.11	0\\
31.12	0\\
31.13	0\\
31.14	0\\
31.15	0\\
31.16	0\\
31.17	0\\
31.18	0\\
31.19	0\\
31.2	0\\
31.21	0\\
31.22	0\\
31.23	0\\
31.24	0\\
31.25	0\\
31.26	0\\
31.27	0\\
31.28	0\\
31.29	0\\
31.3	0\\
31.31	0\\
31.32	0\\
31.33	0\\
31.34	0\\
31.35	0\\
31.36	0\\
31.37	0\\
31.38	0\\
31.39	0\\
31.4	0\\
31.41	0\\
31.42	0\\
31.43	0\\
31.44	0\\
31.45	0\\
31.46	0\\
31.47	0\\
31.48	0\\
31.49	0\\
31.5	0\\
31.51	0\\
31.52	0\\
31.53	0\\
31.54	0\\
31.55	0\\
31.56	0\\
31.57	0\\
31.58	0\\
31.59	0\\
31.6	0\\
31.61	0\\
31.62	0\\
31.63	0\\
31.64	0\\
31.65	0\\
31.66	0\\
31.67	0\\
31.68	0\\
31.69	0\\
31.7	0\\
31.71	0\\
31.72	0\\
31.73	0\\
31.74	0\\
31.75	0\\
31.76	0\\
31.77	0\\
31.78	0\\
31.79	0\\
31.8	0\\
31.81	0\\
31.82	0\\
31.83	0\\
31.84	0\\
31.85	0\\
31.86	0\\
31.87	0\\
31.88	0\\
31.89	0\\
31.9	0\\
31.91	0\\
31.92	0\\
31.93	0\\
31.94	0\\
31.95	0\\
31.96	0\\
31.97	0\\
31.98	0\\
31.99	0\\
32	0\\
32.01	0\\
32.02	0\\
32.03	0\\
32.04	0\\
32.05	0\\
32.06	0\\
32.07	0\\
32.08	0\\
32.09	0\\
32.1	0\\
32.11	0\\
32.12	0\\
32.13	0\\
32.14	0\\
32.15	0\\
32.16	0\\
32.17	0\\
32.18	0\\
32.19	0\\
32.2	0\\
32.21	0\\
32.22	0\\
32.23	0\\
32.24	0\\
32.25	0\\
32.26	0\\
32.27	0\\
32.28	0\\
32.29	0\\
32.3	0\\
32.31	0\\
32.32	0\\
32.33	0\\
32.34	0\\
32.35	0\\
32.36	0\\
32.37	0\\
32.38	0\\
32.39	0\\
32.4	0\\
32.41	0\\
32.42	0\\
32.43	0\\
32.44	0\\
32.45	0\\
32.46	0\\
32.47	0\\
32.48	0\\
32.49	0\\
32.5	0\\
32.51	0\\
32.52	0\\
32.53	0\\
32.54	0\\
32.55	0\\
32.56	0\\
32.57	0\\
32.58	0\\
32.59	0\\
32.6	0\\
32.61	0\\
32.62	0\\
32.63	0\\
32.64	0\\
32.65	0\\
32.66	0\\
32.67	0\\
32.68	0\\
32.69	0\\
32.7	0\\
32.71	0\\
32.72	0\\
32.73	0\\
32.74	0\\
32.75	0\\
32.76	0\\
32.77	0\\
32.78	0\\
32.79	0\\
32.8	0\\
32.81	0\\
32.82	0\\
32.83	0\\
32.84	0\\
32.85	0\\
32.86	0\\
32.87	0\\
32.88	0\\
32.89	0\\
32.9	0\\
32.91	0\\
32.92	0\\
32.93	0\\
32.94	0\\
32.95	0\\
32.96	0\\
32.97	0\\
32.98	0\\
32.99	0\\
33	0\\
33.01	0\\
33.02	0\\
33.03	0\\
33.04	0\\
33.05	0\\
33.06	0\\
33.07	0\\
33.08	0\\
33.09	0\\
33.1	0\\
33.11	0\\
33.12	0\\
33.13	0\\
33.14	0\\
33.15	0\\
33.16	0\\
33.17	0\\
33.18	0\\
33.19	0\\
33.2	0\\
33.21	0\\
33.22	0\\
33.23	0\\
33.24	0\\
33.25	0\\
33.26	0\\
33.27	0\\
33.28	0\\
33.29	0\\
33.3	0\\
33.31	0\\
33.32	0\\
33.33	0\\
33.34	0\\
33.35	0\\
33.36	0\\
33.37	0\\
33.38	0\\
33.39	0\\
33.4	0\\
33.41	0\\
33.42	0\\
33.43	0\\
33.44	0\\
33.45	0\\
33.46	0\\
33.47	0\\
33.48	0\\
33.49	0\\
33.5	0\\
33.51	0\\
33.52	0\\
33.53	0\\
33.54	0\\
33.55	0\\
33.56	0\\
33.57	0\\
33.58	0\\
33.59	0\\
33.6	0\\
33.61	0\\
33.62	0\\
33.63	0\\
33.64	0\\
33.65	0\\
33.66	0\\
33.67	0\\
33.68	0\\
33.69	0\\
33.7	0\\
33.71	0\\
33.72	0\\
33.73	0\\
33.74	0\\
33.75	0\\
33.76	0\\
33.77	0\\
33.78	0\\
33.79	0\\
33.8	0\\
33.81	0\\
33.82	0\\
33.83	0\\
33.84	0\\
33.85	0\\
33.86	0\\
33.87	0\\
33.88	0\\
33.89	0\\
33.9	0\\
33.91	0\\
33.92	0\\
33.93	0\\
33.94	0\\
33.95	0\\
33.96	0\\
33.97	0\\
33.98	0\\
33.99	0\\
34	0\\
34.01	0\\
34.02	0\\
34.03	0\\
34.04	0\\
34.05	0\\
34.06	0\\
34.07	0\\
34.08	0\\
34.09	0\\
34.1	0\\
34.11	0\\
34.12	0\\
34.13	0\\
34.14	0\\
34.15	0\\
34.16	0\\
34.17	0\\
34.18	0\\
34.19	0\\
34.2	0\\
34.21	0\\
34.22	0\\
34.23	0\\
34.24	0\\
34.25	0\\
34.26	0\\
34.27	0\\
34.28	0\\
34.29	0\\
34.3	0\\
34.31	0\\
34.32	0\\
34.33	0\\
34.34	0\\
34.35	0\\
34.36	0\\
34.37	0\\
34.38	0\\
34.39	0\\
34.4	0\\
34.41	0\\
34.42	0\\
34.43	0\\
34.44	0\\
34.45	0\\
34.46	0\\
34.47	0\\
34.48	0\\
34.49	0\\
34.5	0\\
34.51	0\\
34.52	0\\
34.53	0\\
34.54	0\\
34.55	0\\
34.56	0\\
34.57	0\\
34.58	0\\
34.59	0\\
34.6	0\\
34.61	0\\
34.62	0\\
34.63	0\\
34.64	0\\
34.65	0\\
34.66	0\\
34.67	0\\
34.68	0\\
34.69	0\\
34.7	0\\
34.71	0\\
34.72	0\\
34.73	0\\
34.74	0\\
34.75	0\\
34.76	0\\
34.77	0\\
34.78	0\\
34.79	0\\
34.8	0\\
34.81	0\\
34.82	0\\
34.83	0\\
34.84	0\\
34.85	0\\
34.86	0\\
34.87	0\\
34.88	0\\
34.89	0\\
34.9	0\\
34.91	0\\
34.92	0\\
34.93	0\\
34.94	0\\
34.95	0\\
34.96	0\\
34.97	0\\
34.98	0\\
34.99	0\\
35	0\\
35.01	0\\
35.02	0\\
35.03	0\\
35.04	0\\
35.05	0\\
35.06	0\\
35.07	0\\
35.08	0\\
35.09	0\\
35.1	0\\
35.11	0\\
35.12	0\\
35.13	0\\
35.14	0\\
35.15	0\\
35.16	0\\
35.17	0\\
35.18	0\\
35.19	0\\
35.2	0\\
35.21	0\\
35.22	0\\
35.23	0\\
35.24	0\\
35.25	0\\
35.26	0\\
35.27	0\\
35.28	0\\
35.29	0\\
35.3	0\\
35.31	0\\
35.32	0\\
35.33	0\\
35.34	0\\
35.35	0\\
35.36	0\\
35.37	0\\
35.38	0\\
35.39	0\\
35.4	0\\
35.41	0\\
35.42	0\\
35.43	0\\
35.44	0\\
35.45	0\\
35.46	0\\
35.47	0\\
35.48	0\\
35.49	0\\
35.5	0\\
35.51	0\\
35.52	0\\
35.53	0\\
35.54	0\\
35.55	0\\
35.56	0\\
35.57	0\\
35.58	0\\
35.59	0\\
35.6	0\\
35.61	0\\
35.62	0\\
35.63	0\\
35.64	0\\
35.65	0\\
35.66	0\\
35.67	0\\
35.68	0\\
35.69	0\\
35.7	0\\
35.71	0\\
35.72	0\\
35.73	0\\
35.74	0\\
35.75	0\\
35.76	0\\
35.77	0\\
35.78	0\\
35.79	0\\
35.8	0\\
35.81	0\\
35.82	0\\
35.83	0\\
35.84	0\\
35.85	0\\
35.86	0\\
35.87	0\\
35.88	0\\
35.89	0\\
35.9	0\\
35.91	0\\
35.92	0\\
35.93	0\\
35.94	0\\
35.95	0\\
35.96	0\\
35.97	0\\
35.98	0\\
35.99	0\\
36	0\\
36.01	0\\
36.02	0\\
36.03	0\\
36.04	0\\
36.05	0\\
36.06	0\\
36.07	0\\
36.08	0\\
36.09	0\\
36.1	0\\
36.11	0\\
36.12	0\\
36.13	0\\
36.14	0\\
36.15	0\\
36.16	0\\
36.17	0\\
36.18	0\\
36.19	0\\
36.2	0\\
36.21	0\\
36.22	0\\
36.23	0\\
36.24	0\\
36.25	0\\
36.26	0\\
36.27	0\\
36.28	0\\
36.29	0\\
36.3	0\\
36.31	0\\
36.32	0\\
36.33	0\\
36.34	0\\
36.35	0\\
36.36	0\\
36.37	0\\
36.38	0\\
36.39	0\\
36.4	0\\
36.41	0\\
36.42	0\\
36.43	0\\
36.44	0\\
36.45	0\\
36.46	0\\
36.47	0\\
36.48	0\\
36.49	0\\
36.5	0\\
36.51	0\\
36.52	0\\
36.53	0\\
36.54	0\\
36.55	0\\
36.56	0\\
36.57	0\\
36.58	0\\
36.59	0\\
36.6	0\\
36.61	0\\
36.62	0\\
36.63	0\\
36.64	0\\
36.65	0\\
36.66	0\\
36.67	0\\
36.68	0\\
36.69	0\\
36.7	0\\
36.71	0\\
36.72	0\\
36.73	0\\
36.74	0\\
36.75	0\\
36.76	0\\
36.77	0\\
36.78	0\\
36.79	0\\
36.8	0\\
36.81	0\\
36.82	0\\
36.83	0\\
36.84	0\\
36.85	0\\
36.86	0\\
36.87	0\\
36.88	0\\
36.89	0\\
36.9	0\\
36.91	0\\
36.92	0\\
36.93	0\\
36.94	0\\
36.95	0\\
36.96	0\\
36.97	0\\
36.98	0\\
36.99	0\\
37	0\\
37.01	0\\
37.02	0\\
37.03	0\\
37.04	0\\
37.05	0\\
37.06	0\\
37.07	0\\
37.08	0\\
37.09	0\\
37.1	0\\
37.11	0\\
37.12	0\\
37.13	0\\
37.14	0\\
37.15	0\\
37.16	0\\
37.17	0\\
37.18	0\\
37.19	0\\
37.2	0\\
37.21	0\\
37.22	0\\
37.23	0\\
37.24	0\\
37.25	0\\
37.26	0\\
37.27	0\\
37.28	0\\
37.29	0\\
37.3	0\\
37.31	0\\
37.32	0\\
37.33	0\\
37.34	0\\
37.35	0\\
37.36	0\\
37.37	0\\
37.38	0\\
37.39	0\\
37.4	0\\
37.41	0\\
37.42	0\\
37.43	0\\
37.44	0\\
37.45	0\\
37.46	0\\
37.47	0\\
37.48	0\\
37.49	0\\
37.5	0\\
37.51	0\\
37.52	0\\
37.53	0\\
37.54	0\\
37.55	0\\
37.56	0\\
37.57	0\\
37.58	0\\
37.59	0\\
37.6	0\\
37.61	0\\
37.62	0\\
37.63	0\\
37.64	0\\
37.65	0\\
37.66	0\\
37.67	0\\
37.68	0\\
37.69	0\\
37.7	0\\
37.71	0\\
37.72	0\\
37.73	0\\
37.74	0\\
37.75	0\\
37.76	0\\
37.77	0\\
37.78	0\\
37.79	0\\
37.8	0\\
37.81	0\\
37.82	0\\
37.83	0\\
37.84	0\\
37.85	0\\
37.86	0\\
37.87	0\\
37.88	0\\
37.89	0\\
37.9	0\\
37.91	0\\
37.92	0\\
37.93	0\\
37.94	0\\
37.95	0\\
37.96	0\\
37.97	0\\
37.98	0\\
37.99	0\\
38	0\\
38.01	0\\
38.02	0\\
38.03	0\\
38.04	0\\
38.05	0\\
38.06	0\\
38.07	0\\
38.08	0\\
38.09	0\\
38.1	0\\
38.11	0\\
38.12	0\\
38.13	0\\
38.14	0\\
38.15	0\\
38.16	0\\
38.17	0\\
38.18	0\\
38.19	0\\
38.2	0\\
38.21	0\\
38.22	0\\
38.23	0\\
38.24	0\\
38.25	0\\
38.26	0\\
38.27	0\\
38.28	0\\
38.29	0\\
38.3	0\\
38.31	0\\
38.32	0\\
38.33	0\\
38.34	0\\
38.35	0\\
38.36	0\\
38.37	0\\
38.38	0\\
38.39	0\\
38.4	0\\
38.41	0\\
38.42	0\\
38.43	0\\
38.44	0\\
38.45	0\\
38.46	0\\
38.47	0\\
38.48	0\\
38.49	0\\
38.5	0\\
38.51	0\\
38.52	0\\
38.53	0\\
38.54	0\\
38.55	0\\
38.56	0\\
38.57	0\\
38.58	0\\
38.59	0\\
38.6	0\\
38.61	0\\
38.62	0\\
38.63	0\\
38.64	0\\
38.65	0\\
38.66	0\\
38.67	0\\
38.68	0\\
38.69	0\\
38.7	0\\
38.71	0\\
38.72	0\\
38.73	0\\
38.74	0\\
38.75	0\\
38.76	0\\
38.77	0\\
38.78	0\\
38.79	0\\
38.8	0\\
38.81	0\\
38.82	0\\
38.83	0\\
38.84	0\\
38.85	0\\
38.86	0\\
38.87	0\\
38.88	0\\
38.89	0\\
38.9	0\\
38.91	0\\
38.92	0\\
38.93	0\\
38.94	0\\
38.95	0\\
38.96	0\\
38.97	0\\
38.98	0\\
38.99	0\\
39	0\\
39.01	0\\
39.02	0\\
39.03	0\\
39.04	0\\
39.05	0\\
39.06	0\\
39.07	0\\
39.08	0\\
39.09	0\\
39.1	0\\
39.11	0\\
39.12	0\\
39.13	0\\
39.14	0\\
39.15	0\\
39.16	0\\
39.17	0\\
39.18	0\\
39.19	0\\
39.2	0\\
39.21	0\\
39.22	0\\
39.23	0\\
39.24	0\\
39.25	0\\
39.26	0\\
39.27	0\\
39.28	0\\
39.29	0\\
39.3	0\\
39.31	0\\
39.32	0\\
39.33	0\\
39.34	0\\
39.35	0\\
39.36	0\\
39.37	0\\
39.38	0\\
39.39	0\\
39.4	0\\
39.41	0\\
39.42	0\\
39.43	0\\
39.44	0\\
39.45	0\\
39.46	0\\
39.47	0\\
39.48	0\\
39.49	0\\
39.5	0\\
39.51	0\\
39.52	0\\
39.53	0\\
39.54	0\\
39.55	0\\
39.56	0\\
39.57	0\\
39.58	0\\
39.59	0\\
39.6	0\\
39.61	0\\
39.62	0\\
39.63	0\\
39.64	0\\
39.65	0\\
39.66	0\\
39.67	0\\
39.68	0\\
39.69	0\\
39.7	0\\
39.71	0\\
39.72	0\\
39.73	0\\
39.74	0\\
39.75	0\\
39.76	0\\
39.77	0\\
39.78	0\\
39.79	0\\
39.8	0\\
39.81	0\\
39.82	0\\
39.83	0\\
39.84	0\\
39.85	0\\
39.86	0\\
39.87	0\\
39.88	0\\
39.89	0\\
39.9	0\\
39.91	0\\
39.92	0\\
39.93	0\\
39.94	0\\
39.95	0\\
39.96	0\\
39.97	0\\
39.98	0\\
39.99	0\\
40	0\\
40.01	0\\
};
\addplot [color=red,dashed,forget plot]
  table[row sep=crcr]{%
40.01	0\\
40.02	0\\
40.03	0\\
40.04	0\\
40.05	0\\
40.06	0\\
40.07	0\\
40.08	0\\
40.09	0\\
40.1	0\\
40.11	0\\
40.12	0\\
40.13	0\\
40.14	0\\
40.15	0\\
40.16	0\\
40.17	0\\
40.18	0\\
40.19	0\\
40.2	0\\
40.21	0\\
40.22	0\\
40.23	0\\
40.24	0\\
40.25	0\\
40.26	0\\
40.27	0\\
40.28	0\\
40.29	0\\
40.3	0\\
40.31	0\\
40.32	0\\
40.33	0\\
40.34	0\\
40.35	0\\
40.36	0\\
40.37	0\\
40.38	0\\
40.39	0\\
40.4	0\\
40.41	0\\
40.42	0\\
40.43	0\\
40.44	0\\
40.45	0\\
40.46	0\\
40.47	0\\
40.48	0\\
40.49	0\\
40.5	0\\
40.51	0\\
40.52	0\\
40.53	0\\
40.54	0\\
40.55	0\\
40.56	0\\
40.57	0\\
40.58	0\\
40.59	0\\
40.6	0\\
40.61	0\\
40.62	0\\
40.63	0\\
40.64	0\\
40.65	0\\
40.66	0\\
40.67	0\\
40.68	0\\
40.69	0\\
40.7	0\\
40.71	0\\
40.72	0\\
40.73	0\\
40.74	0\\
40.75	0\\
40.76	0\\
40.77	0\\
40.78	0\\
40.79	0\\
40.8	0\\
40.81	0\\
40.82	0\\
40.83	0\\
40.84	0\\
40.85	0\\
40.86	0\\
40.87	0\\
40.88	0\\
40.89	0\\
40.9	0\\
40.91	0\\
40.92	0\\
40.93	0\\
40.94	0\\
40.95	0\\
40.96	0\\
40.97	0\\
40.98	0\\
40.99	0\\
41	0\\
41.01	0\\
41.02	0\\
41.03	0\\
41.04	0\\
41.05	0\\
41.06	0\\
41.07	0\\
41.08	0\\
41.09	0\\
41.1	0\\
41.11	0\\
41.12	0\\
41.13	0\\
41.14	0\\
41.15	0\\
41.16	0\\
41.17	0\\
41.18	0\\
41.19	0\\
41.2	0\\
41.21	0\\
41.22	0\\
41.23	0\\
41.24	0\\
41.25	0\\
41.26	0\\
41.27	0\\
41.28	0\\
41.29	0\\
41.3	0\\
41.31	0\\
41.32	0\\
41.33	0\\
41.34	0\\
41.35	0\\
41.36	0\\
41.37	0\\
41.38	0\\
41.39	0\\
41.4	0\\
41.41	0\\
41.42	0\\
41.43	0\\
41.44	0\\
41.45	0\\
41.46	0\\
41.47	0\\
41.48	0\\
41.49	0\\
41.5	0\\
41.51	0\\
41.52	0\\
41.53	0\\
41.54	0\\
41.55	0\\
41.56	0\\
41.57	0\\
41.58	0\\
41.59	0\\
41.6	0\\
41.61	0\\
41.62	0\\
41.63	0\\
41.64	0\\
41.65	0\\
41.66	0\\
41.67	0\\
41.68	0\\
41.69	0\\
41.7	0\\
41.71	0\\
41.72	0\\
41.73	0\\
41.74	0\\
41.75	0\\
41.76	0\\
41.77	0\\
41.78	0\\
41.79	0\\
41.8	0\\
41.81	0\\
41.82	0\\
41.83	0\\
41.84	0\\
41.85	0\\
41.86	0\\
41.87	0\\
41.88	0\\
41.89	0\\
41.9	0\\
41.91	0\\
41.92	0\\
41.93	0\\
41.94	0\\
41.95	0\\
41.96	0\\
41.97	0\\
41.98	0\\
41.99	0\\
42	0\\
42.01	0\\
42.02	0\\
42.03	0\\
42.04	0\\
42.05	0\\
42.06	0\\
42.07	0\\
42.08	0\\
42.09	0\\
42.1	0\\
42.11	0\\
42.12	0\\
42.13	0\\
42.14	0\\
42.15	0\\
42.16	0\\
42.17	0\\
42.18	0\\
42.19	0\\
42.2	0\\
42.21	0\\
42.22	0\\
42.23	0\\
42.24	0\\
42.25	0\\
42.26	0\\
42.27	0\\
42.28	0\\
42.29	0\\
42.3	0\\
42.31	0\\
42.32	0\\
42.33	0\\
42.34	0\\
42.35	0\\
42.36	0\\
42.37	0\\
42.38	0\\
42.39	0\\
42.4	0\\
42.41	0\\
42.42	0\\
42.43	0\\
42.44	0\\
42.45	0\\
42.46	0\\
42.47	0\\
42.48	0\\
42.49	0\\
42.5	0\\
42.51	0\\
42.52	0\\
42.53	0\\
42.54	0\\
42.55	0\\
42.56	0\\
42.57	0\\
42.58	0\\
42.59	0\\
42.6	0\\
42.61	0\\
42.62	0\\
42.63	0\\
42.64	0\\
42.65	0\\
42.66	0\\
42.67	0\\
42.68	0\\
42.69	0\\
42.7	0\\
42.71	0\\
42.72	0\\
42.73	0\\
42.74	0\\
42.75	0\\
42.76	0\\
42.77	0\\
42.78	0\\
42.79	0\\
42.8	0\\
42.81	0\\
42.82	0\\
42.83	0\\
42.84	0\\
42.85	0\\
42.86	0\\
42.87	0\\
42.88	0\\
42.89	0\\
42.9	0\\
42.91	0\\
42.92	0\\
42.93	0\\
42.94	0\\
42.95	0\\
42.96	0\\
42.97	0\\
42.98	0\\
42.99	0\\
43	0\\
43.01	0\\
43.02	0\\
43.03	0\\
43.04	0\\
43.05	0\\
43.06	0\\
43.07	0\\
43.08	0\\
43.09	0\\
43.1	0\\
43.11	0\\
43.12	0\\
43.13	0\\
43.14	0\\
43.15	0\\
43.16	0\\
43.17	0\\
43.18	0\\
43.19	0\\
43.2	0\\
43.21	0\\
43.22	0\\
43.23	0\\
43.24	0\\
43.25	0\\
43.26	0\\
43.27	0\\
43.28	0\\
43.29	0\\
43.3	0\\
43.31	0\\
43.32	0\\
43.33	0\\
43.34	0\\
43.35	0\\
43.36	0\\
43.37	0\\
43.38	0\\
43.39	0\\
43.4	0\\
43.41	0\\
43.42	0\\
43.43	0\\
43.44	0\\
43.45	0\\
43.46	0\\
43.47	0\\
43.48	0\\
43.49	0\\
43.5	0\\
43.51	0\\
43.52	0\\
43.53	0\\
43.54	0\\
43.55	0\\
43.56	0\\
43.57	0\\
43.58	0\\
43.59	0\\
43.6	0\\
43.61	0\\
43.62	0\\
43.63	0\\
43.64	0\\
43.65	0\\
43.66	0\\
43.67	0\\
43.68	0\\
43.69	0\\
43.7	0\\
43.71	0\\
43.72	0\\
43.73	0\\
43.74	0\\
43.75	0\\
43.76	0\\
43.77	0\\
43.78	0\\
43.79	0\\
43.8	0\\
43.81	0\\
43.82	0\\
43.83	0\\
43.84	0\\
43.85	0\\
43.86	0\\
43.87	0\\
43.88	0\\
43.89	0\\
43.9	0\\
43.91	0\\
43.92	0\\
43.93	0\\
43.94	0\\
43.95	0\\
43.96	0\\
43.97	0\\
43.98	0\\
43.99	0\\
44	0\\
44.01	0\\
44.02	0\\
44.03	0\\
44.04	0\\
44.05	0\\
44.06	0\\
44.07	0\\
44.08	0\\
44.09	0\\
44.1	0\\
44.11	0\\
44.12	0\\
44.13	0\\
44.14	0\\
44.15	0\\
44.16	0\\
44.17	0\\
44.18	0\\
44.19	0\\
44.2	0\\
44.21	0\\
44.22	0\\
44.23	0\\
44.24	0\\
44.25	0\\
44.26	0\\
44.27	0\\
44.28	0\\
44.29	0\\
44.3	0\\
44.31	0\\
44.32	0\\
44.33	0\\
44.34	0\\
44.35	0\\
44.36	0\\
44.37	0\\
44.38	0\\
44.39	0\\
44.4	0\\
44.41	0\\
44.42	0\\
44.43	0\\
44.44	0\\
44.45	0\\
44.46	0\\
44.47	0\\
44.48	0\\
44.49	0\\
44.5	0\\
44.51	0\\
44.52	0\\
44.53	0\\
44.54	0\\
44.55	0\\
44.56	0\\
44.57	0\\
44.58	0\\
44.59	0\\
44.6	0\\
44.61	0\\
44.62	0\\
44.63	0\\
44.64	0\\
44.65	0\\
44.66	0\\
44.67	0\\
44.68	0\\
44.69	0\\
44.7	0\\
44.71	0\\
44.72	0\\
44.73	0\\
44.74	0\\
44.75	0\\
44.76	0\\
44.77	0\\
44.78	0\\
44.79	0\\
44.8	0\\
44.81	0\\
44.82	0\\
44.83	0\\
44.84	0\\
44.85	0\\
44.86	0\\
44.87	0\\
44.88	0\\
44.89	0\\
44.9	0\\
44.91	0\\
44.92	0\\
44.93	0\\
44.94	0\\
44.95	0\\
44.96	0\\
44.97	0\\
44.98	0\\
44.99	0\\
45	0\\
45.01	0\\
45.02	0\\
45.03	0\\
45.04	0\\
45.05	0\\
45.06	0\\
45.07	0\\
45.08	0\\
45.09	0\\
45.1	0\\
45.11	0\\
45.12	0\\
45.13	0\\
45.14	0\\
45.15	0\\
45.16	0\\
45.17	0\\
45.18	0\\
45.19	0\\
45.2	0\\
45.21	0\\
45.22	0\\
45.23	0\\
45.24	0\\
45.25	0\\
45.26	0\\
45.27	0\\
45.28	0\\
45.29	0\\
45.3	0\\
45.31	0\\
45.32	0\\
45.33	0\\
45.34	0\\
45.35	0\\
45.36	0\\
45.37	0\\
45.38	0\\
45.39	0\\
45.4	0\\
45.41	0\\
45.42	0\\
45.43	0\\
45.44	0\\
45.45	0\\
45.46	0\\
45.47	0\\
45.48	0\\
45.49	0\\
45.5	0\\
45.51	0\\
45.52	0\\
45.53	0\\
45.54	0\\
45.55	0\\
45.56	0\\
45.57	0\\
45.58	0\\
45.59	0\\
45.6	0\\
45.61	0\\
45.62	0\\
45.63	0\\
45.64	0\\
45.65	0\\
45.66	0\\
45.67	0\\
45.68	0\\
45.69	0\\
45.7	0\\
45.71	0\\
45.72	0\\
45.73	0\\
45.74	0\\
45.75	0\\
45.76	0\\
45.77	0\\
45.78	0\\
45.79	0\\
45.8	0\\
45.81	0\\
45.82	0\\
45.83	0\\
45.84	0\\
45.85	0\\
45.86	0\\
45.87	0\\
45.88	0\\
45.89	0\\
45.9	0\\
45.91	0\\
45.92	0\\
45.93	0\\
45.94	0\\
45.95	0\\
45.96	0\\
45.97	0\\
45.98	0\\
45.99	0\\
46	0\\
46.01	0\\
46.02	0\\
46.03	0\\
46.04	0\\
46.05	0\\
46.06	0\\
46.07	0\\
46.08	0\\
46.09	0\\
46.1	0\\
46.11	0\\
46.12	0\\
46.13	0\\
46.14	0\\
46.15	0\\
46.16	0\\
46.17	0\\
46.18	0\\
46.19	0\\
46.2	0\\
46.21	0\\
46.22	0\\
46.23	0\\
46.24	0\\
46.25	0\\
46.26	0\\
46.27	0\\
46.28	0\\
46.29	0\\
46.3	0\\
46.31	0\\
46.32	0\\
46.33	0\\
46.34	0\\
46.35	0\\
46.36	0\\
46.37	0\\
46.38	0\\
46.39	0\\
46.4	0\\
46.41	0\\
46.42	0\\
46.43	0\\
46.44	0\\
46.45	0\\
46.46	0\\
46.47	0\\
46.48	0\\
46.49	0\\
46.5	0\\
46.51	0\\
46.52	0\\
46.53	0\\
46.54	0\\
46.55	0\\
46.56	0\\
46.57	0\\
46.58	0\\
46.59	0\\
46.6	0\\
46.61	0\\
46.62	0\\
46.63	0\\
46.64	0\\
46.65	0\\
46.66	0\\
46.67	0\\
46.68	0\\
46.69	0\\
46.7	0\\
46.71	0\\
46.72	0\\
46.73	0\\
46.74	0\\
46.75	0\\
46.76	0\\
46.77	0\\
46.78	0\\
46.79	0\\
46.8	0\\
46.81	0\\
46.82	0\\
46.83	0\\
46.84	0\\
46.85	0\\
46.86	0\\
46.87	0\\
46.88	0\\
46.89	0\\
46.9	0\\
46.91	0\\
46.92	0\\
46.93	0\\
46.94	0\\
46.95	0\\
46.96	0\\
46.97	0\\
46.98	0\\
46.99	0\\
47	0\\
47.01	0\\
47.02	0\\
47.03	0\\
47.04	0\\
47.05	0\\
47.06	0\\
47.07	0\\
47.08	0\\
47.09	0\\
47.1	0\\
47.11	0\\
47.12	0\\
47.13	0\\
47.14	0\\
47.15	0\\
47.16	0\\
47.17	0\\
47.18	0\\
47.19	0\\
47.2	0\\
47.21	0\\
47.22	0\\
47.23	0\\
47.24	0\\
47.25	0\\
47.26	0\\
47.27	0\\
47.28	0\\
47.29	0\\
47.3	0\\
47.31	0\\
47.32	0\\
47.33	0\\
47.34	0\\
47.35	0\\
47.36	0\\
47.37	0\\
47.38	0\\
47.39	0\\
47.4	0\\
47.41	0\\
47.42	0\\
47.43	0\\
47.44	0\\
47.45	0\\
47.46	0\\
47.47	0\\
47.48	0\\
47.49	0\\
47.5	0\\
47.51	0\\
47.52	0\\
47.53	0\\
47.54	0\\
47.55	0\\
47.56	0\\
47.57	0\\
47.58	0\\
47.59	0\\
47.6	0\\
47.61	0\\
47.62	0\\
47.63	0\\
47.64	0\\
47.65	0\\
47.66	0\\
47.67	0\\
47.68	0\\
47.69	0\\
47.7	0\\
47.71	0\\
47.72	0\\
47.73	0\\
47.74	0\\
47.75	0\\
47.76	0\\
47.77	0\\
47.78	0\\
47.79	0\\
47.8	0\\
47.81	0\\
47.82	0\\
47.83	0\\
47.84	0\\
47.85	0\\
47.86	0\\
47.87	0\\
47.88	0\\
47.89	0\\
47.9	0\\
47.91	0\\
47.92	0\\
47.93	0\\
47.94	0\\
47.95	0\\
47.96	0\\
47.97	0\\
47.98	0\\
47.99	0\\
48	0\\
48.01	0\\
48.02	0\\
48.03	0\\
48.04	0\\
48.05	0\\
48.06	0\\
48.07	0\\
48.08	0\\
48.09	0\\
48.1	0\\
48.11	0\\
48.12	0\\
48.13	0\\
48.14	0\\
48.15	0\\
48.16	0\\
48.17	0\\
48.18	0\\
48.19	0\\
48.2	0\\
48.21	0\\
48.22	0\\
48.23	0\\
48.24	0\\
48.25	0\\
48.26	0\\
48.27	0\\
48.28	0\\
48.29	0\\
48.3	0\\
48.31	0\\
48.32	0\\
48.33	0\\
48.34	0\\
48.35	0\\
48.36	0\\
48.37	0\\
48.38	0\\
48.39	0\\
48.4	0\\
48.41	0\\
48.42	0\\
48.43	0\\
48.44	0\\
48.45	0\\
48.46	0\\
48.47	0\\
48.48	0\\
48.49	0\\
48.5	0\\
48.51	0\\
48.52	0\\
48.53	0\\
48.54	0\\
48.55	0\\
48.56	0\\
48.57	0\\
48.58	0\\
48.59	0\\
48.6	0\\
48.61	0\\
48.62	0\\
48.63	0\\
48.64	0\\
48.65	0\\
48.66	0\\
48.67	0\\
48.68	0\\
48.69	0\\
48.7	0\\
48.71	0\\
48.72	0\\
48.73	0\\
48.74	0\\
48.75	0\\
48.76	0\\
48.77	0\\
48.78	0\\
48.79	0\\
48.8	0\\
48.81	0\\
48.82	0\\
48.83	0\\
48.84	0\\
48.85	0\\
48.86	0\\
48.87	0\\
48.88	0\\
48.89	0\\
48.9	0\\
48.91	0\\
48.92	0\\
48.93	0\\
48.94	0\\
48.95	0\\
48.96	0\\
48.97	0\\
48.98	0\\
48.99	0\\
49	0\\
49.01	0\\
49.02	0\\
49.03	0\\
49.04	0\\
49.05	0\\
49.06	0\\
49.07	0\\
49.08	0\\
49.09	0\\
49.1	0\\
49.11	0\\
49.12	0\\
49.13	0\\
49.14	0\\
49.15	0\\
49.16	0\\
49.17	0\\
49.18	0\\
49.19	0\\
49.2	0\\
49.21	0\\
49.22	0\\
49.23	0\\
49.24	0\\
49.25	0\\
49.26	0\\
49.27	0\\
49.28	0\\
49.29	0\\
49.3	0\\
49.31	0\\
49.32	0\\
49.33	0\\
49.34	0\\
49.35	0\\
49.36	0\\
49.37	0\\
49.38	0\\
49.39	0\\
49.4	0\\
49.41	0\\
49.42	0\\
49.43	0\\
49.44	0\\
49.45	0\\
49.46	0\\
49.47	0\\
49.48	0\\
49.49	0\\
49.5	0\\
49.51	0\\
49.52	0\\
49.53	0\\
49.54	0\\
49.55	0\\
49.56	0\\
49.57	0\\
49.58	0\\
49.59	0\\
49.6	0\\
49.61	0\\
49.62	0\\
49.63	0\\
49.64	0\\
49.65	0\\
49.66	0\\
49.67	0\\
49.68	0\\
49.69	0\\
49.7	0\\
49.71	0\\
49.72	0\\
49.73	0\\
49.74	0\\
49.75	0\\
49.76	0\\
49.77	0\\
49.78	0\\
49.79	0\\
49.8	0\\
49.81	0\\
49.82	0\\
49.83	0\\
49.84	0\\
49.85	0\\
49.86	0\\
49.87	0\\
49.88	0\\
49.89	0\\
49.9	0\\
49.91	0\\
49.92	0\\
49.93	0\\
49.94	0\\
49.95	0\\
49.96	0\\
49.97	0\\
49.98	0\\
49.99	0\\
50	0\\
50.01	0\\
50.02	0\\
50.03	0\\
50.04	0\\
50.05	0\\
50.06	0\\
50.07	0\\
50.08	0\\
50.09	0\\
50.1	0\\
50.11	0\\
50.12	0\\
50.13	0\\
50.14	0\\
50.15	0\\
50.16	0\\
50.17	0\\
50.18	0\\
50.19	0\\
50.2	0\\
50.21	0\\
50.22	0\\
50.23	0\\
50.24	0\\
50.25	0\\
50.26	0\\
50.27	0\\
50.28	0\\
50.29	0\\
50.3	0\\
50.31	0\\
50.32	0\\
50.33	0\\
50.34	0\\
50.35	0\\
50.36	0\\
50.37	0\\
50.38	0\\
50.39	0\\
50.4	0\\
50.41	0\\
50.42	0\\
50.43	0\\
50.44	0\\
50.45	0\\
50.46	0\\
50.47	0\\
50.48	0\\
50.49	0\\
50.5	0\\
50.51	0\\
50.52	0\\
50.53	0\\
50.54	0\\
50.55	0\\
50.56	0\\
50.57	0\\
50.58	0\\
50.59	0\\
50.6	0\\
50.61	0\\
50.62	0\\
50.63	0\\
50.64	0\\
50.65	0\\
50.66	0\\
50.67	0\\
50.68	0\\
50.69	0\\
50.7	0\\
50.71	0\\
50.72	0\\
50.73	0\\
50.74	0\\
50.75	0\\
50.76	0\\
50.77	0\\
50.78	0\\
50.79	0\\
50.8	0\\
50.81	0\\
50.82	0\\
50.83	0\\
50.84	0\\
50.85	0\\
50.86	0\\
50.87	0\\
50.88	0\\
50.89	0\\
50.9	0\\
50.91	0\\
50.92	0\\
50.93	0\\
50.94	0\\
50.95	0\\
50.96	0\\
50.97	0\\
50.98	0\\
50.99	0\\
51	0\\
51.01	0\\
51.02	0\\
51.03	0\\
51.04	0\\
51.05	0\\
51.06	0\\
51.07	0\\
51.08	0\\
51.09	0\\
51.1	0\\
51.11	0\\
51.12	0\\
51.13	0\\
51.14	0\\
51.15	0\\
51.16	0\\
51.17	0\\
51.18	0\\
51.19	0\\
51.2	0\\
51.21	0\\
51.22	0\\
51.23	0\\
51.24	0\\
51.25	0\\
51.26	0\\
51.27	0\\
51.28	0\\
51.29	0\\
51.3	0\\
51.31	0\\
51.32	0\\
51.33	0\\
51.34	0\\
51.35	0\\
51.36	0\\
51.37	0\\
51.38	0\\
51.39	0\\
51.4	0\\
51.41	0\\
51.42	0\\
51.43	0\\
51.44	0\\
51.45	0\\
51.46	0\\
51.47	0\\
51.48	0\\
51.49	0\\
51.5	0\\
51.51	0\\
51.52	0\\
51.53	0\\
51.54	0\\
51.55	0\\
51.56	0\\
51.57	0\\
51.58	0\\
51.59	0\\
51.6	0\\
51.61	0\\
51.62	0\\
51.63	0\\
51.64	0\\
51.65	0\\
51.66	0\\
51.67	0\\
51.68	0\\
51.69	0\\
51.7	0\\
51.71	0\\
51.72	0\\
51.73	0\\
51.74	0\\
51.75	0\\
51.76	0\\
51.77	0\\
51.78	0\\
51.79	0\\
51.8	0\\
51.81	0\\
51.82	0\\
51.83	0\\
51.84	0\\
51.85	0\\
51.86	0\\
51.87	0\\
51.88	0\\
51.89	0\\
51.9	0\\
51.91	0\\
51.92	0\\
51.93	0\\
51.94	0\\
51.95	0\\
51.96	0\\
51.97	0\\
51.98	0\\
51.99	0\\
52	0\\
52.01	0\\
52.02	0\\
52.03	0\\
52.04	0\\
52.05	0\\
52.06	0\\
52.07	0\\
52.08	0\\
52.09	0\\
52.1	0\\
52.11	0\\
52.12	0\\
52.13	0\\
52.14	0\\
52.15	0\\
52.16	0\\
52.17	0\\
52.18	0\\
52.19	0\\
52.2	0\\
52.21	0\\
52.22	0\\
52.23	0\\
52.24	0\\
52.25	0\\
52.26	0\\
52.27	0\\
52.28	0\\
52.29	0\\
52.3	0\\
52.31	0\\
52.32	0\\
52.33	0\\
52.34	0\\
52.35	0\\
52.36	0\\
52.37	0\\
52.38	0\\
52.39	0\\
52.4	0\\
52.41	0\\
52.42	0\\
52.43	0\\
52.44	0\\
52.45	0\\
52.46	0\\
52.47	0\\
52.48	0\\
52.49	0\\
52.5	0\\
52.51	0\\
52.52	0\\
52.53	0\\
52.54	0\\
52.55	0\\
52.56	0\\
52.57	0\\
52.58	0\\
52.59	0\\
52.6	0\\
52.61	0\\
52.62	0\\
52.63	0\\
52.64	0\\
52.65	0\\
52.66	0\\
52.67	0\\
52.68	0\\
52.69	0\\
52.7	0\\
52.71	0\\
52.72	0\\
52.73	0\\
52.74	0\\
52.75	0\\
52.76	0\\
52.77	0\\
52.78	0\\
52.79	0\\
52.8	0\\
52.81	0\\
52.82	0\\
52.83	0\\
52.84	0\\
52.85	0\\
52.86	0\\
52.87	0\\
52.88	0\\
52.89	0\\
52.9	0\\
52.91	0\\
52.92	0\\
52.93	0\\
52.94	0\\
52.95	0\\
52.96	0\\
52.97	0\\
52.98	0\\
52.99	0\\
53	0\\
53.01	0\\
53.02	0\\
53.03	0\\
53.04	0\\
53.05	0\\
53.06	0\\
53.07	0\\
53.08	0\\
53.09	0\\
53.1	0\\
53.11	0\\
53.12	0\\
53.13	0\\
53.14	0\\
53.15	0\\
53.16	0\\
53.17	0\\
53.18	0\\
53.19	0\\
53.2	0\\
53.21	0\\
53.22	0\\
53.23	0\\
53.24	0\\
53.25	0\\
53.26	0\\
53.27	0\\
53.28	0\\
53.29	0\\
53.3	0\\
53.31	0\\
53.32	0\\
53.33	0\\
53.34	0\\
53.35	0\\
53.36	0\\
53.37	0\\
53.38	0\\
53.39	0\\
53.4	0\\
53.41	0\\
53.42	0\\
53.43	0\\
53.44	0\\
53.45	0\\
53.46	0\\
53.47	0\\
53.48	0\\
53.49	0\\
53.5	0\\
53.51	0\\
53.52	0\\
53.53	0\\
53.54	0\\
53.55	0\\
53.56	0\\
53.57	0\\
53.58	0\\
53.59	0\\
53.6	0\\
53.61	0\\
53.62	0\\
53.63	0\\
53.64	0\\
53.65	0\\
53.66	0\\
53.67	0\\
53.68	0\\
53.69	0\\
53.7	0\\
53.71	0\\
53.72	0\\
53.73	0\\
53.74	0\\
53.75	0\\
53.76	0\\
53.77	0\\
53.78	0\\
53.79	0\\
53.8	0\\
53.81	0\\
53.82	0\\
53.83	0\\
53.84	0\\
53.85	0\\
53.86	0\\
53.87	0\\
53.88	0\\
53.89	0\\
53.9	0\\
53.91	0\\
53.92	0\\
53.93	0\\
53.94	0\\
53.95	0\\
53.96	0\\
53.97	0\\
53.98	0\\
53.99	0\\
54	0\\
54.01	0\\
54.02	0\\
54.03	0\\
54.04	0\\
54.05	0\\
54.06	0\\
54.07	0\\
54.08	0\\
54.09	0\\
54.1	0\\
54.11	0\\
54.12	0\\
54.13	0\\
54.14	0\\
54.15	0\\
54.16	0\\
54.17	0\\
54.18	0\\
54.19	0\\
54.2	0\\
54.21	0\\
54.22	0\\
54.23	0\\
54.24	0\\
54.25	0\\
54.26	0\\
54.27	0\\
54.28	0\\
54.29	0\\
54.3	0\\
54.31	0\\
54.32	0\\
54.33	0\\
54.34	0\\
54.35	0\\
54.36	0\\
54.37	0\\
54.38	0\\
54.39	0\\
54.4	0\\
54.41	0\\
54.42	0\\
54.43	0\\
54.44	0\\
54.45	0\\
54.46	0\\
54.47	0\\
54.48	0\\
54.49	0\\
54.5	0\\
54.51	0\\
54.52	0\\
54.53	0\\
54.54	0\\
54.55	0\\
54.56	0\\
54.57	0\\
54.58	0\\
54.59	0\\
54.6	0\\
54.61	0\\
54.62	0\\
54.63	0\\
54.64	0\\
54.65	0\\
54.66	0\\
54.67	0\\
54.68	0\\
54.69	0\\
54.7	0\\
54.71	0\\
54.72	0\\
54.73	0\\
54.74	0\\
54.75	0\\
54.76	0\\
54.77	0\\
54.78	0\\
54.79	0\\
54.8	0\\
54.81	0\\
54.82	0\\
54.83	0\\
54.84	0\\
54.85	0\\
54.86	0\\
54.87	0\\
54.88	0\\
54.89	0\\
54.9	0\\
54.91	0\\
54.92	0\\
54.93	0\\
54.94	0\\
54.95	0\\
54.96	0\\
54.97	0\\
54.98	0\\
54.99	0\\
55	0\\
55.01	0\\
55.02	0\\
55.03	0\\
55.04	0\\
55.05	0\\
55.06	0\\
55.07	0\\
55.08	0\\
55.09	0\\
55.1	0\\
55.11	0\\
55.12	0\\
55.13	0\\
55.14	0\\
55.15	0\\
55.16	0\\
55.17	0\\
55.18	0\\
55.19	0\\
55.2	0\\
55.21	0\\
55.22	0\\
55.23	0\\
55.24	0\\
55.25	0\\
55.26	0\\
55.27	0\\
55.28	0\\
55.29	0\\
55.3	0\\
55.31	0\\
55.32	0\\
55.33	0\\
55.34	0\\
55.35	0\\
55.36	0\\
55.37	0\\
55.38	0\\
55.39	0\\
55.4	0\\
55.41	0\\
55.42	0\\
55.43	0\\
55.44	0\\
55.45	0\\
55.46	0\\
55.47	0\\
55.48	0\\
55.49	0\\
55.5	0\\
55.51	0\\
55.52	0\\
55.53	0\\
55.54	0\\
55.55	0\\
55.56	0\\
55.57	0\\
55.58	0\\
55.59	0\\
55.6	0\\
55.61	0\\
55.62	0\\
55.63	0\\
55.64	0\\
55.65	0\\
55.66	0\\
55.67	0\\
55.68	0\\
55.69	0\\
55.7	0\\
55.71	0\\
55.72	0\\
55.73	0\\
55.74	0\\
55.75	0\\
55.76	0\\
55.77	0\\
55.78	0\\
55.79	0\\
55.8	0\\
55.81	0\\
55.82	0\\
55.83	0\\
55.84	0\\
55.85	0\\
55.86	0\\
55.87	0\\
55.88	0\\
55.89	0\\
55.9	0\\
55.91	0\\
55.92	0\\
55.93	0\\
55.94	0\\
55.95	0\\
55.96	0\\
55.97	0\\
55.98	0\\
55.99	0\\
56	0\\
56.01	0\\
56.02	0\\
56.03	0\\
56.04	0\\
56.05	0\\
56.06	0\\
56.07	0\\
56.08	0\\
56.09	0\\
56.1	0\\
56.11	0\\
56.12	0\\
56.13	0\\
56.14	0\\
56.15	0\\
56.16	0\\
56.17	0\\
56.18	0\\
56.19	0\\
56.2	0\\
56.21	0\\
56.22	0\\
56.23	0\\
56.24	0\\
56.25	0\\
56.26	0\\
56.27	0\\
56.28	0\\
56.29	0\\
56.3	0\\
56.31	0\\
56.32	0\\
56.33	0\\
56.34	0\\
56.35	0\\
56.36	0\\
56.37	0\\
56.38	0\\
56.39	0\\
56.4	0\\
56.41	0\\
56.42	0\\
56.43	0\\
56.44	0\\
56.45	0\\
56.46	0\\
56.47	0\\
56.48	0\\
56.49	0\\
56.5	0\\
56.51	0\\
56.52	0\\
56.53	0\\
56.54	0\\
56.55	0\\
56.56	0\\
56.57	0\\
56.58	0\\
56.59	0\\
56.6	0\\
56.61	0\\
56.62	0\\
56.63	0\\
56.64	0\\
56.65	0\\
56.66	0\\
56.67	0\\
56.68	0\\
56.69	0\\
56.7	0\\
56.71	0\\
56.72	0\\
56.73	0\\
56.74	0\\
56.75	0\\
56.76	0\\
56.77	0\\
56.78	0\\
56.79	0\\
56.8	0\\
56.81	0\\
56.82	0\\
56.83	0\\
56.84	0\\
56.85	0\\
56.86	0\\
56.87	0\\
56.88	0\\
56.89	0\\
56.9	0\\
56.91	0\\
56.92	0\\
56.93	0\\
56.94	0\\
56.95	0\\
56.96	0\\
56.97	0\\
56.98	0\\
56.99	0\\
57	0\\
57.01	0\\
57.02	0\\
57.03	0\\
57.04	0\\
57.05	0\\
57.06	0\\
57.07	0\\
57.08	0\\
57.09	0\\
57.1	0\\
57.11	0\\
57.12	0\\
57.13	0\\
57.14	0\\
57.15	0\\
57.16	0\\
57.17	0\\
57.18	0\\
57.19	0\\
57.2	0\\
57.21	0\\
57.22	0\\
57.23	0\\
57.24	0\\
57.25	0\\
57.26	0\\
57.27	0\\
57.28	0\\
57.29	0\\
57.3	0\\
57.31	0\\
57.32	0\\
57.33	0\\
57.34	0\\
57.35	0\\
57.36	0\\
57.37	0\\
57.38	0\\
57.39	0\\
57.4	0\\
57.41	0\\
57.42	0\\
57.43	0\\
57.44	0\\
57.45	0\\
57.46	0\\
57.47	0\\
57.48	0\\
57.49	0\\
57.5	0\\
57.51	0\\
57.52	0\\
57.53	0\\
57.54	0\\
57.55	0\\
57.56	0\\
57.57	0\\
57.58	0\\
57.59	0\\
57.6	0\\
57.61	0\\
57.62	0\\
57.63	0\\
57.64	0\\
57.65	0\\
57.66	0\\
57.67	0\\
57.68	0\\
57.69	0\\
57.7	0\\
57.71	0\\
57.72	0\\
57.73	0\\
57.74	0\\
57.75	0\\
57.76	0\\
57.77	0\\
57.78	0\\
57.79	0\\
57.8	0\\
57.81	0\\
57.82	0\\
57.83	0\\
57.84	0\\
57.85	0\\
57.86	0\\
57.87	0\\
57.88	0\\
57.89	0\\
57.9	0\\
57.91	0\\
57.92	0\\
57.93	0\\
57.94	0\\
57.95	0\\
57.96	0\\
57.97	0\\
57.98	0\\
57.99	0\\
58	0\\
58.01	0\\
58.02	0\\
58.03	0\\
58.04	0\\
58.05	0\\
58.06	0\\
58.07	0\\
58.08	0\\
58.09	0\\
58.1	0\\
58.11	0\\
58.12	0\\
58.13	0\\
58.14	0\\
58.15	0\\
58.16	0\\
58.17	0\\
58.18	0\\
58.19	0\\
58.2	0\\
58.21	0\\
58.22	0\\
58.23	0\\
58.24	0\\
58.25	0\\
58.26	0\\
58.27	0\\
58.28	0\\
58.29	0\\
58.3	0\\
58.31	0\\
58.32	0\\
58.33	0\\
58.34	0\\
58.35	0\\
58.36	0\\
58.37	0\\
58.38	0\\
58.39	0\\
58.4	0\\
58.41	0\\
58.42	0\\
58.43	0\\
58.44	0\\
58.45	0\\
58.46	0\\
58.47	0\\
58.48	0\\
58.49	0\\
58.5	0\\
58.51	0\\
58.52	0\\
58.53	0\\
58.54	0\\
58.55	0\\
58.56	0\\
58.57	0\\
58.58	0\\
58.59	0\\
58.6	0\\
58.61	0\\
58.62	0\\
58.63	0\\
58.64	0\\
58.65	0\\
58.66	0\\
58.67	0\\
58.68	0\\
58.69	0\\
58.7	0\\
58.71	0\\
58.72	0\\
58.73	0\\
58.74	0\\
58.75	0\\
58.76	0\\
58.77	0\\
58.78	0\\
58.79	0\\
58.8	0\\
58.81	0\\
58.82	0\\
58.83	0\\
58.84	0\\
58.85	0\\
58.86	0\\
58.87	0\\
58.88	0\\
58.89	0\\
58.9	0\\
58.91	0\\
58.92	0\\
58.93	0\\
58.94	0\\
58.95	0\\
58.96	0\\
58.97	0\\
58.98	0\\
58.99	0\\
59	0\\
59.01	0\\
59.02	0\\
59.03	0\\
59.04	0\\
59.05	0\\
59.06	0\\
59.07	0\\
59.08	0\\
59.09	0\\
59.1	0\\
59.11	0\\
59.12	0\\
59.13	0\\
59.14	0\\
59.15	0\\
59.16	0\\
59.17	0\\
59.18	0\\
59.19	0\\
59.2	0\\
59.21	0\\
59.22	0\\
59.23	0\\
59.24	0\\
59.25	0\\
59.26	0\\
59.27	0\\
59.28	0\\
59.29	0\\
59.3	0\\
59.31	0\\
59.32	0\\
59.33	0\\
59.34	0\\
59.35	0\\
59.36	0\\
59.37	0\\
59.38	0\\
59.39	0\\
59.4	0\\
59.41	0\\
59.42	0\\
59.43	0\\
59.44	0\\
59.45	0\\
59.46	0\\
59.47	0\\
59.48	0\\
59.49	0\\
59.5	0\\
59.51	0\\
59.52	0\\
59.53	0\\
59.54	0\\
59.55	0\\
59.56	0\\
59.57	0\\
59.58	0\\
59.59	0\\
59.6	0\\
59.61	0\\
59.62	0\\
59.63	0\\
59.64	0\\
59.65	0\\
59.66	0\\
59.67	0\\
59.68	0\\
59.69	0\\
59.7	0\\
59.71	0\\
59.72	0\\
59.73	0\\
59.74	0\\
59.75	0\\
59.76	0\\
59.77	0\\
59.78	0\\
59.79	0\\
59.8	0\\
59.81	0\\
59.82	0\\
59.83	0\\
59.84	0\\
59.85	0\\
59.86	0\\
59.87	0\\
59.88	0\\
59.89	0\\
59.9	0\\
59.91	0\\
59.92	0\\
59.93	0\\
59.94	0\\
59.95	0\\
59.96	0\\
59.97	0\\
59.98	0\\
59.99	0\\
60	0\\
60.01	0\\
60.02	0\\
60.03	0\\
60.04	0\\
60.05	0\\
60.06	0\\
60.07	0\\
60.08	0\\
60.09	0\\
60.1	0\\
60.11	0\\
60.12	0\\
60.13	0\\
60.14	0\\
60.15	0\\
60.16	0\\
60.17	0\\
60.18	0\\
60.19	0\\
60.2	0\\
60.21	0\\
60.22	0\\
60.23	0\\
60.24	0\\
60.25	0\\
60.26	0\\
60.27	0\\
60.28	0\\
60.29	0\\
60.3	0\\
60.31	0\\
60.32	0\\
60.33	0\\
60.34	0\\
60.35	0\\
60.36	0\\
60.37	0\\
60.38	0\\
60.39	0\\
60.4	0\\
60.41	0\\
60.42	0\\
60.43	0\\
60.44	0\\
60.45	0\\
60.46	0\\
60.47	0\\
60.48	0\\
60.49	0\\
60.5	0\\
60.51	0\\
60.52	0\\
60.53	0\\
60.54	0\\
60.55	0\\
60.56	0\\
60.57	0\\
60.58	0\\
60.59	0\\
60.6	0\\
60.61	0\\
60.62	0\\
60.63	0\\
60.64	0\\
60.65	0\\
60.66	0\\
60.67	0\\
60.68	0\\
60.69	0\\
60.7	0\\
60.71	0\\
60.72	0\\
60.73	0\\
60.74	0\\
60.75	0\\
60.76	0\\
60.77	0\\
60.78	0\\
60.79	0\\
60.8	0\\
60.81	0\\
60.82	0\\
60.83	0\\
60.84	0\\
60.85	0\\
60.86	0\\
60.87	0\\
60.88	0\\
60.89	0\\
60.9	0\\
60.91	0\\
60.92	0\\
60.93	0\\
60.94	0\\
60.95	0\\
60.96	0\\
60.97	0\\
60.98	0\\
60.99	0\\
61	0\\
61.01	0\\
61.02	0\\
61.03	0\\
61.04	0\\
61.05	0\\
61.06	0\\
61.07	0\\
61.08	0\\
61.09	0\\
61.1	0\\
61.11	0\\
61.12	0\\
61.13	0\\
61.14	0\\
61.15	0\\
61.16	0\\
61.17	0\\
61.18	0\\
61.19	0\\
61.2	0\\
61.21	0\\
61.22	0\\
61.23	0\\
61.24	0\\
61.25	0\\
61.26	0\\
61.27	0\\
61.28	0\\
61.29	0\\
61.3	0\\
61.31	0\\
61.32	0\\
61.33	0\\
61.34	0\\
61.35	0\\
61.36	0\\
61.37	0\\
61.38	0\\
61.39	0\\
61.4	0\\
61.41	0\\
61.42	0\\
61.43	0\\
61.44	0\\
61.45	0\\
61.46	0\\
61.47	0\\
61.48	0\\
61.49	0\\
61.5	0\\
61.51	0\\
61.52	0\\
61.53	0\\
61.54	0\\
61.55	0\\
61.56	0\\
61.57	0\\
61.58	0\\
61.59	0\\
61.6	0\\
61.61	0\\
61.62	0\\
61.63	0\\
61.64	0\\
61.65	0\\
61.66	0\\
61.67	0\\
61.68	0\\
61.69	0\\
61.7	0\\
61.71	0\\
61.72	0\\
61.73	0\\
61.74	0\\
61.75	0\\
61.76	0\\
61.77	0\\
61.78	0\\
61.79	0\\
61.8	0\\
61.81	0\\
61.82	0\\
61.83	0\\
61.84	0\\
61.85	0\\
61.86	0\\
61.87	0\\
61.88	0\\
61.89	0\\
61.9	0\\
61.91	0\\
61.92	0\\
61.93	0\\
61.94	0\\
61.95	0\\
61.96	0\\
61.97	0\\
61.98	0\\
61.99	0\\
62	0\\
62.01	0\\
62.02	0\\
62.03	0\\
62.04	0\\
62.05	0\\
62.06	0\\
62.07	0\\
62.08	0\\
62.09	0\\
62.1	0\\
62.11	0\\
62.12	0\\
62.13	0\\
62.14	0\\
62.15	0\\
62.16	0\\
62.17	0\\
62.18	0\\
62.19	0\\
62.2	0\\
62.21	0\\
62.22	0\\
62.23	0\\
62.24	0\\
62.25	0\\
62.26	0\\
62.27	0\\
62.28	0\\
62.29	0\\
62.3	0\\
62.31	0\\
62.32	0\\
62.33	0\\
62.34	0\\
62.35	0\\
62.36	0\\
62.37	0\\
62.38	0\\
62.39	0\\
62.4	0\\
62.41	0\\
62.42	0\\
62.43	0\\
62.44	0\\
62.45	0\\
62.46	0\\
62.47	0\\
62.48	0\\
62.49	0\\
62.5	0\\
62.51	0\\
62.52	0\\
62.53	0\\
62.54	0\\
62.55	0\\
62.56	0\\
62.57	0\\
62.58	0\\
62.59	0\\
62.6	0\\
62.61	0\\
62.62	0\\
62.63	0\\
62.64	0\\
62.65	0\\
62.66	0\\
62.67	0\\
62.68	0\\
62.69	0\\
62.7	0\\
62.71	0\\
62.72	0\\
62.73	0\\
62.74	0\\
62.75	0\\
62.76	0\\
62.77	0\\
62.78	0\\
62.79	0\\
62.8	0\\
62.81	0\\
62.82	0\\
62.83	0\\
62.84	0\\
62.85	0\\
62.86	0\\
62.87	0\\
62.88	0\\
62.89	0\\
62.9	0\\
62.91	0\\
62.92	0\\
62.93	0\\
62.94	0\\
62.95	0\\
62.96	0\\
62.97	0\\
62.98	0\\
62.99	0\\
63	0\\
63.01	0\\
63.02	0\\
63.03	0\\
63.04	0\\
63.05	0\\
63.06	0\\
63.07	0\\
63.08	0\\
63.09	0\\
63.1	0\\
63.11	0\\
63.12	0\\
63.13	0\\
63.14	0\\
63.15	0\\
63.16	0\\
63.17	0\\
63.18	0\\
63.19	0\\
63.2	0\\
63.21	0\\
63.22	0\\
63.23	0\\
63.24	0\\
63.25	0\\
63.26	0\\
63.27	0\\
63.28	0\\
63.29	0\\
63.3	0\\
63.31	0\\
63.32	0\\
63.33	0\\
63.34	0\\
63.35	0\\
63.36	0\\
63.37	0\\
63.38	0\\
63.39	0\\
63.4	0\\
63.41	0\\
63.42	0\\
63.43	0\\
63.44	0\\
63.45	0\\
63.46	0\\
63.47	0\\
63.48	0\\
63.49	0\\
63.5	0\\
63.51	0\\
63.52	0\\
63.53	0\\
63.54	0\\
63.55	0\\
63.56	0\\
63.57	0\\
63.58	0\\
63.59	0\\
63.6	0\\
63.61	0\\
63.62	0\\
63.63	0\\
63.64	0\\
63.65	0\\
63.66	0\\
63.67	0\\
63.68	0\\
63.69	0\\
63.7	0\\
63.71	0\\
63.72	0\\
63.73	0\\
63.74	0\\
63.75	0\\
63.76	0\\
63.77	0\\
63.78	0\\
63.79	0\\
63.8	0\\
63.81	0\\
63.82	0\\
63.83	0\\
63.84	0\\
63.85	0\\
63.86	0\\
63.87	0\\
63.88	0\\
63.89	0\\
63.9	0\\
63.91	0\\
63.92	0\\
63.93	0\\
63.94	0\\
63.95	0\\
63.96	0\\
63.97	0\\
63.98	0\\
63.99	0\\
64	0\\
64.01	0\\
64.02	0\\
64.03	0\\
64.04	0\\
64.05	0\\
64.06	0\\
64.07	0\\
64.08	0\\
64.09	0\\
64.1	0\\
64.11	0\\
64.12	0\\
64.13	0\\
64.14	0\\
64.15	0\\
64.16	0\\
64.17	0\\
64.18	0\\
64.19	0\\
64.2	0\\
64.21	0\\
64.22	0\\
64.23	0\\
64.24	0\\
64.25	0\\
64.26	0\\
64.27	0\\
64.28	0\\
64.29	0\\
64.3	0\\
64.31	0\\
64.32	0\\
64.33	0\\
64.34	0\\
64.35	0\\
64.36	0\\
64.37	0\\
64.38	0\\
64.39	0\\
64.4	0\\
64.41	0\\
64.42	0\\
64.43	0\\
64.44	0\\
64.45	0\\
64.46	0\\
64.47	0\\
64.48	0\\
64.49	0\\
64.5	0\\
64.51	0\\
64.52	0\\
64.53	0\\
64.54	0\\
64.55	0\\
64.56	0\\
64.57	0\\
64.58	0\\
64.59	0\\
64.6	0\\
64.61	0\\
64.62	0\\
64.63	0\\
64.64	0\\
64.65	0\\
64.66	0\\
64.67	0\\
64.68	0\\
64.69	0\\
64.7	0\\
64.71	0\\
64.72	0\\
64.73	0\\
64.74	0\\
64.75	0\\
64.76	0\\
64.77	0\\
64.78	0\\
64.79	0\\
64.8	0\\
64.81	0\\
64.82	0\\
64.83	0\\
64.84	0\\
64.85	0\\
64.86	0\\
64.87	0\\
64.88	0\\
64.89	0\\
64.9	0\\
64.91	0\\
64.92	0\\
64.93	0\\
64.94	0\\
64.95	0\\
64.96	0\\
64.97	0\\
64.98	0\\
64.99	0\\
65	0\\
65.01	0\\
65.02	0\\
65.03	0\\
65.04	0\\
65.05	0\\
65.06	0\\
65.07	0\\
65.08	0\\
65.09	0\\
65.1	0\\
65.11	0\\
65.12	0\\
65.13	0\\
65.14	0\\
65.15	0\\
65.16	0\\
65.17	0\\
65.18	0\\
65.19	0\\
65.2	0\\
65.21	0\\
65.22	0\\
65.23	0\\
65.24	0\\
65.25	0\\
65.26	0\\
65.27	0\\
65.28	0\\
65.29	0\\
65.3	0\\
65.31	0\\
65.32	0\\
65.33	0\\
65.34	0\\
65.35	0\\
65.36	0\\
65.37	0\\
65.38	0\\
65.39	0\\
65.4	0\\
65.41	0\\
65.42	0\\
65.43	0\\
65.44	0\\
65.45	0\\
65.46	0\\
65.47	0\\
65.48	0\\
65.49	0\\
65.5	0\\
65.51	0\\
65.52	0\\
65.53	0\\
65.54	0\\
65.55	0\\
65.56	0\\
65.57	0\\
65.58	0\\
65.59	0\\
65.6	0\\
65.61	0\\
65.62	0\\
65.63	0\\
65.64	0\\
65.65	0\\
65.66	0\\
65.67	0\\
65.68	0\\
65.69	0\\
65.7	0\\
65.71	0\\
65.72	0\\
65.73	0\\
65.74	0\\
65.75	0\\
65.76	0\\
65.77	0\\
65.78	0\\
65.79	0\\
65.8	0\\
65.81	0\\
65.82	0\\
65.83	0\\
65.84	0\\
65.85	0\\
65.86	0\\
65.87	0\\
65.88	0\\
65.89	0\\
65.9	0\\
65.91	0\\
65.92	0\\
65.93	0\\
65.94	0\\
65.95	0\\
65.96	0\\
65.97	0\\
65.98	0\\
65.99	0\\
66	0\\
66.01	0\\
66.02	0\\
66.03	0\\
66.04	0\\
66.05	0\\
66.06	0\\
66.07	0\\
66.08	0\\
66.09	0\\
66.1	0\\
66.11	0\\
66.12	0\\
66.13	0\\
66.14	0\\
66.15	0\\
66.16	0\\
66.17	0\\
66.18	0\\
66.19	0\\
66.2	0\\
66.21	0\\
66.22	0\\
66.23	0\\
66.24	0\\
66.25	0\\
66.26	0\\
66.27	0\\
66.28	0\\
66.29	0\\
66.3	0\\
66.31	0\\
66.32	0\\
66.33	0\\
66.34	0\\
66.35	0\\
66.36	0\\
66.37	0\\
66.38	0\\
66.39	0\\
66.4	0\\
66.41	0\\
66.42	0\\
66.43	0\\
66.44	0\\
66.45	0\\
66.46	0\\
66.47	0\\
66.48	0\\
66.49	0\\
66.5	0\\
66.51	0\\
66.52	0\\
66.53	0\\
66.54	0\\
66.55	0\\
66.56	0\\
66.57	0\\
66.58	0\\
66.59	0\\
66.6	0\\
66.61	0\\
66.62	0\\
66.63	0\\
66.64	0\\
66.65	0\\
66.66	0\\
66.67	0\\
66.68	0\\
66.69	0\\
66.7	0\\
66.71	0\\
66.72	0\\
66.73	0\\
66.74	0\\
66.75	0\\
66.76	0\\
66.77	0\\
66.78	0\\
66.79	0\\
66.8	0\\
66.81	0\\
66.82	0\\
66.83	0\\
66.84	0\\
66.85	0\\
66.86	0\\
66.87	0\\
66.88	0\\
66.89	0\\
66.9	0\\
66.91	0\\
66.92	0\\
66.93	0\\
66.94	0\\
66.95	0\\
66.96	0\\
66.97	0\\
66.98	0\\
66.99	0\\
67	0\\
67.01	0\\
67.02	0\\
67.03	0\\
67.04	0\\
67.05	0\\
67.06	0\\
67.07	0\\
67.08	0\\
67.09	0\\
67.1	0\\
67.11	0\\
67.12	0\\
67.13	0\\
67.14	0\\
67.15	0\\
67.16	0\\
67.17	0\\
67.18	0\\
67.19	0\\
67.2	0\\
67.21	0\\
67.22	0\\
67.23	0\\
67.24	0\\
67.25	0\\
67.26	0\\
67.27	0\\
67.28	0\\
67.29	0\\
67.3	0\\
67.31	0\\
67.32	0\\
67.33	0\\
67.34	0\\
67.35	0\\
67.36	0\\
67.37	0\\
67.38	0\\
67.39	0\\
67.4	0\\
67.41	0\\
67.42	0\\
67.43	0\\
67.44	0\\
67.45	0\\
67.46	0\\
67.47	0\\
67.48	0\\
67.49	0\\
67.5	0\\
67.51	0\\
67.52	0\\
67.53	0\\
67.54	0\\
67.55	0\\
67.56	0\\
67.57	0\\
67.58	0\\
67.59	0\\
67.6	0\\
67.61	0\\
67.62	0\\
67.63	0\\
67.64	0\\
67.65	0\\
67.66	0\\
67.67	0\\
67.68	0\\
67.69	0\\
67.7	0\\
67.71	0\\
67.72	0\\
67.73	0\\
67.74	0\\
67.75	0\\
67.76	0\\
67.77	0\\
67.78	0\\
67.79	0\\
67.8	0\\
67.81	0\\
67.82	0\\
67.83	0\\
67.84	0\\
67.85	0\\
67.86	0\\
67.87	0\\
67.88	0\\
67.89	0\\
67.9	0\\
67.91	0\\
67.92	0\\
67.93	0\\
67.94	0\\
67.95	0\\
67.96	0\\
67.97	0\\
67.98	0\\
67.99	0\\
68	0\\
68.01	0\\
68.02	0\\
68.03	0\\
68.04	0\\
68.05	0\\
68.06	0\\
68.07	0\\
68.08	0\\
68.09	0\\
68.1	0\\
68.11	0\\
68.12	0\\
68.13	0\\
68.14	0\\
68.15	0\\
68.16	0\\
68.17	0\\
68.18	0\\
68.19	0\\
68.2	0\\
68.21	0\\
68.22	0\\
68.23	0\\
68.24	0\\
68.25	0\\
68.26	0\\
68.27	0\\
68.28	0\\
68.29	0\\
68.3	0\\
68.31	0\\
68.32	0\\
68.33	0\\
68.34	0\\
68.35	0\\
68.36	0\\
68.37	0\\
68.38	0\\
68.39	0\\
68.4	0\\
68.41	0\\
68.42	0\\
68.43	0\\
68.44	0\\
68.45	0\\
68.46	0\\
68.47	0\\
68.48	0\\
68.49	0\\
68.5	0\\
68.51	0\\
68.52	0\\
68.53	0\\
68.54	0\\
68.55	0\\
68.56	0\\
68.57	0\\
68.58	0\\
68.59	0\\
68.6	0\\
68.61	0\\
68.62	0\\
68.63	0\\
68.64	0\\
68.65	0\\
68.66	0\\
68.67	0\\
68.68	0\\
68.69	0\\
68.7	0\\
68.71	0\\
68.72	0\\
68.73	0\\
68.74	0\\
68.75	0\\
68.76	0\\
68.77	0\\
68.78	0\\
68.79	0\\
68.8	0\\
68.81	0\\
68.82	0\\
68.83	0\\
68.84	0\\
68.85	0\\
68.86	0\\
68.87	0\\
68.88	0\\
68.89	0\\
68.9	0\\
68.91	0\\
68.92	0\\
68.93	0\\
68.94	0\\
68.95	0\\
68.96	0\\
68.97	0\\
68.98	0\\
68.99	0\\
69	0\\
69.01	0\\
69.02	0\\
69.03	0\\
69.04	0\\
69.05	0\\
69.06	0\\
69.07	0\\
69.08	0\\
69.09	0\\
69.1	0\\
69.11	0\\
69.12	0\\
69.13	0\\
69.14	0\\
69.15	0\\
69.16	0\\
69.17	0\\
69.18	0\\
69.19	0\\
69.2	0\\
69.21	0\\
69.22	0\\
69.23	0\\
69.24	0\\
69.25	0\\
69.26	0\\
69.27	0\\
69.28	0\\
69.29	0\\
69.3	0\\
69.31	0\\
69.32	0\\
69.33	0\\
69.34	0\\
69.35	0\\
69.36	0\\
69.37	0\\
69.38	0\\
69.39	0\\
69.4	0\\
69.41	0\\
69.42	0\\
69.43	0\\
69.44	0\\
69.45	0\\
69.46	0\\
69.47	0\\
69.48	0\\
69.49	0\\
69.5	0\\
69.51	0\\
69.52	0\\
69.53	0\\
69.54	0\\
69.55	0\\
69.56	0\\
69.57	0\\
69.58	0\\
69.59	0\\
69.6	0\\
69.61	0\\
69.62	0\\
69.63	0\\
69.64	0\\
69.65	0\\
69.66	0\\
69.67	0\\
69.68	0\\
69.69	0\\
69.7	0\\
69.71	0\\
69.72	0\\
69.73	0\\
69.74	0\\
69.75	0\\
69.76	0\\
69.77	0\\
69.78	0\\
69.79	0\\
69.8	0\\
69.81	0\\
69.82	0\\
69.83	0\\
69.84	0\\
69.85	0\\
69.86	0\\
69.87	0\\
69.88	0\\
69.89	0\\
69.9	0\\
69.91	0\\
69.92	0\\
69.93	0\\
69.94	0\\
69.95	0\\
69.96	0\\
69.97	0\\
69.98	0\\
69.99	0\\
70	0\\
70.01	0\\
70.02	0\\
70.03	0\\
70.04	0\\
70.05	0\\
70.06	0\\
70.07	0\\
70.08	0\\
70.09	0\\
70.1	0\\
70.11	0\\
70.12	0\\
70.13	0\\
70.14	0\\
70.15	0\\
70.16	0\\
70.17	0\\
70.18	0\\
70.19	0\\
70.2	0\\
70.21	0\\
70.22	0\\
70.23	0\\
70.24	0\\
70.25	0\\
70.26	0\\
70.27	0\\
70.28	0\\
70.29	0\\
70.3	0\\
70.31	0\\
70.32	0\\
70.33	0\\
70.34	0\\
70.35	0\\
70.36	0\\
70.37	0\\
70.38	0\\
70.39	0\\
70.4	0\\
70.41	0\\
70.42	0\\
70.43	0\\
70.44	0\\
70.45	0\\
70.46	0\\
70.47	0\\
70.48	0\\
70.49	0\\
70.5	0\\
70.51	0\\
70.52	0\\
70.53	0\\
70.54	0\\
70.55	0\\
70.56	0\\
70.57	0\\
70.58	0\\
70.59	0\\
70.6	0\\
70.61	0\\
70.62	0\\
70.63	0\\
70.64	0\\
70.65	0\\
70.66	0\\
70.67	0\\
70.68	0\\
70.69	0\\
70.7	0\\
70.71	0\\
70.72	0\\
70.73	0\\
70.74	0\\
70.75	0\\
70.76	0\\
70.77	0\\
70.78	0\\
70.79	0\\
70.8	0\\
70.81	0\\
70.82	0\\
70.83	0\\
70.84	0\\
70.85	0\\
70.86	0\\
70.87	0\\
70.88	0\\
70.89	0\\
70.9	0\\
70.91	0\\
70.92	0\\
70.93	0\\
70.94	0\\
70.95	0\\
70.96	0\\
70.97	0\\
70.98	0\\
70.99	0\\
71	0\\
71.01	0\\
71.02	0\\
71.03	0\\
71.04	0\\
71.05	0\\
71.06	0\\
71.07	0\\
71.08	0\\
71.09	0\\
71.1	0\\
71.11	0\\
71.12	0\\
71.13	0\\
71.14	0\\
71.15	0\\
71.16	0\\
71.17	0\\
71.18	0\\
71.19	0\\
71.2	0\\
71.21	0\\
71.22	0\\
71.23	0\\
71.24	0\\
71.25	0\\
71.26	0\\
71.27	0\\
71.28	0\\
71.29	0\\
71.3	0\\
71.31	0\\
71.32	0\\
71.33	0\\
71.34	0\\
71.35	0\\
71.36	0\\
71.37	0\\
71.38	0\\
71.39	0\\
71.4	0\\
71.41	0\\
71.42	0\\
71.43	0\\
71.44	0\\
71.45	0\\
71.46	0\\
71.47	0\\
71.48	0\\
71.49	0\\
71.5	0\\
71.51	0\\
71.52	0\\
71.53	0\\
71.54	0\\
71.55	0\\
71.56	0\\
71.57	0\\
71.58	0\\
71.59	0\\
71.6	0\\
71.61	0\\
71.62	0\\
71.63	0\\
71.64	0\\
71.65	0\\
71.66	0\\
71.67	0\\
71.68	0\\
71.69	0\\
71.7	0\\
71.71	0\\
71.72	0\\
71.73	0\\
71.74	0\\
71.75	0\\
71.76	0\\
71.77	0\\
71.78	0\\
71.79	0\\
71.8	0\\
71.81	0\\
71.82	0\\
71.83	0\\
71.84	0\\
71.85	0\\
71.86	0\\
71.87	0\\
71.88	0\\
71.89	0\\
71.9	0\\
71.91	0\\
71.92	0\\
71.93	0\\
71.94	0\\
71.95	0\\
71.96	0\\
71.97	0\\
71.98	0\\
71.99	0\\
72	0\\
72.01	0\\
72.02	0\\
72.03	0\\
72.04	0\\
72.05	0\\
72.06	0\\
72.07	0\\
72.08	0\\
72.09	0\\
72.1	0\\
72.11	0\\
72.12	0\\
72.13	0\\
72.14	0\\
72.15	0\\
72.16	0\\
72.17	0\\
72.18	0\\
72.19	0\\
72.2	0\\
72.21	0\\
72.22	0\\
72.23	0\\
72.24	0\\
72.25	0\\
72.26	0\\
72.27	0\\
72.28	0\\
72.29	0\\
72.3	0\\
72.31	0\\
72.32	0\\
72.33	0\\
72.34	0\\
72.35	0\\
72.36	0\\
72.37	0\\
72.38	0\\
72.39	0\\
72.4	0\\
72.41	0\\
72.42	0\\
72.43	0\\
72.44	0\\
72.45	0\\
72.46	0\\
72.47	0\\
72.48	0\\
72.49	0\\
72.5	0\\
72.51	0\\
72.52	0\\
72.53	0\\
72.54	0\\
72.55	0\\
72.56	0\\
72.57	0\\
72.58	0\\
72.59	0\\
72.6	0\\
72.61	0\\
72.62	0\\
72.63	0\\
72.64	0\\
72.65	0\\
72.66	0\\
72.67	0\\
72.68	0\\
72.69	0\\
72.7	0\\
72.71	0\\
72.72	0\\
72.73	0\\
72.74	0\\
72.75	0\\
72.76	0\\
72.77	0\\
72.78	0\\
72.79	0\\
72.8	0\\
72.81	0\\
72.82	0\\
72.83	0\\
72.84	0\\
72.85	0\\
72.86	0\\
72.87	0\\
72.88	0\\
72.89	0\\
72.9	0\\
72.91	0\\
72.92	0\\
72.93	0\\
72.94	0\\
72.95	0\\
72.96	0\\
72.97	0\\
72.98	0\\
72.99	0\\
73	0\\
73.01	0\\
73.02	0\\
73.03	0\\
73.04	0\\
73.05	0\\
73.06	0\\
73.07	0\\
73.08	0\\
73.09	0\\
73.1	0\\
73.11	0\\
73.12	0\\
73.13	0\\
73.14	0\\
73.15	0\\
73.16	0\\
73.17	0\\
73.18	0\\
73.19	0\\
73.2	0\\
73.21	0\\
73.22	0\\
73.23	0\\
73.24	0\\
73.25	0\\
73.26	0\\
73.27	0\\
73.28	0\\
73.29	0\\
73.3	0\\
73.31	0\\
73.32	0\\
73.33	0\\
73.34	0\\
73.35	0\\
73.36	0\\
73.37	0\\
73.38	0\\
73.39	0\\
73.4	0\\
73.41	0\\
73.42	0\\
73.43	0\\
73.44	0\\
73.45	0\\
73.46	0\\
73.47	0\\
73.48	0\\
73.49	0\\
73.5	0\\
73.51	0\\
73.52	0\\
73.53	0\\
73.54	0\\
73.55	0\\
73.56	0\\
73.57	0\\
73.58	0\\
73.59	0\\
73.6	0\\
73.61	0\\
73.62	0\\
73.63	0\\
73.64	0\\
73.65	0\\
73.66	0\\
73.67	0\\
73.68	0\\
73.69	0\\
73.7	0\\
73.71	0\\
73.72	0\\
73.73	0\\
73.74	0\\
73.75	0\\
73.76	0\\
73.77	0\\
73.78	0\\
73.79	0\\
73.8	0\\
73.81	0\\
73.82	0\\
73.83	0\\
73.84	0\\
73.85	0\\
73.86	0\\
73.87	0\\
73.88	0\\
73.89	0\\
73.9	0\\
73.91	0\\
73.92	0\\
73.93	0\\
73.94	0\\
73.95	0\\
73.96	0\\
73.97	0\\
73.98	0\\
73.99	0\\
74	0\\
74.01	0\\
74.02	0\\
74.03	0\\
74.04	0\\
74.05	0\\
74.06	0\\
74.07	0\\
74.08	0\\
74.09	0\\
74.1	0\\
74.11	0\\
74.12	0\\
74.13	0\\
74.14	0\\
74.15	0\\
74.16	0\\
74.17	0\\
74.18	0\\
74.19	0\\
74.2	0\\
74.21	0\\
74.22	0\\
74.23	0\\
74.24	0\\
74.25	0\\
74.26	0\\
74.27	0\\
74.28	0\\
74.29	0\\
74.3	0\\
74.31	0\\
74.32	0\\
74.33	0\\
74.34	0\\
74.35	0\\
74.36	0\\
74.37	0\\
74.38	0\\
74.39	0\\
74.4	0\\
74.41	0\\
74.42	0\\
74.43	0\\
74.44	0\\
74.45	0\\
74.46	0\\
74.47	0\\
74.48	0\\
74.49	0\\
74.5	0\\
74.51	0\\
74.52	0\\
74.53	0\\
74.54	0\\
74.55	0\\
74.56	0\\
74.57	0\\
74.58	0\\
74.59	0\\
74.6	0\\
74.61	0\\
74.62	0\\
74.63	0\\
74.64	0\\
74.65	0\\
74.66	0\\
74.67	0\\
74.68	0\\
74.69	0\\
74.7	0\\
74.71	0\\
74.72	0\\
74.73	0\\
74.74	0\\
74.75	0\\
74.76	0\\
74.77	0\\
74.78	0\\
74.79	0\\
74.8	0\\
74.81	0\\
74.82	0\\
74.83	0\\
74.84	0\\
74.85	0\\
74.86	0\\
74.87	0\\
74.88	0\\
74.89	0\\
74.9	0\\
74.91	0\\
74.92	0\\
74.93	0\\
74.94	0\\
74.95	0\\
74.96	0\\
74.97	0\\
74.98	0\\
74.99	0\\
75	0\\
75.01	0\\
75.02	0\\
75.03	0\\
75.04	0\\
75.05	0\\
75.06	0\\
75.07	0\\
75.08	0\\
75.09	0\\
75.1	0\\
75.11	0\\
75.12	0\\
75.13	0\\
75.14	0\\
75.15	0\\
75.16	0\\
75.17	0\\
75.18	0\\
75.19	0\\
75.2	0\\
75.21	0\\
75.22	0\\
75.23	0\\
75.24	0\\
75.25	0\\
75.26	0\\
75.27	0\\
75.28	0\\
75.29	0\\
75.3	0\\
75.31	0\\
75.32	0\\
75.33	0\\
75.34	0\\
75.35	0\\
75.36	0\\
75.37	0\\
75.38	0\\
75.39	0\\
75.4	0\\
75.41	0\\
75.42	0\\
75.43	0\\
75.44	0\\
75.45	0\\
75.46	0\\
75.47	0\\
75.48	0\\
75.49	0\\
75.5	0\\
75.51	0\\
75.52	0\\
75.53	0\\
75.54	0\\
75.55	0\\
75.56	0\\
75.57	0\\
75.58	0\\
75.59	0\\
75.6	0\\
75.61	0\\
75.62	0\\
75.63	0\\
75.64	0\\
75.65	0\\
75.66	0\\
75.67	0\\
75.68	0\\
75.69	0\\
75.7	0\\
75.71	0\\
75.72	0\\
75.73	0\\
75.74	0\\
75.75	0\\
75.76	0\\
75.77	0\\
75.78	0\\
75.79	0\\
75.8	0\\
75.81	0\\
75.82	0\\
75.83	0\\
75.84	0\\
75.85	0\\
75.86	0\\
75.87	0\\
75.88	0\\
75.89	0\\
75.9	0\\
75.91	0\\
75.92	0\\
75.93	0\\
75.94	0\\
75.95	0\\
75.96	0\\
75.97	0\\
75.98	0\\
75.99	0\\
76	0\\
76.01	0\\
76.02	0\\
76.03	0\\
76.04	0\\
76.05	0\\
76.06	0\\
76.07	0\\
76.08	0\\
76.09	0\\
76.1	0\\
76.11	0\\
76.12	0\\
76.13	0\\
76.14	0\\
76.15	0\\
76.16	0\\
76.17	0\\
76.18	0\\
76.19	0\\
76.2	0\\
76.21	0\\
76.22	0\\
76.23	0\\
76.24	0\\
76.25	0\\
76.26	0\\
76.27	0\\
76.28	0\\
76.29	0\\
76.3	0\\
76.31	0\\
76.32	0\\
76.33	0\\
76.34	0\\
76.35	0\\
76.36	0\\
76.37	0\\
76.38	0\\
76.39	0\\
76.4	0\\
76.41	0\\
76.42	0\\
76.43	0\\
76.44	0\\
76.45	0\\
76.46	0\\
76.47	0\\
76.48	0\\
76.49	0\\
76.5	0\\
76.51	0\\
76.52	0\\
76.53	0\\
76.54	0\\
76.55	0\\
76.56	0\\
76.57	0\\
76.58	0\\
76.59	0\\
76.6	0\\
76.61	0\\
76.62	0\\
76.63	0\\
76.64	0\\
76.65	0\\
76.66	0\\
76.67	0\\
76.68	0\\
76.69	0\\
76.7	0\\
76.71	0\\
76.72	0\\
76.73	0\\
76.74	0\\
76.75	0\\
76.76	0\\
76.77	0\\
76.78	0\\
76.79	0\\
76.8	0\\
76.81	0\\
76.82	0\\
76.83	0\\
76.84	0\\
76.85	0\\
76.86	0\\
76.87	0\\
76.88	0\\
76.89	0\\
76.9	0\\
76.91	0\\
76.92	0\\
76.93	0\\
76.94	0\\
76.95	0\\
76.96	0\\
76.97	0\\
76.98	0\\
76.99	0\\
77	0\\
77.01	0\\
77.02	0\\
77.03	0\\
77.04	0\\
77.05	0\\
77.06	0\\
77.07	0\\
77.08	0\\
77.09	0\\
77.1	0\\
77.11	0\\
77.12	0\\
77.13	0\\
77.14	0\\
77.15	0\\
77.16	0\\
77.17	0\\
77.18	0\\
77.19	0\\
77.2	0\\
77.21	0\\
77.22	0\\
77.23	0\\
77.24	0\\
77.25	0\\
77.26	0\\
77.27	0\\
77.28	0\\
77.29	0\\
77.3	0\\
77.31	0\\
77.32	0\\
77.33	0\\
77.34	0\\
77.35	0\\
77.36	0\\
77.37	0\\
77.38	0\\
77.39	0\\
77.4	0\\
77.41	0\\
77.42	0\\
77.43	0\\
77.44	0\\
77.45	0\\
77.46	0\\
77.47	0\\
77.48	0\\
77.49	0\\
77.5	0\\
77.51	0\\
77.52	0\\
77.53	0\\
77.54	0\\
77.55	0\\
77.56	0\\
77.57	0\\
77.58	0\\
77.59	0\\
77.6	0\\
77.61	0\\
77.62	0\\
77.63	0\\
77.64	0\\
77.65	0\\
77.66	0\\
77.67	0\\
77.68	0\\
77.69	0\\
77.7	0\\
77.71	0\\
77.72	0\\
77.73	0\\
77.74	0\\
77.75	0\\
77.76	0\\
77.77	0\\
77.78	0\\
77.79	0\\
77.8	0\\
77.81	0\\
77.82	0\\
77.83	0\\
77.84	0\\
77.85	0\\
77.86	0\\
77.87	0\\
77.88	0\\
77.89	0\\
77.9	0\\
77.91	0\\
77.92	0\\
77.93	0\\
77.94	0\\
77.95	0\\
77.96	0\\
77.97	0\\
77.98	0\\
77.99	0\\
78	0\\
78.01	0\\
78.02	0\\
78.03	0\\
78.04	0\\
78.05	0\\
78.06	0\\
78.07	0\\
78.08	0\\
78.09	0\\
78.1	0\\
78.11	0\\
78.12	0\\
78.13	0\\
78.14	0\\
78.15	0\\
78.16	0\\
78.17	0\\
78.18	0\\
78.19	0\\
78.2	0\\
78.21	0\\
78.22	0\\
78.23	0\\
78.24	0\\
78.25	0\\
78.26	0\\
78.27	0\\
78.28	0\\
78.29	0\\
78.3	0\\
78.31	0\\
78.32	0\\
78.33	0\\
78.34	0\\
78.35	0\\
78.36	0\\
78.37	0\\
78.38	0\\
78.39	0\\
78.4	0\\
78.41	0\\
78.42	0\\
78.43	0\\
78.44	0\\
78.45	0\\
78.46	0\\
78.47	0\\
78.48	0\\
78.49	0\\
78.5	0\\
78.51	0\\
78.52	0\\
78.53	0\\
78.54	0\\
78.55	0\\
78.56	0\\
78.57	0\\
78.58	0\\
78.59	0\\
78.6	0\\
78.61	0\\
78.62	0\\
78.63	0\\
78.64	0\\
78.65	0\\
78.66	0\\
78.67	0\\
78.68	0\\
78.69	0\\
78.7	0\\
78.71	0\\
78.72	0\\
78.73	0\\
78.74	0\\
78.75	0\\
78.76	0\\
78.77	0\\
78.78	0\\
78.79	0\\
78.8	0\\
78.81	0\\
78.82	0\\
78.83	0\\
78.84	0\\
78.85	0\\
78.86	0\\
78.87	0\\
78.88	0\\
78.89	0\\
78.9	0\\
78.91	0\\
78.92	0\\
78.93	0\\
78.94	0\\
78.95	0\\
78.96	0\\
78.97	0\\
78.98	0\\
78.99	0\\
79	0\\
79.01	0\\
79.02	0\\
79.03	0\\
79.04	0\\
79.05	0\\
79.06	0\\
79.07	0\\
79.08	0\\
79.09	0\\
79.1	0\\
79.11	0\\
79.12	0\\
79.13	0\\
79.14	0\\
79.15	0\\
79.16	0\\
79.17	0\\
79.18	0\\
79.19	0\\
79.2	0\\
79.21	0\\
79.22	0\\
79.23	0\\
79.24	0\\
79.25	0\\
79.26	0\\
79.27	0\\
79.28	0\\
79.29	0\\
79.3	0\\
79.31	0\\
79.32	0\\
79.33	0\\
79.34	0\\
79.35	0\\
79.36	0\\
79.37	0\\
79.38	0\\
79.39	0\\
79.4	0\\
79.41	0\\
79.42	0\\
79.43	0\\
79.44	0\\
79.45	0\\
79.46	0\\
79.47	0\\
79.48	0\\
79.49	0\\
79.5	0\\
79.51	0\\
79.52	0\\
79.53	0\\
79.54	0\\
79.55	0\\
79.56	0\\
79.57	0\\
79.58	0\\
79.59	0\\
79.6	0\\
79.61	0\\
79.62	0\\
79.63	0\\
79.64	0\\
79.65	0\\
79.66	0\\
79.67	0\\
79.68	0\\
79.69	0\\
79.7	0\\
79.71	0\\
79.72	0\\
79.73	0\\
79.74	0\\
79.75	0\\
79.76	0\\
79.77	0\\
79.78	0\\
79.79	0\\
79.8	0\\
79.81	0\\
79.82	0\\
79.83	0\\
79.84	0\\
79.85	0\\
79.86	0\\
79.87	0\\
79.88	0\\
79.89	0\\
79.9	0\\
79.91	0\\
79.92	0\\
79.93	0\\
79.94	0\\
79.95	0\\
79.96	0\\
79.97	0\\
79.98	0\\
79.99	0\\
80	0\\
80.01	0\\
};
\addplot [color=red,dashed]
  table[row sep=crcr]{%
80.01	0\\
80.02	0\\
80.03	0\\
80.04	0\\
80.05	0\\
80.06	0\\
80.07	0\\
80.08	0\\
80.09	0\\
80.1	0\\
80.11	0\\
80.12	0\\
80.13	0\\
80.14	0\\
80.15	0\\
80.16	0\\
80.17	0\\
80.18	0\\
80.19	0\\
80.2	0\\
80.21	0\\
80.22	0\\
80.23	0\\
80.24	0\\
80.25	0\\
80.26	0\\
80.27	0\\
80.28	0\\
80.29	0\\
80.3	0\\
80.31	0\\
80.32	0\\
80.33	0\\
80.34	0\\
80.35	0\\
80.36	0\\
80.37	0\\
80.38	0\\
80.39	0\\
80.4	0\\
80.41	0\\
80.42	0\\
80.43	0\\
80.44	0\\
80.45	0\\
80.46	0\\
80.47	0\\
80.48	0\\
80.49	0\\
80.5	0\\
80.51	0\\
80.52	0\\
80.53	0\\
80.54	0\\
80.55	0\\
80.56	0\\
80.57	0\\
80.58	0\\
80.59	0\\
80.6	0\\
80.61	0\\
80.62	0\\
80.63	0\\
80.64	0\\
80.65	0\\
80.66	0\\
80.67	0\\
80.68	0\\
80.69	0\\
80.7	0\\
80.71	0\\
80.72	0\\
80.73	0\\
80.74	0\\
80.75	0\\
80.76	0\\
80.77	0\\
80.78	0\\
80.79	0\\
80.8	0\\
80.81	0\\
80.82	0\\
80.83	0\\
80.84	0\\
80.85	0\\
80.86	0\\
80.87	0\\
80.88	0\\
80.89	0\\
80.9	0\\
80.91	0\\
80.92	0\\
80.93	0\\
80.94	0\\
80.95	0\\
80.96	0\\
80.97	0\\
80.98	0\\
80.99	0\\
81	0\\
81.01	0\\
81.02	0\\
81.03	0\\
81.04	0\\
81.05	0\\
81.06	0\\
81.07	0\\
81.08	0\\
81.09	0\\
81.1	0\\
81.11	0\\
81.12	0\\
81.13	0\\
81.14	0\\
81.15	0\\
81.16	0\\
81.17	0\\
81.18	0\\
81.19	0\\
81.2	0\\
81.21	0\\
81.22	0\\
81.23	0\\
81.24	0\\
81.25	0\\
81.26	0\\
81.27	0\\
81.28	0\\
81.29	0\\
81.3	0\\
81.31	0\\
81.32	0\\
81.33	0\\
81.34	0\\
81.35	0\\
81.36	0\\
81.37	0\\
81.38	0\\
81.39	0\\
81.4	0\\
81.41	0\\
81.42	0\\
81.43	0\\
81.44	0\\
81.45	0\\
81.46	0\\
81.47	0\\
81.48	0\\
81.49	0\\
81.5	0\\
81.51	0\\
81.52	0\\
81.53	0\\
81.54	0\\
81.55	0\\
81.56	0\\
81.57	0\\
81.58	0\\
81.59	0\\
81.6	0\\
81.61	0\\
81.62	0\\
81.63	0\\
81.64	0\\
81.65	0\\
81.66	0\\
81.67	0\\
81.68	0\\
81.69	0\\
81.7	0\\
81.71	0\\
81.72	0\\
81.73	0\\
81.74	0\\
81.75	0\\
81.76	0\\
81.77	0\\
81.78	0\\
81.79	0\\
81.8	0\\
81.81	0\\
81.82	0\\
81.83	0\\
81.84	0\\
81.85	0\\
81.86	0\\
81.87	0\\
81.88	0\\
81.89	0\\
81.9	0\\
81.91	0\\
81.92	0\\
81.93	0\\
81.94	0\\
81.95	0\\
81.96	0\\
81.97	0\\
81.98	0\\
81.99	0\\
82	0\\
82.01	0\\
82.02	0\\
82.03	0\\
82.04	0\\
82.05	0\\
82.06	0\\
82.07	0\\
82.08	0\\
82.09	0\\
82.1	0\\
82.11	0\\
82.12	0\\
82.13	0\\
82.14	0\\
82.15	0\\
82.16	0\\
82.17	0\\
82.18	0\\
82.19	0\\
82.2	0\\
82.21	0\\
82.22	0\\
82.23	0\\
82.24	0\\
82.25	0\\
82.26	0\\
82.27	0\\
82.28	0\\
82.29	0\\
82.3	0\\
82.31	0\\
82.32	0\\
82.33	0\\
82.34	0\\
82.35	0\\
82.36	0\\
82.37	0\\
82.38	0\\
82.39	0\\
82.4	0\\
82.41	0\\
82.42	0\\
82.43	0\\
82.44	0\\
82.45	0\\
82.46	0\\
82.47	0\\
82.48	0\\
82.49	0\\
82.5	0\\
82.51	0\\
82.52	0\\
82.53	0\\
82.54	0\\
82.55	0\\
82.56	0\\
82.57	0\\
82.58	0\\
82.59	0\\
82.6	0\\
82.61	0\\
82.62	0\\
82.63	0\\
82.64	0\\
82.65	0\\
82.66	0\\
82.67	0\\
82.68	0\\
82.69	0\\
82.7	0\\
82.71	0\\
82.72	0\\
82.73	0\\
82.74	0\\
82.75	0\\
82.76	0\\
82.77	0\\
82.78	0\\
82.79	0\\
82.8	0\\
82.81	0\\
82.82	0\\
82.83	0\\
82.84	0\\
82.85	0\\
82.86	0\\
82.87	0\\
82.88	0\\
82.89	0\\
82.9	0\\
82.91	0\\
82.92	0\\
82.93	0\\
82.94	0\\
82.95	0\\
82.96	0\\
82.97	0\\
82.98	0\\
82.99	0\\
83	0\\
83.01	0\\
83.02	0\\
83.03	0\\
83.04	0\\
83.05	0\\
83.06	0\\
83.07	0\\
83.08	0\\
83.09	0\\
83.1	0\\
83.11	0\\
83.12	0\\
83.13	0\\
83.14	0\\
83.15	0\\
83.16	0\\
83.17	0\\
83.18	0\\
83.19	0\\
83.2	0\\
83.21	0\\
83.22	0\\
83.23	0\\
83.24	0\\
83.25	0\\
83.26	0\\
83.27	0\\
83.28	0\\
83.29	0\\
83.3	0\\
83.31	0\\
83.32	0\\
83.33	0\\
83.34	0\\
83.35	0\\
83.36	0\\
83.37	0\\
83.38	0\\
83.39	0\\
83.4	0\\
83.41	0\\
83.42	0\\
83.43	0\\
83.44	0\\
83.45	0\\
83.46	0\\
83.47	0\\
83.48	0\\
83.49	0\\
83.5	0\\
83.51	0\\
83.52	0\\
83.53	0\\
83.54	0\\
83.55	0\\
83.56	0\\
83.57	0\\
83.58	0\\
83.59	0\\
83.6	0\\
83.61	0\\
83.62	0\\
83.63	0\\
83.64	0\\
83.65	0\\
83.66	0\\
83.67	0\\
83.68	0\\
83.69	0\\
83.7	0\\
83.71	0\\
83.72	0\\
83.73	0\\
83.74	0\\
83.75	0\\
83.76	0\\
83.77	0\\
83.78	0\\
83.79	0\\
83.8	0\\
83.81	0\\
83.82	0\\
83.83	0\\
83.84	0\\
83.85	0\\
83.86	0\\
83.87	0\\
83.88	0\\
83.89	0\\
83.9	0\\
83.91	0\\
83.92	0\\
83.93	0\\
83.94	0\\
83.95	0\\
83.96	0\\
83.97	0\\
83.98	0\\
83.99	0\\
84	0\\
84.01	0\\
84.02	0\\
84.03	0\\
84.04	0\\
84.05	0\\
84.06	0\\
84.07	0\\
84.08	0\\
84.09	0\\
84.1	0\\
84.11	0\\
84.12	0\\
84.13	0\\
84.14	0\\
84.15	0\\
84.16	0\\
84.17	0\\
84.18	0\\
84.19	0\\
84.2	0\\
84.21	0\\
84.22	0\\
84.23	0\\
84.24	0\\
84.25	0\\
84.26	0\\
84.27	0\\
84.28	0\\
84.29	0\\
84.3	0\\
84.31	0\\
84.32	0\\
84.33	0\\
84.34	0\\
84.35	0\\
84.36	0\\
84.37	0\\
84.38	0\\
84.39	0\\
84.4	0\\
84.41	0\\
84.42	0\\
84.43	0\\
84.44	0\\
84.45	0\\
84.46	0\\
84.47	0\\
84.48	0\\
84.49	0\\
84.5	0\\
84.51	0\\
84.52	0\\
84.53	0\\
84.54	0\\
84.55	0\\
84.56	0\\
84.57	0\\
84.58	0\\
84.59	0\\
84.6	0\\
84.61	0\\
84.62	0\\
84.63	0\\
84.64	0\\
84.65	0\\
84.66	0\\
84.67	0\\
84.68	0\\
84.69	0\\
84.7	0\\
84.71	0\\
84.72	0\\
84.73	0\\
84.74	0\\
84.75	0\\
84.76	0\\
84.77	0\\
84.78	0\\
84.79	0\\
84.8	0\\
84.81	0\\
84.82	0\\
84.83	0\\
84.84	0\\
84.85	0\\
84.86	0\\
84.87	0\\
84.88	0\\
84.89	0\\
84.9	0\\
84.91	0\\
84.92	0\\
84.93	0\\
84.94	0\\
84.95	0\\
84.96	0\\
84.97	0\\
84.98	0\\
84.99	0\\
85	0\\
85.01	0\\
85.02	0\\
85.03	0\\
85.04	0\\
85.05	0\\
85.06	0\\
85.07	0\\
85.08	0\\
85.09	0\\
85.1	0\\
85.11	0\\
85.12	0\\
85.13	0\\
85.14	0\\
85.15	0\\
85.16	0\\
85.17	0\\
85.18	0\\
85.19	0\\
85.2	0\\
85.21	0\\
85.22	0\\
85.23	0\\
85.24	0\\
85.25	0\\
85.26	0\\
85.27	0\\
85.28	0\\
85.29	0\\
85.3	0\\
85.31	0\\
85.32	0\\
85.33	0\\
85.34	0\\
85.35	0\\
85.36	0\\
85.37	0\\
85.38	0\\
85.39	0\\
85.4	0\\
85.41	0\\
85.42	0\\
85.43	0\\
85.44	0\\
85.45	0\\
85.46	0\\
85.47	0\\
85.48	0\\
85.49	0\\
85.5	0\\
85.51	0\\
85.52	0\\
85.53	0\\
85.54	0\\
85.55	0\\
85.56	0\\
85.57	0\\
85.58	0\\
85.59	0\\
85.6	0\\
85.61	0\\
85.62	0\\
85.63	0\\
85.64	0\\
85.65	0\\
85.66	0\\
85.67	0\\
85.68	0\\
85.69	0\\
85.7	0\\
85.71	0\\
85.72	0\\
85.73	0\\
85.74	0\\
85.75	0\\
85.76	0\\
85.77	0\\
85.78	0\\
85.79	0\\
85.8	0\\
85.81	0\\
85.82	0\\
85.83	0\\
85.84	0\\
85.85	0\\
85.86	0\\
85.87	0\\
85.88	0\\
85.89	0\\
85.9	0\\
85.91	0\\
85.92	0\\
85.93	0\\
85.94	0\\
85.95	0\\
85.96	0\\
85.97	0\\
85.98	0\\
85.99	0\\
86	0\\
86.01	0\\
86.02	0\\
86.03	0\\
86.04	0\\
86.05	0\\
86.06	0\\
86.07	0\\
86.08	0\\
86.09	0\\
86.1	0\\
86.11	0\\
86.12	0\\
86.13	0\\
86.14	0\\
86.15	0\\
86.16	0\\
86.17	0\\
86.18	0\\
86.19	0\\
86.2	0\\
86.21	0\\
86.22	0\\
86.23	0\\
86.24	0\\
86.25	0\\
86.26	0\\
86.27	0\\
86.28	0\\
86.29	0\\
86.3	0\\
86.31	0\\
86.32	0\\
86.33	0\\
86.34	0\\
86.35	0\\
86.36	0\\
86.37	0\\
86.38	0\\
86.39	0\\
86.4	0\\
86.41	0\\
86.42	0\\
86.43	0\\
86.44	0\\
86.45	0\\
86.46	0\\
86.47	0\\
86.48	0\\
86.49	0\\
86.5	0\\
86.51	0\\
86.52	0\\
86.53	0\\
86.54	0\\
86.55	0\\
86.56	0\\
86.57	0\\
86.58	0\\
86.59	0\\
86.6	0\\
86.61	0\\
86.62	0\\
86.63	0\\
86.64	0\\
86.65	0\\
86.66	0\\
86.67	0\\
86.68	0\\
86.69	0\\
86.7	0\\
86.71	0\\
86.72	0\\
86.73	0\\
86.74	0\\
86.75	0\\
86.76	0\\
86.77	0\\
86.78	0\\
86.79	0\\
86.8	0\\
86.81	0\\
86.82	0\\
86.83	0\\
86.84	0\\
86.85	0\\
86.86	0\\
86.87	0\\
86.88	0\\
86.89	0\\
86.9	0\\
86.91	0\\
86.92	0\\
86.93	0\\
86.94	0\\
86.95	0\\
86.96	0\\
86.97	0\\
86.98	0\\
86.99	0\\
87	0\\
87.01	0\\
87.02	0\\
87.03	0\\
87.04	0\\
87.05	0\\
87.06	0\\
87.07	0\\
87.08	0\\
87.09	0\\
87.1	0\\
87.11	0\\
87.12	0\\
87.13	0\\
87.14	0\\
87.15	0\\
87.16	0\\
87.17	0\\
87.18	0\\
87.19	0\\
87.2	0\\
87.21	0\\
87.22	0\\
87.23	0\\
87.24	0\\
87.25	0\\
87.26	0\\
87.27	0\\
87.28	0\\
87.29	0\\
87.3	0\\
87.31	0\\
87.32	0\\
87.33	0\\
87.34	0\\
87.35	0\\
87.36	0\\
87.37	0\\
87.38	0\\
87.39	0\\
87.4	0\\
87.41	0\\
87.42	0\\
87.43	0\\
87.44	0\\
87.45	0\\
87.46	0\\
87.47	0\\
87.48	0\\
87.49	0\\
87.5	0\\
87.51	0\\
87.52	0\\
87.53	0\\
87.54	0\\
87.55	0\\
87.56	0\\
87.57	0\\
87.58	0\\
87.59	0\\
87.6	0\\
87.61	0\\
87.62	0\\
87.63	0\\
87.64	0\\
87.65	0\\
87.66	0\\
87.67	0\\
87.68	0\\
87.69	0\\
87.7	0\\
87.71	0\\
87.72	0\\
87.73	0\\
87.74	0\\
87.75	0\\
87.76	0\\
87.77	0\\
87.78	0\\
87.79	0\\
87.8	0\\
87.81	0\\
87.82	0\\
87.83	0\\
87.84	0\\
87.85	0\\
87.86	0\\
87.87	0\\
87.88	0\\
87.89	0\\
87.9	0\\
87.91	0\\
87.92	0\\
87.93	0\\
87.94	0\\
87.95	0\\
87.96	0\\
87.97	0\\
87.98	0\\
87.99	0\\
88	0\\
88.01	0\\
88.02	0\\
88.03	0\\
88.04	0\\
88.05	0\\
88.06	0\\
88.07	0\\
88.08	0\\
88.09	0\\
88.1	0\\
88.11	0\\
88.12	0\\
88.13	0\\
88.14	0\\
88.15	0\\
88.16	0\\
88.17	0\\
88.18	0\\
88.19	0\\
88.2	0\\
88.21	0\\
88.22	0\\
88.23	0\\
88.24	0\\
88.25	0\\
88.26	0\\
88.27	0\\
88.28	0\\
88.29	0\\
88.3	0\\
88.31	0\\
88.32	0\\
88.33	0\\
88.34	0\\
88.35	0\\
88.36	0\\
88.37	0\\
88.38	0\\
88.39	0\\
88.4	0\\
88.41	0\\
88.42	0\\
88.43	0\\
88.44	0\\
88.45	0\\
88.46	0\\
88.47	0\\
88.48	0\\
88.49	0\\
88.5	0\\
88.51	0\\
88.52	0\\
88.53	0\\
88.54	0\\
88.55	0\\
88.56	0\\
88.57	0\\
88.58	0\\
88.59	0\\
88.6	0\\
88.61	0\\
88.62	0\\
88.63	0\\
88.64	0\\
88.65	0\\
88.66	0\\
88.67	0\\
88.68	0\\
88.69	0\\
88.7	0\\
88.71	0\\
88.72	0\\
88.73	0\\
88.74	0\\
88.75	0\\
88.76	0\\
88.77	0\\
88.78	0\\
88.79	0\\
88.8	0\\
88.81	0\\
88.82	0\\
88.83	0\\
88.84	0\\
88.85	0\\
88.86	0\\
88.87	0\\
88.88	0\\
88.89	0\\
88.9	0\\
88.91	0\\
88.92	0\\
88.93	0\\
88.94	0\\
88.95	0\\
88.96	0\\
88.97	0\\
88.98	0\\
88.99	0\\
89	0\\
89.01	0\\
89.02	0\\
89.03	0\\
89.04	0\\
89.05	0\\
89.06	0\\
89.07	0\\
89.08	0\\
89.09	0\\
89.1	0\\
89.11	0\\
89.12	0\\
89.13	0\\
89.14	0\\
89.15	0\\
89.16	0\\
89.17	0\\
89.18	0\\
89.19	0\\
89.2	0\\
89.21	0\\
89.22	0\\
89.23	0\\
89.24	0\\
89.25	0\\
89.26	0\\
89.27	0\\
89.28	0\\
89.29	0\\
89.3	0\\
89.31	0\\
89.32	0\\
89.33	0\\
89.34	0\\
89.35	0\\
89.36	0\\
89.37	0\\
89.38	0\\
89.39	0\\
89.4	0\\
89.41	0\\
89.42	0\\
89.43	0\\
89.44	0\\
89.45	0\\
89.46	0\\
89.47	0\\
89.48	0\\
89.49	0\\
89.5	0\\
89.51	0\\
89.52	0\\
89.53	0\\
89.54	0\\
89.55	0\\
89.56	0\\
89.57	0\\
89.58	0\\
89.59	0\\
89.6	0\\
89.61	0\\
89.62	0\\
89.63	0\\
89.64	0\\
89.65	0\\
89.66	0\\
89.67	0\\
89.68	0\\
89.69	0\\
89.7	0\\
89.71	0\\
89.72	0\\
89.73	0\\
89.74	0\\
89.75	0\\
89.76	0\\
89.77	0\\
89.78	0\\
89.79	0\\
89.8	0\\
89.81	0\\
89.82	0\\
89.83	0\\
89.84	0\\
89.85	0\\
89.86	0\\
89.87	0\\
89.88	0\\
89.89	0\\
89.9	0\\
89.91	0\\
89.92	0\\
89.93	0\\
89.94	0\\
89.95	0\\
89.96	0\\
89.97	0\\
89.98	0\\
89.99	0\\
90	0\\
90.01	0\\
90.02	0\\
90.03	0\\
90.04	0\\
90.05	0\\
90.06	0\\
90.07	0\\
90.08	0\\
90.09	0\\
90.1	0\\
90.11	0\\
90.12	0\\
90.13	0\\
90.14	0\\
90.15	0\\
90.16	0\\
90.17	0\\
90.18	0\\
90.19	0\\
90.2	0\\
90.21	0\\
90.22	0\\
90.23	0\\
90.24	0\\
90.25	0\\
90.26	0\\
90.27	0\\
90.28	0\\
90.29	0\\
90.3	0\\
90.31	0\\
90.32	0\\
90.33	0\\
90.34	0\\
90.35	0\\
90.36	0\\
90.37	0\\
90.38	0\\
90.39	0\\
90.4	0\\
90.41	0\\
90.42	0\\
90.43	0\\
90.44	0\\
90.45	0\\
90.46	0\\
90.47	0\\
90.48	0\\
90.49	0\\
90.5	0\\
90.51	0\\
90.52	0\\
90.53	0\\
90.54	0\\
90.55	0\\
90.56	0\\
90.57	0\\
90.58	0\\
90.59	0\\
90.6	0\\
90.61	0\\
90.62	0\\
90.63	0\\
90.64	0\\
90.65	0\\
90.66	0\\
90.67	0\\
90.68	0\\
90.69	0\\
90.7	0\\
90.71	0\\
90.72	0\\
90.73	0\\
90.74	0\\
90.75	0\\
90.76	0\\
90.77	0\\
90.78	0\\
90.79	0\\
90.8	0\\
90.81	0\\
90.82	0\\
90.83	0\\
90.84	0\\
90.85	0\\
90.86	0\\
90.87	0\\
90.88	0\\
90.89	0\\
90.9	0\\
90.91	0\\
90.92	0\\
90.93	0\\
90.94	0\\
90.95	0\\
90.96	0\\
90.97	0\\
90.98	0\\
90.99	0\\
91	0\\
91.01	0\\
91.02	0\\
91.03	0\\
91.04	0\\
91.05	0\\
91.06	0\\
91.07	0\\
91.08	0\\
91.09	0\\
91.1	0\\
91.11	0\\
91.12	0\\
91.13	0\\
91.14	0\\
91.15	0\\
91.16	0\\
91.17	0\\
91.18	0\\
91.19	0\\
91.2	0\\
91.21	0\\
91.22	0\\
91.23	0\\
91.24	0\\
91.25	0\\
91.26	0\\
91.27	0\\
91.28	0\\
91.29	0\\
91.3	0\\
91.31	0\\
91.32	0\\
91.33	0\\
91.34	0\\
91.35	0\\
91.36	0\\
91.37	0\\
91.38	0\\
91.39	0\\
91.4	0\\
91.41	0\\
91.42	0\\
91.43	0\\
91.44	0\\
91.45	0\\
91.46	0\\
91.47	0\\
91.48	0\\
91.49	0\\
91.5	0\\
91.51	0\\
91.52	0\\
91.53	0\\
91.54	0\\
91.55	0\\
91.56	0\\
91.57	0\\
91.58	0\\
91.59	0\\
91.6	0\\
91.61	0\\
91.62	0\\
91.63	0\\
91.64	0\\
91.65	0\\
91.66	0\\
91.67	0\\
91.68	0\\
91.69	0\\
91.7	0\\
91.71	0\\
91.72	0\\
91.73	0\\
91.74	0\\
91.75	0\\
91.76	0\\
91.77	0\\
91.78	0\\
91.79	0\\
91.8	0\\
91.81	0\\
91.82	0\\
91.83	0\\
91.84	0\\
91.85	0\\
91.86	0\\
91.87	0\\
91.88	0\\
91.89	0\\
91.9	0\\
91.91	0\\
91.92	0\\
91.93	0\\
91.94	0\\
91.95	0\\
91.96	0\\
91.97	0\\
91.98	0\\
91.99	0\\
92	0\\
92.01	0\\
92.02	0\\
92.03	0\\
92.04	0\\
92.05	0\\
92.06	0\\
92.07	0\\
92.08	0\\
92.09	0\\
92.1	0\\
92.11	0\\
92.12	0\\
92.13	0\\
92.14	0\\
92.15	0\\
92.16	0\\
92.17	0\\
92.18	0\\
92.19	0\\
92.2	0\\
92.21	0\\
92.22	0\\
92.23	0\\
92.24	0\\
92.25	0\\
92.26	0\\
92.27	0\\
92.28	0\\
92.29	0\\
92.3	0\\
92.31	0\\
92.32	0\\
92.33	0\\
92.34	0\\
92.35	0\\
92.36	0\\
92.37	0\\
92.38	0\\
92.39	0\\
92.4	0\\
92.41	0\\
92.42	0\\
92.43	0\\
92.44	0\\
92.45	0\\
92.46	0\\
92.47	0\\
92.48	0\\
92.49	0\\
92.5	0\\
92.51	0\\
92.52	0\\
92.53	0\\
92.54	0\\
92.55	0\\
92.56	0\\
92.57	0\\
92.58	0\\
92.59	0\\
92.6	0\\
92.61	0\\
92.62	0\\
92.63	0\\
92.64	0\\
92.65	0\\
92.66	0\\
92.67	0\\
92.68	0\\
92.69	0\\
92.7	0\\
92.71	0\\
92.72	0\\
92.73	0\\
92.74	0\\
92.75	0\\
92.76	0\\
92.77	0\\
92.78	0\\
92.79	0\\
92.8	0\\
92.81	0\\
92.82	0\\
92.83	0\\
92.84	0\\
92.85	0\\
92.86	0\\
92.87	0\\
92.88	0\\
92.89	0\\
92.9	0\\
92.91	0\\
92.92	0\\
92.93	0\\
92.94	0\\
92.95	0\\
92.96	0\\
92.97	0\\
92.98	0\\
92.99	0\\
93	0\\
93.01	0\\
93.02	0\\
93.03	0\\
93.04	0\\
93.05	0\\
93.06	0\\
93.07	0\\
93.08	0\\
93.09	0\\
93.1	0\\
93.11	0\\
93.12	0\\
93.13	0\\
93.14	0\\
93.15	0\\
93.16	0\\
93.17	0\\
93.18	0\\
93.19	0\\
93.2	0\\
93.21	0\\
93.22	0\\
93.23	0\\
93.24	0\\
93.25	0\\
93.26	0\\
93.27	0\\
93.28	0\\
93.29	0\\
93.3	0\\
93.31	0\\
93.32	0\\
93.33	0\\
93.34	0\\
93.35	0\\
93.36	0\\
93.37	0\\
93.38	0\\
93.39	0\\
93.4	0\\
93.41	0\\
93.42	0\\
93.43	0\\
93.44	0\\
93.45	0\\
93.46	0\\
93.47	0\\
93.48	0\\
93.49	0\\
93.5	0\\
93.51	0\\
93.52	0\\
93.53	0\\
93.54	0\\
93.55	0\\
93.56	0\\
93.57	0\\
93.58	0\\
93.59	0\\
93.6	0\\
93.61	0\\
93.62	0\\
93.63	0\\
93.64	0\\
93.65	0\\
93.66	0\\
93.67	0\\
93.68	0\\
93.69	0\\
93.7	0\\
93.71	0\\
93.72	0\\
93.73	0\\
93.74	0\\
93.75	0\\
93.76	0\\
93.77	0\\
93.78	0\\
93.79	0\\
93.8	0\\
93.81	0\\
93.82	0\\
93.83	0\\
93.84	0\\
93.85	0\\
93.86	0\\
93.87	0\\
93.88	0\\
93.89	0\\
93.9	0\\
93.91	0\\
93.92	0\\
93.93	0\\
93.94	0\\
93.95	0\\
93.96	0\\
93.97	0\\
93.98	0\\
93.99	0\\
94	0\\
94.01	0\\
94.02	0\\
94.03	0\\
94.04	0\\
94.05	0\\
94.06	0\\
94.07	0\\
94.08	0\\
94.09	0\\
94.1	0\\
94.11	0\\
94.12	0\\
94.13	0\\
94.14	0\\
94.15	0\\
94.16	0\\
94.17	0\\
94.18	0\\
94.19	0\\
94.2	0\\
94.21	0\\
94.22	0\\
94.23	0\\
94.24	0\\
94.25	0\\
94.26	0\\
94.27	0\\
94.28	0\\
94.29	0\\
94.3	0\\
94.31	0\\
94.32	0\\
94.33	0\\
94.34	0\\
94.35	0\\
94.36	0\\
94.37	0\\
94.38	0\\
94.39	0\\
94.4	0\\
94.41	0\\
94.42	0\\
94.43	0\\
94.44	0\\
94.45	0\\
94.46	0\\
94.47	0\\
94.48	0\\
94.49	0\\
94.5	0\\
94.51	0\\
94.52	0\\
94.53	0\\
94.54	0\\
94.55	0\\
94.56	0\\
94.57	0\\
94.58	0\\
94.59	0\\
94.6	0\\
94.61	0\\
94.62	0\\
94.63	0\\
94.64	0\\
94.65	0\\
94.66	0\\
94.67	0\\
94.68	0\\
94.69	0\\
94.7	0\\
94.71	0\\
94.72	0\\
94.73	0\\
94.74	0\\
94.75	0\\
94.76	0\\
94.77	0\\
94.78	0\\
94.79	0\\
94.8	0\\
94.81	0\\
94.82	0\\
94.83	0\\
94.84	0\\
94.85	0\\
94.86	0\\
94.87	0\\
94.88	0\\
94.89	0\\
94.9	0\\
94.91	0\\
94.92	0\\
94.93	0\\
94.94	0\\
94.95	0\\
94.96	0\\
94.97	0\\
94.98	0\\
94.99	0\\
95	0\\
95.01	0\\
95.02	0\\
95.03	0\\
95.04	0\\
95.05	0\\
95.06	0\\
95.07	0\\
95.08	0\\
95.09	0\\
95.1	0\\
95.11	0\\
95.12	0\\
95.13	0\\
95.14	0\\
95.15	0\\
95.16	0\\
95.17	0\\
95.18	0\\
95.19	0\\
95.2	0\\
95.21	0\\
95.22	0\\
95.23	0\\
95.24	0\\
95.25	0\\
95.26	0\\
95.27	0\\
95.28	0\\
95.29	0\\
95.3	0\\
95.31	0\\
95.32	0\\
95.33	0\\
95.34	0\\
95.35	0\\
95.36	0\\
95.37	0\\
95.38	0\\
95.39	0\\
95.4	0\\
95.41	0\\
95.42	0\\
95.43	0\\
95.44	0\\
95.45	0\\
95.46	0\\
95.47	0\\
95.48	0\\
95.49	0\\
95.5	0\\
95.51	0\\
95.52	0\\
95.53	0\\
95.54	0\\
95.55	0\\
95.56	0\\
95.57	0\\
95.58	0\\
95.59	0\\
95.6	0\\
95.61	0\\
95.62	0\\
95.63	0\\
95.64	0\\
95.65	0\\
95.66	0\\
95.67	0\\
95.68	0\\
95.69	0\\
95.7	0\\
95.71	0\\
95.72	0\\
95.73	0\\
95.74	0\\
95.75	0\\
95.76	0\\
95.77	0\\
95.78	0\\
95.79	0\\
95.8	0\\
95.81	0\\
95.82	0\\
95.83	0\\
95.84	0\\
95.85	0\\
95.86	0\\
95.87	0\\
95.88	0\\
95.89	0\\
95.9	0\\
95.91	0\\
95.92	0\\
95.93	0\\
95.94	0\\
95.95	0\\
95.96	0\\
95.97	0\\
95.98	0\\
95.99	0\\
96	0\\
96.01	0\\
96.02	0\\
96.03	0\\
96.04	0\\
96.05	0\\
96.06	0\\
96.07	0\\
96.08	0\\
96.09	0\\
96.1	0\\
96.11	0\\
96.12	0\\
96.13	0\\
96.14	0\\
96.15	0\\
96.16	0\\
96.17	0\\
96.18	0\\
96.19	0\\
96.2	0\\
96.21	0\\
96.22	0\\
96.23	0\\
96.24	0\\
96.25	0\\
96.26	0\\
96.27	0\\
96.28	0\\
96.29	0\\
96.3	0\\
96.31	0\\
96.32	0\\
96.33	0\\
96.34	0\\
96.35	0\\
96.36	0\\
96.37	0\\
96.38	0\\
96.39	0\\
96.4	0\\
96.41	0\\
96.42	0\\
96.43	0\\
96.44	0\\
96.45	0\\
96.46	0\\
96.47	0\\
96.48	0\\
96.49	0\\
96.5	0\\
96.51	0\\
96.52	0\\
96.53	0\\
96.54	0\\
96.55	0\\
96.56	0\\
96.57	0\\
96.58	0\\
96.59	0\\
96.6	0\\
96.61	0\\
96.62	0\\
96.63	0\\
96.64	0\\
96.65	0\\
96.66	0\\
96.67	0\\
96.68	0\\
96.69	0\\
96.7	0\\
96.71	0\\
96.72	0\\
96.73	0\\
96.74	0\\
96.75	0\\
96.76	0\\
96.77	0\\
96.78	0\\
96.79	0\\
96.8	0\\
96.81	0\\
96.82	0\\
96.83	0\\
96.84	0\\
96.85	0\\
96.86	0\\
96.87	0\\
96.88	0\\
96.89	0\\
96.9	0\\
96.91	0\\
96.92	0\\
96.93	0\\
96.94	0\\
96.95	0\\
96.96	0\\
96.97	0\\
96.98	0\\
96.99	0\\
97	0\\
97.01	0\\
97.02	0\\
97.03	0\\
97.04	0\\
97.05	0\\
97.06	0\\
97.07	0\\
97.08	0\\
97.09	0\\
97.1	0\\
97.11	0\\
97.12	0\\
97.13	0\\
97.14	0\\
97.15	0\\
97.16	0\\
97.17	0\\
97.18	0\\
97.19	0\\
97.2	0\\
97.21	0\\
97.22	0\\
97.23	0\\
97.24	0\\
97.25	0\\
97.26	0\\
97.27	0\\
97.28	0\\
97.29	0\\
97.3	0\\
97.31	0\\
97.32	0\\
97.33	0\\
97.34	0\\
97.35	0\\
97.36	0\\
97.37	0\\
97.38	0\\
97.39	0\\
97.4	0\\
97.41	0\\
97.42	0\\
97.43	0\\
97.44	0\\
97.45	0\\
97.46	0\\
97.47	0\\
97.48	0\\
97.49	0\\
97.5	0\\
97.51	0\\
97.52	0\\
97.53	0\\
97.54	0\\
97.55	0\\
97.56	0\\
97.57	0\\
97.58	0\\
97.59	0\\
97.6	0\\
97.61	0\\
97.62	0\\
97.63	0\\
97.64	0\\
97.65	0\\
97.66	0\\
97.67	0\\
97.68	0\\
97.69	0\\
97.7	0\\
97.71	0\\
97.72	0\\
97.73	0\\
97.74	0\\
97.75	0\\
97.76	0\\
97.77	0\\
97.78	0\\
97.79	0\\
97.8	0\\
97.81	0\\
97.82	0\\
97.83	0\\
97.84	0\\
97.85	0\\
97.86	0\\
97.87	0\\
97.88	0\\
97.89	0\\
97.9	0\\
97.91	0\\
97.92	0\\
97.93	0\\
97.94	0\\
97.95	0\\
97.96	0\\
97.97	0\\
97.98	0\\
97.99	0\\
98	0\\
98.01	0\\
98.02	0\\
98.03	0\\
98.04	0\\
98.05	0\\
98.06	0\\
98.07	0\\
98.08	0\\
98.09	0\\
98.1	0\\
98.11	0\\
98.12	0\\
98.13	0\\
98.14	0\\
98.15	0\\
98.16	0\\
98.17	0\\
98.18	0\\
98.19	0\\
98.2	0\\
98.21	0\\
98.22	0\\
98.23	0\\
98.24	0\\
98.25	0\\
98.26	0\\
98.27	0\\
98.28	0\\
98.29	0\\
98.3	0\\
98.31	0\\
98.32	0\\
98.33	0\\
98.34	0\\
98.35	0\\
98.36	0\\
98.37	0\\
98.38	0\\
98.39	0\\
98.4	0\\
98.41	0\\
98.42	0\\
98.43	0\\
98.44	0\\
98.45	0\\
98.46	0\\
98.47	0\\
98.48	0\\
98.49	0\\
98.5	0\\
98.51	0\\
98.52	0\\
98.53	0\\
98.54	0\\
98.55	0\\
98.56	0\\
98.57	0\\
98.58	0\\
98.59	0\\
98.6	0\\
98.61	0\\
98.62	0\\
98.63	0\\
98.64	0\\
98.65	0\\
98.66	0\\
98.67	0\\
98.68	0\\
98.69	0\\
98.7	0\\
98.71	0\\
98.72	0\\
98.73	0\\
98.74	0\\
98.75	0\\
98.76	0\\
98.77	0\\
98.78	0\\
98.79	0\\
98.8	0\\
98.81	0\\
98.82	0\\
98.83	0\\
98.84	0\\
98.85	0\\
98.86	0\\
98.87	0\\
98.88	0\\
98.89	0\\
98.9	0\\
98.91	0\\
98.92	0\\
98.93	0\\
98.94	0\\
98.95	0\\
98.96	5.34233723955835e-05\\
98.97	0.000249770323846921\\
98.98	0.000447526972246478\\
98.99	0.000646713529065458\\
99	0.000847350822643051\\
99.01	0.00104946028175264\\
99.02	0.00125306395411695\\
99.03	0.00145818452931908\\
99.04	0.0016648453630583\\
99.05	0.00187301328746817\\
99.06	0.00208271075060248\\
99.07	0.00229396160842808\\
99.08	0.00250679044560442\\
99.09	0.00272122260386963\\
99.1	0.002937284219982\\
99.11	0.0031550022516486\\
99.12	0.00337440450742363\\
99.13	0.00359551968105016\\
99.14	0.00380493250998101\\
99.15	0.00385656934463024\\
99.16	0.00390855237475965\\
99.17	0.00396087846322035\\
99.18	0.00401354417880869\\
99.19	0.00406654578056501\\
99.2	0.00411987921666678\\
99.21	0.00417354011061869\\
99.22	0.00422753536402045\\
99.23	0.00428190027022628\\
99.24	0.00433663114616696\\
99.25	0.0043917239783746\\
99.26	0.00444717440792617\\
99.27	0.00450297771469116\\
99.28	0.0045591288008455\\
99.29	0.00461562217361109\\
99.3	0.00467245192717746\\
99.31	0.00472961172375959\\
99.32	0.0047870947737426\\
99.33	0.0048448938148608\\
99.34	0.00490300109035484\\
99.35	0.00496140832634903\\
99.36	0.00502010670781841\\
99.37	0.00507908684954088\\
99.38	0.0051383387442909\\
99.39	0.00519785176102771\\
99.4	0.00525761461537342\\
99.41	0.00531761533853545\\
99.42	0.00537815543682032\\
99.43	0.00543924041152297\\
99.44	0.00550087518071841\\
99.45	0.00556306471542116\\
99.46	0.0056258140408721\\
99.47	0.00568912823785012\\
99.48	0.00575301234814035\\
99.49	0.00581747134619418\\
99.5	0.00588251024853272\\
99.51	0.00594813411434026\\
99.52	0.00601434804608568\\
99.53	0.00608115719017392\\
99.54	0.00614856673762911\\
99.55	0.00621658192481169\\
99.56	0.00628520803417149\\
99.57	0.00635445039503925\\
99.58	0.00642431438445891\\
99.59	0.00649480542806346\\
99.6	0.00656592900099709\\
99.61	0.00663769062888668\\
99.62	0.00671009588886575\\
99.63	0.00678315041065447\\
99.64	0.00685685987769925\\
99.65	0.00693123002837576\\
99.66	0.00700626665725987\\
99.67	0.00708197561647047\\
99.68	0.00715836281708942\\
99.69	0.00723543421531813\\
99.7	0.00731319582612591\\
99.71	0.00739165372783235\\
99.72	0.00747081406389988\\
99.73	0.00755068304485562\\
99.74	0.00763126695028889\\
99.75	0.00771257213096943\\
99.76	0.00779460501109486\\
99.77	0.00787737209067585\\
99.78	0.00796087994806859\\
99.79	0.0080451352426644\\
99.8	0.00813014471774746\\
99.81	0.00821591520353212\\
99.82	0.00830245362039213\\
99.83	0.00838976698229511\\
99.84	0.00847786240045663\\
99.85	0.008566747087229\\
99.86	0.00865642836024134\\
99.87	0.00874691364680851\\
99.88	0.0088382104886279\\
99.89	0.00893032654678435\\
99.9	0.0090232696070851\\
99.91	0.00911704758574675\\
99.92	0.00921166853545841\\
99.93	0.0093071406518502\\
99.94	0.009403472280396\\
99.95	0.00950067192378203\\
99.96	0.00959874824977478\\
99.97	0.00969771009962486\\
99.98	0.00979756649704546\\
99.99	0.00989832665780802\\
100	0.01\\
};
\addlegendentry{$q=-2$};

\addplot [color=blue,dashed,forget plot]
  table[row sep=crcr]{%
0.01	0\\
0.02	0\\
0.03	0\\
0.04	0\\
0.05	0\\
0.06	0\\
0.07	0\\
0.08	0\\
0.09	0\\
0.1	0\\
0.11	0\\
0.12	0\\
0.13	0\\
0.14	0\\
0.15	0\\
0.16	0\\
0.17	0\\
0.18	0\\
0.19	0\\
0.2	0\\
0.21	0\\
0.22	0\\
0.23	0\\
0.24	0\\
0.25	0\\
0.26	0\\
0.27	0\\
0.28	0\\
0.29	0\\
0.3	0\\
0.31	0\\
0.32	0\\
0.33	0\\
0.34	0\\
0.35	0\\
0.36	0\\
0.37	0\\
0.38	0\\
0.39	0\\
0.4	0\\
0.41	0\\
0.42	0\\
0.43	0\\
0.44	0\\
0.45	0\\
0.46	0\\
0.47	0\\
0.48	0\\
0.49	0\\
0.5	0\\
0.51	0\\
0.52	0\\
0.53	0\\
0.54	0\\
0.55	0\\
0.56	0\\
0.57	0\\
0.58	0\\
0.59	0\\
0.6	0\\
0.61	0\\
0.62	0\\
0.63	0\\
0.64	0\\
0.65	0\\
0.66	0\\
0.67	0\\
0.68	0\\
0.69	0\\
0.7	0\\
0.71	0\\
0.72	0\\
0.73	0\\
0.74	0\\
0.75	0\\
0.76	0\\
0.77	0\\
0.78	0\\
0.79	0\\
0.8	0\\
0.81	0\\
0.82	0\\
0.83	0\\
0.84	0\\
0.85	0\\
0.86	0\\
0.87	0\\
0.88	0\\
0.89	0\\
0.9	0\\
0.91	0\\
0.92	0\\
0.93	0\\
0.94	0\\
0.95	0\\
0.96	0\\
0.97	0\\
0.98	0\\
0.99	0\\
1	0\\
1.01	0\\
1.02	0\\
1.03	0\\
1.04	0\\
1.05	0\\
1.06	0\\
1.07	0\\
1.08	0\\
1.09	0\\
1.1	0\\
1.11	0\\
1.12	0\\
1.13	0\\
1.14	0\\
1.15	0\\
1.16	0\\
1.17	0\\
1.18	0\\
1.19	0\\
1.2	0\\
1.21	0\\
1.22	0\\
1.23	0\\
1.24	0\\
1.25	0\\
1.26	0\\
1.27	0\\
1.28	0\\
1.29	0\\
1.3	0\\
1.31	0\\
1.32	0\\
1.33	0\\
1.34	0\\
1.35	0\\
1.36	0\\
1.37	0\\
1.38	0\\
1.39	0\\
1.4	0\\
1.41	0\\
1.42	0\\
1.43	0\\
1.44	0\\
1.45	0\\
1.46	0\\
1.47	0\\
1.48	0\\
1.49	0\\
1.5	0\\
1.51	0\\
1.52	0\\
1.53	0\\
1.54	0\\
1.55	0\\
1.56	0\\
1.57	0\\
1.58	0\\
1.59	0\\
1.6	0\\
1.61	0\\
1.62	0\\
1.63	0\\
1.64	0\\
1.65	0\\
1.66	0\\
1.67	0\\
1.68	0\\
1.69	0\\
1.7	0\\
1.71	0\\
1.72	0\\
1.73	0\\
1.74	0\\
1.75	0\\
1.76	0\\
1.77	0\\
1.78	0\\
1.79	0\\
1.8	0\\
1.81	0\\
1.82	0\\
1.83	0\\
1.84	0\\
1.85	0\\
1.86	0\\
1.87	0\\
1.88	0\\
1.89	0\\
1.9	0\\
1.91	0\\
1.92	0\\
1.93	0\\
1.94	0\\
1.95	0\\
1.96	0\\
1.97	0\\
1.98	0\\
1.99	0\\
2	0\\
2.01	0\\
2.02	0\\
2.03	0\\
2.04	0\\
2.05	0\\
2.06	0\\
2.07	0\\
2.08	0\\
2.09	0\\
2.1	0\\
2.11	0\\
2.12	0\\
2.13	0\\
2.14	0\\
2.15	0\\
2.16	0\\
2.17	0\\
2.18	0\\
2.19	0\\
2.2	0\\
2.21	0\\
2.22	0\\
2.23	0\\
2.24	0\\
2.25	0\\
2.26	0\\
2.27	0\\
2.28	0\\
2.29	0\\
2.3	0\\
2.31	0\\
2.32	0\\
2.33	0\\
2.34	0\\
2.35	0\\
2.36	0\\
2.37	0\\
2.38	0\\
2.39	0\\
2.4	0\\
2.41	0\\
2.42	0\\
2.43	0\\
2.44	0\\
2.45	0\\
2.46	0\\
2.47	0\\
2.48	0\\
2.49	0\\
2.5	0\\
2.51	0\\
2.52	0\\
2.53	0\\
2.54	0\\
2.55	0\\
2.56	0\\
2.57	0\\
2.58	0\\
2.59	0\\
2.6	0\\
2.61	0\\
2.62	0\\
2.63	0\\
2.64	0\\
2.65	0\\
2.66	0\\
2.67	0\\
2.68	0\\
2.69	0\\
2.7	0\\
2.71	0\\
2.72	0\\
2.73	0\\
2.74	0\\
2.75	0\\
2.76	0\\
2.77	0\\
2.78	0\\
2.79	0\\
2.8	0\\
2.81	0\\
2.82	0\\
2.83	0\\
2.84	0\\
2.85	0\\
2.86	0\\
2.87	0\\
2.88	0\\
2.89	0\\
2.9	0\\
2.91	0\\
2.92	0\\
2.93	0\\
2.94	0\\
2.95	0\\
2.96	0\\
2.97	0\\
2.98	0\\
2.99	0\\
3	0\\
3.01	0\\
3.02	0\\
3.03	0\\
3.04	0\\
3.05	0\\
3.06	0\\
3.07	0\\
3.08	0\\
3.09	0\\
3.1	0\\
3.11	0\\
3.12	0\\
3.13	0\\
3.14	0\\
3.15	0\\
3.16	0\\
3.17	0\\
3.18	0\\
3.19	0\\
3.2	0\\
3.21	0\\
3.22	0\\
3.23	0\\
3.24	0\\
3.25	0\\
3.26	0\\
3.27	0\\
3.28	0\\
3.29	0\\
3.3	0\\
3.31	0\\
3.32	0\\
3.33	0\\
3.34	0\\
3.35	0\\
3.36	0\\
3.37	0\\
3.38	0\\
3.39	0\\
3.4	0\\
3.41	0\\
3.42	0\\
3.43	0\\
3.44	0\\
3.45	0\\
3.46	0\\
3.47	0\\
3.48	0\\
3.49	0\\
3.5	0\\
3.51	0\\
3.52	0\\
3.53	0\\
3.54	0\\
3.55	0\\
3.56	0\\
3.57	0\\
3.58	0\\
3.59	0\\
3.6	0\\
3.61	0\\
3.62	0\\
3.63	0\\
3.64	0\\
3.65	0\\
3.66	0\\
3.67	0\\
3.68	0\\
3.69	0\\
3.7	0\\
3.71	0\\
3.72	0\\
3.73	0\\
3.74	0\\
3.75	0\\
3.76	0\\
3.77	0\\
3.78	0\\
3.79	0\\
3.8	0\\
3.81	0\\
3.82	0\\
3.83	0\\
3.84	0\\
3.85	0\\
3.86	0\\
3.87	0\\
3.88	0\\
3.89	0\\
3.9	0\\
3.91	0\\
3.92	0\\
3.93	0\\
3.94	0\\
3.95	0\\
3.96	0\\
3.97	0\\
3.98	0\\
3.99	0\\
4	0\\
4.01	0\\
4.02	0\\
4.03	0\\
4.04	0\\
4.05	0\\
4.06	0\\
4.07	0\\
4.08	0\\
4.09	0\\
4.1	0\\
4.11	0\\
4.12	0\\
4.13	0\\
4.14	0\\
4.15	0\\
4.16	0\\
4.17	0\\
4.18	0\\
4.19	0\\
4.2	0\\
4.21	0\\
4.22	0\\
4.23	0\\
4.24	0\\
4.25	0\\
4.26	0\\
4.27	0\\
4.28	0\\
4.29	0\\
4.3	0\\
4.31	0\\
4.32	0\\
4.33	0\\
4.34	0\\
4.35	0\\
4.36	0\\
4.37	0\\
4.38	0\\
4.39	0\\
4.4	0\\
4.41	0\\
4.42	0\\
4.43	0\\
4.44	0\\
4.45	0\\
4.46	0\\
4.47	0\\
4.48	0\\
4.49	0\\
4.5	0\\
4.51	0\\
4.52	0\\
4.53	0\\
4.54	0\\
4.55	0\\
4.56	0\\
4.57	0\\
4.58	0\\
4.59	0\\
4.6	0\\
4.61	0\\
4.62	0\\
4.63	0\\
4.64	0\\
4.65	0\\
4.66	0\\
4.67	0\\
4.68	0\\
4.69	0\\
4.7	0\\
4.71	0\\
4.72	0\\
4.73	0\\
4.74	0\\
4.75	0\\
4.76	0\\
4.77	0\\
4.78	0\\
4.79	0\\
4.8	0\\
4.81	0\\
4.82	0\\
4.83	0\\
4.84	0\\
4.85	0\\
4.86	0\\
4.87	0\\
4.88	0\\
4.89	0\\
4.9	0\\
4.91	0\\
4.92	0\\
4.93	0\\
4.94	0\\
4.95	0\\
4.96	0\\
4.97	0\\
4.98	0\\
4.99	0\\
5	0\\
5.01	0\\
5.02	0\\
5.03	0\\
5.04	0\\
5.05	0\\
5.06	0\\
5.07	0\\
5.08	0\\
5.09	0\\
5.1	0\\
5.11	0\\
5.12	0\\
5.13	0\\
5.14	0\\
5.15	0\\
5.16	0\\
5.17	0\\
5.18	0\\
5.19	0\\
5.2	0\\
5.21	0\\
5.22	0\\
5.23	0\\
5.24	0\\
5.25	0\\
5.26	0\\
5.27	0\\
5.28	0\\
5.29	0\\
5.3	0\\
5.31	0\\
5.32	0\\
5.33	0\\
5.34	0\\
5.35	0\\
5.36	0\\
5.37	0\\
5.38	0\\
5.39	0\\
5.4	0\\
5.41	0\\
5.42	0\\
5.43	0\\
5.44	0\\
5.45	0\\
5.46	0\\
5.47	0\\
5.48	0\\
5.49	0\\
5.5	0\\
5.51	0\\
5.52	0\\
5.53	0\\
5.54	0\\
5.55	0\\
5.56	0\\
5.57	0\\
5.58	0\\
5.59	0\\
5.6	0\\
5.61	0\\
5.62	0\\
5.63	0\\
5.64	0\\
5.65	0\\
5.66	0\\
5.67	0\\
5.68	0\\
5.69	0\\
5.7	0\\
5.71	0\\
5.72	0\\
5.73	0\\
5.74	0\\
5.75	0\\
5.76	0\\
5.77	0\\
5.78	0\\
5.79	0\\
5.8	0\\
5.81	0\\
5.82	0\\
5.83	0\\
5.84	0\\
5.85	0\\
5.86	0\\
5.87	0\\
5.88	0\\
5.89	0\\
5.9	0\\
5.91	0\\
5.92	0\\
5.93	0\\
5.94	0\\
5.95	0\\
5.96	0\\
5.97	0\\
5.98	0\\
5.99	0\\
6	0\\
6.01	0\\
6.02	0\\
6.03	0\\
6.04	0\\
6.05	0\\
6.06	0\\
6.07	0\\
6.08	0\\
6.09	0\\
6.1	0\\
6.11	0\\
6.12	0\\
6.13	0\\
6.14	0\\
6.15	0\\
6.16	0\\
6.17	0\\
6.18	0\\
6.19	0\\
6.2	0\\
6.21	0\\
6.22	0\\
6.23	0\\
6.24	0\\
6.25	0\\
6.26	0\\
6.27	0\\
6.28	0\\
6.29	0\\
6.3	0\\
6.31	0\\
6.32	0\\
6.33	0\\
6.34	0\\
6.35	0\\
6.36	0\\
6.37	0\\
6.38	0\\
6.39	0\\
6.4	0\\
6.41	0\\
6.42	0\\
6.43	0\\
6.44	0\\
6.45	0\\
6.46	0\\
6.47	0\\
6.48	0\\
6.49	0\\
6.5	0\\
6.51	0\\
6.52	0\\
6.53	0\\
6.54	0\\
6.55	0\\
6.56	0\\
6.57	0\\
6.58	0\\
6.59	0\\
6.6	0\\
6.61	0\\
6.62	0\\
6.63	0\\
6.64	0\\
6.65	0\\
6.66	0\\
6.67	0\\
6.68	0\\
6.69	0\\
6.7	0\\
6.71	0\\
6.72	0\\
6.73	0\\
6.74	0\\
6.75	0\\
6.76	0\\
6.77	0\\
6.78	0\\
6.79	0\\
6.8	0\\
6.81	0\\
6.82	0\\
6.83	0\\
6.84	0\\
6.85	0\\
6.86	0\\
6.87	0\\
6.88	0\\
6.89	0\\
6.9	0\\
6.91	0\\
6.92	0\\
6.93	0\\
6.94	0\\
6.95	0\\
6.96	0\\
6.97	0\\
6.98	0\\
6.99	0\\
7	0\\
7.01	0\\
7.02	0\\
7.03	0\\
7.04	0\\
7.05	0\\
7.06	0\\
7.07	0\\
7.08	0\\
7.09	0\\
7.1	0\\
7.11	0\\
7.12	0\\
7.13	0\\
7.14	0\\
7.15	0\\
7.16	0\\
7.17	0\\
7.18	0\\
7.19	0\\
7.2	0\\
7.21	0\\
7.22	0\\
7.23	0\\
7.24	0\\
7.25	0\\
7.26	0\\
7.27	0\\
7.28	0\\
7.29	0\\
7.3	0\\
7.31	0\\
7.32	0\\
7.33	0\\
7.34	0\\
7.35	0\\
7.36	0\\
7.37	0\\
7.38	0\\
7.39	0\\
7.4	0\\
7.41	0\\
7.42	0\\
7.43	0\\
7.44	0\\
7.45	0\\
7.46	0\\
7.47	0\\
7.48	0\\
7.49	0\\
7.5	0\\
7.51	0\\
7.52	0\\
7.53	0\\
7.54	0\\
7.55	0\\
7.56	0\\
7.57	0\\
7.58	0\\
7.59	0\\
7.6	0\\
7.61	0\\
7.62	0\\
7.63	0\\
7.64	0\\
7.65	0\\
7.66	0\\
7.67	0\\
7.68	0\\
7.69	0\\
7.7	0\\
7.71	0\\
7.72	0\\
7.73	0\\
7.74	0\\
7.75	0\\
7.76	0\\
7.77	0\\
7.78	0\\
7.79	0\\
7.8	0\\
7.81	0\\
7.82	0\\
7.83	0\\
7.84	0\\
7.85	0\\
7.86	0\\
7.87	0\\
7.88	0\\
7.89	0\\
7.9	0\\
7.91	0\\
7.92	0\\
7.93	0\\
7.94	0\\
7.95	0\\
7.96	0\\
7.97	0\\
7.98	0\\
7.99	0\\
8	0\\
8.01	0\\
8.02	0\\
8.03	0\\
8.04	0\\
8.05	0\\
8.06	0\\
8.07	0\\
8.08	0\\
8.09	0\\
8.1	0\\
8.11	0\\
8.12	0\\
8.13	0\\
8.14	0\\
8.15	0\\
8.16	0\\
8.17	0\\
8.18	0\\
8.19	0\\
8.2	0\\
8.21	0\\
8.22	0\\
8.23	0\\
8.24	0\\
8.25	0\\
8.26	0\\
8.27	0\\
8.28	0\\
8.29	0\\
8.3	0\\
8.31	0\\
8.32	0\\
8.33	0\\
8.34	0\\
8.35	0\\
8.36	0\\
8.37	0\\
8.38	0\\
8.39	0\\
8.4	0\\
8.41	0\\
8.42	0\\
8.43	0\\
8.44	0\\
8.45	0\\
8.46	0\\
8.47	0\\
8.48	0\\
8.49	0\\
8.5	0\\
8.51	0\\
8.52	0\\
8.53	0\\
8.54	0\\
8.55	0\\
8.56	0\\
8.57	0\\
8.58	0\\
8.59	0\\
8.6	0\\
8.61	0\\
8.62	0\\
8.63	0\\
8.64	0\\
8.65	0\\
8.66	0\\
8.67	0\\
8.68	0\\
8.69	0\\
8.7	0\\
8.71	0\\
8.72	0\\
8.73	0\\
8.74	0\\
8.75	0\\
8.76	0\\
8.77	0\\
8.78	0\\
8.79	0\\
8.8	0\\
8.81	0\\
8.82	0\\
8.83	0\\
8.84	0\\
8.85	0\\
8.86	0\\
8.87	0\\
8.88	0\\
8.89	0\\
8.9	0\\
8.91	0\\
8.92	0\\
8.93	0\\
8.94	0\\
8.95	0\\
8.96	0\\
8.97	0\\
8.98	0\\
8.99	0\\
9	0\\
9.01	0\\
9.02	0\\
9.03	0\\
9.04	0\\
9.05	0\\
9.06	0\\
9.07	0\\
9.08	0\\
9.09	0\\
9.1	0\\
9.11	0\\
9.12	0\\
9.13	0\\
9.14	0\\
9.15	0\\
9.16	0\\
9.17	0\\
9.18	0\\
9.19	0\\
9.2	0\\
9.21	0\\
9.22	0\\
9.23	0\\
9.24	0\\
9.25	0\\
9.26	0\\
9.27	0\\
9.28	0\\
9.29	0\\
9.3	0\\
9.31	0\\
9.32	0\\
9.33	0\\
9.34	0\\
9.35	0\\
9.36	0\\
9.37	0\\
9.38	0\\
9.39	0\\
9.4	0\\
9.41	0\\
9.42	0\\
9.43	0\\
9.44	0\\
9.45	0\\
9.46	0\\
9.47	0\\
9.48	0\\
9.49	0\\
9.5	0\\
9.51	0\\
9.52	0\\
9.53	0\\
9.54	0\\
9.55	0\\
9.56	0\\
9.57	0\\
9.58	0\\
9.59	0\\
9.6	0\\
9.61	0\\
9.62	0\\
9.63	0\\
9.64	0\\
9.65	0\\
9.66	0\\
9.67	0\\
9.68	0\\
9.69	0\\
9.7	0\\
9.71	0\\
9.72	0\\
9.73	0\\
9.74	0\\
9.75	0\\
9.76	0\\
9.77	0\\
9.78	0\\
9.79	0\\
9.8	0\\
9.81	0\\
9.82	0\\
9.83	0\\
9.84	0\\
9.85	0\\
9.86	0\\
9.87	0\\
9.88	0\\
9.89	0\\
9.9	0\\
9.91	0\\
9.92	0\\
9.93	0\\
9.94	0\\
9.95	0\\
9.96	0\\
9.97	0\\
9.98	0\\
9.99	0\\
10	0\\
10.01	0\\
10.02	0\\
10.03	0\\
10.04	0\\
10.05	0\\
10.06	0\\
10.07	0\\
10.08	0\\
10.09	0\\
10.1	0\\
10.11	0\\
10.12	0\\
10.13	0\\
10.14	0\\
10.15	0\\
10.16	0\\
10.17	0\\
10.18	0\\
10.19	0\\
10.2	0\\
10.21	0\\
10.22	0\\
10.23	0\\
10.24	0\\
10.25	0\\
10.26	0\\
10.27	0\\
10.28	0\\
10.29	0\\
10.3	0\\
10.31	0\\
10.32	0\\
10.33	0\\
10.34	0\\
10.35	0\\
10.36	0\\
10.37	0\\
10.38	0\\
10.39	0\\
10.4	0\\
10.41	0\\
10.42	0\\
10.43	0\\
10.44	0\\
10.45	0\\
10.46	0\\
10.47	0\\
10.48	0\\
10.49	0\\
10.5	0\\
10.51	0\\
10.52	0\\
10.53	0\\
10.54	0\\
10.55	0\\
10.56	0\\
10.57	0\\
10.58	0\\
10.59	0\\
10.6	0\\
10.61	0\\
10.62	0\\
10.63	0\\
10.64	0\\
10.65	0\\
10.66	0\\
10.67	0\\
10.68	0\\
10.69	0\\
10.7	0\\
10.71	0\\
10.72	0\\
10.73	0\\
10.74	0\\
10.75	0\\
10.76	0\\
10.77	0\\
10.78	0\\
10.79	0\\
10.8	0\\
10.81	0\\
10.82	0\\
10.83	0\\
10.84	0\\
10.85	0\\
10.86	0\\
10.87	0\\
10.88	0\\
10.89	0\\
10.9	0\\
10.91	0\\
10.92	0\\
10.93	0\\
10.94	0\\
10.95	0\\
10.96	0\\
10.97	0\\
10.98	0\\
10.99	0\\
11	0\\
11.01	0\\
11.02	0\\
11.03	0\\
11.04	0\\
11.05	0\\
11.06	0\\
11.07	0\\
11.08	0\\
11.09	0\\
11.1	0\\
11.11	0\\
11.12	0\\
11.13	0\\
11.14	0\\
11.15	0\\
11.16	0\\
11.17	0\\
11.18	0\\
11.19	0\\
11.2	0\\
11.21	0\\
11.22	0\\
11.23	0\\
11.24	0\\
11.25	0\\
11.26	0\\
11.27	0\\
11.28	0\\
11.29	0\\
11.3	0\\
11.31	0\\
11.32	0\\
11.33	0\\
11.34	0\\
11.35	0\\
11.36	0\\
11.37	0\\
11.38	0\\
11.39	0\\
11.4	0\\
11.41	0\\
11.42	0\\
11.43	0\\
11.44	0\\
11.45	0\\
11.46	0\\
11.47	0\\
11.48	0\\
11.49	0\\
11.5	0\\
11.51	0\\
11.52	0\\
11.53	0\\
11.54	0\\
11.55	0\\
11.56	0\\
11.57	0\\
11.58	0\\
11.59	0\\
11.6	0\\
11.61	0\\
11.62	0\\
11.63	0\\
11.64	0\\
11.65	0\\
11.66	0\\
11.67	0\\
11.68	0\\
11.69	0\\
11.7	0\\
11.71	0\\
11.72	0\\
11.73	0\\
11.74	0\\
11.75	0\\
11.76	0\\
11.77	0\\
11.78	0\\
11.79	0\\
11.8	0\\
11.81	0\\
11.82	0\\
11.83	0\\
11.84	0\\
11.85	0\\
11.86	0\\
11.87	0\\
11.88	0\\
11.89	0\\
11.9	0\\
11.91	0\\
11.92	0\\
11.93	0\\
11.94	0\\
11.95	0\\
11.96	0\\
11.97	0\\
11.98	0\\
11.99	0\\
12	0\\
12.01	0\\
12.02	0\\
12.03	0\\
12.04	0\\
12.05	0\\
12.06	0\\
12.07	0\\
12.08	0\\
12.09	0\\
12.1	0\\
12.11	0\\
12.12	0\\
12.13	0\\
12.14	0\\
12.15	0\\
12.16	0\\
12.17	0\\
12.18	0\\
12.19	0\\
12.2	0\\
12.21	0\\
12.22	0\\
12.23	0\\
12.24	0\\
12.25	0\\
12.26	0\\
12.27	0\\
12.28	0\\
12.29	0\\
12.3	0\\
12.31	0\\
12.32	0\\
12.33	0\\
12.34	0\\
12.35	0\\
12.36	0\\
12.37	0\\
12.38	0\\
12.39	0\\
12.4	0\\
12.41	0\\
12.42	0\\
12.43	0\\
12.44	0\\
12.45	0\\
12.46	0\\
12.47	0\\
12.48	0\\
12.49	0\\
12.5	0\\
12.51	0\\
12.52	0\\
12.53	0\\
12.54	0\\
12.55	0\\
12.56	0\\
12.57	0\\
12.58	0\\
12.59	0\\
12.6	0\\
12.61	0\\
12.62	0\\
12.63	0\\
12.64	0\\
12.65	0\\
12.66	0\\
12.67	0\\
12.68	0\\
12.69	0\\
12.7	0\\
12.71	0\\
12.72	0\\
12.73	0\\
12.74	0\\
12.75	0\\
12.76	0\\
12.77	0\\
12.78	0\\
12.79	0\\
12.8	0\\
12.81	0\\
12.82	0\\
12.83	0\\
12.84	0\\
12.85	0\\
12.86	0\\
12.87	0\\
12.88	0\\
12.89	0\\
12.9	0\\
12.91	0\\
12.92	0\\
12.93	0\\
12.94	0\\
12.95	0\\
12.96	0\\
12.97	0\\
12.98	0\\
12.99	0\\
13	0\\
13.01	0\\
13.02	0\\
13.03	0\\
13.04	0\\
13.05	0\\
13.06	0\\
13.07	0\\
13.08	0\\
13.09	0\\
13.1	0\\
13.11	0\\
13.12	0\\
13.13	0\\
13.14	0\\
13.15	0\\
13.16	0\\
13.17	0\\
13.18	0\\
13.19	0\\
13.2	0\\
13.21	0\\
13.22	0\\
13.23	0\\
13.24	0\\
13.25	0\\
13.26	0\\
13.27	0\\
13.28	0\\
13.29	0\\
13.3	0\\
13.31	0\\
13.32	0\\
13.33	0\\
13.34	0\\
13.35	0\\
13.36	0\\
13.37	0\\
13.38	0\\
13.39	0\\
13.4	0\\
13.41	0\\
13.42	0\\
13.43	0\\
13.44	0\\
13.45	0\\
13.46	0\\
13.47	0\\
13.48	0\\
13.49	0\\
13.5	0\\
13.51	0\\
13.52	0\\
13.53	0\\
13.54	0\\
13.55	0\\
13.56	0\\
13.57	0\\
13.58	0\\
13.59	0\\
13.6	0\\
13.61	0\\
13.62	0\\
13.63	0\\
13.64	0\\
13.65	0\\
13.66	0\\
13.67	0\\
13.68	0\\
13.69	0\\
13.7	0\\
13.71	0\\
13.72	0\\
13.73	0\\
13.74	0\\
13.75	0\\
13.76	0\\
13.77	0\\
13.78	0\\
13.79	0\\
13.8	0\\
13.81	0\\
13.82	0\\
13.83	0\\
13.84	0\\
13.85	0\\
13.86	0\\
13.87	0\\
13.88	0\\
13.89	0\\
13.9	0\\
13.91	0\\
13.92	0\\
13.93	0\\
13.94	0\\
13.95	0\\
13.96	0\\
13.97	0\\
13.98	0\\
13.99	0\\
14	0\\
14.01	0\\
14.02	0\\
14.03	0\\
14.04	0\\
14.05	0\\
14.06	0\\
14.07	0\\
14.08	0\\
14.09	0\\
14.1	0\\
14.11	0\\
14.12	0\\
14.13	0\\
14.14	0\\
14.15	0\\
14.16	0\\
14.17	0\\
14.18	0\\
14.19	0\\
14.2	0\\
14.21	0\\
14.22	0\\
14.23	0\\
14.24	0\\
14.25	0\\
14.26	0\\
14.27	0\\
14.28	0\\
14.29	0\\
14.3	0\\
14.31	0\\
14.32	0\\
14.33	0\\
14.34	0\\
14.35	0\\
14.36	0\\
14.37	0\\
14.38	0\\
14.39	0\\
14.4	0\\
14.41	0\\
14.42	0\\
14.43	0\\
14.44	0\\
14.45	0\\
14.46	0\\
14.47	0\\
14.48	0\\
14.49	0\\
14.5	0\\
14.51	0\\
14.52	0\\
14.53	0\\
14.54	0\\
14.55	0\\
14.56	0\\
14.57	0\\
14.58	0\\
14.59	0\\
14.6	0\\
14.61	0\\
14.62	0\\
14.63	0\\
14.64	0\\
14.65	0\\
14.66	0\\
14.67	0\\
14.68	0\\
14.69	0\\
14.7	0\\
14.71	0\\
14.72	0\\
14.73	0\\
14.74	0\\
14.75	0\\
14.76	0\\
14.77	0\\
14.78	0\\
14.79	0\\
14.8	0\\
14.81	0\\
14.82	0\\
14.83	0\\
14.84	0\\
14.85	0\\
14.86	0\\
14.87	0\\
14.88	0\\
14.89	0\\
14.9	0\\
14.91	0\\
14.92	0\\
14.93	0\\
14.94	0\\
14.95	0\\
14.96	0\\
14.97	0\\
14.98	0\\
14.99	0\\
15	0\\
15.01	0\\
15.02	0\\
15.03	0\\
15.04	0\\
15.05	0\\
15.06	0\\
15.07	0\\
15.08	0\\
15.09	0\\
15.1	0\\
15.11	0\\
15.12	0\\
15.13	0\\
15.14	0\\
15.15	0\\
15.16	0\\
15.17	0\\
15.18	0\\
15.19	0\\
15.2	0\\
15.21	0\\
15.22	0\\
15.23	0\\
15.24	0\\
15.25	0\\
15.26	0\\
15.27	0\\
15.28	0\\
15.29	0\\
15.3	0\\
15.31	0\\
15.32	0\\
15.33	0\\
15.34	0\\
15.35	0\\
15.36	0\\
15.37	0\\
15.38	0\\
15.39	0\\
15.4	0\\
15.41	0\\
15.42	0\\
15.43	0\\
15.44	0\\
15.45	0\\
15.46	0\\
15.47	0\\
15.48	0\\
15.49	0\\
15.5	0\\
15.51	0\\
15.52	0\\
15.53	0\\
15.54	0\\
15.55	0\\
15.56	0\\
15.57	0\\
15.58	0\\
15.59	0\\
15.6	0\\
15.61	0\\
15.62	0\\
15.63	0\\
15.64	0\\
15.65	0\\
15.66	0\\
15.67	0\\
15.68	0\\
15.69	0\\
15.7	0\\
15.71	0\\
15.72	0\\
15.73	0\\
15.74	0\\
15.75	0\\
15.76	0\\
15.77	0\\
15.78	0\\
15.79	0\\
15.8	0\\
15.81	0\\
15.82	0\\
15.83	0\\
15.84	0\\
15.85	0\\
15.86	0\\
15.87	0\\
15.88	0\\
15.89	0\\
15.9	0\\
15.91	0\\
15.92	0\\
15.93	0\\
15.94	0\\
15.95	0\\
15.96	0\\
15.97	0\\
15.98	0\\
15.99	0\\
16	0\\
16.01	0\\
16.02	0\\
16.03	0\\
16.04	0\\
16.05	0\\
16.06	0\\
16.07	0\\
16.08	0\\
16.09	0\\
16.1	0\\
16.11	0\\
16.12	0\\
16.13	0\\
16.14	0\\
16.15	0\\
16.16	0\\
16.17	0\\
16.18	0\\
16.19	0\\
16.2	0\\
16.21	0\\
16.22	0\\
16.23	0\\
16.24	0\\
16.25	0\\
16.26	0\\
16.27	0\\
16.28	0\\
16.29	0\\
16.3	0\\
16.31	0\\
16.32	0\\
16.33	0\\
16.34	0\\
16.35	0\\
16.36	0\\
16.37	0\\
16.38	0\\
16.39	0\\
16.4	0\\
16.41	0\\
16.42	0\\
16.43	0\\
16.44	0\\
16.45	0\\
16.46	0\\
16.47	0\\
16.48	0\\
16.49	0\\
16.5	0\\
16.51	0\\
16.52	0\\
16.53	0\\
16.54	0\\
16.55	0\\
16.56	0\\
16.57	0\\
16.58	0\\
16.59	0\\
16.6	0\\
16.61	0\\
16.62	0\\
16.63	0\\
16.64	0\\
16.65	0\\
16.66	0\\
16.67	0\\
16.68	0\\
16.69	0\\
16.7	0\\
16.71	0\\
16.72	0\\
16.73	0\\
16.74	0\\
16.75	0\\
16.76	0\\
16.77	0\\
16.78	0\\
16.79	0\\
16.8	0\\
16.81	0\\
16.82	0\\
16.83	0\\
16.84	0\\
16.85	0\\
16.86	0\\
16.87	0\\
16.88	0\\
16.89	0\\
16.9	0\\
16.91	0\\
16.92	0\\
16.93	0\\
16.94	0\\
16.95	0\\
16.96	0\\
16.97	0\\
16.98	0\\
16.99	0\\
17	0\\
17.01	0\\
17.02	0\\
17.03	0\\
17.04	0\\
17.05	0\\
17.06	0\\
17.07	0\\
17.08	0\\
17.09	0\\
17.1	0\\
17.11	0\\
17.12	0\\
17.13	0\\
17.14	0\\
17.15	0\\
17.16	0\\
17.17	0\\
17.18	0\\
17.19	0\\
17.2	0\\
17.21	0\\
17.22	0\\
17.23	0\\
17.24	0\\
17.25	0\\
17.26	0\\
17.27	0\\
17.28	0\\
17.29	0\\
17.3	0\\
17.31	0\\
17.32	0\\
17.33	0\\
17.34	0\\
17.35	0\\
17.36	0\\
17.37	0\\
17.38	0\\
17.39	0\\
17.4	0\\
17.41	0\\
17.42	0\\
17.43	0\\
17.44	0\\
17.45	0\\
17.46	0\\
17.47	0\\
17.48	0\\
17.49	0\\
17.5	0\\
17.51	0\\
17.52	0\\
17.53	0\\
17.54	0\\
17.55	0\\
17.56	0\\
17.57	0\\
17.58	0\\
17.59	0\\
17.6	0\\
17.61	0\\
17.62	0\\
17.63	0\\
17.64	0\\
17.65	0\\
17.66	0\\
17.67	0\\
17.68	0\\
17.69	0\\
17.7	0\\
17.71	0\\
17.72	0\\
17.73	0\\
17.74	0\\
17.75	0\\
17.76	0\\
17.77	0\\
17.78	0\\
17.79	0\\
17.8	0\\
17.81	0\\
17.82	0\\
17.83	0\\
17.84	0\\
17.85	0\\
17.86	0\\
17.87	0\\
17.88	0\\
17.89	0\\
17.9	0\\
17.91	0\\
17.92	0\\
17.93	0\\
17.94	0\\
17.95	0\\
17.96	0\\
17.97	0\\
17.98	0\\
17.99	0\\
18	0\\
18.01	0\\
18.02	0\\
18.03	0\\
18.04	0\\
18.05	0\\
18.06	0\\
18.07	0\\
18.08	0\\
18.09	0\\
18.1	0\\
18.11	0\\
18.12	0\\
18.13	0\\
18.14	0\\
18.15	0\\
18.16	0\\
18.17	0\\
18.18	0\\
18.19	0\\
18.2	0\\
18.21	0\\
18.22	0\\
18.23	0\\
18.24	0\\
18.25	0\\
18.26	0\\
18.27	0\\
18.28	0\\
18.29	0\\
18.3	0\\
18.31	0\\
18.32	0\\
18.33	0\\
18.34	0\\
18.35	0\\
18.36	0\\
18.37	0\\
18.38	0\\
18.39	0\\
18.4	0\\
18.41	0\\
18.42	0\\
18.43	0\\
18.44	0\\
18.45	0\\
18.46	0\\
18.47	0\\
18.48	0\\
18.49	0\\
18.5	0\\
18.51	0\\
18.52	0\\
18.53	0\\
18.54	0\\
18.55	0\\
18.56	0\\
18.57	0\\
18.58	0\\
18.59	0\\
18.6	0\\
18.61	0\\
18.62	0\\
18.63	0\\
18.64	0\\
18.65	0\\
18.66	0\\
18.67	0\\
18.68	0\\
18.69	0\\
18.7	0\\
18.71	0\\
18.72	0\\
18.73	0\\
18.74	0\\
18.75	0\\
18.76	0\\
18.77	0\\
18.78	0\\
18.79	0\\
18.8	0\\
18.81	0\\
18.82	0\\
18.83	0\\
18.84	0\\
18.85	0\\
18.86	0\\
18.87	0\\
18.88	0\\
18.89	0\\
18.9	0\\
18.91	0\\
18.92	0\\
18.93	0\\
18.94	0\\
18.95	0\\
18.96	0\\
18.97	0\\
18.98	0\\
18.99	0\\
19	0\\
19.01	0\\
19.02	0\\
19.03	0\\
19.04	0\\
19.05	0\\
19.06	0\\
19.07	0\\
19.08	0\\
19.09	0\\
19.1	0\\
19.11	0\\
19.12	0\\
19.13	0\\
19.14	0\\
19.15	0\\
19.16	0\\
19.17	0\\
19.18	0\\
19.19	0\\
19.2	0\\
19.21	0\\
19.22	0\\
19.23	0\\
19.24	0\\
19.25	0\\
19.26	0\\
19.27	0\\
19.28	0\\
19.29	0\\
19.3	0\\
19.31	0\\
19.32	0\\
19.33	0\\
19.34	0\\
19.35	0\\
19.36	0\\
19.37	0\\
19.38	0\\
19.39	0\\
19.4	0\\
19.41	0\\
19.42	0\\
19.43	0\\
19.44	0\\
19.45	0\\
19.46	0\\
19.47	0\\
19.48	0\\
19.49	0\\
19.5	0\\
19.51	0\\
19.52	0\\
19.53	0\\
19.54	0\\
19.55	0\\
19.56	0\\
19.57	0\\
19.58	0\\
19.59	0\\
19.6	0\\
19.61	0\\
19.62	0\\
19.63	0\\
19.64	0\\
19.65	0\\
19.66	0\\
19.67	0\\
19.68	0\\
19.69	0\\
19.7	0\\
19.71	0\\
19.72	0\\
19.73	0\\
19.74	0\\
19.75	0\\
19.76	0\\
19.77	0\\
19.78	0\\
19.79	0\\
19.8	0\\
19.81	0\\
19.82	0\\
19.83	0\\
19.84	0\\
19.85	0\\
19.86	0\\
19.87	0\\
19.88	0\\
19.89	0\\
19.9	0\\
19.91	0\\
19.92	0\\
19.93	0\\
19.94	0\\
19.95	0\\
19.96	0\\
19.97	0\\
19.98	0\\
19.99	0\\
20	0\\
20.01	0\\
20.02	0\\
20.03	0\\
20.04	0\\
20.05	0\\
20.06	0\\
20.07	0\\
20.08	0\\
20.09	0\\
20.1	0\\
20.11	0\\
20.12	0\\
20.13	0\\
20.14	0\\
20.15	0\\
20.16	0\\
20.17	0\\
20.18	0\\
20.19	0\\
20.2	0\\
20.21	0\\
20.22	0\\
20.23	0\\
20.24	0\\
20.25	0\\
20.26	0\\
20.27	0\\
20.28	0\\
20.29	0\\
20.3	0\\
20.31	0\\
20.32	0\\
20.33	0\\
20.34	0\\
20.35	0\\
20.36	0\\
20.37	0\\
20.38	0\\
20.39	0\\
20.4	0\\
20.41	0\\
20.42	0\\
20.43	0\\
20.44	0\\
20.45	0\\
20.46	0\\
20.47	0\\
20.48	0\\
20.49	0\\
20.5	0\\
20.51	0\\
20.52	0\\
20.53	0\\
20.54	0\\
20.55	0\\
20.56	0\\
20.57	0\\
20.58	0\\
20.59	0\\
20.6	0\\
20.61	0\\
20.62	0\\
20.63	0\\
20.64	0\\
20.65	0\\
20.66	0\\
20.67	0\\
20.68	0\\
20.69	0\\
20.7	0\\
20.71	0\\
20.72	0\\
20.73	0\\
20.74	0\\
20.75	0\\
20.76	0\\
20.77	0\\
20.78	0\\
20.79	0\\
20.8	0\\
20.81	0\\
20.82	0\\
20.83	0\\
20.84	0\\
20.85	0\\
20.86	0\\
20.87	0\\
20.88	0\\
20.89	0\\
20.9	0\\
20.91	0\\
20.92	0\\
20.93	0\\
20.94	0\\
20.95	0\\
20.96	0\\
20.97	0\\
20.98	0\\
20.99	0\\
21	0\\
21.01	0\\
21.02	0\\
21.03	0\\
21.04	0\\
21.05	0\\
21.06	0\\
21.07	0\\
21.08	0\\
21.09	0\\
21.1	0\\
21.11	0\\
21.12	0\\
21.13	0\\
21.14	0\\
21.15	0\\
21.16	0\\
21.17	0\\
21.18	0\\
21.19	0\\
21.2	0\\
21.21	0\\
21.22	0\\
21.23	0\\
21.24	0\\
21.25	0\\
21.26	0\\
21.27	0\\
21.28	0\\
21.29	0\\
21.3	0\\
21.31	0\\
21.32	0\\
21.33	0\\
21.34	0\\
21.35	0\\
21.36	0\\
21.37	0\\
21.38	0\\
21.39	0\\
21.4	0\\
21.41	0\\
21.42	0\\
21.43	0\\
21.44	0\\
21.45	0\\
21.46	0\\
21.47	0\\
21.48	0\\
21.49	0\\
21.5	0\\
21.51	0\\
21.52	0\\
21.53	0\\
21.54	0\\
21.55	0\\
21.56	0\\
21.57	0\\
21.58	0\\
21.59	0\\
21.6	0\\
21.61	0\\
21.62	0\\
21.63	0\\
21.64	0\\
21.65	0\\
21.66	0\\
21.67	0\\
21.68	0\\
21.69	0\\
21.7	0\\
21.71	0\\
21.72	0\\
21.73	0\\
21.74	0\\
21.75	0\\
21.76	0\\
21.77	0\\
21.78	0\\
21.79	0\\
21.8	0\\
21.81	0\\
21.82	0\\
21.83	0\\
21.84	0\\
21.85	0\\
21.86	0\\
21.87	0\\
21.88	0\\
21.89	0\\
21.9	0\\
21.91	0\\
21.92	0\\
21.93	0\\
21.94	0\\
21.95	0\\
21.96	0\\
21.97	0\\
21.98	0\\
21.99	0\\
22	0\\
22.01	0\\
22.02	0\\
22.03	0\\
22.04	0\\
22.05	0\\
22.06	0\\
22.07	0\\
22.08	0\\
22.09	0\\
22.1	0\\
22.11	0\\
22.12	0\\
22.13	0\\
22.14	0\\
22.15	0\\
22.16	0\\
22.17	0\\
22.18	0\\
22.19	0\\
22.2	0\\
22.21	0\\
22.22	0\\
22.23	0\\
22.24	0\\
22.25	0\\
22.26	0\\
22.27	0\\
22.28	0\\
22.29	0\\
22.3	0\\
22.31	0\\
22.32	0\\
22.33	0\\
22.34	0\\
22.35	0\\
22.36	0\\
22.37	0\\
22.38	0\\
22.39	0\\
22.4	0\\
22.41	0\\
22.42	0\\
22.43	0\\
22.44	0\\
22.45	0\\
22.46	0\\
22.47	0\\
22.48	0\\
22.49	0\\
22.5	0\\
22.51	0\\
22.52	0\\
22.53	0\\
22.54	0\\
22.55	0\\
22.56	0\\
22.57	0\\
22.58	0\\
22.59	0\\
22.6	0\\
22.61	0\\
22.62	0\\
22.63	0\\
22.64	0\\
22.65	0\\
22.66	0\\
22.67	0\\
22.68	0\\
22.69	0\\
22.7	0\\
22.71	0\\
22.72	0\\
22.73	0\\
22.74	0\\
22.75	0\\
22.76	0\\
22.77	0\\
22.78	0\\
22.79	0\\
22.8	0\\
22.81	0\\
22.82	0\\
22.83	0\\
22.84	0\\
22.85	0\\
22.86	0\\
22.87	0\\
22.88	0\\
22.89	0\\
22.9	0\\
22.91	0\\
22.92	0\\
22.93	0\\
22.94	0\\
22.95	0\\
22.96	0\\
22.97	0\\
22.98	0\\
22.99	0\\
23	0\\
23.01	0\\
23.02	0\\
23.03	0\\
23.04	0\\
23.05	0\\
23.06	0\\
23.07	0\\
23.08	0\\
23.09	0\\
23.1	0\\
23.11	0\\
23.12	0\\
23.13	0\\
23.14	0\\
23.15	0\\
23.16	0\\
23.17	0\\
23.18	0\\
23.19	0\\
23.2	0\\
23.21	0\\
23.22	0\\
23.23	0\\
23.24	0\\
23.25	0\\
23.26	0\\
23.27	0\\
23.28	0\\
23.29	0\\
23.3	0\\
23.31	0\\
23.32	0\\
23.33	0\\
23.34	0\\
23.35	0\\
23.36	0\\
23.37	0\\
23.38	0\\
23.39	0\\
23.4	0\\
23.41	0\\
23.42	0\\
23.43	0\\
23.44	0\\
23.45	0\\
23.46	0\\
23.47	0\\
23.48	0\\
23.49	0\\
23.5	0\\
23.51	0\\
23.52	0\\
23.53	0\\
23.54	0\\
23.55	0\\
23.56	0\\
23.57	0\\
23.58	0\\
23.59	0\\
23.6	0\\
23.61	0\\
23.62	0\\
23.63	0\\
23.64	0\\
23.65	0\\
23.66	0\\
23.67	0\\
23.68	0\\
23.69	0\\
23.7	0\\
23.71	0\\
23.72	0\\
23.73	0\\
23.74	0\\
23.75	0\\
23.76	0\\
23.77	0\\
23.78	0\\
23.79	0\\
23.8	0\\
23.81	0\\
23.82	0\\
23.83	0\\
23.84	0\\
23.85	0\\
23.86	0\\
23.87	0\\
23.88	0\\
23.89	0\\
23.9	0\\
23.91	0\\
23.92	0\\
23.93	0\\
23.94	0\\
23.95	0\\
23.96	0\\
23.97	0\\
23.98	0\\
23.99	0\\
24	0\\
24.01	0\\
24.02	0\\
24.03	0\\
24.04	0\\
24.05	0\\
24.06	0\\
24.07	0\\
24.08	0\\
24.09	0\\
24.1	0\\
24.11	0\\
24.12	0\\
24.13	0\\
24.14	0\\
24.15	0\\
24.16	0\\
24.17	0\\
24.18	0\\
24.19	0\\
24.2	0\\
24.21	0\\
24.22	0\\
24.23	0\\
24.24	0\\
24.25	0\\
24.26	0\\
24.27	0\\
24.28	0\\
24.29	0\\
24.3	0\\
24.31	0\\
24.32	0\\
24.33	0\\
24.34	0\\
24.35	0\\
24.36	0\\
24.37	0\\
24.38	0\\
24.39	0\\
24.4	0\\
24.41	0\\
24.42	0\\
24.43	0\\
24.44	0\\
24.45	0\\
24.46	0\\
24.47	0\\
24.48	0\\
24.49	0\\
24.5	0\\
24.51	0\\
24.52	0\\
24.53	0\\
24.54	0\\
24.55	0\\
24.56	0\\
24.57	0\\
24.58	0\\
24.59	0\\
24.6	0\\
24.61	0\\
24.62	0\\
24.63	0\\
24.64	0\\
24.65	0\\
24.66	0\\
24.67	0\\
24.68	0\\
24.69	0\\
24.7	0\\
24.71	0\\
24.72	0\\
24.73	0\\
24.74	0\\
24.75	0\\
24.76	0\\
24.77	0\\
24.78	0\\
24.79	0\\
24.8	0\\
24.81	0\\
24.82	0\\
24.83	0\\
24.84	0\\
24.85	0\\
24.86	0\\
24.87	0\\
24.88	0\\
24.89	0\\
24.9	0\\
24.91	0\\
24.92	0\\
24.93	0\\
24.94	0\\
24.95	0\\
24.96	0\\
24.97	0\\
24.98	0\\
24.99	0\\
25	0\\
25.01	0\\
25.02	0\\
25.03	0\\
25.04	0\\
25.05	0\\
25.06	0\\
25.07	0\\
25.08	0\\
25.09	0\\
25.1	0\\
25.11	0\\
25.12	0\\
25.13	0\\
25.14	0\\
25.15	0\\
25.16	0\\
25.17	0\\
25.18	0\\
25.19	0\\
25.2	0\\
25.21	0\\
25.22	0\\
25.23	0\\
25.24	0\\
25.25	0\\
25.26	0\\
25.27	0\\
25.28	0\\
25.29	0\\
25.3	0\\
25.31	0\\
25.32	0\\
25.33	0\\
25.34	0\\
25.35	0\\
25.36	0\\
25.37	0\\
25.38	0\\
25.39	0\\
25.4	0\\
25.41	0\\
25.42	0\\
25.43	0\\
25.44	0\\
25.45	0\\
25.46	0\\
25.47	0\\
25.48	0\\
25.49	0\\
25.5	0\\
25.51	0\\
25.52	0\\
25.53	0\\
25.54	0\\
25.55	0\\
25.56	0\\
25.57	0\\
25.58	0\\
25.59	0\\
25.6	0\\
25.61	0\\
25.62	0\\
25.63	0\\
25.64	0\\
25.65	0\\
25.66	0\\
25.67	0\\
25.68	0\\
25.69	0\\
25.7	0\\
25.71	0\\
25.72	0\\
25.73	0\\
25.74	0\\
25.75	0\\
25.76	0\\
25.77	0\\
25.78	0\\
25.79	0\\
25.8	0\\
25.81	0\\
25.82	0\\
25.83	0\\
25.84	0\\
25.85	0\\
25.86	0\\
25.87	0\\
25.88	0\\
25.89	0\\
25.9	0\\
25.91	0\\
25.92	0\\
25.93	0\\
25.94	0\\
25.95	0\\
25.96	0\\
25.97	0\\
25.98	0\\
25.99	0\\
26	0\\
26.01	0\\
26.02	0\\
26.03	0\\
26.04	0\\
26.05	0\\
26.06	0\\
26.07	0\\
26.08	0\\
26.09	0\\
26.1	0\\
26.11	0\\
26.12	0\\
26.13	0\\
26.14	0\\
26.15	0\\
26.16	0\\
26.17	0\\
26.18	0\\
26.19	0\\
26.2	0\\
26.21	0\\
26.22	0\\
26.23	0\\
26.24	0\\
26.25	0\\
26.26	0\\
26.27	0\\
26.28	0\\
26.29	0\\
26.3	0\\
26.31	0\\
26.32	0\\
26.33	0\\
26.34	0\\
26.35	0\\
26.36	0\\
26.37	0\\
26.38	0\\
26.39	0\\
26.4	0\\
26.41	0\\
26.42	0\\
26.43	0\\
26.44	0\\
26.45	0\\
26.46	0\\
26.47	0\\
26.48	0\\
26.49	0\\
26.5	0\\
26.51	0\\
26.52	0\\
26.53	0\\
26.54	0\\
26.55	0\\
26.56	0\\
26.57	0\\
26.58	0\\
26.59	0\\
26.6	0\\
26.61	0\\
26.62	0\\
26.63	0\\
26.64	0\\
26.65	0\\
26.66	0\\
26.67	0\\
26.68	0\\
26.69	0\\
26.7	0\\
26.71	0\\
26.72	0\\
26.73	0\\
26.74	0\\
26.75	0\\
26.76	0\\
26.77	0\\
26.78	0\\
26.79	0\\
26.8	0\\
26.81	0\\
26.82	0\\
26.83	0\\
26.84	0\\
26.85	0\\
26.86	0\\
26.87	0\\
26.88	0\\
26.89	0\\
26.9	0\\
26.91	0\\
26.92	0\\
26.93	0\\
26.94	0\\
26.95	0\\
26.96	0\\
26.97	0\\
26.98	0\\
26.99	0\\
27	0\\
27.01	0\\
27.02	0\\
27.03	0\\
27.04	0\\
27.05	0\\
27.06	0\\
27.07	0\\
27.08	0\\
27.09	0\\
27.1	0\\
27.11	0\\
27.12	0\\
27.13	0\\
27.14	0\\
27.15	0\\
27.16	0\\
27.17	0\\
27.18	0\\
27.19	0\\
27.2	0\\
27.21	0\\
27.22	0\\
27.23	0\\
27.24	0\\
27.25	0\\
27.26	0\\
27.27	0\\
27.28	0\\
27.29	0\\
27.3	0\\
27.31	0\\
27.32	0\\
27.33	0\\
27.34	0\\
27.35	0\\
27.36	0\\
27.37	0\\
27.38	0\\
27.39	0\\
27.4	0\\
27.41	0\\
27.42	0\\
27.43	0\\
27.44	0\\
27.45	0\\
27.46	0\\
27.47	0\\
27.48	0\\
27.49	0\\
27.5	0\\
27.51	0\\
27.52	0\\
27.53	0\\
27.54	0\\
27.55	0\\
27.56	0\\
27.57	0\\
27.58	0\\
27.59	0\\
27.6	0\\
27.61	0\\
27.62	0\\
27.63	0\\
27.64	0\\
27.65	0\\
27.66	0\\
27.67	0\\
27.68	0\\
27.69	0\\
27.7	0\\
27.71	0\\
27.72	0\\
27.73	0\\
27.74	0\\
27.75	0\\
27.76	0\\
27.77	0\\
27.78	0\\
27.79	0\\
27.8	0\\
27.81	0\\
27.82	0\\
27.83	0\\
27.84	0\\
27.85	0\\
27.86	0\\
27.87	0\\
27.88	0\\
27.89	0\\
27.9	0\\
27.91	0\\
27.92	0\\
27.93	0\\
27.94	0\\
27.95	0\\
27.96	0\\
27.97	0\\
27.98	0\\
27.99	0\\
28	0\\
28.01	0\\
28.02	0\\
28.03	0\\
28.04	0\\
28.05	0\\
28.06	0\\
28.07	0\\
28.08	0\\
28.09	0\\
28.1	0\\
28.11	0\\
28.12	0\\
28.13	0\\
28.14	0\\
28.15	0\\
28.16	0\\
28.17	0\\
28.18	0\\
28.19	0\\
28.2	0\\
28.21	0\\
28.22	0\\
28.23	0\\
28.24	0\\
28.25	0\\
28.26	0\\
28.27	0\\
28.28	0\\
28.29	0\\
28.3	0\\
28.31	0\\
28.32	0\\
28.33	0\\
28.34	0\\
28.35	0\\
28.36	0\\
28.37	0\\
28.38	0\\
28.39	0\\
28.4	0\\
28.41	0\\
28.42	0\\
28.43	0\\
28.44	0\\
28.45	0\\
28.46	0\\
28.47	0\\
28.48	0\\
28.49	0\\
28.5	0\\
28.51	0\\
28.52	0\\
28.53	0\\
28.54	0\\
28.55	0\\
28.56	0\\
28.57	0\\
28.58	0\\
28.59	0\\
28.6	0\\
28.61	0\\
28.62	0\\
28.63	0\\
28.64	0\\
28.65	0\\
28.66	0\\
28.67	0\\
28.68	0\\
28.69	0\\
28.7	0\\
28.71	0\\
28.72	0\\
28.73	0\\
28.74	0\\
28.75	0\\
28.76	0\\
28.77	0\\
28.78	0\\
28.79	0\\
28.8	0\\
28.81	0\\
28.82	0\\
28.83	0\\
28.84	0\\
28.85	0\\
28.86	0\\
28.87	0\\
28.88	0\\
28.89	0\\
28.9	0\\
28.91	0\\
28.92	0\\
28.93	0\\
28.94	0\\
28.95	0\\
28.96	0\\
28.97	0\\
28.98	0\\
28.99	0\\
29	0\\
29.01	0\\
29.02	0\\
29.03	0\\
29.04	0\\
29.05	0\\
29.06	0\\
29.07	0\\
29.08	0\\
29.09	0\\
29.1	0\\
29.11	0\\
29.12	0\\
29.13	0\\
29.14	0\\
29.15	0\\
29.16	0\\
29.17	0\\
29.18	0\\
29.19	0\\
29.2	0\\
29.21	0\\
29.22	0\\
29.23	0\\
29.24	0\\
29.25	0\\
29.26	0\\
29.27	0\\
29.28	0\\
29.29	0\\
29.3	0\\
29.31	0\\
29.32	0\\
29.33	0\\
29.34	0\\
29.35	0\\
29.36	0\\
29.37	0\\
29.38	0\\
29.39	0\\
29.4	0\\
29.41	0\\
29.42	0\\
29.43	0\\
29.44	0\\
29.45	0\\
29.46	0\\
29.47	0\\
29.48	0\\
29.49	0\\
29.5	0\\
29.51	0\\
29.52	0\\
29.53	0\\
29.54	0\\
29.55	0\\
29.56	0\\
29.57	0\\
29.58	0\\
29.59	0\\
29.6	0\\
29.61	0\\
29.62	0\\
29.63	0\\
29.64	0\\
29.65	0\\
29.66	0\\
29.67	0\\
29.68	0\\
29.69	0\\
29.7	0\\
29.71	0\\
29.72	0\\
29.73	0\\
29.74	0\\
29.75	0\\
29.76	0\\
29.77	0\\
29.78	0\\
29.79	0\\
29.8	0\\
29.81	0\\
29.82	0\\
29.83	0\\
29.84	0\\
29.85	0\\
29.86	0\\
29.87	0\\
29.88	0\\
29.89	0\\
29.9	0\\
29.91	0\\
29.92	0\\
29.93	0\\
29.94	0\\
29.95	0\\
29.96	0\\
29.97	0\\
29.98	0\\
29.99	0\\
30	0\\
30.01	0\\
30.02	0\\
30.03	0\\
30.04	0\\
30.05	0\\
30.06	0\\
30.07	0\\
30.08	0\\
30.09	0\\
30.1	0\\
30.11	0\\
30.12	0\\
30.13	0\\
30.14	0\\
30.15	0\\
30.16	0\\
30.17	0\\
30.18	0\\
30.19	0\\
30.2	0\\
30.21	0\\
30.22	0\\
30.23	0\\
30.24	0\\
30.25	0\\
30.26	0\\
30.27	0\\
30.28	0\\
30.29	0\\
30.3	0\\
30.31	0\\
30.32	0\\
30.33	0\\
30.34	0\\
30.35	0\\
30.36	0\\
30.37	0\\
30.38	0\\
30.39	0\\
30.4	0\\
30.41	0\\
30.42	0\\
30.43	0\\
30.44	0\\
30.45	0\\
30.46	0\\
30.47	0\\
30.48	0\\
30.49	0\\
30.5	0\\
30.51	0\\
30.52	0\\
30.53	0\\
30.54	0\\
30.55	0\\
30.56	0\\
30.57	0\\
30.58	0\\
30.59	0\\
30.6	0\\
30.61	0\\
30.62	0\\
30.63	0\\
30.64	0\\
30.65	0\\
30.66	0\\
30.67	0\\
30.68	0\\
30.69	0\\
30.7	0\\
30.71	0\\
30.72	0\\
30.73	0\\
30.74	0\\
30.75	0\\
30.76	0\\
30.77	0\\
30.78	0\\
30.79	0\\
30.8	0\\
30.81	0\\
30.82	0\\
30.83	0\\
30.84	0\\
30.85	0\\
30.86	0\\
30.87	0\\
30.88	0\\
30.89	0\\
30.9	0\\
30.91	0\\
30.92	0\\
30.93	0\\
30.94	0\\
30.95	0\\
30.96	0\\
30.97	0\\
30.98	0\\
30.99	0\\
31	0\\
31.01	0\\
31.02	0\\
31.03	0\\
31.04	0\\
31.05	0\\
31.06	0\\
31.07	0\\
31.08	0\\
31.09	0\\
31.1	0\\
31.11	0\\
31.12	0\\
31.13	0\\
31.14	0\\
31.15	0\\
31.16	0\\
31.17	0\\
31.18	0\\
31.19	0\\
31.2	0\\
31.21	0\\
31.22	0\\
31.23	0\\
31.24	0\\
31.25	0\\
31.26	0\\
31.27	0\\
31.28	0\\
31.29	0\\
31.3	0\\
31.31	0\\
31.32	0\\
31.33	0\\
31.34	0\\
31.35	0\\
31.36	0\\
31.37	0\\
31.38	0\\
31.39	0\\
31.4	0\\
31.41	0\\
31.42	0\\
31.43	0\\
31.44	0\\
31.45	0\\
31.46	0\\
31.47	0\\
31.48	0\\
31.49	0\\
31.5	0\\
31.51	0\\
31.52	0\\
31.53	0\\
31.54	0\\
31.55	0\\
31.56	0\\
31.57	0\\
31.58	0\\
31.59	0\\
31.6	0\\
31.61	0\\
31.62	0\\
31.63	0\\
31.64	0\\
31.65	0\\
31.66	0\\
31.67	0\\
31.68	0\\
31.69	0\\
31.7	0\\
31.71	0\\
31.72	0\\
31.73	0\\
31.74	0\\
31.75	0\\
31.76	0\\
31.77	0\\
31.78	0\\
31.79	0\\
31.8	0\\
31.81	0\\
31.82	0\\
31.83	0\\
31.84	0\\
31.85	0\\
31.86	0\\
31.87	0\\
31.88	0\\
31.89	0\\
31.9	0\\
31.91	0\\
31.92	0\\
31.93	0\\
31.94	0\\
31.95	0\\
31.96	0\\
31.97	0\\
31.98	0\\
31.99	0\\
32	0\\
32.01	0\\
32.02	0\\
32.03	0\\
32.04	0\\
32.05	0\\
32.06	0\\
32.07	0\\
32.08	0\\
32.09	0\\
32.1	0\\
32.11	0\\
32.12	0\\
32.13	0\\
32.14	0\\
32.15	0\\
32.16	0\\
32.17	0\\
32.18	0\\
32.19	0\\
32.2	0\\
32.21	0\\
32.22	0\\
32.23	0\\
32.24	0\\
32.25	0\\
32.26	0\\
32.27	0\\
32.28	0\\
32.29	0\\
32.3	0\\
32.31	0\\
32.32	0\\
32.33	0\\
32.34	0\\
32.35	0\\
32.36	0\\
32.37	0\\
32.38	0\\
32.39	0\\
32.4	0\\
32.41	0\\
32.42	0\\
32.43	0\\
32.44	0\\
32.45	0\\
32.46	0\\
32.47	0\\
32.48	0\\
32.49	0\\
32.5	0\\
32.51	0\\
32.52	0\\
32.53	0\\
32.54	0\\
32.55	0\\
32.56	0\\
32.57	0\\
32.58	0\\
32.59	0\\
32.6	0\\
32.61	0\\
32.62	0\\
32.63	0\\
32.64	0\\
32.65	0\\
32.66	0\\
32.67	0\\
32.68	0\\
32.69	0\\
32.7	0\\
32.71	0\\
32.72	0\\
32.73	0\\
32.74	0\\
32.75	0\\
32.76	0\\
32.77	0\\
32.78	0\\
32.79	0\\
32.8	0\\
32.81	0\\
32.82	0\\
32.83	0\\
32.84	0\\
32.85	0\\
32.86	0\\
32.87	0\\
32.88	0\\
32.89	0\\
32.9	0\\
32.91	0\\
32.92	0\\
32.93	0\\
32.94	0\\
32.95	0\\
32.96	0\\
32.97	0\\
32.98	0\\
32.99	0\\
33	0\\
33.01	0\\
33.02	0\\
33.03	0\\
33.04	0\\
33.05	0\\
33.06	0\\
33.07	0\\
33.08	0\\
33.09	0\\
33.1	0\\
33.11	0\\
33.12	0\\
33.13	0\\
33.14	0\\
33.15	0\\
33.16	0\\
33.17	0\\
33.18	0\\
33.19	0\\
33.2	0\\
33.21	0\\
33.22	0\\
33.23	0\\
33.24	0\\
33.25	0\\
33.26	0\\
33.27	0\\
33.28	0\\
33.29	0\\
33.3	0\\
33.31	0\\
33.32	0\\
33.33	0\\
33.34	0\\
33.35	0\\
33.36	0\\
33.37	0\\
33.38	0\\
33.39	0\\
33.4	0\\
33.41	0\\
33.42	0\\
33.43	0\\
33.44	0\\
33.45	0\\
33.46	0\\
33.47	0\\
33.48	0\\
33.49	0\\
33.5	0\\
33.51	0\\
33.52	0\\
33.53	0\\
33.54	0\\
33.55	0\\
33.56	0\\
33.57	0\\
33.58	0\\
33.59	0\\
33.6	0\\
33.61	0\\
33.62	0\\
33.63	0\\
33.64	0\\
33.65	0\\
33.66	0\\
33.67	0\\
33.68	0\\
33.69	0\\
33.7	0\\
33.71	0\\
33.72	0\\
33.73	0\\
33.74	0\\
33.75	0\\
33.76	0\\
33.77	0\\
33.78	0\\
33.79	0\\
33.8	0\\
33.81	0\\
33.82	0\\
33.83	0\\
33.84	0\\
33.85	0\\
33.86	0\\
33.87	0\\
33.88	0\\
33.89	0\\
33.9	0\\
33.91	0\\
33.92	0\\
33.93	0\\
33.94	0\\
33.95	0\\
33.96	0\\
33.97	0\\
33.98	0\\
33.99	0\\
34	0\\
34.01	0\\
34.02	0\\
34.03	0\\
34.04	0\\
34.05	0\\
34.06	0\\
34.07	0\\
34.08	0\\
34.09	0\\
34.1	0\\
34.11	0\\
34.12	0\\
34.13	0\\
34.14	0\\
34.15	0\\
34.16	0\\
34.17	0\\
34.18	0\\
34.19	0\\
34.2	0\\
34.21	0\\
34.22	0\\
34.23	0\\
34.24	0\\
34.25	0\\
34.26	0\\
34.27	0\\
34.28	0\\
34.29	0\\
34.3	0\\
34.31	0\\
34.32	0\\
34.33	0\\
34.34	0\\
34.35	0\\
34.36	0\\
34.37	0\\
34.38	0\\
34.39	0\\
34.4	0\\
34.41	0\\
34.42	0\\
34.43	0\\
34.44	0\\
34.45	0\\
34.46	0\\
34.47	0\\
34.48	0\\
34.49	0\\
34.5	0\\
34.51	0\\
34.52	0\\
34.53	0\\
34.54	0\\
34.55	0\\
34.56	0\\
34.57	0\\
34.58	0\\
34.59	0\\
34.6	0\\
34.61	0\\
34.62	0\\
34.63	0\\
34.64	0\\
34.65	0\\
34.66	0\\
34.67	0\\
34.68	0\\
34.69	0\\
34.7	0\\
34.71	0\\
34.72	0\\
34.73	0\\
34.74	0\\
34.75	0\\
34.76	0\\
34.77	0\\
34.78	0\\
34.79	0\\
34.8	0\\
34.81	0\\
34.82	0\\
34.83	0\\
34.84	0\\
34.85	0\\
34.86	0\\
34.87	0\\
34.88	0\\
34.89	0\\
34.9	0\\
34.91	0\\
34.92	0\\
34.93	0\\
34.94	0\\
34.95	0\\
34.96	0\\
34.97	0\\
34.98	0\\
34.99	0\\
35	0\\
35.01	0\\
35.02	0\\
35.03	0\\
35.04	0\\
35.05	0\\
35.06	0\\
35.07	0\\
35.08	0\\
35.09	0\\
35.1	0\\
35.11	0\\
35.12	0\\
35.13	0\\
35.14	0\\
35.15	0\\
35.16	0\\
35.17	0\\
35.18	0\\
35.19	0\\
35.2	0\\
35.21	0\\
35.22	0\\
35.23	0\\
35.24	0\\
35.25	0\\
35.26	0\\
35.27	0\\
35.28	0\\
35.29	0\\
35.3	0\\
35.31	0\\
35.32	0\\
35.33	0\\
35.34	0\\
35.35	0\\
35.36	0\\
35.37	0\\
35.38	0\\
35.39	0\\
35.4	0\\
35.41	0\\
35.42	0\\
35.43	0\\
35.44	0\\
35.45	0\\
35.46	0\\
35.47	0\\
35.48	0\\
35.49	0\\
35.5	0\\
35.51	0\\
35.52	0\\
35.53	0\\
35.54	0\\
35.55	0\\
35.56	0\\
35.57	0\\
35.58	0\\
35.59	0\\
35.6	0\\
35.61	0\\
35.62	0\\
35.63	0\\
35.64	0\\
35.65	0\\
35.66	0\\
35.67	0\\
35.68	0\\
35.69	0\\
35.7	0\\
35.71	0\\
35.72	0\\
35.73	0\\
35.74	0\\
35.75	0\\
35.76	0\\
35.77	0\\
35.78	0\\
35.79	0\\
35.8	0\\
35.81	0\\
35.82	0\\
35.83	0\\
35.84	0\\
35.85	0\\
35.86	0\\
35.87	0\\
35.88	0\\
35.89	0\\
35.9	0\\
35.91	0\\
35.92	0\\
35.93	0\\
35.94	0\\
35.95	0\\
35.96	0\\
35.97	0\\
35.98	0\\
35.99	0\\
36	0\\
36.01	0\\
36.02	0\\
36.03	0\\
36.04	0\\
36.05	0\\
36.06	0\\
36.07	0\\
36.08	0\\
36.09	0\\
36.1	0\\
36.11	0\\
36.12	0\\
36.13	0\\
36.14	0\\
36.15	0\\
36.16	0\\
36.17	0\\
36.18	0\\
36.19	0\\
36.2	0\\
36.21	0\\
36.22	0\\
36.23	0\\
36.24	0\\
36.25	0\\
36.26	0\\
36.27	0\\
36.28	0\\
36.29	0\\
36.3	0\\
36.31	0\\
36.32	0\\
36.33	0\\
36.34	0\\
36.35	0\\
36.36	0\\
36.37	0\\
36.38	0\\
36.39	0\\
36.4	0\\
36.41	0\\
36.42	0\\
36.43	0\\
36.44	0\\
36.45	0\\
36.46	0\\
36.47	0\\
36.48	0\\
36.49	0\\
36.5	0\\
36.51	0\\
36.52	0\\
36.53	0\\
36.54	0\\
36.55	0\\
36.56	0\\
36.57	0\\
36.58	0\\
36.59	0\\
36.6	0\\
36.61	0\\
36.62	0\\
36.63	0\\
36.64	0\\
36.65	0\\
36.66	0\\
36.67	0\\
36.68	0\\
36.69	0\\
36.7	0\\
36.71	0\\
36.72	0\\
36.73	0\\
36.74	0\\
36.75	0\\
36.76	0\\
36.77	0\\
36.78	0\\
36.79	0\\
36.8	0\\
36.81	0\\
36.82	0\\
36.83	0\\
36.84	0\\
36.85	0\\
36.86	0\\
36.87	0\\
36.88	0\\
36.89	0\\
36.9	0\\
36.91	0\\
36.92	0\\
36.93	0\\
36.94	0\\
36.95	0\\
36.96	0\\
36.97	0\\
36.98	0\\
36.99	0\\
37	0\\
37.01	0\\
37.02	0\\
37.03	0\\
37.04	0\\
37.05	0\\
37.06	0\\
37.07	0\\
37.08	0\\
37.09	0\\
37.1	0\\
37.11	0\\
37.12	0\\
37.13	0\\
37.14	0\\
37.15	0\\
37.16	0\\
37.17	0\\
37.18	0\\
37.19	0\\
37.2	0\\
37.21	0\\
37.22	0\\
37.23	0\\
37.24	0\\
37.25	0\\
37.26	0\\
37.27	0\\
37.28	0\\
37.29	0\\
37.3	0\\
37.31	0\\
37.32	0\\
37.33	0\\
37.34	0\\
37.35	0\\
37.36	0\\
37.37	0\\
37.38	0\\
37.39	0\\
37.4	0\\
37.41	0\\
37.42	0\\
37.43	0\\
37.44	0\\
37.45	0\\
37.46	0\\
37.47	0\\
37.48	0\\
37.49	0\\
37.5	0\\
37.51	0\\
37.52	0\\
37.53	0\\
37.54	0\\
37.55	0\\
37.56	0\\
37.57	0\\
37.58	0\\
37.59	0\\
37.6	0\\
37.61	0\\
37.62	0\\
37.63	0\\
37.64	0\\
37.65	0\\
37.66	0\\
37.67	0\\
37.68	0\\
37.69	0\\
37.7	0\\
37.71	0\\
37.72	0\\
37.73	0\\
37.74	0\\
37.75	0\\
37.76	0\\
37.77	0\\
37.78	0\\
37.79	0\\
37.8	0\\
37.81	0\\
37.82	0\\
37.83	0\\
37.84	0\\
37.85	0\\
37.86	0\\
37.87	0\\
37.88	0\\
37.89	0\\
37.9	0\\
37.91	0\\
37.92	0\\
37.93	0\\
37.94	0\\
37.95	0\\
37.96	0\\
37.97	0\\
37.98	0\\
37.99	0\\
38	0\\
38.01	0\\
38.02	0\\
38.03	0\\
38.04	0\\
38.05	0\\
38.06	0\\
38.07	0\\
38.08	0\\
38.09	0\\
38.1	0\\
38.11	0\\
38.12	0\\
38.13	0\\
38.14	0\\
38.15	0\\
38.16	0\\
38.17	0\\
38.18	0\\
38.19	0\\
38.2	0\\
38.21	0\\
38.22	0\\
38.23	0\\
38.24	0\\
38.25	0\\
38.26	0\\
38.27	0\\
38.28	0\\
38.29	0\\
38.3	0\\
38.31	0\\
38.32	0\\
38.33	0\\
38.34	0\\
38.35	0\\
38.36	0\\
38.37	0\\
38.38	0\\
38.39	0\\
38.4	0\\
38.41	0\\
38.42	0\\
38.43	0\\
38.44	0\\
38.45	0\\
38.46	0\\
38.47	0\\
38.48	0\\
38.49	0\\
38.5	0\\
38.51	0\\
38.52	0\\
38.53	0\\
38.54	0\\
38.55	0\\
38.56	0\\
38.57	0\\
38.58	0\\
38.59	0\\
38.6	0\\
38.61	0\\
38.62	0\\
38.63	0\\
38.64	0\\
38.65	0\\
38.66	0\\
38.67	0\\
38.68	0\\
38.69	0\\
38.7	0\\
38.71	0\\
38.72	0\\
38.73	0\\
38.74	0\\
38.75	0\\
38.76	0\\
38.77	0\\
38.78	0\\
38.79	0\\
38.8	0\\
38.81	0\\
38.82	0\\
38.83	0\\
38.84	0\\
38.85	0\\
38.86	0\\
38.87	0\\
38.88	0\\
38.89	0\\
38.9	0\\
38.91	0\\
38.92	0\\
38.93	0\\
38.94	0\\
38.95	0\\
38.96	0\\
38.97	0\\
38.98	0\\
38.99	0\\
39	0\\
39.01	0\\
39.02	0\\
39.03	0\\
39.04	0\\
39.05	0\\
39.06	0\\
39.07	0\\
39.08	0\\
39.09	0\\
39.1	0\\
39.11	0\\
39.12	0\\
39.13	0\\
39.14	0\\
39.15	0\\
39.16	0\\
39.17	0\\
39.18	0\\
39.19	0\\
39.2	0\\
39.21	0\\
39.22	0\\
39.23	0\\
39.24	0\\
39.25	0\\
39.26	0\\
39.27	0\\
39.28	0\\
39.29	0\\
39.3	0\\
39.31	0\\
39.32	0\\
39.33	0\\
39.34	0\\
39.35	0\\
39.36	0\\
39.37	0\\
39.38	0\\
39.39	0\\
39.4	0\\
39.41	0\\
39.42	0\\
39.43	0\\
39.44	0\\
39.45	0\\
39.46	0\\
39.47	0\\
39.48	0\\
39.49	0\\
39.5	0\\
39.51	0\\
39.52	0\\
39.53	0\\
39.54	0\\
39.55	0\\
39.56	0\\
39.57	0\\
39.58	0\\
39.59	0\\
39.6	0\\
39.61	0\\
39.62	0\\
39.63	0\\
39.64	0\\
39.65	0\\
39.66	0\\
39.67	0\\
39.68	0\\
39.69	0\\
39.7	0\\
39.71	0\\
39.72	0\\
39.73	0\\
39.74	0\\
39.75	0\\
39.76	0\\
39.77	0\\
39.78	0\\
39.79	0\\
39.8	0\\
39.81	0\\
39.82	0\\
39.83	0\\
39.84	0\\
39.85	0\\
39.86	0\\
39.87	0\\
39.88	0\\
39.89	0\\
39.9	0\\
39.91	0\\
39.92	0\\
39.93	0\\
39.94	0\\
39.95	0\\
39.96	0\\
39.97	0\\
39.98	0\\
39.99	0\\
40	0\\
40.01	0\\
};
\addplot [color=blue,dashed,forget plot]
  table[row sep=crcr]{%
40.01	0\\
40.02	0\\
40.03	0\\
40.04	0\\
40.05	0\\
40.06	0\\
40.07	0\\
40.08	0\\
40.09	0\\
40.1	0\\
40.11	0\\
40.12	0\\
40.13	0\\
40.14	0\\
40.15	0\\
40.16	0\\
40.17	0\\
40.18	0\\
40.19	0\\
40.2	0\\
40.21	0\\
40.22	0\\
40.23	0\\
40.24	0\\
40.25	0\\
40.26	0\\
40.27	0\\
40.28	0\\
40.29	0\\
40.3	0\\
40.31	0\\
40.32	0\\
40.33	0\\
40.34	0\\
40.35	0\\
40.36	0\\
40.37	0\\
40.38	0\\
40.39	0\\
40.4	0\\
40.41	0\\
40.42	0\\
40.43	0\\
40.44	0\\
40.45	0\\
40.46	0\\
40.47	0\\
40.48	0\\
40.49	0\\
40.5	0\\
40.51	0\\
40.52	0\\
40.53	0\\
40.54	0\\
40.55	0\\
40.56	0\\
40.57	0\\
40.58	0\\
40.59	0\\
40.6	0\\
40.61	0\\
40.62	0\\
40.63	0\\
40.64	0\\
40.65	0\\
40.66	0\\
40.67	0\\
40.68	0\\
40.69	0\\
40.7	0\\
40.71	0\\
40.72	0\\
40.73	0\\
40.74	0\\
40.75	0\\
40.76	0\\
40.77	0\\
40.78	0\\
40.79	0\\
40.8	0\\
40.81	0\\
40.82	0\\
40.83	0\\
40.84	0\\
40.85	0\\
40.86	0\\
40.87	0\\
40.88	0\\
40.89	0\\
40.9	0\\
40.91	0\\
40.92	0\\
40.93	0\\
40.94	0\\
40.95	0\\
40.96	0\\
40.97	0\\
40.98	0\\
40.99	0\\
41	0\\
41.01	0\\
41.02	0\\
41.03	0\\
41.04	0\\
41.05	0\\
41.06	0\\
41.07	0\\
41.08	0\\
41.09	0\\
41.1	0\\
41.11	0\\
41.12	0\\
41.13	0\\
41.14	0\\
41.15	0\\
41.16	0\\
41.17	0\\
41.18	0\\
41.19	0\\
41.2	0\\
41.21	0\\
41.22	0\\
41.23	0\\
41.24	0\\
41.25	0\\
41.26	0\\
41.27	0\\
41.28	0\\
41.29	0\\
41.3	0\\
41.31	0\\
41.32	0\\
41.33	0\\
41.34	0\\
41.35	0\\
41.36	0\\
41.37	0\\
41.38	0\\
41.39	0\\
41.4	0\\
41.41	0\\
41.42	0\\
41.43	0\\
41.44	0\\
41.45	0\\
41.46	0\\
41.47	0\\
41.48	0\\
41.49	0\\
41.5	0\\
41.51	0\\
41.52	0\\
41.53	0\\
41.54	0\\
41.55	0\\
41.56	0\\
41.57	0\\
41.58	0\\
41.59	0\\
41.6	0\\
41.61	0\\
41.62	0\\
41.63	0\\
41.64	0\\
41.65	0\\
41.66	0\\
41.67	0\\
41.68	0\\
41.69	0\\
41.7	0\\
41.71	0\\
41.72	0\\
41.73	0\\
41.74	0\\
41.75	0\\
41.76	0\\
41.77	0\\
41.78	0\\
41.79	0\\
41.8	0\\
41.81	0\\
41.82	0\\
41.83	0\\
41.84	0\\
41.85	0\\
41.86	0\\
41.87	0\\
41.88	0\\
41.89	0\\
41.9	0\\
41.91	0\\
41.92	0\\
41.93	0\\
41.94	0\\
41.95	0\\
41.96	0\\
41.97	0\\
41.98	0\\
41.99	0\\
42	0\\
42.01	0\\
42.02	0\\
42.03	0\\
42.04	0\\
42.05	0\\
42.06	0\\
42.07	0\\
42.08	0\\
42.09	0\\
42.1	0\\
42.11	0\\
42.12	0\\
42.13	0\\
42.14	0\\
42.15	0\\
42.16	0\\
42.17	0\\
42.18	0\\
42.19	0\\
42.2	0\\
42.21	0\\
42.22	0\\
42.23	0\\
42.24	0\\
42.25	0\\
42.26	0\\
42.27	0\\
42.28	0\\
42.29	0\\
42.3	0\\
42.31	0\\
42.32	0\\
42.33	0\\
42.34	0\\
42.35	0\\
42.36	0\\
42.37	0\\
42.38	0\\
42.39	0\\
42.4	0\\
42.41	0\\
42.42	0\\
42.43	0\\
42.44	0\\
42.45	0\\
42.46	0\\
42.47	0\\
42.48	0\\
42.49	0\\
42.5	0\\
42.51	0\\
42.52	0\\
42.53	0\\
42.54	0\\
42.55	0\\
42.56	0\\
42.57	0\\
42.58	0\\
42.59	0\\
42.6	0\\
42.61	0\\
42.62	0\\
42.63	0\\
42.64	0\\
42.65	0\\
42.66	0\\
42.67	0\\
42.68	0\\
42.69	0\\
42.7	0\\
42.71	0\\
42.72	0\\
42.73	0\\
42.74	0\\
42.75	0\\
42.76	0\\
42.77	0\\
42.78	0\\
42.79	0\\
42.8	0\\
42.81	0\\
42.82	0\\
42.83	0\\
42.84	0\\
42.85	0\\
42.86	0\\
42.87	0\\
42.88	0\\
42.89	0\\
42.9	0\\
42.91	0\\
42.92	0\\
42.93	0\\
42.94	0\\
42.95	0\\
42.96	0\\
42.97	0\\
42.98	0\\
42.99	0\\
43	0\\
43.01	0\\
43.02	0\\
43.03	0\\
43.04	0\\
43.05	0\\
43.06	0\\
43.07	0\\
43.08	0\\
43.09	0\\
43.1	0\\
43.11	0\\
43.12	0\\
43.13	0\\
43.14	0\\
43.15	0\\
43.16	0\\
43.17	0\\
43.18	0\\
43.19	0\\
43.2	0\\
43.21	0\\
43.22	0\\
43.23	0\\
43.24	0\\
43.25	0\\
43.26	0\\
43.27	0\\
43.28	0\\
43.29	0\\
43.3	0\\
43.31	0\\
43.32	0\\
43.33	0\\
43.34	0\\
43.35	0\\
43.36	0\\
43.37	0\\
43.38	0\\
43.39	0\\
43.4	0\\
43.41	0\\
43.42	0\\
43.43	0\\
43.44	0\\
43.45	0\\
43.46	0\\
43.47	0\\
43.48	0\\
43.49	0\\
43.5	0\\
43.51	0\\
43.52	0\\
43.53	0\\
43.54	0\\
43.55	0\\
43.56	0\\
43.57	0\\
43.58	0\\
43.59	0\\
43.6	0\\
43.61	0\\
43.62	0\\
43.63	0\\
43.64	0\\
43.65	0\\
43.66	0\\
43.67	0\\
43.68	0\\
43.69	0\\
43.7	0\\
43.71	0\\
43.72	0\\
43.73	0\\
43.74	0\\
43.75	0\\
43.76	0\\
43.77	0\\
43.78	0\\
43.79	0\\
43.8	0\\
43.81	0\\
43.82	0\\
43.83	0\\
43.84	0\\
43.85	0\\
43.86	0\\
43.87	0\\
43.88	0\\
43.89	0\\
43.9	0\\
43.91	0\\
43.92	0\\
43.93	0\\
43.94	0\\
43.95	0\\
43.96	0\\
43.97	0\\
43.98	0\\
43.99	0\\
44	0\\
44.01	0\\
44.02	0\\
44.03	0\\
44.04	0\\
44.05	0\\
44.06	0\\
44.07	0\\
44.08	0\\
44.09	0\\
44.1	0\\
44.11	0\\
44.12	0\\
44.13	0\\
44.14	0\\
44.15	0\\
44.16	0\\
44.17	0\\
44.18	0\\
44.19	0\\
44.2	0\\
44.21	0\\
44.22	0\\
44.23	0\\
44.24	0\\
44.25	0\\
44.26	0\\
44.27	0\\
44.28	0\\
44.29	0\\
44.3	0\\
44.31	0\\
44.32	0\\
44.33	0\\
44.34	0\\
44.35	0\\
44.36	0\\
44.37	0\\
44.38	0\\
44.39	0\\
44.4	0\\
44.41	0\\
44.42	0\\
44.43	0\\
44.44	0\\
44.45	0\\
44.46	0\\
44.47	0\\
44.48	0\\
44.49	0\\
44.5	0\\
44.51	0\\
44.52	0\\
44.53	0\\
44.54	0\\
44.55	0\\
44.56	0\\
44.57	0\\
44.58	0\\
44.59	0\\
44.6	0\\
44.61	0\\
44.62	0\\
44.63	0\\
44.64	0\\
44.65	0\\
44.66	0\\
44.67	0\\
44.68	0\\
44.69	0\\
44.7	0\\
44.71	0\\
44.72	0\\
44.73	0\\
44.74	0\\
44.75	0\\
44.76	0\\
44.77	0\\
44.78	0\\
44.79	0\\
44.8	0\\
44.81	0\\
44.82	0\\
44.83	0\\
44.84	0\\
44.85	0\\
44.86	0\\
44.87	0\\
44.88	0\\
44.89	0\\
44.9	0\\
44.91	0\\
44.92	0\\
44.93	0\\
44.94	0\\
44.95	0\\
44.96	0\\
44.97	0\\
44.98	0\\
44.99	0\\
45	0\\
45.01	0\\
45.02	0\\
45.03	0\\
45.04	0\\
45.05	0\\
45.06	0\\
45.07	0\\
45.08	0\\
45.09	0\\
45.1	0\\
45.11	0\\
45.12	0\\
45.13	0\\
45.14	0\\
45.15	0\\
45.16	0\\
45.17	0\\
45.18	0\\
45.19	0\\
45.2	0\\
45.21	0\\
45.22	0\\
45.23	0\\
45.24	0\\
45.25	0\\
45.26	0\\
45.27	0\\
45.28	0\\
45.29	0\\
45.3	0\\
45.31	0\\
45.32	0\\
45.33	0\\
45.34	0\\
45.35	0\\
45.36	0\\
45.37	0\\
45.38	0\\
45.39	0\\
45.4	0\\
45.41	0\\
45.42	0\\
45.43	0\\
45.44	0\\
45.45	0\\
45.46	0\\
45.47	0\\
45.48	0\\
45.49	0\\
45.5	0\\
45.51	0\\
45.52	0\\
45.53	0\\
45.54	0\\
45.55	0\\
45.56	0\\
45.57	0\\
45.58	0\\
45.59	0\\
45.6	0\\
45.61	0\\
45.62	0\\
45.63	0\\
45.64	0\\
45.65	0\\
45.66	0\\
45.67	0\\
45.68	0\\
45.69	0\\
45.7	0\\
45.71	0\\
45.72	0\\
45.73	0\\
45.74	0\\
45.75	0\\
45.76	0\\
45.77	0\\
45.78	0\\
45.79	0\\
45.8	0\\
45.81	0\\
45.82	0\\
45.83	0\\
45.84	0\\
45.85	0\\
45.86	0\\
45.87	0\\
45.88	0\\
45.89	0\\
45.9	0\\
45.91	0\\
45.92	0\\
45.93	0\\
45.94	0\\
45.95	0\\
45.96	0\\
45.97	0\\
45.98	0\\
45.99	0\\
46	0\\
46.01	0\\
46.02	0\\
46.03	0\\
46.04	0\\
46.05	0\\
46.06	0\\
46.07	0\\
46.08	0\\
46.09	0\\
46.1	0\\
46.11	0\\
46.12	0\\
46.13	0\\
46.14	0\\
46.15	0\\
46.16	0\\
46.17	0\\
46.18	0\\
46.19	0\\
46.2	0\\
46.21	0\\
46.22	0\\
46.23	0\\
46.24	0\\
46.25	0\\
46.26	0\\
46.27	0\\
46.28	0\\
46.29	0\\
46.3	0\\
46.31	0\\
46.32	0\\
46.33	0\\
46.34	0\\
46.35	0\\
46.36	0\\
46.37	0\\
46.38	0\\
46.39	0\\
46.4	0\\
46.41	0\\
46.42	0\\
46.43	0\\
46.44	0\\
46.45	0\\
46.46	0\\
46.47	0\\
46.48	0\\
46.49	0\\
46.5	0\\
46.51	0\\
46.52	0\\
46.53	0\\
46.54	0\\
46.55	0\\
46.56	0\\
46.57	0\\
46.58	0\\
46.59	0\\
46.6	0\\
46.61	0\\
46.62	0\\
46.63	0\\
46.64	0\\
46.65	0\\
46.66	0\\
46.67	0\\
46.68	0\\
46.69	0\\
46.7	0\\
46.71	0\\
46.72	0\\
46.73	0\\
46.74	0\\
46.75	0\\
46.76	0\\
46.77	0\\
46.78	0\\
46.79	0\\
46.8	0\\
46.81	0\\
46.82	0\\
46.83	0\\
46.84	0\\
46.85	0\\
46.86	0\\
46.87	0\\
46.88	0\\
46.89	0\\
46.9	0\\
46.91	0\\
46.92	0\\
46.93	0\\
46.94	0\\
46.95	0\\
46.96	0\\
46.97	0\\
46.98	0\\
46.99	0\\
47	0\\
47.01	0\\
47.02	0\\
47.03	0\\
47.04	0\\
47.05	0\\
47.06	0\\
47.07	0\\
47.08	0\\
47.09	0\\
47.1	0\\
47.11	0\\
47.12	0\\
47.13	0\\
47.14	0\\
47.15	0\\
47.16	0\\
47.17	0\\
47.18	0\\
47.19	0\\
47.2	0\\
47.21	0\\
47.22	0\\
47.23	0\\
47.24	0\\
47.25	0\\
47.26	0\\
47.27	0\\
47.28	0\\
47.29	0\\
47.3	0\\
47.31	0\\
47.32	0\\
47.33	0\\
47.34	0\\
47.35	0\\
47.36	0\\
47.37	0\\
47.38	0\\
47.39	0\\
47.4	0\\
47.41	0\\
47.42	0\\
47.43	0\\
47.44	0\\
47.45	0\\
47.46	0\\
47.47	0\\
47.48	0\\
47.49	0\\
47.5	0\\
47.51	0\\
47.52	0\\
47.53	0\\
47.54	0\\
47.55	0\\
47.56	0\\
47.57	0\\
47.58	0\\
47.59	0\\
47.6	0\\
47.61	0\\
47.62	0\\
47.63	0\\
47.64	0\\
47.65	0\\
47.66	0\\
47.67	0\\
47.68	0\\
47.69	0\\
47.7	0\\
47.71	0\\
47.72	0\\
47.73	0\\
47.74	0\\
47.75	0\\
47.76	0\\
47.77	0\\
47.78	0\\
47.79	0\\
47.8	0\\
47.81	0\\
47.82	0\\
47.83	0\\
47.84	0\\
47.85	0\\
47.86	0\\
47.87	0\\
47.88	0\\
47.89	0\\
47.9	0\\
47.91	0\\
47.92	0\\
47.93	0\\
47.94	0\\
47.95	0\\
47.96	0\\
47.97	0\\
47.98	0\\
47.99	0\\
48	0\\
48.01	0\\
48.02	0\\
48.03	0\\
48.04	0\\
48.05	0\\
48.06	0\\
48.07	0\\
48.08	0\\
48.09	0\\
48.1	0\\
48.11	0\\
48.12	0\\
48.13	0\\
48.14	0\\
48.15	0\\
48.16	0\\
48.17	0\\
48.18	0\\
48.19	0\\
48.2	0\\
48.21	0\\
48.22	0\\
48.23	0\\
48.24	0\\
48.25	0\\
48.26	0\\
48.27	0\\
48.28	0\\
48.29	0\\
48.3	0\\
48.31	0\\
48.32	0\\
48.33	0\\
48.34	0\\
48.35	0\\
48.36	0\\
48.37	0\\
48.38	0\\
48.39	0\\
48.4	0\\
48.41	0\\
48.42	0\\
48.43	0\\
48.44	0\\
48.45	0\\
48.46	0\\
48.47	0\\
48.48	0\\
48.49	0\\
48.5	0\\
48.51	0\\
48.52	0\\
48.53	0\\
48.54	0\\
48.55	0\\
48.56	0\\
48.57	0\\
48.58	0\\
48.59	0\\
48.6	0\\
48.61	0\\
48.62	0\\
48.63	0\\
48.64	0\\
48.65	0\\
48.66	0\\
48.67	0\\
48.68	0\\
48.69	0\\
48.7	0\\
48.71	0\\
48.72	0\\
48.73	0\\
48.74	0\\
48.75	0\\
48.76	0\\
48.77	0\\
48.78	0\\
48.79	0\\
48.8	0\\
48.81	0\\
48.82	0\\
48.83	0\\
48.84	0\\
48.85	0\\
48.86	0\\
48.87	0\\
48.88	0\\
48.89	0\\
48.9	0\\
48.91	0\\
48.92	0\\
48.93	0\\
48.94	0\\
48.95	0\\
48.96	0\\
48.97	0\\
48.98	0\\
48.99	0\\
49	0\\
49.01	0\\
49.02	0\\
49.03	0\\
49.04	0\\
49.05	0\\
49.06	0\\
49.07	0\\
49.08	0\\
49.09	0\\
49.1	0\\
49.11	0\\
49.12	0\\
49.13	0\\
49.14	0\\
49.15	0\\
49.16	0\\
49.17	0\\
49.18	0\\
49.19	0\\
49.2	0\\
49.21	0\\
49.22	0\\
49.23	0\\
49.24	0\\
49.25	0\\
49.26	0\\
49.27	0\\
49.28	0\\
49.29	0\\
49.3	0\\
49.31	0\\
49.32	0\\
49.33	0\\
49.34	0\\
49.35	0\\
49.36	0\\
49.37	0\\
49.38	0\\
49.39	0\\
49.4	0\\
49.41	0\\
49.42	0\\
49.43	0\\
49.44	0\\
49.45	0\\
49.46	0\\
49.47	0\\
49.48	0\\
49.49	0\\
49.5	0\\
49.51	0\\
49.52	0\\
49.53	0\\
49.54	0\\
49.55	0\\
49.56	0\\
49.57	0\\
49.58	0\\
49.59	0\\
49.6	0\\
49.61	0\\
49.62	0\\
49.63	0\\
49.64	0\\
49.65	0\\
49.66	0\\
49.67	0\\
49.68	0\\
49.69	0\\
49.7	0\\
49.71	0\\
49.72	0\\
49.73	0\\
49.74	0\\
49.75	0\\
49.76	0\\
49.77	0\\
49.78	0\\
49.79	0\\
49.8	0\\
49.81	0\\
49.82	0\\
49.83	0\\
49.84	0\\
49.85	0\\
49.86	0\\
49.87	0\\
49.88	0\\
49.89	0\\
49.9	0\\
49.91	0\\
49.92	0\\
49.93	0\\
49.94	0\\
49.95	0\\
49.96	0\\
49.97	0\\
49.98	0\\
49.99	0\\
50	0\\
50.01	0\\
50.02	0\\
50.03	0\\
50.04	0\\
50.05	0\\
50.06	0\\
50.07	0\\
50.08	0\\
50.09	0\\
50.1	0\\
50.11	0\\
50.12	0\\
50.13	0\\
50.14	0\\
50.15	0\\
50.16	0\\
50.17	0\\
50.18	0\\
50.19	0\\
50.2	0\\
50.21	0\\
50.22	0\\
50.23	0\\
50.24	0\\
50.25	0\\
50.26	0\\
50.27	0\\
50.28	0\\
50.29	0\\
50.3	0\\
50.31	0\\
50.32	0\\
50.33	0\\
50.34	0\\
50.35	0\\
50.36	0\\
50.37	0\\
50.38	0\\
50.39	0\\
50.4	0\\
50.41	0\\
50.42	0\\
50.43	0\\
50.44	0\\
50.45	0\\
50.46	0\\
50.47	0\\
50.48	0\\
50.49	0\\
50.5	0\\
50.51	0\\
50.52	0\\
50.53	0\\
50.54	0\\
50.55	0\\
50.56	0\\
50.57	0\\
50.58	0\\
50.59	0\\
50.6	0\\
50.61	0\\
50.62	0\\
50.63	0\\
50.64	0\\
50.65	0\\
50.66	0\\
50.67	0\\
50.68	0\\
50.69	0\\
50.7	0\\
50.71	0\\
50.72	0\\
50.73	0\\
50.74	0\\
50.75	0\\
50.76	0\\
50.77	0\\
50.78	0\\
50.79	0\\
50.8	0\\
50.81	0\\
50.82	0\\
50.83	0\\
50.84	0\\
50.85	0\\
50.86	0\\
50.87	0\\
50.88	0\\
50.89	0\\
50.9	0\\
50.91	0\\
50.92	0\\
50.93	0\\
50.94	0\\
50.95	0\\
50.96	0\\
50.97	0\\
50.98	0\\
50.99	0\\
51	0\\
51.01	0\\
51.02	0\\
51.03	0\\
51.04	0\\
51.05	0\\
51.06	0\\
51.07	0\\
51.08	0\\
51.09	0\\
51.1	0\\
51.11	0\\
51.12	0\\
51.13	0\\
51.14	0\\
51.15	0\\
51.16	0\\
51.17	0\\
51.18	0\\
51.19	0\\
51.2	0\\
51.21	0\\
51.22	0\\
51.23	0\\
51.24	0\\
51.25	0\\
51.26	0\\
51.27	0\\
51.28	0\\
51.29	0\\
51.3	0\\
51.31	0\\
51.32	0\\
51.33	0\\
51.34	0\\
51.35	0\\
51.36	0\\
51.37	0\\
51.38	0\\
51.39	0\\
51.4	0\\
51.41	0\\
51.42	0\\
51.43	0\\
51.44	0\\
51.45	0\\
51.46	0\\
51.47	0\\
51.48	0\\
51.49	0\\
51.5	0\\
51.51	0\\
51.52	0\\
51.53	0\\
51.54	0\\
51.55	0\\
51.56	0\\
51.57	0\\
51.58	0\\
51.59	0\\
51.6	0\\
51.61	0\\
51.62	0\\
51.63	0\\
51.64	0\\
51.65	0\\
51.66	0\\
51.67	0\\
51.68	0\\
51.69	0\\
51.7	0\\
51.71	0\\
51.72	0\\
51.73	0\\
51.74	0\\
51.75	0\\
51.76	0\\
51.77	0\\
51.78	0\\
51.79	0\\
51.8	0\\
51.81	0\\
51.82	0\\
51.83	0\\
51.84	0\\
51.85	0\\
51.86	0\\
51.87	0\\
51.88	0\\
51.89	0\\
51.9	0\\
51.91	0\\
51.92	0\\
51.93	0\\
51.94	0\\
51.95	0\\
51.96	0\\
51.97	0\\
51.98	0\\
51.99	0\\
52	0\\
52.01	0\\
52.02	0\\
52.03	0\\
52.04	0\\
52.05	0\\
52.06	0\\
52.07	0\\
52.08	0\\
52.09	0\\
52.1	0\\
52.11	0\\
52.12	0\\
52.13	0\\
52.14	0\\
52.15	0\\
52.16	0\\
52.17	0\\
52.18	0\\
52.19	0\\
52.2	0\\
52.21	0\\
52.22	0\\
52.23	0\\
52.24	0\\
52.25	0\\
52.26	0\\
52.27	0\\
52.28	0\\
52.29	0\\
52.3	0\\
52.31	0\\
52.32	0\\
52.33	0\\
52.34	0\\
52.35	0\\
52.36	0\\
52.37	0\\
52.38	0\\
52.39	0\\
52.4	0\\
52.41	0\\
52.42	0\\
52.43	0\\
52.44	0\\
52.45	0\\
52.46	0\\
52.47	0\\
52.48	0\\
52.49	0\\
52.5	0\\
52.51	0\\
52.52	0\\
52.53	0\\
52.54	0\\
52.55	0\\
52.56	0\\
52.57	0\\
52.58	0\\
52.59	0\\
52.6	0\\
52.61	0\\
52.62	0\\
52.63	0\\
52.64	0\\
52.65	0\\
52.66	0\\
52.67	0\\
52.68	0\\
52.69	0\\
52.7	0\\
52.71	0\\
52.72	0\\
52.73	0\\
52.74	0\\
52.75	0\\
52.76	0\\
52.77	0\\
52.78	0\\
52.79	0\\
52.8	0\\
52.81	0\\
52.82	0\\
52.83	0\\
52.84	0\\
52.85	0\\
52.86	0\\
52.87	0\\
52.88	0\\
52.89	0\\
52.9	0\\
52.91	0\\
52.92	0\\
52.93	0\\
52.94	0\\
52.95	0\\
52.96	0\\
52.97	0\\
52.98	0\\
52.99	0\\
53	0\\
53.01	0\\
53.02	0\\
53.03	0\\
53.04	0\\
53.05	0\\
53.06	0\\
53.07	0\\
53.08	0\\
53.09	0\\
53.1	0\\
53.11	0\\
53.12	0\\
53.13	0\\
53.14	0\\
53.15	0\\
53.16	0\\
53.17	0\\
53.18	0\\
53.19	0\\
53.2	0\\
53.21	0\\
53.22	0\\
53.23	0\\
53.24	0\\
53.25	0\\
53.26	0\\
53.27	0\\
53.28	0\\
53.29	0\\
53.3	0\\
53.31	0\\
53.32	0\\
53.33	0\\
53.34	0\\
53.35	0\\
53.36	0\\
53.37	0\\
53.38	0\\
53.39	0\\
53.4	0\\
53.41	0\\
53.42	0\\
53.43	0\\
53.44	0\\
53.45	0\\
53.46	0\\
53.47	0\\
53.48	0\\
53.49	0\\
53.5	0\\
53.51	0\\
53.52	0\\
53.53	0\\
53.54	0\\
53.55	0\\
53.56	0\\
53.57	0\\
53.58	0\\
53.59	0\\
53.6	0\\
53.61	0\\
53.62	0\\
53.63	0\\
53.64	0\\
53.65	0\\
53.66	0\\
53.67	0\\
53.68	0\\
53.69	0\\
53.7	0\\
53.71	0\\
53.72	0\\
53.73	0\\
53.74	0\\
53.75	0\\
53.76	0\\
53.77	0\\
53.78	0\\
53.79	0\\
53.8	0\\
53.81	0\\
53.82	0\\
53.83	0\\
53.84	0\\
53.85	0\\
53.86	0\\
53.87	0\\
53.88	0\\
53.89	0\\
53.9	0\\
53.91	0\\
53.92	0\\
53.93	0\\
53.94	0\\
53.95	0\\
53.96	0\\
53.97	0\\
53.98	0\\
53.99	0\\
54	0\\
54.01	0\\
54.02	0\\
54.03	0\\
54.04	0\\
54.05	0\\
54.06	0\\
54.07	0\\
54.08	0\\
54.09	0\\
54.1	0\\
54.11	0\\
54.12	0\\
54.13	0\\
54.14	0\\
54.15	0\\
54.16	0\\
54.17	0\\
54.18	0\\
54.19	0\\
54.2	0\\
54.21	0\\
54.22	0\\
54.23	0\\
54.24	0\\
54.25	0\\
54.26	0\\
54.27	0\\
54.28	0\\
54.29	0\\
54.3	0\\
54.31	0\\
54.32	0\\
54.33	0\\
54.34	0\\
54.35	0\\
54.36	0\\
54.37	0\\
54.38	0\\
54.39	0\\
54.4	0\\
54.41	0\\
54.42	0\\
54.43	0\\
54.44	0\\
54.45	0\\
54.46	0\\
54.47	0\\
54.48	0\\
54.49	0\\
54.5	0\\
54.51	0\\
54.52	0\\
54.53	0\\
54.54	0\\
54.55	0\\
54.56	0\\
54.57	0\\
54.58	0\\
54.59	0\\
54.6	0\\
54.61	0\\
54.62	0\\
54.63	0\\
54.64	0\\
54.65	0\\
54.66	0\\
54.67	0\\
54.68	0\\
54.69	0\\
54.7	0\\
54.71	0\\
54.72	0\\
54.73	0\\
54.74	0\\
54.75	0\\
54.76	0\\
54.77	0\\
54.78	0\\
54.79	0\\
54.8	0\\
54.81	0\\
54.82	0\\
54.83	0\\
54.84	0\\
54.85	0\\
54.86	0\\
54.87	0\\
54.88	0\\
54.89	0\\
54.9	0\\
54.91	0\\
54.92	0\\
54.93	0\\
54.94	0\\
54.95	0\\
54.96	0\\
54.97	0\\
54.98	0\\
54.99	0\\
55	0\\
55.01	0\\
55.02	0\\
55.03	0\\
55.04	0\\
55.05	0\\
55.06	0\\
55.07	0\\
55.08	0\\
55.09	0\\
55.1	0\\
55.11	0\\
55.12	0\\
55.13	0\\
55.14	0\\
55.15	0\\
55.16	0\\
55.17	0\\
55.18	0\\
55.19	0\\
55.2	0\\
55.21	0\\
55.22	0\\
55.23	0\\
55.24	0\\
55.25	0\\
55.26	0\\
55.27	0\\
55.28	0\\
55.29	0\\
55.3	0\\
55.31	0\\
55.32	0\\
55.33	0\\
55.34	0\\
55.35	0\\
55.36	0\\
55.37	0\\
55.38	0\\
55.39	0\\
55.4	0\\
55.41	0\\
55.42	0\\
55.43	0\\
55.44	0\\
55.45	0\\
55.46	0\\
55.47	0\\
55.48	0\\
55.49	0\\
55.5	0\\
55.51	0\\
55.52	0\\
55.53	0\\
55.54	0\\
55.55	0\\
55.56	0\\
55.57	0\\
55.58	0\\
55.59	0\\
55.6	0\\
55.61	0\\
55.62	0\\
55.63	0\\
55.64	0\\
55.65	0\\
55.66	0\\
55.67	0\\
55.68	0\\
55.69	0\\
55.7	0\\
55.71	0\\
55.72	0\\
55.73	0\\
55.74	0\\
55.75	0\\
55.76	0\\
55.77	0\\
55.78	0\\
55.79	0\\
55.8	0\\
55.81	0\\
55.82	0\\
55.83	0\\
55.84	0\\
55.85	0\\
55.86	0\\
55.87	0\\
55.88	0\\
55.89	0\\
55.9	0\\
55.91	0\\
55.92	0\\
55.93	0\\
55.94	0\\
55.95	0\\
55.96	0\\
55.97	0\\
55.98	0\\
55.99	0\\
56	0\\
56.01	0\\
56.02	0\\
56.03	0\\
56.04	0\\
56.05	0\\
56.06	0\\
56.07	0\\
56.08	0\\
56.09	0\\
56.1	0\\
56.11	0\\
56.12	0\\
56.13	0\\
56.14	0\\
56.15	0\\
56.16	0\\
56.17	0\\
56.18	0\\
56.19	0\\
56.2	0\\
56.21	0\\
56.22	0\\
56.23	0\\
56.24	0\\
56.25	0\\
56.26	0\\
56.27	0\\
56.28	0\\
56.29	0\\
56.3	0\\
56.31	0\\
56.32	0\\
56.33	0\\
56.34	0\\
56.35	0\\
56.36	0\\
56.37	0\\
56.38	0\\
56.39	0\\
56.4	0\\
56.41	0\\
56.42	0\\
56.43	0\\
56.44	0\\
56.45	0\\
56.46	0\\
56.47	0\\
56.48	0\\
56.49	0\\
56.5	0\\
56.51	0\\
56.52	0\\
56.53	0\\
56.54	0\\
56.55	0\\
56.56	0\\
56.57	0\\
56.58	0\\
56.59	0\\
56.6	0\\
56.61	0\\
56.62	0\\
56.63	0\\
56.64	0\\
56.65	0\\
56.66	0\\
56.67	0\\
56.68	0\\
56.69	0\\
56.7	0\\
56.71	0\\
56.72	0\\
56.73	0\\
56.74	0\\
56.75	0\\
56.76	0\\
56.77	0\\
56.78	0\\
56.79	0\\
56.8	0\\
56.81	0\\
56.82	0\\
56.83	0\\
56.84	0\\
56.85	0\\
56.86	0\\
56.87	0\\
56.88	0\\
56.89	0\\
56.9	0\\
56.91	0\\
56.92	0\\
56.93	0\\
56.94	0\\
56.95	0\\
56.96	0\\
56.97	0\\
56.98	0\\
56.99	0\\
57	0\\
57.01	0\\
57.02	0\\
57.03	0\\
57.04	0\\
57.05	0\\
57.06	0\\
57.07	0\\
57.08	0\\
57.09	0\\
57.1	0\\
57.11	0\\
57.12	0\\
57.13	0\\
57.14	0\\
57.15	0\\
57.16	0\\
57.17	0\\
57.18	0\\
57.19	0\\
57.2	0\\
57.21	0\\
57.22	0\\
57.23	0\\
57.24	0\\
57.25	0\\
57.26	0\\
57.27	0\\
57.28	0\\
57.29	0\\
57.3	0\\
57.31	0\\
57.32	0\\
57.33	0\\
57.34	0\\
57.35	0\\
57.36	0\\
57.37	0\\
57.38	0\\
57.39	0\\
57.4	0\\
57.41	0\\
57.42	0\\
57.43	0\\
57.44	0\\
57.45	0\\
57.46	0\\
57.47	0\\
57.48	0\\
57.49	0\\
57.5	0\\
57.51	0\\
57.52	0\\
57.53	0\\
57.54	0\\
57.55	0\\
57.56	0\\
57.57	0\\
57.58	0\\
57.59	0\\
57.6	0\\
57.61	0\\
57.62	0\\
57.63	0\\
57.64	0\\
57.65	0\\
57.66	0\\
57.67	0\\
57.68	0\\
57.69	0\\
57.7	0\\
57.71	0\\
57.72	0\\
57.73	0\\
57.74	0\\
57.75	0\\
57.76	0\\
57.77	0\\
57.78	0\\
57.79	0\\
57.8	0\\
57.81	0\\
57.82	0\\
57.83	0\\
57.84	0\\
57.85	0\\
57.86	0\\
57.87	0\\
57.88	0\\
57.89	0\\
57.9	0\\
57.91	0\\
57.92	0\\
57.93	0\\
57.94	0\\
57.95	0\\
57.96	0\\
57.97	0\\
57.98	0\\
57.99	0\\
58	0\\
58.01	0\\
58.02	0\\
58.03	0\\
58.04	0\\
58.05	0\\
58.06	0\\
58.07	0\\
58.08	0\\
58.09	0\\
58.1	0\\
58.11	0\\
58.12	0\\
58.13	0\\
58.14	0\\
58.15	0\\
58.16	0\\
58.17	0\\
58.18	0\\
58.19	0\\
58.2	0\\
58.21	0\\
58.22	0\\
58.23	0\\
58.24	0\\
58.25	0\\
58.26	0\\
58.27	0\\
58.28	0\\
58.29	0\\
58.3	0\\
58.31	0\\
58.32	0\\
58.33	0\\
58.34	0\\
58.35	0\\
58.36	0\\
58.37	0\\
58.38	0\\
58.39	0\\
58.4	0\\
58.41	0\\
58.42	0\\
58.43	0\\
58.44	0\\
58.45	0\\
58.46	0\\
58.47	0\\
58.48	0\\
58.49	0\\
58.5	0\\
58.51	0\\
58.52	0\\
58.53	0\\
58.54	0\\
58.55	0\\
58.56	0\\
58.57	0\\
58.58	0\\
58.59	0\\
58.6	0\\
58.61	0\\
58.62	0\\
58.63	0\\
58.64	0\\
58.65	0\\
58.66	0\\
58.67	0\\
58.68	0\\
58.69	0\\
58.7	0\\
58.71	0\\
58.72	0\\
58.73	0\\
58.74	0\\
58.75	0\\
58.76	0\\
58.77	0\\
58.78	0\\
58.79	0\\
58.8	0\\
58.81	0\\
58.82	0\\
58.83	0\\
58.84	0\\
58.85	0\\
58.86	0\\
58.87	0\\
58.88	0\\
58.89	0\\
58.9	0\\
58.91	0\\
58.92	0\\
58.93	0\\
58.94	0\\
58.95	0\\
58.96	0\\
58.97	0\\
58.98	0\\
58.99	0\\
59	0\\
59.01	0\\
59.02	0\\
59.03	0\\
59.04	0\\
59.05	0\\
59.06	0\\
59.07	0\\
59.08	0\\
59.09	0\\
59.1	0\\
59.11	0\\
59.12	0\\
59.13	0\\
59.14	0\\
59.15	0\\
59.16	0\\
59.17	0\\
59.18	0\\
59.19	0\\
59.2	0\\
59.21	0\\
59.22	0\\
59.23	0\\
59.24	0\\
59.25	0\\
59.26	0\\
59.27	0\\
59.28	0\\
59.29	0\\
59.3	0\\
59.31	0\\
59.32	0\\
59.33	0\\
59.34	0\\
59.35	0\\
59.36	0\\
59.37	0\\
59.38	0\\
59.39	0\\
59.4	0\\
59.41	0\\
59.42	0\\
59.43	0\\
59.44	0\\
59.45	0\\
59.46	0\\
59.47	0\\
59.48	0\\
59.49	0\\
59.5	0\\
59.51	0\\
59.52	0\\
59.53	0\\
59.54	0\\
59.55	0\\
59.56	0\\
59.57	0\\
59.58	0\\
59.59	0\\
59.6	0\\
59.61	0\\
59.62	0\\
59.63	0\\
59.64	0\\
59.65	0\\
59.66	0\\
59.67	0\\
59.68	0\\
59.69	0\\
59.7	0\\
59.71	0\\
59.72	0\\
59.73	0\\
59.74	0\\
59.75	0\\
59.76	0\\
59.77	0\\
59.78	0\\
59.79	0\\
59.8	0\\
59.81	0\\
59.82	0\\
59.83	0\\
59.84	0\\
59.85	0\\
59.86	0\\
59.87	0\\
59.88	0\\
59.89	0\\
59.9	0\\
59.91	0\\
59.92	0\\
59.93	0\\
59.94	0\\
59.95	0\\
59.96	0\\
59.97	0\\
59.98	0\\
59.99	0\\
60	0\\
60.01	0\\
60.02	0\\
60.03	0\\
60.04	0\\
60.05	0\\
60.06	0\\
60.07	0\\
60.08	0\\
60.09	0\\
60.1	0\\
60.11	0\\
60.12	0\\
60.13	0\\
60.14	0\\
60.15	0\\
60.16	0\\
60.17	0\\
60.18	0\\
60.19	0\\
60.2	0\\
60.21	0\\
60.22	0\\
60.23	0\\
60.24	0\\
60.25	0\\
60.26	0\\
60.27	0\\
60.28	0\\
60.29	0\\
60.3	0\\
60.31	0\\
60.32	0\\
60.33	0\\
60.34	0\\
60.35	0\\
60.36	0\\
60.37	0\\
60.38	0\\
60.39	0\\
60.4	0\\
60.41	0\\
60.42	0\\
60.43	0\\
60.44	0\\
60.45	0\\
60.46	0\\
60.47	0\\
60.48	0\\
60.49	0\\
60.5	0\\
60.51	0\\
60.52	0\\
60.53	0\\
60.54	0\\
60.55	0\\
60.56	0\\
60.57	0\\
60.58	0\\
60.59	0\\
60.6	0\\
60.61	0\\
60.62	0\\
60.63	0\\
60.64	0\\
60.65	0\\
60.66	0\\
60.67	0\\
60.68	0\\
60.69	0\\
60.7	0\\
60.71	0\\
60.72	0\\
60.73	0\\
60.74	0\\
60.75	0\\
60.76	0\\
60.77	0\\
60.78	0\\
60.79	0\\
60.8	0\\
60.81	0\\
60.82	0\\
60.83	0\\
60.84	0\\
60.85	0\\
60.86	0\\
60.87	0\\
60.88	0\\
60.89	0\\
60.9	0\\
60.91	0\\
60.92	0\\
60.93	0\\
60.94	0\\
60.95	0\\
60.96	0\\
60.97	0\\
60.98	0\\
60.99	0\\
61	0\\
61.01	0\\
61.02	0\\
61.03	0\\
61.04	0\\
61.05	0\\
61.06	0\\
61.07	0\\
61.08	0\\
61.09	0\\
61.1	0\\
61.11	0\\
61.12	0\\
61.13	0\\
61.14	0\\
61.15	0\\
61.16	0\\
61.17	0\\
61.18	0\\
61.19	0\\
61.2	0\\
61.21	0\\
61.22	0\\
61.23	0\\
61.24	0\\
61.25	0\\
61.26	0\\
61.27	0\\
61.28	0\\
61.29	0\\
61.3	0\\
61.31	0\\
61.32	0\\
61.33	0\\
61.34	0\\
61.35	0\\
61.36	0\\
61.37	0\\
61.38	0\\
61.39	0\\
61.4	0\\
61.41	0\\
61.42	0\\
61.43	0\\
61.44	0\\
61.45	0\\
61.46	0\\
61.47	0\\
61.48	0\\
61.49	0\\
61.5	0\\
61.51	0\\
61.52	0\\
61.53	0\\
61.54	0\\
61.55	0\\
61.56	0\\
61.57	0\\
61.58	0\\
61.59	0\\
61.6	0\\
61.61	0\\
61.62	0\\
61.63	0\\
61.64	0\\
61.65	0\\
61.66	0\\
61.67	0\\
61.68	0\\
61.69	0\\
61.7	0\\
61.71	0\\
61.72	0\\
61.73	0\\
61.74	0\\
61.75	0\\
61.76	0\\
61.77	0\\
61.78	0\\
61.79	0\\
61.8	0\\
61.81	0\\
61.82	0\\
61.83	0\\
61.84	0\\
61.85	0\\
61.86	0\\
61.87	0\\
61.88	0\\
61.89	0\\
61.9	0\\
61.91	0\\
61.92	0\\
61.93	0\\
61.94	0\\
61.95	0\\
61.96	0\\
61.97	0\\
61.98	0\\
61.99	0\\
62	0\\
62.01	0\\
62.02	0\\
62.03	0\\
62.04	0\\
62.05	0\\
62.06	0\\
62.07	0\\
62.08	0\\
62.09	0\\
62.1	0\\
62.11	0\\
62.12	0\\
62.13	0\\
62.14	0\\
62.15	0\\
62.16	0\\
62.17	0\\
62.18	0\\
62.19	0\\
62.2	0\\
62.21	0\\
62.22	0\\
62.23	0\\
62.24	0\\
62.25	0\\
62.26	0\\
62.27	0\\
62.28	0\\
62.29	0\\
62.3	0\\
62.31	0\\
62.32	0\\
62.33	0\\
62.34	0\\
62.35	0\\
62.36	0\\
62.37	0\\
62.38	0\\
62.39	0\\
62.4	0\\
62.41	0\\
62.42	0\\
62.43	0\\
62.44	0\\
62.45	0\\
62.46	0\\
62.47	0\\
62.48	0\\
62.49	0\\
62.5	0\\
62.51	0\\
62.52	0\\
62.53	0\\
62.54	0\\
62.55	0\\
62.56	0\\
62.57	0\\
62.58	0\\
62.59	0\\
62.6	0\\
62.61	0\\
62.62	0\\
62.63	0\\
62.64	0\\
62.65	0\\
62.66	0\\
62.67	0\\
62.68	0\\
62.69	0\\
62.7	0\\
62.71	0\\
62.72	0\\
62.73	0\\
62.74	0\\
62.75	0\\
62.76	0\\
62.77	0\\
62.78	0\\
62.79	0\\
62.8	0\\
62.81	0\\
62.82	0\\
62.83	0\\
62.84	0\\
62.85	0\\
62.86	0\\
62.87	0\\
62.88	0\\
62.89	0\\
62.9	0\\
62.91	0\\
62.92	0\\
62.93	0\\
62.94	0\\
62.95	0\\
62.96	0\\
62.97	0\\
62.98	0\\
62.99	0\\
63	0\\
63.01	0\\
63.02	0\\
63.03	0\\
63.04	0\\
63.05	0\\
63.06	0\\
63.07	0\\
63.08	0\\
63.09	0\\
63.1	0\\
63.11	0\\
63.12	0\\
63.13	0\\
63.14	0\\
63.15	0\\
63.16	0\\
63.17	0\\
63.18	0\\
63.19	0\\
63.2	0\\
63.21	0\\
63.22	0\\
63.23	0\\
63.24	0\\
63.25	0\\
63.26	0\\
63.27	0\\
63.28	0\\
63.29	0\\
63.3	0\\
63.31	0\\
63.32	0\\
63.33	0\\
63.34	0\\
63.35	0\\
63.36	0\\
63.37	0\\
63.38	0\\
63.39	0\\
63.4	0\\
63.41	0\\
63.42	0\\
63.43	0\\
63.44	0\\
63.45	0\\
63.46	0\\
63.47	0\\
63.48	0\\
63.49	0\\
63.5	0\\
63.51	0\\
63.52	0\\
63.53	0\\
63.54	0\\
63.55	0\\
63.56	0\\
63.57	0\\
63.58	0\\
63.59	0\\
63.6	0\\
63.61	0\\
63.62	0\\
63.63	0\\
63.64	0\\
63.65	0\\
63.66	0\\
63.67	0\\
63.68	0\\
63.69	0\\
63.7	0\\
63.71	0\\
63.72	0\\
63.73	0\\
63.74	0\\
63.75	0\\
63.76	0\\
63.77	0\\
63.78	0\\
63.79	0\\
63.8	0\\
63.81	0\\
63.82	0\\
63.83	0\\
63.84	0\\
63.85	0\\
63.86	0\\
63.87	0\\
63.88	0\\
63.89	0\\
63.9	0\\
63.91	0\\
63.92	0\\
63.93	0\\
63.94	0\\
63.95	0\\
63.96	0\\
63.97	0\\
63.98	0\\
63.99	0\\
64	0\\
64.01	0\\
64.02	0\\
64.03	0\\
64.04	0\\
64.05	0\\
64.06	0\\
64.07	0\\
64.08	0\\
64.09	0\\
64.1	0\\
64.11	0\\
64.12	0\\
64.13	0\\
64.14	0\\
64.15	0\\
64.16	0\\
64.17	0\\
64.18	0\\
64.19	0\\
64.2	0\\
64.21	0\\
64.22	0\\
64.23	0\\
64.24	0\\
64.25	0\\
64.26	0\\
64.27	0\\
64.28	0\\
64.29	0\\
64.3	0\\
64.31	0\\
64.32	0\\
64.33	0\\
64.34	0\\
64.35	0\\
64.36	0\\
64.37	0\\
64.38	0\\
64.39	0\\
64.4	0\\
64.41	0\\
64.42	0\\
64.43	0\\
64.44	0\\
64.45	0\\
64.46	0\\
64.47	0\\
64.48	0\\
64.49	0\\
64.5	0\\
64.51	0\\
64.52	0\\
64.53	0\\
64.54	0\\
64.55	0\\
64.56	0\\
64.57	0\\
64.58	0\\
64.59	0\\
64.6	0\\
64.61	0\\
64.62	0\\
64.63	0\\
64.64	0\\
64.65	0\\
64.66	0\\
64.67	0\\
64.68	0\\
64.69	0\\
64.7	0\\
64.71	0\\
64.72	0\\
64.73	0\\
64.74	0\\
64.75	0\\
64.76	0\\
64.77	0\\
64.78	0\\
64.79	0\\
64.8	0\\
64.81	0\\
64.82	0\\
64.83	0\\
64.84	0\\
64.85	0\\
64.86	0\\
64.87	0\\
64.88	0\\
64.89	0\\
64.9	0\\
64.91	0\\
64.92	0\\
64.93	0\\
64.94	0\\
64.95	0\\
64.96	0\\
64.97	0\\
64.98	0\\
64.99	0\\
65	0\\
65.01	0\\
65.02	0\\
65.03	0\\
65.04	0\\
65.05	0\\
65.06	0\\
65.07	0\\
65.08	0\\
65.09	0\\
65.1	0\\
65.11	0\\
65.12	0\\
65.13	0\\
65.14	0\\
65.15	0\\
65.16	0\\
65.17	0\\
65.18	0\\
65.19	0\\
65.2	0\\
65.21	0\\
65.22	0\\
65.23	0\\
65.24	0\\
65.25	0\\
65.26	0\\
65.27	0\\
65.28	0\\
65.29	0\\
65.3	0\\
65.31	0\\
65.32	0\\
65.33	0\\
65.34	0\\
65.35	0\\
65.36	0\\
65.37	0\\
65.38	0\\
65.39	0\\
65.4	0\\
65.41	0\\
65.42	0\\
65.43	0\\
65.44	0\\
65.45	0\\
65.46	0\\
65.47	0\\
65.48	0\\
65.49	0\\
65.5	0\\
65.51	0\\
65.52	0\\
65.53	0\\
65.54	0\\
65.55	0\\
65.56	0\\
65.57	0\\
65.58	0\\
65.59	0\\
65.6	0\\
65.61	0\\
65.62	0\\
65.63	0\\
65.64	0\\
65.65	0\\
65.66	0\\
65.67	0\\
65.68	0\\
65.69	0\\
65.7	0\\
65.71	0\\
65.72	0\\
65.73	0\\
65.74	0\\
65.75	0\\
65.76	0\\
65.77	0\\
65.78	0\\
65.79	0\\
65.8	0\\
65.81	0\\
65.82	0\\
65.83	0\\
65.84	0\\
65.85	0\\
65.86	0\\
65.87	0\\
65.88	0\\
65.89	0\\
65.9	0\\
65.91	0\\
65.92	0\\
65.93	0\\
65.94	0\\
65.95	0\\
65.96	0\\
65.97	0\\
65.98	0\\
65.99	0\\
66	0\\
66.01	0\\
66.02	0\\
66.03	0\\
66.04	0\\
66.05	0\\
66.06	0\\
66.07	0\\
66.08	0\\
66.09	0\\
66.1	0\\
66.11	0\\
66.12	0\\
66.13	0\\
66.14	0\\
66.15	0\\
66.16	0\\
66.17	0\\
66.18	0\\
66.19	0\\
66.2	0\\
66.21	0\\
66.22	0\\
66.23	0\\
66.24	0\\
66.25	0\\
66.26	0\\
66.27	0\\
66.28	0\\
66.29	0\\
66.3	0\\
66.31	0\\
66.32	0\\
66.33	0\\
66.34	0\\
66.35	0\\
66.36	0\\
66.37	0\\
66.38	0\\
66.39	0\\
66.4	0\\
66.41	0\\
66.42	0\\
66.43	0\\
66.44	0\\
66.45	0\\
66.46	0\\
66.47	0\\
66.48	0\\
66.49	0\\
66.5	0\\
66.51	0\\
66.52	0\\
66.53	0\\
66.54	0\\
66.55	0\\
66.56	0\\
66.57	0\\
66.58	0\\
66.59	0\\
66.6	0\\
66.61	0\\
66.62	0\\
66.63	0\\
66.64	0\\
66.65	0\\
66.66	0\\
66.67	0\\
66.68	0\\
66.69	0\\
66.7	0\\
66.71	0\\
66.72	0\\
66.73	0\\
66.74	0\\
66.75	0\\
66.76	0\\
66.77	0\\
66.78	0\\
66.79	0\\
66.8	0\\
66.81	0\\
66.82	0\\
66.83	0\\
66.84	0\\
66.85	0\\
66.86	0\\
66.87	0\\
66.88	0\\
66.89	0\\
66.9	0\\
66.91	0\\
66.92	0\\
66.93	0\\
66.94	0\\
66.95	0\\
66.96	0\\
66.97	0\\
66.98	0\\
66.99	0\\
67	0\\
67.01	0\\
67.02	0\\
67.03	0\\
67.04	0\\
67.05	0\\
67.06	0\\
67.07	0\\
67.08	0\\
67.09	0\\
67.1	0\\
67.11	0\\
67.12	0\\
67.13	0\\
67.14	0\\
67.15	0\\
67.16	0\\
67.17	0\\
67.18	0\\
67.19	0\\
67.2	0\\
67.21	0\\
67.22	0\\
67.23	0\\
67.24	0\\
67.25	0\\
67.26	0\\
67.27	0\\
67.28	0\\
67.29	0\\
67.3	0\\
67.31	0\\
67.32	0\\
67.33	0\\
67.34	0\\
67.35	0\\
67.36	0\\
67.37	0\\
67.38	0\\
67.39	0\\
67.4	0\\
67.41	0\\
67.42	0\\
67.43	0\\
67.44	0\\
67.45	0\\
67.46	0\\
67.47	0\\
67.48	0\\
67.49	0\\
67.5	0\\
67.51	0\\
67.52	0\\
67.53	0\\
67.54	0\\
67.55	0\\
67.56	0\\
67.57	0\\
67.58	0\\
67.59	0\\
67.6	0\\
67.61	0\\
67.62	0\\
67.63	0\\
67.64	0\\
67.65	0\\
67.66	0\\
67.67	0\\
67.68	0\\
67.69	0\\
67.7	0\\
67.71	0\\
67.72	0\\
67.73	0\\
67.74	0\\
67.75	0\\
67.76	0\\
67.77	0\\
67.78	0\\
67.79	0\\
67.8	0\\
67.81	0\\
67.82	0\\
67.83	0\\
67.84	0\\
67.85	0\\
67.86	0\\
67.87	0\\
67.88	0\\
67.89	0\\
67.9	0\\
67.91	0\\
67.92	0\\
67.93	0\\
67.94	0\\
67.95	0\\
67.96	0\\
67.97	0\\
67.98	0\\
67.99	0\\
68	0\\
68.01	0\\
68.02	0\\
68.03	0\\
68.04	0\\
68.05	0\\
68.06	0\\
68.07	0\\
68.08	0\\
68.09	0\\
68.1	0\\
68.11	0\\
68.12	0\\
68.13	0\\
68.14	0\\
68.15	0\\
68.16	0\\
68.17	0\\
68.18	0\\
68.19	0\\
68.2	0\\
68.21	0\\
68.22	0\\
68.23	0\\
68.24	0\\
68.25	0\\
68.26	0\\
68.27	0\\
68.28	0\\
68.29	0\\
68.3	0\\
68.31	0\\
68.32	0\\
68.33	0\\
68.34	0\\
68.35	0\\
68.36	0\\
68.37	0\\
68.38	0\\
68.39	0\\
68.4	0\\
68.41	0\\
68.42	0\\
68.43	0\\
68.44	0\\
68.45	0\\
68.46	0\\
68.47	0\\
68.48	0\\
68.49	0\\
68.5	0\\
68.51	0\\
68.52	0\\
68.53	0\\
68.54	0\\
68.55	0\\
68.56	0\\
68.57	0\\
68.58	0\\
68.59	0\\
68.6	0\\
68.61	0\\
68.62	0\\
68.63	0\\
68.64	0\\
68.65	0\\
68.66	0\\
68.67	0\\
68.68	0\\
68.69	0\\
68.7	0\\
68.71	0\\
68.72	0\\
68.73	0\\
68.74	0\\
68.75	0\\
68.76	0\\
68.77	0\\
68.78	0\\
68.79	0\\
68.8	0\\
68.81	0\\
68.82	0\\
68.83	0\\
68.84	0\\
68.85	0\\
68.86	0\\
68.87	0\\
68.88	0\\
68.89	0\\
68.9	0\\
68.91	0\\
68.92	0\\
68.93	0\\
68.94	0\\
68.95	0\\
68.96	0\\
68.97	0\\
68.98	0\\
68.99	0\\
69	0\\
69.01	0\\
69.02	0\\
69.03	0\\
69.04	0\\
69.05	0\\
69.06	0\\
69.07	0\\
69.08	0\\
69.09	0\\
69.1	0\\
69.11	0\\
69.12	0\\
69.13	0\\
69.14	0\\
69.15	0\\
69.16	0\\
69.17	0\\
69.18	0\\
69.19	0\\
69.2	0\\
69.21	0\\
69.22	0\\
69.23	0\\
69.24	0\\
69.25	0\\
69.26	0\\
69.27	0\\
69.28	0\\
69.29	0\\
69.3	0\\
69.31	0\\
69.32	0\\
69.33	0\\
69.34	0\\
69.35	0\\
69.36	0\\
69.37	0\\
69.38	0\\
69.39	0\\
69.4	0\\
69.41	0\\
69.42	0\\
69.43	0\\
69.44	0\\
69.45	0\\
69.46	0\\
69.47	0\\
69.48	0\\
69.49	0\\
69.5	0\\
69.51	0\\
69.52	0\\
69.53	0\\
69.54	0\\
69.55	0\\
69.56	0\\
69.57	0\\
69.58	0\\
69.59	0\\
69.6	0\\
69.61	0\\
69.62	0\\
69.63	0\\
69.64	0\\
69.65	0\\
69.66	0\\
69.67	0\\
69.68	0\\
69.69	0\\
69.7	0\\
69.71	0\\
69.72	0\\
69.73	0\\
69.74	0\\
69.75	0\\
69.76	0\\
69.77	0\\
69.78	0\\
69.79	0\\
69.8	0\\
69.81	0\\
69.82	0\\
69.83	0\\
69.84	0\\
69.85	0\\
69.86	0\\
69.87	0\\
69.88	0\\
69.89	0\\
69.9	0\\
69.91	0\\
69.92	0\\
69.93	0\\
69.94	0\\
69.95	0\\
69.96	0\\
69.97	0\\
69.98	0\\
69.99	0\\
70	0\\
70.01	0\\
70.02	0\\
70.03	0\\
70.04	0\\
70.05	0\\
70.06	0\\
70.07	0\\
70.08	0\\
70.09	0\\
70.1	0\\
70.11	0\\
70.12	0\\
70.13	0\\
70.14	0\\
70.15	0\\
70.16	0\\
70.17	0\\
70.18	0\\
70.19	0\\
70.2	0\\
70.21	0\\
70.22	0\\
70.23	0\\
70.24	0\\
70.25	0\\
70.26	0\\
70.27	0\\
70.28	0\\
70.29	0\\
70.3	0\\
70.31	0\\
70.32	0\\
70.33	0\\
70.34	0\\
70.35	0\\
70.36	0\\
70.37	0\\
70.38	0\\
70.39	0\\
70.4	0\\
70.41	0\\
70.42	0\\
70.43	0\\
70.44	0\\
70.45	0\\
70.46	0\\
70.47	0\\
70.48	0\\
70.49	0\\
70.5	0\\
70.51	0\\
70.52	0\\
70.53	0\\
70.54	0\\
70.55	0\\
70.56	0\\
70.57	0\\
70.58	0\\
70.59	0\\
70.6	0\\
70.61	0\\
70.62	0\\
70.63	0\\
70.64	0\\
70.65	0\\
70.66	0\\
70.67	0\\
70.68	0\\
70.69	0\\
70.7	0\\
70.71	0\\
70.72	0\\
70.73	0\\
70.74	0\\
70.75	0\\
70.76	0\\
70.77	0\\
70.78	0\\
70.79	0\\
70.8	0\\
70.81	0\\
70.82	0\\
70.83	0\\
70.84	0\\
70.85	0\\
70.86	0\\
70.87	0\\
70.88	0\\
70.89	0\\
70.9	0\\
70.91	0\\
70.92	0\\
70.93	0\\
70.94	0\\
70.95	0\\
70.96	0\\
70.97	0\\
70.98	0\\
70.99	0\\
71	0\\
71.01	0\\
71.02	0\\
71.03	0\\
71.04	0\\
71.05	0\\
71.06	0\\
71.07	0\\
71.08	0\\
71.09	0\\
71.1	0\\
71.11	0\\
71.12	0\\
71.13	0\\
71.14	0\\
71.15	0\\
71.16	0\\
71.17	0\\
71.18	0\\
71.19	0\\
71.2	0\\
71.21	0\\
71.22	0\\
71.23	0\\
71.24	0\\
71.25	0\\
71.26	0\\
71.27	0\\
71.28	0\\
71.29	0\\
71.3	0\\
71.31	0\\
71.32	0\\
71.33	0\\
71.34	0\\
71.35	0\\
71.36	0\\
71.37	0\\
71.38	0\\
71.39	0\\
71.4	0\\
71.41	0\\
71.42	0\\
71.43	0\\
71.44	0\\
71.45	0\\
71.46	0\\
71.47	0\\
71.48	0\\
71.49	0\\
71.5	0\\
71.51	0\\
71.52	0\\
71.53	0\\
71.54	0\\
71.55	0\\
71.56	0\\
71.57	0\\
71.58	0\\
71.59	0\\
71.6	0\\
71.61	0\\
71.62	0\\
71.63	0\\
71.64	0\\
71.65	0\\
71.66	0\\
71.67	0\\
71.68	0\\
71.69	0\\
71.7	0\\
71.71	0\\
71.72	0\\
71.73	0\\
71.74	0\\
71.75	0\\
71.76	0\\
71.77	0\\
71.78	0\\
71.79	0\\
71.8	0\\
71.81	0\\
71.82	0\\
71.83	0\\
71.84	0\\
71.85	0\\
71.86	0\\
71.87	0\\
71.88	0\\
71.89	0\\
71.9	0\\
71.91	0\\
71.92	0\\
71.93	0\\
71.94	0\\
71.95	0\\
71.96	0\\
71.97	0\\
71.98	0\\
71.99	0\\
72	0\\
72.01	0\\
72.02	0\\
72.03	0\\
72.04	0\\
72.05	0\\
72.06	0\\
72.07	0\\
72.08	0\\
72.09	0\\
72.1	0\\
72.11	0\\
72.12	0\\
72.13	0\\
72.14	0\\
72.15	0\\
72.16	0\\
72.17	0\\
72.18	0\\
72.19	0\\
72.2	0\\
72.21	0\\
72.22	0\\
72.23	0\\
72.24	0\\
72.25	0\\
72.26	0\\
72.27	0\\
72.28	0\\
72.29	0\\
72.3	0\\
72.31	0\\
72.32	0\\
72.33	0\\
72.34	0\\
72.35	0\\
72.36	0\\
72.37	0\\
72.38	0\\
72.39	0\\
72.4	0\\
72.41	0\\
72.42	0\\
72.43	0\\
72.44	0\\
72.45	0\\
72.46	0\\
72.47	0\\
72.48	0\\
72.49	0\\
72.5	0\\
72.51	0\\
72.52	0\\
72.53	0\\
72.54	0\\
72.55	0\\
72.56	0\\
72.57	0\\
72.58	0\\
72.59	0\\
72.6	0\\
72.61	0\\
72.62	0\\
72.63	0\\
72.64	0\\
72.65	0\\
72.66	0\\
72.67	0\\
72.68	0\\
72.69	0\\
72.7	0\\
72.71	0\\
72.72	0\\
72.73	0\\
72.74	0\\
72.75	0\\
72.76	0\\
72.77	0\\
72.78	0\\
72.79	0\\
72.8	0\\
72.81	0\\
72.82	0\\
72.83	0\\
72.84	0\\
72.85	0\\
72.86	0\\
72.87	0\\
72.88	0\\
72.89	0\\
72.9	0\\
72.91	0\\
72.92	0\\
72.93	0\\
72.94	0\\
72.95	0\\
72.96	0\\
72.97	0\\
72.98	0\\
72.99	0\\
73	0\\
73.01	0\\
73.02	0\\
73.03	0\\
73.04	0\\
73.05	0\\
73.06	0\\
73.07	0\\
73.08	0\\
73.09	0\\
73.1	0\\
73.11	0\\
73.12	0\\
73.13	0\\
73.14	0\\
73.15	0\\
73.16	0\\
73.17	0\\
73.18	0\\
73.19	0\\
73.2	0\\
73.21	0\\
73.22	0\\
73.23	0\\
73.24	0\\
73.25	0\\
73.26	0\\
73.27	0\\
73.28	0\\
73.29	0\\
73.3	0\\
73.31	0\\
73.32	0\\
73.33	0\\
73.34	0\\
73.35	0\\
73.36	0\\
73.37	0\\
73.38	0\\
73.39	0\\
73.4	0\\
73.41	0\\
73.42	0\\
73.43	0\\
73.44	0\\
73.45	0\\
73.46	0\\
73.47	0\\
73.48	0\\
73.49	0\\
73.5	0\\
73.51	0\\
73.52	0\\
73.53	0\\
73.54	0\\
73.55	0\\
73.56	0\\
73.57	0\\
73.58	0\\
73.59	0\\
73.6	0\\
73.61	0\\
73.62	0\\
73.63	0\\
73.64	0\\
73.65	0\\
73.66	0\\
73.67	0\\
73.68	0\\
73.69	0\\
73.7	0\\
73.71	0\\
73.72	0\\
73.73	0\\
73.74	0\\
73.75	0\\
73.76	0\\
73.77	0\\
73.78	0\\
73.79	0\\
73.8	0\\
73.81	0\\
73.82	0\\
73.83	0\\
73.84	0\\
73.85	0\\
73.86	0\\
73.87	0\\
73.88	0\\
73.89	0\\
73.9	0\\
73.91	0\\
73.92	0\\
73.93	0\\
73.94	0\\
73.95	0\\
73.96	0\\
73.97	0\\
73.98	0\\
73.99	0\\
74	0\\
74.01	0\\
74.02	0\\
74.03	0\\
74.04	0\\
74.05	0\\
74.06	0\\
74.07	0\\
74.08	0\\
74.09	0\\
74.1	0\\
74.11	0\\
74.12	0\\
74.13	0\\
74.14	0\\
74.15	0\\
74.16	0\\
74.17	0\\
74.18	0\\
74.19	0\\
74.2	0\\
74.21	0\\
74.22	0\\
74.23	0\\
74.24	0\\
74.25	0\\
74.26	0\\
74.27	0\\
74.28	0\\
74.29	0\\
74.3	0\\
74.31	0\\
74.32	0\\
74.33	0\\
74.34	0\\
74.35	0\\
74.36	0\\
74.37	0\\
74.38	0\\
74.39	0\\
74.4	0\\
74.41	0\\
74.42	0\\
74.43	0\\
74.44	0\\
74.45	0\\
74.46	0\\
74.47	0\\
74.48	0\\
74.49	0\\
74.5	0\\
74.51	0\\
74.52	0\\
74.53	0\\
74.54	0\\
74.55	0\\
74.56	0\\
74.57	0\\
74.58	0\\
74.59	0\\
74.6	0\\
74.61	0\\
74.62	0\\
74.63	0\\
74.64	0\\
74.65	0\\
74.66	0\\
74.67	0\\
74.68	0\\
74.69	0\\
74.7	0\\
74.71	0\\
74.72	0\\
74.73	0\\
74.74	0\\
74.75	0\\
74.76	0\\
74.77	0\\
74.78	0\\
74.79	0\\
74.8	0\\
74.81	0\\
74.82	0\\
74.83	0\\
74.84	0\\
74.85	0\\
74.86	0\\
74.87	0\\
74.88	0\\
74.89	0\\
74.9	0\\
74.91	0\\
74.92	0\\
74.93	0\\
74.94	0\\
74.95	0\\
74.96	0\\
74.97	0\\
74.98	0\\
74.99	0\\
75	0\\
75.01	0\\
75.02	0\\
75.03	0\\
75.04	0\\
75.05	0\\
75.06	0\\
75.07	0\\
75.08	0\\
75.09	0\\
75.1	0\\
75.11	0\\
75.12	0\\
75.13	0\\
75.14	0\\
75.15	0\\
75.16	0\\
75.17	0\\
75.18	0\\
75.19	0\\
75.2	0\\
75.21	0\\
75.22	0\\
75.23	0\\
75.24	0\\
75.25	0\\
75.26	0\\
75.27	0\\
75.28	0\\
75.29	0\\
75.3	0\\
75.31	0\\
75.32	0\\
75.33	0\\
75.34	0\\
75.35	0\\
75.36	0\\
75.37	0\\
75.38	0\\
75.39	0\\
75.4	0\\
75.41	0\\
75.42	0\\
75.43	0\\
75.44	0\\
75.45	0\\
75.46	0\\
75.47	0\\
75.48	0\\
75.49	0\\
75.5	0\\
75.51	0\\
75.52	0\\
75.53	0\\
75.54	0\\
75.55	0\\
75.56	0\\
75.57	0\\
75.58	0\\
75.59	0\\
75.6	0\\
75.61	0\\
75.62	0\\
75.63	0\\
75.64	0\\
75.65	0\\
75.66	0\\
75.67	0\\
75.68	0\\
75.69	0\\
75.7	0\\
75.71	0\\
75.72	0\\
75.73	0\\
75.74	0\\
75.75	0\\
75.76	0\\
75.77	0\\
75.78	0\\
75.79	0\\
75.8	0\\
75.81	0\\
75.82	0\\
75.83	0\\
75.84	0\\
75.85	0\\
75.86	0\\
75.87	0\\
75.88	0\\
75.89	0\\
75.9	0\\
75.91	0\\
75.92	0\\
75.93	0\\
75.94	0\\
75.95	0\\
75.96	0\\
75.97	0\\
75.98	0\\
75.99	0\\
76	0\\
76.01	0\\
76.02	0\\
76.03	0\\
76.04	0\\
76.05	0\\
76.06	0\\
76.07	0\\
76.08	0\\
76.09	0\\
76.1	0\\
76.11	0\\
76.12	0\\
76.13	0\\
76.14	0\\
76.15	0\\
76.16	0\\
76.17	0\\
76.18	0\\
76.19	0\\
76.2	0\\
76.21	0\\
76.22	0\\
76.23	0\\
76.24	0\\
76.25	0\\
76.26	0\\
76.27	0\\
76.28	0\\
76.29	0\\
76.3	0\\
76.31	0\\
76.32	0\\
76.33	0\\
76.34	0\\
76.35	0\\
76.36	0\\
76.37	0\\
76.38	0\\
76.39	0\\
76.4	0\\
76.41	0\\
76.42	0\\
76.43	0\\
76.44	0\\
76.45	0\\
76.46	0\\
76.47	0\\
76.48	0\\
76.49	0\\
76.5	0\\
76.51	0\\
76.52	0\\
76.53	0\\
76.54	0\\
76.55	0\\
76.56	0\\
76.57	0\\
76.58	0\\
76.59	0\\
76.6	0\\
76.61	0\\
76.62	0\\
76.63	0\\
76.64	0\\
76.65	0\\
76.66	0\\
76.67	0\\
76.68	0\\
76.69	0\\
76.7	0\\
76.71	0\\
76.72	0\\
76.73	0\\
76.74	0\\
76.75	0\\
76.76	0\\
76.77	0\\
76.78	0\\
76.79	0\\
76.8	0\\
76.81	0\\
76.82	0\\
76.83	0\\
76.84	0\\
76.85	0\\
76.86	0\\
76.87	0\\
76.88	0\\
76.89	0\\
76.9	0\\
76.91	0\\
76.92	0\\
76.93	0\\
76.94	0\\
76.95	0\\
76.96	0\\
76.97	0\\
76.98	0\\
76.99	0\\
77	0\\
77.01	0\\
77.02	0\\
77.03	0\\
77.04	0\\
77.05	0\\
77.06	0\\
77.07	0\\
77.08	0\\
77.09	0\\
77.1	0\\
77.11	0\\
77.12	0\\
77.13	0\\
77.14	0\\
77.15	0\\
77.16	0\\
77.17	0\\
77.18	0\\
77.19	0\\
77.2	0\\
77.21	0\\
77.22	0\\
77.23	0\\
77.24	0\\
77.25	0\\
77.26	0\\
77.27	0\\
77.28	0\\
77.29	0\\
77.3	0\\
77.31	0\\
77.32	0\\
77.33	0\\
77.34	0\\
77.35	0\\
77.36	0\\
77.37	0\\
77.38	0\\
77.39	0\\
77.4	0\\
77.41	0\\
77.42	0\\
77.43	0\\
77.44	0\\
77.45	0\\
77.46	0\\
77.47	0\\
77.48	0\\
77.49	0\\
77.5	0\\
77.51	0\\
77.52	0\\
77.53	0\\
77.54	0\\
77.55	0\\
77.56	0\\
77.57	0\\
77.58	0\\
77.59	0\\
77.6	0\\
77.61	0\\
77.62	0\\
77.63	0\\
77.64	0\\
77.65	0\\
77.66	0\\
77.67	0\\
77.68	0\\
77.69	0\\
77.7	0\\
77.71	0\\
77.72	0\\
77.73	0\\
77.74	0\\
77.75	0\\
77.76	0\\
77.77	0\\
77.78	0\\
77.79	0\\
77.8	0\\
77.81	0\\
77.82	0\\
77.83	0\\
77.84	0\\
77.85	0\\
77.86	0\\
77.87	0\\
77.88	0\\
77.89	0\\
77.9	0\\
77.91	0\\
77.92	0\\
77.93	0\\
77.94	0\\
77.95	0\\
77.96	0\\
77.97	0\\
77.98	0\\
77.99	0\\
78	0\\
78.01	0\\
78.02	0\\
78.03	0\\
78.04	0\\
78.05	0\\
78.06	0\\
78.07	0\\
78.08	0\\
78.09	0\\
78.1	0\\
78.11	0\\
78.12	0\\
78.13	0\\
78.14	0\\
78.15	0\\
78.16	0\\
78.17	0\\
78.18	0\\
78.19	0\\
78.2	0\\
78.21	0\\
78.22	0\\
78.23	0\\
78.24	0\\
78.25	0\\
78.26	0\\
78.27	0\\
78.28	0\\
78.29	0\\
78.3	0\\
78.31	0\\
78.32	0\\
78.33	0\\
78.34	0\\
78.35	0\\
78.36	0\\
78.37	0\\
78.38	0\\
78.39	0\\
78.4	0\\
78.41	0\\
78.42	0\\
78.43	0\\
78.44	0\\
78.45	0\\
78.46	0\\
78.47	0\\
78.48	0\\
78.49	0\\
78.5	0\\
78.51	0\\
78.52	0\\
78.53	0\\
78.54	0\\
78.55	0\\
78.56	0\\
78.57	0\\
78.58	0\\
78.59	0\\
78.6	0\\
78.61	0\\
78.62	0\\
78.63	0\\
78.64	0\\
78.65	0\\
78.66	0\\
78.67	0\\
78.68	0\\
78.69	0\\
78.7	0\\
78.71	0\\
78.72	0\\
78.73	0\\
78.74	0\\
78.75	0\\
78.76	0\\
78.77	0\\
78.78	0\\
78.79	0\\
78.8	0\\
78.81	0\\
78.82	0\\
78.83	0\\
78.84	0\\
78.85	0\\
78.86	0\\
78.87	0\\
78.88	0\\
78.89	0\\
78.9	0\\
78.91	0\\
78.92	0\\
78.93	0\\
78.94	0\\
78.95	0\\
78.96	0\\
78.97	0\\
78.98	0\\
78.99	0\\
79	0\\
79.01	0\\
79.02	0\\
79.03	0\\
79.04	0\\
79.05	0\\
79.06	0\\
79.07	0\\
79.08	0\\
79.09	0\\
79.1	0\\
79.11	0\\
79.12	0\\
79.13	0\\
79.14	0\\
79.15	0\\
79.16	0\\
79.17	0\\
79.18	0\\
79.19	0\\
79.2	0\\
79.21	0\\
79.22	0\\
79.23	0\\
79.24	0\\
79.25	0\\
79.26	0\\
79.27	0\\
79.28	0\\
79.29	0\\
79.3	0\\
79.31	0\\
79.32	0\\
79.33	0\\
79.34	0\\
79.35	0\\
79.36	0\\
79.37	0\\
79.38	0\\
79.39	0\\
79.4	0\\
79.41	0\\
79.42	0\\
79.43	0\\
79.44	0\\
79.45	0\\
79.46	0\\
79.47	0\\
79.48	0\\
79.49	0\\
79.5	0\\
79.51	0\\
79.52	0\\
79.53	0\\
79.54	0\\
79.55	0\\
79.56	0\\
79.57	0\\
79.58	0\\
79.59	0\\
79.6	0\\
79.61	0\\
79.62	0\\
79.63	0\\
79.64	0\\
79.65	0\\
79.66	0\\
79.67	0\\
79.68	0\\
79.69	0\\
79.7	0\\
79.71	0\\
79.72	0\\
79.73	0\\
79.74	0\\
79.75	0\\
79.76	0\\
79.77	0\\
79.78	0\\
79.79	0\\
79.8	0\\
79.81	0\\
79.82	0\\
79.83	0\\
79.84	0\\
79.85	0\\
79.86	0\\
79.87	0\\
79.88	0\\
79.89	0\\
79.9	0\\
79.91	0\\
79.92	0\\
79.93	0\\
79.94	0\\
79.95	0\\
79.96	0\\
79.97	0\\
79.98	0\\
79.99	0\\
80	0\\
80.01	0\\
};
\addplot [color=blue,dashed]
  table[row sep=crcr]{%
80.01	0\\
80.02	0\\
80.03	0\\
80.04	0\\
80.05	0\\
80.06	0\\
80.07	0\\
80.08	0\\
80.09	0\\
80.1	0\\
80.11	0\\
80.12	0\\
80.13	0\\
80.14	0\\
80.15	0\\
80.16	0\\
80.17	0\\
80.18	0\\
80.19	0\\
80.2	0\\
80.21	0\\
80.22	0\\
80.23	0\\
80.24	0\\
80.25	0\\
80.26	0\\
80.27	0\\
80.28	0\\
80.29	0\\
80.3	0\\
80.31	0\\
80.32	0\\
80.33	0\\
80.34	0\\
80.35	0\\
80.36	0\\
80.37	0\\
80.38	0\\
80.39	0\\
80.4	0\\
80.41	0\\
80.42	0\\
80.43	0\\
80.44	0\\
80.45	0\\
80.46	0\\
80.47	0\\
80.48	0\\
80.49	0\\
80.5	0\\
80.51	0\\
80.52	0\\
80.53	0\\
80.54	0\\
80.55	0\\
80.56	0\\
80.57	0\\
80.58	0\\
80.59	0\\
80.6	0\\
80.61	0\\
80.62	0\\
80.63	0\\
80.64	0\\
80.65	0\\
80.66	0\\
80.67	0\\
80.68	0\\
80.69	0\\
80.7	0\\
80.71	0\\
80.72	0\\
80.73	0\\
80.74	0\\
80.75	0\\
80.76	0\\
80.77	0\\
80.78	0\\
80.79	0\\
80.8	0\\
80.81	0\\
80.82	0\\
80.83	0\\
80.84	0\\
80.85	0\\
80.86	0\\
80.87	0\\
80.88	0\\
80.89	0\\
80.9	0\\
80.91	0\\
80.92	0\\
80.93	0\\
80.94	0\\
80.95	0\\
80.96	0\\
80.97	0\\
80.98	0\\
80.99	0\\
81	0\\
81.01	0\\
81.02	0\\
81.03	0\\
81.04	0\\
81.05	0\\
81.06	0\\
81.07	0\\
81.08	0\\
81.09	0\\
81.1	0\\
81.11	0\\
81.12	0\\
81.13	0\\
81.14	0\\
81.15	0\\
81.16	0\\
81.17	0\\
81.18	0\\
81.19	0\\
81.2	0\\
81.21	0\\
81.22	0\\
81.23	0\\
81.24	0\\
81.25	0\\
81.26	0\\
81.27	0\\
81.28	0\\
81.29	0\\
81.3	0\\
81.31	0\\
81.32	0\\
81.33	0\\
81.34	0\\
81.35	0\\
81.36	0\\
81.37	0\\
81.38	0\\
81.39	0\\
81.4	0\\
81.41	0\\
81.42	0\\
81.43	0\\
81.44	0\\
81.45	0\\
81.46	0\\
81.47	0\\
81.48	0\\
81.49	0\\
81.5	0\\
81.51	0\\
81.52	0\\
81.53	0\\
81.54	0\\
81.55	0\\
81.56	0\\
81.57	0\\
81.58	0\\
81.59	0\\
81.6	0\\
81.61	0\\
81.62	0\\
81.63	0\\
81.64	0\\
81.65	0\\
81.66	0\\
81.67	0\\
81.68	0\\
81.69	0\\
81.7	0\\
81.71	0\\
81.72	0\\
81.73	0\\
81.74	0\\
81.75	0\\
81.76	0\\
81.77	0\\
81.78	0\\
81.79	0\\
81.8	0\\
81.81	0\\
81.82	0\\
81.83	0\\
81.84	0\\
81.85	0\\
81.86	0\\
81.87	0\\
81.88	0\\
81.89	0\\
81.9	0\\
81.91	0\\
81.92	0\\
81.93	0\\
81.94	0\\
81.95	0\\
81.96	0\\
81.97	0\\
81.98	0\\
81.99	0\\
82	0\\
82.01	0\\
82.02	0\\
82.03	0\\
82.04	0\\
82.05	0\\
82.06	0\\
82.07	0\\
82.08	0\\
82.09	0\\
82.1	0\\
82.11	0\\
82.12	0\\
82.13	0\\
82.14	0\\
82.15	0\\
82.16	0\\
82.17	0\\
82.18	0\\
82.19	0\\
82.2	0\\
82.21	0\\
82.22	0\\
82.23	0\\
82.24	0\\
82.25	0\\
82.26	0\\
82.27	0\\
82.28	0\\
82.29	0\\
82.3	0\\
82.31	0\\
82.32	0\\
82.33	0\\
82.34	0\\
82.35	0\\
82.36	0\\
82.37	0\\
82.38	0\\
82.39	0\\
82.4	0\\
82.41	0\\
82.42	0\\
82.43	0\\
82.44	0\\
82.45	0\\
82.46	0\\
82.47	0\\
82.48	0\\
82.49	0\\
82.5	0\\
82.51	0\\
82.52	0\\
82.53	0\\
82.54	0\\
82.55	0\\
82.56	0\\
82.57	0\\
82.58	0\\
82.59	0\\
82.6	0\\
82.61	0\\
82.62	0\\
82.63	0\\
82.64	0\\
82.65	0\\
82.66	0\\
82.67	0\\
82.68	0\\
82.69	0\\
82.7	0\\
82.71	0\\
82.72	0\\
82.73	0\\
82.74	0\\
82.75	0\\
82.76	0\\
82.77	0\\
82.78	0\\
82.79	0\\
82.8	0\\
82.81	0\\
82.82	0\\
82.83	0\\
82.84	0\\
82.85	0\\
82.86	0\\
82.87	0\\
82.88	0\\
82.89	0\\
82.9	0\\
82.91	0\\
82.92	0\\
82.93	0\\
82.94	0\\
82.95	0\\
82.96	0\\
82.97	0\\
82.98	0\\
82.99	0\\
83	0\\
83.01	0\\
83.02	0\\
83.03	0\\
83.04	0\\
83.05	0\\
83.06	0\\
83.07	0\\
83.08	0\\
83.09	0\\
83.1	0\\
83.11	0\\
83.12	0\\
83.13	0\\
83.14	0\\
83.15	0\\
83.16	0\\
83.17	0\\
83.18	0\\
83.19	0\\
83.2	0\\
83.21	0\\
83.22	0\\
83.23	0\\
83.24	0\\
83.25	0\\
83.26	0\\
83.27	0\\
83.28	0\\
83.29	0\\
83.3	0\\
83.31	0\\
83.32	0\\
83.33	0\\
83.34	0\\
83.35	0\\
83.36	0\\
83.37	0\\
83.38	0\\
83.39	0\\
83.4	0\\
83.41	0\\
83.42	0\\
83.43	0\\
83.44	0\\
83.45	0\\
83.46	0\\
83.47	0\\
83.48	0\\
83.49	0\\
83.5	0\\
83.51	0\\
83.52	0\\
83.53	0\\
83.54	0\\
83.55	0\\
83.56	0\\
83.57	0\\
83.58	0\\
83.59	0\\
83.6	0\\
83.61	0\\
83.62	0\\
83.63	0\\
83.64	0\\
83.65	0\\
83.66	0\\
83.67	0\\
83.68	0\\
83.69	0\\
83.7	0\\
83.71	0\\
83.72	0\\
83.73	0\\
83.74	0\\
83.75	0\\
83.76	0\\
83.77	0\\
83.78	0\\
83.79	0\\
83.8	0\\
83.81	0\\
83.82	0\\
83.83	0\\
83.84	0\\
83.85	0\\
83.86	0\\
83.87	0\\
83.88	0\\
83.89	0\\
83.9	0\\
83.91	0\\
83.92	0\\
83.93	0\\
83.94	0\\
83.95	0\\
83.96	0\\
83.97	0\\
83.98	0\\
83.99	0\\
84	0\\
84.01	0\\
84.02	0\\
84.03	0\\
84.04	0\\
84.05	0\\
84.06	0\\
84.07	0\\
84.08	0\\
84.09	0\\
84.1	0\\
84.11	0\\
84.12	0\\
84.13	0\\
84.14	0\\
84.15	0\\
84.16	0\\
84.17	0\\
84.18	0\\
84.19	0\\
84.2	0\\
84.21	0\\
84.22	0\\
84.23	0\\
84.24	0\\
84.25	0\\
84.26	0\\
84.27	0\\
84.28	0\\
84.29	0\\
84.3	0\\
84.31	0\\
84.32	0\\
84.33	0\\
84.34	0\\
84.35	0\\
84.36	0\\
84.37	0\\
84.38	0\\
84.39	0\\
84.4	0\\
84.41	0\\
84.42	0\\
84.43	0\\
84.44	0\\
84.45	0\\
84.46	0\\
84.47	0\\
84.48	0\\
84.49	0\\
84.5	0\\
84.51	0\\
84.52	0\\
84.53	0\\
84.54	0\\
84.55	0\\
84.56	0\\
84.57	0\\
84.58	0\\
84.59	0\\
84.6	0\\
84.61	0\\
84.62	0\\
84.63	0\\
84.64	0\\
84.65	0\\
84.66	0\\
84.67	0\\
84.68	0\\
84.69	0\\
84.7	0\\
84.71	0\\
84.72	0\\
84.73	0\\
84.74	0\\
84.75	0\\
84.76	0\\
84.77	0\\
84.78	0\\
84.79	0\\
84.8	0\\
84.81	0\\
84.82	0\\
84.83	0\\
84.84	0\\
84.85	0\\
84.86	0\\
84.87	0\\
84.88	0\\
84.89	0\\
84.9	0\\
84.91	0\\
84.92	0\\
84.93	0\\
84.94	0\\
84.95	0\\
84.96	0\\
84.97	0\\
84.98	0\\
84.99	0\\
85	0\\
85.01	0\\
85.02	0\\
85.03	0\\
85.04	0\\
85.05	0\\
85.06	0\\
85.07	0\\
85.08	0\\
85.09	0\\
85.1	0\\
85.11	0\\
85.12	0\\
85.13	0\\
85.14	0\\
85.15	0\\
85.16	0\\
85.17	0\\
85.18	0\\
85.19	0\\
85.2	0\\
85.21	0\\
85.22	0\\
85.23	0\\
85.24	0\\
85.25	0\\
85.26	0\\
85.27	0\\
85.28	0\\
85.29	0\\
85.3	0\\
85.31	0\\
85.32	0\\
85.33	0\\
85.34	0\\
85.35	0\\
85.36	0\\
85.37	0\\
85.38	0\\
85.39	0\\
85.4	0\\
85.41	0\\
85.42	0\\
85.43	0\\
85.44	0\\
85.45	0\\
85.46	0\\
85.47	0\\
85.48	0\\
85.49	0\\
85.5	0\\
85.51	0\\
85.52	0\\
85.53	0\\
85.54	0\\
85.55	0\\
85.56	0\\
85.57	0\\
85.58	0\\
85.59	0\\
85.6	0\\
85.61	0\\
85.62	0\\
85.63	0\\
85.64	0\\
85.65	0\\
85.66	0\\
85.67	0\\
85.68	0\\
85.69	0\\
85.7	0\\
85.71	0\\
85.72	0\\
85.73	0\\
85.74	0\\
85.75	0\\
85.76	0\\
85.77	0\\
85.78	0\\
85.79	0\\
85.8	0\\
85.81	0\\
85.82	0\\
85.83	0\\
85.84	0\\
85.85	0\\
85.86	0\\
85.87	0\\
85.88	0\\
85.89	0\\
85.9	0\\
85.91	0\\
85.92	0\\
85.93	0\\
85.94	0\\
85.95	0\\
85.96	0\\
85.97	0\\
85.98	0\\
85.99	0\\
86	0\\
86.01	0\\
86.02	0\\
86.03	0\\
86.04	0\\
86.05	0\\
86.06	0\\
86.07	0\\
86.08	0\\
86.09	0\\
86.1	0\\
86.11	0\\
86.12	0\\
86.13	0\\
86.14	0\\
86.15	0\\
86.16	0\\
86.17	0\\
86.18	0\\
86.19	0\\
86.2	0\\
86.21	0\\
86.22	0\\
86.23	0\\
86.24	0\\
86.25	0\\
86.26	0\\
86.27	0\\
86.28	0\\
86.29	0\\
86.3	0\\
86.31	0\\
86.32	0\\
86.33	0\\
86.34	0\\
86.35	0\\
86.36	0\\
86.37	0\\
86.38	0\\
86.39	0\\
86.4	0\\
86.41	0\\
86.42	0\\
86.43	0\\
86.44	0\\
86.45	0\\
86.46	0\\
86.47	0\\
86.48	0\\
86.49	0\\
86.5	0\\
86.51	0\\
86.52	0\\
86.53	0\\
86.54	0\\
86.55	0\\
86.56	0\\
86.57	0\\
86.58	0\\
86.59	0\\
86.6	0\\
86.61	0\\
86.62	0\\
86.63	0\\
86.64	0\\
86.65	0\\
86.66	0\\
86.67	0\\
86.68	0\\
86.69	0\\
86.7	0\\
86.71	0\\
86.72	0\\
86.73	0\\
86.74	0\\
86.75	0\\
86.76	0\\
86.77	0\\
86.78	0\\
86.79	0\\
86.8	0\\
86.81	0\\
86.82	0\\
86.83	0\\
86.84	0\\
86.85	0\\
86.86	0\\
86.87	0\\
86.88	0\\
86.89	0\\
86.9	0\\
86.91	0\\
86.92	0\\
86.93	0\\
86.94	0\\
86.95	0\\
86.96	0\\
86.97	0\\
86.98	0\\
86.99	0\\
87	0\\
87.01	0\\
87.02	0\\
87.03	0\\
87.04	0\\
87.05	0\\
87.06	0\\
87.07	0\\
87.08	0\\
87.09	0\\
87.1	0\\
87.11	0\\
87.12	0\\
87.13	0\\
87.14	0\\
87.15	0\\
87.16	0\\
87.17	0\\
87.18	0\\
87.19	0\\
87.2	0\\
87.21	0\\
87.22	0\\
87.23	0\\
87.24	0\\
87.25	0\\
87.26	0\\
87.27	0\\
87.28	0\\
87.29	0\\
87.3	0\\
87.31	0\\
87.32	0\\
87.33	0\\
87.34	0\\
87.35	0\\
87.36	0\\
87.37	0\\
87.38	0\\
87.39	0\\
87.4	0\\
87.41	0\\
87.42	0\\
87.43	0\\
87.44	0\\
87.45	0\\
87.46	0\\
87.47	0\\
87.48	0\\
87.49	0\\
87.5	0\\
87.51	0\\
87.52	0\\
87.53	0\\
87.54	0\\
87.55	0\\
87.56	0\\
87.57	0\\
87.58	0\\
87.59	0\\
87.6	0\\
87.61	0\\
87.62	0\\
87.63	0\\
87.64	0\\
87.65	0\\
87.66	0\\
87.67	0\\
87.68	0\\
87.69	0\\
87.7	0\\
87.71	0\\
87.72	0\\
87.73	0\\
87.74	0\\
87.75	0\\
87.76	0\\
87.77	0\\
87.78	0\\
87.79	0\\
87.8	0\\
87.81	0\\
87.82	0\\
87.83	0\\
87.84	0\\
87.85	0\\
87.86	0\\
87.87	0\\
87.88	0\\
87.89	0\\
87.9	0\\
87.91	0\\
87.92	0\\
87.93	0\\
87.94	0\\
87.95	0\\
87.96	0\\
87.97	0\\
87.98	0\\
87.99	0\\
88	0\\
88.01	0\\
88.02	0\\
88.03	0\\
88.04	0\\
88.05	0\\
88.06	0\\
88.07	0\\
88.08	0\\
88.09	0\\
88.1	0\\
88.11	0\\
88.12	0\\
88.13	0\\
88.14	0\\
88.15	0\\
88.16	0\\
88.17	0\\
88.18	0\\
88.19	0\\
88.2	0\\
88.21	0\\
88.22	0\\
88.23	0\\
88.24	0\\
88.25	0\\
88.26	0\\
88.27	0\\
88.28	0\\
88.29	0\\
88.3	0\\
88.31	0\\
88.32	0\\
88.33	0\\
88.34	0\\
88.35	0\\
88.36	0\\
88.37	0\\
88.38	0\\
88.39	0\\
88.4	0\\
88.41	0\\
88.42	0\\
88.43	0\\
88.44	0\\
88.45	0\\
88.46	0\\
88.47	0\\
88.48	0\\
88.49	0\\
88.5	0\\
88.51	0\\
88.52	0\\
88.53	0\\
88.54	0\\
88.55	0\\
88.56	0\\
88.57	0\\
88.58	0\\
88.59	0\\
88.6	0\\
88.61	0\\
88.62	0\\
88.63	0\\
88.64	0\\
88.65	0\\
88.66	0\\
88.67	0\\
88.68	0\\
88.69	0\\
88.7	0\\
88.71	0\\
88.72	0\\
88.73	0\\
88.74	0\\
88.75	0\\
88.76	0\\
88.77	0\\
88.78	0\\
88.79	0\\
88.8	0\\
88.81	0\\
88.82	0\\
88.83	0\\
88.84	0\\
88.85	0\\
88.86	0\\
88.87	0\\
88.88	0\\
88.89	0\\
88.9	0\\
88.91	0\\
88.92	0\\
88.93	0\\
88.94	0\\
88.95	0\\
88.96	0\\
88.97	0\\
88.98	0\\
88.99	0\\
89	0\\
89.01	0\\
89.02	0\\
89.03	0\\
89.04	0\\
89.05	0\\
89.06	0\\
89.07	0\\
89.08	0\\
89.09	0\\
89.1	0\\
89.11	0\\
89.12	0\\
89.13	0\\
89.14	0\\
89.15	0\\
89.16	0\\
89.17	0\\
89.18	0\\
89.19	0\\
89.2	0\\
89.21	0\\
89.22	0\\
89.23	0\\
89.24	0\\
89.25	0\\
89.26	0\\
89.27	0\\
89.28	0\\
89.29	0\\
89.3	0\\
89.31	0\\
89.32	0\\
89.33	0\\
89.34	0\\
89.35	0\\
89.36	0\\
89.37	0\\
89.38	0\\
89.39	0\\
89.4	0\\
89.41	0\\
89.42	0\\
89.43	0\\
89.44	0\\
89.45	0\\
89.46	0\\
89.47	0\\
89.48	0\\
89.49	0\\
89.5	0\\
89.51	0\\
89.52	0\\
89.53	0\\
89.54	0\\
89.55	0\\
89.56	0\\
89.57	0\\
89.58	0\\
89.59	0\\
89.6	0\\
89.61	0\\
89.62	0\\
89.63	0\\
89.64	0\\
89.65	0\\
89.66	0\\
89.67	0\\
89.68	0\\
89.69	0\\
89.7	0\\
89.71	0\\
89.72	0\\
89.73	0\\
89.74	0\\
89.75	0\\
89.76	0\\
89.77	0\\
89.78	0\\
89.79	0\\
89.8	0\\
89.81	0\\
89.82	0\\
89.83	0\\
89.84	0\\
89.85	0\\
89.86	0\\
89.87	0\\
89.88	0\\
89.89	0\\
89.9	0\\
89.91	0\\
89.92	0\\
89.93	0\\
89.94	0\\
89.95	0\\
89.96	0\\
89.97	0\\
89.98	0\\
89.99	0\\
90	0\\
90.01	0\\
90.02	0\\
90.03	0\\
90.04	0\\
90.05	0\\
90.06	0\\
90.07	0\\
90.08	0\\
90.09	0\\
90.1	0\\
90.11	0\\
90.12	0\\
90.13	0\\
90.14	0\\
90.15	0\\
90.16	0\\
90.17	0\\
90.18	0\\
90.19	0\\
90.2	0\\
90.21	0\\
90.22	0\\
90.23	0\\
90.24	0\\
90.25	0\\
90.26	0\\
90.27	0\\
90.28	0\\
90.29	0\\
90.3	0\\
90.31	0\\
90.32	0\\
90.33	0\\
90.34	0\\
90.35	0\\
90.36	0\\
90.37	0\\
90.38	0\\
90.39	0\\
90.4	0\\
90.41	0\\
90.42	0\\
90.43	0\\
90.44	0\\
90.45	0\\
90.46	0\\
90.47	0\\
90.48	0\\
90.49	0\\
90.5	0\\
90.51	0\\
90.52	0\\
90.53	0\\
90.54	0\\
90.55	0\\
90.56	0\\
90.57	0\\
90.58	0\\
90.59	0\\
90.6	0\\
90.61	0\\
90.62	0\\
90.63	0\\
90.64	0\\
90.65	0\\
90.66	0\\
90.67	0\\
90.68	0\\
90.69	0\\
90.7	0\\
90.71	0\\
90.72	0\\
90.73	0\\
90.74	0\\
90.75	0\\
90.76	0\\
90.77	0\\
90.78	0\\
90.79	0\\
90.8	0\\
90.81	0\\
90.82	0\\
90.83	0\\
90.84	0\\
90.85	0\\
90.86	0\\
90.87	0\\
90.88	0\\
90.89	0\\
90.9	0\\
90.91	0\\
90.92	0\\
90.93	0\\
90.94	0\\
90.95	0\\
90.96	0\\
90.97	0\\
90.98	0\\
90.99	0\\
91	0\\
91.01	0\\
91.02	0\\
91.03	0\\
91.04	0\\
91.05	0\\
91.06	0\\
91.07	0\\
91.08	0\\
91.09	0\\
91.1	0\\
91.11	0\\
91.12	0\\
91.13	0\\
91.14	0\\
91.15	0\\
91.16	0\\
91.17	0\\
91.18	0\\
91.19	0\\
91.2	0\\
91.21	0\\
91.22	0\\
91.23	0\\
91.24	0\\
91.25	0\\
91.26	0\\
91.27	0\\
91.28	0\\
91.29	0\\
91.3	0\\
91.31	0\\
91.32	0\\
91.33	0\\
91.34	0\\
91.35	0\\
91.36	0\\
91.37	0\\
91.38	0\\
91.39	0\\
91.4	0\\
91.41	0\\
91.42	0\\
91.43	0\\
91.44	0\\
91.45	0\\
91.46	0\\
91.47	0\\
91.48	0\\
91.49	0\\
91.5	0\\
91.51	0\\
91.52	0\\
91.53	0\\
91.54	0\\
91.55	0\\
91.56	0\\
91.57	0\\
91.58	0\\
91.59	0\\
91.6	0\\
91.61	0\\
91.62	0\\
91.63	0\\
91.64	0\\
91.65	0\\
91.66	0\\
91.67	0\\
91.68	0\\
91.69	0\\
91.7	0\\
91.71	0\\
91.72	0\\
91.73	0\\
91.74	0\\
91.75	0\\
91.76	0\\
91.77	0\\
91.78	0\\
91.79	0\\
91.8	0\\
91.81	0\\
91.82	0\\
91.83	0\\
91.84	0\\
91.85	0\\
91.86	0\\
91.87	0\\
91.88	0\\
91.89	0\\
91.9	0\\
91.91	0\\
91.92	0\\
91.93	0\\
91.94	0\\
91.95	0\\
91.96	0\\
91.97	0\\
91.98	0\\
91.99	0\\
92	0\\
92.01	0\\
92.02	0\\
92.03	0\\
92.04	0\\
92.05	0\\
92.06	0\\
92.07	0\\
92.08	0\\
92.09	0\\
92.1	0\\
92.11	0\\
92.12	0\\
92.13	0\\
92.14	0\\
92.15	0\\
92.16	0\\
92.17	0\\
92.18	0\\
92.19	0\\
92.2	0\\
92.21	0\\
92.22	0\\
92.23	0\\
92.24	0\\
92.25	0\\
92.26	0\\
92.27	0\\
92.28	0\\
92.29	0\\
92.3	0\\
92.31	0\\
92.32	0\\
92.33	0\\
92.34	0\\
92.35	0\\
92.36	0\\
92.37	0\\
92.38	0\\
92.39	0\\
92.4	0\\
92.41	0\\
92.42	0\\
92.43	0\\
92.44	0\\
92.45	0\\
92.46	0\\
92.47	0\\
92.48	0\\
92.49	0\\
92.5	0\\
92.51	0\\
92.52	0\\
92.53	0\\
92.54	0\\
92.55	0\\
92.56	0\\
92.57	0\\
92.58	0\\
92.59	0\\
92.6	0\\
92.61	0\\
92.62	0\\
92.63	0\\
92.64	0\\
92.65	0\\
92.66	0\\
92.67	0\\
92.68	0\\
92.69	0\\
92.7	0\\
92.71	0\\
92.72	0\\
92.73	0\\
92.74	0\\
92.75	0\\
92.76	0\\
92.77	0\\
92.78	0\\
92.79	0\\
92.8	0\\
92.81	0\\
92.82	0\\
92.83	0\\
92.84	0\\
92.85	0\\
92.86	0\\
92.87	0\\
92.88	0\\
92.89	0\\
92.9	0\\
92.91	0\\
92.92	0\\
92.93	0\\
92.94	0\\
92.95	0\\
92.96	0\\
92.97	0\\
92.98	0\\
92.99	0\\
93	0\\
93.01	0\\
93.02	0\\
93.03	0\\
93.04	0\\
93.05	0\\
93.06	0\\
93.07	0\\
93.08	0\\
93.09	0\\
93.1	0\\
93.11	0\\
93.12	0\\
93.13	0\\
93.14	0\\
93.15	0\\
93.16	0\\
93.17	0\\
93.18	0\\
93.19	0\\
93.2	0\\
93.21	0\\
93.22	0\\
93.23	0\\
93.24	0\\
93.25	0\\
93.26	0\\
93.27	0\\
93.28	0\\
93.29	0\\
93.3	0\\
93.31	0\\
93.32	0\\
93.33	0\\
93.34	0\\
93.35	0\\
93.36	0\\
93.37	0\\
93.38	0\\
93.39	0\\
93.4	0\\
93.41	0\\
93.42	0\\
93.43	0\\
93.44	0\\
93.45	0\\
93.46	0\\
93.47	0\\
93.48	0\\
93.49	0\\
93.5	0\\
93.51	0\\
93.52	0\\
93.53	0\\
93.54	0\\
93.55	0\\
93.56	0\\
93.57	0\\
93.58	0\\
93.59	0\\
93.6	0\\
93.61	0\\
93.62	0\\
93.63	0\\
93.64	0\\
93.65	0\\
93.66	0\\
93.67	0\\
93.68	0\\
93.69	0\\
93.7	0\\
93.71	0\\
93.72	0\\
93.73	0\\
93.74	0\\
93.75	0\\
93.76	0\\
93.77	0\\
93.78	0\\
93.79	0\\
93.8	0\\
93.81	0\\
93.82	0\\
93.83	0\\
93.84	0\\
93.85	0\\
93.86	0\\
93.87	0\\
93.88	0\\
93.89	0\\
93.9	0\\
93.91	0\\
93.92	0\\
93.93	0\\
93.94	0\\
93.95	0\\
93.96	0\\
93.97	0\\
93.98	0\\
93.99	0\\
94	0\\
94.01	0\\
94.02	0\\
94.03	0\\
94.04	0\\
94.05	0\\
94.06	0\\
94.07	0\\
94.08	0\\
94.09	0\\
94.1	0\\
94.11	0\\
94.12	0\\
94.13	0\\
94.14	0\\
94.15	0\\
94.16	0\\
94.17	0\\
94.18	0\\
94.19	0\\
94.2	0\\
94.21	0\\
94.22	0\\
94.23	0\\
94.24	0\\
94.25	0\\
94.26	0\\
94.27	0\\
94.28	0\\
94.29	0\\
94.3	0\\
94.31	0\\
94.32	0\\
94.33	0\\
94.34	0\\
94.35	0\\
94.36	0\\
94.37	0\\
94.38	0\\
94.39	0\\
94.4	0\\
94.41	0\\
94.42	0\\
94.43	0\\
94.44	0\\
94.45	0\\
94.46	0\\
94.47	0\\
94.48	0\\
94.49	0\\
94.5	0\\
94.51	0\\
94.52	0\\
94.53	0\\
94.54	0\\
94.55	0\\
94.56	0\\
94.57	0\\
94.58	0\\
94.59	0\\
94.6	0\\
94.61	0\\
94.62	0\\
94.63	0\\
94.64	0\\
94.65	0\\
94.66	0\\
94.67	0\\
94.68	0\\
94.69	0\\
94.7	0\\
94.71	0\\
94.72	0\\
94.73	0\\
94.74	0\\
94.75	0\\
94.76	0\\
94.77	0\\
94.78	0\\
94.79	0\\
94.8	0\\
94.81	0\\
94.82	0\\
94.83	0\\
94.84	0\\
94.85	0\\
94.86	0\\
94.87	0\\
94.88	0\\
94.89	0\\
94.9	0\\
94.91	0\\
94.92	0\\
94.93	0\\
94.94	0\\
94.95	0\\
94.96	0\\
94.97	0\\
94.98	0\\
94.99	0\\
95	0\\
95.01	0\\
95.02	0\\
95.03	0\\
95.04	0\\
95.05	0\\
95.06	0\\
95.07	0\\
95.08	0\\
95.09	0\\
95.1	0\\
95.11	0\\
95.12	0\\
95.13	0\\
95.14	0\\
95.15	0\\
95.16	0\\
95.17	0\\
95.18	0\\
95.19	0\\
95.2	0\\
95.21	0\\
95.22	0\\
95.23	0\\
95.24	0\\
95.25	0\\
95.26	0\\
95.27	0\\
95.28	0\\
95.29	0\\
95.3	0\\
95.31	0\\
95.32	0\\
95.33	0\\
95.34	0\\
95.35	0\\
95.36	0\\
95.37	0\\
95.38	0\\
95.39	0\\
95.4	0\\
95.41	0\\
95.42	0\\
95.43	0\\
95.44	0\\
95.45	0\\
95.46	0\\
95.47	0\\
95.48	0\\
95.49	0\\
95.5	0\\
95.51	0\\
95.52	0\\
95.53	0\\
95.54	0\\
95.55	0\\
95.56	0\\
95.57	0\\
95.58	0\\
95.59	0\\
95.6	0\\
95.61	0\\
95.62	0\\
95.63	0\\
95.64	0\\
95.65	0\\
95.66	0\\
95.67	0\\
95.68	0\\
95.69	0\\
95.7	0\\
95.71	0\\
95.72	0\\
95.73	0\\
95.74	0\\
95.75	0\\
95.76	0\\
95.77	0\\
95.78	0\\
95.79	0\\
95.8	0\\
95.81	0\\
95.82	0\\
95.83	0\\
95.84	0\\
95.85	0\\
95.86	0\\
95.87	0\\
95.88	0\\
95.89	0\\
95.9	0\\
95.91	0\\
95.92	0\\
95.93	0\\
95.94	0\\
95.95	0\\
95.96	0\\
95.97	0\\
95.98	0\\
95.99	0\\
96	0\\
96.01	0\\
96.02	0\\
96.03	0\\
96.04	0\\
96.05	0\\
96.06	0\\
96.07	0\\
96.08	0\\
96.09	0\\
96.1	0\\
96.11	0\\
96.12	0\\
96.13	0\\
96.14	0\\
96.15	0\\
96.16	0\\
96.17	0\\
96.18	0\\
96.19	0\\
96.2	0\\
96.21	0\\
96.22	0\\
96.23	0\\
96.24	0\\
96.25	0\\
96.26	0\\
96.27	0\\
96.28	0\\
96.29	0\\
96.3	0\\
96.31	0\\
96.32	0\\
96.33	0\\
96.34	0\\
96.35	0\\
96.36	0\\
96.37	0\\
96.38	0\\
96.39	0\\
96.4	0\\
96.41	0\\
96.42	0\\
96.43	0\\
96.44	0\\
96.45	0\\
96.46	0\\
96.47	0\\
96.48	0\\
96.49	0\\
96.5	0\\
96.51	0\\
96.52	0\\
96.53	0\\
96.54	0\\
96.55	0\\
96.56	0\\
96.57	0\\
96.58	0\\
96.59	0\\
96.6	0\\
96.61	0\\
96.62	0\\
96.63	0\\
96.64	0\\
96.65	0\\
96.66	0\\
96.67	0\\
96.68	0\\
96.69	0\\
96.7	0\\
96.71	0\\
96.72	0\\
96.73	0\\
96.74	0\\
96.75	0\\
96.76	0\\
96.77	0\\
96.78	0\\
96.79	0\\
96.8	0\\
96.81	0\\
96.82	0\\
96.83	0\\
96.84	0\\
96.85	0\\
96.86	0\\
96.87	0\\
96.88	0\\
96.89	0\\
96.9	0\\
96.91	0\\
96.92	0\\
96.93	0\\
96.94	0\\
96.95	0\\
96.96	0\\
96.97	0\\
96.98	0\\
96.99	0\\
97	0\\
97.01	0\\
97.02	0\\
97.03	0\\
97.04	0\\
97.05	0\\
97.06	0\\
97.07	0\\
97.08	0\\
97.09	0\\
97.1	0\\
97.11	0\\
97.12	0\\
97.13	0\\
97.14	0\\
97.15	0\\
97.16	0\\
97.17	0\\
97.18	0\\
97.19	0\\
97.2	0\\
97.21	0\\
97.22	0\\
97.23	0\\
97.24	0\\
97.25	0\\
97.26	0\\
97.27	0\\
97.28	0\\
97.29	0\\
97.3	0\\
97.31	0\\
97.32	0\\
97.33	0\\
97.34	0\\
97.35	0\\
97.36	0\\
97.37	0\\
97.38	0\\
97.39	0\\
97.4	0\\
97.41	0\\
97.42	0\\
97.43	0\\
97.44	0\\
97.45	0\\
97.46	0\\
97.47	0\\
97.48	0\\
97.49	0\\
97.5	0\\
97.51	0\\
97.52	0\\
97.53	0\\
97.54	0\\
97.55	0\\
97.56	0\\
97.57	0\\
97.58	0\\
97.59	0\\
97.6	0\\
97.61	0\\
97.62	0\\
97.63	0\\
97.64	0\\
97.65	0\\
97.66	0\\
97.67	0\\
97.68	0\\
97.69	0\\
97.7	0\\
97.71	0\\
97.72	0\\
97.73	0\\
97.74	0\\
97.75	0\\
97.76	0\\
97.77	0\\
97.78	0\\
97.79	0\\
97.8	0\\
97.81	0\\
97.82	0\\
97.83	0\\
97.84	0\\
97.85	0\\
97.86	0\\
97.87	0\\
97.88	0\\
97.89	0\\
97.9	0\\
97.91	0\\
97.92	0\\
97.93	0\\
97.94	0\\
97.95	0\\
97.96	0\\
97.97	0\\
97.98	0\\
97.99	0\\
98	0\\
98.01	0\\
98.02	0\\
98.03	0\\
98.04	0\\
98.05	0\\
98.06	0\\
98.07	0\\
98.08	0\\
98.09	0\\
98.1	0\\
98.11	0\\
98.12	0\\
98.13	0\\
98.14	0\\
98.15	0\\
98.16	0\\
98.17	0\\
98.18	0\\
98.19	0\\
98.2	0\\
98.21	0\\
98.22	0\\
98.23	0\\
98.24	0\\
98.25	0\\
98.26	0\\
98.27	0\\
98.28	0\\
98.29	0\\
98.3	0\\
98.31	0\\
98.32	0\\
98.33	0\\
98.34	0\\
98.35	0\\
98.36	0\\
98.37	0\\
98.38	0\\
98.39	0\\
98.4	0\\
98.41	0\\
98.42	0\\
98.43	0\\
98.44	0\\
98.45	0\\
98.46	0\\
98.47	0\\
98.48	0\\
98.49	0\\
98.5	0\\
98.51	0\\
98.52	0\\
98.53	0\\
98.54	0\\
98.55	0\\
98.56	0\\
98.57	0\\
98.58	0\\
98.59	0\\
98.6	0\\
98.61	0\\
98.62	0\\
98.63	0\\
98.64	0\\
98.65	0\\
98.66	0\\
98.67	0\\
98.68	0\\
98.69	0\\
98.7	0\\
98.71	0\\
98.72	0\\
98.73	0\\
98.74	0\\
98.75	0\\
98.76	0\\
98.77	0\\
98.78	0\\
98.79	0\\
98.8	0\\
98.81	0\\
98.82	0\\
98.83	0\\
98.84	0\\
98.85	0\\
98.86	0\\
98.87	0\\
98.88	0\\
98.89	0\\
98.9	0\\
98.91	0\\
98.92	0\\
98.93	0\\
98.94	0\\
98.95	0\\
98.96	0\\
98.97	0\\
98.98	0\\
98.99	0\\
99	0\\
99.01	0\\
99.02	0\\
99.03	0\\
99.04	0\\
99.05	0\\
99.06	0\\
99.07	0\\
99.08	0\\
99.09	0\\
99.1	0\\
99.11	0\\
99.12	0\\
99.13	0\\
99.14	1.34192545933262e-05\\
99.15	0.000186052377303365\\
99.16	0.000359788996779725\\
99.17	0.000534644183888877\\
99.18	0.000710633127091898\\
99.19	0.000887771451375849\\
99.2	0.00106607520955938\\
99.21	0.00124556089630498\\
99.22	0.00142623390412353\\
99.23	0.00160807155888041\\
99.24	0.00179109029889443\\
99.25	0.00197530702975846\\
99.26	0.00216073914018688\\
99.27	0.00234740451849972\\
99.28	0.00253532156977716\\
99.29	0.00272450923372016\\
99.3	0.00291498700325548\\
99.31	0.00310677494392576\\
99.32	0.00329989371410826\\
99.33	0.00349436458610888\\
99.34	0.00369020946818107\\
99.35	0.00388745077047987\\
99.36	0.00408611158292441\\
99.37	0.00428621572535727\\
99.38	0.00448778793606267\\
99.39	0.00469085371780874\\
99.4	0.00489543938389655\\
99.41	0.00510157207241651\\
99.42	0.00516645903340951\\
99.43	0.00523160273235865\\
99.44	0.00529730533150437\\
99.45	0.0053635695660718\\
99.46	0.00543039808954691\\
99.47	0.00549779347615066\\
99.48	0.00556577769541155\\
99.49	0.0056343774394848\\
99.5	0.0057035963880993\\
99.51	0.00577343816246014\\
99.52	0.00584390632119317\\
99.53	0.00591500435610443\\
99.54	0.00598673568774476\\
99.55	0.00605910366076959\\
99.56	0.0061321115390831\\
99.57	0.00620576250075546\\
99.58	0.00628005963270098\\
99.59	0.00635500592510444\\
99.6	0.00643060426558196\\
99.61	0.00650685743306189\\
99.62	0.00658376809137045\\
99.63	0.00666133878250572\\
99.64	0.00673957191958258\\
99.65	0.00681846977943027\\
99.66	0.00689803449482267\\
99.67	0.00697826804632059\\
99.68	0.00705917225370364\\
99.69	0.00714074876711371\\
99.7	0.00722299905762269\\
99.71	0.00730592440728626\\
99.72	0.00738952589167773\\
99.73	0.00747380436075851\\
99.74	0.00755876044180094\\
99.75	0.00764439452717997\\
99.76	0.00773070676151409\\
99.77	0.00781769702811535\\
99.78	0.007905364934706\\
99.79	0.00799370979835622\\
99.8	0.00808273062959414\\
99.81	0.008172426115636\\
99.82	0.0082627946026807\\
99.83	0.00835383407720883\\
99.84	0.00844554214622215\\
99.85	0.00853791601635481\\
99.86	0.00863095247178254\\
99.87	0.00872464785085084\\
99.88	0.00881899802133705\\
99.89	0.00891399835425527\\
99.9	0.0090096436961058\\
99.91	0.00910592833996096\\
99.92	0.00920284599502165\\
99.93	0.0093003897536808\\
99.94	0.00939855205657116\\
99.95	0.00949732465545663\\
99.96	0.00959669857381528\\
99.97	0.00969666406494989\\
99.98	0.00979721056744917\\
99.99	0.00989832665780802\\
100	0.01\\
};
\addlegendentry{$q=-1$};

\addplot [color=black,solid,forget plot]
  table[row sep=crcr]{%
0.01	0\\
0.02	0\\
0.03	0\\
0.04	0\\
0.05	0\\
0.06	0\\
0.07	0\\
0.08	0\\
0.09	0\\
0.1	0\\
0.11	0\\
0.12	0\\
0.13	0\\
0.14	0\\
0.15	0\\
0.16	0\\
0.17	0\\
0.18	0\\
0.19	0\\
0.2	0\\
0.21	0\\
0.22	0\\
0.23	0\\
0.24	0\\
0.25	0\\
0.26	0\\
0.27	0\\
0.28	0\\
0.29	0\\
0.3	0\\
0.31	0\\
0.32	0\\
0.33	0\\
0.34	0\\
0.35	0\\
0.36	0\\
0.37	0\\
0.38	0\\
0.39	0\\
0.4	0\\
0.41	0\\
0.42	0\\
0.43	0\\
0.44	0\\
0.45	0\\
0.46	0\\
0.47	0\\
0.48	0\\
0.49	0\\
0.5	0\\
0.51	0\\
0.52	0\\
0.53	0\\
0.54	0\\
0.55	0\\
0.56	0\\
0.57	0\\
0.58	0\\
0.59	0\\
0.6	0\\
0.61	0\\
0.62	0\\
0.63	0\\
0.64	0\\
0.65	0\\
0.66	0\\
0.67	0\\
0.68	0\\
0.69	0\\
0.7	0\\
0.71	0\\
0.72	0\\
0.73	0\\
0.74	0\\
0.75	0\\
0.76	0\\
0.77	0\\
0.78	0\\
0.79	0\\
0.8	0\\
0.81	0\\
0.82	0\\
0.83	0\\
0.84	0\\
0.85	0\\
0.86	0\\
0.87	0\\
0.88	0\\
0.89	0\\
0.9	0\\
0.91	0\\
0.92	0\\
0.93	0\\
0.94	0\\
0.95	0\\
0.96	0\\
0.97	0\\
0.98	0\\
0.99	0\\
1	0\\
1.01	0\\
1.02	0\\
1.03	0\\
1.04	0\\
1.05	0\\
1.06	0\\
1.07	0\\
1.08	0\\
1.09	0\\
1.1	0\\
1.11	0\\
1.12	0\\
1.13	0\\
1.14	0\\
1.15	0\\
1.16	0\\
1.17	0\\
1.18	0\\
1.19	0\\
1.2	0\\
1.21	0\\
1.22	0\\
1.23	0\\
1.24	0\\
1.25	0\\
1.26	0\\
1.27	0\\
1.28	0\\
1.29	0\\
1.3	0\\
1.31	0\\
1.32	0\\
1.33	0\\
1.34	0\\
1.35	0\\
1.36	0\\
1.37	0\\
1.38	0\\
1.39	0\\
1.4	0\\
1.41	0\\
1.42	0\\
1.43	0\\
1.44	0\\
1.45	0\\
1.46	0\\
1.47	0\\
1.48	0\\
1.49	0\\
1.5	0\\
1.51	0\\
1.52	0\\
1.53	0\\
1.54	0\\
1.55	0\\
1.56	0\\
1.57	0\\
1.58	0\\
1.59	0\\
1.6	0\\
1.61	0\\
1.62	0\\
1.63	0\\
1.64	0\\
1.65	0\\
1.66	0\\
1.67	0\\
1.68	0\\
1.69	0\\
1.7	0\\
1.71	0\\
1.72	0\\
1.73	0\\
1.74	0\\
1.75	0\\
1.76	0\\
1.77	0\\
1.78	0\\
1.79	0\\
1.8	0\\
1.81	0\\
1.82	0\\
1.83	0\\
1.84	0\\
1.85	0\\
1.86	0\\
1.87	0\\
1.88	0\\
1.89	0\\
1.9	0\\
1.91	0\\
1.92	0\\
1.93	0\\
1.94	0\\
1.95	0\\
1.96	0\\
1.97	0\\
1.98	0\\
1.99	0\\
2	0\\
2.01	0\\
2.02	0\\
2.03	0\\
2.04	0\\
2.05	0\\
2.06	0\\
2.07	0\\
2.08	0\\
2.09	0\\
2.1	0\\
2.11	0\\
2.12	0\\
2.13	0\\
2.14	0\\
2.15	0\\
2.16	0\\
2.17	0\\
2.18	0\\
2.19	0\\
2.2	0\\
2.21	0\\
2.22	0\\
2.23	0\\
2.24	0\\
2.25	0\\
2.26	0\\
2.27	0\\
2.28	0\\
2.29	0\\
2.3	0\\
2.31	0\\
2.32	0\\
2.33	0\\
2.34	0\\
2.35	0\\
2.36	0\\
2.37	0\\
2.38	0\\
2.39	0\\
2.4	0\\
2.41	0\\
2.42	0\\
2.43	0\\
2.44	0\\
2.45	0\\
2.46	0\\
2.47	0\\
2.48	0\\
2.49	0\\
2.5	0\\
2.51	0\\
2.52	0\\
2.53	0\\
2.54	0\\
2.55	0\\
2.56	0\\
2.57	0\\
2.58	0\\
2.59	0\\
2.6	0\\
2.61	0\\
2.62	0\\
2.63	0\\
2.64	0\\
2.65	0\\
2.66	0\\
2.67	0\\
2.68	0\\
2.69	0\\
2.7	0\\
2.71	0\\
2.72	0\\
2.73	0\\
2.74	0\\
2.75	0\\
2.76	0\\
2.77	0\\
2.78	0\\
2.79	0\\
2.8	0\\
2.81	0\\
2.82	0\\
2.83	0\\
2.84	0\\
2.85	0\\
2.86	0\\
2.87	0\\
2.88	0\\
2.89	0\\
2.9	0\\
2.91	0\\
2.92	0\\
2.93	0\\
2.94	0\\
2.95	0\\
2.96	0\\
2.97	0\\
2.98	0\\
2.99	0\\
3	0\\
3.01	0\\
3.02	0\\
3.03	0\\
3.04	0\\
3.05	0\\
3.06	0\\
3.07	0\\
3.08	0\\
3.09	0\\
3.1	0\\
3.11	0\\
3.12	0\\
3.13	0\\
3.14	0\\
3.15	0\\
3.16	0\\
3.17	0\\
3.18	0\\
3.19	0\\
3.2	0\\
3.21	0\\
3.22	0\\
3.23	0\\
3.24	0\\
3.25	0\\
3.26	0\\
3.27	0\\
3.28	0\\
3.29	0\\
3.3	0\\
3.31	0\\
3.32	0\\
3.33	0\\
3.34	0\\
3.35	0\\
3.36	0\\
3.37	0\\
3.38	0\\
3.39	0\\
3.4	0\\
3.41	0\\
3.42	0\\
3.43	0\\
3.44	0\\
3.45	0\\
3.46	0\\
3.47	0\\
3.48	0\\
3.49	0\\
3.5	0\\
3.51	0\\
3.52	0\\
3.53	0\\
3.54	0\\
3.55	0\\
3.56	0\\
3.57	0\\
3.58	0\\
3.59	0\\
3.6	0\\
3.61	0\\
3.62	0\\
3.63	0\\
3.64	0\\
3.65	0\\
3.66	0\\
3.67	0\\
3.68	0\\
3.69	0\\
3.7	0\\
3.71	0\\
3.72	0\\
3.73	0\\
3.74	0\\
3.75	0\\
3.76	0\\
3.77	0\\
3.78	0\\
3.79	0\\
3.8	0\\
3.81	0\\
3.82	0\\
3.83	0\\
3.84	0\\
3.85	0\\
3.86	0\\
3.87	0\\
3.88	0\\
3.89	0\\
3.9	0\\
3.91	0\\
3.92	0\\
3.93	0\\
3.94	0\\
3.95	0\\
3.96	0\\
3.97	0\\
3.98	0\\
3.99	0\\
4	0\\
4.01	0\\
4.02	0\\
4.03	0\\
4.04	0\\
4.05	0\\
4.06	0\\
4.07	0\\
4.08	0\\
4.09	0\\
4.1	0\\
4.11	0\\
4.12	0\\
4.13	0\\
4.14	0\\
4.15	0\\
4.16	0\\
4.17	0\\
4.18	0\\
4.19	0\\
4.2	0\\
4.21	0\\
4.22	0\\
4.23	0\\
4.24	0\\
4.25	0\\
4.26	0\\
4.27	0\\
4.28	0\\
4.29	0\\
4.3	0\\
4.31	0\\
4.32	0\\
4.33	0\\
4.34	0\\
4.35	0\\
4.36	0\\
4.37	0\\
4.38	0\\
4.39	0\\
4.4	0\\
4.41	0\\
4.42	0\\
4.43	0\\
4.44	0\\
4.45	0\\
4.46	0\\
4.47	0\\
4.48	0\\
4.49	0\\
4.5	0\\
4.51	0\\
4.52	0\\
4.53	0\\
4.54	0\\
4.55	0\\
4.56	0\\
4.57	0\\
4.58	0\\
4.59	0\\
4.6	0\\
4.61	0\\
4.62	0\\
4.63	0\\
4.64	0\\
4.65	0\\
4.66	0\\
4.67	0\\
4.68	0\\
4.69	0\\
4.7	0\\
4.71	0\\
4.72	0\\
4.73	0\\
4.74	0\\
4.75	0\\
4.76	0\\
4.77	0\\
4.78	0\\
4.79	0\\
4.8	0\\
4.81	0\\
4.82	0\\
4.83	0\\
4.84	0\\
4.85	0\\
4.86	0\\
4.87	0\\
4.88	0\\
4.89	0\\
4.9	0\\
4.91	0\\
4.92	0\\
4.93	0\\
4.94	0\\
4.95	0\\
4.96	0\\
4.97	0\\
4.98	0\\
4.99	0\\
5	0\\
5.01	0\\
5.02	0\\
5.03	0\\
5.04	0\\
5.05	0\\
5.06	0\\
5.07	0\\
5.08	0\\
5.09	0\\
5.1	0\\
5.11	0\\
5.12	0\\
5.13	0\\
5.14	0\\
5.15	0\\
5.16	0\\
5.17	0\\
5.18	0\\
5.19	0\\
5.2	0\\
5.21	0\\
5.22	0\\
5.23	0\\
5.24	0\\
5.25	0\\
5.26	0\\
5.27	0\\
5.28	0\\
5.29	0\\
5.3	0\\
5.31	0\\
5.32	0\\
5.33	0\\
5.34	0\\
5.35	0\\
5.36	0\\
5.37	0\\
5.38	0\\
5.39	0\\
5.4	0\\
5.41	0\\
5.42	0\\
5.43	0\\
5.44	0\\
5.45	0\\
5.46	0\\
5.47	0\\
5.48	0\\
5.49	0\\
5.5	0\\
5.51	0\\
5.52	0\\
5.53	0\\
5.54	0\\
5.55	0\\
5.56	0\\
5.57	0\\
5.58	0\\
5.59	0\\
5.6	0\\
5.61	0\\
5.62	0\\
5.63	0\\
5.64	0\\
5.65	0\\
5.66	0\\
5.67	0\\
5.68	0\\
5.69	0\\
5.7	0\\
5.71	0\\
5.72	0\\
5.73	0\\
5.74	0\\
5.75	0\\
5.76	0\\
5.77	0\\
5.78	0\\
5.79	0\\
5.8	0\\
5.81	0\\
5.82	0\\
5.83	0\\
5.84	0\\
5.85	0\\
5.86	0\\
5.87	0\\
5.88	0\\
5.89	0\\
5.9	0\\
5.91	0\\
5.92	0\\
5.93	0\\
5.94	0\\
5.95	0\\
5.96	0\\
5.97	0\\
5.98	0\\
5.99	0\\
6	0\\
6.01	0\\
6.02	0\\
6.03	0\\
6.04	0\\
6.05	0\\
6.06	0\\
6.07	0\\
6.08	0\\
6.09	0\\
6.1	0\\
6.11	0\\
6.12	0\\
6.13	0\\
6.14	0\\
6.15	0\\
6.16	0\\
6.17	0\\
6.18	0\\
6.19	0\\
6.2	0\\
6.21	0\\
6.22	0\\
6.23	0\\
6.24	0\\
6.25	0\\
6.26	0\\
6.27	0\\
6.28	0\\
6.29	0\\
6.3	0\\
6.31	0\\
6.32	0\\
6.33	0\\
6.34	0\\
6.35	0\\
6.36	0\\
6.37	0\\
6.38	0\\
6.39	0\\
6.4	0\\
6.41	0\\
6.42	0\\
6.43	0\\
6.44	0\\
6.45	0\\
6.46	0\\
6.47	0\\
6.48	0\\
6.49	0\\
6.5	0\\
6.51	0\\
6.52	0\\
6.53	0\\
6.54	0\\
6.55	0\\
6.56	0\\
6.57	0\\
6.58	0\\
6.59	0\\
6.6	0\\
6.61	0\\
6.62	0\\
6.63	0\\
6.64	0\\
6.65	0\\
6.66	0\\
6.67	0\\
6.68	0\\
6.69	0\\
6.7	0\\
6.71	0\\
6.72	0\\
6.73	0\\
6.74	0\\
6.75	0\\
6.76	0\\
6.77	0\\
6.78	0\\
6.79	0\\
6.8	0\\
6.81	0\\
6.82	0\\
6.83	0\\
6.84	0\\
6.85	0\\
6.86	0\\
6.87	0\\
6.88	0\\
6.89	0\\
6.9	0\\
6.91	0\\
6.92	0\\
6.93	0\\
6.94	0\\
6.95	0\\
6.96	0\\
6.97	0\\
6.98	0\\
6.99	0\\
7	0\\
7.01	0\\
7.02	0\\
7.03	0\\
7.04	0\\
7.05	0\\
7.06	0\\
7.07	0\\
7.08	0\\
7.09	0\\
7.1	0\\
7.11	0\\
7.12	0\\
7.13	0\\
7.14	0\\
7.15	0\\
7.16	0\\
7.17	0\\
7.18	0\\
7.19	0\\
7.2	0\\
7.21	0\\
7.22	0\\
7.23	0\\
7.24	0\\
7.25	0\\
7.26	0\\
7.27	0\\
7.28	0\\
7.29	0\\
7.3	0\\
7.31	0\\
7.32	0\\
7.33	0\\
7.34	0\\
7.35	0\\
7.36	0\\
7.37	0\\
7.38	0\\
7.39	0\\
7.4	0\\
7.41	0\\
7.42	0\\
7.43	0\\
7.44	0\\
7.45	0\\
7.46	0\\
7.47	0\\
7.48	0\\
7.49	0\\
7.5	0\\
7.51	0\\
7.52	0\\
7.53	0\\
7.54	0\\
7.55	0\\
7.56	0\\
7.57	0\\
7.58	0\\
7.59	0\\
7.6	0\\
7.61	0\\
7.62	0\\
7.63	0\\
7.64	0\\
7.65	0\\
7.66	0\\
7.67	0\\
7.68	0\\
7.69	0\\
7.7	0\\
7.71	0\\
7.72	0\\
7.73	0\\
7.74	0\\
7.75	0\\
7.76	0\\
7.77	0\\
7.78	0\\
7.79	0\\
7.8	0\\
7.81	0\\
7.82	0\\
7.83	0\\
7.84	0\\
7.85	0\\
7.86	0\\
7.87	0\\
7.88	0\\
7.89	0\\
7.9	0\\
7.91	0\\
7.92	0\\
7.93	0\\
7.94	0\\
7.95	0\\
7.96	0\\
7.97	0\\
7.98	0\\
7.99	0\\
8	0\\
8.01	0\\
8.02	0\\
8.03	0\\
8.04	0\\
8.05	0\\
8.06	0\\
8.07	0\\
8.08	0\\
8.09	0\\
8.1	0\\
8.11	0\\
8.12	0\\
8.13	0\\
8.14	0\\
8.15	0\\
8.16	0\\
8.17	0\\
8.18	0\\
8.19	0\\
8.2	0\\
8.21	0\\
8.22	0\\
8.23	0\\
8.24	0\\
8.25	0\\
8.26	0\\
8.27	0\\
8.28	0\\
8.29	0\\
8.3	0\\
8.31	0\\
8.32	0\\
8.33	0\\
8.34	0\\
8.35	0\\
8.36	0\\
8.37	0\\
8.38	0\\
8.39	0\\
8.4	0\\
8.41	0\\
8.42	0\\
8.43	0\\
8.44	0\\
8.45	0\\
8.46	0\\
8.47	0\\
8.48	0\\
8.49	0\\
8.5	0\\
8.51	0\\
8.52	0\\
8.53	0\\
8.54	0\\
8.55	0\\
8.56	0\\
8.57	0\\
8.58	0\\
8.59	0\\
8.6	0\\
8.61	0\\
8.62	0\\
8.63	0\\
8.64	0\\
8.65	0\\
8.66	0\\
8.67	0\\
8.68	0\\
8.69	0\\
8.7	0\\
8.71	0\\
8.72	0\\
8.73	0\\
8.74	0\\
8.75	0\\
8.76	0\\
8.77	0\\
8.78	0\\
8.79	0\\
8.8	0\\
8.81	0\\
8.82	0\\
8.83	0\\
8.84	0\\
8.85	0\\
8.86	0\\
8.87	0\\
8.88	0\\
8.89	0\\
8.9	0\\
8.91	0\\
8.92	0\\
8.93	0\\
8.94	0\\
8.95	0\\
8.96	0\\
8.97	0\\
8.98	0\\
8.99	0\\
9	0\\
9.01	0\\
9.02	0\\
9.03	0\\
9.04	0\\
9.05	0\\
9.06	0\\
9.07	0\\
9.08	0\\
9.09	0\\
9.1	0\\
9.11	0\\
9.12	0\\
9.13	0\\
9.14	0\\
9.15	0\\
9.16	0\\
9.17	0\\
9.18	0\\
9.19	0\\
9.2	0\\
9.21	0\\
9.22	0\\
9.23	0\\
9.24	0\\
9.25	0\\
9.26	0\\
9.27	0\\
9.28	0\\
9.29	0\\
9.3	0\\
9.31	0\\
9.32	0\\
9.33	0\\
9.34	0\\
9.35	0\\
9.36	0\\
9.37	0\\
9.38	0\\
9.39	0\\
9.4	0\\
9.41	0\\
9.42	0\\
9.43	0\\
9.44	0\\
9.45	0\\
9.46	0\\
9.47	0\\
9.48	0\\
9.49	0\\
9.5	0\\
9.51	0\\
9.52	0\\
9.53	0\\
9.54	0\\
9.55	0\\
9.56	0\\
9.57	0\\
9.58	0\\
9.59	0\\
9.6	0\\
9.61	0\\
9.62	0\\
9.63	0\\
9.64	0\\
9.65	0\\
9.66	0\\
9.67	0\\
9.68	0\\
9.69	0\\
9.7	0\\
9.71	0\\
9.72	0\\
9.73	0\\
9.74	0\\
9.75	0\\
9.76	0\\
9.77	0\\
9.78	0\\
9.79	0\\
9.8	0\\
9.81	0\\
9.82	0\\
9.83	0\\
9.84	0\\
9.85	0\\
9.86	0\\
9.87	0\\
9.88	0\\
9.89	0\\
9.9	0\\
9.91	0\\
9.92	0\\
9.93	0\\
9.94	0\\
9.95	0\\
9.96	0\\
9.97	0\\
9.98	0\\
9.99	0\\
10	0\\
10.01	0\\
10.02	0\\
10.03	0\\
10.04	0\\
10.05	0\\
10.06	0\\
10.07	0\\
10.08	0\\
10.09	0\\
10.1	0\\
10.11	0\\
10.12	0\\
10.13	0\\
10.14	0\\
10.15	0\\
10.16	0\\
10.17	0\\
10.18	0\\
10.19	0\\
10.2	0\\
10.21	0\\
10.22	0\\
10.23	0\\
10.24	0\\
10.25	0\\
10.26	0\\
10.27	0\\
10.28	0\\
10.29	0\\
10.3	0\\
10.31	0\\
10.32	0\\
10.33	0\\
10.34	0\\
10.35	0\\
10.36	0\\
10.37	0\\
10.38	0\\
10.39	0\\
10.4	0\\
10.41	0\\
10.42	0\\
10.43	0\\
10.44	0\\
10.45	0\\
10.46	0\\
10.47	0\\
10.48	0\\
10.49	0\\
10.5	0\\
10.51	0\\
10.52	0\\
10.53	0\\
10.54	0\\
10.55	0\\
10.56	0\\
10.57	0\\
10.58	0\\
10.59	0\\
10.6	0\\
10.61	0\\
10.62	0\\
10.63	0\\
10.64	0\\
10.65	0\\
10.66	0\\
10.67	0\\
10.68	0\\
10.69	0\\
10.7	0\\
10.71	0\\
10.72	0\\
10.73	0\\
10.74	0\\
10.75	0\\
10.76	0\\
10.77	0\\
10.78	0\\
10.79	0\\
10.8	0\\
10.81	0\\
10.82	0\\
10.83	0\\
10.84	0\\
10.85	0\\
10.86	0\\
10.87	0\\
10.88	0\\
10.89	0\\
10.9	0\\
10.91	0\\
10.92	0\\
10.93	0\\
10.94	0\\
10.95	0\\
10.96	0\\
10.97	0\\
10.98	0\\
10.99	0\\
11	0\\
11.01	0\\
11.02	0\\
11.03	0\\
11.04	0\\
11.05	0\\
11.06	0\\
11.07	0\\
11.08	0\\
11.09	0\\
11.1	0\\
11.11	0\\
11.12	0\\
11.13	0\\
11.14	0\\
11.15	0\\
11.16	0\\
11.17	0\\
11.18	0\\
11.19	0\\
11.2	0\\
11.21	0\\
11.22	0\\
11.23	0\\
11.24	0\\
11.25	0\\
11.26	0\\
11.27	0\\
11.28	0\\
11.29	0\\
11.3	0\\
11.31	0\\
11.32	0\\
11.33	0\\
11.34	0\\
11.35	0\\
11.36	0\\
11.37	0\\
11.38	0\\
11.39	0\\
11.4	0\\
11.41	0\\
11.42	0\\
11.43	0\\
11.44	0\\
11.45	0\\
11.46	0\\
11.47	0\\
11.48	0\\
11.49	0\\
11.5	0\\
11.51	0\\
11.52	0\\
11.53	0\\
11.54	0\\
11.55	0\\
11.56	0\\
11.57	0\\
11.58	0\\
11.59	0\\
11.6	0\\
11.61	0\\
11.62	0\\
11.63	0\\
11.64	0\\
11.65	0\\
11.66	0\\
11.67	0\\
11.68	0\\
11.69	0\\
11.7	0\\
11.71	0\\
11.72	0\\
11.73	0\\
11.74	0\\
11.75	0\\
11.76	0\\
11.77	0\\
11.78	0\\
11.79	0\\
11.8	0\\
11.81	0\\
11.82	0\\
11.83	0\\
11.84	0\\
11.85	0\\
11.86	0\\
11.87	0\\
11.88	0\\
11.89	0\\
11.9	0\\
11.91	0\\
11.92	0\\
11.93	0\\
11.94	0\\
11.95	0\\
11.96	0\\
11.97	0\\
11.98	0\\
11.99	0\\
12	0\\
12.01	0\\
12.02	0\\
12.03	0\\
12.04	0\\
12.05	0\\
12.06	0\\
12.07	0\\
12.08	0\\
12.09	0\\
12.1	0\\
12.11	0\\
12.12	0\\
12.13	0\\
12.14	0\\
12.15	0\\
12.16	0\\
12.17	0\\
12.18	0\\
12.19	0\\
12.2	0\\
12.21	0\\
12.22	0\\
12.23	0\\
12.24	0\\
12.25	0\\
12.26	0\\
12.27	0\\
12.28	0\\
12.29	0\\
12.3	0\\
12.31	0\\
12.32	0\\
12.33	0\\
12.34	0\\
12.35	0\\
12.36	0\\
12.37	0\\
12.38	0\\
12.39	0\\
12.4	0\\
12.41	0\\
12.42	0\\
12.43	0\\
12.44	0\\
12.45	0\\
12.46	0\\
12.47	0\\
12.48	0\\
12.49	0\\
12.5	0\\
12.51	0\\
12.52	0\\
12.53	0\\
12.54	0\\
12.55	0\\
12.56	0\\
12.57	0\\
12.58	0\\
12.59	0\\
12.6	0\\
12.61	0\\
12.62	0\\
12.63	0\\
12.64	0\\
12.65	0\\
12.66	0\\
12.67	0\\
12.68	0\\
12.69	0\\
12.7	0\\
12.71	0\\
12.72	0\\
12.73	0\\
12.74	0\\
12.75	0\\
12.76	0\\
12.77	0\\
12.78	0\\
12.79	0\\
12.8	0\\
12.81	0\\
12.82	0\\
12.83	0\\
12.84	0\\
12.85	0\\
12.86	0\\
12.87	0\\
12.88	0\\
12.89	0\\
12.9	0\\
12.91	0\\
12.92	0\\
12.93	0\\
12.94	0\\
12.95	0\\
12.96	0\\
12.97	0\\
12.98	0\\
12.99	0\\
13	0\\
13.01	0\\
13.02	0\\
13.03	0\\
13.04	0\\
13.05	0\\
13.06	0\\
13.07	0\\
13.08	0\\
13.09	0\\
13.1	0\\
13.11	0\\
13.12	0\\
13.13	0\\
13.14	0\\
13.15	0\\
13.16	0\\
13.17	0\\
13.18	0\\
13.19	0\\
13.2	0\\
13.21	0\\
13.22	0\\
13.23	0\\
13.24	0\\
13.25	0\\
13.26	0\\
13.27	0\\
13.28	0\\
13.29	0\\
13.3	0\\
13.31	0\\
13.32	0\\
13.33	0\\
13.34	0\\
13.35	0\\
13.36	0\\
13.37	0\\
13.38	0\\
13.39	0\\
13.4	0\\
13.41	0\\
13.42	0\\
13.43	0\\
13.44	0\\
13.45	0\\
13.46	0\\
13.47	0\\
13.48	0\\
13.49	0\\
13.5	0\\
13.51	0\\
13.52	0\\
13.53	0\\
13.54	0\\
13.55	0\\
13.56	0\\
13.57	0\\
13.58	0\\
13.59	0\\
13.6	0\\
13.61	0\\
13.62	0\\
13.63	0\\
13.64	0\\
13.65	0\\
13.66	0\\
13.67	0\\
13.68	0\\
13.69	0\\
13.7	0\\
13.71	0\\
13.72	0\\
13.73	0\\
13.74	0\\
13.75	0\\
13.76	0\\
13.77	0\\
13.78	0\\
13.79	0\\
13.8	0\\
13.81	0\\
13.82	0\\
13.83	0\\
13.84	0\\
13.85	0\\
13.86	0\\
13.87	0\\
13.88	0\\
13.89	0\\
13.9	0\\
13.91	0\\
13.92	0\\
13.93	0\\
13.94	0\\
13.95	0\\
13.96	0\\
13.97	0\\
13.98	0\\
13.99	0\\
14	0\\
14.01	0\\
14.02	0\\
14.03	0\\
14.04	0\\
14.05	0\\
14.06	0\\
14.07	0\\
14.08	0\\
14.09	0\\
14.1	0\\
14.11	0\\
14.12	0\\
14.13	0\\
14.14	0\\
14.15	0\\
14.16	0\\
14.17	0\\
14.18	0\\
14.19	0\\
14.2	0\\
14.21	0\\
14.22	0\\
14.23	0\\
14.24	0\\
14.25	0\\
14.26	0\\
14.27	0\\
14.28	0\\
14.29	0\\
14.3	0\\
14.31	0\\
14.32	0\\
14.33	0\\
14.34	0\\
14.35	0\\
14.36	0\\
14.37	0\\
14.38	0\\
14.39	0\\
14.4	0\\
14.41	0\\
14.42	0\\
14.43	0\\
14.44	0\\
14.45	0\\
14.46	0\\
14.47	0\\
14.48	0\\
14.49	0\\
14.5	0\\
14.51	0\\
14.52	0\\
14.53	0\\
14.54	0\\
14.55	0\\
14.56	0\\
14.57	0\\
14.58	0\\
14.59	0\\
14.6	0\\
14.61	0\\
14.62	0\\
14.63	0\\
14.64	0\\
14.65	0\\
14.66	0\\
14.67	0\\
14.68	0\\
14.69	0\\
14.7	0\\
14.71	0\\
14.72	0\\
14.73	0\\
14.74	0\\
14.75	0\\
14.76	0\\
14.77	0\\
14.78	0\\
14.79	0\\
14.8	0\\
14.81	0\\
14.82	0\\
14.83	0\\
14.84	0\\
14.85	0\\
14.86	0\\
14.87	0\\
14.88	0\\
14.89	0\\
14.9	0\\
14.91	0\\
14.92	0\\
14.93	0\\
14.94	0\\
14.95	0\\
14.96	0\\
14.97	0\\
14.98	0\\
14.99	0\\
15	0\\
15.01	0\\
15.02	0\\
15.03	0\\
15.04	0\\
15.05	0\\
15.06	0\\
15.07	0\\
15.08	0\\
15.09	0\\
15.1	0\\
15.11	0\\
15.12	0\\
15.13	0\\
15.14	0\\
15.15	0\\
15.16	0\\
15.17	0\\
15.18	0\\
15.19	0\\
15.2	0\\
15.21	0\\
15.22	0\\
15.23	0\\
15.24	0\\
15.25	0\\
15.26	0\\
15.27	0\\
15.28	0\\
15.29	0\\
15.3	0\\
15.31	0\\
15.32	0\\
15.33	0\\
15.34	0\\
15.35	0\\
15.36	0\\
15.37	0\\
15.38	0\\
15.39	0\\
15.4	0\\
15.41	0\\
15.42	0\\
15.43	0\\
15.44	0\\
15.45	0\\
15.46	0\\
15.47	0\\
15.48	0\\
15.49	0\\
15.5	0\\
15.51	0\\
15.52	0\\
15.53	0\\
15.54	0\\
15.55	0\\
15.56	0\\
15.57	0\\
15.58	0\\
15.59	0\\
15.6	0\\
15.61	0\\
15.62	0\\
15.63	0\\
15.64	0\\
15.65	0\\
15.66	0\\
15.67	0\\
15.68	0\\
15.69	0\\
15.7	0\\
15.71	0\\
15.72	0\\
15.73	0\\
15.74	0\\
15.75	0\\
15.76	0\\
15.77	0\\
15.78	0\\
15.79	0\\
15.8	0\\
15.81	0\\
15.82	0\\
15.83	0\\
15.84	0\\
15.85	0\\
15.86	0\\
15.87	0\\
15.88	0\\
15.89	0\\
15.9	0\\
15.91	0\\
15.92	0\\
15.93	0\\
15.94	0\\
15.95	0\\
15.96	0\\
15.97	0\\
15.98	0\\
15.99	0\\
16	0\\
16.01	0\\
16.02	0\\
16.03	0\\
16.04	0\\
16.05	0\\
16.06	0\\
16.07	0\\
16.08	0\\
16.09	0\\
16.1	0\\
16.11	0\\
16.12	0\\
16.13	0\\
16.14	0\\
16.15	0\\
16.16	0\\
16.17	0\\
16.18	0\\
16.19	0\\
16.2	0\\
16.21	0\\
16.22	0\\
16.23	0\\
16.24	0\\
16.25	0\\
16.26	0\\
16.27	0\\
16.28	0\\
16.29	0\\
16.3	0\\
16.31	0\\
16.32	0\\
16.33	0\\
16.34	0\\
16.35	0\\
16.36	0\\
16.37	0\\
16.38	0\\
16.39	0\\
16.4	0\\
16.41	0\\
16.42	0\\
16.43	0\\
16.44	0\\
16.45	0\\
16.46	0\\
16.47	0\\
16.48	0\\
16.49	0\\
16.5	0\\
16.51	0\\
16.52	0\\
16.53	0\\
16.54	0\\
16.55	0\\
16.56	0\\
16.57	0\\
16.58	0\\
16.59	0\\
16.6	0\\
16.61	0\\
16.62	0\\
16.63	0\\
16.64	0\\
16.65	0\\
16.66	0\\
16.67	0\\
16.68	0\\
16.69	0\\
16.7	0\\
16.71	0\\
16.72	0\\
16.73	0\\
16.74	0\\
16.75	0\\
16.76	0\\
16.77	0\\
16.78	0\\
16.79	0\\
16.8	0\\
16.81	0\\
16.82	0\\
16.83	0\\
16.84	0\\
16.85	0\\
16.86	0\\
16.87	0\\
16.88	0\\
16.89	0\\
16.9	0\\
16.91	0\\
16.92	0\\
16.93	0\\
16.94	0\\
16.95	0\\
16.96	0\\
16.97	0\\
16.98	0\\
16.99	0\\
17	0\\
17.01	0\\
17.02	0\\
17.03	0\\
17.04	0\\
17.05	0\\
17.06	0\\
17.07	0\\
17.08	0\\
17.09	0\\
17.1	0\\
17.11	0\\
17.12	0\\
17.13	0\\
17.14	0\\
17.15	0\\
17.16	0\\
17.17	0\\
17.18	0\\
17.19	0\\
17.2	0\\
17.21	0\\
17.22	0\\
17.23	0\\
17.24	0\\
17.25	0\\
17.26	0\\
17.27	0\\
17.28	0\\
17.29	0\\
17.3	0\\
17.31	0\\
17.32	0\\
17.33	0\\
17.34	0\\
17.35	0\\
17.36	0\\
17.37	0\\
17.38	0\\
17.39	0\\
17.4	0\\
17.41	0\\
17.42	0\\
17.43	0\\
17.44	0\\
17.45	0\\
17.46	0\\
17.47	0\\
17.48	0\\
17.49	0\\
17.5	0\\
17.51	0\\
17.52	0\\
17.53	0\\
17.54	0\\
17.55	0\\
17.56	0\\
17.57	0\\
17.58	0\\
17.59	0\\
17.6	0\\
17.61	0\\
17.62	0\\
17.63	0\\
17.64	0\\
17.65	0\\
17.66	0\\
17.67	0\\
17.68	0\\
17.69	0\\
17.7	0\\
17.71	0\\
17.72	0\\
17.73	0\\
17.74	0\\
17.75	0\\
17.76	0\\
17.77	0\\
17.78	0\\
17.79	0\\
17.8	0\\
17.81	0\\
17.82	0\\
17.83	0\\
17.84	0\\
17.85	0\\
17.86	0\\
17.87	0\\
17.88	0\\
17.89	0\\
17.9	0\\
17.91	0\\
17.92	0\\
17.93	0\\
17.94	0\\
17.95	0\\
17.96	0\\
17.97	0\\
17.98	0\\
17.99	0\\
18	0\\
18.01	0\\
18.02	0\\
18.03	0\\
18.04	0\\
18.05	0\\
18.06	0\\
18.07	0\\
18.08	0\\
18.09	0\\
18.1	0\\
18.11	0\\
18.12	0\\
18.13	0\\
18.14	0\\
18.15	0\\
18.16	0\\
18.17	0\\
18.18	0\\
18.19	0\\
18.2	0\\
18.21	0\\
18.22	0\\
18.23	0\\
18.24	0\\
18.25	0\\
18.26	0\\
18.27	0\\
18.28	0\\
18.29	0\\
18.3	0\\
18.31	0\\
18.32	0\\
18.33	0\\
18.34	0\\
18.35	0\\
18.36	0\\
18.37	0\\
18.38	0\\
18.39	0\\
18.4	0\\
18.41	0\\
18.42	0\\
18.43	0\\
18.44	0\\
18.45	0\\
18.46	0\\
18.47	0\\
18.48	0\\
18.49	0\\
18.5	0\\
18.51	0\\
18.52	0\\
18.53	0\\
18.54	0\\
18.55	0\\
18.56	0\\
18.57	0\\
18.58	0\\
18.59	0\\
18.6	0\\
18.61	0\\
18.62	0\\
18.63	0\\
18.64	0\\
18.65	0\\
18.66	0\\
18.67	0\\
18.68	0\\
18.69	0\\
18.7	0\\
18.71	0\\
18.72	0\\
18.73	0\\
18.74	0\\
18.75	0\\
18.76	0\\
18.77	0\\
18.78	0\\
18.79	0\\
18.8	0\\
18.81	0\\
18.82	0\\
18.83	0\\
18.84	0\\
18.85	0\\
18.86	0\\
18.87	0\\
18.88	0\\
18.89	0\\
18.9	0\\
18.91	0\\
18.92	0\\
18.93	0\\
18.94	0\\
18.95	0\\
18.96	0\\
18.97	0\\
18.98	0\\
18.99	0\\
19	0\\
19.01	0\\
19.02	0\\
19.03	0\\
19.04	0\\
19.05	0\\
19.06	0\\
19.07	0\\
19.08	0\\
19.09	0\\
19.1	0\\
19.11	0\\
19.12	0\\
19.13	0\\
19.14	0\\
19.15	0\\
19.16	0\\
19.17	0\\
19.18	0\\
19.19	0\\
19.2	0\\
19.21	0\\
19.22	0\\
19.23	0\\
19.24	0\\
19.25	0\\
19.26	0\\
19.27	0\\
19.28	0\\
19.29	0\\
19.3	0\\
19.31	0\\
19.32	0\\
19.33	0\\
19.34	0\\
19.35	0\\
19.36	0\\
19.37	0\\
19.38	0\\
19.39	0\\
19.4	0\\
19.41	0\\
19.42	0\\
19.43	0\\
19.44	0\\
19.45	0\\
19.46	0\\
19.47	0\\
19.48	0\\
19.49	0\\
19.5	0\\
19.51	0\\
19.52	0\\
19.53	0\\
19.54	0\\
19.55	0\\
19.56	0\\
19.57	0\\
19.58	0\\
19.59	0\\
19.6	0\\
19.61	0\\
19.62	0\\
19.63	0\\
19.64	0\\
19.65	0\\
19.66	0\\
19.67	0\\
19.68	0\\
19.69	0\\
19.7	0\\
19.71	0\\
19.72	0\\
19.73	0\\
19.74	0\\
19.75	0\\
19.76	0\\
19.77	0\\
19.78	0\\
19.79	0\\
19.8	0\\
19.81	0\\
19.82	0\\
19.83	0\\
19.84	0\\
19.85	0\\
19.86	0\\
19.87	0\\
19.88	0\\
19.89	0\\
19.9	0\\
19.91	0\\
19.92	0\\
19.93	0\\
19.94	0\\
19.95	0\\
19.96	0\\
19.97	0\\
19.98	0\\
19.99	0\\
20	0\\
20.01	0\\
20.02	0\\
20.03	0\\
20.04	0\\
20.05	0\\
20.06	0\\
20.07	0\\
20.08	0\\
20.09	0\\
20.1	0\\
20.11	0\\
20.12	0\\
20.13	0\\
20.14	0\\
20.15	0\\
20.16	0\\
20.17	0\\
20.18	0\\
20.19	0\\
20.2	0\\
20.21	0\\
20.22	0\\
20.23	0\\
20.24	0\\
20.25	0\\
20.26	0\\
20.27	0\\
20.28	0\\
20.29	0\\
20.3	0\\
20.31	0\\
20.32	0\\
20.33	0\\
20.34	0\\
20.35	0\\
20.36	0\\
20.37	0\\
20.38	0\\
20.39	0\\
20.4	0\\
20.41	0\\
20.42	0\\
20.43	0\\
20.44	0\\
20.45	0\\
20.46	0\\
20.47	0\\
20.48	0\\
20.49	0\\
20.5	0\\
20.51	0\\
20.52	0\\
20.53	0\\
20.54	0\\
20.55	0\\
20.56	0\\
20.57	0\\
20.58	0\\
20.59	0\\
20.6	0\\
20.61	0\\
20.62	0\\
20.63	0\\
20.64	0\\
20.65	0\\
20.66	0\\
20.67	0\\
20.68	0\\
20.69	0\\
20.7	0\\
20.71	0\\
20.72	0\\
20.73	0\\
20.74	0\\
20.75	0\\
20.76	0\\
20.77	0\\
20.78	0\\
20.79	0\\
20.8	0\\
20.81	0\\
20.82	0\\
20.83	0\\
20.84	0\\
20.85	0\\
20.86	0\\
20.87	0\\
20.88	0\\
20.89	0\\
20.9	0\\
20.91	0\\
20.92	0\\
20.93	0\\
20.94	0\\
20.95	0\\
20.96	0\\
20.97	0\\
20.98	0\\
20.99	0\\
21	0\\
21.01	0\\
21.02	0\\
21.03	0\\
21.04	0\\
21.05	0\\
21.06	0\\
21.07	0\\
21.08	0\\
21.09	0\\
21.1	0\\
21.11	0\\
21.12	0\\
21.13	0\\
21.14	0\\
21.15	0\\
21.16	0\\
21.17	0\\
21.18	0\\
21.19	0\\
21.2	0\\
21.21	0\\
21.22	0\\
21.23	0\\
21.24	0\\
21.25	0\\
21.26	0\\
21.27	0\\
21.28	0\\
21.29	0\\
21.3	0\\
21.31	0\\
21.32	0\\
21.33	0\\
21.34	0\\
21.35	0\\
21.36	0\\
21.37	0\\
21.38	0\\
21.39	0\\
21.4	0\\
21.41	0\\
21.42	0\\
21.43	0\\
21.44	0\\
21.45	0\\
21.46	0\\
21.47	0\\
21.48	0\\
21.49	0\\
21.5	0\\
21.51	0\\
21.52	0\\
21.53	0\\
21.54	0\\
21.55	0\\
21.56	0\\
21.57	0\\
21.58	0\\
21.59	0\\
21.6	0\\
21.61	0\\
21.62	0\\
21.63	0\\
21.64	0\\
21.65	0\\
21.66	0\\
21.67	0\\
21.68	0\\
21.69	0\\
21.7	0\\
21.71	0\\
21.72	0\\
21.73	0\\
21.74	0\\
21.75	0\\
21.76	0\\
21.77	0\\
21.78	0\\
21.79	0\\
21.8	0\\
21.81	0\\
21.82	0\\
21.83	0\\
21.84	0\\
21.85	0\\
21.86	0\\
21.87	0\\
21.88	0\\
21.89	0\\
21.9	0\\
21.91	0\\
21.92	0\\
21.93	0\\
21.94	0\\
21.95	0\\
21.96	0\\
21.97	0\\
21.98	0\\
21.99	0\\
22	0\\
22.01	0\\
22.02	0\\
22.03	0\\
22.04	0\\
22.05	0\\
22.06	0\\
22.07	0\\
22.08	0\\
22.09	0\\
22.1	0\\
22.11	0\\
22.12	0\\
22.13	0\\
22.14	0\\
22.15	0\\
22.16	0\\
22.17	0\\
22.18	0\\
22.19	0\\
22.2	0\\
22.21	0\\
22.22	0\\
22.23	0\\
22.24	0\\
22.25	0\\
22.26	0\\
22.27	0\\
22.28	0\\
22.29	0\\
22.3	0\\
22.31	0\\
22.32	0\\
22.33	0\\
22.34	0\\
22.35	0\\
22.36	0\\
22.37	0\\
22.38	0\\
22.39	0\\
22.4	0\\
22.41	0\\
22.42	0\\
22.43	0\\
22.44	0\\
22.45	0\\
22.46	0\\
22.47	0\\
22.48	0\\
22.49	0\\
22.5	0\\
22.51	0\\
22.52	0\\
22.53	0\\
22.54	0\\
22.55	0\\
22.56	0\\
22.57	0\\
22.58	0\\
22.59	0\\
22.6	0\\
22.61	0\\
22.62	0\\
22.63	0\\
22.64	0\\
22.65	0\\
22.66	0\\
22.67	0\\
22.68	0\\
22.69	0\\
22.7	0\\
22.71	0\\
22.72	0\\
22.73	0\\
22.74	0\\
22.75	0\\
22.76	0\\
22.77	0\\
22.78	0\\
22.79	0\\
22.8	0\\
22.81	0\\
22.82	0\\
22.83	0\\
22.84	0\\
22.85	0\\
22.86	0\\
22.87	0\\
22.88	0\\
22.89	0\\
22.9	0\\
22.91	0\\
22.92	0\\
22.93	0\\
22.94	0\\
22.95	0\\
22.96	0\\
22.97	0\\
22.98	0\\
22.99	0\\
23	0\\
23.01	0\\
23.02	0\\
23.03	0\\
23.04	0\\
23.05	0\\
23.06	0\\
23.07	0\\
23.08	0\\
23.09	0\\
23.1	0\\
23.11	0\\
23.12	0\\
23.13	0\\
23.14	0\\
23.15	0\\
23.16	0\\
23.17	0\\
23.18	0\\
23.19	0\\
23.2	0\\
23.21	0\\
23.22	0\\
23.23	0\\
23.24	0\\
23.25	0\\
23.26	0\\
23.27	0\\
23.28	0\\
23.29	0\\
23.3	0\\
23.31	0\\
23.32	0\\
23.33	0\\
23.34	0\\
23.35	0\\
23.36	0\\
23.37	0\\
23.38	0\\
23.39	0\\
23.4	0\\
23.41	0\\
23.42	0\\
23.43	0\\
23.44	0\\
23.45	0\\
23.46	0\\
23.47	0\\
23.48	0\\
23.49	0\\
23.5	0\\
23.51	0\\
23.52	0\\
23.53	0\\
23.54	0\\
23.55	0\\
23.56	0\\
23.57	0\\
23.58	0\\
23.59	0\\
23.6	0\\
23.61	0\\
23.62	0\\
23.63	0\\
23.64	0\\
23.65	0\\
23.66	0\\
23.67	0\\
23.68	0\\
23.69	0\\
23.7	0\\
23.71	0\\
23.72	0\\
23.73	0\\
23.74	0\\
23.75	0\\
23.76	0\\
23.77	0\\
23.78	0\\
23.79	0\\
23.8	0\\
23.81	0\\
23.82	0\\
23.83	0\\
23.84	0\\
23.85	0\\
23.86	0\\
23.87	0\\
23.88	0\\
23.89	0\\
23.9	0\\
23.91	0\\
23.92	0\\
23.93	0\\
23.94	0\\
23.95	0\\
23.96	0\\
23.97	0\\
23.98	0\\
23.99	0\\
24	0\\
24.01	0\\
24.02	0\\
24.03	0\\
24.04	0\\
24.05	0\\
24.06	0\\
24.07	0\\
24.08	0\\
24.09	0\\
24.1	0\\
24.11	0\\
24.12	0\\
24.13	0\\
24.14	0\\
24.15	0\\
24.16	0\\
24.17	0\\
24.18	0\\
24.19	0\\
24.2	0\\
24.21	0\\
24.22	0\\
24.23	0\\
24.24	0\\
24.25	0\\
24.26	0\\
24.27	0\\
24.28	0\\
24.29	0\\
24.3	0\\
24.31	0\\
24.32	0\\
24.33	0\\
24.34	0\\
24.35	0\\
24.36	0\\
24.37	0\\
24.38	0\\
24.39	0\\
24.4	0\\
24.41	0\\
24.42	0\\
24.43	0\\
24.44	0\\
24.45	0\\
24.46	0\\
24.47	0\\
24.48	0\\
24.49	0\\
24.5	0\\
24.51	0\\
24.52	0\\
24.53	0\\
24.54	0\\
24.55	0\\
24.56	0\\
24.57	0\\
24.58	0\\
24.59	0\\
24.6	0\\
24.61	0\\
24.62	0\\
24.63	0\\
24.64	0\\
24.65	0\\
24.66	0\\
24.67	0\\
24.68	0\\
24.69	0\\
24.7	0\\
24.71	0\\
24.72	0\\
24.73	0\\
24.74	0\\
24.75	0\\
24.76	0\\
24.77	0\\
24.78	0\\
24.79	0\\
24.8	0\\
24.81	0\\
24.82	0\\
24.83	0\\
24.84	0\\
24.85	0\\
24.86	0\\
24.87	0\\
24.88	0\\
24.89	0\\
24.9	0\\
24.91	0\\
24.92	0\\
24.93	0\\
24.94	0\\
24.95	0\\
24.96	0\\
24.97	0\\
24.98	0\\
24.99	0\\
25	0\\
25.01	0\\
25.02	0\\
25.03	0\\
25.04	0\\
25.05	0\\
25.06	0\\
25.07	0\\
25.08	0\\
25.09	0\\
25.1	0\\
25.11	0\\
25.12	0\\
25.13	0\\
25.14	0\\
25.15	0\\
25.16	0\\
25.17	0\\
25.18	0\\
25.19	0\\
25.2	0\\
25.21	0\\
25.22	0\\
25.23	0\\
25.24	0\\
25.25	0\\
25.26	0\\
25.27	0\\
25.28	0\\
25.29	0\\
25.3	0\\
25.31	0\\
25.32	0\\
25.33	0\\
25.34	0\\
25.35	0\\
25.36	0\\
25.37	0\\
25.38	0\\
25.39	0\\
25.4	0\\
25.41	0\\
25.42	0\\
25.43	0\\
25.44	0\\
25.45	0\\
25.46	0\\
25.47	0\\
25.48	0\\
25.49	0\\
25.5	0\\
25.51	0\\
25.52	0\\
25.53	0\\
25.54	0\\
25.55	0\\
25.56	0\\
25.57	0\\
25.58	0\\
25.59	0\\
25.6	0\\
25.61	0\\
25.62	0\\
25.63	0\\
25.64	0\\
25.65	0\\
25.66	0\\
25.67	0\\
25.68	0\\
25.69	0\\
25.7	0\\
25.71	0\\
25.72	0\\
25.73	0\\
25.74	0\\
25.75	0\\
25.76	0\\
25.77	0\\
25.78	0\\
25.79	0\\
25.8	0\\
25.81	0\\
25.82	0\\
25.83	0\\
25.84	0\\
25.85	0\\
25.86	0\\
25.87	0\\
25.88	0\\
25.89	0\\
25.9	0\\
25.91	0\\
25.92	0\\
25.93	0\\
25.94	0\\
25.95	0\\
25.96	0\\
25.97	0\\
25.98	0\\
25.99	0\\
26	0\\
26.01	0\\
26.02	0\\
26.03	0\\
26.04	0\\
26.05	0\\
26.06	0\\
26.07	0\\
26.08	0\\
26.09	0\\
26.1	0\\
26.11	0\\
26.12	0\\
26.13	0\\
26.14	0\\
26.15	0\\
26.16	0\\
26.17	0\\
26.18	0\\
26.19	0\\
26.2	0\\
26.21	0\\
26.22	0\\
26.23	0\\
26.24	0\\
26.25	0\\
26.26	0\\
26.27	0\\
26.28	0\\
26.29	0\\
26.3	0\\
26.31	0\\
26.32	0\\
26.33	0\\
26.34	0\\
26.35	0\\
26.36	0\\
26.37	0\\
26.38	0\\
26.39	0\\
26.4	0\\
26.41	0\\
26.42	0\\
26.43	0\\
26.44	0\\
26.45	0\\
26.46	0\\
26.47	0\\
26.48	0\\
26.49	0\\
26.5	0\\
26.51	0\\
26.52	0\\
26.53	0\\
26.54	0\\
26.55	0\\
26.56	0\\
26.57	0\\
26.58	0\\
26.59	0\\
26.6	0\\
26.61	0\\
26.62	0\\
26.63	0\\
26.64	0\\
26.65	0\\
26.66	0\\
26.67	0\\
26.68	0\\
26.69	0\\
26.7	0\\
26.71	0\\
26.72	0\\
26.73	0\\
26.74	0\\
26.75	0\\
26.76	0\\
26.77	0\\
26.78	0\\
26.79	0\\
26.8	0\\
26.81	0\\
26.82	0\\
26.83	0\\
26.84	0\\
26.85	0\\
26.86	0\\
26.87	0\\
26.88	0\\
26.89	0\\
26.9	0\\
26.91	0\\
26.92	0\\
26.93	0\\
26.94	0\\
26.95	0\\
26.96	0\\
26.97	0\\
26.98	0\\
26.99	0\\
27	0\\
27.01	0\\
27.02	0\\
27.03	0\\
27.04	0\\
27.05	0\\
27.06	0\\
27.07	0\\
27.08	0\\
27.09	0\\
27.1	0\\
27.11	0\\
27.12	0\\
27.13	0\\
27.14	0\\
27.15	0\\
27.16	0\\
27.17	0\\
27.18	0\\
27.19	0\\
27.2	0\\
27.21	0\\
27.22	0\\
27.23	0\\
27.24	0\\
27.25	0\\
27.26	0\\
27.27	0\\
27.28	0\\
27.29	0\\
27.3	0\\
27.31	0\\
27.32	0\\
27.33	0\\
27.34	0\\
27.35	0\\
27.36	0\\
27.37	0\\
27.38	0\\
27.39	0\\
27.4	0\\
27.41	0\\
27.42	0\\
27.43	0\\
27.44	0\\
27.45	0\\
27.46	0\\
27.47	0\\
27.48	0\\
27.49	0\\
27.5	0\\
27.51	0\\
27.52	0\\
27.53	0\\
27.54	0\\
27.55	0\\
27.56	0\\
27.57	0\\
27.58	0\\
27.59	0\\
27.6	0\\
27.61	0\\
27.62	0\\
27.63	0\\
27.64	0\\
27.65	0\\
27.66	0\\
27.67	0\\
27.68	0\\
27.69	0\\
27.7	0\\
27.71	0\\
27.72	0\\
27.73	0\\
27.74	0\\
27.75	0\\
27.76	0\\
27.77	0\\
27.78	0\\
27.79	0\\
27.8	0\\
27.81	0\\
27.82	0\\
27.83	0\\
27.84	0\\
27.85	0\\
27.86	0\\
27.87	0\\
27.88	0\\
27.89	0\\
27.9	0\\
27.91	0\\
27.92	0\\
27.93	0\\
27.94	0\\
27.95	0\\
27.96	0\\
27.97	0\\
27.98	0\\
27.99	0\\
28	0\\
28.01	0\\
28.02	0\\
28.03	0\\
28.04	0\\
28.05	0\\
28.06	0\\
28.07	0\\
28.08	0\\
28.09	0\\
28.1	0\\
28.11	0\\
28.12	0\\
28.13	0\\
28.14	0\\
28.15	0\\
28.16	0\\
28.17	0\\
28.18	0\\
28.19	0\\
28.2	0\\
28.21	0\\
28.22	0\\
28.23	0\\
28.24	0\\
28.25	0\\
28.26	0\\
28.27	0\\
28.28	0\\
28.29	0\\
28.3	0\\
28.31	0\\
28.32	0\\
28.33	0\\
28.34	0\\
28.35	0\\
28.36	0\\
28.37	0\\
28.38	0\\
28.39	0\\
28.4	0\\
28.41	0\\
28.42	0\\
28.43	0\\
28.44	0\\
28.45	0\\
28.46	0\\
28.47	0\\
28.48	0\\
28.49	0\\
28.5	0\\
28.51	0\\
28.52	0\\
28.53	0\\
28.54	0\\
28.55	0\\
28.56	0\\
28.57	0\\
28.58	0\\
28.59	0\\
28.6	0\\
28.61	0\\
28.62	0\\
28.63	0\\
28.64	0\\
28.65	0\\
28.66	0\\
28.67	0\\
28.68	0\\
28.69	0\\
28.7	0\\
28.71	0\\
28.72	0\\
28.73	0\\
28.74	0\\
28.75	0\\
28.76	0\\
28.77	0\\
28.78	0\\
28.79	0\\
28.8	0\\
28.81	0\\
28.82	0\\
28.83	0\\
28.84	0\\
28.85	0\\
28.86	0\\
28.87	0\\
28.88	0\\
28.89	0\\
28.9	0\\
28.91	0\\
28.92	0\\
28.93	0\\
28.94	0\\
28.95	0\\
28.96	0\\
28.97	0\\
28.98	0\\
28.99	0\\
29	0\\
29.01	0\\
29.02	0\\
29.03	0\\
29.04	0\\
29.05	0\\
29.06	0\\
29.07	0\\
29.08	0\\
29.09	0\\
29.1	0\\
29.11	0\\
29.12	0\\
29.13	0\\
29.14	0\\
29.15	0\\
29.16	0\\
29.17	0\\
29.18	0\\
29.19	0\\
29.2	0\\
29.21	0\\
29.22	0\\
29.23	0\\
29.24	0\\
29.25	0\\
29.26	0\\
29.27	0\\
29.28	0\\
29.29	0\\
29.3	0\\
29.31	0\\
29.32	0\\
29.33	0\\
29.34	0\\
29.35	0\\
29.36	0\\
29.37	0\\
29.38	0\\
29.39	0\\
29.4	0\\
29.41	0\\
29.42	0\\
29.43	0\\
29.44	0\\
29.45	0\\
29.46	0\\
29.47	0\\
29.48	0\\
29.49	0\\
29.5	0\\
29.51	0\\
29.52	0\\
29.53	0\\
29.54	0\\
29.55	0\\
29.56	0\\
29.57	0\\
29.58	0\\
29.59	0\\
29.6	0\\
29.61	0\\
29.62	0\\
29.63	0\\
29.64	0\\
29.65	0\\
29.66	0\\
29.67	0\\
29.68	0\\
29.69	0\\
29.7	0\\
29.71	0\\
29.72	0\\
29.73	0\\
29.74	0\\
29.75	0\\
29.76	0\\
29.77	0\\
29.78	0\\
29.79	0\\
29.8	0\\
29.81	0\\
29.82	0\\
29.83	0\\
29.84	0\\
29.85	0\\
29.86	0\\
29.87	0\\
29.88	0\\
29.89	0\\
29.9	0\\
29.91	0\\
29.92	0\\
29.93	0\\
29.94	0\\
29.95	0\\
29.96	0\\
29.97	0\\
29.98	0\\
29.99	0\\
30	0\\
30.01	0\\
30.02	0\\
30.03	0\\
30.04	0\\
30.05	0\\
30.06	0\\
30.07	0\\
30.08	0\\
30.09	0\\
30.1	0\\
30.11	0\\
30.12	0\\
30.13	0\\
30.14	0\\
30.15	0\\
30.16	0\\
30.17	0\\
30.18	0\\
30.19	0\\
30.2	0\\
30.21	0\\
30.22	0\\
30.23	0\\
30.24	0\\
30.25	0\\
30.26	0\\
30.27	0\\
30.28	0\\
30.29	0\\
30.3	0\\
30.31	0\\
30.32	0\\
30.33	0\\
30.34	0\\
30.35	0\\
30.36	0\\
30.37	0\\
30.38	0\\
30.39	0\\
30.4	0\\
30.41	0\\
30.42	0\\
30.43	0\\
30.44	0\\
30.45	0\\
30.46	0\\
30.47	0\\
30.48	0\\
30.49	0\\
30.5	0\\
30.51	0\\
30.52	0\\
30.53	0\\
30.54	0\\
30.55	0\\
30.56	0\\
30.57	0\\
30.58	0\\
30.59	0\\
30.6	0\\
30.61	0\\
30.62	0\\
30.63	0\\
30.64	0\\
30.65	0\\
30.66	0\\
30.67	0\\
30.68	0\\
30.69	0\\
30.7	0\\
30.71	0\\
30.72	0\\
30.73	0\\
30.74	0\\
30.75	0\\
30.76	0\\
30.77	0\\
30.78	0\\
30.79	0\\
30.8	0\\
30.81	0\\
30.82	0\\
30.83	0\\
30.84	0\\
30.85	0\\
30.86	0\\
30.87	0\\
30.88	0\\
30.89	0\\
30.9	0\\
30.91	0\\
30.92	0\\
30.93	0\\
30.94	0\\
30.95	0\\
30.96	0\\
30.97	0\\
30.98	0\\
30.99	0\\
31	0\\
31.01	0\\
31.02	0\\
31.03	0\\
31.04	0\\
31.05	0\\
31.06	0\\
31.07	0\\
31.08	0\\
31.09	0\\
31.1	0\\
31.11	0\\
31.12	0\\
31.13	0\\
31.14	0\\
31.15	0\\
31.16	0\\
31.17	0\\
31.18	0\\
31.19	0\\
31.2	0\\
31.21	0\\
31.22	0\\
31.23	0\\
31.24	0\\
31.25	0\\
31.26	0\\
31.27	0\\
31.28	0\\
31.29	0\\
31.3	0\\
31.31	0\\
31.32	0\\
31.33	0\\
31.34	0\\
31.35	0\\
31.36	0\\
31.37	0\\
31.38	0\\
31.39	0\\
31.4	0\\
31.41	0\\
31.42	0\\
31.43	0\\
31.44	0\\
31.45	0\\
31.46	0\\
31.47	0\\
31.48	0\\
31.49	0\\
31.5	0\\
31.51	0\\
31.52	0\\
31.53	0\\
31.54	0\\
31.55	0\\
31.56	0\\
31.57	0\\
31.58	0\\
31.59	0\\
31.6	0\\
31.61	0\\
31.62	0\\
31.63	0\\
31.64	0\\
31.65	0\\
31.66	0\\
31.67	0\\
31.68	0\\
31.69	0\\
31.7	0\\
31.71	0\\
31.72	0\\
31.73	0\\
31.74	0\\
31.75	0\\
31.76	0\\
31.77	0\\
31.78	0\\
31.79	0\\
31.8	0\\
31.81	0\\
31.82	0\\
31.83	0\\
31.84	0\\
31.85	0\\
31.86	0\\
31.87	0\\
31.88	0\\
31.89	0\\
31.9	0\\
31.91	0\\
31.92	0\\
31.93	0\\
31.94	0\\
31.95	0\\
31.96	0\\
31.97	0\\
31.98	0\\
31.99	0\\
32	0\\
32.01	0\\
32.02	0\\
32.03	0\\
32.04	0\\
32.05	0\\
32.06	0\\
32.07	0\\
32.08	0\\
32.09	0\\
32.1	0\\
32.11	0\\
32.12	0\\
32.13	0\\
32.14	0\\
32.15	0\\
32.16	0\\
32.17	0\\
32.18	0\\
32.19	0\\
32.2	0\\
32.21	0\\
32.22	0\\
32.23	0\\
32.24	0\\
32.25	0\\
32.26	0\\
32.27	0\\
32.28	0\\
32.29	0\\
32.3	0\\
32.31	0\\
32.32	0\\
32.33	0\\
32.34	0\\
32.35	0\\
32.36	0\\
32.37	0\\
32.38	0\\
32.39	0\\
32.4	0\\
32.41	0\\
32.42	0\\
32.43	0\\
32.44	0\\
32.45	0\\
32.46	0\\
32.47	0\\
32.48	0\\
32.49	0\\
32.5	0\\
32.51	0\\
32.52	0\\
32.53	0\\
32.54	0\\
32.55	0\\
32.56	0\\
32.57	0\\
32.58	0\\
32.59	0\\
32.6	0\\
32.61	0\\
32.62	0\\
32.63	0\\
32.64	0\\
32.65	0\\
32.66	0\\
32.67	0\\
32.68	0\\
32.69	0\\
32.7	0\\
32.71	0\\
32.72	0\\
32.73	0\\
32.74	0\\
32.75	0\\
32.76	0\\
32.77	0\\
32.78	0\\
32.79	0\\
32.8	0\\
32.81	0\\
32.82	0\\
32.83	0\\
32.84	0\\
32.85	0\\
32.86	0\\
32.87	0\\
32.88	0\\
32.89	0\\
32.9	0\\
32.91	0\\
32.92	0\\
32.93	0\\
32.94	0\\
32.95	0\\
32.96	0\\
32.97	0\\
32.98	0\\
32.99	0\\
33	0\\
33.01	0\\
33.02	0\\
33.03	0\\
33.04	0\\
33.05	0\\
33.06	0\\
33.07	0\\
33.08	0\\
33.09	0\\
33.1	0\\
33.11	0\\
33.12	0\\
33.13	0\\
33.14	0\\
33.15	0\\
33.16	0\\
33.17	0\\
33.18	0\\
33.19	0\\
33.2	0\\
33.21	0\\
33.22	0\\
33.23	0\\
33.24	0\\
33.25	0\\
33.26	0\\
33.27	0\\
33.28	0\\
33.29	0\\
33.3	0\\
33.31	0\\
33.32	0\\
33.33	0\\
33.34	0\\
33.35	0\\
33.36	0\\
33.37	0\\
33.38	0\\
33.39	0\\
33.4	0\\
33.41	0\\
33.42	0\\
33.43	0\\
33.44	0\\
33.45	0\\
33.46	0\\
33.47	0\\
33.48	0\\
33.49	0\\
33.5	0\\
33.51	0\\
33.52	0\\
33.53	0\\
33.54	0\\
33.55	0\\
33.56	0\\
33.57	0\\
33.58	0\\
33.59	0\\
33.6	0\\
33.61	0\\
33.62	0\\
33.63	0\\
33.64	0\\
33.65	0\\
33.66	0\\
33.67	0\\
33.68	0\\
33.69	0\\
33.7	0\\
33.71	0\\
33.72	0\\
33.73	0\\
33.74	0\\
33.75	0\\
33.76	0\\
33.77	0\\
33.78	0\\
33.79	0\\
33.8	0\\
33.81	0\\
33.82	0\\
33.83	0\\
33.84	0\\
33.85	0\\
33.86	0\\
33.87	0\\
33.88	0\\
33.89	0\\
33.9	0\\
33.91	0\\
33.92	0\\
33.93	0\\
33.94	0\\
33.95	0\\
33.96	0\\
33.97	0\\
33.98	0\\
33.99	0\\
34	0\\
34.01	0\\
34.02	0\\
34.03	0\\
34.04	0\\
34.05	0\\
34.06	0\\
34.07	0\\
34.08	0\\
34.09	0\\
34.1	0\\
34.11	0\\
34.12	0\\
34.13	0\\
34.14	0\\
34.15	0\\
34.16	0\\
34.17	0\\
34.18	0\\
34.19	0\\
34.2	0\\
34.21	0\\
34.22	0\\
34.23	0\\
34.24	0\\
34.25	0\\
34.26	0\\
34.27	0\\
34.28	0\\
34.29	0\\
34.3	0\\
34.31	0\\
34.32	0\\
34.33	0\\
34.34	0\\
34.35	0\\
34.36	0\\
34.37	0\\
34.38	0\\
34.39	0\\
34.4	0\\
34.41	0\\
34.42	0\\
34.43	0\\
34.44	0\\
34.45	0\\
34.46	0\\
34.47	0\\
34.48	0\\
34.49	0\\
34.5	0\\
34.51	0\\
34.52	0\\
34.53	0\\
34.54	0\\
34.55	0\\
34.56	0\\
34.57	0\\
34.58	0\\
34.59	0\\
34.6	0\\
34.61	0\\
34.62	0\\
34.63	0\\
34.64	0\\
34.65	0\\
34.66	0\\
34.67	0\\
34.68	0\\
34.69	0\\
34.7	0\\
34.71	0\\
34.72	0\\
34.73	0\\
34.74	0\\
34.75	0\\
34.76	0\\
34.77	0\\
34.78	0\\
34.79	0\\
34.8	0\\
34.81	0\\
34.82	0\\
34.83	0\\
34.84	0\\
34.85	0\\
34.86	0\\
34.87	0\\
34.88	0\\
34.89	0\\
34.9	0\\
34.91	0\\
34.92	0\\
34.93	0\\
34.94	0\\
34.95	0\\
34.96	0\\
34.97	0\\
34.98	0\\
34.99	0\\
35	0\\
35.01	0\\
35.02	0\\
35.03	0\\
35.04	0\\
35.05	0\\
35.06	0\\
35.07	0\\
35.08	0\\
35.09	0\\
35.1	0\\
35.11	0\\
35.12	0\\
35.13	0\\
35.14	0\\
35.15	0\\
35.16	0\\
35.17	0\\
35.18	0\\
35.19	0\\
35.2	0\\
35.21	0\\
35.22	0\\
35.23	0\\
35.24	0\\
35.25	0\\
35.26	0\\
35.27	0\\
35.28	0\\
35.29	0\\
35.3	0\\
35.31	0\\
35.32	0\\
35.33	0\\
35.34	0\\
35.35	0\\
35.36	0\\
35.37	0\\
35.38	0\\
35.39	0\\
35.4	0\\
35.41	0\\
35.42	0\\
35.43	0\\
35.44	0\\
35.45	0\\
35.46	0\\
35.47	0\\
35.48	0\\
35.49	0\\
35.5	0\\
35.51	0\\
35.52	0\\
35.53	0\\
35.54	0\\
35.55	0\\
35.56	0\\
35.57	0\\
35.58	0\\
35.59	0\\
35.6	0\\
35.61	0\\
35.62	0\\
35.63	0\\
35.64	0\\
35.65	0\\
35.66	0\\
35.67	0\\
35.68	0\\
35.69	0\\
35.7	0\\
35.71	0\\
35.72	0\\
35.73	0\\
35.74	0\\
35.75	0\\
35.76	0\\
35.77	0\\
35.78	0\\
35.79	0\\
35.8	0\\
35.81	0\\
35.82	0\\
35.83	0\\
35.84	0\\
35.85	0\\
35.86	0\\
35.87	0\\
35.88	0\\
35.89	0\\
35.9	0\\
35.91	0\\
35.92	0\\
35.93	0\\
35.94	0\\
35.95	0\\
35.96	0\\
35.97	0\\
35.98	0\\
35.99	0\\
36	0\\
36.01	0\\
36.02	0\\
36.03	0\\
36.04	0\\
36.05	0\\
36.06	0\\
36.07	0\\
36.08	0\\
36.09	0\\
36.1	0\\
36.11	0\\
36.12	0\\
36.13	0\\
36.14	0\\
36.15	0\\
36.16	0\\
36.17	0\\
36.18	0\\
36.19	0\\
36.2	0\\
36.21	0\\
36.22	0\\
36.23	0\\
36.24	0\\
36.25	0\\
36.26	0\\
36.27	0\\
36.28	0\\
36.29	0\\
36.3	0\\
36.31	0\\
36.32	0\\
36.33	0\\
36.34	0\\
36.35	0\\
36.36	0\\
36.37	0\\
36.38	0\\
36.39	0\\
36.4	0\\
36.41	0\\
36.42	0\\
36.43	0\\
36.44	0\\
36.45	0\\
36.46	0\\
36.47	0\\
36.48	0\\
36.49	0\\
36.5	0\\
36.51	0\\
36.52	0\\
36.53	0\\
36.54	0\\
36.55	0\\
36.56	0\\
36.57	0\\
36.58	0\\
36.59	0\\
36.6	0\\
36.61	0\\
36.62	0\\
36.63	0\\
36.64	0\\
36.65	0\\
36.66	0\\
36.67	0\\
36.68	0\\
36.69	0\\
36.7	0\\
36.71	0\\
36.72	0\\
36.73	0\\
36.74	0\\
36.75	0\\
36.76	0\\
36.77	0\\
36.78	0\\
36.79	0\\
36.8	0\\
36.81	0\\
36.82	0\\
36.83	0\\
36.84	0\\
36.85	0\\
36.86	0\\
36.87	0\\
36.88	0\\
36.89	0\\
36.9	0\\
36.91	0\\
36.92	0\\
36.93	0\\
36.94	0\\
36.95	0\\
36.96	0\\
36.97	0\\
36.98	0\\
36.99	0\\
37	0\\
37.01	0\\
37.02	0\\
37.03	0\\
37.04	0\\
37.05	0\\
37.06	0\\
37.07	0\\
37.08	0\\
37.09	0\\
37.1	0\\
37.11	0\\
37.12	0\\
37.13	0\\
37.14	0\\
37.15	0\\
37.16	0\\
37.17	0\\
37.18	0\\
37.19	0\\
37.2	0\\
37.21	0\\
37.22	0\\
37.23	0\\
37.24	0\\
37.25	0\\
37.26	0\\
37.27	0\\
37.28	0\\
37.29	0\\
37.3	0\\
37.31	0\\
37.32	0\\
37.33	0\\
37.34	0\\
37.35	0\\
37.36	0\\
37.37	0\\
37.38	0\\
37.39	0\\
37.4	0\\
37.41	0\\
37.42	0\\
37.43	0\\
37.44	0\\
37.45	0\\
37.46	0\\
37.47	0\\
37.48	0\\
37.49	0\\
37.5	0\\
37.51	0\\
37.52	0\\
37.53	0\\
37.54	0\\
37.55	0\\
37.56	0\\
37.57	0\\
37.58	0\\
37.59	0\\
37.6	0\\
37.61	0\\
37.62	0\\
37.63	0\\
37.64	0\\
37.65	0\\
37.66	0\\
37.67	0\\
37.68	0\\
37.69	0\\
37.7	0\\
37.71	0\\
37.72	0\\
37.73	0\\
37.74	0\\
37.75	0\\
37.76	0\\
37.77	0\\
37.78	0\\
37.79	0\\
37.8	0\\
37.81	0\\
37.82	0\\
37.83	0\\
37.84	0\\
37.85	0\\
37.86	0\\
37.87	0\\
37.88	0\\
37.89	0\\
37.9	0\\
37.91	0\\
37.92	0\\
37.93	0\\
37.94	0\\
37.95	0\\
37.96	0\\
37.97	0\\
37.98	0\\
37.99	0\\
38	0\\
38.01	0\\
38.02	0\\
38.03	0\\
38.04	0\\
38.05	0\\
38.06	0\\
38.07	0\\
38.08	0\\
38.09	0\\
38.1	0\\
38.11	0\\
38.12	0\\
38.13	0\\
38.14	0\\
38.15	0\\
38.16	0\\
38.17	0\\
38.18	0\\
38.19	0\\
38.2	0\\
38.21	0\\
38.22	0\\
38.23	0\\
38.24	0\\
38.25	0\\
38.26	0\\
38.27	0\\
38.28	0\\
38.29	0\\
38.3	0\\
38.31	0\\
38.32	0\\
38.33	0\\
38.34	0\\
38.35	0\\
38.36	0\\
38.37	0\\
38.38	0\\
38.39	0\\
38.4	0\\
38.41	0\\
38.42	0\\
38.43	0\\
38.44	0\\
38.45	0\\
38.46	0\\
38.47	0\\
38.48	0\\
38.49	0\\
38.5	0\\
38.51	0\\
38.52	0\\
38.53	0\\
38.54	0\\
38.55	0\\
38.56	0\\
38.57	0\\
38.58	0\\
38.59	0\\
38.6	0\\
38.61	0\\
38.62	0\\
38.63	0\\
38.64	0\\
38.65	0\\
38.66	0\\
38.67	0\\
38.68	0\\
38.69	0\\
38.7	0\\
38.71	0\\
38.72	0\\
38.73	0\\
38.74	0\\
38.75	0\\
38.76	0\\
38.77	0\\
38.78	0\\
38.79	0\\
38.8	0\\
38.81	0\\
38.82	0\\
38.83	0\\
38.84	0\\
38.85	0\\
38.86	0\\
38.87	0\\
38.88	0\\
38.89	0\\
38.9	0\\
38.91	0\\
38.92	0\\
38.93	0\\
38.94	0\\
38.95	0\\
38.96	0\\
38.97	0\\
38.98	0\\
38.99	0\\
39	0\\
39.01	0\\
39.02	0\\
39.03	0\\
39.04	0\\
39.05	0\\
39.06	0\\
39.07	0\\
39.08	0\\
39.09	0\\
39.1	0\\
39.11	0\\
39.12	0\\
39.13	0\\
39.14	0\\
39.15	0\\
39.16	0\\
39.17	0\\
39.18	0\\
39.19	0\\
39.2	0\\
39.21	0\\
39.22	0\\
39.23	0\\
39.24	0\\
39.25	0\\
39.26	0\\
39.27	0\\
39.28	0\\
39.29	0\\
39.3	0\\
39.31	0\\
39.32	0\\
39.33	0\\
39.34	0\\
39.35	0\\
39.36	0\\
39.37	0\\
39.38	0\\
39.39	0\\
39.4	0\\
39.41	0\\
39.42	0\\
39.43	0\\
39.44	0\\
39.45	0\\
39.46	0\\
39.47	0\\
39.48	0\\
39.49	0\\
39.5	0\\
39.51	0\\
39.52	0\\
39.53	0\\
39.54	0\\
39.55	0\\
39.56	0\\
39.57	0\\
39.58	0\\
39.59	0\\
39.6	0\\
39.61	0\\
39.62	0\\
39.63	0\\
39.64	0\\
39.65	0\\
39.66	0\\
39.67	0\\
39.68	0\\
39.69	0\\
39.7	0\\
39.71	0\\
39.72	0\\
39.73	0\\
39.74	0\\
39.75	0\\
39.76	0\\
39.77	0\\
39.78	0\\
39.79	0\\
39.8	0\\
39.81	0\\
39.82	0\\
39.83	0\\
39.84	0\\
39.85	0\\
39.86	0\\
39.87	0\\
39.88	0\\
39.89	0\\
39.9	0\\
39.91	0\\
39.92	0\\
39.93	0\\
39.94	0\\
39.95	0\\
39.96	0\\
39.97	0\\
39.98	0\\
39.99	0\\
40	0\\
40.01	0\\
};
\addplot [color=black,solid,forget plot]
  table[row sep=crcr]{%
40.01	0\\
40.02	0\\
40.03	0\\
40.04	0\\
40.05	0\\
40.06	0\\
40.07	0\\
40.08	0\\
40.09	0\\
40.1	0\\
40.11	0\\
40.12	0\\
40.13	0\\
40.14	0\\
40.15	0\\
40.16	0\\
40.17	0\\
40.18	0\\
40.19	0\\
40.2	0\\
40.21	0\\
40.22	0\\
40.23	0\\
40.24	0\\
40.25	0\\
40.26	0\\
40.27	0\\
40.28	0\\
40.29	0\\
40.3	0\\
40.31	0\\
40.32	0\\
40.33	0\\
40.34	0\\
40.35	0\\
40.36	0\\
40.37	0\\
40.38	0\\
40.39	0\\
40.4	0\\
40.41	0\\
40.42	0\\
40.43	0\\
40.44	0\\
40.45	0\\
40.46	0\\
40.47	0\\
40.48	0\\
40.49	0\\
40.5	0\\
40.51	0\\
40.52	0\\
40.53	0\\
40.54	0\\
40.55	0\\
40.56	0\\
40.57	0\\
40.58	0\\
40.59	0\\
40.6	0\\
40.61	0\\
40.62	0\\
40.63	0\\
40.64	0\\
40.65	0\\
40.66	0\\
40.67	0\\
40.68	0\\
40.69	0\\
40.7	0\\
40.71	0\\
40.72	0\\
40.73	0\\
40.74	0\\
40.75	0\\
40.76	0\\
40.77	0\\
40.78	0\\
40.79	0\\
40.8	0\\
40.81	0\\
40.82	0\\
40.83	0\\
40.84	0\\
40.85	0\\
40.86	0\\
40.87	0\\
40.88	0\\
40.89	0\\
40.9	0\\
40.91	0\\
40.92	0\\
40.93	0\\
40.94	0\\
40.95	0\\
40.96	0\\
40.97	0\\
40.98	0\\
40.99	0\\
41	0\\
41.01	0\\
41.02	0\\
41.03	0\\
41.04	0\\
41.05	0\\
41.06	0\\
41.07	0\\
41.08	0\\
41.09	0\\
41.1	0\\
41.11	0\\
41.12	0\\
41.13	0\\
41.14	0\\
41.15	0\\
41.16	0\\
41.17	0\\
41.18	0\\
41.19	0\\
41.2	0\\
41.21	0\\
41.22	0\\
41.23	0\\
41.24	0\\
41.25	0\\
41.26	0\\
41.27	0\\
41.28	0\\
41.29	0\\
41.3	0\\
41.31	0\\
41.32	0\\
41.33	0\\
41.34	0\\
41.35	0\\
41.36	0\\
41.37	0\\
41.38	0\\
41.39	0\\
41.4	0\\
41.41	0\\
41.42	0\\
41.43	0\\
41.44	0\\
41.45	0\\
41.46	0\\
41.47	0\\
41.48	0\\
41.49	0\\
41.5	0\\
41.51	0\\
41.52	0\\
41.53	0\\
41.54	0\\
41.55	0\\
41.56	0\\
41.57	0\\
41.58	0\\
41.59	0\\
41.6	0\\
41.61	0\\
41.62	0\\
41.63	0\\
41.64	0\\
41.65	0\\
41.66	0\\
41.67	0\\
41.68	0\\
41.69	0\\
41.7	0\\
41.71	0\\
41.72	0\\
41.73	0\\
41.74	0\\
41.75	0\\
41.76	0\\
41.77	0\\
41.78	0\\
41.79	0\\
41.8	0\\
41.81	0\\
41.82	0\\
41.83	0\\
41.84	0\\
41.85	0\\
41.86	0\\
41.87	0\\
41.88	0\\
41.89	0\\
41.9	0\\
41.91	0\\
41.92	0\\
41.93	0\\
41.94	0\\
41.95	0\\
41.96	0\\
41.97	0\\
41.98	0\\
41.99	0\\
42	0\\
42.01	0\\
42.02	0\\
42.03	0\\
42.04	0\\
42.05	0\\
42.06	0\\
42.07	0\\
42.08	0\\
42.09	0\\
42.1	0\\
42.11	0\\
42.12	0\\
42.13	0\\
42.14	0\\
42.15	0\\
42.16	0\\
42.17	0\\
42.18	0\\
42.19	0\\
42.2	0\\
42.21	0\\
42.22	0\\
42.23	0\\
42.24	0\\
42.25	0\\
42.26	0\\
42.27	0\\
42.28	0\\
42.29	0\\
42.3	0\\
42.31	0\\
42.32	0\\
42.33	0\\
42.34	0\\
42.35	0\\
42.36	0\\
42.37	0\\
42.38	0\\
42.39	0\\
42.4	0\\
42.41	0\\
42.42	0\\
42.43	0\\
42.44	0\\
42.45	0\\
42.46	0\\
42.47	0\\
42.48	0\\
42.49	0\\
42.5	0\\
42.51	0\\
42.52	0\\
42.53	0\\
42.54	0\\
42.55	0\\
42.56	0\\
42.57	0\\
42.58	0\\
42.59	0\\
42.6	0\\
42.61	0\\
42.62	0\\
42.63	0\\
42.64	0\\
42.65	0\\
42.66	0\\
42.67	0\\
42.68	0\\
42.69	0\\
42.7	0\\
42.71	0\\
42.72	0\\
42.73	0\\
42.74	0\\
42.75	0\\
42.76	0\\
42.77	0\\
42.78	0\\
42.79	0\\
42.8	0\\
42.81	0\\
42.82	0\\
42.83	0\\
42.84	0\\
42.85	0\\
42.86	0\\
42.87	0\\
42.88	0\\
42.89	0\\
42.9	0\\
42.91	0\\
42.92	0\\
42.93	0\\
42.94	0\\
42.95	0\\
42.96	0\\
42.97	0\\
42.98	0\\
42.99	0\\
43	0\\
43.01	0\\
43.02	0\\
43.03	0\\
43.04	0\\
43.05	0\\
43.06	0\\
43.07	0\\
43.08	0\\
43.09	0\\
43.1	0\\
43.11	0\\
43.12	0\\
43.13	0\\
43.14	0\\
43.15	0\\
43.16	0\\
43.17	0\\
43.18	0\\
43.19	0\\
43.2	0\\
43.21	0\\
43.22	0\\
43.23	0\\
43.24	0\\
43.25	0\\
43.26	0\\
43.27	0\\
43.28	0\\
43.29	0\\
43.3	0\\
43.31	0\\
43.32	0\\
43.33	0\\
43.34	0\\
43.35	0\\
43.36	0\\
43.37	0\\
43.38	0\\
43.39	0\\
43.4	0\\
43.41	0\\
43.42	0\\
43.43	0\\
43.44	0\\
43.45	0\\
43.46	0\\
43.47	0\\
43.48	0\\
43.49	0\\
43.5	0\\
43.51	0\\
43.52	0\\
43.53	0\\
43.54	0\\
43.55	0\\
43.56	0\\
43.57	0\\
43.58	0\\
43.59	0\\
43.6	0\\
43.61	0\\
43.62	0\\
43.63	0\\
43.64	0\\
43.65	0\\
43.66	0\\
43.67	0\\
43.68	0\\
43.69	0\\
43.7	0\\
43.71	0\\
43.72	0\\
43.73	0\\
43.74	0\\
43.75	0\\
43.76	0\\
43.77	0\\
43.78	0\\
43.79	0\\
43.8	0\\
43.81	0\\
43.82	0\\
43.83	0\\
43.84	0\\
43.85	0\\
43.86	0\\
43.87	0\\
43.88	0\\
43.89	0\\
43.9	0\\
43.91	0\\
43.92	0\\
43.93	0\\
43.94	0\\
43.95	0\\
43.96	0\\
43.97	0\\
43.98	0\\
43.99	0\\
44	0\\
44.01	0\\
44.02	0\\
44.03	0\\
44.04	0\\
44.05	0\\
44.06	0\\
44.07	0\\
44.08	0\\
44.09	0\\
44.1	0\\
44.11	0\\
44.12	0\\
44.13	0\\
44.14	0\\
44.15	0\\
44.16	0\\
44.17	0\\
44.18	0\\
44.19	0\\
44.2	0\\
44.21	0\\
44.22	0\\
44.23	0\\
44.24	0\\
44.25	0\\
44.26	0\\
44.27	0\\
44.28	0\\
44.29	0\\
44.3	0\\
44.31	0\\
44.32	0\\
44.33	0\\
44.34	0\\
44.35	0\\
44.36	0\\
44.37	0\\
44.38	0\\
44.39	0\\
44.4	0\\
44.41	0\\
44.42	0\\
44.43	0\\
44.44	0\\
44.45	0\\
44.46	0\\
44.47	0\\
44.48	0\\
44.49	0\\
44.5	0\\
44.51	0\\
44.52	0\\
44.53	0\\
44.54	0\\
44.55	0\\
44.56	0\\
44.57	0\\
44.58	0\\
44.59	0\\
44.6	0\\
44.61	0\\
44.62	0\\
44.63	0\\
44.64	0\\
44.65	0\\
44.66	0\\
44.67	0\\
44.68	0\\
44.69	0\\
44.7	0\\
44.71	0\\
44.72	0\\
44.73	0\\
44.74	0\\
44.75	0\\
44.76	0\\
44.77	0\\
44.78	0\\
44.79	0\\
44.8	0\\
44.81	0\\
44.82	0\\
44.83	0\\
44.84	0\\
44.85	0\\
44.86	0\\
44.87	0\\
44.88	0\\
44.89	0\\
44.9	0\\
44.91	0\\
44.92	0\\
44.93	0\\
44.94	0\\
44.95	0\\
44.96	0\\
44.97	0\\
44.98	0\\
44.99	0\\
45	0\\
45.01	0\\
45.02	0\\
45.03	0\\
45.04	0\\
45.05	0\\
45.06	0\\
45.07	0\\
45.08	0\\
45.09	0\\
45.1	0\\
45.11	0\\
45.12	0\\
45.13	0\\
45.14	0\\
45.15	0\\
45.16	0\\
45.17	0\\
45.18	0\\
45.19	0\\
45.2	0\\
45.21	0\\
45.22	0\\
45.23	0\\
45.24	0\\
45.25	0\\
45.26	0\\
45.27	0\\
45.28	0\\
45.29	0\\
45.3	0\\
45.31	0\\
45.32	0\\
45.33	0\\
45.34	0\\
45.35	0\\
45.36	0\\
45.37	0\\
45.38	0\\
45.39	0\\
45.4	0\\
45.41	0\\
45.42	0\\
45.43	0\\
45.44	0\\
45.45	0\\
45.46	0\\
45.47	0\\
45.48	0\\
45.49	0\\
45.5	0\\
45.51	0\\
45.52	0\\
45.53	0\\
45.54	0\\
45.55	0\\
45.56	0\\
45.57	0\\
45.58	0\\
45.59	0\\
45.6	0\\
45.61	0\\
45.62	0\\
45.63	0\\
45.64	0\\
45.65	0\\
45.66	0\\
45.67	0\\
45.68	0\\
45.69	0\\
45.7	0\\
45.71	0\\
45.72	0\\
45.73	0\\
45.74	0\\
45.75	0\\
45.76	0\\
45.77	0\\
45.78	0\\
45.79	0\\
45.8	0\\
45.81	0\\
45.82	0\\
45.83	0\\
45.84	0\\
45.85	0\\
45.86	0\\
45.87	0\\
45.88	0\\
45.89	0\\
45.9	0\\
45.91	0\\
45.92	0\\
45.93	0\\
45.94	0\\
45.95	0\\
45.96	0\\
45.97	0\\
45.98	0\\
45.99	0\\
46	0\\
46.01	0\\
46.02	0\\
46.03	0\\
46.04	0\\
46.05	0\\
46.06	0\\
46.07	0\\
46.08	0\\
46.09	0\\
46.1	0\\
46.11	0\\
46.12	0\\
46.13	0\\
46.14	0\\
46.15	0\\
46.16	0\\
46.17	0\\
46.18	0\\
46.19	0\\
46.2	0\\
46.21	0\\
46.22	0\\
46.23	0\\
46.24	0\\
46.25	0\\
46.26	0\\
46.27	0\\
46.28	0\\
46.29	0\\
46.3	0\\
46.31	0\\
46.32	0\\
46.33	0\\
46.34	0\\
46.35	0\\
46.36	0\\
46.37	0\\
46.38	0\\
46.39	0\\
46.4	0\\
46.41	0\\
46.42	0\\
46.43	0\\
46.44	0\\
46.45	0\\
46.46	0\\
46.47	0\\
46.48	0\\
46.49	0\\
46.5	0\\
46.51	0\\
46.52	0\\
46.53	0\\
46.54	0\\
46.55	0\\
46.56	0\\
46.57	0\\
46.58	0\\
46.59	0\\
46.6	0\\
46.61	0\\
46.62	0\\
46.63	0\\
46.64	0\\
46.65	0\\
46.66	0\\
46.67	0\\
46.68	0\\
46.69	0\\
46.7	0\\
46.71	0\\
46.72	0\\
46.73	0\\
46.74	0\\
46.75	0\\
46.76	0\\
46.77	0\\
46.78	0\\
46.79	0\\
46.8	0\\
46.81	0\\
46.82	0\\
46.83	0\\
46.84	0\\
46.85	0\\
46.86	0\\
46.87	0\\
46.88	0\\
46.89	0\\
46.9	0\\
46.91	0\\
46.92	0\\
46.93	0\\
46.94	0\\
46.95	0\\
46.96	0\\
46.97	0\\
46.98	0\\
46.99	0\\
47	0\\
47.01	0\\
47.02	0\\
47.03	0\\
47.04	0\\
47.05	0\\
47.06	0\\
47.07	0\\
47.08	0\\
47.09	0\\
47.1	0\\
47.11	0\\
47.12	0\\
47.13	0\\
47.14	0\\
47.15	0\\
47.16	0\\
47.17	0\\
47.18	0\\
47.19	0\\
47.2	0\\
47.21	0\\
47.22	0\\
47.23	0\\
47.24	0\\
47.25	0\\
47.26	0\\
47.27	0\\
47.28	0\\
47.29	0\\
47.3	0\\
47.31	0\\
47.32	0\\
47.33	0\\
47.34	0\\
47.35	0\\
47.36	0\\
47.37	0\\
47.38	0\\
47.39	0\\
47.4	0\\
47.41	0\\
47.42	0\\
47.43	0\\
47.44	0\\
47.45	0\\
47.46	0\\
47.47	0\\
47.48	0\\
47.49	0\\
47.5	0\\
47.51	0\\
47.52	0\\
47.53	0\\
47.54	0\\
47.55	0\\
47.56	0\\
47.57	0\\
47.58	0\\
47.59	0\\
47.6	0\\
47.61	0\\
47.62	0\\
47.63	0\\
47.64	0\\
47.65	0\\
47.66	0\\
47.67	0\\
47.68	0\\
47.69	0\\
47.7	0\\
47.71	0\\
47.72	0\\
47.73	0\\
47.74	0\\
47.75	0\\
47.76	0\\
47.77	0\\
47.78	0\\
47.79	0\\
47.8	0\\
47.81	0\\
47.82	0\\
47.83	0\\
47.84	0\\
47.85	0\\
47.86	0\\
47.87	0\\
47.88	0\\
47.89	0\\
47.9	0\\
47.91	0\\
47.92	0\\
47.93	0\\
47.94	0\\
47.95	0\\
47.96	0\\
47.97	0\\
47.98	0\\
47.99	0\\
48	0\\
48.01	0\\
48.02	0\\
48.03	0\\
48.04	0\\
48.05	0\\
48.06	0\\
48.07	0\\
48.08	0\\
48.09	0\\
48.1	0\\
48.11	0\\
48.12	0\\
48.13	0\\
48.14	0\\
48.15	0\\
48.16	0\\
48.17	0\\
48.18	0\\
48.19	0\\
48.2	0\\
48.21	0\\
48.22	0\\
48.23	0\\
48.24	0\\
48.25	0\\
48.26	0\\
48.27	0\\
48.28	0\\
48.29	0\\
48.3	0\\
48.31	0\\
48.32	0\\
48.33	0\\
48.34	0\\
48.35	0\\
48.36	0\\
48.37	0\\
48.38	0\\
48.39	0\\
48.4	0\\
48.41	0\\
48.42	0\\
48.43	0\\
48.44	0\\
48.45	0\\
48.46	0\\
48.47	0\\
48.48	0\\
48.49	0\\
48.5	0\\
48.51	0\\
48.52	0\\
48.53	0\\
48.54	0\\
48.55	0\\
48.56	0\\
48.57	0\\
48.58	0\\
48.59	0\\
48.6	0\\
48.61	0\\
48.62	0\\
48.63	0\\
48.64	0\\
48.65	0\\
48.66	0\\
48.67	0\\
48.68	0\\
48.69	0\\
48.7	0\\
48.71	0\\
48.72	0\\
48.73	0\\
48.74	0\\
48.75	0\\
48.76	0\\
48.77	0\\
48.78	0\\
48.79	0\\
48.8	0\\
48.81	0\\
48.82	0\\
48.83	0\\
48.84	0\\
48.85	0\\
48.86	0\\
48.87	0\\
48.88	0\\
48.89	0\\
48.9	0\\
48.91	0\\
48.92	0\\
48.93	0\\
48.94	0\\
48.95	0\\
48.96	0\\
48.97	0\\
48.98	0\\
48.99	0\\
49	0\\
49.01	0\\
49.02	0\\
49.03	0\\
49.04	0\\
49.05	0\\
49.06	0\\
49.07	0\\
49.08	0\\
49.09	0\\
49.1	0\\
49.11	0\\
49.12	0\\
49.13	0\\
49.14	0\\
49.15	0\\
49.16	0\\
49.17	0\\
49.18	0\\
49.19	0\\
49.2	0\\
49.21	0\\
49.22	0\\
49.23	0\\
49.24	0\\
49.25	0\\
49.26	0\\
49.27	0\\
49.28	0\\
49.29	0\\
49.3	0\\
49.31	0\\
49.32	0\\
49.33	0\\
49.34	0\\
49.35	0\\
49.36	0\\
49.37	0\\
49.38	0\\
49.39	0\\
49.4	0\\
49.41	0\\
49.42	0\\
49.43	0\\
49.44	0\\
49.45	0\\
49.46	0\\
49.47	0\\
49.48	0\\
49.49	0\\
49.5	0\\
49.51	0\\
49.52	0\\
49.53	0\\
49.54	0\\
49.55	0\\
49.56	0\\
49.57	0\\
49.58	0\\
49.59	0\\
49.6	0\\
49.61	0\\
49.62	0\\
49.63	0\\
49.64	0\\
49.65	0\\
49.66	0\\
49.67	0\\
49.68	0\\
49.69	0\\
49.7	0\\
49.71	0\\
49.72	0\\
49.73	0\\
49.74	0\\
49.75	0\\
49.76	0\\
49.77	0\\
49.78	0\\
49.79	0\\
49.8	0\\
49.81	0\\
49.82	0\\
49.83	0\\
49.84	0\\
49.85	0\\
49.86	0\\
49.87	0\\
49.88	0\\
49.89	0\\
49.9	0\\
49.91	0\\
49.92	0\\
49.93	0\\
49.94	0\\
49.95	0\\
49.96	0\\
49.97	0\\
49.98	0\\
49.99	0\\
50	0\\
50.01	0\\
50.02	0\\
50.03	0\\
50.04	0\\
50.05	0\\
50.06	0\\
50.07	0\\
50.08	0\\
50.09	0\\
50.1	0\\
50.11	0\\
50.12	0\\
50.13	0\\
50.14	0\\
50.15	0\\
50.16	0\\
50.17	0\\
50.18	0\\
50.19	0\\
50.2	0\\
50.21	0\\
50.22	0\\
50.23	0\\
50.24	0\\
50.25	0\\
50.26	0\\
50.27	0\\
50.28	0\\
50.29	0\\
50.3	0\\
50.31	0\\
50.32	0\\
50.33	0\\
50.34	0\\
50.35	0\\
50.36	0\\
50.37	0\\
50.38	0\\
50.39	0\\
50.4	0\\
50.41	0\\
50.42	0\\
50.43	0\\
50.44	0\\
50.45	0\\
50.46	0\\
50.47	0\\
50.48	0\\
50.49	0\\
50.5	0\\
50.51	0\\
50.52	0\\
50.53	0\\
50.54	0\\
50.55	0\\
50.56	0\\
50.57	0\\
50.58	0\\
50.59	0\\
50.6	0\\
50.61	0\\
50.62	0\\
50.63	0\\
50.64	0\\
50.65	0\\
50.66	0\\
50.67	0\\
50.68	0\\
50.69	0\\
50.7	0\\
50.71	0\\
50.72	0\\
50.73	0\\
50.74	0\\
50.75	0\\
50.76	0\\
50.77	0\\
50.78	0\\
50.79	0\\
50.8	0\\
50.81	0\\
50.82	0\\
50.83	0\\
50.84	0\\
50.85	0\\
50.86	0\\
50.87	0\\
50.88	0\\
50.89	0\\
50.9	0\\
50.91	0\\
50.92	0\\
50.93	0\\
50.94	0\\
50.95	0\\
50.96	0\\
50.97	0\\
50.98	0\\
50.99	0\\
51	0\\
51.01	0\\
51.02	0\\
51.03	0\\
51.04	0\\
51.05	0\\
51.06	0\\
51.07	0\\
51.08	0\\
51.09	0\\
51.1	0\\
51.11	0\\
51.12	0\\
51.13	0\\
51.14	0\\
51.15	0\\
51.16	0\\
51.17	0\\
51.18	0\\
51.19	0\\
51.2	0\\
51.21	0\\
51.22	0\\
51.23	0\\
51.24	0\\
51.25	0\\
51.26	0\\
51.27	0\\
51.28	0\\
51.29	0\\
51.3	0\\
51.31	0\\
51.32	0\\
51.33	0\\
51.34	0\\
51.35	0\\
51.36	0\\
51.37	0\\
51.38	0\\
51.39	0\\
51.4	0\\
51.41	0\\
51.42	0\\
51.43	0\\
51.44	0\\
51.45	0\\
51.46	0\\
51.47	0\\
51.48	0\\
51.49	0\\
51.5	0\\
51.51	0\\
51.52	0\\
51.53	0\\
51.54	0\\
51.55	0\\
51.56	0\\
51.57	0\\
51.58	0\\
51.59	0\\
51.6	0\\
51.61	0\\
51.62	0\\
51.63	0\\
51.64	0\\
51.65	0\\
51.66	0\\
51.67	0\\
51.68	0\\
51.69	0\\
51.7	0\\
51.71	0\\
51.72	0\\
51.73	0\\
51.74	0\\
51.75	0\\
51.76	0\\
51.77	0\\
51.78	0\\
51.79	0\\
51.8	0\\
51.81	0\\
51.82	0\\
51.83	0\\
51.84	0\\
51.85	0\\
51.86	0\\
51.87	0\\
51.88	0\\
51.89	0\\
51.9	0\\
51.91	0\\
51.92	0\\
51.93	0\\
51.94	0\\
51.95	0\\
51.96	0\\
51.97	0\\
51.98	0\\
51.99	0\\
52	0\\
52.01	0\\
52.02	0\\
52.03	0\\
52.04	0\\
52.05	0\\
52.06	0\\
52.07	0\\
52.08	0\\
52.09	0\\
52.1	0\\
52.11	0\\
52.12	0\\
52.13	0\\
52.14	0\\
52.15	0\\
52.16	0\\
52.17	0\\
52.18	0\\
52.19	0\\
52.2	0\\
52.21	0\\
52.22	0\\
52.23	0\\
52.24	0\\
52.25	0\\
52.26	0\\
52.27	0\\
52.28	0\\
52.29	0\\
52.3	0\\
52.31	0\\
52.32	0\\
52.33	0\\
52.34	0\\
52.35	0\\
52.36	0\\
52.37	0\\
52.38	0\\
52.39	0\\
52.4	0\\
52.41	0\\
52.42	0\\
52.43	0\\
52.44	0\\
52.45	0\\
52.46	0\\
52.47	0\\
52.48	0\\
52.49	0\\
52.5	0\\
52.51	0\\
52.52	0\\
52.53	0\\
52.54	0\\
52.55	0\\
52.56	0\\
52.57	0\\
52.58	0\\
52.59	0\\
52.6	0\\
52.61	0\\
52.62	0\\
52.63	0\\
52.64	0\\
52.65	0\\
52.66	0\\
52.67	0\\
52.68	0\\
52.69	0\\
52.7	0\\
52.71	0\\
52.72	0\\
52.73	0\\
52.74	0\\
52.75	0\\
52.76	0\\
52.77	0\\
52.78	0\\
52.79	0\\
52.8	0\\
52.81	0\\
52.82	0\\
52.83	0\\
52.84	0\\
52.85	0\\
52.86	0\\
52.87	0\\
52.88	0\\
52.89	0\\
52.9	0\\
52.91	0\\
52.92	0\\
52.93	0\\
52.94	0\\
52.95	0\\
52.96	0\\
52.97	0\\
52.98	0\\
52.99	0\\
53	0\\
53.01	0\\
53.02	0\\
53.03	0\\
53.04	0\\
53.05	0\\
53.06	0\\
53.07	0\\
53.08	0\\
53.09	0\\
53.1	0\\
53.11	0\\
53.12	0\\
53.13	0\\
53.14	0\\
53.15	0\\
53.16	0\\
53.17	0\\
53.18	0\\
53.19	0\\
53.2	0\\
53.21	0\\
53.22	0\\
53.23	0\\
53.24	0\\
53.25	0\\
53.26	0\\
53.27	0\\
53.28	0\\
53.29	0\\
53.3	0\\
53.31	0\\
53.32	0\\
53.33	0\\
53.34	0\\
53.35	0\\
53.36	0\\
53.37	0\\
53.38	0\\
53.39	0\\
53.4	0\\
53.41	0\\
53.42	0\\
53.43	0\\
53.44	0\\
53.45	0\\
53.46	0\\
53.47	0\\
53.48	0\\
53.49	0\\
53.5	0\\
53.51	0\\
53.52	0\\
53.53	0\\
53.54	0\\
53.55	0\\
53.56	0\\
53.57	0\\
53.58	0\\
53.59	0\\
53.6	0\\
53.61	0\\
53.62	0\\
53.63	0\\
53.64	0\\
53.65	0\\
53.66	0\\
53.67	0\\
53.68	0\\
53.69	0\\
53.7	0\\
53.71	0\\
53.72	0\\
53.73	0\\
53.74	0\\
53.75	0\\
53.76	0\\
53.77	0\\
53.78	0\\
53.79	0\\
53.8	0\\
53.81	0\\
53.82	0\\
53.83	0\\
53.84	0\\
53.85	0\\
53.86	0\\
53.87	0\\
53.88	0\\
53.89	0\\
53.9	0\\
53.91	0\\
53.92	0\\
53.93	0\\
53.94	0\\
53.95	0\\
53.96	0\\
53.97	0\\
53.98	0\\
53.99	0\\
54	0\\
54.01	0\\
54.02	0\\
54.03	0\\
54.04	0\\
54.05	0\\
54.06	0\\
54.07	0\\
54.08	0\\
54.09	0\\
54.1	0\\
54.11	0\\
54.12	0\\
54.13	0\\
54.14	0\\
54.15	0\\
54.16	0\\
54.17	0\\
54.18	0\\
54.19	0\\
54.2	0\\
54.21	0\\
54.22	0\\
54.23	0\\
54.24	0\\
54.25	0\\
54.26	0\\
54.27	0\\
54.28	0\\
54.29	0\\
54.3	0\\
54.31	0\\
54.32	0\\
54.33	0\\
54.34	0\\
54.35	0\\
54.36	0\\
54.37	0\\
54.38	0\\
54.39	0\\
54.4	0\\
54.41	0\\
54.42	0\\
54.43	0\\
54.44	0\\
54.45	0\\
54.46	0\\
54.47	0\\
54.48	0\\
54.49	0\\
54.5	0\\
54.51	0\\
54.52	0\\
54.53	0\\
54.54	0\\
54.55	0\\
54.56	0\\
54.57	0\\
54.58	0\\
54.59	0\\
54.6	0\\
54.61	0\\
54.62	0\\
54.63	0\\
54.64	0\\
54.65	0\\
54.66	0\\
54.67	0\\
54.68	0\\
54.69	0\\
54.7	0\\
54.71	0\\
54.72	0\\
54.73	0\\
54.74	0\\
54.75	0\\
54.76	0\\
54.77	0\\
54.78	0\\
54.79	0\\
54.8	0\\
54.81	0\\
54.82	0\\
54.83	0\\
54.84	0\\
54.85	0\\
54.86	0\\
54.87	0\\
54.88	0\\
54.89	0\\
54.9	0\\
54.91	0\\
54.92	0\\
54.93	0\\
54.94	0\\
54.95	0\\
54.96	0\\
54.97	0\\
54.98	0\\
54.99	0\\
55	0\\
55.01	0\\
55.02	0\\
55.03	0\\
55.04	0\\
55.05	0\\
55.06	0\\
55.07	0\\
55.08	0\\
55.09	0\\
55.1	0\\
55.11	0\\
55.12	0\\
55.13	0\\
55.14	0\\
55.15	0\\
55.16	0\\
55.17	0\\
55.18	0\\
55.19	0\\
55.2	0\\
55.21	0\\
55.22	0\\
55.23	0\\
55.24	0\\
55.25	0\\
55.26	0\\
55.27	0\\
55.28	0\\
55.29	0\\
55.3	0\\
55.31	0\\
55.32	0\\
55.33	0\\
55.34	0\\
55.35	0\\
55.36	0\\
55.37	0\\
55.38	0\\
55.39	0\\
55.4	0\\
55.41	0\\
55.42	0\\
55.43	0\\
55.44	0\\
55.45	0\\
55.46	0\\
55.47	0\\
55.48	0\\
55.49	0\\
55.5	0\\
55.51	0\\
55.52	0\\
55.53	0\\
55.54	0\\
55.55	0\\
55.56	0\\
55.57	0\\
55.58	0\\
55.59	0\\
55.6	0\\
55.61	0\\
55.62	0\\
55.63	0\\
55.64	0\\
55.65	0\\
55.66	0\\
55.67	0\\
55.68	0\\
55.69	0\\
55.7	0\\
55.71	0\\
55.72	0\\
55.73	0\\
55.74	0\\
55.75	0\\
55.76	0\\
55.77	0\\
55.78	0\\
55.79	0\\
55.8	0\\
55.81	0\\
55.82	0\\
55.83	0\\
55.84	0\\
55.85	0\\
55.86	0\\
55.87	0\\
55.88	0\\
55.89	0\\
55.9	0\\
55.91	0\\
55.92	0\\
55.93	0\\
55.94	0\\
55.95	0\\
55.96	0\\
55.97	0\\
55.98	0\\
55.99	0\\
56	0\\
56.01	0\\
56.02	0\\
56.03	0\\
56.04	0\\
56.05	0\\
56.06	0\\
56.07	0\\
56.08	0\\
56.09	0\\
56.1	0\\
56.11	0\\
56.12	0\\
56.13	0\\
56.14	0\\
56.15	0\\
56.16	0\\
56.17	0\\
56.18	0\\
56.19	0\\
56.2	0\\
56.21	0\\
56.22	0\\
56.23	0\\
56.24	0\\
56.25	0\\
56.26	0\\
56.27	0\\
56.28	0\\
56.29	0\\
56.3	0\\
56.31	0\\
56.32	0\\
56.33	0\\
56.34	0\\
56.35	0\\
56.36	0\\
56.37	0\\
56.38	0\\
56.39	0\\
56.4	0\\
56.41	0\\
56.42	0\\
56.43	0\\
56.44	0\\
56.45	0\\
56.46	0\\
56.47	0\\
56.48	0\\
56.49	0\\
56.5	0\\
56.51	0\\
56.52	0\\
56.53	0\\
56.54	0\\
56.55	0\\
56.56	0\\
56.57	0\\
56.58	0\\
56.59	0\\
56.6	0\\
56.61	0\\
56.62	0\\
56.63	0\\
56.64	0\\
56.65	0\\
56.66	0\\
56.67	0\\
56.68	0\\
56.69	0\\
56.7	0\\
56.71	0\\
56.72	0\\
56.73	0\\
56.74	0\\
56.75	0\\
56.76	0\\
56.77	0\\
56.78	0\\
56.79	0\\
56.8	0\\
56.81	0\\
56.82	0\\
56.83	0\\
56.84	0\\
56.85	0\\
56.86	0\\
56.87	0\\
56.88	0\\
56.89	0\\
56.9	0\\
56.91	0\\
56.92	0\\
56.93	0\\
56.94	0\\
56.95	0\\
56.96	0\\
56.97	0\\
56.98	0\\
56.99	0\\
57	0\\
57.01	0\\
57.02	0\\
57.03	0\\
57.04	0\\
57.05	0\\
57.06	0\\
57.07	0\\
57.08	0\\
57.09	0\\
57.1	0\\
57.11	0\\
57.12	0\\
57.13	0\\
57.14	0\\
57.15	0\\
57.16	0\\
57.17	0\\
57.18	0\\
57.19	0\\
57.2	0\\
57.21	0\\
57.22	0\\
57.23	0\\
57.24	0\\
57.25	0\\
57.26	0\\
57.27	0\\
57.28	0\\
57.29	0\\
57.3	0\\
57.31	0\\
57.32	0\\
57.33	0\\
57.34	0\\
57.35	0\\
57.36	0\\
57.37	0\\
57.38	0\\
57.39	0\\
57.4	0\\
57.41	0\\
57.42	0\\
57.43	0\\
57.44	0\\
57.45	0\\
57.46	0\\
57.47	0\\
57.48	0\\
57.49	0\\
57.5	0\\
57.51	0\\
57.52	0\\
57.53	0\\
57.54	0\\
57.55	0\\
57.56	0\\
57.57	0\\
57.58	0\\
57.59	0\\
57.6	0\\
57.61	0\\
57.62	0\\
57.63	0\\
57.64	0\\
57.65	0\\
57.66	0\\
57.67	0\\
57.68	0\\
57.69	0\\
57.7	0\\
57.71	0\\
57.72	0\\
57.73	0\\
57.74	0\\
57.75	0\\
57.76	0\\
57.77	0\\
57.78	0\\
57.79	0\\
57.8	0\\
57.81	0\\
57.82	0\\
57.83	0\\
57.84	0\\
57.85	0\\
57.86	0\\
57.87	0\\
57.88	0\\
57.89	0\\
57.9	0\\
57.91	0\\
57.92	0\\
57.93	0\\
57.94	0\\
57.95	0\\
57.96	0\\
57.97	0\\
57.98	0\\
57.99	0\\
58	0\\
58.01	0\\
58.02	0\\
58.03	0\\
58.04	0\\
58.05	0\\
58.06	0\\
58.07	0\\
58.08	0\\
58.09	0\\
58.1	0\\
58.11	0\\
58.12	0\\
58.13	0\\
58.14	0\\
58.15	0\\
58.16	0\\
58.17	0\\
58.18	0\\
58.19	0\\
58.2	0\\
58.21	0\\
58.22	0\\
58.23	0\\
58.24	0\\
58.25	0\\
58.26	0\\
58.27	0\\
58.28	0\\
58.29	0\\
58.3	0\\
58.31	0\\
58.32	0\\
58.33	0\\
58.34	0\\
58.35	0\\
58.36	0\\
58.37	0\\
58.38	0\\
58.39	0\\
58.4	0\\
58.41	0\\
58.42	0\\
58.43	0\\
58.44	0\\
58.45	0\\
58.46	0\\
58.47	0\\
58.48	0\\
58.49	0\\
58.5	0\\
58.51	0\\
58.52	0\\
58.53	0\\
58.54	0\\
58.55	0\\
58.56	0\\
58.57	0\\
58.58	0\\
58.59	0\\
58.6	0\\
58.61	0\\
58.62	0\\
58.63	0\\
58.64	0\\
58.65	0\\
58.66	0\\
58.67	0\\
58.68	0\\
58.69	0\\
58.7	0\\
58.71	0\\
58.72	0\\
58.73	0\\
58.74	0\\
58.75	0\\
58.76	0\\
58.77	0\\
58.78	0\\
58.79	0\\
58.8	0\\
58.81	0\\
58.82	0\\
58.83	0\\
58.84	0\\
58.85	0\\
58.86	0\\
58.87	0\\
58.88	0\\
58.89	0\\
58.9	0\\
58.91	0\\
58.92	0\\
58.93	0\\
58.94	0\\
58.95	0\\
58.96	0\\
58.97	0\\
58.98	0\\
58.99	0\\
59	0\\
59.01	0\\
59.02	0\\
59.03	0\\
59.04	0\\
59.05	0\\
59.06	0\\
59.07	0\\
59.08	0\\
59.09	0\\
59.1	0\\
59.11	0\\
59.12	0\\
59.13	0\\
59.14	0\\
59.15	0\\
59.16	0\\
59.17	0\\
59.18	0\\
59.19	0\\
59.2	0\\
59.21	0\\
59.22	0\\
59.23	0\\
59.24	0\\
59.25	0\\
59.26	0\\
59.27	0\\
59.28	0\\
59.29	0\\
59.3	0\\
59.31	0\\
59.32	0\\
59.33	0\\
59.34	0\\
59.35	0\\
59.36	0\\
59.37	0\\
59.38	0\\
59.39	0\\
59.4	0\\
59.41	0\\
59.42	0\\
59.43	0\\
59.44	0\\
59.45	0\\
59.46	0\\
59.47	0\\
59.48	0\\
59.49	0\\
59.5	0\\
59.51	0\\
59.52	0\\
59.53	0\\
59.54	0\\
59.55	0\\
59.56	0\\
59.57	0\\
59.58	0\\
59.59	0\\
59.6	0\\
59.61	0\\
59.62	0\\
59.63	0\\
59.64	0\\
59.65	0\\
59.66	0\\
59.67	0\\
59.68	0\\
59.69	0\\
59.7	0\\
59.71	0\\
59.72	0\\
59.73	0\\
59.74	0\\
59.75	0\\
59.76	0\\
59.77	0\\
59.78	0\\
59.79	0\\
59.8	0\\
59.81	0\\
59.82	0\\
59.83	0\\
59.84	0\\
59.85	0\\
59.86	0\\
59.87	0\\
59.88	0\\
59.89	0\\
59.9	0\\
59.91	0\\
59.92	0\\
59.93	0\\
59.94	0\\
59.95	0\\
59.96	0\\
59.97	0\\
59.98	0\\
59.99	0\\
60	0\\
60.01	0\\
60.02	0\\
60.03	0\\
60.04	0\\
60.05	0\\
60.06	0\\
60.07	0\\
60.08	0\\
60.09	0\\
60.1	0\\
60.11	0\\
60.12	0\\
60.13	0\\
60.14	0\\
60.15	0\\
60.16	0\\
60.17	0\\
60.18	0\\
60.19	0\\
60.2	0\\
60.21	0\\
60.22	0\\
60.23	0\\
60.24	0\\
60.25	0\\
60.26	0\\
60.27	0\\
60.28	0\\
60.29	0\\
60.3	0\\
60.31	0\\
60.32	0\\
60.33	0\\
60.34	0\\
60.35	0\\
60.36	0\\
60.37	0\\
60.38	0\\
60.39	0\\
60.4	0\\
60.41	0\\
60.42	0\\
60.43	0\\
60.44	0\\
60.45	0\\
60.46	0\\
60.47	0\\
60.48	0\\
60.49	0\\
60.5	0\\
60.51	0\\
60.52	0\\
60.53	0\\
60.54	0\\
60.55	0\\
60.56	0\\
60.57	0\\
60.58	0\\
60.59	0\\
60.6	0\\
60.61	0\\
60.62	0\\
60.63	0\\
60.64	0\\
60.65	0\\
60.66	0\\
60.67	0\\
60.68	0\\
60.69	0\\
60.7	0\\
60.71	0\\
60.72	0\\
60.73	0\\
60.74	0\\
60.75	0\\
60.76	0\\
60.77	0\\
60.78	0\\
60.79	0\\
60.8	0\\
60.81	0\\
60.82	0\\
60.83	0\\
60.84	0\\
60.85	0\\
60.86	0\\
60.87	0\\
60.88	0\\
60.89	0\\
60.9	0\\
60.91	0\\
60.92	0\\
60.93	0\\
60.94	0\\
60.95	0\\
60.96	0\\
60.97	0\\
60.98	0\\
60.99	0\\
61	0\\
61.01	0\\
61.02	0\\
61.03	0\\
61.04	0\\
61.05	0\\
61.06	0\\
61.07	0\\
61.08	0\\
61.09	0\\
61.1	0\\
61.11	0\\
61.12	0\\
61.13	0\\
61.14	0\\
61.15	0\\
61.16	0\\
61.17	0\\
61.18	0\\
61.19	0\\
61.2	0\\
61.21	0\\
61.22	0\\
61.23	0\\
61.24	0\\
61.25	0\\
61.26	0\\
61.27	0\\
61.28	0\\
61.29	0\\
61.3	0\\
61.31	0\\
61.32	0\\
61.33	0\\
61.34	0\\
61.35	0\\
61.36	0\\
61.37	0\\
61.38	0\\
61.39	0\\
61.4	0\\
61.41	0\\
61.42	0\\
61.43	0\\
61.44	0\\
61.45	0\\
61.46	0\\
61.47	0\\
61.48	0\\
61.49	0\\
61.5	0\\
61.51	0\\
61.52	0\\
61.53	0\\
61.54	0\\
61.55	0\\
61.56	0\\
61.57	0\\
61.58	0\\
61.59	0\\
61.6	0\\
61.61	0\\
61.62	0\\
61.63	0\\
61.64	0\\
61.65	0\\
61.66	0\\
61.67	0\\
61.68	0\\
61.69	0\\
61.7	0\\
61.71	0\\
61.72	0\\
61.73	0\\
61.74	0\\
61.75	0\\
61.76	0\\
61.77	0\\
61.78	0\\
61.79	0\\
61.8	0\\
61.81	0\\
61.82	0\\
61.83	0\\
61.84	0\\
61.85	0\\
61.86	0\\
61.87	0\\
61.88	0\\
61.89	0\\
61.9	0\\
61.91	0\\
61.92	0\\
61.93	0\\
61.94	0\\
61.95	0\\
61.96	0\\
61.97	0\\
61.98	0\\
61.99	0\\
62	0\\
62.01	0\\
62.02	0\\
62.03	0\\
62.04	0\\
62.05	0\\
62.06	0\\
62.07	0\\
62.08	0\\
62.09	0\\
62.1	0\\
62.11	0\\
62.12	0\\
62.13	0\\
62.14	0\\
62.15	0\\
62.16	0\\
62.17	0\\
62.18	0\\
62.19	0\\
62.2	0\\
62.21	0\\
62.22	0\\
62.23	0\\
62.24	0\\
62.25	0\\
62.26	0\\
62.27	0\\
62.28	0\\
62.29	0\\
62.3	0\\
62.31	0\\
62.32	0\\
62.33	0\\
62.34	0\\
62.35	0\\
62.36	0\\
62.37	0\\
62.38	0\\
62.39	0\\
62.4	0\\
62.41	0\\
62.42	0\\
62.43	0\\
62.44	0\\
62.45	0\\
62.46	0\\
62.47	0\\
62.48	0\\
62.49	0\\
62.5	0\\
62.51	0\\
62.52	0\\
62.53	0\\
62.54	0\\
62.55	0\\
62.56	0\\
62.57	0\\
62.58	0\\
62.59	0\\
62.6	0\\
62.61	0\\
62.62	0\\
62.63	0\\
62.64	0\\
62.65	0\\
62.66	0\\
62.67	0\\
62.68	0\\
62.69	0\\
62.7	0\\
62.71	0\\
62.72	0\\
62.73	0\\
62.74	0\\
62.75	0\\
62.76	0\\
62.77	0\\
62.78	0\\
62.79	0\\
62.8	0\\
62.81	0\\
62.82	0\\
62.83	0\\
62.84	0\\
62.85	0\\
62.86	0\\
62.87	0\\
62.88	0\\
62.89	0\\
62.9	0\\
62.91	0\\
62.92	0\\
62.93	0\\
62.94	0\\
62.95	0\\
62.96	0\\
62.97	0\\
62.98	0\\
62.99	0\\
63	0\\
63.01	0\\
63.02	0\\
63.03	0\\
63.04	0\\
63.05	0\\
63.06	0\\
63.07	0\\
63.08	0\\
63.09	0\\
63.1	0\\
63.11	0\\
63.12	0\\
63.13	0\\
63.14	0\\
63.15	0\\
63.16	0\\
63.17	0\\
63.18	0\\
63.19	0\\
63.2	0\\
63.21	0\\
63.22	0\\
63.23	0\\
63.24	0\\
63.25	0\\
63.26	0\\
63.27	0\\
63.28	0\\
63.29	0\\
63.3	0\\
63.31	0\\
63.32	0\\
63.33	0\\
63.34	0\\
63.35	0\\
63.36	0\\
63.37	0\\
63.38	0\\
63.39	0\\
63.4	0\\
63.41	0\\
63.42	0\\
63.43	0\\
63.44	0\\
63.45	0\\
63.46	0\\
63.47	0\\
63.48	0\\
63.49	0\\
63.5	0\\
63.51	0\\
63.52	0\\
63.53	0\\
63.54	0\\
63.55	0\\
63.56	0\\
63.57	0\\
63.58	0\\
63.59	0\\
63.6	0\\
63.61	0\\
63.62	0\\
63.63	0\\
63.64	0\\
63.65	0\\
63.66	0\\
63.67	0\\
63.68	0\\
63.69	0\\
63.7	0\\
63.71	0\\
63.72	0\\
63.73	0\\
63.74	0\\
63.75	0\\
63.76	0\\
63.77	0\\
63.78	0\\
63.79	0\\
63.8	0\\
63.81	0\\
63.82	0\\
63.83	0\\
63.84	0\\
63.85	0\\
63.86	0\\
63.87	0\\
63.88	0\\
63.89	0\\
63.9	0\\
63.91	0\\
63.92	0\\
63.93	0\\
63.94	0\\
63.95	0\\
63.96	0\\
63.97	0\\
63.98	0\\
63.99	0\\
64	0\\
64.01	0\\
64.02	0\\
64.03	0\\
64.04	0\\
64.05	0\\
64.06	0\\
64.07	0\\
64.08	0\\
64.09	0\\
64.1	0\\
64.11	0\\
64.12	0\\
64.13	0\\
64.14	0\\
64.15	0\\
64.16	0\\
64.17	0\\
64.18	0\\
64.19	0\\
64.2	0\\
64.21	0\\
64.22	0\\
64.23	0\\
64.24	0\\
64.25	0\\
64.26	0\\
64.27	0\\
64.28	0\\
64.29	0\\
64.3	0\\
64.31	0\\
64.32	0\\
64.33	0\\
64.34	0\\
64.35	0\\
64.36	0\\
64.37	0\\
64.38	0\\
64.39	0\\
64.4	0\\
64.41	0\\
64.42	0\\
64.43	0\\
64.44	0\\
64.45	0\\
64.46	0\\
64.47	0\\
64.48	0\\
64.49	0\\
64.5	0\\
64.51	0\\
64.52	0\\
64.53	0\\
64.54	0\\
64.55	0\\
64.56	0\\
64.57	0\\
64.58	0\\
64.59	0\\
64.6	0\\
64.61	0\\
64.62	0\\
64.63	0\\
64.64	0\\
64.65	0\\
64.66	0\\
64.67	0\\
64.68	0\\
64.69	0\\
64.7	0\\
64.71	0\\
64.72	0\\
64.73	0\\
64.74	0\\
64.75	0\\
64.76	0\\
64.77	0\\
64.78	0\\
64.79	0\\
64.8	0\\
64.81	0\\
64.82	0\\
64.83	0\\
64.84	0\\
64.85	0\\
64.86	0\\
64.87	0\\
64.88	0\\
64.89	0\\
64.9	0\\
64.91	0\\
64.92	0\\
64.93	0\\
64.94	0\\
64.95	0\\
64.96	0\\
64.97	0\\
64.98	0\\
64.99	0\\
65	0\\
65.01	0\\
65.02	0\\
65.03	0\\
65.04	0\\
65.05	0\\
65.06	0\\
65.07	0\\
65.08	0\\
65.09	0\\
65.1	0\\
65.11	0\\
65.12	0\\
65.13	0\\
65.14	0\\
65.15	0\\
65.16	0\\
65.17	0\\
65.18	0\\
65.19	0\\
65.2	0\\
65.21	0\\
65.22	0\\
65.23	0\\
65.24	0\\
65.25	0\\
65.26	0\\
65.27	0\\
65.28	0\\
65.29	0\\
65.3	0\\
65.31	0\\
65.32	0\\
65.33	0\\
65.34	0\\
65.35	0\\
65.36	0\\
65.37	0\\
65.38	0\\
65.39	0\\
65.4	0\\
65.41	0\\
65.42	0\\
65.43	0\\
65.44	0\\
65.45	0\\
65.46	0\\
65.47	0\\
65.48	0\\
65.49	0\\
65.5	0\\
65.51	0\\
65.52	0\\
65.53	0\\
65.54	0\\
65.55	0\\
65.56	0\\
65.57	0\\
65.58	0\\
65.59	0\\
65.6	0\\
65.61	0\\
65.62	0\\
65.63	0\\
65.64	0\\
65.65	0\\
65.66	0\\
65.67	0\\
65.68	0\\
65.69	0\\
65.7	0\\
65.71	0\\
65.72	0\\
65.73	0\\
65.74	0\\
65.75	0\\
65.76	0\\
65.77	0\\
65.78	0\\
65.79	0\\
65.8	0\\
65.81	0\\
65.82	0\\
65.83	0\\
65.84	0\\
65.85	0\\
65.86	0\\
65.87	0\\
65.88	0\\
65.89	0\\
65.9	0\\
65.91	0\\
65.92	0\\
65.93	0\\
65.94	0\\
65.95	0\\
65.96	0\\
65.97	0\\
65.98	0\\
65.99	0\\
66	0\\
66.01	0\\
66.02	0\\
66.03	0\\
66.04	0\\
66.05	0\\
66.06	0\\
66.07	0\\
66.08	0\\
66.09	0\\
66.1	0\\
66.11	0\\
66.12	0\\
66.13	0\\
66.14	0\\
66.15	0\\
66.16	0\\
66.17	0\\
66.18	0\\
66.19	0\\
66.2	0\\
66.21	0\\
66.22	0\\
66.23	0\\
66.24	0\\
66.25	0\\
66.26	0\\
66.27	0\\
66.28	0\\
66.29	0\\
66.3	0\\
66.31	0\\
66.32	0\\
66.33	0\\
66.34	0\\
66.35	0\\
66.36	0\\
66.37	0\\
66.38	0\\
66.39	0\\
66.4	0\\
66.41	0\\
66.42	0\\
66.43	0\\
66.44	0\\
66.45	0\\
66.46	0\\
66.47	0\\
66.48	0\\
66.49	0\\
66.5	0\\
66.51	0\\
66.52	0\\
66.53	0\\
66.54	0\\
66.55	0\\
66.56	0\\
66.57	0\\
66.58	0\\
66.59	0\\
66.6	0\\
66.61	0\\
66.62	0\\
66.63	0\\
66.64	0\\
66.65	0\\
66.66	0\\
66.67	0\\
66.68	0\\
66.69	0\\
66.7	0\\
66.71	0\\
66.72	0\\
66.73	0\\
66.74	0\\
66.75	0\\
66.76	0\\
66.77	0\\
66.78	0\\
66.79	0\\
66.8	0\\
66.81	0\\
66.82	0\\
66.83	0\\
66.84	0\\
66.85	0\\
66.86	0\\
66.87	0\\
66.88	0\\
66.89	0\\
66.9	0\\
66.91	0\\
66.92	0\\
66.93	0\\
66.94	0\\
66.95	0\\
66.96	0\\
66.97	0\\
66.98	0\\
66.99	0\\
67	0\\
67.01	0\\
67.02	0\\
67.03	0\\
67.04	0\\
67.05	0\\
67.06	0\\
67.07	0\\
67.08	0\\
67.09	0\\
67.1	0\\
67.11	0\\
67.12	0\\
67.13	0\\
67.14	0\\
67.15	0\\
67.16	0\\
67.17	0\\
67.18	0\\
67.19	0\\
67.2	0\\
67.21	0\\
67.22	0\\
67.23	0\\
67.24	0\\
67.25	0\\
67.26	0\\
67.27	0\\
67.28	0\\
67.29	0\\
67.3	0\\
67.31	0\\
67.32	0\\
67.33	0\\
67.34	0\\
67.35	0\\
67.36	0\\
67.37	0\\
67.38	0\\
67.39	0\\
67.4	0\\
67.41	0\\
67.42	0\\
67.43	0\\
67.44	0\\
67.45	0\\
67.46	0\\
67.47	0\\
67.48	0\\
67.49	0\\
67.5	0\\
67.51	0\\
67.52	0\\
67.53	0\\
67.54	0\\
67.55	0\\
67.56	0\\
67.57	0\\
67.58	0\\
67.59	0\\
67.6	0\\
67.61	0\\
67.62	0\\
67.63	0\\
67.64	0\\
67.65	0\\
67.66	0\\
67.67	0\\
67.68	0\\
67.69	0\\
67.7	0\\
67.71	0\\
67.72	0\\
67.73	0\\
67.74	0\\
67.75	0\\
67.76	0\\
67.77	0\\
67.78	0\\
67.79	0\\
67.8	0\\
67.81	0\\
67.82	0\\
67.83	0\\
67.84	0\\
67.85	0\\
67.86	0\\
67.87	0\\
67.88	0\\
67.89	0\\
67.9	0\\
67.91	0\\
67.92	0\\
67.93	0\\
67.94	0\\
67.95	0\\
67.96	0\\
67.97	0\\
67.98	0\\
67.99	0\\
68	0\\
68.01	0\\
68.02	0\\
68.03	0\\
68.04	0\\
68.05	0\\
68.06	0\\
68.07	0\\
68.08	0\\
68.09	0\\
68.1	0\\
68.11	0\\
68.12	0\\
68.13	0\\
68.14	0\\
68.15	0\\
68.16	0\\
68.17	0\\
68.18	0\\
68.19	0\\
68.2	0\\
68.21	0\\
68.22	0\\
68.23	0\\
68.24	0\\
68.25	0\\
68.26	0\\
68.27	0\\
68.28	0\\
68.29	0\\
68.3	0\\
68.31	0\\
68.32	0\\
68.33	0\\
68.34	0\\
68.35	0\\
68.36	0\\
68.37	0\\
68.38	0\\
68.39	0\\
68.4	0\\
68.41	0\\
68.42	0\\
68.43	0\\
68.44	0\\
68.45	0\\
68.46	0\\
68.47	0\\
68.48	0\\
68.49	0\\
68.5	0\\
68.51	0\\
68.52	0\\
68.53	0\\
68.54	0\\
68.55	0\\
68.56	0\\
68.57	0\\
68.58	0\\
68.59	0\\
68.6	0\\
68.61	0\\
68.62	0\\
68.63	0\\
68.64	0\\
68.65	0\\
68.66	0\\
68.67	0\\
68.68	0\\
68.69	0\\
68.7	0\\
68.71	0\\
68.72	0\\
68.73	0\\
68.74	0\\
68.75	0\\
68.76	0\\
68.77	0\\
68.78	0\\
68.79	0\\
68.8	0\\
68.81	0\\
68.82	0\\
68.83	0\\
68.84	0\\
68.85	0\\
68.86	0\\
68.87	0\\
68.88	0\\
68.89	0\\
68.9	0\\
68.91	0\\
68.92	0\\
68.93	0\\
68.94	0\\
68.95	0\\
68.96	0\\
68.97	0\\
68.98	0\\
68.99	0\\
69	0\\
69.01	0\\
69.02	0\\
69.03	0\\
69.04	0\\
69.05	0\\
69.06	0\\
69.07	0\\
69.08	0\\
69.09	0\\
69.1	0\\
69.11	0\\
69.12	0\\
69.13	0\\
69.14	0\\
69.15	0\\
69.16	0\\
69.17	0\\
69.18	0\\
69.19	0\\
69.2	0\\
69.21	0\\
69.22	0\\
69.23	0\\
69.24	0\\
69.25	0\\
69.26	0\\
69.27	0\\
69.28	0\\
69.29	0\\
69.3	0\\
69.31	0\\
69.32	0\\
69.33	0\\
69.34	0\\
69.35	0\\
69.36	0\\
69.37	0\\
69.38	0\\
69.39	0\\
69.4	0\\
69.41	0\\
69.42	0\\
69.43	0\\
69.44	0\\
69.45	0\\
69.46	0\\
69.47	0\\
69.48	0\\
69.49	0\\
69.5	0\\
69.51	0\\
69.52	0\\
69.53	0\\
69.54	0\\
69.55	0\\
69.56	0\\
69.57	0\\
69.58	0\\
69.59	0\\
69.6	0\\
69.61	0\\
69.62	0\\
69.63	0\\
69.64	0\\
69.65	0\\
69.66	0\\
69.67	0\\
69.68	0\\
69.69	0\\
69.7	0\\
69.71	0\\
69.72	0\\
69.73	0\\
69.74	0\\
69.75	0\\
69.76	0\\
69.77	0\\
69.78	0\\
69.79	0\\
69.8	0\\
69.81	0\\
69.82	0\\
69.83	0\\
69.84	0\\
69.85	0\\
69.86	0\\
69.87	0\\
69.88	0\\
69.89	0\\
69.9	0\\
69.91	0\\
69.92	0\\
69.93	0\\
69.94	0\\
69.95	0\\
69.96	0\\
69.97	0\\
69.98	0\\
69.99	0\\
70	0\\
70.01	0\\
70.02	0\\
70.03	0\\
70.04	0\\
70.05	0\\
70.06	0\\
70.07	0\\
70.08	0\\
70.09	0\\
70.1	0\\
70.11	0\\
70.12	0\\
70.13	0\\
70.14	0\\
70.15	0\\
70.16	0\\
70.17	0\\
70.18	0\\
70.19	0\\
70.2	0\\
70.21	0\\
70.22	0\\
70.23	0\\
70.24	0\\
70.25	0\\
70.26	0\\
70.27	0\\
70.28	0\\
70.29	0\\
70.3	0\\
70.31	0\\
70.32	0\\
70.33	0\\
70.34	0\\
70.35	0\\
70.36	0\\
70.37	0\\
70.38	0\\
70.39	0\\
70.4	0\\
70.41	0\\
70.42	0\\
70.43	0\\
70.44	0\\
70.45	0\\
70.46	0\\
70.47	0\\
70.48	0\\
70.49	0\\
70.5	0\\
70.51	0\\
70.52	0\\
70.53	0\\
70.54	0\\
70.55	0\\
70.56	0\\
70.57	0\\
70.58	0\\
70.59	0\\
70.6	0\\
70.61	0\\
70.62	0\\
70.63	0\\
70.64	0\\
70.65	0\\
70.66	0\\
70.67	0\\
70.68	0\\
70.69	0\\
70.7	0\\
70.71	0\\
70.72	0\\
70.73	0\\
70.74	0\\
70.75	0\\
70.76	0\\
70.77	0\\
70.78	0\\
70.79	0\\
70.8	0\\
70.81	0\\
70.82	0\\
70.83	0\\
70.84	0\\
70.85	0\\
70.86	0\\
70.87	0\\
70.88	0\\
70.89	0\\
70.9	0\\
70.91	0\\
70.92	0\\
70.93	0\\
70.94	0\\
70.95	0\\
70.96	0\\
70.97	0\\
70.98	0\\
70.99	0\\
71	0\\
71.01	0\\
71.02	0\\
71.03	0\\
71.04	0\\
71.05	0\\
71.06	0\\
71.07	0\\
71.08	0\\
71.09	0\\
71.1	0\\
71.11	0\\
71.12	0\\
71.13	0\\
71.14	0\\
71.15	0\\
71.16	0\\
71.17	0\\
71.18	0\\
71.19	0\\
71.2	0\\
71.21	0\\
71.22	0\\
71.23	0\\
71.24	0\\
71.25	0\\
71.26	0\\
71.27	0\\
71.28	0\\
71.29	0\\
71.3	0\\
71.31	0\\
71.32	0\\
71.33	0\\
71.34	0\\
71.35	0\\
71.36	0\\
71.37	0\\
71.38	0\\
71.39	0\\
71.4	0\\
71.41	0\\
71.42	0\\
71.43	0\\
71.44	0\\
71.45	0\\
71.46	0\\
71.47	0\\
71.48	0\\
71.49	0\\
71.5	0\\
71.51	0\\
71.52	0\\
71.53	0\\
71.54	0\\
71.55	0\\
71.56	0\\
71.57	0\\
71.58	0\\
71.59	0\\
71.6	0\\
71.61	0\\
71.62	0\\
71.63	0\\
71.64	0\\
71.65	0\\
71.66	0\\
71.67	0\\
71.68	0\\
71.69	0\\
71.7	0\\
71.71	0\\
71.72	0\\
71.73	0\\
71.74	0\\
71.75	0\\
71.76	0\\
71.77	0\\
71.78	0\\
71.79	0\\
71.8	0\\
71.81	0\\
71.82	0\\
71.83	0\\
71.84	0\\
71.85	0\\
71.86	0\\
71.87	0\\
71.88	0\\
71.89	0\\
71.9	0\\
71.91	0\\
71.92	0\\
71.93	0\\
71.94	0\\
71.95	0\\
71.96	0\\
71.97	0\\
71.98	0\\
71.99	0\\
72	0\\
72.01	0\\
72.02	0\\
72.03	0\\
72.04	0\\
72.05	0\\
72.06	0\\
72.07	0\\
72.08	0\\
72.09	0\\
72.1	0\\
72.11	0\\
72.12	0\\
72.13	0\\
72.14	0\\
72.15	0\\
72.16	0\\
72.17	0\\
72.18	0\\
72.19	0\\
72.2	0\\
72.21	0\\
72.22	0\\
72.23	0\\
72.24	0\\
72.25	0\\
72.26	0\\
72.27	0\\
72.28	0\\
72.29	0\\
72.3	0\\
72.31	0\\
72.32	0\\
72.33	0\\
72.34	0\\
72.35	0\\
72.36	0\\
72.37	0\\
72.38	0\\
72.39	0\\
72.4	0\\
72.41	0\\
72.42	0\\
72.43	0\\
72.44	0\\
72.45	0\\
72.46	0\\
72.47	0\\
72.48	0\\
72.49	0\\
72.5	0\\
72.51	0\\
72.52	0\\
72.53	0\\
72.54	0\\
72.55	0\\
72.56	0\\
72.57	0\\
72.58	0\\
72.59	0\\
72.6	0\\
72.61	0\\
72.62	0\\
72.63	0\\
72.64	0\\
72.65	0\\
72.66	0\\
72.67	0\\
72.68	0\\
72.69	0\\
72.7	0\\
72.71	0\\
72.72	0\\
72.73	0\\
72.74	0\\
72.75	0\\
72.76	0\\
72.77	0\\
72.78	0\\
72.79	0\\
72.8	0\\
72.81	0\\
72.82	0\\
72.83	0\\
72.84	0\\
72.85	0\\
72.86	0\\
72.87	0\\
72.88	0\\
72.89	0\\
72.9	0\\
72.91	0\\
72.92	0\\
72.93	0\\
72.94	0\\
72.95	0\\
72.96	0\\
72.97	0\\
72.98	0\\
72.99	0\\
73	0\\
73.01	0\\
73.02	0\\
73.03	0\\
73.04	0\\
73.05	0\\
73.06	0\\
73.07	0\\
73.08	0\\
73.09	0\\
73.1	0\\
73.11	0\\
73.12	0\\
73.13	0\\
73.14	0\\
73.15	0\\
73.16	0\\
73.17	0\\
73.18	0\\
73.19	0\\
73.2	0\\
73.21	0\\
73.22	0\\
73.23	0\\
73.24	0\\
73.25	0\\
73.26	0\\
73.27	0\\
73.28	0\\
73.29	0\\
73.3	0\\
73.31	0\\
73.32	0\\
73.33	0\\
73.34	0\\
73.35	0\\
73.36	0\\
73.37	0\\
73.38	0\\
73.39	0\\
73.4	0\\
73.41	0\\
73.42	0\\
73.43	0\\
73.44	0\\
73.45	0\\
73.46	0\\
73.47	0\\
73.48	0\\
73.49	0\\
73.5	0\\
73.51	0\\
73.52	0\\
73.53	0\\
73.54	0\\
73.55	0\\
73.56	0\\
73.57	0\\
73.58	0\\
73.59	0\\
73.6	0\\
73.61	0\\
73.62	0\\
73.63	0\\
73.64	0\\
73.65	0\\
73.66	0\\
73.67	0\\
73.68	0\\
73.69	0\\
73.7	0\\
73.71	0\\
73.72	0\\
73.73	0\\
73.74	0\\
73.75	0\\
73.76	0\\
73.77	0\\
73.78	0\\
73.79	0\\
73.8	0\\
73.81	0\\
73.82	0\\
73.83	0\\
73.84	0\\
73.85	0\\
73.86	0\\
73.87	0\\
73.88	0\\
73.89	0\\
73.9	0\\
73.91	0\\
73.92	0\\
73.93	0\\
73.94	0\\
73.95	0\\
73.96	0\\
73.97	0\\
73.98	0\\
73.99	0\\
74	0\\
74.01	0\\
74.02	0\\
74.03	0\\
74.04	0\\
74.05	0\\
74.06	0\\
74.07	0\\
74.08	0\\
74.09	0\\
74.1	0\\
74.11	0\\
74.12	0\\
74.13	0\\
74.14	0\\
74.15	0\\
74.16	0\\
74.17	0\\
74.18	0\\
74.19	0\\
74.2	0\\
74.21	0\\
74.22	0\\
74.23	0\\
74.24	0\\
74.25	0\\
74.26	0\\
74.27	0\\
74.28	0\\
74.29	0\\
74.3	0\\
74.31	0\\
74.32	0\\
74.33	0\\
74.34	0\\
74.35	0\\
74.36	0\\
74.37	0\\
74.38	0\\
74.39	0\\
74.4	0\\
74.41	0\\
74.42	0\\
74.43	0\\
74.44	0\\
74.45	0\\
74.46	0\\
74.47	0\\
74.48	0\\
74.49	0\\
74.5	0\\
74.51	0\\
74.52	0\\
74.53	0\\
74.54	0\\
74.55	0\\
74.56	0\\
74.57	0\\
74.58	0\\
74.59	0\\
74.6	0\\
74.61	0\\
74.62	0\\
74.63	0\\
74.64	0\\
74.65	0\\
74.66	0\\
74.67	0\\
74.68	0\\
74.69	0\\
74.7	0\\
74.71	0\\
74.72	0\\
74.73	0\\
74.74	0\\
74.75	0\\
74.76	0\\
74.77	0\\
74.78	0\\
74.79	0\\
74.8	0\\
74.81	0\\
74.82	0\\
74.83	0\\
74.84	0\\
74.85	0\\
74.86	0\\
74.87	0\\
74.88	0\\
74.89	0\\
74.9	0\\
74.91	0\\
74.92	0\\
74.93	0\\
74.94	0\\
74.95	0\\
74.96	0\\
74.97	0\\
74.98	0\\
74.99	0\\
75	0\\
75.01	0\\
75.02	0\\
75.03	0\\
75.04	0\\
75.05	0\\
75.06	0\\
75.07	0\\
75.08	0\\
75.09	0\\
75.1	0\\
75.11	0\\
75.12	0\\
75.13	0\\
75.14	0\\
75.15	0\\
75.16	0\\
75.17	0\\
75.18	0\\
75.19	0\\
75.2	0\\
75.21	0\\
75.22	0\\
75.23	0\\
75.24	0\\
75.25	0\\
75.26	0\\
75.27	0\\
75.28	0\\
75.29	0\\
75.3	0\\
75.31	0\\
75.32	0\\
75.33	0\\
75.34	0\\
75.35	0\\
75.36	0\\
75.37	0\\
75.38	0\\
75.39	0\\
75.4	0\\
75.41	0\\
75.42	0\\
75.43	0\\
75.44	0\\
75.45	0\\
75.46	0\\
75.47	0\\
75.48	0\\
75.49	0\\
75.5	0\\
75.51	0\\
75.52	0\\
75.53	0\\
75.54	0\\
75.55	0\\
75.56	0\\
75.57	0\\
75.58	0\\
75.59	0\\
75.6	0\\
75.61	0\\
75.62	0\\
75.63	0\\
75.64	0\\
75.65	0\\
75.66	0\\
75.67	0\\
75.68	0\\
75.69	0\\
75.7	0\\
75.71	0\\
75.72	0\\
75.73	0\\
75.74	0\\
75.75	0\\
75.76	0\\
75.77	0\\
75.78	0\\
75.79	0\\
75.8	0\\
75.81	0\\
75.82	0\\
75.83	0\\
75.84	0\\
75.85	0\\
75.86	0\\
75.87	0\\
75.88	0\\
75.89	0\\
75.9	0\\
75.91	0\\
75.92	0\\
75.93	0\\
75.94	0\\
75.95	0\\
75.96	0\\
75.97	0\\
75.98	0\\
75.99	0\\
76	0\\
76.01	0\\
76.02	0\\
76.03	0\\
76.04	0\\
76.05	0\\
76.06	0\\
76.07	0\\
76.08	0\\
76.09	0\\
76.1	0\\
76.11	0\\
76.12	0\\
76.13	0\\
76.14	0\\
76.15	0\\
76.16	0\\
76.17	0\\
76.18	0\\
76.19	0\\
76.2	0\\
76.21	0\\
76.22	0\\
76.23	0\\
76.24	0\\
76.25	0\\
76.26	0\\
76.27	0\\
76.28	0\\
76.29	0\\
76.3	0\\
76.31	0\\
76.32	0\\
76.33	0\\
76.34	0\\
76.35	0\\
76.36	0\\
76.37	0\\
76.38	0\\
76.39	0\\
76.4	0\\
76.41	0\\
76.42	0\\
76.43	0\\
76.44	0\\
76.45	0\\
76.46	0\\
76.47	0\\
76.48	0\\
76.49	0\\
76.5	0\\
76.51	0\\
76.52	0\\
76.53	0\\
76.54	0\\
76.55	0\\
76.56	0\\
76.57	0\\
76.58	0\\
76.59	0\\
76.6	0\\
76.61	0\\
76.62	0\\
76.63	0\\
76.64	0\\
76.65	0\\
76.66	0\\
76.67	0\\
76.68	0\\
76.69	0\\
76.7	0\\
76.71	0\\
76.72	0\\
76.73	0\\
76.74	0\\
76.75	0\\
76.76	0\\
76.77	0\\
76.78	0\\
76.79	0\\
76.8	0\\
76.81	0\\
76.82	0\\
76.83	0\\
76.84	0\\
76.85	0\\
76.86	0\\
76.87	0\\
76.88	0\\
76.89	0\\
76.9	0\\
76.91	0\\
76.92	0\\
76.93	0\\
76.94	0\\
76.95	0\\
76.96	0\\
76.97	0\\
76.98	0\\
76.99	0\\
77	0\\
77.01	0\\
77.02	0\\
77.03	0\\
77.04	0\\
77.05	0\\
77.06	0\\
77.07	0\\
77.08	0\\
77.09	0\\
77.1	0\\
77.11	0\\
77.12	0\\
77.13	0\\
77.14	0\\
77.15	0\\
77.16	0\\
77.17	0\\
77.18	0\\
77.19	0\\
77.2	0\\
77.21	0\\
77.22	0\\
77.23	0\\
77.24	0\\
77.25	0\\
77.26	0\\
77.27	0\\
77.28	0\\
77.29	0\\
77.3	0\\
77.31	0\\
77.32	0\\
77.33	0\\
77.34	0\\
77.35	0\\
77.36	0\\
77.37	0\\
77.38	0\\
77.39	0\\
77.4	0\\
77.41	0\\
77.42	0\\
77.43	0\\
77.44	0\\
77.45	0\\
77.46	0\\
77.47	0\\
77.48	0\\
77.49	0\\
77.5	0\\
77.51	0\\
77.52	0\\
77.53	0\\
77.54	0\\
77.55	0\\
77.56	0\\
77.57	0\\
77.58	0\\
77.59	0\\
77.6	0\\
77.61	0\\
77.62	0\\
77.63	0\\
77.64	0\\
77.65	0\\
77.66	0\\
77.67	0\\
77.68	0\\
77.69	0\\
77.7	0\\
77.71	0\\
77.72	0\\
77.73	0\\
77.74	0\\
77.75	0\\
77.76	0\\
77.77	0\\
77.78	0\\
77.79	0\\
77.8	0\\
77.81	0\\
77.82	0\\
77.83	0\\
77.84	0\\
77.85	0\\
77.86	0\\
77.87	0\\
77.88	0\\
77.89	0\\
77.9	0\\
77.91	0\\
77.92	0\\
77.93	0\\
77.94	0\\
77.95	0\\
77.96	0\\
77.97	0\\
77.98	0\\
77.99	0\\
78	0\\
78.01	0\\
78.02	0\\
78.03	0\\
78.04	0\\
78.05	0\\
78.06	0\\
78.07	0\\
78.08	0\\
78.09	0\\
78.1	0\\
78.11	0\\
78.12	0\\
78.13	0\\
78.14	0\\
78.15	0\\
78.16	0\\
78.17	0\\
78.18	0\\
78.19	0\\
78.2	0\\
78.21	0\\
78.22	0\\
78.23	0\\
78.24	0\\
78.25	0\\
78.26	0\\
78.27	0\\
78.28	0\\
78.29	0\\
78.3	0\\
78.31	0\\
78.32	0\\
78.33	0\\
78.34	0\\
78.35	0\\
78.36	0\\
78.37	0\\
78.38	0\\
78.39	0\\
78.4	0\\
78.41	0\\
78.42	0\\
78.43	0\\
78.44	0\\
78.45	0\\
78.46	0\\
78.47	0\\
78.48	0\\
78.49	0\\
78.5	0\\
78.51	0\\
78.52	0\\
78.53	0\\
78.54	0\\
78.55	0\\
78.56	0\\
78.57	0\\
78.58	0\\
78.59	0\\
78.6	0\\
78.61	0\\
78.62	0\\
78.63	0\\
78.64	0\\
78.65	0\\
78.66	0\\
78.67	0\\
78.68	0\\
78.69	0\\
78.7	0\\
78.71	0\\
78.72	0\\
78.73	0\\
78.74	0\\
78.75	0\\
78.76	0\\
78.77	0\\
78.78	0\\
78.79	0\\
78.8	0\\
78.81	0\\
78.82	0\\
78.83	0\\
78.84	0\\
78.85	0\\
78.86	0\\
78.87	0\\
78.88	0\\
78.89	0\\
78.9	0\\
78.91	0\\
78.92	0\\
78.93	0\\
78.94	0\\
78.95	0\\
78.96	0\\
78.97	0\\
78.98	0\\
78.99	0\\
79	0\\
79.01	0\\
79.02	0\\
79.03	0\\
79.04	0\\
79.05	0\\
79.06	0\\
79.07	0\\
79.08	0\\
79.09	0\\
79.1	0\\
79.11	0\\
79.12	0\\
79.13	0\\
79.14	0\\
79.15	0\\
79.16	0\\
79.17	0\\
79.18	0\\
79.19	0\\
79.2	0\\
79.21	0\\
79.22	0\\
79.23	0\\
79.24	0\\
79.25	0\\
79.26	0\\
79.27	0\\
79.28	0\\
79.29	0\\
79.3	0\\
79.31	0\\
79.32	0\\
79.33	0\\
79.34	0\\
79.35	0\\
79.36	0\\
79.37	0\\
79.38	0\\
79.39	0\\
79.4	0\\
79.41	0\\
79.42	0\\
79.43	0\\
79.44	0\\
79.45	0\\
79.46	0\\
79.47	0\\
79.48	0\\
79.49	0\\
79.5	0\\
79.51	0\\
79.52	0\\
79.53	0\\
79.54	0\\
79.55	0\\
79.56	0\\
79.57	0\\
79.58	0\\
79.59	0\\
79.6	0\\
79.61	0\\
79.62	0\\
79.63	0\\
79.64	0\\
79.65	0\\
79.66	0\\
79.67	0\\
79.68	0\\
79.69	0\\
79.7	0\\
79.71	0\\
79.72	0\\
79.73	0\\
79.74	0\\
79.75	0\\
79.76	0\\
79.77	0\\
79.78	0\\
79.79	0\\
79.8	0\\
79.81	0\\
79.82	0\\
79.83	0\\
79.84	0\\
79.85	0\\
79.86	0\\
79.87	0\\
79.88	0\\
79.89	0\\
79.9	0\\
79.91	0\\
79.92	0\\
79.93	0\\
79.94	0\\
79.95	0\\
79.96	0\\
79.97	0\\
79.98	0\\
79.99	0\\
80	0\\
80.01	0\\
};
\addplot [color=black,solid]
  table[row sep=crcr]{%
80.01	0\\
80.02	0\\
80.03	0\\
80.04	0\\
80.05	0\\
80.06	0\\
80.07	0\\
80.08	0\\
80.09	0\\
80.1	0\\
80.11	0\\
80.12	0\\
80.13	0\\
80.14	0\\
80.15	0\\
80.16	0\\
80.17	0\\
80.18	0\\
80.19	0\\
80.2	0\\
80.21	0\\
80.22	0\\
80.23	0\\
80.24	0\\
80.25	0\\
80.26	0\\
80.27	0\\
80.28	0\\
80.29	0\\
80.3	0\\
80.31	0\\
80.32	0\\
80.33	0\\
80.34	0\\
80.35	0\\
80.36	0\\
80.37	0\\
80.38	0\\
80.39	0\\
80.4	0\\
80.41	0\\
80.42	0\\
80.43	0\\
80.44	0\\
80.45	0\\
80.46	0\\
80.47	0\\
80.48	0\\
80.49	0\\
80.5	0\\
80.51	0\\
80.52	0\\
80.53	0\\
80.54	0\\
80.55	0\\
80.56	0\\
80.57	0\\
80.58	0\\
80.59	0\\
80.6	0\\
80.61	0\\
80.62	0\\
80.63	0\\
80.64	0\\
80.65	0\\
80.66	0\\
80.67	0\\
80.68	0\\
80.69	0\\
80.7	0\\
80.71	0\\
80.72	0\\
80.73	0\\
80.74	0\\
80.75	0\\
80.76	0\\
80.77	0\\
80.78	0\\
80.79	0\\
80.8	0\\
80.81	0\\
80.82	0\\
80.83	0\\
80.84	0\\
80.85	0\\
80.86	0\\
80.87	0\\
80.88	0\\
80.89	0\\
80.9	0\\
80.91	0\\
80.92	0\\
80.93	0\\
80.94	0\\
80.95	0\\
80.96	0\\
80.97	0\\
80.98	0\\
80.99	0\\
81	0\\
81.01	0\\
81.02	0\\
81.03	0\\
81.04	0\\
81.05	0\\
81.06	0\\
81.07	0\\
81.08	0\\
81.09	0\\
81.1	0\\
81.11	0\\
81.12	0\\
81.13	0\\
81.14	0\\
81.15	0\\
81.16	0\\
81.17	0\\
81.18	0\\
81.19	0\\
81.2	0\\
81.21	0\\
81.22	0\\
81.23	0\\
81.24	0\\
81.25	0\\
81.26	0\\
81.27	0\\
81.28	0\\
81.29	0\\
81.3	0\\
81.31	0\\
81.32	0\\
81.33	0\\
81.34	0\\
81.35	0\\
81.36	0\\
81.37	0\\
81.38	0\\
81.39	0\\
81.4	0\\
81.41	0\\
81.42	0\\
81.43	0\\
81.44	0\\
81.45	0\\
81.46	0\\
81.47	0\\
81.48	0\\
81.49	0\\
81.5	0\\
81.51	0\\
81.52	0\\
81.53	0\\
81.54	0\\
81.55	0\\
81.56	0\\
81.57	0\\
81.58	0\\
81.59	0\\
81.6	0\\
81.61	0\\
81.62	0\\
81.63	0\\
81.64	0\\
81.65	0\\
81.66	0\\
81.67	0\\
81.68	0\\
81.69	0\\
81.7	0\\
81.71	0\\
81.72	0\\
81.73	0\\
81.74	0\\
81.75	0\\
81.76	0\\
81.77	0\\
81.78	0\\
81.79	0\\
81.8	0\\
81.81	0\\
81.82	0\\
81.83	0\\
81.84	0\\
81.85	0\\
81.86	0\\
81.87	0\\
81.88	0\\
81.89	0\\
81.9	0\\
81.91	0\\
81.92	0\\
81.93	0\\
81.94	0\\
81.95	0\\
81.96	0\\
81.97	0\\
81.98	0\\
81.99	0\\
82	0\\
82.01	0\\
82.02	0\\
82.03	0\\
82.04	0\\
82.05	0\\
82.06	0\\
82.07	0\\
82.08	0\\
82.09	0\\
82.1	0\\
82.11	0\\
82.12	0\\
82.13	0\\
82.14	0\\
82.15	0\\
82.16	0\\
82.17	0\\
82.18	0\\
82.19	0\\
82.2	0\\
82.21	0\\
82.22	0\\
82.23	0\\
82.24	0\\
82.25	0\\
82.26	0\\
82.27	0\\
82.28	0\\
82.29	0\\
82.3	0\\
82.31	0\\
82.32	0\\
82.33	0\\
82.34	0\\
82.35	0\\
82.36	0\\
82.37	0\\
82.38	0\\
82.39	0\\
82.4	0\\
82.41	0\\
82.42	0\\
82.43	0\\
82.44	0\\
82.45	0\\
82.46	0\\
82.47	0\\
82.48	0\\
82.49	0\\
82.5	0\\
82.51	0\\
82.52	0\\
82.53	0\\
82.54	0\\
82.55	0\\
82.56	0\\
82.57	0\\
82.58	0\\
82.59	0\\
82.6	0\\
82.61	0\\
82.62	0\\
82.63	0\\
82.64	0\\
82.65	0\\
82.66	0\\
82.67	0\\
82.68	0\\
82.69	0\\
82.7	0\\
82.71	0\\
82.72	0\\
82.73	0\\
82.74	0\\
82.75	0\\
82.76	0\\
82.77	0\\
82.78	0\\
82.79	0\\
82.8	0\\
82.81	0\\
82.82	0\\
82.83	0\\
82.84	0\\
82.85	0\\
82.86	0\\
82.87	0\\
82.88	0\\
82.89	0\\
82.9	0\\
82.91	0\\
82.92	0\\
82.93	0\\
82.94	0\\
82.95	0\\
82.96	0\\
82.97	0\\
82.98	0\\
82.99	0\\
83	0\\
83.01	0\\
83.02	0\\
83.03	0\\
83.04	0\\
83.05	0\\
83.06	0\\
83.07	0\\
83.08	0\\
83.09	0\\
83.1	0\\
83.11	0\\
83.12	0\\
83.13	0\\
83.14	0\\
83.15	0\\
83.16	0\\
83.17	0\\
83.18	0\\
83.19	0\\
83.2	0\\
83.21	0\\
83.22	0\\
83.23	0\\
83.24	0\\
83.25	0\\
83.26	0\\
83.27	0\\
83.28	0\\
83.29	0\\
83.3	0\\
83.31	0\\
83.32	0\\
83.33	0\\
83.34	0\\
83.35	0\\
83.36	0\\
83.37	0\\
83.38	0\\
83.39	0\\
83.4	0\\
83.41	0\\
83.42	0\\
83.43	0\\
83.44	0\\
83.45	0\\
83.46	0\\
83.47	0\\
83.48	0\\
83.49	0\\
83.5	0\\
83.51	0\\
83.52	0\\
83.53	0\\
83.54	0\\
83.55	0\\
83.56	0\\
83.57	0\\
83.58	0\\
83.59	0\\
83.6	0\\
83.61	0\\
83.62	0\\
83.63	0\\
83.64	0\\
83.65	0\\
83.66	0\\
83.67	0\\
83.68	0\\
83.69	0\\
83.7	0\\
83.71	0\\
83.72	0\\
83.73	0\\
83.74	0\\
83.75	0\\
83.76	0\\
83.77	0\\
83.78	0\\
83.79	0\\
83.8	0\\
83.81	0\\
83.82	0\\
83.83	0\\
83.84	0\\
83.85	0\\
83.86	0\\
83.87	0\\
83.88	0\\
83.89	0\\
83.9	0\\
83.91	0\\
83.92	0\\
83.93	0\\
83.94	0\\
83.95	0\\
83.96	0\\
83.97	0\\
83.98	0\\
83.99	0\\
84	0\\
84.01	0\\
84.02	0\\
84.03	0\\
84.04	0\\
84.05	0\\
84.06	0\\
84.07	0\\
84.08	0\\
84.09	0\\
84.1	0\\
84.11	0\\
84.12	0\\
84.13	0\\
84.14	0\\
84.15	0\\
84.16	0\\
84.17	0\\
84.18	0\\
84.19	0\\
84.2	0\\
84.21	0\\
84.22	0\\
84.23	0\\
84.24	0\\
84.25	0\\
84.26	0\\
84.27	0\\
84.28	0\\
84.29	0\\
84.3	0\\
84.31	0\\
84.32	0\\
84.33	0\\
84.34	0\\
84.35	0\\
84.36	0\\
84.37	0\\
84.38	0\\
84.39	0\\
84.4	0\\
84.41	0\\
84.42	0\\
84.43	0\\
84.44	0\\
84.45	0\\
84.46	0\\
84.47	0\\
84.48	0\\
84.49	0\\
84.5	0\\
84.51	0\\
84.52	0\\
84.53	0\\
84.54	0\\
84.55	0\\
84.56	0\\
84.57	0\\
84.58	0\\
84.59	0\\
84.6	0\\
84.61	0\\
84.62	0\\
84.63	0\\
84.64	0\\
84.65	0\\
84.66	0\\
84.67	0\\
84.68	0\\
84.69	0\\
84.7	0\\
84.71	0\\
84.72	0\\
84.73	0\\
84.74	0\\
84.75	0\\
84.76	0\\
84.77	0\\
84.78	0\\
84.79	0\\
84.8	0\\
84.81	0\\
84.82	0\\
84.83	0\\
84.84	0\\
84.85	0\\
84.86	0\\
84.87	0\\
84.88	0\\
84.89	0\\
84.9	0\\
84.91	0\\
84.92	0\\
84.93	0\\
84.94	0\\
84.95	0\\
84.96	0\\
84.97	0\\
84.98	0\\
84.99	0\\
85	0\\
85.01	0\\
85.02	0\\
85.03	0\\
85.04	0\\
85.05	0\\
85.06	0\\
85.07	0\\
85.08	0\\
85.09	0\\
85.1	0\\
85.11	0\\
85.12	0\\
85.13	0\\
85.14	0\\
85.15	0\\
85.16	0\\
85.17	0\\
85.18	0\\
85.19	0\\
85.2	0\\
85.21	0\\
85.22	0\\
85.23	0\\
85.24	0\\
85.25	0\\
85.26	0\\
85.27	0\\
85.28	0\\
85.29	0\\
85.3	0\\
85.31	0\\
85.32	0\\
85.33	0\\
85.34	0\\
85.35	0\\
85.36	0\\
85.37	0\\
85.38	0\\
85.39	0\\
85.4	0\\
85.41	0\\
85.42	0\\
85.43	0\\
85.44	0\\
85.45	0\\
85.46	0\\
85.47	0\\
85.48	0\\
85.49	0\\
85.5	0\\
85.51	0\\
85.52	0\\
85.53	0\\
85.54	0\\
85.55	0\\
85.56	0\\
85.57	0\\
85.58	0\\
85.59	0\\
85.6	0\\
85.61	0\\
85.62	0\\
85.63	0\\
85.64	0\\
85.65	0\\
85.66	0\\
85.67	0\\
85.68	0\\
85.69	0\\
85.7	0\\
85.71	0\\
85.72	0\\
85.73	0\\
85.74	0\\
85.75	0\\
85.76	0\\
85.77	0\\
85.78	0\\
85.79	0\\
85.8	0\\
85.81	0\\
85.82	0\\
85.83	0\\
85.84	0\\
85.85	0\\
85.86	0\\
85.87	0\\
85.88	0\\
85.89	0\\
85.9	0\\
85.91	0\\
85.92	0\\
85.93	0\\
85.94	0\\
85.95	0\\
85.96	0\\
85.97	0\\
85.98	0\\
85.99	0\\
86	0\\
86.01	0\\
86.02	0\\
86.03	0\\
86.04	0\\
86.05	0\\
86.06	0\\
86.07	0\\
86.08	0\\
86.09	0\\
86.1	0\\
86.11	0\\
86.12	0\\
86.13	0\\
86.14	0\\
86.15	0\\
86.16	0\\
86.17	0\\
86.18	0\\
86.19	0\\
86.2	0\\
86.21	0\\
86.22	0\\
86.23	0\\
86.24	0\\
86.25	0\\
86.26	0\\
86.27	0\\
86.28	0\\
86.29	0\\
86.3	0\\
86.31	0\\
86.32	0\\
86.33	0\\
86.34	0\\
86.35	0\\
86.36	0\\
86.37	0\\
86.38	0\\
86.39	0\\
86.4	0\\
86.41	0\\
86.42	0\\
86.43	0\\
86.44	0\\
86.45	0\\
86.46	0\\
86.47	0\\
86.48	0\\
86.49	0\\
86.5	0\\
86.51	0\\
86.52	0\\
86.53	0\\
86.54	0\\
86.55	0\\
86.56	0\\
86.57	0\\
86.58	0\\
86.59	0\\
86.6	0\\
86.61	0\\
86.62	0\\
86.63	0\\
86.64	0\\
86.65	0\\
86.66	0\\
86.67	0\\
86.68	0\\
86.69	0\\
86.7	0\\
86.71	0\\
86.72	0\\
86.73	0\\
86.74	0\\
86.75	0\\
86.76	0\\
86.77	0\\
86.78	0\\
86.79	0\\
86.8	0\\
86.81	0\\
86.82	0\\
86.83	0\\
86.84	0\\
86.85	0\\
86.86	0\\
86.87	0\\
86.88	0\\
86.89	0\\
86.9	0\\
86.91	0\\
86.92	0\\
86.93	0\\
86.94	0\\
86.95	0\\
86.96	0\\
86.97	0\\
86.98	0\\
86.99	0\\
87	0\\
87.01	0\\
87.02	0\\
87.03	0\\
87.04	0\\
87.05	0\\
87.06	0\\
87.07	0\\
87.08	0\\
87.09	0\\
87.1	0\\
87.11	0\\
87.12	0\\
87.13	0\\
87.14	0\\
87.15	0\\
87.16	0\\
87.17	0\\
87.18	0\\
87.19	0\\
87.2	0\\
87.21	0\\
87.22	0\\
87.23	0\\
87.24	0\\
87.25	0\\
87.26	0\\
87.27	0\\
87.28	0\\
87.29	0\\
87.3	0\\
87.31	0\\
87.32	0\\
87.33	0\\
87.34	0\\
87.35	0\\
87.36	0\\
87.37	0\\
87.38	0\\
87.39	0\\
87.4	0\\
87.41	0\\
87.42	0\\
87.43	0\\
87.44	0\\
87.45	0\\
87.46	0\\
87.47	0\\
87.48	0\\
87.49	0\\
87.5	0\\
87.51	0\\
87.52	0\\
87.53	0\\
87.54	0\\
87.55	0\\
87.56	0\\
87.57	0\\
87.58	0\\
87.59	0\\
87.6	0\\
87.61	0\\
87.62	0\\
87.63	0\\
87.64	0\\
87.65	0\\
87.66	0\\
87.67	0\\
87.68	0\\
87.69	0\\
87.7	0\\
87.71	0\\
87.72	0\\
87.73	0\\
87.74	0\\
87.75	0\\
87.76	0\\
87.77	0\\
87.78	0\\
87.79	0\\
87.8	0\\
87.81	0\\
87.82	0\\
87.83	0\\
87.84	0\\
87.85	0\\
87.86	0\\
87.87	0\\
87.88	0\\
87.89	0\\
87.9	0\\
87.91	0\\
87.92	0\\
87.93	0\\
87.94	0\\
87.95	0\\
87.96	0\\
87.97	0\\
87.98	0\\
87.99	0\\
88	0\\
88.01	0\\
88.02	0\\
88.03	0\\
88.04	0\\
88.05	0\\
88.06	0\\
88.07	0\\
88.08	0\\
88.09	0\\
88.1	0\\
88.11	0\\
88.12	0\\
88.13	0\\
88.14	0\\
88.15	0\\
88.16	0\\
88.17	0\\
88.18	0\\
88.19	0\\
88.2	0\\
88.21	0\\
88.22	0\\
88.23	0\\
88.24	0\\
88.25	0\\
88.26	0\\
88.27	0\\
88.28	0\\
88.29	0\\
88.3	0\\
88.31	0\\
88.32	0\\
88.33	0\\
88.34	0\\
88.35	0\\
88.36	0\\
88.37	0\\
88.38	0\\
88.39	0\\
88.4	0\\
88.41	0\\
88.42	0\\
88.43	0\\
88.44	0\\
88.45	0\\
88.46	0\\
88.47	0\\
88.48	0\\
88.49	0\\
88.5	0\\
88.51	0\\
88.52	0\\
88.53	0\\
88.54	0\\
88.55	0\\
88.56	0\\
88.57	0\\
88.58	0\\
88.59	0\\
88.6	0\\
88.61	0\\
88.62	0\\
88.63	0\\
88.64	0\\
88.65	0\\
88.66	0\\
88.67	0\\
88.68	0\\
88.69	0\\
88.7	0\\
88.71	0\\
88.72	0\\
88.73	0\\
88.74	0\\
88.75	0\\
88.76	0\\
88.77	0\\
88.78	0\\
88.79	0\\
88.8	0\\
88.81	0\\
88.82	0\\
88.83	0\\
88.84	0\\
88.85	0\\
88.86	0\\
88.87	0\\
88.88	0\\
88.89	0\\
88.9	0\\
88.91	0\\
88.92	0\\
88.93	0\\
88.94	0\\
88.95	0\\
88.96	0\\
88.97	0\\
88.98	0\\
88.99	0\\
89	0\\
89.01	0\\
89.02	0\\
89.03	0\\
89.04	0\\
89.05	0\\
89.06	0\\
89.07	0\\
89.08	0\\
89.09	0\\
89.1	0\\
89.11	0\\
89.12	0\\
89.13	0\\
89.14	0\\
89.15	0\\
89.16	0\\
89.17	0\\
89.18	0\\
89.19	0\\
89.2	0\\
89.21	0\\
89.22	0\\
89.23	0\\
89.24	0\\
89.25	0\\
89.26	0\\
89.27	0\\
89.28	0\\
89.29	0\\
89.3	0\\
89.31	0\\
89.32	0\\
89.33	0\\
89.34	0\\
89.35	0\\
89.36	0\\
89.37	0\\
89.38	0\\
89.39	0\\
89.4	0\\
89.41	0\\
89.42	0\\
89.43	0\\
89.44	0\\
89.45	0\\
89.46	0\\
89.47	0\\
89.48	0\\
89.49	0\\
89.5	0\\
89.51	0\\
89.52	0\\
89.53	0\\
89.54	0\\
89.55	0\\
89.56	0\\
89.57	0\\
89.58	0\\
89.59	0\\
89.6	0\\
89.61	0\\
89.62	0\\
89.63	0\\
89.64	0\\
89.65	0\\
89.66	0\\
89.67	0\\
89.68	0\\
89.69	0\\
89.7	0\\
89.71	0\\
89.72	0\\
89.73	0\\
89.74	0\\
89.75	0\\
89.76	0\\
89.77	0\\
89.78	0\\
89.79	0\\
89.8	0\\
89.81	0\\
89.82	0\\
89.83	0\\
89.84	0\\
89.85	0\\
89.86	0\\
89.87	0\\
89.88	0\\
89.89	0\\
89.9	0\\
89.91	0\\
89.92	0\\
89.93	0\\
89.94	0\\
89.95	0\\
89.96	0\\
89.97	0\\
89.98	0\\
89.99	0\\
90	0\\
90.01	0\\
90.02	0\\
90.03	0\\
90.04	0\\
90.05	0\\
90.06	0\\
90.07	0\\
90.08	0\\
90.09	0\\
90.1	0\\
90.11	0\\
90.12	0\\
90.13	0\\
90.14	0\\
90.15	0\\
90.16	0\\
90.17	0\\
90.18	0\\
90.19	0\\
90.2	0\\
90.21	0\\
90.22	0\\
90.23	0\\
90.24	0\\
90.25	0\\
90.26	0\\
90.27	0\\
90.28	0\\
90.29	0\\
90.3	0\\
90.31	0\\
90.32	0\\
90.33	0\\
90.34	0\\
90.35	0\\
90.36	0\\
90.37	0\\
90.38	0\\
90.39	0\\
90.4	0\\
90.41	0\\
90.42	0\\
90.43	0\\
90.44	0\\
90.45	0\\
90.46	0\\
90.47	0\\
90.48	0\\
90.49	0\\
90.5	0\\
90.51	0\\
90.52	0\\
90.53	0\\
90.54	0\\
90.55	0\\
90.56	0\\
90.57	0\\
90.58	0\\
90.59	0\\
90.6	0\\
90.61	0\\
90.62	0\\
90.63	0\\
90.64	0\\
90.65	0\\
90.66	0\\
90.67	0\\
90.68	0\\
90.69	0\\
90.7	0\\
90.71	0\\
90.72	0\\
90.73	0\\
90.74	0\\
90.75	0\\
90.76	0\\
90.77	0\\
90.78	0\\
90.79	0\\
90.8	0\\
90.81	0\\
90.82	0\\
90.83	0\\
90.84	0\\
90.85	0\\
90.86	0\\
90.87	0\\
90.88	0\\
90.89	0\\
90.9	0\\
90.91	0\\
90.92	0\\
90.93	0\\
90.94	0\\
90.95	0\\
90.96	0\\
90.97	0\\
90.98	0\\
90.99	0\\
91	0\\
91.01	0\\
91.02	0\\
91.03	0\\
91.04	0\\
91.05	0\\
91.06	0\\
91.07	0\\
91.08	0\\
91.09	0\\
91.1	0\\
91.11	0\\
91.12	0\\
91.13	0\\
91.14	0\\
91.15	0\\
91.16	0\\
91.17	0\\
91.18	0\\
91.19	0\\
91.2	0\\
91.21	0\\
91.22	0\\
91.23	0\\
91.24	0\\
91.25	0\\
91.26	0\\
91.27	0\\
91.28	0\\
91.29	0\\
91.3	0\\
91.31	0\\
91.32	0\\
91.33	0\\
91.34	0\\
91.35	0\\
91.36	0\\
91.37	0\\
91.38	0\\
91.39	0\\
91.4	0\\
91.41	0\\
91.42	0\\
91.43	0\\
91.44	0\\
91.45	0\\
91.46	0\\
91.47	0\\
91.48	0\\
91.49	0\\
91.5	0\\
91.51	0\\
91.52	0\\
91.53	0\\
91.54	0\\
91.55	0\\
91.56	0\\
91.57	0\\
91.58	0\\
91.59	0\\
91.6	0\\
91.61	0\\
91.62	0\\
91.63	0\\
91.64	0\\
91.65	0\\
91.66	0\\
91.67	0\\
91.68	0\\
91.69	0\\
91.7	0\\
91.71	0\\
91.72	0\\
91.73	0\\
91.74	0\\
91.75	0\\
91.76	0\\
91.77	0\\
91.78	0\\
91.79	0\\
91.8	0\\
91.81	0\\
91.82	0\\
91.83	0\\
91.84	0\\
91.85	0\\
91.86	0\\
91.87	0\\
91.88	0\\
91.89	0\\
91.9	0\\
91.91	0\\
91.92	0\\
91.93	0\\
91.94	0\\
91.95	0\\
91.96	0\\
91.97	0\\
91.98	0\\
91.99	0\\
92	0\\
92.01	0\\
92.02	0\\
92.03	0\\
92.04	0\\
92.05	0\\
92.06	0\\
92.07	0\\
92.08	0\\
92.09	0\\
92.1	0\\
92.11	0\\
92.12	0\\
92.13	0\\
92.14	0\\
92.15	0\\
92.16	0\\
92.17	0\\
92.18	0\\
92.19	0\\
92.2	0\\
92.21	0\\
92.22	0\\
92.23	0\\
92.24	0\\
92.25	0\\
92.26	0\\
92.27	0\\
92.28	0\\
92.29	0\\
92.3	0\\
92.31	0\\
92.32	0\\
92.33	0\\
92.34	0\\
92.35	0\\
92.36	0\\
92.37	0\\
92.38	0\\
92.39	0\\
92.4	0\\
92.41	0\\
92.42	0\\
92.43	0\\
92.44	0\\
92.45	0\\
92.46	0\\
92.47	0\\
92.48	0\\
92.49	0\\
92.5	0\\
92.51	0\\
92.52	0\\
92.53	0\\
92.54	0\\
92.55	0\\
92.56	0\\
92.57	0\\
92.58	0\\
92.59	0\\
92.6	0\\
92.61	0\\
92.62	0\\
92.63	0\\
92.64	0\\
92.65	0\\
92.66	0\\
92.67	0\\
92.68	0\\
92.69	0\\
92.7	0\\
92.71	0\\
92.72	0\\
92.73	0\\
92.74	0\\
92.75	0\\
92.76	0\\
92.77	0\\
92.78	0\\
92.79	0\\
92.8	0\\
92.81	0\\
92.82	0\\
92.83	0\\
92.84	0\\
92.85	0\\
92.86	0\\
92.87	0\\
92.88	0\\
92.89	0\\
92.9	0\\
92.91	0\\
92.92	0\\
92.93	0\\
92.94	0\\
92.95	0\\
92.96	0\\
92.97	0\\
92.98	0\\
92.99	0\\
93	0\\
93.01	0\\
93.02	0\\
93.03	0\\
93.04	0\\
93.05	0\\
93.06	0\\
93.07	0\\
93.08	0\\
93.09	0\\
93.1	0\\
93.11	0\\
93.12	0\\
93.13	0\\
93.14	0\\
93.15	0\\
93.16	0\\
93.17	0\\
93.18	0\\
93.19	0\\
93.2	0\\
93.21	0\\
93.22	0\\
93.23	0\\
93.24	0\\
93.25	0\\
93.26	0\\
93.27	0\\
93.28	0\\
93.29	0\\
93.3	0\\
93.31	0\\
93.32	0\\
93.33	0\\
93.34	0\\
93.35	0\\
93.36	0\\
93.37	0\\
93.38	0\\
93.39	0\\
93.4	0\\
93.41	0\\
93.42	0\\
93.43	0\\
93.44	0\\
93.45	0\\
93.46	0\\
93.47	0\\
93.48	0\\
93.49	0\\
93.5	0\\
93.51	0\\
93.52	0\\
93.53	0\\
93.54	0\\
93.55	0\\
93.56	0\\
93.57	0\\
93.58	0\\
93.59	0\\
93.6	0\\
93.61	0\\
93.62	0\\
93.63	0\\
93.64	0\\
93.65	0\\
93.66	0\\
93.67	0\\
93.68	0\\
93.69	0\\
93.7	0\\
93.71	0\\
93.72	0\\
93.73	0\\
93.74	0\\
93.75	0\\
93.76	0\\
93.77	0\\
93.78	0\\
93.79	0\\
93.8	0\\
93.81	0\\
93.82	0\\
93.83	0\\
93.84	0\\
93.85	0\\
93.86	0\\
93.87	0\\
93.88	0\\
93.89	0\\
93.9	0\\
93.91	0\\
93.92	0\\
93.93	0\\
93.94	0\\
93.95	0\\
93.96	0\\
93.97	0\\
93.98	0\\
93.99	0\\
94	0\\
94.01	0\\
94.02	0\\
94.03	0\\
94.04	0\\
94.05	0\\
94.06	0\\
94.07	0\\
94.08	0\\
94.09	0\\
94.1	0\\
94.11	0\\
94.12	0\\
94.13	0\\
94.14	0\\
94.15	0\\
94.16	0\\
94.17	0\\
94.18	0\\
94.19	0\\
94.2	0\\
94.21	0\\
94.22	0\\
94.23	0\\
94.24	0\\
94.25	0\\
94.26	0\\
94.27	0\\
94.28	0\\
94.29	0\\
94.3	0\\
94.31	0\\
94.32	0\\
94.33	0\\
94.34	0\\
94.35	0\\
94.36	0\\
94.37	0\\
94.38	0\\
94.39	0\\
94.4	0\\
94.41	0\\
94.42	0\\
94.43	0\\
94.44	0\\
94.45	0\\
94.46	0\\
94.47	0\\
94.48	0\\
94.49	0\\
94.5	0\\
94.51	0\\
94.52	0\\
94.53	0\\
94.54	0\\
94.55	0\\
94.56	0\\
94.57	0\\
94.58	0\\
94.59	0\\
94.6	0\\
94.61	0\\
94.62	0\\
94.63	0\\
94.64	0\\
94.65	0\\
94.66	0\\
94.67	0\\
94.68	0\\
94.69	0\\
94.7	0\\
94.71	0\\
94.72	0\\
94.73	0\\
94.74	0\\
94.75	0\\
94.76	0\\
94.77	0\\
94.78	0\\
94.79	0\\
94.8	0\\
94.81	0\\
94.82	0\\
94.83	0\\
94.84	0\\
94.85	0\\
94.86	0\\
94.87	0\\
94.88	0\\
94.89	0\\
94.9	0\\
94.91	0\\
94.92	0\\
94.93	0\\
94.94	0\\
94.95	0\\
94.96	0\\
94.97	0\\
94.98	0\\
94.99	0\\
95	0\\
95.01	0\\
95.02	0\\
95.03	0\\
95.04	0\\
95.05	0\\
95.06	0\\
95.07	0\\
95.08	0\\
95.09	0\\
95.1	0\\
95.11	0\\
95.12	0\\
95.13	0\\
95.14	0\\
95.15	0\\
95.16	0\\
95.17	0\\
95.18	0\\
95.19	0\\
95.2	0\\
95.21	0\\
95.22	0\\
95.23	0\\
95.24	0\\
95.25	0\\
95.26	0\\
95.27	0\\
95.28	0\\
95.29	0\\
95.3	0\\
95.31	0\\
95.32	0\\
95.33	0\\
95.34	0\\
95.35	0\\
95.36	0\\
95.37	0\\
95.38	0\\
95.39	0\\
95.4	0\\
95.41	0\\
95.42	0\\
95.43	0\\
95.44	0\\
95.45	0\\
95.46	0\\
95.47	0\\
95.48	0\\
95.49	0\\
95.5	0\\
95.51	0\\
95.52	0\\
95.53	0\\
95.54	0\\
95.55	0\\
95.56	0\\
95.57	0\\
95.58	0\\
95.59	0\\
95.6	0\\
95.61	0\\
95.62	0\\
95.63	0\\
95.64	0\\
95.65	0\\
95.66	0\\
95.67	0\\
95.68	0\\
95.69	0\\
95.7	0\\
95.71	0\\
95.72	0\\
95.73	0\\
95.74	0\\
95.75	0\\
95.76	0\\
95.77	0\\
95.78	0\\
95.79	0\\
95.8	0\\
95.81	0\\
95.82	0\\
95.83	0\\
95.84	0\\
95.85	0\\
95.86	0\\
95.87	0\\
95.88	0\\
95.89	0\\
95.9	0\\
95.91	0\\
95.92	0\\
95.93	0\\
95.94	0\\
95.95	0\\
95.96	0\\
95.97	0\\
95.98	0\\
95.99	0\\
96	0\\
96.01	0\\
96.02	0\\
96.03	0\\
96.04	0\\
96.05	0\\
96.06	0\\
96.07	0\\
96.08	0\\
96.09	0\\
96.1	0\\
96.11	0\\
96.12	0\\
96.13	0\\
96.14	0\\
96.15	0\\
96.16	0\\
96.17	0\\
96.18	0\\
96.19	0\\
96.2	0\\
96.21	0\\
96.22	0\\
96.23	0\\
96.24	0\\
96.25	0\\
96.26	0\\
96.27	0\\
96.28	0\\
96.29	0\\
96.3	0\\
96.31	0\\
96.32	0\\
96.33	0\\
96.34	0\\
96.35	0\\
96.36	0\\
96.37	0\\
96.38	0\\
96.39	0\\
96.4	0\\
96.41	0\\
96.42	0\\
96.43	0\\
96.44	0\\
96.45	0\\
96.46	0\\
96.47	0\\
96.48	0\\
96.49	0\\
96.5	0\\
96.51	0\\
96.52	0\\
96.53	0\\
96.54	0\\
96.55	0\\
96.56	0\\
96.57	0\\
96.58	0\\
96.59	0\\
96.6	0\\
96.61	0\\
96.62	0\\
96.63	0\\
96.64	0\\
96.65	0\\
96.66	0\\
96.67	0\\
96.68	0\\
96.69	0\\
96.7	0\\
96.71	0\\
96.72	0\\
96.73	0\\
96.74	0\\
96.75	0\\
96.76	0\\
96.77	0\\
96.78	0\\
96.79	0\\
96.8	0\\
96.81	0\\
96.82	0\\
96.83	0\\
96.84	0\\
96.85	0\\
96.86	0\\
96.87	0\\
96.88	0\\
96.89	0\\
96.9	0\\
96.91	0\\
96.92	0\\
96.93	0\\
96.94	0\\
96.95	0\\
96.96	0\\
96.97	0\\
96.98	0\\
96.99	0\\
97	0\\
97.01	0\\
97.02	0\\
97.03	0\\
97.04	0\\
97.05	0\\
97.06	0\\
97.07	0\\
97.08	0\\
97.09	0\\
97.1	0\\
97.11	0\\
97.12	0\\
97.13	0\\
97.14	0\\
97.15	0\\
97.16	0\\
97.17	0\\
97.18	0\\
97.19	0\\
97.2	0\\
97.21	0\\
97.22	0\\
97.23	0\\
97.24	0\\
97.25	0\\
97.26	0\\
97.27	0\\
97.28	0\\
97.29	0\\
97.3	0\\
97.31	0\\
97.32	0\\
97.33	0\\
97.34	0\\
97.35	0\\
97.36	0\\
97.37	0\\
97.38	0\\
97.39	0\\
97.4	0\\
97.41	0\\
97.42	0\\
97.43	0\\
97.44	0\\
97.45	0\\
97.46	0\\
97.47	0\\
97.48	0\\
97.49	0\\
97.5	0\\
97.51	0\\
97.52	0\\
97.53	0\\
97.54	0\\
97.55	0\\
97.56	0\\
97.57	0\\
97.58	0\\
97.59	0\\
97.6	0\\
97.61	0\\
97.62	0\\
97.63	0\\
97.64	0\\
97.65	0\\
97.66	0\\
97.67	0\\
97.68	0\\
97.69	0\\
97.7	0\\
97.71	0\\
97.72	0\\
97.73	0\\
97.74	0\\
97.75	0\\
97.76	0\\
97.77	0\\
97.78	0\\
97.79	0\\
97.8	0\\
97.81	0\\
97.82	0\\
97.83	0\\
97.84	0\\
97.85	0\\
97.86	0\\
97.87	0\\
97.88	0\\
97.89	0\\
97.9	0\\
97.91	0\\
97.92	0\\
97.93	0\\
97.94	0\\
97.95	0\\
97.96	0\\
97.97	0\\
97.98	0\\
97.99	0\\
98	0\\
98.01	0\\
98.02	0\\
98.03	0\\
98.04	0\\
98.05	0\\
98.06	0\\
98.07	0\\
98.08	0\\
98.09	0\\
98.1	0\\
98.11	0\\
98.12	0\\
98.13	0\\
98.14	0\\
98.15	0\\
98.16	0\\
98.17	0\\
98.18	0\\
98.19	0\\
98.2	0\\
98.21	0\\
98.22	0\\
98.23	0\\
98.24	0\\
98.25	0\\
98.26	0\\
98.27	0\\
98.28	0\\
98.29	0\\
98.3	0\\
98.31	0\\
98.32	0\\
98.33	0\\
98.34	0\\
98.35	0\\
98.36	0\\
98.37	0\\
98.38	0\\
98.39	0\\
98.4	0\\
98.41	0\\
98.42	0\\
98.43	0\\
98.44	0\\
98.45	0\\
98.46	0\\
98.47	0\\
98.48	0\\
98.49	0\\
98.5	0\\
98.51	0\\
98.52	0\\
98.53	0\\
98.54	0\\
98.55	0\\
98.56	0\\
98.57	0\\
98.58	0\\
98.59	0\\
98.6	0\\
98.61	0\\
98.62	0\\
98.63	0\\
98.64	0\\
98.65	0\\
98.66	0\\
98.67	0\\
98.68	0\\
98.69	0\\
98.7	0\\
98.71	0\\
98.72	0\\
98.73	0\\
98.74	0\\
98.75	0\\
98.76	0\\
98.77	0\\
98.78	0\\
98.79	0\\
98.8	0\\
98.81	0\\
98.82	0\\
98.83	0\\
98.84	0\\
98.85	0\\
98.86	0\\
98.87	0\\
98.88	0\\
98.89	0\\
98.9	0\\
98.91	0\\
98.92	0\\
98.93	0\\
98.94	0\\
98.95	0\\
98.96	0\\
98.97	0\\
98.98	0\\
98.99	0\\
99	0\\
99.01	0\\
99.02	0\\
99.03	0\\
99.04	0\\
99.05	0\\
99.06	0\\
99.07	0\\
99.08	0\\
99.09	0\\
99.1	0\\
99.11	0\\
99.12	0\\
99.13	0\\
99.14	0\\
99.15	0\\
99.16	0\\
99.17	0\\
99.18	0\\
99.19	0\\
99.2	0\\
99.21	0\\
99.22	0\\
99.23	0\\
99.24	0\\
99.25	0\\
99.26	0\\
99.27	0\\
99.28	0\\
99.29	0\\
99.3	0\\
99.31	0\\
99.32	0\\
99.33	0\\
99.34	0\\
99.35	0\\
99.36	0\\
99.37	0\\
99.38	0\\
99.39	0\\
99.4	0\\
99.41	0\\
99.42	0.000142507484711799\\
99.43	0.00028603094158923\\
99.44	0.000430279959184088\\
99.45	0.000575263732640401\\
99.46	0.000720991715020906\\
99.47	0.000867473608317394\\
99.48	0.00101470001362642\\
99.49	0.00116265714149537\\
99.5	0.00131135440874501\\
99.51	0.00146080148760354\\
99.52	0.00161100831282696\\
99.53	0.00176198508903214\\
99.54	0.00191374229825145\\
99.55	0.00206629070771789\\
99.56	0.00221964137789047\\
99.57	0.00237380567073002\\
99.58	0.00252879525823614\\
99.59	0.00268462213125676\\
99.6	0.00284129860858223\\
99.61	0.00299883734633685\\
99.62	0.00315725134768115\\
99.63	0.00331655397283938\\
99.64	0.00347675894946723\\
99.65	0.00363788038337595\\
99.66	0.00379993276962971\\
99.67	0.00396293100403452\\
99.68	0.0041268903950377\\
99.69	0.00429182667605687\\
99.7	0.00445775601826144\\
99.71	0.00462469504382956\\
99.72	0.00479266083977007\\
99.73	0.00496167097234289\\
99.74	0.00513174350189687\\
99.75	0.00530289699829066\\
99.76	0.00547515055692814\\
99.77	0.00564852381544173\\
99.78	0.00582303697105938\\
99.79	0.00599871079869321\\
99.8	0.00617556666979046\\
99.81	0.00635362657199012\\
99.82	0.00653291312963158\\
99.83	0.00671344962516493\\
99.84	0.00689526002151581\\
99.85	0.00707836898546157\\
99.86	0.00726280191207965\\
99.87	0.00744858495033297\\
99.88	0.00763574502986251\\
99.89	0.0078243098890616\\
99.9	0.00801430810451265\\
99.91	0.0082057689535157\\
99.92	0.00839872238447791\\
99.93	0.00859319927334343\\
99.94	0.00878923145914393\\
99.95	0.00898685178139854\\
99.96	0.0091860941194868\\
99.97	0.0093869934341281\\
99.98	0.00958958581111154\\
99.99	0.0097939085074319\\
100	0.01\\
};
\addlegendentry{$q=0$};

\addplot [color=blue,solid,forget plot]
  table[row sep=crcr]{%
0.01	0.00922742732202166\\
0.02	0.00922742719080925\\
0.03	0.00922742705922687\\
0.04	0.00922742692727674\\
0.05	0.00922742679496116\\
0.06	0.00922742666228257\\
0.07	0.00922742652924349\\
0.08	0.00922742639584658\\
0.09	0.00922742626209461\\
0.1	0.00922742612799046\\
0.11	0.00922742599353714\\
0.12	0.00922742585873776\\
0.13	0.00922742572359556\\
0.14	0.0092274255881139\\
0.15	0.00922742545229624\\
0.16	0.00922742531614618\\
0.17	0.00922742517966744\\
0.18	0.00922742504286384\\
0.19	0.00922742490573932\\
0.2	0.00922742476829796\\
0.21	0.00922742463054393\\
0.22	0.00922742449248154\\
0.23	0.0092274243541152\\
0.24	0.00922742421544945\\
0.25	0.00922742407648893\\
0.26	0.00922742393723838\\
0.27	0.00922742379770269\\
0.28	0.00922742365788682\\
0.29	0.00922742351779585\\
0.3	0.00922742337743497\\
0.31	0.00922742323680945\\
0.32	0.00922742309592468\\
0.33	0.00922742295478614\\
0.34	0.00922742281339938\\
0.35	0.00922742267177007\\
0.36	0.00922742252990393\\
0.37	0.00922742238780678\\
0.38	0.0092274222454845\\
0.39	0.00922742210294303\\
0.4	0.0092274219601884\\
0.41	0.00922742181722666\\
0.42	0.00922742167406391\\
0.43	0.00922742153070633\\
0.44	0.00922742138716008\\
0.45	0.00922742124343138\\
0.46	0.00922742109952644\\
0.47	0.00922742095545149\\
0.48	0.00922742081121276\\
0.49	0.00922742066681644\\
0.5	0.00922742052226872\\
0.51	0.00922742037757572\\
0.52	0.00922742023274354\\
0.53	0.00922742008777819\\
0.54	0.00922741994268561\\
0.55	0.00922741979747165\\
0.56	0.00922741965214201\\
0.57	0.00922741950670232\\
0.58	0.00922741936115801\\
0.59	0.00922741921551437\\
0.6	0.00922741906977651\\
0.61	0.0092274189239493\\
0.62	0.00922741877803741\\
0.63	0.00922741863204525\\
0.64	0.00922741848597694\\
0.65	0.00922741833983631\\
0.66	0.00922741819362685\\
0.67	0.0092274180473517\\
0.68	0.0092274179010136\\
0.69	0.00922741775461488\\
0.7	0.0092274176081574\\
0.71	0.00922741746164255\\
0.72	0.00922741731507119\\
0.73	0.0092274171684436\\
0.74	0.00922741702175977\\
0.75	0.00922741687501967\\
0.76	0.00922741672822328\\
0.77	0.00922741658137057\\
0.78	0.00922741643446153\\
0.79	0.00922741628749612\\
0.8	0.00922741614047432\\
0.81	0.00922741599339612\\
0.82	0.00922741584626149\\
0.83	0.00922741569907039\\
0.84	0.00922741555182282\\
0.85	0.00922741540451874\\
0.86	0.00922741525715814\\
0.87	0.00922741510974098\\
0.88	0.00922741496226725\\
0.89	0.00922741481473693\\
0.9	0.00922741466714998\\
0.91	0.00922741451950638\\
0.92	0.00922741437180612\\
0.93	0.00922741422404916\\
0.94	0.00922741407623548\\
0.95	0.00922741392836507\\
0.96	0.00922741378043789\\
0.97	0.00922741363245392\\
0.98	0.00922741348441315\\
0.99	0.00922741333631553\\
1	0.00922741318816106\\
1.01	0.0092274130399497\\
1.02	0.00922741289168144\\
1.03	0.00922741274335625\\
1.04	0.0092274125949741\\
1.05	0.00922741244653497\\
1.06	0.00922741229803885\\
1.07	0.00922741214948569\\
1.08	0.00922741200087549\\
1.09	0.00922741185220821\\
1.1	0.00922741170348384\\
1.11	0.00922741155470235\\
1.12	0.0092274114058637\\
1.13	0.00922741125696789\\
1.14	0.00922741110801489\\
1.15	0.00922741095900467\\
1.16	0.00922741080993721\\
1.17	0.00922741066081248\\
1.18	0.00922741051163047\\
1.19	0.00922741036239114\\
1.2	0.00922741021309447\\
1.21	0.00922741006374044\\
1.22	0.00922740991432903\\
1.23	0.00922740976486021\\
1.24	0.00922740961533396\\
1.25	0.00922740946575024\\
1.26	0.00922740931610905\\
1.27	0.00922740916641035\\
1.28	0.00922740901665412\\
1.29	0.00922740886684034\\
1.3	0.00922740871696898\\
1.31	0.00922740856704002\\
1.32	0.00922740841705343\\
1.33	0.00922740826700919\\
1.34	0.00922740811690728\\
1.35	0.00922740796674766\\
1.36	0.00922740781653033\\
1.37	0.00922740766625525\\
1.38	0.00922740751592239\\
1.39	0.00922740736553175\\
1.4	0.00922740721508328\\
1.41	0.00922740706457696\\
1.42	0.00922740691401278\\
1.43	0.00922740676339071\\
1.44	0.00922740661271072\\
1.45	0.00922740646197279\\
1.46	0.00922740631117689\\
1.47	0.009227406160323\\
1.48	0.0092274060094111\\
1.49	0.00922740585844116\\
1.5	0.00922740570741316\\
1.51	0.00922740555632707\\
1.52	0.00922740540518287\\
1.53	0.00922740525398053\\
1.54	0.00922740510272003\\
1.55	0.00922740495140135\\
1.56	0.00922740480002445\\
1.57	0.00922740464858932\\
1.58	0.00922740449709594\\
1.59	0.00922740434554427\\
1.6	0.0092274041939343\\
1.61	0.00922740404226599\\
1.62	0.00922740389053932\\
1.63	0.00922740373875428\\
1.64	0.00922740358691083\\
1.65	0.00922740343500895\\
1.66	0.00922740328304861\\
1.67	0.0092274031310298\\
1.68	0.00922740297895248\\
1.69	0.00922740282681663\\
1.7	0.00922740267462224\\
1.71	0.00922740252236926\\
1.72	0.00922740237005768\\
1.73	0.00922740221768748\\
1.74	0.00922740206525862\\
1.75	0.00922740191277109\\
1.76	0.00922740176022485\\
1.77	0.00922740160761989\\
1.78	0.00922740145495618\\
1.79	0.00922740130223369\\
1.8	0.00922740114945241\\
1.81	0.0092274009966123\\
1.82	0.00922740084371334\\
1.83	0.00922740069075551\\
1.84	0.00922740053773878\\
1.85	0.00922740038466312\\
1.86	0.00922740023152852\\
1.87	0.00922740007833494\\
1.88	0.00922739992508236\\
1.89	0.00922739977177076\\
1.9	0.00922739961840012\\
1.91	0.0092273994649704\\
1.92	0.00922739931148158\\
1.93	0.00922739915793364\\
1.94	0.00922739900432656\\
1.95	0.0092273988506603\\
1.96	0.00922739869693484\\
1.97	0.00922739854315017\\
1.98	0.00922739838930624\\
1.99	0.00922739823540304\\
2	0.00922739808144055\\
2.01	0.00922739792741873\\
2.02	0.00922739777333757\\
2.03	0.00922739761919703\\
2.04	0.0092273974649971\\
2.05	0.00922739731073774\\
2.06	0.00922739715641894\\
2.07	0.00922739700204066\\
2.08	0.00922739684760289\\
2.09	0.00922739669310559\\
2.1	0.00922739653854875\\
2.11	0.00922739638393233\\
2.12	0.00922739622925632\\
2.13	0.00922739607452068\\
2.14	0.00922739591972539\\
2.15	0.00922739576487044\\
2.16	0.00922739560995578\\
2.17	0.0092273954549814\\
2.18	0.00922739529994727\\
2.19	0.00922739514485336\\
2.2	0.00922739498969966\\
2.21	0.00922739483448613\\
2.22	0.00922739467921275\\
2.23	0.00922739452387949\\
2.24	0.00922739436848634\\
2.25	0.00922739421303326\\
2.26	0.00922739405752023\\
2.27	0.00922739390194721\\
2.28	0.0092273937463142\\
2.29	0.00922739359062117\\
2.3	0.00922739343486807\\
2.31	0.00922739327905491\\
2.32	0.00922739312318163\\
2.33	0.00922739296724823\\
2.34	0.00922739281125468\\
2.35	0.00922739265520094\\
2.36	0.009227392499087\\
2.37	0.00922739234291283\\
2.38	0.0092273921866784\\
2.39	0.00922739203038369\\
2.4	0.00922739187402868\\
2.41	0.00922739171761333\\
2.42	0.00922739156113763\\
2.43	0.00922739140460154\\
2.44	0.00922739124800504\\
2.45	0.0092273910913481\\
2.46	0.00922739093463071\\
2.47	0.00922739077785283\\
2.48	0.00922739062101443\\
2.49	0.00922739046411551\\
2.5	0.00922739030715602\\
2.51	0.00922739015013594\\
2.52	0.00922738999305524\\
2.53	0.00922738983591391\\
2.54	0.00922738967871191\\
2.55	0.00922738952144922\\
2.56	0.00922738936412582\\
2.57	0.00922738920674167\\
2.58	0.00922738904929675\\
2.59	0.00922738889179104\\
2.6	0.00922738873422451\\
2.61	0.00922738857659714\\
2.62	0.00922738841890889\\
2.63	0.00922738826115975\\
2.64	0.00922738810334968\\
2.65	0.00922738794547867\\
2.66	0.00922738778754668\\
2.67	0.00922738762955369\\
2.68	0.00922738747149967\\
2.69	0.0092273873133846\\
2.7	0.00922738715520846\\
2.71	0.00922738699697121\\
2.72	0.00922738683867283\\
2.73	0.00922738668031329\\
2.74	0.00922738652189258\\
2.75	0.00922738636341065\\
2.76	0.0092273862048675\\
2.77	0.00922738604626308\\
2.78	0.00922738588759738\\
2.79	0.00922738572887036\\
2.8	0.00922738557008201\\
2.81	0.0092273854112323\\
2.82	0.00922738525232119\\
2.83	0.00922738509334867\\
2.84	0.00922738493431471\\
2.85	0.00922738477521928\\
2.86	0.00922738461606236\\
2.87	0.00922738445684391\\
2.88	0.00922738429756392\\
2.89	0.00922738413822236\\
2.9	0.00922738397881921\\
2.91	0.00922738381935442\\
2.92	0.00922738365982798\\
2.93	0.00922738350023987\\
2.94	0.00922738334059006\\
2.95	0.00922738318087851\\
2.96	0.00922738302110522\\
2.97	0.00922738286127013\\
2.98	0.00922738270137325\\
2.99	0.00922738254141452\\
3	0.00922738238139393\\
3.01	0.00922738222131146\\
3.02	0.00922738206116708\\
3.03	0.00922738190096076\\
3.04	0.00922738174069247\\
3.05	0.00922738158036218\\
3.06	0.00922738141996988\\
3.07	0.00922738125951554\\
3.08	0.00922738109899912\\
3.09	0.00922738093842061\\
3.1	0.00922738077777997\\
3.11	0.00922738061707718\\
3.12	0.00922738045631221\\
3.13	0.00922738029548504\\
3.14	0.00922738013459564\\
3.15	0.00922737997364398\\
3.16	0.00922737981263004\\
3.17	0.00922737965155379\\
3.18	0.00922737949041521\\
3.19	0.00922737932921426\\
3.2	0.00922737916795093\\
3.21	0.00922737900662518\\
3.22	0.00922737884523698\\
3.23	0.00922737868378632\\
3.24	0.00922737852227316\\
3.25	0.00922737836069747\\
3.26	0.00922737819905925\\
3.27	0.00922737803735845\\
3.28	0.00922737787559504\\
3.29	0.009227377713769\\
3.3	0.00922737755188031\\
3.31	0.00922737738992894\\
3.32	0.00922737722791486\\
3.33	0.00922737706583804\\
3.34	0.00922737690369846\\
3.35	0.00922737674149609\\
3.36	0.00922737657923091\\
3.37	0.00922737641690288\\
3.38	0.00922737625451199\\
3.39	0.0092273760920582\\
3.4	0.00922737592954148\\
3.41	0.00922737576696182\\
3.42	0.00922737560431918\\
3.43	0.00922737544161354\\
3.44	0.00922737527884486\\
3.45	0.00922737511601313\\
3.46	0.00922737495311831\\
3.47	0.00922737479016039\\
3.48	0.00922737462713932\\
3.49	0.00922737446405509\\
3.5	0.00922737430090767\\
3.51	0.00922737413769703\\
3.52	0.00922737397442314\\
3.53	0.00922737381108598\\
3.54	0.00922737364768552\\
3.55	0.00922737348422173\\
3.56	0.00922737332069459\\
3.57	0.00922737315710407\\
3.58	0.00922737299345014\\
3.59	0.00922737282973277\\
3.6	0.00922737266595195\\
3.61	0.00922737250210763\\
3.62	0.00922737233819979\\
3.63	0.00922737217422842\\
3.64	0.00922737201019347\\
3.65	0.00922737184609492\\
3.66	0.00922737168193275\\
3.67	0.00922737151770693\\
3.68	0.00922737135341742\\
3.69	0.00922737118906421\\
3.7	0.00922737102464727\\
3.71	0.00922737086016656\\
3.72	0.00922737069562207\\
3.73	0.00922737053101375\\
3.74	0.0092273703663416\\
3.75	0.00922737020160558\\
3.76	0.00922737003680566\\
3.77	0.00922736987194181\\
3.78	0.00922736970701401\\
3.79	0.00922736954202223\\
3.8	0.00922736937696644\\
3.81	0.00922736921184662\\
3.82	0.00922736904666274\\
3.83	0.00922736888141476\\
3.84	0.00922736871610267\\
3.85	0.00922736855072644\\
3.86	0.00922736838528603\\
3.87	0.00922736821978142\\
3.88	0.00922736805421259\\
3.89	0.0092273678885795\\
3.9	0.00922736772288213\\
3.91	0.00922736755712046\\
3.92	0.00922736739129444\\
3.93	0.00922736722540406\\
3.94	0.00922736705944929\\
3.95	0.0092273668934301\\
3.96	0.00922736672734646\\
3.97	0.00922736656119834\\
3.98	0.00922736639498573\\
3.99	0.00922736622870858\\
4	0.00922736606236687\\
4.01	0.00922736589596059\\
4.02	0.00922736572948968\\
4.03	0.00922736556295414\\
4.04	0.00922736539635392\\
4.05	0.00922736522968901\\
4.06	0.00922736506295938\\
4.07	0.009227364896165\\
4.08	0.00922736472930583\\
4.09	0.00922736456238186\\
4.1	0.00922736439539305\\
4.11	0.00922736422833938\\
4.12	0.00922736406122082\\
4.13	0.00922736389403733\\
4.14	0.00922736372678891\\
4.15	0.0092273635594755\\
4.16	0.0092273633920971\\
4.17	0.00922736322465366\\
4.18	0.00922736305714517\\
4.19	0.00922736288957159\\
4.2	0.0092273627219329\\
4.21	0.00922736255422906\\
4.22	0.00922736238646005\\
4.23	0.00922736221862585\\
4.24	0.00922736205072642\\
4.25	0.00922736188276173\\
4.26	0.00922736171473176\\
4.27	0.00922736154663649\\
4.28	0.00922736137847587\\
4.29	0.00922736121024989\\
4.3	0.00922736104195851\\
4.31	0.00922736087360171\\
4.32	0.00922736070517946\\
4.33	0.00922736053669173\\
4.34	0.0092273603681385\\
4.35	0.00922736019951973\\
4.36	0.0092273600308354\\
4.37	0.00922735986208547\\
4.38	0.00922735969326992\\
4.39	0.00922735952438873\\
4.4	0.00922735935544186\\
4.41	0.00922735918642928\\
4.42	0.00922735901735097\\
4.43	0.00922735884820691\\
4.44	0.00922735867899705\\
4.45	0.00922735850972137\\
4.46	0.00922735834037985\\
4.47	0.00922735817097245\\
4.48	0.00922735800149915\\
4.49	0.00922735783195992\\
4.5	0.00922735766235473\\
4.51	0.00922735749268355\\
4.52	0.00922735732294636\\
4.53	0.00922735715314312\\
4.54	0.00922735698327381\\
4.55	0.00922735681333839\\
4.56	0.00922735664333685\\
4.57	0.00922735647326915\\
4.58	0.00922735630313525\\
4.59	0.00922735613293515\\
4.6	0.0092273559626688\\
4.61	0.00922735579233618\\
4.62	0.00922735562193725\\
4.63	0.009227355451472\\
4.64	0.00922735528094038\\
4.65	0.00922735511034238\\
4.66	0.00922735493967797\\
4.67	0.00922735476894711\\
4.68	0.00922735459814977\\
4.69	0.00922735442728594\\
4.7	0.00922735425635557\\
4.71	0.00922735408535865\\
4.72	0.00922735391429513\\
4.73	0.009227353743165\\
4.74	0.00922735357196823\\
4.75	0.00922735340070477\\
4.76	0.00922735322937462\\
4.77	0.00922735305797774\\
4.78	0.00922735288651409\\
4.79	0.00922735271498366\\
4.8	0.0092273525433864\\
4.81	0.00922735237172231\\
4.82	0.00922735219999133\\
4.83	0.00922735202819345\\
4.84	0.00922735185632864\\
4.85	0.00922735168439686\\
4.86	0.00922735151239809\\
4.87	0.0092273513403323\\
4.88	0.00922735116819946\\
4.89	0.00922735099599955\\
4.9	0.00922735082373253\\
4.91	0.00922735065139837\\
4.92	0.00922735047899704\\
4.93	0.00922735030652852\\
4.94	0.00922735013399278\\
4.95	0.00922734996138978\\
4.96	0.0092273497887195\\
4.97	0.00922734961598191\\
4.98	0.00922734944317699\\
4.99	0.00922734927030469\\
5	0.00922734909736499\\
5.01	0.00922734892435787\\
5.02	0.00922734875128329\\
5.03	0.00922734857814122\\
5.04	0.00922734840493164\\
5.05	0.00922734823165452\\
5.06	0.00922734805830982\\
5.07	0.00922734788489752\\
5.08	0.00922734771141759\\
5.09	0.00922734753786999\\
5.1	0.00922734736425471\\
5.11	0.00922734719057171\\
5.12	0.00922734701682095\\
5.13	0.00922734684300242\\
5.14	0.00922734666911608\\
5.15	0.00922734649516191\\
5.16	0.00922734632113986\\
5.17	0.00922734614704993\\
5.18	0.00922734597289206\\
5.19	0.00922734579866624\\
5.2	0.00922734562437244\\
5.21	0.00922734545001062\\
5.22	0.00922734527558076\\
5.23	0.00922734510108283\\
5.24	0.00922734492651679\\
5.25	0.00922734475188263\\
5.26	0.0092273445771803\\
5.27	0.00922734440240978\\
5.28	0.00922734422757105\\
5.29	0.00922734405266405\\
5.3	0.00922734387768879\\
5.31	0.00922734370264521\\
5.32	0.0092273435275333\\
5.33	0.00922734335235301\\
5.34	0.00922734317710432\\
5.35	0.00922734300178721\\
5.36	0.00922734282640164\\
5.37	0.00922734265094759\\
5.38	0.00922734247542502\\
5.39	0.0092273422998339\\
5.4	0.0092273421241742\\
5.41	0.0092273419484459\\
5.42	0.00922734177264896\\
5.43	0.00922734159678336\\
5.44	0.00922734142084906\\
5.45	0.00922734124484603\\
5.46	0.00922734106877425\\
5.47	0.00922734089263369\\
5.48	0.00922734071642431\\
5.49	0.00922734054014608\\
5.5	0.00922734036379898\\
5.51	0.00922734018738297\\
5.52	0.00922734001089803\\
5.53	0.00922733983434412\\
5.54	0.00922733965772123\\
5.55	0.0092273394810293\\
5.56	0.00922733930426832\\
5.57	0.00922733912743826\\
5.58	0.00922733895053907\\
5.59	0.00922733877357075\\
5.6	0.00922733859653325\\
5.61	0.00922733841942655\\
5.62	0.0092273382422506\\
5.63	0.0092273380650054\\
5.64	0.0092273378876909\\
5.65	0.00922733771030707\\
5.66	0.00922733753285389\\
5.67	0.00922733735533132\\
5.68	0.00922733717773934\\
5.69	0.00922733700007791\\
5.7	0.009227336822347\\
5.71	0.00922733664454659\\
5.72	0.00922733646667663\\
5.73	0.00922733628873711\\
5.74	0.009227336110728\\
5.75	0.00922733593264925\\
5.76	0.00922733575450085\\
5.77	0.00922733557628276\\
5.78	0.00922733539799495\\
5.79	0.00922733521963739\\
5.8	0.00922733504121005\\
5.81	0.00922733486271289\\
5.82	0.00922733468414591\\
5.83	0.00922733450550904\\
5.84	0.00922733432680228\\
5.85	0.00922733414802558\\
5.86	0.00922733396917892\\
5.87	0.00922733379026227\\
5.88	0.00922733361127559\\
5.89	0.00922733343221886\\
5.9	0.00922733325309205\\
5.91	0.00922733307389512\\
5.92	0.00922733289462804\\
5.93	0.00922733271529079\\
5.94	0.00922733253588333\\
5.95	0.00922733235640563\\
5.96	0.00922733217685767\\
5.97	0.00922733199723941\\
5.98	0.00922733181755081\\
5.99	0.00922733163779186\\
6	0.00922733145796251\\
6.01	0.00922733127806274\\
6.02	0.00922733109809252\\
6.03	0.00922733091805182\\
6.04	0.0092273307379406\\
6.05	0.00922733055775883\\
6.06	0.0092273303775065\\
6.07	0.00922733019718355\\
6.08	0.00922733001678997\\
6.09	0.00922732983632571\\
6.1	0.00922732965579076\\
6.11	0.00922732947518508\\
6.12	0.00922732929450864\\
6.13	0.0092273291137614\\
6.14	0.00922732893294335\\
6.15	0.00922732875205443\\
6.16	0.00922732857109464\\
6.17	0.00922732839006393\\
6.18	0.00922732820896227\\
6.19	0.00922732802778963\\
6.2	0.00922732784654598\\
6.21	0.0092273276652313\\
6.22	0.00922732748384554\\
6.23	0.00922732730238868\\
6.24	0.00922732712086068\\
6.25	0.00922732693926153\\
6.26	0.00922732675759117\\
6.27	0.00922732657584959\\
6.28	0.00922732639403675\\
6.29	0.00922732621215262\\
6.3	0.00922732603019718\\
6.31	0.00922732584817038\\
6.32	0.0092273256660722\\
6.33	0.0092273254839026\\
6.34	0.00922732530166156\\
6.35	0.00922732511934904\\
6.36	0.00922732493696501\\
6.37	0.00922732475450944\\
6.38	0.00922732457198231\\
6.39	0.00922732438938357\\
6.4	0.0092273242067132\\
6.41	0.00922732402397116\\
6.42	0.00922732384115743\\
6.43	0.00922732365827197\\
6.44	0.00922732347531475\\
6.45	0.00922732329228574\\
6.46	0.00922732310918491\\
6.47	0.00922732292601222\\
6.48	0.00922732274276765\\
6.49	0.00922732255945117\\
6.5	0.00922732237606273\\
6.51	0.00922732219260232\\
6.52	0.00922732200906989\\
6.53	0.00922732182546542\\
6.54	0.00922732164178888\\
6.55	0.00922732145804023\\
6.56	0.00922732127421944\\
6.57	0.00922732109032648\\
6.58	0.00922732090636133\\
6.59	0.00922732072232394\\
6.6	0.00922732053821428\\
6.61	0.00922732035403233\\
6.62	0.00922732016977805\\
6.63	0.00922731998545141\\
6.64	0.00922731980105238\\
6.65	0.00922731961658093\\
6.66	0.00922731943203702\\
6.67	0.00922731924742063\\
6.68	0.00922731906273171\\
6.69	0.00922731887797025\\
6.7	0.00922731869313619\\
6.71	0.00922731850822953\\
6.72	0.00922731832325023\\
6.73	0.00922731813819824\\
6.74	0.00922731795307354\\
6.75	0.0092273177678761\\
6.76	0.00922731758260589\\
6.77	0.00922731739726287\\
6.78	0.00922731721184701\\
6.79	0.00922731702635828\\
6.8	0.00922731684079665\\
6.81	0.00922731665516209\\
6.82	0.00922731646945455\\
6.83	0.00922731628367402\\
6.84	0.00922731609782046\\
6.85	0.00922731591189384\\
6.86	0.00922731572589412\\
6.87	0.00922731553982128\\
6.88	0.00922731535367527\\
6.89	0.00922731516745608\\
6.9	0.00922731498116366\\
6.91	0.00922731479479798\\
6.92	0.00922731460835902\\
6.93	0.00922731442184673\\
6.94	0.0092273142352611\\
6.95	0.00922731404860208\\
6.96	0.00922731386186963\\
6.97	0.00922731367506374\\
6.98	0.00922731348818437\\
6.99	0.00922731330123149\\
7	0.00922731311420506\\
7.01	0.00922731292710504\\
7.02	0.00922731273993142\\
7.03	0.00922731255268415\\
7.04	0.00922731236536321\\
7.05	0.00922731217796855\\
7.06	0.00922731199050016\\
7.07	0.00922731180295799\\
7.08	0.00922731161534201\\
7.09	0.0092273114276522\\
7.1	0.00922731123988851\\
7.11	0.00922731105205092\\
7.12	0.00922731086413939\\
7.13	0.00922731067615389\\
7.14	0.0092273104880944\\
7.15	0.00922731029996086\\
7.16	0.00922731011175326\\
7.17	0.00922730992347155\\
7.18	0.00922730973511572\\
7.19	0.00922730954668572\\
7.2	0.00922730935818152\\
7.21	0.00922730916960309\\
7.22	0.0092273089809504\\
7.23	0.0092273087922234\\
7.24	0.00922730860342208\\
7.25	0.0092273084145464\\
7.26	0.00922730822559632\\
7.27	0.00922730803657182\\
7.28	0.00922730784747285\\
7.29	0.00922730765829939\\
7.3	0.00922730746905141\\
7.31	0.00922730727972886\\
7.32	0.00922730709033173\\
7.33	0.00922730690085996\\
7.34	0.00922730671131354\\
7.35	0.00922730652169242\\
7.36	0.00922730633199659\\
7.37	0.00922730614222599\\
7.38	0.00922730595238061\\
7.39	0.0092273057624604\\
7.4	0.00922730557246533\\
7.41	0.00922730538239538\\
7.42	0.0092273051922505\\
7.43	0.00922730500203066\\
7.44	0.00922730481173584\\
7.45	0.00922730462136599\\
7.46	0.00922730443092109\\
7.47	0.0092273042404011\\
7.48	0.00922730404980599\\
7.49	0.00922730385913572\\
7.5	0.00922730366839026\\
7.51	0.00922730347756959\\
7.52	0.00922730328667365\\
7.53	0.00922730309570243\\
7.54	0.00922730290465589\\
7.55	0.009227302713534\\
7.56	0.00922730252233671\\
7.57	0.009227302331064\\
7.58	0.00922730213971584\\
7.59	0.0092273019482922\\
7.6	0.00922730175679302\\
7.61	0.0092273015652183\\
7.62	0.00922730137356798\\
7.63	0.00922730118184204\\
7.64	0.00922730099004045\\
7.65	0.00922730079816316\\
7.66	0.00922730060621016\\
7.67	0.0092273004141814\\
7.68	0.00922730022207685\\
7.69	0.00922730002989647\\
7.7	0.00922729983764023\\
7.71	0.00922729964530811\\
7.72	0.00922729945290006\\
7.73	0.00922729926041606\\
7.74	0.00922729906785606\\
7.75	0.00922729887522004\\
7.76	0.00922729868250795\\
7.77	0.00922729848971978\\
7.78	0.00922729829685548\\
7.79	0.00922729810391502\\
7.8	0.00922729791089836\\
7.81	0.00922729771780548\\
7.82	0.00922729752463633\\
7.83	0.00922729733139089\\
7.84	0.00922729713806912\\
7.85	0.00922729694467099\\
7.86	0.00922729675119646\\
7.87	0.0092272965576455\\
7.88	0.00922729636401808\\
7.89	0.00922729617031415\\
7.9	0.0092272959765337\\
7.91	0.00922729578267667\\
7.92	0.00922729558874305\\
7.93	0.00922729539473279\\
7.94	0.00922729520064587\\
7.95	0.00922729500648224\\
7.96	0.00922729481224187\\
7.97	0.00922729461792473\\
7.98	0.00922729442353079\\
7.99	0.00922729422906001\\
8	0.00922729403451236\\
8.01	0.00922729383988779\\
8.02	0.00922729364518629\\
8.03	0.00922729345040781\\
8.04	0.00922729325555233\\
8.05	0.0092272930606198\\
8.06	0.00922729286561019\\
8.07	0.00922729267052347\\
8.08	0.0092272924753596\\
8.09	0.00922729228011855\\
8.1	0.00922729208480028\\
8.11	0.00922729188940477\\
8.12	0.00922729169393197\\
8.13	0.00922729149838186\\
8.14	0.00922729130275439\\
8.15	0.00922729110704954\\
8.16	0.00922729091126726\\
8.17	0.00922729071540753\\
8.18	0.00922729051947031\\
8.19	0.00922729032345557\\
8.2	0.00922729012736327\\
8.21	0.00922728993119337\\
8.22	0.00922728973494584\\
8.23	0.00922728953862066\\
8.24	0.00922728934221777\\
8.25	0.00922728914573716\\
8.26	0.00922728894917878\\
8.27	0.0092272887525426\\
8.28	0.00922728855582858\\
8.29	0.00922728835903669\\
8.3	0.0092272881621669\\
8.31	0.00922728796521917\\
8.32	0.00922728776819347\\
8.33	0.00922728757108976\\
8.34	0.009227287373908\\
8.35	0.00922728717664817\\
8.36	0.00922728697931023\\
8.37	0.00922728678189414\\
8.38	0.00922728658439986\\
8.39	0.00922728638682738\\
8.4	0.00922728618917664\\
8.41	0.00922728599144761\\
8.42	0.00922728579364026\\
8.43	0.00922728559575455\\
8.44	0.00922728539779046\\
8.45	0.00922728519974794\\
8.46	0.00922728500162696\\
8.47	0.00922728480342748\\
8.48	0.00922728460514948\\
8.49	0.00922728440679291\\
8.5	0.00922728420835774\\
8.51	0.00922728400984393\\
8.52	0.00922728381125145\\
8.53	0.00922728361258028\\
8.54	0.00922728341383035\\
8.55	0.00922728321500166\\
8.56	0.00922728301609415\\
8.57	0.0092272828171078\\
8.58	0.00922728261804257\\
8.59	0.00922728241889843\\
8.6	0.00922728221967533\\
8.61	0.00922728202037325\\
8.62	0.00922728182099215\\
8.63	0.00922728162153199\\
8.64	0.00922728142199274\\
8.65	0.00922728122237437\\
8.66	0.00922728102267683\\
8.67	0.0092272808229001\\
8.68	0.00922728062304414\\
8.69	0.00922728042310891\\
8.7	0.00922728022309438\\
8.71	0.0092272800230005\\
8.72	0.00922727982282726\\
8.73	0.00922727962257461\\
8.74	0.00922727942224252\\
8.75	0.00922727922183095\\
8.76	0.00922727902133986\\
8.77	0.00922727882076922\\
8.78	0.009227278620119\\
8.79	0.00922727841938916\\
8.8	0.00922727821857967\\
8.81	0.00922727801769048\\
8.82	0.00922727781672157\\
8.83	0.0092272776156729\\
8.84	0.00922727741454442\\
8.85	0.00922727721333612\\
8.86	0.00922727701204795\\
8.87	0.00922727681067987\\
8.88	0.00922727660923186\\
8.89	0.00922727640770387\\
8.9	0.00922727620609587\\
8.91	0.00922727600440782\\
8.92	0.0092272758026397\\
8.93	0.00922727560079145\\
8.94	0.00922727539886306\\
8.95	0.00922727519685448\\
8.96	0.00922727499476567\\
8.97	0.0092272747925966\\
8.98	0.00922727459034724\\
8.99	0.00922727438801754\\
9	0.00922727418560749\\
9.01	0.00922727398311702\\
9.02	0.00922727378054612\\
9.03	0.00922727357789476\\
9.04	0.00922727337516287\\
9.05	0.00922727317235045\\
9.06	0.00922727296945744\\
9.07	0.00922727276648382\\
9.08	0.00922727256342954\\
9.09	0.00922727236029458\\
9.1	0.00922727215707889\\
9.11	0.00922727195378244\\
9.12	0.0092272717504052\\
9.13	0.00922727154694713\\
9.14	0.00922727134340819\\
9.15	0.00922727113978834\\
9.16	0.00922727093608756\\
9.17	0.0092272707323058\\
9.18	0.00922727052844304\\
9.19	0.00922727032449922\\
9.2	0.00922727012047433\\
9.21	0.00922726991636831\\
9.22	0.00922726971218115\\
9.23	0.00922726950791279\\
9.24	0.0092272693035632\\
9.25	0.00922726909913235\\
9.26	0.00922726889462021\\
9.27	0.00922726869002673\\
9.28	0.00922726848535188\\
9.29	0.00922726828059562\\
9.3	0.00922726807575793\\
9.31	0.00922726787083875\\
9.32	0.00922726766583806\\
9.33	0.00922726746075582\\
9.34	0.00922726725559199\\
9.35	0.00922726705034655\\
9.36	0.00922726684501944\\
9.37	0.00922726663961063\\
9.38	0.0092272664341201\\
9.39	0.00922726622854781\\
9.4	0.0092272660228937\\
9.41	0.00922726581715776\\
9.42	0.00922726561133994\\
9.43	0.00922726540544022\\
9.44	0.00922726519945854\\
9.45	0.00922726499339488\\
9.46	0.0092272647872492\\
9.47	0.00922726458102147\\
9.48	0.00922726437471164\\
9.49	0.00922726416831969\\
9.5	0.00922726396184557\\
9.51	0.00922726375528925\\
9.52	0.0092272635486507\\
9.53	0.00922726334192987\\
9.54	0.00922726313512673\\
9.55	0.00922726292824124\\
9.56	0.00922726272127338\\
9.57	0.00922726251422309\\
9.58	0.00922726230709035\\
9.59	0.00922726209987512\\
9.6	0.00922726189257737\\
9.61	0.00922726168519705\\
9.62	0.00922726147773413\\
9.63	0.00922726127018858\\
9.64	0.00922726106256035\\
9.65	0.00922726085484942\\
9.66	0.00922726064705574\\
9.67	0.00922726043917928\\
9.68	0.00922726023122001\\
9.69	0.00922726002317788\\
9.7	0.00922725981505286\\
9.71	0.00922725960684492\\
9.72	0.00922725939855401\\
9.73	0.0092272591901801\\
9.74	0.00922725898172316\\
9.75	0.00922725877318316\\
9.76	0.00922725856456004\\
9.77	0.00922725835585378\\
9.78	0.00922725814706434\\
9.79	0.00922725793819167\\
9.8	0.00922725772923577\\
9.81	0.00922725752019657\\
9.82	0.00922725731107404\\
9.83	0.00922725710186816\\
9.84	0.00922725689257887\\
9.85	0.00922725668320615\\
9.86	0.00922725647374996\\
9.87	0.00922725626421027\\
9.88	0.00922725605458703\\
9.89	0.00922725584488021\\
9.9	0.00922725563508977\\
9.91	0.00922725542521568\\
9.92	0.00922725521525791\\
9.93	0.00922725500521641\\
9.94	0.00922725479509114\\
9.95	0.00922725458488209\\
9.96	0.00922725437458919\\
9.97	0.00922725416421243\\
9.98	0.00922725395375176\\
9.99	0.00922725374320715\\
10	0.00922725353257856\\
10.01	0.00922725332186595\\
10.02	0.00922725311106929\\
10.03	0.00922725290018855\\
10.04	0.00922725268922367\\
10.05	0.00922725247817465\\
10.06	0.00922725226704142\\
10.07	0.00922725205582396\\
10.08	0.00922725184452223\\
10.09	0.00922725163313619\\
10.1	0.00922725142166582\\
10.11	0.00922725121011106\\
10.12	0.0092272509984719\\
10.13	0.00922725078674828\\
10.14	0.00922725057494017\\
10.15	0.00922725036304755\\
10.16	0.00922725015107036\\
10.17	0.00922724993900858\\
10.18	0.00922724972686217\\
10.19	0.00922724951463109\\
10.2	0.00922724930231531\\
10.21	0.00922724908991479\\
10.22	0.00922724887742949\\
10.23	0.00922724866485938\\
10.24	0.00922724845220442\\
10.25	0.00922724823946458\\
10.26	0.00922724802663982\\
10.27	0.0092272478137301\\
10.28	0.00922724760073539\\
10.29	0.00922724738765565\\
10.3	0.00922724717449085\\
10.31	0.00922724696124095\\
10.32	0.00922724674790591\\
10.33	0.0092272465344857\\
10.34	0.00922724632098028\\
10.35	0.00922724610738962\\
10.36	0.00922724589371368\\
10.37	0.00922724567995242\\
10.38	0.00922724546610581\\
10.39	0.00922724525217382\\
10.4	0.0092272450381564\\
10.41	0.00922724482405352\\
10.42	0.00922724460986514\\
10.43	0.00922724439559124\\
10.44	0.00922724418123177\\
10.45	0.0092272439667867\\
10.46	0.00922724375225598\\
10.47	0.0092272435376396\\
10.48	0.0092272433229375\\
10.49	0.00922724310814967\\
10.5	0.00922724289327605\\
10.51	0.00922724267831661\\
10.52	0.00922724246327132\\
10.53	0.00922724224814015\\
10.54	0.00922724203292305\\
10.55	0.00922724181761999\\
10.56	0.00922724160223094\\
10.57	0.00922724138675586\\
10.58	0.00922724117119472\\
10.59	0.00922724095554748\\
10.6	0.0092272407398141\\
10.61	0.00922724052399455\\
10.62	0.00922724030808879\\
10.63	0.00922724009209679\\
10.64	0.00922723987601852\\
10.65	0.00922723965985394\\
10.66	0.00922723944360301\\
10.67	0.00922723922726569\\
10.68	0.00922723901084196\\
10.69	0.00922723879433178\\
10.7	0.00922723857773511\\
10.71	0.00922723836105192\\
10.72	0.00922723814428217\\
10.73	0.00922723792742582\\
10.74	0.00922723771048285\\
10.75	0.00922723749345322\\
10.76	0.0092272372763369\\
10.77	0.00922723705913384\\
10.78	0.00922723684184402\\
10.79	0.00922723662446739\\
10.8	0.00922723640700393\\
10.81	0.0092272361894536\\
10.82	0.00922723597181637\\
10.83	0.00922723575409219\\
10.84	0.00922723553628104\\
10.85	0.00922723531838289\\
10.86	0.00922723510039769\\
10.87	0.00922723488232542\\
10.88	0.00922723466416603\\
10.89	0.0092272344459195\\
10.9	0.00922723422758579\\
10.91	0.00922723400916486\\
10.92	0.00922723379065669\\
10.93	0.00922723357206123\\
10.94	0.00922723335337846\\
10.95	0.00922723313460834\\
10.96	0.00922723291575084\\
10.97	0.00922723269680592\\
10.98	0.00922723247777354\\
10.99	0.00922723225865369\\
11	0.00922723203944631\\
11.01	0.00922723182015138\\
11.02	0.00922723160076886\\
11.03	0.00922723138129873\\
11.04	0.00922723116174094\\
11.05	0.00922723094209546\\
11.06	0.00922723072236227\\
11.07	0.00922723050254132\\
11.08	0.00922723028263259\\
11.09	0.00922723006263603\\
11.1	0.00922722984255162\\
11.11	0.00922722962237933\\
11.12	0.00922722940211912\\
11.13	0.00922722918177096\\
11.14	0.00922722896133481\\
11.15	0.00922722874081065\\
11.16	0.00922722852019843\\
11.17	0.00922722829949814\\
11.18	0.00922722807870973\\
11.19	0.00922722785783316\\
11.2	0.00922722763686843\\
11.21	0.00922722741581547\\
11.22	0.00922722719467428\\
11.23	0.0092272269734448\\
11.24	0.00922722675212702\\
11.25	0.00922722653072089\\
11.26	0.00922722630922639\\
11.27	0.00922722608764349\\
11.28	0.00922722586597215\\
11.29	0.00922722564421234\\
11.3	0.00922722542236403\\
11.31	0.00922722520042719\\
11.32	0.00922722497840178\\
11.33	0.00922722475628778\\
11.34	0.00922722453408515\\
11.35	0.00922722431179386\\
11.36	0.00922722408941389\\
11.37	0.00922722386694519\\
11.38	0.00922722364438775\\
11.39	0.00922722342174151\\
11.4	0.00922722319900647\\
11.41	0.00922722297618258\\
11.42	0.00922722275326982\\
11.43	0.00922722253026815\\
11.44	0.00922722230717754\\
11.45	0.00922722208399797\\
11.46	0.0092272218607294\\
11.47	0.00922722163737181\\
11.48	0.00922722141392515\\
11.49	0.00922722119038942\\
11.5	0.00922722096676455\\
11.51	0.00922722074305055\\
11.52	0.00922722051924737\\
11.53	0.00922722029535498\\
11.54	0.00922722007137335\\
11.55	0.00922721984730246\\
11.56	0.00922721962314227\\
11.57	0.00922721939889276\\
11.58	0.00922721917455389\\
11.59	0.00922721895012564\\
11.6	0.00922721872560797\\
11.61	0.00922721850100087\\
11.62	0.0092272182763043\\
11.63	0.00922721805151822\\
11.64	0.00922721782664262\\
11.65	0.00922721760167747\\
11.66	0.00922721737662272\\
11.67	0.00922721715147837\\
11.68	0.00922721692624437\\
11.69	0.00922721670092071\\
11.7	0.00922721647550735\\
11.71	0.00922721625000426\\
11.72	0.00922721602441142\\
11.73	0.0092272157987288\\
11.74	0.00922721557295638\\
11.75	0.00922721534709411\\
11.76	0.00922721512114198\\
11.77	0.00922721489509997\\
11.78	0.00922721466896803\\
11.79	0.00922721444274615\\
11.8	0.0092272142164343\\
11.81	0.00922721399003245\\
11.82	0.00922721376354058\\
11.83	0.00922721353695865\\
11.84	0.00922721331028665\\
11.85	0.00922721308352454\\
11.86	0.00922721285667231\\
11.87	0.00922721262972991\\
11.88	0.00922721240269734\\
11.89	0.00922721217557455\\
11.9	0.00922721194836153\\
11.91	0.00922721172105826\\
11.92	0.00922721149366469\\
11.93	0.00922721126618081\\
11.94	0.0092272110386066\\
11.95	0.00922721081094203\\
11.96	0.00922721058318707\\
11.97	0.00922721035534169\\
11.98	0.00922721012740589\\
11.99	0.00922720989937962\\
12	0.00922720967126287\\
12.01	0.0092272094430556\\
12.02	0.0092272092147578\\
12.03	0.00922720898636944\\
12.04	0.0092272087578905\\
12.05	0.00922720852932095\\
12.06	0.00922720830066077\\
12.07	0.00922720807190993\\
12.08	0.00922720784306842\\
12.09	0.0092272076141362\\
12.1	0.00922720738511326\\
12.11	0.00922720715599957\\
12.12	0.00922720692679511\\
12.13	0.00922720669749985\\
12.14	0.00922720646811377\\
12.15	0.00922720623863685\\
12.16	0.00922720600906907\\
12.17	0.0092272057794104\\
12.18	0.00922720554966082\\
12.19	0.00922720531982031\\
12.2	0.00922720508988885\\
12.21	0.0092272048598664\\
12.22	0.00922720462975297\\
12.23	0.00922720439954851\\
12.24	0.009227204169253\\
12.25	0.00922720393886644\\
12.26	0.00922720370838879\\
12.27	0.00922720347782003\\
12.28	0.00922720324716014\\
12.29	0.0092272030164091\\
12.3	0.00922720278556689\\
12.31	0.00922720255463349\\
12.32	0.00922720232360887\\
12.33	0.00922720209249302\\
12.34	0.00922720186128592\\
12.35	0.00922720162998753\\
12.36	0.00922720139859785\\
12.37	0.00922720116711685\\
12.38	0.00922720093554451\\
12.39	0.00922720070388081\\
12.4	0.00922720047212574\\
12.41	0.00922720024027926\\
12.42	0.00922720000834137\\
12.43	0.00922719977631203\\
12.44	0.00922719954419124\\
12.45	0.00922719931197896\\
12.46	0.00922719907967519\\
12.47	0.00922719884727989\\
12.48	0.00922719861479306\\
12.49	0.00922719838221467\\
12.5	0.0092271981495447\\
12.51	0.00922719791678313\\
12.52	0.00922719768392994\\
12.53	0.00922719745098512\\
12.54	0.00922719721794864\\
12.55	0.00922719698482048\\
12.56	0.00922719675160063\\
12.57	0.00922719651828907\\
12.58	0.00922719628488577\\
12.59	0.00922719605139072\\
12.6	0.0092271958178039\\
12.61	0.00922719558412529\\
12.62	0.00922719535035487\\
12.63	0.00922719511649262\\
12.64	0.00922719488253852\\
12.65	0.00922719464849255\\
12.66	0.00922719441435471\\
12.67	0.00922719418012495\\
12.68	0.00922719394580327\\
12.69	0.00922719371138965\\
12.7	0.00922719347688407\\
12.71	0.00922719324228651\\
12.72	0.00922719300759695\\
12.73	0.00922719277281537\\
12.74	0.00922719253794175\\
12.75	0.00922719230297608\\
12.76	0.00922719206791833\\
12.77	0.00922719183276848\\
12.78	0.00922719159752653\\
12.79	0.00922719136219244\\
12.8	0.0092271911267662\\
12.81	0.00922719089124779\\
12.82	0.00922719065563718\\
12.83	0.00922719041993437\\
12.84	0.00922719018413933\\
12.85	0.00922718994825204\\
12.86	0.00922718971227249\\
12.87	0.00922718947620064\\
12.88	0.00922718924003649\\
12.89	0.00922718900378002\\
12.9	0.00922718876743119\\
12.91	0.00922718853098999\\
12.92	0.00922718829445642\\
12.93	0.00922718805783043\\
12.94	0.00922718782111201\\
12.95	0.00922718758430115\\
12.96	0.00922718734739781\\
12.97	0.00922718711040199\\
12.98	0.00922718687331365\\
12.99	0.00922718663613279\\
13	0.00922718639885936\\
13.01	0.00922718616149337\\
13.02	0.00922718592403477\\
13.03	0.00922718568648356\\
13.04	0.00922718544883971\\
13.05	0.00922718521110319\\
13.06	0.00922718497327399\\
13.07	0.00922718473535208\\
13.08	0.00922718449733744\\
13.09	0.00922718425923004\\
13.1	0.00922718402102987\\
13.11	0.0092271837827369\\
13.12	0.00922718354435109\\
13.13	0.00922718330587245\\
13.14	0.00922718306730092\\
13.15	0.0092271828286365\\
13.16	0.00922718258987915\\
13.17	0.00922718235102886\\
13.18	0.00922718211208559\\
13.19	0.00922718187304932\\
13.2	0.00922718163392002\\
13.21	0.00922718139469767\\
13.22	0.00922718115538223\\
13.23	0.00922718091597369\\
13.24	0.00922718067647201\\
13.25	0.00922718043687717\\
13.26	0.00922718019718914\\
13.27	0.00922717995740788\\
13.28	0.00922717971753337\\
13.29	0.00922717947756558\\
13.3	0.00922717923750448\\
13.31	0.00922717899735004\\
13.32	0.00922717875710223\\
13.33	0.00922717851676101\\
13.34	0.00922717827632635\\
13.35	0.00922717803579823\\
13.36	0.0092271777951766\\
13.37	0.00922717755446144\\
13.38	0.00922717731365271\\
13.39	0.00922717707275038\\
13.4	0.00922717683175441\\
13.41	0.00922717659066477\\
13.42	0.00922717634948141\\
13.43	0.00922717610820431\\
13.44	0.00922717586683342\\
13.45	0.00922717562536872\\
13.46	0.00922717538381015\\
13.47	0.00922717514215769\\
13.48	0.00922717490041128\\
13.49	0.0092271746585709\\
13.5	0.0092271744166365\\
13.51	0.00922717417460804\\
13.52	0.00922717393248548\\
13.53	0.00922717369026877\\
13.54	0.00922717344795787\\
13.55	0.00922717320555274\\
13.56	0.00922717296305333\\
13.57	0.0092271727204596\\
13.58	0.0092271724777715\\
13.59	0.00922717223498898\\
13.6	0.009227171992112\\
13.61	0.00922717174914051\\
13.62	0.00922717150607445\\
13.63	0.00922717126291379\\
13.64	0.00922717101965847\\
13.65	0.00922717077630843\\
13.66	0.00922717053286363\\
13.67	0.009227170289324\\
13.68	0.00922717004568952\\
13.69	0.0092271698019601\\
13.7	0.0092271695581357\\
13.71	0.00922716931421627\\
13.72	0.00922716907020174\\
13.73	0.00922716882609206\\
13.74	0.00922716858188717\\
13.75	0.009227168337587\\
13.76	0.00922716809319151\\
13.77	0.00922716784870063\\
13.78	0.0092271676041143\\
13.79	0.00922716735943245\\
13.8	0.00922716711465502\\
13.81	0.00922716686978196\\
13.82	0.00922716662481318\\
13.83	0.00922716637974862\\
13.84	0.00922716613458823\\
13.85	0.00922716588933192\\
13.86	0.00922716564397963\\
13.87	0.0092271653985313\\
13.88	0.00922716515298684\\
13.89	0.0092271649073462\\
13.9	0.00922716466160929\\
13.91	0.00922716441577604\\
13.92	0.00922716416984638\\
13.93	0.00922716392382023\\
13.94	0.00922716367769753\\
13.95	0.00922716343147818\\
13.96	0.00922716318516212\\
13.97	0.00922716293874926\\
13.98	0.00922716269223953\\
13.99	0.00922716244563284\\
14	0.00922716219892911\\
14.01	0.00922716195212826\\
14.02	0.00922716170523022\\
14.03	0.00922716145823488\\
14.04	0.00922716121114218\\
14.05	0.00922716096395202\\
14.06	0.00922716071666432\\
14.07	0.00922716046927898\\
14.08	0.00922716022179594\\
14.09	0.00922715997421508\\
14.1	0.00922715972653632\\
14.11	0.00922715947875959\\
14.12	0.00922715923088477\\
14.13	0.00922715898291178\\
14.14	0.00922715873484054\\
14.15	0.00922715848667093\\
14.16	0.00922715823840288\\
14.17	0.00922715799003628\\
14.18	0.00922715774157105\\
14.19	0.00922715749300708\\
14.2	0.00922715724434428\\
14.21	0.00922715699558254\\
14.22	0.00922715674672178\\
14.23	0.00922715649776189\\
14.24	0.00922715624870278\\
14.25	0.00922715599954434\\
14.26	0.00922715575028647\\
14.27	0.00922715550092909\\
14.28	0.00922715525147207\\
14.29	0.00922715500191532\\
14.3	0.00922715475225875\\
14.31	0.00922715450250224\\
14.32	0.0092271542526457\\
14.33	0.00922715400268902\\
14.34	0.00922715375263209\\
14.35	0.00922715350247482\\
14.36	0.0092271532522171\\
14.37	0.00922715300185883\\
14.38	0.00922715275139989\\
14.39	0.0092271525008402\\
14.4	0.00922715225017964\\
14.41	0.0092271519994181\\
14.42	0.00922715174855549\\
14.43	0.0092271514975917\\
14.44	0.00922715124652662\\
14.45	0.00922715099536015\\
14.46	0.00922715074409219\\
14.47	0.00922715049272262\\
14.48	0.00922715024125135\\
14.49	0.00922714998967828\\
14.5	0.00922714973800329\\
14.51	0.00922714948622628\\
14.52	0.00922714923434715\\
14.53	0.0092271489823658\\
14.54	0.00922714873028213\\
14.55	0.00922714847809603\\
14.56	0.00922714822580739\\
14.57	0.00922714797341613\\
14.58	0.00922714772092214\\
14.59	0.00922714746832531\\
14.6	0.00922714721562556\\
14.61	0.00922714696282277\\
14.62	0.00922714670991685\\
14.63	0.00922714645690771\\
14.64	0.00922714620379524\\
14.65	0.00922714595057936\\
14.66	0.00922714569725997\\
14.67	0.00922714544383695\\
14.68	0.00922714519031025\\
14.69	0.00922714493667974\\
14.7	0.00922714468294535\\
14.71	0.00922714442910698\\
14.72	0.00922714417516454\\
14.73	0.00922714392111794\\
14.74	0.0092271436669671\\
14.75	0.00922714341271193\\
14.76	0.00922714315835235\\
14.77	0.00922714290388826\\
14.78	0.00922714264931958\\
14.79	0.00922714239464624\\
14.8	0.00922714213986815\\
14.81	0.00922714188498524\\
14.82	0.00922714162999742\\
14.83	0.00922714137490461\\
14.84	0.00922714111970674\\
14.85	0.00922714086440374\\
14.86	0.00922714060899552\\
14.87	0.00922714035348203\\
14.88	0.00922714009786318\\
14.89	0.0092271398421389\\
14.9	0.00922713958630913\\
14.91	0.00922713933037379\\
14.92	0.00922713907433283\\
14.93	0.00922713881818617\\
14.94	0.00922713856193374\\
14.95	0.0092271383055755\\
14.96	0.00922713804911136\\
14.97	0.00922713779254128\\
14.98	0.00922713753586518\\
14.99	0.00922713727908302\\
15	0.00922713702219472\\
15.01	0.00922713676520024\\
15.02	0.00922713650809951\\
15.03	0.00922713625089247\\
15.04	0.00922713599357908\\
15.05	0.00922713573615928\\
15.06	0.00922713547863301\\
15.07	0.00922713522100022\\
15.08	0.00922713496326085\\
15.09	0.00922713470541485\\
15.1	0.00922713444746217\\
15.11	0.00922713418940276\\
15.12	0.00922713393123657\\
15.13	0.00922713367296353\\
15.14	0.0092271334145836\\
15.15	0.00922713315609674\\
15.16	0.00922713289750288\\
15.17	0.00922713263880198\\
15.18	0.00922713237999398\\
15.19	0.00922713212107884\\
15.2	0.0092271318620565\\
15.21	0.00922713160292691\\
15.22	0.00922713134369002\\
15.23	0.00922713108434578\\
15.24	0.00922713082489413\\
15.25	0.00922713056533503\\
15.26	0.00922713030566843\\
15.27	0.00922713004589426\\
15.28	0.00922712978601249\\
15.29	0.00922712952602305\\
15.3	0.0092271292659259\\
15.31	0.00922712900572099\\
15.32	0.00922712874540826\\
15.33	0.00922712848498766\\
15.34	0.00922712822445915\\
15.35	0.00922712796382266\\
15.36	0.00922712770307815\\
15.37	0.00922712744222557\\
15.38	0.00922712718126485\\
15.39	0.00922712692019596\\
15.4	0.00922712665901884\\
15.41	0.00922712639773343\\
15.42	0.00922712613633969\\
15.43	0.00922712587483756\\
15.44	0.009227125613227\\
15.45	0.00922712535150794\\
15.46	0.00922712508968034\\
15.47	0.00922712482774415\\
15.48	0.0092271245656993\\
15.49	0.00922712430354576\\
15.5	0.00922712404128347\\
15.51	0.00922712377891237\\
15.52	0.00922712351643242\\
15.53	0.00922712325384355\\
15.54	0.00922712299114573\\
15.55	0.00922712272833889\\
15.56	0.00922712246542299\\
15.57	0.00922712220239797\\
15.58	0.00922712193926377\\
15.59	0.00922712167602036\\
15.6	0.00922712141266766\\
15.61	0.00922712114920564\\
15.62	0.00922712088563424\\
15.63	0.00922712062195341\\
15.64	0.00922712035816308\\
15.65	0.00922712009426322\\
15.66	0.00922711983025377\\
15.67	0.00922711956613467\\
15.68	0.00922711930190588\\
15.69	0.00922711903756733\\
15.7	0.00922711877311898\\
15.71	0.00922711850856078\\
15.72	0.00922711824389267\\
15.73	0.00922711797911459\\
15.74	0.00922711771422651\\
15.75	0.00922711744922835\\
15.76	0.00922711718412007\\
15.77	0.00922711691890162\\
15.78	0.00922711665357294\\
15.79	0.00922711638813399\\
15.8	0.00922711612258469\\
15.81	0.00922711585692502\\
15.82	0.0092271155911549\\
15.83	0.00922711532527429\\
15.84	0.00922711505928314\\
15.85	0.00922711479318138\\
15.86	0.00922711452696897\\
15.87	0.00922711426064586\\
15.88	0.00922711399421199\\
15.89	0.0092271137276673\\
15.9	0.00922711346101175\\
15.91	0.00922711319424528\\
15.92	0.00922711292736783\\
15.93	0.00922711266037936\\
15.94	0.00922711239327981\\
15.95	0.00922711212606912\\
15.96	0.00922711185874724\\
15.97	0.00922711159131413\\
15.98	0.00922711132376972\\
15.99	0.00922711105611396\\
16	0.0092271107883468\\
16.01	0.00922711052046818\\
16.02	0.00922711025247805\\
16.03	0.00922710998437636\\
16.04	0.00922710971616304\\
16.05	0.00922710944783806\\
16.06	0.00922710917940135\\
16.07	0.00922710891085286\\
16.08	0.00922710864219255\\
16.09	0.00922710837342034\\
16.1	0.00922710810453618\\
16.11	0.00922710783554003\\
16.12	0.00922710756643184\\
16.13	0.00922710729721153\\
16.14	0.00922710702787907\\
16.15	0.0092271067584344\\
16.16	0.00922710648887746\\
16.17	0.00922710621920819\\
16.18	0.00922710594942656\\
16.19	0.00922710567953249\\
16.2	0.00922710540952594\\
16.21	0.00922710513940684\\
16.22	0.00922710486917516\\
16.23	0.00922710459883083\\
16.24	0.0092271043283738\\
16.25	0.009227104057804\\
16.26	0.0092271037871214\\
16.27	0.00922710351632593\\
16.28	0.00922710324541754\\
16.29	0.00922710297439617\\
16.3	0.00922710270326178\\
16.31	0.0092271024320143\\
16.32	0.00922710216065368\\
16.33	0.00922710188917986\\
16.34	0.0092271016175928\\
16.35	0.00922710134589243\\
16.36	0.0092271010740787\\
16.37	0.00922710080215156\\
16.38	0.00922710053011096\\
16.39	0.00922710025795683\\
16.4	0.00922709998568911\\
16.41	0.00922709971330776\\
16.42	0.00922709944081273\\
16.43	0.00922709916820395\\
16.44	0.00922709889548137\\
16.45	0.00922709862264494\\
16.46	0.0092270983496946\\
16.47	0.00922709807663029\\
16.48	0.00922709780345196\\
16.49	0.00922709753015956\\
16.5	0.00922709725675303\\
16.51	0.00922709698323231\\
16.52	0.00922709670959735\\
16.53	0.00922709643584809\\
16.54	0.00922709616198448\\
16.55	0.00922709588800646\\
16.56	0.00922709561391397\\
16.57	0.00922709533970697\\
16.58	0.0092270950653854\\
16.59	0.00922709479094919\\
16.6	0.0092270945163983\\
16.61	0.00922709424173267\\
16.62	0.00922709396695223\\
16.63	0.00922709369205695\\
16.64	0.00922709341704676\\
16.65	0.0092270931419216\\
16.66	0.00922709286668142\\
16.67	0.00922709259132617\\
16.68	0.00922709231585578\\
16.69	0.0092270920402702\\
16.7	0.00922709176456938\\
16.71	0.00922709148875326\\
16.72	0.00922709121282179\\
16.73	0.0092270909367749\\
16.74	0.00922709066061254\\
16.75	0.00922709038433466\\
16.76	0.0092270901079412\\
16.77	0.0092270898314321\\
16.78	0.00922708955480731\\
16.79	0.00922708927806676\\
16.8	0.00922708900121042\\
16.81	0.0092270887242382\\
16.82	0.00922708844715008\\
16.83	0.00922708816994597\\
16.84	0.00922708789262584\\
16.85	0.00922708761518961\\
16.86	0.00922708733763724\\
16.87	0.00922708705996868\\
16.88	0.00922708678218385\\
16.89	0.00922708650428271\\
16.9	0.0092270862262652\\
16.91	0.00922708594813126\\
16.92	0.00922708566988084\\
16.93	0.00922708539151388\\
16.94	0.00922708511303032\\
16.95	0.00922708483443011\\
16.96	0.00922708455571318\\
16.97	0.00922708427687949\\
16.98	0.00922708399792897\\
16.99	0.00922708371886157\\
17	0.00922708343967723\\
17.01	0.0092270831603759\\
17.02	0.00922708288095751\\
17.03	0.00922708260142202\\
17.04	0.00922708232176935\\
17.05	0.00922708204199947\\
17.06	0.0092270817621123\\
17.07	0.0092270814821078\\
17.08	0.00922708120198589\\
17.09	0.00922708092174654\\
17.1	0.00922708064138968\\
17.11	0.00922708036091524\\
17.12	0.00922708008032319\\
17.13	0.00922707979961345\\
17.14	0.00922707951878597\\
17.15	0.0092270792378407\\
17.16	0.00922707895677757\\
17.17	0.00922707867559653\\
17.18	0.00922707839429752\\
17.19	0.00922707811288048\\
17.2	0.00922707783134536\\
17.21	0.00922707754969209\\
17.22	0.00922707726792063\\
17.23	0.00922707698603091\\
17.24	0.00922707670402287\\
17.25	0.00922707642189646\\
17.26	0.00922707613965162\\
17.27	0.00922707585728829\\
17.28	0.00922707557480641\\
17.29	0.00922707529220593\\
17.3	0.00922707500948679\\
17.31	0.00922707472664893\\
17.32	0.00922707444369228\\
17.33	0.0092270741606168\\
17.34	0.00922707387742243\\
17.35	0.0092270735941091\\
17.36	0.00922707331067676\\
17.37	0.00922707302712535\\
17.38	0.00922707274345482\\
17.39	0.0092270724596651\\
17.4	0.00922707217575614\\
17.41	0.00922707189172788\\
17.42	0.00922707160758025\\
17.43	0.00922707132331321\\
17.44	0.00922707103892669\\
17.45	0.00922707075442063\\
17.46	0.00922707046979498\\
17.47	0.00922707018504967\\
17.48	0.00922706990018466\\
17.49	0.00922706961519987\\
17.5	0.00922706933009526\\
17.51	0.00922706904487076\\
17.52	0.00922706875952631\\
17.53	0.00922706847406186\\
17.54	0.00922706818847734\\
17.55	0.00922706790277271\\
17.56	0.0092270676169479\\
17.57	0.00922706733100284\\
17.58	0.00922706704493748\\
17.59	0.00922706675875177\\
17.6	0.00922706647244565\\
17.61	0.00922706618601904\\
17.62	0.0092270658994719\\
17.63	0.00922706561280417\\
17.64	0.00922706532601579\\
17.65	0.00922706503910669\\
17.66	0.00922706475207683\\
17.67	0.00922706446492613\\
17.68	0.00922706417765455\\
17.69	0.00922706389026202\\
17.7	0.00922706360274848\\
17.71	0.00922706331511387\\
17.72	0.00922706302735813\\
17.73	0.00922706273948122\\
17.74	0.00922706245148305\\
17.75	0.00922706216336358\\
17.76	0.00922706187512275\\
17.77	0.00922706158676049\\
17.78	0.00922706129827675\\
17.79	0.00922706100967148\\
17.8	0.00922706072094459\\
17.81	0.00922706043209604\\
17.82	0.00922706014312578\\
17.83	0.00922705985403373\\
17.84	0.00922705956481983\\
17.85	0.00922705927548404\\
17.86	0.00922705898602629\\
17.87	0.00922705869644652\\
17.88	0.00922705840674466\\
17.89	0.00922705811692067\\
17.9	0.00922705782697448\\
17.91	0.00922705753690602\\
17.92	0.00922705724671524\\
17.93	0.00922705695640209\\
17.94	0.0092270566659665\\
17.95	0.0092270563754084\\
17.96	0.00922705608472774\\
17.97	0.00922705579392446\\
17.98	0.0092270555029985\\
17.99	0.0092270552119498\\
18	0.00922705492077829\\
18.01	0.00922705462948393\\
18.02	0.00922705433806664\\
18.03	0.00922705404652637\\
18.04	0.00922705375486305\\
18.05	0.00922705346307663\\
18.06	0.00922705317116705\\
18.07	0.00922705287913424\\
18.08	0.00922705258697814\\
18.09	0.0092270522946987\\
18.1	0.00922705200229585\\
18.11	0.00922705170976954\\
18.12	0.00922705141711969\\
18.13	0.00922705112434626\\
18.14	0.00922705083144917\\
18.15	0.00922705053842838\\
18.16	0.00922705024528381\\
18.17	0.00922704995201541\\
18.18	0.00922704965862312\\
18.19	0.00922704936510687\\
18.2	0.00922704907146661\\
18.21	0.00922704877770227\\
18.22	0.0092270484838138\\
18.23	0.00922704818980113\\
18.24	0.00922704789566419\\
18.25	0.00922704760140294\\
18.26	0.0092270473070173\\
18.27	0.00922704701250722\\
18.28	0.00922704671787264\\
18.29	0.00922704642311349\\
18.3	0.00922704612822971\\
18.31	0.00922704583322124\\
18.32	0.00922704553808802\\
18.33	0.00922704524283\\
18.34	0.00922704494744709\\
18.35	0.00922704465193926\\
18.36	0.00922704435630642\\
18.37	0.00922704406054853\\
18.38	0.00922704376466552\\
18.39	0.00922704346865733\\
18.4	0.0092270431725239\\
18.41	0.00922704287626516\\
18.42	0.00922704257988106\\
18.43	0.00922704228337153\\
18.44	0.00922704198673651\\
18.45	0.00922704168997593\\
18.46	0.00922704139308974\\
18.47	0.00922704109607788\\
18.48	0.00922704079894028\\
18.49	0.00922704050167688\\
18.5	0.00922704020428762\\
18.51	0.00922703990677243\\
18.52	0.00922703960913126\\
18.53	0.00922703931136404\\
18.54	0.00922703901347071\\
18.55	0.00922703871545121\\
18.56	0.00922703841730547\\
18.57	0.00922703811903344\\
18.58	0.00922703782063505\\
18.59	0.00922703752211023\\
18.6	0.00922703722345893\\
18.61	0.00922703692468109\\
18.62	0.00922703662577663\\
18.63	0.00922703632674551\\
18.64	0.00922703602758765\\
18.65	0.00922703572830299\\
18.66	0.00922703542889147\\
18.67	0.00922703512935304\\
18.68	0.00922703482968761\\
18.69	0.00922703452989515\\
18.7	0.00922703422997556\\
18.71	0.00922703392992881\\
18.72	0.00922703362975482\\
18.73	0.00922703332945354\\
18.74	0.00922703302902489\\
18.75	0.00922703272846881\\
18.76	0.00922703242778525\\
18.77	0.00922703212697414\\
18.78	0.00922703182603542\\
18.79	0.00922703152496902\\
18.8	0.00922703122377488\\
18.81	0.00922703092245293\\
18.82	0.00922703062100312\\
18.83	0.00922703031942538\\
18.84	0.00922703001771965\\
18.85	0.00922702971588586\\
18.86	0.00922702941392396\\
18.87	0.00922702911183387\\
18.88	0.00922702880961553\\
18.89	0.00922702850726889\\
18.9	0.00922702820479387\\
18.91	0.00922702790219042\\
18.92	0.00922702759945847\\
18.93	0.00922702729659795\\
18.94	0.00922702699360881\\
18.95	0.00922702669049097\\
18.96	0.00922702638724438\\
18.97	0.00922702608386898\\
18.98	0.00922702578036469\\
18.99	0.00922702547673145\\
19	0.00922702517296921\\
19.01	0.00922702486907789\\
19.02	0.00922702456505743\\
19.03	0.00922702426090778\\
19.04	0.00922702395662886\\
19.05	0.00922702365222061\\
19.06	0.00922702334768296\\
19.07	0.00922702304301586\\
19.08	0.00922702273821924\\
19.09	0.00922702243329303\\
19.1	0.00922702212823717\\
19.11	0.0092270218230516\\
19.12	0.00922702151773624\\
19.13	0.00922702121229105\\
19.14	0.00922702090671594\\
19.15	0.00922702060101087\\
19.16	0.00922702029517576\\
19.17	0.00922701998921055\\
19.18	0.00922701968311518\\
19.19	0.00922701937688957\\
19.2	0.00922701907053367\\
19.21	0.00922701876404742\\
19.22	0.00922701845743074\\
19.23	0.00922701815068357\\
19.24	0.00922701784380585\\
19.25	0.00922701753679751\\
19.26	0.00922701722965849\\
19.27	0.00922701692238872\\
19.28	0.00922701661498814\\
19.29	0.00922701630745668\\
19.3	0.00922701599979429\\
19.31	0.00922701569200089\\
19.32	0.00922701538407641\\
19.33	0.0092270150760208\\
19.34	0.00922701476783398\\
19.35	0.0092270144595159\\
19.36	0.00922701415106649\\
19.37	0.00922701384248568\\
19.38	0.0092270135337734\\
19.39	0.0092270132249296\\
19.4	0.00922701291595421\\
19.41	0.00922701260684716\\
19.42	0.00922701229760838\\
19.43	0.00922701198823781\\
19.44	0.0092270116787354\\
19.45	0.00922701136910105\\
19.46	0.00922701105933473\\
19.47	0.00922701074943635\\
19.48	0.00922701043940585\\
19.49	0.00922701012924318\\
19.5	0.00922700981894825\\
19.51	0.00922700950852101\\
19.52	0.00922700919796139\\
19.53	0.00922700888726933\\
19.54	0.00922700857644475\\
19.55	0.0092270082654876\\
19.56	0.0092270079543978\\
19.57	0.0092270076431753\\
19.58	0.00922700733182002\\
19.59	0.0092270070203319\\
19.6	0.00922700670871087\\
19.61	0.00922700639695687\\
19.62	0.00922700608506983\\
19.63	0.00922700577304968\\
19.64	0.00922700546089637\\
19.65	0.00922700514860981\\
19.66	0.00922700483618995\\
19.67	0.00922700452363672\\
19.68	0.00922700421095005\\
19.69	0.00922700389812988\\
19.7	0.00922700358517614\\
19.71	0.00922700327208877\\
19.72	0.00922700295886769\\
19.73	0.00922700264551284\\
19.74	0.00922700233202416\\
19.75	0.00922700201840157\\
19.76	0.00922700170464502\\
19.77	0.00922700139075442\\
19.78	0.00922700107672973\\
19.79	0.00922700076257087\\
19.8	0.00922700044827777\\
19.81	0.00922700013385036\\
19.82	0.00922699981928859\\
19.83	0.00922699950459238\\
19.84	0.00922699918976167\\
19.85	0.00922699887479639\\
19.86	0.00922699855969646\\
19.87	0.00922699824446184\\
19.88	0.00922699792909244\\
19.89	0.0092269976135882\\
19.9	0.00922699729794906\\
19.91	0.00922699698217494\\
19.92	0.00922699666626578\\
19.93	0.00922699635022151\\
19.94	0.00922699603404207\\
19.95	0.00922699571772739\\
19.96	0.00922699540127739\\
19.97	0.00922699508469202\\
19.98	0.0092269947679712\\
19.99	0.00922699445111486\\
20	0.00922699413412295\\
20.01	0.00922699381699539\\
20.02	0.00922699349973211\\
20.03	0.00922699318233306\\
20.04	0.00922699286479814\\
20.05	0.00922699254712731\\
20.06	0.00922699222932049\\
20.07	0.00922699191137762\\
20.08	0.00922699159329862\\
20.09	0.00922699127508344\\
20.1	0.009226990956732\\
20.11	0.00922699063824423\\
20.12	0.00922699031962006\\
20.13	0.00922699000085943\\
20.14	0.00922698968196227\\
20.15	0.00922698936292851\\
20.16	0.00922698904375809\\
20.17	0.00922698872445092\\
20.18	0.00922698840500696\\
20.19	0.00922698808542612\\
20.2	0.00922698776570834\\
20.21	0.00922698744585355\\
20.22	0.00922698712586169\\
20.23	0.00922698680573268\\
20.24	0.00922698648546645\\
20.25	0.00922698616506294\\
20.26	0.00922698584452208\\
20.27	0.0092269855238438\\
20.28	0.00922698520302802\\
20.29	0.00922698488207469\\
20.3	0.00922698456098374\\
20.31	0.00922698423975509\\
20.32	0.00922698391838867\\
20.33	0.00922698359688442\\
20.34	0.00922698327524227\\
20.35	0.00922698295346215\\
20.36	0.00922698263154398\\
20.37	0.00922698230948771\\
20.38	0.00922698198729326\\
20.39	0.00922698166496056\\
20.4	0.00922698134248955\\
20.41	0.00922698101988014\\
20.42	0.00922698069713229\\
20.43	0.0092269803742459\\
20.44	0.00922698005122093\\
20.45	0.00922697972805729\\
20.46	0.00922697940475492\\
20.47	0.00922697908131374\\
20.48	0.00922697875773369\\
20.49	0.0092269784340147\\
20.5	0.00922697811015671\\
20.51	0.00922697778615962\\
20.52	0.00922697746202339\\
20.53	0.00922697713774794\\
20.54	0.0092269768133332\\
20.55	0.0092269764887791\\
20.56	0.00922697616408557\\
20.57	0.00922697583925254\\
20.58	0.00922697551427994\\
20.59	0.0092269751891677\\
20.6	0.00922697486391575\\
20.61	0.00922697453852402\\
20.62	0.00922697421299244\\
20.63	0.00922697388732094\\
20.64	0.00922697356150945\\
20.65	0.0092269732355579\\
20.66	0.00922697290946622\\
20.67	0.00922697258323434\\
20.68	0.00922697225686218\\
20.69	0.00922697193034969\\
20.7	0.00922697160369678\\
20.71	0.00922697127690339\\
20.72	0.00922697094996944\\
20.73	0.00922697062289487\\
20.74	0.00922697029567961\\
20.75	0.00922696996832358\\
20.76	0.00922696964082671\\
20.77	0.00922696931318894\\
20.78	0.00922696898541019\\
20.79	0.00922696865749039\\
20.8	0.00922696832942947\\
20.81	0.00922696800122736\\
20.82	0.00922696767288399\\
20.83	0.00922696734439929\\
20.84	0.00922696701577318\\
20.85	0.0092269666870056\\
20.86	0.00922696635809647\\
20.87	0.00922696602904572\\
20.88	0.00922696569985328\\
20.89	0.00922696537051909\\
20.9	0.00922696504104306\\
20.91	0.00922696471142513\\
20.92	0.00922696438166523\\
20.93	0.00922696405176327\\
20.94	0.0092269637217192\\
20.95	0.00922696339153295\\
20.96	0.00922696306120443\\
20.97	0.00922696273073358\\
20.98	0.00922696240012033\\
20.99	0.0092269620693646\\
21	0.00922696173846632\\
21.01	0.00922696140742542\\
21.02	0.00922696107624183\\
21.03	0.00922696074491548\\
21.04	0.00922696041344629\\
21.05	0.00922696008183419\\
21.06	0.00922695975007912\\
21.07	0.00922695941818099\\
21.08	0.00922695908613974\\
21.09	0.00922695875395528\\
21.1	0.00922695842162756\\
21.11	0.0092269580891565\\
21.12	0.00922695775654203\\
21.13	0.00922695742378407\\
21.14	0.00922695709088255\\
21.15	0.0092269567578374\\
21.16	0.00922695642464855\\
21.17	0.00922695609131591\\
21.18	0.00922695575783943\\
21.19	0.00922695542421902\\
21.2	0.00922695509045463\\
21.21	0.00922695475654616\\
21.22	0.00922695442249355\\
21.23	0.00922695408829673\\
21.24	0.00922695375395562\\
21.25	0.00922695341947015\\
21.26	0.00922695308484024\\
21.27	0.00922695275006583\\
21.28	0.00922695241514685\\
21.29	0.0092269520800832\\
21.3	0.00922695174487483\\
21.31	0.00922695140952166\\
21.32	0.00922695107402362\\
21.33	0.00922695073838063\\
21.34	0.00922695040259262\\
21.35	0.00922695006665952\\
21.36	0.00922694973058124\\
21.37	0.00922694939435773\\
21.38	0.0092269490579889\\
21.39	0.00922694872147468\\
21.4	0.009226948384815\\
21.41	0.00922694804800978\\
21.42	0.00922694771105895\\
21.43	0.00922694737396243\\
21.44	0.00922694703672016\\
21.45	0.00922694669933205\\
21.46	0.00922694636179803\\
21.47	0.00922694602411803\\
21.48	0.00922694568629198\\
21.49	0.00922694534831979\\
21.5	0.0092269450102014\\
21.51	0.00922694467193672\\
21.52	0.0092269443335257\\
21.53	0.00922694399496824\\
21.54	0.00922694365626428\\
21.55	0.00922694331741373\\
21.56	0.00922694297841654\\
21.57	0.00922694263927262\\
21.58	0.00922694229998189\\
21.59	0.00922694196054428\\
21.6	0.00922694162095972\\
21.61	0.00922694128122813\\
21.62	0.00922694094134944\\
21.63	0.00922694060132357\\
21.64	0.00922694026115044\\
21.65	0.00922693992082998\\
21.66	0.00922693958036211\\
21.67	0.00922693923974676\\
21.68	0.00922693889898386\\
21.69	0.00922693855807332\\
21.7	0.00922693821701508\\
21.71	0.00922693787580905\\
21.72	0.00922693753445516\\
21.73	0.00922693719295333\\
21.74	0.00922693685130349\\
21.75	0.00922693650950557\\
21.76	0.00922693616755947\\
21.77	0.00922693582546514\\
21.78	0.00922693548322249\\
21.79	0.00922693514083145\\
21.8	0.00922693479829194\\
21.81	0.00922693445560387\\
21.82	0.00922693411276719\\
21.83	0.00922693376978181\\
21.84	0.00922693342664765\\
21.85	0.00922693308336464\\
21.86	0.00922693273993269\\
21.87	0.00922693239635175\\
21.88	0.00922693205262171\\
21.89	0.00922693170874252\\
21.9	0.00922693136471408\\
21.91	0.00922693102053633\\
21.92	0.00922693067620919\\
21.93	0.00922693033173257\\
21.94	0.00922692998710641\\
21.95	0.00922692964233062\\
21.96	0.00922692929740513\\
21.97	0.00922692895232986\\
21.98	0.00922692860710473\\
21.99	0.00922692826172966\\
22	0.00922692791620458\\
22.01	0.0092269275705294\\
22.02	0.00922692722470405\\
22.03	0.00922692687872846\\
22.04	0.00922692653260253\\
22.05	0.0092269261863262\\
22.06	0.00922692583989938\\
22.07	0.00922692549332201\\
22.08	0.00922692514659399\\
22.09	0.00922692479971525\\
22.1	0.00922692445268571\\
22.11	0.00922692410550529\\
22.12	0.00922692375817391\\
22.13	0.0092269234106915\\
22.14	0.00922692306305798\\
22.15	0.00922692271527325\\
22.16	0.00922692236733725\\
22.17	0.0092269220192499\\
22.18	0.00922692167101111\\
22.19	0.00922692132262081\\
22.2	0.00922692097407892\\
22.21	0.00922692062538535\\
22.22	0.00922692027654004\\
22.23	0.00922691992754288\\
22.24	0.00922691957839381\\
22.25	0.00922691922909275\\
22.26	0.00922691887963962\\
22.27	0.00922691853003433\\
22.28	0.00922691818027681\\
22.29	0.00922691783036697\\
22.3	0.00922691748030473\\
22.31	0.00922691713009001\\
22.32	0.00922691677972274\\
22.33	0.00922691642920282\\
22.34	0.00922691607853019\\
22.35	0.00922691572770475\\
22.36	0.00922691537672642\\
22.37	0.00922691502559513\\
22.38	0.0092269146743108\\
22.39	0.00922691432287333\\
22.4	0.00922691397128265\\
22.41	0.00922691361953868\\
22.42	0.00922691326764133\\
22.43	0.00922691291559053\\
22.44	0.00922691256338618\\
22.45	0.00922691221102821\\
22.46	0.00922691185851653\\
22.47	0.00922691150585107\\
22.48	0.00922691115303173\\
22.49	0.00922691080005844\\
22.5	0.00922691044693111\\
22.51	0.00922691009364966\\
22.52	0.009226909740214\\
22.53	0.00922690938662406\\
22.54	0.00922690903287974\\
22.55	0.00922690867898096\\
22.56	0.00922690832492765\\
22.57	0.00922690797071971\\
22.58	0.00922690761635706\\
22.59	0.00922690726183961\\
22.6	0.00922690690716729\\
22.61	0.00922690655234001\\
22.62	0.00922690619735768\\
22.63	0.00922690584222021\\
22.64	0.00922690548692753\\
22.65	0.00922690513147954\\
22.66	0.00922690477587617\\
22.67	0.00922690442011732\\
22.68	0.0092269040642029\\
22.69	0.00922690370813285\\
22.7	0.00922690335190706\\
22.71	0.00922690299552544\\
22.72	0.00922690263898792\\
22.73	0.00922690228229441\\
22.74	0.00922690192544482\\
22.75	0.00922690156843907\\
22.76	0.00922690121127706\\
22.77	0.00922690085395871\\
22.78	0.00922690049648392\\
22.79	0.00922690013885263\\
22.8	0.00922689978106473\\
22.81	0.00922689942312013\\
22.82	0.00922689906501876\\
22.83	0.00922689870676051\\
22.84	0.00922689834834531\\
22.85	0.00922689798977305\\
22.86	0.00922689763104367\\
22.87	0.00922689727215706\\
22.88	0.00922689691311313\\
22.89	0.00922689655391179\\
22.9	0.00922689619455297\\
22.91	0.00922689583503655\\
22.92	0.00922689547536247\\
22.93	0.00922689511553061\\
22.94	0.0092268947555409\\
22.95	0.00922689439539325\\
22.96	0.00922689403508755\\
22.97	0.00922689367462372\\
22.98	0.00922689331400167\\
22.99	0.00922689295322131\\
23	0.00922689259228254\\
23.01	0.00922689223118527\\
23.02	0.00922689186992942\\
23.03	0.00922689150851488\\
23.04	0.00922689114694156\\
23.05	0.00922689078520937\\
23.06	0.00922689042331822\\
23.07	0.00922689006126801\\
23.08	0.00922688969905865\\
23.09	0.00922688933669004\\
23.1	0.00922688897416209\\
23.11	0.00922688861147471\\
23.12	0.00922688824862779\\
23.13	0.00922688788562125\\
23.14	0.00922688752245499\\
23.15	0.0092268871591289\\
23.16	0.0092268867956429\\
23.17	0.0092268864319969\\
23.18	0.00922688606819078\\
23.19	0.00922688570422446\\
23.2	0.00922688534009784\\
23.21	0.00922688497581081\\
23.22	0.00922688461136329\\
23.23	0.00922688424675517\\
23.24	0.00922688388198636\\
23.25	0.00922688351705675\\
23.26	0.00922688315196625\\
23.27	0.00922688278671475\\
23.28	0.00922688242130216\\
23.29	0.00922688205572837\\
23.3	0.00922688168999329\\
23.31	0.00922688132409682\\
23.32	0.00922688095803884\\
23.33	0.00922688059181928\\
23.34	0.009226880225438\\
23.35	0.00922687985889492\\
23.36	0.00922687949218994\\
23.37	0.00922687912532295\\
23.38	0.00922687875829384\\
23.39	0.00922687839110251\\
23.4	0.00922687802374887\\
23.41	0.0092268776562328\\
23.42	0.00922687728855419\\
23.43	0.00922687692071295\\
23.44	0.00922687655270897\\
23.45	0.00922687618454214\\
23.46	0.00922687581621234\\
23.47	0.00922687544771949\\
23.48	0.00922687507906346\\
23.49	0.00922687471024416\\
23.5	0.00922687434126146\\
23.51	0.00922687397211527\\
23.52	0.00922687360280547\\
23.53	0.00922687323333196\\
23.54	0.00922687286369462\\
23.55	0.00922687249389334\\
23.56	0.00922687212392801\\
23.57	0.00922687175379852\\
23.58	0.00922687138350476\\
23.59	0.0092268710130466\\
23.6	0.00922687064242395\\
23.61	0.00922687027163669\\
23.62	0.00922686990068469\\
23.63	0.00922686952956786\\
23.64	0.00922686915828606\\
23.65	0.00922686878683919\\
23.66	0.00922686841522713\\
23.67	0.00922686804344976\\
23.68	0.00922686767150696\\
23.69	0.00922686729939863\\
23.7	0.00922686692712462\\
23.71	0.00922686655468483\\
23.72	0.00922686618207915\\
23.73	0.00922686580930744\\
23.74	0.00922686543636958\\
23.75	0.00922686506326546\\
23.76	0.00922686468999495\\
23.77	0.00922686431655792\\
23.78	0.00922686394295426\\
23.79	0.00922686356918384\\
23.8	0.00922686319524654\\
23.81	0.00922686282114222\\
23.82	0.00922686244687076\\
23.83	0.00922686207243204\\
23.84	0.00922686169782593\\
23.85	0.0092268613230523\\
23.86	0.00922686094811101\\
23.87	0.00922686057300194\\
23.88	0.00922686019772497\\
23.89	0.00922685982227994\\
23.9	0.00922685944666674\\
23.91	0.00922685907088524\\
23.92	0.00922685869493529\\
23.93	0.00922685831881676\\
23.94	0.00922685794252953\\
23.95	0.00922685756607344\\
23.96	0.00922685718944836\\
23.97	0.00922685681265416\\
23.98	0.0092268564356907\\
23.99	0.00922685605855783\\
24	0.00922685568125542\\
24.01	0.00922685530378333\\
24.02	0.00922685492614141\\
24.03	0.00922685454832952\\
24.04	0.00922685417034751\\
24.05	0.00922685379219524\\
24.06	0.00922685341387257\\
24.07	0.00922685303537934\\
24.08	0.00922685265671541\\
24.09	0.00922685227788064\\
24.1	0.00922685189887486\\
24.11	0.00922685151969793\\
24.12	0.00922685114034969\\
24.13	0.00922685076083\\
24.14	0.0092268503811387\\
24.15	0.00922685000127563\\
24.16	0.00922684962124064\\
24.17	0.00922684924103357\\
24.18	0.00922684886065425\\
24.19	0.00922684848010255\\
24.2	0.00922684809937827\\
24.21	0.00922684771848128\\
24.22	0.00922684733741141\\
24.23	0.00922684695616848\\
24.24	0.00922684657475233\\
24.25	0.00922684619316281\\
24.26	0.00922684581139973\\
24.27	0.00922684542946293\\
24.28	0.00922684504735225\\
24.29	0.00922684466506749\\
24.3	0.0092268442826085\\
24.31	0.0092268438999751\\
24.32	0.00922684351716711\\
24.33	0.00922684313418436\\
24.34	0.00922684275102666\\
24.35	0.00922684236769383\\
24.36	0.0092268419841857\\
24.37	0.00922684160050208\\
24.38	0.00922684121664278\\
24.39	0.00922684083260762\\
24.4	0.00922684044839642\\
24.41	0.00922684006400897\\
24.42	0.00922683967944509\\
24.43	0.00922683929470459\\
24.44	0.00922683890978728\\
24.45	0.00922683852469295\\
24.46	0.00922683813942141\\
24.47	0.00922683775397246\\
24.48	0.00922683736834591\\
24.49	0.00922683698254154\\
24.5	0.00922683659655915\\
24.51	0.00922683621039854\\
24.52	0.0092268358240595\\
24.53	0.00922683543754182\\
24.54	0.00922683505084528\\
24.55	0.00922683466396968\\
24.56	0.00922683427691479\\
24.57	0.00922683388968041\\
24.58	0.00922683350226631\\
24.59	0.00922683311467226\\
24.6	0.00922683272689805\\
24.61	0.00922683233894346\\
24.62	0.00922683195080825\\
24.63	0.00922683156249219\\
24.64	0.00922683117399505\\
24.65	0.0092268307853166\\
24.66	0.0092268303964566\\
24.67	0.00922683000741481\\
24.68	0.009226829618191\\
24.69	0.00922682922878492\\
24.7	0.00922682883919631\\
24.71	0.00922682844942495\\
24.72	0.00922682805947057\\
24.73	0.00922682766933293\\
24.74	0.00922682727901176\\
24.75	0.00922682688850682\\
24.76	0.00922682649781784\\
24.77	0.00922682610694455\\
24.78	0.0092268257158867\\
24.79	0.00922682532464402\\
24.8	0.00922682493321623\\
24.81	0.00922682454160306\\
24.82	0.00922682414980425\\
24.83	0.00922682375781949\\
24.84	0.00922682336564853\\
24.85	0.00922682297329106\\
24.86	0.00922682258074682\\
24.87	0.00922682218801549\\
24.88	0.0092268217950968\\
24.89	0.00922682140199044\\
24.9	0.00922682100869611\\
24.91	0.00922682061521352\\
24.92	0.00922682022154236\\
24.93	0.00922681982768231\\
24.94	0.00922681943363307\\
24.95	0.00922681903939432\\
24.96	0.00922681864496574\\
24.97	0.00922681825034701\\
24.98	0.00922681785553781\\
24.99	0.0092268174605378\\
25	0.00922681706534665\\
25.01	0.00922681666996403\\
25.02	0.00922681627438959\\
25.03	0.009226815878623\\
25.04	0.00922681548266391\\
25.05	0.00922681508651195\\
25.06	0.00922681469016679\\
25.07	0.00922681429362806\\
25.08	0.0092268138968954\\
25.09	0.00922681349996844\\
25.1	0.00922681310284681\\
25.11	0.00922681270553013\\
25.12	0.00922681230801804\\
25.13	0.00922681191031013\\
25.14	0.00922681151240603\\
25.15	0.00922681111430535\\
25.16	0.00922681071600768\\
25.17	0.00922681031751263\\
25.18	0.00922680991881979\\
25.19	0.00922680951992875\\
25.2	0.0092268091208391\\
25.21	0.00922680872155041\\
25.22	0.00922680832206227\\
25.23	0.00922680792237424\\
25.24	0.00922680752248589\\
25.25	0.00922680712239678\\
25.26	0.00922680672210647\\
25.27	0.0092268063216145\\
25.28	0.00922680592092044\\
25.29	0.00922680552002379\\
25.3	0.00922680511892413\\
25.31	0.00922680471762096\\
25.32	0.00922680431611381\\
25.33	0.00922680391440219\\
25.34	0.00922680351248564\\
25.35	0.00922680311036364\\
25.36	0.00922680270803571\\
25.37	0.00922680230550133\\
25.38	0.00922680190275999\\
25.39	0.00922680149981119\\
25.4	0.00922680109665439\\
25.41	0.00922680069328907\\
25.42	0.00922680028971469\\
25.43	0.00922679988593071\\
25.44	0.00922679948193658\\
25.45	0.00922679907773175\\
25.46	0.00922679867331566\\
25.47	0.00922679826868773\\
25.48	0.00922679786384739\\
25.49	0.00922679745879406\\
25.5	0.00922679705352716\\
25.51	0.00922679664804607\\
25.52	0.00922679624235021\\
25.53	0.00922679583643895\\
25.54	0.00922679543031168\\
25.55	0.00922679502396778\\
25.56	0.00922679461740661\\
25.57	0.00922679421062753\\
25.58	0.00922679380362989\\
25.59	0.00922679339641303\\
25.6	0.00922679298897629\\
25.61	0.009226792581319\\
25.62	0.00922679217344046\\
25.63	0.00922679176534\\
25.64	0.00922679135701691\\
25.65	0.00922679094847049\\
25.66	0.00922679053970001\\
25.67	0.00922679013070476\\
25.68	0.00922678972148401\\
25.69	0.00922678931203699\\
25.7	0.00922678890236297\\
25.71	0.00922678849246118\\
25.72	0.00922678808233086\\
25.73	0.00922678767197122\\
25.74	0.00922678726138146\\
25.75	0.00922678685056079\\
25.76	0.0092267864395084\\
25.77	0.00922678602822346\\
25.78	0.00922678561670515\\
25.79	0.00922678520495262\\
25.8	0.00922678479296503\\
25.81	0.0092267843807415\\
25.82	0.00922678396828116\\
25.83	0.00922678355558313\\
25.84	0.00922678314264652\\
25.85	0.0092267827294704\\
25.86	0.00922678231605388\\
25.87	0.009226781902396\\
25.88	0.00922678148849583\\
25.89	0.00922678107435243\\
25.9	0.00922678065996481\\
25.91	0.009226780245332\\
25.92	0.00922677983045301\\
25.93	0.00922677941532684\\
25.94	0.00922677899995246\\
25.95	0.00922677858432886\\
25.96	0.00922677816845497\\
25.97	0.00922677775232975\\
25.98	0.00922677733595214\\
25.99	0.00922677691932104\\
26	0.00922677650243535\\
26.01	0.00922677608529397\\
26.02	0.00922677566789576\\
26.03	0.0092267752502396\\
26.04	0.00922677483232432\\
26.05	0.00922677441414875\\
26.06	0.00922677399571171\\
26.07	0.009226773577012\\
26.08	0.00922677315804841\\
26.09	0.00922677273881969\\
26.1	0.00922677231932461\\
26.11	0.0092267718995619\\
26.12	0.00922677147953029\\
26.13	0.00922677105922847\\
26.14	0.00922677063865513\\
26.15	0.00922677021780894\\
26.16	0.00922676979668857\\
26.17	0.00922676937529263\\
26.18	0.00922676895361976\\
26.19	0.00922676853166855\\
26.2	0.00922676810943758\\
26.21	0.00922676768692541\\
26.22	0.0092267672641306\\
26.23	0.00922676684105167\\
26.24	0.00922676641768712\\
26.25	0.00922676599403545\\
26.26	0.00922676557009512\\
26.27	0.00922676514586457\\
26.28	0.00922676472134225\\
26.29	0.00922676429652655\\
26.3	0.00922676387141586\\
26.31	0.00922676344600856\\
26.32	0.00922676302030298\\
26.33	0.00922676259429745\\
26.34	0.00922676216799028\\
26.35	0.00922676174137973\\
26.36	0.00922676131446407\\
26.37	0.00922676088724153\\
26.38	0.00922676045971032\\
26.39	0.00922676003186864\\
26.4	0.00922675960371465\\
26.41	0.00922675917524649\\
26.42	0.00922675874646228\\
26.43	0.0092267583173601\\
26.44	0.00922675788793803\\
26.45	0.00922675745819411\\
26.46	0.00922675702812637\\
26.47	0.00922675659773277\\
26.48	0.0092267561670113\\
26.49	0.0092267557359599\\
26.5	0.00922675530457647\\
26.51	0.0092267548728589\\
26.52	0.00922675444080505\\
26.53	0.00922675400841275\\
26.54	0.00922675357567979\\
26.55	0.00922675314260396\\
26.56	0.00922675270918299\\
26.57	0.0092267522754146\\
26.58	0.00922675184129647\\
26.59	0.00922675140682626\\
26.6	0.0092267509720016\\
26.61	0.00922675053682007\\
26.62	0.00922675010127924\\
26.63	0.00922674966537664\\
26.64	0.00922674922910977\\
26.65	0.00922674879247609\\
26.66	0.00922674835547303\\
26.67	0.009226747918098\\
26.68	0.00922674748034835\\
26.69	0.00922674704222143\\
26.7	0.00922674660371453\\
26.71	0.00922674616482491\\
26.72	0.0092267457255498\\
26.73	0.00922674528588638\\
26.74	0.00922674484583182\\
26.75	0.00922674440538322\\
26.76	0.00922674396453767\\
26.77	0.00922674352329221\\
26.78	0.00922674308164384\\
26.79	0.00922674263958953\\
26.8	0.0092267421971262\\
26.81	0.00922674175425074\\
26.82	0.00922674131095999\\
26.83	0.00922674086725076\\
26.84	0.0092267404231198\\
26.85	0.00922673997856385\\
26.86	0.00922673953357958\\
26.87	0.00922673908816362\\
26.88	0.00922673864231257\\
26.89	0.00922673819602298\\
26.9	0.00922673774929134\\
26.91	0.00922673730211412\\
26.92	0.00922673685448773\\
26.93	0.00922673640640854\\
26.94	0.00922673595787287\\
26.95	0.00922673550887698\\
26.96	0.0092267350594171\\
26.97	0.00922673460948942\\
26.98	0.00922673415909006\\
26.99	0.00922673370821508\\
27	0.00922673325686052\\
27.01	0.00922673280502235\\
27.02	0.0092267323526965\\
27.03	0.00922673189987883\\
27.04	0.00922673144656518\\
27.05	0.00922673099275129\\
27.06	0.00922673053843287\\
27.07	0.00922673008360559\\
27.08	0.00922672962826505\\
27.09	0.00922672917240677\\
27.1	0.00922672871602626\\
27.11	0.00922672825911893\\
27.12	0.00922672780168017\\
27.13	0.00922672734370527\\
27.14	0.00922672688518948\\
27.15	0.00922672642612801\\
27.16	0.00922672596651597\\
27.17	0.00922672550634843\\
27.18	0.0092267250456204\\
27.19	0.0092267245843268\\
27.2	0.00922672412246253\\
27.21	0.00922672366002238\\
27.22	0.0092267231970011\\
27.23	0.00922672273339337\\
27.24	0.00922672226919379\\
27.25	0.00922672180439692\\
27.26	0.00922672133899721\\
27.27	0.00922672087298906\\
27.28	0.00922672040636682\\
27.29	0.00922671993912473\\
27.3	0.00922671947125697\\
27.31	0.00922671900275768\\
27.32	0.00922671853362087\\
27.33	0.00922671806384051\\
27.34	0.00922671759341048\\
27.35	0.00922671712232461\\
27.36	0.0092267166505766\\
27.37	0.00922671617816012\\
27.38	0.00922671570506873\\
27.39	0.00922671523129593\\
27.4	0.00922671475683512\\
27.41	0.00922671428167963\\
27.42	0.00922671380582269\\
27.43	0.00922671332925747\\
27.44	0.00922671285197703\\
27.45	0.00922671237397435\\
27.46	0.00922671189524234\\
27.47	0.00922671141577379\\
27.48	0.00922671093556142\\
27.49	0.00922671045459786\\
27.5	0.00922670997287564\\
27.51	0.0092267094903872\\
27.52	0.00922670900712487\\
27.53	0.00922670852308092\\
27.54	0.00922670803824749\\
27.55	0.00922670755261664\\
27.56	0.00922670706618031\\
27.57	0.00922670657893038\\
27.58	0.00922670609085858\\
27.59	0.00922670560195657\\
27.6	0.00922670511221591\\
27.61	0.00922670462162802\\
27.62	0.00922670413018426\\
27.63	0.00922670363787584\\
27.64	0.00922670314469389\\
27.65	0.00922670265062941\\
27.66	0.00922670215567331\\
27.67	0.00922670165981638\\
27.68	0.00922670116304928\\
27.69	0.00922670066536258\\
27.7	0.00922670016674671\\
27.71	0.009226699667192\\
27.72	0.00922669916668866\\
27.73	0.00922669866522677\\
27.74	0.00922669816279628\\
27.75	0.00922669765938705\\
27.76	0.00922669715498878\\
27.77	0.00922669664959106\\
27.78	0.00922669614318335\\
27.79	0.00922669563575499\\
27.8	0.00922669512729516\\
27.81	0.00922669461779295\\
27.82	0.00922669410723728\\
27.83	0.00922669359561697\\
27.84	0.00922669308292066\\
27.85	0.00922669256913689\\
27.86	0.00922669205425404\\
27.87	0.00922669153826035\\
27.88	0.00922669102114393\\
27.89	0.00922669050289273\\
27.9	0.00922668998349456\\
27.91	0.00922668946293709\\
27.92	0.00922668894120783\\
27.93	0.00922668841829413\\
27.94	0.00922668789418321\\
27.95	0.00922668736886212\\
27.96	0.00922668684231776\\
27.97	0.00922668631453688\\
27.98	0.00922668578550603\\
27.99	0.00922668525521165\\
28	0.00922668472364\\
28.01	0.00922668419077714\\
28.02	0.00922668365660903\\
28.03	0.00922668312112138\\
28.04	0.00922668258429981\\
28.05	0.00922668204612971\\
28.06	0.00922668150659631\\
28.07	0.00922668096568467\\
28.08	0.00922668042337967\\
28.09	0.00922667987966601\\
28.1	0.0092266793345282\\
28.11	0.00922667878795057\\
28.12	0.00922667823991727\\
28.13	0.00922667769041224\\
28.14	0.00922667713941926\\
28.15	0.00922667658692188\\
28.16	0.00922667603290349\\
28.17	0.00922667547734727\\
28.18	0.00922667492023618\\
28.19	0.00922667436155302\\
28.2	0.00922667380128034\\
28.21	0.00922667323940051\\
28.22	0.0092266726758957\\
28.23	0.00922667211074785\\
28.24	0.0092266715439387\\
28.25	0.00922667097544975\\
28.26	0.00922667040526232\\
28.27	0.00922666983335749\\
28.28	0.00922666925971611\\
28.29	0.00922666868431881\\
28.3	0.00922666810714601\\
28.31	0.00922666752817789\\
28.32	0.00922666694739437\\
28.33	0.00922666636477518\\
28.34	0.00922666578029979\\
28.35	0.00922666519394743\\
28.36	0.00922666460569709\\
28.37	0.00922666401552753\\
28.38	0.00922666342341722\\
28.39	0.00922666282934444\\
28.4	0.00922666223328716\\
28.41	0.00922666163522313\\
28.42	0.00922666103512984\\
28.43	0.00922666043298449\\
28.44	0.00922665982876406\\
28.45	0.00922665922244522\\
28.46	0.00922665861400441\\
28.47	0.00922665800341777\\
28.48	0.00922665739066117\\
28.49	0.00922665677571021\\
28.5	0.00922665615854021\\
28.51	0.0092266555391262\\
28.52	0.00922665491744293\\
28.53	0.00922665429346484\\
28.54	0.00922665366716611\\
28.55	0.00922665303852061\\
28.56	0.00922665240750189\\
28.57	0.00922665177408324\\
28.58	0.00922665113823761\\
28.59	0.00922665049993766\\
28.6	0.00922664985915573\\
28.61	0.00922664921586387\\
28.62	0.00922664857003377\\
28.63	0.00922664792163684\\
28.64	0.00922664727064414\\
28.65	0.00922664661702642\\
28.66	0.00922664596075409\\
28.67	0.00922664530179723\\
28.68	0.00922664464012559\\
28.69	0.00922664397570858\\
28.7	0.00922664330851525\\
28.71	0.00922664263851432\\
28.72	0.00922664196567416\\
28.73	0.00922664128996278\\
28.74	0.00922664061134784\\
28.75	0.00922663992979664\\
28.76	0.00922663924527612\\
28.77	0.00922663855775284\\
28.78	0.009226637867193\\
28.79	0.00922663717356245\\
28.8	0.00922663647682661\\
28.81	0.00922663577695058\\
28.82	0.00922663507389903\\
28.83	0.00922663436763627\\
28.84	0.00922663365812623\\
28.85	0.00922663294533241\\
28.86	0.00922663222921795\\
28.87	0.00922663150974557\\
28.88	0.00922663078687759\\
28.89	0.00922663006057594\\
28.9	0.00922662933080212\\
28.91	0.00922662859751724\\
28.92	0.00922662786068196\\
28.93	0.00922662712025656\\
28.94	0.00922662637620088\\
28.95	0.00922662562847433\\
28.96	0.0092266248770359\\
28.97	0.00922662412184416\\
28.98	0.00922662336285721\\
28.99	0.00922662260003277\\
29	0.00922662183332807\\
29.01	0.00922662106269991\\
29.02	0.00922662028810466\\
29.03	0.00922661950949824\\
29.04	0.00922661872683611\\
29.05	0.00922661794007328\\
29.06	0.00922661714916431\\
29.07	0.0092266163540633\\
29.08	0.0092266155547239\\
29.09	0.00922661475109928\\
29.1	0.00922661394314216\\
29.11	0.0092266131308048\\
29.12	0.00922661231403899\\
29.13	0.00922661149279604\\
29.14	0.00922661066702681\\
29.15	0.00922660983668167\\
29.16	0.00922660900171055\\
29.17	0.00922660816206288\\
29.18	0.00922660731768763\\
29.19	0.00922660646853329\\
29.2	0.00922660561454789\\
29.21	0.00922660475567897\\
29.22	0.00922660389187363\\
29.23	0.00922660302307847\\
29.24	0.00922660214923962\\
29.25	0.00922660127030277\\
29.26	0.00922660038621311\\
29.27	0.00922659949691538\\
29.28	0.00922659860235386\\
29.29	0.00922659770247237\\
29.3	0.00922659679721427\\
29.31	0.00922659588652246\\
29.32	0.00922659497033941\\
29.33	0.00922659404860712\\
29.34	0.00922659312126718\\
29.35	0.00922659218826073\\
29.36	0.00922659124952848\\
29.37	0.00922659030501071\\
29.38	0.00922658935464732\\
29.39	0.00922658839837776\\
29.4	0.00922658743614112\\
29.41	0.00922658646787608\\
29.42	0.00922658549352094\\
29.43	0.00922658451301366\\
29.44	0.00922658352629179\\
29.45	0.0092265825332926\\
29.46	0.00922658153395297\\
29.47	0.00922658052820952\\
29.48	0.00922657951599852\\
29.49	0.00922657849725598\\
29.5	0.00922657747191764\\
29.51	0.00922657643991899\\
29.52	0.00922657540119528\\
29.53	0.00922657435568157\\
29.54	0.00922657330331272\\
29.55	0.00922657224402342\\
29.56	0.00922657117774822\\
29.57	0.00922657010442157\\
29.58	0.00922656902397778\\
29.59	0.00922656793635108\\
29.6	0.00922656684147565\\
29.61	0.00922656573928561\\
29.62	0.00922656462971509\\
29.63	0.00922656351269818\\
29.64	0.00922656238816905\\
29.65	0.00922656125606189\\
29.66	0.00922656011631099\\
29.67	0.00922655896885074\\
29.68	0.00922655781361569\\
29.69	0.00922655665054054\\
29.7	0.0092265554795602\\
29.71	0.00922655430060985\\
29.72	0.00922655311362488\\
29.73	0.00922655191854104\\
29.74	0.0092265507152944\\
29.75	0.00922654950382142\\
29.76	0.00922654828405896\\
29.77	0.00922654705594438\\
29.78	0.00922654581941553\\
29.79	0.0092265445744108\\
29.8	0.00922654332086919\\
29.81	0.00922654205873035\\
29.82	0.0092265407879346\\
29.83	0.00922653950842304\\
29.84	0.00922653822013756\\
29.85	0.00922653692302087\\
29.86	0.00922653561701664\\
29.87	0.00922653430206949\\
29.88	0.00922653297812507\\
29.89	0.00922653164513012\\
29.9	0.00922653030303256\\
29.91	0.00922652895178153\\
29.92	0.00922652759132745\\
29.93	0.00922652622162216\\
29.94	0.00922652484261889\\
29.95	0.00922652345427243\\
29.96	0.00922652205653919\\
29.97	0.00922652064937722\\
29.98	0.00922651923274638\\
29.99	0.0092265178066084\\
30	0.00922651637092693\\
30.01	0.00922651492566772\\
30.02	0.00922651347079862\\
30.03	0.00922651200628976\\
30.04	0.00922651053211362\\
30.05	0.00922650904824514\\
30.06	0.00922650755466184\\
30.07	0.00922650605134393\\
30.08	0.00922650453827442\\
30.09	0.00922650301543928\\
30.1	0.00922650148282751\\
30.11	0.00922649994043132\\
30.12	0.00922649838824625\\
30.13	0.00922649682627131\\
30.14	0.00922649525450911\\
30.15	0.00922649367296603\\
30.16	0.00922649208165238\\
30.17	0.00922649048058252\\
30.18	0.00922648886977507\\
30.19	0.00922648724925306\\
30.2	0.0092264856190441\\
30.21	0.00922648397918054\\
30.22	0.00922648232969974\\
30.23	0.00922648067064415\\
30.24	0.00922647900206157\\
30.25	0.00922647732400538\\
30.26	0.00922647563653466\\
30.27	0.00922647393971453\\
30.28	0.00922647223361625\\
30.29	0.00922647051831753\\
30.3	0.00922646879390277\\
30.31	0.00922646706046324\\
30.32	0.00922646531809741\\
30.33	0.00922646356691118\\
30.34	0.00922646180701813\\
30.35	0.00922646003853982\\
30.36	0.00922645826160609\\
30.37	0.0092264564763553\\
30.38	0.00922645468293471\\
30.39	0.00922645288150073\\
30.4	0.00922645107221931\\
30.41	0.00922644925526617\\
30.42	0.00922644743082726\\
30.43	0.00922644559909903\\
30.44	0.00922644376028885\\
30.45	0.00922644191461534\\
30.46	0.00922644006230879\\
30.47	0.00922643820361154\\
30.48	0.00922643633877841\\
30.49	0.00922643446807711\\
30.5	0.00922643259178868\\
30.51	0.00922643071020794\\
30.52	0.00922642882364395\\
30.53	0.00922642693242051\\
30.54	0.00922642503687661\\
30.55	0.00922642313736698\\
30.56	0.00922642123426259\\
30.57	0.00922641932795118\\
30.58	0.00922641741883784\\
30.59	0.00922641550734557\\
30.6	0.00922641359391584\\
30.61	0.00922641167900926\\
30.62	0.00922640976310612\\
30.63	0.00922640784646683\\
30.64	0.00922640592909104\\
30.65	0.00922640401097839\\
30.66	0.00922640209212853\\
30.67	0.0092264001725411\\
30.68	0.00922639825221576\\
30.69	0.00922639633115214\\
30.7	0.00922639440934989\\
30.71	0.00922639248680865\\
30.72	0.00922639056352807\\
30.73	0.0092263886395078\\
30.74	0.00922638671474747\\
30.75	0.00922638478924672\\
30.76	0.00922638286300522\\
30.77	0.00922638093602258\\
30.78	0.00922637900829847\\
30.79	0.00922637707983251\\
30.8	0.00922637515062436\\
30.81	0.00922637322067365\\
30.82	0.00922637128998003\\
30.83	0.00922636935854313\\
30.84	0.00922636742636261\\
30.85	0.00922636549343809\\
30.86	0.00922636355976922\\
30.87	0.00922636162535565\\
30.88	0.009226359690197\\
30.89	0.00922635775429292\\
30.9	0.00922635581764306\\
30.91	0.00922635388024704\\
30.92	0.00922635194210451\\
30.93	0.00922635000321511\\
30.94	0.00922634806357847\\
30.95	0.00922634612319424\\
30.96	0.00922634418206205\\
30.97	0.00922634224018154\\
30.98	0.00922634029755235\\
30.99	0.00922633835417411\\
31	0.00922633641004646\\
31.01	0.00922633446516904\\
31.02	0.00922633251954148\\
31.03	0.00922633057316343\\
31.04	0.00922632862603451\\
31.05	0.00922632667815436\\
31.06	0.00922632472952263\\
31.07	0.00922632278013893\\
31.08	0.00922632083000292\\
31.09	0.00922631887911421\\
31.1	0.00922631692747245\\
31.11	0.00922631497507728\\
31.12	0.00922631302192831\\
31.13	0.0092263110680252\\
31.14	0.00922630911336757\\
31.15	0.00922630715795505\\
31.16	0.00922630520178728\\
31.17	0.00922630324486389\\
31.18	0.00922630128718451\\
31.19	0.00922629932874878\\
31.2	0.00922629736955632\\
31.21	0.00922629540960677\\
31.22	0.00922629344889976\\
31.23	0.00922629148743491\\
31.24	0.00922628952521187\\
31.25	0.00922628756223025\\
31.26	0.0092262855984897\\
31.27	0.00922628363398984\\
31.28	0.00922628166873029\\
31.29	0.0092262797027107\\
31.3	0.00922627773593068\\
31.31	0.00922627576838987\\
31.32	0.0092262738000879\\
31.33	0.00922627183102439\\
31.34	0.00922626986119897\\
31.35	0.00922626789061127\\
31.36	0.00922626591926092\\
31.37	0.00922626394714753\\
31.38	0.00922626197427076\\
31.39	0.0092262600006302\\
31.4	0.0092262580262255\\
31.41	0.00922625605105628\\
31.42	0.00922625407512216\\
31.43	0.00922625209842278\\
31.44	0.00922625012095775\\
31.45	0.0092262481427267\\
31.46	0.00922624616372926\\
31.47	0.00922624418396504\\
31.48	0.00922624220343368\\
31.49	0.0092262402221348\\
31.5	0.00922623824006802\\
31.51	0.00922623625723297\\
31.52	0.00922623427362926\\
31.53	0.00922623228925652\\
31.54	0.00922623030411438\\
31.55	0.00922622831820245\\
31.56	0.00922622633152035\\
31.57	0.00922622434406772\\
31.58	0.00922622235584416\\
31.59	0.00922622036684931\\
31.6	0.00922621837708277\\
31.61	0.00922621638654418\\
31.62	0.00922621439523315\\
31.63	0.0092262124031493\\
31.64	0.00922621041029226\\
31.65	0.00922620841666164\\
31.66	0.00922620642225706\\
31.67	0.00922620442707813\\
31.68	0.00922620243112449\\
31.69	0.00922620043439574\\
31.7	0.0092261984368915\\
31.71	0.0092261964386114\\
31.72	0.00922619443955505\\
31.73	0.00922619243972206\\
31.74	0.00922619043911205\\
31.75	0.00922618843772465\\
31.76	0.00922618643555946\\
31.77	0.0092261844326161\\
31.78	0.00922618242889419\\
31.79	0.00922618042439335\\
31.8	0.00922617841911318\\
31.81	0.0092261764130533\\
31.82	0.00922617440621334\\
31.83	0.00922617239859289\\
31.84	0.00922617039019159\\
31.85	0.00922616838100903\\
31.86	0.00922616637104483\\
31.87	0.00922616436029861\\
31.88	0.00922616234876998\\
31.89	0.00922616033645855\\
31.9	0.00922615832336394\\
31.91	0.00922615630948575\\
31.92	0.00922615429482361\\
31.93	0.0092261522793771\\
31.94	0.00922615026314586\\
31.95	0.00922614824612949\\
31.96	0.00922614622832761\\
31.97	0.00922614420973981\\
31.98	0.00922614219036571\\
31.99	0.00922614017020493\\
32	0.00922613814925707\\
32.01	0.00922613612752173\\
32.02	0.00922613410499854\\
32.03	0.0092261320816871\\
32.04	0.00922613005758701\\
32.05	0.00922612803269788\\
32.06	0.00922612600701933\\
32.07	0.00922612398055095\\
32.08	0.00922612195329236\\
32.09	0.00922611992524316\\
32.1	0.00922611789640296\\
32.11	0.00922611586677136\\
32.12	0.00922611383634797\\
32.13	0.00922611180513241\\
32.14	0.00922610977312426\\
32.15	0.00922610774032314\\
32.16	0.00922610570672864\\
32.17	0.00922610367234039\\
32.18	0.00922610163715797\\
32.19	0.009226099601181\\
32.2	0.00922609756440908\\
32.21	0.0092260955268418\\
32.22	0.00922609348847878\\
32.23	0.00922609144931961\\
32.24	0.0092260894093639\\
32.25	0.00922608736861125\\
32.26	0.00922608532706126\\
32.27	0.00922608328471354\\
32.28	0.00922608124156767\\
32.29	0.00922607919762328\\
32.3	0.00922607715287995\\
32.31	0.00922607510733728\\
32.32	0.00922607306099488\\
32.33	0.00922607101385234\\
32.34	0.00922606896590927\\
32.35	0.00922606691716526\\
32.36	0.00922606486761991\\
32.37	0.00922606281727282\\
32.38	0.00922606076612359\\
32.39	0.00922605871417182\\
32.4	0.0092260566614171\\
32.41	0.00922605460785903\\
32.42	0.00922605255349721\\
32.43	0.00922605049833122\\
32.44	0.00922604844236069\\
32.45	0.00922604638558518\\
32.46	0.00922604432800431\\
32.47	0.00922604226961767\\
32.48	0.00922604021042484\\
32.49	0.00922603815042544\\
32.5	0.00922603608961904\\
32.51	0.00922603402800525\\
32.52	0.00922603196558366\\
32.53	0.00922602990235386\\
32.54	0.00922602783831544\\
32.55	0.00922602577346801\\
32.56	0.00922602370781114\\
32.57	0.00922602164134444\\
32.58	0.00922601957406749\\
32.59	0.00922601750597989\\
32.6	0.00922601543708123\\
32.61	0.0092260133673711\\
32.62	0.00922601129684908\\
32.63	0.00922600922551478\\
32.64	0.00922600715336777\\
32.65	0.00922600508040765\\
32.66	0.00922600300663401\\
32.67	0.00922600093204645\\
32.68	0.00922599885664453\\
32.69	0.00922599678042786\\
32.7	0.00922599470339603\\
32.71	0.00922599262554862\\
32.72	0.00922599054688521\\
32.73	0.00922598846740541\\
32.74	0.00922598638710878\\
32.75	0.00922598430599492\\
32.76	0.00922598222406342\\
32.77	0.00922598014131386\\
32.78	0.00922597805774582\\
32.79	0.0092259759733589\\
32.8	0.00922597388815268\\
32.81	0.00922597180212673\\
32.82	0.00922596971528066\\
32.83	0.00922596762761403\\
32.84	0.00922596553912644\\
32.85	0.00922596344981747\\
32.86	0.00922596135968669\\
32.87	0.0092259592687337\\
32.88	0.00922595717695807\\
32.89	0.00922595508435939\\
32.9	0.00922595299093724\\
32.91	0.00922595089669121\\
32.92	0.00922594880162086\\
32.93	0.00922594670572578\\
32.94	0.00922594460900556\\
32.95	0.00922594251145977\\
32.96	0.009225940413088\\
32.97	0.00922593831388981\\
32.98	0.00922593621386481\\
32.99	0.00922593411301254\\
33	0.00922593201133261\\
33.01	0.00922592990882459\\
33.02	0.00922592780548805\\
33.03	0.00922592570132257\\
33.04	0.00922592359632773\\
33.05	0.00922592149050311\\
33.06	0.00922591938384828\\
33.07	0.00922591727636282\\
33.08	0.00922591516804631\\
33.09	0.00922591305889831\\
33.1	0.00922591094891841\\
33.11	0.00922590883810619\\
33.12	0.00922590672646121\\
33.13	0.00922590461398305\\
33.14	0.00922590250067128\\
33.15	0.00922590038652548\\
33.16	0.00922589827154521\\
33.17	0.00922589615573007\\
33.18	0.00922589403907961\\
33.19	0.0092258919215934\\
33.2	0.00922588980327103\\
33.21	0.00922588768411206\\
33.22	0.00922588556411605\\
33.23	0.0092258834432826\\
33.24	0.00922588132161125\\
33.25	0.00922587919910159\\
33.26	0.00922587707575319\\
33.27	0.00922587495156561\\
33.28	0.00922587282653842\\
33.29	0.00922587070067119\\
33.3	0.00922586857396349\\
33.31	0.0092258664464149\\
33.32	0.00922586431802497\\
33.33	0.00922586218879327\\
33.34	0.00922586005871937\\
33.35	0.00922585792780284\\
33.36	0.00922585579604325\\
33.37	0.00922585366344015\\
33.38	0.00922585152999313\\
33.39	0.00922584939570173\\
33.4	0.00922584726056553\\
33.41	0.00922584512458409\\
33.42	0.00922584298775698\\
33.43	0.00922584085008376\\
33.44	0.00922583871156399\\
33.45	0.00922583657219724\\
33.46	0.00922583443198307\\
33.47	0.00922583229092104\\
33.48	0.00922583014901072\\
33.49	0.00922582800625167\\
33.5	0.00922582586264344\\
33.51	0.00922582371818561\\
33.52	0.00922582157287773\\
33.53	0.00922581942671936\\
33.54	0.00922581727971006\\
33.55	0.00922581513184939\\
33.56	0.00922581298313692\\
33.57	0.0092258108335722\\
33.58	0.00922580868315479\\
33.59	0.00922580653188424\\
33.6	0.00922580437976013\\
33.61	0.00922580222678199\\
33.62	0.0092258000729494\\
33.63	0.00922579791826191\\
33.64	0.00922579576271908\\
33.65	0.00922579360632046\\
33.66	0.0092257914490656\\
33.67	0.00922578929095407\\
33.68	0.00922578713198542\\
33.69	0.00922578497215921\\
33.7	0.00922578281147498\\
33.71	0.0092257806499323\\
33.72	0.00922577848753072\\
33.73	0.00922577632426978\\
33.74	0.00922577416014905\\
33.75	0.00922577199516808\\
33.76	0.00922576982932642\\
33.77	0.00922576766262363\\
33.78	0.00922576549505924\\
33.79	0.00922576332663282\\
33.8	0.00922576115734392\\
33.81	0.00922575898719208\\
33.82	0.00922575681617687\\
33.83	0.00922575464429781\\
33.84	0.00922575247155448\\
33.85	0.00922575029794642\\
33.86	0.00922574812347316\\
33.87	0.00922574594813428\\
33.88	0.0092257437719293\\
33.89	0.00922574159485779\\
33.9	0.00922573941691928\\
33.91	0.00922573723811333\\
33.92	0.00922573505843948\\
33.93	0.00922573287789728\\
33.94	0.00922573069648627\\
33.95	0.00922572851420601\\
33.96	0.00922572633105602\\
33.97	0.00922572414703587\\
33.98	0.00922572196214509\\
33.99	0.00922571977638323\\
34	0.00922571758974984\\
34.01	0.00922571540224445\\
34.02	0.0092257132138666\\
34.03	0.00922571102461586\\
34.04	0.00922570883449174\\
34.05	0.0092257066434938\\
34.06	0.00922570445162158\\
34.07	0.00922570225887462\\
34.08	0.00922570006525245\\
34.09	0.00922569787075463\\
34.1	0.00922569567538069\\
34.11	0.00922569347913017\\
34.12	0.00922569128200261\\
34.13	0.00922568908399755\\
34.14	0.00922568688511452\\
34.15	0.00922568468535307\\
34.16	0.00922568248471274\\
34.17	0.00922568028319305\\
34.18	0.00922567808079356\\
34.19	0.00922567587751378\\
34.2	0.00922567367335327\\
34.21	0.00922567146831156\\
34.22	0.00922566926238818\\
34.23	0.00922566705558267\\
34.24	0.00922566484789456\\
34.25	0.0092256626393234\\
34.26	0.0092256604298687\\
34.27	0.00922565821953\\
34.28	0.00922565600830686\\
34.29	0.00922565379619877\\
34.3	0.0092256515832053\\
34.31	0.00922564936932596\\
34.32	0.00922564715456029\\
34.33	0.00922564493890781\\
34.34	0.00922564272236808\\
34.35	0.0092256405049406\\
34.36	0.00922563828662491\\
34.37	0.00922563606742055\\
34.38	0.00922563384732703\\
34.39	0.00922563162634391\\
34.4	0.00922562940447069\\
34.41	0.0092256271817069\\
34.42	0.00922562495805209\\
34.43	0.00922562273350576\\
34.44	0.00922562050806746\\
34.45	0.00922561828173671\\
34.46	0.00922561605451303\\
34.47	0.00922561382639595\\
34.48	0.009225611597385\\
34.49	0.00922560936747969\\
34.5	0.00922560713667957\\
34.51	0.00922560490498414\\
34.52	0.00922560267239294\\
34.53	0.00922560043890549\\
34.54	0.00922559820452131\\
34.55	0.00922559596923992\\
34.56	0.00922559373306085\\
34.57	0.00922559149598363\\
34.58	0.00922558925800776\\
34.59	0.00922558701913277\\
34.6	0.00922558477935819\\
34.61	0.00922558253868353\\
34.62	0.00922558029710832\\
34.63	0.00922557805463207\\
34.64	0.00922557581125431\\
34.65	0.00922557356697455\\
34.66	0.0092255713217923\\
34.67	0.0092255690757071\\
34.68	0.00922556682871845\\
34.69	0.00922556458082588\\
34.7	0.0092255623320289\\
34.71	0.00922556008232703\\
34.72	0.00922555783171978\\
34.73	0.00922555558020668\\
34.74	0.00922555332778722\\
34.75	0.00922555107446094\\
34.76	0.00922554882022735\\
34.77	0.00922554656508595\\
34.78	0.00922554430903626\\
34.79	0.0092255420520778\\
34.8	0.00922553979421008\\
34.81	0.00922553753543262\\
34.82	0.00922553527574491\\
34.83	0.00922553301514648\\
34.84	0.00922553075363684\\
34.85	0.0092255284912155\\
34.86	0.00922552622788197\\
34.87	0.00922552396363575\\
34.88	0.00922552169847637\\
34.89	0.00922551943240332\\
34.9	0.00922551716541612\\
34.91	0.00922551489751427\\
34.92	0.00922551262869728\\
34.93	0.00922551035896467\\
34.94	0.00922550808831593\\
34.95	0.00922550581675058\\
34.96	0.00922550354426812\\
34.97	0.00922550127086806\\
34.98	0.00922549899654991\\
34.99	0.00922549672131316\\
35	0.00922549444515732\\
35.01	0.0092254921680819\\
35.02	0.0092254898900864\\
35.03	0.00922548761117033\\
35.04	0.00922548533133318\\
35.05	0.00922548305057447\\
35.06	0.00922548076889368\\
35.07	0.00922547848629034\\
35.08	0.00922547620276393\\
35.09	0.00922547391831395\\
35.1	0.00922547163293992\\
35.11	0.00922546934664131\\
35.12	0.00922546705941765\\
35.13	0.00922546477126844\\
35.14	0.00922546248219315\\
35.15	0.00922546019219129\\
35.16	0.00922545790126237\\
35.17	0.00922545560940588\\
35.18	0.00922545331662132\\
35.19	0.00922545102290819\\
35.2	0.00922544872826597\\
35.21	0.00922544643269417\\
35.22	0.00922544413619228\\
35.23	0.0092254418387598\\
35.24	0.00922543954039622\\
35.25	0.00922543724110104\\
35.26	0.00922543494087374\\
35.27	0.00922543263971383\\
35.28	0.0092254303376208\\
35.29	0.00922542803459414\\
35.3	0.00922542573063334\\
35.31	0.00922542342573788\\
35.32	0.00922542111990728\\
35.33	0.00922541881314101\\
35.34	0.00922541650543856\\
35.35	0.00922541419679943\\
35.36	0.00922541188722311\\
35.37	0.00922540957670907\\
35.38	0.00922540726525682\\
35.39	0.00922540495286584\\
35.4	0.00922540263953562\\
35.41	0.00922540032526564\\
35.42	0.0092253980100554\\
35.43	0.00922539569390437\\
35.44	0.00922539337681205\\
35.45	0.00922539105877792\\
35.46	0.00922538873980147\\
35.47	0.00922538641988217\\
35.48	0.00922538409901953\\
35.49	0.00922538177721301\\
35.5	0.0092253794544621\\
35.51	0.00922537713076629\\
35.52	0.00922537480612505\\
35.53	0.00922537248053788\\
35.54	0.00922537015400425\\
35.55	0.00922536782652363\\
35.56	0.00922536549809552\\
35.57	0.00922536316871939\\
35.58	0.00922536083839473\\
35.59	0.00922535850712102\\
35.6	0.00922535617489772\\
35.61	0.00922535384172432\\
35.62	0.0092253515076003\\
35.63	0.00922534917252514\\
35.64	0.0092253468364983\\
35.65	0.00922534449951928\\
35.66	0.00922534216158754\\
35.67	0.00922533982270257\\
35.68	0.00922533748286383\\
35.69	0.00922533514207081\\
35.7	0.00922533280032297\\
35.71	0.00922533045761979\\
35.72	0.00922532811396075\\
35.73	0.00922532576934531\\
35.74	0.00922532342377295\\
35.75	0.00922532107724315\\
35.76	0.00922531872975537\\
35.77	0.00922531638130908\\
35.78	0.00922531403190376\\
35.79	0.00922531168153888\\
35.8	0.0092253093302139\\
35.81	0.0092253069779283\\
35.82	0.00922530462468154\\
35.83	0.00922530227047309\\
35.84	0.00922529991530243\\
35.85	0.00922529755916902\\
35.86	0.00922529520207232\\
35.87	0.00922529284401181\\
35.88	0.00922529048498694\\
35.89	0.00922528812499719\\
35.9	0.00922528576404202\\
35.91	0.00922528340212089\\
35.92	0.00922528103923328\\
35.93	0.00922527867537863\\
35.94	0.00922527631055643\\
35.95	0.00922527394476612\\
35.96	0.00922527157800718\\
35.97	0.00922526921027906\\
35.98	0.00922526684158123\\
35.99	0.00922526447191314\\
36	0.00922526210127426\\
36.01	0.00922525972966405\\
36.02	0.00922525735708197\\
36.03	0.00922525498352747\\
36.04	0.00922525260900002\\
36.05	0.00922525023349908\\
36.06	0.00922524785702409\\
36.07	0.00922524547957453\\
36.08	0.00922524310114984\\
36.09	0.00922524072174949\\
36.1	0.00922523834137292\\
36.11	0.0092252359600196\\
36.12	0.00922523357768897\\
36.13	0.0092252311943805\\
36.14	0.00922522881009364\\
36.15	0.00922522642482784\\
36.16	0.00922522403858255\\
36.17	0.00922522165135723\\
36.18	0.00922521926315133\\
36.19	0.0092252168739643\\
36.2	0.00922521448379559\\
36.21	0.00922521209264465\\
36.22	0.00922520970051094\\
36.23	0.0092252073073939\\
36.24	0.00922520491329298\\
36.25	0.00922520251820763\\
36.26	0.0092252001221373\\
36.27	0.00922519772508144\\
36.28	0.00922519532703949\\
36.29	0.0092251929280109\\
36.3	0.00922519052799512\\
36.31	0.00922518812699159\\
36.32	0.00922518572499976\\
36.33	0.00922518332201908\\
36.34	0.00922518091804898\\
36.35	0.00922517851308891\\
36.36	0.00922517610713832\\
36.37	0.00922517370019665\\
36.38	0.00922517129226333\\
36.39	0.00922516888333782\\
36.4	0.00922516647341956\\
36.41	0.00922516406250797\\
36.42	0.00922516165060251\\
36.43	0.00922515923770262\\
36.44	0.00922515682380774\\
36.45	0.00922515440891729\\
36.46	0.00922515199303073\\
36.47	0.00922514957614749\\
36.48	0.009225147158267\\
36.49	0.00922514473938872\\
36.5	0.00922514231951206\\
36.51	0.00922513989863647\\
36.52	0.00922513747676138\\
36.53	0.00922513505388623\\
36.54	0.00922513263001045\\
36.55	0.00922513020513348\\
36.56	0.00922512777925475\\
36.57	0.00922512535237369\\
36.58	0.00922512292448974\\
36.59	0.00922512049560232\\
36.6	0.00922511806571087\\
36.61	0.00922511563481483\\
36.62	0.00922511320291361\\
36.63	0.00922511077000665\\
36.64	0.00922510833609339\\
36.65	0.00922510590117324\\
36.66	0.00922510346524564\\
36.67	0.00922510102831001\\
36.68	0.00922509859036579\\
36.69	0.0092250961514124\\
36.7	0.00922509371144926\\
36.71	0.00922509127047581\\
36.72	0.00922508882849146\\
36.73	0.00922508638549564\\
36.74	0.00922508394148779\\
36.75	0.00922508149646731\\
36.76	0.00922507905043364\\
36.77	0.0092250766033862\\
36.78	0.0092250741553244\\
36.79	0.00922507170624767\\
36.8	0.00922506925615544\\
36.81	0.00922506680504712\\
36.82	0.00922506435292214\\
36.83	0.0092250618997799\\
36.84	0.00922505944561984\\
36.85	0.00922505699044138\\
36.86	0.00922505453424392\\
36.87	0.00922505207702688\\
36.88	0.0092250496187897\\
36.89	0.00922504715953177\\
36.9	0.00922504469925252\\
36.91	0.00922504223795136\\
36.92	0.00922503977562772\\
36.93	0.00922503731228099\\
36.94	0.0092250348479106\\
36.95	0.00922503238251596\\
36.96	0.00922502991609648\\
36.97	0.00922502744865158\\
36.98	0.00922502498018067\\
36.99	0.00922502251068316\\
37	0.00922502004015845\\
37.01	0.00922501756860596\\
37.02	0.00922501509602511\\
37.03	0.00922501262241529\\
37.04	0.00922501014777592\\
37.05	0.00922500767210641\\
37.06	0.00922500519540616\\
37.07	0.00922500271767458\\
37.08	0.00922500023891108\\
37.09	0.00922499775911507\\
37.1	0.00922499527828594\\
37.11	0.00922499279642311\\
37.12	0.00922499031352598\\
37.13	0.00922498782959395\\
37.14	0.00922498534462642\\
37.15	0.00922498285862281\\
37.16	0.00922498037158251\\
37.17	0.00922497788350492\\
37.18	0.00922497539438944\\
37.19	0.00922497290423548\\
37.2	0.00922497041304244\\
37.21	0.00922496792080972\\
37.22	0.0092249654275367\\
37.23	0.0092249629332228\\
37.24	0.00922496043786742\\
37.25	0.00922495794146993\\
37.26	0.00922495544402976\\
37.27	0.00922495294554629\\
37.28	0.00922495044601891\\
37.29	0.00922494794544704\\
37.3	0.00922494544383004\\
37.31	0.00922494294116733\\
37.32	0.0092249404374583\\
37.33	0.00922493793270234\\
37.34	0.00922493542689883\\
37.35	0.00922493292004718\\
37.36	0.00922493041214678\\
37.37	0.00922492790319702\\
37.38	0.00922492539319728\\
37.39	0.00922492288214695\\
37.4	0.00922492037004544\\
37.41	0.00922491785689211\\
37.42	0.00922491534268637\\
37.43	0.00922491282742761\\
37.44	0.00922491031111519\\
37.45	0.00922490779374852\\
37.46	0.00922490527532699\\
37.47	0.00922490275584997\\
37.48	0.00922490023531685\\
37.49	0.00922489771372701\\
37.5	0.00922489519107984\\
37.51	0.00922489266737473\\
37.52	0.00922489014261105\\
37.53	0.00922488761678818\\
37.54	0.00922488508990551\\
37.55	0.00922488256196243\\
37.56	0.0092248800329583\\
37.57	0.00922487750289251\\
37.58	0.00922487497176444\\
37.59	0.00922487243957346\\
37.6	0.00922486990631896\\
37.61	0.00922486737200032\\
37.62	0.0092248648366169\\
37.63	0.00922486230016809\\
37.64	0.00922485976265327\\
37.65	0.0092248572240718\\
37.66	0.00922485468442306\\
37.67	0.00922485214370643\\
37.68	0.00922484960192128\\
37.69	0.00922484705906698\\
37.7	0.00922484451514291\\
37.71	0.00922484197014844\\
37.72	0.00922483942408294\\
37.73	0.00922483687694577\\
37.74	0.00922483432873632\\
37.75	0.00922483177945394\\
37.76	0.00922482922909802\\
37.77	0.00922482667766791\\
37.78	0.00922482412516299\\
37.79	0.00922482157158262\\
37.8	0.00922481901692617\\
37.81	0.009224816461193\\
37.82	0.00922481390438249\\
37.83	0.00922481134649399\\
37.84	0.00922480878752687\\
37.85	0.0092248062274805\\
37.86	0.00922480366635423\\
37.87	0.00922480110414743\\
37.88	0.00922479854085947\\
37.89	0.0092247959764897\\
37.9	0.00922479341103749\\
37.91	0.00922479084450219\\
37.92	0.00922478827688317\\
37.93	0.00922478570817978\\
37.94	0.00922478313839139\\
37.95	0.00922478056751735\\
37.96	0.00922477799555701\\
37.97	0.00922477542250974\\
37.98	0.0092247728483749\\
37.99	0.00922477027315183\\
38	0.00922476769683989\\
38.01	0.00922476511943845\\
38.02	0.00922476254094684\\
38.03	0.00922475996136443\\
38.04	0.00922475738069057\\
38.05	0.00922475479892461\\
38.06	0.0092247522160659\\
38.07	0.00922474963211379\\
38.08	0.00922474704706763\\
38.09	0.00922474446092678\\
38.1	0.00922474187369057\\
38.11	0.00922473928535838\\
38.12	0.00922473669592952\\
38.13	0.00922473410540337\\
38.14	0.00922473151377926\\
38.15	0.00922472892105654\\
38.16	0.00922472632723455\\
38.17	0.00922472373231264\\
38.18	0.00922472113629016\\
38.19	0.00922471853916645\\
38.2	0.00922471594094085\\
38.21	0.0092247133416127\\
38.22	0.00922471074118135\\
38.23	0.00922470813964614\\
38.24	0.00922470553700641\\
38.25	0.00922470293326149\\
38.26	0.00922470032841074\\
38.27	0.00922469772245348\\
38.28	0.00922469511538907\\
38.29	0.00922469250721682\\
38.3	0.00922468989793609\\
38.31	0.00922468728754621\\
38.32	0.00922468467604651\\
38.33	0.00922468206343633\\
38.34	0.009224679449715\\
38.35	0.00922467683488187\\
38.36	0.00922467421893625\\
38.37	0.0092246716018775\\
38.38	0.00922466898370493\\
38.39	0.00922466636441789\\
38.4	0.00922466374401569\\
38.41	0.00922466112249768\\
38.42	0.00922465849986317\\
38.43	0.00922465587611151\\
38.44	0.00922465325124202\\
38.45	0.00922465062525403\\
38.46	0.00922464799814686\\
38.47	0.00922464536991985\\
38.48	0.00922464274057231\\
38.49	0.00922464011010358\\
38.5	0.00922463747851298\\
38.51	0.00922463484579984\\
38.52	0.00922463221196347\\
38.53	0.00922462957700321\\
38.54	0.00922462694091837\\
38.55	0.00922462430370827\\
38.56	0.00922462166537224\\
38.57	0.0092246190259096\\
38.58	0.00922461638531967\\
38.59	0.00922461374360176\\
38.6	0.00922461110075521\\
38.61	0.00922460845677932\\
38.62	0.00922460581167341\\
38.63	0.0092246031654368\\
38.64	0.00922460051806881\\
38.65	0.00922459786956875\\
38.66	0.00922459521993594\\
38.67	0.00922459256916969\\
38.68	0.00922458991726931\\
38.69	0.00922458726423413\\
38.7	0.00922458461006344\\
38.71	0.00922458195475657\\
38.72	0.00922457929831282\\
38.73	0.00922457664073152\\
38.74	0.00922457398201195\\
38.75	0.00922457132215345\\
38.76	0.0092245686611553\\
38.77	0.00922456599901683\\
38.78	0.00922456333573734\\
38.79	0.00922456067131614\\
38.8	0.00922455800575253\\
38.81	0.00922455533904582\\
38.82	0.00922455267119531\\
38.83	0.00922455000220031\\
38.84	0.00922454733206013\\
38.85	0.00922454466077406\\
38.86	0.00922454198834141\\
38.87	0.00922453931476148\\
38.88	0.00922453664003357\\
38.89	0.00922453396415699\\
38.9	0.00922453128713103\\
38.91	0.00922452860895499\\
38.92	0.00922452592962817\\
38.93	0.00922452324914986\\
38.94	0.00922452056751938\\
38.95	0.009224517884736\\
38.96	0.00922451520079904\\
38.97	0.00922451251570778\\
38.98	0.00922450982946151\\
38.99	0.00922450714205955\\
39	0.00922450445350117\\
39.01	0.00922450176378566\\
39.02	0.00922449907291234\\
39.03	0.00922449638088047\\
39.04	0.00922449368768936\\
39.05	0.0092244909933383\\
39.06	0.00922448829782657\\
39.07	0.00922448560115346\\
39.08	0.00922448290331826\\
39.09	0.00922448020432027\\
39.1	0.00922447750415876\\
39.11	0.00922447480283303\\
39.12	0.00922447210034235\\
39.13	0.00922446939668602\\
39.14	0.00922446669186332\\
39.15	0.00922446398587353\\
39.16	0.00922446127871593\\
39.17	0.00922445857038981\\
39.18	0.00922445586089445\\
39.19	0.00922445315022914\\
39.2	0.00922445043839314\\
39.21	0.00922444772538575\\
39.22	0.00922444501120623\\
39.23	0.00922444229585388\\
39.24	0.00922443957932796\\
39.25	0.00922443686162775\\
39.26	0.00922443414275254\\
39.27	0.00922443142270159\\
39.28	0.00922442870147418\\
39.29	0.00922442597906959\\
39.3	0.00922442325548708\\
39.31	0.00922442053072594\\
39.32	0.00922441780478543\\
39.33	0.00922441507766483\\
39.34	0.00922441234936341\\
39.35	0.00922440961988044\\
39.36	0.00922440688921519\\
39.37	0.00922440415736692\\
39.38	0.0092244014243349\\
39.39	0.00922439869011842\\
39.4	0.00922439595471672\\
39.41	0.00922439321812908\\
39.42	0.00922439048035476\\
39.43	0.00922438774139302\\
39.44	0.00922438500124314\\
39.45	0.00922438225990437\\
39.46	0.00922437951737599\\
39.47	0.00922437677365724\\
39.48	0.0092243740287474\\
39.49	0.00922437128264572\\
39.5	0.00922436853535146\\
39.51	0.00922436578686388\\
39.52	0.00922436303718224\\
39.53	0.00922436028630581\\
39.54	0.00922435753423383\\
39.55	0.00922435478096557\\
39.56	0.00922435202650027\\
39.57	0.0092243492708372\\
39.58	0.00922434651397561\\
39.59	0.00922434375591475\\
39.6	0.00922434099665388\\
39.61	0.00922433823619225\\
39.62	0.00922433547452911\\
39.63	0.00922433271166371\\
39.64	0.0092243299475953\\
39.65	0.00922432718232313\\
39.66	0.00922432441584646\\
39.67	0.00922432164816452\\
39.68	0.00922431887927657\\
39.69	0.00922431610918185\\
39.7	0.00922431333787961\\
39.71	0.0092243105653691\\
39.72	0.00922430779164955\\
39.73	0.00922430501672022\\
39.74	0.00922430224058034\\
39.75	0.00922429946322916\\
39.76	0.00922429668466593\\
39.77	0.00922429390488987\\
39.78	0.00922429112390023\\
39.79	0.00922428834169626\\
39.8	0.00922428555827719\\
39.81	0.00922428277364225\\
39.82	0.00922427998779069\\
39.83	0.00922427720072174\\
39.84	0.00922427441243463\\
39.85	0.00922427162292861\\
39.86	0.0092242688322029\\
39.87	0.00922426604025675\\
39.88	0.00922426324708937\\
39.89	0.00922426045270002\\
39.9	0.00922425765708791\\
39.91	0.00922425486025227\\
39.92	0.00922425206219234\\
39.93	0.00922424926290734\\
39.94	0.00922424646239651\\
39.95	0.00922424366065907\\
39.96	0.00922424085769425\\
39.97	0.00922423805350127\\
39.98	0.00922423524807936\\
39.99	0.00922423244142775\\
40	0.00922422963354564\\
40.01	0.00922422682443228\\
};
\addplot [color=blue,solid,forget plot]
  table[row sep=crcr]{%
40.01	0.00922422682443228\\
40.02	0.00922422401408688\\
40.03	0.00922422120250866\\
40.04	0.00922421838969685\\
40.05	0.00922421557565065\\
40.06	0.0092242127603693\\
40.07	0.009224209943852\\
40.08	0.00922420712609799\\
40.09	0.00922420430710646\\
40.1	0.00922420148687665\\
40.11	0.00922419866540775\\
40.12	0.00922419584269899\\
40.13	0.00922419301874959\\
40.14	0.00922419019355875\\
40.15	0.00922418736712567\\
40.16	0.00922418453944959\\
40.17	0.0092241817105297\\
40.18	0.00922417888036521\\
40.19	0.00922417604895534\\
40.2	0.00922417321629927\\
40.21	0.00922417038239625\\
40.22	0.00922416754724544\\
40.23	0.00922416471084608\\
40.24	0.00922416187319735\\
40.25	0.00922415903429847\\
40.26	0.00922415619414863\\
40.27	0.00922415335274704\\
40.28	0.00922415051009289\\
40.29	0.0092241476661854\\
40.3	0.00922414482102374\\
40.31	0.00922414197460713\\
40.32	0.00922413912693476\\
40.33	0.00922413627800583\\
40.34	0.00922413342781952\\
40.35	0.00922413057637504\\
40.36	0.00922412772367158\\
40.37	0.00922412486970834\\
40.38	0.00922412201448449\\
40.39	0.00922411915799924\\
40.4	0.00922411630025177\\
40.41	0.00922411344124127\\
40.42	0.00922411058096693\\
40.43	0.00922410771942794\\
40.44	0.00922410485662348\\
40.45	0.00922410199255273\\
40.46	0.00922409912721489\\
40.47	0.00922409626060914\\
40.48	0.00922409339273465\\
40.49	0.00922409052359061\\
40.5	0.0092240876531762\\
40.51	0.00922408478149059\\
40.52	0.00922408190853297\\
40.53	0.00922407903430252\\
40.54	0.0092240761587984\\
40.55	0.0092240732820198\\
40.56	0.00922407040396589\\
40.57	0.00922406752463585\\
40.58	0.00922406464402883\\
40.59	0.00922406176214402\\
40.6	0.00922405887898059\\
40.61	0.00922405599453771\\
40.62	0.00922405310881454\\
40.63	0.00922405022181025\\
40.64	0.009224047333524\\
40.65	0.00922404444395497\\
40.66	0.00922404155310232\\
40.67	0.0092240386609652\\
40.68	0.00922403576754279\\
40.69	0.00922403287283424\\
40.7	0.00922402997683871\\
40.71	0.00922402707955536\\
40.72	0.00922402418098335\\
40.73	0.00922402128112183\\
40.74	0.00922401837996997\\
40.75	0.0092240154775269\\
40.76	0.0092240125737918\\
40.77	0.0092240096687638\\
40.78	0.00922400676244206\\
40.79	0.00922400385482573\\
40.8	0.00922400094591396\\
40.81	0.0092239980357059\\
40.82	0.00922399512420068\\
40.83	0.00922399221139746\\
40.84	0.00922398929729539\\
40.85	0.00922398638189359\\
40.86	0.00922398346519123\\
40.87	0.00922398054718742\\
40.88	0.00922397762788132\\
40.89	0.00922397470727206\\
40.9	0.00922397178535878\\
40.91	0.00922396886214062\\
40.92	0.0092239659376167\\
40.93	0.00922396301178616\\
40.94	0.00922396008464814\\
40.95	0.00922395715620176\\
40.96	0.00922395422644615\\
40.97	0.00922395129538044\\
40.98	0.00922394836300375\\
40.99	0.00922394542931522\\
41	0.00922394249431397\\
41.01	0.00922393955799911\\
41.02	0.00922393662036977\\
41.03	0.00922393368142507\\
41.04	0.00922393074116413\\
41.05	0.00922392779958607\\
41.06	0.00922392485668999\\
41.07	0.00922392191247503\\
41.08	0.00922391896694028\\
41.09	0.00922391602008486\\
41.1	0.00922391307190789\\
41.11	0.00922391012240847\\
41.12	0.0092239071715857\\
41.13	0.0092239042194387\\
41.14	0.00922390126596657\\
41.15	0.00922389831116841\\
41.16	0.00922389535504333\\
41.17	0.00922389239759043\\
41.18	0.00922388943880879\\
41.19	0.00922388647869753\\
41.2	0.00922388351725574\\
41.21	0.00922388055448251\\
41.22	0.00922387759037694\\
41.23	0.00922387462493812\\
41.24	0.00922387165816513\\
41.25	0.00922386869005707\\
41.26	0.00922386572061302\\
41.27	0.00922386274983207\\
41.28	0.00922385977771331\\
41.29	0.00922385680425581\\
41.3	0.00922385382945866\\
41.31	0.00922385085332094\\
41.32	0.00922384787584171\\
41.33	0.00922384489702007\\
41.34	0.00922384191685509\\
41.35	0.00922383893534583\\
41.36	0.00922383595249137\\
41.37	0.00922383296829078\\
41.38	0.00922382998274313\\
41.39	0.00922382699584748\\
41.4	0.0092238240076029\\
41.41	0.00922382101800845\\
41.42	0.0092238180270632\\
41.43	0.00922381503476619\\
41.44	0.00922381204111651\\
41.45	0.00922380904611318\\
41.46	0.00922380604975529\\
41.47	0.00922380305204186\\
41.48	0.00922380005297197\\
41.49	0.00922379705254466\\
41.5	0.00922379405075897\\
41.51	0.00922379104761395\\
41.52	0.00922378804310865\\
41.53	0.00922378503724211\\
41.54	0.00922378203001337\\
41.55	0.00922377902142147\\
41.56	0.00922377601146545\\
41.57	0.00922377300014435\\
41.58	0.00922376998745719\\
41.59	0.00922376697340301\\
41.6	0.00922376395798085\\
41.61	0.00922376094118972\\
41.62	0.00922375792302866\\
41.63	0.0092237549034967\\
41.64	0.00922375188259284\\
41.65	0.00922374886031613\\
41.66	0.00922374583666558\\
41.67	0.0092237428116402\\
41.68	0.00922373978523901\\
41.69	0.00922373675746103\\
41.7	0.00922373372830527\\
41.71	0.00922373069777074\\
41.72	0.00922372766585645\\
41.73	0.0092237246325614\\
41.74	0.00922372159788461\\
41.75	0.00922371856182507\\
41.76	0.00922371552438179\\
41.77	0.00922371248555377\\
41.78	0.00922370944534\\
41.79	0.00922370640373949\\
41.8	0.00922370336075122\\
41.81	0.00922370031637419\\
41.82	0.00922369727060738\\
41.83	0.0092236942234498\\
41.84	0.00922369117490042\\
41.85	0.00922368812495824\\
41.86	0.00922368507362223\\
41.87	0.00922368202089137\\
41.88	0.00922367896676466\\
41.89	0.00922367591124106\\
41.9	0.00922367285431955\\
41.91	0.00922366979599911\\
41.92	0.0092236667362787\\
41.93	0.00922366367515731\\
41.94	0.00922366061263389\\
41.95	0.00922365754870743\\
41.96	0.00922365448337687\\
41.97	0.0092236514166412\\
41.98	0.00922364834849936\\
41.99	0.00922364527895033\\
42	0.00922364220799305\\
42.01	0.00922363913562648\\
42.02	0.00922363606184959\\
42.03	0.00922363298666133\\
42.04	0.00922362991006064\\
42.05	0.00922362683204648\\
42.06	0.0092236237526178\\
42.07	0.00922362067177354\\
42.08	0.00922361758951265\\
42.09	0.00922361450583407\\
42.1	0.00922361142073675\\
42.11	0.00922360833421962\\
42.12	0.00922360524628163\\
42.13	0.00922360215692172\\
42.14	0.00922359906613881\\
42.15	0.00922359597393185\\
42.16	0.00922359288029976\\
42.17	0.00922358978524148\\
42.18	0.00922358668875595\\
42.19	0.00922358359084208\\
42.2	0.0092235804914988\\
42.21	0.00922357739072505\\
42.22	0.00922357428851974\\
42.23	0.0092235711848818\\
42.24	0.00922356807981015\\
42.25	0.00922356497330371\\
42.26	0.00922356186536141\\
42.27	0.00922355875598215\\
42.28	0.00922355564516486\\
42.29	0.00922355253290845\\
42.3	0.00922354941921183\\
42.31	0.00922354630407393\\
42.32	0.00922354318749364\\
42.33	0.00922354006946989\\
42.34	0.00922353695000158\\
42.35	0.00922353382908763\\
42.36	0.00922353070672693\\
42.37	0.0092235275829184\\
42.38	0.00922352445766094\\
42.39	0.00922352133095346\\
42.4	0.00922351820279486\\
42.41	0.00922351507318405\\
42.42	0.00922351194211992\\
42.43	0.00922350880960139\\
42.44	0.00922350567562734\\
42.45	0.00922350254019668\\
42.46	0.00922349940330832\\
42.47	0.00922349626496115\\
42.48	0.00922349312515406\\
42.49	0.00922348998388595\\
42.5	0.00922348684115574\\
42.51	0.00922348369696229\\
42.52	0.00922348055130453\\
42.53	0.00922347740418133\\
42.54	0.0092234742555916\\
42.55	0.00922347110553423\\
42.56	0.00922346795400811\\
42.57	0.00922346480101214\\
42.58	0.00922346164654521\\
42.59	0.00922345849060621\\
42.6	0.00922345533319405\\
42.61	0.0092234521743076\\
42.62	0.00922344901394577\\
42.63	0.00922344585210744\\
42.64	0.00922344268879151\\
42.65	0.00922343952399687\\
42.66	0.00922343635772241\\
42.67	0.00922343318996703\\
42.68	0.00922343002072961\\
42.69	0.00922342685000904\\
42.7	0.00922342367780423\\
42.71	0.00922342050411405\\
42.72	0.0092234173289374\\
42.73	0.00922341415227319\\
42.74	0.00922341097412029\\
42.75	0.00922340779447759\\
42.76	0.009223404613344\\
42.77	0.00922340143071841\\
42.78	0.00922339824659969\\
42.79	0.00922339506098675\\
42.8	0.00922339187387849\\
42.81	0.00922338868527379\\
42.82	0.00922338549517155\\
42.83	0.00922338230357066\\
42.84	0.00922337911047001\\
42.85	0.00922337591586849\\
42.86	0.00922337271976501\\
42.87	0.00922336952215846\\
42.88	0.00922336632304772\\
42.89	0.00922336312243169\\
42.9	0.00922335992030928\\
42.91	0.00922335671667936\\
42.92	0.00922335351154084\\
42.93	0.00922335030489261\\
42.94	0.00922334709673357\\
42.95	0.00922334388706261\\
42.96	0.00922334067587862\\
42.97	0.0092233374631805\\
42.98	0.00922333424896715\\
42.99	0.00922333103323745\\
43	0.00922332781599031\\
43.01	0.00922332459722462\\
43.02	0.00922332137693927\\
43.03	0.00922331815513315\\
43.04	0.00922331493180517\\
43.05	0.0092233117069542\\
43.06	0.00922330848057915\\
43.07	0.00922330525267892\\
43.08	0.00922330202325238\\
43.09	0.00922329879229843\\
43.1	0.00922329555981597\\
43.11	0.00922329232580388\\
43.12	0.00922328909026105\\
43.13	0.00922328585318638\\
43.14	0.00922328261457874\\
43.15	0.00922327937443704\\
43.16	0.00922327613276015\\
43.17	0.00922327288954696\\
43.18	0.00922326964479635\\
43.19	0.00922326639850721\\
43.2	0.00922326315067843\\
43.21	0.00922325990130887\\
43.22	0.00922325665039742\\
43.23	0.00922325339794297\\
43.24	0.00922325014394439\\
43.25	0.00922324688840055\\
43.26	0.00922324363131033\\
43.27	0.00922324037267261\\
43.28	0.00922323711248625\\
43.29	0.00922323385075014\\
43.3	0.00922323058746313\\
43.31	0.00922322732262411\\
43.32	0.00922322405623193\\
43.33	0.00922322078828546\\
43.34	0.00922321751878358\\
43.35	0.00922321424772513\\
43.36	0.009223210975109\\
43.37	0.00922320770093403\\
43.38	0.00922320442519909\\
43.39	0.00922320114790304\\
43.4	0.00922319786904474\\
43.41	0.00922319458862304\\
43.42	0.0092231913066368\\
43.43	0.00922318802308487\\
43.44	0.00922318473796611\\
43.45	0.00922318145127936\\
43.46	0.00922317816302348\\
43.47	0.00922317487319733\\
43.48	0.00922317158179973\\
43.49	0.00922316828882956\\
43.5	0.00922316499428563\\
43.51	0.00922316169816682\\
43.52	0.00922315840047194\\
43.53	0.00922315510119987\\
43.54	0.00922315180034941\\
43.55	0.00922314849791943\\
43.56	0.00922314519390875\\
43.57	0.00922314188831622\\
43.58	0.00922313858114066\\
43.59	0.00922313527238093\\
43.6	0.00922313196203584\\
43.61	0.00922312865010423\\
43.62	0.00922312533658492\\
43.63	0.00922312202147676\\
43.64	0.00922311870477857\\
43.65	0.00922311538648917\\
43.66	0.0092231120666074\\
43.67	0.00922310874513207\\
43.68	0.00922310542206201\\
43.69	0.00922310209739605\\
43.7	0.009223098771133\\
43.71	0.00922309544327168\\
43.72	0.00922309211381091\\
43.73	0.00922308878274951\\
43.74	0.0092230854500863\\
43.75	0.00922308211582008\\
43.76	0.00922307877994968\\
43.77	0.0092230754424739\\
43.78	0.00922307210339157\\
43.79	0.00922306876270148\\
43.8	0.00922306542040245\\
43.81	0.00922306207649328\\
43.82	0.00922305873097278\\
43.83	0.00922305538383975\\
43.84	0.00922305203509301\\
43.85	0.00922304868473134\\
43.86	0.00922304533275356\\
43.87	0.00922304197915847\\
43.88	0.00922303862394485\\
43.89	0.00922303526711152\\
43.9	0.00922303190865727\\
43.91	0.00922302854858089\\
43.92	0.00922302518688117\\
43.93	0.00922302182355692\\
43.94	0.00922301845860691\\
43.95	0.00922301509202996\\
43.96	0.00922301172382483\\
43.97	0.00922300835399033\\
43.98	0.00922300498252523\\
43.99	0.00922300160942833\\
44	0.00922299823469841\\
44.01	0.00922299485833425\\
44.02	0.00922299148033463\\
44.03	0.00922298810069835\\
44.04	0.00922298471942416\\
44.05	0.00922298133651087\\
44.06	0.00922297795195724\\
44.07	0.00922297456576204\\
44.08	0.00922297117792407\\
44.09	0.00922296778844208\\
44.1	0.00922296439731486\\
44.11	0.00922296100454118\\
44.12	0.00922295761011979\\
44.13	0.00922295421404949\\
44.14	0.00922295081632903\\
44.15	0.00922294741695718\\
44.16	0.00922294401593271\\
44.17	0.00922294061325438\\
44.18	0.00922293720892096\\
44.19	0.00922293380293121\\
44.2	0.00922293039528389\\
44.21	0.00922292698597777\\
44.22	0.00922292357501159\\
44.23	0.00922292016238413\\
44.24	0.00922291674809414\\
44.25	0.00922291333214038\\
44.26	0.00922290991452159\\
44.27	0.00922290649523654\\
44.28	0.00922290307428397\\
44.29	0.00922289965166265\\
44.3	0.00922289622737131\\
44.31	0.00922289280140872\\
44.32	0.00922288937377361\\
44.33	0.00922288594446474\\
44.34	0.00922288251348085\\
44.35	0.0092228790808207\\
44.36	0.00922287564648301\\
44.37	0.00922287221046655\\
44.38	0.00922286877277004\\
44.39	0.00922286533339223\\
44.4	0.00922286189233187\\
44.41	0.00922285844958768\\
44.42	0.00922285500515842\\
44.43	0.00922285155904281\\
44.44	0.00922284811123959\\
44.45	0.00922284466174751\\
44.46	0.00922284121056528\\
44.47	0.00922283775769166\\
44.48	0.00922283430312537\\
44.49	0.00922283084686513\\
44.5	0.00922282738890969\\
44.51	0.00922282392925776\\
44.52	0.00922282046790809\\
44.53	0.0092228170048594\\
44.54	0.00922281354011041\\
44.55	0.00922281007365985\\
44.56	0.00922280660550644\\
44.57	0.00922280313564892\\
44.58	0.00922279966408599\\
44.59	0.0092227961908164\\
44.6	0.00922279271583885\\
44.61	0.00922278923915206\\
44.62	0.00922278576075477\\
44.63	0.00922278228064568\\
44.64	0.00922277879882351\\
44.65	0.00922277531528699\\
44.66	0.00922277183003482\\
44.67	0.00922276834306573\\
44.68	0.00922276485437842\\
44.69	0.00922276136397162\\
44.7	0.00922275787184403\\
44.71	0.00922275437799437\\
44.72	0.00922275088242135\\
44.73	0.00922274738512367\\
44.74	0.00922274388610006\\
44.75	0.00922274038534922\\
44.76	0.00922273688286986\\
44.77	0.00922273337866068\\
44.78	0.00922272987272039\\
44.79	0.00922272636504771\\
44.8	0.00922272285564133\\
44.81	0.00922271934449996\\
44.82	0.00922271583162231\\
44.83	0.00922271231700708\\
44.84	0.00922270880065297\\
44.85	0.00922270528255868\\
44.86	0.00922270176272293\\
44.87	0.0092226982411444\\
44.88	0.0092226947178218\\
44.89	0.00922269119275383\\
44.9	0.00922268766593919\\
44.91	0.00922268413737659\\
44.92	0.0092226806070647\\
44.93	0.00922267707500225\\
44.94	0.00922267354118792\\
44.95	0.0092226700056204\\
44.96	0.00922266646829841\\
44.97	0.00922266292922063\\
44.98	0.00922265938838576\\
44.99	0.0092226558457925\\
45	0.00922265230143954\\
45.01	0.00922264875532557\\
45.02	0.00922264520744929\\
45.03	0.0092226416578094\\
45.04	0.00922263810640457\\
45.05	0.00922263455323352\\
45.06	0.00922263099829493\\
45.07	0.0092226274415875\\
45.08	0.00922262388310991\\
45.09	0.00922262032286087\\
45.1	0.00922261676083906\\
45.11	0.00922261319704316\\
45.12	0.00922260963147188\\
45.13	0.00922260606412391\\
45.14	0.00922260249499794\\
45.15	0.00922259892409265\\
45.16	0.00922259535140675\\
45.17	0.00922259177693891\\
45.18	0.00922258820068783\\
45.19	0.00922258462265219\\
45.2	0.00922258104283071\\
45.21	0.00922257746122205\\
45.22	0.00922257387782492\\
45.23	0.00922257029263799\\
45.24	0.00922256670565997\\
45.25	0.00922256311688954\\
45.26	0.0092225595263254\\
45.27	0.00922255593396623\\
45.28	0.00922255233981073\\
45.29	0.00922254874385759\\
45.3	0.00922254514610549\\
45.31	0.00922254154655314\\
45.32	0.00922253794519921\\
45.33	0.00922253434204241\\
45.34	0.00922253073708143\\
45.35	0.00922252713031495\\
45.36	0.00922252352174168\\
45.37	0.00922251991136029\\
45.38	0.0092225162991695\\
45.39	0.00922251268516799\\
45.4	0.00922250906935445\\
45.41	0.00922250545172759\\
45.42	0.0092225018322861\\
45.43	0.00922249821102866\\
45.44	0.00922249458795399\\
45.45	0.00922249096306077\\
45.46	0.00922248733634771\\
45.47	0.0092224837078135\\
45.48	0.00922248007745684\\
45.49	0.00922247644527643\\
45.5	0.00922247281127097\\
45.51	0.00922246917543917\\
45.52	0.00922246553777972\\
45.53	0.00922246189829133\\
45.54	0.00922245825697269\\
45.55	0.00922245461382253\\
45.56	0.00922245096883953\\
45.57	0.00922244732202242\\
45.58	0.00922244367336988\\
45.59	0.00922244002288064\\
45.6	0.00922243637055341\\
45.61	0.00922243271638688\\
45.62	0.00922242906037978\\
45.63	0.00922242540253082\\
45.64	0.00922242174283871\\
45.65	0.00922241808130217\\
45.66	0.00922241441791991\\
45.67	0.00922241075269065\\
45.68	0.00922240708561311\\
45.69	0.00922240341668601\\
45.7	0.00922239974590808\\
45.71	0.00922239607327803\\
45.72	0.00922239239879459\\
45.73	0.00922238872245649\\
45.74	0.00922238504426246\\
45.75	0.00922238136421123\\
45.76	0.00922237768230152\\
45.77	0.00922237399853207\\
45.78	0.00922237031290161\\
45.79	0.00922236662540889\\
45.8	0.00922236293605264\\
45.81	0.0092223592448316\\
45.82	0.00922235555174451\\
45.83	0.00922235185679013\\
45.84	0.00922234815996718\\
45.85	0.00922234446127443\\
45.86	0.00922234076071062\\
45.87	0.00922233705827451\\
45.88	0.00922233335396484\\
45.89	0.00922232964778039\\
45.9	0.0092223259397199\\
45.91	0.00922232222978214\\
45.92	0.00922231851796587\\
45.93	0.00922231480426987\\
45.94	0.00922231108869289\\
45.95	0.00922230737123371\\
45.96	0.00922230365189111\\
45.97	0.00922229993066386\\
45.98	0.00922229620755075\\
45.99	0.00922229248255055\\
46	0.00922228875566204\\
46.01	0.00922228502688403\\
46.02	0.00922228129621529\\
46.03	0.00922227756365463\\
46.04	0.00922227382920084\\
46.05	0.00922227009285271\\
46.06	0.00922226635460907\\
46.07	0.0092222626144687\\
46.08	0.00922225887243042\\
46.09	0.00922225512849305\\
46.1	0.0092222513826554\\
46.11	0.00922224763491628\\
46.12	0.00922224388527453\\
46.13	0.00922224013372898\\
46.14	0.00922223638027844\\
46.15	0.00922223262492176\\
46.16	0.00922222886765778\\
46.17	0.00922222510848533\\
46.18	0.00922222134740326\\
46.19	0.00922221758441043\\
46.2	0.00922221381950568\\
46.21	0.00922221005268788\\
46.22	0.00922220628395588\\
46.23	0.00922220251330856\\
46.24	0.00922219874074477\\
46.25	0.00922219496626341\\
46.26	0.00922219118986334\\
46.27	0.00922218741154345\\
46.28	0.00922218363130264\\
46.29	0.00922217984913978\\
46.3	0.00922217606505378\\
46.31	0.00922217227904355\\
46.32	0.00922216849110799\\
46.33	0.00922216470124601\\
46.34	0.00922216090945652\\
46.35	0.00922215711573846\\
46.36	0.00922215332009074\\
46.37	0.00922214952251231\\
46.38	0.00922214572300209\\
46.39	0.00922214192155904\\
46.4	0.00922213811818209\\
46.41	0.00922213431287021\\
46.42	0.00922213050562235\\
46.43	0.00922212669643748\\
46.44	0.00922212288531457\\
46.45	0.0092221190722526\\
46.46	0.00922211525725054\\
46.47	0.00922211144030738\\
46.48	0.00922210762142213\\
46.49	0.00922210380059379\\
46.5	0.00922209997782134\\
46.51	0.00922209615310383\\
46.52	0.00922209232644025\\
46.53	0.00922208849782964\\
46.54	0.00922208466727103\\
46.55	0.00922208083476346\\
46.56	0.00922207700030597\\
46.57	0.00922207316389762\\
46.58	0.00922206932553746\\
46.59	0.00922206548522457\\
46.6	0.00922206164295801\\
46.61	0.00922205779873687\\
46.62	0.00922205395256022\\
46.63	0.00922205010442718\\
46.64	0.00922204625433684\\
46.65	0.0092220424022883\\
46.66	0.00922203854828069\\
46.67	0.00922203469231313\\
46.68	0.00922203083438475\\
46.69	0.0092220269744947\\
46.7	0.00922202311264212\\
46.71	0.00922201924882616\\
46.72	0.00922201538304599\\
46.73	0.00922201151530079\\
46.74	0.00922200764558973\\
46.75	0.009222003773912\\
46.76	0.00922199990026679\\
46.77	0.00922199602465333\\
46.78	0.0092219921470708\\
46.79	0.00922198826751845\\
46.8	0.0092219843859955\\
46.81	0.00922198050250119\\
46.82	0.00922197661703476\\
46.83	0.00922197272959548\\
46.84	0.00922196884018262\\
46.85	0.00922196494879545\\
46.86	0.00922196105543326\\
46.87	0.00922195716009534\\
46.88	0.00922195326278099\\
46.89	0.00922194936348953\\
46.9	0.00922194546222028\\
46.91	0.00922194155897257\\
46.92	0.00922193765374575\\
46.93	0.00922193374653917\\
46.94	0.00922192983735219\\
46.95	0.00922192592618419\\
46.96	0.00922192201303454\\
46.97	0.00922191809790264\\
46.98	0.00922191418078789\\
46.99	0.00922191026168971\\
47	0.00922190634060751\\
47.01	0.00922190241754074\\
47.02	0.00922189849248884\\
47.03	0.00922189456545125\\
47.04	0.00922189063642745\\
47.05	0.00922188670541692\\
47.06	0.00922188277241914\\
47.07	0.00922187883743361\\
47.08	0.00922187490045984\\
47.09	0.00922187096149735\\
47.1	0.00922186702054567\\
47.11	0.00922186307760435\\
47.12	0.00922185913267294\\
47.13	0.009221855185751\\
47.14	0.00922185123683812\\
47.15	0.00922184728593388\\
47.16	0.00922184333303788\\
47.17	0.00922183937814974\\
47.18	0.00922183542126907\\
47.19	0.00922183146239552\\
47.2	0.00922182750152872\\
47.21	0.00922182353866834\\
47.22	0.00922181957381406\\
47.23	0.00922181560696554\\
47.24	0.0092218116381225\\
47.25	0.00922180766728462\\
47.26	0.00922180369445164\\
47.27	0.00922179971962329\\
47.28	0.0092217957427993\\
47.29	0.00922179176397943\\
47.3	0.00922178778316345\\
47.31	0.00922178380035115\\
47.32	0.00922177981554231\\
47.33	0.00922177582873674\\
47.34	0.00922177183993425\\
47.35	0.00922176784913469\\
47.36	0.00922176385633787\\
47.37	0.00922175986154367\\
47.38	0.00922175586475195\\
47.39	0.00922175186596259\\
47.4	0.00922174786517548\\
47.41	0.00922174386239052\\
47.42	0.00922173985760764\\
47.43	0.00922173585082676\\
47.44	0.00922173184204782\\
47.45	0.00922172783127079\\
47.46	0.00922172381849561\\
47.47	0.00922171980372229\\
47.48	0.0092217157869508\\
47.49	0.00922171176818115\\
47.5	0.00922170774741336\\
47.51	0.00922170372464746\\
47.52	0.00922169969988349\\
47.53	0.0092216956731215\\
47.54	0.00922169164436156\\
47.55	0.00922168761360375\\
47.56	0.00922168358084815\\
47.57	0.00922167954609487\\
47.58	0.00922167550934403\\
47.59	0.00922167147059575\\
47.6	0.00922166742985016\\
47.61	0.00922166338710743\\
47.62	0.0092216593423677\\
47.63	0.00922165529563116\\
47.64	0.00922165124689798\\
47.65	0.00922164719616837\\
47.66	0.00922164314344253\\
47.67	0.00922163908872068\\
47.68	0.00922163503200305\\
47.69	0.00922163097328989\\
47.7	0.00922162691258143\\
47.71	0.00922162284987795\\
47.72	0.00922161878517971\\
47.73	0.009221614718487\\
47.74	0.00922161064980012\\
47.75	0.00922160657911936\\
47.76	0.00922160250644504\\
47.77	0.00922159843177747\\
47.78	0.00922159435511701\\
47.79	0.00922159027646397\\
47.8	0.00922158619581872\\
47.81	0.00922158211318161\\
47.82	0.00922157802855301\\
47.83	0.00922157394193329\\
47.84	0.00922156985332285\\
47.85	0.00922156576272206\\
47.86	0.00922156167013134\\
47.87	0.00922155757555109\\
47.88	0.00922155347898172\\
47.89	0.00922154938042364\\
47.9	0.0092215452798773\\
47.91	0.00922154117734312\\
47.92	0.00922153707282154\\
47.93	0.00922153296631302\\
47.94	0.00922152885781798\\
47.95	0.00922152474733689\\
47.96	0.00922152063487021\\
47.97	0.0092215165204184\\
47.98	0.00922151240398193\\
47.99	0.00922150828556126\\
48	0.00922150416515688\\
48.01	0.00922150004276925\\
48.02	0.00922149591839885\\
48.03	0.00922149179204616\\
48.04	0.00922148766371167\\
48.05	0.00922148353339585\\
48.06	0.00922147940109917\\
48.07	0.00922147526682214\\
48.08	0.00922147113056521\\
48.09	0.00922146699232888\\
48.1	0.00922146285211361\\
48.11	0.00922145870991989\\
48.12	0.00922145456574818\\
48.13	0.00922145041959895\\
48.14	0.00922144627147266\\
48.15	0.00922144212136978\\
48.16	0.00922143796929076\\
48.17	0.00922143381523605\\
48.18	0.00922142965920608\\
48.19	0.0092214255012013\\
48.2	0.00922142134122212\\
48.21	0.00922141717926897\\
48.22	0.00922141301534226\\
48.23	0.00922140884944239\\
48.24	0.00922140468156973\\
48.25	0.00922140051172468\\
48.26	0.00922139633990759\\
48.27	0.00922139216611882\\
48.28	0.0092213879903587\\
48.29	0.00922138381262757\\
48.3	0.00922137963292572\\
48.31	0.00922137545125346\\
48.32	0.00922137126761105\\
48.33	0.00922136708199876\\
48.34	0.00922136289441683\\
48.35	0.00922135870486547\\
48.36	0.00922135451334488\\
48.37	0.00922135031985524\\
48.38	0.00922134612439672\\
48.39	0.00922134192696943\\
48.4	0.0092213377275735\\
48.41	0.009221333526209\\
48.42	0.00922132932287598\\
48.43	0.0092213251175745\\
48.44	0.00922132091030453\\
48.45	0.00922131670106605\\
48.46	0.00922131248985902\\
48.47	0.00922130827668334\\
48.48	0.00922130406153888\\
48.49	0.0092212998444255\\
48.5	0.009221295625343\\
48.51	0.00922129140429116\\
48.52	0.00922128718126972\\
48.53	0.00922128295627838\\
48.54	0.0092212787293168\\
48.55	0.0092212745003846\\
48.56	0.00922127026948135\\
48.57	0.00922126603660661\\
48.58	0.00922126180175984\\
48.59	0.00922125756494052\\
48.6	0.00922125332614803\\
48.61	0.00922124908538173\\
48.62	0.00922124484264092\\
48.63	0.00922124059792486\\
48.64	0.00922123635123275\\
48.65	0.00922123210256373\\
48.66	0.00922122785191691\\
48.67	0.00922122359929133\\
48.68	0.00922121934468596\\
48.69	0.00922121508809974\\
48.7	0.00922121082953153\\
48.71	0.00922120656898013\\
48.72	0.00922120230644429\\
48.73	0.00922119804192269\\
48.74	0.00922119377541394\\
48.75	0.00922118950691659\\
48.76	0.00922118523642913\\
48.77	0.00922118096394996\\
48.78	0.00922117668947743\\
48.79	0.00922117241300982\\
48.8	0.0092211681345453\\
48.81	0.00922116385408202\\
48.82	0.00922115957161801\\
48.83	0.00922115528715126\\
48.84	0.00922115100067965\\
48.85	0.00922114671220099\\
48.86	0.00922114242171302\\
48.87	0.00922113812921338\\
48.88	0.00922113383469965\\
48.89	0.0092211295381693\\
48.9	0.00922112523961973\\
48.91	0.00922112093904824\\
48.92	0.00922111663645205\\
48.93	0.0092211123318283\\
48.94	0.00922110802517401\\
48.95	0.00922110371648613\\
48.96	0.00922109940576152\\
48.97	0.00922109509299692\\
48.98	0.009221090778189\\
48.99	0.00922108646133433\\
49	0.00922108214242935\\
49.01	0.00922107782147045\\
49.02	0.00922107349845389\\
49.03	0.00922106917337582\\
49.04	0.00922106484623231\\
49.05	0.00922106051701932\\
49.06	0.0092210561857327\\
49.07	0.00922105185236821\\
49.08	0.00922104751692148\\
49.09	0.00922104317938805\\
49.1	0.00922103883976336\\
49.11	0.00922103449804272\\
49.12	0.00922103015422134\\
49.13	0.00922102580829435\\
49.14	0.00922102146025673\\
49.15	0.00922101711010336\\
49.16	0.00922101275782902\\
49.17	0.00922100840342839\\
49.18	0.009221004046896\\
49.19	0.00922099968822632\\
49.2	0.00922099532741367\\
49.21	0.00922099096445227\\
49.22	0.00922098659933624\\
49.23	0.00922098223205958\\
49.24	0.00922097786261617\\
49.25	0.00922097349099981\\
49.26	0.00922096911720416\\
49.27	0.00922096474122278\\
49.28	0.00922096036304914\\
49.29	0.00922095598267657\\
49.3	0.00922095160009832\\
49.31	0.00922094721530752\\
49.32	0.0092209428282972\\
49.33	0.00922093843906028\\
49.34	0.0092209340475896\\
49.35	0.00922092965387787\\
49.36	0.00922092525791771\\
49.37	0.00922092085970165\\
49.38	0.00922091645922212\\
49.39	0.00922091205647145\\
49.4	0.00922090765144189\\
49.41	0.00922090324412559\\
49.42	0.0092208988345146\\
49.43	0.00922089442260092\\
49.44	0.00922089000837644\\
49.45	0.00922088559183297\\
49.46	0.00922088117296226\\
49.47	0.00922087675175597\\
49.48	0.00922087232820568\\
49.49	0.00922086790230294\\
49.5	0.0092208634740392\\
49.51	0.00922085904340586\\
49.52	0.00922085461039427\\
49.53	0.00922085017499571\\
49.54	0.00922084573720143\\
49.55	0.00922084129700263\\
49.56	0.00922083685439046\\
49.57	0.00922083240935604\\
49.58	0.00922082796189047\\
49.59	0.0092208235119848\\
49.6	0.00922081905963008\\
49.61	0.00922081460481733\\
49.62	0.00922081014753757\\
49.63	0.00922080568778181\\
49.64	0.00922080122554105\\
49.65	0.00922079676080631\\
49.66	0.00922079229356862\\
49.67	0.00922078782381901\\
49.68	0.00922078335154856\\
49.69	0.00922077887674836\\
49.7	0.00922077439940953\\
49.71	0.00922076991952326\\
49.72	0.00922076543708076\\
49.73	0.00922076095207332\\
49.74	0.00922075646449227\\
49.75	0.00922075197432902\\
49.76	0.00922074748157505\\
49.77	0.00922074298622192\\
49.78	0.00922073848826129\\
49.79	0.00922073398768489\\
49.8	0.00922072948448458\\
49.81	0.00922072497865232\\
49.82	0.00922072047018017\\
49.83	0.00922071595906033\\
49.84	0.0092207114452851\\
49.85	0.00922070692884695\\
49.86	0.00922070240973846\\
49.87	0.00922069788795237\\
49.88	0.00922069336348156\\
49.89	0.00922068883631908\\
49.9	0.00922068430645813\\
49.91	0.00922067977389207\\
49.92	0.00922067523861443\\
49.93	0.00922067070061893\\
49.94	0.00922066615989945\\
49.95	0.00922066161645005\\
49.96	0.00922065707026497\\
49.97	0.00922065252133864\\
49.98	0.00922064796966566\\
49.99	0.00922064341524083\\
50	0.0092206388580591\\
50.01	0.00922063429811563\\
50.02	0.00922062973540575\\
50.03	0.00922062516992494\\
50.04	0.00922062060166887\\
50.05	0.00922061603063336\\
50.06	0.00922061145681438\\
50.07	0.00922060688020806\\
50.08	0.00922060230081064\\
50.09	0.00922059771861847\\
50.1	0.00922059313362804\\
50.11	0.0092205885458359\\
50.12	0.00922058395523868\\
50.13	0.00922057936183304\\
50.14	0.00922057476561569\\
50.15	0.00922057016658332\\
50.16	0.00922056556473262\\
50.17	0.00922056096006026\\
50.18	0.00922055635256293\\
50.19	0.00922055174223728\\
50.2	0.00922054712907996\\
50.21	0.00922054251308763\\
50.22	0.00922053789425691\\
50.23	0.00922053327258444\\
50.24	0.00922052864806684\\
50.25	0.00922052402070071\\
50.26	0.00922051939048266\\
50.27	0.00922051475740928\\
50.28	0.00922051012147714\\
50.29	0.00922050548268282\\
50.3	0.00922050084102288\\
50.31	0.00922049619649388\\
50.32	0.00922049154909235\\
50.33	0.00922048689881482\\
50.34	0.00922048224565781\\
50.35	0.00922047758961783\\
50.36	0.00922047293069139\\
50.37	0.00922046826887495\\
50.38	0.00922046360416501\\
50.39	0.00922045893655801\\
50.4	0.00922045426605043\\
50.41	0.00922044959263869\\
50.42	0.00922044491631922\\
50.43	0.00922044023708843\\
50.44	0.00922043555494274\\
50.45	0.00922043086987852\\
50.46	0.00922042618189216\\
50.47	0.00922042149098001\\
50.48	0.00922041679713845\\
50.49	0.00922041210036379\\
50.5	0.00922040740065235\\
50.51	0.00922040269800047\\
50.52	0.00922039799240442\\
50.53	0.0092203932838605\\
50.54	0.00922038857236498\\
50.55	0.0092203838579141\\
50.56	0.00922037914050411\\
50.57	0.00922037442013124\\
50.58	0.0092203696967917\\
50.59	0.00922036497048169\\
50.6	0.00922036024119739\\
50.61	0.00922035550893496\\
50.62	0.00922035077369057\\
50.63	0.00922034603546036\\
50.64	0.00922034129424043\\
50.65	0.0092203365500269\\
50.66	0.00922033180281586\\
50.67	0.00922032705260338\\
50.68	0.00922032229938552\\
50.69	0.00922031754315834\\
50.7	0.00922031278391784\\
50.71	0.00922030802166005\\
50.72	0.00922030325638095\\
50.73	0.00922029848807653\\
50.74	0.00922029371674273\\
50.75	0.00922028894237551\\
50.76	0.00922028416497078\\
50.77	0.00922027938452446\\
50.78	0.00922027460103242\\
50.79	0.00922026981449055\\
50.8	0.00922026502489469\\
50.81	0.00922026023224067\\
50.82	0.00922025543652432\\
50.83	0.00922025063774142\\
50.84	0.00922024583588777\\
50.85	0.0092202410309591\\
50.86	0.00922023622295115\\
50.87	0.00922023141185966\\
50.88	0.00922022659768031\\
50.89	0.00922022178040879\\
50.9	0.00922021696004075\\
50.91	0.00922021213657183\\
50.92	0.00922020730999764\\
50.93	0.00922020248031379\\
50.94	0.00922019764751583\\
50.95	0.00922019281159933\\
50.96	0.00922018797255982\\
50.97	0.0092201831303928\\
50.98	0.00922017828509375\\
50.99	0.00922017343665815\\
51	0.00922016858508142\\
51.01	0.009220163730359\\
51.02	0.00922015887248626\\
51.03	0.00922015401145859\\
51.04	0.00922014914727133\\
51.05	0.00922014427991978\\
51.06	0.00922013940939927\\
51.07	0.00922013453570505\\
51.08	0.00922012965883237\\
51.09	0.00922012477877646\\
51.1	0.0092201198955325\\
51.11	0.00922011500909568\\
51.12	0.00922011011946113\\
51.13	0.00922010522662397\\
51.14	0.00922010033057929\\
51.15	0.00922009543132214\\
51.16	0.00922009052884758\\
51.17	0.00922008562315059\\
51.18	0.00922008071422617\\
51.19	0.00922007580206926\\
51.2	0.00922007088667478\\
51.21	0.00922006596803762\\
51.22	0.00922006104615264\\
51.23	0.00922005612101468\\
51.24	0.00922005119261854\\
51.25	0.00922004626095899\\
51.26	0.00922004132603076\\
51.27	0.00922003638782857\\
51.28	0.00922003144634709\\
51.29	0.00922002650158097\\
51.3	0.0092200215535248\\
51.31	0.00922001660217318\\
51.32	0.00922001164752064\\
51.33	0.00922000668956171\\
51.34	0.00922000172829084\\
51.35	0.00921999676370249\\
51.36	0.00921999179579105\\
51.37	0.00921998682455091\\
51.38	0.00921998184997638\\
51.39	0.00921997687206178\\
51.4	0.00921997189080136\\
51.41	0.00921996690618934\\
51.42	0.00921996191821991\\
51.43	0.00921995692688722\\
51.44	0.00921995193218537\\
51.45	0.00921994693410842\\
51.46	0.00921994193265043\\
51.47	0.00921993692780535\\
51.48	0.00921993191956715\\
51.49	0.00921992690792973\\
51.5	0.00921992189288696\\
51.51	0.00921991687443265\\
51.52	0.00921991185256058\\
51.53	0.00921990682726449\\
51.54	0.00921990179853807\\
51.55	0.00921989676637496\\
51.56	0.00921989173076877\\
51.57	0.00921988669171305\\
51.58	0.00921988164920131\\
51.59	0.00921987660322701\\
51.6	0.00921987155378356\\
51.61	0.00921986650086433\\
51.62	0.00921986144446264\\
51.63	0.00921985638457175\\
51.64	0.00921985132118488\\
51.65	0.00921984625429521\\
51.66	0.00921984118389583\\
51.67	0.00921983610997982\\
51.68	0.00921983103254019\\
51.69	0.0092198259515699\\
51.7	0.00921982086706184\\
51.71	0.00921981577900887\\
51.72	0.00921981068740378\\
51.73	0.0092198055922393\\
51.74	0.00921980049350811\\
51.75	0.00921979539120285\\
51.76	0.00921979028531606\\
51.77	0.00921978517584026\\
51.78	0.00921978006276789\\
51.79	0.00921977494609134\\
51.8	0.00921976982580292\\
51.81	0.00921976470189489\\
51.82	0.00921975957435946\\
51.83	0.00921975444318875\\
51.84	0.00921974930837485\\
51.85	0.00921974416990973\\
51.86	0.00921973902778535\\
51.87	0.00921973388199357\\
51.88	0.0092197287325262\\
51.89	0.00921972357937495\\
51.9	0.00921971842253151\\
51.91	0.00921971326198744\\
51.92	0.00921970809773429\\
51.93	0.00921970292976348\\
51.94	0.0092196977580664\\
51.95	0.00921969258263433\\
51.96	0.00921968740345851\\
51.97	0.00921968222053007\\
51.98	0.00921967703384009\\
51.99	0.00921967184337954\\
52	0.00921966664913935\\
52.01	0.00921966145111034\\
52.02	0.00921965624928325\\
52.03	0.00921965104364875\\
52.04	0.00921964583419742\\
52.05	0.00921964062091974\\
52.06	0.00921963540380614\\
52.07	0.00921963018284692\\
52.08	0.00921962495803232\\
52.09	0.00921961972935249\\
52.1	0.00921961449679747\\
52.11	0.00921960926035721\\
52.12	0.0092196040200216\\
52.13	0.0092195987757804\\
52.14	0.00921959352762327\\
52.15	0.00921958827553981\\
52.16	0.00921958301951949\\
52.17	0.00921957775955169\\
52.18	0.0092195724956257\\
52.19	0.00921956722773067\\
52.2	0.0092195619558557\\
52.21	0.00921955667998975\\
52.22	0.00921955140012168\\
52.23	0.00921954611624024\\
52.24	0.00921954082833408\\
52.25	0.00921953553639173\\
52.26	0.00921953024040162\\
52.27	0.00921952494035205\\
52.28	0.00921951963623122\\
52.29	0.0092195143280272\\
52.3	0.00921950901572794\\
52.31	0.0092195036993213\\
52.32	0.00921949837879498\\
52.33	0.00921949305413658\\
52.34	0.00921948772533357\\
52.35	0.0092194823923733\\
52.36	0.00921947705524297\\
52.37	0.00921947171392968\\
52.38	0.00921946636842037\\
52.39	0.00921946101870188\\
52.4	0.00921945566476088\\
52.41	0.00921945030658393\\
52.42	0.00921944494415744\\
52.43	0.00921943957746769\\
52.44	0.00921943420650079\\
52.45	0.00921942883124274\\
52.46	0.00921942345167937\\
52.47	0.00921941806779639\\
52.48	0.00921941267957933\\
52.49	0.00921940728701358\\
52.5	0.00921940189008438\\
52.51	0.00921939648877682\\
52.52	0.00921939108307581\\
52.53	0.00921938567296612\\
52.54	0.00921938025843235\\
52.55	0.00921937483945894\\
52.56	0.00921936941603017\\
52.57	0.00921936398813013\\
52.58	0.00921935855574276\\
52.59	0.00921935311885183\\
52.6	0.00921934767744091\\
52.61	0.00921934223149341\\
52.62	0.00921933678099257\\
52.63	0.00921933132592142\\
52.64	0.00921932586626284\\
52.65	0.00921932040199949\\
52.66	0.00921931493311386\\
52.67	0.00921930945958825\\
52.68	0.00921930398140475\\
52.69	0.00921929849854526\\
52.7	0.0092192930109915\\
52.71	0.00921928751872496\\
52.72	0.00921928202172693\\
52.73	0.00921927651997852\\
52.74	0.00921927101346059\\
52.75	0.00921926550215381\\
52.76	0.00921925998603865\\
52.77	0.00921925446509531\\
52.78	0.00921924893930383\\
52.79	0.00921924340864398\\
52.8	0.00921923787309534\\
52.81	0.00921923233263722\\
52.82	0.00921922678724873\\
52.83	0.00921922123690872\\
52.84	0.00921921568159583\\
52.85	0.00921921012128844\\
52.86	0.00921920455596467\\
52.87	0.00921919898560242\\
52.88	0.00921919341017932\\
52.89	0.00921918782967277\\
52.9	0.00921918224405986\\
52.91	0.00921917665331748\\
52.92	0.00921917105742222\\
52.93	0.00921916545635041\\
52.94	0.00921915985007809\\
52.95	0.00921915423858107\\
52.96	0.00921914862183485\\
52.97	0.00921914299981463\\
52.98	0.00921913737249537\\
52.99	0.0092191317398517\\
53	0.00921912610185799\\
53.01	0.00921912045848829\\
53.02	0.00921911480971635\\
53.03	0.00921910915551563\\
53.04	0.00921910349585927\\
53.05	0.00921909783072011\\
53.06	0.00921909216007065\\
53.07	0.0092190864838831\\
53.08	0.00921908080212932\\
53.09	0.00921907511478086\\
53.1	0.00921906942180893\\
53.11	0.00921906372318441\\
53.12	0.00921905801887782\\
53.13	0.00921905230885935\\
53.14	0.00921904659309884\\
53.15	0.00921904087156578\\
53.16	0.00921903514422929\\
53.17	0.00921902941105814\\
53.18	0.00921902367202071\\
53.19	0.00921901792708505\\
53.2	0.00921901217621878\\
53.21	0.00921900641938919\\
53.22	0.00921900065656314\\
53.23	0.00921899488770714\\
53.24	0.00921898911278726\\
53.25	0.00921898333176922\\
53.26	0.00921897754461829\\
53.27	0.00921897175129935\\
53.28	0.00921896595177687\\
53.29	0.00921896014601488\\
53.3	0.00921895433397701\\
53.31	0.00921894851562643\\
53.32	0.0092189426909259\\
53.33	0.00921893685983772\\
53.34	0.00921893102232376\\
53.35	0.00921892517834541\\
53.36	0.00921891932786364\\
53.37	0.00921891347083892\\
53.38	0.00921890760723128\\
53.39	0.00921890173700026\\
53.4	0.00921889586010491\\
53.41	0.00921888997650382\\
53.42	0.00921888408615506\\
53.43	0.00921887818901623\\
53.44	0.00921887228504441\\
53.45	0.00921886637419616\\
53.46	0.00921886045642755\\
53.47	0.00921885453169411\\
53.48	0.00921884859995084\\
53.49	0.00921884266115221\\
53.5	0.00921883671525215\\
53.51	0.00921883076220405\\
53.52	0.00921882480196073\\
53.53	0.00921881883447446\\
53.54	0.00921881285969694\\
53.55	0.0092188068775793\\
53.56	0.00921880088807209\\
53.57	0.00921879489112526\\
53.58	0.00921878888668819\\
53.59	0.00921878287470964\\
53.6	0.00921877685513776\\
53.61	0.00921877082792011\\
53.62	0.00921876479300359\\
53.63	0.00921875875033452\\
53.64	0.00921875269985853\\
53.65	0.00921874664152065\\
53.66	0.00921874057526525\\
53.67	0.00921873450103601\\
53.68	0.00921872841877599\\
53.69	0.00921872232842756\\
53.7	0.00921871622993239\\
53.71	0.00921871012323148\\
53.72	0.00921870400826514\\
53.73	0.00921869788497298\\
53.74	0.00921869175329386\\
53.75	0.00921868561316596\\
53.76	0.00921867946452673\\
53.77	0.00921867330731287\\
53.78	0.00921866714146034\\
53.79	0.00921866096690435\\
53.8	0.00921865478357935\\
53.81	0.00921864859141902\\
53.82	0.00921864239035628\\
53.83	0.00921863618032324\\
53.84	0.00921862996125123\\
53.85	0.00921862373307078\\
53.86	0.00921861749571161\\
53.87	0.00921861124910261\\
53.88	0.00921860499317186\\
53.89	0.0092185987278466\\
53.9	0.0092185924530532\\
53.91	0.00921858616871721\\
53.92	0.0092185798747633\\
53.93	0.00921857357111527\\
53.94	0.00921856725769605\\
53.95	0.00921856093442766\\
53.96	0.00921855460123124\\
53.97	0.009218548258027\\
53.98	0.00921854190473427\\
53.99	0.00921853554127141\\
54	0.00921852916755588\\
54.01	0.00921852278350417\\
54.02	0.00921851638903182\\
54.03	0.00921850998405342\\
54.04	0.00921850356848257\\
54.05	0.0092184971422319\\
54.06	0.00921849070521303\\
54.07	0.0092184842573366\\
54.08	0.00921847779851222\\
54.09	0.00921847132864848\\
54.1	0.00921846484765294\\
54.11	0.00921845835543213\\
54.12	0.00921845185189151\\
54.13	0.00921844533693549\\
54.14	0.00921843881046741\\
54.15	0.00921843227238952\\
54.16	0.009218425722603\\
54.17	0.00921841916100788\\
54.18	0.00921841258750315\\
54.19	0.00921840600198661\\
54.2	0.00921839940435498\\
54.21	0.00921839279450382\\
54.22	0.00921838617232752\\
54.23	0.00921837953771934\\
54.24	0.00921837289057135\\
54.25	0.00921836623077445\\
54.26	0.00921835955821832\\
54.27	0.00921835287279149\\
54.28	0.00921834617438122\\
54.29	0.00921833946287359\\
54.3	0.00921833273815342\\
54.31	0.00921832600010432\\
54.32	0.00921831924860861\\
54.33	0.00921831248354738\\
54.34	0.00921830570480041\\
54.35	0.00921829891224624\\
54.36	0.00921829210576209\\
54.37	0.00921828528522388\\
54.38	0.00921827845050623\\
54.39	0.00921827160148242\\
54.4	0.00921826473802442\\
54.41	0.00921825786000282\\
54.42	0.0092182509672869\\
54.43	0.00921824405974455\\
54.44	0.0092182371372423\\
54.45	0.0092182301996453\\
54.46	0.00921822324681729\\
54.47	0.00921821627862064\\
54.48	0.0092182092949163\\
54.49	0.00921820229556377\\
54.5	0.00921819528042117\\
54.51	0.00921818824934515\\
54.52	0.00921818120219092\\
54.53	0.00921817413881225\\
54.54	0.00921816705906141\\
54.55	0.00921815996278924\\
54.56	0.00921815284984506\\
54.57	0.00921814572007672\\
54.58	0.00921813857333057\\
54.59	0.00921813140945144\\
54.6	0.00921812422828267\\
54.61	0.00921811702966604\\
54.62	0.00921810981344184\\
54.63	0.00921810257944879\\
54.64	0.00921809532752407\\
54.65	0.00921808805750332\\
54.66	0.0092180807692206\\
54.67	0.00921807346250842\\
54.68	0.0092180661371977\\
54.69	0.0092180587931178\\
54.7	0.00921805143009647\\
54.71	0.00921804404795987\\
54.72	0.00921803664653258\\
54.73	0.00921802922563756\\
54.74	0.00921802178509616\\
54.75	0.00921801432472812\\
54.76	0.00921800684435156\\
54.77	0.00921799934378296\\
54.78	0.0092179918228372\\
54.79	0.0092179842813275\\
54.8	0.00921797671906546\\
54.81	0.00921796913586102\\
54.82	0.00921796153152251\\
54.83	0.00921795390585657\\
54.84	0.00921794625866824\\
54.85	0.00921793858976087\\
54.86	0.00921793089893616\\
54.87	0.00921792318599419\\
54.88	0.00921791545073334\\
54.89	0.00921790769295037\\
54.9	0.00921789991244036\\
54.91	0.00921789210899674\\
54.92	0.00921788428241129\\
54.93	0.00921787643247412\\
54.94	0.00921786855897371\\
54.95	0.00921786066169687\\
54.96	0.00921785274042876\\
54.97	0.0092178447949529\\
54.98	0.00921783682505116\\
54.99	0.00921782883050378\\
55	0.00921782081108934\\
55.01	0.00921781276658482\\
55.02	0.00921780469676555\\
55.03	0.00921779660140524\\
55.04	0.009217788480276\\
55.05	0.00921778033314831\\
55.06	0.00921777215979106\\
55.07	0.00921776395997156\\
55.08	0.00921775573345551\\
55.09	0.00921774748000705\\
55.1	0.00921773919938875\\
55.11	0.00921773089136162\\
55.12	0.00921772255568512\\
55.13	0.0092177141921172\\
55.14	0.00921770580041426\\
55.15	0.0092176973803312\\
55.16	0.00921768893162144\\
55.17	0.00921768045403689\\
55.18	0.00921767194732802\\
55.19	0.00921766341124382\\
55.2	0.00921765484553188\\
55.21	0.00921764624993834\\
55.22	0.00921763762420794\\
55.23	0.00921762896808408\\
55.24	0.00921762028130874\\
55.25	0.00921761156362259\\
55.26	0.00921760281476497\\
55.27	0.00921759403447391\\
55.28	0.00921758522248617\\
55.29	0.00921757637853726\\
55.3	0.00921756750236143\\
55.31	0.00921755859369175\\
55.32	0.00921754965226009\\
55.33	0.00921754067779717\\
55.34	0.00921753167003257\\
55.35	0.00921752262869478\\
55.36	0.00921751355351121\\
55.37	0.00921750444420822\\
55.38	0.00921749530051116\\
55.39	0.00921748612214439\\
55.4	0.00921747690883133\\
55.41	0.00921746766029446\\
55.42	0.00921745837625539\\
55.43	0.00921744905643487\\
55.44	0.00921743970055284\\
55.45	0.00921743030832844\\
55.46	0.00921742087948008\\
55.47	0.00921741141372546\\
55.48	0.00921740191078159\\
55.49	0.00921739237036488\\
55.5	0.00921738279219113\\
55.51	0.00921737317597557\\
55.52	0.00921736352143296\\
55.53	0.00921735382827754\\
55.54	0.00921734409622317\\
55.55	0.0092173343249833\\
55.56	0.00921732451427103\\
55.57	0.00921731466379917\\
55.58	0.00921730477328029\\
55.59	0.00921729484242674\\
55.6	0.00921728487095071\\
55.61	0.00921727485856426\\
55.62	0.0092172648049794\\
55.63	0.00921725470990811\\
55.64	0.00921724457306238\\
55.65	0.0092172343941543\\
55.66	0.00921722417289604\\
55.67	0.00921721390899996\\
55.68	0.00921720360217862\\
55.69	0.00921719325214487\\
55.7	0.00921718285861184\\
55.71	0.00921717242129302\\
55.72	0.00921716193990231\\
55.73	0.00921715141415407\\
55.74	0.00921714084376313\\
55.75	0.00921713022844489\\
55.76	0.00921711956791533\\
55.77	0.00921710886189106\\
55.78	0.00921709811008938\\
55.79	0.00921708731222831\\
55.8	0.00921707646802661\\
55.81	0.00921706557720389\\
55.82	0.00921705463948056\\
55.83	0.00921704365457794\\
55.84	0.00921703262221828\\
55.85	0.00921702154212475\\
55.86	0.00921701041402153\\
55.87	0.00921699923763383\\
55.88	0.0092169880126879\\
55.89	0.00921697673891105\\
55.9	0.00921696541603171\\
55.91	0.00921695404377942\\
55.92	0.00921694262188486\\
55.93	0.00921693115007986\\
55.94	0.00921691962809741\\
55.95	0.00921690805567168\\
55.96	0.009216896432538\\
55.97	0.00921688475843288\\
55.98	0.00921687303309399\\
55.99	0.00921686125626017\\
56	0.00921684942767136\\
56.01	0.00921683754706864\\
56.02	0.00921682561419419\\
56.03	0.00921681362879122\\
56.04	0.00921680159060395\\
56.05	0.00921678949937757\\
56.06	0.00921677735485818\\
56.07	0.00921676515679272\\
56.08	0.0092167529049289\\
56.09	0.00921674059901511\\
56.1	0.00921672823880034\\
56.11	0.00921671582403409\\
56.12	0.00921670335446623\\
56.13	0.00921669082984688\\
56.14	0.00921667824992632\\
56.15	0.00921666561445478\\
56.16	0.00921665292318233\\
56.17	0.00921664017585868\\
56.18	0.00921662737223301\\
56.19	0.00921661451205373\\
56.2	0.00921660159506833\\
56.21	0.00921658862102307\\
56.22	0.00921657558966278\\
56.23	0.00921656250073055\\
56.24	0.00921654935396749\\
56.25	0.00921653614911238\\
56.26	0.00921652288590134\\
56.27	0.00921650956407195\\
56.28	0.00921649618337176\\
56.29	0.0092164827435588\\
56.3	0.00921646924440211\\
56.31	0.00921645568568232\\
56.32	0.00921644206719218\\
56.33	0.00921642838873722\\
56.34	0.00921641465013629\\
56.35	0.00921640085122223\\
56.36	0.00921638699184253\\
56.37	0.00921637307186001\\
56.38	0.0092163590911535\\
56.39	0.00921634504961859\\
56.4	0.00921633094716833\\
56.41	0.00921631678373406\\
56.42	0.00921630255926619\\
56.43	0.00921628827373502\\
56.44	0.00921627392713158\\
56.45	0.00921625951946852\\
56.46	0.00921624505078105\\
56.47	0.00921623052112782\\
56.48	0.00921621593059195\\
56.49	0.00921620127928199\\
56.5	0.00921618656733297\\
56.51	0.00921617179490747\\
56.52	0.00921615696219673\\
56.53	0.00921614206942175\\
56.54	0.00921612711683452\\
56.55	0.00921611210471921\\
56.56	0.00921609703339339\\
56.57	0.00921608190320937\\
56.58	0.00921606671455553\\
56.59	0.00921605146785764\\
56.6	0.00921603616358036\\
56.61	0.00921602080222866\\
56.62	0.0092160053843493\\
56.63	0.00921598991053249\\
56.64	0.0092159743814134\\
56.65	0.00921595879767386\\
56.66	0.00921594316004408\\
56.67	0.00921592746930442\\
56.68	0.00921591172628718\\
56.69	0.0092158959318785\\
56.7	0.00921588008702033\\
56.71	0.00921586419271238\\
56.72	0.00921584825001418\\
56.73	0.00921583226004728\\
56.74	0.00921581622399736\\
56.75	0.00921580014311651\\
56.76	0.00921578401872562\\
56.77	0.00921576785221669\\
56.78	0.0092157516450554\\
56.79	0.00921573539878361\\
56.8	0.009215719115022\\
56.81	0.00921570279547282\\
56.82	0.00921568644192264\\
56.83	0.0092156700562453\\
56.84	0.00921565364040481\\
56.85	0.00921563719645851\\
56.86	0.00921562072656013\\
56.87	0.00921560423296312\\
56.88	0.00921558771802402\\
56.89	0.00921557118420586\\
56.9	0.00921555463408175\\
56.91	0.00921553807033861\\
56.92	0.0092155214957809\\
56.93	0.00921550491333453\\
56.94	0.00921548832340527\\
56.95	0.00921547172598803\\
56.96	0.00921545512107772\\
56.97	0.00921543850866926\\
56.98	0.00921542188875754\\
56.99	0.00921540526133745\\
57	0.00921538862640389\\
57.01	0.00921537198395175\\
57.02	0.0092153553339759\\
57.03	0.00921533867647122\\
57.04	0.00921532201143257\\
57.05	0.00921530533885483\\
57.06	0.00921528865873284\\
57.07	0.00921527197106147\\
57.08	0.00921525527583557\\
57.09	0.00921523857304997\\
57.1	0.00921522186269952\\
57.11	0.00921520514477904\\
57.12	0.00921518841928337\\
57.13	0.00921517168620732\\
57.14	0.00921515494554572\\
57.15	0.00921513819729337\\
57.16	0.00921512144144508\\
57.17	0.00921510467799565\\
57.18	0.00921508790693988\\
57.19	0.00921507112827255\\
57.2	0.00921505434198845\\
57.21	0.00921503754808236\\
57.22	0.00921502074654906\\
57.23	0.00921500393738331\\
57.24	0.00921498712057988\\
57.25	0.00921497029613351\\
57.26	0.00921495346403897\\
57.27	0.009214936624291\\
57.28	0.00921491977688435\\
57.29	0.00921490292181374\\
57.3	0.00921488605907391\\
57.31	0.00921486918865959\\
57.32	0.00921485231056549\\
57.33	0.00921483542478633\\
57.34	0.00921481853131682\\
57.35	0.00921480163015166\\
57.36	0.00921478472128554\\
57.37	0.00921476780471316\\
57.38	0.00921475088042921\\
57.39	0.00921473394842837\\
57.4	0.00921471700870531\\
57.41	0.0092147000612547\\
57.42	0.00921468310607121\\
57.43	0.0092146661431495\\
57.44	0.00921464917248422\\
57.45	0.00921463219407002\\
57.46	0.00921461520790154\\
57.47	0.00921459821397341\\
57.48	0.00921458121228028\\
57.49	0.00921456420281677\\
57.5	0.00921454718557748\\
57.51	0.00921453016055705\\
57.52	0.00921451312775007\\
57.53	0.00921449608715116\\
57.54	0.0092144790387549\\
57.55	0.0092144619825559\\
57.56	0.00921444491854873\\
57.57	0.00921442784672799\\
57.58	0.00921441076708823\\
57.59	0.00921439367962404\\
57.6	0.00921437658432998\\
57.61	0.0092143594812006\\
57.62	0.00921434237023046\\
57.63	0.00921432525141411\\
57.64	0.00921430812474607\\
57.65	0.0092142909902209\\
57.66	0.00921427384783312\\
57.67	0.00921425669757726\\
57.68	0.00921423953944782\\
57.69	0.00921422237343933\\
57.7	0.00921420519954629\\
57.71	0.00921418801776321\\
57.72	0.00921417082808457\\
57.73	0.00921415363050486\\
57.74	0.00921413642501858\\
57.75	0.0092141192116202\\
57.76	0.00921410199030418\\
57.77	0.009214084761065\\
57.78	0.00921406752389711\\
57.79	0.00921405027879498\\
57.8	0.00921403302575305\\
57.81	0.00921401576476577\\
57.82	0.00921399849582756\\
57.83	0.00921398121893286\\
57.84	0.0092139639340761\\
57.85	0.00921394664125169\\
57.86	0.00921392934045406\\
57.87	0.0092139120316776\\
57.88	0.00921389471491672\\
57.89	0.00921387739016581\\
57.9	0.00921386005741927\\
57.91	0.00921384271667148\\
57.92	0.00921382536791682\\
57.93	0.00921380801114966\\
57.94	0.00921379064636436\\
57.95	0.00921377327355529\\
57.96	0.0092137558927168\\
57.97	0.00921373850384323\\
57.98	0.00921372110692894\\
57.99	0.00921370370196826\\
58	0.00921368628895552\\
58.01	0.00921366886788504\\
58.02	0.00921365143875114\\
58.03	0.00921363400154813\\
58.04	0.00921361655627032\\
58.05	0.00921359910291202\\
58.06	0.00921358164146751\\
58.07	0.00921356417193108\\
58.08	0.00921354669429702\\
58.09	0.0092135292085596\\
58.1	0.00921351171471309\\
58.11	0.00921349421275176\\
58.12	0.00921347670266986\\
58.13	0.00921345918446165\\
58.14	0.00921344165812137\\
58.15	0.00921342412364325\\
58.16	0.00921340658102155\\
58.17	0.00921338903025047\\
58.18	0.00921337147132425\\
58.19	0.00921335390423709\\
58.2	0.00921333632898321\\
58.21	0.00921331874555681\\
58.22	0.00921330115395208\\
58.23	0.00921328355416322\\
58.24	0.00921326594618441\\
58.25	0.00921324833000983\\
58.26	0.00921323070563364\\
58.27	0.00921321307305002\\
58.28	0.00921319543225312\\
58.29	0.0092131777832371\\
58.3	0.00921316012599611\\
58.31	0.00921314246052428\\
58.32	0.00921312478681574\\
58.33	0.00921310710486463\\
58.34	0.00921308941466507\\
58.35	0.00921307171621117\\
58.36	0.00921305400949705\\
58.37	0.00921303629451679\\
58.38	0.00921301857126451\\
58.39	0.00921300083973429\\
58.4	0.00921298309992021\\
58.41	0.00921296535181635\\
58.42	0.00921294759541679\\
58.43	0.00921292983071558\\
58.44	0.00921291205770679\\
58.45	0.00921289427638446\\
58.46	0.00921287648674265\\
58.47	0.00921285868877539\\
58.48	0.00921284088247671\\
58.49	0.00921282306784064\\
58.5	0.0092128052448612\\
58.51	0.0092127874135324\\
58.52	0.00921276957384826\\
58.53	0.00921275172580276\\
58.54	0.00921273386938991\\
58.55	0.00921271600460368\\
58.56	0.00921269813143807\\
58.57	0.00921268024988705\\
58.58	0.00921266235994458\\
58.59	0.00921264446160462\\
58.6	0.00921262655486114\\
58.61	0.00921260863970807\\
58.62	0.00921259071613936\\
58.63	0.00921257278414895\\
58.64	0.00921255484373077\\
58.65	0.00921253689487872\\
58.66	0.00921251893758674\\
58.67	0.00921250097184873\\
58.68	0.00921248299765859\\
58.69	0.00921246501501022\\
58.7	0.0092124470238975\\
58.71	0.00921242902431433\\
58.72	0.00921241101625457\\
58.73	0.00921239299971209\\
58.74	0.00921237497468076\\
58.75	0.00921235694115442\\
58.76	0.00921233889912694\\
58.77	0.00921232084859215\\
58.78	0.00921230278954389\\
58.79	0.00921228472197599\\
58.8	0.00921226664588226\\
58.81	0.00921224856125654\\
58.82	0.00921223046809261\\
58.83	0.00921221236638428\\
58.84	0.00921219425612536\\
58.85	0.00921217613730962\\
58.86	0.00921215800993085\\
58.87	0.00921213987398283\\
58.88	0.00921212172945932\\
58.89	0.00921210357635408\\
58.9	0.00921208541466088\\
58.91	0.00921206724437344\\
58.92	0.00921204906548552\\
58.93	0.00921203087799084\\
58.94	0.00921201268188315\\
58.95	0.00921199447715615\\
58.96	0.00921197626380355\\
58.97	0.00921195804181908\\
58.98	0.00921193981119641\\
58.99	0.00921192157192925\\
59	0.00921190332401129\\
59.01	0.00921188506743619\\
59.02	0.00921186680219763\\
59.03	0.00921184852828928\\
59.04	0.00921183024570479\\
59.05	0.00921181195443781\\
59.06	0.00921179365448199\\
59.07	0.00921177534583096\\
59.08	0.00921175702847836\\
59.09	0.0092117387024178\\
59.1	0.0092117203676429\\
59.11	0.00921170202414727\\
59.12	0.00921168367192451\\
59.13	0.00921166531096822\\
59.14	0.00921164694127197\\
59.15	0.00921162856282936\\
59.16	0.00921161017563395\\
59.17	0.00921159177967931\\
59.18	0.009211573374959\\
59.19	0.00921155496146656\\
59.2	0.00921153653919555\\
59.21	0.0092115181081395\\
59.22	0.00921149966829195\\
59.23	0.0092114812196464\\
59.24	0.00921146276219639\\
59.25	0.00921144429593541\\
59.26	0.00921142582085697\\
59.27	0.00921140733695456\\
59.28	0.00921138884422167\\
59.29	0.00921137034265179\\
59.3	0.00921135183223837\\
59.31	0.00921133331297489\\
59.32	0.00921131478485479\\
59.33	0.00921129624787155\\
59.34	0.00921127770201858\\
59.35	0.00921125914728934\\
59.36	0.00921124058367725\\
59.37	0.00921122201117573\\
59.38	0.00921120342977819\\
59.39	0.00921118483947804\\
59.4	0.00921116624026868\\
59.41	0.0092111476321435\\
59.42	0.00921112901509588\\
59.43	0.00921111038911921\\
59.44	0.00921109175420684\\
59.45	0.00921107311035215\\
59.46	0.00921105445754847\\
59.47	0.00921103579578917\\
59.48	0.00921101712506758\\
59.49	0.00921099844537704\\
59.5	0.00921097975671086\\
59.51	0.00921096105906236\\
59.52	0.00921094235242486\\
59.53	0.00921092363679165\\
59.54	0.00921090491215602\\
59.55	0.00921088617851128\\
59.56	0.00921086743585068\\
59.57	0.00921084868416751\\
59.58	0.00921082992345503\\
59.59	0.00921081115370649\\
59.6	0.00921079237491514\\
59.61	0.00921077358707422\\
59.62	0.00921075479017697\\
59.63	0.00921073598421661\\
59.64	0.00921071716918636\\
59.65	0.00921069834507943\\
59.66	0.00921067951188901\\
59.67	0.00921066066960832\\
59.68	0.00921064181823052\\
59.69	0.00921062295774881\\
59.7	0.00921060408815634\\
59.71	0.0092105852094463\\
59.72	0.00921056632161182\\
59.73	0.00921054742464606\\
59.74	0.00921052851854216\\
59.75	0.00921050960329326\\
59.76	0.00921049067889247\\
59.77	0.00921047174533291\\
59.78	0.00921045280260768\\
59.79	0.0092104338507099\\
59.8	0.00921041488963266\\
59.81	0.00921039591936903\\
59.82	0.0092103769399121\\
59.83	0.00921035795125493\\
59.84	0.00921033895339059\\
59.85	0.00921031994631213\\
59.86	0.00921030093001258\\
59.87	0.009210281904485\\
59.88	0.00921026286972241\\
59.89	0.00921024382571783\\
59.9	0.00921022477246427\\
59.91	0.00921020570995474\\
59.92	0.00921018663818223\\
59.93	0.00921016755713974\\
59.94	0.00921014846682024\\
59.95	0.00921012936721671\\
59.96	0.0092101102583221\\
59.97	0.00921009114012939\\
59.98	0.0092100720126315\\
59.99	0.0092100528758214\\
60	0.009210033729692\\
60.01	0.00921001457423623\\
60.02	0.009209995409447\\
60.03	0.00920997623531722\\
60.04	0.00920995705183979\\
60.05	0.0092099378590076\\
60.06	0.00920991865681354\\
60.07	0.00920989944525047\\
60.08	0.00920988022431126\\
60.09	0.00920986099398878\\
60.1	0.00920984175427585\\
60.11	0.00920982250516534\\
60.12	0.00920980324665007\\
60.13	0.00920978397872285\\
60.14	0.00920976470137652\\
60.15	0.00920974541460387\\
60.16	0.00920972611839771\\
60.17	0.00920970681275082\\
60.18	0.00920968749765598\\
60.19	0.00920966817310597\\
60.2	0.00920964883909356\\
60.21	0.00920962949561149\\
60.22	0.00920961014265251\\
60.23	0.00920959078020937\\
60.24	0.00920957140827479\\
60.25	0.0092095520268415\\
60.26	0.0092095326359022\\
60.27	0.00920951323544961\\
60.28	0.00920949382547641\\
60.29	0.0092094744059753\\
60.3	0.00920945497693895\\
60.31	0.00920943553836004\\
60.32	0.00920941609023121\\
60.33	0.00920939663254513\\
60.34	0.00920937716529444\\
60.35	0.00920935768847178\\
60.36	0.00920933820206976\\
60.37	0.009209318706081\\
60.38	0.00920929920049812\\
60.39	0.00920927968531372\\
60.4	0.00920926016052037\\
60.41	0.00920924062611067\\
60.42	0.00920922108207719\\
60.43	0.00920920152841249\\
60.44	0.00920918196510912\\
60.45	0.00920916239215964\\
60.46	0.00920914280955657\\
60.47	0.00920912321729245\\
60.48	0.0092091036153598\\
60.49	0.00920908400375111\\
60.5	0.00920906438245891\\
60.51	0.00920904475147567\\
60.52	0.00920902511079387\\
60.53	0.00920900546040601\\
60.54	0.00920898580030452\\
60.55	0.00920896613048188\\
60.56	0.00920894645093053\\
60.57	0.0092089267616429\\
60.58	0.00920890706261142\\
60.59	0.00920888735382852\\
60.6	0.00920886763528659\\
60.61	0.00920884790697804\\
60.62	0.00920882816889526\\
60.63	0.00920880842103063\\
60.64	0.00920878866337652\\
60.65	0.0092087688959253\\
60.66	0.00920874911866931\\
60.67	0.00920872933160091\\
60.68	0.00920870953471242\\
60.69	0.00920868972799617\\
60.7	0.00920866991144448\\
60.71	0.00920865008504965\\
60.72	0.00920863024880397\\
60.73	0.00920861040269974\\
60.74	0.00920859054672923\\
60.75	0.00920857068088471\\
60.76	0.00920855080515844\\
60.77	0.00920853091954266\\
60.78	0.00920851102402961\\
60.79	0.00920849111861153\\
60.8	0.00920847120328063\\
60.81	0.00920845127802911\\
60.82	0.0092084313428492\\
60.83	0.00920841139773306\\
60.84	0.00920839144267289\\
60.85	0.00920837147766084\\
60.86	0.00920835150268909\\
60.87	0.00920833151774979\\
60.88	0.00920831152283507\\
60.89	0.00920829151793706\\
60.9	0.00920827150304789\\
60.91	0.00920825147815967\\
60.92	0.0092082314432645\\
60.93	0.00920821139835447\\
60.94	0.00920819134342166\\
60.95	0.00920817127845814\\
60.96	0.00920815120345597\\
60.97	0.00920813111840721\\
60.98	0.0092081110233039\\
60.99	0.00920809091813806\\
61	0.0092080708029017\\
61.01	0.00920805067758686\\
61.02	0.00920803054218553\\
61.03	0.00920801039668968\\
61.04	0.00920799024109131\\
61.05	0.00920797007538238\\
61.06	0.00920794989955486\\
61.07	0.00920792971360068\\
61.08	0.00920790951751179\\
61.09	0.00920788931128011\\
61.1	0.00920786909489756\\
61.11	0.00920784886835605\\
61.12	0.00920782863164747\\
61.13	0.00920780838476372\\
61.14	0.00920778812769665\\
61.15	0.00920776786043814\\
61.16	0.00920774758298004\\
61.17	0.0092077272953142\\
61.18	0.00920770699743244\\
61.19	0.00920768668932658\\
61.2	0.00920766637098845\\
61.21	0.00920764604240982\\
61.22	0.0092076257035825\\
61.23	0.00920760535449827\\
61.24	0.00920758499514888\\
61.25	0.00920756462552611\\
61.26	0.00920754424562168\\
61.27	0.00920752385542734\\
61.28	0.0092075034549348\\
61.29	0.00920748304413579\\
61.3	0.009207462623022\\
61.31	0.00920744219158512\\
61.32	0.00920742174981684\\
61.33	0.00920740129770881\\
61.34	0.00920738083525269\\
61.35	0.00920736036244014\\
61.36	0.00920733987926278\\
61.37	0.00920731938571223\\
61.38	0.00920729888178011\\
61.39	0.00920727836745802\\
61.4	0.00920725784273754\\
61.41	0.00920723730761026\\
61.42	0.00920721676206773\\
61.43	0.00920719620610152\\
61.44	0.00920717563970316\\
61.45	0.00920715506286418\\
61.46	0.00920713447557611\\
61.47	0.00920711387783044\\
61.48	0.00920709326961868\\
61.49	0.00920707265093231\\
61.5	0.00920705202176281\\
61.51	0.00920703138210162\\
61.52	0.00920701073194021\\
61.53	0.00920699007127\\
61.54	0.00920696940008242\\
61.55	0.00920694871836889\\
61.56	0.00920692802612079\\
61.57	0.00920690732332953\\
61.58	0.00920688660998648\\
61.59	0.009206865886083\\
61.6	0.00920684515161044\\
61.61	0.00920682440656015\\
61.62	0.00920680365092344\\
61.63	0.00920678288469164\\
61.64	0.00920676210785606\\
61.65	0.00920674132040796\\
61.66	0.00920672052233865\\
61.67	0.00920669971363937\\
61.68	0.00920667889430138\\
61.69	0.00920665806431593\\
61.7	0.00920663722367424\\
61.71	0.00920661637236752\\
61.72	0.00920659551038699\\
61.73	0.00920657463772382\\
61.74	0.00920655375436918\\
61.75	0.00920653286031426\\
61.76	0.00920651195555019\\
61.77	0.00920649104006811\\
61.78	0.00920647011385915\\
61.79	0.00920644917691443\\
61.8	0.00920642822922503\\
61.81	0.00920640727078204\\
61.82	0.00920638630157654\\
61.83	0.00920636532159959\\
61.84	0.00920634433084222\\
61.85	0.00920632332929547\\
61.86	0.00920630231695037\\
61.87	0.0092062812937979\\
61.88	0.00920626025982908\\
61.89	0.00920623921503487\\
61.9	0.00920621815940624\\
61.91	0.00920619709293413\\
61.92	0.00920617601560949\\
61.93	0.00920615492742325\\
61.94	0.0092061338283663\\
61.95	0.00920611271842954\\
61.96	0.00920609159760386\\
61.97	0.00920607046588012\\
61.98	0.00920604932324918\\
61.99	0.00920602816970188\\
62	0.00920600700522903\\
62.01	0.00920598582982146\\
62.02	0.00920596464346994\\
62.03	0.00920594344616528\\
62.04	0.00920592223789824\\
62.05	0.00920590101865956\\
62.06	0.00920587978844\\
62.07	0.00920585854723026\\
62.08	0.00920583729502107\\
62.09	0.00920581603180311\\
62.1	0.00920579475756707\\
62.11	0.0092057734723036\\
62.12	0.00920575217600337\\
62.13	0.009205730868657\\
62.14	0.00920570955025511\\
62.15	0.00920568822078832\\
62.16	0.0092056668802472\\
62.17	0.00920564552862233\\
62.18	0.00920562416590427\\
62.19	0.00920560279208357\\
62.2	0.00920558140715075\\
62.21	0.00920556001109633\\
62.22	0.00920553860391079\\
62.23	0.00920551718558462\\
62.24	0.0092054957561083\\
62.25	0.00920547431547225\\
62.26	0.00920545286366693\\
62.27	0.00920543140068275\\
62.28	0.0092054099265101\\
62.29	0.00920538844113937\\
62.3	0.00920536694456094\\
62.31	0.00920534543676515\\
62.32	0.00920532391774234\\
62.33	0.00920530238748284\\
62.34	0.00920528084597694\\
62.35	0.00920525929321493\\
62.36	0.00920523772918708\\
62.37	0.00920521615388365\\
62.38	0.00920519456729486\\
62.39	0.00920517296941095\\
62.4	0.00920515136022211\\
62.41	0.00920512973971852\\
62.42	0.00920510810789036\\
62.43	0.00920508646472777\\
62.44	0.0092050648102209\\
62.45	0.00920504314435985\\
62.46	0.00920502146713472\\
62.47	0.0092049997785356\\
62.48	0.00920497807855255\\
62.49	0.00920495636717561\\
62.5	0.00920493464439481\\
62.51	0.00920491291020016\\
62.52	0.00920489116458165\\
62.53	0.00920486940752925\\
62.54	0.00920484763903292\\
62.55	0.00920482585908259\\
62.56	0.00920480406766819\\
62.57	0.00920478226477961\\
62.58	0.00920476045040673\\
62.59	0.00920473862453941\\
62.6	0.0092047167871675\\
62.61	0.00920469493828082\\
62.62	0.00920467307786918\\
62.63	0.00920465120592236\\
62.64	0.00920462932243012\\
62.65	0.00920460742738222\\
62.66	0.00920458552076838\\
62.67	0.00920456360257831\\
62.68	0.0092045416728017\\
62.69	0.00920451973142821\\
62.7	0.0092044977784475\\
62.71	0.00920447581384919\\
62.72	0.00920445383762289\\
62.73	0.0092044318497582\\
62.74	0.00920440985024466\\
62.75	0.00920438783907184\\
62.76	0.00920436581622926\\
62.77	0.00920434378170642\\
62.78	0.00920432173549281\\
62.79	0.0092042996775779\\
62.8	0.00920427760795112\\
62.81	0.0092042555266019\\
62.82	0.00920423343351963\\
62.83	0.0092042113286937\\
62.84	0.00920418921211347\\
62.85	0.00920416708376826\\
62.86	0.00920414494364739\\
62.87	0.00920412279174015\\
62.88	0.00920410062803582\\
62.89	0.00920407845252364\\
62.9	0.00920405626519282\\
62.91	0.00920403406603258\\
62.92	0.0092040118550321\\
62.93	0.00920398963218053\\
62.94	0.00920396739746701\\
62.95	0.00920394515088063\\
62.96	0.00920392289241051\\
62.97	0.00920390062204569\\
62.98	0.00920387833977522\\
62.99	0.00920385604558811\\
63	0.00920383373947336\\
63.01	0.00920381142141993\\
63.02	0.00920378909141678\\
63.03	0.00920376674945283\\
63.04	0.00920374439551696\\
63.05	0.00920372202959805\\
63.06	0.00920369965168495\\
63.07	0.00920367726176649\\
63.08	0.00920365485983145\\
63.09	0.00920363244586861\\
63.1	0.00920361001986672\\
63.11	0.0092035875818145\\
63.12	0.00920356513170064\\
63.13	0.00920354266951382\\
63.14	0.00920352019524268\\
63.15	0.00920349770887583\\
63.16	0.00920347521040188\\
63.17	0.00920345269980937\\
63.18	0.00920343017708687\\
63.19	0.00920340764222286\\
63.2	0.00920338509520584\\
63.21	0.00920336253602426\\
63.22	0.00920333996466657\\
63.23	0.00920331738112115\\
63.24	0.00920329478537638\\
63.25	0.00920327217742061\\
63.26	0.00920324955724216\\
63.27	0.00920322692482932\\
63.28	0.00920320428017035\\
63.29	0.00920318162325349\\
63.3	0.00920315895406693\\
63.31	0.00920313627259886\\
63.32	0.00920311357883742\\
63.33	0.00920309087277073\\
63.34	0.00920306815438687\\
63.35	0.0092030454236739\\
63.36	0.00920302268061985\\
63.37	0.00920299992521271\\
63.38	0.00920297715744046\\
63.39	0.00920295437729103\\
63.4	0.00920293158475232\\
63.41	0.0092029087798122\\
63.42	0.00920288596245853\\
63.43	0.0092028631326791\\
63.44	0.00920284029046171\\
63.45	0.0092028174357941\\
63.46	0.00920279456866398\\
63.47	0.00920277168905903\\
63.48	0.00920274879696691\\
63.49	0.00920272589237523\\
63.5	0.00920270297527158\\
63.51	0.00920268004564352\\
63.52	0.00920265710347854\\
63.53	0.00920263414876415\\
63.54	0.00920261118148778\\
63.55	0.00920258820163687\\
63.56	0.00920256520919877\\
63.57	0.00920254220416085\\
63.58	0.00920251918651041\\
63.59	0.00920249615623474\\
63.6	0.00920247311332106\\
63.61	0.00920245005775659\\
63.62	0.00920242698952848\\
63.63	0.00920240390862389\\
63.64	0.0092023808150299\\
63.65	0.00920235770873356\\
63.66	0.00920233458972191\\
63.67	0.00920231145798193\\
63.68	0.00920228831350057\\
63.69	0.00920226515626473\\
63.7	0.00920224198626128\\
63.71	0.00920221880347706\\
63.72	0.00920219560789887\\
63.73	0.00920217239951346\\
63.74	0.00920214917830754\\
63.75	0.00920212594426779\\
63.76	0.00920210269738085\\
63.77	0.00920207943763331\\
63.78	0.00920205616501174\\
63.79	0.00920203287950264\\
63.8	0.00920200958109249\\
63.81	0.00920198626976773\\
63.82	0.00920196294551474\\
63.83	0.00920193960831988\\
63.84	0.00920191625816945\\
63.85	0.00920189289504973\\
63.86	0.00920186951894692\\
63.87	0.00920184612984722\\
63.88	0.00920182272773676\\
63.89	0.00920179931260162\\
63.9	0.00920177588442786\\
63.91	0.00920175244320148\\
63.92	0.00920172898890845\\
63.93	0.00920170552153467\\
63.94	0.00920168204106601\\
63.95	0.00920165854748829\\
63.96	0.0092016350407873\\
63.97	0.00920161152094876\\
63.98	0.00920158798795835\\
63.99	0.00920156444180172\\
64	0.00920154088246446\\
64.01	0.00920151730993209\\
64.02	0.00920149372419012\\
64.03	0.00920147012522399\\
64.04	0.0092014465130191\\
64.05	0.00920142288756079\\
64.06	0.00920139924883436\\
64.07	0.00920137559682506\\
64.08	0.00920135193151808\\
64.09	0.00920132825289858\\
64.1	0.00920130456095163\\
64.11	0.00920128085566231\\
64.12	0.00920125713701557\\
64.13	0.00920123340499638\\
64.14	0.00920120965958962\\
64.15	0.00920118590078012\\
64.16	0.00920116212855266\\
64.17	0.00920113834289197\\
64.18	0.00920111454378273\\
64.19	0.00920109073120953\\
64.2	0.00920106690515697\\
64.21	0.00920104306560954\\
64.22	0.00920101921255169\\
64.23	0.00920099534596782\\
64.24	0.00920097146584226\\
64.25	0.00920094757215931\\
64.26	0.00920092366490319\\
64.27	0.00920089974405805\\
64.28	0.00920087580960802\\
64.29	0.00920085186153714\\
64.3	0.0092008278998294\\
64.31	0.00920080392446873\\
64.32	0.00920077993543901\\
64.33	0.00920075593272405\\
64.34	0.00920073191630759\\
64.35	0.00920070788617333\\
64.36	0.00920068384230488\\
64.37	0.00920065978468582\\
64.38	0.00920063571329966\\
64.39	0.00920061162812982\\
64.4	0.00920058752915968\\
64.41	0.00920056341637255\\
64.42	0.00920053928975169\\
64.43	0.00920051514928027\\
64.44	0.0092004909949414\\
64.45	0.00920046682671815\\
64.46	0.00920044264459349\\
64.47	0.00920041844855034\\
64.48	0.00920039423857155\\
64.49	0.00920037001463991\\
64.5	0.00920034577673812\\
64.51	0.00920032152484884\\
64.52	0.00920029725895463\\
64.53	0.009200272979038\\
64.54	0.0092002486850814\\
64.55	0.00920022437706718\\
64.56	0.00920020005497764\\
64.57	0.00920017571879501\\
64.58	0.00920015136850142\\
64.59	0.00920012700407896\\
64.6	0.00920010262550964\\
64.61	0.00920007823277538\\
64.62	0.00920005382585804\\
64.63	0.00920002940473941\\
64.64	0.00920000496940118\\
64.65	0.00919998051982498\\
64.66	0.00919995605599238\\
64.67	0.00919993157788485\\
64.68	0.0091999070854838\\
64.69	0.00919988257877053\\
64.7	0.0091998580577263\\
64.71	0.00919983352233227\\
64.72	0.00919980897256953\\
64.73	0.00919978440841909\\
64.74	0.00919975982986186\\
64.75	0.0091997352368787\\
64.76	0.00919971062945036\\
64.77	0.00919968600755754\\
64.78	0.00919966137118082\\
64.79	0.00919963672030073\\
64.8	0.00919961205489769\\
64.81	0.00919958737495205\\
64.82	0.00919956268044408\\
64.83	0.00919953797135396\\
64.84	0.00919951324766177\\
64.85	0.00919948850934753\\
64.86	0.00919946375639116\\
64.87	0.00919943898877248\\
64.88	0.00919941420647124\\
64.89	0.0091993894094671\\
64.9	0.00919936459773963\\
64.91	0.0091993397712683\\
64.92	0.0091993149300325\\
64.93	0.00919929007401154\\
64.94	0.00919926520318461\\
64.95	0.00919924031753084\\
64.96	0.00919921541702925\\
64.97	0.00919919050165877\\
64.98	0.00919916557139823\\
64.99	0.00919914062622639\\
65	0.0091991156661219\\
65.01	0.00919909069106331\\
65.02	0.00919906570102908\\
65.03	0.00919904069599758\\
65.04	0.00919901567594709\\
65.05	0.00919899064085576\\
65.06	0.0091989655907017\\
65.07	0.00919894052546286\\
65.08	0.00919891544511714\\
65.09	0.00919889034964232\\
65.1	0.00919886523901608\\
65.11	0.00919884011321601\\
65.12	0.0091988149722196\\
65.13	0.00919878981600423\\
65.14	0.00919876464454719\\
65.15	0.00919873945782565\\
65.16	0.00919871425581672\\
65.17	0.00919868903849736\\
65.18	0.00919866380584445\\
65.19	0.00919863855783478\\
65.2	0.00919861329444502\\
65.21	0.00919858801565173\\
65.22	0.00919856272143138\\
65.23	0.00919853741176034\\
65.24	0.00919851208661486\\
65.25	0.00919848674597109\\
65.26	0.00919846138980509\\
65.27	0.00919843601809279\\
65.28	0.00919841063081004\\
65.29	0.00919838522793255\\
65.3	0.00919835980943597\\
65.31	0.00919833437529579\\
65.32	0.00919830892548744\\
65.33	0.00919828345998623\\
65.34	0.00919825797876733\\
65.35	0.00919823248180585\\
65.36	0.00919820696907676\\
65.37	0.00919818144055493\\
65.38	0.00919815589621515\\
65.39	0.00919813033603205\\
65.4	0.00919810475998019\\
65.41	0.009198079168034\\
65.42	0.00919805356016783\\
65.43	0.00919802793635589\\
65.44	0.0091980022965723\\
65.45	0.00919797664079107\\
65.46	0.00919795096898608\\
65.47	0.00919792528113114\\
65.48	0.00919789957719992\\
65.49	0.009197873857166\\
65.5	0.00919784812100284\\
65.51	0.00919782236868378\\
65.52	0.0091977966001821\\
65.53	0.00919777081547092\\
65.54	0.00919774501452328\\
65.55	0.0091977191973121\\
65.56	0.00919769336381021\\
65.57	0.0091976675139903\\
65.58	0.00919764164782501\\
65.59	0.00919761576528682\\
65.6	0.00919758986634813\\
65.61	0.00919756395098122\\
65.62	0.0091975380191583\\
65.63	0.00919751207085143\\
65.64	0.00919748610603261\\
65.65	0.00919746012467369\\
65.66	0.00919743412674648\\
65.67	0.00919740811222263\\
65.68	0.00919738208107372\\
65.69	0.00919735603327123\\
65.7	0.00919732996878653\\
65.71	0.00919730388759091\\
65.72	0.00919727778965554\\
65.73	0.00919725167495152\\
65.74	0.00919722554344983\\
65.75	0.00919719939512138\\
65.76	0.00919717322993698\\
65.77	0.00919714704786734\\
65.78	0.00919712084888309\\
65.79	0.00919709463295478\\
65.8	0.00919706840005285\\
65.81	0.00919704215014768\\
65.82	0.00919701588320956\\
65.83	0.00919698959920869\\
65.84	0.00919696329811519\\
65.85	0.00919693697989913\\
65.86	0.00919691064453046\\
65.87	0.0091968842919791\\
65.88	0.00919685792221487\\
65.89	0.00919683153520753\\
65.9	0.00919680513092677\\
65.91	0.00919677870934223\\
65.92	0.00919675227042347\\
65.93	0.00919672581414\\
65.94	0.00919669934046126\\
65.95	0.00919667284935667\\
65.96	0.00919664634079556\\
65.97	0.00919661981474724\\
65.98	0.00919659327118096\\
65.99	0.00919656671006593\\
66	0.00919654013137135\\
66.01	0.00919651353506634\\
66.02	0.00919648692112003\\
66.03	0.00919646028950149\\
66.04	0.0091964336401798\\
66.05	0.00919640697312399\\
66.06	0.00919638028830311\\
66.07	0.00919635358568617\\
66.08	0.00919632686524218\\
66.09	0.00919630012694018\\
66.1	0.00919627337074918\\
66.11	0.0091962465966382\\
66.12	0.0091962198045763\\
66.13	0.00919619299453254\\
66.14	0.00919616616647601\\
66.15	0.00919613932037584\\
66.16	0.00919611245620119\\
66.17	0.00919608557392125\\
66.18	0.00919605867350529\\
66.19	0.00919603175492262\\
66.2	0.00919600481814261\\
66.21	0.00919597786313471\\
66.22	0.00919595088986844\\
66.23	0.00919592389831341\\
66.24	0.00919589688843932\\
66.25	0.00919586986021598\\
66.26	0.0091958428136133\\
66.27	0.0091958157486013\\
66.28	0.00919578866515015\\
66.29	0.00919576156323014\\
66.3	0.0091957344428117\\
66.31	0.00919570730386543\\
66.32	0.00919568014636209\\
66.33	0.0091956529702726\\
66.34	0.00919562577556808\\
66.35	0.00919559856221986\\
66.36	0.00919557133019943\\
66.37	0.00919554407947855\\
66.38	0.00919551681002919\\
66.39	0.00919548952182356\\
66.4	0.00919546221483413\\
66.41	0.00919543488903364\\
66.42	0.00919540754439511\\
66.43	0.00919538018089185\\
66.44	0.0091953527984975\\
66.45	0.00919532539718601\\
66.46	0.00919529797693166\\
66.47	0.00919527053770911\\
66.48	0.00919524307949338\\
66.49	0.00919521560225988\\
66.5	0.00919518810598441\\
66.51	0.00919516059064323\\
66.52	0.009195133056213\\
66.53	0.00919510550267087\\
66.54	0.00919507792999445\\
66.55	0.00919505033816185\\
66.56	0.00919502272715171\\
66.57	0.00919499509694321\\
66.58	0.00919496744751606\\
66.59	0.00919493977885059\\
66.6	0.00919491209092772\\
66.61	0.009194884383729\\
66.62	0.00919485665723662\\
66.63	0.00919482891143348\\
66.64	0.00919480114630315\\
66.65	0.00919477336182994\\
66.66	0.00919474555799892\\
66.67	0.00919471773479593\\
66.68	0.00919468989220766\\
66.69	0.00919466203022158\\
66.7	0.00919463414882608\\
66.71	0.00919460624801044\\
66.72	0.00919457832776486\\
66.73	0.00919455038808052\\
66.74	0.0091945224289496\\
66.75	0.00919449445036531\\
66.76	0.00919446645232192\\
66.77	0.00919443843481483\\
66.78	0.00919441039784056\\
66.79	0.00919438234139683\\
66.8	0.00919435426548258\\
66.81	0.009194326170098\\
66.82	0.00919429805524459\\
66.83	0.0091942699209252\\
66.84	0.00919424176714409\\
66.85	0.00919421359390693\\
66.86	0.00919418540122087\\
66.87	0.00919415718909463\\
66.88	0.00919412895753847\\
66.89	0.0091941007065643\\
66.9	0.00919407243618572\\
66.91	0.00919404414641805\\
66.92	0.00919401583727842\\
66.93	0.00919398750878579\\
66.94	0.00919395916096105\\
66.95	0.00919393079382704\\
66.96	0.00919390240740862\\
66.97	0.00919387400173276\\
66.98	0.00919384557682856\\
66.99	0.00919381713272737\\
67	0.0091937886694628\\
67.01	0.00919376018707084\\
67.02	0.0091937316855899\\
67.03	0.00919370316506089\\
67.04	0.0091936746255273\\
67.05	0.0091936460670353\\
67.06	0.00919361748963376\\
67.07	0.00919358889337439\\
67.08	0.0091935602783118\\
67.09	0.00919353164450359\\
67.1	0.00919350299192515\\
67.11	0.00919347432053666\\
67.12	0.00919344563029793\\
67.13	0.00919341692116841\\
67.14	0.0091933881931072\\
67.15	0.00919335944607302\\
67.16	0.0091933306800242\\
67.17	0.00919330189491873\\
67.18	0.00919327309071415\\
67.19	0.00919324426736767\\
67.2	0.00919321542483606\\
67.21	0.00919318656307572\\
67.22	0.00919315768204262\\
67.23	0.00919312878169231\\
67.24	0.00919309986197995\\
67.25	0.00919307092286023\\
67.26	0.00919304196428746\\
67.27	0.00919301298621546\\
67.28	0.00919298398859764\\
67.29	0.00919295497138696\\
67.3	0.0091929259345359\\
67.31	0.00919289687799649\\
67.32	0.00919286780172029\\
67.33	0.00919283870565839\\
67.34	0.00919280958976138\\
67.35	0.00919278045397938\\
67.36	0.009192751298262\\
67.37	0.00919272212255834\\
67.38	0.00919269292681701\\
67.39	0.00919266371098608\\
67.4	0.00919263447501311\\
67.41	0.00919260521884511\\
67.42	0.00919257594242857\\
67.43	0.00919254664570942\\
67.44	0.00919251732863304\\
67.45	0.00919248799114423\\
67.46	0.00919245863318724\\
67.47	0.00919242925470573\\
67.48	0.00919239985564276\\
67.49	0.00919237043594082\\
67.5	0.00919234099554177\\
67.51	0.00919231153438687\\
67.52	0.00919228205241675\\
67.53	0.00919225254957141\\
67.54	0.00919222302579022\\
67.55	0.00919219348101189\\
67.56	0.00919216391517448\\
67.57	0.00919213432821536\\
67.58	0.00919210472007126\\
67.59	0.00919207509067819\\
67.6	0.00919204543997148\\
67.61	0.00919201576788575\\
67.62	0.00919198607435489\\
67.63	0.0091919563593121\\
67.64	0.00919192662268979\\
67.65	0.00919189686441968\\
67.66	0.00919186708443268\\
67.67	0.00919183728265896\\
67.68	0.0091918074590279\\
67.69	0.00919177761346811\\
67.7	0.00919174774590735\\
67.71	0.00919171785627262\\
67.72	0.00919168794449006\\
67.73	0.00919165801048499\\
67.74	0.00919162805418186\\
67.75	0.00919159807550428\\
67.76	0.00919156807437499\\
67.77	0.00919153805071582\\
67.78	0.00919150800444772\\
67.79	0.00919147793549073\\
67.8	0.00919144784376395\\
67.81	0.00919141772918555\\
67.82	0.00919138759167276\\
67.83	0.00919135743114184\\
67.84	0.00919132724750806\\
67.85	0.00919129704068571\\
67.86	0.00919126681058807\\
67.87	0.00919123655712741\\
67.88	0.00919120628021494\\
67.89	0.00919117597976085\\
67.9	0.00919114565567425\\
67.91	0.00919111530786317\\
67.92	0.00919108493623455\\
67.93	0.00919105454069422\\
67.94	0.00919102412114686\\
67.95	0.00919099367749606\\
67.96	0.00919096320964419\\
67.97	0.00919093271749249\\
67.98	0.009190902200941\\
67.99	0.00919087165988852\\
68	0.00919084109423267\\
68.01	0.0091908105038698\\
68.02	0.00919077988869499\\
68.03	0.00919074924860208\\
68.04	0.00919071858348357\\
68.05	0.00919068789323066\\
68.06	0.00919065717773324\\
68.07	0.00919062643687981\\
68.08	0.0091905956705575\\
68.09	0.0091905648786521\\
68.1	0.00919053406104792\\
68.11	0.00919050321762787\\
68.12	0.00919047234827343\\
68.13	0.00919044145286457\\
68.14	0.0091904105312798\\
68.15	0.00919037958339607\\
68.16	0.00919034860908884\\
68.17	0.00919031760823201\\
68.18	0.00919028658069787\\
68.19	0.00919025552635713\\
68.2	0.00919022444507888\\
68.21	0.00919019333673055\\
68.22	0.0091901622011779\\
68.23	0.00919013103828501\\
68.24	0.00919009984791423\\
68.25	0.00919006862992618\\
68.26	0.00919003738417969\\
68.27	0.00919000611053183\\
68.28	0.00918997480883783\\
68.29	0.00918994347895109\\
68.3	0.00918991212072313\\
68.31	0.0091898807340036\\
68.32	0.0091898493186402\\
68.33	0.00918981787447871\\
68.34	0.00918978640136291\\
68.35	0.00918975489913459\\
68.36	0.00918972336763352\\
68.37	0.00918969180669738\\
68.38	0.0091896602161618\\
68.39	0.00918962859586026\\
68.4	0.00918959694562412\\
68.41	0.00918956526528254\\
68.42	0.00918953355466249\\
68.43	0.0091895018135887\\
68.44	0.00918947004188362\\
68.45	0.0091894382393674\\
68.46	0.00918940640585788\\
68.47	0.00918937454117051\\
68.48	0.00918934264511835\\
68.49	0.00918931071751204\\
68.5	0.00918927875815973\\
68.51	0.00918924676686708\\
68.52	0.00918921474343724\\
68.53	0.00918918268767076\\
68.54	0.00918915059936561\\
68.55	0.00918911847831709\\
68.56	0.00918908632431785\\
68.57	0.00918905413715782\\
68.58	0.00918902191662416\\
68.59	0.00918898966250127\\
68.6	0.0091889573745707\\
68.61	0.00918892505261114\\
68.62	0.00918889269639839\\
68.63	0.00918886030570527\\
68.64	0.00918882788030164\\
68.65	0.00918879541995432\\
68.66	0.00918876292442708\\
68.67	0.00918873039348056\\
68.68	0.00918869782687225\\
68.69	0.00918866522435644\\
68.7	0.0091886325856842\\
68.71	0.0091885999106033\\
68.72	0.00918856719885818\\
68.73	0.00918853445018992\\
68.74	0.00918850166433616\\
68.75	0.00918846884103109\\
68.76	0.00918843598000538\\
68.77	0.00918840308098614\\
68.78	0.00918837014369689\\
68.79	0.00918833716785745\\
68.8	0.00918830415318398\\
68.81	0.00918827109938886\\
68.82	0.00918823800618066\\
68.83	0.00918820487326409\\
68.84	0.00918817170033998\\
68.85	0.00918813848710517\\
68.86	0.00918810523325248\\
68.87	0.00918807193847069\\
68.88	0.00918803860244443\\
68.89	0.00918800522485416\\
68.9	0.00918797180537612\\
68.91	0.00918793834368225\\
68.92	0.00918790483944014\\
68.93	0.00918787129231298\\
68.94	0.0091878377019595\\
68.95	0.00918780406803391\\
68.96	0.00918777039018583\\
68.97	0.00918773666806026\\
68.98	0.00918770290129747\\
68.99	0.009187669089533\\
69	0.00918763523239755\\
69.01	0.00918760132951691\\
69.02	0.00918756738051196\\
69.03	0.00918753338499853\\
69.04	0.00918749934258739\\
69.05	0.00918746525288414\\
69.06	0.00918743111548921\\
69.07	0.00918739692999769\\
69.08	0.00918736269599937\\
69.09	0.0091873284130786\\
69.1	0.00918729408081425\\
69.11	0.00918725969877964\\
69.12	0.00918722526654243\\
69.13	0.00918719078366462\\
69.14	0.00918715624970243\\
69.15	0.00918712166420621\\
69.16	0.00918708702672041\\
69.17	0.0091870523367835\\
69.18	0.00918701759392785\\
69.19	0.00918698279767971\\
69.2	0.0091869479475591\\
69.21	0.00918691304307974\\
69.22	0.00918687808374896\\
69.23	0.00918684306906766\\
69.24	0.00918680799853018\\
69.25	0.00918677287162425\\
69.26	0.0091867376878309\\
69.27	0.00918670244662438\\
69.28	0.00918666714747207\\
69.29	0.00918663178983442\\
69.3	0.00918659637316483\\
69.31	0.00918656089690958\\
69.32	0.00918652536050776\\
69.33	0.00918648976339116\\
69.34	0.00918645410498419\\
69.35	0.00918641838470381\\
69.36	0.00918638260195941\\
69.37	0.00918634675615274\\
69.38	0.00918631084667782\\
69.39	0.00918627487292083\\
69.4	0.00918623883426005\\
69.41	0.00918620273006574\\
69.42	0.00918616655970005\\
69.43	0.00918613032251695\\
69.44	0.00918609401786211\\
69.45	0.00918605764507281\\
69.46	0.00918602120347785\\
69.47	0.00918598469239746\\
69.48	0.00918594811114317\\
69.49	0.00918591145901775\\
69.5	0.00918587473531512\\
69.51	0.00918583793932017\\
69.52	0.00918580107030879\\
69.53	0.00918576412754764\\
69.54	0.00918572711029415\\
69.55	0.00918569001779634\\
69.56	0.00918565284929278\\
69.57	0.00918561560401247\\
69.58	0.0091855782811747\\
69.59	0.009185540879989\\
69.6	0.00918550339965501\\
69.61	0.00918546583936237\\
69.62	0.00918542819829064\\
69.63	0.00918539047560917\\
69.64	0.00918535267047702\\
69.65	0.00918531478204281\\
69.66	0.00918527680944466\\
69.67	0.0091852387518101\\
69.68	0.00918520060825588\\
69.69	0.00918516237788796\\
69.7	0.00918512405980131\\
69.71	0.00918508565307992\\
69.72	0.00918504715679657\\
69.73	0.00918500857001282\\
69.74	0.00918496989177883\\
69.75	0.00918493112113331\\
69.76	0.0091848922571034\\
69.77	0.00918485329870452\\
69.78	0.00918481424494035\\
69.79	0.00918477509480262\\
69.8	0.00918473584727109\\
69.81	0.00918469650131342\\
69.82	0.00918465705588505\\
69.83	0.00918461750992909\\
69.84	0.00918457786237625\\
69.85	0.00918453811214471\\
69.86	0.00918449825814005\\
69.87	0.0091844582992551\\
69.88	0.00918441823436987\\
69.89	0.00918437806235146\\
69.9	0.00918433778205393\\
69.91	0.00918429739231823\\
69.92	0.0091842568919721\\
69.93	0.00918421627982997\\
69.94	0.00918417555469285\\
69.95	0.00918413471534826\\
69.96	0.00918409376057015\\
69.97	0.00918405268911877\\
69.98	0.00918401149974063\\
69.99	0.00918397019116838\\
70	0.00918392876212074\\
70.01	0.00918388721130242\\
70.02	0.00918384553740405\\
70.03	0.00918380373910207\\
70.04	0.00918376181505871\\
70.05	0.00918371976392187\\
70.06	0.00918367758432507\\
70.07	0.00918363527488739\\
70.08	0.0091835928342134\\
70.09	0.00918355026089311\\
70.1	0.00918350755350191\\
70.11	0.00918346471060049\\
70.12	0.00918342173073486\\
70.13	0.00918337861243623\\
70.14	0.00918333535422102\\
70.15	0.0091832919545908\\
70.16	0.00918324841203229\\
70.17	0.00918320472501727\\
70.18	0.00918316089200264\\
70.19	0.00918311691143033\\
70.2	0.00918307278172736\\
70.21	0.00918302850130575\\
70.22	0.00918298406856262\\
70.23	0.0091829394818801\\
70.24	0.00918289473962543\\
70.25	0.00918284984015091\\
70.26	0.00918280478179395\\
70.27	0.00918275956287716\\
70.28	0.00918271418170829\\
70.29	0.00918266863658036\\
70.3	0.00918262292577168\\
70.31	0.00918257704754596\\
70.32	0.00918253100015231\\
70.33	0.00918248478182541\\
70.34	0.00918243839078553\\
70.35	0.0091823918252387\\
70.36	0.00918234508337677\\
70.37	0.00918229816337758\\
70.38	0.00918225106340508\\
70.39	0.00918220378160947\\
70.4	0.00918215631612736\\
70.41	0.00918210866508196\\
70.42	0.00918206082658327\\
70.43	0.00918201279872826\\
70.44	0.00918196457960108\\
70.45	0.00918191616727333\\
70.46	0.00918186755980426\\
70.47	0.00918181875524106\\
70.48	0.00918176975161909\\
70.49	0.00918172054696226\\
70.5	0.00918167113928323\\
70.51	0.00918162152658383\\
70.52	0.00918157170685535\\
70.53	0.00918152167807895\\
70.54	0.00918147143822602\\
70.55	0.00918142098525856\\
70.56	0.00918137031712969\\
70.57	0.00918131943178402\\
70.58	0.00918126832715817\\
70.59	0.00918121700118125\\
70.6	0.00918116545177539\\
70.61	0.00918111367685629\\
70.62	0.00918106167433379\\
70.63	0.00918100944211248\\
70.64	0.00918095697809233\\
70.65	0.00918090428016933\\
70.66	0.00918085134623623\\
70.67	0.0091807981741832\\
70.68	0.00918074476189864\\
70.69	0.00918069110726993\\
70.7	0.00918063720818426\\
70.71	0.00918058306252953\\
70.72	0.00918052866819519\\
70.73	0.0091804740230732\\
70.74	0.00918041912505899\\
70.75	0.00918036397205253\\
70.76	0.00918030856195931\\
70.77	0.00918025289269148\\
70.78	0.00918019696216902\\
70.79	0.00918014076832088\\
70.8	0.00918008430908627\\
70.81	0.00918002758241592\\
70.82	0.00917997058627346\\
70.83	0.00917991331863676\\
70.84	0.00917985577749944\\
70.85	0.00917979796087235\\
70.86	0.00917973986678513\\
70.87	0.00917968149328788\\
70.88	0.00917962283845279\\
70.89	0.00917956390037594\\
70.9	0.00917950467717911\\
70.91	0.00917944516701167\\
70.92	0.00917938536805255\\
70.93	0.00917932527851226\\
70.94	0.009179264896635\\
70.95	0.0091792042207009\\
70.96	0.00917914324902821\\
70.97	0.00917908197997573\\
70.98	0.00917902041194519\\
70.99	0.00917895854338385\\
71	0.00917889637278704\\
71.01	0.00917883389870093\\
71.02	0.00917877111972533\\
71.03	0.00917870803451663\\
71.04	0.00917864464179076\\
71.05	0.00917858094032637\\
71.06	0.00917851692896804\\
71.07	0.00917845260662968\\
71.08	0.0091783879722979\\
71.09	0.00917832302503572\\
71.1	0.00917825776398621\\
71.11	0.00917819218837636\\
71.12	0.00917812629752105\\
71.13	0.00917806009082722\\
71.14	0.00917799356779806\\
71.15	0.00917792672803744\\
71.16	0.00917785957125453\\
71.17	0.00917779209726841\\
71.18	0.00917772430601306\\
71.19	0.00917765619754232\\
71.2	0.00917758777203513\\
71.21	0.00917751902980094\\
71.22	0.00917744997128527\\
71.23	0.00917738059707549\\
71.24	0.00917731090790674\\
71.25	0.00917724090466813\\
71.26	0.00917717058840907\\
71.27	0.00917709996034588\\
71.28	0.00917702902186855\\
71.29	0.0091769577745478\\
71.3	0.00917688622014231\\
71.31	0.00917681436060623\\
71.32	0.0091767421980969\\
71.33	0.0091766697349829\\
71.34	0.00917659697385229\\
71.35	0.00917652391752113\\
71.36	0.00917645056904232\\
71.37	0.0091763769317147\\
71.38	0.00917630300909248\\
71.39	0.00917622880499492\\
71.4	0.00917615432351637\\
71.41	0.00917607956903666\\
71.42	0.00917600454623173\\
71.43	0.00917592926008475\\
71.44	0.00917585371589746\\
71.45	0.009175777919302\\
71.46	0.00917570187627301\\
71.47	0.00917562559314018\\
71.48	0.00917554907660124\\
71.49	0.0091754723337353\\
71.5	0.00917539537201665\\
71.51	0.00917531819932899\\
71.52	0.00917524082398015\\
71.53	0.00917516325471726\\
71.54	0.00917508550074239\\
71.55	0.00917500757172871\\
71.56	0.00917492947783719\\
71.57	0.00917485122973377\\
71.58	0.00917477283764932\\
71.59	0.00917469429973362\\
71.6	0.00917461561410559\\
71.61	0.00917453677885284\\
71.62	0.00917445779203104\\
71.63	0.00917437865166342\\
71.64	0.00917429935574017\\
71.65	0.00917421990221786\\
71.66	0.00917414028901885\\
71.67	0.0091740605140307\\
71.68	0.00917398057510553\\
71.69	0.00917390047005941\\
71.7	0.00917382019667171\\
71.71	0.0091737397526845\\
71.72	0.00917365913580179\\
71.73	0.00917357834368894\\
71.74	0.00917349737397195\\
71.75	0.00917341622423675\\
71.76	0.00917333489202849\\
71.77	0.0091732533748508\\
71.78	0.0091731716701651\\
71.79	0.00917308977538978\\
71.8	0.00917300768789946\\
71.81	0.00917292540502422\\
71.82	0.00917284292404879\\
71.83	0.00917276024221173\\
71.84	0.0091726773567046\\
71.85	0.00917259426467111\\
71.86	0.00917251096320631\\
71.87	0.00917242744935563\\
71.88	0.00917234372011406\\
71.89	0.00917225977242517\\
71.9	0.00917217560318025\\
71.91	0.0091720912092173\\
71.92	0.00917200658732012\\
71.93	0.00917192173421729\\
71.94	0.00917183664658116\\
71.95	0.00917175132102688\\
71.96	0.00917166575411127\\
71.97	0.00917157994233186\\
71.98	0.00917149388212572\\
71.99	0.00917140756986841\\
72	0.00917132100187282\\
72.01	0.00917123417438804\\
72.02	0.00917114708359818\\
72.03	0.00917105972562121\\
72.04	0.00917097209650766\\
72.05	0.00917088419223949\\
72.06	0.00917079600872871\\
72.07	0.00917070754181618\\
72.08	0.00917061878727027\\
72.09	0.00917052974078546\\
72.1	0.00917044039798106\\
72.11	0.00917035075439975\\
72.12	0.00917026080550617\\
72.13	0.00917017054668548\\
72.14	0.00917007997324188\\
72.15	0.00916998908039707\\
72.16	0.00916989786328873\\
72.17	0.00916980631696894\\
72.18	0.00916971443640261\\
72.19	0.00916962221646575\\
72.2	0.0091695296519439\\
72.21	0.00916943673753038\\
72.22	0.00916934346782455\\
72.23	0.00916924983733003\\
72.24	0.00916915584045291\\
72.25	0.0091690614714999\\
72.26	0.00916896672467641\\
72.27	0.00916887159408468\\
72.28	0.00916877607372176\\
72.29	0.00916868015747754\\
72.3	0.0091685838391327\\
72.31	0.00916848711235662\\
72.32	0.00916838997070524\\
72.33	0.00916829240761888\\
72.34	0.00916819441642004\\
72.35	0.00916809599031113\\
72.36	0.0091679971223721\\
72.37	0.00916789780555818\\
72.38	0.00916779803269737\\
72.39	0.00916769779648801\\
72.4	0.00916759708949629\\
72.41	0.00916749590415365\\
72.42	0.00916739423275416\\
72.43	0.00916729206745184\\
72.44	0.00916718940025793\\
72.45	0.00916708622303811\\
72.46	0.0091669825275096\\
72.47	0.0091668783052383\\
72.48	0.00916677354763577\\
72.49	0.0091666682459562\\
72.5	0.00916656239129332\\
72.51	0.00916645597457718\\
72.52	0.00916634898657097\\
72.53	0.00916624141786764\\
72.54	0.00916613325888656\\
72.55	0.00916602449987004\\
72.56	0.00916591513087981\\
72.57	0.00916580514179336\\
72.58	0.00916569452230031\\
72.59	0.00916558326189859\\
72.6	0.0091654713498906\\
72.61	0.00916535877537928\\
72.62	0.00916524552726404\\
72.63	0.00916513159423668\\
72.64	0.00916501696477714\\
72.65	0.00916490162714924\\
72.66	0.00916478556939623\\
72.67	0.00916466877933634\\
72.68	0.0091645512445581\\
72.69	0.0091644329524157\\
72.7	0.00916431389002417\\
72.71	0.00916419404425442\\
72.72	0.00916407340172823\\
72.73	0.0091639519488131\\
72.74	0.00916382967161697\\
72.75	0.00916370655598286\\
72.76	0.00916358258748336\\
72.77	0.00916345775141508\\
72.78	0.00916333203279291\\
72.79	0.00916320541634416\\
72.8	0.00916307788650261\\
72.81	0.00916294942740238\\
72.82	0.00916282002287167\\
72.83	0.00916268965642637\\
72.84	0.00916255831126355\\
72.85	0.0091624259702547\\
72.86	0.00916229261593898\\
72.87	0.00916215823051616\\
72.88	0.00916202279583947\\
72.89	0.00916188629340829\\
72.9	0.00916174870436067\\
72.91	0.00916161000946565\\
72.92	0.00916147018911544\\
72.93	0.00916132922331738\\
72.94	0.00916118709168573\\
72.95	0.00916104377343334\\
72.96	0.00916089924736296\\
72.97	0.00916075349185854\\
72.98	0.00916060648487617\\
72.99	0.00916045909894929\\
73	0.0091603116004552\\
73.01	0.00916016398907375\\
73.02	0.00916001626448528\\
73.03	0.00915986842637057\\
73.04	0.00915972047441105\\
73.05	0.00915957240828873\\
73.06	0.00915942422768635\\
73.07	0.00915927593228741\\
73.08	0.00915912752177624\\
73.09	0.00915897899583808\\
73.1	0.00915883035415912\\
73.11	0.00915868159642661\\
73.12	0.00915853272232887\\
73.13	0.00915838373155543\\
73.14	0.00915823462379702\\
73.15	0.00915808539874571\\
73.16	0.00915793605609492\\
73.17	0.00915778659553954\\
73.18	0.00915763701677597\\
73.19	0.00915748731950218\\
73.2	0.00915733750341778\\
73.21	0.00915718756822413\\
73.22	0.00915703751362436\\
73.23	0.00915688733932345\\
73.24	0.00915673704502829\\
73.25	0.00915658663044777\\
73.26	0.0091564360952928\\
73.27	0.00915628543927642\\
73.28	0.00915613466211381\\
73.29	0.0091559837635224\\
73.3	0.00915583274322188\\
73.31	0.00915568160093431\\
73.32	0.00915553033638411\\
73.33	0.00915537894929814\\
73.34	0.00915522743940575\\
73.35	0.00915507580643884\\
73.36	0.00915492405013186\\
73.37	0.00915477217022186\\
73.38	0.00915462016644857\\
73.39	0.00915446803855437\\
73.4	0.00915431578628431\\
73.41	0.0091541634093862\\
73.42	0.00915401090761057\\
73.43	0.00915385828071067\\
73.44	0.00915370552844253\\
73.45	0.0091535526505649\\
73.46	0.00915339964683925\\
73.47	0.0091532465170298\\
73.48	0.00915309326090345\\
73.49	0.00915293987822973\\
73.5	0.00915278636878084\\
73.51	0.0091526327323315\\
73.52	0.00915247896865897\\
73.53	0.00915232507754294\\
73.54	0.00915217105876545\\
73.55	0.0091520169121108\\
73.56	0.00915186263736546\\
73.57	0.00915170823431795\\
73.58	0.00915155370275868\\
73.59	0.00915139904247983\\
73.6	0.00915124425327518\\
73.61	0.00915108933493993\\
73.62	0.00915093428727054\\
73.63	0.00915077911006445\\
73.64	0.00915062380311991\\
73.65	0.00915046836623569\\
73.66	0.00915031279921086\\
73.67	0.00915015710184443\\
73.68	0.00915000127393508\\
73.69	0.00914984531528081\\
73.7	0.00914968922567856\\
73.71	0.00914953300492385\\
73.72	0.00914937665281033\\
73.73	0.00914922016912933\\
73.74	0.00914906355366942\\
73.75	0.00914890680621584\\
73.76	0.00914874992655\\
73.77	0.0091485929144489\\
73.78	0.00914843576968446\\
73.79	0.00914827849202292\\
73.8	0.00914812108122414\\
73.81	0.0091479635370408\\
73.82	0.0091478058592177\\
73.83	0.00914764804749088\\
73.84	0.00914749010158674\\
73.85	0.00914733202122112\\
73.86	0.00914717380609834\\
73.87	0.00914701545591012\\
73.88	0.00914685697033451\\
73.89	0.00914669834903472\\
73.9	0.00914653959165793\\
73.91	0.00914638069783396\\
73.92	0.00914622166717395\\
73.93	0.00914606249926891\\
73.94	0.00914590319368823\\
73.95	0.00914574374997811\\
73.96	0.00914558416765986\\
73.97	0.0091454244462282\\
73.98	0.00914526458514939\\
73.99	0.00914510458385932\\
74	0.00914494444176145\\
74.01	0.00914478415822474\\
74.02	0.00914462373258136\\
74.03	0.00914446316412438\\
74.04	0.00914430245210532\\
74.05	0.00914414159573153\\
74.06	0.00914398059416355\\
74.07	0.00914381944651224\\
74.08	0.00914365815183583\\
74.09	0.00914349670913682\\
74.1	0.00914333511739848\\
74.11	0.00914317337558812\\
74.12	0.00914301148265682\\
74.13	0.00914284943753908\\
74.14	0.00914268723915252\\
74.15	0.00914252488639755\\
74.16	0.00914236237815701\\
74.17	0.00914219971329584\\
74.18	0.00914203689066076\\
74.19	0.00914187390907985\\
74.2	0.00914171076736221\\
74.21	0.0091415474642976\\
74.22	0.00914138399865601\\
74.23	0.00914122036918731\\
74.24	0.00914105657462082\\
74.25	0.00914089261366489\\
74.26	0.0091407284850065\\
74.27	0.0091405641873108\\
74.28	0.00914039971922069\\
74.29	0.00914023507935633\\
74.3	0.00914007026631471\\
74.31	0.00913990527866916\\
74.32	0.00913974011496886\\
74.33	0.00913957477373831\\
74.34	0.00913940925347689\\
74.35	0.00913924355265825\\
74.36	0.00913907766972983\\
74.37	0.00913891160311229\\
74.38	0.00913874535119893\\
74.39	0.00913857891235514\\
74.4	0.00913841228491779\\
74.41	0.0091382454671946\\
74.42	0.00913807845746358\\
74.43	0.0091379112539723\\
74.44	0.00913774385493734\\
74.45	0.00913757625854352\\
74.46	0.00913740846294328\\
74.47	0.00913724046625593\\
74.48	0.00913707226656697\\
74.49	0.0091369038619273\\
74.5	0.0091367352503525\\
74.51	0.00913656642982202\\
74.52	0.00913639739827839\\
74.53	0.00913622815362641\\
74.54	0.00913605869373228\\
74.55	0.00913588901642278\\
74.56	0.00913571911948431\\
74.57	0.00913554900066206\\
74.58	0.00913537865765905\\
74.59	0.00913520808813512\\
74.6	0.00913503728970602\\
74.61	0.00913486625994238\\
74.62	0.00913469499636863\\
74.63	0.00913452349646198\\
74.64	0.00913435175765135\\
74.65	0.00913417977731617\\
74.66	0.00913400755278532\\
74.67	0.00913383508133586\\
74.68	0.0091336623601919\\
74.69	0.00913348938652331\\
74.7	0.00913331615744442\\
74.71	0.00913314267001275\\
74.72	0.00913296892122765\\
74.73	0.0091327949080289\\
74.74	0.00913262062729528\\
74.75	0.00913244607584312\\
74.76	0.00913227125042482\\
74.77	0.00913209614772723\\
74.78	0.00913192076437018\\
74.79	0.00913174509690472\\
74.8	0.00913156914181154\\
74.81	0.00913139294031263\\
74.82	0.00913121660251869\\
74.83	0.00913104012781277\\
74.84	0.00913086351557005\\
74.85	0.00913068676515781\\
74.86	0.00913050987593527\\
74.87	0.00913033284725348\\
74.88	0.00913015567845514\\
74.89	0.00912997836887456\\
74.9	0.00912980091783747\\
74.91	0.00912962332466091\\
74.92	0.00912944558865305\\
74.93	0.00912926770911313\\
74.94	0.00912908968533126\\
74.95	0.00912891151658831\\
74.96	0.00912873320215573\\
74.97	0.00912855474129546\\
74.98	0.00912837613325974\\
74.99	0.00912819737729094\\
75	0.00912801847262148\\
75.01	0.0091278394184736\\
75.02	0.00912766021405923\\
75.03	0.00912748085857985\\
75.04	0.00912730135122628\\
75.05	0.00912712169117856\\
75.06	0.00912694187760573\\
75.07	0.00912676190966573\\
75.08	0.00912658178650515\\
75.09	0.00912640150725909\\
75.1	0.00912622107105099\\
75.11	0.00912604047699241\\
75.12	0.00912585972418289\\
75.13	0.00912567881170969\\
75.14	0.0091254977386477\\
75.15	0.00912531650405913\\
75.16	0.00912513510699341\\
75.17	0.00912495354648692\\
75.18	0.0091247718215628\\
75.19	0.00912458993123077\\
75.2	0.00912440787448689\\
75.21	0.00912422565031334\\
75.22	0.00912404325767821\\
75.23	0.00912386069553528\\
75.24	0.0091236779628238\\
75.25	0.00912349505846823\\
75.26	0.00912331198137801\\
75.27	0.00912312873044738\\
75.28	0.00912294530455504\\
75.29	0.00912276170256397\\
75.3	0.00912257792332118\\
75.31	0.00912239396565743\\
75.32	0.00912220982838696\\
75.33	0.00912202551030726\\
75.34	0.00912184101019879\\
75.35	0.00912165632682467\\
75.36	0.00912147145893048\\
75.37	0.00912128640524392\\
75.38	0.00912110116447451\\
75.39	0.00912091573531336\\
75.4	0.00912073011643282\\
75.41	0.0091205443064862\\
75.42	0.00912035830410746\\
75.43	0.0091201721079109\\
75.44	0.00911998571649084\\
75.45	0.00911979912842131\\
75.46	0.00911961234225569\\
75.47	0.00911942535652642\\
75.48	0.00911923816974462\\
75.49	0.00911905078039977\\
75.5	0.00911886318695936\\
75.51	0.00911867538786852\\
75.52	0.00911848738154967\\
75.53	0.00911829916640213\\
75.54	0.00911811074080177\\
75.55	0.00911792210310061\\
75.56	0.00911773325162645\\
75.57	0.00911754418468246\\
75.58	0.00911735490054678\\
75.59	0.00911716539747211\\
75.6	0.0091169756736853\\
75.61	0.00911678572738694\\
75.62	0.0091165955567509\\
75.63	0.00911640515992391\\
75.64	0.00911621453502512\\
75.65	0.00911602368014564\\
75.66	0.00911583259334807\\
75.67	0.00911564127266605\\
75.68	0.00911544971610378\\
75.69	0.00911525792163552\\
75.7	0.00911506588720512\\
75.71	0.00911487361072549\\
75.72	0.00911468109007812\\
75.73	0.00911448832311254\\
75.74	0.0091142953076458\\
75.75	0.00911410204146192\\
75.76	0.00911390852231138\\
75.77	0.00911371474791049\\
75.78	0.00911352071594093\\
75.79	0.00911332642404905\\
75.8	0.00911313186984541\\
75.81	0.00911293705090407\\
75.82	0.00911274196476206\\
75.83	0.00911254660891875\\
75.84	0.00911235098083517\\
75.85	0.00911215507793344\\
75.86	0.00911195889759608\\
75.87	0.00911176243716534\\
75.88	0.00911156569394255\\
75.89	0.00911136866518745\\
75.9	0.00911117134811743\\
75.91	0.00911097373990686\\
75.92	0.00911077583768639\\
75.93	0.00911057763854215\\
75.94	0.00911037913951508\\
75.95	0.00911018033760008\\
75.96	0.00910998122974531\\
75.97	0.00910978181285135\\
75.98	0.00910958208377044\\
75.99	0.00910938203930559\\
76	0.00910918167620985\\
76.01	0.00910898099118535\\
76.02	0.00910877998088253\\
76.03	0.0091085786418992\\
76.04	0.00910837697077968\\
76.05	0.00910817496401385\\
76.06	0.0091079726180363\\
76.07	0.00910776992922528\\
76.08	0.00910756689390183\\
76.09	0.00910736350832875\\
76.1	0.0091071597687096\\
76.11	0.00910695567118774\\
76.12	0.00910675121184524\\
76.13	0.00910654638670185\\
76.14	0.00910634119171393\\
76.15	0.00910613562277335\\
76.16	0.0091059296757064\\
76.17	0.00910572334627264\\
76.18	0.00910551663016375\\
76.19	0.00910530952300236\\
76.2	0.00910510202034087\\
76.21	0.00910489411766021\\
76.22	0.00910468581036862\\
76.23	0.00910447709380037\\
76.24	0.0091042679632145\\
76.25	0.00910405841379349\\
76.26	0.00910384844064194\\
76.27	0.0091036380387852\\
76.28	0.00910342720316802\\
76.29	0.00910321592865307\\
76.3	0.0091030042100196\\
76.31	0.00910279204196191\\
76.32	0.00910257941908787\\
76.33	0.00910236633591745\\
76.34	0.00910215278688112\\
76.35	0.00910193876631829\\
76.36	0.00910172426847573\\
76.37	0.00910150928750592\\
76.38	0.00910129381746538\\
76.39	0.00910107785231298\\
76.4	0.00910086138590824\\
76.41	0.00910064441200951\\
76.42	0.00910042692427222\\
76.43	0.00910020891624706\\
76.44	0.00909999038137809\\
76.45	0.00909977131300085\\
76.46	0.00909955170434044\\
76.47	0.00909933154850953\\
76.48	0.00909911083850638\\
76.49	0.00909888956721276\\
76.5	0.00909866772739188\\
76.51	0.00909844531168628\\
76.52	0.00909822231261564\\
76.53	0.00909799872257458\\
76.54	0.00909777453383041\\
76.55	0.00909754973852085\\
76.56	0.00909732432865167\\
76.57	0.00909709829609431\\
76.58	0.00909687163258347\\
76.59	0.00909664432971462\\
76.6	0.00909641637894147\\
76.61	0.0090961877715734\\
76.62	0.00909595849877286\\
76.63	0.00909572855155265\\
76.64	0.00909549792077324\\
76.65	0.00909526659713995\\
76.66	0.00909503457120015\\
76.67	0.00909480183334033\\
76.68	0.0090945683737832\\
76.69	0.00909433418258466\\
76.7	0.00909409924963072\\
76.71	0.0090938635646344\\
76.72	0.00909362711713253\\
76.73	0.00909338989648253\\
76.74	0.00909315189185904\\
76.75	0.0090929130922506\\
76.76	0.00909267348645616\\
76.77	0.00909243306308158\\
76.78	0.00909219181053606\\
76.79	0.00909194971702844\\
76.8	0.00909170677056353\\
76.81	0.00909146295893826\\
76.82	0.00909121826973779\\
76.83	0.0090909726903316\\
76.84	0.00909072620786942\\
76.85	0.00909047880927708\\
76.86	0.00909023048125239\\
76.87	0.00908998121026075\\
76.88	0.00908973098253084\\
76.89	0.00908947978405016\\
76.9	0.00908922760056043\\
76.91	0.00908897441755297\\
76.92	0.00908872022026396\\
76.93	0.00908846499366959\\
76.94	0.00908820872248113\\
76.95	0.0090879513911399\\
76.96	0.00908769298381207\\
76.97	0.00908743348438351\\
76.98	0.00908717287645434\\
76.99	0.00908691114333352\\
77	0.00908664826803326\\
77.01	0.0090863842332633\\
77.02	0.00908611902142511\\
77.03	0.009085852614606\\
77.04	0.00908558499457298\\
77.05	0.00908531614276665\\
77.06	0.00908504604029484\\
77.07	0.00908477466792621\\
77.08	0.00908450200608362\\
77.09	0.00908422803483745\\
77.1	0.00908395273389877\\
77.11	0.00908367608261225\\
77.12	0.00908339805994911\\
77.13	0.00908311864449973\\
77.14	0.00908283781446629\\
77.15	0.00908255554765509\\
77.16	0.00908227182146879\\
77.17	0.00908198661289849\\
77.18	0.00908169989851563\\
77.19	0.00908141165446367\\
77.2	0.0090811218564497\\
77.21	0.00908083047973572\\
77.22	0.00908053749912991\\
77.23	0.00908024288897753\\
77.24	0.00907994662315181\\
77.25	0.00907964867504447\\
77.26	0.00907934901755615\\
77.27	0.0090790476230866\\
77.28	0.00907874446352466\\
77.29	0.00907843951023801\\
77.3	0.0090781327340627\\
77.31	0.00907782410529249\\
77.32	0.00907751359366794\\
77.33	0.00907720116836529\\
77.34	0.00907688679798507\\
77.35	0.00907657045054053\\
77.36	0.00907625209344569\\
77.37	0.00907593169350327\\
77.38	0.00907560921689231\\
77.39	0.00907528462915543\\
77.4	0.00907495789518598\\
77.41	0.00907462897921476\\
77.42	0.00907429784479652\\
77.43	0.00907396445479614\\
77.44	0.00907362877137455\\
77.45	0.00907329075597427\\
77.46	0.00907295036930469\\
77.47	0.00907260757132699\\
77.48	0.00907226232123875\\
77.49	0.00907191457745819\\
77.5	0.0090715642976081\\
77.51	0.00907121143849939\\
77.52	0.00907085595611425\\
77.53	0.00907049780558899\\
77.54	0.00907013694119646\\
77.55	0.00906977331632808\\
77.56	0.00906940688347549\\
77.57	0.00906903759421174\\
77.58	0.00906866539917211\\
77.59	0.0090682902480345\\
77.6	0.00906791208949931\\
77.61	0.00906753087126894\\
77.62	0.0090671467006994\\
77.63	0.00906676235063087\\
77.64	0.00906637782102065\\
77.65	0.00906599311182264\\
77.66	0.0090656082229872\\
77.67	0.00906522315446093\\
77.68	0.00906483790618662\\
77.69	0.00906445247810299\\
77.7	0.00906406687014459\\
77.71	0.00906368108224157\\
77.72	0.00906329511431958\\
77.73	0.0090629089662995\\
77.74	0.00906252263809734\\
77.75	0.00906213612962403\\
77.76	0.00906174944078519\\
77.77	0.00906136257148099\\
77.78	0.00906097552160591\\
77.79	0.00906058829104859\\
77.8	0.00906020087969156\\
77.81	0.00905981328741108\\
77.82	0.00905942551407688\\
77.83	0.009059037559552\\
77.84	0.00905864942369251\\
77.85	0.00905826110634731\\
77.86	0.00905787260735789\\
77.87	0.00905748392655811\\
77.88	0.00905709506377395\\
77.89	0.00905670601882325\\
77.9	0.00905631679151547\\
77.91	0.00905592738165146\\
77.92	0.00905553778902318\\
77.93	0.00905514801341344\\
77.94	0.00905475805459562\\
77.95	0.00905436791233346\\
77.96	0.00905397758638071\\
77.97	0.00905358707648089\\
77.98	0.00905319638236701\\
77.99	0.0090528055037613\\
78	0.00905241444037486\\
78.01	0.00905202319190744\\
78.02	0.00905163175804709\\
78.03	0.00905124013846988\\
78.04	0.00905084833283961\\
78.05	0.00905045634080746\\
78.06	0.00905006416201174\\
78.07	0.00904967179607751\\
78.08	0.00904927924261632\\
78.09	0.00904888650122586\\
78.1	0.00904849357148963\\
78.11	0.00904810045297665\\
78.12	0.00904770714524109\\
78.13	0.00904731364782198\\
78.14	0.00904691996024287\\
78.15	0.00904652608201146\\
78.16	0.00904613201261934\\
78.17	0.00904573775154157\\
78.18	0.00904534329823642\\
78.19	0.00904494865214499\\
78.2	0.00904455381269089\\
78.21	0.00904415877927988\\
78.22	0.00904376355129961\\
78.23	0.00904336812811919\\
78.24	0.00904297250908891\\
78.25	0.00904257669353994\\
78.26	0.00904218068078391\\
78.27	0.0090417844701127\\
78.28	0.00904138806079805\\
78.29	0.00904099145209122\\
78.3	0.00904059464322278\\
78.31	0.0090401976334022\\
78.32	0.00903980042181762\\
78.33	0.00903940300763553\\
78.34	0.00903900539000047\\
78.35	0.0090386075680348\\
78.36	0.00903820954083838\\
78.37	0.00903781130748838\\
78.38	0.00903741286703895\\
78.39	0.00903701421852107\\
78.4	0.00903661536094229\\
78.41	0.00903621629328651\\
78.42	0.00903581701451382\\
78.43	0.00903541752356034\\
78.44	0.009035017819338\\
78.45	0.00903461790073449\\
78.46	0.00903421776661307\\
78.47	0.00903381741581252\\
78.48	0.00903341684714706\\
78.49	0.00903301605940631\\
78.5	0.00903261505135524\\
78.51	0.00903221382173425\\
78.52	0.00903181236925915\\
78.53	0.00903141069262124\\
78.54	0.00903100879048744\\
78.55	0.00903060666150046\\
78.56	0.00903020430427889\\
78.57	0.00902980171741754\\
78.58	0.00902939889948763\\
78.59	0.00902899584903712\\
78.6	0.00902859256459109\\
78.61	0.00902818904465213\\
78.62	0.00902778528770084\\
78.63	0.00902738129219631\\
78.64	0.00902697705657674\\
78.65	0.00902657257926008\\
78.66	0.00902616785864474\\
78.67	0.00902576289311035\\
78.68	0.0090253576810187\\
78.69	0.00902495222071457\\
78.7	0.00902454651052684\\
78.71	0.00902414054876954\\
78.72	0.00902373433374306\\
78.73	0.00902332786373547\\
78.74	0.00902292113702388\\
78.75	0.00902251415187592\\
78.76	0.0090221069065514\\
78.77	0.00902169939930399\\
78.78	0.00902129162838308\\
78.79	0.00902088359203576\\
78.8	0.00902047528850889\\
78.81	0.00902006671605137\\
78.82	0.00901965787291652\\
78.83	0.00901924875736464\\
78.84	0.00901883936766569\\
78.85	0.00901842970210216\\
78.86	0.00901801975897215\\
78.87	0.00901760953659254\\
78.88	0.00901719903330248\\
78.89	0.00901678824746698\\
78.9	0.00901637717748071\\
78.91	0.00901596582177214\\
78.92	0.00901555417880774\\
78.93	0.00901514224709653\\
78.94	0.00901473002519489\\
78.95	0.00901431751171155\\
78.96	0.00901390470531291\\
78.97	0.00901349160472866\\
78.98	0.00901307820875765\\
78.99	0.0090126645162741\\
79	0.00901225052616202\\
79.01	0.00901183623729243\\
79.02	0.00901142164852318\\
79.03	0.00901100675869878\\
79.04	0.0090105915666502\\
79.05	0.00901017607119471\\
79.06	0.00900976027113573\\
79.07	0.00900934416526256\\
79.08	0.00900892775235028\\
79.09	0.00900851103115952\\
79.1	0.00900809400043627\\
79.11	0.00900767665891171\\
79.12	0.00900725900530199\\
79.13	0.00900684103830803\\
79.14	0.00900642275661538\\
79.15	0.00900600415889393\\
79.16	0.00900558524379778\\
79.17	0.009005166009965\\
79.18	0.00900474645601743\\
79.19	0.00900432658056049\\
79.2	0.00900390638218292\\
79.21	0.00900348585945662\\
79.22	0.0090030650109364\\
79.23	0.0090026438351598\\
79.24	0.00900222233064681\\
79.25	0.00900180049589972\\
79.26	0.00900137832940282\\
79.27	0.00900095582962224\\
79.28	0.0090005329950057\\
79.29	0.00900010982398226\\
79.3	0.00899968631496211\\
79.31	0.00899926246633632\\
79.32	0.00899883827647663\\
79.33	0.00899841374373519\\
79.34	0.0089979888664443\\
79.35	0.00899756364291622\\
79.36	0.00899713807144287\\
79.37	0.00899671215029564\\
79.38	0.00899628587772507\\
79.39	0.00899585925196069\\
79.4	0.00899543227121067\\
79.41	0.00899500493366165\\
79.42	0.00899457723747842\\
79.43	0.0089941491808037\\
79.44	0.00899372076175788\\
79.45	0.00899329197843874\\
79.46	0.00899286282892118\\
79.47	0.00899243331125698\\
79.48	0.00899200342347454\\
79.49	0.00899157316357856\\
79.5	0.00899114252954982\\
79.51	0.00899071151934491\\
79.52	0.0089902801308959\\
79.53	0.00898984836211014\\
79.54	0.00898941621086994\\
79.55	0.00898898367503229\\
79.56	0.00898855075242862\\
79.57	0.00898811744086445\\
79.58	0.00898768373811921\\
79.59	0.00898724964194585\\
79.6	0.00898681515007065\\
79.61	0.00898638026019287\\
79.62	0.00898594496998452\\
79.63	0.00898550927709002\\
79.64	0.00898507317912594\\
79.65	0.00898463667368075\\
79.66	0.00898419975831447\\
79.67	0.00898376243055843\\
79.68	0.00898332468791495\\
79.69	0.00898288652785709\\
79.7	0.00898244794782833\\
79.71	0.0089820089452423\\
79.72	0.00898156951748248\\
79.73	0.00898112966190196\\
79.74	0.00898068937582308\\
79.75	0.0089802486565372\\
79.76	0.00897980750130442\\
79.77	0.00897936590735324\\
79.78	0.00897892387188035\\
79.79	0.00897848139205033\\
79.8	0.00897803846499534\\
79.81	0.00897759508781485\\
79.82	0.00897715125757542\\
79.83	0.00897670697131037\\
79.84	0.00897626222601952\\
79.85	0.00897581701866897\\
79.86	0.00897537134619075\\
79.87	0.00897492520548265\\
79.88	0.0089744785934079\\
79.89	0.00897403150679496\\
79.9	0.00897358394243722\\
79.91	0.00897313589709281\\
79.92	0.00897268736748432\\
79.93	0.00897223835029858\\
79.94	0.00897178884218643\\
79.95	0.00897133883976249\\
79.96	0.00897088833960493\\
79.97	0.00897043733825528\\
79.98	0.00896998583221821\\
79.99	0.00896953381796133\\
80	0.008969081291915\\
80.01	0.00896862825047214\\
};
\addplot [color=blue,solid]
  table[row sep=crcr]{%
80.01	0.00896862825047214\\
80.02	0.00896817468998805\\
80.03	0.00896772060678026\\
80.04	0.00896726599712836\\
80.05	0.00896681085727383\\
80.06	0.00896635518341994\\
80.07	0.00896589897173158\\
80.08	0.00896544221833519\\
80.09	0.00896498491931859\\
80.1	0.00896452707073094\\
80.11	0.00896406866858261\\
80.12	0.00896360970884518\\
80.13	0.0089631501874513\\
80.14	0.0089626901002947\\
80.15	0.00896222944323016\\
80.16	0.00896176821207351\\
80.17	0.00896130640260157\\
80.18	0.00896084401055227\\
80.19	0.00896038103162462\\
80.2	0.0089599174614788\\
80.21	0.00895945329573624\\
80.22	0.00895898852997969\\
80.23	0.00895852315975339\\
80.24	0.00895805718056317\\
80.25	0.00895759058787663\\
80.26	0.00895712337712335\\
80.27	0.00895665554369506\\
80.28	0.00895618708294592\\
80.29	0.00895571799019276\\
80.3	0.00895524826071542\\
80.31	0.00895477788975699\\
80.32	0.00895430687252425\\
80.33	0.00895383520418802\\
80.34	0.00895336287988356\\
80.35	0.00895288989471106\\
80.36	0.00895241624373607\\
80.37	0.00895194192199008\\
80.38	0.00895146692447108\\
80.39	0.00895099124614411\\
80.4	0.00895051488194191\\
80.41	0.00895003782676567\\
80.42	0.00894956007548566\\
80.43	0.00894908162294207\\
80.44	0.00894860246394579\\
80.45	0.00894812259327929\\
80.46	0.00894764200569757\\
80.47	0.00894716069592904\\
80.48	0.00894667865867664\\
80.49	0.00894619588861884\\
80.5	0.00894571238041086\\
80.51	0.00894522812868578\\
80.52	0.00894474312805584\\
80.53	0.00894425737311377\\
80.54	0.00894377085843416\\
80.55	0.00894328357857493\\
80.56	0.00894279552807886\\
80.57	0.00894230670147515\\
80.58	0.00894181709328118\\
80.59	0.00894132669800419\\
80.6	0.00894083551014317\\
80.61	0.00894034352419072\\
80.62	0.00893985073463515\\
80.63	0.00893935713596249\\
80.64	0.00893886272265875\\
80.65	0.00893836748921215\\
80.66	0.00893787143011555\\
80.67	0.00893737453986892\\
80.68	0.00893687681298196\\
80.69	0.00893637824397675\\
80.7	0.00893587882739064\\
80.71	0.00893537855777912\\
80.72	0.00893487742971894\\
80.73	0.00893437543781121\\
80.74	0.0089338725766848\\
80.75	0.00893336884099971\\
80.76	0.00893286422545068\\
80.77	0.0089323587247709\\
80.78	0.00893185233373591\\
80.79	0.00893134504716755\\
80.8	0.00893083685993822\\
80.81	0.00893032776697514\\
80.82	0.00892981776326489\\
80.83	0.00892930684385807\\
80.84	0.00892879500387416\\
80.85	0.00892828223850652\\
80.86	0.00892776854302767\\
80.87	0.00892725391279462\\
80.88	0.00892673834325458\\
80.89	0.00892622182995067\\
80.9	0.00892570436852807\\
80.91	0.00892518595474018\\
80.92	0.00892466658445516\\
80.93	0.00892414625366259\\
80.94	0.0089236249584805\\
80.95	0.00892310269516249\\
80.96	0.00892257946010525\\
80.97	0.00892205524985626\\
80.98	0.00892153006112181\\
80.99	0.00892100389077524\\
81	0.00892047673586553\\
81.01	0.00891994859362619\\
81.02	0.0089194194614844\\
81.03	0.00891888933707051\\
81.04	0.00891835821822788\\
81.05	0.00891782610302301\\
81.06	0.00891729298975606\\
81.07	0.00891675887697173\\
81.08	0.00891622376347045\\
81.09	0.00891568764832002\\
81.1	0.00891515053086763\\
81.11	0.00891461241075224\\
81.12	0.0089140732879174\\
81.13	0.00891353316262454\\
81.14	0.00891299203546665\\
81.15	0.0089124499073824\\
81.16	0.00891190677967082\\
81.17	0.00891136265400634\\
81.18	0.00891081753245444\\
81.19	0.00891027141748774\\
81.2	0.00890972431200266\\
81.21	0.00890917621933656\\
81.22	0.00890862714328556\\
81.23	0.00890807708812278\\
81.24	0.00890752605861733\\
81.25	0.00890697406005379\\
81.26	0.00890642109825239\\
81.27	0.00890586717958983\\
81.28	0.00890531231102072\\
81.29	0.00890475650009978\\
81.3	0.00890419975500469\\
81.31	0.00890364208455971\\
81.32	0.00890308349826003\\
81.33	0.00890252400629688\\
81.34	0.00890196361958349\\
81.35	0.00890140234978177\\
81.36	0.00890084020932999\\
81.37	0.00890027721147116\\
81.38	0.00889971337028239\\
81.39	0.00889914870070519\\
81.4	0.00889858321857668\\
81.41	0.00889801694066176\\
81.42	0.00889744988468634\\
81.43	0.00889688206937154\\
81.44	0.00889631351446902\\
81.45	0.00889574424079731\\
81.46	0.00889517427027941\\
81.47	0.00889460362598137\\
81.48	0.00889403233215222\\
81.49	0.00889346041426507\\
81.5	0.00889288789905939\\
81.51	0.00889231481458477\\
81.52	0.00889174116481141\\
81.53	0.00889116694502753\\
81.54	0.00889059215044787\\
81.55	0.00889001677621241\\
81.56	0.00888944081738511\\
81.57	0.00888886426895263\\
81.58	0.00888828712582296\\
81.59	0.00888770938282414\\
81.6	0.00888713103470285\\
81.61	0.00888655207612301\\
81.62	0.00888597250166441\\
81.63	0.00888539230582122\\
81.64	0.00888481148300052\\
81.65	0.00888423002752088\\
81.66	0.00888364793361073\\
81.67	0.00888306519540688\\
81.68	0.00888248180695294\\
81.69	0.00888189776219768\\
81.7	0.00888131305499342\\
81.71	0.00888072767909437\\
81.72	0.00888014162815492\\
81.73	0.00887955489572793\\
81.74	0.00887896747526294\\
81.75	0.00887837936010443\\
81.76	0.00887779054348998\\
81.77	0.00887720101854839\\
81.78	0.00887661077829784\\
81.79	0.00887601981564391\\
81.8	0.00887542812337767\\
81.81	0.00887483569417368\\
81.82	0.00887424252058793\\
81.83	0.00887364859505583\\
81.84	0.00887305390989002\\
81.85	0.00887245845727831\\
81.86	0.00887186222928146\\
81.87	0.00887126521783096\\
81.88	0.00887066741472674\\
81.89	0.00887006881163492\\
81.9	0.00886946940008542\\
81.91	0.00886886917146959\\
81.92	0.00886826811703775\\
81.93	0.00886766622789671\\
81.94	0.0088670634950073\\
81.95	0.00886645990918169\\
81.96	0.00886585546108088\\
81.97	0.00886525014121194\\
81.98	0.00886464393992535\\
81.99	0.0088640368474122\\
82	0.00886342885370134\\
82.01	0.00886281994865658\\
82.02	0.00886221012197367\\
82.03	0.0088615993631774\\
82.04	0.00886098766161847\\
82.05	0.00886037500647048\\
82.06	0.00885976138672668\\
82.07	0.00885914679119687\\
82.08	0.00885853120850398\\
82.09	0.00885791462708086\\
82.1	0.0088572970351668\\
82.11	0.0088566784208041\\
82.12	0.00885605877183451\\
82.13	0.00885543807589563\\
82.14	0.00885481632041728\\
82.15	0.00885419349261772\\
82.16	0.00885356957949985\\
82.17	0.00885294456784734\\
82.18	0.00885231844422066\\
82.19	0.00885169119495304\\
82.2	0.00885106280614638\\
82.21	0.00885043326366705\\
82.22	0.00884980255314162\\
82.23	0.00884917065995249\\
82.24	0.00884853756923349\\
82.25	0.00884790326586533\\
82.26	0.00884726773447099\\
82.27	0.00884663095941104\\
82.28	0.00884599292477884\\
82.29	0.00884535361439564\\
82.3	0.00884471301180563\\
82.31	0.00884407110027084\\
82.32	0.00884342786276598\\
82.33	0.00884278328197315\\
82.34	0.00884213734027651\\
82.35	0.0088414900197567\\
82.36	0.00884084130218534\\
82.37	0.00884019116901928\\
82.38	0.00883953960139481\\
82.39	0.00883888658012171\\
82.4	0.00883823208567722\\
82.41	0.00883757609819986\\
82.42	0.00883691859748315\\
82.43	0.00883625956296921\\
82.44	0.00883559897374219\\
82.45	0.00883493680852165\\
82.46	0.00883427304565569\\
82.47	0.00883360766311407\\
82.48	0.00883294063848111\\
82.49	0.00883227194894849\\
82.5	0.00883160157130783\\
82.51	0.00883092948194326\\
82.52	0.00883025565682369\\
82.53	0.00882958007149505\\
82.54	0.00882890270107227\\
82.55	0.00882822352023117\\
82.56	0.00882754250320016\\
82.57	0.00882685962375177\\
82.58	0.00882617485519401\\
82.59	0.00882548817036157\\
82.6	0.00882479954160677\\
82.61	0.00882410894079049\\
82.62	0.0088234163392727\\
82.63	0.00882272170790297\\
82.64	0.00882202501701068\\
82.65	0.00882132623639511\\
82.66	0.00882062533531524\\
82.67	0.00881992228247943\\
82.68	0.00881921704603481\\
82.69	0.00881850959355649\\
82.7	0.00881779989203656\\
82.71	0.00881708790787283\\
82.72	0.00881637360685734\\
82.73	0.00881565695416467\\
82.74	0.00881493791433995\\
82.75	0.00881421645128667\\
82.76	0.00881349252825421\\
82.77	0.00881276610782508\\
82.78	0.00881203715190201\\
82.79	0.00881130562169459\\
82.8	0.0088105714777058\\
82.81	0.00880983467971813\\
82.82	0.00880909518677952\\
82.83	0.00880835295718887\\
82.84	0.00880760794848139\\
82.85	0.00880686011741353\\
82.86	0.00880610941994762\\
82.87	0.00880535581123623\\
82.88	0.00880459924560612\\
82.89	0.00880383967654193\\
82.9	0.00880307705666944\\
82.91	0.00880231133773855\\
82.92	0.00880154247060582\\
82.93	0.00880077040521672\\
82.94	0.00879999509058742\\
82.95	0.00879921647478622\\
82.96	0.00879843450491462\\
82.97	0.0087976491270879\\
82.98	0.00879686028641534\\
82.99	0.00879606792698004\\
83	0.00879527199181817\\
83.01	0.00879447242289795\\
83.02	0.00879366916109801\\
83.03	0.0087928621461854\\
83.04	0.00879205131679304\\
83.05	0.00879123661039674\\
83.06	0.00879041796329164\\
83.07	0.00878959531056823\\
83.08	0.00878876858608774\\
83.09	0.00878793772245709\\
83.1	0.00878710265100322\\
83.11	0.00878626330174684\\
83.12	0.0087854196033757\\
83.13	0.00878457148321714\\
83.14	0.00878371886721013\\
83.15	0.00878286167987664\\
83.16	0.00878199984429237\\
83.17	0.00878113328205689\\
83.18	0.00878026191326301\\
83.19	0.00877938565646554\\
83.2	0.00877850442864939\\
83.21	0.00877761814519679\\
83.22	0.00877672671985398\\
83.23	0.008775830064697\\
83.24	0.00877492809009679\\
83.25	0.00877402070468346\\
83.26	0.00877310781530975\\
83.27	0.00877218932701372\\
83.28	0.00877126514298052\\
83.29	0.00877033516450336\\
83.3	0.00876939929094351\\
83.31	0.00876845741968949\\
83.32	0.00876750944611523\\
83.33	0.00876655526353735\\
83.34	0.00876559476317141\\
83.35	0.00876462783408721\\
83.36	0.008763654363163\\
83.37	0.00876267423503867\\
83.38	0.00876168733206787\\
83.39	0.00876069353426903\\
83.4	0.00875969271927522\\
83.41	0.00875868476228282\\
83.42	0.00875766953599915\\
83.43	0.00875664691058863\\
83.44	0.00875561675361799\\
83.45	0.00875457892999991\\
83.46	0.00875353330193557\\
83.47	0.00875247972885574\\
83.48	0.00875141806736052\\
83.49	0.00875034817115766\\
83.5	0.00874926989099948\\
83.51	0.00874818307461819\\
83.52	0.00874708756665986\\
83.53	0.00874598320861663\\
83.54	0.00874486983875752\\
83.55	0.00874374729205742\\
83.56	0.00874261540012455\\
83.57	0.00874147682646632\\
83.58	0.00874033772022233\\
83.59	0.0087391980808133\\
83.6	0.00873805790765819\\
83.61	0.00873691720017422\\
83.62	0.00873577595777679\\
83.63	0.00873463417987954\\
83.64	0.00873349186589427\\
83.65	0.00873234901523093\\
83.66	0.00873120562729763\\
83.67	0.00873006170150061\\
83.68	0.00872891723724418\\
83.69	0.00872777223393077\\
83.7	0.00872662669096085\\
83.71	0.00872548060773295\\
83.72	0.0087243339836436\\
83.73	0.00872318681808736\\
83.74	0.00872203911045674\\
83.75	0.00872089086014223\\
83.76	0.00871974206653225\\
83.77	0.00871859272901312\\
83.78	0.00871744284696906\\
83.79	0.00871629241978217\\
83.8	0.00871514144683236\\
83.81	0.00871398992749739\\
83.82	0.00871283786115281\\
83.83	0.00871168524717192\\
83.84	0.00871053208492579\\
83.85	0.00870937837378319\\
83.86	0.00870822411311059\\
83.87	0.00870706930227213\\
83.88	0.0087059139406296\\
83.89	0.00870475802754238\\
83.9	0.00870360156236745\\
83.91	0.00870244454445934\\
83.92	0.00870128697317013\\
83.93	0.00870012884784937\\
83.94	0.0086989701678441\\
83.95	0.00869781093249879\\
83.96	0.00869665114115533\\
83.97	0.00869549079315298\\
83.98	0.00869432988782836\\
83.99	0.00869316842451539\\
84	0.0086920064025453\\
84.01	0.00869084382124653\\
84.02	0.00868968067994478\\
84.03	0.00868851697796291\\
84.04	0.00868735271462094\\
84.05	0.00868618788923598\\
84.06	0.00868502250112225\\
84.07	0.008683856549591\\
84.08	0.00868269003395048\\
84.09	0.00868152295350592\\
84.1	0.00868035530755946\\
84.11	0.00867918709541015\\
84.12	0.00867801831635389\\
84.13	0.00867684896968339\\
84.14	0.00867567905468813\\
84.15	0.00867450857065432\\
84.16	0.00867333751686486\\
84.17	0.00867216589259931\\
84.18	0.00867099369713381\\
84.19	0.00866982092974107\\
84.2	0.00866864758969031\\
84.21	0.00866747367624725\\
84.22	0.008666299188674\\
84.23	0.00866512412622904\\
84.24	0.00866394848816722\\
84.25	0.00866277227373963\\
84.26	0.00866159548219362\\
84.27	0.00866041811277271\\
84.28	0.00865924016471654\\
84.29	0.00865806163726085\\
84.3	0.00865688252963741\\
84.31	0.00865570284107394\\
84.32	0.0086545225707941\\
84.33	0.00865334171801741\\
84.34	0.00865216028195919\\
84.35	0.00865097826183052\\
84.36	0.00864979565683818\\
84.37	0.00864861246618458\\
84.38	0.0086474286890677\\
84.39	0.00864624432468105\\
84.4	0.00864505937221358\\
84.41	0.00864387383084964\\
84.42	0.0086426876997689\\
84.43	0.00864150097814629\\
84.44	0.00864031366515196\\
84.45	0.00863912575995116\\
84.46	0.00863793726170422\\
84.47	0.00863674816956647\\
84.48	0.00863555848268813\\
84.49	0.0086343682002143\\
84.5	0.00863317732128485\\
84.51	0.00863198584503437\\
84.52	0.00863079377059205\\
84.53	0.00862960109708165\\
84.54	0.00862840782362142\\
84.55	0.00862721394932398\\
84.56	0.00862601947329628\\
84.57	0.00862482439463952\\
84.58	0.00862362871244903\\
84.59	0.00862243242581422\\
84.6	0.00862123553381848\\
84.61	0.00862003803553909\\
84.62	0.00861883993004717\\
84.63	0.00861764121640751\\
84.64	0.00861644189367857\\
84.65	0.00861524196091232\\
84.66	0.00861404141715418\\
84.67	0.00861284026144289\\
84.68	0.00861163849281048\\
84.69	0.00861043611028209\\
84.7	0.00860923311287593\\
84.71	0.00860802949960312\\
84.72	0.00860682526946765\\
84.73	0.00860562042146624\\
84.74	0.00860441495458823\\
84.75	0.00860320886781547\\
84.76	0.00860200216012222\\
84.77	0.00860079483047503\\
84.78	0.00859958687783264\\
84.79	0.00859837830114582\\
84.8	0.0085971690993573\\
84.81	0.00859595927140163\\
84.82	0.00859474881620505\\
84.83	0.00859353773268537\\
84.84	0.00859232601975185\\
84.85	0.00859111367630506\\
84.86	0.00858990070123675\\
84.87	0.00858868709342972\\
84.88	0.00858747285175769\\
84.89	0.00858625797508515\\
84.9	0.00858504246226722\\
84.91	0.00858382631214953\\
84.92	0.00858260952356802\\
84.93	0.00858139209534887\\
84.94	0.00858017402630829\\
84.95	0.00857895531525236\\
84.96	0.00857773596097696\\
84.97	0.00857651596226748\\
84.98	0.00857529531789879\\
84.99	0.008574074026635\\
85	0.00857285208722929\\
85.01	0.00857162949842379\\
85.02	0.00857040625894937\\
85.03	0.00856918236752548\\
85.04	0.00856795782285997\\
85.05	0.00856673262364891\\
85.06	0.00856550676857639\\
85.07	0.00856428025631436\\
85.08	0.00856305308552244\\
85.09	0.00856182525484769\\
85.1	0.00856059676292446\\
85.11	0.00855936760837416\\
85.12	0.00855813778980505\\
85.13	0.00855690730581208\\
85.14	0.00855567615497664\\
85.15	0.00855444433586633\\
85.16	0.00855321184703481\\
85.17	0.0085519786870215\\
85.18	0.00855074485435141\\
85.19	0.00854951034753489\\
85.2	0.00854827516506741\\
85.21	0.00854703930542931\\
85.22	0.00854580276708556\\
85.23	0.00854456554848553\\
85.24	0.00854332764806271\\
85.25	0.00854208906423452\\
85.26	0.00854084979540197\\
85.27	0.00853960983994947\\
85.28	0.00853836919624451\\
85.29	0.00853712786263744\\
85.3	0.00853588583746114\\
85.31	0.0085346431190308\\
85.32	0.00853339970564356\\
85.33	0.00853215559557833\\
85.34	0.00853091078709536\\
85.35	0.00852966527843606\\
85.36	0.00852841906782263\\
85.37	0.00852717215345775\\
85.38	0.00852592453352431\\
85.39	0.00852467620618503\\
85.4	0.0085234271695822\\
85.41	0.00852217742183729\\
85.42	0.00852092696105062\\
85.43	0.00851967578530108\\
85.44	0.00851842389264568\\
85.45	0.00851717128111928\\
85.46	0.00851591794873419\\
85.47	0.00851466389347981\\
85.48	0.00851340911332226\\
85.49	0.00851215360620399\\
85.5	0.0085108973700434\\
85.51	0.00850964040273444\\
85.52	0.00850838270214624\\
85.53	0.00850712426612265\\
85.54	0.00850586509248187\\
85.55	0.00850460517901602\\
85.56	0.0085033445234907\\
85.57	0.00850208312364456\\
85.58	0.00850082097718884\\
85.59	0.00849955808180697\\
85.6	0.00849829443515405\\
85.61	0.00849703003485642\\
85.62	0.00849576487851118\\
85.63	0.0084944989636857\\
85.64	0.00849323228791713\\
85.65	0.00849196484871189\\
85.66	0.00849069664354519\\
85.67	0.0084894276698605\\
85.68	0.00848815792506901\\
85.69	0.00848688740654908\\
85.7	0.00848561611164577\\
85.71	0.00848434403767019\\
85.72	0.00848307118189901\\
85.73	0.00848179754157384\\
85.74	0.00848052311390067\\
85.75	0.00847924789604926\\
85.76	0.00847797188515258\\
85.77	0.00847669507830611\\
85.78	0.00847541747256731\\
85.79	0.0084741390649549\\
85.8	0.00847285985244829\\
85.81	0.00847157983198686\\
85.82	0.00847029900046932\\
85.83	0.00846901735475301\\
85.84	0.00846773489165322\\
85.85	0.0084664516079425\\
85.86	0.0084651675003499\\
85.87	0.00846388256556026\\
85.88	0.00846259680021348\\
85.89	0.00846131020090374\\
85.9	0.00846002276417876\\
85.91	0.00845873448653896\\
85.92	0.00845744536443673\\
85.93	0.00845615539427558\\
85.94	0.00845486457240931\\
85.95	0.00845357289514118\\
85.96	0.00845228035872308\\
85.97	0.00845098695935462\\
85.98	0.00844969269318224\\
85.99	0.00844839755629837\\
86	0.00844710154474044\\
86.01	0.00844580465448999\\
86.02	0.0084445068814717\\
86.03	0.00844320822155245\\
86.04	0.0084419086705403\\
86.05	0.00844060822418352\\
86.06	0.00843930687816955\\
86.07	0.00843800462812397\\
86.08	0.00843670146960946\\
86.09	0.00843539739812471\\
86.1	0.0084340924091033\\
86.11	0.00843278649791265\\
86.12	0.00843147965985285\\
86.13	0.00843017189015549\\
86.14	0.00842886318398254\\
86.15	0.0084275535364251\\
86.16	0.00842624294250222\\
86.17	0.00842493139715966\\
86.18	0.00842361889526862\\
86.19	0.00842230543162444\\
86.2	0.00842099100094536\\
86.21	0.00841967559787111\\
86.22	0.00841835921696163\\
86.23	0.00841704185269566\\
86.24	0.00841572349946933\\
86.25	0.00841440415159475\\
86.26	0.00841308380329858\\
86.27	0.0084117624487205\\
86.28	0.00841044008191175\\
86.29	0.00840911669683359\\
86.3	0.00840779228735575\\
86.31	0.0084064668472548\\
86.32	0.0084051403702126\\
86.33	0.0084038128498146\\
86.34	0.00840248427954821\\
86.35	0.00840115465280105\\
86.36	0.00839982396285927\\
86.37	0.00839849220290575\\
86.38	0.00839715936601828\\
86.39	0.0083958254451678\\
86.4	0.00839449043321645\\
86.41	0.00839315432291576\\
86.42	0.00839181710690463\\
86.43	0.00839047877770744\\
86.44	0.008389139327732\\
86.45	0.00838779874926753\\
86.46	0.00838645703448256\\
86.47	0.00838511417542287\\
86.48	0.00838377016400925\\
86.49	0.00838242499203541\\
86.5	0.00838107865116566\\
86.51	0.00837973113293268\\
86.52	0.00837838242873521\\
86.53	0.00837703252983568\\
86.54	0.00837568142735778\\
86.55	0.00837432911228407\\
86.56	0.00837297557545348\\
86.57	0.00837162080755873\\
86.58	0.0083702647991438\\
86.59	0.00836890754060125\\
86.6	0.00836754902216962\\
86.61	0.00836618923393061\\
86.62	0.00836482816580635\\
86.63	0.0083634658075566\\
86.64	0.0083621021487758\\
86.65	0.00836073717889018\\
86.66	0.00835937088715477\\
86.67	0.00835800326265033\\
86.68	0.00835663429428028\\
86.69	0.00835526397076753\\
86.7	0.00835389228065123\\
86.71	0.00835251921228355\\
86.72	0.00835114475382629\\
86.73	0.00834976889324753\\
86.74	0.00834839161831809\\
86.75	0.00834701291660809\\
86.76	0.0083456327754833\\
86.77	0.00834425118210145\\
86.78	0.0083428681234086\\
86.79	0.00834148358613522\\
86.8	0.00834009755679241\\
86.81	0.00833871002166788\\
86.82	0.00833732096682199\\
86.83	0.00833593037808363\\
86.84	0.00833453824104603\\
86.85	0.00833314454106253\\
86.86	0.00833174926324225\\
86.87	0.00833035239244564\\
86.88	0.00832895391328004\\
86.89	0.00832755381009503\\
86.9	0.00832615206697782\\
86.91	0.00832474866774846\\
86.92	0.00832334359595498\\
86.93	0.00832193683486846\\
86.94	0.008320528367478\\
86.95	0.00831911817648559\\
86.96	0.00831770624430084\\
86.97	0.00831629255303569\\
86.98	0.00831487708449898\\
86.99	0.00831345982019088\\
87	0.00831204074129725\\
87.01	0.00831061982868391\\
87.02	0.00830919706289076\\
87.03	0.00830777242412582\\
87.04	0.00830634589225909\\
87.05	0.00830491744681642\\
87.06	0.0083034870669731\\
87.07	0.00830205473154746\\
87.08	0.00830062041899425\\
87.09	0.008299184107398\\
87.1	0.0082977457744661\\
87.11	0.00829630539752189\\
87.12	0.00829486295349754\\
87.13	0.00829341841892681\\
87.14	0.00829197176993765\\
87.15	0.00829052298224468\\
87.16	0.00828907203114151\\
87.17	0.00828761889149292\\
87.18	0.00828616353772688\\
87.19	0.00828470594382639\\
87.2	0.0082832460833212\\
87.21	0.00828178392927935\\
87.22	0.00828031945429853\\
87.23	0.00827885263049728\\
87.24	0.00827738342950603\\
87.25	0.00827591182245792\\
87.26	0.0082744377799795\\
87.27	0.00827296127218118\\
87.28	0.00827148226864755\\
87.29	0.00827000073842744\\
87.3	0.00826851665002387\\
87.31	0.00826702997138369\\
87.32	0.00826554066988712\\
87.33	0.00826404871233702\\
87.34	0.00826255406494792\\
87.35	0.00826105669333493\\
87.36	0.0082595565625023\\
87.37	0.00825805363683184\\
87.38	0.0082565478800711\\
87.39	0.00825503925532125\\
87.4	0.00825352772502476\\
87.41	0.00825201325095285\\
87.42	0.00825049579419265\\
87.43	0.00824897531513407\\
87.44	0.0082474517734565\\
87.45	0.00824592512811515\\
87.46	0.00824439533732712\\
87.47	0.00824286235855725\\
87.48	0.00824132614850361\\
87.49	0.00823978666308272\\
87.5	0.00823824385741446\\
87.51	0.00823669768580669\\
87.52	0.00823514810173951\\
87.53	0.00823359505784926\\
87.54	0.00823203850591212\\
87.55	0.00823047839682743\\
87.56	0.00822891468060058\\
87.57	0.00822734730632568\\
87.58	0.00822577622216773\\
87.59	0.0082242013753445\\
87.6	0.00822262271210801\\
87.61	0.00822104017772562\\
87.62	0.00821945371646074\\
87.63	0.00821786327155311\\
87.64	0.00821626878519868\\
87.65	0.00821467019852908\\
87.66	0.0082130674515906\\
87.67	0.00821146048332286\\
87.68	0.00820984923153684\\
87.69	0.00820823363289259\\
87.7	0.00820661362287644\\
87.71	0.00820498913577769\\
87.72	0.00820336010466478\\
87.73	0.00820172646136108\\
87.74	0.00820008813642001\\
87.75	0.00819844505909972\\
87.76	0.00819679715733719\\
87.77	0.00819514435772183\\
87.78	0.00819348658546841\\
87.79	0.00819182376438951\\
87.8	0.00819015581686733\\
87.81	0.00818848266382488\\
87.82	0.00818680422469657\\
87.83	0.00818512041739817\\
87.84	0.00818343115829608\\
87.85	0.00818173636217599\\
87.86	0.00818003594221076\\
87.87	0.00817832980992775\\
87.88	0.00817661787517527\\
87.89	0.00817490004608843\\
87.9	0.00817317622905415\\
87.91	0.00817144632867546\\
87.92	0.00816971024773496\\
87.93	0.00816796788715757\\
87.94	0.00816621914597232\\
87.95	0.00816446392127342\\
87.96	0.0081627021081804\\
87.97	0.00816093359979737\\
87.98	0.00815915828717142\\
87.99	0.00815737605925002\\
88	0.00815558680283755\\
88.01	0.00815379040255077\\
88.02	0.00815198674077338\\
88.03	0.00815017569760953\\
88.04	0.00814835715083622\\
88.05	0.00814653097585477\\
88.06	0.00814469704564101\\
88.07	0.00814285523069455\\
88.08	0.00814100539898673\\
88.09	0.00813914741590747\\
88.1	0.00813728114421094\\
88.11	0.0081354064439599\\
88.12	0.00813352317246887\\
88.13	0.00813163118424593\\
88.14	0.00812973033093324\\
88.15	0.00812782046124615\\
88.16	0.00812590142091099\\
88.17	0.00812397305260129\\
88.18	0.00812203519587271\\
88.19	0.00812008768709631\\
88.2	0.00811813035939035\\
88.21	0.00811616304255056\\
88.22	0.00811418556297863\\
88.23	0.00811219774360923\\
88.24	0.00811019940383522\\
88.25	0.00810819035943109\\
88.26	0.00810617042247471\\
88.27	0.00810413940126718\\
88.28	0.00810209710025077\\
88.29	0.00810004331992503\\
88.3	0.00809797785676082\\
88.31	0.0080959005031124\\
88.32	0.00809381104712736\\
88.33	0.00809170927265446\\
88.34	0.0080895949591493\\
88.35	0.00808746788157773\\
88.36	0.00808532781031691\\
88.37	0.00808317451105411\\
88.38	0.00808100774468298\\
88.39	0.00807882726719745\\
88.4	0.00807663282958296\\
88.41	0.0080744241777052\\
88.42	0.00807220105219605\\
88.43	0.00806996318833688\\
88.44	0.00806771031593892\\
88.45	0.00806544215922084\\
88.46	0.00806315843668325\\
88.47	0.00806085886098027\\
88.48	0.0080585431387878\\
88.49	0.00805621097066879\\
88.5	0.00805386205093502\\
88.51	0.00805149606750563\\
88.52	0.00804911270176207\\
88.53	0.00804671162839957\\
88.54	0.0080442925152749\\
88.55	0.00804185502325038\\
88.56	0.00803939880603404\\
88.57	0.00803692351001578\\
88.58	0.00803442877409954\\
88.59	0.00803191422953122\\
88.6	0.0080293794997223\\
88.61	0.00802683539440995\\
88.62	0.00802429030663428\\
88.63	0.00802174423558546\\
88.64	0.00801919718045263\\
88.65	0.00801664914042385\\
88.66	0.00801410011468615\\
88.67	0.00801155010242549\\
88.68	0.00800899910282674\\
88.69	0.00800644711507374\\
88.7	0.00800389413834923\\
88.71	0.00800134017183493\\
88.72	0.00799878521471144\\
88.73	0.0079962292661583\\
88.74	0.00799367232535401\\
88.75	0.00799111439147594\\
88.76	0.00798855546370043\\
88.77	0.00798599554120272\\
88.78	0.00798343462315696\\
88.79	0.00798087270873622\\
88.8	0.00797830979711251\\
88.81	0.00797574588745673\\
88.82	0.00797318097893869\\
88.83	0.00797061507072711\\
88.84	0.00796804816198962\\
88.85	0.00796548025189277\\
88.86	0.00796291133960199\\
88.87	0.00796034142428163\\
88.88	0.00795777050509493\\
88.89	0.00795519858120402\\
88.9	0.00795262565176995\\
88.91	0.00795005171595264\\
88.92	0.00794747677291091\\
88.93	0.00794490082180248\\
88.94	0.00794232386178395\\
88.95	0.0079397458920108\\
88.96	0.0079371669116374\\
88.97	0.00793458691981702\\
88.98	0.00793200591570177\\
88.99	0.00792942389844268\\
89	0.00792684086718964\\
89.01	0.0079242568210914\\
89.02	0.00792167175929561\\
89.03	0.00791908568094876\\
89.04	0.00791649858519625\\
89.05	0.00791391047118231\\
89.06	0.00791132133805005\\
89.07	0.00790873118494143\\
89.08	0.0079061400109973\\
89.09	0.00790354781535734\\
89.1	0.0079009545971601\\
89.11	0.00789836035554299\\
89.12	0.00789576508964226\\
89.13	0.007893168798593\\
89.14	0.0078905714815292\\
89.15	0.00788797313758364\\
89.16	0.00788537376588797\\
89.17	0.00788277336557268\\
89.18	0.00788017193576711\\
89.19	0.00787756947559942\\
89.2	0.00787496598419663\\
89.21	0.00787236146068458\\
89.22	0.00786975590418793\\
89.23	0.0078671493138302\\
89.24	0.00786454168873373\\
89.25	0.00786193302801967\\
89.26	0.00785932333080801\\
89.27	0.00785671259621756\\
89.28	0.00785410082336596\\
89.29	0.00785148801136965\\
89.3	0.00784887415934389\\
89.31	0.00784625926640276\\
89.32	0.00784364333165916\\
89.33	0.00784102635422478\\
89.34	0.00783840833321015\\
89.35	0.00783578926772458\\
89.36	0.00783316915687617\\
89.37	0.00783054799977188\\
89.38	0.0078279257955174\\
89.39	0.00782530254321727\\
89.4	0.0078226782419748\\
89.41	0.00782005289089209\\
89.42	0.00781742648907007\\
89.43	0.0078147990356084\\
89.44	0.00781217052960558\\
89.45	0.00780954097015886\\
89.46	0.00780691035636429\\
89.47	0.00780427868731671\\
89.48	0.00780164596210972\\
89.49	0.00779901217983569\\
89.5	0.0077963773395858\\
89.51	0.00779374144044997\\
89.52	0.0077911044815169\\
89.53	0.00778846646187407\\
89.54	0.0077858273806077\\
89.55	0.0077831872368028\\
89.56	0.00778054602954313\\
89.57	0.00777790375791121\\
89.58	0.00777526042098831\\
89.59	0.00777261601785449\\
89.6	0.00776997054758851\\
89.61	0.00776732400926792\\
89.62	0.007764676401969\\
89.63	0.0077620277247668\\
89.64	0.00775937797673508\\
89.65	0.00775672715694638\\
89.66	0.00775407526447194\\
89.67	0.00775142229838177\\
89.68	0.00774876825774461\\
89.69	0.00774611314162792\\
89.7	0.00774345694909791\\
89.71	0.00774079967921949\\
89.72	0.00773814133105634\\
89.73	0.00773548190367083\\
89.74	0.00773282139612406\\
89.75	0.00773015980747586\\
89.76	0.00772749713678477\\
89.77	0.00772483338310805\\
89.78	0.00772216854550167\\
89.79	0.00771950262302033\\
89.8	0.0077168356147174\\
89.81	0.007714167519645\\
89.82	0.00771149833685392\\
89.83	0.00770882806539369\\
89.84	0.0077061567043125\\
89.85	0.00770348425265726\\
89.86	0.00770081070947359\\
89.87	0.00769813607380577\\
89.88	0.00769546034469679\\
89.89	0.00769278352118835\\
89.9	0.00769010560232079\\
89.91	0.00768742658713318\\
89.92	0.00768474647466324\\
89.93	0.0076820652639474\\
89.94	0.00767938295402073\\
89.95	0.00767669954391702\\
89.96	0.0076740150326687\\
89.97	0.00767132941930689\\
89.98	0.00766864270286137\\
89.99	0.00766595488236059\\
90	0.00766326595683167\\
90.01	0.00766057592530038\\
90.02	0.00765788478679116\\
90.03	0.0076551925403271\\
90.04	0.00765249918492996\\
90.05	0.00764980471962014\\
90.06	0.0076471091434167\\
90.07	0.00764441245533735\\
90.08	0.00764171465439843\\
90.09	0.00763901573961494\\
90.1	0.00763631571000051\\
90.11	0.00763361456456744\\
90.12	0.00763091230232663\\
90.13	0.00762820892228763\\
90.14	0.00762550442345864\\
90.15	0.00762279880484646\\
90.16	0.00762009206545654\\
90.17	0.00761738420429295\\
90.18	0.00761467522035838\\
90.19	0.00761196511265416\\
90.2	0.00760925388018021\\
90.21	0.0076065415219351\\
90.22	0.00760382803691599\\
90.23	0.00760111342411865\\
90.24	0.0075983976825375\\
90.25	0.00759568081116552\\
90.26	0.00759296280899432\\
90.27	0.00759024367501411\\
90.28	0.00758752340821371\\
90.29	0.00758480200758052\\
90.3	0.00758207947210056\\
90.31	0.00757935580075843\\
90.32	0.00757663099253732\\
90.33	0.00757390504641902\\
90.34	0.00757117796138391\\
90.35	0.00756844973641094\\
90.36	0.00756572037047767\\
90.37	0.00756298986256021\\
90.38	0.00756025821163327\\
90.39	0.00755752541667014\\
90.4	0.00755479147664267\\
90.41	0.00755205639052129\\
90.42	0.007549320157275\\
90.43	0.00754658277587137\\
90.44	0.00754384424527653\\
90.45	0.00754110456445518\\
90.46	0.00753836373237058\\
90.47	0.00753562174798455\\
90.48	0.00753287861025747\\
90.49	0.00753013431814826\\
90.5	0.00752738887061441\\
90.51	0.00752464226661196\\
90.52	0.00752189450509549\\
90.53	0.00751914558501812\\
90.54	0.00751639550533154\\
90.55	0.00751364426498596\\
90.56	0.00751089186293013\\
90.57	0.00750813829811135\\
90.58	0.00750538356947544\\
90.59	0.00750262767596678\\
90.6	0.00749987061652825\\
90.61	0.00749711239010129\\
90.62	0.00749435299562583\\
90.63	0.00749159243204036\\
90.64	0.00748883069828188\\
90.65	0.00748606779328591\\
90.66	0.00748330371598649\\
90.67	0.00748053846531617\\
90.68	0.00747777204020604\\
90.69	0.00747500443958566\\
90.7	0.00747223566238314\\
90.71	0.00746946570752509\\
90.72	0.00746669457393661\\
90.73	0.00746392226054132\\
90.74	0.00746114876626135\\
90.75	0.00745837409001731\\
90.76	0.00745559823072832\\
90.77	0.007452821187312\\
90.78	0.00745004295868447\\
90.79	0.00744726354376032\\
90.8	0.00744448294145266\\
90.81	0.00744170115067307\\
90.82	0.00743891817033163\\
90.83	0.0074361339993369\\
90.84	0.00743334863659592\\
90.85	0.00743056208101423\\
90.86	0.00742777433149582\\
90.87	0.00742498538694319\\
90.88	0.0074221952462573\\
90.89	0.00741940390833759\\
90.9	0.00741661137208196\\
90.91	0.00741381763638682\\
90.92	0.007411022700147\\
90.93	0.00740822656225584\\
90.94	0.00740542922160512\\
90.95	0.0074026306770851\\
90.96	0.0073998309275845\\
90.97	0.00739702997199051\\
90.98	0.00739422780918875\\
90.99	0.00739142443806334\\
91	0.00738861985749683\\
91.01	0.00738581406637025\\
91.02	0.00738300706356305\\
91.03	0.00738019884795316\\
91.04	0.00737738941841696\\
91.05	0.00737457877382928\\
91.06	0.00737176691306338\\
91.07	0.00736895383499099\\
91.08	0.00736613953848229\\
91.09	0.00736332402240588\\
91.1	0.00736050728562883\\
91.11	0.00735768932701663\\
91.12	0.00735487014543325\\
91.13	0.00735204973974105\\
91.14	0.00734922810880086\\
91.15	0.00734640525147195\\
91.16	0.00734358116661203\\
91.17	0.00734075585307722\\
91.18	0.00733792930972211\\
91.19	0.0073351015353997\\
91.2	0.00733227252896143\\
91.21	0.00732944228925719\\
91.22	0.00732661081513529\\
91.23	0.00732377810544246\\
91.24	0.00732094415902388\\
91.25	0.00731810897472314\\
91.26	0.0073152725513823\\
91.27	0.0073124348878418\\
91.28	0.00730959598294055\\
91.29	0.00730675583551585\\
91.3	0.00730391444440346\\
91.31	0.00730107180843756\\
91.32	0.00729822792645075\\
91.33	0.00729538279727406\\
91.34	0.00729253641973696\\
91.35	0.00728968879266732\\
91.36	0.00728683991489147\\
91.37	0.00728398978523414\\
91.38	0.0072811384025185\\
91.39	0.00727828576556614\\
91.4	0.00727543187319709\\
91.41	0.0072725767242298\\
91.42	0.00726972031748114\\
91.43	0.00726686265176644\\
91.44	0.00726400372589941\\
91.45	0.00726114353869222\\
91.46	0.00725828208895548\\
91.47	0.00725541937549821\\
91.48	0.00725255539712787\\
91.49	0.00724969015265035\\
91.5	0.00724682364086997\\
91.51	0.00724395586058949\\
91.52	0.00724108681061012\\
91.53	0.00723821648973147\\
91.54	0.00723534489675162\\
91.55	0.00723247203046708\\
91.56	0.0072295978896728\\
91.57	0.00722672247316217\\
91.58	0.00722384577972703\\
91.59	0.00722096780815766\\
91.6	0.00721808855724279\\
91.61	0.00721520802576959\\
91.62	0.00721232621252371\\
91.63	0.00720944311628922\\
91.64	0.00720655873584867\\
91.65	0.00720367306998306\\
91.66	0.00720078611747184\\
91.67	0.00719789787709294\\
91.68	0.00719500834762275\\
91.69	0.00719211752783614\\
91.7	0.00718922541650643\\
91.71	0.00718633201240543\\
91.72	0.00718343731430342\\
91.73	0.00718054132096917\\
91.74	0.00717764403116994\\
91.75	0.00717474544367146\\
91.76	0.00717184555723798\\
91.77	0.00716894437063221\\
91.78	0.0071660418826154\\
91.79	0.00716313809194727\\
91.8	0.00716023299738608\\
91.81	0.00715732659768859\\
91.82	0.00715441889161007\\
91.83	0.00715150987790433\\
91.84	0.00714859955532369\\
91.85	0.00714568792261902\\
91.86	0.00714277497853972\\
91.87	0.00713986072183375\\
91.88	0.00713694515124759\\
91.89	0.00713402826552631\\
91.9	0.00713111006341352\\
91.91	0.0071281905436514\\
91.92	0.00712526970498072\\
91.93	0.00712234754614082\\
91.94	0.00711942406586963\\
91.95	0.00711649926290369\\
91.96	0.0071135731359781\\
91.97	0.00711064568382663\\
91.98	0.00710771690518163\\
91.99	0.00710478679877409\\
92	0.00710185536333362\\
92.01	0.00709892259758851\\
92.02	0.00709598850026565\\
92.03	0.00709305307009064\\
92.04	0.00709011630578772\\
92.05	0.00708717820607981\\
92.06	0.00708423876968855\\
92.07	0.00708129799533424\\
92.08	0.00707835588173591\\
92.09	0.00707541242761132\\
92.1	0.00707246763167695\\
92.11	0.00706952149264801\\
92.12	0.00706657400923848\\
92.13	0.00706362518016111\\
92.14	0.00706067500412742\\
92.15	0.00705772347984771\\
92.16	0.00705477060603111\\
92.17	0.00705181638138554\\
92.18	0.00704886080461776\\
92.19	0.00704590387443338\\
92.2	0.00704294558953686\\
92.21	0.00703998594863152\\
92.22	0.00703702495041959\\
92.23	0.00703406259360218\\
92.24	0.00703109887687934\\
92.25	0.00702813379895004\\
92.26	0.00702516735851219\\
92.27	0.00702219955426268\\
92.28	0.00701923038489738\\
92.29	0.00701625984911117\\
92.3	0.00701328794559793\\
92.31	0.00701031467305059\\
92.32	0.00700734003016113\\
92.33	0.00700436401562062\\
92.34	0.0070013866281192\\
92.35	0.00699840786634614\\
92.36	0.00699542772898985\\
92.37	0.00699244621473791\\
92.38	0.00698946332227703\\
92.39	0.00698647905029317\\
92.4	0.0069834933974715\\
92.41	0.00698050636249643\\
92.42	0.00697751794405164\\
92.43	0.00697452814082011\\
92.44	0.00697153695148414\\
92.45	0.00696854437472539\\
92.46	0.00696555040922485\\
92.47	0.00696255505366296\\
92.48	0.00695955830671953\\
92.49	0.00695656016707387\\
92.5	0.00695356063340475\\
92.51	0.00695055970439044\\
92.52	0.00694755737870877\\
92.53	0.00694455365503712\\
92.54	0.0069415485320525\\
92.55	0.0069385420084315\\
92.56	0.00693553408285043\\
92.57	0.00693252475398527\\
92.58	0.00692951402051172\\
92.59	0.00692650188110527\\
92.6	0.00692348833444118\\
92.61	0.00692047337919458\\
92.62	0.00691745701404044\\
92.63	0.00691443923765366\\
92.64	0.00691142004870908\\
92.65	0.00690839944588152\\
92.66	0.00690537742784582\\
92.67	0.00690235399327691\\
92.68	0.00689932914084981\\
92.69	0.00689630286923968\\
92.7	0.00689327517712188\\
92.71	0.00689024606317201\\
92.72	0.00688721552606594\\
92.73	0.00688418356447987\\
92.74	0.00688115017709037\\
92.75	0.00687811536257442\\
92.76	0.00687507911960947\\
92.77	0.00687204144687349\\
92.78	0.006869002343045\\
92.79	0.00686596180680315\\
92.8	0.00686291983682775\\
92.81	0.00685987643179932\\
92.82	0.00685683159039916\\
92.83	0.00685378531130941\\
92.84	0.00685073759321306\\
92.85	0.00684768843479407\\
92.86	0.00684463783473738\\
92.87	0.00684158579172898\\
92.88	0.00683853230445598\\
92.89	0.00683547737160667\\
92.9	0.00683242099187057\\
92.91	0.00682936316393849\\
92.92	0.00682630388650261\\
92.93	0.00682324315825653\\
92.94	0.00682018097789536\\
92.95	0.00681711734411574\\
92.96	0.00681405225561595\\
92.97	0.00681098571109597\\
92.98	0.00680791770925753\\
92.99	0.00680484824880419\\
93	0.00680177732844145\\
93.01	0.00679870494687675\\
93.02	0.0067956311028196\\
93.03	0.00679255579498165\\
93.04	0.00678947902207673\\
93.05	0.00678640078282096\\
93.06	0.00678332107593283\\
93.07	0.00678023990013325\\
93.08	0.00677715725414567\\
93.09	0.00677407313669611\\
93.1	0.0067709875465133\\
93.11	0.00676790048232872\\
93.12	0.00676481194287669\\
93.13	0.00676172192689448\\
93.14	0.00675863043312237\\
93.15	0.00675553746030373\\
93.16	0.00675244300718515\\
93.17	0.00674934707251648\\
93.18	0.00674624965505095\\
93.19	0.00674315075354524\\
93.2	0.00674005036675958\\
93.21	0.00673694849345786\\
93.22	0.00673384513240767\\
93.23	0.00673074028238045\\
93.24	0.00672763394215155\\
93.25	0.00672452611050033\\
93.26	0.00672141678621027\\
93.27	0.00671830596806909\\
93.28	0.00671519365486887\\
93.29	0.00671207984540615\\
93.3	0.0067089645384821\\
93.31	0.00670584773290257\\
93.32	0.0067027294274783\\
93.33	0.00669960962102498\\
93.34	0.00669648831236342\\
93.35	0.00669336550031966\\
93.36	0.00669024118372513\\
93.37	0.00668711536141677\\
93.38	0.00668398803223719\\
93.39	0.00668085919503478\\
93.4	0.0066777288486639\\
93.41	0.00667459699198497\\
93.42	0.0066714636238647\\
93.43	0.00666832874317616\\
93.44	0.00666519234879899\\
93.45	0.00666205443961956\\
93.46	0.00665891501453109\\
93.47	0.00665577407243386\\
93.48	0.00665263161223536\\
93.49	0.00664948763285045\\
93.5	0.00664634213320155\\
93.51	0.0066431951122188\\
93.52	0.00664004656884026\\
93.53	0.00663689650201209\\
93.54	0.0066337449106887\\
93.55	0.006630591793833\\
93.56	0.00662743715041655\\
93.57	0.00662428097941974\\
93.58	0.00662112327983205\\
93.59	0.00661796405065221\\
93.6	0.00661480329088839\\
93.61	0.00661164099955844\\
93.62	0.00660847717569011\\
93.63	0.00660531181832125\\
93.64	0.00660214492650001\\
93.65	0.0065989764992851\\
93.66	0.00659580653574599\\
93.67	0.00659263503496318\\
93.68	0.00658946199602838\\
93.69	0.00658628741804479\\
93.7	0.00658311130012732\\
93.71	0.00657993364140287\\
93.72	0.00657675444101053\\
93.73	0.00657357369810188\\
93.74	0.00657039141184122\\
93.75	0.00656720758140588\\
93.76	0.0065640222059864\\
93.77	0.0065608352847869\\
93.78	0.00655764681702529\\
93.79	0.00655445680193357\\
93.8	0.00655126523875814\\
93.81	0.00654807212676004\\
93.82	0.00654487746521529\\
93.83	0.00654168125341517\\
93.84	0.00653848349066652\\
93.85	0.00653528417629205\\
93.86	0.00653208330963066\\
93.87	0.00652888089003776\\
93.88	0.00652567691688559\\
93.89	0.00652247138956353\\
93.9	0.00651926430747846\\
93.91	0.00651605567005509\\
93.92	0.0065128454767363\\
93.93	0.00650963372698347\\
93.94	0.00650642042027688\\
93.95	0.00650320555611601\\
93.96	0.00649998913401998\\
93.97	0.00649677115352783\\
93.98	0.00649355161419897\\
93.99	0.00649033051561351\\
94	0.00648710785737267\\
94.01	0.00648388363909919\\
94.02	0.00648065786043769\\
94.03	0.00647743052105507\\
94.04	0.00647420162064096\\
94.05	0.00647097115890812\\
94.06	0.00646773913559281\\
94.07	0.0064645055504553\\
94.08	0.00646127040328024\\
94.09	0.00645803369387711\\
94.1	0.00645479542208068\\
94.11	0.00645155558775145\\
94.12	0.00644831419077609\\
94.13	0.00644507123106793\\
94.14	0.0064418267085674\\
94.15	0.00643858062324253\\
94.16	0.00643533297508941\\
94.17	0.00643208376413267\\
94.18	0.006428832990426\\
94.19	0.00642558065405262\\
94.2	0.00642232675512581\\
94.21	0.00641907129378937\\
94.22	0.00641581427021821\\
94.23	0.00641255568461881\\
94.24	0.00640929553722976\\
94.25	0.00640603382832234\\
94.26	0.00640277055820099\\
94.27	0.0063995057272039\\
94.28	0.00639623933570357\\
94.29	0.00639297138410732\\
94.3	0.00638970187285793\\
94.31	0.00638643080243412\\
94.32	0.00638315817335119\\
94.33	0.00637988398616159\\
94.34	0.0063766082414555\\
94.35	0.0063733309398614\\
94.36	0.00637005208204673\\
94.37	0.00636677166871842\\
94.38	0.00636348970062356\\
94.39	0.00636020617854997\\
94.4	0.00635692110332685\\
94.41	0.00635363447582538\\
94.42	0.0063503462969594\\
94.43	0.00634705656768597\\
94.44	0.00634376528900606\\
94.45	0.00634047246196518\\
94.46	0.00633717808765402\\
94.47	0.00633388216720914\\
94.48	0.00633058470181355\\
94.49	0.00632728569269746\\
94.5	0.00632398514113885\\
94.51	0.00632068304846423\\
94.52	0.00631737941604925\\
94.53	0.00631407424531936\\
94.54	0.00631076753775054\\
94.55	0.00630745929486992\\
94.56	0.0063041495182565\\
94.57	0.00630083820954181\\
94.58	0.00629752537041059\\
94.59	0.00629421100260147\\
94.6	0.00629089510790768\\
94.61	0.00628757768817767\\
94.62	0.00628425874531586\\
94.63	0.00628093828128328\\
94.64	0.00627761629809826\\
94.65	0.00627429279783709\\
94.66	0.00627096778263474\\
94.67	0.00626764125468547\\
94.68	0.00626431321624355\\
94.69	0.0062609836696239\\
94.7	0.00625765261720273\\
94.71	0.00625432006141824\\
94.72	0.00625098600477123\\
94.73	0.00624765044982574\\
94.74	0.00624431339920971\\
94.75	0.00624097485561557\\
94.76	0.00623763482180089\\
94.77	0.00623429330058894\\
94.78	0.00623095029486932\\
94.79	0.00622760580759851\\
94.8	0.00622425984180049\\
94.81	0.0062209124005672\\
94.82	0.00621756348705919\\
94.83	0.00621421310450603\\
94.84	0.00621086125620691\\
94.85	0.00620750794553105\\
94.86	0.00620415317591821\\
94.87	0.0062007969508791\\
94.88	0.00619743927399584\\
94.89	0.0061940801489223\\
94.9	0.00619071957938451\\
94.91	0.00618735756918099\\
94.92	0.00618399412218304\\
94.93	0.00618062924233505\\
94.94	0.00617726293365473\\
94.95	0.00617389520023333\\
94.96	0.00617052604623579\\
94.97	0.00616715547590091\\
94.98	0.00616378349354142\\
94.99	0.00616041010354405\\
95	0.00615703531036951\\
95.01	0.00615365911855247\\
95.02	0.00615028153270145\\
95.03	0.0061469025574987\\
95.04	0.00614352219769996\\
95.05	0.00614014045813423\\
95.06	0.00613675734370342\\
95.07	0.00613337285938207\\
95.08	0.00612998701021685\\
95.09	0.00612659980132612\\
95.1	0.00612321123789937\\
95.11	0.00611982132519654\\
95.12	0.00611643006854734\\
95.13	0.00611303747335045\\
95.14	0.00610964354507262\\
95.15	0.00610624828924772\\
95.16	0.00610285171147563\\
95.17	0.00609945381742111\\
95.18	0.00609605461281251\\
95.19	0.0060926541034404\\
95.2	0.00608925229515605\\
95.21	0.00608584919386986\\
95.22	0.00608244480554959\\
95.23	0.00607903913621852\\
95.24	0.00607563219195346\\
95.25	0.00607222397888259\\
95.26	0.00606881450318319\\
95.27	0.00606540377107922\\
95.28	0.0060619917888387\\
95.29	0.00605857856277099\\
95.3	0.00605516409922385\\
95.31	0.00605174840458029\\
95.32	0.00604833148525536\\
95.33	0.00604491334769256\\
95.34	0.00604149399836025\\
95.35	0.00603807344374767\\
95.36	0.00603465169036088\\
95.37	0.00603122874471834\\
95.38	0.00602780461334639\\
95.39	0.0060243793027744\\
95.4	0.00602095281952964\\
95.41	0.00601752517013196\\
95.42	0.00601409636108816\\
95.43	0.00601066639888601\\
95.44	0.0060072352899881\\
95.45	0.00600380304082522\\
95.46	0.00600036965778956\\
95.47	0.00599693514722748\\
95.48	0.00599349951543194\\
95.49	0.00599006276863461\\
95.5	0.00598662491299756\\
95.51	0.00598318595460457\\
95.52	0.00597974589945201\\
95.53	0.00597630475343936\\
95.54	0.0059728625223592\\
95.55	0.0059694192118868\\
95.56	0.00596597482756922\\
95.57	0.0059625293748139\\
95.58	0.00595908285887679\\
95.59	0.00595563528484986\\
95.6	0.00595218665764815\\
95.61	0.00594873698199617\\
95.62	0.00594528626241376\\
95.63	0.0059418345032013\\
95.64	0.00593838170842428\\
95.65	0.00593492788189724\\
95.66	0.00593147302716697\\
95.67	0.00592801714749502\\
95.68	0.00592456024583947\\
95.69	0.00592110232483595\\
95.7	0.00591764338677777\\
95.71	0.00591418343359536\\
95.72	0.00591072246683473\\
95.73	0.00590726048763511\\
95.74	0.00590379749670561\\
95.75	0.00590033349430102\\
95.76	0.00589686848019647\\
95.77	0.00589340245366119\\
95.78	0.00588993541343115\\
95.79	0.00588646735768057\\
95.8	0.00588299828399233\\
95.81	0.00587952818932717\\
95.82	0.00587605706999163\\
95.83	0.00587258492160477\\
95.84	0.00586911173906351\\
95.85	0.00586563751650662\\
95.86	0.00586216224727731\\
95.87	0.00585868592388429\\
95.88	0.00585520853796136\\
95.89	0.00585173008022541\\
95.9	0.00584825054043271\\
95.91	0.00584476990733362\\
95.92	0.0058412881686254\\
95.93	0.00583780531090334\\
95.94	0.00583432131960989\\
95.95	0.0058308361789819\\
95.96	0.00582734987199577\\
95.97	0.00582386238031058\\
95.98	0.00582037368420893\\
95.99	0.00581688376253568\\
96	0.00581339259263423\\
96.01	0.00580990015028039\\
96.02	0.00580640640961383\\
96.03	0.00580291134306688\\
96.04	0.00579941492129059\\
96.05	0.00579591711307814\\
96.06	0.00579241788528524\\
96.07	0.00578891720274764\\
96.08	0.00578541502819546\\
96.09	0.0057819113221644\\
96.1	0.0057784060429036\\
96.11	0.00577489914628007\\
96.12	0.00577139058567956\\
96.13	0.00576788031190376\\
96.14	0.00576436827306367\\
96.15	0.00576085441446907\\
96.16	0.00575733867851384\\
96.17	0.00575382100455711\\
96.18	0.00575030132879995\\
96.19	0.00574677958415764\\
96.2	0.00574325570012711\\
96.21	0.00573972960264963\\
96.22	0.00573620121396835\\
96.23	0.00573267045248077\\
96.24	0.00572913723258566\\
96.25	0.00572560146452449\\
96.26	0.00572206305421703\\
96.27	0.00571852190309097\\
96.28	0.0057149779079053\\
96.29	0.00571143101720071\\
96.3	0.00570788121361917\\
96.31	0.00570432847949472\\
96.32	0.00570077279684701\\
96.33	0.00569721414737485\\
96.34	0.00569365251244949\\
96.35	0.00569008787310787\\
96.36	0.00568652021004563\\
96.37	0.00568294950361001\\
96.38	0.00567937573379266\\
96.39	0.00567579888022218\\
96.4	0.00567221892215656\\
96.41	0.00566863583847552\\
96.42	0.00566504960767254\\
96.43	0.00566146020784689\\
96.44	0.00565786761669534\\
96.45	0.00565427181150378\\
96.46	0.00565067276913867\\
96.47	0.00564707046603822\\
96.48	0.00564346487820349\\
96.49	0.00563985598118921\\
96.5	0.00563624375009448\\
96.51	0.0056326281595532\\
96.52	0.00562900918372433\\
96.53	0.00562538679628198\\
96.54	0.00562176097040519\\
96.55	0.00561813167876761\\
96.56	0.00561449889352684\\
96.57	0.00561086258631366\\
96.58	0.0056072227282209\\
96.59	0.00560357928979222\\
96.6	0.0055999322410105\\
96.61	0.0055962815512861\\
96.62	0.00559262718944481\\
96.63	0.00558896912371551\\
96.64	0.00558530732171769\\
96.65	0.00558164175044853\\
96.66	0.00557797237626985\\
96.67	0.00557429916489471\\
96.68	0.00557062208137371\\
96.69	0.00556694109008106\\
96.7	0.00556325615470026\\
96.71	0.00555956723820957\\
96.72	0.00555587430286709\\
96.73	0.00555217731019556\\
96.74	0.00554847622096682\\
96.75	0.00554477099518593\\
96.76	0.00554106159207497\\
96.77	0.00553734797005646\\
96.78	0.00553363008673646\\
96.79	0.00552990789888725\\
96.8	0.00552618136242972\\
96.81	0.00552245043241529\\
96.82	0.0055187150630075\\
96.83	0.00551497520746319\\
96.84	0.00551123081811325\\
96.85	0.00550748184634301\\
96.86	0.00550372824257212\\
96.87	0.00549996995623409\\
96.88	0.00549620693575531\\
96.89	0.00549243912853368\\
96.9	0.00548866648091671\\
96.91	0.00548488893817924\\
96.92	0.00548110644450056\\
96.93	0.00547731894294114\\
96.94	0.0054735263754188\\
96.95	0.00546972868268438\\
96.96	0.00546592580429687\\
96.97	0.00546211767859804\\
96.98	0.00545830424268645\\
96.99	0.00545448543239097\\
97	0.0054506611822437\\
97.01	0.00544683142545227\\
97.02	0.00544299609387157\\
97.03	0.00543915511797489\\
97.04	0.00543530842682438\\
97.05	0.0054314559480409\\
97.06	0.0054275976077732\\
97.07	0.00542373333066646\\
97.08	0.00541986303983011\\
97.09	0.00541598665680497\\
97.1	0.00541210410152966\\
97.11	0.0054082152923063\\
97.12	0.00540432014576545\\
97.13	0.00540041857683028\\
97.14	0.00539651049867997\\
97.15	0.00539259582271234\\
97.16	0.00538867445850557\\
97.17	0.00538474631377921\\
97.18	0.00538081129435426\\
97.19	0.00537686930411238\\
97.2	0.00537292024495425\\
97.21	0.00536896401675699\\
97.22	0.00536500051733065\\
97.23	0.00536102964237377\\
97.24	0.00535705128542793\\
97.25	0.00535306533783138\\
97.26	0.00534907168867152\\
97.27	0.00534507022473652\\
97.28	0.0053410608304657\\
97.29	0.00533704338789897\\
97.3	0.00533301777662507\\
97.31	0.0053289838737287\\
97.32	0.00532494155373651\\
97.33	0.00532089068856186\\
97.34	0.00531683114744844\\
97.35	0.00531276279691258\\
97.36	0.00530868550068429\\
97.37	0.0053045991196471\\
97.38	0.00530050351177645\\
97.39	0.00529639853207682\\
97.4	0.0052922840325174\\
97.41	0.00528815986196644\\
97.42	0.00528402586612404\\
97.43	0.00527988188745356\\
97.44	0.00527572776511142\\
97.45	0.00527156333487545\\
97.46	0.00526738842907159\\
97.47	0.005263202876499\\
97.48	0.00525900650235349\\
97.49	0.00525479912814928\\
97.5	0.00525058057163902\\
97.51	0.00524635064673208\\
97.52	0.00524210916341094\\
97.53	0.0052378559276458\\
97.54	0.00523359074130728\\
97.55	0.0052293134020772\\
97.56	0.00522502370335732\\
97.57	0.00522072143417615\\
97.58	0.00521640637909359\\
97.59	0.00521207831810351\\
97.6	0.00520773702653417\\
97.61	0.00520338227494636\\
97.62	0.00519901382902937\\
97.63	0.00519463144949453\\
97.64	0.0051902348919665\\
97.65	0.00518582390687201\\
97.66	0.00518139823932621\\
97.67	0.00517695762901644\\
97.68	0.0051725018100834\\
97.69	0.0051680305109997\\
97.7	0.00516354345444049\\
97.71	0.00515904035715187\\
97.72	0.0051545209298204\\
97.73	0.00514998487693972\\
97.74	0.00514543189667416\\
97.75	0.00514086168071926\\
97.76	0.00513627391415914\\
97.77	0.00513166827532061\\
97.78	0.00512704443562405\\
97.79	0.00512240205943088\\
97.8	0.00511774080388754\\
97.81	0.00511306031876596\\
97.82	0.00510836024630044\\
97.83	0.00510364022102074\\
97.84	0.00509889986958149\\
97.85	0.00509413881058757\\
97.86	0.00508935665441567\\
97.87	0.00508455300303164\\
97.88	0.0050797274498038\\
97.89	0.00507487957931185\\
97.9	0.00507000896715153\\
97.91	0.00506511517973472\\
97.92	0.00506019777408504\\
97.93	0.00505525629762869\\
97.94	0.00505029028798053\\
97.95	0.00504529927272522\\
97.96	0.00504028276919333\\
97.97	0.00503524028423235\\
97.98	0.00503017131397231\\
97.99	0.00502507534358607\\
98	0.00501995184704409\\
98.01	0.00501480028686344\\
98.02	0.00500962011385106\\
98.03	0.00500441076684108\\
98.04	0.004999171672426\\
98.05	0.00499390224468163\\
98.06	0.00498860188488569\\
98.07	0.00498326998122972\\
98.08	0.00497790590852446\\
98.09	0.00497250902789814\\
98.1	0.00496707868648791\\
98.11	0.0049616142171239\\
98.12	0.00495611493800592\\
98.13	0.00495058015237261\\
98.14	0.00494500914816276\\
98.15	0.00493940119766865\\
98.16	0.00493375555718128\\
98.17	0.00492807146662713\\
98.18	0.00492234814919641\\
98.19	0.00491658481096249\\
98.2	0.00491078064049223\\
98.21	0.00490493480844718\\
98.22	0.00489904646717519\\
98.23	0.0048931147502923\\
98.24	0.00488713877225473\\
98.25	0.0048811176279205\\
98.26	0.00487505039210069\\
98.27	0.00486893611909984\\
98.28	0.00486277384224534\\
98.29	0.00485656257340549\\
98.3	0.00485030130249593\\
98.31	0.00484398899697416\\
98.32	0.00483762460132176\\
98.33	0.00483120703651418\\
98.34	0.00482473519947751\\
98.35	0.00481820796253208\\
98.36	0.00481162417282255\\
98.37	0.00480498265173392\\
98.38	0.00479828219429342\\
98.39	0.00479152156855752\\
98.4	0.00478469951498404\\
98.41	0.00477781474578868\\
98.42	0.00477086594428569\\
98.43	0.00476385176421216\\
98.44	0.00475677082903566\\
98.45	0.00474962173124451\\
98.46	0.00474240303162048\\
98.47	0.00473511325849326\\
98.48	0.00472775090697621\\
98.49	0.00472031443818303\\
98.5	0.0047128022784246\\
98.51	0.00470521281838562\\
98.52	0.00469754441228036\\
98.53	0.00468979537698703\\
98.54	0.00468196399116014\\
98.55	0.00467404849432016\\
98.56	0.00466604708591992\\
98.57	0.00465795792438712\\
98.58	0.00464977912614212\\
98.59	0.00464150876459045\\
98.6	0.00463314486908921\\
98.61	0.00462468542388671\\
98.62	0.00461612836703442\\
98.63	0.00460747158927059\\
98.64	0.00459871293287459\\
98.65	0.00458985019049112\\
98.66	0.0045808811039235\\
98.67	0.00457180336289497\\
98.68	0.00456261460377712\\
98.69	0.00455331240828454\\
98.7	0.00454389430213449\\
98.71	0.0045343577536707\\
98.72	0.00452470017245008\\
98.73	0.00451491890779132\\
98.74	0.00450501124728411\\
98.75	0.0044949744152578\\
98.76	0.00448480557120822\\
98.77	0.00447450180818146\\
98.78	0.004464060151113\\
98.79	0.0044534775551211\\
98.8	0.00444275090375275\\
98.81	0.00443187700718079\\
98.82	0.00442085260035056\\
98.83	0.00440967434107453\\
98.84	0.00439833880807312\\
98.85	0.00438684249896001\\
98.86	0.00437518182817017\\
98.87	0.00436335312482856\\
98.88	0.00435135263055766\\
98.89	0.00433917649718688\\
98.9	0.00432682078439225\\
98.91	0.00431428145730063\\
98.92	0.00430155438402334\\
98.93	0.00428863533311712\\
98.94	0.00427551997096969\\
98.95	0.00426220385910736\\
98.96	0.0042486824514219\\
98.97	0.00423495109131445\\
98.98	0.00422100500875479\\
98.99	0.00420683931724858\\
99	0.00419244901071166\\
99.01	0.00417782896024826\\
99.02	0.00416297391082926\\
99.03	0.00414787847786715\\
99.04	0.00413253714368365\\
99.05	0.00411694425386603\\
99.06	0.00410109401350807\\
99.07	0.00408498048333106\\
99.08	0.00406859757568049\\
99.09	0.00405193905039347\\
99.1	0.0040349985105342\\
99.11	0.00401776939798825\\
99.12	0.00400024498891175\\
99.13	0.00398241838903006\\
99.14	0.0039642825287796\\
99.15	0.00394583015828669\\
99.16	0.00392705384217749\\
99.17	0.00390794595421406\\
99.18	0.00388849867174659\\
99.19	0.00386870396997163\\
99.2	0.00384855361599089\\
99.21	0.00382803916266256\\
99.22	0.003807151942236\\
99.23	0.00378588305976082\\
99.24	0.0037642233862604\\
99.25	0.0037421635516597\\
99.26	0.00371969393745665\\
99.27	0.0036968046691257\\
99.28	0.00367348560824153\\
99.29	0.0036497263443105\\
99.3	0.00362551618629647\\
99.31	0.00360084413655559\\
99.32	0.00357569888582327\\
99.33	0.00355006881594236\\
99.34	0.00352394198998781\\
99.35	0.00349730614200647\\
99.36	0.00347014866635337\\
99.37	0.00344245660660522\\
99.38	0.00341425888549942\\
99.39	0.0033855659980539\\
99.4	0.00335636581065441\\
99.41	0.00332664584999986\\
99.42	0.00329639329011699\\
99.43	0.00326559492261193\\
99.44	0.00323425675308237\\
99.45	0.00320238362600423\\
99.46	0.0031699622610237\\
99.47	0.00313697899702963\\
99.48	0.00310341977913769\\
99.49	0.00306927014512682\\
99.5	0.00303451521130009\\
99.51	0.00299913965773969\\
99.52	0.00296312771292437\\
99.53	0.00292646313767548\\
99.54	0.00288912920839561\\
99.55	0.00285110869956154\\
99.56	0.00281238386543081\\
99.57	0.00277293642091844\\
99.58	0.00273274752159758\\
99.59	0.00269179774277483\\
99.6	0.00265006705758736\\
99.61	0.00260753481406595\\
99.62	0.00256417971110353\\
99.63	0.0025199797732653\\
99.64	0.00247491232437151\\
99.65	0.0024289539597794\\
99.66	0.00238208051728548\\
99.67	0.00233426704656365\\
99.68	0.00228548777704836\\
99.69	0.00223571608416403\\
99.7	0.00218492445379856\\
99.71	0.00213308444490778\\
99.72	0.00208016665012936\\
99.73	0.00202614065427561\\
99.74	0.00197097499056455\\
99.75	0.00191463709443744\\
99.76	0.00185709325479893\\
99.77	0.00179830856250254\\
99.78	0.00173824685588996\\
99.79	0.00167687066317651\\
99.8	0.00161414114145789\\
99.81	0.00155001801209408\\
99.82	0.00148445949220548\\
99.83	0.00141742222199301\\
99.84	0.00134886118756874\\
99.85	0.00127872963895532\\
99.86	0.00120697900288187\\
99.87	0.00113355878996945\\
99.88	0.00105841649586151\\
99.89	0.000981497495812871\\
99.9	0.000902744932204142\\
99.91	0.000822099594396612\\
99.92	0.000739499790285361\\
99.93	0.000654881208843966\\
99.94	0.000568176772882604\\
99.95	0.000479316481161323\\
99.96	0.000388227238910565\\
99.97	0.000294832675710451\\
99.98	0.000199052949567459\\
99.99	0.000100804535899853\\
100	0\\
};
\addlegendentry{$q=1$};

\addplot [color=red,solid,forget plot]
  table[row sep=crcr]{%
0.01	0.00402362825950001\\
0.02	0.00402362734089105\\
0.03	0.00402362641795564\\
0.04	0.00402362549065522\\
0.05	0.00402362455895159\\
0.06	0.00402362362280698\\
0.07	0.00402362268218406\\
0.08	0.00402362173704602\\
0.09	0.00402362078735659\\
0.1	0.00402361983308011\\
0.11	0.00402361887418157\\
0.12	0.00402361791062666\\
0.13	0.00402361694238187\\
0.14	0.00402361596941448\\
0.15	0.00402361499169268\\
0.16	0.00402361400918561\\
0.17	0.00402361302186344\\
0.18	0.00402361202969741\\
0.19	0.00402361103265996\\
0.2	0.00402361003072477\\
0.21	0.00402360902386682\\
0.22	0.00402360801206254\\
0.23	0.00402360699528982\\
0.24	0.00402360597352817\\
0.25	0.00402360494675876\\
0.26	0.00402360391496456\\
0.27	0.00402360287813041\\
0.28	0.00402360183624317\\
0.29	0.00402360078929177\\
0.3	0.00402359973726739\\
0.31	0.00402359868016353\\
0.32	0.00402359761797617\\
0.33	0.00402359655070388\\
0.34	0.00402359547834797\\
0.35	0.00402359440091262\\
0.36	0.00402359331840504\\
0.37	0.00402359223083561\\
0.38	0.00402359113821804\\
0.39	0.00402359004056954\\
0.4	0.00402358893791101\\
0.41	0.00402358783026716\\
0.42	0.00402358671766677\\
0.43	0.00402358560014283\\
0.44	0.00402358447773277\\
0.45	0.00402358335047864\\
0.46	0.00402358221842734\\
0.47	0.00402358108163085\\
0.48	0.00402357994014646\\
0.49	0.004023578794037\\
0.5	0.00402357764337111\\
0.51	0.00402357648822348\\
0.52	0.00402357532867513\\
0.53	0.00402357416481369\\
0.54	0.00402357299673368\\
0.55	0.00402357182453686\\
0.56	0.00402357064833246\\
0.57	0.00402356946823757\\
0.58	0.00402356828437746\\
0.59	0.00402356709688592\\
0.6	0.00402356590590564\\
0.61	0.00402356471158856\\
0.62	0.00402356351409629\\
0.63	0.0040235623136005\\
0.64	0.00402356111028332\\
0.65	0.00402355990433783\\
0.66	0.00402355869596844\\
0.67	0.00402355748539142\\
0.68	0.00402355627283534\\
0.69	0.0040235550585416\\
0.7	0.00402355384276494\\
0.71	0.00402355262577399\\
0.72	0.00402355140785179\\
0.73	0.00402355018929643\\
0.74	0.00402354897027454\\
0.75	0.00402354775078594\\
0.76	0.00402354653083042\\
0.77	0.00402354531040779\\
0.78	0.00402354408951786\\
0.79	0.00402354286816043\\
0.8	0.0040235416463353\\
0.81	0.0040235404240423\\
0.82	0.00402353920128121\\
0.83	0.00402353797805185\\
0.84	0.00402353675435402\\
0.85	0.00402353553018753\\
0.86	0.00402353430555218\\
0.87	0.00402353308044777\\
0.88	0.00402353185487412\\
0.89	0.00402353062883102\\
0.9	0.00402352940231829\\
0.91	0.00402352817533572\\
0.92	0.00402352694788313\\
0.93	0.0040235257199603\\
0.94	0.00402352449156706\\
0.95	0.00402352326270321\\
0.96	0.00402352203336854\\
0.97	0.00402352080356286\\
0.98	0.00402351957328599\\
0.99	0.00402351834253771\\
1	0.00402351711131784\\
1.01	0.00402351587962618\\
1.02	0.00402351464746253\\
1.03	0.00402351341482669\\
1.04	0.00402351218171848\\
1.05	0.00402351094813769\\
1.06	0.00402350971408412\\
1.07	0.00402350847955759\\
1.08	0.00402350724455789\\
1.09	0.00402350600908482\\
1.1	0.00402350477313819\\
1.11	0.0040235035367178\\
1.12	0.00402350229982346\\
1.13	0.00402350106245497\\
1.14	0.00402349982461212\\
1.15	0.00402349858629473\\
1.16	0.00402349734750259\\
1.17	0.00402349610823551\\
1.18	0.00402349486849328\\
1.19	0.00402349362827572\\
1.2	0.00402349238758262\\
1.21	0.00402349114641378\\
1.22	0.00402348990476901\\
1.23	0.00402348866264811\\
1.24	0.00402348742005088\\
1.25	0.00402348617697712\\
1.26	0.00402348493342663\\
1.27	0.00402348368939922\\
1.28	0.00402348244489469\\
1.29	0.00402348119991282\\
1.3	0.00402347995445344\\
1.31	0.00402347870851633\\
1.32	0.0040234774621013\\
1.33	0.00402347621520816\\
1.34	0.00402347496783669\\
1.35	0.0040234737199867\\
1.36	0.004023472471658\\
1.37	0.00402347122285038\\
1.38	0.00402346997356363\\
1.39	0.00402346872379757\\
1.4	0.004023467473552\\
1.41	0.0040234662228267\\
1.42	0.00402346497162148\\
1.43	0.00402346371993615\\
1.44	0.0040234624677705\\
1.45	0.00402346121512433\\
1.46	0.00402345996199744\\
1.47	0.00402345870838962\\
1.48	0.00402345745430069\\
1.49	0.00402345619973044\\
1.5	0.00402345494467865\\
1.51	0.00402345368914515\\
1.52	0.00402345243312972\\
1.53	0.00402345117663216\\
1.54	0.00402344991965228\\
1.55	0.00402344866218987\\
1.56	0.00402344740424472\\
1.57	0.00402344614581665\\
1.58	0.00402344488690544\\
1.59	0.00402344362751089\\
1.6	0.0040234423676328\\
1.61	0.00402344110727098\\
1.62	0.00402343984642521\\
1.63	0.00402343858509529\\
1.64	0.00402343732328103\\
1.65	0.00402343606098223\\
1.66	0.00402343479819866\\
1.67	0.00402343353493015\\
1.68	0.00402343227117647\\
1.69	0.00402343100693744\\
1.7	0.00402342974221284\\
1.71	0.00402342847700247\\
1.72	0.00402342721130614\\
1.73	0.00402342594512363\\
1.74	0.00402342467845474\\
1.75	0.00402342341129928\\
1.76	0.00402342214365704\\
1.77	0.00402342087552781\\
1.78	0.00402341960691138\\
1.79	0.00402341833780757\\
1.8	0.00402341706821615\\
1.81	0.00402341579813693\\
1.82	0.00402341452756971\\
1.83	0.00402341325651427\\
1.84	0.00402341198497042\\
1.85	0.00402341071293796\\
1.86	0.00402340944041666\\
1.87	0.00402340816740634\\
1.88	0.00402340689390678\\
1.89	0.00402340561991779\\
1.9	0.00402340434543915\\
1.91	0.00402340307047067\\
1.92	0.00402340179501213\\
1.93	0.00402340051906333\\
1.94	0.00402339924262406\\
1.95	0.00402339796569413\\
1.96	0.00402339668827332\\
1.97	0.00402339541036143\\
1.98	0.00402339413195826\\
1.99	0.00402339285306359\\
2	0.00402339157367722\\
2.01	0.00402339029379895\\
2.02	0.00402338901342857\\
2.03	0.00402338773256588\\
2.04	0.00402338645121066\\
2.05	0.0040233851693627\\
2.06	0.00402338388702181\\
2.07	0.00402338260418778\\
2.08	0.0040233813208604\\
2.09	0.00402338003703946\\
2.1	0.00402337875272476\\
2.11	0.00402337746791609\\
2.12	0.00402337618261324\\
2.13	0.004023374896816\\
2.14	0.00402337361052417\\
2.15	0.00402337232373754\\
2.16	0.0040233710364559\\
2.17	0.00402336974867904\\
2.18	0.00402336846040676\\
2.19	0.00402336717163885\\
2.2	0.0040233658823751\\
2.21	0.00402336459261531\\
2.22	0.00402336330235926\\
2.23	0.00402336201160674\\
2.24	0.00402336072035755\\
2.25	0.00402335942861148\\
2.26	0.00402335813636832\\
2.27	0.00402335684362786\\
2.28	0.00402335555038989\\
2.29	0.0040233542566542\\
2.3	0.0040233529624206\\
2.31	0.00402335166768885\\
2.32	0.00402335037245876\\
2.33	0.00402334907673012\\
2.34	0.00402334778050272\\
2.35	0.00402334648377634\\
2.36	0.00402334518655078\\
2.37	0.00402334388882584\\
2.38	0.00402334259060128\\
2.39	0.00402334129187692\\
2.4	0.00402333999265253\\
2.41	0.00402333869292791\\
2.42	0.00402333739270285\\
2.43	0.00402333609197713\\
2.44	0.00402333479075056\\
2.45	0.00402333348902291\\
2.46	0.00402333218679397\\
2.47	0.00402333088406354\\
2.48	0.0040233295808314\\
2.49	0.00402332827709734\\
2.5	0.00402332697286116\\
2.51	0.00402332566812264\\
2.52	0.00402332436288156\\
2.53	0.00402332305713773\\
2.54	0.00402332175089091\\
2.55	0.00402332044414091\\
2.56	0.00402331913688751\\
2.57	0.00402331782913051\\
2.58	0.00402331652086968\\
2.59	0.00402331521210482\\
2.6	0.00402331390283571\\
2.61	0.00402331259306214\\
2.62	0.0040233112827839\\
2.63	0.00402330997200078\\
2.64	0.00402330866071256\\
2.65	0.00402330734891904\\
2.66	0.00402330603661999\\
2.67	0.00402330472381521\\
2.68	0.00402330341050448\\
2.69	0.00402330209668759\\
2.7	0.00402330078236432\\
2.71	0.00402329946753447\\
2.72	0.00402329815219781\\
2.73	0.00402329683635415\\
2.74	0.00402329552000325\\
2.75	0.00402329420314492\\
2.76	0.00402329288577892\\
2.77	0.00402329156790506\\
2.78	0.00402329024952311\\
2.79	0.00402328893063287\\
2.8	0.0040232876112341\\
2.81	0.00402328629132662\\
2.82	0.00402328497091019\\
2.83	0.00402328364998461\\
2.84	0.00402328232854965\\
2.85	0.00402328100660511\\
2.86	0.00402327968415076\\
2.87	0.00402327836118641\\
2.88	0.00402327703771182\\
2.89	0.00402327571372678\\
2.9	0.00402327438923108\\
2.91	0.00402327306422451\\
2.92	0.00402327173870684\\
2.93	0.00402327041267786\\
2.94	0.00402326908613736\\
2.95	0.00402326775908511\\
2.96	0.00402326643152092\\
2.97	0.00402326510344454\\
2.98	0.00402326377485578\\
2.99	0.00402326244575441\\
3	0.00402326111614022\\
3.01	0.00402325978601299\\
3.02	0.00402325845537251\\
3.03	0.00402325712421855\\
3.04	0.0040232557925509\\
3.05	0.00402325446036935\\
3.06	0.00402325312767367\\
3.07	0.00402325179446366\\
3.08	0.00402325046073908\\
3.09	0.00402324912649973\\
3.1	0.00402324779174539\\
3.11	0.00402324645647583\\
3.12	0.00402324512069085\\
3.13	0.00402324378439022\\
3.14	0.00402324244757373\\
3.15	0.00402324111024115\\
3.16	0.00402323977239227\\
3.17	0.00402323843402687\\
3.18	0.00402323709514473\\
3.19	0.00402323575574564\\
3.2	0.00402323441582937\\
3.21	0.00402323307539571\\
3.22	0.00402323173444444\\
3.23	0.00402323039297534\\
3.24	0.00402322905098818\\
3.25	0.00402322770848275\\
3.26	0.00402322636545883\\
3.27	0.00402322502191621\\
3.28	0.00402322367785465\\
3.29	0.00402322233327395\\
3.3	0.00402322098817388\\
3.31	0.00402321964255422\\
3.32	0.00402321829641475\\
3.33	0.00402321694975525\\
3.34	0.00402321560257551\\
3.35	0.0040232142548753\\
3.36	0.0040232129066544\\
3.37	0.00402321155791259\\
3.38	0.00402321020864965\\
3.39	0.00402320885886536\\
3.4	0.00402320750855949\\
3.41	0.00402320615773183\\
3.42	0.00402320480638216\\
3.43	0.00402320345451025\\
3.44	0.00402320210211588\\
3.45	0.00402320074919884\\
3.46	0.0040231993957589\\
3.47	0.00402319804179583\\
3.48	0.00402319668730943\\
3.49	0.00402319533229945\\
3.5	0.00402319397676569\\
3.51	0.00402319262070792\\
3.52	0.00402319126412592\\
3.53	0.00402318990701946\\
3.54	0.00402318854938833\\
3.55	0.0040231871912323\\
3.56	0.00402318583255114\\
3.57	0.00402318447334464\\
3.58	0.00402318311361258\\
3.59	0.00402318175335472\\
3.6	0.00402318039257085\\
3.61	0.00402317903126074\\
3.62	0.00402317766942418\\
3.63	0.00402317630706092\\
3.64	0.00402317494417076\\
3.65	0.00402317358075347\\
3.66	0.00402317221680882\\
3.67	0.0040231708523366\\
3.68	0.00402316948733657\\
3.69	0.00402316812180851\\
3.7	0.00402316675575221\\
3.71	0.00402316538916743\\
3.72	0.00402316402205394\\
3.73	0.00402316265441154\\
3.74	0.00402316128623998\\
3.75	0.00402315991753904\\
3.76	0.00402315854830851\\
3.77	0.00402315717854815\\
3.78	0.00402315580825774\\
3.79	0.00402315443743706\\
3.8	0.00402315306608588\\
3.81	0.00402315169420397\\
3.82	0.00402315032179111\\
3.83	0.00402314894884707\\
3.84	0.00402314757537163\\
3.85	0.00402314620136456\\
3.86	0.00402314482682563\\
3.87	0.00402314345175462\\
3.88	0.0040231420761513\\
3.89	0.00402314070001545\\
3.9	0.00402313932334684\\
3.91	0.00402313794614524\\
3.92	0.00402313656841042\\
3.93	0.00402313519014217\\
3.94	0.00402313381134024\\
3.95	0.00402313243200442\\
3.96	0.00402313105213448\\
3.97	0.00402312967173018\\
3.98	0.00402312829079131\\
3.99	0.00402312690931763\\
4	0.00402312552730892\\
4.01	0.00402312414476495\\
4.02	0.00402312276168548\\
4.03	0.00402312137807031\\
4.04	0.00402311999391918\\
4.05	0.00402311860923188\\
4.06	0.00402311722400818\\
4.07	0.00402311583824785\\
4.08	0.00402311445195066\\
4.09	0.00402311306511638\\
4.1	0.00402311167774479\\
4.11	0.00402311028983564\\
4.12	0.00402310890138873\\
4.13	0.00402310751240381\\
4.14	0.00402310612288066\\
4.15	0.00402310473281904\\
4.16	0.00402310334221873\\
4.17	0.0040231019510795\\
4.18	0.00402310055940111\\
4.19	0.00402309916718335\\
4.2	0.00402309777442597\\
4.21	0.00402309638112875\\
4.22	0.00402309498729145\\
4.23	0.00402309359291386\\
4.24	0.00402309219799573\\
4.25	0.00402309080253683\\
4.26	0.00402308940653694\\
4.27	0.00402308800999582\\
4.28	0.00402308661291325\\
4.29	0.00402308521528899\\
4.3	0.00402308381712281\\
4.31	0.00402308241841448\\
4.32	0.00402308101916376\\
4.33	0.00402307961937044\\
4.34	0.00402307821903426\\
4.35	0.00402307681815501\\
4.36	0.00402307541673245\\
4.37	0.00402307401476635\\
4.38	0.00402307261225647\\
4.39	0.00402307120920259\\
4.4	0.00402306980560447\\
4.41	0.00402306840146188\\
4.42	0.00402306699677459\\
4.43	0.00402306559154236\\
4.44	0.00402306418576496\\
4.45	0.00402306277944216\\
4.46	0.00402306137257372\\
4.47	0.00402305996515941\\
4.48	0.004023058557199\\
4.49	0.00402305714869226\\
4.5	0.00402305573963895\\
4.51	0.00402305433003883\\
4.52	0.00402305291989168\\
4.53	0.00402305150919726\\
4.54	0.00402305009795533\\
4.55	0.00402304868616567\\
4.56	0.00402304727382802\\
4.57	0.00402304586094218\\
4.58	0.00402304444750788\\
4.59	0.00402304303352492\\
4.6	0.00402304161899304\\
4.61	0.00402304020391201\\
4.62	0.0040230387882816\\
4.63	0.00402303737210158\\
4.64	0.00402303595537169\\
4.65	0.00402303453809173\\
4.66	0.00402303312026144\\
4.67	0.00402303170188059\\
4.68	0.00402303028294895\\
4.69	0.00402302886346627\\
4.7	0.00402302744343233\\
4.71	0.00402302602284688\\
4.72	0.0040230246017097\\
4.73	0.00402302318002054\\
4.74	0.00402302175777917\\
4.75	0.00402302033498535\\
4.76	0.00402301891163885\\
4.77	0.00402301748773942\\
4.78	0.00402301606328683\\
4.79	0.00402301463828084\\
4.8	0.00402301321272122\\
4.81	0.00402301178660773\\
4.82	0.00402301035994014\\
4.83	0.00402300893271819\\
4.84	0.00402300750494166\\
4.85	0.00402300607661031\\
4.86	0.0040230046477239\\
4.87	0.00402300321828218\\
4.88	0.00402300178828494\\
4.89	0.00402300035773192\\
4.9	0.00402299892662288\\
4.91	0.0040229974949576\\
4.92	0.00402299606273582\\
4.93	0.00402299462995732\\
4.94	0.00402299319662184\\
4.95	0.00402299176272916\\
4.96	0.00402299032827903\\
4.97	0.00402298889327122\\
4.98	0.00402298745770548\\
4.99	0.00402298602158158\\
5	0.00402298458489927\\
5.01	0.00402298314765832\\
5.02	0.00402298170985848\\
5.03	0.00402298027149951\\
5.04	0.00402297883258118\\
5.05	0.00402297739310325\\
5.06	0.00402297595306547\\
5.07	0.00402297451246761\\
5.08	0.00402297307130942\\
5.09	0.00402297162959065\\
5.1	0.00402297018731109\\
5.11	0.00402296874447047\\
5.12	0.00402296730106856\\
5.13	0.00402296585710512\\
5.14	0.0040229644125799\\
5.15	0.00402296296749267\\
5.16	0.00402296152184319\\
5.17	0.0040229600756312\\
5.18	0.00402295862885647\\
5.19	0.00402295718151876\\
5.2	0.00402295573361783\\
5.21	0.00402295428515343\\
5.22	0.00402295283612531\\
5.23	0.00402295138653324\\
5.24	0.00402294993637699\\
5.25	0.00402294848565629\\
5.26	0.00402294703437091\\
5.27	0.0040229455825206\\
5.28	0.00402294413010514\\
5.29	0.00402294267712425\\
5.3	0.00402294122357772\\
5.31	0.00402293976946529\\
5.32	0.00402293831478671\\
5.33	0.00402293685954176\\
5.34	0.00402293540373017\\
5.35	0.0040229339473517\\
5.36	0.00402293249040612\\
5.37	0.00402293103289318\\
5.38	0.00402292957481263\\
5.39	0.00402292811616423\\
5.4	0.00402292665694774\\
5.41	0.0040229251971629\\
5.42	0.00402292373680948\\
5.43	0.00402292227588723\\
5.44	0.0040229208143959\\
5.45	0.00402291935233525\\
5.46	0.00402291788970503\\
5.47	0.00402291642650501\\
5.48	0.00402291496273492\\
5.49	0.00402291349839453\\
5.5	0.00402291203348359\\
5.51	0.00402291056800185\\
5.52	0.00402290910194908\\
5.53	0.00402290763532501\\
5.54	0.0040229061681294\\
5.55	0.00402290470036202\\
5.56	0.00402290323202261\\
5.57	0.00402290176311091\\
5.58	0.0040229002936267\\
5.59	0.00402289882356972\\
5.6	0.00402289735293972\\
5.61	0.00402289588173645\\
5.62	0.00402289440995968\\
5.63	0.00402289293760914\\
5.64	0.0040228914646846\\
5.65	0.0040228899911858\\
5.66	0.0040228885171125\\
5.67	0.00402288704246444\\
5.68	0.00402288556724138\\
5.69	0.00402288409144308\\
5.7	0.00402288261506928\\
5.71	0.00402288113811973\\
5.72	0.00402287966059419\\
5.73	0.0040228781824924\\
5.74	0.00402287670381412\\
5.75	0.0040228752245591\\
5.76	0.00402287374472709\\
5.77	0.00402287226431784\\
5.78	0.0040228707833311\\
5.79	0.00402286930176661\\
5.8	0.00402286781962414\\
5.81	0.00402286633690342\\
5.82	0.00402286485360422\\
5.83	0.00402286336972627\\
5.84	0.00402286188526933\\
5.85	0.00402286040023316\\
5.86	0.00402285891461749\\
5.87	0.00402285742842207\\
5.88	0.00402285594164667\\
5.89	0.00402285445429101\\
5.9	0.00402285296635486\\
5.91	0.00402285147783796\\
5.92	0.00402284998874007\\
5.93	0.00402284849906092\\
5.94	0.00402284700880027\\
5.95	0.00402284551795787\\
5.96	0.00402284402653346\\
5.97	0.00402284253452679\\
5.98	0.00402284104193761\\
5.99	0.00402283954876567\\
6	0.00402283805501071\\
6.01	0.00402283656067248\\
6.02	0.00402283506575073\\
6.03	0.00402283357024521\\
6.04	0.00402283207415566\\
6.05	0.00402283057748184\\
6.06	0.00402282908022348\\
6.07	0.00402282758238033\\
6.08	0.00402282608395214\\
6.09	0.00402282458493866\\
6.1	0.00402282308533963\\
6.11	0.00402282158515479\\
6.12	0.0040228200843839\\
6.13	0.00402281858302671\\
6.14	0.00402281708108294\\
6.15	0.00402281557855236\\
6.16	0.0040228140754347\\
6.17	0.00402281257172971\\
6.18	0.00402281106743715\\
6.19	0.00402280956255674\\
6.2	0.00402280805708824\\
6.21	0.00402280655103139\\
6.22	0.00402280504438593\\
6.23	0.00402280353715161\\
6.24	0.00402280202932819\\
6.25	0.00402280052091538\\
6.26	0.00402279901191295\\
6.27	0.00402279750232064\\
6.28	0.00402279599213818\\
6.29	0.00402279448136533\\
6.3	0.00402279297000183\\
6.31	0.00402279145804742\\
6.32	0.00402278994550184\\
6.33	0.00402278843236483\\
6.34	0.00402278691863615\\
6.35	0.00402278540431553\\
6.36	0.00402278388940271\\
6.37	0.00402278237389745\\
6.38	0.00402278085779947\\
6.39	0.00402277934110852\\
6.4	0.00402277782382435\\
6.41	0.00402277630594669\\
6.42	0.00402277478747529\\
6.43	0.00402277326840989\\
6.44	0.00402277174875023\\
6.45	0.00402277022849606\\
6.46	0.00402276870764711\\
6.47	0.00402276718620312\\
6.48	0.00402276566416384\\
6.49	0.00402276414152901\\
6.5	0.00402276261829837\\
6.51	0.00402276109447166\\
6.52	0.00402275957004861\\
6.53	0.00402275804502897\\
6.54	0.00402275651941249\\
6.55	0.00402275499319889\\
6.56	0.00402275346638792\\
6.57	0.00402275193897933\\
6.58	0.00402275041097284\\
6.59	0.0040227488823682\\
6.6	0.00402274735316515\\
6.61	0.00402274582336342\\
6.62	0.00402274429296276\\
6.63	0.00402274276196291\\
6.64	0.0040227412303636\\
6.65	0.00402273969816458\\
6.66	0.00402273816536557\\
6.67	0.00402273663196633\\
6.68	0.00402273509796658\\
6.69	0.00402273356336607\\
6.7	0.00402273202816453\\
6.71	0.00402273049236171\\
6.72	0.00402272895595733\\
6.73	0.00402272741895115\\
6.74	0.00402272588134289\\
6.75	0.00402272434313229\\
6.76	0.00402272280431909\\
6.77	0.00402272126490302\\
6.78	0.00402271972488384\\
6.79	0.00402271818426126\\
6.8	0.00402271664303502\\
6.81	0.00402271510120487\\
6.82	0.00402271355877054\\
6.83	0.00402271201573177\\
6.84	0.00402271047208828\\
6.85	0.00402270892783983\\
6.86	0.00402270738298614\\
6.87	0.00402270583752694\\
6.88	0.00402270429146198\\
6.89	0.004022702744791\\
6.9	0.00402270119751371\\
6.91	0.00402269964962987\\
6.92	0.0040226981011392\\
6.93	0.00402269655204144\\
6.94	0.00402269500233633\\
6.95	0.00402269345202359\\
6.96	0.00402269190110297\\
6.97	0.0040226903495742\\
6.98	0.00402268879743701\\
6.99	0.00402268724469113\\
7	0.00402268569133631\\
7.01	0.00402268413737226\\
7.02	0.00402268258279874\\
7.03	0.00402268102761547\\
7.04	0.00402267947182217\\
7.05	0.00402267791541859\\
7.06	0.00402267635840447\\
7.07	0.00402267480077952\\
7.08	0.00402267324254349\\
7.09	0.00402267168369611\\
7.1	0.00402267012423711\\
7.11	0.00402266856416622\\
7.12	0.00402266700348317\\
7.13	0.0040226654421877\\
7.14	0.00402266388027954\\
7.15	0.00402266231775841\\
7.16	0.00402266075462406\\
7.17	0.00402265919087622\\
7.18	0.0040226576265146\\
7.19	0.00402265606153895\\
7.2	0.00402265449594899\\
7.21	0.00402265292974447\\
7.22	0.0040226513629251\\
7.23	0.00402264979549062\\
7.24	0.00402264822744075\\
7.25	0.00402264665877524\\
7.26	0.00402264508949381\\
7.27	0.00402264351959618\\
7.28	0.00402264194908209\\
7.29	0.00402264037795127\\
7.3	0.00402263880620344\\
7.31	0.00402263723383835\\
7.32	0.00402263566085571\\
7.33	0.00402263408725525\\
7.34	0.00402263251303671\\
7.35	0.00402263093819981\\
7.36	0.00402262936274429\\
7.37	0.00402262778666986\\
7.38	0.00402262620997627\\
7.39	0.00402262463266323\\
7.4	0.00402262305473048\\
7.41	0.00402262147617774\\
7.42	0.00402261989700474\\
7.43	0.00402261831721121\\
7.44	0.00402261673679687\\
7.45	0.00402261515576146\\
7.46	0.0040226135741047\\
7.47	0.00402261199182631\\
7.48	0.00402261040892603\\
7.49	0.00402260882540358\\
7.5	0.00402260724125869\\
7.51	0.00402260565649109\\
7.52	0.0040226040711005\\
7.53	0.00402260248508664\\
7.54	0.00402260089844925\\
7.55	0.00402259931118804\\
7.56	0.00402259772330275\\
7.57	0.0040225961347931\\
7.58	0.00402259454565881\\
7.59	0.00402259295589962\\
7.6	0.00402259136551525\\
7.61	0.00402258977450541\\
7.62	0.00402258818286984\\
7.63	0.00402258659060826\\
7.64	0.0040225849977204\\
7.65	0.00402258340420598\\
7.66	0.00402258181006472\\
7.67	0.00402258021529634\\
7.68	0.00402257861990059\\
7.69	0.00402257702387717\\
7.7	0.00402257542722581\\
7.71	0.00402257382994623\\
7.72	0.00402257223203816\\
7.73	0.00402257063350132\\
7.74	0.00402256903433544\\
7.75	0.00402256743454023\\
7.76	0.00402256583411542\\
7.77	0.00402256423306073\\
7.78	0.00402256263137589\\
7.79	0.00402256102906061\\
7.8	0.00402255942611462\\
7.81	0.00402255782253765\\
7.82	0.0040225562183294\\
7.83	0.00402255461348961\\
7.84	0.004022553008018\\
7.85	0.00402255140191429\\
7.86	0.00402254979517819\\
7.87	0.00402254818780944\\
7.88	0.00402254657980775\\
7.89	0.00402254497117284\\
7.9	0.00402254336190443\\
7.91	0.00402254175200225\\
7.92	0.00402254014146601\\
7.93	0.00402253853029544\\
7.94	0.00402253691849025\\
7.95	0.00402253530605017\\
7.96	0.00402253369297491\\
7.97	0.00402253207926419\\
7.98	0.00402253046491775\\
7.99	0.00402252884993528\\
8	0.00402252723431652\\
8.01	0.00402252561806117\\
8.02	0.00402252400116897\\
8.03	0.00402252238363963\\
8.04	0.00402252076547287\\
8.05	0.0040225191466684\\
8.06	0.00402251752722595\\
8.07	0.00402251590714523\\
8.08	0.00402251428642596\\
8.09	0.00402251266506787\\
8.1	0.00402251104307066\\
8.11	0.00402250942043405\\
8.12	0.00402250779715777\\
8.13	0.00402250617324153\\
8.14	0.00402250454868504\\
8.15	0.00402250292348803\\
8.16	0.00402250129765021\\
8.17	0.0040224996711713\\
8.18	0.00402249804405101\\
8.19	0.00402249641628906\\
8.2	0.00402249478788516\\
8.21	0.00402249315883904\\
8.22	0.00402249152915041\\
8.23	0.00402248989881898\\
8.24	0.00402248826784448\\
8.25	0.0040224866362266\\
8.26	0.00402248500396508\\
8.27	0.00402248337105963\\
8.28	0.00402248173750996\\
8.29	0.00402248010331578\\
8.3	0.00402247846847681\\
8.31	0.00402247683299277\\
8.32	0.00402247519686338\\
8.33	0.00402247356008833\\
8.34	0.00402247192266736\\
8.35	0.00402247028460016\\
8.36	0.00402246864588647\\
8.37	0.00402246700652598\\
8.38	0.00402246536651842\\
8.39	0.00402246372586349\\
8.4	0.00402246208456092\\
8.41	0.00402246044261041\\
8.42	0.00402245880001167\\
8.43	0.00402245715676443\\
8.44	0.00402245551286839\\
8.45	0.00402245386832326\\
8.46	0.00402245222312875\\
8.47	0.00402245057728459\\
8.48	0.00402244893079048\\
8.49	0.00402244728364613\\
8.5	0.00402244563585125\\
8.51	0.00402244398740557\\
8.52	0.00402244233830878\\
8.53	0.0040224406885606\\
8.54	0.00402243903816074\\
8.55	0.00402243738710891\\
8.56	0.00402243573540482\\
8.57	0.00402243408304818\\
8.58	0.00402243243003871\\
8.59	0.00402243077637611\\
8.6	0.0040224291220601\\
8.61	0.00402242746709038\\
8.62	0.00402242581146666\\
8.63	0.00402242415518866\\
8.64	0.00402242249825608\\
8.65	0.00402242084066863\\
8.66	0.00402241918242603\\
8.67	0.00402241752352797\\
8.68	0.00402241586397418\\
8.69	0.00402241420376435\\
8.7	0.00402241254289821\\
8.71	0.00402241088137545\\
8.72	0.00402240921919578\\
8.73	0.00402240755635892\\
8.74	0.00402240589286457\\
8.75	0.00402240422871243\\
8.76	0.00402240256390223\\
8.77	0.00402240089843365\\
8.78	0.00402239923230643\\
8.79	0.00402239756552025\\
8.8	0.00402239589807482\\
8.81	0.00402239422996987\\
8.82	0.00402239256120508\\
8.83	0.00402239089178017\\
8.84	0.00402238922169484\\
8.85	0.0040223875509488\\
8.86	0.00402238587954176\\
8.87	0.00402238420747342\\
8.88	0.00402238253474349\\
8.89	0.00402238086135168\\
8.9	0.00402237918729769\\
8.91	0.00402237751258122\\
8.92	0.00402237583720198\\
8.93	0.00402237416115968\\
8.94	0.00402237248445403\\
8.95	0.00402237080708471\\
8.96	0.00402236912905145\\
8.97	0.00402236745035395\\
8.98	0.0040223657709919\\
8.99	0.00402236409096503\\
9	0.00402236241027302\\
9.01	0.00402236072891558\\
9.02	0.00402235904689242\\
9.03	0.00402235736420324\\
9.04	0.00402235568084775\\
9.05	0.00402235399682565\\
9.06	0.00402235231213663\\
9.07	0.00402235062678041\\
9.08	0.00402234894075668\\
9.09	0.00402234725406516\\
9.1	0.00402234556670554\\
9.11	0.00402234387867752\\
9.12	0.00402234218998081\\
9.13	0.00402234050061511\\
9.14	0.00402233881058013\\
9.15	0.00402233711987556\\
9.16	0.0040223354285011\\
9.17	0.00402233373645647\\
9.18	0.00402233204374135\\
9.19	0.00402233035035545\\
9.2	0.00402232865629847\\
9.21	0.00402232696157011\\
9.22	0.00402232526617009\\
9.23	0.00402232357009808\\
9.24	0.0040223218733538\\
9.25	0.00402232017593694\\
9.26	0.00402231847784721\\
9.27	0.00402231677908431\\
9.28	0.00402231507964793\\
9.29	0.00402231337953778\\
9.3	0.00402231167875355\\
9.31	0.00402230997729495\\
9.32	0.00402230827516168\\
9.33	0.00402230657235342\\
9.34	0.0040223048688699\\
9.35	0.00402230316471079\\
9.36	0.00402230145987581\\
9.37	0.00402229975436465\\
9.38	0.004022298048177\\
9.39	0.00402229634131257\\
9.4	0.00402229463377106\\
9.41	0.00402229292555216\\
9.42	0.00402229121665557\\
9.43	0.00402228950708099\\
9.44	0.00402228779682812\\
9.45	0.00402228608589665\\
9.46	0.00402228437428629\\
9.47	0.00402228266199673\\
9.48	0.00402228094902766\\
9.49	0.00402227923537878\\
9.5	0.0040222775210498\\
9.51	0.00402227580604041\\
9.52	0.0040222740903503\\
9.53	0.00402227237397917\\
9.54	0.00402227065692672\\
9.55	0.00402226893919264\\
9.56	0.00402226722077663\\
9.57	0.00402226550167838\\
9.58	0.0040222637818976\\
9.59	0.00402226206143397\\
9.6	0.0040222603402872\\
9.61	0.00402225861845697\\
9.62	0.00402225689594299\\
9.63	0.00402225517274494\\
9.64	0.00402225344886253\\
9.65	0.00402225172429544\\
9.66	0.00402224999904338\\
9.67	0.00402224827310604\\
9.68	0.00402224654648311\\
9.69	0.00402224481917428\\
9.7	0.00402224309117926\\
9.71	0.00402224136249772\\
9.72	0.00402223963312937\\
9.73	0.00402223790307391\\
9.74	0.00402223617233103\\
9.75	0.00402223444090041\\
9.76	0.00402223270878175\\
9.77	0.00402223097597475\\
9.78	0.0040222292424791\\
9.79	0.0040222275082945\\
9.8	0.00402222577342062\\
9.81	0.00402222403785718\\
9.82	0.00402222230160385\\
9.83	0.00402222056466034\\
9.84	0.00402221882702633\\
9.85	0.00402221708870152\\
9.86	0.0040222153496856\\
9.87	0.00402221360997827\\
9.88	0.00402221186957921\\
9.89	0.00402221012848811\\
9.9	0.00402220838670467\\
9.91	0.00402220664422857\\
9.92	0.00402220490105952\\
9.93	0.0040222031571972\\
9.94	0.00402220141264131\\
9.95	0.00402219966739153\\
9.96	0.00402219792144755\\
9.97	0.00402219617480907\\
9.98	0.00402219442747577\\
9.99	0.00402219267944735\\
10	0.0040221909307235\\
10.01	0.00402218918130391\\
10.02	0.00402218743118826\\
10.03	0.00402218568037626\\
10.04	0.00402218392886758\\
10.05	0.00402218217666193\\
10.06	0.00402218042375898\\
10.07	0.00402217867015843\\
10.08	0.00402217691585997\\
10.09	0.00402217516086328\\
10.1	0.00402217340516807\\
10.11	0.004022171648774\\
10.12	0.00402216989168079\\
10.13	0.00402216813388811\\
10.14	0.00402216637539566\\
10.15	0.00402216461620311\\
10.16	0.00402216285631017\\
10.17	0.00402216109571653\\
10.18	0.00402215933442186\\
10.19	0.00402215757242586\\
10.2	0.00402215580972822\\
10.21	0.00402215404632862\\
10.22	0.00402215228222676\\
10.23	0.00402215051742231\\
10.24	0.00402214875191498\\
10.25	0.00402214698570444\\
10.26	0.00402214521879039\\
10.27	0.00402214345117252\\
10.28	0.00402214168285051\\
10.29	0.00402213991382405\\
10.3	0.00402213814409282\\
10.31	0.00402213637365652\\
10.32	0.00402213460251483\\
10.33	0.00402213283066744\\
10.34	0.00402213105811404\\
10.35	0.00402212928485432\\
10.36	0.00402212751088796\\
10.37	0.00402212573621464\\
10.38	0.00402212396083407\\
10.39	0.00402212218474592\\
10.4	0.00402212040794987\\
10.41	0.00402211863044563\\
10.42	0.00402211685223287\\
10.43	0.00402211507331128\\
10.44	0.00402211329368055\\
10.45	0.00402211151334037\\
10.46	0.00402210973229042\\
10.47	0.00402210795053038\\
10.48	0.00402210616805996\\
10.49	0.00402210438487882\\
10.5	0.00402210260098666\\
10.51	0.00402210081638316\\
10.52	0.00402209903106802\\
10.53	0.00402209724504092\\
10.54	0.00402209545830153\\
10.55	0.00402209367084957\\
10.56	0.00402209188268469\\
10.57	0.0040220900938066\\
10.58	0.00402208830421498\\
10.59	0.00402208651390951\\
10.6	0.00402208472288989\\
10.61	0.00402208293115579\\
10.62	0.00402208113870691\\
10.63	0.00402207934554293\\
10.64	0.00402207755166354\\
10.65	0.00402207575706842\\
10.66	0.00402207396175726\\
10.67	0.00402207216572974\\
10.68	0.00402207036898556\\
10.69	0.00402206857152439\\
10.7	0.00402206677334593\\
10.71	0.00402206497444986\\
10.72	0.00402206317483586\\
10.73	0.00402206137450363\\
10.74	0.00402205957345285\\
10.75	0.0040220577716832\\
10.76	0.00402205596919438\\
10.77	0.00402205416598606\\
10.78	0.00402205236205793\\
10.79	0.00402205055740969\\
10.8	0.00402204875204102\\
10.81	0.0040220469459516\\
10.82	0.00402204513914112\\
10.83	0.00402204333160926\\
10.84	0.00402204152335572\\
10.85	0.00402203971438018\\
10.86	0.00402203790468233\\
10.87	0.00402203609426185\\
10.88	0.00402203428311843\\
10.89	0.00402203247125176\\
10.9	0.00402203065866153\\
10.91	0.00402202884534742\\
10.92	0.00402202703130911\\
10.93	0.00402202521654631\\
10.94	0.00402202340105868\\
10.95	0.00402202158484593\\
10.96	0.00402201976790774\\
10.97	0.00402201795024379\\
10.98	0.00402201613185378\\
10.99	0.00402201431273739\\
11	0.00402201249289431\\
11.01	0.00402201067232423\\
11.02	0.00402200885102684\\
11.03	0.00402200702900182\\
11.04	0.00402200520624886\\
11.05	0.00402200338276766\\
11.06	0.0040220015585579\\
11.07	0.00402199973361927\\
11.08	0.00402199790795145\\
11.09	0.00402199608155415\\
11.1	0.00402199425442704\\
11.11	0.00402199242656982\\
11.12	0.00402199059798218\\
11.13	0.0040219887686638\\
11.14	0.00402198693861438\\
11.15	0.00402198510783361\\
11.16	0.00402198327632118\\
11.17	0.00402198144407677\\
11.18	0.00402197961110009\\
11.19	0.00402197777739081\\
11.2	0.00402197594294865\\
11.21	0.00402197410777327\\
11.22	0.00402197227186438\\
11.23	0.00402197043522166\\
11.24	0.00402196859784482\\
11.25	0.00402196675973353\\
11.26	0.00402196492088751\\
11.27	0.00402196308130643\\
11.28	0.00402196124098999\\
11.29	0.00402195939993789\\
11.3	0.00402195755814982\\
11.31	0.00402195571562547\\
11.32	0.00402195387236453\\
11.33	0.00402195202836672\\
11.34	0.00402195018363171\\
11.35	0.0040219483381592\\
11.36	0.00402194649194889\\
11.37	0.00402194464500047\\
11.38	0.00402194279731365\\
11.39	0.00402194094888811\\
11.4	0.00402193909972356\\
11.41	0.00402193724981969\\
11.42	0.0040219353991762\\
11.43	0.00402193354779279\\
11.44	0.00402193169566916\\
11.45	0.00402192984280501\\
11.46	0.00402192798920003\\
11.47	0.00402192613485392\\
11.48	0.00402192427976639\\
11.49	0.00402192242393713\\
11.5	0.00402192056736586\\
11.51	0.00402191871005226\\
11.52	0.00402191685199604\\
11.53	0.00402191499319691\\
11.54	0.00402191313365457\\
11.55	0.00402191127336871\\
11.56	0.00402190941233904\\
11.57	0.00402190755056528\\
11.58	0.00402190568804711\\
11.59	0.00402190382478426\\
11.6	0.00402190196077642\\
11.61	0.00402190009602329\\
11.62	0.0040218982305246\\
11.63	0.00402189636428004\\
11.64	0.00402189449728933\\
11.65	0.00402189262955217\\
11.66	0.00402189076106827\\
11.67	0.00402188889183734\\
11.68	0.00402188702185909\\
11.69	0.00402188515113324\\
11.7	0.00402188327965949\\
11.71	0.00402188140743756\\
11.72	0.00402187953446716\\
11.73	0.00402187766074801\\
11.74	0.00402187578627981\\
11.75	0.00402187391106229\\
11.76	0.00402187203509516\\
11.77	0.00402187015837813\\
11.78	0.00402186828091093\\
11.79	0.00402186640269328\\
11.8	0.00402186452372488\\
11.81	0.00402186264400546\\
11.82	0.00402186076353475\\
11.83	0.00402185888231246\\
11.84	0.00402185700033831\\
11.85	0.00402185511761203\\
11.86	0.00402185323413333\\
11.87	0.00402185134990195\\
11.88	0.00402184946491762\\
11.89	0.00402184757918004\\
11.9	0.00402184569268897\\
11.91	0.00402184380544411\\
11.92	0.0040218419174452\\
11.93	0.00402184002869197\\
11.94	0.00402183813918415\\
11.95	0.00402183624892148\\
11.96	0.00402183435790368\\
11.97	0.00402183246613049\\
11.98	0.00402183057360164\\
11.99	0.00402182868031687\\
12	0.00402182678627591\\
12.01	0.00402182489147851\\
12.02	0.0040218229959244\\
12.03	0.00402182109961332\\
12.04	0.00402181920254501\\
12.05	0.00402181730471921\\
12.06	0.00402181540613568\\
12.07	0.00402181350679414\\
12.08	0.00402181160669435\\
12.09	0.00402180970583606\\
12.1	0.004021807804219\\
12.11	0.00402180590184294\\
12.12	0.00402180399870762\\
12.13	0.00402180209481278\\
12.14	0.00402180019015819\\
12.15	0.0040217982847436\\
12.16	0.00402179637856876\\
12.17	0.00402179447163343\\
12.18	0.00402179256393737\\
12.19	0.00402179065548033\\
12.2	0.00402178874626207\\
12.21	0.00402178683628237\\
12.22	0.00402178492554097\\
12.23	0.00402178301403766\\
12.24	0.00402178110177218\\
12.25	0.00402177918874431\\
12.26	0.00402177727495382\\
12.27	0.00402177536040047\\
12.28	0.00402177344508404\\
12.29	0.00402177152900431\\
12.3	0.00402176961216104\\
12.31	0.00402176769455402\\
12.32	0.00402176577618301\\
12.33	0.00402176385704781\\
12.34	0.00402176193714819\\
12.35	0.00402176001648393\\
12.36	0.00402175809505482\\
12.37	0.00402175617286064\\
12.38	0.00402175424990118\\
12.39	0.00402175232617623\\
12.4	0.00402175040168558\\
12.41	0.00402174847642903\\
12.42	0.00402174655040636\\
12.43	0.00402174462361738\\
12.44	0.00402174269606187\\
12.45	0.00402174076773965\\
12.46	0.0040217388386505\\
12.47	0.00402173690879425\\
12.48	0.00402173497817068\\
12.49	0.0040217330467796\\
12.5	0.00402173111462083\\
12.51	0.00402172918169418\\
12.52	0.00402172724799945\\
12.53	0.00402172531353646\\
12.54	0.00402172337830503\\
12.55	0.00402172144230498\\
12.56	0.00402171950553612\\
12.57	0.00402171756799828\\
12.58	0.00402171562969128\\
12.59	0.00402171369061494\\
12.6	0.00402171175076911\\
12.61	0.00402170981015359\\
12.62	0.00402170786876823\\
12.63	0.00402170592661287\\
12.64	0.00402170398368733\\
12.65	0.00402170203999146\\
12.66	0.00402170009552509\\
12.67	0.00402169815028808\\
12.68	0.00402169620428025\\
12.69	0.00402169425750147\\
12.7	0.00402169230995158\\
12.71	0.00402169036163042\\
12.72	0.00402168841253786\\
12.73	0.00402168646267374\\
12.74	0.00402168451203794\\
12.75	0.0040216825606303\\
12.76	0.00402168060845069\\
12.77	0.00402167865549897\\
12.78	0.004021676701775\\
12.79	0.00402167474727867\\
12.8	0.00402167279200984\\
12.81	0.00402167083596838\\
12.82	0.00402166887915417\\
12.83	0.00402166692156708\\
12.84	0.004021664963207\\
12.85	0.00402166300407381\\
12.86	0.00402166104416738\\
12.87	0.00402165908348762\\
12.88	0.0040216571220344\\
12.89	0.00402165515980763\\
12.9	0.00402165319680719\\
12.91	0.00402165123303298\\
12.92	0.0040216492684849\\
12.93	0.00402164730316285\\
12.94	0.00402164533706673\\
12.95	0.00402164337019645\\
12.96	0.00402164140255191\\
12.97	0.00402163943413303\\
12.98	0.00402163746493972\\
12.99	0.00402163549497189\\
13	0.00402163352422946\\
13.01	0.00402163155271235\\
13.02	0.00402162958042047\\
13.03	0.00402162760735376\\
13.04	0.00402162563351213\\
13.05	0.00402162365889552\\
13.06	0.00402162168350386\\
13.07	0.00402161970733706\\
13.08	0.00402161773039509\\
13.09	0.00402161575267786\\
13.1	0.00402161377418532\\
13.11	0.00402161179491741\\
13.12	0.00402160981487406\\
13.13	0.00402160783405523\\
13.14	0.00402160585246087\\
13.15	0.00402160387009091\\
13.16	0.00402160188694532\\
13.17	0.00402159990302403\\
13.18	0.00402159791832702\\
13.19	0.00402159593285423\\
13.2	0.00402159394660562\\
13.21	0.00402159195958116\\
13.22	0.0040215899717808\\
13.23	0.00402158798320451\\
13.24	0.00402158599385226\\
13.25	0.004021584003724\\
13.26	0.00402158201281972\\
13.27	0.00402158002113937\\
13.28	0.00402157802868292\\
13.29	0.00402157603545037\\
13.3	0.00402157404144167\\
13.31	0.0040215720466568\\
13.32	0.00402157005109574\\
13.33	0.00402156805475846\\
13.34	0.00402156605764495\\
13.35	0.00402156405975518\\
13.36	0.00402156206108914\\
13.37	0.00402156006164681\\
13.38	0.00402155806142816\\
13.39	0.0040215560604332\\
13.4	0.00402155405866188\\
13.41	0.00402155205611421\\
13.42	0.00402155005279017\\
13.43	0.00402154804868973\\
13.44	0.0040215460438129\\
13.45	0.00402154403815964\\
13.46	0.00402154203172996\\
13.47	0.00402154002452383\\
13.48	0.00402153801654125\\
13.49	0.00402153600778219\\
13.5	0.00402153399824664\\
13.51	0.0040215319879346\\
13.52	0.00402152997684603\\
13.53	0.00402152796498093\\
13.54	0.00402152595233928\\
13.55	0.00402152393892106\\
13.56	0.00402152192472626\\
13.57	0.00402151990975485\\
13.58	0.0040215178940068\\
13.59	0.00402151587748211\\
13.6	0.00402151386018074\\
13.61	0.00402151184210267\\
13.62	0.00402150982324786\\
13.63	0.0040215078036163\\
13.64	0.00402150578320795\\
13.65	0.00402150376202277\\
13.66	0.00402150174006072\\
13.67	0.00402149971732178\\
13.68	0.00402149769380588\\
13.69	0.004021495669513\\
13.7	0.00402149364444308\\
13.71	0.00402149161859606\\
13.72	0.0040214895919719\\
13.73	0.00402148756457052\\
13.74	0.00402148553639188\\
13.75	0.0040214835074359\\
13.76	0.00402148147770251\\
13.77	0.00402147944719163\\
13.78	0.00402147741590319\\
13.79	0.00402147538383709\\
13.8	0.00402147335099324\\
13.81	0.00402147131737156\\
13.82	0.00402146928297193\\
13.83	0.00402146724779425\\
13.84	0.0040214652118384\\
13.85	0.00402146317510427\\
13.86	0.00402146113759172\\
13.87	0.00402145909930063\\
13.88	0.00402145706023085\\
13.89	0.00402145502038223\\
13.9	0.00402145297975463\\
13.91	0.00402145093834787\\
13.92	0.00402144889616179\\
13.93	0.0040214468531962\\
13.94	0.00402144480945093\\
13.95	0.00402144276492576\\
13.96	0.00402144071962051\\
13.97	0.00402143867353494\\
13.98	0.00402143662666885\\
13.99	0.00402143457902199\\
14	0.00402143253059413\\
14.01	0.004021430481385\\
14.02	0.00402142843139435\\
14.03	0.00402142638062189\\
14.04	0.00402142432906735\\
14.05	0.00402142227673042\\
14.06	0.00402142022361079\\
14.07	0.00402141816970814\\
14.08	0.00402141611502215\\
14.09	0.00402141405955246\\
14.1	0.00402141200329871\\
14.11	0.00402140994626054\\
14.12	0.00402140788843756\\
14.13	0.00402140582982937\\
14.14	0.00402140377043556\\
14.15	0.0040214017102557\\
14.16	0.00402139964928935\\
14.17	0.00402139758753606\\
14.18	0.00402139552499536\\
14.19	0.00402139346166676\\
14.2	0.00402139139754977\\
14.21	0.00402138933264387\\
14.22	0.00402138726694854\\
14.23	0.00402138520046321\\
14.24	0.00402138313318734\\
14.25	0.00402138106512034\\
14.26	0.00402137899626161\\
14.27	0.00402137692661056\\
14.28	0.00402137485616653\\
14.29	0.0040213727849289\\
14.3	0.004021370712897\\
14.31	0.00402136864007014\\
14.32	0.00402136656644764\\
14.33	0.00402136449202877\\
14.34	0.00402136241681279\\
14.35	0.00402136034079898\\
14.36	0.00402135826398654\\
14.37	0.0040213561863747\\
14.38	0.00402135410796264\\
14.39	0.00402135202874956\\
14.4	0.0040213499487346\\
14.41	0.0040213478679169\\
14.42	0.0040213457862956\\
14.43	0.0040213437038698\\
14.44	0.00402134162063857\\
14.45	0.00402133953660099\\
14.46	0.00402133745175612\\
14.47	0.00402133536610298\\
14.48	0.00402133327964058\\
14.49	0.00402133119236794\\
14.5	0.00402132910428402\\
14.51	0.0040213270153878\\
14.52	0.00402132492567821\\
14.53	0.00402132283515419\\
14.54	0.00402132074381466\\
14.55	0.0040213186516585\\
14.56	0.00402131655868461\\
14.57	0.00402131446489186\\
14.58	0.00402131237027908\\
14.59	0.00402131027484513\\
14.6	0.00402130817858883\\
14.61	0.00402130608150898\\
14.62	0.00402130398360438\\
14.63	0.00402130188487383\\
14.64	0.00402129978531609\\
14.65	0.00402129768492994\\
14.66	0.0040212955837141\\
14.67	0.00402129348166735\\
14.68	0.0040212913787884\\
14.69	0.00402128927507599\\
14.7	0.00402128717052883\\
14.71	0.00402128506514563\\
14.72	0.00402128295892511\\
14.73	0.00402128085186597\\
14.74	0.0040212787439669\\
14.75	0.0040212766352266\\
14.76	0.00402127452564376\\
14.77	0.00402127241521709\\
14.78	0.00402127030394527\\
14.79	0.004021268191827\\
14.8	0.00402126607886098\\
14.81	0.00402126396504592\\
14.82	0.00402126185038051\\
14.83	0.00402125973486349\\
14.84	0.00402125761849358\\
14.85	0.0040212555012695\\
14.86	0.00402125338318999\\
14.87	0.00402125126425383\\
14.88	0.00402124914445976\\
14.89	0.00402124702380659\\
14.9	0.00402124490229309\\
14.91	0.0040212427799181\\
14.92	0.00402124065668044\\
14.93	0.00402123853257897\\
14.94	0.00402123640761256\\
14.95	0.00402123428178012\\
14.96	0.00402123215508056\\
14.97	0.00402123002751284\\
14.98	0.00402122789907591\\
14.99	0.0040212257697688\\
15	0.00402122363959052\\
15.01	0.00402122150854012\\
15.02	0.0040212193766167\\
15.03	0.00402121724381938\\
15.04	0.00402121511014731\\
15.05	0.00402121297559966\\
15.06	0.00402121084017566\\
15.07	0.00402120870387454\\
15.08	0.00402120656669559\\
15.09	0.00402120442863812\\
15.1	0.00402120228970147\\
15.11	0.00402120014988502\\
15.12	0.00402119800918818\\
15.13	0.00402119586761037\\
15.14	0.00402119372515105\\
15.15	0.00402119158180973\\
15.16	0.00402118943758591\\
15.17	0.00402118729247912\\
15.18	0.00402118514648892\\
15.19	0.00402118299961486\\
15.2	0.00402118085185652\\
15.21	0.00402117870321349\\
15.22	0.00402117655368533\\
15.23	0.00402117440327165\\
15.24	0.00402117225197201\\
15.25	0.00402117009978599\\
15.26	0.00402116794671318\\
15.27	0.00402116579275315\\
15.28	0.00402116363790548\\
15.29	0.00402116148216976\\
15.3	0.00402115932554556\\
15.31	0.00402115716803246\\
15.32	0.00402115500963003\\
15.33	0.00402115285033786\\
15.34	0.00402115069015552\\
15.35	0.00402114852908259\\
15.36	0.00402114636711865\\
15.37	0.00402114420426328\\
15.38	0.00402114204051605\\
15.39	0.00402113987587654\\
15.4	0.00402113771034433\\
15.41	0.00402113554391898\\
15.42	0.00402113337660009\\
15.43	0.00402113120838722\\
15.44	0.00402112903927995\\
15.45	0.00402112686927785\\
15.46	0.00402112469838049\\
15.47	0.00402112252658746\\
15.48	0.00402112035389834\\
15.49	0.00402111818031268\\
15.5	0.00402111600583007\\
15.51	0.00402111383045008\\
15.52	0.00402111165417228\\
15.53	0.00402110947699625\\
15.54	0.00402110729892156\\
15.55	0.00402110511994778\\
15.56	0.00402110294007449\\
15.57	0.00402110075930125\\
15.58	0.00402109857762764\\
15.59	0.00402109639505323\\
15.6	0.0040210942115776\\
15.61	0.00402109202720031\\
15.62	0.00402108984192094\\
15.63	0.00402108765573905\\
15.64	0.00402108546865422\\
15.65	0.00402108328066601\\
15.66	0.004021081091774\\
15.67	0.00402107890197776\\
15.68	0.00402107671127686\\
15.69	0.00402107451967085\\
15.7	0.00402107232715933\\
15.71	0.00402107013374185\\
15.72	0.00402106793941798\\
15.73	0.00402106574418729\\
15.74	0.00402106354804935\\
15.75	0.00402106135100373\\
15.76	0.00402105915304999\\
15.77	0.0040210569541877\\
15.78	0.00402105475441644\\
15.79	0.00402105255373575\\
15.8	0.00402105035214522\\
15.81	0.00402104814964441\\
15.82	0.00402104594623288\\
15.83	0.00402104374191021\\
15.84	0.00402104153667595\\
15.85	0.00402103933052968\\
15.86	0.00402103712347095\\
15.87	0.00402103491549933\\
15.88	0.00402103270661439\\
15.89	0.00402103049681569\\
15.9	0.0040210282861028\\
15.91	0.00402102607447527\\
15.92	0.00402102386193268\\
15.93	0.00402102164847459\\
15.94	0.00402101943410055\\
15.95	0.00402101721881013\\
15.96	0.00402101500260291\\
15.97	0.00402101278547843\\
15.98	0.00402101056743626\\
15.99	0.00402100834847596\\
16	0.0040210061285971\\
16.01	0.00402100390779924\\
16.02	0.00402100168608192\\
16.03	0.00402099946344473\\
16.04	0.00402099723988721\\
16.05	0.00402099501540893\\
16.06	0.00402099279000945\\
16.07	0.00402099056368833\\
16.08	0.00402098833644513\\
16.09	0.0040209861082794\\
16.1	0.00402098387919071\\
16.11	0.00402098164917862\\
16.12	0.00402097941824269\\
16.13	0.00402097718638246\\
16.14	0.00402097495359751\\
16.15	0.00402097271988738\\
16.16	0.00402097048525164\\
16.17	0.00402096824968985\\
16.18	0.00402096601320156\\
16.19	0.00402096377578633\\
16.2	0.00402096153744371\\
16.21	0.00402095929817326\\
16.22	0.00402095705797455\\
16.23	0.00402095481684712\\
16.24	0.00402095257479052\\
16.25	0.00402095033180432\\
16.26	0.00402094808788807\\
16.27	0.00402094584304133\\
16.28	0.00402094359726364\\
16.29	0.00402094135055457\\
16.3	0.00402093910291367\\
16.31	0.00402093685434049\\
16.32	0.00402093460483459\\
16.33	0.00402093235439551\\
16.34	0.00402093010302282\\
16.35	0.00402092785071606\\
16.36	0.00402092559747479\\
16.37	0.00402092334329856\\
16.38	0.00402092108818693\\
16.39	0.00402091883213943\\
16.4	0.00402091657515564\\
16.41	0.00402091431723509\\
16.42	0.00402091205837734\\
16.43	0.00402090979858194\\
16.44	0.00402090753784844\\
16.45	0.00402090527617639\\
16.46	0.00402090301356534\\
16.47	0.00402090075001484\\
16.48	0.00402089848552444\\
16.49	0.00402089622009369\\
16.5	0.00402089395372214\\
16.51	0.00402089168640934\\
16.52	0.00402088941815483\\
16.53	0.00402088714895816\\
16.54	0.00402088487881889\\
16.55	0.00402088260773656\\
16.56	0.00402088033571071\\
16.57	0.0040208780627409\\
16.58	0.00402087578882667\\
16.59	0.00402087351396757\\
16.6	0.00402087123816314\\
16.61	0.00402086896141294\\
16.62	0.0040208666837165\\
16.63	0.00402086440507338\\
16.64	0.00402086212548311\\
16.65	0.00402085984494525\\
16.66	0.00402085756345933\\
16.67	0.00402085528102491\\
16.68	0.00402085299764153\\
16.69	0.00402085071330873\\
16.7	0.00402084842802605\\
16.71	0.00402084614179305\\
16.72	0.00402084385460926\\
16.73	0.00402084156647422\\
16.74	0.00402083927738748\\
16.75	0.00402083698734858\\
16.76	0.00402083469635707\\
16.77	0.00402083240441249\\
16.78	0.00402083011151437\\
16.79	0.00402082781766226\\
16.8	0.0040208255228557\\
16.81	0.00402082322709424\\
16.82	0.0040208209303774\\
16.83	0.00402081863270474\\
16.84	0.0040208163340758\\
16.85	0.00402081403449011\\
16.86	0.00402081173394721\\
16.87	0.00402080943244664\\
16.88	0.00402080712998795\\
16.89	0.00402080482657066\\
16.9	0.00402080252219433\\
16.91	0.00402080021685848\\
16.92	0.00402079791056267\\
16.93	0.00402079560330642\\
16.94	0.00402079329508926\\
16.95	0.00402079098591075\\
16.96	0.00402078867577041\\
16.97	0.00402078636466779\\
16.98	0.00402078405260242\\
16.99	0.00402078173957384\\
17	0.00402077942558157\\
17.01	0.00402077711062517\\
17.02	0.00402077479470416\\
17.03	0.00402077247781808\\
17.04	0.00402077015996646\\
17.05	0.00402076784114884\\
17.06	0.00402076552136476\\
17.07	0.00402076320061375\\
17.08	0.00402076087889533\\
17.09	0.00402075855620906\\
17.1	0.00402075623255445\\
17.11	0.00402075390793104\\
17.12	0.00402075158233837\\
17.13	0.00402074925577597\\
17.14	0.00402074692824337\\
17.15	0.0040207445997401\\
17.16	0.00402074227026569\\
17.17	0.00402073993981968\\
17.18	0.0040207376084016\\
17.19	0.00402073527601098\\
17.2	0.00402073294264734\\
17.21	0.00402073060831023\\
17.22	0.00402072827299917\\
17.23	0.00402072593671369\\
17.24	0.00402072359945331\\
17.25	0.00402072126121758\\
17.26	0.00402071892200601\\
17.27	0.00402071658181814\\
17.28	0.0040207142406535\\
17.29	0.00402071189851161\\
17.3	0.004020709555392\\
17.31	0.00402070721129421\\
17.32	0.00402070486621775\\
17.33	0.00402070252016216\\
17.34	0.00402070017312697\\
17.35	0.00402069782511169\\
17.36	0.00402069547611586\\
17.37	0.004020693126139\\
17.38	0.00402069077518064\\
17.39	0.0040206884232403\\
17.4	0.00402068607031751\\
17.41	0.0040206837164118\\
17.42	0.00402068136152268\\
17.43	0.00402067900564969\\
17.44	0.00402067664879235\\
17.45	0.00402067429095018\\
17.46	0.00402067193212271\\
17.47	0.00402066957230946\\
17.48	0.00402066721150996\\
17.49	0.00402066484972372\\
17.5	0.00402066248695027\\
17.51	0.00402066012318913\\
17.52	0.00402065775843983\\
17.53	0.00402065539270188\\
17.54	0.00402065302597481\\
17.55	0.00402065065825814\\
17.56	0.00402064828955139\\
17.57	0.00402064591985408\\
17.58	0.00402064354916574\\
17.59	0.00402064117748588\\
17.6	0.00402063880481402\\
17.61	0.00402063643114968\\
17.62	0.00402063405649238\\
17.63	0.00402063168084164\\
17.64	0.00402062930419699\\
17.65	0.00402062692655793\\
17.66	0.00402062454792399\\
17.67	0.00402062216829469\\
17.68	0.00402061978766954\\
17.69	0.00402061740604806\\
17.7	0.00402061502342977\\
17.71	0.00402061263981419\\
17.72	0.00402061025520082\\
17.73	0.0040206078695892\\
17.74	0.00402060548297883\\
17.75	0.00402060309536923\\
17.76	0.00402060070675992\\
17.77	0.00402059831715041\\
17.78	0.00402059592654022\\
17.79	0.00402059353492886\\
17.8	0.00402059114231585\\
17.81	0.0040205887487007\\
17.82	0.00402058635408292\\
17.83	0.00402058395846204\\
17.84	0.00402058156183755\\
17.85	0.00402057916420898\\
17.86	0.00402057676557585\\
17.87	0.00402057436593765\\
17.88	0.00402057196529391\\
17.89	0.00402056956364413\\
17.9	0.00402056716098783\\
17.91	0.00402056475732453\\
17.92	0.00402056235265372\\
17.93	0.00402055994697493\\
17.94	0.00402055754028766\\
17.95	0.00402055513259142\\
17.96	0.00402055272388572\\
17.97	0.00402055031417008\\
17.98	0.00402054790344401\\
17.99	0.004020545491707\\
18	0.00402054307895858\\
18.01	0.00402054066519825\\
18.02	0.00402053825042552\\
18.03	0.0040205358346399\\
18.04	0.00402053341784088\\
18.05	0.004020531000028\\
18.06	0.00402052858120075\\
18.07	0.00402052616135863\\
18.08	0.00402052374050117\\
18.09	0.00402052131862784\\
18.1	0.00402051889573819\\
18.11	0.00402051647183169\\
18.12	0.00402051404690787\\
18.13	0.00402051162096622\\
18.14	0.00402050919400625\\
18.15	0.00402050676602747\\
18.16	0.00402050433702937\\
18.17	0.00402050190701148\\
18.18	0.00402049947597327\\
18.19	0.00402049704391428\\
18.2	0.00402049461083399\\
18.21	0.00402049217673191\\
18.22	0.00402048974160755\\
18.23	0.00402048730546039\\
18.24	0.00402048486828996\\
18.25	0.00402048243009574\\
18.26	0.00402047999087725\\
18.27	0.00402047755063398\\
18.28	0.00402047510936544\\
18.29	0.00402047266707113\\
18.3	0.00402047022375054\\
18.31	0.00402046777940318\\
18.32	0.00402046533402855\\
18.33	0.00402046288762614\\
18.34	0.00402046044019546\\
18.35	0.00402045799173601\\
18.36	0.00402045554224728\\
18.37	0.00402045309172878\\
18.38	0.00402045064018\\
18.39	0.00402044818760045\\
18.4	0.00402044573398961\\
18.41	0.00402044327934699\\
18.42	0.00402044082367209\\
18.43	0.00402043836696439\\
18.44	0.00402043590922341\\
18.45	0.00402043345044862\\
18.46	0.00402043099063955\\
18.47	0.00402042852979567\\
18.48	0.00402042606791648\\
18.49	0.00402042360500148\\
18.5	0.00402042114105016\\
18.51	0.00402041867606203\\
18.52	0.00402041621003656\\
18.53	0.00402041374297326\\
18.54	0.00402041127487163\\
18.55	0.00402040880573115\\
18.56	0.00402040633555132\\
18.57	0.00402040386433163\\
18.58	0.00402040139207158\\
18.59	0.00402039891877065\\
18.6	0.00402039644442834\\
18.61	0.00402039396904415\\
18.62	0.00402039149261756\\
18.63	0.00402038901514806\\
18.64	0.00402038653663516\\
18.65	0.00402038405707832\\
18.66	0.00402038157647706\\
18.67	0.00402037909483085\\
18.68	0.00402037661213919\\
18.69	0.00402037412840157\\
18.7	0.00402037164361748\\
18.71	0.0040203691577864\\
18.72	0.00402036667090783\\
18.73	0.00402036418298126\\
18.74	0.00402036169400616\\
18.75	0.00402035920398204\\
18.76	0.00402035671290838\\
18.77	0.00402035422078466\\
18.78	0.00402035172761037\\
18.79	0.00402034923338501\\
18.8	0.00402034673810805\\
18.81	0.00402034424177898\\
18.82	0.0040203417443973\\
18.83	0.00402033924596247\\
18.84	0.00402033674647401\\
18.85	0.00402033424593137\\
18.86	0.00402033174433405\\
18.87	0.00402032924168154\\
18.88	0.00402032673797333\\
18.89	0.00402032423320888\\
18.9	0.00402032172738769\\
18.91	0.00402031922050924\\
18.92	0.00402031671257301\\
18.93	0.00402031420357849\\
18.94	0.00402031169352517\\
18.95	0.00402030918241251\\
18.96	0.00402030667024\\
18.97	0.00402030415700714\\
18.98	0.00402030164271339\\
18.99	0.00402029912735823\\
19	0.00402029661094116\\
19.01	0.00402029409346165\\
19.02	0.00402029157491917\\
19.03	0.00402028905531322\\
19.04	0.00402028653464326\\
19.05	0.00402028401290878\\
19.06	0.00402028149010927\\
19.07	0.00402027896624419\\
19.08	0.00402027644131302\\
19.09	0.00402027391531526\\
19.1	0.00402027138825035\\
19.11	0.00402026886011781\\
19.12	0.00402026633091709\\
19.13	0.00402026380064767\\
19.14	0.00402026126930903\\
19.15	0.00402025873690065\\
19.16	0.00402025620342201\\
19.17	0.00402025366887257\\
19.18	0.00402025113325182\\
19.19	0.00402024859655923\\
19.2	0.00402024605879428\\
19.21	0.00402024351995644\\
19.22	0.00402024098004518\\
19.23	0.00402023843905998\\
19.24	0.00402023589700032\\
19.25	0.00402023335386566\\
19.26	0.00402023080965549\\
19.27	0.00402022826436927\\
19.28	0.00402022571800647\\
19.29	0.00402022317056657\\
19.3	0.00402022062204904\\
19.31	0.00402021807245336\\
19.32	0.00402021552177899\\
19.33	0.0040202129700254\\
19.34	0.00402021041719207\\
19.35	0.00402020786327847\\
19.36	0.00402020530828407\\
19.37	0.00402020275220833\\
19.38	0.00402020019505073\\
19.39	0.00402019763681074\\
19.4	0.00402019507748783\\
19.41	0.00402019251708145\\
19.42	0.0040201899555911\\
19.43	0.00402018739301623\\
19.44	0.0040201848293563\\
19.45	0.0040201822646108\\
19.46	0.00402017969877919\\
19.47	0.00402017713186092\\
19.48	0.00402017456385548\\
19.49	0.00402017199476232\\
19.5	0.00402016942458091\\
19.51	0.00402016685331073\\
19.52	0.00402016428095122\\
19.53	0.00402016170750187\\
19.54	0.00402015913296213\\
19.55	0.00402015655733148\\
19.56	0.00402015398060937\\
19.57	0.00402015140279526\\
19.58	0.00402014882388863\\
19.59	0.00402014624388893\\
19.6	0.00402014366279564\\
19.61	0.0040201410806082\\
19.62	0.0040201384973261\\
19.63	0.00402013591294878\\
19.64	0.00402013332747571\\
19.65	0.00402013074090635\\
19.66	0.00402012815324017\\
19.67	0.00402012556447662\\
19.68	0.00402012297461517\\
19.69	0.00402012038365527\\
19.7	0.00402011779159639\\
19.71	0.00402011519843799\\
19.72	0.00402011260417953\\
19.73	0.00402011000882046\\
19.74	0.00402010741236025\\
19.75	0.00402010481479835\\
19.76	0.00402010221613423\\
19.77	0.00402009961636734\\
19.78	0.00402009701549713\\
19.79	0.00402009441352308\\
19.8	0.00402009181044463\\
19.81	0.00402008920626124\\
19.82	0.00402008660097237\\
19.83	0.00402008399457747\\
19.84	0.00402008138707601\\
19.85	0.00402007877846743\\
19.86	0.0040200761687512\\
19.87	0.00402007355792676\\
19.88	0.00402007094599358\\
19.89	0.00402006833295111\\
19.9	0.00402006571879879\\
19.91	0.0040200631035361\\
19.92	0.00402006048716247\\
19.93	0.00402005786967738\\
19.94	0.00402005525108025\\
19.95	0.00402005263137056\\
19.96	0.00402005001054775\\
19.97	0.00402004738861128\\
19.98	0.00402004476556059\\
19.99	0.00402004214139515\\
20	0.0040200395161144\\
20.01	0.00402003688971778\\
20.02	0.00402003426220476\\
20.03	0.00402003163357478\\
20.04	0.00402002900382729\\
20.05	0.00402002637296175\\
20.06	0.0040200237409776\\
20.07	0.00402002110787429\\
20.08	0.00402001847365128\\
20.09	0.004020015838308\\
20.1	0.00402001320184392\\
20.11	0.00402001056425847\\
20.12	0.0040200079255511\\
20.13	0.00402000528572127\\
20.14	0.00402000264476842\\
20.15	0.00402000000269199\\
20.16	0.00401999735949144\\
20.17	0.0040199947151662\\
20.18	0.00401999206971573\\
20.19	0.00401998942313947\\
20.2	0.00401998677543687\\
20.21	0.00401998412660736\\
20.22	0.00401998147665041\\
20.23	0.00401997882556545\\
20.24	0.00401997617335192\\
20.25	0.00401997352000927\\
20.26	0.00401997086553694\\
20.27	0.00401996820993438\\
20.28	0.00401996555320103\\
20.29	0.00401996289533633\\
20.3	0.00401996023633973\\
20.31	0.00401995757621065\\
20.32	0.00401995491494856\\
20.33	0.00401995225255289\\
20.34	0.00401994958902308\\
20.35	0.00401994692435857\\
20.36	0.00401994425855879\\
20.37	0.00401994159162321\\
20.38	0.00401993892355124\\
20.39	0.00401993625434234\\
20.4	0.00401993358399594\\
20.41	0.00401993091251147\\
20.42	0.00401992823988839\\
20.43	0.00401992556612613\\
20.44	0.00401992289122412\\
20.45	0.0040199202151818\\
20.46	0.00401991753799862\\
20.47	0.004019914859674\\
20.48	0.00401991218020738\\
20.49	0.00401990949959822\\
20.5	0.00401990681784592\\
20.51	0.00401990413494994\\
20.52	0.00401990145090972\\
20.53	0.00401989876572468\\
20.54	0.00401989607939425\\
20.55	0.00401989339191788\\
20.56	0.004019890703295\\
20.57	0.00401988801352504\\
20.58	0.00401988532260744\\
20.59	0.00401988263054163\\
20.6	0.00401987993732705\\
20.61	0.00401987724296312\\
20.62	0.00401987454744928\\
20.63	0.00401987185078497\\
20.64	0.0040198691529696\\
20.65	0.00401986645400262\\
20.66	0.00401986375388346\\
20.67	0.00401986105261155\\
20.68	0.00401985835018631\\
20.69	0.00401985564660719\\
20.7	0.0040198529418736\\
20.71	0.00401985023598498\\
20.72	0.00401984752894076\\
20.73	0.00401984482074037\\
20.74	0.00401984211138323\\
20.75	0.00401983940086878\\
20.76	0.00401983668919645\\
20.77	0.00401983397636565\\
20.78	0.00401983126237582\\
20.79	0.00401982854722639\\
20.8	0.00401982583091679\\
20.81	0.00401982311344643\\
20.82	0.00401982039481475\\
20.83	0.00401981767502118\\
20.84	0.00401981495406513\\
20.85	0.00401981223194604\\
20.86	0.00401980950866332\\
20.87	0.00401980678421641\\
20.88	0.00401980405860473\\
20.89	0.00401980133182769\\
20.9	0.00401979860388474\\
20.91	0.00401979587477529\\
20.92	0.00401979314449876\\
20.93	0.00401979041305458\\
20.94	0.00401978768044216\\
20.95	0.00401978494666094\\
20.96	0.00401978221171033\\
20.97	0.00401977947558975\\
20.98	0.00401977673829864\\
20.99	0.0040197739998364\\
21	0.00401977126020246\\
21.01	0.00401976851939623\\
21.02	0.00401976577741715\\
21.03	0.00401976303426462\\
21.04	0.00401976028993808\\
21.05	0.00401975754443692\\
21.06	0.0040197547977606\\
21.07	0.0040197520499085\\
21.08	0.00401974930088005\\
21.09	0.00401974655067468\\
21.1	0.0040197437992918\\
21.11	0.00401974104673082\\
21.12	0.00401973829299116\\
21.13	0.00401973553807225\\
21.14	0.00401973278197349\\
21.15	0.0040197300246943\\
21.16	0.0040197272662341\\
21.17	0.0040197245065923\\
21.18	0.00401972174576833\\
21.19	0.00401971898376158\\
21.2	0.00401971622057148\\
21.21	0.00401971345619745\\
21.22	0.00401971069063889\\
21.23	0.00401970792389521\\
21.24	0.00401970515596585\\
21.25	0.00401970238685019\\
21.26	0.00401969961654767\\
21.27	0.00401969684505768\\
21.28	0.00401969407237964\\
21.29	0.00401969129851297\\
21.3	0.00401968852345707\\
21.31	0.00401968574721136\\
21.32	0.00401968296977524\\
21.33	0.00401968019114813\\
21.34	0.00401967741132943\\
21.35	0.00401967463031856\\
21.36	0.00401967184811492\\
21.37	0.00401966906471793\\
21.38	0.00401966628012699\\
21.39	0.00401966349434151\\
21.4	0.0040196607073609\\
21.41	0.00401965791918456\\
21.42	0.00401965512981191\\
21.43	0.00401965233924235\\
21.44	0.00401964954747528\\
21.45	0.00401964675451012\\
21.46	0.00401964396034627\\
21.47	0.00401964116498313\\
21.48	0.00401963836842012\\
21.49	0.00401963557065663\\
21.5	0.00401963277169207\\
21.51	0.00401962997152585\\
21.52	0.00401962717015736\\
21.53	0.00401962436758602\\
21.54	0.00401962156381122\\
21.55	0.00401961875883238\\
21.56	0.00401961595264888\\
21.57	0.00401961314526014\\
21.58	0.00401961033666556\\
21.59	0.00401960752686453\\
21.6	0.00401960471585646\\
21.61	0.00401960190364076\\
21.62	0.00401959909021681\\
21.63	0.00401959627558404\\
21.64	0.00401959345974181\\
21.65	0.00401959064268955\\
21.66	0.00401958782442666\\
21.67	0.00401958500495251\\
21.68	0.00401958218426654\\
21.69	0.00401957936236812\\
21.7	0.00401957653925665\\
21.71	0.00401957371493153\\
21.72	0.00401957088939217\\
21.73	0.00401956806263795\\
21.74	0.00401956523466828\\
21.75	0.00401956240548254\\
21.76	0.00401955957508015\\
21.77	0.00401955674346049\\
21.78	0.00401955391062295\\
21.79	0.00401955107656694\\
21.8	0.00401954824129184\\
21.81	0.00401954540479706\\
21.82	0.00401954256708198\\
21.83	0.004019539728146\\
21.84	0.00401953688798851\\
21.85	0.00401953404660891\\
21.86	0.00401953120400659\\
21.87	0.00401952836018094\\
21.88	0.00401952551513134\\
21.89	0.00401952266885721\\
21.9	0.00401951982135791\\
21.91	0.00401951697263286\\
21.92	0.00401951412268143\\
21.93	0.00401951127150301\\
21.94	0.00401950841909701\\
21.95	0.00401950556546279\\
21.96	0.00401950271059977\\
21.97	0.00401949985450731\\
21.98	0.00401949699718482\\
21.99	0.00401949413863168\\
22	0.00401949127884728\\
22.01	0.004019488417831\\
22.02	0.00401948555558223\\
22.03	0.00401948269210037\\
22.04	0.00401947982738479\\
22.05	0.00401947696143488\\
22.06	0.00401947409425002\\
22.07	0.00401947122582961\\
22.08	0.00401946835617302\\
22.09	0.00401946548527965\\
22.1	0.00401946261314887\\
22.11	0.00401945973978006\\
22.12	0.00401945686517262\\
22.13	0.00401945398932593\\
22.14	0.00401945111223936\\
22.15	0.00401944823391231\\
22.16	0.00401944535434414\\
22.17	0.00401944247353425\\
22.18	0.00401943959148201\\
22.19	0.00401943670818682\\
22.2	0.00401943382364803\\
22.21	0.00401943093786504\\
22.22	0.00401942805083723\\
22.23	0.00401942516256397\\
22.24	0.00401942227304465\\
22.25	0.00401941938227864\\
22.26	0.00401941649026532\\
22.27	0.00401941359700407\\
22.28	0.00401941070249426\\
22.29	0.00401940780673528\\
22.3	0.0040194049097265\\
22.31	0.00401940201146729\\
22.32	0.00401939911195704\\
22.33	0.00401939621119511\\
22.34	0.00401939330918089\\
22.35	0.00401939040591374\\
22.36	0.00401938750139304\\
22.37	0.00401938459561817\\
22.38	0.0040193816885885\\
22.39	0.00401937878030339\\
22.4	0.00401937587076224\\
22.41	0.0040193729599644\\
22.42	0.00401937004790925\\
22.43	0.00401936713459616\\
22.44	0.00401936422002451\\
22.45	0.00401936130419366\\
22.46	0.00401935838710298\\
22.47	0.00401935546875184\\
22.48	0.00401935254913961\\
22.49	0.00401934962826567\\
22.5	0.00401934670612938\\
22.51	0.0040193437827301\\
22.52	0.00401934085806721\\
22.53	0.00401933793214008\\
22.54	0.00401933500494807\\
22.55	0.00401933207649054\\
22.56	0.00401932914676687\\
22.57	0.00401932621577642\\
22.58	0.00401932328351856\\
22.59	0.00401932034999264\\
22.6	0.00401931741519804\\
22.61	0.00401931447913412\\
22.62	0.00401931154180024\\
22.63	0.00401930860319577\\
22.64	0.00401930566332006\\
22.65	0.00401930272217249\\
22.66	0.00401929977975241\\
22.67	0.00401929683605918\\
22.68	0.00401929389109217\\
22.69	0.00401929094485074\\
22.7	0.00401928799733424\\
22.71	0.00401928504854204\\
22.72	0.00401928209847349\\
22.73	0.00401927914712796\\
22.74	0.0040192761945048\\
22.75	0.00401927324060337\\
22.76	0.00401927028542303\\
22.77	0.00401926732896313\\
22.78	0.00401926437122304\\
22.79	0.0040192614122021\\
22.8	0.00401925845189968\\
22.81	0.00401925549031512\\
22.82	0.00401925252744778\\
22.83	0.00401924956329703\\
22.84	0.0040192465978622\\
22.85	0.00401924363114266\\
22.86	0.00401924066313775\\
22.87	0.00401923769384683\\
22.88	0.00401923472326926\\
22.89	0.00401923175140437\\
22.9	0.00401922877825153\\
22.91	0.00401922580381008\\
22.92	0.00401922282807938\\
22.93	0.00401921985105876\\
22.94	0.00401921687274759\\
22.95	0.0040192138931452\\
22.96	0.00401921091225095\\
22.97	0.00401920793006419\\
22.98	0.00401920494658425\\
22.99	0.00401920196181049\\
23	0.00401919897574225\\
23.01	0.00401919598837888\\
23.02	0.00401919299971972\\
23.03	0.00401919000976412\\
23.04	0.00401918701851141\\
23.05	0.00401918402596095\\
23.06	0.00401918103211208\\
23.07	0.00401917803696412\\
23.08	0.00401917504051644\\
23.09	0.00401917204276836\\
23.1	0.00401916904371924\\
23.11	0.0040191660433684\\
23.12	0.00401916304171518\\
23.13	0.00401916003875894\\
23.14	0.004019157034499\\
23.15	0.00401915402893469\\
23.16	0.00401915102206537\\
23.17	0.00401914801389035\\
23.18	0.00401914500440899\\
23.19	0.00401914199362061\\
23.2	0.00401913898152455\\
23.21	0.00401913596812014\\
23.22	0.00401913295340671\\
23.23	0.0040191299373836\\
23.24	0.00401912692005015\\
23.25	0.00401912390140566\\
23.26	0.00401912088144949\\
23.27	0.00401911786018096\\
23.28	0.00401911483759941\\
23.29	0.00401911181370414\\
23.3	0.00401910878849451\\
23.31	0.00401910576196982\\
23.32	0.00401910273412943\\
23.33	0.00401909970497263\\
23.34	0.00401909667449876\\
23.35	0.00401909364270716\\
23.36	0.00401909060959713\\
23.37	0.004019087575168\\
23.38	0.0040190845394191\\
23.39	0.00401908150234975\\
23.4	0.00401907846395926\\
23.41	0.00401907542424696\\
23.42	0.00401907238321218\\
23.43	0.00401906934085421\\
23.44	0.0040190662971724\\
23.45	0.00401906325216604\\
23.46	0.00401906020583446\\
23.47	0.00401905715817698\\
23.48	0.00401905410919291\\
23.49	0.00401905105888156\\
23.5	0.00401904800724225\\
23.51	0.00401904495427429\\
23.52	0.00401904189997699\\
23.53	0.00401903884434967\\
23.54	0.00401903578739162\\
23.55	0.00401903272910218\\
23.56	0.00401902966948063\\
23.57	0.0040190266085263\\
23.58	0.00401902354623848\\
23.59	0.00401902048261649\\
23.6	0.00401901741765962\\
23.61	0.00401901435136719\\
23.62	0.00401901128373849\\
23.63	0.00401900821477283\\
23.64	0.0040190051444695\\
23.65	0.00401900207282783\\
23.66	0.00401899899984709\\
23.67	0.00401899592552659\\
23.68	0.00401899284986563\\
23.69	0.0040189897728635\\
23.7	0.00401898669451951\\
23.71	0.00401898361483294\\
23.72	0.00401898053380309\\
23.73	0.00401897745142925\\
23.74	0.00401897436771072\\
23.75	0.00401897128264679\\
23.76	0.00401896819623675\\
23.77	0.00401896510847988\\
23.78	0.00401896201937547\\
23.79	0.00401895892892282\\
23.8	0.0040189558371212\\
23.81	0.00401895274396991\\
23.82	0.00401894964946822\\
23.83	0.00401894655361541\\
23.84	0.00401894345641078\\
23.85	0.0040189403578536\\
23.86	0.00401893725794315\\
23.87	0.0040189341566787\\
23.88	0.00401893105405954\\
23.89	0.00401892795008494\\
23.9	0.00401892484475417\\
23.91	0.00401892173806651\\
23.92	0.00401891863002124\\
23.93	0.00401891552061761\\
23.94	0.00401891240985491\\
23.95	0.00401890929773239\\
23.96	0.00401890618424933\\
23.97	0.00401890306940499\\
23.98	0.00401889995319864\\
23.99	0.00401889683562954\\
24	0.00401889371669695\\
24.01	0.00401889059640013\\
24.02	0.00401888747473834\\
24.03	0.00401888435171084\\
24.04	0.00401888122731689\\
24.05	0.00401887810155573\\
24.06	0.00401887497442664\\
24.07	0.00401887184592884\\
24.08	0.0040188687160616\\
24.09	0.00401886558482416\\
24.1	0.00401886245221577\\
24.11	0.00401885931823569\\
24.12	0.00401885618288314\\
24.13	0.00401885304615738\\
24.14	0.00401884990805764\\
24.15	0.00401884676858317\\
24.16	0.0040188436277332\\
24.17	0.00401884048550697\\
24.18	0.00401883734190372\\
24.19	0.00401883419692266\\
24.2	0.00401883105056305\\
24.21	0.00401882790282411\\
24.22	0.00401882475370506\\
24.23	0.00401882160320513\\
24.24	0.00401881845132354\\
24.25	0.00401881529805952\\
24.26	0.00401881214341229\\
24.27	0.00401880898738105\\
24.28	0.00401880582996504\\
24.29	0.00401880267116347\\
24.3	0.00401879951097555\\
24.31	0.00401879634940048\\
24.32	0.00401879318643748\\
24.33	0.00401879002208575\\
24.34	0.00401878685634451\\
24.35	0.00401878368921294\\
24.36	0.00401878052069025\\
24.37	0.00401877735077564\\
24.38	0.0040187741794683\\
24.39	0.00401877100676743\\
24.4	0.00401876783267222\\
24.41	0.00401876465718186\\
24.42	0.00401876148029553\\
24.43	0.00401875830201242\\
24.44	0.00401875512233172\\
24.45	0.00401875194125259\\
24.46	0.00401874875877422\\
24.47	0.00401874557489578\\
24.48	0.00401874238961645\\
24.49	0.00401873920293539\\
24.5	0.00401873601485177\\
24.51	0.00401873282536476\\
24.52	0.00401872963447351\\
24.53	0.00401872644217719\\
24.54	0.00401872324847495\\
24.55	0.00401872005336596\\
24.56	0.00401871685684934\\
24.57	0.00401871365892426\\
24.58	0.00401871045958986\\
24.59	0.00401870725884529\\
24.6	0.00401870405668967\\
24.61	0.00401870085312215\\
24.62	0.00401869764814186\\
24.63	0.00401869444174793\\
24.64	0.00401869123393949\\
24.65	0.00401868802471566\\
24.66	0.00401868481407556\\
24.67	0.00401868160201831\\
24.68	0.00401867838854302\\
24.69	0.00401867517364881\\
24.7	0.00401867195733477\\
24.71	0.00401866873960002\\
24.72	0.00401866552044365\\
24.73	0.00401866229986476\\
24.74	0.00401865907786245\\
24.75	0.00401865585443579\\
24.76	0.00401865262958388\\
24.77	0.00401864940330581\\
24.78	0.00401864617560064\\
24.79	0.00401864294646745\\
24.8	0.0040186397159053\\
24.81	0.00401863648391328\\
24.82	0.00401863325049044\\
24.83	0.00401863001563582\\
24.84	0.00401862677934851\\
24.85	0.00401862354162752\\
24.86	0.00401862030247193\\
24.87	0.00401861706188076\\
24.88	0.00401861381985305\\
24.89	0.00401861057638783\\
24.9	0.00401860733148413\\
24.91	0.00401860408514098\\
24.92	0.00401860083735739\\
24.93	0.00401859758813236\\
24.94	0.00401859433746492\\
24.95	0.00401859108535407\\
24.96	0.0040185878317988\\
24.97	0.0040185845767981\\
24.98	0.00401858132035096\\
24.99	0.00401857806245638\\
25	0.00401857480311332\\
25.01	0.00401857154232075\\
25.02	0.00401856828007764\\
25.03	0.00401856501638297\\
25.04	0.00401856175123566\\
25.05	0.0040185584846347\\
25.06	0.004018555216579\\
25.07	0.00401855194706751\\
25.08	0.00401854867609917\\
25.09	0.0040185454036729\\
25.1	0.00401854212978761\\
25.11	0.00401853885444223\\
25.12	0.00401853557763566\\
25.13	0.0040185322993668\\
25.14	0.00401852901963454\\
25.15	0.00401852573843778\\
25.16	0.00401852245577539\\
25.17	0.00401851917164624\\
25.18	0.00401851588604921\\
25.19	0.00401851259898316\\
25.2	0.00401850931044694\\
25.21	0.00401850602043939\\
25.22	0.00401850272895934\\
25.23	0.00401849943600565\\
25.24	0.00401849614157712\\
25.25	0.00401849284567257\\
25.26	0.00401848954829081\\
25.27	0.00401848624943064\\
25.28	0.00401848294909084\\
25.29	0.00401847964727021\\
25.3	0.00401847634396752\\
25.31	0.00401847303918153\\
25.32	0.00401846973291101\\
25.33	0.0040184664251547\\
25.34	0.00401846311591134\\
25.35	0.00401845980517967\\
25.36	0.0040184564929584\\
25.37	0.00401845317924625\\
25.38	0.00401844986404193\\
25.39	0.00401844654734412\\
25.4	0.00401844322915152\\
25.41	0.00401843990946278\\
25.42	0.00401843658827659\\
25.43	0.0040184332655916\\
25.44	0.00401842994140644\\
25.45	0.00401842661571976\\
25.46	0.00401842328853017\\
25.47	0.00401841995983628\\
25.48	0.00401841662963671\\
25.49	0.00401841329793003\\
25.5	0.00401840996471483\\
25.51	0.00401840662998967\\
25.52	0.00401840329375312\\
25.53	0.0040183999560037\\
25.54	0.00401839661673996\\
25.55	0.00401839327596042\\
25.56	0.00401838993366357\\
25.57	0.00401838658984792\\
25.58	0.00401838324451195\\
25.59	0.00401837989765412\\
25.6	0.00401837654927289\\
25.61	0.00401837319936671\\
25.62	0.004018369847934\\
25.63	0.00401836649497317\\
25.64	0.00401836314048262\\
25.65	0.00401835978446076\\
25.66	0.00401835642690593\\
25.67	0.0040183530678165\\
25.68	0.00401834970719082\\
25.69	0.0040183463450272\\
25.7	0.00401834298132397\\
25.71	0.00401833961607941\\
25.72	0.00401833624929181\\
25.73	0.00401833288095942\\
25.74	0.00401832951108051\\
25.75	0.0040183261396533\\
25.76	0.00401832276667599\\
25.77	0.0040183193921468\\
25.78	0.00401831601606389\\
25.79	0.00401831263842543\\
25.8	0.00401830925922956\\
25.81	0.00401830587847441\\
25.82	0.00401830249615809\\
25.83	0.00401829911227868\\
25.84	0.00401829572683424\\
25.85	0.00401829233982284\\
25.86	0.00401828895124249\\
25.87	0.00401828556109122\\
25.88	0.004018282169367\\
25.89	0.0040182787760678\\
25.9	0.00401827538119158\\
25.91	0.00401827198473626\\
25.92	0.00401826858669973\\
25.93	0.0040182651870799\\
25.94	0.0040182617858746\\
25.95	0.00401825838308168\\
25.96	0.00401825497869896\\
25.97	0.00401825157272423\\
25.98	0.00401824816515524\\
25.99	0.00401824475598975\\
26	0.00401824134522547\\
26.01	0.0040182379328601\\
26.02	0.0040182345188913\\
26.03	0.00401823110331671\\
26.04	0.00401822768613395\\
26.05	0.00401822426734061\\
26.06	0.00401822084693425\\
26.07	0.0040182174249124\\
26.08	0.00401821400127258\\
26.09	0.00401821057601226\\
26.1	0.00401820714912888\\
26.11	0.00401820372061988\\
26.12	0.00401820029048265\\
26.13	0.00401819685871453\\
26.14	0.00401819342531287\\
26.15	0.00401818999027496\\
26.16	0.00401818655359808\\
26.17	0.00401818311527945\\
26.18	0.00401817967531628\\
26.19	0.00401817623370574\\
26.2	0.00401817279044498\\
26.21	0.00401816934553107\\
26.22	0.00401816589896112\\
26.23	0.00401816245073213\\
26.24	0.00401815900084111\\
26.25	0.00401815554928502\\
26.26	0.00401815209606078\\
26.27	0.00401814864116529\\
26.28	0.00401814518459538\\
26.29	0.00401814172634788\\
26.3	0.00401813826641955\\
26.31	0.00401813480480712\\
26.32	0.00401813134150728\\
26.33	0.00401812787651669\\
26.34	0.00401812440983194\\
26.35	0.00401812094144962\\
26.36	0.00401811747136623\\
26.37	0.00401811399957826\\
26.38	0.00401811052608214\\
26.39	0.00401810705087427\\
26.4	0.00401810357395098\\
26.41	0.00401810009530858\\
26.42	0.00401809661494331\\
26.43	0.00401809313285138\\
26.44	0.00401808964902894\\
26.45	0.00401808616347208\\
26.46	0.00401808267617688\\
26.47	0.00401807918713933\\
26.48	0.00401807569635538\\
26.49	0.00401807220382092\\
26.5	0.00401806870953181\\
26.51	0.00401806521348384\\
26.52	0.00401806171567274\\
26.53	0.00401805821609419\\
26.54	0.0040180547147438\\
26.55	0.00401805121161716\\
26.56	0.00401804770670976\\
26.57	0.00401804420001706\\
26.58	0.00401804069153443\\
26.59	0.0040180371812572\\
26.6	0.00401803366918064\\
26.61	0.00401803015529995\\
26.62	0.00401802663961025\\
26.63	0.00401802312210662\\
26.64	0.00401801960278407\\
26.65	0.00401801608163753\\
26.66	0.00401801255866186\\
26.67	0.00401800903385188\\
26.68	0.00401800550720229\\
26.69	0.00401800197870776\\
26.7	0.00401799844836288\\
26.71	0.00401799491616216\\
26.72	0.00401799138210002\\
26.73	0.00401798784617083\\
26.74	0.00401798430836887\\
26.75	0.00401798076868834\\
26.76	0.00401797722712336\\
26.77	0.00401797368366798\\
26.78	0.00401797013831616\\
26.79	0.00401796659106178\\
26.8	0.00401796304189862\\
26.81	0.0040179594908204\\
26.82	0.00401795593782073\\
26.83	0.00401795238289314\\
26.84	0.00401794882603108\\
26.85	0.00401794526722788\\
26.86	0.00401794170647682\\
26.87	0.00401793814377105\\
26.88	0.00401793457910364\\
26.89	0.00401793101246754\\
26.9	0.00401792744385565\\
26.91	0.00401792387326073\\
26.92	0.00401792030067544\\
26.93	0.00401791672609235\\
26.94	0.00401791314950393\\
26.95	0.00401790957090252\\
26.96	0.00401790599028039\\
26.97	0.00401790240762965\\
26.98	0.00401789882294235\\
26.99	0.00401789523621039\\
27	0.00401789164742557\\
27.01	0.00401788805657958\\
27.02	0.00401788446366397\\
27.03	0.00401788086867019\\
27.04	0.00401787727158958\\
27.05	0.00401787367241332\\
27.06	0.00401787007113249\\
27.07	0.00401786646773803\\
27.08	0.00401786286222077\\
27.09	0.0040178592545714\\
27.1	0.00401785564478045\\
27.11	0.00401785203283836\\
27.12	0.00401784841873541\\
27.13	0.00401784480246173\\
27.14	0.00401784118400733\\
27.15	0.00401783756336207\\
27.16	0.00401783394051566\\
27.17	0.00401783031545766\\
27.18	0.00401782668817749\\
27.19	0.00401782305866441\\
27.2	0.00401781942690754\\
27.21	0.00401781579289581\\
27.22	0.00401781215661804\\
27.23	0.00401780851806284\\
27.24	0.00401780487721869\\
27.25	0.0040178012340739\\
27.26	0.00401779758861658\\
27.27	0.00401779394083472\\
27.28	0.00401779029071609\\
27.29	0.00401778663824832\\
27.3	0.00401778298341883\\
27.31	0.00401777932621489\\
27.32	0.00401777566662356\\
27.33	0.00401777200463172\\
27.34	0.00401776834022609\\
27.35	0.00401776467339314\\
27.36	0.00401776100411919\\
27.37	0.00401775733239036\\
27.38	0.00401775365819255\\
27.39	0.00401774998151147\\
27.4	0.00401774630233262\\
27.41	0.00401774262064129\\
27.42	0.00401773893642255\\
27.43	0.00401773524966128\\
27.44	0.00401773156034212\\
27.45	0.00401772786844949\\
27.46	0.00401772417396758\\
27.47	0.00401772047688038\\
27.48	0.00401771677717161\\
27.49	0.00401771307482478\\
27.5	0.00401770936982316\\
27.51	0.00401770566214978\\
27.52	0.00401770195178741\\
27.53	0.00401769823871858\\
27.54	0.00401769452292559\\
27.55	0.00401769080439044\\
27.56	0.00401768708309491\\
27.57	0.0040176833590205\\
27.58	0.00401767963214845\\
27.59	0.00401767590245973\\
27.6	0.00401767216993502\\
27.61	0.00401766843455474\\
27.62	0.00401766469629903\\
27.63	0.00401766095514775\\
27.64	0.00401765721108042\\
27.65	0.00401765346407635\\
27.66	0.00401764971411447\\
27.67	0.00401764596117348\\
27.68	0.00401764220523171\\
27.69	0.00401763844626722\\
27.7	0.00401763468425775\\
27.71	0.0040176309191807\\
27.72	0.00401762715101318\\
27.73	0.00401762337973195\\
27.74	0.00401761960531345\\
27.75	0.00401761582773376\\
27.76	0.00401761204696863\\
27.77	0.00401760826299348\\
27.78	0.00401760447578337\\
27.79	0.00401760068531299\\
27.8	0.00401759689155668\\
27.81	0.00401759309448843\\
27.82	0.00401758929408183\\
27.83	0.00401758549031012\\
27.84	0.00401758168314613\\
27.85	0.00401757787256234\\
27.86	0.00401757405853081\\
27.87	0.00401757024102323\\
27.88	0.00401756642001086\\
27.89	0.00401756259546457\\
27.9	0.00401755876735481\\
27.91	0.00401755493565162\\
27.92	0.00401755110032462\\
27.93	0.00401754726134298\\
27.94	0.00401754341867546\\
27.95	0.00401753957229035\\
27.96	0.00401753572215553\\
27.97	0.00401753186823838\\
27.98	0.00401752801050587\\
27.99	0.00401752414892448\\
28	0.00401752028346022\\
28.01	0.00401751641407862\\
28.02	0.00401751254074473\\
28.03	0.00401750866342312\\
28.04	0.00401750478207785\\
28.05	0.00401750089667246\\
28.06	0.00401749700717003\\
28.07	0.00401749311353308\\
28.08	0.00401748921572362\\
28.09	0.00401748531370313\\
28.1	0.00401748140743255\\
28.11	0.00401747749687227\\
28.12	0.00401747358198216\\
28.13	0.00401746966272147\\
28.14	0.00401746573904895\\
28.15	0.00401746181092272\\
28.16	0.00401745787830036\\
28.17	0.00401745394113884\\
28.18	0.00401744999939453\\
28.19	0.00401744605302321\\
28.2	0.00401744210198004\\
28.21	0.00401743814621954\\
28.22	0.00401743418569563\\
28.23	0.00401743022036158\\
28.24	0.00401742625017\\
28.25	0.00401742227507288\\
28.26	0.0040174182950215\\
28.27	0.00401741430996651\\
28.28	0.00401741031985784\\
28.29	0.00401740632464477\\
28.3	0.00401740232427584\\
28.31	0.00401739831869891\\
28.32	0.0040173943078611\\
28.33	0.00401739029170882\\
28.34	0.00401738627018773\\
28.35	0.00401738224324273\\
28.36	0.00401737821081799\\
28.37	0.00401737417285688\\
28.38	0.00401737012930202\\
28.39	0.00401736608009521\\
28.4	0.00401736202517746\\
28.41	0.00401735796448898\\
28.42	0.00401735389796914\\
28.43	0.00401734982555647\\
28.44	0.00401734574718866\\
28.45	0.00401734166280256\\
28.46	0.0040173375723341\\
28.47	0.00401733347571838\\
28.48	0.00401732937288956\\
28.49	0.00401732526378091\\
28.5	0.00401732114832478\\
28.51	0.00401731702645256\\
28.52	0.00401731289809473\\
28.53	0.00401730876318077\\
28.54	0.00401730462163918\\
28.55	0.00401730047339749\\
28.56	0.0040172963183822\\
28.57	0.0040172921565188\\
28.58	0.00401728798773173\\
28.59	0.00401728381194438\\
28.6	0.00401727962907906\\
28.61	0.00401727543905699\\
28.62	0.00401727124179829\\
28.63	0.00401726703722195\\
28.64	0.00401726282524581\\
28.65	0.00401725860578656\\
28.66	0.00401725437875971\\
28.67	0.00401725014407955\\
28.68	0.00401724590165916\\
28.69	0.00401724165141038\\
28.7	0.00401723739324379\\
28.71	0.00401723312706866\\
28.72	0.00401722885279299\\
28.73	0.00401722457032341\\
28.74	0.00401722027956522\\
28.75	0.00401721598042232\\
28.76	0.00401721167279723\\
28.77	0.004017207356591\\
28.78	0.00401720303170325\\
28.79	0.00401719869803211\\
28.8	0.00401719435547419\\
28.81	0.00401719000392454\\
28.82	0.00401718564327666\\
28.83	0.00401718127342241\\
28.84	0.00401717689425204\\
28.85	0.00401717250565412\\
28.86	0.00401716810751548\\
28.87	0.00401716369972125\\
28.88	0.00401715928215473\\
28.89	0.00401715485469745\\
28.9	0.00401715041722904\\
28.91	0.00401714596962723\\
28.92	0.00401714151176783\\
28.93	0.00401713704352463\\
28.94	0.00401713256476941\\
28.95	0.00401712807537185\\
28.96	0.00401712357519949\\
28.97	0.00401711906411772\\
28.98	0.00401711454198966\\
28.99	0.00401711000867615\\
29	0.00401710546403569\\
29.01	0.00401710090792436\\
29.02	0.00401709634019577\\
29.03	0.004017091760701\\
29.04	0.00401708716928855\\
29.05	0.00401708256580423\\
29.06	0.00401707795009112\\
29.07	0.00401707332198947\\
29.08	0.00401706868133667\\
29.09	0.00401706402796712\\
29.1	0.00401705936171217\\
29.11	0.00401705468240004\\
29.12	0.00401704998985571\\
29.13	0.00401704528390084\\
29.14	0.00401704056435368\\
29.15	0.00401703583102894\\
29.16	0.00401703108373773\\
29.17	0.00401702632228741\\
29.18	0.00401702154648149\\
29.19	0.00401701675611952\\
29.2	0.00401701195099697\\
29.21	0.00401700713090507\\
29.22	0.00401700229563072\\
29.23	0.00401699744495633\\
29.24	0.00401699257865967\\
29.25	0.00401698769651373\\
29.26	0.00401698279828655\\
29.27	0.0040169778837411\\
29.28	0.00401697295263505\\
29.29	0.00401696800472063\\
29.3	0.00401696303974444\\
29.31	0.00401695805744728\\
29.32	0.00401695305756389\\
29.33	0.00401694803982281\\
29.34	0.00401694300394614\\
29.35	0.00401693794964928\\
29.36	0.00401693287664078\\
29.37	0.00401692778462199\\
29.38	0.00401692267328691\\
29.39	0.00401691754232183\\
29.4	0.00401691239140515\\
29.41	0.00401690722020699\\
29.42	0.00401690202838901\\
29.43	0.00401689681560397\\
29.44	0.00401689158149554\\
29.45	0.00401688632569782\\
29.46	0.00401688104783513\\
29.47	0.00401687574752153\\
29.48	0.00401687042436051\\
29.49	0.00401686507794458\\
29.5	0.00401685970785481\\
29.51	0.00401685431366045\\
29.52	0.00401684889491849\\
29.53	0.00401684345117314\\
29.54	0.0040168379819554\\
29.55	0.00401683248678248\\
29.56	0.00401682696516046\\
29.57	0.00401682141658447\\
29.58	0.0040168158405384\\
29.59	0.00401681023649468\\
29.6	0.00401680460391397\\
29.61	0.00401679894224492\\
29.62	0.00401679325092381\\
29.63	0.00401678752937431\\
29.64	0.00401678177700716\\
29.65	0.00401677599321985\\
29.66	0.00401677017739627\\
29.67	0.00401676432890647\\
29.68	0.00401675844710621\\
29.69	0.00401675253133669\\
29.7	0.00401674658092417\\
29.71	0.00401674059517959\\
29.72	0.00401673457339822\\
29.73	0.00401672851485926\\
29.74	0.00401672241882546\\
29.75	0.00401671628454269\\
29.76	0.00401671011123957\\
29.77	0.00401670389812697\\
29.78	0.00401669764439765\\
29.79	0.00401669134922578\\
29.8	0.00401668501176646\\
29.81	0.00401667863115528\\
29.82	0.00401667220650781\\
29.83	0.00401666573691915\\
29.84	0.00401665922146333\\
29.85	0.00401665265919288\\
29.86	0.00401664604913825\\
29.87	0.00401663939030724\\
29.88	0.00401663268168448\\
29.89	0.00401662592223081\\
29.9	0.0040166191108827\\
29.91	0.00401661224655163\\
29.92	0.0040166053281235\\
29.93	0.0040165983544579\\
29.94	0.00401659132438755\\
29.95	0.00401658423671753\\
29.96	0.00401657709022466\\
29.97	0.00401656988365671\\
29.98	0.00401656261573174\\
29.99	0.00401655528513729\\
30	0.0040165478905296\\
30.01	0.00401654043053287\\
30.02	0.0040165329037384\\
30.03	0.00401652530870376\\
30.04	0.00401651764395193\\
30.05	0.00401650990797043\\
30.06	0.00401650209921036\\
30.07	0.00401649421608555\\
30.08	0.00401648625697155\\
30.09	0.00401647822020464\\
30.1	0.00401647010408086\\
30.11	0.00401646190685494\\
30.12	0.00401645362673926\\
30.13	0.00401644526190273\\
30.14	0.00401643681046969\\
30.15	0.00401642827051877\\
30.16	0.00401641964008167\\
30.17	0.00401641091714196\\
30.18	0.00401640209963384\\
30.19	0.00401639318544085\\
30.2	0.00401638417239457\\
30.21	0.00401637505827324\\
30.22	0.00401636584080038\\
30.23	0.00401635651764339\\
30.24	0.00401634708641203\\
30.25	0.00401633754465696\\
30.26	0.00401632788986819\\
30.27	0.00401631811947349\\
30.28	0.0040163082308367\\
30.29	0.00401629822125617\\
30.3	0.00401628808796292\\
30.31	0.00401627782811896\\
30.32	0.0040162674388154\\
30.33	0.00401625691707064\\
30.34	0.00401624625982843\\
30.35	0.0040162354639559\\
30.36	0.00401622452624154\\
30.37	0.00401621344339309\\
30.38	0.00401620221203546\\
30.39	0.00401619082870848\\
30.4	0.00401617928986467\\
30.41	0.00401616759186695\\
30.42	0.00401615573098617\\
30.43	0.00401614370339877\\
30.44	0.00401613150518417\\
30.45	0.00401611913232228\\
30.46	0.00401610658069075\\
30.47	0.00401609384606231\\
30.48	0.00401608092410193\\
30.49	0.00401606781036395\\
30.5	0.00401605450028914\\
30.51	0.00401604098920161\\
30.52	0.00401602727230571\\
30.53	0.00401601334468283\\
30.54	0.00401599920128808\\
30.55	0.00401598483694688\\
30.56	0.00401597024635151\\
30.57	0.00401595542405747\\
30.58	0.00401594036447986\\
30.59	0.00401592506188954\\
30.6	0.00401590951040926\\
30.61	0.00401589370400966\\
30.62	0.00401587763650515\\
30.63	0.0040158614257348\\
30.64	0.00401584520873793\\
30.65	0.00401582898551156\\
30.66	0.00401581275605272\\
30.67	0.00401579652035844\\
30.68	0.00401578027842573\\
30.69	0.0040157640302516\\
30.7	0.00401574777583307\\
30.71	0.00401573151516716\\
30.72	0.00401571524825087\\
30.73	0.0040156989750812\\
30.74	0.00401568269565518\\
30.75	0.00401566640996979\\
30.76	0.00401565011802205\\
30.77	0.00401563381980894\\
30.78	0.00401561751532748\\
30.79	0.00401560120457464\\
30.8	0.00401558488754744\\
30.81	0.00401556856424285\\
30.82	0.00401555223465787\\
30.83	0.00401553589878949\\
30.84	0.00401551955663468\\
30.85	0.00401550320819044\\
30.86	0.00401548685345373\\
30.87	0.00401547049242155\\
30.88	0.00401545412509086\\
30.89	0.00401543775145865\\
30.9	0.00401542137152187\\
30.91	0.00401540498527751\\
30.92	0.00401538859272252\\
30.93	0.00401537219385388\\
30.94	0.00401535578866856\\
30.95	0.0040153393771635\\
30.96	0.00401532295933567\\
30.97	0.00401530653518202\\
30.98	0.00401529010469952\\
30.99	0.00401527366788511\\
31	0.00401525722473574\\
31.01	0.00401524077524837\\
31.02	0.00401522431941993\\
31.03	0.00401520785724738\\
31.04	0.00401519138872765\\
31.05	0.00401517491385768\\
31.06	0.00401515843263441\\
31.07	0.00401514194505478\\
31.08	0.00401512545111572\\
31.09	0.00401510895081416\\
31.1	0.00401509244414703\\
31.11	0.00401507593111125\\
31.12	0.00401505941170375\\
31.13	0.00401504288592146\\
31.14	0.00401502635376129\\
31.15	0.00401500981522015\\
31.16	0.00401499327029497\\
31.17	0.00401497671898267\\
31.18	0.00401496016128014\\
31.19	0.0040149435971843\\
31.2	0.00401492702669205\\
31.21	0.00401491044980031\\
31.22	0.00401489386650596\\
31.23	0.00401487727680593\\
31.24	0.00401486068069708\\
31.25	0.00401484407817634\\
31.26	0.00401482746924059\\
31.27	0.00401481085388672\\
31.28	0.00401479423211161\\
31.29	0.00401477760391217\\
31.3	0.00401476096928527\\
31.31	0.00401474432822779\\
31.32	0.00401472768073661\\
31.33	0.00401471102680862\\
31.34	0.00401469436644068\\
31.35	0.00401467769962968\\
31.36	0.00401466102637247\\
31.37	0.00401464434666594\\
31.38	0.00401462766050694\\
31.39	0.00401461096789235\\
31.4	0.00401459426881902\\
31.41	0.00401457756328382\\
31.42	0.00401456085128359\\
31.43	0.00401454413281521\\
31.44	0.00401452740787551\\
31.45	0.00401451067646136\\
31.46	0.00401449393856959\\
31.47	0.00401447719419706\\
31.48	0.00401446044334061\\
31.49	0.00401444368599708\\
31.5	0.00401442692216332\\
31.51	0.00401441015183615\\
31.52	0.00401439337501241\\
31.53	0.00401437659168894\\
31.54	0.00401435980186257\\
31.55	0.00401434300553012\\
31.56	0.00401432620268843\\
31.57	0.0040143093933343\\
31.58	0.00401429257746457\\
31.59	0.00401427575507605\\
31.6	0.00401425892616556\\
31.61	0.00401424209072992\\
31.62	0.00401422524876594\\
31.63	0.00401420840027042\\
31.64	0.00401419154524018\\
31.65	0.00401417468367201\\
31.66	0.00401415781556272\\
31.67	0.00401414094090912\\
31.68	0.00401412405970799\\
31.69	0.00401410717195614\\
31.7	0.00401409027765036\\
31.71	0.00401407337678744\\
31.72	0.00401405646936417\\
31.73	0.00401403955537733\\
31.74	0.00401402263482371\\
31.75	0.00401400570770009\\
31.76	0.00401398877400325\\
31.77	0.00401397183372997\\
31.78	0.00401395488687701\\
31.79	0.00401393793344117\\
31.8	0.0040139209734192\\
31.81	0.00401390400680787\\
31.82	0.00401388703360394\\
31.83	0.00401387005380419\\
31.84	0.00401385306740537\\
31.85	0.00401383607440423\\
31.86	0.00401381907479754\\
31.87	0.00401380206858204\\
31.88	0.0040137850557545\\
31.89	0.00401376803631165\\
31.9	0.00401375101025025\\
31.91	0.00401373397756703\\
31.92	0.00401371693825874\\
31.93	0.00401369989232213\\
31.94	0.00401368283975391\\
31.95	0.00401366578055084\\
31.96	0.00401364871470964\\
31.97	0.00401363164222703\\
31.98	0.00401361456309976\\
31.99	0.00401359747732454\\
32	0.00401358038489809\\
32.01	0.00401356328581714\\
32.02	0.0040135461800784\\
32.03	0.00401352906767859\\
32.04	0.00401351194861442\\
32.05	0.0040134948228826\\
32.06	0.00401347769047984\\
32.07	0.00401346055140285\\
32.08	0.00401344340564832\\
32.09	0.00401342625321295\\
32.1	0.00401340909409346\\
32.11	0.00401339192828653\\
32.12	0.00401337475578885\\
32.13	0.00401335757659711\\
32.14	0.00401334039070801\\
32.15	0.00401332319811824\\
32.16	0.00401330599882446\\
32.17	0.00401328879282337\\
32.18	0.00401327158011164\\
32.19	0.00401325436068595\\
32.2	0.00401323713454298\\
32.21	0.00401321990167939\\
32.22	0.00401320266209185\\
32.23	0.00401318541577703\\
32.24	0.00401316816273159\\
32.25	0.0040131509029522\\
32.26	0.00401313363643552\\
32.27	0.00401311636317819\\
32.28	0.00401309908317687\\
32.29	0.00401308179642823\\
32.3	0.00401306450292889\\
32.31	0.00401304720267552\\
32.32	0.00401302989566475\\
32.33	0.00401301258189322\\
32.34	0.00401299526135758\\
32.35	0.00401297793405446\\
32.36	0.0040129605999805\\
32.37	0.00401294325913232\\
32.38	0.00401292591150656\\
32.39	0.00401290855709984\\
32.4	0.00401289119590879\\
32.41	0.00401287382793002\\
32.42	0.00401285645316016\\
32.43	0.00401283907159582\\
32.44	0.00401282168323363\\
32.45	0.00401280428807017\\
32.46	0.00401278688610208\\
32.47	0.00401276947732594\\
32.48	0.00401275206173837\\
32.49	0.00401273463933597\\
32.5	0.00401271721011534\\
32.51	0.00401269977407307\\
32.52	0.00401268233120575\\
32.53	0.00401266488150998\\
32.54	0.00401264742498235\\
32.55	0.00401262996161943\\
32.56	0.00401261249141782\\
32.57	0.0040125950143741\\
32.58	0.00401257753048484\\
32.59	0.00401256003974662\\
32.6	0.00401254254215601\\
32.61	0.00401252503770958\\
32.62	0.0040125075264039\\
32.63	0.00401249000823554\\
32.64	0.00401247248320106\\
32.65	0.00401245495129701\\
32.66	0.00401243741251996\\
32.67	0.00401241986686646\\
32.68	0.00401240231433306\\
32.69	0.00401238475491632\\
32.7	0.00401236718861278\\
32.71	0.00401234961541898\\
32.72	0.00401233203533146\\
32.73	0.00401231444834677\\
32.74	0.00401229685446145\\
32.75	0.00401227925367203\\
32.76	0.00401226164597502\\
32.77	0.00401224403136698\\
32.78	0.00401222640984443\\
32.79	0.00401220878140387\\
32.8	0.00401219114604185\\
32.81	0.00401217350375487\\
32.82	0.00401215585453945\\
32.83	0.00401213819839211\\
32.84	0.00401212053530936\\
32.85	0.00401210286528769\\
32.86	0.00401208518832363\\
32.87	0.00401206750441366\\
32.88	0.0040120498135543\\
32.89	0.00401203211574203\\
32.9	0.00401201441097336\\
32.91	0.00401199669924477\\
32.92	0.00401197898055275\\
32.93	0.0040119612548938\\
32.94	0.00401194352226438\\
32.95	0.004011925782661\\
32.96	0.00401190803608011\\
32.97	0.00401189028251821\\
32.98	0.00401187252197176\\
32.99	0.00401185475443723\\
33	0.00401183697991109\\
33.01	0.00401181919838981\\
33.02	0.00401180140986984\\
33.03	0.00401178361434765\\
33.04	0.0040117658118197\\
33.05	0.00401174800228244\\
33.06	0.00401173018573231\\
33.07	0.00401171236216578\\
33.08	0.00401169453157928\\
33.09	0.00401167669396926\\
33.1	0.00401165884933215\\
33.11	0.00401164099766441\\
33.12	0.00401162313896246\\
33.13	0.00401160527322273\\
33.14	0.00401158740044166\\
33.15	0.00401156952061567\\
33.16	0.00401155163374118\\
33.17	0.00401153373981462\\
33.18	0.00401151583883242\\
33.19	0.00401149793079097\\
33.2	0.00401148001568669\\
33.21	0.00401146209351601\\
33.22	0.00401144416427531\\
33.23	0.00401142622796102\\
33.24	0.00401140828456953\\
33.25	0.00401139033409724\\
33.26	0.00401137237654055\\
33.27	0.00401135441189585\\
33.28	0.00401133644015952\\
33.29	0.00401131846132798\\
33.3	0.00401130047539759\\
33.31	0.00401128248236474\\
33.32	0.0040112644822258\\
33.33	0.00401124647497716\\
33.34	0.0040112284606152\\
33.35	0.00401121043913628\\
33.36	0.00401119241053676\\
33.37	0.00401117437481303\\
33.38	0.00401115633196144\\
33.39	0.00401113828197835\\
33.4	0.00401112022486011\\
33.41	0.0040111021606031\\
33.42	0.00401108408920364\\
33.43	0.0040110660106581\\
33.44	0.00401104792496282\\
33.45	0.00401102983211414\\
33.46	0.00401101173210841\\
33.47	0.00401099362494197\\
33.48	0.00401097551061113\\
33.49	0.00401095738911225\\
33.5	0.00401093926044165\\
33.51	0.00401092112459565\\
33.52	0.00401090298157058\\
33.53	0.00401088483136276\\
33.54	0.00401086667396851\\
33.55	0.00401084850938413\\
33.56	0.00401083033760596\\
33.57	0.00401081215863029\\
33.58	0.00401079397245342\\
33.59	0.00401077577907167\\
33.6	0.00401075757848133\\
33.61	0.00401073937067871\\
33.62	0.00401072115566008\\
33.63	0.00401070293342176\\
33.64	0.00401068470396002\\
33.65	0.00401066646727116\\
33.66	0.00401064822335145\\
33.67	0.00401062997219718\\
33.68	0.00401061171380462\\
33.69	0.00401059344817005\\
33.7	0.00401057517528974\\
33.71	0.00401055689515996\\
33.72	0.00401053860777698\\
33.73	0.00401052031313704\\
33.74	0.00401050201123643\\
33.75	0.00401048370207139\\
33.76	0.00401046538563818\\
33.77	0.00401044706193304\\
33.78	0.00401042873095223\\
33.79	0.004010410392692\\
33.8	0.00401039204714857\\
33.81	0.0040103736943182\\
33.82	0.00401035533419712\\
33.83	0.00401033696678157\\
33.84	0.00401031859206776\\
33.85	0.00401030021005194\\
33.86	0.00401028182073032\\
33.87	0.00401026342409913\\
33.88	0.00401024502015458\\
33.89	0.00401022660889289\\
33.9	0.00401020819031028\\
33.91	0.00401018976440294\\
33.92	0.0040101713311671\\
33.93	0.00401015289059895\\
33.94	0.00401013444269469\\
33.95	0.00401011598745052\\
33.96	0.00401009752486263\\
33.97	0.00401007905492723\\
33.98	0.00401006057764048\\
33.99	0.00401004209299859\\
34	0.00401002360099773\\
34.01	0.00401000510163408\\
34.02	0.00400998659490382\\
34.03	0.00400996808080312\\
34.04	0.00400994955932815\\
34.05	0.00400993103047509\\
34.06	0.00400991249424008\\
34.07	0.0040098939506193\\
34.08	0.00400987539960891\\
34.09	0.00400985684120505\\
34.1	0.00400983827540388\\
34.11	0.00400981970220156\\
34.12	0.00400980112159421\\
34.13	0.004009782533578\\
34.14	0.00400976393814905\\
34.15	0.00400974533530351\\
34.16	0.0040097267250375\\
34.17	0.00400970810734717\\
34.18	0.00400968948222863\\
34.19	0.004009670849678\\
34.2	0.00400965220969143\\
34.21	0.00400963356226501\\
34.22	0.00400961490739487\\
34.23	0.00400959624507711\\
34.24	0.00400957757530785\\
34.25	0.00400955889808319\\
34.26	0.00400954021339923\\
34.27	0.00400952152125208\\
34.28	0.00400950282163783\\
34.29	0.00400948411455257\\
34.3	0.00400946539999239\\
34.31	0.00400944667795339\\
34.32	0.00400942794843164\\
34.33	0.00400940921142322\\
34.34	0.00400939046692422\\
34.35	0.00400937171493071\\
34.36	0.00400935295543875\\
34.37	0.00400933418844442\\
34.38	0.00400931541394379\\
34.39	0.0040092966319329\\
34.4	0.00400927784240782\\
34.41	0.00400925904536462\\
34.42	0.00400924024079933\\
34.43	0.00400922142870802\\
34.44	0.00400920260908672\\
34.45	0.00400918378193147\\
34.46	0.00400916494723833\\
34.47	0.00400914610500331\\
34.48	0.00400912725522247\\
34.49	0.00400910839789182\\
34.5	0.00400908953300739\\
34.51	0.00400907066056521\\
34.52	0.0040090517805613\\
34.53	0.00400903289299167\\
34.54	0.00400901399785234\\
34.55	0.00400899509513932\\
34.56	0.00400897618484861\\
34.57	0.00400895726697623\\
34.58	0.00400893834151817\\
34.59	0.00400891940847042\\
34.6	0.004008900467829\\
34.61	0.00400888151958988\\
34.62	0.00400886256374904\\
34.63	0.00400884360030249\\
34.64	0.0040088246292462\\
34.65	0.00400880565057615\\
34.66	0.0040087866642883\\
34.67	0.00400876767037864\\
34.68	0.00400874866884314\\
34.69	0.00400872965967775\\
34.7	0.00400871064287844\\
34.71	0.00400869161844117\\
34.72	0.0040086725863619\\
34.73	0.00400865354663657\\
34.74	0.00400863449926113\\
34.75	0.00400861544423153\\
34.76	0.00400859638154371\\
34.77	0.00400857731119362\\
34.78	0.00400855823317718\\
34.79	0.00400853914749033\\
34.8	0.00400852005412899\\
34.81	0.0040085009530891\\
34.82	0.00400848184436657\\
34.83	0.00400846272795732\\
34.84	0.00400844360385727\\
34.85	0.00400842447206233\\
34.86	0.00400840533256842\\
34.87	0.00400838618537143\\
34.88	0.00400836703046726\\
34.89	0.00400834786785182\\
34.9	0.004008328697521\\
34.91	0.00400830951947069\\
34.92	0.00400829033369678\\
34.93	0.00400827114019516\\
34.94	0.00400825193896172\\
34.95	0.00400823272999231\\
34.96	0.00400821351328283\\
34.97	0.00400819428882914\\
34.98	0.00400817505662712\\
34.99	0.00400815581667262\\
35	0.00400813656896151\\
35.01	0.00400811731348965\\
35.02	0.00400809805025289\\
35.03	0.00400807877924708\\
35.04	0.00400805950046807\\
35.05	0.0040080402139117\\
35.06	0.00400802091957382\\
35.07	0.00400800161745027\\
35.08	0.00400798230753687\\
35.09	0.00400796298982946\\
35.1	0.00400794366432386\\
35.11	0.00400792433101591\\
35.12	0.00400790498990141\\
35.13	0.00400788564097619\\
35.14	0.00400786628423606\\
35.15	0.00400784691967682\\
35.16	0.0040078275472943\\
35.17	0.00400780816708428\\
35.18	0.00400778877904257\\
35.19	0.00400776938316496\\
35.2	0.00400774997944725\\
35.21	0.00400773056788523\\
35.22	0.00400771114847468\\
35.23	0.00400769172121138\\
35.24	0.00400767228609111\\
35.25	0.00400765284310966\\
35.26	0.00400763339226277\\
35.27	0.00400761393354624\\
35.28	0.00400759446695581\\
35.29	0.00400757499248726\\
35.3	0.00400755551013633\\
35.31	0.00400753601989879\\
35.32	0.00400751652177039\\
35.33	0.00400749701574686\\
35.34	0.00400747750182395\\
35.35	0.00400745797999741\\
35.36	0.00400743845026297\\
35.37	0.00400741891261636\\
35.38	0.00400739936705331\\
35.39	0.00400737981356954\\
35.4	0.00400736025216078\\
35.41	0.00400734068282275\\
35.42	0.00400732110555116\\
35.43	0.00400730152034172\\
35.44	0.00400728192719015\\
35.45	0.00400726232609213\\
35.46	0.00400724271704338\\
35.47	0.00400722310003959\\
35.48	0.00400720347507645\\
35.49	0.00400718384214965\\
35.5	0.00400716420125489\\
35.51	0.00400714455238783\\
35.52	0.00400712489554417\\
35.53	0.00400710523071956\\
35.54	0.00400708555790969\\
35.55	0.00400706587711023\\
35.56	0.00400704618831684\\
35.57	0.00400702649152517\\
35.58	0.00400700678673088\\
35.59	0.00400698707392964\\
35.6	0.00400696735311708\\
35.61	0.00400694762428885\\
35.62	0.0040069278874406\\
35.63	0.00400690814256796\\
35.64	0.00400688838966657\\
35.65	0.00400686862873206\\
35.66	0.00400684885976005\\
35.67	0.00400682908274618\\
35.68	0.00400680929768605\\
35.69	0.00400678950457529\\
35.7	0.00400676970340951\\
35.71	0.00400674989418432\\
35.72	0.00400673007689532\\
35.73	0.00400671025153811\\
35.74	0.00400669041810829\\
35.75	0.00400667057660146\\
35.76	0.00400665072701321\\
35.77	0.00400663086933912\\
35.78	0.00400661100357477\\
35.79	0.00400659112971574\\
35.8	0.00400657124775761\\
35.81	0.00400655135769595\\
35.82	0.00400653145952633\\
35.83	0.00400651155324431\\
35.84	0.00400649163884545\\
35.85	0.00400647171632531\\
35.86	0.00400645178567944\\
35.87	0.00400643184690339\\
35.88	0.00400641189999271\\
35.89	0.00400639194494293\\
35.9	0.00400637198174958\\
35.91	0.00400635201040822\\
35.92	0.00400633203091436\\
35.93	0.00400631204326354\\
35.94	0.00400629204745126\\
35.95	0.00400627204347307\\
35.96	0.00400625203132445\\
35.97	0.00400623201100093\\
35.98	0.00400621198249802\\
35.99	0.00400619194581121\\
36	0.00400617190093601\\
36.01	0.00400615184786791\\
36.02	0.0040061317866024\\
36.03	0.00400611171713497\\
36.04	0.0040060916394611\\
36.05	0.00400607155357628\\
36.06	0.00400605145947597\\
36.07	0.00400603135715565\\
36.08	0.0040060112466108\\
36.09	0.00400599112783686\\
36.1	0.00400597100082931\\
36.11	0.0040059508655836\\
36.12	0.00400593072209517\\
36.13	0.0040059105703595\\
36.14	0.004005890410372\\
36.15	0.00400587024212813\\
36.16	0.00400585006562332\\
36.17	0.00400582988085301\\
36.18	0.00400580968781262\\
36.19	0.00400578948649758\\
36.2	0.00400576927690331\\
36.21	0.00400574905902523\\
36.22	0.00400572883285875\\
36.23	0.00400570859839927\\
36.24	0.00400568835564221\\
36.25	0.00400566810458297\\
36.26	0.00400564784521694\\
36.27	0.00400562757753952\\
36.28	0.00400560730154609\\
36.29	0.00400558701723204\\
36.3	0.00400556672459275\\
36.31	0.0040055464236236\\
36.32	0.00400552611431997\\
36.33	0.00400550579667721\\
36.34	0.0040054854706907\\
36.35	0.0040054651363558\\
36.36	0.00400544479366786\\
36.37	0.00400542444262224\\
36.38	0.00400540408321429\\
36.39	0.00400538371543934\\
36.4	0.00400536333929276\\
36.41	0.00400534295476986\\
36.42	0.00400532256186598\\
36.43	0.00400530216057646\\
36.44	0.00400528175089661\\
36.45	0.00400526133282176\\
36.46	0.00400524090634723\\
36.47	0.00400522047146831\\
36.48	0.00400520002818034\\
36.49	0.0040051795764786\\
36.5	0.0040051591163584\\
36.51	0.00400513864781503\\
36.52	0.00400511817084379\\
36.53	0.00400509768543997\\
36.54	0.00400507719159884\\
36.55	0.00400505668931569\\
36.56	0.0040050361785858\\
36.57	0.00400501565940443\\
36.58	0.00400499513176685\\
36.59	0.00400497459566833\\
36.6	0.00400495405110412\\
36.61	0.00400493349806948\\
36.62	0.00400491293655966\\
36.63	0.0040048923665699\\
36.64	0.00400487178809545\\
36.65	0.00400485120113155\\
36.66	0.00400483060567343\\
36.67	0.00400481000171632\\
36.68	0.00400478938925544\\
36.69	0.00400476876828602\\
36.7	0.00400474813880327\\
36.71	0.00400472750080241\\
36.72	0.00400470685427865\\
36.73	0.00400468619922718\\
36.74	0.00400466553564321\\
36.75	0.00400464486352194\\
36.76	0.00400462418285856\\
36.77	0.00400460349364825\\
36.78	0.0040045827958862\\
36.79	0.00400456208956759\\
36.8	0.00400454137468759\\
36.81	0.00400452065124137\\
36.82	0.00400449991922409\\
36.83	0.00400447917863094\\
36.84	0.00400445842945704\\
36.85	0.00400443767169758\\
36.86	0.00400441690534767\\
36.87	0.00400439613040249\\
36.88	0.00400437534685716\\
36.89	0.00400435455470682\\
36.9	0.00400433375394661\\
36.91	0.00400431294457165\\
36.92	0.00400429212657706\\
36.93	0.00400427129995797\\
36.94	0.00400425046470949\\
36.95	0.00400422962082673\\
36.96	0.00400420876830478\\
36.97	0.00400418790713877\\
36.98	0.00400416703732378\\
36.99	0.0040041461588549\\
37	0.00400412527172723\\
37.01	0.00400410437593584\\
37.02	0.00400408347147582\\
37.03	0.00400406255834225\\
37.04	0.00400404163653019\\
37.05	0.00400402070603472\\
37.06	0.00400399976685088\\
37.07	0.00400397881897375\\
37.08	0.00400395786239837\\
37.09	0.00400393689711979\\
37.1	0.00400391592313306\\
37.11	0.00400389494043322\\
37.12	0.0040038739490153\\
37.13	0.00400385294887433\\
37.14	0.00400383194000535\\
37.15	0.00400381092240337\\
37.16	0.0040037898960634\\
37.17	0.00400376886098048\\
37.18	0.00400374781714959\\
37.19	0.00400372676456576\\
37.2	0.00400370570322397\\
37.21	0.00400368463311922\\
37.22	0.00400366355424651\\
37.23	0.00400364246660082\\
37.24	0.00400362137017713\\
37.25	0.00400360026497041\\
37.26	0.00400357915097566\\
37.27	0.00400355802818782\\
37.28	0.00400353689660187\\
37.29	0.00400351575621276\\
37.3	0.00400349460701545\\
37.31	0.00400347344900489\\
37.32	0.00400345228217603\\
37.33	0.00400343110652381\\
37.34	0.00400340992204315\\
37.35	0.004003388728729\\
37.36	0.00400336752657629\\
37.37	0.00400334631557992\\
37.38	0.00400332509573484\\
37.39	0.00400330386703594\\
37.4	0.00400328262947814\\
37.41	0.00400326138305634\\
37.42	0.00400324012776544\\
37.43	0.00400321886360034\\
37.44	0.00400319759055592\\
37.45	0.00400317630862708\\
37.46	0.0040031550178087\\
37.47	0.00400313371809564\\
37.48	0.00400311240948279\\
37.49	0.00400309109196501\\
37.5	0.00400306976553717\\
37.51	0.00400304843019412\\
37.52	0.00400302708593072\\
37.53	0.00400300573274181\\
37.54	0.00400298437062224\\
37.55	0.00400296299956685\\
37.56	0.00400294161957048\\
37.57	0.00400292023062795\\
37.58	0.00400289883273409\\
37.59	0.00400287742588372\\
37.6	0.00400285601007166\\
37.61	0.00400283458529271\\
37.62	0.0040028131515417\\
37.63	0.0040027917088134\\
37.64	0.00400277025710264\\
37.65	0.00400274879640419\\
37.66	0.00400272732671284\\
37.67	0.00400270584802338\\
37.68	0.00400268436033059\\
37.69	0.00400266286362923\\
37.7	0.00400264135791409\\
37.71	0.00400261984317992\\
37.72	0.00400259831942148\\
37.73	0.00400257678663352\\
37.74	0.0040025552448108\\
37.75	0.00400253369394806\\
37.76	0.00400251213404003\\
37.77	0.00400249056508147\\
37.78	0.00400246898706708\\
37.79	0.00400244739999161\\
37.8	0.00400242580384977\\
37.81	0.00400240419863628\\
37.82	0.00400238258434584\\
37.83	0.00400236096097317\\
37.84	0.00400233932851297\\
37.85	0.00400231768695993\\
37.86	0.00400229603630874\\
37.87	0.00400227437655409\\
37.88	0.00400225270769067\\
37.89	0.00400223102971315\\
37.9	0.0040022093426162\\
37.91	0.00400218764639449\\
37.92	0.00400216594104268\\
37.93	0.00400214422655543\\
37.94	0.00400212250292739\\
37.95	0.00400210077015321\\
37.96	0.00400207902822753\\
37.97	0.00400205727714499\\
37.98	0.00400203551690023\\
37.99	0.00400201374748786\\
38	0.00400199196890252\\
38.01	0.00400197018113882\\
38.02	0.00400194838419137\\
38.03	0.00400192657805479\\
38.04	0.00400190476272367\\
38.05	0.00400188293819262\\
38.06	0.00400186110445622\\
38.07	0.00400183926150907\\
38.08	0.00400181740934575\\
38.09	0.00400179554796083\\
38.1	0.00400177367734889\\
38.11	0.0040017517975045\\
38.12	0.00400172990842223\\
38.13	0.00400170801009661\\
38.14	0.00400168610252223\\
38.15	0.00400166418569361\\
38.16	0.00400164225960531\\
38.17	0.00400162032425186\\
38.18	0.00400159837962779\\
38.19	0.00400157642572764\\
38.2	0.00400155446254592\\
38.21	0.00400153249007716\\
38.22	0.00400151050831586\\
38.23	0.00400148851725653\\
38.24	0.00400146651689368\\
38.25	0.0040014445072218\\
38.26	0.00400142248823538\\
38.27	0.00400140045992892\\
38.28	0.0040013784222969\\
38.29	0.00400135637533378\\
38.3	0.00400133431903405\\
38.31	0.00400131225339217\\
38.32	0.0040012901784026\\
38.33	0.0040012680940598\\
38.34	0.00400124600035822\\
38.35	0.00400122389729231\\
38.36	0.0040012017848565\\
38.37	0.00400117966304523\\
38.38	0.00400115753185295\\
38.39	0.00400113539127405\\
38.4	0.00400111324130298\\
38.41	0.00400109108193415\\
38.42	0.00400106891316195\\
38.43	0.0040010467349808\\
38.44	0.00400102454738511\\
38.45	0.00400100235036925\\
38.46	0.00400098014392763\\
38.47	0.00400095792805463\\
38.48	0.00400093570274462\\
38.49	0.00400091346799198\\
38.5	0.00400089122379107\\
38.51	0.00400086897013626\\
38.52	0.00400084670702191\\
38.53	0.00400082443444237\\
38.54	0.00400080215239198\\
38.55	0.0040007798608651\\
38.56	0.00400075755985604\\
38.57	0.00400073524935915\\
38.58	0.00400071292936876\\
38.59	0.00400069059987918\\
38.6	0.00400066826088472\\
38.61	0.00400064591237971\\
38.62	0.00400062355435844\\
38.63	0.00400060118681521\\
38.64	0.00400057880974431\\
38.65	0.00400055642314005\\
38.66	0.00400053402699669\\
38.67	0.00400051162130852\\
38.68	0.00400048920606981\\
38.69	0.00400046678127483\\
38.7	0.00400044434691784\\
38.71	0.00400042190299309\\
38.72	0.00400039944949484\\
38.73	0.00400037698641734\\
38.74	0.00400035451375483\\
38.75	0.00400033203150153\\
38.76	0.00400030953965168\\
38.77	0.00400028703819951\\
38.78	0.00400026452713923\\
38.79	0.00400024200646506\\
38.8	0.0040002194761712\\
38.81	0.00400019693625186\\
38.82	0.00400017438670124\\
38.83	0.00400015182751353\\
38.84	0.0040001292586829\\
38.85	0.00400010668020356\\
38.86	0.00400008409206967\\
38.87	0.00400006149427539\\
38.88	0.00400003888681491\\
38.89	0.00400001626968237\\
38.9	0.00399999364287192\\
38.91	0.00399997100637773\\
38.92	0.00399994836019393\\
38.93	0.00399992570431465\\
38.94	0.00399990303873403\\
38.95	0.0039998803634462\\
38.96	0.00399985767844528\\
38.97	0.00399983498372538\\
38.98	0.00399981227928061\\
38.99	0.00399978956510508\\
39	0.00399976684119288\\
39.01	0.00399974410753812\\
39.02	0.00399972136413486\\
39.03	0.00399969861097721\\
39.04	0.00399967584805923\\
39.05	0.003999653075375\\
39.06	0.00399963029291858\\
39.07	0.00399960750068404\\
39.08	0.00399958469866542\\
39.09	0.00399956188685678\\
39.1	0.00399953906525215\\
39.11	0.00399951623384558\\
39.12	0.0039994933926311\\
39.13	0.00399947054160274\\
39.14	0.00399944768075451\\
39.15	0.00399942481008043\\
39.16	0.00399940192957452\\
39.17	0.00399937903923077\\
39.18	0.00399935613904319\\
39.19	0.00399933322900576\\
39.2	0.00399931030911247\\
39.21	0.00399928737935732\\
39.22	0.00399926443973426\\
39.23	0.00399924149023728\\
39.24	0.00399921853086033\\
39.25	0.00399919556159738\\
39.26	0.00399917258244238\\
39.27	0.00399914959338927\\
39.28	0.00399912659443201\\
39.29	0.00399910358556452\\
39.3	0.00399908056678074\\
39.31	0.00399905753807459\\
39.32	0.00399903449943998\\
39.33	0.00399901145087085\\
39.34	0.00399898839236109\\
39.35	0.00399896532390459\\
39.36	0.00399894224549528\\
39.37	0.00399891915712703\\
39.38	0.00399889605879371\\
39.39	0.00399887295048923\\
39.4	0.00399884983220745\\
39.41	0.00399882670394224\\
39.42	0.00399880356568746\\
39.43	0.00399878041743696\\
39.44	0.0039987572591846\\
39.45	0.00399873409092423\\
39.46	0.00399871091264967\\
39.47	0.00399868772435477\\
39.48	0.00399866452603336\\
39.49	0.00399864131767925\\
39.5	0.00399861809928626\\
39.51	0.0039985948708482\\
39.52	0.00399857163235888\\
39.53	0.00399854838381209\\
39.54	0.00399852512520163\\
39.55	0.00399850185652129\\
39.56	0.00399847857776484\\
39.57	0.00399845528892607\\
39.58	0.00399843198999874\\
39.59	0.00399840868097661\\
39.6	0.00399838536185344\\
39.61	0.003998362032623\\
39.62	0.00399833869327902\\
39.63	0.00399831534381524\\
39.64	0.0039982919842254\\
39.65	0.00399826861450323\\
39.66	0.00399824523464245\\
39.67	0.00399822184463678\\
39.68	0.00399819844447992\\
39.69	0.00399817503416559\\
39.7	0.00399815161368749\\
39.71	0.0039981281830393\\
39.72	0.00399810474221473\\
39.73	0.00399808129120744\\
39.74	0.00399805783001113\\
39.75	0.00399803435861944\\
39.76	0.00399801087702605\\
39.77	0.00399798738522462\\
39.78	0.00399796388320881\\
39.79	0.00399794037097225\\
39.8	0.0039979168485086\\
39.81	0.00399789331581147\\
39.82	0.00399786977287451\\
39.83	0.00399784621969134\\
39.84	0.00399782265625557\\
39.85	0.00399779908256081\\
39.86	0.00399777549860068\\
39.87	0.00399775190436877\\
39.88	0.00399772829985866\\
39.89	0.00399770468506396\\
39.9	0.00399768105997824\\
39.91	0.00399765742459508\\
39.92	0.00399763377890805\\
39.93	0.0039976101229107\\
39.94	0.00399758645659661\\
39.95	0.00399756277995931\\
39.96	0.00399753909299237\\
39.97	0.0039975153956893\\
39.98	0.00399749168804366\\
39.99	0.00399746797004896\\
40	0.00399744424169874\\
40.01	0.0039974205029865\\
};
\addplot [color=red,solid,forget plot]
  table[row sep=crcr]{%
40.01	0.0039974205029865\\
40.02	0.00399739675390575\\
40.03	0.00399737299445\\
40.04	0.00399734922461275\\
40.05	0.00399732544438748\\
40.06	0.0039973016537677\\
40.07	0.00399727785274686\\
40.08	0.00399725404131845\\
40.09	0.00399723021947593\\
40.1	0.00399720638721277\\
40.11	0.00399718254452242\\
40.12	0.00399715869139833\\
40.13	0.00399713482783394\\
40.14	0.00399711095382269\\
40.15	0.00399708706935802\\
40.16	0.00399706317443334\\
40.17	0.00399703926904208\\
40.18	0.00399701535317765\\
40.19	0.00399699142683346\\
40.2	0.0039969674900029\\
40.21	0.00399694354267937\\
40.22	0.00399691958485626\\
40.23	0.00399689561652695\\
40.24	0.00399687163768482\\
40.25	0.00399684764832324\\
40.26	0.00399682364843557\\
40.27	0.00399679963801517\\
40.28	0.00399677561705538\\
40.29	0.00399675158554957\\
40.3	0.00399672754349105\\
40.31	0.00399670349087318\\
40.32	0.00399667942768927\\
40.33	0.00399665535393265\\
40.34	0.00399663126959662\\
40.35	0.00399660717467451\\
40.36	0.0039965830691596\\
40.37	0.0039965589530452\\
40.38	0.0039965348263246\\
40.39	0.00399651068899107\\
40.4	0.0039964865410379\\
40.41	0.00399646238245836\\
40.42	0.0039964382132457\\
40.43	0.00399641403339319\\
40.44	0.00399638984289408\\
40.45	0.00399636564174162\\
40.46	0.00399634142992904\\
40.47	0.00399631720744958\\
40.48	0.00399629297429647\\
40.49	0.00399626873046291\\
40.5	0.00399624447594214\\
40.51	0.00399622021072735\\
40.52	0.00399619593481175\\
40.53	0.00399617164818853\\
40.54	0.00399614735085087\\
40.55	0.00399612304279198\\
40.56	0.003996098724005\\
40.57	0.00399607439448312\\
40.58	0.0039960500542195\\
40.59	0.00399602570320729\\
40.6	0.00399600134143964\\
40.61	0.0039959769689097\\
40.62	0.00399595258561061\\
40.63	0.00399592819153548\\
40.64	0.00399590378667745\\
40.65	0.00399587937102963\\
40.66	0.00399585494458514\\
40.67	0.00399583050733707\\
40.68	0.00399580605927853\\
40.69	0.0039957816004026\\
40.7	0.00399575713070236\\
40.71	0.0039957326501709\\
40.72	0.00399570815880128\\
40.73	0.00399568365658657\\
40.74	0.00399565914351983\\
40.75	0.0039956346195941\\
40.76	0.00399561008480243\\
40.77	0.00399558553913786\\
40.78	0.00399556098259341\\
40.79	0.00399553641516211\\
40.8	0.00399551183683697\\
40.81	0.003995487247611\\
40.82	0.00399546264747721\\
40.83	0.00399543803642859\\
40.84	0.00399541341445812\\
40.85	0.0039953887815588\\
40.86	0.00399536413772359\\
40.87	0.00399533948294546\\
40.88	0.00399531481721738\\
40.89	0.00399529014053228\\
40.9	0.00399526545288313\\
40.91	0.00399524075426286\\
40.92	0.00399521604466439\\
40.93	0.00399519132408066\\
40.94	0.00399516659250459\\
40.95	0.00399514184992908\\
40.96	0.00399511709634703\\
40.97	0.00399509233175134\\
40.98	0.0039950675561349\\
40.99	0.00399504276949059\\
41	0.00399501797181128\\
41.01	0.00399499316308984\\
41.02	0.00399496834331911\\
41.03	0.00399494351249197\\
41.04	0.00399491867060124\\
41.05	0.00399489381763976\\
41.06	0.00399486895360036\\
41.07	0.00399484407847586\\
41.08	0.00399481919225907\\
41.09	0.0039947942949428\\
41.1	0.00399476938651983\\
41.11	0.00399474446698296\\
41.12	0.00399471953632497\\
41.13	0.00399469459453863\\
41.14	0.00399466964161671\\
41.15	0.00399464467755196\\
41.16	0.00399461970233712\\
41.17	0.00399459471596494\\
41.18	0.00399456971842814\\
41.19	0.00399454470971946\\
41.2	0.00399451968983161\\
41.21	0.00399449465875728\\
41.22	0.00399446961648918\\
41.23	0.00399444456302\\
41.24	0.00399441949834242\\
41.25	0.0039943944224491\\
41.26	0.00399436933533272\\
41.27	0.00399434423698592\\
41.28	0.00399431912740135\\
41.29	0.00399429400657164\\
41.3	0.00399426887448943\\
41.31	0.00399424373114732\\
41.32	0.00399421857653793\\
41.33	0.00399419341065386\\
41.34	0.0039941682334877\\
41.35	0.00399414304503202\\
41.36	0.0039941178452794\\
41.37	0.00399409263422239\\
41.38	0.00399406741185356\\
41.39	0.00399404217816544\\
41.4	0.00399401693315056\\
41.41	0.00399399167680145\\
41.42	0.00399396640911062\\
41.43	0.00399394113007057\\
41.44	0.00399391583967378\\
41.45	0.00399389053791275\\
41.46	0.00399386522477994\\
41.47	0.00399383990026782\\
41.48	0.00399381456436884\\
41.49	0.00399378921707542\\
41.5	0.00399376385838001\\
41.51	0.00399373848827501\\
41.52	0.00399371310675284\\
41.53	0.0039936877138059\\
41.54	0.00399366230942655\\
41.55	0.00399363689360719\\
41.56	0.00399361146634017\\
41.57	0.00399358602761783\\
41.58	0.00399356057743253\\
41.59	0.00399353511577658\\
41.6	0.0039935096426423\\
41.61	0.00399348415802199\\
41.62	0.00399345866190794\\
41.63	0.00399343315429245\\
41.64	0.00399340763516775\\
41.65	0.00399338210452612\\
41.66	0.00399335656235979\\
41.67	0.00399333100866099\\
41.68	0.00399330544342193\\
41.69	0.00399327986663483\\
41.7	0.00399325427829186\\
41.71	0.0039932286783852\\
41.72	0.00399320306690702\\
41.73	0.00399317744384947\\
41.74	0.00399315180920468\\
41.75	0.00399312616296476\\
41.76	0.00399310050512184\\
41.77	0.003993074835668\\
41.78	0.00399304915459532\\
41.79	0.00399302346189587\\
41.8	0.0039929977575617\\
41.81	0.00399297204158485\\
41.82	0.00399294631395732\\
41.83	0.00399292057467113\\
41.84	0.00399289482371828\\
41.85	0.00399286906109073\\
41.86	0.00399284328678045\\
41.87	0.00399281750077938\\
41.88	0.00399279170307944\\
41.89	0.00399276589367256\\
41.9	0.00399274007255063\\
41.91	0.00399271423970552\\
41.92	0.00399268839512911\\
41.93	0.00399266253881325\\
41.94	0.00399263667074976\\
41.95	0.00399261079093045\\
41.96	0.00399258489934713\\
41.97	0.00399255899599158\\
41.98	0.00399253308085555\\
41.99	0.00399250715393081\\
42	0.00399248121520906\\
42.01	0.00399245526468203\\
42.02	0.0039924293023414\\
42.03	0.00399240332817885\\
42.04	0.00399237734218604\\
42.05	0.00399235134435461\\
42.06	0.00399232533467618\\
42.07	0.00399229931314233\\
42.08	0.00399227327974467\\
42.09	0.00399224723447476\\
42.1	0.00399222117732412\\
42.11	0.00399219510828431\\
42.12	0.00399216902734681\\
42.13	0.00399214293450312\\
42.14	0.00399211682974471\\
42.15	0.00399209071306301\\
42.16	0.00399206458444946\\
42.17	0.00399203844389548\\
42.18	0.00399201229139244\\
42.19	0.00399198612693172\\
42.2	0.00399195995050466\\
42.21	0.00399193376210259\\
42.22	0.00399190756171682\\
42.23	0.00399188134933863\\
42.24	0.0039918551249593\\
42.25	0.00399182888857007\\
42.26	0.00399180264016216\\
42.27	0.00399177637972677\\
42.28	0.0039917501072551\\
42.29	0.00399172382273831\\
42.3	0.00399169752616753\\
42.31	0.00399167121753389\\
42.32	0.0039916448968285\\
42.33	0.00399161856404242\\
42.34	0.00399159221916673\\
42.35	0.00399156586219244\\
42.36	0.0039915394931106\\
42.37	0.00399151311191218\\
42.38	0.00399148671858817\\
42.39	0.00399146031312953\\
42.4	0.00399143389552718\\
42.41	0.00399140746577203\\
42.42	0.00399138102385498\\
42.43	0.00399135456976691\\
42.44	0.00399132810349867\\
42.45	0.00399130162504109\\
42.46	0.00399127513438497\\
42.47	0.00399124863152112\\
42.48	0.00399122211644031\\
42.49	0.00399119558913328\\
42.5	0.00399116904959078\\
42.51	0.00399114249780351\\
42.52	0.00399111593376218\\
42.53	0.00399108935745745\\
42.54	0.00399106276887999\\
42.55	0.00399103616802044\\
42.56	0.00399100955486942\\
42.57	0.00399098292941754\\
42.58	0.00399095629165538\\
42.59	0.00399092964157353\\
42.6	0.00399090297916251\\
42.61	0.00399087630441289\\
42.62	0.00399084961731518\\
42.63	0.0039908229178599\\
42.64	0.00399079620603752\\
42.65	0.00399076948183854\\
42.66	0.00399074274525343\\
42.67	0.00399071599627262\\
42.68	0.00399068923488656\\
42.69	0.00399066246108568\\
42.7	0.0039906356748604\\
42.71	0.00399060887620112\\
42.72	0.00399058206509822\\
42.73	0.00399055524154211\\
42.74	0.00399052840552315\\
42.75	0.00399050155703171\\
42.76	0.00399047469605815\\
42.77	0.00399044782259283\\
42.78	0.00399042093662609\\
42.79	0.00399039403814827\\
42.8	0.00399036712714971\\
42.81	0.00399034020362074\\
42.82	0.00399031326755168\\
42.83	0.00399028631893288\\
42.84	0.00399025935775465\\
42.85	0.00399023238400731\\
42.86	0.0039902053976812\\
42.87	0.00399017839876664\\
42.88	0.00399015138725396\\
42.89	0.00399012436313349\\
42.9	0.00399009732639556\\
42.91	0.00399007027703051\\
42.92	0.00399004321502869\\
42.93	0.00399001614038045\\
42.94	0.00398998905307616\\
42.95	0.00398996195310616\\
42.96	0.00398993484046085\\
42.97	0.0039899077151306\\
42.98	0.00398988057710581\\
42.99	0.00398985342637689\\
43	0.00398982626293425\\
43.01	0.00398979908676834\\
43.02	0.00398977189786958\\
43.03	0.00398974469622845\\
43.04	0.00398971748183541\\
43.05	0.00398969025468096\\
43.06	0.00398966301475559\\
43.07	0.00398963576204983\\
43.08	0.00398960849655421\\
43.09	0.00398958121825929\\
43.1	0.00398955392715562\\
43.11	0.00398952662323381\\
43.12	0.00398949930648444\\
43.13	0.00398947197689813\\
43.14	0.00398944463446552\\
43.15	0.00398941727917725\\
43.16	0.00398938991102399\\
43.17	0.0039893625299964\\
43.18	0.00398933513608517\\
43.19	0.00398930772928099\\
43.2	0.00398928030957459\\
43.21	0.00398925287695665\\
43.22	0.00398922543141791\\
43.23	0.00398919797294909\\
43.24	0.00398917050154089\\
43.25	0.00398914301718404\\
43.26	0.00398911551986926\\
43.27	0.00398908800958723\\
43.28	0.00398906048632864\\
43.29	0.00398903295008419\\
43.3	0.00398900540084452\\
43.31	0.0039889778386003\\
43.32	0.00398895026334217\\
43.33	0.00398892267506078\\
43.34	0.00398889507374673\\
43.35	0.00398886745939066\\
43.36	0.00398883983198314\\
43.37	0.00398881219151479\\
43.38	0.00398878453797618\\
43.39	0.00398875687135789\\
43.4	0.00398872919165045\\
43.41	0.00398870149884444\\
43.42	0.00398867379293038\\
43.43	0.0039886460738988\\
43.44	0.00398861834174021\\
43.45	0.00398859059644512\\
43.46	0.00398856283800402\\
43.47	0.0039885350664074\\
43.48	0.00398850728164572\\
43.49	0.00398847948370944\\
43.5	0.00398845167258902\\
43.51	0.00398842384827488\\
43.52	0.00398839601075747\\
43.53	0.00398836816002718\\
43.54	0.00398834029607444\\
43.55	0.00398831241888962\\
43.56	0.00398828452846312\\
43.57	0.0039882566247853\\
43.58	0.00398822870784652\\
43.59	0.00398820077763713\\
43.6	0.00398817283414747\\
43.61	0.00398814487736787\\
43.62	0.00398811690728864\\
43.63	0.00398808892390009\\
43.64	0.0039880609271925\\
43.65	0.00398803291715616\\
43.66	0.00398800489378133\\
43.67	0.00398797685705828\\
43.68	0.00398794880697726\\
43.69	0.00398792074352849\\
43.7	0.00398789266670221\\
43.71	0.00398786457648862\\
43.72	0.00398783647287793\\
43.73	0.00398780835586034\\
43.74	0.00398778022542601\\
43.75	0.00398775208156511\\
43.76	0.00398772392426781\\
43.77	0.00398769575352425\\
43.78	0.00398766756932455\\
43.79	0.00398763937165885\\
43.8	0.00398761116051724\\
43.81	0.00398758293588984\\
43.82	0.00398755469776673\\
43.83	0.00398752644613798\\
43.84	0.00398749818099365\\
43.85	0.0039874699023238\\
43.86	0.00398744161011848\\
43.87	0.00398741330436771\\
43.88	0.00398738498506152\\
43.89	0.00398735665218989\\
43.9	0.00398732830574283\\
43.91	0.00398729994571032\\
43.92	0.00398727157208235\\
43.93	0.00398724318484885\\
43.94	0.00398721478399979\\
43.95	0.00398718636952509\\
43.96	0.00398715794141469\\
43.97	0.0039871294996585\\
43.98	0.00398710104424641\\
43.99	0.00398707257516832\\
44	0.0039870440924141\\
44.01	0.00398701559597362\\
44.02	0.00398698708583674\\
44.03	0.0039869585619933\\
44.04	0.00398693002443313\\
44.05	0.00398690147314604\\
44.06	0.00398687290812185\\
44.07	0.00398684432935035\\
44.08	0.00398681573682132\\
44.09	0.00398678713052453\\
44.1	0.00398675851044975\\
44.11	0.00398672987658672\\
44.12	0.00398670122892518\\
44.13	0.00398667256745485\\
44.14	0.00398664389216545\\
44.15	0.00398661520304667\\
44.16	0.0039865865000882\\
44.17	0.00398655778327972\\
44.18	0.0039865290526109\\
44.19	0.00398650030807138\\
44.2	0.00398647154965081\\
44.21	0.00398644277733881\\
44.22	0.00398641399112501\\
44.23	0.003986385190999\\
44.24	0.00398635637695038\\
44.25	0.00398632754896873\\
44.26	0.00398629870704362\\
44.27	0.0039862698511646\\
44.28	0.00398624098132122\\
44.29	0.00398621209750302\\
44.3	0.00398618319969951\\
44.31	0.0039861542879002\\
44.32	0.00398612536209458\\
44.33	0.00398609642227215\\
44.34	0.00398606746842236\\
44.35	0.0039860385005347\\
44.36	0.00398600951859859\\
44.37	0.00398598052260348\\
44.38	0.0039859515125388\\
44.39	0.00398592248839394\\
44.4	0.00398589345015832\\
44.41	0.00398586439782132\\
44.42	0.00398583533137231\\
44.43	0.00398580625080066\\
44.44	0.00398577715609572\\
44.45	0.00398574804724683\\
44.46	0.00398571892424331\\
44.47	0.00398568978707448\\
44.48	0.00398566063572965\\
44.49	0.00398563147019809\\
44.5	0.00398560229046909\\
44.51	0.00398557309653192\\
44.52	0.00398554388837583\\
44.53	0.00398551466599006\\
44.54	0.00398548542936384\\
44.55	0.00398545617848639\\
44.56	0.00398542691334691\\
44.57	0.0039853976339346\\
44.58	0.00398536834023863\\
44.59	0.00398533903224817\\
44.6	0.00398530970995239\\
44.61	0.00398528037334042\\
44.62	0.0039852510224014\\
44.63	0.00398522165712444\\
44.64	0.00398519227749865\\
44.65	0.00398516288351314\\
44.66	0.00398513347515697\\
44.67	0.00398510405241923\\
44.68	0.00398507461528897\\
44.69	0.00398504516375524\\
44.7	0.00398501569780707\\
44.71	0.00398498621743349\\
44.72	0.0039849567226235\\
44.73	0.0039849272133661\\
44.74	0.00398489768965028\\
44.75	0.00398486815146502\\
44.76	0.00398483859879926\\
44.77	0.00398480903164197\\
44.78	0.00398477944998208\\
44.79	0.00398474985380851\\
44.8	0.00398472024311018\\
44.81	0.00398469061787599\\
44.82	0.00398466097809482\\
44.83	0.00398463132375556\\
44.84	0.00398460165484706\\
44.85	0.00398457197135819\\
44.86	0.00398454227327778\\
44.87	0.00398451256059465\\
44.88	0.00398448283329763\\
44.89	0.00398445309137552\\
44.9	0.00398442333481711\\
44.91	0.00398439356361118\\
44.92	0.0039843637777465\\
44.93	0.00398433397721182\\
44.94	0.00398430416199589\\
44.95	0.00398427433208745\\
44.96	0.0039842444874752\\
44.97	0.00398421462814787\\
44.98	0.00398418475409414\\
44.99	0.0039841548653027\\
45	0.00398412496176223\\
45.01	0.00398409504346138\\
45.02	0.0039840651103888\\
45.03	0.00398403516253313\\
45.04	0.003984005199883\\
45.05	0.00398397522242702\\
45.06	0.00398394523015378\\
45.07	0.00398391522305189\\
45.08	0.00398388520110992\\
45.09	0.00398385516431643\\
45.1	0.00398382511265998\\
45.11	0.00398379504612912\\
45.12	0.00398376496471237\\
45.13	0.00398373486839826\\
45.14	0.0039837047571753\\
45.15	0.00398367463103197\\
45.16	0.00398364448995678\\
45.17	0.00398361433393819\\
45.18	0.00398358416296466\\
45.19	0.00398355397702465\\
45.2	0.0039835237761066\\
45.21	0.00398349356019893\\
45.22	0.00398346332929006\\
45.23	0.00398343308336841\\
45.24	0.00398340282242236\\
45.25	0.00398337254644029\\
45.26	0.00398334225541059\\
45.27	0.00398331194932161\\
45.28	0.00398328162816171\\
45.29	0.00398325129191921\\
45.3	0.00398322094058247\\
45.31	0.00398319057413978\\
45.32	0.00398316019257946\\
45.33	0.0039831297958898\\
45.34	0.00398309938405909\\
45.35	0.0039830689570756\\
45.36	0.0039830385149276\\
45.37	0.00398300805760335\\
45.38	0.00398297758509107\\
45.39	0.00398294709737902\\
45.4	0.0039829165944554\\
45.41	0.00398288607630844\\
45.42	0.00398285554292632\\
45.43	0.00398282499429725\\
45.44	0.00398279443040941\\
45.45	0.00398276385125096\\
45.46	0.00398273325681006\\
45.47	0.00398270264707487\\
45.48	0.00398267202203353\\
45.49	0.00398264138167417\\
45.5	0.00398261072598491\\
45.51	0.00398258005495386\\
45.52	0.00398254936856912\\
45.53	0.00398251866681879\\
45.54	0.00398248794969094\\
45.55	0.00398245721717366\\
45.56	0.00398242646925501\\
45.57	0.00398239570592303\\
45.58	0.00398236492716579\\
45.59	0.0039823341329713\\
45.6	0.00398230332332761\\
45.61	0.00398227249822273\\
45.62	0.00398224165764467\\
45.63	0.00398221080158144\\
45.64	0.00398217993002102\\
45.65	0.0039821490429514\\
45.66	0.00398211814036056\\
45.67	0.00398208722223646\\
45.68	0.00398205628856707\\
45.69	0.00398202533934033\\
45.7	0.00398199437454419\\
45.71	0.00398196339416659\\
45.72	0.00398193239819545\\
45.73	0.00398190138661869\\
45.74	0.00398187035942423\\
45.75	0.00398183931659997\\
45.76	0.00398180825813382\\
45.77	0.00398177718401365\\
45.78	0.00398174609422737\\
45.79	0.00398171498876284\\
45.8	0.00398168386760793\\
45.81	0.00398165273075051\\
45.82	0.00398162157817844\\
45.83	0.00398159040987957\\
45.84	0.00398155922584174\\
45.85	0.0039815280260528\\
45.86	0.00398149681050058\\
45.87	0.0039814655791729\\
45.88	0.00398143433205758\\
45.89	0.00398140306914246\\
45.9	0.00398137179041533\\
45.91	0.003981340495864\\
45.92	0.00398130918547628\\
45.93	0.00398127785923996\\
45.94	0.00398124651714284\\
45.95	0.0039812151591727\\
45.96	0.00398118378531733\\
45.97	0.00398115239556451\\
45.98	0.00398112098990202\\
45.99	0.00398108956831762\\
46	0.00398105813079909\\
46.01	0.0039810266773342\\
46.02	0.00398099520791071\\
46.03	0.00398096372251637\\
46.04	0.00398093222113896\\
46.05	0.00398090070376621\\
46.06	0.0039808691703859\\
46.07	0.00398083762098577\\
46.08	0.00398080605555356\\
46.09	0.00398077447407704\\
46.1	0.00398074287654394\\
46.11	0.00398071126294201\\
46.12	0.00398067963325901\\
46.13	0.00398064798748267\\
46.14	0.00398061632560075\\
46.15	0.00398058464760099\\
46.16	0.00398055295347112\\
46.17	0.00398052124319891\\
46.18	0.0039804895167721\\
46.19	0.00398045777417844\\
46.2	0.00398042601540567\\
46.21	0.00398039424044156\\
46.22	0.00398036244927385\\
46.23	0.00398033064189031\\
46.24	0.00398029881827868\\
46.25	0.00398026697842674\\
46.26	0.00398023512232225\\
46.27	0.00398020324995298\\
46.28	0.0039801713613067\\
46.29	0.0039801394563712\\
46.3	0.00398010753513424\\
46.31	0.00398007559758362\\
46.32	0.00398004364370714\\
46.33	0.00398001167349259\\
46.34	0.00397997968692778\\
46.35	0.00397994768400052\\
46.36	0.00397991566469862\\
46.37	0.0039798836290099\\
46.38	0.00397985157692222\\
46.39	0.00397981950842339\\
46.4	0.00397978742350128\\
46.41	0.00397975532214374\\
46.42	0.00397972320433864\\
46.43	0.00397969107007385\\
46.44	0.00397965891933727\\
46.45	0.00397962675211679\\
46.46	0.00397959456840031\\
46.47	0.00397956236817577\\
46.48	0.00397953015143108\\
46.49	0.0039794979181542\\
46.5	0.00397946566833309\\
46.51	0.00397943340195571\\
46.52	0.00397940111901004\\
46.53	0.0039793688194841\\
46.54	0.00397933650336588\\
46.55	0.00397930417064343\\
46.56	0.00397927182130478\\
46.57	0.00397923945533801\\
46.58	0.00397920707273118\\
46.59	0.00397917467347239\\
46.6	0.00397914225754977\\
46.61	0.00397910982495143\\
46.62	0.00397907737566554\\
46.63	0.00397904490968027\\
46.64	0.00397901242698381\\
46.65	0.00397897992756437\\
46.66	0.00397894741141019\\
46.67	0.00397891487850954\\
46.68	0.00397888232885069\\
46.69	0.00397884976242194\\
46.7	0.00397881717921163\\
46.71	0.00397878457920811\\
46.72	0.00397875196239977\\
46.73	0.00397871932877501\\
46.74	0.00397868667832225\\
46.75	0.00397865401102998\\
46.76	0.00397862132688667\\
46.77	0.00397858862588085\\
46.78	0.00397855590800106\\
46.79	0.00397852317323589\\
46.8	0.00397849042157396\\
46.81	0.0039784576530039\\
46.82	0.00397842486751439\\
46.83	0.00397839206509415\\
46.84	0.00397835924573193\\
46.85	0.00397832640941651\\
46.86	0.0039782935561367\\
46.87	0.00397826068588136\\
46.88	0.0039782277986394\\
46.89	0.00397819489439974\\
46.9	0.00397816197315136\\
46.91	0.00397812903488327\\
46.92	0.00397809607958454\\
46.93	0.00397806310724426\\
46.94	0.00397803011785158\\
46.95	0.0039779971113957\\
46.96	0.00397796408786583\\
46.97	0.00397793104725128\\
46.98	0.00397789798954136\\
46.99	0.00397786491472546\\
47	0.00397783182279301\\
47.01	0.00397779871373349\\
47.02	0.00397776558753642\\
47.03	0.0039777324441914\\
47.04	0.00397769928368807\\
47.05	0.00397766610601612\\
47.06	0.0039776329111653\\
47.07	0.00397759969912542\\
47.08	0.00397756646988636\\
47.09	0.00397753322343804\\
47.1	0.00397749995977045\\
47.11	0.00397746667887365\\
47.12	0.00397743338073775\\
47.13	0.00397740006535293\\
47.14	0.00397736673270945\\
47.15	0.00397733338279761\\
47.16	0.00397730001560781\\
47.17	0.0039772666311305\\
47.18	0.0039772332293562\\
47.19	0.00397719981027551\\
47.2	0.00397716637387911\\
47.21	0.00397713292015776\\
47.22	0.00397709944910227\\
47.23	0.00397706596070354\\
47.24	0.00397703245495258\\
47.25	0.00397699893184045\\
47.26	0.00397696539135829\\
47.27	0.00397693183349733\\
47.28	0.00397689825824891\\
47.29	0.00397686466560443\\
47.3	0.00397683105555539\\
47.31	0.00397679742809336\\
47.32	0.00397676378321005\\
47.33	0.00397673012089722\\
47.34	0.00397669644114674\\
47.35	0.00397666274395058\\
47.36	0.00397662902930081\\
47.37	0.00397659529718959\\
47.38	0.0039765615476092\\
47.39	0.003976527780552\\
47.4	0.00397649399601048\\
47.41	0.00397646019397722\\
47.42	0.00397642637444493\\
47.43	0.0039763925374064\\
47.44	0.00397635868285457\\
47.45	0.00397632481078247\\
47.46	0.00397629092118324\\
47.47	0.00397625701405017\\
47.48	0.00397622308937664\\
47.49	0.00397618914715617\\
47.5	0.0039761551873824\\
47.51	0.00397612121004909\\
47.52	0.00397608721515014\\
47.53	0.00397605320267958\\
47.54	0.00397601917263155\\
47.55	0.00397598512500036\\
47.56	0.00397595105978044\\
47.57	0.00397591697696633\\
47.58	0.00397588287655277\\
47.59	0.00397584875853459\\
47.6	0.00397581462290679\\
47.61	0.00397578046966452\\
47.62	0.00397574629880306\\
47.63	0.00397571211031787\\
47.64	0.00397567790420453\\
47.65	0.0039756436804588\\
47.66	0.0039756094390766\\
47.67	0.00397557518005399\\
47.68	0.0039755409033872\\
47.69	0.00397550660907264\\
47.7	0.00397547229710688\\
47.71	0.00397543796748665\\
47.72	0.00397540362020885\\
47.73	0.00397536925527057\\
47.74	0.00397533487266907\\
47.75	0.00397530047240178\\
47.76	0.00397526605446633\\
47.77	0.00397523161886051\\
47.78	0.0039751971655823\\
47.79	0.0039751626946299\\
47.8	0.00397512820600167\\
47.81	0.00397509369969615\\
47.82	0.00397505917571212\\
47.83	0.00397502463404851\\
47.84	0.00397499007470449\\
47.85	0.00397495549767942\\
47.86	0.00397492090297285\\
47.87	0.00397488629058455\\
47.88	0.0039748516605145\\
47.89	0.0039748170127629\\
47.9	0.00397478234733015\\
47.91	0.00397474766421687\\
47.92	0.00397471296342391\\
47.93	0.00397467824495233\\
47.94	0.00397464350880342\\
47.95	0.00397460875497871\\
47.96	0.00397457398347993\\
47.97	0.00397453919430907\\
47.98	0.00397450438746834\\
47.99	0.00397446956296018\\
48	0.00397443472078729\\
48.01	0.00397439986095259\\
48.02	0.00397436498345926\\
48.03	0.00397433008831071\\
48.04	0.00397429517551061\\
48.05	0.00397426024506288\\
48.06	0.00397422529697168\\
48.07	0.00397419033124144\\
48.08	0.00397415534787683\\
48.09	0.0039741203468828\\
48.1	0.00397408532826455\\
48.11	0.00397405029202753\\
48.12	0.00397401523817748\\
48.13	0.00397398016672039\\
48.14	0.00397394507766253\\
48.15	0.00397390997101041\\
48.16	0.00397387484677086\\
48.17	0.00397383970495096\\
48.18	0.00397380454555806\\
48.19	0.00397376936859979\\
48.2	0.00397373417408408\\
48.21	0.00397369896201911\\
48.22	0.00397366373241335\\
48.23	0.00397362848527558\\
48.24	0.00397359322061483\\
48.25	0.00397355793844044\\
48.26	0.00397352263876203\\
48.27	0.00397348732158949\\
48.28	0.00397345198693304\\
48.29	0.00397341663480314\\
48.3	0.00397338126521059\\
48.31	0.00397334587816644\\
48.32	0.00397331047368206\\
48.33	0.0039732750517691\\
48.34	0.00397323961243951\\
48.35	0.00397320415570552\\
48.36	0.00397316868157966\\
48.37	0.00397313319007477\\
48.38	0.00397309768120395\\
48.39	0.00397306215498061\\
48.4	0.00397302661141847\\
48.41	0.0039729910505315\\
48.42	0.003972955472334\\
48.43	0.00397291987684053\\
48.44	0.00397288426406596\\
48.45	0.00397284863402543\\
48.46	0.00397281298673438\\
48.47	0.00397277732220853\\
48.48	0.00397274164046386\\
48.49	0.00397270594151667\\
48.5	0.00397267022538351\\
48.51	0.00397263449208121\\
48.52	0.00397259874162687\\
48.53	0.00397256297403787\\
48.54	0.00397252718933184\\
48.55	0.00397249138752669\\
48.56	0.00397245556864057\\
48.57	0.0039724197326919\\
48.58	0.00397238387969935\\
48.59	0.00397234800968183\\
48.6	0.00397231212265848\\
48.61	0.00397227621864871\\
48.62	0.00397224029767213\\
48.63	0.00397220435974858\\
48.64	0.00397216840489813\\
48.65	0.00397213243314105\\
48.66	0.00397209644449782\\
48.67	0.00397206043898913\\
48.68	0.00397202441663585\\
48.69	0.00397198837745903\\
48.7	0.0039719523214799\\
48.71	0.00397191624871986\\
48.72	0.00397188015920048\\
48.73	0.00397184405294344\\
48.74	0.00397180792997059\\
48.75	0.00397177179030391\\
48.76	0.00397173563396548\\
48.77	0.00397169946097749\\
48.78	0.00397166327136225\\
48.79	0.00397162706514211\\
48.8	0.00397159084233952\\
48.81	0.00397155460297699\\
48.82	0.00397151834707707\\
48.83	0.00397148207466231\\
48.84	0.00397144578575532\\
48.85	0.00397140948037869\\
48.86	0.00397137315855499\\
48.87	0.00397133682030677\\
48.88	0.00397130046565651\\
48.89	0.00397126409462665\\
48.9	0.00397122770723955\\
48.91	0.00397119130351742\\
48.92	0.00397115488348241\\
48.93	0.00397111844715648\\
48.94	0.00397108199456146\\
48.95	0.00397104552571899\\
48.96	0.00397100904065048\\
48.97	0.00397097253937715\\
48.98	0.00397093602191995\\
48.99	0.00397089948829956\\
49	0.00397086293853636\\
49.01	0.0039708263726504\\
49.02	0.00397078979066139\\
49.03	0.00397075319258867\\
49.04	0.00397071657845115\\
49.05	0.00397067994826734\\
49.06	0.00397064330205526\\
49.07	0.00397060663983246\\
49.08	0.00397056996161597\\
49.09	0.00397053326742225\\
49.1	0.00397049655726718\\
49.11	0.00397045983116605\\
49.12	0.00397042308913346\\
49.13	0.00397038633118335\\
49.14	0.00397034955732894\\
49.15	0.00397031276758269\\
49.16	0.00397027596195626\\
49.17	0.00397023914046049\\
49.18	0.00397020230310536\\
49.19	0.00397016544989991\\
49.2	0.00397012858085228\\
49.21	0.00397009169596958\\
49.22	0.00397005479525793\\
49.23	0.00397001787872235\\
49.24	0.00396998094636675\\
49.25	0.0039699439981939\\
49.26	0.00396990703420536\\
49.27	0.00396987005440143\\
49.28	0.00396983305878115\\
49.29	0.0039697960473422\\
49.3	0.00396975902008087\\
49.31	0.00396972197699203\\
49.32	0.00396968491806907\\
49.33	0.00396964784330385\\
49.34	0.00396961075268665\\
49.35	0.00396957364620612\\
49.36	0.00396953652384924\\
49.37	0.00396949938560125\\
49.38	0.00396946223144564\\
49.39	0.00396942506136404\\
49.4	0.0039693878753362\\
49.41	0.00396935067333997\\
49.42	0.0039693134553512\\
49.43	0.00396927622134368\\
49.44	0.00396923897128915\\
49.45	0.0039692017051572\\
49.46	0.00396916442291524\\
49.47	0.00396912712452842\\
49.48	0.00396908980995963\\
49.49	0.0039690524791694\\
49.5	0.00396901513211588\\
49.51	0.0039689777687548\\
49.52	0.00396894038903937\\
49.53	0.00396890299292031\\
49.54	0.00396886558034574\\
49.55	0.00396882815126118\\
49.56	0.00396879070560948\\
49.57	0.00396875324333078\\
49.58	0.00396871576436251\\
49.59	0.0039686782686393\\
49.6	0.00396864075609296\\
49.61	0.00396860322665249\\
49.62	0.003968565680244\\
49.63	0.00396852811679069\\
49.64	0.00396849053621285\\
49.65	0.00396845293842784\\
49.66	0.00396841532335005\\
49.67	0.0039683776908909\\
49.68	0.00396834004095884\\
49.69	0.00396830237345934\\
49.7	0.0039682646882949\\
49.71	0.00396822698536503\\
49.72	0.00396818926456629\\
49.73	0.00396815152579234\\
49.74	0.00396811376893388\\
49.75	0.00396807599387876\\
49.76	0.00396803820051199\\
49.77	0.00396800038871581\\
49.78	0.0039679625583697\\
49.79	0.0039679247093505\\
49.8	0.00396788684153247\\
49.81	0.00396784895478736\\
49.82	0.00396781104898452\\
49.83	0.00396777312399101\\
49.84	0.00396773517967174\\
49.85	0.00396769721588956\\
49.86	0.00396765923250545\\
49.87	0.00396762122937867\\
49.88	0.00396758320636693\\
49.89	0.00396754516332659\\
49.9	0.00396750710011289\\
49.91	0.00396746901658016\\
49.92	0.00396743091258207\\
49.93	0.00396739278797191\\
49.94	0.00396735464260288\\
49.95	0.00396731647632841\\
49.96	0.00396727828900248\\
49.97	0.00396724008048001\\
49.98	0.00396720185061723\\
49.99	0.00396716359927209\\
50	0.00396712532630475\\
50.01	0.003967087031578\\
50.02	0.00396704871495782\\
50.03	0.00396701037631391\\
50.04	0.00396697201552025\\
50.05	0.00396693363245573\\
50.06	0.00396689522700484\\
50.07	0.00396685679905832\\
50.08	0.00396681834851393\\
50.09	0.00396677987527724\\
50.1	0.00396674137926246\\
50.11	0.00396670286039334\\
50.12	0.00396666431860408\\
50.13	0.00396662575384037\\
50.14	0.00396658716606041\\
50.15	0.00396654855523605\\
50.16	0.00396650992134251\\
50.17	0.00396647126435496\\
50.18	0.00396643258424857\\
50.19	0.00396639388099848\\
50.2	0.00396635515457982\\
50.21	0.00396631640496769\\
50.22	0.00396627763213716\\
50.23	0.00396623883606329\\
50.24	0.00396620001672114\\
50.25	0.00396616117408574\\
50.26	0.00396612230813208\\
50.27	0.00396608341883518\\
50.28	0.00396604450617001\\
50.29	0.00396600557011155\\
50.3	0.00396596661063474\\
50.31	0.00396592762771453\\
50.32	0.00396588862132583\\
50.33	0.00396584959144357\\
50.34	0.00396581053804264\\
50.35	0.00396577146109792\\
50.36	0.00396573236058428\\
50.37	0.00396569323647658\\
50.38	0.00396565408874965\\
50.39	0.0039656149173783\\
50.4	0.00396557572233733\\
50.41	0.00396553650360152\\
50.42	0.00396549726114561\\
50.43	0.00396545799494433\\
50.44	0.00396541870497238\\
50.45	0.00396537939120441\\
50.46	0.00396534005361506\\
50.47	0.00396530069217891\\
50.48	0.0039652613068705\\
50.49	0.00396522189766431\\
50.5	0.0039651824645348\\
50.51	0.00396514300745635\\
50.52	0.00396510352640331\\
50.53	0.00396506402134996\\
50.54	0.00396502449227056\\
50.55	0.00396498493913929\\
50.56	0.0039649453619303\\
50.57	0.00396490576061767\\
50.58	0.00396486613517545\\
50.59	0.00396482648557763\\
50.6	0.00396478681179813\\
50.61	0.00396474711381085\\
50.62	0.00396470739158962\\
50.63	0.00396466764510823\\
50.64	0.00396462787434039\\
50.65	0.00396458807925978\\
50.66	0.00396454825984002\\
50.67	0.00396450841605469\\
50.68	0.00396446854787729\\
50.69	0.00396442865528129\\
50.7	0.00396438873824009\\
50.71	0.00396434879672705\\
50.72	0.00396430883071545\\
50.73	0.00396426884017855\\
50.74	0.00396422882508952\\
50.75	0.00396418878542151\\
50.76	0.00396414872114757\\
50.77	0.00396410863224073\\
50.78	0.00396406851867395\\
50.79	0.00396402838042013\\
50.8	0.00396398821745212\\
50.81	0.0039639480297427\\
50.82	0.00396390781726461\\
50.83	0.00396386757999052\\
50.84	0.00396382731789304\\
50.85	0.00396378703094472\\
50.86	0.00396374671911807\\
50.87	0.00396370638238551\\
50.88	0.00396366602071942\\
50.89	0.00396362563409211\\
50.9	0.00396358522247584\\
50.91	0.00396354478584279\\
50.92	0.00396350432416511\\
50.93	0.00396346383741485\\
50.94	0.00396342332556402\\
50.95	0.00396338278858456\\
50.96	0.00396334222644836\\
50.97	0.00396330163912723\\
50.98	0.00396326102659293\\
50.99	0.00396322038881713\\
51	0.00396317972577146\\
51.01	0.00396313903742749\\
51.02	0.00396309832375669\\
51.03	0.00396305758473051\\
51.04	0.00396301682032029\\
51.05	0.00396297603049732\\
51.06	0.00396293521523284\\
51.07	0.003962894374498\\
51.08	0.00396285350826388\\
51.09	0.00396281261650151\\
51.1	0.00396277169918183\\
51.11	0.00396273075627573\\
51.12	0.00396268978775401\\
51.13	0.00396264879358741\\
51.14	0.0039626077737466\\
51.15	0.00396256672820217\\
51.16	0.00396252565692465\\
51.17	0.00396248455988448\\
51.18	0.00396244343705205\\
51.19	0.00396240228839764\\
51.2	0.00396236111389151\\
51.21	0.00396231991350378\\
51.22	0.00396227868720454\\
51.23	0.00396223743496379\\
51.24	0.00396219615675146\\
51.25	0.00396215485253738\\
51.26	0.00396211352229133\\
51.27	0.00396207216598299\\
51.28	0.00396203078358198\\
51.29	0.00396198937505782\\
51.3	0.00396194794037996\\
51.31	0.00396190647951778\\
51.32	0.00396186499244055\\
51.33	0.00396182347911748\\
51.34	0.00396178193951769\\
51.35	0.00396174037361021\\
51.36	0.003961698781364\\
51.37	0.00396165716274793\\
51.38	0.00396161551773076\\
51.39	0.00396157384628121\\
51.4	0.00396153214836786\\
51.41	0.00396149042395924\\
51.42	0.00396144867302378\\
51.43	0.00396140689552982\\
51.44	0.0039613650914456\\
51.45	0.00396132326073927\\
51.46	0.00396128140337892\\
51.47	0.0039612395193325\\
51.48	0.00396119760856789\\
51.49	0.00396115567105287\\
51.5	0.00396111370675515\\
51.51	0.00396107171564229\\
51.52	0.0039610296976818\\
51.53	0.00396098765284108\\
51.54	0.00396094558108742\\
51.55	0.00396090348238801\\
51.56	0.00396086135670996\\
51.57	0.00396081920402025\\
51.58	0.00396077702428578\\
51.59	0.00396073481747333\\
51.6	0.00396069258354958\\
51.61	0.00396065032248112\\
51.62	0.0039606080342344\\
51.63	0.0039605657187758\\
51.64	0.00396052337607156\\
51.65	0.00396048100608782\\
51.66	0.00396043860879063\\
51.67	0.00396039618414589\\
51.68	0.00396035373211941\\
51.69	0.00396031125267688\\
51.7	0.00396026874578389\\
51.71	0.00396022621140587\\
51.72	0.00396018364950819\\
51.73	0.00396014106005606\\
51.74	0.00396009844301457\\
51.75	0.00396005579834872\\
51.76	0.00396001312602334\\
51.77	0.00395997042600319\\
51.78	0.00395992769825286\\
51.79	0.00395988494273683\\
51.8	0.00395984215941945\\
51.81	0.00395979934826496\\
51.82	0.00395975650923741\\
51.83	0.0039597136423008\\
51.84	0.00395967074741893\\
51.85	0.0039596278245555\\
51.86	0.00395958487367405\\
51.87	0.00395954189473799\\
51.88	0.00395949888771062\\
51.89	0.00395945585255504\\
51.9	0.00395941278923426\\
51.91	0.00395936969771111\\
51.92	0.00395932657794829\\
51.93	0.00395928342990836\\
51.94	0.00395924025355371\\
51.95	0.00395919704884659\\
51.96	0.00395915381574909\\
51.97	0.00395911055422316\\
51.98	0.00395906726423059\\
51.99	0.00395902394573298\\
52	0.00395898059869182\\
52.01	0.00395893722306839\\
52.02	0.00395889381882385\\
52.03	0.00395885038591916\\
52.04	0.00395880692431513\\
52.05	0.00395876343397238\\
52.06	0.00395871991485139\\
52.07	0.00395867636691243\\
52.08	0.00395863279011562\\
52.09	0.00395858918442088\\
52.1	0.00395854554978799\\
52.11	0.00395850188617648\\
52.12	0.00395845819354577\\
52.13	0.00395841447185503\\
52.14	0.00395837072106327\\
52.15	0.00395832694112931\\
52.16	0.00395828313201176\\
52.17	0.00395823929366905\\
52.18	0.0039581954260594\\
52.19	0.00395815152914083\\
52.2	0.00395810760287115\\
52.21	0.00395806364720797\\
52.22	0.00395801966210869\\
52.23	0.00395797564753049\\
52.24	0.00395793160343035\\
52.25	0.00395788752976501\\
52.26	0.003957843426491\\
52.27	0.00395779929356463\\
52.28	0.00395775513094199\\
52.29	0.00395771093857891\\
52.3	0.00395766671643101\\
52.31	0.00395762246445368\\
52.32	0.00395757818260206\\
52.33	0.00395753387083104\\
52.34	0.00395748952909528\\
52.35	0.00395744515734918\\
52.36	0.0039574007555469\\
52.37	0.00395735632364232\\
52.38	0.0039573118615891\\
52.39	0.00395726736934059\\
52.4	0.00395722284684991\\
52.41	0.0039571782940699\\
52.42	0.00395713371095312\\
52.43	0.00395708909745187\\
52.44	0.00395704445351813\\
52.45	0.00395699977910364\\
52.46	0.00395695507415983\\
52.47	0.00395691033863783\\
52.48	0.00395686557248849\\
52.49	0.00395682077566235\\
52.5	0.00395677594810963\\
52.51	0.00395673108978027\\
52.52	0.00395668620062387\\
52.53	0.00395664128058972\\
52.54	0.0039565963296268\\
52.55	0.00395655134768373\\
52.56	0.00395650633470884\\
52.57	0.00395646129065007\\
52.58	0.00395641621545508\\
52.59	0.00395637110907112\\
52.6	0.00395632597144513\\
52.61	0.00395628080252369\\
52.62	0.00395623560225299\\
52.63	0.00395619037057887\\
52.64	0.00395614510744681\\
52.65	0.00395609981280189\\
52.66	0.0039560544865888\\
52.67	0.00395600912875187\\
52.68	0.00395596373923502\\
52.69	0.00395591831798175\\
52.7	0.00395587286493517\\
52.71	0.00395582738003798\\
52.72	0.00395578186323246\\
52.73	0.00395573631446045\\
52.74	0.00395569073366336\\
52.75	0.00395564512078218\\
52.76	0.00395559947575743\\
52.77	0.0039555537985292\\
52.78	0.0039555080890371\\
52.79	0.00395546234722029\\
52.8	0.00395541657301746\\
52.81	0.00395537076636679\\
52.82	0.003955324927206\\
52.83	0.00395527905547232\\
52.84	0.00395523315110245\\
52.85	0.0039551872140326\\
52.86	0.00395514124419846\\
52.87	0.00395509524153518\\
52.88	0.0039550492059774\\
52.89	0.0039550031374592\\
52.9	0.00395495703591411\\
52.91	0.0039549109012751\\
52.92	0.00395486473347457\\
52.93	0.00395481853244436\\
52.94	0.00395477229811571\\
52.95	0.00395472603041927\\
52.96	0.00395467972928507\\
52.97	0.00395463339464254\\
52.98	0.00395458702642049\\
52.99	0.0039545406245471\\
53	0.00395449418894988\\
53.01	0.00395444771955571\\
53.02	0.0039544012162908\\
53.03	0.00395435467908069\\
53.04	0.00395430810785023\\
53.05	0.00395426150252358\\
53.06	0.00395421486302417\\
53.07	0.00395416818927474\\
53.08	0.00395412148119729\\
53.09	0.00395407473871308\\
53.1	0.00395402796174261\\
53.11	0.00395398115020563\\
53.12	0.00395393430402108\\
53.13	0.00395388742310715\\
53.14	0.0039538405073812\\
53.15	0.00395379355675979\\
53.16	0.00395374657115863\\
53.17	0.00395369955049261\\
53.18	0.00395365249467575\\
53.19	0.00395360540362122\\
53.2	0.00395355827724127\\
53.21	0.00395351111544728\\
53.22	0.00395346391814971\\
53.23	0.0039534166852581\\
53.24	0.00395336941668103\\
53.25	0.00395332211232613\\
53.26	0.00395327477210005\\
53.27	0.00395322739590846\\
53.28	0.00395317998365603\\
53.29	0.00395313253524637\\
53.3	0.00395308505058211\\
53.31	0.00395303752956476\\
53.32	0.00395298997209481\\
53.33	0.00395294237807161\\
53.34	0.00395289474739345\\
53.35	0.00395284707995745\\
53.36	0.0039527993756596\\
53.37	0.00395275163439473\\
53.38	0.00395270385605646\\
53.39	0.00395265604053725\\
53.4	0.00395260818772829\\
53.41	0.00395256029751956\\
53.42	0.00395251236979974\\
53.43	0.00395246440445625\\
53.44	0.0039524164013752\\
53.45	0.00395236836044135\\
53.46	0.00395232028153813\\
53.47	0.00395227216454758\\
53.48	0.00395222400935036\\
53.49	0.00395217581582568\\
53.5	0.00395212758385134\\
53.51	0.00395207931330364\\
53.52	0.0039520310040574\\
53.53	0.00395198265598593\\
53.54	0.00395193426896098\\
53.55	0.00395188584285274\\
53.56	0.00395183737752979\\
53.57	0.00395178887285912\\
53.58	0.00395174032870602\\
53.59	0.00395169174493415\\
53.6	0.00395164312140541\\
53.61	0.00395159445798003\\
53.62	0.00395154575451642\\
53.63	0.00395149701087122\\
53.64	0.00395144822689924\\
53.65	0.00395139940245344\\
53.66	0.00395135053738488\\
53.67	0.00395130163154271\\
53.68	0.00395125268477414\\
53.69	0.00395120369692437\\
53.7	0.0039511546678366\\
53.71	0.00395110559735198\\
53.72	0.00395105648530954\\
53.73	0.00395100733154625\\
53.74	0.00395095813589686\\
53.75	0.00395090889819395\\
53.76	0.00395085961826788\\
53.77	0.00395081029594672\\
53.78	0.00395076093105624\\
53.79	0.00395071152341986\\
53.8	0.00395066207285862\\
53.81	0.00395061257919112\\
53.82	0.00395056304223349\\
53.83	0.00395051346179935\\
53.84	0.00395046383769978\\
53.85	0.00395041416974325\\
53.86	0.00395036445773558\\
53.87	0.00395031470147992\\
53.88	0.00395026490077666\\
53.89	0.00395021505542345\\
53.9	0.00395016516521509\\
53.91	0.0039501152299435\\
53.92	0.00395006524939768\\
53.93	0.00395001522336367\\
53.94	0.00394996515162448\\
53.95	0.00394991503396003\\
53.96	0.00394986487014711\\
53.97	0.00394981465995937\\
53.98	0.00394976440316718\\
53.99	0.00394971409953763\\
54	0.00394966374883447\\
54.01	0.00394961335081805\\
54.02	0.00394956290524524\\
54.03	0.00394951241186942\\
54.04	0.00394946187044035\\
54.05	0.00394941128070418\\
54.06	0.00394936064240335\\
54.07	0.00394930995527652\\
54.08	0.00394925921905852\\
54.09	0.00394920843348032\\
54.1	0.00394915759826888\\
54.11	0.00394910671314715\\
54.12	0.00394905577783398\\
54.13	0.00394900479204407\\
54.14	0.00394895375548785\\
54.15	0.00394890266787147\\
54.16	0.00394885152889667\\
54.17	0.00394880033826075\\
54.18	0.00394874909565647\\
54.19	0.00394869780077198\\
54.2	0.00394864645329074\\
54.21	0.00394859505289144\\
54.22	0.00394854359924792\\
54.23	0.0039484920920291\\
54.24	0.00394844053089888\\
54.25	0.00394838891551605\\
54.26	0.00394833724553421\\
54.27	0.00394828552060172\\
54.28	0.00394823374036156\\
54.29	0.00394818190445123\\
54.3	0.00394813001250272\\
54.31	0.00394807806414237\\
54.32	0.00394802605899079\\
54.33	0.00394797399666275\\
54.34	0.0039479218767671\\
54.35	0.00394786969890666\\
54.36	0.00394781746267812\\
54.37	0.00394776516767193\\
54.38	0.00394771281347223\\
54.39	0.00394766039965669\\
54.4	0.00394760792579643\\
54.41	0.00394755539145594\\
54.42	0.00394750279619289\\
54.43	0.00394745013955811\\
54.44	0.00394739742109541\\
54.45	0.00394734464034147\\
54.46	0.00394729179682575\\
54.47	0.00394723889007037\\
54.48	0.00394718591958994\\
54.49	0.00394713288489149\\
54.5	0.0039470797854743\\
54.51	0.00394702662082982\\
54.52	0.0039469733904415\\
54.53	0.00394692009378466\\
54.54	0.00394686673032639\\
54.55	0.00394681329952535\\
54.56	0.00394675980083174\\
54.57	0.00394670623368702\\
54.58	0.00394665259752388\\
54.59	0.00394659889176604\\
54.6	0.00394654511582813\\
54.61	0.00394649126911551\\
54.62	0.00394643735102416\\
54.63	0.00394638336094048\\
54.64	0.00394632929824116\\
54.65	0.00394627516229304\\
54.66	0.00394622095245291\\
54.67	0.00394616666806735\\
54.68	0.00394611230847263\\
54.69	0.00394605787299443\\
54.7	0.0039460033609478\\
54.71	0.00394594877163687\\
54.72	0.00394589410435479\\
54.73	0.00394583935838348\\
54.74	0.00394578453299349\\
54.75	0.00394572962744382\\
54.76	0.00394567464098175\\
54.77	0.00394561957284263\\
54.78	0.00394556442224973\\
54.79	0.00394550918841404\\
54.8	0.00394545387053409\\
54.81	0.00394539846779575\\
54.82	0.00394534297937207\\
54.83	0.00394528740442302\\
54.84	0.0039452317420954\\
54.85	0.00394517599152253\\
54.86	0.00394512015182415\\
54.87	0.00394506422210614\\
54.88	0.00394500820146038\\
54.89	0.00394495208896449\\
54.9	0.00394489588368167\\
54.91	0.00394483958466048\\
54.92	0.00394478319093462\\
54.93	0.00394472670152271\\
54.94	0.00394467011542811\\
54.95	0.00394461343163866\\
54.96	0.00394455664912652\\
54.97	0.00394449976684789\\
54.98	0.00394444278374283\\
54.99	0.00394438569873505\\
55	0.00394432851073164\\
55.01	0.00394427121862288\\
55.02	0.00394421382128202\\
55.03	0.00394415631756503\\
55.04	0.00394409870631039\\
55.05	0.00394404098633885\\
55.06	0.00394398315645322\\
55.07	0.00394392521543812\\
55.08	0.00394386716205976\\
55.09	0.0039438089950657\\
55.1	0.00394375071318463\\
55.11	0.00394369231512613\\
55.12	0.00394363379958044\\
55.13	0.00394357516521823\\
55.14	0.00394351641069038\\
55.15	0.00394345753462769\\
55.16	0.00394339853564074\\
55.17	0.0039433394123196\\
55.18	0.00394328016323359\\
55.19	0.00394322078693109\\
55.2	0.0039431612819393\\
55.21	0.00394310164676398\\
55.22	0.0039430418798893\\
55.23	0.00394298197977751\\
55.24	0.00394292194486882\\
55.25	0.00394286177358113\\
55.26	0.00394280146430981\\
55.27	0.00394274101542751\\
55.28	0.00394268042528393\\
55.29	0.00394261969220561\\
55.3	0.00394255881449576\\
55.31	0.00394249779043403\\
55.32	0.00394243661827631\\
55.33	0.00394237529625457\\
55.34	0.00394231382257663\\
55.35	0.00394225219542603\\
55.36	0.00394219041296182\\
55.37	0.00394212847331841\\
55.38	0.00394206637460541\\
55.39	0.00394200411490747\\
55.4	0.00394194169228411\\
55.41	0.00394187910476964\\
55.42	0.00394181635037298\\
55.43	0.00394175342707757\\
55.44	0.00394169033284125\\
55.45	0.00394162706559615\\
55.46	0.00394156362324863\\
55.47	0.00394150000367916\\
55.48	0.00394143620474232\\
55.49	0.00394137222426667\\
55.5	0.00394130806005479\\
55.51	0.00394124370988321\\
55.52	0.00394117917150241\\
55.53	0.00394111444263684\\
55.54	0.00394104952098494\\
55.55	0.0039409844042192\\
55.56	0.00394091908998618\\
55.57	0.00394085357590665\\
55.58	0.00394078785957566\\
55.59	0.00394072193856265\\
55.6	0.00394065581041166\\
55.61	0.00394058947264141\\
55.62	0.0039405229227456\\
55.63	0.00394045615819306\\
55.64	0.00394038917642803\\
55.65	0.00394032197487044\\
55.66	0.00394025455091622\\
55.67	0.00394018690193763\\
55.68	0.00394011902528366\\
55.69	0.00394005091828043\\
55.7	0.00393998257823164\\
55.71	0.00393991400241904\\
55.72	0.00393984518810296\\
55.73	0.00393977613252291\\
55.74	0.00393970683289814\\
55.75	0.00393963728642831\\
55.76	0.00393956749029424\\
55.77	0.00393949744165856\\
55.78	0.00393942713766662\\
55.79	0.00393935657544727\\
55.8	0.00393928575211379\\
55.81	0.00393921466476491\\
55.82	0.00393914331048574\\
55.83	0.00393907168634896\\
55.84	0.00393899978941591\\
55.85	0.00393892761673788\\
55.86	0.00393885516535732\\
55.87	0.00393878243230932\\
55.88	0.00393870941462296\\
55.89	0.0039386361093229\\
55.9	0.00393856251343099\\
55.91	0.00393848862396795\\
55.92	0.00393841443795516\\
55.93	0.00393833995241656\\
55.94	0.00393826516438063\\
55.95	0.00393819007088246\\
55.96	0.00393811466896599\\
55.97	0.00393803895568623\\
55.98	0.00393796292811175\\
55.99	0.00393788658332717\\
56	0.00393780991843583\\
56.01	0.0039377329305626\\
56.02	0.00393765561685675\\
56.03	0.00393757797449507\\
56.04	0.003937500000685\\
56.05	0.00393742169266805\\
56.06	0.00393734304772327\\
56.07	0.00393726406317092\\
56.08	0.00393718473637632\\
56.09	0.00393710506475384\\
56.1	0.0039370250457711\\
56.11	0.00393694467695335\\
56.12	0.00393686395588802\\
56.13	0.00393678288022955\\
56.14	0.00393670144770427\\
56.15	0.00393661965611569\\
56.16	0.00393653750334989\\
56.17	0.00393645498738117\\
56.18	0.00393637210627795\\
56.19	0.00393628885820891\\
56.2	0.00393620524144945\\
56.21	0.0039361212543883\\
56.22	0.00393603689553452\\
56.23	0.00393595216352474\\
56.24	0.00393586705713072\\
56.25	0.0039357815752672\\
56.26	0.00393569571700009\\
56.27	0.0039356094792325\\
56.28	0.0039355228544476\\
56.29	0.00393543583494631\\
56.3	0.00393534841284249\\
56.31	0.00393526058005811\\
56.32	0.0039351723283182\\
56.33	0.00393508364914574\\
56.34	0.00393499453385632\\
56.35	0.00393490497355275\\
56.36	0.00393481495911945\\
56.37	0.00393472448121671\\
56.38	0.0039346335302748\\
56.39	0.00393454209648791\\
56.4	0.00393445016980792\\
56.41	0.003934357739938\\
56.42	0.00393426479632605\\
56.43	0.00393417132815792\\
56.44	0.00393407732435049\\
56.45	0.0039339827735445\\
56.46	0.00393388766409727\\
56.47	0.00393379198407512\\
56.48	0.00393369572124563\\
56.49	0.00393359886306972\\
56.5	0.0039335013966934\\
56.51	0.00393340330893944\\
56.52	0.00393330458629863\\
56.53	0.003933205214921\\
56.54	0.00393310518060664\\
56.55	0.00393300446879629\\
56.56	0.00393290306456174\\
56.57	0.00393280095259591\\
56.58	0.00393269811720259\\
56.59	0.00393259454228606\\
56.6	0.00393249021134026\\
56.61	0.00393238510743771\\
56.62	0.00393227921321817\\
56.63	0.00393217251087688\\
56.64	0.0039320649821526\\
56.65	0.00393195660831517\\
56.66	0.00393184737015284\\
56.67	0.0039317372479592\\
56.68	0.00393162622151969\\
56.69	0.00393151427009784\\
56.7	0.00393140137242102\\
56.71	0.00393128750666586\\
56.72	0.0039311726504432\\
56.73	0.00393105678078269\\
56.74	0.00393093987411685\\
56.75	0.0039308219062648\\
56.76	0.00393070285241541\\
56.77	0.00393058268711011\\
56.78	0.00393046138422506\\
56.79	0.00393033891695293\\
56.8	0.00393021525778416\\
56.81	0.00393009037848757\\
56.82	0.00392996425009059\\
56.83	0.00392983684285879\\
56.84	0.00392970812627491\\
56.85	0.00392957806901729\\
56.86	0.00392944663893759\\
56.87	0.00392931380303801\\
56.88	0.00392917952744783\\
56.89	0.00392904377739916\\
56.9	0.00392890651720219\\
56.91	0.00392876771021959\\
56.92	0.00392862731884025\\
56.93	0.00392848530445224\\
56.94	0.0039283430067023\\
56.95	0.00392820064335426\\
56.96	0.00392805821436291\\
56.97	0.00392791571968297\\
56.98	0.00392777315926912\\
56.99	0.00392763053307602\\
57	0.00392748784105824\\
57.01	0.00392734508317033\\
57.02	0.00392720225936678\\
57.03	0.00392705936960206\\
57.04	0.00392691641383056\\
57.05	0.00392677339200665\\
57.06	0.00392663030408462\\
57.07	0.00392648715001874\\
57.08	0.00392634392976323\\
57.09	0.00392620064327227\\
57.1	0.00392605729049995\\
57.11	0.00392591387140037\\
57.12	0.00392577038592755\\
57.13	0.00392562683403546\\
57.14	0.00392548321567805\\
57.15	0.00392533953080918\\
57.16	0.00392519577938271\\
57.17	0.00392505196135241\\
57.18	0.00392490807667204\\
57.19	0.00392476412529527\\
57.2	0.00392462010717576\\
57.21	0.00392447602226711\\
57.22	0.00392433187052285\\
57.23	0.00392418765189651\\
57.24	0.00392404336634151\\
57.25	0.00392389901381128\\
57.26	0.00392375459425917\\
57.27	0.00392361010763848\\
57.28	0.00392346555390248\\
57.29	0.00392332093300438\\
57.3	0.00392317624489733\\
57.31	0.00392303148953446\\
57.32	0.00392288666686883\\
57.33	0.00392274177685345\\
57.34	0.0039225968194413\\
57.35	0.00392245179458529\\
57.36	0.00392230670223829\\
57.37	0.00392216154235312\\
57.38	0.00392201631488256\\
57.39	0.00392187101977934\\
57.4	0.00392172565699611\\
57.41	0.00392158022648551\\
57.42	0.00392143472820012\\
57.43	0.00392128916209245\\
57.44	0.00392114352811499\\
57.45	0.00392099782622016\\
57.46	0.00392085205636034\\
57.47	0.00392070621848786\\
57.48	0.00392056031255499\\
57.49	0.00392041433851396\\
57.5	0.00392026829631695\\
57.51	0.00392012218591609\\
57.52	0.00391997600726346\\
57.53	0.00391982976031109\\
57.54	0.00391968344501096\\
57.55	0.00391953706131498\\
57.56	0.00391939060917504\\
57.57	0.00391924408854297\\
57.58	0.00391909749937054\\
57.59	0.00391895084160949\\
57.6	0.00391880411521148\\
57.61	0.00391865732012814\\
57.62	0.00391851045631105\\
57.63	0.00391836352371173\\
57.64	0.00391821652228166\\
57.65	0.00391806945197224\\
57.66	0.00391792231273487\\
57.67	0.00391777510452085\\
57.68	0.00391762782728146\\
57.69	0.00391748048096792\\
57.7	0.00391733306553138\\
57.71	0.00391718558092297\\
57.72	0.00391703802709376\\
57.73	0.00391689040399474\\
57.74	0.0039167427115769\\
57.75	0.00391659494979112\\
57.76	0.00391644711858828\\
57.77	0.00391629921791918\\
57.78	0.00391615124773457\\
57.79	0.00391600320798515\\
57.8	0.00391585509862158\\
57.81	0.00391570691959445\\
57.82	0.00391555867085431\\
57.83	0.00391541035235166\\
57.84	0.00391526196403693\\
57.85	0.00391511350586052\\
57.86	0.00391496497777277\\
57.87	0.00391481637972396\\
57.88	0.00391466771166432\\
57.89	0.00391451897354403\\
57.9	0.00391437016531322\\
57.91	0.00391422128692196\\
57.92	0.00391407233832028\\
57.93	0.00391392331945814\\
57.94	0.00391377423028546\\
57.95	0.00391362507075211\\
57.96	0.00391347584080788\\
57.97	0.00391332654040255\\
57.98	0.00391317716948581\\
57.99	0.00391302772800731\\
58	0.00391287821591665\\
58.01	0.00391272863316338\\
58.02	0.00391257897969698\\
58.03	0.00391242925546688\\
58.04	0.00391227946042248\\
58.05	0.0039121295945131\\
58.06	0.00391197965768801\\
58.07	0.00391182964989644\\
58.08	0.00391167957108756\\
58.09	0.00391152942121046\\
58.1	0.00391137920021423\\
58.11	0.00391122890804785\\
58.12	0.00391107854466028\\
58.13	0.00391092811000042\\
58.14	0.0039107776040171\\
58.15	0.00391062702665912\\
58.16	0.0039104763778752\\
58.17	0.00391032565761403\\
58.18	0.00391017486582423\\
58.19	0.00391002400245436\\
58.2	0.00390987306745294\\
58.21	0.00390972206076843\\
58.22	0.00390957098234924\\
58.23	0.0039094198321437\\
58.24	0.00390926861010012\\
58.25	0.00390911731616672\\
58.26	0.0039089659502917\\
58.27	0.00390881451242318\\
58.28	0.00390866300250924\\
58.29	0.00390851142049788\\
58.3	0.00390835976633706\\
58.31	0.00390820803997471\\
58.32	0.00390805624135866\\
58.33	0.0039079043704367\\
58.34	0.00390775242715657\\
58.35	0.00390760041146596\\
58.36	0.00390744832331249\\
58.37	0.00390729616264373\\
58.38	0.00390714392940718\\
58.39	0.00390699162355031\\
58.4	0.00390683924502052\\
58.41	0.00390668679376515\\
58.42	0.00390653426973148\\
58.43	0.00390638167286675\\
58.44	0.00390622900311812\\
58.45	0.00390607626043273\\
58.46	0.00390592344475762\\
58.47	0.0039057705560398\\
58.48	0.00390561759422621\\
58.49	0.00390546455926374\\
58.5	0.00390531145109922\\
58.51	0.00390515826967943\\
58.52	0.00390500501495109\\
58.53	0.00390485168686084\\
58.54	0.0039046982853553\\
58.55	0.00390454481038101\\
58.56	0.00390439126188445\\
58.57	0.00390423763981205\\
58.58	0.00390408394411018\\
58.59	0.00390393017472517\\
58.6	0.00390377633160325\\
58.61	0.00390362241469062\\
58.62	0.00390346842393343\\
58.63	0.00390331435927775\\
58.64	0.00390316022066961\\
58.65	0.00390300600805497\\
58.66	0.00390285172137973\\
58.67	0.00390269736058975\\
58.68	0.0039025429256308\\
58.69	0.00390238841644861\\
58.7	0.00390223383298886\\
58.71	0.00390207917519715\\
58.72	0.00390192444301905\\
58.73	0.00390176963640004\\
58.74	0.00390161475528555\\
58.75	0.00390145979962096\\
58.76	0.00390130476935158\\
58.77	0.00390114966442268\\
58.78	0.00390099448477944\\
58.79	0.003900839230367\\
58.8	0.00390068390113043\\
58.81	0.00390052849701477\\
58.82	0.00390037301796495\\
58.83	0.00390021746392589\\
58.84	0.00390006183484242\\
58.85	0.00389990613065931\\
58.86	0.00389975035132128\\
58.87	0.00389959449677298\\
58.88	0.00389943856695902\\
58.89	0.00389928256182393\\
58.9	0.00389912648131218\\
58.91	0.0038989703253682\\
58.92	0.00389881409393633\\
58.93	0.00389865778696086\\
58.94	0.00389850140438604\\
58.95	0.00389834494615602\\
58.96	0.00389818841221493\\
58.97	0.0038980318025068\\
58.98	0.00389787511697563\\
58.99	0.00389771835556535\\
59	0.00389756151821981\\
59.01	0.00389740460488283\\
59.02	0.00389724761549815\\
59.03	0.00389709055000943\\
59.04	0.00389693340836032\\
59.05	0.00389677619049435\\
59.06	0.00389661889635503\\
59.07	0.00389646152588579\\
59.08	0.00389630407903\\
59.09	0.00389614655573096\\
59.1	0.00389598895593193\\
59.11	0.00389583127957609\\
59.12	0.00389567352660656\\
59.13	0.00389551569696641\\
59.14	0.00389535779059861\\
59.15	0.00389519980744611\\
59.16	0.00389504174745179\\
59.17	0.00389488361055844\\
59.18	0.00389472539670882\\
59.19	0.0038945671058456\\
59.2	0.00389440873791141\\
59.21	0.00389425029284879\\
59.22	0.00389409177060024\\
59.23	0.00389393317110819\\
59.24	0.00389377449431501\\
59.25	0.00389361574016299\\
59.26	0.00389345690859438\\
59.27	0.00389329799955134\\
59.28	0.003893139012976\\
59.29	0.00389297994881038\\
59.3	0.00389282080699648\\
59.31	0.00389266158747621\\
59.32	0.00389250229019143\\
59.33	0.00389234291508392\\
59.34	0.00389218346209541\\
59.35	0.00389202393116756\\
59.36	0.00389186432224197\\
59.37	0.00389170463526016\\
59.38	0.00389154487016361\\
59.39	0.0038913850268937\\
59.4	0.00389122510539178\\
59.41	0.00389106510559913\\
59.42	0.00389090502745693\\
59.43	0.00389074487090634\\
59.44	0.00389058463588843\\
59.45	0.00389042432234421\\
59.46	0.00389026393021462\\
59.47	0.00389010345944053\\
59.48	0.00388994290996278\\
59.49	0.00388978228172209\\
59.5	0.00388962157465914\\
59.51	0.00388946078871456\\
59.52	0.0038892999238289\\
59.53	0.00388913897994262\\
59.54	0.00388897795699616\\
59.55	0.00388881685492985\\
59.56	0.00388865567368398\\
59.57	0.00388849441319877\\
59.58	0.00388833307341436\\
59.59	0.00388817165427084\\
59.6	0.00388801015570822\\
59.61	0.00388784857766646\\
59.62	0.00388768692008542\\
59.63	0.00388752518290494\\
59.64	0.00388736336606474\\
59.65	0.00388720146950451\\
59.66	0.00388703949316388\\
59.67	0.00388687743698236\\
59.68	0.00388671530089945\\
59.69	0.00388655308485455\\
59.7	0.00388639078878701\\
59.71	0.00388622841263609\\
59.72	0.003886065956341\\
59.73	0.00388590341984088\\
59.74	0.00388574080307478\\
59.75	0.00388557810598172\\
59.76	0.00388541532850063\\
59.77	0.00388525247057036\\
59.78	0.00388508953212971\\
59.79	0.0038849265131174\\
59.8	0.00388476341347209\\
59.81	0.00388460023313237\\
59.82	0.00388443697203675\\
59.83	0.00388427363012369\\
59.84	0.00388411020733156\\
59.85	0.00388394670359866\\
59.86	0.00388378311886325\\
59.87	0.00388361945306349\\
59.88	0.00388345570613748\\
59.89	0.00388329187802325\\
59.9	0.00388312796865876\\
59.91	0.00388296397798191\\
59.92	0.00388279990593051\\
59.93	0.00388263575244232\\
59.94	0.003882471517455\\
59.95	0.00388230720090619\\
59.96	0.0038821428027334\\
59.97	0.00388197832287412\\
59.98	0.00388181376126574\\
59.99	0.00388164911784557\\
60	0.00388148439255089\\
60.01	0.00388131958531887\\
60.02	0.00388115469608663\\
60.03	0.00388098972479121\\
60.04	0.00388082467136957\\
60.05	0.00388065953575863\\
60.06	0.0038804943178952\\
60.07	0.00388032901771604\\
60.08	0.00388016363515784\\
60.09	0.0038799981701572\\
60.1	0.00387983262265067\\
60.11	0.0038796669925747\\
60.12	0.00387950127986571\\
60.13	0.00387933548446\\
60.14	0.00387916960629384\\
60.15	0.00387900364530339\\
60.16	0.00387883760142477\\
60.17	0.003878671474594\\
60.18	0.00387850526474705\\
60.19	0.00387833897181979\\
60.2	0.00387817259574804\\
60.21	0.00387800613646755\\
60.22	0.00387783959391397\\
60.23	0.00387767296802291\\
60.24	0.00387750625872988\\
60.25	0.00387733946597031\\
60.26	0.0038771725896796\\
60.27	0.00387700562979303\\
60.28	0.00387683858624582\\
60.29	0.00387667145897314\\
60.3	0.00387650424791004\\
60.31	0.00387633695299153\\
60.32	0.00387616957415255\\
60.33	0.00387600211132793\\
60.34	0.00387583456445246\\
60.35	0.00387566693346083\\
60.36	0.00387549921828768\\
60.37	0.00387533141886756\\
60.38	0.00387516353513494\\
60.39	0.00387499556702423\\
60.4	0.00387482751446975\\
60.41	0.00387465937740575\\
60.42	0.0038744911557664\\
60.43	0.00387432284948581\\
60.44	0.003874154458498\\
60.45	0.00387398598273691\\
60.46	0.00387381742213642\\
60.47	0.00387364877663032\\
60.48	0.00387348004615233\\
60.49	0.00387331123063609\\
60.5	0.00387314233001517\\
60.51	0.00387297334422305\\
60.52	0.00387280427319315\\
60.53	0.0038726351168588\\
60.54	0.00387246587515326\\
60.55	0.00387229654800971\\
60.56	0.00387212713536125\\
60.57	0.00387195763714092\\
60.58	0.00387178805328165\\
60.59	0.00387161838371632\\
60.6	0.00387144862837773\\
60.61	0.00387127878719858\\
60.62	0.00387110886011151\\
60.63	0.00387093884704909\\
60.64	0.00387076874794379\\
60.65	0.00387059856272802\\
60.66	0.0038704282913341\\
60.67	0.00387025793369427\\
60.68	0.0038700874897407\\
60.69	0.00386991695940548\\
60.7	0.00386974634262062\\
60.71	0.00386957563931805\\
60.72	0.00386940484942961\\
60.73	0.00386923397288708\\
60.74	0.00386906300962215\\
60.75	0.00386889195956643\\
60.76	0.00386872082265145\\
60.77	0.00386854959880867\\
60.78	0.00386837828796946\\
60.79	0.0038682068900651\\
60.8	0.00386803540502681\\
60.81	0.00386786383278573\\
60.82	0.0038676921732729\\
60.83	0.00386752042641929\\
60.84	0.0038673485921558\\
60.85	0.00386717667041322\\
60.86	0.0038670046611223\\
60.87	0.00386683256421367\\
60.88	0.00386666037961791\\
60.89	0.00386648810726548\\
60.9	0.0038663157470868\\
60.91	0.00386614329901219\\
60.92	0.00386597076297189\\
60.93	0.00386579813889605\\
60.94	0.00386562542671475\\
60.95	0.00386545262635798\\
60.96	0.00386527973775565\\
60.97	0.00386510676083759\\
60.98	0.00386493369553355\\
60.99	0.00386476054177318\\
61	0.00386458729948608\\
61.01	0.00386441396860172\\
61.02	0.00386424054904953\\
61.03	0.00386406704075885\\
61.04	0.00386389344365891\\
61.05	0.00386371975767888\\
61.06	0.00386354598274785\\
61.07	0.0038633721187948\\
61.08	0.00386319816574864\\
61.09	0.00386302412353822\\
61.1	0.00386284999209227\\
61.11	0.00386267577133945\\
61.12	0.00386250146120834\\
61.13	0.00386232706162742\\
61.14	0.00386215257252511\\
61.15	0.00386197799382973\\
61.16	0.00386180332546951\\
61.17	0.0038616285673726\\
61.18	0.00386145371946708\\
61.19	0.00386127878168091\\
61.2	0.003861103753942\\
61.21	0.00386092863617815\\
61.22	0.00386075342831708\\
61.23	0.00386057813028645\\
61.24	0.00386040274201379\\
61.25	0.00386022726342657\\
61.26	0.00386005169445218\\
61.27	0.0038598760350179\\
61.28	0.00385970028505093\\
61.29	0.00385952444447841\\
61.3	0.00385934851322734\\
61.31	0.0038591724912247\\
61.32	0.00385899637839732\\
61.33	0.00385882017467199\\
61.34	0.00385864387997537\\
61.35	0.00385846749423408\\
61.36	0.00385829101737461\\
61.37	0.00385811444932338\\
61.38	0.00385793779000673\\
61.39	0.0038577610393509\\
61.4	0.00385758419728204\\
61.41	0.00385740726372621\\
61.42	0.00385723023860939\\
61.43	0.00385705312185748\\
61.44	0.00385687591339627\\
61.45	0.00385669861315147\\
61.46	0.00385652122104869\\
61.47	0.00385634373701348\\
61.48	0.00385616616097127\\
61.49	0.00385598849284742\\
61.5	0.00385581073256718\\
61.51	0.00385563288005574\\
61.52	0.00385545493523816\\
61.53	0.00385527689803945\\
61.54	0.00385509876838449\\
61.55	0.0038549205461981\\
61.56	0.00385474223140501\\
61.57	0.00385456382392983\\
61.58	0.00385438532369711\\
61.59	0.00385420673063129\\
61.6	0.00385402804465673\\
61.61	0.00385384926569769\\
61.62	0.00385367039367833\\
61.63	0.00385349142852274\\
61.64	0.00385331237015491\\
61.65	0.00385313321849873\\
61.66	0.003852953973478\\
61.67	0.00385277463501643\\
61.68	0.00385259520303765\\
61.69	0.00385241567746516\\
61.7	0.00385223605822241\\
61.71	0.00385205634523273\\
61.72	0.00385187653841937\\
61.73	0.00385169663770548\\
61.74	0.00385151664301412\\
61.75	0.00385133655426824\\
61.76	0.00385115637139073\\
61.77	0.00385097609430435\\
61.78	0.00385079572293179\\
61.79	0.00385061525719563\\
61.8	0.00385043469701836\\
61.81	0.00385025404232239\\
61.82	0.00385007329303002\\
61.83	0.00384989244906344\\
61.84	0.00384971151034478\\
61.85	0.00384953047679605\\
61.86	0.00384934934833918\\
61.87	0.00384916812489599\\
61.88	0.00384898680638819\\
61.89	0.00384880539273744\\
61.9	0.00384862388386526\\
61.91	0.0038484422796931\\
61.92	0.0038482605801423\\
61.93	0.00384807878513411\\
61.94	0.00384789689458967\\
61.95	0.00384771490843005\\
61.96	0.00384753282657618\\
61.97	0.00384735064894894\\
61.98	0.00384716837546908\\
61.99	0.00384698600605727\\
62	0.00384680354063406\\
62.01	0.00384662097911993\\
62.02	0.00384643832143525\\
62.03	0.00384625556750027\\
62.04	0.00384607271723518\\
62.05	0.00384588977056004\\
62.06	0.00384570672739483\\
62.07	0.00384552358765942\\
62.08	0.00384534035127358\\
62.09	0.00384515701815699\\
62.1	0.00384497358822923\\
62.11	0.00384479006140975\\
62.12	0.00384460643761795\\
62.13	0.0038444227167731\\
62.14	0.00384423889879436\\
62.15	0.00384405498360082\\
62.16	0.00384387097111144\\
62.17	0.0038436868612451\\
62.18	0.00384350265392056\\
62.19	0.0038433183490565\\
62.2	0.00384313394657147\\
62.21	0.00384294944638395\\
62.22	0.0038427648484123\\
62.23	0.00384258015257478\\
62.24	0.00384239535878955\\
62.25	0.00384221046697466\\
62.26	0.00384202547704806\\
62.27	0.00384184038892762\\
62.28	0.00384165520253107\\
62.29	0.00384146991777606\\
62.3	0.00384128453458014\\
62.31	0.00384109905286073\\
62.32	0.00384091347253517\\
62.33	0.00384072779352069\\
62.34	0.00384054201573442\\
62.35	0.00384035613909337\\
62.36	0.00384017016351447\\
62.37	0.00383998408891452\\
62.38	0.00383979791521023\\
62.39	0.00383961164231821\\
62.4	0.00383942527015494\\
62.41	0.00383923879863682\\
62.42	0.00383905222768013\\
62.43	0.00383886555720105\\
62.44	0.00383867878711566\\
62.45	0.00383849191733992\\
62.46	0.00383830494778969\\
62.47	0.00383811787838073\\
62.48	0.00383793070902867\\
62.49	0.00383774343964907\\
62.5	0.00383755607015734\\
62.51	0.00383736860046882\\
62.52	0.00383718103049872\\
62.53	0.00383699336016215\\
62.54	0.00383680558937412\\
62.55	0.00383661771804951\\
62.56	0.0038364297461031\\
62.57	0.00383624167344957\\
62.58	0.0038360535000035\\
62.59	0.00383586522567933\\
62.6	0.00383567685039142\\
62.61	0.003835488374054\\
62.62	0.0038352997965812\\
62.63	0.00383511111788704\\
62.64	0.00383492233788543\\
62.65	0.00383473345649017\\
62.66	0.00383454447361494\\
62.67	0.00383435538917333\\
62.68	0.00383416620307879\\
62.69	0.00383397691524469\\
62.7	0.00383378752558427\\
62.71	0.00383359803401066\\
62.72	0.00383340844043687\\
62.73	0.00383321874477583\\
62.74	0.00383302894694033\\
62.75	0.00383283904684305\\
62.76	0.00383264904439656\\
62.77	0.00383245893951332\\
62.78	0.00383226873210568\\
62.79	0.00383207842208587\\
62.8	0.00383188800936601\\
62.81	0.0038316974938581\\
62.82	0.00383150687547404\\
62.83	0.00383131615412561\\
62.84	0.00383112532972446\\
62.85	0.00383093440218216\\
62.86	0.00383074337141013\\
62.87	0.00383055223731969\\
62.88	0.00383036099982205\\
62.89	0.0038301696588283\\
62.9	0.00382997821424941\\
62.91	0.00382978666599624\\
62.92	0.00382959501397954\\
62.93	0.00382940325810992\\
62.94	0.0038292113982979\\
62.95	0.00382901943445388\\
62.96	0.00382882736648813\\
62.97	0.00382863519431081\\
62.98	0.00382844291783196\\
62.99	0.00382825053696151\\
63	0.00382805805160926\\
63.01	0.00382786546168491\\
63.02	0.00382767276709803\\
63.03	0.00382747996775808\\
63.04	0.00382728706357438\\
63.05	0.00382709405445616\\
63.06	0.00382690094031251\\
63.07	0.00382670772105242\\
63.08	0.00382651439658473\\
63.09	0.0038263209668182\\
63.1	0.00382612743166145\\
63.11	0.00382593379102296\\
63.12	0.00382574004481114\\
63.13	0.00382554619293422\\
63.14	0.00382535223530037\\
63.15	0.00382515817181759\\
63.16	0.00382496400239378\\
63.17	0.00382476972693672\\
63.18	0.00382457534535407\\
63.19	0.00382438085755336\\
63.2	0.00382418626344201\\
63.21	0.0038239915629273\\
63.22	0.0038237967559164\\
63.23	0.00382360184231636\\
63.24	0.00382340682203411\\
63.25	0.00382321169497644\\
63.26	0.00382301646105003\\
63.27	0.00382282112016143\\
63.28	0.00382262567221709\\
63.29	0.00382243011712331\\
63.3	0.00382223445478626\\
63.31	0.00382203868511201\\
63.32	0.0038218428080065\\
63.33	0.00382164682337554\\
63.34	0.00382145073112482\\
63.35	0.00382125453115989\\
63.36	0.0038210582233862\\
63.37	0.00382086180770905\\
63.38	0.00382066528403364\\
63.39	0.00382046865226503\\
63.4	0.00382027191230815\\
63.41	0.00382007506406781\\
63.42	0.00381987810744869\\
63.43	0.00381968104235536\\
63.44	0.00381948386869224\\
63.45	0.00381928658636364\\
63.46	0.00381908919527374\\
63.47	0.00381889169532658\\
63.48	0.00381869408642609\\
63.49	0.00381849636847607\\
63.5	0.00381829854138018\\
63.51	0.00381810060504196\\
63.52	0.00381790255936483\\
63.53	0.00381770440425208\\
63.54	0.00381750613960685\\
63.55	0.00381730776533218\\
63.56	0.00381710928133096\\
63.57	0.00381691068750598\\
63.58	0.00381671198375985\\
63.59	0.00381651316999511\\
63.6	0.00381631424611413\\
63.61	0.00381611521201918\\
63.62	0.00381591606761236\\
63.63	0.00381571681279568\\
63.64	0.00381551744747101\\
63.65	0.00381531797154007\\
63.66	0.00381511838490448\\
63.67	0.0038149186874657\\
63.68	0.00381471887912508\\
63.69	0.00381451895978384\\
63.7	0.00381431892934305\\
63.71	0.00381411878770368\\
63.72	0.00381391853476653\\
63.73	0.00381371817043231\\
63.74	0.00381351769460156\\
63.75	0.00381331710717472\\
63.76	0.00381311640805208\\
63.77	0.00381291559713381\\
63.78	0.00381271467431994\\
63.79	0.00381251363951037\\
63.8	0.00381231249260487\\
63.81	0.00381211123350307\\
63.82	0.00381190986210448\\
63.83	0.00381170837830848\\
63.84	0.0038115067820143\\
63.85	0.00381130507312105\\
63.86	0.00381110325152771\\
63.87	0.00381090131713313\\
63.88	0.003810699269836\\
63.89	0.00381049710953491\\
63.9	0.0038102948361283\\
63.91	0.00381009244951449\\
63.92	0.00380988994959165\\
63.93	0.00380968733625784\\
63.94	0.00380948460941096\\
63.95	0.00380928176894879\\
63.96	0.00380907881476899\\
63.97	0.00380887574676907\\
63.98	0.0038086725648464\\
63.99	0.00380846926889825\\
64	0.00380826585882171\\
64.01	0.00380806233451378\\
64.02	0.0038078586958713\\
64.03	0.00380765494279098\\
64.04	0.00380745107516942\\
64.05	0.00380724709290306\\
64.06	0.00380704299588821\\
64.07	0.00380683878402106\\
64.08	0.00380663445719767\\
64.09	0.00380643001531394\\
64.1	0.00380622545826566\\
64.11	0.00380602078594847\\
64.12	0.00380581599825792\\
64.13	0.00380561109508936\\
64.14	0.00380540607633806\\
64.15	0.00380520094189914\\
64.16	0.00380499569166759\\
64.17	0.00380479032553825\\
64.18	0.00380458484340586\\
64.19	0.003804379245165\\
64.2	0.00380417353071013\\
64.21	0.00380396769993559\\
64.22	0.00380376175273555\\
64.23	0.00380355568900408\\
64.24	0.00380334950863512\\
64.25	0.00380314321152248\\
64.26	0.0038029367975598\\
64.27	0.00380273026664063\\
64.28	0.00380252361865838\\
64.29	0.00380231685350632\\
64.3	0.0038021099710776\\
64.31	0.00380190297126523\\
64.32	0.0038016958539621\\
64.33	0.00380148861906096\\
64.34	0.00380128126645443\\
64.35	0.00380107379603502\\
64.36	0.00380086620769509\\
64.37	0.00380065850132688\\
64.38	0.0038004506768225\\
64.39	0.00380024273407392\\
64.4	0.00380003467297301\\
64.41	0.00379982649341149\\
64.42	0.00379961819528095\\
64.43	0.00379940977847288\\
64.44	0.0037992012428786\\
64.45	0.00379899258838936\\
64.46	0.00379878381489623\\
64.47	0.00379857492229018\\
64.48	0.00379836591046207\\
64.49	0.0037981567793026\\
64.5	0.00379794752870237\\
64.51	0.00379773815855186\\
64.52	0.0037975286687414\\
64.53	0.00379731905916123\\
64.54	0.00379710932970145\\
64.55	0.00379689948025203\\
64.56	0.00379668951070284\\
64.57	0.00379647942094362\\
64.58	0.00379626921086398\\
64.59	0.00379605888035343\\
64.6	0.00379584842930133\\
64.61	0.00379563785759696\\
64.62	0.00379542716512947\\
64.63	0.00379521635178787\\
64.64	0.00379500541746107\\
64.65	0.00379479436203788\\
64.66	0.00379458318540696\\
64.67	0.00379437188745688\\
64.68	0.0037941604680761\\
64.69	0.00379394892715295\\
64.7	0.00379373726457566\\
64.71	0.00379352548023234\\
64.72	0.00379331357401099\\
64.73	0.0037931015457995\\
64.74	0.00379288939548566\\
64.75	0.00379267712295714\\
64.76	0.00379246472810151\\
64.77	0.00379225221080623\\
64.78	0.00379203957095865\\
64.79	0.00379182680844603\\
64.8	0.0037916139231555\\
64.81	0.00379140091497413\\
64.82	0.00379118778378882\\
64.83	0.00379097452948645\\
64.84	0.00379076115195373\\
64.85	0.00379054765107732\\
64.86	0.00379033402674374\\
64.87	0.00379012027883946\\
64.88	0.00378990640725081\\
64.89	0.00378969241186405\\
64.9	0.00378947829256535\\
64.91	0.00378926404924076\\
64.92	0.00378904968177627\\
64.93	0.00378883519005776\\
64.94	0.00378862057397104\\
64.95	0.0037884058334018\\
64.96	0.00378819096823567\\
64.97	0.00378797597835819\\
64.98	0.00378776086365482\\
64.99	0.00378754562401092\\
65	0.0037873302593118\\
65.01	0.00378711476944264\\
65.02	0.00378689915428859\\
65.03	0.00378668341373469\\
65.04	0.00378646754766595\\
65.05	0.00378625155596724\\
65.06	0.00378603543852341\\
65.07	0.00378581919521921\\
65.08	0.00378560282593934\\
65.09	0.00378538633056841\\
65.1	0.00378516970899099\\
65.11	0.00378495296109157\\
65.12	0.00378473608675456\\
65.13	0.00378451908586434\\
65.14	0.00378430195830521\\
65.15	0.00378408470396142\\
65.16	0.00378386732271714\\
65.17	0.00378364981445654\\
65.18	0.00378343217906367\\
65.19	0.00378321441642257\\
65.2	0.00378299652641722\\
65.21	0.00378277850893156\\
65.22	0.00378256036384946\\
65.23	0.00378234209105477\\
65.24	0.00378212369043129\\
65.25	0.00378190516186278\\
65.26	0.00378168650523295\\
65.27	0.00378146772042549\\
65.28	0.00378124880732404\\
65.29	0.00378102976581223\\
65.3	0.00378081059577364\\
65.31	0.00378059129709181\\
65.32	0.00378037186965028\\
65.33	0.00378015231333256\\
65.34	0.00377993262802213\\
65.35	0.00377971281360245\\
65.36	0.00377949286995696\\
65.37	0.00377927279696908\\
65.38	0.00377905259452223\\
65.39	0.00377883226249981\\
65.4	0.00377861180078521\\
65.41	0.00377839120926181\\
65.42	0.00377817048781299\\
65.43	0.00377794963632212\\
65.44	0.00377772865467257\\
65.45	0.00377750754274771\\
65.46	0.00377728630043093\\
65.47	0.0037770649276056\\
65.48	0.00377684342415511\\
65.49	0.00377662178996287\\
65.5	0.00377640002491227\\
65.51	0.00377617812888676\\
65.52	0.00377595610176976\\
65.53	0.00377573394344475\\
65.54	0.0037755116537952\\
65.55	0.00377528923270462\\
65.56	0.00377506668005654\\
65.57	0.00377484399573451\\
65.58	0.00377462117962214\\
65.59	0.00377439823160303\\
65.6	0.00377417515156084\\
65.61	0.00377395193937927\\
65.62	0.00377372859494204\\
65.63	0.00377350511813293\\
65.64	0.00377328150883575\\
65.65	0.00377305776693436\\
65.66	0.00377283389231266\\
65.67	0.00377260988485462\\
65.68	0.00377238574444424\\
65.69	0.00377216147096556\\
65.7	0.00377193706430273\\
65.71	0.0037717125243399\\
65.72	0.00377148785096131\\
65.73	0.00377126304405124\\
65.74	0.00377103810349406\\
65.75	0.00377081302917418\\
65.76	0.00377058782097608\\
65.77	0.00377036247878433\\
65.78	0.00377013700248354\\
65.79	0.0037699113919584\\
65.8	0.00376968564709367\\
65.81	0.0037694597677742\\
65.82	0.0037692337538849\\
65.83	0.00376900760531075\\
65.84	0.00376878132193682\\
65.85	0.00376855490364826\\
65.86	0.00376832835033028\\
65.87	0.00376810166186818\\
65.88	0.00376787483814736\\
65.89	0.00376764787905327\\
65.9	0.00376742078447145\\
65.91	0.00376719355428754\\
65.92	0.00376696618838725\\
65.93	0.00376673868665636\\
65.94	0.00376651104898074\\
65.95	0.00376628327524636\\
65.96	0.00376605536533925\\
65.97	0.00376582731914553\\
65.98	0.00376559913655139\\
65.99	0.00376537081744312\\
66	0.00376514236170706\\
66.01	0.00376491376922966\\
66.02	0.00376468503989743\\
66.03	0.00376445617359695\\
66.04	0.00376422717021487\\
66.05	0.00376399802963792\\
66.06	0.0037637687517529\\
66.07	0.00376353933644666\\
66.08	0.00376330978360613\\
66.09	0.00376308009311829\\
66.1	0.00376285026487018\\
66.11	0.00376262029874889\\
66.12	0.00376239019464156\\
66.13	0.00376215995243537\\
66.14	0.00376192957201754\\
66.15	0.00376169905327533\\
66.16	0.00376146839609603\\
66.17	0.00376123760036695\\
66.18	0.00376100666597542\\
66.19	0.00376077559280877\\
66.2	0.00376054438075436\\
66.21	0.00376031302969951\\
66.22	0.00376008153953157\\
66.23	0.00375984991013785\\
66.24	0.00375961814140563\\
66.25	0.00375938623322216\\
66.26	0.00375915418547465\\
66.27	0.00375892199805024\\
66.28	0.00375868967083603\\
66.29	0.00375845720371901\\
66.3	0.0037582245965861\\
66.31	0.00375799184932411\\
66.32	0.00375775896181975\\
66.33	0.00375752593395957\\
66.34	0.00375729276563001\\
66.35	0.00375705945671732\\
66.36	0.00375682600710759\\
66.37	0.00375659241668668\\
66.38	0.00375635868534028\\
66.39	0.00375612481295383\\
66.4	0.00375589079941248\\
66.41	0.00375565664460115\\
66.42	0.00375542234840444\\
66.43	0.00375518791070665\\
66.44	0.00375495333139169\\
66.45	0.00375471861034314\\
66.46	0.00375448374744418\\
66.47	0.00375424874257755\\
66.48	0.00375401359562555\\
66.49	0.00375377830646999\\
66.5	0.00375354287499218\\
66.51	0.00375330730107287\\
66.52	0.00375307158459226\\
66.53	0.0037528357254299\\
66.54	0.00375259972346473\\
66.55	0.00375236357857497\\
66.56	0.00375212729063814\\
66.57	0.003751890859531\\
66.58	0.00375165428512949\\
66.59	0.0037514175673087\\
66.6	0.00375118070594283\\
66.61	0.00375094370090515\\
66.62	0.00375070655206791\\
66.63	0.00375046925930237\\
66.64	0.00375023182247864\\
66.65	0.00374999424146574\\
66.66	0.00374975651613144\\
66.67	0.00374951864634228\\
66.68	0.00374928063196347\\
66.69	0.00374904247285884\\
66.7	0.00374880416889077\\
66.71	0.00374856571992014\\
66.72	0.00374832712580623\\
66.73	0.00374808838640667\\
66.74	0.00374784950157736\\
66.75	0.00374761047117242\\
66.76	0.00374737129504405\\
66.77	0.0037471319730425\\
66.78	0.003746892505016\\
66.79	0.00374665289081058\\
66.8	0.0037464131302701\\
66.81	0.00374617322323606\\
66.82	0.00374593316954757\\
66.83	0.0037456929690412\\
66.84	0.00374545262155092\\
66.85	0.00374521212690795\\
66.86	0.00374497148494069\\
66.87	0.00374473069547459\\
66.88	0.00374448975833201\\
66.89	0.00374424867333214\\
66.9	0.00374400744029085\\
66.91	0.00374376605902058\\
66.92	0.00374352452933018\\
66.93	0.00374328285102478\\
66.94	0.00374304102390566\\
66.95	0.00374279904777011\\
66.96	0.00374255692241126\\
66.97	0.00374231464761793\\
66.98	0.00374207222317448\\
66.99	0.00374182964886062\\
67	0.00374158692445128\\
67.01	0.0037413440497164\\
67.02	0.00374110102442075\\
67.03	0.00374085784832377\\
67.04	0.00374061452117936\\
67.05	0.00374037104273566\\
67.06	0.00374012741273491\\
67.07	0.00373988363091316\\
67.08	0.00373963969700011\\
67.09	0.00373939561071888\\
67.1	0.00373915137187342\\
67.11	0.00373890698028196\\
67.12	0.00373866243576211\\
67.13	0.00373841773813087\\
67.14	0.0037381728872046\\
67.15	0.00373792788279905\\
67.16	0.00373768272472932\\
67.17	0.00373743741280987\\
67.18	0.0037371919468545\\
67.19	0.00373694632667636\\
67.2	0.00373670055208792\\
67.21	0.00373645462290099\\
67.22	0.00373620853892669\\
67.23	0.00373596229997545\\
67.24	0.00373571590585701\\
67.25	0.00373546935638041\\
67.26	0.00373522265135395\\
67.27	0.00373497579058523\\
67.28	0.00373472877388111\\
67.29	0.00373448160104774\\
67.3	0.00373423427189047\\
67.31	0.00373398678621395\\
67.32	0.00373373914382202\\
67.33	0.00373349134451778\\
67.34	0.00373324338810353\\
67.35	0.00373299527438078\\
67.36	0.00373274700315024\\
67.37	0.00373249857421181\\
67.38	0.00373224998736457\\
67.39	0.00373200124240676\\
67.4	0.0037317523391358\\
67.41	0.00373150327734822\\
67.42	0.00373125405683974\\
67.43	0.00373100467740515\\
67.44	0.00373075513883842\\
67.45	0.00373050544093256\\
67.46	0.00373025558347974\\
67.47	0.00373000556627114\\
67.48	0.00372975538909709\\
67.49	0.00372950505174692\\
67.5	0.00372925455400905\\
67.51	0.00372900389567089\\
67.52	0.00372875307651891\\
67.53	0.00372850209633858\\
67.54	0.00372825095491438\\
67.55	0.00372799965202974\\
67.56	0.00372774818746711\\
67.57	0.00372749656100786\\
67.58	0.00372724477243233\\
67.59	0.00372699282151976\\
67.6	0.00372674070804836\\
67.61	0.00372648843179518\\
67.62	0.00372623599253621\\
67.63	0.00372598339004628\\
67.64	0.00372573062409909\\
67.65	0.0037254776944672\\
67.66	0.00372522460092197\\
67.67	0.00372497134323358\\
67.68	0.003724717921171\\
67.69	0.00372446433450201\\
67.7	0.00372421058299311\\
67.71	0.00372395666640957\\
67.72	0.00372370258451538\\
67.73	0.00372344833707324\\
67.74	0.00372319392384455\\
67.75	0.00372293934458938\\
67.76	0.00372268459906646\\
67.77	0.00372242968703314\\
67.78	0.00372217460824541\\
67.79	0.00372191936245785\\
67.8	0.0037216639494236\\
67.81	0.0037214083688944\\
67.82	0.00372115262062049\\
67.83	0.00372089670435063\\
67.84	0.0037206406198321\\
67.85	0.00372038436681065\\
67.86	0.00372012794503046\\
67.87	0.00371987135423415\\
67.88	0.00371961459416275\\
67.89	0.00371935766455569\\
67.9	0.00371910056515072\\
67.91	0.00371884329568396\\
67.92	0.00371858585588983\\
67.93	0.00371832824550104\\
67.94	0.00371807046424857\\
67.95	0.00371781251186161\\
67.96	0.00371755438806758\\
67.97	0.00371729609259209\\
67.98	0.00371703762515888\\
67.99	0.00371677898548983\\
68	0.00371652017330495\\
68.01	0.00371626118832227\\
68.02	0.00371600203025789\\
68.03	0.00371574269882593\\
68.04	0.00371548319373847\\
68.05	0.00371522351470556\\
68.06	0.00371496366143516\\
68.07	0.00371470363363313\\
68.08	0.00371444343100316\\
68.09	0.00371418305324681\\
68.1	0.00371392250006337\\
68.11	0.00371366177114994\\
68.12	0.0037134008662013\\
68.13	0.00371313978490993\\
68.14	0.00371287852696597\\
68.15	0.00371261709205715\\
68.16	0.00371235547986879\\
68.17	0.00371209369008373\\
68.18	0.00371183172238233\\
68.19	0.00371156957644238\\
68.2	0.00371130725193912\\
68.21	0.00371104474854513\\
68.22	0.00371078206593035\\
68.23	0.00371051920376199\\
68.24	0.00371025616170454\\
68.25	0.00370999293941966\\
68.26	0.00370972953656618\\
68.27	0.00370946595280006\\
68.28	0.0037092021877743\\
68.29	0.00370893824113893\\
68.3	0.00370867411254096\\
68.31	0.00370840980162429\\
68.32	0.00370814530802973\\
68.33	0.00370788063139488\\
68.34	0.00370761577135411\\
68.35	0.00370735072753852\\
68.36	0.00370708549957584\\
68.37	0.00370682008709044\\
68.38	0.00370655448970321\\
68.39	0.00370628870703153\\
68.4	0.00370602273868923\\
68.41	0.00370575658428649\\
68.42	0.00370549024342983\\
68.43	0.00370522371572198\\
68.44	0.00370495700076188\\
68.45	0.0037046900981446\\
68.46	0.00370442300746124\\
68.47	0.00370415572829891\\
68.48	0.00370388826024064\\
68.49	0.00370362060286529\\
68.5	0.00370335275574752\\
68.51	0.00370308471845771\\
68.52	0.00370281649056184\\
68.53	0.00370254807162149\\
68.54	0.00370227946119368\\
68.55	0.00370201065883089\\
68.56	0.00370174166408086\\
68.57	0.00370147247648664\\
68.58	0.00370120309558641\\
68.59	0.00370093352091342\\
68.6	0.00370066375199594\\
68.61	0.00370039378835712\\
68.62	0.00370012362951493\\
68.63	0.00369985327498209\\
68.64	0.00369958272426593\\
68.65	0.00369931197686831\\
68.66	0.00369904103228555\\
68.67	0.0036987698900083\\
68.68	0.00369849854952147\\
68.69	0.00369822701030407\\
68.7	0.00369795527182919\\
68.71	0.00369768333356383\\
68.72	0.0036974111949688\\
68.73	0.00369713885549862\\
68.74	0.00369686631460143\\
68.75	0.00369659357171882\\
68.76	0.00369632062628577\\
68.77	0.00369604747773048\\
68.78	0.00369577412547427\\
68.79	0.00369550056893148\\
68.8	0.00369522680750929\\
68.81	0.00369495284060763\\
68.82	0.00369467866761903\\
68.83	0.0036944042879285\\
68.84	0.00369412970091335\\
68.85	0.00369385490594313\\
68.86	0.00369357990237938\\
68.87	0.00369330468957557\\
68.88	0.00369302926687693\\
68.89	0.00369275363362024\\
68.9	0.00369247778913376\\
68.91	0.00369220173273701\\
68.92	0.00369192546374064\\
68.93	0.00369164898144624\\
68.94	0.00369137228514617\\
68.95	0.00369109537412343\\
68.96	0.00369081824765141\\
68.97	0.00369054090499378\\
68.98	0.00369026334540426\\
68.99	0.00368998556812647\\
69	0.00368970757239368\\
69.01	0.00368942935742867\\
69.02	0.0036891509224435\\
69.03	0.00368887226663934\\
69.04	0.00368859338920621\\
69.05	0.00368831428932278\\
69.06	0.0036880349661562\\
69.07	0.00368775541886181\\
69.08	0.00368747564658296\\
69.09	0.00368719564845077\\
69.1	0.00368691542358386\\
69.11	0.00368663497108815\\
69.12	0.0036863542900566\\
69.13	0.00368607337956893\\
69.14	0.00368579223869142\\
69.15	0.00368551086647657\\
69.16	0.00368522926196291\\
69.17	0.00368494742417466\\
69.18	0.00368466535212148\\
69.19	0.00368438304479818\\
69.2	0.00368410050118444\\
69.21	0.00368381772024446\\
69.22	0.00368353470092672\\
69.23	0.00368325144216361\\
69.24	0.00368296794287115\\
69.25	0.00368268420194864\\
69.26	0.00368240021827834\\
69.27	0.0036821159907251\\
69.28	0.00368183151813606\\
69.29	0.00368154679934025\\
69.3	0.00368126183314822\\
69.31	0.00368097661835172\\
69.32	0.00368069115372326\\
69.33	0.00368040543801577\\
69.34	0.00368011946996214\\
69.35	0.00367983324827488\\
69.36	0.00367954677164566\\
69.37	0.00367926003874489\\
69.38	0.00367897304822131\\
69.39	0.0036786857987015\\
69.4	0.00367839828878947\\
69.41	0.00367811051706615\\
69.42	0.00367782248208898\\
69.43	0.00367753418239132\\
69.44	0.00367724561648207\\
69.45	0.00367695678284507\\
69.46	0.00367666767993861\\
69.47	0.00367637830619492\\
69.48	0.00367608866001959\\
69.49	0.00367579873979101\\
69.5	0.00367550854385984\\
69.51	0.0036752180705484\\
69.52	0.00367492731815006\\
69.53	0.00367463628492865\\
69.54	0.00367434496911782\\
69.55	0.00367405336892042\\
69.56	0.00367376148250783\\
69.57	0.00367346930801929\\
69.58	0.00367317684356122\\
69.59	0.00367288408720653\\
69.6	0.00367259103699387\\
69.61	0.00367229769092693\\
69.62	0.00367200404697367\\
69.63	0.00367171010306556\\
69.64	0.00367141585709679\\
69.65	0.00367112130692344\\
69.66	0.0036708264503627\\
69.67	0.00367053128519199\\
69.68	0.00367023580914812\\
69.69	0.00366994001992639\\
69.7	0.00366964391517967\\
69.71	0.0036693474925175\\
69.72	0.00366905074950514\\
69.73	0.00366875368366257\\
69.74	0.00366845629246353\\
69.75	0.00366815857333445\\
69.76	0.00366786052365349\\
69.77	0.00366756214074935\\
69.78	0.00366726342190031\\
69.79	0.00366696436433298\\
69.8	0.00366666496522128\\
69.81	0.00366636522168513\\
69.82	0.00366606513078936\\
69.83	0.00366576468954241\\
69.84	0.00366546389489509\\
69.85	0.00366516274373926\\
69.86	0.00366486123290655\\
69.87	0.00366455935916695\\
69.88	0.00366425711922745\\
69.89	0.0036639545097306\\
69.9	0.00366365152725306\\
69.91	0.00366334816830409\\
69.92	0.00366304442932402\\
69.93	0.00366274030668267\\
69.94	0.00366243579667775\\
69.95	0.00366213089553323\\
69.96	0.00366182559939758\\
69.97	0.00366151990434211\\
69.98	0.00366121380635914\\
69.99	0.00366090730136022\\
70	0.00366060038517421\\
70.01	0.00366029305354543\\
70.02	0.00365998530213165\\
70.03	0.00365967712650209\\
70.04	0.00365936852213538\\
70.05	0.00365905948441739\\
70.06	0.00365875000863915\\
70.07	0.00365844008999455\\
70.08	0.00365812972357808\\
70.09	0.00365781890438253\\
70.1	0.00365750762729657\\
70.11	0.00365719588710228\\
70.12	0.00365688367847266\\
70.13	0.00365657099596902\\
70.14	0.00365625783403835\\
70.15	0.00365594418701061\\
70.16	0.00365563004909591\\
70.17	0.00365531541438171\\
70.18	0.00365500027682983\\
70.19	0.00365468463027347\\
70.2	0.00365436846841415\\
70.21	0.00365405178481851\\
70.22	0.00365373457291506\\
70.23	0.00365341682599091\\
70.24	0.00365309853718829\\
70.25	0.00365277969950107\\
70.26	0.00365246030577121\\
70.27	0.00365214034868499\\
70.28	0.00365181982076928\\
70.29	0.00365149871438764\\
70.3	0.00365117702173636\\
70.31	0.00365085473484035\\
70.32	0.00365053184554892\\
70.33	0.00365020834553154\\
70.34	0.00364988422627337\\
70.35	0.00364955947907073\\
70.36	0.00364923409502647\\
70.37	0.00364890806504517\\
70.38	0.00364858137982824\\
70.39	0.00364825402986891\\
70.4	0.00364792600544704\\
70.41	0.0036475972966238\\
70.42	0.00364726789323627\\
70.43	0.00364693778489182\\
70.44	0.00364660696096241\\
70.45	0.00364627541057861\\
70.46	0.00364594312262367\\
70.47	0.00364561008572721\\
70.48	0.00364527628825892\\
70.49	0.00364494171832194\\
70.5	0.00364460636374619\\
70.51	0.00364427021208148\\
70.52	0.00364393325059034\\
70.53	0.00364359546624082\\
70.54	0.00364325684569898\\
70.55	0.00364291737532121\\
70.56	0.00364257704114633\\
70.57	0.00364223582888754\\
70.58	0.00364189372392403\\
70.59	0.00364155071129249\\
70.6	0.0036412067756783\\
70.61	0.00364086190140653\\
70.62	0.00364051607243265\\
70.63	0.00364016927233307\\
70.64	0.00363982148429529\\
70.65	0.0036394726911079\\
70.66	0.00363912287515026\\
70.67	0.0036387720183819\\
70.68	0.0036384201023316\\
70.69	0.00363806710808623\\
70.7	0.00363771301627922\\
70.71	0.00363735780707879\\
70.72	0.00363700146017576\\
70.73	0.00363664395477108\\
70.74	0.00363628526956303\\
70.75	0.00363592538273403\\
70.76	0.00363556427193707\\
70.77	0.00363520191428185\\
70.78	0.00363483828632041\\
70.79	0.00363447336403249\\
70.8	0.00363410712281035\\
70.81	0.00363373953744329\\
70.82	0.00363337058210167\\
70.83	0.00363300023032048\\
70.84	0.00363262845498249\\
70.85	0.00363225522830089\\
70.86	0.00363188052180149\\
70.87	0.00363150430630432\\
70.88	0.00363112655190489\\
70.89	0.00363074722795473\\
70.9	0.00363036630304151\\
70.91	0.00362998374496854\\
70.92	0.00362959952073376\\
70.93	0.00362921359650801\\
70.94	0.00362882593761284\\
70.95	0.00362843650849757\\
70.96	0.0036280452727158\\
70.97	0.00362765219290112\\
70.98	0.0036272572307423\\
70.99	0.00362686034695764\\
71	0.00362646150126865\\
71.01	0.00362606065237299\\
71.02	0.00362565775791659\\
71.03	0.00362525277446503\\
71.04	0.00362484565747405\\
71.05	0.0036244363612593\\
71.06	0.00362402483896512\\
71.07	0.00362361104253248\\
71.08	0.00362319492266606\\
71.09	0.00362277642880025\\
71.1	0.00362235550906429\\
71.11	0.00362193211024637\\
71.12	0.00362150617775667\\
71.13	0.00362107765558941\\
71.14	0.0036206464862837\\
71.15	0.00362021261088338\\
71.16	0.00361977596889558\\
71.17	0.0036193364982482\\
71.18	0.00361889413524607\\
71.19	0.00361844881452588\\
71.2	0.00361800046900982\\
71.21	0.00361754902985784\\
71.22	0.00361709442641855\\
71.23	0.00361663658617865\\
71.24	0.00361617543471097\\
71.25	0.00361571089562095\\
71.26	0.00361524289049154\\
71.27	0.00361477133882651\\
71.28	0.00361429615799217\\
71.29	0.00361381726315725\\
71.3	0.00361333456723119\\
71.31	0.00361284798080048\\
71.32	0.00361235741206325\\
71.33	0.00361186276676183\\
71.34	0.00361136394811342\\
71.35	0.00361086085673876\\
71.36	0.00361035339058852\\
71.37	0.00360984144486778\\
71.38	0.00360932491195804\\
71.39	0.00360880368133717\\
71.4	0.00360827763949679\\
71.41	0.00360774666985735\\
71.42	0.00360721065268063\\
71.43	0.00360666946497974\\
71.44	0.00360612298042635\\
71.45	0.00360557106925525\\
71.46	0.00360501359816605\\
71.47	0.00360445043022201\\
71.48	0.00360388142474581\\
71.49	0.00360330643721224\\
71.5	0.00360272531913777\\
71.51	0.00360213791796673\\
71.52	0.00360154407695417\\
71.53	0.00360094363504524\\
71.54	0.00360033642675093\\
71.55	0.00359972228202009\\
71.56	0.00359910102610775\\
71.57	0.00359847247943934\\
71.58	0.00359783697350581\\
71.59	0.00359720111091926\\
71.6	0.0035965648913805\\
71.61	0.00359592831460129\\
71.62	0.0035952913803048\\
71.63	0.00359465408822606\\
71.64	0.00359401643811257\\
71.65	0.00359337842972476\\
71.66	0.00359274006283655\\
71.67	0.00359210133723597\\
71.68	0.00359146225272568\\
71.69	0.00359082280912361\\
71.7	0.00359018300626359\\
71.71	0.00358954284399598\\
71.72	0.00358890232218837\\
71.73	0.00358826144072621\\
71.74	0.0035876201995136\\
71.75	0.00358697859847395\\
71.76	0.0035863366375508\\
71.77	0.00358569431670856\\
71.78	0.00358505163593335\\
71.79	0.00358440859523381\\
71.8	0.00358376519464198\\
71.81	0.00358312143421419\\
71.82	0.00358247731403198\\
71.83	0.00358183283420302\\
71.84	0.00358118799486212\\
71.85	0.00358054279617224\\
71.86	0.00357989723832553\\
71.87	0.00357925132154438\\
71.88	0.00357860504608257\\
71.89	0.00357795841222638\\
71.9	0.00357731142029582\\
71.91	0.0035766640706458\\
71.92	0.00357601636366742\\
71.93	0.00357536829978925\\
71.94	0.00357471987947873\\
71.95	0.00357407110324345\\
71.96	0.00357342197163268\\
71.97	0.0035727724852388\\
71.98	0.00357212264469881\\
71.99	0.00357147245069594\\
72	0.00357082190396121\\
72.01	0.00357017100527519\\
72.02	0.00356951975546962\\
72.03	0.00356886815542928\\
72.04	0.00356821620609379\\
72.05	0.00356756390845948\\
72.06	0.0035669112635814\\
72.07	0.0035662582725753\\
72.08	0.00356560493661971\\
72.09	0.00356495125695814\\
72.1	0.0035642972349012\\
72.11	0.00356364287182899\\
72.12	0.00356298816919341\\
72.13	0.00356233312852059\\
72.14	0.0035616777514134\\
72.15	0.0035610220395541\\
72.16	0.00356036599470692\\
72.17	0.00355970961872092\\
72.18	0.00355905291353277\\
72.19	0.00355839588116971\\
72.2	0.00355773852375259\\
72.21	0.00355708084349898\\
72.22	0.00355642284272646\\
72.23	0.00355576452385584\\
72.24	0.00355510588941472\\
72.25	0.00355444694204096\\
72.26	0.00355378768448637\\
72.27	0.00355312811962049\\
72.28	0.00355246825043448\\
72.29	0.00355180808004515\\
72.3	0.00355114761169912\\
72.31	0.00355048684877709\\
72.32	0.00354982579479827\\
72.33	0.00354916445342496\\
72.34	0.00354850282846724\\
72.35	0.00354784092388782\\
72.36	0.00354717874380711\\
72.37	0.00354651629250834\\
72.38	0.00354585357444295\\
72.39	0.00354519059423608\\
72.4	0.00354452735669227\\
72.41	0.00354386386680134\\
72.42	0.00354320012974444\\
72.43	0.00354253615090037\\
72.44	0.00354187193585196\\
72.45	0.0035412074903928\\
72.46	0.0035405428205341\\
72.47	0.00353987793251184\\
72.48	0.00353921283279403\\
72.49	0.00353854752808837\\
72.5	0.00353788202535004\\
72.51	0.00353721633178975\\
72.52	0.00353655045488211\\
72.53	0.00353588440237423\\
72.54	0.00353521818229461\\
72.55	0.00353455180296227\\
72.56	0.00353388527299631\\
72.57	0.0035332186013256\\
72.58	0.00353255179719894\\
72.59	0.00353188487019548\\
72.6	0.00353121783023544\\
72.61	0.00353055068759132\\
72.62	0.00352988345289929\\
72.63	0.0035292161371711\\
72.64	0.00352854875180633\\
72.65	0.00352788130860502\\
72.66	0.00352721381978072\\
72.67	0.00352654629797401\\
72.68	0.00352587875626642\\
72.69	0.00352521120819482\\
72.7	0.00352454366776628\\
72.71	0.00352387614947344\\
72.72	0.00352320866831034\\
72.73	0.00352254123978884\\
72.74	0.00352187387995551\\
72.75	0.00352120660540601\\
72.76	0.00352053943328332\\
72.77	0.00351987238129443\\
72.78	0.00351920546772744\\
72.79	0.0035185387114692\\
72.8	0.00351787213202353\\
72.81	0.00351720574952991\\
72.82	0.00351653958478275\\
72.83	0.00351587365925126\\
72.84	0.00351520799509985\\
72.85	0.00351454261520925\\
72.86	0.00351387754319815\\
72.87	0.00351321280344552\\
72.88	0.00351254842111362\\
72.89	0.00351188442217169\\
72.9	0.00351122083342032\\
72.91	0.00351055768251656\\
72.92	0.00350989499799982\\
72.93	0.00350923280931847\\
72.94	0.00350857114685728\\
72.95	0.00350791004196569\\
72.96	0.00350724952698689\\
72.97	0.00350658963528779\\
72.98	0.00350593040128989\\
72.99	0.003505270953085\\
73	0.00350461102128345\\
73.01	0.0035039506031637\\
73.02	0.0035032896959746\\
73.03	0.00350262829693553\\
73.04	0.00350196640323653\\
73.05	0.00350130401203851\\
73.06	0.00350064112047339\\
73.07	0.00349997772564443\\
73.08	0.00349931382462642\\
73.09	0.00349864941446601\\
73.1	0.00349798449218209\\
73.11	0.00349731905476614\\
73.12	0.00349665309918262\\
73.13	0.00349598662236952\\
73.14	0.00349531962123879\\
73.15	0.00349465209267694\\
73.16	0.00349398403354562\\
73.17	0.0034933154406823\\
73.18	0.00349264631090096\\
73.19	0.00349197664099287\\
73.2	0.00349130642772739\\
73.21	0.00349063566785289\\
73.22	0.00348996435809766\\
73.23	0.00348929249517096\\
73.24	0.00348862007576409\\
73.25	0.00348794709655153\\
73.26	0.0034872735541922\\
73.27	0.00348659944533077\\
73.28	0.00348592476659907\\
73.29	0.00348524951461752\\
73.3	0.00348457368599677\\
73.31	0.0034838972773393\\
73.32	0.00348322028524126\\
73.33	0.00348254270629427\\
73.34	0.00348186453708746\\
73.35	0.00348118577420947\\
73.36	0.00348050641425076\\
73.37	0.00347982645380582\\
73.38	0.00347914588947573\\
73.39	0.00347846471787066\\
73.4	0.00347778293561259\\
73.41	0.00347710053933824\\
73.42	0.00347641752570199\\
73.43	0.00347573389137911\\
73.44	0.00347504963306904\\
73.45	0.00347436474749888\\
73.46	0.00347367923142711\\
73.47	0.00347299308164733\\
73.48	0.00347230629499241\\
73.49	0.00347161886833862\\
73.5	0.00347093079861013\\
73.51	0.00347024208278362\\
73.52	0.00346955271789316\\
73.53	0.00346886270103528\\
73.54	0.00346817202937433\\
73.55	0.00346748070014805\\
73.56	0.00346678871067339\\
73.57	0.00346609605835264\\
73.58	0.0034654027406798\\
73.59	0.00346470875524724\\
73.6	0.00346401409975271\\
73.61	0.00346331877200655\\
73.62	0.00346262276993938\\
73.63	0.00346192609161001\\
73.64	0.00346122873521369\\
73.65	0.00346053069909085\\
73.66	0.00345983198173605\\
73.67	0.00345913258180748\\
73.68	0.00345843249813672\\
73.69	0.00345773172973904\\
73.7	0.0034570302758241\\
73.71	0.00345632813580702\\
73.72	0.00345562530932011\\
73.73	0.00345492179622489\\
73.74	0.00345421759662473\\
73.75	0.00345351271087801\\
73.76	0.00345280713961182\\
73.77	0.00345210088373615\\
73.78	0.00345139394445884\\
73.79	0.00345068632330098\\
73.8	0.00344997802211301\\
73.81	0.00344926904309147\\
73.82	0.00344855938879647\\
73.83	0.0034478490621698\\
73.84	0.00344713806655383\\
73.85	0.00344642640571113\\
73.86	0.00344571408384494\\
73.87	0.00344500110562037\\
73.88	0.00344428747618653\\
73.89	0.00344357320119952\\
73.9	0.00344285828684629\\
73.91	0.00344214273986953\\
73.92	0.00344142656759343\\
73.93	0.00344070977795058\\
73.94	0.00343999237950981\\
73.95	0.00343927438150524\\
73.96	0.00343855579386634\\
73.97	0.00343783662724926\\
73.98	0.00343711689306936\\
73.99	0.00343639660353491\\
74	0.00343567577168228\\
74.01	0.00343495441141227\\
74.02	0.00343423253752802\\
74.03	0.0034335101657743\\
74.04	0.00343278731287825\\
74.05	0.00343206399659183\\
74.06	0.00343134023573572\\
74.07	0.00343061605024504\\
74.08	0.0034298914612167\\
74.09	0.0034291664909586\\
74.1	0.00342844114126226\\
74.11	0.00342771541193641\\
74.12	0.00342698930280977\\
74.13	0.00342626281373202\\
74.14	0.00342553594457469\\
74.15	0.0034248086952323\\
74.16	0.00342408106562326\\
74.17	0.00342335305569106\\
74.18	0.00342262466540534\\
74.19	0.00342189589476307\\
74.2	0.00342116674378974\\
74.21	0.00342043721254064\\
74.22	0.00341970730110212\\
74.23	0.00341897700959296\\
74.24	0.00341824633816576\\
74.25	0.00341751528700837\\
74.26	0.00341678385634538\\
74.27	0.00341605204643972\\
74.28	0.00341531985759417\\
74.29	0.00341458729015313\\
74.3	0.00341385434450426\\
74.31	0.0034131210210803\\
74.32	0.00341238732036092\\
74.33	0.00341165324287463\\
74.34	0.00341091878920075\\
74.35	0.00341018395997146\\
74.36	0.00340944875587399\\
74.37	0.00340871317765272\\
74.38	0.00340797722611154\\
74.39	0.00340724090211615\\
74.4	0.00340650420659657\\
74.41	0.00340576714054961\\
74.42	0.00340502970504153\\
74.43	0.00340429190121073\\
74.44	0.00340355373027055\\
74.45	0.00340281519351221\\
74.46	0.00340207629230777\\
74.47	0.00340133702811328\\
74.48	0.00340059740247198\\
74.49	0.00339985741701763\\
74.5	0.00339911707347799\\
74.51	0.00339837637367835\\
74.52	0.00339763531954524\\
74.53	0.00339689391311027\\
74.54	0.00339615215651404\\
74.55	0.00339541005201029\\
74.56	0.00339466760197006\\
74.57	0.00339392480888614\\
74.58	0.00339318167537756\\
74.59	0.00339243820419429\\
74.6	0.00339169439822207\\
74.61	0.00339095026048746\\
74.62	0.00339020579416299\\
74.63	0.00338946100257255\\
74.64	0.00338871588919692\\
74.65	0.00338797045767954\\
74.66	0.00338722471183242\\
74.67	0.00338647865564228\\
74.68	0.00338573229327692\\
74.69	0.00338498562909178\\
74.7	0.00338423866763671\\
74.71	0.00338349141366302\\
74.72	0.00338274387213073\\
74.73	0.00338199604821609\\
74.74	0.00338124794731931\\
74.75	0.00338049957507266\\
74.76	0.00337975093734869\\
74.77	0.00337900204026887\\
74.78	0.00337825289021244\\
74.79	0.00337750349382559\\
74.8	0.00337675385803097\\
74.81	0.00337600394381009\\
74.82	0.00337525363984766\\
74.83	0.003374502945545\\
74.84	0.00337375186030001\\
74.85	0.00337300038350716\\
74.86	0.00337224851455745\\
74.87	0.00337149625283841\\
74.88	0.0033707435977341\\
74.89	0.00336999054862505\\
74.9	0.0033692371048883\\
74.91	0.00336848326589735\\
74.92	0.00336772903102215\\
74.93	0.0033669743996291\\
74.94	0.00336621937108104\\
74.95	0.00336546394473721\\
74.96	0.00336470811995328\\
74.97	0.0033639518960813\\
74.98	0.00336319527246974\\
74.99	0.00336243824846342\\
75	0.00336168082340355\\
75.01	0.00336092299662772\\
75.02	0.00336016476746987\\
75.03	0.00335940613526031\\
75.04	0.0033586470993257\\
75.05	0.00335788765898904\\
75.06	0.00335712781356971\\
75.07	0.00335636756238344\\
75.08	0.0033556069047423\\
75.09	0.00335484583995471\\
75.1	0.00335408436732548\\
75.11	0.00335332248615576\\
75.12	0.00335256019574308\\
75.13	0.00335179749538136\\
75.14	0.0033510343843609\\
75.15	0.00335027086196841\\
75.16	0.003349506927487\\
75.17	0.00334874258019622\\
75.18	0.00334797781937208\\
75.19	0.00334721264428702\\
75.2	0.00334644705420999\\
75.21	0.00334568104840645\\
75.22	0.00334491462613836\\
75.23	0.00334414778666427\\
75.24	0.00334338052923929\\
75.25	0.00334261285311517\\
75.26	0.00334184475754029\\
75.27	0.00334107624175972\\
75.28	0.00334030730501525\\
75.29	0.00333953794654545\\
75.3	0.00333876816558568\\
75.31	0.00333799796136815\\
75.32	0.00333722733312199\\
75.33	0.00333645628007327\\
75.34	0.00333568480144508\\
75.35	0.00333491289645757\\
75.36	0.00333414056432803\\
75.37	0.00333336780427093\\
75.38	0.00333259461549801\\
75.39	0.00333182099721836\\
75.4	0.00333104694863847\\
75.41	0.00333027246896232\\
75.42	0.00332949755739147\\
75.43	0.00332872221312516\\
75.44	0.00332794643536035\\
75.45	0.00332717022329189\\
75.46	0.00332639357611257\\
75.47	0.00332561649301325\\
75.48	0.00332483897318295\\
75.49	0.00332406101580901\\
75.5	0.00332328262007713\\
75.51	0.0033225037851716\\
75.52	0.00332172451027534\\
75.53	0.0033209447945701\\
75.54	0.00332016463723658\\
75.55	0.00331938403745458\\
75.56	0.00331860299440315\\
75.57	0.00331782150726077\\
75.58	0.0033170395752055\\
75.59	0.00331625719741518\\
75.6	0.00331547437306759\\
75.61	0.00331469110134065\\
75.62	0.00331390738141263\\
75.63	0.00331312321246232\\
75.64	0.0033123385936693\\
75.65	0.0033115535242141\\
75.66	0.00331076800327848\\
75.67	0.00330998203004565\\
75.68	0.00330919560370052\\
75.69	0.00330840872342993\\
75.7	0.00330762138842298\\
75.71	0.00330683359787125\\
75.72	0.00330604535096913\\
75.73	0.00330525664691408\\
75.74	0.00330446748490694\\
75.75	0.00330367786415231\\
75.76	0.0033028877838588\\
75.77	0.00330209724323942\\
75.78	0.00330130624151193\\
75.79	0.0033005147778992\\
75.8	0.00329972285162961\\
75.81	0.00329893046193739\\
75.82	0.00329813760806312\\
75.83	0.00329734428925404\\
75.84	0.0032965505047646\\
75.85	0.00329575625385682\\
75.86	0.0032949615358008\\
75.87	0.0032941663498752\\
75.88	0.00329337069536773\\
75.89	0.00329257457157569\\
75.9	0.00329177797780645\\
75.91	0.00329098091337806\\
75.92	0.00329018337761978\\
75.93	0.00328938536987269\\
75.94	0.00328858688949028\\
75.95	0.0032877879358391\\
75.96	0.00328698850829937\\
75.97	0.0032861886062657\\
75.98	0.00328538822914774\\
75.99	0.0032845873763709\\
76	0.00328378604737713\\
76.01	0.00328298424162559\\
76.02	0.00328218195859352\\
76.03	0.00328137919777704\\
76.04	0.00328057595869192\\
76.05	0.00327977224087451\\
76.06	0.00327896804388261\\
76.07	0.00327816336729637\\
76.08	0.00327735821071926\\
76.09	0.00327655257377904\\
76.1	0.00327574645612877\\
76.11	0.00327493985744783\\
76.12	0.003274132777443\\
76.13	0.0032733252158496\\
76.14	0.00327251717243258\\
76.15	0.0032717086469877\\
76.16	0.00327089963934281\\
76.17	0.00327009014935898\\
76.18	0.00326928017693189\\
76.19	0.00326846972199312\\
76.2	0.0032676587845115\\
76.21	0.00326684736449454\\
76.22	0.00326603546198988\\
76.23	0.0032652230770868\\
76.24	0.00326441020991773\\
76.25	0.00326359686065985\\
76.26	0.00326278302953677\\
76.27	0.00326196871682015\\
76.28	0.00326115392283149\\
76.29	0.00326033864794391\\
76.3	0.00325952289258396\\
76.31	0.00325870665723358\\
76.32	0.00325788994243201\\
76.33	0.00325707274877784\\
76.34	0.00325625507693105\\
76.35	0.00325543692761519\\
76.36	0.00325461830161956\\
76.37	0.00325379919980149\\
76.38	0.00325297962308866\\
76.39	0.00325215957248152\\
76.4	0.00325133904905579\\
76.41	0.00325051805396495\\
76.42	0.00324969658844293\\
76.43	0.00324887465380678\\
76.44	0.00324805225145948\\
76.45	0.0032472293828928\\
76.46	0.00324640604969021\\
76.47	0.00324558225352999\\
76.48	0.00324475799618832\\
76.49	0.00324393327954249\\
76.5	0.00324310810557426\\
76.51	0.00324228247637321\\
76.52	0.00324145639414032\\
76.53	0.00324062986119155\\
76.54	0.00323980287996154\\
76.55	0.00323897545300752\\
76.56	0.00323814758301314\\
76.57	0.00323731927279265\\
76.58	0.00323649052529498\\
76.59	0.00323566134360809\\
76.6	0.00323483173096337\\
76.61	0.00323400169074023\\
76.62	0.00323317122647073\\
76.63	0.00323234034184445\\
76.64	0.00323150904071342\\
76.65	0.00323067732709724\\
76.66	0.00322984520518834\\
76.67	0.00322901267935737\\
76.68	0.00322817975415878\\
76.69	0.00322734643433654\\
76.7	0.00322651272483\\
76.71	0.00322567863078001\\
76.72	0.00322484415753512\\
76.73	0.00322400931065796\\
76.74	0.00322317409593194\\
76.75	0.00322233851936792\\
76.76	0.0032215025872113\\
76.77	0.00322066630594916\\
76.78	0.00321982968231767\\
76.79	0.00321899272330969\\
76.8	0.0032181554361826\\
76.81	0.00321731782846639\\
76.82	0.00321647990797188\\
76.83	0.00321564168279932\\
76.84	0.0032148031613471\\
76.85	0.00321396435232081\\
76.86	0.00321312526474251\\
76.87	0.00321228590796027\\
76.88	0.00321144629165802\\
76.89	0.00321060642586564\\
76.9	0.00320976632096933\\
76.91	0.00320892598772238\\
76.92	0.0032080854372561\\
76.93	0.00320724468109114\\
76.94	0.00320640373114921\\
76.95	0.00320556259976495\\
76.96	0.00320472129969836\\
76.97	0.00320387984414736\\
76.98	0.00320303824676094\\
76.99	0.0032021965216525\\
77	0.00320135468341366\\
77.01	0.00320051274712849\\
77.02	0.00319967072838803\\
77.03	0.0031988286433054\\
77.04	0.0031979865085312\\
77.05	0.00319714434126941\\
77.06	0.00319630215929371\\
77.07	0.00319545998096436\\
77.08	0.00319461782524545\\
77.09	0.00319377571172269\\
77.1	0.00319293366062176\\
77.11	0.00319209169282708\\
77.12	0.00319124982990123\\
77.13	0.00319040809410485\\
77.14	0.00318956650841715\\
77.15	0.00318872509655697\\
77.16	0.00318788388300449\\
77.17	0.00318704289302354\\
77.18	0.00318620215268455\\
77.19	0.00318536168888814\\
77.2	0.00318452152938945\\
77.21	0.00318368170282307\\
77.22	0.00318284223872879\\
77.23	0.00318200316757801\\
77.24	0.00318116452080094\\
77.25	0.00318032633081459\\
77.26	0.00317948863105155\\
77.27	0.0031786514559896\\
77.28	0.00317781484118216\\
77.29	0.00317697882328963\\
77.3	0.00317614344011164\\
77.31	0.00317530873061603\\
77.32	0.00317447473496592\\
77.33	0.0031736414945539\\
77.34	0.0031728090520373\\
77.35	0.00317197745137431\\
77.36	0.00317114673786123\\
77.37	0.00317031695817069\\
77.38	0.00316948816039101\\
77.39	0.0031686603940666\\
77.4	0.00316783371023952\\
77.41	0.00316700816149229\\
77.42	0.00316618380199174\\
77.43	0.00316536068753426\\
77.44	0.0031645388755922\\
77.45	0.00316371842536164\\
77.46	0.00316289939781149\\
77.47	0.00316208185573399\\
77.48	0.00316126586379656\\
77.49	0.00316045148859521\\
77.5	0.00315963879870937\\
77.51	0.00315882786475831\\
77.52	0.00315801875945915\\
77.53	0.00315721155768645\\
77.54	0.00315640633653355\\
77.55	0.00315560317537562\\
77.56	0.00315480215593439\\
77.57	0.00315400336234489\\
77.58	0.00315320688122388\\
77.59	0.00315241280174034\\
77.6	0.00315162121568789\\
77.61	0.00315083221755925\\
77.62	0.00315004574164604\\
77.63	0.00314925901796608\\
77.64	0.00314847204947123\\
77.65	0.00314768483916832\\
77.66	0.00314689739011981\\
77.67	0.00314610970544434\\
77.68	0.00314532178831732\\
77.69	0.00314453364197156\\
77.7	0.00314374526969775\\
77.71	0.00314295667484509\\
77.72	0.00314216786082179\\
77.73	0.00314137883109562\\
77.74	0.00314058958919438\\
77.75	0.00313980013870644\\
77.76	0.0031390104832812\\
77.77	0.00313822062662952\\
77.78	0.00313743057252418\\
77.79	0.0031366403248003\\
77.8	0.00313584988735568\\
77.81	0.0031350592641512\\
77.82	0.00313426845921116\\
77.83	0.00313347747662353\\
77.84	0.00313268632054027\\
77.85	0.00313189499517754\\
77.86	0.0031311035048159\\
77.87	0.00313031185380047\\
77.88	0.00312952004654108\\
77.89	0.00312872808751231\\
77.9	0.0031279359812535\\
77.91	0.00312714373236882\\
77.92	0.00312635134552708\\
77.93	0.00312555882546173\\
77.94	0.00312476617697059\\
77.95	0.00312397340491562\\
77.96	0.00312318051422268\\
77.97	0.00312238750988107\\
77.98	0.00312159439694314\\
77.99	0.00312080118052377\\
78	0.00312000786579972\\
78.01	0.00311921445800905\\
78.02	0.00311842096245027\\
78.03	0.00311762738448149\\
78.04	0.00311683372951951\\
78.05	0.00311604000303874\\
78.06	0.00311524621057003\\
78.07	0.00311445235769942\\
78.08	0.00311365845006675\\
78.09	0.00311286449336411\\
78.1	0.00311207049333426\\
78.11	0.00311127645576885\\
78.12	0.00311048238650646\\
78.13	0.00310968829143058\\
78.14	0.0031088941764674\\
78.15	0.00310810004758341\\
78.16	0.00310730591078288\\
78.17	0.00310651177210512\\
78.18	0.00310571763762159\\
78.19	0.00310492351343278\\
78.2	0.00310412940566494\\
78.21	0.00310333532046655\\
78.22	0.00310254126400458\\
78.23	0.00310174724246056\\
78.24	0.00310095326202633\\
78.25	0.00310015932889963\\
78.26	0.00309936544927935\\
78.27	0.00309857162936056\\
78.28	0.0030977778753292\\
78.29	0.00309698419335657\\
78.3	0.00309619058959337\\
78.31	0.00309539707016354\\
78.32	0.00309460364115766\\
78.33	0.00309381030862611\\
78.34	0.00309301707857173\\
78.35	0.00309222395694222\\
78.36	0.00309143094962204\\
78.37	0.00309063806242396\\
78.38	0.00308984530108015\\
78.39	0.0030890526712328\\
78.4	0.0030882601784243\\
78.41	0.0030874678280869\\
78.42	0.00308667562553191\\
78.43	0.00308588357593826\\
78.44	0.00308509168434063\\
78.45	0.00308429995561692\\
78.46	0.00308350839447515\\
78.47	0.0030827170054397\\
78.48	0.00308192579283696\\
78.49	0.00308113476078024\\
78.5	0.00308034391315399\\
78.51	0.0030795532535973\\
78.52	0.00307876278548663\\
78.53	0.00307797251191774\\
78.54	0.00307718243568677\\
78.55	0.0030763925592705\\
78.56	0.00307560288480572\\
78.57	0.00307481341406755\\
78.58	0.00307402414844701\\
78.59	0.00307323508892736\\
78.6	0.00307244623605952\\
78.61	0.00307165758993638\\
78.62	0.00307086915016597\\
78.63	0.00307008091584342\\
78.64	0.00306929288552178\\
78.65	0.00306850505718144\\
78.66	0.0030677174281984\\
78.67	0.00306692999531102\\
78.68	0.00306614275458538\\
78.69	0.00306535570137922\\
78.7	0.00306456883030426\\
78.71	0.00306378213518698\\
78.72	0.0030629956090277\\
78.73	0.00306220924395793\\
78.74	0.00306142303119598\\
78.75	0.00306063696100063\\
78.76	0.00305985102262292\\
78.77	0.00305906520425594\\
78.78	0.00305827949298245\\
78.79	0.00305749387472046\\
78.8	0.00305670833416643\\
78.81	0.00305592285473622\\
78.82	0.0030551374185036\\
78.83	0.00305435200613621\\
78.84	0.00305356659682892\\
78.85	0.00305278116823447\\
78.86	0.00305199569639135\\
78.87	0.00305121015564863\\
78.88	0.00305042451858789\\
78.89	0.00304963875594189\\
78.9	0.00304885283651003\\
78.91	0.00304806672707038\\
78.92	0.00304728039228823\\
78.93	0.00304649379462091\\
78.94	0.00304570689421888\\
78.95	0.00304491964882284\\
78.96	0.0030441320136568\\
78.97	0.00304334394131681\\
78.98	0.00304255538165541\\
78.99	0.0030417662816614\\
79	0.00304097662640358\\
79.01	0.00304018641402235\\
79.02	0.00303939564263039\\
79.03	0.00303860431031211\\
79.04	0.00303781241512298\\
79.05	0.00303701995508898\\
79.06	0.00303622692820592\\
79.07	0.0030354333324388\\
79.08	0.00303463916572115\\
79.09	0.00303384442595436\\
79.1	0.00303304911100695\\
79.11	0.00303225321871392\\
79.12	0.00303145674687593\\
79.13	0.00303065969325864\\
79.14	0.00302986205559189\\
79.15	0.00302906383156894\\
79.16	0.00302826501884565\\
79.17	0.0030274656150397\\
79.18	0.00302666561772969\\
79.19	0.00302586502445432\\
79.2	0.00302506383271152\\
79.21	0.00302426203995751\\
79.22	0.00302345964360588\\
79.23	0.00302265664102667\\
79.24	0.00302185302954541\\
79.25	0.00302104880644207\\
79.26	0.00302024396895008\\
79.27	0.00301943851425532\\
79.28	0.003018632439495\\
79.29	0.00301782574175659\\
79.3	0.00301701841807672\\
79.31	0.00301621046543998\\
79.32	0.00301540188077784\\
79.33	0.00301459266096733\\
79.34	0.00301378280282993\\
79.35	0.00301297230313022\\
79.36	0.00301216115857461\\
79.37	0.00301134936581005\\
79.38	0.00301053692142264\\
79.39	0.00300972382193625\\
79.4	0.00300891006381111\\
79.41	0.00300809564344233\\
79.42	0.00300728055715843\\
79.43	0.0030064648012198\\
79.44	0.0030056483718171\\
79.45	0.00300483126506974\\
79.46	0.00300401347702413\\
79.47	0.00300319500365208\\
79.48	0.003002375840849\\
79.49	0.00300155598443221\\
79.5	0.00300073543013903\\
79.51	0.00299991417362501\\
79.52	0.00299909221046197\\
79.53	0.00299826953613606\\
79.54	0.00299744614604577\\
79.55	0.00299662203549986\\
79.56	0.00299579719971528\\
79.57	0.002994971633815\\
79.58	0.0029941453328258\\
79.59	0.00299331829167602\\
79.6	0.00299249050519323\\
79.61	0.00299166196810184\\
79.62	0.00299083267502072\\
79.63	0.00299000262046061\\
79.64	0.00298917179882163\\
79.65	0.00298834020439066\\
79.66	0.00298750783133856\\
79.67	0.00298667467371753\\
79.68	0.0029858407254582\\
79.69	0.00298500598036676\\
79.7	0.00298417043212198\\
79.71	0.00298333407427217\\
79.72	0.00298249690023204\\
79.73	0.00298165890327953\\
79.74	0.0029808200765525\\
79.75	0.00297998041304533\\
79.76	0.00297913990560555\\
79.77	0.00297829854693023\\
79.78	0.00297745632956237\\
79.79	0.00297661324588719\\
79.8	0.0029757692881283\\
79.81	0.00297492444834379\\
79.82	0.00297407871842221\\
79.83	0.00297323209007848\\
79.84	0.00297238455484961\\
79.85	0.00297153610409046\\
79.86	0.00297068672896921\\
79.87	0.00296983642046288\\
79.88	0.0029689851693526\\
79.89	0.00296813296621889\\
79.9	0.00296727980143669\\
79.91	0.00296642566517034\\
79.92	0.00296557054736842\\
79.93	0.00296471443775847\\
79.94	0.0029638573258415\\
79.95	0.0029629992008865\\
79.96	0.00296214005192463\\
79.97	0.00296127986774342\\
79.98	0.00296041863688073\\
79.99	0.00295955634761859\\
80	0.00295869298797683\\
80.01	0.00295782854570667\\
};
\addplot [color=red,solid]
  table[row sep=crcr]{%
80.01	0.00295782854570667\\
80.02	0.00295696300828396\\
80.03	0.00295609636290242\\
80.04	0.0029552285964666\\
80.05	0.00295435969558469\\
80.06	0.00295348964656114\\
80.07	0.0029526184353891\\
80.08	0.00295174604774266\\
80.09	0.00295087246896885\\
80.1	0.00294999768407951\\
80.11	0.00294912167774288\\
80.12	0.002948244434275\\
80.13	0.00294736593763087\\
80.14	0.00294648617139543\\
80.15	0.00294560511877422\\
80.16	0.00294472276258388\\
80.17	0.00294383908524232\\
80.18	0.00294295406875876\\
80.19	0.00294206769472335\\
80.2	0.00294117994429662\\
80.21	0.00294029079819866\\
80.22	0.00293940023669798\\
80.23	0.00293850823960004\\
80.24	0.0029376147862356\\
80.25	0.00293671985544865\\
80.26	0.00293582342558406\\
80.27	0.00293492547447495\\
80.28	0.00293402597942964\\
80.29	0.00293312491721835\\
80.3	0.00293222226405949\\
80.31	0.00293131799560561\\
80.32	0.00293041208692898\\
80.33	0.00292950451250678\\
80.34	0.00292859524620591\\
80.35	0.00292768426126743\\
80.36	0.0029267715302905\\
80.37	0.002925857025216\\
80.38	0.00292494071730966\\
80.39	0.00292402257714477\\
80.4	0.00292310257458437\\
80.41	0.00292218067876311\\
80.42	0.00292125685806842\\
80.43	0.00292033108012144\\
80.44	0.00291940331175716\\
80.45	0.0029184735190043\\
80.46	0.0029175416670645\\
80.47	0.00291660772029096\\
80.48	0.00291567164216659\\
80.49	0.00291473339528153\\
80.5	0.00291379294131008\\
80.51	0.00291285024098702\\
80.52	0.00291190525408331\\
80.53	0.00291095793938111\\
80.54	0.00291000825464818\\
80.55	0.00290905615661158\\
80.56	0.00290810160093067\\
80.57	0.00290714454216938\\
80.58	0.00290618493376774\\
80.59	0.00290522272801269\\
80.6	0.00290425787600805\\
80.61	0.00290329032764376\\
80.62	0.00290232003156423\\
80.63	0.00290134693513586\\
80.64	0.00290037098441376\\
80.65	0.00289939212410748\\
80.66	0.00289841029754589\\
80.67	0.0028974254466411\\
80.68	0.00289643751185138\\
80.69	0.00289544643214319\\
80.7	0.00289445214495205\\
80.71	0.0028934545861425\\
80.72	0.00289245368996685\\
80.73	0.00289144938902296\\
80.74	0.00289044161421078\\
80.75	0.0028894302946878\\
80.76	0.00288841535782319\\
80.77	0.0028873967291509\\
80.78	0.00288637433232128\\
80.79	0.00288534808905154\\
80.8	0.00288431791907485\\
80.81	0.00288328374008803\\
80.82	0.0028822454676979\\
80.83	0.00288120301536609\\
80.84	0.00288015629435246\\
80.85	0.00287910521365691\\
80.86	0.00287804967995969\\
80.87	0.00287698959756006\\
80.88	0.00287592486831329\\
80.89	0.00287485539156598\\
80.9	0.0028737810640896\\
80.91	0.0028727017800123\\
80.92	0.00287161743074878\\
80.93	0.00287052790492832\\
80.94	0.00286943308832089\\
80.95	0.00286833286376117\\
80.96	0.00286722711107061\\
80.97	0.00286611570697729\\
80.98	0.00286499852503366\\
80.99	0.00286387543553207\\
81	0.00286274630541788\\
81.01	0.00286161099820042\\
81.02	0.00286046937386132\\
81.03	0.00285932128876049\\
81.04	0.00285816659553948\\
81.05	0.00285700514302222\\
81.06	0.0028558367761131\\
81.07	0.00285466133569217\\
81.08	0.00285347865850764\\
81.09	0.00285228857706532\\
81.1	0.00285109091951508\\
81.11	0.00284988550953426\\
81.12	0.00284867216620786\\
81.13	0.00284745070390544\\
81.14	0.00284622093215469\\
81.15	0.00284498265551157\\
81.16	0.00284373567342682\\
81.17	0.00284247978010889\\
81.18	0.00284121476438312\\
81.19	0.002839940409547\\
81.2	0.00283865649322151\\
81.21	0.00283736278719844\\
81.22	0.0028360590572834\\
81.23	0.00283474506313464\\
81.24	0.00283342055809738\\
81.25	0.00283208528903362\\
81.26	0.00283073899614727\\
81.27	0.00282938141280447\\
81.28	0.00282801226534897\\
81.29	0.00282663127291246\\
81.3	0.00282523814721959\\
81.31	0.00282383259238777\\
81.32	0.00282241430472132\\
81.33	0.00282098297249998\\
81.34	0.00281953827576167\\
81.35	0.0028180798860791\\
81.36	0.00281660746633036\\
81.37	0.0028151206704631\\
81.38	0.00281361914325221\\
81.39	0.00281210252005076\\
81.4	0.0028105704265341\\
81.41	0.00280902247843687\\
81.42	0.00280745828128268\\
81.43	0.00280587743010634\\
81.44	0.00280427950916834\\
81.45	0.0028026640916615\\
81.46	0.0028010307394093\\
81.47	0.00279937900255604\\
81.48	0.0027977084192482\\
81.49	0.00279601851530703\\
81.5	0.00279430880389201\\
81.51	0.00279257878515491\\
81.52	0.00279084279639984\\
81.53	0.00278910594816802\\
81.54	0.00278736823819391\\
81.55	0.00278562966420099\\
81.56	0.00278389022390198\\
81.57	0.00278214991499922\\
81.58	0.00278040873518503\\
81.59	0.00277866668214203\\
81.6	0.00277692375354358\\
81.61	0.00277517994705412\\
81.62	0.0027734352603297\\
81.63	0.00277168969101834\\
81.64	0.00276994323676058\\
81.65	0.00276819589518991\\
81.66	0.00276644766393336\\
81.67	0.00276469854061203\\
81.68	0.00276294852284165\\
81.69	0.00276119760823323\\
81.7	0.00275944579439364\\
81.71	0.00275769307892632\\
81.72	0.00275593945943197\\
81.73	0.00275418493350925\\
81.74	0.00275242949875559\\
81.75	0.00275067315276794\\
81.76	0.00274891589314361\\
81.77	0.00274715771748116\\
81.78	0.00274539862338127\\
81.79	0.00274363860844773\\
81.8	0.00274187767028837\\
81.81	0.00274011580651612\\
81.82	0.00273835301475007\\
81.83	0.00273658929261658\\
81.84	0.00273482463775044\\
81.85	0.00273305904779606\\
81.86	0.00273129252040876\\
81.87	0.00272952505325604\\
81.88	0.00272775664401894\\
81.89	0.00272598729039349\\
81.9	0.00272421699009213\\
81.91	0.00272244574084524\\
81.92	0.00272067354040276\\
81.93	0.00271890038653579\\
81.94	0.00271712627703831\\
81.95	0.002715351209729\\
81.96	0.00271357518245299\\
81.97	0.00271179819308386\\
81.98	0.00271002023952556\\
81.99	0.00270824131971449\\
82	0.00270646143162166\\
82.01	0.00270468057325484\\
82.02	0.00270289874266088\\
82.03	0.00270111593792813\\
82.04	0.00269933215718885\\
82.05	0.00269754739862176\\
82.06	0.00269576166045474\\
82.07	0.00269397494096752\\
82.08	0.00269218723849453\\
82.09	0.00269039855142786\\
82.1	0.00268860887822029\\
82.11	0.00268681821738844\\
82.12	0.00268502656751604\\
82.13	0.00268323392725733\\
82.14	0.0026814402953405\\
82.15	0.00267964567057137\\
82.16	0.00267785005183712\\
82.17	0.00267605343811013\\
82.18	0.00267425582845204\\
82.19	0.00267245722201786\\
82.2	0.00267065761806027\\
82.21	0.00266885701593408\\
82.22	0.00266705541510076\\
82.23	0.00266525281513326\\
82.24	0.00266344921572082\\
82.25	0.00266164461667415\\
82.26	0.00265983901793056\\
82.27	0.00265803241955947\\
82.28	0.00265622482176793\\
82.29	0.00265441622490646\\
82.3	0.00265260662947502\\
82.31	0.00265079603612914\\
82.32	0.00264898444568636\\
82.33	0.0026471718591328\\
82.34	0.00264535827762994\\
82.35	0.00264354370252171\\
82.36	0.0026417281353417\\
82.37	0.00263991157782069\\
82.38	0.00263809403189442\\
82.39	0.00263627549971152\\
82.4	0.00263445598364185\\
82.41	0.00263263548628496\\
82.42	0.00263081401047897\\
82.43	0.00262899155930958\\
82.44	0.0026271681361195\\
82.45	0.00262534374451811\\
82.46	0.00262351838839151\\
82.47	0.00262169207191275\\
82.48	0.00261986479955256\\
82.49	0.00261803657609028\\
82.5	0.00261620740662522\\
82.51	0.00261437729658838\\
82.52	0.00261254625175447\\
82.53	0.0026107142782544\\
82.54	0.00260888138258815\\
82.55	0.00260704757163795\\
82.56	0.00260521285268203\\
82.57	0.00260337723340868\\
82.58	0.00260154072193083\\
82.59	0.002599703326801\\
82.6	0.00259786505702685\\
82.61	0.00259602592208709\\
82.62	0.00259418593194795\\
82.63	0.00259234509708017\\
82.64	0.00259050342847647\\
82.65	0.00258866093766966\\
82.66	0.00258681763675119\\
82.67	0.00258497353839036\\
82.68	0.00258312865585412\\
82.69	0.00258128300302746\\
82.7	0.00257943659443442\\
82.71	0.00257758944525978\\
82.72	0.00257574157137145\\
82.73	0.00257389298934342\\
82.74	0.00257204371647959\\
82.75	0.0025701937708382\\
82.76	0.00256834317125708\\
82.77	0.00256649193737964\\
82.78	0.00256464008968167\\
82.79	0.00256278764949898\\
82.8	0.00256093463905584\\
82.81	0.00255908108149434\\
82.82	0.00255722700090461\\
82.83	0.00255537242235595\\
82.84	0.00255351737192899\\
82.85	0.00255166187674871\\
82.86	0.00254980596501853\\
82.87	0.00254794966605549\\
82.88	0.00254609301032634\\
82.89	0.00254423602948491\\
82.9	0.00254237875641045\\
82.91	0.00254052122524725\\
82.92	0.00253866347144538\\
82.93	0.00253680553180271\\
82.94	0.00253494744450817\\
82.95	0.0025330892491863\\
82.96	0.0025312309869432\\
82.97	0.0025293727004138\\
82.98	0.00252751443381058\\
82.99	0.00252565623297376\\
83	0.002523798145423\\
83.01	0.0025219402204106\\
83.02	0.00252008250897639\\
83.03	0.00251822506400414\\
83.04	0.00251636794027979\\
83.05	0.00251451119455134\\
83.06	0.00251265488559051\\
83.07	0.00251079907425633\\
83.08	0.00250894382356057\\
83.09	0.00250708919873509\\
83.1	0.00250523526730128\\
83.11	0.0025033820991415\\
83.12	0.00250152976657267\\
83.13	0.00249967834442207\\
83.14	0.00249782791010535\\
83.15	0.00249597854370688\\
83.16	0.00249413032806248\\
83.17	0.00249228334884462\\
83.18	0.00249043769465014\\
83.19	0.00248859345709053\\
83.2	0.00248675073088495\\
83.21	0.00248490961395594\\
83.22	0.00248307020752798\\
83.23	0.00248123261622903\\
83.24	0.00247939694819493\\
83.25	0.002477563315177\\
83.26	0.0024757318326528\\
83.27	0.00247390261994009\\
83.28	0.00247207580031416\\
83.29	0.00247025150112877\\
83.3	0.00246842985394035\\
83.31	0.00246661099463619\\
83.32	0.00246479506356614\\
83.33	0.00246298220567828\\
83.34	0.00246117257065864\\
83.35	0.00245936631307486\\
83.36	0.00245756359252424\\
83.37	0.00245576457378599\\
83.38	0.00245396942697804\\
83.39	0.00245217832771839\\
83.4	0.00245039145729119\\
83.41	0.00244860900281768\\
83.42	0.00244683115743216\\
83.43	0.00244505812046309\\
83.44	0.00244329009761942\\
83.45	0.00244152730118244\\
83.46	0.00243976995020322\\
83.47	0.00243801827070578\\
83.48	0.00243627249589617\\
83.49	0.00243453286637767\\
83.5	0.00243279963037228\\
83.51	0.00243107304394863\\
83.52	0.00242935337125647\\
83.53	0.0024276408847681\\
83.54	0.00242593586552671\\
83.55	0.00242423860340197\\
83.56	0.00242254939735304\\
83.57	0.00242086567621757\\
83.58	0.00241918128975166\\
83.59	0.00241749623742718\\
83.6	0.00241581051871453\\
83.61	0.00241412413308275\\
83.62	0.0024124370799994\\
83.63	0.00241074935893059\\
83.64	0.00240906096934098\\
83.65	0.00240737191069374\\
83.66	0.00240568218245052\\
83.67	0.00240399178407149\\
83.68	0.00240230071501527\\
83.69	0.00240060897473895\\
83.7	0.00239891656269807\\
83.71	0.00239722347834656\\
83.72	0.00239552972113681\\
83.73	0.00239383529051956\\
83.74	0.00239214018594395\\
83.75	0.00239044440685748\\
83.76	0.00238874795270599\\
83.77	0.00238705082293363\\
83.78	0.00238535301698289\\
83.79	0.00238365453429453\\
83.8	0.0023819553743076\\
83.81	0.00238025553645937\\
83.82	0.00237855502018539\\
83.83	0.00237685382491939\\
83.84	0.00237515195009332\\
83.85	0.00237344939513731\\
83.86	0.00237174615947964\\
83.87	0.00237004224254672\\
83.88	0.0023683376437631\\
83.89	0.00236663236255141\\
83.9	0.00236492639833237\\
83.91	0.00236321975052475\\
83.92	0.00236151241854535\\
83.93	0.00235980440180898\\
83.94	0.00235809569972845\\
83.95	0.00235638631171455\\
83.96	0.00235467623717597\\
83.97	0.00235296547551937\\
83.98	0.00235125402614928\\
83.99	0.0023495418884681\\
84	0.00234782906187611\\
84.01	0.00234611554577139\\
84.02	0.00234440133954981\\
84.03	0.00234268644260504\\
84.04	0.00234097085432848\\
84.05	0.00233925457410927\\
84.06	0.00233753760133423\\
84.07	0.00233581993538784\\
84.08	0.00233410157565224\\
84.09	0.00233238252150719\\
84.1	0.00233066277233001\\
84.11	0.00232894232749559\\
84.12	0.00232722118637634\\
84.13	0.00232549934834217\\
84.14	0.00232377681276049\\
84.15	0.00232205357899608\\
84.16	0.00232032964641117\\
84.17	0.00231860501436538\\
84.18	0.00231687968221563\\
84.19	0.00231515364931618\\
84.2	0.00231342691501856\\
84.21	0.00231169947867153\\
84.22	0.00230997133962109\\
84.23	0.00230824249721039\\
84.24	0.00230651295077973\\
84.25	0.00230478269966652\\
84.26	0.00230305174320523\\
84.27	0.00230132008072738\\
84.28	0.00229958771156147\\
84.29	0.00229785463503296\\
84.3	0.00229612085046425\\
84.31	0.00229438635717459\\
84.32	0.00229265115448009\\
84.33	0.00229091524169367\\
84.34	0.002289178618125\\
84.35	0.00228744128308047\\
84.36	0.00228570323586317\\
84.37	0.00228396447577279\\
84.38	0.00228222500210564\\
84.39	0.00228048481415457\\
84.4	0.00227874391120895\\
84.41	0.0022770022925546\\
84.42	0.00227525995747374\\
84.43	0.00227351690524498\\
84.44	0.00227177313514326\\
84.45	0.00227002864643975\\
84.46	0.0022682834384019\\
84.47	0.0022665375102933\\
84.48	0.00226479086137367\\
84.49	0.00226304349089882\\
84.5	0.00226129539812058\\
84.51	0.00225954658228674\\
84.52	0.00225779704264101\\
84.53	0.00225604677842297\\
84.54	0.00225429578886803\\
84.55	0.00225254407320732\\
84.56	0.00225079163066768\\
84.57	0.00224903846047162\\
84.58	0.00224728456183719\\
84.59	0.00224552993397798\\
84.6	0.00224377457610307\\
84.61	0.0022420184874169\\
84.62	0.0022402616671193\\
84.63	0.00223850411440534\\
84.64	0.00223674582846534\\
84.65	0.00223498680848475\\
84.66	0.00223322705364413\\
84.67	0.00223146656311903\\
84.68	0.00222970533607998\\
84.69	0.00222794337169239\\
84.7	0.0022261806691165\\
84.71	0.00222441722750727\\
84.72	0.00222265304601436\\
84.73	0.00222088812378202\\
84.74	0.00221912245994904\\
84.75	0.00221735605364867\\
84.76	0.00221558890400853\\
84.77	0.00221382101015054\\
84.78	0.00221205237119087\\
84.79	0.00221028298623981\\
84.8	0.00220851285440175\\
84.81	0.00220674197477505\\
84.82	0.00220497034645196\\
84.83	0.00220319796851858\\
84.84	0.00220142484005473\\
84.85	0.00219965096013388\\
84.86	0.00219787632782309\\
84.87	0.00219610094218286\\
84.88	0.00219432480226709\\
84.89	0.00219254790712298\\
84.9	0.00219077025579094\\
84.91	0.00218899184730448\\
84.92	0.00218721268069011\\
84.93	0.0021854327549673\\
84.94	0.00218365206914829\\
84.95	0.00218187062223809\\
84.96	0.0021800884132343\\
84.97	0.00217830544112704\\
84.98	0.00217652170489889\\
84.99	0.00217473720352467\\
85	0.00217295193597148\\
85.01	0.00217116590119847\\
85.02	0.00216937909815681\\
85.03	0.00216759152578952\\
85.04	0.00216580318303141\\
85.05	0.00216401406880895\\
85.06	0.00216222418204011\\
85.07	0.00216043352163433\\
85.08	0.00215864208649232\\
85.09	0.00215684987550599\\
85.1	0.00215505688755831\\
85.11	0.00215326312152317\\
85.12	0.00215146857626529\\
85.13	0.0021496732506401\\
85.14	0.00214787714349354\\
85.15	0.002146080253662\\
85.16	0.00214428257997218\\
85.17	0.00214248412124093\\
85.18	0.00214068487627513\\
85.19	0.00213888484387155\\
85.2	0.0021370840228167\\
85.21	0.00213528241188673\\
85.22	0.00213348000984723\\
85.23	0.00213167681545313\\
85.24	0.00212987282744852\\
85.25	0.00212806804456654\\
85.26	0.00212626246552919\\
85.27	0.00212445608904722\\
85.28	0.00212264891381993\\
85.29	0.00212084093853504\\
85.3	0.00211903216186854\\
85.31	0.00211722258248452\\
85.32	0.00211541219903498\\
85.33	0.00211360101015972\\
85.34	0.00211178901448614\\
85.35	0.00210997621062907\\
85.36	0.00210816259719063\\
85.37	0.00210634817276\\
85.38	0.00210453293591332\\
85.39	0.00210271688521345\\
85.4	0.00210090001920983\\
85.41	0.00209908233643828\\
85.42	0.00209726383542081\\
85.43	0.00209544451466549\\
85.44	0.00209362437266617\\
85.45	0.00209180340790237\\
85.46	0.00208998161883907\\
85.47	0.00208815900392648\\
85.48	0.00208633556159989\\
85.49	0.00208451129027945\\
85.5	0.00208268618836996\\
85.51	0.0020808602542607\\
85.52	0.00207903348632519\\
85.53	0.00207720588292101\\
85.54	0.00207537744238956\\
85.55	0.00207354816305589\\
85.56	0.00207171804322846\\
85.57	0.00206988708119892\\
85.58	0.0020680552752419\\
85.59	0.00206622262361479\\
85.6	0.00206438912455754\\
85.61	0.00206255477629239\\
85.62	0.00206071957702365\\
85.63	0.00205888352493753\\
85.64	0.00205704661820184\\
85.65	0.00205520885496577\\
85.66	0.00205337023335969\\
85.67	0.00205153075149489\\
85.68	0.00204969040746331\\
85.69	0.00204784919933735\\
85.7	0.00204600712516958\\
85.71	0.00204416418299256\\
85.72	0.00204232037081849\\
85.73	0.00204047568663904\\
85.74	0.0020386301284251\\
85.75	0.00203678369412645\\
85.76	0.0020349363816716\\
85.77	0.00203308818896746\\
85.78	0.00203123911389909\\
85.79	0.00202938915432948\\
85.8	0.00202753830809925\\
85.81	0.00202568657302638\\
85.82	0.00202383394690598\\
85.83	0.00202198042750997\\
85.84	0.00202012601258685\\
85.85	0.0020182706998614\\
85.86	0.00201641448703444\\
85.87	0.00201455737178252\\
85.88	0.00201269935175766\\
85.89	0.00201084042458708\\
85.9	0.00200898058787288\\
85.91	0.00200711983919182\\
85.92	0.00200525817609502\\
85.93	0.00200339559610762\\
85.94	0.00200153209672859\\
85.95	0.00199966767543038\\
85.96	0.00199780232965866\\
85.97	0.00199593605683204\\
85.98	0.00199406885434174\\
85.99	0.0019922007195514\\
86	0.00199033164979667\\
86.01	0.00198846164238502\\
86.02	0.00198659069459543\\
86.03	0.00198471880367807\\
86.04	0.00198284596685407\\
86.05	0.00198097218131517\\
86.06	0.0019790974442235\\
86.07	0.00197722175271125\\
86.08	0.0019753451038804\\
86.09	0.00197346749480246\\
86.1	0.00197158892251818\\
86.11	0.00196970938403724\\
86.12	0.001967828876338\\
86.13	0.00196594739636725\\
86.14	0.00196406494103989\\
86.15	0.00196218150723869\\
86.16	0.00196029709181401\\
86.17	0.00195841169158354\\
86.18	0.00195652530333206\\
86.19	0.00195463792381113\\
86.2	0.0019527495497389\\
86.21	0.00195086017779982\\
86.22	0.0019489698046444\\
86.23	0.00194707842688899\\
86.24	0.00194518604111552\\
86.25	0.00194329264387129\\
86.26	0.00194139823166873\\
86.27	0.00193950280098519\\
86.28	0.00193760634826276\\
86.29	0.00193570886990799\\
86.3	0.00193381036229179\\
86.31	0.00193191082174917\\
86.32	0.00193001024457906\\
86.33	0.00192810862704422\\
86.34	0.00192620596537096\\
86.35	0.00192430225574911\\
86.36	0.00192239749433177\\
86.37	0.00192049167723525\\
86.38	0.00191858480053893\\
86.39	0.00191667686028515\\
86.4	0.00191476785247912\\
86.41	0.00191285777308883\\
86.42	0.00191094661804502\\
86.43	0.00190903438324107\\
86.44	0.00190712106453303\\
86.45	0.00190520665773951\\
86.46	0.0019032911586418\\
86.47	0.00190137456298377\\
86.48	0.00189945686647197\\
86.49	0.00189753806477567\\
86.5	0.0018956181535269\\
86.51	0.00189369712832061\\
86.52	0.00189177498471472\\
86.53	0.0018898517182303\\
86.54	0.00188792732435173\\
86.55	0.00188600179852688\\
86.56	0.00188407513616731\\
86.57	0.00188214733264856\\
86.58	0.00188021838331037\\
86.59	0.00187828828345703\\
86.6	0.00187635702835767\\
86.61	0.00187442461324664\\
86.62	0.00187249103332392\\
86.63	0.00187055628375556\\
86.64	0.00186862035967413\\
86.65	0.00186668325617921\\
86.66	0.00186474496833799\\
86.67	0.00186280549118584\\
86.68	0.00186086481972691\\
86.69	0.00185892294893484\\
86.7	0.00185697987375348\\
86.71	0.00185503558909764\\
86.72	0.00185309008985395\\
86.73	0.00185114337088168\\
86.74	0.00184919542701369\\
86.75	0.00184724625305743\\
86.76	0.00184529584379596\\
86.77	0.00184334419398905\\
86.78	0.00184139129837432\\
86.79	0.00183943715166854\\
86.8	0.00183748174856884\\
86.81	0.00183552508375412\\
86.82	0.00183356715188647\\
86.83	0.00183160794761269\\
86.84	0.00182964746556587\\
86.85	0.00182768570036707\\
86.86	0.00182572264662707\\
86.87	0.0018237582989482\\
86.88	0.00182179265192628\\
86.89	0.00181982570015264\\
86.9	0.00181785743821622\\
86.91	0.00181588786070584\\
86.92	0.00181391696221245\\
86.93	0.00181194473733162\\
86.94	0.00180997118066604\\
86.95	0.00180799628682821\\
86.96	0.00180602005044319\\
86.97	0.00180404246615148\\
86.98	0.00180206352861209\\
86.99	0.00180008323250567\\
87	0.00179810157253781\\
87.01	0.00179611854344243\\
87.02	0.00179413413998546\\
87.03	0.00179214835696848\\
87.04	0.00179016118923265\\
87.05	0.00178817263166273\\
87.06	0.00178618267919137\\
87.07	0.0017841913268034\\
87.08	0.00178219856954049\\
87.09	0.00178020440250586\\
87.1	0.00177820882086925\\
87.11	0.00177621181987204\\
87.12	0.00177421339483264\\
87.13	0.00177221354115204\\
87.14	0.00177021225431957\\
87.15	0.00176820952991894\\
87.16	0.0017662053636345\\
87.17	0.00176419975125769\\
87.18	0.00176219268869378\\
87.19	0.00176018417196892\\
87.2	0.00175817419723736\\
87.21	0.001756162760789\\
87.22	0.00175414985905721\\
87.23	0.00175213548862699\\
87.24	0.00175011964624337\\
87.25	0.00174810232882016\\
87.26	0.00174608353344902\\
87.27	0.0017440632574089\\
87.28	0.00174204149817573\\
87.29	0.00174001825343261\\
87.3	0.00173799352108026\\
87.31	0.00173596729924787\\
87.32	0.00173393958630443\\
87.33	0.00173191038087033\\
87.34	0.00172987968182955\\
87.35	0.0017278474883421\\
87.36	0.00172581379985708\\
87.37	0.00172377861612611\\
87.38	0.00172174193721726\\
87.39	0.00171970376352948\\
87.4	0.00171766409580754\\
87.41	0.00171562293515754\\
87.42	0.00171358028306289\\
87.43	0.00171153614140089\\
87.44	0.00170949051245994\\
87.45	0.00170744339895728\\
87.46	0.00170539480405738\\
87.47	0.00170334473139099\\
87.48	0.00170129318507481\\
87.49	0.00169924016973188\\
87.5	0.00169718569051268\\
87.51	0.00169512975311691\\
87.52	0.00169307236381607\\
87.53	0.00169101352947681\\
87.54	0.00168895325758503\\
87.55	0.0016868915562709\\
87.56	0.00168482843433462\\
87.57	0.00168276390127316\\
87.58	0.00168069796730786\\
87.59	0.00167863064341295\\
87.6	0.00167656194134508\\
87.61	0.00167449187367383\\
87.62	0.00167242045381326\\
87.63	0.00167034769605445\\
87.64	0.00166827361559929\\
87.65	0.00166619822859523\\
87.66	0.00166412155217133\\
87.67	0.0016620436044754\\
87.68	0.00165996440471252\\
87.69	0.0016578839731847\\
87.7	0.00165580233133196\\
87.71	0.00165371950177473\\
87.72	0.00165163550835766\\
87.73	0.00164955037619492\\
87.74	0.00164746413171695\\
87.75	0.00164537680271875\\
87.76	0.00164328841840983\\
87.77	0.00164119900946574\\
87.78	0.00163910860808133\\
87.79	0.00163701724802571\\
87.8	0.00163492496469912\\
87.81	0.0016328317951915\\
87.82	0.00163073777834316\\
87.83	0.0016286429548073\\
87.84	0.00162654736711458\\
87.85	0.00162445105973992\\
87.86	0.00162235407917129\\
87.87	0.00162025647398087\\
87.88	0.0016181582948985\\
87.89	0.00161605959488747\\
87.9	0.00161396042922276\\
87.91	0.00161186085557193\\
87.92	0.00160976093407846\\
87.93	0.00160766072744788\\
87.94	0.0016055603010367\\
87.95	0.00160345972294411\\
87.96	0.00160135906410674\\
87.97	0.00159925839839637\\
87.98	0.00159715780272086\\
87.99	0.00159505735712831\\
88	0.00159295714491448\\
88.01	0.00159085725273375\\
88.02	0.0015887577707136\\
88.03	0.00158665879257272\\
88.04	0.00158456041574294\\
88.05	0.00158246274149502\\
88.06	0.00158036587506847\\
88.07	0.00157826992580553\\
88.08	0.00157617500728932\\
88.09	0.00157408123748651\\
88.1	0.00157198873889449\\
88.11	0.00156989763869312\\
88.12	0.00156780806890144\\
88.13	0.00156572016653924\\
88.14	0.00156363407379382\\
88.15	0.00156154993819199\\
88.16	0.00155946791277748\\
88.17	0.0015573881562941\\
88.18	0.0015553108333745\\
88.19	0.001553236114735\\
88.2	0.00155116417737656\\
88.21	0.00154909520479208\\
88.22	0.00154702938718014\\
88.23	0.00154496692166566\\
88.24	0.00154290801252727\\
88.25	0.00154085287143202\\
88.26	0.00153880171767736\\
88.27	0.00153675477844073\\
88.28	0.00153471228903699\\
88.29	0.00153267449318393\\
88.3	0.00153064164327606\\
88.31	0.00152861400066701\\
88.32	0.00152659183596075\\
88.33	0.00152457542931187\\
88.34	0.00152256507073534\\
88.35	0.00152056106042584\\
88.36	0.0015185637090871\\
88.37	0.0015165733382715\\
88.38	0.00151459028073025\\
88.39	0.00151261488077449\\
88.4	0.00151064749464758\\
88.41	0.00150868849090901\\
88.42	0.00150673825083023\\
88.43	0.00150479716880274\\
88.44	0.00150286565275893\\
88.45	0.00150094412460582\\
88.46	0.00149903302067243\\
88.47	0.00149713279217082\\
88.48	0.00149524390567155\\
88.49	0.00149336684359376\\
88.5	0.0014915021047104\\
88.51	0.00148965020466912\\
88.52	0.00148781167652918\\
88.53	0.00148598707131491\\
88.54	0.00148417695858631\\
88.55	0.00148238192702715\\
88.56	0.00148060258505116\\
88.57	0.00147883956142693\\
88.58	0.00147709350592196\\
88.59	0.0014753650899665\\
88.6	0.00147365500733776\\
88.61	0.00147195258785159\\
88.62	0.00147024943724989\\
88.63	0.00146854555512318\\
88.64	0.00146684094106171\\
88.65	0.00146513559465547\\
88.66	0.00146342951549422\\
88.67	0.00146172270316742\\
88.68	0.00146001515726431\\
88.69	0.00145830687737387\\
88.7	0.0014565978630848\\
88.71	0.00145488811398557\\
88.72	0.00145317762966437\\
88.73	0.00145146640970915\\
88.74	0.0014497544537076\\
88.75	0.00144804176124715\\
88.76	0.00144632833191499\\
88.77	0.00144461416529802\\
88.78	0.00144289926098292\\
88.79	0.00144118361855609\\
88.8	0.00143946723760368\\
88.81	0.00143775011771158\\
88.82	0.00143603225846544\\
88.83	0.00143431365945063\\
88.84	0.00143259432025228\\
88.85	0.00143087424045526\\
88.86	0.00142915341964419\\
88.87	0.00142743185740342\\
88.88	0.00142570955331705\\
88.89	0.00142398650696892\\
88.9	0.00142226271794264\\
88.91	0.00142053818582152\\
88.92	0.00141881291018865\\
88.93	0.00141708689062685\\
88.94	0.00141536012671868\\
88.95	0.00141363261804645\\
88.96	0.00141190436419222\\
88.97	0.00141017536473778\\
88.98	0.00140844561926468\\
88.99	0.0014067151273542\\
89	0.00140498388858738\\
89.01	0.00140325190254498\\
89.02	0.00140151916880753\\
89.03	0.0013997856869553\\
89.04	0.00139805145656829\\
89.05	0.00139631647722625\\
89.06	0.00139458074850869\\
89.07	0.00139284426999484\\
89.08	0.00139110704126371\\
89.09	0.00138936906189402\\
89.1	0.00138763033146424\\
89.11	0.0013858908495526\\
89.12	0.00138415061573708\\
89.13	0.00138240962959539\\
89.14	0.00138066789070498\\
89.15	0.00137892539864305\\
89.16	0.00137718215298657\\
89.17	0.00137543815331222\\
89.18	0.00137369339919644\\
89.19	0.00137194789021543\\
89.2	0.00137020162594511\\
89.21	0.00136845460596117\\
89.22	0.00136670682983904\\
89.23	0.00136495829715387\\
89.24	0.00136320900748059\\
89.25	0.00136145896039386\\
89.26	0.0013597081554681\\
89.27	0.00135795659227746\\
89.28	0.00135620427039584\\
89.29	0.0013544511893969\\
89.3	0.00135269734885402\\
89.31	0.00135094274834036\\
89.32	0.0013491873874288\\
89.33	0.00134743126569199\\
89.34	0.00134567438270231\\
89.35	0.0013439167380319\\
89.36	0.00134215833125263\\
89.37	0.00134039916193613\\
89.38	0.00133863922965378\\
89.39	0.00133687853397671\\
89.4	0.00133511707447579\\
89.41	0.00133335485072163\\
89.42	0.00133159186228462\\
89.43	0.00132982810873486\\
89.44	0.00132806358964223\\
89.45	0.00132629830457633\\
89.46	0.00132453225310655\\
89.47	0.00132276543480197\\
89.48	0.00132099784923149\\
89.49	0.0013192294959637\\
89.5	0.00131746037456697\\
89.51	0.00131569048460942\\
89.52	0.0013139198256589\\
89.53	0.00131214839728303\\
89.54	0.00131037619904917\\
89.55	0.00130860323052444\\
89.56	0.00130682949127571\\
89.57	0.00130505498086958\\
89.58	0.00130327969887243\\
89.59	0.00130150364485038\\
89.6	0.00129972681836929\\
89.61	0.00129794921899479\\
89.62	0.00129617084629225\\
89.63	0.00129439169982681\\
89.64	0.00129261177916333\\
89.65	0.00129083108386646\\
89.66	0.00128904961350057\\
89.67	0.0012872673676298\\
89.68	0.00128548434581806\\
89.69	0.00128370054762897\\
89.7	0.00128191597262594\\
89.71	0.00128013062037213\\
89.72	0.00127834449043044\\
89.73	0.00127655758236353\\
89.74	0.00127476989573382\\
89.75	0.00127298143010348\\
89.76	0.00127119218503443\\
89.77	0.00126940216008838\\
89.78	0.00126761135482674\\
89.79	0.00126581976881073\\
89.8	0.00126402740160128\\
89.81	0.0012622342527591\\
89.82	0.00126044032184467\\
89.83	0.00125864560841821\\
89.84	0.0012568501120397\\
89.85	0.00125505383226888\\
89.86	0.00125325676866524\\
89.87	0.00125145892078804\\
89.88	0.0012496602881963\\
89.89	0.00124786087044879\\
89.9	0.00124606066710404\\
89.91	0.00124425967772035\\
89.92	0.00124245790185578\\
89.93	0.00124065533906814\\
89.94	0.001238851988915\\
89.95	0.00123704785095371\\
89.96	0.00123524292474136\\
89.97	0.00123343720983481\\
89.98	0.0012316307057907\\
89.99	0.0012298234121654\\
90	0.00122801532851507\\
90.01	0.00122620645439562\\
90.02	0.00122439678936273\\
90.03	0.00122258633297185\\
90.04	0.00122077508477818\\
90.05	0.00121896304433669\\
90.06	0.00121715021120214\\
90.07	0.00121533658492901\\
90.08	0.00121352216507159\\
90.09	0.00121170695118391\\
90.1	0.00120989094281979\\
90.11	0.00120807413953279\\
90.12	0.00120625654087626\\
90.13	0.00120443814640332\\
90.14	0.00120261895566685\\
90.15	0.00120079896821951\\
90.16	0.0011989781836137\\
90.17	0.00119715660140164\\
90.18	0.00119533422113528\\
90.19	0.00119351104236637\\
90.2	0.00119168706464642\\
90.21	0.0011898622875267\\
90.22	0.00118803671055829\\
90.23	0.00118621033329201\\
90.24	0.00118438315527848\\
90.25	0.00118255517606807\\
90.26	0.00118072639521094\\
90.27	0.00117889681225704\\
90.28	0.00117706642675607\\
90.29	0.00117523523825753\\
90.3	0.00117340324631068\\
90.31	0.00117157045046459\\
90.32	0.00116973685026807\\
90.33	0.00116790244526974\\
90.34	0.00116606723501799\\
90.35	0.00116423121906099\\
90.36	0.0011623943969467\\
90.37	0.00116055676822286\\
90.38	0.001158718332437\\
90.39	0.00115687908913641\\
90.4	0.0011550390378682\\
90.41	0.00115319817817924\\
90.42	0.0011513565096162\\
90.43	0.00114951403172555\\
90.44	0.00114767074405351\\
90.45	0.00114582664614613\\
90.46	0.00114398173754922\\
90.47	0.00114213601780841\\
90.48	0.00114028948646909\\
90.49	0.00113844214307647\\
90.5	0.00113659398717554\\
90.51	0.00113474501831108\\
90.52	0.00113289523602769\\
90.53	0.00113104463986972\\
90.54	0.00112919322938137\\
90.55	0.00112734100410661\\
90.56	0.00112548796358921\\
90.57	0.00112363410737274\\
90.58	0.00112177943500057\\
90.59	0.00111992394601588\\
90.6	0.00111806763996166\\
90.61	0.00111621051638068\\
90.62	0.00111435257481552\\
90.63	0.00111249381480859\\
90.64	0.00111063423590208\\
90.65	0.00110877383763801\\
90.66	0.00110691261955819\\
90.67	0.00110505058120425\\
90.68	0.00110318772211763\\
90.69	0.00110132404183958\\
90.7	0.00109945953991118\\
90.71	0.0010975942158733\\
90.72	0.00109572806926665\\
90.73	0.00109386109963173\\
90.74	0.00109199330650889\\
90.75	0.00109012468943827\\
90.76	0.00108825524795986\\
90.77	0.00108638498161345\\
90.78	0.00108451388993868\\
90.79	0.00108264197247498\\
90.8	0.00108076922876162\\
90.81	0.00107889565833773\\
90.82	0.00107702126074221\\
90.83	0.00107514603551385\\
90.84	0.00107326998219124\\
90.85	0.00107139310031279\\
90.86	0.0010695153894168\\
90.87	0.00106763684904134\\
90.88	0.00106575747872437\\
90.89	0.00106387727800366\\
90.9	0.00106199624641685\\
90.91	0.00106011438350138\\
90.92	0.00105823168879457\\
90.93	0.00105634816183359\\
90.94	0.00105446380215543\\
90.95	0.00105257860929695\\
90.96	0.00105069258279486\\
90.97	0.00104880572218572\\
90.98	0.00104691802700594\\
90.99	0.00104502949679179\\
91	0.00104314013107942\\
91.01	0.00104124992940481\\
91.02	0.00103935889130382\\
91.03	0.00103746701631216\\
91.04	0.00103557430396544\\
91.05	0.00103368075379909\\
91.06	0.00103178636534846\\
91.07	0.00102989113814874\\
91.08	0.001027995071735\\
91.09	0.0010260981656422\\
91.1	0.00102420041940517\\
91.11	0.00102230183255863\\
91.12	0.00102040240463717\\
91.13	0.00101850213517527\\
91.14	0.0010166010237073\\
91.15	0.00101469906976753\\
91.16	0.00101279627289012\\
91.17	0.00101089263260912\\
91.18	0.00100898814845847\\
91.19	0.00100708281997203\\
91.2	0.00100517664668354\\
91.21	0.00100326962812667\\
91.22	0.00100136176383499\\
91.23	0.000999453053341967\\
91.24	0.000997543496181001\\
91.25	0.0009956330918854\\
91.26	0.000993721839988396\\
91.27	0.000991809740023124\\
91.28	0.000989896791522673\\
91.29	0.00098798299402004\\
91.3	0.000986068347048159\\
91.31	0.000984152850139898\\
91.32	0.00098223650282806\\
91.33	0.000980319304645391\\
91.34	0.000978401255124583\\
91.35	0.000976482353798264\\
91.36	0.000974562600199017\\
91.37	0.000972641993859387\\
91.38	0.00097072053431186\\
91.39	0.000968798221088894\\
91.4	0.000966875053722902\\
91.41	0.000964951031746271\\
91.42	0.000963026154691351\\
91.43	0.000961100422090478\\
91.44	0.000959173833475949\\
91.45	0.000957246388380057\\
91.46	0.000955318086335076\\
91.47	0.000953388926873269\\
91.48	0.000951458909526885\\
91.49	0.000949528033828183\\
91.5	0.000947596299309421\\
91.51	0.000945663705502852\\
91.52	0.000943730251940745\\
91.53	0.000941795938155389\\
91.54	0.000939860763679084\\
91.55	0.000937924728044158\\
91.56	0.000935987830782956\\
91.57	0.000934050071427871\\
91.58	0.000932111449511318\\
91.59	0.000930171964565759\\
91.6	0.000928231616123711\\
91.61	0.00092629040371772\\
91.62	0.000924348326880408\\
91.63	0.00092240538514445\\
91.64	0.000920461578042589\\
91.65	0.000918516905107643\\
91.66	0.000916571365872495\\
91.67	0.000914624959870118\\
91.68	0.000912677686633575\\
91.69	0.00091072954569602\\
91.7	0.0009087805365907\\
91.71	0.000906830658850968\\
91.72	0.0009048799120103\\
91.73	0.000902928295602274\\
91.74	0.000900975809160596\\
91.75	0.000899022452219098\\
91.76	0.000897068224311758\\
91.77	0.000895113124972674\\
91.78	0.000893157153736111\\
91.79	0.000891200310136488\\
91.8	0.000889242593708367\\
91.81	0.000887284003986502\\
91.82	0.000885324540505803\\
91.83	0.000883364202801379\\
91.84	0.000881402990408507\\
91.85	0.000879440902862681\\
91.86	0.000877477939699594\\
91.87	0.000875514100455147\\
91.88	0.000873549384665466\\
91.89	0.000871583791866901\\
91.9	0.000869617321596039\\
91.91	0.000867649973389712\\
91.92	0.000865681746785014\\
91.93	0.000863712641319276\\
91.94	0.000861742656530129\\
91.95	0.000859771791955464\\
91.96	0.000857800047133457\\
91.97	0.000855827421602609\\
91.98	0.000853853914901693\\
91.99	0.000851879526569817\\
92	0.00084990425614642\\
92.01	0.00084792810317127\\
92.02	0.000845951067184481\\
92.03	0.000843973147726535\\
92.04	0.000841994344338267\\
92.05	0.000840014656560898\\
92.06	0.000838034083936052\\
92.07	0.00083605262600573\\
92.08	0.000834070282312371\\
92.09	0.000832087052398827\\
92.1	0.00083010293580839\\
92.11	0.000828117932084807\\
92.12	0.000826132040772275\\
92.13	0.00082414526141549\\
92.14	0.000822157593559617\\
92.15	0.000820169036750336\\
92.16	0.000818179590533848\\
92.17	0.000816189254456865\\
92.18	0.000814198028066673\\
92.19	0.000812205910911105\\
92.2	0.000810212902538573\\
92.21	0.000808219002498082\\
92.22	0.000806224210339254\\
92.23	0.000804228525612329\\
92.24	0.000802231947868196\\
92.25	0.000800234476658397\\
92.26	0.000798236111535167\\
92.27	0.000796236852051423\\
92.28	0.000794236697760806\\
92.29	0.000792235648217701\\
92.3	0.000790233702977237\\
92.31	0.000788230861595319\\
92.32	0.000786227123628665\\
92.33	0.000784222488634795\\
92.34	0.000782216956172081\\
92.35	0.000780210525799765\\
92.36	0.000778203197077965\\
92.37	0.000776194969567724\\
92.38	0.000774185842831015\\
92.39	0.000772175816430782\\
92.4	0.000770164889930954\\
92.41	0.000768153062896489\\
92.42	0.000766140334893375\\
92.43	0.000764126705488694\\
92.44	0.000762112174250621\\
92.45	0.000760096740748478\\
92.46	0.000758080404552748\\
92.47	0.00075606316523512\\
92.48	0.000754045022368519\\
92.49	0.000752025975527143\\
92.5	0.000750006024286501\\
92.51	0.000747985168223437\\
92.52	0.000745963406916189\\
92.53	0.000743940739944418\\
92.54	0.000741917166889249\\
92.55	0.000739892687333312\\
92.56	0.000737867300860806\\
92.57	0.00073584100705751\\
92.58	0.000733813805510857\\
92.59	0.000731785695809986\\
92.6	0.000729756677545773\\
92.61	0.000727726750310888\\
92.62	0.000725695913699866\\
92.63	0.000723664167309152\\
92.64	0.00072163151073716\\
92.65	0.000719597943584324\\
92.66	0.000717563465453186\\
92.67	0.00071552807594844\\
92.68	0.000713491774677003\\
92.69	0.000711454561248095\\
92.7	0.000709416435273299\\
92.71	0.000707377396366648\\
92.72	0.000705337444144702\\
92.73	0.000703296578226615\\
92.74	0.000701254798234236\\
92.75	0.000699212103792193\\
92.76	0.000697168494527984\\
92.77	0.000695123970072068\\
92.78	0.000693078530057961\\
92.79	0.000691032174122353\\
92.8	0.000688984901905195\\
92.81	0.000686936713049825\\
92.82	0.000684887607203073\\
92.83	0.000682837584015382\\
92.84	0.000680786643140941\\
92.85	0.000678734784237802\\
92.86	0.00067668200696802\\
92.87	0.000674628310997792\\
92.88	0.000672573695997604\\
92.89	0.000670518161642375\\
92.9	0.000668461707611626\\
92.91	0.000666404333589632\\
92.92	0.000664346039265604\\
92.93	0.000662286824333849\\
92.94	0.000660226688493979\\
92.95	0.000658165631451076\\
92.96	0.000656103652915912\\
92.97	0.000654040752605152\\
92.98	0.000651976930241565\\
92.99	0.000649912185554253\\
93	0.000647846518278893\\
93.01	0.000645779928157975\\
93.02	0.000643712414941053\\
93.03	0.000641643978385025\\
93.04	0.000639574618254391\\
93.05	0.000637504334321555\\
93.06	0.000635433126367128\\
93.07	0.000633360994180229\\
93.08	0.000631287937558818\\
93.09	0.000629213956310036\\
93.1	0.000627139050250561\\
93.11	0.000625063219206974\\
93.12	0.00062298646301614\\
93.13	0.00062090878152562\\
93.14	0.000618830174594085\\
93.15	0.000616750642091735\\
93.16	0.000614670183900795\\
93.17	0.000612588799915935\\
93.18	0.000610506490044804\\
93.19	0.000608423254208526\\
93.2	0.000606339092342246\\
93.21	0.000604254004395673\\
93.22	0.000602167990333691\\
93.23	0.000600081050136933\\
93.24	0.000597993183802442\\
93.25	0.000595904391344318\\
93.26	0.000593814672789033\\
93.27	0.000591724028175056\\
93.28	0.00058963245755313\\
93.29	0.000587539960986578\\
93.3	0.000585446538551603\\
93.31	0.000583352190337614\\
93.32	0.00058125691644754\\
93.33	0.000579160716998151\\
93.34	0.000577063592120413\\
93.35	0.000574965541959837\\
93.36	0.000572866566676813\\
93.37	0.000570766666447\\
93.38	0.00056866584146169\\
93.39	0.000566564091928195\\
93.4	0.00056446141807024\\
93.41	0.000562357820128376\\
93.42	0.000560253298360387\\
93.43	0.000558147853041717\\
93.44	0.000556041484465921\\
93.45	0.000553934192945098\\
93.46	0.00055182597881036\\
93.47	0.000549716842412307\\
93.48	0.000547606784121512\\
93.49	0.000545495804329015\\
93.5	0.000543383903446841\\
93.51	0.00054127108190853\\
93.52	0.00053915734016966\\
93.53	0.000537042678708416\\
93.54	0.000534927098026149\\
93.55	0.000532810598647975\\
93.56	0.000530693181123356\\
93.57	0.00052857484602672\\
93.58	0.000526455593958102\\
93.59	0.000524335425543783\\
93.6	0.000522214341436964\\
93.61	0.000520092342318442\\
93.62	0.000517969428897315\\
93.63	0.000515845601911712\\
93.64	0.000513720862129521\\
93.65	0.000511595210349153\\
93.66	0.000509468647400332\\
93.67	0.000507341174144887\\
93.68	0.000505212791477586\\
93.69	0.000503083500326979\\
93.7	0.000500953301656266\\
93.71	0.00049882219646419\\
93.72	0.000496690185785959\\
93.73	0.000494557270694184\\
93.74	0.000492423452299865\\
93.75	0.000490288731753356\\
93.76	0.000488153110245424\\
93.77	0.000486016589008272\\
93.78	0.000483879169316621\\
93.79	0.000481740852488848\\
93.8	0.000479601639888095\\
93.81	0.000477461532923449\\
93.82	0.000475320533051165\\
93.83	0.000473178641775875\\
93.84	0.000471035860651883\\
93.85	0.000468892191284471\\
93.86	0.000466747635331223\\
93.87	0.000464602194503432\\
93.88	0.000462455870567506\\
93.89	0.000460308665346429\\
93.9	0.000458160580721271\\
93.91	0.000456011618632715\\
93.92	0.000453861781082665\\
93.93	0.000451711070135851\\
93.94	0.000449559487921527\\
93.95	0.000447407036635181\\
93.96	0.000445253718540322\\
93.97	0.000443099535970286\\
93.98	0.000440944491330121\\
93.99	0.000438788587098516\\
94	0.000436631825829764\\
94.01	0.000434474210155834\\
94.02	0.000432315742788429\\
94.03	0.000430156426521174\\
94.04	0.000427996264231807\\
94.05	0.000425835258884477\\
94.06	0.000423673413532085\\
94.07	0.000421510731318695\\
94.08	0.000419347215482006\\
94.09	0.000417182869355924\\
94.1	0.000415017696373176\\
94.11	0.000412851700068008\\
94.12	0.000410684884078975\\
94.13	0.000408517252151781\\
94.14	0.000406348808142254\\
94.15	0.000404179556019331\\
94.16	0.000402009499868199\\
94.17	0.000399838643893494\\
94.18	0.000397666992422583\\
94.19	0.000395494549908986\\
94.2	0.00039332132093583\\
94.21	0.000391147310219464\\
94.22	0.000388972522613147\\
94.23	0.000386796963110862\\
94.24	0.000384620636851208\\
94.25	0.000382443549121458\\
94.26	0.000380265705361675\\
94.27	0.000378087111169014\\
94.28	0.000375907772302071\\
94.29	0.000373727694685447\\
94.3	0.000371546884414375\\
94.31	0.000369365347759517\\
94.32	0.000367183091171898\\
94.33	0.000365000121287982\\
94.34	0.000362816444934895\\
94.35	0.000360632069135796\\
94.36	0.000358447001115437\\
94.37	0.000356261248305843\\
94.38	0.0003540748183522\\
94.39	0.000351887719118879\\
94.4	0.000349699958695672\\
94.41	0.000347511545404208\\
94.42	0.000345322487804526\\
94.43	0.000343132794701896\\
94.44	0.000340942475153786\\
94.45	0.000338751538477093\\
94.46	0.000336559994255541\\
94.47	0.000334367852347332\\
94.48	0.000332175122893011\\
94.49	0.000329981816323563\\
94.5	0.000327787943368752\\
94.51	0.000325593515065744\\
94.52	0.000323398542767927\\
94.53	0.000321203038154036\\
94.54	0.000319007013237553\\
94.55	0.000316810480376366\\
94.56	0.000314613452282741\\
94.57	0.000312415942033585\\
94.58	0.000310217963081011\\
94.59	0.000308019529263228\\
94.6	0.000305820654815761\\
94.61	0.000303621354383016\\
94.62	0.000301421643030172\\
94.63	0.00029922153625545\\
94.64	0.000297021050002759\\
94.65	0.000294820200674699\\
94.66	0.00029261900514599\\
94.67	0.000290417480777285\\
94.68	0.000288215645429414\\
94.69	0.000286013517478038\\
94.7	0.000283811115828796\\
94.71	0.000281608459932867\\
94.72	0.000279405569803015\\
94.73	0.000277202466030146\\
94.74	0.00027499916980034\\
94.75	0.000272795702912434\\
94.76	0.000270592087796092\\
94.77	0.000268388347530501\\
94.78	0.000266184505863557\\
94.79	0.000263980587231701\\
94.8	0.000261776616780342\\
94.81	0.00025957262038488\\
94.82	0.000257368624672421\\
94.83	0.000255164657044117\\
94.84	0.00025296074569822\\
94.85	0.000250756919653827\\
94.86	0.000248553208775351\\
94.87	0.000246349643797784\\
94.88	0.000244146256352676\\
94.89	0.00024194307899496\\
94.9	0.000239740145230609\\
94.91	0.000237537489545091\\
94.92	0.000235335147432772\\
94.93	0.000233133155427175\\
94.94	0.000230931551132197\\
94.95	0.000228730373254309\\
94.96	0.000226529661635687\\
94.97	0.000224329457288465\\
94.98	0.000222129802429952\\
94.99	0.000219930740519019\\
95	0.000217732316293571\\
95.01	0.000215534575809176\\
95.02	0.000213337566478936\\
95.03	0.000211141337114562\\
95.04	0.000208945937968727\\
95.05	0.000206751420778762\\
95.06	0.000204557838805495\\
95.07	0.000202365246874782\\
95.08	0.000200173701424665\\
95.09	0.00019798326055402\\
95.1	0.000195793984072696\\
95.11	0.000193605933553198\\
95.12	0.000191419172383992\\
95.13	0.000189233765824441\\
95.14	0.000187049781061445\\
95.15	0.000184867287267862\\
95.16	0.0001826863556627\\
95.17	0.000180507059573207\\
95.18	0.000178329474498888\\
95.19	0.000176153678177483\\
95.2	0.000173979750653032\\
95.21	0.00017180777434603\\
95.22	0.000169637834125774\\
95.23	0.000167470017384956\\
95.24	0.000165304414116597\\
95.25	0.000163141116993329\\
95.26	0.000160980221449215\\
95.27	0.000158821825764063\\
95.28	0.000156666031150389\\
95.29	0.000154512941843097\\
95.3	0.000152362665191962\\
95.31	0.00015021531175697\\
95.32	0.000148070995406669\\
95.33	0.000145929833419583\\
95.34	0.00014379194658878\\
95.35	0.000141657459329741\\
95.36	0.000139526499791549\\
95.37	0.000137399199971621\\
95.38	0.000135275695833992\\
95.39	0.000133156127431292\\
95.4	0.000131040639030572\\
95.41	0.000128929379243069\\
95.42	0.000126822501158003\\
95.43	0.000124720162480616\\
95.44	0.000122622525674489\\
95.45	0.000120529758108371\\
95.46	0.000118442032207572\\
95.47	0.000116359525610147\\
95.48	0.000114282421327929\\
95.49	0.000112210907912682\\
95.5	0.000110145179627414\\
95.51	0.000108085436623105\\
95.52	0.000106031885120981\\
95.53	0.000103984737600503\\
95.54	0.000101944212993284\\
95.55	9.99105368830813e-05\\
95.56	9.78839417121024e-05\\
95.57	9.5864666993772e-05\\
95.58	9.38529595322073e-05\\
95.59	9.18490736485939e-05\\
95.6	8.98532714146919e-05\\
95.61	8.78658228936815e-05\\
95.62	8.58870063885821e-05\\
95.63	8.39171086985061e-05\\
95.64	8.19564253829851e-05\\
95.65	8.00052610345752e-05\\
95.66	7.80639295601231e-05\\
95.67	7.61327544707962e-05\\
95.68	7.42120691813262e-05\\
95.69	7.23022173186299e-05\\
95.7	7.04035530401624e-05\\
95.71	6.85164413623176e-05\\
95.72	6.66412584991328e-05\\
95.73	6.47783922116847e-05\\
95.74	6.2928242168506e-05\\
95.75	6.10912203173192e-05\\
95.76	5.92677512684953e-05\\
95.77	5.74582726906043e-05\\
95.78	5.56632357183902e-05\\
95.79	5.38831053736224e-05\\
95.8	5.21183609991701e-05\\
95.81	5.03694967067762e-05\\
95.82	4.86370218389046e-05\\
95.83	4.69214614451149e-05\\
95.84	4.52233567734637e-05\\
95.85	4.35432657773325e-05\\
95.86	4.1881763638214e-05\\
95.87	4.02394433049527e-05\\
95.88	3.86169160499095e-05\\
95.89	3.70148120426572e-05\\
95.9	3.54337809416513e-05\\
95.91	3.38744925045438e-05\\
95.92	3.23376372176245e-05\\
95.93	3.08239269450564e-05\\
95.94	2.93340955984638e-05\\
95.95	2.78688998275635e-05\\
95.96	2.64291197324008e-05\\
95.97	2.50155595979962e-05\\
95.98	2.36290486519217e-05\\
95.99	2.22704418456673e-05\\
96	2.09406206604628e-05\\
96.01	1.96404939382718e-05\\
96.02	1.83709987388262e-05\\
96.03	1.7133101223402e-05\\
96.04	1.59277975662325e-05\\
96.05	1.47561148943227e-05\\
96.06	1.36191122566427e-05\\
96.07	1.25178816234809e-05\\
96.08	1.14535489169273e-05\\
96.09	1.0427275073447e-05\\
96.1	9.44025713949317e-06\\
96.11	8.49372940114022e-06\\
96.12	7.58896454880649e-06\\
96.13	6.72727487807147e-06\\
96.14	5.91001352772715e-06\\
96.15	5.13857575613029e-06\\
96.16	4.41440025701449e-06\\
96.17	3.73897051593637e-06\\
96.18	3.11381620859107e-06\\
96.19	2.54051464217316e-06\\
96.2	2.02069224111874e-06\\
96.21	1.55602607849505e-06\\
96.22	1.14824545438368e-06\\
96.23	7.99133522650514e-07\\
96.24	5.10528967472695e-07\\
96.25	2.84327731119721e-07\\
96.26	1.22484794414651e-07\\
96.27	2.70160114383688e-08\\
96.28	0\\
96.29	0\\
96.3	0\\
96.31	0\\
96.32	1.73472347597681e-18\\
96.33	1.73472347597681e-18\\
96.34	1.73472347597681e-18\\
96.35	1.73472347597681e-18\\
96.36	0\\
96.37	0\\
96.38	1.73472347597681e-18\\
96.39	1.73472347597681e-18\\
96.4	0\\
96.41	1.73472347597681e-18\\
96.42	0\\
96.43	1.73472347597681e-18\\
96.44	0\\
96.45	0\\
96.46	1.73472347597681e-18\\
96.47	0\\
96.48	1.73472347597681e-18\\
96.49	0\\
96.5	0\\
96.51	1.73472347597681e-18\\
96.52	1.73472347597681e-18\\
96.53	0\\
96.54	1.73472347597681e-18\\
96.55	0\\
96.56	0\\
96.57	0\\
96.58	1.73472347597681e-18\\
96.59	0\\
96.6	0\\
96.61	0\\
96.62	0\\
96.63	0\\
96.64	0\\
96.65	0\\
96.66	0\\
96.67	0\\
96.68	1.73472347597681e-18\\
96.69	0\\
96.7	1.73472347597681e-18\\
96.71	1.73472347597681e-18\\
96.72	0\\
96.73	0\\
96.74	0\\
96.75	0\\
96.76	0\\
96.77	1.73472347597681e-18\\
96.78	0\\
96.79	0\\
96.8	1.73472347597681e-18\\
96.81	0\\
96.82	1.73472347597681e-18\\
96.83	1.73472347597681e-18\\
96.84	1.73472347597681e-18\\
96.85	1.73472347597681e-18\\
96.86	0\\
96.87	0\\
96.88	1.73472347597681e-18\\
96.89	1.73472347597681e-18\\
96.9	0\\
96.91	1.73472347597681e-18\\
96.92	0\\
96.93	0\\
96.94	0\\
96.95	1.73472347597681e-18\\
96.96	1.73472347597681e-18\\
96.97	0\\
96.98	0\\
96.99	0\\
97	0\\
97.01	0\\
97.02	0\\
97.03	0\\
97.04	0\\
97.05	0\\
97.06	1.73472347597681e-18\\
97.07	0\\
97.08	1.73472347597681e-18\\
97.09	0\\
97.1	1.73472347597681e-18\\
97.11	0\\
97.12	1.73472347597681e-18\\
97.13	1.73472347597681e-18\\
97.14	1.73472347597681e-18\\
97.15	0\\
97.16	1.73472347597681e-18\\
97.17	0\\
97.18	0\\
97.19	0\\
97.2	0\\
97.21	0\\
97.22	0\\
97.23	0\\
97.24	0\\
97.25	0\\
97.26	0\\
97.27	0\\
97.28	0\\
97.29	0\\
97.3	0\\
97.31	1.73472347597681e-18\\
97.32	1.73472347597681e-18\\
97.33	1.73472347597681e-18\\
97.34	0\\
97.35	1.73472347597681e-18\\
97.36	0\\
97.37	1.73472347597681e-18\\
97.38	0\\
97.39	0\\
97.4	1.73472347597681e-18\\
97.41	0\\
97.42	0\\
97.43	0\\
97.44	1.73472347597681e-18\\
97.45	0\\
97.46	0\\
97.47	0\\
97.48	0\\
97.49	1.73472347597681e-18\\
97.5	1.73472347597681e-18\\
97.51	0\\
97.52	0\\
97.53	0\\
97.54	1.73472347597681e-18\\
97.55	0\\
97.56	0\\
97.57	0\\
97.58	0\\
97.59	0\\
97.6	1.73472347597681e-18\\
97.61	0\\
97.62	1.73472347597681e-18\\
97.63	0\\
97.64	1.73472347597681e-18\\
97.65	0\\
97.66	0\\
97.67	0\\
97.68	0\\
97.69	0\\
97.7	0\\
97.71	0\\
97.72	0\\
97.73	0\\
97.74	0\\
97.75	0\\
97.76	1.73472347597681e-18\\
97.77	1.73472347597681e-18\\
97.78	0\\
97.79	1.73472347597681e-18\\
97.8	0\\
97.81	0\\
97.82	0\\
97.83	0\\
97.84	0\\
97.85	0\\
97.86	0\\
97.87	0\\
97.88	0\\
97.89	0\\
97.9	0\\
97.91	0\\
97.92	0\\
97.93	0\\
97.94	0\\
97.95	1.73472347597681e-18\\
97.96	0\\
97.97	1.73472347597681e-18\\
97.98	1.73472347597681e-18\\
97.99	0\\
98	0\\
98.01	1.73472347597681e-18\\
98.02	0\\
98.03	0\\
98.04	0\\
98.05	0\\
98.06	0\\
98.07	0\\
98.08	0\\
98.09	1.73472347597681e-18\\
98.1	0\\
98.11	0\\
98.12	0\\
98.13	1.73472347597681e-18\\
98.14	0\\
98.15	0\\
98.16	0\\
98.17	1.73472347597681e-18\\
98.18	0\\
98.19	0\\
98.2	0\\
98.21	0\\
98.22	1.73472347597681e-18\\
98.23	0\\
98.24	1.73472347597681e-18\\
98.25	0\\
98.26	0\\
98.27	0\\
98.28	0\\
98.29	0\\
98.3	0\\
98.31	0\\
98.32	0\\
98.33	0\\
98.34	0\\
98.35	0\\
98.36	0\\
98.37	0\\
98.38	0\\
98.39	0\\
98.4	0\\
98.41	0\\
98.42	0\\
98.43	0\\
98.44	0\\
98.45	0\\
98.46	0\\
98.47	1.73472347597681e-18\\
98.48	0\\
98.49	0\\
98.5	0\\
98.51	1.73472347597681e-18\\
98.52	0\\
98.53	0\\
98.54	0\\
98.55	1.73472347597681e-18\\
98.56	1.73472347597681e-18\\
98.57	0\\
98.58	0\\
98.59	0\\
98.6	0\\
98.61	0\\
98.62	1.73472347597681e-18\\
98.63	0\\
98.64	0\\
98.65	0\\
98.66	0\\
98.67	0\\
98.68	0\\
98.69	0\\
98.7	0\\
98.71	0\\
98.72	0\\
98.73	1.73472347597681e-18\\
98.74	0\\
98.75	0\\
98.76	0\\
98.77	0\\
98.78	0\\
98.79	1.73472347597681e-18\\
98.8	0\\
98.81	0\\
98.82	1.73472347597681e-18\\
98.83	0\\
98.84	1.73472347597681e-18\\
98.85	0\\
98.86	0\\
98.87	1.73472347597681e-18\\
98.88	0\\
98.89	0\\
98.9	0\\
98.91	0\\
98.92	0\\
98.93	0\\
98.94	1.73472347597681e-18\\
98.95	0\\
98.96	0\\
98.97	1.73472347597681e-18\\
98.98	0\\
98.99	0\\
99	0\\
99.01	0\\
99.02	0\\
99.03	0\\
99.04	0\\
99.05	1.73472347597681e-18\\
99.06	0\\
99.07	0\\
99.08	1.73472347597681e-18\\
99.09	0\\
99.1	0\\
99.11	1.73472347597681e-18\\
99.12	0\\
99.13	0\\
99.14	0\\
99.15	0\\
99.16	0\\
99.17	1.73472347597681e-18\\
99.18	0\\
99.19	0\\
99.2	0\\
99.21	0\\
99.22	0\\
99.23	0\\
99.24	1.73472347597681e-18\\
99.25	0\\
99.26	0\\
99.27	1.73472347597681e-18\\
99.28	0\\
99.29	0\\
99.3	0\\
99.31	0\\
99.32	0\\
99.33	0\\
99.34	0\\
99.35	0\\
99.36	0\\
99.37	0\\
99.38	0\\
99.39	0\\
99.4	0\\
99.41	0\\
99.42	0\\
99.43	0\\
99.44	0\\
99.45	0\\
99.46	0\\
99.47	0\\
99.48	0\\
99.49	0\\
99.5	0\\
99.51	1.73472347597681e-18\\
99.52	0\\
99.53	0\\
99.54	0\\
99.55	0\\
99.56	0\\
99.57	0\\
99.58	0\\
99.59	0\\
99.6	0\\
99.61	1.73472347597681e-18\\
99.62	0\\
99.63	0\\
99.64	0\\
99.65	1.73472347597681e-18\\
99.66	0\\
99.67	1.73472347597681e-18\\
99.68	0\\
99.69	0\\
99.7	0\\
99.71	0\\
99.72	0\\
99.73	0\\
99.74	0\\
99.75	0\\
99.76	0\\
99.77	1.73472347597681e-18\\
99.78	0\\
99.79	0\\
99.8	0\\
99.81	0\\
99.82	0\\
99.83	0\\
99.84	0\\
99.85	0\\
99.86	0\\
99.87	0\\
99.88	0\\
99.89	0\\
99.9	0\\
99.91	0\\
99.92	0\\
99.93	0\\
99.94	0\\
99.95	0\\
99.96	0\\
99.97	0\\
99.98	0\\
99.99	0\\
100	0\\
};
\addlegendentry{$q=2$};

\addplot [color=mycolor1,solid,forget plot]
  table[row sep=crcr]{%
0.01	0.00180108844777369\\
0.02	0.00180107971834339\\
0.03	0.00180107095164265\\
0.04	0.00180106214686632\\
0.05	0.00180105330318802\\
0.06	0.00180104441975947\\
0.07	0.00180103549570996\\
0.08	0.00180102653014564\\
0.09	0.0018010175221489\\
0.1	0.00180100847077766\\
0.11	0.00180099937506472\\
0.12	0.00180099023401702\\
0.13	0.00180098104661488\\
0.14	0.00180097181181126\\
0.15	0.001800962528531\\
0.16	0.00180095319566995\\
0.17	0.00180094381209421\\
0.18	0.00180093437663922\\
0.19	0.00180092488810888\\
0.2	0.00180091534527469\\
0.21	0.00180090574687477\\
0.22	0.00180089609161289\\
0.23	0.00180088637815751\\
0.24	0.00180087660514073\\
0.25	0.00180086677115723\\
0.26	0.00180085687476322\\
0.27	0.00180084691447527\\
0.28	0.00180083688876921\\
0.29	0.0018008267960789\\
0.3	0.001800816634795\\
0.31	0.00180080640326374\\
0.32	0.00180079609978561\\
0.33	0.00180078572261401\\
0.34	0.00180077526995387\\
0.35	0.00180076473996021\\
0.36	0.00180075413073673\\
0.37	0.00180074344033423\\
0.38	0.00180073266674909\\
0.39	0.00180072180792165\\
0.4	0.00180071086173457\\
0.41	0.00180069982601111\\
0.42	0.00180068869851338\\
0.43	0.0018006774769405\\
0.44	0.00180066615892679\\
0.45	0.00180065474203981\\
0.46	0.00180064322377833\\
0.47	0.00180063160157039\\
0.48	0.00180061987277108\\
0.49	0.00180060803466047\\
0.5	0.00180059608444127\\
0.51	0.00180058401923659\\
0.52	0.00180057183608754\\
0.53	0.00180055953195076\\
0.54	0.0018005471036959\\
0.55	0.00180053454810303\\
0.56	0.0018005218618599\\
0.57	0.00180050904155921\\
0.58	0.00180049608369576\\
0.59	0.00180048298466345\\
0.6	0.0018004697407523\\
0.61	0.00180045634814526\\
0.62	0.00180044280291501\\
0.63	0.00180042910102065\\
0.64	0.00180041523830421\\
0.65	0.00180040121048716\\
0.66	0.00180038701316672\\
0.67	0.00180037264181211\\
0.68	0.00180035809176067\\
0.69	0.00180034335821385\\
0.7	0.00180032843623307\\
0.71	0.0018003133207355\\
0.72	0.00180029800648962\\
0.73	0.00180028248811072\\
0.74	0.00180026685431318\\
0.75	0.00180025121451621\\
0.76	0.00180023556871731\\
0.77	0.00180021991691399\\
0.78	0.00180020425910375\\
0.79	0.00180018859528411\\
0.8	0.00180017292545256\\
0.81	0.00180015724960661\\
0.82	0.00180014156774376\\
0.83	0.00180012587986151\\
0.84	0.00180011018595735\\
0.85	0.00180009448602879\\
0.86	0.00180007878007333\\
0.87	0.00180006306808845\\
0.88	0.00180004735007165\\
0.89	0.00180003162602043\\
0.9	0.00180001589593227\\
0.91	0.00180000015980467\\
0.92	0.00179998441763511\\
0.93	0.0017999686694211\\
0.94	0.0017999529151601\\
0.95	0.00179993715484961\\
0.96	0.00179992138848711\\
0.97	0.00179990561607008\\
0.98	0.00179988983759601\\
0.99	0.00179987405306238\\
1	0.00179985826246666\\
1.01	0.00179984246580634\\
1.02	0.00179982666307889\\
1.03	0.00179981085428179\\
1.04	0.00179979503941252\\
1.05	0.00179977921846854\\
1.06	0.00179976339144733\\
1.07	0.00179974755834635\\
1.08	0.00179973171916309\\
1.09	0.00179971587389501\\
1.1	0.00179970002253957\\
1.11	0.00179968416509424\\
1.12	0.00179966830155649\\
1.13	0.00179965243192378\\
1.14	0.00179963655619357\\
1.15	0.00179962067436332\\
1.16	0.0017996047864305\\
1.17	0.00179958889239256\\
1.18	0.00179957299224697\\
1.19	0.00179955708599117\\
1.2	0.00179954117362262\\
1.21	0.00179952525513877\\
1.22	0.00179950933053709\\
1.23	0.00179949339981501\\
1.24	0.00179947746297\\
1.25	0.00179946151999949\\
1.26	0.00179944557090094\\
1.27	0.00179942961567179\\
1.28	0.00179941365430949\\
1.29	0.00179939768681148\\
1.3	0.00179938171317521\\
1.31	0.00179936573339811\\
1.32	0.00179934974747763\\
1.33	0.00179933375541121\\
1.34	0.00179931775719628\\
1.35	0.00179930175283028\\
1.36	0.00179928574231064\\
1.37	0.0017992697256348\\
1.38	0.0017992537028002\\
1.39	0.00179923767380426\\
1.4	0.00179922163864441\\
1.41	0.00179920559731808\\
1.42	0.00179918954982271\\
1.43	0.00179917349615571\\
1.44	0.00179915743631452\\
1.45	0.00179914137029655\\
1.46	0.00179912529809923\\
1.47	0.00179910921971998\\
1.48	0.00179909313515622\\
1.49	0.00179907704440538\\
1.5	0.00179906094746486\\
1.51	0.00179904484433208\\
1.52	0.00179902873500447\\
1.53	0.00179901261947943\\
1.54	0.00179899649775438\\
1.55	0.00179898036982672\\
1.56	0.00179896423569388\\
1.57	0.00179894809535325\\
1.58	0.00179893194880224\\
1.59	0.00179891579603827\\
1.6	0.00179889963705874\\
1.61	0.00179888347186104\\
1.62	0.00179886730044259\\
1.63	0.00179885112280079\\
1.64	0.00179883493893303\\
1.65	0.00179881874883671\\
1.66	0.00179880255250923\\
1.67	0.001798786349948\\
1.68	0.0017987701411504\\
1.69	0.00179875392611382\\
1.7	0.00179873770483566\\
1.71	0.00179872147731332\\
1.72	0.00179870524354417\\
1.73	0.00179868900352562\\
1.74	0.00179867275725504\\
1.75	0.00179865650472982\\
1.76	0.00179864024594736\\
1.77	0.00179862398090502\\
1.78	0.0017986077096002\\
1.79	0.00179859143203027\\
1.8	0.00179857514819262\\
1.81	0.00179855885808462\\
1.82	0.00179854256170366\\
1.83	0.0017985262590471\\
1.84	0.00179850995011232\\
1.85	0.0017984936348967\\
1.86	0.0017984773133976\\
1.87	0.0017984609856124\\
1.88	0.00179844465153847\\
1.89	0.00179842831117318\\
1.9	0.00179841196451389\\
1.91	0.00179839561155796\\
1.92	0.00179837925230277\\
1.93	0.00179836288674568\\
1.94	0.00179834651488404\\
1.95	0.00179833013671522\\
1.96	0.00179831375223658\\
1.97	0.00179829736144548\\
1.98	0.00179828096433926\\
1.99	0.0017982645609153\\
2	0.00179824815117094\\
2.01	0.00179823173510353\\
2.02	0.00179821531271043\\
2.03	0.00179819888398898\\
2.04	0.00179818244893654\\
2.05	0.00179816600755046\\
2.06	0.00179814955982807\\
2.07	0.00179813310576673\\
2.08	0.00179811664536378\\
2.09	0.00179810017861655\\
2.1	0.0017980837055224\\
2.11	0.00179806722607866\\
2.12	0.00179805074028267\\
2.13	0.00179803424813177\\
2.14	0.00179801774962329\\
2.15	0.00179800124475456\\
2.16	0.00179798473352294\\
2.17	0.00179796821592573\\
2.18	0.00179795169196027\\
2.19	0.0017979351616239\\
2.2	0.00179791862491394\\
2.21	0.00179790208182772\\
2.22	0.00179788553236256\\
2.23	0.00179786897651579\\
2.24	0.00179785241428473\\
2.25	0.0017978358456667\\
2.26	0.00179781927065903\\
2.27	0.00179780268925902\\
2.28	0.00179778610146401\\
2.29	0.0017977695072713\\
2.3	0.0017977529066782\\
2.31	0.00179773629968205\\
2.32	0.00179771968628014\\
2.33	0.00179770306646978\\
2.34	0.00179768644024829\\
2.35	0.00179766980761298\\
2.36	0.00179765316856115\\
2.37	0.0017976365230901\\
2.38	0.00179761987119715\\
2.39	0.00179760321287959\\
2.4	0.00179758654813473\\
2.41	0.00179756987695987\\
2.42	0.0017975531993523\\
2.43	0.00179753651530933\\
2.44	0.00179751982482825\\
2.45	0.00179750312790635\\
2.46	0.00179748642454093\\
2.47	0.00179746971472929\\
2.48	0.0017974529984687\\
2.49	0.00179743627575647\\
2.5	0.00179741954658988\\
2.51	0.00179740281096622\\
2.52	0.00179738606888277\\
2.53	0.00179736932033682\\
2.54	0.00179735256532565\\
2.55	0.00179733580384654\\
2.56	0.00179731903589678\\
2.57	0.00179730226147364\\
2.58	0.0017972854805744\\
2.59	0.00179726869319634\\
2.6	0.00179725189933673\\
2.61	0.00179723509899284\\
2.62	0.00179721829216195\\
2.63	0.00179720147884133\\
2.64	0.00179718465902825\\
2.65	0.00179716783271998\\
2.66	0.00179715099991378\\
2.67	0.00179713416060692\\
2.68	0.00179711731479665\\
2.69	0.00179710046248027\\
2.7	0.001797083603655\\
2.71	0.00179706673831813\\
2.72	0.0017970498664669\\
2.73	0.00179703298809858\\
2.74	0.00179701610321042\\
2.75	0.00179699921179967\\
2.76	0.00179698231386359\\
2.77	0.00179696540939944\\
2.78	0.00179694849840446\\
2.79	0.00179693158087589\\
2.8	0.001796914656811\\
2.81	0.00179689772620703\\
2.82	0.00179688078906121\\
2.83	0.0017968638453708\\
2.84	0.00179684689513305\\
2.85	0.00179682993834518\\
2.86	0.00179681297500443\\
2.87	0.00179679600510806\\
2.88	0.00179677902865329\\
2.89	0.00179676204563737\\
2.9	0.00179674505605751\\
2.91	0.00179672805991097\\
2.92	0.00179671105719496\\
2.93	0.00179669404790673\\
2.94	0.00179667703204349\\
2.95	0.00179666000960249\\
2.96	0.00179664298058093\\
2.97	0.00179662594497605\\
2.98	0.00179660890278507\\
2.99	0.00179659185400522\\
3	0.00179657479863371\\
3.01	0.00179655773666776\\
3.02	0.0017965406681046\\
3.03	0.00179652359294143\\
3.04	0.00179650651117547\\
3.05	0.00179648942280394\\
3.06	0.00179647232782405\\
3.07	0.00179645522623301\\
3.08	0.00179643811802803\\
3.09	0.00179642100320632\\
3.1	0.00179640388176508\\
3.11	0.00179638675370153\\
3.12	0.00179636961901285\\
3.13	0.00179635247769627\\
3.14	0.00179633532974897\\
3.15	0.00179631817516817\\
3.16	0.00179630101395105\\
3.17	0.00179628384609481\\
3.18	0.00179626667159666\\
3.19	0.00179624949045377\\
3.2	0.00179623230266336\\
3.21	0.00179621510822261\\
3.22	0.0017961979071287\\
3.23	0.00179618069937884\\
3.24	0.00179616348497019\\
3.25	0.00179614626389996\\
3.26	0.00179612903616533\\
3.27	0.00179611180176347\\
3.28	0.00179609456069158\\
3.29	0.00179607731294683\\
3.3	0.0017960600585264\\
3.31	0.00179604279742747\\
3.32	0.00179602552964721\\
3.33	0.0017960082551828\\
3.34	0.00179599097403141\\
3.35	0.00179597368619022\\
3.36	0.0017959563916564\\
3.37	0.00179593909042711\\
3.38	0.00179592178249952\\
3.39	0.0017959044678708\\
3.4	0.00179588714653811\\
3.41	0.00179586981849863\\
3.42	0.0017958524837495\\
3.43	0.00179583514228789\\
3.44	0.00179581779411095\\
3.45	0.00179580043921586\\
3.46	0.00179578307759977\\
3.47	0.00179576570925981\\
3.48	0.00179574833419317\\
3.49	0.00179573095239697\\
3.5	0.00179571356386839\\
3.51	0.00179569616860456\\
3.52	0.00179567876660263\\
3.53	0.00179566135785975\\
3.54	0.00179564394237306\\
3.55	0.00179562652013972\\
3.56	0.00179560909115685\\
3.57	0.00179559165542161\\
3.58	0.00179557421293112\\
3.59	0.00179555676368254\\
3.6	0.00179553930767299\\
3.61	0.00179552184489961\\
3.62	0.00179550437535954\\
3.63	0.0017954868990499\\
3.64	0.00179546941596783\\
3.65	0.00179545192611046\\
3.66	0.00179543442947491\\
3.67	0.00179541692605831\\
3.68	0.00179539941585779\\
3.69	0.00179538189887047\\
3.7	0.00179536437509348\\
3.71	0.00179534684452392\\
3.72	0.00179532930715893\\
3.73	0.00179531176299562\\
3.74	0.0017952942120311\\
3.75	0.0017952766542625\\
3.76	0.00179525908968693\\
3.77	0.00179524151830149\\
3.78	0.0017952239401033\\
3.79	0.00179520635508947\\
3.8	0.00179518876325711\\
3.81	0.00179517116460331\\
3.82	0.00179515355912519\\
3.83	0.00179513594681985\\
3.84	0.0017951183276844\\
3.85	0.00179510070171593\\
3.86	0.00179508306891154\\
3.87	0.00179506542926833\\
3.88	0.00179504778278339\\
3.89	0.00179503012945383\\
3.9	0.00179501246927673\\
3.91	0.00179499480224919\\
3.92	0.0017949771283683\\
3.93	0.00179495944763114\\
3.94	0.0017949417600348\\
3.95	0.00179492406557637\\
3.96	0.00179490636425293\\
3.97	0.00179488865606158\\
3.98	0.00179487094099937\\
3.99	0.00179485321906341\\
4	0.00179483549025076\\
4.01	0.00179481775455851\\
4.02	0.00179480001198373\\
4.03	0.00179478226252349\\
4.04	0.00179476450617487\\
4.05	0.00179474674293494\\
4.06	0.00179472897280077\\
4.07	0.00179471119576942\\
4.08	0.00179469341183797\\
4.09	0.00179467562100347\\
4.1	0.001794657823263\\
4.11	0.00179464001861362\\
4.12	0.00179462220705238\\
4.13	0.00179460438857635\\
4.14	0.00179458656318258\\
4.15	0.00179456873086814\\
4.16	0.00179455089163007\\
4.17	0.00179453304546543\\
4.18	0.00179451519237127\\
4.19	0.00179449733234465\\
4.2	0.00179447946538261\\
4.21	0.00179446159148219\\
4.22	0.00179444371064046\\
4.23	0.00179442582285444\\
4.24	0.00179440792812119\\
4.25	0.00179439002643774\\
4.26	0.00179437211780114\\
4.27	0.00179435420220843\\
4.28	0.00179433627965664\\
4.29	0.00179431835014281\\
4.3	0.00179430041366397\\
4.31	0.00179428247021715\\
4.32	0.00179426451979939\\
4.33	0.00179424656240772\\
4.34	0.00179422859803916\\
4.35	0.00179421062669075\\
4.36	0.0017941926483595\\
4.37	0.00179417466304244\\
4.38	0.0017941566707366\\
4.39	0.00179413867143899\\
4.4	0.00179412066514663\\
4.41	0.00179410265185655\\
4.42	0.00179408463156575\\
4.43	0.00179406660427126\\
4.44	0.00179404856997008\\
4.45	0.00179403052865923\\
4.46	0.00179401248033572\\
4.47	0.00179399442499655\\
4.48	0.00179397636263875\\
4.49	0.00179395829325929\\
4.5	0.00179394021685521\\
4.51	0.00179392213342349\\
4.52	0.00179390404296114\\
4.53	0.00179388594546515\\
4.54	0.00179386784093254\\
4.55	0.00179384972936028\\
4.56	0.00179383161074539\\
4.57	0.00179381348508485\\
4.58	0.00179379535237565\\
4.59	0.00179377721261479\\
4.6	0.00179375906579924\\
4.61	0.00179374091192601\\
4.62	0.00179372275099208\\
4.63	0.00179370458299443\\
4.64	0.00179368640793005\\
4.65	0.00179366822579591\\
4.66	0.001793650036589\\
4.67	0.00179363184030629\\
4.68	0.00179361363694477\\
4.69	0.00179359542650141\\
4.7	0.00179357720897317\\
4.71	0.00179355898435703\\
4.72	0.00179354075264998\\
4.73	0.00179352251384897\\
4.74	0.00179350426795096\\
4.75	0.00179348601495293\\
4.76	0.00179346775485185\\
4.77	0.00179344948764467\\
4.78	0.00179343121332835\\
4.79	0.00179341293189987\\
4.8	0.00179339464335617\\
4.81	0.00179337634769421\\
4.82	0.00179335804491095\\
4.83	0.00179333973500334\\
4.84	0.00179332141796833\\
4.85	0.00179330309380288\\
4.86	0.00179328476250393\\
4.87	0.00179326642406844\\
4.88	0.00179324807849334\\
4.89	0.00179322972577558\\
4.9	0.00179321136591211\\
4.91	0.00179319299889986\\
4.92	0.00179317462473579\\
4.93	0.00179315624341681\\
4.94	0.00179313785493988\\
4.95	0.00179311945930192\\
4.96	0.00179310105649988\\
4.97	0.00179308264653067\\
4.98	0.00179306422939124\\
4.99	0.00179304580507852\\
5	0.00179302737358942\\
5.01	0.00179300893492088\\
5.02	0.00179299048906982\\
5.03	0.00179297203603316\\
5.04	0.00179295357580782\\
5.05	0.00179293510839073\\
5.06	0.00179291663377881\\
5.07	0.00179289815196895\\
5.08	0.0017928796629581\\
5.09	0.00179286116674315\\
5.1	0.00179284266332101\\
5.11	0.0017928241526886\\
5.12	0.00179280563484284\\
5.13	0.00179278710978061\\
5.14	0.00179276857749884\\
5.15	0.00179275003799442\\
5.16	0.00179273149126426\\
5.17	0.00179271293730525\\
5.18	0.0017926943761143\\
5.19	0.0017926758076883\\
5.2	0.00179265723202416\\
5.21	0.00179263864911875\\
5.22	0.00179262005896899\\
5.23	0.00179260146157175\\
5.24	0.00179258285692393\\
5.25	0.00179256424502241\\
5.26	0.00179254562586409\\
5.27	0.00179252699944585\\
5.28	0.00179250836576456\\
5.29	0.00179248972481712\\
5.3	0.0017924710766004\\
5.31	0.00179245242111128\\
5.32	0.00179243375834664\\
5.33	0.00179241508830336\\
5.34	0.0017923964109783\\
5.35	0.00179237772636834\\
5.36	0.00179235903447035\\
5.37	0.00179234033528121\\
5.38	0.00179232162879776\\
5.39	0.00179230291501689\\
5.4	0.00179228419393546\\
5.41	0.00179226546555033\\
5.42	0.00179224672985836\\
5.43	0.0017922279868564\\
5.44	0.00179220923654133\\
5.45	0.00179219047890999\\
5.46	0.00179217171395924\\
5.47	0.00179215294168593\\
5.48	0.00179213416208691\\
5.49	0.00179211537515903\\
5.5	0.00179209658089914\\
5.51	0.00179207777930409\\
5.52	0.00179205897037072\\
5.53	0.00179204015409588\\
5.54	0.0017920213304764\\
5.55	0.00179200249950913\\
5.56	0.00179198366119091\\
5.57	0.00179196481551857\\
5.58	0.00179194596248895\\
5.59	0.00179192710209887\\
5.6	0.00179190823434518\\
5.61	0.0017918893592247\\
5.62	0.00179187047673427\\
5.63	0.0017918515868707\\
5.64	0.00179183268963083\\
5.65	0.00179181378501148\\
5.66	0.00179179487300946\\
5.67	0.00179177595362161\\
5.68	0.00179175702684474\\
5.69	0.00179173809267567\\
5.7	0.0017917191511112\\
5.71	0.00179170020214817\\
5.72	0.00179168124578338\\
5.73	0.00179166228201363\\
5.74	0.00179164331083575\\
5.75	0.00179162433224653\\
5.76	0.00179160534624278\\
5.77	0.00179158635282131\\
5.78	0.00179156735197892\\
5.79	0.00179154834371241\\
5.8	0.00179152932801858\\
5.81	0.00179151030489423\\
5.82	0.00179149127433615\\
5.83	0.00179147223634114\\
5.84	0.00179145319090599\\
5.85	0.0017914341380275\\
5.86	0.00179141507770244\\
5.87	0.0017913960099276\\
5.88	0.00179137693469979\\
5.89	0.00179135785201578\\
5.9	0.00179133876187235\\
5.91	0.00179131966426628\\
5.92	0.00179130055919435\\
5.93	0.00179128144665334\\
5.94	0.00179126232664003\\
5.95	0.00179124319915119\\
5.96	0.0017912240641836\\
5.97	0.00179120492173402\\
5.98	0.00179118577179923\\
5.99	0.00179116661437599\\
6	0.00179114744946107\\
6.01	0.00179112827705123\\
6.02	0.00179110909714324\\
6.03	0.00179108990973385\\
6.04	0.00179107071481983\\
6.05	0.00179105151239794\\
6.06	0.00179103230246493\\
6.07	0.00179101308501755\\
6.08	0.00179099386005257\\
6.09	0.00179097462756672\\
6.1	0.00179095538755676\\
6.11	0.00179093614001944\\
6.12	0.0017909168849515\\
6.13	0.00179089762234969\\
6.14	0.00179087835221075\\
6.15	0.00179085907453142\\
6.16	0.00179083978930844\\
6.17	0.00179082049653855\\
6.18	0.00179080119621849\\
6.19	0.00179078188834498\\
6.2	0.00179076257291477\\
6.21	0.00179074324992458\\
6.22	0.00179072391937114\\
6.23	0.00179070458125118\\
6.24	0.00179068523556143\\
6.25	0.00179066588229861\\
6.26	0.00179064652145944\\
6.27	0.00179062715304064\\
6.28	0.00179060777703893\\
6.29	0.00179058839345103\\
6.3	0.00179056900227367\\
6.31	0.00179054960350354\\
6.32	0.00179053019713736\\
6.33	0.00179051078317185\\
6.34	0.0017904913616037\\
6.35	0.00179047193242964\\
6.36	0.00179045249564636\\
6.37	0.00179043305125056\\
6.38	0.00179041359923897\\
6.39	0.00179039413960826\\
6.4	0.00179037467235513\\
6.41	0.0017903551974763\\
6.42	0.00179033571496845\\
6.43	0.00179031622482828\\
6.44	0.00179029672705247\\
6.45	0.00179027722163772\\
6.46	0.00179025770858071\\
6.47	0.00179023818787814\\
6.48	0.00179021865952669\\
6.49	0.00179019912352304\\
6.5	0.00179017957986387\\
6.51	0.00179016002854586\\
6.52	0.0017901404695657\\
6.53	0.00179012090292005\\
6.54	0.00179010132860559\\
6.55	0.001790081746619\\
6.56	0.00179006215695694\\
6.57	0.00179004255961609\\
6.58	0.00179002295459311\\
6.59	0.00179000334188468\\
6.6	0.00178998372148744\\
6.61	0.00178996409339807\\
6.62	0.00178994445761323\\
6.63	0.00178992481412958\\
6.64	0.00178990516294377\\
6.65	0.00178988550405246\\
6.66	0.0017898658374523\\
6.67	0.00178984616313995\\
6.68	0.00178982648111206\\
6.69	0.00178980679136526\\
6.7	0.00178978709389623\\
6.71	0.00178976738870158\\
6.72	0.00178974767577798\\
6.73	0.00178972795512206\\
6.74	0.00178970822673046\\
6.75	0.00178968849059983\\
6.76	0.00178966874672679\\
6.77	0.00178964899510797\\
6.78	0.00178962923574003\\
6.79	0.00178960946861958\\
6.8	0.00178958969374326\\
6.81	0.00178956991110768\\
6.82	0.00178955012070949\\
6.83	0.0017895303225453\\
6.84	0.00178951051661174\\
6.85	0.00178949070290542\\
6.86	0.00178947088142297\\
6.87	0.001789451052161\\
6.88	0.00178943121511612\\
6.89	0.00178941137028496\\
6.9	0.00178939151766412\\
6.91	0.00178937165725022\\
6.92	0.00178935178903985\\
6.93	0.00178933191302964\\
6.94	0.00178931202921617\\
6.95	0.00178929213759607\\
6.96	0.00178927223816592\\
6.97	0.00178925233092232\\
6.98	0.00178923241586189\\
6.99	0.0017892124929812\\
7	0.00178919256227686\\
7.01	0.00178917262374545\\
7.02	0.00178915267738357\\
7.03	0.00178913272318781\\
7.04	0.00178911276115475\\
7.05	0.00178909279128098\\
7.06	0.00178907281356308\\
7.07	0.00178905282799764\\
7.08	0.00178903283458123\\
7.09	0.00178901283331043\\
7.1	0.00178899282418183\\
7.11	0.00178897280719198\\
7.12	0.00178895278233747\\
7.13	0.00178893274961486\\
7.14	0.00178891270902073\\
7.15	0.00178889266055165\\
7.16	0.00178887260420417\\
7.17	0.00178885253997487\\
7.18	0.0017888324678603\\
7.19	0.00178881238785703\\
7.2	0.0017887922999616\\
7.21	0.00178877220417059\\
7.22	0.00178875210048054\\
7.23	0.00178873198888801\\
7.24	0.00178871186938955\\
7.25	0.00178869174198171\\
7.26	0.00178867160666103\\
7.27	0.00178865146342407\\
7.28	0.00178863131226735\\
7.29	0.00178861115318744\\
7.3	0.00178859098618086\\
7.31	0.00178857081124416\\
7.32	0.00178855062837388\\
7.33	0.00178853043756654\\
7.34	0.00178851023881868\\
7.35	0.00178849003212683\\
7.36	0.00178846981748753\\
7.37	0.0017884495948973\\
7.38	0.00178842936435266\\
7.39	0.00178840912585014\\
7.4	0.00178838887938626\\
7.41	0.00178836862495754\\
7.42	0.0017883483625605\\
7.43	0.00178832809219166\\
7.44	0.00178830781384753\\
7.45	0.00178828752752463\\
7.46	0.00178826723321946\\
7.47	0.00178824693092854\\
7.48	0.00178822662064837\\
7.49	0.00178820630237546\\
7.5	0.00178818597610631\\
7.51	0.00178816564183742\\
7.52	0.0017881452995653\\
7.53	0.00178812494928645\\
7.54	0.00178810459099735\\
7.55	0.0017880842246945\\
7.56	0.00178806385037441\\
7.57	0.00178804346803354\\
7.58	0.00178802307766841\\
7.59	0.00178800267927549\\
7.6	0.00178798227285128\\
7.61	0.00178796185839224\\
7.62	0.00178794143589487\\
7.63	0.00178792100535564\\
7.64	0.00178790056677104\\
7.65	0.00178788012013754\\
7.66	0.00178785966545161\\
7.67	0.00178783920270973\\
7.68	0.00178781873190836\\
7.69	0.00178779825304398\\
7.7	0.00178777776611305\\
7.71	0.00178775727111204\\
7.72	0.00178773676803741\\
7.73	0.00178771625688562\\
7.74	0.00178769573765313\\
7.75	0.0017876752103364\\
7.76	0.00178765467493189\\
7.77	0.00178763413143605\\
7.78	0.00178761357984532\\
7.79	0.00178759302015617\\
7.8	0.00178757245236503\\
7.81	0.00178755187646836\\
7.82	0.00178753129246261\\
7.83	0.0017875107003442\\
7.84	0.00178749010010959\\
7.85	0.00178746949175521\\
7.86	0.0017874488752775\\
7.87	0.00178742825067289\\
7.88	0.00178740761793782\\
7.89	0.00178738697706872\\
7.9	0.00178736632806202\\
7.91	0.00178734567091414\\
7.92	0.00178732500562152\\
7.93	0.00178730433218057\\
7.94	0.00178728365058772\\
7.95	0.00178726296083939\\
7.96	0.001787242262932\\
7.97	0.00178722155686195\\
7.98	0.00178720084262568\\
7.99	0.00178718012021958\\
8	0.00178715938964007\\
8.01	0.00178713865088356\\
8.02	0.00178711790394646\\
8.03	0.00178709714882516\\
8.04	0.00178707638551608\\
8.05	0.00178705561401561\\
8.06	0.00178703483432016\\
8.07	0.00178701404642612\\
8.08	0.00178699325032988\\
8.09	0.00178697244602785\\
8.1	0.0017869516335164\\
8.11	0.00178693081279193\\
8.12	0.00178690998385084\\
8.13	0.0017868891466895\\
8.14	0.0017868683013043\\
8.15	0.00178684744769161\\
8.16	0.00178682658584783\\
8.17	0.00178680571576933\\
8.18	0.00178678483745249\\
8.19	0.00178676395089367\\
8.2	0.00178674305608926\\
8.21	0.00178672215303561\\
8.22	0.00178670124172911\\
8.23	0.00178668032216612\\
8.24	0.00178665939434301\\
8.25	0.00178663845825612\\
8.26	0.00178661751390184\\
8.27	0.00178659656127651\\
8.28	0.00178657560037649\\
8.29	0.00178655463119815\\
8.3	0.00178653365373782\\
8.31	0.00178651266799186\\
8.32	0.00178649167395663\\
8.33	0.00178647067162846\\
8.34	0.00178644966100371\\
8.35	0.00178642864207872\\
8.36	0.00178640761484983\\
8.37	0.00178638657931338\\
8.38	0.00178636553546571\\
8.39	0.00178634448330315\\
8.4	0.00178632342282203\\
8.41	0.00178630235401871\\
8.42	0.00178628127688948\\
8.43	0.00178626019143069\\
8.44	0.00178623909763867\\
8.45	0.00178621799550973\\
8.46	0.00178619688504021\\
8.47	0.00178617576622641\\
8.48	0.00178615463906466\\
8.49	0.00178613350355128\\
8.5	0.00178611235968257\\
8.51	0.00178609120745486\\
8.52	0.00178607004686445\\
8.53	0.00178604887790765\\
8.54	0.00178602770058076\\
8.55	0.0017860065148801\\
8.56	0.00178598532080196\\
8.57	0.00178596411834265\\
8.58	0.00178594290749846\\
8.59	0.00178592168826569\\
8.6	0.00178590046064064\\
8.61	0.00178587922461959\\
8.62	0.00178585798019885\\
8.63	0.00178583672737469\\
8.64	0.0017858154661434\\
8.65	0.00178579419650128\\
8.66	0.0017857729184446\\
8.67	0.00178575163196964\\
8.68	0.00178573033707269\\
8.69	0.00178570903375002\\
8.7	0.0017856877219979\\
8.71	0.00178566640181261\\
8.72	0.00178564507319042\\
8.73	0.0017856237361276\\
8.74	0.00178560239062042\\
8.75	0.00178558103666513\\
8.76	0.00178555967425801\\
8.77	0.00178553830339532\\
8.78	0.00178551692407331\\
8.79	0.00178549553628825\\
8.8	0.00178547414003638\\
8.81	0.00178545273531396\\
8.82	0.00178543132211725\\
8.83	0.00178540990044248\\
8.84	0.00178538847028592\\
8.85	0.0017853670316438\\
8.86	0.00178534558451237\\
8.87	0.00178532412888787\\
8.88	0.00178530266476653\\
8.89	0.0017852811921446\\
8.9	0.00178525971101832\\
8.91	0.00178523822138391\\
8.92	0.0017852167232376\\
8.93	0.00178519521657563\\
8.94	0.00178517370139422\\
8.95	0.00178515217768961\\
8.96	0.001785130645458\\
8.97	0.00178510910469563\\
8.98	0.00178508755539871\\
8.99	0.00178506599756346\\
9	0.0017850444311861\\
9.01	0.00178502285626283\\
9.02	0.00178500127278988\\
9.03	0.00178497968076344\\
9.04	0.00178495808017974\\
9.05	0.00178493647103496\\
9.06	0.00178491485332533\\
9.07	0.00178489322704703\\
9.08	0.00178487159219626\\
9.09	0.00178484994876924\\
9.1	0.00178482829676213\\
9.11	0.00178480663617116\\
9.12	0.00178478496699249\\
9.13	0.00178476328922233\\
9.14	0.00178474160285686\\
9.15	0.00178471990789227\\
9.16	0.00178469820432473\\
9.17	0.00178467649215043\\
9.18	0.00178465477136556\\
9.19	0.00178463304196628\\
9.2	0.00178461130394877\\
9.21	0.00178458955730921\\
9.22	0.00178456780204376\\
9.23	0.0017845460381486\\
9.24	0.00178452426561989\\
9.25	0.0017845024844538\\
9.26	0.00178448069464648\\
9.27	0.00178445889619411\\
9.28	0.00178443708909284\\
9.29	0.00178441527333882\\
9.3	0.00178439344892822\\
9.31	0.00178437161585718\\
9.32	0.00178434977412185\\
9.33	0.00178432792371839\\
9.34	0.00178430606464294\\
9.35	0.00178428419689165\\
9.36	0.00178426232046065\\
9.37	0.0017842404353461\\
9.38	0.00178421854154412\\
9.39	0.00178419663905086\\
9.4	0.00178417472786245\\
9.41	0.00178415280797501\\
9.42	0.00178413087938469\\
9.43	0.00178410894208761\\
9.44	0.00178408699607989\\
9.45	0.00178406504135767\\
9.46	0.00178404307791707\\
9.47	0.00178402110575419\\
9.48	0.00178399912486517\\
9.49	0.00178397713524612\\
9.5	0.00178395513689315\\
9.51	0.00178393312980237\\
9.52	0.0017839111139699\\
9.53	0.00178388908939184\\
9.54	0.0017838670560643\\
9.55	0.00178384501398339\\
9.56	0.00178382296314519\\
9.57	0.00178380090354583\\
9.58	0.00178377883518139\\
9.59	0.00178375675804797\\
9.6	0.00178373467214166\\
9.61	0.00178371257745855\\
9.62	0.00178369047399474\\
9.63	0.00178366836174631\\
9.64	0.00178364624070935\\
9.65	0.00178362411087993\\
9.66	0.00178360197225415\\
9.67	0.00178357982482809\\
9.68	0.00178355766859781\\
9.69	0.0017835355035594\\
9.7	0.00178351332970893\\
9.71	0.00178349114704247\\
9.72	0.00178346895555609\\
9.73	0.00178344675524585\\
9.74	0.00178342454610783\\
9.75	0.00178340232813808\\
9.76	0.00178338010133267\\
9.77	0.00178335786568765\\
9.78	0.00178333562119908\\
9.79	0.00178331336786303\\
9.8	0.00178329110567553\\
9.81	0.00178326883463265\\
9.82	0.00178324655473042\\
9.83	0.0017832242659649\\
9.84	0.00178320196833214\\
9.85	0.00178317966182817\\
9.86	0.00178315734644904\\
9.87	0.00178313502219078\\
9.88	0.00178311268904943\\
9.89	0.00178309034702103\\
9.9	0.00178306799610161\\
9.91	0.0017830456362872\\
9.92	0.00178302326757382\\
9.93	0.00178300088995751\\
9.94	0.00178297850343429\\
9.95	0.00178295610800018\\
9.96	0.00178293370365121\\
9.97	0.00178291129038338\\
9.98	0.00178288886819272\\
9.99	0.00178286643707525\\
10	0.00178284399702697\\
10.01	0.00178282154804389\\
10.02	0.00178279909012202\\
10.03	0.00178277662325737\\
10.04	0.00178275414744596\\
10.05	0.00178273166268375\\
10.06	0.00178270916896678\\
10.07	0.00178268666629104\\
10.08	0.00178266415465251\\
10.09	0.0017826416340472\\
10.1	0.0017826191044711\\
10.11	0.00178259656592019\\
10.12	0.00178257401839047\\
10.13	0.00178255146187792\\
10.14	0.00178252889637853\\
10.15	0.00178250632188827\\
10.16	0.00178248373840313\\
10.17	0.0017824611459191\\
10.18	0.00178243854443213\\
10.19	0.00178241593393821\\
10.2	0.00178239331443331\\
10.21	0.0017823706859134\\
10.22	0.00178234804837445\\
10.23	0.00178232540181242\\
10.24	0.00178230274622329\\
10.25	0.001782280081603\\
10.26	0.00178225740794753\\
10.27	0.00178223472525282\\
10.28	0.00178221203351485\\
10.29	0.00178218933272955\\
10.3	0.00178216662289288\\
10.31	0.0017821439040008\\
10.32	0.00178212117604926\\
10.33	0.0017820984390342\\
10.34	0.00178207569295155\\
10.35	0.00178205293779728\\
10.36	0.00178203017356731\\
10.37	0.00178200740025759\\
10.38	0.00178198461786406\\
10.39	0.00178196182638264\\
10.4	0.00178193902580927\\
10.41	0.00178191621613989\\
10.42	0.00178189339737042\\
10.43	0.00178187056949679\\
10.44	0.00178184773251493\\
10.45	0.00178182488642076\\
10.46	0.0017818020312102\\
10.47	0.00178177916687916\\
10.48	0.00178175629342357\\
10.49	0.00178173341083936\\
10.5	0.00178171051912241\\
10.51	0.00178168761826866\\
10.52	0.00178166470827401\\
10.53	0.00178164178913437\\
10.54	0.00178161886084564\\
10.55	0.00178159592340374\\
10.56	0.00178157297680456\\
10.57	0.001781550021044\\
10.58	0.00178152705611797\\
10.59	0.00178150408202236\\
10.6	0.00178148109875307\\
10.61	0.00178145810630599\\
10.62	0.00178143510467701\\
10.63	0.00178141209386203\\
10.64	0.00178138907385692\\
10.65	0.00178136604465759\\
10.66	0.00178134300625991\\
10.67	0.00178131995865976\\
10.68	0.00178129690185303\\
10.69	0.0017812738358356\\
10.7	0.00178125076060334\\
10.71	0.00178122767615213\\
10.72	0.00178120458247784\\
10.73	0.00178118147957636\\
10.74	0.00178115836744354\\
10.75	0.00178113524607526\\
10.76	0.00178111211546738\\
10.77	0.00178108897561578\\
10.78	0.0017810658265163\\
10.79	0.00178104266816483\\
10.8	0.00178101950055721\\
10.81	0.00178099632368931\\
10.82	0.00178097313755698\\
10.83	0.00178094994215608\\
10.84	0.00178092673748247\\
10.85	0.00178090352353199\\
10.86	0.00178088030030049\\
10.87	0.00178085706778384\\
10.88	0.00178083382597787\\
10.89	0.00178081057487843\\
10.9	0.00178078731448136\\
10.91	0.00178076404478251\\
10.92	0.00178074076577772\\
10.93	0.00178071747746283\\
10.94	0.00178069417983368\\
10.95	0.0017806708728861\\
10.96	0.00178064755661593\\
10.97	0.001780624231019\\
10.98	0.00178060089609116\\
10.99	0.00178057755182822\\
11	0.00178055419822602\\
11.01	0.00178053083528039\\
11.02	0.00178050746298714\\
11.03	0.00178048408134212\\
11.04	0.00178046069034115\\
11.05	0.00178043728998004\\
11.06	0.00178041388025461\\
11.07	0.00178039046116069\\
11.08	0.0017803670326941\\
11.09	0.00178034359485065\\
11.1	0.00178032014762616\\
11.11	0.00178029669101645\\
11.12	0.00178027322501733\\
11.13	0.00178024974962461\\
11.14	0.0017802262648341\\
11.15	0.00178020277064162\\
11.16	0.00178017926704297\\
11.17	0.00178015575403396\\
11.18	0.0017801322316104\\
11.19	0.00178010869976809\\
11.2	0.00178008515850284\\
11.21	0.00178006160781046\\
11.22	0.00178003804768674\\
11.23	0.00178001447812748\\
11.24	0.00177999089912849\\
11.25	0.00177996731068557\\
11.26	0.00177994371279452\\
11.27	0.00177992010545113\\
11.28	0.0017798964886512\\
11.29	0.00177987286239053\\
11.3	0.00177984922666491\\
11.31	0.00177982558147014\\
11.32	0.00177980192680201\\
11.33	0.00177977826265632\\
11.34	0.00177975458902885\\
11.35	0.00177973090591539\\
11.36	0.00177970721331175\\
11.37	0.00177968351121371\\
11.38	0.00177965979961706\\
11.39	0.00177963607851758\\
11.4	0.00177961234791107\\
11.41	0.00177958860779332\\
11.42	0.00177956485816011\\
11.43	0.00177954109900724\\
11.44	0.00177951733033048\\
11.45	0.00177949355212561\\
11.46	0.00177946976438845\\
11.47	0.00177944596711475\\
11.48	0.00177942216030031\\
11.49	0.00177939834394092\\
11.5	0.00177937451803237\\
11.51	0.00177935068257043\\
11.52	0.00177932683755088\\
11.53	0.00177930298296953\\
11.54	0.00177927911882214\\
11.55	0.00177925524510451\\
11.56	0.00177923136181241\\
11.57	0.00177920746894164\\
11.58	0.00177918356648798\\
11.59	0.00177915965444722\\
11.6	0.00177913573281513\\
11.61	0.0017791118015875\\
11.62	0.00177908786076013\\
11.63	0.0017790639103288\\
11.64	0.00177903995028929\\
11.65	0.00177901598063739\\
11.66	0.00177899200136888\\
11.67	0.00177896801247957\\
11.68	0.00177894401396523\\
11.69	0.00177892000582165\\
11.7	0.00177889598804463\\
11.71	0.00177887196062996\\
11.72	0.00177884792357341\\
11.73	0.00177882387687081\\
11.74	0.00177879982051792\\
11.75	0.00177877575451056\\
11.76	0.00177875167884451\\
11.77	0.00177872759351557\\
11.78	0.00177870349851954\\
11.79	0.00177867939385222\\
11.8	0.0017786552795094\\
11.81	0.00177863115548689\\
11.82	0.0017786070217805\\
11.83	0.00177858287838603\\
11.84	0.00177855872529928\\
11.85	0.00177853456251607\\
11.86	0.0017785103900322\\
11.87	0.00177848620784348\\
11.88	0.00177846201594573\\
11.89	0.00177843781433477\\
11.9	0.0017784136030064\\
11.91	0.00177838938195645\\
11.92	0.00177836515118075\\
11.93	0.00177834091067511\\
11.94	0.00177831666043537\\
11.95	0.00177829240045734\\
11.96	0.00177826813073686\\
11.97	0.00177824385126977\\
11.98	0.00177821956205191\\
11.99	0.0017781952630791\\
12	0.0017781709543472\\
12.01	0.00177814663585204\\
12.02	0.00177812230758949\\
12.03	0.00177809796955537\\
12.04	0.00177807362174557\\
12.05	0.00177804926415592\\
12.06	0.00177802489678229\\
12.07	0.00177800051962055\\
12.08	0.00177797613266656\\
12.09	0.0017779517359162\\
12.1	0.00177792732936534\\
12.11	0.00177790291300987\\
12.12	0.00177787848684566\\
12.13	0.0017778540508686\\
12.14	0.00177782960507459\\
12.15	0.00177780514945953\\
12.16	0.00177778068401931\\
12.17	0.00177775620874985\\
12.18	0.00177773172364704\\
12.19	0.00177770722870682\\
12.2	0.00177768272392509\\
12.21	0.00177765820929779\\
12.22	0.00177763368482084\\
12.23	0.0017776091504902\\
12.24	0.00177758460630178\\
12.25	0.00177756005225155\\
12.26	0.00177753548833546\\
12.27	0.00177751091454946\\
12.28	0.00177748633088953\\
12.29	0.00177746173735163\\
12.3	0.00177743713393175\\
12.31	0.00177741252062586\\
12.32	0.00177738789742997\\
12.33	0.00177736326434006\\
12.34	0.00177733862135215\\
12.35	0.00177731396846225\\
12.36	0.00177728930566637\\
12.37	0.00177726463296056\\
12.38	0.00177723995034083\\
12.39	0.00177721525780324\\
12.4	0.00177719055534383\\
12.41	0.00177716584295868\\
12.42	0.00177714112064383\\
12.43	0.00177711638839537\\
12.44	0.00177709164620939\\
12.45	0.00177706689408198\\
12.46	0.00177704213200923\\
12.47	0.00177701735998728\\
12.48	0.00177699257801223\\
12.49	0.00177696778608022\\
12.5	0.00177694298418738\\
12.51	0.00177691817232988\\
12.52	0.00177689335050388\\
12.53	0.00177686851870553\\
12.54	0.00177684367693104\\
12.55	0.00177681882517659\\
12.56	0.00177679396343839\\
12.57	0.00177676909171265\\
12.58	0.0017767442099956\\
12.59	0.00177671931828349\\
12.6	0.00177669441657257\\
12.61	0.00177666950485909\\
12.62	0.00177664458313933\\
12.63	0.0017766196514096\\
12.64	0.00177659470966618\\
12.65	0.00177656975790539\\
12.66	0.00177654479612357\\
12.67	0.00177651982431706\\
12.68	0.00177649484248221\\
12.69	0.00177646985061539\\
12.7	0.00177644484871299\\
12.71	0.00177641983677142\\
12.72	0.00177639481478708\\
12.73	0.00177636978275641\\
12.74	0.00177634474067585\\
12.75	0.00177631968854187\\
12.76	0.00177629462635094\\
12.77	0.00177626955409957\\
12.78	0.00177624447178425\\
12.79	0.00177621937940152\\
12.8	0.00177619427694793\\
12.81	0.00177616916442004\\
12.82	0.00177614404181443\\
12.83	0.00177611890912771\\
12.84	0.00177609376635647\\
12.85	0.00177606861349738\\
12.86	0.00177604345054707\\
12.87	0.00177601827750223\\
12.88	0.00177599309435956\\
12.89	0.00177596790111576\\
12.9	0.00177594269776758\\
12.91	0.00177591748431176\\
12.92	0.00177589226074509\\
12.93	0.00177586702706436\\
12.94	0.00177584178326641\\
12.95	0.00177581652934805\\
12.96	0.00177579126530618\\
12.97	0.00177576599113766\\
12.98	0.00177574070683942\\
12.99	0.00177571541240838\\
13	0.0017756901078415\\
13.01	0.00177566479313577\\
13.02	0.0017756394682882\\
13.03	0.00177561413329581\\
13.04	0.00177558878815565\\
13.05	0.00177556343286483\\
13.06	0.00177553806742043\\
13.07	0.0017755126918196\\
13.08	0.0017754873060595\\
13.09	0.00177546191013731\\
13.1	0.00177543650405025\\
13.11	0.00177541108779557\\
13.12	0.00177538566137053\\
13.13	0.00177536022477244\\
13.14	0.00177533477799863\\
13.15	0.00177530932104644\\
13.16	0.00177528385391328\\
13.17	0.00177525837659656\\
13.18	0.00177523288909372\\
13.19	0.00177520739140226\\
13.2	0.00177518188351967\\
13.21	0.00177515636544349\\
13.22	0.00177513083717131\\
13.23	0.00177510529870073\\
13.24	0.00177507975002938\\
13.25	0.00177505419115495\\
13.26	0.00177502862207512\\
13.27	0.00177500304278765\\
13.28	0.0017749774532903\\
13.29	0.00177495185358089\\
13.3	0.00177492624365725\\
13.31	0.00177490062351726\\
13.32	0.00177487499315884\\
13.33	0.00177484935257993\\
13.34	0.00177482370177851\\
13.35	0.00177479804075262\\
13.36	0.0017747723695003\\
13.37	0.00177474668801966\\
13.38	0.00177472099630882\\
13.39	0.00177469529436596\\
13.4	0.00177466958218928\\
13.41	0.00177464385977704\\
13.42	0.00177461812712752\\
13.43	0.00177459238423904\\
13.44	0.00177456663110997\\
13.45	0.00177454086773871\\
13.46	0.00177451509412371\\
13.47	0.00177448931026345\\
13.48	0.00177446351615646\\
13.49	0.00177443771180129\\
13.5	0.00177441189719655\\
13.51	0.00177438607234089\\
13.52	0.001774360237233\\
13.53	0.0017743343918716\\
13.54	0.00177430853625546\\
13.55	0.0017742826703834\\
13.56	0.00177425679425426\\
13.57	0.00177423090786695\\
13.58	0.00177420501122039\\
13.59	0.00177417910431356\\
13.6	0.00177415318714549\\
13.61	0.00177412725971523\\
13.62	0.00177410132202188\\
13.63	0.0017740753740646\\
13.64	0.00177404941584256\\
13.65	0.001774023447355\\
13.66	0.00177399746860118\\
13.67	0.00177397147958042\\
13.68	0.00177394548029206\\
13.69	0.0017739194707355\\
13.7	0.00177389345091017\\
13.71	0.00177386742081555\\
13.72	0.00177384138045115\\
13.73	0.00177381532981652\\
13.74	0.00177378926891126\\
13.75	0.00177376319773499\\
13.76	0.00177373711628739\\
13.77	0.00177371102456818\\
13.78	0.00177368492257708\\
13.79	0.00177365881031389\\
13.8	0.00177363268777842\\
13.81	0.00177360655497054\\
13.82	0.00177358041189012\\
13.83	0.0017735542585371\\
13.84	0.00177352809491142\\
13.85	0.00177350192101308\\
13.86	0.00177347573684211\\
13.87	0.00177344954239854\\
13.88	0.00177342333768246\\
13.89	0.00177339712269398\\
13.9	0.00177337089743324\\
13.91	0.00177334466190038\\
13.92	0.00177331841609561\\
13.93	0.00177329216001913\\
13.94	0.00177326589367116\\
13.95	0.00177323961705197\\
13.96	0.00177321333016181\\
13.97	0.00177318703300097\\
13.98	0.00177316072556976\\
13.99	0.00177313440786849\\
14	0.00177310807989748\\
14.01	0.00177308174165706\\
14.02	0.00177305539314757\\
14.03	0.00177302903436938\\
14.04	0.00177300266532281\\
14.05	0.00177297628600822\\
14.06	0.00177294989642595\\
14.07	0.00177292349657635\\
14.08	0.00177289708645975\\
14.09	0.00177287066607647\\
14.1	0.00177284423542682\\
14.11	0.00177281779451109\\
14.12	0.00177279134332956\\
14.13	0.00177276488188248\\
14.14	0.00177273841017008\\
14.15	0.00177271192819254\\
14.16	0.00177268543595003\\
14.17	0.00177265893344268\\
14.18	0.00177263242067056\\
14.19	0.00177260589763373\\
14.2	0.00177257936433216\\
14.21	0.0017725528207658\\
14.22	0.00177252626693452\\
14.23	0.00177249970283815\\
14.24	0.00177247312847643\\
14.25	0.00177244654384904\\
14.26	0.00177241994895559\\
14.27	0.00177239334379559\\
14.28	0.00177236672836848\\
14.29	0.00177234010267361\\
14.3	0.00177231346671021\\
14.31	0.00177228682047743\\
14.32	0.00177226016397431\\
14.33	0.00177223349719975\\
14.34	0.00177220682015255\\
14.35	0.00177218013283138\\
14.36	0.00177215343523478\\
14.37	0.00177212672736115\\
14.38	0.00177210000920871\\
14.39	0.00177207328077559\\
14.4	0.00177204654205969\\
14.41	0.00177201979305879\\
14.42	0.00177199303377048\\
14.43	0.00177196626419217\\
14.44	0.00177193948432107\\
14.45	0.0017719126941542\\
14.46	0.00177188589368837\\
14.47	0.00177185908292019\\
14.48	0.00177183226184604\\
14.49	0.00177180543046206\\
14.5	0.00177177858876416\\
14.51	0.00177175173674803\\
14.52	0.00177172487440906\\
14.53	0.00177169800174241\\
14.54	0.00177167111874295\\
14.55	0.00177164422540529\\
14.56	0.00177161732172375\\
14.57	0.00177159040769232\\
14.58	0.00177156348330473\\
14.59	0.00177153654855438\\
14.6	0.00177150960343432\\
14.61	0.00177148264793732\\
14.62	0.00177145568205577\\
14.63	0.00177142870578173\\
14.64	0.0017714017191069\\
14.65	0.00177137472202263\\
14.66	0.00177134771451988\\
14.67	0.00177132069658922\\
14.68	0.00177129366822088\\
14.69	0.00177126662940465\\
14.7	0.00177123958012993\\
14.71	0.00177121252038574\\
14.72	0.00177118545016064\\
14.73	0.00177115836944281\\
14.74	0.00177113127821999\\
14.75	0.00177110417647949\\
14.76	0.00177107706420819\\
14.77	0.00177104994139252\\
14.78	0.0017710228080185\\
14.79	0.00177099566407168\\
14.8	0.00177096850953717\\
14.81	0.00177094134439964\\
14.82	0.00177091416864331\\
14.83	0.00177088698225199\\
14.84	0.001770859785209\\
14.85	0.00177083257749725\\
14.86	0.00177080535909924\\
14.87	0.001770778129997\\
14.88	0.00177075089017219\\
14.89	0.00177072363960602\\
14.9	0.00177069637827936\\
14.91	0.00177066910617265\\
14.92	0.00177064182326598\\
14.93	0.0017706145295391\\
14.94	0.00177058722497142\\
14.95	0.00177055990954206\\
14.96	0.00177053258322985\\
14.97	0.00177050524601336\\
14.98	0.00177047789787094\\
14.99	0.00177045053878076\\
15	0.00177042316872083\\
15.01	0.00177039578766906\\
15.02	0.00177036839560328\\
15.03	0.00177034099250133\\
15.04	0.00177031357834105\\
15.05	0.00177028615310039\\
15.06	0.00177025871675748\\
15.07	0.00177023126929064\\
15.08	0.00177020381067849\\
15.09	0.00177017634090007\\
15.1	0.00177014885993483\\
15.11	0.0017701213677628\\
15.12	0.00177009386436467\\
15.13	0.00177006634972188\\
15.14	0.00177003882381673\\
15.15	0.00177001128663254\\
15.16	0.00176998373815374\\
15.17	0.00176995617836603\\
15.18	0.00176992860725654\\
15.19	0.00176990102481396\\
15.2	0.00176987343102872\\
15.21	0.0017698458258932\\
15.22	0.00176981820940189\\
15.23	0.00176979058154947\\
15.24	0.00176976294233062\\
15.25	0.00176973529174004\\
15.26	0.0017697076297724\\
15.27	0.00176967995642237\\
15.28	0.00176965227168463\\
15.29	0.00176962457555385\\
15.3	0.0017695968680247\\
15.31	0.00176956914909183\\
15.32	0.00176954141874992\\
15.33	0.00176951367699361\\
15.34	0.00176948592381757\\
15.35	0.00176945815921644\\
15.36	0.00176943038318489\\
15.37	0.00176940259571754\\
15.38	0.00176937479680906\\
15.39	0.00176934698645407\\
15.4	0.00176931916464722\\
15.41	0.00176929133138314\\
15.42	0.00176926348665647\\
15.43	0.00176923563046182\\
15.44	0.00176920776279383\\
15.45	0.00176917988364713\\
15.46	0.00176915199301633\\
15.47	0.00176912409089604\\
15.48	0.00176909617728089\\
15.49	0.00176906825216549\\
15.5	0.00176904031554443\\
15.51	0.00176901236741234\\
15.52	0.00176898440776381\\
15.53	0.00176895643659344\\
15.54	0.00176892845389583\\
15.55	0.00176890045966557\\
15.56	0.00176887245389725\\
15.57	0.00176884443658547\\
15.58	0.0017688164077248\\
15.59	0.00176878836730982\\
15.6	0.00176876031533512\\
15.61	0.00176873225179527\\
15.62	0.00176870417668485\\
15.63	0.00176867608999842\\
15.64	0.00176864799173055\\
15.65	0.00176861988187582\\
15.66	0.00176859176042876\\
15.67	0.00176856362738395\\
15.68	0.00176853548273594\\
15.69	0.00176850732647929\\
15.7	0.00176847915860854\\
15.71	0.00176845097911823\\
15.72	0.00176842278800291\\
15.73	0.00176839458525713\\
15.74	0.00176836637087541\\
15.75	0.0017683381448523\\
15.76	0.00176830990718232\\
15.77	0.00176828165786\\
15.78	0.00176825339687987\\
15.79	0.00176822512423645\\
15.8	0.00176819683992426\\
15.81	0.00176816854393782\\
15.82	0.00176814023627163\\
15.83	0.00176811191692021\\
15.84	0.00176808358587808\\
15.85	0.00176805524313971\\
15.86	0.00176802688869963\\
15.87	0.00176799852255234\\
15.88	0.00176797014469231\\
15.89	0.00176794175511406\\
15.9	0.00176791335381206\\
15.91	0.00176788494078081\\
15.92	0.00176785651601478\\
15.93	0.00176782807950846\\
15.94	0.00176779963125632\\
15.95	0.00176777117125284\\
15.96	0.00176774269949249\\
15.97	0.00176771421596974\\
15.98	0.00176768572067905\\
15.99	0.00176765721361488\\
16	0.0017676286947717\\
16.01	0.00176760016414396\\
16.02	0.00176757162172611\\
16.03	0.0017675430675126\\
16.04	0.00176751450149788\\
16.05	0.0017674859236764\\
16.06	0.00176745733404259\\
16.07	0.00176742873259089\\
16.08	0.00176740011931574\\
16.09	0.00176737149421157\\
16.1	0.0017673428572728\\
16.11	0.00176731420849387\\
16.12	0.0017672855478692\\
16.13	0.0017672568753932\\
16.14	0.00176722819106028\\
16.15	0.00176719949486488\\
16.16	0.0017671707868014\\
16.17	0.00176714206686423\\
16.18	0.00176711333504779\\
16.19	0.00176708459134649\\
16.2	0.00176705583575471\\
16.21	0.00176702706826685\\
16.22	0.0017669982888773\\
16.23	0.00176696949758046\\
16.24	0.00176694069437071\\
16.25	0.00176691187924243\\
16.26	0.00176688305219\\
16.27	0.0017668542132078\\
16.28	0.0017668253622902\\
16.29	0.00176679649943158\\
16.3	0.00176676762462629\\
16.31	0.00176673873786871\\
16.32	0.0017667098391532\\
16.33	0.00176668092847411\\
16.34	0.0017666520058258\\
16.35	0.00176662307120263\\
16.36	0.00176659412459893\\
16.37	0.00176656516600907\\
16.38	0.00176653619542737\\
16.39	0.00176650721284818\\
16.4	0.00176647821826585\\
16.41	0.00176644921167469\\
16.42	0.00176642019306904\\
16.43	0.00176639116244323\\
16.44	0.00176636211979158\\
16.45	0.00176633306510842\\
16.46	0.00176630399838805\\
16.47	0.00176627491962481\\
16.48	0.001766245828813\\
16.49	0.00176621672594692\\
16.5	0.00176618761102089\\
16.51	0.0017661584840292\\
16.52	0.00176612934496616\\
16.53	0.00176610019382607\\
16.54	0.0017660710306032\\
16.55	0.00176604185529187\\
16.56	0.00176601266788635\\
16.57	0.00176598346838093\\
16.58	0.00176595425676988\\
16.59	0.0017659250330475\\
16.6	0.00176589579720804\\
16.61	0.00176586654924578\\
16.62	0.001765837289155\\
16.63	0.00176580801692995\\
16.64	0.0017657787325649\\
16.65	0.00176574943605411\\
16.66	0.00176572012739183\\
16.67	0.00176569080657232\\
16.68	0.00176566147358982\\
16.69	0.00176563212843858\\
16.7	0.00176560277111285\\
16.71	0.00176557340160687\\
16.72	0.00176554401991487\\
16.73	0.00176551462603108\\
16.74	0.00176548521994975\\
16.75	0.00176545580166509\\
16.76	0.00176542637117133\\
16.77	0.00176539692846269\\
16.78	0.00176536747353339\\
16.79	0.00176533800637765\\
16.8	0.00176530852698968\\
16.81	0.00176527903536368\\
16.82	0.00176524953149386\\
16.83	0.00176522001537443\\
16.84	0.00176519048699958\\
16.85	0.00176516094636351\\
16.86	0.00176513139346041\\
16.87	0.00176510182828448\\
16.88	0.0017650722508299\\
16.89	0.00176504266109085\\
16.9	0.00176501305906151\\
16.91	0.00176498344473606\\
16.92	0.00176495381810868\\
16.93	0.00176492417917354\\
16.94	0.0017648945279248\\
16.95	0.00176486486435662\\
16.96	0.00176483518846318\\
16.97	0.00176480550023863\\
16.98	0.00176477579967711\\
16.99	0.00176474608677279\\
17	0.00176471636151982\\
17.01	0.00176468662391234\\
17.02	0.00176465687394449\\
17.03	0.0017646271116104\\
17.04	0.00176459733690423\\
17.05	0.00176456754982009\\
17.06	0.00176453775035213\\
17.07	0.00176450793849446\\
17.08	0.00176447811424121\\
17.09	0.0017644482775865\\
17.1	0.00176441842852444\\
17.11	0.00176438856704916\\
17.12	0.00176435869315475\\
17.13	0.00176432880683533\\
17.14	0.001764298908085\\
17.15	0.00176426899689787\\
17.16	0.00176423907326803\\
17.17	0.00176420913718957\\
17.18	0.00176417918865658\\
17.19	0.00176414922766317\\
17.2	0.0017641192542034\\
17.21	0.00176408926827135\\
17.22	0.00176405926986112\\
17.23	0.00176402925896678\\
17.24	0.00176399923558239\\
17.25	0.00176396919970202\\
17.26	0.00176393915131975\\
17.27	0.00176390909042963\\
17.28	0.00176387901702572\\
17.29	0.00176384893110208\\
17.3	0.00176381883265276\\
17.31	0.00176378872167181\\
17.32	0.00176375859815328\\
17.33	0.0017637284620912\\
17.34	0.00176369831347964\\
17.35	0.0017636681523126\\
17.36	0.00176363797858413\\
17.37	0.00176360779228827\\
17.38	0.00176357759341903\\
17.39	0.00176354738197044\\
17.4	0.00176351715793652\\
17.41	0.00176348692131129\\
17.42	0.00176345667208876\\
17.43	0.00176342641026295\\
17.44	0.00176339613582785\\
17.45	0.00176336584877749\\
17.46	0.00176333554910584\\
17.47	0.00176330523680692\\
17.48	0.00176327491187472\\
17.49	0.00176324457430323\\
17.5	0.00176321422408643\\
17.51	0.00176318386121832\\
17.52	0.00176315348569286\\
17.53	0.00176312309750406\\
17.54	0.00176309269664587\\
17.55	0.00176306228311227\\
17.56	0.00176303185689722\\
17.57	0.0017630014179947\\
17.58	0.00176297096639866\\
17.59	0.00176294050210307\\
17.6	0.00176291002510187\\
17.61	0.00176287953538902\\
17.62	0.00176284903295848\\
17.63	0.00176281851780418\\
17.64	0.00176278798992006\\
17.65	0.00176275744930008\\
17.66	0.00176272689593816\\
17.67	0.00176269632982823\\
17.68	0.00176266575096423\\
17.69	0.00176263515934008\\
17.7	0.0017626045549497\\
17.71	0.00176257393778701\\
17.72	0.00176254330784594\\
17.73	0.00176251266512039\\
17.74	0.00176248200960427\\
17.75	0.00176245134129148\\
17.76	0.00176242066017594\\
17.77	0.00176238996625154\\
17.78	0.00176235925951218\\
17.79	0.00176232853995175\\
17.8	0.00176229780756414\\
17.81	0.00176226706234323\\
17.82	0.00176223630428292\\
17.83	0.00176220553337708\\
17.84	0.00176217474961958\\
17.85	0.00176214395300431\\
17.86	0.00176211314352513\\
17.87	0.00176208232117591\\
17.88	0.0017620514859505\\
17.89	0.00176202063784278\\
17.9	0.0017619897768466\\
17.91	0.00176195890295581\\
17.92	0.00176192801616427\\
17.93	0.00176189711646581\\
17.94	0.00176186620385429\\
17.95	0.00176183527832354\\
17.96	0.00176180433986741\\
17.97	0.00176177338847972\\
17.98	0.0017617424241543\\
17.99	0.00176171144688499\\
18	0.00176168045666561\\
18.01	0.00176164945348996\\
18.02	0.00176161843735189\\
18.03	0.0017615874082452\\
18.04	0.00176155636616369\\
18.05	0.00176152531110118\\
18.06	0.00176149424305147\\
18.07	0.00176146316200836\\
18.08	0.00176143206796565\\
18.09	0.00176140096091713\\
18.1	0.00176136984085659\\
18.11	0.00176133870777782\\
18.12	0.00176130756167461\\
18.13	0.00176127640254073\\
18.14	0.00176124523036996\\
18.15	0.00176121404515608\\
18.16	0.00176118284689286\\
18.17	0.00176115163557406\\
18.18	0.00176112041119344\\
18.19	0.00176108917374478\\
18.2	0.00176105792322182\\
18.21	0.00176102665961832\\
18.22	0.00176099538292803\\
18.23	0.0017609640931447\\
18.24	0.00176093279026206\\
18.25	0.00176090147427388\\
18.26	0.00176087014517387\\
18.27	0.00176083880295577\\
18.28	0.00176080744761332\\
18.29	0.00176077607914023\\
18.3	0.00176074469753024\\
18.31	0.00176071330277707\\
18.32	0.00176068189487443\\
18.33	0.00176065047381603\\
18.34	0.00176061903959559\\
18.35	0.00176058759220681\\
18.36	0.0017605561316434\\
18.37	0.00176052465789905\\
18.38	0.00176049317096747\\
18.39	0.00176046167084235\\
18.4	0.00176043015751737\\
18.41	0.00176039863098623\\
18.42	0.00176036709124261\\
18.43	0.00176033553828018\\
18.44	0.00176030397209263\\
18.45	0.00176027239267362\\
18.46	0.00176024080001684\\
18.47	0.00176020919411594\\
18.48	0.00176017757496458\\
18.49	0.00176014594255644\\
18.5	0.00176011429688516\\
18.51	0.00176008263794439\\
18.52	0.00176005096572779\\
18.53	0.00176001928022901\\
18.54	0.00175998758144168\\
18.55	0.00175995586935944\\
18.56	0.00175992414397594\\
18.57	0.0017598924052848\\
18.58	0.00175986065327966\\
18.59	0.00175982888795414\\
18.6	0.00175979710930186\\
18.61	0.00175976531731644\\
18.62	0.0017597335119915\\
18.63	0.00175970169332064\\
18.64	0.00175966986129748\\
18.65	0.00175963801591564\\
18.66	0.0017596061571687\\
18.67	0.00175957428505026\\
18.68	0.00175954239955393\\
18.69	0.00175951050067329\\
18.7	0.00175947858840193\\
18.71	0.00175944666273344\\
18.72	0.00175941472366139\\
18.73	0.00175938277117938\\
18.74	0.00175935080528097\\
18.75	0.00175931882595973\\
18.76	0.00175928683320924\\
18.77	0.00175925482702305\\
18.78	0.00175922280739473\\
18.79	0.00175919077431783\\
18.8	0.00175915872778592\\
18.81	0.00175912666779254\\
18.82	0.00175909459433125\\
18.83	0.00175906250739557\\
18.84	0.00175903040697906\\
18.85	0.00175899829307525\\
18.86	0.00175896616567769\\
18.87	0.00175893402477988\\
18.88	0.00175890187037538\\
18.89	0.00175886970245769\\
18.9	0.00175883752102034\\
18.91	0.00175880532605685\\
18.92	0.00175877311756072\\
18.93	0.00175874089552548\\
18.94	0.00175870865994463\\
18.95	0.00175867641081166\\
18.96	0.00175864414812008\\
18.97	0.00175861187186339\\
18.98	0.00175857958203508\\
18.99	0.00175854727862863\\
19	0.00175851496163755\\
19.01	0.0017584826310553\\
19.02	0.00175845028687536\\
19.03	0.00175841792909122\\
19.04	0.00175838555769635\\
19.05	0.0017583531726842\\
19.06	0.00175832077404826\\
19.07	0.00175828836178198\\
19.08	0.00175825593587882\\
19.09	0.00175822349633223\\
19.1	0.00175819104313567\\
19.11	0.00175815857628258\\
19.12	0.00175812609576641\\
19.13	0.00175809360158061\\
19.14	0.00175806109371861\\
19.15	0.00175802857217384\\
19.16	0.00175799603693973\\
19.17	0.00175796348800972\\
19.18	0.00175793092537722\\
19.19	0.00175789834903566\\
19.2	0.00175786575897846\\
19.21	0.00175783315519902\\
19.22	0.00175780053769076\\
19.23	0.00175776790644709\\
19.24	0.0017577352614614\\
19.25	0.00175770260272711\\
19.26	0.0017576699302376\\
19.27	0.00175763724398627\\
19.28	0.00175760454396651\\
19.29	0.0017575718301717\\
19.3	0.00175753910259523\\
19.31	0.00175750636123048\\
19.32	0.00175747360607081\\
19.33	0.00175744083710961\\
19.34	0.00175740805434025\\
19.35	0.00175737525775608\\
19.36	0.00175734244735047\\
19.37	0.00175730962311678\\
19.38	0.00175727678504836\\
19.39	0.00175724393313857\\
19.4	0.00175721106738075\\
19.41	0.00175717818776825\\
19.42	0.0017571452942944\\
19.43	0.00175711238695255\\
19.44	0.00175707946573603\\
19.45	0.00175704653063817\\
19.46	0.0017570135816523\\
19.47	0.00175698061877174\\
19.48	0.00175694764198982\\
19.49	0.00175691465129984\\
19.5	0.00175688164669512\\
19.51	0.00175684862816897\\
19.52	0.0017568155957147\\
19.53	0.00175678254932561\\
19.54	0.001756749488995\\
19.55	0.00175671641471616\\
19.56	0.00175668332648239\\
19.57	0.00175665022428698\\
19.58	0.00175661710812321\\
19.59	0.00175658397798437\\
19.6	0.00175655083386372\\
19.61	0.00175651767575455\\
19.62	0.00175648450365013\\
19.63	0.00175645131754372\\
19.64	0.0017564181174286\\
19.65	0.00175638490329802\\
19.66	0.00175635167514523\\
19.67	0.0017563184329635\\
19.68	0.00175628517674607\\
19.69	0.00175625190648619\\
19.7	0.00175621862217711\\
19.71	0.00175618532381205\\
19.72	0.00175615201138427\\
19.73	0.00175611868488699\\
19.74	0.00175608534431344\\
19.75	0.00175605198965685\\
19.76	0.00175601862091044\\
19.77	0.00175598523806743\\
19.78	0.00175595184112103\\
19.79	0.00175591843006446\\
19.8	0.00175588500489093\\
19.81	0.00175585156559364\\
19.82	0.00175581811216578\\
19.83	0.00175578464460057\\
19.84	0.0017557511628912\\
19.85	0.00175571766703084\\
19.86	0.0017556841570127\\
19.87	0.00175565063282995\\
19.88	0.00175561709447578\\
19.89	0.00175558354194336\\
19.9	0.00175554997522587\\
19.91	0.00175551639431648\\
19.92	0.00175548279920834\\
19.93	0.00175544918989464\\
19.94	0.00175541556636852\\
19.95	0.00175538192862313\\
19.96	0.00175534827665164\\
19.97	0.0017553146104472\\
19.98	0.00175528093000293\\
19.99	0.001755247235312\\
20	0.00175521352636753\\
20.01	0.00175517980316267\\
20.02	0.00175514606569053\\
20.03	0.00175511231394426\\
20.04	0.00175507854791697\\
20.05	0.00175504476760179\\
20.06	0.00175501097299183\\
20.07	0.0017549771640802\\
20.08	0.00175494334086002\\
20.09	0.00175490950332439\\
20.1	0.00175487565146642\\
20.11	0.00175484178527919\\
20.12	0.00175480790475582\\
20.13	0.00175477400988939\\
20.14	0.00175474010067299\\
20.15	0.00175470617709971\\
20.16	0.00175467223916263\\
20.17	0.00175463828685483\\
20.18	0.00175460432016937\\
20.19	0.00175457033909934\\
20.2	0.0017545363436378\\
20.21	0.00175450233377782\\
20.22	0.00175446830951245\\
20.23	0.00175443427083475\\
20.24	0.00175440021773778\\
20.25	0.00175436615021459\\
20.26	0.00175433206825821\\
20.27	0.00175429797186171\\
20.28	0.00175426386101811\\
20.29	0.00175422973572045\\
20.3	0.00175419559596176\\
20.31	0.00175416144173508\\
20.32	0.00175412727303342\\
20.33	0.00175409308984981\\
20.34	0.00175405889217727\\
20.35	0.00175402468000881\\
20.36	0.00175399045333745\\
20.37	0.00175395621215618\\
20.38	0.00175392195645803\\
20.39	0.00175388768623597\\
20.4	0.00175385340148302\\
20.41	0.00175381910219217\\
20.42	0.0017537847883564\\
20.43	0.0017537504599687\\
20.44	0.00175371611702206\\
20.45	0.00175368175950946\\
20.46	0.00175364738742386\\
20.47	0.00175361300075825\\
20.48	0.00175357859950559\\
20.49	0.00175354418365884\\
20.5	0.00175350975321097\\
20.51	0.00175347530815494\\
20.52	0.00175344084848369\\
20.53	0.00175340637419019\\
20.54	0.00175337188526738\\
20.55	0.0017533373817082\\
20.56	0.0017533028635056\\
20.57	0.0017532683306525\\
20.58	0.00175323378314184\\
20.59	0.00175319922096656\\
20.6	0.00175316464411958\\
20.61	0.00175313005259382\\
20.62	0.0017530954463822\\
20.63	0.00175306082547763\\
20.64	0.00175302618987304\\
20.65	0.00175299153956131\\
20.66	0.00175295687453538\\
20.67	0.00175292219478811\\
20.68	0.00175288750031244\\
20.69	0.00175285279110123\\
20.7	0.00175281806714739\\
20.71	0.00175278332844381\\
20.72	0.00175274857498335\\
20.73	0.00175271380675891\\
20.74	0.00175267902376337\\
20.75	0.00175264422598959\\
20.76	0.00175260941343044\\
20.77	0.00175257458607879\\
20.78	0.0017525397439275\\
20.79	0.00175250488696944\\
20.8	0.00175247001519744\\
20.81	0.00175243512860438\\
20.82	0.00175240022718309\\
20.83	0.00175236531092642\\
20.84	0.00175233037982721\\
20.85	0.0017522954338783\\
20.86	0.00175226047307252\\
20.87	0.0017522254974027\\
20.88	0.00175219050686167\\
20.89	0.00175215550144225\\
20.9	0.00175212048113726\\
20.91	0.00175208544593952\\
20.92	0.00175205039584184\\
20.93	0.00175201533083703\\
20.94	0.00175198025091789\\
20.95	0.00175194515607722\\
20.96	0.00175191004630783\\
20.97	0.00175187492160251\\
20.98	0.00175183978195404\\
20.99	0.00175180462735523\\
21	0.00175176945779884\\
21.01	0.00175173427327766\\
21.02	0.00175169907378448\\
21.03	0.00175166385931205\\
21.04	0.00175162862985316\\
21.05	0.00175159338540056\\
21.06	0.00175155812594702\\
21.07	0.0017515228514853\\
21.08	0.00175148756200815\\
21.09	0.00175145225750834\\
21.1	0.00175141693797859\\
21.11	0.00175138160341167\\
21.12	0.0017513462538003\\
21.13	0.00175131088913723\\
21.14	0.0017512755094152\\
21.15	0.00175124011462693\\
21.16	0.00175120470476515\\
21.17	0.00175116927982258\\
21.18	0.00175113383979195\\
21.19	0.00175109838466597\\
21.2	0.00175106291443735\\
21.21	0.0017510274290988\\
21.22	0.00175099192864303\\
21.23	0.00175095641306274\\
21.24	0.00175092088235063\\
21.25	0.00175088533649938\\
21.26	0.00175084977550171\\
21.27	0.00175081419935029\\
21.28	0.0017507786080378\\
21.29	0.00175074300155693\\
21.3	0.00175070737990035\\
21.31	0.00175067174306075\\
21.32	0.00175063609103078\\
21.33	0.00175060042380311\\
21.34	0.00175056474137041\\
21.35	0.00175052904372534\\
21.36	0.00175049333086055\\
21.37	0.0017504576027687\\
21.38	0.00175042185944243\\
21.39	0.00175038610087439\\
21.4	0.00175035032705722\\
21.41	0.00175031453798355\\
21.42	0.00175027873364603\\
21.43	0.00175024291403728\\
21.44	0.00175020707914994\\
21.45	0.00175017122897662\\
21.46	0.00175013536350994\\
21.47	0.00175009948274252\\
21.48	0.00175006358666698\\
21.49	0.00175002767527592\\
21.5	0.00174999174856195\\
21.51	0.00174995580651767\\
21.52	0.00174991984913569\\
21.53	0.00174988387640859\\
21.54	0.00174984788832896\\
21.55	0.00174981188488941\\
21.56	0.0017497758660825\\
21.57	0.00174973983190083\\
21.58	0.00174970378233696\\
21.59	0.00174966771738347\\
21.6	0.00174963163703294\\
21.61	0.00174959554127793\\
21.62	0.001749559430111\\
21.63	0.00174952330352471\\
21.64	0.00174948716151162\\
21.65	0.00174945100406429\\
21.66	0.00174941483117525\\
21.67	0.00174937864283706\\
21.68	0.00174934243904225\\
21.69	0.00174930621978337\\
21.7	0.00174926998505295\\
21.71	0.00174923373484352\\
21.72	0.00174919746914761\\
21.73	0.00174916118795774\\
21.74	0.00174912489126644\\
21.75	0.00174908857906621\\
21.76	0.00174905225134958\\
21.77	0.00174901590810905\\
21.78	0.00174897954933712\\
21.79	0.00174894317502632\\
21.8	0.00174890678516911\\
21.81	0.00174887037975802\\
21.82	0.00174883395878552\\
21.83	0.00174879752224411\\
21.84	0.00174876107012628\\
21.85	0.00174872460242449\\
21.86	0.00174868811913124\\
21.87	0.00174865162023899\\
21.88	0.00174861510574021\\
21.89	0.00174857857562737\\
21.9	0.00174854202989294\\
21.91	0.00174850546852938\\
21.92	0.00174846889152913\\
21.93	0.00174843229888467\\
21.94	0.00174839569058842\\
21.95	0.00174835906663284\\
21.96	0.00174832242701038\\
21.97	0.00174828577171346\\
21.98	0.00174824910073453\\
21.99	0.00174821241406601\\
22	0.00174817571170034\\
22.01	0.00174813899362994\\
22.02	0.00174810225984722\\
22.03	0.00174806551034461\\
22.04	0.00174802874511451\\
22.05	0.00174799196414935\\
22.06	0.00174795516744153\\
22.07	0.00174791835498344\\
22.08	0.00174788152676749\\
22.09	0.00174784468278607\\
22.1	0.00174780782303158\\
22.11	0.00174777094749639\\
22.12	0.00174773405617291\\
22.13	0.00174769714905352\\
22.14	0.00174766022613058\\
22.15	0.00174762328739647\\
22.16	0.00174758633284357\\
22.17	0.00174754936246424\\
22.18	0.00174751237625084\\
22.19	0.00174747537419574\\
22.2	0.00174743835629128\\
22.21	0.00174740132252983\\
22.22	0.00174736427290374\\
22.23	0.00174732720740534\\
22.24	0.00174729012602699\\
22.25	0.00174725302876102\\
22.26	0.00174721591559976\\
22.27	0.00174717878653555\\
22.28	0.00174714164156071\\
22.29	0.00174710448066758\\
22.3	0.00174706730384846\\
22.31	0.00174703011109568\\
22.32	0.00174699290240155\\
22.33	0.00174695567775839\\
22.34	0.00174691843715848\\
22.35	0.00174688118059416\\
22.36	0.0017468439080577\\
22.37	0.00174680661954141\\
22.38	0.00174676931503758\\
22.39	0.00174673199453849\\
22.4	0.00174669465803643\\
22.41	0.0017466573055237\\
22.42	0.00174661993699254\\
22.43	0.00174658255243526\\
22.44	0.00174654515184411\\
22.45	0.00174650773521137\\
22.46	0.0017464703025293\\
22.47	0.00174643285379016\\
22.48	0.0017463953889862\\
22.49	0.00174635790810967\\
22.5	0.00174632041115283\\
22.51	0.00174628289810793\\
22.52	0.0017462453689672\\
22.53	0.00174620782372289\\
22.54	0.00174617026236722\\
22.55	0.00174613268489243\\
22.56	0.00174609509129076\\
22.57	0.00174605748155441\\
22.58	0.00174601985567562\\
22.59	0.00174598221364659\\
22.6	0.00174594455545955\\
22.61	0.00174590688110671\\
22.62	0.00174586919058026\\
22.63	0.00174583148387242\\
22.64	0.00174579376097537\\
22.65	0.00174575602188133\\
22.66	0.00174571826658247\\
22.67	0.001745680495071\\
22.68	0.00174564270733908\\
22.69	0.00174560490337891\\
22.7	0.00174556708318266\\
22.71	0.0017455292467425\\
22.72	0.00174549139405061\\
22.73	0.00174545352509916\\
22.74	0.00174541563988029\\
22.75	0.00174537773838618\\
22.76	0.00174533982060899\\
22.77	0.00174530188654086\\
22.78	0.00174526393617393\\
22.79	0.00174522596950037\\
22.8	0.0017451879865123\\
22.81	0.00174514998720188\\
22.82	0.00174511197156122\\
22.83	0.00174507393958247\\
22.84	0.00174503589125774\\
22.85	0.00174499782657916\\
22.86	0.00174495974553886\\
22.87	0.00174492164812894\\
22.88	0.00174488353434153\\
22.89	0.00174484540416872\\
22.9	0.00174480725760262\\
22.91	0.00174476909463535\\
22.92	0.00174473091525898\\
22.93	0.00174469271946563\\
22.94	0.00174465450724737\\
22.95	0.00174461627859629\\
22.96	0.00174457803350449\\
22.97	0.00174453977196404\\
22.98	0.00174450149396701\\
22.99	0.00174446319950548\\
23	0.00174442488857151\\
23.01	0.00174438656115718\\
23.02	0.00174434821725454\\
23.03	0.00174430985685564\\
23.04	0.00174427147995256\\
23.05	0.00174423308653732\\
23.06	0.00174419467660198\\
23.07	0.0017441562501386\\
23.08	0.00174411780713919\\
23.09	0.00174407934759581\\
23.1	0.00174404087150047\\
23.11	0.00174400237884521\\
23.12	0.00174396386962206\\
23.13	0.00174392534382303\\
23.14	0.00174388680144014\\
23.15	0.00174384824246541\\
23.16	0.00174380966689084\\
23.17	0.00174377107470844\\
23.18	0.00174373246591022\\
23.19	0.00174369384048817\\
23.2	0.00174365519843428\\
23.21	0.00174361653974055\\
23.22	0.00174357786439896\\
23.23	0.0017435391724015\\
23.24	0.00174350046374015\\
23.25	0.00174346173840688\\
23.26	0.00174342299639367\\
23.27	0.00174338423769248\\
23.28	0.00174334546229528\\
23.29	0.00174330667019403\\
23.3	0.00174326786138068\\
23.31	0.00174322903584719\\
23.32	0.00174319019358552\\
23.33	0.00174315133458759\\
23.34	0.00174311245884537\\
23.35	0.00174307356635078\\
23.36	0.00174303465709576\\
23.37	0.00174299573107223\\
23.38	0.00174295678827214\\
23.39	0.00174291782868739\\
23.4	0.0017428788523099\\
23.41	0.0017428398591316\\
23.42	0.00174280084914439\\
23.43	0.00174276182234019\\
23.44	0.00174272277871088\\
23.45	0.00174268371824839\\
23.46	0.00174264464094459\\
23.47	0.00174260554679137\\
23.48	0.00174256643578064\\
23.49	0.00174252730790427\\
23.5	0.00174248816315414\\
23.51	0.00174244900152213\\
23.52	0.0017424098230001\\
23.53	0.00174237062757994\\
23.54	0.00174233141525349\\
23.55	0.00174229218601262\\
23.56	0.0017422529398492\\
23.57	0.00174221367675506\\
23.58	0.00174217439672206\\
23.59	0.00174213509974204\\
23.6	0.00174209578580684\\
23.61	0.0017420564549083\\
23.62	0.00174201710703825\\
23.63	0.00174197774218851\\
23.64	0.00174193836035091\\
23.65	0.00174189896151728\\
23.66	0.00174185954567941\\
23.67	0.00174182011282914\\
23.68	0.00174178066295826\\
23.69	0.00174174119605858\\
23.7	0.00174170171212189\\
23.71	0.00174166221113999\\
23.72	0.00174162269310467\\
23.73	0.00174158315800772\\
23.74	0.00174154360584092\\
23.75	0.00174150403659605\\
23.76	0.00174146445026488\\
23.77	0.00174142484683918\\
23.78	0.00174138522631071\\
23.79	0.00174134558867124\\
23.8	0.00174130593391253\\
23.81	0.00174126626202631\\
23.82	0.00174122657300435\\
23.83	0.00174118686683839\\
23.84	0.00174114714352017\\
23.85	0.00174110740304141\\
23.86	0.00174106764539386\\
23.87	0.00174102787056923\\
23.88	0.00174098807855926\\
23.89	0.00174094826935564\\
23.9	0.0017409084429501\\
23.91	0.00174086859933436\\
23.92	0.00174082873850009\\
23.93	0.00174078886043902\\
23.94	0.00174074896514284\\
23.95	0.00174070905260322\\
23.96	0.00174066912281186\\
23.97	0.00174062917576044\\
23.98	0.00174058921144063\\
23.99	0.00174054922984411\\
24	0.00174050923096254\\
24.01	0.00174046921478758\\
24.02	0.0017404291813109\\
24.03	0.00174038913052414\\
24.04	0.00174034906241895\\
24.05	0.00174030897698697\\
24.06	0.00174026887421984\\
24.07	0.00174022875410919\\
24.08	0.00174018861664666\\
24.09	0.00174014846182385\\
24.1	0.00174010828963241\\
24.11	0.00174006810006392\\
24.12	0.00174002789311001\\
24.13	0.00173998766876228\\
24.14	0.00173994742701232\\
24.15	0.00173990716785172\\
24.16	0.00173986689127208\\
24.17	0.00173982659726498\\
24.18	0.00173978628582199\\
24.19	0.00173974595693468\\
24.2	0.00173970561059462\\
24.21	0.00173966524679338\\
24.22	0.00173962486552251\\
24.23	0.00173958446677355\\
24.24	0.00173954405053806\\
24.25	0.00173950361680758\\
24.26	0.00173946316557363\\
24.27	0.00173942269682775\\
24.28	0.00173938221056145\\
24.29	0.00173934170676626\\
24.3	0.00173930118543368\\
24.31	0.00173926064655523\\
24.32	0.00173922009012241\\
24.33	0.0017391795161267\\
24.34	0.00173913892455959\\
24.35	0.00173909831541257\\
24.36	0.00173905768867712\\
24.37	0.0017390170443447\\
24.38	0.00173897638240677\\
24.39	0.00173893570285481\\
24.4	0.00173889500568025\\
24.41	0.00173885429087455\\
24.42	0.00173881355842914\\
24.43	0.00173877280833546\\
24.44	0.00173873204058493\\
24.45	0.00173869125516897\\
24.46	0.001738650452079\\
24.47	0.00173860963130643\\
24.48	0.00173856879284266\\
24.49	0.00173852793667908\\
24.5	0.00173848706280707\\
24.51	0.00173844617121802\\
24.52	0.0017384052619033\\
24.53	0.00173836433485428\\
24.54	0.00173832339006232\\
24.55	0.00173828242751877\\
24.56	0.00173824144721498\\
24.57	0.00173820044914228\\
24.58	0.001738159433292\\
24.59	0.00173811839965548\\
24.6	0.00173807734822402\\
24.61	0.00173803627898894\\
24.62	0.00173799519194153\\
24.63	0.00173795408707309\\
24.64	0.0017379129643749\\
24.65	0.00173787182383825\\
24.66	0.00173783066545439\\
24.67	0.0017377894892146\\
24.68	0.00173774829511013\\
24.69	0.00173770708313223\\
24.7	0.00173766585327212\\
24.71	0.00173762460552105\\
24.72	0.00173758333987022\\
24.73	0.00173754205631086\\
24.74	0.00173750075483417\\
24.75	0.00173745943543134\\
24.76	0.00173741809809355\\
24.77	0.00173737674281199\\
24.78	0.00173733536957782\\
24.79	0.0017372939783822\\
24.8	0.00173725256921628\\
24.81	0.0017372111420712\\
24.82	0.00173716969693809\\
24.83	0.00173712823380807\\
24.84	0.00173708675267226\\
24.85	0.00173704525352175\\
24.86	0.00173700373634763\\
24.87	0.00173696220114099\\
24.88	0.00173692064789289\\
24.89	0.00173687907659441\\
24.9	0.00173683748723658\\
24.91	0.00173679587981044\\
24.92	0.00173675425430704\\
24.93	0.00173671261071737\\
24.94	0.00173667094903246\\
24.95	0.0017366292692433\\
24.96	0.00173658757134086\\
24.97	0.00173654585531612\\
24.98	0.00173650412116005\\
24.99	0.00173646236886359\\
25	0.00173642059841768\\
25.01	0.00173637880981325\\
25.02	0.0017363370030412\\
25.03	0.00173629517809244\\
25.04	0.00173625333495786\\
25.05	0.00173621147362833\\
25.06	0.00173616959409471\\
25.07	0.00173612769634785\\
25.08	0.00173608578037859\\
25.09	0.00173604384617774\\
25.1	0.00173600189373613\\
25.11	0.00173595992304453\\
25.12	0.00173591793409373\\
25.13	0.0017358759268745\\
25.14	0.00173583390137758\\
25.15	0.00173579185759371\\
25.16	0.00173574979551362\\
25.17	0.00173570771512801\\
25.18	0.00173566561642757\\
25.19	0.00173562349940297\\
25.2	0.00173558136404488\\
25.21	0.00173553921034394\\
25.22	0.00173549703829076\\
25.23	0.00173545484787597\\
25.24	0.00173541263909016\\
25.25	0.0017353704119239\\
25.26	0.00173532816636775\\
25.27	0.00173528590241225\\
25.28	0.00173524362004792\\
25.29	0.00173520131926527\\
25.3	0.00173515900005478\\
25.31	0.00173511666240692\\
25.32	0.00173507430631214\\
25.33	0.00173503193176088\\
25.34	0.00173498953874352\\
25.35	0.00173494712725048\\
25.36	0.0017349046972721\\
25.37	0.00173486224879876\\
25.38	0.00173481978182076\\
25.39	0.00173477729632841\\
25.4	0.001734734792312\\
25.41	0.0017346922697618\\
25.42	0.00173464972866804\\
25.43	0.00173460716902094\\
25.44	0.00173456459081069\\
25.45	0.00173452199402746\\
25.46	0.0017344793786614\\
25.47	0.00173443674470264\\
25.48	0.00173439409214126\\
25.49	0.00173435142096735\\
25.5	0.00173430873117094\\
25.51	0.00173426602274207\\
25.52	0.00173422329567073\\
25.53	0.00173418054994687\\
25.54	0.00173413778556046\\
25.55	0.00173409500250139\\
25.56	0.00173405220075957\\
25.57	0.00173400938032483\\
25.58	0.00173396654118701\\
25.59	0.00173392368333591\\
25.6	0.0017338808067613\\
25.61	0.00173383791145291\\
25.62	0.00173379499740045\\
25.63	0.00173375206459359\\
25.64	0.00173370911302199\\
25.65	0.00173366614267524\\
25.66	0.00173362315354293\\
25.67	0.00173358014561459\\
25.68	0.00173353711887974\\
25.69	0.00173349407332786\\
25.7	0.00173345100894837\\
25.71	0.00173340792573068\\
25.72	0.00173336482366414\\
25.73	0.00173332170273811\\
25.74	0.00173327856294185\\
25.75	0.00173323540426463\\
25.76	0.00173319222669565\\
25.77	0.00173314903022408\\
25.78	0.00173310581483905\\
25.79	0.00173306258052965\\
25.8	0.00173301932728492\\
25.81	0.00173297605509388\\
25.82	0.00173293276394547\\
25.83	0.00173288945382861\\
25.84	0.00173284612473217\\
25.85	0.00173280277664496\\
25.86	0.00173275940955578\\
25.87	0.00173271602345334\\
25.88	0.00173267261832632\\
25.89	0.00173262919416335\\
25.9	0.00173258575095301\\
25.91	0.00173254228868383\\
25.92	0.00173249880734428\\
25.93	0.00173245530692277\\
25.94	0.00173241178740769\\
25.95	0.00173236824878734\\
25.96	0.00173232469104997\\
25.97	0.00173228111418379\\
25.98	0.00173223751817693\\
25.99	0.00173219390301747\\
26	0.00173215026869344\\
26.01	0.00173210661519279\\
26.02	0.00173206294250342\\
26.03	0.00173201925061316\\
26.04	0.00173197553950977\\
26.05	0.00173193180918097\\
26.06	0.00173188805961438\\
26.07	0.00173184429079756\\
26.08	0.00173180050271802\\
26.09	0.00173175669536317\\
26.1	0.00173171286872038\\
26.11	0.0017316690227769\\
26.12	0.00173162515751996\\
26.13	0.00173158127293667\\
26.14	0.00173153736901409\\
26.15	0.00173149344573919\\
26.16	0.00173144950309883\\
26.17	0.00173140554107985\\
26.18	0.00173136155966896\\
26.19	0.0017313175588528\\
26.2	0.0017312735386179\\
26.21	0.00173122949895075\\
26.22	0.0017311854398377\\
26.23	0.00173114136126505\\
26.24	0.00173109726321896\\
26.25	0.00173105314568554\\
26.26	0.00173100900865077\\
26.27	0.00173096485210056\\
26.28	0.00173092067602069\\
26.29	0.00173087648039684\\
26.3	0.00173083226521463\\
26.31	0.00173078803045951\\
26.32	0.00173074377611686\\
26.33	0.00173069950217195\\
26.34	0.00173065520860991\\
26.35	0.00173061089541579\\
26.36	0.0017305665625745\\
26.37	0.00173052221007083\\
26.38	0.00173047783788947\\
26.39	0.00173043344601496\\
26.4	0.00173038903443173\\
26.41	0.00173034460312409\\
26.42	0.00173030015207619\\
26.43	0.00173025568127208\\
26.44	0.00173021119069564\\
26.45	0.00173016668033064\\
26.46	0.00173012215016071\\
26.47	0.00173007760016931\\
26.48	0.00173003303033979\\
26.49	0.0017299884406553\\
26.5	0.0017299438310989\\
26.51	0.00172989920165345\\
26.52	0.00172985455230167\\
26.53	0.00172980988302612\\
26.54	0.00172976519380918\\
26.55	0.00172972048463309\\
26.56	0.00172967575547991\\
26.57	0.00172963100633151\\
26.58	0.00172958623716961\\
26.59	0.00172954144797572\\
26.6	0.0017294966387312\\
26.61	0.00172945180941721\\
26.62	0.00172940696001471\\
26.63	0.00172936209050448\\
26.64	0.00172931720086709\\
26.65	0.00172927229108292\\
26.66	0.00172922736113214\\
26.67	0.00172918241099471\\
26.68	0.00172913744065039\\
26.69	0.00172909245007871\\
26.7	0.00172904743925897\\
26.71	0.00172900240817027\\
26.72	0.00172895735679146\\
26.73	0.00172891228510117\\
26.74	0.00172886719307777\\
26.75	0.00172882208069942\\
26.76	0.00172877694794402\\
26.77	0.00172873179478919\\
26.78	0.00172868662121234\\
26.79	0.00172864142719059\\
26.8	0.00172859621270078\\
26.81	0.00172855097771952\\
26.82	0.00172850572222311\\
26.83	0.00172846044618758\\
26.84	0.00172841514958867\\
26.85	0.00172836983240182\\
26.86	0.00172832449460219\\
26.87	0.00172827913616462\\
26.88	0.00172823375706364\\
26.89	0.00172818835727347\\
26.9	0.00172814293676802\\
26.91	0.00172809749552084\\
26.92	0.00172805203350518\\
26.93	0.00172800655069392\\
26.94	0.00172796104705962\\
26.95	0.00172791552257447\\
26.96	0.0017278699772103\\
26.97	0.00172782441093858\\
26.98	0.0017277788237304\\
26.99	0.00172773321555647\\
27	0.00172768758638711\\
27.01	0.00172764193619225\\
27.02	0.00172759626494143\\
27.03	0.00172755057260373\\
27.04	0.00172750485914788\\
27.05	0.00172745912454212\\
27.06	0.0017274133687543\\
27.07	0.0017273675917518\\
27.08	0.00172732179350156\\
27.09	0.00172727597397007\\
27.1	0.00172723013312334\\
27.11	0.00172718427092689\\
27.12	0.00172713838734578\\
27.13	0.00172709248234457\\
27.14	0.00172704655588729\\
27.15	0.00172700060793749\\
27.16	0.00172695463845818\\
27.17	0.00172690864741182\\
27.18	0.00172686263476037\\
27.19	0.00172681660046519\\
27.2	0.00172677054448711\\
27.21	0.00172672446678635\\
27.22	0.00172667836732259\\
27.23	0.00172663224605488\\
27.24	0.00172658610294167\\
27.25	0.0017265399379408\\
27.26	0.00172649375100946\\
27.27	0.00172644754210423\\
27.28	0.00172640131118101\\
27.29	0.00172635505819503\\
27.3	0.00172630878310087\\
27.31	0.00172626248585239\\
27.32	0.00172621616640276\\
27.33	0.00172616982470443\\
27.34	0.00172612346070913\\
27.35	0.00172607707436783\\
27.36	0.00172603066563075\\
27.37	0.00172598423444734\\
27.38	0.00172593778076627\\
27.39	0.0017258913045354\\
27.4	0.00172584480570178\\
27.41	0.00172579828421162\\
27.42	0.00172575174001031\\
27.43	0.00172570517304237\\
27.44	0.00172565858325143\\
27.45	0.00172561197058024\\
27.46	0.00172556533497065\\
27.47	0.00172551867636356\\
27.48	0.00172547199469897\\
27.49	0.00172542528991589\\
27.5	0.00172537856195236\\
27.51	0.00172533181074544\\
27.52	0.00172528503623116\\
27.53	0.00172523823834455\\
27.54	0.00172519141701956\\
27.55	0.0017251445721891\\
27.56	0.00172509770378498\\
27.57	0.00172505081173792\\
27.58	0.00172500389597751\\
27.59	0.00172495695643219\\
27.6	0.00172490999302924\\
27.61	0.00172486300569477\\
27.62	0.00172481599435366\\
27.63	0.0017247689589296\\
27.64	0.00172472189934499\\
27.65	0.00172467481552099\\
27.66	0.00172462770737747\\
27.67	0.00172458057483296\\
27.68	0.0017245334178047\\
27.69	0.00172448623620852\\
27.7	0.0017244390299589\\
27.71	0.0017243917989689\\
27.72	0.00172434454315017\\
27.73	0.00172429726241289\\
27.74	0.00172424995666575\\
27.75	0.00172420262581595\\
27.76	0.00172415526976918\\
27.77	0.00172410788842954\\
27.78	0.00172406048169957\\
27.79	0.0017240130494802\\
27.8	0.00172396559167073\\
27.81	0.00172391810816878\\
27.82	0.00172387059887032\\
27.83	0.00172382306366956\\
27.84	0.00172377550245901\\
27.85	0.00172372791512938\\
27.86	0.0017236803015696\\
27.87	0.00172363266166674\\
27.88	0.00172358499530606\\
27.89	0.00172353730237089\\
27.9	0.00172348958274267\\
27.91	0.00172344183630087\\
27.92	0.00172339406292302\\
27.93	0.0017233462624846\\
27.94	0.00172329843485908\\
27.95	0.00172325057991785\\
27.96	0.00172320269753021\\
27.97	0.0017231547875633\\
27.98	0.00172310684988212\\
27.99	0.00172305888434948\\
28	0.00172301089082593\\
28.01	0.00172296286916978\\
28.02	0.00172291481923705\\
28.03	0.00172286674088143\\
28.04	0.00172281863395422\\
28.05	0.00172277049830437\\
28.06	0.00172272233377838\\
28.07	0.00172267414022028\\
28.08	0.00172262591747164\\
28.09	0.00172257766537146\\
28.1	0.00172252938375621\\
28.11	0.00172248107245973\\
28.12	0.00172243273131327\\
28.13	0.00172238436014539\\
28.14	0.00172233595878196\\
28.15	0.00172228752704611\\
28.16	0.00172223906475821\\
28.17	0.00172219057173584\\
28.18	0.00172214204779374\\
28.19	0.00172209349274377\\
28.2	0.0017220449063949\\
28.21	0.00172199628855317\\
28.22	0.00172194763902167\\
28.23	0.00172189895760045\\
28.24	0.00172185024408657\\
28.25	0.00172180149827401\\
28.26	0.00172175271995366\\
28.27	0.00172170390891328\\
28.28	0.00172165506493749\\
28.29	0.00172160618780772\\
28.3	0.00172155727730218\\
28.31	0.00172150833319585\\
28.32	0.00172145935526044\\
28.33	0.00172141034326436\\
28.34	0.00172136129697271\\
28.35	0.00172131221614722\\
28.36	0.00172126310054628\\
28.37	0.00172121394992488\\
28.38	0.00172116476403459\\
28.39	0.00172111554262355\\
28.4	0.00172106628543646\\
28.41	0.00172101699221452\\
28.42	0.00172096766269549\\
28.43	0.00172091829661361\\
28.44	0.00172086889369959\\
28.45	0.00172081945368065\\
28.46	0.00172076997628048\\
28.47	0.00172072046121921\\
28.48	0.00172067090821345\\
28.49	0.00172062131697627\\
28.5	0.0017205716872172\\
28.51	0.00172052201864223\\
28.52	0.00172047231095382\\
28.53	0.00172042256385092\\
28.54	0.00172037277702898\\
28.55	0.00172032295017993\\
28.56	0.00172027308299226\\
28.57	0.001720223175151\\
28.58	0.00172017322633777\\
28.59	0.00172012323623077\\
28.6	0.00172007320450489\\
28.61	0.00172002313083168\\
28.62	0.00171997301487942\\
28.63	0.00171992285631318\\
28.64	0.00171987265479488\\
28.65	0.00171982240998329\\
28.66	0.0017197721215342\\
28.67	0.00171972178910039\\
28.68	0.00171967141233177\\
28.69	0.00171962099087545\\
28.7	0.00171957052437583\\
28.71	0.00171952001247471\\
28.72	0.00171946945481134\\
28.73	0.00171941885102264\\
28.74	0.00171936820074324\\
28.75	0.00171931750360562\\
28.76	0.00171926675924029\\
28.77	0.00171921596727591\\
28.78	0.00171916512733945\\
28.79	0.0017191142390564\\
28.8	0.00171906330205088\\
28.81	0.00171901231594589\\
28.82	0.00171896128036352\\
28.83	0.00171891019492511\\
28.84	0.00171885905925156\\
28.85	0.00171880787296349\\
28.86	0.00171875663568155\\
28.87	0.00171870534702671\\
28.88	0.00171865400662049\\
28.89	0.0017186026140853\\
28.9	0.00171855116904476\\
28.91	0.00171849967112404\\
28.92	0.0017184481199502\\
28.93	0.00171839651515257\\
28.94	0.00171834485636318\\
28.95	0.0017182931432171\\
28.96	0.00171824137535297\\
28.97	0.00171818955241339\\
28.98	0.00171813767404543\\
28.99	0.00171808573990114\\
29	0.00171803374963807\\
29.01	0.00171798170291983\\
29.02	0.0017179295994167\\
29.03	0.00171787743880621\\
29.04	0.00171782522077382\\
29.05	0.00171777294501356\\
29.06	0.00171772061122875\\
29.07	0.00171766821913277\\
29.08	0.00171761576844979\\
29.09	0.00171756325891561\\
29.1	0.00171751069027852\\
29.11	0.00171745806230017\\
29.12	0.00171740537475648\\
29.13	0.00171735262743866\\
29.14	0.00171729982015422\\
29.15	0.00171724695272799\\
29.16	0.00171719402500327\\
29.17	0.00171714103684299\\
29.18	0.0017170879881309\\
29.19	0.00171703487877286\\
29.2	0.00171698170869816\\
29.21	0.00171692847786087\\
29.22	0.00171687518624132\\
29.23	0.00171682183384758\\
29.24	0.00171676842071705\\
29.25	0.00171671494691806\\
29.26	0.00171666141255162\\
29.27	0.00171660781775318\\
29.28	0.00171655416269448\\
29.29	0.00171650044758548\\
29.3	0.00171644667267643\\
29.31	0.00171639283825989\\
29.32	0.00171633894467299\\
29.33	0.00171628499229969\\
29.34	0.00171623098157314\\
29.35	0.00171617691297821\\
29.36	0.00171612278705401\\
29.37	0.00171606860439661\\
29.38	0.00171601436566184\\
29.39	0.00171596007156823\\
29.4	0.00171590572289997\\
29.41	0.00171585132051016\\
29.42	0.00171579686532404\\
29.43	0.00171574235834243\\
29.44	0.00171568780064529\\
29.45	0.00171563319339544\\
29.46	0.0017155785378424\\
29.47	0.00171552383532642\\
29.48	0.00171546908728261\\
29.49	0.00171541429524537\\
29.5	0.00171535946085281\\
29.51	0.0017153045858515\\
29.52	0.00171524967210135\\
29.53	0.00171519472158067\\
29.54	0.00171513973639147\\
29.55	0.00171508471876497\\
29.56	0.00171502966906673\\
29.57	0.00171497458735218\\
29.58	0.00171491947368483\\
29.59	0.00171486432813668\\
29.6	0.00171480915078852\\
29.61	0.00171475394173037\\
29.62	0.00171469870106184\\
29.63	0.00171464342889253\\
29.64	0.0017145881253425\\
29.65	0.00171453279054261\\
29.66	0.00171447742463508\\
29.67	0.00171442202777388\\
29.68	0.00171436660012524\\
29.69	0.00171431114186816\\
29.7	0.0017142556531949\\
29.71	0.00171420013431153\\
29.72	0.00171414458543852\\
29.73	0.00171408900681123\\
29.74	0.00171403339868058\\
29.75	0.00171397776131365\\
29.76	0.00171392209499428\\
29.77	0.00171386640002376\\
29.78	0.00171381067672151\\
29.79	0.00171375492542576\\
29.8	0.00171369914649432\\
29.81	0.00171364334030528\\
29.82	0.00171358750725785\\
29.83	0.00171353164777311\\
29.84	0.00171347576229487\\
29.85	0.00171341985129056\\
29.86	0.00171336391525205\\
29.87	0.00171330795469666\\
29.88	0.00171325197016804\\
29.89	0.00171319596223719\\
29.9	0.00171313993150349\\
29.91	0.00171308387859574\\
29.92	0.00171302780417323\\
29.93	0.00171297170892692\\
29.94	0.00171291559358056\\
29.95	0.00171285945889192\\
29.96	0.00171280330565399\\
29.97	0.00171274713469636\\
29.98	0.00171269094688644\\
29.99	0.00171263474313094\\
30	0.00171257852437719\\
30.01	0.00171252229161471\\
30.02	0.00171246604587666\\
30.03	0.00171240978824142\\
30.04	0.00171235351983425\\
30.05	0.00171229724182891\\
30.06	0.00171224095544946\\
30.07	0.00171218466197197\\
30.08	0.00171212836272645\\
30.09	0.0017120720590987\\
30.1	0.00171201575253234\\
30.11	0.00171195944453079\\
30.12	0.00171190313665943\\
30.13	0.00171184683054776\\
30.14	0.00171179052789165\\
30.15	0.00171173423045565\\
30.16	0.00171167794007543\\
30.17	0.00171162165866023\\
30.18	0.00171156538819543\\
30.19	0.00171150913074522\\
30.2	0.00171145288845531\\
30.21	0.00171139666355577\\
30.22	0.00171134045836393\\
30.23	0.00171128427528744\\
30.24	0.00171122811682731\\
30.25	0.00171117198558123\\
30.26	0.00171111588424678\\
30.27	0.00171105981562495\\
30.28	0.00171100378262363\\
30.29	0.00171094778826131\\
30.3	0.00171089183567084\\
30.31	0.00171083592810332\\
30.32	0.00171078006893214\\
30.33	0.00171072426165718\\
30.34	0.00171066850990906\\
30.35	0.0017106128174536\\
30.36	0.00171055718819641\\
30.37	0.00171050162618762\\
30.38	0.00171044613562675\\
30.39	0.0017103907208678\\
30.4	0.00171033538642442\\
30.41	0.00171028013697533\\
30.42	0.00171022497736982\\
30.43	0.00171016991263357\\
30.44	0.00171011494797449\\
30.45	0.00171006008878889\\
30.46	0.00171000534066778\\
30.47	0.0017099507094034\\
30.48	0.0017098962009959\\
30.49	0.00170984182166039\\
30.5	0.00170978757783399\\
30.51	0.00170973347618334\\
30.52	0.00170967952361221\\
30.53	0.00170962572726937\\
30.54	0.00170957209455677\\
30.55	0.00170951863313793\\
30.56	0.00170946535094666\\
30.57	0.00170941225619598\\
30.58	0.00170935935738741\\
30.59	0.00170930666332049\\
30.6	0.00170925418310267\\
30.61	0.00170920192615947\\
30.62	0.00170914990224501\\
30.63	0.00170909799545791\\
30.64	0.00170904606874648\\
30.65	0.00170899412210136\\
30.66	0.00170894215551316\\
30.67	0.00170889016897251\\
30.68	0.00170883816247\\
30.69	0.00170878613599623\\
30.7	0.0017087340895418\\
30.71	0.0017086820230973\\
30.72	0.0017086299366533\\
30.73	0.00170857783020038\\
30.74	0.00170852570372912\\
30.75	0.00170847355723006\\
30.76	0.00170842139069376\\
30.77	0.00170836920411078\\
30.78	0.00170831699747166\\
30.79	0.00170826477076693\\
30.8	0.00170821252398713\\
30.81	0.00170816025712277\\
30.82	0.00170810797016437\\
30.83	0.00170805566310245\\
30.84	0.00170800333592751\\
30.85	0.00170795098863004\\
30.86	0.00170789862120055\\
30.87	0.0017078462336295\\
30.88	0.00170779382590738\\
30.89	0.00170774139802467\\
30.9	0.00170768894997182\\
30.91	0.0017076364817393\\
30.92	0.00170758399331757\\
30.93	0.00170753148469706\\
30.94	0.00170747895586821\\
30.95	0.00170742640682146\\
30.96	0.00170737383754724\\
30.97	0.00170732124803596\\
30.98	0.00170726863827804\\
30.99	0.00170721600826389\\
31	0.00170716335798391\\
31.01	0.00170711068742848\\
31.02	0.00170705799658801\\
31.03	0.00170700528545286\\
31.04	0.00170695255401342\\
31.05	0.00170689980226004\\
31.06	0.0017068470301831\\
31.07	0.00170679423777295\\
31.08	0.00170674142501993\\
31.09	0.00170668859191439\\
31.1	0.00170663573844666\\
31.11	0.00170658286460708\\
31.12	0.00170652997038595\\
31.13	0.0017064770557736\\
31.14	0.00170642412076034\\
31.15	0.00170637116533646\\
31.16	0.00170631818949227\\
31.17	0.00170626519321805\\
31.18	0.00170621217650409\\
31.19	0.00170615913934065\\
31.2	0.00170610608171801\\
31.21	0.00170605300362644\\
31.22	0.00170599990505617\\
31.23	0.00170594678599748\\
31.24	0.00170589364644058\\
31.25	0.00170584048637573\\
31.26	0.00170578730579314\\
31.27	0.00170573410468306\\
31.28	0.00170568088303568\\
31.29	0.00170562764084121\\
31.3	0.00170557437808986\\
31.31	0.00170552109477183\\
31.32	0.00170546779087729\\
31.33	0.00170541446639644\\
31.34	0.00170536112131945\\
31.35	0.00170530775563647\\
31.36	0.0017052543693377\\
31.37	0.00170520096241326\\
31.38	0.0017051475348533\\
31.39	0.00170509408664798\\
31.4	0.00170504061778743\\
31.41	0.00170498712826177\\
31.42	0.00170493361806111\\
31.43	0.00170488008717559\\
31.44	0.0017048265355953\\
31.45	0.00170477296331034\\
31.46	0.00170471937031081\\
31.47	0.0017046657565868\\
31.48	0.00170461212212838\\
31.49	0.00170455846692563\\
31.5	0.0017045047909686\\
31.51	0.00170445109424737\\
31.52	0.00170439737675198\\
31.53	0.00170434363847248\\
31.54	0.00170428987939891\\
31.55	0.0017042360995213\\
31.56	0.00170418229882967\\
31.57	0.00170412847731404\\
31.58	0.00170407463496443\\
31.59	0.00170402077177083\\
31.6	0.00170396688772325\\
31.61	0.00170391298281167\\
31.62	0.00170385905702608\\
31.63	0.00170380511035645\\
31.64	0.00170375114279275\\
31.65	0.00170369715432495\\
31.66	0.00170364314494299\\
31.67	0.00170358911463684\\
31.68	0.00170353506339642\\
31.69	0.00170348099121168\\
31.7	0.00170342689807254\\
31.71	0.00170337278396893\\
31.72	0.00170331864889075\\
31.73	0.00170326449282791\\
31.74	0.00170321031577031\\
31.75	0.00170315611770784\\
31.76	0.0017031018986304\\
31.77	0.00170304765852785\\
31.78	0.00170299339739007\\
31.79	0.00170293911520692\\
31.8	0.00170288481196826\\
31.81	0.00170283048766394\\
31.82	0.0017027761422838\\
31.83	0.00170272177581769\\
31.84	0.00170266738825541\\
31.85	0.00170261297958681\\
31.86	0.00170255854980169\\
31.87	0.00170250409888987\\
31.88	0.00170244962684113\\
31.89	0.00170239513364527\\
31.9	0.00170234061929209\\
31.91	0.00170228608377135\\
31.92	0.00170223152707284\\
31.93	0.00170217694918631\\
31.94	0.00170212235010152\\
31.95	0.00170206772980823\\
31.96	0.00170201308829618\\
31.97	0.00170195842555509\\
31.98	0.00170190374157471\\
31.99	0.00170184903634474\\
32	0.00170179430985492\\
32.01	0.00170173956209494\\
32.02	0.0017016847930545\\
32.03	0.00170163000272329\\
32.04	0.00170157519109101\\
32.05	0.00170152035814733\\
32.06	0.00170146550388192\\
32.07	0.00170141062828444\\
32.08	0.00170135573134455\\
32.09	0.0017013008130519\\
32.1	0.00170124587339613\\
32.11	0.00170119091236687\\
32.12	0.00170113592995376\\
32.13	0.00170108092614641\\
32.14	0.00170102590093443\\
32.15	0.00170097085430744\\
32.16	0.00170091578625502\\
32.17	0.00170086069676676\\
32.18	0.00170080558583227\\
32.19	0.0017007504534411\\
32.2	0.00170069529958282\\
32.21	0.001700640124247\\
32.22	0.0017005849274232\\
32.23	0.00170052970910094\\
32.24	0.00170047446926979\\
32.25	0.00170041920791926\\
32.26	0.00170036392503889\\
32.27	0.00170030862061818\\
32.28	0.00170025329464666\\
32.29	0.00170019794711381\\
32.3	0.00170014257800914\\
32.31	0.00170008718732213\\
32.32	0.00170003177504226\\
32.33	0.001699976341159\\
32.34	0.00169992088566183\\
32.35	0.00169986540854018\\
32.36	0.00169980990978353\\
32.37	0.0016997543893813\\
32.38	0.00169969884732293\\
32.39	0.00169964328359786\\
32.4	0.00169958769819549\\
32.41	0.00169953209110524\\
32.42	0.00169947646231652\\
32.43	0.00169942081181873\\
32.44	0.00169936513960124\\
32.45	0.00169930944565345\\
32.46	0.00169925372996474\\
32.47	0.00169919799252445\\
32.48	0.00169914223332197\\
32.49	0.00169908645234663\\
32.5	0.00169903064958779\\
32.51	0.00169897482503477\\
32.52	0.00169891897867692\\
32.53	0.00169886311050354\\
32.54	0.00169880722050395\\
32.55	0.00169875130866746\\
32.56	0.00169869537498337\\
32.57	0.00169863941944097\\
32.58	0.00169858344202954\\
32.59	0.00169852744273836\\
32.6	0.00169847142155669\\
32.61	0.0016984153784738\\
32.62	0.00169835931347894\\
32.63	0.00169830322656135\\
32.64	0.00169824711771028\\
32.65	0.00169819098691494\\
32.66	0.00169813483416456\\
32.67	0.00169807865944836\\
32.68	0.00169802246275554\\
32.69	0.0016979662440753\\
32.7	0.00169791000339684\\
32.71	0.00169785374070932\\
32.72	0.00169779745600194\\
32.73	0.00169774114926385\\
32.74	0.00169768482048423\\
32.75	0.0016976284696522\\
32.76	0.00169757209675694\\
32.77	0.00169751570178756\\
32.78	0.0016974592847332\\
32.79	0.00169740284558298\\
32.8	0.00169734638432601\\
32.81	0.0016972899009514\\
32.82	0.00169723339544825\\
32.83	0.00169717686780564\\
32.84	0.00169712031801265\\
32.85	0.00169706374605836\\
32.86	0.00169700715193184\\
32.87	0.00169695053562214\\
32.88	0.00169689389711831\\
32.89	0.0016968372364094\\
32.9	0.00169678055348443\\
32.91	0.00169672384833244\\
32.92	0.00169666712094245\\
32.93	0.00169661037130346\\
32.94	0.00169655359940448\\
32.95	0.00169649680523449\\
32.96	0.0016964399887825\\
32.97	0.00169638315003747\\
32.98	0.00169632628898839\\
32.99	0.0016962694056242\\
33	0.00169621249993386\\
33.01	0.00169615557190633\\
33.02	0.00169609862153053\\
33.03	0.00169604164879541\\
33.04	0.00169598465368988\\
33.05	0.00169592763620286\\
33.06	0.00169587059632325\\
33.07	0.00169581353403996\\
33.08	0.00169575644934187\\
33.09	0.00169569934221787\\
33.1	0.00169564221265682\\
33.11	0.00169558506064761\\
33.12	0.00169552788617908\\
33.13	0.00169547068924008\\
33.14	0.00169541346981946\\
33.15	0.00169535622790605\\
33.16	0.00169529896348868\\
33.17	0.00169524167655616\\
33.18	0.00169518436709731\\
33.19	0.00169512703510092\\
33.2	0.00169506968055579\\
33.21	0.0016950123034507\\
33.22	0.00169495490377443\\
33.23	0.00169489748151575\\
33.24	0.00169484003666342\\
33.25	0.00169478256920619\\
33.26	0.00169472507913281\\
33.27	0.001694667566432\\
33.28	0.00169461003109251\\
33.29	0.00169455247310304\\
33.3	0.00169449489245231\\
33.31	0.00169443728912902\\
33.32	0.00169437966312187\\
33.33	0.00169432201441955\\
33.34	0.00169426434301072\\
33.35	0.00169420664888406\\
33.36	0.00169414893202823\\
33.37	0.00169409119243189\\
33.38	0.00169403343008368\\
33.39	0.00169397564497223\\
33.4	0.00169391783708618\\
33.41	0.00169386000641414\\
33.42	0.00169380215294473\\
33.43	0.00169374427666654\\
33.44	0.00169368637756818\\
33.45	0.00169362845563824\\
33.46	0.00169357051086527\\
33.47	0.00169351254323787\\
33.48	0.00169345455274458\\
33.49	0.00169339653937397\\
33.5	0.00169333850311456\\
33.51	0.00169328044395492\\
33.52	0.00169322236188355\\
33.53	0.00169316425688898\\
33.54	0.00169310612895971\\
33.55	0.00169304797808426\\
33.56	0.00169298980425111\\
33.57	0.00169293160744875\\
33.58	0.00169287338766566\\
33.59	0.0016928151448903\\
33.6	0.00169275687911114\\
33.61	0.00169269859031662\\
33.62	0.00169264027849519\\
33.63	0.00169258194363528\\
33.64	0.00169252358572532\\
33.65	0.00169246520475373\\
33.66	0.0016924068007089\\
33.67	0.00169234837357925\\
33.68	0.00169228992335316\\
33.69	0.00169223145001902\\
33.7	0.0016921729535652\\
33.71	0.00169211443398007\\
33.72	0.00169205589125198\\
33.73	0.00169199732536928\\
33.74	0.00169193873632032\\
33.75	0.00169188012409341\\
33.76	0.00169182148867689\\
33.77	0.00169176283005906\\
33.78	0.00169170414822824\\
33.79	0.00169164544317272\\
33.8	0.00169158671488078\\
33.81	0.00169152796334071\\
33.82	0.00169146918854077\\
33.83	0.00169141039046923\\
33.84	0.00169135156911434\\
33.85	0.00169129272446435\\
33.86	0.00169123385650748\\
33.87	0.00169117496523197\\
33.88	0.00169111605062603\\
33.89	0.00169105711267787\\
33.9	0.0016909981513757\\
33.91	0.0016909391667077\\
33.92	0.00169088015866205\\
33.93	0.00169082112722694\\
33.94	0.00169076207239052\\
33.95	0.00169070299414095\\
33.96	0.00169064389246637\\
33.97	0.00169058476735494\\
33.98	0.00169052561879477\\
33.99	0.00169046644677398\\
34	0.0016904072512807\\
34.01	0.00169034803230301\\
34.02	0.00169028878982902\\
34.03	0.00169022952384681\\
34.04	0.00169017023434445\\
34.05	0.00169011092131002\\
34.06	0.00169005158473158\\
34.07	0.00168999222459716\\
34.08	0.00168993284089481\\
34.09	0.00168987343361257\\
34.1	0.00168981400273845\\
34.11	0.00168975454826047\\
34.12	0.00168969507016664\\
34.13	0.00168963556844494\\
34.14	0.00168957604308337\\
34.15	0.0016895164940699\\
34.16	0.0016894569213925\\
34.17	0.00168939732503913\\
34.18	0.00168933770499775\\
34.19	0.00168927806125628\\
34.2	0.00168921839380268\\
34.21	0.00168915870262484\\
34.22	0.0016890989877107\\
34.23	0.00168903924904816\\
34.24	0.0016889794866251\\
34.25	0.00168891970042943\\
34.26	0.00168885989044901\\
34.27	0.00168880005667172\\
34.28	0.00168874019908541\\
34.29	0.00168868031767793\\
34.3	0.00168862041243713\\
34.31	0.00168856048335084\\
34.32	0.00168850053040688\\
34.33	0.00168844055359306\\
34.34	0.00168838055289718\\
34.35	0.00168832052830705\\
34.36	0.00168826047981044\\
34.37	0.00168820040739514\\
34.38	0.00168814031104891\\
34.39	0.00168808019075951\\
34.4	0.00168802004651468\\
34.41	0.00168795987830217\\
34.42	0.00168789968610971\\
34.43	0.00168783946992502\\
34.44	0.0016877792297358\\
34.45	0.00168771896552977\\
34.46	0.00168765867729461\\
34.47	0.00168759836501802\\
34.48	0.00168753802868764\\
34.49	0.00168747766829117\\
34.5	0.00168741728381626\\
34.51	0.00168735687525054\\
34.52	0.00168729644258165\\
34.53	0.00168723598579723\\
34.54	0.00168717550488489\\
34.55	0.00168711499983223\\
34.56	0.00168705447062687\\
34.57	0.00168699391725638\\
34.58	0.00168693333970835\\
34.59	0.00168687273797034\\
34.6	0.00168681211202993\\
34.61	0.00168675146187465\\
34.62	0.00168669078749206\\
34.63	0.00168663008886969\\
34.64	0.00168656936599505\\
34.65	0.00168650861885567\\
34.66	0.00168644784743904\\
34.67	0.00168638705173266\\
34.68	0.00168632623172402\\
34.69	0.00168626538740059\\
34.7	0.00168620451874984\\
34.71	0.00168614362575922\\
34.72	0.00168608270841619\\
34.73	0.00168602176670817\\
34.74	0.00168596080062259\\
34.75	0.00168589981014688\\
34.76	0.00168583879526844\\
34.77	0.00168577775597467\\
34.78	0.00168571669225296\\
34.79	0.00168565560409069\\
34.8	0.00168559449147523\\
34.81	0.00168553335439393\\
34.82	0.00168547219283414\\
34.83	0.00168541100678322\\
34.84	0.00168534979622849\\
34.85	0.00168528856115726\\
34.86	0.00168522730155686\\
34.87	0.00168516601741457\\
34.88	0.00168510470871771\\
34.89	0.00168504337545354\\
34.9	0.00168498201760933\\
34.91	0.00168492063517236\\
34.92	0.00168485922812987\\
34.93	0.0016847977964691\\
34.94	0.0016847363401773\\
34.95	0.00168467485924167\\
34.96	0.00168461335364945\\
34.97	0.00168455182338781\\
34.98	0.00168449026844398\\
34.99	0.00168442868880511\\
35	0.0016843670844584\\
35.01	0.001684305455391\\
35.02	0.00168424380159006\\
35.03	0.00168418212304274\\
35.04	0.00168412041973615\\
35.05	0.00168405869165744\\
35.06	0.00168399693879371\\
35.07	0.00168393516113207\\
35.08	0.00168387335865961\\
35.09	0.00168381153136342\\
35.1	0.00168374967923056\\
35.11	0.00168368780224811\\
35.12	0.00168362590040312\\
35.13	0.00168356397368263\\
35.14	0.00168350202207367\\
35.15	0.00168344004556328\\
35.16	0.00168337804413846\\
35.17	0.00168331601778621\\
35.18	0.00168325396649354\\
35.19	0.00168319189024742\\
35.2	0.00168312978903483\\
35.21	0.00168306766284273\\
35.22	0.00168300551165807\\
35.23	0.00168294333546781\\
35.24	0.00168288113425886\\
35.25	0.00168281890801816\\
35.26	0.00168275665673262\\
35.27	0.00168269438038913\\
35.28	0.0016826320789746\\
35.29	0.00168256975247589\\
35.3	0.0016825074008799\\
35.31	0.00168244502417347\\
35.32	0.00168238262234345\\
35.33	0.0016823201953767\\
35.34	0.00168225774326003\\
35.35	0.00168219526598028\\
35.36	0.00168213276352424\\
35.37	0.00168207023587872\\
35.38	0.00168200768303052\\
35.39	0.0016819451049664\\
35.4	0.00168188250167313\\
35.41	0.00168181987313749\\
35.42	0.0016817572193462\\
35.43	0.00168169454028602\\
35.44	0.00168163183594367\\
35.45	0.00168156910630586\\
35.46	0.00168150635135931\\
35.47	0.0016814435710907\\
35.48	0.00168138076548673\\
35.49	0.00168131793453407\\
35.5	0.00168125507821938\\
35.51	0.00168119219652932\\
35.52	0.00168112928945054\\
35.53	0.00168106635696965\\
35.54	0.0016810033990733\\
35.55	0.00168094041574809\\
35.56	0.00168087740698063\\
35.57	0.0016808143727575\\
35.58	0.00168075131306528\\
35.59	0.00168068822789054\\
35.6	0.00168062511721986\\
35.61	0.00168056198103976\\
35.62	0.0016804988193368\\
35.63	0.0016804356320975\\
35.64	0.00168037241930838\\
35.65	0.00168030918095594\\
35.66	0.00168024591702668\\
35.67	0.00168018262750709\\
35.68	0.00168011931238363\\
35.69	0.00168005597164278\\
35.7	0.00167999260527099\\
35.71	0.00167992921325469\\
35.72	0.00167986579558033\\
35.73	0.00167980235223432\\
35.74	0.00167973888320307\\
35.75	0.00167967538847299\\
35.76	0.00167961186803045\\
35.77	0.00167954832186185\\
35.78	0.00167948474995354\\
35.79	0.00167942115229188\\
35.8	0.00167935752886323\\
35.81	0.0016792938796539\\
35.82	0.00167923020465024\\
35.83	0.00167916650383854\\
35.84	0.00167910277720512\\
35.85	0.00167903902473627\\
35.86	0.00167897524641825\\
35.87	0.00167891144223736\\
35.88	0.00167884761217983\\
35.89	0.00167878375623193\\
35.9	0.00167871987437989\\
35.91	0.00167865596660993\\
35.92	0.00167859203290827\\
35.93	0.00167852807326112\\
35.94	0.00167846408765466\\
35.95	0.00167840007607508\\
35.96	0.00167833603850854\\
35.97	0.00167827197494122\\
35.98	0.00167820788535926\\
35.99	0.00167814376974879\\
36	0.00167807962809594\\
36.01	0.00167801546038683\\
36.02	0.00167795126660757\\
36.03	0.00167788704674425\\
36.04	0.00167782280078294\\
36.05	0.00167775852870973\\
36.06	0.00167769423051067\\
36.07	0.00167762990617181\\
36.08	0.00167756555567919\\
36.09	0.00167750117901884\\
36.1	0.00167743677617676\\
36.11	0.00167737234713897\\
36.12	0.00167730789189146\\
36.13	0.00167724341042022\\
36.14	0.0016771789027112\\
36.15	0.00167711436875038\\
36.16	0.0016770498085237\\
36.17	0.00167698522201709\\
36.18	0.00167692060921649\\
36.19	0.0016768559701078\\
36.2	0.00167679130467693\\
36.21	0.00167672661290978\\
36.22	0.00167666189479222\\
36.23	0.00167659715031012\\
36.24	0.00167653237944934\\
36.25	0.00167646758219573\\
36.26	0.00167640275853512\\
36.27	0.00167633790845334\\
36.28	0.00167627303193618\\
36.29	0.00167620812896947\\
36.3	0.00167614319953899\\
36.31	0.00167607824363051\\
36.32	0.00167601326122981\\
36.33	0.00167594825232262\\
36.34	0.00167588321689472\\
36.35	0.00167581815493181\\
36.36	0.00167575306641963\\
36.37	0.00167568795134388\\
36.38	0.00167562280969026\\
36.39	0.00167555764144445\\
36.4	0.00167549244659214\\
36.41	0.00167542722511897\\
36.42	0.00167536197701061\\
36.43	0.00167529670225269\\
36.44	0.00167523140083085\\
36.45	0.00167516607273069\\
36.46	0.00167510071793782\\
36.47	0.00167503533643784\\
36.48	0.00167496992821633\\
36.49	0.00167490449325885\\
36.5	0.00167483903155097\\
36.51	0.00167477354307823\\
36.52	0.00167470802782616\\
36.53	0.00167464248578029\\
36.54	0.00167457691692613\\
36.55	0.00167451132124918\\
36.56	0.00167444569873492\\
36.57	0.00167438004936884\\
36.58	0.00167431437313639\\
36.59	0.00167424867002303\\
36.6	0.0016741829400142\\
36.61	0.00167411718309532\\
36.62	0.00167405139925182\\
36.63	0.00167398558846909\\
36.64	0.00167391975073253\\
36.65	0.00167385388602753\\
36.66	0.00167378799433945\\
36.67	0.00167372207565364\\
36.68	0.00167365612995546\\
36.69	0.00167359015723023\\
36.7	0.00167352415746328\\
36.71	0.00167345813063992\\
36.72	0.00167339207674545\\
36.73	0.00167332599576513\\
36.74	0.00167325988768427\\
36.75	0.00167319375248811\\
36.76	0.0016731275901619\\
36.77	0.00167306140069088\\
36.78	0.00167299518406027\\
36.79	0.0016729289402553\\
36.8	0.00167286266926115\\
36.81	0.00167279637106302\\
36.82	0.00167273004564609\\
36.83	0.00167266369299552\\
36.84	0.00167259731309646\\
36.85	0.00167253090593405\\
36.86	0.00167246447149342\\
36.87	0.00167239800975968\\
36.88	0.00167233152071795\\
36.89	0.0016722650043533\\
36.9	0.00167219846065083\\
36.91	0.00167213188959559\\
36.92	0.00167206529117264\\
36.93	0.00167199866536703\\
36.94	0.00167193201216378\\
36.95	0.00167186533154791\\
36.96	0.00167179862350443\\
36.97	0.00167173188801833\\
36.98	0.00167166512507459\\
36.99	0.00167159833465818\\
37	0.00167153151675405\\
37.01	0.00167146467134716\\
37.02	0.00167139779842242\\
37.03	0.00167133089796476\\
37.04	0.00167126396995909\\
37.05	0.0016711970143903\\
37.06	0.00167113003124327\\
37.07	0.00167106302050286\\
37.08	0.00167099598215395\\
37.09	0.00167092891618136\\
37.1	0.00167086182256994\\
37.11	0.0016707947013045\\
37.12	0.00167072755236984\\
37.13	0.00167066037575076\\
37.14	0.00167059317143205\\
37.15	0.00167052593939847\\
37.16	0.00167045867963478\\
37.17	0.00167039139212571\\
37.18	0.00167032407685601\\
37.19	0.00167025673381038\\
37.2	0.00167018936297355\\
37.21	0.00167012196433019\\
37.22	0.00167005453786498\\
37.23	0.00166998708356261\\
37.24	0.00166991960140771\\
37.25	0.00166985209138494\\
37.26	0.00166978455347892\\
37.27	0.00166971698767426\\
37.28	0.00166964939395558\\
37.29	0.00166958177230746\\
37.3	0.00166951412271448\\
37.31	0.00166944644516121\\
37.32	0.0016693787396322\\
37.33	0.00166931100611199\\
37.34	0.00166924324458509\\
37.35	0.00166917545503605\\
37.36	0.00166910763744934\\
37.37	0.00166903979180945\\
37.38	0.00166897191810087\\
37.39	0.00166890401630806\\
37.4	0.00166883608641547\\
37.41	0.00166876812840752\\
37.42	0.00166870014226865\\
37.43	0.00166863212798327\\
37.44	0.00166856408553576\\
37.45	0.00166849601491053\\
37.46	0.00166842791609193\\
37.47	0.00166835978906433\\
37.48	0.00166829163381207\\
37.49	0.00166822345031949\\
37.5	0.0016681552385709\\
37.51	0.00166808699855061\\
37.52	0.0016680187302429\\
37.53	0.00166795043363208\\
37.54	0.00166788210870238\\
37.55	0.00166781375543809\\
37.56	0.00166774537382342\\
37.57	0.00166767696384261\\
37.58	0.00166760852547987\\
37.59	0.0016675400587194\\
37.6	0.0016674715635454\\
37.61	0.00166740303994203\\
37.62	0.00166733448789346\\
37.63	0.00166726590738382\\
37.64	0.00166719729839727\\
37.65	0.0016671286609179\\
37.66	0.00166705999492985\\
37.67	0.0016669913004172\\
37.68	0.00166692257736402\\
37.69	0.00166685382575439\\
37.7	0.00166678504557237\\
37.71	0.00166671623680198\\
37.72	0.00166664739942726\\
37.73	0.00166657853343223\\
37.74	0.00166650963880088\\
37.75	0.0016664407155172\\
37.76	0.00166637176356516\\
37.77	0.00166630278292873\\
37.78	0.00166623377359184\\
37.79	0.00166616473553844\\
37.8	0.00166609566875244\\
37.81	0.00166602657321775\\
37.82	0.00166595744891826\\
37.83	0.00166588829583785\\
37.84	0.00166581911396038\\
37.85	0.0016657499032697\\
37.86	0.00166568066374966\\
37.87	0.00166561139538408\\
37.88	0.00166554209815676\\
37.89	0.0016654727720515\\
37.9	0.00166540341705209\\
37.91	0.0016653340331423\\
37.92	0.00166526462030588\\
37.93	0.00166519517852657\\
37.94	0.00166512570778811\\
37.95	0.0016650562080742\\
37.96	0.00166498667936855\\
37.97	0.00166491712165485\\
37.98	0.00166484753491676\\
37.99	0.00166477791913795\\
38	0.00166470827430205\\
38.01	0.00166463860039271\\
38.02	0.00166456889739354\\
38.03	0.00166449916528815\\
38.04	0.00166442940406012\\
38.05	0.00166435961369303\\
38.06	0.00166428979417044\\
38.07	0.00166421994547591\\
38.08	0.00166415006759295\\
38.09	0.0016640801605051\\
38.1	0.00166401022419586\\
38.11	0.00166394025864873\\
38.12	0.00166387026384717\\
38.13	0.00166380023977466\\
38.14	0.00166373018641464\\
38.15	0.00166366010375056\\
38.16	0.00166358999176582\\
38.17	0.00166351985044385\\
38.18	0.00166344967976803\\
38.19	0.00166337947972174\\
38.2	0.00166330925028834\\
38.21	0.0016632389914512\\
38.22	0.00166316870319365\\
38.23	0.00166309838549901\\
38.24	0.00166302803835058\\
38.25	0.00166295766173166\\
38.26	0.00166288725562554\\
38.27	0.00166281682001548\\
38.28	0.00166274635488473\\
38.29	0.00166267586021652\\
38.3	0.00166260533599409\\
38.31	0.00166253478220063\\
38.32	0.00166246419881935\\
38.33	0.00166239358583342\\
38.34	0.00166232294322601\\
38.35	0.00166225227098028\\
38.36	0.00166218156907934\\
38.37	0.00166211083750635\\
38.38	0.00166204007624438\\
38.39	0.00166196928527656\\
38.4	0.00166189846458594\\
38.41	0.0016618276141556\\
38.42	0.00166175673396859\\
38.43	0.00166168582400793\\
38.44	0.00166161488425667\\
38.45	0.0016615439146978\\
38.46	0.0016614729153143\\
38.47	0.00166140188608917\\
38.48	0.00166133082700537\\
38.49	0.00166125973804584\\
38.5	0.00166118861919351\\
38.51	0.00166111747043131\\
38.52	0.00166104629174214\\
38.53	0.0016609750831089\\
38.54	0.00166090384451445\\
38.55	0.00166083257594166\\
38.56	0.00166076127737337\\
38.57	0.00166068994879242\\
38.58	0.00166061859018162\\
38.59	0.00166054720152377\\
38.6	0.00166047578280167\\
38.61	0.00166040433399808\\
38.62	0.00166033285509576\\
38.63	0.00166026134607746\\
38.64	0.0016601898069259\\
38.65	0.00166011823762379\\
38.66	0.00166004663815384\\
38.67	0.00165997500849873\\
38.68	0.00165990334864113\\
38.69	0.00165983165856369\\
38.7	0.00165975993824905\\
38.71	0.00165968818767983\\
38.72	0.00165961640683865\\
38.73	0.0016595445957081\\
38.74	0.00165947275427075\\
38.75	0.00165940088250918\\
38.76	0.00165932898040593\\
38.77	0.00165925704794353\\
38.78	0.00165918508510451\\
38.79	0.00165911309187137\\
38.8	0.0016590410682266\\
38.81	0.00165896901415267\\
38.82	0.00165889692963205\\
38.83	0.00165882481464718\\
38.84	0.00165875266918047\\
38.85	0.00165868049321437\\
38.86	0.00165860828673125\\
38.87	0.0016585360497135\\
38.88	0.0016584637821435\\
38.89	0.00165839148400359\\
38.9	0.00165831915527612\\
38.91	0.00165824679594341\\
38.92	0.00165817440598776\\
38.93	0.00165810198539146\\
38.94	0.00165802953413681\\
38.95	0.00165795705220605\\
38.96	0.00165788453958144\\
38.97	0.0016578119962452\\
38.98	0.00165773942217957\\
38.99	0.00165766681736672\\
39	0.00165759418178885\\
39.01	0.00165752151542814\\
39.02	0.00165744881826673\\
39.03	0.00165737609028678\\
39.04	0.00165730333147039\\
39.05	0.00165723054179968\\
39.06	0.00165715772125675\\
39.07	0.00165708486982367\\
39.08	0.00165701198748251\\
39.09	0.00165693907421531\\
39.1	0.00165686613000411\\
39.11	0.00165679315483092\\
39.12	0.00165672014867775\\
39.13	0.00165664711152658\\
39.14	0.00165657404335937\\
39.15	0.0016565009441581\\
39.16	0.00165642781390468\\
39.17	0.00165635465258106\\
39.18	0.00165628146016913\\
39.19	0.0016562082366508\\
39.2	0.00165613498200793\\
39.21	0.00165606169622239\\
39.22	0.00165598837927603\\
39.23	0.00165591503115066\\
39.24	0.00165584165182813\\
39.25	0.00165576824129021\\
39.26	0.0016556947995187\\
39.27	0.00165562132649535\\
39.28	0.00165554782220193\\
39.29	0.00165547428662017\\
39.3	0.0016554007197318\\
39.31	0.00165532712151851\\
39.32	0.001655253491962\\
39.33	0.00165517983104394\\
39.34	0.00165510613874599\\
39.35	0.00165503241504978\\
39.36	0.00165495865993696\\
39.37	0.00165488487338912\\
39.38	0.00165481105538788\\
39.39	0.00165473720591479\\
39.4	0.00165466332495143\\
39.41	0.00165458941247935\\
39.42	0.00165451546848008\\
39.43	0.00165444149293513\\
39.44	0.00165436748582601\\
39.45	0.0016542934471342\\
39.46	0.00165421937684116\\
39.47	0.00165414527492836\\
39.48	0.00165407114137723\\
39.49	0.00165399697616919\\
39.5	0.00165392277928564\\
39.51	0.00165384855070798\\
39.52	0.00165377429041757\\
39.53	0.00165369999839578\\
39.54	0.00165362567462395\\
39.55	0.00165355131908341\\
39.56	0.00165347693175546\\
39.57	0.0016534025126214\\
39.58	0.0016533280616625\\
39.59	0.00165325357886004\\
39.6	0.00165317906419524\\
39.61	0.00165310451764936\\
39.62	0.00165302993920359\\
39.63	0.00165295532883914\\
39.64	0.00165288068653719\\
39.65	0.0016528060122789\\
39.66	0.00165273130604543\\
39.67	0.00165265656781791\\
39.68	0.00165258179757746\\
39.69	0.00165250699530518\\
39.7	0.00165243216098216\\
39.71	0.00165235729458946\\
39.72	0.00165228239610814\\
39.73	0.00165220746551924\\
39.74	0.00165213250280378\\
39.75	0.00165205750794277\\
39.76	0.00165198248091719\\
39.77	0.00165190742170803\\
39.78	0.00165183233029623\\
39.79	0.00165175720666273\\
39.8	0.00165168205078847\\
39.81	0.00165160686265436\\
39.82	0.00165153164224128\\
39.83	0.00165145638953011\\
39.84	0.00165138110450172\\
39.85	0.00165130578713694\\
39.86	0.00165123043741662\\
39.87	0.00165115505532155\\
39.88	0.00165107964083254\\
39.89	0.00165100419393036\\
39.9	0.00165092871459579\\
39.91	0.00165085320280956\\
39.92	0.00165077765855241\\
39.93	0.00165070208180506\\
39.94	0.00165062647254821\\
39.95	0.00165055083076252\\
39.96	0.00165047515642869\\
39.97	0.00165039944952735\\
39.98	0.00165032371003914\\
39.99	0.00165024793794467\\
40	0.00165017213322456\\
40.01	0.00165009629585938\\
};
\addplot [color=mycolor1,solid,forget plot]
  table[row sep=crcr]{%
40.01	0.00165009629585938\\
40.02	0.00165002042582972\\
40.03	0.00164994452311611\\
40.04	0.0016498685876991\\
40.05	0.00164979261955921\\
40.06	0.00164971661867694\\
40.07	0.00164964058503279\\
40.08	0.00164956451860722\\
40.09	0.00164948841938069\\
40.1	0.00164941228733365\\
40.11	0.00164933612244651\\
40.12	0.00164925992469969\\
40.13	0.00164918369407357\\
40.14	0.00164910743054854\\
40.15	0.00164903113410494\\
40.16	0.00164895480472313\\
40.17	0.00164887844238343\\
40.18	0.00164880204706614\\
40.19	0.00164872561875158\\
40.2	0.00164864915742001\\
40.21	0.00164857266305169\\
40.22	0.00164849613562687\\
40.23	0.00164841957512578\\
40.24	0.00164834298152863\\
40.25	0.00164826635481562\\
40.26	0.00164818969496694\\
40.27	0.00164811300196274\\
40.28	0.00164803627578317\\
40.29	0.00164795951640837\\
40.3	0.00164788272381845\\
40.31	0.00164780589799352\\
40.32	0.00164772903891365\\
40.33	0.00164765214655891\\
40.34	0.00164757522090935\\
40.35	0.00164749826194501\\
40.36	0.00164742126964591\\
40.37	0.00164734424399205\\
40.38	0.00164726718496341\\
40.39	0.00164719009253997\\
40.4	0.00164711296670167\\
40.41	0.00164703580742846\\
40.42	0.00164695861470027\\
40.43	0.00164688138849698\\
40.44	0.0016468041287985\\
40.45	0.00164672683558469\\
40.46	0.00164664950883542\\
40.47	0.00164657214853052\\
40.48	0.00164649475464981\\
40.49	0.00164641732717311\\
40.5	0.00164633986608021\\
40.51	0.00164626237135088\\
40.52	0.00164618484296488\\
40.53	0.00164610728090195\\
40.54	0.00164602968514183\\
40.55	0.00164595205566421\\
40.56	0.00164587439244881\\
40.57	0.0016457966954753\\
40.58	0.00164571896472333\\
40.59	0.00164564120017256\\
40.6	0.00164556340180261\\
40.61	0.0016454855695931\\
40.62	0.00164540770352362\\
40.63	0.00164532980357377\\
40.64	0.0016452518697231\\
40.65	0.00164517390195116\\
40.66	0.00164509590023749\\
40.67	0.0016450178645616\\
40.68	0.00164493979490299\\
40.69	0.00164486169124114\\
40.7	0.00164478355355553\\
40.71	0.00164470538182561\\
40.72	0.00164462717603082\\
40.73	0.00164454893615056\\
40.74	0.00164447066216425\\
40.75	0.00164439235405127\\
40.76	0.00164431401179099\\
40.77	0.00164423563536277\\
40.78	0.00164415722474595\\
40.79	0.00164407877991984\\
40.8	0.00164400030086376\\
40.81	0.00164392178755699\\
40.82	0.00164384323997881\\
40.83	0.00164376465810847\\
40.84	0.00164368604192522\\
40.85	0.00164360739140828\\
40.86	0.00164352870653685\\
40.87	0.00164344998729014\\
40.88	0.0016433712336473\\
40.89	0.00164329244558751\\
40.9	0.00164321362308991\\
40.91	0.00164313476613361\\
40.92	0.00164305587469775\\
40.93	0.00164297694876139\\
40.94	0.00164289798830363\\
40.95	0.00164281899330352\\
40.96	0.00164273996374011\\
40.97	0.00164266089959242\\
40.98	0.00164258180083947\\
40.99	0.00164250266746024\\
41	0.00164242349943373\\
41.01	0.00164234429673888\\
41.02	0.00164226505935464\\
41.03	0.00164218578725995\\
41.04	0.0016421064804337\\
41.05	0.0016420271388548\\
41.06	0.00164194776250212\\
41.07	0.00164186835135453\\
41.08	0.00164178890539086\\
41.09	0.00164170942458994\\
41.1	0.00164162990893059\\
41.11	0.00164155035839159\\
41.12	0.00164147077295172\\
41.13	0.00164139115258975\\
41.14	0.00164131149728439\\
41.15	0.00164123180701439\\
41.16	0.00164115208175845\\
41.17	0.00164107232149526\\
41.18	0.00164099252620349\\
41.19	0.00164091269586179\\
41.2	0.00164083283044879\\
41.21	0.00164075292994313\\
41.22	0.00164067299432339\\
41.23	0.00164059302356816\\
41.24	0.00164051301765601\\
41.25	0.00164043297656548\\
41.26	0.0016403529002751\\
41.27	0.00164027278876338\\
41.28	0.00164019264200882\\
41.29	0.00164011245998987\\
41.3	0.00164003224268501\\
41.31	0.00163995199007267\\
41.32	0.00163987170213126\\
41.33	0.00163979137883917\\
41.34	0.0016397110201748\\
41.35	0.0016396306261165\\
41.36	0.00163955019664261\\
41.37	0.00163946973173146\\
41.38	0.00163938923136133\\
41.39	0.00163930869551051\\
41.4	0.00163922812415728\\
41.41	0.00163914751727986\\
41.42	0.00163906687485647\\
41.43	0.00163898619686533\\
41.44	0.00163890548328459\\
41.45	0.00163882473409243\\
41.46	0.00163874394926699\\
41.47	0.00163866312878637\\
41.48	0.00163858227262866\\
41.49	0.00163850138077196\\
41.5	0.00163842045319429\\
41.51	0.0016383394898737\\
41.52	0.00163825849078819\\
41.53	0.00163817745591573\\
41.54	0.00163809638523429\\
41.55	0.00163801527872181\\
41.56	0.00163793413635619\\
41.57	0.00163785295811532\\
41.58	0.00163777174397708\\
41.59	0.00163769049391928\\
41.6	0.00163760920791976\\
41.61	0.0016375278859563\\
41.62	0.00163744652800666\\
41.63	0.00163736513404858\\
41.64	0.00163728370405977\\
41.65	0.00163720223801792\\
41.66	0.00163712073590067\\
41.67	0.00163703919768566\\
41.68	0.00163695762335049\\
41.69	0.00163687601287273\\
41.7	0.00163679436622992\\
41.71	0.00163671268339958\\
41.72	0.0016366309643592\\
41.73	0.00163654920908621\\
41.74	0.00163646741755806\\
41.75	0.00163638558975212\\
41.76	0.00163630372564575\\
41.77	0.00163622182521629\\
41.78	0.00163613988844102\\
41.79	0.0016360579152972\\
41.8	0.00163597590576206\\
41.81	0.00163589385981277\\
41.82	0.00163581177742651\\
41.83	0.00163572965858038\\
41.84	0.00163564750325147\\
41.85	0.00163556531141681\\
41.86	0.0016354830830534\\
41.87	0.00163540081813822\\
41.88	0.00163531851664819\\
41.89	0.00163523617856018\\
41.9	0.00163515380385103\\
41.91	0.00163507139249755\\
41.92	0.00163498894447649\\
41.93	0.00163490645976455\\
41.94	0.00163482393833841\\
41.95	0.00163474138017467\\
41.96	0.00163465878524992\\
41.97	0.00163457615354066\\
41.98	0.00163449348502339\\
41.99	0.0016344107796745\\
42	0.00163432803747039\\
42.01	0.00163424525838736\\
42.02	0.00163416244240168\\
42.03	0.00163407958948956\\
42.04	0.00163399669962715\\
42.05	0.00163391377279055\\
42.06	0.0016338308089558\\
42.07	0.00163374780809886\\
42.08	0.00163366477019567\\
42.09	0.00163358169522206\\
42.1	0.00163349858315384\\
42.11	0.00163341543396673\\
42.12	0.00163333224763637\\
42.13	0.00163324902413837\\
42.14	0.00163316576344824\\
42.15	0.00163308246554142\\
42.16	0.00163299913039331\\
42.17	0.00163291575797917\\
42.18	0.00163283234827426\\
42.19	0.0016327489012537\\
42.2	0.00163266541689258\\
42.21	0.00163258189516586\\
42.22	0.00163249833604845\\
42.23	0.00163241473951517\\
42.24	0.00163233110554075\\
42.25	0.00163224743409983\\
42.26	0.00163216372516695\\
42.27	0.00163207997871656\\
42.28	0.00163199619472305\\
42.29	0.00163191237316066\\
42.3	0.00163182851400356\\
42.31	0.00163174461722583\\
42.32	0.00163166068280141\\
42.33	0.00163157671070416\\
42.34	0.00163149270090783\\
42.35	0.00163140865338607\\
42.36	0.00163132456811239\\
42.37	0.00163124044506021\\
42.38	0.00163115628420282\\
42.39	0.0016310720855134\\
42.4	0.00163098784896499\\
42.41	0.00163090357453054\\
42.42	0.00163081926218285\\
42.43	0.00163073491189458\\
42.44	0.00163065052363829\\
42.45	0.00163056609738637\\
42.46	0.00163048163311111\\
42.47	0.00163039713078463\\
42.48	0.00163031259037894\\
42.49	0.00163022801186586\\
42.5	0.00163014339521711\\
42.51	0.00163005874040425\\
42.52	0.00162997404739866\\
42.53	0.0016298893161716\\
42.54	0.00162980454669416\\
42.55	0.00162971973893726\\
42.56	0.00162963489287167\\
42.57	0.00162955000846801\\
42.58	0.0016294650856967\\
42.59	0.00162938012452801\\
42.6	0.00162929512493204\\
42.61	0.00162921008687872\\
42.62	0.0016291250103378\\
42.63	0.00162903989527882\\
42.64	0.0016289547416712\\
42.65	0.00162886954948414\\
42.66	0.00162878431868665\\
42.67	0.00162869904924758\\
42.68	0.00162861374113557\\
42.69	0.00162852839431908\\
42.7	0.00162844300876639\\
42.71	0.00162835758444556\\
42.72	0.00162827212132449\\
42.73	0.00162818661937086\\
42.74	0.00162810107855217\\
42.75	0.00162801549883572\\
42.76	0.0016279298801886\\
42.77	0.00162784422257774\\
42.78	0.00162775852596983\\
42.79	0.00162767279033141\\
42.8	0.00162758701562879\\
42.81	0.00162750120182809\\
42.82	0.00162741534889527\\
42.83	0.00162732945679606\\
42.84	0.00162724352549604\\
42.85	0.00162715755496055\\
42.86	0.00162707154515482\\
42.87	0.00162698549604384\\
42.88	0.00162689940759246\\
42.89	0.00162681327976534\\
42.9	0.001626727112527\\
42.91	0.00162664090584178\\
42.92	0.00162655465967387\\
42.93	0.00162646837398732\\
42.94	0.00162638204874604\\
42.95	0.00162629568391382\\
42.96	0.00162620927945432\\
42.97	0.0016261228353311\\
42.98	0.00162603635150762\\
42.99	0.00162594982794726\\
43	0.00162586326461332\\
43.01	0.00162577666146905\\
43.02	0.00162569001847767\\
43.03	0.00162560333560235\\
43.04	0.00162551661280628\\
43.05	0.00162542985005267\\
43.06	0.00162534304730473\\
43.07	0.00162525620452578\\
43.08	0.00162516932167917\\
43.09	0.00162508239872842\\
43.1	0.00162499543563713\\
43.11	0.00162490843236912\\
43.12	0.00162482138888838\\
43.13	0.00162473430515914\\
43.14	0.0016246471811459\\
43.15	0.00162456001681349\\
43.16	0.00162447281212708\\
43.17	0.00162438556705224\\
43.18	0.00162429828155498\\
43.19	0.0016242109556018\\
43.2	0.00162412358915977\\
43.21	0.00162403618219654\\
43.22	0.00162394873468044\\
43.23	0.00162386124658053\\
43.24	0.00162377371786664\\
43.25	0.00162368614850952\\
43.26	0.00162359853848079\\
43.27	0.00162351088775316\\
43.28	0.00162342319629936\\
43.29	0.0016233354640921\\
43.3	0.00162324769110406\\
43.31	0.0016231598773079\\
43.32	0.00162307202267624\\
43.33	0.00162298412718169\\
43.34	0.00162289619079682\\
43.35	0.00162280821349416\\
43.36	0.00162272019524624\\
43.37	0.00162263213602555\\
43.38	0.00162254403580454\\
43.39	0.00162245589455565\\
43.4	0.00162236771225128\\
43.41	0.00162227948886381\\
43.42	0.00162219122436558\\
43.43	0.00162210291872893\\
43.44	0.00162201457192613\\
43.45	0.00162192618392946\\
43.46	0.00162183775471115\\
43.47	0.00162174928424341\\
43.48	0.00162166077249842\\
43.49	0.00162157221944833\\
43.5	0.00162148362506526\\
43.51	0.00162139498932131\\
43.52	0.00162130631218854\\
43.53	0.001621217593639\\
43.54	0.00162112883364469\\
43.55	0.0016210400321776\\
43.56	0.00162095118920967\\
43.57	0.00162086230471283\\
43.58	0.00162077337865898\\
43.59	0.00162068441101999\\
43.6	0.00162059540176769\\
43.61	0.00162050635087389\\
43.62	0.00162041725831037\\
43.63	0.0016203281240489\\
43.64	0.00162023894806118\\
43.65	0.00162014973031892\\
43.66	0.00162006047079378\\
43.67	0.0016199711694574\\
43.68	0.00161988182628138\\
43.69	0.00161979244123732\\
43.7	0.00161970301429675\\
43.71	0.00161961354543121\\
43.72	0.00161952403461218\\
43.73	0.00161943448181112\\
43.74	0.00161934488699948\\
43.75	0.00161925525014865\\
43.76	0.00161916557123001\\
43.77	0.00161907585021492\\
43.78	0.00161898608707469\\
43.79	0.0016188962817806\\
43.8	0.00161880643430392\\
43.81	0.00161871654461588\\
43.82	0.00161862661268768\\
43.83	0.00161853663849049\\
43.84	0.00161844662199545\\
43.85	0.00161835656317368\\
43.86	0.00161826646199626\\
43.87	0.00161817631843425\\
43.88	0.00161808613245866\\
43.89	0.0016179959040405\\
43.9	0.00161790563315074\\
43.91	0.00161781531976029\\
43.92	0.00161772496384008\\
43.93	0.00161763456536098\\
43.94	0.00161754412429384\\
43.95	0.00161745364060947\\
43.96	0.00161736311427867\\
43.97	0.00161727254527218\\
43.98	0.00161718193356075\\
43.99	0.00161709127911507\\
44	0.0016170005819058\\
44.01	0.00161690984190359\\
44.02	0.00161681905907905\\
44.03	0.00161672823340275\\
44.04	0.00161663736484525\\
44.05	0.00161654645337705\\
44.06	0.00161645549896867\\
44.07	0.00161636450159054\\
44.08	0.00161627346121311\\
44.09	0.00161618237780676\\
44.1	0.00161609125134188\\
44.11	0.0016160000817888\\
44.12	0.00161590886911784\\
44.13	0.00161581761329926\\
44.14	0.00161572631430331\\
44.15	0.00161563497210023\\
44.16	0.00161554358666019\\
44.17	0.00161545215795336\\
44.18	0.00161536068594986\\
44.19	0.00161526917061978\\
44.2	0.00161517761193321\\
44.21	0.00161508600986016\\
44.22	0.00161499436437067\\
44.23	0.00161490267543469\\
44.24	0.00161481094302217\\
44.25	0.00161471916710304\\
44.26	0.00161462734764716\\
44.27	0.00161453548462442\\
44.28	0.00161444357800462\\
44.29	0.00161435162775756\\
44.3	0.001614259633853\\
44.31	0.00161416759626069\\
44.32	0.00161407551495031\\
44.33	0.00161398338989155\\
44.34	0.00161389122105404\\
44.35	0.0016137990084074\\
44.36	0.00161370675192121\\
44.37	0.00161361445156501\\
44.38	0.00161352210730832\\
44.39	0.00161342971912065\\
44.4	0.00161333728697144\\
44.41	0.00161324481083012\\
44.42	0.0016131522906661\\
44.43	0.00161305972644872\\
44.44	0.00161296711814734\\
44.45	0.00161287446573126\\
44.46	0.00161278176916975\\
44.47	0.00161268902843206\\
44.48	0.0016125962434874\\
44.49	0.00161250341430495\\
44.5	0.00161241054085386\\
44.51	0.00161231762310327\\
44.52	0.00161222466102224\\
44.53	0.00161213165457987\\
44.54	0.00161203860374515\\
44.55	0.0016119455084871\\
44.56	0.00161185236877468\\
44.57	0.00161175918457684\\
44.58	0.00161166595586247\\
44.59	0.00161157268260046\\
44.6	0.00161147936475965\\
44.61	0.00161138600230886\\
44.62	0.00161129259521686\\
44.63	0.00161119914345242\\
44.64	0.00161110564698425\\
44.65	0.00161101210578105\\
44.66	0.00161091851981148\\
44.67	0.00161082488904417\\
44.68	0.00161073121344772\\
44.69	0.0016106374929907\\
44.7	0.00161054372764166\\
44.71	0.00161044991736909\\
44.72	0.00161035606214148\\
44.73	0.00161026216192728\\
44.74	0.0016101682166949\\
44.75	0.00161007422641273\\
44.76	0.00160998019104913\\
44.77	0.00160988611057242\\
44.78	0.0016097919849509\\
44.79	0.00160969781415283\\
44.8	0.00160960359814645\\
44.81	0.00160950933689997\\
44.82	0.00160941503038155\\
44.83	0.00160932067855934\\
44.84	0.00160922628140146\\
44.85	0.00160913183887599\\
44.86	0.00160903735095098\\
44.87	0.00160894281759446\\
44.88	0.00160884823877441\\
44.89	0.00160875361445881\\
44.9	0.00160865894461559\\
44.91	0.00160856422921265\\
44.92	0.00160846946821786\\
44.93	0.00160837466159906\\
44.94	0.00160827980932408\\
44.95	0.0016081849113607\\
44.96	0.00160808996767666\\
44.97	0.0016079949782397\\
44.98	0.0016078999430175\\
44.99	0.00160780486197775\\
45	0.00160770973508807\\
45.01	0.00160761456231607\\
45.02	0.00160751934362932\\
45.03	0.00160742407899538\\
45.04	0.00160732876838176\\
45.05	0.00160723341175596\\
45.06	0.00160713800908544\\
45.07	0.00160704256033762\\
45.08	0.00160694706547992\\
45.09	0.00160685152447971\\
45.1	0.00160675593730432\\
45.11	0.00160666030392109\\
45.12	0.0016065646242973\\
45.13	0.00160646889840021\\
45.14	0.00160637312619705\\
45.15	0.00160627730765502\\
45.16	0.00160618144274131\\
45.17	0.00160608553142305\\
45.18	0.00160598957366738\\
45.19	0.00160589356944138\\
45.2	0.00160579751871211\\
45.21	0.00160570142144662\\
45.22	0.00160560527761192\\
45.23	0.00160550908717498\\
45.24	0.00160541285010276\\
45.25	0.00160531656636219\\
45.26	0.00160522023592017\\
45.27	0.00160512385874358\\
45.28	0.00160502743479926\\
45.29	0.00160493096405404\\
45.3	0.0016048344464747\\
45.31	0.00160473788202803\\
45.32	0.00160464127068076\\
45.33	0.00160454461239961\\
45.34	0.00160444790715127\\
45.35	0.00160435115490241\\
45.36	0.00160425435561967\\
45.37	0.00160415750926967\\
45.38	0.001604060615819\\
45.39	0.00160396367523422\\
45.4	0.00160386668748188\\
45.41	0.00160376965252849\\
45.42	0.00160367257034055\\
45.43	0.00160357544088452\\
45.44	0.00160347826412686\\
45.45	0.00160338104003399\\
45.46	0.0016032837685723\\
45.47	0.00160318644970817\\
45.48	0.00160308908340797\\
45.49	0.001602991669638\\
45.5	0.0016028942083646\\
45.51	0.00160279669955404\\
45.52	0.00160269914317258\\
45.53	0.00160260153918648\\
45.54	0.00160250388756196\\
45.55	0.00160240618826521\\
45.56	0.00160230844126243\\
45.57	0.00160221064651977\\
45.58	0.00160211280400337\\
45.59	0.00160201491367936\\
45.6	0.00160191697551384\\
45.61	0.00160181898947289\\
45.62	0.00160172095552257\\
45.63	0.00160162287362894\\
45.64	0.00160152474375802\\
45.65	0.00160142656587583\\
45.66	0.00160132833994835\\
45.67	0.00160123006594157\\
45.68	0.00160113174382145\\
45.69	0.00160103337355393\\
45.7	0.00160093495510494\\
45.71	0.0016008364884404\\
45.72	0.0016007379735262\\
45.73	0.00160063941032823\\
45.74	0.00160054079881236\\
45.75	0.00160044213894446\\
45.76	0.00160034343069036\\
45.77	0.00160024467401589\\
45.78	0.00160014586888689\\
45.79	0.00160004701526916\\
45.8	0.00159994811312848\\
45.81	0.00159984916243067\\
45.82	0.00159975016314149\\
45.83	0.00159965111522671\\
45.84	0.00159955201865208\\
45.85	0.00159945287338338\\
45.86	0.00159935367938633\\
45.87	0.00159925443662667\\
45.88	0.00159915514507014\\
45.89	0.00159905580468246\\
45.9	0.00159895641542936\\
45.91	0.00159885697727653\\
45.92	0.00159875749018971\\
45.93	0.0015986579541346\\
45.94	0.0015985583690769\\
45.95	0.00159845873498232\\
45.96	0.00159835905181656\\
45.97	0.00159825931954533\\
45.98	0.00159815953813433\\
45.99	0.00159805970754926\\
46	0.00159795982775584\\
46.01	0.00159785989871978\\
46.02	0.00159775992040679\\
46.03	0.00159765989278259\\
46.04	0.0015975598158129\\
46.05	0.00159745968946346\\
46.06	0.00159735951370001\\
46.07	0.00159725928848829\\
46.08	0.00159715901379406\\
46.09	0.0015970586895831\\
46.1	0.00159695831582116\\
46.11	0.00159685789247405\\
46.12	0.00159675741950757\\
46.13	0.00159665689688755\\
46.14	0.0015965563245798\\
46.15	0.00159645570255019\\
46.16	0.00159635503076457\\
46.17	0.00159625430918885\\
46.18	0.00159615353778892\\
46.19	0.00159605271653072\\
46.2	0.00159595184538019\\
46.21	0.00159585092430332\\
46.22	0.00159574995326609\\
46.23	0.00159564893223454\\
46.24	0.00159554786117473\\
46.25	0.00159544674005273\\
46.26	0.00159534556883467\\
46.27	0.00159524434748668\\
46.28	0.00159514307597495\\
46.29	0.00159504175426569\\
46.3	0.00159494038232515\\
46.31	0.00159483896011964\\
46.32	0.00159473748761546\\
46.33	0.001594635964779\\
46.34	0.00159453439157667\\
46.35	0.00159443276797492\\
46.36	0.00159433109394028\\
46.37	0.00159422936943928\\
46.38	0.00159412759443854\\
46.39	0.00159402576890471\\
46.4	0.0015939238928045\\
46.41	0.00159382196610467\\
46.42	0.00159371998877206\\
46.43	0.00159361796077355\\
46.44	0.00159351588207608\\
46.45	0.00159341375264666\\
46.46	0.00159331157245238\\
46.47	0.00159320934146037\\
46.48	0.00159310705963786\\
46.49	0.00159300472695214\\
46.5	0.00159290234337057\\
46.51	0.00159279990886059\\
46.52	0.00159269742338974\\
46.53	0.00159259488692562\\
46.54	0.00159249229943591\\
46.55	0.00159238966088841\\
46.56	0.00159228697125099\\
46.57	0.0015921842304916\\
46.58	0.00159208143857832\\
46.59	0.00159197859547931\\
46.6	0.00159187570116283\\
46.61	0.00159177275559724\\
46.62	0.00159166975875104\\
46.63	0.00159156671059281\\
46.64	0.00159146361109125\\
46.65	0.00159136046021519\\
46.66	0.00159125725793357\\
46.67	0.00159115400421546\\
46.68	0.00159105069903005\\
46.69	0.00159094734234668\\
46.7	0.00159084393413481\\
46.71	0.00159074047436404\\
46.72	0.00159063696300412\\
46.73	0.00159053340002495\\
46.74	0.00159042978539656\\
46.75	0.00159032611908915\\
46.76	0.00159022240107309\\
46.77	0.00159011863131889\\
46.78	0.00159001480979725\\
46.79	0.00158991093647901\\
46.8	0.00158980701133522\\
46.81	0.0015897030343371\\
46.82	0.00158959900545605\\
46.83	0.00158949492466366\\
46.84	0.00158939079193172\\
46.85	0.00158928660723221\\
46.86	0.00158918237053735\\
46.87	0.00158907808181951\\
46.88	0.00158897374105134\\
46.89	0.00158886934820567\\
46.9	0.00158876490325556\\
46.91	0.00158866040617433\\
46.92	0.00158855585693551\\
46.93	0.00158845125551288\\
46.94	0.00158834660188048\\
46.95	0.0015882418960126\\
46.96	0.00158813713788379\\
46.97	0.00158803232746888\\
46.98	0.00158792746474296\\
46.99	0.0015878225496814\\
47	0.00158771758225989\\
47.01	0.00158761256245439\\
47.02	0.00158750749024117\\
47.03	0.0015874023655968\\
47.04	0.00158729718849818\\
47.05	0.00158719195892254\\
47.06	0.00158708667684744\\
47.07	0.00158698134225076\\
47.08	0.00158687595511076\\
47.09	0.00158677051540606\\
47.1	0.0015866650231156\\
47.11	0.00158655947821875\\
47.12	0.00158645388069524\\
47.13	0.00158634823052517\\
47.14	0.00158624252768908\\
47.15	0.00158613677216791\\
47.16	0.001586030963943\\
47.17	0.00158592510299615\\
47.18	0.00158581918930958\\
47.19	0.00158571322286596\\
47.2	0.00158560720364844\\
47.21	0.00158550113164061\\
47.22	0.00158539500682658\\
47.23	0.00158528882919093\\
47.24	0.00158518259871875\\
47.25	0.00158507631539564\\
47.26	0.00158496997920773\\
47.27	0.0015848635901417\\
47.28	0.00158475714818477\\
47.29	0.00158465065332474\\
47.3	0.00158454410554997\\
47.31	0.00158443750484941\\
47.32	0.00158433085121262\\
47.33	0.0015842241446298\\
47.34	0.00158411738509174\\
47.35	0.00158401057258991\\
47.36	0.00158390370711641\\
47.37	0.00158379678866404\\
47.38	0.00158368981722627\\
47.39	0.00158358279279729\\
47.4	0.00158347571537201\\
47.41	0.00158336858494606\\
47.42	0.00158326140151584\\
47.43	0.00158315416507851\\
47.44	0.00158304687563202\\
47.45	0.00158293953317513\\
47.46	0.00158283213770742\\
47.47	0.00158272468922929\\
47.48	0.00158261718774202\\
47.49	0.00158250963324776\\
47.5	0.00158240202574957\\
47.51	0.00158229436525139\\
47.52	0.00158218665175814\\
47.53	0.00158207888527566\\
47.54	0.00158197106581079\\
47.55	0.00158186319337136\\
47.56	0.00158175526796621\\
47.57	0.00158164728960523\\
47.58	0.0015815392582994\\
47.59	0.00158143117406074\\
47.6	0.00158132303690242\\
47.61	0.00158121484683871\\
47.62	0.00158110660388508\\
47.63	0.00158099830805816\\
47.64	0.00158088995937579\\
47.65	0.00158078155785704\\
47.66	0.00158067310352227\\
47.67	0.0015805645963931\\
47.68	0.00158045603649247\\
47.69	0.00158034742384468\\
47.7	0.0015802387584754\\
47.71	0.00158013004041169\\
47.72	0.00158002126968205\\
47.73	0.00157991244631644\\
47.74	0.00157980357034632\\
47.75	0.00157969464180466\\
47.76	0.00157958566072602\\
47.77	0.00157947662714651\\
47.78	0.00157936754110388\\
47.79	0.00157925840263755\\
47.8	0.00157914921178861\\
47.81	0.00157903996859989\\
47.82	0.00157893067311599\\
47.83	0.0015788213253833\\
47.84	0.00157871192545004\\
47.85	0.00157860247336633\\
47.86	0.00157849296918418\\
47.87	0.00157838341295759\\
47.88	0.00157827380474251\\
47.89	0.00157816414459695\\
47.9	0.00157805443258102\\
47.91	0.00157794466875691\\
47.92	0.001577834853189\\
47.93	0.00157772498594389\\
47.94	0.00157761506709039\\
47.95	0.00157750509669967\\
47.96	0.0015773950748452\\
47.97	0.00157728500160287\\
47.98	0.00157717487705101\\
47.99	0.00157706470127044\\
48	0.00157695447434454\\
48.01	0.00157684419635927\\
48.02	0.00157673386740323\\
48.03	0.00157662348756776\\
48.04	0.00157651305694691\\
48.05	0.00157640257563758\\
48.06	0.00157629204373951\\
48.07	0.00157618146135538\\
48.08	0.00157607082859085\\
48.09	0.00157596014555464\\
48.1	0.00157584941235854\\
48.11	0.00157573862911753\\
48.12	0.00157562779594981\\
48.13	0.00157551691297687\\
48.14	0.00157540598032356\\
48.15	0.00157529499811814\\
48.16	0.00157518396649238\\
48.17	0.0015750728855816\\
48.18	0.00157496175552473\\
48.19	0.0015748505764644\\
48.2	0.00157473934854704\\
48.21	0.00157462807192288\\
48.22	0.00157451674674609\\
48.23	0.00157440537317483\\
48.24	0.00157429395137129\\
48.25	0.00157418248150186\\
48.26	0.00157407096373712\\
48.27	0.00157395939825197\\
48.28	0.00157384778522567\\
48.29	0.00157373612484198\\
48.3	0.00157362441728917\\
48.31	0.00157351266276019\\
48.32	0.0015734008614527\\
48.33	0.00157328901356915\\
48.34	0.00157317711931692\\
48.35	0.00157306517890836\\
48.36	0.00157295319256092\\
48.37	0.00157284116049721\\
48.38	0.00157272908294511\\
48.39	0.00157261696013789\\
48.4	0.00157250479231425\\
48.41	0.00157239257971849\\
48.42	0.00157228032260055\\
48.43	0.00157216802121611\\
48.44	0.00157205567582676\\
48.45	0.00157194328670004\\
48.46	0.00157183085410955\\
48.47	0.00157171837833509\\
48.48	0.00157160585966275\\
48.49	0.00157149329838501\\
48.5	0.00157138069480085\\
48.51	0.00157126804921588\\
48.52	0.00157115536194245\\
48.53	0.00157104263329974\\
48.54	0.00157092986361389\\
48.55	0.00157081705321814\\
48.56	0.00157070420245292\\
48.57	0.00157059131166595\\
48.58	0.00157047838121242\\
48.59	0.00157036541145504\\
48.6	0.00157025240276424\\
48.61	0.00157013935551821\\
48.62	0.00157002627010309\\
48.63	0.00156991314691306\\
48.64	0.00156979998635049\\
48.65	0.00156968678882604\\
48.66	0.0015695735547588\\
48.67	0.00156946028457643\\
48.68	0.00156934697871526\\
48.69	0.00156923363762045\\
48.7	0.00156912026174612\\
48.71	0.00156900685155545\\
48.72	0.00156889340752082\\
48.73	0.00156877993012398\\
48.74	0.00156866641985614\\
48.75	0.00156855287721811\\
48.76	0.00156843930272043\\
48.77	0.00156832569688352\\
48.78	0.0015682120602378\\
48.79	0.0015680983933238\\
48.8	0.00156798469669234\\
48.81	0.00156787097090461\\
48.82	0.00156775721653234\\
48.83	0.00156764343415787\\
48.84	0.00156752962437436\\
48.85	0.00156741578778586\\
48.86	0.00156730192500743\\
48.87	0.00156718803666531\\
48.88	0.00156707412339698\\
48.89	0.00156696018585135\\
48.9	0.00156684622468882\\
48.91	0.00156673224058142\\
48.92	0.00156661823421292\\
48.93	0.00156650420627895\\
48.94	0.00156639015748707\\
48.95	0.00156627608855694\\
48.96	0.00156616200022035\\
48.97	0.00156604789322136\\
48.98	0.00156593376831637\\
48.99	0.00156581962627422\\
49	0.00156570546787628\\
49.01	0.00156559129391649\\
49.02	0.0015654771052015\\
49.03	0.00156536290255063\\
49.04	0.00156524868679605\\
49.05	0.00156513445878275\\
49.06	0.00156502021936859\\
49.07	0.0015649059694244\\
49.08	0.00156479170983392\\
49.09	0.00156467744149388\\
49.1	0.001564563165314\\
49.11	0.00156444888221699\\
49.12	0.00156433459313852\\
49.13	0.00156422029902722\\
49.14	0.00156410600084463\\
49.15	0.0015639916995652\\
49.16	0.00156387739617616\\
49.17	0.00156376309167751\\
49.18	0.00156364878708188\\
49.19	0.0015635344834145\\
49.2	0.001563420181713\\
49.21	0.00156330588302732\\
49.22	0.00156319158841957\\
49.23	0.00156307729896379\\
49.24	0.00156296301574585\\
49.25	0.00156284873986316\\
49.26	0.00156273447242448\\
49.27	0.00156262021454962\\
49.28	0.0015625059673692\\
49.29	0.00156239173202432\\
49.3	0.00156227750966621\\
49.31	0.00156216330145588\\
49.32	0.00156204910856376\\
49.33	0.0015619349321692\\
49.34	0.00156182077346008\\
49.35	0.00156170663363228\\
49.36	0.00156159251388918\\
49.37	0.00156147841544106\\
49.38	0.00156136433950453\\
49.39	0.00156125028730189\\
49.4	0.00156113626006041\\
49.41	0.0015610222590116\\
49.42	0.00156090828539048\\
49.43	0.00156079434043467\\
49.44	0.0015606804253836\\
49.45	0.00156056654147751\\
49.46	0.00156045268995649\\
49.47	0.00156033887205943\\
49.48	0.00156022508902289\\
49.49	0.00156011134207996\\
49.5	0.00155999763245901\\
49.51	0.00155988396138234\\
49.52	0.00155977033006486\\
49.53	0.00155965673971259\\
49.54	0.00155954319152113\\
49.55	0.00155942968667405\\
49.56	0.00155931622634118\\
49.57	0.00155920281167678\\
49.58	0.00155908944381771\\
49.59	0.00155897612388136\\
49.6	0.00155886285296365\\
49.61	0.00155874963213674\\
49.62	0.0015586364624468\\
49.63	0.00155852334491151\\
49.64	0.00155841028051757\\
49.65	0.00155829727021801\\
49.66	0.00155818431492935\\
49.67	0.00155807141552872\\
49.68	0.00155795857285071\\
49.69	0.00155784578768422\\
49.7	0.00155773306076896\\
49.71	0.00155762039279202\\
49.72	0.00155750778438406\\
49.73	0.00155739523611549\\
49.74	0.00155728274849232\\
49.75	0.00155717032195197\\
49.76	0.00155705795685878\\
49.77	0.00155694565349933\\
49.78	0.00155683341207758\\
49.79	0.00155672123270976\\
49.8	0.00155660911541906\\
49.81	0.00155649706013\\
49.82	0.00155638506666268\\
49.83	0.00155627313472661\\
49.84	0.00155616126391444\\
49.85	0.00155604945369527\\
49.86	0.0015559377034077\\
49.87	0.00155582601225265\\
49.88	0.00155571437928575\\
49.89	0.00155560280340949\\
49.9	0.00155549128336496\\
49.91	0.00155537981772329\\
49.92	0.00155526840487668\\
49.93	0.00155515704302904\\
49.94	0.00155504573018629\\
49.95	0.00155493446414612\\
49.96	0.00155482324248748\\
49.97	0.00155471206255946\\
49.98	0.00155460092146982\\
49.99	0.00155448981607295\\
50	0.00155437874295737\\
50.01	0.00155426769843272\\
50.02	0.00155415667851614\\
50.03	0.00155404567891811\\
50.04	0.00155393469502775\\
50.05	0.00155382372189744\\
50.06	0.00155371275422679\\
50.07	0.00155360178634606\\
50.08	0.00155349081219874\\
50.09	0.00155337982532352\\
50.1	0.0015532688188355\\
50.11	0.00155315778540661\\
50.12	0.00155304671724521\\
50.13	0.00155293560607491\\
50.14	0.00155282444311247\\
50.15	0.00155271321904489\\
50.16	0.00155260193152102\\
50.17	0.00155249058046949\\
50.18	0.00155237916581857\\
50.19	0.00155226768749618\\
50.2	0.00155215614542993\\
50.21	0.0015520445395471\\
50.22	0.0015519328697747\\
50.23	0.00155182113603948\\
50.24	0.00155170933826795\\
50.25	0.00155159747638641\\
50.26	0.00155148555032097\\
50.27	0.00155137355999765\\
50.28	0.00155126150534229\\
50.29	0.00155114938628072\\
50.3	0.00155103720273872\\
50.31	0.00155092495464211\\
50.32	0.00155081264191676\\
50.33	0.00155070026448869\\
50.34	0.00155058782228408\\
50.35	0.00155047531522936\\
50.36	0.00155036274325129\\
50.37	0.001550250106277\\
50.38	0.00155013740423407\\
50.39	0.00155002463705062\\
50.4	0.00154991180465541\\
50.41	0.00154979890697791\\
50.42	0.0015496859439484\\
50.43	0.00154957291549809\\
50.44	0.00154945982155921\\
50.45	0.00154934666206517\\
50.46	0.00154923343695061\\
50.47	0.00154912014615163\\
50.48	0.00154900678960586\\
50.49	0.00154889336725209\\
50.5	0.00154877987902903\\
50.51	0.00154866632487528\\
50.52	0.00154855270472938\\
50.53	0.00154843901852973\\
50.54	0.00154832526621466\\
50.55	0.00154821144772241\\
50.56	0.00154809756299111\\
50.57	0.00154798361195879\\
50.58	0.0015478695945634\\
50.59	0.00154775551074277\\
50.6	0.00154764136043465\\
50.61	0.0015475271435767\\
50.62	0.00154741286010645\\
50.63	0.00154729850996136\\
50.64	0.00154718409307878\\
50.65	0.00154706960939596\\
50.66	0.00154695505885006\\
50.67	0.00154684044137811\\
50.68	0.00154672575691708\\
50.69	0.0015466110054038\\
50.7	0.00154649618677504\\
50.71	0.00154638130096743\\
50.72	0.00154626634791752\\
50.73	0.00154615132756174\\
50.74	0.00154603623983643\\
50.75	0.00154592108467782\\
50.76	0.00154580586202204\\
50.77	0.00154569057180512\\
50.78	0.00154557521396296\\
50.79	0.0015454597884314\\
50.8	0.00154534429514611\\
50.81	0.00154522873404272\\
50.82	0.00154511310505671\\
50.83	0.00154499740812346\\
50.84	0.00154488164317825\\
50.85	0.00154476581015625\\
50.86	0.00154464990899251\\
50.87	0.00154453393962199\\
50.88	0.00154441790197952\\
50.89	0.00154430179599983\\
50.9	0.00154418562161754\\
50.91	0.00154406937876714\\
50.92	0.00154395306738304\\
50.93	0.0015438366873995\\
50.94	0.00154372023875071\\
50.95	0.00154360372137069\\
50.96	0.0015434871351934\\
50.97	0.00154337048015265\\
50.98	0.00154325375618215\\
50.99	0.00154313696321548\\
51	0.00154302010118612\\
51.01	0.00154290317002742\\
51.02	0.00154278616967261\\
51.03	0.00154266910005481\\
51.04	0.00154255196110702\\
51.05	0.00154243475276211\\
51.06	0.00154231747495283\\
51.07	0.00154220012761181\\
51.08	0.00154208271067158\\
51.09	0.0015419652240645\\
51.1	0.00154184766772285\\
51.11	0.00154173004157876\\
51.12	0.00154161234556425\\
51.13	0.0015414945796112\\
51.14	0.00154137674365137\\
51.15	0.00154125883761638\\
51.16	0.00154114086143776\\
51.17	0.00154102281504687\\
51.18	0.00154090469837494\\
51.19	0.00154078651135311\\
51.2	0.00154066825391233\\
51.21	0.00154054992598347\\
51.22	0.00154043152749724\\
51.23	0.00154031305838422\\
51.24	0.00154019451857486\\
51.25	0.00154007590799946\\
51.26	0.0015399572265882\\
51.27	0.0015398384742711\\
51.28	0.00153971965097808\\
51.29	0.00153960075663887\\
51.3	0.00153948179118311\\
51.31	0.00153936275454025\\
51.32	0.00153924364663964\\
51.33	0.00153912446741045\\
51.34	0.00153900521678173\\
51.35	0.00153888589468238\\
51.36	0.00153876650104114\\
51.37	0.00153864703578662\\
51.38	0.00153852749884727\\
51.39	0.00153840789015138\\
51.4	0.00153828820962711\\
51.41	0.00153816845720245\\
51.42	0.00153804863280526\\
51.43	0.00153792873636322\\
51.44	0.00153780876780386\\
51.45	0.00153768872705456\\
51.46	0.00153756861404254\\
51.47	0.00153744842869485\\
51.48	0.00153732817093839\\
51.49	0.00153720784069989\\
51.5	0.00153708743790593\\
51.51	0.00153696696248291\\
51.52	0.00153684641435707\\
51.53	0.00153672579345448\\
51.54	0.00153660509970104\\
51.55	0.00153648433302248\\
51.56	0.00153636349334436\\
51.57	0.00153624258059207\\
51.58	0.00153612159469081\\
51.59	0.00153600053556562\\
51.6	0.00153587940314135\\
51.61	0.00153575819734268\\
51.62	0.0015356369180941\\
51.63	0.00153551556531992\\
51.64	0.00153539413894427\\
51.65	0.00153527263889108\\
51.66	0.0015351510650841\\
51.67	0.0015350294174469\\
51.68	0.00153490769590284\\
51.69	0.00153478590037509\\
51.7	0.00153466403078664\\
51.71	0.00153454208706026\\
51.72	0.00153442006911854\\
51.73	0.00153429797688386\\
51.74	0.00153417581027839\\
51.75	0.00153405356922411\\
51.76	0.00153393125364277\\
51.77	0.00153380886345594\\
51.78	0.00153368639858495\\
51.79	0.00153356385895093\\
51.8	0.0015334412444748\\
51.81	0.00153331855507725\\
51.82	0.00153319579067875\\
51.83	0.00153307295119955\\
51.84	0.00153295003655968\\
51.85	0.00153282704667893\\
51.86	0.00153270398147686\\
51.87	0.00153258084087282\\
51.88	0.0015324576247859\\
51.89	0.00153233433313496\\
51.9	0.00153221096583861\\
51.91	0.00153208752281524\\
51.92	0.00153196400398297\\
51.93	0.00153184040925968\\
51.94	0.00153171673856301\\
51.95	0.00153159299181032\\
51.96	0.00153146916891873\\
51.97	0.00153134526980509\\
51.98	0.00153122129438601\\
51.99	0.0015310972425778\\
52	0.00153097311429651\\
52.01	0.00153084890945793\\
52.02	0.00153072462797756\\
52.03	0.00153060026977063\\
52.04	0.00153047583475207\\
52.05	0.00153035132283654\\
52.06	0.00153022673393841\\
52.07	0.00153010206797174\\
52.08	0.00152997732485031\\
52.09	0.00152985250448757\\
52.1	0.00152972760679671\\
52.11	0.00152960263169056\\
52.12	0.00152947757908168\\
52.13	0.00152935244888229\\
52.14	0.00152922724100428\\
52.15	0.00152910195535925\\
52.16	0.00152897659185842\\
52.17	0.00152885115041272\\
52.18	0.00152872563093271\\
52.19	0.00152860003332863\\
52.2	0.00152847435751036\\
52.21	0.00152834860338742\\
52.22	0.00152822277086899\\
52.23	0.00152809685986388\\
52.24	0.00152797087028052\\
52.25	0.00152784480202699\\
52.26	0.00152771865501097\\
52.27	0.00152759242913976\\
52.28	0.0015274661243203\\
52.29	0.0015273397404591\\
52.3	0.00152721327746229\\
52.31	0.00152708673523557\\
52.32	0.00152696011368428\\
52.33	0.00152683341271328\\
52.34	0.00152670663222706\\
52.35	0.00152657977212964\\
52.36	0.00152645283232463\\
52.37	0.00152632581271519\\
52.38	0.00152619871320403\\
52.39	0.00152607153369339\\
52.4	0.00152594427408509\\
52.41	0.00152581693428042\\
52.42	0.00152568951418026\\
52.43	0.00152556201368494\\
52.44	0.00152543443269436\\
52.45	0.00152530677110787\\
52.46	0.00152517902882435\\
52.47	0.00152505120574215\\
52.48	0.00152492330175909\\
52.49	0.00152479531677249\\
52.5	0.00152466725067909\\
52.51	0.00152453910337512\\
52.52	0.00152441087475622\\
52.53	0.0015242825647175\\
52.54	0.00152415417315347\\
52.55	0.00152402569995807\\
52.56	0.00152389714502464\\
52.57	0.00152376850824592\\
52.58	0.00152363978951405\\
52.59	0.00152351098872053\\
52.6	0.00152338210575625\\
52.61	0.00152325314051142\\
52.62	0.00152312409287565\\
52.63	0.00152299496273783\\
52.64	0.00152286574998623\\
52.65	0.0015227364545084\\
52.66	0.00152260707619119\\
52.67	0.00152247761492075\\
52.68	0.00152234807058252\\
52.69	0.0015222184430612\\
52.7	0.00152208873224072\\
52.71	0.0015219589380043\\
52.72	0.00152182906023434\\
52.73	0.00152169909881248\\
52.74	0.00152156905361955\\
52.75	0.00152143892453559\\
52.76	0.00152130871143979\\
52.77	0.00152117841421051\\
52.78	0.00152104803272526\\
52.79	0.00152091756686065\\
52.8	0.00152078701649244\\
52.81	0.00152065638149548\\
52.82	0.00152052566174369\\
52.83	0.00152039485711006\\
52.84	0.00152026396746664\\
52.85	0.00152013299268451\\
52.86	0.00152000193263376\\
52.87	0.00151987078718347\\
52.88	0.00151973955620171\\
52.89	0.00151960823955552\\
52.9	0.00151947683711087\\
52.91	0.00151934534873264\\
52.92	0.00151921377428464\\
52.93	0.00151908211362955\\
52.94	0.00151895036662892\\
52.95	0.00151881853314312\\
52.96	0.00151868661303137\\
52.97	0.00151855460615166\\
52.98	0.00151842251236078\\
52.99	0.00151829033151426\\
53	0.00151815806346637\\
53.01	0.00151802570807007\\
53.02	0.00151789326517701\\
53.03	0.00151776073463751\\
53.04	0.0015176281163005\\
53.05	0.00151749541001353\\
53.06	0.00151736261562273\\
53.07	0.00151722973297277\\
53.08	0.00151709676190686\\
53.09	0.00151696370226669\\
53.1	0.00151683055389243\\
53.11	0.00151669731662269\\
53.12	0.00151656399029449\\
53.13	0.0015164305747432\\
53.14	0.00151629706980258\\
53.15	0.00151616347530467\\
53.16	0.0015160297910798\\
53.17	0.00151589601695656\\
53.18	0.00151576215276174\\
53.19	0.00151562819832032\\
53.2	0.00151549415345541\\
53.21	0.00151536001798823\\
53.22	0.00151522579173808\\
53.23	0.00151509147452228\\
53.24	0.00151495706615615\\
53.25	0.00151482256645295\\
53.26	0.00151468797522388\\
53.27	0.00151455329227797\\
53.28	0.00151441851742211\\
53.29	0.00151428365046096\\
53.3	0.00151414869119692\\
53.31	0.0015140136394301\\
53.32	0.00151387849495824\\
53.33	0.00151374325757668\\
53.34	0.00151360792707834\\
53.35	0.0015134725032536\\
53.36	0.00151333698589034\\
53.37	0.0015132013747738\\
53.38	0.0015130656696866\\
53.39	0.00151292987040864\\
53.4	0.00151279397671705\\
53.41	0.00151265798838616\\
53.42	0.00151252190518742\\
53.43	0.00151238572688933\\
53.44	0.00151224945325743\\
53.45	0.00151211308405417\\
53.46	0.00151197661903889\\
53.47	0.00151184005796777\\
53.48	0.00151170340059371\\
53.49	0.00151156664666631\\
53.5	0.00151142979593178\\
53.51	0.00151129284813288\\
53.52	0.00151115580300884\\
53.53	0.00151101866029528\\
53.54	0.00151088141972416\\
53.55	0.00151074408102367\\
53.56	0.00151060664391818\\
53.57	0.00151046910812816\\
53.58	0.00151033147337005\\
53.59	0.00151019373935625\\
53.6	0.00151005590579497\\
53.61	0.00150991797239018\\
53.62	0.00150977993884151\\
53.63	0.00150964180484413\\
53.64	0.00150950357008873\\
53.65	0.00150936523426133\\
53.66	0.00150922679704326\\
53.67	0.00150908825811101\\
53.68	0.00150894961713616\\
53.69	0.00150881087378522\\
53.7	0.00150867202771962\\
53.71	0.00150853307859549\\
53.72	0.00150839402606364\\
53.73	0.00150825486976938\\
53.74	0.00150811560935243\\
53.75	0.0015079762444468\\
53.76	0.00150783677468065\\
53.77	0.00150769719967622\\
53.78	0.00150755751904959\\
53.79	0.00150741773241068\\
53.8	0.00150727783936303\\
53.81	0.00150713783950366\\
53.82	0.00150699773242299\\
53.83	0.00150685751770464\\
53.84	0.00150671719492531\\
53.85	0.00150657676365459\\
53.86	0.00150643622345489\\
53.87	0.00150629557388119\\
53.88	0.00150615481448092\\
53.89	0.00150601394479379\\
53.9	0.00150587296435165\\
53.91	0.00150573187267826\\
53.92	0.00150559066928914\\
53.93	0.00150544935369141\\
53.94	0.00150530792538357\\
53.95	0.00150516638385535\\
53.96	0.00150502472858745\\
53.97	0.00150488295905143\\
53.98	0.00150474107470942\\
53.99	0.001504599075014\\
54	0.00150445695940788\\
54.01	0.0015043147273238\\
54.02	0.00150417237818422\\
54.03	0.00150402991140112\\
54.04	0.00150388732637579\\
54.05	0.00150374462249855\\
54.06	0.00150360179914855\\
54.07	0.00150345885569348\\
54.08	0.00150331579148934\\
54.09	0.00150317260588017\\
54.1	0.00150302929819777\\
54.11	0.00150288586776147\\
54.12	0.0015027423138778\\
54.13	0.00150259863584022\\
54.14	0.00150245483292884\\
54.15	0.0015023109044101\\
54.16	0.00150216684953649\\
54.17	0.00150202266754619\\
54.18	0.00150187835766281\\
54.19	0.00150173391909501\\
54.2	0.00150158935103618\\
54.21	0.0015014446526641\\
54.22	0.00150129982314057\\
54.23	0.00150115486161109\\
54.24	0.00150100976720443\\
54.25	0.0015008645390323\\
54.26	0.00150071917618896\\
54.27	0.00150057367775079\\
54.28	0.0015004280427759\\
54.29	0.00150028227030374\\
54.3	0.00150013635935465\\
54.31	0.00149999030892941\\
54.32	0.00149984411800886\\
54.33	0.00149969778555334\\
54.34	0.00149955131050234\\
54.35	0.00149940469177393\\
54.36	0.00149925792826433\\
54.37	0.00149911101884736\\
54.38	0.00149896396237399\\
54.39	0.00149881675767177\\
54.4	0.00149866940354428\\
54.41	0.00149852189877066\\
54.42	0.00149837424210495\\
54.43	0.00149822643227559\\
54.44	0.00149807846798478\\
54.45	0.00149793034790793\\
54.46	0.00149778207069298\\
54.47	0.00149763363495984\\
54.48	0.00149748503929967\\
54.49	0.00149733628227429\\
54.5	0.00149718736241543\\
54.51	0.00149703827822412\\
54.52	0.00149688902816991\\
54.53	0.00149673961069016\\
54.54	0.00149659002418935\\
54.55	0.00149644026703823\\
54.56	0.00149629033757312\\
54.57	0.00149614023409507\\
54.58	0.00149598995486907\\
54.59	0.00149583949812316\\
54.6	0.00149568886204766\\
54.61	0.00149553804479423\\
54.62	0.001495387044475\\
54.63	0.00149523585916164\\
54.64	0.00149508448688444\\
54.65	0.00149493292563133\\
54.66	0.0014947811733469\\
54.67	0.00149462922793141\\
54.68	0.00149447708723974\\
54.69	0.00149432474907977\\
54.7	0.00149417221121002\\
54.71	0.00149401947133839\\
54.72	0.00149386652712092\\
54.73	0.00149371337616051\\
54.74	0.00149356001600559\\
54.75	0.00149340644414876\\
54.76	0.00149325265802543\\
54.77	0.00149309865501236\\
54.78	0.00149294443242623\\
54.79	0.0014927899875221\\
54.8	0.0014926353174919\\
54.81	0.00149248041946284\\
54.82	0.0014923252904958\\
54.83	0.00149216992758363\\
54.84	0.0014920143276495\\
54.85	0.00149185848754508\\
54.86	0.00149170240404882\\
54.87	0.00149154607386403\\
54.88	0.00149138949361701\\
54.89	0.00149123265985516\\
54.9	0.00149107556904486\\
54.91	0.00149091821756959\\
54.92	0.00149076060172767\\
54.93	0.0014906027177302\\
54.94	0.00149044456169882\\
54.95	0.00149028612966342\\
54.96	0.00149012741755982\\
54.97	0.00148996842122739\\
54.98	0.00148980913640656\\
54.99	0.00148964955873632\\
55	0.00148948968375162\\
55.01	0.00148932950688072\\
55.02	0.00148916902344243\\
55.03	0.00148900822864334\\
55.04	0.00148884711757492\\
55.05	0.00148868568521059\\
55.06	0.00148852392640264\\
55.07	0.00148836183587917\\
55.08	0.00148819940824085\\
55.09	0.00148803663795765\\
55.1	0.00148787351936545\\
55.11	0.00148771004666263\\
55.12	0.00148754621390644\\
55.13	0.0014873820150094\\
55.14	0.00148721744373553\\
55.15	0.00148705249369649\\
55.16	0.00148688715834766\\
55.17	0.00148672143098403\\
55.18	0.00148655530473607\\
55.19	0.00148638877256542\\
55.2	0.00148622182726048\\
55.21	0.00148605446143194\\
55.22	0.00148588666750807\\
55.23	0.00148571843773002\\
55.24	0.00148554976414685\\
55.25	0.00148538063861057\\
55.26	0.00148521105277091\\
55.27	0.00148504099807007\\
55.28	0.00148487046573721\\
55.29	0.00148469944678291\\
55.3	0.00148452793199334\\
55.31	0.00148435591192442\\
55.32	0.0014841833768957\\
55.33	0.00148401031698414\\
55.34	0.0014838367220177\\
55.35	0.00148366258156874\\
55.36	0.00148348788494727\\
55.37	0.00148331262119399\\
55.38	0.00148313677907315\\
55.39	0.0014829603470652\\
55.4	0.00148278331335924\\
55.41	0.00148260566584532\\
55.42	0.0014824273921064\\
55.43	0.00148224847941025\\
55.44	0.00148206891470096\\
55.45	0.00148188868459036\\
55.46	0.00148170777534912\\
55.47	0.00148152617289762\\
55.48	0.00148134386279656\\
55.49	0.00148116083023735\\
55.5	0.00148097706003219\\
55.51	0.00148079253660387\\
55.52	0.00148060724397533\\
55.53	0.00148042116575892\\
55.54	0.00148023428514531\\
55.55	0.00148004658489214\\
55.56	0.00147985804731235\\
55.57	0.00147966865426217\\
55.58	0.00147947838712878\\
55.59	0.00147928722681764\\
55.6	0.00147909515373943\\
55.61	0.00147890214779665\\
55.62	0.00147870818836985\\
55.63	0.00147851325430348\\
55.64	0.00147831732389129\\
55.65	0.0014781203748614\\
55.66	0.00147792238436087\\
55.67	0.00147772332893991\\
55.68	0.00147752318453559\\
55.69	0.00147732192645514\\
55.7	0.00147711952935873\\
55.71	0.00147691596724178\\
55.72	0.0014767112134168\\
55.73	0.00147650524049468\\
55.74	0.00147629802036552\\
55.75	0.00147608952417876\\
55.76	0.00147587972232295\\
55.77	0.00147566858440482\\
55.78	0.00147545607922778\\
55.79	0.00147524217476981\\
55.8	0.00147502683816077\\
55.81	0.001474810035659\\
55.82	0.00147459173262724\\
55.83	0.00147437189350801\\
55.84	0.00147415048179811\\
55.85	0.00147392746002246\\
55.86	0.00147370278970728\\
55.87	0.00147347643135235\\
55.88	0.00147324834440262\\
55.89	0.00147301848721893\\
55.9	0.00147278681704786\\
55.91	0.00147255328999087\\
55.92	0.00147231786097239\\
55.93	0.00147208048370705\\
55.94	0.00147184111066604\\
55.95	0.00147159969304242\\
55.96	0.00147135618071546\\
55.97	0.00147111052221397\\
55.98	0.00147086266467856\\
55.99	0.00147061255382283\\
56	0.00147036013389345\\
56.01	0.00147010534762902\\
56.02	0.00146984813621789\\
56.03	0.0014695884392546\\
56.04	0.00146932619469515\\
56.05	0.00146906133881103\\
56.06	0.00146879380614176\\
56.07	0.00146852352944626\\
56.08	0.00146825043965262\\
56.09	0.00146797446580659\\
56.1	0.0014676955350184\\
56.11	0.00146741357240825\\
56.12	0.00146712850105001\\
56.13	0.00146684024191341\\
56.14	0.00146654871380454\\
56.15	0.00146625383330456\\
56.16	0.00146595551470672\\
56.17	0.00146565366995145\\
56.18	0.00146534820855965\\
56.19	0.00146503903756395\\
56.2	0.00146472606143803\\
56.21	0.00146440918202388\\
56.22	0.00146408829845687\\
56.23	0.0014637633070887\\
56.24	0.00146343410140807\\
56.25	0.00146310057195903\\
56.26	0.00146276260625694\\
56.27	0.00146242162683012\\
56.28	0.00146208051637791\\
56.29	0.0014617392798962\\
56.3	0.00146139792262132\\
56.31	0.00146105645003917\\
56.32	0.00146071486789444\\
56.33	0.00146037318220032\\
56.34	0.00146003139924853\\
56.35	0.00145968952561955\\
56.36	0.00145934756819337\\
56.37	0.00145900553416049\\
56.38	0.00145866343103336\\
56.39	0.00145832126665817\\
56.4	0.00145797904922703\\
56.41	0.0014576367872906\\
56.42	0.00145729448977115\\
56.43	0.00145695216597599\\
56.44	0.00145660982561143\\
56.45	0.00145626747879717\\
56.46	0.00145592513608118\\
56.47	0.00145558280845508\\
56.48	0.00145524050737004\\
56.49	0.00145489824475322\\
56.5	0.0014545560330247\\
56.51	0.00145421388511507\\
56.52	0.00145387181448352\\
56.53	0.00145352983513661\\
56.54	0.00145318796164756\\
56.55	0.00145284620917631\\
56.56	0.00145250459349013\\
56.57	0.00145216313098498\\
56.58	0.00145182183870755\\
56.59	0.00145148073437805\\
56.6	0.00145113983641371\\
56.61	0.00145079916395309\\
56.62	0.00145045873688123\\
56.63	0.0014501185758555\\
56.64	0.00144977870233244\\
56.65	0.00144943913859541\\
56.66	0.00144909990778314\\
56.67	0.00144876103391925\\
56.68	0.00144842254194272\\
56.69	0.00144808445773935\\
56.7	0.00144774680817428\\
56.71	0.0014474096211255\\
56.72	0.00144707292551855\\
56.73	0.00144673675136225\\
56.74	0.00144640112978567\\
56.75	0.0014460660930763\\
56.76	0.0014457316747194\\
56.77	0.00144539790943864\\
56.78	0.00144506483323813\\
56.79	0.00144473248344574\\
56.8	0.00144440089875784\\
56.81	0.0014440701192855\\
56.82	0.00144374018660216\\
56.83	0.00144341114379286\\
56.84	0.00144308303550504\\
56.85	0.00144275590800101\\
56.86	0.00144242980921207\\
56.87	0.00144210478879437\\
56.88	0.00144178089818663\\
56.89	0.00144145819066962\\
56.9	0.00144113672142762\\
56.91	0.00144081654761182\\
56.92	0.00144049772840577\\
56.93	0.00144018032509289\\
56.94	0.00143986300110103\\
56.95	0.00143954553854252\\
56.96	0.00143922793733861\\
56.97	0.0014389101974105\\
56.98	0.0014385923186793\\
56.99	0.00143827430106611\\
57	0.00143795614449191\\
57.01	0.00143763784887765\\
57.02	0.0014373194141442\\
57.03	0.00143700084021238\\
57.04	0.00143668212700294\\
57.05	0.00143636327443654\\
57.06	0.00143604428243383\\
57.07	0.00143572515091535\\
57.08	0.00143540587980159\\
57.09	0.00143508646901299\\
57.1	0.0014347669184699\\
57.11	0.00143444722809262\\
57.12	0.0014341273978014\\
57.13	0.0014338074275164\\
57.14	0.00143348731715772\\
57.15	0.00143316706664542\\
57.16	0.00143284667589945\\
57.17	0.00143252614483975\\
57.18	0.00143220547338616\\
57.19	0.00143188466145846\\
57.2	0.00143156370897637\\
57.21	0.00143124261585955\\
57.22	0.00143092138202759\\
57.23	0.00143060000740002\\
57.24	0.00143027849189629\\
57.25	0.00142995683543581\\
57.26	0.00142963503793791\\
57.27	0.00142931309932185\\
57.28	0.00142899101950685\\
57.29	0.00142866879841203\\
57.3	0.00142834643595648\\
57.31	0.0014280239320592\\
57.32	0.00142770128663914\\
57.33	0.00142737849961519\\
57.34	0.00142705557090615\\
57.35	0.00142673250043078\\
57.36	0.00142640928810777\\
57.37	0.00142608593385574\\
57.38	0.00142576243759325\\
57.39	0.00142543879923878\\
57.4	0.00142511501871078\\
57.41	0.00142479109592761\\
57.42	0.00142446703080756\\
57.43	0.00142414282326888\\
57.44	0.00142381847322972\\
57.45	0.0014234939806082\\
57.46	0.00142316934532236\\
57.47	0.00142284456729018\\
57.48	0.00142251964642955\\
57.49	0.00142219458265834\\
57.5	0.0014218693758943\\
57.51	0.00142154402605517\\
57.52	0.00142121853305859\\
57.53	0.00142089289682214\\
57.54	0.00142056711726334\\
57.55	0.00142024119429964\\
57.56	0.00141991512784843\\
57.57	0.00141958891782703\\
57.58	0.00141926256415269\\
57.59	0.0014189360667426\\
57.6	0.00141860942551388\\
57.61	0.00141828264038359\\
57.62	0.00141795571126872\\
57.63	0.00141762863808619\\
57.64	0.00141730142075287\\
57.65	0.00141697405918553\\
57.66	0.00141664655330092\\
57.67	0.00141631890301567\\
57.68	0.0014159911082464\\
57.69	0.00141566316890961\\
57.7	0.00141533508492177\\
57.71	0.00141500685619928\\
57.72	0.00141467848265845\\
57.73	0.00141434996421554\\
57.74	0.00141402130078674\\
57.75	0.00141369249228818\\
57.76	0.00141336353863591\\
57.77	0.00141303443974592\\
57.78	0.00141270519553413\\
57.79	0.0014123758059164\\
57.8	0.00141204627080851\\
57.81	0.00141171659012618\\
57.82	0.00141138676378505\\
57.83	0.00141105679170073\\
57.84	0.00141072667378872\\
57.85	0.00141039640996446\\
57.86	0.00141006600014334\\
57.87	0.00140973544424068\\
57.88	0.0014094047421717\\
57.89	0.0014090738938516\\
57.9	0.00140874289919547\\
57.91	0.00140841175811836\\
57.92	0.00140808047053523\\
57.93	0.00140774903636099\\
57.94	0.00140741745551047\\
57.95	0.00140708572789844\\
57.96	0.0014067538534396\\
57.97	0.00140642183204856\\
57.98	0.00140608966363989\\
57.99	0.00140575734812807\\
58	0.00140542488542754\\
58.01	0.00140509227545263\\
58.02	0.00140475951811764\\
58.03	0.00140442661333677\\
58.04	0.00140409356102417\\
58.05	0.00140376036109391\\
58.06	0.00140342701346\\
58.07	0.00140309351803637\\
58.08	0.00140275987473689\\
58.09	0.00140242608347535\\
58.1	0.00140209214416548\\
58.11	0.00140175805672094\\
58.12	0.00140142382105531\\
58.13	0.00140108943708211\\
58.14	0.00140075490471479\\
58.15	0.00140042022386672\\
58.16	0.0014000853944512\\
58.17	0.00139975041638147\\
58.18	0.0013994152895707\\
58.19	0.00139908001393198\\
58.2	0.00139874458937834\\
58.21	0.00139840901582272\\
58.22	0.00139807329317801\\
58.23	0.00139773742135702\\
58.24	0.0013974014002725\\
58.25	0.00139706522983711\\
58.26	0.00139672890996344\\
58.27	0.00139639244056404\\
58.28	0.00139605582155135\\
58.29	0.00139571905283776\\
58.3	0.00139538213433558\\
58.31	0.00139504506595705\\
58.32	0.00139470784761436\\
58.33	0.00139437047921958\\
58.34	0.00139403296068476\\
58.35	0.00139369529192184\\
58.36	0.00139335747284272\\
58.37	0.00139301950335919\\
58.38	0.00139268138338301\\
58.39	0.00139234311282583\\
58.4	0.00139200469159926\\
58.41	0.00139166611961482\\
58.42	0.00139132739678396\\
58.43	0.00139098852301805\\
58.44	0.00139064949822841\\
58.45	0.00139031032232626\\
58.46	0.00138997099522276\\
58.47	0.00138963151682901\\
58.48	0.00138929188705601\\
58.49	0.00138895210581472\\
58.5	0.001388612173016\\
58.51	0.00138827208857063\\
58.52	0.00138793185238936\\
58.53	0.00138759146438282\\
58.54	0.00138725092446158\\
58.55	0.00138691023253617\\
58.56	0.00138656938851699\\
58.57	0.00138622839231442\\
58.58	0.00138588724383873\\
58.59	0.00138554594300012\\
58.6	0.00138520448970873\\
58.61	0.00138486288387463\\
58.62	0.0013845211254078\\
58.63	0.00138417921421814\\
58.64	0.0013838371502155\\
58.65	0.00138349493330964\\
58.66	0.00138315256341026\\
58.67	0.00138281004042695\\
58.68	0.00138246736426927\\
58.69	0.00138212453484668\\
58.7	0.00138178155206858\\
58.71	0.00138143841584427\\
58.72	0.00138109512608299\\
58.73	0.00138075168269392\\
58.74	0.00138040808558615\\
58.75	0.00138006433466868\\
58.76	0.00137972042985047\\
58.77	0.00137937637104037\\
58.78	0.00137903215814718\\
58.79	0.00137868779107962\\
58.8	0.00137834326974631\\
58.81	0.00137799859405583\\
58.82	0.00137765376391665\\
58.83	0.0013773087792372\\
58.84	0.00137696363992581\\
58.85	0.00137661834589073\\
58.86	0.00137627289704016\\
58.87	0.00137592729328219\\
58.88	0.00137558153452487\\
58.89	0.00137523562067614\\
58.9	0.00137488955164388\\
58.91	0.0013745433273359\\
58.92	0.00137419694765992\\
58.93	0.00137385041252358\\
58.94	0.00137350372183446\\
58.95	0.00137315687550006\\
58.96	0.0013728098734278\\
58.97	0.001372462715525\\
58.98	0.00137211540169894\\
58.99	0.00137176793185681\\
59	0.00137142030590571\\
59.01	0.00137107252375266\\
59.02	0.00137072458530464\\
59.03	0.00137037649046851\\
59.04	0.00137002823915107\\
59.05	0.00136967983125904\\
59.06	0.00136933126669906\\
59.07	0.00136898254537771\\
59.08	0.00136863366720147\\
59.09	0.00136828463207674\\
59.1	0.00136793543990986\\
59.11	0.00136758609060707\\
59.12	0.00136723658407456\\
59.13	0.00136688692021841\\
59.14	0.00136653709894465\\
59.15	0.00136618712015921\\
59.16	0.00136583698376794\\
59.17	0.00136548668967664\\
59.18	0.00136513623779099\\
59.19	0.00136478562801663\\
59.2	0.0013644348602591\\
59.21	0.00136408393442385\\
59.22	0.00136373285041628\\
59.23	0.00136338160814168\\
59.24	0.00136303020750528\\
59.25	0.00136267864841224\\
59.26	0.00136232693076761\\
59.27	0.00136197505447637\\
59.28	0.00136162301944345\\
59.29	0.00136127082557365\\
59.3	0.00136091847277173\\
59.31	0.00136056596094236\\
59.32	0.00136021328999011\\
59.33	0.00135986045981949\\
59.34	0.00135950747033494\\
59.35	0.00135915432144078\\
59.36	0.00135880101304129\\
59.37	0.00135844754504064\\
59.38	0.00135809391734294\\
59.39	0.00135774012985221\\
59.4	0.00135738618247238\\
59.41	0.00135703207510732\\
59.42	0.0013566778076608\\
59.43	0.00135632338003652\\
59.44	0.0013559687921381\\
59.45	0.00135561404386905\\
59.46	0.00135525913513285\\
59.47	0.00135490406583284\\
59.48	0.00135454883587233\\
59.49	0.00135419344515452\\
59.5	0.00135383789358252\\
59.51	0.00135348218105938\\
59.52	0.00135312630748807\\
59.53	0.00135277027277145\\
59.54	0.00135241407681232\\
59.55	0.0013520577195134\\
59.56	0.0013517012007773\\
59.57	0.00135134452050658\\
59.58	0.0013509876786037\\
59.59	0.00135063067497104\\
59.6	0.0013502735095109\\
59.61	0.00134991618212549\\
59.62	0.00134955869271695\\
59.63	0.00134920104118731\\
59.64	0.00134884322743855\\
59.65	0.00134848525137255\\
59.66	0.00134812711289109\\
59.67	0.00134776881189591\\
59.68	0.00134741034828861\\
59.69	0.00134705172197076\\
59.7	0.00134669293284381\\
59.71	0.00134633398080914\\
59.72	0.00134597486576804\\
59.73	0.00134561558762173\\
59.74	0.00134525614627131\\
59.75	0.00134489654161784\\
59.76	0.00134453677356227\\
59.77	0.00134417684200547\\
59.78	0.00134381674684823\\
59.79	0.00134345648799124\\
59.8	0.00134309606533512\\
59.81	0.00134273547878041\\
59.82	0.00134237472822754\\
59.83	0.00134201381357687\\
59.84	0.00134165273472869\\
59.85	0.00134129149158317\\
59.86	0.00134093008404042\\
59.87	0.00134056851200046\\
59.88	0.00134020677536322\\
59.89	0.00133984487402855\\
59.9	0.00133948280789619\\
59.91	0.00133912057686583\\
59.92	0.00133875818083706\\
59.93	0.00133839561970936\\
59.94	0.00133803289338216\\
59.95	0.00133767000175478\\
59.96	0.00133730694472647\\
59.97	0.00133694372219637\\
59.98	0.00133658033406355\\
59.99	0.00133621678022699\\
60	0.00133585306058558\\
60.01	0.00133548917503814\\
60.02	0.00133512512348336\\
60.03	0.00133476090581989\\
60.04	0.00133439652194627\\
60.05	0.00133403197176095\\
60.06	0.00133366725516229\\
60.07	0.00133330237204858\\
60.08	0.00133293732231802\\
60.09	0.00133257210586868\\
60.1	0.0013322067225986\\
60.11	0.00133184117240571\\
60.12	0.00133147545518783\\
60.13	0.00133110957084272\\
60.14	0.00133074351926803\\
60.15	0.00133037730036134\\
60.16	0.00133001091402013\\
60.17	0.00132964436014179\\
60.18	0.00132927763862364\\
60.19	0.00132891074936287\\
60.2	0.00132854369225662\\
60.21	0.00132817646720193\\
60.22	0.00132780907409574\\
60.23	0.0013274415128349\\
60.24	0.00132707378331619\\
60.25	0.00132670588543629\\
60.26	0.00132633781909177\\
60.27	0.00132596958417914\\
60.28	0.0013256011805948\\
60.29	0.00132523260823508\\
60.3	0.00132486386699618\\
60.31	0.00132449495677426\\
60.32	0.00132412587746536\\
60.33	0.00132375662896542\\
60.34	0.00132338721117032\\
60.35	0.00132301762397582\\
60.36	0.00132264786727761\\
60.37	0.00132227794097126\\
60.38	0.00132190784495229\\
60.39	0.0013215375791161\\
60.4	0.001321167143358\\
60.41	0.00132079653757321\\
60.42	0.00132042576165687\\
60.43	0.00132005481550402\\
60.44	0.0013196836990096\\
60.45	0.00131931241206847\\
60.46	0.00131894095457539\\
60.47	0.00131856932642502\\
60.48	0.00131819752751196\\
60.49	0.00131782555773068\\
60.5	0.00131745341697558\\
60.51	0.00131708110514094\\
60.52	0.00131670862212099\\
60.53	0.00131633596780984\\
60.54	0.00131596314210149\\
60.55	0.00131559014488988\\
60.56	0.00131521697606884\\
60.57	0.00131484363553212\\
60.58	0.00131447012317335\\
60.59	0.00131409643888608\\
60.6	0.00131372258256379\\
60.61	0.00131334855409982\\
60.62	0.00131297435338745\\
60.63	0.00131259998031985\\
60.64	0.0013122254347901\\
60.65	0.00131185071669119\\
60.66	0.00131147582591601\\
60.67	0.00131110076235736\\
60.68	0.00131072552590793\\
60.69	0.00131035011646033\\
60.7	0.00130997453390709\\
60.71	0.0013095987781406\\
60.72	0.00130922284905318\\
60.73	0.00130884674653707\\
60.74	0.0013084704704844\\
60.75	0.00130809402078719\\
60.76	0.00130771739733739\\
60.77	0.00130734060002683\\
60.78	0.00130696362874725\\
60.79	0.00130658648339032\\
60.8	0.00130620916384757\\
60.81	0.00130583167001047\\
60.82	0.00130545400177038\\
60.83	0.00130507615901854\\
60.84	0.00130469814164614\\
60.85	0.00130431994954424\\
60.86	0.00130394158260381\\
60.87	0.00130356304071571\\
60.88	0.00130318432377075\\
60.89	0.00130280543165957\\
60.9	0.00130242636427277\\
60.91	0.00130204712150084\\
60.92	0.00130166770323414\\
60.93	0.00130128810936299\\
60.94	0.00130090833977754\\
60.95	0.00130052839436792\\
60.96	0.00130014827302409\\
60.97	0.00129976797563596\\
60.98	0.00129938750209331\\
60.99	0.00129900685228586\\
61	0.00129862602610318\\
61.01	0.00129824502343478\\
61.02	0.00129786384417007\\
61.03	0.00129748248819832\\
61.04	0.00129710095540875\\
61.05	0.00129671924569045\\
61.06	0.00129633735893243\\
61.07	0.00129595529502358\\
61.08	0.0012955730538527\\
61.09	0.00129519063530849\\
61.1	0.00129480803927956\\
61.11	0.00129442526565439\\
61.12	0.0012940423143214\\
61.13	0.00129365918516887\\
61.14	0.001293275878085\\
61.15	0.00129289239295789\\
61.16	0.00129250872967552\\
61.17	0.00129212488812581\\
61.18	0.00129174086819653\\
61.19	0.00129135666977538\\
61.2	0.00129097229274995\\
61.21	0.00129058773700771\\
61.22	0.00129020300243607\\
61.23	0.00128981808892229\\
61.24	0.00128943299635357\\
61.25	0.00128904772461697\\
61.26	0.00128866227359948\\
61.27	0.00128827664318796\\
61.28	0.00128789083326919\\
61.29	0.00128750484372984\\
61.3	0.00128711867445647\\
61.31	0.00128673232533553\\
61.32	0.0012863457962534\\
61.33	0.00128595908709631\\
61.34	0.00128557219775044\\
61.35	0.00128518512810181\\
61.36	0.00128479787803637\\
61.37	0.00128441044743996\\
61.38	0.00128402283619832\\
61.39	0.00128363504419708\\
61.4	0.00128324707132177\\
61.41	0.00128285891745781\\
61.42	0.0012824705824905\\
61.43	0.00128208206630507\\
61.44	0.00128169336878662\\
61.45	0.00128130448982016\\
61.46	0.00128091542929058\\
61.47	0.00128052618708267\\
61.48	0.00128013676308112\\
61.49	0.00127974715717051\\
61.5	0.0012793573692353\\
61.51	0.00127896739915988\\
61.52	0.0012785772468285\\
61.53	0.00127818691212531\\
61.54	0.00127779639493437\\
61.55	0.00127740569513962\\
61.56	0.00127701481262489\\
61.57	0.00127662374727392\\
61.58	0.00127623249897032\\
61.59	0.00127584106759761\\
61.6	0.0012754494530392\\
61.61	0.00127505765517839\\
61.62	0.00127466567389836\\
61.63	0.00127427350908221\\
61.64	0.00127388116061291\\
61.65	0.00127348862837333\\
61.66	0.00127309591224623\\
61.67	0.00127270301211426\\
61.68	0.00127230992785996\\
61.69	0.00127191665936578\\
61.7	0.00127152320651403\\
61.71	0.00127112956918693\\
61.72	0.0012707357472666\\
61.73	0.00127034174063502\\
61.74	0.0012699475491741\\
61.75	0.0012695531727656\\
61.76	0.0012691586112912\\
61.77	0.00126876386463246\\
61.78	0.00126836893267083\\
61.79	0.00126797381528764\\
61.8	0.00126757851236414\\
61.81	0.00126718302378144\\
61.82	0.00126678734942054\\
61.83	0.00126639148916235\\
61.84	0.00126599544288765\\
61.85	0.00126559921047712\\
61.86	0.00126520279181132\\
61.87	0.00126480618677071\\
61.88	0.00126440939523563\\
61.89	0.0012640124170863\\
61.9	0.00126361525220286\\
61.91	0.00126321790046529\\
61.92	0.0012628203617535\\
61.93	0.00126242263594727\\
61.94	0.00126202472292627\\
61.95	0.00126162662257006\\
61.96	0.00126122833475807\\
61.97	0.00126082985936965\\
61.98	0.001260431196284\\
61.99	0.00126003234538024\\
62	0.00125963330653736\\
62.01	0.00125923407963422\\
62.02	0.00125883466454961\\
62.03	0.00125843506116216\\
62.04	0.00125803526935041\\
62.05	0.00125763528899279\\
62.06	0.00125723511996761\\
62.07	0.00125683476215306\\
62.08	0.00125643421542721\\
62.09	0.00125603347966804\\
62.1	0.00125563255475338\\
62.11	0.00125523144056098\\
62.12	0.00125483013696845\\
62.13	0.00125442864385329\\
62.14	0.0012540269610929\\
62.15	0.00125362508856455\\
62.16	0.00125322302614539\\
62.17	0.00125282077371246\\
62.18	0.00125241833114268\\
62.19	0.00125201569831287\\
62.2	0.00125161287509971\\
62.21	0.00125120986137979\\
62.22	0.00125080665702955\\
62.23	0.00125040326192533\\
62.24	0.00124999967594337\\
62.25	0.00124959589895976\\
62.26	0.0012491919308505\\
62.27	0.00124878777149145\\
62.28	0.00124838342075837\\
62.29	0.0012479788785269\\
62.3	0.00124757414467255\\
62.31	0.00124716921907071\\
62.32	0.00124676410159668\\
62.33	0.00124635879212561\\
62.34	0.00124595329053254\\
62.35	0.00124554759669241\\
62.36	0.00124514171048\\
62.37	0.00124473563177002\\
62.38	0.00124432936043703\\
62.39	0.00124392289635547\\
62.4	0.00124351623939968\\
62.41	0.00124310938944385\\
62.42	0.00124270234636208\\
62.43	0.00124229511002834\\
62.44	0.00124188768031647\\
62.45	0.00124148005710019\\
62.46	0.00124107224025312\\
62.47	0.00124066422964874\\
62.48	0.00124025602516041\\
62.49	0.00123984762666138\\
62.5	0.00123943903402477\\
62.51	0.00123903024712358\\
62.52	0.00123862126583068\\
62.53	0.00123821209001885\\
62.54	0.0012378027195607\\
62.55	0.00123739315432876\\
62.56	0.00123698339419541\\
62.57	0.00123657343903293\\
62.58	0.00123616328871346\\
62.59	0.00123575294310903\\
62.6	0.00123534240209153\\
62.61	0.00123493166553275\\
62.62	0.00123452073330434\\
62.63	0.00123410960527782\\
62.64	0.00123369828132462\\
62.65	0.00123328676131601\\
62.66	0.00123287504512316\\
62.67	0.0012324631326171\\
62.68	0.00123205102366875\\
62.69	0.00123163871814888\\
62.7	0.00123122621592819\\
62.71	0.00123081351687719\\
62.72	0.0012304006208663\\
62.73	0.00122998752776583\\
62.74	0.00122957423744593\\
62.75	0.00122916074977664\\
62.76	0.00122874706462788\\
62.77	0.00122833318186945\\
62.78	0.001227919101371\\
62.79	0.00122750482300208\\
62.8	0.0012270903466321\\
62.81	0.00122667567213034\\
62.82	0.00122626079936598\\
62.83	0.00122584572820805\\
62.84	0.00122543045852545\\
62.85	0.00122501499018697\\
62.86	0.00122459932306127\\
62.87	0.00122418345701687\\
62.88	0.00122376739192218\\
62.89	0.00122335112764548\\
62.9	0.00122293466405492\\
62.91	0.0012225180010185\\
62.92	0.00122210113840414\\
62.93	0.0012216840760796\\
62.94	0.00122126681391251\\
62.95	0.00122084935177039\\
62.96	0.00122043168952063\\
62.97	0.00122001382703047\\
62.98	0.00121959576416705\\
62.99	0.00121917750079737\\
63	0.00121875903678828\\
63.01	0.00121834037200655\\
63.02	0.00121792150631879\\
63.03	0.00121750243959146\\
63.04	0.00121708317169095\\
63.05	0.00121666370248347\\
63.06	0.00121624403183512\\
63.07	0.00121582415961186\\
63.08	0.00121540408567955\\
63.09	0.00121498380990389\\
63.1	0.00121456333215046\\
63.11	0.00121414265228472\\
63.12	0.00121372177017197\\
63.13	0.00121330068567743\\
63.14	0.00121287939866615\\
63.15	0.00121245790900306\\
63.16	0.00121203621655296\\
63.17	0.00121161432118053\\
63.18	0.00121119222275031\\
63.19	0.00121076992112671\\
63.2	0.001210347416174\\
63.21	0.00120992470775635\\
63.22	0.00120950179573778\\
63.23	0.00120907867998216\\
63.24	0.00120865536035326\\
63.25	0.00120823183671471\\
63.26	0.00120780810893\\
63.27	0.00120738417686251\\
63.28	0.00120696004037545\\
63.29	0.00120653569933194\\
63.3	0.00120611115359496\\
63.31	0.00120568640302734\\
63.32	0.00120526144749178\\
63.33	0.00120483628685088\\
63.34	0.00120441092096708\\
63.35	0.00120398534970268\\
63.36	0.00120355957291989\\
63.37	0.00120313359048075\\
63.38	0.00120270740224718\\
63.39	0.00120228100808098\\
63.4	0.00120185440784379\\
63.41	0.00120142760139715\\
63.42	0.00120100058860245\\
63.43	0.00120057336932096\\
63.44	0.0012001459434138\\
63.45	0.00119971831074198\\
63.46	0.00119929047116636\\
63.47	0.00119886242454768\\
63.48	0.00119843417074655\\
63.49	0.00119800570962343\\
63.5	0.00119757704103867\\
63.51	0.00119714816485247\\
63.52	0.00119671908092491\\
63.53	0.00119628978911595\\
63.54	0.00119586028928538\\
63.55	0.0011954305812929\\
63.56	0.00119500066499805\\
63.57	0.00119457054026025\\
63.58	0.0011941402069388\\
63.59	0.00119370966489284\\
63.6	0.00119327891398139\\
63.61	0.00119284795406336\\
63.62	0.0011924167849975\\
63.63	0.00119198540664244\\
63.64	0.00119155381885668\\
63.65	0.00119112202149858\\
63.66	0.00119069001442639\\
63.67	0.0011902577974982\\
63.68	0.00118982537057198\\
63.69	0.0011893927335056\\
63.7	0.00118895988615674\\
63.71	0.00118852682838299\\
63.72	0.00118809356004182\\
63.73	0.00118766008099052\\
63.74	0.00118722639108629\\
63.75	0.0011867924901862\\
63.76	0.00118635837814716\\
63.77	0.00118592405482598\\
63.78	0.00118548952007933\\
63.79	0.00118505477376374\\
63.8	0.00118461981573562\\
63.81	0.00118418464585126\\
63.82	0.0011837492639668\\
63.83	0.00118331366993826\\
63.84	0.00118287786362153\\
63.85	0.00118244184487238\\
63.86	0.00118200561354645\\
63.87	0.00118156916949923\\
63.88	0.0011811325125861\\
63.89	0.00118069564266232\\
63.9	0.001180258559583\\
63.91	0.00117982126320314\\
63.92	0.0011793837533776\\
63.93	0.00117894602996113\\
63.94	0.00117850809280833\\
63.95	0.00117806994177369\\
63.96	0.00117763157671158\\
63.97	0.00117719299747621\\
63.98	0.0011767542039217\\
63.99	0.00117631519590202\\
64	0.00117587597327105\\
64.01	0.00117543653588249\\
64.02	0.00117499688358996\\
64.03	0.00117455701624695\\
64.04	0.0011741169337068\\
64.05	0.00117367663582275\\
64.06	0.00117323612244791\\
64.07	0.00117279539343526\\
64.08	0.00117235444863766\\
64.09	0.00117191328790786\\
64.1	0.00117147191109848\\
64.11	0.00117103031806201\\
64.12	0.00117058850865083\\
64.13	0.00117014648271718\\
64.14	0.00116970424011321\\
64.15	0.00116926178069092\\
64.16	0.0011688191043022\\
64.17	0.00116837621079884\\
64.18	0.00116793310003249\\
64.19	0.00116748977185466\\
64.2	0.0011670462261168\\
64.21	0.0011666024626702\\
64.22	0.00116615848136603\\
64.23	0.00116571428205537\\
64.24	0.00116526986458916\\
64.25	0.00116482522881824\\
64.26	0.00116438037459332\\
64.27	0.00116393530176501\\
64.28	0.0011634900101838\\
64.29	0.00116304449970006\\
64.3	0.00116259877016404\\
64.31	0.00116215282142591\\
64.32	0.0011617066533357\\
64.33	0.00116126026574333\\
64.34	0.00116081365849861\\
64.35	0.00116036683145125\\
64.36	0.00115991978445085\\
64.37	0.00115947251734688\\
64.38	0.00115902502998872\\
64.39	0.00115857732222564\\
64.4	0.00115812939390679\\
64.41	0.00115768124488124\\
64.42	0.00115723287499792\\
64.43	0.00115678428410567\\
64.44	0.00115633547205323\\
64.45	0.00115588643868924\\
64.46	0.00115543718386221\\
64.47	0.00115498770742057\\
64.48	0.00115453800921266\\
64.49	0.00115408808908667\\
64.5	0.00115363794689074\\
64.51	0.00115318758247288\\
64.52	0.00115273699568102\\
64.53	0.00115228618636297\\
64.54	0.00115183515436645\\
64.55	0.00115138389953909\\
64.56	0.00115093242172842\\
64.57	0.00115048072078187\\
64.58	0.00115002879654678\\
64.59	0.0011495766488704\\
64.6	0.00114912427759986\\
64.61	0.00114867168258223\\
64.62	0.00114821886366448\\
64.63	0.00114776582069346\\
64.64	0.00114731255351598\\
64.65	0.00114685906197871\\
64.66	0.00114640534592827\\
64.67	0.00114595140521117\\
64.68	0.00114549723967383\\
64.69	0.00114504284916261\\
64.7	0.00114458823352375\\
64.71	0.00114413339260342\\
64.72	0.00114367832624772\\
64.73	0.00114322303430265\\
64.74	0.00114276751661412\\
64.75	0.00114231177302798\\
64.76	0.00114185580338998\\
64.77	0.0011413996075458\\
64.78	0.00114094318534106\\
64.79	0.00114048653662125\\
64.8	0.00114002966123182\\
64.81	0.00113957255901815\\
64.82	0.00113911522982552\\
64.83	0.00113865767349914\\
64.84	0.00113819988988417\\
64.85	0.00113774187882565\\
64.86	0.00113728364016859\\
64.87	0.0011368251737579\\
64.88	0.00113636647943845\\
64.89	0.001135907557055\\
64.9	0.00113544840645227\\
64.91	0.00113498902747489\\
64.92	0.00113452941996744\\
64.93	0.00113406958377443\\
64.94	0.00113360951874028\\
64.95	0.00113314922470937\\
64.96	0.00113268870152601\\
64.97	0.00113222794903443\\
64.98	0.00113176696707882\\
64.99	0.00113130575550327\\
65	0.00113084431415185\\
65.01	0.00113038264286854\\
65.02	0.00112992074149726\\
65.03	0.00112945860988188\\
65.04	0.00112899624786619\\
65.05	0.00112853365529394\\
65.06	0.00112807083200881\\
65.07	0.00112760777785443\\
65.08	0.00112714449267435\\
65.09	0.00112668097631207\\
65.1	0.00112621722861105\\
65.11	0.00112575324941467\\
65.12	0.00112528903856626\\
65.13	0.00112482459590908\\
65.14	0.00112435992128636\\
65.15	0.00112389501454125\\
65.16	0.00112342987551685\\
65.17	0.0011229645040562\\
65.18	0.00112249890000228\\
65.19	0.00112203306319802\\
65.2	0.0011215669934863\\
65.21	0.00112110069070991\\
65.22	0.00112063415471162\\
65.23	0.00112016738533413\\
65.24	0.00111970038242006\\
65.25	0.001119233145812\\
65.26	0.00111876567535246\\
65.27	0.00111829797088392\\
65.28	0.00111783003224875\\
65.29	0.0011173618592893\\
65.3	0.00111689345184786\\
65.31	0.00111642480976662\\
65.32	0.00111595593288775\\
65.33	0.00111548682105332\\
65.34	0.00111501747410535\\
65.35	0.00111454789188579\\
65.36	0.00111407807423653\\
65.37	0.00111360802099939\\
65.38	0.00111313773201609\\
65.39	0.00111266720712832\\
65.4	0.00111219644617766\\
65.41	0.00111172544900564\\
65.42	0.0011112542154537\\
65.43	0.00111078274536319\\
65.44	0.00111031103857539\\
65.45	0.0011098390949315\\
65.46	0.00110936691427261\\
65.47	0.00110889449643976\\
65.48	0.00110842184127385\\
65.49	0.00110794894861571\\
65.5	0.00110747581830609\\
65.51	0.00110700245018558\\
65.52	0.00110652884409473\\
65.53	0.00110605499987394\\
65.54	0.00110558091736352\\
65.55	0.00110510659640365\\
65.56	0.00110463203683439\\
65.57	0.00110415723849568\\
65.58	0.00110368220122733\\
65.59	0.00110320692486903\\
65.6	0.0011027314092603\\
65.61	0.00110225565424055\\
65.62	0.00110177965964902\\
65.63	0.0011013034253248\\
65.64	0.00110082695110683\\
65.65	0.00110035023683388\\
65.66	0.00109987328234453\\
65.67	0.00109939608747722\\
65.68	0.00109891865207017\\
65.69	0.00109844097596142\\
65.7	0.00109796305898882\\
65.71	0.00109748490098999\\
65.72	0.00109700650180236\\
65.73	0.00109652786126313\\
65.74	0.00109604897920927\\
65.75	0.0010955698554775\\
65.76	0.00109509048990431\\
65.77	0.00109461088232592\\
65.78	0.00109413103257829\\
65.79	0.00109365094049709\\
65.8	0.00109317060591772\\
65.81	0.00109269002867527\\
65.82	0.00109220920860453\\
65.83	0.00109172814553996\\
65.84	0.00109124683931571\\
65.85	0.00109076528976555\\
65.86	0.00109028349672293\\
65.87	0.00108980146002091\\
65.88	0.00108931917949217\\
65.89	0.001088836654969\\
65.9	0.00108835388628327\\
65.91	0.00108787087326644\\
65.92	0.0010873876157495\\
65.93	0.00108690411356304\\
65.94	0.00108642036653711\\
65.95	0.00108593637450134\\
65.96	0.00108545213728479\\
65.97	0.00108496765471604\\
65.98	0.00108448292662314\\
65.99	0.00108399795283353\\
66	0.00108351273317412\\
66.01	0.00108302726747121\\
66.02	0.00108254155555048\\
66.03	0.00108205559723696\\
66.04	0.00108156939235503\\
66.05	0.00108108294072841\\
66.06	0.0010805962421801\\
66.07	0.00108010929653235\\
66.08	0.00107962210360671\\
66.09	0.00107913466322391\\
66.1	0.00107864697520392\\
66.11	0.00107815903936586\\
66.12	0.001077670855528\\
66.13	0.00107718242350776\\
66.14	0.00107669374312161\\
66.15	0.00107620481418515\\
66.16	0.00107571563651295\\
66.17	0.00107522620991864\\
66.18	0.0010747365342148\\
66.19	0.00107424660921298\\
66.2	0.00107375643472363\\
66.21	0.00107326601055609\\
66.22	0.00107277533651856\\
66.23	0.00107228441241803\\
66.24	0.00107179323806029\\
66.25	0.00107130181324988\\
66.26	0.00107081013779005\\
66.27	0.0010703182114827\\
66.28	0.00106982603412839\\
66.29	0.00106933360552628\\
66.3	0.00106884092547405\\
66.31	0.00106834799376842\\
66.32	0.00106785481020529\\
66.33	0.00106736137457974\\
66.34	0.00106686768668605\\
66.35	0.00106637374631762\\
66.36	0.001065879553267\\
66.37	0.00106538510732584\\
66.38	0.00106489040828488\\
66.39	0.00106439545593395\\
66.4	0.00106390025006192\\
66.41	0.00106340479045669\\
66.42	0.00106290907690518\\
66.43	0.00106241310919329\\
66.44	0.00106191688710593\\
66.45	0.00106142041042691\\
66.46	0.00106092367893903\\
66.47	0.00106042669242397\\
66.48	0.00105992945066231\\
66.49	0.00105943195343353\\
66.5	0.00105893420051595\\
66.51	0.00105843619168674\\
66.52	0.00105793792672189\\
66.53	0.00105743940539619\\
66.54	0.00105694062748323\\
66.55	0.00105644159275536\\
66.56	0.0010559423009837\\
66.57	0.0010554427519381\\
66.58	0.00105494294538713\\
66.59	0.0010544428810981\\
66.6	0.00105394255883699\\
66.61	0.00105344197836846\\
66.62	0.00105294113945588\\
66.63	0.00105244004186126\\
66.64	0.00105193868534527\\
66.65	0.00105143706966721\\
66.66	0.00105093519458506\\
66.67	0.0010504330598554\\
66.68	0.00104993066523345\\
66.69	0.00104942801047307\\
66.7	0.00104892509532673\\
66.71	0.00104842191954555\\
66.72	0.00104791848287926\\
66.73	0.00104741478507623\\
66.74	0.0010469108258835\\
66.75	0.00104640660504675\\
66.76	0.00104590212231031\\
66.77	0.00104539737741721\\
66.78	0.00104489237010917\\
66.79	0.00104438710012665\\
66.8	0.00104388156720882\\
66.81	0.00104337577109365\\
66.82	0.0010428697115179\\
66.83	0.00104236338821714\\
66.84	0.00104185680092587\\
66.85	0.00104134994937744\\
66.86	0.00104084283330421\\
66.87	0.00104033545243752\\
66.88	0.00103982780650781\\
66.89	0.00103931989524461\\
66.9	0.00103881171837667\\
66.91	0.001038303275632\\
66.92	0.00103779456673795\\
66.93	0.0010372855914213\\
66.94	0.00103677634940836\\
66.95	0.00103626684042502\\
66.96	0.00103575706419691\\
66.97	0.0010352470204495\\
66.98	0.00103473670890819\\
66.99	0.00103422612929846\\
67	0.00103371528134599\\
67.01	0.00103320416477684\\
67.02	0.00103269277931756\\
67.03	0.00103218112469539\\
67.04	0.00103166920063841\\
67.05	0.00103115700687573\\
67.06	0.00103064454313769\\
67.07	0.00103013180915609\\
67.08	0.00102961880466436\\
67.09	0.00102910552939784\\
67.1	0.0010285919830915\\
67.11	0.00102807816547957\\
67.12	0.00102756407629545\\
67.13	0.0010270497152718\\
67.14	0.00102653508214043\\
67.15	0.0010260201766324\\
67.16	0.00102550499847792\\
67.17	0.00102498954740638\\
67.18	0.00102447382314635\\
67.19	0.00102395782542557\\
67.2	0.00102344155397089\\
67.21	0.00102292500850834\\
67.22	0.00102240818876309\\
67.23	0.00102189109445939\\
67.24	0.00102137372532063\\
67.25	0.00102085608106932\\
67.26	0.00102033816142703\\
67.27	0.00101981996611445\\
67.28	0.0010193014948513\\
67.29	0.00101878274735641\\
67.3	0.00101826372334764\\
67.31	0.00101774442254189\\
67.32	0.0010172248446551\\
67.33	0.00101670498940222\\
67.34	0.00101618485649724\\
67.35	0.0010156644456531\\
67.36	0.00101514375658176\\
67.37	0.00101462278899416\\
67.38	0.00101410154260017\\
67.39	0.00101358001710865\\
67.4	0.00101305821222737\\
67.41	0.00101253612766303\\
67.42	0.00101201376312126\\
67.43	0.00101149111830658\\
67.44	0.00101096819292238\\
67.45	0.00101044498667095\\
67.46	0.00100992149925342\\
67.47	0.00100939773036979\\
67.48	0.00100887367971886\\
67.49	0.00100834934699828\\
67.5	0.00100782473190447\\
67.51	0.00100729983413268\\
67.52	0.00100677465337689\\
67.53	0.00100624918932985\\
67.54	0.00100572344168309\\
67.55	0.0010051974101268\\
67.56	0.00100467109434995\\
67.57	0.00100414449404014\\
67.58	0.00100361760888371\\
67.59	0.0010030904385656\\
67.6	0.00100256298276944\\
67.61	0.00100203524117747\\
67.62	0.00100150721347052\\
67.63	0.00100097889932805\\
67.64	0.00100045029842805\\
67.65	0.000999921410447099\\
67.66	0.00099939223506028\\
67.67	0.000998862771941212\\
67.68	0.000998333020762002\\
67.69	0.000997802981193232\\
67.7	0.00099727265290394\\
67.71	0.000996742035561597\\
67.72	0.000996211128832093\\
67.73	0.000995679932379706\\
67.74	0.000995148445867081\\
67.75	0.00099461666895522\\
67.76	0.000994084601303444\\
67.77	0.000993552242569381\\
67.78	0.000993019592408941\\
67.79	0.000992486650476285\\
67.8	0.000991953416423811\\
67.81	0.000991419889902128\\
67.82	0.000990886070560014\\
67.83	0.000990351958044431\\
67.84	0.000989817552000448\\
67.85	0.000989282852071253\\
67.86	0.000988747857898112\\
67.87	0.000988212569120349\\
67.88	0.000987676985375312\\
67.89	0.000987141106298337\\
67.9	0.000986604931522745\\
67.91	0.000986068460679784\\
67.92	0.000985531693398628\\
67.93	0.00098499462930632\\
67.94	0.000984457268027761\\
67.95	0.000983919609185676\\
67.96	0.000983381652400563\\
67.97	0.000982843397290704\\
67.98	0.000982304843472087\\
67.99	0.000981765990558393\\
68	0.000981226838160971\\
68.01	0.000980687385888791\\
68.02	0.000980147633348411\\
68.03	0.00097960758014394\\
68.04	0.00097906722587702\\
68.05	0.000978526570146769\\
68.06	0.000977985612549742\\
68.07	0.000977444352679916\\
68.08	0.000976902790128634\\
68.09	0.000976360924484563\\
68.1	0.000975818755333667\\
68.11	0.000975276282259168\\
68.12	0.000974733504841487\\
68.13	0.000974190422658221\\
68.14	0.000973647035284091\\
68.15	0.000973103342290905\\
68.16	0.000972559343247508\\
68.17	0.000972015037719744\\
68.18	0.000971470425270404\\
68.19	0.000970925505459184\\
68.2	0.000970380277842639\\
68.21	0.000969834741974139\\
68.22	0.000969288897403813\\
68.23	0.000968742743678501\\
68.24	0.000968196280341706\\
68.25	0.00096764950693355\\
68.26	0.000967102422990713\\
68.27	0.000966555028046384\\
68.28	0.000966007321630206\\
68.29	0.000965459303268227\\
68.3	0.000964910972482845\\
68.31	0.000964362328792741\\
68.32	0.000963813371712833\\
68.33	0.000963264100754225\\
68.34	0.00096271451542412\\
68.35	0.000962164615225797\\
68.36	0.000961614399658523\\
68.37	0.000961063868217503\\
68.38	0.000960513020393818\\
68.39	0.000959961855674357\\
68.4	0.000959410373541756\\
68.41	0.000958858573474327\\
68.42	0.000958306454945998\\
68.43	0.000957754017426239\\
68.44	0.000957201260380002\\
68.45	0.000956648183267643\\
68.46	0.000956094785544844\\
68.47	0.00095554106666257\\
68.48	0.000954987026066957\\
68.49	0.000954432663199259\\
68.5	0.000953877977495783\\
68.51	0.000953322968387784\\
68.52	0.000952767635301405\\
68.53	0.000952211977657597\\
68.54	0.000951655994872026\\
68.55	0.00095109968635502\\
68.56	0.000950543051511448\\
68.57	0.000949986089740656\\
68.58	0.000949428800436391\\
68.59	0.000948871182986689\\
68.6	0.000948313236773808\\
68.61	0.000947754961174127\\
68.62	0.000947196355558056\\
68.63	0.000946637419289955\\
68.64	0.000946078151728013\\
68.65	0.000945518552224186\\
68.66	0.00094495862012408\\
68.67	0.000944398354766852\\
68.68	0.000943837755485129\\
68.69	0.000943276821604874\\
68.7	0.000942715552445319\\
68.71	0.000942153947318844\\
68.72	0.000941592005530857\\
68.73	0.00094102972637972\\
68.74	0.000940467109156615\\
68.75	0.000939904153145436\\
68.76	0.00093934085762269\\
68.77	0.000938777221857374\\
68.78	0.000938213245110861\\
68.79	0.000937648926636779\\
68.8	0.000937084265680899\\
68.81	0.000936519261481023\\
68.82	0.000935953913266844\\
68.83	0.000935388220259833\\
68.84	0.000934822181673111\\
68.85	0.000934255796711326\\
68.86	0.000933689064570527\\
68.87	0.000933121984438018\\
68.88	0.000932554555492253\\
68.89	0.000931986776902675\\
68.9	0.000931418647829597\\
68.91	0.000930850167424065\\
68.92	0.000930281334827707\\
68.93	0.000929712149172613\\
68.94	0.000929142609581168\\
68.95	0.000928572715165929\\
68.96	0.000928002465029465\\
68.97	0.000927431858264225\\
68.98	0.00092686089395237\\
68.99	0.000926289571165636\\
69	0.000925717888965174\\
69.01	0.000925145846401404\\
69.02	0.00092457344251385\\
69.03	0.00092400067633098\\
69.04	0.000923427546870055\\
69.05	0.00092285405313697\\
69.06	0.000922280194126076\\
69.07	0.000921705968820028\\
69.08	0.000921131376189615\\
69.09	0.000920556415193593\\
69.1	0.000919981084778506\\
69.11	0.000919405383878531\\
69.12	0.000918829311415301\\
69.13	0.000918252866297709\\
69.14	0.00091767604742176\\
69.15	0.000917098853670383\\
69.16	0.000916521283913245\\
69.17	0.000915943337006586\\
69.18	0.000915365011793016\\
69.19	0.00091478630710135\\
69.2	0.000914207221746415\\
69.21	0.000913627754528868\\
69.22	0.000913047904235\\
69.23	0.00091246766963656\\
69.24	0.000911887049490563\\
69.25	0.000911306042539091\\
69.26	0.000910724647509107\\
69.27	0.000910142863112263\\
69.28	0.000909560688044708\\
69.29	0.000908978120986895\\
69.3	0.000908395160603382\\
69.31	0.000907811805542641\\
69.32	0.000907228054436854\\
69.33	0.000906643905901727\\
69.34	0.000906059358536299\\
69.35	0.000905474410922725\\
69.36	0.000904889061626094\\
69.37	0.000904303309194235\\
69.38	0.000903717152157518\\
69.39	0.000903130589028646\\
69.4	0.000902543618302476\\
69.41	0.000901956238455818\\
69.42	0.000901368447947241\\
69.43	0.000900780245216885\\
69.44	0.000900191628686261\\
69.45	0.000899602596758062\\
69.46	0.000899013147815984\\
69.47	0.000898423280224528\\
69.48	0.000897832992328819\\
69.49	0.000897242282454426\\
69.5	0.000896651148907177\\
69.51	0.000896059589972981\\
69.52	0.000895467603917661\\
69.53	0.000894875188986769\\
69.54	0.000894282343405432\\
69.55	0.000893689065378182\\
69.56	0.000893095353088787\\
69.57	0.00089250120470011\\
69.58	0.000891906618353956\\
69.59	0.000891311592170908\\
69.6	0.000890716124250211\\
69.61	0.000890120212669619\\
69.62	0.000889523855485281\\
69.63	0.000888927050731604\\
69.64	0.000888329796421158\\
69.65	0.000887732090544543\\
69.66	0.00088713393107032\\
69.67	0.000886535315944894\\
69.68	0.000885936243092451\\
69.69	0.000885336710414884\\
69.7	0.000884736715791709\\
69.71	0.000884136257080049\\
69.72	0.000883535332114578\\
69.73	0.000882933938707486\\
69.74	0.000882332074648471\\
69.75	0.000881729737704754\\
69.76	0.000881126925621051\\
69.77	0.000880523636119653\\
69.78	0.000879919866900425\\
69.79	0.000879315615640884\\
69.8	0.000878710879996279\\
69.81	0.000878105657599684\\
69.82	0.000877499946062099\\
69.83	0.000876893742972607\\
69.84	0.000876287045898497\\
69.85	0.000875679852385482\\
69.86	0.000875072159957867\\
69.87	0.000874463966118772\\
69.88	0.000873855268350411\\
69.89	0.000873246064114331\\
69.9	0.000872636350851745\\
69.91	0.000872026125983852\\
69.92	0.000871415386912188\\
69.93	0.000870804131019051\\
69.94	0.000870192355667898\\
69.95	0.000869580058203829\\
69.96	0.000868967235954064\\
69.97	0.000868353886228506\\
69.98	0.000867740006320306\\
69.99	0.000867125593506463\\
70	0.000866510645048517\\
70.01	0.000865895158193228\\
70.02	0.000865279130173334\\
70.03	0.000864662558208369\\
70.04	0.000864045439505495\\
70.05	0.00086342777126042\\
70.06	0.00086280955065835\\
70.07	0.000862190774875019\\
70.08	0.00086157144107776\\
70.09	0.000860951546426649\\
70.1	0.000860331088075714\\
70.11	0.000859710063174209\\
70.12	0.000859088468867968\\
70.13	0.000858466302300817\\
70.14	0.000857843560616068\\
70.15	0.000857220240958117\\
70.16	0.000856596340474075\\
70.17	0.000855971856315548\\
70.18	0.00085534678564044\\
70.19	0.000854721125614927\\
70.2	0.000854094873415446\\
70.21	0.000853468026230847\\
70.22	0.000852840581264631\\
70.23	0.000852212535737282\\
70.24	0.000851583886888744\\
70.25	0.000850954631980995\\
70.26	0.000850324768300732\\
70.27	0.000849694293162236\\
70.28	0.000849063203910302\\
70.29	0.000848431497923361\\
70.3	0.000847799172616712\\
70.31	0.000847166225445945\\
70.32	0.00084653265391044\\
70.33	0.00084589845555714\\
70.34	0.000845263627984372\\
70.35	0.000844628168845942\\
70.36	0.000843992075855346\\
70.37	0.000843355346790196\\
70.38	0.000842717979496827\\
70.39	0.000842079971895126\\
70.4	0.000841441321983542\\
70.41	0.000840802027844348\\
70.42	0.000840162087649083\\
70.43	0.00083952149966428\\
70.44	0.000838880262257395\\
70.45	0.000838238373903007\\
70.46	0.000837595833189283\\
70.47	0.000836952638824702\\
70.48	0.000836308789645071\\
70.49	0.000835664284620815\\
70.5	0.000835019122864595\\
70.51	0.000834373303639231\\
70.52	0.000833726826365936\\
70.53	0.000833079690632897\\
70.54	0.000832431896204227\\
70.55	0.000831783443029232\\
70.56	0.000831134331252105\\
70.57	0.000830484561221973\\
70.58	0.000829834133503378\\
70.59	0.000829183048887152\\
70.6	0.000828531308401737\\
70.61	0.000827878913324944\\
70.62	0.000827225865196219\\
70.63	0.000826572165829344\\
70.64	0.000825917817325645\\
70.65	0.000825262822087761\\
70.66	0.000824607182833908\\
70.67	0.000823950902612716\\
70.68	0.000823293984818641\\
70.69	0.000822636433207994\\
70.7	0.000821978251915547\\
70.71	0.000821319445471834\\
70.72	0.000820660018821081\\
70.73	0.000819999977339839\\
70.74	0.000819339326856322\\
70.75	0.000818678073670507\\
70.76	0.000818016224574976\\
70.77	0.000817353786876543\\
70.78	0.000816690768418765\\
70.79	0.000816027177605204\\
70.8	0.000815363023423667\\
70.81	0.000814698315471304\\
70.82	0.000814033063980653\\
70.83	0.000813367279846683\\
70.84	0.000812700974654844\\
70.85	0.000812034160710162\\
70.86	0.000811366851067421\\
70.87	0.000810699059562462\\
70.88	0.000810030800844668\\
70.89	0.000809362090410619\\
70.9	0.00080869294463904\\
70.91	0.000808023380826991\\
70.92	0.000807353417227444\\
70.93	0.00080668307308818\\
70.94	0.000806012368692197\\
70.95	0.000805341325399528\\
70.96	0.000804669965690629\\
70.97	0.000803998313211367\\
70.98	0.000803326392819605\\
70.99	0.00080265423063354\\
71	0.000801981854081748\\
71.01	0.000801309291955115\\
71.02	0.000800636574460582\\
71.03	0.000799963733276884\\
71.04	0.000799290801612312\\
71.05	0.000798617814264505\\
71.06	0.000797944807682499\\
71.07	0.000797271820030906\\
71.08	0.000796598891256483\\
71.09	0.00079592606315706\\
71.1	0.000795253379452937\\
71.11	0.000794580885860874\\
71.12	0.000793908630170698\\
71.13	0.000793236662324646\\
71.14	0.000792565034499589\\
71.15	0.000791893801192151\\
71.16	0.000791223019306847\\
71.17	0.00079055274824741\\
71.18	0.000789883050011328\\
71.19	0.00078921398928773\\
71.2	0.000788545633558789\\
71.21	0.000787878053204667\\
71.22	0.000787211321612209\\
71.23	0.000786545515287467\\
71.24	0.000785880713972188\\
71.25	0.000785217000764447\\
71.26	0.000784554462243483\\
71.27	0.000783893188598967\\
71.28	0.000783233273764802\\
71.29	0.000782574815557623\\
71.3	0.000781917915820163\\
71.31	0.000781262680569654\\
71.32	0.0007806092201514\\
71.33	0.000779957649397756\\
71.34	0.000779308087792653\\
71.35	0.000778660659641854\\
71.36	0.00077801549424919\\
71.37	0.000777372726098903\\
71.38	0.000776732495044369\\
71.39	0.00077609494650336\\
71.4	0.000775460231660136\\
71.41	0.000774828507674533\\
71.42	0.000774199937898293\\
71.43	0.000773574692098955\\
71.44	0.000772952946691443\\
71.45	0.0007723348849777\\
71.46	0.000771720697394632\\
71.47	0.000771110581770585\\
71.48	0.000770504743590712\\
71.49	0.000769903396271476\\
71.5	0.000769306761444656\\
71.51	0.000768715069251077\\
71.52	0.000768128558644523\\
71.53	0.000767547477706068\\
71.54	0.000766972083969204\\
71.55	0.00076640264475618\\
71.56	0.000765839437525834\\
71.57	0.000765282750233387\\
71.58	0.00076473235718031\\
71.59	0.000764181655740617\\
71.6	0.000763630646223236\\
71.61	0.000763079328950358\\
71.62	0.000762527704257514\\
71.63	0.000761975772493635\\
71.64	0.000761423534021116\\
71.65	0.000760870989215835\\
71.66	0.000760318138467192\\
71.67	0.00075976498217812\\
71.68	0.000759211520765087\\
71.69	0.000758657754658066\\
71.7	0.000758103684300512\\
71.71	0.000757549310149319\\
71.72	0.000756994632674727\\
71.73	0.000756439652360255\\
71.74	0.000755884369702583\\
71.75	0.000755328785211428\\
71.76	0.000754772899409372\\
71.77	0.0007542167128317\\
71.78	0.000753660226026191\\
71.79	0.000753103439552875\\
71.8	0.000752546353983786\\
71.81	0.000751988969902672\\
71.82	0.000751431287904645\\
71.83	0.000750873308595858\\
71.84	0.000750315032593082\\
71.85	0.000749756460523289\\
71.86	0.000749197593023186\\
71.87	0.000748638430738682\\
71.88	0.000748078974324357\\
71.89	0.000747519224442826\\
71.9	0.000746959181764133\\
71.91	0.000746398846964999\\
71.92	0.000745838220728096\\
71.93	0.000745277303741231\\
71.94	0.00074471609669647\\
71.95	0.000744154600289197\\
71.96	0.000743592815217135\\
71.97	0.000743030742179261\\
71.98	0.000742468381874677\\
71.99	0.000741905735001389\\
72	0.000741342802255025\\
72.01	0.000740779584327448\\
72.02	0.000740216081905307\\
72.03	0.000739652295668476\\
72.04	0.000739088226288412\\
72.05	0.000738523874426406\\
72.06	0.000737959240731749\\
72.07	0.000737394325839743\\
72.08	0.000736829130369659\\
72.09	0.000736263654922539\\
72.1	0.000735697900078861\\
72.11	0.000735131866396113\\
72.12	0.000734565554406222\\
72.13	0.000733998964612799\\
72.14	0.000733432097488284\\
72.15	0.00073286495347092\\
72.16	0.000732297532961555\\
72.17	0.000731729836320284\\
72.18	0.000731161863862914\\
72.19	0.000730593615857238\\
72.2	0.000730025092519107\\
72.21	0.000729456294008342\\
72.22	0.000728887220424359\\
72.23	0.000728317871801658\\
72.24	0.000727748248105032\\
72.25	0.000727178349224521\\
72.26	0.000726608174970181\\
72.27	0.000726037725066528\\
72.28	0.00072546699914673\\
72.29	0.000724895996746524\\
72.3	0.000724324717297829\\
72.31	0.000723753160122055\\
72.32	0.000723181324423061\\
72.33	0.000722609209279823\\
72.34	0.000722036813638711\\
72.35	0.00072146413630542\\
72.36	0.000720891175936506\\
72.37	0.000720317931030537\\
72.38	0.000719744399918827\\
72.39	0.000719170580755718\\
72.4	0.00071859647150844\\
72.41	0.000718022069946485\\
72.42	0.000717447373630475\\
72.43	0.000716872379900568\\
72.44	0.00071629708586428\\
72.45	0.000715721488383777\\
72.46	0.000715145584062585\\
72.47	0.000714569369231716\\
72.48	0.000713992839935146\\
72.49	0.000713415991914625\\
72.5	0.000712838820593868\\
72.51	0.00071226132106196\\
72.52	0.000711683488056112\\
72.53	0.000711105315943546\\
72.54	0.000710526798702685\\
72.55	0.000709947929903434\\
72.56	0.000709368702686643\\
72.57	0.00070878910974263\\
72.58	0.000708209143288806\\
72.59	0.000707628795046282\\
72.6	0.000707048056215496\\
72.61	0.000706466917450759\\
72.62	0.000705885368833698\\
72.63	0.000705303399845574\\
72.64	0.000704720999338387\\
72.65	0.000704138155504744\\
72.66	0.000703554855846436\\
72.67	0.000702971087141679\\
72.68	0.000702386835410952\\
72.69	0.000701802085881366\\
72.7	0.000701216822949541\\
72.71	0.000700631030142904\\
72.72	0.000700044690079336\\
72.73	0.000699457784425131\\
72.74	0.000698870293851161\\
72.75	0.000698282199058549\\
72.76	0.000697693486412385\\
72.77	0.000697104141642261\\
72.78	0.000696514149815683\\
72.79	0.000695923495310556\\
72.8	0.000695332161786532\\
72.81	0.00069474013215534\\
72.82	0.000694147388549938\\
72.83	0.000693553912292495\\
72.84	0.000692959683861178\\
72.85	0.000692364682855665\\
72.86	0.000691768887961351\\
72.87	0.000691172276912217\\
72.88	0.000690574826452283\\
72.89	0.000689976512295616\\
72.9	0.000689377309084851\\
72.91	0.000688777190348126\\
72.92	0.000688176128454419\\
72.93	0.000687574094567224\\
72.94	0.000686971058596487\\
72.95	0.000686366989148752\\
72.96	0.000685761853475452\\
72.97	0.000685155617419275\\
72.98	0.000684548245358555\\
72.99	0.000683939712777222\\
73	0.000683329998198319\\
73.01	0.00068271907955952\\
73.02	0.000682106934196312\\
73.03	0.000681493538824723\\
73.04	0.000680878869523506\\
73.05	0.000680262901715809\\
73.06	0.000679645610150305\\
73.07	0.000679026968881759\\
73.08	0.000678406951251011\\
73.09	0.000677785529864387\\
73.1	0.000677162676572456\\
73.11	0.000676538362448213\\
73.12	0.000675912557764559\\
73.13	0.00067528523197115\\
73.14	0.000674656353670523\\
73.15	0.000674025890593543\\
73.16	0.000673393809574102\\
73.17	0.000672760076523043\\
73.18	0.000672124656401351\\
73.19	0.000671487513192487\\
73.2	0.000670848609873945\\
73.21	0.000670207908387932\\
73.22	0.000669565369611165\\
73.23	0.000668920953323781\\
73.24	0.000668274618177296\\
73.25	0.00066762632166162\\
73.26	0.000666976020071057\\
73.27	0.000666323668469316\\
73.28	0.000665669220653446\\
73.29	0.000665012629116671\\
73.3	0.00066435384501016\\
73.31	0.000663692818103597\\
73.32	0.000663029496744568\\
73.33	0.000662363827816729\\
73.34	0.000661695756696726\\
73.35	0.000661025227209799\\
73.36	0.000660352181584035\\
73.37	0.000659676560403244\\
73.38	0.000658998302558427\\
73.39	0.000658317345197749\\
73.4	0.000657633623674998\\
73.41	0.000656947071496492\\
73.42	0.000656257620266373\\
73.43	0.000655565199630243\\
73.44	0.000654869737217057\\
73.45	0.000654171158579296\\
73.46	0.000653469387131284\\
73.47	0.000652764344085642\\
73.48	0.000652055948387801\\
73.49	0.000651344116648531\\
73.5	0.000650628763074375\\
73.51	0.000649909799396017\\
73.52	0.000649187134794418\\
73.53	0.000648460675824693\\
73.54	0.000647730326337697\\
73.55	0.00064699598739917\\
73.56	0.000646257557206448\\
73.57	0.000645514931002567\\
73.58	0.000644768000987787\\
73.59	0.000644016656228378\\
73.6	0.000643260782562596\\
73.61	0.000642500262503795\\
73.62	0.000641734975140543\\
73.63	0.000640964796033667\\
73.64	0.000640189597110137\\
73.65	0.000639409246553676\\
73.66	0.000638623608692015\\
73.67	0.000637832543880636\\
73.68	0.000637035908382964\\
73.69	0.000636233554246846\\
73.7	0.000635425329177217\\
73.71	0.00063461107640481\\
73.72	0.000633790634550841\\
73.73	0.000632963837487475\\
73.74	0.000632130514193995\\
73.75	0.000631290488608485\\
73.76	0.000630443579474931\\
73.77	0.00062958960018559\\
73.78	0.000628728358618423\\
73.79	0.000627859656969534\\
73.8	0.000626983291580341\\
73.81	0.000626099052759403\\
73.82	0.000625206724598709\\
73.83	0.000624306084784209\\
73.84	0.000623396904400466\\
73.85	0.000622478947729205\\
73.86	0.000621551972041574\\
73.87	0.00062061572738391\\
73.88	0.000619669956356823\\
73.89	0.000618714393887368\\
73.9	0.000617748766994051\\
73.91	0.000616772794544539\\
73.92	0.000615786187005708\\
73.93	0.000614788646185898\\
73.94	0.000613779864969062\\
73.95	0.000612759527040565\\
73.96	0.000611727306604358\\
73.97	0.000610682868091277\\
73.98	0.000609625865858117\\
73.99	0.000608555943877293\\
74	0.000607472735416649\\
74.01	0.000606375862709216\\
74.02	0.000605264936612514\\
74.03	0.000604139556257119\\
74.04	0.000602999308684073\\
74.05	0.000601843768470878\\
74.06	0.000600672497345613\\
74.07	0.00059948504378881\\
74.08	0.000598280942622731\\
74.09	0.000597059714587603\\
74.1	0.000595836582752201\\
74.11	0.000594613027984978\\
74.12	0.000593389050399002\\
74.13	0.000592164650113894\\
74.14	0.000590939827255697\\
74.15	0.000589714581956767\\
74.16	0.00058848891435561\\
74.17	0.000587262824596695\\
74.18	0.000586036312830277\\
74.19	0.000584809379212179\\
74.2	0.000583582023903548\\
74.21	0.000582354247070586\\
74.22	0.00058112604888428\\
74.23	0.00057989742952006\\
74.24	0.000578668389157471\\
74.25	0.000577438927979785\\
74.26	0.000576209046173606\\
74.27	0.000574978743928407\\
74.28	0.000573748021436069\\
74.29	0.000572516878890364\\
74.3	0.000571285316486387\\
74.31	0.00057005333441997\\
74.32	0.000568820932887043\\
74.33	0.000567588112082944\\
74.34	0.000566354872201679\\
74.35	0.000565121213435145\\
74.36	0.000563887135972293\\
74.37	0.000562652639998223\\
74.38	0.000561417725693252\\
74.39	0.00056018239323187\\
74.4	0.000558946642781695\\
74.41	0.0005577104745023\\
74.42	0.00055647388854402\\
74.43	0.000555236885046637\\
74.44	0.000553999464138026\\
74.45	0.000552761625932707\\
74.46	0.00055152337053031\\
74.47	0.000550284698013933\\
74.48	0.000549045608448451\\
74.49	0.000547806101878675\\
74.5	0.000546566178327448\\
74.51	0.000545325837793627\\
74.52	0.000544085080249919\\
74.53	0.000542843905640659\\
74.54	0.000541602313879402\\
74.55	0.000540360304846443\\
74.56	0.000539117878386163\\
74.57	0.000537875034304244\\
74.58	0.000536631772364772\\
74.59	0.000535388092287122\\
74.6	0.000534143993742749\\
74.61	0.00053289947635176\\
74.62	0.000531654539679351\\
74.63	0.00053040918323204\\
74.64	0.000529163406453714\\
74.65	0.000527917208721484\\
74.66	0.000526670589341327\\
74.67	0.000525423547543508\\
74.68	0.000524176082477784\\
74.69	0.000522928193208384\\
74.7	0.000521679878708716\\
74.71	0.000520431137855833\\
74.72	0.000519181969424642\\
74.73	0.000517932372081802\\
74.74	0.000516682344379396\\
74.75	0.000515431884748209\\
74.76	0.000514180991490778\\
74.77	0.000512929662774044\\
74.78	0.000511677896621725\\
74.79	0.000510425690906265\\
74.8	0.000509173043340465\\
74.81	0.00050791995211334\\
74.82	0.000506666416912861\\
74.83	0.00050541243742735\\
74.84	0.000504158013345503\\
74.85	0.000502903144356467\\
74.86	0.000501647830149889\\
74.87	0.000500392070415983\\
74.88	0.000499135864845566\\
74.89	0.000497879213130152\\
74.9	0.000496622114962002\\
74.91	0.000495364570034188\\
74.92	0.000494106578040653\\
74.93	0.000492848138676318\\
74.94	0.000491589251637109\\
74.95	0.000490329916620067\\
74.96	0.0004890701333234\\
74.97	0.000487809901446595\\
74.98	0.000486549220690455\\
74.99	0.00048528809075723\\
75	0.00048402651135067\\
75.01	0.000482764482176146\\
75.02	0.00048150200294071\\
75.03	0.000480239073353206\\
75.04	0.000478975693124382\\
75.05	0.000477711861966964\\
75.06	0.000476447579595777\\
75.07	0.000475182845727857\\
75.08	0.000473917660082531\\
75.09	0.000472652022381569\\
75.1	0.000471385932349281\\
75.11	0.00047011938971263\\
75.12	0.000468852394201368\\
75.13	0.000467584945548164\\
75.14	0.000466317043488712\\
75.15	0.000465048687761892\\
75.16	0.000463779878109893\\
75.17	0.000462510614278355\\
75.18	0.000461240896016516\\
75.19	0.000459970723077354\\
75.2	0.000458700095217753\\
75.21	0.000457429012198649\\
75.22	0.000456157473785192\\
75.23	0.000454885479746921\\
75.24	0.00045361302985792\\
75.25	0.000452340123896992\\
75.26	0.000451066761647861\\
75.27	0.000449792942899324\\
75.28	0.000448518667445464\\
75.29	0.000447243935085821\\
75.3	0.000445968745625608\\
75.31	0.00044469309887591\\
75.32	0.000443416994653889\\
75.33	0.000442140432782996\\
75.34	0.000440863413093193\\
75.35	0.000439585935421181\\
75.36	0.000438307999610621\\
75.37	0.000437029605512382\\
75.38	0.000435750752984767\\
75.39	0.000434471441893776\\
75.4	0.000433191672113354\\
75.41	0.00043191144352564\\
75.42	0.000430630756021257\\
75.43	0.000429349609499566\\
75.44	0.000428068003868956\\
75.45	0.00042678593904711\\
75.46	0.000425503414961337\\
75.47	0.000424220431548826\\
75.48	0.00042293698875699\\
75.49	0.000421653086543763\\
75.5	0.000420368724877921\\
75.51	0.000419083903739417\\
75.52	0.000417798623119721\\
75.53	0.000416512883022148\\
75.54	0.000415226683462243\\
75.55	0.000413940024468097\\
75.56	0.000412652906080764\\
75.57	0.000411365328354603\\
75.58	0.000410077291357679\\
75.59	0.000408788795172157\\
75.6	0.00040749983989471\\
75.61	0.000406210425636903\\
75.62	0.000404920552525669\\
75.63	0.000403630220703688\\
75.64	0.000402339430329858\\
75.65	0.000401048181579726\\
75.66	0.000399756474645968\\
75.67	0.000398464309738836\\
75.68	0.000397171687086667\\
75.69	0.000395878606936342\\
75.7	0.000394585069553813\\
75.71	0.000393291075224586\\
75.72	0.000391996624254274\\
75.73	0.000390701716969104\\
75.74	0.000389406353716478\\
75.75	0.000388110534865524\\
75.76	0.000386814260807654\\
75.77	0.000385517531957149\\
75.78	0.000384220348751746\\
75.79	0.000382922711653242\\
75.8	0.000381624621148104\\
75.81	0.000380326077748096\\
75.82	0.000379027081990915\\
75.83	0.00037772763444085\\
75.84	0.000376427735689423\\
75.85	0.000375127386356089\\
75.86	0.000373826587088906\\
75.87	0.000372525338565254\\
75.88	0.000371223641492531\\
75.89	0.000369921496608883\\
75.9	0.000368618904683963\\
75.91	0.000367315866519653\\
75.92	0.00036601238295087\\
75.93	0.00036470845484632\\
75.94	0.0003634040831093\\
75.95	0.000362099268678513\\
75.96	0.000360794012528892\\
75.97	0.000359488315672424\\
75.98	0.00035818217915902\\
75.99	0.000356875604077362\\
76	0.00035556859155581\\
76.01	0.000354261142763266\\
76.02	0.000352953258910103\\
76.03	0.000351644941249082\\
76.04	0.000350336191076297\\
76.05	0.000349027009732118\\
76.06	0.000347717398602161\\
76.07	0.000346407359118278\\
76.08	0.000345096892759545\\
76.09	0.000343786001053266\\
76.1	0.000342474685576027\\
76.11	0.000341162947954693\\
76.12	0.000339850789867504\\
76.13	0.000338538213045107\\
76.14	0.000337225219271675\\
76.15	0.000335911810385964\\
76.16	0.00033459798828245\\
76.17	0.00033328375491244\\
76.18	0.000331969112285222\\
76.19	0.000330654062469194\\
76.2	0.000329338607593044\\
76.21	0.000328022749846921\\
76.22	0.000326706491483619\\
76.23	0.000325389834819776\\
76.24	0.000324072782237091\\
76.25	0.000322755336183547\\
76.26	0.000321437499174633\\
76.27	0.0003201192737946\\
76.28	0.000318800662697702\\
76.29	0.00031748166860948\\
76.3	0.000316162294328006\\
76.31	0.000314842542725178\\
76.32	0.000313522416747997\\
76.33	0.00031220191941988\\
76.34	0.000310881053841937\\
76.35	0.000309559823194276\\
76.36	0.000308238230737333\\
76.37	0.000306916279813146\\
76.38	0.000305593973846717\\
76.39	0.000304271316347282\\
76.4	0.000302948310909648\\
76.41	0.000301624961215495\\
76.42	0.000300301271034706\\
76.43	0.00029897724422666\\
76.44	0.000297652884741531\\
76.45	0.000296328196621586\\
76.46	0.000295003184002477\\
76.47	0.00029367785111452\\
76.48	0.000292352202283958\\
76.49	0.000291026241934202\\
76.5	0.000289699974587086\\
76.51	0.000288373404864073\\
76.52	0.000287046537487462\\
76.53	0.000285719377281557\\
76.54	0.000284391929173839\\
76.55	0.000283064198196065\\
76.56	0.000281736189485385\\
76.57	0.000280407908285403\\
76.58	0.000279079359947202\\
76.59	0.000277750549930338\\
76.6	0.000276421483803793\\
76.61	0.000275092167246874\\
76.62	0.000273762606050066\\
76.63	0.000272432806115858\\
76.64	0.000271102773459465\\
76.65	0.000269772514209536\\
76.66	0.000268442034608764\\
76.67	0.000267111341014453\\
76.68	0.000265780439898999\\
76.69	0.000264449337850282\\
76.7	0.000263118041572013\\
76.71	0.000261786557883939\\
76.72	0.000260454893721984\\
76.73	0.000259123056138304\\
76.74	0.000257791052301212\\
76.75	0.000256458889494989\\
76.76	0.000255126575119587\\
76.77	0.000253794116690199\\
76.78	0.000252461521836715\\
76.79	0.000251128798303002\\
76.8	0.000249795953946062\\
76.81	0.000248462996735017\\
76.82	0.000247129934749955\\
76.83	0.000245796776180553\\
76.84	0.000244463529324572\\
76.85	0.000243130202586112\\
76.86	0.000241796804473676\\
76.87	0.000240463343598044\\
76.88	0.000239129828669882\\
76.89	0.000237796268497141\\
76.9	0.000236462671982195\\
76.91	0.000235129048118713\\
76.92	0.000233795405988261\\
76.93	0.000232461754756618\\
76.94	0.000231128103669776\\
76.95	0.000229794462049654\\
76.96	0.000228460839289417\\
76.97	0.000227127244848512\\
76.98	0.000225793688247289\\
76.99	0.000224460179061265\\
77	0.000223126726914985\\
77.01	0.000221793341475427\\
77.02	0.000220460032445021\\
77.03	0.000219126809554158\\
77.04	0.000217793682553247\\
77.05	0.000216460661204253\\
77.06	0.000215127755271704\\
77.07	0.000213794974513166\\
77.08	0.000212462328669108\\
77.09	0.000211129827452198\\
77.1	0.000209797480535947\\
77.11	0.000208465297542697\\
77.12	0.000207133288030939\\
77.13	0.000205801461481882\\
77.14	0.000204469827285312\\
77.15	0.000203138394724623\\
77.16	0.000201807172961071\\
77.17	0.000200476171017165\\
77.18	0.000199145397759158\\
77.19	0.000197814861878608\\
77.2	0.000196484571873011\\
77.21	0.000195154536025359\\
77.22	0.000193824762382729\\
77.23	0.000192495258733681\\
77.24	0.000191166032584613\\
77.25	0.00018983709113485\\
77.26	0.000188508441250523\\
77.27	0.000187180089437154\\
77.28	0.000185852041810902\\
77.29	0.00018452430406838\\
77.3	0.000183196881455036\\
77.31	0.000181869780463114\\
77.32	0.00018054300995844\\
77.33	0.000179216578489025\\
77.34	0.000177890494257416\\
77.35	0.000176564765091656\\
77.36	0.000175239398414878\\
77.37	0.000173914401213364\\
77.38	0.000172589780003118\\
77.39	0.000171265540794835\\
77.4	0.000169941689057179\\
77.41	0.000168618229678378\\
77.42	0.00016729516692596\\
77.43	0.000165972504404695\\
77.44	0.000164650245012502\\
77.45	0.00016332839089442\\
77.46	0.000162006943394384\\
77.47	0.000160685903004892\\
77.48	0.000159365269314328\\
77.49	0.00015804504095196\\
77.5	0.000156725215530458\\
77.51	0.000155405789585842\\
77.52	0.000154086758514799\\
77.53	0.000152768116509211\\
77.54	0.000151449856487788\\
77.55	0.000150131970024791\\
77.56	0.000148814447275549\\
77.57	0.000147497276898832\\
77.58	0.000146180445975841\\
77.59	0.000144863939925704\\
77.6	0.000143547742417392\\
77.61	0.000142231835277853\\
77.62	0.000140916200747658\\
77.63	0.000139600860513692\\
77.64	0.000138285836886708\\
77.65	0.000136971152819623\\
77.66	0.000135656831926406\\
77.67	0.000134342898501512\\
77.68	0.000133029377539933\\
77.69	0.000131716294757819\\
77.7	0.000130403676613756\\
77.71	0.000129091550330649\\
77.72	0.000127779943918344\\
77.73	0.000126468886196834\\
77.74	0.000125158406820269\\
77.75	0.000123848536301626\\
77.76	0.00012253930603817\\
77.77	0.000121230748337671\\
77.78	0.000119922896445419\\
77.79	0.000118615784572085\\
77.8	0.000117309447922407\\
77.81	0.000116003922724759\\
77.82	0.00011469924626163\\
77.83	0.00011339545690104\\
77.84	0.000112092594128883\\
77.85	0.00011079069858231\\
77.86	0.000109489812084104\\
77.87	0.000108189977678108\\
77.88	0.000106891239665757\\
77.89	0.000105593643643712\\
77.9	0.000104297236542653\\
77.91	0.000103002066667296\\
77.92	0.000101708183737594\\
77.93	0.000100415638931225\\
77.94	9.91244849274028e-05\\
77.95	9.78347759520194e-05\\
77.96	9.65465678241795e-05\\
77.97	9.52599180041869e-05\\
77.98	9.39748856429931e-05\\
77.99	9.26915316331687e-05\\
78	9.14099186614861e-05\\
78.01	9.0130111263095e-05\\
78.02	8.88521758773703e-05\\
78.03	8.75761809055169e-05\\
78.04	8.63021967699405e-05\\
78.05	8.50302959754402e-05\\
78.06	8.37605531723573e-05\\
78.07	8.24930452216158e-05\\
78.08	8.12278512618159e-05\\
78.09	7.99650527784166e-05\\
78.1	7.87047336750536e-05\\
78.11	7.74469803470779e-05\\
78.12	7.61918817573804e-05\\
78.13	7.49395295145932e-05\\
78.14	7.36900179536919e-05\\
78.15	7.24434442191599e-05\\
78.16	7.11999083506838e-05\\
78.17	6.99595133715775e-05\\
78.18	6.87223653799262e-05\\
78.19	6.74885736426014e-05\\
78.2	6.6258250692201e-05\\
78.21	6.50315124270141e-05\\
78.22	6.38084782141248e-05\\
78.23	6.2589270995719e-05\\
78.24	6.13740173987624e-05\\
78.25	6.01628478480443e-05\\
78.26	5.89558966828316e-05\\
78.27	5.77533022771328e-05\\
78.28	5.6555207163746e-05\\
78.29	5.53617581622069e-05\\
78.3	5.41731065107215e-05\\
78.31	5.29894080023002e-05\\
78.32	5.18108231250865e-05\\
78.33	5.06375172071548e-05\\
78.34	4.94696605658467e-05\\
78.35	4.83074286617736e-05\\
78.36	4.71510022577188e-05\\
78.37	4.60005675824646e-05\\
78.38	4.48563164998448e-05\\
78.39	4.37184466830845e-05\\
78.4	4.25871617946317e-05\\
78.41	4.14626716716739e-05\\
78.42	4.03451925174818e-05\\
78.43	3.92349470987978e-05\\
78.44	3.81321649494393e-05\\
78.45	3.70370825803386e-05\\
78.46	3.59499436962119e-05\\
78.47	3.4870999419033e-05\\
78.48	3.38005085186305e-05\\
78.49	3.27387376504971e-05\\
78.5	3.16859616011564e-05\\
78.51	3.06424635412843e-05\\
78.52	2.96085352867806e-05\\
78.53	2.85844775681689e-05\\
78.54	2.75706003084724e-05\\
78.55	2.65672229098537e-05\\
78.56	2.55746745494029e-05\\
78.57	2.45932944841886e-05\\
78.58	2.36234323660357e-05\\
78.59	2.26654485662361e-05\\
78.6	2.17197145105479e-05\\
78.61	2.07866130247751e-05\\
78.62	1.98665386913384e-05\\
78.63	1.89598982170917e-05\\
78.64	1.80671108128156e-05\\
78.65	1.71886085847117e-05\\
78.66	1.63248369383293e-05\\
78.67	1.54762549952475e-05\\
78.68	1.4643336022982e-05\\
78.69	1.38265678785252e-05\\
78.7	1.30264534658857e-05\\
78.71	1.2243511208189e-05\\
78.72	1.14782755346928e-05\\
78.73	1.07312973832395e-05\\
78.74	1.0003144718617e-05\\
78.75	9.29440306736418e-06\\
78.76	8.6056760694956e-06\\
78.77	7.93758604768213e-06\\
78.78	7.2907745945032e-06\\
78.79	6.66590317823998e-06\\
78.8	6.06365376788071e-06\\
78.81	5.48472947787919e-06\\
78.82	4.92985523334842e-06\\
78.83	4.39977845624953e-06\\
78.84	3.89526977333937e-06\\
78.85	3.41712374645961e-06\\
78.86	2.96615962591847e-06\\
78.87	2.54322212769366e-06\\
78.88	2.14918223516262e-06\\
78.89	1.7849380261753e-06\\
78.9	1.45141552619112e-06\\
78.91	1.14956958838057e-06\\
78.92	8.80384801430281e-07\\
78.93	6.44876426014393e-07\\
78.94	4.4409136070768e-07\\
78.95	2.79109138363876e-07\\
78.96	1.51042953824843e-07\\
78.97	6.10407239424321e-08\\
78.98	1.02861809209154e-08\\
78.99	1.73472347597681e-18\\
79	0\\
79.01	0\\
79.02	0\\
79.03	0\\
79.04	0\\
79.05	0\\
79.06	1.73472347597681e-18\\
79.07	1.73472347597681e-18\\
79.08	0\\
79.09	1.73472347597681e-18\\
79.1	0\\
79.11	0\\
79.12	0\\
79.13	1.73472347597681e-18\\
79.14	1.73472347597681e-18\\
79.15	0\\
79.16	0\\
79.17	0\\
79.18	0\\
79.19	0\\
79.2	0\\
79.21	0\\
79.22	1.73472347597681e-18\\
79.23	0\\
79.24	1.73472347597681e-18\\
79.25	1.73472347597681e-18\\
79.26	0\\
79.27	0\\
79.28	1.73472347597681e-18\\
79.29	0\\
79.3	1.73472347597681e-18\\
79.31	0\\
79.32	0\\
79.33	0\\
79.34	0\\
79.35	1.73472347597681e-18\\
79.36	0\\
79.37	0\\
79.38	0\\
79.39	1.73472347597681e-18\\
79.4	0\\
79.41	0\\
79.42	0\\
79.43	0\\
79.44	0\\
79.45	0\\
79.46	0\\
79.47	0\\
79.48	1.73472347597681e-18\\
79.49	0\\
79.5	0\\
79.51	0\\
79.52	1.73472347597681e-18\\
79.53	0\\
79.54	0\\
79.55	1.73472347597681e-18\\
79.56	0\\
79.57	0\\
79.58	0\\
79.59	0\\
79.6	0\\
79.61	0\\
79.62	1.73472347597681e-18\\
79.63	0\\
79.64	0\\
79.65	1.73472347597681e-18\\
79.66	0\\
79.67	0\\
79.68	0\\
79.69	0\\
79.7	1.73472347597681e-18\\
79.71	1.73472347597681e-18\\
79.72	0\\
79.73	0\\
79.74	1.73472347597681e-18\\
79.75	0\\
79.76	0\\
79.77	0\\
79.78	0\\
79.79	0\\
79.8	0\\
79.81	0\\
79.82	0\\
79.83	0\\
79.84	0\\
79.85	1.73472347597681e-18\\
79.86	1.73472347597681e-18\\
79.87	1.73472347597681e-18\\
79.88	0\\
79.89	0\\
79.9	1.73472347597681e-18\\
79.91	1.73472347597681e-18\\
79.92	0\\
79.93	0\\
79.94	0\\
79.95	0\\
79.96	0\\
79.97	0\\
79.98	0\\
79.99	1.73472347597681e-18\\
80	0\\
80.01	0\\
};
\addplot [color=mycolor1,solid]
  table[row sep=crcr]{%
80.01	0\\
80.02	0\\
80.03	0\\
80.04	1.73472347597681e-18\\
80.05	0\\
80.06	1.73472347597681e-18\\
80.07	0\\
80.08	0\\
80.09	1.73472347597681e-18\\
80.1	1.73472347597681e-18\\
80.11	0\\
80.12	0\\
80.13	0\\
80.14	0\\
80.15	0\\
80.16	1.73472347597681e-18\\
80.17	0\\
80.18	1.73472347597681e-18\\
80.19	0\\
80.2	1.73472347597681e-18\\
80.21	0\\
80.22	0\\
80.23	0\\
80.24	1.73472347597681e-18\\
80.25	0\\
80.26	0\\
80.27	1.73472347597681e-18\\
80.28	0\\
80.29	1.73472347597681e-18\\
80.3	0\\
80.31	0\\
80.32	0\\
80.33	0\\
80.34	0\\
80.35	0\\
80.36	1.73472347597681e-18\\
80.37	1.73472347597681e-18\\
80.38	0\\
80.39	1.73472347597681e-18\\
80.4	0\\
80.41	1.73472347597681e-18\\
80.42	0\\
80.43	1.73472347597681e-18\\
80.44	1.73472347597681e-18\\
80.45	0\\
80.46	0\\
80.47	0\\
80.48	0\\
80.49	1.73472347597681e-18\\
80.5	0\\
80.51	0\\
80.52	0\\
80.53	0\\
80.54	1.73472347597681e-18\\
80.55	0\\
80.56	0\\
80.57	1.73472347597681e-18\\
80.58	1.73472347597681e-18\\
80.59	0\\
80.6	1.73472347597681e-18\\
80.61	0\\
80.62	1.73472347597681e-18\\
80.63	0\\
80.64	1.73472347597681e-18\\
80.65	0\\
80.66	0\\
80.67	0\\
80.68	0\\
80.69	1.73472347597681e-18\\
80.7	0\\
80.71	1.73472347597681e-18\\
80.72	1.73472347597681e-18\\
80.73	0\\
80.74	1.73472347597681e-18\\
80.75	1.73472347597681e-18\\
80.76	0\\
80.77	0\\
80.78	0\\
80.79	0\\
80.8	0\\
80.81	0\\
80.82	0\\
80.83	0\\
80.84	1.73472347597681e-18\\
80.85	0\\
80.86	0\\
80.87	0\\
80.88	0\\
80.89	0\\
80.9	0\\
80.91	0\\
80.92	1.73472347597681e-18\\
80.93	0\\
80.94	1.73472347597681e-18\\
80.95	0\\
80.96	0\\
80.97	0\\
80.98	0\\
80.99	1.73472347597681e-18\\
81	1.73472347597681e-18\\
81.01	0\\
81.02	1.73472347597681e-18\\
81.03	0\\
81.04	0\\
81.05	1.73472347597681e-18\\
81.06	1.73472347597681e-18\\
81.07	0\\
81.08	0\\
81.09	0\\
81.1	1.73472347597681e-18\\
81.11	0\\
81.12	0\\
81.13	0\\
81.14	0\\
81.15	0\\
81.16	1.73472347597681e-18\\
81.17	1.73472347597681e-18\\
81.18	0\\
81.19	0\\
81.2	0\\
81.21	0\\
81.22	0\\
81.23	0\\
81.24	0\\
81.25	1.73472347597681e-18\\
81.26	0\\
81.27	0\\
81.28	0\\
81.29	0\\
81.3	0\\
81.31	1.73472347597681e-18\\
81.32	1.73472347597681e-18\\
81.33	0\\
81.34	0\\
81.35	1.73472347597681e-18\\
81.36	1.73472347597681e-18\\
81.37	0\\
81.38	0\\
81.39	1.73472347597681e-18\\
81.4	0\\
81.41	0\\
81.42	0\\
81.43	0\\
81.44	0\\
81.45	0\\
81.46	0\\
81.47	0\\
81.48	1.73472347597681e-18\\
81.49	0\\
81.5	0\\
81.51	0\\
81.52	0\\
81.53	0\\
81.54	0\\
81.55	0\\
81.56	0\\
81.57	0\\
81.58	0\\
81.59	0\\
81.6	1.73472347597681e-18\\
81.61	0\\
81.62	1.73472347597681e-18\\
81.63	0\\
81.64	0\\
81.65	0\\
81.66	0\\
81.67	1.73472347597681e-18\\
81.68	1.73472347597681e-18\\
81.69	0\\
81.7	0\\
81.71	0\\
81.72	1.73472347597681e-18\\
81.73	0\\
81.74	0\\
81.75	0\\
81.76	0\\
81.77	0\\
81.78	0\\
81.79	0\\
81.8	0\\
81.81	1.73472347597681e-18\\
81.82	0\\
81.83	0\\
81.84	0\\
81.85	0\\
81.86	0\\
81.87	1.73472347597681e-18\\
81.88	0\\
81.89	0\\
81.9	1.73472347597681e-18\\
81.91	0\\
81.92	0\\
81.93	1.73472347597681e-18\\
81.94	0\\
81.95	0\\
81.96	1.73472347597681e-18\\
81.97	1.73472347597681e-18\\
81.98	1.73472347597681e-18\\
81.99	0\\
82	0\\
82.01	1.73472347597681e-18\\
82.02	1.73472347597681e-18\\
82.03	0\\
82.04	1.73472347597681e-18\\
82.05	0\\
82.06	0\\
82.07	0\\
82.08	0\\
82.09	0\\
82.1	0\\
82.11	0\\
82.12	0\\
82.13	1.73472347597681e-18\\
82.14	0\\
82.15	1.73472347597681e-18\\
82.16	1.73472347597681e-18\\
82.17	0\\
82.18	0\\
82.19	0\\
82.2	1.73472347597681e-18\\
82.21	0\\
82.22	0\\
82.23	1.73472347597681e-18\\
82.24	1.73472347597681e-18\\
82.25	1.73472347597681e-18\\
82.26	0\\
82.27	0\\
82.28	0\\
82.29	0\\
82.3	1.73472347597681e-18\\
82.31	0\\
82.32	0\\
82.33	0\\
82.34	0\\
82.35	0\\
82.36	0\\
82.37	0\\
82.38	0\\
82.39	1.73472347597681e-18\\
82.4	0\\
82.41	0\\
82.42	0\\
82.43	1.73472347597681e-18\\
82.44	0\\
82.45	0\\
82.46	0\\
82.47	0\\
82.48	0\\
82.49	0\\
82.5	0\\
82.51	0\\
82.52	0\\
82.53	0\\
82.54	0\\
82.55	0\\
82.56	0\\
82.57	0\\
82.58	1.73472347597681e-18\\
82.59	1.73472347597681e-18\\
82.6	0\\
82.61	0\\
82.62	0\\
82.63	1.73472347597681e-18\\
82.64	0\\
82.65	0\\
82.66	0\\
82.67	1.73472347597681e-18\\
82.68	0\\
82.69	1.73472347597681e-18\\
82.7	0\\
82.71	0\\
82.72	0\\
82.73	0\\
82.74	0\\
82.75	0\\
82.76	0\\
82.77	0\\
82.78	1.73472347597681e-18\\
82.79	0\\
82.8	0\\
82.81	0\\
82.82	1.73472347597681e-18\\
82.83	1.73472347597681e-18\\
82.84	1.73472347597681e-18\\
82.85	1.73472347597681e-18\\
82.86	0\\
82.87	0\\
82.88	0\\
82.89	0\\
82.9	0\\
82.91	1.73472347597681e-18\\
82.92	0\\
82.93	1.73472347597681e-18\\
82.94	0\\
82.95	0\\
82.96	0\\
82.97	1.73472347597681e-18\\
82.98	0\\
82.99	0\\
83	0\\
83.01	0\\
83.02	0\\
83.03	0\\
83.04	0\\
83.05	0\\
83.06	1.73472347597681e-18\\
83.07	1.73472347597681e-18\\
83.08	1.73472347597681e-18\\
83.09	0\\
83.1	1.73472347597681e-18\\
83.11	0\\
83.12	1.73472347597681e-18\\
83.13	0\\
83.14	0\\
83.15	0\\
83.16	1.73472347597681e-18\\
83.17	0\\
83.18	0\\
83.19	0\\
83.2	1.73472347597681e-18\\
83.21	1.73472347597681e-18\\
83.22	0\\
83.23	0\\
83.24	0\\
83.25	0\\
83.26	0\\
83.27	1.73472347597681e-18\\
83.28	0\\
83.29	1.73472347597681e-18\\
83.3	0\\
83.31	0\\
83.32	0\\
83.33	0\\
83.34	0\\
83.35	0\\
83.36	1.73472347597681e-18\\
83.37	0\\
83.38	0\\
83.39	0\\
83.4	0\\
83.41	1.73472347597681e-18\\
83.42	0\\
83.43	0\\
83.44	1.73472347597681e-18\\
83.45	0\\
83.46	0\\
83.47	0\\
83.48	1.73472347597681e-18\\
83.49	0\\
83.5	1.73472347597681e-18\\
83.51	1.73472347597681e-18\\
83.52	0\\
83.53	0\\
83.54	1.73472347597681e-18\\
83.55	0\\
83.56	0\\
83.57	0\\
83.58	0\\
83.59	0\\
83.6	0\\
83.61	1.73472347597681e-18\\
83.62	1.73472347597681e-18\\
83.63	0\\
83.64	0\\
83.65	0\\
83.66	0\\
83.67	1.73472347597681e-18\\
83.68	1.73472347597681e-18\\
83.69	0\\
83.7	0\\
83.71	0\\
83.72	0\\
83.73	1.73472347597681e-18\\
83.74	0\\
83.75	1.73472347597681e-18\\
83.76	1.73472347597681e-18\\
83.77	1.73472347597681e-18\\
83.78	0\\
83.79	0\\
83.8	0\\
83.81	0\\
83.82	0\\
83.83	0\\
83.84	0\\
83.85	1.73472347597681e-18\\
83.86	0\\
83.87	0\\
83.88	1.73472347597681e-18\\
83.89	0\\
83.9	0\\
83.91	0\\
83.92	0\\
83.93	0\\
83.94	0\\
83.95	1.73472347597681e-18\\
83.96	1.73472347597681e-18\\
83.97	0\\
83.98	0\\
83.99	0\\
84	0\\
84.01	0\\
84.02	0\\
84.03	1.73472347597681e-18\\
84.04	0\\
84.05	0\\
84.06	1.73472347597681e-18\\
84.07	0\\
84.08	0\\
84.09	0\\
84.1	0\\
84.11	1.73472347597681e-18\\
84.12	0\\
84.13	0\\
84.14	0\\
84.15	1.73472347597681e-18\\
84.16	0\\
84.17	0\\
84.18	0\\
84.19	0\\
84.2	0\\
84.21	0\\
84.22	0\\
84.23	0\\
84.24	0\\
84.25	0\\
84.26	0\\
84.27	0\\
84.28	0\\
84.29	0\\
84.3	0\\
84.31	0\\
84.32	0\\
84.33	1.73472347597681e-18\\
84.34	0\\
84.35	0\\
84.36	1.73472347597681e-18\\
84.37	0\\
84.38	1.73472347597681e-18\\
84.39	0\\
84.4	0\\
84.41	1.73472347597681e-18\\
84.42	0\\
84.43	0\\
84.44	0\\
84.45	0\\
84.46	1.73472347597681e-18\\
84.47	0\\
84.48	0\\
84.49	0\\
84.5	0\\
84.51	1.73472347597681e-18\\
84.52	0\\
84.53	0\\
84.54	0\\
84.55	1.73472347597681e-18\\
84.56	0\\
84.57	0\\
84.58	1.73472347597681e-18\\
84.59	0\\
84.6	1.73472347597681e-18\\
84.61	1.73472347597681e-18\\
84.62	1.73472347597681e-18\\
84.63	0\\
84.64	1.73472347597681e-18\\
84.65	0\\
84.66	0\\
84.67	0\\
84.68	0\\
84.69	0\\
84.7	0\\
84.71	0\\
84.72	0\\
84.73	1.73472347597681e-18\\
84.74	0\\
84.75	0\\
84.76	1.73472347597681e-18\\
84.77	0\\
84.78	0\\
84.79	0\\
84.8	0\\
84.81	0\\
84.82	0\\
84.83	0\\
84.84	1.73472347597681e-18\\
84.85	0\\
84.86	0\\
84.87	0\\
84.88	0\\
84.89	0\\
84.9	0\\
84.91	0\\
84.92	0\\
84.93	1.73472347597681e-18\\
84.94	0\\
84.95	1.73472347597681e-18\\
84.96	1.73472347597681e-18\\
84.97	0\\
84.98	1.73472347597681e-18\\
84.99	0\\
85	0\\
85.01	0\\
85.02	0\\
85.03	0\\
85.04	0\\
85.05	0\\
85.06	0\\
85.07	0\\
85.08	0\\
85.09	1.73472347597681e-18\\
85.1	0\\
85.11	0\\
85.12	0\\
85.13	1.73472347597681e-18\\
85.14	0\\
85.15	0\\
85.16	0\\
85.17	0\\
85.18	0\\
85.19	0\\
85.2	0\\
85.21	1.73472347597681e-18\\
85.22	0\\
85.23	1.73472347597681e-18\\
85.24	0\\
85.25	0\\
85.26	0\\
85.27	1.73472347597681e-18\\
85.28	0\\
85.29	0\\
85.3	0\\
85.31	0\\
85.32	1.73472347597681e-18\\
85.33	0\\
85.34	1.73472347597681e-18\\
85.35	0\\
85.36	0\\
85.37	0\\
85.38	0\\
85.39	0\\
85.4	1.73472347597681e-18\\
85.41	0\\
85.42	0\\
85.43	0\\
85.44	0\\
85.45	1.73472347597681e-18\\
85.46	0\\
85.47	0\\
85.48	0\\
85.49	0\\
85.5	0\\
85.51	0\\
85.52	0\\
85.53	0\\
85.54	1.73472347597681e-18\\
85.55	0\\
85.56	0\\
85.57	1.73472347597681e-18\\
85.58	0\\
85.59	0\\
85.6	0\\
85.61	0\\
85.62	0\\
85.63	1.73472347597681e-18\\
85.64	0\\
85.65	0\\
85.66	0\\
85.67	0\\
85.68	0\\
85.69	1.73472347597681e-18\\
85.7	0\\
85.71	0\\
85.72	1.73472347597681e-18\\
85.73	0\\
85.74	0\\
85.75	0\\
85.76	0\\
85.77	1.73472347597681e-18\\
85.78	0\\
85.79	0\\
85.8	1.73472347597681e-18\\
85.81	0\\
85.82	0\\
85.83	0\\
85.84	0\\
85.85	0\\
85.86	1.73472347597681e-18\\
85.87	0\\
85.88	0\\
85.89	1.73472347597681e-18\\
85.9	1.73472347597681e-18\\
85.91	0\\
85.92	1.73472347597681e-18\\
85.93	0\\
85.94	1.73472347597681e-18\\
85.95	0\\
85.96	0\\
85.97	0\\
85.98	0\\
85.99	0\\
86	1.73472347597681e-18\\
86.01	1.73472347597681e-18\\
86.02	1.73472347597681e-18\\
86.03	0\\
86.04	0\\
86.05	0\\
86.06	0\\
86.07	1.73472347597681e-18\\
86.08	0\\
86.09	0\\
86.1	0\\
86.11	0\\
86.12	0\\
86.13	0\\
86.14	0\\
86.15	0\\
86.16	0\\
86.17	0\\
86.18	0\\
86.19	0\\
86.2	0\\
86.21	0\\
86.22	0\\
86.23	0\\
86.24	0\\
86.25	0\\
86.26	0\\
86.27	0\\
86.28	1.73472347597681e-18\\
86.29	0\\
86.3	1.73472347597681e-18\\
86.31	1.73472347597681e-18\\
86.32	0\\
86.33	1.73472347597681e-18\\
86.34	0\\
86.35	0\\
86.36	1.73472347597681e-18\\
86.37	0\\
86.38	0\\
86.39	1.73472347597681e-18\\
86.4	1.73472347597681e-18\\
86.41	0\\
86.42	0\\
86.43	0\\
86.44	1.73472347597681e-18\\
86.45	0\\
86.46	0\\
86.47	0\\
86.48	1.73472347597681e-18\\
86.49	0\\
86.5	0\\
86.51	0\\
86.52	0\\
86.53	0\\
86.54	0\\
86.55	0\\
86.56	0\\
86.57	0\\
86.58	0\\
86.59	0\\
86.6	0\\
86.61	0\\
86.62	0\\
86.63	0\\
86.64	1.73472347597681e-18\\
86.65	0\\
86.66	0\\
86.67	0\\
86.68	0\\
86.69	0\\
86.7	0\\
86.71	0\\
86.72	0\\
86.73	1.73472347597681e-18\\
86.74	0\\
86.75	0\\
86.76	0\\
86.77	1.73472347597681e-18\\
86.78	0\\
86.79	0\\
86.8	0\\
86.81	0\\
86.82	0\\
86.83	0\\
86.84	0\\
86.85	0\\
86.86	0\\
86.87	0\\
86.88	0\\
86.89	1.73472347597681e-18\\
86.9	0\\
86.91	0\\
86.92	1.73472347597681e-18\\
86.93	0\\
86.94	0\\
86.95	0\\
86.96	0\\
86.97	1.73472347597681e-18\\
86.98	0\\
86.99	0\\
87	0\\
87.01	0\\
87.02	0\\
87.03	1.73472347597681e-18\\
87.04	0\\
87.05	0\\
87.06	1.73472347597681e-18\\
87.07	1.73472347597681e-18\\
87.08	1.73472347597681e-18\\
87.09	1.73472347597681e-18\\
87.1	1.73472347597681e-18\\
87.11	0\\
87.12	0\\
87.13	0\\
87.14	1.73472347597681e-18\\
87.15	0\\
87.16	0\\
87.17	1.73472347597681e-18\\
87.18	1.73472347597681e-18\\
87.19	0\\
87.2	0\\
87.21	0\\
87.22	0\\
87.23	0\\
87.24	1.73472347597681e-18\\
87.25	1.73472347597681e-18\\
87.26	0\\
87.27	0\\
87.28	0\\
87.29	1.73472347597681e-18\\
87.3	0\\
87.31	0\\
87.32	0\\
87.33	0\\
87.34	0\\
87.35	0\\
87.36	0\\
87.37	0\\
87.38	0\\
87.39	1.73472347597681e-18\\
87.4	0\\
87.41	0\\
87.42	0\\
87.43	0\\
87.44	0\\
87.45	0\\
87.46	0\\
87.47	0\\
87.48	0\\
87.49	0\\
87.5	0\\
87.51	0\\
87.52	1.73472347597681e-18\\
87.53	1.73472347597681e-18\\
87.54	0\\
87.55	0\\
87.56	1.73472347597681e-18\\
87.57	0\\
87.58	0\\
87.59	0\\
87.6	0\\
87.61	0\\
87.62	0\\
87.63	0\\
87.64	0\\
87.65	0\\
87.66	0\\
87.67	0\\
87.68	1.73472347597681e-18\\
87.69	0\\
87.7	0\\
87.71	0\\
87.72	0\\
87.73	0\\
87.74	1.73472347597681e-18\\
87.75	1.73472347597681e-18\\
87.76	1.73472347597681e-18\\
87.77	0\\
87.78	1.73472347597681e-18\\
87.79	0\\
87.8	0\\
87.81	0\\
87.82	1.73472347597681e-18\\
87.83	1.73472347597681e-18\\
87.84	0\\
87.85	1.73472347597681e-18\\
87.86	0\\
87.87	0\\
87.88	0\\
87.89	0\\
87.9	0\\
87.91	0\\
87.92	1.73472347597681e-18\\
87.93	0\\
87.94	0\\
87.95	0\\
87.96	0\\
87.97	1.73472347597681e-18\\
87.98	0\\
87.99	0\\
88	0\\
88.01	1.73472347597681e-18\\
88.02	1.73472347597681e-18\\
88.03	0\\
88.04	1.73472347597681e-18\\
88.05	0\\
88.06	0\\
88.07	1.73472347597681e-18\\
88.08	1.73472347597681e-18\\
88.09	0\\
88.1	0\\
88.11	0\\
88.12	1.73472347597681e-18\\
88.13	0\\
88.14	0\\
88.15	1.73472347597681e-18\\
88.16	0\\
88.17	1.73472347597681e-18\\
88.18	0\\
88.19	0\\
88.2	0\\
88.21	0\\
88.22	0\\
88.23	0\\
88.24	0\\
88.25	0\\
88.26	0\\
88.27	0\\
88.28	0\\
88.29	1.73472347597681e-18\\
88.3	0\\
88.31	0\\
88.32	0\\
88.33	0\\
88.34	0\\
88.35	0\\
88.36	0\\
88.37	0\\
88.38	0\\
88.39	0\\
88.4	1.73472347597681e-18\\
88.41	0\\
88.42	0\\
88.43	0\\
88.44	1.73472347597681e-18\\
88.45	0\\
88.46	1.73472347597681e-18\\
88.47	1.73472347597681e-18\\
88.48	1.73472347597681e-18\\
88.49	1.73472347597681e-18\\
88.5	0\\
88.51	0\\
88.52	0\\
88.53	1.73472347597681e-18\\
88.54	0\\
88.55	0\\
88.56	1.73472347597681e-18\\
88.57	0\\
88.58	0\\
88.59	1.73472347597681e-18\\
88.6	0\\
88.61	0\\
88.62	1.73472347597681e-18\\
88.63	0\\
88.64	0\\
88.65	1.73472347597681e-18\\
88.66	1.73472347597681e-18\\
88.67	1.73472347597681e-18\\
88.68	0\\
88.69	0\\
88.7	0\\
88.71	0\\
88.72	0\\
88.73	1.73472347597681e-18\\
88.74	0\\
88.75	0\\
88.76	0\\
88.77	0\\
88.78	0\\
88.79	1.73472347597681e-18\\
88.8	0\\
88.81	0\\
88.82	1.73472347597681e-18\\
88.83	0\\
88.84	0\\
88.85	0\\
88.86	0\\
88.87	0\\
88.88	0\\
88.89	0\\
88.9	1.73472347597681e-18\\
88.91	0\\
88.92	0\\
88.93	0\\
88.94	0\\
88.95	0\\
88.96	0\\
88.97	0\\
88.98	0\\
88.99	1.73472347597681e-18\\
89	1.73472347597681e-18\\
89.01	1.73472347597681e-18\\
89.02	0\\
89.03	0\\
89.04	1.73472347597681e-18\\
89.05	0\\
89.06	0\\
89.07	0\\
89.08	0\\
89.09	0\\
89.1	0\\
89.11	0\\
89.12	0\\
89.13	1.73472347597681e-18\\
89.14	1.73472347597681e-18\\
89.15	0\\
89.16	0\\
89.17	1.73472347597681e-18\\
89.18	1.73472347597681e-18\\
89.19	0\\
89.2	0\\
89.21	0\\
89.22	1.73472347597681e-18\\
89.23	0\\
89.24	0\\
89.25	0\\
89.26	0\\
89.27	0\\
89.28	0\\
89.29	1.73472347597681e-18\\
89.3	0\\
89.31	0\\
89.32	0\\
89.33	0\\
89.34	0\\
89.35	1.73472347597681e-18\\
89.36	0\\
89.37	1.73472347597681e-18\\
89.38	0\\
89.39	0\\
89.4	1.73472347597681e-18\\
89.41	0\\
89.42	1.73472347597681e-18\\
89.43	0\\
89.44	0\\
89.45	0\\
89.46	1.73472347597681e-18\\
89.47	0\\
89.48	0\\
89.49	0\\
89.5	0\\
89.51	0\\
89.52	1.73472347597681e-18\\
89.53	0\\
89.54	0\\
89.55	1.73472347597681e-18\\
89.56	1.73472347597681e-18\\
89.57	0\\
89.58	0\\
89.59	0\\
89.6	1.73472347597681e-18\\
89.61	0\\
89.62	1.73472347597681e-18\\
89.63	1.73472347597681e-18\\
89.64	0\\
89.65	0\\
89.66	0\\
89.67	0\\
89.68	0\\
89.69	0\\
89.7	0\\
89.71	1.73472347597681e-18\\
89.72	1.73472347597681e-18\\
89.73	0\\
89.74	1.73472347597681e-18\\
89.75	0\\
89.76	0\\
89.77	1.73472347597681e-18\\
89.78	0\\
89.79	1.73472347597681e-18\\
89.8	1.73472347597681e-18\\
89.81	0\\
89.82	0\\
89.83	1.73472347597681e-18\\
89.84	1.73472347597681e-18\\
89.85	1.73472347597681e-18\\
89.86	1.73472347597681e-18\\
89.87	0\\
89.88	1.73472347597681e-18\\
89.89	0\\
89.9	0\\
89.91	0\\
89.92	1.73472347597681e-18\\
89.93	1.73472347597681e-18\\
89.94	0\\
89.95	0\\
89.96	0\\
89.97	1.73472347597681e-18\\
89.98	1.73472347597681e-18\\
89.99	0\\
90	0\\
90.01	0\\
90.02	1.73472347597681e-18\\
90.03	0\\
90.04	0\\
90.05	0\\
90.06	1.73472347597681e-18\\
90.07	0\\
90.08	1.73472347597681e-18\\
90.09	1.73472347597681e-18\\
90.1	1.73472347597681e-18\\
90.11	0\\
90.12	0\\
90.13	0\\
90.14	0\\
90.15	1.73472347597681e-18\\
90.16	0\\
90.17	1.73472347597681e-18\\
90.18	1.73472347597681e-18\\
90.19	0\\
90.2	0\\
90.21	0\\
90.22	0\\
90.23	0\\
90.24	0\\
90.25	0\\
90.26	0\\
90.27	0\\
90.28	0\\
90.29	0\\
90.3	0\\
90.31	0\\
90.32	0\\
90.33	0\\
90.34	1.73472347597681e-18\\
90.35	0\\
90.36	0\\
90.37	0\\
90.38	0\\
90.39	0\\
90.4	1.73472347597681e-18\\
90.41	0\\
90.42	0\\
90.43	1.73472347597681e-18\\
90.44	0\\
90.45	0\\
90.46	0\\
90.47	1.73472347597681e-18\\
90.48	0\\
90.49	1.73472347597681e-18\\
90.5	1.73472347597681e-18\\
90.51	0\\
90.52	0\\
90.53	0\\
90.54	0\\
90.55	0\\
90.56	0\\
90.57	0\\
90.58	0\\
90.59	0\\
90.6	0\\
90.61	1.73472347597681e-18\\
90.62	1.73472347597681e-18\\
90.63	0\\
90.64	0\\
90.65	0\\
90.66	0\\
90.67	0\\
90.68	0\\
90.69	0\\
90.7	1.73472347597681e-18\\
90.71	0\\
90.72	0\\
90.73	0\\
90.74	0\\
90.75	0\\
90.76	0\\
90.77	0\\
90.78	0\\
90.79	0\\
90.8	0\\
90.81	0\\
90.82	0\\
90.83	0\\
90.84	1.73472347597681e-18\\
90.85	0\\
90.86	1.73472347597681e-18\\
90.87	0\\
90.88	0\\
90.89	0\\
90.9	1.73472347597681e-18\\
90.91	0\\
90.92	0\\
90.93	0\\
90.94	0\\
90.95	0\\
90.96	0\\
90.97	1.73472347597681e-18\\
90.98	0\\
90.99	0\\
91	0\\
91.01	1.73472347597681e-18\\
91.02	1.73472347597681e-18\\
91.03	0\\
91.04	0\\
91.05	0\\
91.06	0\\
91.07	1.73472347597681e-18\\
91.08	0\\
91.09	0\\
91.1	1.73472347597681e-18\\
91.11	0\\
91.12	0\\
91.13	1.73472347597681e-18\\
91.14	0\\
91.15	0\\
91.16	0\\
91.17	0\\
91.18	0\\
91.19	0\\
91.2	0\\
91.21	0\\
91.22	1.73472347597681e-18\\
91.23	0\\
91.24	0\\
91.25	0\\
91.26	1.73472347597681e-18\\
91.27	0\\
91.28	0\\
91.29	1.73472347597681e-18\\
91.3	0\\
91.31	1.73472347597681e-18\\
91.32	0\\
91.33	0\\
91.34	0\\
91.35	0\\
91.36	0\\
91.37	0\\
91.38	0\\
91.39	1.73472347597681e-18\\
91.4	1.73472347597681e-18\\
91.41	1.73472347597681e-18\\
91.42	0\\
91.43	1.73472347597681e-18\\
91.44	0\\
91.45	0\\
91.46	0\\
91.47	1.73472347597681e-18\\
91.48	0\\
91.49	0\\
91.5	1.73472347597681e-18\\
91.51	1.73472347597681e-18\\
91.52	0\\
91.53	0\\
91.54	0\\
91.55	1.73472347597681e-18\\
91.56	0\\
91.57	0\\
91.58	0\\
91.59	0\\
91.6	1.73472347597681e-18\\
91.61	1.73472347597681e-18\\
91.62	0\\
91.63	0\\
91.64	0\\
91.65	0\\
91.66	0\\
91.67	0\\
91.68	0\\
91.69	0\\
91.7	1.73472347597681e-18\\
91.71	0\\
91.72	0\\
91.73	0\\
91.74	0\\
91.75	0\\
91.76	1.73472347597681e-18\\
91.77	0\\
91.78	0\\
91.79	0\\
91.8	0\\
91.81	0\\
91.82	0\\
91.83	1.73472347597681e-18\\
91.84	0\\
91.85	0\\
91.86	0\\
91.87	0\\
91.88	0\\
91.89	0\\
91.9	0\\
91.91	0\\
91.92	0\\
91.93	0\\
91.94	0\\
91.95	1.73472347597681e-18\\
91.96	0\\
91.97	1.73472347597681e-18\\
91.98	0\\
91.99	0\\
92	0\\
92.01	0\\
92.02	0\\
92.03	0\\
92.04	1.73472347597681e-18\\
92.05	0\\
92.06	1.73472347597681e-18\\
92.07	0\\
92.08	0\\
92.09	0\\
92.1	0\\
92.11	1.73472347597681e-18\\
92.12	0\\
92.13	0\\
92.14	0\\
92.15	0\\
92.16	1.73472347597681e-18\\
92.17	0\\
92.18	0\\
92.19	1.73472347597681e-18\\
92.2	1.73472347597681e-18\\
92.21	0\\
92.22	0\\
92.23	0\\
92.24	0\\
92.25	0\\
92.26	1.73472347597681e-18\\
92.27	0\\
92.28	0\\
92.29	0\\
92.3	1.73472347597681e-18\\
92.31	0\\
92.32	1.73472347597681e-18\\
92.33	1.73472347597681e-18\\
92.34	0\\
92.35	0\\
92.36	0\\
92.37	0\\
92.38	0\\
92.39	1.73472347597681e-18\\
92.4	0\\
92.41	1.73472347597681e-18\\
92.42	0\\
92.43	0\\
92.44	0\\
92.45	0\\
92.46	0\\
92.47	0\\
92.48	0\\
92.49	0\\
92.5	0\\
92.51	0\\
92.52	0\\
92.53	0\\
92.54	0\\
92.55	0\\
92.56	1.73472347597681e-18\\
92.57	1.73472347597681e-18\\
92.58	0\\
92.59	0\\
92.6	1.73472347597681e-18\\
92.61	1.73472347597681e-18\\
92.62	0\\
92.63	0\\
92.64	1.73472347597681e-18\\
92.65	0\\
92.66	0\\
92.67	0\\
92.68	1.73472347597681e-18\\
92.69	1.73472347597681e-18\\
92.7	1.73472347597681e-18\\
92.71	0\\
92.72	1.73472347597681e-18\\
92.73	1.73472347597681e-18\\
92.74	0\\
92.75	0\\
92.76	0\\
92.77	1.73472347597681e-18\\
92.78	0\\
92.79	0\\
92.8	0\\
92.81	1.73472347597681e-18\\
92.82	1.73472347597681e-18\\
92.83	0\\
92.84	1.73472347597681e-18\\
92.85	1.73472347597681e-18\\
92.86	0\\
92.87	1.73472347597681e-18\\
92.88	1.73472347597681e-18\\
92.89	1.73472347597681e-18\\
92.9	1.73472347597681e-18\\
92.91	0\\
92.92	1.73472347597681e-18\\
92.93	0\\
92.94	0\\
92.95	0\\
92.96	0\\
92.97	1.73472347597681e-18\\
92.98	1.73472347597681e-18\\
92.99	1.73472347597681e-18\\
93	0\\
93.01	0\\
93.02	0\\
93.03	1.73472347597681e-18\\
93.04	1.73472347597681e-18\\
93.05	0\\
93.06	0\\
93.07	0\\
93.08	0\\
93.09	0\\
93.1	0\\
93.11	1.73472347597681e-18\\
93.12	0\\
93.13	1.73472347597681e-18\\
93.14	1.73472347597681e-18\\
93.15	0\\
93.16	1.73472347597681e-18\\
93.17	1.73472347597681e-18\\
93.18	0\\
93.19	0\\
93.2	1.73472347597681e-18\\
93.21	0\\
93.22	0\\
93.23	0\\
93.24	0\\
93.25	0\\
93.26	0\\
93.27	0\\
93.28	0\\
93.29	1.73472347597681e-18\\
93.3	0\\
93.31	0\\
93.32	1.73472347597681e-18\\
93.33	1.73472347597681e-18\\
93.34	0\\
93.35	0\\
93.36	1.73472347597681e-18\\
93.37	0\\
93.38	0\\
93.39	0\\
93.4	0\\
93.41	0\\
93.42	0\\
93.43	0\\
93.44	0\\
93.45	0\\
93.46	0\\
93.47	0\\
93.48	0\\
93.49	1.73472347597681e-18\\
93.5	0\\
93.51	0\\
93.52	0\\
93.53	0\\
93.54	0\\
93.55	0\\
93.56	1.73472347597681e-18\\
93.57	0\\
93.58	0\\
93.59	0\\
93.6	0\\
93.61	0\\
93.62	0\\
93.63	0\\
93.64	0\\
93.65	1.73472347597681e-18\\
93.66	1.73472347597681e-18\\
93.67	1.73472347597681e-18\\
93.68	0\\
93.69	0\\
93.7	0\\
93.71	0\\
93.72	1.73472347597681e-18\\
93.73	0\\
93.74	0\\
93.75	0\\
93.76	0\\
93.77	1.73472347597681e-18\\
93.78	0\\
93.79	0\\
93.8	1.73472347597681e-18\\
93.81	0\\
93.82	1.73472347597681e-18\\
93.83	1.73472347597681e-18\\
93.84	0\\
93.85	1.73472347597681e-18\\
93.86	1.73472347597681e-18\\
93.87	0\\
93.88	1.73472347597681e-18\\
93.89	0\\
93.9	0\\
93.91	0\\
93.92	0\\
93.93	0\\
93.94	1.73472347597681e-18\\
93.95	0\\
93.96	0\\
93.97	0\\
93.98	0\\
93.99	1.73472347597681e-18\\
94	0\\
94.01	1.73472347597681e-18\\
94.02	0\\
94.03	1.73472347597681e-18\\
94.04	1.73472347597681e-18\\
94.05	0\\
94.06	0\\
94.07	0\\
94.08	0\\
94.09	0\\
94.1	0\\
94.11	0\\
94.12	1.73472347597681e-18\\
94.13	0\\
94.14	0\\
94.15	1.73472347597681e-18\\
94.16	0\\
94.17	1.73472347597681e-18\\
94.18	0\\
94.19	0\\
94.2	1.73472347597681e-18\\
94.21	1.73472347597681e-18\\
94.22	1.73472347597681e-18\\
94.23	1.73472347597681e-18\\
94.24	0\\
94.25	1.73472347597681e-18\\
94.26	0\\
94.27	1.73472347597681e-18\\
94.28	0\\
94.29	1.73472347597681e-18\\
94.3	1.73472347597681e-18\\
94.31	0\\
94.32	0\\
94.33	0\\
94.34	1.73472347597681e-18\\
94.35	0\\
94.36	0\\
94.37	0\\
94.38	0\\
94.39	1.73472347597681e-18\\
94.4	0\\
94.41	0\\
94.42	0\\
94.43	0\\
94.44	1.73472347597681e-18\\
94.45	1.73472347597681e-18\\
94.46	1.73472347597681e-18\\
94.47	0\\
94.48	0\\
94.49	1.73472347597681e-18\\
94.5	0\\
94.51	0\\
94.52	0\\
94.53	0\\
94.54	1.73472347597681e-18\\
94.55	0\\
94.56	0\\
94.57	0\\
94.58	0\\
94.59	1.73472347597681e-18\\
94.6	0\\
94.61	0\\
94.62	0\\
94.63	0\\
94.64	0\\
94.65	0\\
94.66	0\\
94.67	0\\
94.68	1.73472347597681e-18\\
94.69	0\\
94.7	0\\
94.71	1.73472347597681e-18\\
94.72	1.73472347597681e-18\\
94.73	1.73472347597681e-18\\
94.74	1.73472347597681e-18\\
94.75	1.73472347597681e-18\\
94.76	0\\
94.77	0\\
94.78	1.73472347597681e-18\\
94.79	0\\
94.8	0\\
94.81	0\\
94.82	0\\
94.83	0\\
94.84	0\\
94.85	1.73472347597681e-18\\
94.86	1.73472347597681e-18\\
94.87	1.73472347597681e-18\\
94.88	1.73472347597681e-18\\
94.89	0\\
94.9	0\\
94.91	0\\
94.92	0\\
94.93	0\\
94.94	0\\
94.95	1.73472347597681e-18\\
94.96	0\\
94.97	0\\
94.98	0\\
94.99	0\\
95	1.73472347597681e-18\\
95.01	1.73472347597681e-18\\
95.02	0\\
95.03	0\\
95.04	0\\
95.05	1.73472347597681e-18\\
95.06	1.73472347597681e-18\\
95.07	1.73472347597681e-18\\
95.08	0\\
95.09	0\\
95.1	0\\
95.11	0\\
95.12	0\\
95.13	1.73472347597681e-18\\
95.14	0\\
95.15	0\\
95.16	1.73472347597681e-18\\
95.17	0\\
95.18	0\\
95.19	0\\
95.2	0\\
95.21	0\\
95.22	1.73472347597681e-18\\
95.23	0\\
95.24	1.73472347597681e-18\\
95.25	1.73472347597681e-18\\
95.26	1.73472347597681e-18\\
95.27	1.73472347597681e-18\\
95.28	1.73472347597681e-18\\
95.29	0\\
95.3	1.73472347597681e-18\\
95.31	1.73472347597681e-18\\
95.32	0\\
95.33	1.73472347597681e-18\\
95.34	0\\
95.35	1.73472347597681e-18\\
95.36	1.73472347597681e-18\\
95.37	0\\
95.38	0\\
95.39	1.73472347597681e-18\\
95.4	0\\
95.41	1.73472347597681e-18\\
95.42	0\\
95.43	0\\
95.44	1.73472347597681e-18\\
95.45	1.73472347597681e-18\\
95.46	0\\
95.47	0\\
95.48	0\\
95.49	0\\
95.5	1.73472347597681e-18\\
95.51	0\\
95.52	1.73472347597681e-18\\
95.53	0\\
95.54	1.73472347597681e-18\\
95.55	0\\
95.56	0\\
95.57	0\\
95.58	1.73472347597681e-18\\
95.59	1.73472347597681e-18\\
95.6	0\\
95.61	1.73472347597681e-18\\
95.62	1.73472347597681e-18\\
95.63	1.73472347597681e-18\\
95.64	1.73472347597681e-18\\
95.65	0\\
95.66	0\\
95.67	0\\
95.68	0\\
95.69	1.73472347597681e-18\\
95.7	0\\
95.71	0\\
95.72	1.73472347597681e-18\\
95.73	0\\
95.74	0\\
95.75	1.73472347597681e-18\\
95.76	1.73472347597681e-18\\
95.77	0\\
95.78	0\\
95.79	1.73472347597681e-18\\
95.8	0\\
95.81	0\\
95.82	1.73472347597681e-18\\
95.83	1.73472347597681e-18\\
95.84	0\\
95.85	1.73472347597681e-18\\
95.86	0\\
95.87	1.73472347597681e-18\\
95.88	0\\
95.89	0\\
95.9	0\\
95.91	0\\
95.92	0\\
95.93	0\\
95.94	0\\
95.95	1.73472347597681e-18\\
95.96	0\\
95.97	0\\
95.98	0\\
95.99	0\\
96	0\\
96.01	1.73472347597681e-18\\
96.02	1.73472347597681e-18\\
96.03	0\\
96.04	1.73472347597681e-18\\
96.05	0\\
96.06	0\\
96.07	0\\
96.08	1.73472347597681e-18\\
96.09	1.73472347597681e-18\\
96.1	0\\
96.11	0\\
96.12	1.73472347597681e-18\\
96.13	1.73472347597681e-18\\
96.14	1.73472347597681e-18\\
96.15	0\\
96.16	1.73472347597681e-18\\
96.17	1.73472347597681e-18\\
96.18	1.73472347597681e-18\\
96.19	1.73472347597681e-18\\
96.2	0\\
96.21	0\\
96.22	0\\
96.23	0\\
96.24	0\\
96.25	1.73472347597681e-18\\
96.26	1.73472347597681e-18\\
96.27	0\\
96.28	0\\
96.29	0\\
96.3	0\\
96.31	0\\
96.32	1.73472347597681e-18\\
96.33	1.73472347597681e-18\\
96.34	1.73472347597681e-18\\
96.35	1.73472347597681e-18\\
96.36	0\\
96.37	0\\
96.38	1.73472347597681e-18\\
96.39	1.73472347597681e-18\\
96.4	0\\
96.41	1.73472347597681e-18\\
96.42	0\\
96.43	1.73472347597681e-18\\
96.44	0\\
96.45	0\\
96.46	1.73472347597681e-18\\
96.47	0\\
96.48	1.73472347597681e-18\\
96.49	0\\
96.5	0\\
96.51	1.73472347597681e-18\\
96.52	1.73472347597681e-18\\
96.53	0\\
96.54	1.73472347597681e-18\\
96.55	0\\
96.56	0\\
96.57	0\\
96.58	1.73472347597681e-18\\
96.59	0\\
96.6	0\\
96.61	0\\
96.62	0\\
96.63	0\\
96.64	0\\
96.65	0\\
96.66	0\\
96.67	0\\
96.68	1.73472347597681e-18\\
96.69	0\\
96.7	1.73472347597681e-18\\
96.71	1.73472347597681e-18\\
96.72	0\\
96.73	0\\
96.74	0\\
96.75	0\\
96.76	0\\
96.77	1.73472347597681e-18\\
96.78	0\\
96.79	0\\
96.8	1.73472347597681e-18\\
96.81	0\\
96.82	1.73472347597681e-18\\
96.83	1.73472347597681e-18\\
96.84	1.73472347597681e-18\\
96.85	1.73472347597681e-18\\
96.86	0\\
96.87	0\\
96.88	1.73472347597681e-18\\
96.89	1.73472347597681e-18\\
96.9	0\\
96.91	1.73472347597681e-18\\
96.92	0\\
96.93	0\\
96.94	0\\
96.95	1.73472347597681e-18\\
96.96	1.73472347597681e-18\\
96.97	0\\
96.98	0\\
96.99	0\\
97	0\\
97.01	0\\
97.02	0\\
97.03	0\\
97.04	0\\
97.05	0\\
97.06	1.73472347597681e-18\\
97.07	0\\
97.08	1.73472347597681e-18\\
97.09	0\\
97.1	1.73472347597681e-18\\
97.11	0\\
97.12	1.73472347597681e-18\\
97.13	1.73472347597681e-18\\
97.14	1.73472347597681e-18\\
97.15	0\\
97.16	1.73472347597681e-18\\
97.17	0\\
97.18	0\\
97.19	0\\
97.2	0\\
97.21	0\\
97.22	0\\
97.23	0\\
97.24	0\\
97.25	0\\
97.26	0\\
97.27	0\\
97.28	0\\
97.29	0\\
97.3	0\\
97.31	1.73472347597681e-18\\
97.32	1.73472347597681e-18\\
97.33	1.73472347597681e-18\\
97.34	0\\
97.35	1.73472347597681e-18\\
97.36	0\\
97.37	1.73472347597681e-18\\
97.38	0\\
97.39	0\\
97.4	1.73472347597681e-18\\
97.41	0\\
97.42	0\\
97.43	0\\
97.44	1.73472347597681e-18\\
97.45	0\\
97.46	0\\
97.47	0\\
97.48	0\\
97.49	1.73472347597681e-18\\
97.5	1.73472347597681e-18\\
97.51	0\\
97.52	0\\
97.53	0\\
97.54	1.73472347597681e-18\\
97.55	0\\
97.56	0\\
97.57	0\\
97.58	0\\
97.59	0\\
97.6	1.73472347597681e-18\\
97.61	0\\
97.62	1.73472347597681e-18\\
97.63	0\\
97.64	1.73472347597681e-18\\
97.65	0\\
97.66	0\\
97.67	0\\
97.68	0\\
97.69	0\\
97.7	0\\
97.71	0\\
97.72	0\\
97.73	0\\
97.74	0\\
97.75	0\\
97.76	1.73472347597681e-18\\
97.77	1.73472347597681e-18\\
97.78	0\\
97.79	1.73472347597681e-18\\
97.8	0\\
97.81	0\\
97.82	0\\
97.83	0\\
97.84	0\\
97.85	0\\
97.86	0\\
97.87	0\\
97.88	0\\
97.89	0\\
97.9	0\\
97.91	0\\
97.92	0\\
97.93	0\\
97.94	0\\
97.95	1.73472347597681e-18\\
97.96	0\\
97.97	1.73472347597681e-18\\
97.98	1.73472347597681e-18\\
97.99	0\\
98	0\\
98.01	1.73472347597681e-18\\
98.02	0\\
98.03	0\\
98.04	0\\
98.05	0\\
98.06	0\\
98.07	0\\
98.08	0\\
98.09	1.73472347597681e-18\\
98.1	0\\
98.11	0\\
98.12	0\\
98.13	1.73472347597681e-18\\
98.14	0\\
98.15	0\\
98.16	0\\
98.17	1.73472347597681e-18\\
98.18	0\\
98.19	0\\
98.2	0\\
98.21	0\\
98.22	1.73472347597681e-18\\
98.23	0\\
98.24	1.73472347597681e-18\\
98.25	0\\
98.26	0\\
98.27	0\\
98.28	0\\
98.29	0\\
98.3	0\\
98.31	0\\
98.32	0\\
98.33	0\\
98.34	0\\
98.35	0\\
98.36	0\\
98.37	0\\
98.38	0\\
98.39	0\\
98.4	0\\
98.41	0\\
98.42	0\\
98.43	0\\
98.44	0\\
98.45	0\\
98.46	0\\
98.47	1.73472347597681e-18\\
98.48	0\\
98.49	0\\
98.5	0\\
98.51	1.73472347597681e-18\\
98.52	0\\
98.53	0\\
98.54	0\\
98.55	1.73472347597681e-18\\
98.56	1.73472347597681e-18\\
98.57	0\\
98.58	0\\
98.59	0\\
98.6	0\\
98.61	0\\
98.62	1.73472347597681e-18\\
98.63	0\\
98.64	0\\
98.65	0\\
98.66	0\\
98.67	0\\
98.68	0\\
98.69	0\\
98.7	0\\
98.71	0\\
98.72	0\\
98.73	1.73472347597681e-18\\
98.74	0\\
98.75	0\\
98.76	0\\
98.77	0\\
98.78	0\\
98.79	1.73472347597681e-18\\
98.8	0\\
98.81	0\\
98.82	1.73472347597681e-18\\
98.83	0\\
98.84	1.73472347597681e-18\\
98.85	0\\
98.86	0\\
98.87	1.73472347597681e-18\\
98.88	0\\
98.89	0\\
98.9	0\\
98.91	0\\
98.92	0\\
98.93	0\\
98.94	1.73472347597681e-18\\
98.95	0\\
98.96	0\\
98.97	1.73472347597681e-18\\
98.98	0\\
98.99	0\\
99	0\\
99.01	0\\
99.02	0\\
99.03	0\\
99.04	0\\
99.05	1.73472347597681e-18\\
99.06	0\\
99.07	0\\
99.08	1.73472347597681e-18\\
99.09	0\\
99.1	0\\
99.11	1.73472347597681e-18\\
99.12	0\\
99.13	0\\
99.14	0\\
99.15	0\\
99.16	0\\
99.17	1.73472347597681e-18\\
99.18	0\\
99.19	0\\
99.2	0\\
99.21	0\\
99.22	0\\
99.23	0\\
99.24	1.73472347597681e-18\\
99.25	0\\
99.26	0\\
99.27	1.73472347597681e-18\\
99.28	0\\
99.29	0\\
99.3	0\\
99.31	0\\
99.32	0\\
99.33	0\\
99.34	0\\
99.35	0\\
99.36	0\\
99.37	0\\
99.38	0\\
99.39	0\\
99.4	0\\
99.41	0\\
99.42	0\\
99.43	0\\
99.44	0\\
99.45	0\\
99.46	0\\
99.47	0\\
99.48	0\\
99.49	0\\
99.5	0\\
99.51	1.73472347597681e-18\\
99.52	0\\
99.53	0\\
99.54	0\\
99.55	0\\
99.56	0\\
99.57	0\\
99.58	0\\
99.59	0\\
99.6	0\\
99.61	1.73472347597681e-18\\
99.62	0\\
99.63	0\\
99.64	0\\
99.65	1.73472347597681e-18\\
99.66	0\\
99.67	1.73472347597681e-18\\
99.68	0\\
99.69	0\\
99.7	0\\
99.71	0\\
99.72	0\\
99.73	0\\
99.74	0\\
99.75	0\\
99.76	0\\
99.77	1.73472347597681e-18\\
99.78	0\\
99.79	0\\
99.8	0\\
99.81	0\\
99.82	0\\
99.83	0\\
99.84	0\\
99.85	0\\
99.86	0\\
99.87	0\\
99.88	0\\
99.89	0\\
99.9	0\\
99.91	0\\
99.92	0\\
99.93	0\\
99.94	0\\
99.95	0\\
99.96	0\\
99.97	0\\
99.98	0\\
99.99	0\\
100	0\\
};
\addlegendentry{$q=3$};

\addplot [color=green,solid,forget plot]
  table[row sep=crcr]{%
0.01	0.000436939440003583\\
0.02	0.000436921277715224\\
0.03	0.000436903102319499\\
0.04	0.000436884914176433\\
0.05	0.000436866713669996\\
0.06	0.000436848501209126\\
0.07	0.000436830277228836\\
0.08	0.000436812042191342\\
0.09	0.000436793796587212\\
0.1	0.000436775540936624\\
0.11	0.000436757275790581\\
0.12	0.000436739001732241\\
0.13	0.000436720719378302\\
0.14	0.000436702429380351\\
0.15	0.000436684132426398\\
0.16	0.000436665829242336\\
0.17	0.00043664752059356\\
0.18	0.000436629207286564\\
0.19	0.00043661089017067\\
0.2	0.000436592570139768\\
0.21	0.000436574248134151\\
0.22	0.000436555925142385\\
0.23	0.000436537602203297\\
0.24	0.000436519280408007\\
0.25	0.000436500960902026\\
0.26	0.000436482644887444\\
0.27	0.000436464333625212\\
0.28	0.000436446028437485\\
0.29	0.000436427730710067\\
0.3	0.000436409441894926\\
0.31	0.000436391163512824\\
0.32	0.000436372897156053\\
0.33	0.000436354644491202\\
0.34	0.000436336407262124\\
0.35	0.000436318187292931\\
0.36	0.000436299986491145\\
0.37	0.000436281806850935\\
0.38	0.00043626365045649\\
0.39	0.000436245519485496\\
0.4	0.000436227416212773\\
0.41	0.00043620934301399\\
0.42	0.000436191302369561\\
0.43	0.000436173296868663\\
0.44	0.000436155329213397\\
0.45	0.000436137402223095\\
0.46	0.0004361195188388\\
0.47	0.00043610168212788\\
0.48	0.000436083895288829\\
0.49	0.000436066161656224\\
0.5	0.000436048484705874\\
0.51	0.000436030868060138\\
0.52	0.000436013315493445\\
0.53	0.000435995830937983\\
0.54	0.000435978418489653\\
0.55	0.00043596108241414\\
0.56	0.00043594382715329\\
0.57	0.000435926657331656\\
0.58	0.000435909577763299\\
0.59	0.000435892593458817\\
0.6	0.000435875709632633\\
0.61	0.000435858931710534\\
0.62	0.000435842265337468\\
0.63	0.000435825716385641\\
0.64	0.000435809290962864\\
0.65	0.000435792995421203\\
0.66	0.000435776836365967\\
0.67	0.000435760820664953\\
0.68	0.000435744955458051\\
0.69	0.000435729248167193\\
0.7	0.000435713706506602\\
0.71	0.000435698338493462\\
0.72	0.000435683152458902\\
0.73	0.000435668157059402\\
0.74	0.00043565326541053\\
0.75	0.000435638368088192\\
0.76	0.000435623465089989\\
0.77	0.000435608556413504\\
0.78	0.000435593642056337\\
0.79	0.000435578722016069\\
0.8	0.000435563796290291\\
0.81	0.000435548864876593\\
0.82	0.000435533927772553\\
0.83	0.000435518984975767\\
0.84	0.000435504036483809\\
0.85	0.000435489082294261\\
0.86	0.000435474122404706\\
0.87	0.000435459156812722\\
0.88	0.000435444185515889\\
0.89	0.00043542920851178\\
0.9	0.000435414225797967\\
0.91	0.000435399237372031\\
0.92	0.000435384243231541\\
0.93	0.000435369243374069\\
0.94	0.000435354237797185\\
0.95	0.000435339226498454\\
0.96	0.000435324209475446\\
0.97	0.000435309186725729\\
0.98	0.000435294158246863\\
0.99	0.000435279124036417\\
1	0.000435264084091946\\
1.01	0.000435249038411013\\
1.02	0.00043523398699118\\
1.03	0.000435218929830006\\
1.04	0.000435203866925037\\
1.05	0.000435188798273839\\
1.06	0.000435173723873965\\
1.07	0.000435158643722965\\
1.08	0.000435143557818389\\
1.09	0.000435128466157788\\
1.1	0.000435113368738707\\
1.11	0.000435098265558702\\
1.12	0.00043508315661531\\
1.13	0.000435068041906083\\
1.14	0.00043505292142856\\
1.15	0.000435037795180284\\
1.16	0.000435022663158794\\
1.17	0.000435007525361629\\
1.18	0.00043499238178633\\
1.19	0.000434977232430429\\
1.2	0.000434962077291467\\
1.21	0.000434946916366974\\
1.22	0.000434931749654477\\
1.23	0.00043491657715152\\
1.24	0.000434901398855624\\
1.25	0.00043488621476432\\
1.26	0.000434871024875128\\
1.27	0.000434855829185586\\
1.28	0.00043484062769321\\
1.29	0.000434825420395525\\
1.3	0.000434810207290056\\
1.31	0.000434794988374319\\
1.32	0.00043477976364583\\
1.33	0.000434764533102117\\
1.34	0.000434749296740686\\
1.35	0.000434734054559058\\
1.36	0.000434718806554742\\
1.37	0.000434703552725251\\
1.38	0.000434688293068097\\
1.39	0.00043467302758079\\
1.4	0.00043465775626084\\
1.41	0.000434642479105745\\
1.42	0.000434627196113022\\
1.43	0.000434611907280167\\
1.44	0.000434596612604686\\
1.45	0.000434581312084079\\
1.46	0.000434566005715845\\
1.47	0.000434550693497479\\
1.48	0.000434535375426487\\
1.49	0.000434520051500357\\
1.5	0.000434504721716588\\
1.51	0.000434489386072667\\
1.52	0.000434474044566093\\
1.53	0.000434458697194351\\
1.54	0.000434443343954934\\
1.55	0.000434427984845324\\
1.56	0.00043441261986301\\
1.57	0.000434397249005474\\
1.58	0.000434381872270203\\
1.59	0.000434366489654678\\
1.6	0.000434351101156378\\
1.61	0.000434335706772782\\
1.62	0.000434320306501368\\
1.63	0.000434304900339612\\
1.64	0.000434289488284993\\
1.65	0.000434274070334976\\
1.66	0.000434258646487037\\
1.67	0.000434243216738654\\
1.68	0.000434227781087284\\
1.69	0.000434212339530405\\
1.7	0.000434196892065475\\
1.71	0.000434181438689965\\
1.72	0.000434165979401338\\
1.73	0.000434150514197054\\
1.74	0.000434135043074576\\
1.75	0.000434119566031359\\
1.76	0.000434104083064867\\
1.77	0.000434088594172556\\
1.78	0.000434073099351874\\
1.79	0.000434057598600284\\
1.8	0.00043404209191523\\
1.81	0.000434026579294172\\
1.82	0.000434011060734556\\
1.83	0.000433995536233824\\
1.84	0.000433980005789432\\
1.85	0.00043396446939882\\
1.86	0.000433948927059431\\
1.87	0.000433933378768714\\
1.88	0.000433917824524101\\
1.89	0.000433902264323036\\
1.9	0.000433886698162958\\
1.91	0.000433871126041303\\
1.92	0.000433855547955506\\
1.93	0.000433839963903004\\
1.94	0.000433824373881224\\
1.95	0.000433808777887604\\
1.96	0.000433793175919565\\
1.97	0.000433777567974544\\
1.98	0.00043376195404996\\
1.99	0.000433746334143241\\
2	0.000433730708251815\\
2.01	0.000433715076373102\\
2.02	0.000433699438504517\\
2.03	0.000433683794643491\\
2.04	0.00043366814478743\\
2.05	0.000433652488933759\\
2.06	0.000433636827079888\\
2.07	0.000433621159223237\\
2.08	0.000433605485361218\\
2.09	0.000433589805491233\\
2.1	0.000433574119610702\\
2.11	0.000433558427717026\\
2.12	0.00043354272980761\\
2.13	0.000433527025879867\\
2.14	0.000433511315931197\\
2.15	0.000433495599959003\\
2.16	0.000433479877960682\\
2.17	0.000433464149933638\\
2.18	0.000433448415875268\\
2.19	0.000433432675782962\\
2.2	0.000433416929654123\\
2.21	0.000433401177486144\\
2.22	0.000433385419276412\\
2.23	0.000433369655022322\\
2.24	0.000433353884721259\\
2.25	0.000433338108370615\\
2.26	0.00043332232596777\\
2.27	0.000433306537510116\\
2.28	0.000433290742995031\\
2.29	0.000433274942419898\\
2.3	0.000433259135782101\\
2.31	0.000433243323079013\\
2.32	0.000433227504308013\\
2.33	0.000433211679466482\\
2.34	0.000433195848551785\\
2.35	0.000433180011561302\\
2.36	0.000433164168492404\\
2.37	0.000433148319342457\\
2.38	0.000433132464108834\\
2.39	0.000433116602788898\\
2.4	0.000433100735380011\\
2.41	0.000433084861879545\\
2.42	0.000433068982284863\\
2.43	0.000433053096593318\\
2.44	0.000433037204802275\\
2.45	0.000433021306909094\\
2.46	0.000433005402911128\\
2.47	0.000432989492805732\\
2.48	0.000432973576590261\\
2.49	0.000432957654262066\\
2.5	0.000432941725818495\\
2.51	0.000432925791256903\\
2.52	0.00043290985057463\\
2.53	0.000432893903769034\\
2.54	0.000432877950837443\\
2.55	0.000432861991777216\\
2.56	0.000432846026585687\\
2.57	0.000432830055260194\\
2.58	0.000432814077798079\\
2.59	0.00043279809419668\\
2.6	0.000432782104453328\\
2.61	0.000432766108565361\\
2.62	0.000432750106530108\\
2.63	0.000432734098344906\\
2.64	0.000432718084007078\\
2.65	0.000432702063513953\\
2.66	0.000432686036862863\\
2.67	0.000432670004051125\\
2.68	0.000432653965076073\\
2.69	0.000432637919935015\\
2.7	0.000432621868625279\\
2.71	0.000432605811144187\\
2.72	0.000432589747489054\\
2.73	0.000432573677657189\\
2.74	0.000432557601645915\\
2.75	0.000432541519452538\\
2.76	0.000432525431074379\\
2.77	0.000432509336508733\\
2.78	0.000432493235752925\\
2.79	0.000432477128804246\\
2.8	0.000432461015660012\\
2.81	0.000432444896317521\\
2.82	0.000432428770774079\\
2.83	0.000432412639026981\\
2.84	0.000432396501073531\\
2.85	0.000432380356911021\\
2.86	0.000432364206536754\\
2.87	0.000432348049948019\\
2.88	0.000432331887142107\\
2.89	0.000432315718116312\\
2.9	0.000432299542867927\\
2.91	0.000432283361394238\\
2.92	0.000432267173692529\\
2.93	0.000432250979760089\\
2.94	0.000432234779594197\\
2.95	0.000432218573192132\\
2.96	0.000432202360551186\\
2.97	0.000432186141668631\\
2.98	0.00043216991654174\\
2.99	0.000432153685167797\\
3	0.000432137447544067\\
3.01	0.000432121203667831\\
3.02	0.000432104953536353\\
3.03	0.000432088697146909\\
3.04	0.000432072434496762\\
3.05	0.000432056165583177\\
3.06	0.000432039890403425\\
3.07	0.000432023608954762\\
3.08	0.000432007321234457\\
3.09	0.000431991027239767\\
3.1	0.000431974726967946\\
3.11	0.000431958420416254\\
3.12	0.000431942107581948\\
3.13	0.000431925788462278\\
3.14	0.000431909463054502\\
3.15	0.000431893131355859\\
3.16	0.000431876793363614\\
3.17	0.000431860449074999\\
3.18	0.000431844098487271\\
3.19	0.00043182774159767\\
3.2	0.000431811378403436\\
3.21	0.000431795008901817\\
3.22	0.000431778633090043\\
3.23	0.000431762250965357\\
3.24	0.000431745862524997\\
3.25	0.000431729467766194\\
3.26	0.000431713066686183\\
3.27	0.000431696659282197\\
3.28	0.000431680245551461\\
3.29	0.000431663825491205\\
3.3	0.000431647399098655\\
3.31	0.000431630966371041\\
3.32	0.00043161452730558\\
3.33	0.000431598081899498\\
3.34	0.000431581630150014\\
3.35	0.000431565172054347\\
3.36	0.000431548707609711\\
3.37	0.000431532236813328\\
3.38	0.000431515759662408\\
3.39	0.000431499276154157\\
3.4	0.000431482786285792\\
3.41	0.000431466290054524\\
3.42	0.000431449787457558\\
3.43	0.000431433278492094\\
3.44	0.000431416763155347\\
3.45	0.000431400241444507\\
3.46	0.000431383713356786\\
3.47	0.000431367178889376\\
3.48	0.000431350638039479\\
3.49	0.00043133409080429\\
3.5	0.000431317537180998\\
3.51	0.000431300977166803\\
3.52	0.000431284410758889\\
3.53	0.000431267837954451\\
3.54	0.000431251258750674\\
3.55	0.000431234673144746\\
3.56	0.00043121808113385\\
3.57	0.000431201482715169\\
3.58	0.000431184877885887\\
3.59	0.000431168266643177\\
3.6	0.000431151648984223\\
3.61	0.0004311350249062\\
3.62	0.00043111839440628\\
3.63	0.000431101757481636\\
3.64	0.000431085114129445\\
3.65	0.000431068464346871\\
3.66	0.000431051808131085\\
3.67	0.000431035145479252\\
3.68	0.000431018476388535\\
3.69	0.0004310018008561\\
3.7	0.000430985118879105\\
3.71	0.000430968430454714\\
3.72	0.000430951735580084\\
3.73	0.000430935034252365\\
3.74	0.000430918326468721\\
3.75	0.000430901612226302\\
3.76	0.000430884891522255\\
3.77	0.000430868164353736\\
3.78	0.00043085143071789\\
3.79	0.000430834690611863\\
3.8	0.000430817944032798\\
3.81	0.000430801190977844\\
3.82	0.000430784431444136\\
3.83	0.000430767665428818\\
3.84	0.00043075089292903\\
3.85	0.0004307341139419\\
3.86	0.00043071732846457\\
3.87	0.000430700536494171\\
3.88	0.000430683738027833\\
3.89	0.000430666933062682\\
3.9	0.000430650121595856\\
3.91	0.000430633303624474\\
3.92	0.000430616479145662\\
3.93	0.000430599648156539\\
3.94	0.000430582810654234\\
3.95	0.000430565966635861\\
3.96	0.000430549116098538\\
3.97	0.000430532259039383\\
3.98	0.000430515395455508\\
3.99	0.000430498525344032\\
4	0.000430481648702057\\
4.01	0.000430464765526696\\
4.02	0.00043044787581506\\
4.03	0.000430430979564248\\
4.04	0.000430414076771371\\
4.05	0.000430397167433526\\
4.06	0.000430380251547818\\
4.07	0.000430363329111342\\
4.08	0.000430346400121196\\
4.09	0.000430329464574478\\
4.1	0.000430312522468277\\
4.11	0.000430295573799696\\
4.12	0.000430278618565814\\
4.13	0.000430261656763721\\
4.14	0.000430244688390511\\
4.15	0.000430227713443261\\
4.16	0.00043021073191906\\
4.17	0.000430193743814989\\
4.18	0.000430176749128128\\
4.19	0.000430159747855553\\
4.2	0.00043014273999434\\
4.21	0.000430125725541566\\
4.22	0.000430108704494303\\
4.23	0.000430091676849627\\
4.24	0.000430074642604602\\
4.25	0.0004300576017563\\
4.26	0.000430040554301781\\
4.27	0.000430023500238118\\
4.28	0.000430006439562366\\
4.29	0.000429989372271591\\
4.3	0.000429972298362846\\
4.31	0.000429955217833193\\
4.32	0.000429938130679692\\
4.33	0.00042992103689939\\
4.34	0.00042990393648934\\
4.35	0.000429886829446592\\
4.36	0.000429869715768198\\
4.37	0.000429852595451204\\
4.38	0.000429835468492653\\
4.39	0.000429818334889587\\
4.4	0.000429801194639052\\
4.41	0.000429784047738087\\
4.42	0.000429766894183725\\
4.43	0.000429749733973011\\
4.44	0.000429732567102971\\
4.45	0.000429715393570644\\
4.46	0.000429698213373056\\
4.47	0.000429681026507242\\
4.48	0.000429663832970219\\
4.49	0.000429646632759028\\
4.5	0.000429629425870682\\
4.51	0.000429612212302199\\
4.52	0.000429594992050612\\
4.53	0.00042957776511293\\
4.54	0.000429560531486177\\
4.55	0.000429543291167364\\
4.56	0.000429526044153502\\
4.57	0.000429508790441605\\
4.58	0.000429491530028678\\
4.59	0.000429474262911738\\
4.6	0.000429456989087784\\
4.61	0.000429439708553822\\
4.62	0.000429422421306853\\
4.63	0.000429405127343878\\
4.64	0.0004293878266619\\
4.65	0.000429370519257913\\
4.66	0.000429353205128909\\
4.67	0.000429335884271886\\
4.68	0.000429318556683829\\
4.69	0.000429301222361735\\
4.7	0.000429283881302591\\
4.71	0.000429266533503376\\
4.72	0.000429249178961083\\
4.73	0.000429231817672694\\
4.74	0.000429214449635184\\
4.75	0.000429197074845534\\
4.76	0.000429179693300721\\
4.77	0.000429162304997725\\
4.78	0.00042914490993351\\
4.79	0.000429127508105054\\
4.8	0.000429110099509323\\
4.81	0.000429092684143291\\
4.82	0.00042907526200392\\
4.83	0.000429057833088173\\
4.84	0.000429040397393011\\
4.85	0.0004290229549154\\
4.86	0.000429005505652294\\
4.87	0.000428988049600652\\
4.88	0.000428970586757433\\
4.89	0.000428953117119581\\
4.9	0.000428935640684056\\
4.91	0.000428918157447802\\
4.92	0.000428900667407764\\
4.93	0.000428883170560893\\
4.94	0.000428865666904133\\
4.95	0.00042884815643443\\
4.96	0.000428830639148715\\
4.97	0.000428813115043931\\
4.98	0.000428795584117013\\
4.99	0.000428778046364901\\
5	0.00042876050178452\\
5.01	0.000428742950372802\\
5.02	0.000428725392126683\\
5.03	0.000428707827043084\\
5.04	0.000428690255118934\\
5.05	0.00042867267635115\\
5.06	0.00042865509073666\\
5.07	0.000428637498272381\\
5.08	0.000428619898955231\\
5.09	0.000428602292782127\\
5.1	0.000428584679749981\\
5.11	0.000428567059855707\\
5.12	0.000428549433096214\\
5.13	0.000428531799468414\\
5.14	0.000428514158969209\\
5.15	0.000428496511595504\\
5.16	0.000428478857344204\\
5.17	0.000428461196212209\\
5.18	0.000428443528196419\\
5.19	0.000428425853293727\\
5.2	0.000428408171501034\\
5.21	0.000428390482815226\\
5.22	0.000428372787233199\\
5.23	0.000428355084751844\\
5.24	0.000428337375368052\\
5.25	0.000428319659078694\\
5.26	0.000428301935880671\\
5.27	0.000428284205770851\\
5.28	0.000428266468746122\\
5.29	0.000428248724803359\\
5.3	0.000428230973939439\\
5.31	0.000428213216151239\\
5.32	0.000428195451435624\\
5.33	0.000428177679789471\\
5.34	0.000428159901209646\\
5.35	0.000428142115693014\\
5.36	0.000428124323236443\\
5.37	0.000428106523836791\\
5.38	0.000428088717490921\\
5.39	0.000428070904195693\\
5.4	0.000428053083947966\\
5.41	0.000428035256744591\\
5.42	0.00042801742258242\\
5.43	0.000427999581458309\\
5.44	0.000427981733369099\\
5.45	0.000427963878311647\\
5.46	0.000427946016282788\\
5.47	0.000427928147279375\\
5.48	0.000427910271298246\\
5.49	0.000427892388336239\\
5.5	0.000427874498390192\\
5.51	0.000427856601456935\\
5.52	0.000427838697533312\\
5.53	0.000427820786616146\\
5.54	0.000427802868702275\\
5.55	0.00042778494378852\\
5.56	0.000427767011871704\\
5.57	0.000427749072948659\\
5.58	0.000427731127016205\\
5.59	0.000427713174071155\\
5.6	0.000427695214110336\\
5.61	0.000427677247130559\\
5.62	0.000427659273128635\\
5.63	0.000427641292101379\\
5.64	0.000427623304045601\\
5.65	0.00042760530895811\\
5.66	0.000427587306835713\\
5.67	0.000427569297675208\\
5.68	0.000427551281473407\\
5.69	0.000427533258227102\\
5.7	0.000427515227933087\\
5.71	0.000427497190588171\\
5.72	0.00042747914618914\\
5.73	0.000427461094732786\\
5.74	0.0004274430362159\\
5.75	0.000427424970635271\\
5.76	0.000427406897987684\\
5.77	0.000427388818269923\\
5.78	0.000427370731478771\\
5.79	0.00042735263761101\\
5.8	0.000427334536663413\\
5.81	0.000427316428632762\\
5.82	0.000427298313515826\\
5.83	0.000427280191309377\\
5.84	0.000427262062010192\\
5.85	0.000427243925615032\\
5.86	0.000427225782120663\\
5.87	0.000427207631523851\\
5.88	0.00042718947382136\\
5.89	0.000427171309009947\\
5.9	0.000427153137086371\\
5.91	0.000427134958047384\\
5.92	0.00042711677188975\\
5.93	0.000427098578610211\\
5.94	0.000427080378205522\\
5.95	0.000427062170672428\\
5.96	0.000427043956007674\\
5.97	0.000427025734208008\\
5.98	0.000427007505270171\\
5.99	0.000426989269190894\\
6	0.00042697102596693\\
6.01	0.000426952775594999\\
6.02	0.000426934518071843\\
6.03	0.00042691625339419\\
6.04	0.000426897981558774\\
6.05	0.000426879702562315\\
6.06	0.000426861416401549\\
6.07	0.000426843123073188\\
6.08	0.00042682482257396\\
6.09	0.000426806514900581\\
6.1	0.000426788200049768\\
6.11	0.000426769878018238\\
6.12	0.000426751548802702\\
6.13	0.000426733212399872\\
6.14	0.000426714868806457\\
6.15	0.000426696518019163\\
6.16	0.000426678160034697\\
6.17	0.000426659794849755\\
6.18	0.000426641422461042\\
6.19	0.000426623042865261\\
6.2	0.000426604656059097\\
6.21	0.000426586262039255\\
6.22	0.000426567860802422\\
6.23	0.00042654945234529\\
6.24	0.000426531036664542\\
6.25	0.000426512613756872\\
6.26	0.000426494183618955\\
6.27	0.000426475746247478\\
6.28	0.000426457301639122\\
6.29	0.000426438849790557\\
6.3	0.000426420390698467\\
6.31	0.000426401924359521\\
6.32	0.00042638345077039\\
6.33	0.000426364969927743\\
6.34	0.000426346481828248\\
6.35	0.000426327986468571\\
6.36	0.000426309483845371\\
6.37	0.000426290973955313\\
6.38	0.000426272456795052\\
6.39	0.000426253932361243\\
6.4	0.000426235400650543\\
6.41	0.000426216861659608\\
6.42	0.000426198315385084\\
6.43	0.000426179761823613\\
6.44	0.000426161200971852\\
6.45	0.000426142632826436\\
6.46	0.000426124057384012\\
6.47	0.000426105474641213\\
6.48	0.000426086884594681\\
6.49	0.00042606828724105\\
6.5	0.000426049682576951\\
6.51	0.00042603107059902\\
6.52	0.000426012451303878\\
6.53	0.000425993824688159\\
6.54	0.000425975190748483\\
6.55	0.000425956549481469\\
6.56	0.000425937900883744\\
6.57	0.000425919244951922\\
6.58	0.000425900581682619\\
6.59	0.00042588191107245\\
6.6	0.000425863233118023\\
6.61	0.00042584454781595\\
6.62	0.000425825855162838\\
6.63	0.000425807155155293\\
6.64	0.000425788447789913\\
6.65	0.000425769733063304\\
6.66	0.000425751010972059\\
6.67	0.00042573228151278\\
6.68	0.000425713544682054\\
6.69	0.000425694800476481\\
6.7	0.000425676048892642\\
6.71	0.00042565728992713\\
6.72	0.000425638523576528\\
6.73	0.00042561974983742\\
6.74	0.000425600968706388\\
6.75	0.000425582180180005\\
6.76	0.000425563384254856\\
6.77	0.000425544580927511\\
6.78	0.000425525770194538\\
6.79	0.000425506952052511\\
6.8	0.000425488126497999\\
6.81	0.000425469293527565\\
6.82	0.000425450453137771\\
6.83	0.000425431605325183\\
6.84	0.000425412750086352\\
6.85	0.000425393887417845\\
6.86	0.000425375017316208\\
6.87	0.000425356139778\\
6.88	0.000425337254799762\\
6.89	0.000425318362378045\\
6.9	0.000425299462509398\\
6.91	0.000425280555190363\\
6.92	0.000425261640417476\\
6.93	0.000425242718187282\\
6.94	0.000425223788496317\\
6.95	0.000425204851341114\\
6.96	0.000425185906718204\\
6.97	0.000425166954624114\\
6.98	0.000425147995055376\\
6.99	0.000425129028008515\\
7	0.000425110053480055\\
7.01	0.000425091071466512\\
7.02	0.000425072081964415\\
7.03	0.000425053084970265\\
7.04	0.000425034080480587\\
7.05	0.000425015068491888\\
7.06	0.000424996049000681\\
7.07	0.00042497702200347\\
7.08	0.000424957987496765\\
7.09	0.000424938945477064\\
7.1	0.000424919895940868\\
7.11	0.000424900838884674\\
7.12	0.000424881774304981\\
7.13	0.000424862702198282\\
7.14	0.00042484362256107\\
7.15	0.00042482453538983\\
7.16	0.000424805440681049\\
7.17	0.000424786338431215\\
7.18	0.000424767228636807\\
7.19	0.00042474811129431\\
7.2	0.000424728986400193\\
7.21	0.000424709853950938\\
7.22	0.000424690713943018\\
7.23	0.0004246715663729\\
7.24	0.000424652411237054\\
7.25	0.000424633248531947\\
7.26	0.000424614078254042\\
7.27	0.000424594900399805\\
7.28	0.000424575714965687\\
7.29	0.000424556521948149\\
7.3	0.000424537321343647\\
7.31	0.000424518113148632\\
7.32	0.000424498897359557\\
7.33	0.00042447967397286\\
7.34	0.000424460442984996\\
7.35	0.0004244412043924\\
7.36	0.000424421958191522\\
7.37	0.000424402704378796\\
7.38	0.000424383442950654\\
7.39	0.000424364173903539\\
7.4	0.000424344897233869\\
7.41	0.000424325612938087\\
7.42	0.000424306321012608\\
7.43	0.000424287021453862\\
7.44	0.000424267714258273\\
7.45	0.000424248399422254\\
7.46	0.00042422907694223\\
7.47	0.000424209746814608\\
7.48	0.00042419040903581\\
7.49	0.000424171063602234\\
7.5	0.000424151710510294\\
7.51	0.000424132349756398\\
7.52	0.00042411298133695\\
7.53	0.000424093605248347\\
7.54	0.000424074221486985\\
7.55	0.000424054830049262\\
7.56	0.000424035430931573\\
7.57	0.000424016024130313\\
7.58	0.000423996609641862\\
7.59	0.00042397718746261\\
7.6	0.000423957757588942\\
7.61	0.000423938320017242\\
7.62	0.000423918874743887\\
7.63	0.000423899421765253\\
7.64	0.000423879961077714\\
7.65	0.000423860492677647\\
7.66	0.000423841016561413\\
7.67	0.000423821532725384\\
7.68	0.000423802041165929\\
7.69	0.000423782541879405\\
7.7	0.000423763034862176\\
7.71	0.000423743520110592\\
7.72	0.00042372399762102\\
7.73	0.000423704467389804\\
7.74	0.000423684929413299\\
7.75	0.000423665383687846\\
7.76	0.0004236458302098\\
7.77	0.000423626268975495\\
7.78	0.000423606699981281\\
7.79	0.00042358712322349\\
7.8	0.000423567538698462\\
7.81	0.000423547946402527\\
7.82	0.000423528346332016\\
7.83	0.000423508738483257\\
7.84	0.000423489122852581\\
7.85	0.000423469499436308\\
7.86	0.000423449868230757\\
7.87	0.000423430229232255\\
7.88	0.00042341058243711\\
7.89	0.000423390927841636\\
7.9	0.000423371265442151\\
7.91	0.000423351595234961\\
7.92	0.000423331917216371\\
7.93	0.000423312231382684\\
7.94	0.000423292537730205\\
7.95	0.000423272836255232\\
7.96	0.00042325312695406\\
7.97	0.000423233409822987\\
7.98	0.000423213684858302\\
7.99	0.000423193952056294\\
8	0.000423174211413246\\
8.01	0.000423154462925453\\
8.02	0.000423134706589185\\
8.03	0.000423114942400729\\
8.04	0.000423095170356359\\
8.05	0.000423075390452349\\
8.06	0.000423055602684975\\
8.07	0.0004230358070505\\
8.08	0.000423016003545197\\
8.09	0.000422996192165322\\
8.1	0.000422976372907146\\
8.11	0.000422956545766923\\
8.12	0.000422936710740913\\
8.13	0.000422916867825366\\
8.14	0.000422897017016541\\
8.15	0.000422877158310677\\
8.16	0.000422857291704028\\
8.17	0.000422837417192838\\
8.18	0.000422817534773346\\
8.19	0.000422797644441794\\
8.2	0.000422777746194414\\
8.21	0.000422757840027446\\
8.22	0.000422737925937123\\
8.23	0.000422718003919667\\
8.24	0.000422698073971306\\
8.25	0.000422678136088268\\
8.26	0.00042265819026677\\
8.27	0.000422638236503035\\
8.28	0.000422618274793278\\
8.29	0.000422598305133711\\
8.3	0.000422578327520549\\
8.31	0.000422558341949998\\
8.32	0.000422538348418265\\
8.33	0.000422518346921554\\
8.34	0.000422498337456062\\
8.35	0.000422478320017993\\
8.36	0.00042245829460354\\
8.37	0.000422438261208897\\
8.38	0.000422418219830255\\
8.39	0.000422398170463802\\
8.4	0.000422378113105725\\
8.41	0.000422358047752208\\
8.42	0.000422337974399426\\
8.43	0.000422317893043563\\
8.44	0.000422297803680791\\
8.45	0.000422277706307282\\
8.46	0.000422257600919206\\
8.47	0.000422237487512734\\
8.48	0.000422217366084027\\
8.49	0.000422197236629254\\
8.5	0.000422177099144569\\
8.51	0.000422156953626128\\
8.52	0.000422136800070088\\
8.53	0.000422116638472605\\
8.54	0.000422096468829821\\
8.55	0.000422076291137885\\
8.56	0.000422056105392947\\
8.57	0.000422035911591135\\
8.58	0.000422015709728604\\
8.59	0.000421995499801481\\
8.6	0.000421975281805901\\
8.61	0.000421955055738001\\
8.62	0.000421934821593896\\
8.63	0.000421914579369726\\
8.64	0.000421894329061603\\
8.65	0.000421874070665655\\
8.66	0.000421853804177995\\
8.67	0.000421833529594741\\
8.68	0.000421813246912008\\
8.69	0.000421792956125898\\
8.7	0.000421772657232524\\
8.71	0.000421752350227992\\
8.72	0.000421732035108402\\
8.73	0.000421711711869851\\
8.74	0.000421691380508439\\
8.75	0.00042167104102026\\
8.76	0.000421650693401401\\
8.77	0.000421630337647954\\
8.78	0.000421609973756008\\
8.79	0.000421589601721641\\
8.8	0.000421569221540938\\
8.81	0.000421548833209976\\
8.82	0.000421528436724827\\
8.83	0.000421508032081568\\
8.84	0.000421487619276266\\
8.85	0.000421467198304987\\
8.86	0.000421446769163801\\
8.87	0.00042142633184877\\
8.88	0.000421405886355945\\
8.89	0.000421385432681391\\
8.9	0.000421364970821158\\
8.91	0.000421344500771297\\
8.92	0.000421324022527857\\
8.93	0.000421303536086886\\
8.94	0.000421283041444422\\
8.95	0.000421262538596513\\
8.96	0.000421242027539192\\
8.97	0.000421221508268491\\
8.98	0.000421200980780445\\
8.99	0.000421180445071088\\
9	0.000421159901136441\\
9.01	0.000421139348972532\\
9.02	0.000421118788575378\\
9.03	0.000421098219941003\\
9.04	0.000421077643065416\\
9.05	0.000421057057944636\\
9.06	0.000421036464574672\\
9.07	0.000421015862951533\\
9.08	0.000420995253071221\\
9.09	0.000420974634929744\\
9.1	0.000420954008523095\\
9.11	0.000420933373847274\\
9.12	0.000420912730898273\\
9.13	0.000420892079672089\\
9.14	0.000420871420164707\\
9.15	0.000420850752372106\\
9.16	0.000420830076290282\\
9.17	0.000420809391915208\\
9.18	0.000420788699242858\\
9.19	0.000420767998269216\\
9.2	0.000420747288990249\\
9.21	0.000420726571401929\\
9.22	0.000420705845500219\\
9.23	0.000420685111281084\\
9.24	0.000420664368740482\\
9.25	0.000420643617874384\\
9.26	0.000420622858678731\\
9.27	0.000420602091149483\\
9.28	0.000420581315282588\\
9.29	0.000420560531073994\\
9.3	0.000420539738519648\\
9.31	0.000420518937615487\\
9.32	0.00042049812835745\\
9.33	0.000420477310741477\\
9.34	0.0004204564847635\\
9.35	0.000420435650419448\\
9.36	0.000420414807705249\\
9.37	0.000420393956616834\\
9.38	0.000420373097150118\\
9.39	0.000420352229301021\\
9.4	0.00042033135306546\\
9.41	0.00042031046843935\\
9.42	0.000420289575418604\\
9.43	0.00042026867399913\\
9.44	0.000420247764176835\\
9.45	0.000420226845947617\\
9.46	0.000420205919307375\\
9.47	0.00042018498425201\\
9.48	0.000420164040777415\\
9.49	0.000420143088879487\\
9.5	0.000420122128554105\\
9.51	0.000420101159797159\\
9.52	0.000420080182604537\\
9.53	0.000420059196972114\\
9.54	0.000420038202895769\\
9.55	0.000420017200371376\\
9.56	0.000419996189394809\\
9.57	0.000419975169961933\\
9.58	0.000419954142068616\\
9.59	0.000419933105710726\\
9.6	0.000419912060884115\\
9.61	0.000419891007584649\\
9.62	0.000419869945808183\\
9.63	0.000419848875550563\\
9.64	0.000419827796807642\\
9.65	0.000419806709575267\\
9.66	0.000419785613849283\\
9.67	0.000419764509625524\\
9.68	0.000419743396899834\\
9.69	0.000419722275668048\\
9.7	0.000419701145926002\\
9.71	0.00041968000766952\\
9.72	0.000419658860894428\\
9.73	0.000419637705596559\\
9.74	0.000419616541771724\\
9.75	0.000419595369415748\\
9.76	0.000419574188524443\\
9.77	0.000419552999093626\\
9.78	0.000419531801119101\\
9.79	0.000419510594596681\\
9.8	0.000419489379522167\\
9.81	0.000419468155891363\\
9.82	0.000419446923700065\\
9.83	0.00041942568294407\\
9.84	0.000419404433619172\\
9.85	0.000419383175721159\\
9.86	0.000419361909245819\\
9.87	0.000419340634188943\\
9.88	0.000419319350546303\\
9.89	0.000419298058313686\\
9.9	0.000419276757486863\\
9.91	0.000419255448061616\\
9.92	0.000419234130033704\\
9.93	0.000419212803398896\\
9.94	0.000419191468152965\\
9.95	0.000419170124291672\\
9.96	0.000419148771810771\\
9.97	0.00041912741070602\\
9.98	0.000419106040973177\\
9.99	0.000419084662607991\\
10	0.000419063275606205\\
10.01	0.000419041879963573\\
10.02	0.000419020475675831\\
10.03	0.000418999062738725\\
10.04	0.000418977641147985\\
10.05	0.00041895621089935\\
10.06	0.000418934771988549\\
10.07	0.000418913324411317\\
10.08	0.00041889186816337\\
10.09	0.000418870403240439\\
10.1	0.000418848929638236\\
10.11	0.000418827447352488\\
10.12	0.000418805956378908\\
10.13	0.000418784456713204\\
10.14	0.00041876294835109\\
10.15	0.000418741431288268\\
10.16	0.000418719905520445\\
10.17	0.000418698371043316\\
10.18	0.000418676827852588\\
10.19	0.000418655275943952\\
10.2	0.000418633715313102\\
10.21	0.000418612145955729\\
10.22	0.000418590567867519\\
10.23	0.000418568981044153\\
10.24	0.000418547385481319\\
10.25	0.000418525781174696\\
10.26	0.000418504168119954\\
10.27	0.000418482546312774\\
10.28	0.000418460915748826\\
10.29	0.000418439276423774\\
10.3	0.000418417628333285\\
10.31	0.000418395971473027\\
10.32	0.000418374305838654\\
10.33	0.000418352631425829\\
10.34	0.000418330948230204\\
10.35	0.000418309256247432\\
10.36	0.000418287555473164\\
10.37	0.000418265845903043\\
10.38	0.000418244127532717\\
10.39	0.000418222400357831\\
10.4	0.000418200664374015\\
10.41	0.000418178919576913\\
10.42	0.000418157165962158\\
10.43	0.00041813540352538\\
10.44	0.000418113632262206\\
10.45	0.000418091852168264\\
10.46	0.000418070063239179\\
10.47	0.00041804826547057\\
10.48	0.000418026458858056\\
10.49	0.000418004643397253\\
10.5	0.000417982819083781\\
10.51	0.00041796098591324\\
10.52	0.000417939143881245\\
10.53	0.000417917292983398\\
10.54	0.000417895433215307\\
10.55	0.000417873564572568\\
10.56	0.000417851687050788\\
10.57	0.000417829800645551\\
10.58	0.000417807905352462\\
10.59	0.000417786001167104\\
10.6	0.000417764088085072\\
10.61	0.000417742166101943\\
10.62	0.000417720235213314\\
10.63	0.00041769829541476\\
10.64	0.000417676346701859\\
10.65	0.000417654389070187\\
10.66	0.000417632422515322\\
10.67	0.000417610447032835\\
10.68	0.0004175884626183\\
10.69	0.000417566469267276\\
10.7	0.000417544466975336\\
10.71	0.000417522455738039\\
10.72	0.000417500435550951\\
10.73	0.000417478406409624\\
10.74	0.000417456368309619\\
10.75	0.000417434321246486\\
10.76	0.000417412265215784\\
10.77	0.000417390200213058\\
10.78	0.000417368126233856\\
10.79	0.000417346043273729\\
10.8	0.000417323951328215\\
10.81	0.000417301850392859\\
10.82	0.000417279740463201\\
10.83	0.000417257621534776\\
10.84	0.000417235493603124\\
10.85	0.000417213356663775\\
10.86	0.000417191210712261\\
10.87	0.000417169055744113\\
10.88	0.000417146891754861\\
10.89	0.000417124718740025\\
10.9	0.000417102536695141\\
10.91	0.000417080345615724\\
10.92	0.000417058145497291\\
10.93	0.000417035936335367\\
10.94	0.00041701371812547\\
10.95	0.000416991490863113\\
10.96	0.000416969254543815\\
10.97	0.000416947009163081\\
10.98	0.000416924754716424\\
10.99	0.000416902491199356\\
11	0.000416880218607382\\
11.01	0.000416857936936011\\
11.02	0.000416835646180747\\
11.03	0.000416813346337091\\
11.04	0.000416791037400551\\
11.05	0.000416768719366621\\
11.06	0.000416746392230806\\
11.07	0.000416724055988599\\
11.08	0.0004167017106355\\
11.09	0.000416679356167005\\
11.1	0.000416656992578609\\
11.11	0.000416634619865802\\
11.12	0.000416612238024083\\
11.13	0.000416589847048938\\
11.14	0.000416567446935859\\
11.15	0.000416545037680335\\
11.16	0.000416522619277857\\
11.17	0.000416500191723913\\
11.18	0.000416477755013986\\
11.19	0.000416455309143568\\
11.2	0.00041643285410814\\
11.21	0.000416410389903189\\
11.22	0.000416387916524203\\
11.23	0.000416365433966663\\
11.24	0.000416342942226052\\
11.25	0.000416320441297857\\
11.26	0.00041629793117756\\
11.27	0.000416275411860638\\
11.28	0.000416252883342579\\
11.29	0.000416230345618869\\
11.3	0.000416207798684985\\
11.31	0.000416185242536411\\
11.32	0.000416162677168631\\
11.33	0.000416140102577124\\
11.34	0.00041611751875738\\
11.35	0.000416094925704878\\
11.36	0.000416072323415098\\
11.37	0.000416049711883529\\
11.38	0.000416027091105655\\
11.39	0.000416004461076966\\
11.4	0.00041598182179294\\
11.41	0.000415959173249065\\
11.42	0.000415936515440833\\
11.43	0.000415913848363729\\
11.44	0.000415891172013244\\
11.45	0.00041586848638487\\
11.46	0.000415845791474094\\
11.47	0.00041582308727642\\
11.48	0.000415800373787334\\
11.49	0.000415777651002333\\
11.5	0.000415754918916916\\
11.51	0.000415732177526584\\
11.52	0.00041570942682684\\
11.53	0.00041568666681318\\
11.54	0.000415663897481116\\
11.55	0.000415641118826155\\
11.56	0.000415618330843808\\
11.57	0.000415595533529588\\
11.58	0.000415572726879008\\
11.59	0.000415549910887584\\
11.6	0.00041552708555084\\
11.61	0.000415504250864295\\
11.62	0.000415481406823483\\
11.63	0.000415458553423925\\
11.64	0.000415435690661162\\
11.65	0.000415412818530723\\
11.66	0.000415389937028153\\
11.67	0.000415367046148994\\
11.68	0.000415344145888799\\
11.69	0.000415321236243113\\
11.7	0.000415298317207491\\
11.71	0.000415275388777505\\
11.72	0.000415252450948705\\
11.73	0.00041522950371667\\
11.74	0.000415206547076975\\
11.75	0.000415183581025193\\
11.76	0.00041516060555692\\
11.77	0.000415137620667735\\
11.78	0.000415114626353246\\
11.79	0.000415091622609045\\
11.8	0.000415068609430748\\
11.81	0.00041504558681396\\
11.82	0.000415022554754309\\
11.83	0.000414999513247419\\
11.84	0.00041497646228893\\
11.85	0.000414953401874481\\
11.86	0.000414930331999715\\
11.87	0.000414907252660291\\
11.88	0.000414884163851877\\
11.89	0.000414861065570144\\
11.9	0.000414837957810768\\
11.91	0.000414814840569443\\
11.92	0.000414791713841867\\
11.93	0.000414768577623743\\
11.94	0.000414745431910788\\
11.95	0.000414722276698733\\
11.96	0.000414699111983306\\
11.97	0.000414675937760258\\
11.98	0.00041465275402534\\
11.99	0.000414629560774323\\
12	0.000414606358002983\\
12.01	0.000414583145707113\\
12.02	0.000414559923882512\\
12.03	0.000414536692524994\\
12.04	0.000414513451630379\\
12.05	0.000414490201194511\\
12.06	0.000414466941213236\\
12.07	0.000414443671682427\\
12.08	0.000414420392597953\\
12.09	0.000414397103955712\\
12.1	0.000414373805751607\\
12.11	0.000414350497981563\\
12.12	0.00041432718064152\\
12.13	0.000414303853727425\\
12.14	0.000414280517235251\\
12.15	0.000414257171160982\\
12.16	0.000414233815500618\\
12.17	0.000414210450250179\\
12.18	0.000414187075405713\\
12.19	0.000414163690963261\\
12.2	0.00041414029691891\\
12.21	0.000414116893268748\\
12.22	0.000414093480008891\\
12.23	0.000414070057135474\\
12.24	0.000414046624644648\\
12.25	0.000414023182532594\\
12.26	0.000413999730795511\\
12.27	0.000413976269429617\\
12.28	0.00041395279843116\\
12.29	0.000413929317796406\\
12.3	0.000413905827521642\\
12.31	0.00041388232760319\\
12.32	0.000413858818037386\\
12.33	0.000413835298820605\\
12.34	0.00041381176994924\\
12.35	0.000413788231419705\\
12.36	0.00041376468322846\\
12.37	0.000413741125371979\\
12.38	0.000413717557846766\\
12.39	0.000413693980649367\\
12.4	0.000413670393776345\\
12.41	0.000413646797224299\\
12.42	0.000413623190989868\\
12.43	0.000413599575069715\\
12.44	0.000413575949460536\\
12.45	0.00041355231415907\\
12.46	0.000413528669162088\\
12.47	0.00041350501446639\\
12.48	0.000413481350068828\\
12.49	0.000413457675966277\\
12.5	0.000413433992155661\\
12.51	0.000413410298633943\\
12.52	0.000413386595398123\\
12.53	0.000413362882445241\\
12.54	0.000413339159772392\\
12.55	0.000413315427376701\\
12.56	0.000413291685255345\\
12.57	0.000413267933405549\\
12.58	0.000413244171824577\\
12.59	0.000413220400509749\\
12.6	0.000413196619458435\\
12.61	0.000413172828668049\\
12.62	0.000413149028136058\\
12.63	0.000413125217859987\\
12.64	0.00041310139783741\\
12.65	0.000413077568065965\\
12.66	0.000413053728543333\\
12.67	0.00041302987926726\\
12.68	0.000413006020235557\\
12.69	0.000412982151446089\\
12.7	0.000412958272896781\\
12.71	0.000412934384585627\\
12.72	0.000412910486510688\\
12.73	0.000412886578670082\\
12.74	0.000412862661062007\\
12.75	0.000412838733684721\\
12.76	0.000412814796536555\\
12.77	0.000412790849615919\\
12.78	0.000412766892921289\\
12.79	0.000412742926451225\\
12.8	0.000412718950204361\\
12.81	0.000412694964179412\\
12.82	0.000412670968375172\\
12.83	0.000412646962790522\\
12.84	0.000412622947424425\\
12.85	0.000412598922275936\\
12.86	0.000412574887344194\\
12.87	0.000412550842628435\\
12.88	0.000412526788127981\\
12.89	0.000412502723842256\\
12.9	0.000412478649770776\\
12.91	0.000412454565913167\\
12.92	0.000412430472269144\\
12.93	0.000412406368838537\\
12.94	0.00041238225562127\\
12.95	0.000412358132617398\\
12.96	0.000412333999827062\\
12.97	0.000412309857250538\\
12.98	0.000412285704888205\\
12.99	0.000412261542740563\\
13	0.00041223737080825\\
13.01	0.000412213189092001\\
13.02	0.0004121889975927\\
13.03	0.000412164796311355\\
13.04	0.000412140585249103\\
13.05	0.000412116364407225\\
13.06	0.000412092133787135\\
13.07	0.000412067893390385\\
13.08	0.000412043643218689\\
13.09	0.000412019383273891\\
13.1	0.000411995113557995\\
13.11	0.000411970834073168\\
13.12	0.000411946544821717\\
13.13	0.000411922245806131\\
13.14	0.000411897937029048\\
13.15	0.000411873618493288\\
13.16	0.000411849290201834\\
13.17	0.000411824952157854\\
13.18	0.000411800604364692\\
13.19	0.000411776246825873\\
13.2	0.000411751879545126\\
13.21	0.000411727502526351\\
13.22	0.000411703115773663\\
13.23	0.000411678719291365\\
13.24	0.000411654313083976\\
13.25	0.000411629897156219\\
13.26	0.000411605471513036\\
13.27	0.000411581036159576\\
13.28	0.000411556591101229\\
13.29	0.000411532136343604\\
13.3	0.000411507671892542\\
13.31	0.000411483197754126\\
13.32	0.000411458713934686\\
13.33	0.000411434220440793\\
13.34	0.000411409717279277\\
13.35	0.00041138520445722\\
13.36	0.000411360681981988\\
13.37	0.000411336149861198\\
13.38	0.000411311608102745\\
13.39	0.00041128705671482\\
13.4	0.000411262495705887\\
13.41	0.000411237925084715\\
13.42	0.000411213344860365\\
13.43	0.000411188755042214\\
13.44	0.000411164155639937\\
13.45	0.000411139546663539\\
13.46	0.000411114928123353\\
13.47	0.000411090300030037\\
13.48	0.000411065662394594\\
13.49	0.000411041015228377\\
13.5	0.000411016358543079\\
13.51	0.000410991692350766\\
13.52	0.00041096701666387\\
13.53	0.000410942331495195\\
13.54	0.000410917636857931\\
13.55	0.000410892932765661\\
13.56	0.00041086821923236\\
13.57	0.000410843496272413\\
13.58	0.000410818763900627\\
13.59	0.000410794022132226\\
13.6	0.000410769270982856\\
13.61	0.000410744510468627\\
13.62	0.000410719740606073\\
13.63	0.000410694961412205\\
13.64	0.000410670172904484\\
13.65	0.000410645375100861\\
13.66	0.000410620568019758\\
13.67	0.00041059575168011\\
13.68	0.000410570926101333\\
13.69	0.000410546091303371\\
13.7	0.000410521247306683\\
13.71	0.000410496394132275\\
13.72	0.000410471531801676\\
13.73	0.00041044666033698\\
13.74	0.000410421779760844\\
13.75	0.000410396890096486\\
13.76	0.000410371991367734\\
13.77	0.000410347083598982\\
13.78	0.000410322166815243\\
13.79	0.000410297241042147\\
13.8	0.000410272306305947\\
13.81	0.00041024736263354\\
13.82	0.000410222410052466\\
13.83	0.000410197448590929\\
13.84	0.000410172478277803\\
13.85	0.000410147499142647\\
13.86	0.000410122511215716\\
13.87	0.000410097514527976\\
13.88	0.000410072509111101\\
13.89	0.000410047494997505\\
13.9	0.000410022472220339\\
13.91	0.000409997440813511\\
13.92	0.000409972400811691\\
13.93	0.000409947352250339\\
13.94	0.000409922295165692\\
13.95	0.000409897229594794\\
13.96	0.000409872155575508\\
13.97	0.000409847073146518\\
13.98	0.000409821982347347\\
13.99	0.000409796883218372\\
14	0.000409771775800837\\
14.01	0.00040974666013685\\
14.02	0.000409721536269411\\
14.03	0.000409696404242415\\
14.04	0.000409671264100681\\
14.05	0.000409646115889933\\
14.06	0.000409620959656836\\
14.07	0.000409595795449001\\
14.08	0.00040957062331499\\
14.09	0.000409545443304331\\
14.1	0.00040952025546754\\
14.11	0.000409495059856102\\
14.12	0.00040946985652252\\
14.13	0.00040944464552029\\
14.14	0.000409419426903923\\
14.15	0.000409394200728972\\
14.16	0.00040936896705201\\
14.17	0.000409343725930655\\
14.18	0.000409318477423584\\
14.19	0.000409293221590513\\
14.2	0.000409267958492244\\
14.21	0.000409242688190639\\
14.22	0.000409217410748626\\
14.23	0.000409192126230225\\
14.24	0.000409166834700533\\
14.25	0.000409141536225746\\
14.26	0.000409116230873125\\
14.27	0.000409090918711042\\
14.28	0.000409065599808949\\
14.29	0.00040904027423738\\
14.3	0.000409014942067978\\
14.31	0.000408989603373438\\
14.32	0.00040896425822756\\
14.33	0.000408938906705202\\
14.34	0.000408913548882285\\
14.35	0.000408888184835796\\
14.36	0.000408862814643751\\
14.37	0.000408837438385214\\
14.38	0.000408812056140252\\
14.39	0.000408786667989946\\
14.4	0.00040876127401635\\
14.41	0.000408735874302481\\
14.42	0.000408710468932294\\
14.43	0.000408685057990662\\
14.44	0.000408659641563327\\
14.45	0.000408634219736899\\
14.46	0.000408608792598802\\
14.47	0.000408583360237236\\
14.48	0.000408557922741153\\
14.49	0.000408532480200192\\
14.5	0.000408507032704647\\
14.51	0.000408481580345401\\
14.52	0.000408456123213894\\
14.53	0.00040843066140204\\
14.54	0.000408405195002176\\
14.55	0.000408379724106981\\
14.56	0.000408354248809412\\
14.57	0.000408328769202632\\
14.58	0.000408303285379902\\
14.59	0.000408277797434518\\
14.6	0.000408252305459687\\
14.61	0.000408226809548448\\
14.62	0.000408201309793545\\
14.63	0.000408175806287325\\
14.64	0.000408150299121604\\
14.65	0.00040812478838753\\
14.66	0.00040809927417545\\
14.67	0.000408073756574777\\
14.68	0.000408048235673794\\
14.69	0.000408022711559522\\
14.7	0.00040799718431753\\
14.71	0.000407971654031744\\
14.72	0.00040794612078425\\
14.73	0.000407920584655079\\
14.74	0.00040789504572201\\
14.75	0.000407869504060297\\
14.76	0.000407843959742445\\
14.77	0.000407818412837947\\
14.78	0.000407792863413003\\
14.79	0.000407767311530232\\
14.8	0.000407741757248352\\
14.81	0.000407716200621872\\
14.82	0.000407690641700742\\
14.83	0.000407665080529996\\
14.84	0.000407639517149368\\
14.85	0.000407613951592901\\
14.86	0.000407588383888513\\
14.87	0.000407562814057561\\
14.88	0.000407537242114387\\
14.89	0.000407511668065802\\
14.9	0.000407486091910604\\
14.91	0.000407460513639\\
14.92	0.00040743493323207\\
14.93	0.000407409350661145\\
14.94	0.000407383765887189\\
14.95	0.000407358178860135\\
14.96	0.000407332589518198\\
14.97	0.00040730699778713\\
14.98	0.000407281403579481\\
14.99	0.000407255806793775\\
15	0.000407230207313671\\
15.01	0.000407204605007089\\
15.02	0.000407178999725272\\
15.03	0.000407153391301821\\
15.04	0.000407127779551668\\
15.05	0.000407102164270015\\
15.06	0.000407076545231205\\
15.07	0.000407050922187542\\
15.08	0.000407025294868078\\
15.09	0.000406999662977298\\
15.1	0.000406974026193779\\
15.11	0.000406948384168769\\
15.12	0.000406922736524706\\
15.13	0.000406897082853653\\
15.14	0.000406871422715668\\
15.15	0.000406845755637117\\
15.16	0.000406820081108853\\
15.17	0.000406794398584376\\
15.18	0.000406768707477856\\
15.19	0.000406743007162072\\
15.2	0.000406717296966301\\
15.21	0.000406691576174038\\
15.22	0.000406665844020649\\
15.23	0.000406640100428334\\
15.24	0.000406614345390119\\
15.25	0.000406588578899006\\
15.26	0.000406562800948003\\
15.27	0.000406537011530112\\
15.28	0.000406511210638321\\
15.29	0.000406485398265615\\
15.3	0.000406459574404971\\
15.31	0.000406433739049363\\
15.32	0.00040640789219175\\
15.33	0.000406382033825096\\
15.34	0.000406356163942344\\
15.35	0.000406330282536444\\
15.36	0.000406304389600329\\
15.37	0.000406278485126933\\
15.38	0.000406252569109176\\
15.39	0.000406226641539977\\
15.4	0.000406200702412242\\
15.41	0.000406174751718875\\
15.42	0.000406148789452773\\
15.43	0.000406122815606823\\
15.44	0.000406096830173909\\
15.45	0.000406070833146902\\
15.46	0.000406044824518672\\
15.47	0.000406018804282081\\
15.48	0.000405992772429987\\
15.49	0.000405966728955225\\
15.5	0.000405940673850646\\
15.51	0.000405914607109082\\
15.52	0.000405888528723355\\
15.53	0.000405862438686287\\
15.54	0.000405836336990685\\
15.55	0.000405810223629363\\
15.56	0.000405784098595116\\
15.57	0.000405757961880727\\
15.58	0.000405731813478994\\
15.59	0.000405705653382679\\
15.6	0.000405679481584561\\
15.61	0.000405653298077405\\
15.62	0.000405627102853957\\
15.63	0.000405600895906974\\
15.64	0.000405574677229193\\
15.65	0.000405548446813351\\
15.66	0.000405522204652174\\
15.67	0.000405495950738383\\
15.68	0.000405469685064688\\
15.69	0.000405443407623798\\
15.7	0.00040541711840841\\
15.71	0.000405390817411214\\
15.72	0.000405364504624894\\
15.73	0.00040533818004213\\
15.74	0.000405311843655594\\
15.75	0.000405285495457939\\
15.76	0.000405259135441828\\
15.77	0.000405232763599907\\
15.78	0.000405206379924819\\
15.79	0.000405179984409188\\
15.8	0.000405153577045654\\
15.81	0.000405127157826826\\
15.82	0.00040510072674532\\
15.83	0.000405074283793739\\
15.84	0.000405047828964677\\
15.85	0.000405021362250731\\
15.86	0.000404994883644478\\
15.87	0.000404968393138493\\
15.88	0.000404941890725345\\
15.89	0.000404915376397592\\
15.9	0.000404888850147795\\
15.91	0.000404862311968488\\
15.92	0.000404835761852216\\
15.93	0.000404809199791507\\
15.94	0.000404782625778885\\
15.95	0.000404756039806865\\
15.96	0.000404729441867957\\
15.97	0.000404702831954661\\
15.98	0.000404676210059473\\
15.99	0.000404649576174875\\
16	0.000404622930293344\\
16.01	0.000404596272407359\\
16.02	0.000404569602509376\\
16.03	0.000404542920591855\\
16.04	0.000404516226647238\\
16.05	0.000404489520667971\\
16.06	0.00040446280264649\\
16.07	0.000404436072575214\\
16.08	0.000404409330446569\\
16.09	0.000404382576252955\\
16.1	0.000404355809986785\\
16.11	0.000404329031640453\\
16.12	0.00040430224120634\\
16.13	0.000404275438676832\\
16.14	0.000404248624044298\\
16.15	0.000404221797301107\\
16.16	0.000404194958439612\\
16.17	0.000404168107452162\\
16.18	0.000404141244331106\\
16.19	0.000404114369068767\\
16.2	0.000404087481657486\\
16.21	0.000404060582089567\\
16.22	0.000404033670357325\\
16.23	0.000404006746453071\\
16.24	0.000403979810369094\\
16.25	0.000403952862097681\\
16.26	0.000403925901631113\\
16.27	0.000403898928961664\\
16.28	0.000403871944081596\\
16.29	0.000403844946983166\\
16.3	0.000403817937658625\\
16.31	0.000403790916100209\\
16.32	0.00040376388230016\\
16.33	0.000403736836250694\\
16.34	0.000403709777944031\\
16.35	0.00040368270737238\\
16.36	0.000403655624527943\\
16.37	0.000403628529402919\\
16.38	0.000403601421989483\\
16.39	0.000403574302279821\\
16.4	0.0004035471702661\\
16.41	0.000403520025940485\\
16.42	0.000403492869295132\\
16.43	0.000403465700322177\\
16.44	0.000403438519013767\\
16.45	0.000403411325362028\\
16.46	0.000403384119359081\\
16.47	0.000403356900997043\\
16.48	0.000403329670268023\\
16.49	0.000403302427164109\\
16.5	0.000403275171677405\\
16.51	0.000403247903799977\\
16.52	0.000403220623523912\\
16.53	0.000403193330841269\\
16.54	0.000403166025744106\\
16.55	0.000403138708224476\\
16.56	0.000403111378274415\\
16.57	0.000403084035885962\\
16.58	0.000403056681051138\\
16.59	0.000403029313761967\\
16.6	0.000403001934010447\\
16.61	0.000402974541788583\\
16.62	0.000402947137088373\\
16.63	0.000402919719901791\\
16.64	0.000402892290220823\\
16.65	0.000402864848037435\\
16.66	0.000402837393343578\\
16.67	0.00040280992613121\\
16.68	0.00040278244639228\\
16.69	0.000402754954118709\\
16.7	0.000402727449302438\\
16.71	0.000402699931935372\\
16.72	0.000402672402009434\\
16.73	0.000402644859516514\\
16.74	0.000402617304448512\\
16.75	0.000402589736797307\\
16.76	0.000402562156554778\\
16.77	0.0004025345637128\\
16.78	0.000402506958263224\\
16.79	0.000402479340197905\\
16.8	0.000402451709508684\\
16.81	0.000402424066187401\\
16.82	0.000402396410225875\\
16.83	0.000402368741615927\\
16.84	0.000402341060349365\\
16.85	0.000402313366417991\\
16.86	0.0004022856598136\\
16.87	0.000402257940527969\\
16.88	0.000402230208552876\\
16.89	0.00040220246388009\\
16.9	0.000402174706501369\\
16.91	0.00040214693640846\\
16.92	0.000402119153593106\\
16.93	0.00040209135804704\\
16.94	0.000402063549761987\\
16.95	0.000402035728729658\\
16.96	0.000402007894941761\\
16.97	0.000401980048389998\\
16.98	0.000401952189066057\\
16.99	0.000401924316961615\\
17	0.000401896432068343\\
17.01	0.000401868534377913\\
17.02	0.000401840623881974\\
17.03	0.000401812700572175\\
17.04	0.000401784764440149\\
17.05	0.000401756815477527\\
17.06	0.000401728853675933\\
17.07	0.000401700879026971\\
17.08	0.000401672891522245\\
17.09	0.000401644891153356\\
17.1	0.000401616877911879\\
17.11	0.000401588851789396\\
17.12	0.000401560812777473\\
17.13	0.000401532760867662\\
17.14	0.000401504696051524\\
17.15	0.000401476618320592\\
17.16	0.000401448527666401\\
17.17	0.000401420424080473\\
17.18	0.000401392307554321\\
17.19	0.000401364178079455\\
17.2	0.000401336035647366\\
17.21	0.000401307880249543\\
17.22	0.000401279711877466\\
17.23	0.0004012515305226\\
17.24	0.000401223336176407\\
17.25	0.000401195128830339\\
17.26	0.000401166908475841\\
17.27	0.000401138675104344\\
17.28	0.000401110428707276\\
17.29	0.000401082169276043\\
17.3	0.00040105389680206\\
17.31	0.000401025611276721\\
17.32	0.00040099731269142\\
17.33	0.000400969001037524\\
17.34	0.000400940676306414\\
17.35	0.000400912338489448\\
17.36	0.000400883987577974\\
17.37	0.000400855623563333\\
17.38	0.000400827246436867\\
17.39	0.000400798856189892\\
17.4	0.000400770452813729\\
17.41	0.000400742036299683\\
17.42	0.000400713606639045\\
17.43	0.000400685163823111\\
17.44	0.000400656707843148\\
17.45	0.000400628238690432\\
17.46	0.000400599756356229\\
17.47	0.000400571260831777\\
17.48	0.00040054275210832\\
17.49	0.000400514230177094\\
17.5	0.00040048569502932\\
17.51	0.000400457146656211\\
17.52	0.000400428585048967\\
17.53	0.00040040001019879\\
17.54	0.000400371422096852\\
17.55	0.000400342820734341\\
17.56	0.000400314206102417\\
17.57	0.000400285578192239\\
17.58	0.000400256936994954\\
17.59	0.0004002282825017\\
17.6	0.000400199614703601\\
17.61	0.000400170933591779\\
17.62	0.000400142239157342\\
17.63	0.000400113531391397\\
17.64	0.000400084810285023\\
17.65	0.000400056075829309\\
17.66	0.00040002732801532\\
17.67	0.00039999856683412\\
17.68	0.000399969792276769\\
17.69	0.000399941004334298\\
17.7	0.000399912202997741\\
17.71	0.000399883388258125\\
17.72	0.000399854560106463\\
17.73	0.00039982571853376\\
17.74	0.000399796863531006\\
17.75	0.00039976799508919\\
17.76	0.000399739113199286\\
17.77	0.000399710217852256\\
17.78	0.000399681309039057\\
17.79	0.000399652386750635\\
17.8	0.000399623450977923\\
17.81	0.000399594501711854\\
17.82	0.000399565538943337\\
17.83	0.000399536562663283\\
17.84	0.000399507572862586\\
17.85	0.000399478569532133\\
17.86	0.000399449552662802\\
17.87	0.00039942052224546\\
17.88	0.000399391478270962\\
17.89	0.000399362420730159\\
17.9	0.000399333349613883\\
17.91	0.00039930426491297\\
17.92	0.000399275166618229\\
17.93	0.00039924605472047\\
17.94	0.000399216929210496\\
17.95	0.000399187790079086\\
17.96	0.000399158637317024\\
17.97	0.000399129470915072\\
17.98	0.000399100290863997\\
17.99	0.000399071097154537\\
18	0.000399041889777433\\
18.01	0.000399012668723412\\
18.02	0.000398983433983192\\
18.03	0.000398954185547477\\
18.04	0.000398924923406969\\
18.05	0.000398895647552352\\
18.06	0.000398866357974302\\
18.07	0.000398837054663488\\
18.08	0.000398807737610566\\
18.09	0.000398778406806178\\
18.1	0.000398749062240968\\
18.11	0.000398719703905552\\
18.12	0.000398690331790552\\
18.13	0.000398660945886569\\
18.14	0.000398631546184205\\
18.15	0.00039860213267404\\
18.16	0.000398572705346645\\
18.17	0.000398543264192593\\
18.18	0.000398513809202425\\
18.19	0.000398484340366696\\
18.2	0.00039845485767593\\
18.21	0.000398425361120657\\
18.22	0.000398395850691386\\
18.23	0.000398366326378618\\
18.24	0.000398336788172844\\
18.25	0.000398307236064543\\
18.26	0.000398277670044191\\
18.27	0.000398248090102242\\
18.28	0.00039821849622915\\
18.29	0.000398188888415354\\
18.3	0.000398159266651278\\
18.31	0.000398129630927339\\
18.32	0.000398099981233949\\
18.33	0.000398070317561504\\
18.34	0.000398040639900381\\
18.35	0.000398010948240969\\
18.36	0.000397981242573625\\
18.37	0.000397951522888703\\
18.38	0.000397921789176551\\
18.39	0.000397892041427497\\
18.4	0.000397862279631862\\
18.41	0.000397832503779963\\
18.42	0.000397802713862094\\
18.43	0.000397772909868548\\
18.44	0.000397743091789603\\
18.45	0.00039771325961553\\
18.46	0.000397683413336583\\
18.47	0.000397653552943009\\
18.48	0.000397623678425049\\
18.49	0.000397593789772922\\
18.5	0.000397563886976841\\
18.51	0.000397533970027014\\
18.52	0.000397504038913633\\
18.53	0.000397474093626872\\
18.54	0.000397444134156913\\
18.55	0.000397414160493907\\
18.56	0.000397384172628008\\
18.57	0.000397354170549346\\
18.58	0.000397324154248056\\
18.59	0.000397294123714248\\
18.6	0.000397264078938028\\
18.61	0.000397234019909485\\
18.62	0.000397203946618714\\
18.63	0.00039717385905577\\
18.64	0.000397143757210726\\
18.65	0.000397113641073626\\
18.66	0.000397083510634506\\
18.67	0.000397053365883393\\
18.68	0.000397023206810307\\
18.69	0.000396993033405244\\
18.7	0.000396962845658207\\
18.71	0.000396932643559172\\
18.72	0.00039690242709811\\
18.73	0.000396872196264981\\
18.74	0.000396841951049737\\
18.75	0.000396811691442307\\
18.76	0.000396781417432624\\
18.77	0.000396751129010591\\
18.78	0.000396720826166123\\
18.79	0.00039669050888911\\
18.8	0.000396660177169425\\
18.81	0.000396629830996944\\
18.82	0.000396599470361516\\
18.83	0.000396569095252996\\
18.84	0.000396538705661208\\
18.85	0.000396508301575988\\
18.86	0.000396477882987136\\
18.87	0.000396447449884454\\
18.88	0.000396417002257734\\
18.89	0.000396386540096749\\
18.9	0.000396356063391267\\
18.91	0.000396325572131038\\
18.92	0.000396295066305808\\
18.93	0.000396264545905306\\
18.94	0.000396234010919247\\
18.95	0.000396203461337343\\
18.96	0.000396172897149291\\
18.97	0.000396142318344769\\
18.98	0.000396111724913453\\
18.99	0.000396081116844997\\
19	0.000396050494129057\\
19.01	0.000396019856755268\\
19.02	0.000395989204713251\\
19.03	0.000395958537992618\\
19.04	0.000395927856582979\\
19.05	0.000395897160473918\\
19.06	0.000395866449655011\\
19.07	0.000395835724115827\\
19.08	0.000395804983845916\\
19.09	0.000395774228834824\\
19.1	0.000395743459072076\\
19.11	0.000395712674547197\\
19.12	0.000395681875249685\\
19.13	0.000395651061169042\\
19.14	0.000395620232294744\\
19.15	0.000395589388616261\\
19.16	0.000395558530123052\\
19.17	0.000395527656804565\\
19.18	0.000395496768650234\\
19.19	0.000395465865649477\\
19.2	0.000395434947791701\\
19.21	0.000395404015066311\\
19.22	0.000395373067462691\\
19.23	0.000395342104970208\\
19.24	0.000395311127578226\\
19.25	0.000395280135276095\\
19.26	0.000395249128053152\\
19.27	0.000395218105898716\\
19.28	0.000395187068802099\\
19.29	0.00039515601675261\\
19.3	0.000395124949739527\\
19.31	0.000395093867752126\\
19.32	0.000395062770779669\\
19.33	0.000395031658811412\\
19.34	0.000395000531836585\\
19.35	0.000394969389844419\\
19.36	0.000394938232824125\\
19.37	0.000394907060764898\\
19.38	0.00039487587365593\\
19.39	0.0003948446714864\\
19.4	0.000394813454245463\\
19.41	0.000394782221922279\\
19.42	0.000394750974505977\\
19.43	0.000394719711985682\\
19.44	0.000394688434350518\\
19.45	0.00039465714158957\\
19.46	0.000394625833691932\\
19.47	0.000394594510646679\\
19.48	0.000394563172442873\\
19.49	0.000394531819069562\\
19.5	0.00039450045051578\\
19.51	0.000394469066770558\\
19.52	0.000394437667822902\\
19.53	0.000394406253661812\\
19.54	0.00039437482427627\\
19.55	0.000394343379655248\\
19.56	0.000394311919787711\\
19.57	0.000394280444662602\\
19.58	0.000394248954268851\\
19.59	0.000394217448595391\\
19.6	0.00039418592763112\\
19.61	0.000394154391364938\\
19.62	0.000394122839785726\\
19.63	0.000394091272882346\\
19.64	0.000394059690643668\\
19.65	0.00039402809305852\\
19.66	0.000393996480115746\\
19.67	0.000393964851804155\\
19.68	0.000393933208112551\\
19.69	0.000393901549029728\\
19.7	0.000393869874544459\\
19.71	0.000393838184645515\\
19.72	0.000393806479321638\\
19.73	0.000393774758561574\\
19.74	0.000393743022354051\\
19.75	0.000393711270687771\\
19.76	0.000393679503551434\\
19.77	0.000393647720933728\\
19.78	0.000393615922823328\\
19.79	0.000393584109208886\\
19.8	0.000393552280079047\\
19.81	0.000393520435422446\\
19.82	0.000393488575227701\\
19.83	0.000393456699483421\\
19.84	0.000393424808178189\\
19.85	0.000393392901300587\\
19.86	0.000393360978839179\\
19.87	0.000393329040782516\\
19.88	0.000393297087119137\\
19.89	0.000393265117837559\\
19.9	0.000393233132926305\\
19.91	0.000393201132373861\\
19.92	0.000393169116168716\\
19.93	0.000393137084299331\\
19.94	0.000393105036754172\\
19.95	0.000393072973521676\\
19.96	0.000393040894590272\\
19.97	0.000393008799948373\\
19.98	0.000392976689584384\\
19.99	0.000392944563486691\\
20	0.000392912421643668\\
20.01	0.000392880264043667\\
20.02	0.000392848090675043\\
20.03	0.000392815901526128\\
20.04	0.000392783696585228\\
20.05	0.000392751475840662\\
20.06	0.000392719239280711\\
20.07	0.000392686986893659\\
20.08	0.000392654718667756\\
20.09	0.000392622434591258\\
20.1	0.0003925901346524\\
20.11	0.000392557818839396\\
20.12	0.000392525487140463\\
20.13	0.000392493139543778\\
20.14	0.000392460776037535\\
20.15	0.000392428396609886\\
20.16	0.000392396001248985\\
20.17	0.000392363589942966\\
20.18	0.00039233116267995\\
20.19	0.000392298719448046\\
20.2	0.000392266260235346\\
20.21	0.000392233785029927\\
20.22	0.000392201293819852\\
20.23	0.000392168786593178\\
20.24	0.000392136263337929\\
20.25	0.000392103724042133\\
20.26	0.000392071168693794\\
20.27	0.000392038597280913\\
20.28	0.000392006009791456\\
20.29	0.00039197340621339\\
20.3	0.000391940786534664\\
20.31	0.000391908150743213\\
20.32	0.000391875498826959\\
20.33	0.000391842830773802\\
20.34	0.000391810146571638\\
20.35	0.000391777446208342\\
20.36	0.000391744729671774\\
20.37	0.000391711996949781\\
20.38	0.000391679248030189\\
20.39	0.000391646482900828\\
20.4	0.000391613701549489\\
20.41	0.000391580903963967\\
20.42	0.000391548090132032\\
20.43	0.000391515260041442\\
20.44	0.000391482413679943\\
20.45	0.000391449551035261\\
20.46	0.000391416672095113\\
20.47	0.000391383776847191\\
20.48	0.000391350865279182\\
20.49	0.00039131793737876\\
20.5	0.000391284993133574\\
20.51	0.000391252032531262\\
20.52	0.000391219055559453\\
20.53	0.000391186062205751\\
20.54	0.000391153052457753\\
20.55	0.000391120026303034\\
20.56	0.000391086983729156\\
20.57	0.000391053924723675\\
20.58	0.00039102084927412\\
20.59	0.000390987757368009\\
20.6	0.000390954648992844\\
20.61	0.000390921524136115\\
20.62	0.000390888382785292\\
20.63	0.000390855224927836\\
20.64	0.000390822050551182\\
20.65	0.000390788859642759\\
20.66	0.000390755652189977\\
20.67	0.000390722428180236\\
20.68	0.000390689187600912\\
20.69	0.000390655930439368\\
20.7	0.000390622656682956\\
20.71	0.000390589366319014\\
20.72	0.000390556059334852\\
20.73	0.000390522735717776\\
20.74	0.000390489395455072\\
20.75	0.000390456038534015\\
20.76	0.000390422664941853\\
20.77	0.00039038927466583\\
20.78	0.000390355867693174\\
20.79	0.00039032244401109\\
20.8	0.00039028900360677\\
20.81	0.000390255546467391\\
20.82	0.000390222072580116\\
20.83	0.000390188581932089\\
20.84	0.000390155074510441\\
20.85	0.000390121550302287\\
20.86	0.000390088009294721\\
20.87	0.000390054451474825\\
20.88	0.000390020876829668\\
20.89	0.000389987285346293\\
20.9	0.000389953677011742\\
20.91	0.00038992005181303\\
20.92	0.00038988640973716\\
20.93	0.000389852750771118\\
20.94	0.000389819074901869\\
20.95	0.000389785382116366\\
20.96	0.000389751672401555\\
20.97	0.00038971794574435\\
20.98	0.000389684202131656\\
20.99	0.000389650441550366\\
21	0.000389616663987345\\
21.01	0.000389582869429453\\
21.02	0.000389549057863531\\
21.03	0.000389515229276402\\
21.04	0.000389481383654871\\
21.05	0.00038944752098573\\
21.06	0.000389413641255748\\
21.07	0.00038937974445169\\
21.08	0.000389345830560292\\
21.09	0.000389311899568279\\
21.1	0.000389277951462359\\
21.11	0.000389243986229222\\
21.12	0.000389210003855551\\
21.13	0.000389176004327991\\
21.14	0.000389141987633193\\
21.15	0.000389107953757776\\
21.16	0.00038907390268835\\
21.17	0.000389039834411508\\
21.18	0.000389005748913821\\
21.19	0.000388971646181849\\
21.2	0.000388937526202129\\
21.21	0.000388903388961189\\
21.22	0.000388869234445534\\
21.23	0.000388835062641653\\
21.24	0.000388800873536025\\
21.25	0.000388766667115096\\
21.26	0.000388732443365311\\
21.27	0.00038869820227309\\
21.28	0.000388663943824837\\
21.29	0.000388629668006945\\
21.3	0.00038859537480578\\
21.31	0.000388561064207694\\
21.32	0.000388526736199026\\
21.33	0.000388492390766096\\
21.34	0.000388458027895205\\
21.35	0.000388423647572633\\
21.36	0.000388389249784651\\
21.37	0.000388354834517509\\
21.38	0.000388320401757437\\
21.39	0.000388285951490652\\
21.4	0.000388251483703353\\
21.41	0.000388216998381716\\
21.42	0.000388182495511905\\
21.43	0.000388147975080069\\
21.44	0.00038811343707233\\
21.45	0.000388078881474794\\
21.46	0.000388044308273565\\
21.47	0.000388009717454708\\
21.48	0.000387975109004287\\
21.49	0.00038794048290834\\
21.5	0.00038790583915288\\
21.51	0.000387871177723919\\
21.52	0.000387836498607439\\
21.53	0.000387801801789413\\
21.54	0.000387767087255787\\
21.55	0.000387732354992492\\
21.56	0.000387697604985445\\
21.57	0.000387662837220542\\
21.58	0.000387628051683664\\
21.59	0.000387593248360664\\
21.6	0.000387558427237393\\
21.61	0.000387523588299668\\
21.62	0.000387488731533296\\
21.63	0.000387453856924073\\
21.64	0.000387418964457757\\
21.65	0.000387384054120112\\
21.66	0.000387349125896861\\
21.67	0.000387314179773722\\
21.68	0.000387279215736397\\
21.69	0.000387244233770555\\
21.7	0.000387209233861867\\
21.71	0.000387174215995968\\
21.72	0.00038713918015848\\
21.73	0.000387104126335008\\
21.74	0.000387069054511142\\
21.75	0.000387033964672445\\
21.76	0.000386998856804469\\
21.77	0.000386963730892746\\
21.78	0.000386928586922788\\
21.79	0.000386893424880084\\
21.8	0.000386858244750111\\
21.81	0.000386823046518325\\
21.82	0.000386787830170165\\
21.83	0.000386752595691042\\
21.84	0.000386717343066363\\
21.85	0.000386682072281508\\
21.86	0.000386646783321835\\
21.87	0.000386611476172687\\
21.88	0.000386576150819388\\
21.89	0.000386540807247244\\
21.9	0.000386505445441544\\
21.91	0.000386470065387549\\
21.92	0.00038643466707051\\
21.93	0.000386399250475653\\
21.94	0.000386363815588188\\
21.95	0.000386328362393306\\
21.96	0.000386292890876175\\
21.97	0.000386257401021949\\
21.98	0.00038622189281576\\
21.99	0.000386186366242722\\
22	0.000386150821287928\\
22.01	0.000386115257936448\\
22.02	0.000386079676173339\\
22.03	0.000386044075983643\\
22.04	0.000386008457352363\\
22.05	0.000385972820264506\\
22.06	0.000385937164705044\\
22.07	0.000385901490658933\\
22.08	0.000385865798111111\\
22.09	0.000385830087046493\\
22.1	0.000385794357449985\\
22.11	0.000385758609306461\\
22.12	0.000385722842600777\\
22.13	0.000385687057317773\\
22.14	0.000385651253442264\\
22.15	0.000385615430959056\\
22.16	0.000385579589852919\\
22.17	0.000385543730108621\\
22.18	0.000385507851710891\\
22.19	0.000385471954644454\\
22.2	0.000385436038894009\\
22.21	0.00038540010444423\\
22.22	0.000385364151279783\\
22.23	0.000385328179385294\\
22.24	0.000385292188745391\\
22.25	0.000385256179344668\\
22.26	0.000385220151167702\\
22.27	0.000385184104199051\\
22.28	0.000385148038423251\\
22.29	0.000385111953824815\\
22.3	0.000385075850388246\\
22.31	0.000385039728098013\\
22.32	0.000385003586938571\\
22.33	0.000384967426894357\\
22.34	0.000384931247949782\\
22.35	0.000384895050089243\\
22.36	0.000384858833297107\\
22.37	0.000384822597557724\\
22.38	0.00038478634285543\\
22.39	0.000384750069174536\\
22.4	0.000384713776499325\\
22.41	0.000384677464814064\\
22.42	0.00038464113410301\\
22.43	0.000384604784350377\\
22.44	0.00038456841554038\\
22.45	0.000384532027657199\\
22.46	0.000384495620684997\\
22.47	0.000384459194607912\\
22.48	0.000384422749410073\\
22.49	0.000384386285075573\\
22.5	0.000384349801588496\\
22.51	0.000384313298932895\\
22.52	0.000384276777092806\\
22.53	0.000384240236052245\\
22.54	0.000384203675795204\\
22.55	0.000384167096305654\\
22.56	0.000384130497567546\\
22.57	0.000384093879564805\\
22.58	0.000384057242281344\\
22.59	0.000384020585701047\\
22.6	0.000383983909807779\\
22.61	0.000383947214585373\\
22.62	0.00038391050001766\\
22.63	0.000383873766088432\\
22.64	0.000383837012781472\\
22.65	0.000383800240080527\\
22.66	0.000383763447969334\\
22.67	0.000383726636431604\\
22.68	0.00038368980545103\\
22.69	0.000383652955011271\\
22.7	0.000383616085095979\\
22.71	0.000383579195688771\\
22.72	0.000383542286773252\\
22.73	0.000383505358333\\
22.74	0.000383468410351567\\
22.75	0.000383431442812499\\
22.76	0.000383394455699293\\
22.77	0.000383357448995446\\
22.78	0.00038332042268442\\
22.79	0.000383283376749668\\
22.8	0.000383246311174605\\
22.81	0.000383209225942633\\
22.82	0.000383172121037126\\
22.83	0.00038313499644144\\
22.84	0.000383097852138912\\
22.85	0.000383060688112842\\
22.86	0.000383023504346519\\
22.87	0.000382986300823208\\
22.88	0.000382949077526147\\
22.89	0.000382911834438554\\
22.9	0.000382874571543625\\
22.91	0.000382837288824532\\
22.92	0.00038279998626442\\
22.93	0.000382762663846415\\
22.94	0.000382725321553621\\
22.95	0.000382687959369117\\
22.96	0.000382650577275958\\
22.97	0.000382613175257174\\
22.98	0.00038257575329578\\
22.99	0.000382538311374754\\
23	0.000382500849477066\\
23.01	0.000382463367585647\\
23.02	0.000382425865683421\\
23.03	0.000382388343753273\\
23.04	0.000382350801778073\\
23.05	0.000382313239740665\\
23.06	0.000382275657623872\\
23.07	0.000382238055410489\\
23.08	0.000382200433083289\\
23.09	0.000382162790625022\\
23.1	0.000382125128018414\\
23.11	0.000382087445246164\\
23.12	0.000382049742290948\\
23.13	0.000382012019135427\\
23.14	0.000381974275762222\\
23.15	0.00038193651215394\\
23.16	0.000381898728293166\\
23.17	0.000381860924162451\\
23.18	0.000381823099744326\\
23.19	0.000381785255021305\\
23.2	0.000381747389975866\\
23.21	0.000381709504590468\\
23.22	0.00038167159884755\\
23.23	0.000381633672729521\\
23.24	0.000381595726218755\\
23.25	0.000381557759297628\\
23.26	0.000381519771948467\\
23.27	0.000381481764153584\\
23.28	0.000381443735895264\\
23.29	0.000381405687155766\\
23.3	0.000381367617917332\\
23.31	0.00038132952816217\\
23.32	0.000381291417872461\\
23.33	0.000381253287030371\\
23.34	0.000381215135618037\\
23.35	0.000381176963617559\\
23.36	0.000381138771011032\\
23.37	0.000381100557780515\\
23.38	0.000381062323908036\\
23.39	0.000381024069375607\\
23.4	0.000380985794165206\\
23.41	0.0003809474982588\\
23.42	0.000380909181638312\\
23.43	0.000380870844285654\\
23.44	0.000380832486182701\\
23.45	0.000380794107311313\\
23.46	0.000380755707653311\\
23.47	0.0003807172871905\\
23.48	0.000380678845904659\\
23.49	0.000380640383777534\\
23.5	0.000380601900790849\\
23.51	0.000380563396926306\\
23.52	0.000380524872165574\\
23.53	0.000380486326490294\\
23.54	0.000380447759882092\\
23.55	0.000380409172322555\\
23.56	0.000380370563793251\\
23.57	0.000380331934275722\\
23.58	0.000380293283751469\\
23.59	0.000380254612201989\\
23.6	0.000380215919608734\\
23.61	0.000380177205953137\\
23.62	0.000380138471216609\\
23.63	0.000380099715380517\\
23.64	0.000380060938426218\\
23.65	0.000380022140335036\\
23.66	0.000379983321088266\\
23.67	0.000379944480667171\\
23.68	0.000379905619053003\\
23.69	0.00037986673622697\\
23.7	0.000379827832170261\\
23.71	0.000379788906864027\\
23.72	0.000379749960289413\\
23.73	0.000379710992427512\\
23.74	0.000379672003259406\\
23.75	0.000379632992766136\\
23.76	0.000379593960928723\\
23.77	0.000379554907728166\\
23.78	0.000379515833145425\\
23.79	0.000379476737161429\\
23.8	0.000379437619757091\\
23.81	0.000379398480913293\\
23.82	0.000379359320610881\\
23.83	0.000379320138830673\\
23.84	0.000379280935553465\\
23.85	0.000379241710760031\\
23.86	0.000379202464431091\\
23.87	0.000379163196547364\\
23.88	0.000379123907089518\\
23.89	0.00037908459603821\\
23.9	0.000379045263374057\\
23.91	0.000379005909077643\\
23.92	0.00037896653312954\\
23.93	0.000378927135510276\\
23.94	0.000378887716200354\\
23.95	0.000378848275180243\\
23.96	0.000378808812430388\\
23.97	0.000378769327931204\\
23.98	0.000378729821663076\\
23.99	0.000378690293606352\\
24	0.000378650743741363\\
24.01	0.000378611172048396\\
24.02	0.00037857157850772\\
24.03	0.00037853196309956\\
24.04	0.000378492325804128\\
24.05	0.000378452666601591\\
24.06	0.000378412985472091\\
24.07	0.000378373282395738\\
24.08	0.000378333557352612\\
24.09	0.000378293810322767\\
24.1	0.000378254041286216\\
24.11	0.000378214250222947\\
24.12	0.000378174437112917\\
24.13	0.00037813460193605\\
24.14	0.000378094744672239\\
24.15	0.000378054865301342\\
24.16	0.000378014963803193\\
24.17	0.000377975040157588\\
24.18	0.000377935094344297\\
24.19	0.000377895126343047\\
24.2	0.000377855136133544\\
24.21	0.00037781512369546\\
24.22	0.000377775089008426\\
24.23	0.000377735032052053\\
24.24	0.000377694952805908\\
24.25	0.000377654851249534\\
24.26	0.000377614727362434\\
24.27	0.000377574581124087\\
24.28	0.00037753441251393\\
24.29	0.000377494221511373\\
24.3	0.000377454008095784\\
24.31	0.000377413772246512\\
24.32	0.000377373513942856\\
24.33	0.000377333233164089\\
24.34	0.000377292929889455\\
24.35	0.000377252604098155\\
24.36	0.000377212255769364\\
24.37	0.000377171884882217\\
24.38	0.000377131491415814\\
24.39	0.000377091075349223\\
24.4	0.000377050636661476\\
24.41	0.000377010175331573\\
24.42	0.00037696969133847\\
24.43	0.000376929184661104\\
24.44	0.000376888655278358\\
24.45	0.000376848103169096\\
24.46	0.000376807528312132\\
24.47	0.000376766930686254\\
24.48	0.000376726310270206\\
24.49	0.000376685667042707\\
24.5	0.00037664500098243\\
24.51	0.000376604312068016\\
24.52	0.000376563600278065\\
24.53	0.000376522865591145\\
24.54	0.000376482107985783\\
24.55	0.000376441327440467\\
24.56	0.000376400523933659\\
24.57	0.000376359697443769\\
24.58	0.000376318847949178\\
24.59	0.000376277975428227\\
24.6	0.000376237079859214\\
24.61	0.000376196161220407\\
24.62	0.000376155219490032\\
24.63	0.00037611425464627\\
24.64	0.000376073266667274\\
24.65	0.000376032255531145\\
24.66	0.000375991221215958\\
24.67	0.000375950163699739\\
24.68	0.000375909082960474\\
24.69	0.000375867978976116\\
24.7	0.00037582685172457\\
24.71	0.000375785701183708\\
24.72	0.000375744527331353\\
24.73	0.000375703330145291\\
24.74	0.000375662109603266\\
24.75	0.000375620865682984\\
24.76	0.000375579598362102\\
24.77	0.000375538307618243\\
24.78	0.000375496993428979\\
24.79	0.000375455655771845\\
24.8	0.000375414294624333\\
24.81	0.000375372909963891\\
24.82	0.000375331501767925\\
24.83	0.000375290070013794\\
24.84	0.000375248614678815\\
24.85	0.000375207135740262\\
24.86	0.00037516563317536\\
24.87	0.000375124106961293\\
24.88	0.000375082557075202\\
24.89	0.000375040983494175\\
24.9	0.000374999386195262\\
24.91	0.00037495776515546\\
24.92	0.000374916120351728\\
24.93	0.000374874451760971\\
24.94	0.000374832759360049\\
24.95	0.00037479104312577\\
24.96	0.000374749303034907\\
24.97	0.000374707539064174\\
24.98	0.000374665751190236\\
24.99	0.000374623939389715\\
25	0.000374582103639181\\
25.01	0.000374540243915152\\
25.02	0.000374498360194103\\
25.03	0.000374456452452451\\
25.04	0.000374414520666565\\
25.05	0.000374372564812762\\
25.06	0.00037433058486731\\
25.07	0.000374288580806424\\
25.08	0.000374246552606265\\
25.09	0.000374204500242937\\
25.1	0.000374162423692499\\
25.11	0.000374120322930955\\
25.12	0.00037407819793425\\
25.13	0.000374036048678271\\
25.14	0.000373993875138862\\
25.15	0.000373951677291803\\
25.16	0.000373909455112816\\
25.17	0.000373867208577574\\
25.18	0.000373824937661681\\
25.19	0.000373782642340698\\
25.2	0.000373740322590115\\
25.21	0.000373697978385367\\
25.22	0.000373655609701835\\
25.23	0.000373613216514834\\
25.24	0.000373570798799612\\
25.25	0.000373528356531377\\
25.26	0.000373485889685251\\
25.27	0.000373443398236308\\
25.28	0.000373400882159558\\
25.29	0.000373358341429944\\
25.3	0.000373315776022346\\
25.31	0.000373273185911571\\
25.32	0.000373230571072377\\
25.33	0.000373187931479443\\
25.34	0.000373145267107387\\
25.35	0.000373102577930752\\
25.36	0.000373059863924023\\
25.37	0.000373017125061605\\
25.38	0.000372974361317838\\
25.39	0.000372931572666995\\
25.4	0.000372888759083275\\
25.41	0.000372845920540796\\
25.42	0.000372803057013621\\
25.43	0.000372760168475719\\
25.44	0.000372717254900998\\
25.45	0.000372674316263286\\
25.46	0.000372631352536333\\
25.47	0.000372588363693813\\
25.48	0.000372545349709321\\
25.49	0.000372502310556378\\
25.5	0.000372459246208413\\
25.51	0.00037241615663878\\
25.52	0.000372373041820761\\
25.53	0.000372329901727536\\
25.54	0.00037228673633221\\
25.55	0.000372243545607802\\
25.56	0.000372200329527245\\
25.57	0.00037215708806339\\
25.58	0.000372113821188987\\
25.59	0.0003720705288767\\
25.6	0.000372027211099111\\
25.61	0.000371983867828701\\
25.62	0.000371940499037856\\
25.63	0.000371897104698871\\
25.64	0.000371853684783952\\
25.65	0.000371810239265193\\
25.66	0.000371766768114595\\
25.67	0.00037172327130407\\
25.68	0.000371679748805412\\
25.69	0.000371636200590324\\
25.7	0.000371592626630399\\
25.71	0.00037154902689712\\
25.72	0.000371505401361874\\
25.73	0.000371461749995933\\
25.74	0.00037141807277046\\
25.75	0.000371374369656502\\
25.76	0.000371330640624997\\
25.77	0.000371286885646768\\
25.78	0.000371243104692515\\
25.79	0.000371199297732829\\
25.8	0.00037115546473817\\
25.81	0.000371111605678881\\
25.82	0.000371067720525182\\
25.83	0.000371023809247166\\
25.84	0.000370979871814794\\
25.85	0.000370935908197903\\
25.86	0.000370891918366196\\
25.87	0.000370847902289235\\
25.88	0.000370803859936453\\
25.89	0.000370759791277148\\
25.9	0.000370715696280463\\
25.91	0.000370671574915419\\
25.92	0.000370627427150867\\
25.93	0.000370583252955532\\
25.94	0.000370539052297976\\
25.95	0.000370494825146618\\
25.96	0.000370450571469708\\
25.97	0.000370406291235355\\
25.98	0.000370361984411491\\
25.99	0.000370317650965901\\
26	0.000370273290866191\\
26.01	0.000370228904079811\\
26.02	0.000370184490574021\\
26.03	0.000370140050315932\\
26.04	0.000370095583272457\\
26.05	0.000370051089410328\\
26.06	0.000370006568696115\\
26.07	0.000369962021096175\\
26.08	0.000369917446576691\\
26.09	0.000369872845103651\\
26.1	0.000369828216642835\\
26.11	0.000369783561159838\\
26.12	0.000369738878620036\\
26.13	0.00036969416898861\\
26.14	0.000369649432230524\\
26.15	0.000369604668310522\\
26.16	0.000369559877193134\\
26.17	0.000369515058842669\\
26.18	0.000369470213223198\\
26.19	0.000369425340298575\\
26.2	0.0003693804400324\\
26.21	0.000369335512388052\\
26.22	0.000369290557328653\\
26.23	0.000369245574817077\\
26.24	0.000369200564815941\\
26.25	0.000369155527287604\\
26.26	0.000369110462194167\\
26.27	0.000369065369497459\\
26.28	0.00036902024915903\\
26.29	0.000368975101140156\\
26.3	0.00036892992540182\\
26.31	0.000368884721904722\\
26.32	0.000368839490609268\\
26.33	0.000368794231475546\\
26.34	0.000368748944463352\\
26.35	0.000368703629532159\\
26.36	0.000368658286641127\\
26.37	0.000368612915749076\\
26.38	0.000368567516814515\\
26.39	0.000368522089795584\\
26.4	0.000368476634650094\\
26.41	0.000368431151335506\\
26.42	0.000368385639808906\\
26.43	0.000368340100027018\\
26.44	0.000368294531946192\\
26.45	0.000368248935522389\\
26.46	0.00036820331071118\\
26.47	0.000368157657467741\\
26.48	0.000368111975746837\\
26.49	0.000368066265502809\\
26.5	0.000368020526689586\\
26.51	0.000367974759260654\\
26.52	0.00036792896316906\\
26.53	0.000367883138367397\\
26.54	0.00036783728480779\\
26.55	0.000367791402441907\\
26.56	0.000367745491220921\\
26.57	0.000367699551095517\\
26.58	0.000367653582015882\\
26.59	0.000367607583931683\\
26.6	0.000367561556792066\\
26.61	0.000367515500545642\\
26.62	0.000367469415140475\\
26.63	0.000367423300524066\\
26.64	0.000367377156643358\\
26.65	0.000367330983444691\\
26.66	0.000367284780873828\\
26.67	0.000367238548875907\\
26.68	0.000367192287395455\\
26.69	0.000367145996376361\\
26.7	0.000367099675761859\\
26.71	0.000367053325494523\\
26.72	0.000367006945516245\\
26.73	0.000366960535768228\\
26.74	0.000366914096190961\\
26.75	0.000366867626724205\\
26.76	0.000366821127306984\\
26.77	0.000366774597877569\\
26.78	0.000366728038373445\\
26.79	0.00036668144873131\\
26.8	0.00036663482888705\\
26.81	0.000366588178775724\\
26.82	0.000366541498331541\\
26.83	0.000366494787487846\\
26.84	0.000366448046177095\\
26.85	0.000366401274330842\\
26.86	0.000366354471879706\\
26.87	0.000366307638753366\\
26.88	0.000366260774880519\\
26.89	0.000366213880188884\\
26.9	0.000366166954605146\\
26.91	0.00036611999805497\\
26.92	0.000366073010462944\\
26.93	0.000366025991752582\\
26.94	0.000365978941846262\\
26.95	0.000365931860665249\\
26.96	0.000365884748129625\\
26.97	0.000365837604158285\\
26.98	0.000365790428668907\\
26.99	0.000365743221577907\\
27	0.000365695982800441\\
27.01	0.000365648712250342\\
27.02	0.000365601409840101\\
27.03	0.000365554075480853\\
27.04	0.00036550670908231\\
27.05	0.000365459310552763\\
27.06	0.000365411879799018\\
27.07	0.000365364416726383\\
27.08	0.000365316921238623\\
27.09	0.000365269393237919\\
27.1	0.000365221832624836\\
27.11	0.000365174239298289\\
27.12	0.000365126613155483\\
27.13	0.000365078954091901\\
27.14	0.000365031262001244\\
27.15	0.000364983536775379\\
27.16	0.000364935778304322\\
27.17	0.000364887986476173\\
27.18	0.000364840161177072\\
27.19	0.000364792302291153\\
27.2	0.000364744409700503\\
27.21	0.000364696483285095\\
27.22	0.000364648522922756\\
27.23	0.000364600528489098\\
27.24	0.000364552499857465\\
27.25	0.000364504436898899\\
27.26	0.000364456339482051\\
27.27	0.000364408207473139\\
27.28	0.000364360040735902\\
27.29	0.000364311839131501\\
27.3	0.000364263602518499\\
27.31	0.000364215330752753\\
27.32	0.000364167023687391\\
27.33	0.000364118681172705\\
27.34	0.000364070303056104\\
27.35	0.000364021889182034\\
27.36	0.000363973439391908\\
27.37	0.000363924953524017\\
27.38	0.000363876431413472\\
27.39	0.00036382787289211\\
27.4	0.000363779277788412\\
27.41	0.000363730645927429\\
27.42	0.000363681977130681\\
27.43	0.000363633271216078\\
27.44	0.000363584527997821\\
27.45	0.000363535747286321\\
27.46	0.000363486928888091\\
27.47	0.000363438072605644\\
27.48	0.000363389178237404\\
27.49	0.000363340245577591\\
27.5	0.000363291274416125\\
27.51	0.000363242264538493\\
27.52	0.000363193215725673\\
27.53	0.000363144127753973\\
27.54	0.000363095000394952\\
27.55	0.000363045833415275\\
27.56	0.000362996626576594\\
27.57	0.000362947379635411\\
27.58	0.000362898092342958\\
27.59	0.000362848764445048\\
27.6	0.000362799395681948\\
27.61	0.000362749985788221\\
27.62	0.00036270053449259\\
27.63	0.000362651041517772\\
27.64	0.000362601506580344\\
27.65	0.00036255192939056\\
27.66	0.000362502309652193\\
27.67	0.000362452647062382\\
27.68	0.000362402941311429\\
27.69	0.000362353192082637\\
27.7	0.000362303399052142\\
27.71	0.00036225356188868\\
27.72	0.000362203680253441\\
27.73	0.000362153753799841\\
27.74	0.000362103782173323\\
27.75	0.000362053765011156\\
27.76	0.000362003701942198\\
27.77	0.0003619535925867\\
27.78	0.000361903436556057\\
27.79	0.000361853233452577\\
27.8	0.000361802982869254\\
27.81	0.000361752684389496\\
27.82	0.000361702337586889\\
27.83	0.000361651942024924\\
27.84	0.000361601497256735\\
27.85	0.000361551002824825\\
27.86	0.000361500458260763\\
27.87	0.000361449863084907\\
27.88	0.000361399216806106\\
27.89	0.00036134851892137\\
27.9	0.000361297768915581\\
27.91	0.000361246966261134\\
27.92	0.000361196110417628\\
27.93	0.000361145200831506\\
27.94	0.000361094236935696\\
27.95	0.000361043218149264\\
27.96	0.000360992143877015\\
27.97	0.000360941013509123\\
27.98	0.000360889826420721\\
27.99	0.00036083858197151\\
28	0.000360787279505306\\
28.01	0.000360735918349649\\
28.02	0.000360684497815332\\
28.03	0.00036063301719594\\
28.04	0.000360581475767402\\
28.05	0.000360529872787481\\
28.06	0.0003604782074953\\
28.07	0.000360426479110809\\
28.08	0.00036037468683427\\
28.09	0.000360322829845727\\
28.1	0.00036027090730441\\
28.11	0.000360218918348214\\
28.12	0.000360166862093062\\
28.13	0.00036011473763233\\
28.14	0.000360062544036205\\
28.15	0.000360010280351054\\
28.16	0.000359957945598755\\
28.17	0.000359905538776014\\
28.18	0.000359853058853691\\
28.19	0.000359800504776048\\
28.2	0.000359747875460021\\
28.21	0.000359695169794469\\
28.22	0.000359642386639374\\
28.23	0.000359589524825036\\
28.24	0.000359536583151262\\
28.25	0.000359483560386479\\
28.26	0.000359430455266891\\
28.27	0.000359377266495544\\
28.28	0.000359323992741415\\
28.29	0.000359270632638433\\
28.3	0.000359217184784519\\
28.31	0.000359163647740551\\
28.32	0.000359110020029325\\
28.33	0.00035905630013448\\
28.34	0.00035900248649939\\
28.35	0.000358948577526027\\
28.36	0.00035889457157378\\
28.37	0.000358840466958261\\
28.38	0.000358786261950055\\
28.39	0.000358731954773433\\
28.4	0.000358677543605039\\
28.41	0.00035862302657255\\
28.42	0.000358568401753251\\
28.43	0.000358513667172626\\
28.44	0.00035845882080285\\
28.45	0.000358403860561294\\
28.46	0.000358348784308935\\
28.47	0.000358293589848744\\
28.48	0.00035823827492403\\
28.49	0.000358182837216727\\
28.5	0.000358127274345621\\
28.51	0.000358071583864536\\
28.52	0.000358015763260477\\
28.53	0.000357959809951685\\
28.54	0.000357903721285672\\
28.55	0.000357847494537154\\
28.56	0.000357791126905978\\
28.57	0.000357734615514934\\
28.58	0.000357677957407532\\
28.59	0.000357621149545696\\
28.6	0.000357564188807408\\
28.61	0.000357507071984261\\
28.62	0.000357449795778952\\
28.63	0.000357392356802695\\
28.64	0.000357334751572557\\
28.65	0.000357276976508714\\
28.66	0.000357219027931623\\
28.67	0.000357160902059105\\
28.68	0.000357102595003352\\
28.69	0.000357044102767837\\
28.7	0.000356985421244118\\
28.71	0.000356926546208568\\
28.72	0.000356867473319004\\
28.73	0.000356808198111179\\
28.74	0.000356748715995218\\
28.75	0.000356689022251919\\
28.76	0.000356629112028925\\
28.77	0.000356568980336829\\
28.78	0.000356508622045108\\
28.79	0.000356448031877963\\
28.8	0.000356387204410035\\
28.81	0.000356326134061974\\
28.82	0.000356264815095876\\
28.83	0.000356203241610597\\
28.84	0.000356141407536889\\
28.85	0.000356079306632446\\
28.86	0.000356016932476739\\
28.87	0.000355954278465711\\
28.88	0.000355891337806341\\
28.89	0.000355828103510992\\
28.9	0.000355764568391632\\
28.91	0.00035570072505383\\
28.92	0.000355636565890613\\
28.93	0.00035557208307611\\
28.94	0.000355507268559005\\
28.95	0.000355442114055781\\
28.96	0.000355376611043777\\
28.97	0.000355310750754009\\
28.98	0.000355244524163778\\
28.99	0.000355177921989053\\
29	0.000355110934676625\\
29.01	0.000355043552395995\\
29.02	0.000354975765031029\\
29.03	0.000354907562171367\\
29.04	0.000354838933103526\\
29.05	0.000354769866801776\\
29.06	0.000354700351918697\\
29.07	0.00035463037677546\\
29.08	0.000354559929351796\\
29.09	0.000354488997275674\\
29.1	0.000354417567812624\\
29.11	0.000354345627854765\\
29.12	0.000354273163909463\\
29.13	0.000354200162087651\\
29.14	0.000354126608091787\\
29.15	0.000354052487203421\\
29.16	0.000353977784270381\\
29.17	0.000353902483693585\\
29.18	0.000353826569413384\\
29.19	0.000353750024895547\\
29.2	0.000353672833116733\\
29.21	0.000353594976549588\\
29.22	0.000353516437147308\\
29.23	0.000353437196327754\\
29.24	0.000353357234957061\\
29.25	0.000353276533332732\\
29.26	0.000353195071166193\\
29.27	0.000353112827564823\\
29.28	0.000353029781013398\\
29.29	0.000352945909354941\\
29.3	0.000352861189771014\\
29.31	0.000352775598761344\\
29.32	0.000352689112122828\\
29.33	0.000352601704927877\\
29.34	0.000352513351502057\\
29.35	0.000352424025401058\\
29.36	0.000352333699386911\\
29.37	0.000352242345403452\\
29.38	0.000352149934551023\\
29.39	0.000352056437060377\\
29.4	0.000351961822265726\\
29.41	0.000351866058576971\\
29.42	0.00035176911345104\\
29.43	0.000351670953362301\\
29.44	0.000351571543772055\\
29.45	0.000351470849097063\\
29.46	0.000351368832677048\\
29.47	0.000351265456741211\\
29.48	0.00035116068237361\\
29.49	0.000351054469477516\\
29.5	0.000350946776738576\\
29.51	0.000350837561586836\\
29.52	0.00035072678015754\\
29.53	0.000350614387250672\\
29.54	0.000350500336289224\\
29.55	0.000350384579276131\\
29.56	0.000350268543120801\\
29.57	0.000350152469212061\\
29.58	0.000350036357866564\\
29.59	0.00034992020940937\\
29.6	0.000349804024174069\\
29.61	0.000349687802502861\\
29.62	0.000349571544746647\\
29.63	0.000349455251265151\\
29.64	0.000349338922426976\\
29.65	0.0003492225586097\\
29.66	0.000349106160199952\\
29.67	0.000348989727593486\\
29.68	0.00034887326119524\\
29.69	0.000348756761419416\\
29.7	0.0003486402286895\\
29.71	0.000348523663438344\\
29.72	0.000348407066108187\\
29.73	0.000348290437150676\\
29.74	0.000348173777026905\\
29.75	0.000348057086207425\\
29.76	0.000347940365172231\\
29.77	0.000347823614410766\\
29.78	0.000347706834421903\\
29.79	0.000347590025713896\\
29.8	0.000347473188804352\\
29.81	0.000347356324220159\\
29.82	0.000347239432497422\\
29.83	0.000347122514181369\\
29.84	0.000347005569826239\\
29.85	0.000346888599995165\\
29.86	0.000346771605260036\\
29.87	0.000346654586201331\\
29.88	0.000346537543407941\\
29.89	0.000346420477476957\\
29.9	0.000346303389013458\\
29.91	0.000346186278630253\\
29.92	0.000346069146947608\\
29.93	0.000345951994592938\\
29.94	0.000345834822200497\\
29.95	0.000345717630410993\\
29.96	0.000345600419871212\\
29.97	0.000345483191233599\\
29.98	0.000345365945155785\\
29.99	0.000345248682300106\\
30	0.000345131403333072\\
30.01	0.00034501410892479\\
30.02	0.000344896799748353\\
30.03	0.000344779476479175\\
30.04	0.000344662139794307\\
30.05	0.000344544790371664\\
30.06	0.00034442742888924\\
30.07	0.000344310056024234\\
30.08	0.00034419267245216\\
30.09	0.000344075278845868\\
30.1	0.000343957875874506\\
30.11	0.000343840464202435\\
30.12	0.000343723044488072\\
30.13	0.00034360561738264\\
30.14	0.000343488183528884\\
30.15	0.000343370743559675\\
30.16	0.000343253298096558\\
30.17	0.000343135847748201\\
30.18	0.000343018393108765\\
30.19	0.00034290093475616\\
30.2	0.000342783473250263\\
30.21	0.000342666009130955\\
30.22	0.000342548542916122\\
30.23	0.000342431075099504\\
30.24	0.000342313606148462\\
30.25	0.000342196136501589\\
30.26	0.00034207866656623\\
30.27	0.000341961196715855\\
30.28	0.000341843727287296\\
30.29	0.000341726258577861\\
30.3	0.00034160879084226\\
30.31	0.000341491324289424\\
30.32	0.000341373859079138\\
30.33	0.000341256395318498\\
30.34	0.000341138933058226\\
30.35	0.000341021472288748\\
30.36	0.00034090401293616\\
30.37	0.000340786554857919\\
30.38	0.000340669097838366\\
30.39	0.000340551641584032\\
30.4	0.000340434185718719\\
30.41	0.000340316729778344\\
30.42	0.00034019927320552\\
30.43	0.00034008181534393\\
30.44	0.000339964355432397\\
30.45	0.00033984689259868\\
30.46	0.000339729425853015\\
30.47	0.000339611954081335\\
30.48	0.000339494476038168\\
30.49	0.000339376990339237\\
30.5	0.000339259495453704\\
30.51	0.000339141989696081\\
30.52	0.000339024471217772\\
30.53	0.000338906937998212\\
30.54	0.000338789387835645\\
30.55	0.000338671818337471\\
30.56	0.000338554226910176\\
30.57	0.000338436610748805\\
30.58	0.000338318966825988\\
30.59	0.000338201291880474\\
30.6	0.000338083582405169\\
30.61	0.000337965834634647\\
30.62	0.000337848044532138\\
30.63	0.000337730209867783\\
30.64	0.000337612330622485\\
30.65	0.000337494406777146\\
30.66	0.000337376438312654\\
30.67	0.000337258425209893\\
30.68	0.000337140367449728\\
30.69	0.000337022265013023\\
30.7	0.000336904117880624\\
30.71	0.000336785926033367\\
30.72	0.000336667689452086\\
30.73	0.000336549408117597\\
30.74	0.000336431082010711\\
30.75	0.00033631271111223\\
30.76	0.00033619429540294\\
30.77	0.000336075834863621\\
30.78	0.000335957329475044\\
30.79	0.000335838779217968\\
30.8	0.000335720184073136\\
30.81	0.000335601544021298\\
30.82	0.000335482859043171\\
30.83	0.00033536412911948\\
30.84	0.000335245354230936\\
30.85	0.000335126534358233\\
30.86	0.000335007669482067\\
30.87	0.000334888759583105\\
30.88	0.000334769804642027\\
30.89	0.000334650804639484\\
30.9	0.000334531759556125\\
30.91	0.000334412669372593\\
30.92	0.000334293534069508\\
30.93	0.000334174353627497\\
30.94	0.000334055128027162\\
30.95	0.000333935857249102\\
30.96	0.0003338165412739\\
30.97	0.000333697180082139\\
30.98	0.000333577773654385\\
30.99	0.000333458321971195\\
31	0.000333338825013116\\
31.01	0.000333219282760679\\
31.02	0.000333099695194414\\
31.03	0.000332980062294843\\
31.04	0.000332860384042465\\
31.05	0.000332740660417774\\
31.06	0.000332620891401261\\
31.07	0.000332501076973397\\
31.08	0.000332381217114651\\
31.09	0.000332261311805471\\
31.1	0.000332141361026309\\
31.11	0.000332021364757594\\
31.12	0.000331901322979757\\
31.13	0.000331781235673205\\
31.14	0.000331661102818339\\
31.15	0.00033154092439556\\
31.16	0.000331420700385247\\
31.17	0.000331300430767768\\
31.18	0.000331180115523496\\
31.19	0.000331059754632771\\
31.2	0.000330939348075942\\
31.21	0.000330818895833336\\
31.22	0.000330698397885279\\
31.23	0.000330577854212076\\
31.24	0.000330457264794033\\
31.25	0.000330336629611436\\
31.26	0.000330215948644566\\
31.27	0.000330095221873689\\
31.28	0.000329974449279066\\
31.29	0.000329853630840947\\
31.3	0.000329732766539569\\
31.31	0.000329611856355157\\
31.32	0.000329490900267928\\
31.33	0.000329369898258094\\
31.34	0.000329248850305849\\
31.35	0.000329127756391377\\
31.36	0.000329006616494857\\
31.37	0.000328885430596447\\
31.38	0.000328764198676307\\
31.39	0.000328642920714585\\
31.4	0.000328521596691405\\
31.41	0.000328400226586898\\
31.42	0.000328278810381175\\
31.43	0.000328157348054336\\
31.44	0.000328035839586474\\
31.45	0.000327914284957671\\
31.46	0.000327792684147997\\
31.47	0.000327671037137513\\
31.48	0.000327549343906266\\
31.49	0.000327427604434301\\
31.5	0.000327305818701646\\
31.51	0.000327183986688309\\
31.52	0.000327062108374311\\
31.53	0.000326940183739641\\
31.54	0.000326818212764287\\
31.55	0.000326696195428229\\
31.56	0.00032657413171143\\
31.57	0.000326452021593839\\
31.58	0.000326329865055409\\
31.59	0.000326207662076071\\
31.6	0.000326085412635742\\
31.61	0.000325963116714343\\
31.62	0.000325840774291774\\
31.63	0.000325718385347923\\
31.64	0.000325595949862673\\
31.65	0.000325473467815895\\
31.66	0.000325350939187445\\
31.67	0.000325228363957174\\
31.68	0.00032510574210492\\
31.69	0.000324983073610507\\
31.7	0.000324860358453754\\
31.71	0.000324737596614468\\
31.72	0.000324614788072445\\
31.73	0.000324491932807466\\
31.74	0.000324369030799309\\
31.75	0.000324246082027741\\
31.76	0.000324123086472504\\
31.77	0.000324000044113343\\
31.78	0.000323876954929993\\
31.79	0.000323753818902175\\
31.8	0.000323630636009592\\
31.81	0.00032350740623195\\
31.82	0.000323384129548935\\
31.83	0.00032326080594022\\
31.84	0.000323137435385477\\
31.85	0.000323014017864358\\
31.86	0.000322890553356511\\
31.87	0.000322767041841568\\
31.88	0.000322643483299155\\
31.89	0.000322519877708883\\
31.9	0.000322396225050352\\
31.91	0.000322272525303161\\
31.92	0.000322148778446879\\
31.93	0.000322024984461083\\
31.94	0.000321901143325323\\
31.95	0.000321777255019157\\
31.96	0.000321653319522119\\
31.97	0.000321529336813729\\
31.98	0.000321405306873511\\
31.99	0.000321281229680958\\
32	0.000321157105215575\\
32.01	0.000321032933456837\\
32.02	0.000320908714384218\\
32.03	0.000320784447977181\\
32.04	0.000320660134215169\\
32.05	0.000320535773077625\\
32.06	0.000320411364543981\\
32.07	0.000320286908593646\\
32.08	0.000320162405206029\\
32.09	0.000320037854360527\\
32.1	0.000319913256036523\\
32.11	0.000319788610213388\\
32.12	0.000319663916870491\\
32.13	0.000319539175987172\\
32.14	0.000319414387542778\\
32.15	0.000319289551516637\\
32.16	0.00031916466788807\\
32.17	0.000319039736636375\\
32.18	0.00031891475774086\\
32.19	0.000318789731180803\\
32.2	0.000318664656935476\\
32.21	0.00031853953498415\\
32.22	0.000318414365306077\\
32.23	0.000318289147880488\\
32.24	0.000318163882686615\\
32.25	0.000318038569703683\\
32.26	0.0003179132089109\\
32.27	0.000317787800287456\\
32.28	0.000317662343812539\\
32.29	0.000317536839465327\\
32.3	0.00031741128722498\\
32.31	0.000317285687070653\\
32.32	0.000317160038981482\\
32.33	0.0003170343429366\\
32.34	0.000316908598915127\\
32.35	0.000316782806896168\\
32.36	0.000316656966858824\\
32.37	0.000316531078782175\\
32.38	0.000316405142645299\\
32.39	0.000316279158427257\\
32.4	0.0003161531261071\\
32.41	0.000316027045663872\\
32.42	0.000315900917076603\\
32.43	0.000315774740324303\\
32.44	0.000315648515385989\\
32.45	0.000315522242240648\\
32.46	0.000315395920867271\\
32.47	0.000315269551244833\\
32.48	0.00031514313335229\\
32.49	0.000315016667168599\\
32.5	0.000314890152672692\\
32.51	0.000314763589843501\\
32.52	0.000314636978659947\\
32.53	0.00031451031910093\\
32.54	0.000314383611145349\\
32.55	0.000314256854772085\\
32.56	0.000314130049960008\\
32.57	0.00031400319668798\\
32.58	0.000313876294934855\\
32.59	0.000313749344679466\\
32.6	0.000313622345900641\\
32.61	0.000313495298577191\\
32.62	0.000313368202687925\\
32.63	0.000313241058211636\\
32.64	0.000313113865127102\\
32.65	0.000312986623413097\\
32.66	0.000312859333048372\\
32.67	0.000312731994011684\\
32.68	0.00031260460628176\\
32.69	0.000312477169837328\\
32.7	0.000312349684657098\\
32.71	0.000312222150719776\\
32.72	0.000312094568004052\\
32.73	0.000311966936488597\\
32.74	0.000311839256152089\\
32.75	0.000311711526973176\\
32.76	0.000311583748930507\\
32.77	0.000311455922002709\\
32.78	0.000311328046168409\\
32.79	0.000311200121406218\\
32.8	0.000311072147694725\\
32.81	0.000310944125012529\\
32.82	0.00031081605333819\\
32.83	0.000310687932650286\\
32.84	0.00031055976292736\\
32.85	0.000310431544147963\\
32.86	0.000310303276290617\\
32.87	0.000310174959333836\\
32.88	0.000310046593256133\\
32.89	0.000309918178036\\
32.9	0.000309789713651924\\
32.91	0.000309661200082368\\
32.92	0.000309532637305798\\
32.93	0.00030940402530066\\
32.94	0.000309275364045394\\
32.95	0.000309146653518421\\
32.96	0.000309017893698155\\
32.97	0.000308889084562999\\
32.98	0.000308760226091342\\
32.99	0.000308631318261565\\
33	0.000308502361052032\\
33.01	0.000308373354441102\\
33.02	0.000308244298407111\\
33.03	0.0003081151929284\\
33.04	0.000307986037983286\\
33.05	0.000307856833550075\\
33.06	0.000307727579607063\\
33.07	0.000307598276132541\\
33.08	0.000307468923104778\\
33.09	0.000307339520502038\\
33.1	0.000307210068302569\\
33.11	0.000307080566484612\\
33.12	0.00030695101502639\\
33.13	0.000306821413906118\\
33.14	0.000306691763102\\
33.15	0.000306562062592228\\
33.16	0.000306432312354983\\
33.17	0.000306302512368427\\
33.18	0.000306172662610725\\
33.19	0.000306042763060014\\
33.2	0.000305912813694426\\
33.21	0.000305782814492088\\
33.22	0.0003056527654311\\
33.23	0.000305522666489566\\
33.24	0.000305392517645567\\
33.25	0.000305262318877174\\
33.26	0.000305132070162452\\
33.27	0.000305001771479451\\
33.28	0.000304871422806207\\
33.29	0.000304741024120746\\
33.3	0.00030461057540108\\
33.31	0.000304480076625216\\
33.32	0.00030434952777114\\
33.33	0.000304218928816829\\
33.34	0.000304088279740253\\
33.35	0.000303957580519362\\
33.36	0.000303826831132099\\
33.37	0.000303696031556396\\
33.38	0.000303565181770172\\
33.39	0.000303434281751334\\
33.4	0.000303303331477772\\
33.41	0.00030317233092737\\
33.42	0.000303041280078003\\
33.43	0.000302910178907526\\
33.44	0.000302779027393781\\
33.45	0.000302647825514613\\
33.46	0.000302516573247834\\
33.47	0.00030238527057126\\
33.48	0.00030225391746269\\
33.49	0.000302122513899907\\
33.5	0.000301991059860689\\
33.51	0.000301859555322796\\
33.52	0.000301728000263981\\
33.53	0.000301596394661976\\
33.54	0.00030146473849451\\
33.55	0.000301333031739301\\
33.56	0.000301201274374042\\
33.57	0.000301069466376433\\
33.58	0.000300937607724145\\
33.59	0.000300805698394841\\
33.6	0.00030067373836618\\
33.61	0.000300541727615806\\
33.62	0.000300409666121338\\
33.63	0.000300277553860404\\
33.64	0.000300145390810597\\
33.65	0.000300013176949521\\
33.66	0.000299880912254748\\
33.67	0.000299748596703849\\
33.68	0.000299616230274379\\
33.69	0.000299483812943882\\
33.7	0.000299351344689891\\
33.71	0.000299218825489925\\
33.72	0.00029908625532149\\
33.73	0.000298953634162078\\
33.74	0.000298820961989178\\
33.75	0.000298688238780256\\
33.76	0.000298555464512771\\
33.77	0.000298422639164166\\
33.78	0.000298289762711878\\
33.79	0.00029815683513333\\
33.8	0.00029802385640593\\
33.81	0.000297890826507066\\
33.82	0.000297757745414134\\
33.83	0.000297624613104498\\
33.84	0.00029749142955552\\
33.85	0.00029735819474455\\
33.86	0.000297224908648919\\
33.87	0.000297091571245949\\
33.88	0.000296958182512954\\
33.89	0.000296824742427227\\
33.9	0.000296691250966055\\
33.91	0.000296557708106713\\
33.92	0.00029642411382646\\
33.93	0.000296290468102546\\
33.94	0.000296156770912204\\
33.95	0.000296023022232654\\
33.96	0.000295889222041115\\
33.97	0.000295755370314779\\
33.98	0.000295621467030835\\
33.99	0.000295487512166452\\
34	0.0002953535056988\\
34.01	0.00029521944760502\\
34.02	0.000295085337862251\\
34.03	0.000294951176447613\\
34.04	0.000294816963338222\\
34.05	0.000294682698511173\\
34.06	0.000294548381943548\\
34.07	0.000294414013612432\\
34.08	0.000294279593494874\\
34.09	0.000294145121567928\\
34.1	0.00029401059780863\\
34.11	0.000293876022193999\\
34.12	0.000293741394701053\\
34.13	0.000293606715306781\\
34.14	0.000293471983988176\\
34.15	0.000293337200722204\\
34.16	0.000293202365485828\\
34.17	0.000293067478255997\\
34.18	0.000292932539009647\\
34.19	0.000292797547723694\\
34.2	0.000292662504375056\\
34.21	0.000292527408940622\\
34.22	0.00029239226139728\\
34.23	0.000292257061721903\\
34.24	0.000292121809891346\\
34.25	0.00029198650588246\\
34.26	0.000291851149672074\\
34.27	0.000291715741237012\\
34.28	0.00029158028055408\\
34.29	0.000291444767600076\\
34.3	0.000291309202351778\\
34.31	0.000291173584785955\\
34.32	0.000291037914879375\\
34.33	0.000290902192608767\\
34.34	0.000290766417950877\\
34.35	0.000290630590882412\\
34.36	0.000290494711380087\\
34.37	0.000290358779420586\\
34.38	0.000290222794980602\\
34.39	0.000290086758036788\\
34.4	0.000289950668565808\\
34.41	0.000289814526544306\\
34.42	0.000289678331948899\\
34.43	0.000289542084756213\\
34.44	0.00028940578494285\\
34.45	0.000289269432485398\\
34.46	0.000289133027360437\\
34.47	0.000288996569544528\\
34.48	0.000288860059014224\\
34.49	0.000288723495746069\\
34.5	0.000288586879716581\\
34.51	0.000288450210902283\\
34.52	0.00028831348927966\\
34.53	0.000288176714825214\\
34.54	0.00028803988751541\\
34.55	0.000287903007326711\\
34.56	0.000287766074235569\\
34.57	0.000287629088218416\\
34.58	0.000287492049251671\\
34.59	0.000287354957311749\\
34.6	0.000287217812375047\\
34.61	0.000287080614417939\\
34.62	0.000286943363416807\\
34.63	0.000286806059347998\\
34.64	0.000286668702187862\\
34.65	0.000286531291912728\\
34.66	0.000286393828498915\\
34.67	0.000286256311922724\\
34.68	0.000286118742160455\\
34.69	0.000285981119188377\\
34.7	0.000285843442982764\\
34.71	0.000285705713519864\\
34.72	0.000285567930775917\\
34.73	0.000285430094727146\\
34.74	0.000285292205349766\\
34.75	0.000285154262619979\\
34.76	0.000285016266513975\\
34.77	0.000284878217007919\\
34.78	0.000284740114077977\\
34.79	0.000284601957700296\\
34.8	0.000284463747851006\\
34.81	0.000284325484506232\\
34.82	0.000284187167642079\\
34.83	0.000284048797234648\\
34.84	0.00028391037326001\\
34.85	0.00028377189569424\\
34.86	0.000283633364513388\\
34.87	0.000283494779693498\\
34.88	0.000283356141210597\\
34.89	0.000283217449040706\\
34.9	0.000283078703159818\\
34.91	0.000282939903543928\\
34.92	0.000282801050169004\\
34.93	0.000282662143011012\\
34.94	0.000282523182045899\\
34.95	0.000282384167249603\\
34.96	0.000282245098598042\\
34.97	0.000282105976067126\\
34.98	0.000281966799632745\\
34.99	0.000281827569270789\\
35	0.000281688284957117\\
35.01	0.000281548946667593\\
35.02	0.000281409554378055\\
35.03	0.000281270108064323\\
35.04	0.000281130607702225\\
35.05	0.000280991053267553\\
35.06	0.000280851444736097\\
35.07	0.00028071178208363\\
35.08	0.000280572065285919\\
35.09	0.000280432294318698\\
35.1	0.000280292469157712\\
35.11	0.000280152589778683\\
35.12	0.000280012656157307\\
35.13	0.000279872668269286\\
35.14	0.000279732626090297\\
35.15	0.000279592529596006\\
35.16	0.000279452378762066\\
35.17	0.000279312173564117\\
35.18	0.000279171913977784\\
35.19	0.000279031599978675\\
35.2	0.000278891231542398\\
35.21	0.000278750808644528\\
35.22	0.000278610331260639\\
35.23	0.000278469799366294\\
35.24	0.000278329212937028\\
35.25	0.000278188571948381\\
35.26	0.000278047876375864\\
35.27	0.000277907126194982\\
35.28	0.000277766321381224\\
35.29	0.000277625461910064\\
35.3	0.000277484547756965\\
35.31	0.000277343578897379\\
35.32	0.000277202555306739\\
35.33	0.000277061476960461\\
35.34	0.000276920343833958\\
35.35	0.000276779155902624\\
35.36	0.000276637913141838\\
35.37	0.000276496615526962\\
35.38	0.000276355263033353\\
35.39	0.000276213855636347\\
35.4	0.000276072393311273\\
35.41	0.000275930876033437\\
35.42	0.000275789303778138\\
35.43	0.000275647676520666\\
35.44	0.000275505994236279\\
35.45	0.000275364256900244\\
35.46	0.000275222464487794\\
35.47	0.000275080616974161\\
35.48	0.000274938714334563\\
35.49	0.000274796756544194\\
35.5	0.000274654743578247\\
35.51	0.000274512675411889\\
35.52	0.000274370552020281\\
35.53	0.000274228373378571\\
35.54	0.000274086139461883\\
35.55	0.000273943850245342\\
35.56	0.000273801505704045\\
35.57	0.000273659105813088\\
35.58	0.000273516650547541\\
35.59	0.000273374139882467\\
35.6	0.000273231573792911\\
35.61	0.000273088952253912\\
35.62	0.000272946275240485\\
35.63	0.000272803542727636\\
35.64	0.000272660754690356\\
35.65	0.000272517911103621\\
35.66	0.000272375011942402\\
35.67	0.000272232057181639\\
35.68	0.000272089046796271\\
35.69	0.000271945980761223\\
35.7	0.000271802859051399\\
35.71	0.000271659681641688\\
35.72	0.000271516448506976\\
35.73	0.000271373159622123\\
35.74	0.000271229814961979\\
35.75	0.00027108641450139\\
35.76	0.000270942958215167\\
35.77	0.000270799446078124\\
35.78	0.000270655878065055\\
35.79	0.000270512254150739\\
35.8	0.00027036857430994\\
35.81	0.000270224838517415\\
35.82	0.000270081046747901\\
35.83	0.000269937198976118\\
35.84	0.000269793295176774\\
35.85	0.000269649335324565\\
35.86	0.000269505319394174\\
35.87	0.000269361247360263\\
35.88	0.000269217119197493\\
35.89	0.000269072934880491\\
35.9	0.000268928694383889\\
35.91	0.000268784397682291\\
35.92	0.000268640044750293\\
35.93	0.000268495635562482\\
35.94	0.000268351170093412\\
35.95	0.00026820664831765\\
35.96	0.000268062070209723\\
35.97	0.000267917435744155\\
35.98	0.000267772744895459\\
35.99	0.000267627997638133\\
36	0.000267483193946649\\
36.01	0.000267338333795478\\
36.02	0.00026719341715907\\
36.03	0.000267048444011861\\
36.04	0.000266903414328275\\
36.05	0.000266758328082721\\
36.06	0.000266613185249592\\
36.07	0.00026646798580327\\
36.08	0.000266322729718113\\
36.09	0.000266177416968482\\
36.1	0.000266032047528704\\
36.11	0.000265886621373104\\
36.12	0.000265741138475988\\
36.13	0.000265595598811648\\
36.14	0.000265450002354364\\
36.15	0.000265304349078399\\
36.16	0.000265158638958003\\
36.17	0.000265012871967406\\
36.18	0.000264867048080829\\
36.19	0.000264721167272483\\
36.2	0.000264575229516553\\
36.21	0.000264429234787216\\
36.22	0.000264283183058628\\
36.23	0.000264137074304945\\
36.24	0.000263990908500298\\
36.25	0.000263844685618802\\
36.26	0.000263698405634557\\
36.27	0.000263552068521656\\
36.28	0.000263405674254175\\
36.29	0.000263259222806161\\
36.3	0.000263112714151668\\
36.31	0.000262966148264727\\
36.32	0.000262819525119345\\
36.33	0.000262672844689527\\
36.34	0.000262526106949256\\
36.35	0.000262379311872505\\
36.36	0.000262232459433227\\
36.37	0.000262085549605369\\
36.38	0.000261938582362849\\
36.39	0.000261791557679582\\
36.4	0.000261644475529465\\
36.41	0.000261497335886377\\
36.42	0.000261350138724196\\
36.43	0.000261202884016756\\
36.44	0.000261055571737912\\
36.45	0.000260908201861476\\
36.46	0.000260760774361258\\
36.47	0.000260613289211051\\
36.48	0.000260465746384633\\
36.49	0.000260318145855766\\
36.5	0.000260170487598202\\
36.51	0.000260022771585672\\
36.52	0.000259874997791893\\
36.53	0.000259727166190566\\
36.54	0.000259579276755382\\
36.55	0.000259431329460012\\
36.56	0.000259283324278121\\
36.57	0.000259135261183345\\
36.58	0.000258987140149318\\
36.59	0.00025883896114965\\
36.6	0.000258690724157941\\
36.61	0.000258542429147777\\
36.62	0.000258394076092718\\
36.63	0.000258245664966329\\
36.64	0.000258097195742135\\
36.65	0.000257948668393671\\
36.66	0.000257800082894439\\
36.67	0.00025765143921793\\
36.68	0.000257502737337625\\
36.69	0.000257353977226992\\
36.7	0.000257205158859467\\
36.71	0.000257056282208495\\
36.72	0.000256907347247483\\
36.73	0.000256758353949836\\
36.74	0.000256609302288947\\
36.75	0.000256460192238182\\
36.76	0.000256311023770899\\
36.77	0.00025616179686044\\
36.78	0.000256012511480129\\
36.79	0.00025586316760328\\
36.8	0.00025571376520319\\
36.81	0.000255564304253135\\
36.82	0.000255414784726384\\
36.83	0.000255265206596186\\
36.84	0.000255115569835774\\
36.85	0.00025496587441837\\
36.86	0.000254816120317176\\
36.87	0.000254666307505383\\
36.88	0.000254516435956165\\
36.89	0.000254366505642672\\
36.9	0.000254216516538059\\
36.91	0.000254066468615446\\
36.92	0.000253916361847951\\
36.93	0.000253766196208661\\
36.94	0.000253615971670666\\
36.95	0.000253465688207031\\
36.96	0.000253315345790803\\
36.97	0.000253164944395021\\
36.98	0.000253014483992698\\
36.99	0.000252863964556848\\
37	0.000252713386060449\\
37.01	0.000252562748476483\\
37.02	0.000252412051777909\\
37.03	0.000252261295937662\\
37.04	0.000252110480928672\\
37.05	0.000251959606723851\\
37.06	0.000251808673296092\\
37.07	0.000251657680618279\\
37.08	0.000251506628663276\\
37.09	0.000251355517403927\\
37.1	0.000251204346813073\\
37.11	0.000251053116863529\\
37.12	0.000250901827528096\\
37.13	0.000250750478779566\\
37.14	0.000250599070590704\\
37.15	0.000250447602934268\\
37.16	0.000250296075782998\\
37.17	0.000250144489109622\\
37.18	0.000249992842886841\\
37.19	0.000249841137087351\\
37.2	0.00024968937168383\\
37.21	0.000249537546648942\\
37.22	0.000249385661955332\\
37.23	0.000249233717575631\\
37.24	0.000249081713482447\\
37.25	0.000248929649648386\\
37.26	0.000248777526046028\\
37.27	0.000248625342647938\\
37.28	0.000248473099426673\\
37.29	0.000248320796354767\\
37.3	0.000248168433404734\\
37.31	0.000248016010549086\\
37.32	0.000247863527760303\\
37.33	0.000247710985010868\\
37.34	0.000247558382273229\\
37.35	0.00024740571951983\\
37.36	0.000247252996723095\\
37.37	0.000247100213855435\\
37.38	0.000246947370889237\\
37.39	0.000246794467796888\\
37.4	0.000246641504550739\\
37.41	0.000246488481123144\\
37.42	0.000246335397486426\\
37.43	0.000246182253612901\\
37.44	0.00024602904947487\\
37.45	0.000245875785044612\\
37.46	0.000245722460294388\\
37.47	0.000245569075196454\\
37.48	0.000245415629723038\\
37.49	0.000245262123846367\\
37.5	0.000245108557538632\\
37.51	0.000244954930772022\\
37.52	0.00024480124351871\\
37.53	0.000244647495750844\\
37.54	0.000244493687440566\\
37.55	0.000244339818559999\\
37.56	0.000244185889081242\\
37.57	0.000244031898976389\\
37.58	0.000243877848217507\\
37.59	0.00024372373677666\\
37.6	0.000243569564625882\\
37.61	0.000243415331737206\\
37.62	0.000243261038082634\\
37.63	0.000243106683634156\\
37.64	0.000242952268363756\\
37.65	0.00024279779224339\\
37.66	0.000242643255245001\\
37.67	0.000242488657340516\\
37.68	0.000242333998501846\\
37.69	0.000242179278700893\\
37.7	0.000242024497909525\\
37.71	0.000241869656099611\\
37.72	0.000241714753242996\\
37.73	0.000241559789311511\\
37.74	0.000241404764276967\\
37.75	0.000241249678111158\\
37.76	0.000241094530785876\\
37.77	0.000240939322272878\\
37.78	0.000240784052543911\\
37.79	0.000240628721570714\\
37.8	0.000240473329324994\\
37.81	0.000240317875778459\\
37.82	0.000240162360902784\\
37.83	0.000240006784669641\\
37.84	0.000239851147050676\\
37.85	0.000239695448017526\\
37.86	0.000239539687541807\\
37.87	0.000239383865595119\\
37.88	0.000239227982149048\\
37.89	0.00023907203717516\\
37.9	0.000238916030645006\\
37.91	0.000238759962530128\\
37.92	0.000238603832802031\\
37.93	0.000238447641432228\\
37.94	0.000238291388392202\\
37.95	0.000238135073653417\\
37.96	0.000237978697187334\\
37.97	0.000237822258965379\\
37.98	0.000237665758958975\\
37.99	0.000237509197139524\\
38	0.000237352573478419\\
38.01	0.000237195887947015\\
38.02	0.000237039140516681\\
38.03	0.000236882331158745\\
38.04	0.000236725459844522\\
38.05	0.000236568526545326\\
38.06	0.000236411531232432\\
38.07	0.000236254473877118\\
38.08	0.00023609735445063\\
38.09	0.000235940172924207\\
38.1	0.000235782929269074\\
38.11	0.000235625623456423\\
38.12	0.000235468255457449\\
38.13	0.000235310825243317\\
38.14	0.000235153332785183\\
38.15	0.000234995778054175\\
38.16	0.000234838161021423\\
38.17	0.00023468048165802\\
38.18	0.000234522739935057\\
38.19	0.000234364935823602\\
38.2	0.000234207069294704\\
38.21	0.000234049140319397\\
38.22	0.000233891148868704\\
38.23	0.00023373309491362\\
38.24	0.000233574978425136\\
38.25	0.00023341679937422\\
38.26	0.000233258557731816\\
38.27	0.000233100253468864\\
38.28	0.000232941886556277\\
38.29	0.000232783456964956\\
38.3	0.000232624964665785\\
38.31	0.000232466409629629\\
38.32	0.000232307791827339\\
38.33	0.000232149111229743\\
38.34	0.000231990367807666\\
38.35	0.000231831561531894\\
38.36	0.000231672692373217\\
38.37	0.000231513760302397\\
38.38	0.00023135476529018\\
38.39	0.0002311957073073\\
38.4	0.000231036586324463\\
38.41	0.00023087740231237\\
38.42	0.000230718155241701\\
38.43	0.000230558845083119\\
38.44	0.000230399471807263\\
38.45	0.000230240035384766\\
38.46	0.000230080535786239\\
38.47	0.000229920972982271\\
38.48	0.000229761346943446\\
38.49	0.000229601657640317\\
38.5	0.000229441905043431\\
38.51	0.00022928208912331\\
38.52	0.000229122209850463\\
38.53	0.000228962267195383\\
38.54	0.000228802261128539\\
38.55	0.000228642191620391\\
38.56	0.000228482058641382\\
38.57	0.000228321862161924\\
38.58	0.000228161602152435\\
38.59	0.000228001278583289\\
38.6	0.000227840891424864\\
38.61	0.000227680440647512\\
38.62	0.000227519926221574\\
38.63	0.000227359348117359\\
38.64	0.000227198706305175\\
38.65	0.000227038000755303\\
38.66	0.000226877231438013\\
38.67	0.000226716398323551\\
38.68	0.000226555501382153\\
38.69	0.00022639454058403\\
38.7	0.000226233515899382\\
38.71	0.000226072427298385\\
38.72	0.000225911274751207\\
38.73	0.000225750058227992\\
38.74	0.000225588777698866\\
38.75	0.000225427433133939\\
38.76	0.000225266024503307\\
38.77	0.00022510455177705\\
38.78	0.000224943014925214\\
38.79	0.000224781413917853\\
38.8	0.000224619748724985\\
38.81	0.000224458019316615\\
38.82	0.000224296225662734\\
38.83	0.000224134367733312\\
38.84	0.000223972445498299\\
38.85	0.00022381045892764\\
38.86	0.000223648407991243\\
38.87	0.000223486292659017\\
38.88	0.000223324112900845\\
38.89	0.000223161868686594\\
38.9	0.000222999559986108\\
38.91	0.000222837186769223\\
38.92	0.000222674749005747\\
38.93	0.000222512246665479\\
38.94	0.000222349679718202\\
38.95	0.000222187048133669\\
38.96	0.000222024351881627\\
38.97	0.000221861590931806\\
38.98	0.000221698765253903\\
38.99	0.00022153587481762\\
39	0.000221372919592623\\
39.01	0.000221209899548568\\
39.02	0.000221046814655096\\
39.03	0.000220883664881825\\
39.04	0.000220720450198353\\
39.05	0.000220557170574273\\
39.06	0.000220393825979149\\
39.07	0.000220230416382524\\
39.08	0.000220066941753938\\
39.09	0.000219903402062901\\
39.1	0.00021973979727891\\
39.11	0.000219576127371441\\
39.12	0.000219412392309956\\
39.13	0.000219248592063898\\
39.14	0.000219084726602699\\
39.15	0.000218920795895755\\
39.16	0.000218756799912463\\
39.17	0.000218592738622196\\
39.18	0.000218428611994302\\
39.19	0.000218264419998121\\
39.2	0.000218100162602975\\
39.21	0.000217935839778163\\
39.22	0.000217771451492961\\
39.23	0.000217606997716642\\
39.24	0.000217442478418454\\
39.25	0.000217277893567621\\
39.26	0.00021711324313336\\
39.27	0.000216948527084862\\
39.28	0.000216783745391301\\
39.29	0.000216618898021844\\
39.3	0.000216453984945625\\
39.31	0.000216289006131767\\
39.32	0.000216123961549373\\
39.33	0.000215958851167529\\
39.34	0.000215793674955311\\
39.35	0.000215628432881764\\
39.36	0.000215463124915922\\
39.37	0.000215297751026805\\
39.38	0.000215132311183402\\
39.39	0.000214966805354693\\
39.4	0.00021480123350965\\
39.41	0.000214635595617204\\
39.42	0.000214469891646285\\
39.43	0.000214304121565805\\
39.44	0.000214138285344645\\
39.45	0.000213972382951684\\
39.46	0.000213806414355771\\
39.47	0.000213640379525743\\
39.48	0.000213474278430419\\
39.49	0.000213308111038599\\
39.5	0.000213141877319061\\
39.51	0.000212975577240574\\
39.52	0.000212809210771877\\
39.53	0.000212642777881706\\
39.54	0.000212476278538761\\
39.55	0.000212309712711736\\
39.56	0.000212143080369308\\
39.57	0.00021197638148013\\
39.58	0.000211809616012847\\
39.59	0.000211642783936067\\
39.6	0.000211475885218393\\
39.61	0.000211308919828415\\
39.62	0.000211141887734693\\
39.63	0.000210974788905773\\
39.64	0.00021080762331019\\
39.65	0.000210640390916447\\
39.66	0.000210473091693041\\
39.67	0.000210305725608446\\
39.68	0.000210138292631118\\
39.69	0.000209970792729493\\
39.7	0.000209803225872\\
39.71	0.00020963559202703\\
39.72	0.000209467891162973\\
39.73	0.000209300123248192\\
39.74	0.000209132288251037\\
39.75	0.000208964386139833\\
39.76	0.000208796416882899\\
39.77	0.000208628380448521\\
39.78	0.000208460276804975\\
39.79	0.000208292105920519\\
39.8	0.000208123867763393\\
39.81	0.000207955562301817\\
39.82	0.000207787189503986\\
39.83	0.000207618749338091\\
39.84	0.000207450241772296\\
39.85	0.000207281666774748\\
39.86	0.000207113024313574\\
39.87	0.000206944314356888\\
39.88	0.000206775536872778\\
39.89	0.000206606691829321\\
39.9	0.000206437779194575\\
39.91	0.000206268798936572\\
39.92	0.000206099751023339\\
39.93	0.000205930635422867\\
39.94	0.000205761452103147\\
39.95	0.000205592201032142\\
39.96	0.000205422882177795\\
39.97	0.000205253495508036\\
39.98	0.000205084040990776\\
39.99	0.000204914518593903\\
40	0.000204744928285292\\
40.01	0.000204575270032792\\
};
\addplot [color=green,solid,forget plot]
  table[row sep=crcr]{%
40.01	0.000204575270032792\\
40.02	0.000204405543804244\\
40.03	0.000204235749567462\\
40.04	0.00020406588729025\\
40.05	0.000203895956940382\\
40.06	0.000203725958485624\\
40.07	0.00020355589189372\\
40.08	0.000203385757132392\\
40.09	0.000203215554169354\\
40.1	0.000203045282972285\\
40.11	0.000202874943508862\\
40.12	0.000202704535746731\\
40.13	0.000202534059653528\\
40.14	0.000202363515196867\\
40.15	0.000202192902344342\\
40.16	0.00020202222106353\\
40.17	0.000201851471321991\\
40.18	0.000201680653087266\\
40.19	0.000201509766326876\\
40.2	0.00020133881100832\\
40.21	0.000201167787099082\\
40.22	0.000200996694566635\\
40.23	0.000200825533378418\\
40.24	0.000200654303501859\\
40.25	0.000200483004904373\\
40.26	0.000200311637553347\\
40.27	0.000200140201416153\\
40.28	0.000199968696460145\\
40.29	0.000199797122652653\\
40.3	0.000199625479960995\\
40.31	0.000199453768352469\\
40.32	0.000199281987794356\\
40.33	0.000199110138253909\\
40.34	0.00019893821969837\\
40.35	0.000198766232094961\\
40.36	0.000198594175410877\\
40.37	0.00019842204961331\\
40.38	0.000198249854669425\\
40.39	0.000198077590546358\\
40.4	0.00019790525721124\\
40.41	0.000197732854631177\\
40.42	0.000197560382773262\\
40.43	0.000197387841604555\\
40.44	0.00019721523109211\\
40.45	0.000197042551202956\\
40.46	0.000196869801904101\\
40.47	0.000196696983162542\\
40.48	0.000196524094945251\\
40.49	0.000196351137219173\\
40.5	0.000196178109951251\\
40.51	0.000196005013108395\\
40.52	0.000195831846657498\\
40.53	0.000195658610565436\\
40.54	0.000195485304799063\\
40.55	0.000195311929325221\\
40.56	0.000195138484110719\\
40.57	0.000194964969122354\\
40.58	0.000194791384326909\\
40.59	0.00019461772969113\\
40.6	0.000194444005181763\\
40.61	0.000194270210765518\\
40.62	0.000194096346409099\\
40.63	0.000193922412079175\\
40.64	0.00019374840774241\\
40.65	0.000193574333365436\\
40.66	0.000193400188914871\\
40.67	0.00019322597435731\\
40.68	0.000193051689659333\\
40.69	0.000192877334787489\\
40.7	0.000192702909708313\\
40.71	0.000192528414388327\\
40.72	0.000192353848794014\\
40.73	0.000192179212891851\\
40.74	0.000192004506648294\\
40.75	0.000191829730029769\\
40.76	0.000191654883002686\\
40.77	0.000191479965533441\\
40.78	0.000191304977588394\\
40.79	0.000191129919133894\\
40.8	0.000190954790136269\\
40.81	0.000190779590561815\\
40.82	0.000190604320376824\\
40.83	0.000190428979547554\\
40.84	0.000190253568040236\\
40.85	0.0001900780858211\\
40.86	0.000189902532856331\\
40.87	0.00018972690911211\\
40.88	0.00018955121455458\\
40.89	0.000189375449149874\\
40.9	0.000189199612864096\\
40.91	0.000189023705663333\\
40.92	0.000188847727513644\\
40.93	0.000188671678381063\\
40.94	0.000188495558231614\\
40.95	0.000188319367031278\\
40.96	0.000188143104746036\\
40.97	0.000187966771341823\\
40.98	0.000187790366784566\\
40.99	0.000187613891040165\\
41	0.000187437344074485\\
41.01	0.000187260725853388\\
41.02	0.00018708403634269\\
41.03	0.000186907275508196\\
41.04	0.000186730443315689\\
41.05	0.000186553539730912\\
41.06	0.000186376564719598\\
41.07	0.000186199518247446\\
41.08	0.000186022400280138\\
41.09	0.00018584521078332\\
41.1	0.00018566794972262\\
41.11	0.00018549061706364\\
41.12	0.000185313212771949\\
41.13	0.000185135736813094\\
41.14	0.000184958189152602\\
41.15	0.000184780569755962\\
41.16	0.00018460287858864\\
41.17	0.000184425115616079\\
41.18	0.000184247280803688\\
41.19	0.000184069374116851\\
41.2	0.000183891395520927\\
41.21	0.000183713344981237\\
41.22	0.000183535222463086\\
41.23	0.000183357027931748\\
41.24	0.000183178761352451\\
41.25	0.00018300042269042\\
41.26	0.000182822011910824\\
41.27	0.000182643528978821\\
41.28	0.000182464973859535\\
41.29	0.000182286346518054\\
41.3	0.00018210764691943\\
41.31	0.000181928875028704\\
41.32	0.00018175003081086\\
41.33	0.00018157111423087\\
41.34	0.000181392125253659\\
41.35	0.000181213063844126\\
41.36	0.000181033929967146\\
41.37	0.00018085472358754\\
41.38	0.000180675444670108\\
41.39	0.000180496093179621\\
41.4	0.000180316669080803\\
41.41	0.000180137172338342\\
41.42	0.000179957602916902\\
41.43	0.000179777960781105\\
41.44	0.000179598245895534\\
41.45	0.000179418458224744\\
41.46	0.000179238597733234\\
41.47	0.000179058664385489\\
41.48	0.000178878658145932\\
41.49	0.00017869857897897\\
41.5	0.000178518426848955\\
41.51	0.000178338201720203\\
41.52	0.000178157903556995\\
41.53	0.000177977532323561\\
41.54	0.0001777970879841\\
41.55	0.000177616570502762\\
41.56	0.000177435979843654\\
41.57	0.000177255315970848\\
41.58	0.000177074578848366\\
41.59	0.000176893768440187\\
41.6	0.000176712884710242\\
41.61	0.000176531927622427\\
41.62	0.000176350897140579\\
41.63	0.000176169793228494\\
41.64	0.000175988615849923\\
41.65	0.000175807364968571\\
41.66	0.000175626040548084\\
41.67	0.000175444642552066\\
41.68	0.000175263170944078\\
41.69	0.000175081625687619\\
41.7	0.000174900006746137\\
41.71	0.000174718314083033\\
41.72	0.000174536547661662\\
41.73	0.000174354707445312\\
41.74	0.000174172793397221\\
41.75	0.000173990805480579\\
41.76	0.000173808743658512\\
41.77	0.000173626607894096\\
41.78	0.000173444398150346\\
41.79	0.000173262114390218\\
41.8	0.000173079756576617\\
41.81	0.000172897324672381\\
41.82	0.000172714818640292\\
41.83	0.000172532238443063\\
41.84	0.000172349584043363\\
41.85	0.000172166855403775\\
41.86	0.00017198405248684\\
41.87	0.000171801175255015\\
41.88	0.000171618223670709\\
41.89	0.000171435197696258\\
41.9	0.000171252097293932\\
41.91	0.000171068922425927\\
41.92	0.000170885673054384\\
41.93	0.000170702349141356\\
41.94	0.000170518950648848\\
41.95	0.00017033547753877\\
41.96	0.000170151929772985\\
41.97	0.000169968307313256\\
41.98	0.0001697846101213\\
41.99	0.000169600838158736\\
42	0.00016941699138712\\
42.01	0.000169233069767932\\
42.02	0.000169049073262567\\
42.03	0.000168865001832344\\
42.04	0.000168680855438507\\
42.05	0.000168496634042222\\
42.06	0.000168312337604566\\
42.07	0.000168127966086539\\
42.08	0.000167943519449053\\
42.09	0.00016775899765295\\
42.1	0.000167574400658971\\
42.11	0.000167389728427782\\
42.12	0.00016720498091996\\
42.13	0.000167020158095998\\
42.14	0.000166835259916304\\
42.15	0.000166650286341182\\
42.16	0.000166465237330857\\
42.17	0.000166280112845477\\
42.18	0.000166094912845076\\
42.19	0.000165909637289614\\
42.2	0.000165724286138945\\
42.21	0.00016553885935284\\
42.22	0.000165353356890975\\
42.23	0.00016516777871293\\
42.24	0.000164982124778192\\
42.25	0.000164796395046147\\
42.26	0.000164610589476095\\
42.27	0.000164424708027236\\
42.28	0.000164238750658657\\
42.29	0.00016405271732938\\
42.3	0.000163866607998296\\
42.31	0.000163680422624223\\
42.32	0.00016349416116586\\
42.33	0.000163307823581829\\
42.34	0.00016312140983063\\
42.35	0.000162934919870681\\
42.36	0.000162748353660298\\
42.37	0.000162561711157689\\
42.38	0.000162374992320965\\
42.39	0.000162188197108148\\
42.4	0.000162001325477151\\
42.41	0.000161814377385784\\
42.42	0.000161627352791767\\
42.43	0.000161440251652718\\
42.44	0.00016125307392616\\
42.45	0.000161065819569508\\
42.46	0.000160878488540092\\
42.47	0.000160691080795135\\
42.48	0.000160503596291774\\
42.49	0.000160316034987042\\
42.5	0.000160128396837877\\
42.51	0.000159940681801139\\
42.52	0.000159752889833569\\
42.53	0.000159565020891839\\
42.54	0.000159377074932524\\
42.55	0.000159189051912109\\
42.56	0.000159000951786989\\
42.57	0.000158812774513481\\
42.58	0.000158624520047813\\
42.59	0.000158436188346133\\
42.6	0.000158247779364509\\
42.61	0.000158059293058925\\
42.62	0.000157870729385301\\
42.63	0.000157682088299475\\
42.64	0.000157493369757209\\
42.65	0.000157304573714205\\
42.66	0.000157115700126106\\
42.67	0.000156926748948476\\
42.68	0.000156737720136826\\
42.69	0.000156548613646609\\
42.7	0.000156359429433236\\
42.71	0.000156170167452055\\
42.72	0.000155980827658367\\
42.73	0.000155791410007438\\
42.74	0.000155601914454491\\
42.75	0.000155412340954718\\
42.76	0.00015522268946328\\
42.77	0.000155032959935311\\
42.78	0.000154843152325923\\
42.79	0.000154653266590209\\
42.8	0.000154463302683256\\
42.81	0.000154273260560146\\
42.82	0.000154083140175948\\
42.83	0.00015389294148575\\
42.84	0.000153702664444632\\
42.85	0.000153512309007708\\
42.86	0.000153321875130098\\
42.87	0.000153131362766949\\
42.88	0.000152940771873448\\
42.89	0.000152750102404819\\
42.9	0.000152559354316326\\
42.91	0.00015236852756329\\
42.92	0.000152177622101082\\
42.93	0.000151986637885144\\
42.94	0.00015179557487098\\
42.95	0.000151604433014181\\
42.96	0.000151413212270416\\
42.97	0.000151221912595438\\
42.98	0.000151030533945112\\
42.99	0.000150839076275396\\
43	0.000150647539542352\\
43.01	0.000150455923702175\\
43.02	0.000150264228711165\\
43.03	0.000150072454525767\\
43.04	0.000149880601102546\\
43.05	0.000149688668398222\\
43.06	0.000149496656369656\\
43.07	0.000149304564973854\\
43.08	0.000149112394167988\\
43.09	0.000148920143909386\\
43.1	0.000148727814155546\\
43.11	0.000148535404864139\\
43.12	0.000148342915993003\\
43.13	0.00014815034750015\\
43.14	0.000147957699343778\\
43.15	0.000147764971482261\\
43.16	0.000147572163874151\\
43.17	0.000147379276478184\\
43.18	0.000147186309253268\\
43.19	0.000146993262158498\\
43.2	0.000146800135153127\\
43.21	0.000146606928196598\\
43.22	0.00014641364124849\\
43.23	0.000146220274268562\\
43.24	0.000146026827216697\\
43.25	0.000145833300052927\\
43.26	0.000145639692737405\\
43.27	0.000145446005230386\\
43.28	0.000145252237492108\\
43.29	0.0001450583894828\\
43.3	0.000144864461162639\\
43.31	0.000144670452491802\\
43.32	0.00014447636343042\\
43.33	0.000144282193938616\\
43.34	0.000144087943976474\\
43.35	0.000143893613504056\\
43.36	0.000143699202481406\\
43.37	0.000143504710868526\\
43.38	0.000143310138625412\\
43.39	0.000143115485712017\\
43.4	0.000142920752088277\\
43.41	0.000142725937714106\\
43.42	0.000142531042549382\\
43.43	0.000142336066553962\\
43.44	0.000142141009687683\\
43.45	0.000141945871910345\\
43.46	0.000141750653181734\\
43.47	0.000141555353461598\\
43.48	0.000141359972709672\\
43.49	0.000141164510885651\\
43.5	0.000140968967949214\\
43.51	0.000140773343860014\\
43.52	0.000140577638577673\\
43.53	0.000140381852061795\\
43.54	0.000140185984271946\\
43.55	0.000139990035167677\\
43.56	0.000139794004708507\\
43.57	0.00013959789285393\\
43.58	0.00013940169956342\\
43.59	0.000139205424796415\\
43.6	0.000139009068512334\\
43.61	0.000138812630670565\\
43.62	0.000138616111230476\\
43.63	0.000138419510151408\\
43.64	0.000138222827392667\\
43.65	0.000138026062913544\\
43.66	0.0001378292166733\\
43.67	0.000137632288631166\\
43.68	0.000137435278746352\\
43.69	0.00013723818697804\\
43.7	0.000137041013285389\\
43.71	0.000136843757627519\\
43.72	0.000136646419963547\\
43.73	0.00013644900025254\\
43.74	0.000136251498453553\\
43.75	0.000136053914525612\\
43.76	0.000135856248427713\\
43.77	0.000135658500118834\\
43.78	0.000135460669557916\\
43.79	0.000135262756703881\\
43.8	0.000135064761515626\\
43.81	0.00013486668395201\\
43.82	0.000134668523971884\\
43.83	0.000134470281534059\\
43.84	0.000134271956597324\\
43.85	0.000134073549120442\\
43.86	0.000133875059062148\\
43.87	0.000133676486381152\\
43.88	0.000133477831036141\\
43.89	0.000133279092985772\\
43.9	0.00013308027218867\\
43.91	0.000132881368603451\\
43.92	0.000132682382188682\\
43.93	0.000132483312902923\\
43.94	0.000132284160704698\\
43.95	0.000132084925552505\\
43.96	0.000131885607404819\\
43.97	0.000131686206220087\\
43.98	0.000131486721956731\\
43.99	0.00013128715457314\\
44	0.000131087504027686\\
44.01	0.000130887770278712\\
44.02	0.000130687953284532\\
44.03	0.000130488053003433\\
44.04	0.000130288069393686\\
44.05	0.000130088002413518\\
44.06	0.000129887852021145\\
44.07	0.000129687618174746\\
44.08	0.000129487300832483\\
44.09	0.000129286899952487\\
44.1	0.000129086415492865\\
44.11	0.00012888584741169\\
44.12	0.000128685195667014\\
44.13	0.000128484460216873\\
44.14	0.00012828364101926\\
44.15	0.000128082738032149\\
44.16	0.000127881751213494\\
44.17	0.000127680680521207\\
44.18	0.000127479525913191\\
44.19	0.000127278287347307\\
44.2	0.000127076964781408\\
44.21	0.000126875558173305\\
44.22	0.00012667406748079\\
44.23	0.000126472492661626\\
44.24	0.000126270833673552\\
44.25	0.000126069090474282\\
44.26	0.0001258672630215\\
44.27	0.000125665351272869\\
44.28	0.00012546335518602\\
44.29	0.000125261274718569\\
44.3	0.000125059109828086\\
44.31	0.000124856860472132\\
44.32	0.000124654526608241\\
44.33	0.000124452108193918\\
44.34	0.000124249605186639\\
44.35	0.000124047017543854\\
44.36	0.000123844345222997\\
44.37	0.000123641588181461\\
44.38	0.000123438746376629\\
44.39	0.000123235819765843\\
44.4	0.000123032808306436\\
44.41	0.000122829711955697\\
44.42	0.000122626530670907\\
44.43	0.000122423264409309\\
44.44	0.000122219913128123\\
44.45	0.000122016476784548\\
44.46	0.000121812955335756\\
44.47	0.000121609348738888\\
44.48	0.00012140565695107\\
44.49	0.000121201879929389\\
44.5	0.00012099801763092\\
44.51	0.000120794070012706\\
44.52	0.000120590037031769\\
44.53	0.0001203859186451\\
44.54	0.000120181714809665\\
44.55	0.000119977425482413\\
44.56	0.000119773050620262\\
44.57	0.000119568590180101\\
44.58	0.00011936404411881\\
44.59	0.000119159412393223\\
44.6	0.000118954694960167\\
44.61	0.000118749891776434\\
44.62	0.000118545002798794\\
44.63	0.000118340027983995\\
44.64	0.00011813496728876\\
44.65	0.000117929820669785\\
44.66	0.000117724588083748\\
44.67	0.000117519269487294\\
44.68	0.000117313864837048\\
44.69	0.000117108374089611\\
44.7	0.000116902797201566\\
44.71	0.000116697134129465\\
44.72	0.000116491384829833\\
44.73	0.00011628554925918\\
44.74	0.000116079627373988\\
44.75	0.000115873619130716\\
44.76	0.000115667524485807\\
44.77	0.000115461343395665\\
44.78	0.000115255075816688\\
44.79	0.000115048721705237\\
44.8	0.000114842281017656\\
44.81	0.000114635753710271\\
44.82	0.000114429139739382\\
44.83	0.000114222439061264\\
44.84	0.000114015651632167\\
44.85	0.000113808777408332\\
44.86	0.000113601816345961\\
44.87	0.000113394768401245\\
44.88	0.000113187633530354\\
44.89	0.000112980411689428\\
44.9	0.000112773102834592\\
44.91	0.000112565706921948\\
44.92	0.00011235822390758\\
44.93	0.000112150653747546\\
44.94	0.000111942996397885\\
44.95	0.000111735251814614\\
44.96	0.000111527419953737\\
44.97	0.000111319500771229\\
44.98	0.000111111494223047\\
44.99	0.000110903400265126\\
45	0.000110695218853389\\
45.01	0.000110486949943736\\
45.02	0.000110278593492042\\
45.03	0.00011007014945417\\
45.04	0.000109861617785966\\
45.05	0.000109652998443246\\
45.06	0.000109444291381822\\
45.07	0.00010923549655747\\
45.08	0.000109026613925972\\
45.09	0.000108817643443072\\
45.1	0.000108608585064505\\
45.11	0.000108399438745984\\
45.12	0.000108190204443219\\
45.13	0.000107980882111879\\
45.14	0.000107771471707641\\
45.15	0.000107561973186156\\
45.16	0.000107352386503052\\
45.17	0.000107142711613956\\
45.18	0.000106932948474466\\
45.19	0.000106723097040176\\
45.2	0.000106513157266661\\
45.21	0.000106303129109486\\
45.22	0.000106093012524188\\
45.23	0.000105882807466311\\
45.24	0.000105672513891366\\
45.25	0.00010546213175487\\
45.26	0.000105251661012315\\
45.27	0.000105041101619182\\
45.28	0.000104830453530949\\
45.29	0.000104619716703074\\
45.3	0.000104408891091003\\
45.31	0.000104197976650181\\
45.32	0.000103986973336034\\
45.33	0.000103775881103985\\
45.34	0.000103564699909445\\
45.35	0.000103353429707818\\
45.36	0.000103142070454495\\
45.37	0.00010293062210486\\
45.38	0.000102719084614299\\
45.39	0.000102507457938186\\
45.4	0.000102295742031884\\
45.41	0.000102083936850758\\
45.42	0.000101872042350161\\
45.43	0.000101660058485449\\
45.44	0.000101447985211967\\
45.45	0.000101235822485061\\
45.46	0.00010102357026007\\
45.47	0.000100811228492341\\
45.48	0.000100598797137209\\
45.49	0.00010038627615001\\
45.5	0.00010017366548609\\
45.51	9.99609651007819e-05\\
45.52	9.97481749494197e-05\\
45.53	9.9535294987357e-05\\
45.54	9.93223251699316e-05\\
45.55	9.91092654524899e-05\\
45.56	9.88961157903836e-05\\
45.57	9.86828761389748e-05\\
45.58	9.84695464536223e-05\\
45.59	9.82561266896984e-05\\
45.6	9.80426168025739e-05\\
45.61	9.78290167476421e-05\\
45.62	9.76153264802913e-05\\
45.63	9.74015459559285e-05\\
45.64	9.71876751299661e-05\\
45.65	9.6973713957834e-05\\
45.66	9.67596623949723e-05\\
45.67	9.65455203968315e-05\\
45.68	9.63312879188848e-05\\
45.69	9.61169649166139e-05\\
45.7	9.59025513455249e-05\\
45.71	9.56880471611344e-05\\
45.72	9.54734523189778e-05\\
45.73	9.52587667746169e-05\\
45.74	9.50439904836373e-05\\
45.75	9.4829123401639e-05\\
45.76	9.46141654842494e-05\\
45.77	9.43991166871185e-05\\
45.78	9.41839769659295e-05\\
45.79	9.39687462763895e-05\\
45.8	9.37534245742284e-05\\
45.81	9.35380118152161e-05\\
45.82	9.33225079551535e-05\\
45.83	9.31069129498693e-05\\
45.84	9.28912267552286e-05\\
45.85	9.26754493271349e-05\\
45.86	9.24595806215296e-05\\
45.87	9.22436205943924e-05\\
45.88	9.20275692017428e-05\\
45.89	9.18114263996454e-05\\
45.9	9.15951921442067e-05\\
45.91	9.13788663915813e-05\\
45.92	9.11624490979729e-05\\
45.93	9.09459402196316e-05\\
45.94	9.07293397128632e-05\\
45.95	9.05126475340221e-05\\
45.96	9.02958636395234e-05\\
45.97	9.00789879858307e-05\\
45.98	8.98620205294822e-05\\
45.99	8.96449612270665e-05\\
46	8.94278100352346e-05\\
46.01	8.92105669107138e-05\\
46.02	8.89932318102887e-05\\
46.03	8.87758046908202e-05\\
46.04	8.85582855092437e-05\\
46.05	8.83406742225659e-05\\
46.06	8.81229707878749e-05\\
46.07	8.79051751623387e-05\\
46.08	8.76872873032033e-05\\
46.09	8.74693071678066e-05\\
46.1	8.72512347135732e-05\\
46.11	8.70330698980182e-05\\
46.12	8.68148126787552e-05\\
46.13	8.65964630134795e-05\\
46.14	8.63780208600114e-05\\
46.15	8.61594861762491e-05\\
46.16	8.59408589202158e-05\\
46.17	8.5722139050037e-05\\
46.18	8.55033265239457e-05\\
46.19	8.52844213002947e-05\\
46.2	8.50654233375582e-05\\
46.21	8.48463325943283e-05\\
46.22	8.46271490293326e-05\\
46.23	8.44078726014129e-05\\
46.24	8.41885032695604e-05\\
46.25	8.39690409928963e-05\\
46.26	8.37494857306789e-05\\
46.27	8.35298374423209e-05\\
46.28	8.33100960873773e-05\\
46.29	8.30902616255659e-05\\
46.3	8.28703340167471e-05\\
46.31	8.26503132209614e-05\\
46.32	8.24301991983953e-05\\
46.33	8.22099919094328e-05\\
46.34	8.19896913146143e-05\\
46.35	8.17692973746655e-05\\
46.36	8.15488100505048e-05\\
46.37	8.13282293032295e-05\\
46.38	8.11075550941481e-05\\
46.39	8.08867873847603e-05\\
46.4	8.06659261367738e-05\\
46.41	8.04449713121148e-05\\
46.42	8.02239228729282e-05\\
46.43	8.00027807815722e-05\\
46.44	7.97815450006498e-05\\
46.45	7.95602154929909e-05\\
46.46	7.93387922216719e-05\\
46.47	7.91172751500209e-05\\
46.48	7.8895664241619e-05\\
46.49	7.86739594603077e-05\\
46.5	7.84521607702059e-05\\
46.51	7.82302681357049e-05\\
46.52	7.80082815214789e-05\\
46.53	7.77862008924986e-05\\
46.54	7.75640262140331e-05\\
46.55	7.73417574516499e-05\\
46.56	7.71193945712512e-05\\
46.57	7.68969375390446e-05\\
46.58	7.6674386321576e-05\\
46.59	7.64517408857381e-05\\
46.6	7.62290011987708e-05\\
46.61	7.60061672282607e-05\\
46.62	7.57832389421849e-05\\
46.63	7.55602163088759e-05\\
46.64	7.53370992970723e-05\\
46.65	7.51138878758995e-05\\
46.66	7.48905820148939e-05\\
46.67	7.46671816840155e-05\\
46.68	7.44436868536473e-05\\
46.69	7.42200974946094e-05\\
46.7	7.39964135781888e-05\\
46.71	7.37726350761199e-05\\
46.72	7.35487619606175e-05\\
46.73	7.33247942043963e-05\\
46.74	7.31007317806547e-05\\
46.75	7.28765746631083e-05\\
46.76	7.2652322826005e-05\\
46.77	7.24279762441323e-05\\
46.78	7.22035348928278e-05\\
46.79	7.19789987479926e-05\\
46.8	7.17543677861178e-05\\
46.81	7.15296419842911e-05\\
46.82	7.13048213202144e-05\\
46.83	7.10799057722124e-05\\
46.84	7.08548953192584e-05\\
46.85	7.0629789940985e-05\\
46.86	7.04045896177067e-05\\
46.87	7.01792943304247e-05\\
46.88	6.99539040608637e-05\\
46.89	6.97284187914718e-05\\
46.9	6.95028385054551e-05\\
46.91	6.92771631867779e-05\\
46.92	6.90513928202005e-05\\
46.93	6.88255273912936e-05\\
46.94	6.859956688645e-05\\
46.95	6.83735112929264e-05\\
46.96	6.81473605988348e-05\\
46.97	6.79211147931943e-05\\
46.98	6.76947738659315e-05\\
46.99	6.7468337807932e-05\\
47	6.72418066110271e-05\\
47.01	6.70151802680437e-05\\
47.02	6.67884587728321e-05\\
47.03	6.65616421202681e-05\\
47.04	6.6334730306308e-05\\
47.05	6.61077233279957e-05\\
47.06	6.58806211834995e-05\\
47.07	6.56534238721411e-05\\
47.08	6.54261313944183e-05\\
47.09	6.51987437520415e-05\\
47.1	6.49712609479668e-05\\
47.11	6.47436829864147e-05\\
47.12	6.4516009872919e-05\\
47.13	6.42882416143544e-05\\
47.14	6.40603782189608e-05\\
47.15	6.38324196963903e-05\\
47.16	6.36043660577467e-05\\
47.17	6.33762173156034e-05\\
47.18	6.31479734840654e-05\\
47.19	6.29196345787904e-05\\
47.2	6.26912006170301e-05\\
47.21	6.24626716176843e-05\\
47.22	6.22340476013252e-05\\
47.23	6.20053285902613e-05\\
47.24	6.17765146085549e-05\\
47.25	6.15476056820968e-05\\
47.26	6.13186018386305e-05\\
47.27	6.10895031078144e-05\\
47.28	6.08603095212552e-05\\
47.29	6.06310211125703e-05\\
47.3	6.04016379174444e-05\\
47.31	6.01721599736704e-05\\
47.32	5.99425873212024e-05\\
47.33	5.97129200022271e-05\\
47.34	5.94831580612021e-05\\
47.35	5.92533015449301e-05\\
47.36	5.90233505026114e-05\\
47.37	5.8793304985906e-05\\
47.38	5.85631650490016e-05\\
47.39	5.83329307486738e-05\\
47.4	5.81026021443509e-05\\
47.41	5.78721792982e-05\\
47.42	5.76416622751642e-05\\
47.43	5.74110511430796e-05\\
47.44	5.71803459727023e-05\\
47.45	5.69495468378169e-05\\
47.46	5.67186538153151e-05\\
47.47	5.64876669852526e-05\\
47.48	5.62565864309465e-05\\
47.49	5.60254122390687e-05\\
47.5	5.57941444997209e-05\\
47.51	5.55627833065172e-05\\
47.52	5.53313287566977e-05\\
47.53	5.50997809511938e-05\\
47.54	5.4868139994755e-05\\
47.55	5.46364059960322e-05\\
47.56	5.44045790676802e-05\\
47.57	5.41726593264649e-05\\
47.58	5.39406468933572e-05\\
47.59	5.37085418936736e-05\\
47.6	5.34763444571627e-05\\
47.61	5.3244054718132e-05\\
47.62	5.30116728155641e-05\\
47.63	5.27791988932449e-05\\
47.64	5.25466330998821e-05\\
47.65	5.23139755892448e-05\\
47.66	5.20812265202789e-05\\
47.67	5.18483860572662e-05\\
47.68	5.16154543699458e-05\\
47.69	5.13824316336689e-05\\
47.7	5.11493180295388e-05\\
47.71	5.09161137445743e-05\\
47.72	5.06828189718537e-05\\
47.73	5.04494339106792e-05\\
47.74	5.02159587667493e-05\\
47.75	4.99823937523194e-05\\
47.76	4.97487390863638e-05\\
47.77	4.95149949947815e-05\\
47.78	4.92811617105668e-05\\
47.79	4.90472394739858e-05\\
47.8	4.88132285327934e-05\\
47.81	4.85791291424068e-05\\
47.82	4.83449415661294e-05\\
47.83	4.81106660753568e-05\\
47.84	4.78763029497786e-05\\
47.85	4.76418524776225e-05\\
47.86	4.74073149558593e-05\\
47.87	4.71726906904561e-05\\
47.88	4.69379799965981e-05\\
47.89	4.67031831989581e-05\\
47.9	4.64683006319317e-05\\
47.91	4.6233332639898e-05\\
47.92	4.59982795775037e-05\\
47.93	4.57631418099239e-05\\
47.94	4.55279197131495e-05\\
47.95	4.52926136742804e-05\\
47.96	4.50572240918226e-05\\
47.97	4.48217513759862e-05\\
47.98	4.45861959490288e-05\\
47.99	4.43505582455456e-05\\
48	4.41148387128228e-05\\
48.01	4.38790378111831e-05\\
48.02	4.36431560143294e-05\\
48.03	4.34071938097032e-05\\
48.04	4.31711516988668e-05\\
48.05	4.29350301978902e-05\\
48.06	4.26988298377288e-05\\
48.07	4.24625511646383e-05\\
48.08	4.22261947405911e-05\\
48.09	4.19897611436994e-05\\
48.1	4.17532509686628e-05\\
48.11	4.15166648272073e-05\\
48.12	4.12800033485605e-05\\
48.13	4.10432671799268e-05\\
48.14	4.08064569869821e-05\\
48.15	4.05695734543663e-05\\
48.16	4.03326172862192e-05\\
48.17	4.00955892067012e-05\\
48.18	3.98584899605516e-05\\
48.19	3.9621320313649e-05\\
48.2	3.93840810535926e-05\\
48.21	3.91467729902971e-05\\
48.22	3.89093969566205e-05\\
48.23	3.86719538089681e-05\\
48.24	3.84344444279653e-05\\
48.25	3.81968697191221e-05\\
48.26	3.79592306135133e-05\\
48.27	3.77215280684892e-05\\
48.28	3.74837630683997e-05\\
48.29	3.72459366253467e-05\\
48.3	3.70080497799543e-05\\
48.31	3.6770103602141e-05\\
48.32	3.65320991919625e-05\\
48.33	3.6294037680424e-05\\
48.34	3.60559202303608e-05\\
48.35	3.58177480372975e-05\\
48.36	3.55795223303899e-05\\
48.37	3.53412443733405e-05\\
48.38	3.51029154653706e-05\\
48.39	3.48645369422119e-05\\
48.4	3.46261101771255e-05\\
48.41	3.43876365819507e-05\\
48.42	3.4149117608186e-05\\
48.43	3.39105547481049e-05\\
48.44	3.36719495358916e-05\\
48.45	3.34333035488191e-05\\
48.46	3.31946184084705e-05\\
48.47	3.29558957819706e-05\\
48.48	3.27171373832747e-05\\
48.49	3.24783449744889e-05\\
48.5	3.22395203672266e-05\\
48.51	3.20006654239945e-05\\
48.52	3.17617820596548e-05\\
48.53	3.15228722428795e-05\\
48.54	3.12839379976847e-05\\
48.55	3.10449814050033e-05\\
48.56	3.08060046042887e-05\\
48.57	3.05670097951669e-05\\
48.58	3.03279992391638e-05\\
48.59	3.00889752614664e-05\\
48.6	2.98499402526971e-05\\
48.61	2.9610896670822e-05\\
48.62	2.93718470430436e-05\\
48.63	2.91327939677694e-05\\
48.64	2.88937401166714e-05\\
48.65	2.86546882367449e-05\\
48.66	2.84156411524943e-05\\
48.67	2.81766017681158e-05\\
48.68	2.79375730698212e-05\\
48.69	2.76985581281457e-05\\
48.7	2.74595601004006e-05\\
48.71	2.72205822331329e-05\\
48.72	2.69816278647226e-05\\
48.73	2.67427004280018e-05\\
48.74	2.65038034529836e-05\\
48.75	2.62649405696463e-05\\
48.76	2.60261155108337e-05\\
48.77	2.57873321152201e-05\\
48.78	2.55485943303491e-05\\
48.79	2.53099062157997e-05\\
48.8	2.50712719464287e-05\\
48.81	2.48326958156957e-05\\
48.82	2.45941822390929e-05\\
48.83	2.43557357577171e-05\\
48.84	2.41173610418775e-05\\
48.85	2.38790628948743e-05\\
48.86	2.36408462568633e-05\\
48.87	2.34027162088167e-05\\
48.88	2.31646779766548e-05\\
48.89	2.29267369354514e-05\\
48.9	2.26888986137805e-05\\
48.91	2.24511686982305e-05\\
48.92	2.22135530379679e-05\\
48.93	2.19760576495465e-05\\
48.94	2.1738688721764e-05\\
48.95	2.15014526207345e-05\\
48.96	2.1264355895086e-05\\
48.97	2.10274052812841e-05\\
48.98	2.07906077091969e-05\\
48.99	2.05539703077365e-05\\
49	2.03175004107343e-05\\
49.01	2.00812055629711e-05\\
49.02	1.98450935263873e-05\\
49.03	1.96091722864822e-05\\
49.04	1.93734500589342e-05\\
49.05	1.91379352963653e-05\\
49.06	1.8902636695366e-05\\
49.07	1.86675632037076e-05\\
49.08	1.84327240277705e-05\\
49.09	1.81981286402223e-05\\
49.1	1.79637867879015e-05\\
49.11	1.77297084999455e-05\\
49.12	1.74959040961948e-05\\
49.13	1.72623841958163e-05\\
49.14	1.70291597262253e-05\\
49.15	1.67962419322376e-05\\
49.16	1.65636423855465e-05\\
49.17	1.63313729944561e-05\\
49.18	1.60994460139258e-05\\
49.19	1.58678740559397e-05\\
49.2	1.56366701001519e-05\\
49.21	1.54058475048992e-05\\
49.22	1.51754200185348e-05\\
49.23	1.49454017910992e-05\\
49.24	1.47158073863785e-05\\
49.25	1.44866517943071e-05\\
49.26	1.42579504437655e-05\\
49.27	1.4029719215769e-05\\
49.28	1.38019744570662e-05\\
49.29	1.35747329941451e-05\\
49.3	1.33480121477011e-05\\
49.31	1.3121829747503e-05\\
49.32	1.28962041477734e-05\\
49.33	1.26711542430043e-05\\
49.34	1.24466994842858e-05\\
49.35	1.22228598961188e-05\\
49.36	1.19996560937769e-05\\
49.37	1.17771093011733e-05\\
49.38	1.15552413693203e-05\\
49.39	1.13340747953366e-05\\
49.4	1.11136327420446e-05\\
49.41	1.08939390581712e-05\\
49.42	1.06750182992232e-05\\
49.43	1.04568957489497e-05\\
49.44	1.02395974415186e-05\\
49.45	1.00231501843507e-05\\
49.46	9.80758158169656e-06\\
49.47	9.5929200589432e-06\\
49.48	9.37919488764281e-06\\
49.49	9.16643621139655e-06\\
49.5	8.95467507245999e-06\\
49.51	8.7439434392541e-06\\
49.52	8.53427423470193e-06\\
49.53	8.32570136545877e-06\\
49.54	8.1182597520494e-06\\
49.55	7.91198535999259e-06\\
49.56	7.70691523182578e-06\\
49.57	7.50308752021055e-06\\
49.58	7.3005415220477e-06\\
49.59	7.09931771364358e-06\\
49.6	6.89945778702648e-06\\
49.61	6.70100468736466e-06\\
49.62	6.50400265159862e-06\\
49.63	6.30849724824779e-06\\
49.64	6.11453541852873e-06\\
49.65	5.92216551871354e-06\\
49.66	5.73143736386054e-06\\
49.67	5.54240227291364e-06\\
49.68	5.35511311522581e-06\\
49.69	5.16962435856416e-06\\
49.7	4.98599211861371e-06\\
49.71	4.80427421007366e-06\\
49.72	4.6245301993323e-06\\
49.73	4.44682145886444e-06\\
49.74	4.2712112232994e-06\\
49.75	4.09776464731053e-06\\
49.76	3.92654886530348e-06\\
49.77	3.7576330530302e-06\\
49.78	3.59108849115088e-06\\
49.79	3.42698863078569e-06\\
49.8	3.26540916121588e-06\\
49.81	3.10642807966992e-06\\
49.82	2.95012576337524e-06\\
49.83	2.79658504387034e-06\\
49.84	2.64589128372815e-06\\
49.85	2.49813245567845e-06\\
49.86	2.35339922428218e-06\\
49.87	2.21178503019386e-06\\
49.88	2.07338617713368e-06\\
49.89	1.93830192161265e-06\\
49.9	1.80663456554771e-06\\
49.91	1.67848955179642e-06\\
49.92	1.55397556281939e-06\\
49.93	1.43320462240269e-06\\
49.94	1.31629220072663e-06\\
49.95	1.2033573227252e-06\\
49.96	1.09452267997232e-06\\
49.97	9.89914746108572e-07\\
49.98	8.8966389599765e-07\\
49.99	7.93904528686995e-07\\
50	7.02775194325392e-07\\
50.01	6.16418725146728e-07\\
50.02	5.34982370643117e-07\\
50.03	4.58617937116462e-07\\
50.04	3.87481931683059e-07\\
50.05	3.21735710913373e-07\\
50.06	2.61545634249252e-07\\
50.07	2.070832223703e-07\\
50.08	1.58525320644726e-07\\
50.09	1.16054267820798e-07\\
50.1	7.9858070203484e-08\\
50.11	5.01305813648684e-08\\
50.12	2.7071687723132e-08\\
50.13	1.08875000473518e-08\\
50.14	1.79055115699656e-09\\
50.15	0\\
50.16	0\\
50.17	0\\
50.18	0\\
50.19	0\\
50.2	0\\
50.21	0\\
50.22	0\\
50.23	0\\
50.24	0\\
50.25	0\\
50.26	0\\
50.27	0\\
50.28	0\\
50.29	0\\
50.3	0\\
50.31	0\\
50.32	0\\
50.33	0\\
50.34	0\\
50.35	0\\
50.36	0\\
50.37	0\\
50.38	0\\
50.39	0\\
50.4	1.73472347597681e-18\\
50.41	0\\
50.42	1.73472347597681e-18\\
50.43	0\\
50.44	0\\
50.45	0\\
50.46	0\\
50.47	0\\
50.48	0\\
50.49	0\\
50.5	0\\
50.51	0\\
50.52	0\\
50.53	0\\
50.54	0\\
50.55	0\\
50.56	0\\
50.57	0\\
50.58	0\\
50.59	0\\
50.6	0\\
50.61	1.73472347597681e-18\\
50.62	0\\
50.63	0\\
50.64	0\\
50.65	0\\
50.66	1.73472347597681e-18\\
50.67	0\\
50.68	0\\
50.69	0\\
50.7	1.73472347597681e-18\\
50.71	0\\
50.72	1.73472347597681e-18\\
50.73	1.73472347597681e-18\\
50.74	0\\
50.75	0\\
50.76	0\\
50.77	0\\
50.78	0\\
50.79	1.73472347597681e-18\\
50.8	0\\
50.81	1.73472347597681e-18\\
50.82	0\\
50.83	1.73472347597681e-18\\
50.84	0\\
50.85	1.73472347597681e-18\\
50.86	0\\
50.87	0\\
50.88	0\\
50.89	0\\
50.9	0\\
50.91	1.73472347597681e-18\\
50.92	1.73472347597681e-18\\
50.93	0\\
50.94	0\\
50.95	0\\
50.96	1.73472347597681e-18\\
50.97	0\\
50.98	0\\
50.99	0\\
51	0\\
51.01	0\\
51.02	1.73472347597681e-18\\
51.03	0\\
51.04	0\\
51.05	1.73472347597681e-18\\
51.06	0\\
51.07	0\\
51.08	0\\
51.09	0\\
51.1	0\\
51.11	0\\
51.12	1.73472347597681e-18\\
51.13	0\\
51.14	1.73472347597681e-18\\
51.15	0\\
51.16	1.73472347597681e-18\\
51.17	1.73472347597681e-18\\
51.18	0\\
51.19	1.73472347597681e-18\\
51.2	0\\
51.21	0\\
51.22	0\\
51.23	0\\
51.24	0\\
51.25	0\\
51.26	1.73472347597681e-18\\
51.27	1.73472347597681e-18\\
51.28	0\\
51.29	0\\
51.3	1.73472347597681e-18\\
51.31	0\\
51.32	0\\
51.33	0\\
51.34	0\\
51.35	0\\
51.36	0\\
51.37	0\\
51.38	0\\
51.39	0\\
51.4	0\\
51.41	1.73472347597681e-18\\
51.42	0\\
51.43	1.73472347597681e-18\\
51.44	1.73472347597681e-18\\
51.45	0\\
51.46	0\\
51.47	1.73472347597681e-18\\
51.48	0\\
51.49	0\\
51.5	1.73472347597681e-18\\
51.51	0\\
51.52	0\\
51.53	0\\
51.54	0\\
51.55	0\\
51.56	0\\
51.57	1.73472347597681e-18\\
51.58	1.73472347597681e-18\\
51.59	1.73472347597681e-18\\
51.6	0\\
51.61	0\\
51.62	1.73472347597681e-18\\
51.63	0\\
51.64	1.73472347597681e-18\\
51.65	0\\
51.66	0\\
51.67	0\\
51.68	0\\
51.69	1.73472347597681e-18\\
51.7	0\\
51.71	0\\
51.72	0\\
51.73	0\\
51.74	0\\
51.75	0\\
51.76	0\\
51.77	0\\
51.78	1.73472347597681e-18\\
51.79	1.73472347597681e-18\\
51.8	0\\
51.81	0\\
51.82	0\\
51.83	0\\
51.84	1.73472347597681e-18\\
51.85	0\\
51.86	1.73472347597681e-18\\
51.87	0\\
51.88	1.73472347597681e-18\\
51.89	1.73472347597681e-18\\
51.9	0\\
51.91	0\\
51.92	1.73472347597681e-18\\
51.93	0\\
51.94	0\\
51.95	0\\
51.96	0\\
51.97	0\\
51.98	0\\
51.99	1.73472347597681e-18\\
52	0\\
52.01	0\\
52.02	1.73472347597681e-18\\
52.03	0\\
52.04	1.73472347597681e-18\\
52.05	0\\
52.06	0\\
52.07	0\\
52.08	0\\
52.09	0\\
52.1	0\\
52.11	0\\
52.12	0\\
52.13	1.73472347597681e-18\\
52.14	0\\
52.15	1.73472347597681e-18\\
52.16	1.73472347597681e-18\\
52.17	1.73472347597681e-18\\
52.18	0\\
52.19	0\\
52.2	0\\
52.21	0\\
52.22	0\\
52.23	0\\
52.24	0\\
52.25	0\\
52.26	0\\
52.27	0\\
52.28	1.73472347597681e-18\\
52.29	0\\
52.3	1.73472347597681e-18\\
52.31	0\\
52.32	0\\
52.33	0\\
52.34	0\\
52.35	0\\
52.36	0\\
52.37	1.73472347597681e-18\\
52.38	0\\
52.39	0\\
52.4	0\\
52.41	0\\
52.42	0\\
52.43	1.73472347597681e-18\\
52.44	0\\
52.45	1.73472347597681e-18\\
52.46	1.73472347597681e-18\\
52.47	1.73472347597681e-18\\
52.48	0\\
52.49	0\\
52.5	0\\
52.51	0\\
52.52	1.73472347597681e-18\\
52.53	0\\
52.54	0\\
52.55	0\\
52.56	1.73472347597681e-18\\
52.57	0\\
52.58	0\\
52.59	0\\
52.6	1.73472347597681e-18\\
52.61	0\\
52.62	0\\
52.63	0\\
52.64	0\\
52.65	0\\
52.66	0\\
52.67	1.73472347597681e-18\\
52.68	1.73472347597681e-18\\
52.69	1.73472347597681e-18\\
52.7	0\\
52.71	0\\
52.72	0\\
52.73	0\\
52.74	0\\
52.75	0\\
52.76	0\\
52.77	1.73472347597681e-18\\
52.78	1.73472347597681e-18\\
52.79	0\\
52.8	0\\
52.81	0\\
52.82	0\\
52.83	0\\
52.84	0\\
52.85	0\\
52.86	0\\
52.87	0\\
52.88	0\\
52.89	0\\
52.9	0\\
52.91	0\\
52.92	0\\
52.93	0\\
52.94	0\\
52.95	1.73472347597681e-18\\
52.96	0\\
52.97	0\\
52.98	0\\
52.99	0\\
53	1.73472347597681e-18\\
53.01	0\\
53.02	0\\
53.03	1.73472347597681e-18\\
53.04	0\\
53.05	0\\
53.06	0\\
53.07	0\\
53.08	0\\
53.09	0\\
53.1	0\\
53.11	0\\
53.12	0\\
53.13	0\\
53.14	0\\
53.15	0\\
53.16	1.73472347597681e-18\\
53.17	0\\
53.18	1.73472347597681e-18\\
53.19	0\\
53.2	1.73472347597681e-18\\
53.21	1.73472347597681e-18\\
53.22	0\\
53.23	0\\
53.24	0\\
53.25	0\\
53.26	1.73472347597681e-18\\
53.27	0\\
53.28	1.73472347597681e-18\\
53.29	0\\
53.3	0\\
53.31	0\\
53.32	0\\
53.33	0\\
53.34	1.73472347597681e-18\\
53.35	0\\
53.36	0\\
53.37	0\\
53.38	1.73472347597681e-18\\
53.39	1.73472347597681e-18\\
53.4	0\\
53.41	0\\
53.42	0\\
53.43	0\\
53.44	1.73472347597681e-18\\
53.45	0\\
53.46	0\\
53.47	0\\
53.48	0\\
53.49	1.73472347597681e-18\\
53.5	1.73472347597681e-18\\
53.51	0\\
53.52	1.73472347597681e-18\\
53.53	1.73472347597681e-18\\
53.54	0\\
53.55	0\\
53.56	0\\
53.57	1.73472347597681e-18\\
53.58	0\\
53.59	0\\
53.6	0\\
53.61	0\\
53.62	0\\
53.63	0\\
53.64	0\\
53.65	0\\
53.66	0\\
53.67	0\\
53.68	1.73472347597681e-18\\
53.69	0\\
53.7	0\\
53.71	0\\
53.72	1.73472347597681e-18\\
53.73	0\\
53.74	0\\
53.75	1.73472347597681e-18\\
53.76	0\\
53.77	1.73472347597681e-18\\
53.78	1.73472347597681e-18\\
53.79	0\\
53.8	0\\
53.81	1.73472347597681e-18\\
53.82	0\\
53.83	0\\
53.84	1.73472347597681e-18\\
53.85	0\\
53.86	0\\
53.87	0\\
53.88	1.73472347597681e-18\\
53.89	0\\
53.9	1.73472347597681e-18\\
53.91	0\\
53.92	0\\
53.93	0\\
53.94	0\\
53.95	0\\
53.96	0\\
53.97	0\\
53.98	0\\
53.99	0\\
54	0\\
54.01	1.73472347597681e-18\\
54.02	0\\
54.03	0\\
54.04	1.73472347597681e-18\\
54.05	0\\
54.06	0\\
54.07	0\\
54.08	0\\
54.09	0\\
54.1	0\\
54.11	0\\
54.12	0\\
54.13	0\\
54.14	1.73472347597681e-18\\
54.15	0\\
54.16	1.73472347597681e-18\\
54.17	0\\
54.18	0\\
54.19	0\\
54.2	1.73472347597681e-18\\
54.21	0\\
54.22	0\\
54.23	0\\
54.24	0\\
54.25	0\\
54.26	0\\
54.27	1.73472347597681e-18\\
54.28	0\\
54.29	0\\
54.3	0\\
54.31	0\\
54.32	1.73472347597681e-18\\
54.33	0\\
54.34	0\\
54.35	0\\
54.36	0\\
54.37	0\\
54.38	0\\
54.39	0\\
54.4	0\\
54.41	0\\
54.42	0\\
54.43	0\\
54.44	0\\
54.45	0\\
54.46	0\\
54.47	0\\
54.48	0\\
54.49	0\\
54.5	0\\
54.51	1.73472347597681e-18\\
54.52	1.73472347597681e-18\\
54.53	0\\
54.54	0\\
54.55	0\\
54.56	0\\
54.57	0\\
54.58	0\\
54.59	0\\
54.6	0\\
54.61	1.73472347597681e-18\\
54.62	1.73472347597681e-18\\
54.63	0\\
54.64	0\\
54.65	0\\
54.66	1.73472347597681e-18\\
54.67	0\\
54.68	1.73472347597681e-18\\
54.69	0\\
54.7	1.73472347597681e-18\\
54.71	0\\
54.72	0\\
54.73	0\\
54.74	0\\
54.75	0\\
54.76	0\\
54.77	0\\
54.78	0\\
54.79	1.73472347597681e-18\\
54.8	0\\
54.81	0\\
54.82	0\\
54.83	0\\
54.84	0\\
54.85	0\\
54.86	0\\
54.87	0\\
54.88	0\\
54.89	1.73472347597681e-18\\
54.9	0\\
54.91	0\\
54.92	1.73472347597681e-18\\
54.93	0\\
54.94	0\\
54.95	0\\
54.96	1.73472347597681e-18\\
54.97	0\\
54.98	0\\
54.99	0\\
55	0\\
55.01	1.73472347597681e-18\\
55.02	1.73472347597681e-18\\
55.03	0\\
55.04	0\\
55.05	0\\
55.06	0\\
55.07	0\\
55.08	1.73472347597681e-18\\
55.09	0\\
55.1	0\\
55.11	1.73472347597681e-18\\
55.12	0\\
55.13	0\\
55.14	1.73472347597681e-18\\
55.15	0\\
55.16	1.73472347597681e-18\\
55.17	0\\
55.18	0\\
55.19	0\\
55.2	1.73472347597681e-18\\
55.21	1.73472347597681e-18\\
55.22	0\\
55.23	0\\
55.24	0\\
55.25	0\\
55.26	0\\
55.27	1.73472347597681e-18\\
55.28	0\\
55.29	0\\
55.3	1.73472347597681e-18\\
55.31	0\\
55.32	0\\
55.33	1.73472347597681e-18\\
55.34	0\\
55.35	0\\
55.36	0\\
55.37	0\\
55.38	0\\
55.39	0\\
55.4	0\\
55.41	0\\
55.42	0\\
55.43	0\\
55.44	0\\
55.45	0\\
55.46	0\\
55.47	0\\
55.48	0\\
55.49	0\\
55.5	1.73472347597681e-18\\
55.51	1.73472347597681e-18\\
55.52	0\\
55.53	0\\
55.54	1.73472347597681e-18\\
55.55	0\\
55.56	0\\
55.57	0\\
55.58	0\\
55.59	0\\
55.6	0\\
55.61	0\\
55.62	0\\
55.63	0\\
55.64	0\\
55.65	0\\
55.66	0\\
55.67	0\\
55.68	0\\
55.69	0\\
55.7	1.73472347597681e-18\\
55.71	0\\
55.72	1.73472347597681e-18\\
55.73	0\\
55.74	0\\
55.75	0\\
55.76	0\\
55.77	1.73472347597681e-18\\
55.78	1.73472347597681e-18\\
55.79	0\\
55.8	0\\
55.81	1.73472347597681e-18\\
55.82	0\\
55.83	0\\
55.84	0\\
55.85	1.73472347597681e-18\\
55.86	0\\
55.87	0\\
55.88	0\\
55.89	1.73472347597681e-18\\
55.9	0\\
55.91	0\\
55.92	0\\
55.93	0\\
55.94	0\\
55.95	0\\
55.96	0\\
55.97	0\\
55.98	0\\
55.99	0\\
56	1.73472347597681e-18\\
56.01	0\\
56.02	0\\
56.03	1.73472347597681e-18\\
56.04	0\\
56.05	0\\
56.06	0\\
56.07	0\\
56.08	0\\
56.09	0\\
56.1	0\\
56.11	0\\
56.12	0\\
56.13	0\\
56.14	0\\
56.15	0\\
56.16	0\\
56.17	0\\
56.18	0\\
56.19	1.73472347597681e-18\\
56.2	0\\
56.21	0\\
56.22	0\\
56.23	1.73472347597681e-18\\
56.24	0\\
56.25	1.73472347597681e-18\\
56.26	0\\
56.27	0\\
56.28	0\\
56.29	1.73472347597681e-18\\
56.3	0\\
56.31	0\\
56.32	1.73472347597681e-18\\
56.33	0\\
56.34	0\\
56.35	0\\
56.36	0\\
56.37	0\\
56.38	0\\
56.39	0\\
56.4	1.73472347597681e-18\\
56.41	0\\
56.42	0\\
56.43	0\\
56.44	0\\
56.45	0\\
56.46	0\\
56.47	0\\
56.48	0\\
56.49	0\\
56.5	0\\
56.51	0\\
56.52	0\\
56.53	1.73472347597681e-18\\
56.54	1.73472347597681e-18\\
56.55	0\\
56.56	1.73472347597681e-18\\
56.57	0\\
56.58	0\\
56.59	1.73472347597681e-18\\
56.6	1.73472347597681e-18\\
56.61	0\\
56.62	0\\
56.63	0\\
56.64	0\\
56.65	0\\
56.66	0\\
56.67	1.73472347597681e-18\\
56.68	1.73472347597681e-18\\
56.69	1.73472347597681e-18\\
56.7	0\\
56.71	0\\
56.72	0\\
56.73	1.73472347597681e-18\\
56.74	0\\
56.75	0\\
56.76	0\\
56.77	1.73472347597681e-18\\
56.78	1.73472347597681e-18\\
56.79	0\\
56.8	0\\
56.81	0\\
56.82	0\\
56.83	0\\
56.84	0\\
56.85	0\\
56.86	0\\
56.87	1.73472347597681e-18\\
56.88	0\\
56.89	0\\
56.9	0\\
56.91	0\\
56.92	0\\
56.93	0\\
56.94	0\\
56.95	0\\
56.96	0\\
56.97	1.73472347597681e-18\\
56.98	0\\
56.99	0\\
57	0\\
57.01	0\\
57.02	0\\
57.03	0\\
57.04	1.73472347597681e-18\\
57.05	0\\
57.06	0\\
57.07	0\\
57.08	0\\
57.09	0\\
57.1	1.73472347597681e-18\\
57.11	0\\
57.12	0\\
57.13	0\\
57.14	1.73472347597681e-18\\
57.15	0\\
57.16	0\\
57.17	0\\
57.18	0\\
57.19	0\\
57.2	0\\
57.21	1.73472347597681e-18\\
57.22	0\\
57.23	0\\
57.24	0\\
57.25	1.73472347597681e-18\\
57.26	0\\
57.27	0\\
57.28	1.73472347597681e-18\\
57.29	1.73472347597681e-18\\
57.3	0\\
57.31	0\\
57.32	0\\
57.33	0\\
57.34	0\\
57.35	1.73472347597681e-18\\
57.36	0\\
57.37	0\\
57.38	0\\
57.39	0\\
57.4	0\\
57.41	0\\
57.42	0\\
57.43	1.73472347597681e-18\\
57.44	1.73472347597681e-18\\
57.45	0\\
57.46	0\\
57.47	0\\
57.48	0\\
57.49	0\\
57.5	0\\
57.51	0\\
57.52	0\\
57.53	0\\
57.54	0\\
57.55	0\\
57.56	0\\
57.57	0\\
57.58	0\\
57.59	1.73472347597681e-18\\
57.6	0\\
57.61	0\\
57.62	1.73472347597681e-18\\
57.63	0\\
57.64	1.73472347597681e-18\\
57.65	0\\
57.66	0\\
57.67	0\\
57.68	0\\
57.69	1.73472347597681e-18\\
57.7	0\\
57.71	1.73472347597681e-18\\
57.72	1.73472347597681e-18\\
57.73	0\\
57.74	0\\
57.75	0\\
57.76	0\\
57.77	1.73472347597681e-18\\
57.78	1.73472347597681e-18\\
57.79	0\\
57.8	1.73472347597681e-18\\
57.81	1.73472347597681e-18\\
57.82	0\\
57.83	0\\
57.84	0\\
57.85	0\\
57.86	0\\
57.87	1.73472347597681e-18\\
57.88	0\\
57.89	0\\
57.9	0\\
57.91	1.73472347597681e-18\\
57.92	0\\
57.93	1.73472347597681e-18\\
57.94	0\\
57.95	0\\
57.96	0\\
57.97	0\\
57.98	0\\
57.99	0\\
58	0\\
58.01	0\\
58.02	0\\
58.03	0\\
58.04	0\\
58.05	1.73472347597681e-18\\
58.06	0\\
58.07	0\\
58.08	0\\
58.09	0\\
58.1	0\\
58.11	0\\
58.12	0\\
58.13	0\\
58.14	1.73472347597681e-18\\
58.15	1.73472347597681e-18\\
58.16	0\\
58.17	0\\
58.18	0\\
58.19	0\\
58.2	1.73472347597681e-18\\
58.21	0\\
58.22	0\\
58.23	0\\
58.24	0\\
58.25	0\\
58.26	0\\
58.27	0\\
58.28	0\\
58.29	1.73472347597681e-18\\
58.3	0\\
58.31	0\\
58.32	0\\
58.33	0\\
58.34	0\\
58.35	0\\
58.36	1.73472347597681e-18\\
58.37	0\\
58.38	0\\
58.39	0\\
58.4	0\\
58.41	0\\
58.42	0\\
58.43	0\\
58.44	0\\
58.45	1.73472347597681e-18\\
58.46	0\\
58.47	0\\
58.48	0\\
58.49	0\\
58.5	1.73472347597681e-18\\
58.51	1.73472347597681e-18\\
58.52	1.73472347597681e-18\\
58.53	1.73472347597681e-18\\
58.54	0\\
58.55	1.73472347597681e-18\\
58.56	0\\
58.57	0\\
58.58	0\\
58.59	0\\
58.6	0\\
58.61	0\\
58.62	1.73472347597681e-18\\
58.63	0\\
58.64	0\\
58.65	0\\
58.66	0\\
58.67	1.73472347597681e-18\\
58.68	1.73472347597681e-18\\
58.69	0\\
58.7	0\\
58.71	1.73472347597681e-18\\
58.72	0\\
58.73	0\\
58.74	1.73472347597681e-18\\
58.75	0\\
58.76	0\\
58.77	0\\
58.78	0\\
58.79	1.73472347597681e-18\\
58.8	0\\
58.81	1.73472347597681e-18\\
58.82	0\\
58.83	0\\
58.84	0\\
58.85	1.73472347597681e-18\\
58.86	1.73472347597681e-18\\
58.87	0\\
58.88	0\\
58.89	0\\
58.9	0\\
58.91	0\\
58.92	1.73472347597681e-18\\
58.93	0\\
58.94	0\\
58.95	0\\
58.96	0\\
58.97	0\\
58.98	0\\
58.99	0\\
59	1.73472347597681e-18\\
59.01	0\\
59.02	0\\
59.03	1.73472347597681e-18\\
59.04	0\\
59.05	0\\
59.06	0\\
59.07	0\\
59.08	0\\
59.09	0\\
59.1	1.73472347597681e-18\\
59.11	0\\
59.12	0\\
59.13	1.73472347597681e-18\\
59.14	0\\
59.15	1.73472347597681e-18\\
59.16	0\\
59.17	0\\
59.18	1.73472347597681e-18\\
59.19	1.73472347597681e-18\\
59.2	0\\
59.21	1.73472347597681e-18\\
59.22	0\\
59.23	0\\
59.24	0\\
59.25	1.73472347597681e-18\\
59.26	0\\
59.27	1.73472347597681e-18\\
59.28	0\\
59.29	0\\
59.3	0\\
59.31	1.73472347597681e-18\\
59.32	0\\
59.33	0\\
59.34	0\\
59.35	0\\
59.36	1.73472347597681e-18\\
59.37	0\\
59.38	0\\
59.39	0\\
59.4	1.73472347597681e-18\\
59.41	0\\
59.42	0\\
59.43	0\\
59.44	1.73472347597681e-18\\
59.45	0\\
59.46	1.73472347597681e-18\\
59.47	0\\
59.48	1.73472347597681e-18\\
59.49	1.73472347597681e-18\\
59.5	0\\
59.51	0\\
59.52	0\\
59.53	0\\
59.54	0\\
59.55	1.73472347597681e-18\\
59.56	0\\
59.57	0\\
59.58	0\\
59.59	0\\
59.6	0\\
59.61	0\\
59.62	1.73472347597681e-18\\
59.63	1.73472347597681e-18\\
59.64	0\\
59.65	0\\
59.66	0\\
59.67	0\\
59.68	0\\
59.69	0\\
59.7	0\\
59.71	0\\
59.72	1.73472347597681e-18\\
59.73	1.73472347597681e-18\\
59.74	0\\
59.75	0\\
59.76	0\\
59.77	0\\
59.78	0\\
59.79	0\\
59.8	0\\
59.81	1.73472347597681e-18\\
59.82	0\\
59.83	0\\
59.84	0\\
59.85	1.73472347597681e-18\\
59.86	0\\
59.87	1.73472347597681e-18\\
59.88	0\\
59.89	0\\
59.9	0\\
59.91	0\\
59.92	0\\
59.93	0\\
59.94	1.73472347597681e-18\\
59.95	0\\
59.96	0\\
59.97	0\\
59.98	0\\
59.99	1.73472347597681e-18\\
60	0\\
60.01	0\\
60.02	1.73472347597681e-18\\
60.03	0\\
60.04	0\\
60.05	0\\
60.06	0\\
60.07	0\\
60.08	0\\
60.09	0\\
60.1	0\\
60.11	0\\
60.12	0\\
60.13	0\\
60.14	1.73472347597681e-18\\
60.15	0\\
60.16	0\\
60.17	0\\
60.18	0\\
60.19	0\\
60.2	0\\
60.21	0\\
60.22	1.73472347597681e-18\\
60.23	0\\
60.24	0\\
60.25	0\\
60.26	0\\
60.27	1.73472347597681e-18\\
60.28	0\\
60.29	0\\
60.3	0\\
60.31	0\\
60.32	0\\
60.33	0\\
60.34	0\\
60.35	1.73472347597681e-18\\
60.36	1.73472347597681e-18\\
60.37	0\\
60.38	1.73472347597681e-18\\
60.39	0\\
60.4	1.73472347597681e-18\\
60.41	0\\
60.42	1.73472347597681e-18\\
60.43	1.73472347597681e-18\\
60.44	0\\
60.45	1.73472347597681e-18\\
60.46	0\\
60.47	0\\
60.48	0\\
60.49	0\\
60.5	0\\
60.51	0\\
60.52	0\\
60.53	1.73472347597681e-18\\
60.54	0\\
60.55	0\\
60.56	0\\
60.57	0\\
60.58	0\\
60.59	0\\
60.6	0\\
60.61	1.73472347597681e-18\\
60.62	0\\
60.63	0\\
60.64	1.73472347597681e-18\\
60.65	0\\
60.66	0\\
60.67	0\\
60.68	0\\
60.69	0\\
60.7	0\\
60.71	1.73472347597681e-18\\
60.72	0\\
60.73	0\\
60.74	1.73472347597681e-18\\
60.75	0\\
60.76	0\\
60.77	0\\
60.78	0\\
60.79	1.73472347597681e-18\\
60.8	0\\
60.81	0\\
60.82	1.73472347597681e-18\\
60.83	0\\
60.84	0\\
60.85	0\\
60.86	0\\
60.87	0\\
60.88	1.73472347597681e-18\\
60.89	0\\
60.9	1.73472347597681e-18\\
60.91	0\\
60.92	0\\
60.93	1.73472347597681e-18\\
60.94	0\\
60.95	0\\
60.96	1.73472347597681e-18\\
60.97	1.73472347597681e-18\\
60.98	0\\
60.99	0\\
61	0\\
61.01	0\\
61.02	0\\
61.03	0\\
61.04	0\\
61.05	0\\
61.06	0\\
61.07	1.73472347597681e-18\\
61.08	1.73472347597681e-18\\
61.09	0\\
61.1	0\\
61.11	1.73472347597681e-18\\
61.12	1.73472347597681e-18\\
61.13	0\\
61.14	0\\
61.15	0\\
61.16	0\\
61.17	0\\
61.18	0\\
61.19	1.73472347597681e-18\\
61.2	0\\
61.21	0\\
61.22	0\\
61.23	0\\
61.24	0\\
61.25	0\\
61.26	0\\
61.27	0\\
61.28	0\\
61.29	0\\
61.3	0\\
61.31	0\\
61.32	0\\
61.33	0\\
61.34	1.73472347597681e-18\\
61.35	1.73472347597681e-18\\
61.36	0\\
61.37	0\\
61.38	0\\
61.39	0\\
61.4	0\\
61.41	1.73472347597681e-18\\
61.42	0\\
61.43	0\\
61.44	0\\
61.45	1.73472347597681e-18\\
61.46	0\\
61.47	0\\
61.48	0\\
61.49	0\\
61.5	0\\
61.51	1.73472347597681e-18\\
61.52	1.73472347597681e-18\\
61.53	0\\
61.54	0\\
61.55	1.73472347597681e-18\\
61.56	0\\
61.57	1.73472347597681e-18\\
61.58	1.73472347597681e-18\\
61.59	0\\
61.6	0\\
61.61	0\\
61.62	0\\
61.63	0\\
61.64	0\\
61.65	1.73472347597681e-18\\
61.66	1.73472347597681e-18\\
61.67	1.73472347597681e-18\\
61.68	0\\
61.69	0\\
61.7	1.73472347597681e-18\\
61.71	0\\
61.72	0\\
61.73	0\\
61.74	0\\
61.75	0\\
61.76	0\\
61.77	0\\
61.78	0\\
61.79	0\\
61.8	0\\
61.81	0\\
61.82	0\\
61.83	0\\
61.84	1.73472347597681e-18\\
61.85	0\\
61.86	1.73472347597681e-18\\
61.87	0\\
61.88	0\\
61.89	1.73472347597681e-18\\
61.9	0\\
61.91	0\\
61.92	0\\
61.93	1.73472347597681e-18\\
61.94	0\\
61.95	0\\
61.96	0\\
61.97	1.73472347597681e-18\\
61.98	0\\
61.99	0\\
62	1.73472347597681e-18\\
62.01	0\\
62.02	1.73472347597681e-18\\
62.03	0\\
62.04	0\\
62.05	0\\
62.06	0\\
62.07	0\\
62.08	0\\
62.09	1.73472347597681e-18\\
62.1	0\\
62.11	0\\
62.12	0\\
62.13	0\\
62.14	1.73472347597681e-18\\
62.15	0\\
62.16	0\\
62.17	0\\
62.18	0\\
62.19	0\\
62.2	0\\
62.21	0\\
62.22	1.73472347597681e-18\\
62.23	0\\
62.24	0\\
62.25	1.73472347597681e-18\\
62.26	0\\
62.27	0\\
62.28	0\\
62.29	0\\
62.3	1.73472347597681e-18\\
62.31	0\\
62.32	0\\
62.33	0\\
62.34	0\\
62.35	1.73472347597681e-18\\
62.36	0\\
62.37	0\\
62.38	0\\
62.39	0\\
62.4	1.73472347597681e-18\\
62.41	0\\
62.42	1.73472347597681e-18\\
62.43	0\\
62.44	0\\
62.45	0\\
62.46	0\\
62.47	0\\
62.48	0\\
62.49	0\\
62.5	0\\
62.51	0\\
62.52	0\\
62.53	0\\
62.54	1.73472347597681e-18\\
62.55	0\\
62.56	0\\
62.57	0\\
62.58	0\\
62.59	1.73472347597681e-18\\
62.6	0\\
62.61	0\\
62.62	0\\
62.63	0\\
62.64	0\\
62.65	0\\
62.66	0\\
62.67	0\\
62.68	0\\
62.69	0\\
62.7	0\\
62.71	0\\
62.72	0\\
62.73	0\\
62.74	1.73472347597681e-18\\
62.75	0\\
62.76	0\\
62.77	0\\
62.78	0\\
62.79	0\\
62.8	0\\
62.81	0\\
62.82	0\\
62.83	0\\
62.84	1.73472347597681e-18\\
62.85	0\\
62.86	0\\
62.87	0\\
62.88	0\\
62.89	0\\
62.9	0\\
62.91	0\\
62.92	0\\
62.93	0\\
62.94	0\\
62.95	0\\
62.96	0\\
62.97	1.73472347597681e-18\\
62.98	0\\
62.99	1.73472347597681e-18\\
63	0\\
63.01	0\\
63.02	1.73472347597681e-18\\
63.03	0\\
63.04	0\\
63.05	0\\
63.06	1.73472347597681e-18\\
63.07	0\\
63.08	0\\
63.09	0\\
63.1	0\\
63.11	0\\
63.12	0\\
63.13	0\\
63.14	0\\
63.15	0\\
63.16	0\\
63.17	0\\
63.18	0\\
63.19	0\\
63.2	0\\
63.21	0\\
63.22	0\\
63.23	0\\
63.24	1.73472347597681e-18\\
63.25	0\\
63.26	0\\
63.27	0\\
63.28	0\\
63.29	0\\
63.3	1.73472347597681e-18\\
63.31	1.73472347597681e-18\\
63.32	0\\
63.33	0\\
63.34	0\\
63.35	0\\
63.36	1.73472347597681e-18\\
63.37	0\\
63.38	0\\
63.39	1.73472347597681e-18\\
63.4	0\\
63.41	0\\
63.42	0\\
63.43	0\\
63.44	0\\
63.45	0\\
63.46	0\\
63.47	0\\
63.48	0\\
63.49	0\\
63.5	0\\
63.51	0\\
63.52	0\\
63.53	1.73472347597681e-18\\
63.54	1.73472347597681e-18\\
63.55	1.73472347597681e-18\\
63.56	0\\
63.57	0\\
63.58	1.73472347597681e-18\\
63.59	0\\
63.6	0\\
63.61	0\\
63.62	0\\
63.63	1.73472347597681e-18\\
63.64	0\\
63.65	0\\
63.66	1.73472347597681e-18\\
63.67	0\\
63.68	0\\
63.69	0\\
63.7	0\\
63.71	0\\
63.72	0\\
63.73	0\\
63.74	0\\
63.75	1.73472347597681e-18\\
63.76	1.73472347597681e-18\\
63.77	0\\
63.78	0\\
63.79	0\\
63.8	1.73472347597681e-18\\
63.81	1.73472347597681e-18\\
63.82	0\\
63.83	0\\
63.84	0\\
63.85	0\\
63.86	0\\
63.87	1.73472347597681e-18\\
63.88	0\\
63.89	0\\
63.9	0\\
63.91	0\\
63.92	0\\
63.93	0\\
63.94	0\\
63.95	0\\
63.96	0\\
63.97	1.73472347597681e-18\\
63.98	0\\
63.99	0\\
64	0\\
64.01	0\\
64.02	0\\
64.03	0\\
64.04	0\\
64.05	1.73472347597681e-18\\
64.06	0\\
64.07	0\\
64.08	0\\
64.09	0\\
64.1	0\\
64.11	0\\
64.12	0\\
64.13	0\\
64.14	0\\
64.15	0\\
64.16	0\\
64.17	0\\
64.18	1.73472347597681e-18\\
64.19	0\\
64.2	0\\
64.21	1.73472347597681e-18\\
64.22	0\\
64.23	0\\
64.24	0\\
64.25	0\\
64.26	1.73472347597681e-18\\
64.27	0\\
64.28	1.73472347597681e-18\\
64.29	1.73472347597681e-18\\
64.3	0\\
64.31	1.73472347597681e-18\\
64.32	0\\
64.33	0\\
64.34	1.73472347597681e-18\\
64.35	0\\
64.36	1.73472347597681e-18\\
64.37	0\\
64.38	1.73472347597681e-18\\
64.39	1.73472347597681e-18\\
64.4	1.73472347597681e-18\\
64.41	1.73472347597681e-18\\
64.42	1.73472347597681e-18\\
64.43	0\\
64.44	1.73472347597681e-18\\
64.45	0\\
64.46	0\\
64.47	0\\
64.48	0\\
64.49	1.73472347597681e-18\\
64.5	0\\
64.51	0\\
64.52	1.73472347597681e-18\\
64.53	1.73472347597681e-18\\
64.54	0\\
64.55	1.73472347597681e-18\\
64.56	0\\
64.57	0\\
64.58	0\\
64.59	0\\
64.6	0\\
64.61	0\\
64.62	0\\
64.63	0\\
64.64	1.73472347597681e-18\\
64.65	0\\
64.66	0\\
64.67	0\\
64.68	0\\
64.69	0\\
64.7	1.73472347597681e-18\\
64.71	0\\
64.72	0\\
64.73	0\\
64.74	0\\
64.75	1.73472347597681e-18\\
64.76	0\\
64.77	0\\
64.78	1.73472347597681e-18\\
64.79	1.73472347597681e-18\\
64.8	0\\
64.81	1.73472347597681e-18\\
64.82	1.73472347597681e-18\\
64.83	0\\
64.84	0\\
64.85	0\\
64.86	0\\
64.87	0\\
64.88	0\\
64.89	0\\
64.9	0\\
64.91	1.73472347597681e-18\\
64.92	1.73472347597681e-18\\
64.93	0\\
64.94	0\\
64.95	0\\
64.96	0\\
64.97	0\\
64.98	0\\
64.99	0\\
65	0\\
65.01	0\\
65.02	0\\
65.03	1.73472347597681e-18\\
65.04	1.73472347597681e-18\\
65.05	0\\
65.06	0\\
65.07	0\\
65.08	0\\
65.09	0\\
65.1	1.73472347597681e-18\\
65.11	0\\
65.12	0\\
65.13	0\\
65.14	0\\
65.15	1.73472347597681e-18\\
65.16	0\\
65.17	0\\
65.18	1.73472347597681e-18\\
65.19	1.73472347597681e-18\\
65.2	0\\
65.21	0\\
65.22	0\\
65.23	0\\
65.24	1.73472347597681e-18\\
65.25	0\\
65.26	0\\
65.27	1.73472347597681e-18\\
65.28	0\\
65.29	0\\
65.3	0\\
65.31	0\\
65.32	0\\
65.33	1.73472347597681e-18\\
65.34	0\\
65.35	0\\
65.36	0\\
65.37	0\\
65.38	1.73472347597681e-18\\
65.39	0\\
65.4	0\\
65.41	0\\
65.42	1.73472347597681e-18\\
65.43	0\\
65.44	0\\
65.45	0\\
65.46	0\\
65.47	0\\
65.48	1.73472347597681e-18\\
65.49	0\\
65.5	1.73472347597681e-18\\
65.51	0\\
65.52	1.73472347597681e-18\\
65.53	0\\
65.54	0\\
65.55	1.73472347597681e-18\\
65.56	0\\
65.57	0\\
65.58	1.73472347597681e-18\\
65.59	0\\
65.6	0\\
65.61	0\\
65.62	1.73472347597681e-18\\
65.63	0\\
65.64	0\\
65.65	0\\
65.66	0\\
65.67	1.73472347597681e-18\\
65.68	0\\
65.69	0\\
65.7	0\\
65.71	1.73472347597681e-18\\
65.72	0\\
65.73	0\\
65.74	0\\
65.75	0\\
65.76	0\\
65.77	0\\
65.78	0\\
65.79	0\\
65.8	1.73472347597681e-18\\
65.81	0\\
65.82	0\\
65.83	0\\
65.84	0\\
65.85	0\\
65.86	0\\
65.87	0\\
65.88	0\\
65.89	0\\
65.9	0\\
65.91	1.73472347597681e-18\\
65.92	0\\
65.93	0\\
65.94	0\\
65.95	1.73472347597681e-18\\
65.96	0\\
65.97	0\\
65.98	1.73472347597681e-18\\
65.99	0\\
66	0\\
66.01	0\\
66.02	1.73472347597681e-18\\
66.03	1.73472347597681e-18\\
66.04	1.73472347597681e-18\\
66.05	0\\
66.06	0\\
66.07	0\\
66.08	0\\
66.09	0\\
66.1	0\\
66.11	0\\
66.12	0\\
66.13	1.73472347597681e-18\\
66.14	0\\
66.15	1.73472347597681e-18\\
66.16	1.73472347597681e-18\\
66.17	1.73472347597681e-18\\
66.18	0\\
66.19	0\\
66.2	0\\
66.21	0\\
66.22	0\\
66.23	0\\
66.24	0\\
66.25	0\\
66.26	1.73472347597681e-18\\
66.27	0\\
66.28	1.73472347597681e-18\\
66.29	0\\
66.3	1.73472347597681e-18\\
66.31	0\\
66.32	0\\
66.33	0\\
66.34	1.73472347597681e-18\\
66.35	0\\
66.36	0\\
66.37	0\\
66.38	0\\
66.39	0\\
66.4	1.73472347597681e-18\\
66.41	0\\
66.42	0\\
66.43	0\\
66.44	1.73472347597681e-18\\
66.45	0\\
66.46	0\\
66.47	1.73472347597681e-18\\
66.48	0\\
66.49	0\\
66.5	0\\
66.51	0\\
66.52	0\\
66.53	0\\
66.54	1.73472347597681e-18\\
66.55	0\\
66.56	0\\
66.57	0\\
66.58	0\\
66.59	0\\
66.6	1.73472347597681e-18\\
66.61	0\\
66.62	1.73472347597681e-18\\
66.63	0\\
66.64	1.73472347597681e-18\\
66.65	0\\
66.66	0\\
66.67	0\\
66.68	0\\
66.69	0\\
66.7	0\\
66.71	1.73472347597681e-18\\
66.72	0\\
66.73	0\\
66.74	0\\
66.75	0\\
66.76	0\\
66.77	1.73472347597681e-18\\
66.78	0\\
66.79	1.73472347597681e-18\\
66.8	0\\
66.81	0\\
66.82	0\\
66.83	0\\
66.84	0\\
66.85	1.73472347597681e-18\\
66.86	0\\
66.87	0\\
66.88	0\\
66.89	0\\
66.9	0\\
66.91	0\\
66.92	0\\
66.93	0\\
66.94	1.73472347597681e-18\\
66.95	0\\
66.96	0\\
66.97	0\\
66.98	1.73472347597681e-18\\
66.99	0\\
67	1.73472347597681e-18\\
67.01	0\\
67.02	0\\
67.03	0\\
67.04	1.73472347597681e-18\\
67.05	0\\
67.06	0\\
67.07	0\\
67.08	0\\
67.09	0\\
67.1	0\\
67.11	1.73472347597681e-18\\
67.12	0\\
67.13	1.73472347597681e-18\\
67.14	0\\
67.15	1.73472347597681e-18\\
67.16	0\\
67.17	1.73472347597681e-18\\
67.18	1.73472347597681e-18\\
67.19	1.73472347597681e-18\\
67.2	0\\
67.21	0\\
67.22	0\\
67.23	0\\
67.24	1.73472347597681e-18\\
67.25	1.73472347597681e-18\\
67.26	1.73472347597681e-18\\
67.27	1.73472347597681e-18\\
67.28	1.73472347597681e-18\\
67.29	0\\
67.3	0\\
67.31	0\\
67.32	0\\
67.33	0\\
67.34	1.73472347597681e-18\\
67.35	1.73472347597681e-18\\
67.36	0\\
67.37	1.73472347597681e-18\\
67.38	0\\
67.39	0\\
67.4	0\\
67.41	0\\
67.42	0\\
67.43	0\\
67.44	0\\
67.45	1.73472347597681e-18\\
67.46	0\\
67.47	0\\
67.48	0\\
67.49	1.73472347597681e-18\\
67.5	0\\
67.51	1.73472347597681e-18\\
67.52	1.73472347597681e-18\\
67.53	0\\
67.54	1.73472347597681e-18\\
67.55	0\\
67.56	0\\
67.57	0\\
67.58	0\\
67.59	0\\
67.6	0\\
67.61	1.73472347597681e-18\\
67.62	0\\
67.63	1.73472347597681e-18\\
67.64	0\\
67.65	0\\
67.66	0\\
67.67	0\\
67.68	0\\
67.69	0\\
67.7	0\\
67.71	0\\
67.72	0\\
67.73	1.73472347597681e-18\\
67.74	1.73472347597681e-18\\
67.75	1.73472347597681e-18\\
67.76	1.73472347597681e-18\\
67.77	0\\
67.78	0\\
67.79	0\\
67.8	0\\
67.81	1.73472347597681e-18\\
67.82	0\\
67.83	1.73472347597681e-18\\
67.84	1.73472347597681e-18\\
67.85	1.73472347597681e-18\\
67.86	0\\
67.87	0\\
67.88	1.73472347597681e-18\\
67.89	0\\
67.9	0\\
67.91	0\\
67.92	0\\
67.93	0\\
67.94	0\\
67.95	1.73472347597681e-18\\
67.96	0\\
67.97	0\\
67.98	1.73472347597681e-18\\
67.99	0\\
68	0\\
68.01	0\\
68.02	1.73472347597681e-18\\
68.03	0\\
68.04	0\\
68.05	0\\
68.06	0\\
68.07	0\\
68.08	1.73472347597681e-18\\
68.09	0\\
68.1	0\\
68.11	0\\
68.12	0\\
68.13	0\\
68.14	0\\
68.15	0\\
68.16	0\\
68.17	0\\
68.18	0\\
68.19	0\\
68.2	0\\
68.21	0\\
68.22	0\\
68.23	0\\
68.24	0\\
68.25	1.73472347597681e-18\\
68.26	1.73472347597681e-18\\
68.27	1.73472347597681e-18\\
68.28	1.73472347597681e-18\\
68.29	0\\
68.3	0\\
68.31	0\\
68.32	0\\
68.33	1.73472347597681e-18\\
68.34	0\\
68.35	1.73472347597681e-18\\
68.36	0\\
68.37	0\\
68.38	0\\
68.39	0\\
68.4	0\\
68.41	0\\
68.42	1.73472347597681e-18\\
68.43	0\\
68.44	0\\
68.45	1.73472347597681e-18\\
68.46	0\\
68.47	1.73472347597681e-18\\
68.48	1.73472347597681e-18\\
68.49	0\\
68.5	0\\
68.51	0\\
68.52	0\\
68.53	1.73472347597681e-18\\
68.54	0\\
68.55	0\\
68.56	0\\
68.57	0\\
68.58	0\\
68.59	0\\
68.6	0\\
68.61	0\\
68.62	0\\
68.63	1.73472347597681e-18\\
68.64	0\\
68.65	0\\
68.66	0\\
68.67	0\\
68.68	0\\
68.69	0\\
68.7	0\\
68.71	1.73472347597681e-18\\
68.72	0\\
68.73	0\\
68.74	1.73472347597681e-18\\
68.75	0\\
68.76	0\\
68.77	0\\
68.78	1.73472347597681e-18\\
68.79	1.73472347597681e-18\\
68.8	0\\
68.81	0\\
68.82	0\\
68.83	1.73472347597681e-18\\
68.84	0\\
68.85	0\\
68.86	1.73472347597681e-18\\
68.87	0\\
68.88	0\\
68.89	0\\
68.9	0\\
68.91	1.73472347597681e-18\\
68.92	0\\
68.93	0\\
68.94	0\\
68.95	1.73472347597681e-18\\
68.96	0\\
68.97	0\\
68.98	0\\
68.99	0\\
69	0\\
69.01	0\\
69.02	0\\
69.03	0\\
69.04	0\\
69.05	0\\
69.06	0\\
69.07	0\\
69.08	0\\
69.09	1.73472347597681e-18\\
69.1	0\\
69.11	0\\
69.12	1.73472347597681e-18\\
69.13	1.73472347597681e-18\\
69.14	0\\
69.15	0\\
69.16	0\\
69.17	0\\
69.18	0\\
69.19	0\\
69.2	0\\
69.21	1.73472347597681e-18\\
69.22	0\\
69.23	0\\
69.24	0\\
69.25	0\\
69.26	1.73472347597681e-18\\
69.27	1.73472347597681e-18\\
69.28	0\\
69.29	0\\
69.3	0\\
69.31	1.73472347597681e-18\\
69.32	1.73472347597681e-18\\
69.33	0\\
69.34	0\\
69.35	1.73472347597681e-18\\
69.36	0\\
69.37	0\\
69.38	0\\
69.39	1.73472347597681e-18\\
69.4	1.73472347597681e-18\\
69.41	1.73472347597681e-18\\
69.42	0\\
69.43	0\\
69.44	0\\
69.45	0\\
69.46	0\\
69.47	0\\
69.48	0\\
69.49	0\\
69.5	0\\
69.51	0\\
69.52	0\\
69.53	0\\
69.54	0\\
69.55	1.73472347597681e-18\\
69.56	1.73472347597681e-18\\
69.57	0\\
69.58	0\\
69.59	0\\
69.6	0\\
69.61	0\\
69.62	1.73472347597681e-18\\
69.63	0\\
69.64	1.73472347597681e-18\\
69.65	0\\
69.66	0\\
69.67	0\\
69.68	0\\
69.69	1.73472347597681e-18\\
69.7	1.73472347597681e-18\\
69.71	0\\
69.72	0\\
69.73	0\\
69.74	0\\
69.75	1.73472347597681e-18\\
69.76	0\\
69.77	0\\
69.78	0\\
69.79	0\\
69.8	0\\
69.81	0\\
69.82	0\\
69.83	1.73472347597681e-18\\
69.84	0\\
69.85	0\\
69.86	1.73472347597681e-18\\
69.87	0\\
69.88	1.73472347597681e-18\\
69.89	1.73472347597681e-18\\
69.9	0\\
69.91	1.73472347597681e-18\\
69.92	0\\
69.93	0\\
69.94	0\\
69.95	1.73472347597681e-18\\
69.96	0\\
69.97	0\\
69.98	0\\
69.99	0\\
70	0\\
70.01	1.73472347597681e-18\\
70.02	0\\
70.03	0\\
70.04	0\\
70.05	0\\
70.06	0\\
70.07	0\\
70.08	0\\
70.09	1.73472347597681e-18\\
70.1	1.73472347597681e-18\\
70.11	0\\
70.12	0\\
70.13	0\\
70.14	0\\
70.15	0\\
70.16	0\\
70.17	1.73472347597681e-18\\
70.18	0\\
70.19	0\\
70.2	0\\
70.21	0\\
70.22	1.73472347597681e-18\\
70.23	1.73472347597681e-18\\
70.24	0\\
70.25	1.73472347597681e-18\\
70.26	0\\
70.27	0\\
70.28	1.73472347597681e-18\\
70.29	1.73472347597681e-18\\
70.3	0\\
70.31	1.73472347597681e-18\\
70.32	0\\
70.33	1.73472347597681e-18\\
70.34	0\\
70.35	0\\
70.36	0\\
70.37	1.73472347597681e-18\\
70.38	1.73472347597681e-18\\
70.39	1.73472347597681e-18\\
70.4	0\\
70.41	1.73472347597681e-18\\
70.42	0\\
70.43	0\\
70.44	1.73472347597681e-18\\
70.45	1.73472347597681e-18\\
70.46	0\\
70.47	0\\
70.48	0\\
70.49	1.73472347597681e-18\\
70.5	0\\
70.51	0\\
70.52	0\\
70.53	0\\
70.54	0\\
70.55	0\\
70.56	0\\
70.57	0\\
70.58	0\\
70.59	0\\
70.6	1.73472347597681e-18\\
70.61	0\\
70.62	0\\
70.63	1.73472347597681e-18\\
70.64	1.73472347597681e-18\\
70.65	1.73472347597681e-18\\
70.66	0\\
70.67	0\\
70.68	0\\
70.69	0\\
70.7	0\\
70.71	0\\
70.72	0\\
70.73	1.73472347597681e-18\\
70.74	0\\
70.75	0\\
70.76	1.73472347597681e-18\\
70.77	0\\
70.78	0\\
70.79	0\\
70.8	0\\
70.81	0\\
70.82	1.73472347597681e-18\\
70.83	1.73472347597681e-18\\
70.84	1.73472347597681e-18\\
70.85	0\\
70.86	0\\
70.87	0\\
70.88	1.73472347597681e-18\\
70.89	0\\
70.9	1.73472347597681e-18\\
70.91	0\\
70.92	1.73472347597681e-18\\
70.93	0\\
70.94	0\\
70.95	1.73472347597681e-18\\
70.96	0\\
70.97	0\\
70.98	0\\
70.99	1.73472347597681e-18\\
71	0\\
71.01	0\\
71.02	0\\
71.03	0\\
71.04	1.73472347597681e-18\\
71.05	0\\
71.06	0\\
71.07	0\\
71.08	0\\
71.09	1.73472347597681e-18\\
71.1	0\\
71.11	0\\
71.12	1.73472347597681e-18\\
71.13	0\\
71.14	0\\
71.15	1.73472347597681e-18\\
71.16	0\\
71.17	0\\
71.18	1.73472347597681e-18\\
71.19	0\\
71.2	0\\
71.21	0\\
71.22	0\\
71.23	1.73472347597681e-18\\
71.24	0\\
71.25	0\\
71.26	0\\
71.27	0\\
71.28	0\\
71.29	0\\
71.3	0\\
71.31	0\\
71.32	1.73472347597681e-18\\
71.33	0\\
71.34	1.73472347597681e-18\\
71.35	0\\
71.36	0\\
71.37	0\\
71.38	0\\
71.39	0\\
71.4	0\\
71.41	1.73472347597681e-18\\
71.42	0\\
71.43	0\\
71.44	0\\
71.45	0\\
71.46	0\\
71.47	0\\
71.48	0\\
71.49	0\\
71.5	1.73472347597681e-18\\
71.51	1.73472347597681e-18\\
71.52	0\\
71.53	1.73472347597681e-18\\
71.54	0\\
71.55	0\\
71.56	0\\
71.57	0\\
71.58	0\\
71.59	0\\
71.6	0\\
71.61	0\\
71.62	0\\
71.63	0\\
71.64	0\\
71.65	0\\
71.66	0\\
71.67	0\\
71.68	0\\
71.69	1.73472347597681e-18\\
71.7	0\\
71.71	0\\
71.72	0\\
71.73	0\\
71.74	0\\
71.75	0\\
71.76	0\\
71.77	0\\
71.78	1.73472347597681e-18\\
71.79	1.73472347597681e-18\\
71.8	0\\
71.81	1.73472347597681e-18\\
71.82	0\\
71.83	1.73472347597681e-18\\
71.84	1.73472347597681e-18\\
71.85	0\\
71.86	1.73472347597681e-18\\
71.87	0\\
71.88	1.73472347597681e-18\\
71.89	0\\
71.9	1.73472347597681e-18\\
71.91	1.73472347597681e-18\\
71.92	0\\
71.93	0\\
71.94	1.73472347597681e-18\\
71.95	1.73472347597681e-18\\
71.96	0\\
71.97	0\\
71.98	0\\
71.99	0\\
72	0\\
72.01	0\\
72.02	0\\
72.03	0\\
72.04	1.73472347597681e-18\\
72.05	0\\
72.06	0\\
72.07	0\\
72.08	0\\
72.09	1.73472347597681e-18\\
72.1	1.73472347597681e-18\\
72.11	0\\
72.12	0\\
72.13	0\\
72.14	0\\
72.15	1.73472347597681e-18\\
72.16	1.73472347597681e-18\\
72.17	0\\
72.18	0\\
72.19	1.73472347597681e-18\\
72.2	0\\
72.21	1.73472347597681e-18\\
72.22	0\\
72.23	0\\
72.24	1.73472347597681e-18\\
72.25	0\\
72.26	0\\
72.27	0\\
72.28	0\\
72.29	0\\
72.3	0\\
72.31	0\\
72.32	0\\
72.33	0\\
72.34	0\\
72.35	0\\
72.36	1.73472347597681e-18\\
72.37	0\\
72.38	0\\
72.39	0\\
72.4	0\\
72.41	1.73472347597681e-18\\
72.42	0\\
72.43	0\\
72.44	0\\
72.45	1.73472347597681e-18\\
72.46	0\\
72.47	0\\
72.48	0\\
72.49	0\\
72.5	1.73472347597681e-18\\
72.51	0\\
72.52	1.73472347597681e-18\\
72.53	0\\
72.54	1.73472347597681e-18\\
72.55	0\\
72.56	0\\
72.57	0\\
72.58	0\\
72.59	0\\
72.6	0\\
72.61	1.73472347597681e-18\\
72.62	1.73472347597681e-18\\
72.63	0\\
72.64	0\\
72.65	1.73472347597681e-18\\
72.66	1.73472347597681e-18\\
72.67	1.73472347597681e-18\\
72.68	1.73472347597681e-18\\
72.69	1.73472347597681e-18\\
72.7	0\\
72.71	0\\
72.72	0\\
72.73	1.73472347597681e-18\\
72.74	0\\
72.75	1.73472347597681e-18\\
72.76	0\\
72.77	0\\
72.78	0\\
72.79	1.73472347597681e-18\\
72.8	1.73472347597681e-18\\
72.81	0\\
72.82	0\\
72.83	0\\
72.84	0\\
72.85	0\\
72.86	0\\
72.87	1.73472347597681e-18\\
72.88	1.73472347597681e-18\\
72.89	0\\
72.9	0\\
72.91	0\\
72.92	0\\
72.93	0\\
72.94	0\\
72.95	0\\
72.96	1.73472347597681e-18\\
72.97	0\\
72.98	0\\
72.99	0\\
73	0\\
73.01	1.73472347597681e-18\\
73.02	1.73472347597681e-18\\
73.03	0\\
73.04	1.73472347597681e-18\\
73.05	1.73472347597681e-18\\
73.06	0\\
73.07	0\\
73.08	0\\
73.09	0\\
73.1	0\\
73.11	0\\
73.12	0\\
73.13	1.73472347597681e-18\\
73.14	1.73472347597681e-18\\
73.15	0\\
73.16	0\\
73.17	0\\
73.18	1.73472347597681e-18\\
73.19	1.73472347597681e-18\\
73.2	0\\
73.21	0\\
73.22	0\\
73.23	0\\
73.24	0\\
73.25	1.73472347597681e-18\\
73.26	0\\
73.27	0\\
73.28	1.73472347597681e-18\\
73.29	1.73472347597681e-18\\
73.3	0\\
73.31	0\\
73.32	0\\
73.33	1.73472347597681e-18\\
73.34	0\\
73.35	0\\
73.36	1.73472347597681e-18\\
73.37	0\\
73.38	0\\
73.39	0\\
73.4	1.73472347597681e-18\\
73.41	1.73472347597681e-18\\
73.42	0\\
73.43	0\\
73.44	0\\
73.45	0\\
73.46	0\\
73.47	0\\
73.48	0\\
73.49	1.73472347597681e-18\\
73.5	0\\
73.51	0\\
73.52	0\\
73.53	0\\
73.54	1.73472347597681e-18\\
73.55	0\\
73.56	0\\
73.57	0\\
73.58	0\\
73.59	0\\
73.6	0\\
73.61	0\\
73.62	0\\
73.63	1.73472347597681e-18\\
73.64	1.73472347597681e-18\\
73.65	0\\
73.66	1.73472347597681e-18\\
73.67	1.73472347597681e-18\\
73.68	0\\
73.69	0\\
73.7	0\\
73.71	0\\
73.72	1.73472347597681e-18\\
73.73	0\\
73.74	0\\
73.75	0\\
73.76	0\\
73.77	1.73472347597681e-18\\
73.78	0\\
73.79	0\\
73.8	1.73472347597681e-18\\
73.81	0\\
73.82	0\\
73.83	0\\
73.84	0\\
73.85	0\\
73.86	1.73472347597681e-18\\
73.87	1.73472347597681e-18\\
73.88	0\\
73.89	0\\
73.9	0\\
73.91	0\\
73.92	1.73472347597681e-18\\
73.93	1.73472347597681e-18\\
73.94	0\\
73.95	0\\
73.96	0\\
73.97	1.73472347597681e-18\\
73.98	0\\
73.99	0\\
74	0\\
74.01	0\\
74.02	0\\
74.03	1.73472347597681e-18\\
74.04	1.73472347597681e-18\\
74.05	0\\
74.06	0\\
74.07	1.73472347597681e-18\\
74.08	1.73472347597681e-18\\
74.09	0\\
74.1	0\\
74.11	0\\
74.12	0\\
74.13	1.73472347597681e-18\\
74.14	0\\
74.15	0\\
74.16	0\\
74.17	0\\
74.18	0\\
74.19	0\\
74.2	1.73472347597681e-18\\
74.21	0\\
74.22	0\\
74.23	0\\
74.24	0\\
74.25	0\\
74.26	1.73472347597681e-18\\
74.27	1.73472347597681e-18\\
74.28	0\\
74.29	1.73472347597681e-18\\
74.3	1.73472347597681e-18\\
74.31	0\\
74.32	0\\
74.33	0\\
74.34	0\\
74.35	0\\
74.36	0\\
74.37	0\\
74.38	1.73472347597681e-18\\
74.39	0\\
74.4	0\\
74.41	0\\
74.42	0\\
74.43	1.73472347597681e-18\\
74.44	0\\
74.45	0\\
74.46	0\\
74.47	0\\
74.48	0\\
74.49	1.73472347597681e-18\\
74.5	0\\
74.51	1.73472347597681e-18\\
74.52	0\\
74.53	1.73472347597681e-18\\
74.54	0\\
74.55	0\\
74.56	1.73472347597681e-18\\
74.57	0\\
74.58	0\\
74.59	0\\
74.6	0\\
74.61	0\\
74.62	0\\
74.63	0\\
74.64	0\\
74.65	0\\
74.66	0\\
74.67	1.73472347597681e-18\\
74.68	0\\
74.69	0\\
74.7	0\\
74.71	0\\
74.72	1.73472347597681e-18\\
74.73	0\\
74.74	0\\
74.75	0\\
74.76	0\\
74.77	0\\
74.78	0\\
74.79	0\\
74.8	0\\
74.81	0\\
74.82	0\\
74.83	0\\
74.84	0\\
74.85	0\\
74.86	0\\
74.87	1.73472347597681e-18\\
74.88	0\\
74.89	0\\
74.9	0\\
74.91	1.73472347597681e-18\\
74.92	0\\
74.93	0\\
74.94	0\\
74.95	1.73472347597681e-18\\
74.96	0\\
74.97	1.73472347597681e-18\\
74.98	0\\
74.99	1.73472347597681e-18\\
75	0\\
75.01	0\\
75.02	1.73472347597681e-18\\
75.03	0\\
75.04	0\\
75.05	0\\
75.06	0\\
75.07	1.73472347597681e-18\\
75.08	1.73472347597681e-18\\
75.09	0\\
75.1	0\\
75.11	0\\
75.12	0\\
75.13	1.73472347597681e-18\\
75.14	0\\
75.15	0\\
75.16	0\\
75.17	0\\
75.18	1.73472347597681e-18\\
75.19	0\\
75.2	0\\
75.21	0\\
75.22	0\\
75.23	0\\
75.24	1.73472347597681e-18\\
75.25	0\\
75.26	0\\
75.27	0\\
75.28	0\\
75.29	1.73472347597681e-18\\
75.3	0\\
75.31	0\\
75.32	0\\
75.33	0\\
75.34	0\\
75.35	0\\
75.36	0\\
75.37	0\\
75.38	0\\
75.39	0\\
75.4	1.73472347597681e-18\\
75.41	0\\
75.42	0\\
75.43	0\\
75.44	1.73472347597681e-18\\
75.45	0\\
75.46	0\\
75.47	0\\
75.48	0\\
75.49	0\\
75.5	0\\
75.51	0\\
75.52	1.73472347597681e-18\\
75.53	0\\
75.54	1.73472347597681e-18\\
75.55	0\\
75.56	0\\
75.57	0\\
75.58	0\\
75.59	0\\
75.6	1.73472347597681e-18\\
75.61	0\\
75.62	0\\
75.63	0\\
75.64	0\\
75.65	0\\
75.66	0\\
75.67	0\\
75.68	0\\
75.69	0\\
75.7	1.73472347597681e-18\\
75.71	0\\
75.72	1.73472347597681e-18\\
75.73	1.73472347597681e-18\\
75.74	0\\
75.75	0\\
75.76	0\\
75.77	0\\
75.78	0\\
75.79	1.73472347597681e-18\\
75.8	1.73472347597681e-18\\
75.81	1.73472347597681e-18\\
75.82	0\\
75.83	1.73472347597681e-18\\
75.84	1.73472347597681e-18\\
75.85	1.73472347597681e-18\\
75.86	0\\
75.87	0\\
75.88	0\\
75.89	0\\
75.9	1.73472347597681e-18\\
75.91	0\\
75.92	0\\
75.93	1.73472347597681e-18\\
75.94	0\\
75.95	0\\
75.96	0\\
75.97	0\\
75.98	0\\
75.99	0\\
76	0\\
76.01	0\\
76.02	0\\
76.03	0\\
76.04	0\\
76.05	0\\
76.06	0\\
76.07	0\\
76.08	1.73472347597681e-18\\
76.09	0\\
76.1	1.73472347597681e-18\\
76.11	0\\
76.12	1.73472347597681e-18\\
76.13	0\\
76.14	0\\
76.15	0\\
76.16	0\\
76.17	0\\
76.18	0\\
76.19	0\\
76.2	0\\
76.21	0\\
76.22	0\\
76.23	0\\
76.24	0\\
76.25	0\\
76.26	0\\
76.27	1.73472347597681e-18\\
76.28	0\\
76.29	0\\
76.3	0\\
76.31	1.73472347597681e-18\\
76.32	0\\
76.33	0\\
76.34	0\\
76.35	0\\
76.36	1.73472347597681e-18\\
76.37	0\\
76.38	0\\
76.39	0\\
76.4	1.73472347597681e-18\\
76.41	0\\
76.42	0\\
76.43	0\\
76.44	0\\
76.45	1.73472347597681e-18\\
76.46	0\\
76.47	0\\
76.48	0\\
76.49	0\\
76.5	0\\
76.51	0\\
76.52	1.73472347597681e-18\\
76.53	0\\
76.54	0\\
76.55	0\\
76.56	0\\
76.57	0\\
76.58	0\\
76.59	0\\
76.6	0\\
76.61	1.73472347597681e-18\\
76.62	0\\
76.63	0\\
76.64	0\\
76.65	1.73472347597681e-18\\
76.66	1.73472347597681e-18\\
76.67	1.73472347597681e-18\\
76.68	1.73472347597681e-18\\
76.69	0\\
76.7	0\\
76.71	1.73472347597681e-18\\
76.72	1.73472347597681e-18\\
76.73	0\\
76.74	0\\
76.75	0\\
76.76	1.73472347597681e-18\\
76.77	0\\
76.78	0\\
76.79	0\\
76.8	1.73472347597681e-18\\
76.81	0\\
76.82	0\\
76.83	0\\
76.84	0\\
76.85	1.73472347597681e-18\\
76.86	1.73472347597681e-18\\
76.87	1.73472347597681e-18\\
76.88	1.73472347597681e-18\\
76.89	0\\
76.9	0\\
76.91	0\\
76.92	0\\
76.93	1.73472347597681e-18\\
76.94	0\\
76.95	0\\
76.96	0\\
76.97	1.73472347597681e-18\\
76.98	1.73472347597681e-18\\
76.99	0\\
77	0\\
77.01	0\\
77.02	0\\
77.03	0\\
77.04	0\\
77.05	0\\
77.06	0\\
77.07	0\\
77.08	0\\
77.09	0\\
77.1	0\\
77.11	0\\
77.12	0\\
77.13	0\\
77.14	0\\
77.15	1.73472347597681e-18\\
77.16	0\\
77.17	0\\
77.18	1.73472347597681e-18\\
77.19	0\\
77.2	0\\
77.21	0\\
77.22	1.73472347597681e-18\\
77.23	0\\
77.24	0\\
77.25	0\\
77.26	0\\
77.27	0\\
77.28	0\\
77.29	0\\
77.3	1.73472347597681e-18\\
77.31	0\\
77.32	1.73472347597681e-18\\
77.33	0\\
77.34	1.73472347597681e-18\\
77.35	0\\
77.36	0\\
77.37	1.73472347597681e-18\\
77.38	0\\
77.39	0\\
77.4	0\\
77.41	0\\
77.42	0\\
77.43	1.73472347597681e-18\\
77.44	0\\
77.45	0\\
77.46	0\\
77.47	0\\
77.48	0\\
77.49	0\\
77.5	1.73472347597681e-18\\
77.51	0\\
77.52	0\\
77.53	1.73472347597681e-18\\
77.54	0\\
77.55	0\\
77.56	0\\
77.57	0\\
77.58	0\\
77.59	0\\
77.6	0\\
77.61	1.73472347597681e-18\\
77.62	1.73472347597681e-18\\
77.63	0\\
77.64	0\\
77.65	0\\
77.66	1.73472347597681e-18\\
77.67	0\\
77.68	0\\
77.69	0\\
77.7	1.73472347597681e-18\\
77.71	0\\
77.72	1.73472347597681e-18\\
77.73	0\\
77.74	0\\
77.75	0\\
77.76	0\\
77.77	0\\
77.78	0\\
77.79	0\\
77.8	0\\
77.81	0\\
77.82	0\\
77.83	1.73472347597681e-18\\
77.84	1.73472347597681e-18\\
77.85	0\\
77.86	0\\
77.87	0\\
77.88	0\\
77.89	1.73472347597681e-18\\
77.9	0\\
77.91	0\\
77.92	0\\
77.93	0\\
77.94	0\\
77.95	0\\
77.96	0\\
77.97	0\\
77.98	1.73472347597681e-18\\
77.99	1.73472347597681e-18\\
78	0\\
78.01	0\\
78.02	0\\
78.03	0\\
78.04	1.73472347597681e-18\\
78.05	0\\
78.06	0\\
78.07	0\\
78.08	0\\
78.09	0\\
78.1	0\\
78.11	0\\
78.12	0\\
78.13	1.73472347597681e-18\\
78.14	0\\
78.15	0\\
78.16	0\\
78.17	0\\
78.18	0\\
78.19	0\\
78.2	0\\
78.21	0\\
78.22	1.73472347597681e-18\\
78.23	0\\
78.24	0\\
78.25	0\\
78.26	0\\
78.27	0\\
78.28	0\\
78.29	1.73472347597681e-18\\
78.3	0\\
78.31	0\\
78.32	1.73472347597681e-18\\
78.33	0\\
78.34	1.73472347597681e-18\\
78.35	0\\
78.36	0\\
78.37	0\\
78.38	0\\
78.39	1.73472347597681e-18\\
78.4	1.73472347597681e-18\\
78.41	0\\
78.42	0\\
78.43	0\\
78.44	0\\
78.45	0\\
78.46	1.73472347597681e-18\\
78.47	0\\
78.48	0\\
78.49	1.73472347597681e-18\\
78.5	0\\
78.51	1.73472347597681e-18\\
78.52	1.73472347597681e-18\\
78.53	0\\
78.54	1.73472347597681e-18\\
78.55	0\\
78.56	0\\
78.57	1.73472347597681e-18\\
78.58	1.73472347597681e-18\\
78.59	0\\
78.6	1.73472347597681e-18\\
78.61	0\\
78.62	0\\
78.63	0\\
78.64	0\\
78.65	0\\
78.66	0\\
78.67	0\\
78.68	0\\
78.69	1.73472347597681e-18\\
78.7	1.73472347597681e-18\\
78.71	0\\
78.72	0\\
78.73	0\\
78.74	0\\
78.75	0\\
78.76	1.73472347597681e-18\\
78.77	1.73472347597681e-18\\
78.78	1.73472347597681e-18\\
78.79	0\\
78.8	0\\
78.81	0\\
78.82	1.73472347597681e-18\\
78.83	1.73472347597681e-18\\
78.84	0\\
78.85	0\\
78.86	0\\
78.87	0\\
78.88	0\\
78.89	0\\
78.9	0\\
78.91	1.73472347597681e-18\\
78.92	0\\
78.93	1.73472347597681e-18\\
78.94	1.73472347597681e-18\\
78.95	0\\
78.96	0\\
78.97	0\\
78.98	0\\
78.99	1.73472347597681e-18\\
79	0\\
79.01	0\\
79.02	0\\
79.03	0\\
79.04	0\\
79.05	0\\
79.06	1.73472347597681e-18\\
79.07	1.73472347597681e-18\\
79.08	0\\
79.09	1.73472347597681e-18\\
79.1	0\\
79.11	0\\
79.12	0\\
79.13	1.73472347597681e-18\\
79.14	1.73472347597681e-18\\
79.15	0\\
79.16	0\\
79.17	0\\
79.18	0\\
79.19	0\\
79.2	0\\
79.21	0\\
79.22	1.73472347597681e-18\\
79.23	0\\
79.24	1.73472347597681e-18\\
79.25	1.73472347597681e-18\\
79.26	0\\
79.27	0\\
79.28	1.73472347597681e-18\\
79.29	0\\
79.3	1.73472347597681e-18\\
79.31	0\\
79.32	0\\
79.33	0\\
79.34	0\\
79.35	1.73472347597681e-18\\
79.36	0\\
79.37	0\\
79.38	0\\
79.39	1.73472347597681e-18\\
79.4	0\\
79.41	0\\
79.42	0\\
79.43	0\\
79.44	0\\
79.45	0\\
79.46	0\\
79.47	0\\
79.48	1.73472347597681e-18\\
79.49	0\\
79.5	0\\
79.51	0\\
79.52	1.73472347597681e-18\\
79.53	0\\
79.54	0\\
79.55	1.73472347597681e-18\\
79.56	0\\
79.57	0\\
79.58	0\\
79.59	0\\
79.6	0\\
79.61	0\\
79.62	1.73472347597681e-18\\
79.63	0\\
79.64	0\\
79.65	1.73472347597681e-18\\
79.66	0\\
79.67	0\\
79.68	0\\
79.69	0\\
79.7	1.73472347597681e-18\\
79.71	1.73472347597681e-18\\
79.72	0\\
79.73	0\\
79.74	1.73472347597681e-18\\
79.75	0\\
79.76	0\\
79.77	0\\
79.78	0\\
79.79	0\\
79.8	0\\
79.81	0\\
79.82	0\\
79.83	0\\
79.84	0\\
79.85	1.73472347597681e-18\\
79.86	1.73472347597681e-18\\
79.87	1.73472347597681e-18\\
79.88	0\\
79.89	0\\
79.9	1.73472347597681e-18\\
79.91	1.73472347597681e-18\\
79.92	0\\
79.93	0\\
79.94	0\\
79.95	0\\
79.96	0\\
79.97	0\\
79.98	0\\
79.99	1.73472347597681e-18\\
80	0\\
80.01	0\\
};
\addplot [color=green,solid]
  table[row sep=crcr]{%
80.01	0\\
80.02	0\\
80.03	0\\
80.04	1.73472347597681e-18\\
80.05	0\\
80.06	1.73472347597681e-18\\
80.07	0\\
80.08	0\\
80.09	1.73472347597681e-18\\
80.1	1.73472347597681e-18\\
80.11	0\\
80.12	0\\
80.13	0\\
80.14	0\\
80.15	0\\
80.16	1.73472347597681e-18\\
80.17	0\\
80.18	1.73472347597681e-18\\
80.19	0\\
80.2	1.73472347597681e-18\\
80.21	0\\
80.22	0\\
80.23	0\\
80.24	1.73472347597681e-18\\
80.25	0\\
80.26	0\\
80.27	1.73472347597681e-18\\
80.28	0\\
80.29	1.73472347597681e-18\\
80.3	0\\
80.31	0\\
80.32	0\\
80.33	0\\
80.34	0\\
80.35	0\\
80.36	1.73472347597681e-18\\
80.37	1.73472347597681e-18\\
80.38	0\\
80.39	1.73472347597681e-18\\
80.4	0\\
80.41	1.73472347597681e-18\\
80.42	0\\
80.43	1.73472347597681e-18\\
80.44	1.73472347597681e-18\\
80.45	0\\
80.46	0\\
80.47	0\\
80.48	0\\
80.49	1.73472347597681e-18\\
80.5	0\\
80.51	0\\
80.52	0\\
80.53	0\\
80.54	1.73472347597681e-18\\
80.55	0\\
80.56	0\\
80.57	1.73472347597681e-18\\
80.58	1.73472347597681e-18\\
80.59	0\\
80.6	1.73472347597681e-18\\
80.61	0\\
80.62	1.73472347597681e-18\\
80.63	0\\
80.64	1.73472347597681e-18\\
80.65	0\\
80.66	0\\
80.67	0\\
80.68	0\\
80.69	1.73472347597681e-18\\
80.7	0\\
80.71	1.73472347597681e-18\\
80.72	1.73472347597681e-18\\
80.73	0\\
80.74	1.73472347597681e-18\\
80.75	1.73472347597681e-18\\
80.76	0\\
80.77	0\\
80.78	0\\
80.79	0\\
80.8	0\\
80.81	0\\
80.82	0\\
80.83	0\\
80.84	1.73472347597681e-18\\
80.85	0\\
80.86	0\\
80.87	0\\
80.88	0\\
80.89	0\\
80.9	0\\
80.91	0\\
80.92	1.73472347597681e-18\\
80.93	0\\
80.94	1.73472347597681e-18\\
80.95	0\\
80.96	0\\
80.97	0\\
80.98	0\\
80.99	1.73472347597681e-18\\
81	1.73472347597681e-18\\
81.01	0\\
81.02	1.73472347597681e-18\\
81.03	0\\
81.04	0\\
81.05	1.73472347597681e-18\\
81.06	1.73472347597681e-18\\
81.07	0\\
81.08	0\\
81.09	0\\
81.1	1.73472347597681e-18\\
81.11	0\\
81.12	0\\
81.13	0\\
81.14	0\\
81.15	0\\
81.16	1.73472347597681e-18\\
81.17	1.73472347597681e-18\\
81.18	0\\
81.19	0\\
81.2	0\\
81.21	0\\
81.22	0\\
81.23	0\\
81.24	0\\
81.25	1.73472347597681e-18\\
81.26	0\\
81.27	0\\
81.28	0\\
81.29	0\\
81.3	0\\
81.31	1.73472347597681e-18\\
81.32	1.73472347597681e-18\\
81.33	0\\
81.34	0\\
81.35	1.73472347597681e-18\\
81.36	1.73472347597681e-18\\
81.37	0\\
81.38	0\\
81.39	1.73472347597681e-18\\
81.4	0\\
81.41	0\\
81.42	0\\
81.43	0\\
81.44	0\\
81.45	0\\
81.46	0\\
81.47	0\\
81.48	1.73472347597681e-18\\
81.49	0\\
81.5	0\\
81.51	0\\
81.52	0\\
81.53	0\\
81.54	0\\
81.55	0\\
81.56	0\\
81.57	0\\
81.58	0\\
81.59	0\\
81.6	1.73472347597681e-18\\
81.61	0\\
81.62	1.73472347597681e-18\\
81.63	0\\
81.64	0\\
81.65	0\\
81.66	0\\
81.67	1.73472347597681e-18\\
81.68	1.73472347597681e-18\\
81.69	0\\
81.7	0\\
81.71	0\\
81.72	1.73472347597681e-18\\
81.73	0\\
81.74	0\\
81.75	0\\
81.76	0\\
81.77	0\\
81.78	0\\
81.79	0\\
81.8	0\\
81.81	1.73472347597681e-18\\
81.82	0\\
81.83	0\\
81.84	0\\
81.85	0\\
81.86	0\\
81.87	1.73472347597681e-18\\
81.88	0\\
81.89	0\\
81.9	1.73472347597681e-18\\
81.91	0\\
81.92	0\\
81.93	1.73472347597681e-18\\
81.94	0\\
81.95	0\\
81.96	1.73472347597681e-18\\
81.97	1.73472347597681e-18\\
81.98	1.73472347597681e-18\\
81.99	0\\
82	0\\
82.01	1.73472347597681e-18\\
82.02	1.73472347597681e-18\\
82.03	0\\
82.04	1.73472347597681e-18\\
82.05	0\\
82.06	0\\
82.07	0\\
82.08	0\\
82.09	0\\
82.1	0\\
82.11	0\\
82.12	0\\
82.13	1.73472347597681e-18\\
82.14	0\\
82.15	1.73472347597681e-18\\
82.16	1.73472347597681e-18\\
82.17	0\\
82.18	0\\
82.19	0\\
82.2	1.73472347597681e-18\\
82.21	0\\
82.22	0\\
82.23	1.73472347597681e-18\\
82.24	1.73472347597681e-18\\
82.25	1.73472347597681e-18\\
82.26	0\\
82.27	0\\
82.28	0\\
82.29	0\\
82.3	1.73472347597681e-18\\
82.31	0\\
82.32	0\\
82.33	0\\
82.34	0\\
82.35	0\\
82.36	0\\
82.37	0\\
82.38	0\\
82.39	1.73472347597681e-18\\
82.4	0\\
82.41	0\\
82.42	0\\
82.43	1.73472347597681e-18\\
82.44	0\\
82.45	0\\
82.46	0\\
82.47	0\\
82.48	0\\
82.49	0\\
82.5	0\\
82.51	0\\
82.52	0\\
82.53	0\\
82.54	0\\
82.55	0\\
82.56	0\\
82.57	0\\
82.58	1.73472347597681e-18\\
82.59	1.73472347597681e-18\\
82.6	0\\
82.61	0\\
82.62	0\\
82.63	1.73472347597681e-18\\
82.64	0\\
82.65	0\\
82.66	0\\
82.67	1.73472347597681e-18\\
82.68	0\\
82.69	1.73472347597681e-18\\
82.7	0\\
82.71	0\\
82.72	0\\
82.73	0\\
82.74	0\\
82.75	0\\
82.76	0\\
82.77	0\\
82.78	1.73472347597681e-18\\
82.79	0\\
82.8	0\\
82.81	0\\
82.82	1.73472347597681e-18\\
82.83	1.73472347597681e-18\\
82.84	1.73472347597681e-18\\
82.85	1.73472347597681e-18\\
82.86	0\\
82.87	0\\
82.88	0\\
82.89	0\\
82.9	0\\
82.91	1.73472347597681e-18\\
82.92	0\\
82.93	1.73472347597681e-18\\
82.94	0\\
82.95	0\\
82.96	0\\
82.97	1.73472347597681e-18\\
82.98	0\\
82.99	0\\
83	0\\
83.01	0\\
83.02	0\\
83.03	0\\
83.04	0\\
83.05	0\\
83.06	1.73472347597681e-18\\
83.07	1.73472347597681e-18\\
83.08	1.73472347597681e-18\\
83.09	0\\
83.1	1.73472347597681e-18\\
83.11	0\\
83.12	1.73472347597681e-18\\
83.13	0\\
83.14	0\\
83.15	0\\
83.16	1.73472347597681e-18\\
83.17	0\\
83.18	0\\
83.19	0\\
83.2	1.73472347597681e-18\\
83.21	1.73472347597681e-18\\
83.22	0\\
83.23	0\\
83.24	0\\
83.25	0\\
83.26	0\\
83.27	1.73472347597681e-18\\
83.28	0\\
83.29	1.73472347597681e-18\\
83.3	0\\
83.31	0\\
83.32	0\\
83.33	0\\
83.34	0\\
83.35	0\\
83.36	1.73472347597681e-18\\
83.37	0\\
83.38	0\\
83.39	0\\
83.4	0\\
83.41	1.73472347597681e-18\\
83.42	0\\
83.43	0\\
83.44	1.73472347597681e-18\\
83.45	0\\
83.46	0\\
83.47	0\\
83.48	1.73472347597681e-18\\
83.49	0\\
83.5	1.73472347597681e-18\\
83.51	1.73472347597681e-18\\
83.52	0\\
83.53	0\\
83.54	1.73472347597681e-18\\
83.55	0\\
83.56	0\\
83.57	0\\
83.58	0\\
83.59	0\\
83.6	0\\
83.61	1.73472347597681e-18\\
83.62	1.73472347597681e-18\\
83.63	0\\
83.64	0\\
83.65	0\\
83.66	0\\
83.67	1.73472347597681e-18\\
83.68	1.73472347597681e-18\\
83.69	0\\
83.7	0\\
83.71	0\\
83.72	0\\
83.73	1.73472347597681e-18\\
83.74	0\\
83.75	1.73472347597681e-18\\
83.76	1.73472347597681e-18\\
83.77	1.73472347597681e-18\\
83.78	0\\
83.79	0\\
83.8	0\\
83.81	0\\
83.82	0\\
83.83	0\\
83.84	0\\
83.85	1.73472347597681e-18\\
83.86	0\\
83.87	0\\
83.88	1.73472347597681e-18\\
83.89	0\\
83.9	0\\
83.91	0\\
83.92	0\\
83.93	0\\
83.94	0\\
83.95	1.73472347597681e-18\\
83.96	1.73472347597681e-18\\
83.97	0\\
83.98	0\\
83.99	0\\
84	0\\
84.01	0\\
84.02	0\\
84.03	1.73472347597681e-18\\
84.04	0\\
84.05	0\\
84.06	1.73472347597681e-18\\
84.07	0\\
84.08	0\\
84.09	0\\
84.1	0\\
84.11	1.73472347597681e-18\\
84.12	0\\
84.13	0\\
84.14	0\\
84.15	1.73472347597681e-18\\
84.16	0\\
84.17	0\\
84.18	0\\
84.19	0\\
84.2	0\\
84.21	0\\
84.22	0\\
84.23	0\\
84.24	0\\
84.25	0\\
84.26	0\\
84.27	0\\
84.28	0\\
84.29	0\\
84.3	0\\
84.31	0\\
84.32	0\\
84.33	1.73472347597681e-18\\
84.34	0\\
84.35	0\\
84.36	1.73472347597681e-18\\
84.37	0\\
84.38	1.73472347597681e-18\\
84.39	0\\
84.4	0\\
84.41	1.73472347597681e-18\\
84.42	0\\
84.43	0\\
84.44	0\\
84.45	0\\
84.46	1.73472347597681e-18\\
84.47	0\\
84.48	0\\
84.49	0\\
84.5	0\\
84.51	1.73472347597681e-18\\
84.52	0\\
84.53	0\\
84.54	0\\
84.55	1.73472347597681e-18\\
84.56	0\\
84.57	0\\
84.58	1.73472347597681e-18\\
84.59	0\\
84.6	1.73472347597681e-18\\
84.61	1.73472347597681e-18\\
84.62	1.73472347597681e-18\\
84.63	0\\
84.64	1.73472347597681e-18\\
84.65	0\\
84.66	0\\
84.67	0\\
84.68	0\\
84.69	0\\
84.7	0\\
84.71	0\\
84.72	0\\
84.73	1.73472347597681e-18\\
84.74	0\\
84.75	0\\
84.76	1.73472347597681e-18\\
84.77	0\\
84.78	0\\
84.79	0\\
84.8	0\\
84.81	0\\
84.82	0\\
84.83	0\\
84.84	1.73472347597681e-18\\
84.85	0\\
84.86	0\\
84.87	0\\
84.88	0\\
84.89	0\\
84.9	0\\
84.91	0\\
84.92	0\\
84.93	1.73472347597681e-18\\
84.94	0\\
84.95	1.73472347597681e-18\\
84.96	1.73472347597681e-18\\
84.97	0\\
84.98	1.73472347597681e-18\\
84.99	0\\
85	0\\
85.01	0\\
85.02	0\\
85.03	0\\
85.04	0\\
85.05	0\\
85.06	0\\
85.07	0\\
85.08	0\\
85.09	1.73472347597681e-18\\
85.1	0\\
85.11	0\\
85.12	0\\
85.13	1.73472347597681e-18\\
85.14	0\\
85.15	0\\
85.16	0\\
85.17	0\\
85.18	0\\
85.19	0\\
85.2	0\\
85.21	1.73472347597681e-18\\
85.22	0\\
85.23	1.73472347597681e-18\\
85.24	0\\
85.25	0\\
85.26	0\\
85.27	1.73472347597681e-18\\
85.28	0\\
85.29	0\\
85.3	0\\
85.31	0\\
85.32	1.73472347597681e-18\\
85.33	0\\
85.34	1.73472347597681e-18\\
85.35	0\\
85.36	0\\
85.37	0\\
85.38	0\\
85.39	0\\
85.4	1.73472347597681e-18\\
85.41	0\\
85.42	0\\
85.43	0\\
85.44	0\\
85.45	1.73472347597681e-18\\
85.46	0\\
85.47	0\\
85.48	0\\
85.49	0\\
85.5	0\\
85.51	0\\
85.52	0\\
85.53	0\\
85.54	1.73472347597681e-18\\
85.55	0\\
85.56	0\\
85.57	1.73472347597681e-18\\
85.58	0\\
85.59	0\\
85.6	0\\
85.61	0\\
85.62	0\\
85.63	1.73472347597681e-18\\
85.64	0\\
85.65	0\\
85.66	0\\
85.67	0\\
85.68	0\\
85.69	1.73472347597681e-18\\
85.7	0\\
85.71	0\\
85.72	1.73472347597681e-18\\
85.73	0\\
85.74	0\\
85.75	0\\
85.76	0\\
85.77	1.73472347597681e-18\\
85.78	0\\
85.79	0\\
85.8	1.73472347597681e-18\\
85.81	0\\
85.82	0\\
85.83	0\\
85.84	0\\
85.85	0\\
85.86	1.73472347597681e-18\\
85.87	0\\
85.88	0\\
85.89	1.73472347597681e-18\\
85.9	1.73472347597681e-18\\
85.91	0\\
85.92	1.73472347597681e-18\\
85.93	0\\
85.94	1.73472347597681e-18\\
85.95	0\\
85.96	0\\
85.97	0\\
85.98	0\\
85.99	0\\
86	1.73472347597681e-18\\
86.01	1.73472347597681e-18\\
86.02	1.73472347597681e-18\\
86.03	0\\
86.04	0\\
86.05	0\\
86.06	0\\
86.07	1.73472347597681e-18\\
86.08	0\\
86.09	0\\
86.1	0\\
86.11	0\\
86.12	0\\
86.13	0\\
86.14	0\\
86.15	0\\
86.16	0\\
86.17	0\\
86.18	0\\
86.19	0\\
86.2	0\\
86.21	0\\
86.22	0\\
86.23	0\\
86.24	0\\
86.25	0\\
86.26	0\\
86.27	0\\
86.28	1.73472347597681e-18\\
86.29	0\\
86.3	1.73472347597681e-18\\
86.31	1.73472347597681e-18\\
86.32	0\\
86.33	1.73472347597681e-18\\
86.34	0\\
86.35	0\\
86.36	1.73472347597681e-18\\
86.37	0\\
86.38	0\\
86.39	1.73472347597681e-18\\
86.4	1.73472347597681e-18\\
86.41	0\\
86.42	0\\
86.43	0\\
86.44	1.73472347597681e-18\\
86.45	0\\
86.46	0\\
86.47	0\\
86.48	1.73472347597681e-18\\
86.49	0\\
86.5	0\\
86.51	0\\
86.52	0\\
86.53	0\\
86.54	0\\
86.55	0\\
86.56	0\\
86.57	0\\
86.58	0\\
86.59	0\\
86.6	0\\
86.61	0\\
86.62	0\\
86.63	0\\
86.64	1.73472347597681e-18\\
86.65	0\\
86.66	0\\
86.67	0\\
86.68	0\\
86.69	0\\
86.7	0\\
86.71	0\\
86.72	0\\
86.73	1.73472347597681e-18\\
86.74	0\\
86.75	0\\
86.76	0\\
86.77	1.73472347597681e-18\\
86.78	0\\
86.79	0\\
86.8	0\\
86.81	0\\
86.82	0\\
86.83	0\\
86.84	0\\
86.85	0\\
86.86	0\\
86.87	0\\
86.88	0\\
86.89	1.73472347597681e-18\\
86.9	0\\
86.91	0\\
86.92	1.73472347597681e-18\\
86.93	0\\
86.94	0\\
86.95	0\\
86.96	0\\
86.97	1.73472347597681e-18\\
86.98	0\\
86.99	0\\
87	0\\
87.01	0\\
87.02	0\\
87.03	1.73472347597681e-18\\
87.04	0\\
87.05	0\\
87.06	1.73472347597681e-18\\
87.07	1.73472347597681e-18\\
87.08	1.73472347597681e-18\\
87.09	1.73472347597681e-18\\
87.1	1.73472347597681e-18\\
87.11	0\\
87.12	0\\
87.13	0\\
87.14	1.73472347597681e-18\\
87.15	0\\
87.16	0\\
87.17	1.73472347597681e-18\\
87.18	1.73472347597681e-18\\
87.19	0\\
87.2	0\\
87.21	0\\
87.22	0\\
87.23	0\\
87.24	1.73472347597681e-18\\
87.25	1.73472347597681e-18\\
87.26	0\\
87.27	0\\
87.28	0\\
87.29	1.73472347597681e-18\\
87.3	0\\
87.31	0\\
87.32	0\\
87.33	0\\
87.34	0\\
87.35	0\\
87.36	0\\
87.37	0\\
87.38	0\\
87.39	1.73472347597681e-18\\
87.4	0\\
87.41	0\\
87.42	0\\
87.43	0\\
87.44	0\\
87.45	0\\
87.46	0\\
87.47	0\\
87.48	0\\
87.49	0\\
87.5	0\\
87.51	0\\
87.52	1.73472347597681e-18\\
87.53	1.73472347597681e-18\\
87.54	0\\
87.55	0\\
87.56	1.73472347597681e-18\\
87.57	0\\
87.58	0\\
87.59	0\\
87.6	0\\
87.61	0\\
87.62	0\\
87.63	0\\
87.64	0\\
87.65	0\\
87.66	0\\
87.67	0\\
87.68	1.73472347597681e-18\\
87.69	0\\
87.7	0\\
87.71	0\\
87.72	0\\
87.73	0\\
87.74	1.73472347597681e-18\\
87.75	1.73472347597681e-18\\
87.76	1.73472347597681e-18\\
87.77	0\\
87.78	1.73472347597681e-18\\
87.79	0\\
87.8	0\\
87.81	0\\
87.82	1.73472347597681e-18\\
87.83	1.73472347597681e-18\\
87.84	0\\
87.85	1.73472347597681e-18\\
87.86	0\\
87.87	0\\
87.88	0\\
87.89	0\\
87.9	0\\
87.91	0\\
87.92	1.73472347597681e-18\\
87.93	0\\
87.94	0\\
87.95	0\\
87.96	0\\
87.97	1.73472347597681e-18\\
87.98	0\\
87.99	0\\
88	0\\
88.01	1.73472347597681e-18\\
88.02	1.73472347597681e-18\\
88.03	0\\
88.04	1.73472347597681e-18\\
88.05	0\\
88.06	0\\
88.07	1.73472347597681e-18\\
88.08	1.73472347597681e-18\\
88.09	0\\
88.1	0\\
88.11	0\\
88.12	1.73472347597681e-18\\
88.13	0\\
88.14	0\\
88.15	1.73472347597681e-18\\
88.16	0\\
88.17	1.73472347597681e-18\\
88.18	0\\
88.19	0\\
88.2	0\\
88.21	0\\
88.22	0\\
88.23	0\\
88.24	0\\
88.25	0\\
88.26	0\\
88.27	0\\
88.28	0\\
88.29	1.73472347597681e-18\\
88.3	0\\
88.31	0\\
88.32	0\\
88.33	0\\
88.34	0\\
88.35	0\\
88.36	0\\
88.37	0\\
88.38	0\\
88.39	0\\
88.4	1.73472347597681e-18\\
88.41	0\\
88.42	0\\
88.43	0\\
88.44	1.73472347597681e-18\\
88.45	0\\
88.46	1.73472347597681e-18\\
88.47	1.73472347597681e-18\\
88.48	1.73472347597681e-18\\
88.49	1.73472347597681e-18\\
88.5	0\\
88.51	0\\
88.52	0\\
88.53	1.73472347597681e-18\\
88.54	0\\
88.55	0\\
88.56	1.73472347597681e-18\\
88.57	0\\
88.58	0\\
88.59	1.73472347597681e-18\\
88.6	0\\
88.61	0\\
88.62	1.73472347597681e-18\\
88.63	0\\
88.64	0\\
88.65	1.73472347597681e-18\\
88.66	1.73472347597681e-18\\
88.67	1.73472347597681e-18\\
88.68	0\\
88.69	0\\
88.7	0\\
88.71	0\\
88.72	0\\
88.73	1.73472347597681e-18\\
88.74	0\\
88.75	0\\
88.76	0\\
88.77	0\\
88.78	0\\
88.79	1.73472347597681e-18\\
88.8	0\\
88.81	0\\
88.82	1.73472347597681e-18\\
88.83	0\\
88.84	0\\
88.85	0\\
88.86	0\\
88.87	0\\
88.88	0\\
88.89	0\\
88.9	1.73472347597681e-18\\
88.91	0\\
88.92	0\\
88.93	0\\
88.94	0\\
88.95	0\\
88.96	0\\
88.97	0\\
88.98	0\\
88.99	1.73472347597681e-18\\
89	1.73472347597681e-18\\
89.01	1.73472347597681e-18\\
89.02	0\\
89.03	0\\
89.04	1.73472347597681e-18\\
89.05	0\\
89.06	0\\
89.07	0\\
89.08	0\\
89.09	0\\
89.1	0\\
89.11	0\\
89.12	0\\
89.13	1.73472347597681e-18\\
89.14	1.73472347597681e-18\\
89.15	0\\
89.16	0\\
89.17	1.73472347597681e-18\\
89.18	1.73472347597681e-18\\
89.19	0\\
89.2	0\\
89.21	0\\
89.22	1.73472347597681e-18\\
89.23	0\\
89.24	0\\
89.25	0\\
89.26	0\\
89.27	0\\
89.28	0\\
89.29	1.73472347597681e-18\\
89.3	0\\
89.31	0\\
89.32	0\\
89.33	0\\
89.34	0\\
89.35	1.73472347597681e-18\\
89.36	0\\
89.37	1.73472347597681e-18\\
89.38	0\\
89.39	0\\
89.4	1.73472347597681e-18\\
89.41	0\\
89.42	1.73472347597681e-18\\
89.43	0\\
89.44	0\\
89.45	0\\
89.46	1.73472347597681e-18\\
89.47	0\\
89.48	0\\
89.49	0\\
89.5	0\\
89.51	0\\
89.52	1.73472347597681e-18\\
89.53	0\\
89.54	0\\
89.55	1.73472347597681e-18\\
89.56	1.73472347597681e-18\\
89.57	0\\
89.58	0\\
89.59	0\\
89.6	1.73472347597681e-18\\
89.61	0\\
89.62	1.73472347597681e-18\\
89.63	1.73472347597681e-18\\
89.64	0\\
89.65	0\\
89.66	0\\
89.67	0\\
89.68	0\\
89.69	0\\
89.7	0\\
89.71	1.73472347597681e-18\\
89.72	1.73472347597681e-18\\
89.73	0\\
89.74	1.73472347597681e-18\\
89.75	0\\
89.76	0\\
89.77	1.73472347597681e-18\\
89.78	0\\
89.79	1.73472347597681e-18\\
89.8	1.73472347597681e-18\\
89.81	0\\
89.82	0\\
89.83	1.73472347597681e-18\\
89.84	1.73472347597681e-18\\
89.85	1.73472347597681e-18\\
89.86	1.73472347597681e-18\\
89.87	0\\
89.88	1.73472347597681e-18\\
89.89	0\\
89.9	0\\
89.91	0\\
89.92	1.73472347597681e-18\\
89.93	1.73472347597681e-18\\
89.94	0\\
89.95	0\\
89.96	0\\
89.97	1.73472347597681e-18\\
89.98	1.73472347597681e-18\\
89.99	0\\
90	0\\
90.01	0\\
90.02	1.73472347597681e-18\\
90.03	0\\
90.04	0\\
90.05	0\\
90.06	1.73472347597681e-18\\
90.07	0\\
90.08	1.73472347597681e-18\\
90.09	1.73472347597681e-18\\
90.1	1.73472347597681e-18\\
90.11	0\\
90.12	0\\
90.13	0\\
90.14	0\\
90.15	1.73472347597681e-18\\
90.16	0\\
90.17	1.73472347597681e-18\\
90.18	1.73472347597681e-18\\
90.19	0\\
90.2	0\\
90.21	0\\
90.22	0\\
90.23	0\\
90.24	0\\
90.25	0\\
90.26	0\\
90.27	0\\
90.28	0\\
90.29	0\\
90.3	0\\
90.31	0\\
90.32	0\\
90.33	0\\
90.34	1.73472347597681e-18\\
90.35	0\\
90.36	0\\
90.37	0\\
90.38	0\\
90.39	0\\
90.4	1.73472347597681e-18\\
90.41	0\\
90.42	0\\
90.43	1.73472347597681e-18\\
90.44	0\\
90.45	0\\
90.46	0\\
90.47	1.73472347597681e-18\\
90.48	0\\
90.49	1.73472347597681e-18\\
90.5	1.73472347597681e-18\\
90.51	0\\
90.52	0\\
90.53	0\\
90.54	0\\
90.55	0\\
90.56	0\\
90.57	0\\
90.58	0\\
90.59	0\\
90.6	0\\
90.61	1.73472347597681e-18\\
90.62	1.73472347597681e-18\\
90.63	0\\
90.64	0\\
90.65	0\\
90.66	0\\
90.67	0\\
90.68	0\\
90.69	0\\
90.7	1.73472347597681e-18\\
90.71	0\\
90.72	0\\
90.73	0\\
90.74	0\\
90.75	0\\
90.76	0\\
90.77	0\\
90.78	0\\
90.79	0\\
90.8	0\\
90.81	0\\
90.82	0\\
90.83	0\\
90.84	1.73472347597681e-18\\
90.85	0\\
90.86	1.73472347597681e-18\\
90.87	0\\
90.88	0\\
90.89	0\\
90.9	1.73472347597681e-18\\
90.91	0\\
90.92	0\\
90.93	0\\
90.94	0\\
90.95	0\\
90.96	0\\
90.97	1.73472347597681e-18\\
90.98	0\\
90.99	0\\
91	0\\
91.01	1.73472347597681e-18\\
91.02	1.73472347597681e-18\\
91.03	0\\
91.04	0\\
91.05	0\\
91.06	0\\
91.07	1.73472347597681e-18\\
91.08	0\\
91.09	0\\
91.1	1.73472347597681e-18\\
91.11	0\\
91.12	0\\
91.13	1.73472347597681e-18\\
91.14	0\\
91.15	0\\
91.16	0\\
91.17	0\\
91.18	0\\
91.19	0\\
91.2	0\\
91.21	0\\
91.22	1.73472347597681e-18\\
91.23	0\\
91.24	0\\
91.25	0\\
91.26	1.73472347597681e-18\\
91.27	0\\
91.28	0\\
91.29	1.73472347597681e-18\\
91.3	0\\
91.31	1.73472347597681e-18\\
91.32	0\\
91.33	0\\
91.34	0\\
91.35	0\\
91.36	0\\
91.37	0\\
91.38	0\\
91.39	1.73472347597681e-18\\
91.4	1.73472347597681e-18\\
91.41	1.73472347597681e-18\\
91.42	0\\
91.43	1.73472347597681e-18\\
91.44	0\\
91.45	0\\
91.46	0\\
91.47	1.73472347597681e-18\\
91.48	0\\
91.49	0\\
91.5	1.73472347597681e-18\\
91.51	1.73472347597681e-18\\
91.52	0\\
91.53	0\\
91.54	0\\
91.55	1.73472347597681e-18\\
91.56	0\\
91.57	0\\
91.58	0\\
91.59	0\\
91.6	1.73472347597681e-18\\
91.61	1.73472347597681e-18\\
91.62	0\\
91.63	0\\
91.64	0\\
91.65	0\\
91.66	0\\
91.67	0\\
91.68	0\\
91.69	0\\
91.7	1.73472347597681e-18\\
91.71	0\\
91.72	0\\
91.73	0\\
91.74	0\\
91.75	0\\
91.76	1.73472347597681e-18\\
91.77	0\\
91.78	0\\
91.79	0\\
91.8	0\\
91.81	0\\
91.82	0\\
91.83	1.73472347597681e-18\\
91.84	0\\
91.85	0\\
91.86	0\\
91.87	0\\
91.88	0\\
91.89	0\\
91.9	0\\
91.91	0\\
91.92	0\\
91.93	0\\
91.94	0\\
91.95	1.73472347597681e-18\\
91.96	0\\
91.97	1.73472347597681e-18\\
91.98	0\\
91.99	0\\
92	0\\
92.01	0\\
92.02	0\\
92.03	0\\
92.04	1.73472347597681e-18\\
92.05	0\\
92.06	1.73472347597681e-18\\
92.07	0\\
92.08	0\\
92.09	0\\
92.1	0\\
92.11	1.73472347597681e-18\\
92.12	0\\
92.13	0\\
92.14	0\\
92.15	0\\
92.16	1.73472347597681e-18\\
92.17	0\\
92.18	0\\
92.19	1.73472347597681e-18\\
92.2	1.73472347597681e-18\\
92.21	0\\
92.22	0\\
92.23	0\\
92.24	0\\
92.25	0\\
92.26	1.73472347597681e-18\\
92.27	0\\
92.28	0\\
92.29	0\\
92.3	1.73472347597681e-18\\
92.31	0\\
92.32	1.73472347597681e-18\\
92.33	1.73472347597681e-18\\
92.34	0\\
92.35	0\\
92.36	0\\
92.37	0\\
92.38	0\\
92.39	1.73472347597681e-18\\
92.4	0\\
92.41	1.73472347597681e-18\\
92.42	0\\
92.43	0\\
92.44	0\\
92.45	0\\
92.46	0\\
92.47	0\\
92.48	0\\
92.49	0\\
92.5	0\\
92.51	0\\
92.52	0\\
92.53	0\\
92.54	0\\
92.55	0\\
92.56	1.73472347597681e-18\\
92.57	1.73472347597681e-18\\
92.58	0\\
92.59	0\\
92.6	1.73472347597681e-18\\
92.61	1.73472347597681e-18\\
92.62	0\\
92.63	0\\
92.64	1.73472347597681e-18\\
92.65	0\\
92.66	0\\
92.67	0\\
92.68	1.73472347597681e-18\\
92.69	1.73472347597681e-18\\
92.7	1.73472347597681e-18\\
92.71	0\\
92.72	1.73472347597681e-18\\
92.73	1.73472347597681e-18\\
92.74	0\\
92.75	0\\
92.76	0\\
92.77	1.73472347597681e-18\\
92.78	0\\
92.79	0\\
92.8	0\\
92.81	1.73472347597681e-18\\
92.82	1.73472347597681e-18\\
92.83	0\\
92.84	1.73472347597681e-18\\
92.85	1.73472347597681e-18\\
92.86	0\\
92.87	1.73472347597681e-18\\
92.88	1.73472347597681e-18\\
92.89	1.73472347597681e-18\\
92.9	1.73472347597681e-18\\
92.91	0\\
92.92	1.73472347597681e-18\\
92.93	0\\
92.94	0\\
92.95	0\\
92.96	0\\
92.97	1.73472347597681e-18\\
92.98	1.73472347597681e-18\\
92.99	1.73472347597681e-18\\
93	0\\
93.01	0\\
93.02	0\\
93.03	1.73472347597681e-18\\
93.04	1.73472347597681e-18\\
93.05	0\\
93.06	0\\
93.07	0\\
93.08	0\\
93.09	0\\
93.1	0\\
93.11	1.73472347597681e-18\\
93.12	0\\
93.13	1.73472347597681e-18\\
93.14	1.73472347597681e-18\\
93.15	0\\
93.16	1.73472347597681e-18\\
93.17	1.73472347597681e-18\\
93.18	0\\
93.19	0\\
93.2	1.73472347597681e-18\\
93.21	0\\
93.22	0\\
93.23	0\\
93.24	0\\
93.25	0\\
93.26	0\\
93.27	0\\
93.28	0\\
93.29	1.73472347597681e-18\\
93.3	0\\
93.31	0\\
93.32	1.73472347597681e-18\\
93.33	1.73472347597681e-18\\
93.34	0\\
93.35	0\\
93.36	1.73472347597681e-18\\
93.37	0\\
93.38	0\\
93.39	0\\
93.4	0\\
93.41	0\\
93.42	0\\
93.43	0\\
93.44	0\\
93.45	0\\
93.46	0\\
93.47	0\\
93.48	0\\
93.49	1.73472347597681e-18\\
93.5	0\\
93.51	0\\
93.52	0\\
93.53	0\\
93.54	0\\
93.55	0\\
93.56	1.73472347597681e-18\\
93.57	0\\
93.58	0\\
93.59	0\\
93.6	0\\
93.61	0\\
93.62	0\\
93.63	0\\
93.64	0\\
93.65	1.73472347597681e-18\\
93.66	1.73472347597681e-18\\
93.67	1.73472347597681e-18\\
93.68	0\\
93.69	0\\
93.7	0\\
93.71	0\\
93.72	1.73472347597681e-18\\
93.73	0\\
93.74	0\\
93.75	0\\
93.76	0\\
93.77	1.73472347597681e-18\\
93.78	0\\
93.79	0\\
93.8	1.73472347597681e-18\\
93.81	0\\
93.82	1.73472347597681e-18\\
93.83	1.73472347597681e-18\\
93.84	0\\
93.85	1.73472347597681e-18\\
93.86	1.73472347597681e-18\\
93.87	0\\
93.88	1.73472347597681e-18\\
93.89	0\\
93.9	0\\
93.91	0\\
93.92	0\\
93.93	0\\
93.94	1.73472347597681e-18\\
93.95	0\\
93.96	0\\
93.97	0\\
93.98	0\\
93.99	1.73472347597681e-18\\
94	0\\
94.01	1.73472347597681e-18\\
94.02	0\\
94.03	1.73472347597681e-18\\
94.04	1.73472347597681e-18\\
94.05	0\\
94.06	0\\
94.07	0\\
94.08	0\\
94.09	0\\
94.1	0\\
94.11	0\\
94.12	1.73472347597681e-18\\
94.13	0\\
94.14	0\\
94.15	1.73472347597681e-18\\
94.16	0\\
94.17	1.73472347597681e-18\\
94.18	0\\
94.19	0\\
94.2	1.73472347597681e-18\\
94.21	1.73472347597681e-18\\
94.22	1.73472347597681e-18\\
94.23	1.73472347597681e-18\\
94.24	0\\
94.25	1.73472347597681e-18\\
94.26	0\\
94.27	1.73472347597681e-18\\
94.28	0\\
94.29	1.73472347597681e-18\\
94.3	1.73472347597681e-18\\
94.31	0\\
94.32	0\\
94.33	0\\
94.34	1.73472347597681e-18\\
94.35	0\\
94.36	0\\
94.37	0\\
94.38	0\\
94.39	1.73472347597681e-18\\
94.4	0\\
94.41	0\\
94.42	0\\
94.43	0\\
94.44	1.73472347597681e-18\\
94.45	1.73472347597681e-18\\
94.46	1.73472347597681e-18\\
94.47	0\\
94.48	0\\
94.49	1.73472347597681e-18\\
94.5	0\\
94.51	0\\
94.52	0\\
94.53	0\\
94.54	1.73472347597681e-18\\
94.55	0\\
94.56	0\\
94.57	0\\
94.58	0\\
94.59	1.73472347597681e-18\\
94.6	0\\
94.61	0\\
94.62	0\\
94.63	0\\
94.64	0\\
94.65	0\\
94.66	0\\
94.67	0\\
94.68	1.73472347597681e-18\\
94.69	0\\
94.7	0\\
94.71	1.73472347597681e-18\\
94.72	1.73472347597681e-18\\
94.73	1.73472347597681e-18\\
94.74	1.73472347597681e-18\\
94.75	1.73472347597681e-18\\
94.76	0\\
94.77	0\\
94.78	1.73472347597681e-18\\
94.79	0\\
94.8	0\\
94.81	0\\
94.82	0\\
94.83	0\\
94.84	0\\
94.85	1.73472347597681e-18\\
94.86	1.73472347597681e-18\\
94.87	1.73472347597681e-18\\
94.88	1.73472347597681e-18\\
94.89	0\\
94.9	0\\
94.91	0\\
94.92	0\\
94.93	0\\
94.94	0\\
94.95	1.73472347597681e-18\\
94.96	0\\
94.97	0\\
94.98	0\\
94.99	0\\
95	1.73472347597681e-18\\
95.01	1.73472347597681e-18\\
95.02	0\\
95.03	0\\
95.04	0\\
95.05	1.73472347597681e-18\\
95.06	1.73472347597681e-18\\
95.07	1.73472347597681e-18\\
95.08	0\\
95.09	0\\
95.1	0\\
95.11	0\\
95.12	0\\
95.13	1.73472347597681e-18\\
95.14	0\\
95.15	0\\
95.16	1.73472347597681e-18\\
95.17	0\\
95.18	0\\
95.19	0\\
95.2	0\\
95.21	0\\
95.22	1.73472347597681e-18\\
95.23	0\\
95.24	1.73472347597681e-18\\
95.25	1.73472347597681e-18\\
95.26	1.73472347597681e-18\\
95.27	1.73472347597681e-18\\
95.28	1.73472347597681e-18\\
95.29	0\\
95.3	1.73472347597681e-18\\
95.31	1.73472347597681e-18\\
95.32	0\\
95.33	1.73472347597681e-18\\
95.34	0\\
95.35	1.73472347597681e-18\\
95.36	1.73472347597681e-18\\
95.37	0\\
95.38	0\\
95.39	1.73472347597681e-18\\
95.4	0\\
95.41	1.73472347597681e-18\\
95.42	0\\
95.43	0\\
95.44	1.73472347597681e-18\\
95.45	1.73472347597681e-18\\
95.46	0\\
95.47	0\\
95.48	0\\
95.49	0\\
95.5	1.73472347597681e-18\\
95.51	0\\
95.52	1.73472347597681e-18\\
95.53	0\\
95.54	1.73472347597681e-18\\
95.55	0\\
95.56	0\\
95.57	0\\
95.58	1.73472347597681e-18\\
95.59	1.73472347597681e-18\\
95.6	0\\
95.61	1.73472347597681e-18\\
95.62	1.73472347597681e-18\\
95.63	1.73472347597681e-18\\
95.64	1.73472347597681e-18\\
95.65	0\\
95.66	0\\
95.67	0\\
95.68	0\\
95.69	1.73472347597681e-18\\
95.7	0\\
95.71	0\\
95.72	1.73472347597681e-18\\
95.73	0\\
95.74	0\\
95.75	1.73472347597681e-18\\
95.76	1.73472347597681e-18\\
95.77	0\\
95.78	0\\
95.79	1.73472347597681e-18\\
95.8	0\\
95.81	0\\
95.82	1.73472347597681e-18\\
95.83	1.73472347597681e-18\\
95.84	0\\
95.85	1.73472347597681e-18\\
95.86	0\\
95.87	1.73472347597681e-18\\
95.88	0\\
95.89	0\\
95.9	0\\
95.91	0\\
95.92	0\\
95.93	0\\
95.94	0\\
95.95	1.73472347597681e-18\\
95.96	0\\
95.97	0\\
95.98	0\\
95.99	0\\
96	0\\
96.01	1.73472347597681e-18\\
96.02	1.73472347597681e-18\\
96.03	0\\
96.04	1.73472347597681e-18\\
96.05	0\\
96.06	0\\
96.07	0\\
96.08	1.73472347597681e-18\\
96.09	1.73472347597681e-18\\
96.1	0\\
96.11	0\\
96.12	1.73472347597681e-18\\
96.13	1.73472347597681e-18\\
96.14	1.73472347597681e-18\\
96.15	0\\
96.16	1.73472347597681e-18\\
96.17	1.73472347597681e-18\\
96.18	1.73472347597681e-18\\
96.19	1.73472347597681e-18\\
96.2	0\\
96.21	0\\
96.22	0\\
96.23	0\\
96.24	0\\
96.25	1.73472347597681e-18\\
96.26	1.73472347597681e-18\\
96.27	0\\
96.28	0\\
96.29	0\\
96.3	0\\
96.31	0\\
96.32	1.73472347597681e-18\\
96.33	1.73472347597681e-18\\
96.34	1.73472347597681e-18\\
96.35	1.73472347597681e-18\\
96.36	0\\
96.37	0\\
96.38	1.73472347597681e-18\\
96.39	1.73472347597681e-18\\
96.4	0\\
96.41	1.73472347597681e-18\\
96.42	0\\
96.43	1.73472347597681e-18\\
96.44	0\\
96.45	0\\
96.46	1.73472347597681e-18\\
96.47	0\\
96.48	1.73472347597681e-18\\
96.49	0\\
96.5	0\\
96.51	1.73472347597681e-18\\
96.52	1.73472347597681e-18\\
96.53	0\\
96.54	1.73472347597681e-18\\
96.55	0\\
96.56	0\\
96.57	0\\
96.58	1.73472347597681e-18\\
96.59	0\\
96.6	0\\
96.61	0\\
96.62	0\\
96.63	0\\
96.64	0\\
96.65	0\\
96.66	0\\
96.67	0\\
96.68	1.73472347597681e-18\\
96.69	0\\
96.7	1.73472347597681e-18\\
96.71	1.73472347597681e-18\\
96.72	0\\
96.73	0\\
96.74	0\\
96.75	0\\
96.76	0\\
96.77	1.73472347597681e-18\\
96.78	0\\
96.79	0\\
96.8	1.73472347597681e-18\\
96.81	0\\
96.82	1.73472347597681e-18\\
96.83	1.73472347597681e-18\\
96.84	1.73472347597681e-18\\
96.85	1.73472347597681e-18\\
96.86	0\\
96.87	0\\
96.88	1.73472347597681e-18\\
96.89	1.73472347597681e-18\\
96.9	0\\
96.91	1.73472347597681e-18\\
96.92	0\\
96.93	0\\
96.94	0\\
96.95	1.73472347597681e-18\\
96.96	1.73472347597681e-18\\
96.97	0\\
96.98	0\\
96.99	0\\
97	0\\
97.01	0\\
97.02	0\\
97.03	0\\
97.04	0\\
97.05	0\\
97.06	1.73472347597681e-18\\
97.07	0\\
97.08	1.73472347597681e-18\\
97.09	0\\
97.1	1.73472347597681e-18\\
97.11	0\\
97.12	1.73472347597681e-18\\
97.13	1.73472347597681e-18\\
97.14	1.73472347597681e-18\\
97.15	0\\
97.16	1.73472347597681e-18\\
97.17	0\\
97.18	0\\
97.19	0\\
97.2	0\\
97.21	0\\
97.22	0\\
97.23	0\\
97.24	0\\
97.25	0\\
97.26	0\\
97.27	0\\
97.28	0\\
97.29	0\\
97.3	0\\
97.31	1.73472347597681e-18\\
97.32	1.73472347597681e-18\\
97.33	1.73472347597681e-18\\
97.34	0\\
97.35	1.73472347597681e-18\\
97.36	0\\
97.37	1.73472347597681e-18\\
97.38	0\\
97.39	0\\
97.4	1.73472347597681e-18\\
97.41	0\\
97.42	0\\
97.43	0\\
97.44	1.73472347597681e-18\\
97.45	0\\
97.46	0\\
97.47	0\\
97.48	0\\
97.49	1.73472347597681e-18\\
97.5	1.73472347597681e-18\\
97.51	0\\
97.52	0\\
97.53	0\\
97.54	1.73472347597681e-18\\
97.55	0\\
97.56	0\\
97.57	0\\
97.58	0\\
97.59	0\\
97.6	1.73472347597681e-18\\
97.61	0\\
97.62	1.73472347597681e-18\\
97.63	0\\
97.64	1.73472347597681e-18\\
97.65	0\\
97.66	0\\
97.67	0\\
97.68	0\\
97.69	0\\
97.7	0\\
97.71	0\\
97.72	0\\
97.73	0\\
97.74	0\\
97.75	0\\
97.76	1.73472347597681e-18\\
97.77	1.73472347597681e-18\\
97.78	0\\
97.79	1.73472347597681e-18\\
97.8	0\\
97.81	0\\
97.82	0\\
97.83	0\\
97.84	0\\
97.85	0\\
97.86	0\\
97.87	0\\
97.88	0\\
97.89	0\\
97.9	0\\
97.91	0\\
97.92	0\\
97.93	0\\
97.94	0\\
97.95	1.73472347597681e-18\\
97.96	0\\
97.97	1.73472347597681e-18\\
97.98	1.73472347597681e-18\\
97.99	0\\
98	0\\
98.01	1.73472347597681e-18\\
98.02	0\\
98.03	0\\
98.04	0\\
98.05	0\\
98.06	0\\
98.07	0\\
98.08	0\\
98.09	1.73472347597681e-18\\
98.1	0\\
98.11	0\\
98.12	0\\
98.13	1.73472347597681e-18\\
98.14	0\\
98.15	0\\
98.16	0\\
98.17	1.73472347597681e-18\\
98.18	0\\
98.19	0\\
98.2	0\\
98.21	0\\
98.22	1.73472347597681e-18\\
98.23	0\\
98.24	1.73472347597681e-18\\
98.25	0\\
98.26	0\\
98.27	0\\
98.28	0\\
98.29	0\\
98.3	0\\
98.31	0\\
98.32	0\\
98.33	0\\
98.34	0\\
98.35	0\\
98.36	0\\
98.37	0\\
98.38	0\\
98.39	0\\
98.4	0\\
98.41	0\\
98.42	0\\
98.43	0\\
98.44	0\\
98.45	0\\
98.46	0\\
98.47	1.73472347597681e-18\\
98.48	0\\
98.49	0\\
98.5	0\\
98.51	1.73472347597681e-18\\
98.52	0\\
98.53	0\\
98.54	0\\
98.55	1.73472347597681e-18\\
98.56	1.73472347597681e-18\\
98.57	0\\
98.58	0\\
98.59	0\\
98.6	0\\
98.61	0\\
98.62	1.73472347597681e-18\\
98.63	0\\
98.64	0\\
98.65	0\\
98.66	0\\
98.67	0\\
98.68	0\\
98.69	0\\
98.7	0\\
98.71	0\\
98.72	0\\
98.73	1.73472347597681e-18\\
98.74	0\\
98.75	0\\
98.76	0\\
98.77	0\\
98.78	0\\
98.79	1.73472347597681e-18\\
98.8	0\\
98.81	0\\
98.82	1.73472347597681e-18\\
98.83	0\\
98.84	1.73472347597681e-18\\
98.85	0\\
98.86	0\\
98.87	1.73472347597681e-18\\
98.88	0\\
98.89	0\\
98.9	0\\
98.91	0\\
98.92	0\\
98.93	0\\
98.94	1.73472347597681e-18\\
98.95	0\\
98.96	0\\
98.97	1.73472347597681e-18\\
98.98	0\\
98.99	0\\
99	0\\
99.01	0\\
99.02	0\\
99.03	0\\
99.04	0\\
99.05	1.73472347597681e-18\\
99.06	0\\
99.07	0\\
99.08	1.73472347597681e-18\\
99.09	0\\
99.1	0\\
99.11	1.73472347597681e-18\\
99.12	0\\
99.13	0\\
99.14	0\\
99.15	0\\
99.16	0\\
99.17	1.73472347597681e-18\\
99.18	0\\
99.19	0\\
99.2	0\\
99.21	0\\
99.22	0\\
99.23	0\\
99.24	1.73472347597681e-18\\
99.25	0\\
99.26	0\\
99.27	1.73472347597681e-18\\
99.28	0\\
99.29	0\\
99.3	0\\
99.31	0\\
99.32	0\\
99.33	0\\
99.34	0\\
99.35	0\\
99.36	0\\
99.37	0\\
99.38	0\\
99.39	0\\
99.4	0\\
99.41	0\\
99.42	0\\
99.43	0\\
99.44	0\\
99.45	0\\
99.46	0\\
99.47	0\\
99.48	0\\
99.49	0\\
99.5	0\\
99.51	1.73472347597681e-18\\
99.52	0\\
99.53	0\\
99.54	0\\
99.55	0\\
99.56	0\\
99.57	0\\
99.58	0\\
99.59	0\\
99.6	0\\
99.61	1.73472347597681e-18\\
99.62	0\\
99.63	0\\
99.64	0\\
99.65	1.73472347597681e-18\\
99.66	0\\
99.67	1.73472347597681e-18\\
99.68	0\\
99.69	0\\
99.7	0\\
99.71	0\\
99.72	0\\
99.73	0\\
99.74	0\\
99.75	0\\
99.76	0\\
99.77	1.73472347597681e-18\\
99.78	0\\
99.79	0\\
99.8	0\\
99.81	0\\
99.82	0\\
99.83	0\\
99.84	0\\
99.85	0\\
99.86	0\\
99.87	0\\
99.88	0\\
99.89	0\\
99.9	0\\
99.91	0\\
99.92	0\\
99.93	0\\
99.94	0\\
99.95	0\\
99.96	0\\
99.97	0\\
99.98	0\\
99.99	0\\
100	0\\
};
\addlegendentry{$q=4$};

\end{axis}
\end{tikzpicture}%
  \caption{Continuous Time}
\end{subfigure}%
\hfill%
\begin{subfigure}{.45\linewidth}
  \centering
  \setlength\figureheight{\linewidth} 
  \setlength\figurewidth{\linewidth}
  \tikzsetnextfilename{dm_dscr_z15}
  % This file was created by matlab2tikz.
%
%The latest updates can be retrieved from
%  http://www.mathworks.com/matlabcentral/fileexchange/22022-matlab2tikz-matlab2tikz
%where you can also make suggestions and rate matlab2tikz.
%
\definecolor{mycolor1}{rgb}{0.00000,1.00000,0.14286}%
\definecolor{mycolor2}{rgb}{0.00000,1.00000,0.28571}%
\definecolor{mycolor3}{rgb}{0.00000,1.00000,0.42857}%
\definecolor{mycolor4}{rgb}{0.00000,1.00000,0.57143}%
\definecolor{mycolor5}{rgb}{0.00000,1.00000,0.71429}%
\definecolor{mycolor6}{rgb}{0.00000,1.00000,0.85714}%
\definecolor{mycolor7}{rgb}{0.00000,1.00000,1.00000}%
\definecolor{mycolor8}{rgb}{0.00000,0.87500,1.00000}%
\definecolor{mycolor9}{rgb}{0.00000,0.62500,1.00000}%
\definecolor{mycolor10}{rgb}{0.12500,0.00000,1.00000}%
\definecolor{mycolor11}{rgb}{0.25000,0.00000,1.00000}%
\definecolor{mycolor12}{rgb}{0.37500,0.00000,1.00000}%
\definecolor{mycolor13}{rgb}{0.50000,0.00000,1.00000}%
\definecolor{mycolor14}{rgb}{0.62500,0.00000,1.00000}%
\definecolor{mycolor15}{rgb}{0.75000,0.00000,1.00000}%
\definecolor{mycolor16}{rgb}{0.87500,0.00000,1.00000}%
\definecolor{mycolor17}{rgb}{1.00000,0.00000,1.00000}%
\definecolor{mycolor18}{rgb}{1.00000,0.00000,0.87500}%
\definecolor{mycolor19}{rgb}{1.00000,0.00000,0.62500}%
\definecolor{mycolor20}{rgb}{0.85714,0.00000,0.00000}%
\definecolor{mycolor21}{rgb}{0.71429,0.00000,0.00000}%
%
\begin{tikzpicture}

\begin{axis}[%
width=4.1in,
height=3.803in,
at={(0.809in,0.513in)},
scale only axis,
point meta min=0,
point meta max=1,
every outer x axis line/.append style={black},
every x tick label/.append style={font=\color{black}},
xmin=0,
xmax=600,
every outer y axis line/.append style={black},
every y tick label/.append style={font=\color{black}},
ymin=0,
ymax=0.014,
axis background/.style={fill=white},
axis x line*=bottom,
axis y line*=left,
colormap={mymap}{[1pt] rgb(0pt)=(0,1,0); rgb(7pt)=(0,1,1); rgb(15pt)=(0,0,1); rgb(23pt)=(1,0,1); rgb(31pt)=(1,0,0); rgb(38pt)=(0,0,0)},
colorbar,
colorbar style={separate axis lines,every outer x axis line/.append style={black},every x tick label/.append style={font=\color{black}},every outer y axis line/.append style={black},every y tick label/.append style={font=\color{black}},yticklabels={{-19},{-17},{-15},{-13},{-11},{-9},{-7},{-5},{-3},{-1},{1},{3},{5},{7},{9},{11},{13},{15},{17},{19}}}
]
\addplot [color=green,solid,forget plot]
  table[row sep=crcr]{%
1	0.0112532687603981\\
2	0.0112532699026608\\
3	0.0112532710662638\\
4	0.0112532722516089\\
5	0.0112532734591055\\
6	0.0112532746891708\\
7	0.0112532759422298\\
8	0.0112532772187157\\
9	0.01125327851907\\
10	0.0112532798437422\\
11	0.0112532811931907\\
12	0.0112532825678825\\
13	0.0112532839682934\\
14	0.0112532853949084\\
15	0.0112532868482215\\
16	0.0112532883287362\\
17	0.0112532898369656\\
18	0.0112532913734325\\
19	0.0112532929386696\\
20	0.0112532945332198\\
21	0.0112532961576363\\
22	0.0112532978124828\\
23	0.0112532994983338\\
24	0.0112533012157746\\
25	0.0112533029654016\\
26	0.0112533047478228\\
27	0.0112533065636575\\
28	0.0112533084135369\\
29	0.0112533102981043\\
30	0.011253312218015\\
31	0.0112533141739371\\
32	0.0112533161665511\\
33	0.0112533181965506\\
34	0.0112533202646425\\
35	0.011253322371547\\
36	0.0112533245179982\\
37	0.0112533267047441\\
38	0.0112533289325467\\
39	0.0112533312021831\\
40	0.0112533335144446\\
41	0.011253335870138\\
42	0.0112533382700853\\
43	0.0112533407151242\\
44	0.0112533432061085\\
45	0.011253345743908\\
46	0.0112533483294094\\
47	0.0112533509635161\\
48	0.0112533536471489\\
49	0.0112533563812462\\
50	0.011253359166764\\
51	0.0112533620046768\\
52	0.0112533648959778\\
53	0.0112533678416789\\
54	0.0112533708428115\\
55	0.0112533739004266\\
56	0.0112533770155955\\
57	0.0112533801894096\\
58	0.0112533834229815\\
59	0.0112533867174449\\
60	0.011253390073955\\
61	0.0112533934936895\\
62	0.0112533969778483\\
63	0.0112534005276543\\
64	0.0112534041443538\\
65	0.0112534078292171\\
66	0.0112534115835384\\
67	0.0112534154086371\\
68	0.0112534193058576\\
69	0.0112534232765701\\
70	0.0112534273221709\\
71	0.0112534314440833\\
72	0.0112534356437576\\
73	0.011253439922672\\
74	0.0112534442823328\\
75	0.0112534487242756\\
76	0.0112534532500649\\
77	0.0112534578612956\\
78	0.0112534625595927\\
79	0.0112534673466128\\
80	0.0112534722240441\\
81	0.0112534771936069\\
82	0.0112534822570549\\
83	0.0112534874161752\\
84	0.0112534926727891\\
85	0.011253498028753\\
86	0.0112535034859588\\
87	0.0112535090463348\\
88	0.0112535147118463\\
89	0.0112535204844961\\
90	0.0112535263663258\\
91	0.0112535323594158\\
92	0.0112535384658868\\
93	0.0112535446879\\
94	0.0112535510276581\\
95	0.0112535574874064\\
96	0.0112535640694328\\
97	0.0112535707760698\\
98	0.0112535776096942\\
99	0.0112535845727288\\
100	0.0112535916676428\\
101	0.0112535988969529\\
102	0.0112536062632243\\
103	0.0112536137690715\\
104	0.0112536214171592\\
105	0.0112536292102033\\
106	0.0112536371509723\\
107	0.0112536452422876\\
108	0.011253653487025\\
109	0.0112536618881159\\
110	0.0112536704485478\\
111	0.011253679171366\\
112	0.0112536880596742\\
113	0.0112536971166362\\
114	0.0112537063454764\\
115	0.0112537157494815\\
116	0.0112537253320017\\
117	0.0112537350964515\\
118	0.0112537450463114\\
119	0.0112537551851289\\
120	0.0112537655165201\\
121	0.0112537760441705\\
122	0.0112537867718372\\
123	0.0112537977033494\\
124	0.0112538088426103\\
125	0.0112538201935984\\
126	0.0112538317603692\\
127	0.0112538435470563\\
128	0.0112538555578733\\
129	0.0112538677971149\\
130	0.0112538802691592\\
131	0.0112538929784685\\
132	0.0112539059295916\\
133	0.0112539191271652\\
134	0.0112539325759154\\
135	0.01125394628066\\
136	0.01125396024631\\
137	0.011253974477871\\
138	0.011253988980446\\
139	0.0112540037592363\\
140	0.0112540188195443\\
141	0.0112540341667746\\
142	0.0112540498064368\\
143	0.0112540657441471\\
144	0.0112540819856307\\
145	0.0112540985367233\\
146	0.0112541154033742\\
147	0.0112541325916476\\
148	0.0112541501077257\\
149	0.0112541679579103\\
150	0.0112541861486255\\
151	0.0112542046864202\\
152	0.0112542235779703\\
153	0.0112542428300813\\
154	0.0112542624496907\\
155	0.0112542824438711\\
156	0.0112543028198323\\
157	0.011254323584924\\
158	0.011254344746639\\
159	0.0112543663126158\\
160	0.0112543882906414\\
161	0.0112544106886541\\
162	0.011254433514747\\
163	0.0112544567771703\\
164	0.0112544804843351\\
165	0.0112545046448164\\
166	0.0112545292673558\\
167	0.0112545543608657\\
168	0.011254579934432\\
169	0.0112546059973177\\
170	0.0112546325589665\\
171	0.0112546596290065\\
172	0.0112546872172535\\
173	0.011254715333715\\
174	0.011254743988594\\
175	0.0112547731922928\\
176	0.0112548029554169\\
177	0.0112548332887792\\
178	0.0112548642034043\\
179	0.0112548957105321\\
180	0.0112549278216228\\
181	0.011254960548361\\
182	0.01125499390266\\
183	0.0112550278966667\\
184	0.0112550625427663\\
185	0.0112550978535867\\
186	0.0112551338420038\\
187	0.0112551705211462\\
188	0.0112552079044005\\
189	0.0112552460054162\\
190	0.0112552848381115\\
191	0.0112553244166782\\
192	0.0112553647555875\\
193	0.0112554058695958\\
194	0.0112554477737507\\
195	0.0112554904833967\\
196	0.0112555340141818\\
197	0.011255578382062\\
198	0.0112556236033085\\
199	0.0112556696945146\\
200	0.0112557166726019\\
201	0.0112557645548274\\
202	0.0112558133587902\\
203	0.0112558631024383\\
204	0.0112559138040762\\
205	0.0112559654823721\\
206	0.0112560181563652\\
207	0.0112560718454738\\
208	0.0112561265695027\\
209	0.0112561823486517\\
210	0.0112562392035231\\
211	0.0112562971551311\\
212	0.0112563562249094\\
213	0.0112564164347205\\
214	0.0112564778068648\\
215	0.0112565403640894\\
216	0.0112566041295975\\
217	0.0112566691270586\\
218	0.0112567353806174\\
219	0.0112568029149049\\
220	0.0112568717550477\\
221	0.0112569419266792\\
222	0.0112570134559501\\
223	0.0112570863695396\\
224	0.0112571606946663\\
225	0.0112572364591004\\
226	0.0112573136911746\\
227	0.0112573924197972\\
228	0.0112574726744636\\
229	0.0112575544852696\\
230	0.011257637882924\\
231	0.0112577228987617\\
232	0.0112578095647579\\
233	0.0112578979135413\\
234	0.0112579879784087\\
235	0.0112580797933394\\
236	0.0112581733930099\\
237	0.0112582688128096\\
238	0.0112583660888555\\
239	0.0112584652580092\\
240	0.011258566357892\\
241	0.0112586694269024\\
242	0.0112587745042325\\
243	0.0112588816298857\\
244	0.0112589908446942\\
245	0.0112591021903371\\
246	0.0112592157093589\\
247	0.0112593314451881\\
248	0.0112594494421567\\
249	0.0112595697455197\\
250	0.0112596924014744\\
251	0.0112598174571816\\
252	0.0112599449607854\\
253	0.0112600749614344\\
254	0.0112602075093029\\
255	0.0112603426556124\\
256	0.0112604804526533\\
257	0.0112606209538069\\
258	0.011260764213568\\
259	0.0112609102875668\\
260	0.0112610592325917\\
261	0.0112612111066118\\
262	0.0112613659688\\
263	0.0112615238795553\\
264	0.0112616849005253\\
265	0.0112618490946292\\
266	0.0112620165260794\\
267	0.0112621872604036\\
268	0.011262361364466\\
269	0.011262538906487\\
270	0.0112627199560613\\
271	0.0112629045841716\\
272	0.0112630928631914\\
273	0.0112632848668682\\
274	0.0112634806702647\\
275	0.0112636803496342\\
276	0.0112638839822626\\
277	0.0112640916466497\\
278	0.0112643034248528\\
279	0.0112645194021022\\
280	0.0112647396624035\\
281	0.0112649642914616\\
282	0.0112651933767158\\
283	0.0112654270073747\\
284	0.0112656652744531\\
285	0.0112659082708092\\
286	0.0112661560911821\\
287	0.0112664088322304\\
288	0.0112666665925723\\
289	0.0112669294728254\\
290	0.0112671975756479\\
291	0.0112674710057809\\
292	0.0112677498700911\\
293	0.0112680342776147\\
294	0.011268324339602\\
295	0.0112686201695635\\
296	0.0112689218833159\\
297	0.0112692295990304\\
298	0.0112695434372808\\
299	0.0112698635210935\\
300	0.0112701899759983\\
301	0.0112705229300796\\
302	0.0112708625140301\\
303	0.0112712088612041\\
304	0.0112715621076731\\
305	0.0112719223922818\\
306	0.0112722898567058\\
307	0.0112726646455104\\
308	0.0112730469062103\\
309	0.0112734367893311\\
310	0.0112738344484716\\
311	0.0112742400403679\\
312	0.0112746537249585\\
313	0.0112750756654509\\
314	0.0112755060283896\\
315	0.0112759449837256\\
316	0.0112763927048876\\
317	0.0112768493688539\\
318	0.0112773151562267\\
319	0.0112777902513079\\
320	0.0112782748421758\\
321	0.0112787691207639\\
322	0.0112792732829419\\
323	0.0112797875285972\\
324	0.0112803120617194\\
325	0.0112808470904854\\
326	0.0112813928273475\\
327	0.0112819494891221\\
328	0.0112825172970816\\
329	0.0112830964770469\\
330	0.0112836872594831\\
331	0.0112842898795961\\
332	0.0112849045774321\\
333	0.0112855315979786\\
334	0.011286171191268\\
335	0.0112868236124828\\
336	0.0112874891220635\\
337	0.0112881679858184\\
338	0.011288860475036\\
339	0.0112895668665993\\
340	0.0112902874431027\\
341	0.0112910224929716\\
342	0.0112917723105838\\
343	0.0112925371963936\\
344	0.0112933174570591\\
345	0.0112941134055709\\
346	0.0112949253613844\\
347	0.0112957536505542\\
348	0.0112965986058711\\
349	0.0112974605670026\\
350	0.0112983398806354\\
351	0.0112992369006206\\
352	0.0113001519881228\\
353	0.0113010855117708\\
354	0.0113020378478121\\
355	0.0113030093802701\\
356	0.0113040005011042\\
357	0.0113050116103734\\
358	0.0113060431164024\\
359	0.011307095435951\\
360	0.0113081689943874\\
361	0.0113092642258635\\
362	0.011310381573495\\
363	0.0113115214895429\\
364	0.0113126844356008\\
365	0.0113138708827835\\
366	0.0113150813119201\\
367	0.0113163162137513\\
368	0.011317576089129\\
369	0.0113188614492209\\
370	0.0113201728157185\\
371	0.0113215107210487\\
372	0.0113228757085908\\
373	0.0113242683328963\\
374	0.0113256891599139\\
375	0.0113271387672193\\
376	0.0113286177442489\\
377	0.0113301266925384\\
378	0.0113316662259654\\
379	0.0113332369709945\\
380	0.0113348395669233\\
381	0.0113364746661287\\
382	0.0113381429343176\\
383	0.011339845050813\\
384	0.0113415817089591\\
385	0.0113433536167269\\
386	0.0113451614968069\\
387	0.0113470060863007\\
388	0.0113488881375493\\
389	0.0113508084184482\\
390	0.0113527677127685\\
391	0.0113547668204836\\
392	0.0113568065581035\\
393	0.0113588877590146\\
394	0.0113610112738261\\
395	0.011363177970724\\
396	0.0113653887358305\\
397	0.0113676444735715\\
398	0.0113699461070501\\
399	0.0113722945784273\\
400	0.0113746908493106\\
401	0.0113771359011482\\
402	0.0113796307356319\\
403	0.0113821763751068\\
404	0.0113847738629876\\
405	0.0113874242641834\\
406	0.0113901286655294\\
407	0.0113928881762258\\
408	0.0113957039282847\\
409	0.011398577076984\\
410	0.0114015088013289\\
411	0.0114045003045212\\
412	0.0114075528144347\\
413	0.0114106675840995\\
414	0.0114138458921919\\
415	0.0114170890435317\\
416	0.0114203983695865\\
417	0.0114237752289818\\
418	0.0114272210080167\\
419	0.0114307371211865\\
420	0.0114343250117111\\
421	0.0114379861520722\\
422	0.0114417220445615\\
423	0.011445534221831\\
424	0.0114494242474103\\
425	0.0114533937162365\\
426	0.0114574442552246\\
427	0.0114615775238116\\
428	0.0114657952144996\\
429	0.011470099053398\\
430	0.0114744908007607\\
431	0.01147897225152\\
432	0.0114835452358132\\
433	0.0114882116195013\\
434	0.0114929733046774\\
435	0.0114978322301627\\
436	0.0115027903719873\\
437	0.0115078497438535\\
438	0.0115130123975779\\
439	0.0115182804235104\\
440	0.0115236559509238\\
441	0.0115291411483723\\
442	0.0115347382240134\\
443	0.0115404494258877\\
444	0.0115462770421518\\
445	0.0115522234012577\\
446	0.011558290872072\\
447	0.0115644818639277\\
448	0.0115707988265971\\
449	0.0115772442501793\\
450	0.0115838206648954\\
451	0.0115905306407745\\
452	0.01159737678722\\
453	0.0116043617524433\\
454	0.0116114882227481\\
455	0.0116187589216505\\
456	0.0116261766088137\\
457	0.0116337440787798\\
458	0.0116414641594738\\
459	0.0116493397104556\\
460	0.0116573736208924\\
461	0.0116655688072201\\
462	0.0116739282104558\\
463	0.0116824547931162\\
464	0.011691151535667\\
465	0.0117000214323632\\
466	0.011709067486144\\
467	0.0117182927016689\\
468	0.011727700073867\\
469	0.0117372925641882\\
470	0.0117470730410027\\
471	0.011757044112307\\
472	0.0117672076294984\\
473	0.0117775631743661\\
474	0.01178810336984\\
475	0.0117988334177848\\
476	0.0118097773734434\\
477	0.0118209447552968\\
478	0.0118323402618766\\
479	0.0118439672533229\\
480	0.0118558289563955\\
481	0.0118679284734506\\
482	0.0118802687620906\\
483	0.0118928526119643\\
484	0.0119056826195373\\
485	0.0119187611611653\\
486	0.0119320903647791\\
487	0.0119456720806265\\
488	0.01195950785181\\
489	0.0119735988858536\\
490	0.0119879460295273\\
491	0.0120025497514546\\
492	0.0120174101430532\\
493	0.0120325269654678\\
494	0.0120478998211435\\
495	0.0120635286852891\\
496	0.012079415523849\\
497	0.0120955692953053\\
498	0.0121120217504468\\
499	0.0121288784953111\\
500	0.0121492200434839\\
501	0.0121805823533343\\
502	0.012212480160662\\
503	0.0122428675404684\\
504	0.0122629773579806\\
505	0.0122791333586272\\
506	0.012293039209729\\
507	0.0123049439605129\\
508	0.0123140109558925\\
509	0.0123228481488757\\
510	0.0123314452400056\\
511	0.0123397252426882\\
512	0.0123481270524362\\
513	0.0123566704589909\\
514	0.0123653735381654\\
515	0.0123742482061504\\
516	0.0123832999768181\\
517	0.0123925190306154\\
518	0.0124019204071479\\
519	0.0124115299016542\\
520	0.0124213521012647\\
521	0.0124313915350337\\
522	0.0124416527706151\\
523	0.0124521404175238\\
524	0.0124628591285542\\
525	0.0124738136246502\\
526	0.0124850087478262\\
527	0.0124964496311903\\
528	0.0125081422222362\\
529	0.0125200948793841\\
530	0.0125327166443861\\
531	0.0125466030129406\\
532	0.0125602143170317\\
533	0.0125724298226389\\
534	0.0125847215275143\\
535	0.0125969318824986\\
536	0.0126092075618591\\
537	0.0126216960159315\\
538	0.0126344089773676\\
539	0.0126473486000177\\
540	0.0126605165736733\\
541	0.0126739141877061\\
542	0.0126875424056075\\
543	0.0127014017427442\\
544	0.0127154919186042\\
545	0.0127298109255144\\
546	0.0127443524719401\\
547	0.0127590989427433\\
548	0.0127740579513386\\
549	0.0127892536849125\\
550	0.0128046905323789\\
551	0.0128222833008422\\
552	0.0128386517601674\\
553	0.012854041701367\\
554	0.0128692201193779\\
555	0.012884594762811\\
556	0.012900165549626\\
557	0.0129159297924771\\
558	0.0129318843547016\\
559	0.0129480256379725\\
560	0.0129643495822428\\
561	0.0129808517156201\\
562	0.0129975273440607\\
563	0.0130143721298674\\
564	0.0130313837529099\\
565	0.0130495369065494\\
566	0.0130664447891471\\
567	0.0130832622072266\\
568	0.0131002111569796\\
569	0.0131172839941916\\
570	0.0131344722037186\\
571	0.0131517663252309\\
572	0.0131691558724101\\
573	0.013186629244912\\
574	0.0132041736323075\\
575	0.013221774909103\\
576	0.0132394175198042\\
577	0.0132570843528331\\
578	0.0132747566019253\\
579	0.0132924136134348\\
580	0.0133100327177776\\
581	0.0133275890431348\\
582	0.0133450553097115\\
583	0.0133624016039255\\
584	0.0133795951354571\\
585	0.0133965999904454\\
586	0.0134133769227543\\
587	0.0134298833028437\\
588	0.0134460735522599\\
589	0.0134619009495395\\
590	0.0134776600725805\\
591	0.0134935706885156\\
592	0.0135096572289508\\
593	0.0135260152836207\\
594	0.0135429334162338\\
595	0.0135612173030331\\
596	0.0135830518096797\\
597	0.0136142933738059\\
598	0.0136705746214114\\
599	0\\
600	0\\
};
\addplot [color=mycolor1,solid,forget plot]
  table[row sep=crcr]{%
1	0.0112524830521932\\
2	0.0112524842682691\\
3	0.0112524855069177\\
4	0.0112524867685612\\
5	0.0112524880536299\\
6	0.0112524893625619\\
7	0.011252490695804\\
8	0.0112524920538111\\
9	0.0112524934370468\\
10	0.0112524948459833\\
11	0.011252496281102\\
12	0.0112524977428932\\
13	0.0112524992318564\\
14	0.0112525007485005\\
15	0.0112525022933443\\
16	0.0112525038669161\\
17	0.0112525054697542\\
18	0.0112525071024072\\
19	0.011252508765434\\
20	0.0112525104594041\\
21	0.0112525121848976\\
22	0.0112525139425058\\
23	0.0112525157328309\\
24	0.0112525175564867\\
25	0.0112525194140986\\
26	0.0112525213063036\\
27	0.0112525232337509\\
28	0.0112525251971022\\
29	0.0112525271970312\\
30	0.0112525292342248\\
31	0.0112525313093826\\
32	0.0112525334232177\\
33	0.0112525355764565\\
34	0.0112525377698392\\
35	0.0112525400041199\\
36	0.0112525422800672\\
37	0.0112525445984642\\
38	0.0112525469601086\\
39	0.0112525493658135\\
40	0.0112525518164072\\
41	0.0112525543127338\\
42	0.0112525568556533\\
43	0.0112525594460422\\
44	0.0112525620847933\\
45	0.0112525647728165\\
46	0.011252567511039\\
47	0.0112525703004055\\
48	0.0112525731418785\\
49	0.0112525760364391\\
50	0.0112525789850866\\
51	0.0112525819888394\\
52	0.0112525850487355\\
53	0.0112525881658321\\
54	0.0112525913412068\\
55	0.0112525945759576\\
56	0.0112525978712032\\
57	0.0112526012280838\\
58	0.0112526046477609\\
59	0.0112526081314184\\
60	0.0112526116802625\\
61	0.0112526152955222\\
62	0.0112526189784502\\
63	0.0112526227303226\\
64	0.0112526265524399\\
65	0.0112526304461275\\
66	0.0112526344127358\\
67	0.0112526384536408\\
68	0.0112526425702449\\
69	0.0112526467639768\\
70	0.0112526510362928\\
71	0.0112526553886767\\
72	0.0112526598226403\\
73	0.0112526643397245\\
74	0.0112526689414994\\
75	0.0112526736295649\\
76	0.0112526784055514\\
77	0.0112526832711204\\
78	0.0112526882279649\\
79	0.0112526932778101\\
80	0.0112526984224144\\
81	0.0112527036635692\\
82	0.0112527090031002\\
83	0.0112527144428681\\
84	0.0112527199847687\\
85	0.0112527256307342\\
86	0.0112527313827333\\
87	0.0112527372427726\\
88	0.0112527432128966\\
89	0.0112527492951889\\
90	0.011252755491773\\
91	0.0112527618048126\\
92	0.0112527682365128\\
93	0.0112527747891207\\
94	0.0112527814649262\\
95	0.0112527882662632\\
96	0.0112527951955097\\
97	0.0112528022550894\\
98	0.0112528094474719\\
99	0.0112528167751743\\
100	0.0112528242407616\\
101	0.0112528318468476\\
102	0.0112528395960962\\
103	0.0112528474912222\\
104	0.0112528555349919\\
105	0.011252863730225\\
106	0.0112528720797944\\
107	0.0112528805866285\\
108	0.0112528892537111\\
109	0.0112528980840834\\
110	0.0112529070808445\\
111	0.0112529162471529\\
112	0.0112529255862273\\
113	0.011252935101348\\
114	0.011252944795858\\
115	0.0112529546731642\\
116	0.0112529647367387\\
117	0.01125297499012\\
118	0.0112529854369143\\
119	0.0112529960807966\\
120	0.0112530069255124\\
121	0.0112530179748789\\
122	0.0112530292327861\\
123	0.0112530407031986\\
124	0.011253052390157\\
125	0.011253064297779\\
126	0.0112530764302613\\
127	0.0112530887918808\\
128	0.0112531013869965\\
129	0.0112531142200508\\
130	0.011253127295571\\
131	0.0112531406181714\\
132	0.0112531541925545\\
133	0.0112531680235129\\
134	0.0112531821159312\\
135	0.0112531964747874\\
136	0.0112532111051549\\
137	0.0112532260122047\\
138	0.0112532412012064\\
139	0.011253256677531\\
140	0.0112532724466523\\
141	0.0112532885141492\\
142	0.0112533048857074\\
143	0.0112533215671218\\
144	0.0112533385642984\\
145	0.0112533558832564\\
146	0.0112533735301306\\
147	0.0112533915111736\\
148	0.0112534098327576\\
149	0.0112534285013775\\
150	0.0112534475236526\\
151	0.0112534669063294\\
152	0.0112534866562836\\
153	0.0112535067805233\\
154	0.0112535272861908\\
155	0.0112535481805658\\
156	0.0112535694710675\\
157	0.0112535911652578\\
158	0.0112536132708436\\
159	0.01125363579568\\
160	0.0112536587477729\\
161	0.0112536821352819\\
162	0.0112537059665233\\
163	0.0112537302499732\\
164	0.0112537549942706\\
165	0.0112537802082204\\
166	0.0112538059007965\\
167	0.0112538320811453\\
168	0.0112538587585888\\
169	0.0112538859426284\\
170	0.0112539136429476\\
171	0.0112539418694161\\
172	0.0112539706320933\\
173	0.0112539999412315\\
174	0.0112540298072804\\
175	0.0112540602408898\\
176	0.0112540912529146\\
177	0.0112541228544177\\
178	0.0112541550566747\\
179	0.0112541878711776\\
180	0.0112542213096389\\
181	0.011254255383996\\
182	0.0112542901064154\\
183	0.0112543254892971\\
184	0.0112543615452787\\
185	0.0112543982872405\\
186	0.0112544357283096\\
187	0.0112544738818647\\
188	0.0112545127615409\\
189	0.0112545523812343\\
190	0.0112545927551067\\
191	0.0112546338975906\\
192	0.0112546758233932\\
193	0.0112547185475021\\
194	0.0112547620851906\\
195	0.0112548064520278\\
196	0.0112548516638952\\
197	0.0112548977370016\\
198	0.0112549446878254\\
199	0.0112549925331587\\
200	0.0112550412901154\\
201	0.0112550909761368\\
202	0.0112551416089982\\
203	0.0112551932068148\\
204	0.0112552457880488\\
205	0.0112552993715154\\
206	0.0112553539763898\\
207	0.011255409622214\\
208	0.0112554663289035\\
209	0.0112555241167546\\
210	0.0112555830064519\\
211	0.011255643019075\\
212	0.0112557041761064\\
213	0.0112557664994391\\
214	0.0112558300113845\\
215	0.0112558947346801\\
216	0.0112559606924975\\
217	0.0112560279084513\\
218	0.0112560964066066\\
219	0.0112561662114884\\
220	0.0112562373480898\\
221	0.0112563098418813\\
222	0.01125638371882\\
223	0.0112564590053583\\
224	0.0112565357284544\\
225	0.0112566139155812\\
226	0.0112566935947369\\
227	0.0112567747944548\\
228	0.0112568575438139\\
229	0.0112569418724499\\
230	0.0112570278105655\\
231	0.0112571153889423\\
232	0.0112572046389523\\
233	0.0112572955925695\\
234	0.0112573882823822\\
235	0.0112574827416059\\
236	0.0112575790040958\\
237	0.0112576771043608\\
238	0.0112577770775767\\
239	0.0112578789596008\\
240	0.0112579827869868\\
241	0.011258088597\\
242	0.0112581964276333\\
243	0.0112583063176235\\
244	0.0112584183064691\\
245	0.0112585324344476\\
246	0.0112586487426347\\
247	0.0112587672729237\\
248	0.0112588880680461\\
249	0.011259011171593\\
250	0.011259136628038\\
251	0.0112592644827603\\
252	0.0112593947820703\\
253	0.0112595275732355\\
254	0.0112596629045085\\
255	0.0112598008251563\\
256	0.011259941385491\\
257	0.011260084636903\\
258	0.0112602306318954\\
259	0.0112603794241206\\
260	0.0112605310684197\\
261	0.0112606856208631\\
262	0.0112608431387948\\
263	0.0112610036808777\\
264	0.0112611673071434\\
265	0.011261334079043\\
266	0.0112615040595019\\
267	0.0112616773129768\\
268	0.0112618539055155\\
269	0.0112620339048183\\
270	0.0112622173802994\\
271	0.0112624044031419\\
272	0.0112625950463256\\
273	0.0112627893845708\\
274	0.0112629874940154\\
275	0.0112631894510961\\
276	0.011263395329147\\
277	0.01126360518903\\
278	0.0112638190572748\\
279	0.0112640370224375\\
280	0.0112642592747547\\
281	0.0112644858991333\\
282	0.0112647169821746\\
283	0.0112649526122084\\
284	0.011265192879328\\
285	0.0112654378754256\\
286	0.0112656876942285\\
287	0.011265942431336\\
288	0.0112662021842575\\
289	0.0112664670524505\\
290	0.0112667371373603\\
291	0.0112670125424597\\
292	0.0112672933732905\\
293	0.0112675797375044\\
294	0.0112678717449064\\
295	0.0112681695074977\\
296	0.0112684731395206\\
297	0.0112687827575031\\
298	0.0112690984803057\\
299	0.0112694204291684\\
300	0.0112697487277585\\
301	0.0112700835022203\\
302	0.0112704248812247\\
303	0.0112707729960207\\
304	0.0112711279804874\\
305	0.0112714899711876\\
306	0.0112718591074218\\
307	0.0112722355312845\\
308	0.0112726193877199\\
309	0.0112730108245807\\
310	0.0112734099926867\\
311	0.0112738170458853\\
312	0.0112742321411132\\
313	0.0112746554384592\\
314	0.0112750871012285\\
315	0.0112755272960083\\
316	0.011275976192735\\
317	0.0112764339647622\\
318	0.011276900788931\\
319	0.0112773768456412\\
320	0.0112778623189245\\
321	0.0112783573965185\\
322	0.0112788622699432\\
323	0.011279377134579\\
324	0.0112799021897456\\
325	0.0112804376387835\\
326	0.0112809836891372\\
327	0.0112815405524396\\
328	0.0112821084445994\\
329	0.0112826875858888\\
330	0.0112832782010351\\
331	0.0112838805193128\\
332	0.0112844947746385\\
333	0.0112851212056678\\
334	0.0112857600558946\\
335	0.0112864115737523\\
336	0.0112870760127176\\
337	0.0112877536314165\\
338	0.0112884446937331\\
339	0.0112891494689206\\
340	0.0112898682317148\\
341	0.0112906012624508\\
342	0.0112913488471817\\
343	0.0112921112778008\\
344	0.0112928888521662\\
345	0.0112936818742286\\
346	0.0112944906541617\\
347	0.0112953155084965\\
348	0.0112961567602577\\
349	0.0112970147391043\\
350	0.011297889781473\\
351	0.0112987822307247\\
352	0.0112996924372949\\
353	0.0113006207588475\\
354	0.0113015675604316\\
355	0.0113025332146426\\
356	0.011303518101786\\
357	0.0113045226100456\\
358	0.0113055471356546\\
359	0.0113065920830707\\
360	0.0113076578651547\\
361	0.0113087449033519\\
362	0.0113098536278781\\
363	0.0113109844779076\\
364	0.0113121379017654\\
365	0.0113133143571217\\
366	0.0113145143111892\\
367	0.0113157382409236\\
368	0.0113169866332251\\
369	0.0113182599851432\\
370	0.0113195588040815\\
371	0.0113208836080046\\
372	0.0113222349256451\\
373	0.0113236132967097\\
374	0.0113250192720858\\
375	0.0113264534140446\\
376	0.0113279162964421\\
377	0.011329408504914\\
378	0.0113309306370588\\
379	0.0113324833025986\\
380	0.0113340671234859\\
381	0.0113356827338897\\
382	0.0113373307799219\\
383	0.0113390119189026\\
384	0.0113407268182457\\
385	0.0113424761562293\\
386	0.0113442606364118\\
387	0.0113460809876018\\
388	0.011347937934634\\
389	0.0113498322170013\\
390	0.0113517645891429\\
391	0.0113537358207374\\
392	0.0113557466969994\\
393	0.0113577980189812\\
394	0.0113598906038787\\
395	0.0113620252853419\\
396	0.0113642029137888\\
397	0.0113664243567246\\
398	0.0113686904990635\\
399	0.0113710022434556\\
400	0.0113733605106167\\
401	0.0113757662396622\\
402	0.0113782203884447\\
403	0.0113807239338946\\
404	0.0113832778723645\\
405	0.0113858832199758\\
406	0.0113885410129683\\
407	0.0113912523080531\\
408	0.0113940181827673\\
409	0.0113968397358307\\
410	0.0113997180875058\\
411	0.0114026543799584\\
412	0.0114056497776206\\
413	0.0114087054675554\\
414	0.0114118226598219\\
415	0.011415002587841\\
416	0.0114182465087578\\
417	0.0114215557038086\\
418	0.0114249314786793\\
419	0.011428375163855\\
420	0.0114318881149549\\
421	0.0114354717130622\\
422	0.0114391273651115\\
423	0.0114428565045169\\
424	0.0114466605922073\\
425	0.0114505411163034\\
426	0.0114544995916544\\
427	0.0114585375610441\\
428	0.0114626565955319\\
429	0.0114668582947825\\
430	0.011471144287384\\
431	0.0114755162311521\\
432	0.011479975813417\\
433	0.0114845247512944\\
434	0.0114891647919347\\
435	0.0114938977127504\\
436	0.011498725321618\\
437	0.011503649457052\\
438	0.0115086719883473\\
439	0.0115137948156877\\
440	0.0115190198702147\\
441	0.0115243491140548\\
442	0.0115297845402993\\
443	0.0115353281729315\\
444	0.0115409820666968\\
445	0.0115467483069109\\
446	0.011552629009202\\
447	0.0115586263191873\\
448	0.0115647424120749\\
449	0.0115709794921507\\
450	0.0115773397920783\\
451	0.0115838255722394\\
452	0.0115904391198868\\
453	0.0115971827481654\\
454	0.0116040587949895\\
455	0.0116110696217641\\
456	0.011618217611935\\
457	0.0116255051693528\\
458	0.0116329347164351\\
459	0.0116405086921079\\
460	0.0116482295495076\\
461	0.0116560997534235\\
462	0.0116641217774544\\
463	0.0116722981008551\\
464	0.0116806312050312\\
465	0.011689123569621\\
466	0.0116977776680358\\
467	0.0117065959621533\\
468	0.0117155808954044\\
469	0.011724734882305\\
470	0.0117340602894447\\
471	0.0117435593956014\\
472	0.0117532343021368\\
473	0.0117630867333219\\
474	0.0117731176313312\\
475	0.0117833248584306\\
476	0.011793692140582\\
477	0.0118042509672604\\
478	0.0118150079442964\\
479	0.0118259668392118\\
480	0.0118371342960018\\
481	0.0118485126582112\\
482	0.0118601040918559\\
483	0.0118719106056965\\
484	0.0118839340430588\\
485	0.0118961760654111\\
486	0.0119086381338444\\
487	0.0119213214901426\\
488	0.0119342271377476\\
489	0.0119473558232064\\
490	0.0119607080190216\\
491	0.0119742839096594\\
492	0.0119880833844439\\
493	0.0120021060460078\\
494	0.0120163512553132\\
495	0.0120308182640109\\
496	0.0120455065501811\\
497	0.012060416584893\\
498	0.0120755512808906\\
499	0.0120909172749872\\
500	0.0121065174217606\\
501	0.0121223566082075\\
502	0.0121385259556374\\
503	0.0121552584342686\\
504	0.0121830012063432\\
505	0.0122136192446264\\
506	0.0122446972813977\\
507	0.012273480306238\\
508	0.0122910327289581\\
509	0.0123074638300017\\
510	0.0123224565276307\\
511	0.0123339757309958\\
512	0.0123436032414682\\
513	0.0123530262532023\\
514	0.0123622786725007\\
515	0.0123713719086139\\
516	0.0123804133831952\\
517	0.0123895935199106\\
518	0.0123989237281152\\
519	0.012408441379587\\
520	0.012418162897774\\
521	0.0124280951638259\\
522	0.0124382428452668\\
523	0.0124486105373701\\
524	0.0124592028331678\\
525	0.0124700240507275\\
526	0.0124810785167217\\
527	0.0124923706108423\\
528	0.0125039048332589\\
529	0.0125156858225964\\
530	0.0125277176757696\\
531	0.0125400076340011\\
532	0.0125525974509021\\
533	0.0125669174131264\\
534	0.0125810880498951\\
535	0.0125943232888783\\
536	0.0126070868988795\\
537	0.0126198902274787\\
538	0.0126325901523641\\
539	0.0126455042818192\\
540	0.0126586421428423\\
541	0.0126720095926186\\
542	0.012685608166418\\
543	0.0126994385965093\\
544	0.0127135009151143\\
545	0.0127277937006524\\
546	0.0127423119792135\\
547	0.0127570412082338\\
548	0.0127719752814566\\
549	0.0127871457226094\\
550	0.0128025525532398\\
551	0.0128181955814237\\
552	0.012835412943187\\
553	0.0128527562713104\\
554	0.0128685314380216\\
555	0.0128839719719866\\
556	0.0128995465837712\\
557	0.0129153156319256\\
558	0.0129312761394756\\
559	0.0129474245584713\\
560	0.0129637568594933\\
561	0.0129802685487319\\
562	0.0129969547536164\\
563	0.0130138104743898\\
564	0.0130308312004876\\
565	0.0130482511036168\\
566	0.0130662623676731\\
567	0.0130832622072162\\
568	0.0131002111569789\\
569	0.013117283994191\\
570	0.0131344722037186\\
571	0.0131517663252309\\
572	0.0131691558724101\\
573	0.013186629244912\\
574	0.0132041736323075\\
575	0.013221774909103\\
576	0.0132394175198042\\
577	0.0132570843528331\\
578	0.0132747566019253\\
579	0.0132924136134348\\
580	0.0133100327177776\\
581	0.0133275890431348\\
582	0.0133450553097115\\
583	0.0133624016039255\\
584	0.0133795951354571\\
585	0.0133965999904454\\
586	0.0134133769227543\\
587	0.0134298833028437\\
588	0.0134460735522599\\
589	0.0134619009495395\\
590	0.0134776600725805\\
591	0.0134935706885156\\
592	0.0135096572289508\\
593	0.0135260152836207\\
594	0.0135429334162338\\
595	0.0135612173030331\\
596	0.0135830518096797\\
597	0.0136142933738059\\
598	0.0136705746214114\\
599	0\\
600	0\\
};
\addplot [color=mycolor2,solid,forget plot]
  table[row sep=crcr]{%
1	0.0112495588969524\\
2	0.0112495603570775\\
3	0.0112495618438726\\
4	0.0112495633578285\\
5	0.0112495648994451\\
6	0.0112495664692315\\
7	0.0112495680677063\\
8	0.0112495696953977\\
9	0.0112495713528437\\
10	0.0112495730405923\\
11	0.0112495747592017\\
12	0.0112495765092403\\
13	0.0112495782912874\\
14	0.0112495801059325\\
15	0.0112495819537767\\
16	0.0112495838354317\\
17	0.0112495857515209\\
18	0.0112495877026792\\
19	0.0112495896895531\\
20	0.0112495917128015\\
21	0.0112495937730953\\
22	0.0112495958711177\\
23	0.011249598007565\\
24	0.0112496001831462\\
25	0.0112496023985835\\
26	0.0112496046546125\\
27	0.0112496069519826\\
28	0.0112496092914571\\
29	0.0112496116738134\\
30	0.0112496140998435\\
31	0.011249616570354\\
32	0.0112496190861667\\
33	0.0112496216481186\\
34	0.0112496242570622\\
35	0.0112496269138659\\
36	0.0112496296194144\\
37	0.0112496323746087\\
38	0.0112496351803668\\
39	0.0112496380376236\\
40	0.0112496409473314\\
41	0.0112496439104606\\
42	0.0112496469279991\\
43	0.0112496500009538\\
44	0.01124965313035\\
45	0.0112496563172322\\
46	0.0112496595626644\\
47	0.0112496628677305\\
48	0.0112496662335345\\
49	0.0112496696612011\\
50	0.0112496731518758\\
51	0.0112496767067258\\
52	0.0112496803269397\\
53	0.0112496840137287\\
54	0.0112496877683261\\
55	0.0112496915919887\\
56	0.0112496954859965\\
57	0.0112496994516535\\
58	0.0112497034902879\\
59	0.011249707603253\\
60	0.0112497117919271\\
61	0.0112497160577143\\
62	0.0112497204020451\\
63	0.0112497248263766\\
64	0.0112497293321932\\
65	0.0112497339210068\\
66	0.011249738594358\\
67	0.0112497433538158\\
68	0.0112497482009787\\
69	0.0112497531374752\\
70	0.0112497581649641\\
71	0.0112497632851352\\
72	0.0112497684997101\\
73	0.0112497738104424\\
74	0.0112497792191186\\
75	0.0112497847275588\\
76	0.0112497903376167\\
77	0.0112497960511813\\
78	0.0112498018701765\\
79	0.0112498077965623\\
80	0.0112498138323356\\
81	0.0112498199795305\\
82	0.0112498262402193\\
83	0.0112498326165131\\
84	0.0112498391105623\\
85	0.011249845724558\\
86	0.011249852460732\\
87	0.0112498593213582\\
88	0.0112498663087528\\
89	0.0112498734252757\\
90	0.0112498806733308\\
91	0.0112498880553672\\
92	0.0112498955738798\\
93	0.0112499032314105\\
94	0.0112499110305486\\
95	0.0112499189739319\\
96	0.011249927064248\\
97	0.0112499353042347\\
98	0.011249943696681\\
99	0.0112499522444285\\
100	0.0112499609503719\\
101	0.0112499698174603\\
102	0.011249978848698\\
103	0.0112499880471459\\
104	0.0112499974159222\\
105	0.0112500069582034\\
106	0.011250016677226\\
107	0.0112500265762871\\
108	0.0112500366587454\\
109	0.0112500469280231\\
110	0.0112500573876063\\
111	0.0112500680410466\\
112	0.0112500788919625\\
113	0.0112500899440401\\
114	0.0112501012010349\\
115	0.0112501126667729\\
116	0.0112501243451519\\
117	0.0112501362401428\\
118	0.0112501483557913\\
119	0.0112501606962189\\
120	0.0112501732656244\\
121	0.0112501860682858\\
122	0.011250199108561\\
123	0.0112502123908901\\
124	0.0112502259197965\\
125	0.0112502396998885\\
126	0.0112502537358608\\
127	0.0112502680324967\\
128	0.011250282594669\\
129	0.011250297427342\\
130	0.0112503125355736\\
131	0.0112503279245163\\
132	0.0112503435994196\\
133	0.0112503595656316\\
134	0.0112503758286008\\
135	0.011250392393878\\
136	0.0112504092671185\\
137	0.0112504264540835\\
138	0.0112504439606426\\
139	0.0112504617927756\\
140	0.0112504799565746\\
141	0.0112504984582462\\
142	0.0112505173041136\\
143	0.0112505365006185\\
144	0.0112505560543238\\
145	0.0112505759719158\\
146	0.0112505962602059\\
147	0.0112506169261338\\
148	0.0112506379767694\\
149	0.0112506594193153\\
150	0.0112506812611094\\
151	0.0112507035096272\\
152	0.0112507261724849\\
153	0.0112507492574412\\
154	0.0112507727724008\\
155	0.0112507967254167\\
156	0.0112508211246928\\
157	0.0112508459785873\\
158	0.0112508712956149\\
159	0.0112508970844504\\
160	0.011250923353931\\
161	0.0112509501130599\\
162	0.0112509773710091\\
163	0.0112510051371226\\
164	0.0112510334209195\\
165	0.0112510622320976\\
166	0.0112510915805363\\
167	0.0112511214763002\\
168	0.0112511519296425\\
169	0.0112511829510085\\
170	0.0112512145510392\\
171	0.0112512467405748\\
172	0.0112512795306586\\
173	0.0112513129325403\\
174	0.0112513469576804\\
175	0.0112513816177537\\
176	0.0112514169246535\\
177	0.0112514528904952\\
178	0.0112514895276208\\
179	0.0112515268486029\\
180	0.0112515648662489\\
181	0.0112516035936052\\
182	0.0112516430439617\\
183	0.0112516832308562\\
184	0.0112517241680788\\
185	0.0112517658696764\\
186	0.0112518083499575\\
187	0.0112518516234965\\
188	0.0112518957051383\\
189	0.0112519406100018\\
190	0.0112519863534837\\
191	0.0112520329512587\\
192	0.0112520804192762\\
193	0.0112521287737508\\
194	0.0112521780311441\\
195	0.0112522282081536\\
196	0.0112522793217732\\
197	0.011252331389639\\
198	0.0112523844311411\\
199	0.0112524384644899\\
200	0.0112524935081596\\
201	0.011252549580961\\
202	0.0112526067020474\\
203	0.0112526648909199\\
204	0.0112527241674335\\
205	0.0112527845518025\\
206	0.0112528460646068\\
207	0.0112529087267977\\
208	0.0112529725597037\\
209	0.0112530375850368\\
210	0.0112531038248981\\
211	0.0112531713017845\\
212	0.0112532400385947\\
213	0.0112533100586349\\
214	0.0112533813856257\\
215	0.0112534540437077\\
216	0.0112535280574485\\
217	0.0112536034518482\\
218	0.0112536802523461\\
219	0.011253758484827\\
220	0.0112538381756274\\
221	0.0112539193515416\\
222	0.0112540020398281\\
223	0.0112540862682158\\
224	0.0112541720649101\\
225	0.0112542594585992\\
226	0.0112543484784597\\
227	0.011254439154163\\
228	0.011254531515881\\
229	0.0112546255942918\\
230	0.0112547214205855\\
231	0.0112548190264698\\
232	0.0112549184441751\\
233	0.0112550197064599\\
234	0.0112551228466158\\
235	0.0112552278984725\\
236	0.0112553348964022\\
237	0.0112554438753241\\
238	0.0112555548707086\\
239	0.0112556679185811\\
240	0.011255783055526\\
241	0.0112559003186893\\
242	0.0112560197457823\\
243	0.0112561413750839\\
244	0.0112562652454431\\
245	0.011256391396281\\
246	0.0112565198675928\\
247	0.0112566506999486\\
248	0.0112567839344951\\
249	0.0112569196129558\\
250	0.011257057777632\\
251	0.0112571984714032\\
252	0.011257341737727\\
253	0.0112574876206394\\
254	0.0112576361647553\\
255	0.0112577874152692\\
256	0.0112579414179554\\
257	0.0112580982191708\\
258	0.0112582578658567\\
259	0.011258420405543\\
260	0.0112585858863542\\
261	0.0112587543570171\\
262	0.0112589258668722\\
263	0.011259100465888\\
264	0.0112592782046807\\
265	0.0112594591345397\\
266	0.0112596433074592\\
267	0.01125983077618\\
268	0.011260021594241\\
269	0.0112602158160431\\
270	0.0112604134969274\\
271	0.0112606146932656\\
272	0.0112608194625467\\
273	0.0112610278633887\\
274	0.0112612399551857\\
275	0.0112614557962181\\
276	0.0112616754355016\\
277	0.0112618988790575\\
278	0.0112621259504319\\
279	0.0112623543704692\\
280	0.0112625830980049\\
281	0.0112628162453201\\
282	0.0112630538980323\\
283	0.0112632961434183\\
284	0.011263543070446\\
285	0.0112637947698079\\
286	0.0112640513339533\\
287	0.011264312857123\\
288	0.0112645794353831\\
289	0.0112648511666608\\
290	0.0112651281507799\\
291	0.0112654104894972\\
292	0.0112656982865402\\
293	0.0112659916476442\\
294	0.0112662906805915\\
295	0.0112665954952508\\
296	0.0112669062036166\\
297	0.0112672229198507\\
298	0.0112675457603235\\
299	0.0112678748436561\\
300	0.011268210290764\\
301	0.0112685522249006\\
302	0.0112689007717021\\
303	0.0112692560592331\\
304	0.0112696182180331\\
305	0.0112699873811637\\
306	0.0112703636842569\\
307	0.0112707472655641\\
308	0.0112711382660062\\
309	0.0112715368292244\\
310	0.0112719431016321\\
311	0.0112723572324679\\
312	0.0112727793738494\\
313	0.0112732096808276\\
314	0.0112736483114433\\
315	0.0112740954267838\\
316	0.0112745511910407\\
317	0.0112750157715691\\
318	0.0112754893389479\\
319	0.0112759720670405\\
320	0.011276464133058\\
321	0.0112769657176224\\
322	0.0112774770048314\\
323	0.0112779981823246\\
324	0.0112785294413511\\
325	0.0112790709768378\\
326	0.0112796229874595\\
327	0.0112801856757107\\
328	0.0112807592479777\\
329	0.0112813439146138\\
330	0.0112819398900144\\
331	0.0112825473926947\\
332	0.0112831666453688\\
333	0.01128379787503\\
334	0.0112844413130339\\
335	0.0112850971951821\\
336	0.0112857657618086\\
337	0.0112864472578678\\
338	0.0112871419330251\\
339	0.0112878500417485\\
340	0.0112885718434041\\
341	0.0112893076023526\\
342	0.0112900575880488\\
343	0.0112908220751439\\
344	0.0112916013435909\\
345	0.0112923956787516\\
346	0.0112932053715086\\
347	0.0112940307183792\\
348	0.0112948720216337\\
349	0.0112957295894168\\
350	0.011296603735874\\
351	0.0112974947812813\\
352	0.0112984030521799\\
353	0.0112993288815165\\
354	0.0113002726087878\\
355	0.0113012345801915\\
356	0.0113022151487832\\
357	0.0113032146746397\\
358	0.0113042335250297\\
359	0.0113052720745917\\
360	0.0113063307055204\\
361	0.0113074098077615\\
362	0.0113085097792163\\
363	0.011309631025956\\
364	0.0113107739624468\\
365	0.0113119390117867\\
366	0.0113131266059543\\
367	0.0113143371860712\\
368	0.0113155712026775\\
369	0.0113168291160224\\
370	0.0113181113963706\\
371	0.0113194185243241\\
372	0.0113207509911615\\
373	0.0113221092991948\\
374	0.0113234939621433\\
375	0.011324905505526\\
376	0.01132634446707\\
377	0.0113278113971315\\
378	0.0113293068591164\\
379	0.0113308314298622\\
380	0.0113323856998622\\
381	0.0113339702729717\\
382	0.0113355857645042\\
383	0.0113372327945159\\
384	0.0113389119673765\\
385	0.0113406238156731\\
386	0.0113423686701932\\
387	0.0113441472092685\\
388	0.0113459607778204\\
389	0.01134781006563\\
390	0.0113496957760585\\
391	0.0113516186263208\\
392	0.0113535793477657\\
393	0.0113555786861575\\
394	0.0113576174019493\\
395	0.0113596962705739\\
396	0.0113618160827468\\
397	0.0113639776447653\\
398	0.0113661817788114\\
399	0.011368429323258\\
400	0.0113707211329771\\
401	0.0113730580796517\\
402	0.0113754410520882\\
403	0.0113778709565312\\
404	0.0113803487169902\\
405	0.0113828752755666\\
406	0.0113854515927638\\
407	0.0113880786478069\\
408	0.0113907574389615\\
409	0.0113934889838488\\
410	0.0113962743197575\\
411	0.0113991145039511\\
412	0.0114020106139695\\
413	0.0114049637479274\\
414	0.0114079750248104\\
415	0.0114110455847629\\
416	0.0114141765893302\\
417	0.011417369221582\\
418	0.0114206246863381\\
419	0.0114239442102216\\
420	0.0114273290414423\\
421	0.0114307804490434\\
422	0.0114342997212324\\
423	0.0114378881631094\\
424	0.0114415470988912\\
425	0.0114452779036574\\
426	0.0114490819871701\\
427	0.0114529607538761\\
428	0.0114569156328187\\
429	0.0114609480779095\\
430	0.0114650595681918\\
431	0.0114692516080948\\
432	0.0114735257276784\\
433	0.0114778834828656\\
434	0.0114823264556622\\
435	0.0114868562543634\\
436	0.0114914745137455\\
437	0.0114961828952417\\
438	0.0115009830871001\\
439	0.0115058768045239\\
440	0.0115108657897911\\
441	0.0115159518123515\\
442	0.0115211366688999\\
443	0.0115264221834176\\
444	0.0115318102071775\\
445	0.0115373026186999\\
446	0.011542901323659\\
447	0.0115486082547876\\
448	0.0115544253719716\\
449	0.0115603546629268\\
450	0.0115663981439583\\
451	0.0115725578561502\\
452	0.011578835867275\\
453	0.0115852342714826\\
454	0.0115917551889297\\
455	0.0115984007653465\\
456	0.0116051731715332\\
457	0.0116120746027831\\
458	0.0116191072782236\\
459	0.0116262734400695\\
460	0.0116335753527804\\
461	0.0116410153021135\\
462	0.0116485955940617\\
463	0.0116563185536657\\
464	0.0116641865236859\\
465	0.0116722018631164\\
466	0.0116803669455109\\
467	0.0116886841570585\\
468	0.0116971558942384\\
469	0.0117057845606337\\
470	0.0117145725624089\\
471	0.0117235223007023\\
472	0.0117326361569343\\
473	0.0117419164608546\\
474	0.0117513654123277\\
475	0.0117609849512421\\
476	0.0117707768689435\\
477	0.0117807399770495\\
478	0.0117908655562956\\
479	0.0118011617268628\\
480	0.0118116514230274\\
481	0.011822334062811\\
482	0.0118332183983219\\
483	0.0118443086771084\\
484	0.0118556078098222\\
485	0.0118671186114295\\
486	0.01187884384088\\
487	0.0118907861906651\\
488	0.0119029482763203\\
489	0.0119153326237612\\
490	0.0119279416557111\\
491	0.0119407776774582\\
492	0.0119538428626137\\
493	0.0119671392402373\\
494	0.0119806686859706\\
495	0.0119944329214376\\
496	0.0120084335260613\\
497	0.0120226719559619\\
498	0.0120371495318123\\
499	0.0120518673429515\\
500	0.0120668266351968\\
501	0.0120820304230073\\
502	0.0120974844612003\\
503	0.0121131971168465\\
504	0.0121291437198603\\
505	0.0121453627958154\\
506	0.0121619961882843\\
507	0.0121815443539876\\
508	0.012213429998921\\
509	0.0122447632873064\\
510	0.0122755592386346\\
511	0.0123013158101747\\
512	0.0123194706314159\\
513	0.0123364566204591\\
514	0.0123520919361984\\
515	0.0123641896744675\\
516	0.0123744562744946\\
517	0.0123845189755243\\
518	0.0123944068302158\\
519	0.01240432809404\\
520	0.0124140710555115\\
521	0.0124239474419283\\
522	0.0124340239328815\\
523	0.0124443063139268\\
524	0.0124548010089261\\
525	0.0124655214367356\\
526	0.0124764726054971\\
527	0.0124876593863447\\
528	0.0124990866772153\\
529	0.0125107592067008\\
530	0.0125226816636942\\
531	0.0125348590301723\\
532	0.0125472961796001\\
533	0.0125599957605532\\
534	0.0125729779788394\\
535	0.0125871711315262\\
536	0.0126019389678803\\
537	0.0126163386922887\\
538	0.0126295949239761\\
539	0.0126429435746475\\
540	0.0126562165524934\\
541	0.012669573720808\\
542	0.0126831516341576\\
543	0.0126969627284164\\
544	0.0127110078664425\\
545	0.0127252866437267\\
546	0.0127397957008046\\
547	0.0127545247444511\\
548	0.0127694529822806\\
549	0.0127846192099572\\
550	0.0128000241398811\\
551	0.0128156676347905\\
552	0.0128315486179904\\
553	0.0128482923356843\\
554	0.0128665773615686\\
555	0.0128828579104104\\
556	0.0128987644428459\\
557	0.0129145466087058\\
558	0.0129305193875654\\
559	0.0129466816218336\\
560	0.0129630293293708\\
561	0.0129795579970991\\
562	0.0129962626181625\\
563	0.0130131377802041\\
564	0.0130301779213039\\
565	0.0130473780268321\\
566	0.0130651559477278\\
567	0.0130831050302261\\
568	0.0131002111568546\\
569	0.013117283994186\\
570	0.0131344722037162\\
571	0.0131517663252298\\
572	0.0131691558724095\\
573	0.0131866292449119\\
574	0.0132041736323075\\
575	0.013221774909103\\
576	0.0132394175198042\\
577	0.0132570843528331\\
578	0.0132747566019253\\
579	0.0132924136134348\\
580	0.0133100327177776\\
581	0.0133275890431348\\
582	0.0133450553097115\\
583	0.0133624016039255\\
584	0.0133795951354571\\
585	0.0133965999904454\\
586	0.0134133769227543\\
587	0.0134298833028437\\
588	0.0134460735522599\\
589	0.0134619009495395\\
590	0.0134776600725805\\
591	0.0134935706885156\\
592	0.0135096572289508\\
593	0.0135260152836207\\
594	0.0135429334162338\\
595	0.0135612173030331\\
596	0.0135830518096797\\
597	0.0136142933738059\\
598	0.0136705746214114\\
599	0\\
600	0\\
};
\addplot [color=mycolor3,solid,forget plot]
  table[row sep=crcr]{%
1	0.0112398531188097\\
2	0.0112398550540577\\
3	0.0112398570239577\\
4	0.011239859029134\\
5	0.0112398610702224\\
6	0.0112398631478702\\
7	0.0112398652627364\\
8	0.0112398674154918\\
9	0.0112398696068196\\
10	0.0112398718374154\\
11	0.0112398741079874\\
12	0.0112398764192564\\
13	0.0112398787719567\\
14	0.0112398811668357\\
15	0.0112398836046544\\
16	0.0112398860861876\\
17	0.0112398886122243\\
18	0.0112398911835677\\
19	0.0112398938010356\\
20	0.0112398964654609\\
21	0.0112398991776913\\
22	0.0112399019385901\\
23	0.0112399047490363\\
24	0.0112399076099248\\
25	0.0112399105221669\\
26	0.0112399134866902\\
27	0.0112399165044396\\
28	0.0112399195763769\\
29	0.0112399227034813\\
30	0.0112399258867502\\
31	0.0112399291271989\\
32	0.0112399324258611\\
33	0.0112399357837895\\
34	0.0112399392020559\\
35	0.0112399426817517\\
36	0.0112399462239879\\
37	0.0112399498298962\\
38	0.0112399535006284\\
39	0.0112399572373577\\
40	0.0112399610412784\\
41	0.0112399649136067\\
42	0.011239968855581\\
43	0.0112399728684622\\
44	0.0112399769535342\\
45	0.0112399811121042\\
46	0.0112399853455035\\
47	0.0112399896550875\\
48	0.0112399940422363\\
49	0.0112399985083552\\
50	0.0112400030548751\\
51	0.0112400076832533\\
52	0.0112400123949733\\
53	0.0112400171915459\\
54	0.0112400220745093\\
55	0.0112400270454301\\
56	0.0112400321059031\\
57	0.0112400372575525\\
58	0.0112400425020321\\
59	0.0112400478410258\\
60	0.0112400532762485\\
61	0.0112400588094461\\
62	0.0112400644423966\\
63	0.0112400701769106\\
64	0.0112400760148315\\
65	0.0112400819580366\\
66	0.0112400880084374\\
67	0.0112400941679805\\
68	0.0112401004386479\\
69	0.0112401068224579\\
70	0.0112401133214658\\
71	0.0112401199377644\\
72	0.0112401266734849\\
73	0.0112401335307973\\
74	0.0112401405119114\\
75	0.0112401476190774\\
76	0.0112401548545868\\
77	0.011240162220773\\
78	0.0112401697200118\\
79	0.0112401773547229\\
80	0.0112401851273702\\
81	0.0112401930404625\\
82	0.0112402010965548\\
83	0.0112402092982487\\
84	0.0112402176481935\\
85	0.0112402261490872\\
86	0.0112402348036769\\
87	0.0112402436147602\\
88	0.011240252585186\\
89	0.0112402617178553\\
90	0.0112402710157223\\
91	0.0112402804817953\\
92	0.0112402901191378\\
93	0.0112402999308695\\
94	0.011240309920167\\
95	0.0112403200902654\\
96	0.0112403304444591\\
97	0.0112403409861027\\
98	0.0112403517186126\\
99	0.0112403626454675\\
100	0.0112403737702102\\
101	0.0112403850964482\\
102	0.0112403966278554\\
103	0.0112404083681729\\
104	0.0112404203212105\\
105	0.0112404324908479\\
106	0.011240444881036\\
107	0.0112404574957979\\
108	0.0112404703392308\\
109	0.0112404834155069\\
110	0.0112404967288751\\
111	0.0112405102836621\\
112	0.0112405240842741\\
113	0.0112405381351979\\
114	0.011240552441003\\
115	0.0112405670063423\\
116	0.0112405818359544\\
117	0.0112405969346648\\
118	0.0112406123073873\\
119	0.0112406279591261\\
120	0.0112406438949772\\
121	0.0112406601201299\\
122	0.0112406766398688\\
123	0.0112406934595756\\
124	0.0112407105847305\\
125	0.0112407280209144\\
126	0.0112407457738105\\
127	0.0112407638492062\\
128	0.0112407822529951\\
129	0.0112408009911791\\
130	0.0112408200698697\\
131	0.011240839495291\\
132	0.011240859273781\\
133	0.0112408794117937\\
134	0.0112408999159018\\
135	0.0112409207927984\\
136	0.0112409420492992\\
137	0.0112409636923449\\
138	0.0112409857290034\\
139	0.0112410081664721\\
140	0.0112410310120803\\
141	0.0112410542732917\\
142	0.0112410779577066\\
143	0.0112411020730646\\
144	0.0112411266272471\\
145	0.0112411516282797\\
146	0.011241177084335\\
147	0.0112412030037353\\
148	0.0112412293949553\\
149	0.0112412562666246\\
150	0.011241283627531\\
151	0.0112413114866229\\
152	0.0112413398530125\\
153	0.0112413687359787\\
154	0.01124139814497\\
155	0.0112414280896079\\
156	0.0112414585796896\\
157	0.0112414896251916\\
158	0.0112415212362726\\
159	0.011241553423277\\
160	0.0112415861967383\\
161	0.0112416195673825\\
162	0.0112416535461314\\
163	0.0112416881441063\\
164	0.0112417233726318\\
165	0.0112417592432392\\
166	0.0112417957676701\\
167	0.011241832957881\\
168	0.0112418708260462\\
169	0.0112419093845625\\
170	0.0112419486460529\\
171	0.0112419886233707\\
172	0.0112420293296039\\
173	0.0112420707780794\\
174	0.0112421129823674\\
175	0.0112421559562855\\
176	0.0112421997139037\\
177	0.011242244269549\\
178	0.0112422896378097\\
179	0.0112423358335407\\
180	0.011242382871868\\
181	0.0112424307681941\\
182	0.0112424795382031\\
183	0.0112425291978655\\
184	0.0112425797634442\\
185	0.0112426312514994\\
186	0.0112426836788943\\
187	0.0112427370628005\\
188	0.0112427914207028\\
189	0.0112428467704027\\
190	0.0112429031300189\\
191	0.0112429605179801\\
192	0.0112430189530018\\
193	0.0112430784540319\\
194	0.0112431390401441\\
195	0.0112432007303894\\
196	0.011243263543812\\
197	0.0112433275007307\\
198	0.0112433926299122\\
199	0.0112434589530833\\
200	0.0112435264920208\\
201	0.0112435952688908\\
202	0.011243665306255\\
203	0.0112437366270774\\
204	0.0112438092547309\\
205	0.0112438832130041\\
206	0.0112439585261077\\
207	0.0112440352186819\\
208	0.0112441133158028\\
209	0.0112441928429899\\
210	0.0112442738262124\\
211	0.0112443562918972\\
212	0.0112444402669353\\
213	0.0112445257786893\\
214	0.0112446128550005\\
215	0.0112447015241964\\
216	0.0112447918150977\\
217	0.0112448837570255\\
218	0.0112449773798089\\
219	0.0112450727137921\\
220	0.0112451697898416\\
221	0.0112452686393536\\
222	0.0112453692942608\\
223	0.01124547178704\\
224	0.0112455761507184\\
225	0.0112456824188812\\
226	0.0112457906256776\\
227	0.011245900805828\\
228	0.0112460129946297\\
229	0.0112461272279635\\
230	0.011246243542299\\
231	0.0112463619747005\\
232	0.0112464825628315\\
233	0.0112466053449598\\
234	0.0112467303599609\\
235	0.011246857647322\\
236	0.0112469872471444\\
237	0.0112471192001454\\
238	0.0112472535476591\\
239	0.0112473903316368\\
240	0.011247529594645\\
241	0.0112476713798632\\
242	0.0112478157310796\\
243	0.0112479626926845\\
244	0.0112481123096631\\
245	0.0112482646275842\\
246	0.0112484196925875\\
247	0.0112485775513678\\
248	0.0112487382511545\\
249	0.0112489018396894\\
250	0.0112490683651977\\
251	0.0112492378763555\\
252	0.011249410422251\\
253	0.0112495860523386\\
254	0.0112497648163866\\
255	0.0112499467644149\\
256	0.0112501319466237\\
257	0.0112503204133108\\
258	0.0112505122147758\\
259	0.0112507074012097\\
260	0.0112509060225672\\
261	0.0112511081284203\\
262	0.0112513137677893\\
263	0.0112515229889477\\
264	0.011251735839199\\
265	0.0112519523646189\\
266	0.0112521726097594\\
267	0.0112523966173079\\
268	0.0112526244276964\\
269	0.0112528560786523\\
270	0.0112530916046879\\
271	0.0112533310365277\\
272	0.0112535744005041\\
273	0.0112538217180451\\
274	0.0112540730057057\\
275	0.011254328277375\\
276	0.0112545875545408\\
277	0.0112548509062643\\
278	0.0112551186004842\\
279	0.0112553815469959\\
280	0.0112556331914203\\
281	0.0112558895529067\\
282	0.0112561507194839\\
283	0.0112564167808071\\
284	0.0112566878281876\\
285	0.0112569639546223\\
286	0.0112572452548246\\
287	0.0112575318252547\\
288	0.0112578237641519\\
289	0.0112581211715659\\
290	0.0112584241493897\\
291	0.0112587328013923\\
292	0.011259047233253\\
293	0.0112593675525949\\
294	0.0112596938690198\\
295	0.0112600262941438\\
296	0.0112603649416329\\
297	0.0112607099272394\\
298	0.0112610613688392\\
299	0.0112614193864689\\
300	0.0112617841023647\\
301	0.0112621556410006\\
302	0.011262534129128\\
303	0.0112629196958157\\
304	0.0112633124724904\\
305	0.011263712592978\\
306	0.011264120193545\\
307	0.0112645354129416\\
308	0.0112649583924437\\
309	0.0112653892758975\\
310	0.0112658282097626\\
311	0.0112662753431576\\
312	0.011266730827905\\
313	0.0112671948185771\\
314	0.0112676674725427\\
315	0.0112681489500138\\
316	0.0112686394140932\\
317	0.0112691390308228\\
318	0.0112696479692321\\
319	0.0112701664013871\\
320	0.0112706945024399\\
321	0.0112712324506792\\
322	0.0112717804275797\\
323	0.0112723386178541\\
324	0.0112729072095033\\
325	0.0112734863938684\\
326	0.0112740763656825\\
327	0.0112746773231224\\
328	0.0112752894678612\\
329	0.0112759130051204\\
330	0.0112765481437225\\
331	0.0112771950961434\\
332	0.0112778540785648\\
333	0.0112785253109268\\
334	0.0112792090169795\\
335	0.0112799054243355\\
336	0.0112806147645211\\
337	0.0112813372730273\\
338	0.0112820731893605\\
339	0.0112828227570923\\
340	0.0112835862239086\\
341	0.0112843638416577\\
342	0.0112851558663978\\
343	0.011285962558443\\
344	0.0112867841824079\\
345	0.0112876210072516\\
346	0.011288473306319\\
347	0.0112893413573817\\
348	0.0112902254426759\\
349	0.0112911258489398\\
350	0.0112920428674475\\
351	0.0112929767940417\\
352	0.0112939279291644\\
353	0.0112948965778843\\
354	0.011295883049923\\
355	0.0112968876596781\\
356	0.0112979107262447\\
357	0.0112989525734344\\
358	0.0113000135297927\\
359	0.0113010939286142\\
360	0.0113021941079574\\
361	0.0113033144106585\\
362	0.0113044551843449\\
363	0.0113056167814503\\
364	0.0113067995592318\\
365	0.0113080038797902\\
366	0.0113092301100968\\
367	0.0113104786220266\\
368	0.011311749792403\\
369	0.011313044003056\\
370	0.011314361640898\\
371	0.0113157030980229\\
372	0.0113170687718335\\
373	0.0113184590652045\\
374	0.0113198743866902\\
375	0.0113213151507862\\
376	0.0113227817782561\\
377	0.0113242746965376\\
378	0.0113257943402352\\
379	0.0113273411516913\\
380	0.011328915581545\\
381	0.0113305180888589\\
382	0.0113321491390907\\
383	0.0113338091929641\\
384	0.0113354986581789\\
385	0.01133721768925\\
386	0.0113389653597892\\
387	0.0113407286749514\\
388	0.0113425003424735\\
389	0.0113443057519865\\
390	0.0113461455390459\\
391	0.0113480203509496\\
392	0.011349930847039\\
393	0.011351877699096\\
394	0.0113538615917902\\
395	0.0113558832224729\\
396	0.0113579433013773\\
397	0.0113600425522479\\
398	0.011362181712684\\
399	0.0113643615344995\\
400	0.0113665827840942\\
401	0.0113688462428322\\
402	0.0113711527074176\\
403	0.0113735029902307\\
404	0.011375897919598\\
405	0.0113783383404624\\
406	0.0113808251152458\\
407	0.0113833591239784\\
408	0.0113859412648179\\
409	0.0113885724545928\\
410	0.011391253629367\\
411	0.0113939857450233\\
412	0.011396769777868\\
413	0.0113996067252743\\
414	0.0114024976064391\\
415	0.0114054434634291\\
416	0.0114084453627361\\
417	0.0114115043967779\\
418	0.0114146216802207\\
419	0.011417798352494\\
420	0.0114210355774191\\
421	0.0114243345391087\\
422	0.0114276964279419\\
423	0.0114311223985326\\
424	0.0114346134562151\\
425	0.0114381702031162\\
426	0.0114417941998605\\
427	0.0114454878012625\\
428	0.0114492523274892\\
429	0.0114530891232117\\
430	0.0114569995580219\\
431	0.0114609850268254\\
432	0.0114650469502567\\
433	0.0114691867751049\\
434	0.0114734059747216\\
435	0.0114777060494223\\
436	0.0114820885268803\\
437	0.0114865549625127\\
438	0.0114911069398559\\
439	0.0114957460709191\\
440	0.0115004739965148\\
441	0.0115052923865645\\
442	0.0115102029403635\\
443	0.0115152073867757\\
444	0.0115203074842916\\
445	0.0115255050207952\\
446	0.0115308018127076\\
447	0.011536199702938\\
448	0.0115417005572822\\
449	0.0115473062622548\\
450	0.0115530187439165\\
451	0.0115588400383558\\
452	0.0115647721402734\\
453	0.0115708170767832\\
454	0.0115769769076162\\
455	0.0115832537252857\\
456	0.0115896496552114\\
457	0.0115961668557969\\
458	0.0116028075184547\\
459	0.0116095738675752\\
460	0.011616468160423\\
461	0.0116234926869567\\
462	0.0116306497695561\\
463	0.0116379417626397\\
464	0.0116453710521476\\
465	0.011652940054867\\
466	0.0116606512176171\\
467	0.0116685070164475\\
468	0.0116765099562864\\
469	0.0116846625711881\\
470	0.0116929674193775\\
471	0.0117014270811989\\
472	0.0117100441567484\\
473	0.0117188212604146\\
474	0.011727761012032\\
475	0.0117368660194445\\
476	0.0117461388197515\\
477	0.0117555818509529\\
478	0.0117651975457344\\
479	0.0117749874007065\\
480	0.0117849495662895\\
481	0.0117950683519135\\
482	0.0118053751625677\\
483	0.0118158786655222\\
484	0.0118265789306442\\
485	0.0118374864357845\\
486	0.0118486044751374\\
487	0.0118599362846002\\
488	0.0118714850054656\\
489	0.0118832537105701\\
490	0.0118952453896086\\
491	0.0119074629337356\\
492	0.0119199091179936\\
493	0.011932586582716\\
494	0.0119454978141011\\
495	0.0119586451241416\\
496	0.011972030629777\\
497	0.0119856562312711\\
498	0.0119995235941901\\
499	0.0120136341565611\\
500	0.0120279891734304\\
501	0.0120425897351124\\
502	0.0120574367165883\\
503	0.0120725305977687\\
504	0.0120878736782381\\
505	0.0121034705492995\\
506	0.012119328402592\\
507	0.0121354360780582\\
508	0.0121517664015308\\
509	0.0121686145309986\\
510	0.0121862704871729\\
511	0.0122112620172072\\
512	0.0122440440222808\\
513	0.0122762600254102\\
514	0.0123076690699707\\
515	0.0123302400833033\\
516	0.0123491247969454\\
517	0.0123667981092209\\
518	0.0123830493986286\\
519	0.012395998317316\\
520	0.0124071173383049\\
521	0.0124178720894091\\
522	0.0124285027323542\\
523	0.0124392339915711\\
524	0.0124499928605226\\
525	0.0124606599612539\\
526	0.0124715411306862\\
527	0.0124826436028659\\
528	0.0124939718506357\\
529	0.0125055367495732\\
530	0.0125173475614404\\
531	0.012529409260057\\
532	0.0125417266354928\\
533	0.0125543049143144\\
534	0.0125671489273281\\
535	0.0125802617527497\\
536	0.0125936558856777\\
537	0.0126076037435631\\
538	0.0126229993775001\\
539	0.0126382562441207\\
540	0.0126525553135615\\
541	0.0126664288287315\\
542	0.0126803824849375\\
543	0.01269419443014\\
544	0.0127082291520141\\
545	0.012722491244862\\
546	0.0127369862321245\\
547	0.012751706727023\\
548	0.0127666303338323\\
549	0.0127817860228746\\
550	0.0127971825719994\\
551	0.0128128189922586\\
552	0.0128286943673395\\
553	0.0128448070893748\\
554	0.0128611619901994\\
555	0.0128796429233807\\
556	0.0128972235126808\\
557	0.0129135973940944\\
558	0.0129296493701558\\
559	0.0129458266155032\\
560	0.0129621905822272\\
561	0.0129787371564156\\
562	0.012995461304492\\
563	0.0130123574527557\\
564	0.0130294196035139\\
565	0.013046641638502\\
566	0.0130640176451161\\
567	0.0130820854068037\\
568	0.0131000574438603\\
569	0.0131172839926588\\
570	0.0131344722036746\\
571	0.0131517663252136\\
572	0.0131691558724018\\
573	0.013186629244908\\
574	0.0132041736323058\\
575	0.0132217749091023\\
576	0.0132394175198041\\
577	0.013257084352833\\
578	0.0132747566019253\\
579	0.0132924136134348\\
580	0.0133100327177776\\
581	0.0133275890431348\\
582	0.0133450553097115\\
583	0.0133624016039255\\
584	0.0133795951354571\\
585	0.0133965999904454\\
586	0.0134133769227543\\
587	0.0134298833028437\\
588	0.0134460735522599\\
589	0.0134619009495395\\
590	0.0134776600725805\\
591	0.0134935706885156\\
592	0.0135096572289508\\
593	0.0135260152836207\\
594	0.0135429334162338\\
595	0.0135612173030331\\
596	0.0135830518096797\\
597	0.0136142933738059\\
598	0.0136705746214114\\
599	0\\
600	0\\
};
\addplot [color=mycolor4,solid,forget plot]
  table[row sep=crcr]{%
1	0.0112219631116023\\
2	0.0112219647918909\\
3	0.011221966502336\\
4	0.0112219682434822\\
5	0.0112219700158839\\
6	0.0112219718201059\\
7	0.0112219736567231\\
8	0.0112219755263207\\
9	0.0112219774294949\\
10	0.0112219793668525\\
11	0.0112219813390113\\
12	0.0112219833466005\\
13	0.0112219853902607\\
14	0.0112219874706442\\
15	0.0112219895884151\\
16	0.0112219917442496\\
17	0.0112219939388363\\
18	0.0112219961728761\\
19	0.0112219984470831\\
20	0.0112220007621839\\
21	0.0112220031189188\\
22	0.0112220055180414\\
23	0.0112220079603188\\
24	0.0112220104465325\\
25	0.011222012977478\\
26	0.0112220155539653\\
27	0.0112220181768193\\
28	0.0112220208468799\\
29	0.0112220235650022\\
30	0.011222026332057\\
31	0.0112220291489311\\
32	0.0112220320165273\\
33	0.0112220349357649\\
34	0.01122203790758\\
35	0.011222040932926\\
36	0.0112220440127733\\
37	0.0112220471481103\\
38	0.0112220503399434\\
39	0.0112220535892975\\
40	0.0112220568972159\\
41	0.0112220602647613\\
42	0.0112220636930157\\
43	0.0112220671830807\\
44	0.0112220707360783\\
45	0.0112220743531509\\
46	0.0112220780354618\\
47	0.0112220817841956\\
48	0.0112220856005584\\
49	0.0112220894857787\\
50	0.0112220934411071\\
51	0.0112220974678173\\
52	0.0112221015672063\\
53	0.0112221057405948\\
54	0.0112221099893277\\
55	0.0112221143147744\\
56	0.0112221187183296\\
57	0.0112221232014134\\
58	0.0112221277654718\\
59	0.0112221324119776\\
60	0.0112221371424303\\
61	0.0112221419583571\\
62	0.0112221468613129\\
63	0.0112221518528814\\
64	0.0112221569346752\\
65	0.0112221621083365\\
66	0.0112221673755375\\
67	0.0112221727379813\\
68	0.0112221781974019\\
69	0.0112221837555656\\
70	0.0112221894142706\\
71	0.0112221951753484\\
72	0.0112222010406641\\
73	0.0112222070121169\\
74	0.0112222130916411\\
75	0.0112222192812062\\
76	0.0112222255828182\\
77	0.0112222319985198\\
78	0.0112222385303911\\
79	0.0112222451805506\\
80	0.0112222519511557\\
81	0.0112222588444035\\
82	0.0112222658625314\\
83	0.0112222730078179\\
84	0.0112222802825834\\
85	0.0112222876891912\\
86	0.0112222952300479\\
87	0.0112223029076042\\
88	0.0112223107243562\\
89	0.0112223186828458\\
90	0.0112223267856616\\
91	0.0112223350354401\\
92	0.011222343434866\\
93	0.0112223519866738\\
94	0.011222360693648\\
95	0.0112223695586246\\
96	0.0112223785844918\\
97	0.011222387774191\\
98	0.0112223971307178\\
99	0.011222406657123\\
100	0.0112224163565138\\
101	0.0112224262320543\\
102	0.0112224362869672\\
103	0.0112224465245346\\
104	0.0112224569480991\\
105	0.011222467561065\\
106	0.0112224783668992\\
107	0.0112224893691328\\
108	0.011222500571362\\
109	0.0112225119772492\\
110	0.0112225235905244\\
111	0.0112225354149868\\
112	0.0112225474545053\\
113	0.0112225597130203\\
114	0.0112225721945453\\
115	0.0112225849031673\\
116	0.0112225978430494\\
117	0.0112226110184312\\
118	0.0112226244336306\\
119	0.0112226380930454\\
120	0.0112226520011547\\
121	0.0112226661625201\\
122	0.0112226805817876\\
123	0.0112226952636891\\
124	0.0112227102130438\\
125	0.0112227254347602\\
126	0.0112227409338371\\
127	0.011222756715366\\
128	0.0112227727845324\\
129	0.0112227891466176\\
130	0.0112228058070005\\
131	0.0112228227711595\\
132	0.0112228400446741\\
133	0.011222857633227\\
134	0.0112228755426059\\
135	0.0112228937787054\\
136	0.0112229123475292\\
137	0.0112229312551917\\
138	0.0112229505079206\\
139	0.0112229701120584\\
140	0.0112229900740648\\
141	0.0112230104005189\\
142	0.0112230310981215\\
143	0.0112230521736968\\
144	0.0112230736341955\\
145	0.0112230954866962\\
146	0.0112231177384085\\
147	0.011223140396675\\
148	0.011223163468974\\
149	0.0112231869629215\\
150	0.0112232108862745\\
151	0.0112232352469327\\
152	0.0112232600529419\\
153	0.011223285312496\\
154	0.0112233110339402\\
155	0.0112233372257736\\
156	0.0112233638966519\\
157	0.0112233910553902\\
158	0.0112234187109664\\
159	0.0112234468725235\\
160	0.011223475549373\\
161	0.011223504750998\\
162	0.011223534487056\\
163	0.0112235647673823\\
164	0.0112235956019935\\
165	0.0112236270010902\\
166	0.0112236589750609\\
167	0.0112236915344848\\
168	0.0112237246901362\\
169	0.0112237584529871\\
170	0.0112237928342113\\
171	0.0112238278451881\\
172	0.0112238634975058\\
173	0.0112238998029658\\
174	0.0112239367735862\\
175	0.011223974421606\\
176	0.0112240127594891\\
177	0.0112240517999281\\
178	0.0112240915558491\\
179	0.0112241320404155\\
180	0.0112241732670322\\
181	0.0112242152493509\\
182	0.0112242580012735\\
183	0.0112243015369575\\
184	0.0112243458708205\\
185	0.0112243910175447\\
186	0.011224436992082\\
187	0.0112244838096589\\
188	0.0112245314857814\\
189	0.0112245800362401\\
190	0.0112246294771154\\
191	0.0112246798247824\\
192	0.0112247310959163\\
193	0.0112247833074971\\
194	0.0112248364768145\\
195	0.0112248906214715\\
196	0.0112249457593875\\
197	0.0112250019087989\\
198	0.0112250590882603\\
199	0.0112251173166673\\
200	0.0112251766132634\\
201	0.0112252369976472\\
202	0.0112252984897786\\
203	0.0112253611099861\\
204	0.0112254248789741\\
205	0.0112254898178298\\
206	0.0112255559480311\\
207	0.0112256232914541\\
208	0.0112256918703809\\
209	0.0112257617075078\\
210	0.0112258328259538\\
211	0.0112259052492689\\
212	0.0112259790014429\\
213	0.0112260541069146\\
214	0.011226130590581\\
215	0.0112262084778067\\
216	0.0112262877944345\\
217	0.0112263685667948\\
218	0.0112264508217167\\
219	0.0112265345865386\\
220	0.0112266198891199\\
221	0.0112267067578525\\
222	0.0112267952216732\\
223	0.011226885310076\\
224	0.0112269770531259\\
225	0.0112270704814721\\
226	0.0112271656263629\\
227	0.0112272625196605\\
228	0.0112273611938567\\
229	0.0112274616820899\\
230	0.0112275640181621\\
231	0.0112276682365574\\
232	0.0112277743724619\\
233	0.0112278824617834\\
234	0.0112279925411741\\
235	0.0112281046480534\\
236	0.0112282188206324\\
237	0.0112283350979411\\
238	0.0112284535198558\\
239	0.0112285741271304\\
240	0.0112286969614284\\
241	0.0112288220653588\\
242	0.0112289494825141\\
243	0.0112290792575118\\
244	0.0112292114360396\\
245	0.0112293460649046\\
246	0.0112294831920871\\
247	0.0112296228667998\\
248	0.0112297651395524\\
249	0.011229910062223\\
250	0.0112300576881372\\
251	0.0112302080721552\\
252	0.011230361270769\\
253	0.0112305173422095\\
254	0.0112306763465671\\
255	0.0112308383459257\\
256	0.0112310034045123\\
257	0.0112311715888661\\
258	0.0112313429680266\\
259	0.0112315176137475\\
260	0.0112316956007361\\
261	0.0112318770069248\\
262	0.0112320619137785\\
263	0.0112322504066425\\
264	0.0112324425751378\\
265	0.0112326385136114\\
266	0.0112328383216482\\
267	0.0112330421046556\\
268	0.011233249974532\\
269	0.0112334620504303\\
270	0.011233678459635\\
271	0.0112338993385677\\
272	0.0112341248339464\\
273	0.0112343551041282\\
274	0.0112345903206872\\
275	0.0112348306703359\\
276	0.0112350763574919\\
277	0.0112353276084654\\
278	0.0112355846807008\\
279	0.0112358592472386\\
280	0.0112361590995526\\
281	0.0112364643715693\\
282	0.011236775159877\\
283	0.0112370915627436\\
284	0.0112374136801452\\
285	0.0112377416137945\\
286	0.0112380754671694\\
287	0.0112384153455433\\
288	0.0112387613560147\\
289	0.0112391136075373\\
290	0.0112394722109517\\
291	0.0112398372790162\\
292	0.0112402089264391\\
293	0.011240587269911\\
294	0.0112409724281377\\
295	0.011241364521874\\
296	0.0112417636739574\\
297	0.0112421700093428\\
298	0.0112425836551374\\
299	0.0112430047406367\\
300	0.0112434333973604\\
301	0.0112438697590895\\
302	0.0112443139619032\\
303	0.0112447661442172\\
304	0.0112452264468221\\
305	0.0112456950129225\\
306	0.0112461719881768\\
307	0.0112466575207371\\
308	0.0112471517612909\\
309	0.0112476548631018\\
310	0.011248166982052\\
311	0.0112486882766852\\
312	0.0112492189082497\\
313	0.0112497590407424\\
314	0.0112503088409535\\
315	0.0112508684785118\\
316	0.01125143812593\\
317	0.0112520179586518\\
318	0.0112526081550983\\
319	0.0112532088967158\\
320	0.0112538203680237\\
321	0.0112544427566635\\
322	0.0112550762534474\\
323	0.0112557210524085\\
324	0.0112563773508502\\
325	0.0112570453493971\\
326	0.0112577252520458\\
327	0.0112584172662153\\
328	0.0112591216027994\\
329	0.0112598384762171\\
330	0.0112605681044646\\
331	0.011261310709167\\
332	0.0112620665156287\\
333	0.0112628357528854\\
334	0.0112636186537538\\
335	0.011264415454882\\
336	0.0112652263967984\\
337	0.0112660517239597\\
338	0.0112668916847973\\
339	0.0112677465317629\\
340	0.0112686165213711\\
341	0.0112695019142402\\
342	0.0112704029751301\\
343	0.0112713199729775\\
344	0.0112722531809255\\
345	0.0112732028763512\\
346	0.0112741693408857\\
347	0.0112751528604296\\
348	0.0112761537251607\\
349	0.0112771722295339\\
350	0.011278208672272\\
351	0.0112792633563449\\
352	0.0112803365889376\\
353	0.0112814286814034\\
354	0.0112825399492015\\
355	0.0112836707118152\\
356	0.01128482129265\\
357	0.011285992018907\\
358	0.0112871832214287\\
359	0.0112883952345134\\
360	0.0112896283956934\\
361	0.0112908830454722\\
362	0.0112921595270145\\
363	0.0112934581857824\\
364	0.0112947793691106\\
365	0.0112961234257108\\
366	0.0112974907050965\\
367	0.0112988815569155\\
368	0.0113002963301778\\
369	0.011301735372363\\
370	0.011303199028391\\
371	0.0113046876394337\\
372	0.0113062015415474\\
373	0.0113077410640972\\
374	0.0113093065279452\\
375	0.0113108982433649\\
376	0.0113125165076473\\
377	0.0113141616023556\\
378	0.0113158337901991\\
379	0.0113175333115385\\
380	0.0113192603806968\\
381	0.0113210151828205\\
382	0.0113227978740234\\
383	0.0113246085945726\\
384	0.0113264475302268\\
385	0.0113283151505388\\
386	0.0113302131080716\\
387	0.011332089795487\\
388	0.0113339007916197\\
389	0.0113357446154091\\
390	0.0113376218438527\\
391	0.0113395330621137\\
392	0.0113414788623804\\
393	0.0113434598442122\\
394	0.0113454766181136\\
395	0.0113475298186756\\
396	0.0113496200852277\\
397	0.0113517480532125\\
398	0.0113539143663893\\
399	0.0113561196770324\\
400	0.0113583646462316\\
401	0.0113606499442819\\
402	0.0113629762510956\\
403	0.0113653442563665\\
404	0.0113677546581359\\
405	0.0113702081537199\\
406	0.0113727054440872\\
407	0.0113752472542535\\
408	0.0113778343190787\\
409	0.0113804673834061\\
410	0.0113831472021713\\
411	0.0113858745404084\\
412	0.0113886501730216\\
413	0.011391474884139\\
414	0.0113943494660112\\
415	0.0113972747184201\\
416	0.0114002514533194\\
417	0.011403280519896\\
418	0.0114063628838984\\
419	0.0114094994360129\\
420	0.01141269106977\\
421	0.0114159386917957\\
422	0.0114192431985357\\
423	0.011422605375286\\
424	0.0114260254774431\\
425	0.0114295014979882\\
426	0.0114330050965013\\
427	0.011436534905361\\
428	0.0114401306301286\\
429	0.0114437935101228\\
430	0.0114475248101129\\
431	0.0114513258213043\\
432	0.0114551978612684\\
433	0.0114591422745426\\
434	0.0114631604339751\\
435	0.0114672537415922\\
436	0.0114714236294925\\
437	0.0114756715607744\\
438	0.0114799990305633\\
439	0.0114844075672708\\
440	0.0114888987336818\\
441	0.0114934741279946\\
442	0.011498135384972\\
443	0.0115028841770253\\
444	0.0115077222149977\\
445	0.0115126512479479\\
446	0.0115176730598178\\
447	0.0115227894567328\\
448	0.0115280022272361\\
449	0.0115333130297676\\
450	0.0115387231140325\\
451	0.0115442332895658\\
452	0.0115498480846319\\
453	0.011555569487059\\
454	0.0115613995210984\\
455	0.0115673402477864\\
456	0.0115733937652064\\
457	0.0115795622086724\\
458	0.0115858477508069\\
459	0.0115922526015383\\
460	0.0115987790081117\\
461	0.0116054292548502\\
462	0.0116122056628004\\
463	0.0116191105891438\\
464	0.0116261464262109\\
465	0.0116333155997407\\
466	0.0116406205657165\\
467	0.0116480638050156\\
468	0.0116556478173543\\
469	0.0116633751293362\\
470	0.0116712483819547\\
471	0.0116792702134409\\
472	0.011687443253407\\
473	0.011695770156151\\
474	0.0117042535976241\\
475	0.0117128962710672\\
476	0.0117217008817136\\
477	0.0117306701358281\\
478	0.01173980670663\\
479	0.011749113183503\\
480	0.0117585920207511\\
481	0.0117682458612118\\
482	0.0117780754743744\\
483	0.0117880776250861\\
484	0.0117982382958586\\
485	0.0118085958701235\\
486	0.0118191522215327\\
487	0.01182990946247\\
488	0.0118408767476743\\
489	0.0118520574067276\\
490	0.0118634547032141\\
491	0.01187507181162\\
492	0.0118869118390668\\
493	0.0118989778130187\\
494	0.0119112726685232\\
495	0.0119237992336361\\
496	0.0119365602140999\\
497	0.0119495581777026\\
498	0.0119627955392805\\
499	0.0119762745461344\\
500	0.0119899972612726\\
501	0.0120039655464266\\
502	0.0120181810550988\\
503	0.0120326452815836\\
504	0.0120473595567745\\
505	0.0120623250379274\\
506	0.012077542644494\\
507	0.0120930141182895\\
508	0.0121087442768055\\
509	0.0121247291509203\\
510	0.0121409615873548\\
511	0.0121575180636179\\
512	0.012174624778109\\
513	0.0121923924818932\\
514	0.0122111363681238\\
515	0.0122409719170328\\
516	0.0122747684081621\\
517	0.0123079808729794\\
518	0.0123403346499847\\
519	0.0123605770382437\\
520	0.0123801047733365\\
521	0.0123985686319261\\
522	0.0124155659603693\\
523	0.0124290737853909\\
524	0.0124419530231794\\
525	0.0124535382169708\\
526	0.0124650217823087\\
527	0.0124766234842425\\
528	0.0124883482818219\\
529	0.0125000172563007\\
530	0.0125117601209739\\
531	0.0125237389387467\\
532	0.0125359603508334\\
533	0.0125484274094362\\
534	0.0125611554279267\\
535	0.0125741502329515\\
536	0.0125874170402883\\
537	0.012600960392213\\
538	0.012614785398535\\
539	0.0126289124619574\\
540	0.0126443067400762\\
541	0.0126602468513451\\
542	0.0126759829699894\\
543	0.0126903979723286\\
544	0.01270491746666\\
545	0.0127194414524905\\
546	0.012733944604642\\
547	0.012748665469186\\
548	0.0127635932559511\\
549	0.0127787442047143\\
550	0.0127941396153654\\
551	0.0128097788369895\\
552	0.0128256607314815\\
553	0.0128417840478628\\
554	0.0128581477898656\\
555	0.0128747493877848\\
556	0.0128924913528037\\
557	0.012911335290226\\
558	0.0129282472267946\\
559	0.0129448443662115\\
560	0.0129612339774888\\
561	0.0129777991218968\\
562	0.0129945433838557\\
563	0.0130114613985217\\
564	0.0130285469905477\\
565	0.0130457935051899\\
566	0.0130631941735076\\
567	0.0130807421299556\\
568	0.0130990292238604\\
569	0.0131171111613697\\
570	0.0131344721844243\\
571	0.0131517663248718\\
572	0.0131691558722871\\
573	0.0131866292448517\\
574	0.0132041736322785\\
575	0.0132217749090898\\
576	0.0132394175197985\\
577	0.0132570843528308\\
578	0.0132747566019245\\
579	0.0132924136134347\\
580	0.0133100327177776\\
581	0.0133275890431348\\
582	0.0133450553097115\\
583	0.0133624016039255\\
584	0.0133795951354571\\
585	0.0133965999904454\\
586	0.0134133769227543\\
587	0.0134298833028437\\
588	0.0134460735522599\\
589	0.0134619009495395\\
590	0.0134776600725805\\
591	0.0134935706885156\\
592	0.0135096572289508\\
593	0.0135260152836207\\
594	0.0135429334162338\\
595	0.0135612173030331\\
596	0.0135830518096797\\
597	0.0136142933738059\\
598	0.0136705746214114\\
599	0\\
600	0\\
};
\addplot [color=mycolor5,solid,forget plot]
  table[row sep=crcr]{%
1	0.0112101608668663\\
2	0.0112101627483165\\
3	0.0112101646631256\\
4	0.0112101666118882\\
5	0.0112101685952093\\
6	0.0112101706137049\\
7	0.011210172668002\\
8	0.0112101747587388\\
9	0.0112101768865651\\
10	0.0112101790521421\\
11	0.0112101812561431\\
12	0.0112101834992532\\
13	0.0112101857821702\\
14	0.0112101881056039\\
15	0.0112101904702772\\
16	0.0112101928769259\\
17	0.011210195326299\\
18	0.0112101978191588\\
19	0.0112102003562814\\
20	0.0112102029384568\\
21	0.0112102055664893\\
22	0.0112102082411974\\
23	0.0112102109634145\\
24	0.0112102137339889\\
25	0.0112102165537842\\
26	0.0112102194236792\\
27	0.011210222344569\\
28	0.0112102253173643\\
29	0.0112102283429924\\
30	0.0112102314223974\\
31	0.01121023455654\\
32	0.0112102377463984\\
33	0.0112102409929684\\
34	0.0112102442972637\\
35	0.0112102476603161\\
36	0.011210251083176\\
37	0.0112102545669128\\
38	0.0112102581126151\\
39	0.0112102617213909\\
40	0.0112102653943684\\
41	0.011210269132696\\
42	0.0112102729375427\\
43	0.0112102768100986\\
44	0.0112102807515752\\
45	0.0112102847632059\\
46	0.0112102888462462\\
47	0.0112102930019743\\
48	0.0112102972316913\\
49	0.0112103015367219\\
50	0.0112103059184145\\
51	0.0112103103781421\\
52	0.011210314917302\\
53	0.011210319537317\\
54	0.0112103242396355\\
55	0.011210329025732\\
56	0.0112103338971075\\
57	0.0112103388552903\\
58	0.0112103439018361\\
59	0.0112103490383287\\
60	0.0112103542663804\\
61	0.0112103595876328\\
62	0.0112103650037569\\
63	0.0112103705164541\\
64	0.0112103761274565\\
65	0.0112103818385272\\
66	0.0112103876514615\\
67	0.011210393568087\\
68	0.0112103995902642\\
69	0.0112104057198875\\
70	0.0112104119588854\\
71	0.0112104183092212\\
72	0.0112104247728938\\
73	0.0112104313519383\\
74	0.0112104380484265\\
75	0.0112104448644678\\
76	0.0112104518022098\\
77	0.0112104588638388\\
78	0.0112104660515809\\
79	0.0112104733677023\\
80	0.0112104808145106\\
81	0.0112104883943549\\
82	0.0112104961096269\\
83	0.0112105039627617\\
84	0.0112105119562388\\
85	0.0112105200925823\\
86	0.0112105283743623\\
87	0.0112105368041955\\
88	0.0112105453847459\\
89	0.0112105541187263\\
90	0.0112105630088983\\
91	0.011210572058074\\
92	0.0112105812691163\\
93	0.0112105906449404\\
94	0.0112106001885143\\
95	0.01121060990286\\
96	0.0112106197910547\\
97	0.0112106298562312\\
98	0.0112106401015795\\
99	0.0112106505303478\\
100	0.0112106611458431\\
101	0.0112106719514331\\
102	0.0112106829505464\\
103	0.0112106941466744\\
104	0.011210705543372\\
105	0.011210717144259\\
106	0.0112107289530212\\
107	0.0112107409734116\\
108	0.0112107532092519\\
109	0.0112107656644334\\
110	0.0112107783429185\\
111	0.0112107912487422\\
112	0.0112108043860129\\
113	0.0112108177589144\\
114	0.0112108313717071\\
115	0.011210845228729\\
116	0.0112108593343977\\
117	0.011210873693212\\
118	0.0112108883097525\\
119	0.0112109031886842\\
120	0.0112109183347574\\
121	0.0112109337528097\\
122	0.0112109494477671\\
123	0.0112109654246465\\
124	0.0112109816885564\\
125	0.0112109982446995\\
126	0.0112110150983738\\
127	0.0112110322549748\\
128	0.0112110497199973\\
129	0.011211067499037\\
130	0.0112110855977926\\
131	0.0112111040220678\\
132	0.011211122777773\\
133	0.0112111418709275\\
134	0.0112111613076616\\
135	0.0112111810942185\\
136	0.0112112012369567\\
137	0.0112112217423517\\
138	0.0112112426169986\\
139	0.0112112638676143\\
140	0.0112112855010397\\
141	0.0112113075242418\\
142	0.0112113299443166\\
143	0.0112113527684911\\
144	0.0112113760041257\\
145	0.0112113996587171\\
146	0.0112114237399006\\
147	0.0112114482554524\\
148	0.0112114732132929\\
149	0.0112114986214889\\
150	0.0112115244882562\\
151	0.011211550821963\\
152	0.0112115776311319\\
153	0.0112116049244435\\
154	0.0112116327107389\\
155	0.0112116609990228\\
156	0.0112116897984664\\
157	0.0112117191184107\\
158	0.0112117489683692\\
159	0.0112117793580315\\
160	0.0112118102972661\\
161	0.011211841796124\\
162	0.0112118738648418\\
163	0.0112119065138451\\
164	0.0112119397537518\\
165	0.0112119735953757\\
166	0.0112120080497299\\
167	0.0112120431280304\\
168	0.0112120788416995\\
169	0.0112121152023699\\
170	0.0112121522218876\\
171	0.0112121899123166\\
172	0.0112122282859419\\
173	0.0112122673552738\\
174	0.0112123071330514\\
175	0.0112123476322472\\
176	0.0112123888660702\\
177	0.011212430847971\\
178	0.0112124735916449\\
179	0.0112125171110368\\
180	0.0112125614203455\\
181	0.0112126065340273\\
182	0.0112126524668013\\
183	0.0112126992336536\\
184	0.0112127468498415\\
185	0.0112127953308991\\
186	0.0112128446926414\\
187	0.0112128949511697\\
188	0.0112129461228766\\
189	0.0112129982244515\\
190	0.0112130512728861\\
191	0.0112131052854801\\
192	0.0112131602798474\\
193	0.0112132162739224\\
194	0.011213273285967\\
195	0.0112133313345773\\
196	0.0112133904386918\\
197	0.011213450617599\\
198	0.0112135118909466\\
199	0.0112135742787491\\
200	0.0112136378013945\\
201	0.0112137024796527\\
202	0.0112137683346823\\
203	0.0112138353880394\\
204	0.0112139036616854\\
205	0.011213973177995\\
206	0.0112140439597656\\
207	0.0112141160302252\\
208	0.0112141894130422\\
209	0.0112142641323342\\
210	0.011214340212678\\
211	0.011214417679119\\
212	0.0112144965571815\\
213	0.0112145768728792\\
214	0.0112146586527258\\
215	0.011214741923746\\
216	0.011214826713487\\
217	0.0112149130500298\\
218	0.011215000962002\\
219	0.0112150904785895\\
220	0.0112151816295498\\
221	0.0112152744452248\\
222	0.0112153689565549\\
223	0.011215465195093\\
224	0.0112155631930187\\
225	0.011215662983154\\
226	0.0112157645989779\\
227	0.0112158680746434\\
228	0.0112159734449934\\
229	0.0112160807455783\\
230	0.0112161900126733\\
231	0.0112163012832969\\
232	0.0112164145952297\\
233	0.011216529987034\\
234	0.0112166474980739\\
235	0.0112167671685357\\
236	0.0112168890394501\\
237	0.0112170131527134\\
238	0.0112171395511106\\
239	0.0112172682783388\\
240	0.0112173993790307\\
241	0.0112175328987793\\
242	0.0112176688841627\\
243	0.0112178073827698\\
244	0.0112179484432256\\
245	0.0112180921152175\\
246	0.0112182384495214\\
247	0.0112183874980273\\
248	0.0112185393137654\\
249	0.011218693950931\\
250	0.0112188514649088\\
251	0.0112190119122953\\
252	0.0112191753509203\\
253	0.011219341839865\\
254	0.0112195114394775\\
255	0.0112196842113839\\
256	0.0112198602184943\\
257	0.0112200395250024\\
258	0.0112202221963767\\
259	0.0112204082993429\\
260	0.0112205979018526\\
261	0.0112207910730386\\
262	0.0112209878831517\\
263	0.0112211884034755\\
264	0.0112213927062158\\
265	0.011221600864357\\
266	0.0112218129514816\\
267	0.011222029041543\\
268	0.011222249208583\\
269	0.0112224735263839\\
270	0.0112227020680405\\
271	0.0112229349054374\\
272	0.0112231721086131\\
273	0.0112234137449901\\
274	0.011223659878448\\
275	0.0112239105682173\\
276	0.011224165867576\\
277	0.0112244258223135\\
278	0.0112246904687509\\
279	0.0112249598299463\\
280	0.011225233943631\\
281	0.0112255128911928\\
282	0.0112257967553395\\
283	0.011226085620119\\
284	0.0112263795709384\\
285	0.0112266786945845\\
286	0.0112269830792447\\
287	0.0112272928145267\\
288	0.0112276079914804\\
289	0.0112279287026191\\
290	0.0112282550419406\\
291	0.0112285871049499\\
292	0.0112289249886813\\
293	0.0112292687917207\\
294	0.0112296186142295\\
295	0.0112299745579671\\
296	0.0112303367263155\\
297	0.0112307052243034\\
298	0.0112310801586304\\
299	0.0112314616376929\\
300	0.0112318497716092\\
301	0.0112322446722458\\
302	0.0112326464532441\\
303	0.0112330552300474\\
304	0.0112334711199289\\
305	0.0112338942420198\\
306	0.0112343247173387\\
307	0.0112347626688207\\
308	0.0112352082213485\\
309	0.011235661501783\\
310	0.0112361226389957\\
311	0.0112365917639011\\
312	0.0112370690094907\\
313	0.0112375545108679\\
314	0.0112380484052834\\
315	0.0112385508321722\\
316	0.0112390619331918\\
317	0.0112395818522613\\
318	0.0112401107356023\\
319	0.0112406487317809\\
320	0.0112411959917516\\
321	0.0112417526689025\\
322	0.0112423189191027\\
323	0.0112428949007515\\
324	0.0112434807748298\\
325	0.0112440767049532\\
326	0.0112446828574289\\
327	0.0112452994013136\\
328	0.0112459265084757\\
329	0.0112465643536596\\
330	0.0112472131145542\\
331	0.0112478729718642\\
332	0.0112485441093865\\
333	0.0112492267140897\\
334	0.0112499209761991\\
335	0.0112506270892869\\
336	0.0112513452503675\\
337	0.0112520756599988\\
338	0.011252818522391\\
339	0.011253574045522\\
340	0.0112543424412604\\
341	0.0112551239254985\\
342	0.0112559187182929\\
343	0.0112567270440178\\
344	0.0112575491315278\\
345	0.0112583852143343\\
346	0.0112592355307959\\
347	0.0112601003243235\\
348	0.0112609798436029\\
349	0.0112618743428354\\
350	0.0112627840819995\\
351	0.0112637093271356\\
352	0.0112646503506553\\
353	0.0112656074316792\\
354	0.0112665808564065\\
355	0.0112675709185182\\
356	0.0112685779196207\\
357	0.0112696021697308\\
358	0.0112706439878113\\
359	0.0112717037023593\\
360	0.0112727816520561\\
361	0.0112738781864858\\
362	0.0112749936669305\\
363	0.0112761284672526\\
364	0.0112772829748754\\
365	0.0112784575918742\\
366	0.0112796527361938\\
367	0.0112808688430078\\
368	0.0112821063662403\\
369	0.0112833657802722\\
370	0.0112846475818575\\
371	0.01128595229228\\
372	0.011287280459784\\
373	0.0112886326623202\\
374	0.0112900095106518\\
375	0.0112914116518759\\
376	0.0112928397734214\\
377	0.0112942946075978\\
378	0.0112957769367801\\
379	0.0112972875993332\\
380	0.0112988274964037\\
381	0.0113003975997556\\
382	0.0113019989609403\\
383	0.0113036327224349\\
384	0.011305300132499\\
385	0.011307002569437\\
386	0.0113087415953288\\
387	0.0113105838560625\\
388	0.0113125824178272\\
389	0.0113146151381459\\
390	0.0113166826165156\\
391	0.0113187854618731\\
392	0.0113209242858353\\
393	0.0113230997009292\\
394	0.0113253123319215\\
395	0.011327562892189\\
396	0.0113298520904981\\
397	0.0113321805795407\\
398	0.0113345490148004\\
399	0.0113369580537151\\
400	0.0113394083553105\\
401	0.0113419005805524\\
402	0.0113444353928242\\
403	0.0113470134555411\\
404	0.0113496354276487\\
405	0.0113523019203198\\
406	0.0113550135060556\\
407	0.0113577708266785\\
408	0.0113605745205874\\
409	0.0113634252207116\\
410	0.0113663235519349\\
411	0.0113692701276438\\
412	0.0113722655447279\\
413	0.0113753103758503\\
414	0.0113784051574498\\
415	0.0113815503738958\\
416	0.0113847464520389\\
417	0.0113879938432376\\
418	0.011391293518341\\
419	0.0113946459536104\\
420	0.0113980515008217\\
421	0.0114015104609364\\
422	0.0114050231131851\\
423	0.011408589844386\\
424	0.0114122116456016\\
425	0.0114158919786296\\
426	0.0114195123458516\\
427	0.0114230419398553\\
428	0.0114266340568707\\
429	0.0114302897729072\\
430	0.0114340101796216\\
431	0.0114377963902975\\
432	0.0114416495610825\\
433	0.0114455708599083\\
434	0.0114495614521669\\
435	0.0114536225211449\\
436	0.0114577552682258\\
437	0.0114619609125729\\
438	0.0114662406897633\\
439	0.0114705958506114\\
440	0.0114750276700149\\
441	0.0114795374485723\\
442	0.0114841265099247\\
443	0.0114887962023601\\
444	0.0114935479005594\\
445	0.0114983830068167\\
446	0.011503302948886\\
447	0.0115083091629009\\
448	0.0115134030147409\\
449	0.0115185854702249\\
450	0.0115238557332828\\
451	0.0115292009736181\\
452	0.0115345523717564\\
453	0.0115400037712459\\
454	0.0115455570896299\\
455	0.0115512142829569\\
456	0.0115569773477176\\
457	0.0115628483221231\\
458	0.0115688292873682\\
459	0.0115749223676445\\
460	0.011581129729191\\
461	0.0115874535847438\\
462	0.01159389619288\\
463	0.0116004598584973\\
464	0.0116071469322786\\
465	0.0116139598073329\\
466	0.0116209009072982\\
467	0.0116279726495704\\
468	0.0116351773399628\\
469	0.0116425168999938\\
470	0.011649992308052\\
471	0.0116576079759241\\
472	0.0116653677075215\\
473	0.0116732741169187\\
474	0.0116813298472841\\
475	0.0116895375690131\\
476	0.0116978999770756\\
477	0.0117064197878423\\
478	0.0117150997348694\\
479	0.0117239425627341\\
480	0.0117329510148544\\
481	0.0117421277830298\\
482	0.011751475469707\\
483	0.0117609965391068\\
484	0.0117706934857502\\
485	0.011780566259578\\
486	0.0117906095026309\\
487	0.0118008193942787\\
488	0.0118112275191317\\
489	0.0118218354975734\\
490	0.0118326468488757\\
491	0.0118436698765661\\
492	0.0118549079861272\\
493	0.0118663645256787\\
494	0.0118780427679161\\
495	0.0118899459311535\\
496	0.0119020771680832\\
497	0.0119144395546933\\
498	0.0119270360777342\\
499	0.0119398696226666\\
500	0.011952942959958\\
501	0.0119662587243142\\
502	0.0119798194057985\\
503	0.0119936273349418\\
504	0.0120076846689582\\
505	0.012021993382422\\
506	0.0120365552827791\\
507	0.0120513720258793\\
508	0.0120664450540655\\
509	0.0120817761847404\\
510	0.0120973679222172\\
511	0.012113219445742\\
512	0.0121293207801988\\
513	0.0121456634130861\\
514	0.0121625176813984\\
515	0.0121799157322206\\
516	0.0121979291211385\\
517	0.012216728126238\\
518	0.0122366161148493\\
519	0.0122704251274466\\
520	0.0123053504243618\\
521	0.0123396890013103\\
522	0.0123731009104478\\
523	0.0123930194170206\\
524	0.0124123264858313\\
525	0.0124317772324751\\
526	0.0124497989810461\\
527	0.0124642979658527\\
528	0.0124785250836554\\
529	0.0124916535760422\\
530	0.0125040759982029\\
531	0.0125166073242571\\
532	0.0125292823906416\\
533	0.0125421004655522\\
534	0.012554775781831\\
535	0.0125676804653018\\
536	0.0125808414237412\\
537	0.0125942639871736\\
538	0.0126079545685592\\
539	0.0126219262038969\\
540	0.0126361820347617\\
541	0.0126507342581248\\
542	0.0126656769091821\\
543	0.0126823216961768\\
544	0.0126988447472338\\
545	0.0127147776620054\\
546	0.0127298607063255\\
547	0.012745041866463\\
548	0.0127601555513124\\
549	0.012775307509993\\
550	0.0127906926872055\\
551	0.0128063197606412\\
552	0.0128221917182514\\
553	0.0128383075591147\\
554	0.0128546656018517\\
555	0.0128712640631596\\
556	0.0128880997214041\\
557	0.0129051788074997\\
558	0.0129243096547846\\
559	0.0129428503854944\\
560	0.0129598882759485\\
561	0.0129767248056191\\
562	0.0129934960861159\\
563	0.013010437382405\\
564	0.0130275479376466\\
565	0.0130448210998571\\
566	0.0130622494904639\\
567	0.0130798255633339\\
568	0.0130975414203238\\
569	0.0131159748673745\\
570	0.0131342565343236\\
571	0.0131517660703654\\
572	0.0131691558693043\\
573	0.0131866292440528\\
574	0.0132041736318864\\
575	0.0132217749088948\\
576	0.0132394175197059\\
577	0.0132570843527896\\
578	0.0132747566019076\\
579	0.0132924136134283\\
580	0.0133100327177755\\
581	0.0133275890431342\\
582	0.0133450553097114\\
583	0.0133624016039255\\
584	0.0133795951354571\\
585	0.0133965999904454\\
586	0.0134133769227543\\
587	0.0134298833028437\\
588	0.0134460735522599\\
589	0.0134619009495395\\
590	0.0134776600725805\\
591	0.0134935706885156\\
592	0.0135096572289508\\
593	0.0135260152836207\\
594	0.0135429334162338\\
595	0.0135612173030331\\
596	0.0135830518096797\\
597	0.0136142933738059\\
598	0.0136705746214114\\
599	0\\
600	0\\
};
\addplot [color=mycolor6,solid,forget plot]
  table[row sep=crcr]{%
1	0.0111879847775439\\
2	0.011187987163077\\
3	0.0111879895902234\\
4	0.0111879920597111\\
5	0.0111879945722808\\
6	0.0111879971286865\\
7	0.0111879997296951\\
8	0.011188002376087\\
9	0.0111880050686565\\
10	0.0111880078082116\\
11	0.0111880105955747\\
12	0.0111880134315825\\
13	0.0111880163170865\\
14	0.0111880192529528\\
15	0.0111880222400633\\
16	0.0111880252793148\\
17	0.0111880283716203\\
18	0.0111880315179085\\
19	0.0111880347191246\\
20	0.0111880379762302\\
21	0.0111880412902041\\
22	0.0111880446620421\\
23	0.0111880480927574\\
24	0.0111880515833811\\
25	0.0111880551349625\\
26	0.0111880587485692\\
27	0.0111880624252877\\
28	0.0111880661662233\\
29	0.011188069972501\\
30	0.0111880738452655\\
31	0.0111880777856816\\
32	0.0111880817949345\\
33	0.0111880858742303\\
34	0.0111880900247963\\
35	0.0111880942478814\\
36	0.0111880985447563\\
37	0.0111881029167142\\
38	0.0111881073650711\\
39	0.0111881118911659\\
40	0.0111881164963611\\
41	0.0111881211820434\\
42	0.0111881259496236\\
43	0.0111881308005373\\
44	0.0111881357362457\\
45	0.0111881407582354\\
46	0.0111881458680191\\
47	0.0111881510671362\\
48	0.0111881563571535\\
49	0.0111881617396648\\
50	0.0111881672162923\\
51	0.0111881727886867\\
52	0.0111881784585277\\
53	0.0111881842275245\\
54	0.0111881900974167\\
55	0.0111881960699741\\
56	0.011188202146998\\
57	0.0111882083303211\\
58	0.0111882146218087\\
59	0.0111882210233588\\
60	0.0111882275369026\\
61	0.0111882341644058\\
62	0.0111882409078684\\
63	0.0111882477693257\\
64	0.0111882547508488\\
65	0.0111882618545455\\
66	0.0111882690825606\\
67	0.0111882764370766\\
68	0.0111882839203148\\
69	0.0111882915345353\\
70	0.0111882992820383\\
71	0.0111883071651646\\
72	0.011188315186296\\
73	0.0111883233478566\\
74	0.0111883316523132\\
75	0.011188340102176\\
76	0.0111883486999998\\
77	0.0111883574483841\\
78	0.0111883663499746\\
79	0.0111883754074637\\
80	0.0111883846235911\\
81	0.0111883940011453\\
82	0.0111884035429636\\
83	0.0111884132519339\\
84	0.0111884231309948\\
85	0.011188433183137\\
86	0.011188443411404\\
87	0.0111884538188934\\
88	0.0111884644087571\\
89	0.0111884751842033\\
90	0.0111884861484966\\
91	0.0111884973049596\\
92	0.0111885086569737\\
93	0.0111885202079801\\
94	0.011188531961481\\
95	0.0111885439210407\\
96	0.0111885560902868\\
97	0.011188568472911\\
98	0.0111885810726706\\
99	0.0111885938933896\\
100	0.0111886069389598\\
101	0.0111886202133421\\
102	0.0111886337205678\\
103	0.0111886474647399\\
104	0.0111886614500342\\
105	0.0111886756807007\\
106	0.0111886901610652\\
107	0.0111887048955304\\
108	0.0111887198885773\\
109	0.0111887351447668\\
110	0.011188750668741\\
111	0.0111887664652249\\
112	0.0111887825390276\\
113	0.0111887988950442\\
114	0.0111888155382571\\
115	0.0111888324737376\\
116	0.0111888497066479\\
117	0.0111888672422424\\
118	0.0111888850858694\\
119	0.0111889032429732\\
120	0.0111889217190954\\
121	0.0111889405198773\\
122	0.011188959651061\\
123	0.011188979118492\\
124	0.0111889989281206\\
125	0.0111890190860044\\
126	0.0111890395983097\\
127	0.0111890604713138\\
128	0.0111890817114075\\
129	0.0111891033250965\\
130	0.011189125319004\\
131	0.0111891476998732\\
132	0.0111891704745688\\
133	0.0111891936500801\\
134	0.0111892172335228\\
135	0.0111892412321418\\
136	0.0111892656533136\\
137	0.0111892905045486\\
138	0.011189315793494\\
139	0.0111893415279361\\
140	0.0111893677158034\\
141	0.0111893943651689\\
142	0.0111894214842535\\
143	0.0111894490814282\\
144	0.0111894771652175\\
145	0.0111895057443024\\
146	0.0111895348275231\\
147	0.0111895644238826\\
148	0.0111895945425496\\
149	0.0111896251928619\\
150	0.0111896563843299\\
151	0.0111896881266395\\
152	0.0111897204296562\\
153	0.0111897533034283\\
154	0.0111897867581909\\
155	0.0111898208043691\\
156	0.0111898554525823\\
157	0.0111898907136477\\
158	0.0111899265985847\\
159	0.0111899631186185\\
160	0.0111900002851843\\
161	0.0111900381099318\\
162	0.0111900766047293\\
163	0.0111901157816678\\
164	0.0111901556530658\\
165	0.0111901962314736\\
166	0.011190237529678\\
167	0.0111902795607068\\
168	0.0111903223378334\\
169	0.011190365874582\\
170	0.0111904101847319\\
171	0.0111904552823226\\
172	0.0111905011816588\\
173	0.0111905478973148\\
174	0.0111905954441402\\
175	0.0111906438372643\\
176	0.0111906930921011\\
177	0.0111907432243544\\
178	0.0111907942500224\\
179	0.0111908461854027\\
180	0.0111908990470971\\
181	0.0111909528520158\\
182	0.0111910076173823\\
183	0.0111910633607378\\
184	0.0111911200999447\\
185	0.0111911778531913\\
186	0.0111912366389949\\
187	0.0111912964762052\\
188	0.0111913573840078\\
189	0.0111914193819259\\
190	0.0111914824898235\\
191	0.0111915467279064\\
192	0.0111916121167237\\
193	0.0111916786771687\\
194	0.0111917464304798\\
195	0.0111918153982412\\
196	0.0111918856023842\\
197	0.0111919570651895\\
198	0.0111920298092964\\
199	0.0111921038577525\\
200	0.0111921792340235\\
201	0.0111922559620008\\
202	0.0111923340660094\\
203	0.0111924135708154\\
204	0.011192494501635\\
205	0.0111925768841417\\
206	0.0111926607444755\\
207	0.0111927461092511\\
208	0.0111928330055667\\
209	0.0111929214610126\\
210	0.0111930115036804\\
211	0.0111931031621719\\
212	0.0111931964656087\\
213	0.0111932914436409\\
214	0.0111933881264576\\
215	0.0111934865447956\\
216	0.0111935867299496\\
217	0.0111936887137824\\
218	0.0111937925287342\\
219	0.0111938982078333\\
220	0.0111940057847063\\
221	0.0111941152935881\\
222	0.0111942267693325\\
223	0.0111943402474227\\
224	0.0111944557639819\\
225	0.0111945733557838\\
226	0.0111946930602629\\
227	0.0111948149155257\\
228	0.0111949389603607\\
229	0.0111950652342495\\
230	0.0111951937773766\\
231	0.0111953246306401\\
232	0.0111954578356622\\
233	0.0111955934347984\\
234	0.0111957314711479\\
235	0.011195871988563\\
236	0.0111960150316579\\
237	0.0111961606458179\\
238	0.0111963088772072\\
239	0.0111964597727771\\
240	0.0111966133802726\\
241	0.0111967697482393\\
242	0.0111969289260285\\
243	0.0111970909638022\\
244	0.0111972559125368\\
245	0.0111974238240253\\
246	0.0111975947508787\\
247	0.0111977687465261\\
248	0.0111979458652127\\
249	0.0111981261619966\\
250	0.0111983096927437\\
251	0.0111984965141203\\
252	0.0111986866835844\\
253	0.0111988802593735\\
254	0.0111990773004914\\
255	0.0111992778666916\\
256	0.0111994820184582\\
257	0.0111996898169853\\
258	0.0111999013241523\\
259	0.0112001166024978\\
260	0.0112003357151917\\
261	0.0112005587260041\\
262	0.0112007856992746\\
263	0.0112010166998798\\
264	0.0112012517932024\\
265	0.0112014910451018\\
266	0.0112017345218889\\
267	0.0112019822903078\\
268	0.0112022344175265\\
269	0.011202490971143\\
270	0.0112027520192095\\
271	0.0112030176302845\\
272	0.011203287873519\\
273	0.0112035628187903\\
274	0.0112038425368954\\
275	0.0112041270998234\\
276	0.0112044165811265\\
277	0.0112047110564122\\
278	0.0112050106039834\\
279	0.0112053153057215\\
280	0.0112056252466757\\
281	0.011205940513176\\
282	0.0112062611928486\\
283	0.0112065873746312\\
284	0.0112069191487878\\
285	0.0112072566069249\\
286	0.0112075998420059\\
287	0.0112079489483672\\
288	0.0112083040217331\\
289	0.0112086651592315\\
290	0.0112090324594093\\
291	0.0112094060222478\\
292	0.0112097859491782\\
293	0.0112101723430971\\
294	0.0112105653083819\\
295	0.0112109649509062\\
296	0.0112113713780555\\
297	0.0112117846987427\\
298	0.0112122050234231\\
299	0.0112126324641105\\
300	0.0112130671343922\\
301	0.0112135091494448\\
302	0.0112139586260494\\
303	0.0112144156826071\\
304	0.0112148804391548\\
305	0.0112153530173802\\
306	0.0112158335406378\\
307	0.0112163221339642\\
308	0.0112168189240935\\
309	0.0112173240394736\\
310	0.0112178376102813\\
311	0.0112183597684384\\
312	0.0112188906476277\\
313	0.011219430383309\\
314	0.0112199791127354\\
315	0.0112205369749696\\
316	0.0112211041109008\\
317	0.0112216806632619\\
318	0.0112222667766462\\
319	0.0112228625975256\\
320	0.0112234682742688\\
321	0.0112240839571594\\
322	0.011224709798416\\
323	0.0112253459522111\\
324	0.0112259925746929\\
325	0.0112266498240057\\
326	0.0112273178603132\\
327	0.011227996845822\\
328	0.0112286869448059\\
329	0.0112293883236327\\
330	0.0112301011507916\\
331	0.0112308255969226\\
332	0.0112315618348483\\
333	0.0112323100396066\\
334	0.0112330703884871\\
335	0.0112338430610692\\
336	0.0112346282392635\\
337	0.0112354261073558\\
338	0.011236236852055\\
339	0.0112370606625449\\
340	0.0112378977305394\\
341	0.0112387482503428\\
342	0.0112396124189146\\
343	0.0112404904359398\\
344	0.0112413825039048\\
345	0.0112422888281792\\
346	0.0112432096171045\\
347	0.0112441450820899\\
348	0.0112450954377148\\
349	0.0112460609018405\\
350	0.0112470416957294\\
351	0.0112480380441729\\
352	0.0112490501756291\\
353	0.0112500783223696\\
354	0.0112511227206358\\
355	0.0112521836108048\\
356	0.0112532612375661\\
357	0.0112543558501058\\
358	0.0112554677023017\\
359	0.0112565970529245\\
360	0.0112577441658474\\
361	0.0112589093102599\\
362	0.011260092760884\\
363	0.0112612947981909\\
364	0.0112625157086126\\
365	0.0112637557847435\\
366	0.011265015325527\\
367	0.0112662946364165\\
368	0.0112675940295027\\
369	0.0112689138235934\\
370	0.0112702543442306\\
371	0.0112716159236259\\
372	0.0112729989004906\\
373	0.0112744036197322\\
374	0.0112758304319824\\
375	0.0112772796929156\\
376	0.011278751762305\\
377	0.0112802470027573\\
378	0.0112817657780495\\
379	0.011283308450982\\
380	0.0112848753806408\\
381	0.0112864669189496\\
382	0.0112880834063808\\
383	0.0112897251666954\\
384	0.0112913925006017\\
385	0.011293085678142\\
386	0.0112948049285913\\
387	0.0112965504199716\\
388	0.0112983224157504\\
389	0.011300121462252\\
390	0.0113019481331103\\
391	0.0113038030305512\\
392	0.0113056867865641\\
393	0.0113076000639022\\
394	0.0113095435568472\\
395	0.0113115179916952\\
396	0.0113135241270456\\
397	0.0113155627535793\\
398	0.0113176346931037\\
399	0.0113197407970645\\
400	0.0113218819452529\\
401	0.0113240590469026\\
402	0.0113262730472155\\
403	0.0113285249330422\\
404	0.0113308156256021\\
405	0.0113331458849139\\
406	0.0113355165001309\\
407	0.0113379282920113\\
408	0.0113403821158681\\
409	0.0113428788648933\\
410	0.0113454194739131\\
411	0.0113480049236316\\
412	0.0113506362454351\\
413	0.0113533145268316\\
414	0.0113560409176102\\
415	0.0113588166368083\\
416	0.0113616429805733\\
417	0.0113645213310354\\
418	0.0113674531665187\\
419	0.0113704400747423\\
420	0.0113734837670294\\
421	0.011376586094814\\
422	0.011379749069825\\
423	0.011382974891436\\
424	0.0113862659926398\\
425	0.0113896251453567\\
426	0.0113932026472093\\
427	0.0113970331840979\\
428	0.0114009269787139\\
429	0.0114048850043423\\
430	0.0114089082330413\\
431	0.0114129976538403\\
432	0.0114171543953875\\
433	0.0114213795754421\\
434	0.0114256742137696\\
435	0.0114300393292169\\
436	0.0114344759361172\\
437	0.0114389850376395\\
438	0.0114435676118684\\
439	0.0114482245920296\\
440	0.0114529568996611\\
441	0.0114577654422773\\
442	0.0114626510849059\\
443	0.0114676146402823\\
444	0.0114726568581525\\
445	0.0114777784149752\\
446	0.0114829799088692\\
447	0.0114882618770652\\
448	0.0114936248973209\\
449	0.0114990699957101\\
450	0.0115046001839403\\
451	0.011510189560996\\
452	0.0115155073743129\\
453	0.0115209189922193\\
454	0.0115264260600362\\
455	0.0115320302180047\\
456	0.0115377331130457\\
457	0.0115435364164562\\
458	0.0115494418325196\\
459	0.0115554511233982\\
460	0.0115615660949551\\
461	0.0115677884840345\\
462	0.0115741201141942\\
463	0.0115805628433339\\
464	0.011587118567149\\
465	0.0115937892175781\\
466	0.0116005767512082\\
467	0.0116074830893976\\
468	0.0116145098592357\\
469	0.0116216573171839\\
470	0.0116289198979487\\
471	0.0116362215049877\\
472	0.0116436155834413\\
473	0.0116511472586251\\
474	0.0116588190978156\\
475	0.0116666337097441\\
476	0.0116745937462632\\
477	0.0116827018910367\\
478	0.0116909608632551\\
479	0.0116993734175702\\
480	0.0117079423406494\\
481	0.0117166704457158\\
482	0.0117255605552234\\
483	0.0117346154519109\\
484	0.0117438377268174\\
485	0.0117532294934983\\
486	0.0117627918414849\\
487	0.0117725273976635\\
488	0.0117824385323567\\
489	0.0117925184431706\\
490	0.0118027701199059\\
491	0.0118132206664018\\
492	0.0118238716898225\\
493	0.01183472726414\\
494	0.0118457955321458\\
495	0.0118570800272307\\
496	0.0118685842367818\\
497	0.011880311566037\\
498	0.0118922653599716\\
499	0.0119044488909241\\
500	0.011916865404098\\
501	0.0119295181971444\\
502	0.0119424103480829\\
503	0.011955544836209\\
504	0.011968924529171\\
505	0.0119825521709068\\
506	0.0119964303703711\\
507	0.012010561590512\\
508	0.0120249481458056\\
509	0.0120395921766152\\
510	0.0120544956305107\\
511	0.012069660522155\\
512	0.0120850896310963\\
513	0.0121007867246678\\
514	0.0121167420785275\\
515	0.0121329441821784\\
516	0.0121494912815763\\
517	0.012166605931786\\
518	0.0121843043937916\\
519	0.0122026186341278\\
520	0.0122216398657011\\
521	0.012241505706393\\
522	0.0122625957292709\\
523	0.0122992080608472\\
524	0.0123353882993247\\
525	0.0123709988176049\\
526	0.0124055941493639\\
527	0.0124265473647618\\
528	0.0124465012094762\\
529	0.0124664817033657\\
530	0.0124858067773764\\
531	0.0125017301000397\\
532	0.0125170900344906\\
533	0.0125322269587333\\
534	0.0125457692661988\\
535	0.0125592992856946\\
536	0.0125729816056132\\
537	0.012586827243525\\
538	0.0126007503092802\\
539	0.0126146422890886\\
540	0.012628803109012\\
541	0.0126432401785412\\
542	0.0126579573564621\\
543	0.0126729628292463\\
544	0.0126882799837062\\
545	0.0127044550013907\\
546	0.0127217397286062\\
547	0.0127388976942583\\
548	0.0127551857632212\\
549	0.0127709183873788\\
550	0.012786753411821\\
551	0.012802519587187\\
552	0.0128184044427045\\
553	0.0128345237872479\\
554	0.0128508847341814\\
555	0.0128674892295274\\
556	0.0128843354164019\\
557	0.0129014211146582\\
558	0.0129187417855153\\
559	0.0129368450096861\\
560	0.0129564410756787\\
561	0.0129747448770842\\
562	0.0129921292222182\\
563	0.0130092533046203\\
564	0.0130263961357387\\
565	0.0130436991350803\\
566	0.0130611590542131\\
567	0.0130787679451915\\
568	0.0130965173617682\\
569	0.0131143982640872\\
570	0.0131329043089112\\
571	0.0131514795455432\\
572	0.01316915215331\\
573	0.0131866292157513\\
574	0.0132041736263707\\
575	0.0132217749062404\\
576	0.0132394175183562\\
577	0.0132570843521267\\
578	0.0132747566016014\\
579	0.0132924136132975\\
580	0.013310032717725\\
581	0.0133275890431169\\
582	0.0133450553097063\\
583	0.0133624016039243\\
584	0.013379595135457\\
585	0.0133965999904454\\
586	0.0134133769227543\\
587	0.0134298833028437\\
588	0.0134460735522599\\
589	0.0134619009495395\\
590	0.0134776600725805\\
591	0.0134935706885156\\
592	0.0135096572289508\\
593	0.0135260152836207\\
594	0.0135429334162338\\
595	0.0135612173030331\\
596	0.0135830518096797\\
597	0.0136142933738059\\
598	0.0136705746214114\\
599	0\\
600	0\\
};
\addplot [color=mycolor7,solid,forget plot]
  table[row sep=crcr]{%
1	0.0111345123431092\\
2	0.0111345157870999\\
3	0.011134519290245\\
4	0.011134522853561\\
5	0.0111345264780816\\
6	0.0111345301648587\\
7	0.0111345339149619\\
8	0.0111345377294792\\
9	0.0111345416095176\\
10	0.0111345455562027\\
11	0.0111345495706798\\
12	0.0111345536541135\\
13	0.0111345578076887\\
14	0.0111345620326106\\
15	0.0111345663301049\\
16	0.0111345707014187\\
17	0.0111345751478201\\
18	0.0111345796705993\\
19	0.0111345842710685\\
20	0.0111345889505625\\
21	0.0111345937104389\\
22	0.0111345985520789\\
23	0.011134603476887\\
24	0.011134608486292\\
25	0.0111346135817475\\
26	0.0111346187647315\\
27	0.0111346240367479\\
28	0.0111346293993262\\
29	0.0111346348540219\\
30	0.0111346404024177\\
31	0.011134646046123\\
32	0.0111346517867752\\
33	0.0111346576260395\\
34	0.0111346635656099\\
35	0.0111346696072093\\
36	0.0111346757525903\\
37	0.0111346820035354\\
38	0.0111346883618579\\
39	0.011134694829402\\
40	0.0111347014080438\\
41	0.0111347080996913\\
42	0.0111347149062854\\
43	0.0111347218298002\\
44	0.0111347288722437\\
45	0.0111347360356585\\
46	0.0111347433221219\\
47	0.0111347507337472\\
48	0.0111347582726836\\
49	0.0111347659411173\\
50	0.0111347737412721\\
51	0.0111347816754097\\
52	0.0111347897458309\\
53	0.0111347979548755\\
54	0.0111348063049239\\
55	0.0111348147983969\\
56	0.011134823437757\\
57	0.0111348322255089\\
58	0.0111348411642001\\
59	0.0111348502564221\\
60	0.0111348595048103\\
61	0.0111348689120457\\
62	0.011134878480855\\
63	0.0111348882140117\\
64	0.0111348981143368\\
65	0.0111349081846997\\
66	0.0111349184280189\\
67	0.0111349288472629\\
68	0.0111349394454509\\
69	0.0111349502256542\\
70	0.0111349611909963\\
71	0.0111349723446543\\
72	0.0111349836898599\\
73	0.0111349952298999\\
74	0.0111350069681175\\
75	0.0111350189079132\\
76	0.0111350310527455\\
77	0.0111350434061324\\
78	0.0111350559716519\\
79	0.0111350687529433\\
80	0.0111350817537082\\
81	0.0111350949777116\\
82	0.0111351084287829\\
83	0.0111351221108171\\
84	0.0111351360277757\\
85	0.0111351501836881\\
86	0.0111351645826527\\
87	0.0111351792288379\\
88	0.0111351941264836\\
89	0.0111352092799019\\
90	0.0111352246934791\\
91	0.0111352403716761\\
92	0.0111352563190306\\
93	0.0111352725401575\\
94	0.0111352890397508\\
95	0.0111353058225848\\
96	0.0111353228935154\\
97	0.0111353402574817\\
98	0.0111353579195071\\
99	0.011135375884701\\
100	0.0111353941582602\\
101	0.0111354127454703\\
102	0.0111354316517075\\
103	0.0111354508824396\\
104	0.0111354704432282\\
105	0.0111354903397299\\
106	0.0111355105776981\\
107	0.0111355311629847\\
108	0.0111355521015414\\
109	0.011135573399422\\
110	0.0111355950627836\\
111	0.011135617097889\\
112	0.0111356395111078\\
113	0.0111356623089186\\
114	0.0111356854979112\\
115	0.0111357090847877\\
116	0.0111357330763654\\
117	0.0111357574795778\\
118	0.0111357823014775\\
119	0.0111358075492378\\
120	0.0111358332301546\\
121	0.0111358593516491\\
122	0.0111358859212695\\
123	0.0111359129466933\\
124	0.0111359404357299\\
125	0.0111359683963222\\
126	0.0111359968365498\\
127	0.0111360257646307\\
128	0.011136055188924\\
129	0.0111360851179323\\
130	0.0111361155603045\\
131	0.0111361465248381\\
132	0.0111361780204818\\
133	0.0111362100563384\\
134	0.0111362426416676\\
135	0.0111362757858885\\
136	0.0111363094985827\\
137	0.0111363437894972\\
138	0.0111363786685473\\
139	0.0111364141458198\\
140	0.0111364502315762\\
141	0.0111364869362555\\
142	0.0111365242704781\\
143	0.0111365622450488\\
144	0.0111366008709601\\
145	0.0111366401593962\\
146	0.0111366801217362\\
147	0.0111367207695583\\
148	0.011136762114643\\
149	0.0111368041689778\\
150	0.0111368469447604\\
151	0.0111368904544038\\
152	0.0111369347105398\\
153	0.0111369797260238\\
154	0.0111370255139395\\
155	0.0111370720876032\\
156	0.011137119460569\\
157	0.0111371676466339\\
158	0.0111372166598426\\
159	0.0111372665144932\\
160	0.011137317225143\\
161	0.0111373688066137\\
162	0.0111374212739981\\
163	0.0111374746426657\\
164	0.0111375289282699\\
165	0.0111375841467542\\
166	0.0111376403143597\\
167	0.0111376974476321\\
168	0.0111377555634296\\
169	0.0111378146789312\\
170	0.0111378748116444\\
171	0.0111379359794147\\
172	0.0111379982004342\\
173	0.0111380614932519\\
174	0.0111381258767829\\
175	0.0111381913703198\\
176	0.0111382579935432\\
177	0.0111383257665335\\
178	0.0111383947097829\\
179	0.0111384648442082\\
180	0.011138536191164\\
181	0.0111386087724565\\
182	0.0111386826103579\\
183	0.0111387577276218\\
184	0.0111388341474982\\
185	0.0111389118937501\\
186	0.0111389909906702\\
187	0.0111390714630976\\
188	0.0111391533364355\\
189	0.0111392366366689\\
190	0.0111393213903817\\
191	0.0111394076247738\\
192	0.011139495367677\\
193	0.0111395846475682\\
194	0.011139675493578\\
195	0.0111397679354881\\
196	0.0111398620036999\\
197	0.0111399577291125\\
198	0.011140055142707\\
199	0.0111401542742476\\
200	0.0111402551539413\\
201	0.0111403578125196\\
202	0.0111404622812472\\
203	0.0111405685919302\\
204	0.0111406767769259\\
205	0.0111407868691506\\
206	0.0111408989020894\\
207	0.011141012909805\\
208	0.0111411289269468\\
209	0.0111412469887604\\
210	0.0111413671310968\\
211	0.0111414893904221\\
212	0.0111416138038267\\
213	0.0111417404090351\\
214	0.0111418692444155\\
215	0.0111420003489897\\
216	0.0111421337624427\\
217	0.0111422695251327\\
218	0.0111424076781009\\
219	0.0111425482630814\\
220	0.0111426913225114\\
221	0.011142836899541\\
222	0.011142985038043\\
223	0.0111431357826235\\
224	0.0111432891786313\\
225	0.0111434452721684\\
226	0.0111436041100996\\
227	0.0111437657400626\\
228	0.011143930210478\\
229	0.011144097570559\\
230	0.0111442678703212\\
231	0.0111444411605923\\
232	0.0111446174930221\\
233	0.0111447969200915\\
234	0.0111449794951222\\
235	0.011145165272286\\
236	0.011145354306614\\
237	0.0111455466540058\\
238	0.0111457423712383\\
239	0.0111459415159746\\
240	0.0111461441467725\\
241	0.0111463503230937\\
242	0.0111465601053115\\
243	0.0111467735547197\\
244	0.0111469907335405\\
245	0.0111472117049333\\
246	0.0111474365330023\\
247	0.0111476652828051\\
248	0.0111478980203611\\
249	0.0111481348126599\\
250	0.0111483757276695\\
251	0.0111486208343463\\
252	0.0111488702026437\\
253	0.0111491239035225\\
254	0.0111493820089612\\
255	0.0111496445919683\\
256	0.0111499117265941\\
257	0.011150183487945\\
258	0.011150459952199\\
259	0.0111507411966224\\
260	0.0111510272995898\\
261	0.0111513183406048\\
262	0.0111516144003252\\
263	0.0111519155605894\\
264	0.0111522219044477\\
265	0.0111525335161963\\
266	0.011152850481415\\
267	0.0111531728870095\\
268	0.0111535008212575\\
269	0.0111538343738584\\
270	0.011154173635986\\
271	0.0111545187003448\\
272	0.011154869661226\\
273	0.011155226614563\\
274	0.011155589657983\\
275	0.0111559588908481\\
276	0.0111563344142832\\
277	0.0111567163311784\\
278	0.0111571047461569\\
279	0.0111574997654921\\
280	0.0111579014970348\\
281	0.0111583100502301\\
282	0.0111587255361337\\
283	0.0111591480674291\\
284	0.0111595777584429\\
285	0.0111600147251617\\
286	0.0111604590852475\\
287	0.0111609109580533\\
288	0.0111613704646386\\
289	0.0111618377277839\\
290	0.0111623128720057\\
291	0.0111627960235701\\
292	0.0111632873105074\\
293	0.0111637868626245\\
294	0.0111642948115179\\
295	0.0111648112905864\\
296	0.0111653364350419\\
297	0.0111658703819207\\
298	0.0111664132700937\\
299	0.0111669652402759\\
300	0.0111675264350347\\
301	0.0111680969987982\\
302	0.0111686770778615\\
303	0.0111692668203924\\
304	0.0111698663764365\\
305	0.0111704758979205\\
306	0.0111710955386542\\
307	0.0111717254543317\\
308	0.0111723658025307\\
309	0.0111730167427103\\
310	0.0111736784362075\\
311	0.0111743510462315\\
312	0.0111750347378564\\
313	0.0111757296780119\\
314	0.0111764360354716\\
315	0.0111771539808395\\
316	0.0111778836865335\\
317	0.011178625326767\\
318	0.011179379077527\\
319	0.0111801451165501\\
320	0.0111809236232947\\
321	0.01118171477891\\
322	0.0111825187662021\\
323	0.0111833357695958\\
324	0.0111841659750923\\
325	0.0111850095702232\\
326	0.0111858667439998\\
327	0.0111867376868576\\
328	0.0111876225905956\\
329	0.0111885216483107\\
330	0.0111894350543262\\
331	0.0111903630041141\\
332	0.011191305694211\\
333	0.011192263322128\\
334	0.0111932360862525\\
335	0.0111942241857427\\
336	0.0111952278204145\\
337	0.0111962471906195\\
338	0.0111972824971142\\
339	0.0111983339409198\\
340	0.0111994017231719\\
341	0.0112004860449598\\
342	0.0112015871071548\\
343	0.0112027051102267\\
344	0.0112038402540483\\
345	0.0112049927376869\\
346	0.0112061627591824\\
347	0.0112073505153117\\
348	0.0112085562013379\\
349	0.0112097800107449\\
350	0.0112110221349559\\
351	0.0112122827630356\\
352	0.0112135620813758\\
353	0.0112148602733636\\
354	0.0112161775190324\\
355	0.0112175139946947\\
356	0.0112188698725573\\
357	0.0112202453203193\\
358	0.0112216405007514\\
359	0.0112230555712602\\
360	0.0112244906834351\\
361	0.0112259459825816\\
362	0.0112274216072419\\
363	0.0112289176887048\\
364	0.0112304343505098\\
365	0.0112319717079471\\
366	0.0112335298675609\\
367	0.011235108926661\\
368	0.0112367089728516\\
369	0.0112383300835861\\
370	0.0112399723257614\\
371	0.0112416357553648\\
372	0.0112433204171928\\
373	0.0112450263446626\\
374	0.0112467535597421\\
375	0.0112485020730314\\
376	0.0112502718840328\\
377	0.0112520629816553\\
378	0.0112538753450092\\
379	0.0112557089445574\\
380	0.0112575637437033\\
381	0.0112594397009123\\
382	0.011261336772484\\
383	0.011263254916113\\
384	0.0112651940953997\\
385	0.0112671542854823\\
386	0.0112691354799856\\
387	0.0112711376998812\\
388	0.0112731609958665\\
389	0.011275205442138\\
390	0.0112772711419582\\
391	0.0112793582341545\\
392	0.0112814669004692\\
393	0.0112835973737467\\
394	0.0112857499469497\\
395	0.011287924982974\\
396	0.0112901229261886\\
397	0.0112923443143758\\
398	0.0112945897895973\\
399	0.011296860107229\\
400	0.0112991561414874\\
401	0.0113014788897761\\
402	0.0113038294993952\\
403	0.0113062094033161\\
404	0.0113086226958257\\
405	0.0113110751510743\\
406	0.0113135673699815\\
407	0.0113160999615969\\
408	0.0113186735427542\\
409	0.0113212887375818\\
410	0.0113239461768355\\
411	0.0113266464970148\\
412	0.0113293903392128\\
413	0.0113321783476421\\
414	0.0113350111677682\\
415	0.0113378894439685\\
416	0.0113408138166206\\
417	0.0113437849185099\\
418	0.0113468033704163\\
419	0.0113498697756232\\
420	0.0113529847132047\\
421	0.0113561487298956\\
422	0.011359362330353\\
423	0.0113626259656622\\
424	0.0113659400197551\\
425	0.0113693047912446\\
426	0.0113727204542197\\
427	0.0113761874244677\\
428	0.0113797065484491\\
429	0.0113832787140749\\
430	0.0113869048551724\\
431	0.0113905859563956\\
432	0.0113943230586703\\
433	0.0113981172655079\\
434	0.0114019697499337\\
435	0.0114058817619046\\
436	0.0114098546364875\\
437	0.0114138898028763\\
438	0.0114179887943226\\
439	0.0114221532590524\\
440	0.0114263849722265\\
441	0.0114306858491446\\
442	0.0114350579598497\\
443	0.011439503545246\\
444	0.0114440250349267\\
445	0.0114486250669778\\
446	0.0114533065102145\\
447	0.0114580724898613\\
448	0.0114629264195495\\
449	0.0114678720490076\\
450	0.0114729135605266\\
451	0.0114780967282399\\
452	0.0114838247603427\\
453	0.0114896438072392\\
454	0.0114955557322134\\
455	0.0115015623651431\\
456	0.0115076656011001\\
457	0.0115138667117468\\
458	0.0115201668715035\\
459	0.0115265673724257\\
460	0.0115330695970858\\
461	0.0115396743507591\\
462	0.0115463826483999\\
463	0.0115531954320997\\
464	0.0115601135678285\\
465	0.0115671378326153\\
466	0.0115742689280181\\
467	0.0115815075656714\\
468	0.0115888548048529\\
469	0.0115963132839257\\
470	0.0116038917100389\\
471	0.0116112470452104\\
472	0.0116185579486169\\
473	0.0116259950960515\\
474	0.0116335606989619\\
475	0.0116412569949167\\
476	0.011649086308021\\
477	0.0116570512504184\\
478	0.0116651542494176\\
479	0.0116733976982518\\
480	0.0116817840088591\\
481	0.0116903155920849\\
482	0.0116989948720478\\
483	0.0117078242215766\\
484	0.0117168057224236\\
485	0.0117259402094452\\
486	0.011735223353074\\
487	0.0117445967645339\\
488	0.0117540438960194\\
489	0.0117636627293605\\
490	0.011773455458317\\
491	0.01178342163743\\
492	0.0117935540185383\\
493	0.0118038579983944\\
494	0.0118143595453667\\
495	0.0118250607769372\\
496	0.0118359644041274\\
497	0.0118470798895441\\
498	0.0118584108270909\\
499	0.0118699604173308\\
500	0.0118817308249362\\
501	0.0118937242629477\\
502	0.0119059486953561\\
503	0.0119184075796668\\
504	0.0119311043163338\\
505	0.0119440422400084\\
506	0.0119572246089145\\
507	0.0119706545955775\\
508	0.0119843352751079\\
509	0.0119982696139005\\
510	0.0120124604612387\\
511	0.0120269105311997\\
512	0.0120416223585578\\
513	0.0120565983145842\\
514	0.0120718414555345\\
515	0.0120873556252044\\
516	0.0121031401616912\\
517	0.0121191831131971\\
518	0.0121354714227797\\
519	0.0121522355660203\\
520	0.0121695817635656\\
521	0.0121875257516508\\
522	0.0122061547629794\\
523	0.0122254552750192\\
524	0.0122455098893775\\
525	0.0122664602206702\\
526	0.0122887936321264\\
527	0.0123267608992047\\
528	0.0123643335330852\\
529	0.0124013799674802\\
530	0.0124375264266978\\
531	0.0124609174558013\\
532	0.0124822410236508\\
533	0.0125025495458929\\
534	0.0125233303641362\\
535	0.0125412936178678\\
536	0.0125579257775076\\
537	0.0125743168196365\\
538	0.0125900935401606\\
539	0.0126047006867955\\
540	0.0126194620420817\\
541	0.0126343951220848\\
542	0.0126495061887113\\
543	0.0126646122999745\\
544	0.0126798240621971\\
545	0.0126953209458609\\
546	0.0127111114582881\\
547	0.0127272116698324\\
548	0.0127444883070642\\
549	0.0127624381672173\\
550	0.0127802358560681\\
551	0.0127970371041276\\
552	0.012813503488888\\
553	0.0128300811739142\\
554	0.0128465893914093\\
555	0.0128632159712371\\
556	0.0128800747340768\\
557	0.0128971702537141\\
558	0.0129145052118848\\
559	0.0129320759630228\\
560	0.0129498830694804\\
561	0.012968997322975\\
562	0.0129888628245528\\
563	0.0130071646924787\\
564	0.0130249079364167\\
565	0.0130423674263173\\
566	0.0130598681838456\\
567	0.0130775145513471\\
568	0.0130953029356719\\
569	0.0131132243488511\\
570	0.013131268655502\\
571	0.0131497744261586\\
572	0.0131687373925793\\
573	0.0131865642062699\\
574	0.0132041733275976\\
575	0.0132217748679167\\
576	0.0132394175008641\\
577	0.0132570843430895\\
578	0.0132747565970175\\
579	0.0132924136110985\\
580	0.0133100327167465\\
581	0.013327589042721\\
582	0.0133450553095644\\
583	0.0133624016038809\\
584	0.0133795951354462\\
585	0.0133965999904435\\
586	0.0134133769227543\\
587	0.0134298833028437\\
588	0.0134460735522599\\
589	0.0134619009495395\\
590	0.0134776600725805\\
591	0.0134935706885156\\
592	0.0135096572289508\\
593	0.0135260152836207\\
594	0.0135429334162338\\
595	0.0135612173030331\\
596	0.0135830518096797\\
597	0.0136142933738059\\
598	0.0136705746214114\\
599	0\\
600	0\\
};
\addplot [color=mycolor8,solid,forget plot]
  table[row sep=crcr]{%
1	0.0109994391104229\\
2	0.0109994446589932\\
3	0.0109994503018498\\
4	0.0109994560405924\\
5	0.0109994618768477\\
6	0.0109994678122701\\
7	0.0109994738485418\\
8	0.0109994799873737\\
9	0.0109994862305054\\
10	0.0109994925797059\\
11	0.0109994990367743\\
12	0.0109995056035401\\
13	0.0109995122818636\\
14	0.0109995190736366\\
15	0.010999525980783\\
16	0.010999533005259\\
17	0.010999540149054\\
18	0.0109995474141911\\
19	0.0109995548027273\\
20	0.0109995623167547\\
21	0.0109995699584005\\
22	0.0109995777298278\\
23	0.0109995856332365\\
24	0.0109995936708634\\
25	0.0109996018449832\\
26	0.010999610157909\\
27	0.0109996186119928\\
28	0.0109996272096264\\
29	0.0109996359532421\\
30	0.0109996448453131\\
31	0.0109996538883542\\
32	0.010999663084923\\
33	0.0109996724376198\\
34	0.0109996819490891\\
35	0.0109996916220198\\
36	0.0109997014591462\\
37	0.0109997114632488\\
38	0.0109997216371548\\
39	0.0109997319837392\\
40	0.0109997425059252\\
41	0.0109997532066856\\
42	0.0109997640890429\\
43	0.0109997751560709\\
44	0.0109997864108948\\
45	0.0109997978566928\\
46	0.0109998094966961\\
47	0.0109998213341909\\
48	0.0109998333725181\\
49	0.0109998456150752\\
50	0.0109998580653168\\
51	0.0109998707267556\\
52	0.0109998836029631\\
53	0.0109998966975713\\
54	0.0109999100142729\\
55	0.0109999235568229\\
56	0.0109999373290391\\
57	0.0109999513348038\\
58	0.0109999655780643\\
59	0.0109999800628341\\
60	0.0109999947931943\\
61	0.0110000097732943\\
62	0.0110000250073534\\
63	0.0110000404996613\\
64	0.01100005625458\\
65	0.0110000722765445\\
66	0.0110000885700641\\
67	0.0110001051397238\\
68	0.0110001219901854\\
69	0.0110001391261887\\
70	0.011000156552553\\
71	0.0110001742741782\\
72	0.0110001922960463\\
73	0.0110002106232226\\
74	0.0110002292608572\\
75	0.0110002482141862\\
76	0.0110002674885334\\
77	0.0110002870893114\\
78	0.0110003070220235\\
79	0.0110003272922646\\
80	0.0110003479057234\\
81	0.0110003688681831\\
82	0.0110003901855237\\
83	0.0110004118637233\\
84	0.0110004339088596\\
85	0.0110004563271114\\
86	0.0110004791247609\\
87	0.0110005023081945\\
88	0.0110005258839051\\
89	0.0110005498584938\\
90	0.0110005742386712\\
91	0.0110005990312597\\
92	0.011000624243195\\
93	0.0110006498815281\\
94	0.0110006759534271\\
95	0.0110007024661789\\
96	0.0110007294271915\\
97	0.0110007568439956\\
98	0.0110007847242467\\
99	0.0110008130757273\\
100	0.0110008419063486\\
101	0.0110008712241526\\
102	0.0110009010373146\\
103	0.0110009313541447\\
104	0.0110009621830904\\
105	0.0110009935327388\\
106	0.0110010254118184\\
107	0.0110010578292018\\
108	0.0110010907939079\\
109	0.0110011243151037\\
110	0.0110011584021074\\
111	0.0110011930643904\\
112	0.0110012283115794\\
113	0.0110012641534595\\
114	0.0110013005999762\\
115	0.011001337661238\\
116	0.0110013753475189\\
117	0.0110014136692612\\
118	0.0110014526370777\\
119	0.0110014922617548\\
120	0.0110015325542547\\
121	0.0110015735257185\\
122	0.0110016151874687\\
123	0.011001657551012\\
124	0.011001700628042\\
125	0.0110017444304424\\
126	0.0110017889702894\\
127	0.011001834259855\\
128	0.0110018803116096\\
129	0.011001927138225\\
130	0.0110019747525776\\
131	0.0110020231677513\\
132	0.0110020723970406\\
133	0.0110021224539534\\
134	0.0110021733522146\\
135	0.0110022251057688\\
136	0.0110022777287837\\
137	0.0110023312356532\\
138	0.0110023856410006\\
139	0.0110024409596821\\
140	0.0110024972067895\\
141	0.0110025543976541\\
142	0.0110026125478498\\
143	0.0110026716731963\\
144	0.0110027317897624\\
145	0.0110027929138699\\
146	0.0110028550620963\\
147	0.0110029182512787\\
148	0.0110029824985171\\
149	0.0110030478211777\\
150	0.0110031142368966\\
151	0.0110031817635831\\
152	0.0110032504194232\\
153	0.0110033202228833\\
154	0.0110033911927135\\
155	0.0110034633479512\\
156	0.0110035367079249\\
157	0.0110036112922572\\
158	0.011003687120869\\
159	0.0110037642139829\\
160	0.0110038425921269\\
161	0.0110039222761379\\
162	0.0110040032871657\\
163	0.0110040856466769\\
164	0.0110041693764584\\
165	0.0110042544986219\\
166	0.0110043410356076\\
167	0.0110044290101885\\
168	0.0110045184454753\\
169	0.0110046093649202\\
170	0.0110047017923228\\
171	0.0110047957518344\\
172	0.0110048912679641\\
173	0.011004988365585\\
174	0.0110050870699405\\
175	0.0110051874066518\\
176	0.0110052894017257\\
177	0.0110053930815645\\
178	0.0110054984729752\\
179	0.011005605603182\\
180	0.0110057144998394\\
181	0.0110058251910472\\
182	0.0110059377053685\\
183	0.0110060520718501\\
184	0.0110061683200464\\
185	0.0110062864800466\\
186	0.011006406582508\\
187	0.0110065286586925\\
188	0.0110066527405103\\
189	0.0110067788605677\\
190	0.0110069070522158\\
191	0.0110070373495887\\
192	0.0110071697875911\\
193	0.0110073044017306\\
194	0.011007441227491\\
195	0.0110075802983981\\
196	0.0110077216403673\\
197	0.0110078652554368\\
198	0.0110080110748471\\
199	0.0110081594194041\\
200	0.0110083103450839\\
201	0.0110084638965014\\
202	0.0110086201190287\\
203	0.011008779058807\\
204	0.0110089407627596\\
205	0.0110091052786033\\
206	0.0110092726548623\\
207	0.0110094429408799\\
208	0.0110096161868322\\
209	0.0110097924437409\\
210	0.0110099717634867\\
211	0.0110101541988227\\
212	0.0110103398033883\\
213	0.0110105286317227\\
214	0.011010720739279\\
215	0.0110109161824386\\
216	0.0110111150185252\\
217	0.0110113173058194\\
218	0.0110115231035734\\
219	0.0110117324720262\\
220	0.0110119454724182\\
221	0.0110121621670066\\
222	0.0110123826190809\\
223	0.0110126068929785\\
224	0.0110128350541005\\
225	0.0110130671689278\\
226	0.0110133033050371\\
227	0.0110135435311177\\
228	0.0110137879169877\\
229	0.0110140365336114\\
230	0.0110142894531159\\
231	0.0110145467488088\\
232	0.0110148084951956\\
233	0.0110150747679978\\
234	0.0110153456441705\\
235	0.0110156212019216\\
236	0.0110159015207298\\
237	0.011016186681364\\
238	0.0110164767659025\\
239	0.0110167718577528\\
240	0.0110170720416716\\
241	0.0110173774037854\\
242	0.0110176880316113\\
243	0.0110180040140781\\
244	0.0110183254415486\\
245	0.0110186524058414\\
246	0.0110189850002539\\
247	0.0110193233195858\\
248	0.0110196674601628\\
249	0.0110200175198613\\
250	0.0110203735981335\\
251	0.0110207357960336\\
252	0.0110211042162438\\
253	0.0110214789631022\\
254	0.0110218601426307\\
255	0.0110222478625635\\
256	0.011022642232377\\
257	0.0110230433633204\\
258	0.0110234513684464\\
259	0.0110238663626431\\
260	0.0110242884626669\\
261	0.0110247177871755\\
262	0.0110251544567617\\
263	0.0110255985939875\\
264	0.0110260503234186\\
265	0.0110265097716592\\
266	0.0110269770673861\\
267	0.0110274523413836\\
268	0.0110279357265764\\
269	0.0110284273580626\\
270	0.0110289273731457\\
271	0.0110294359113637\\
272	0.0110299531145182\\
273	0.0110304791267001\\
274	0.0110310140943141\\
275	0.0110315581661008\\
276	0.0110321114931575\\
277	0.011032674228959\\
278	0.0110332465293796\\
279	0.0110338285527195\\
280	0.0110344204597362\\
281	0.0110350224136753\\
282	0.0110356345803016\\
283	0.0110362571279302\\
284	0.0110368902274578\\
285	0.0110375340523935\\
286	0.0110381887788897\\
287	0.0110388545857732\\
288	0.0110395316545758\\
289	0.0110402201695645\\
290	0.0110409203177724\\
291	0.0110416322890279\\
292	0.0110423562759852\\
293	0.0110430924741531\\
294	0.0110438410819241\\
295	0.0110446023006026\\
296	0.0110453763344332\\
297	0.0110461633906277\\
298	0.0110469636793918\\
299	0.0110477774139507\\
300	0.0110486048105746\\
301	0.0110494460886026\\
302	0.0110503014704658\\
303	0.0110511711817093\\
304	0.0110520554510129\\
305	0.011052954510211\\
306	0.0110538685943101\\
307	0.0110547979415057\\
308	0.011055742793197\\
309	0.0110567033939995\\
310	0.0110576799917563\\
311	0.0110586728375466\\
312	0.0110596821856921\\
313	0.0110607082937609\\
314	0.0110617514225685\\
315	0.0110628118361758\\
316	0.0110638898018839\\
317	0.0110649855902258\\
318	0.0110660994749535\\
319	0.0110672317330219\\
320	0.0110683826445682\\
321	0.011069552492886\\
322	0.0110707415643949\\
323	0.0110719501486047\\
324	0.0110731785380734\\
325	0.0110744270283593\\
326	0.0110756959179657\\
327	0.0110769855082791\\
328	0.0110782961034987\\
329	0.0110796280105582\\
330	0.0110809815390383\\
331	0.011082357001069\\
332	0.011083754711222\\
333	0.0110851749863913\\
334	0.011086618145662\\
335	0.0110880845101662\\
336	0.0110895744029238\\
337	0.01109108814867\\
338	0.0110926260736644\\
339	0.0110941885054844\\
340	0.0110957757727991\\
341	0.0110973882051225\\
342	0.0110990261325459\\
343	0.0111006898854459\\
344	0.0111023797941684\\
345	0.0111040961886843\\
346	0.0111058393982166\\
347	0.0111076097508366\\
348	0.0111094075730254\\
349	0.0111112331892001\\
350	0.0111130869212001\\
351	0.0111149690877322\\
352	0.01111688000377\\
353	0.0111188199799051\\
354	0.0111207893216455\\
355	0.0111227883286592\\
356	0.0111248172939556\\
357	0.0111268765030038\\
358	0.0111289662327803\\
359	0.0111310867507424\\
360	0.0111332383137213\\
361	0.0111354211667287\\
362	0.0111376355416725\\
363	0.0111398816559719\\
364	0.0111421597110686\\
365	0.0111444698908236\\
366	0.0111468123597938\\
367	0.01114918726138\\
368	0.0111515947158367\\
369	0.0111540348181367\\
370	0.0111565076356795\\
371	0.0111590132058357\\
372	0.0111615515333163\\
373	0.0111641225873575\\
374	0.0111667262987121\\
375	0.0111693625564347\\
376	0.0111720312044536\\
377	0.0111747320379172\\
378	0.0111774647993051\\
379	0.0111802291742958\\
380	0.0111830247873785\\
381	0.0111858511972009\\
382	0.0111887078916425\\
383	0.0111915942826005\\
384	0.0111945097004746\\
385	0.0111974533883317\\
386	0.0112004244957219\\
387	0.0112034220720975\\
388	0.0112064450602456\\
389	0.0112094922904391\\
390	0.0112125624836319\\
391	0.0112156542396862\\
392	0.0112187660310133\\
393	0.0112218961985045\\
394	0.0112250429481726\\
395	0.0112282043475872\\
396	0.0112313782932352\\
397	0.0112345625144164\\
398	0.011237754596442\\
399	0.0112409519725482\\
400	0.0112441518921304\\
401	0.0112473513225935\\
402	0.0112505466989671\\
403	0.0112537334697729\\
404	0.0112568439820669\\
405	0.0112598418843373\\
406	0.0112628848280278\\
407	0.0112659732222988\\
408	0.0112691074629742\\
409	0.0112722879313295\\
410	0.0112755149928624\\
411	0.011278788996058\\
412	0.0112821102711688\\
413	0.0112854791290313\\
414	0.0112888958599498\\
415	0.0112923607326802\\
416	0.0112958739935511\\
417	0.0112994358657629\\
418	0.0113030465489109\\
419	0.0113067062194533\\
420	0.0113104150313417\\
421	0.0113141731172676\\
422	0.0113179805907699\\
423	0.0113218375494261\\
424	0.011325744079357\\
425	0.0113297002613238\\
426	0.011333706179562\\
427	0.0113377619142812\\
428	0.0113418675244678\\
429	0.0113460230480884\\
430	0.0113502285027755\\
431	0.0113544838867843\\
432	0.0113587891802689\\
433	0.0113631443485436\\
434	0.0113675493458466\\
435	0.011372004118536\\
436	0.0113765086088889\\
437	0.0113810627595569\\
438	0.0113856665187299\\
439	0.0113903198460437\\
440	0.0113950227192466\\
441	0.0113997751416779\\
442	0.0114045771505625\\
443	0.01140942882607\\
444	0.0114143303010686\\
445	0.011419281771453\\
446	0.0114242835068744\\
447	0.011429335861635\\
448	0.0114344392854164\\
449	0.0114395943331726\\
450	0.0114448016716341\\
451	0.0114500620695723\\
452	0.0114553764480879\\
453	0.0114607469815602\\
454	0.0114661761969173\\
455	0.0114716671186106\\
456	0.0114772238733225\\
457	0.0114828650018824\\
458	0.0114885945508254\\
459	0.0114944143765377\\
460	0.0115003264698249\\
461	0.0115063329744043\\
462	0.0115124362058791\\
463	0.0115186386738397\\
464	0.0115249431069522\\
465	0.0115313524816497\\
466	0.0115378700555127\\
467	0.0115444994080129\\
468	0.011551244496638\\
469	0.0115581097554603\\
470	0.011565100333571\\
471	0.0115726406136334\\
472	0.0115805039425119\\
473	0.0115884808319944\\
474	0.011596572840121\\
475	0.0116047814820966\\
476	0.0116131084447845\\
477	0.0116215568644889\\
478	0.0116301288729188\\
479	0.0116388262998099\\
480	0.011647650930361\\
481	0.0116566043880182\\
482	0.0116656882736436\\
483	0.0116749041918865\\
484	0.0116842540274581\\
485	0.0116937411516271\\
486	0.011703374770306\\
487	0.0117129167721184\\
488	0.0117222364425427\\
489	0.0117317130571502\\
490	0.0117413490544015\\
491	0.0117511468336866\\
492	0.0117611088865898\\
493	0.0117712372314286\\
494	0.011781531608925\\
495	0.011791986129479\\
496	0.0118025981494018\\
497	0.0118133993103486\\
498	0.0118243921411096\\
499	0.0118355745707611\\
500	0.0118469504849285\\
501	0.0118584840801549\\
502	0.0118700845232673\\
503	0.0118819065861743\\
504	0.0118939541837207\\
505	0.0119062311546815\\
506	0.0119187412937486\\
507	0.0119314883274837\\
508	0.0119444759490848\\
509	0.0119577078120563\\
510	0.0119711874662541\\
511	0.0119849182319479\\
512	0.0119989028881973\\
513	0.0120131429831058\\
514	0.0120276408824161\\
515	0.0120424042789703\\
516	0.0120574363174658\\
517	0.0120727409301603\\
518	0.012088323150752\\
519	0.0121041766363579\\
520	0.0121202892111064\\
521	0.012136647059301\\
522	0.0121535841005046\\
523	0.0121711193189029\\
524	0.0121892769414689\\
525	0.0122081380070635\\
526	0.0122277262064328\\
527	0.0122480282710536\\
528	0.0122691265074252\\
529	0.0122911574298672\\
530	0.0123144413487278\\
531	0.0123524072050768\\
532	0.0123915122039917\\
533	0.0124301395128839\\
534	0.0124678798396923\\
535	0.0124958954394859\\
536	0.0125188551149724\\
537	0.0125408431676882\\
538	0.012562157324602\\
539	0.0125827931588255\\
540	0.0126008544131521\\
541	0.0126186577195187\\
542	0.0126362389580849\\
543	0.0126527445975886\\
544	0.0126686679535967\\
545	0.012684760060125\\
546	0.0127010381773125\\
547	0.0127175029632303\\
548	0.0127338785602641\\
549	0.0127504733471904\\
550	0.0127673529200804\\
551	0.0127855629428391\\
552	0.0128042456652251\\
553	0.0128228101224692\\
554	0.0128403790221128\\
555	0.0128576152448109\\
556	0.0128749610736061\\
557	0.0128922661924296\\
558	0.0129096369206464\\
559	0.0129272337051127\\
560	0.0129450570584863\\
561	0.012963109982889\\
562	0.0129813975417506\\
563	0.013001304118468\\
564	0.0130214934907333\\
565	0.0130400364960564\\
566	0.0130581481420238\\
567	0.0130759889286681\\
568	0.0130938267363126\\
569	0.0131117895203391\\
570	0.0131298756555198\\
571	0.0131480743817308\\
572	0.0131665007135915\\
573	0.0131857960098377\\
574	0.0132039776933286\\
575	0.0132217712134681\\
576	0.0132394172255701\\
577	0.0132570842302988\\
578	0.0132747565385405\\
579	0.0132924135805215\\
580	0.0133100327015016\\
581	0.013327589035641\\
582	0.0133450553065625\\
583	0.0133624016027492\\
584	0.0133795951350809\\
585	0.0133965999903484\\
586	0.0134133769227364\\
587	0.0134298833028419\\
588	0.0134460735522599\\
589	0.0134619009495395\\
590	0.0134776600725805\\
591	0.0134935706885156\\
592	0.0135096572289508\\
593	0.0135260152836207\\
594	0.0135429334162338\\
595	0.0135612173030331\\
596	0.0135830518096797\\
597	0.0136142933738059\\
598	0.0136705746214114\\
599	0\\
600	0\\
};
\addplot [color=blue!25!mycolor7,solid,forget plot]
  table[row sep=crcr]{%
1	0.0107466297875576\\
2	0.0107466377329611\\
3	0.0107466458127207\\
4	0.0107466540291026\\
5	0.0107466623844111\\
6	0.0107466708809892\\
7	0.0107466795212193\\
8	0.0107466883075236\\
9	0.0107466972423653\\
10	0.0107467063282487\\
11	0.0107467155677201\\
12	0.0107467249633689\\
13	0.0107467345178274\\
14	0.0107467442337726\\
15	0.0107467541139261\\
16	0.0107467641610552\\
17	0.0107467743779738\\
18	0.0107467847675426\\
19	0.0107467953326706\\
20	0.0107468060763156\\
21	0.0107468170014847\\
22	0.0107468281112355\\
23	0.010746839408677\\
24	0.0107468508969702\\
25	0.0107468625793289\\
26	0.0107468744590209\\
27	0.0107468865393687\\
28	0.0107468988237503\\
29	0.0107469113156005\\
30	0.0107469240184113\\
31	0.0107469369357334\\
32	0.0107469500711767\\
33	0.0107469634284115\\
34	0.0107469770111697\\
35	0.0107469908232453\\
36	0.010747004868496\\
37	0.0107470191508438\\
38	0.0107470336742763\\
39	0.0107470484428477\\
40	0.0107470634606801\\
41	0.0107470787319641\\
42	0.0107470942609606\\
43	0.0107471100520014\\
44	0.0107471261094907\\
45	0.0107471424379063\\
46	0.0107471590418004\\
47	0.0107471759258014\\
48	0.0107471930946146\\
49	0.0107472105530241\\
50	0.0107472283058935\\
51	0.0107472463581673\\
52	0.0107472647148727\\
53	0.0107472833811203\\
54	0.0107473023621059\\
55	0.0107473216631118\\
56	0.0107473412895082\\
57	0.0107473612467545\\
58	0.010747381540401\\
59	0.0107474021760902\\
60	0.0107474231595584\\
61	0.0107474444966371\\
62	0.0107474661932548\\
63	0.0107474882554383\\
64	0.0107475106893144\\
65	0.0107475335011117\\
66	0.0107475566971619\\
67	0.0107475802839018\\
68	0.0107476042678748\\
69	0.0107476286557327\\
70	0.0107476534542375\\
71	0.0107476786702631\\
72	0.010747704310797\\
73	0.0107477303829425\\
74	0.0107477568939202\\
75	0.01074778385107\\
76	0.0107478112618531\\
77	0.0107478391338539\\
78	0.0107478674747818\\
79	0.0107478962924736\\
80	0.010747925594895\\
81	0.0107479553901433\\
82	0.0107479856864488\\
83	0.0107480164921777\\
84	0.0107480478158333\\
85	0.0107480796660593\\
86	0.010748112051641\\
87	0.0107481449815082\\
88	0.0107481784647372\\
89	0.0107482125105532\\
90	0.0107482471283327\\
91	0.0107482823276057\\
92	0.0107483181180584\\
93	0.0107483545095351\\
94	0.0107483915120416\\
95	0.0107484291357468\\
96	0.0107484673909856\\
97	0.0107485062882616\\
98	0.0107485458382497\\
99	0.0107485860517984\\
100	0.010748626939933\\
101	0.010748668513858\\
102	0.0107487107849599\\
103	0.0107487537648098\\
104	0.0107487974651668\\
105	0.0107488418979801\\
106	0.0107488870753927\\
107	0.0107489330097434\\
108	0.0107489797135708\\
109	0.0107490271996153\\
110	0.0107490754808228\\
111	0.0107491245703474\\
112	0.0107491744815549\\
113	0.0107492252280254\\
114	0.0107492768235567\\
115	0.0107493292821678\\
116	0.0107493826181014\\
117	0.0107494368458279\\
118	0.0107494919800482\\
119	0.0107495480356972\\
120	0.010749605027947\\
121	0.0107496629722104\\
122	0.0107497218841441\\
123	0.0107497817796525\\
124	0.0107498426748904\\
125	0.0107499045862672\\
126	0.0107499675304499\\
127	0.0107500315243666\\
128	0.0107500965852103\\
129	0.0107501627304419\\
130	0.010750229977794\\
131	0.0107502983452743\\
132	0.0107503678511693\\
133	0.0107504385140473\\
134	0.0107505103527623\\
135	0.0107505833864573\\
136	0.0107506576345675\\
137	0.0107507331168241\\
138	0.0107508098532573\\
139	0.0107508878641998\\
140	0.01075096717029\\
141	0.010751047792475\\
142	0.010751129752014\\
143	0.010751213070481\\
144	0.0107512977697681\\
145	0.0107513838720878\\
146	0.0107514713999761\\
147	0.0107515603762946\\
148	0.0107516508242333\\
149	0.0107517427673124\\
150	0.010751836229384\\
151	0.0107519312346346\\
152	0.0107520278075856\\
153	0.0107521259730948\\
154	0.0107522257563571\\
155	0.010752327182905\\
156	0.0107524302786079\\
157	0.0107525350696724\\
158	0.0107526415826402\\
159	0.0107527498443872\\
160	0.0107528598821204\\
161	0.0107529717233749\\
162	0.0107530853960097\\
163	0.0107532009282028\\
164	0.0107533183484448\\
165	0.0107534376855312\\
166	0.0107535589685542\\
167	0.0107536822268916\\
168	0.0107538074901949\\
169	0.0107539347883753\\
170	0.0107540641515868\\
171	0.0107541956102075\\
172	0.0107543291948178\\
173	0.0107544649361747\\
174	0.010754602865184\\
175	0.0107547430128661\\
176	0.0107548854103191\\
177	0.0107550300886754\\
178	0.010755177079052\\
179	0.0107553264124948\\
180	0.0107554781199136\\
181	0.0107556322320091\\
182	0.010755788779189\\
183	0.0107559477914715\\
184	0.0107561092983765\\
185	0.0107562733287989\\
186	0.010756439910865\\
187	0.0107566090717648\\
188	0.0107567808375558\\
189	0.0107569552329245\\
190	0.0107571322808771\\
191	0.0107573120022847\\
192	0.0107574944150894\\
193	0.01075767953264\\
194	0.0107578673596988\\
195	0.0107580578820589\\
196	0.0107582510384095\\
197	0.010758446642402\\
198	0.0107586441638536\\
199	0.0107588450346474\\
200	0.0107590493731681\\
201	0.0107592572387948\\
202	0.010759468691906\\
203	0.0107596837938964\\
204	0.0107599026071935\\
205	0.0107601251952738\\
206	0.0107603516226804\\
207	0.0107605819550399\\
208	0.0107608162590803\\
209	0.0107610546026486\\
210	0.010761297054729\\
211	0.0107615436854611\\
212	0.0107617945661588\\
213	0.0107620497693295\\
214	0.0107623093686926\\
215	0.0107625734391999\\
216	0.0107628420570549\\
217	0.0107631152997331\\
218	0.0107633932460029\\
219	0.010763675975946\\
220	0.0107639635709784\\
221	0.0107642561138727\\
222	0.010764553688779\\
223	0.0107648563812477\\
224	0.0107651642782519\\
225	0.0107654774682104\\
226	0.0107657960410108\\
227	0.0107661200880335\\
228	0.0107664497021753\\
229	0.0107667849778745\\
230	0.0107671260111353\\
231	0.0107674728995534\\
232	0.0107678257423416\\
233	0.0107681846403561\\
234	0.0107685496961228\\
235	0.0107689210138648\\
236	0.0107692986995299\\
237	0.0107696828608181\\
238	0.0107700736072108\\
239	0.0107704710499994\\
240	0.0107708753023149\\
241	0.0107712864791577\\
242	0.0107717046974286\\
243	0.0107721300759593\\
244	0.0107725627355442\\
245	0.0107730027989728\\
246	0.0107734503910616\\
247	0.0107739056386881\\
248	0.0107743686708241\\
249	0.0107748396185697\\
250	0.0107753186151889\\
251	0.0107758057961439\\
252	0.0107763012991318\\
253	0.0107768052641204\\
254	0.0107773178333855\\
255	0.0107778391515478\\
256	0.0107783693656109\\
257	0.0107789086249996\\
258	0.0107794570815984\\
259	0.0107800148897907\\
260	0.0107805822064981\\
261	0.0107811591912201\\
262	0.0107817460060742\\
263	0.0107823428158363\\
264	0.0107829497879811\\
265	0.0107835670927229\\
266	0.0107841949030567\\
267	0.0107848333947992\\
268	0.0107854827466304\\
269	0.0107861431401351\\
270	0.0107868147598448\\
271	0.01078749779328\\
272	0.0107881924309928\\
273	0.01078889886661\\
274	0.0107896172968771\\
275	0.0107903479217025\\
276	0.0107910909442036\\
277	0.0107918465707524\\
278	0.0107926150110235\\
279	0.0107933964780418\\
280	0.0107941911882312\\
281	0.010794999361464\\
282	0.0107958212211098\\
283	0.0107966569940863\\
284	0.0107975069109096\\
285	0.0107983712057455\\
286	0.0107992501164606\\
287	0.0108001438846751\\
288	0.0108010527558148\\
289	0.0108019769791644\\
290	0.010802916807921\\
291	0.0108038724992477\\
292	0.0108048443143286\\
293	0.010805832518423\\
294	0.010806837380921\\
295	0.0108078591753994\\
296	0.0108088981796769\\
297	0.0108099546758714\\
298	0.0108110289504566\\
299	0.0108121212943189\\
300	0.0108132320028151\\
301	0.0108143613758303\\
302	0.0108155097178359\\
303	0.0108166773379481\\
304	0.0108178645499866\\
305	0.0108190716725334\\
306	0.0108202990289921\\
307	0.010821546947647\\
308	0.010822815761723\\
309	0.0108241058094446\\
310	0.0108254174340965\\
311	0.0108267509840825\\
312	0.0108281068129862\\
313	0.0108294852796301\\
314	0.0108308867481362\\
315	0.0108323115879853\\
316	0.0108337601740768\\
317	0.0108352328867884\\
318	0.0108367301120353\\
319	0.0108382522413292\\
320	0.0108397996718376\\
321	0.0108413728064419\\
322	0.0108429720537957\\
323	0.0108445978283822\\
324	0.0108462505505719\\
325	0.0108479306466784\\
326	0.0108496385490153\\
327	0.0108513746959503\\
328	0.0108531395319606\\
329	0.0108549335076857\\
330	0.0108567570799803\\
331	0.0108586107119654\\
332	0.0108604948730791\\
333	0.0108624100391246\\
334	0.0108643566923187\\
335	0.0108663353213365\\
336	0.0108683464213564\\
337	0.0108703904941016\\
338	0.0108724680478803\\
339	0.0108745795976231\\
340	0.0108767256649177\\
341	0.0108789067780412\\
342	0.0108811234719883\\
343	0.0108833762884964\\
344	0.0108856657760662\\
345	0.0108879924899781\\
346	0.0108903569923027\\
347	0.0108927598519062\\
348	0.0108952016444483\\
349	0.010897682952373\\
350	0.0109002043648908\\
351	0.0109027664779504\\
352	0.0109053698941998\\
353	0.0109080152229337\\
354	0.0109107030800264\\
355	0.0109134340878475\\
356	0.010916208875157\\
357	0.0109190280769783\\
358	0.0109218923344437\\
359	0.0109248022946092\\
360	0.0109277586102336\\
361	0.0109307619395154\\
362	0.0109338129457811\\
363	0.0109369122971168\\
364	0.0109400606659336\\
365	0.0109432587284563\\
366	0.0109465071641213\\
367	0.01094980665487\\
368	0.0109531578843195\\
369	0.0109565615367899\\
370	0.0109600182961636\\
371	0.01096352884455\\
372	0.010967093860721\\
373	0.0109707140182797\\
374	0.0109743899835176\\
375	0.0109781224129081\\
376	0.010981911950175\\
377	0.0109857592228662\\
378	0.0109896648383503\\
379	0.0109936293791415\\
380	0.010997653397445\\
381	0.0110017374088008\\
382	0.0110058818846985\\
383	0.0110100872440552\\
384	0.0110143538435583\\
385	0.0110186819672918\\
386	0.0110230718174987\\
387	0.0110275235132301\\
388	0.0110320371203289\\
389	0.0110366127939116\\
390	0.0110412495201906\\
391	0.0110459466091877\\
392	0.0110507032498267\\
393	0.0110555182645491\\
394	0.0110603898742527\\
395	0.01106531509703\\
396	0.0110702936137899\\
397	0.0110753249844707\\
398	0.0110804060063078\\
399	0.0110855327944353\\
400	0.0110907006471332\\
401	0.0110959037944916\\
402	0.0111011347073483\\
403	0.0111063816609873\\
404	0.0111113236409798\\
405	0.01111570990855\\
406	0.0111201630238673\\
407	0.0111246836564568\\
408	0.011129272458569\\
409	0.0111339300631169\\
410	0.0111386570814743\\
411	0.0111434541011258\\
412	0.0111483216831647\\
413	0.0111532603596372\\
414	0.011158270630746\\
415	0.0111633529619588\\
416	0.0111685077811431\\
417	0.0111737354760475\\
418	0.0111790363929673\\
419	0.0111844108146664\\
420	0.0111898589682061\\
421	0.0111953810219278\\
422	0.0112009770798376\\
423	0.0112066471753408\\
424	0.011212391264194\\
425	0.0112182092165031\\
426	0.0112241008075525\\
427	0.0112300657083503\\
428	0.0112361034862987\\
429	0.0112422135944056\\
430	0.0112483953527047\\
431	0.0112546479364083\\
432	0.0112609703612202\\
433	0.0112673614189749\\
434	0.0112738196701308\\
435	0.0112803434700234\\
436	0.0112869309537705\\
437	0.0112935800206169\\
438	0.0113002883178527\\
439	0.0113070532246436\\
440	0.0113138718363307\\
441	0.0113207409477575\\
442	0.0113276570368504\\
443	0.0113346162496617\\
444	0.011341614386802\\
445	0.0113486468918518\\
446	0.0113557088424331\\
447	0.0113627949446761\\
448	0.0113698995317313\\
449	0.011377016566452\\
450	0.0113841396465713\\
451	0.0113912620056216\\
452	0.0113983764840891\\
453	0.0114054753577054\\
454	0.0114125499601084\\
455	0.0114195897940586\\
456	0.011426581285533\\
457	0.0114331440586039\\
458	0.0114397203007937\\
459	0.0114463786536373\\
460	0.011453119016052\\
461	0.0114599412068445\\
462	0.0114668449766635\\
463	0.0114738300041075\\
464	0.0114808958913022\\
465	0.0114880421588272\\
466	0.0114952682398705\\
467	0.0115025734735701\\
468	0.0115099570976205\\
469	0.0115174182397183\\
470	0.0115249559018402\\
471	0.0115325688961541\\
472	0.0115402569338782\\
473	0.011548020327366\\
474	0.0115558596095909\\
475	0.0115637755800602\\
476	0.0115717693561007\\
477	0.0115798424300584\\
478	0.0115879967375929\\
479	0.011596234728786\\
480	0.0116045594418825\\
481	0.0116129745760054\\
482	0.0116214845552705\\
483	0.0116300945888013\\
484	0.0116388107931745\\
485	0.0116476406653904\\
486	0.0116565977051972\\
487	0.0116660129631392\\
488	0.0116760382990391\\
489	0.0116862029418582\\
490	0.0116965081914692\\
491	0.0117069553306078\\
492	0.0117175455988874\\
493	0.0117282801581354\\
494	0.0117391600799758\\
495	0.0117501864587584\\
496	0.0117613603505247\\
497	0.0117726813501743\\
498	0.0117841468866146\\
499	0.0117957378514784\\
500	0.0118074921072991\\
501	0.0118192996705101\\
502	0.0118306980409187\\
503	0.0118422850852284\\
504	0.0118540650658937\\
505	0.011866049772292\\
506	0.0118782440898461\\
507	0.011890653015385\\
508	0.0119032803565499\\
509	0.0119161297087328\\
510	0.0119292050077963\\
511	0.0119425096887588\\
512	0.0119560455753848\\
513	0.0119698088735144\\
514	0.011983745912769\\
515	0.0119977821223298\\
516	0.0120120755612036\\
517	0.0120266301586594\\
518	0.0120414497002757\\
519	0.0120565384462788\\
520	0.0120719015972237\\
521	0.0120875456005788\\
522	0.0121034601280463\\
523	0.0121196329661394\\
524	0.0121360994941164\\
525	0.0121531615295992\\
526	0.0121708319651537\\
527	0.0121891410657405\\
528	0.0122081708257204\\
529	0.0122279455328579\\
530	0.0122484907427356\\
531	0.0122697965392347\\
532	0.012291936167081\\
533	0.012315035584419\\
534	0.0123393769489866\\
535	0.0123751637239336\\
536	0.0124159530873286\\
537	0.0124563393135354\\
538	0.0124959307877364\\
539	0.012530695077453\\
540	0.0125555631167354\\
541	0.0125795297041052\\
542	0.0126024024293412\\
543	0.0126251144855324\\
544	0.0126455089895527\\
545	0.0126649093133018\\
546	0.0126840555140884\\
547	0.0127029922190132\\
548	0.012720442782498\\
549	0.0127377865369162\\
550	0.012755294901288\\
551	0.0127729477055679\\
552	0.0127907691676174\\
553	0.0128084966990175\\
554	0.0128275737386326\\
555	0.012847119273552\\
556	0.0128665362193994\\
557	0.0128851039411577\\
558	0.0129031450891604\\
559	0.0129212911312746\\
560	0.0129394565977156\\
561	0.0129575756203802\\
562	0.0129759095087654\\
563	0.0129944520705167\\
564	0.0130132207588255\\
565	0.0130336906041906\\
566	0.0130542691102903\\
567	0.0130733199477811\\
568	0.013091815343303\\
569	0.0131100896839242\\
570	0.0131282466996527\\
571	0.0131465012870392\\
572	0.0131648559372097\\
573	0.0131832996784548\\
574	0.013202588645439\\
575	0.013221383677246\\
576	0.0132393615744666\\
577	0.013257082089171\\
578	0.0132747558182093\\
579	0.0132924132146983\\
580	0.01331003250515\\
581	0.0133275889338842\\
582	0.0133450552571701\\
583	0.0133624015807581\\
584	0.0133795951263379\\
585	0.0133965999873593\\
586	0.0134133769219085\\
587	0.0134298833026759\\
588	0.0134460735522417\\
589	0.0134619009495395\\
590	0.0134776600725805\\
591	0.0134935706885156\\
592	0.0135096572289508\\
593	0.0135260152836207\\
594	0.0135429334162338\\
595	0.0135612173030331\\
596	0.0135830518096797\\
597	0.0136142933738059\\
598	0.0136705746214114\\
599	0\\
600	0\\
};
\addplot [color=mycolor9,solid,forget plot]
  table[row sep=crcr]{%
1	0.0105882821173149\\
2	0.0105882869733345\\
3	0.0105882919114163\\
4	0.0105882969329423\\
5	0.0105883020393175\\
6	0.0105883072319704\\
7	0.0105883125123535\\
8	0.0105883178819437\\
9	0.0105883233422422\\
10	0.0105883288947758\\
11	0.0105883345410964\\
12	0.010588340282782\\
13	0.010588346121437\\
14	0.0105883520586925\\
15	0.010588358096207\\
16	0.0105883642356664\\
17	0.010588370478785\\
18	0.0105883768273057\\
19	0.0105883832830002\\
20	0.01058838984767\\
21	0.0105883965231466\\
22	0.010588403311292\\
23	0.0105884102139992\\
24	0.0105884172331927\\
25	0.0105884243708291\\
26	0.0105884316288976\\
27	0.0105884390094205\\
28	0.0105884465144538\\
29	0.0105884541460874\\
30	0.0105884619064463\\
31	0.0105884697976908\\
32	0.0105884778220169\\
33	0.0105884859816571\\
34	0.0105884942788811\\
35	0.0105885027159963\\
36	0.0105885112953482\\
37	0.0105885200193214\\
38	0.0105885288903399\\
39	0.0105885379108679\\
40	0.0105885470834106\\
41	0.0105885564105144\\
42	0.0105885658947681\\
43	0.0105885755388033\\
44	0.0105885853452949\\
45	0.0105885953169624\\
46	0.0105886054565701\\
47	0.0105886157669277\\
48	0.0105886262508917\\
49	0.0105886369113655\\
50	0.0105886477513004\\
51	0.0105886587736966\\
52	0.0105886699816035\\
53	0.0105886813781207\\
54	0.0105886929663991\\
55	0.0105887047496412\\
56	0.0105887167311022\\
57	0.010588728914091\\
58	0.0105887413019706\\
59	0.0105887538981594\\
60	0.0105887667061317\\
61	0.010588779729419\\
62	0.0105887929716106\\
63	0.0105888064363544\\
64	0.0105888201273583\\
65	0.0105888340483907\\
66	0.0105888482032816\\
67	0.0105888625959235\\
68	0.0105888772302725\\
69	0.0105888921103493\\
70	0.01058890724024\\
71	0.0105889226240974\\
72	0.0105889382661417\\
73	0.0105889541706618\\
74	0.0105889703420163\\
75	0.0105889867846347\\
76	0.010589003503018\\
77	0.0105890205017405\\
78	0.0105890377854502\\
79	0.0105890553588707\\
80	0.0105890732268017\\
81	0.0105890913941204\\
82	0.0105891098657826\\
83	0.0105891286468242\\
84	0.010589147742362\\
85	0.010589167157595\\
86	0.0105891868978058\\
87	0.0105892069683619\\
88	0.0105892273747165\\
89	0.0105892481224103\\
90	0.0105892692170725\\
91	0.0105892906644224\\
92	0.0105893124702702\\
93	0.0105893346405187\\
94	0.0105893571811649\\
95	0.0105893800983006\\
96	0.0105894033981145\\
97	0.0105894270868932\\
98	0.0105894511710228\\
99	0.0105894756569903\\
100	0.0105895005513848\\
101	0.0105895258608992\\
102	0.0105895515923315\\
103	0.0105895777525865\\
104	0.010589604348677\\
105	0.0105896313877254\\
106	0.0105896588769653\\
107	0.010589686823743\\
108	0.0105897152355188\\
109	0.0105897441198688\\
110	0.0105897734844864\\
111	0.0105898033371838\\
112	0.0105898336858936\\
113	0.0105898645386703\\
114	0.0105898959036922\\
115	0.0105899277892624\\
116	0.010589960203811\\
117	0.0105899931558965\\
118	0.0105900266542071\\
119	0.0105900607075629\\
120	0.010590095324917\\
121	0.0105901305153575\\
122	0.0105901662881089\\
123	0.0105902026525337\\
124	0.0105902396181343\\
125	0.0105902771945543\\
126	0.0105903153915804\\
127	0.0105903542191437\\
128	0.0105903936873219\\
129	0.01059043380634\\
130	0.010590474586573\\
131	0.0105905160385466\\
132	0.0105905581729392\\
133	0.0105906010005834\\
134	0.0105906445324679\\
135	0.0105906887797384\\
136	0.0105907337536996\\
137	0.0105907794658169\\
138	0.0105908259277175\\
139	0.0105908731511921\\
140	0.0105909211481965\\
141	0.0105909699308531\\
142	0.0105910195114521\\
143	0.0105910699024533\\
144	0.0105911211164874\\
145	0.0105911731663574\\
146	0.0105912260650402\\
147	0.010591279825688\\
148	0.0105913344616296\\
149	0.0105913899863722\\
150	0.0105914464136025\\
151	0.0105915037571885\\
152	0.0105915620311808\\
153	0.0105916212498144\\
154	0.0105916814275104\\
155	0.0105917425788775\\
156	0.0105918047187143\\
157	0.0105918678620109\\
158	0.0105919320239512\\
159	0.0105919972199156\\
160	0.0105920634654831\\
161	0.0105921307764348\\
162	0.0105921991687571\\
163	0.0105922686586453\\
164	0.0105923392625082\\
165	0.0105924109969731\\
166	0.0105924838788913\\
167	0.010592557925345\\
168	0.010592633153655\\
169	0.0105927095813896\\
170	0.0105927872263753\\
171	0.0105928661067088\\
172	0.0105929462407716\\
173	0.0105930276472464\\
174	0.0105931103451372\\
175	0.0105931943537915\\
176	0.0105932796929281\\
177	0.0105933663826675\\
178	0.0105934544435695\\
179	0.0105935438966759\\
180	0.010593634763561\\
181	0.0105937270663907\\
182	0.0105938208279913\\
183	0.0105939160719308\\
184	0.010594012822613\\
185	0.0105941111053886\\
186	0.010594210946684\\
187	0.0105943123741526\\
188	0.01059441541685\\
189	0.0105945201054364\\
190	0.0105946264724068\\
191	0.010594734552342\\
192	0.0105948443821615\\
193	0.0105949560013257\\
194	0.0105950694518863\\
195	0.0105951847782612\\
196	0.0105953020269754\\
197	0.010595421249013\\
198	0.0105955425190109\\
199	0.0105956658734649\\
200	0.0105957913477456\\
201	0.0105959189778089\\
202	0.0105960488002058\\
203	0.0105961808520913\\
204	0.0105963151712341\\
205	0.0105964517960267\\
206	0.0105965907654947\\
207	0.0105967321193071\\
208	0.0105968758977866\\
209	0.0105970221419194\\
210	0.0105971708933663\\
211	0.0105973221944729\\
212	0.0105974760882805\\
213	0.0105976326185371\\
214	0.0105977918297087\\
215	0.01059795376699\\
216	0.0105981184763166\\
217	0.0105982860043763\\
218	0.0105984563986208\\
219	0.0105986297072781\\
220	0.0105988059793641\\
221	0.0105989852646958\\
222	0.010599167613903\\
223	0.0105993530784418\\
224	0.0105995417106073\\
225	0.0105997335635464\\
226	0.0105999286912719\\
227	0.0106001271486755\\
228	0.0106003289915419\\
229	0.0106005342765627\\
230	0.0106007430613507\\
231	0.010600955404454\\
232	0.0106011713653713\\
233	0.0106013910045662\\
234	0.0106016143834825\\
235	0.0106018415645598\\
236	0.0106020726112485\\
237	0.0106023075880262\\
238	0.0106025465604132\\
239	0.0106027895949891\\
240	0.010603036759409\\
241	0.0106032881224207\\
242	0.0106035437538808\\
243	0.0106038037247729\\
244	0.0106040681072241\\
245	0.010604336974523\\
246	0.0106046104011377\\
247	0.0106048884627337\\
248	0.0106051712361918\\
249	0.0106054587996274\\
250	0.0106057512324086\\
251	0.0106060486151752\\
252	0.0106063510298581\\
253	0.0106066585596983\\
254	0.0106069712892664\\
255	0.0106072893044829\\
256	0.0106076126926373\\
257	0.0106079415424086\\
258	0.0106082759438857\\
259	0.0106086159885877\\
260	0.0106089617694844\\
261	0.0106093133810178\\
262	0.0106096709191222\\
263	0.0106100344812465\\
264	0.0106104041663751\\
265	0.0106107800750498\\
266	0.0106111623093918\\
267	0.0106115509731239\\
268	0.0106119461715934\\
269	0.0106123480117944\\
270	0.0106127566023916\\
271	0.0106131720537435\\
272	0.0106135944779263\\
273	0.0106140239887585\\
274	0.0106144607018254\\
275	0.0106149047345041\\
276	0.0106153562059893\\
277	0.0106158152373189\\
278	0.0106162819514002\\
279	0.0106167564730366\\
280	0.0106172389289546\\
281	0.0106177294478309\\
282	0.0106182281603205\\
283	0.0106187351990843\\
284	0.0106192506988181\\
285	0.0106197747962812\\
286	0.0106203076303263\\
287	0.0106208493419286\\
288	0.0106214000742169\\
289	0.0106219599725039\\
290	0.0106225291843177\\
291	0.0106231078594339\\
292	0.0106236961499075\\
293	0.0106242942101063\\
294	0.0106249021967442\\
295	0.010625520268916\\
296	0.0106261485881314\\
297	0.0106267873183512\\
298	0.0106274366260237\\
299	0.0106280966801212\\
300	0.0106287676521784\\
301	0.0106294497163311\\
302	0.0106301430493555\\
303	0.0106308478307091\\
304	0.0106315642425725\\
305	0.0106322924698915\\
306	0.0106330327004218\\
307	0.0106337851247731\\
308	0.0106345499364563\\
309	0.0106353273319309\\
310	0.010636117510654\\
311	0.0106369206751317\\
312	0.0106377370309709\\
313	0.010638566786934\\
314	0.0106394101549951\\
315	0.0106402673503983\\
316	0.0106411385917179\\
317	0.0106420241009216\\
318	0.010642924103436\\
319	0.0106438388282147\\
320	0.0106447685078092\\
321	0.0106457133784438\\
322	0.010646673680093\\
323	0.0106476496565634\\
324	0.0106486415555785\\
325	0.0106496496288695\\
326	0.0106506741322689\\
327	0.0106517153258107\\
328	0.0106527734738349\\
329	0.0106538488450987\\
330	0.0106549417128936\\
331	0.0106560523551696\\
332	0.0106571810546672\\
333	0.0106583280990565\\
334	0.0106594937810867\\
335	0.0106606783987437\\
336	0.0106618822554192\\
337	0.0106631056600903\\
338	0.0106643489275123\\
339	0.010665612378424\\
340	0.0106668963397684\\
341	0.0106682011449292\\
342	0.0106695271339841\\
343	0.0106708746539783\\
344	0.0106722440592176\\
345	0.0106736357115852\\
346	0.0106750499808831\\
347	0.0106764872452005\\
348	0.010677947891313\\
349	0.0106794323151141\\
350	0.0106809409220833\\
351	0.0106824741277937\\
352	0.0106840323584643\\
353	0.0106856160515599\\
354	0.0106872256564456\\
355	0.0106888616350993\\
356	0.0106905244628912\\
357	0.0106922146294349\\
358	0.0106939326395195\\
359	0.0106956790141321\\
360	0.0106974542915789\\
361	0.0106992590287188\\
362	0.0107010938023203\\
363	0.0107029592105574\\
364	0.0107048558746606\\
365	0.0107067844407423\\
366	0.0107087455818163\\
367	0.0107107400000378\\
368	0.0107127684291883\\
369	0.0107148316374397\\
370	0.0107169304304309\\
371	0.0107190656546975\\
372	0.0107212382015024\\
373	0.0107234490111175\\
374	0.0107256990776188\\
375	0.0107279894542626\\
376	0.0107303212595217\\
377	0.0107326956838715\\
378	0.0107351139974308\\
379	0.010737577558577\\
380	0.0107400878236808\\
381	0.0107426463581461\\
382	0.0107452548490297\\
383	0.0107479151197616\\
384	0.0107506291482134\\
385	0.0107533990916765\\
386	0.0107562273298109\\
387	0.0107591165613434\\
388	0.0107620700726854\\
389	0.0107650925740865\\
390	0.0107681839027551\\
391	0.0107713467298691\\
392	0.0107745845866781\\
393	0.0107779008727301\\
394	0.0107812979535853\\
395	0.0107847745353099\\
396	0.0107883459649614\\
397	0.0107920308416895\\
398	0.0107958379074395\\
399	0.0107997769873472\\
400	0.0108038591971218\\
401	0.0108080973237868\\
402	0.0108125067615679\\
403	0.0108171083567521\\
404	0.0108222939710062\\
405	0.0108283517081059\\
406	0.0108345037024479\\
407	0.0108407512128546\\
408	0.0108470955043369\\
409	0.0108535378472801\\
410	0.0108600795165473\\
411	0.0108667217905024\\
412	0.0108734659499767\\
413	0.0108803132772462\\
414	0.0108872650551996\\
415	0.0108943225671414\\
416	0.0109014870983052\\
417	0.0109087599416243\\
418	0.0109161424138475\\
419	0.0109236357570528\\
420	0.0109312411973846\\
421	0.0109389599534928\\
422	0.0109467932348981\\
423	0.010954742241243\\
424	0.0109628081651141\\
425	0.0109709922074698\\
426	0.0109792956363064\\
427	0.0109877199935107\\
428	0.0109962658135513\\
429	0.0110049333756634\\
430	0.0110137234563363\\
431	0.0110226363181366\\
432	0.011031670821521\\
433	0.0110408307285183\\
434	0.0110501203599107\\
435	0.0110595402806774\\
436	0.0110690909020248\\
437	0.0110787724542857\\
438	0.0110885849550169\\
439	0.0110985281718859\\
440	0.0111086015895066\\
441	0.0111188043639162\\
442	0.0111291352640944\\
443	0.0111395926071258\\
444	0.0111501741827598\\
445	0.0111608771657314\\
446	0.0111716980139924\\
447	0.0111826323507832\\
448	0.0111936748282306\\
449	0.0112048189697973\\
450	0.011216056988005\\
451	0.0112273795703483\\
452	0.0112387756118446\\
453	0.0112502318139983\\
454	0.0112617318140141\\
455	0.011273253485234\\
456	0.0112847590516309\\
457	0.0112944063106285\\
458	0.0113038963625982\\
459	0.0113135001692406\\
460	0.0113232168485268\\
461	0.0113330451735367\\
462	0.0113429831217872\\
463	0.0113530283549678\\
464	0.0113631781965573\\
465	0.011373429608432\\
466	0.0113837791674379\\
467	0.0113942230421108\\
468	0.0114047569699496\\
469	0.0114153762355743\\
470	0.0114260756498903\\
471	0.0114368495401707\\
472	0.0114476916898557\\
473	0.0114585952848931\\
474	0.0114695528843065\\
475	0.0114805563932684\\
476	0.0114915970403096\\
477	0.0115026653615329\\
478	0.0115137511375012\\
479	0.0115248434044239\\
480	0.0115359304427172\\
481	0.0115469997346382\\
482	0.0115580378768194\\
483	0.0115690302582419\\
484	0.011579960241225\\
485	0.0115908075019726\\
486	0.0116014527819778\\
487	0.0116114232324257\\
488	0.0116214988866434\\
489	0.0116316794477841\\
490	0.0116419647365788\\
491	0.0116523547301622\\
492	0.0116628496071647\\
493	0.0116734497998251\\
494	0.0116841560534704\\
495	0.0116949694884054\\
496	0.0117058916589316\\
497	0.0117169246366177\\
498	0.0117280710951212\\
499	0.0117393351197034\\
500	0.0117507210941049\\
501	0.0117623964024197\\
502	0.0117749175192852\\
503	0.0117875220285785\\
504	0.0118002096586272\\
505	0.0118130185042998\\
506	0.0118259526254834\\
507	0.0118390180919553\\
508	0.0118522567247901\\
509	0.0118656701057629\\
510	0.011879259132052\\
511	0.0118930374071062\\
512	0.0119070082630888\\
513	0.0119211814661936\\
514	0.0119353650254472\\
515	0.0119491573245559\\
516	0.0119631723649265\\
517	0.0119774150521919\\
518	0.0119918905469862\\
519	0.0120066040924986\\
520	0.0120215609423588\\
521	0.0120367663735222\\
522	0.0120522265399429\\
523	0.0120679484874184\\
524	0.0120839363495991\\
525	0.012100171117324\\
526	0.0121165803990822\\
527	0.012133108289083\\
528	0.0121502430616949\\
529	0.0121680012019229\\
530	0.012186398179634\\
531	0.0122055196227654\\
532	0.012225399098003\\
533	0.0122460625176041\\
534	0.0122675370969204\\
535	0.01228982255668\\
536	0.0123129682944442\\
537	0.0123370780876139\\
538	0.0123623730605596\\
539	0.0123937450259851\\
540	0.0124363547764866\\
541	0.0124786602990867\\
542	0.0125203281603286\\
543	0.0125613615910063\\
544	0.0125912739359988\\
545	0.0126175155014747\\
546	0.0126427869464446\\
547	0.0126669245066154\\
548	0.0126912012121315\\
549	0.0127124236645317\\
550	0.0127333664161979\\
551	0.0127540418041656\\
552	0.0127743761770403\\
553	0.0127930347629488\\
554	0.012811842867462\\
555	0.0128308225928537\\
556	0.0128500137122757\\
557	0.0128701860574235\\
558	0.0128906988843134\\
559	0.012911073974177\\
560	0.0129309301324748\\
561	0.012949835704869\\
562	0.0129688253261797\\
563	0.0129878931867068\\
564	0.0130068213234517\\
565	0.0130258909374129\\
566	0.0130451618827012\\
567	0.0130659648405829\\
568	0.0130870114454352\\
569	0.0131068557406383\\
570	0.0131257575325211\\
571	0.0131445257604034\\
572	0.013162987270045\\
573	0.0131815165110982\\
574	0.0132001107773793\\
575	0.0132191264797513\\
576	0.0132385381519454\\
577	0.0132568110091697\\
578	0.0132747362574936\\
579	0.0132924085254632\\
580	0.0133100302842094\\
581	0.0133275877217758\\
582	0.013345054604795\\
583	0.0133624012494927\\
584	0.0133795949711217\\
585	0.0133965999220869\\
586	0.0134133768981876\\
587	0.0134298832956531\\
588	0.0134460735507277\\
589	0.0134619009493599\\
590	0.0134776600725805\\
591	0.0134935706885156\\
592	0.0135096572289508\\
593	0.0135260152836207\\
594	0.0135429334162338\\
595	0.0135612173030331\\
596	0.0135830518096797\\
597	0.0136142933738059\\
598	0.0136705746214114\\
599	0\\
600	0\\
};
\addplot [color=blue!50!mycolor7,solid,forget plot]
  table[row sep=crcr]{%
1	0.0104618560317627\\
2	0.010461860056769\\
3	0.0104618641496725\\
4	0.0104618683116139\\
5	0.010461872543753\\
6	0.0104618768472689\\
7	0.0104618812233603\\
8	0.010461885673246\\
9	0.0104618901981649\\
10	0.0104618947993768\\
11	0.0104618994781623\\
12	0.0104619042358235\\
13	0.0104619090736838\\
14	0.0104619139930891\\
15	0.0104619189954075\\
16	0.0104619240820296\\
17	0.0104619292543695\\
18	0.0104619345138647\\
19	0.0104619398619764\\
20	0.0104619453001904\\
21	0.0104619508300168\\
22	0.0104619564529911\\
23	0.0104619621706741\\
24	0.0104619679846526\\
25	0.0104619738965396\\
26	0.0104619799079749\\
27	0.0104619860206256\\
28	0.0104619922361863\\
29	0.0104619985563796\\
30	0.0104620049829567\\
31	0.0104620115176979\\
32	0.0104620181624129\\
33	0.0104620249189411\\
34	0.0104620317891526\\
35	0.0104620387749483\\
36	0.0104620458782604\\
37	0.0104620531010533\\
38	0.0104620604453234\\
39	0.0104620679131005\\
40	0.0104620755064475\\
41	0.0104620832274615\\
42	0.0104620910782741\\
43	0.0104620990610519\\
44	0.0104621071779975\\
45	0.0104621154313493\\
46	0.0104621238233828\\
47	0.0104621323564109\\
48	0.0104621410327843\\
49	0.0104621498548926\\
50	0.0104621588251643\\
51	0.0104621679460679\\
52	0.0104621772201123\\
53	0.0104621866498477\\
54	0.0104621962378659\\
55	0.010462205986801\\
56	0.0104622158993305\\
57	0.0104622259781753\\
58	0.010462236226101\\
59	0.0104622466459184\\
60	0.010462257240484\\
61	0.0104622680127009\\
62	0.0104622789655196\\
63	0.0104622901019384\\
64	0.0104623014250048\\
65	0.0104623129378156\\
66	0.0104623246435178\\
67	0.0104623365453098\\
68	0.0104623486464417\\
69	0.0104623609502163\\
70	0.0104623734599901\\
71	0.0104623861791737\\
72	0.0104623991112331\\
73	0.01046241225969\\
74	0.0104624256281233\\
75	0.0104624392201696\\
76	0.0104624530395239\\
77	0.0104624670899409\\
78	0.0104624813752359\\
79	0.0104624958992853\\
80	0.0104625106660278\\
81	0.0104625256794654\\
82	0.0104625409436643\\
83	0.0104625564627558\\
84	0.0104625722409374\\
85	0.0104625882824737\\
86	0.0104626045916973\\
87	0.0104626211730101\\
88	0.0104626380308841\\
89	0.0104626551698627\\
90	0.0104626725945613\\
91	0.0104626903096689\\
92	0.0104627083199489\\
93	0.0104627266302401\\
94	0.0104627452454582\\
95	0.0104627641705965\\
96	0.0104627834107274\\
97	0.0104628029710032\\
98	0.0104628228566578\\
99	0.0104628430730071\\
100	0.0104628636254512\\
101	0.0104628845194746\\
102	0.0104629057606482\\
103	0.0104629273546304\\
104	0.0104629493071679\\
105	0.0104629716240975\\
106	0.0104629943113474\\
107	0.0104630173749381\\
108	0.0104630408209843\\
109	0.0104630646556957\\
110	0.0104630888853788\\
111	0.0104631135164382\\
112	0.0104631385553777\\
113	0.0104631640088024\\
114	0.0104631898834195\\
115	0.0104632161860402\\
116	0.0104632429235811\\
117	0.0104632701030655\\
118	0.0104632977316256\\
119	0.0104633258165034\\
120	0.0104633543650526\\
121	0.0104633833847404\\
122	0.0104634128831489\\
123	0.0104634428679772\\
124	0.0104634733470427\\
125	0.0104635043282833\\
126	0.010463535819759\\
127	0.0104635678296537\\
128	0.0104636003662776\\
129	0.0104636334380687\\
130	0.0104636670535949\\
131	0.0104637012215563\\
132	0.0104637359507873\\
133	0.0104637712502586\\
134	0.0104638071290799\\
135	0.0104638435965018\\
136	0.0104638806619187\\
137	0.0104639183348708\\
138	0.0104639566250475\\
139	0.0104639955422894\\
140	0.0104640350965916\\
141	0.0104640752981067\\
142	0.0104641161571476\\
143	0.010464157684191\\
144	0.0104641998898809\\
145	0.010464242785032\\
146	0.0104642863806336\\
147	0.0104643306878535\\
148	0.0104643757180422\\
149	0.0104644214827377\\
150	0.0104644679936694\\
151	0.0104645152627639\\
152	0.0104645633021498\\
153	0.0104646121241635\\
154	0.0104646617413549\\
155	0.010464712166494\\
156	0.0104647634125775\\
157	0.0104648154928361\\
158	0.0104648684207417\\
159	0.0104649222100164\\
160	0.0104649768746404\\
161	0.0104650324288617\\
162	0.0104650888872059\\
163	0.0104651462644871\\
164	0.0104652045758186\\
165	0.0104652638366254\\
166	0.0104653240626571\\
167	0.0104653852700012\\
168	0.0104654474750979\\
169	0.0104655106947555\\
170	0.0104655749461667\\
171	0.0104656402469258\\
172	0.0104657066150471\\
173	0.0104657740689839\\
174	0.0104658426276485\\
175	0.010465912310433\\
176	0.0104659831372309\\
177	0.0104660551284591\\
178	0.0104661283050806\\
179	0.0104662026886267\\
180	0.0104662783012198\\
181	0.0104663551655947\\
182	0.0104664333051195\\
183	0.0104665127438136\\
184	0.0104665935063635\\
185	0.0104666756181343\\
186	0.0104667591051759\\
187	0.0104668439942217\\
188	0.010466930312678\\
189	0.010467018088601\\
190	0.0104671073506577\\
191	0.0104671981280662\\
192	0.0104672904505093\\
193	0.0104673843480157\\
194	0.0104674798508062\\
195	0.0104675769891198\\
196	0.0104676757930652\\
197	0.0104677762925177\\
198	0.0104678785163132\\
199	0.0104679824937038\\
200	0.0104680882544284\\
201	0.0104681958287215\\
202	0.0104683052473206\\
203	0.0104684165414749\\
204	0.0104685297429532\\
205	0.0104686448840527\\
206	0.0104687619976074\\
207	0.0104688811169971\\
208	0.010469002276156\\
209	0.0104691255095817\\
210	0.0104692508523449\\
211	0.0104693783400981\\
212	0.0104695080090855\\
213	0.0104696398961526\\
214	0.0104697740387557\\
215	0.0104699104749725\\
216	0.0104700492435114\\
217	0.0104701903837225\\
218	0.0104703339356078\\
219	0.0104704799398317\\
220	0.0104706284377322\\
221	0.0104707794713315\\
222	0.0104709330833474\\
223	0.0104710893172048\\
224	0.0104712482170467\\
225	0.0104714098277466\\
226	0.0104715741949202\\
227	0.0104717413649372\\
228	0.0104719113849342\\
229	0.0104720843028267\\
230	0.0104722601673225\\
231	0.0104724390279336\\
232	0.0104726209349907\\
233	0.0104728059396554\\
234	0.0104729940939343\\
235	0.0104731854506931\\
236	0.01047338006367\\
237	0.0104735779874903\\
238	0.0104737792776811\\
239	0.0104739839906856\\
240	0.0104741921838784\\
241	0.0104744039155806\\
242	0.0104746192450751\\
243	0.0104748382326227\\
244	0.0104750609394776\\
245	0.010475287427904\\
246	0.0104755177611925\\
247	0.0104757520036766\\
248	0.0104759902207501\\
249	0.0104762324788842\\
250	0.0104764788456454\\
251	0.0104767293897128\\
252	0.010476984180897\\
253	0.0104772432901583\\
254	0.0104775067896255\\
255	0.0104777747526152\\
256	0.0104780472536514\\
257	0.0104783243684852\\
258	0.0104786061741152\\
259	0.0104788927488081\\
260	0.0104791841721198\\
261	0.0104794805249166\\
262	0.0104797818893977\\
263	0.0104800883491168\\
264	0.0104803999890052\\
265	0.0104807168953954\\
266	0.0104810391560442\\
267	0.0104813668601578\\
268	0.0104817000984159\\
269	0.0104820389629975\\
270	0.010482383547607\\
271	0.0104827339475003\\
272	0.0104830902595123\\
273	0.010483452582085\\
274	0.0104838210152953\\
275	0.0104841956608848\\
276	0.0104845766222894\\
277	0.0104849640046699\\
278	0.0104853579149435\\
279	0.0104857584618159\\
280	0.0104861657558143\\
281	0.0104865799093212\\
282	0.0104870010366094\\
283	0.0104874292538773\\
284	0.0104878646792861\\
285	0.0104883074329971\\
286	0.0104887576372105\\
287	0.0104892154162056\\
288	0.0104896808963815\\
289	0.0104901542062998\\
290	0.0104906354767278\\
291	0.0104911248406835\\
292	0.0104916224334821\\
293	0.0104921283927839\\
294	0.0104926428586432\\
295	0.0104931659735596\\
296	0.0104936978825309\\
297	0.010494238733107\\
298	0.0104947886754468\\
299	0.0104953478623762\\
300	0.0104959164494488\\
301	0.0104964945950083\\
302	0.0104970824602532\\
303	0.0104976802093047\\
304	0.010498288009276\\
305	0.0104989060303451\\
306	0.0104995344458298\\
307	0.0105001734322668\\
308	0.0105008231694921\\
309	0.0105014838407266\\
310	0.0105021556326637\\
311	0.0105028387355615\\
312	0.0105035333433382\\
313	0.0105042396536723\\
314	0.0105049578681062\\
315	0.0105056881921549\\
316	0.0105064308354195\\
317	0.0105071860117052\\
318	0.0105079539391454\\
319	0.0105087348403305\\
320	0.0105095289424433\\
321	0.0105103364774002\\
322	0.0105111576819992\\
323	0.0105119927980751\\
324	0.0105128420726611\\
325	0.010513705758159\\
326	0.0105145841125174\\
327	0.0105154773994183\\
328	0.0105163858884724\\
329	0.0105173098554254\\
330	0.0105182495823722\\
331	0.0105192053579842\\
332	0.0105201774777458\\
333	0.0105211662442038\\
334	0.0105221719672294\\
335	0.0105231949642927\\
336	0.010524235560752\\
337	0.0105252940901566\\
338	0.0105263708945663\\
339	0.0105274663248859\\
340	0.0105285807412178\\
341	0.0105297145132322\\
342	0.0105308680205556\\
343	0.01053204165318\\
344	0.0105332358118926\\
345	0.0105344509087271\\
346	0.0105356873674379\\
347	0.0105369456239981\\
348	0.0105382261271231\\
349	0.0105395293388189\\
350	0.0105408557349589\\
351	0.0105422058058875\\
352	0.0105435800570538\\
353	0.0105449790096746\\
354	0.0105464032014287\\
355	0.010547853187183\\
356	0.0105493295397505\\
357	0.0105508328506814\\
358	0.0105523637310874\\
359	0.0105539228124989\\
360	0.0105555107477546\\
361	0.0105571282119233\\
362	0.0105587759032566\\
363	0.0105604545441695\\
364	0.0105621648822476\\
365	0.0105639076912768\\
366	0.0105656837722904\\
367	0.0105674939546281\\
368	0.0105693390970002\\
369	0.0105712200885454\\
370	0.0105731378498737\\
371	0.010575093334078\\
372	0.0105770875276986\\
373	0.0105791214516199\\
374	0.0105811961618747\\
375	0.0105833127503257\\
376	0.0105854723451908\\
377	0.0105876761113692\\
378	0.0105899252505203\\
379	0.0105922210008376\\
380	0.0105945646364528\\
381	0.0105969574664004\\
382	0.0105994008330703\\
383	0.0106018961101059\\
384	0.010604444699769\\
385	0.0106070480299474\\
386	0.0106097075511195\\
387	0.0106124247327716\\
388	0.0106152010516376\\
389	0.0106180379251555\\
390	0.0106209368573581\\
391	0.010623899352566\\
392	0.0106269268723012\\
393	0.0106300208055151\\
394	0.0106331824587128\\
395	0.0106364132202592\\
396	0.0106397140817526\\
397	0.0106430854708272\\
398	0.0106465275001525\\
399	0.010650039906089\\
400	0.0106536219861182\\
401	0.0106572725450429\\
402	0.0106609898616465\\
403	0.0106647716310313\\
404	0.0106686140380545\\
405	0.0106725135220054\\
406	0.0106764707781298\\
407	0.0106804865193999\\
408	0.0106845614692874\\
409	0.0106886963628523\\
410	0.0106928919479593\\
411	0.0106971489866214\\
412	0.0107014682564653\\
413	0.0107058505522971\\
414	0.0107102966877097\\
415	0.0107148074965538\\
416	0.0107193838336482\\
417	0.0107240265724335\\
418	0.0107287365916548\\
419	0.0107335147276244\\
420	0.0107383618319532\\
421	0.0107432787841063\\
422	0.0107482664978915\\
423	0.0107533259346295\\
424	0.01075845813697\\
425	0.0107636643291846\\
426	0.0107689462360031\\
427	0.010774307129683\\
428	0.0107797452619875\\
429	0.0107852576133016\\
430	0.0107908438139842\\
431	0.0107965011654771\\
432	0.0108022203616186\\
433	0.0108080180536345\\
434	0.0108139153864526\\
435	0.0108199158551454\\
436	0.0108260233088902\\
437	0.0108322419996992\\
438	0.0108385766462874\\
439	0.0108450325139448\\
440	0.0108516152105582\\
441	0.0108583307907356\\
442	0.0108651860064563\\
443	0.010872188405137\\
444	0.0108793464418683\\
445	0.0108866696078573\\
446	0.010894168577415\\
447	0.0109018553762157\\
448	0.0109097435741712\\
449	0.0109178485075051\\
450	0.0109261875378234\\
451	0.0109347803655851\\
452	0.0109436494463235\\
453	0.0109528206616975\\
454	0.0109623247555887\\
455	0.0109722013074967\\
456	0.0109825058726065\\
457	0.01099545753385\\
458	0.0110089126592302\\
459	0.0110225302357228\\
460	0.0110363128007854\\
461	0.0110502731381233\\
462	0.0110644319979453\\
463	0.0110787903669355\\
464	0.0110933490097762\\
465	0.0111081084434564\\
466	0.0111230688953945\\
467	0.0111382302543363\\
468	0.0111535920123783\\
469	0.0111691531954047\\
470	0.0111849122764116\\
471	0.0112008671089404\\
472	0.0112170148359317\\
473	0.0112333517472362\\
474	0.0112498731450758\\
475	0.0112665731930428\\
476	0.0112834447524771\\
477	0.0113004792192295\\
478	0.0113176660396245\\
479	0.0113349925622321\\
480	0.0113524437520336\\
481	0.0113700017896465\\
482	0.0113876457231932\\
483	0.011405350622941\\
484	0.0114230856081108\\
485	0.0114408073216547\\
486	0.011457983511922\\
487	0.011472199867011\\
488	0.0114865411775527\\
489	0.0115010012507893\\
490	0.0115155732302646\\
491	0.0115302495490025\\
492	0.0115450218804916\\
493	0.0115598810875777\\
494	0.0115748171693553\\
495	0.0115898192062262\\
496	0.0116048753035576\\
497	0.0116199725341832\\
498	0.0116350968830373\\
499	0.0116502331681932\\
500	0.0116653649892731\\
501	0.0116804746195786\\
502	0.011695543210039\\
503	0.0117105526352493\\
504	0.0117254830521333\\
505	0.0117403106903827\\
506	0.0117550074041313\\
507	0.0117695381018975\\
508	0.011783013940493\\
509	0.0117964502549902\\
510	0.0118099809815712\\
511	0.0118236600594836\\
512	0.0118374898629259\\
513	0.0118514737529174\\
514	0.0118658665923287\\
515	0.0118811491201193\\
516	0.0118965571693358\\
517	0.0119120850408738\\
518	0.0119277273082647\\
519	0.0119434929809049\\
520	0.0119593943446705\\
521	0.0119754343954329\\
522	0.0119916172427691\\
523	0.0120079487568465\\
524	0.0120244462732379\\
525	0.0120411699911785\\
526	0.0120579255312488\\
527	0.0120742467757627\\
528	0.01209080789069\\
529	0.0121075970045605\\
530	0.0121246006526744\\
531	0.0121421464440662\\
532	0.0121603012456835\\
533	0.0121790836751113\\
534	0.0121985137317768\\
535	0.0122186995944744\\
536	0.0122396638898131\\
537	0.0122614166858521\\
538	0.0122838046561501\\
539	0.0123070247766448\\
540	0.0123311275844313\\
541	0.0123562014373352\\
542	0.0123823918895241\\
543	0.0124100078596024\\
544	0.0124509639187759\\
545	0.0124953241835517\\
546	0.0125392519891677\\
547	0.0125826448821141\\
548	0.0126243945697375\\
549	0.0126531879732461\\
550	0.0126811672151636\\
551	0.0127081372590588\\
552	0.012734095733079\\
553	0.0127601828272216\\
554	0.0127831409655147\\
555	0.0128057551240993\\
556	0.0128281107156632\\
557	0.012849915479683\\
558	0.0128702202302514\\
559	0.0128906869227948\\
560	0.0129117696110277\\
561	0.0129339377115742\\
562	0.0129557725987332\\
563	0.0129770789041954\\
564	0.0129971434859475\\
565	0.0130170612757732\\
566	0.0130370255858518\\
567	0.0130569582570077\\
568	0.0130767861265584\\
569	0.0130977213797593\\
570	0.013119342908312\\
571	0.0131402949957096\\
572	0.0131596331873905\\
573	0.0131788550623476\\
574	0.0131977785716585\\
575	0.013216557170478\\
576	0.0132353572130932\\
577	0.0132549395653295\\
578	0.0132740795174773\\
579	0.0132921691907292\\
580	0.0133099968721648\\
581	0.0133275744670578\\
582	0.0133450474098974\\
583	0.0133623972394724\\
584	0.0133795928411548\\
585	0.0133965988697241\\
586	0.0134133764286198\\
587	0.0134298831135822\\
588	0.0134460734928525\\
589	0.0134619009358821\\
590	0.0134776600708209\\
591	0.0134935706885156\\
592	0.0135096572289508\\
593	0.0135260152836207\\
594	0.0135429334162338\\
595	0.0135612173030331\\
596	0.0135830518096797\\
597	0.0136142933738059\\
598	0.0136705746214114\\
599	0\\
600	0\\
};
\addplot [color=blue!40!mycolor9,solid,forget plot]
  table[row sep=crcr]{%
1	0.0101521730102996\\
2	0.0101521777761881\\
3	0.0101521826223211\\
4	0.0101521875500457\\
5	0.0101521925607317\\
6	0.0101521976557714\\
7	0.0101522028365809\\
8	0.0101522081045996\\
9	0.010152213461291\\
10	0.0101522189081431\\
11	0.0101522244466689\\
12	0.0101522300784065\\
13	0.0101522358049197\\
14	0.0101522416277984\\
15	0.0101522475486593\\
16	0.0101522535691457\\
17	0.0101522596909286\\
18	0.0101522659157069\\
19	0.0101522722452078\\
20	0.0101522786811872\\
21	0.0101522852254305\\
22	0.0101522918797529\\
23	0.0101522986459996\\
24	0.0101523055260469\\
25	0.0101523125218022\\
26	0.0101523196352048\\
27	0.0101523268682264\\
28	0.0101523342228714\\
29	0.0101523417011777\\
30	0.0101523493052173\\
31	0.0101523570370964\\
32	0.0101523648989566\\
33	0.010152372892975\\
34	0.0101523810213652\\
35	0.0101523892863774\\
36	0.0101523976902994\\
37	0.010152406235457\\
38	0.010152414924215\\
39	0.0101524237589771\\
40	0.0101524327421876\\
41	0.0101524418763308\\
42	0.0101524511639329\\
43	0.0101524606075619\\
44	0.0101524702098284\\
45	0.0101524799733866\\
46	0.0101524899009348\\
47	0.0101524999952161\\
48	0.0101525102590192\\
49	0.0101525206951792\\
50	0.0101525313065783\\
51	0.0101525420961465\\
52	0.0101525530668627\\
53	0.0101525642217549\\
54	0.0101525755639017\\
55	0.0101525870964328\\
56	0.0101525988225297\\
57	0.0101526107454267\\
58	0.010152622868412\\
59	0.0101526351948281\\
60	0.0101526477280732\\
61	0.0101526604716015\\
62	0.0101526734289249\\
63	0.0101526866036133\\
64	0.0101526999992958\\
65	0.0101527136196618\\
66	0.0101527274684615\\
67	0.0101527415495077\\
68	0.0101527558666761\\
69	0.0101527704239066\\
70	0.0101527852252044\\
71	0.0101528002746412\\
72	0.0101528155763558\\
73	0.0101528311345557\\
74	0.0101528469535181\\
75	0.010152863037591\\
76	0.0101528793911941\\
77	0.0101528960188206\\
78	0.0101529129250379\\
79	0.0101529301144888\\
80	0.0101529475918932\\
81	0.0101529653620489\\
82	0.010152983429833\\
83	0.0101530018002034\\
84	0.0101530204781999\\
85	0.0101530394689458\\
86	0.0101530587776489\\
87	0.0101530784096033\\
88	0.0101530983701906\\
89	0.0101531186648812\\
90	0.0101531392992364\\
91	0.0101531602789089\\
92	0.0101531816096455\\
93	0.0101532032972876\\
94	0.0101532253477733\\
95	0.0101532477671391\\
96	0.0101532705615214\\
97	0.0101532937371579\\
98	0.0101533173003898\\
99	0.0101533412576631\\
100	0.0101533656155308\\
101	0.0101533903806542\\
102	0.0101534155598052\\
103	0.0101534411598677\\
104	0.0101534671878401\\
105	0.0101534936508368\\
106	0.0101535205560901\\
107	0.0101535479109526\\
108	0.0101535757228993\\
109	0.0101536039995291\\
110	0.0101536327485676\\
111	0.0101536619778692\\
112	0.0101536916954189\\
113	0.0101537219093351\\
114	0.0101537526278719\\
115	0.0101537838594211\\
116	0.0101538156125153\\
117	0.0101538478958296\\
118	0.0101538807181851\\
119	0.0101539140885509\\
120	0.0101539480160468\\
121	0.0101539825099466\\
122	0.0101540175796804\\
123	0.0101540532348379\\
124	0.010154089485171\\
125	0.0101541263405972\\
126	0.0101541638112027\\
127	0.0101542019072455\\
128	0.010154240639159\\
129	0.0101542800175551\\
130	0.0101543200532279\\
131	0.0101543607571572\\
132	0.0101544021405125\\
133	0.0101544442146567\\
134	0.0101544869911499\\
135	0.0101545304817537\\
136	0.0101545746984351\\
137	0.0101546196533714\\
138	0.0101546653589541\\
139	0.0101547118277938\\
140	0.0101547590727248\\
141	0.0101548071068101\\
142	0.0101548559433467\\
143	0.0101549055958702\\
144	0.0101549560781609\\
145	0.010155007404249\\
146	0.0101550595884204\\
147	0.0101551126452226\\
148	0.0101551665894708\\
149	0.0101552214362547\\
150	0.0101552772009441\\
151	0.0101553338991966\\
152	0.0101553915469637\\
153	0.0101554501604984\\
154	0.0101555097563625\\
155	0.010155570351434\\
156	0.0101556319629147\\
157	0.0101556946083386\\
158	0.0101557583055793\\
159	0.0101558230728591\\
160	0.0101558889287568\\
161	0.0101559558922166\\
162	0.0101560239825568\\
163	0.0101560932194787\\
164	0.0101561636230754\\
165	0.010156235213841\\
166	0.0101563080126793\\
167	0.0101563820409131\\
168	0.0101564573202925\\
169	0.0101565338730045\\
170	0.0101566117216807\\
171	0.010156690889406\\
172	0.0101567713997261\\
173	0.0101568532766555\\
174	0.0101569365446836\\
175	0.0101570212287818\\
176	0.0101571073544078\\
177	0.0101571949475112\\
178	0.0101572840345363\\
179	0.0101573746424247\\
180	0.0101574667986162\\
181	0.0101575605310488\\
182	0.0101576558681566\\
183	0.0101577528388666\\
184	0.0101578514725936\\
185	0.0101579517992339\\
186	0.0101580538491573\\
187	0.0101581576531977\\
188	0.0101582632426433\\
189	0.0101583706492257\\
190	0.0101584799051101\\
191	0.0101585910428867\\
192	0.010158704095565\\
193	0.0101588190965734\\
194	0.010158936079768\\
195	0.0101590550794517\\
196	0.0101591761304091\\
197	0.0101592992679512\\
198	0.0101594245279956\\
199	0.0101595519470797\\
200	0.010159681562372\\
201	0.0101598134116827\\
202	0.0101599475334757\\
203	0.0101600839668792\\
204	0.0101602227516983\\
205	0.0101603639284261\\
206	0.0101605075382566\\
207	0.0101606536230964\\
208	0.010160802225578\\
209	0.010160953389072\\
210	0.0101611071577005\\
211	0.0101612635763504\\
212	0.0101614226906869\\
213	0.0101615845471674\\
214	0.0101617491930558\\
215	0.0101619166764363\\
216	0.0101620870462286\\
217	0.0101622603522023\\
218	0.0101624366449928\\
219	0.010162615976116\\
220	0.0101627983979846\\
221	0.0101629839639239\\
222	0.0101631727281883\\
223	0.0101633647459782\\
224	0.0101635600734565\\
225	0.0101637587677665\\
226	0.0101639608870494\\
227	0.010164166490462\\
228	0.0101643756381957\\
229	0.0101645883914948\\
230	0.0101648048126756\\
231	0.0101650249651461\\
232	0.0101652489134259\\
233	0.0101654767231662\\
234	0.0101657084611705\\
235	0.010165944195416\\
236	0.0101661839950746\\
237	0.0101664279305354\\
238	0.0101666760734264\\
239	0.010166928496638\\
240	0.0101671852743458\\
241	0.0101674464820346\\
242	0.0101677121965224\\
243	0.0101679824959856\\
244	0.0101682574599839\\
245	0.0101685371694862\\
246	0.010168821706897\\
247	0.010169111156083\\
248	0.0101694056024013\\
249	0.0101697051327267\\
250	0.0101700098354806\\
251	0.0101703198006606\\
252	0.0101706351198701\\
253	0.0101709558863491\\
254	0.0101712821950053\\
255	0.0101716141424465\\
256	0.0101719518270126\\
257	0.0101722953488099\\
258	0.0101726448097448\\
259	0.0101730003135588\\
260	0.0101733619658647\\
261	0.0101737298741831\\
262	0.0101741041479797\\
263	0.0101744848987043\\
264	0.0101748722398296\\
265	0.0101752662868918\\
266	0.0101756671575322\\
267	0.0101760749715387\\
268	0.0101764898508899\\
269	0.0101769119197994\\
270	0.010177341304761\\
271	0.0101777781345957\\
272	0.0101782225404995\\
273	0.0101786746560924\\
274	0.0101791346174689\\
275	0.0101796025632494\\
276	0.0101800786346335\\
277	0.0101805629754543\\
278	0.010181055732234\\
279	0.0101815570542416\\
280	0.0101820670935513\\
281	0.0101825860051034\\
282	0.010183113946766\\
283	0.0101836510793991\\
284	0.0101841975669197\\
285	0.0101847535763699\\
286	0.0101853192779853\\
287	0.010185894845267\\
288	0.0101864804550541\\
289	0.0101870762875997\\
290	0.0101876825266477\\
291	0.0101882993595132\\
292	0.010188926977164\\
293	0.0101895655743053\\
294	0.0101902153494668\\
295	0.0101908765050917\\
296	0.0101915492476294\\
297	0.01019223378763\\
298	0.0101929303398423\\
299	0.0101936391233147\\
300	0.0101943603614986\\
301	0.0101950942823555\\
302	0.0101958411184676\\
303	0.0101966011071506\\
304	0.0101973744905715\\
305	0.0101981615158692\\
306	0.0101989624352786\\
307	0.0101997775062591\\
308	0.0102006069916272\\
309	0.0102014511596922\\
310	0.0102023102843976\\
311	0.0102031846454653\\
312	0.010204074528546\\
313	0.0102049802253725\\
314	0.0102059020339192\\
315	0.0102068402585656\\
316	0.0102077952102652\\
317	0.0102087672067198\\
318	0.0102097565725584\\
319	0.0102107636395223\\
320	0.0102117887466554\\
321	0.0102128322405001\\
322	0.0102138944752991\\
323	0.010214975813203\\
324	0.0102160766244844\\
325	0.0102171972877572\\
326	0.0102183381902027\\
327	0.0102194997278019\\
328	0.0102206823055736\\
329	0.0102218863378193\\
330	0.0102231122483739\\
331	0.0102243604708631\\
332	0.0102256314489667\\
333	0.010226925636688\\
334	0.010228243498629\\
335	0.0102295855102721\\
336	0.0102309521582663\\
337	0.0102323439407195\\
338	0.0102337613674954\\
339	0.0102352049605147\\
340	0.0102366752540611\\
341	0.0102381727950898\\
342	0.0102396981435393\\
343	0.0102412518726456\\
344	0.0102428345692564\\
345	0.0102444468341469\\
346	0.0102460892823331\\
347	0.0102477625433841\\
348	0.0102494672617296\\
349	0.0102512040969626\\
350	0.0102529737241346\\
351	0.0102547768340413\\
352	0.0102566141334967\\
353	0.0102584863455935\\
354	0.0102603942099457\\
355	0.010262338482912\\
356	0.0102643199377946\\
357	0.0102663393650119\\
358	0.0102683975722391\\
359	0.0102704953845133\\
360	0.0102726336442982\\
361	0.0102748132115023\\
362	0.0102770349634467\\
363	0.0102792997947745\\
364	0.0102816086172966\\
365	0.0102839623597673\\
366	0.0102863619675811\\
367	0.0102888084023842\\
368	0.0102913026415931\\
369	0.0102938456778111\\
370	0.0102964385181349\\
371	0.0102990821833433\\
372	0.010301777706959\\
373	0.0103045261341761\\
374	0.0103073285206452\\
375	0.0103101859311094\\
376	0.0103130994378854\\
377	0.0103160701191857\\
378	0.0103190990572785\\
379	0.0103221873364865\\
380	0.0103253360410274\\
381	0.010328546252704\\
382	0.0103318190484577\\
383	0.0103351554978056\\
384	0.0103385566601879\\
385	0.0103420235822479\\
386	0.0103455572950313\\
387	0.0103491588109932\\
388	0.0103528291207502\\
389	0.0103565691918346\\
390	0.0103603799616753\\
391	0.01036426233328\\
392	0.0103682171720335\\
393	0.0103722453036614\\
394	0.0103763475141626\\
395	0.0103805245451411\\
396	0.010384777107279\\
397	0.0103891059013596\\
398	0.0103935116344279\\
399	0.0103979950505131\\
400	0.0104025570011102\\
401	0.0104071986370694\\
402	0.0104119219939701\\
403	0.0104167318974506\\
404	0.010421642426985\\
405	0.0104266990131087\\
406	0.0104319519551126\\
407	0.0104373589661464\\
408	0.0104428544564227\\
409	0.0104484395925339\\
410	0.0104541154705981\\
411	0.0104598829986168\\
412	0.0104657425736062\\
413	0.0104716931882777\\
414	0.0104777299442254\\
415	0.0104838370863948\\
416	0.0104899683536016\\
417	0.0104959910722292\\
418	0.0105015254344689\\
419	0.0105070757755747\\
420	0.0105127194406027\\
421	0.0105184577594436\\
422	0.0105242920680508\\
423	0.0105302237102971\\
424	0.0105362540416916\\
425	0.010542384434842\\
426	0.0105486162778084\\
427	0.0105549509077097\\
428	0.0105613897452537\\
429	0.0105679343326811\\
430	0.0105745862255363\\
431	0.0105813470222317\\
432	0.0105882186822336\\
433	0.0105952028197434\\
434	0.010602300496586\\
435	0.0106095127192765\\
436	0.0106168404245531\\
437	0.0106242844665697\\
438	0.0106318456491375\\
439	0.0106395249496099\\
440	0.0106473331375697\\
441	0.0106552780179475\\
442	0.0106633601250206\\
443	0.0106715796125061\\
444	0.0106799361833479\\
445	0.0106884290089753\\
446	0.0106970566366745\\
447	0.0107058168836664\\
448	0.0107147067166302\\
449	0.0107237221161687\\
450	0.0107328579283542\\
451	0.010742107713919\\
452	0.0107514636319723\\
453	0.0107609164768863\\
454	0.0107704562408228\\
455	0.0107800743617254\\
456	0.0107897434103097\\
457	0.0107994366960155\\
458	0.0108091371184326\\
459	0.0108188193609434\\
460	0.0108284712915606\\
461	0.0108381194117552\\
462	0.0108470901632276\\
463	0.0108562188080657\\
464	0.0108655109294113\\
465	0.0108749720161409\\
466	0.0108846080636622\\
467	0.0108944256416825\\
468	0.0109044319719155\\
469	0.0109146350177565\\
470	0.0109250435902589\\
471	0.0109356674842239\\
472	0.0109465176097905\\
473	0.0109576061228972\\
474	0.0109689465968613\\
475	0.0109805542163584\\
476	0.0109924459928782\\
477	0.0110046409896906\\
478	0.0110171605011407\\
479	0.0110300285903488\\
480	0.0110432727475884\\
481	0.0110569251361986\\
482	0.0110710179041852\\
483	0.0110855901270968\\
484	0.0111006896577753\\
485	0.011116379162969\\
486	0.0111333025782048\\
487	0.0111543331995493\\
488	0.0111756113126896\\
489	0.0111971332039886\\
490	0.0112188944796201\\
491	0.0112408899887983\\
492	0.0112631137362845\\
493	0.0112855587824233\\
494	0.0113082171286406\\
495	0.0113310795858977\\
496	0.0113541356229181\\
497	0.0113773731898385\\
498	0.0114007785106466\\
499	0.0114243358353171\\
500	0.0114480271382083\\
501	0.0114718317610983\\
502	0.0114957260380835\\
503	0.0115196832677241\\
504	0.0115436897958312\\
505	0.0115677292082954\\
506	0.0115917449629598\\
507	0.0116156431775283\\
508	0.0116350959200895\\
509	0.0116540948596634\\
510	0.011673201803359\\
511	0.0116924004703743\\
512	0.0117116729905464\\
513	0.0117309999482889\\
514	0.0117503605951538\\
515	0.0117697344773078\\
516	0.0117890985566605\\
517	0.0118083999697607\\
518	0.0118275814188709\\
519	0.0118466542777181\\
520	0.0118656432483732\\
521	0.011884520478095\\
522	0.0119032551908507\\
523	0.0119218105428036\\
524	0.0119399423834209\\
525	0.0119567734017832\\
526	0.0119740333048423\\
527	0.0119922756848448\\
528	0.0120106395997655\\
529	0.012029150780514\\
530	0.0120478109148558\\
531	0.0120666047753112\\
532	0.012085513580832\\
533	0.0121045182717183\\
534	0.0121236101728657\\
535	0.0121432193639268\\
536	0.0121633701205425\\
537	0.0121841056838724\\
538	0.0122046113986208\\
539	0.0122256980303088\\
540	0.0122475877393751\\
541	0.0122703087015464\\
542	0.0122938917342383\\
543	0.0123183693234578\\
544	0.012343742595612\\
545	0.012370073595347\\
546	0.0123974679814966\\
547	0.0124260978139957\\
548	0.012457513495557\\
549	0.012504025781355\\
550	0.0125503649777074\\
551	0.0125961887607179\\
552	0.0126417282556439\\
553	0.0126844274225608\\
554	0.0127153954857073\\
555	0.0127455569832322\\
556	0.0127747731226373\\
557	0.0128032121345103\\
558	0.0128316786093553\\
559	0.012856621498466\\
560	0.0128812655709474\\
561	0.0129056057316157\\
562	0.0129292961273491\\
563	0.012951515087047\\
564	0.0129750342727103\\
565	0.0129985209800742\\
566	0.0130216904241917\\
567	0.0130441542759827\\
568	0.0130651350621362\\
569	0.0130861137363929\\
570	0.0131070844581785\\
571	0.0131282247529408\\
572	0.0131506116875849\\
573	0.0131725321131203\\
574	0.0131930925593435\\
575	0.0132127992350075\\
576	0.0132323268920917\\
577	0.0132514318291406\\
578	0.0132706075505943\\
579	0.0132903298291615\\
580	0.0133092474054333\\
581	0.0133272671056936\\
582	0.013344966345282\\
583	0.013362355964056\\
584	0.0133795691997061\\
585	0.0133965857690501\\
586	0.0134133695948726\\
587	0.0134298798679046\\
588	0.0134460721451588\\
589	0.013461900474034\\
590	0.0134776599524643\\
591	0.0134935706714433\\
592	0.0135096572289508\\
593	0.0135260152836207\\
594	0.0135429334162338\\
595	0.0135612173030331\\
596	0.0135830518096797\\
597	0.0136142933738059\\
598	0.0136705746214114\\
599	0\\
600	0\\
};
\addplot [color=blue!75!mycolor7,solid,forget plot]
  table[row sep=crcr]{%
1	0.00922176541936403\\
2	0.00922177510630055\\
3	0.00922178495619226\\
4	0.00922179497177658\\
5	0.00922180515583688\\
6	0.0092218155112033\\
7	0.00922182604075349\\
8	0.00922183674741344\\
9	0.0092218476341583\\
10	0.00922185870401319\\
11	0.00922186996005403\\
12	0.00922188140540845\\
13	0.0092218930432566\\
14	0.0092219048768321\\
15	0.00922191690942286\\
16	0.00922192914437207\\
17	0.0092219415850791\\
18	0.00922195423500044\\
19	0.0092219670976507\\
20	0.00922198017660355\\
21	0.00922199347549273\\
22	0.00922200699801311\\
23	0.00922202074792166\\
24	0.00922203472903851\\
25	0.00922204894524805\\
26	0.0092220634005\\
27	0.00922207809881052\\
28	0.00922209304426331\\
29	0.00922210824101081\\
30	0.00922212369327528\\
31	0.00922213940535008\\
32	0.00922215538160077\\
33	0.00922217162646647\\
34	0.00922218814446094\\
35	0.00922220494017399\\
36	0.00922222201827269\\
37	0.00922223938350272\\
38	0.00922225704068968\\
39	0.00922227499474047\\
40	0.00922229325064465\\
41	0.00922231181347586\\
42	0.00922233068839326\\
43	0.00922234988064296\\
44	0.00922236939555952\\
45	0.00922238923856743\\
46	0.00922240941518271\\
47	0.00922242993101437\\
48	0.00922245079176609\\
49	0.00922247200323779\\
50	0.00922249357132726\\
51	0.00922251550203189\\
52	0.0092225378014503\\
53	0.00922256047578412\\
54	0.00922258353133975\\
55	0.00922260697453014\\
56	0.00922263081187662\\
57	0.00922265505001077\\
58	0.0092226796956763\\
59	0.00922270475573098\\
60	0.00922273023714857\\
61	0.00922275614702089\\
62	0.00922278249255978\\
63	0.0092228092810992\\
64	0.00922283652009732\\
65	0.00922286421713867\\
66	0.00922289237993634\\
67	0.00922292101633414\\
68	0.00922295013430894\\
69	0.00922297974197292\\
70	0.00922300984757595\\
71	0.0092230404595079\\
72	0.0092230715863012\\
73	0.0092231032366332\\
74	0.00922313541932871\\
75	0.00922316814336263\\
76	0.0092232014178625\\
77	0.0092232352521112\\
78	0.00922326965554965\\
79	0.00922330463777953\\
80	0.00922334020856617\\
81	0.00922337637784137\\
82	0.00922341315570634\\
83	0.00922345055243465\\
84	0.00922348857847528\\
85	0.0092235272444557\\
86	0.00922356656118503\\
87	0.00922360653965723\\
88	0.00922364719105439\\
89	0.00922368852675001\\
90	0.00922373055831247\\
91	0.00922377329750844\\
92	0.00922381675630638\\
93	0.00922386094688022\\
94	0.00922390588161292\\
95	0.00922395157310031\\
96	0.00922399803415481\\
97	0.00922404527780935\\
98	0.00922409331732136\\
99	0.00922414216617673\\
100	0.00922419183809403\\
101	0.00922424234702859\\
102	0.00922429370717689\\
103	0.00922434593298087\\
104	0.0092243990391324\\
105	0.00922445304057782\\
106	0.00922450795252258\\
107	0.00922456379043599\\
108	0.009224620570056\\
109	0.00922467830739418\\
110	0.00922473701874073\\
111	0.00922479672066958\\
112	0.00922485743004365\\
113	0.00922491916402019\\
114	0.00922498194005624\\
115	0.00922504577591418\\
116	0.0092251106896674\\
117	0.00922517669970611\\
118	0.00922524382474331\\
119	0.00922531208382073\\
120	0.00922538149631509\\
121	0.00922545208194433\\
122	0.00922552386077408\\
123	0.00922559685322421\\
124	0.00922567108007552\\
125	0.00922574656247661\\
126	0.00922582332195081\\
127	0.00922590138040333\\
128	0.00922598076012857\\
129	0.00922606148381745\\
130	0.00922614357456509\\
131	0.00922622705587848\\
132	0.00922631195168438\\
133	0.00922639828633737\\
134	0.00922648608462803\\
135	0.00922657537179138\\
136	0.00922666617351533\\
137	0.00922675851594947\\
138	0.00922685242571384\\
139	0.00922694792990807\\
140	0.00922704505612051\\
141	0.00922714383243762\\
142	0.00922724428745353\\
143	0.00922734645027974\\
144	0.009227450350555\\
145	0.00922755601845531\\
146	0.00922766348470417\\
147	0.00922777278058289\\
148	0.00922788393794112\\
149	0.00922799698920753\\
150	0.00922811196740055\\
151	0.00922822890613937\\
152	0.00922834783965497\\
153	0.00922846880280132\\
154	0.00922859183106666\\
155	0.00922871696058494\\
156	0.00922884422814724\\
157	0.0092289736712134\\
158	0.00922910532792358\\
159	0.00922923923711002\\
160	0.00922937543830867\\
161	0.00922951397177095\\
162	0.00922965487847546\\
163	0.00922979820013968\\
164	0.00922994397923166\\
165	0.00923009225898162\\
166	0.00923024308339349\\
167	0.00923039649725635\\
168	0.00923055254615586\\
169	0.00923071127648541\\
170	0.00923087273545727\\
171	0.00923103697111363\\
172	0.00923120403233739\\
173	0.00923137396886293\\
174	0.00923154683128675\\
175	0.00923172267107802\\
176	0.00923190154058909\\
177	0.00923208349306591\\
178	0.00923226858265869\\
179	0.00923245686443253\\
180	0.00923264839437841\\
181	0.00923284322942438\\
182	0.00923304142744748\\
183	0.00923324304728612\\
184	0.00923344814875334\\
185	0.00923365679265125\\
186	0.00923386904078663\\
187	0.00923408495598802\\
188	0.00923430460212466\\
189	0.00923452804412722\\
190	0.00923475534801082\\
191	0.00923498658090009\\
192	0.0092352218110565\\
193	0.00923546110790764\\
194	0.00923570454207776\\
195	0.00923595218541896\\
196	0.00923620411104124\\
197	0.00923646039334055\\
198	0.00923672110802232\\
199	0.00923698633212508\\
200	0.00923725614404467\\
201	0.00923753062355887\\
202	0.00923780985185254\\
203	0.00923809391154307\\
204	0.00923838288670658\\
205	0.00923867686290444\\
206	0.00923897592721031\\
207	0.00923928016823781\\
208	0.00923958967616859\\
209	0.00923990454278103\\
210	0.00924022486147947\\
211	0.009240550727324\\
212	0.00924088223706081\\
213	0.00924121948915319\\
214	0.00924156258381306\\
215	0.00924191162303316\\
216	0.00924226671061986\\
217	0.00924262795222659\\
218	0.00924299545538796\\
219	0.00924336932955448\\
220	0.00924374968612811\\
221	0.0092441366384983\\
222	0.00924453030207892\\
223	0.00924493079434582\\
224	0.00924533823487517\\
225	0.00924575274538257\\
226	0.00924617444976287\\
227	0.00924660347413091\\
228	0.00924703994686292\\
229	0.00924748399863883\\
230	0.00924793576248547\\
231	0.00924839537382055\\
232	0.00924886297049755\\
233	0.00924933869285152\\
234	0.00924982268374586\\
235	0.00925031508861991\\
236	0.00925081605553768\\
237	0.00925132573523742\\
238	0.00925184428118231\\
239	0.0092523718496121\\
240	0.00925290859959584\\
241	0.00925345469308579\\
242	0.0092540102949722\\
243	0.00925457557313952\\
244	0.00925515069852353\\
245	0.00925573584516979\\
246	0.00925633119029325\\
247	0.00925693691433913\\
248	0.00925755320104504\\
249	0.00925818023750445\\
250	0.00925881821423144\\
251	0.00925946732522686\\
252	0.00926012776804584\\
253	0.00926079974386677\\
254	0.00926148345756172\\
255	0.00926217911776837\\
256	0.00926288693696345\\
257	0.00926360713153776\\
258	0.00926433992187281\\
259	0.00926508553241909\\
260	0.00926584419177596\\
261	0.0092666161327734\\
262	0.00926740159255541\\
263	0.00926820081266518\\
264	0.00926901403913226\\
265	0.00926984152256144\\
266	0.00927068351822368\\
267	0.00927154028614894\\
268	0.00927241209122112\\
269	0.00927329920327497\\
270	0.00927420189719518\\
271	0.00927512045301763\\
272	0.00927605515603283\\
273	0.00927700629689167\\
274	0.00927797417171348\\
275	0.0092789590821964\\
276	0.00927996133573031\\
277	0.00928098124551214\\
278	0.00928201913066376\\
279	0.00928307531635248\\
280	0.00928415013391424\\
281	0.00928524392097943\\
282	0.0092863570216016\\
283	0.00928748978638893\\
284	0.00928864257263872\\
285	0.0092898157444747\\
286	0.00929100967298753\\
287	0.00929222473637829\\
288	0.00929346132010526\\
289	0.0092947198170338\\
290	0.00929600062758963\\
291	0.0092973041599155\\
292	0.0092986308300312\\
293	0.00929998106199721\\
294	0.00930135528808186\\
295	0.00930275394893213\\
296	0.00930417749374822\\
297	0.00930562638046187\\
298	0.00930710107591857\\
299	0.00930860205606364\\
300	0.00931012980613236\\
301	0.00931168482084413\\
302	0.00931326760460073\\
303	0.00931487867168877\\
304	0.00931651854648642\\
305	0.00931818776367434\\
306	0.00931988686845102\\
307	0.00932161641675252\\
308	0.00932337697547663\\
309	0.00932516912271149\\
310	0.00932699344796876\\
311	0.00932885055242129\\
312	0.00933074104914542\\
313	0.00933266556336777\\
314	0.00933462473271667\\
315	0.0093366192074782\\
316	0.0093386496508567\\
317	0.00934071673923992\\
318	0.00934282116246862\\
319	0.00934496362411067\\
320	0.00934714484173949\\
321	0.00934936554721683\\
322	0.00935162648697977\\
323	0.00935392842233177\\
324	0.00935627212973774\\
325	0.00935865840112288\\
326	0.00936108804417512\\
327	0.00936356188265101\\
328	0.00936608075668469\\
329	0.00936864552309989\\
330	0.00937125705572445\\
331	0.00937391624570719\\
332	0.00937662400183659\\
333	0.00937938125086114\\
334	0.00938218893781062\\
335	0.0093850480263181\\
336	0.00938795949894196\\
337	0.00939092435748741\\
338	0.00939394362332695\\
339	0.0093970183377191\\
340	0.00940014956212455\\
341	0.00940333837851931\\
342	0.00940658588970362\\
343	0.00940989321960621\\
344	0.00941326151358273\\
345	0.00941669193870744\\
346	0.00942018568405746\\
347	0.00942374396098814\\
348	0.00942736800339893\\
349	0.00943105906798835\\
350	0.00943481843449734\\
351	0.00943864740593952\\
352	0.00944254730881765\\
353	0.00944651949332511\\
354	0.0094505653335313\\
355	0.00945468622755038\\
356	0.00945888359769211\\
357	0.00946315889059441\\
358	0.00946751357733702\\
359	0.00947194915353585\\
360	0.00947646713941792\\
361	0.0094810690798772\\
362	0.00948575654451175\\
363	0.00949053112764309\\
364	0.00949539444831929\\
365	0.0095003481503034\\
366	0.00950539390205022\\
367	0.00951053339667405\\
368	0.00951576835191173\\
369	0.00952110051008562\\
370	0.00952653163807222\\
371	0.00953206352728304\\
372	0.00953769799366559\\
373	0.0095434368777332\\
374	0.00954928204463381\\
375	0.00955523538426882\\
376	0.0095612988114743\\
377	0.00956747426627823\\
378	0.00957376371424812\\
379	0.00958016914694436\\
380	0.00958669258249545\\
381	0.00959333606631133\\
382	0.0096001016719508\\
383	0.00960699150215806\\
384	0.00961400769008074\\
385	0.00962115240067878\\
386	0.00962842783233244\\
387	0.0096358362186675\\
388	0.00964337983063555\\
389	0.00965106097876177\\
390	0.00965888201595166\\
391	0.00966684534071529\\
392	0.00967495340078923\\
393	0.00968320869714935\\
394	0.00969161378837674\\
395	0.00970017129594287\\
396	0.0097088839104627\\
397	0.00971775440251562\\
398	0.00972678565142761\\
399	0.00973598073337622\\
400	0.0097453432040831\\
401	0.00975487802159254\\
402	0.00976459459042191\\
403	0.00977451690478917\\
404	0.00978471770413746\\
405	0.00979543480095023\\
406	0.00980672439485846\\
407	0.00981820954074321\\
408	0.00982989260509757\\
409	0.00984177568662119\\
410	0.0098538603668624\\
411	0.00986614707165071\\
412	0.00987863336605915\\
413	0.00989130930699013\\
414	0.00990414463258818\\
415	0.00991705317669946\\
416	0.00992979333490768\\
417	0.00994168760042892\\
418	0.00995082544966875\\
419	0.00995974583856609\\
420	0.00996883414839947\\
421	0.00997809255919277\\
422	0.00998752310365268\\
423	0.00999712763745113\\
424	0.0100069078051161\\
425	0.0100168650006486\\
426	0.0100270003218523\\
427	0.0100373145199697\\
428	0.0100478079363083\\
429	0.0100584804283815\\
430	0.0100693312904025\\
431	0.0100803591632609\\
432	0.0100915619185157\\
433	0.010102936541866\\
434	0.0101144789592561\\
435	0.0101261836357617\\
436	0.0101380424161917\\
437	0.0101500405088747\\
438	0.0101621416545168\\
439	0.0101742324546107\\
440	0.0101859630963881\\
441	0.0101975800194278\\
442	0.0102094052043279\\
443	0.0102214400103279\\
444	0.0102336855026604\\
445	0.0102461424131265\\
446	0.0102588110956913\\
447	0.0102716914761963\\
448	0.0102847829951067\\
449	0.0102980845420447\\
450	0.0103115943808136\\
451	0.0103253100637786\\
452	0.0103392283346019\\
453	0.0103533450155238\\
454	0.0103676548535381\\
455	0.0103821511842752\\
456	0.0103968262973786\\
457	0.0104116714638779\\
458	0.0104266774095858\\
459	0.0104418383289617\\
460	0.0104571644562596\\
461	0.0104726917585141\\
462	0.0104850268203094\\
463	0.0104975503262077\\
464	0.0105102382752083\\
465	0.0105231973151078\\
466	0.0105364346018183\\
467	0.0105499598829411\\
468	0.0105637916865503\\
469	0.0105779786155313\\
470	0.0105926733276726\\
471	0.0106080569705353\\
472	0.0106239653988795\\
473	0.0106401572453594\\
474	0.0106565915162975\\
475	0.0106731471339501\\
476	0.0106894771165114\\
477	0.0107045898247293\\
478	0.0107196779927161\\
479	0.0107350503103319\\
480	0.010750706278824\\
481	0.0107666487928968\\
482	0.0107828569292666\\
483	0.010799314577009\\
484	0.0108160061593955\\
485	0.0108329072316817\\
486	0.0108499714021265\\
487	0.010867192906201\\
488	0.0108846458556959\\
489	0.0109023293867002\\
490	0.0109202424722349\\
491	0.0109383839628969\\
492	0.0109567526429629\\
493	0.0109753473056063\\
494	0.0109941668516027\\
495	0.0110132104166574\\
496	0.0110324775330822\\
497	0.0110519683313517\\
498	0.0110716837843369\\
499	0.0110916259873147\\
500	0.0111117984413357\\
501	0.0111322062583276\\
502	0.0111528561949722\\
503	0.0111737479508317\\
504	0.0111942780392007\\
505	0.0112144362425563\\
506	0.0112353726727469\\
507	0.0112572058356874\\
508	0.0112851106963396\\
509	0.0113140649587009\\
510	0.0113433381985308\\
511	0.0113729185583065\\
512	0.0114027917312098\\
513	0.0114329410441506\\
514	0.0114633471294422\\
515	0.0114939874340903\\
516	0.0115248355679867\\
517	0.0115558614448134\\
518	0.0115870311418205\\
519	0.0116183046531186\\
520	0.0116496344260743\\
521	0.0116809651035511\\
522	0.0117122299533089\\
523	0.0117433385733353\\
524	0.0117731808271651\\
525	0.0117972249248136\\
526	0.0118212445200562\\
527	0.0118451872175872\\
528	0.0118689922018106\\
529	0.0118927593064552\\
530	0.0119164917716417\\
531	0.0119402071834083\\
532	0.0119638766427322\\
533	0.0119874708997969\\
534	0.0120109603929543\\
535	0.0120342938460484\\
536	0.0120574048543298\\
537	0.0120802021104621\\
538	0.0121033854437855\\
539	0.0121250445564356\\
540	0.0121472598920891\\
541	0.0121700424227883\\
542	0.012193403435077\\
543	0.0122173544468347\\
544	0.0122419095294658\\
545	0.0122671735220755\\
546	0.0122932150546455\\
547	0.0123200743279906\\
548	0.012347689445395\\
549	0.0123752240142216\\
550	0.0124036961956132\\
551	0.0124333052553468\\
552	0.0124642678860752\\
553	0.0124995625454783\\
554	0.0125487329135216\\
555	0.0125976654247791\\
556	0.012646296724314\\
557	0.0126947144300475\\
558	0.012741997672469\\
559	0.0127756216176028\\
560	0.0128084988655746\\
561	0.0128404345214355\\
562	0.0128716087008204\\
563	0.0129030991433651\\
564	0.0129315920026561\\
565	0.0129586184867447\\
566	0.0129852886048376\\
567	0.0130117367769548\\
568	0.0130374380016178\\
569	0.0130625401577994\\
570	0.0130872432876624\\
571	0.0131110245177824\\
572	0.0131337680617313\\
573	0.0131559960932691\\
574	0.0131791081047528\\
575	0.0132023497317156\\
576	0.0132247926415096\\
577	0.0132454573710228\\
578	0.0132655976738446\\
579	0.0132854517096758\\
580	0.0133051087401945\\
581	0.0133249129365691\\
582	0.0133438260377146\\
583	0.0133617745810026\\
584	0.0133793354722062\\
585	0.0133964510047924\\
586	0.013413292509456\\
587	0.0134298374631793\\
588	0.0134460506448502\\
589	0.0134618908837145\\
590	0.013477656335182\\
591	0.0134935696493031\\
592	0.0135096570639433\\
593	0.0135260152836207\\
594	0.0135429334162338\\
595	0.0135612173030331\\
596	0.0135830518096797\\
597	0.0136142933738059\\
598	0.0136705746214114\\
599	0\\
600	0\\
};
\addplot [color=blue!80!mycolor9,solid,forget plot]
  table[row sep=crcr]{%
1	0.0081165121904482\\
2	0.0081165209829513\\
3	0.00811652992332978\\
4	0.00811653901406917\\
5	0.00811654825769683\\
6	0.00811655765678267\\
7	0.00811656721393991\\
8	0.00811657693182576\\
9	0.0081165868131422\\
10	0.00811659686063674\\
11	0.00811660707710317\\
12	0.00811661746538236\\
13	0.00811662802836308\\
14	0.00811663876898278\\
15	0.00811664969022844\\
16	0.00811666079513742\\
17	0.00811667208679829\\
18	0.00811668356835174\\
19	0.00811669524299143\\
20	0.00811670711396491\\
21	0.00811671918457455\\
22	0.00811673145817844\\
23	0.00811674393819141\\
24	0.00811675662808592\\
25	0.00811676953139309\\
26	0.00811678265170372\\
27	0.00811679599266924\\
28	0.00811680955800282\\
29	0.00811682335148041\\
30	0.00811683737694177\\
31	0.00811685163829166\\
32	0.00811686613950082\\
33	0.00811688088460724\\
34	0.00811689587771723\\
35	0.00811691112300661\\
36	0.00811692662472193\\
37	0.00811694238718165\\
38	0.00811695841477742\\
39	0.00811697471197529\\
40	0.00811699128331704\\
41	0.00811700813342147\\
42	0.00811702526698574\\
43	0.00811704268878667\\
44	0.00811706040368218\\
45	0.00811707841661269\\
46	0.0081170967326025\\
47	0.00811711535676129\\
48	0.00811713429428558\\
49	0.00811715355046021\\
50	0.00811717313065997\\
51	0.00811719304035102\\
52	0.00811721328509261\\
53	0.00811723387053863\\
54	0.00811725480243926\\
55	0.00811727608664269\\
56	0.00811729772909676\\
57	0.0081173197358508\\
58	0.0081173421130573\\
59	0.0081173648669738\\
60	0.00811738800396468\\
61	0.00811741153050302\\
62	0.00811743545317257\\
63	0.00811745977866964\\
64	0.00811748451380507\\
65	0.00811750966550632\\
66	0.00811753524081943\\
67	0.00811756124691118\\
68	0.00811758769107118\\
69	0.00811761458071406\\
70	0.00811764192338166\\
71	0.00811766972674531\\
72	0.00811769799860811\\
73	0.00811772674690726\\
74	0.00811775597971646\\
75	0.00811778570524831\\
76	0.00811781593185679\\
77	0.00811784666803981\\
78	0.00811787792244172\\
79	0.00811790970385599\\
80	0.00811794202122778\\
81	0.00811797488365679\\
82	0.00811800830039986\\
83	0.00811804228087395\\
84	0.00811807683465891\\
85	0.00811811197150043\\
86	0.00811814770131305\\
87	0.00811818403418315\\
88	0.0081182209803721\\
89	0.00811825855031938\\
90	0.00811829675464582\\
91	0.00811833560415686\\
92	0.00811837510984593\\
93	0.00811841528289779\\
94	0.00811845613469206\\
95	0.00811849767680674\\
96	0.00811853992102179\\
97	0.00811858287932285\\
98	0.00811862656390496\\
99	0.00811867098717635\\
100	0.00811871616176239\\
101	0.00811876210050951\\
102	0.00811880881648924\\
103	0.00811885632300237\\
104	0.00811890463358307\\
105	0.00811895376200325\\
106	0.00811900372227683\\
107	0.00811905452866427\\
108	0.00811910619567702\\
109	0.00811915873808216\\
110	0.00811921217090709\\
111	0.00811926650944433\\
112	0.00811932176925635\\
113	0.00811937796618058\\
114	0.00811943511633444\\
115	0.00811949323612052\\
116	0.00811955234223177\\
117	0.00811961245165688\\
118	0.00811967358168571\\
119	0.00811973574991484\\
120	0.00811979897425319\\
121	0.00811986327292775\\
122	0.00811992866448945\\
123	0.00811999516781905\\
124	0.00812006280213324\\
125	0.00812013158699077\\
126	0.0081202015422987\\
127	0.00812027268831876\\
128	0.00812034504567383\\
129	0.00812041863535454\\
130	0.00812049347872589\\
131	0.0081205695975341\\
132	0.00812064701391352\\
133	0.00812072575039357\\
134	0.00812080582990595\\
135	0.00812088727579181\\
136	0.00812097011180911\\
137	0.00812105436214005\\
138	0.00812114005139864\\
139	0.00812122720463836\\
140	0.00812131584735992\\
141	0.00812140600551913\\
142	0.00812149770553491\\
143	0.00812159097429732\\
144	0.00812168583917578\\
145	0.00812178232802736\\
146	0.00812188046920513\\
147	0.0081219802915667\\
148	0.00812208182448274\\
149	0.00812218509784572\\
150	0.00812229014207863\\
151	0.0081223969881439\\
152	0.00812250566755228\\
153	0.00812261621237198\\
154	0.00812272865523775\\
155	0.00812284302936012\\
156	0.00812295936853478\\
157	0.00812307770715191\\
158	0.00812319808020572\\
159	0.00812332052330409\\
160	0.00812344507267821\\
161	0.00812357176519242\\
162	0.00812370063835409\\
163	0.00812383173032368\\
164	0.00812396507992485\\
165	0.00812410072665475\\
166	0.00812423871069441\\
167	0.00812437907291934\\
168	0.00812452185491026\\
169	0.00812466709896402\\
170	0.0081248148481048\\
171	0.00812496514609536\\
172	0.00812511803744881\\
173	0.00812527356744041\\
174	0.0081254317821199\\
175	0.00812559272832401\\
176	0.00812575645368949\\
177	0.00812592300666644\\
178	0.00812609243653222\\
179	0.00812626479340577\\
180	0.00812644012826249\\
181	0.00812661849294973\\
182	0.00812679994020282\\
183	0.00812698452366175\\
184	0.00812717229788854\\
185	0.00812736331838516\\
186	0.0081275576416122\\
187	0.00812775532500808\\
188	0.00812795642700887\\
189	0.00812816100706864\\
190	0.00812836912568017\\
191	0.00812858084439607\\
192	0.00812879622585006\\
193	0.0081290153337783\\
194	0.00812923823304076\\
195	0.00812946498964239\\
196	0.00812969567075431\\
197	0.00812993034473499\\
198	0.00813016908115185\\
199	0.0081304119508033\\
200	0.00813065902574112\\
201	0.00813091037929333\\
202	0.0081311660860875\\
203	0.00813142622207443\\
204	0.0081316908645524\\
205	0.00813196009219178\\
206	0.00813223398506012\\
207	0.00813251262464784\\
208	0.00813279609389419\\
209	0.00813308447721397\\
210	0.00813337786052454\\
211	0.00813367633127348\\
212	0.00813397997846676\\
213	0.00813428889269737\\
214	0.00813460316617463\\
215	0.00813492289275397\\
216	0.00813524816796733\\
217	0.00813557908905412\\
218	0.0081359157549928\\
219	0.00813625826653308\\
220	0.00813660672622869\\
221	0.00813696123847083\\
222	0.00813732190952231\\
223	0.00813768884755227\\
224	0.00813806216267156\\
225	0.00813844196696896\\
226	0.00813882837454793\\
227	0.00813922150156417\\
228	0.00813962146626386\\
229	0.00814002838902276\\
230	0.0081404423923859\\
231	0.00814086360110819\\
232	0.00814129214219575\\
233	0.00814172814494806\\
234	0.00814217174100099\\
235	0.00814262306437056\\
236	0.00814308225149767\\
237	0.00814354944129367\\
238	0.00814402477518676\\
239	0.00814450839716939\\
240	0.00814500045384658\\
241	0.00814550109448509\\
242	0.00814601047106372\\
243	0.0081465287383244\\
244	0.0081470560538245\\
245	0.00814759257798996\\
246	0.0081481384741696\\
247	0.00814869390869043\\
248	0.00814925905091411\\
249	0.00814983407329441\\
250	0.0081504191514359\\
251	0.00815101446415373\\
252	0.0081516201935346\\
253	0.0081522365249989\\
254	0.00815286364736412\\
255	0.0081535017529094\\
256	0.00815415103744143\\
257	0.00815481170036157\\
258	0.00815548394473434\\
259	0.00815616797735711\\
260	0.00815686400883133\\
261	0.00815757225363494\\
262	0.0081582929301963\\
263	0.00815902626096952\\
264	0.00815977247251119\\
265	0.00816053179555865\\
266	0.00816130446510967\\
267	0.00816209072050374\\
268	0.0081628908055048\\
269	0.00816370496838566\\
270	0.0081645334620139\\
271	0.00816537654393942\\
272	0.00816623447648373\\
273	0.00816710752683067\\
274	0.0081679959671191\\
275	0.00816890007453707\\
276	0.00816982013141791\\
277	0.00817075642533793\\
278	0.00817170924921599\\
279	0.00817267890141484\\
280	0.00817366568584434\\
281	0.00817466991206645\\
282	0.00817569189540215\\
283	0.00817673195704025\\
284	0.0081777904241481\\
285	0.00817886762998421\\
286	0.00817996391401288\\
287	0.00818107962202075\\
288	0.00818221510623535\\
289	0.00818337072544566\\
290	0.0081845468451247\\
291	0.00818574383755414\\
292	0.00818696208195093\\
293	0.00818820196459604\\
294	0.0081894638789652\\
295	0.0081907482258618\\
296	0.00819205541355182\\
297	0.00819338585790086\\
298	0.00819473998251322\\
299	0.0081961182188732\\
300	0.00819752100648833\\
301	0.00819894879303487\\
302	0.0082004020345052\\
303	0.00820188119535754\\
304	0.00820338674866751\\
305	0.00820491917628193\\
306	0.00820647896897459\\
307	0.0082080666266041\\
308	0.00820968265827377\\
309	0.00821132758249347\\
310	0.0082130019273435\\
311	0.00821470623064046\\
312	0.008216441040105\\
313	0.00821820691353153\\
314	0.00822000441895981\\
315	0.0082218341348484\\
316	0.00822369665024994\\
317	0.00822559256498827\\
318	0.00822752248983718\\
319	0.00822948704670112\\
320	0.00823148686879736\\
321	0.00823352260084004\\
322	0.00823559489922572\\
323	0.00823770443222061\\
324	0.00823985188014938\\
325	0.00824203793558549\\
326	0.00824426330354317\\
327	0.00824652870167089\\
328	0.00824883486044632\\
329	0.00825118252337302\\
330	0.00825357244717849\\
331	0.00825600540201401\\
332	0.00825848217165602\\
333	0.00826100355370934\\
334	0.0082635703598121\\
335	0.0082661834158427\\
336	0.00826884356212877\\
337	0.00827155165365844\\
338	0.00827430856029407\\
339	0.00827711516698859\\
340	0.00827997237400502\\
341	0.00828288109713916\\
342	0.00828584226794615\\
343	0.00828885683397118\\
344	0.008291925758985\\
345	0.00829505002322474\\
346	0.00829823062364081\\
347	0.00830146857415064\\
348	0.00830476490590011\\
349	0.00830812066753378\\
350	0.00831153692547487\\
351	0.0083150147642164\\
352	0.00831855528662479\\
353	0.0083221596142574\\
354	0.00832582888769591\\
355	0.00832956426689705\\
356	0.00833336693156314\\
357	0.00833723808153432\\
358	0.00834117893720499\\
359	0.00834519073996716\\
360	0.00834927475268341\\
361	0.00835343226019253\\
362	0.008357664569851\\
363	0.00836197301211378\\
364	0.00836635894115799\\
365	0.0083708237355532\\
366	0.00837536879898222\\
367	0.00837999556101659\\
368	0.0083847054779506\\
369	0.00838950003369818\\
370	0.00839438074075668\\
371	0.00839934914124156\\
372	0.00840440680799587\\
373	0.00840955534577799\\
374	0.00841479639253082\\
375	0.00842013162073507\\
376	0.00842556273884871\\
377	0.00843109149283356\\
378	0.0084367196677694\\
379	0.00844244908955445\\
380	0.00844828162669001\\
381	0.00845421919214522\\
382	0.00846026374529627\\
383	0.00846641729393258\\
384	0.0084726818963201\\
385	0.0084790596633102\\
386	0.00848555276048066\\
387	0.00849216341029267\\
388	0.00849889389424329\\
389	0.00850574655499458\\
390	0.00851272379843273\\
391	0.00851982809561501\\
392	0.00852706198456085\\
393	0.00853442807184914\\
394	0.00854192903402625\\
395	0.00854956761893354\\
396	0.00855734664747615\\
397	0.00856526901733551\\
398	0.00857333771270361\\
399	0.00858155583096706\\
400	0.00858992665408947\\
401	0.00859845382885909\\
402	0.00860714177589527\\
403	0.00861599640864139\\
404	0.00862502533045057\\
405	0.00863423083139975\\
406	0.00864360919849129\\
407	0.00865316449920479\\
408	0.00866290095978554\\
409	0.00867282296335026\\
410	0.00868293502129544\\
411	0.00869324167453038\\
412	0.00870374722149663\\
413	0.00871445504627727\\
414	0.00872536611641857\\
415	0.00873647612976628\\
416	0.00874777218247386\\
417	0.00875923947307966\\
418	0.00877093593298967\\
419	0.00878287439067812\\
420	0.00879506014721533\\
421	0.00880749852824258\\
422	0.00882019485880757\\
423	0.00883315443211037\\
424	0.00884638247088236\\
425	0.00885988407987879\\
426	0.00887366418769777\\
427	0.00888772747564478\\
428	0.00890207829144161\\
429	0.00891672054489455\\
430	0.00893165758164668\\
431	0.00894689202985107\\
432	0.00896242560939518\\
433	0.00897825886361876\\
434	0.00899439065340819\\
435	0.00901081674514154\\
436	0.00902752475715883\\
437	0.00904447457134704\\
438	0.00906152162457224\\
439	0.00907811873321334\\
440	0.00909216464998948\\
441	0.00910478596108619\\
442	0.00911769449911386\\
443	0.00913090038330905\\
444	0.00914441460684081\\
445	0.00915824916513284\\
446	0.00917241720551568\\
447	0.00918693320210463\\
448	0.00920181316065282\\
449	0.00921707485925254\\
450	0.00923273813234726\\
451	0.00924882520808249\\
452	0.00926536111414272\\
453	0.00928237418009274\\
454	0.00929989670227323\\
455	0.00931796595744039\\
456	0.00933662603222151\\
457	0.00935593204257199\\
458	0.00937596161613087\\
459	0.00939684915558759\\
460	0.00941889345154167\\
461	0.00944290679040185\\
462	0.00947147760970131\\
463	0.00950013651964995\\
464	0.00952845519842749\\
465	0.00955607668922808\\
466	0.00958424534270779\\
467	0.0096129912240588\\
468	0.00964239370149766\\
469	0.00967269831215308\\
470	0.00970471609603819\\
471	0.00973826024318917\\
472	0.00977237046428122\\
473	0.00980696880938624\\
474	0.00984183082307433\\
475	0.0098763243549342\\
476	0.00990867316575703\\
477	0.00993386348751344\\
478	0.00995796957564751\\
479	0.00998257010911525\\
480	0.0100076730167105\\
481	0.0100332859277783\\
482	0.0100594168402533\\
483	0.010086073894941\\
484	0.0101132651813247\\
485	0.0101409987550907\\
486	0.0101692833711498\\
487	0.0101981274852583\\
488	0.0102275362602819\\
489	0.0102575134551582\\
490	0.0102880611181513\\
491	0.0103191792243276\\
492	0.0103508652470469\\
493	0.0103831136515238\\
494	0.0104159152969657\\
495	0.010449256733512\\
496	0.0104831193845817\\
497	0.0105174786243152\\
498	0.0105523028230258\\
499	0.0105875526261451\\
500	0.0106231812972926\\
501	0.0106591385752409\\
502	0.0106953850198763\\
503	0.010731891621282\\
504	0.0107653796212684\\
505	0.0107939231262624\\
506	0.0108224570606551\\
507	0.0108513128367884\\
508	0.0108802292053025\\
509	0.0109088245746912\\
510	0.0109355777907873\\
511	0.0109613186383503\\
512	0.0109875239548034\\
513	0.0110142043290339\\
514	0.0110413715571839\\
515	0.011069038932675\\
516	0.0110972216109926\\
517	0.0111259370422628\\
518	0.0111552054993332\\
519	0.0111850508248709\\
520	0.0112155015664\\
521	0.011246593074209\\
522	0.0112783727589124\\
523	0.0113109162007254\\
524	0.0113455466053302\\
525	0.0113876301501583\\
526	0.0114297844239418\\
527	0.0114719261991653\\
528	0.0115139458834622\\
529	0.0115558231579011\\
530	0.0115965474110717\\
531	0.0116357045430165\\
532	0.0116750045571061\\
533	0.0117143887800835\\
534	0.0117537918345834\\
535	0.0117931449685202\\
536	0.0118323210527927\\
537	0.0118711540249626\\
538	0.0119085042694377\\
539	0.0119383183259512\\
540	0.0119681507194159\\
541	0.0119979399091869\\
542	0.0120276192023278\\
543	0.0120571167483817\\
544	0.0120863452087069\\
545	0.0121156589710805\\
546	0.0121452070216667\\
547	0.0121749217925998\\
548	0.0122048631449959\\
549	0.0122360612888342\\
550	0.0122673796090885\\
551	0.0122970750201469\\
552	0.0123269572677888\\
553	0.0123577396764707\\
554	0.0123894181594779\\
555	0.0124220769203335\\
556	0.0124558196782977\\
557	0.0124908220522919\\
558	0.0125282569506533\\
559	0.0125805693688947\\
560	0.0126325142867498\\
561	0.0126841940904036\\
562	0.0127357204439438\\
563	0.0127870054197745\\
564	0.0128287971685281\\
565	0.0128654200923061\\
566	0.0129010804831224\\
567	0.0129359985896982\\
568	0.0129713746365488\\
569	0.013004953897912\\
570	0.0130348028860955\\
571	0.0130648173985361\\
572	0.013095056860187\\
573	0.0131229341138437\\
574	0.0131498585303791\\
575	0.0131756188420202\\
576	0.0132001639874162\\
577	0.0132257687143828\\
578	0.0132506135884907\\
579	0.0132743534992253\\
580	0.0132958297758235\\
581	0.0133166490375325\\
582	0.0133373261130961\\
583	0.0133576999200791\\
584	0.013377012209116\\
585	0.0133950331218095\\
586	0.0134125207835145\\
587	0.0134293997995027\\
588	0.0134457992364619\\
589	0.0134617544855999\\
590	0.0134775895179212\\
591	0.0134935419150657\\
592	0.0135096483159889\\
593	0.0135260136751798\\
594	0.0135429334162338\\
595	0.0135612173030331\\
596	0.0135830518096797\\
597	0.0136142933738059\\
598	0.0136705746214114\\
599	0\\
600	0\\
};
\addplot [color=blue,solid,forget plot]
  table[row sep=crcr]{%
1	0.00665849622214942\\
2	0.00665850113433685\\
3	0.0066585061291329\\
4	0.00665851120792658\\
5	0.00665851637213032\\
6	0.00665852162318037\\
7	0.00665852696253721\\
8	0.00665853239168595\\
9	0.00665853791213677\\
10	0.00665854352542533\\
11	0.00665854923311318\\
12	0.00665855503678829\\
13	0.0066585609380654\\
14	0.00665856693858654\\
15	0.00665857304002145\\
16	0.00665857924406812\\
17	0.00665858555245319\\
18	0.00665859196693253\\
19	0.00665859848929166\\
20	0.00665860512134631\\
21	0.00665861186494292\\
22	0.00665861872195918\\
23	0.00665862569430452\\
24	0.00665863278392072\\
25	0.00665863999278245\\
26	0.00665864732289778\\
27	0.00665865477630882\\
28	0.00665866235509228\\
29	0.00665867006136007\\
30	0.00665867789725992\\
31	0.00665868586497596\\
32	0.00665869396672938\\
33	0.00665870220477906\\
34	0.00665871058142224\\
35	0.00665871909899516\\
36	0.00665872775987374\\
37	0.00665873656647427\\
38	0.00665874552125411\\
39	0.00665875462671241\\
40	0.00665876388539082\\
41	0.00665877329987424\\
42	0.00665878287279159\\
43	0.00665879260681651\\
44	0.00665880250466824\\
45	0.00665881256911231\\
46	0.00665882280296143\\
47	0.00665883320907628\\
48	0.00665884379036632\\
49	0.00665885454979072\\
50	0.00665886549035913\\
51	0.00665887661513266\\
52	0.00665888792722473\\
53	0.00665889942980198\\
54	0.00665891112608524\\
55	0.00665892301935046\\
56	0.0066589351129297\\
57	0.00665894741021209\\
58	0.00665895991464485\\
59	0.00665897262973433\\
60	0.00665898555904702\\
61	0.00665899870621068\\
62	0.00665901207491532\\
63	0.0066590256689144\\
64	0.00665903949202591\\
65	0.00665905354813353\\
66	0.00665906784118777\\
67	0.0066590823752072\\
68	0.00665909715427963\\
69	0.00665911218256337\\
70	0.00665912746428844\\
71	0.00665914300375792\\
72	0.0066591588053492\\
73	0.00665917487351534\\
74	0.00665919121278639\\
75	0.00665920782777083\\
76	0.00665922472315691\\
77	0.00665924190371413\\
78	0.00665925937429468\\
79	0.00665927713983493\\
80	0.00665929520535695\\
81	0.00665931357597005\\
82	0.00665933225687238\\
83	0.00665935125335247\\
84	0.00665937057079094\\
85	0.00665939021466211\\
86	0.00665941019053571\\
87	0.00665943050407865\\
88	0.00665945116105671\\
89	0.00665947216733639\\
90	0.00665949352888671\\
91	0.00665951525178108\\
92	0.0066595373421992\\
93	0.006659559806429\\
94	0.00665958265086857\\
95	0.00665960588202824\\
96	0.00665962950653255\\
97	0.00665965353112238\\
98	0.00665967796265703\\
99	0.00665970280811642\\
100	0.00665972807460327\\
101	0.00665975376934532\\
102	0.00665977989969765\\
103	0.00665980647314499\\
104	0.00665983349730405\\
105	0.00665986097992599\\
106	0.0066598889288988\\
107	0.00665991735224986\\
108	0.00665994625814843\\
109	0.00665997565490828\\
110	0.00666000555099027\\
111	0.00666003595500507\\
112	0.00666006687571586\\
113	0.00666009832204109\\
114	0.00666013030305734\\
115	0.00666016282800212\\
116	0.00666019590627686\\
117	0.00666022954744981\\
118	0.00666026376125908\\
119	0.00666029855761572\\
120	0.00666033394660677\\
121	0.0066603699384985\\
122	0.00666040654373956\\
123	0.00666044377296428\\
124	0.00666048163699598\\
125	0.0066605201468503\\
126	0.00666055931373867\\
127	0.00666059914907174\\
128	0.00666063966446292\\
129	0.00666068087173192\\
130	0.00666072278290844\\
131	0.00666076541023577\\
132	0.00666080876617454\\
133	0.00666085286340656\\
134	0.00666089771483855\\
135	0.00666094333360608\\
136	0.00666098973307751\\
137	0.00666103692685797\\
138	0.00666108492879338\\
139	0.00666113375297455\\
140	0.00666118341374135\\
141	0.00666123392568687\\
142	0.00666128530366166\\
143	0.0066613375627781\\
144	0.00666139071841468\\
145	0.00666144478622045\\
146	0.00666149978211946\\
147	0.00666155572231532\\
148	0.00666161262329571\\
149	0.0066616705018371\\
150	0.00666172937500936\\
151	0.0066617892601806\\
152	0.00666185017502188\\
153	0.00666191213751222\\
154	0.00666197516594347\\
155	0.00666203927892535\\
156	0.00666210449539056\\
157	0.00666217083459997\\
158	0.00666223831614785\\
159	0.00666230695996726\\
160	0.00666237678633548\\
161	0.00666244781587952\\
162	0.00666252006958187\\
163	0.00666259356878615\\
164	0.00666266833520307\\
165	0.00666274439091644\\
166	0.00666282175838932\\
167	0.00666290046047031\\
168	0.00666298052040002\\
169	0.00666306196181772\\
170	0.00666314480876811\\
171	0.00666322908570831\\
172	0.0066633148175151\\
173	0.00666340202949223\\
174	0.00666349074737809\\
175	0.0066635809973535\\
176	0.0066636728060498\\
177	0.00666376620055708\\
178	0.00666386120843277\\
179	0.00666395785771039\\
180	0.00666405617690852\\
181	0.00666415619504008\\
182	0.00666425794162179\\
183	0.00666436144668384\\
184	0.00666446674077982\\
185	0.00666457385499676\\
186	0.00666468282096548\\
187	0.00666479367087095\\
188	0.00666490643746289\\
189	0.00666502115406647\\
190	0.00666513785459315\\
191	0.00666525657355155\\
192	0.00666537734605853\\
193	0.00666550020785031\\
194	0.00666562519529374\\
195	0.00666575234539776\\
196	0.00666588169582502\\
197	0.00666601328490372\\
198	0.0066661471516397\\
199	0.00666628333572868\\
200	0.00666642187756887\\
201	0.00666656281827369\\
202	0.00666670619968474\\
203	0.00666685206438511\\
204	0.00666700045571283\\
205	0.00666715141777465\\
206	0.00666730499546001\\
207	0.00666746123445537\\
208	0.0066676201812587\\
209	0.00666778188319431\\
210	0.00666794638842798\\
211	0.00666811374598228\\
212	0.0066682840057523\\
213	0.00666845721852158\\
214	0.0066686334359784\\
215	0.00666881271073233\\
216	0.00666899509633112\\
217	0.00666918064727797\\
218	0.00666936941904897\\
219	0.00666956146811102\\
220	0.00666975685194002\\
221	0.00666995562903942\\
222	0.0066701578589591\\
223	0.00667036360231465\\
224	0.00667057292080694\\
225	0.00667078587724214\\
226	0.00667100253555211\\
227	0.00667122296081508\\
228	0.00667144721927682\\
229	0.00667167537837222\\
230	0.00667190750674715\\
231	0.00667214367428088\\
232	0.00667238395210882\\
233	0.00667262841264574\\
234	0.0066728771296094\\
235	0.00667313017804463\\
236	0.00667338763434788\\
237	0.00667364957629219\\
238	0.00667391608305264\\
239	0.00667418723523232\\
240	0.0066744631148887\\
241	0.00667474380556055\\
242	0.00667502939229537\\
243	0.00667531996167725\\
244	0.00667561560185533\\
245	0.00667591640257274\\
246	0.00667622245519608\\
247	0.00667653385274543\\
248	0.00667685068992492\\
249	0.00667717306315387\\
250	0.00667750107059846\\
251	0.006677834812204\\
252	0.00667817438972775\\
253	0.0066785199067724\\
254	0.00667887146882005\\
255	0.00667922918326686\\
256	0.00667959315945831\\
257	0.00667996350872506\\
258	0.00668034034441943\\
259	0.00668072378195259\\
260	0.00668111393883226\\
261	0.0066815109347012\\
262	0.00668191489137632\\
263	0.00668232593288834\\
264	0.00668274418552232\\
265	0.00668316977785873\\
266	0.00668360284081523\\
267	0.00668404350768921\\
268	0.00668449191420091\\
269	0.00668494819853736\\
270	0.00668541250139695\\
271	0.00668588496603479\\
272	0.00668636573830867\\
273	0.00668685496672588\\
274	0.00668735280249064\\
275	0.00668785939955237\\
276	0.00668837491465457\\
277	0.00668889950738452\\
278	0.00668943334022372\\
279	0.006689976578599\\
280	0.00669052939093442\\
281	0.00669109194870391\\
282	0.00669166442648461\\
283	0.00669224700201097\\
284	0.00669283985622963\\
285	0.0066934431733549\\
286	0.00669405714092509\\
287	0.00669468194985955\\
288	0.00669531779451635\\
289	0.00669596487275075\\
290	0.00669662338597434\\
291	0.0066972935392149\\
292	0.00669797554117692\\
293	0.00669866960430282\\
294	0.00669937594483489\\
295	0.00670009478287777\\
296	0.0067008263424617\\
297	0.00670157085160638\\
298	0.00670232854238541\\
299	0.00670309965099143\\
300	0.00670388441780176\\
301	0.00670468308744472\\
302	0.0067054959088665\\
303	0.00670632313539859\\
304	0.00670716502482569\\
305	0.00670802183945435\\
306	0.00670889384618187\\
307	0.00670978131656597\\
308	0.00671068452689479\\
309	0.00671160375825744\\
310	0.00671253929661506\\
311	0.0067134914328723\\
312	0.00671446046294938\\
313	0.00671544668785447\\
314	0.00671645041375672\\
315	0.00671747195205964\\
316	0.00671851161947508\\
317	0.00671956973809763\\
318	0.00672064663547964\\
319	0.00672174264470676\\
320	0.00672285810447408\\
321	0.00672399335916299\\
322	0.0067251487589186\\
323	0.00672632465972813\\
324	0.00672752142349997\\
325	0.00672873941814376\\
326	0.00672997901765157\\
327	0.00673124060218013\\
328	0.0067325245581345\\
329	0.00673383127825313\\
330	0.00673516116169457\\
331	0.00673651461412614\\
332	0.00673789204781458\\
333	0.00673929388171909\\
334	0.00674072054158706\\
335	0.00674217246005268\\
336	0.0067436500767389\\
337	0.00674515383836314\\
338	0.00674668419884707\\
339	0.00674824161943102\\
340	0.00674982656879358\\
341	0.00675143952317685\\
342	0.00675308096651796\\
343	0.00675475139058773\\
344	0.00675645129513686\\
345	0.00675818118805073\\
346	0.0067599415855135\\
347	0.00676173301218243\\
348	0.00676355600137348\\
349	0.00676541109525899\\
350	0.00676729884507889\\
351	0.00676921981136624\\
352	0.00677117456418858\\
353	0.00677316368340628\\
354	0.00677518775894926\\
355	0.00677724739111349\\
356	0.00677934319087883\\
357	0.00678147578024961\\
358	0.00678364579261961\\
359	0.00678585387316322\\
360	0.00678810067925414\\
361	0.00679038688091361\\
362	0.00679271316128984\\
363	0.00679508021717031\\
364	0.0067974887595289\\
365	0.00679993951410946\\
366	0.00680243322204776\\
367	0.00680497064053357\\
368	0.00680755254351461\\
369	0.00681017972244433\\
370	0.00681285298707516\\
371	0.00681557316629916\\
372	0.00681834110903801\\
373	0.00682115768518413\\
374	0.00682402378659517\\
375	0.00682694032814399\\
376	0.00682990824882669\\
377	0.00683292851293138\\
378	0.00683600211127108\\
379	0.00683913006248447\\
380	0.00684231341440902\\
381	0.00684555324553202\\
382	0.0068488506665262\\
383	0.00685220682187785\\
384	0.00685562289161735\\
385	0.00685910009316387\\
386	0.00686263968329851\\
387	0.0068662429602831\\
388	0.00686991126614492\\
389	0.00687364598915218\\
390	0.00687744856651073\\
391	0.00688132048732035\\
392	0.00688526329583872\\
393	0.00688927859511548\\
394	0.00689336805108251\\
395	0.00689753339722902\\
396	0.00690177644006987\\
397	0.00690609906577753\\
398	0.00691050324866354\\
399	0.00691499106269409\\
400	0.00691956469759967\\
401	0.00692422647966834\\
402	0.0069289788883072\\
403	0.0069338245323167\\
404	0.00693876602679916\\
405	0.00694380606938479\\
406	0.0069489477020684\\
407	0.00695419414055362\\
408	0.00695954878540159\\
409	0.00696501523163511\\
410	0.00697059727437339\\
411	0.00697629890610602\\
412	0.00698212429918936\\
413	0.0069880777695456\\
414	0.00699416374032204\\
415	0.00700038680644838\\
416	0.00700675217320869\\
417	0.0070132666123024\\
418	0.00701993616375867\\
419	0.0070267671474367\\
420	0.00703376639630928\\
421	0.00704094131947502\\
422	0.0070482999746153\\
423	0.00705585115151404\\
424	0.00706360446857164\\
425	0.00707157048462881\\
426	0.0070797608288982\\
427	0.00708818835245344\\
428	0.00709686730565987\\
429	0.00710581354759963\\
430	0.00711504479716378\\
431	0.00712458094469018\\
432	0.00713444446922644\\
433	0.00714466108658206\\
434	0.00715526100755638\\
435	0.00716628201787495\\
436	0.00717777838324267\\
437	0.00718984915182086\\
438	0.00720273285852934\\
439	0.00721713445505476\\
440	0.00723537922833009\\
441	0.00725607315897028\\
442	0.00727725241962505\\
443	0.00729893202542735\\
444	0.00732112752649359\\
445	0.00734385500640205\\
446	0.00736713107474147\\
447	0.00739097285231631\\
448	0.00741539794732106\\
449	0.00744042442050513\\
450	0.00746607073711496\\
451	0.00749235570345761\\
452	0.00751929838694749\\
453	0.00754691802223254\\
454	0.00757523391670673\\
455	0.00760426539565564\\
456	0.00763403189515005\\
457	0.00766455344327985\\
458	0.00769585194898445\\
459	0.00772795347472628\\
460	0.00776088815975503\\
461	0.00779466392027246\\
462	0.00782908482898321\\
463	0.00786373614512882\\
464	0.00789693336848542\\
465	0.00792244574097877\\
466	0.00794845148259389\\
467	0.00797496672597139\\
468	0.00800201278952988\\
469	0.00802961600514142\\
470	0.00805778032836273\\
471	0.00808650054645856\\
472	0.00811578866789925\\
473	0.00814565072770774\\
474	0.00817607922135795\\
475	0.00820704414099572\\
476	0.00823850998024071\\
477	0.0082706296751561\\
478	0.00830345600717334\\
479	0.0083370114466712\\
480	0.00837131972360764\\
481	0.00840640592812267\\
482	0.00844229657474578\\
483	0.00847901968494245\\
484	0.00851660489173914\\
485	0.00855508356182589\\
486	0.00859448888711043\\
487	0.00863485603336607\\
488	0.00867622246665473\\
489	0.00871862817916558\\
490	0.00876211594140733\\
491	0.00880673158398479\\
492	0.00885252431312617\\
493	0.0088995470664619\\
494	0.00894785692198749\\
495	0.00899751559156356\\
496	0.00904859008379621\\
497	0.00910115377721846\\
498	0.00915528860086066\\
499	0.00921109035243528\\
500	0.0092686830738449\\
501	0.00932825970775418\\
502	0.00939019888150805\\
503	0.00945540342962495\\
504	0.00952646573751785\\
505	0.00959839935473565\\
506	0.00966367308746111\\
507	0.00972970790853321\\
508	0.0097955311013139\\
509	0.00985837338788833\\
510	0.00991040864687504\\
511	0.00995592602006868\\
512	0.0100024931174222\\
513	0.0100501223406701\\
514	0.0100988230096538\\
515	0.0101486007112808\\
516	0.0101994565439792\\
517	0.0102513862177785\\
518	0.0103043788747779\\
519	0.010358415106622\\
520	0.0104134621060141\\
521	0.010469457815001\\
522	0.0105262521459318\\
523	0.0105833808074022\\
524	0.0106393515589638\\
525	0.0106932074588815\\
526	0.0107477862954605\\
527	0.0108030094449667\\
528	0.0108587580832945\\
529	0.0109156340675751\\
530	0.010969329471756\\
531	0.0110176713793229\\
532	0.0110692555168902\\
533	0.0111221588697866\\
534	0.0111733727061059\\
535	0.0112193060340414\\
536	0.0112659227110087\\
537	0.0113110686881058\\
538	0.0113558276637581\\
539	0.0114106298288132\\
540	0.0114658751159879\\
541	0.0115214899992542\\
542	0.0115773928948543\\
543	0.0116334873174785\\
544	0.0116896387386176\\
545	0.0117456820604746\\
546	0.0118014402687732\\
547	0.0118566887386902\\
548	0.0119110223679965\\
549	0.0119643346036823\\
550	0.0120134307766296\\
551	0.0120535342679049\\
552	0.0120912402234394\\
553	0.0121293920607468\\
554	0.0121679036058302\\
555	0.0122066756646262\\
556	0.012245811799337\\
557	0.0122852552494042\\
558	0.0123248881493484\\
559	0.0123659593523865\\
560	0.0124072965241059\\
561	0.0124488738878768\\
562	0.0124895577254884\\
563	0.0125299810993788\\
564	0.0125822630974104\\
565	0.0126399020762001\\
566	0.012697328174826\\
567	0.0127547133241035\\
568	0.0128115287896205\\
569	0.0128668965172786\\
570	0.0129082681532976\\
571	0.0129485801586111\\
572	0.0129878421888096\\
573	0.0130279149923698\\
574	0.0130668210080391\\
575	0.0131043817580682\\
576	0.0131389702761102\\
577	0.0131725361252934\\
578	0.0132039569120243\\
579	0.0132334216392856\\
580	0.0132621199093817\\
581	0.0132897806008646\\
582	0.0133162496629243\\
583	0.0133407217484723\\
584	0.0133633646912738\\
585	0.0133855289310785\\
586	0.0134064486197059\\
587	0.013425403755595\\
588	0.0134432920613732\\
589	0.0134603205714337\\
590	0.013476735350403\\
591	0.0134930854747703\\
592	0.0135094374760753\\
593	0.0135259382450398\\
594	0.0135429172056608\\
595	0.0135612173030331\\
596	0.0135830518096797\\
597	0.0136142933738059\\
598	0.0136705746214114\\
599	0\\
600	0\\
};
\addplot [color=mycolor10,solid,forget plot]
  table[row sep=crcr]{%
1	0.0060262081908944\\
2	0.00602620839363852\\
3	0.00602620859979209\\
4	0.00602620880941243\\
5	0.00602620902255784\\
6	0.00602620923928761\\
7	0.00602620945966199\\
8	0.00602620968374231\\
9	0.00602620991159087\\
10	0.00602621014327107\\
11	0.00602621037884736\\
12	0.00602621061838527\\
13	0.00602621086195144\\
14	0.00602621110961367\\
15	0.00602621136144084\\
16	0.00602621161750307\\
17	0.0060262118778716\\
18	0.00602621214261891\\
19	0.00602621241181869\\
20	0.00602621268554589\\
21	0.00602621296387672\\
22	0.00602621324688868\\
23	0.00602621353466058\\
24	0.00602621382727257\\
25	0.00602621412480614\\
26	0.00602621442734419\\
27	0.00602621473497101\\
28	0.00602621504777231\\
29	0.00602621536583526\\
30	0.00602621568924851\\
31	0.00602621601810221\\
32	0.00602621635248805\\
33	0.00602621669249926\\
34	0.00602621703823067\\
35	0.00602621738977871\\
36	0.00602621774724144\\
37	0.00602621811071861\\
38	0.00602621848031165\\
39	0.00602621885612372\\
40	0.00602621923825973\\
41	0.00602621962682638\\
42	0.00602622002193219\\
43	0.00602622042368753\\
44	0.00602622083220464\\
45	0.00602622124759767\\
46	0.00602622166998274\\
47	0.00602622209947792\\
48	0.00602622253620334\\
49	0.00602622298028113\\
50	0.00602622343183554\\
51	0.00602622389099292\\
52	0.00602622435788182\\
53	0.00602622483263295\\
54	0.00602622531537926\\
55	0.00602622580625601\\
56	0.00602622630540073\\
57	0.00602622681295334\\
58	0.00602622732905616\\
59	0.00602622785385393\\
60	0.00602622838749389\\
61	0.00602622893012581\\
62	0.00602622948190203\\
63	0.0060262300429775\\
64	0.00602623061350985\\
65	0.00602623119365941\\
66	0.0060262317835893\\
67	0.00602623238346541\\
68	0.00602623299345651\\
69	0.00602623361373429\\
70	0.00602623424447339\\
71	0.00602623488585147\\
72	0.00602623553804927\\
73	0.00602623620125063\\
74	0.00602623687564261\\
75	0.00602623756141547\\
76	0.00602623825876279\\
77	0.00602623896788151\\
78	0.00602623968897196\\
79	0.00602624042223797\\
80	0.00602624116788691\\
81	0.00602624192612974\\
82	0.0060262426971811\\
83	0.00602624348125937\\
84	0.00602624427858672\\
85	0.0060262450893892\\
86	0.0060262459138968\\
87	0.00602624675234353\\
88	0.00602624760496748\\
89	0.00602624847201089\\
90	0.00602624935372026\\
91	0.00602625025034639\\
92	0.00602625116214446\\
93	0.00602625208937414\\
94	0.00602625303229965\\
95	0.00602625399118983\\
96	0.00602625496631825\\
97	0.0060262559579633\\
98	0.00602625696640824\\
99	0.00602625799194133\\
100	0.00602625903485587\\
101	0.00602626009545038\\
102	0.00602626117402859\\
103	0.00602626227089962\\
104	0.00602626338637801\\
105	0.00602626452078389\\
106	0.00602626567444301\\
107	0.00602626684768691\\
108	0.00602626804085295\\
109	0.0060262692542845\\
110	0.00602627048833096\\
111	0.00602627174334797\\
112	0.00602627301969742\\
113	0.00602627431774764\\
114	0.00602627563787349\\
115	0.00602627698045645\\
116	0.0060262783458848\\
117	0.00602627973455367\\
118	0.00602628114686526\\
119	0.00602628258322883\\
120	0.00602628404406098\\
121	0.00602628552978567\\
122	0.00602628704083437\\
123	0.00602628857764627\\
124	0.00602629014066829\\
125	0.00602629173035533\\
126	0.00602629334717035\\
127	0.00602629499158454\\
128	0.00602629666407744\\
129	0.0060262983651371\\
130	0.00602630009526024\\
131	0.00602630185495237\\
132	0.00602630364472799\\
133	0.00602630546511067\\
134	0.0060263073166333\\
135	0.00602630919983816\\
136	0.00602631111527715\\
137	0.00602631306351192\\
138	0.00602631504511403\\
139	0.00602631706066512\\
140	0.00602631911075712\\
141	0.00602632119599236\\
142	0.00602632331698376\\
143	0.00602632547435505\\
144	0.00602632766874089\\
145	0.00602632990078708\\
146	0.00602633217115073\\
147	0.00602633448050048\\
148	0.00602633682951661\\
149	0.00602633921889132\\
150	0.00602634164932886\\
151	0.00602634412154573\\
152	0.00602634663627092\\
153	0.00602634919424604\\
154	0.00602635179622561\\
155	0.00602635444297717\\
156	0.00602635713528157\\
157	0.00602635987393312\\
158	0.00602636265973987\\
159	0.00602636549352376\\
160	0.0060263683761209\\
161	0.00602637130838178\\
162	0.00602637429117146\\
163	0.00602637732536991\\
164	0.00602638041187214\\
165	0.00602638355158851\\
166	0.00602638674544499\\
167	0.0060263899943834\\
168	0.00602639329936168\\
169	0.00602639666135416\\
170	0.00602640008135188\\
171	0.00602640356036284\\
172	0.0060264070994123\\
173	0.00602641069954312\\
174	0.00602641436181605\\
175	0.00602641808731006\\
176	0.0060264218771227\\
177	0.0060264257323704\\
178	0.00602642965418887\\
179	0.00602643364373343\\
180	0.0060264377021794\\
181	0.0060264418307225\\
182	0.00602644603057918\\
183	0.00602645030298709\\
184	0.00602645464920544\\
185	0.00602645907051544\\
186	0.00602646356822071\\
187	0.00602646814364768\\
188	0.00602647279814605\\
189	0.00602647753308922\\
190	0.00602648234987471\\
191	0.00602648724992462\\
192	0.00602649223468608\\
193	0.00602649730563167\\
194	0.00602650246425992\\
195	0.00602650771209576\\
196	0.00602651305069098\\
197	0.00602651848162475\\
198	0.00602652400650407\\
199	0.00602652962696432\\
200	0.00602653534466975\\
201	0.00602654116131398\\
202	0.00602654707862057\\
203	0.00602655309834353\\
204	0.00602655922226792\\
205	0.00602656545221034\\
206	0.00602657179001955\\
207	0.00602657823757705\\
208	0.00602658479679768\\
209	0.00602659146963018\\
210	0.00602659825805787\\
211	0.00602660516409922\\
212	0.00602661218980854\\
213	0.00602661933727658\\
214	0.00602662660863123\\
215	0.00602663400603821\\
216	0.00602664153170171\\
217	0.00602664918786511\\
218	0.00602665697681171\\
219	0.00602666490086545\\
220	0.00602667296239164\\
221	0.00602668116379773\\
222	0.00602668950753408\\
223	0.00602669799609471\\
224	0.00602670663201815\\
225	0.00602671541788821\\
226	0.00602672435633482\\
227	0.00602673345003489\\
228	0.00602674270171315\\
229	0.00602675211414303\\
230	0.00602676169014754\\
231	0.0060267714326002\\
232	0.00602678134442593\\
233	0.00602679142860204\\
234	0.00602680168815915\\
235	0.00602681212618216\\
236	0.00602682274581126\\
237	0.00602683355024297\\
238	0.0060268445427311\\
239	0.00602685572658787\\
240	0.00602686710518492\\
241	0.00602687868195444\\
242	0.00602689046039027\\
243	0.006026902444049\\
244	0.00602691463655112\\
245	0.00602692704158224\\
246	0.00602693966289421\\
247	0.00602695250430637\\
248	0.00602696556970677\\
249	0.00602697886305341\\
250	0.00602699238837555\\
251	0.00602700614977496\\
252	0.00602702015142728\\
253	0.00602703439758332\\
254	0.00602704889257048\\
255	0.00602706364079408\\
256	0.00602707864673883\\
257	0.0060270939149702\\
258	0.00602710945013595\\
259	0.00602712525696756\\
260	0.00602714134028177\\
261	0.00602715770498208\\
262	0.00602717435606038\\
263	0.00602719129859843\\
264	0.00602720853776959\\
265	0.00602722607884032\\
266	0.00602724392717198\\
267	0.0060272620882224\\
268	0.00602728056754769\\
269	0.00602729937080389\\
270	0.0060273185037488\\
271	0.00602733797224375\\
272	0.00602735778225541\\
273	0.00602737793985767\\
274	0.00602739845123345\\
275	0.00602741932267666\\
276	0.00602744056059409\\
277	0.00602746217150738\\
278	0.00602748416205497\\
279	0.00602750653899414\\
280	0.00602752930920301\\
281	0.0060275524796826\\
282	0.00602757605755895\\
283	0.00602760005008516\\
284	0.00602762446464357\\
285	0.00602764930874793\\
286	0.00602767459004552\\
287	0.00602770031631942\\
288	0.00602772649549072\\
289	0.00602775313562074\\
290	0.00602778024491336\\
291	0.00602780783171729\\
292	0.0060278359045284\\
293	0.00602786447199206\\
294	0.00602789354290549\\
295	0.00602792312622018\\
296	0.00602795323104424\\
297	0.00602798386664489\\
298	0.00602801504245083\\
299	0.00602804676805475\\
300	0.00602807905321579\\
301	0.006028111907862\\
302	0.0060281453420929\\
303	0.00602817936618195\\
304	0.00602821399057912\\
305	0.00602824922591342\\
306	0.00602828508299544\\
307	0.00602832157281995\\
308	0.00602835870656849\\
309	0.00602839649561191\\
310	0.00602843495151302\\
311	0.00602847408602918\\
312	0.00602851391111494\\
313	0.00602855443892461\\
314	0.00602859568181498\\
315	0.00602863765234792\\
316	0.00602868036329304\\
317	0.00602872382763037\\
318	0.00602876805855302\\
319	0.0060288130694699\\
320	0.00602885887400841\\
321	0.00602890548601717\\
322	0.00602895291956875\\
323	0.00602900118896245\\
324	0.00602905030872707\\
325	0.00602910029362375\\
326	0.00602915115864882\\
327	0.00602920291903669\\
328	0.00602925559026276\\
329	0.00602930918804647\\
330	0.00602936372835432\\
331	0.00602941922740298\\
332	0.00602947570166253\\
333	0.00602953316785975\\
334	0.00602959164298151\\
335	0.00602965114427829\\
336	0.00602971168926786\\
337	0.00602977329573907\\
338	0.00602983598175586\\
339	0.00602989976566139\\
340	0.00602996466608246\\
341	0.00603003070193412\\
342	0.0060300978924246\\
343	0.00603016625706043\\
344	0.00603023581565202\\
345	0.00603030658831949\\
346	0.00603037859549897\\
347	0.00603045185794925\\
348	0.00603052639675901\\
349	0.00603060223335448\\
350	0.00603067938950769\\
351	0.00603075788734535\\
352	0.00603083774935834\\
353	0.00603091899841199\\
354	0.00603100165775707\\
355	0.00603108575104161\\
356	0.00603117130232361\\
357	0.00603125833608473\\
358	0.00603134687724496\\
359	0.00603143695117832\\
360	0.00603152858372978\\
361	0.00603162180123334\\
362	0.00603171663053133\\
363	0.00603181309899515\\
364	0.00603191123454731\\
365	0.006032011065685\\
366	0.00603211262150514\\
367	0.00603221593173108\\
368	0.00603232102674091\\
369	0.0060324279375975\\
370	0.00603253669608041\\
371	0.00603264733471953\\
372	0.00603275988683074\\
373	0.00603287438655359\\
374	0.00603299086889096\\
375	0.00603310936975102\\
376	0.00603322992599142\\
377	0.00603335257546594\\
378	0.00603347735707363\\
379	0.00603360431081078\\
380	0.00603373347782586\\
381	0.00603386490047765\\
382	0.00603399862239702\\
383	0.00603413468855266\\
384	0.00603427314532129\\
385	0.00603441404056296\\
386	0.00603455742370213\\
387	0.00603470334581539\\
388	0.00603485185972694\\
389	0.00603500302011306\\
390	0.00603515688361736\\
391	0.00603531350897904\\
392	0.00603547295717703\\
393	0.00603563529159472\\
394	0.00603580057821183\\
395	0.006035968885835\\
396	0.00603614028638504\\
397	0.00603631485526851\\
398	0.00603649267186337\\
399	0.00603667382011449\\
400	0.00603685838909168\\
401	0.00603704647302444\\
402	0.0060372381700037\\
403	0.00603743357983671\\
404	0.00603763280780192\\
405	0.00603783597026249\\
406	0.00603804319018601\\
407	0.00603825459750199\\
408	0.00603847032935031\\
409	0.00603869053011631\\
410	0.00603891535112242\\
411	0.00603914494992886\\
412	0.00603937948963411\\
413	0.00603961913978187\\
414	0.00603986408248705\\
415	0.00604011452688038\\
416	0.00604037071844429\\
417	0.00604063288679885\\
418	0.00604090127474831\\
419	0.00604117614521222\\
420	0.0060414577837333\\
421	0.00604174650136423\\
422	0.00604204263799895\\
423	0.0060423465662284\\
424	0.00604265869581927\\
425	0.00604297947894383\\
426	0.00604330941634252\\
427	0.00604364906471344\\
428	0.00604399904589498\\
429	0.00604436005910483\\
430	0.00604473289934236\\
431	0.00604511848998202\\
432	0.00604551795064977\\
433	0.0060459327555038\\
434	0.0060463651224831\\
435	0.00604681897408983\\
436	0.00604730221186276\\
437	0.0060478315297225\\
438	0.00604843974775084\\
439	0.00604917167982035\\
440	0.00604997319065317\\
441	0.00605079186423843\\
442	0.00605162816771526\\
443	0.00605248258122923\\
444	0.00605335559729303\\
445	0.00605424771987512\\
446	0.00605515946316192\\
447	0.00605609134993318\\
448	0.0060570439094919\\
449	0.00605801767511521\\
450	0.00605901318108042\\
451	0.00606003095957424\\
452	0.00606107153846132\\
453	0.00606213544257997\\
454	0.00606322320543609\\
455	0.00606433540865586\\
456	0.00606547279269126\\
457	0.00606663655078591\\
458	0.00606782911307416\\
459	0.00606905634946132\\
460	0.00607033435113452\\
461	0.00607171268419969\\
462	0.00607336243346001\\
463	0.00607585835269156\\
464	0.00608117797779962\\
465	0.00609594182528355\\
466	0.00611098540167769\\
467	0.00612631745965242\\
468	0.00614194719582217\\
469	0.00615788377119556\\
470	0.00617413679078783\\
471	0.00619071673591015\\
472	0.00620763434969619\\
473	0.00622490050695596\\
474	0.00624252636059664\\
475	0.00626052452077361\\
476	0.00627891167846384\\
477	0.00629770247295594\\
478	0.00631691164122175\\
479	0.00633655487585932\\
480	0.00635664891024622\\
481	0.00637721161382601\\
482	0.00639826210190065\\
483	0.00641982086191638\\
484	0.00644190989831576\\
485	0.00646455289814938\\
486	0.00648777542332625\\
487	0.00651160513269779\\
488	0.00653607203053365\\
489	0.0065612087540536\\
490	0.00658705090785074\\
491	0.00661363745506758\\
492	0.00664101117812462\\
493	0.00666921922662537\\
494	0.00669831377935474\\
495	0.00672835286817136\\
496	0.00675940146414622\\
497	0.00679153306788021\\
498	0.00682483244214689\\
499	0.00685940125878525\\
500	0.00689537172734906\\
501	0.00693294298757802\\
502	0.00697248411689079\\
503	0.00701483593466218\\
504	0.00706220544010819\\
505	0.00712173865658605\\
506	0.00719157327987468\\
507	0.00726284911407143\\
508	0.00733555442072288\\
509	0.00740965963627356\\
510	0.0074853860819512\\
511	0.00756297456351322\\
512	0.00764252144057168\\
513	0.00772413236236931\\
514	0.00780792340672302\\
515	0.00789402236775618\\
516	0.00798257018116807\\
517	0.00807372234193788\\
518	0.00816764957844863\\
519	0.0082645346062017\\
520	0.00836455204343796\\
521	0.00846778046942138\\
522	0.0085738483798231\\
523	0.0086805512988779\\
524	0.00877851642951977\\
525	0.00886073097854157\\
526	0.00894591267348451\\
527	0.00903445663020803\\
528	0.00912717900691978\\
529	0.00922616115061996\\
530	0.00933795947270032\\
531	0.00946623449705119\\
532	0.00959860209578009\\
533	0.00972879537376874\\
534	0.00984818273103183\\
535	0.00993831111639377\\
536	0.0100305916175822\\
537	0.0101133474864796\\
538	0.0101824496736979\\
539	0.0102531854727262\\
540	0.0103255978535515\\
541	0.010399732004836\\
542	0.0104756435006355\\
543	0.0105533655181092\\
544	0.0106327870420982\\
545	0.0107138148427025\\
546	0.0107964278336208\\
547	0.0108811809951747\\
548	0.0109682689598058\\
549	0.0110571105521811\\
550	0.0111322777113094\\
551	0.01122096415446\\
552	0.0113152857402254\\
553	0.0114144917605989\\
554	0.011515548378353\\
555	0.0116051695313629\\
556	0.0116862928095314\\
557	0.0117659796145721\\
558	0.0118449978005538\\
559	0.0119229810477401\\
560	0.0119940897157269\\
561	0.0120623351265066\\
562	0.0121240557717538\\
563	0.0121793359699178\\
564	0.0122349664418353\\
565	0.0122907476562111\\
566	0.0123422673317996\\
567	0.0123947274613003\\
568	0.01244875718908\\
569	0.0125053391756342\\
570	0.0125779831679375\\
571	0.0126495344895391\\
572	0.0127199187496675\\
573	0.0127869654547992\\
574	0.0128521614820326\\
575	0.0129154078031304\\
576	0.0129647402996792\\
577	0.0130128345568331\\
578	0.0130591997830326\\
579	0.0131053439193897\\
580	0.0131511516774424\\
581	0.0131946629971335\\
582	0.0132339963244058\\
583	0.0132720337247357\\
584	0.0133071158931735\\
585	0.0133394974957768\\
586	0.013369936986104\\
587	0.0133975929139217\\
588	0.0134234536684543\\
589	0.0134472304868347\\
590	0.0134680458102776\\
591	0.0134876177197991\\
592	0.0135062927468828\\
593	0.0135243044590959\\
594	0.0135422416189903\\
595	0.0135610405805831\\
596	0.0135830518096797\\
597	0.0136142933738059\\
598	0.0136705746214114\\
599	0\\
600	0\\
};
\addplot [color=mycolor11,solid,forget plot]
  table[row sep=crcr]{%
1	0.00598969048168012\\
2	0.00598969049058498\\
3	0.00598969049963959\\
4	0.00598969050884647\\
5	0.00598969051820818\\
6	0.00598969052772731\\
7	0.00598969053740653\\
8	0.00598969054724852\\
9	0.00598969055725602\\
10	0.0059896905674318\\
11	0.00598969057777872\\
12	0.00598969058829963\\
13	0.00598969059899747\\
14	0.00598969060987523\\
15	0.00598969062093591\\
16	0.00598969063218262\\
17	0.00598969064361846\\
18	0.00598969065524663\\
19	0.00598969066707036\\
20	0.00598969067909296\\
21	0.00598969069131775\\
22	0.00598969070374816\\
23	0.00598969071638763\\
24	0.0059896907292397\\
25	0.00598969074230794\\
26	0.00598969075559599\\
27	0.00598969076910756\\
28	0.00598969078284641\\
29	0.00598969079681637\\
30	0.00598969081102134\\
31	0.00598969082546527\\
32	0.0059896908401522\\
33	0.00598969085508621\\
34	0.00598969087027148\\
35	0.00598969088571225\\
36	0.00598969090141281\\
37	0.00598969091737756\\
38	0.00598969093361094\\
39	0.0059896909501175\\
40	0.00598969096690184\\
41	0.00598969098396864\\
42	0.00598969100132268\\
43	0.00598969101896881\\
44	0.00598969103691195\\
45	0.00598969105515712\\
46	0.00598969107370943\\
47	0.00598969109257406\\
48	0.00598969111175629\\
49	0.00598969113126148\\
50	0.0059896911510951\\
51	0.0059896911712627\\
52	0.00598969119176993\\
53	0.00598969121262252\\
54	0.00598969123382632\\
55	0.00598969125538728\\
56	0.00598969127731144\\
57	0.00598969129960494\\
58	0.00598969132227404\\
59	0.00598969134532511\\
60	0.0059896913687646\\
61	0.00598969139259911\\
62	0.00598969141683532\\
63	0.00598969144148006\\
64	0.00598969146654023\\
65	0.0059896914920229\\
66	0.00598969151793522\\
67	0.0059896915442845\\
68	0.00598969157107814\\
69	0.00598969159832369\\
70	0.00598969162602883\\
71	0.00598969165420137\\
72	0.00598969168284925\\
73	0.00598969171198055\\
74	0.00598969174160349\\
75	0.00598969177172644\\
76	0.0059896918023579\\
77	0.00598969183350653\\
78	0.00598969186518113\\
79	0.00598969189739067\\
80	0.00598969193014426\\
81	0.00598969196345118\\
82	0.00598969199732085\\
83	0.00598969203176289\\
84	0.00598969206678705\\
85	0.00598969210240327\\
86	0.00598969213862166\\
87	0.00598969217545252\\
88	0.00598969221290631\\
89	0.00598969225099367\\
90	0.00598969228972545\\
91	0.00598969232911267\\
92	0.00598969236916656\\
93	0.00598969240989852\\
94	0.00598969245132018\\
95	0.00598969249344335\\
96	0.00598969253628006\\
97	0.00598969257984256\\
98	0.0059896926241433\\
99	0.00598969266919496\\
100	0.00598969271501043\\
101	0.00598969276160283\\
102	0.00598969280898553\\
103	0.00598969285717211\\
104	0.00598969290617641\\
105	0.00598969295601251\\
106	0.00598969300669473\\
107	0.00598969305823765\\
108	0.0059896931106561\\
109	0.00598969316396519\\
110	0.00598969321818029\\
111	0.00598969327331702\\
112	0.00598969332939132\\
113	0.00598969338641937\\
114	0.00598969344441766\\
115	0.00598969350340296\\
116	0.00598969356339237\\
117	0.00598969362440325\\
118	0.00598969368645328\\
119	0.00598969374956048\\
120	0.00598969381374316\\
121	0.00598969387901997\\
122	0.00598969394540988\\
123	0.00598969401293221\\
124	0.00598969408160662\\
125	0.00598969415145311\\
126	0.00598969422249204\\
127	0.00598969429474414\\
128	0.00598969436823049\\
129	0.00598969444297255\\
130	0.00598969451899219\\
131	0.00598969459631161\\
132	0.00598969467495346\\
133	0.00598969475494075\\
134	0.00598969483629692\\
135	0.00598969491904581\\
136	0.0059896950032117\\
137	0.00598969508881928\\
138	0.00598969517589369\\
139	0.0059896952644605\\
140	0.00598969535454574\\
141	0.00598969544617589\\
142	0.0059896955393779\\
143	0.00598969563417919\\
144	0.00598969573060767\\
145	0.00598969582869173\\
146	0.00598969592846025\\
147	0.00598969602994264\\
148	0.0059896961331688\\
149	0.00598969623816915\\
150	0.00598969634497465\\
151	0.00598969645361679\\
152	0.00598969656412762\\
153	0.00598969667653974\\
154	0.00598969679088629\\
155	0.00598969690720103\\
156	0.00598969702551825\\
157	0.00598969714587288\\
158	0.00598969726830042\\
159	0.00598969739283699\\
160	0.00598969751951933\\
161	0.00598969764838481\\
162	0.00598969777947145\\
163	0.0059896979128179\\
164	0.0059896980484635\\
165	0.00598969818644825\\
166	0.00598969832681284\\
167	0.00598969846959864\\
168	0.00598969861484774\\
169	0.00598969876260296\\
170	0.00598969891290783\\
171	0.00598969906580667\\
172	0.0059896992213445\\
173	0.00598969937956715\\
174	0.00598969954052125\\
175	0.0059896997042542\\
176	0.00598969987081423\\
177	0.00598970004025042\\
178	0.00598970021261266\\
179	0.00598970038795174\\
180	0.00598970056631932\\
181	0.00598970074776795\\
182	0.0059897009323511\\
183	0.00598970112012319\\
184	0.00598970131113956\\
185	0.00598970150545653\\
186	0.00598970170313142\\
187	0.00598970190422255\\
188	0.00598970210878924\\
189	0.0059897023168919\\
190	0.00598970252859195\\
191	0.00598970274395193\\
192	0.00598970296303545\\
193	0.00598970318590728\\
194	0.00598970341263329\\
195	0.00598970364328055\\
196	0.00598970387791726\\
197	0.00598970411661288\\
198	0.00598970435943806\\
199	0.00598970460646471\\
200	0.00598970485776601\\
201	0.00598970511341643\\
202	0.00598970537349175\\
203	0.00598970563806911\\
204	0.00598970590722699\\
205	0.00598970618104528\\
206	0.00598970645960527\\
207	0.0059897067429897\\
208	0.00598970703128277\\
209	0.00598970732457018\\
210	0.00598970762293914\\
211	0.00598970792647842\\
212	0.00598970823527834\\
213	0.00598970854943087\\
214	0.00598970886902956\\
215	0.00598970919416967\\
216	0.00598970952494813\\
217	0.00598970986146359\\
218	0.00598971020381647\\
219	0.00598971055210897\\
220	0.00598971090644512\\
221	0.00598971126693079\\
222	0.00598971163367374\\
223	0.00598971200678367\\
224	0.00598971238637221\\
225	0.005989712772553\\
226	0.00598971316544172\\
227	0.00598971356515609\\
228	0.00598971397181595\\
229	0.00598971438554328\\
230	0.00598971480646224\\
231	0.00598971523469922\\
232	0.00598971567038286\\
233	0.00598971611364411\\
234	0.00598971656461627\\
235	0.00598971702343503\\
236	0.00598971749023849\\
237	0.00598971796516726\\
238	0.00598971844836446\\
239	0.00598971893997576\\
240	0.00598971944014948\\
241	0.00598971994903657\\
242	0.00598972046679071\\
243	0.00598972099356835\\
244	0.00598972152952872\\
245	0.00598972207483395\\
246	0.00598972262964905\\
247	0.00598972319414204\\
248	0.00598972376848392\\
249	0.00598972435284879\\
250	0.00598972494741388\\
251	0.00598972555235963\\
252	0.00598972616786968\\
253	0.00598972679413104\\
254	0.00598972743133402\\
255	0.00598972807967241\\
256	0.00598972873934347\\
257	0.00598972941054801\\
258	0.00598973009349048\\
259	0.00598973078837898\\
260	0.00598973149542537\\
261	0.00598973221484534\\
262	0.00598973294685845\\
263	0.00598973369168821\\
264	0.00598973444956217\\
265	0.00598973522071196\\
266	0.00598973600537339\\
267	0.00598973680378651\\
268	0.00598973761619567\\
269	0.00598973844284964\\
270	0.00598973928400165\\
271	0.00598974013990946\\
272	0.0059897410108355\\
273	0.00598974189704686\\
274	0.00598974279881546\\
275	0.00598974371641807\\
276	0.00598974465013644\\
277	0.00598974560025733\\
278	0.00598974656707267\\
279	0.00598974755087957\\
280	0.00598974855198046\\
281	0.00598974957068318\\
282	0.00598975060730102\\
283	0.00598975166215288\\
284	0.00598975273556331\\
285	0.00598975382786263\\
286	0.00598975493938703\\
287	0.00598975607047862\\
288	0.00598975722148561\\
289	0.00598975839276231\\
290	0.00598975958466932\\
291	0.00598976079757356\\
292	0.0059897620318484\\
293	0.00598976328787378\\
294	0.00598976456603627\\
295	0.0059897658667292\\
296	0.00598976719035277\\
297	0.00598976853731413\\
298	0.0059897699080275\\
299	0.0059897713029143\\
300	0.00598977272240319\\
301	0.00598977416693026\\
302	0.00598977563693906\\
303	0.00598977713288078\\
304	0.0059897786552143\\
305	0.00598978020440634\\
306	0.00598978178093154\\
307	0.00598978338527261\\
308	0.00598978501792038\\
309	0.00598978667937398\\
310	0.00598978837014089\\
311	0.00598979009073711\\
312	0.00598979184168722\\
313	0.00598979362352453\\
314	0.00598979543679117\\
315	0.00598979728203823\\
316	0.00598979915982583\\
317	0.00598980107072329\\
318	0.00598980301530919\\
319	0.00598980499417155\\
320	0.00598980700790786\\
321	0.00598980905712528\\
322	0.00598981114244073\\
323	0.00598981326448099\\
324	0.00598981542388283\\
325	0.00598981762129316\\
326	0.00598981985736911\\
327	0.0059898221327782\\
328	0.00598982444819843\\
329	0.00598982680431844\\
330	0.00598982920183763\\
331	0.00598983164146632\\
332	0.00598983412392584\\
333	0.00598983664994874\\
334	0.00598983922027892\\
335	0.00598984183567177\\
336	0.00598984449689435\\
337	0.00598984720472559\\
338	0.00598984995995642\\
339	0.00598985276338999\\
340	0.0059898556158419\\
341	0.00598985851814036\\
342	0.00598986147112644\\
343	0.00598986447565435\\
344	0.00598986753259162\\
345	0.00598987064281947\\
346	0.00598987380723302\\
347	0.00598987702674165\\
348	0.00598988030226935\\
349	0.00598988363475505\\
350	0.00598988702515304\\
351	0.00598989047443338\\
352	0.00598989398358235\\
353	0.00598989755360293\\
354	0.00598990118551534\\
355	0.00598990488035761\\
356	0.00598990863918613\\
357	0.00598991246307639\\
358	0.00598991635312357\\
359	0.00598992031044336\\
360	0.00598992433617275\\
361	0.00598992843147083\\
362	0.00598993259751976\\
363	0.00598993683552573\\
364	0.00598994114671998\\
365	0.00598994553235991\\
366	0.00598994999373024\\
367	0.0059899545321443\\
368	0.0059899591489453\\
369	0.00598996384550775\\
370	0.00598996862323893\\
371	0.00598997348358046\\
372	0.00598997842800998\\
373	0.00598998345804282\\
374	0.00598998857523389\\
375	0.00598999378117962\\
376	0.00598999907751997\\
377	0.00599000446594064\\
378	0.00599000994817529\\
379	0.005990015526008\\
380	0.00599002120127579\\
381	0.0059900269758714\\
382	0.00599003285174613\\
383	0.005990038830913\\
384	0.00599004491545003\\
385	0.00599005110750391\\
386	0.00599005740929387\\
387	0.00599006382311597\\
388	0.0059900703513478\\
389	0.0059900769964537\\
390	0.00599008376099058\\
391	0.0059900906476146\\
392	0.00599009765908881\\
393	0.00599010479829229\\
394	0.00599011206823125\\
395	0.00599011947205285\\
396	0.00599012701306268\\
397	0.00599013469474558\\
398	0.00599014252078715\\
399	0.00599015049508643\\
400	0.005990158621744\\
401	0.00599016690501903\\
402	0.00599017534931239\\
403	0.00599018395931498\\
404	0.00599019274013309\\
405	0.00599020169716107\\
406	0.00599021083609386\\
407	0.00599022016293347\\
408	0.005990229683986\\
409	0.00599023940584818\\
410	0.00599024933539231\\
411	0.00599025947978273\\
412	0.00599026984659439\\
413	0.0059902804441112\\
414	0.00599029128171498\\
415	0.00599030236977833\\
416	0.00599031371842137\\
417	0.00599032533843712\\
418	0.00599033724153081\\
419	0.0059903494404351\\
420	0.00599036194904302\\
421	0.0059903747825618\\
422	0.0059903879576918\\
423	0.00599040149283606\\
424	0.00599041540834937\\
425	0.00599042972684191\\
426	0.00599044447356845\\
427	0.00599045967697119\\
428	0.00599047536953586\\
429	0.0059904915893392\\
430	0.00599050838317822\\
431	0.00599052581331632\\
432	0.00599054397226421\\
433	0.00599056301435532\\
434	0.00599058321864331\\
435	0.00599060509727728\\
436	0.0059906295247385\\
437	0.00599065769421434\\
438	0.00599069028369172\\
439	0.0059907251829658\\
440	0.00599076082971954\\
441	0.00599079724424999\\
442	0.0059908344473961\\
443	0.00599087246050493\\
444	0.00599091130538495\\
445	0.00599095100424439\\
446	0.005990991579613\\
447	0.00599103305424788\\
448	0.00599107545103038\\
449	0.0059911187928791\\
450	0.00599116310274929\\
451	0.00599120840390616\\
452	0.00599125472095694\\
453	0.00599130208288913\\
454	0.00599135053134084\\
455	0.0059914001425336\\
456	0.00599145108523068\\
457	0.00599150377480639\\
458	0.00599155928554306\\
459	0.00599162045232058\\
460	0.00599169475124916\\
461	0.00599180136142713\\
462	0.00599198565317281\\
463	0.00599233962071602\\
464	0.00599297402333325\\
465	0.00599361955866146\\
466	0.00599427655214808\\
467	0.00599494534095863\\
468	0.00599562626483714\\
469	0.00599631968201205\\
470	0.00599702597610692\\
471	0.005997745538598\\
472	0.0059984787701488\\
473	0.00599922609748762\\
474	0.0059999880151636\\
475	0.00600076511262502\\
476	0.0060015579194166\\
477	0.00600236697463054\\
478	0.00600319285212695\\
479	0.00600403616365375\\
480	0.00600489756237234\\
481	0.00600577774693629\\
482	0.00600667746619168\\
483	0.00600759752456801\\
484	0.00600853878825506\\
485	0.00600950219236508\\
486	0.00601048874916899\\
487	0.00601149955738371\\
488	0.0060125358129608\\
489	0.00601359882170875\\
490	0.0060146900142099\\
491	0.00601581096374016\\
492	0.00601696340843476\\
493	0.00601814928020679\\
494	0.00601937074604632\\
495	0.00602063027518965\\
496	0.00602193076547143\\
497	0.00602327581159899\\
498	0.00602467031839013\\
499	0.00602612194269185\\
500	0.00602764445355785\\
501	0.00602926520377614\\
502	0.00603103999267444\\
503	0.00603307472568061\\
504	0.0060355189938237\\
505	0.00603824130571579\\
506	0.0060410001268793\\
507	0.00604379470173271\\
508	0.00604662570211901\\
509	0.00604949992360775\\
510	0.00605242333406577\\
511	0.00605539817255656\\
512	0.00605842690935983\\
513	0.00606151228632606\\
514	0.0060646573895223\\
515	0.0060678658129364\\
516	0.00607114209163605\\
517	0.00607449295707083\\
518	0.00607793116295935\\
519	0.00608148753397307\\
520	0.00608524991503473\\
521	0.00608949209344558\\
522	0.00609511012840298\\
523	0.0061051301833072\\
524	0.00613002127605857\\
525	0.00617681469959294\\
526	0.00622505923225858\\
527	0.00627489120333235\\
528	0.00632650112595021\\
529	0.00638014070034895\\
530	0.00643592221275672\\
531	0.00649370166373161\\
532	0.00655353678715795\\
533	0.00661556945805705\\
534	0.00668026597790861\\
535	0.0067496732133194\\
536	0.00682766226147303\\
537	0.00692679819215324\\
538	0.0070461791972775\\
539	0.00716881303397234\\
540	0.00729492360348167\\
541	0.00742475977709956\\
542	0.00755859804459732\\
543	0.00769674804940313\\
544	0.00783957121605228\\
545	0.00798750786172132\\
546	0.00814114880730967\\
547	0.00830141710698412\\
548	0.00847025612688123\\
549	0.00865284739684703\\
550	0.00886188484321872\\
551	0.00907838149041119\\
552	0.00930658073482916\\
553	0.00954395599849905\\
554	0.00976664870602384\\
555	0.00993634648258792\\
556	0.010068111863646\\
557	0.0101969944427767\\
558	0.0103294538541867\\
559	0.0104658877669568\\
560	0.0105743212645119\\
561	0.0106743064569711\\
562	0.010785164267641\\
563	0.010907325968073\\
564	0.0110316075704457\\
565	0.0111582449964869\\
566	0.0112628434642702\\
567	0.0113695049917534\\
568	0.0114784694850645\\
569	0.0115895342149261\\
570	0.0117031671732253\\
571	0.0118202709229055\\
572	0.0119444987730682\\
573	0.0120625053760952\\
574	0.0121625080828611\\
575	0.0122633372533387\\
576	0.0123795470139676\\
577	0.0124958982751172\\
578	0.0126129595122869\\
579	0.0127061356894566\\
580	0.0127925425266601\\
581	0.0128735284462991\\
582	0.0129432208264589\\
583	0.0130080792244955\\
584	0.0130705156478675\\
585	0.0131322524866778\\
586	0.013191742015231\\
587	0.0132490730426834\\
588	0.0133027203032354\\
589	0.0133499992419486\\
590	0.0133923367684799\\
591	0.0134304428998188\\
592	0.0134653360886603\\
593	0.0134971271898046\\
594	0.0135267686953181\\
595	0.0135538870981491\\
596	0.0135808269897195\\
597	0.0136142933738059\\
598	0.0136705746214114\\
599	0\\
600	0\\
};
\addplot [color=mycolor12,solid,forget plot]
  table[row sep=crcr]{%
1	0.00581832489790387\\
2	0.00581832489829987\\
3	0.00581832489870253\\
4	0.00581832489911196\\
5	0.00581832489952827\\
6	0.00581832489995159\\
7	0.00581832490038203\\
8	0.0058183249008197\\
9	0.00581832490126473\\
10	0.00581832490171725\\
11	0.00581832490217738\\
12	0.00581832490264524\\
13	0.00581832490312098\\
14	0.00581832490360471\\
15	0.00581832490409658\\
16	0.00581832490459672\\
17	0.00581832490510528\\
18	0.00581832490562238\\
19	0.00581832490614818\\
20	0.00581832490668283\\
21	0.00581832490722647\\
22	0.00581832490777925\\
23	0.00581832490834133\\
24	0.00581832490891287\\
25	0.00581832490949401\\
26	0.00581832491008494\\
27	0.0058183249106858\\
28	0.00581832491129677\\
29	0.00581832491191802\\
30	0.00581832491254972\\
31	0.00581832491319205\\
32	0.00581832491384519\\
33	0.00581832491450931\\
34	0.00581832491518461\\
35	0.00581832491587127\\
36	0.00581832491656948\\
37	0.00581832491727944\\
38	0.00581832491800136\\
39	0.00581832491873542\\
40	0.00581832491948183\\
41	0.00581832492024081\\
42	0.00581832492101255\\
43	0.0058183249217973\\
44	0.00581832492259525\\
45	0.00581832492340663\\
46	0.00581832492423167\\
47	0.00581832492507061\\
48	0.00581832492592367\\
49	0.00581832492679109\\
50	0.00581832492767312\\
51	0.00581832492857001\\
52	0.005818324929482\\
53	0.00581832493040935\\
54	0.00581832493135232\\
55	0.00581832493231117\\
56	0.00581832493328618\\
57	0.00581832493427762\\
58	0.00581832493528577\\
59	0.0058183249363109\\
60	0.00581832493735331\\
61	0.00581832493841329\\
62	0.00581832493949114\\
63	0.00581832494058715\\
64	0.00581832494170165\\
65	0.00581832494283494\\
66	0.00581832494398734\\
67	0.00581832494515917\\
68	0.00581832494635077\\
69	0.00581832494756247\\
70	0.00581832494879462\\
71	0.00581832495004755\\
72	0.00581832495132163\\
73	0.00581832495261722\\
74	0.00581832495393467\\
75	0.00581832495527436\\
76	0.00581832495663668\\
77	0.005818324958022\\
78	0.00581832495943072\\
79	0.00581832496086323\\
80	0.00581832496231995\\
81	0.00581832496380128\\
82	0.00581832496530765\\
83	0.00581832496683948\\
84	0.0058183249683972\\
85	0.00581832496998126\\
86	0.00581832497159212\\
87	0.00581832497323021\\
88	0.00581832497489603\\
89	0.00581832497659003\\
90	0.0058183249783127\\
91	0.00581832498006453\\
92	0.00581832498184602\\
93	0.00581832498365768\\
94	0.00581832498550002\\
95	0.00581832498737357\\
96	0.00581832498927887\\
97	0.00581832499121647\\
98	0.00581832499318691\\
99	0.00581832499519076\\
100	0.00581832499722859\\
101	0.00581832499930099\\
102	0.00581832500140856\\
103	0.0058183250035519\\
104	0.00581832500573162\\
105	0.00581832500794835\\
106	0.00581832501020274\\
107	0.00581832501249542\\
108	0.00581832501482706\\
109	0.00581832501719833\\
110	0.00581832501960992\\
111	0.00581832502206253\\
112	0.00581832502455685\\
113	0.00581832502709361\\
114	0.00581832502967356\\
115	0.00581832503229742\\
116	0.00581832503496597\\
117	0.00581832503767998\\
118	0.00581832504044023\\
119	0.00581832504324753\\
120	0.00581832504610269\\
121	0.00581832504900654\\
122	0.00581832505195994\\
123	0.00581832505496373\\
124	0.00581832505801879\\
125	0.00581832506112603\\
126	0.00581832506428633\\
127	0.00581832506750062\\
128	0.00581832507076985\\
129	0.00581832507409496\\
130	0.00581832507747694\\
131	0.00581832508091677\\
132	0.00581832508441546\\
133	0.00581832508797403\\
134	0.00581832509159353\\
135	0.00581832509527501\\
136	0.00581832509901957\\
137	0.0058183251028283\\
138	0.00581832510670231\\
139	0.00581832511064276\\
140	0.00581832511465078\\
141	0.00581832511872757\\
142	0.00581832512287433\\
143	0.00581832512709227\\
144	0.00581832513138264\\
145	0.0058183251357467\\
146	0.00581832514018573\\
147	0.00581832514470105\\
148	0.00581832514929399\\
149	0.00581832515396589\\
150	0.00581832515871815\\
151	0.00581832516355215\\
152	0.00581832516846933\\
153	0.00581832517347114\\
154	0.00581832517855905\\
155	0.00581832518373456\\
156	0.0058183251889992\\
157	0.00581832519435452\\
158	0.00581832519980211\\
159	0.00581832520534357\\
160	0.00581832521098053\\
161	0.00581832521671466\\
162	0.00581832522254765\\
163	0.00581832522848121\\
164	0.00581832523451711\\
165	0.00581832524065711\\
166	0.00581832524690304\\
167	0.00581832525325672\\
168	0.00581832525972003\\
169	0.00581832526629488\\
170	0.00581832527298321\\
171	0.00581832527978698\\
172	0.0058183252867082\\
173	0.00581832529374891\\
174	0.00581832530091118\\
175	0.00581832530819712\\
176	0.00581832531560888\\
177	0.00581832532314865\\
178	0.00581832533081863\\
179	0.0058183253386211\\
180	0.00581832534655835\\
181	0.00581832535463273\\
182	0.0058183253628466\\
183	0.0058183253712024\\
184	0.00581832537970259\\
185	0.00581832538834967\\
186	0.00581832539714619\\
187	0.00581832540609476\\
188	0.00581832541519801\\
189	0.00581832542445864\\
190	0.00581832543387937\\
191	0.005818325443463\\
192	0.00581832545321235\\
193	0.0058183254631303\\
194	0.0058183254732198\\
195	0.00581832548348382\\
196	0.00581832549392541\\
197	0.00581832550454765\\
198	0.00581832551535369\\
199	0.00581832552634673\\
200	0.00581832553753004\\
201	0.00581832554890692\\
202	0.00581832556048076\\
203	0.00581832557225498\\
204	0.00581832558423308\\
205	0.00581832559641862\\
206	0.00581832560881522\\
207	0.00581832562142656\\
208	0.00581832563425639\\
209	0.00581832564730853\\
210	0.00581832566058686\\
211	0.00581832567409534\\
212	0.00581832568783798\\
213	0.00581832570181889\\
214	0.00581832571604222\\
215	0.00581832573051222\\
216	0.00581832574523321\\
217	0.00581832576020958\\
218	0.0058183257754458\\
219	0.00581832579094643\\
220	0.00581832580671609\\
221	0.00581832582275951\\
222	0.00581832583908147\\
223	0.00581832585568688\\
224	0.00581832587258069\\
225	0.00581832588976797\\
226	0.00581832590725388\\
227	0.00581832592504365\\
228	0.00581832594314262\\
229	0.00581832596155623\\
230	0.00581832598029001\\
231	0.00581832599934958\\
232	0.00581832601874068\\
233	0.00581832603846913\\
234	0.00581832605854089\\
235	0.00581832607896198\\
236	0.00581832609973856\\
237	0.0058183261208769\\
238	0.00581832614238337\\
239	0.00581832616426445\\
240	0.00581832618652676\\
241	0.00581832620917702\\
242	0.00581832623222208\\
243	0.0058183262556689\\
244	0.00581832627952458\\
245	0.00581832630379633\\
246	0.00581832632849153\\
247	0.00581832635361763\\
248	0.00581832637918226\\
249	0.00581832640519319\\
250	0.0058183264316583\\
251	0.00581832645858564\\
252	0.00581832648598337\\
253	0.00581832651385984\\
254	0.00581832654222353\\
255	0.00581832657108305\\
256	0.00581832660044721\\
257	0.00581832663032495\\
258	0.00581832666072537\\
259	0.00581832669165776\\
260	0.00581832672313154\\
261	0.00581832675515632\\
262	0.00581832678774189\\
263	0.00581832682089819\\
264	0.00581832685463538\\
265	0.00581832688896377\\
266	0.00581832692389386\\
267	0.00581832695943634\\
268	0.00581832699560211\\
269	0.00581832703240224\\
270	0.00581832706984801\\
271	0.0058183271079509\\
272	0.00581832714672261\\
273	0.00581832718617504\\
274	0.00581832722632029\\
275	0.00581832726717068\\
276	0.00581832730873879\\
277	0.00581832735103737\\
278	0.00581832739407942\\
279	0.00581832743787819\\
280	0.00581832748244713\\
281	0.00581832752779997\\
282	0.00581832757395065\\
283	0.00581832762091337\\
284	0.0058183276687026\\
285	0.00581832771733303\\
286	0.00581832776681964\\
287	0.00581832781717766\\
288	0.0058183278684226\\
289	0.00581832792057023\\
290	0.00581832797363661\\
291	0.00581832802763808\\
292	0.00581832808259125\\
293	0.00581832813851304\\
294	0.00581832819542066\\
295	0.00581832825333161\\
296	0.00581832831226372\\
297	0.00581832837223509\\
298	0.00581832843326418\\
299	0.00581832849536972\\
300	0.0058183285585708\\
301	0.00581832862288682\\
302	0.00581832868833752\\
303	0.00581832875494297\\
304	0.00581832882272359\\
305	0.00581832889170014\\
306	0.00581832896189372\\
307	0.00581832903332582\\
308	0.00581832910601826\\
309	0.00581832917999324\\
310	0.00581832925527333\\
311	0.00581832933188147\\
312	0.00581832940984098\\
313	0.00581832948917557\\
314	0.00581832956990935\\
315	0.00581832965206681\\
316	0.00581832973567284\\
317	0.00581832982075275\\
318	0.00581832990733225\\
319	0.00581832999543746\\
320	0.00581833008509494\\
321	0.00581833017633166\\
322	0.00581833026917502\\
323	0.00581833036365287\\
324	0.00581833045979349\\
325	0.00581833055762561\\
326	0.00581833065717842\\
327	0.00581833075848155\\
328	0.00581833086156511\\
329	0.00581833096645968\\
330	0.00581833107319632\\
331	0.00581833118180655\\
332	0.00581833129232241\\
333	0.00581833140477641\\
334	0.00581833151920157\\
335	0.00581833163563143\\
336	0.00581833175410004\\
337	0.00581833187464199\\
338	0.00581833199729237\\
339	0.00581833212208685\\
340	0.00581833224906163\\
341	0.00581833237825349\\
342	0.00581833250969978\\
343	0.00581833264343843\\
344	0.00581833277950797\\
345	0.00581833291794755\\
346	0.00581833305879693\\
347	0.00581833320209652\\
348	0.0058183333478874\\
349	0.0058183334962113\\
350	0.00581833364711067\\
351	0.00581833380062865\\
352	0.00581833395680912\\
353	0.00581833411569675\\
354	0.00581833427733696\\
355	0.005818334441776\\
356	0.00581833460906095\\
357	0.00581833477923977\\
358	0.00581833495236131\\
359	0.00581833512847541\\
360	0.00581833530763282\\
361	0.00581833548988536\\
362	0.00581833567528591\\
363	0.00581833586388843\\
364	0.00581833605574808\\
365	0.00581833625092119\\
366	0.00581833644946541\\
367	0.00581833665143966\\
368	0.00581833685690428\\
369	0.00581833706592107\\
370	0.00581833727855333\\
371	0.00581833749486599\\
372	0.00581833771492562\\
373	0.00581833793880058\\
374	0.00581833816656104\\
375	0.00581833839827914\\
376	0.00581833863402902\\
377	0.00581833887388698\\
378	0.00581833911793154\\
379	0.00581833936624357\\
380	0.00581833961890645\\
381	0.00581833987600612\\
382	0.00581834013763132\\
383	0.00581834040387365\\
384	0.00581834067482777\\
385	0.0058183409505916\\
386	0.00581834123126648\\
387	0.00581834151695741\\
388	0.00581834180777326\\
389	0.00581834210382709\\
390	0.0058183424052364\\
391	0.00581834271212356\\
392	0.0058183430246162\\
393	0.00581834334284778\\
394	0.00581834366695822\\
395	0.00581834399709469\\
396	0.00581834433341246\\
397	0.00581834467607558\\
398	0.00581834502525709\\
399	0.00581834538113839\\
400	0.00581834574390831\\
401	0.00581834611376418\\
402	0.00581834649091711\\
403	0.00581834687559454\\
404	0.00581834726803663\\
405	0.00581834766849666\\
406	0.00581834807724116\\
407	0.00581834849454981\\
408	0.00581834892071511\\
409	0.00581834935604276\\
410	0.00581834980085438\\
411	0.00581835025549441\\
412	0.0058183507203415\\
413	0.00581835119581651\\
414	0.00581835168237091\\
415	0.00581835218045908\\
416	0.00581835269056865\\
417	0.00581835321322928\\
418	0.00581835374901806\\
419	0.00581835429856574\\
420	0.00581835486256403\\
421	0.0058183554417741\\
422	0.00581835603703681\\
423	0.0058183566492854\\
424	0.00581835727956199\\
425	0.00581835792904121\\
426	0.00581835859906795\\
427	0.00581835929122464\\
428	0.00581836000746178\\
429	0.00581836075036148\\
430	0.00581836152366976\\
431	0.00581836233333405\\
432	0.00581836318937482\\
433	0.00581836410880263\\
434	0.0058183651188651\\
435	0.00581836625698095\\
436	0.00581836755806561\\
437	0.00581836901814071\\
438	0.00581837055847237\\
439	0.00581837213191744\\
440	0.00581837373937672\\
441	0.00581837538177419\\
442	0.00581837706005524\\
443	0.00581837877518435\\
444	0.00581838052814224\\
445	0.00581838231992242\\
446	0.00581838415152801\\
447	0.00581838602397008\\
448	0.00581838793827228\\
449	0.00581838989549261\\
450	0.00581839189679052\\
451	0.00581839394360907\\
452	0.00581839603814442\\
453	0.00581839818452696\\
454	0.00581840039174946\\
455	0.00581840268083287\\
456	0.00581840510205807\\
457	0.00581840777524533\\
458	0.00581841097945058\\
459	0.00581841533616052\\
460	0.0058184221246906\\
461	0.00581843362535705\\
462	0.00581845283954372\\
463	0.00581848060156154\\
464	0.00581850885187073\\
465	0.00581853760472771\\
466	0.00581856687484483\\
467	0.00581859667727031\\
468	0.00581862702788363\\
469	0.00581865794346984\\
470	0.00581868944128959\\
471	0.00581872153937687\\
472	0.00581875425738129\\
473	0.0058187876176879\\
474	0.0058188216450338\\
475	0.00581885636261482\\
476	0.00581889179439671\\
477	0.00581892796591022\\
478	0.0058189649043949\\
479	0.00581900263896281\\
480	0.00581904120078762\\
481	0.00581908062332225\\
482	0.00581912094254823\\
483	0.00581916219726215\\
484	0.00581920442940712\\
485	0.00581924768445369\\
486	0.00581929201183502\\
487	0.00581933746545656\\
488	0.00581938410430132\\
489	0.00581943199316437\\
490	0.00581948120357993\\
491	0.00581953181506984\\
492	0.00581958391699713\\
493	0.00581963761166931\\
494	0.00581969302016694\\
495	0.0058197502942304\\
496	0.00581980964150004\\
497	0.00581987137922757\\
498	0.00581993604496821\\
499	0.00582000460877228\\
500	0.00582007882511986\\
501	0.00582016163828671\\
502	0.00582025705246961\\
503	0.0058203676341965\\
504	0.00582048711063806\\
505	0.00582060816568061\\
506	0.00582073080923204\\
507	0.00582085514349397\\
508	0.00582098143430155\\
509	0.00582110989038627\\
510	0.00582124061416979\\
511	0.00582137372091961\\
512	0.00582150934460099\\
513	0.00582164765212948\\
514	0.00582178888055439\\
515	0.00582193343701455\\
516	0.00582208216996372\\
517	0.00582223710324883\\
518	0.00582240339829175\\
519	0.00582259446859486\\
520	0.00582284470057707\\
521	0.00582323828647359\\
522	0.00582396130457422\\
523	0.00582532759938242\\
524	0.00582735445406723\\
525	0.00582943648832064\\
526	0.0058315794036397\\
527	0.00583379036269136\\
528	0.00583607638110128\\
529	0.00583843875706317\\
530	0.00584087441474408\\
531	0.00584339014964481\\
532	0.00584600470657951\\
533	0.0058487672646121\\
534	0.00585179527688337\\
535	0.00585528381838834\\
536	0.00585951716742279\\
537	0.00586430375082576\\
538	0.00586916465514541\\
539	0.00587410490302073\\
540	0.00587913008051202\\
541	0.00588424649089568\\
542	0.00588946166068076\\
543	0.00589478565644572\\
544	0.00590023352335655\\
545	0.00590583073064758\\
546	0.00591162472129789\\
547	0.00591770858640413\\
548	0.00592424385061118\\
549	0.00593137227716606\\
550	0.00593865121852761\\
551	0.00594618889539113\\
552	0.00595405382312025\\
553	0.00596264717864731\\
554	0.00597481321461326\\
555	0.00600213735957934\\
556	0.00608128927895747\\
557	0.00617255740290365\\
558	0.00627195508725874\\
559	0.0063875690700433\\
560	0.00654652108013211\\
561	0.00672363767769963\\
562	0.00690716650038917\\
563	0.00709798667936095\\
564	0.0072983495517545\\
565	0.00751412197944582\\
566	0.00776420888634558\\
567	0.00802128522556084\\
568	0.00828548199163197\\
569	0.00855780269317548\\
570	0.00884029145496344\\
571	0.00913843112296926\\
572	0.00945182182634264\\
573	0.00975000108028537\\
574	0.00999141621540801\\
575	0.010240604889184\\
576	0.0104977720813063\\
577	0.010765592277293\\
578	0.0110457072995958\\
579	0.0112112997926998\\
580	0.0113495343978163\\
581	0.0114659670122866\\
582	0.0115981835042571\\
583	0.0117369240052751\\
584	0.0118584748988712\\
585	0.0119801841401753\\
586	0.012104185300918\\
587	0.0122319943176634\\
588	0.0123625100193166\\
589	0.0124826535075719\\
590	0.0126149618042466\\
591	0.0127742928229332\\
592	0.0129309160813475\\
593	0.0130804870866522\\
594	0.0132100115689649\\
595	0.0133231932805825\\
596	0.0134311271277434\\
597	0.0135416354105032\\
598	0.0136705746214114\\
599	0\\
600	0\\
};
\addplot [color=mycolor13,solid,forget plot]
  table[row sep=crcr]{%
1	0.00628147121413001\\
2	0.00628147121414736\\
3	0.00628147121416499\\
4	0.00628147121418293\\
5	0.00628147121420117\\
6	0.00628147121421971\\
7	0.00628147121423857\\
8	0.00628147121425774\\
9	0.00628147121427723\\
10	0.00628147121429706\\
11	0.00628147121431721\\
12	0.00628147121433771\\
13	0.00628147121435855\\
14	0.00628147121437973\\
15	0.00628147121440128\\
16	0.00628147121442319\\
17	0.00628147121444547\\
18	0.00628147121446812\\
19	0.00628147121449115\\
20	0.00628147121451457\\
21	0.00628147121453839\\
22	0.0062814712145626\\
23	0.00628147121458722\\
24	0.00628147121461226\\
25	0.00628147121463772\\
26	0.0062814712146636\\
27	0.00628147121468992\\
28	0.00628147121471668\\
29	0.0062814712147439\\
30	0.00628147121477157\\
31	0.00628147121479971\\
32	0.00628147121482831\\
33	0.00628147121485741\\
34	0.00628147121488699\\
35	0.00628147121491707\\
36	0.00628147121494765\\
37	0.00628147121497875\\
38	0.00628147121501037\\
39	0.00628147121504253\\
40	0.00628147121507523\\
41	0.00628147121510847\\
42	0.00628147121514228\\
43	0.00628147121517665\\
44	0.0062814712152116\\
45	0.00628147121524715\\
46	0.00628147121528329\\
47	0.00628147121532004\\
48	0.0062814712153574\\
49	0.0062814712153954\\
50	0.00628147121543404\\
51	0.00628147121547332\\
52	0.00628147121551327\\
53	0.0062814712155539\\
54	0.0062814712155952\\
55	0.0062814712156372\\
56	0.00628147121567992\\
57	0.00628147121572334\\
58	0.0062814712157675\\
59	0.00628147121581241\\
60	0.00628147121585807\\
61	0.0062814712159045\\
62	0.00628147121595172\\
63	0.00628147121599973\\
64	0.00628147121604855\\
65	0.00628147121609819\\
66	0.00628147121614867\\
67	0.00628147121620001\\
68	0.0062814712162522\\
69	0.00628147121630528\\
70	0.00628147121635926\\
71	0.00628147121641414\\
72	0.00628147121646995\\
73	0.00628147121652671\\
74	0.00628147121658442\\
75	0.0062814712166431\\
76	0.00628147121670278\\
77	0.00628147121676347\\
78	0.00628147121682518\\
79	0.00628147121688793\\
80	0.00628147121695174\\
81	0.00628147121701664\\
82	0.00628147121708262\\
83	0.00628147121714973\\
84	0.00628147121721797\\
85	0.00628147121728736\\
86	0.00628147121735793\\
87	0.00628147121742969\\
88	0.00628147121750266\\
89	0.00628147121757687\\
90	0.00628147121765234\\
91	0.00628147121772908\\
92	0.00628147121780712\\
93	0.00628147121788649\\
94	0.0062814712179672\\
95	0.00628147121804928\\
96	0.00628147121813274\\
97	0.00628147121821763\\
98	0.00628147121830395\\
99	0.00628147121839174\\
100	0.00628147121848101\\
101	0.00628147121857181\\
102	0.00628147121866414\\
103	0.00628147121875804\\
104	0.00628147121885353\\
105	0.00628147121895065\\
106	0.00628147121904941\\
107	0.00628147121914986\\
108	0.00628147121925201\\
109	0.0062814712193559\\
110	0.00628147121946155\\
111	0.00628147121956901\\
112	0.00628147121967829\\
113	0.00628147121978943\\
114	0.00628147121990246\\
115	0.00628147122001742\\
116	0.00628147122013434\\
117	0.00628147122025325\\
118	0.00628147122037419\\
119	0.00628147122049719\\
120	0.00628147122062228\\
121	0.00628147122074951\\
122	0.00628147122087892\\
123	0.00628147122101053\\
124	0.00628147122114439\\
125	0.00628147122128053\\
126	0.006281471221419\\
127	0.00628147122155984\\
128	0.00628147122170309\\
129	0.00628147122184879\\
130	0.00628147122199697\\
131	0.0062814712221477\\
132	0.006281471222301\\
133	0.00628147122245693\\
134	0.00628147122261553\\
135	0.00628147122277685\\
136	0.00628147122294093\\
137	0.00628147122310783\\
138	0.00628147122327759\\
139	0.00628147122345026\\
140	0.00628147122362589\\
141	0.00628147122380453\\
142	0.00628147122398625\\
143	0.00628147122417108\\
144	0.00628147122435909\\
145	0.00628147122455033\\
146	0.00628147122474485\\
147	0.00628147122494272\\
148	0.006281471225144\\
149	0.00628147122534873\\
150	0.00628147122555698\\
151	0.00628147122576882\\
152	0.00628147122598431\\
153	0.0062814712262035\\
154	0.00628147122642648\\
155	0.00628147122665329\\
156	0.006281471226884\\
157	0.00628147122711869\\
158	0.00628147122735743\\
159	0.00628147122760028\\
160	0.00628147122784732\\
161	0.00628147122809861\\
162	0.00628147122835424\\
163	0.00628147122861428\\
164	0.0062814712288788\\
165	0.00628147122914789\\
166	0.00628147122942161\\
167	0.00628147122970007\\
168	0.00628147122998333\\
169	0.00628147123027147\\
170	0.00628147123056459\\
171	0.00628147123086277\\
172	0.00628147123116609\\
173	0.00628147123147466\\
174	0.00628147123178855\\
175	0.00628147123210786\\
176	0.00628147123243269\\
177	0.00628147123276313\\
178	0.00628147123309927\\
179	0.00628147123344122\\
180	0.00628147123378908\\
181	0.00628147123414295\\
182	0.00628147123450293\\
183	0.00628147123486914\\
184	0.00628147123524167\\
185	0.00628147123562064\\
186	0.00628147123600616\\
187	0.00628147123639834\\
188	0.00628147123679731\\
189	0.00628147123720317\\
190	0.00628147123761605\\
191	0.00628147123803607\\
192	0.00628147123846335\\
193	0.00628147123889802\\
194	0.00628147123934022\\
195	0.00628147123979006\\
196	0.00628147124024768\\
197	0.00628147124071323\\
198	0.00628147124118683\\
199	0.00628147124166862\\
200	0.00628147124215876\\
201	0.00628147124265738\\
202	0.00628147124316464\\
203	0.00628147124368067\\
204	0.00628147124420565\\
205	0.00628147124473972\\
206	0.00628147124528304\\
207	0.00628147124583577\\
208	0.00628147124639809\\
209	0.00628147124697014\\
210	0.00628147124755211\\
211	0.00628147124814417\\
212	0.0062814712487465\\
213	0.00628147124935927\\
214	0.00628147124998266\\
215	0.00628147125061687\\
216	0.00628147125126209\\
217	0.00628147125191849\\
218	0.00628147125258629\\
219	0.00628147125326568\\
220	0.00628147125395687\\
221	0.00628147125466005\\
222	0.00628147125537545\\
223	0.00628147125610327\\
224	0.00628147125684374\\
225	0.00628147125759707\\
226	0.0062814712583635\\
227	0.00628147125914325\\
228	0.00628147125993655\\
229	0.00628147126074364\\
230	0.00628147126156478\\
231	0.00628147126240019\\
232	0.00628147126325015\\
233	0.00628147126411489\\
234	0.00628147126499468\\
235	0.00628147126588979\\
236	0.00628147126680049\\
237	0.00628147126772705\\
238	0.00628147126866974\\
239	0.00628147126962887\\
240	0.0062814712706047\\
241	0.00628147127159755\\
242	0.00628147127260771\\
243	0.00628147127363548\\
244	0.00628147127468119\\
245	0.00628147127574513\\
246	0.00628147127682764\\
247	0.00628147127792905\\
248	0.00628147127904968\\
249	0.00628147128018989\\
250	0.00628147128135\\
251	0.00628147128253039\\
252	0.0062814712837314\\
253	0.00628147128495341\\
254	0.00628147128619678\\
255	0.00628147128746189\\
256	0.00628147128874913\\
257	0.0062814712900589\\
258	0.00628147129139158\\
259	0.00628147129274759\\
260	0.00628147129412734\\
261	0.00628147129553125\\
262	0.00628147129695975\\
263	0.00628147129841328\\
264	0.00628147129989229\\
265	0.00628147130139722\\
266	0.00628147130292854\\
267	0.00628147130448671\\
268	0.00628147130607221\\
269	0.00628147130768554\\
270	0.00628147130932718\\
271	0.00628147131099763\\
272	0.00628147131269741\\
273	0.00628147131442705\\
274	0.00628147131618707\\
275	0.00628147131797801\\
276	0.00628147131980043\\
277	0.00628147132165488\\
278	0.00628147132354193\\
279	0.00628147132546217\\
280	0.00628147132741619\\
281	0.00628147132940458\\
282	0.00628147133142796\\
283	0.00628147133348695\\
284	0.00628147133558219\\
285	0.00628147133771431\\
286	0.00628147133988398\\
287	0.00628147134209187\\
288	0.00628147134433864\\
289	0.00628147134662501\\
290	0.00628147134895166\\
291	0.00628147135131931\\
292	0.00628147135372871\\
293	0.00628147135618057\\
294	0.00628147135867566\\
295	0.00628147136121475\\
296	0.00628147136379862\\
297	0.00628147136642805\\
298	0.00628147136910387\\
299	0.00628147137182688\\
300	0.00628147137459793\\
301	0.00628147137741787\\
302	0.00628147138028755\\
303	0.00628147138320787\\
304	0.00628147138617971\\
305	0.00628147138920397\\
306	0.0062814713922816\\
307	0.00628147139541353\\
308	0.0062814713986007\\
309	0.0062814714018441\\
310	0.00628147140514471\\
311	0.00628147140850354\\
312	0.00628147141192161\\
313	0.00628147141539995\\
314	0.00628147141893962\\
315	0.00628147142254169\\
316	0.00628147142620725\\
317	0.00628147142993741\\
318	0.00628147143373329\\
319	0.00628147143759603\\
320	0.0062814714415268\\
321	0.00628147144552677\\
322	0.00628147144959715\\
323	0.00628147145373914\\
324	0.00628147145795399\\
325	0.00628147146224296\\
326	0.00628147146660731\\
327	0.00628147147104835\\
328	0.00628147147556739\\
329	0.00628147148016576\\
330	0.00628147148484483\\
331	0.00628147148960596\\
332	0.00628147149445057\\
333	0.00628147149938007\\
334	0.0062814715043959\\
335	0.00628147150949953\\
336	0.00628147151469245\\
337	0.00628147151997617\\
338	0.00628147152535223\\
339	0.00628147153082217\\
340	0.00628147153638758\\
341	0.00628147154205008\\
342	0.00628147154781129\\
343	0.00628147155367287\\
344	0.00628147155963651\\
345	0.00628147156570392\\
346	0.00628147157187684\\
347	0.00628147157815703\\
348	0.0062814715845463\\
349	0.00628147159104647\\
350	0.0062814715976594\\
351	0.00628147160438699\\
352	0.00628147161123114\\
353	0.00628147161819383\\
354	0.00628147162527704\\
355	0.00628147163248281\\
356	0.0062814716398132\\
357	0.00628147164727032\\
358	0.00628147165485632\\
359	0.00628147166257339\\
360	0.00628147167042376\\
361	0.00628147167840972\\
362	0.00628147168653359\\
363	0.00628147169479776\\
364	0.00628147170320466\\
365	0.00628147171175679\\
366	0.00628147172045669\\
367	0.00628147172930696\\
368	0.00628147173831029\\
369	0.0062814717474694\\
370	0.00628147175678711\\
371	0.0062814717662663\\
372	0.00628147177590994\\
373	0.00628147178572105\\
374	0.00628147179570277\\
375	0.0062814718058583\\
376	0.00628147181619096\\
377	0.00628147182670415\\
378	0.00628147183740138\\
379	0.00628147184828627\\
380	0.00628147185936255\\
381	0.00628147187063407\\
382	0.00628147188210481\\
383	0.00628147189377888\\
384	0.00628147190566055\\
385	0.00628147191775422\\
386	0.00628147193006447\\
387	0.00628147194259602\\
388	0.00628147195535382\\
389	0.006281471968343\\
390	0.00628147198156888\\
391	0.00628147199503706\\
392	0.00628147200875338\\
393	0.00628147202272398\\
394	0.00628147203695533\\
395	0.00628147205145424\\
396	0.00628147206622789\\
397	0.00628147208128388\\
398	0.0062814720966301\\
399	0.00628147211227487\\
400	0.0062814721282269\\
401	0.00628147214449555\\
402	0.0062814721610908\\
403	0.0062814721780232\\
404	0.00628147219530383\\
405	0.00628147221294438\\
406	0.00628147223095706\\
407	0.00628147224935472\\
408	0.00628147226815083\\
409	0.00628147228735969\\
410	0.00628147230699677\\
411	0.00628147232707903\\
412	0.00628147234762495\\
413	0.0062814723686541\\
414	0.00628147239018669\\
415	0.00628147241224449\\
416	0.0062814724348512\\
417	0.00628147245803268\\
418	0.00628147248181727\\
419	0.00628147250623608\\
420	0.00628147253132343\\
421	0.00628147255711733\\
422	0.00628147258366013\\
423	0.00628147261099941\\
424	0.00628147263918949\\
425	0.00628147266829411\\
426	0.00628147269839173\\
427	0.00628147272958618\\
428	0.00628147276202756\\
429	0.00628147279595017\\
430	0.00628147283173346\\
431	0.00628147286998213\\
432	0.00628147291160126\\
433	0.00628147295781249\\
434	0.00628147300990588\\
435	0.00628147306852271\\
436	0.0062814731326216\\
437	0.00628147319949654\\
438	0.00628147326781429\\
439	0.00628147333761401\\
440	0.00628147340893564\\
441	0.00628147348181964\\
442	0.00628147355630642\\
443	0.00628147363243526\\
444	0.00628147371024214\\
445	0.00628147378975588\\
446	0.00628147387099159\\
447	0.00628147395394181\\
448	0.00628147403857065\\
449	0.00628147412482909\\
450	0.00628147421273357\\
451	0.00628147430255719\\
452	0.0062814743950599\\
453	0.00628147449133799\\
454	0.00628147459370044\\
455	0.00628147470719128\\
456	0.00628147484247193\\
457	0.00628147502095944\\
458	0.00628147528250066\\
459	0.00628147569274341\\
460	0.00628147633863498\\
461	0.0062814772866681\\
462	0.00628147848319528\\
463	0.00628147970090063\\
464	0.00628148094040389\\
465	0.00628148220234519\\
466	0.0062814834873863\\
467	0.00628148479622463\\
468	0.00628148612959329\\
469	0.00628148748825522\\
470	0.00628148887302158\\
471	0.00628149028478149\\
472	0.00628149172452026\\
473	0.00628149319328032\\
474	0.00628149469207649\\
475	0.00628149622196933\\
476	0.00628149778408973\\
477	0.00628149937964554\\
478	0.00628150100992922\\
479	0.00628150267632656\\
480	0.00628150438032678\\
481	0.00628150612353411\\
482	0.0062815079076811\\
483	0.006281509734644\\
484	0.00628151160646064\\
485	0.00628151352535114\\
486	0.00628151549374275\\
487	0.00628151751430047\\
488	0.00628151958996674\\
489	0.00628152172401678\\
490	0.00628152392014375\\
491	0.00628152618260347\\
492	0.00628152851648245\\
493	0.00628153092821948\\
494	0.00628153342663656\\
495	0.00628153602494413\\
496	0.00628153874446304\\
497	0.00628154162096281\\
498	0.00628154471386078\\
499	0.00628154811530101\\
500	0.00628155194729449\\
501	0.00628155631978493\\
502	0.00628156121970753\\
503	0.00628156639622401\\
504	0.00628157164205764\\
505	0.00628157695994823\\
506	0.00628158235621212\\
507	0.00628158784092407\\
508	0.00628159342168317\\
509	0.00628159910347436\\
510	0.0062816048923583\\
511	0.00628161079648248\\
512	0.00628161682856814\\
513	0.00628162301204054\\
514	0.00628162939597606\\
515	0.0062816360907879\\
516	0.00628164335075137\\
517	0.00628165175605578\\
518	0.00628166258614611\\
519	0.00628167849526165\\
520	0.00628170444794106\\
521	0.00628174808475764\\
522	0.0062818164758985\\
523	0.00628190435369219\\
524	0.0062819947145791\\
525	0.00628208781728392\\
526	0.00628218394352367\\
527	0.0062822833220021\\
528	0.00628238604484887\\
529	0.00628249229023778\\
530	0.00628260284199454\\
531	0.00628271940566175\\
532	0.0062828453828084\\
533	0.00628298658580129\\
534	0.00628315017153576\\
535	0.00628334073157942\\
536	0.0062835480693423\\
537	0.00628375861837817\\
538	0.00628397264786947\\
539	0.00628419042589189\\
540	0.00628441227817807\\
541	0.00628463864079491\\
542	0.00628487016622327\\
543	0.00628510790754584\\
544	0.006285353636253\\
545	0.00628561029851052\\
546	0.006285882376759\\
547	0.00628617496290806\\
548	0.00628648931540072\\
549	0.00628681771498464\\
550	0.00628717555884154\\
551	0.00628759886100056\\
552	0.00628818765462217\\
553	0.00628920296487979\\
554	0.00629118004849084\\
555	0.00629474950528016\\
556	0.0062988030043819\\
557	0.00630335050631058\\
558	0.00630875392036417\\
559	0.00631544151441322\\
560	0.00632259192686827\\
561	0.00632990229743072\\
562	0.00633743013254443\\
563	0.00634529534406617\\
564	0.00635369936582436\\
565	0.00636283085700966\\
566	0.0063719863020908\\
567	0.00638117229966024\\
568	0.00639041686327628\\
569	0.0063997609999331\\
570	0.00640922342078908\\
571	0.00641855821935511\\
572	0.00642710414803505\\
573	0.00643465772054127\\
574	0.00644284560486014\\
575	0.00645283916946108\\
576	0.0064680788191831\\
577	0.00649840856303665\\
578	0.00657251775791106\\
579	0.00677655935682935\\
580	0.0070217576200483\\
581	0.00729254469105682\\
582	0.0075717849880937\\
583	0.00786229006219552\\
584	0.00816640480651304\\
585	0.00847778550003963\\
586	0.00879571311756991\\
587	0.00912237943904783\\
588	0.00946326359254427\\
589	0.0098145825724013\\
590	0.01019532223335\\
591	0.0105794353774955\\
592	0.0109592601665937\\
593	0.0113136891797993\\
594	0.0115893659094649\\
595	0.0118468086384495\\
596	0.0120848708026119\\
597	0.0123651760751447\\
598	0.0127378231003575\\
599	0\\
600	0\\
};
\addplot [color=mycolor14,solid,forget plot]
  table[row sep=crcr]{%
1	0.0136309658319098\\
2	0.0136309658319092\\
3	0.0136309658319086\\
4	0.013630965831908\\
5	0.0136309658319074\\
6	0.0136309658319068\\
7	0.0136309658319061\\
8	0.0136309658319055\\
9	0.0136309658319048\\
10	0.0136309658319041\\
11	0.0136309658319034\\
12	0.0136309658319027\\
13	0.013630965831902\\
14	0.0136309658319013\\
15	0.0136309658319006\\
16	0.0136309658318998\\
17	0.013630965831899\\
18	0.0136309658318983\\
19	0.0136309658318975\\
20	0.0136309658318966\\
21	0.0136309658318958\\
22	0.013630965831895\\
23	0.0136309658318941\\
24	0.0136309658318933\\
25	0.0136309658318924\\
26	0.0136309658318915\\
27	0.0136309658318906\\
28	0.0136309658318896\\
29	0.0136309658318887\\
30	0.0136309658318877\\
31	0.0136309658318867\\
32	0.0136309658318857\\
33	0.0136309658318847\\
34	0.0136309658318837\\
35	0.0136309658318826\\
36	0.0136309658318815\\
37	0.0136309658318804\\
38	0.0136309658318793\\
39	0.0136309658318782\\
40	0.013630965831877\\
41	0.0136309658318759\\
42	0.0136309658318747\\
43	0.0136309658318735\\
44	0.0136309658318722\\
45	0.013630965831871\\
46	0.0136309658318697\\
47	0.0136309658318684\\
48	0.013630965831867\\
49	0.0136309658318657\\
50	0.0136309658318643\\
51	0.0136309658318629\\
52	0.0136309658318615\\
53	0.01363096583186\\
54	0.0136309658318586\\
55	0.013630965831857\\
56	0.0136309658318555\\
57	0.013630965831854\\
58	0.0136309658318524\\
59	0.0136309658318508\\
60	0.0136309658318491\\
61	0.0136309658318474\\
62	0.0136309658318457\\
63	0.013630965831844\\
64	0.0136309658318422\\
65	0.0136309658318404\\
66	0.0136309658318386\\
67	0.0136309658318368\\
68	0.0136309658318349\\
69	0.0136309658318329\\
70	0.013630965831831\\
71	0.013630965831829\\
72	0.0136309658318269\\
73	0.0136309658318249\\
74	0.0136309658318228\\
75	0.0136309658318206\\
76	0.0136309658318184\\
77	0.0136309658318162\\
78	0.013630965831814\\
79	0.0136309658318117\\
80	0.0136309658318093\\
81	0.0136309658318069\\
82	0.0136309658318045\\
83	0.013630965831802\\
84	0.0136309658317995\\
85	0.013630965831797\\
86	0.0136309658317944\\
87	0.0136309658317917\\
88	0.013630965831789\\
89	0.0136309658317863\\
90	0.0136309658317835\\
91	0.0136309658317806\\
92	0.0136309658317777\\
93	0.0136309658317748\\
94	0.0136309658317718\\
95	0.0136309658317687\\
96	0.0136309658317656\\
97	0.0136309658317625\\
98	0.0136309658317593\\
99	0.013630965831756\\
100	0.0136309658317526\\
101	0.0136309658317492\\
102	0.0136309658317458\\
103	0.0136309658317423\\
104	0.0136309658317387\\
105	0.013630965831735\\
106	0.0136309658317313\\
107	0.0136309658317276\\
108	0.0136309658317237\\
109	0.0136309658317198\\
110	0.0136309658317158\\
111	0.0136309658317118\\
112	0.0136309658317077\\
113	0.0136309658317035\\
114	0.0136309658316992\\
115	0.0136309658316948\\
116	0.0136309658316904\\
117	0.0136309658316859\\
118	0.0136309658316813\\
119	0.0136309658316767\\
120	0.0136309658316719\\
121	0.0136309658316671\\
122	0.0136309658316622\\
123	0.0136309658316571\\
124	0.013630965831652\\
125	0.0136309658316469\\
126	0.0136309658316416\\
127	0.0136309658316362\\
128	0.0136309658316307\\
129	0.0136309658316252\\
130	0.0136309658316195\\
131	0.0136309658316137\\
132	0.0136309658316078\\
133	0.0136309658316019\\
134	0.0136309658315958\\
135	0.0136309658315896\\
136	0.0136309658315833\\
137	0.0136309658315769\\
138	0.0136309658315703\\
139	0.0136309658315637\\
140	0.0136309658315569\\
141	0.01363096583155\\
142	0.013630965831543\\
143	0.0136309658315359\\
144	0.0136309658315286\\
145	0.0136309658315212\\
146	0.0136309658315137\\
147	0.0136309658315061\\
148	0.0136309658314983\\
149	0.0136309658314903\\
150	0.0136309658314823\\
151	0.013630965831474\\
152	0.0136309658314657\\
153	0.0136309658314572\\
154	0.0136309658314485\\
155	0.0136309658314397\\
156	0.0136309658314307\\
157	0.0136309658314215\\
158	0.0136309658314122\\
159	0.0136309658314028\\
160	0.0136309658313931\\
161	0.0136309658313833\\
162	0.0136309658313733\\
163	0.0136309658313632\\
164	0.0136309658313528\\
165	0.0136309658313423\\
166	0.0136309658313316\\
167	0.0136309658313207\\
168	0.0136309658313096\\
169	0.0136309658312982\\
170	0.0136309658312867\\
171	0.013630965831275\\
172	0.0136309658312631\\
173	0.013630965831251\\
174	0.0136309658312386\\
175	0.0136309658312261\\
176	0.0136309658312133\\
177	0.0136309658312002\\
178	0.013630965831187\\
179	0.0136309658311735\\
180	0.0136309658311597\\
181	0.0136309658311458\\
182	0.0136309658311315\\
183	0.013630965831117\\
184	0.0136309658311023\\
185	0.0136309658310873\\
186	0.013630965831072\\
187	0.0136309658310564\\
188	0.0136309658310406\\
189	0.0136309658310245\\
190	0.0136309658310081\\
191	0.0136309658309914\\
192	0.0136309658309744\\
193	0.0136309658309571\\
194	0.0136309658309395\\
195	0.0136309658309215\\
196	0.0136309658309033\\
197	0.0136309658308847\\
198	0.0136309658308658\\
199	0.0136309658308466\\
200	0.013630965830827\\
201	0.013630965830807\\
202	0.0136309658307867\\
203	0.013630965830766\\
204	0.013630965830745\\
205	0.0136309658307236\\
206	0.0136309658307018\\
207	0.0136309658306796\\
208	0.013630965830657\\
209	0.013630965830634\\
210	0.0136309658306106\\
211	0.0136309658305868\\
212	0.0136309658305625\\
213	0.0136309658305378\\
214	0.0136309658305127\\
215	0.0136309658304871\\
216	0.013630965830461\\
217	0.0136309658304345\\
218	0.0136309658304075\\
219	0.01363096583038\\
220	0.013630965830352\\
221	0.0136309658303235\\
222	0.0136309658302946\\
223	0.013630965830265\\
224	0.013630965830235\\
225	0.0136309658302044\\
226	0.0136309658301732\\
227	0.0136309658301415\\
228	0.0136309658301092\\
229	0.0136309658300764\\
230	0.0136309658300429\\
231	0.0136309658300089\\
232	0.0136309658299742\\
233	0.0136309658299389\\
234	0.0136309658299029\\
235	0.0136309658298663\\
236	0.0136309658298291\\
237	0.0136309658297911\\
238	0.0136309658297525\\
239	0.0136309658297132\\
240	0.0136309658296732\\
241	0.0136309658296324\\
242	0.0136309658295909\\
243	0.0136309658295487\\
244	0.0136309658295057\\
245	0.0136309658294619\\
246	0.0136309658294173\\
247	0.0136309658293719\\
248	0.0136309658293257\\
249	0.0136309658292786\\
250	0.0136309658292307\\
251	0.0136309658291819\\
252	0.0136309658291322\\
253	0.0136309658290816\\
254	0.0136309658290301\\
255	0.0136309658289777\\
256	0.0136309658289243\\
257	0.0136309658288699\\
258	0.0136309658288145\\
259	0.0136309658287582\\
260	0.0136309658287008\\
261	0.0136309658286423\\
262	0.0136309658285828\\
263	0.0136309658285222\\
264	0.0136309658284605\\
265	0.0136309658283976\\
266	0.0136309658283336\\
267	0.0136309658282684\\
268	0.0136309658282021\\
269	0.0136309658281345\\
270	0.0136309658280657\\
271	0.0136309658279956\\
272	0.0136309658279242\\
273	0.0136309658278515\\
274	0.0136309658277775\\
275	0.0136309658277021\\
276	0.0136309658276254\\
277	0.0136309658275472\\
278	0.0136309658274676\\
279	0.0136309658273865\\
280	0.0136309658273039\\
281	0.0136309658272198\\
282	0.0136309658271342\\
283	0.0136309658270469\\
284	0.0136309658269581\\
285	0.0136309658268676\\
286	0.0136309658267754\\
287	0.0136309658266816\\
288	0.013630965826586\\
289	0.0136309658264886\\
290	0.0136309658263895\\
291	0.0136309658262885\\
292	0.0136309658261856\\
293	0.0136309658260808\\
294	0.0136309658259741\\
295	0.0136309658258654\\
296	0.0136309658257547\\
297	0.0136309658256419\\
298	0.013630965825527\\
299	0.01363096582541\\
300	0.0136309658252908\\
301	0.0136309658251694\\
302	0.0136309658250458\\
303	0.0136309658249198\\
304	0.0136309658247915\\
305	0.0136309658246608\\
306	0.0136309658245277\\
307	0.0136309658243921\\
308	0.013630965824254\\
309	0.0136309658241133\\
310	0.0136309658239699\\
311	0.0136309658238239\\
312	0.0136309658236752\\
313	0.0136309658235237\\
314	0.0136309658233694\\
315	0.0136309658232122\\
316	0.013630965823052\\
317	0.0136309658228889\\
318	0.0136309658227227\\
319	0.0136309658225534\\
320	0.013630965822381\\
321	0.0136309658222053\\
322	0.0136309658220263\\
323	0.0136309658218441\\
324	0.0136309658216584\\
325	0.0136309658214692\\
326	0.0136309658212765\\
327	0.0136309658210802\\
328	0.0136309658208802\\
329	0.0136309658206765\\
330	0.013630965820469\\
331	0.0136309658202576\\
332	0.0136309658200423\\
333	0.013630965819823\\
334	0.0136309658195995\\
335	0.0136309658193719\\
336	0.0136309658191401\\
337	0.0136309658189039\\
338	0.0136309658186633\\
339	0.0136309658184183\\
340	0.0136309658181687\\
341	0.0136309658179144\\
342	0.0136309658176554\\
343	0.0136309658173915\\
344	0.0136309658171228\\
345	0.0136309658168491\\
346	0.0136309658165702\\
347	0.0136309658162862\\
348	0.0136309658159969\\
349	0.0136309658157022\\
350	0.0136309658154021\\
351	0.0136309658150963\\
352	0.0136309658147849\\
353	0.0136309658144678\\
354	0.0136309658141447\\
355	0.0136309658138156\\
356	0.0136309658134805\\
357	0.0136309658131391\\
358	0.0136309658127914\\
359	0.0136309658124372\\
360	0.0136309658120765\\
361	0.0136309658117091\\
362	0.0136309658113348\\
363	0.0136309658109537\\
364	0.0136309658105654\\
365	0.01363096581017\\
366	0.0136309658097672\\
367	0.013630965809357\\
368	0.0136309658089391\\
369	0.0136309658085135\\
370	0.0136309658080799\\
371	0.0136309658076383\\
372	0.0136309658071885\\
373	0.0136309658067302\\
374	0.0136309658062634\\
375	0.0136309658057879\\
376	0.0136309658053034\\
377	0.0136309658048099\\
378	0.013630965804307\\
379	0.0136309658037947\\
380	0.0136309658032727\\
381	0.0136309658027408\\
382	0.0136309658021988\\
383	0.0136309658016464\\
384	0.0136309658010834\\
385	0.0136309658005096\\
386	0.0136309657999247\\
387	0.0136309657993284\\
388	0.0136309657987204\\
389	0.0136309657981005\\
390	0.0136309657974682\\
391	0.0136309657968233\\
392	0.0136309657961654\\
393	0.0136309657954942\\
394	0.0136309657948092\\
395	0.01363096579411\\
396	0.0136309657933961\\
397	0.0136309657926671\\
398	0.0136309657919225\\
399	0.0136309657911618\\
400	0.0136309657903844\\
401	0.0136309657895898\\
402	0.0136309657887772\\
403	0.0136309657879462\\
404	0.013630965787096\\
405	0.0136309657862258\\
406	0.0136309657853349\\
407	0.0136309657844224\\
408	0.0136309657834875\\
409	0.0136309657825293\\
410	0.0136309657815466\\
411	0.0136309657805385\\
412	0.0136309657795036\\
413	0.0136309657784407\\
414	0.0136309657773484\\
415	0.0136309657762249\\
416	0.0136309657750685\\
417	0.0136309657738773\\
418	0.0136309657726488\\
419	0.0136309657713804\\
420	0.0136309657700688\\
421	0.0136309657687097\\
422	0.0136309657672969\\
423	0.0136309657658208\\
424	0.0136309657642657\\
425	0.0136309657626042\\
426	0.0136309657607904\\
427	0.0136309657587515\\
428	0.0136309657563858\\
429	0.0136309657535858\\
430	0.0136309657503083\\
431	0.0136309657466805\\
432	0.0136309657429676\\
433	0.0136309657391668\\
434	0.0136309657352752\\
435	0.0136309657312896\\
436	0.0136309657272067\\
437	0.013630965723023\\
438	0.0136309657187344\\
439	0.0136309657143366\\
440	0.0136309657098235\\
441	0.0136309657051866\\
442	0.0136309657004106\\
443	0.0136309656954656\\
444	0.0136309656902881\\
445	0.0136309656847411\\
446	0.0136309656785315\\
447	0.013630965671055\\
448	0.0136309656611344\\
449	0.0136309656466795\\
450	0.0136309656245269\\
451	0.0136309655913045\\
452	0.0136309655469044\\
453	0.0136309654998666\\
454	0.0136309654519184\\
455	0.0136309654030474\\
456	0.0136309653532393\\
457	0.0136309653024774\\
458	0.013630965250742\\
459	0.0136309651980122\\
460	0.0136309651442694\\
461	0.0136309650895013\\
462	0.0136309650336988\\
463	0.0136309649768395\\
464	0.0136309649188926\\
465	0.0136309648598258\\
466	0.0136309647996054\\
467	0.0136309647381964\\
468	0.0136309646755616\\
469	0.0136309646116622\\
470	0.0136309645464572\\
471	0.0136309644799027\\
472	0.0136309644119504\\
473	0.0136309643425497\\
474	0.013630964271651\\
475	0.0136309641992018\\
476	0.0136309641251456\\
477	0.0136309640494224\\
478	0.0136309639719679\\
479	0.0136309638927129\\
480	0.0136309638115836\\
481	0.0136309637285\\
482	0.0136309636433764\\
483	0.0136309635561199\\
484	0.0136309634666299\\
485	0.0136309633747972\\
486	0.013630963280503\\
487	0.0136309631836174\\
488	0.0136309630839976\\
489	0.0136309629814868\\
490	0.0136309628759107\\
491	0.0136309627670747\\
492	0.0136309626547577\\
493	0.0136309625387017\\
494	0.0136309624185915\\
495	0.0136309622940133\\
496	0.0136309621643663\\
497	0.013630962028683\\
498	0.0136309618852705\\
499	0.0136309617310475\\
500	0.0136309615604597\\
501	0.0136309613641172\\
502	0.0136309611282811\\
503	0.013630960838769\\
504	0.013630960495943\\
505	0.0136309601374296\\
506	0.0136309597720466\\
507	0.0136309593992937\\
508	0.0136309590186467\\
509	0.0136309586297127\\
510	0.0136309582320433\\
511	0.0136309578251015\\
512	0.0136309574081761\\
513	0.0136309569801729\\
514	0.0136309565391112\\
515	0.0136309560809131\\
516	0.0136309555965559\\
517	0.0136309550656014\\
518	0.0136309544422492\\
519	0.0136309536277669\\
520	0.0136309524244403\\
521	0.0136309504887083\\
522	0.0136309474053044\\
523	0.0136309433583529\\
524	0.0136309391912565\\
525	0.013630934891648\\
526	0.0136309304417525\\
527	0.0136309258155281\\
528	0.0136309209774774\\
529	0.0136309158897506\\
530	0.0136309105385125\\
531	0.0136309049660484\\
532	0.0136308991100912\\
533	0.0136308928258591\\
534	0.0136308858304809\\
535	0.013630877613074\\
536	0.0136308673464894\\
537	0.0136308542106188\\
538	0.0136308395397956\\
539	0.0136308245085861\\
540	0.0136308090866307\\
541	0.0136307932383445\\
542	0.0136307769202442\\
543	0.0136307600738726\\
544	0.0136307426035434\\
545	0.0136307243249696\\
546	0.0136307048793325\\
547	0.0136306836169608\\
548	0.0136306594396842\\
549	0.0136306309167202\\
550	0.0136305979611434\\
551	0.0136305635345089\\
552	0.0136305263721561\\
553	0.0136304830193048\\
554	0.0136304240864966\\
555	0.0136303274829646\\
556	0.0136288328597215\\
557	0.0136271291649729\\
558	0.013625322902744\\
559	0.0136233838755326\\
560	0.0136209301368864\\
561	0.0136157382692607\\
562	0.0136103287534191\\
563	0.0136046753881353\\
564	0.0135987351491936\\
565	0.0135924396775756\\
566	0.0135828185746823\\
567	0.0135702057495042\\
568	0.0135570775097813\\
569	0.0135433711508054\\
570	0.0135290093188268\\
571	0.0135138881562953\\
572	0.0134978576477121\\
573	0.0134710351308557\\
574	0.0134410716924379\\
575	0.0134097084547733\\
576	0.0133768221605966\\
577	0.0133420423097614\\
578	0.0133048030495129\\
579	0.013248127640383\\
580	0.0131755790204489\\
581	0.0130742536442268\\
582	0.0129476602131191\\
583	0.0128068643687614\\
584	0.0126134852512881\\
585	0.0124117489980442\\
586	0.01220885155371\\
587	0.0120013107491774\\
588	0.0117804111508331\\
589	0.0114523476566414\\
590	0.0111087673967892\\
591	0.0107473412609005\\
592	0.0103596458212847\\
593	0.00991258394449295\\
594	0.00943759941930204\\
595	0.00890136474132322\\
596	0.00784893456155462\\
597	0.00634254856127415\\
598	0.00367057462141144\\
599	0\\
600	0\\
};
\addplot [color=mycolor15,solid,forget plot]
  table[row sep=crcr]{%
1	0.0135932197841419\\
2	0.0135932197841296\\
3	0.0135932197841171\\
4	0.0135932197841044\\
5	0.0135932197840914\\
6	0.0135932197840782\\
7	0.0135932197840647\\
8	0.0135932197840511\\
9	0.0135932197840371\\
10	0.013593219784023\\
11	0.0135932197840085\\
12	0.0135932197839938\\
13	0.0135932197839789\\
14	0.0135932197839637\\
15	0.0135932197839482\\
16	0.0135932197839324\\
17	0.0135932197839164\\
18	0.0135932197839\\
19	0.0135932197838834\\
20	0.0135932197838665\\
21	0.0135932197838492\\
22	0.0135932197838317\\
23	0.0135932197838139\\
24	0.0135932197837957\\
25	0.0135932197837772\\
26	0.0135932197837584\\
27	0.0135932197837392\\
28	0.0135932197837197\\
29	0.0135932197836999\\
30	0.0135932197836797\\
31	0.0135932197836591\\
32	0.0135932197836382\\
33	0.0135932197836169\\
34	0.0135932197835952\\
35	0.0135932197835731\\
36	0.0135932197835506\\
37	0.0135932197835278\\
38	0.0135932197835045\\
39	0.0135932197834808\\
40	0.0135932197834567\\
41	0.0135932197834321\\
42	0.0135932197834071\\
43	0.0135932197833817\\
44	0.0135932197833558\\
45	0.0135932197833295\\
46	0.0135932197833026\\
47	0.0135932197832753\\
48	0.0135932197832475\\
49	0.0135932197832193\\
50	0.0135932197831905\\
51	0.0135932197831612\\
52	0.0135932197831313\\
53	0.013593219783101\\
54	0.0135932197830701\\
55	0.0135932197830386\\
56	0.0135932197830066\\
57	0.013593219782974\\
58	0.0135932197829409\\
59	0.0135932197829071\\
60	0.0135932197828727\\
61	0.0135932197828378\\
62	0.0135932197828021\\
63	0.0135932197827659\\
64	0.013593219782729\\
65	0.0135932197826915\\
66	0.0135932197826533\\
67	0.0135932197826144\\
68	0.0135932197825748\\
69	0.0135932197825345\\
70	0.0135932197824935\\
71	0.0135932197824517\\
72	0.0135932197824092\\
73	0.013593219782366\\
74	0.013593219782322\\
75	0.0135932197822772\\
76	0.0135932197822315\\
77	0.0135932197821851\\
78	0.0135932197821379\\
79	0.0135932197820898\\
80	0.0135932197820408\\
81	0.013593219781991\\
82	0.0135932197819403\\
83	0.0135932197818887\\
84	0.0135932197818361\\
85	0.0135932197817826\\
86	0.0135932197817282\\
87	0.0135932197816728\\
88	0.0135932197816164\\
89	0.013593219781559\\
90	0.0135932197815006\\
91	0.0135932197814411\\
92	0.0135932197813806\\
93	0.013593219781319\\
94	0.0135932197812563\\
95	0.0135932197811924\\
96	0.0135932197811275\\
97	0.0135932197810614\\
98	0.013593219780994\\
99	0.0135932197809255\\
100	0.0135932197808558\\
101	0.0135932197807848\\
102	0.0135932197807126\\
103	0.0135932197806391\\
104	0.0135932197805642\\
105	0.0135932197804881\\
106	0.0135932197804105\\
107	0.0135932197803316\\
108	0.0135932197802513\\
109	0.0135932197801695\\
110	0.0135932197800863\\
111	0.0135932197800016\\
112	0.0135932197799154\\
113	0.0135932197798277\\
114	0.0135932197797384\\
115	0.0135932197796474\\
116	0.0135932197795549\\
117	0.0135932197794607\\
118	0.0135932197793649\\
119	0.0135932197792673\\
120	0.013593219779168\\
121	0.0135932197790669\\
122	0.013593219778964\\
123	0.0135932197788593\\
124	0.0135932197787527\\
125	0.0135932197786442\\
126	0.0135932197785338\\
127	0.0135932197784214\\
128	0.013593219778307\\
129	0.0135932197781905\\
130	0.013593219778072\\
131	0.0135932197779514\\
132	0.0135932197778286\\
133	0.0135932197777036\\
134	0.0135932197775764\\
135	0.0135932197774469\\
136	0.0135932197773151\\
137	0.013593219777181\\
138	0.0135932197770445\\
139	0.0135932197769055\\
140	0.013593219776764\\
141	0.01359321977662\\
142	0.0135932197764735\\
143	0.0135932197763243\\
144	0.0135932197761725\\
145	0.013593219776018\\
146	0.0135932197758607\\
147	0.0135932197757006\\
148	0.0135932197755376\\
149	0.0135932197753718\\
150	0.0135932197752029\\
151	0.0135932197750311\\
152	0.0135932197748562\\
153	0.0135932197746782\\
154	0.0135932197744969\\
155	0.0135932197743125\\
156	0.0135932197741248\\
157	0.0135932197739337\\
158	0.0135932197737392\\
159	0.0135932197735412\\
160	0.0135932197733397\\
161	0.0135932197731346\\
162	0.0135932197729258\\
163	0.0135932197727133\\
164	0.013593219772497\\
165	0.0135932197722768\\
166	0.0135932197720527\\
167	0.0135932197718246\\
168	0.0135932197715924\\
169	0.0135932197713561\\
170	0.0135932197711156\\
171	0.0135932197708707\\
172	0.0135932197706215\\
173	0.0135932197703678\\
174	0.0135932197701096\\
175	0.0135932197698467\\
176	0.0135932197695792\\
177	0.0135932197693069\\
178	0.0135932197690297\\
179	0.0135932197687476\\
180	0.0135932197684604\\
181	0.013593219768168\\
182	0.0135932197678705\\
183	0.0135932197675676\\
184	0.0135932197672593\\
185	0.0135932197669455\\
186	0.013593219766626\\
187	0.0135932197663009\\
188	0.0135932197659699\\
189	0.013593219765633\\
190	0.01359321976529\\
191	0.013593219764941\\
192	0.0135932197645856\\
193	0.0135932197642239\\
194	0.0135932197638557\\
195	0.013593219763481\\
196	0.0135932197630995\\
197	0.0135932197627111\\
198	0.0135932197623158\\
199	0.0135932197619135\\
200	0.0135932197615039\\
201	0.0135932197610869\\
202	0.0135932197606625\\
203	0.0135932197602304\\
204	0.0135932197597906\\
205	0.0135932197593429\\
206	0.0135932197588872\\
207	0.0135932197584233\\
208	0.0135932197579511\\
209	0.0135932197574703\\
210	0.013593219756981\\
211	0.0135932197564828\\
212	0.0135932197559757\\
213	0.0135932197554595\\
214	0.013593219754934\\
215	0.013593219754399\\
216	0.0135932197538544\\
217	0.0135932197533\\
218	0.0135932197527357\\
219	0.0135932197521612\\
220	0.0135932197515763\\
221	0.0135932197509809\\
222	0.0135932197503748\\
223	0.0135932197497577\\
224	0.0135932197491295\\
225	0.0135932197484901\\
226	0.013593219747839\\
227	0.0135932197471762\\
228	0.0135932197465015\\
229	0.0135932197458146\\
230	0.0135932197451153\\
231	0.0135932197444034\\
232	0.0135932197436786\\
233	0.0135932197429407\\
234	0.0135932197421894\\
235	0.0135932197414246\\
236	0.0135932197406459\\
237	0.0135932197398532\\
238	0.0135932197390461\\
239	0.0135932197382244\\
240	0.0135932197373878\\
241	0.0135932197365361\\
242	0.0135932197356689\\
243	0.013593219734786\\
244	0.0135932197338871\\
245	0.0135932197329719\\
246	0.0135932197320401\\
247	0.0135932197310914\\
248	0.0135932197301254\\
249	0.0135932197291419\\
250	0.0135932197281405\\
251	0.0135932197271209\\
252	0.0135932197260827\\
253	0.0135932197250257\\
254	0.0135932197239494\\
255	0.0135932197228535\\
256	0.0135932197217376\\
257	0.0135932197206015\\
258	0.0135932197194446\\
259	0.0135932197182665\\
260	0.013593219717067\\
261	0.0135932197158456\\
262	0.0135932197146019\\
263	0.0135932197133355\\
264	0.0135932197120459\\
265	0.0135932197107327\\
266	0.0135932197093955\\
267	0.0135932197080338\\
268	0.0135932197066472\\
269	0.0135932197052352\\
270	0.0135932197037972\\
271	0.013593219702333\\
272	0.0135932197008418\\
273	0.0135932196993233\\
274	0.0135932196977769\\
275	0.0135932196962021\\
276	0.0135932196945983\\
277	0.0135932196929651\\
278	0.0135932196913018\\
279	0.0135932196896079\\
280	0.0135932196878827\\
281	0.0135932196861258\\
282	0.0135932196843365\\
283	0.0135932196825143\\
284	0.0135932196806584\\
285	0.0135932196787682\\
286	0.0135932196768431\\
287	0.0135932196748825\\
288	0.0135932196728856\\
289	0.0135932196708518\\
290	0.0135932196687803\\
291	0.0135932196666706\\
292	0.0135932196645217\\
293	0.013593219662333\\
294	0.0135932196601038\\
295	0.0135932196578332\\
296	0.0135932196555206\\
297	0.013593219653165\\
298	0.0135932196507656\\
299	0.0135932196483217\\
300	0.0135932196458324\\
301	0.0135932196432968\\
302	0.0135932196407141\\
303	0.0135932196380833\\
304	0.0135932196354036\\
305	0.0135932196326739\\
306	0.0135932196298935\\
307	0.0135932196270613\\
308	0.0135932196241762\\
309	0.0135932196212375\\
310	0.0135932196182439\\
311	0.0135932196151945\\
312	0.0135932196120882\\
313	0.013593219608924\\
314	0.0135932196057007\\
315	0.0135932196024172\\
316	0.0135932195990725\\
317	0.0135932195956653\\
318	0.0135932195921944\\
319	0.0135932195886588\\
320	0.0135932195850571\\
321	0.0135932195813881\\
322	0.0135932195776505\\
323	0.0135932195738432\\
324	0.0135932195699646\\
325	0.0135932195660137\\
326	0.0135932195619888\\
327	0.0135932195578888\\
328	0.0135932195537122\\
329	0.0135932195494576\\
330	0.0135932195451234\\
331	0.0135932195407083\\
332	0.0135932195362108\\
333	0.0135932195316292\\
334	0.0135932195269621\\
335	0.0135932195222079\\
336	0.013593219517365\\
337	0.0135932195124317\\
338	0.0135932195074063\\
339	0.0135932195022872\\
340	0.0135932194970726\\
341	0.0135932194917608\\
342	0.01359321948635\\
343	0.0135932194808384\\
344	0.0135932194752241\\
345	0.0135932194695053\\
346	0.01359321946368\\
347	0.0135932194577463\\
348	0.0135932194517021\\
349	0.0135932194455456\\
350	0.0135932194392746\\
351	0.013593219432887\\
352	0.0135932194263807\\
353	0.0135932194197535\\
354	0.0135932194130032\\
355	0.0135932194061276\\
356	0.0135932193991243\\
357	0.0135932193919911\\
358	0.0135932193847255\\
359	0.013593219377325\\
360	0.0135932193697873\\
361	0.0135932193621098\\
362	0.0135932193542898\\
363	0.0135932193463248\\
364	0.013593219338212\\
365	0.0135932193299487\\
366	0.0135932193215321\\
367	0.0135932193129592\\
368	0.0135932193042271\\
369	0.0135932192953328\\
370	0.0135932192862731\\
371	0.0135932192770449\\
372	0.0135932192676449\\
373	0.0135932192580698\\
374	0.013593219248316\\
375	0.01359321923838\\
376	0.0135932192282581\\
377	0.0135932192179467\\
378	0.0135932192074416\\
379	0.0135932191967391\\
380	0.0135932191858347\\
381	0.0135932191747243\\
382	0.0135932191634034\\
383	0.0135932191518672\\
384	0.0135932191401111\\
385	0.0135932191281298\\
386	0.0135932191159182\\
387	0.0135932191034708\\
388	0.0135932190907817\\
389	0.0135932190778449\\
390	0.013593219064654\\
391	0.0135932190512022\\
392	0.0135932190374824\\
393	0.0135932190234872\\
394	0.0135932190092088\\
395	0.0135932189946388\\
396	0.0135932189797684\\
397	0.0135932189645881\\
398	0.013593218949088\\
399	0.0135932189332576\\
400	0.0135932189170861\\
401	0.0135932189005617\\
402	0.0135932188836725\\
403	0.0135932188664059\\
404	0.0135932188487484\\
405	0.0135932188306858\\
406	0.0135932188122028\\
407	0.0135932187932826\\
408	0.013593218773908\\
409	0.0135932187540608\\
410	0.0135932187337219\\
411	0.0135932187128703\\
412	0.0135932186914832\\
413	0.0135932186695359\\
414	0.0135932186470013\\
415	0.0135932186238498\\
416	0.0135932186000486\\
417	0.0135932185755616\\
418	0.0135932185503485\\
419	0.0135932185243641\\
420	0.013593218497557\\
421	0.013593218469867\\
422	0.013593218441221\\
423	0.013593218411523\\
424	0.0135932183806336\\
425	0.0135932183483272\\
426	0.0135932183142044\\
427	0.0135932182775288\\
428	0.0135932182369605\\
429	0.013593218190242\\
430	0.0135932181341764\\
431	0.013593218065893\\
432	0.0135932179868151\\
433	0.0135932179058867\\
434	0.013593217823047\\
435	0.0135932177382321\\
436	0.0135932176513749\\
437	0.0135932175624045\\
438	0.013593217471246\\
439	0.0135932173778201\\
440	0.0135932172820408\\
441	0.0135932171838123\\
442	0.0135932170830186\\
443	0.0135932169794986\\
444	0.0135932168729815\\
445	0.0135932167629235\\
446	0.0135932166481035\\
447	0.0135932165256474\\
448	0.0135932163887724\\
449	0.0135932162218852\\
450	0.0135932159910419\\
451	0.0135932156292496\\
452	0.0135932150282004\\
453	0.0135932140971164\\
454	0.0135932130911099\\
455	0.0135932120654681\\
456	0.0135932110199364\\
457	0.0135932099542239\\
458	0.0135932088679815\\
459	0.0135932077607833\\
460	0.0135932066321342\\
461	0.0135932054815525\\
462	0.0135932043087469\\
463	0.0135932031136975\\
464	0.0135932018960189\\
465	0.0135932006550508\\
466	0.013593199390102\\
467	0.0135931981004553\\
468	0.0135931967853607\\
469	0.0135931954440204\\
470	0.0135931940756077\\
471	0.0135931926792591\\
472	0.0135931912540512\\
473	0.0135931897989649\\
474	0.0135931883128931\\
475	0.0135931867947886\\
476	0.0135931852435366\\
477	0.0135931836579486\\
478	0.0135931820367557\\
479	0.0135931803786008\\
480	0.0135931786820304\\
481	0.0135931769454848\\
482	0.0135931751672878\\
483	0.0135931733456344\\
484	0.013593171478577\\
485	0.0135931695640099\\
486	0.0135931675996496\\
487	0.0135931655830132\\
488	0.0135931635113931\\
489	0.0135931613818266\\
490	0.0135931591910602\\
491	0.0135931569355063\\
492	0.0135931546111897\\
493	0.0135931522136725\\
494	0.0135931497379362\\
495	0.0135931471781747\\
496	0.0135931445273783\\
497	0.0135931417764218\\
498	0.0135931389119859\\
499	0.013593135911803\\
500	0.013593132734053\\
501	0.013593129294998\\
502	0.0135931254265748\\
503	0.0135931208129123\\
504	0.0135931149581395\\
505	0.0135931074611348\\
506	0.0135930995062341\\
507	0.0135930913989287\\
508	0.0135930831285117\\
509	0.0135930746823357\\
510	0.0135930660520342\\
511	0.013593057228361\\
512	0.0135930482011203\\
513	0.0135930389586948\\
514	0.0135930294869083\\
515	0.013593019766179\\
516	0.0135930097637886\\
517	0.013592999412673\\
518	0.0135929885533615\\
519	0.0135929767769817\\
520	0.0135929630110051\\
521	0.0135929444739167\\
522	0.013592914254782\\
523	0.0135926139817828\\
524	0.0135917764855341\\
525	0.0135909138463869\\
526	0.0135900244244648\\
527	0.0135891063177543\\
528	0.0135881572440032\\
529	0.0135871743808378\\
530	0.0135861542693009\\
531	0.0135850932917075\\
532	0.0135839903275175\\
533	0.013582840726992\\
534	0.0135816380376323\\
535	0.0135803732307596\\
536	0.0135790324388508\\
537	0.0135775901783382\\
538	0.0135749761387627\\
539	0.0135714785929435\\
540	0.0135678714532827\\
541	0.0135641459664595\\
542	0.0135602922420424\\
543	0.0135562990667041\\
544	0.0135521536976276\\
545	0.0135478413944967\\
546	0.0135433442830304\\
547	0.0135386393883875\\
548	0.0135336960639342\\
549	0.0135284683137031\\
550	0.0135220830673666\\
551	0.0135120280228964\\
552	0.0135016098785498\\
553	0.0134907954391512\\
554	0.0134795394141261\\
555	0.0134677311343968\\
556	0.0134499208573963\\
557	0.0134307715056464\\
558	0.0134106655327996\\
559	0.0133894455856878\\
560	0.0133655570771487\\
561	0.0133301155115372\\
562	0.0132880438621706\\
563	0.0132388632035299\\
564	0.0131874769572801\\
565	0.0131335520270486\\
566	0.0130652957374099\\
567	0.0129839692940525\\
568	0.0128992898165501\\
569	0.0128109303966229\\
570	0.0127184098303629\\
571	0.0126211504570585\\
572	0.0125183702779609\\
573	0.0123710604313286\\
574	0.0122092834627664\\
575	0.0120333012433002\\
576	0.0118335896846035\\
577	0.0116243980867168\\
578	0.0114045767228592\\
579	0.0111990107889634\\
580	0.0110013257164267\\
581	0.0108232897877673\\
582	0.0105506744218719\\
583	0.0102637014066868\\
584	0.0100262185391095\\
585	0.00977915643767162\\
586	0.00952148512625224\\
587	0.00925143808347156\\
588	0.00896945464650578\\
589	0.0087817577863876\\
590	0.00855226441525957\\
591	0.00832092118641841\\
592	0.00804511381779336\\
593	0.00749792355242844\\
594	0.00690888881459738\\
595	0.00623921690448287\\
596	0.0058271594182152\\
597	0.0051299581010561\\
598	0.00367057462141144\\
599	0\\
600	0\\
};
\addplot [color=mycolor16,solid,forget plot]
  table[row sep=crcr]{%
1	0.0135908104290373\\
2	0.0135908104287727\\
3	0.0135908104285034\\
4	0.0135908104282292\\
5	0.0135908104279502\\
6	0.0135908104276662\\
7	0.0135908104273771\\
8	0.0135908104270828\\
9	0.0135908104267833\\
10	0.0135908104264784\\
11	0.0135908104261681\\
12	0.0135908104258522\\
13	0.0135908104255307\\
14	0.0135908104252034\\
15	0.0135908104248703\\
16	0.0135908104245312\\
17	0.0135908104241861\\
18	0.0135908104238348\\
19	0.0135908104234772\\
20	0.0135908104231132\\
21	0.0135908104227428\\
22	0.0135908104223656\\
23	0.0135908104219818\\
24	0.0135908104215911\\
25	0.0135908104211934\\
26	0.0135908104207886\\
27	0.0135908104203766\\
28	0.0135908104199572\\
29	0.0135908104195304\\
30	0.0135908104190958\\
31	0.0135908104186536\\
32	0.0135908104182034\\
33	0.0135908104177452\\
34	0.0135908104172788\\
35	0.013590810416804\\
36	0.0135908104163208\\
37	0.013590810415829\\
38	0.0135908104153283\\
39	0.0135908104148188\\
40	0.0135908104143001\\
41	0.0135908104137721\\
42	0.0135908104132348\\
43	0.0135908104126878\\
44	0.013590810412131\\
45	0.0135908104115643\\
46	0.0135908104109875\\
47	0.0135908104104004\\
48	0.0135908104098028\\
49	0.0135908104091945\\
50	0.0135908104085754\\
51	0.0135908104079452\\
52	0.0135908104073038\\
53	0.0135908104066509\\
54	0.0135908104059863\\
55	0.0135908104053099\\
56	0.0135908104046214\\
57	0.0135908104039205\\
58	0.0135908104032072\\
59	0.0135908104024812\\
60	0.0135908104017422\\
61	0.0135908104009899\\
62	0.0135908104002243\\
63	0.013590810399445\\
64	0.0135908103986518\\
65	0.0135908103978444\\
66	0.0135908103970226\\
67	0.0135908103961861\\
68	0.0135908103953347\\
69	0.0135908103944681\\
70	0.013590810393586\\
71	0.0135908103926882\\
72	0.0135908103917744\\
73	0.0135908103908442\\
74	0.0135908103898974\\
75	0.0135908103889338\\
76	0.0135908103879529\\
77	0.0135908103869546\\
78	0.0135908103859384\\
79	0.013590810384904\\
80	0.0135908103838513\\
81	0.0135908103827797\\
82	0.013590810381689\\
83	0.0135908103805788\\
84	0.0135908103794489\\
85	0.0135908103782987\\
86	0.0135908103771281\\
87	0.0135908103759365\\
88	0.0135908103747237\\
89	0.0135908103734892\\
90	0.0135908103722327\\
91	0.0135908103709538\\
92	0.0135908103696521\\
93	0.0135908103683271\\
94	0.0135908103669785\\
95	0.0135908103656059\\
96	0.0135908103642087\\
97	0.0135908103627866\\
98	0.0135908103613392\\
99	0.0135908103598659\\
100	0.0135908103583663\\
101	0.01359081035684\\
102	0.0135908103552864\\
103	0.0135908103537051\\
104	0.0135908103520956\\
105	0.0135908103504574\\
106	0.01359081034879\\
107	0.0135908103470928\\
108	0.0135908103453653\\
109	0.013590810343607\\
110	0.0135908103418174\\
111	0.0135908103399958\\
112	0.0135908103381417\\
113	0.0135908103362546\\
114	0.0135908103343337\\
115	0.0135908103323786\\
116	0.0135908103303887\\
117	0.0135908103283632\\
118	0.0135908103263016\\
119	0.0135908103242032\\
120	0.0135908103220673\\
121	0.0135908103198934\\
122	0.0135908103176807\\
123	0.0135908103154285\\
124	0.0135908103131361\\
125	0.0135908103108028\\
126	0.0135908103084279\\
127	0.0135908103060106\\
128	0.0135908103035502\\
129	0.0135908103010459\\
130	0.0135908102984969\\
131	0.0135908102959025\\
132	0.0135908102932617\\
133	0.0135908102905739\\
134	0.013590810287838\\
135	0.0135908102850534\\
136	0.013590810282219\\
137	0.0135908102793341\\
138	0.0135908102763977\\
139	0.0135908102734089\\
140	0.0135908102703668\\
141	0.0135908102672704\\
142	0.0135908102641187\\
143	0.0135908102609108\\
144	0.0135908102576456\\
145	0.0135908102543222\\
146	0.0135908102509394\\
147	0.0135908102474963\\
148	0.0135908102439917\\
149	0.0135908102404245\\
150	0.0135908102367937\\
151	0.0135908102330981\\
152	0.0135908102293364\\
153	0.0135908102255077\\
154	0.0135908102216105\\
155	0.0135908102176438\\
156	0.0135908102136063\\
157	0.0135908102094967\\
158	0.0135908102053137\\
159	0.013590810201056\\
160	0.0135908101967222\\
161	0.0135908101923111\\
162	0.0135908101878212\\
163	0.0135908101832511\\
164	0.0135908101785993\\
165	0.0135908101738644\\
166	0.013590810169045\\
167	0.0135908101641395\\
168	0.0135908101591463\\
169	0.0135908101540639\\
170	0.0135908101488906\\
171	0.0135908101436249\\
172	0.0135908101382652\\
173	0.0135908101328096\\
174	0.0135908101272564\\
175	0.013590810121604\\
176	0.0135908101158506\\
177	0.0135908101099943\\
178	0.0135908101040332\\
179	0.0135908100979656\\
180	0.0135908100917894\\
181	0.0135908100855028\\
182	0.0135908100791037\\
183	0.0135908100725901\\
184	0.01359081006596\\
185	0.0135908100592112\\
186	0.0135908100523417\\
187	0.0135908100453492\\
188	0.0135908100382315\\
189	0.0135908100309864\\
190	0.0135908100236116\\
191	0.0135908100161047\\
192	0.0135908100084634\\
193	0.0135908100006852\\
194	0.0135908099927677\\
195	0.0135908099847083\\
196	0.0135908099765044\\
197	0.0135908099681536\\
198	0.013590809959653\\
199	0.0135908099510001\\
200	0.013590809942192\\
201	0.013590809933226\\
202	0.0135908099240991\\
203	0.0135908099148085\\
204	0.0135908099053512\\
205	0.0135908098957242\\
206	0.0135908098859244\\
207	0.0135908098759487\\
208	0.0135908098657939\\
209	0.0135908098554567\\
210	0.0135908098449338\\
211	0.0135908098342219\\
212	0.0135908098233175\\
213	0.0135908098122171\\
214	0.0135908098009171\\
215	0.013590809789414\\
216	0.013590809777704\\
217	0.0135908097657833\\
218	0.0135908097536482\\
219	0.0135908097412947\\
220	0.0135908097287187\\
221	0.0135908097159164\\
222	0.0135908097028835\\
223	0.0135908096896158\\
224	0.0135908096761091\\
225	0.0135908096623589\\
226	0.0135908096483609\\
227	0.0135908096341104\\
228	0.0135908096196029\\
229	0.0135908096048336\\
230	0.0135908095897978\\
231	0.0135908095744905\\
232	0.0135908095589068\\
233	0.0135908095430416\\
234	0.0135908095268898\\
235	0.013590809510446\\
236	0.013590809493705\\
237	0.0135908094766611\\
238	0.013590809459309\\
239	0.0135908094416428\\
240	0.0135908094236568\\
241	0.0135908094053451\\
242	0.0135908093867016\\
243	0.0135908093677204\\
244	0.013590809348395\\
245	0.0135908093287192\\
246	0.0135908093086865\\
247	0.0135908092882903\\
248	0.0135908092675238\\
249	0.0135908092463802\\
250	0.0135908092248524\\
251	0.0135908092029334\\
252	0.013590809180616\\
253	0.0135908091578926\\
254	0.0135908091347557\\
255	0.0135908091111977\\
256	0.0135908090872107\\
257	0.0135908090627868\\
258	0.0135908090379177\\
259	0.0135908090125952\\
260	0.0135908089868108\\
261	0.013590808960556\\
262	0.0135908089338218\\
263	0.0135908089065994\\
264	0.0135908088788796\\
265	0.0135908088506531\\
266	0.0135908088219104\\
267	0.0135908087926419\\
268	0.0135908087628376\\
269	0.0135908087324876\\
270	0.0135908087015815\\
271	0.013590808670109\\
272	0.0135908086380593\\
273	0.0135908086054217\\
274	0.013590808572185\\
275	0.013590808538338\\
276	0.0135908085038693\\
277	0.0135908084687669\\
278	0.0135908084330191\\
279	0.0135908083966136\\
280	0.0135908083595381\\
281	0.0135908083217798\\
282	0.0135908082833258\\
283	0.0135908082441631\\
284	0.0135908082042781\\
285	0.0135908081636573\\
286	0.0135908081222867\\
287	0.0135908080801521\\
288	0.0135908080372391\\
289	0.0135908079935328\\
290	0.0135908079490182\\
291	0.0135908079036801\\
292	0.0135908078575028\\
293	0.0135908078104703\\
294	0.0135908077625665\\
295	0.0135908077137748\\
296	0.0135908076640784\\
297	0.01359080761346\\
298	0.0135908075619023\\
299	0.0135908075093872\\
300	0.0135908074558968\\
301	0.0135908074014124\\
302	0.0135908073459151\\
303	0.0135908072893859\\
304	0.013590807231805\\
305	0.0135908071731525\\
306	0.0135908071134081\\
307	0.013590807052551\\
308	0.0135908069905602\\
309	0.0135908069274142\\
310	0.013590806863091\\
311	0.0135908067975685\\
312	0.0135908067308238\\
313	0.0135908066628338\\
314	0.0135908065935751\\
315	0.0135908065230236\\
316	0.0135908064511549\\
317	0.0135908063779441\\
318	0.0135908063033659\\
319	0.0135908062273946\\
320	0.0135908061500039\\
321	0.0135908060711671\\
322	0.013590805990857\\
323	0.013590805909046\\
324	0.0135908058257058\\
325	0.0135908057408079\\
326	0.013590805654323\\
327	0.0135908055662214\\
328	0.0135908054764731\\
329	0.0135908053850471\\
330	0.0135908052919122\\
331	0.0135908051970366\\
332	0.013590805100388\\
333	0.0135908050019333\\
334	0.013590804901639\\
335	0.013590804799471\\
336	0.0135908046953946\\
337	0.0135908045893745\\
338	0.0135908044813748\\
339	0.0135908043713589\\
340	0.0135908042592896\\
341	0.013590804145129\\
342	0.0135908040288387\\
343	0.0135908039103794\\
344	0.0135908037897114\\
345	0.0135908036667939\\
346	0.0135908035415858\\
347	0.013590803414045\\
348	0.0135908032841286\\
349	0.0135908031517933\\
350	0.0135908030169945\\
351	0.0135908028796873\\
352	0.0135908027398257\\
353	0.0135908025973629\\
354	0.0135908024522513\\
355	0.0135908023044422\\
356	0.0135908021538863\\
357	0.0135908020005332\\
358	0.0135908018443314\\
359	0.0135908016852288\\
360	0.0135908015231717\\
361	0.0135908013581058\\
362	0.0135908011899755\\
363	0.0135908010187241\\
364	0.0135908008442935\\
365	0.0135908006666246\\
366	0.0135908004856569\\
367	0.0135908003013285\\
368	0.0135908001135761\\
369	0.013590799922335\\
370	0.0135907997275387\\
371	0.0135907995291192\\
372	0.0135907993270068\\
373	0.0135907991211299\\
374	0.0135907989114149\\
375	0.0135907986977863\\
376	0.0135907984801662\\
377	0.0135907982584747\\
378	0.0135907980326293\\
379	0.0135907978025448\\
380	0.0135907975681333\\
381	0.013590797329304\\
382	0.0135907970859629\\
383	0.0135907968380124\\
384	0.0135907965853515\\
385	0.0135907963278752\\
386	0.0135907960654739\\
387	0.0135907957980338\\
388	0.0135907955254355\\
389	0.0135907952475541\\
390	0.0135907949642585\\
391	0.0135907946754102\\
392	0.0135907943808632\\
393	0.0135907940804639\\
394	0.0135907937740516\\
395	0.0135907934614587\\
396	0.0135907931425058\\
397	0.0135907928170009\\
398	0.0135907924847411\\
399	0.0135907921455125\\
400	0.0135907917990889\\
401	0.0135907914452324\\
402	0.0135907910836928\\
403	0.0135907907142078\\
404	0.0135907903365033\\
405	0.0135907899502926\\
406	0.0135907895552718\\
407	0.0135907891511126\\
408	0.0135907887374555\\
409	0.0135907883139227\\
410	0.0135907878801319\\
411	0.0135907874356714\\
412	0.0135907869800964\\
413	0.0135907865129246\\
414	0.0135907860336319\\
415	0.0135907855416464\\
416	0.0135907850363421\\
417	0.0135907845170314\\
418	0.0135907839829558\\
419	0.0135907834332754\\
420	0.0135907828670547\\
421	0.0135907822832443\\
422	0.0135907816806521\\
423	0.0135907810578927\\
424	0.0135907804132849\\
425	0.0135907797446252\\
426	0.0135907790486649\\
427	0.0135907783199013\\
428	0.013590777547864\\
429	0.0135907767113967\\
430	0.013590775768038\\
431	0.0135907746394975\\
432	0.0135907732119976\\
433	0.0135907714372479\\
434	0.0135907696209855\\
435	0.0135907677618541\\
436	0.0135907658584281\\
437	0.0135907639092082\\
438	0.0135907619126173\\
439	0.0135907598669951\\
440	0.013590757770592\\
441	0.0135907556215592\\
442	0.0135907534179282\\
443	0.0135907511575594\\
444	0.0135907488380055\\
445	0.0135907464561295\\
446	0.0135907440070411\\
447	0.0135907414811241\\
448	0.0135907388557906\\
449	0.0135907360728446\\
450	0.0135907329774583\\
451	0.0135907291586135\\
452	0.0135907235539922\\
453	0.013590713572431\\
454	0.0135905369218037\\
455	0.0135903226356763\\
456	0.0135901040800384\\
457	0.013589881203469\\
458	0.0135896539510847\\
459	0.0135894222623881\\
460	0.0135891860686443\\
461	0.0135889452901721\\
462	0.013588699834955\\
463	0.0135884496014782\\
464	0.0135881944857666\\
465	0.0135879343578735\\
466	0.0135876690680143\\
467	0.013587398457499\\
468	0.0135871223581931\\
469	0.0135868405916892\\
470	0.0135865529686578\\
471	0.0135862592901814\\
472	0.0135859593477668\\
473	0.0135856529233033\\
474	0.0135853397888447\\
475	0.0135850197069876\\
476	0.0135846924357881\\
477	0.013584357717821\\
478	0.013584015278822\\
479	0.0135836648261947\\
480	0.0135833060473495\\
481	0.0135829386078454\\
482	0.0135825621493043\\
483	0.0135821762870612\\
484	0.0135817806075048\\
485	0.0135813746650585\\
486	0.0135809579787303\\
487	0.0135805300281369\\
488	0.0135800902489075\\
489	0.0135796380274215\\
490	0.0135791726946506\\
491	0.0135786935190064\\
492	0.013578199697949\\
493	0.0135776903481839\\
494	0.0135771644940534\\
495	0.0135766210536061\\
496	0.0135760588218378\\
497	0.0135754764495512\\
498	0.0135748724148539\\
499	0.0135742449802419\\
500	0.0135735921176499\\
501	0.0135729113567644\\
502	0.0135721994446177\\
503	0.0135714515478408\\
504	0.0135706594115897\\
505	0.0135698073976464\\
506	0.0135681068946364\\
507	0.0135661566953231\\
508	0.0135641595633222\\
509	0.0135621129635245\\
510	0.0135600141867689\\
511	0.0135578605816622\\
512	0.0135556492573261\\
513	0.0135533770613859\\
514	0.01355104054395\\
515	0.0135486359163115\\
516	0.0135461590048053\\
517	0.0135436051849832\\
518	0.0135409692621817\\
519	0.0135382451872574\\
520	0.0135354252328481\\
521	0.0135324973402145\\
522	0.0135294361079529\\
523	0.0135252162630884\\
524	0.0135187677670761\\
525	0.0135121079540014\\
526	0.0135052211040621\\
527	0.013498089192789\\
528	0.0134906909676028\\
529	0.0134829998034698\\
530	0.0134749793009951\\
531	0.0134665742762438\\
532	0.0134551454254896\\
533	0.0134430442177135\\
534	0.0134304998074297\\
535	0.0134174679131611\\
536	0.0134038785892172\\
537	0.0133896345947009\\
538	0.0133704932703215\\
539	0.0133475835735536\\
540	0.0133239522344869\\
541	0.0132995439297899\\
542	0.0132742962221328\\
543	0.0132481382185071\\
544	0.0132209888182322\\
545	0.0131927590803361\\
546	0.0131633551268653\\
547	0.0131326636812784\\
548	0.0131005021831366\\
549	0.013066625085545\\
550	0.0130275100032482\\
551	0.0129680684138083\\
552	0.0129030636340772\\
553	0.0128357209451848\\
554	0.0127659206929571\\
555	0.0126935089383745\\
556	0.0126250651689276\\
557	0.0125544504967602\\
558	0.0124808251137211\\
559	0.0124039057504147\\
560	0.0123251588096792\\
561	0.0122555320740259\\
562	0.0121626363514115\\
563	0.0120450432553115\\
564	0.0119228010699805\\
565	0.0117956695672022\\
566	0.0116782872122191\\
567	0.0115688671451299\\
568	0.0114545948702469\\
569	0.0113278280186925\\
570	0.0111807175583761\\
571	0.0110272440274142\\
572	0.0108668484800807\\
573	0.0107523932132728\\
574	0.0106439375041794\\
575	0.0105038700875634\\
576	0.0102939715526381\\
577	0.01007868824332\\
578	0.00985783771767488\\
579	0.00963066371746887\\
580	0.00939599427996934\\
581	0.00915311264567365\\
582	0.00900965264746922\\
583	0.00888046166828602\\
584	0.00875004454059655\\
585	0.00862047927770268\\
586	0.00849436294964468\\
587	0.00831683286812368\\
588	0.00815218297991306\\
589	0.00792856360702847\\
590	0.00746027952429212\\
591	0.00698674329305618\\
592	0.00654854605102128\\
593	0.00639716114546897\\
594	0.0062374075909446\\
595	0.00604937501958872\\
596	0.00574702137257038\\
597	0.0051299581010561\\
598	0.00367057462141144\\
599	0\\
600	0\\
};
\addplot [color=mycolor17,solid,forget plot]
  table[row sep=crcr]{%
1	0.0135741955054137\\
2	0.0135741954475592\\
3	0.01357419538867\\
4	0.0135741953287279\\
5	0.0135741952677141\\
6	0.0135741952056092\\
7	0.013574195142394\\
8	0.0135741950780486\\
9	0.0135741950125527\\
10	0.0135741949458859\\
11	0.0135741948780272\\
12	0.0135741948089554\\
13	0.0135741947386488\\
14	0.0135741946670854\\
15	0.0135741945942427\\
16	0.013574194520098\\
17	0.0135741944446279\\
18	0.0135741943678087\\
19	0.0135741942896166\\
20	0.0135741942100268\\
21	0.0135741941290145\\
22	0.0135741940465543\\
23	0.0135741939626204\\
24	0.0135741938771864\\
25	0.0135741937902256\\
26	0.0135741937017107\\
27	0.013574193611614\\
28	0.0135741935199073\\
29	0.0135741934265618\\
30	0.0135741933315483\\
31	0.013574193234837\\
32	0.0135741931363977\\
33	0.0135741930361995\\
34	0.0135741929342111\\
35	0.0135741928304004\\
36	0.013574192724735\\
37	0.0135741926171817\\
38	0.013574192507707\\
39	0.0135741923962765\\
40	0.0135741922828553\\
41	0.013574192167408\\
42	0.0135741920498983\\
43	0.0135741919302896\\
44	0.0135741918085442\\
45	0.0135741916846242\\
46	0.0135741915584908\\
47	0.0135741914301044\\
48	0.0135741912994248\\
49	0.0135741911664113\\
50	0.0135741910310221\\
51	0.0135741908932148\\
52	0.0135741907529464\\
53	0.013574190610173\\
54	0.0135741904648498\\
55	0.0135741903169314\\
56	0.0135741901663716\\
57	0.0135741900131231\\
58	0.0135741898571381\\
59	0.0135741896983677\\
60	0.0135741895367623\\
61	0.0135741893722714\\
62	0.0135741892048434\\
63	0.013574189034426\\
64	0.0135741888609659\\
65	0.0135741886844088\\
66	0.0135741885046995\\
67	0.0135741883217818\\
68	0.0135741881355985\\
69	0.0135741879460914\\
70	0.0135741877532011\\
71	0.0135741875568674\\
72	0.0135741873570288\\
73	0.0135741871536229\\
74	0.013574186946586\\
75	0.0135741867358535\\
76	0.0135741865213594\\
77	0.0135741863030366\\
78	0.0135741860808169\\
79	0.0135741858546308\\
80	0.0135741856244075\\
81	0.0135741853900752\\
82	0.0135741851515605\\
83	0.0135741849087888\\
84	0.0135741846616843\\
85	0.0135741844101696\\
86	0.0135741841541662\\
87	0.0135741838935941\\
88	0.0135741836283717\\
89	0.0135741833584161\\
90	0.013574183083643\\
91	0.0135741828039664\\
92	0.013574182519299\\
93	0.0135741822295516\\
94	0.0135741819346338\\
95	0.0135741816344533\\
96	0.0135741813289163\\
97	0.0135741810179273\\
98	0.013574180701389\\
99	0.0135741803792025\\
100	0.0135741800512671\\
101	0.0135741797174803\\
102	0.0135741793777376\\
103	0.0135741790319329\\
104	0.0135741786799581\\
105	0.0135741783217031\\
106	0.0135741779570559\\
107	0.0135741775859025\\
108	0.0135741772081269\\
109	0.0135741768236109\\
110	0.0135741764322343\\
111	0.0135741760338747\\
112	0.0135741756284076\\
113	0.0135741752157061\\
114	0.0135741747956413\\
115	0.0135741743680817\\
116	0.0135741739328936\\
117	0.0135741734899409\\
118	0.0135741730390851\\
119	0.0135741725801851\\
120	0.0135741721130975\\
121	0.013574171637676\\
122	0.013574171153772\\
123	0.0135741706612341\\
124	0.0135741701599082\\
125	0.0135741696496373\\
126	0.013574169130262\\
127	0.0135741686016195\\
128	0.0135741680635445\\
129	0.0135741675158686\\
130	0.0135741669584203\\
131	0.0135741663910252\\
132	0.0135741658135056\\
133	0.0135741652256807\\
134	0.0135741646273664\\
135	0.0135741640183755\\
136	0.0135741633985171\\
137	0.0135741627675972\\
138	0.0135741621254181\\
139	0.0135741614717787\\
140	0.013574160806474\\
141	0.0135741601292958\\
142	0.0135741594400318\\
143	0.0135741587384658\\
144	0.0135741580243782\\
145	0.0135741572975448\\
146	0.013574156557738\\
147	0.0135741558047256\\
148	0.0135741550382715\\
149	0.0135741542581353\\
150	0.0135741534640722\\
151	0.0135741526558331\\
152	0.0135741518331643\\
153	0.0135741509958076\\
154	0.0135741501435001\\
155	0.0135741492759742\\
156	0.0135741483929575\\
157	0.0135741474941725\\
158	0.0135741465793369\\
159	0.0135741456481633\\
160	0.0135741447003589\\
161	0.0135741437356258\\
162	0.0135741427536607\\
163	0.0135741417541547\\
164	0.0135741407367934\\
165	0.0135741397012567\\
166	0.0135741386472187\\
167	0.0135741375743476\\
168	0.0135741364823055\\
169	0.0135741353707486\\
170	0.0135741342393265\\
171	0.0135741330876828\\
172	0.0135741319154545\\
173	0.0135741307222719\\
174	0.0135741295077587\\
175	0.0135741282715317\\
176	0.0135741270132009\\
177	0.013574125732369\\
178	0.0135741244286314\\
179	0.0135741231015765\\
180	0.0135741217507848\\
181	0.0135741203758293\\
182	0.0135741189762754\\
183	0.0135741175516802\\
184	0.013574116101593\\
185	0.0135741146255548\\
186	0.0135741131230981\\
187	0.013574111593747\\
188	0.0135741100370168\\
189	0.013574108452414\\
190	0.0135741068394358\\
191	0.0135741051975707\\
192	0.0135741035262972\\
193	0.0135741018250847\\
194	0.0135741000933927\\
195	0.0135740983306706\\
196	0.013574096536358\\
197	0.013574094709884\\
198	0.0135740928506671\\
199	0.0135740909581152\\
200	0.0135740890316253\\
201	0.0135740870705833\\
202	0.0135740850743636\\
203	0.0135740830423293\\
204	0.0135740809738314\\
205	0.0135740788682092\\
206	0.0135740767247896\\
207	0.0135740745428871\\
208	0.0135740723218033\\
209	0.0135740700608272\\
210	0.0135740677592343\\
211	0.0135740654162867\\
212	0.0135740630312328\\
213	0.0135740606033071\\
214	0.0135740581317298\\
215	0.0135740556157064\\
216	0.0135740530544277\\
217	0.0135740504470695\\
218	0.0135740477927921\\
219	0.01357404509074\\
220	0.0135740423400419\\
221	0.01357403953981\\
222	0.01357403668914\\
223	0.0135740337871105\\
224	0.0135740308327828\\
225	0.0135740278252007\\
226	0.01357402476339\\
227	0.013574021646358\\
228	0.0135740184730935\\
229	0.0135740152425659\\
230	0.0135740119537257\\
231	0.013574008605503\\
232	0.0135740051968081\\
233	0.0135740017265304\\
234	0.0135739981935385\\
235	0.0135739945966793\\
236	0.013573990934778\\
237	0.0135739872066375\\
238	0.0135739834110379\\
239	0.0135739795467361\\
240	0.0135739756124653\\
241	0.0135739716069347\\
242	0.0135739675288287\\
243	0.0135739633768068\\
244	0.0135739591495026\\
245	0.0135739548455239\\
246	0.0135739504634516\\
247	0.0135739460018397\\
248	0.0135739414592142\\
249	0.0135739368340729\\
250	0.013573932124885\\
251	0.01357392733009\\
252	0.0135739224480976\\
253	0.0135739174772866\\
254	0.0135739124160049\\
255	0.0135739072625685\\
256	0.0135739020152608\\
257	0.0135738966723321\\
258	0.013573891231999\\
259	0.0135738856924435\\
260	0.0135738800518126\\
261	0.0135738743082173\\
262	0.0135738684597322\\
263	0.0135738625043943\\
264	0.0135738564402028\\
265	0.013573850265118\\
266	0.0135738439770607\\
267	0.0135738375739111\\
268	0.0135738310535084\\
269	0.0135738244136499\\
270	0.0135738176520899\\
271	0.013573810766539\\
272	0.0135738037546634\\
273	0.0135737966140839\\
274	0.0135737893423748\\
275	0.0135737819370633\\
276	0.0135737743956285\\
277	0.0135737667155004\\
278	0.0135737588940588\\
279	0.0135737509286326\\
280	0.0135737428164986\\
281	0.0135737345548809\\
282	0.013573726140949\\
283	0.0135737175718178\\
284	0.0135737088445457\\
285	0.0135736999561343\\
286	0.0135736909035265\\
287	0.0135736816836059\\
288	0.0135736722931957\\
289	0.0135736627290574\\
290	0.0135736529878895\\
291	0.0135736430663268\\
292	0.0135736329609387\\
293	0.0135736226682283\\
294	0.0135736121846311\\
295	0.0135736015065138\\
296	0.0135735906301731\\
297	0.0135735795518343\\
298	0.01357356826765\\
299	0.0135735567736991\\
300	0.0135735450659852\\
301	0.0135735331404354\\
302	0.013573520992899\\
303	0.013573508619146\\
304	0.013573496014866\\
305	0.0135734831756665\\
306	0.013573470097072\\
307	0.0135734567745219\\
308	0.0135734432033698\\
309	0.0135734293788816\\
310	0.0135734152962343\\
311	0.0135734009505145\\
312	0.0135733863367168\\
313	0.0135733714497427\\
314	0.0135733562843987\\
315	0.0135733408353949\\
316	0.0135733250973436\\
317	0.0135733090647578\\
318	0.0135732927320495\\
319	0.0135732760935282\\
320	0.0135732591433993\\
321	0.0135732418757626\\
322	0.0135732242846107\\
323	0.0135732063638268\\
324	0.013573188107184\\
325	0.0135731695083428\\
326	0.0135731505608497\\
327	0.0135731312581352\\
328	0.0135731115935125\\
329	0.0135730915601748\\
330	0.0135730711511945\\
331	0.01357305035952\\
332	0.013573029177975\\
333	0.0135730075992552\\
334	0.0135729856159271\\
335	0.0135729632204255\\
336	0.0135729404050508\\
337	0.0135729171619674\\
338	0.0135728934832007\\
339	0.013572869360635\\
340	0.0135728447860103\\
341	0.0135728197509202\\
342	0.013572794246809\\
343	0.0135727682649682\\
344	0.0135727417965344\\
345	0.0135727148324855\\
346	0.0135726873636376\\
347	0.0135726593806418\\
348	0.0135726308739804\\
349	0.0135726018339635\\
350	0.0135725722507248\\
351	0.0135725421142178\\
352	0.0135725114142113\\
353	0.0135724801402848\\
354	0.013572448281824\\
355	0.0135724158280149\\
356	0.0135723827678391\\
357	0.013572349090067\\
358	0.0135723147832523\\
359	0.0135722798357247\\
360	0.0135722442355825\\
361	0.0135722079706851\\
362	0.0135721710286443\\
363	0.0135721333968153\\
364	0.0135720950622867\\
365	0.0135720560118702\\
366	0.0135720162320888\\
367	0.0135719757091648\\
368	0.0135719344290062\\
369	0.0135718923771922\\
370	0.0135718495389577\\
371	0.0135718058991766\\
372	0.013571761442343\\
373	0.0135717161525518\\
374	0.0135716700134768\\
375	0.0135716230083479\\
376	0.0135715751199254\\
377	0.0135715263304724\\
378	0.0135714766217254\\
379	0.0135714259748613\\
380	0.0135713743704622\\
381	0.0135713217884766\\
382	0.0135712682081768\\
383	0.0135712136081128\\
384	0.0135711579660606\\
385	0.0135711012589663\\
386	0.0135710434628831\\
387	0.0135709845529021\\
388	0.0135709245030741\\
389	0.0135708632863219\\
390	0.0135708008743411\\
391	0.0135707372374862\\
392	0.0135706723446406\\
393	0.0135706061630698\\
394	0.0135705386582601\\
395	0.0135704697937514\\
396	0.0135703995309475\\
397	0.0135703278287667\\
398	0.0135702546435763\\
399	0.0135701799292728\\
400	0.0135701036371131\\
401	0.0135700257155419\\
402	0.0135699461100216\\
403	0.0135698647628673\\
404	0.0135697816130975\\
405	0.0135696965963048\\
406	0.0135696096445462\\
407	0.01356952068623\\
408	0.0135694296459575\\
409	0.0135693364443535\\
410	0.0135692409984055\\
411	0.013569143221091\\
412	0.0135690430187631\\
413	0.0135689402903559\\
414	0.0135688349264708\\
415	0.0135687268083254\\
416	0.0135686158065393\\
417	0.0135685017797287\\
418	0.0135683845728762\\
419	0.0135682640154346\\
420	0.0135681399191123\\
421	0.0135680120752813\\
422	0.0135678802519238\\
423	0.0135677441900068\\
424	0.013567603599107\\
425	0.0135674581519571\\
426	0.0135673074771931\\
427	0.0135671511485479\\
428	0.0135669886659655\\
429	0.0135668194168035\\
430	0.0135666425867245\\
431	0.0135664569461465\\
432	0.0135662603514933\\
433	0.0135660296918102\\
434	0.013565585858322\\
435	0.0135651312574447\\
436	0.0135646655231995\\
437	0.0135641882703674\\
438	0.0135636990932423\\
439	0.0135631975643376\\
440	0.0135626832330595\\
441	0.0135621556243659\\
442	0.0135616142374356\\
443	0.013561058544373\\
444	0.0135604879889636\\
445	0.0135599019854251\\
446	0.0135592999168544\\
447	0.0135586811322268\\
448	0.0135580449379783\\
449	0.0135573905707127\\
450	0.0135567171052215\\
451	0.0135560231400375\\
452	0.0135553057086453\\
453	0.013554556454497\\
454	0.0135531309702458\\
455	0.0135515386588616\\
456	0.0135499110572848\\
457	0.0135482475507965\\
458	0.0135465475425931\\
459	0.0135448104577816\\
460	0.0135430357459824\\
461	0.0135412228801507\\
462	0.0135393713510089\\
463	0.0135374806756365\\
464	0.0135355505238831\\
465	0.0135335813477377\\
466	0.0135315727239963\\
467	0.0135295234964529\\
468	0.0135274324421771\\
469	0.0135252982721042\\
470	0.0135231196230671\\
471	0.0135208950272386\\
472	0.0135186229395503\\
473	0.0135163017378163\\
474	0.0135139297210525\\
475	0.0135115050936955\\
476	0.0135090259046485\\
477	0.0135064902976824\\
478	0.0135038963009287\\
479	0.0135012418163034\\
480	0.0134985246080424\\
481	0.0134957422899304\\
482	0.0134928923110229\\
483	0.0134899719396252\\
484	0.0134869782452648\\
485	0.0134839080783718\\
486	0.013480758047361\\
487	0.0134775244926006\\
488	0.0134742034560052\\
489	0.0134707906447316\\
490	0.0134672813912341\\
491	0.0134636706041583\\
492	0.0134599527119994\\
493	0.0134561215952258\\
494	0.013452170508445\\
495	0.0134480919876111\\
496	0.0134438777324719\\
497	0.0134395184711273\\
498	0.0134350037936513\\
499	0.0134303219397871\\
500	0.013425459505737\\
501	0.0134204009752132\\
502	0.0134151277930801\\
503	0.0134096161006349\\
504	0.0134038302887151\\
505	0.0133977042829194\\
506	0.0133879465419463\\
507	0.0133754544360152\\
508	0.0133626639172835\\
509	0.0133495625463163\\
510	0.0133361337217632\\
511	0.0133223551433638\\
512	0.0133082101002117\\
513	0.0132936804120145\\
514	0.0132787466056319\\
515	0.0132633877357703\\
516	0.0132475810977353\\
517	0.0132313020759999\\
518	0.0132145240381542\\
519	0.0131972184178252\\
520	0.0131793554778035\\
521	0.0131609074507852\\
522	0.0131418600117765\\
523	0.0131234761229606\\
524	0.0131071025687588\\
525	0.0130901586555513\\
526	0.0130726066266726\\
527	0.0130543941694972\\
528	0.0130354622267326\\
529	0.0130157451510546\\
530	0.0129951497867033\\
531	0.0129735007122376\\
532	0.0129404169122399\\
533	0.0129052760869902\\
534	0.0128690784379187\\
535	0.0128317675385575\\
536	0.0127932288138182\\
537	0.0127534524658115\\
538	0.0127177616286965\\
539	0.0126805995499357\\
540	0.0126394834502458\\
541	0.0125970662185484\\
542	0.0125532652522324\\
543	0.0125079873925301\\
544	0.0124611239148045\\
545	0.0124125701118195\\
546	0.0123622577298588\\
547	0.0123101348875378\\
548	0.0122559512929347\\
549	0.0121993873122374\\
550	0.0121443497414123\\
551	0.0120833739284788\\
552	0.0120067729157643\\
553	0.0119277978982168\\
554	0.0118463635751454\\
555	0.0117623795920937\\
556	0.0116755637917591\\
557	0.0115858191339125\\
558	0.0114931515957708\\
559	0.0113973569553004\\
560	0.0112980500590257\\
561	0.0111947190179295\\
562	0.0111125893429125\\
563	0.0110513680167058\\
564	0.0109673962746709\\
565	0.0108807371784593\\
566	0.010790655527241\\
567	0.0106972226228035\\
568	0.0106020507050947\\
569	0.0104721994434009\\
570	0.0102777838501053\\
571	0.0100790432323702\\
572	0.00987605874912886\\
573	0.00966695196358534\\
574	0.00945084046506507\\
575	0.00926112530070261\\
576	0.00915063224355679\\
577	0.0090418074459414\\
578	0.00893567193291057\\
579	0.00883270072371829\\
580	0.00873396835181544\\
581	0.00864355247881464\\
582	0.00855576529386706\\
583	0.00846823376216357\\
584	0.00835103096980105\\
585	0.00821023311266957\\
586	0.0080753363346691\\
587	0.00762857331744418\\
588	0.00717164652229217\\
589	0.00677887491642115\\
590	0.00666490690787525\\
591	0.00655457425094847\\
592	0.00645089505528701\\
593	0.00634160927221849\\
594	0.00621539581768703\\
595	0.00604257880714497\\
596	0.00574702137257038\\
597	0.0051299581010561\\
598	0.00367057462141144\\
599	0\\
600	0\\
};
\addplot [color=mycolor18,solid,forget plot]
  table[row sep=crcr]{%
1	0.0134561822659322\\
2	0.0134561818216159\\
3	0.0134561813693538\\
4	0.013456180909004\\
5	0.0134561804404222\\
6	0.0134561799634613\\
7	0.0134561794779717\\
8	0.0134561789838011\\
9	0.0134561784807946\\
10	0.0134561779687943\\
11	0.0134561774476397\\
12	0.0134561769171674\\
13	0.0134561763772109\\
14	0.013456175827601\\
15	0.0134561752681653\\
16	0.0134561746987284\\
17	0.0134561741191118\\
18	0.0134561735291337\\
19	0.0134561729286093\\
20	0.0134561723173502\\
21	0.0134561716951648\\
22	0.0134561710618581\\
23	0.0134561704172316\\
24	0.0134561697610833\\
25	0.0134561690932074\\
26	0.0134561684133948\\
27	0.0134561677214323\\
28	0.0134561670171032\\
29	0.0134561663001867\\
30	0.0134561655704582\\
31	0.013456164827689\\
32	0.0134561640716465\\
33	0.0134561633020938\\
34	0.0134561625187898\\
35	0.0134561617214891\\
36	0.013456160909942\\
37	0.0134561600838943\\
38	0.0134561592430872\\
39	0.0134561583872575\\
40	0.013456157516137\\
41	0.0134561566294531\\
42	0.0134561557269279\\
43	0.0134561548082789\\
44	0.0134561538732184\\
45	0.0134561529214537\\
46	0.0134561519526868\\
47	0.0134561509666143\\
48	0.0134561499629275\\
49	0.0134561489413122\\
50	0.0134561479014486\\
51	0.0134561468430113\\
52	0.0134561457656688\\
53	0.013456144669084\\
54	0.0134561435529136\\
55	0.0134561424168083\\
56	0.0134561412604126\\
57	0.0134561400833644\\
58	0.0134561388852956\\
59	0.0134561376658311\\
60	0.0134561364245894\\
61	0.013456135161182\\
62	0.0134561338752137\\
63	0.0134561325662821\\
64	0.0134561312339776\\
65	0.0134561298778835\\
66	0.0134561284975754\\
67	0.0134561270926215\\
68	0.0134561256625823\\
69	0.0134561242070105\\
70	0.0134561227254507\\
71	0.0134561212174396\\
72	0.0134561196825053\\
73	0.0134561181201677\\
74	0.0134561165299383\\
75	0.0134561149113196\\
76	0.0134561132638054\\
77	0.0134561115868805\\
78	0.0134561098800202\\
79	0.0134561081426909\\
80	0.0134561063743493\\
81	0.0134561045744422\\
82	0.0134561027424069\\
83	0.0134561008776705\\
84	0.0134560989796497\\
85	0.0134560970477511\\
86	0.0134560950813706\\
87	0.0134560930798933\\
88	0.0134560910426934\\
89	0.0134560889691338\\
90	0.0134560868585662\\
91	0.0134560847103305\\
92	0.0134560825237553\\
93	0.0134560802981566\\
94	0.0134560780328386\\
95	0.0134560757270931\\
96	0.013456073380199\\
97	0.0134560709914226\\
98	0.0134560685600168\\
99	0.0134560660852216\\
100	0.0134560635662631\\
101	0.0134560610023535\\
102	0.0134560583926914\\
103	0.0134560557364606\\
104	0.0134560530328305\\
105	0.0134560502809559\\
106	0.0134560474799761\\
107	0.0134560446290153\\
108	0.0134560417271819\\
109	0.0134560387735686\\
110	0.0134560357672517\\
111	0.0134560327072909\\
112	0.0134560295927294\\
113	0.0134560264225931\\
114	0.0134560231958904\\
115	0.0134560199116123\\
116	0.0134560165687313\\
117	0.013456013166202\\
118	0.0134560097029599\\
119	0.0134560061779218\\
120	0.0134560025899849\\
121	0.0134559989380266\\
122	0.0134559952209046\\
123	0.0134559914374558\\
124	0.0134559875864962\\
125	0.0134559836668211\\
126	0.0134559796772036\\
127	0.0134559756163954\\
128	0.0134559714831254\\
129	0.0134559672761\\
130	0.0134559629940023\\
131	0.0134559586354918\\
132	0.0134559541992041\\
133	0.0134559496837501\\
134	0.013455945087716\\
135	0.0134559404096626\\
136	0.0134559356481248\\
137	0.0134559308016113\\
138	0.0134559258686039\\
139	0.0134559208475574\\
140	0.0134559157368985\\
141	0.013455910535026\\
142	0.0134559052403098\\
143	0.0134558998510904\\
144	0.0134558943656785\\
145	0.0134558887823547\\
146	0.0134558830993683\\
147	0.0134558773149373\\
148	0.0134558714272477\\
149	0.0134558654344528\\
150	0.0134558593346725\\
151	0.0134558531259931\\
152	0.0134558468064663\\
153	0.0134558403741087\\
154	0.0134558338269011\\
155	0.0134558271627881\\
156	0.0134558203796769\\
157	0.0134558134754373\\
158	0.0134558064479004\\
159	0.0134557992948583\\
160	0.0134557920140632\\
161	0.0134557846032265\\
162	0.0134557770600186\\
163	0.0134557693820675\\
164	0.0134557615669585\\
165	0.0134557536122329\\
166	0.0134557455153879\\
167	0.0134557372738751\\
168	0.0134557288851002\\
169	0.0134557203464215\\
170	0.0134557116551498\\
171	0.013455702808547\\
172	0.0134556938038253\\
173	0.0134556846381462\\
174	0.0134556753086198\\
175	0.0134556658123037\\
176	0.013455656146202\\
177	0.0134556463072642\\
178	0.0134556362923846\\
179	0.0134556260984006\\
180	0.0134556157220923\\
181	0.0134556051601809\\
182	0.0134555944093281\\
183	0.0134555834661346\\
184	0.0134555723271389\\
185	0.0134555609888166\\
186	0.0134555494475788\\
187	0.0134555376997708\\
188	0.0134555257416714\\
189	0.0134555135694912\\
190	0.0134555011793712\\
191	0.0134554885673821\\
192	0.0134554757295224\\
193	0.013455462661717\\
194	0.0134554493598163\\
195	0.0134554358195945\\
196	0.0134554220367479\\
197	0.013455408006894\\
198	0.0134553937255694\\
199	0.0134553791882288\\
200	0.0134553643902429\\
201	0.0134553493268971\\
202	0.0134553339933901\\
203	0.0134553183848317\\
204	0.0134553024962414\\
205	0.0134552863225467\\
206	0.0134552698585814\\
207	0.0134552530990834\\
208	0.0134552360386933\\
209	0.0134552186719522\\
210	0.0134552009933001\\
211	0.0134551829970736\\
212	0.0134551646775043\\
213	0.0134551460287164\\
214	0.0134551270447246\\
215	0.0134551077194323\\
216	0.0134550880466294\\
217	0.0134550680199896\\
218	0.0134550476330689\\
219	0.0134550268793026\\
220	0.0134550057520034\\
221	0.0134549842443587\\
222	0.0134549623494284\\
223	0.0134549400601424\\
224	0.0134549173692977\\
225	0.0134548942695562\\
226	0.0134548707534419\\
227	0.013454846813338\\
228	0.0134548224414844\\
229	0.0134547976299749\\
230	0.013454772370754\\
231	0.013454746655614\\
232	0.0134547204761925\\
233	0.0134546938239686\\
234	0.0134546666902603\\
235	0.0134546390662209\\
236	0.0134546109428363\\
237	0.0134545823109209\\
238	0.0134545531611148\\
239	0.0134545234838801\\
240	0.0134544932694973\\
241	0.0134544625080619\\
242	0.0134544311894801\\
243	0.0134543993034661\\
244	0.013454366839537\\
245	0.0134543337870097\\
246	0.0134543001349968\\
247	0.0134542658724021\\
248	0.0134542309879167\\
249	0.0134541954700147\\
250	0.0134541593069487\\
251	0.0134541224867456\\
252	0.0134540849972019\\
253	0.0134540468258789\\
254	0.0134540079600983\\
255	0.0134539683869371\\
256	0.0134539280932227\\
257	0.013453887065528\\
258	0.0134538452901662\\
259	0.0134538027531855\\
260	0.0134537594403638\\
261	0.0134537153372034\\
262	0.0134536704289249\\
263	0.0134536247004623\\
264	0.0134535781364565\\
265	0.01345353072125\\
266	0.0134534824388804\\
267	0.0134534332730744\\
268	0.0134533832072417\\
269	0.0134533322244685\\
270	0.0134532803075109\\
271	0.0134532274387883\\
272	0.013453173600377\\
273	0.0134531187740026\\
274	0.0134530629410338\\
275	0.0134530060824745\\
276	0.0134529481789571\\
277	0.0134528892107348\\
278	0.0134528291576741\\
279	0.013452767999247\\
280	0.0134527057145233\\
281	0.0134526422821629\\
282	0.0134525776804071\\
283	0.0134525118870709\\
284	0.0134524448795342\\
285	0.0134523766347336\\
286	0.0134523071291536\\
287	0.0134522363388178\\
288	0.0134521642392799\\
289	0.0134520908056147\\
290	0.0134520160124087\\
291	0.013451939833751\\
292	0.0134518622432235\\
293	0.0134517832138913\\
294	0.0134517027182931\\
295	0.0134516207284309\\
296	0.0134515372157603\\
297	0.0134514521511798\\
298	0.0134513655050211\\
299	0.0134512772470379\\
300	0.0134511873463957\\
301	0.013451095771661\\
302	0.0134510024907901\\
303	0.0134509074711185\\
304	0.0134508106793495\\
305	0.0134507120815428\\
306	0.0134506116431034\\
307	0.0134505093287697\\
308	0.0134504051026023\\
309	0.0134502989279716\\
310	0.0134501907675466\\
311	0.0134500805832822\\
312	0.0134499683364077\\
313	0.0134498539874139\\
314	0.0134497374960412\\
315	0.0134496188212669\\
316	0.0134494979212926\\
317	0.0134493747535314\\
318	0.0134492492745948\\
319	0.0134491214402804\\
320	0.0134489912055577\\
321	0.0134488585245557\\
322	0.0134487233505486\\
323	0.0134485856359428\\
324	0.0134484453322628\\
325	0.0134483023901368\\
326	0.0134481567592831\\
327	0.0134480083884949\\
328	0.0134478572256259\\
329	0.0134477032175752\\
330	0.0134475463102717\\
331	0.0134473864486583\\
332	0.013447223576676\\
333	0.0134470576372473\\
334	0.0134468885722589\\
335	0.0134467163225443\\
336	0.0134465408278658\\
337	0.0134463620268957\\
338	0.0134461798571971\\
339	0.0134459942552039\\
340	0.0134458051562003\\
341	0.0134456124942993\\
342	0.0134454162024208\\
343	0.0134452162122685\\
344	0.0134450124543068\\
345	0.0134448048577359\\
346	0.0134445933504669\\
347	0.0134443778590958\\
348	0.0134441583088762\\
349	0.0134439346236921\\
350	0.0134437067260281\\
351	0.01344347453694\\
352	0.0134432379760234\\
353	0.0134429969613803\\
354	0.0134427514095857\\
355	0.0134425012356502\\
356	0.0134422463529823\\
357	0.0134419866733473\\
358	0.0134417221068238\\
359	0.0134414525617582\\
360	0.013441177944715\\
361	0.0134408981604247\\
362	0.0134406131117277\\
363	0.0134403226995144\\
364	0.0134400268226607\\
365	0.0134397253779595\\
366	0.0134394182600462\\
367	0.0134391053613194\\
368	0.013438786571855\\
369	0.0134384617793145\\
370	0.0134381308688452\\
371	0.0134377937229734\\
372	0.0134374502214887\\
373	0.0134371002413198\\
374	0.0134367436563995\\
375	0.0134363803375195\\
376	0.0134360101521733\\
377	0.013435632964386\\
378	0.0134352486345302\\
379	0.013434857019127\\
380	0.0134344579706292\\
381	0.0134340513371866\\
382	0.0134336369623904\\
383	0.0134332146849954\\
384	0.0134327843386154\\
385	0.0134323457513912\\
386	0.0134318987456256\\
387	0.0134314431373806\\
388	0.0134309787360328\\
389	0.013430505343777\\
390	0.0134300227550688\\
391	0.0134295307559893\\
392	0.0134290291235104\\
393	0.0134285176246247\\
394	0.0134279960153166\\
395	0.0134274640394291\\
396	0.0134269214278475\\
397	0.0134263678994751\\
398	0.0134258031545394\\
399	0.013425226871955\\
400	0.013424638713397\\
401	0.0134240383221035\\
402	0.0134234253216468\\
403	0.0134227993147096\\
404	0.0134221598819214\\
405	0.0134215065808535\\
406	0.0134208389453062\\
407	0.0134201564849819\\
408	0.0134194586852379\\
409	0.0134187450050662\\
410	0.0134180148673948\\
411	0.0134172676718487\\
412	0.0134165028246402\\
413	0.0134157196929958\\
414	0.0134149176008566\\
415	0.0134140958239427\\
416	0.0134132535840813\\
417	0.0134123900426708\\
418	0.0134115042931249\\
419	0.0134105953521124\\
420	0.0134096621493659\\
421	0.0134087035157896\\
422	0.0134077181695276\\
423	0.013406704699578\\
424	0.0134056615464017\\
425	0.0134045869787614\\
426	0.0134034790655271\\
427	0.0134023356398417\\
428	0.0134011542489361\\
429	0.0133999320695075\\
430	0.0133986657233074\\
431	0.0133973507706399\\
432	0.0133959801050795\\
433	0.0133944636519077\\
434	0.0133920903990077\\
435	0.0133896605293285\\
436	0.0133871722050095\\
437	0.0133846234966891\\
438	0.0133820123775732\\
439	0.0133793367172082\\
440	0.0133765942749932\\
441	0.0133737826934872\\
442	0.0133708994915988\\
443	0.0133679420577969\\
444	0.0133649076435679\\
445	0.0133617933575372\\
446	0.0133585961611518\\
447	0.0133553128681827\\
448	0.0133519401544736\\
449	0.0133484745976945\\
450	0.0133449128108547\\
451	0.0133412518818472\\
452	0.013337490843202\\
453	0.0133336356947361\\
454	0.0133305044376716\\
455	0.0133274581932079\\
456	0.013324328432145\\
457	0.0133211121291874\\
458	0.0133178061982601\\
459	0.0133144075078554\\
460	0.0133109128800677\\
461	0.0133073190286252\\
462	0.0133036223198203\\
463	0.0132998180546342\\
464	0.0132958986183272\\
465	0.013291350788174\\
466	0.0132864159568648\\
467	0.0132813790580278\\
468	0.01327623735875\\
469	0.0132709879209172\\
470	0.0132656278679858\\
471	0.0132601547667313\\
472	0.0132545653409825\\
473	0.0132488561999474\\
474	0.0132430239161477\\
475	0.0132370651843567\\
476	0.0132309770125634\\
477	0.0132247504722809\\
478	0.0132183814955756\\
479	0.0132118658217303\\
480	0.0132051989737917\\
481	0.0131983762391065\\
482	0.013191392647137\\
483	0.0131842429438813\\
484	0.013176921562027\\
485	0.0131694225859451\\
486	0.0131617397113751\\
487	0.0131538662021061\\
488	0.0131457948484457\\
489	0.01313751790961\\
490	0.0131290269927436\\
491	0.0131203131150731\\
492	0.013111366592337\\
493	0.0131021769818272\\
494	0.0130927329210816\\
495	0.0130830220613133\\
496	0.0130730311174757\\
497	0.0130627455236987\\
498	0.013052149236027\\
499	0.0130412244979829\\
500	0.013029951567043\\
501	0.0130183084184418\\
502	0.0130062705173229\\
503	0.0129938110360469\\
504	0.0129809029682741\\
505	0.0129675397169172\\
506	0.0129570570296516\\
507	0.0129407381016821\\
508	0.0129240360578046\\
509	0.0129069501953385\\
510	0.0128894706158807\\
511	0.0128715498320805\\
512	0.0128531746672378\\
513	0.0128343337158503\\
514	0.0128150036452325\\
515	0.0127951630904035\\
516	0.0127747953998423\\
517	0.0127538835191894\\
518	0.0127324100281054\\
519	0.0127103572073009\\
520	0.0126877071249932\\
521	0.0126644416485809\\
522	0.0126405416028984\\
523	0.012615954652976\\
524	0.0125905834978737\\
525	0.0125637948285747\\
526	0.0125313743910036\\
527	0.0124981032987448\\
528	0.0124639489826938\\
529	0.0124289175463783\\
530	0.0123930309866679\\
531	0.0123563020828988\\
532	0.0123319006649354\\
533	0.0123080205162911\\
534	0.0122832992946603\\
535	0.0122576544907124\\
536	0.0122309860425432\\
537	0.0122031694844117\\
538	0.0121738349436389\\
539	0.0121263582211654\\
540	0.0120664541393345\\
541	0.0120048499405961\\
542	0.0119414845967801\\
543	0.0118762958203566\\
544	0.0118092209176181\\
545	0.0117401817905396\\
546	0.0116691100229229\\
547	0.0115960169466127\\
548	0.0115209745984561\\
549	0.0114437511641843\\
550	0.0113642635933633\\
551	0.0113081491469601\\
552	0.0112672593220329\\
553	0.0112246829503894\\
554	0.0111802869992823\\
555	0.0111339864737577\\
556	0.0110856940865832\\
557	0.0110352304885566\\
558	0.0109704696552243\\
559	0.0108935488940699\\
560	0.0108142526902001\\
561	0.0107324097570252\\
562	0.010648265430956\\
563	0.010553146137476\\
564	0.0103708300584357\\
565	0.0101835860386593\\
566	0.00999107795264215\\
567	0.00979300937704845\\
568	0.00958906268226325\\
569	0.00941818431876402\\
570	0.00932179459357979\\
571	0.00922733619922007\\
572	0.00913661137543609\\
573	0.00905066532319974\\
574	0.00897105299605528\\
575	0.00889698503277137\\
576	0.00882324482386913\\
577	0.00875022682640823\\
578	0.00867788261042622\\
579	0.00860728333174988\\
580	0.0085378672241195\\
581	0.00846680204803803\\
582	0.00834866481965409\\
583	0.00822246407517227\\
584	0.00794063544024274\\
585	0.00748623886716219\\
586	0.00701867506221063\\
587	0.00691290225686209\\
588	0.00681549064247339\\
589	0.00672507127560368\\
590	0.00663371654684208\\
591	0.00654064038451895\\
592	0.00644381571862552\\
593	0.00633887846033801\\
594	0.00621472953405218\\
595	0.00604257880714497\\
596	0.00574702137257038\\
597	0.0051299581010561\\
598	0.00367057462141144\\
599	0\\
600	0\\
};
\addplot [color=red!25!mycolor17,solid,forget plot]
  table[row sep=crcr]{%
1	0.0132415262699758\\
2	0.0132415251533705\\
3	0.0132415240167911\\
4	0.013241522859881\\
5	0.013241521682277\\
6	0.0132415204836094\\
7	0.0132415192635019\\
8	0.0132415180215717\\
9	0.0132415167574289\\
10	0.0132415154706768\\
11	0.0132415141609116\\
12	0.0132415128277223\\
13	0.0132415114706905\\
14	0.0132415100893905\\
15	0.0132415086833888\\
16	0.0132415072522443\\
17	0.0132415057955081\\
18	0.0132415043127231\\
19	0.0132415028034242\\
20	0.013241501267138\\
21	0.0132414997033826\\
22	0.0132414981116675\\
23	0.0132414964914934\\
24	0.0132414948423524\\
25	0.0132414931637272\\
26	0.0132414914550913\\
27	0.0132414897159091\\
28	0.0132414879456352\\
29	0.0132414861437146\\
30	0.0132414843095822\\
31	0.0132414824426631\\
32	0.0132414805423719\\
33	0.0132414786081131\\
34	0.0132414766392802\\
35	0.0132414746352561\\
36	0.0132414725954128\\
37	0.0132414705191108\\
38	0.0132414684056994\\
39	0.0132414662545163\\
40	0.0132414640648873\\
41	0.0132414618361264\\
42	0.0132414595675351\\
43	0.0132414572584026\\
44	0.0132414549080054\\
45	0.0132414525156071\\
46	0.0132414500804582\\
47	0.0132414476017958\\
48	0.0132414450788434\\
49	0.0132414425108107\\
50	0.0132414398968932\\
51	0.0132414372362724\\
52	0.0132414345281148\\
53	0.0132414317715722\\
54	0.0132414289657813\\
55	0.0132414261098634\\
56	0.0132414232029243\\
57	0.0132414202440535\\
58	0.0132414172323246\\
59	0.0132414141667945\\
60	0.0132414110465034\\
61	0.0132414078704743\\
62	0.0132414046377129\\
63	0.0132414013472071\\
64	0.0132413979979268\\
65	0.0132413945888234\\
66	0.0132413911188298\\
67	0.0132413875868599\\
68	0.013241383991808\\
69	0.0132413803325488\\
70	0.0132413766079371\\
71	0.0132413728168072\\
72	0.0132413689579724\\
73	0.0132413650302251\\
74	0.0132413610323362\\
75	0.0132413569630545\\
76	0.0132413528211066\\
77	0.0132413486051963\\
78	0.0132413443140045\\
79	0.0132413399461885\\
80	0.0132413355003814\\
81	0.0132413309751923\\
82	0.0132413263692053\\
83	0.0132413216809793\\
84	0.0132413169090475\\
85	0.0132413120519169\\
86	0.0132413071080681\\
87	0.0132413020759542\\
88	0.0132412969540011\\
89	0.0132412917406063\\
90	0.0132412864341391\\
91	0.0132412810329393\\
92	0.0132412755353173\\
93	0.0132412699395535\\
94	0.0132412642438973\\
95	0.0132412584465672\\
96	0.0132412525457496\\
97	0.0132412465395987\\
98	0.0132412404262358\\
99	0.0132412342037487\\
100	0.013241227870191\\
101	0.0132412214235814\\
102	0.0132412148619036\\
103	0.0132412081831051\\
104	0.0132412013850967\\
105	0.013241194465752\\
106	0.0132411874229067\\
107	0.0132411802543578\\
108	0.0132411729578628\\
109	0.0132411655311394\\
110	0.0132411579718645\\
111	0.0132411502776733\\
112	0.013241142446159\\
113	0.0132411344748716\\
114	0.0132411263613176\\
115	0.0132411181029588\\
116	0.0132411096972114\\
117	0.0132411011414458\\
118	0.0132410924329853\\
119	0.0132410835691052\\
120	0.0132410745470322\\
121	0.0132410653639434\\
122	0.0132410560169654\\
123	0.0132410465031736\\
124	0.0132410368195906\\
125	0.0132410269631864\\
126	0.0132410169308761\\
127	0.0132410067195202\\
128	0.0132409963259227\\
129	0.0132409857468303\\
130	0.0132409749789317\\
131	0.0132409640188562\\
132	0.0132409528631727\\
133	0.0132409415083887\\
134	0.0132409299509492\\
135	0.0132409181872353\\
136	0.0132409062135634\\
137	0.0132408940261838\\
138	0.0132408816212798\\
139	0.0132408689949659\\
140	0.0132408561432871\\
141	0.0132408430622176\\
142	0.0132408297476591\\
143	0.0132408161954398\\
144	0.0132408024013133\\
145	0.0132407883609566\\
146	0.0132407740699692\\
147	0.0132407595238717\\
148	0.0132407447181041\\
149	0.0132407296480244\\
150	0.0132407143089072\\
151	0.0132406986959422\\
152	0.0132406828042326\\
153	0.0132406666287935\\
154	0.0132406501645501\\
155	0.0132406334063366\\
156	0.013240616348894\\
157	0.0132405989868686\\
158	0.0132405813148103\\
159	0.0132405633271708\\
160	0.0132405450183018\\
161	0.013240526382453\\
162	0.0132405074137706\\
163	0.0132404881062951\\
164	0.0132404684539595\\
165	0.0132404484505871\\
166	0.0132404280898899\\
167	0.0132404073654661\\
168	0.0132403862707982\\
169	0.013240364799251\\
170	0.0132403429440692\\
171	0.0132403206983753\\
172	0.0132402980551673\\
173	0.0132402750073164\\
174	0.0132402515475647\\
175	0.0132402276685227\\
176	0.0132402033626671\\
177	0.013240178622338\\
178	0.0132401534397365\\
179	0.0132401278069223\\
180	0.0132401017158106\\
181	0.0132400751581699\\
182	0.0132400481256192\\
183	0.0132400206096247\\
184	0.0132399926014977\\
185	0.013239964092391\\
186	0.0132399350732966\\
187	0.0132399055350422\\
188	0.013239875468288\\
189	0.0132398448635244\\
190	0.0132398137110677\\
191	0.0132397820010577\\
192	0.0132397497234538\\
193	0.0132397168680321\\
194	0.0132396834243815\\
195	0.0132396493819003\\
196	0.0132396147297929\\
197	0.0132395794570657\\
198	0.0132395435525235\\
199	0.0132395070047658\\
200	0.0132394698021827\\
201	0.0132394319329511\\
202	0.0132393933850307\\
203	0.0132393541461594\\
204	0.0132393142038497\\
205	0.0132392735453839\\
206	0.0132392321578101\\
207	0.0132391900279373\\
208	0.0132391471423312\\
209	0.0132391034873092\\
210	0.0132390590489359\\
211	0.0132390138130182\\
212	0.0132389677651001\\
213	0.0132389208904578\\
214	0.0132388731740947\\
215	0.0132388246007358\\
216	0.0132387751548225\\
217	0.0132387248205071\\
218	0.013238673581647\\
219	0.0132386214217994\\
220	0.0132385683242149\\
221	0.0132385142718319\\
222	0.0132384592472705\\
223	0.013238403232826\\
224	0.013238346210463\\
225	0.0132382881618084\\
226	0.0132382290681452\\
227	0.0132381689104054\\
228	0.0132381076691634\\
229	0.0132380453246289\\
230	0.0132379818566394\\
231	0.0132379172446532\\
232	0.0132378514677418\\
233	0.0132377845045821\\
234	0.0132377163334487\\
235	0.013237646932206\\
236	0.0132375762782997\\
237	0.0132375043487489\\
238	0.0132374311201372\\
239	0.0132373565686043\\
240	0.0132372806698368\\
241	0.0132372033990594\\
242	0.0132371247310255\\
243	0.0132370446400079\\
244	0.0132369630997888\\
245	0.0132368800836501\\
246	0.0132367955643635\\
247	0.0132367095141798\\
248	0.0132366219048186\\
249	0.0132365327074575\\
250	0.0132364418927213\\
251	0.0132363494306702\\
252	0.013236255290789\\
253	0.0132361594419749\\
254	0.0132360618525257\\
255	0.0132359624901277\\
256	0.0132358613218429\\
257	0.0132357583140964\\
258	0.0132356534326633\\
259	0.0132355466426553\\
260	0.0132354379085075\\
261	0.0132353271939638\\
262	0.0132352144620631\\
263	0.0132350996751248\\
264	0.0132349827947339\\
265	0.0132348637817255\\
266	0.0132347425961699\\
267	0.0132346191973565\\
268	0.0132344935437774\\
269	0.0132343655931113\\
270	0.0132342353022065\\
271	0.0132341026270634\\
272	0.0132339675228174\\
273	0.0132338299437205\\
274	0.0132336898431233\\
275	0.0132335471734559\\
276	0.0132334018862092\\
277	0.0132332539319148\\
278	0.0132331032601256\\
279	0.0132329498193953\\
280	0.0132327935572574\\
281	0.0132326344202041\\
282	0.0132324723536648\\
283	0.0132323073019836\\
284	0.0132321392083974\\
285	0.0132319680150123\\
286	0.0132317936627803\\
287	0.0132316160914755\\
288	0.0132314352396695\\
289	0.0132312510447069\\
290	0.0132310634426792\\
291	0.0132308723683997\\
292	0.0132306777553765\\
293	0.0132304795357861\\
294	0.0132302776404457\\
295	0.0132300719987855\\
296	0.0132298625388203\\
297	0.0132296491871205\\
298	0.0132294318687825\\
299	0.0132292105073993\\
300	0.0132289850250294\\
301	0.0132287553421661\\
302	0.0132285213777063\\
303	0.0132282830489177\\
304	0.013228040271407\\
305	0.0132277929590863\\
306	0.01322754102414\\
307	0.0132272843769902\\
308	0.0132270229262625\\
309	0.0132267565787506\\
310	0.0132264852393812\\
311	0.013226208811177\\
312	0.0132259271952214\\
313	0.0132256402906201\\
314	0.0132253479944647\\
315	0.0132250502017942\\
316	0.0132247468055566\\
317	0.0132244376965701\\
318	0.013224122763484\\
319	0.0132238018927385\\
320	0.0132234749685251\\
321	0.0132231418727457\\
322	0.0132228024849717\\
323	0.0132224566824025\\
324	0.0132221043398238\\
325	0.0132217453295655\\
326	0.0132213795214587\\
327	0.0132210067827927\\
328	0.0132206269782713\\
329	0.0132202399699687\\
330	0.0132198456172849\\
331	0.0132194437769003\\
332	0.0132190343027295\\
333	0.013218617045875\\
334	0.0132181918545792\\
335	0.0132177585741762\\
336	0.0132173170470419\\
337	0.0132168671125437\\
338	0.0132164086069879\\
339	0.0132159413635665\\
340	0.0132154652123019\\
341	0.0132149799799896\\
342	0.0132144854901395\\
343	0.0132139815629144\\
344	0.0132134680150674\\
345	0.0132129446598759\\
346	0.0132124113070751\\
347	0.0132118677627892\\
348	0.0132113138294605\\
349	0.0132107493057777\\
350	0.0132101739866024\\
351	0.0132095876628947\\
352	0.0132089901216384\\
353	0.0132083811457651\\
354	0.0132077605140792\\
355	0.0132071280011815\\
356	0.0132064833773925\\
357	0.0132058264086726\\
358	0.0132051568565399\\
359	0.0132044744779866\\
360	0.013203779025391\\
361	0.013203070246428\\
362	0.0132023478839752\\
363	0.0132016116760163\\
364	0.0132008613555399\\
365	0.0132000966504343\\
366	0.0131993172833777\\
367	0.0131985229717232\\
368	0.0131977134273782\\
369	0.0131968883566777\\
370	0.0131960474602515\\
371	0.0131951904328834\\
372	0.0131943169633634\\
373	0.0131934267343314\\
374	0.0131925194221113\\
375	0.0131915946965365\\
376	0.0131906522207639\\
377	0.0131896916510776\\
378	0.0131887126366802\\
379	0.0131877148194721\\
380	0.0131866978338174\\
381	0.0131856613062957\\
382	0.01318460485544\\
383	0.0131835280914591\\
384	0.0131824306159447\\
385	0.013181312021563\\
386	0.0131801718917299\\
387	0.0131790098002692\\
388	0.0131778253110535\\
389	0.0131766179776225\\
390	0.0131753873427703\\
391	0.0131741329380676\\
392	0.0131728542832394\\
393	0.0131715508851725\\
394	0.0131702222359954\\
395	0.0131688678089182\\
396	0.0131674870492313\\
397	0.0131660793574355\\
398	0.0131646442566986\\
399	0.013163181267577\\
400	0.013161689758668\\
401	0.0131601690740179\\
402	0.0131586185313543\\
403	0.0131570374202713\\
404	0.0131554250005576\\
405	0.0131537805011747\\
406	0.0131521031211144\\
407	0.0131503920349413\\
408	0.0131486464087313\\
409	0.0131468654354145\\
410	0.0131450483918036\\
411	0.0131431940429545\\
412	0.0131413004633047\\
413	0.0131393667585046\\
414	0.0131373920202426\\
415	0.0131353753282662\\
416	0.0131333157523212\\
417	0.0131312123545382\\
418	0.013129064192344\\
419	0.0131268703219884\\
420	0.0131246298028038\\
421	0.0131223417023092\\
422	0.0131200051023475\\
423	0.0131176191063866\\
424	0.0131151828484111\\
425	0.0131126955036803\\
426	0.0131101563022144\\
427	0.013107564547345\\
428	0.0131049196466035\\
429	0.0131022211791484\\
430	0.0130994690827802\\
431	0.013096664250864\\
432	0.0130938105683327\\
433	0.0130910178774083\\
434	0.013089193712143\\
435	0.01308733275998\\
436	0.0130854344015374\\
437	0.013083498022295\\
438	0.0130815230133461\\
439	0.013079508771824\\
440	0.0130774547008407\\
441	0.0130753602087279\\
442	0.0130732247072978\\
443	0.013071047608764\\
444	0.0130688283208625\\
445	0.0130665662396346\\
446	0.0130642607393224\\
447	0.0130619111590918\\
448	0.0130595167872844\\
449	0.0130570768465023\\
450	0.0130545904877712\\
451	0.0130520568050717\\
452	0.0130494748379425\\
453	0.0130468431830869\\
454	0.0130441428595212\\
455	0.0130413660450392\\
456	0.0130385073400341\\
457	0.0130355604497909\\
458	0.013032517983756\\
459	0.0130293711955058\\
460	0.0130261096246619\\
461	0.0130227205239324\\
462	0.0130191876399106\\
463	0.0130154883663957\\
464	0.0130115849059912\\
465	0.0130054352098086\\
466	0.0129979931854274\\
467	0.0129903921644399\\
468	0.0129826280291561\\
469	0.0129746960169894\\
470	0.0129665921457124\\
471	0.0129583160340487\\
472	0.0129498638325548\\
473	0.01294123186774\\
474	0.0129324170083104\\
475	0.0129234178456119\\
476	0.0129142379027734\\
477	0.0129048471550044\\
478	0.0128952405002771\\
479	0.0128854128115593\\
480	0.0128753589184546\\
481	0.0128650736218173\\
482	0.0128545517102967\\
483	0.0128437879777004\\
484	0.0128327772380637\\
485	0.0128215143314799\\
486	0.0128099941090638\\
487	0.0127982113917951\\
488	0.0127861609734407\\
489	0.0127738381297506\\
490	0.0127612393771385\\
491	0.0127483576272744\\
492	0.0127351873813235\\
493	0.0127217224325908\\
494	0.0127079579749715\\
495	0.0126938860840931\\
496	0.012679493393753\\
497	0.0126647739832218\\
498	0.0126497226348489\\
499	0.0126343350995647\\
500	0.012618608458065\\
501	0.012602541653393\\
502	0.0125861363883721\\
503	0.0125693989099062\\
504	0.0125523440985248\\
505	0.0125350051060213\\
506	0.0125180848221119\\
507	0.0125089710820178\\
508	0.0124995252002239\\
509	0.0124894433542382\\
510	0.0124752829349699\\
511	0.01246080610946\\
512	0.0124460057344027\\
513	0.0124308876487257\\
514	0.0124153948404277\\
515	0.0123994759881515\\
516	0.0123831014017383\\
517	0.0123662367976273\\
518	0.0123488422592002\\
519	0.0123308709516625\\
520	0.0123122679440976\\
521	0.0122929679109663\\
522	0.0122728913588624\\
523	0.0122519378767479\\
524	0.0122299670373399\\
525	0.0122045054387753\\
526	0.0121590006316697\\
527	0.0121125078323892\\
528	0.012065024976333\\
529	0.0120165374372497\\
530	0.011967043124584\\
531	0.011916556063086\\
532	0.0118647041909667\\
533	0.0118113961845816\\
534	0.0117565693270968\\
535	0.0117001575218594\\
536	0.011642101753244\\
537	0.0115824294338461\\
538	0.0115211836958824\\
539	0.0114793811479352\\
540	0.0114502313060085\\
541	0.0114202030625028\\
542	0.0113892817640803\\
543	0.0113574556823843\\
544	0.0113247194433691\\
545	0.0112909221186956\\
546	0.0112561032077554\\
547	0.011220300534579\\
548	0.011183567530313\\
549	0.011145958747388\\
550	0.0111075448660224\\
551	0.0110673825392227\\
552	0.0110245748736544\\
553	0.0109619295452036\\
554	0.0108911368784767\\
555	0.0108167203253289\\
556	0.0107382497142737\\
557	0.0106560108701348\\
558	0.0105224456115407\\
559	0.010344136337633\\
560	0.0101611647601714\\
561	0.00997317843492067\\
562	0.00977987499177368\\
563	0.009591092903384\\
564	0.0095049193983546\\
565	0.00942090995560901\\
566	0.00934017128376091\\
567	0.00926433826453039\\
568	0.00919533827758744\\
569	0.00913280724376426\\
570	0.00907102639874058\\
571	0.00900862339496446\\
572	0.00894649810592607\\
573	0.00888527034687089\\
574	0.00882454238005712\\
575	0.00876439434152953\\
576	0.00870519652559366\\
577	0.00864443692047735\\
578	0.00858133840500743\\
579	0.0085191396279465\\
580	0.00840466935341735\\
581	0.00828719617608019\\
582	0.00790817058368755\\
583	0.00744209006934761\\
584	0.00715566805048741\\
585	0.00706286052921099\\
586	0.00697869861467483\\
587	0.00689384575807169\\
588	0.00680801778338822\\
589	0.00672074193547112\\
590	0.00663152074497564\\
591	0.00653964628779212\\
592	0.00644347343018583\\
593	0.00633880868635237\\
594	0.00621472953405218\\
595	0.00604257880714497\\
596	0.00574702137257038\\
597	0.0051299581010561\\
598	0.00367057462141144\\
599	0\\
600	0\\
};
\addplot [color=mycolor19,solid,forget plot]
  table[row sep=crcr]{%
1	0.0130516765041814\\
2	0.0130516748158668\\
3	0.0130516730973378\\
4	0.0130516713480545\\
5	0.0130516695674671\\
6	0.0130516677550162\\
7	0.0130516659101325\\
8	0.0130516640322362\\
9	0.0130516621207375\\
10	0.0130516601750358\\
11	0.01305165819452\\
12	0.013051656178568\\
13	0.0130516541265465\\
14	0.0130516520378112\\
15	0.013051649911706\\
16	0.0130516477475632\\
17	0.0130516455447032\\
18	0.0130516433024343\\
19	0.0130516410200524\\
20	0.0130516386968409\\
21	0.0130516363320704\\
22	0.0130516339249985\\
23	0.0130516314748694\\
24	0.0130516289809141\\
25	0.0130516264423496\\
26	0.0130516238583792\\
27	0.0130516212281918\\
28	0.0130516185509618\\
29	0.0130516158258491\\
30	0.0130516130519985\\
31	0.0130516102285394\\
32	0.0130516073545858\\
33	0.0130516044292361\\
34	0.0130516014515722\\
35	0.01305159842066\\
36	0.0130515953355485\\
37	0.0130515921952698\\
38	0.013051588998839\\
39	0.0130515857452531\\
40	0.0130515824334917\\
41	0.0130515790625161\\
42	0.0130515756312688\\
43	0.0130515721386738\\
44	0.0130515685836358\\
45	0.0130515649650398\\
46	0.0130515612817512\\
47	0.0130515575326149\\
48	0.0130515537164554\\
49	0.0130515498320762\\
50	0.0130515458782594\\
51	0.0130515418537654\\
52	0.0130515377573327\\
53	0.0130515335876769\\
54	0.0130515293434911\\
55	0.0130515250234449\\
56	0.0130515206261843\\
57	0.013051516150331\\
58	0.0130515115944822\\
59	0.0130515069572101\\
60	0.0130515022370616\\
61	0.0130514974325574\\
62	0.0130514925421921\\
63	0.0130514875644333\\
64	0.0130514824977213\\
65	0.0130514773404687\\
66	0.0130514720910596\\
67	0.0130514667478496\\
68	0.0130514613091647\\
69	0.0130514557733012\\
70	0.013051450138525\\
71	0.0130514444030711\\
72	0.0130514385651431\\
73	0.0130514326229123\\
74	0.0130514265745179\\
75	0.0130514204180654\\
76	0.0130514141516268\\
77	0.0130514077732399\\
78	0.0130514012809071\\
79	0.0130513946725955\\
80	0.0130513879462359\\
81	0.0130513810997222\\
82	0.0130513741309106\\
83	0.0130513670376194\\
84	0.0130513598176277\\
85	0.0130513524686753\\
86	0.0130513449884614\\
87	0.0130513373746444\\
88	0.0130513296248409\\
89	0.0130513217366249\\
90	0.0130513137075273\\
91	0.013051305535035\\
92	0.0130512972165898\\
93	0.0130512887495884\\
94	0.0130512801313805\\
95	0.0130512713592689\\
96	0.0130512624305084\\
97	0.0130512533423044\\
98	0.013051244091813\\
99	0.0130512346761392\\
100	0.0130512250923365\\
101	0.0130512153374059\\
102	0.0130512054082949\\
103	0.0130511953018964\\
104	0.0130511850150481\\
105	0.0130511745445312\\
106	0.0130511638870695\\
107	0.0130511530393283\\
108	0.0130511419979135\\
109	0.0130511307593704\\
110	0.0130511193201826\\
111	0.0130511076767711\\
112	0.013051095825493\\
113	0.0130510837626403\\
114	0.0130510714844389\\
115	0.0130510589870474\\
116	0.0130510462665557\\
117	0.013051033318984\\
118	0.0130510201402816\\
119	0.0130510067263252\\
120	0.0130509930729181\\
121	0.0130509791757886\\
122	0.0130509650305888\\
123	0.0130509506328929\\
124	0.0130509359781965\\
125	0.0130509210619144\\
126	0.0130509058793797\\
127	0.0130508904258419\\
128	0.0130508746964659\\
129	0.0130508586863302\\
130	0.0130508423904251\\
131	0.0130508258036516\\
132	0.0130508089208196\\
133	0.0130507917366461\\
134	0.0130507742457536\\
135	0.0130507564426688\\
136	0.0130507383218202\\
137	0.0130507198775367\\
138	0.013050701104046\\
139	0.0130506819954724\\
140	0.013050662545835\\
141	0.0130506427490461\\
142	0.0130506225989089\\
143	0.0130506020891158\\
144	0.0130505812132463\\
145	0.0130505599647647\\
146	0.0130505383370188\\
147	0.0130505163232368\\
148	0.0130504939165257\\
149	0.0130504711098693\\
150	0.0130504478961254\\
151	0.013050424268024\\
152	0.0130504002181645\\
153	0.0130503757390141\\
154	0.0130503508229045\\
155	0.0130503254620302\\
156	0.0130502996484454\\
157	0.013050273374062\\
158	0.0130502466306466\\
159	0.0130502194098182\\
160	0.0130501917030451\\
161	0.0130501635016426\\
162	0.01305013479677\\
163	0.0130501055794277\\
164	0.0130500758404543\\
165	0.0130500455705241\\
166	0.0130500147601433\\
167	0.0130499833996476\\
168	0.0130499514791989\\
169	0.013049918988782\\
170	0.0130498859182015\\
171	0.0130498522570785\\
172	0.013049817994847\\
173	0.0130497831207509\\
174	0.0130497476238403\\
175	0.0130497114929677\\
176	0.0130496747167849\\
177	0.0130496372837389\\
178	0.0130495991820682\\
179	0.0130495603997991\\
180	0.0130495209247419\\
181	0.0130494807444865\\
182	0.0130494398463986\\
183	0.0130493982176159\\
184	0.0130493558450431\\
185	0.0130493127153484\\
186	0.0130492688149586\\
187	0.0130492241300549\\
188	0.0130491786465682\\
189	0.0130491323501746\\
190	0.0130490852262906\\
191	0.0130490372600683\\
192	0.0130489884363904\\
193	0.0130489387398653\\
194	0.013048888154822\\
195	0.0130488366653049\\
196	0.0130487842550684\\
197	0.0130487309075715\\
198	0.0130486766059724\\
199	0.0130486213331227\\
200	0.0130485650715619\\
201	0.0130485078035113\\
202	0.0130484495108681\\
203	0.0130483901751996\\
204	0.0130483297777364\\
205	0.0130482682993668\\
206	0.0130482057206296\\
207	0.0130481420217081\\
208	0.0130480771824232\\
209	0.0130480111822264\\
210	0.0130479440001928\\
211	0.0130478756150145\\
212	0.0130478060049924\\
213	0.0130477351480298\\
214	0.0130476630216239\\
215	0.0130475896028589\\
216	0.0130475148683973\\
217	0.0130474387944725\\
218	0.0130473613568805\\
219	0.0130472825309711\\
220	0.0130472022916399\\
221	0.013047120613319\\
222	0.0130470374699688\\
223	0.0130469528350683\\
224	0.013046866681606\\
225	0.0130467789820707\\
226	0.0130466897084414\\
227	0.0130465988321777\\
228	0.0130465063242097\\
229	0.0130464121549278\\
230	0.0130463162941718\\
231	0.0130462187112209\\
232	0.0130461193747818\\
233	0.0130460182529784\\
234	0.0130459153133399\\
235	0.0130458105227893\\
236	0.0130457038476314\\
237	0.0130455952535407\\
238	0.0130454847055489\\
239	0.0130453721680323\\
240	0.013045257604699\\
241	0.0130451409785752\\
242	0.0130450222519919\\
243	0.0130449013865714\\
244	0.0130447783432126\\
245	0.013044653082077\\
246	0.0130445255625736\\
247	0.0130443957433443\\
248	0.0130442635822479\\
249	0.013044129036345\\
250	0.0130439920618813\\
251	0.0130438526142715\\
252	0.0130437106480825\\
253	0.013043566117016\\
254	0.0130434189738911\\
255	0.0130432691706263\\
256	0.0130431166582211\\
257	0.0130429613867371\\
258	0.013042803305279\\
259	0.0130426423619749\\
260	0.0130424785039561\\
261	0.0130423116773365\\
262	0.013042141827192\\
263	0.0130419688975385\\
264	0.01304179283131\\
265	0.0130416135703364\\
266	0.0130414310553202\\
267	0.013041245225813\\
268	0.0130410560201916\\
269	0.0130408633756329\\
270	0.0130406672280892\\
271	0.013040467512262\\
272	0.0130402641615757\\
273	0.0130400571081508\\
274	0.0130398462827756\\
275	0.0130396316148786\\
276	0.0130394130324989\\
277	0.0130391904622571\\
278	0.0130389638293242\\
279	0.0130387330573912\\
280	0.0130384980686369\\
281	0.0130382587836952\\
282	0.0130380151216222\\
283	0.0130377669998614\\
284	0.0130375143342093\\
285	0.0130372570387791\\
286	0.0130369950259644\\
287	0.0130367282064015\\
288	0.0130364564889311\\
289	0.0130361797805588\\
290	0.0130358979864146\\
291	0.0130356110097124\\
292	0.0130353187517069\\
293	0.013035021111651\\
294	0.0130347179867514\\
295	0.013034409272123\\
296	0.0130340948607431\\
297	0.0130337746434036\\
298	0.0130334485086625\\
299	0.0130331163427945\\
300	0.0130327780297398\\
301	0.0130324334510526\\
302	0.0130320824858476\\
303	0.0130317250107456\\
304	0.0130313608998186\\
305	0.0130309900245321\\
306	0.0130306122536879\\
307	0.0130302274533643\\
308	0.0130298354868558\\
309	0.0130294362146114\\
310	0.0130290294941714\\
311	0.0130286151801028\\
312	0.0130281931239344\\
313	0.0130277631740892\\
314	0.0130273251758168\\
315	0.0130268789711236\\
316	0.0130264243987026\\
317	0.0130259612938612\\
318	0.0130254894884485\\
319	0.0130250088107809\\
320	0.0130245190855671\\
321	0.0130240201338317\\
322	0.0130235117728376\\
323	0.013022993816008\\
324	0.0130224660728469\\
325	0.0130219283488592\\
326	0.0130213804454698\\
327	0.013020822159942\\
328	0.0130202532852953\\
329	0.0130196736102227\\
330	0.0130190829190078\\
331	0.0130184809914405\\
332	0.0130178676027338\\
333	0.0130172425234396\\
334	0.0130166055193638\\
335	0.0130159563514822\\
336	0.0130152947758558\\
337	0.0130146205435453\\
338	0.0130139334005264\\
339	0.0130132330876038\\
340	0.0130125193403252\\
341	0.0130117918888936\\
342	0.0130110504580786\\
343	0.0130102947671241\\
344	0.0130095245296542\\
345	0.0130087394535696\\
346	0.0130079392409385\\
347	0.0130071235878833\\
348	0.0130062921844613\\
349	0.0130054447145392\\
350	0.0130045808556609\\
351	0.013003700278908\\
352	0.0130028026487518\\
353	0.0130018876229034\\
354	0.0130009548521656\\
355	0.0130000039802669\\
356	0.0129990346437371\\
357	0.0129980464717848\\
358	0.0129970390861772\\
359	0.0129960121011191\\
360	0.0129949651231335\\
361	0.0129938977509431\\
362	0.0129928095753534\\
363	0.0129917001791371\\
364	0.0129905691369206\\
365	0.0129894160150729\\
366	0.0129882403715969\\
367	0.0129870417560232\\
368	0.0129858197093076\\
369	0.0129845737637313\\
370	0.0129833034428057\\
371	0.0129820082611807\\
372	0.0129806877245568\\
373	0.0129793413296034\\
374	0.01297796856388\\
375	0.0129765689057642\\
376	0.012975141824384\\
377	0.0129736867795565\\
378	0.0129722032217318\\
379	0.0129706905919433\\
380	0.0129691483217645\\
381	0.0129675758332717\\
382	0.0129659725390139\\
383	0.01296433784199\\
384	0.0129626711356329\\
385	0.0129609718038026\\
386	0.0129592392207875\\
387	0.0129574727513155\\
388	0.0129556717505738\\
389	0.0129538355642351\\
390	0.0129519635284722\\
391	0.0129500549699064\\
392	0.0129481092053009\\
393	0.0129461255403884\\
394	0.0129441032658347\\
395	0.0129420416437695\\
396	0.0129399398630855\\
397	0.0129377968904506\\
398	0.0129356124541433\\
399	0.0129333864983498\\
400	0.012931118324202\\
401	0.0129288072300517\\
402	0.012926452512348\\
403	0.0129240534668051\\
404	0.0129216093903211\\
405	0.012919119584949\\
406	0.0129165833675789\\
407	0.0129140000955766\\
408	0.012911369237111\\
409	0.0129086905669634\\
410	0.0129059647158666\\
411	0.0129031898772679\\
412	0.0129003598726659\\
413	0.0128974736987815\\
414	0.0128945303215682\\
415	0.0128915286961883\\
416	0.0128884677850603\\
417	0.0128853465626767\\
418	0.0128821640213123\\
419	0.0128789191776147\\
420	0.012875611080104\\
421	0.0128722388185868\\
422	0.0128688015348229\\
423	0.0128652984370008\\
424	0.0128617288121446\\
425	0.012858092045354\\
426	0.0128543876437718\\
427	0.012850615267453\\
428	0.0128467747714451\\
429	0.0128428662678894\\
430	0.0128388902242682\\
431	0.0128348476133164\\
432	0.0128307400416881\\
433	0.012826567172327\\
434	0.0128223073711684\\
435	0.0128179587180579\\
436	0.0128135192539167\\
437	0.0128089869806677\\
438	0.0128043598612878\\
439	0.0127996358200218\\
440	0.0127948127428096\\
441	0.0127898884779919\\
442	0.0127848608373853\\
443	0.0127797275978492\\
444	0.0127744865035141\\
445	0.0127691352689071\\
446	0.0127636715833263\\
447	0.0127580931170181\\
448	0.0127523975301309\\
449	0.0127465824863504\\
450	0.0127406456753264\\
451	0.0127345848295087\\
452	0.0127283977101353\\
453	0.0127220820909179\\
454	0.0127156362368398\\
455	0.0127090588341981\\
456	0.0127023488828684\\
457	0.0126955057658321\\
458	0.0126885293260203\\
459	0.0126814199572238\\
460	0.0126741787687839\\
461	0.0126668081232145\\
462	0.0126593138819075\\
463	0.0126517044233373\\
464	0.0126440037340591\\
465	0.0126388151430028\\
466	0.0126349683817853\\
467	0.0126310358904958\\
468	0.0126270146369492\\
469	0.0126229013671404\\
470	0.012618692530352\\
471	0.0126143841460786\\
472	0.0126099734952427\\
473	0.0126054570996451\\
474	0.0126008319442539\\
475	0.0125960919907125\\
476	0.0125912290291176\\
477	0.0125862381601618\\
478	0.0125811142190499\\
479	0.0125758517682402\\
480	0.0125704450943308\\
481	0.0125648882100243\\
482	0.0125591748615398\\
483	0.0125532985382366\\
484	0.0125472524692487\\
485	0.0125410295557079\\
486	0.012534622083348\\
487	0.0125280207796136\\
488	0.0125212120904648\\
489	0.0125137888838374\\
490	0.0125037442339984\\
491	0.0124935209462086\\
492	0.0124831235887651\\
493	0.0124725526369678\\
494	0.012461816114435\\
495	0.0124509057701628\\
496	0.0124397798573642\\
497	0.0124284301287739\\
498	0.0124168472390404\\
499	0.0124050205282825\\
500	0.0123929377680932\\
501	0.0123805848740705\\
502	0.0123679456030925\\
503	0.0123550012684404\\
504	0.0123417304543679\\
505	0.0123281081715599\\
506	0.012314084922381\\
507	0.0122993881772031\\
508	0.0122839055285751\\
509	0.0122664192327332\\
510	0.0122332872494546\\
511	0.0121994087672017\\
512	0.0121647589209\\
513	0.0121293106808105\\
514	0.0120930377294443\\
515	0.0120559140741343\\
516	0.0120179128511751\\
517	0.0119790076498087\\
518	0.0119391749572796\\
519	0.0118983912798381\\
520	0.0118566193072385\\
521	0.011813828035859\\
522	0.0117699935037526\\
523	0.0117251002282769\\
524	0.0116791627581568\\
525	0.0116352019900711\\
526	0.0116146912723895\\
527	0.0115937610253761\\
528	0.0115724415945748\\
529	0.0115506754588339\\
530	0.0115284308052128\\
531	0.0115057079127346\\
532	0.0114825226316448\\
533	0.011458899981276\\
534	0.0114348741640457\\
535	0.0114104534089532\\
536	0.0113857063861111\\
537	0.011360774069598\\
538	0.0113357872502022\\
539	0.0113100537136924\\
540	0.0112830304820355\\
541	0.011254619635868\\
542	0.0112247115766432\\
543	0.01119319513065\\
544	0.0111599239169194\\
545	0.0111244645894079\\
546	0.0110867628805137\\
547	0.0110465359967391\\
548	0.0109867114941703\\
549	0.0109190101635827\\
550	0.0108478598474244\\
551	0.0107725282472264\\
552	0.010692374534789\\
553	0.0105420689983671\\
554	0.0103680459264671\\
555	0.010189781003786\\
556	0.0100068603494441\\
557	0.00981865392500542\\
558	0.00968971146961484\\
559	0.00961261036854086\\
560	0.00953876628330234\\
561	0.00946940009233044\\
562	0.0094062327273572\\
563	0.00935164670908242\\
564	0.00929765586947655\\
565	0.00924432435086066\\
566	0.00919156443186727\\
567	0.00913822618044222\\
568	0.00908427431447258\\
569	0.0090303868279074\\
570	0.0089765412868726\\
571	0.00892417214096602\\
572	0.00887165297246957\\
573	0.0088181602642672\\
574	0.00876393377480993\\
575	0.00870555987223758\\
576	0.00864785099385888\\
577	0.00859245881627625\\
578	0.00849014574700841\\
579	0.00837359678813039\\
580	0.00796206160435481\\
581	0.00748331129242383\\
582	0.00730441616660587\\
583	0.00722016428778616\\
584	0.00713992059634869\\
585	0.0070587537323187\\
586	0.00697632805286418\\
587	0.00689255466272374\\
588	0.00680731363119408\\
589	0.00672040282431112\\
590	0.00663138255463875\\
591	0.00653960395099572\\
592	0.00644346597070807\\
593	0.00633880868635237\\
594	0.00621472953405218\\
595	0.00604257880714497\\
596	0.00574702137257038\\
597	0.0051299581010561\\
598	0.00367057462141144\\
599	0\\
600	0\\
};
\addplot [color=red!50!mycolor17,solid,forget plot]
  table[row sep=crcr]{%
1	0.0127545985020357\\
2	0.0127545975764642\\
3	0.0127545966342993\\
4	0.012754595675244\\
5	0.0127545946989958\\
6	0.0127545937052469\\
7	0.0127545926936838\\
8	0.0127545916639875\\
9	0.0127545906158334\\
10	0.012754589548891\\
11	0.0127545884628239\\
12	0.0127545873572896\\
13	0.0127545862319396\\
14	0.0127545850864192\\
15	0.0127545839203672\\
16	0.0127545827334161\\
17	0.0127545815251917\\
18	0.0127545802953132\\
19	0.012754579043393\\
20	0.0127545777690365\\
21	0.0127545764718421\\
22	0.0127545751514011\\
23	0.0127545738072972\\
24	0.0127545724391071\\
25	0.0127545710463995\\
26	0.0127545696287358\\
27	0.0127545681856692\\
28	0.0127545667167452\\
29	0.012754565221501\\
30	0.0127545636994655\\
31	0.0127545621501595\\
32	0.0127545605730948\\
33	0.0127545589677748\\
34	0.0127545573336938\\
35	0.0127545556703374\\
36	0.0127545539771817\\
37	0.0127545522536935\\
38	0.0127545504993301\\
39	0.0127545487135393\\
40	0.0127545468957587\\
41	0.0127545450454161\\
42	0.012754543161929\\
43	0.0127545412447045\\
44	0.0127545392931392\\
45	0.0127545373066188\\
46	0.012754535284518\\
47	0.0127545332262006\\
48	0.0127545311310187\\
49	0.0127545289983131\\
50	0.0127545268274128\\
51	0.0127545246176346\\
52	0.0127545223682833\\
53	0.0127545200786514\\
54	0.0127545177480185\\
55	0.0127545153756515\\
56	0.0127545129608043\\
57	0.0127545105027172\\
58	0.0127545080006174\\
59	0.0127545054537178\\
60	0.0127545028612177\\
61	0.0127545002223018\\
62	0.0127544975361404\\
63	0.012754494801889\\
64	0.0127544920186881\\
65	0.0127544891856627\\
66	0.0127544863019223\\
67	0.0127544833665605\\
68	0.0127544803786549\\
69	0.0127544773372663\\
70	0.0127544742414391\\
71	0.0127544710902005\\
72	0.0127544678825604\\
73	0.012754464617511\\
74	0.0127544612940265\\
75	0.0127544579110631\\
76	0.012754454467558\\
77	0.0127544509624298\\
78	0.0127544473945775\\
79	0.0127544437628808\\
80	0.0127544400661994\\
81	0.0127544363033724\\
82	0.0127544324732185\\
83	0.0127544285745352\\
84	0.0127544246060988\\
85	0.0127544205666636\\
86	0.0127544164549617\\
87	0.0127544122697028\\
88	0.0127544080095735\\
89	0.012754403673237\\
90	0.0127543992593327\\
91	0.012754394766476\\
92	0.0127543901932573\\
93	0.0127543855382422\\
94	0.0127543807999705\\
95	0.0127543759769564\\
96	0.0127543710676873\\
97	0.0127543660706238\\
98	0.0127543609841992\\
99	0.0127543558068188\\
100	0.0127543505368596\\
101	0.0127543451726699\\
102	0.0127543397125682\\
103	0.0127543341548435\\
104	0.0127543284977542\\
105	0.0127543227395277\\
106	0.01275431687836\\
107	0.012754310912415\\
108	0.0127543048398239\\
109	0.0127542986586848\\
110	0.0127542923670618\\
111	0.0127542859629848\\
112	0.0127542794444486\\
113	0.0127542728094125\\
114	0.0127542660557992\\
115	0.0127542591814949\\
116	0.0127542521843482\\
117	0.0127542450621692\\
118	0.0127542378127294\\
119	0.0127542304337606\\
120	0.0127542229229543\\
121	0.0127542152779613\\
122	0.0127542074963904\\
123	0.0127541995758079\\
124	0.0127541915137373\\
125	0.0127541833076578\\
126	0.012754174955004\\
127	0.012754166453165\\
128	0.0127541577994835\\
129	0.0127541489912554\\
130	0.0127541400257281\\
131	0.0127541309001007\\
132	0.0127541216115224\\
133	0.0127541121570918\\
134	0.0127541025338563\\
135	0.0127540927388107\\
136	0.0127540827688968\\
137	0.0127540726210021\\
138	0.0127540622919589\\
139	0.0127540517785435\\
140	0.012754041077475\\
141	0.0127540301854145\\
142	0.012754019098964\\
143	0.0127540078146651\\
144	0.0127539963289985\\
145	0.0127539846383825\\
146	0.0127539727391718\\
147	0.0127539606276569\\
148	0.0127539483000625\\
149	0.0127539357525466\\
150	0.0127539229811992\\
151	0.0127539099820411\\
152	0.0127538967510228\\
153	0.0127538832840233\\
154	0.0127538695768486\\
155	0.0127538556252307\\
156	0.0127538414248259\\
157	0.0127538269712142\\
158	0.0127538122598971\\
159	0.0127537972862969\\
160	0.0127537820457549\\
161	0.0127537665335303\\
162	0.0127537507447983\\
163	0.0127537346746493\\
164	0.0127537183180867\\
165	0.012753701670026\\
166	0.0127536847252928\\
167	0.0127536674786215\\
168	0.0127536499246536\\
169	0.012753632057936\\
170	0.0127536138729196\\
171	0.0127535953639574\\
172	0.0127535765253028\\
173	0.012753557351108\\
174	0.0127535378354221\\
175	0.0127535179721893\\
176	0.0127534977552471\\
177	0.0127534771783246\\
178	0.0127534562350404\\
179	0.0127534349189007\\
180	0.0127534132232973\\
181	0.0127533911415059\\
182	0.0127533686666835\\
183	0.0127533457918671\\
184	0.0127533225099708\\
185	0.0127532988137844\\
186	0.0127532746959706\\
187	0.0127532501490633\\
188	0.0127532251654649\\
189	0.0127531997374444\\
190	0.0127531738571348\\
191	0.0127531475165309\\
192	0.0127531207074869\\
193	0.0127530934217136\\
194	0.0127530656507767\\
195	0.0127530373860931\\
196	0.0127530086189296\\
197	0.0127529793403993\\
198	0.0127529495414593\\
199	0.0127529192129081\\
200	0.0127528883453825\\
201	0.0127528569293552\\
202	0.0127528249551316\\
203	0.0127527924128471\\
204	0.0127527592924639\\
205	0.0127527255837684\\
206	0.0127526912763677\\
207	0.0127526563596866\\
208	0.0127526208229646\\
209	0.0127525846552525\\
210	0.0127525478454092\\
211	0.0127525103820984\\
212	0.0127524722537848\\
213	0.0127524334487313\\
214	0.012752393954995\\
215	0.0127523537604237\\
216	0.0127523128526523\\
217	0.012752271219099\\
218	0.0127522288469618\\
219	0.0127521857232142\\
220	0.0127521418346016\\
221	0.0127520971676372\\
222	0.0127520517085981\\
223	0.0127520054435208\\
224	0.0127519583581974\\
225	0.0127519104381711\\
226	0.0127518616687317\\
227	0.0127518120349114\\
228	0.0127517615214804\\
229	0.0127517101129418\\
230	0.0127516577935274\\
231	0.0127516045471924\\
232	0.0127515503576112\\
233	0.0127514952081719\\
234	0.0127514390819715\\
235	0.0127513819618104\\
236	0.0127513238301878\\
237	0.0127512646692958\\
238	0.0127512044610141\\
239	0.0127511431869044\\
240	0.012751080828205\\
241	0.0127510173658248\\
242	0.0127509527803373\\
243	0.0127508870519749\\
244	0.0127508201606226\\
245	0.0127507520858118\\
246	0.0127506828067137\\
247	0.0127506123021335\\
248	0.0127505405505028\\
249	0.0127504675298737\\
250	0.0127503932179114\\
251	0.0127503175918872\\
252	0.0127502406286714\\
253	0.0127501623047262\\
254	0.0127500825960975\\
255	0.0127500014784081\\
256	0.0127499189268491\\
257	0.0127498349161723\\
258	0.0127497494206821\\
259	0.0127496624142267\\
260	0.01274957387019\\
261	0.0127494837614825\\
262	0.0127493920605326\\
263	0.0127492987392772\\
264	0.0127492037691524\\
265	0.0127491071210841\\
266	0.0127490087654777\\
267	0.0127489086722083\\
268	0.0127488068106105\\
269	0.0127487031494674\\
270	0.012748597657\\
271	0.0127484903008559\\
272	0.0127483810480979\\
273	0.0127482698651925\\
274	0.0127481567179971\\
275	0.0127480415717485\\
276	0.0127479243910493\\
277	0.0127478051398555\\
278	0.0127476837814625\\
279	0.0127475602784914\\
280	0.0127474345928748\\
281	0.0127473066858419\\
282	0.0127471765179033\\
283	0.0127470440488357\\
284	0.0127469092376654\\
285	0.0127467720426516\\
286	0.0127466324212699\\
287	0.0127464903301936\\
288	0.012746345725276\\
289	0.0127461985615316\\
290	0.0127460487931158\\
291	0.0127458963733052\\
292	0.0127457412544767\\
293	0.0127455833880852\\
294	0.0127454227246418\\
295	0.0127452592136898\\
296	0.012745092803781\\
297	0.0127449234424505\\
298	0.0127447510761907\\
299	0.0127445756504238\\
300	0.0127443971094745\\
301	0.0127442153965401\\
302	0.0127440304536609\\
303	0.0127438422216878\\
304	0.0127436506402503\\
305	0.0127434556477219\\
306	0.0127432571811847\\
307	0.0127430551763925\\
308	0.0127428495677323\\
309	0.0127426402881842\\
310	0.0127424272692796\\
311	0.0127422104410578\\
312	0.0127419897320208\\
313	0.012741765069086\\
314	0.0127415363775372\\
315	0.0127413035809734\\
316	0.0127410666012558\\
317	0.0127408253584519\\
318	0.0127405797707788\\
319	0.0127403297545424\\
320	0.0127400752240757\\
321	0.0127398160916744\\
322	0.0127395522675295\\
323	0.0127392836596579\\
324	0.0127390101738314\\
325	0.0127387317135014\\
326	0.0127384481797234\\
327	0.0127381594710778\\
328	0.0127378654835887\\
329	0.0127375661106419\\
330	0.0127372612428994\\
331	0.0127369507682136\\
332	0.0127366345715401\\
333	0.012736312534849\\
334	0.0127359845370369\\
335	0.012735650453838\\
336	0.0127353101577365\\
337	0.0127349635178803\\
338	0.0127346103999963\\
339	0.0127342506663095\\
340	0.0127338841754691\\
341	0.0127335107824876\\
342	0.0127331303387021\\
343	0.0127327426917254\\
344	0.0127323476853464\\
345	0.0127319451596681\\
346	0.0127315349510807\\
347	0.0127311168922394\\
348	0.0127306908120478\\
349	0.0127302565356426\\
350	0.0127298138843792\\
351	0.0127293626758022\\
352	0.0127289027235709\\
353	0.0127284338372886\\
354	0.0127279558222645\\
355	0.0127274684799608\\
356	0.0127269716067945\\
357	0.0127264649939122\\
358	0.0127259484270419\\
359	0.0127254216863439\\
360	0.012724884546259\\
361	0.012724336775356\\
362	0.0127237781361776\\
363	0.0127232083850859\\
364	0.0127226272721078\\
365	0.0127220345407813\\
366	0.0127214299280029\\
367	0.0127208131638767\\
368	0.0127201839715677\\
369	0.0127195420671578\\
370	0.0127188871595081\\
371	0.0127182189501275\\
372	0.0127175371330486\\
373	0.0127168413947144\\
374	0.0127161314138752\\
375	0.0127154068615002\\
376	0.0127146674007034\\
377	0.0127139126866888\\
378	0.0127131423667156\\
379	0.0127123560800878\\
380	0.0127115534581704\\
381	0.0127107341244374\\
382	0.0127098976945537\\
383	0.0127090437764968\\
384	0.0127081719707228\\
385	0.0127072818703824\\
386	0.0127063730615918\\
387	0.0127054451237666\\
388	0.0127044976300246\\
389	0.0127035301476653\\
390	0.0127025422387345\\
391	0.0127015334606794\\
392	0.0127005033671021\\
393	0.0126994515086136\\
394	0.0126983774337974\\
395	0.0126972806903118\\
396	0.0126961608261695\\
397	0.0126950173907244\\
398	0.0126938499298565\\
399	0.0126926579880649\\
400	0.01269144111644\\
401	0.012690198875348\\
402	0.012688930837548\\
403	0.0126876365918084\\
404	0.0126863157470926\\
405	0.012684967937378\\
406	0.0126835928271214\\
407	0.012682190117211\\
408	0.0126807595507965\\
409	0.0126793009176372\\
410	0.012677814056609\\
411	0.0126762988749632\\
412	0.0126747553613267\\
413	0.0126731835940945\\
414	0.0126715841468356\\
415	0.012669956950282\\
416	0.0126683014638589\\
417	0.0126666171275712\\
418	0.0126649033616651\\
419	0.0126631595709315\\
420	0.0126613851474721\\
421	0.0126595794456335\\
422	0.0126577417826272\\
423	0.0126558713975911\\
424	0.0126539675972278\\
425	0.0126520296465964\\
426	0.0126500567628509\\
427	0.0126480481081597\\
428	0.012646002781848\\
429	0.0126439198113872\\
430	0.0126417981395917\\
431	0.0126396365981381\\
432	0.0126374338464782\\
433	0.0126351883554846\\
434	0.0126328990827063\\
435	0.0126305649454395\\
436	0.0126281848187238\\
437	0.0126257575333351\\
438	0.0126232818738159\\
439	0.012620756576594\\
440	0.0126181803282611\\
441	0.0126155517641018\\
442	0.0126128694669882\\
443	0.0126101319667811\\
444	0.0126073377403891\\
445	0.0126044852125958\\
446	0.0126015727575683\\
447	0.012598598700343\\
448	0.0125955613159838\\
449	0.0125924588206286\\
450	0.0125892893434939\\
451	0.0125860513142852\\
452	0.0125827439930391\\
453	0.012579366662723\\
454	0.0125759184148184\\
455	0.012572394257863\\
456	0.0125687932374642\\
457	0.0125651147312644\\
458	0.0125613584348809\\
459	0.0125575240805595\\
460	0.0125536103695203\\
461	0.012549611895864\\
462	0.0125439334591976\\
463	0.0125380933174566\\
464	0.0125322362780321\\
465	0.0125263047483982\\
466	0.0125202625361127\\
467	0.0125141061375706\\
468	0.0125078317092378\\
469	0.0125014349388582\\
470	0.0124949106999592\\
471	0.0124882520171283\\
472	0.0124814592380077\\
473	0.0124745295651245\\
474	0.0124674641365366\\
475	0.0124602493519749\\
476	0.0124528589170659\\
477	0.0124452819135758\\
478	0.0124375058674562\\
479	0.0124295164536896\\
480	0.0124212971364666\\
481	0.0124128287278014\\
482	0.0124040888414927\\
483	0.0123950512069913\\
484	0.0123856847780035\\
485	0.012375952485136\\
486	0.0123658092054592\\
487	0.0123551975629615\\
488	0.0123440366873667\\
489	0.0123306854120152\\
490	0.0123072077027676\\
491	0.0122832692163129\\
492	0.0122588579573395\\
493	0.0122339612282659\\
494	0.0122085652419333\\
495	0.0121826558858596\\
496	0.0121562202974452\\
497	0.0121292452727705\\
498	0.0121017172968511\\
499	0.0120736225881178\\
500	0.0120449471626263\\
501	0.0120156769289558\\
502	0.0119857978154159\\
503	0.0119552959497213\\
504	0.0119241578916636\\
505	0.011892370543465\\
506	0.0118599226093484\\
507	0.0118268153722785\\
508	0.0117930734350693\\
509	0.0117601392619305\\
510	0.0117456407789512\\
511	0.0117308877860919\\
512	0.0117158752030065\\
513	0.0117006091320383\\
514	0.0116850980239546\\
515	0.011669352891289\\
516	0.011653386793121\\
517	0.011637221513277\\
518	0.0116209020889976\\
519	0.01160448580294\\
520	0.0115879639692391\\
521	0.0115713622457898\\
522	0.0115547407383224\\
523	0.0115381767642198\\
524	0.0115217698764425\\
525	0.0115054968142768\\
526	0.0114886004483281\\
527	0.0114710728693751\\
528	0.0114528710104204\\
529	0.0114339476386273\\
530	0.0114142470182112\\
531	0.0113937053536593\\
532	0.0113722606700371\\
533	0.0113498424035028\\
534	0.0113263587921211\\
535	0.0113015253326078\\
536	0.0112752176297207\\
537	0.0112474747522305\\
538	0.0112181493619588\\
539	0.0111871038555331\\
540	0.0111541590929274\\
541	0.0111190891785521\\
542	0.0110815734281057\\
543	0.011028957280358\\
544	0.0109642141017049\\
545	0.0108964245416155\\
546	0.0108247704650098\\
547	0.0107482813245176\\
548	0.0106012588279682\\
549	0.0104318132978333\\
550	0.0102582519623652\\
551	0.0100799513088546\\
552	0.00989729974345482\\
553	0.00980075684378092\\
554	0.00973144082473939\\
555	0.00966604864384597\\
556	0.00960598532552121\\
557	0.00955381705766953\\
558	0.00950598076603648\\
559	0.00945881344723994\\
560	0.0094122614807116\\
561	0.00936608740260415\\
562	0.00932005111636898\\
563	0.00927326833710882\\
564	0.00922458047462828\\
565	0.00917586510537503\\
566	0.00912708432677337\\
567	0.00907937173966379\\
568	0.00903227121704845\\
569	0.00898438864080491\\
570	0.00893495145044245\\
571	0.00888468490012829\\
572	0.00883163213810392\\
573	0.0087768296892846\\
574	0.0087229479267433\\
575	0.00867430838405559\\
576	0.00859066016595079\\
577	0.00847302793356834\\
578	0.00809134841304196\\
579	0.00760258303780249\\
580	0.00745290055685877\\
581	0.00737433701842538\\
582	0.00729710956722606\\
583	0.00721880290108836\\
584	0.00713923376619486\\
585	0.0070583505526549\\
586	0.00697610643114973\\
587	0.00689244206250444\\
588	0.00680726342880715\\
589	0.00672038422413477\\
590	0.00663137743078028\\
591	0.00653960315341497\\
592	0.00644346597070806\\
593	0.00633880868635237\\
594	0.00621472953405217\\
595	0.00604257880714497\\
596	0.00574702137257038\\
597	0.0051299581010561\\
598	0.00367057462141144\\
599	0\\
600	0\\
};
\addplot [color=red!40!mycolor19,solid,forget plot]
  table[row sep=crcr]{%
1	0.0126821384982309\\
2	0.0126821375246621\\
3	0.0126821365336159\\
4	0.0126821355247788\\
5	0.0126821344978318\\
6	0.0126821334524499\\
7	0.0126821323883027\\
8	0.0126821313050534\\
9	0.0126821302023596\\
10	0.0126821290798726\\
11	0.0126821279372373\\
12	0.0126821267740925\\
13	0.0126821255900704\\
14	0.0126821243847965\\
15	0.0126821231578899\\
16	0.0126821219089625\\
17	0.0126821206376196\\
18	0.0126821193434593\\
19	0.0126821180260724\\
20	0.0126821166850424\\
21	0.0126821153199456\\
22	0.0126821139303503\\
23	0.0126821125158175\\
24	0.0126821110759\\
25	0.0126821096101427\\
26	0.0126821081180825\\
27	0.0126821065992478\\
28	0.0126821050531588\\
29	0.0126821034793267\\
30	0.0126821018772545\\
31	0.0126821002464359\\
32	0.0126820985863556\\
33	0.0126820968964893\\
34	0.012682095176303\\
35	0.0126820934252534\\
36	0.0126820916427875\\
37	0.0126820898283421\\
38	0.0126820879813443\\
39	0.0126820861012106\\
40	0.0126820841873474\\
41	0.0126820822391503\\
42	0.012682080256004\\
43	0.0126820782372825\\
44	0.0126820761823483\\
45	0.0126820740905527\\
46	0.0126820719612352\\
47	0.0126820697937237\\
48	0.012682067587334\\
49	0.0126820653413697\\
50	0.0126820630551218\\
51	0.0126820607278689\\
52	0.0126820583588764\\
53	0.0126820559473969\\
54	0.0126820534926694\\
55	0.0126820509939193\\
56	0.0126820484503583\\
57	0.012682045861184\\
58	0.0126820432255795\\
59	0.0126820405427136\\
60	0.0126820378117398\\
61	0.0126820350317971\\
62	0.0126820322020085\\
63	0.0126820293214817\\
64	0.0126820263893084\\
65	0.0126820234045641\\
66	0.0126820203663078\\
67	0.0126820172735816\\
68	0.0126820141254105\\
69	0.0126820109208023\\
70	0.0126820076587468\\
71	0.0126820043382161\\
72	0.0126820009581637\\
73	0.0126819975175245\\
74	0.0126819940152145\\
75	0.0126819904501301\\
76	0.0126819868211484\\
77	0.0126819831271261\\
78	0.0126819793668998\\
79	0.0126819755392851\\
80	0.0126819716430768\\
81	0.0126819676770478\\
82	0.0126819636399495\\
83	0.0126819595305109\\
84	0.0126819553474382\\
85	0.0126819510894149\\
86	0.0126819467551007\\
87	0.0126819423431316\\
88	0.0126819378521192\\
89	0.0126819332806505\\
90	0.0126819286272873\\
91	0.0126819238905656\\
92	0.0126819190689956\\
93	0.0126819141610609\\
94	0.0126819091652179\\
95	0.0126819040798959\\
96	0.012681898903496\\
97	0.0126818936343908\\
98	0.0126818882709242\\
99	0.0126818828114104\\
100	0.0126818772541336\\
101	0.0126818715973478\\
102	0.0126818658392757\\
103	0.0126818599781084\\
104	0.0126818540120049\\
105	0.0126818479390916\\
106	0.0126818417574614\\
107	0.0126818354651734\\
108	0.0126818290602522\\
109	0.0126818225406875\\
110	0.0126818159044331\\
111	0.0126818091494065\\
112	0.0126818022734883\\
113	0.0126817952745214\\
114	0.0126817881503104\\
115	0.0126817808986213\\
116	0.0126817735171799\\
117	0.0126817660036721\\
118	0.0126817583557426\\
119	0.0126817505709945\\
120	0.0126817426469881\\
121	0.0126817345812408\\
122	0.0126817263712259\\
123	0.0126817180143718\\
124	0.0126817095080616\\
125	0.0126817008496319\\
126	0.0126816920363722\\
127	0.0126816830655241\\
128	0.0126816739342803\\
129	0.0126816646397839\\
130	0.0126816551791273\\
131	0.0126816455493519\\
132	0.0126816357474463\\
133	0.0126816257703463\\
134	0.0126816156149333\\
135	0.0126816052780337\\
136	0.0126815947564179\\
137	0.0126815840467992\\
138	0.012681573145833\\
139	0.0126815620501155\\
140	0.0126815507561832\\
141	0.0126815392605111\\
142	0.0126815275595123\\
143	0.0126815156495368\\
144	0.0126815035268699\\
145	0.0126814911877319\\
146	0.0126814786282762\\
147	0.0126814658445888\\
148	0.0126814528326865\\
149	0.0126814395885163\\
150	0.0126814261079537\\
151	0.0126814123868019\\
152	0.0126813984207902\\
153	0.0126813842055729\\
154	0.012681369736728\\
155	0.0126813550097559\\
156	0.0126813400200779\\
157	0.0126813247630351\\
158	0.0126813092338869\\
159	0.0126812934278095\\
160	0.0126812773398946\\
161	0.012681260965148\\
162	0.012681244298488\\
163	0.0126812273347438\\
164	0.0126812100686544\\
165	0.0126811924948666\\
166	0.0126811746079336\\
167	0.0126811564023137\\
168	0.012681137872368\\
169	0.0126811190123594\\
170	0.0126810998164506\\
171	0.0126810802787026\\
172	0.0126810603930726\\
173	0.0126810401534127\\
174	0.0126810195534679\\
175	0.0126809985868741\\
176	0.0126809772471566\\
177	0.0126809555277281\\
178	0.0126809334218867\\
179	0.012680910922814\\
180	0.0126808880235734\\
181	0.0126808647171076\\
182	0.0126808409962371\\
183	0.0126808168536577\\
184	0.0126807922819388\\
185	0.0126807672735209\\
186	0.0126807418207138\\
187	0.0126807159156941\\
188	0.0126806895505034\\
189	0.0126806627170454\\
190	0.0126806354070842\\
191	0.0126806076122416\\
192	0.0126805793239951\\
193	0.012680550533675\\
194	0.0126805212324624\\
195	0.0126804914113865\\
196	0.0126804610613223\\
197	0.0126804301729875\\
198	0.0126803987369408\\
199	0.0126803667435784\\
200	0.0126803341831318\\
201	0.0126803010456651\\
202	0.012680267321072\\
203	0.0126802329990731\\
204	0.0126801980692133\\
205	0.0126801625208585\\
206	0.012680126343193\\
207	0.0126800895252164\\
208	0.0126800520557408\\
209	0.0126800139233873\\
210	0.0126799751165833\\
211	0.0126799356235593\\
212	0.0126798954323454\\
213	0.0126798545307686\\
214	0.0126798129064489\\
215	0.0126797705467965\\
216	0.0126797274390081\\
217	0.0126796835700635\\
218	0.0126796389267223\\
219	0.0126795934955201\\
220	0.0126795472627651\\
221	0.0126795002145345\\
222	0.0126794523366707\\
223	0.0126794036147774\\
224	0.0126793540342164\\
225	0.0126793035801028\\
226	0.0126792522373022\\
227	0.0126791999904257\\
228	0.0126791468238268\\
229	0.0126790927215966\\
230	0.0126790376675601\\
231	0.0126789816452718\\
232	0.0126789246380116\\
233	0.0126788666287805\\
234	0.012678807600296\\
235	0.012678747534988\\
236	0.0126786864149941\\
237	0.012678624222155\\
238	0.0126785609380102\\
239	0.0126784965437929\\
240	0.0126784310204256\\
241	0.0126783643485151\\
242	0.0126782965083477\\
243	0.0126782274798843\\
244	0.0126781572427552\\
245	0.0126780857762554\\
246	0.0126780130593389\\
247	0.012677939070614\\
248	0.0126778637883377\\
249	0.0126777871904103\\
250	0.0126777092543704\\
251	0.0126776299573888\\
252	0.0126775492762631\\
253	0.0126774671874123\\
254	0.0126773836668708\\
255	0.0126772986902824\\
256	0.0126772122328948\\
257	0.0126771242695532\\
258	0.0126770347746946\\
259	0.0126769437223411\\
260	0.0126768510860941\\
261	0.012676756839128\\
262	0.012676660954183\\
263	0.0126765634035596\\
264	0.0126764641591111\\
265	0.0126763631922373\\
266	0.0126762604738773\\
267	0.0126761559745031\\
268	0.0126760496641118\\
269	0.012675941512219\\
270	0.0126758314878511\\
271	0.0126757195595383\\
272	0.0126756056953065\\
273	0.01267548986267\\
274	0.0126753720286235\\
275	0.0126752521596342\\
276	0.0126751302216336\\
277	0.0126750061800094\\
278	0.012674879999597\\
279	0.0126747516446705\\
280	0.0126746210789348\\
281	0.0126744882655155\\
282	0.0126743531669508\\
283	0.0126742157451815\\
284	0.0126740759615413\\
285	0.0126739337767478\\
286	0.0126737891508914\\
287	0.0126736420434256\\
288	0.0126734924131564\\
289	0.012673340218231\\
290	0.0126731854161273\\
291	0.0126730279636416\\
292	0.0126728678168772\\
293	0.012672704931232\\
294	0.012672539261386\\
295	0.012672370761288\\
296	0.0126721993841421\\
297	0.0126720250823937\\
298	0.0126718478077151\\
299	0.01267166751099\\
300	0.012671484142298\\
301	0.012671297650898\\
302	0.0126711079852113\\
303	0.0126709150928033\\
304	0.0126707189203651\\
305	0.0126705194136937\\
306	0.0126703165176713\\
307	0.0126701101762437\\
308	0.0126699003323975\\
309	0.0126696869281357\\
310	0.0126694699044525\\
311	0.0126692492013064\\
312	0.0126690247575915\\
313	0.0126687965111075\\
314	0.0126685643985272\\
315	0.0126683283553624\\
316	0.0126680883159277\\
317	0.0126678442133014\\
318	0.0126675959792837\\
319	0.0126673435443526\\
320	0.0126670868376161\\
321	0.0126668257867612\\
322	0.0126665603179988\\
323	0.012666290356005\\
324	0.0126660158238573\\
325	0.0126657366429665\\
326	0.0126654527330029\\
327	0.0126651640118166\\
328	0.0126648703953516\\
329	0.0126645717975529\\
330	0.0126642681302662\\
331	0.0126639593031291\\
332	0.0126636452234532\\
333	0.0126633257960971\\
334	0.0126630009233276\\
335	0.0126626705046691\\
336	0.0126623344367383\\
337	0.0126619926130599\\
338	0.0126616449238555\\
339	0.0126612912557909\\
340	0.0126609314916649\\
341	0.012660565510051\\
342	0.0126601931850861\\
343	0.0126598143873731\\
344	0.0126594289848655\\
345	0.0126590368355522\\
346	0.0126586377925764\\
347	0.0126582317041404\\
348	0.0126578184134729\\
349	0.0126573977589036\\
350	0.012656969574118\\
351	0.0126565336886903\\
352	0.0126560899288936\\
353	0.0126556381181686\\
354	0.0126551780741605\\
355	0.0126547095911987\\
356	0.0126542324912549\\
357	0.0126537465935831\\
358	0.0126532517119574\\
359	0.0126527476544005\\
360	0.0126522342228926\\
361	0.0126517112130605\\
362	0.012651178413846\\
363	0.0126506356071492\\
364	0.0126500825674476\\
365	0.0126495190613878\\
366	0.0126489448473467\\
367	0.0126483596749608\\
368	0.0126477632846213\\
369	0.0126471554069301\\
370	0.0126465357621159\\
371	0.0126459040594051\\
372	0.0126452599963446\\
373	0.012644603258072\\
374	0.012643933516529\\
375	0.0126432504296128\\
376	0.0126425536402607\\
377	0.0126418427754615\\
378	0.0126411174451877\\
379	0.0126403772412426\\
380	0.0126396217360134\\
381	0.0126388504811248\\
382	0.0126380630059827\\
383	0.0126372588162003\\
384	0.0126364373918982\\
385	0.0126355981858662\\
386	0.0126347406215805\\
387	0.0126338640910623\\
388	0.0126329679525716\\
389	0.0126320515281226\\
390	0.0126311141008155\\
391	0.0126301549119749\\
392	0.0126291731580922\\
393	0.0126281679875721\\
394	0.0126271384972958\\
395	0.0126260837290172\\
396	0.0126250026656035\\
397	0.0126238942270721\\
398	0.0126227572667663\\
399	0.0126215905676232\\
400	0.0126203928383898\\
401	0.0126191627101315\\
402	0.0126178987332571\\
403	0.0126165993753612\\
404	0.0126152630202705\\
405	0.012613887968738\\
406	0.012612472441149\\
407	0.0126110145820429\\
408	0.0126095124642345\\
409	0.0126079640840799\\
410	0.0126063673212525\\
411	0.0126047197843032\\
412	0.0126030183340735\\
413	0.0126012578028076\\
414	0.0125989304301161\\
415	0.0125962901405522\\
416	0.0125936070902231\\
417	0.012590880505345\\
418	0.0125881095475069\\
419	0.0125852932704924\\
420	0.0125824308695115\\
421	0.0125795224034734\\
422	0.0125765676112143\\
423	0.0125735663641194\\
424	0.0125705136768825\\
425	0.0125674085614011\\
426	0.0125642499913995\\
427	0.0125610368994781\\
428	0.012557768173878\\
429	0.0125544426547109\\
430	0.01255105912919\\
431	0.0125476163258642\\
432	0.0125441129112809\\
433	0.0125405474937155\\
434	0.0125369185937896\\
435	0.0125332246362759\\
436	0.0125294639373186\\
437	0.0125256346895387\\
438	0.0125217349446237\\
439	0.012517762592922\\
440	0.0125137153394577\\
441	0.012509590675649\\
442	0.0125053858458513\\
443	0.012501097807614\\
444	0.0124967231841954\\
445	0.0124922582072418\\
446	0.0124876986460775\\
447	0.0124830397160824\\
448	0.0124782759466124\\
449	0.0124734009505523\\
450	0.0124684069118317\\
451	0.0124632861797514\\
452	0.0124580339859791\\
453	0.0124526440836535\\
454	0.0124471078146902\\
455	0.0124413907491602\\
456	0.0124354742367119\\
457	0.0124293360230302\\
458	0.0124229491411886\\
459	0.0124162798077128\\
460	0.0124092822279207\\
461	0.0124018826791984\\
462	0.0123877276286462\\
463	0.0123727775787413\\
464	0.0123575532216385\\
465	0.0123420478104492\\
466	0.012326255220179\\
467	0.0123101691581969\\
468	0.0122937831742189\\
469	0.0122770906921087\\
470	0.0122600850165231\\
471	0.012242759287295\\
472	0.0122251062073171\\
473	0.01220711815554\\
474	0.0121887869584476\\
475	0.0121701045083871\\
476	0.0121510631665305\\
477	0.0121316560674415\\
478	0.0121118764953822\\
479	0.012091717978397\\
480	0.012071174408962\\
481	0.0120502401973416\\
482	0.0120289104678494\\
483	0.0120071813159212\\
484	0.0119850501682907\\
485	0.0119625163753743\\
486	0.0119395824922343\\
487	0.0119162579718089\\
488	0.0118925719857662\\
489	0.0118705673501859\\
490	0.0118605304669374\\
491	0.0118503396459747\\
492	0.011839997339345\\
493	0.0118295068834464\\
494	0.0118188726781294\\
495	0.0118081003765547\\
496	0.0117971970598913\\
497	0.0117861715028001\\
498	0.0117750344858835\\
499	0.01176379917563\\
500	0.0117524815701264\\
501	0.0117411009919618\\
502	0.0117296810282637\\
503	0.0117182503583083\\
504	0.0117068441894678\\
505	0.0116955081135955\\
506	0.0116842875654741\\
507	0.0116732455064873\\
508	0.0116624622193177\\
509	0.0116519843840293\\
510	0.011641246824729\\
511	0.0116302568092112\\
512	0.0116189542477841\\
513	0.0116073228885564\\
514	0.0115953393642289\\
515	0.0115829312548133\\
516	0.0115701394049692\\
517	0.0115569389123067\\
518	0.011543301569114\\
519	0.0115291953933751\\
520	0.0115145843875783\\
521	0.0114994273081138\\
522	0.0114836762423314\\
523	0.0114672749160674\\
524	0.0114501550903314\\
525	0.0114320820014225\\
526	0.0114130524724843\\
527	0.0113931211410729\\
528	0.0113722232503991\\
529	0.0113502871174845\\
530	0.0113272097629686\\
531	0.0113028572075059\\
532	0.0112771285416074\\
533	0.0112499113236196\\
534	0.0112210926444908\\
535	0.0111906872246862\\
536	0.0111585186218426\\
537	0.0111241047832793\\
538	0.011082933573668\\
539	0.0110216252168996\\
540	0.010957205380482\\
541	0.0108892609878906\\
542	0.0108169205862074\\
543	0.0106913176793697\\
544	0.0105257624230604\\
545	0.0103567523001861\\
546	0.0101838360539744\\
547	0.0100073144777143\\
548	0.00991657440060152\\
549	0.00985343438137391\\
550	0.00979458317156496\\
551	0.00974203412149505\\
552	0.00969726603262051\\
553	0.00965485660769349\\
554	0.00961313061874853\\
555	0.00957197600187563\\
556	0.00953116132464891\\
557	0.00948983153602096\\
558	0.00944828000425516\\
559	0.00940622891548437\\
560	0.00936245232514134\\
561	0.00931770832586766\\
562	0.00927281369592761\\
563	0.00922816330438962\\
564	0.0091851585840644\\
565	0.00914141022825503\\
566	0.00909687310998814\\
567	0.00905144120132996\\
568	0.0090041904811284\\
569	0.00895622432265022\\
570	0.00890419797882667\\
571	0.00885210644247857\\
572	0.00880340548299123\\
573	0.00875700766877701\\
574	0.00870316920464919\\
575	0.00858293100054045\\
576	0.00828880804391675\\
577	0.00779225449873448\\
578	0.0075994314578478\\
579	0.00752414904281851\\
580	0.00744956084147144\\
581	0.0073738431920374\\
582	0.00729689511120333\\
583	0.00721867978526622\\
584	0.00713916438478356\\
585	0.0070583142646163\\
586	0.00697608921767156\\
587	0.00689243494774153\\
588	0.00680726100967728\\
589	0.00672038361872311\\
590	0.00663137734622848\\
591	0.00653960315341498\\
592	0.00644346597070807\\
593	0.00633880868635237\\
594	0.00621472953405218\\
595	0.00604257880714498\\
596	0.00574702137257038\\
597	0.0051299581010561\\
598	0.00367057462141144\\
599	0\\
600	0\\
};
\addplot [color=red!75!mycolor17,solid,forget plot]
  table[row sep=crcr]{%
1	0.0126301263351651\\
2	0.0126301243954216\\
3	0.0126301224208585\\
4	0.0126301204108513\\
5	0.0126301183647642\\
6	0.0126301162819502\\
7	0.0126301141617505\\
8	0.0126301120034946\\
9	0.0126301098065001\\
10	0.0126301075700721\\
11	0.0126301052935036\\
12	0.0126301029760746\\
13	0.0126301006170523\\
14	0.0126300982156911\\
15	0.0126300957712315\\
16	0.0126300932829008\\
17	0.0126300907499123\\
18	0.0126300881714653\\
19	0.0126300855467446\\
20	0.0126300828749205\\
21	0.0126300801551485\\
22	0.012630077386569\\
23	0.0126300745683067\\
24	0.0126300716994711\\
25	0.0126300687791554\\
26	0.0126300658064366\\
27	0.0126300627803754\\
28	0.0126300597000154\\
29	0.0126300565643833\\
30	0.0126300533724882\\
31	0.0126300501233216\\
32	0.0126300468158569\\
33	0.0126300434490491\\
34	0.0126300400218346\\
35	0.0126300365331305\\
36	0.0126300329818348\\
37	0.0126300293668256\\
38	0.0126300256869611\\
39	0.0126300219410789\\
40	0.0126300181279958\\
41	0.0126300142465075\\
42	0.0126300102953883\\
43	0.0126300062733901\\
44	0.0126300021792431\\
45	0.0126299980116542\\
46	0.0126299937693075\\
47	0.0126299894508635\\
48	0.0126299850549587\\
49	0.0126299805802052\\
50	0.0126299760251902\\
51	0.012629971388476\\
52	0.0126299666685986\\
53	0.0126299618640684\\
54	0.0126299569733687\\
55	0.0126299519949559\\
56	0.0126299469272588\\
57	0.0126299417686781\\
58	0.0126299365175857\\
59	0.0126299311723248\\
60	0.0126299257312085\\
61	0.0126299201925202\\
62	0.0126299145545123\\
63	0.012629908815406\\
64	0.0126299029733908\\
65	0.0126298970266238\\
66	0.0126298909732292\\
67	0.0126298848112976\\
68	0.0126298785388855\\
69	0.0126298721540147\\
70	0.0126298656546717\\
71	0.0126298590388069\\
72	0.0126298523043342\\
73	0.0126298454491301\\
74	0.0126298384710334\\
75	0.0126298313678439\\
76	0.0126298241373226\\
77	0.01262981677719\\
78	0.0126298092851264\\
79	0.0126298016587703\\
80	0.0126297938957181\\
81	0.0126297859935234\\
82	0.012629777949696\\
83	0.0126297697617014\\
84	0.0126297614269597\\
85	0.0126297529428451\\
86	0.0126297443066847\\
87	0.012629735515758\\
88	0.0126297265672962\\
89	0.0126297174584807\\
90	0.0126297081864429\\
91	0.0126296987482629\\
92	0.0126296891409686\\
93	0.0126296793615352\\
94	0.0126296694068836\\
95	0.0126296592738799\\
96	0.0126296489593345\\
97	0.0126296384600007\\
98	0.012629627772574\\
99	0.0126296168936909\\
100	0.0126296058199281\\
101	0.0126295945478011\\
102	0.0126295830737635\\
103	0.0126295713942054\\
104	0.0126295595054529\\
105	0.0126295474037664\\
106	0.0126295350853399\\
107	0.0126295225462992\\
108	0.0126295097827017\\
109	0.0126294967905339\\
110	0.0126294835657114\\
111	0.0126294701040767\\
112	0.0126294564013985\\
113	0.01262944245337\\
114	0.0126294282556079\\
115	0.0126294138036508\\
116	0.0126293990929578\\
117	0.0126293841189075\\
118	0.012629368876796\\
119	0.0126293533618358\\
120	0.0126293375691542\\
121	0.0126293214937921\\
122	0.0126293051307018\\
123	0.0126292884747462\\
124	0.0126292715206968\\
125	0.012629254263232\\
126	0.0126292366969358\\
127	0.0126292188162958\\
128	0.0126292006157019\\
129	0.0126291820894442\\
130	0.0126291632317113\\
131	0.0126291440365886\\
132	0.0126291244980568\\
133	0.0126291046099892\\
134	0.0126290843661509\\
135	0.012629063760196\\
136	0.0126290427856661\\
137	0.0126290214359881\\
138	0.0126289997044726\\
139	0.0126289775843111\\
140	0.0126289550685749\\
141	0.0126289321502121\\
142	0.0126289088220459\\
143	0.0126288850767722\\
144	0.0126288609069578\\
145	0.0126288363050375\\
146	0.0126288112633121\\
147	0.0126287857739463\\
148	0.012628759828966\\
149	0.0126287334202557\\
150	0.0126287065395565\\
151	0.0126286791784635\\
152	0.0126286513284227\\
153	0.0126286229807292\\
154	0.0126285941265241\\
155	0.0126285647567916\\
156	0.0126285348623569\\
157	0.012628504433883\\
158	0.0126284734618678\\
159	0.0126284419366415\\
160	0.0126284098483636\\
161	0.0126283771870197\\
162	0.0126283439424188\\
163	0.0126283101041902\\
164	0.01262827566178\\
165	0.0126282406044483\\
166	0.012628204921266\\
167	0.012628168601111\\
168	0.0126281316326657\\
169	0.0126280940044128\\
170	0.0126280557046323\\
171	0.0126280167213977\\
172	0.012627977042573\\
173	0.0126279366558081\\
174	0.0126278955485361\\
175	0.0126278537079691\\
176	0.0126278111210941\\
177	0.0126277677746697\\
178	0.0126277236552218\\
179	0.0126276787490397\\
180	0.012627633042172\\
181	0.0126275865204223\\
182	0.0126275391693455\\
183	0.0126274909742428\\
184	0.0126274419201579\\
185	0.0126273919918724\\
186	0.0126273411739013\\
187	0.0126272894504882\\
188	0.0126272368056013\\
189	0.0126271832229278\\
190	0.0126271286858701\\
191	0.0126270731775398\\
192	0.0126270166807538\\
193	0.0126269591780286\\
194	0.0126269006515755\\
195	0.012626841083295\\
196	0.0126267804547721\\
197	0.0126267187472705\\
198	0.0126266559417273\\
199	0.0126265920187474\\
200	0.0126265269585979\\
201	0.0126264607412023\\
202	0.0126263933461349\\
203	0.0126263247526146\\
204	0.0126262549394991\\
205	0.0126261838852787\\
206	0.0126261115680702\\
207	0.0126260379656104\\
208	0.01262596305525\\
209	0.0126258868139469\\
210	0.0126258092182596\\
211	0.0126257302443406\\
212	0.0126256498679296\\
213	0.0126255680643464\\
214	0.012625484808484\\
215	0.0126254000748017\\
216	0.0126253138373173\\
217	0.0126252260696004\\
218	0.0126251367447643\\
219	0.0126250458354588\\
220	0.0126249533138627\\
221	0.0126248591516752\\
222	0.0126247633201088\\
223	0.0126246657898808\\
224	0.0126245665312052\\
225	0.0126244655137843\\
226	0.0126243627068005\\
227	0.0126242580789074\\
228	0.0126241515982216\\
229	0.0126240432323132\\
230	0.0126239329481974\\
231	0.0126238207123253\\
232	0.0126237064905744\\
233	0.0126235902482395\\
234	0.0126234719500231\\
235	0.012623351560026\\
236	0.0126232290417368\\
237	0.012623104358023\\
238	0.0126229774711201\\
239	0.0126228483426217\\
240	0.012622716933469\\
241	0.0126225832039407\\
242	0.0126224471136415\\
243	0.0126223086214923\\
244	0.0126221676857184\\
245	0.0126220242638387\\
246	0.0126218783126544\\
247	0.0126217297882374\\
248	0.0126215786459191\\
249	0.012621424840278\\
250	0.0126212683251282\\
251	0.0126211090535072\\
252	0.0126209469776637\\
253	0.0126207820490449\\
254	0.0126206142182841\\
255	0.0126204434351879\\
256	0.0126202696487232\\
257	0.0126200928070042\\
258	0.0126199128572787\\
259	0.0126197297459151\\
260	0.0126195434183887\\
261	0.0126193538192675\\
262	0.0126191608921984\\
263	0.0126189645798931\\
264	0.0126187648241136\\
265	0.0126185615656575\\
266	0.0126183547443438\\
267	0.0126181442989972\\
268	0.0126179301674336\\
269	0.0126177122864445\\
270	0.0126174905917815\\
271	0.0126172650181408\\
272	0.0126170354991473\\
273	0.0126168019673384\\
274	0.0126165643541477\\
275	0.0126163225898891\\
276	0.0126160766037393\\
277	0.0126158263237219\\
278	0.0126155716766898\\
279	0.0126153125883082\\
280	0.0126150489830371\\
281	0.0126147807841141\\
282	0.0126145079135361\\
283	0.0126142302920418\\
284	0.0126139478390933\\
285	0.0126136604728581\\
286	0.01261336811019\\
287	0.0126130706666111\\
288	0.0126127680562923\\
289	0.0126124601920349\\
290	0.0126121469852507\\
291	0.0126118283459428\\
292	0.0126115041826862\\
293	0.0126111744026078\\
294	0.0126108389113665\\
295	0.0126104976131329\\
296	0.0126101504105693\\
297	0.0126097972048087\\
298	0.0126094378954348\\
299	0.0126090723804603\\
300	0.0126087005563068\\
301	0.0126083223177827\\
302	0.0126079375580625\\
303	0.012607546168665\\
304	0.0126071480394313\\
305	0.0126067430585033\\
306	0.0126063311123015\\
307	0.0126059120855025\\
308	0.0126054858610167\\
309	0.0126050523199655\\
310	0.0126046113416588\\
311	0.0126041628035711\\
312	0.0126037065813192\\
313	0.0126032425486376\\
314	0.0126027705773554\\
315	0.0126022905373719\\
316	0.012601802296632\\
317	0.0126013057211017\\
318	0.0126008006747429\\
319	0.0126002870194874\\
320	0.0125997646152115\\
321	0.0125992333197087\\
322	0.0125986929886628\\
323	0.0125981434756198\\
324	0.0125975846319589\\
325	0.0125970163068631\\
326	0.0125964383472874\\
327	0.0125958505979271\\
328	0.0125952529011835\\
329	0.0125946450971278\\
330	0.0125940270234625\\
331	0.0125933985154807\\
332	0.0125927594060206\\
333	0.0125921095254162\\
334	0.0125914487014409\\
335	0.0125907767592406\\
336	0.0125900935212472\\
337	0.0125893988070474\\
338	0.0125886924331457\\
339	0.0125879742124495\\
340	0.0125872439530371\\
341	0.0125865014550807\\
342	0.012585746503249\\
343	0.0125849788489805\\
344	0.0125841982565308\\
345	0.0125834047090503\\
346	0.0125825979796988\\
347	0.0125817778336415\\
348	0.012580944027497\\
349	0.012580096309226\\
350	0.012579234419293\\
351	0.0125783580951219\\
352	0.0125774670834935\\
353	0.0125765611706726\\
354	0.0125756402473431\\
355	0.0125747044227837\\
356	0.0125737521125687\\
357	0.0125727829180165\\
358	0.012571796531002\\
359	0.0125707926364278\\
360	0.0125697709119313\\
361	0.0125687310275672\\
362	0.0125676726454652\\
363	0.0125665954194582\\
364	0.0125654989946783\\
365	0.0125643830071177\\
366	0.0125632470831488\\
367	0.0125620908390009\\
368	0.0125609138801855\\
369	0.0125597158008666\\
370	0.0125584961831669\\
371	0.0125572545964039\\
372	0.0125559905962442\\
373	0.0125547037237683\\
374	0.0125533935044309\\
375	0.0125520594469041\\
376	0.0125507010417873\\
377	0.0125493177601642\\
378	0.0125479090519839\\
379	0.0125464743442433\\
380	0.0125450130389367\\
381	0.0125435245107407\\
382	0.0125420081043909\\
383	0.0125404631317023\\
384	0.0125388888681758\\
385	0.0125372845491215\\
386	0.0125356493652188\\
387	0.0125339824574161\\
388	0.0125322829110544\\
389	0.0125305497490767\\
390	0.0125287819241584\\
391	0.0125269783095575\\
392	0.0125251376884492\\
393	0.0125232587414591\\
394	0.0125213400320687\\
395	0.0125193799895396\\
396	0.0125173768890709\\
397	0.0125153288292992\\
398	0.0125132336990394\\
399	0.0125110891411607\\
400	0.0125088925148683\\
401	0.0125066408459497\\
402	0.0125043307666397\\
403	0.0125019584426398\\
404	0.0124995194841509\\
405	0.012497008836821\\
406	0.0124944206470281\\
407	0.0124917480932079\\
408	0.0124889831688415\\
409	0.0124861163862751\\
410	0.0124831363200509\\
411	0.0124800287723485\\
412	0.0124767747369785\\
413	0.0124733441581986\\
414	0.0124677231507463\\
415	0.0124609157159468\\
416	0.0124539913982103\\
417	0.0124469482630345\\
418	0.0124397842081562\\
419	0.012432496628115\\
420	0.0124250835271574\\
421	0.0124175477264243\\
422	0.0124098909318436\\
423	0.0124021171596884\\
424	0.0123942039555683\\
425	0.0123861487965927\\
426	0.0123779491073991\\
427	0.0123696022586704\\
428	0.0123611055656747\\
429	0.0123524562869015\\
430	0.012343651622965\\
431	0.0123346887160222\\
432	0.012325564649277\\
433	0.012316276442285\\
434	0.0123068210370234\\
435	0.0122971953110777\\
436	0.0122873960773541\\
437	0.0122774200842912\\
438	0.0122672640167197\\
439	0.012256924497553\\
440	0.0122463980905396\\
441	0.0122356813043599\\
442	0.0122247705984248\\
443	0.0122136623908186\\
444	0.0122023530689487\\
445	0.0121908390036253\\
446	0.0121791165675565\\
447	0.0121671821597444\\
448	0.0121550322384044\\
449	0.0121426633679102\\
450	0.0121300722925003\\
451	0.0121172559356511\\
452	0.0121042112740916\\
453	0.0120909354256852\\
454	0.0120774258116741\\
455	0.0120636814375333\\
456	0.0120497022455232\\
457	0.0120354894554633\\
458	0.0120210462186853\\
459	0.0120063793115629\\
460	0.0119915046605629\\
461	0.0119764676725778\\
462	0.011969450781025\\
463	0.0119629969679124\\
464	0.0119564443901573\\
465	0.0119497934637275\\
466	0.0119430448783009\\
467	0.0119361996372254\\
468	0.0119292591031179\\
469	0.011922225049423\\
470	0.0119150997195856\\
471	0.0119078858962934\\
472	0.0119005869906802\\
473	0.0118932071431425\\
474	0.011885751350509\\
475	0.0118782255973526\\
476	0.0118706369978939\\
477	0.011862993956694\\
478	0.0118553063813451\\
479	0.011847585935194\\
480	0.0118398463438389\\
481	0.0118321037640612\\
482	0.0118243772317733\\
483	0.0118166892182287\\
484	0.0118090663426918\\
485	0.0118015403351149\\
486	0.0117941494263459\\
487	0.0117869403838369\\
488	0.0117799705842822\\
489	0.0117732383175681\\
490	0.0117663792179719\\
491	0.0117593901270192\\
492	0.0117522676774891\\
493	0.0117450080963839\\
494	0.011737607262509\\
495	0.0117300606794943\\
496	0.0117223634082928\\
497	0.0117145098868796\\
498	0.0117064941319923\\
499	0.0116983096287429\\
500	0.0116899495581544\\
501	0.0116814070833733\\
502	0.0116726662353675\\
503	0.0116637088558584\\
504	0.0116544783696465\\
505	0.0116450354036406\\
506	0.0116353288008985\\
507	0.0116253266968029\\
508	0.011614997017994\\
509	0.0116043005947005\\
510	0.0115932146103219\\
511	0.0115817144396793\\
512	0.0115697736648359\\
513	0.0115573631441542\\
514	0.0115444291593998\\
515	0.011530730753585\\
516	0.0115164833331356\\
517	0.0115016502335145\\
518	0.0114861915172601\\
519	0.0114700635708654\\
520	0.0114532192088373\\
521	0.0114356071218876\\
522	0.0114171709926675\\
523	0.0113978448271879\\
524	0.0113775563957385\\
525	0.0113564349576774\\
526	0.0113343224781104\\
527	0.0113109260008469\\
528	0.0112861424734332\\
529	0.0112598578158543\\
530	0.0112319652088311\\
531	0.0112022911603164\\
532	0.0111706140006836\\
533	0.0111365823150043\\
534	0.0110859869924623\\
535	0.0110252202971323\\
536	0.0109613291311454\\
537	0.0108936236909409\\
538	0.0108048965222318\\
539	0.0106439683502646\\
540	0.0104795420851456\\
541	0.010312000564974\\
542	0.0101415474699112\\
543	0.010033504946196\\
544	0.0099750691464064\\
545	0.00992139945061239\\
546	0.00987382224989625\\
547	0.00983353511884946\\
548	0.00979560651967778\\
549	0.00975836306835898\\
550	0.0097216642086969\\
551	0.00968520430602083\\
552	0.00964845595521371\\
553	0.00961118730246861\\
554	0.00957299316228189\\
555	0.0095343050149351\\
556	0.00949509657678232\\
557	0.00945364062129668\\
558	0.00941189115537966\\
559	0.00937000851108659\\
560	0.00932945475487657\\
561	0.00928923880376672\\
562	0.00924829214340743\\
563	0.00920656941793769\\
564	0.00916395871585244\\
565	0.00912039783292629\\
566	0.00907501404237021\\
567	0.00902888639429137\\
568	0.00897816611491892\\
569	0.00892821669162519\\
570	0.00888337075780473\\
571	0.00883890057749335\\
572	0.00879372198462122\\
573	0.00870342529973373\\
574	0.00855024016950544\\
575	0.00804768227117276\\
576	0.00774342736384662\\
577	0.00766942347551535\\
578	0.0075972472245315\\
579	0.00752396298911323\\
580	0.00744949243362982\\
581	0.0073738054700987\\
582	0.00729687377624669\\
583	0.00721866834500435\\
584	0.00713915872884316\\
585	0.00705831174958944\\
586	0.00697608824927152\\
587	0.00689243464330991\\
588	0.00680726093981205\\
589	0.00672038360986785\\
590	0.00663137734622847\\
591	0.00653960315341497\\
592	0.00644346597070806\\
593	0.00633880868635237\\
594	0.00621472953405217\\
595	0.00604257880714497\\
596	0.00574702137257038\\
597	0.0051299581010561\\
598	0.00367057462141144\\
599	0\\
600	0\\
};
\addplot [color=red!80!mycolor19,solid,forget plot]
  table[row sep=crcr]{%
1	0.0124755144686956\\
2	0.0124755095498818\\
3	0.0124755045428053\\
4	0.0124754994458837\\
5	0.0124754942575062\\
6	0.0124754889760332\\
7	0.0124754835997957\\
8	0.0124754781270948\\
9	0.0124754725562011\\
10	0.0124754668853543\\
11	0.0124754611127628\\
12	0.0124754552366025\\
13	0.0124754492550169\\
14	0.0124754431661163\\
15	0.0124754369679769\\
16	0.0124754306586408\\
17	0.0124754242361148\\
18	0.01247541769837\\
19	0.0124754110433413\\
20	0.0124754042689267\\
21	0.0124753973729863\\
22	0.0124753903533421\\
23	0.012475383207777\\
24	0.0124753759340344\\
25	0.0124753685298171\\
26	0.0124753609927868\\
27	0.0124753533205634\\
28	0.0124753455107243\\
29	0.0124753375608034\\
30	0.0124753294682904\\
31	0.0124753212306302\\
32	0.012475312845222\\
33	0.0124753043094183\\
34	0.0124752956205244\\
35	0.0124752867757972\\
36	0.0124752777724447\\
37	0.0124752686076248\\
38	0.0124752592784445\\
39	0.0124752497819593\\
40	0.0124752401151716\\
41	0.0124752302750306\\
42	0.0124752202584306\\
43	0.0124752100622104\\
44	0.0124751996831525\\
45	0.0124751891179814\\
46	0.0124751783633635\\
47	0.012475167415905\\
48	0.0124751562721519\\
49	0.0124751449285882\\
50	0.0124751333816349\\
51	0.0124751216276492\\
52	0.012475109662923\\
53	0.0124750974836821\\
54	0.0124750850860844\\
55	0.0124750724662194\\
56	0.0124750596201066\\
57	0.0124750465436943\\
58	0.0124750332328585\\
59	0.012475019683401\\
60	0.0124750058910492\\
61	0.0124749918514535\\
62	0.0124749775601868\\
63	0.012474963012743\\
64	0.0124749482045352\\
65	0.0124749331308945\\
66	0.0124749177870687\\
67	0.0124749021682205\\
68	0.0124748862694263\\
69	0.0124748700856742\\
70	0.0124748536118632\\
71	0.0124748368428006\\
72	0.0124748197732012\\
73	0.0124748023976853\\
74	0.012474784710777\\
75	0.0124747667069027\\
76	0.0124747483803891\\
77	0.0124747297254614\\
78	0.0124747107362418\\
79	0.0124746914067474\\
80	0.0124746717308886\\
81	0.0124746517024667\\
82	0.0124746313151724\\
83	0.0124746105625838\\
84	0.0124745894381641\\
85	0.01247456793526\\
86	0.0124745460470991\\
87	0.0124745237667881\\
88	0.0124745010873108\\
89	0.0124744780015254\\
90	0.0124744545021628\\
91	0.012474430581824\\
92	0.0124744062329779\\
93	0.0124743814479589\\
94	0.0124743562189647\\
95	0.0124743305380536\\
96	0.012474304397142\\
97	0.0124742777880021\\
98	0.0124742507022595\\
99	0.0124742231313898\\
100	0.012474195066717\\
101	0.0124741664994098\\
102	0.0124741374204796\\
103	0.0124741078207774\\
104	0.0124740776909908\\
105	0.0124740470216413\\
106	0.0124740158030812\\
107	0.012473984025491\\
108	0.0124739516788757\\
109	0.0124739187530621\\
110	0.0124738852376955\\
111	0.0124738511222367\\
112	0.0124738163959584\\
113	0.012473781047942\\
114	0.0124737450670742\\
115	0.0124737084420435\\
116	0.0124736711613368\\
117	0.0124736332132357\\
118	0.0124735945858129\\
119	0.0124735552669285\\
120	0.0124735152442262\\
121	0.0124734745051293\\
122	0.0124734330368373\\
123	0.0124733908263212\\
124	0.0124733478603202\\
125	0.0124733041253369\\
126	0.0124732596076337\\
127	0.012473214293228\\
128	0.0124731681678882\\
129	0.0124731212171292\\
130	0.0124730734262081\\
131	0.0124730247801191\\
132	0.0124729752635894\\
133	0.0124729248610741\\
134	0.0124728735567517\\
135	0.0124728213345188\\
136	0.0124727681779855\\
137	0.0124727140704701\\
138	0.0124726589949939\\
139	0.0124726029342761\\
140	0.0124725458707284\\
141	0.0124724877864496\\
142	0.0124724286632197\\
143	0.0124723684824948\\
144	0.0124723072254011\\
145	0.0124722448727291\\
146	0.0124721814049275\\
147	0.0124721168020975\\
148	0.0124720510439864\\
149	0.0124719841099812\\
150	0.0124719159791028\\
151	0.012471846629999\\
152	0.0124717760409379\\
153	0.0124717041898016\\
154	0.0124716310540791\\
155	0.0124715566108595\\
156	0.0124714808368246\\
157	0.0124714037082422\\
158	0.0124713252009584\\
159	0.0124712452903905\\
160	0.0124711639515191\\
161	0.0124710811588806\\
162	0.0124709968865593\\
163	0.0124709111081795\\
164	0.0124708237968973\\
165	0.0124707349253925\\
166	0.0124706444658599\\
167	0.0124705523900012\\
168	0.0124704586690159\\
169	0.0124703632735927\\
170	0.0124702661739004\\
171	0.012470167339579\\
172	0.0124700667397299\\
173	0.012469964342907\\
174	0.0124698601171067\\
175	0.0124697540297581\\
176	0.0124696460477132\\
177	0.0124695361372367\\
178	0.0124694242639958\\
179	0.0124693103930494\\
180	0.0124691944888377\\
181	0.0124690765151712\\
182	0.0124689564352197\\
183	0.0124688342115012\\
184	0.01246870980587\\
185	0.0124685831795059\\
186	0.0124684542929014\\
187	0.0124683231058504\\
188	0.0124681895774356\\
189	0.0124680536660162\\
190	0.0124679153292151\\
191	0.0124677745239063\\
192	0.0124676312062013\\
193	0.0124674853314363\\
194	0.0124673368541583\\
195	0.0124671857281114\\
196	0.0124670319062229\\
197	0.0124668753405888\\
198	0.0124667159824595\\
199	0.0124665537822249\\
200	0.0124663886893995\\
201	0.012466220652607\\
202	0.0124660496195648\\
203	0.0124658755370684\\
204	0.012465698350975\\
205	0.0124655180061875\\
206	0.0124653344466377\\
207	0.0124651476152693\\
208	0.012464957454021\\
209	0.0124647639038086\\
210	0.0124645669045079\\
211	0.0124643663949358\\
212	0.0124641623128326\\
213	0.0124639545948428\\
214	0.0124637431764965\\
215	0.0124635279921896\\
216	0.0124633089751648\\
217	0.0124630860574907\\
218	0.0124628591700425\\
219	0.0124626282424805\\
220	0.0124623932032292\\
221	0.0124621539794565\\
222	0.0124619104970512\\
223	0.0124616626806014\\
224	0.0124614104533717\\
225	0.0124611537372806\\
226	0.0124608924528771\\
227	0.0124606265193171\\
228	0.0124603558543393\\
229	0.0124600803742411\\
230	0.012459799993853\\
231	0.0124595146265143\\
232	0.0124592241840467\\
233	0.0124589285767284\\
234	0.0124586277132677\\
235	0.0124583215007757\\
236	0.0124580098447393\\
237	0.0124576926489928\\
238	0.0124573698156898\\
239	0.0124570412452744\\
240	0.0124567068364517\\
241	0.0124563664861581\\
242	0.012456020089531\\
243	0.0124556675398781\\
244	0.0124553087286456\\
245	0.0124549435453872\\
246	0.0124545718777309\\
247	0.0124541936113465\\
248	0.0124538086299123\\
249	0.0124534168150808\\
250	0.0124530180464441\\
251	0.0124526122014992\\
252	0.0124521991556118\\
253	0.0124517787819804\\
254	0.0124513509515991\\
255	0.0124509155332203\\
256	0.0124504723933165\\
257	0.0124500213960417\\
258	0.0124495624031916\\
259	0.0124490952741642\\
260	0.0124486198659188\\
261	0.0124481360329345\\
262	0.0124476436271686\\
263	0.0124471424980134\\
264	0.0124466324922534\\
265	0.0124461134540206\\
266	0.0124455852247502\\
267	0.0124450476431348\\
268	0.012444500545078\\
269	0.0124439437636476\\
270	0.012443377129028\\
271	0.0124428004684709\\
272	0.012442213606247\\
273	0.0124416163635948\\
274	0.0124410085586706\\
275	0.012440390006496\\
276	0.0124397605189058\\
277	0.0124391199044944\\
278	0.0124384679685614\\
279	0.0124378045130567\\
280	0.0124371293365243\\
281	0.0124364422340456\\
282	0.0124357429971812\\
283	0.0124350314139127\\
284	0.0124343072685826\\
285	0.0124335703418339\\
286	0.0124328204105486\\
287	0.0124320572477854\\
288	0.0124312806227157\\
289	0.0124304903005596\\
290	0.0124296860425203\\
291	0.0124288676057178\\
292	0.012428034743121\\
293	0.0124271872034799\\
294	0.0124263247312553\\
295	0.0124254470665486\\
296	0.0124245539450301\\
297	0.0124236450978663\\
298	0.0124227202516459\\
299	0.012421779128305\\
300	0.0124208214450511\\
301	0.0124198469142856\\
302	0.0124188552435263\\
303	0.0124178461353272\\
304	0.0124168192871985\\
305	0.0124157743915246\\
306	0.0124147111354816\\
307	0.0124136292009532\\
308	0.0124125282644455\\
309	0.0124114079970009\\
310	0.0124102680641107\\
311	0.0124091081256264\\
312	0.0124079278356704\\
313	0.0124067268425448\\
314	0.0124055047886397\\
315	0.0124042613103402\\
316	0.0124029960379324\\
317	0.0124017085955083\\
318	0.0124003986008694\\
319	0.0123990656654303\\
320	0.0123977093941203\\
321	0.0123963293852845\\
322	0.0123949252305847\\
323	0.0123934965148989\\
324	0.0123920428162207\\
325	0.0123905637055579\\
326	0.0123890587468315\\
327	0.0123875274967739\\
328	0.0123859695048278\\
329	0.0123843843130454\\
330	0.0123827714559882\\
331	0.0123811304606287\\
332	0.0123794608462527\\
333	0.0123777621243635\\
334	0.0123760337985872\\
335	0.0123742753645745\\
336	0.0123724863098848\\
337	0.0123706661138055\\
338	0.0123688142469559\\
339	0.0123669301701858\\
340	0.0123650133311963\\
341	0.012363063153758\\
342	0.0123610790027423\\
343	0.0123590600695378\\
344	0.0123570055652523\\
345	0.0123549159404961\\
346	0.012352790636389\\
347	0.0123506290894347\\
348	0.0123484307327675\\
349	0.0123461949989316\\
350	0.0123439213268083\\
351	0.0123416091799461\\
352	0.0123392580965572\\
353	0.0123368678280272\\
354	0.0123344387262578\\
355	0.0123319728342118\\
356	0.0123294630371904\\
357	0.0123269080524546\\
358	0.0123243070641003\\
359	0.0123216592407893\\
360	0.0123189637353936\\
361	0.0123162196846282\\
362	0.0123134262086713\\
363	0.0123105824107722\\
364	0.012307687376846\\
365	0.0123047401750551\\
366	0.012301739855378\\
367	0.0122986854491638\\
368	0.0122955759686743\\
369	0.012292410406613\\
370	0.0122891877356416\\
371	0.0122859069078851\\
372	0.0122825668544261\\
373	0.0122791664847915\\
374	0.0122757046864314\\
375	0.012272180324195\\
376	0.012268592239807\\
377	0.0122649392513482\\
378	0.0122612201527471\\
379	0.0122574337132903\\
380	0.0122535786771605\\
381	0.012249653763014\\
382	0.0122456576636126\\
383	0.0122415890455273\\
384	0.0122374465489363\\
385	0.0122332287875445\\
386	0.0122289343486576\\
387	0.0122245617934523\\
388	0.0122201096574936\\
389	0.0122155764515605\\
390	0.0122109606628598\\
391	0.0122062607567266\\
392	0.0122014751789435\\
393	0.0121966023588697\\
394	0.0121916407136976\\
395	0.0121865886544708\\
396	0.012181444595367\\
397	0.0121762069702899\\
398	0.012170874221064\\
399	0.0121654448019182\\
400	0.0121599172133104\\
401	0.012154290021758\\
402	0.0121485618844481\\
403	0.0121427315797532\\
404	0.0121367980450469\\
405	0.0121307604235598\\
406	0.0121246181225056\\
407	0.0121183708856747\\
408	0.0121120188864918\\
409	0.0121055628579482\\
410	0.0120990043174232\\
411	0.0120923462678039\\
412	0.0120855953355706\\
413	0.0120787688313277\\
414	0.0120744561000278\\
415	0.0120714358844984\\
416	0.0120683651912251\\
417	0.0120652433232628\\
418	0.0120620695816869\\
419	0.0120588432659593\\
420	0.0120555636668341\\
421	0.0120522300525845\\
422	0.0120488417032285\\
423	0.0120453979485009\\
424	0.0120418982987207\\
425	0.0120383422999723\\
426	0.0120347295391993\\
427	0.0120310596498969\\
428	0.0120273323184802\\
429	0.0120235472914195\\
430	0.0120197043832394\\
431	0.0120158034854813\\
432	0.0120118445767341\\
433	0.0120078277340416\\
434	0.0120037531461588\\
435	0.0119996211281254\\
436	0.0119954321377541\\
437	0.0119911867943202\\
438	0.0119868858997872\\
439	0.0119825304629599\\
440	0.0119781217270309\\
441	0.011973661201072\\
442	0.0119691506961265\\
443	0.0119645923666888\\
444	0.0119599887585186\\
445	0.0119553428639312\\
446	0.0119506581859499\\
447	0.0119459388129916\\
448	0.0119411895061344\\
449	0.0119364158011844\\
450	0.0119316241279064\\
451	0.0119268219529787\\
452	0.0119220179571923\\
453	0.0119172222490321\\
454	0.0119124466236521\\
455	0.0119077048423679\\
456	0.0119030130184252\\
457	0.0118983901307399\\
458	0.0118938587590958\\
459	0.011889446192401\\
460	0.0118851858737732\\
461	0.0118811169339689\\
462	0.0118770042570216\\
463	0.0118728278465224\\
464	0.0118685871501279\\
465	0.0118642816185919\\
466	0.0118599107068718\\
467	0.011855473870292\\
468	0.0118509705592331\\
469	0.0118464002121612\\
470	0.0118417622466001\\
471	0.0118370560475929\\
472	0.0118322809526955\\
473	0.0118274362327803\\
474	0.0118225210670258\\
475	0.0118175345110285\\
476	0.0118124754567716\\
477	0.0118073425830492\\
478	0.0118021342960878\\
479	0.0117968486709527\\
480	0.0117914832566156\\
481	0.0117860349795989\\
482	0.0117805000473632\\
483	0.0117748737599653\\
484	0.0117691502756782\\
485	0.0117633223154481\\
486	0.0117573807602555\\
487	0.0117513139920008\\
488	0.0117451066550057\\
489	0.0117387405664958\\
490	0.0117322088193201\\
491	0.0117255039213907\\
492	0.0117185745061438\\
493	0.0117114499942753\\
494	0.0117041309699967\\
495	0.0116966078567005\\
496	0.0116888708328851\\
497	0.0116809116505156\\
498	0.011672712298576\\
499	0.0116642605909642\\
500	0.0116555467018985\\
501	0.0116465673322376\\
502	0.0116372753564269\\
503	0.0116276236993866\\
504	0.0116174137163944\\
505	0.01160684827428\\
506	0.0115959065251878\\
507	0.0115845661206818\\
508	0.0115728033240156\\
509	0.011560592925425\\
510	0.0115479073256344\\
511	0.011534717232526\\
512	0.011520990851716\\
513	0.0115066928470181\\
514	0.0114918105639298\\
515	0.011476571160296\\
516	0.0114606154885302\\
517	0.0114438924086322\\
518	0.0114263460681601\\
519	0.0114079149735465\\
520	0.0113885271937836\\
521	0.0113681057030035\\
522	0.0113465699163814\\
523	0.0113238029023321\\
524	0.0112996753806748\\
525	0.0112740738089133\\
526	0.011246880896024\\
527	0.0112179085057993\\
528	0.0111867755097057\\
529	0.011153297647489\\
530	0.0110967843545523\\
531	0.0110369415962543\\
532	0.0109738639105913\\
533	0.0109065470699926\\
534	0.0107799680402887\\
535	0.0106210329400423\\
536	0.0104592225649713\\
537	0.0102945757697561\\
538	0.0101492830449432\\
539	0.0100944462849399\\
540	0.0100443356891933\\
541	0.00999949231265718\\
542	0.00996146131563589\\
543	0.00992717745260455\\
544	0.00989358898635408\\
545	0.00986053639053389\\
546	0.00982772065993645\\
547	0.00979464812207893\\
548	0.00976113312255739\\
549	0.00972709243347414\\
550	0.00969238968500708\\
551	0.00965664011152814\\
552	0.00962028708606027\\
553	0.00958312670422892\\
554	0.0095437605781611\\
555	0.00950437567988051\\
556	0.00946501730068039\\
557	0.00942772980228527\\
558	0.00938994916628365\\
559	0.00935147416764861\\
560	0.00931221857669507\\
561	0.00927211919806968\\
562	0.00923114718594447\\
563	0.00918926919285994\\
564	0.00914561773113805\\
565	0.00910110002904921\\
566	0.00905202071254307\\
567	0.00900385292250346\\
568	0.00896159721627496\\
569	0.00891882226376778\\
570	0.00887527973107765\\
571	0.00883089342360094\\
572	0.00871047700603365\\
573	0.00836591746156936\\
574	0.00788469092837864\\
575	0.00781023316667708\\
576	0.0077403005223293\\
577	0.00766933232583592\\
578	0.00759722494416737\\
579	0.00752395143952015\\
580	0.00744948595110907\\
581	0.00737380194212672\\
582	0.00729687197614712\\
583	0.00721866750263022\\
584	0.00713915837631075\\
585	0.00705831162252266\\
586	0.00697608821211005\\
587	0.00689243463543015\\
588	0.00680726093889688\\
589	0.00672038360986784\\
590	0.00663137734622847\\
591	0.00653960315341497\\
592	0.00644346597070806\\
593	0.00633880868635238\\
594	0.00621472953405218\\
595	0.00604257880714497\\
596	0.00574702137257038\\
597	0.0051299581010561\\
598	0.00367057462141144\\
599	0\\
600	0\\
};
\addplot [color=red,solid,forget plot]
  table[row sep=crcr]{%
1	0.0121486575729673\\
2	0.0121486553573578\\
3	0.0121486531018636\\
4	0.0121486508057667\\
5	0.0121486484683363\\
6	0.0121486460888284\\
7	0.0121486436664856\\
8	0.0121486412005368\\
9	0.0121486386901972\\
10	0.0121486361346677\\
11	0.0121486335331349\\
12	0.0121486308847707\\
13	0.0121486281887323\\
14	0.0121486254441614\\
15	0.0121486226501845\\
16	0.0121486198059123\\
17	0.0121486169104396\\
18	0.0121486139628446\\
19	0.0121486109621892\\
20	0.0121486079075184\\
21	0.0121486047978599\\
22	0.0121486016322239\\
23	0.0121485984096028\\
24	0.012148595128971\\
25	0.0121485917892843\\
26	0.0121485883894798\\
27	0.0121485849284752\\
28	0.0121485814051691\\
29	0.0121485778184402\\
30	0.0121485741671467\\
31	0.0121485704501265\\
32	0.0121485666661968\\
33	0.012148562814153\\
34	0.0121485588927693\\
35	0.0121485549007975\\
36	0.0121485508369671\\
37	0.0121485466999847\\
38	0.0121485424885336\\
39	0.0121485382012736\\
40	0.01214853383684\\
41	0.0121485293938438\\
42	0.0121485248708709\\
43	0.012148520266482\\
44	0.0121485155792115\\
45	0.0121485108075676\\
46	0.0121485059500319\\
47	0.0121485010050582\\
48	0.0121484959710728\\
49	0.0121484908464736\\
50	0.0121484856296297\\
51	0.0121484803188807\\
52	0.0121484749125366\\
53	0.0121484694088767\\
54	0.0121484638061495\\
55	0.0121484581025719\\
56	0.0121484522963287\\
57	0.0121484463855722\\
58	0.0121484403684213\\
59	0.0121484342429609\\
60	0.0121484280072418\\
61	0.0121484216592795\\
62	0.0121484151970538\\
63	0.0121484086185081\\
64	0.012148401921549\\
65	0.0121483951040451\\
66	0.012148388163827\\
67	0.0121483810986861\\
68	0.0121483739063739\\
69	0.0121483665846017\\
70	0.0121483591310396\\
71	0.0121483515433154\\
72	0.0121483438190148\\
73	0.0121483359556796\\
74	0.0121483279508076\\
75	0.0121483198018514\\
76	0.012148311506218\\
77	0.0121483030612675\\
78	0.0121482944643127\\
79	0.0121482857126178\\
80	0.0121482768033982\\
81	0.0121482677338187\\
82	0.0121482585009935\\
83	0.0121482491019845\\
84	0.0121482395338012\\
85	0.0121482297933989\\
86	0.0121482198776783\\
87	0.0121482097834842\\
88	0.0121481995076049\\
89	0.0121481890467708\\
90	0.0121481783976533\\
91	0.0121481675568641\\
92	0.0121481565209539\\
93	0.0121481452864115\\
94	0.0121481338496623\\
95	0.0121481222070675\\
96	0.0121481103549229\\
97	0.0121480982894575\\
98	0.0121480860068328\\
99	0.0121480735031409\\
100	0.0121480607744037\\
101	0.0121480478165718\\
102	0.0121480346255227\\
103	0.0121480211970599\\
104	0.0121480075269112\\
105	0.0121479936107278\\
106	0.0121479794440827\\
107	0.0121479650224692\\
108	0.0121479503412995\\
109	0.0121479353959035\\
110	0.0121479201815268\\
111	0.0121479046933299\\
112	0.012147888926386\\
113	0.0121478728756798\\
114	0.0121478565361059\\
115	0.012147839902467\\
116	0.0121478229694724\\
117	0.0121478057317362\\
118	0.012147788183776\\
119	0.0121477703200104\\
120	0.012147752134758\\
121	0.0121477336222352\\
122	0.0121477147765543\\
123	0.012147695591722\\
124	0.0121476760616371\\
125	0.0121476561800888\\
126	0.0121476359407548\\
127	0.012147615337199\\
128	0.0121475943628699\\
129	0.0121475730110979\\
130	0.012147551275094\\
131	0.0121475291479469\\
132	0.0121475066226213\\
133	0.0121474836919554\\
134	0.0121474603486589\\
135	0.0121474365853104\\
136	0.0121474123943552\\
137	0.012147387768103\\
138	0.0121473626987254\\
139	0.0121473371782534\\
140	0.0121473111985749\\
141	0.0121472847514319\\
142	0.0121472578284186\\
143	0.0121472304209779\\
144	0.0121472025203992\\
145	0.0121471741178156\\
146	0.012147145204201\\
147	0.0121471157703674\\
148	0.0121470858069616\\
149	0.0121470553044629\\
150	0.0121470242531796\\
151	0.0121469926432462\\
152	0.0121469604646201\\
153	0.0121469277070787\\
154	0.0121468943602158\\
155	0.0121468604134388\\
156	0.012146825855965\\
157	0.0121467906768184\\
158	0.0121467548648263\\
159	0.0121467184086155\\
160	0.0121466812966091\\
161	0.0121466435170225\\
162	0.0121466050578601\\
163	0.0121465659069113\\
164	0.0121465260517467\\
165	0.0121464854797141\\
166	0.0121464441779348\\
167	0.0121464021332994\\
168	0.0121463593324638\\
169	0.0121463157618449\\
170	0.0121462714076164\\
171	0.0121462262557045\\
172	0.0121461802917835\\
173	0.0121461335012714\\
174	0.0121460858693252\\
175	0.0121460373808365\\
176	0.0121459880204266\\
177	0.0121459377724416\\
178	0.0121458866209479\\
179	0.012145834549727\\
180	0.0121457815422705\\
181	0.012145727581775\\
182	0.0121456726511367\\
183	0.0121456167329465\\
184	0.0121455598094841\\
185	0.012145501862713\\
186	0.0121454428742745\\
187	0.0121453828254823\\
188	0.0121453216973166\\
189	0.0121452594704183\\
190	0.0121451961250829\\
191	0.0121451316412545\\
192	0.0121450659985196\\
193	0.012144999176101\\
194	0.0121449311528509\\
195	0.012144861907245\\
196	0.0121447914173756\\
197	0.0121447196609447\\
198	0.0121446466152575\\
199	0.0121445722572154\\
200	0.0121444965633087\\
201	0.0121444195096095\\
202	0.0121443410717644\\
203	0.0121442612249874\\
204	0.0121441799440517\\
205	0.0121440972032826\\
206	0.0121440129765493\\
207	0.012143927237257\\
208	0.0121438399583394\\
209	0.0121437511122495\\
210	0.012143660670952\\
211	0.0121435686059148\\
212	0.0121434748880999\\
213	0.0121433794879549\\
214	0.0121432823754045\\
215	0.0121431835198406\\
216	0.012143082890114\\
217	0.0121429804545242\\
218	0.0121428761808106\\
219	0.0121427700361425\\
220	0.0121426619871094\\
221	0.0121425519997109\\
222	0.0121424400393465\\
223	0.0121423260708059\\
224	0.0121422100582577\\
225	0.0121420919652392\\
226	0.0121419717546459\\
227	0.0121418493887199\\
228	0.0121417248290392\\
229	0.0121415980365061\\
230	0.0121414689713361\\
231	0.012141337593046\\
232	0.0121412038604417\\
233	0.0121410677316068\\
234	0.0121409291638899\\
235	0.0121407881138926\\
236	0.0121406445374561\\
237	0.0121404983896496\\
238	0.012140349624756\\
239	0.0121401981962597\\
240	0.0121400440568328\\
241	0.0121398871583213\\
242	0.0121397274517316\\
243	0.0121395648872165\\
244	0.0121393994140608\\
245	0.0121392309806669\\
246	0.01213905953454\\
247	0.0121388850222737\\
248	0.0121387073895343\\
249	0.012138526581046\\
250	0.012138342540575\\
251	0.0121381552109141\\
252	0.0121379645338661\\
253	0.0121377704502283\\
254	0.0121375728997756\\
255	0.012137371821244\\
256	0.0121371671523135\\
257	0.0121369588295912\\
258	0.0121367467885934\\
259	0.0121365309637287\\
260	0.0121363112882795\\
261	0.0121360876943841\\
262	0.0121358601130182\\
263	0.0121356284739764\\
264	0.0121353927058533\\
265	0.0121351527360242\\
266	0.0121349084906261\\
267	0.0121346598945373\\
268	0.0121344068713585\\
269	0.0121341493433917\\
270	0.0121338872316202\\
271	0.0121336204556882\\
272	0.012133348933879\\
273	0.0121330725830949\\
274	0.0121327913188345\\
275	0.0121325050551721\\
276	0.0121322137047349\\
277	0.0121319171786811\\
278	0.0121316153866772\\
279	0.0121313082368751\\
280	0.0121309956358896\\
281	0.0121306774887741\\
282	0.012130353698998\\
283	0.012130024168422\\
284	0.0121296887972746\\
285	0.0121293474841274\\
286	0.0121290001258704\\
287	0.0121286466176875\\
288	0.0121282868530308\\
289	0.0121279207235955\\
290	0.0121275481192944\\
291	0.0121271689282316\\
292	0.012126783036677\\
293	0.0121263903290391\\
294	0.0121259906878391\\
295	0.0121255839936839\\
296	0.012125170125239\\
297	0.0121247489592012\\
298	0.0121243203702714\\
299	0.0121238842311268\\
300	0.0121234404123932\\
301	0.0121229887826169\\
302	0.0121225292082366\\
303	0.0121220615535552\\
304	0.0121215856807112\\
305	0.0121211014496501\\
306	0.012120608718096\\
307	0.0121201073415223\\
308	0.012119597173123\\
309	0.0121190780637839\\
310	0.012118549862053\\
311	0.0121180124141118\\
312	0.0121174655637456\\
313	0.0121169091523144\\
314	0.0121163430187234\\
315	0.0121157669993936\\
316	0.0121151809282318\\
317	0.0121145846366014\\
318	0.0121139779532924\\
319	0.0121133607044909\\
320	0.0121127327137497\\
321	0.0121120938019569\\
322	0.0121114437873058\\
323	0.0121107824852631\\
324	0.0121101097085369\\
325	0.0121094252670445\\
326	0.0121087289678786\\
327	0.0121080206152725\\
328	0.0121073000105644\\
329	0.0121065669521591\\
330	0.0121058212354887\\
331	0.0121050626529694\\
332	0.0121042909939564\\
333	0.0121035060446943\\
334	0.0121027075882625\\
335	0.0121018954045155\\
336	0.0121010692700143\\
337	0.0121002289579479\\
338	0.0120993742380377\\
339	0.0120985048764186\\
340	0.0120976206354822\\
341	0.0120967212736718\\
342	0.0120958065452151\\
343	0.0120948761996008\\
344	0.012093929978778\\
345	0.0120929676110897\\
346	0.0120919888184208\\
347	0.0120909933158207\\
348	0.012089980811115\\
349	0.012088951004529\\
350	0.0120879035883678\\
351	0.0120868382468304\\
352	0.0120857546560703\\
353	0.0120846524846538\\
354	0.0120835313949953\\
355	0.0120823910504231\\
356	0.012081231162159\\
357	0.0120800514439489\\
358	0.0120788516087237\\
359	0.012077631368911\\
360	0.0120763904367823\\
361	0.012075128524837\\
362	0.0120738453462294\\
363	0.0120725406152422\\
364	0.0120712140478106\\
365	0.0120698653621035\\
366	0.0120684942791678\\
367	0.0120671005236422\\
368	0.0120656838245477\\
369	0.0120642439161646\\
370	0.0120627805390039\\
371	0.0120612934408844\\
372	0.0120597823781285\\
373	0.0120582471168876\\
374	0.0120566874346142\\
375	0.0120551031216976\\
376	0.0120534939832807\\
377	0.0120518598412809\\
378	0.0120502005366398\\
379	0.0120485159318281\\
380	0.0120468059136381\\
381	0.0120450703962999\\
382	0.0120433093249618\\
383	0.0120415226795811\\
384	0.0120397104792815\\
385	0.0120378727872352\\
386	0.0120360097161446\\
387	0.0120341214344039\\
388	0.0120322081730374\\
389	0.0120302702335255\\
390	0.0120283079966497\\
391	0.0120263219325096\\
392	0.0120243126118914\\
393	0.0120222807192014\\
394	0.0120202270672099\\
395	0.012018152613888\\
396	0.0120160584816315\\
397	0.0120139459791124\\
398	0.01201181662723\\
399	0.0120096721894834\\
400	0.0120075147067693\\
401	0.0120053465380903\\
402	0.0120031704083988\\
403	0.0120009894651167\\
404	0.0119988073452851\\
405	0.0119966282558997\\
406	0.011994457070941\\
407	0.0119922994503889\\
408	0.0119901619902689\\
409	0.0119880524212338\\
410	0.0119859798909106\\
411	0.0119839553847053\\
412	0.0119819922874344\\
413	0.0119801063862311\\
414	0.0119782292968253\\
415	0.0119763220281277\\
416	0.011974384243097\\
417	0.0119724156116453\\
418	0.011970415811962\\
419	0.0119683845318106\\
420	0.0119663214698275\\
421	0.0119642263376199\\
422	0.0119620988612987\\
423	0.0119599387837195\\
424	0.0119577458604926\\
425	0.0119555198610505\\
426	0.0119532605697376\\
427	0.0119509677869256\\
428	0.0119486413301049\\
429	0.0119462810349096\\
430	0.0119438867561095\\
431	0.0119414583685532\\
432	0.0119389957680382\\
433	0.011936498872076\\
434	0.0119339676204997\\
435	0.011931401975878\\
436	0.0119288019236768\\
437	0.0119261674720978\\
438	0.0119234986515062\\
439	0.0119207955133384\\
440	0.0119180581283537\\
441	0.0119152865840648\\
442	0.0119124809811352\\
443	0.0119096414284857\\
444	0.0119067680367877\\
445	0.0119038609099581\\
446	0.0119009201342003\\
447	0.0118979457639653\\
448	0.0118949378032542\\
449	0.0118918961843493\\
450	0.0118888207393674\\
451	0.0118857111624711\\
452	0.0118825669614717\\
453	0.0118793873959793\\
454	0.0118761713984931\\
455	0.0118729174761976\\
456	0.0118696235884254\\
457	0.0118662869932465\\
458	0.0118629040443387\\
459	0.0118594698713335\\
460	0.0118559777557352\\
461	0.0118524180554707\\
462	0.0118487883882409\\
463	0.0118450868974798\\
464	0.0118413116271318\\
465	0.0118374605350129\\
466	0.011833531339564\\
467	0.0118295216253688\\
468	0.0118254288369277\\
469	0.0118212502680508\\
470	0.0118169830505261\\
471	0.0118126241420442\\
472	0.0118081703132798\\
473	0.0118036181340029\\
474	0.0117989639578617\\
475	0.0117942039046307\\
476	0.0117893338363998\\
477	0.0117843493176102\\
478	0.0117792455353164\\
479	0.0117739890451117\\
480	0.0117685878562875\\
481	0.011763054615547\\
482	0.0117573831025024\\
483	0.0117515666660712\\
484	0.011745598212998\\
485	0.0117394702015855\\
486	0.011733174637755\\
487	0.011726703081249\\
488	0.011720046715736\\
489	0.0117131964735088\\
490	0.0117061423273602\\
491	0.011698872717233\\
492	0.0116912051018767\\
493	0.0116832812225449\\
494	0.0116751286653533\\
495	0.0116667339601145\\
496	0.0116580886877917\\
497	0.0116491982560047\\
498	0.0116400184802184\\
499	0.0116305330077348\\
500	0.0116207255076699\\
501	0.0116105782378705\\
502	0.0116000718340359\\
503	0.011589217928098\\
504	0.0115782249348314\\
505	0.0115667881795456\\
506	0.0115548799805786\\
507	0.0115424703773344\\
508	0.0115295268932274\\
509	0.0115160142552167\\
510	0.011501894127167\\
511	0.0114871248363757\\
512	0.0114716612082866\\
513	0.0114554550165393\\
514	0.0114384557029834\\
515	0.0114205961190073\\
516	0.0114018111999489\\
517	0.0113820243800961\\
518	0.011361151130595\\
519	0.0113391024612881\\
520	0.0113157492534324\\
521	0.0112909579438292\\
522	0.0112644557240849\\
523	0.0112362380528972\\
524	0.0112060556584573\\
525	0.0111705333741596\\
526	0.0111145800698175\\
527	0.0110560033300487\\
528	0.010993561345724\\
529	0.0109269264217574\\
530	0.0107758577895813\\
531	0.0106198319484714\\
532	0.0104611309642711\\
533	0.0103001006596234\\
534	0.0102103227226019\\
535	0.0101619564029839\\
536	0.0101183109325057\\
537	0.0100807661575314\\
538	0.0100494325687633\\
539	0.01001880946129\\
540	0.00998875324713256\\
541	0.00995900398450516\\
542	0.00992913045420666\\
543	0.00989885955338671\\
544	0.00986811562927428\\
545	0.00983682239612696\\
546	0.009804909422787\\
547	0.00977232236247919\\
548	0.00973862628930834\\
549	0.0097043343856294\\
550	0.00966920294108186\\
551	0.0096319250017616\\
552	0.0095945746685891\\
553	0.00955741345163244\\
554	0.00952230398503872\\
555	0.00948672101974251\\
556	0.00945050720506747\\
557	0.00941353068151448\\
558	0.0093757602167804\\
559	0.0093371699520098\\
560	0.00929773624990541\\
561	0.00925743407409785\\
562	0.00921558914890249\\
563	0.00917274139189328\\
564	0.00912551878140976\\
565	0.00907884223381979\\
566	0.00903816143795629\\
567	0.00899698969452701\\
568	0.00895504782782192\\
569	0.00891230184922443\\
570	0.00884837858667458\\
571	0.00872490508201813\\
572	0.0082159119842798\\
573	0.00794672081996317\\
574	0.00787889116878201\\
575	0.0078101379022131\\
576	0.00774029240671172\\
577	0.00766932877310036\\
578	0.00759722299306871\\
579	0.00752395037121544\\
580	0.00744948539355952\\
581	0.00737380167099611\\
582	0.0072968718557514\\
583	0.00721866745503845\\
584	0.00713915836018309\\
585	0.00705831161811151\\
586	0.00697608821124038\\
587	0.00689243463533689\\
588	0.00680726093889689\\
589	0.00672038360986785\\
590	0.00663137734622848\\
591	0.00653960315341498\\
592	0.00644346597070808\\
593	0.00633880868635238\\
594	0.00621472953405218\\
595	0.00604257880714498\\
596	0.00574702137257039\\
597	0.0051299581010561\\
598	0.00367057462141144\\
599	0\\
600	0\\
};
\addplot [color=mycolor20,solid,forget plot]
  table[row sep=crcr]{%
1	0.0120558961099599\\
2	0.0120558947994649\\
3	0.0120558934652485\\
4	0.0120558921068809\\
5	0.0120558907239243\\
6	0.0120558893159334\\
7	0.0120558878824543\\
8	0.0120558864230252\\
9	0.012055884937176\\
10	0.0120558834244277\\
11	0.0120558818842929\\
12	0.0120558803162754\\
13	0.0120558787198697\\
14	0.0120558770945613\\
15	0.0120558754398265\\
16	0.0120558737551318\\
17	0.0120558720399343\\
18	0.012055870293681\\
19	0.0120558685158091\\
20	0.0120558667057454\\
21	0.0120558648629064\\
22	0.0120558629866979\\
23	0.012055861076515\\
24	0.0120558591317419\\
25	0.0120558571517514\\
26	0.0120558551359052\\
27	0.012055853083553\\
28	0.0120558509940331\\
29	0.0120558488666716\\
30	0.0120558467007824\\
31	0.012055844495667\\
32	0.012055842250614\\
33	0.0120558399648993\\
34	0.0120558376377856\\
35	0.0120558352685222\\
36	0.0120558328563447\\
37	0.012055830400475\\
38	0.0120558279001205\\
39	0.0120558253544748\\
40	0.0120558227627162\\
41	0.0120558201240085\\
42	0.0120558174375002\\
43	0.0120558147023243\\
44	0.0120558119175981\\
45	0.0120558090824228\\
46	0.0120558061958833\\
47	0.0120558032570479\\
48	0.0120558002649681\\
49	0.0120557972186778\\
50	0.0120557941171936\\
51	0.0120557909595144\\
52	0.0120557877446206\\
53	0.0120557844714741\\
54	0.0120557811390181\\
55	0.0120557777461766\\
56	0.0120557742918539\\
57	0.0120557707749344\\
58	0.0120557671942824\\
59	0.0120557635487413\\
60	0.0120557598371337\\
61	0.0120557560582607\\
62	0.0120557522109017\\
63	0.0120557482938138\\
64	0.0120557443057315\\
65	0.0120557402453664\\
66	0.0120557361114066\\
67	0.0120557319025163\\
68	0.0120557276173357\\
69	0.0120557232544799\\
70	0.012055718812539\\
71	0.0120557142900776\\
72	0.0120557096856339\\
73	0.0120557049977197\\
74	0.0120557002248198\\
75	0.0120556953653914\\
76	0.0120556904178634\\
77	0.0120556853806366\\
78	0.0120556802520824\\
79	0.0120556750305425\\
80	0.0120556697143288\\
81	0.0120556643017221\\
82	0.0120556587909722\\
83	0.0120556531802971\\
84	0.0120556474678821\\
85	0.0120556416518797\\
86	0.0120556357304088\\
87	0.0120556297015541\\
88	0.0120556235633653\\
89	0.0120556173138568\\
90	0.0120556109510068\\
91	0.0120556044727567\\
92	0.0120555978770105\\
93	0.0120555911616341\\
94	0.0120555843244543\\
95	0.0120555773632589\\
96	0.0120555702757948\\
97	0.0120555630597685\\
98	0.0120555557128444\\
99	0.0120555482326444\\
100	0.0120555406167475\\
101	0.0120555328626881\\
102	0.0120555249679562\\
103	0.0120555169299959\\
104	0.0120555087462048\\
105	0.0120555004139332\\
106	0.0120554919304832\\
107	0.0120554832931077\\
108	0.0120554744990096\\
109	0.0120554655453411\\
110	0.0120554564292024\\
111	0.0120554471476408\\
112	0.0120554376976503\\
113	0.0120554280761696\\
114	0.0120554182800823\\
115	0.0120554083062147\\
116	0.0120553981513358\\
117	0.0120553878121556\\
118	0.0120553772853241\\
119	0.0120553665674305\\
120	0.012055355655002\\
121	0.0120553445445023\\
122	0.0120553332323309\\
123	0.0120553217148219\\
124	0.0120553099882425\\
125	0.0120552980487919\\
126	0.0120552858926003\\
127	0.0120552735157275\\
128	0.0120552609141612\\
129	0.0120552480838166\\
130	0.0120552350205342\\
131	0.0120552217200789\\
132	0.0120552081781385\\
133	0.0120551943903224\\
134	0.01205518035216\\
135	0.0120551660590993\\
136	0.0120551515065057\\
137	0.0120551366896599\\
138	0.0120551216037569\\
139	0.0120551062439044\\
140	0.0120550906051209\\
141	0.0120550746823342\\
142	0.0120550584703799\\
143	0.0120550419639998\\
144	0.0120550251578396\\
145	0.0120550080464481\\
146	0.0120549906242744\\
147	0.0120549728856671\\
148	0.0120549548248717\\
149	0.012054936436029\\
150	0.0120549177131734\\
151	0.0120548986502308\\
152	0.0120548792410164\\
153	0.012054859479233\\
154	0.0120548393584691\\
155	0.0120548188721965\\
156	0.012054798013768\\
157	0.012054776776416\\
158	0.0120547551532496\\
159	0.0120547331372528\\
160	0.0120547107212818\\
161	0.0120546878980632\\
162	0.0120546646601914\\
163	0.0120546410001261\\
164	0.0120546169101901\\
165	0.0120545923825667\\
166	0.0120545674092973\\
167	0.0120545419822784\\
168	0.0120545160932598\\
169	0.0120544897338411\\
170	0.0120544628954696\\
171	0.0120544355694372\\
172	0.0120544077468779\\
173	0.0120543794187645\\
174	0.0120543505759064\\
175	0.0120543212089458\\
176	0.0120542913083555\\
177	0.0120542608644354\\
178	0.0120542298673094\\
179	0.0120541983069224\\
180	0.012054166173037\\
181	0.0120541334552304\\
182	0.0120541001428908\\
183	0.0120540662252141\\
184	0.0120540316912007\\
185	0.0120539965296516\\
186	0.0120539607291653\\
187	0.0120539242781336\\
188	0.0120538871647387\\
189	0.0120538493769486\\
190	0.0120538109025139\\
191	0.0120537717289637\\
192	0.0120537318436016\\
193	0.0120536912335019\\
194	0.0120536498855052\\
195	0.0120536077862146\\
196	0.0120535649219915\\
197	0.0120535212789508\\
198	0.0120534768429571\\
199	0.0120534315996201\\
200	0.0120533855342899\\
201	0.0120533386320529\\
202	0.0120532908777265\\
203	0.0120532422558548\\
204	0.0120531927507038\\
205	0.0120531423462564\\
206	0.0120530910262073\\
207	0.0120530387739583\\
208	0.0120529855726128\\
209	0.0120529314049708\\
210	0.0120528762535235\\
211	0.012052820100448\\
212	0.012052762927602\\
213	0.0120527047165178\\
214	0.0120526454483968\\
215	0.0120525851041043\\
216	0.0120525236641628\\
217	0.0120524611087467\\
218	0.0120523974176763\\
219	0.0120523325704111\\
220	0.0120522665460444\\
221	0.0120521993232965\\
222	0.0120521308805085\\
223	0.0120520611956355\\
224	0.0120519902462407\\
225	0.0120519180094879\\
226	0.0120518444621354\\
227	0.0120517695805287\\
228	0.0120516933405936\\
229	0.0120516157178294\\
230	0.0120515366873012\\
231	0.012051456223633\\
232	0.0120513743010002\\
233	0.0120512908931221\\
234	0.0120512059732541\\
235	0.0120511195141804\\
236	0.0120510314882057\\
237	0.0120509418671476\\
238	0.0120508506223285\\
239	0.0120507577245672\\
240	0.012050663144171\\
241	0.0120505668509272\\
242	0.0120504688140943\\
243	0.012050369002394\\
244	0.012050267384002\\
245	0.0120501639265395\\
246	0.0120500585970638\\
247	0.01204995136206\\
248	0.0120498421874311\\
249	0.0120497310384893\\
250	0.0120496178799461\\
251	0.0120495026759034\\
252	0.0120493853898434\\
253	0.0120492659846191\\
254	0.0120491444224443\\
255	0.012049020664884\\
256	0.0120488946728441\\
257	0.0120487664065611\\
258	0.0120486358255922\\
259	0.0120485028888045\\
260	0.0120483675543648\\
261	0.012048229779729\\
262	0.0120480895216312\\
263	0.0120479467360728\\
264	0.012047801378312\\
265	0.0120476534028523\\
266	0.0120475027634316\\
267	0.0120473494130109\\
268	0.0120471933037629\\
269	0.0120470343870606\\
270	0.0120468726134655\\
271	0.0120467079327166\\
272	0.0120465402937177\\
273	0.0120463696445262\\
274	0.012046195932341\\
275	0.0120460191034902\\
276	0.0120458391034193\\
277	0.0120456558766785\\
278	0.0120454693669107\\
279	0.012045279516839\\
280	0.0120450862682538\\
281	0.0120448895620007\\
282	0.0120446893379672\\
283	0.0120444855350702\\
284	0.012044278091243\\
285	0.0120440669434223\\
286	0.0120438520275348\\
287	0.0120436332784844\\
288	0.0120434106301385\\
289	0.0120431840153146\\
290	0.0120429533657672\\
291	0.0120427186121737\\
292	0.0120424796841208\\
293	0.0120422365100906\\
294	0.0120419890174471\\
295	0.0120417371324214\\
296	0.0120414807800983\\
297	0.0120412198844013\\
298	0.0120409543680789\\
299	0.0120406841526896\\
300	0.0120404091585876\\
301	0.0120401293049076\\
302	0.0120398445095504\\
303	0.0120395546891677\\
304	0.0120392597591468\\
305	0.0120389596335958\\
306	0.0120386542253275\\
307	0.0120383434458443\\
308	0.0120380272053227\\
309	0.0120377054125969\\
310	0.0120373779751435\\
311	0.0120370447990648\\
312	0.0120367057890732\\
313	0.0120363608484745\\
314	0.0120360098791517\\
315	0.0120356527815486\\
316	0.0120352894546527\\
317	0.0120349197959794\\
318	0.0120345437015546\\
319	0.0120341610658988\\
320	0.0120337717820098\\
321	0.0120333757413469\\
322	0.0120329728338137\\
323	0.012032562947743\\
324	0.01203214596988\\
325	0.0120317217853672\\
326	0.0120312902777294\\
327	0.0120308513288588\\
328	0.0120304048190017\\
329	0.0120299506267452\\
330	0.0120294886290055\\
331	0.0120290187010171\\
332	0.0120285407163239\\
333	0.012028054546772\\
334	0.0120275600625041\\
335	0.0120270571319576\\
336	0.0120265456218643\\
337	0.0120260253972545\\
338	0.0120254963214651\\
339	0.0120249582561545\\
340	0.0120244110613285\\
341	0.0120238545953867\\
342	0.0120232887151924\\
343	0.0120227132761363\\
344	0.0120221281322192\\
345	0.0120215331364493\\
346	0.0120209281410148\\
347	0.0120203129974856\\
348	0.0120196875570469\\
349	0.0120190516707685\\
350	0.012018405189907\\
351	0.0120177479662231\\
352	0.012017079852279\\
353	0.012016400701676\\
354	0.012015710369332\\
355	0.0120150087122074\\
356	0.0120142955878684\\
357	0.0120135708545346\\
358	0.0120128343712579\\
359	0.0120120859981134\\
360	0.012011325596402\\
361	0.0120105530288663\\
362	0.0120097681599188\\
363	0.0120089708558844\\
364	0.012008160985257\\
365	0.0120073384189709\\
366	0.0120065030306867\\
367	0.0120056546970943\\
368	0.0120047932982301\\
369	0.0120039187178122\\
370	0.0120030308435907\\
371	0.0120021295677155\\
372	0.0120012147871207\\
373	0.0120002864039242\\
374	0.011999344325845\\
375	0.0119983884666335\\
376	0.0119974187465173\\
377	0.0119964350926588\\
378	0.0119954374396229\\
379	0.0119944257298526\\
380	0.0119933999141482\\
381	0.0119923599521462\\
382	0.0119913058127918\\
383	0.0119902374747974\\
384	0.0119891549270788\\
385	0.0119880581691568\\
386	0.0119869472115111\\
387	0.0119858220758679\\
388	0.0119846827954001\\
389	0.0119835294148125\\
390	0.0119823619902784\\
391	0.0119811805891868\\
392	0.0119799852896457\\
393	0.0119787761796792\\
394	0.0119775533560361\\
395	0.0119763169225096\\
396	0.0119750669876436\\
397	0.0119738036616751\\
398	0.0119725270524897\\
399	0.0119712372603143\\
400	0.0119699343707916\\
401	0.0119686184460676\\
402	0.0119672895133641\\
403	0.0119659475503653\\
404	0.0119645924665719\\
405	0.011963224079575\\
406	0.0119618420849918\\
407	0.0119604460186275\\
408	0.0119590352092387\\
409	0.011957608719507\\
410	0.0119561652687368\\
411	0.0119547031146815\\
412	0.0119532198291131\\
413	0.0119517118993435\\
414	0.0119501776903453\\
415	0.0119486167686459\\
416	0.0119470286925225\\
417	0.0119454130114934\\
418	0.0119437692657273\\
419	0.0119420969853653\\
420	0.0119403956897872\\
421	0.011938664886804\\
422	0.0119369040718153\\
423	0.0119351127267194\\
424	0.0119332903189692\\
425	0.0119314363006095\\
426	0.0119295501073336\\
427	0.0119276311572083\\
428	0.0119256788502798\\
429	0.0119236925689134\\
430	0.0119216716765493\\
431	0.0119196155163274\\
432	0.0119175234095643\\
433	0.0119153946540672\\
434	0.0119132285222656\\
435	0.0119110242591409\\
436	0.011908781079933\\
437	0.0119064981675982\\
438	0.0119041746699945\\
439	0.0119018096967643\\
440	0.0118994023158834\\
441	0.0118969515498391\\
442	0.0118944563713856\\
443	0.0118919156988048\\
444	0.0118893283905572\\
445	0.011886693239206\\
446	0.0118840089648164\\
447	0.0118812742099465\\
448	0.0118784875460524\\
449	0.0118756473944823\\
450	0.0118727520450643\\
451	0.0118697996850693\\
452	0.0118667883952393\\
453	0.0118637161473063\\
454	0.0118605808038732\\
455	0.0118573801216766\\
456	0.0118541117594177\\
457	0.0118507732904503\\
458	0.0118473622188061\\
459	0.0118438759981468\\
460	0.0118403120792765\\
461	0.0118366680908792\\
462	0.0118329415779152\\
463	0.0118291299510877\\
464	0.0118252304152145\\
465	0.0118212029068933\\
466	0.0118170834181897\\
467	0.0118128700235029\\
468	0.0118085593773549\\
469	0.0118041479408268\\
470	0.0117996319699352\\
471	0.011795007480524\\
472	0.0117902702359601\\
473	0.0117854157284375\\
474	0.0117804391419369\\
475	0.011775335308251\\
476	0.0117700985719009\\
477	0.011764722791759\\
478	0.0117592007003346\\
479	0.0117534130281492\\
480	0.0117474070309391\\
481	0.0117412502508626\\
482	0.0117349369635803\\
483	0.0117284611499936\\
484	0.0117218164845919\\
485	0.0117149963199486\\
486	0.0117079936688997\\
487	0.0117008011377979\\
488	0.0116934111225914\\
489	0.0116858159221389\\
490	0.0116780084831023\\
491	0.0116699756013328\\
492	0.011661927954636\\
493	0.0116536745259642\\
494	0.0116451502114539\\
495	0.0116363218151761\\
496	0.0116271721240532\\
497	0.0116176825840665\\
498	0.0116078334773694\\
499	0.0115976035622183\\
500	0.0115869699789431\\
501	0.0115759083311207\\
502	0.0115643933721359\\
503	0.0115523992589964\\
504	0.0115398877841433\\
505	0.0115268251184635\\
506	0.0115131745027327\\
507	0.0114988959261132\\
508	0.0114839457447316\\
509	0.0114682762507652\\
510	0.0114518352111497\\
511	0.0114345648497484\\
512	0.0114164019116341\\
513	0.0113972763558837\\
514	0.0113771061164495\\
515	0.0113557955821106\\
516	0.0113331025054401\\
517	0.0113090636446581\\
518	0.0112835383950993\\
519	0.0112563454227976\\
520	0.0112272291396666\\
521	0.0111913720425386\\
522	0.0111362062617216\\
523	0.01107855116565\\
524	0.0110177410618636\\
525	0.0109409571798995\\
526	0.0107900790728774\\
527	0.0106373072066772\\
528	0.0104828689988428\\
529	0.0103261862728256\\
530	0.0102751715307638\\
531	0.010231859562221\\
532	0.0101938110775907\\
533	0.0101625100999561\\
534	0.0101343077825933\\
535	0.0101067256577019\\
536	0.0100795670013488\\
537	0.0100524902798381\\
538	0.0100250689368841\\
539	0.00999723629863891\\
540	0.00996892071170054\\
541	0.00994005394302687\\
542	0.00991059416402076\\
543	0.00988051328945524\\
544	0.00984978299992253\\
545	0.00981801969296898\\
546	0.00978559496720072\\
547	0.0097525285655252\\
548	0.00971714045512619\\
549	0.00968170762435967\\
550	0.00964653922855263\\
551	0.00961334193970775\\
552	0.00957957457993395\\
553	0.00954523081058604\\
554	0.00951026190411773\\
555	0.00947465459575065\\
556	0.00943827697808934\\
557	0.00940110865372952\\
558	0.0093631290502357\\
559	0.00932431715805278\\
560	0.0092842998410953\\
561	0.00924288756956627\\
562	0.00919803203251526\\
563	0.00915233007413679\\
564	0.00911247654191991\\
565	0.00907275099311367\\
566	0.00903228031873804\\
567	0.00899103145666645\\
568	0.00894898410983059\\
569	0.00886794777257589\\
570	0.00864416974350737\\
571	0.00810251177679691\\
572	0.00801319312275873\\
573	0.00794656299291458\\
574	0.00787888633733788\\
575	0.0078101367625929\\
576	0.00774029182210559\\
577	0.00766932845435543\\
578	0.00759722282425729\\
579	0.00752395028671346\\
580	0.00744948535432137\\
581	0.00737380165442037\\
582	0.00729687184954222\\
583	0.00721866745305259\\
584	0.00713915835967279\\
585	0.0070583116180176\\
586	0.00697608821123101\\
587	0.00689243463533688\\
588	0.00680726093889689\\
589	0.00672038360986785\\
590	0.00663137734622847\\
591	0.00653960315341497\\
592	0.00644346597070806\\
593	0.00633880868635237\\
594	0.00621472953405218\\
595	0.00604257880714497\\
596	0.00574702137257038\\
597	0.0051299581010561\\
598	0.00367057462141144\\
599	0\\
600	0\\
};
\addplot [color=mycolor21,solid,forget plot]
  table[row sep=crcr]{%
1	0.0120255297121637\\
2	0.0120255287899248\\
3	0.0120255278508754\\
4	0.0120255268947083\\
5	0.012025525921111\\
6	0.0120255249297648\\
7	0.0120255239203456\\
8	0.012025522892523\\
9	0.0120255218459608\\
10	0.0120255207803166\\
11	0.0120255196952416\\
12	0.0120255185903807\\
13	0.0120255174653723\\
14	0.0120255163198482\\
15	0.0120255151534332\\
16	0.0120255139657457\\
17	0.0120255127563966\\
18	0.0120255115249899\\
19	0.0120255102711225\\
20	0.0120255089943837\\
21	0.0120255076943552\\
22	0.0120255063706112\\
23	0.012025505022718\\
24	0.012025503650234\\
25	0.0120255022527095\\
26	0.0120255008296866\\
27	0.0120254993806987\\
28	0.012025497905271\\
29	0.0120254964029199\\
30	0.0120254948731528\\
31	0.0120254933154682\\
32	0.0120254917293553\\
33	0.012025490114294\\
34	0.0120254884697545\\
35	0.0120254867951976\\
36	0.0120254850900739\\
37	0.0120254833538239\\
38	0.0120254815858781\\
39	0.0120254797856564\\
40	0.0120254779525679\\
41	0.012025476086011\\
42	0.0120254741853731\\
43	0.0120254722500303\\
44	0.0120254702793472\\
45	0.0120254682726767\\
46	0.01202546622936\\
47	0.0120254641487259\\
48	0.0120254620300912\\
49	0.0120254598727598\\
50	0.0120254576760232\\
51	0.0120254554391595\\
52	0.0120254531614337\\
53	0.0120254508420975\\
54	0.0120254484803884\\
55	0.0120254460755304\\
56	0.0120254436267327\\
57	0.0120254411331903\\
58	0.0120254385940834\\
59	0.0120254360085771\\
60	0.0120254333758209\\
61	0.012025430694949\\
62	0.0120254279650795\\
63	0.0120254251853144\\
64	0.0120254223547389\\
65	0.0120254194724217\\
66	0.0120254165374143\\
67	0.0120254135487506\\
68	0.0120254105054468\\
69	0.012025407406501\\
70	0.012025404250893\\
71	0.0120254010375835\\
72	0.0120253977655144\\
73	0.0120253944336079\\
74	0.0120253910407665\\
75	0.0120253875858725\\
76	0.0120253840677874\\
77	0.0120253804853521\\
78	0.0120253768373858\\
79	0.0120253731226862\\
80	0.0120253693400287\\
81	0.0120253654881663\\
82	0.0120253615658289\\
83	0.012025357571723\\
84	0.0120253535045314\\
85	0.0120253493629125\\
86	0.0120253451455\\
87	0.0120253408509025\\
88	0.0120253364777028\\
89	0.0120253320244578\\
90	0.0120253274896976\\
91	0.0120253228719254\\
92	0.0120253181696166\\
93	0.0120253133812186\\
94	0.0120253085051502\\
95	0.0120253035398009\\
96	0.0120252984835307\\
97	0.0120252933346692\\
98	0.0120252880915152\\
99	0.0120252827523363\\
100	0.0120252773153678\\
101	0.0120252717788129\\
102	0.0120252661408411\\
103	0.0120252603995887\\
104	0.0120252545531572\\
105	0.0120252485996131\\
106	0.0120252425369874\\
107	0.0120252363632746\\
108	0.0120252300764323\\
109	0.0120252236743803\\
110	0.012025217155\\
111	0.0120252105161337\\
112	0.0120252037555838\\
113	0.0120251968711123\\
114	0.0120251898604397\\
115	0.0120251827212444\\
116	0.0120251754511619\\
117	0.0120251680477842\\
118	0.0120251605086586\\
119	0.0120251528312871\\
120	0.0120251450131258\\
121	0.0120251370515834\\
122	0.012025128944021\\
123	0.0120251206877508\\
124	0.0120251122800353\\
125	0.0120251037180866\\
126	0.012025094999065\\
127	0.0120250861200784\\
128	0.0120250770781812\\
129	0.0120250678703735\\
130	0.0120250584935997\\
131	0.0120250489447477\\
132	0.0120250392206481\\
133	0.0120250293180726\\
134	0.0120250192337333\\
135	0.0120250089642815\\
136	0.0120249985063064\\
137	0.0120249878563342\\
138	0.012024977010827\\
139	0.0120249659661811\\
140	0.0120249547187263\\
141	0.0120249432647244\\
142	0.0120249316003682\\
143	0.0120249197217798\\
144	0.0120249076250097\\
145	0.012024895306035\\
146	0.0120248827607587\\
147	0.0120248699850077\\
148	0.0120248569745318\\
149	0.0120248437250019\\
150	0.012024830232009\\
151	0.0120248164910621\\
152	0.0120248024975875\\
153	0.0120247882469263\\
154	0.0120247737343338\\
155	0.0120247589549769\\
156	0.0120247439039335\\
157	0.01202472857619\\
158	0.0120247129666401\\
159	0.0120246970700826\\
160	0.0120246808812204\\
161	0.012024664394658\\
162	0.0120246476049\\
163	0.0120246305063491\\
164	0.0120246130933045\\
165	0.0120245953599595\\
166	0.0120245773004001\\
167	0.0120245589086025\\
168	0.0120245401784315\\
169	0.012024521103638\\
170	0.0120245016778573\\
171	0.0120244818946067\\
172	0.0120244617472833\\
173	0.012024441229162\\
174	0.0120244203333929\\
175	0.0120243990529995\\
176	0.0120243773808757\\
177	0.0120243553097839\\
178	0.0120243328323525\\
179	0.012024309941073\\
180	0.0120242866282982\\
181	0.012024262886239\\
182	0.0120242387069621\\
183	0.012024214082387\\
184	0.0120241890042839\\
185	0.0120241634642703\\
186	0.0120241374538086\\
187	0.0120241109642029\\
188	0.0120240839865965\\
189	0.0120240565119685\\
190	0.0120240285311311\\
191	0.0120240000347263\\
192	0.012023971013223\\
193	0.0120239414569136\\
194	0.0120239113559108\\
195	0.0120238807001443\\
196	0.0120238494793577\\
197	0.0120238176831045\\
198	0.0120237853007452\\
199	0.0120237523214436\\
200	0.0120237187341628\\
201	0.012023684527662\\
202	0.0120236496904927\\
203	0.0120236142109946\\
204	0.0120235780772922\\
205	0.0120235412772903\\
206	0.0120235037986705\\
207	0.0120234656288867\\
208	0.0120234267551615\\
209	0.0120233871644813\\
210	0.0120233468435926\\
211	0.0120233057789973\\
212	0.0120232639569483\\
213	0.0120232213634452\\
214	0.0120231779842295\\
215	0.0120231338047797\\
216	0.0120230888103073\\
217	0.0120230429857512\\
218	0.012022996315773\\
219	0.0120229487847523\\
220	0.0120229003767814\\
221	0.01202285107566\\
222	0.0120228008648902\\
223	0.012022749727671\\
224	0.012022697646893\\
225	0.0120226446051329\\
226	0.0120225905846477\\
227	0.0120225355673691\\
228	0.0120224795348979\\
229	0.012022422468498\\
230	0.0120223643490903\\
231	0.0120223051572467\\
232	0.0120222448731842\\
233	0.0120221834767582\\
234	0.0120221209474565\\
235	0.0120220572643929\\
236	0.0120219924063001\\
237	0.0120219263515237\\
238	0.0120218590780153\\
239	0.0120217905633252\\
240	0.012021720784596\\
241	0.0120216497185554\\
242	0.0120215773415087\\
243	0.0120215036293322\\
244	0.0120214285574651\\
245	0.0120213521009027\\
246	0.0120212742341882\\
247	0.0120211949314056\\
248	0.0120211141661716\\
249	0.0120210319116277\\
250	0.0120209481404323\\
251	0.0120208628247527\\
252	0.0120207759362566\\
253	0.0120206874461041\\
254	0.0120205973249391\\
255	0.012020505542881\\
256	0.0120204120695156\\
257	0.0120203168738871\\
258	0.0120202199244884\\
259	0.012020121189253\\
260	0.0120200206355455\\
261	0.0120199182301524\\
262	0.0120198139392732\\
263	0.0120197077285109\\
264	0.0120195995628623\\
265	0.0120194894067089\\
266	0.0120193772238071\\
267	0.0120192629772784\\
268	0.0120191466295996\\
269	0.012019028142593\\
270	0.0120189074774163\\
271	0.0120187845945524\\
272	0.0120186594537994\\
273	0.0120185320142606\\
274	0.0120184022343333\\
275	0.0120182700716994\\
276	0.0120181354833139\\
277	0.0120179984253953\\
278	0.012017858853414\\
279	0.0120177167220822\\
280	0.0120175719853427\\
281	0.0120174245963582\\
282	0.0120172745075003\\
283	0.0120171216703383\\
284	0.0120169660356283\\
285	0.0120168075533019\\
286	0.0120166461724548\\
287	0.0120164818413355\\
288	0.0120163145073344\\
289	0.0120161441169713\\
290	0.0120159706158846\\
291	0.0120157939488196\\
292	0.0120156140596162\\
293	0.0120154308911977\\
294	0.0120152443855585\\
295	0.0120150544837518\\
296	0.0120148611258783\\
297	0.0120146642510729\\
298	0.0120144637974928\\
299	0.012014259702305\\
300	0.0120140519016733\\
301	0.0120138403307454\\
302	0.0120136249236402\\
303	0.0120134056134338\\
304	0.0120131823321465\\
305	0.0120129550107287\\
306	0.0120127235790471\\
307	0.01201248796587\\
308	0.0120122480988529\\
309	0.0120120039045231\\
310	0.0120117553082645\\
311	0.0120115022343018\\
312	0.012011244605684\\
313	0.0120109823442679\\
314	0.0120107153707006\\
315	0.0120104436044021\\
316	0.0120101669635469\\
317	0.0120098853650448\\
318	0.0120095987245223\\
319	0.0120093069563021\\
320	0.0120090099733829\\
321	0.0120087076874185\\
322	0.012008400008696\\
323	0.0120080868461143\\
324	0.0120077681071616\\
325	0.0120074436978927\\
326	0.0120071135229068\\
327	0.0120067774853242\\
328	0.012006435486764\\
329	0.0120060874273217\\
330	0.0120057332055475\\
331	0.012005372718426\\
332	0.0120050058613559\\
333	0.0120046325281325\\
334	0.0120042526109316\\
335	0.0120038660002957\\
336	0.0120034725851233\\
337	0.0120030722526616\\
338	0.0120026648885033\\
339	0.0120022503765879\\
340	0.0120018285992089\\
341	0.0120013994370252\\
342	0.0120009627690759\\
343	0.0120005184727982\\
344	0.0120000664240578\\
345	0.0119996064971714\\
346	0.0119991385649306\\
347	0.0119986624986228\\
348	0.0119981781680293\\
349	0.0119976854414182\\
350	0.0119971841855298\\
351	0.0119966742655507\\
352	0.0119961555450808\\
353	0.0119956278861042\\
354	0.0119950911489764\\
355	0.0119945451923468\\
356	0.0119939898731386\\
357	0.0119934250465321\\
358	0.0119928505659465\\
359	0.0119922662830178\\
360	0.0119916720475749\\
361	0.0119910677076115\\
362	0.0119904531092541\\
363	0.011989828096725\\
364	0.0119891925122997\\
365	0.0119885461962584\\
366	0.0119878889868294\\
367	0.0119872207201245\\
368	0.0119865412300644\\
369	0.0119858503482939\\
370	0.0119851479040839\\
371	0.0119844337242197\\
372	0.0119837076328735\\
373	0.0119829694514593\\
374	0.0119822189984671\\
375	0.0119814560892761\\
376	0.0119806805359414\\
377	0.0119798921469546\\
378	0.0119790907269721\\
379	0.0119782760765105\\
380	0.011977447991604\\
381	0.0119766062634203\\
382	0.0119757506778313\\
383	0.0119748810149338\\
384	0.0119739970485174\\
385	0.011973098545472\\
386	0.0119721852651333\\
387	0.01197125695856\\
388	0.0119703133677385\\
389	0.0119693542247097\\
390	0.011968379250616\\
391	0.0119673881546618\\
392	0.0119663806329876\\
393	0.0119653563674543\\
394	0.0119643150243398\\
395	0.0119632562529589\\
396	0.0119621796842344\\
397	0.0119610849292579\\
398	0.0119599715777179\\
399	0.0119588391965376\\
400	0.0119576873296264\\
401	0.0119565154964725\\
402	0.0119553231908543\\
403	0.0119541098797745\\
404	0.011952875002771\\
405	0.0119516179718289\\
406	0.0119503381722104\\
407	0.011949034964616\\
408	0.0119477076891184\\
409	0.0119463556710709\\
410	0.0119449782285423\\
411	0.0119435746809689\\
412	0.0119421443668154\\
413	0.0119406867053521\\
414	0.0119392011453393\\
415	0.0119376871230316\\
416	0.0119361440617144\\
417	0.0119345713711832\\
418	0.0119329684471561\\
419	0.0119313346706092\\
420	0.011929669407016\\
421	0.0119279720054608\\
422	0.0119262417975704\\
423	0.0119244780962016\\
424	0.0119226801937789\\
425	0.011920847360326\\
426	0.0119189788422595\\
427	0.0119170738673206\\
428	0.0119151316132169\\
429	0.011913151185794\\
430	0.0119111316603661\\
431	0.011909072080582\\
432	0.0119069714572969\\
433	0.0119048287674624\\
434	0.0119026429530481\\
435	0.0119004129200023\\
436	0.0118981375372731\\
437	0.0118958156359115\\
438	0.0118934460082797\\
439	0.0118910274073887\\
440	0.0118885585463875\\
441	0.0118860380982064\\
442	0.0118834646952959\\
443	0.0118808369291995\\
444	0.0118781533491193\\
445	0.0118754124569799\\
446	0.0118726126921438\\
447	0.0118697523890269\\
448	0.0118668097195243\\
449	0.0118637941635262\\
450	0.0118607151672787\\
451	0.0118575708300137\\
452	0.0118543591578061\\
453	0.0118510780548184\\
454	0.0118477253108345\\
455	0.0118442986198687\\
456	0.0118407955566509\\
457	0.0118372135661188\\
458	0.0118335499506417\\
459	0.011829801855534\\
460	0.0118259662520849\\
461	0.011822039896919\\
462	0.0118180192658186\\
463	0.0118139003743498\\
464	0.0118096781843979\\
465	0.0118052012457311\\
466	0.0118006209417538\\
467	0.0117959392998883\\
468	0.0117911533512572\\
469	0.0117862600409853\\
470	0.0117812562440118\\
471	0.0117761386715563\\
472	0.0117709038999233\\
473	0.0117655483792318\\
474	0.0117600683696803\\
475	0.0117544598835507\\
476	0.0117487182923039\\
477	0.011742838994337\\
478	0.0117368162978626\\
479	0.0117307889119633\\
480	0.0117246862925758\\
481	0.0117184096550395\\
482	0.0117119511788036\\
483	0.0117053025335702\\
484	0.0116984547992348\\
485	0.0116913986720072\\
486	0.0116841246685197\\
487	0.0116766242337286\\
488	0.0116688830653201\\
489	0.0116608897836263\\
490	0.0116526430078128\\
491	0.0116441179130111\\
492	0.0116352827189334\\
493	0.0116261185857869\\
494	0.0116166066322599\\
495	0.0116067266804755\\
496	0.0115964569798114\\
497	0.0115857740584845\\
498	0.0115746525523927\\
499	0.011563065031309\\
500	0.0115509817941974\\
501	0.0115383706370639\\
502	0.0115251965467394\\
503	0.0115114212636823\\
504	0.0114970031860585\\
505	0.0114818963047486\\
506	0.0114660501940957\\
507	0.0114494094458906\\
508	0.0114319131859898\\
509	0.0114134794522809\\
510	0.0113939225676525\\
511	0.0113732997418437\\
512	0.0113515103310387\\
513	0.011328450721176\\
514	0.0113039980327086\\
515	0.0112779313281883\\
516	0.0112494297986435\\
517	0.0112149622997858\\
518	0.0111616261246485\\
519	0.0111058754752424\\
520	0.0110470938945192\\
521	0.0109668132577287\\
522	0.0108201144889326\\
523	0.0106714453465997\\
524	0.0105205839305904\\
525	0.0103836515294843\\
526	0.0103400772576323\\
527	0.0103008937326336\\
528	0.0102672910308798\\
529	0.0102408006238487\\
530	0.0102152165337159\\
531	0.0101901710867317\\
532	0.0101654133787905\\
533	0.0101405240497298\\
534	0.0101152824311588\\
535	0.0100896210583419\\
536	0.010063475628275\\
537	0.0100367961053353\\
538	0.0100095584854258\\
539	0.00998174661979945\\
540	0.00995333423090456\\
541	0.00992430024113237\\
542	0.00989439212460905\\
543	0.00986369056406941\\
544	0.00983243438806496\\
545	0.00979914248651857\\
546	0.00976544414150726\\
547	0.00973183325738434\\
548	0.00970048766839892\\
549	0.00966860155975083\\
550	0.00963614879663857\\
551	0.00960302846261999\\
552	0.00956922922488635\\
553	0.00953475772430091\\
554	0.00949967761422997\\
555	0.00946389230032819\\
556	0.00942733039872636\\
557	0.00938997245792145\\
558	0.00935179497832293\\
559	0.00931183749946102\\
560	0.00926974633768217\\
561	0.00922491902592286\\
562	0.00918483364423364\\
563	0.00914649480811267\\
564	0.0091074664117141\\
565	0.00906769525172241\\
566	0.00902716361243558\\
567	0.00898585384011895\\
568	0.0088935945560552\\
569	0.00858031079120776\\
570	0.0081439331999808\\
571	0.00807878169078327\\
572	0.0080131873432107\\
573	0.00794656252939614\\
574	0.0078788861595522\\
575	0.00781013666848245\\
576	0.00774029177194786\\
577	0.00766932842871205\\
578	0.00759722281193334\\
579	0.00752395028123868\\
580	0.00744948535211585\\
581	0.00737380165363516\\
582	0.0072968718493044\\
583	0.00721866745299509\\
584	0.00713915835966283\\
585	0.00705831161801654\\
586	0.00697608821123101\\
587	0.00689243463533689\\
588	0.00680726093889688\\
589	0.00672038360986784\\
590	0.00663137734622847\\
591	0.00653960315341498\\
592	0.00644346597070807\\
593	0.00633880868635237\\
594	0.00621472953405218\\
595	0.00604257880714497\\
596	0.00574702137257039\\
597	0.0051299581010561\\
598	0.00367057462141144\\
599	0\\
600	0\\
};
\addplot [color=black!20!mycolor21,solid,forget plot]
  table[row sep=crcr]{%
1	0.0120148151515662\\
2	0.012014814396408\\
3	0.0120148136273969\\
4	0.012014812844278\\
5	0.0120148120467916\\
6	0.0120148112346734\\
7	0.012014810407654\\
8	0.0120148095654591\\
9	0.0120148087078094\\
10	0.0120148078344204\\
11	0.0120148069450022\\
12	0.0120148060392599\\
13	0.0120148051168927\\
14	0.0120148041775945\\
15	0.0120148032210536\\
16	0.0120148022469524\\
17	0.0120148012549673\\
18	0.0120148002447691\\
19	0.0120147992160222\\
20	0.0120147981683849\\
21	0.0120147971015092\\
22	0.0120147960150405\\
23	0.0120147949086179\\
24	0.0120147937818736\\
25	0.0120147926344331\\
26	0.0120147914659151\\
27	0.0120147902759309\\
28	0.0120147890640848\\
29	0.012014787829974\\
30	0.0120147865731878\\
31	0.0120147852933082\\
32	0.0120147839899094\\
33	0.0120147826625577\\
34	0.0120147813108113\\
35	0.0120147799342204\\
36	0.0120147785323266\\
37	0.0120147771046632\\
38	0.0120147756507549\\
39	0.0120147741701174\\
40	0.0120147726622577\\
41	0.0120147711266733\\
42	0.0120147695628526\\
43	0.0120147679702747\\
44	0.0120147663484086\\
45	0.0120147646967139\\
46	0.0120147630146399\\
47	0.0120147613016258\\
48	0.0120147595571003\\
49	0.0120147577804816\\
50	0.0120147559711771\\
51	0.0120147541285832\\
52	0.0120147522520852\\
53	0.0120147503410567\\
54	0.01201474839486\\
55	0.0120147464128454\\
56	0.0120147443943511\\
57	0.0120147423387032\\
58	0.0120147402452152\\
59	0.0120147381131877\\
60	0.0120147359419086\\
61	0.0120147337306523\\
62	0.0120147314786799\\
63	0.0120147291852388\\
64	0.0120147268495622\\
65	0.0120147244708694\\
66	0.0120147220483648\\
67	0.0120147195812383\\
68	0.0120147170686647\\
69	0.0120147145098034\\
70	0.0120147119037981\\
71	0.0120147092497768\\
72	0.0120147065468511\\
73	0.0120147037941162\\
74	0.0120147009906504\\
75	0.0120146981355149\\
76	0.0120146952277533\\
77	0.0120146922663916\\
78	0.0120146892504377\\
79	0.0120146861788808\\
80	0.0120146830506914\\
81	0.012014679864821\\
82	0.0120146766202014\\
83	0.0120146733157446\\
84	0.0120146699503422\\
85	0.0120146665228654\\
86	0.0120146630321641\\
87	0.0120146594770671\\
88	0.012014655856381\\
89	0.0120146521688905\\
90	0.0120146484133576\\
91	0.012014644588521\\
92	0.0120146406930961\\
93	0.0120146367257743\\
94	0.0120146326852227\\
95	0.0120146285700833\\
96	0.012014624378973\\
97	0.0120146201104829\\
98	0.0120146157631776\\
99	0.0120146113355953\\
100	0.0120146068262465\\
101	0.0120146022336142\\
102	0.012014597556153\\
103	0.0120145927922886\\
104	0.0120145879404174\\
105	0.0120145829989059\\
106	0.0120145779660898\\
107	0.0120145728402742\\
108	0.0120145676197322\\
109	0.0120145623027046\\
110	0.0120145568873995\\
111	0.0120145513719916\\
112	0.0120145457546211\\
113	0.0120145400333939\\
114	0.0120145342063803\\
115	0.0120145282716143\\
116	0.0120145222270934\\
117	0.0120145160707777\\
118	0.0120145098005889\\
119	0.0120145034144099\\
120	0.0120144969100841\\
121	0.0120144902854145\\
122	0.0120144835381628\\
123	0.0120144766660491\\
124	0.0120144696667505\\
125	0.0120144625379007\\
126	0.0120144552770893\\
127	0.0120144478818605\\
128	0.0120144403497125\\
129	0.0120144326780968\\
130	0.0120144248644168\\
131	0.0120144169060277\\
132	0.0120144088002347\\
133	0.0120144005442926\\
134	0.0120143921354049\\
135	0.0120143835707223\\
136	0.0120143748473423\\
137	0.0120143659623079\\
138	0.0120143569126066\\
139	0.0120143476951692\\
140	0.0120143383068691\\
141	0.0120143287445208\\
142	0.0120143190048792\\
143	0.012014309084638\\
144	0.012014298980429\\
145	0.0120142886888205\\
146	0.0120142782063165\\
147	0.0120142675293552\\
148	0.0120142566543077\\
149	0.0120142455774771\\
150	0.0120142342950968\\
151	0.0120142228033295\\
152	0.0120142110982654\\
153	0.0120141991759216\\
154	0.0120141870322397\\
155	0.0120141746630853\\
156	0.012014162064246\\
157	0.01201414923143\\
158	0.0120141361602648\\
159	0.0120141228462953\\
160	0.0120141092849828\\
161	0.0120140954717027\\
162	0.0120140814017435\\
163	0.0120140670703045\\
164	0.0120140524724949\\
165	0.0120140376033313\\
166	0.0120140224577364\\
167	0.0120140070305371\\
168	0.0120139913164627\\
169	0.0120139753101428\\
170	0.0120139590061059\\
171	0.0120139423987769\\
172	0.0120139254824754\\
173	0.0120139082514138\\
174	0.012013890699695\\
175	0.0120138728213105\\
176	0.0120138546101381\\
177	0.01201383605994\\
178	0.0120138171643603\\
179	0.0120137979169227\\
180	0.0120137783110285\\
181	0.0120137583399542\\
182	0.0120137379968488\\
183	0.0120137172747316\\
184	0.0120136961664898\\
185	0.0120136746648758\\
186	0.0120136527625046\\
187	0.0120136304518512\\
188	0.012013607725248\\
189	0.0120135845748821\\
190	0.0120135609927922\\
191	0.0120135369708662\\
192	0.0120135125008378\\
193	0.012013487574284\\
194	0.0120134621826219\\
195	0.0120134363171056\\
196	0.0120134099688229\\
197	0.0120133831286925\\
198	0.0120133557874605\\
199	0.0120133279356972\\
200	0.0120132995637934\\
201	0.0120132706619573\\
202	0.0120132412202109\\
203	0.0120132112283864\\
204	0.0120131806761224\\
205	0.0120131495528604\\
206	0.012013117847841\\
207	0.0120130855500995\\
208	0.0120130526484629\\
209	0.012013019131545\\
210	0.0120129849877428\\
211	0.012012950205232\\
212	0.0120129147719632\\
213	0.0120128786756568\\
214	0.0120128419037995\\
215	0.0120128044436389\\
216	0.0120127662821795\\
217	0.0120127274061778\\
218	0.0120126878021374\\
219	0.0120126474563042\\
220	0.0120126063546617\\
221	0.0120125644829254\\
222	0.0120125218265382\\
223	0.0120124783706645\\
224	0.0120124341001855\\
225	0.0120123889996933\\
226	0.0120123430534853\\
227	0.012012296245559\\
228	0.0120122485596054\\
229	0.0120121999790039\\
230	0.0120121504868159\\
231	0.0120121000657785\\
232	0.0120120486982986\\
233	0.0120119963664464\\
234	0.0120119430519488\\
235	0.0120118887361829\\
236	0.0120118334001694\\
237	0.0120117770245654\\
238	0.0120117195896577\\
239	0.0120116610753555\\
240	0.0120116014611832\\
241	0.0120115407262731\\
242	0.0120114788493576\\
243	0.0120114158087619\\
244	0.0120113515823961\\
245	0.0120112861477467\\
246	0.0120112194818694\\
247	0.0120111515613802\\
248	0.0120110823624473\\
249	0.0120110118607822\\
250	0.0120109400316316\\
251	0.012010866849768\\
252	0.0120107922894809\\
253	0.0120107163245675\\
254	0.0120106389283238\\
255	0.0120105600735343\\
256	0.0120104797324633\\
257	0.0120103978768443\\
258	0.0120103144778703\\
259	0.0120102295061839\\
260	0.0120101429318664\\
261	0.0120100547244279\\
262	0.0120099648527963\\
263	0.0120098732853064\\
264	0.0120097799896891\\
265	0.0120096849330603\\
266	0.0120095880819093\\
267	0.0120094894020874\\
268	0.012009388858796\\
269	0.0120092864165754\\
270	0.012009182039292\\
271	0.0120090756901264\\
272	0.0120089673315612\\
273	0.0120088569253685\\
274	0.0120087444325966\\
275	0.0120086298135582\\
276	0.0120085130278165\\
277	0.0120083940341724\\
278	0.0120082727906513\\
279	0.0120081492544897\\
280	0.0120080233821211\\
281	0.0120078951291631\\
282	0.0120077644504031\\
283	0.0120076312997841\\
284	0.0120074956303915\\
285	0.0120073573944379\\
286	0.0120072165432496\\
287	0.0120070730272519\\
288	0.012006926795955\\
289	0.012006777797939\\
290	0.0120066259808396\\
291	0.0120064712913339\\
292	0.0120063136751249\\
293	0.0120061530769276\\
294	0.012005989440454\\
295	0.0120058227083983\\
296	0.0120056528224226\\
297	0.0120054797231417\\
298	0.0120053033501088\\
299	0.0120051236418009\\
300	0.0120049405356039\\
301	0.0120047539677984\\
302	0.0120045638735449\\
303	0.0120043701868697\\
304	0.0120041728406498\\
305	0.0120039717665995\\
306	0.012003766895255\\
307	0.0120035581559606\\
308	0.0120033454768544\\
309	0.0120031287848533\\
310	0.0120029080056388\\
311	0.0120026830636425\\
312	0.0120024538820306\\
313	0.0120022203826891\\
314	0.0120019824862085\\
315	0.0120017401118669\\
316	0.0120014931776145\\
317	0.0120012416000552\\
318	0.0120009852944292\\
319	0.0120007241745931\\
320	0.0120004581529997\\
321	0.0120001871406757\\
322	0.0119999110471983\\
323	0.0119996297806686\\
324	0.011999343247684\\
325	0.0119990513533073\\
326	0.0119987540010322\\
327	0.0119984510927465\\
328	0.0119981425286905\\
329	0.0119978282074116\\
330	0.0119975080257138\\
331	0.0119971818786024\\
332	0.0119968496592225\\
333	0.0119965112587923\\
334	0.0119961665665284\\
335	0.0119958154695659\\
336	0.0119954578528703\\
337	0.0119950935991419\\
338	0.0119947225887137\\
339	0.0119943446994406\\
340	0.0119939598065818\\
341	0.0119935677826759\\
342	0.0119931684974103\\
343	0.0119927618174892\\
344	0.0119923476065101\\
345	0.0119919257248634\\
346	0.0119914960295956\\
347	0.0119910583743842\\
348	0.0119906126099686\\
349	0.0119901585840743\\
350	0.0119896961413355\\
351	0.0119892251232151\\
352	0.011988745367925\\
353	0.0119882567103439\\
354	0.0119877589819316\\
355	0.0119872520106436\\
356	0.0119867356208439\\
357	0.0119862096332166\\
358	0.0119856738646753\\
359	0.0119851281282704\\
360	0.011984572233095\\
361	0.0119840059841886\\
362	0.0119834291824397\\
363	0.0119828416244863\\
364	0.011982243102616\\
365	0.0119816334046634\\
366	0.0119810123139079\\
367	0.0119803796089693\\
368	0.0119797350637038\\
369	0.0119790784470985\\
370	0.0119784095231658\\
371	0.0119777280508379\\
372	0.0119770337838602\\
373	0.011976326470686\\
374	0.0119756058543701\\
375	0.0119748716724633\\
376	0.0119741236569066\\
377	0.0119733615339256\\
378	0.0119725850239242\\
379	0.0119717938413781\\
380	0.0119709876947268\\
381	0.011970166286264\\
382	0.0119693293120246\\
383	0.0119684764616675\\
384	0.011967607418353\\
385	0.011966721858612\\
386	0.0119658194522041\\
387	0.0119648998619628\\
388	0.0119639627436229\\
389	0.0119630077456251\\
390	0.0119620345088918\\
391	0.0119610426665645\\
392	0.0119600318436879\\
393	0.0119590016568161\\
394	0.0119579517134957\\
395	0.0119568816115666\\
396	0.0119557909382804\\
397	0.0119546792697361\\
398	0.011953546173347\\
399	0.0119523911980287\\
400	0.0119512138450451\\
401	0.0119500136023912\\
402	0.0119487899447784\\
403	0.0119475423336915\\
404	0.0119462702175204\\
405	0.0119449730317661\\
406	0.0119436501993072\\
407	0.0119423011306811\\
408	0.0119409252242862\\
409	0.0119395218663627\\
410	0.0119380904306992\\
411	0.0119366302784273\\
412	0.0119351407584722\\
413	0.0119336212046226\\
414	0.011932070933786\\
415	0.0119304892459382\\
416	0.011928875424166\\
417	0.0119272287348222\\
418	0.0119255484278127\\
419	0.0119238337370338\\
420	0.0119220838809695\\
421	0.0119202980634271\\
422	0.0119184754742528\\
423	0.0119166152895395\\
424	0.0119147166698111\\
425	0.0119127787522184\\
426	0.0119108006265467\\
427	0.0119087718946531\\
428	0.0119066888269649\\
429	0.0119045644272019\\
430	0.0119023976914949\\
431	0.011900187575492\\
432	0.0118979329919033\\
433	0.0118956328073956\\
434	0.0118932858362306\\
435	0.0118908908381313\\
436	0.0118884465138412\\
437	0.0118859515001852\\
438	0.0118834043645596\\
439	0.0118808035987626\\
440	0.0118781476120355\\
441	0.0118754347230965\\
442	0.0118726631506793\\
443	0.0118698310012506\\
444	0.0118669362497418\\
445	0.0118639766992243\\
446	0.0118609498695208\\
447	0.0118578526301097\\
448	0.0118546024231345\\
449	0.0118512417510506\\
450	0.0118478136967337\\
451	0.0118443167692112\\
452	0.0118407494316341\\
453	0.0118371100868808\\
454	0.0118333970487651\\
455	0.0118296086723464\\
456	0.0118257432814313\\
457	0.0118217991701318\\
458	0.0118177746046532\\
459	0.0118136678244231\\
460	0.011809477039181\\
461	0.0118052004140623\\
462	0.0118008360057085\\
463	0.0117963815603824\\
464	0.011791833854806\\
465	0.0117873800831699\\
466	0.0117828235662098\\
467	0.0117781542543439\\
468	0.0117733679856552\\
469	0.0117684603509974\\
470	0.0117634266720834\\
471	0.0117582619801473\\
472	0.0117529609920099\\
473	0.0117475180840029\\
474	0.0117419272694589\\
475	0.0117361821936444\\
476	0.011730276219622\\
477	0.0117242027536418\\
478	0.0117179565469448\\
479	0.0117115270828871\\
480	0.0117049035952951\\
481	0.0116980770116598\\
482	0.0116910378841657\\
483	0.0116837766125426\\
484	0.0116762845475008\\
485	0.0116685472255938\\
486	0.0116605531669874\\
487	0.0116523009053517\\
488	0.0116437588288315\\
489	0.0116349021796429\\
490	0.0116257130224917\\
491	0.0116161720429362\\
492	0.0116062586564251\\
493	0.0115959506966648\\
494	0.0115852242189069\\
495	0.0115740533118988\\
496	0.0115624099024444\\
497	0.0115502635786817\\
498	0.0115375810487048\\
499	0.0115243256981167\\
500	0.0115104575963668\\
501	0.0114959330534962\\
502	0.0114807040238229\\
503	0.0114646845642786\\
504	0.0114477507286357\\
505	0.0114299576265125\\
506	0.0114112338486864\\
507	0.0113914980993016\\
508	0.0113706532935275\\
509	0.0113485314383598\\
510	0.0113246014510131\\
511	0.0112992693342318\\
512	0.0112722853019366\\
513	0.0112418133667483\\
514	0.011190476444293\\
515	0.0111368547786893\\
516	0.0110811703984723\\
517	0.0110064593453029\\
518	0.0108631391289739\\
519	0.0107178228881732\\
520	0.0105706695680367\\
521	0.0104452483799196\\
522	0.0104048571550337\\
523	0.0103689904946693\\
524	0.0103388715678438\\
525	0.0103148924279247\\
526	0.0102915574748177\\
527	0.0102686997839379\\
528	0.0102460351536893\\
529	0.0102230987996568\\
530	0.010199813024634\\
531	0.0101761149546873\\
532	0.0101519490036182\\
533	0.0101272821960117\\
534	0.0101020936181511\\
535	0.010076364515993\\
536	0.0100500785516816\\
537	0.0100232214492585\\
538	0.00999577977008527\\
539	0.00996766898662057\\
540	0.0099385820974458\\
541	0.00990898377379937\\
542	0.00987791551347699\\
543	0.00984578740576401\\
544	0.00981370911235043\\
545	0.00978360879208303\\
546	0.00975349543905166\\
547	0.00972286137938807\\
548	0.00969159666524659\\
549	0.0096596823109867\\
550	0.00962710017684199\\
551	0.00959383861630612\\
552	0.0095598940535381\\
553	0.00952528583815998\\
554	0.00949008609645875\\
555	0.00945413422832545\\
556	0.0094174022457297\\
557	0.0093794304509989\\
558	0.00934018782568999\\
559	0.00929636247481347\\
560	0.00925518661613193\\
561	0.00921814086148142\\
562	0.00918049591341593\\
563	0.00914214072293134\\
564	0.00910305761380159\\
565	0.0090632302776897\\
566	0.0090226414295871\\
567	0.00892472303347574\\
568	0.00855041708090625\\
569	0.00820697227840564\\
570	0.00814336486304254\\
571	0.00807878133810083\\
572	0.00801318728233301\\
573	0.00794656250157514\\
574	0.0078788861448692\\
575	0.00781013666087204\\
576	0.00774029176819273\\
577	0.00766932842697722\\
578	0.00759722281119467\\
579	0.00752395028095427\\
580	0.00744948535201936\\
581	0.00737380165360752\\
582	0.00729687184929805\\
583	0.0072186674529939\\
584	0.00713915835966273\\
585	0.00705831161801653\\
586	0.00697608821123101\\
587	0.00689243463533689\\
588	0.00680726093889689\\
589	0.00672038360986786\\
590	0.00663137734622847\\
591	0.00653960315341497\\
592	0.00644346597070807\\
593	0.00633880868635237\\
594	0.00621472953405217\\
595	0.00604257880714497\\
596	0.00574702137257037\\
597	0.0051299581010561\\
598	0.00367057462141144\\
599	0\\
600	0\\
};
\addplot [color=black!50!mycolor20,solid,forget plot]
  table[row sep=crcr]{%
1	0.0120106151151093\\
2	0.0120106144127983\\
3	0.0120106136975531\\
4	0.0120106129691348\\
5	0.0120106122272997\\
6	0.0120106114717999\\
7	0.0120106107023829\\
8	0.0120106099187912\\
9	0.0120106091207628\\
10	0.0120106083080308\\
11	0.0120106074803233\\
12	0.0120106066373633\\
13	0.0120106057788687\\
14	0.0120106049045523\\
15	0.0120106040141213\\
16	0.0120106031072776\\
17	0.0120106021837177\\
18	0.0120106012431321\\
19	0.012010600285206\\
20	0.0120105993096184\\
21	0.0120105983160424\\
22	0.0120105973041451\\
23	0.0120105962735875\\
24	0.012010595224024\\
25	0.0120105941551029\\
26	0.0120105930664658\\
27	0.0120105919577476\\
28	0.0120105908285765\\
29	0.0120105896785737\\
30	0.0120105885073534\\
31	0.0120105873145227\\
32	0.0120105860996813\\
33	0.0120105848624213\\
34	0.0120105836023276\\
35	0.012010582318977\\
36	0.0120105810119387\\
37	0.0120105796807736\\
38	0.0120105783250346\\
39	0.0120105769442663\\
40	0.0120105755380047\\
41	0.0120105741057774\\
42	0.0120105726471028\\
43	0.0120105711614907\\
44	0.0120105696484416\\
45	0.0120105681074468\\
46	0.0120105665379878\\
47	0.0120105649395369\\
48	0.0120105633115563\\
49	0.0120105616534982\\
50	0.0120105599648045\\
51	0.012010558244907\\
52	0.0120105564932266\\
53	0.0120105547091735\\
54	0.0120105528921469\\
55	0.0120105510415349\\
56	0.012010549156714\\
57	0.0120105472370491\\
58	0.0120105452818935\\
59	0.012010543290588\\
60	0.0120105412624615\\
61	0.0120105391968302\\
62	0.0120105370929975\\
63	0.0120105349502539\\
64	0.0120105327678765\\
65	0.012010530545129\\
66	0.0120105282812615\\
67	0.0120105259755097\\
68	0.0120105236270954\\
69	0.0120105212352258\\
70	0.012010518799093\\
71	0.0120105163178744\\
72	0.0120105137907317\\
73	0.0120105112168111\\
74	0.0120105085952428\\
75	0.0120105059251407\\
76	0.0120105032056023\\
77	0.0120105004357081\\
78	0.0120104976145212\\
79	0.0120104947410875\\
80	0.0120104918144349\\
81	0.0120104888335731\\
82	0.0120104857974933\\
83	0.0120104827051678\\
84	0.0120104795555498\\
85	0.0120104763475726\\
86	0.0120104730801498\\
87	0.0120104697521745\\
88	0.0120104663625193\\
89	0.0120104629100352\\
90	0.0120104593935523\\
91	0.0120104558118782\\
92	0.0120104521637986\\
93	0.0120104484480761\\
94	0.0120104446634504\\
95	0.0120104408086373\\
96	0.0120104368823288\\
97	0.0120104328831921\\
98	0.0120104288098695\\
99	0.0120104246609779\\
100	0.0120104204351082\\
101	0.0120104161308249\\
102	0.0120104117466653\\
103	0.0120104072811397\\
104	0.0120104027327301\\
105	0.0120103980998901\\
106	0.0120103933810441\\
107	0.0120103885745872\\
108	0.0120103836788841\\
109	0.0120103786922688\\
110	0.0120103736130443\\
111	0.0120103684394812\\
112	0.0120103631698179\\
113	0.0120103578022597\\
114	0.012010352334978\\
115	0.0120103467661099\\
116	0.0120103410937574\\
117	0.0120103353159868\\
118	0.0120103294308281\\
119	0.012010323436274\\
120	0.0120103173302796\\
121	0.0120103111107616\\
122	0.012010304775597\\
123	0.0120102983226234\\
124	0.0120102917496371\\
125	0.0120102850543934\\
126	0.0120102782346049\\
127	0.0120102712879412\\
128	0.012010264212028\\
129	0.0120102570044462\\
130	0.0120102496627311\\
131	0.0120102421843713\\
132	0.0120102345668083\\
133	0.0120102268074351\\
134	0.0120102189035956\\
135	0.0120102108525835\\
136	0.0120102026516413\\
137	0.0120101942979596\\
138	0.0120101857886756\\
139	0.0120101771208728\\
140	0.0120101682915792\\
141	0.0120101592977669\\
142	0.0120101501363504\\
143	0.0120101408041862\\
144	0.012010131298071\\
145	0.0120101216147411\\
146	0.0120101117508708\\
147	0.0120101017030716\\
148	0.0120100914678908\\
149	0.0120100810418102\\
150	0.0120100704212449\\
151	0.0120100596025422\\
152	0.0120100485819801\\
153	0.0120100373557658\\
154	0.0120100259200349\\
155	0.0120100142708493\\
156	0.0120100024041964\\
157	0.0120099903159874\\
158	0.0120099780020557\\
159	0.0120099654581557\\
160	0.0120099526799609\\
161	0.0120099396630629\\
162	0.0120099264029691\\
163	0.0120099128951018\\
164	0.0120098991347959\\
165	0.0120098851172977\\
166	0.0120098708377629\\
167	0.0120098562912551\\
168	0.0120098414727436\\
169	0.0120098263771021\\
170	0.0120098109991063\\
171	0.0120097953334326\\
172	0.0120097793746555\\
173	0.0120097631172463\\
174	0.0120097465555705\\
175	0.0120097296838862\\
176	0.0120097124963416\\
177	0.0120096949869731\\
178	0.0120096771497034\\
179	0.0120096589783386\\
180	0.0120096404665665\\
181	0.0120096216079539\\
182	0.0120096023959448\\
183	0.0120095828238571\\
184	0.012009562884881\\
185	0.0120095425720762\\
186	0.012009521878369\\
187	0.0120095007965503\\
188	0.0120094793192725\\
189	0.0120094574390466\\
190	0.0120094351482403\\
191	0.0120094124390742\\
192	0.0120093893036193\\
193	0.0120093657337943\\
194	0.0120093417213621\\
195	0.0120093172579272\\
196	0.0120092923349323\\
197	0.012009266943655\\
198	0.0120092410752048\\
199	0.0120092147205197\\
200	0.0120091878703627\\
201	0.0120091605153184\\
202	0.0120091326457895\\
203	0.0120091042519931\\
204	0.0120090753239569\\
205	0.0120090458515157\\
206	0.0120090158243075\\
207	0.0120089852317694\\
208	0.0120089540631336\\
209	0.0120089223074236\\
210	0.0120088899534495\\
211	0.0120088569898044\\
212	0.0120088234048595\\
213	0.0120087891867597\\
214	0.0120087543234193\\
215	0.0120087188025175\\
216	0.0120086826114929\\
217	0.0120086457375396\\
218	0.0120086081676015\\
219	0.0120085698883678\\
220	0.0120085308862674\\
221	0.012008491147464\\
222	0.0120084506578505\\
223	0.0120084094030435\\
224	0.0120083673683779\\
225	0.0120083245389008\\
226	0.0120082808993659\\
227	0.0120082364342275\\
228	0.0120081911276344\\
229	0.0120081449634234\\
230	0.0120080979251131\\
231	0.0120080499958975\\
232	0.0120080011586392\\
233	0.0120079513958623\\
234	0.012007900689746\\
235	0.0120078490221171\\
236	0.0120077963744426\\
237	0.0120077427278228\\
238	0.0120076880629831\\
239	0.0120076323602666\\
240	0.0120075755996261\\
241	0.0120075177606157\\
242	0.012007458822383\\
243	0.0120073987636601\\
244	0.0120073375627554\\
245	0.0120072751975444\\
246	0.0120072116454606\\
247	0.0120071468834864\\
248	0.0120070808881437\\
249	0.0120070136354841\\
250	0.0120069451010787\\
251	0.0120068752600085\\
252	0.0120068040868536\\
253	0.012006731555683\\
254	0.0120066576400432\\
255	0.0120065823129479\\
256	0.0120065055468659\\
257	0.0120064273137099\\
258	0.0120063475848248\\
259	0.012006266330975\\
260	0.0120061835223329\\
261	0.0120060991284653\\
262	0.0120060131183211\\
263	0.0120059254602179\\
264	0.0120058361218281\\
265	0.0120057450701658\\
266	0.0120056522715718\\
267	0.0120055576916998\\
268	0.0120054612955012\\
269	0.0120053630472102\\
270	0.0120052629103285\\
271	0.0120051608476088\\
272	0.0120050568210398\\
273	0.0120049507918286\\
274	0.0120048427203847\\
275	0.0120047325663023\\
276	0.0120046202883432\\
277	0.0120045058444183\\
278	0.01200438919157\\
279	0.012004270285953\\
280	0.0120041490828156\\
281	0.0120040255364801\\
282	0.0120038996003232\\
283	0.0120037712267557\\
284	0.0120036403672023\\
285	0.0120035069720803\\
286	0.0120033709907785\\
287	0.0120032323716361\\
288	0.0120030910619198\\
289	0.0120029470078025\\
290	0.0120028001543402\\
291	0.012002650445449\\
292	0.0120024978238824\\
293	0.0120023422312073\\
294	0.0120021836077808\\
295	0.012002021892726\\
296	0.0120018570239078\\
297	0.0120016889379093\\
298	0.0120015175700066\\
299	0.012001342854145\\
300	0.0120011647229147\\
301	0.012000983107526\\
302	0.0120007979377855\\
303	0.0120006091420718\\
304	0.0120004166473124\\
305	0.0120002203789592\\
306	0.012000020260967\\
307	0.0119998162157699\\
308	0.0119996081642604\\
309	0.0119993960257683\\
310	0.0119991797180408\\
311	0.0119989591572233\\
312	0.0119987342578421\\
313	0.0119985049327881\\
314	0.0119982710933014\\
315	0.0119980326489588\\
316	0.0119977895076622\\
317	0.0119975415756292\\
318	0.011997288757386\\
319	0.0119970309557628\\
320	0.0119967680718914\\
321	0.0119965000052049\\
322	0.0119962266534415\\
323	0.0119959479126495\\
324	0.0119956636771962\\
325	0.0119953738397788\\
326	0.0119950782914387\\
327	0.0119947769215776\\
328	0.0119944696179763\\
329	0.0119941562668149\\
330	0.0119938367526944\\
331	0.0119935109586588\\
332	0.0119931787662169\\
333	0.0119928400553615\\
334	0.0119924947045867\\
335	0.011992142590899\\
336	0.0119917835898216\\
337	0.0119914175753879\\
338	0.0119910444201217\\
339	0.0119906639949972\\
340	0.011990276169373\\
341	0.0119898808108867\\
342	0.0119894777852901\\
343	0.011989066956197\\
344	0.0119886481847435\\
345	0.0119882213293679\\
346	0.0119877862468527\\
347	0.0119873427879657\\
348	0.0119868907855941\\
349	0.0119864300689015\\
350	0.0119859604632224\\
351	0.0119854817899529\\
352	0.0119849938664379\\
353	0.0119844965058539\\
354	0.0119839895170894\\
355	0.011983472704621\\
356	0.0119829458683665\\
357	0.0119824088035481\\
358	0.0119818613005513\\
359	0.0119813031447808\\
360	0.0119807341165107\\
361	0.01198015399073\\
362	0.0119795625369843\\
363	0.011978959519213\\
364	0.0119783446955818\\
365	0.0119777178183117\\
366	0.0119770786335048\\
367	0.0119764268809662\\
368	0.0119757622940242\\
369	0.0119750845993482\\
370	0.011974393516766\\
371	0.011973688759081\\
372	0.0119729700318911\\
373	0.0119722370334106\\
374	0.0119714894542969\\
375	0.0119707269774837\\
376	0.0119699492780246\\
377	0.0119691560229485\\
378	0.0119683468711306\\
379	0.0119675214731846\\
380	0.011966679471378\\
381	0.011965820499578\\
382	0.0119649441832337\\
383	0.0119640501394009\\
384	0.0119631379768187\\
385	0.0119622072960469\\
386	0.0119612576896748\\
387	0.0119602887426134\\
388	0.0119593000324861\\
389	0.0119582911301307\\
390	0.0119572616002302\\
391	0.0119562110020801\\
392	0.011955138890486\\
393	0.0119540448167192\\
394	0.0119529283292776\\
395	0.0119517889736703\\
396	0.0119506262890282\\
397	0.0119494397959426\\
398	0.0119482251195641\\
399	0.0119469755424342\\
400	0.0119457017853809\\
401	0.0119444033435415\\
402	0.0119430796982215\\
403	0.011941730316154\\
404	0.0119403546487081\\
405	0.0119389521310679\\
406	0.0119375221814206\\
407	0.0119360642001498\\
408	0.0119345775687085\\
409	0.0119330616463279\\
410	0.0119315157567108\\
411	0.0119299391885975\\
412	0.0119283312231195\\
413	0.0119266911136817\\
414	0.0119250180802577\\
415	0.0119233113066078\\
416	0.0119215699382179\\
417	0.0119197930787502\\
418	0.0119179797862459\\
419	0.0119161290697883\\
420	0.0119142398853442\\
421	0.0119123111245265\\
422	0.0119103416055459\\
423	0.0119083300512209\\
424	0.0119062750940739\\
425	0.0119041751959248\\
426	0.0119020284396605\\
427	0.0118997953672165\\
428	0.0118974635740099\\
429	0.0118950875155099\\
430	0.0118926663850904\\
431	0.0118901993685025\\
432	0.0118876856474064\\
433	0.0118851244013332\\
434	0.0118825147954175\\
435	0.0118798559913842\\
436	0.011877147149608\\
437	0.011874387431559\\
438	0.0118715760026899\\
439	0.0118687120358291\\
440	0.0118657947151098\\
441	0.0118628232403625\\
442	0.0118597968315135\\
443	0.0118567147312705\\
444	0.0118535762002721\\
445	0.0118503804854885\\
446	0.011847126698962\\
447	0.0118438134010928\\
448	0.0118405418911741\\
449	0.0118372564163759\\
450	0.0118338991808079\\
451	0.0118304681060618\\
452	0.0118269610110146\\
453	0.0118233756039658\\
454	0.0118197094760942\\
455	0.0118159600901843\\
456	0.0118121247700754\\
457	0.0118082006891739\\
458	0.0118041848582846\\
459	0.011800074113977\\
460	0.0117958651122213\\
461	0.0117915543441192\\
462	0.011787138234213\\
463	0.0117826135366853\\
464	0.0117779788034406\\
465	0.0117732250141751\\
466	0.0117683475127057\\
467	0.0117633415352039\\
468	0.0117582020424558\\
469	0.011752923699961\\
470	0.0117475008565411\\
471	0.0117419275213265\\
472	0.0117361973392071\\
473	0.0117303035651736\\
474	0.0117242390385759\\
475	0.0117179961607447\\
476	0.0117115668769278\\
477	0.0117049426402691\\
478	0.0116981142336251\\
479	0.0116910720678922\\
480	0.011683806465364\\
481	0.0116763086146761\\
482	0.011668564253965\\
483	0.011660561632207\\
484	0.0116522992044375\\
485	0.0116437448122876\\
486	0.0116348737459636\\
487	0.0116256678107017\\
488	0.0116161075874448\\
489	0.0116061721996318\\
490	0.0115958391076566\\
491	0.0115850840131436\\
492	0.0115738801518583\\
493	0.0115621983141678\\
494	0.0115500067552809\\
495	0.011537270864827\\
496	0.0115239526995846\\
497	0.0115099530042766\\
498	0.0114952191859894\\
499	0.0114797856347734\\
500	0.0114635972973228\\
501	0.011446592916784\\
502	0.0114287029652318\\
503	0.0114097171984323\\
504	0.0113892898865262\\
505	0.0113678037999125\\
506	0.0113451679302032\\
507	0.011321258044426\\
508	0.011295858921333\\
509	0.0112687074987874\\
510	0.0112226560914108\\
511	0.0111719528845412\\
512	0.01111856301296\\
513	0.0110556138408896\\
514	0.0109157644427097\\
515	0.0107738972814499\\
516	0.0106305018015266\\
517	0.0105057503087455\\
518	0.0104678344588721\\
519	0.0104344661035434\\
520	0.0104068452357926\\
521	0.0103847982977021\\
522	0.0103633618154138\\
523	0.0103423529381083\\
524	0.0103214662003717\\
525	0.0103003009616101\\
526	0.0102787966292368\\
527	0.0102568947141158\\
528	0.0102345466566108\\
529	0.0102117314714084\\
530	0.0101884310406324\\
531	0.0101646294416194\\
532	0.0101403129400221\\
533	0.0101154687499199\\
534	0.0100900843226428\\
535	0.0100641467156925\\
536	0.0100376393920876\\
537	0.0100102412688675\\
538	0.00998219994812182\\
539	0.0099534004265039\\
540	0.00992268727859505\\
541	0.00989203974073431\\
542	0.0098627115636891\\
543	0.00983425094425008\\
544	0.0098053070252231\\
545	0.00977578618017489\\
546	0.00974565412076706\\
547	0.0097148919859461\\
548	0.00968348490934231\\
549	0.00965141830360132\\
550	0.00961867881385174\\
551	0.00958525680464518\\
552	0.0095511543400607\\
553	0.00951640895634259\\
554	0.00948103117320042\\
555	0.00944488185616451\\
556	0.00940704990361946\\
557	0.00936665220095536\\
558	0.00932392445885881\\
559	0.00928762399433467\\
560	0.00925130106126414\\
561	0.0092142996705098\\
562	0.0091766000002963\\
563	0.00913818673266915\\
564	0.00909904397822049\\
565	0.00905915473912828\\
566	0.00896097057670297\\
567	0.00855382259809112\\
568	0.0082695782015922\\
569	0.00820695731592837\\
570	0.00814336483038243\\
571	0.00807878132936148\\
572	0.00801318727806461\\
573	0.00794656249935759\\
574	0.00787888614375365\\
575	0.00781013666034054\\
576	0.00774029176795637\\
577	0.0076693284268806\\
578	0.00759722281115925\\
579	0.00752395028094279\\
580	0.0074494853520162\\
581	0.0073738016536067\\
582	0.00729687184929793\\
583	0.00721866745299389\\
584	0.00713915835966271\\
585	0.00705831161801653\\
586	0.00697608821123101\\
587	0.00689243463533687\\
588	0.00680726093889688\\
589	0.00672038360986785\\
590	0.00663137734622848\\
591	0.00653960315341497\\
592	0.00644346597070807\\
593	0.00633880868635237\\
594	0.00621472953405218\\
595	0.00604257880714497\\
596	0.00574702137257038\\
597	0.0051299581010561\\
598	0.00367057462141144\\
599	0\\
600	0\\
};
\addplot [color=black!60!mycolor21,solid,forget plot]
  table[row sep=crcr]{%
1	0.0120083999454855\\
2	0.0120083992290398\\
3	0.0120083984993837\\
4	0.0120083977562728\\
5	0.0120083969994581\\
6	0.0120083962286859\\
7	0.012008395443698\\
8	0.0120083946442312\\
9	0.0120083938300175\\
10	0.0120083930007839\\
11	0.0120083921562523\\
12	0.0120083912961395\\
13	0.0120083904201569\\
14	0.0120083895280107\\
15	0.0120083886194014\\
16	0.0120083876940243\\
17	0.0120083867515687\\
18	0.0120083857917183\\
19	0.0120083848141508\\
20	0.012008383818538\\
21	0.0120083828045455\\
22	0.0120083817718327\\
23	0.0120083807200528\\
24	0.0120083796488524\\
25	0.0120083785578716\\
26	0.0120083774467437\\
27	0.0120083763150952\\
28	0.0120083751625458\\
29	0.0120083739887079\\
30	0.0120083727931868\\
31	0.0120083715755804\\
32	0.0120083703354793\\
33	0.0120083690724662\\
34	0.0120083677861161\\
35	0.0120083664759962\\
36	0.0120083651416656\\
37	0.012008363782675\\
38	0.012008362398567\\
39	0.0120083609888756\\
40	0.0120083595531259\\
41	0.0120083580908344\\
42	0.0120083566015086\\
43	0.0120083550846465\\
44	0.0120083535397371\\
45	0.0120083519662597\\
46	0.012008350363684\\
47	0.0120083487314695\\
48	0.0120083470690661\\
49	0.012008345375913\\
50	0.0120083436514392\\
51	0.012008341895063\\
52	0.0120083401061917\\
53	0.0120083382842217\\
54	0.0120083364285381\\
55	0.0120083345385145\\
56	0.0120083326135128\\
57	0.012008330652883\\
58	0.012008328655963\\
59	0.0120083266220782\\
60	0.0120083245505415\\
61	0.012008322440653\\
62	0.0120083202916997\\
63	0.0120083181029553\\
64	0.0120083158736798\\
65	0.0120083136031195\\
66	0.0120083112905066\\
67	0.012008308935059\\
68	0.0120083065359798\\
69	0.0120083040924573\\
70	0.0120083016036647\\
71	0.0120082990687596\\
72	0.012008296486884\\
73	0.0120082938571638\\
74	0.0120082911787085\\
75	0.0120082884506108\\
76	0.0120082856719468\\
77	0.0120082828417751\\
78	0.0120082799591365\\
79	0.0120082770230542\\
80	0.0120082740325329\\
81	0.0120082709865587\\
82	0.0120082678840986\\
83	0.0120082647241006\\
84	0.0120082615054927\\
85	0.012008258227183\\
86	0.0120082548880588\\
87	0.012008251486987\\
88	0.0120082480228131\\
89	0.0120082444943608\\
90	0.012008240900432\\
91	0.012008237239806\\
92	0.0120082335112392\\
93	0.0120082297134649\\
94	0.0120082258451923\\
95	0.0120082219051066\\
96	0.0120082178918685\\
97	0.0120082138041131\\
98	0.0120082096404505\\
99	0.0120082053994641\\
100	0.0120082010797112\\
101	0.0120081966797219\\
102	0.0120081921979984\\
103	0.0120081876330152\\
104	0.012008182983218\\
105	0.0120081782470232\\
106	0.0120081734228176\\
107	0.0120081685089576\\
108	0.0120081635037689\\
109	0.0120081584055454\\
110	0.0120081532125492\\
111	0.0120081479230098\\
112	0.0120081425351231\\
113	0.0120081370470514\\
114	0.0120081314569222\\
115	0.012008125762828\\
116	0.0120081199628252\\
117	0.0120081140549339\\
118	0.0120081080371366\\
119	0.0120081019073781\\
120	0.0120080956635644\\
121	0.0120080893035621\\
122	0.0120080828251976\\
123	0.0120080762262566\\
124	0.0120080695044826\\
125	0.0120080626575771\\
126	0.012008055683198\\
127	0.0120080485789591\\
128	0.0120080413424293\\
129	0.0120080339711315\\
130	0.0120080264625422\\
131	0.01200801881409\\
132	0.0120080110231552\\
133	0.0120080030870685\\
134	0.0120079950031104\\
135	0.0120079867685099\\
136	0.0120079783804438\\
137	0.0120079698360357\\
138	0.0120079611323546\\
139	0.0120079522664144\\
140	0.0120079432351724\\
141	0.0120079340355286\\
142	0.0120079246643242\\
143	0.0120079151183407\\
144	0.012007905394299\\
145	0.0120078954888577\\
146	0.0120078853986122\\
147	0.0120078751200938\\
148	0.0120078646497676\\
149	0.0120078539840324\\
150	0.0120078431192184\\
151	0.0120078320515864\\
152	0.0120078207773265\\
153	0.0120078092925563\\
154	0.0120077975933202\\
155	0.0120077856755873\\
156	0.0120077735352504\\
157	0.0120077611681245\\
158	0.012007748569945\\
159	0.0120077357363665\\
160	0.012007722662961\\
161	0.0120077093452167\\
162	0.0120076957785356\\
163	0.0120076819582329\\
164	0.0120076678795345\\
165	0.0120076535375756\\
166	0.0120076389273987\\
167	0.0120076240439524\\
168	0.0120076088820889\\
169	0.0120075934365625\\
170	0.0120075777020277\\
171	0.0120075616730371\\
172	0.0120075453440396\\
173	0.0120075287093783\\
174	0.0120075117632884\\
175	0.0120074944998953\\
176	0.0120074769132122\\
177	0.0120074589971382\\
178	0.0120074407454559\\
179	0.0120074221518291\\
180	0.0120074032098006\\
181	0.0120073839127898\\
182	0.0120073642540903\\
183	0.0120073442268675\\
184	0.012007323824156\\
185	0.012007303038857\\
186	0.0120072818637358\\
187	0.0120072602914194\\
188	0.012007238314393\\
189	0.0120072159249981\\
190	0.0120071931154292\\
191	0.012007169877731\\
192	0.0120071462037955\\
193	0.012007122085359\\
194	0.0120070975139989\\
195	0.0120070724811309\\
196	0.0120070469780052\\
197	0.0120070209957041\\
198	0.0120069945251379\\
199	0.0120069675570418\\
200	0.0120069400819725\\
201	0.0120069120903047\\
202	0.0120068835722272\\
203	0.0120068545177394\\
204	0.0120068249166477\\
205	0.0120067947585614\\
206	0.0120067640328888\\
207	0.0120067327288332\\
208	0.012006700835389\\
209	0.0120066683413374\\
210	0.0120066352352419\\
211	0.0120066015054443\\
212	0.0120065671400599\\
213	0.0120065321269735\\
214	0.012006496453834\\
215	0.0120064601080502\\
216	0.012006423076786\\
217	0.0120063853469548\\
218	0.0120063469052153\\
219	0.0120063077379657\\
220	0.0120062678313385\\
221	0.0120062271711956\\
222	0.012006185743122\\
223	0.0120061435324209\\
224	0.0120061005241073\\
225	0.0120060567029026\\
226	0.0120060120532283\\
227	0.0120059665592\\
228	0.0120059202046209\\
229	0.0120058729729756\\
230	0.0120058248474234\\
231	0.0120057758107916\\
232	0.0120057258455685\\
233	0.0120056749338965\\
234	0.0120056230575649\\
235	0.0120055701980025\\
236	0.0120055163362699\\
237	0.0120054614530522\\
238	0.0120054055286508\\
239	0.0120053485429752\\
240	0.012005290475535\\
241	0.0120052313054316\\
242	0.0120051710113491\\
243	0.0120051095715457\\
244	0.0120050469638447\\
245	0.012004983165625\\
246	0.0120049181538117\\
247	0.0120048519048665\\
248	0.0120047843947774\\
249	0.0120047155990489\\
250	0.0120046454926912\\
251	0.0120045740502095\\
252	0.0120045012455933\\
253	0.0120044270523047\\
254	0.0120043514432674\\
255	0.0120042743908542\\
256	0.0120041958668756\\
257	0.0120041158425667\\
258	0.0120040342885749\\
259	0.0120039511749465\\
260	0.0120038664711137\\
261	0.0120037801458803\\
262	0.0120036921674079\\
263	0.0120036025032013\\
264	0.0120035111200936\\
265	0.0120034179842311\\
266	0.0120033230610573\\
267	0.0120032263152968\\
268	0.0120031277109391\\
269	0.0120030272112213\\
270	0.0120029247786104\\
271	0.012002820374786\\
272	0.0120027139606211\\
273	0.0120026054961635\\
274	0.0120024949406162\\
275	0.0120023822523173\\
276	0.0120022673887192\\
277	0.0120021503063676\\
278	0.0120020309608793\\
279	0.0120019093069199\\
280	0.0120017852981803\\
281	0.0120016588873531\\
282	0.0120015300261077\\
283	0.0120013986650651\\
284	0.0120012647537715\\
285	0.0120011282406717\\
286	0.0120009890730808\\
287	0.012000847197156\\
288	0.0120007025578665\\
289	0.0120005550989635\\
290	0.0120004047629481\\
291	0.0120002514910393\\
292	0.01200009522314\\
293	0.0119999358978029\\
294	0.0119997734521938\\
295	0.0119996078220554\\
296	0.011999438941669\\
297	0.0119992667438148\\
298	0.0119990911597312\\
299	0.011998912119073\\
300	0.011998729549868\\
301	0.0119985433784717\\
302	0.0119983535295216\\
303	0.0119981599258894\\
304	0.0119979624886314\\
305	0.0119977611369384\\
306	0.0119975557880829\\
307	0.0119973463573659\\
308	0.0119971327580614\\
309	0.0119969149013599\\
310	0.0119966926963101\\
311	0.0119964660497599\\
312	0.0119962348662954\\
313	0.0119959990481792\\
314	0.0119957584952878\\
315	0.0119955131050481\\
316	0.0119952627723735\\
317	0.0119950073895998\\
318	0.0119947468464216\\
319	0.0119944810298287\\
320	0.011994209824045\\
321	0.011993933110468\\
322	0.011993650767612\\
323	0.0119933626710548\\
324	0.0119930686933893\\
325	0.0119927687041809\\
326	0.011992462569933\\
327	0.0119921501540617\\
328	0.0119918313168827\\
329	0.0119915059156111\\
330	0.0119911738043787\\
331	0.0119908348342715\\
332	0.0119904888533901\\
333	0.0119901357069385\\
334	0.0119897752373463\\
335	0.0119894072844285\\
336	0.0119890316855904\\
337	0.0119886482760851\\
338	0.0119882568893294\\
339	0.0119878573572838\\
340	0.0119874495108946\\
341	0.0119870331805696\\
342	0.0119866081965822\\
343	0.0119861743890787\\
344	0.0119857315867723\\
345	0.0119852796119969\\
346	0.0119848166276532\\
347	0.0119843399739015\\
348	0.0119838540644975\\
349	0.0119833587172524\\
350	0.0119828537462952\\
351	0.0119823389620415\\
352	0.0119818141711709\\
353	0.0119812791764767\\
354	0.0119807337759276\\
355	0.0119801777583165\\
356	0.011979610912911\\
357	0.0119790330248667\\
358	0.0119784438747102\\
359	0.0119778432381981\\
360	0.011977230886168\\
361	0.0119766065843782\\
362	0.0119759700933378\\
363	0.0119753211681232\\
364	0.0119746595581818\\
365	0.0119739850071199\\
366	0.0119732972524738\\
367	0.0119725960254619\\
368	0.0119718810507148\\
369	0.0119711520459823\\
370	0.0119704087218128\\
371	0.011969650781203\\
372	0.0119688779192128\\
373	0.0119680898225433\\
374	0.0119672861690705\\
375	0.0119664666273313\\
376	0.0119656308559546\\
377	0.0119647785030307\\
378	0.0119639092054113\\
379	0.0119630225879305\\
380	0.0119621182625372\\
381	0.0119611958273259\\
382	0.0119602548654549\\
383	0.0119592949439332\\
384	0.0119583156122614\\
385	0.0119573164009048\\
386	0.0119562968195756\\
387	0.0119552563552967\\
388	0.0119541944702143\\
389	0.0119531105991187\\
390	0.0119520041466189\\
391	0.0119508744838829\\
392	0.0119497209447728\\
393	0.0119485428209458\\
394	0.0119473393546667\\
395	0.0119461097252672\\
396	0.0119448530151993\\
397	0.0119435681047047\\
398	0.0119422383568506\\
399	0.0119408381933469\\
400	0.0119394115029898\\
401	0.0119379578218509\\
402	0.0119364766815374\\
403	0.0119349676097115\\
404	0.0119334301308569\\
405	0.0119318637674778\\
406	0.0119302680419956\\
407	0.0119286424794024\\
408	0.0119269866090724\\
409	0.0119252999565169\\
410	0.011923581986418\\
411	0.0119218321164078\\
412	0.0119200498658816\\
413	0.0119182347695394\\
414	0.0119163863634334\\
415	0.0119145041873502\\
416	0.0119125877928737\\
417	0.0119106367464193\\
418	0.0119086506337675\\
419	0.0119066290700301\\
420	0.0119045717092323\\
421	0.0119024782236954\\
422	0.0119003483023212\\
423	0.0118981816019366\\
424	0.0118959778788493\\
425	0.0118937368455432\\
426	0.0118914579408957\\
427	0.0118891887091366\\
428	0.0118869441862681\\
429	0.0118846541518084\\
430	0.0118823175737925\\
431	0.0118799333843454\\
432	0.0118775004771255\\
433	0.011875017704343\\
434	0.0118724838739019\\
435	0.0118698977460056\\
436	0.0118672580293897\\
437	0.0118645633771347\\
438	0.0118618123820097\\
439	0.0118590035713047\\
440	0.011856135401147\\
441	0.0118532062504605\\
442	0.0118502144153103\\
443	0.0118471581064472\\
444	0.0118440354602067\\
445	0.0118408445989951\\
446	0.0118375838710563\\
447	0.0118342527368183\\
448	0.0118308471871242\\
449	0.0118273636308956\\
450	0.0118237997097628\\
451	0.0118201529545751\\
452	0.0118164207783623\\
453	0.0118126004688413\\
454	0.0118086891803352\\
455	0.0118046839252094\\
456	0.0118005815647688\\
457	0.0117963787996397\\
458	0.0117920721597538\\
459	0.011787657994278\\
460	0.0117831324622632\\
461	0.0117784915252294\\
462	0.0117737309408854\\
463	0.0117688462388229\\
464	0.0117638325589291\\
465	0.0117586848956967\\
466	0.0117533979540501\\
467	0.0117479661232025\\
468	0.0117423834527318\\
469	0.0117366436266832\\
470	0.0117307399356738\\
471	0.0117246652471789\\
472	0.0117184119746498\\
473	0.0117119720472593\\
474	0.011705336884405\\
475	0.0116984973597942\\
476	0.0116914438311097\\
477	0.0116841665175684\\
478	0.0116766562487449\\
479	0.0116688996236285\\
480	0.0116608841063311\\
481	0.0116526075184698\\
482	0.0116440391102595\\
483	0.0116351530867053\\
484	0.0116259310856454\\
485	0.0116163535250666\\
486	0.0116063987569243\\
487	0.01159604338395\\
488	0.0115852620541966\\
489	0.0115740271862129\\
490	0.0115623085538313\\
491	0.0115499825006261\\
492	0.011537082765613\\
493	0.011523610037454\\
494	0.0115095211315003\\
495	0.0114947679840367\\
496	0.0114792960318934\\
497	0.0114628164424628\\
498	0.0114452402107816\\
499	0.0114268386863519\\
500	0.0114075418343798\\
501	0.0113872698905347\\
502	0.0113659372749331\\
503	0.0113436049016238\\
504	0.0113204389430254\\
505	0.0112955664340549\\
506	0.0112566571397118\\
507	0.011208047680716\\
508	0.0111569488689571\\
509	0.0111025534637219\\
510	0.0109759252031797\\
511	0.0108376578587082\\
512	0.0106977706400336\\
513	0.0105644467407458\\
514	0.0105283856891288\\
515	0.0104967788563659\\
516	0.0104707771345457\\
517	0.0104503883291214\\
518	0.0104305912912045\\
519	0.0104111987509055\\
520	0.0103919022665037\\
521	0.0103723416624619\\
522	0.0103524595977559\\
523	0.0103322022561332\\
524	0.0103115282126467\\
525	0.0102904195643863\\
526	0.0102688603755856\\
527	0.0102468368505073\\
528	0.0102243372233087\\
529	0.0102013500035187\\
530	0.0101778638005055\\
531	0.0101538672704206\\
532	0.0101293490401102\\
533	0.0101042954728937\\
534	0.0100786224197955\\
535	0.0100520275724269\\
536	0.0100249756415505\\
537	0.00999625088143839\\
538	0.00996689579833516\\
539	0.00993799859609607\\
540	0.00991109342893662\\
541	0.00988373880067814\\
542	0.00985587160983944\\
543	0.00982743186874199\\
544	0.00979840193510684\\
545	0.00976876727679838\\
546	0.00973851392679122\\
547	0.00970762788032122\\
548	0.00967609499478674\\
549	0.0096439014059571\\
550	0.00961103476249886\\
551	0.00957748784851001\\
552	0.0095432689828405\\
553	0.00950843188070146\\
554	0.00947271650210162\\
555	0.00943541624537776\\
556	0.0093938705541174\\
557	0.00935497609595499\\
558	0.00931990416966743\\
559	0.00928420250295137\\
560	0.00924783077694059\\
561	0.00921077438624748\\
562	0.00917301857794923\\
563	0.00913454792749566\\
564	0.00909534588454923\\
565	0.00900221725383939\\
566	0.00858983448223343\\
567	0.0083312443361884\\
568	0.00826957769629984\\
569	0.00820695731222601\\
570	0.00814336482910577\\
571	0.00807878132872334\\
572	0.0080131872777402\\
573	0.0079465624991993\\
574	0.00787888614368079\\
575	0.00781013666030913\\
576	0.0077402917679442\\
577	0.00766932842687632\\
578	0.00759722281115773\\
579	0.00752395028094238\\
580	0.00744948535201612\\
581	0.0073738016536067\\
582	0.00729687184929793\\
583	0.0072186674529939\\
584	0.00713915835966273\\
585	0.00705831161801654\\
586	0.00697608821123101\\
587	0.00689243463533689\\
588	0.00680726093889688\\
589	0.00672038360986785\\
590	0.00663137734622847\\
591	0.00653960315341497\\
592	0.00644346597070807\\
593	0.00633880868635238\\
594	0.00621472953405218\\
595	0.00604257880714497\\
596	0.00574702137257039\\
597	0.0051299581010561\\
598	0.00367057462141144\\
599	0\\
600	0\\
};
\addplot [color=black!80!mycolor21,solid,forget plot]
  table[row sep=crcr]{%
1	0.0120061774987932\\
2	0.012006176709492\\
3	0.0120061759056528\\
4	0.0120061750870068\\
5	0.0120061742532806\\
6	0.0120061734041952\\
7	0.012006172539467\\
8	0.0120061716588066\\
9	0.0120061707619196\\
10	0.012006169848506\\
11	0.0120061689182602\\
12	0.012006167970871\\
13	0.0120061670060214\\
14	0.0120061660233885\\
15	0.0120061650226435\\
16	0.0120061640034514\\
17	0.0120061629654708\\
18	0.0120061619083544\\
19	0.0120061608317482\\
20	0.0120061597352914\\
21	0.012006158618617\\
22	0.0120061574813508\\
23	0.0120061563231117\\
24	0.0120061551435118\\
25	0.0120061539421556\\
26	0.0120061527186404\\
27	0.0120061514725562\\
28	0.0120061502034852\\
29	0.0120061489110019\\
30	0.0120061475946727\\
31	0.0120061462540562\\
32	0.0120061448887028\\
33	0.0120061434981543\\
34	0.0120061420819442\\
35	0.0120061406395972\\
36	0.0120061391706293\\
37	0.0120061376745472\\
38	0.0120061361508489\\
39	0.0120061345990227\\
40	0.0120061330185474\\
41	0.0120061314088923\\
42	0.0120061297695167\\
43	0.0120061280998698\\
44	0.0120061263993906\\
45	0.0120061246675078\\
46	0.0120061229036392\\
47	0.012006121107192\\
48	0.0120061192775622\\
49	0.0120061174141346\\
50	0.0120061155162827\\
51	0.0120061135833681\\
52	0.0120061116147406\\
53	0.0120061096097381\\
54	0.0120061075676858\\
55	0.0120061054878966\\
56	0.0120061033696705\\
57	0.0120061012122945\\
58	0.0120060990150423\\
59	0.0120060967771741\\
60	0.0120060944979362\\
61	0.0120060921765611\\
62	0.0120060898122667\\
63	0.0120060874042565\\
64	0.0120060849517192\\
65	0.0120060824538282\\
66	0.0120060799097416\\
67	0.0120060773186019\\
68	0.0120060746795354\\
69	0.0120060719916522\\
70	0.0120060692540459\\
71	0.0120060664657931\\
72	0.0120060636259531\\
73	0.0120060607335679\\
74	0.0120060577876614\\
75	0.0120060547872393\\
76	0.0120060517312889\\
77	0.0120060486187785\\
78	0.0120060454486571\\
79	0.0120060422198542\\
80	0.0120060389312793\\
81	0.0120060355818214\\
82	0.012006032170349\\
83	0.0120060286957092\\
84	0.0120060251567279\\
85	0.0120060215522086\\
86	0.012006017880933\\
87	0.0120060141416597\\
88	0.0120060103331242\\
89	0.0120060064540383\\
90	0.0120060025030899\\
91	0.0120059984789423\\
92	0.0120059943802338\\
93	0.0120059902055772\\
94	0.0120059859535594\\
95	0.012005981622741\\
96	0.0120059772116557\\
97	0.0120059727188095\\
98	0.0120059681426808\\
99	0.0120059634817193\\
100	0.0120059587343459\\
101	0.0120059538989518\\
102	0.0120059489738983\\
103	0.0120059439575158\\
104	0.0120059388481037\\
105	0.0120059336439294\\
106	0.0120059283432279\\
107	0.0120059229442013\\
108	0.0120059174450179\\
109	0.0120059118438117\\
110	0.012005906138682\\
111	0.0120059003276922\\
112	0.0120058944088697\\
113	0.0120058883802047\\
114	0.01200588223965\\
115	0.0120058759851198\\
116	0.0120058696144895\\
117	0.0120058631255946\\
118	0.0120058565162298\\
119	0.0120058497841488\\
120	0.0120058429270631\\
121	0.0120058359426411\\
122	0.0120058288285079\\
123	0.0120058215822436\\
124	0.0120058142013833\\
125	0.0120058066834156\\
126	0.0120057990257822\\
127	0.0120057912258768\\
128	0.0120057832810439\\
129	0.0120057751885786\\
130	0.012005766945725\\
131	0.0120057585496756\\
132	0.0120057499975699\\
133	0.0120057412864942\\
134	0.0120057324134796\\
135	0.0120057233755018\\
136	0.0120057141694796\\
137	0.0120057047922737\\
138	0.0120056952406863\\
139	0.0120056855114591\\
140	0.0120056756012728\\
141	0.0120056655067455\\
142	0.0120056552244321\\
143	0.0120056447508224\\
144	0.0120056340823403\\
145	0.0120056232153427\\
146	0.0120056121461175\\
147	0.0120056008708833\\
148	0.0120055893857872\\
149	0.0120055776869039\\
150	0.0120055657702344\\
151	0.0120055536317043\\
152	0.0120055412671624\\
153	0.0120055286723796\\
154	0.0120055158430471\\
155	0.0120055027747747\\
156	0.0120054894630899\\
157	0.0120054759034358\\
158	0.0120054620911694\\
159	0.0120054480215607\\
160	0.01200543368979\\
161	0.0120054190909473\\
162	0.0120054042200296\\
163	0.0120053890719397\\
164	0.0120053736414844\\
165	0.0120053579233724\\
166	0.0120053419122125\\
167	0.012005325602512\\
168	0.0120053089886743\\
169	0.0120052920649972\\
170	0.0120052748256711\\
171	0.0120052572647763\\
172	0.0120052393762816\\
173	0.0120052211540416\\
174	0.0120052025917949\\
175	0.0120051836831619\\
176	0.0120051644216422\\
177	0.0120051448006126\\
178	0.0120051248133246\\
179	0.012005104452902\\
180	0.0120050837123387\\
181	0.0120050625844957\\
182	0.0120050410620992\\
183	0.0120050191377376\\
184	0.0120049968038587\\
185	0.0120049740527676\\
186	0.0120049508766235\\
187	0.0120049272674367\\
188	0.0120049032170666\\
189	0.0120048787172178\\
190	0.0120048537594379\\
191	0.0120048283351138\\
192	0.0120048024354693\\
193	0.0120047760515615\\
194	0.0120047491742776\\
195	0.0120047217943319\\
196	0.0120046939022623\\
197	0.0120046654884267\\
198	0.012004636543\\
199	0.0120046070559701\\
200	0.0120045770171345\\
201	0.0120045464160967\\
202	0.0120045152422621\\
203	0.0120044834848347\\
204	0.0120044511328128\\
205	0.012004418174985\\
206	0.0120043845999262\\
207	0.0120043503959938\\
208	0.0120043155513228\\
209	0.0120042800538222\\
210	0.0120042438911698\\
211	0.0120042070508086\\
212	0.0120041695199414\\
213	0.0120041312855266\\
214	0.0120040923342734\\
215	0.0120040526526368\\
216	0.0120040122268126\\
217	0.0120039710427323\\
218	0.0120039290860581\\
219	0.0120038863421777\\
220	0.0120038427961986\\
221	0.0120037984329425\\
222	0.0120037532369403\\
223	0.0120037071924258\\
224	0.01200366028333\\
225	0.0120036124932753\\
226	0.012003563805569\\
227	0.0120035142031977\\
228	0.0120034636688202\\
229	0.0120034121847617\\
230	0.0120033597330066\\
231	0.0120033062951921\\
232	0.0120032518526013\\
233	0.0120031963861558\\
234	0.0120031398764089\\
235	0.0120030823035379\\
236	0.0120030236473368\\
237	0.0120029638872086\\
238	0.0120029030021574\\
239	0.0120028409707803\\
240	0.0120027777712593\\
241	0.012002713381353\\
242	0.012002647778388\\
243	0.0120025809392501\\
244	0.0120025128403753\\
245	0.0120024434577409\\
246	0.0120023727668561\\
247	0.0120023007427523\\
248	0.0120022273599734\\
249	0.012002152592566\\
250	0.012002076414069\\
251	0.0120019987975034\\
252	0.0120019197153613\\
253	0.0120018391395953\\
254	0.0120017570416072\\
255	0.0120016733922367\\
256	0.0120015881617494\\
257	0.0120015013198254\\
258	0.0120014128355465\\
259	0.0120013226773841\\
260	0.0120012308131859\\
261	0.0120011372101635\\
262	0.0120010418348782\\
263	0.0120009446532278\\
264	0.0120008456304322\\
265	0.0120007447310191\\
266	0.0120006419188091\\
267	0.0120005371569006\\
268	0.0120004304076541\\
269	0.0120003216326768\\
270	0.0120002107928054\\
271	0.01200009784809\\
272	0.0119999827577768\\
273	0.0119998654802903\\
274	0.0119997459732153\\
275	0.0119996241932781\\
276	0.0119995000963277\\
277	0.0119993736373161\\
278	0.0119992447702781\\
279	0.0119991134483105\\
280	0.0119989796235511\\
281	0.0119988432471563\\
282	0.0119987042692789\\
283	0.0119985626390449\\
284	0.0119984183045294\\
285	0.011998271212732\\
286	0.0119981213095513\\
287	0.0119979685397588\\
288	0.0119978128469717\\
289	0.0119976541736248\\
290	0.0119974924609419\\
291	0.0119973276489054\\
292	0.0119971596762256\\
293	0.0119969884803081\\
294	0.011996813997221\\
295	0.0119966361616595\\
296	0.0119964549069105\\
297	0.0119962701648143\\
298	0.011996081865726\\
299	0.0119958899384739\\
300	0.0119956943103174\\
301	0.0119954949069015\\
302	0.0119952916522098\\
303	0.0119950844685152\\
304	0.0119948732763274\\
305	0.011994657994338\\
306	0.0119944385393621\\
307	0.0119942148262765\\
308	0.011993986767954\\
309	0.0119937542751936\\
310	0.0119935172566454\\
311	0.011993275618731\\
312	0.0119930292655574\\
313	0.0119927780988248\\
314	0.0119925220177266\\
315	0.011992260918842\\
316	0.0119919946960187\\
317	0.0119917232402458\\
318	0.0119914464395154\\
319	0.0119911641786705\\
320	0.0119908763392395\\
321	0.0119905827992535\\
322	0.0119902834330452\\
323	0.011989978111028\\
324	0.0119896666994501\\
325	0.0119893490601235\\
326	0.011989025050122\\
327	0.0119886945214457\\
328	0.0119883573206474\\
329	0.0119880132884141\\
330	0.0119876622591007\\
331	0.0119873040602062\\
332	0.0119869385117864\\
333	0.0119865654257935\\
334	0.0119861846053333\\
335	0.0119857958438264\\
336	0.0119853989240613\\
337	0.0119849936171217\\
338	0.0119845796811653\\
339	0.0119841568600243\\
340	0.0119837248815598\\
341	0.0119832834556125\\
342	0.0119828322710647\\
343	0.0119823709904413\\
344	0.0119818992355815\\
345	0.0119814165400692\\
346	0.0119809158212517\\
347	0.0119803868719465\\
348	0.0119798476623257\\
349	0.0119792979956718\\
350	0.0119787376720487\\
351	0.0119781664885482\\
352	0.011977584239588\\
353	0.0119769907165609\\
354	0.0119763857035599\\
355	0.0119757689563677\\
356	0.0119751402511398\\
357	0.0119744993623042\\
358	0.011973846060654\\
359	0.0119731801133499\\
360	0.0119725012839301\\
361	0.0119718093323295\\
362	0.0119711040149076\\
363	0.0119703850844885\\
364	0.0119696522904124\\
365	0.0119689053786018\\
366	0.0119681440916433\\
367	0.0119673681688878\\
368	0.01196657734657\\
369	0.0119657713579523\\
370	0.0119649499334936\\
371	0.0119641128010481\\
372	0.0119632596860968\\
373	0.0119623903120172\\
374	0.0119615044003952\\
375	0.0119606016713853\\
376	0.0119596818441255\\
377	0.0119587446372139\\
378	0.0119577897692559\\
379	0.0119568169594911\\
380	0.0119558259285101\\
381	0.0119548163990757\\
382	0.0119537880970601\\
383	0.0119527407525164\\
384	0.0119516741009033\\
385	0.0119505878844829\\
386	0.0119494818539196\\
387	0.0119483557701061\\
388	0.01194720940625\\
389	0.0119460425502514\\
390	0.0119448550073976\\
391	0.0119436466033572\\
392	0.0119424171873401\\
393	0.0119411666349045\\
394	0.0119398948487485\\
395	0.0119386017523298\\
396	0.011937287260403\\
397	0.0119359511770441\\
398	0.0119346129130317\\
399	0.0119333045783006\\
400	0.0119319700990121\\
401	0.0119306089503952\\
402	0.0119292205958899\\
403	0.0119278044866776\\
404	0.0119263600611732\\
405	0.0119248867444738\\
406	0.0119233839477297\\
407	0.0119218510673435\\
408	0.0119202874837817\\
409	0.0119186925598882\\
410	0.0119170656409285\\
411	0.0119154060558225\\
412	0.0119137131138388\\
413	0.0119119861028877\\
414	0.0119102242880122\\
415	0.0119084269097514\\
416	0.011906593182167\\
417	0.0119047222906505\\
418	0.0119028133894989\\
419	0.0119008655990366\\
420	0.0118988780020343\\
421	0.0118968496407517\\
422	0.0118947795156533\\
423	0.0118926665943251\\
424	0.0118905098435751\\
425	0.0118883083631071\\
426	0.0118860618791172\\
427	0.0118837689398346\\
428	0.0118814267482418\\
429	0.0118790340959398\\
430	0.0118765897331702\\
431	0.0118740923666976\\
432	0.0118715406575671\\
433	0.0118689332187405\\
434	0.0118662686125926\\
435	0.0118635453482694\\
436	0.0118607618789043\\
437	0.0118579165986911\\
438	0.0118550078398167\\
439	0.0118520338692655\\
440	0.0118489928855326\\
441	0.0118458830153412\\
442	0.0118427023106056\\
443	0.011839448746161\\
444	0.0118361202191271\\
445	0.0118327145498347\\
446	0.0118292294741894\\
447	0.011825662560535\\
448	0.0118220113037117\\
449	0.011818273122127\\
450	0.0118144453145193\\
451	0.0118105250521574\\
452	0.0118065093704263\\
453	0.0118023951597424\\
454	0.0117981791557413\\
455	0.0117938579286678\\
456	0.0117894278719001\\
457	0.0117848851895276\\
458	0.0117802258829056\\
459	0.0117754457360854\\
460	0.0117705402999514\\
461	0.0117655048746742\\
462	0.0117603344896774\\
463	0.0117550238808451\\
464	0.0117495674744412\\
465	0.0117439593584629\\
466	0.0117381932558002\\
467	0.0117322624952381\\
468	0.011726159980219\\
469	0.0117198781558529\\
470	0.0117134089756929\\
471	0.0117067438721048\\
472	0.0116998737387487\\
473	0.0116927889417047\\
474	0.011685479399855\\
475	0.011677935278597\\
476	0.0116701460952378\\
477	0.0116620969594819\\
478	0.0116537838744283\\
479	0.0116451812208078\\
480	0.0116362593065418\\
481	0.0116269990106955\\
482	0.0116173798686302\\
483	0.0116073797521621\\
484	0.0115969646076894\\
485	0.0115860096822867\\
486	0.0115746130075835\\
487	0.0115627437848719\\
488	0.0115503681119168\\
489	0.0115374482284512\\
490	0.0115239406477969\\
491	0.0115094395139539\\
492	0.0114941769734634\\
493	0.0114782667784652\\
494	0.0114616601452876\\
495	0.0114443017805102\\
496	0.0114261274901885\\
497	0.0114073571887202\\
498	0.0113879503821884\\
499	0.0113673979688879\\
500	0.0113455460417468\\
501	0.0113221550045847\\
502	0.011291235525038\\
503	0.011244706045718\\
504	0.0111958847102135\\
505	0.0111440786457605\\
506	0.0110416353212969\\
507	0.0109070246505103\\
508	0.0107706358227658\\
509	0.0106332026419147\\
510	0.0105863672131381\\
511	0.0105559005716357\\
512	0.0105307881680949\\
513	0.0105118292321733\\
514	0.0104934587917271\\
515	0.0104754960071344\\
516	0.0104576455041051\\
517	0.01043954987302\\
518	0.0104211547918156\\
519	0.0104024100947646\\
520	0.0103832787818694\\
521	0.0103637446992966\\
522	0.0103437936222919\\
523	0.0103234133542831\\
524	0.010302593373959\\
525	0.0102813233510255\\
526	0.0102595930522392\\
527	0.0102373922284501\\
528	0.0102147104578364\\
529	0.010191537102131\\
530	0.010167861290119\\
531	0.0101436685171106\\
532	0.0101187003218401\\
533	0.0100930566867924\\
534	0.0100667056447144\\
535	0.0100385143062766\\
536	0.0100104463396591\\
537	0.00998410516278209\\
538	0.00995823425319883\\
539	0.0099319192649669\\
540	0.00990506725908214\\
541	0.00987766172597634\\
542	0.00984968787746924\\
543	0.00982113263253384\\
544	0.0097919828010765\\
545	0.00976222489079139\\
546	0.00973184506412253\\
547	0.00970082916457133\\
548	0.00966916287773498\\
549	0.00963683223760079\\
550	0.00960382512425271\\
551	0.00957013557458552\\
552	0.00953577600451852\\
553	0.00950030228599682\\
554	0.00946282763855322\\
555	0.00942151480211031\\
556	0.00938636600758131\\
557	0.00935191421889609\\
558	0.00931682028034871\\
559	0.00928106926782935\\
560	0.00924464730730252\\
561	0.00920754010879173\\
562	0.00916973275239304\\
563	0.00913120920249568\\
564	0.00904848868442004\\
565	0.00865759593881743\\
566	0.00839197492902687\\
567	0.0083312443109848\\
568	0.00826957769583891\\
569	0.00820695731204064\\
570	0.00814336482901296\\
571	0.00807878132867731\\
572	0.00801318727771863\\
573	0.00794656249918969\\
574	0.00787888614367679\\
575	0.00781013666030784\\
576	0.0077402917679436\\
577	0.0076693284268761\\
578	0.00759722281115764\\
579	0.00752395028094237\\
580	0.00744948535201613\\
581	0.00737380165360671\\
582	0.00729687184929793\\
583	0.00721866745299389\\
584	0.00713915835966271\\
585	0.00705831161801653\\
586	0.00697608821123102\\
587	0.0068924346353369\\
588	0.0068072609388969\\
589	0.00672038360986785\\
590	0.00663137734622847\\
591	0.00653960315341497\\
592	0.00644346597070807\\
593	0.00633880868635238\\
594	0.00621472953405218\\
595	0.00604257880714497\\
596	0.00574702137257038\\
597	0.0051299581010561\\
598	0.00367057462141144\\
599	0\\
600	0\\
};
\addplot [color=black,solid,forget plot]
  table[row sep=crcr]{%
1	0.012002334618024\\
2	0.0120023337085181\\
3	0.0120023327822967\\
4	0.0120023318390514\\
5	0.0120023308784687\\
6	0.0120023299002287\\
7	0.012002328904006\\
8	0.012002327889469\\
9	0.01200232685628\\
10	0.0120023258040951\\
11	0.0120023247325639\\
12	0.0120023236413297\\
13	0.0120023225300291\\
14	0.0120023213982919\\
15	0.0120023202457412\\
16	0.0120023190719931\\
17	0.0120023178766565\\
18	0.012002316659333\\
19	0.0120023154196172\\
20	0.0120023141570956\\
21	0.0120023128713477\\
22	0.0120023115619447\\
23	0.0120023102284501\\
24	0.0120023088704192\\
25	0.0120023074873992\\
26	0.0120023060789289\\
27	0.0120023046445384\\
28	0.0120023031837493\\
29	0.0120023016960741\\
30	0.0120023001810165\\
31	0.0120022986380709\\
32	0.0120022970667224\\
33	0.0120022954664463\\
34	0.0120022938367086\\
35	0.0120022921769651\\
36	0.0120022904866616\\
37	0.0120022887652336\\
38	0.0120022870121062\\
39	0.0120022852266938\\
40	0.0120022834083999\\
41	0.012002281556617\\
42	0.0120022796707264\\
43	0.0120022777500978\\
44	0.0120022757940893\\
45	0.012002273802047\\
46	0.012002271773305\\
47	0.0120022697071849\\
48	0.0120022676029959\\
49	0.0120022654600342\\
50	0.0120022632775831\\
51	0.0120022610549125\\
52	0.0120022587912788\\
53	0.0120022564859245\\
54	0.0120022541380783\\
55	0.0120022517469543\\
56	0.0120022493117521\\
57	0.0120022468316566\\
58	0.0120022443058373\\
59	0.0120022417334486\\
60	0.0120022391136289\\
61	0.0120022364455008\\
62	0.0120022337281705\\
63	0.0120022309607276\\
64	0.012002228142245\\
65	0.0120022252717781\\
66	0.012002222348365\\
67	0.0120022193710256\\
68	0.012002216338762\\
69	0.0120022132505575\\
70	0.0120022101053766\\
71	0.0120022069021646\\
72	0.012002203639847\\
73	0.0120022003173296\\
74	0.0120021969334977\\
75	0.012002193487216\\
76	0.012002189977328\\
77	0.0120021864026556\\
78	0.0120021827619991\\
79	0.0120021790541363\\
80	0.0120021752778223\\
81	0.0120021714317891\\
82	0.012002167514745\\
83	0.0120021635253744\\
84	0.0120021594623373\\
85	0.0120021553242685\\
86	0.0120021511097777\\
87	0.0120021468174487\\
88	0.0120021424458387\\
89	0.0120021379934784\\
90	0.012002133458871\\
91	0.0120021288404917\\
92	0.0120021241367877\\
93	0.012002119346177\\
94	0.0120021144670481\\
95	0.0120021094977599\\
96	0.0120021044366404\\
97	0.0120020992819865\\
98	0.0120020940320636\\
99	0.0120020886851047\\
100	0.0120020832393098\\
101	0.0120020776928455\\
102	0.0120020720438443\\
103	0.0120020662904038\\
104	0.0120020604305862\\
105	0.0120020544624177\\
106	0.0120020483838877\\
107	0.012002042192948\\
108	0.0120020358875124\\
109	0.0120020294654558\\
110	0.0120020229246135\\
111	0.0120020162627805\\
112	0.0120020094777106\\
113	0.0120020025671159\\
114	0.0120019955286657\\
115	0.012001988359986\\
116	0.0120019810586583\\
117	0.0120019736222194\\
118	0.0120019660481598\\
119	0.0120019583339236\\
120	0.0120019504769068\\
121	0.0120019424744572\\
122	0.0120019343238729\\
123	0.0120019260224019\\
124	0.0120019175672405\\
125	0.0120019089555329\\
126	0.0120019001843702\\
127	0.0120018912507889\\
128	0.0120018821517705\\
129	0.0120018728842399\\
130	0.0120018634450648\\
131	0.0120018538310544\\
132	0.0120018440389585\\
133	0.012001834065466\\
134	0.0120018239072042\\
135	0.0120018135607372\\
136	0.0120018030225652\\
137	0.012001792289123\\
138	0.0120017813567788\\
139	0.0120017702218329\\
140	0.0120017588805167\\
141	0.0120017473289912\\
142	0.0120017355633456\\
143	0.012001723579596\\
144	0.0120017113736844\\
145	0.0120016989414765\\
146	0.0120016862787613\\
147	0.0120016733812488\\
148	0.0120016602445688\\
149	0.0120016468642698\\
150	0.0120016332358167\\
151	0.0120016193545899\\
152	0.0120016052158834\\
153	0.0120015908149033\\
154	0.0120015761467659\\
155	0.0120015612064965\\
156	0.0120015459890271\\
157	0.0120015304891952\\
158	0.0120015147017416\\
159	0.0120014986213088\\
160	0.0120014822424392\\
161	0.0120014655595731\\
162	0.0120014485670468\\
163	0.0120014312590906\\
164	0.0120014136298271\\
165	0.0120013956732686\\
166	0.0120013773833157\\
167	0.0120013587537547\\
168	0.0120013397782558\\
169	0.0120013204503706\\
170	0.0120013007635302\\
171	0.0120012807110426\\
172	0.0120012602860907\\
173	0.0120012394817296\\
174	0.0120012182908846\\
175	0.0120011967063485\\
176	0.0120011747207792\\
177	0.0120011523266972\\
178	0.0120011295164827\\
179	0.0120011062823736\\
180	0.0120010826164622\\
181	0.0120010585106926\\
182	0.0120010339568584\\
183	0.0120010089465993\\
184	0.0120009834713983\\
185	0.0120009575225789\\
186	0.0120009310913023\\
187	0.0120009041685639\\
188	0.0120008767451903\\
189	0.0120008488118364\\
190	0.012000820358982\\
191	0.0120007913769284\\
192	0.0120007618557951\\
193	0.0120007317855166\\
194	0.0120007011558386\\
195	0.0120006699563145\\
196	0.0120006381763023\\
197	0.0120006058049599\\
198	0.0120005728312424\\
199	0.0120005392438976\\
200	0.0120005050314624\\
201	0.0120004701822586\\
202	0.0120004346843894\\
203	0.0120003985257344\\
204	0.0120003616939461\\
205	0.0120003241764456\\
206	0.0120002859604177\\
207	0.012000247032807\\
208	0.0120002073803132\\
209	0.0120001669893865\\
210	0.0120001258462228\\
211	0.012000083936759\\
212	0.0120000412466682\\
213	0.0119999977613545\\
214	0.0119999534659483\\
215	0.0119999083453008\\
216	0.011999862383979\\
217	0.01199981556626\\
218	0.0119997678761261\\
219	0.0119997192972586\\
220	0.0119996698130327\\
221	0.0119996194065114\\
222	0.0119995680604398\\
223	0.0119995157572388\\
224	0.0119994624789995\\
225	0.0119994082074765\\
226	0.0119993529240819\\
227	0.0119992966098788\\
228	0.0119992392455744\\
229	0.0119991808115139\\
230	0.011999121287673\\
231	0.0119990606536515\\
232	0.0119989988886659\\
233	0.0119989359715423\\
234	0.0119988718807091\\
235	0.0119988065941894\\
236	0.0119987400895934\\
237	0.0119986723441107\\
238	0.0119986033345021\\
239	0.0119985330370921\\
240	0.01199846142776\\
241	0.0119983884819321\\
242	0.0119983141745727\\
243	0.0119982384801758\\
244	0.011998161372756\\
245	0.0119980828258397\\
246	0.0119980028124555\\
247	0.0119979213051253\\
248	0.0119978382758545\\
249	0.0119977536961225\\
250	0.0119976675368724\\
251	0.0119975797685015\\
252	0.0119974903608506\\
253	0.0119973992831935\\
254	0.0119973065042269\\
255	0.011997211992059\\
256	0.0119971157141985\\
257	0.0119970176375438\\
258	0.0119969177283711\\
259	0.0119968159523229\\
260	0.0119967122743963\\
261	0.0119966066589306\\
262	0.0119964990695953\\
263	0.0119963894693772\\
264	0.0119962778205682\\
265	0.011996164084752\\
266	0.0119960482227908\\
267	0.0119959301948122\\
268	0.0119958099601953\\
269	0.011995687477557\\
270	0.0119955627047376\\
271	0.011995435598787\\
272	0.0119953061159494\\
273	0.0119951742116491\\
274	0.0119950398404751\\
275	0.011994902956166\\
276	0.0119947635115943\\
277	0.0119946214587509\\
278	0.0119944767487288\\
279	0.0119943293317075\\
280	0.0119941791569359\\
281	0.0119940261727162\\
282	0.0119938703263869\\
283	0.0119937115643061\\
284	0.011993549831834\\
285	0.0119933850733158\\
286	0.0119932172320643\\
287	0.0119930462503424\\
288	0.0119928720693453\\
289	0.0119926946291831\\
290	0.0119925138688629\\
291	0.0119923297262713\\
292	0.0119921421381568\\
293	0.0119919510401122\\
294	0.0119917563665574\\
295	0.0119915580507223\\
296	0.0119913560246303\\
297	0.0119911502190819\\
298	0.0119909405636389\\
299	0.0119907269866094\\
300	0.0119905094150335\\
301	0.0119902877746699\\
302	0.0119900619899838\\
303	0.0119898319841357\\
304	0.0119895976789724\\
305	0.011989358995019\\
306	0.0119891158514741\\
307	0.011988868166206\\
308	0.0119886158557532\\
309	0.0119883588353271\\
310	0.0119880970188189\\
311	0.0119878303188108\\
312	0.0119875586465922\\
313	0.0119872819121812\\
314	0.011987000024353\\
315	0.0119867128906755\\
316	0.0119864204175532\\
317	0.0119861225102816\\
318	0.011985819073111\\
319	0.0119855100093246\\
320	0.0119851952213295\\
321	0.0119848746107651\\
322	0.0119845480786284\\
323	0.0119842155254224\\
324	0.0119838768513265\\
325	0.0119835319563952\\
326	0.0119831807407872\\
327	0.0119828231050301\\
328	0.0119824589503244\\
329	0.011982088178894\\
330	0.0119817106943872\\
331	0.0119813264023383\\
332	0.0119809352106954\\
333	0.0119805370304261\\
334	0.0119801317762108\\
335	0.0119797193672385\\
336	0.0119792997281199\\
337	0.0119788727899369\\
338	0.0119784384914534\\
339	0.0119779967805106\\
340	0.0119775476156215\\
341	0.0119770909677074\\
342	0.011976626821642\\
343	0.0119761551763533\\
344	0.0119756760341156\\
345	0.0119751893419361\\
346	0.0119747033967705\\
347	0.0119742309999435\\
348	0.0119737492122328\\
349	0.0119732578467016\\
350	0.0119727567127762\\
351	0.011972245616176\\
352	0.0119717243588084\\
353	0.0119711927385808\\
354	0.0119706505491817\\
355	0.0119700975808259\\
356	0.0119695336196882\\
357	0.0119689584477857\\
358	0.0119683718428982\\
359	0.0119677735784865\\
360	0.0119671634236081\\
361	0.0119665411428297\\
362	0.0119659064961368\\
363	0.0119652592388384\\
364	0.0119645991214684\\
365	0.0119639258896817\\
366	0.0119632392841436\\
367	0.0119625390404143\\
368	0.011961824888824\\
369	0.0119610965543401\\
370	0.0119603537564244\\
371	0.0119595962088778\\
372	0.011958823619672\\
373	0.0119580356907652\\
374	0.0119572321178996\\
375	0.0119564125903778\\
376	0.0119555767908153\\
377	0.0119547243948638\\
378	0.0119538550709033\\
379	0.0119529684796948\\
380	0.0119520642739912\\
381	0.0119511420980965\\
382	0.0119502015873672\\
383	0.0119492423676465\\
384	0.0119482640546203\\
385	0.0119472662530831\\
386	0.0119462485561005\\
387	0.0119452105440506\\
388	0.0119441517835291\\
389	0.0119430718261012\\
390	0.0119419702068957\\
391	0.0119408464430847\\
392	0.0119397000324562\\
393	0.0119385304528739\\
394	0.0119373371654809\\
395	0.0119361196318338\\
396	0.0119348773814805\\
397	0.0119336102617314\\
398	0.0119323179017876\\
399	0.011930998433343\\
400	0.0119296512573555\\
401	0.0119282757592654\\
402	0.01192687130843\\
403	0.0119254372575299\\
404	0.0119239729419453\\
405	0.0119224776790995\\
406	0.0119209507677654\\
407	0.0119193914873339\\
408	0.0119177990970507\\
409	0.0119161728352613\\
410	0.0119145119186195\\
411	0.0119128155411731\\
412	0.0119110828734673\\
413	0.0119093130616237\\
414	0.0119075052263901\\
415	0.0119056584621601\\
416	0.0119037718359704\\
417	0.0119018443864826\\
418	0.0118998751229586\\
419	0.0118978630242564\\
420	0.0118958070379309\\
421	0.0118937060795839\\
422	0.0118915590328094\\
423	0.0118893647501689\\
424	0.01188712205556\\
425	0.0118848297432207\\
426	0.0118824865392958\\
427	0.0118800911165304\\
428	0.0118776421479879\\
429	0.0118751382618341\\
430	0.0118725780391294\\
431	0.011869960011477\\
432	0.0118672826585175\\
433	0.0118645444052564\\
434	0.0118617436192115\\
435	0.0118588786073649\\
436	0.0118559476129055\\
437	0.011852948811741\\
438	0.0118498803087644\\
439	0.0118467401338529\\
440	0.0118435262375814\\
441	0.0118402364866329\\
442	0.0118368686588801\\
443	0.0118334204380804\\
444	0.0118298894079996\\
445	0.011826273045521\\
446	0.011822568712481\\
447	0.0118187736511096\\
448	0.0118148849756283\\
449	0.011810899662542\\
450	0.011806814541411\\
451	0.011802626284876\\
452	0.0117983313978688\\
453	0.0117939262059311\\
454	0.01178940684256\\
455	0.0117847692354875\\
456	0.0117800090917911\\
457	0.0117751218817222\\
458	0.011770102821123\\
459	0.0117649468522852\\
460	0.0117596486230801\\
461	0.0117542024641812\\
462	0.0117486023642506\\
463	0.0117428419430494\\
464	0.0117369144216768\\
465	0.0117308125902503\\
466	0.0117245287727546\\
467	0.0117180547892735\\
468	0.0117113819168138\\
469	0.0117045008527483\\
470	0.0116974016927045\\
471	0.0116900739559267\\
472	0.0116825067475515\\
473	0.0116746892931557\\
474	0.0116666121365498\\
475	0.0116582613131673\\
476	0.0116496268573872\\
477	0.0116406685036736\\
478	0.0116313666646186\\
479	0.0116215744592333\\
480	0.0116114135645324\\
481	0.0116008584521903\\
482	0.0115898813035436\\
483	0.0115784517523586\\
484	0.0115664965352428\\
485	0.0115535810913429\\
486	0.0115401743097363\\
487	0.0115262433284871\\
488	0.011511752025207\\
489	0.0114966601315676\\
490	0.0114809211781825\\
491	0.0114649474117142\\
492	0.011448337177884\\
493	0.011430820228943\\
494	0.0114123134926886\\
495	0.0113927201633375\\
496	0.0113719218609005\\
497	0.0113497430388157\\
498	0.0113258959944548\\
499	0.0112842679757464\\
500	0.0112367561145483\\
501	0.0111870588310392\\
502	0.0111120749231243\\
503	0.0109810087170838\\
504	0.0108482609581411\\
505	0.0107143438737388\\
506	0.0106426438237272\\
507	0.0106128813992842\\
508	0.0105881563836952\\
509	0.0105697122651866\\
510	0.0105525874182717\\
511	0.0105358944372637\\
512	0.0105193627982374\\
513	0.0105026074458054\\
514	0.0104855765004467\\
515	0.0104682221468864\\
516	0.010450508956189\\
517	0.0104324220664008\\
518	0.0104139484762821\\
519	0.0103950771022635\\
520	0.0103757983274288\\
521	0.0103561027077202\\
522	0.0103359808848344\\
523	0.0103154234842467\\
524	0.010294420997174\\
525	0.0102729637503076\\
526	0.010251041857233\\
527	0.0102286451118995\\
528	0.0102057627336832\\
529	0.0101823827464436\\
530	0.0101581175006428\\
531	0.0101333585166204\\
532	0.0101072338011138\\
533	0.0100802137631616\\
534	0.0100537070293058\\
535	0.0100292218104736\\
536	0.0100043358418639\\
537	0.00997897559600942\\
538	0.00995309617091416\\
539	0.00992668211141196\\
540	0.0098997212087853\\
541	0.00987220115444026\\
542	0.0098441094345192\\
543	0.00981543320732683\\
544	0.00978615927592536\\
545	0.00975627407587246\\
546	0.00972576367919629\\
547	0.00969461384096915\\
548	0.00966281016553393\\
549	0.00963033861433157\\
550	0.00959718698669629\\
551	0.00956334918709186\\
552	0.00952808643524388\\
553	0.00948996768636199\\
554	0.00945084811108654\\
555	0.00941754878703731\\
556	0.00938368152452256\\
557	0.00934918331733607\\
558	0.00931404097641799\\
559	0.00927824106494997\\
560	0.00924176973245556\\
561	0.00920461251390034\\
562	0.0091667538515289\\
563	0.00910005807242574\\
564	0.00875624526545568\\
565	0.00845178678033124\\
566	0.008391974927306\\
567	0.00833124431092456\\
568	0.00826957769581224\\
569	0.00820695731202763\\
570	0.00814336482900666\\
571	0.0080787813286744\\
572	0.00801318727771716\\
573	0.00794656249918909\\
574	0.00787888614367657\\
575	0.00781013666030771\\
576	0.00774029176794356\\
577	0.0076693284268761\\
578	0.00759722281115766\\
579	0.00752395028094236\\
580	0.00744948535201613\\
581	0.00737380165360669\\
582	0.00729687184929792\\
583	0.00721866745299389\\
584	0.00713915835966272\\
585	0.00705831161801653\\
586	0.006976088211231\\
587	0.00689243463533687\\
588	0.00680726093889688\\
589	0.00672038360986785\\
590	0.00663137734622847\\
591	0.00653960315341497\\
592	0.00644346597070806\\
593	0.00633880868635236\\
594	0.00621472953405217\\
595	0.00604257880714497\\
596	0.00574702137257038\\
597	0.00512995810105609\\
598	0.00367057462141144\\
599	0\\
600	0\\
};
\end{axis}
\end{tikzpicture}% 
  \caption{Discrete Time}
\end{subfigure}\\
\vspace{1cm}
\begin{subfigure}{.45\linewidth}
  \centering
  \setlength\figureheight{\linewidth} 
  \setlength\figurewidth{\linewidth}
  \tikzsetnextfilename{dm_cts_nFPC_z15}
  % This file was created by matlab2tikz.
%
%The latest updates can be retrieved from
%  http://www.mathworks.com/matlabcentral/fileexchange/22022-matlab2tikz-matlab2tikz
%where you can also make suggestions and rate matlab2tikz.
%
\definecolor{mycolor1}{rgb}{1.00000,0.00000,1.00000}%
%
\begin{tikzpicture}

\begin{axis}[%
width=4.564in,
height=3.803in,
at={(1.067in,0.513in)},
scale only axis,
every outer x axis line/.append style={black},
every x tick label/.append style={font=\color{black}},
xmin=0,
xmax=100,
xlabel={Time},
every outer y axis line/.append style={black},
every y tick label/.append style={font=\color{black}},
ymin=0,
ymax=0.01,
ylabel={Depth $\delta$},
axis background/.style={fill=white},
title={Z=15},
axis x line*=bottom,
axis y line*=left,
legend style={legend cell align=left,align=left,draw=black}
]
\addplot [color=green,dashed,forget plot]
  table[row sep=crcr]{%
0.01	0\\
0.02	0\\
0.03	0\\
0.04	0\\
0.05	0\\
0.06	0\\
0.07	0\\
0.08	0\\
0.09	0\\
0.1	0\\
0.11	0\\
0.12	0\\
0.13	0\\
0.14	0\\
0.15	0\\
0.16	0\\
0.17	0\\
0.18	0\\
0.19	0\\
0.2	0\\
0.21	0\\
0.22	0\\
0.23	0\\
0.24	0\\
0.25	0\\
0.26	0\\
0.27	0\\
0.28	0\\
0.29	0\\
0.3	0\\
0.31	0\\
0.32	0\\
0.33	0\\
0.34	0\\
0.35	0\\
0.36	0\\
0.37	0\\
0.38	0\\
0.39	0\\
0.4	0\\
0.41	0\\
0.42	0\\
0.43	0\\
0.44	0\\
0.45	0\\
0.46	0\\
0.47	0\\
0.48	0\\
0.49	0\\
0.5	0\\
0.51	0\\
0.52	0\\
0.53	0\\
0.54	0\\
0.55	0\\
0.56	0\\
0.57	0\\
0.58	0\\
0.59	0\\
0.6	0\\
0.61	0\\
0.62	0\\
0.63	0\\
0.64	0\\
0.65	0\\
0.66	0\\
0.67	0\\
0.68	0\\
0.69	0\\
0.7	0\\
0.71	0\\
0.72	0\\
0.73	0\\
0.74	0\\
0.75	0\\
0.76	0\\
0.77	0\\
0.78	0\\
0.79	0\\
0.8	0\\
0.81	0\\
0.82	0\\
0.83	0\\
0.84	0\\
0.85	0\\
0.86	0\\
0.87	0\\
0.88	0\\
0.89	0\\
0.9	0\\
0.91	0\\
0.92	0\\
0.93	0\\
0.94	0\\
0.95	0\\
0.96	0\\
0.97	0\\
0.98	0\\
0.99	0\\
1	0\\
1.01	0\\
1.02	0\\
1.03	0\\
1.04	0\\
1.05	0\\
1.06	0\\
1.07	0\\
1.08	0\\
1.09	0\\
1.1	0\\
1.11	0\\
1.12	0\\
1.13	0\\
1.14	0\\
1.15	0\\
1.16	0\\
1.17	0\\
1.18	0\\
1.19	0\\
1.2	0\\
1.21	0\\
1.22	0\\
1.23	0\\
1.24	0\\
1.25	0\\
1.26	0\\
1.27	0\\
1.28	0\\
1.29	0\\
1.3	0\\
1.31	0\\
1.32	0\\
1.33	0\\
1.34	0\\
1.35	0\\
1.36	0\\
1.37	0\\
1.38	0\\
1.39	0\\
1.4	0\\
1.41	0\\
1.42	0\\
1.43	0\\
1.44	0\\
1.45	0\\
1.46	0\\
1.47	0\\
1.48	0\\
1.49	0\\
1.5	0\\
1.51	0\\
1.52	0\\
1.53	0\\
1.54	0\\
1.55	0\\
1.56	0\\
1.57	0\\
1.58	0\\
1.59	0\\
1.6	0\\
1.61	0\\
1.62	0\\
1.63	0\\
1.64	0\\
1.65	0\\
1.66	0\\
1.67	0\\
1.68	0\\
1.69	0\\
1.7	0\\
1.71	0\\
1.72	0\\
1.73	0\\
1.74	0\\
1.75	0\\
1.76	0\\
1.77	0\\
1.78	0\\
1.79	0\\
1.8	0\\
1.81	0\\
1.82	0\\
1.83	0\\
1.84	0\\
1.85	0\\
1.86	0\\
1.87	0\\
1.88	0\\
1.89	0\\
1.9	0\\
1.91	0\\
1.92	0\\
1.93	0\\
1.94	0\\
1.95	0\\
1.96	0\\
1.97	0\\
1.98	0\\
1.99	0\\
2	0\\
2.01	0\\
2.02	0\\
2.03	0\\
2.04	0\\
2.05	0\\
2.06	0\\
2.07	0\\
2.08	0\\
2.09	0\\
2.1	0\\
2.11	0\\
2.12	0\\
2.13	0\\
2.14	0\\
2.15	0\\
2.16	0\\
2.17	0\\
2.18	0\\
2.19	0\\
2.2	0\\
2.21	0\\
2.22	0\\
2.23	0\\
2.24	0\\
2.25	0\\
2.26	0\\
2.27	0\\
2.28	0\\
2.29	0\\
2.3	0\\
2.31	0\\
2.32	0\\
2.33	0\\
2.34	0\\
2.35	0\\
2.36	0\\
2.37	0\\
2.38	0\\
2.39	0\\
2.4	0\\
2.41	0\\
2.42	0\\
2.43	0\\
2.44	0\\
2.45	0\\
2.46	0\\
2.47	0\\
2.48	0\\
2.49	0\\
2.5	0\\
2.51	0\\
2.52	0\\
2.53	0\\
2.54	0\\
2.55	0\\
2.56	0\\
2.57	0\\
2.58	0\\
2.59	0\\
2.6	0\\
2.61	0\\
2.62	0\\
2.63	0\\
2.64	0\\
2.65	0\\
2.66	0\\
2.67	0\\
2.68	0\\
2.69	0\\
2.7	0\\
2.71	0\\
2.72	0\\
2.73	0\\
2.74	0\\
2.75	0\\
2.76	0\\
2.77	0\\
2.78	0\\
2.79	0\\
2.8	0\\
2.81	0\\
2.82	0\\
2.83	0\\
2.84	0\\
2.85	0\\
2.86	0\\
2.87	0\\
2.88	0\\
2.89	0\\
2.9	0\\
2.91	0\\
2.92	0\\
2.93	0\\
2.94	0\\
2.95	0\\
2.96	0\\
2.97	0\\
2.98	0\\
2.99	0\\
3	0\\
3.01	0\\
3.02	0\\
3.03	0\\
3.04	0\\
3.05	0\\
3.06	0\\
3.07	0\\
3.08	0\\
3.09	0\\
3.1	0\\
3.11	0\\
3.12	0\\
3.13	0\\
3.14	0\\
3.15	0\\
3.16	0\\
3.17	0\\
3.18	0\\
3.19	0\\
3.2	0\\
3.21	0\\
3.22	0\\
3.23	0\\
3.24	0\\
3.25	0\\
3.26	0\\
3.27	0\\
3.28	0\\
3.29	0\\
3.3	0\\
3.31	0\\
3.32	0\\
3.33	0\\
3.34	0\\
3.35	0\\
3.36	0\\
3.37	0\\
3.38	0\\
3.39	0\\
3.4	0\\
3.41	0\\
3.42	0\\
3.43	0\\
3.44	0\\
3.45	0\\
3.46	0\\
3.47	0\\
3.48	0\\
3.49	0\\
3.5	0\\
3.51	0\\
3.52	0\\
3.53	0\\
3.54	0\\
3.55	0\\
3.56	0\\
3.57	0\\
3.58	0\\
3.59	0\\
3.6	0\\
3.61	0\\
3.62	0\\
3.63	0\\
3.64	0\\
3.65	0\\
3.66	0\\
3.67	0\\
3.68	0\\
3.69	0\\
3.7	0\\
3.71	0\\
3.72	0\\
3.73	0\\
3.74	0\\
3.75	0\\
3.76	0\\
3.77	0\\
3.78	0\\
3.79	0\\
3.8	0\\
3.81	0\\
3.82	0\\
3.83	0\\
3.84	0\\
3.85	0\\
3.86	0\\
3.87	0\\
3.88	0\\
3.89	0\\
3.9	0\\
3.91	0\\
3.92	0\\
3.93	0\\
3.94	0\\
3.95	0\\
3.96	0\\
3.97	0\\
3.98	0\\
3.99	0\\
4	0\\
4.01	0\\
4.02	0\\
4.03	0\\
4.04	0\\
4.05	0\\
4.06	0\\
4.07	0\\
4.08	0\\
4.09	0\\
4.1	0\\
4.11	0\\
4.12	0\\
4.13	0\\
4.14	0\\
4.15	0\\
4.16	0\\
4.17	0\\
4.18	0\\
4.19	0\\
4.2	0\\
4.21	0\\
4.22	0\\
4.23	0\\
4.24	0\\
4.25	0\\
4.26	0\\
4.27	0\\
4.28	0\\
4.29	0\\
4.3	0\\
4.31	0\\
4.32	0\\
4.33	0\\
4.34	0\\
4.35	0\\
4.36	0\\
4.37	0\\
4.38	0\\
4.39	0\\
4.4	0\\
4.41	0\\
4.42	0\\
4.43	0\\
4.44	0\\
4.45	0\\
4.46	0\\
4.47	0\\
4.48	0\\
4.49	0\\
4.5	0\\
4.51	0\\
4.52	0\\
4.53	0\\
4.54	0\\
4.55	0\\
4.56	0\\
4.57	0\\
4.58	0\\
4.59	0\\
4.6	0\\
4.61	0\\
4.62	0\\
4.63	0\\
4.64	0\\
4.65	0\\
4.66	0\\
4.67	0\\
4.68	0\\
4.69	0\\
4.7	0\\
4.71	0\\
4.72	0\\
4.73	0\\
4.74	0\\
4.75	0\\
4.76	0\\
4.77	0\\
4.78	0\\
4.79	0\\
4.8	0\\
4.81	0\\
4.82	0\\
4.83	0\\
4.84	0\\
4.85	0\\
4.86	0\\
4.87	0\\
4.88	0\\
4.89	0\\
4.9	0\\
4.91	0\\
4.92	0\\
4.93	0\\
4.94	0\\
4.95	0\\
4.96	0\\
4.97	0\\
4.98	0\\
4.99	0\\
5	0\\
5.01	0\\
5.02	0\\
5.03	0\\
5.04	0\\
5.05	0\\
5.06	0\\
5.07	0\\
5.08	0\\
5.09	0\\
5.1	0\\
5.11	0\\
5.12	0\\
5.13	0\\
5.14	0\\
5.15	0\\
5.16	0\\
5.17	0\\
5.18	0\\
5.19	0\\
5.2	0\\
5.21	0\\
5.22	0\\
5.23	0\\
5.24	0\\
5.25	0\\
5.26	0\\
5.27	0\\
5.28	0\\
5.29	0\\
5.3	0\\
5.31	0\\
5.32	0\\
5.33	0\\
5.34	0\\
5.35	0\\
5.36	0\\
5.37	0\\
5.38	0\\
5.39	0\\
5.4	0\\
5.41	0\\
5.42	0\\
5.43	0\\
5.44	0\\
5.45	0\\
5.46	0\\
5.47	0\\
5.48	0\\
5.49	0\\
5.5	0\\
5.51	0\\
5.52	0\\
5.53	0\\
5.54	0\\
5.55	0\\
5.56	0\\
5.57	0\\
5.58	0\\
5.59	0\\
5.6	0\\
5.61	0\\
5.62	0\\
5.63	0\\
5.64	0\\
5.65	0\\
5.66	0\\
5.67	0\\
5.68	0\\
5.69	0\\
5.7	0\\
5.71	0\\
5.72	0\\
5.73	0\\
5.74	0\\
5.75	0\\
5.76	0\\
5.77	0\\
5.78	0\\
5.79	0\\
5.8	0\\
5.81	0\\
5.82	0\\
5.83	0\\
5.84	0\\
5.85	0\\
5.86	0\\
5.87	0\\
5.88	0\\
5.89	0\\
5.9	0\\
5.91	0\\
5.92	0\\
5.93	0\\
5.94	0\\
5.95	0\\
5.96	0\\
5.97	0\\
5.98	0\\
5.99	0\\
6	0\\
6.01	0\\
6.02	0\\
6.03	0\\
6.04	0\\
6.05	0\\
6.06	0\\
6.07	0\\
6.08	0\\
6.09	0\\
6.1	0\\
6.11	0\\
6.12	0\\
6.13	0\\
6.14	0\\
6.15	0\\
6.16	0\\
6.17	0\\
6.18	0\\
6.19	0\\
6.2	0\\
6.21	0\\
6.22	0\\
6.23	0\\
6.24	0\\
6.25	0\\
6.26	0\\
6.27	0\\
6.28	0\\
6.29	0\\
6.3	0\\
6.31	0\\
6.32	0\\
6.33	0\\
6.34	0\\
6.35	0\\
6.36	0\\
6.37	0\\
6.38	0\\
6.39	0\\
6.4	0\\
6.41	0\\
6.42	0\\
6.43	0\\
6.44	0\\
6.45	0\\
6.46	0\\
6.47	0\\
6.48	0\\
6.49	0\\
6.5	0\\
6.51	0\\
6.52	0\\
6.53	0\\
6.54	0\\
6.55	0\\
6.56	0\\
6.57	0\\
6.58	0\\
6.59	0\\
6.6	0\\
6.61	0\\
6.62	0\\
6.63	0\\
6.64	0\\
6.65	0\\
6.66	0\\
6.67	0\\
6.68	0\\
6.69	0\\
6.7	0\\
6.71	0\\
6.72	0\\
6.73	0\\
6.74	0\\
6.75	0\\
6.76	0\\
6.77	0\\
6.78	0\\
6.79	0\\
6.8	0\\
6.81	0\\
6.82	0\\
6.83	0\\
6.84	0\\
6.85	0\\
6.86	0\\
6.87	0\\
6.88	0\\
6.89	0\\
6.9	0\\
6.91	0\\
6.92	0\\
6.93	0\\
6.94	0\\
6.95	0\\
6.96	0\\
6.97	0\\
6.98	0\\
6.99	0\\
7	0\\
7.01	0\\
7.02	0\\
7.03	0\\
7.04	0\\
7.05	0\\
7.06	0\\
7.07	0\\
7.08	0\\
7.09	0\\
7.1	0\\
7.11	0\\
7.12	0\\
7.13	0\\
7.14	0\\
7.15	0\\
7.16	0\\
7.17	0\\
7.18	0\\
7.19	0\\
7.2	0\\
7.21	0\\
7.22	0\\
7.23	0\\
7.24	0\\
7.25	0\\
7.26	0\\
7.27	0\\
7.28	0\\
7.29	0\\
7.3	0\\
7.31	0\\
7.32	0\\
7.33	0\\
7.34	0\\
7.35	0\\
7.36	0\\
7.37	0\\
7.38	0\\
7.39	0\\
7.4	0\\
7.41	0\\
7.42	0\\
7.43	0\\
7.44	0\\
7.45	0\\
7.46	0\\
7.47	0\\
7.48	0\\
7.49	0\\
7.5	0\\
7.51	0\\
7.52	0\\
7.53	0\\
7.54	0\\
7.55	0\\
7.56	0\\
7.57	0\\
7.58	0\\
7.59	0\\
7.6	0\\
7.61	0\\
7.62	0\\
7.63	0\\
7.64	0\\
7.65	0\\
7.66	0\\
7.67	0\\
7.68	0\\
7.69	0\\
7.7	0\\
7.71	0\\
7.72	0\\
7.73	0\\
7.74	0\\
7.75	0\\
7.76	0\\
7.77	0\\
7.78	0\\
7.79	0\\
7.8	0\\
7.81	0\\
7.82	0\\
7.83	0\\
7.84	0\\
7.85	0\\
7.86	0\\
7.87	0\\
7.88	0\\
7.89	0\\
7.9	0\\
7.91	0\\
7.92	0\\
7.93	0\\
7.94	0\\
7.95	0\\
7.96	0\\
7.97	0\\
7.98	0\\
7.99	0\\
8	0\\
8.01	0\\
8.02	0\\
8.03	0\\
8.04	0\\
8.05	0\\
8.06	0\\
8.07	0\\
8.08	0\\
8.09	0\\
8.1	0\\
8.11	0\\
8.12	0\\
8.13	0\\
8.14	0\\
8.15	0\\
8.16	0\\
8.17	0\\
8.18	0\\
8.19	0\\
8.2	0\\
8.21	0\\
8.22	0\\
8.23	0\\
8.24	0\\
8.25	0\\
8.26	0\\
8.27	0\\
8.28	0\\
8.29	0\\
8.3	0\\
8.31	0\\
8.32	0\\
8.33	0\\
8.34	0\\
8.35	0\\
8.36	0\\
8.37	0\\
8.38	0\\
8.39	0\\
8.4	0\\
8.41	0\\
8.42	0\\
8.43	0\\
8.44	0\\
8.45	0\\
8.46	0\\
8.47	0\\
8.48	0\\
8.49	0\\
8.5	0\\
8.51	0\\
8.52	0\\
8.53	0\\
8.54	0\\
8.55	0\\
8.56	0\\
8.57	0\\
8.58	0\\
8.59	0\\
8.6	0\\
8.61	0\\
8.62	0\\
8.63	0\\
8.64	0\\
8.65	0\\
8.66	0\\
8.67	0\\
8.68	0\\
8.69	0\\
8.7	0\\
8.71	0\\
8.72	0\\
8.73	0\\
8.74	0\\
8.75	0\\
8.76	0\\
8.77	0\\
8.78	0\\
8.79	0\\
8.8	0\\
8.81	0\\
8.82	0\\
8.83	0\\
8.84	0\\
8.85	0\\
8.86	0\\
8.87	0\\
8.88	0\\
8.89	0\\
8.9	0\\
8.91	0\\
8.92	0\\
8.93	0\\
8.94	0\\
8.95	0\\
8.96	0\\
8.97	0\\
8.98	0\\
8.99	0\\
9	0\\
9.01	0\\
9.02	0\\
9.03	0\\
9.04	0\\
9.05	0\\
9.06	0\\
9.07	0\\
9.08	0\\
9.09	0\\
9.1	0\\
9.11	0\\
9.12	0\\
9.13	0\\
9.14	0\\
9.15	0\\
9.16	0\\
9.17	0\\
9.18	0\\
9.19	0\\
9.2	0\\
9.21	0\\
9.22	0\\
9.23	0\\
9.24	0\\
9.25	0\\
9.26	0\\
9.27	0\\
9.28	0\\
9.29	0\\
9.3	0\\
9.31	0\\
9.32	0\\
9.33	0\\
9.34	0\\
9.35	0\\
9.36	0\\
9.37	0\\
9.38	0\\
9.39	0\\
9.4	0\\
9.41	0\\
9.42	0\\
9.43	0\\
9.44	0\\
9.45	0\\
9.46	0\\
9.47	0\\
9.48	0\\
9.49	0\\
9.5	0\\
9.51	0\\
9.52	0\\
9.53	0\\
9.54	0\\
9.55	0\\
9.56	0\\
9.57	0\\
9.58	0\\
9.59	0\\
9.6	0\\
9.61	0\\
9.62	0\\
9.63	0\\
9.64	0\\
9.65	0\\
9.66	0\\
9.67	0\\
9.68	0\\
9.69	0\\
9.7	0\\
9.71	0\\
9.72	0\\
9.73	0\\
9.74	0\\
9.75	0\\
9.76	0\\
9.77	0\\
9.78	0\\
9.79	0\\
9.8	0\\
9.81	0\\
9.82	0\\
9.83	0\\
9.84	0\\
9.85	0\\
9.86	0\\
9.87	0\\
9.88	0\\
9.89	0\\
9.9	0\\
9.91	0\\
9.92	0\\
9.93	0\\
9.94	0\\
9.95	0\\
9.96	0\\
9.97	0\\
9.98	0\\
9.99	0\\
10	0\\
10.01	0\\
10.02	0\\
10.03	0\\
10.04	0\\
10.05	0\\
10.06	0\\
10.07	0\\
10.08	0\\
10.09	0\\
10.1	0\\
10.11	0\\
10.12	0\\
10.13	0\\
10.14	0\\
10.15	0\\
10.16	0\\
10.17	0\\
10.18	0\\
10.19	0\\
10.2	0\\
10.21	0\\
10.22	0\\
10.23	0\\
10.24	0\\
10.25	0\\
10.26	0\\
10.27	0\\
10.28	0\\
10.29	0\\
10.3	0\\
10.31	0\\
10.32	0\\
10.33	0\\
10.34	0\\
10.35	0\\
10.36	0\\
10.37	0\\
10.38	0\\
10.39	0\\
10.4	0\\
10.41	0\\
10.42	0\\
10.43	0\\
10.44	0\\
10.45	0\\
10.46	0\\
10.47	0\\
10.48	0\\
10.49	0\\
10.5	0\\
10.51	0\\
10.52	0\\
10.53	0\\
10.54	0\\
10.55	0\\
10.56	0\\
10.57	0\\
10.58	0\\
10.59	0\\
10.6	0\\
10.61	0\\
10.62	0\\
10.63	0\\
10.64	0\\
10.65	0\\
10.66	0\\
10.67	0\\
10.68	0\\
10.69	0\\
10.7	0\\
10.71	0\\
10.72	0\\
10.73	0\\
10.74	0\\
10.75	0\\
10.76	0\\
10.77	0\\
10.78	0\\
10.79	0\\
10.8	0\\
10.81	0\\
10.82	0\\
10.83	0\\
10.84	0\\
10.85	0\\
10.86	0\\
10.87	0\\
10.88	0\\
10.89	0\\
10.9	0\\
10.91	0\\
10.92	0\\
10.93	0\\
10.94	0\\
10.95	0\\
10.96	0\\
10.97	0\\
10.98	0\\
10.99	0\\
11	0\\
11.01	0\\
11.02	0\\
11.03	0\\
11.04	0\\
11.05	0\\
11.06	0\\
11.07	0\\
11.08	0\\
11.09	0\\
11.1	0\\
11.11	0\\
11.12	0\\
11.13	0\\
11.14	0\\
11.15	0\\
11.16	0\\
11.17	0\\
11.18	0\\
11.19	0\\
11.2	0\\
11.21	0\\
11.22	0\\
11.23	0\\
11.24	0\\
11.25	0\\
11.26	0\\
11.27	0\\
11.28	0\\
11.29	0\\
11.3	0\\
11.31	0\\
11.32	0\\
11.33	0\\
11.34	0\\
11.35	0\\
11.36	0\\
11.37	0\\
11.38	0\\
11.39	0\\
11.4	0\\
11.41	0\\
11.42	0\\
11.43	0\\
11.44	0\\
11.45	0\\
11.46	0\\
11.47	0\\
11.48	0\\
11.49	0\\
11.5	0\\
11.51	0\\
11.52	0\\
11.53	0\\
11.54	0\\
11.55	0\\
11.56	0\\
11.57	0\\
11.58	0\\
11.59	0\\
11.6	0\\
11.61	0\\
11.62	0\\
11.63	0\\
11.64	0\\
11.65	0\\
11.66	0\\
11.67	0\\
11.68	0\\
11.69	0\\
11.7	0\\
11.71	0\\
11.72	0\\
11.73	0\\
11.74	0\\
11.75	0\\
11.76	0\\
11.77	0\\
11.78	0\\
11.79	0\\
11.8	0\\
11.81	0\\
11.82	0\\
11.83	0\\
11.84	0\\
11.85	0\\
11.86	0\\
11.87	0\\
11.88	0\\
11.89	0\\
11.9	0\\
11.91	0\\
11.92	0\\
11.93	0\\
11.94	0\\
11.95	0\\
11.96	0\\
11.97	0\\
11.98	0\\
11.99	0\\
12	0\\
12.01	0\\
12.02	0\\
12.03	0\\
12.04	0\\
12.05	0\\
12.06	0\\
12.07	0\\
12.08	0\\
12.09	0\\
12.1	0\\
12.11	0\\
12.12	0\\
12.13	0\\
12.14	0\\
12.15	0\\
12.16	0\\
12.17	0\\
12.18	0\\
12.19	0\\
12.2	0\\
12.21	0\\
12.22	0\\
12.23	0\\
12.24	0\\
12.25	0\\
12.26	0\\
12.27	0\\
12.28	0\\
12.29	0\\
12.3	0\\
12.31	0\\
12.32	0\\
12.33	0\\
12.34	0\\
12.35	0\\
12.36	0\\
12.37	0\\
12.38	0\\
12.39	0\\
12.4	0\\
12.41	0\\
12.42	0\\
12.43	0\\
12.44	0\\
12.45	0\\
12.46	0\\
12.47	0\\
12.48	0\\
12.49	0\\
12.5	0\\
12.51	0\\
12.52	0\\
12.53	0\\
12.54	0\\
12.55	0\\
12.56	0\\
12.57	0\\
12.58	0\\
12.59	0\\
12.6	0\\
12.61	0\\
12.62	0\\
12.63	0\\
12.64	0\\
12.65	0\\
12.66	0\\
12.67	0\\
12.68	0\\
12.69	0\\
12.7	0\\
12.71	0\\
12.72	0\\
12.73	0\\
12.74	0\\
12.75	0\\
12.76	0\\
12.77	0\\
12.78	0\\
12.79	0\\
12.8	0\\
12.81	0\\
12.82	0\\
12.83	0\\
12.84	0\\
12.85	0\\
12.86	0\\
12.87	0\\
12.88	0\\
12.89	0\\
12.9	0\\
12.91	0\\
12.92	0\\
12.93	0\\
12.94	0\\
12.95	0\\
12.96	0\\
12.97	0\\
12.98	0\\
12.99	0\\
13	0\\
13.01	0\\
13.02	0\\
13.03	0\\
13.04	0\\
13.05	0\\
13.06	0\\
13.07	0\\
13.08	0\\
13.09	0\\
13.1	0\\
13.11	0\\
13.12	0\\
13.13	0\\
13.14	0\\
13.15	0\\
13.16	0\\
13.17	0\\
13.18	0\\
13.19	0\\
13.2	0\\
13.21	0\\
13.22	0\\
13.23	0\\
13.24	0\\
13.25	0\\
13.26	0\\
13.27	0\\
13.28	0\\
13.29	0\\
13.3	0\\
13.31	0\\
13.32	0\\
13.33	0\\
13.34	0\\
13.35	0\\
13.36	0\\
13.37	0\\
13.38	0\\
13.39	0\\
13.4	0\\
13.41	0\\
13.42	0\\
13.43	0\\
13.44	0\\
13.45	0\\
13.46	0\\
13.47	0\\
13.48	0\\
13.49	0\\
13.5	0\\
13.51	0\\
13.52	0\\
13.53	0\\
13.54	0\\
13.55	0\\
13.56	0\\
13.57	0\\
13.58	0\\
13.59	0\\
13.6	0\\
13.61	0\\
13.62	0\\
13.63	0\\
13.64	0\\
13.65	0\\
13.66	0\\
13.67	0\\
13.68	0\\
13.69	0\\
13.7	0\\
13.71	0\\
13.72	0\\
13.73	0\\
13.74	0\\
13.75	0\\
13.76	0\\
13.77	0\\
13.78	0\\
13.79	0\\
13.8	0\\
13.81	0\\
13.82	0\\
13.83	0\\
13.84	0\\
13.85	0\\
13.86	0\\
13.87	0\\
13.88	0\\
13.89	0\\
13.9	0\\
13.91	0\\
13.92	0\\
13.93	0\\
13.94	0\\
13.95	0\\
13.96	0\\
13.97	0\\
13.98	0\\
13.99	0\\
14	0\\
14.01	0\\
14.02	0\\
14.03	0\\
14.04	0\\
14.05	0\\
14.06	0\\
14.07	0\\
14.08	0\\
14.09	0\\
14.1	0\\
14.11	0\\
14.12	0\\
14.13	0\\
14.14	0\\
14.15	0\\
14.16	0\\
14.17	0\\
14.18	0\\
14.19	0\\
14.2	0\\
14.21	0\\
14.22	0\\
14.23	0\\
14.24	0\\
14.25	0\\
14.26	0\\
14.27	0\\
14.28	0\\
14.29	0\\
14.3	0\\
14.31	0\\
14.32	0\\
14.33	0\\
14.34	0\\
14.35	0\\
14.36	0\\
14.37	0\\
14.38	0\\
14.39	0\\
14.4	0\\
14.41	0\\
14.42	0\\
14.43	0\\
14.44	0\\
14.45	0\\
14.46	0\\
14.47	0\\
14.48	0\\
14.49	0\\
14.5	0\\
14.51	0\\
14.52	0\\
14.53	0\\
14.54	0\\
14.55	0\\
14.56	0\\
14.57	0\\
14.58	0\\
14.59	0\\
14.6	0\\
14.61	0\\
14.62	0\\
14.63	0\\
14.64	0\\
14.65	0\\
14.66	0\\
14.67	0\\
14.68	0\\
14.69	0\\
14.7	0\\
14.71	0\\
14.72	0\\
14.73	0\\
14.74	0\\
14.75	0\\
14.76	0\\
14.77	0\\
14.78	0\\
14.79	0\\
14.8	0\\
14.81	0\\
14.82	0\\
14.83	0\\
14.84	0\\
14.85	0\\
14.86	0\\
14.87	0\\
14.88	0\\
14.89	0\\
14.9	0\\
14.91	0\\
14.92	0\\
14.93	0\\
14.94	0\\
14.95	0\\
14.96	0\\
14.97	0\\
14.98	0\\
14.99	0\\
15	0\\
15.01	0\\
15.02	0\\
15.03	0\\
15.04	0\\
15.05	0\\
15.06	0\\
15.07	0\\
15.08	0\\
15.09	0\\
15.1	0\\
15.11	0\\
15.12	0\\
15.13	0\\
15.14	0\\
15.15	0\\
15.16	0\\
15.17	0\\
15.18	0\\
15.19	0\\
15.2	0\\
15.21	0\\
15.22	0\\
15.23	0\\
15.24	0\\
15.25	0\\
15.26	0\\
15.27	0\\
15.28	0\\
15.29	0\\
15.3	0\\
15.31	0\\
15.32	0\\
15.33	0\\
15.34	0\\
15.35	0\\
15.36	0\\
15.37	0\\
15.38	0\\
15.39	0\\
15.4	0\\
15.41	0\\
15.42	0\\
15.43	0\\
15.44	0\\
15.45	0\\
15.46	0\\
15.47	0\\
15.48	0\\
15.49	0\\
15.5	0\\
15.51	0\\
15.52	0\\
15.53	0\\
15.54	0\\
15.55	0\\
15.56	0\\
15.57	0\\
15.58	0\\
15.59	0\\
15.6	0\\
15.61	0\\
15.62	0\\
15.63	0\\
15.64	0\\
15.65	0\\
15.66	0\\
15.67	0\\
15.68	0\\
15.69	0\\
15.7	0\\
15.71	0\\
15.72	0\\
15.73	0\\
15.74	0\\
15.75	0\\
15.76	0\\
15.77	0\\
15.78	0\\
15.79	0\\
15.8	0\\
15.81	0\\
15.82	0\\
15.83	0\\
15.84	0\\
15.85	0\\
15.86	0\\
15.87	0\\
15.88	0\\
15.89	0\\
15.9	0\\
15.91	0\\
15.92	0\\
15.93	0\\
15.94	0\\
15.95	0\\
15.96	0\\
15.97	0\\
15.98	0\\
15.99	0\\
16	0\\
16.01	0\\
16.02	0\\
16.03	0\\
16.04	0\\
16.05	0\\
16.06	0\\
16.07	0\\
16.08	0\\
16.09	0\\
16.1	0\\
16.11	0\\
16.12	0\\
16.13	0\\
16.14	0\\
16.15	0\\
16.16	0\\
16.17	0\\
16.18	0\\
16.19	0\\
16.2	0\\
16.21	0\\
16.22	0\\
16.23	0\\
16.24	0\\
16.25	0\\
16.26	0\\
16.27	0\\
16.28	0\\
16.29	0\\
16.3	0\\
16.31	0\\
16.32	0\\
16.33	0\\
16.34	0\\
16.35	0\\
16.36	0\\
16.37	0\\
16.38	0\\
16.39	0\\
16.4	0\\
16.41	0\\
16.42	0\\
16.43	0\\
16.44	0\\
16.45	0\\
16.46	0\\
16.47	0\\
16.48	0\\
16.49	0\\
16.5	0\\
16.51	0\\
16.52	0\\
16.53	0\\
16.54	0\\
16.55	0\\
16.56	0\\
16.57	0\\
16.58	0\\
16.59	0\\
16.6	0\\
16.61	0\\
16.62	0\\
16.63	0\\
16.64	0\\
16.65	0\\
16.66	0\\
16.67	0\\
16.68	0\\
16.69	0\\
16.7	0\\
16.71	0\\
16.72	0\\
16.73	0\\
16.74	0\\
16.75	0\\
16.76	0\\
16.77	0\\
16.78	0\\
16.79	0\\
16.8	0\\
16.81	0\\
16.82	0\\
16.83	0\\
16.84	0\\
16.85	0\\
16.86	0\\
16.87	0\\
16.88	0\\
16.89	0\\
16.9	0\\
16.91	0\\
16.92	0\\
16.93	0\\
16.94	0\\
16.95	0\\
16.96	0\\
16.97	0\\
16.98	0\\
16.99	0\\
17	0\\
17.01	0\\
17.02	0\\
17.03	0\\
17.04	0\\
17.05	0\\
17.06	0\\
17.07	0\\
17.08	0\\
17.09	0\\
17.1	0\\
17.11	0\\
17.12	0\\
17.13	0\\
17.14	0\\
17.15	0\\
17.16	0\\
17.17	0\\
17.18	0\\
17.19	0\\
17.2	0\\
17.21	0\\
17.22	0\\
17.23	0\\
17.24	0\\
17.25	0\\
17.26	0\\
17.27	0\\
17.28	0\\
17.29	0\\
17.3	0\\
17.31	0\\
17.32	0\\
17.33	0\\
17.34	0\\
17.35	0\\
17.36	0\\
17.37	0\\
17.38	0\\
17.39	0\\
17.4	0\\
17.41	0\\
17.42	0\\
17.43	0\\
17.44	0\\
17.45	0\\
17.46	0\\
17.47	0\\
17.48	0\\
17.49	0\\
17.5	0\\
17.51	0\\
17.52	0\\
17.53	0\\
17.54	0\\
17.55	0\\
17.56	0\\
17.57	0\\
17.58	0\\
17.59	0\\
17.6	0\\
17.61	0\\
17.62	0\\
17.63	0\\
17.64	0\\
17.65	0\\
17.66	0\\
17.67	0\\
17.68	0\\
17.69	0\\
17.7	0\\
17.71	0\\
17.72	0\\
17.73	0\\
17.74	0\\
17.75	0\\
17.76	0\\
17.77	0\\
17.78	0\\
17.79	0\\
17.8	0\\
17.81	0\\
17.82	0\\
17.83	0\\
17.84	0\\
17.85	0\\
17.86	0\\
17.87	0\\
17.88	0\\
17.89	0\\
17.9	0\\
17.91	0\\
17.92	0\\
17.93	0\\
17.94	0\\
17.95	0\\
17.96	0\\
17.97	0\\
17.98	0\\
17.99	0\\
18	0\\
18.01	0\\
18.02	0\\
18.03	0\\
18.04	0\\
18.05	0\\
18.06	0\\
18.07	0\\
18.08	0\\
18.09	0\\
18.1	0\\
18.11	0\\
18.12	0\\
18.13	0\\
18.14	0\\
18.15	0\\
18.16	0\\
18.17	0\\
18.18	0\\
18.19	0\\
18.2	0\\
18.21	0\\
18.22	0\\
18.23	0\\
18.24	0\\
18.25	0\\
18.26	0\\
18.27	0\\
18.28	0\\
18.29	0\\
18.3	0\\
18.31	0\\
18.32	0\\
18.33	0\\
18.34	0\\
18.35	0\\
18.36	0\\
18.37	0\\
18.38	0\\
18.39	0\\
18.4	0\\
18.41	0\\
18.42	0\\
18.43	0\\
18.44	0\\
18.45	0\\
18.46	0\\
18.47	0\\
18.48	0\\
18.49	0\\
18.5	0\\
18.51	0\\
18.52	0\\
18.53	0\\
18.54	0\\
18.55	0\\
18.56	0\\
18.57	0\\
18.58	0\\
18.59	0\\
18.6	0\\
18.61	0\\
18.62	0\\
18.63	0\\
18.64	0\\
18.65	0\\
18.66	0\\
18.67	0\\
18.68	0\\
18.69	0\\
18.7	0\\
18.71	0\\
18.72	0\\
18.73	0\\
18.74	0\\
18.75	0\\
18.76	0\\
18.77	0\\
18.78	0\\
18.79	0\\
18.8	0\\
18.81	0\\
18.82	0\\
18.83	0\\
18.84	0\\
18.85	0\\
18.86	0\\
18.87	0\\
18.88	0\\
18.89	0\\
18.9	0\\
18.91	0\\
18.92	0\\
18.93	0\\
18.94	0\\
18.95	0\\
18.96	0\\
18.97	0\\
18.98	0\\
18.99	0\\
19	0\\
19.01	0\\
19.02	0\\
19.03	0\\
19.04	0\\
19.05	0\\
19.06	0\\
19.07	0\\
19.08	0\\
19.09	0\\
19.1	0\\
19.11	0\\
19.12	0\\
19.13	0\\
19.14	0\\
19.15	0\\
19.16	0\\
19.17	0\\
19.18	0\\
19.19	0\\
19.2	0\\
19.21	0\\
19.22	0\\
19.23	0\\
19.24	0\\
19.25	0\\
19.26	0\\
19.27	0\\
19.28	0\\
19.29	0\\
19.3	0\\
19.31	0\\
19.32	0\\
19.33	0\\
19.34	0\\
19.35	0\\
19.36	0\\
19.37	0\\
19.38	0\\
19.39	0\\
19.4	0\\
19.41	0\\
19.42	0\\
19.43	0\\
19.44	0\\
19.45	0\\
19.46	0\\
19.47	0\\
19.48	0\\
19.49	0\\
19.5	0\\
19.51	0\\
19.52	0\\
19.53	0\\
19.54	0\\
19.55	0\\
19.56	0\\
19.57	0\\
19.58	0\\
19.59	0\\
19.6	0\\
19.61	0\\
19.62	0\\
19.63	0\\
19.64	0\\
19.65	0\\
19.66	0\\
19.67	0\\
19.68	0\\
19.69	0\\
19.7	0\\
19.71	0\\
19.72	0\\
19.73	0\\
19.74	0\\
19.75	0\\
19.76	0\\
19.77	0\\
19.78	0\\
19.79	0\\
19.8	0\\
19.81	0\\
19.82	0\\
19.83	0\\
19.84	0\\
19.85	0\\
19.86	0\\
19.87	0\\
19.88	0\\
19.89	0\\
19.9	0\\
19.91	0\\
19.92	0\\
19.93	0\\
19.94	0\\
19.95	0\\
19.96	0\\
19.97	0\\
19.98	0\\
19.99	0\\
20	0\\
20.01	0\\
20.02	0\\
20.03	0\\
20.04	0\\
20.05	0\\
20.06	0\\
20.07	0\\
20.08	0\\
20.09	0\\
20.1	0\\
20.11	0\\
20.12	0\\
20.13	0\\
20.14	0\\
20.15	0\\
20.16	0\\
20.17	0\\
20.18	0\\
20.19	0\\
20.2	0\\
20.21	0\\
20.22	0\\
20.23	0\\
20.24	0\\
20.25	0\\
20.26	0\\
20.27	0\\
20.28	0\\
20.29	0\\
20.3	0\\
20.31	0\\
20.32	0\\
20.33	0\\
20.34	0\\
20.35	0\\
20.36	0\\
20.37	0\\
20.38	0\\
20.39	0\\
20.4	0\\
20.41	0\\
20.42	0\\
20.43	0\\
20.44	0\\
20.45	0\\
20.46	0\\
20.47	0\\
20.48	0\\
20.49	0\\
20.5	0\\
20.51	0\\
20.52	0\\
20.53	0\\
20.54	0\\
20.55	0\\
20.56	0\\
20.57	0\\
20.58	0\\
20.59	0\\
20.6	0\\
20.61	0\\
20.62	0\\
20.63	0\\
20.64	0\\
20.65	0\\
20.66	0\\
20.67	0\\
20.68	0\\
20.69	0\\
20.7	0\\
20.71	0\\
20.72	0\\
20.73	0\\
20.74	0\\
20.75	0\\
20.76	0\\
20.77	0\\
20.78	0\\
20.79	0\\
20.8	0\\
20.81	0\\
20.82	0\\
20.83	0\\
20.84	0\\
20.85	0\\
20.86	0\\
20.87	0\\
20.88	0\\
20.89	0\\
20.9	0\\
20.91	0\\
20.92	0\\
20.93	0\\
20.94	0\\
20.95	0\\
20.96	0\\
20.97	0\\
20.98	0\\
20.99	0\\
21	0\\
21.01	0\\
21.02	0\\
21.03	0\\
21.04	0\\
21.05	0\\
21.06	0\\
21.07	0\\
21.08	0\\
21.09	0\\
21.1	0\\
21.11	0\\
21.12	0\\
21.13	0\\
21.14	0\\
21.15	0\\
21.16	0\\
21.17	0\\
21.18	0\\
21.19	0\\
21.2	0\\
21.21	0\\
21.22	0\\
21.23	0\\
21.24	0\\
21.25	0\\
21.26	0\\
21.27	0\\
21.28	0\\
21.29	0\\
21.3	0\\
21.31	0\\
21.32	0\\
21.33	0\\
21.34	0\\
21.35	0\\
21.36	0\\
21.37	0\\
21.38	0\\
21.39	0\\
21.4	0\\
21.41	0\\
21.42	0\\
21.43	0\\
21.44	0\\
21.45	0\\
21.46	0\\
21.47	0\\
21.48	0\\
21.49	0\\
21.5	0\\
21.51	0\\
21.52	0\\
21.53	0\\
21.54	0\\
21.55	0\\
21.56	0\\
21.57	0\\
21.58	0\\
21.59	0\\
21.6	0\\
21.61	0\\
21.62	0\\
21.63	0\\
21.64	0\\
21.65	0\\
21.66	0\\
21.67	0\\
21.68	0\\
21.69	0\\
21.7	0\\
21.71	0\\
21.72	0\\
21.73	0\\
21.74	0\\
21.75	0\\
21.76	0\\
21.77	0\\
21.78	0\\
21.79	0\\
21.8	0\\
21.81	0\\
21.82	0\\
21.83	0\\
21.84	0\\
21.85	0\\
21.86	0\\
21.87	0\\
21.88	0\\
21.89	0\\
21.9	0\\
21.91	0\\
21.92	0\\
21.93	0\\
21.94	0\\
21.95	0\\
21.96	0\\
21.97	0\\
21.98	0\\
21.99	0\\
22	0\\
22.01	0\\
22.02	0\\
22.03	0\\
22.04	0\\
22.05	0\\
22.06	0\\
22.07	0\\
22.08	0\\
22.09	0\\
22.1	0\\
22.11	0\\
22.12	0\\
22.13	0\\
22.14	0\\
22.15	0\\
22.16	0\\
22.17	0\\
22.18	0\\
22.19	0\\
22.2	0\\
22.21	0\\
22.22	0\\
22.23	0\\
22.24	0\\
22.25	0\\
22.26	0\\
22.27	0\\
22.28	0\\
22.29	0\\
22.3	0\\
22.31	0\\
22.32	0\\
22.33	0\\
22.34	0\\
22.35	0\\
22.36	0\\
22.37	0\\
22.38	0\\
22.39	0\\
22.4	0\\
22.41	0\\
22.42	0\\
22.43	0\\
22.44	0\\
22.45	0\\
22.46	0\\
22.47	0\\
22.48	0\\
22.49	0\\
22.5	0\\
22.51	0\\
22.52	0\\
22.53	0\\
22.54	0\\
22.55	0\\
22.56	0\\
22.57	0\\
22.58	0\\
22.59	0\\
22.6	0\\
22.61	0\\
22.62	0\\
22.63	0\\
22.64	0\\
22.65	0\\
22.66	0\\
22.67	0\\
22.68	0\\
22.69	0\\
22.7	0\\
22.71	0\\
22.72	0\\
22.73	0\\
22.74	0\\
22.75	0\\
22.76	0\\
22.77	0\\
22.78	0\\
22.79	0\\
22.8	0\\
22.81	0\\
22.82	0\\
22.83	0\\
22.84	0\\
22.85	0\\
22.86	0\\
22.87	0\\
22.88	0\\
22.89	0\\
22.9	0\\
22.91	0\\
22.92	0\\
22.93	0\\
22.94	0\\
22.95	0\\
22.96	0\\
22.97	0\\
22.98	0\\
22.99	0\\
23	0\\
23.01	0\\
23.02	0\\
23.03	0\\
23.04	0\\
23.05	0\\
23.06	0\\
23.07	0\\
23.08	0\\
23.09	0\\
23.1	0\\
23.11	0\\
23.12	0\\
23.13	0\\
23.14	0\\
23.15	0\\
23.16	0\\
23.17	0\\
23.18	0\\
23.19	0\\
23.2	0\\
23.21	0\\
23.22	0\\
23.23	0\\
23.24	0\\
23.25	0\\
23.26	0\\
23.27	0\\
23.28	0\\
23.29	0\\
23.3	0\\
23.31	0\\
23.32	0\\
23.33	0\\
23.34	0\\
23.35	0\\
23.36	0\\
23.37	0\\
23.38	0\\
23.39	0\\
23.4	0\\
23.41	0\\
23.42	0\\
23.43	0\\
23.44	0\\
23.45	0\\
23.46	0\\
23.47	0\\
23.48	0\\
23.49	0\\
23.5	0\\
23.51	0\\
23.52	0\\
23.53	0\\
23.54	0\\
23.55	0\\
23.56	0\\
23.57	0\\
23.58	0\\
23.59	0\\
23.6	0\\
23.61	0\\
23.62	0\\
23.63	0\\
23.64	0\\
23.65	0\\
23.66	0\\
23.67	0\\
23.68	0\\
23.69	0\\
23.7	0\\
23.71	0\\
23.72	0\\
23.73	0\\
23.74	0\\
23.75	0\\
23.76	0\\
23.77	0\\
23.78	0\\
23.79	0\\
23.8	0\\
23.81	0\\
23.82	0\\
23.83	0\\
23.84	0\\
23.85	0\\
23.86	0\\
23.87	0\\
23.88	0\\
23.89	0\\
23.9	0\\
23.91	0\\
23.92	0\\
23.93	0\\
23.94	0\\
23.95	0\\
23.96	0\\
23.97	0\\
23.98	0\\
23.99	0\\
24	0\\
24.01	0\\
24.02	0\\
24.03	0\\
24.04	0\\
24.05	0\\
24.06	0\\
24.07	0\\
24.08	0\\
24.09	0\\
24.1	0\\
24.11	0\\
24.12	0\\
24.13	0\\
24.14	0\\
24.15	0\\
24.16	0\\
24.17	0\\
24.18	0\\
24.19	0\\
24.2	0\\
24.21	0\\
24.22	0\\
24.23	0\\
24.24	0\\
24.25	0\\
24.26	0\\
24.27	0\\
24.28	0\\
24.29	0\\
24.3	0\\
24.31	0\\
24.32	0\\
24.33	0\\
24.34	0\\
24.35	0\\
24.36	0\\
24.37	0\\
24.38	0\\
24.39	0\\
24.4	0\\
24.41	0\\
24.42	0\\
24.43	0\\
24.44	0\\
24.45	0\\
24.46	0\\
24.47	0\\
24.48	0\\
24.49	0\\
24.5	0\\
24.51	0\\
24.52	0\\
24.53	0\\
24.54	0\\
24.55	0\\
24.56	0\\
24.57	0\\
24.58	0\\
24.59	0\\
24.6	0\\
24.61	0\\
24.62	0\\
24.63	0\\
24.64	0\\
24.65	0\\
24.66	0\\
24.67	0\\
24.68	0\\
24.69	0\\
24.7	0\\
24.71	0\\
24.72	0\\
24.73	0\\
24.74	0\\
24.75	0\\
24.76	0\\
24.77	0\\
24.78	0\\
24.79	0\\
24.8	0\\
24.81	0\\
24.82	0\\
24.83	0\\
24.84	0\\
24.85	0\\
24.86	0\\
24.87	0\\
24.88	0\\
24.89	0\\
24.9	0\\
24.91	0\\
24.92	0\\
24.93	0\\
24.94	0\\
24.95	0\\
24.96	0\\
24.97	0\\
24.98	0\\
24.99	0\\
25	0\\
25.01	0\\
25.02	0\\
25.03	0\\
25.04	0\\
25.05	0\\
25.06	0\\
25.07	0\\
25.08	0\\
25.09	0\\
25.1	0\\
25.11	0\\
25.12	0\\
25.13	0\\
25.14	0\\
25.15	0\\
25.16	0\\
25.17	0\\
25.18	0\\
25.19	0\\
25.2	0\\
25.21	0\\
25.22	0\\
25.23	0\\
25.24	0\\
25.25	0\\
25.26	0\\
25.27	0\\
25.28	0\\
25.29	0\\
25.3	0\\
25.31	0\\
25.32	0\\
25.33	0\\
25.34	0\\
25.35	0\\
25.36	0\\
25.37	0\\
25.38	0\\
25.39	0\\
25.4	0\\
25.41	0\\
25.42	0\\
25.43	0\\
25.44	0\\
25.45	0\\
25.46	0\\
25.47	0\\
25.48	0\\
25.49	0\\
25.5	0\\
25.51	0\\
25.52	0\\
25.53	0\\
25.54	0\\
25.55	0\\
25.56	0\\
25.57	0\\
25.58	0\\
25.59	0\\
25.6	0\\
25.61	0\\
25.62	0\\
25.63	0\\
25.64	0\\
25.65	0\\
25.66	0\\
25.67	0\\
25.68	0\\
25.69	0\\
25.7	0\\
25.71	0\\
25.72	0\\
25.73	0\\
25.74	0\\
25.75	0\\
25.76	0\\
25.77	0\\
25.78	0\\
25.79	0\\
25.8	0\\
25.81	0\\
25.82	0\\
25.83	0\\
25.84	0\\
25.85	0\\
25.86	0\\
25.87	0\\
25.88	0\\
25.89	0\\
25.9	0\\
25.91	0\\
25.92	0\\
25.93	0\\
25.94	0\\
25.95	0\\
25.96	0\\
25.97	0\\
25.98	0\\
25.99	0\\
26	0\\
26.01	0\\
26.02	0\\
26.03	0\\
26.04	0\\
26.05	0\\
26.06	0\\
26.07	0\\
26.08	0\\
26.09	0\\
26.1	0\\
26.11	0\\
26.12	0\\
26.13	0\\
26.14	0\\
26.15	0\\
26.16	0\\
26.17	0\\
26.18	0\\
26.19	0\\
26.2	0\\
26.21	0\\
26.22	0\\
26.23	0\\
26.24	0\\
26.25	0\\
26.26	0\\
26.27	0\\
26.28	0\\
26.29	0\\
26.3	0\\
26.31	0\\
26.32	0\\
26.33	0\\
26.34	0\\
26.35	0\\
26.36	0\\
26.37	0\\
26.38	0\\
26.39	0\\
26.4	0\\
26.41	0\\
26.42	0\\
26.43	0\\
26.44	0\\
26.45	0\\
26.46	0\\
26.47	0\\
26.48	0\\
26.49	0\\
26.5	0\\
26.51	0\\
26.52	0\\
26.53	0\\
26.54	0\\
26.55	0\\
26.56	0\\
26.57	0\\
26.58	0\\
26.59	0\\
26.6	0\\
26.61	0\\
26.62	0\\
26.63	0\\
26.64	0\\
26.65	0\\
26.66	0\\
26.67	0\\
26.68	0\\
26.69	0\\
26.7	0\\
26.71	0\\
26.72	0\\
26.73	0\\
26.74	0\\
26.75	0\\
26.76	0\\
26.77	0\\
26.78	0\\
26.79	0\\
26.8	0\\
26.81	0\\
26.82	0\\
26.83	0\\
26.84	0\\
26.85	0\\
26.86	0\\
26.87	0\\
26.88	0\\
26.89	0\\
26.9	0\\
26.91	0\\
26.92	0\\
26.93	0\\
26.94	0\\
26.95	0\\
26.96	0\\
26.97	0\\
26.98	0\\
26.99	0\\
27	0\\
27.01	0\\
27.02	0\\
27.03	0\\
27.04	0\\
27.05	0\\
27.06	0\\
27.07	0\\
27.08	0\\
27.09	0\\
27.1	0\\
27.11	0\\
27.12	0\\
27.13	0\\
27.14	0\\
27.15	0\\
27.16	0\\
27.17	0\\
27.18	0\\
27.19	0\\
27.2	0\\
27.21	0\\
27.22	0\\
27.23	0\\
27.24	0\\
27.25	0\\
27.26	0\\
27.27	0\\
27.28	0\\
27.29	0\\
27.3	0\\
27.31	0\\
27.32	0\\
27.33	0\\
27.34	0\\
27.35	0\\
27.36	0\\
27.37	0\\
27.38	0\\
27.39	0\\
27.4	0\\
27.41	0\\
27.42	0\\
27.43	0\\
27.44	0\\
27.45	0\\
27.46	0\\
27.47	0\\
27.48	0\\
27.49	0\\
27.5	0\\
27.51	0\\
27.52	0\\
27.53	0\\
27.54	0\\
27.55	0\\
27.56	0\\
27.57	0\\
27.58	0\\
27.59	0\\
27.6	0\\
27.61	0\\
27.62	0\\
27.63	0\\
27.64	0\\
27.65	0\\
27.66	0\\
27.67	0\\
27.68	0\\
27.69	0\\
27.7	0\\
27.71	0\\
27.72	0\\
27.73	0\\
27.74	0\\
27.75	0\\
27.76	0\\
27.77	0\\
27.78	0\\
27.79	0\\
27.8	0\\
27.81	0\\
27.82	0\\
27.83	0\\
27.84	0\\
27.85	0\\
27.86	0\\
27.87	0\\
27.88	0\\
27.89	0\\
27.9	0\\
27.91	0\\
27.92	0\\
27.93	0\\
27.94	0\\
27.95	0\\
27.96	0\\
27.97	0\\
27.98	0\\
27.99	0\\
28	0\\
28.01	0\\
28.02	0\\
28.03	0\\
28.04	0\\
28.05	0\\
28.06	0\\
28.07	0\\
28.08	0\\
28.09	0\\
28.1	0\\
28.11	0\\
28.12	0\\
28.13	0\\
28.14	0\\
28.15	0\\
28.16	0\\
28.17	0\\
28.18	0\\
28.19	0\\
28.2	0\\
28.21	0\\
28.22	0\\
28.23	0\\
28.24	0\\
28.25	0\\
28.26	0\\
28.27	0\\
28.28	0\\
28.29	0\\
28.3	0\\
28.31	0\\
28.32	0\\
28.33	0\\
28.34	0\\
28.35	0\\
28.36	0\\
28.37	0\\
28.38	0\\
28.39	0\\
28.4	0\\
28.41	0\\
28.42	0\\
28.43	0\\
28.44	0\\
28.45	0\\
28.46	0\\
28.47	0\\
28.48	0\\
28.49	0\\
28.5	0\\
28.51	0\\
28.52	0\\
28.53	0\\
28.54	0\\
28.55	0\\
28.56	0\\
28.57	0\\
28.58	0\\
28.59	0\\
28.6	0\\
28.61	0\\
28.62	0\\
28.63	0\\
28.64	0\\
28.65	0\\
28.66	0\\
28.67	0\\
28.68	0\\
28.69	0\\
28.7	0\\
28.71	0\\
28.72	0\\
28.73	0\\
28.74	0\\
28.75	0\\
28.76	0\\
28.77	0\\
28.78	0\\
28.79	0\\
28.8	0\\
28.81	0\\
28.82	0\\
28.83	0\\
28.84	0\\
28.85	0\\
28.86	0\\
28.87	0\\
28.88	0\\
28.89	0\\
28.9	0\\
28.91	0\\
28.92	0\\
28.93	0\\
28.94	0\\
28.95	0\\
28.96	0\\
28.97	0\\
28.98	0\\
28.99	0\\
29	0\\
29.01	0\\
29.02	0\\
29.03	0\\
29.04	0\\
29.05	0\\
29.06	0\\
29.07	0\\
29.08	0\\
29.09	0\\
29.1	0\\
29.11	0\\
29.12	0\\
29.13	0\\
29.14	0\\
29.15	0\\
29.16	0\\
29.17	0\\
29.18	0\\
29.19	0\\
29.2	0\\
29.21	0\\
29.22	0\\
29.23	0\\
29.24	0\\
29.25	0\\
29.26	0\\
29.27	0\\
29.28	0\\
29.29	0\\
29.3	0\\
29.31	0\\
29.32	0\\
29.33	0\\
29.34	0\\
29.35	0\\
29.36	0\\
29.37	0\\
29.38	0\\
29.39	0\\
29.4	0\\
29.41	0\\
29.42	0\\
29.43	0\\
29.44	0\\
29.45	0\\
29.46	0\\
29.47	0\\
29.48	0\\
29.49	0\\
29.5	0\\
29.51	0\\
29.52	0\\
29.53	0\\
29.54	0\\
29.55	0\\
29.56	0\\
29.57	0\\
29.58	0\\
29.59	0\\
29.6	0\\
29.61	0\\
29.62	0\\
29.63	0\\
29.64	0\\
29.65	0\\
29.66	0\\
29.67	0\\
29.68	0\\
29.69	0\\
29.7	0\\
29.71	0\\
29.72	0\\
29.73	0\\
29.74	0\\
29.75	0\\
29.76	0\\
29.77	0\\
29.78	0\\
29.79	0\\
29.8	0\\
29.81	0\\
29.82	0\\
29.83	0\\
29.84	0\\
29.85	0\\
29.86	0\\
29.87	0\\
29.88	0\\
29.89	0\\
29.9	0\\
29.91	0\\
29.92	0\\
29.93	0\\
29.94	0\\
29.95	0\\
29.96	0\\
29.97	0\\
29.98	0\\
29.99	0\\
30	0\\
30.01	0\\
30.02	0\\
30.03	0\\
30.04	0\\
30.05	0\\
30.06	0\\
30.07	0\\
30.08	0\\
30.09	0\\
30.1	0\\
30.11	0\\
30.12	0\\
30.13	0\\
30.14	0\\
30.15	0\\
30.16	0\\
30.17	0\\
30.18	0\\
30.19	0\\
30.2	0\\
30.21	0\\
30.22	0\\
30.23	0\\
30.24	0\\
30.25	0\\
30.26	0\\
30.27	0\\
30.28	0\\
30.29	0\\
30.3	0\\
30.31	0\\
30.32	0\\
30.33	0\\
30.34	0\\
30.35	0\\
30.36	0\\
30.37	0\\
30.38	0\\
30.39	0\\
30.4	0\\
30.41	0\\
30.42	0\\
30.43	0\\
30.44	0\\
30.45	0\\
30.46	0\\
30.47	0\\
30.48	0\\
30.49	0\\
30.5	0\\
30.51	0\\
30.52	0\\
30.53	0\\
30.54	0\\
30.55	0\\
30.56	0\\
30.57	0\\
30.58	0\\
30.59	0\\
30.6	0\\
30.61	0\\
30.62	0\\
30.63	0\\
30.64	0\\
30.65	0\\
30.66	0\\
30.67	0\\
30.68	0\\
30.69	0\\
30.7	0\\
30.71	0\\
30.72	0\\
30.73	0\\
30.74	0\\
30.75	0\\
30.76	0\\
30.77	0\\
30.78	0\\
30.79	0\\
30.8	0\\
30.81	0\\
30.82	0\\
30.83	0\\
30.84	0\\
30.85	0\\
30.86	0\\
30.87	0\\
30.88	0\\
30.89	0\\
30.9	0\\
30.91	0\\
30.92	0\\
30.93	0\\
30.94	0\\
30.95	0\\
30.96	0\\
30.97	0\\
30.98	0\\
30.99	0\\
31	0\\
31.01	0\\
31.02	0\\
31.03	0\\
31.04	0\\
31.05	0\\
31.06	0\\
31.07	0\\
31.08	0\\
31.09	0\\
31.1	0\\
31.11	0\\
31.12	0\\
31.13	0\\
31.14	0\\
31.15	0\\
31.16	0\\
31.17	0\\
31.18	0\\
31.19	0\\
31.2	0\\
31.21	0\\
31.22	0\\
31.23	0\\
31.24	0\\
31.25	0\\
31.26	0\\
31.27	0\\
31.28	0\\
31.29	0\\
31.3	0\\
31.31	0\\
31.32	0\\
31.33	0\\
31.34	0\\
31.35	0\\
31.36	0\\
31.37	0\\
31.38	0\\
31.39	0\\
31.4	0\\
31.41	0\\
31.42	0\\
31.43	0\\
31.44	0\\
31.45	0\\
31.46	0\\
31.47	0\\
31.48	0\\
31.49	0\\
31.5	0\\
31.51	0\\
31.52	0\\
31.53	0\\
31.54	0\\
31.55	0\\
31.56	0\\
31.57	0\\
31.58	0\\
31.59	0\\
31.6	0\\
31.61	0\\
31.62	0\\
31.63	0\\
31.64	0\\
31.65	0\\
31.66	0\\
31.67	0\\
31.68	0\\
31.69	0\\
31.7	0\\
31.71	0\\
31.72	0\\
31.73	0\\
31.74	0\\
31.75	0\\
31.76	0\\
31.77	0\\
31.78	0\\
31.79	0\\
31.8	0\\
31.81	0\\
31.82	0\\
31.83	0\\
31.84	0\\
31.85	0\\
31.86	0\\
31.87	0\\
31.88	0\\
31.89	0\\
31.9	0\\
31.91	0\\
31.92	0\\
31.93	0\\
31.94	0\\
31.95	0\\
31.96	0\\
31.97	0\\
31.98	0\\
31.99	0\\
32	0\\
32.01	0\\
32.02	0\\
32.03	0\\
32.04	0\\
32.05	0\\
32.06	0\\
32.07	0\\
32.08	0\\
32.09	0\\
32.1	0\\
32.11	0\\
32.12	0\\
32.13	0\\
32.14	0\\
32.15	0\\
32.16	0\\
32.17	0\\
32.18	0\\
32.19	0\\
32.2	0\\
32.21	0\\
32.22	0\\
32.23	0\\
32.24	0\\
32.25	0\\
32.26	0\\
32.27	0\\
32.28	0\\
32.29	0\\
32.3	0\\
32.31	0\\
32.32	0\\
32.33	0\\
32.34	0\\
32.35	0\\
32.36	0\\
32.37	0\\
32.38	0\\
32.39	0\\
32.4	0\\
32.41	0\\
32.42	0\\
32.43	0\\
32.44	0\\
32.45	0\\
32.46	0\\
32.47	0\\
32.48	0\\
32.49	0\\
32.5	0\\
32.51	0\\
32.52	0\\
32.53	0\\
32.54	0\\
32.55	0\\
32.56	0\\
32.57	0\\
32.58	0\\
32.59	0\\
32.6	0\\
32.61	0\\
32.62	0\\
32.63	0\\
32.64	0\\
32.65	0\\
32.66	0\\
32.67	0\\
32.68	0\\
32.69	0\\
32.7	0\\
32.71	0\\
32.72	0\\
32.73	0\\
32.74	0\\
32.75	0\\
32.76	0\\
32.77	0\\
32.78	0\\
32.79	0\\
32.8	0\\
32.81	0\\
32.82	0\\
32.83	0\\
32.84	0\\
32.85	0\\
32.86	0\\
32.87	0\\
32.88	0\\
32.89	0\\
32.9	0\\
32.91	0\\
32.92	0\\
32.93	0\\
32.94	0\\
32.95	0\\
32.96	0\\
32.97	0\\
32.98	0\\
32.99	0\\
33	0\\
33.01	0\\
33.02	0\\
33.03	0\\
33.04	0\\
33.05	0\\
33.06	0\\
33.07	0\\
33.08	0\\
33.09	0\\
33.1	0\\
33.11	0\\
33.12	0\\
33.13	0\\
33.14	0\\
33.15	0\\
33.16	0\\
33.17	0\\
33.18	0\\
33.19	0\\
33.2	0\\
33.21	0\\
33.22	0\\
33.23	0\\
33.24	0\\
33.25	0\\
33.26	0\\
33.27	0\\
33.28	0\\
33.29	0\\
33.3	0\\
33.31	0\\
33.32	0\\
33.33	0\\
33.34	0\\
33.35	0\\
33.36	0\\
33.37	0\\
33.38	0\\
33.39	0\\
33.4	0\\
33.41	0\\
33.42	0\\
33.43	0\\
33.44	0\\
33.45	0\\
33.46	0\\
33.47	0\\
33.48	0\\
33.49	0\\
33.5	0\\
33.51	0\\
33.52	0\\
33.53	0\\
33.54	0\\
33.55	0\\
33.56	0\\
33.57	0\\
33.58	0\\
33.59	0\\
33.6	0\\
33.61	0\\
33.62	0\\
33.63	0\\
33.64	0\\
33.65	0\\
33.66	0\\
33.67	0\\
33.68	0\\
33.69	0\\
33.7	0\\
33.71	0\\
33.72	0\\
33.73	0\\
33.74	0\\
33.75	0\\
33.76	0\\
33.77	0\\
33.78	0\\
33.79	0\\
33.8	0\\
33.81	0\\
33.82	0\\
33.83	0\\
33.84	0\\
33.85	0\\
33.86	0\\
33.87	0\\
33.88	0\\
33.89	0\\
33.9	0\\
33.91	0\\
33.92	0\\
33.93	0\\
33.94	0\\
33.95	0\\
33.96	0\\
33.97	0\\
33.98	0\\
33.99	0\\
34	0\\
34.01	0\\
34.02	0\\
34.03	0\\
34.04	0\\
34.05	0\\
34.06	0\\
34.07	0\\
34.08	0\\
34.09	0\\
34.1	0\\
34.11	0\\
34.12	0\\
34.13	0\\
34.14	0\\
34.15	0\\
34.16	0\\
34.17	0\\
34.18	0\\
34.19	0\\
34.2	0\\
34.21	0\\
34.22	0\\
34.23	0\\
34.24	0\\
34.25	0\\
34.26	0\\
34.27	0\\
34.28	0\\
34.29	0\\
34.3	0\\
34.31	0\\
34.32	0\\
34.33	0\\
34.34	0\\
34.35	0\\
34.36	0\\
34.37	0\\
34.38	0\\
34.39	0\\
34.4	0\\
34.41	0\\
34.42	0\\
34.43	0\\
34.44	0\\
34.45	0\\
34.46	0\\
34.47	0\\
34.48	0\\
34.49	0\\
34.5	0\\
34.51	0\\
34.52	0\\
34.53	0\\
34.54	0\\
34.55	0\\
34.56	0\\
34.57	0\\
34.58	0\\
34.59	0\\
34.6	0\\
34.61	0\\
34.62	0\\
34.63	0\\
34.64	0\\
34.65	0\\
34.66	0\\
34.67	0\\
34.68	0\\
34.69	0\\
34.7	0\\
34.71	0\\
34.72	0\\
34.73	0\\
34.74	0\\
34.75	0\\
34.76	0\\
34.77	0\\
34.78	0\\
34.79	0\\
34.8	0\\
34.81	0\\
34.82	0\\
34.83	0\\
34.84	0\\
34.85	0\\
34.86	0\\
34.87	0\\
34.88	0\\
34.89	0\\
34.9	0\\
34.91	0\\
34.92	0\\
34.93	0\\
34.94	0\\
34.95	0\\
34.96	0\\
34.97	0\\
34.98	0\\
34.99	0\\
35	0\\
35.01	0\\
35.02	0\\
35.03	0\\
35.04	0\\
35.05	0\\
35.06	0\\
35.07	0\\
35.08	0\\
35.09	0\\
35.1	0\\
35.11	0\\
35.12	0\\
35.13	0\\
35.14	0\\
35.15	0\\
35.16	0\\
35.17	0\\
35.18	0\\
35.19	0\\
35.2	0\\
35.21	0\\
35.22	0\\
35.23	0\\
35.24	0\\
35.25	0\\
35.26	0\\
35.27	0\\
35.28	0\\
35.29	0\\
35.3	0\\
35.31	0\\
35.32	0\\
35.33	0\\
35.34	0\\
35.35	0\\
35.36	0\\
35.37	0\\
35.38	0\\
35.39	0\\
35.4	0\\
35.41	0\\
35.42	0\\
35.43	0\\
35.44	0\\
35.45	0\\
35.46	0\\
35.47	0\\
35.48	0\\
35.49	0\\
35.5	0\\
35.51	0\\
35.52	0\\
35.53	0\\
35.54	0\\
35.55	0\\
35.56	0\\
35.57	0\\
35.58	0\\
35.59	0\\
35.6	0\\
35.61	0\\
35.62	0\\
35.63	0\\
35.64	0\\
35.65	0\\
35.66	0\\
35.67	0\\
35.68	0\\
35.69	0\\
35.7	0\\
35.71	0\\
35.72	0\\
35.73	0\\
35.74	0\\
35.75	0\\
35.76	0\\
35.77	0\\
35.78	0\\
35.79	0\\
35.8	0\\
35.81	0\\
35.82	0\\
35.83	0\\
35.84	0\\
35.85	0\\
35.86	0\\
35.87	0\\
35.88	0\\
35.89	0\\
35.9	0\\
35.91	0\\
35.92	0\\
35.93	0\\
35.94	0\\
35.95	0\\
35.96	0\\
35.97	0\\
35.98	0\\
35.99	0\\
36	0\\
36.01	0\\
36.02	0\\
36.03	0\\
36.04	0\\
36.05	0\\
36.06	0\\
36.07	0\\
36.08	0\\
36.09	0\\
36.1	0\\
36.11	0\\
36.12	0\\
36.13	0\\
36.14	0\\
36.15	0\\
36.16	0\\
36.17	0\\
36.18	0\\
36.19	0\\
36.2	0\\
36.21	0\\
36.22	0\\
36.23	0\\
36.24	0\\
36.25	0\\
36.26	0\\
36.27	0\\
36.28	0\\
36.29	0\\
36.3	0\\
36.31	0\\
36.32	0\\
36.33	0\\
36.34	0\\
36.35	0\\
36.36	0\\
36.37	0\\
36.38	0\\
36.39	0\\
36.4	0\\
36.41	0\\
36.42	0\\
36.43	0\\
36.44	0\\
36.45	0\\
36.46	0\\
36.47	0\\
36.48	0\\
36.49	0\\
36.5	0\\
36.51	0\\
36.52	0\\
36.53	0\\
36.54	0\\
36.55	0\\
36.56	0\\
36.57	0\\
36.58	0\\
36.59	0\\
36.6	0\\
36.61	0\\
36.62	0\\
36.63	0\\
36.64	0\\
36.65	0\\
36.66	0\\
36.67	0\\
36.68	0\\
36.69	0\\
36.7	0\\
36.71	0\\
36.72	0\\
36.73	0\\
36.74	0\\
36.75	0\\
36.76	0\\
36.77	0\\
36.78	0\\
36.79	0\\
36.8	0\\
36.81	0\\
36.82	0\\
36.83	0\\
36.84	0\\
36.85	0\\
36.86	0\\
36.87	0\\
36.88	0\\
36.89	0\\
36.9	0\\
36.91	0\\
36.92	0\\
36.93	0\\
36.94	0\\
36.95	0\\
36.96	0\\
36.97	0\\
36.98	0\\
36.99	0\\
37	0\\
37.01	0\\
37.02	0\\
37.03	0\\
37.04	0\\
37.05	0\\
37.06	0\\
37.07	0\\
37.08	0\\
37.09	0\\
37.1	0\\
37.11	0\\
37.12	0\\
37.13	0\\
37.14	0\\
37.15	0\\
37.16	0\\
37.17	0\\
37.18	0\\
37.19	0\\
37.2	0\\
37.21	0\\
37.22	0\\
37.23	0\\
37.24	0\\
37.25	0\\
37.26	0\\
37.27	0\\
37.28	0\\
37.29	0\\
37.3	0\\
37.31	0\\
37.32	0\\
37.33	0\\
37.34	0\\
37.35	0\\
37.36	0\\
37.37	0\\
37.38	0\\
37.39	0\\
37.4	0\\
37.41	0\\
37.42	0\\
37.43	0\\
37.44	0\\
37.45	0\\
37.46	0\\
37.47	0\\
37.48	0\\
37.49	0\\
37.5	0\\
37.51	0\\
37.52	0\\
37.53	0\\
37.54	0\\
37.55	0\\
37.56	0\\
37.57	0\\
37.58	0\\
37.59	0\\
37.6	0\\
37.61	0\\
37.62	0\\
37.63	0\\
37.64	0\\
37.65	0\\
37.66	0\\
37.67	0\\
37.68	0\\
37.69	0\\
37.7	0\\
37.71	0\\
37.72	0\\
37.73	0\\
37.74	0\\
37.75	0\\
37.76	0\\
37.77	0\\
37.78	0\\
37.79	0\\
37.8	0\\
37.81	0\\
37.82	0\\
37.83	0\\
37.84	0\\
37.85	0\\
37.86	0\\
37.87	0\\
37.88	0\\
37.89	0\\
37.9	0\\
37.91	0\\
37.92	0\\
37.93	0\\
37.94	0\\
37.95	0\\
37.96	0\\
37.97	0\\
37.98	0\\
37.99	0\\
38	0\\
38.01	0\\
38.02	0\\
38.03	0\\
38.04	0\\
38.05	0\\
38.06	0\\
38.07	0\\
38.08	0\\
38.09	0\\
38.1	0\\
38.11	0\\
38.12	0\\
38.13	0\\
38.14	0\\
38.15	0\\
38.16	0\\
38.17	0\\
38.18	0\\
38.19	0\\
38.2	0\\
38.21	0\\
38.22	0\\
38.23	0\\
38.24	0\\
38.25	0\\
38.26	0\\
38.27	0\\
38.28	0\\
38.29	0\\
38.3	0\\
38.31	0\\
38.32	0\\
38.33	0\\
38.34	0\\
38.35	0\\
38.36	0\\
38.37	0\\
38.38	0\\
38.39	0\\
38.4	0\\
38.41	0\\
38.42	0\\
38.43	0\\
38.44	0\\
38.45	0\\
38.46	0\\
38.47	0\\
38.48	0\\
38.49	0\\
38.5	0\\
38.51	0\\
38.52	0\\
38.53	0\\
38.54	0\\
38.55	0\\
38.56	0\\
38.57	0\\
38.58	0\\
38.59	0\\
38.6	0\\
38.61	0\\
38.62	0\\
38.63	0\\
38.64	0\\
38.65	0\\
38.66	0\\
38.67	0\\
38.68	0\\
38.69	0\\
38.7	0\\
38.71	0\\
38.72	0\\
38.73	0\\
38.74	0\\
38.75	0\\
38.76	0\\
38.77	0\\
38.78	0\\
38.79	0\\
38.8	0\\
38.81	0\\
38.82	0\\
38.83	0\\
38.84	0\\
38.85	0\\
38.86	0\\
38.87	0\\
38.88	0\\
38.89	0\\
38.9	0\\
38.91	0\\
38.92	0\\
38.93	0\\
38.94	0\\
38.95	0\\
38.96	0\\
38.97	0\\
38.98	0\\
38.99	0\\
39	0\\
39.01	0\\
39.02	0\\
39.03	0\\
39.04	0\\
39.05	0\\
39.06	0\\
39.07	0\\
39.08	0\\
39.09	0\\
39.1	0\\
39.11	0\\
39.12	0\\
39.13	0\\
39.14	0\\
39.15	0\\
39.16	0\\
39.17	0\\
39.18	0\\
39.19	0\\
39.2	0\\
39.21	0\\
39.22	0\\
39.23	0\\
39.24	0\\
39.25	0\\
39.26	0\\
39.27	0\\
39.28	0\\
39.29	0\\
39.3	0\\
39.31	0\\
39.32	0\\
39.33	0\\
39.34	0\\
39.35	0\\
39.36	0\\
39.37	0\\
39.38	0\\
39.39	0\\
39.4	0\\
39.41	0\\
39.42	0\\
39.43	0\\
39.44	0\\
39.45	0\\
39.46	0\\
39.47	0\\
39.48	0\\
39.49	0\\
39.5	0\\
39.51	0\\
39.52	0\\
39.53	0\\
39.54	0\\
39.55	0\\
39.56	0\\
39.57	0\\
39.58	0\\
39.59	0\\
39.6	0\\
39.61	0\\
39.62	0\\
39.63	0\\
39.64	0\\
39.65	0\\
39.66	0\\
39.67	0\\
39.68	0\\
39.69	0\\
39.7	0\\
39.71	0\\
39.72	0\\
39.73	0\\
39.74	0\\
39.75	0\\
39.76	0\\
39.77	0\\
39.78	0\\
39.79	0\\
39.8	0\\
39.81	0\\
39.82	0\\
39.83	0\\
39.84	0\\
39.85	0\\
39.86	0\\
39.87	0\\
39.88	0\\
39.89	0\\
39.9	0\\
39.91	0\\
39.92	0\\
39.93	0\\
39.94	0\\
39.95	0\\
39.96	0\\
39.97	0\\
39.98	0\\
39.99	0\\
40	0\\
40.01	0\\
};
\addplot [color=green,dashed,forget plot]
  table[row sep=crcr]{%
40.01	0\\
40.02	0\\
40.03	0\\
40.04	0\\
40.05	0\\
40.06	0\\
40.07	0\\
40.08	0\\
40.09	0\\
40.1	0\\
40.11	0\\
40.12	0\\
40.13	0\\
40.14	0\\
40.15	0\\
40.16	0\\
40.17	0\\
40.18	0\\
40.19	0\\
40.2	0\\
40.21	0\\
40.22	0\\
40.23	0\\
40.24	0\\
40.25	0\\
40.26	0\\
40.27	0\\
40.28	0\\
40.29	0\\
40.3	0\\
40.31	0\\
40.32	0\\
40.33	0\\
40.34	0\\
40.35	0\\
40.36	0\\
40.37	0\\
40.38	0\\
40.39	0\\
40.4	0\\
40.41	0\\
40.42	0\\
40.43	0\\
40.44	0\\
40.45	0\\
40.46	0\\
40.47	0\\
40.48	0\\
40.49	0\\
40.5	0\\
40.51	0\\
40.52	0\\
40.53	0\\
40.54	0\\
40.55	0\\
40.56	0\\
40.57	0\\
40.58	0\\
40.59	0\\
40.6	0\\
40.61	0\\
40.62	0\\
40.63	0\\
40.64	0\\
40.65	0\\
40.66	0\\
40.67	0\\
40.68	0\\
40.69	0\\
40.7	0\\
40.71	0\\
40.72	0\\
40.73	0\\
40.74	0\\
40.75	0\\
40.76	0\\
40.77	0\\
40.78	0\\
40.79	0\\
40.8	0\\
40.81	0\\
40.82	0\\
40.83	0\\
40.84	0\\
40.85	0\\
40.86	0\\
40.87	0\\
40.88	0\\
40.89	0\\
40.9	0\\
40.91	0\\
40.92	0\\
40.93	0\\
40.94	0\\
40.95	0\\
40.96	0\\
40.97	0\\
40.98	0\\
40.99	0\\
41	0\\
41.01	0\\
41.02	0\\
41.03	0\\
41.04	0\\
41.05	0\\
41.06	0\\
41.07	0\\
41.08	0\\
41.09	0\\
41.1	0\\
41.11	0\\
41.12	0\\
41.13	0\\
41.14	0\\
41.15	0\\
41.16	0\\
41.17	0\\
41.18	0\\
41.19	0\\
41.2	0\\
41.21	0\\
41.22	0\\
41.23	0\\
41.24	0\\
41.25	0\\
41.26	0\\
41.27	0\\
41.28	0\\
41.29	0\\
41.3	0\\
41.31	0\\
41.32	0\\
41.33	0\\
41.34	0\\
41.35	0\\
41.36	0\\
41.37	0\\
41.38	0\\
41.39	0\\
41.4	0\\
41.41	0\\
41.42	0\\
41.43	0\\
41.44	0\\
41.45	0\\
41.46	0\\
41.47	0\\
41.48	0\\
41.49	0\\
41.5	0\\
41.51	0\\
41.52	0\\
41.53	0\\
41.54	0\\
41.55	0\\
41.56	0\\
41.57	0\\
41.58	0\\
41.59	0\\
41.6	0\\
41.61	0\\
41.62	0\\
41.63	0\\
41.64	0\\
41.65	0\\
41.66	0\\
41.67	0\\
41.68	0\\
41.69	0\\
41.7	0\\
41.71	0\\
41.72	0\\
41.73	0\\
41.74	0\\
41.75	0\\
41.76	0\\
41.77	0\\
41.78	0\\
41.79	0\\
41.8	0\\
41.81	0\\
41.82	0\\
41.83	0\\
41.84	0\\
41.85	0\\
41.86	0\\
41.87	0\\
41.88	0\\
41.89	0\\
41.9	0\\
41.91	0\\
41.92	0\\
41.93	0\\
41.94	0\\
41.95	0\\
41.96	0\\
41.97	0\\
41.98	0\\
41.99	0\\
42	0\\
42.01	0\\
42.02	0\\
42.03	0\\
42.04	0\\
42.05	0\\
42.06	0\\
42.07	0\\
42.08	0\\
42.09	0\\
42.1	0\\
42.11	0\\
42.12	0\\
42.13	0\\
42.14	0\\
42.15	0\\
42.16	0\\
42.17	0\\
42.18	0\\
42.19	0\\
42.2	0\\
42.21	0\\
42.22	0\\
42.23	0\\
42.24	0\\
42.25	0\\
42.26	0\\
42.27	0\\
42.28	0\\
42.29	0\\
42.3	0\\
42.31	0\\
42.32	0\\
42.33	0\\
42.34	0\\
42.35	0\\
42.36	0\\
42.37	0\\
42.38	0\\
42.39	0\\
42.4	0\\
42.41	0\\
42.42	0\\
42.43	0\\
42.44	0\\
42.45	0\\
42.46	0\\
42.47	0\\
42.48	0\\
42.49	0\\
42.5	0\\
42.51	0\\
42.52	0\\
42.53	0\\
42.54	0\\
42.55	0\\
42.56	0\\
42.57	0\\
42.58	0\\
42.59	0\\
42.6	0\\
42.61	0\\
42.62	0\\
42.63	0\\
42.64	0\\
42.65	0\\
42.66	0\\
42.67	0\\
42.68	0\\
42.69	0\\
42.7	0\\
42.71	0\\
42.72	0\\
42.73	0\\
42.74	0\\
42.75	0\\
42.76	0\\
42.77	0\\
42.78	0\\
42.79	0\\
42.8	0\\
42.81	0\\
42.82	0\\
42.83	0\\
42.84	0\\
42.85	0\\
42.86	0\\
42.87	0\\
42.88	0\\
42.89	0\\
42.9	0\\
42.91	0\\
42.92	0\\
42.93	0\\
42.94	0\\
42.95	0\\
42.96	0\\
42.97	0\\
42.98	0\\
42.99	0\\
43	0\\
43.01	0\\
43.02	0\\
43.03	0\\
43.04	0\\
43.05	0\\
43.06	0\\
43.07	0\\
43.08	0\\
43.09	0\\
43.1	0\\
43.11	0\\
43.12	0\\
43.13	0\\
43.14	0\\
43.15	0\\
43.16	0\\
43.17	0\\
43.18	0\\
43.19	0\\
43.2	0\\
43.21	0\\
43.22	0\\
43.23	0\\
43.24	0\\
43.25	0\\
43.26	0\\
43.27	0\\
43.28	0\\
43.29	0\\
43.3	0\\
43.31	0\\
43.32	0\\
43.33	0\\
43.34	0\\
43.35	0\\
43.36	0\\
43.37	0\\
43.38	0\\
43.39	0\\
43.4	0\\
43.41	0\\
43.42	0\\
43.43	0\\
43.44	0\\
43.45	0\\
43.46	0\\
43.47	0\\
43.48	0\\
43.49	0\\
43.5	0\\
43.51	0\\
43.52	0\\
43.53	0\\
43.54	0\\
43.55	0\\
43.56	0\\
43.57	0\\
43.58	0\\
43.59	0\\
43.6	0\\
43.61	0\\
43.62	0\\
43.63	0\\
43.64	0\\
43.65	0\\
43.66	0\\
43.67	0\\
43.68	0\\
43.69	0\\
43.7	0\\
43.71	0\\
43.72	0\\
43.73	0\\
43.74	0\\
43.75	0\\
43.76	0\\
43.77	0\\
43.78	0\\
43.79	0\\
43.8	0\\
43.81	0\\
43.82	0\\
43.83	0\\
43.84	0\\
43.85	0\\
43.86	0\\
43.87	0\\
43.88	0\\
43.89	0\\
43.9	0\\
43.91	0\\
43.92	0\\
43.93	0\\
43.94	0\\
43.95	0\\
43.96	0\\
43.97	0\\
43.98	0\\
43.99	0\\
44	0\\
44.01	0\\
44.02	0\\
44.03	0\\
44.04	0\\
44.05	0\\
44.06	0\\
44.07	0\\
44.08	0\\
44.09	0\\
44.1	0\\
44.11	0\\
44.12	0\\
44.13	0\\
44.14	0\\
44.15	0\\
44.16	0\\
44.17	0\\
44.18	0\\
44.19	0\\
44.2	0\\
44.21	0\\
44.22	0\\
44.23	0\\
44.24	0\\
44.25	0\\
44.26	0\\
44.27	0\\
44.28	0\\
44.29	0\\
44.3	0\\
44.31	0\\
44.32	0\\
44.33	0\\
44.34	0\\
44.35	0\\
44.36	0\\
44.37	0\\
44.38	0\\
44.39	0\\
44.4	0\\
44.41	0\\
44.42	0\\
44.43	0\\
44.44	0\\
44.45	0\\
44.46	0\\
44.47	0\\
44.48	0\\
44.49	0\\
44.5	0\\
44.51	0\\
44.52	0\\
44.53	0\\
44.54	0\\
44.55	0\\
44.56	0\\
44.57	0\\
44.58	0\\
44.59	0\\
44.6	0\\
44.61	0\\
44.62	0\\
44.63	0\\
44.64	0\\
44.65	0\\
44.66	0\\
44.67	0\\
44.68	0\\
44.69	0\\
44.7	0\\
44.71	0\\
44.72	0\\
44.73	0\\
44.74	0\\
44.75	0\\
44.76	0\\
44.77	0\\
44.78	0\\
44.79	0\\
44.8	0\\
44.81	0\\
44.82	0\\
44.83	0\\
44.84	0\\
44.85	0\\
44.86	0\\
44.87	0\\
44.88	0\\
44.89	0\\
44.9	0\\
44.91	0\\
44.92	0\\
44.93	0\\
44.94	0\\
44.95	0\\
44.96	0\\
44.97	0\\
44.98	0\\
44.99	0\\
45	0\\
45.01	0\\
45.02	0\\
45.03	0\\
45.04	0\\
45.05	0\\
45.06	0\\
45.07	0\\
45.08	0\\
45.09	0\\
45.1	0\\
45.11	0\\
45.12	0\\
45.13	0\\
45.14	0\\
45.15	0\\
45.16	0\\
45.17	0\\
45.18	0\\
45.19	0\\
45.2	0\\
45.21	0\\
45.22	0\\
45.23	0\\
45.24	0\\
45.25	0\\
45.26	0\\
45.27	0\\
45.28	0\\
45.29	0\\
45.3	0\\
45.31	0\\
45.32	0\\
45.33	0\\
45.34	0\\
45.35	0\\
45.36	0\\
45.37	0\\
45.38	0\\
45.39	0\\
45.4	0\\
45.41	0\\
45.42	0\\
45.43	0\\
45.44	0\\
45.45	0\\
45.46	0\\
45.47	0\\
45.48	0\\
45.49	0\\
45.5	0\\
45.51	0\\
45.52	0\\
45.53	0\\
45.54	0\\
45.55	0\\
45.56	0\\
45.57	0\\
45.58	0\\
45.59	0\\
45.6	0\\
45.61	0\\
45.62	0\\
45.63	0\\
45.64	0\\
45.65	0\\
45.66	0\\
45.67	0\\
45.68	0\\
45.69	0\\
45.7	0\\
45.71	0\\
45.72	0\\
45.73	0\\
45.74	0\\
45.75	0\\
45.76	0\\
45.77	0\\
45.78	0\\
45.79	0\\
45.8	0\\
45.81	0\\
45.82	0\\
45.83	0\\
45.84	0\\
45.85	0\\
45.86	0\\
45.87	0\\
45.88	0\\
45.89	0\\
45.9	0\\
45.91	0\\
45.92	0\\
45.93	0\\
45.94	0\\
45.95	0\\
45.96	0\\
45.97	0\\
45.98	0\\
45.99	0\\
46	0\\
46.01	0\\
46.02	0\\
46.03	0\\
46.04	0\\
46.05	0\\
46.06	0\\
46.07	0\\
46.08	0\\
46.09	0\\
46.1	0\\
46.11	0\\
46.12	0\\
46.13	0\\
46.14	0\\
46.15	0\\
46.16	0\\
46.17	0\\
46.18	0\\
46.19	0\\
46.2	0\\
46.21	0\\
46.22	0\\
46.23	0\\
46.24	0\\
46.25	0\\
46.26	0\\
46.27	0\\
46.28	0\\
46.29	0\\
46.3	0\\
46.31	0\\
46.32	0\\
46.33	0\\
46.34	0\\
46.35	0\\
46.36	0\\
46.37	0\\
46.38	0\\
46.39	0\\
46.4	0\\
46.41	0\\
46.42	0\\
46.43	0\\
46.44	0\\
46.45	0\\
46.46	0\\
46.47	0\\
46.48	0\\
46.49	0\\
46.5	0\\
46.51	0\\
46.52	0\\
46.53	0\\
46.54	0\\
46.55	0\\
46.56	0\\
46.57	0\\
46.58	0\\
46.59	0\\
46.6	0\\
46.61	0\\
46.62	0\\
46.63	0\\
46.64	0\\
46.65	0\\
46.66	0\\
46.67	0\\
46.68	0\\
46.69	0\\
46.7	0\\
46.71	0\\
46.72	0\\
46.73	0\\
46.74	0\\
46.75	0\\
46.76	0\\
46.77	0\\
46.78	0\\
46.79	0\\
46.8	0\\
46.81	0\\
46.82	0\\
46.83	0\\
46.84	0\\
46.85	0\\
46.86	0\\
46.87	0\\
46.88	0\\
46.89	0\\
46.9	0\\
46.91	0\\
46.92	0\\
46.93	0\\
46.94	0\\
46.95	0\\
46.96	0\\
46.97	0\\
46.98	0\\
46.99	0\\
47	0\\
47.01	0\\
47.02	0\\
47.03	0\\
47.04	0\\
47.05	0\\
47.06	0\\
47.07	0\\
47.08	0\\
47.09	0\\
47.1	0\\
47.11	0\\
47.12	0\\
47.13	0\\
47.14	0\\
47.15	0\\
47.16	0\\
47.17	0\\
47.18	0\\
47.19	0\\
47.2	0\\
47.21	0\\
47.22	0\\
47.23	0\\
47.24	0\\
47.25	0\\
47.26	0\\
47.27	0\\
47.28	0\\
47.29	0\\
47.3	0\\
47.31	0\\
47.32	0\\
47.33	0\\
47.34	0\\
47.35	0\\
47.36	0\\
47.37	0\\
47.38	0\\
47.39	0\\
47.4	0\\
47.41	0\\
47.42	0\\
47.43	0\\
47.44	0\\
47.45	0\\
47.46	0\\
47.47	0\\
47.48	0\\
47.49	0\\
47.5	0\\
47.51	0\\
47.52	0\\
47.53	0\\
47.54	0\\
47.55	0\\
47.56	0\\
47.57	0\\
47.58	0\\
47.59	0\\
47.6	0\\
47.61	0\\
47.62	0\\
47.63	0\\
47.64	0\\
47.65	0\\
47.66	0\\
47.67	0\\
47.68	0\\
47.69	0\\
47.7	0\\
47.71	0\\
47.72	0\\
47.73	0\\
47.74	0\\
47.75	0\\
47.76	0\\
47.77	0\\
47.78	0\\
47.79	0\\
47.8	0\\
47.81	0\\
47.82	0\\
47.83	0\\
47.84	0\\
47.85	0\\
47.86	0\\
47.87	0\\
47.88	0\\
47.89	0\\
47.9	0\\
47.91	0\\
47.92	0\\
47.93	0\\
47.94	0\\
47.95	0\\
47.96	0\\
47.97	0\\
47.98	0\\
47.99	0\\
48	0\\
48.01	0\\
48.02	0\\
48.03	0\\
48.04	0\\
48.05	0\\
48.06	0\\
48.07	0\\
48.08	0\\
48.09	0\\
48.1	0\\
48.11	0\\
48.12	0\\
48.13	0\\
48.14	0\\
48.15	0\\
48.16	0\\
48.17	0\\
48.18	0\\
48.19	0\\
48.2	0\\
48.21	0\\
48.22	0\\
48.23	0\\
48.24	0\\
48.25	0\\
48.26	0\\
48.27	0\\
48.28	0\\
48.29	0\\
48.3	0\\
48.31	0\\
48.32	0\\
48.33	0\\
48.34	0\\
48.35	0\\
48.36	0\\
48.37	0\\
48.38	0\\
48.39	0\\
48.4	0\\
48.41	0\\
48.42	0\\
48.43	0\\
48.44	0\\
48.45	0\\
48.46	0\\
48.47	0\\
48.48	0\\
48.49	0\\
48.5	0\\
48.51	0\\
48.52	0\\
48.53	0\\
48.54	0\\
48.55	0\\
48.56	0\\
48.57	0\\
48.58	0\\
48.59	0\\
48.6	0\\
48.61	0\\
48.62	0\\
48.63	0\\
48.64	0\\
48.65	0\\
48.66	0\\
48.67	0\\
48.68	0\\
48.69	0\\
48.7	0\\
48.71	0\\
48.72	0\\
48.73	0\\
48.74	0\\
48.75	0\\
48.76	0\\
48.77	0\\
48.78	0\\
48.79	0\\
48.8	0\\
48.81	0\\
48.82	0\\
48.83	0\\
48.84	0\\
48.85	0\\
48.86	0\\
48.87	0\\
48.88	0\\
48.89	0\\
48.9	0\\
48.91	0\\
48.92	0\\
48.93	0\\
48.94	0\\
48.95	0\\
48.96	0\\
48.97	0\\
48.98	0\\
48.99	0\\
49	0\\
49.01	0\\
49.02	0\\
49.03	0\\
49.04	0\\
49.05	0\\
49.06	0\\
49.07	0\\
49.08	0\\
49.09	0\\
49.1	0\\
49.11	0\\
49.12	0\\
49.13	0\\
49.14	0\\
49.15	0\\
49.16	0\\
49.17	0\\
49.18	0\\
49.19	0\\
49.2	0\\
49.21	0\\
49.22	0\\
49.23	0\\
49.24	0\\
49.25	0\\
49.26	0\\
49.27	0\\
49.28	0\\
49.29	0\\
49.3	0\\
49.31	0\\
49.32	0\\
49.33	0\\
49.34	0\\
49.35	0\\
49.36	0\\
49.37	0\\
49.38	0\\
49.39	0\\
49.4	0\\
49.41	0\\
49.42	0\\
49.43	0\\
49.44	0\\
49.45	0\\
49.46	0\\
49.47	0\\
49.48	0\\
49.49	0\\
49.5	0\\
49.51	0\\
49.52	0\\
49.53	0\\
49.54	0\\
49.55	0\\
49.56	0\\
49.57	0\\
49.58	0\\
49.59	0\\
49.6	0\\
49.61	0\\
49.62	0\\
49.63	0\\
49.64	0\\
49.65	0\\
49.66	0\\
49.67	0\\
49.68	0\\
49.69	0\\
49.7	0\\
49.71	0\\
49.72	0\\
49.73	0\\
49.74	0\\
49.75	0\\
49.76	0\\
49.77	0\\
49.78	0\\
49.79	0\\
49.8	0\\
49.81	0\\
49.82	0\\
49.83	0\\
49.84	0\\
49.85	0\\
49.86	0\\
49.87	0\\
49.88	0\\
49.89	0\\
49.9	0\\
49.91	0\\
49.92	0\\
49.93	0\\
49.94	0\\
49.95	0\\
49.96	0\\
49.97	0\\
49.98	0\\
49.99	0\\
50	0\\
50.01	0\\
50.02	0\\
50.03	0\\
50.04	0\\
50.05	0\\
50.06	0\\
50.07	0\\
50.08	0\\
50.09	0\\
50.1	0\\
50.11	0\\
50.12	0\\
50.13	0\\
50.14	0\\
50.15	0\\
50.16	0\\
50.17	0\\
50.18	0\\
50.19	0\\
50.2	0\\
50.21	0\\
50.22	0\\
50.23	0\\
50.24	0\\
50.25	0\\
50.26	0\\
50.27	0\\
50.28	0\\
50.29	0\\
50.3	0\\
50.31	0\\
50.32	0\\
50.33	0\\
50.34	0\\
50.35	0\\
50.36	0\\
50.37	0\\
50.38	0\\
50.39	0\\
50.4	0\\
50.41	0\\
50.42	0\\
50.43	0\\
50.44	0\\
50.45	0\\
50.46	0\\
50.47	0\\
50.48	0\\
50.49	0\\
50.5	0\\
50.51	0\\
50.52	0\\
50.53	0\\
50.54	0\\
50.55	0\\
50.56	0\\
50.57	0\\
50.58	0\\
50.59	0\\
50.6	0\\
50.61	0\\
50.62	0\\
50.63	0\\
50.64	0\\
50.65	0\\
50.66	0\\
50.67	0\\
50.68	0\\
50.69	0\\
50.7	0\\
50.71	0\\
50.72	0\\
50.73	0\\
50.74	0\\
50.75	0\\
50.76	0\\
50.77	0\\
50.78	0\\
50.79	0\\
50.8	0\\
50.81	0\\
50.82	0\\
50.83	0\\
50.84	0\\
50.85	0\\
50.86	0\\
50.87	0\\
50.88	0\\
50.89	0\\
50.9	0\\
50.91	0\\
50.92	0\\
50.93	0\\
50.94	0\\
50.95	0\\
50.96	0\\
50.97	0\\
50.98	0\\
50.99	0\\
51	0\\
51.01	0\\
51.02	0\\
51.03	0\\
51.04	0\\
51.05	0\\
51.06	0\\
51.07	0\\
51.08	0\\
51.09	0\\
51.1	0\\
51.11	0\\
51.12	0\\
51.13	0\\
51.14	0\\
51.15	0\\
51.16	0\\
51.17	0\\
51.18	0\\
51.19	0\\
51.2	0\\
51.21	0\\
51.22	0\\
51.23	0\\
51.24	0\\
51.25	0\\
51.26	0\\
51.27	0\\
51.28	0\\
51.29	0\\
51.3	0\\
51.31	0\\
51.32	0\\
51.33	0\\
51.34	0\\
51.35	0\\
51.36	0\\
51.37	0\\
51.38	0\\
51.39	0\\
51.4	0\\
51.41	0\\
51.42	0\\
51.43	0\\
51.44	0\\
51.45	0\\
51.46	0\\
51.47	0\\
51.48	0\\
51.49	0\\
51.5	0\\
51.51	0\\
51.52	0\\
51.53	0\\
51.54	0\\
51.55	0\\
51.56	0\\
51.57	0\\
51.58	0\\
51.59	0\\
51.6	0\\
51.61	0\\
51.62	0\\
51.63	0\\
51.64	0\\
51.65	0\\
51.66	0\\
51.67	0\\
51.68	0\\
51.69	0\\
51.7	0\\
51.71	0\\
51.72	0\\
51.73	0\\
51.74	0\\
51.75	0\\
51.76	0\\
51.77	0\\
51.78	0\\
51.79	0\\
51.8	0\\
51.81	0\\
51.82	0\\
51.83	0\\
51.84	0\\
51.85	0\\
51.86	0\\
51.87	0\\
51.88	0\\
51.89	0\\
51.9	0\\
51.91	0\\
51.92	0\\
51.93	0\\
51.94	0\\
51.95	0\\
51.96	0\\
51.97	0\\
51.98	0\\
51.99	0\\
52	0\\
52.01	0\\
52.02	0\\
52.03	0\\
52.04	0\\
52.05	0\\
52.06	0\\
52.07	0\\
52.08	0\\
52.09	0\\
52.1	0\\
52.11	0\\
52.12	0\\
52.13	0\\
52.14	0\\
52.15	0\\
52.16	0\\
52.17	0\\
52.18	0\\
52.19	0\\
52.2	0\\
52.21	0\\
52.22	0\\
52.23	0\\
52.24	0\\
52.25	0\\
52.26	0\\
52.27	0\\
52.28	0\\
52.29	0\\
52.3	0\\
52.31	0\\
52.32	0\\
52.33	0\\
52.34	0\\
52.35	0\\
52.36	0\\
52.37	0\\
52.38	0\\
52.39	0\\
52.4	0\\
52.41	0\\
52.42	0\\
52.43	0\\
52.44	0\\
52.45	0\\
52.46	0\\
52.47	0\\
52.48	0\\
52.49	0\\
52.5	0\\
52.51	0\\
52.52	0\\
52.53	0\\
52.54	0\\
52.55	0\\
52.56	0\\
52.57	0\\
52.58	0\\
52.59	0\\
52.6	0\\
52.61	0\\
52.62	0\\
52.63	0\\
52.64	0\\
52.65	0\\
52.66	0\\
52.67	0\\
52.68	0\\
52.69	0\\
52.7	0\\
52.71	0\\
52.72	0\\
52.73	0\\
52.74	0\\
52.75	0\\
52.76	0\\
52.77	0\\
52.78	0\\
52.79	0\\
52.8	0\\
52.81	0\\
52.82	0\\
52.83	0\\
52.84	0\\
52.85	0\\
52.86	0\\
52.87	0\\
52.88	0\\
52.89	0\\
52.9	0\\
52.91	0\\
52.92	0\\
52.93	0\\
52.94	0\\
52.95	0\\
52.96	0\\
52.97	0\\
52.98	0\\
52.99	0\\
53	0\\
53.01	0\\
53.02	0\\
53.03	0\\
53.04	0\\
53.05	0\\
53.06	0\\
53.07	0\\
53.08	0\\
53.09	0\\
53.1	0\\
53.11	0\\
53.12	0\\
53.13	0\\
53.14	0\\
53.15	0\\
53.16	0\\
53.17	0\\
53.18	0\\
53.19	0\\
53.2	0\\
53.21	0\\
53.22	0\\
53.23	0\\
53.24	0\\
53.25	0\\
53.26	0\\
53.27	0\\
53.28	0\\
53.29	0\\
53.3	0\\
53.31	0\\
53.32	0\\
53.33	0\\
53.34	0\\
53.35	0\\
53.36	0\\
53.37	0\\
53.38	0\\
53.39	0\\
53.4	0\\
53.41	0\\
53.42	0\\
53.43	0\\
53.44	0\\
53.45	0\\
53.46	0\\
53.47	0\\
53.48	0\\
53.49	0\\
53.5	0\\
53.51	0\\
53.52	0\\
53.53	0\\
53.54	0\\
53.55	0\\
53.56	0\\
53.57	0\\
53.58	0\\
53.59	0\\
53.6	0\\
53.61	0\\
53.62	0\\
53.63	0\\
53.64	0\\
53.65	0\\
53.66	0\\
53.67	0\\
53.68	0\\
53.69	0\\
53.7	0\\
53.71	0\\
53.72	0\\
53.73	0\\
53.74	0\\
53.75	0\\
53.76	0\\
53.77	0\\
53.78	0\\
53.79	0\\
53.8	0\\
53.81	0\\
53.82	0\\
53.83	0\\
53.84	0\\
53.85	0\\
53.86	0\\
53.87	0\\
53.88	0\\
53.89	0\\
53.9	0\\
53.91	0\\
53.92	0\\
53.93	0\\
53.94	0\\
53.95	0\\
53.96	0\\
53.97	0\\
53.98	0\\
53.99	0\\
54	0\\
54.01	0\\
54.02	0\\
54.03	0\\
54.04	0\\
54.05	0\\
54.06	0\\
54.07	0\\
54.08	0\\
54.09	0\\
54.1	0\\
54.11	0\\
54.12	0\\
54.13	0\\
54.14	0\\
54.15	0\\
54.16	0\\
54.17	0\\
54.18	0\\
54.19	0\\
54.2	0\\
54.21	0\\
54.22	0\\
54.23	0\\
54.24	0\\
54.25	0\\
54.26	0\\
54.27	0\\
54.28	0\\
54.29	0\\
54.3	0\\
54.31	0\\
54.32	0\\
54.33	0\\
54.34	0\\
54.35	0\\
54.36	0\\
54.37	0\\
54.38	0\\
54.39	0\\
54.4	0\\
54.41	0\\
54.42	0\\
54.43	0\\
54.44	0\\
54.45	0\\
54.46	0\\
54.47	0\\
54.48	0\\
54.49	0\\
54.5	0\\
54.51	0\\
54.52	0\\
54.53	0\\
54.54	0\\
54.55	0\\
54.56	0\\
54.57	0\\
54.58	0\\
54.59	0\\
54.6	0\\
54.61	0\\
54.62	0\\
54.63	0\\
54.64	0\\
54.65	0\\
54.66	0\\
54.67	0\\
54.68	0\\
54.69	0\\
54.7	0\\
54.71	0\\
54.72	0\\
54.73	0\\
54.74	0\\
54.75	0\\
54.76	0\\
54.77	0\\
54.78	0\\
54.79	0\\
54.8	0\\
54.81	0\\
54.82	0\\
54.83	0\\
54.84	0\\
54.85	0\\
54.86	0\\
54.87	0\\
54.88	0\\
54.89	0\\
54.9	0\\
54.91	0\\
54.92	0\\
54.93	0\\
54.94	0\\
54.95	0\\
54.96	0\\
54.97	0\\
54.98	0\\
54.99	0\\
55	0\\
55.01	0\\
55.02	0\\
55.03	0\\
55.04	0\\
55.05	0\\
55.06	0\\
55.07	0\\
55.08	0\\
55.09	0\\
55.1	0\\
55.11	0\\
55.12	0\\
55.13	0\\
55.14	0\\
55.15	0\\
55.16	0\\
55.17	0\\
55.18	0\\
55.19	0\\
55.2	0\\
55.21	0\\
55.22	0\\
55.23	0\\
55.24	0\\
55.25	0\\
55.26	0\\
55.27	0\\
55.28	0\\
55.29	0\\
55.3	0\\
55.31	0\\
55.32	0\\
55.33	0\\
55.34	0\\
55.35	0\\
55.36	0\\
55.37	0\\
55.38	0\\
55.39	0\\
55.4	0\\
55.41	0\\
55.42	0\\
55.43	0\\
55.44	0\\
55.45	0\\
55.46	0\\
55.47	0\\
55.48	0\\
55.49	0\\
55.5	0\\
55.51	0\\
55.52	0\\
55.53	0\\
55.54	0\\
55.55	0\\
55.56	0\\
55.57	0\\
55.58	0\\
55.59	0\\
55.6	0\\
55.61	0\\
55.62	0\\
55.63	0\\
55.64	0\\
55.65	0\\
55.66	0\\
55.67	0\\
55.68	0\\
55.69	0\\
55.7	0\\
55.71	0\\
55.72	0\\
55.73	0\\
55.74	0\\
55.75	0\\
55.76	0\\
55.77	0\\
55.78	0\\
55.79	0\\
55.8	0\\
55.81	0\\
55.82	0\\
55.83	0\\
55.84	0\\
55.85	0\\
55.86	0\\
55.87	0\\
55.88	0\\
55.89	0\\
55.9	0\\
55.91	0\\
55.92	0\\
55.93	0\\
55.94	0\\
55.95	0\\
55.96	0\\
55.97	0\\
55.98	0\\
55.99	0\\
56	0\\
56.01	0\\
56.02	0\\
56.03	0\\
56.04	0\\
56.05	0\\
56.06	0\\
56.07	0\\
56.08	0\\
56.09	0\\
56.1	0\\
56.11	0\\
56.12	0\\
56.13	0\\
56.14	0\\
56.15	0\\
56.16	0\\
56.17	0\\
56.18	0\\
56.19	0\\
56.2	0\\
56.21	0\\
56.22	0\\
56.23	0\\
56.24	0\\
56.25	0\\
56.26	0\\
56.27	0\\
56.28	0\\
56.29	0\\
56.3	0\\
56.31	0\\
56.32	0\\
56.33	0\\
56.34	0\\
56.35	0\\
56.36	0\\
56.37	0\\
56.38	0\\
56.39	0\\
56.4	0\\
56.41	0\\
56.42	0\\
56.43	0\\
56.44	0\\
56.45	0\\
56.46	0\\
56.47	0\\
56.48	0\\
56.49	0\\
56.5	0\\
56.51	0\\
56.52	0\\
56.53	0\\
56.54	0\\
56.55	0\\
56.56	0\\
56.57	0\\
56.58	0\\
56.59	0\\
56.6	0\\
56.61	0\\
56.62	0\\
56.63	0\\
56.64	0\\
56.65	0\\
56.66	0\\
56.67	0\\
56.68	0\\
56.69	0\\
56.7	0\\
56.71	0\\
56.72	0\\
56.73	0\\
56.74	0\\
56.75	0\\
56.76	0\\
56.77	0\\
56.78	0\\
56.79	0\\
56.8	0\\
56.81	0\\
56.82	0\\
56.83	0\\
56.84	0\\
56.85	0\\
56.86	0\\
56.87	0\\
56.88	0\\
56.89	0\\
56.9	0\\
56.91	0\\
56.92	0\\
56.93	0\\
56.94	0\\
56.95	0\\
56.96	0\\
56.97	0\\
56.98	0\\
56.99	0\\
57	0\\
57.01	0\\
57.02	0\\
57.03	0\\
57.04	0\\
57.05	0\\
57.06	0\\
57.07	0\\
57.08	0\\
57.09	0\\
57.1	0\\
57.11	0\\
57.12	0\\
57.13	0\\
57.14	0\\
57.15	0\\
57.16	0\\
57.17	0\\
57.18	0\\
57.19	0\\
57.2	0\\
57.21	0\\
57.22	0\\
57.23	0\\
57.24	0\\
57.25	0\\
57.26	0\\
57.27	0\\
57.28	0\\
57.29	0\\
57.3	0\\
57.31	0\\
57.32	0\\
57.33	0\\
57.34	0\\
57.35	0\\
57.36	0\\
57.37	0\\
57.38	0\\
57.39	0\\
57.4	0\\
57.41	0\\
57.42	0\\
57.43	0\\
57.44	0\\
57.45	0\\
57.46	0\\
57.47	0\\
57.48	0\\
57.49	0\\
57.5	0\\
57.51	0\\
57.52	0\\
57.53	0\\
57.54	0\\
57.55	0\\
57.56	0\\
57.57	0\\
57.58	0\\
57.59	0\\
57.6	0\\
57.61	0\\
57.62	0\\
57.63	0\\
57.64	0\\
57.65	0\\
57.66	0\\
57.67	0\\
57.68	0\\
57.69	0\\
57.7	0\\
57.71	0\\
57.72	0\\
57.73	0\\
57.74	0\\
57.75	0\\
57.76	0\\
57.77	0\\
57.78	0\\
57.79	0\\
57.8	0\\
57.81	0\\
57.82	0\\
57.83	0\\
57.84	0\\
57.85	0\\
57.86	0\\
57.87	0\\
57.88	0\\
57.89	0\\
57.9	0\\
57.91	0\\
57.92	0\\
57.93	0\\
57.94	0\\
57.95	0\\
57.96	0\\
57.97	0\\
57.98	0\\
57.99	0\\
58	0\\
58.01	0\\
58.02	0\\
58.03	0\\
58.04	0\\
58.05	0\\
58.06	0\\
58.07	0\\
58.08	0\\
58.09	0\\
58.1	0\\
58.11	0\\
58.12	0\\
58.13	0\\
58.14	0\\
58.15	0\\
58.16	0\\
58.17	0\\
58.18	0\\
58.19	0\\
58.2	0\\
58.21	0\\
58.22	0\\
58.23	0\\
58.24	0\\
58.25	0\\
58.26	0\\
58.27	0\\
58.28	0\\
58.29	0\\
58.3	0\\
58.31	0\\
58.32	0\\
58.33	0\\
58.34	0\\
58.35	0\\
58.36	0\\
58.37	0\\
58.38	0\\
58.39	0\\
58.4	0\\
58.41	0\\
58.42	0\\
58.43	0\\
58.44	0\\
58.45	0\\
58.46	0\\
58.47	0\\
58.48	0\\
58.49	0\\
58.5	0\\
58.51	0\\
58.52	0\\
58.53	0\\
58.54	0\\
58.55	0\\
58.56	0\\
58.57	0\\
58.58	0\\
58.59	0\\
58.6	0\\
58.61	0\\
58.62	0\\
58.63	0\\
58.64	0\\
58.65	0\\
58.66	0\\
58.67	0\\
58.68	0\\
58.69	0\\
58.7	0\\
58.71	0\\
58.72	0\\
58.73	0\\
58.74	0\\
58.75	0\\
58.76	0\\
58.77	0\\
58.78	0\\
58.79	0\\
58.8	0\\
58.81	0\\
58.82	0\\
58.83	0\\
58.84	0\\
58.85	0\\
58.86	0\\
58.87	0\\
58.88	0\\
58.89	0\\
58.9	0\\
58.91	0\\
58.92	0\\
58.93	0\\
58.94	0\\
58.95	0\\
58.96	0\\
58.97	0\\
58.98	0\\
58.99	0\\
59	0\\
59.01	0\\
59.02	0\\
59.03	0\\
59.04	0\\
59.05	0\\
59.06	0\\
59.07	0\\
59.08	0\\
59.09	0\\
59.1	0\\
59.11	0\\
59.12	0\\
59.13	0\\
59.14	0\\
59.15	0\\
59.16	0\\
59.17	0\\
59.18	0\\
59.19	0\\
59.2	0\\
59.21	0\\
59.22	0\\
59.23	0\\
59.24	0\\
59.25	0\\
59.26	0\\
59.27	0\\
59.28	0\\
59.29	0\\
59.3	0\\
59.31	0\\
59.32	0\\
59.33	0\\
59.34	0\\
59.35	0\\
59.36	0\\
59.37	0\\
59.38	0\\
59.39	0\\
59.4	0\\
59.41	0\\
59.42	0\\
59.43	0\\
59.44	0\\
59.45	0\\
59.46	0\\
59.47	0\\
59.48	0\\
59.49	0\\
59.5	0\\
59.51	0\\
59.52	0\\
59.53	0\\
59.54	0\\
59.55	0\\
59.56	0\\
59.57	0\\
59.58	0\\
59.59	0\\
59.6	0\\
59.61	0\\
59.62	0\\
59.63	0\\
59.64	0\\
59.65	0\\
59.66	0\\
59.67	0\\
59.68	0\\
59.69	0\\
59.7	0\\
59.71	0\\
59.72	0\\
59.73	0\\
59.74	0\\
59.75	0\\
59.76	0\\
59.77	0\\
59.78	0\\
59.79	0\\
59.8	0\\
59.81	0\\
59.82	0\\
59.83	0\\
59.84	0\\
59.85	0\\
59.86	0\\
59.87	0\\
59.88	0\\
59.89	0\\
59.9	0\\
59.91	0\\
59.92	0\\
59.93	0\\
59.94	0\\
59.95	0\\
59.96	0\\
59.97	0\\
59.98	0\\
59.99	0\\
60	0\\
60.01	0\\
60.02	0\\
60.03	0\\
60.04	0\\
60.05	0\\
60.06	0\\
60.07	0\\
60.08	0\\
60.09	0\\
60.1	0\\
60.11	0\\
60.12	0\\
60.13	0\\
60.14	0\\
60.15	0\\
60.16	0\\
60.17	0\\
60.18	0\\
60.19	0\\
60.2	0\\
60.21	0\\
60.22	0\\
60.23	0\\
60.24	0\\
60.25	0\\
60.26	0\\
60.27	0\\
60.28	0\\
60.29	0\\
60.3	0\\
60.31	0\\
60.32	0\\
60.33	0\\
60.34	0\\
60.35	0\\
60.36	0\\
60.37	0\\
60.38	0\\
60.39	0\\
60.4	0\\
60.41	0\\
60.42	0\\
60.43	0\\
60.44	0\\
60.45	0\\
60.46	0\\
60.47	0\\
60.48	0\\
60.49	0\\
60.5	0\\
60.51	0\\
60.52	0\\
60.53	0\\
60.54	0\\
60.55	0\\
60.56	0\\
60.57	0\\
60.58	0\\
60.59	0\\
60.6	0\\
60.61	0\\
60.62	0\\
60.63	0\\
60.64	0\\
60.65	0\\
60.66	0\\
60.67	0\\
60.68	0\\
60.69	0\\
60.7	0\\
60.71	0\\
60.72	0\\
60.73	0\\
60.74	0\\
60.75	0\\
60.76	0\\
60.77	0\\
60.78	0\\
60.79	0\\
60.8	0\\
60.81	0\\
60.82	0\\
60.83	0\\
60.84	0\\
60.85	0\\
60.86	0\\
60.87	0\\
60.88	0\\
60.89	0\\
60.9	0\\
60.91	0\\
60.92	0\\
60.93	0\\
60.94	0\\
60.95	0\\
60.96	0\\
60.97	0\\
60.98	0\\
60.99	0\\
61	0\\
61.01	0\\
61.02	0\\
61.03	0\\
61.04	0\\
61.05	0\\
61.06	0\\
61.07	0\\
61.08	0\\
61.09	0\\
61.1	0\\
61.11	0\\
61.12	0\\
61.13	0\\
61.14	0\\
61.15	0\\
61.16	0\\
61.17	0\\
61.18	0\\
61.19	0\\
61.2	0\\
61.21	0\\
61.22	0\\
61.23	0\\
61.24	0\\
61.25	0\\
61.26	0\\
61.27	0\\
61.28	0\\
61.29	0\\
61.3	0\\
61.31	0\\
61.32	0\\
61.33	0\\
61.34	0\\
61.35	0\\
61.36	0\\
61.37	0\\
61.38	0\\
61.39	0\\
61.4	0\\
61.41	0\\
61.42	0\\
61.43	0\\
61.44	0\\
61.45	0\\
61.46	0\\
61.47	0\\
61.48	0\\
61.49	0\\
61.5	0\\
61.51	0\\
61.52	0\\
61.53	0\\
61.54	0\\
61.55	0\\
61.56	0\\
61.57	0\\
61.58	0\\
61.59	0\\
61.6	0\\
61.61	0\\
61.62	0\\
61.63	0\\
61.64	0\\
61.65	0\\
61.66	0\\
61.67	0\\
61.68	0\\
61.69	0\\
61.7	0\\
61.71	0\\
61.72	0\\
61.73	0\\
61.74	0\\
61.75	0\\
61.76	0\\
61.77	0\\
61.78	0\\
61.79	0\\
61.8	0\\
61.81	0\\
61.82	0\\
61.83	0\\
61.84	0\\
61.85	0\\
61.86	0\\
61.87	0\\
61.88	0\\
61.89	0\\
61.9	0\\
61.91	0\\
61.92	0\\
61.93	0\\
61.94	0\\
61.95	0\\
61.96	0\\
61.97	0\\
61.98	0\\
61.99	0\\
62	0\\
62.01	0\\
62.02	0\\
62.03	0\\
62.04	0\\
62.05	0\\
62.06	0\\
62.07	0\\
62.08	0\\
62.09	0\\
62.1	0\\
62.11	0\\
62.12	0\\
62.13	0\\
62.14	0\\
62.15	0\\
62.16	0\\
62.17	0\\
62.18	0\\
62.19	0\\
62.2	0\\
62.21	0\\
62.22	0\\
62.23	0\\
62.24	0\\
62.25	0\\
62.26	0\\
62.27	0\\
62.28	0\\
62.29	0\\
62.3	0\\
62.31	0\\
62.32	0\\
62.33	0\\
62.34	0\\
62.35	0\\
62.36	0\\
62.37	0\\
62.38	0\\
62.39	0\\
62.4	0\\
62.41	0\\
62.42	0\\
62.43	0\\
62.44	0\\
62.45	0\\
62.46	0\\
62.47	0\\
62.48	0\\
62.49	0\\
62.5	0\\
62.51	0\\
62.52	0\\
62.53	0\\
62.54	0\\
62.55	0\\
62.56	0\\
62.57	0\\
62.58	0\\
62.59	0\\
62.6	0\\
62.61	0\\
62.62	0\\
62.63	0\\
62.64	0\\
62.65	0\\
62.66	0\\
62.67	0\\
62.68	0\\
62.69	0\\
62.7	0\\
62.71	0\\
62.72	0\\
62.73	0\\
62.74	0\\
62.75	0\\
62.76	0\\
62.77	0\\
62.78	0\\
62.79	0\\
62.8	0\\
62.81	0\\
62.82	0\\
62.83	0\\
62.84	0\\
62.85	0\\
62.86	0\\
62.87	0\\
62.88	0\\
62.89	0\\
62.9	0\\
62.91	0\\
62.92	0\\
62.93	0\\
62.94	0\\
62.95	0\\
62.96	0\\
62.97	0\\
62.98	0\\
62.99	0\\
63	0\\
63.01	0\\
63.02	0\\
63.03	0\\
63.04	0\\
63.05	0\\
63.06	0\\
63.07	0\\
63.08	0\\
63.09	0\\
63.1	0\\
63.11	0\\
63.12	0\\
63.13	0\\
63.14	0\\
63.15	0\\
63.16	0\\
63.17	0\\
63.18	0\\
63.19	0\\
63.2	0\\
63.21	0\\
63.22	0\\
63.23	0\\
63.24	0\\
63.25	0\\
63.26	0\\
63.27	0\\
63.28	0\\
63.29	0\\
63.3	0\\
63.31	0\\
63.32	0\\
63.33	0\\
63.34	0\\
63.35	0\\
63.36	0\\
63.37	0\\
63.38	0\\
63.39	0\\
63.4	0\\
63.41	0\\
63.42	0\\
63.43	0\\
63.44	0\\
63.45	0\\
63.46	0\\
63.47	0\\
63.48	0\\
63.49	0\\
63.5	0\\
63.51	0\\
63.52	0\\
63.53	0\\
63.54	0\\
63.55	0\\
63.56	0\\
63.57	0\\
63.58	0\\
63.59	0\\
63.6	0\\
63.61	0\\
63.62	0\\
63.63	0\\
63.64	0\\
63.65	0\\
63.66	0\\
63.67	0\\
63.68	0\\
63.69	0\\
63.7	0\\
63.71	0\\
63.72	0\\
63.73	0\\
63.74	0\\
63.75	0\\
63.76	0\\
63.77	0\\
63.78	0\\
63.79	0\\
63.8	0\\
63.81	0\\
63.82	0\\
63.83	0\\
63.84	0\\
63.85	0\\
63.86	0\\
63.87	0\\
63.88	0\\
63.89	0\\
63.9	0\\
63.91	0\\
63.92	0\\
63.93	0\\
63.94	0\\
63.95	0\\
63.96	0\\
63.97	0\\
63.98	0\\
63.99	0\\
64	0\\
64.01	0\\
64.02	0\\
64.03	0\\
64.04	0\\
64.05	0\\
64.06	0\\
64.07	0\\
64.08	0\\
64.09	0\\
64.1	0\\
64.11	0\\
64.12	0\\
64.13	0\\
64.14	0\\
64.15	0\\
64.16	0\\
64.17	0\\
64.18	0\\
64.19	0\\
64.2	0\\
64.21	0\\
64.22	0\\
64.23	0\\
64.24	0\\
64.25	0\\
64.26	0\\
64.27	0\\
64.28	0\\
64.29	0\\
64.3	0\\
64.31	0\\
64.32	0\\
64.33	0\\
64.34	0\\
64.35	0\\
64.36	0\\
64.37	0\\
64.38	0\\
64.39	0\\
64.4	0\\
64.41	0\\
64.42	0\\
64.43	0\\
64.44	0\\
64.45	0\\
64.46	0\\
64.47	0\\
64.48	0\\
64.49	0\\
64.5	0\\
64.51	0\\
64.52	0\\
64.53	0\\
64.54	0\\
64.55	0\\
64.56	0\\
64.57	0\\
64.58	0\\
64.59	0\\
64.6	0\\
64.61	0\\
64.62	0\\
64.63	0\\
64.64	0\\
64.65	0\\
64.66	0\\
64.67	0\\
64.68	0\\
64.69	0\\
64.7	0\\
64.71	0\\
64.72	0\\
64.73	0\\
64.74	0\\
64.75	0\\
64.76	0\\
64.77	0\\
64.78	0\\
64.79	0\\
64.8	0\\
64.81	0\\
64.82	0\\
64.83	0\\
64.84	0\\
64.85	0\\
64.86	0\\
64.87	0\\
64.88	0\\
64.89	0\\
64.9	0\\
64.91	0\\
64.92	0\\
64.93	0\\
64.94	0\\
64.95	0\\
64.96	0\\
64.97	0\\
64.98	0\\
64.99	0\\
65	0\\
65.01	0\\
65.02	0\\
65.03	0\\
65.04	0\\
65.05	0\\
65.06	0\\
65.07	0\\
65.08	0\\
65.09	0\\
65.1	0\\
65.11	0\\
65.12	0\\
65.13	0\\
65.14	0\\
65.15	0\\
65.16	0\\
65.17	0\\
65.18	0\\
65.19	0\\
65.2	0\\
65.21	0\\
65.22	0\\
65.23	0\\
65.24	0\\
65.25	0\\
65.26	0\\
65.27	0\\
65.28	0\\
65.29	0\\
65.3	0\\
65.31	0\\
65.32	0\\
65.33	0\\
65.34	0\\
65.35	0\\
65.36	0\\
65.37	0\\
65.38	0\\
65.39	0\\
65.4	0\\
65.41	0\\
65.42	0\\
65.43	0\\
65.44	0\\
65.45	0\\
65.46	0\\
65.47	0\\
65.48	0\\
65.49	0\\
65.5	0\\
65.51	0\\
65.52	0\\
65.53	0\\
65.54	0\\
65.55	0\\
65.56	0\\
65.57	0\\
65.58	0\\
65.59	0\\
65.6	0\\
65.61	0\\
65.62	0\\
65.63	0\\
65.64	0\\
65.65	0\\
65.66	0\\
65.67	0\\
65.68	0\\
65.69	0\\
65.7	0\\
65.71	0\\
65.72	0\\
65.73	0\\
65.74	0\\
65.75	0\\
65.76	0\\
65.77	0\\
65.78	0\\
65.79	0\\
65.8	0\\
65.81	0\\
65.82	0\\
65.83	0\\
65.84	0\\
65.85	0\\
65.86	0\\
65.87	0\\
65.88	0\\
65.89	0\\
65.9	0\\
65.91	0\\
65.92	0\\
65.93	0\\
65.94	0\\
65.95	0\\
65.96	0\\
65.97	0\\
65.98	0\\
65.99	0\\
66	0\\
66.01	0\\
66.02	0\\
66.03	0\\
66.04	0\\
66.05	0\\
66.06	0\\
66.07	0\\
66.08	0\\
66.09	0\\
66.1	0\\
66.11	0\\
66.12	0\\
66.13	0\\
66.14	0\\
66.15	0\\
66.16	0\\
66.17	0\\
66.18	0\\
66.19	0\\
66.2	0\\
66.21	0\\
66.22	0\\
66.23	0\\
66.24	0\\
66.25	0\\
66.26	0\\
66.27	0\\
66.28	0\\
66.29	0\\
66.3	0\\
66.31	0\\
66.32	0\\
66.33	0\\
66.34	0\\
66.35	0\\
66.36	0\\
66.37	0\\
66.38	0\\
66.39	0\\
66.4	0\\
66.41	0\\
66.42	0\\
66.43	0\\
66.44	0\\
66.45	0\\
66.46	0\\
66.47	0\\
66.48	0\\
66.49	0\\
66.5	0\\
66.51	0\\
66.52	0\\
66.53	0\\
66.54	0\\
66.55	0\\
66.56	0\\
66.57	0\\
66.58	0\\
66.59	0\\
66.6	0\\
66.61	0\\
66.62	0\\
66.63	0\\
66.64	0\\
66.65	0\\
66.66	0\\
66.67	0\\
66.68	0\\
66.69	0\\
66.7	0\\
66.71	0\\
66.72	0\\
66.73	0\\
66.74	0\\
66.75	0\\
66.76	0\\
66.77	0\\
66.78	0\\
66.79	0\\
66.8	0\\
66.81	0\\
66.82	0\\
66.83	0\\
66.84	0\\
66.85	0\\
66.86	0\\
66.87	0\\
66.88	0\\
66.89	0\\
66.9	0\\
66.91	0\\
66.92	0\\
66.93	0\\
66.94	0\\
66.95	0\\
66.96	0\\
66.97	0\\
66.98	0\\
66.99	0\\
67	0\\
67.01	0\\
67.02	0\\
67.03	0\\
67.04	0\\
67.05	0\\
67.06	0\\
67.07	0\\
67.08	0\\
67.09	0\\
67.1	0\\
67.11	0\\
67.12	0\\
67.13	0\\
67.14	0\\
67.15	0\\
67.16	0\\
67.17	0\\
67.18	0\\
67.19	0\\
67.2	0\\
67.21	0\\
67.22	0\\
67.23	0\\
67.24	0\\
67.25	0\\
67.26	0\\
67.27	0\\
67.28	0\\
67.29	0\\
67.3	0\\
67.31	0\\
67.32	0\\
67.33	0\\
67.34	0\\
67.35	0\\
67.36	0\\
67.37	0\\
67.38	0\\
67.39	0\\
67.4	0\\
67.41	0\\
67.42	0\\
67.43	0\\
67.44	0\\
67.45	0\\
67.46	0\\
67.47	0\\
67.48	0\\
67.49	0\\
67.5	0\\
67.51	0\\
67.52	0\\
67.53	0\\
67.54	0\\
67.55	0\\
67.56	0\\
67.57	0\\
67.58	0\\
67.59	0\\
67.6	0\\
67.61	0\\
67.62	0\\
67.63	0\\
67.64	0\\
67.65	0\\
67.66	0\\
67.67	0\\
67.68	0\\
67.69	0\\
67.7	0\\
67.71	0\\
67.72	0\\
67.73	0\\
67.74	0\\
67.75	0\\
67.76	0\\
67.77	0\\
67.78	0\\
67.79	0\\
67.8	0\\
67.81	0\\
67.82	0\\
67.83	0\\
67.84	0\\
67.85	0\\
67.86	0\\
67.87	0\\
67.88	0\\
67.89	0\\
67.9	0\\
67.91	0\\
67.92	0\\
67.93	0\\
67.94	0\\
67.95	0\\
67.96	0\\
67.97	0\\
67.98	0\\
67.99	0\\
68	0\\
68.01	0\\
68.02	0\\
68.03	0\\
68.04	0\\
68.05	0\\
68.06	0\\
68.07	0\\
68.08	0\\
68.09	0\\
68.1	0\\
68.11	0\\
68.12	0\\
68.13	0\\
68.14	0\\
68.15	0\\
68.16	0\\
68.17	0\\
68.18	0\\
68.19	0\\
68.2	0\\
68.21	0\\
68.22	0\\
68.23	0\\
68.24	0\\
68.25	0\\
68.26	0\\
68.27	0\\
68.28	0\\
68.29	0\\
68.3	0\\
68.31	0\\
68.32	0\\
68.33	0\\
68.34	0\\
68.35	0\\
68.36	0\\
68.37	0\\
68.38	0\\
68.39	0\\
68.4	0\\
68.41	0\\
68.42	0\\
68.43	0\\
68.44	0\\
68.45	0\\
68.46	0\\
68.47	0\\
68.48	0\\
68.49	0\\
68.5	0\\
68.51	0\\
68.52	0\\
68.53	0\\
68.54	0\\
68.55	0\\
68.56	0\\
68.57	0\\
68.58	0\\
68.59	0\\
68.6	0\\
68.61	0\\
68.62	0\\
68.63	0\\
68.64	0\\
68.65	0\\
68.66	0\\
68.67	0\\
68.68	0\\
68.69	0\\
68.7	0\\
68.71	0\\
68.72	0\\
68.73	0\\
68.74	0\\
68.75	0\\
68.76	0\\
68.77	0\\
68.78	0\\
68.79	0\\
68.8	0\\
68.81	0\\
68.82	0\\
68.83	0\\
68.84	0\\
68.85	0\\
68.86	0\\
68.87	0\\
68.88	0\\
68.89	0\\
68.9	0\\
68.91	0\\
68.92	0\\
68.93	0\\
68.94	0\\
68.95	0\\
68.96	0\\
68.97	0\\
68.98	0\\
68.99	0\\
69	0\\
69.01	0\\
69.02	0\\
69.03	0\\
69.04	0\\
69.05	0\\
69.06	0\\
69.07	0\\
69.08	0\\
69.09	0\\
69.1	0\\
69.11	0\\
69.12	0\\
69.13	0\\
69.14	0\\
69.15	0\\
69.16	0\\
69.17	0\\
69.18	0\\
69.19	0\\
69.2	0\\
69.21	0\\
69.22	0\\
69.23	0\\
69.24	0\\
69.25	0\\
69.26	0\\
69.27	0\\
69.28	0\\
69.29	0\\
69.3	0\\
69.31	0\\
69.32	0\\
69.33	0\\
69.34	0\\
69.35	0\\
69.36	0\\
69.37	0\\
69.38	0\\
69.39	0\\
69.4	0\\
69.41	0\\
69.42	0\\
69.43	0\\
69.44	0\\
69.45	0\\
69.46	0\\
69.47	0\\
69.48	0\\
69.49	0\\
69.5	0\\
69.51	0\\
69.52	0\\
69.53	0\\
69.54	0\\
69.55	0\\
69.56	0\\
69.57	0\\
69.58	0\\
69.59	0\\
69.6	0\\
69.61	0\\
69.62	0\\
69.63	0\\
69.64	0\\
69.65	0\\
69.66	0\\
69.67	0\\
69.68	0\\
69.69	0\\
69.7	0\\
69.71	0\\
69.72	0\\
69.73	0\\
69.74	0\\
69.75	0\\
69.76	0\\
69.77	0\\
69.78	0\\
69.79	0\\
69.8	0\\
69.81	0\\
69.82	0\\
69.83	0\\
69.84	0\\
69.85	0\\
69.86	0\\
69.87	0\\
69.88	0\\
69.89	0\\
69.9	0\\
69.91	0\\
69.92	0\\
69.93	0\\
69.94	0\\
69.95	0\\
69.96	0\\
69.97	0\\
69.98	0\\
69.99	0\\
70	0\\
70.01	0\\
70.02	0\\
70.03	0\\
70.04	0\\
70.05	0\\
70.06	0\\
70.07	0\\
70.08	0\\
70.09	0\\
70.1	0\\
70.11	0\\
70.12	0\\
70.13	0\\
70.14	0\\
70.15	0\\
70.16	0\\
70.17	0\\
70.18	0\\
70.19	0\\
70.2	0\\
70.21	0\\
70.22	0\\
70.23	0\\
70.24	0\\
70.25	0\\
70.26	0\\
70.27	0\\
70.28	0\\
70.29	0\\
70.3	0\\
70.31	0\\
70.32	0\\
70.33	0\\
70.34	0\\
70.35	0\\
70.36	0\\
70.37	0\\
70.38	0\\
70.39	0\\
70.4	0\\
70.41	0\\
70.42	0\\
70.43	0\\
70.44	0\\
70.45	0\\
70.46	0\\
70.47	0\\
70.48	0\\
70.49	0\\
70.5	0\\
70.51	0\\
70.52	0\\
70.53	0\\
70.54	0\\
70.55	0\\
70.56	0\\
70.57	0\\
70.58	0\\
70.59	0\\
70.6	0\\
70.61	0\\
70.62	0\\
70.63	0\\
70.64	0\\
70.65	0\\
70.66	0\\
70.67	0\\
70.68	0\\
70.69	0\\
70.7	0\\
70.71	0\\
70.72	0\\
70.73	0\\
70.74	0\\
70.75	0\\
70.76	0\\
70.77	0\\
70.78	0\\
70.79	0\\
70.8	0\\
70.81	0\\
70.82	0\\
70.83	0\\
70.84	0\\
70.85	0\\
70.86	0\\
70.87	0\\
70.88	0\\
70.89	0\\
70.9	0\\
70.91	0\\
70.92	0\\
70.93	0\\
70.94	0\\
70.95	0\\
70.96	0\\
70.97	0\\
70.98	0\\
70.99	0\\
71	0\\
71.01	0\\
71.02	0\\
71.03	0\\
71.04	0\\
71.05	0\\
71.06	0\\
71.07	0\\
71.08	0\\
71.09	0\\
71.1	0\\
71.11	0\\
71.12	0\\
71.13	0\\
71.14	0\\
71.15	0\\
71.16	0\\
71.17	0\\
71.18	0\\
71.19	0\\
71.2	0\\
71.21	0\\
71.22	0\\
71.23	0\\
71.24	0\\
71.25	0\\
71.26	0\\
71.27	0\\
71.28	0\\
71.29	0\\
71.3	0\\
71.31	0\\
71.32	0\\
71.33	0\\
71.34	0\\
71.35	0\\
71.36	0\\
71.37	0\\
71.38	0\\
71.39	0\\
71.4	0\\
71.41	0\\
71.42	0\\
71.43	0\\
71.44	0\\
71.45	0\\
71.46	0\\
71.47	0\\
71.48	0\\
71.49	0\\
71.5	0\\
71.51	0\\
71.52	0\\
71.53	0\\
71.54	0\\
71.55	0\\
71.56	0\\
71.57	0\\
71.58	0\\
71.59	0\\
71.6	0\\
71.61	0\\
71.62	0\\
71.63	0\\
71.64	0\\
71.65	0\\
71.66	0\\
71.67	0\\
71.68	0\\
71.69	0\\
71.7	0\\
71.71	0\\
71.72	0\\
71.73	0\\
71.74	0\\
71.75	0\\
71.76	0\\
71.77	0\\
71.78	0\\
71.79	0\\
71.8	0\\
71.81	0\\
71.82	0\\
71.83	0\\
71.84	0\\
71.85	0\\
71.86	0\\
71.87	0\\
71.88	0\\
71.89	0\\
71.9	0\\
71.91	0\\
71.92	0\\
71.93	0\\
71.94	0\\
71.95	0\\
71.96	0\\
71.97	0\\
71.98	0\\
71.99	0\\
72	0\\
72.01	0\\
72.02	0\\
72.03	0\\
72.04	0\\
72.05	0\\
72.06	0\\
72.07	0\\
72.08	0\\
72.09	0\\
72.1	0\\
72.11	0\\
72.12	0\\
72.13	0\\
72.14	0\\
72.15	0\\
72.16	0\\
72.17	0\\
72.18	0\\
72.19	0\\
72.2	0\\
72.21	0\\
72.22	0\\
72.23	0\\
72.24	0\\
72.25	0\\
72.26	0\\
72.27	0\\
72.28	0\\
72.29	0\\
72.3	0\\
72.31	0\\
72.32	0\\
72.33	0\\
72.34	0\\
72.35	0\\
72.36	0\\
72.37	0\\
72.38	0\\
72.39	0\\
72.4	0\\
72.41	0\\
72.42	0\\
72.43	0\\
72.44	0\\
72.45	0\\
72.46	0\\
72.47	0\\
72.48	0\\
72.49	0\\
72.5	0\\
72.51	0\\
72.52	0\\
72.53	0\\
72.54	0\\
72.55	0\\
72.56	0\\
72.57	0\\
72.58	0\\
72.59	0\\
72.6	0\\
72.61	0\\
72.62	0\\
72.63	0\\
72.64	0\\
72.65	0\\
72.66	0\\
72.67	0\\
72.68	0\\
72.69	0\\
72.7	0\\
72.71	0\\
72.72	0\\
72.73	0\\
72.74	0\\
72.75	0\\
72.76	0\\
72.77	0\\
72.78	0\\
72.79	0\\
72.8	0\\
72.81	0\\
72.82	0\\
72.83	0\\
72.84	0\\
72.85	0\\
72.86	0\\
72.87	0\\
72.88	0\\
72.89	0\\
72.9	0\\
72.91	0\\
72.92	0\\
72.93	0\\
72.94	0\\
72.95	0\\
72.96	0\\
72.97	0\\
72.98	0\\
72.99	0\\
73	0\\
73.01	0\\
73.02	0\\
73.03	0\\
73.04	0\\
73.05	0\\
73.06	0\\
73.07	0\\
73.08	0\\
73.09	0\\
73.1	0\\
73.11	0\\
73.12	0\\
73.13	0\\
73.14	0\\
73.15	0\\
73.16	0\\
73.17	0\\
73.18	0\\
73.19	0\\
73.2	0\\
73.21	0\\
73.22	0\\
73.23	0\\
73.24	0\\
73.25	0\\
73.26	0\\
73.27	0\\
73.28	0\\
73.29	0\\
73.3	0\\
73.31	0\\
73.32	0\\
73.33	0\\
73.34	0\\
73.35	0\\
73.36	0\\
73.37	0\\
73.38	0\\
73.39	0\\
73.4	0\\
73.41	0\\
73.42	0\\
73.43	0\\
73.44	0\\
73.45	0\\
73.46	0\\
73.47	0\\
73.48	0\\
73.49	0\\
73.5	0\\
73.51	0\\
73.52	0\\
73.53	0\\
73.54	0\\
73.55	0\\
73.56	0\\
73.57	0\\
73.58	0\\
73.59	0\\
73.6	0\\
73.61	0\\
73.62	0\\
73.63	0\\
73.64	0\\
73.65	0\\
73.66	0\\
73.67	0\\
73.68	0\\
73.69	0\\
73.7	0\\
73.71	0\\
73.72	0\\
73.73	0\\
73.74	0\\
73.75	0\\
73.76	0\\
73.77	0\\
73.78	0\\
73.79	0\\
73.8	0\\
73.81	0\\
73.82	0\\
73.83	0\\
73.84	0\\
73.85	0\\
73.86	0\\
73.87	0\\
73.88	0\\
73.89	0\\
73.9	0\\
73.91	0\\
73.92	0\\
73.93	0\\
73.94	0\\
73.95	0\\
73.96	0\\
73.97	0\\
73.98	0\\
73.99	0\\
74	0\\
74.01	0\\
74.02	0\\
74.03	0\\
74.04	0\\
74.05	0\\
74.06	0\\
74.07	0\\
74.08	0\\
74.09	0\\
74.1	0\\
74.11	0\\
74.12	0\\
74.13	0\\
74.14	0\\
74.15	0\\
74.16	0\\
74.17	0\\
74.18	0\\
74.19	0\\
74.2	0\\
74.21	0\\
74.22	0\\
74.23	0\\
74.24	0\\
74.25	0\\
74.26	0\\
74.27	0\\
74.28	0\\
74.29	0\\
74.3	0\\
74.31	0\\
74.32	0\\
74.33	0\\
74.34	0\\
74.35	0\\
74.36	0\\
74.37	0\\
74.38	0\\
74.39	0\\
74.4	0\\
74.41	0\\
74.42	0\\
74.43	0\\
74.44	0\\
74.45	0\\
74.46	0\\
74.47	0\\
74.48	0\\
74.49	0\\
74.5	0\\
74.51	0\\
74.52	0\\
74.53	0\\
74.54	0\\
74.55	0\\
74.56	0\\
74.57	0\\
74.58	0\\
74.59	0\\
74.6	0\\
74.61	0\\
74.62	0\\
74.63	0\\
74.64	0\\
74.65	0\\
74.66	0\\
74.67	0\\
74.68	0\\
74.69	0\\
74.7	0\\
74.71	0\\
74.72	0\\
74.73	0\\
74.74	0\\
74.75	0\\
74.76	0\\
74.77	0\\
74.78	0\\
74.79	0\\
74.8	0\\
74.81	0\\
74.82	0\\
74.83	0\\
74.84	0\\
74.85	0\\
74.86	0\\
74.87	0\\
74.88	0\\
74.89	0\\
74.9	0\\
74.91	0\\
74.92	0\\
74.93	0\\
74.94	0\\
74.95	0\\
74.96	0\\
74.97	0\\
74.98	0\\
74.99	0\\
75	0\\
75.01	0\\
75.02	0\\
75.03	0\\
75.04	0\\
75.05	0\\
75.06	0\\
75.07	0\\
75.08	0\\
75.09	0\\
75.1	0\\
75.11	0\\
75.12	0\\
75.13	0\\
75.14	0\\
75.15	0\\
75.16	0\\
75.17	0\\
75.18	0\\
75.19	0\\
75.2	0\\
75.21	0\\
75.22	0\\
75.23	0\\
75.24	0\\
75.25	0\\
75.26	0\\
75.27	0\\
75.28	0\\
75.29	0\\
75.3	0\\
75.31	0\\
75.32	0\\
75.33	0\\
75.34	0\\
75.35	0\\
75.36	0\\
75.37	0\\
75.38	0\\
75.39	0\\
75.4	0\\
75.41	0\\
75.42	0\\
75.43	0\\
75.44	0\\
75.45	0\\
75.46	0\\
75.47	0\\
75.48	0\\
75.49	0\\
75.5	0\\
75.51	0\\
75.52	0\\
75.53	0\\
75.54	0\\
75.55	0\\
75.56	0\\
75.57	0\\
75.58	0\\
75.59	0\\
75.6	0\\
75.61	0\\
75.62	0\\
75.63	0\\
75.64	0\\
75.65	0\\
75.66	0\\
75.67	0\\
75.68	0\\
75.69	0\\
75.7	0\\
75.71	0\\
75.72	0\\
75.73	0\\
75.74	0\\
75.75	0\\
75.76	0\\
75.77	0\\
75.78	0\\
75.79	0\\
75.8	0\\
75.81	0\\
75.82	0\\
75.83	0\\
75.84	0\\
75.85	0\\
75.86	0\\
75.87	0\\
75.88	0\\
75.89	0\\
75.9	0\\
75.91	0\\
75.92	0\\
75.93	0\\
75.94	0\\
75.95	0\\
75.96	0\\
75.97	0\\
75.98	0\\
75.99	0\\
76	0\\
76.01	0\\
76.02	0\\
76.03	0\\
76.04	0\\
76.05	0\\
76.06	0\\
76.07	0\\
76.08	0\\
76.09	0\\
76.1	0\\
76.11	0\\
76.12	0\\
76.13	0\\
76.14	0\\
76.15	0\\
76.16	0\\
76.17	0\\
76.18	0\\
76.19	0\\
76.2	0\\
76.21	0\\
76.22	0\\
76.23	0\\
76.24	0\\
76.25	0\\
76.26	0\\
76.27	0\\
76.28	0\\
76.29	0\\
76.3	0\\
76.31	0\\
76.32	0\\
76.33	0\\
76.34	0\\
76.35	0\\
76.36	0\\
76.37	0\\
76.38	0\\
76.39	0\\
76.4	0\\
76.41	0\\
76.42	0\\
76.43	0\\
76.44	0\\
76.45	0\\
76.46	0\\
76.47	0\\
76.48	0\\
76.49	0\\
76.5	0\\
76.51	0\\
76.52	0\\
76.53	0\\
76.54	0\\
76.55	0\\
76.56	0\\
76.57	0\\
76.58	0\\
76.59	0\\
76.6	0\\
76.61	0\\
76.62	0\\
76.63	0\\
76.64	0\\
76.65	0\\
76.66	0\\
76.67	0\\
76.68	0\\
76.69	0\\
76.7	0\\
76.71	0\\
76.72	0\\
76.73	0\\
76.74	0\\
76.75	0\\
76.76	0\\
76.77	0\\
76.78	0\\
76.79	0\\
76.8	0\\
76.81	0\\
76.82	0\\
76.83	0\\
76.84	0\\
76.85	0\\
76.86	0\\
76.87	0\\
76.88	0\\
76.89	0\\
76.9	0\\
76.91	0\\
76.92	0\\
76.93	0\\
76.94	0\\
76.95	0\\
76.96	0\\
76.97	0\\
76.98	0\\
76.99	0\\
77	0\\
77.01	0\\
77.02	0\\
77.03	0\\
77.04	0\\
77.05	0\\
77.06	0\\
77.07	0\\
77.08	0\\
77.09	0\\
77.1	0\\
77.11	0\\
77.12	0\\
77.13	0\\
77.14	0\\
77.15	0\\
77.16	0\\
77.17	0\\
77.18	0\\
77.19	0\\
77.2	0\\
77.21	0\\
77.22	0\\
77.23	0\\
77.24	0\\
77.25	0\\
77.26	0\\
77.27	0\\
77.28	0\\
77.29	0\\
77.3	0\\
77.31	0\\
77.32	0\\
77.33	0\\
77.34	0\\
77.35	0\\
77.36	0\\
77.37	0\\
77.38	0\\
77.39	0\\
77.4	0\\
77.41	0\\
77.42	0\\
77.43	0\\
77.44	0\\
77.45	0\\
77.46	0\\
77.47	0\\
77.48	0\\
77.49	0\\
77.5	0\\
77.51	0\\
77.52	0\\
77.53	0\\
77.54	0\\
77.55	0\\
77.56	0\\
77.57	0\\
77.58	0\\
77.59	0\\
77.6	0\\
77.61	0\\
77.62	0\\
77.63	0\\
77.64	0\\
77.65	0\\
77.66	0\\
77.67	0\\
77.68	0\\
77.69	0\\
77.7	0\\
77.71	0\\
77.72	0\\
77.73	0\\
77.74	0\\
77.75	0\\
77.76	0\\
77.77	0\\
77.78	0\\
77.79	0\\
77.8	0\\
77.81	0\\
77.82	0\\
77.83	0\\
77.84	0\\
77.85	0\\
77.86	0\\
77.87	0\\
77.88	0\\
77.89	0\\
77.9	0\\
77.91	0\\
77.92	0\\
77.93	0\\
77.94	0\\
77.95	0\\
77.96	0\\
77.97	0\\
77.98	0\\
77.99	0\\
78	0\\
78.01	0\\
78.02	0\\
78.03	0\\
78.04	0\\
78.05	0\\
78.06	0\\
78.07	0\\
78.08	0\\
78.09	0\\
78.1	0\\
78.11	0\\
78.12	0\\
78.13	0\\
78.14	0\\
78.15	0\\
78.16	0\\
78.17	0\\
78.18	0\\
78.19	0\\
78.2	0\\
78.21	0\\
78.22	0\\
78.23	0\\
78.24	0\\
78.25	0\\
78.26	0\\
78.27	0\\
78.28	0\\
78.29	0\\
78.3	0\\
78.31	0\\
78.32	0\\
78.33	0\\
78.34	0\\
78.35	0\\
78.36	0\\
78.37	0\\
78.38	0\\
78.39	0\\
78.4	0\\
78.41	0\\
78.42	0\\
78.43	0\\
78.44	0\\
78.45	0\\
78.46	0\\
78.47	0\\
78.48	0\\
78.49	0\\
78.5	0\\
78.51	0\\
78.52	0\\
78.53	0\\
78.54	0\\
78.55	0\\
78.56	0\\
78.57	0\\
78.58	0\\
78.59	0\\
78.6	0\\
78.61	0\\
78.62	0\\
78.63	0\\
78.64	0\\
78.65	0\\
78.66	0\\
78.67	0\\
78.68	0\\
78.69	0\\
78.7	0\\
78.71	0\\
78.72	0\\
78.73	0\\
78.74	0\\
78.75	0\\
78.76	0\\
78.77	0\\
78.78	0\\
78.79	0\\
78.8	0\\
78.81	0\\
78.82	0\\
78.83	0\\
78.84	0\\
78.85	0\\
78.86	0\\
78.87	0\\
78.88	0\\
78.89	0\\
78.9	0\\
78.91	0\\
78.92	0\\
78.93	0\\
78.94	0\\
78.95	0\\
78.96	0\\
78.97	0\\
78.98	0\\
78.99	0\\
79	0\\
79.01	0\\
79.02	0\\
79.03	0\\
79.04	0\\
79.05	0\\
79.06	0\\
79.07	0\\
79.08	0\\
79.09	0\\
79.1	0\\
79.11	0\\
79.12	0\\
79.13	0\\
79.14	0\\
79.15	0\\
79.16	0\\
79.17	0\\
79.18	0\\
79.19	0\\
79.2	0\\
79.21	0\\
79.22	0\\
79.23	0\\
79.24	0\\
79.25	0\\
79.26	0\\
79.27	0\\
79.28	0\\
79.29	0\\
79.3	0\\
79.31	0\\
79.32	0\\
79.33	0\\
79.34	0\\
79.35	0\\
79.36	0\\
79.37	0\\
79.38	0\\
79.39	0\\
79.4	0\\
79.41	0\\
79.42	0\\
79.43	0\\
79.44	0\\
79.45	0\\
79.46	0\\
79.47	0\\
79.48	0\\
79.49	0\\
79.5	0\\
79.51	0\\
79.52	0\\
79.53	0\\
79.54	0\\
79.55	0\\
79.56	0\\
79.57	0\\
79.58	0\\
79.59	0\\
79.6	0\\
79.61	0\\
79.62	0\\
79.63	0\\
79.64	0\\
79.65	0\\
79.66	0\\
79.67	0\\
79.68	0\\
79.69	0\\
79.7	0\\
79.71	0\\
79.72	0\\
79.73	0\\
79.74	0\\
79.75	0\\
79.76	0\\
79.77	0\\
79.78	0\\
79.79	0\\
79.8	0\\
79.81	0\\
79.82	0\\
79.83	0\\
79.84	0\\
79.85	0\\
79.86	0\\
79.87	0\\
79.88	0\\
79.89	0\\
79.9	0\\
79.91	0\\
79.92	0\\
79.93	0\\
79.94	0\\
79.95	0\\
79.96	0\\
79.97	0\\
79.98	0\\
79.99	0\\
80	0\\
80.01	0\\
};
\addplot [color=green,dashed]
  table[row sep=crcr]{%
80.01	0\\
80.02	0\\
80.03	0\\
80.04	0\\
80.05	0\\
80.06	0\\
80.07	0\\
80.08	0\\
80.09	0\\
80.1	0\\
80.11	0\\
80.12	0\\
80.13	0\\
80.14	0\\
80.15	0\\
80.16	0\\
80.17	0\\
80.18	0\\
80.19	0\\
80.2	0\\
80.21	0\\
80.22	0\\
80.23	0\\
80.24	0\\
80.25	0\\
80.26	0\\
80.27	0\\
80.28	0\\
80.29	0\\
80.3	0\\
80.31	0\\
80.32	0\\
80.33	0\\
80.34	0\\
80.35	0\\
80.36	0\\
80.37	0\\
80.38	0\\
80.39	0\\
80.4	0\\
80.41	0\\
80.42	0\\
80.43	0\\
80.44	0\\
80.45	0\\
80.46	0\\
80.47	0\\
80.48	0\\
80.49	0\\
80.5	0\\
80.51	0\\
80.52	0\\
80.53	0\\
80.54	0\\
80.55	0\\
80.56	0\\
80.57	0\\
80.58	0\\
80.59	0\\
80.6	0\\
80.61	0\\
80.62	0\\
80.63	0\\
80.64	0\\
80.65	0\\
80.66	0\\
80.67	0\\
80.68	0\\
80.69	0\\
80.7	0\\
80.71	0\\
80.72	0\\
80.73	0\\
80.74	0\\
80.75	0\\
80.76	0\\
80.77	0\\
80.78	0\\
80.79	0\\
80.8	0\\
80.81	0\\
80.82	0\\
80.83	0\\
80.84	0\\
80.85	0\\
80.86	0\\
80.87	0\\
80.88	0\\
80.89	0\\
80.9	0\\
80.91	0\\
80.92	0\\
80.93	0\\
80.94	0\\
80.95	0\\
80.96	0\\
80.97	0\\
80.98	0\\
80.99	0\\
81	0\\
81.01	0\\
81.02	0\\
81.03	0\\
81.04	0\\
81.05	0\\
81.06	0\\
81.07	0\\
81.08	0\\
81.09	0\\
81.1	0\\
81.11	0\\
81.12	0\\
81.13	0\\
81.14	0\\
81.15	0\\
81.16	0\\
81.17	0\\
81.18	0\\
81.19	0\\
81.2	0\\
81.21	0\\
81.22	0\\
81.23	0\\
81.24	0\\
81.25	0\\
81.26	0\\
81.27	0\\
81.28	0\\
81.29	0\\
81.3	0\\
81.31	0\\
81.32	0\\
81.33	0\\
81.34	0\\
81.35	0\\
81.36	0\\
81.37	0\\
81.38	0\\
81.39	0\\
81.4	0\\
81.41	0\\
81.42	0\\
81.43	0\\
81.44	0\\
81.45	0\\
81.46	0\\
81.47	0\\
81.48	0\\
81.49	0\\
81.5	0\\
81.51	0\\
81.52	0\\
81.53	0\\
81.54	0\\
81.55	0\\
81.56	0\\
81.57	0\\
81.58	0\\
81.59	0\\
81.6	0\\
81.61	0\\
81.62	0\\
81.63	0\\
81.64	0\\
81.65	0\\
81.66	0\\
81.67	0\\
81.68	0\\
81.69	0\\
81.7	0\\
81.71	0\\
81.72	0\\
81.73	0\\
81.74	0\\
81.75	0\\
81.76	0\\
81.77	0\\
81.78	0\\
81.79	0\\
81.8	0\\
81.81	0\\
81.82	0\\
81.83	0\\
81.84	0\\
81.85	0\\
81.86	0\\
81.87	0\\
81.88	0\\
81.89	0\\
81.9	0\\
81.91	0\\
81.92	0\\
81.93	0\\
81.94	0\\
81.95	0\\
81.96	0\\
81.97	0\\
81.98	0\\
81.99	0\\
82	0\\
82.01	0\\
82.02	0\\
82.03	0\\
82.04	0\\
82.05	0\\
82.06	0\\
82.07	0\\
82.08	0\\
82.09	0\\
82.1	0\\
82.11	0\\
82.12	0\\
82.13	0\\
82.14	0\\
82.15	0\\
82.16	0\\
82.17	0\\
82.18	0\\
82.19	0\\
82.2	0\\
82.21	0\\
82.22	0\\
82.23	0\\
82.24	0\\
82.25	0\\
82.26	0\\
82.27	0\\
82.28	0\\
82.29	0\\
82.3	0\\
82.31	0\\
82.32	0\\
82.33	0\\
82.34	0\\
82.35	0\\
82.36	0\\
82.37	0\\
82.38	0\\
82.39	0\\
82.4	0\\
82.41	0\\
82.42	0\\
82.43	0\\
82.44	0\\
82.45	0\\
82.46	0\\
82.47	0\\
82.48	0\\
82.49	0\\
82.5	0\\
82.51	0\\
82.52	0\\
82.53	0\\
82.54	0\\
82.55	0\\
82.56	0\\
82.57	0\\
82.58	0\\
82.59	0\\
82.6	0\\
82.61	0\\
82.62	0\\
82.63	0\\
82.64	0\\
82.65	0\\
82.66	0\\
82.67	0\\
82.68	0\\
82.69	0\\
82.7	0\\
82.71	0\\
82.72	0\\
82.73	0\\
82.74	0\\
82.75	0\\
82.76	0\\
82.77	0\\
82.78	0\\
82.79	0\\
82.8	0\\
82.81	0\\
82.82	0\\
82.83	0\\
82.84	0\\
82.85	0\\
82.86	0\\
82.87	0\\
82.88	0\\
82.89	0\\
82.9	0\\
82.91	0\\
82.92	0\\
82.93	0\\
82.94	0\\
82.95	0\\
82.96	0\\
82.97	0\\
82.98	0\\
82.99	0\\
83	0\\
83.01	0\\
83.02	0\\
83.03	0\\
83.04	0\\
83.05	0\\
83.06	0\\
83.07	0\\
83.08	0\\
83.09	0\\
83.1	0\\
83.11	0\\
83.12	0\\
83.13	0\\
83.14	0\\
83.15	0\\
83.16	0\\
83.17	0\\
83.18	0\\
83.19	0\\
83.2	0\\
83.21	0\\
83.22	0\\
83.23	0\\
83.24	0\\
83.25	0\\
83.26	0\\
83.27	0\\
83.28	0\\
83.29	0\\
83.3	0\\
83.31	0\\
83.32	0\\
83.33	0\\
83.34	0\\
83.35	0\\
83.36	0\\
83.37	0\\
83.38	0\\
83.39	0\\
83.4	0\\
83.41	0\\
83.42	0\\
83.43	0\\
83.44	0\\
83.45	0\\
83.46	0\\
83.47	0\\
83.48	0\\
83.49	0\\
83.5	0\\
83.51	0\\
83.52	0\\
83.53	0\\
83.54	0\\
83.55	0\\
83.56	0\\
83.57	0\\
83.58	0\\
83.59	0\\
83.6	0\\
83.61	0\\
83.62	0\\
83.63	0\\
83.64	0\\
83.65	0\\
83.66	0\\
83.67	0\\
83.68	0\\
83.69	0\\
83.7	0\\
83.71	0\\
83.72	0\\
83.73	0\\
83.74	0\\
83.75	0\\
83.76	0\\
83.77	0\\
83.78	0\\
83.79	0\\
83.8	0\\
83.81	0\\
83.82	0\\
83.83	0\\
83.84	0\\
83.85	0\\
83.86	0\\
83.87	0\\
83.88	0\\
83.89	0\\
83.9	0\\
83.91	0\\
83.92	0\\
83.93	0\\
83.94	0\\
83.95	0\\
83.96	0\\
83.97	0\\
83.98	0\\
83.99	0\\
84	0\\
84.01	0\\
84.02	0\\
84.03	0\\
84.04	0\\
84.05	0\\
84.06	0\\
84.07	0\\
84.08	0\\
84.09	0\\
84.1	0\\
84.11	0\\
84.12	0\\
84.13	0\\
84.14	0\\
84.15	0\\
84.16	0\\
84.17	0\\
84.18	0\\
84.19	0\\
84.2	0\\
84.21	0\\
84.22	0\\
84.23	0\\
84.24	0\\
84.25	0\\
84.26	0\\
84.27	0\\
84.28	0\\
84.29	0\\
84.3	0\\
84.31	0\\
84.32	0\\
84.33	0\\
84.34	0\\
84.35	0\\
84.36	0\\
84.37	0\\
84.38	0\\
84.39	0\\
84.4	0\\
84.41	0\\
84.42	0\\
84.43	0\\
84.44	0\\
84.45	0\\
84.46	0\\
84.47	0\\
84.48	0\\
84.49	0\\
84.5	0\\
84.51	0\\
84.52	0\\
84.53	0\\
84.54	0\\
84.55	0\\
84.56	0\\
84.57	0\\
84.58	0\\
84.59	0\\
84.6	0\\
84.61	0\\
84.62	0\\
84.63	0\\
84.64	0\\
84.65	0\\
84.66	0\\
84.67	0\\
84.68	0\\
84.69	0\\
84.7	0\\
84.71	0\\
84.72	0\\
84.73	0\\
84.74	0\\
84.75	0\\
84.76	0\\
84.77	0\\
84.78	0\\
84.79	0\\
84.8	0\\
84.81	0\\
84.82	0\\
84.83	0\\
84.84	0\\
84.85	0\\
84.86	0\\
84.87	0\\
84.88	0\\
84.89	0\\
84.9	0\\
84.91	0\\
84.92	0\\
84.93	0\\
84.94	0\\
84.95	0\\
84.96	0\\
84.97	0\\
84.98	0\\
84.99	0\\
85	0\\
85.01	0\\
85.02	0\\
85.03	0\\
85.04	0\\
85.05	0\\
85.06	0\\
85.07	0\\
85.08	0\\
85.09	0\\
85.1	0\\
85.11	0\\
85.12	0\\
85.13	0\\
85.14	0\\
85.15	0\\
85.16	0\\
85.17	0\\
85.18	0\\
85.19	0\\
85.2	0\\
85.21	0\\
85.22	0\\
85.23	0\\
85.24	0\\
85.25	0\\
85.26	0\\
85.27	0\\
85.28	0\\
85.29	0\\
85.3	0\\
85.31	0\\
85.32	0\\
85.33	0\\
85.34	0\\
85.35	0\\
85.36	0\\
85.37	0\\
85.38	0\\
85.39	0\\
85.4	0\\
85.41	0\\
85.42	0\\
85.43	0\\
85.44	0\\
85.45	0\\
85.46	0\\
85.47	0\\
85.48	0\\
85.49	0\\
85.5	0\\
85.51	0\\
85.52	0\\
85.53	0\\
85.54	0\\
85.55	0\\
85.56	0\\
85.57	0\\
85.58	0\\
85.59	0\\
85.6	0\\
85.61	0\\
85.62	0\\
85.63	0\\
85.64	0\\
85.65	0\\
85.66	0\\
85.67	0\\
85.68	0\\
85.69	0\\
85.7	0\\
85.71	0\\
85.72	0\\
85.73	0\\
85.74	0\\
85.75	0\\
85.76	0\\
85.77	0\\
85.78	0\\
85.79	0\\
85.8	0\\
85.81	0\\
85.82	0\\
85.83	0\\
85.84	0\\
85.85	0\\
85.86	0\\
85.87	0\\
85.88	0\\
85.89	0\\
85.9	0\\
85.91	0\\
85.92	0\\
85.93	0\\
85.94	0\\
85.95	0\\
85.96	0\\
85.97	0\\
85.98	0\\
85.99	0\\
86	0\\
86.01	0\\
86.02	0\\
86.03	0\\
86.04	0\\
86.05	0\\
86.06	0\\
86.07	0\\
86.08	0\\
86.09	0\\
86.1	0\\
86.11	0\\
86.12	0\\
86.13	0\\
86.14	0\\
86.15	0\\
86.16	0\\
86.17	0\\
86.18	0\\
86.19	0\\
86.2	0\\
86.21	0\\
86.22	0\\
86.23	0\\
86.24	0\\
86.25	0\\
86.26	0\\
86.27	0\\
86.28	0\\
86.29	0\\
86.3	0\\
86.31	0\\
86.32	0\\
86.33	0\\
86.34	0\\
86.35	0\\
86.36	0\\
86.37	0\\
86.38	0\\
86.39	0\\
86.4	0\\
86.41	0\\
86.42	0\\
86.43	0\\
86.44	0\\
86.45	0\\
86.46	0\\
86.47	0\\
86.48	0\\
86.49	0\\
86.5	0\\
86.51	0\\
86.52	0\\
86.53	0\\
86.54	0\\
86.55	0\\
86.56	0\\
86.57	0\\
86.58	0\\
86.59	0\\
86.6	0\\
86.61	0\\
86.62	0\\
86.63	0\\
86.64	0\\
86.65	0\\
86.66	0\\
86.67	0\\
86.68	0\\
86.69	0\\
86.7	0\\
86.71	0\\
86.72	0\\
86.73	0\\
86.74	0\\
86.75	0\\
86.76	0\\
86.77	0\\
86.78	0\\
86.79	0\\
86.8	0\\
86.81	0\\
86.82	0\\
86.83	0\\
86.84	0\\
86.85	0\\
86.86	0\\
86.87	0\\
86.88	0\\
86.89	0\\
86.9	0\\
86.91	0\\
86.92	0\\
86.93	0\\
86.94	0\\
86.95	0\\
86.96	0\\
86.97	0\\
86.98	0\\
86.99	0\\
87	0\\
87.01	0\\
87.02	0\\
87.03	0\\
87.04	0\\
87.05	0\\
87.06	0\\
87.07	0\\
87.08	0\\
87.09	0\\
87.1	0\\
87.11	0\\
87.12	0\\
87.13	0\\
87.14	0\\
87.15	0\\
87.16	0\\
87.17	0\\
87.18	0\\
87.19	0\\
87.2	0\\
87.21	0\\
87.22	0\\
87.23	0\\
87.24	0\\
87.25	0\\
87.26	0\\
87.27	0\\
87.28	0\\
87.29	0\\
87.3	0\\
87.31	0\\
87.32	0\\
87.33	0\\
87.34	0\\
87.35	0\\
87.36	0\\
87.37	0\\
87.38	0\\
87.39	0\\
87.4	0\\
87.41	0\\
87.42	0\\
87.43	0\\
87.44	0\\
87.45	0\\
87.46	0\\
87.47	0\\
87.48	0\\
87.49	0\\
87.5	0\\
87.51	0\\
87.52	0\\
87.53	0\\
87.54	0\\
87.55	0\\
87.56	0\\
87.57	0\\
87.58	0\\
87.59	0\\
87.6	0\\
87.61	0\\
87.62	0\\
87.63	0\\
87.64	0\\
87.65	0\\
87.66	0\\
87.67	0\\
87.68	0\\
87.69	0\\
87.7	0\\
87.71	0\\
87.72	0\\
87.73	0\\
87.74	0\\
87.75	0\\
87.76	0\\
87.77	0\\
87.78	0\\
87.79	0\\
87.8	0\\
87.81	0\\
87.82	0\\
87.83	0\\
87.84	0\\
87.85	0\\
87.86	0\\
87.87	0\\
87.88	0\\
87.89	0\\
87.9	0\\
87.91	0\\
87.92	0\\
87.93	0\\
87.94	0\\
87.95	0\\
87.96	0\\
87.97	0\\
87.98	0\\
87.99	0\\
88	0\\
88.01	0\\
88.02	0\\
88.03	0\\
88.04	0\\
88.05	0\\
88.06	0\\
88.07	0\\
88.08	0\\
88.09	0\\
88.1	0\\
88.11	0\\
88.12	0\\
88.13	0\\
88.14	0\\
88.15	0\\
88.16	0\\
88.17	0\\
88.18	0\\
88.19	0\\
88.2	0\\
88.21	0\\
88.22	0\\
88.23	0\\
88.24	0\\
88.25	0\\
88.26	0\\
88.27	0\\
88.28	0\\
88.29	0\\
88.3	0\\
88.31	0\\
88.32	0\\
88.33	0\\
88.34	0\\
88.35	0\\
88.36	0\\
88.37	0\\
88.38	0\\
88.39	0\\
88.4	0\\
88.41	0\\
88.42	0\\
88.43	0\\
88.44	0\\
88.45	0\\
88.46	0\\
88.47	0\\
88.48	0\\
88.49	0\\
88.5	0\\
88.51	0\\
88.52	0\\
88.53	0\\
88.54	0\\
88.55	0\\
88.56	0\\
88.57	0\\
88.58	0\\
88.59	0\\
88.6	0\\
88.61	0\\
88.62	0\\
88.63	0\\
88.64	0\\
88.65	0\\
88.66	0\\
88.67	0\\
88.68	0\\
88.69	0\\
88.7	0\\
88.71	0\\
88.72	0\\
88.73	0\\
88.74	0\\
88.75	0\\
88.76	0\\
88.77	0\\
88.78	0\\
88.79	0\\
88.8	0\\
88.81	0\\
88.82	0\\
88.83	0\\
88.84	0\\
88.85	0\\
88.86	0\\
88.87	0\\
88.88	0\\
88.89	0\\
88.9	0\\
88.91	0\\
88.92	0\\
88.93	0\\
88.94	0\\
88.95	0\\
88.96	0\\
88.97	0\\
88.98	0\\
88.99	0\\
89	0\\
89.01	0\\
89.02	0\\
89.03	0\\
89.04	0\\
89.05	0\\
89.06	0\\
89.07	0\\
89.08	0\\
89.09	0\\
89.1	0\\
89.11	0\\
89.12	0\\
89.13	0\\
89.14	0\\
89.15	0\\
89.16	0\\
89.17	0\\
89.18	0\\
89.19	0\\
89.2	0\\
89.21	0\\
89.22	0\\
89.23	0\\
89.24	0\\
89.25	0\\
89.26	0\\
89.27	0\\
89.28	0\\
89.29	0\\
89.3	0\\
89.31	0\\
89.32	0\\
89.33	0\\
89.34	0\\
89.35	0\\
89.36	0\\
89.37	0\\
89.38	0\\
89.39	0\\
89.4	0\\
89.41	0\\
89.42	0\\
89.43	0\\
89.44	0\\
89.45	0\\
89.46	0\\
89.47	0\\
89.48	0\\
89.49	0\\
89.5	0\\
89.51	0\\
89.52	0\\
89.53	0\\
89.54	0\\
89.55	0\\
89.56	0\\
89.57	0\\
89.58	0\\
89.59	0\\
89.6	0\\
89.61	0\\
89.62	0\\
89.63	0\\
89.64	0\\
89.65	0\\
89.66	0\\
89.67	0\\
89.68	0\\
89.69	0\\
89.7	0\\
89.71	0\\
89.72	0\\
89.73	0\\
89.74	0\\
89.75	0\\
89.76	0\\
89.77	0\\
89.78	0\\
89.79	0\\
89.8	0\\
89.81	0\\
89.82	0\\
89.83	0\\
89.84	0\\
89.85	0\\
89.86	0\\
89.87	0\\
89.88	0\\
89.89	0\\
89.9	0\\
89.91	0\\
89.92	0\\
89.93	0\\
89.94	0\\
89.95	0\\
89.96	0\\
89.97	0\\
89.98	0\\
89.99	0\\
90	0\\
90.01	0\\
90.02	0\\
90.03	0\\
90.04	0\\
90.05	0\\
90.06	0\\
90.07	0\\
90.08	0\\
90.09	0\\
90.1	0\\
90.11	0\\
90.12	0\\
90.13	0\\
90.14	0\\
90.15	0\\
90.16	0\\
90.17	0\\
90.18	0\\
90.19	0\\
90.2	0\\
90.21	0\\
90.22	0\\
90.23	0\\
90.24	0\\
90.25	0\\
90.26	0\\
90.27	0\\
90.28	0\\
90.29	0\\
90.3	0\\
90.31	0\\
90.32	0\\
90.33	0\\
90.34	0\\
90.35	0\\
90.36	0\\
90.37	0\\
90.38	0\\
90.39	0\\
90.4	0\\
90.41	0\\
90.42	0\\
90.43	0\\
90.44	0\\
90.45	0\\
90.46	0\\
90.47	0\\
90.48	0\\
90.49	0\\
90.5	0\\
90.51	0\\
90.52	0\\
90.53	0\\
90.54	0\\
90.55	0\\
90.56	0\\
90.57	0\\
90.58	0\\
90.59	0\\
90.6	0\\
90.61	0\\
90.62	0\\
90.63	0\\
90.64	0\\
90.65	0\\
90.66	0\\
90.67	0\\
90.68	0\\
90.69	0\\
90.7	0\\
90.71	0\\
90.72	0\\
90.73	0\\
90.74	0\\
90.75	0\\
90.76	0\\
90.77	0\\
90.78	0\\
90.79	0\\
90.8	0\\
90.81	0\\
90.82	0\\
90.83	0\\
90.84	0\\
90.85	0\\
90.86	0\\
90.87	0\\
90.88	0\\
90.89	0\\
90.9	0\\
90.91	0\\
90.92	0\\
90.93	0\\
90.94	0\\
90.95	0\\
90.96	0\\
90.97	0\\
90.98	0\\
90.99	0\\
91	0\\
91.01	0\\
91.02	0\\
91.03	0\\
91.04	0\\
91.05	0\\
91.06	0\\
91.07	0\\
91.08	0\\
91.09	0\\
91.1	0\\
91.11	0\\
91.12	0\\
91.13	0\\
91.14	0\\
91.15	0\\
91.16	0\\
91.17	0\\
91.18	0\\
91.19	0\\
91.2	0\\
91.21	0\\
91.22	0\\
91.23	0\\
91.24	0\\
91.25	0\\
91.26	0\\
91.27	0\\
91.28	0\\
91.29	0\\
91.3	0\\
91.31	0\\
91.32	0\\
91.33	0\\
91.34	0\\
91.35	0\\
91.36	0\\
91.37	0\\
91.38	0\\
91.39	0\\
91.4	0\\
91.41	0\\
91.42	0\\
91.43	0\\
91.44	0\\
91.45	0\\
91.46	0\\
91.47	0\\
91.48	0\\
91.49	0\\
91.5	0\\
91.51	0\\
91.52	0\\
91.53	0\\
91.54	0\\
91.55	0\\
91.56	0\\
91.57	0\\
91.58	0\\
91.59	0\\
91.6	0\\
91.61	0\\
91.62	0\\
91.63	0\\
91.64	0\\
91.65	0\\
91.66	0\\
91.67	0\\
91.68	0\\
91.69	0\\
91.7	0\\
91.71	0\\
91.72	0\\
91.73	0\\
91.74	0\\
91.75	0\\
91.76	0\\
91.77	0\\
91.78	0\\
91.79	0\\
91.8	0\\
91.81	0\\
91.82	0\\
91.83	0\\
91.84	0\\
91.85	0\\
91.86	0\\
91.87	0\\
91.88	0\\
91.89	0\\
91.9	0\\
91.91	0\\
91.92	0\\
91.93	0\\
91.94	0\\
91.95	0\\
91.96	0\\
91.97	0\\
91.98	0\\
91.99	0\\
92	0\\
92.01	0\\
92.02	0\\
92.03	0\\
92.04	0\\
92.05	0\\
92.06	0\\
92.07	0\\
92.08	0\\
92.09	0\\
92.1	0\\
92.11	0\\
92.12	0\\
92.13	0\\
92.14	0\\
92.15	0\\
92.16	0\\
92.17	0\\
92.18	0\\
92.19	0\\
92.2	0\\
92.21	0\\
92.22	0\\
92.23	0\\
92.24	0\\
92.25	0\\
92.26	0\\
92.27	0\\
92.28	0\\
92.29	0\\
92.3	0\\
92.31	0\\
92.32	0\\
92.33	0\\
92.34	0\\
92.35	0\\
92.36	0\\
92.37	0\\
92.38	0\\
92.39	0\\
92.4	0\\
92.41	0\\
92.42	0\\
92.43	0\\
92.44	0\\
92.45	0\\
92.46	0\\
92.47	0\\
92.48	0\\
92.49	0\\
92.5	0\\
92.51	0\\
92.52	0\\
92.53	0\\
92.54	0\\
92.55	0\\
92.56	0\\
92.57	0\\
92.58	0\\
92.59	0\\
92.6	0\\
92.61	0\\
92.62	0\\
92.63	0\\
92.64	0\\
92.65	0\\
92.66	0\\
92.67	0\\
92.68	0\\
92.69	0\\
92.7	0\\
92.71	0\\
92.72	0\\
92.73	0\\
92.74	0\\
92.75	0\\
92.76	0\\
92.77	0\\
92.78	0\\
92.79	0\\
92.8	0\\
92.81	0\\
92.82	0\\
92.83	0\\
92.84	0\\
92.85	0\\
92.86	0\\
92.87	0\\
92.88	0\\
92.89	0\\
92.9	0\\
92.91	0\\
92.92	0\\
92.93	0\\
92.94	0\\
92.95	0\\
92.96	0\\
92.97	0\\
92.98	0\\
92.99	0\\
93	0\\
93.01	0\\
93.02	0\\
93.03	0\\
93.04	0\\
93.05	0\\
93.06	0\\
93.07	0\\
93.08	0\\
93.09	0\\
93.1	0\\
93.11	0\\
93.12	0\\
93.13	0\\
93.14	0\\
93.15	0\\
93.16	0\\
93.17	0\\
93.18	0\\
93.19	0\\
93.2	0\\
93.21	0\\
93.22	0\\
93.23	0\\
93.24	0\\
93.25	0\\
93.26	0\\
93.27	0\\
93.28	0\\
93.29	0\\
93.3	0\\
93.31	0\\
93.32	0\\
93.33	0\\
93.34	0\\
93.35	0\\
93.36	0\\
93.37	0\\
93.38	0\\
93.39	0\\
93.4	0\\
93.41	0\\
93.42	0\\
93.43	0\\
93.44	0\\
93.45	0\\
93.46	0\\
93.47	0\\
93.48	0\\
93.49	0\\
93.5	0\\
93.51	0\\
93.52	0\\
93.53	0\\
93.54	0\\
93.55	0\\
93.56	0\\
93.57	0\\
93.58	0\\
93.59	0\\
93.6	0\\
93.61	0\\
93.62	0\\
93.63	0\\
93.64	0\\
93.65	0\\
93.66	0\\
93.67	0\\
93.68	0\\
93.69	0\\
93.7	0\\
93.71	0\\
93.72	0\\
93.73	0\\
93.74	0\\
93.75	0\\
93.76	0\\
93.77	0\\
93.78	0\\
93.79	0\\
93.8	0\\
93.81	0\\
93.82	0\\
93.83	0\\
93.84	0\\
93.85	0\\
93.86	0\\
93.87	0\\
93.88	0\\
93.89	0\\
93.9	0\\
93.91	0\\
93.92	0\\
93.93	0\\
93.94	0\\
93.95	0\\
93.96	0\\
93.97	0\\
93.98	0\\
93.99	0\\
94	0\\
94.01	0\\
94.02	0\\
94.03	0\\
94.04	0\\
94.05	0\\
94.06	0\\
94.07	0\\
94.08	0\\
94.09	0\\
94.1	0\\
94.11	0\\
94.12	0\\
94.13	0\\
94.14	0\\
94.15	0\\
94.16	0\\
94.17	0\\
94.18	0\\
94.19	0\\
94.2	0\\
94.21	0\\
94.22	0\\
94.23	0\\
94.24	0\\
94.25	0\\
94.26	0\\
94.27	0\\
94.28	0\\
94.29	0\\
94.3	0\\
94.31	0\\
94.32	0\\
94.33	0\\
94.34	0\\
94.35	0\\
94.36	0\\
94.37	0\\
94.38	0\\
94.39	0\\
94.4	0\\
94.41	0\\
94.42	0\\
94.43	0\\
94.44	0\\
94.45	0\\
94.46	0\\
94.47	0\\
94.48	0\\
94.49	0\\
94.5	0\\
94.51	0\\
94.52	0\\
94.53	0\\
94.54	0\\
94.55	0\\
94.56	0\\
94.57	0\\
94.58	0\\
94.59	0\\
94.6	0\\
94.61	0\\
94.62	0\\
94.63	0\\
94.64	0\\
94.65	0\\
94.66	0\\
94.67	0\\
94.68	0\\
94.69	0\\
94.7	0\\
94.71	0\\
94.72	0\\
94.73	0\\
94.74	0\\
94.75	0\\
94.76	0\\
94.77	0\\
94.78	0\\
94.79	0\\
94.8	0\\
94.81	0\\
94.82	0\\
94.83	0\\
94.84	0\\
94.85	0\\
94.86	0\\
94.87	0\\
94.88	0\\
94.89	0\\
94.9	0\\
94.91	0\\
94.92	0\\
94.93	0\\
94.94	0\\
94.95	0\\
94.96	0\\
94.97	0\\
94.98	0\\
94.99	0\\
95	0\\
95.01	0\\
95.02	0\\
95.03	0\\
95.04	0\\
95.05	0\\
95.06	0\\
95.07	0\\
95.08	0\\
95.09	0\\
95.1	0\\
95.11	0\\
95.12	0\\
95.13	0\\
95.14	0\\
95.15	0\\
95.16	0\\
95.17	0\\
95.18	0\\
95.19	0\\
95.2	0\\
95.21	0\\
95.22	0\\
95.23	0\\
95.24	0\\
95.25	0\\
95.26	0\\
95.27	0\\
95.28	0\\
95.29	0\\
95.3	0\\
95.31	0\\
95.32	0\\
95.33	0\\
95.34	0\\
95.35	0\\
95.36	0\\
95.37	0\\
95.38	0\\
95.39	0\\
95.4	0\\
95.41	0\\
95.42	0\\
95.43	0\\
95.44	0\\
95.45	0\\
95.46	0\\
95.47	0\\
95.48	0\\
95.49	0\\
95.5	0\\
95.51	0\\
95.52	0\\
95.53	0\\
95.54	0\\
95.55	0\\
95.56	0\\
95.57	0\\
95.58	0\\
95.59	0\\
95.6	0\\
95.61	0\\
95.62	0\\
95.63	0\\
95.64	0\\
95.65	0\\
95.66	0\\
95.67	0\\
95.68	0\\
95.69	0\\
95.7	0\\
95.71	0\\
95.72	0\\
95.73	0\\
95.74	0\\
95.75	0\\
95.76	0\\
95.77	0\\
95.78	0\\
95.79	0\\
95.8	0\\
95.81	0\\
95.82	0\\
95.83	0\\
95.84	0\\
95.85	0\\
95.86	0\\
95.87	0\\
95.88	0\\
95.89	0\\
95.9	0\\
95.91	0\\
95.92	0\\
95.93	0\\
95.94	0\\
95.95	0\\
95.96	0\\
95.97	0\\
95.98	0\\
95.99	0\\
96	0\\
96.01	0\\
96.02	0\\
96.03	0\\
96.04	0\\
96.05	0\\
96.06	0\\
96.07	0\\
96.08	0\\
96.09	0\\
96.1	0\\
96.11	0\\
96.12	0\\
96.13	0\\
96.14	0\\
96.15	0\\
96.16	0\\
96.17	0\\
96.18	0\\
96.19	0\\
96.2	0\\
96.21	0\\
96.22	0\\
96.23	0\\
96.24	0\\
96.25	0\\
96.26	0\\
96.27	0\\
96.28	0\\
96.29	0\\
96.3	0\\
96.31	0\\
96.32	0\\
96.33	0\\
96.34	0\\
96.35	0\\
96.36	0\\
96.37	0\\
96.38	0\\
96.39	0\\
96.4	0\\
96.41	0\\
96.42	0\\
96.43	0\\
96.44	0\\
96.45	0\\
96.46	0\\
96.47	0\\
96.48	0\\
96.49	0\\
96.5	0\\
96.51	0\\
96.52	0\\
96.53	0\\
96.54	0\\
96.55	0\\
96.56	0\\
96.57	0\\
96.58	0\\
96.59	0\\
96.6	0\\
96.61	0\\
96.62	0\\
96.63	0\\
96.64	0\\
96.65	0\\
96.66	0\\
96.67	0\\
96.68	0\\
96.69	0\\
96.7	0\\
96.71	0\\
96.72	0\\
96.73	0\\
96.74	0\\
96.75	0\\
96.76	0\\
96.77	0\\
96.78	0\\
96.79	0\\
96.8	0\\
96.81	0\\
96.82	0\\
96.83	0\\
96.84	0\\
96.85	0\\
96.86	0\\
96.87	0\\
96.88	0\\
96.89	0\\
96.9	0\\
96.91	0\\
96.92	0\\
96.93	0\\
96.94	0\\
96.95	0\\
96.96	0\\
96.97	0\\
96.98	0\\
96.99	0\\
97	0\\
97.01	0\\
97.02	0\\
97.03	0\\
97.04	0\\
97.05	0\\
97.06	0\\
97.07	0\\
97.08	0\\
97.09	0\\
97.1	0\\
97.11	0\\
97.12	0\\
97.13	0\\
97.14	0\\
97.15	0\\
97.16	0\\
97.17	0\\
97.18	0\\
97.19	0\\
97.2	0\\
97.21	0\\
97.22	0\\
97.23	0\\
97.24	0\\
97.25	0\\
97.26	0\\
97.27	0\\
97.28	0\\
97.29	0\\
97.3	0\\
97.31	0\\
97.32	0\\
97.33	0\\
97.34	0\\
97.35	0\\
97.36	0\\
97.37	0\\
97.38	0\\
97.39	0\\
97.4	0\\
97.41	0\\
97.42	0\\
97.43	0\\
97.44	0\\
97.45	0\\
97.46	0\\
97.47	0\\
97.48	0\\
97.49	0\\
97.5	0\\
97.51	0\\
97.52	0\\
97.53	0\\
97.54	0\\
97.55	0\\
97.56	0\\
97.57	0\\
97.58	0\\
97.59	0\\
97.6	0\\
97.61	0\\
97.62	0\\
97.63	0\\
97.64	0\\
97.65	0\\
97.66	0\\
97.67	0\\
97.68	0\\
97.69	0\\
97.7	0\\
97.71	0\\
97.72	0\\
97.73	0\\
97.74	0\\
97.75	0\\
97.76	0\\
97.77	0\\
97.78	0\\
97.79	0\\
97.8	0\\
97.81	0\\
97.82	0\\
97.83	0\\
97.84	0\\
97.85	0\\
97.86	0\\
97.87	0\\
97.88	0\\
97.89	0\\
97.9	0\\
97.91	0\\
97.92	0\\
97.93	0\\
97.94	0\\
97.95	0\\
97.96	0\\
97.97	1.81059676260947e-06\\
97.98	9.68777241342825e-05\\
97.99	0.000192691523931837\\
98	0.000289259362742912\\
98.01	0.000386588687883091\\
98.02	0.00048468703068575\\
98.03	0.00058356200867795\\
98.04	0.00068322132366324\\
98.05	0.000747389874110008\\
98.06	0.000764875133034824\\
98.07	0.000782498710588532\\
98.08	0.000800261589913295\\
98.09	0.000818164755492318\\
98.1	0.00083620919161486\\
98.11	0.000854395874875999\\
98.12	0.000872725782148753\\
98.13	0.000891199890344987\\
98.14	0.000909819176166692\\
98.15	0.00092858461584721\\
98.16	0.000947511321890492\\
98.17	0.000966607561238772\\
98.18	0.000985874972731539\\
98.19	0.00100531520682857\\
98.2	0.00102492992836113\\
98.21	0.00104472081518063\\
98.22	0.00106468956135957\\
98.23	0.00108483787735098\\
98.24	0.00110516749014948\\
98.25	0.00112568014345399\\
98.26	0.00114637759783211\\
98.27	0.00116726163088626\\
98.28	0.00118833403742152\\
98.29	0.00120959662961514\\
98.3	0.001231051237188\\
98.31	0.00125269970757764\\
98.32	0.00127454390611327\\
98.33	0.00129658571112819\\
98.34	0.00131882701862342\\
98.35	0.00134126973834978\\
98.36	0.00136391575154592\\
98.37	0.0013867669570129\\
98.38	0.00140982527127774\\
98.39	0.0014330926287585\\
98.4	0.00145657098193082\\
98.41	0.00148026230149669\\
98.42	0.00150416857655937\\
98.43	0.00152829181479498\\
98.44	0.00155263404262572\\
98.45	0.00157719730539465\\
98.46	0.00160198366754217\\
98.47	0.00162699521278405\\
98.48	0.00165223404429121\\
98.49	0.00167770228487111\\
98.5	0.00170340207715083\\
98.51	0.00172933558376194\\
98.52	0.00175550498752694\\
98.53	0.00178191249164758\\
98.54	0.00180856031989485\\
98.55	0.00183545071680079\\
98.56	0.00186258594785198\\
98.57	0.00188996829968496\\
98.58	0.00191760008028335\\
98.59	0.00194548361917685\\
98.6	0.00197362126764206\\
98.61	0.00200201539890522\\
98.62	0.00203066840834677\\
98.63	0.00205958271370804\\
98.64	0.002088760755301\\
98.65	0.00211820499621877\\
98.66	0.002147917922548\\
98.67	0.0021779020435833\\
98.68	0.00220815989204362\\
98.69	0.00223869402429069\\
98.7	0.00226950702054944\\
98.71	0.00230060148513054\\
98.72	0.00233198004665493\\
98.73	0.00236364535828055\\
98.74	0.00239560009793104\\
98.75	0.00242784696852669\\
98.76	0.0024603886982175\\
98.77	0.00249322804061834\\
98.78	0.00252636777504639\\
98.79	0.00255981070676076\\
98.8	0.00259355966720435\\
98.81	0.00262761751424789\\
98.82	0.00266198713243635\\
98.83	0.00269667143323762\\
98.84	0.0027316733552934\\
98.85	0.00276699586467263\\
98.86	0.00280264195512706\\
98.87	0.00283861464834933\\
98.88	0.00287491699423346\\
98.89	0.00291155207113766\\
98.9	0.00294852298614972\\
98.91	0.00298583287535475\\
98.92	0.00302348490410548\\
98.93	0.00306148226729509\\
98.94	0.00309982818963248\\
98.95	0.00313852592592026\\
98.96	0.00317757876133517\\
98.97	0.00321699001171122\\
98.98	0.0032567630238254\\
98.99	0.00329690117568609\\
99	0.00333740787682413\\
99.01	0.00337828656858661\\
99.02	0.00341954072443339\\
99.03	0.00346117385023638\\
99.04	0.00350318948458161\\
99.05	0.00354559119907409\\
99.06	0.00358838259864556\\
99.07	0.00363156732186504\\
99.08	0.00367514904125229\\
99.09	0.00371913146359422\\
99.1	0.0037635183302642\\
99.11	0.00380831341754438\\
99.12	0.00385352053695097\\
99.13	0.00389914353556257\\
99.14	0.00394518629635159\\
99.15	0.00399165273851862\\
99.16	0.00403854681783014\\
99.17	0.00408587252695911\\
99.18	0.00413363389582897\\
99.19	0.00418183495431657\\
99.2	0.00423047976694747\\
99.21	0.00427957243576882\\
99.22	0.00432911710069517\\
99.23	0.00437911793985739\\
99.24	0.00442957916995488\\
99.25	0.00448050504661093\\
99.26	0.00453189986473142\\
99.27	0.0045837679588668\\
99.28	0.00463611370357735\\
99.29	0.00468894151380188\\
99.3	0.00474225584522977\\
99.31	0.00479606119467643\\
99.32	0.00485036210046225\\
99.33	0.00490516314279497\\
99.34	0.00496046894415566\\
99.35	0.00501628416968814\\
99.36	0.00507261352759206\\
99.37	0.0051294617695196\\
99.38	0.00518683369097572\\
99.39	0.00524473413172222\\
99.4	0.00530316797618543\\
99.41	0.00536214015386766\\
99.42	0.00542165563976247\\
99.43	0.00548171945477372\\
99.44	0.00554233666613847\\
99.45	0.00560351238785381\\
99.46	0.00566525178110757\\
99.47	0.00572756005471303\\
99.48	0.00579044246554756\\
99.49	0.00585390431899538\\
99.5	0.00591795096939432\\
99.51	0.00598258782048674\\
99.52	0.00604782032587452\\
99.53	0.00611365398947834\\
99.54	0.00618009436600108\\
99.55	0.00624714706139553\\
99.56	0.00631481773333638\\
99.57	0.00638311209169663\\
99.58	0.00645203589902817\\
99.59	0.00652159497104708\\
99.6	0.00659179517712308\\
99.61	0.00666264244077372\\
99.62	0.00673414273681778\\
99.63	0.00680630208973471\\
99.64	0.00687912657950685\\
99.65	0.00695262234212994\\
99.66	0.0070267955701283\\
99.67	0.00710165251307478\\
99.68	0.00717719947811546\\
99.69	0.00725344283049917\\
99.7	0.00733038899411186\\
99.71	0.0074080444520159\\
99.72	0.00748641574699432\\
99.73	0.00756550948210005\\
99.74	0.00764533232121015\\
99.75	0.00772589098958526\\
99.76	0.00780719227443403\\
99.77	0.00788924302548281\\
99.78	0.00797205015555061\\
99.79	0.00805562064112923\\
99.8	0.00813996152296881\\
99.81	0.00822507990666871\\
99.82	0.00831098296327383\\
99.83	0.00839767792987635\\
99.84	0.00848517211022314\\
99.85	0.00857347287532854\\
99.86	0.00866258766409292\\
99.87	0.00875252398392687\\
99.88	0.00884328941138109\\
99.89	0.00893489159278211\\
99.9	0.00902733824487381\\
99.91	0.00912063715546484\\
99.92	0.00921479618408196\\
99.93	0.00930982326262945\\
99.94	0.00940572639605445\\
99.95	0.00950251366301854\\
99.96	0.00960019321657535\\
99.97	0.00969877328485451\\
99.98	0.00979826217175179\\
99.99	0.00989866825762563\\
100	0.01\\
};
\addlegendentry{$q=-4$};

\addplot [color=mycolor1,dashed,forget plot]
  table[row sep=crcr]{%
0.01	0\\
0.02	0\\
0.03	0\\
0.04	0\\
0.05	0\\
0.06	0\\
0.07	0\\
0.08	0\\
0.09	0\\
0.1	0\\
0.11	0\\
0.12	0\\
0.13	0\\
0.14	0\\
0.15	0\\
0.16	0\\
0.17	0\\
0.18	0\\
0.19	0\\
0.2	0\\
0.21	0\\
0.22	0\\
0.23	0\\
0.24	0\\
0.25	0\\
0.26	0\\
0.27	0\\
0.28	0\\
0.29	0\\
0.3	0\\
0.31	0\\
0.32	0\\
0.33	0\\
0.34	0\\
0.35	0\\
0.36	0\\
0.37	0\\
0.38	0\\
0.39	0\\
0.4	0\\
0.41	0\\
0.42	0\\
0.43	0\\
0.44	0\\
0.45	0\\
0.46	0\\
0.47	0\\
0.48	0\\
0.49	0\\
0.5	0\\
0.51	0\\
0.52	0\\
0.53	0\\
0.54	0\\
0.55	0\\
0.56	0\\
0.57	0\\
0.58	0\\
0.59	0\\
0.6	0\\
0.61	0\\
0.62	0\\
0.63	0\\
0.64	0\\
0.65	0\\
0.66	0\\
0.67	0\\
0.68	0\\
0.69	0\\
0.7	0\\
0.71	0\\
0.72	0\\
0.73	0\\
0.74	0\\
0.75	0\\
0.76	0\\
0.77	0\\
0.78	0\\
0.79	0\\
0.8	0\\
0.81	0\\
0.82	0\\
0.83	0\\
0.84	0\\
0.85	0\\
0.86	0\\
0.87	0\\
0.88	0\\
0.89	0\\
0.9	0\\
0.91	0\\
0.92	0\\
0.93	0\\
0.94	0\\
0.95	0\\
0.96	0\\
0.97	0\\
0.98	0\\
0.99	0\\
1	0\\
1.01	0\\
1.02	0\\
1.03	0\\
1.04	0\\
1.05	0\\
1.06	0\\
1.07	0\\
1.08	0\\
1.09	0\\
1.1	0\\
1.11	0\\
1.12	0\\
1.13	0\\
1.14	0\\
1.15	0\\
1.16	0\\
1.17	0\\
1.18	0\\
1.19	0\\
1.2	0\\
1.21	0\\
1.22	0\\
1.23	0\\
1.24	0\\
1.25	0\\
1.26	0\\
1.27	0\\
1.28	0\\
1.29	0\\
1.3	0\\
1.31	0\\
1.32	0\\
1.33	0\\
1.34	0\\
1.35	0\\
1.36	0\\
1.37	0\\
1.38	0\\
1.39	0\\
1.4	0\\
1.41	0\\
1.42	0\\
1.43	0\\
1.44	0\\
1.45	0\\
1.46	0\\
1.47	0\\
1.48	0\\
1.49	0\\
1.5	0\\
1.51	0\\
1.52	0\\
1.53	0\\
1.54	0\\
1.55	0\\
1.56	0\\
1.57	0\\
1.58	0\\
1.59	0\\
1.6	0\\
1.61	0\\
1.62	0\\
1.63	0\\
1.64	0\\
1.65	0\\
1.66	0\\
1.67	0\\
1.68	0\\
1.69	0\\
1.7	0\\
1.71	0\\
1.72	0\\
1.73	0\\
1.74	0\\
1.75	0\\
1.76	0\\
1.77	0\\
1.78	0\\
1.79	0\\
1.8	0\\
1.81	0\\
1.82	0\\
1.83	0\\
1.84	0\\
1.85	0\\
1.86	0\\
1.87	0\\
1.88	0\\
1.89	0\\
1.9	0\\
1.91	0\\
1.92	0\\
1.93	0\\
1.94	0\\
1.95	0\\
1.96	0\\
1.97	0\\
1.98	0\\
1.99	0\\
2	0\\
2.01	0\\
2.02	0\\
2.03	0\\
2.04	0\\
2.05	0\\
2.06	0\\
2.07	0\\
2.08	0\\
2.09	0\\
2.1	0\\
2.11	0\\
2.12	0\\
2.13	0\\
2.14	0\\
2.15	0\\
2.16	0\\
2.17	0\\
2.18	0\\
2.19	0\\
2.2	0\\
2.21	0\\
2.22	0\\
2.23	0\\
2.24	0\\
2.25	0\\
2.26	0\\
2.27	0\\
2.28	0\\
2.29	0\\
2.3	0\\
2.31	0\\
2.32	0\\
2.33	0\\
2.34	0\\
2.35	0\\
2.36	0\\
2.37	0\\
2.38	0\\
2.39	0\\
2.4	0\\
2.41	0\\
2.42	0\\
2.43	0\\
2.44	0\\
2.45	0\\
2.46	0\\
2.47	0\\
2.48	0\\
2.49	0\\
2.5	0\\
2.51	0\\
2.52	0\\
2.53	0\\
2.54	0\\
2.55	0\\
2.56	0\\
2.57	0\\
2.58	0\\
2.59	0\\
2.6	0\\
2.61	0\\
2.62	0\\
2.63	0\\
2.64	0\\
2.65	0\\
2.66	0\\
2.67	0\\
2.68	0\\
2.69	0\\
2.7	0\\
2.71	0\\
2.72	0\\
2.73	0\\
2.74	0\\
2.75	0\\
2.76	0\\
2.77	0\\
2.78	0\\
2.79	0\\
2.8	0\\
2.81	0\\
2.82	0\\
2.83	0\\
2.84	0\\
2.85	0\\
2.86	0\\
2.87	0\\
2.88	0\\
2.89	0\\
2.9	0\\
2.91	0\\
2.92	0\\
2.93	0\\
2.94	0\\
2.95	0\\
2.96	0\\
2.97	0\\
2.98	0\\
2.99	0\\
3	0\\
3.01	0\\
3.02	0\\
3.03	0\\
3.04	0\\
3.05	0\\
3.06	0\\
3.07	0\\
3.08	0\\
3.09	0\\
3.1	0\\
3.11	0\\
3.12	0\\
3.13	0\\
3.14	0\\
3.15	0\\
3.16	0\\
3.17	0\\
3.18	0\\
3.19	0\\
3.2	0\\
3.21	0\\
3.22	0\\
3.23	0\\
3.24	0\\
3.25	0\\
3.26	0\\
3.27	0\\
3.28	0\\
3.29	0\\
3.3	0\\
3.31	0\\
3.32	0\\
3.33	0\\
3.34	0\\
3.35	0\\
3.36	0\\
3.37	0\\
3.38	0\\
3.39	0\\
3.4	0\\
3.41	0\\
3.42	0\\
3.43	0\\
3.44	0\\
3.45	0\\
3.46	0\\
3.47	0\\
3.48	0\\
3.49	0\\
3.5	0\\
3.51	0\\
3.52	0\\
3.53	0\\
3.54	0\\
3.55	0\\
3.56	0\\
3.57	0\\
3.58	0\\
3.59	0\\
3.6	0\\
3.61	0\\
3.62	0\\
3.63	0\\
3.64	0\\
3.65	0\\
3.66	0\\
3.67	0\\
3.68	0\\
3.69	0\\
3.7	0\\
3.71	0\\
3.72	0\\
3.73	0\\
3.74	0\\
3.75	0\\
3.76	0\\
3.77	0\\
3.78	0\\
3.79	0\\
3.8	0\\
3.81	0\\
3.82	0\\
3.83	0\\
3.84	0\\
3.85	0\\
3.86	0\\
3.87	0\\
3.88	0\\
3.89	0\\
3.9	0\\
3.91	0\\
3.92	0\\
3.93	0\\
3.94	0\\
3.95	0\\
3.96	0\\
3.97	0\\
3.98	0\\
3.99	0\\
4	0\\
4.01	0\\
4.02	0\\
4.03	0\\
4.04	0\\
4.05	0\\
4.06	0\\
4.07	0\\
4.08	0\\
4.09	0\\
4.1	0\\
4.11	0\\
4.12	0\\
4.13	0\\
4.14	0\\
4.15	0\\
4.16	0\\
4.17	0\\
4.18	0\\
4.19	0\\
4.2	0\\
4.21	0\\
4.22	0\\
4.23	0\\
4.24	0\\
4.25	0\\
4.26	0\\
4.27	0\\
4.28	0\\
4.29	0\\
4.3	0\\
4.31	0\\
4.32	0\\
4.33	0\\
4.34	0\\
4.35	0\\
4.36	0\\
4.37	0\\
4.38	0\\
4.39	0\\
4.4	0\\
4.41	0\\
4.42	0\\
4.43	0\\
4.44	0\\
4.45	0\\
4.46	0\\
4.47	0\\
4.48	0\\
4.49	0\\
4.5	0\\
4.51	0\\
4.52	0\\
4.53	0\\
4.54	0\\
4.55	0\\
4.56	0\\
4.57	0\\
4.58	0\\
4.59	0\\
4.6	0\\
4.61	0\\
4.62	0\\
4.63	0\\
4.64	0\\
4.65	0\\
4.66	0\\
4.67	0\\
4.68	0\\
4.69	0\\
4.7	0\\
4.71	0\\
4.72	0\\
4.73	0\\
4.74	0\\
4.75	0\\
4.76	0\\
4.77	0\\
4.78	0\\
4.79	0\\
4.8	0\\
4.81	0\\
4.82	0\\
4.83	0\\
4.84	0\\
4.85	0\\
4.86	0\\
4.87	0\\
4.88	0\\
4.89	0\\
4.9	0\\
4.91	0\\
4.92	0\\
4.93	0\\
4.94	0\\
4.95	0\\
4.96	0\\
4.97	0\\
4.98	0\\
4.99	0\\
5	0\\
5.01	0\\
5.02	0\\
5.03	0\\
5.04	0\\
5.05	0\\
5.06	0\\
5.07	0\\
5.08	0\\
5.09	0\\
5.1	0\\
5.11	0\\
5.12	0\\
5.13	0\\
5.14	0\\
5.15	0\\
5.16	0\\
5.17	0\\
5.18	0\\
5.19	0\\
5.2	0\\
5.21	0\\
5.22	0\\
5.23	0\\
5.24	0\\
5.25	0\\
5.26	0\\
5.27	0\\
5.28	0\\
5.29	0\\
5.3	0\\
5.31	0\\
5.32	0\\
5.33	0\\
5.34	0\\
5.35	0\\
5.36	0\\
5.37	0\\
5.38	0\\
5.39	0\\
5.4	0\\
5.41	0\\
5.42	0\\
5.43	0\\
5.44	0\\
5.45	0\\
5.46	0\\
5.47	0\\
5.48	0\\
5.49	0\\
5.5	0\\
5.51	0\\
5.52	0\\
5.53	0\\
5.54	0\\
5.55	0\\
5.56	0\\
5.57	0\\
5.58	0\\
5.59	0\\
5.6	0\\
5.61	0\\
5.62	0\\
5.63	0\\
5.64	0\\
5.65	0\\
5.66	0\\
5.67	0\\
5.68	0\\
5.69	0\\
5.7	0\\
5.71	0\\
5.72	0\\
5.73	0\\
5.74	0\\
5.75	0\\
5.76	0\\
5.77	0\\
5.78	0\\
5.79	0\\
5.8	0\\
5.81	0\\
5.82	0\\
5.83	0\\
5.84	0\\
5.85	0\\
5.86	0\\
5.87	0\\
5.88	0\\
5.89	0\\
5.9	0\\
5.91	0\\
5.92	0\\
5.93	0\\
5.94	0\\
5.95	0\\
5.96	0\\
5.97	0\\
5.98	0\\
5.99	0\\
6	0\\
6.01	0\\
6.02	0\\
6.03	0\\
6.04	0\\
6.05	0\\
6.06	0\\
6.07	0\\
6.08	0\\
6.09	0\\
6.1	0\\
6.11	0\\
6.12	0\\
6.13	0\\
6.14	0\\
6.15	0\\
6.16	0\\
6.17	0\\
6.18	0\\
6.19	0\\
6.2	0\\
6.21	0\\
6.22	0\\
6.23	0\\
6.24	0\\
6.25	0\\
6.26	0\\
6.27	0\\
6.28	0\\
6.29	0\\
6.3	0\\
6.31	0\\
6.32	0\\
6.33	0\\
6.34	0\\
6.35	0\\
6.36	0\\
6.37	0\\
6.38	0\\
6.39	0\\
6.4	0\\
6.41	0\\
6.42	0\\
6.43	0\\
6.44	0\\
6.45	0\\
6.46	0\\
6.47	0\\
6.48	0\\
6.49	0\\
6.5	0\\
6.51	0\\
6.52	0\\
6.53	0\\
6.54	0\\
6.55	0\\
6.56	0\\
6.57	0\\
6.58	0\\
6.59	0\\
6.6	0\\
6.61	0\\
6.62	0\\
6.63	0\\
6.64	0\\
6.65	0\\
6.66	0\\
6.67	0\\
6.68	0\\
6.69	0\\
6.7	0\\
6.71	0\\
6.72	0\\
6.73	0\\
6.74	0\\
6.75	0\\
6.76	0\\
6.77	0\\
6.78	0\\
6.79	0\\
6.8	0\\
6.81	0\\
6.82	0\\
6.83	0\\
6.84	0\\
6.85	0\\
6.86	0\\
6.87	0\\
6.88	0\\
6.89	0\\
6.9	0\\
6.91	0\\
6.92	0\\
6.93	0\\
6.94	0\\
6.95	0\\
6.96	0\\
6.97	0\\
6.98	0\\
6.99	0\\
7	0\\
7.01	0\\
7.02	0\\
7.03	0\\
7.04	0\\
7.05	0\\
7.06	0\\
7.07	0\\
7.08	0\\
7.09	0\\
7.1	0\\
7.11	0\\
7.12	0\\
7.13	0\\
7.14	0\\
7.15	0\\
7.16	0\\
7.17	0\\
7.18	0\\
7.19	0\\
7.2	0\\
7.21	0\\
7.22	0\\
7.23	0\\
7.24	0\\
7.25	0\\
7.26	0\\
7.27	0\\
7.28	0\\
7.29	0\\
7.3	0\\
7.31	0\\
7.32	0\\
7.33	0\\
7.34	0\\
7.35	0\\
7.36	0\\
7.37	0\\
7.38	0\\
7.39	0\\
7.4	0\\
7.41	0\\
7.42	0\\
7.43	0\\
7.44	0\\
7.45	0\\
7.46	0\\
7.47	0\\
7.48	0\\
7.49	0\\
7.5	0\\
7.51	0\\
7.52	0\\
7.53	0\\
7.54	0\\
7.55	0\\
7.56	0\\
7.57	0\\
7.58	0\\
7.59	0\\
7.6	0\\
7.61	0\\
7.62	0\\
7.63	0\\
7.64	0\\
7.65	0\\
7.66	0\\
7.67	0\\
7.68	0\\
7.69	0\\
7.7	0\\
7.71	0\\
7.72	0\\
7.73	0\\
7.74	0\\
7.75	0\\
7.76	0\\
7.77	0\\
7.78	0\\
7.79	0\\
7.8	0\\
7.81	0\\
7.82	0\\
7.83	0\\
7.84	0\\
7.85	0\\
7.86	0\\
7.87	0\\
7.88	0\\
7.89	0\\
7.9	0\\
7.91	0\\
7.92	0\\
7.93	0\\
7.94	0\\
7.95	0\\
7.96	0\\
7.97	0\\
7.98	0\\
7.99	0\\
8	0\\
8.01	0\\
8.02	0\\
8.03	0\\
8.04	0\\
8.05	0\\
8.06	0\\
8.07	0\\
8.08	0\\
8.09	0\\
8.1	0\\
8.11	0\\
8.12	0\\
8.13	0\\
8.14	0\\
8.15	0\\
8.16	0\\
8.17	0\\
8.18	0\\
8.19	0\\
8.2	0\\
8.21	0\\
8.22	0\\
8.23	0\\
8.24	0\\
8.25	0\\
8.26	0\\
8.27	0\\
8.28	0\\
8.29	0\\
8.3	0\\
8.31	0\\
8.32	0\\
8.33	0\\
8.34	0\\
8.35	0\\
8.36	0\\
8.37	0\\
8.38	0\\
8.39	0\\
8.4	0\\
8.41	0\\
8.42	0\\
8.43	0\\
8.44	0\\
8.45	0\\
8.46	0\\
8.47	0\\
8.48	0\\
8.49	0\\
8.5	0\\
8.51	0\\
8.52	0\\
8.53	0\\
8.54	0\\
8.55	0\\
8.56	0\\
8.57	0\\
8.58	0\\
8.59	0\\
8.6	0\\
8.61	0\\
8.62	0\\
8.63	0\\
8.64	0\\
8.65	0\\
8.66	0\\
8.67	0\\
8.68	0\\
8.69	0\\
8.7	0\\
8.71	0\\
8.72	0\\
8.73	0\\
8.74	0\\
8.75	0\\
8.76	0\\
8.77	0\\
8.78	0\\
8.79	0\\
8.8	0\\
8.81	0\\
8.82	0\\
8.83	0\\
8.84	0\\
8.85	0\\
8.86	0\\
8.87	0\\
8.88	0\\
8.89	0\\
8.9	0\\
8.91	0\\
8.92	0\\
8.93	0\\
8.94	0\\
8.95	0\\
8.96	0\\
8.97	0\\
8.98	0\\
8.99	0\\
9	0\\
9.01	0\\
9.02	0\\
9.03	0\\
9.04	0\\
9.05	0\\
9.06	0\\
9.07	0\\
9.08	0\\
9.09	0\\
9.1	0\\
9.11	0\\
9.12	0\\
9.13	0\\
9.14	0\\
9.15	0\\
9.16	0\\
9.17	0\\
9.18	0\\
9.19	0\\
9.2	0\\
9.21	0\\
9.22	0\\
9.23	0\\
9.24	0\\
9.25	0\\
9.26	0\\
9.27	0\\
9.28	0\\
9.29	0\\
9.3	0\\
9.31	0\\
9.32	0\\
9.33	0\\
9.34	0\\
9.35	0\\
9.36	0\\
9.37	0\\
9.38	0\\
9.39	0\\
9.4	0\\
9.41	0\\
9.42	0\\
9.43	0\\
9.44	0\\
9.45	0\\
9.46	0\\
9.47	0\\
9.48	0\\
9.49	0\\
9.5	0\\
9.51	0\\
9.52	0\\
9.53	0\\
9.54	0\\
9.55	0\\
9.56	0\\
9.57	0\\
9.58	0\\
9.59	0\\
9.6	0\\
9.61	0\\
9.62	0\\
9.63	0\\
9.64	0\\
9.65	0\\
9.66	0\\
9.67	0\\
9.68	0\\
9.69	0\\
9.7	0\\
9.71	0\\
9.72	0\\
9.73	0\\
9.74	0\\
9.75	0\\
9.76	0\\
9.77	0\\
9.78	0\\
9.79	0\\
9.8	0\\
9.81	0\\
9.82	0\\
9.83	0\\
9.84	0\\
9.85	0\\
9.86	0\\
9.87	0\\
9.88	0\\
9.89	0\\
9.9	0\\
9.91	0\\
9.92	0\\
9.93	0\\
9.94	0\\
9.95	0\\
9.96	0\\
9.97	0\\
9.98	0\\
9.99	0\\
10	0\\
10.01	0\\
10.02	0\\
10.03	0\\
10.04	0\\
10.05	0\\
10.06	0\\
10.07	0\\
10.08	0\\
10.09	0\\
10.1	0\\
10.11	0\\
10.12	0\\
10.13	0\\
10.14	0\\
10.15	0\\
10.16	0\\
10.17	0\\
10.18	0\\
10.19	0\\
10.2	0\\
10.21	0\\
10.22	0\\
10.23	0\\
10.24	0\\
10.25	0\\
10.26	0\\
10.27	0\\
10.28	0\\
10.29	0\\
10.3	0\\
10.31	0\\
10.32	0\\
10.33	0\\
10.34	0\\
10.35	0\\
10.36	0\\
10.37	0\\
10.38	0\\
10.39	0\\
10.4	0\\
10.41	0\\
10.42	0\\
10.43	0\\
10.44	0\\
10.45	0\\
10.46	0\\
10.47	0\\
10.48	0\\
10.49	0\\
10.5	0\\
10.51	0\\
10.52	0\\
10.53	0\\
10.54	0\\
10.55	0\\
10.56	0\\
10.57	0\\
10.58	0\\
10.59	0\\
10.6	0\\
10.61	0\\
10.62	0\\
10.63	0\\
10.64	0\\
10.65	0\\
10.66	0\\
10.67	0\\
10.68	0\\
10.69	0\\
10.7	0\\
10.71	0\\
10.72	0\\
10.73	0\\
10.74	0\\
10.75	0\\
10.76	0\\
10.77	0\\
10.78	0\\
10.79	0\\
10.8	0\\
10.81	0\\
10.82	0\\
10.83	0\\
10.84	0\\
10.85	0\\
10.86	0\\
10.87	0\\
10.88	0\\
10.89	0\\
10.9	0\\
10.91	0\\
10.92	0\\
10.93	0\\
10.94	0\\
10.95	0\\
10.96	0\\
10.97	0\\
10.98	0\\
10.99	0\\
11	0\\
11.01	0\\
11.02	0\\
11.03	0\\
11.04	0\\
11.05	0\\
11.06	0\\
11.07	0\\
11.08	0\\
11.09	0\\
11.1	0\\
11.11	0\\
11.12	0\\
11.13	0\\
11.14	0\\
11.15	0\\
11.16	0\\
11.17	0\\
11.18	0\\
11.19	0\\
11.2	0\\
11.21	0\\
11.22	0\\
11.23	0\\
11.24	0\\
11.25	0\\
11.26	0\\
11.27	0\\
11.28	0\\
11.29	0\\
11.3	0\\
11.31	0\\
11.32	0\\
11.33	0\\
11.34	0\\
11.35	0\\
11.36	0\\
11.37	0\\
11.38	0\\
11.39	0\\
11.4	0\\
11.41	0\\
11.42	0\\
11.43	0\\
11.44	0\\
11.45	0\\
11.46	0\\
11.47	0\\
11.48	0\\
11.49	0\\
11.5	0\\
11.51	0\\
11.52	0\\
11.53	0\\
11.54	0\\
11.55	0\\
11.56	0\\
11.57	0\\
11.58	0\\
11.59	0\\
11.6	0\\
11.61	0\\
11.62	0\\
11.63	0\\
11.64	0\\
11.65	0\\
11.66	0\\
11.67	0\\
11.68	0\\
11.69	0\\
11.7	0\\
11.71	0\\
11.72	0\\
11.73	0\\
11.74	0\\
11.75	0\\
11.76	0\\
11.77	0\\
11.78	0\\
11.79	0\\
11.8	0\\
11.81	0\\
11.82	0\\
11.83	0\\
11.84	0\\
11.85	0\\
11.86	0\\
11.87	0\\
11.88	0\\
11.89	0\\
11.9	0\\
11.91	0\\
11.92	0\\
11.93	0\\
11.94	0\\
11.95	0\\
11.96	0\\
11.97	0\\
11.98	0\\
11.99	0\\
12	0\\
12.01	0\\
12.02	0\\
12.03	0\\
12.04	0\\
12.05	0\\
12.06	0\\
12.07	0\\
12.08	0\\
12.09	0\\
12.1	0\\
12.11	0\\
12.12	0\\
12.13	0\\
12.14	0\\
12.15	0\\
12.16	0\\
12.17	0\\
12.18	0\\
12.19	0\\
12.2	0\\
12.21	0\\
12.22	0\\
12.23	0\\
12.24	0\\
12.25	0\\
12.26	0\\
12.27	0\\
12.28	0\\
12.29	0\\
12.3	0\\
12.31	0\\
12.32	0\\
12.33	0\\
12.34	0\\
12.35	0\\
12.36	0\\
12.37	0\\
12.38	0\\
12.39	0\\
12.4	0\\
12.41	0\\
12.42	0\\
12.43	0\\
12.44	0\\
12.45	0\\
12.46	0\\
12.47	0\\
12.48	0\\
12.49	0\\
12.5	0\\
12.51	0\\
12.52	0\\
12.53	0\\
12.54	0\\
12.55	0\\
12.56	0\\
12.57	0\\
12.58	0\\
12.59	0\\
12.6	0\\
12.61	0\\
12.62	0\\
12.63	0\\
12.64	0\\
12.65	0\\
12.66	0\\
12.67	0\\
12.68	0\\
12.69	0\\
12.7	0\\
12.71	0\\
12.72	0\\
12.73	0\\
12.74	0\\
12.75	0\\
12.76	0\\
12.77	0\\
12.78	0\\
12.79	0\\
12.8	0\\
12.81	0\\
12.82	0\\
12.83	0\\
12.84	0\\
12.85	0\\
12.86	0\\
12.87	0\\
12.88	0\\
12.89	0\\
12.9	0\\
12.91	0\\
12.92	0\\
12.93	0\\
12.94	0\\
12.95	0\\
12.96	0\\
12.97	0\\
12.98	0\\
12.99	0\\
13	0\\
13.01	0\\
13.02	0\\
13.03	0\\
13.04	0\\
13.05	0\\
13.06	0\\
13.07	0\\
13.08	0\\
13.09	0\\
13.1	0\\
13.11	0\\
13.12	0\\
13.13	0\\
13.14	0\\
13.15	0\\
13.16	0\\
13.17	0\\
13.18	0\\
13.19	0\\
13.2	0\\
13.21	0\\
13.22	0\\
13.23	0\\
13.24	0\\
13.25	0\\
13.26	0\\
13.27	0\\
13.28	0\\
13.29	0\\
13.3	0\\
13.31	0\\
13.32	0\\
13.33	0\\
13.34	0\\
13.35	0\\
13.36	0\\
13.37	0\\
13.38	0\\
13.39	0\\
13.4	0\\
13.41	0\\
13.42	0\\
13.43	0\\
13.44	0\\
13.45	0\\
13.46	0\\
13.47	0\\
13.48	0\\
13.49	0\\
13.5	0\\
13.51	0\\
13.52	0\\
13.53	0\\
13.54	0\\
13.55	0\\
13.56	0\\
13.57	0\\
13.58	0\\
13.59	0\\
13.6	0\\
13.61	0\\
13.62	0\\
13.63	0\\
13.64	0\\
13.65	0\\
13.66	0\\
13.67	0\\
13.68	0\\
13.69	0\\
13.7	0\\
13.71	0\\
13.72	0\\
13.73	0\\
13.74	0\\
13.75	0\\
13.76	0\\
13.77	0\\
13.78	0\\
13.79	0\\
13.8	0\\
13.81	0\\
13.82	0\\
13.83	0\\
13.84	0\\
13.85	0\\
13.86	0\\
13.87	0\\
13.88	0\\
13.89	0\\
13.9	0\\
13.91	0\\
13.92	0\\
13.93	0\\
13.94	0\\
13.95	0\\
13.96	0\\
13.97	0\\
13.98	0\\
13.99	0\\
14	0\\
14.01	0\\
14.02	0\\
14.03	0\\
14.04	0\\
14.05	0\\
14.06	0\\
14.07	0\\
14.08	0\\
14.09	0\\
14.1	0\\
14.11	0\\
14.12	0\\
14.13	0\\
14.14	0\\
14.15	0\\
14.16	0\\
14.17	0\\
14.18	0\\
14.19	0\\
14.2	0\\
14.21	0\\
14.22	0\\
14.23	0\\
14.24	0\\
14.25	0\\
14.26	0\\
14.27	0\\
14.28	0\\
14.29	0\\
14.3	0\\
14.31	0\\
14.32	0\\
14.33	0\\
14.34	0\\
14.35	0\\
14.36	0\\
14.37	0\\
14.38	0\\
14.39	0\\
14.4	0\\
14.41	0\\
14.42	0\\
14.43	0\\
14.44	0\\
14.45	0\\
14.46	0\\
14.47	0\\
14.48	0\\
14.49	0\\
14.5	0\\
14.51	0\\
14.52	0\\
14.53	0\\
14.54	0\\
14.55	0\\
14.56	0\\
14.57	0\\
14.58	0\\
14.59	0\\
14.6	0\\
14.61	0\\
14.62	0\\
14.63	0\\
14.64	0\\
14.65	0\\
14.66	0\\
14.67	0\\
14.68	0\\
14.69	0\\
14.7	0\\
14.71	0\\
14.72	0\\
14.73	0\\
14.74	0\\
14.75	0\\
14.76	0\\
14.77	0\\
14.78	0\\
14.79	0\\
14.8	0\\
14.81	0\\
14.82	0\\
14.83	0\\
14.84	0\\
14.85	0\\
14.86	0\\
14.87	0\\
14.88	0\\
14.89	0\\
14.9	0\\
14.91	0\\
14.92	0\\
14.93	0\\
14.94	0\\
14.95	0\\
14.96	0\\
14.97	0\\
14.98	0\\
14.99	0\\
15	0\\
15.01	0\\
15.02	0\\
15.03	0\\
15.04	0\\
15.05	0\\
15.06	0\\
15.07	0\\
15.08	0\\
15.09	0\\
15.1	0\\
15.11	0\\
15.12	0\\
15.13	0\\
15.14	0\\
15.15	0\\
15.16	0\\
15.17	0\\
15.18	0\\
15.19	0\\
15.2	0\\
15.21	0\\
15.22	0\\
15.23	0\\
15.24	0\\
15.25	0\\
15.26	0\\
15.27	0\\
15.28	0\\
15.29	0\\
15.3	0\\
15.31	0\\
15.32	0\\
15.33	0\\
15.34	0\\
15.35	0\\
15.36	0\\
15.37	0\\
15.38	0\\
15.39	0\\
15.4	0\\
15.41	0\\
15.42	0\\
15.43	0\\
15.44	0\\
15.45	0\\
15.46	0\\
15.47	0\\
15.48	0\\
15.49	0\\
15.5	0\\
15.51	0\\
15.52	0\\
15.53	0\\
15.54	0\\
15.55	0\\
15.56	0\\
15.57	0\\
15.58	0\\
15.59	0\\
15.6	0\\
15.61	0\\
15.62	0\\
15.63	0\\
15.64	0\\
15.65	0\\
15.66	0\\
15.67	0\\
15.68	0\\
15.69	0\\
15.7	0\\
15.71	0\\
15.72	0\\
15.73	0\\
15.74	0\\
15.75	0\\
15.76	0\\
15.77	0\\
15.78	0\\
15.79	0\\
15.8	0\\
15.81	0\\
15.82	0\\
15.83	0\\
15.84	0\\
15.85	0\\
15.86	0\\
15.87	0\\
15.88	0\\
15.89	0\\
15.9	0\\
15.91	0\\
15.92	0\\
15.93	0\\
15.94	0\\
15.95	0\\
15.96	0\\
15.97	0\\
15.98	0\\
15.99	0\\
16	0\\
16.01	0\\
16.02	0\\
16.03	0\\
16.04	0\\
16.05	0\\
16.06	0\\
16.07	0\\
16.08	0\\
16.09	0\\
16.1	0\\
16.11	0\\
16.12	0\\
16.13	0\\
16.14	0\\
16.15	0\\
16.16	0\\
16.17	0\\
16.18	0\\
16.19	0\\
16.2	0\\
16.21	0\\
16.22	0\\
16.23	0\\
16.24	0\\
16.25	0\\
16.26	0\\
16.27	0\\
16.28	0\\
16.29	0\\
16.3	0\\
16.31	0\\
16.32	0\\
16.33	0\\
16.34	0\\
16.35	0\\
16.36	0\\
16.37	0\\
16.38	0\\
16.39	0\\
16.4	0\\
16.41	0\\
16.42	0\\
16.43	0\\
16.44	0\\
16.45	0\\
16.46	0\\
16.47	0\\
16.48	0\\
16.49	0\\
16.5	0\\
16.51	0\\
16.52	0\\
16.53	0\\
16.54	0\\
16.55	0\\
16.56	0\\
16.57	0\\
16.58	0\\
16.59	0\\
16.6	0\\
16.61	0\\
16.62	0\\
16.63	0\\
16.64	0\\
16.65	0\\
16.66	0\\
16.67	0\\
16.68	0\\
16.69	0\\
16.7	0\\
16.71	0\\
16.72	0\\
16.73	0\\
16.74	0\\
16.75	0\\
16.76	0\\
16.77	0\\
16.78	0\\
16.79	0\\
16.8	0\\
16.81	0\\
16.82	0\\
16.83	0\\
16.84	0\\
16.85	0\\
16.86	0\\
16.87	0\\
16.88	0\\
16.89	0\\
16.9	0\\
16.91	0\\
16.92	0\\
16.93	0\\
16.94	0\\
16.95	0\\
16.96	0\\
16.97	0\\
16.98	0\\
16.99	0\\
17	0\\
17.01	0\\
17.02	0\\
17.03	0\\
17.04	0\\
17.05	0\\
17.06	0\\
17.07	0\\
17.08	0\\
17.09	0\\
17.1	0\\
17.11	0\\
17.12	0\\
17.13	0\\
17.14	0\\
17.15	0\\
17.16	0\\
17.17	0\\
17.18	0\\
17.19	0\\
17.2	0\\
17.21	0\\
17.22	0\\
17.23	0\\
17.24	0\\
17.25	0\\
17.26	0\\
17.27	0\\
17.28	0\\
17.29	0\\
17.3	0\\
17.31	0\\
17.32	0\\
17.33	0\\
17.34	0\\
17.35	0\\
17.36	0\\
17.37	0\\
17.38	0\\
17.39	0\\
17.4	0\\
17.41	0\\
17.42	0\\
17.43	0\\
17.44	0\\
17.45	0\\
17.46	0\\
17.47	0\\
17.48	0\\
17.49	0\\
17.5	0\\
17.51	0\\
17.52	0\\
17.53	0\\
17.54	0\\
17.55	0\\
17.56	0\\
17.57	0\\
17.58	0\\
17.59	0\\
17.6	0\\
17.61	0\\
17.62	0\\
17.63	0\\
17.64	0\\
17.65	0\\
17.66	0\\
17.67	0\\
17.68	0\\
17.69	0\\
17.7	0\\
17.71	0\\
17.72	0\\
17.73	0\\
17.74	0\\
17.75	0\\
17.76	0\\
17.77	0\\
17.78	0\\
17.79	0\\
17.8	0\\
17.81	0\\
17.82	0\\
17.83	0\\
17.84	0\\
17.85	0\\
17.86	0\\
17.87	0\\
17.88	0\\
17.89	0\\
17.9	0\\
17.91	0\\
17.92	0\\
17.93	0\\
17.94	0\\
17.95	0\\
17.96	0\\
17.97	0\\
17.98	0\\
17.99	0\\
18	0\\
18.01	0\\
18.02	0\\
18.03	0\\
18.04	0\\
18.05	0\\
18.06	0\\
18.07	0\\
18.08	0\\
18.09	0\\
18.1	0\\
18.11	0\\
18.12	0\\
18.13	0\\
18.14	0\\
18.15	0\\
18.16	0\\
18.17	0\\
18.18	0\\
18.19	0\\
18.2	0\\
18.21	0\\
18.22	0\\
18.23	0\\
18.24	0\\
18.25	0\\
18.26	0\\
18.27	0\\
18.28	0\\
18.29	0\\
18.3	0\\
18.31	0\\
18.32	0\\
18.33	0\\
18.34	0\\
18.35	0\\
18.36	0\\
18.37	0\\
18.38	0\\
18.39	0\\
18.4	0\\
18.41	0\\
18.42	0\\
18.43	0\\
18.44	0\\
18.45	0\\
18.46	0\\
18.47	0\\
18.48	0\\
18.49	0\\
18.5	0\\
18.51	0\\
18.52	0\\
18.53	0\\
18.54	0\\
18.55	0\\
18.56	0\\
18.57	0\\
18.58	0\\
18.59	0\\
18.6	0\\
18.61	0\\
18.62	0\\
18.63	0\\
18.64	0\\
18.65	0\\
18.66	0\\
18.67	0\\
18.68	0\\
18.69	0\\
18.7	0\\
18.71	0\\
18.72	0\\
18.73	0\\
18.74	0\\
18.75	0\\
18.76	0\\
18.77	0\\
18.78	0\\
18.79	0\\
18.8	0\\
18.81	0\\
18.82	0\\
18.83	0\\
18.84	0\\
18.85	0\\
18.86	0\\
18.87	0\\
18.88	0\\
18.89	0\\
18.9	0\\
18.91	0\\
18.92	0\\
18.93	0\\
18.94	0\\
18.95	0\\
18.96	0\\
18.97	0\\
18.98	0\\
18.99	0\\
19	0\\
19.01	0\\
19.02	0\\
19.03	0\\
19.04	0\\
19.05	0\\
19.06	0\\
19.07	0\\
19.08	0\\
19.09	0\\
19.1	0\\
19.11	0\\
19.12	0\\
19.13	0\\
19.14	0\\
19.15	0\\
19.16	0\\
19.17	0\\
19.18	0\\
19.19	0\\
19.2	0\\
19.21	0\\
19.22	0\\
19.23	0\\
19.24	0\\
19.25	0\\
19.26	0\\
19.27	0\\
19.28	0\\
19.29	0\\
19.3	0\\
19.31	0\\
19.32	0\\
19.33	0\\
19.34	0\\
19.35	0\\
19.36	0\\
19.37	0\\
19.38	0\\
19.39	0\\
19.4	0\\
19.41	0\\
19.42	0\\
19.43	0\\
19.44	0\\
19.45	0\\
19.46	0\\
19.47	0\\
19.48	0\\
19.49	0\\
19.5	0\\
19.51	0\\
19.52	0\\
19.53	0\\
19.54	0\\
19.55	0\\
19.56	0\\
19.57	0\\
19.58	0\\
19.59	0\\
19.6	0\\
19.61	0\\
19.62	0\\
19.63	0\\
19.64	0\\
19.65	0\\
19.66	0\\
19.67	0\\
19.68	0\\
19.69	0\\
19.7	0\\
19.71	0\\
19.72	0\\
19.73	0\\
19.74	0\\
19.75	0\\
19.76	0\\
19.77	0\\
19.78	0\\
19.79	0\\
19.8	0\\
19.81	0\\
19.82	0\\
19.83	0\\
19.84	0\\
19.85	0\\
19.86	0\\
19.87	0\\
19.88	0\\
19.89	0\\
19.9	0\\
19.91	0\\
19.92	0\\
19.93	0\\
19.94	0\\
19.95	0\\
19.96	0\\
19.97	0\\
19.98	0\\
19.99	0\\
20	0\\
20.01	0\\
20.02	0\\
20.03	0\\
20.04	0\\
20.05	0\\
20.06	0\\
20.07	0\\
20.08	0\\
20.09	0\\
20.1	0\\
20.11	0\\
20.12	0\\
20.13	0\\
20.14	0\\
20.15	0\\
20.16	0\\
20.17	0\\
20.18	0\\
20.19	0\\
20.2	0\\
20.21	0\\
20.22	0\\
20.23	0\\
20.24	0\\
20.25	0\\
20.26	0\\
20.27	0\\
20.28	0\\
20.29	0\\
20.3	0\\
20.31	0\\
20.32	0\\
20.33	0\\
20.34	0\\
20.35	0\\
20.36	0\\
20.37	0\\
20.38	0\\
20.39	0\\
20.4	0\\
20.41	0\\
20.42	0\\
20.43	0\\
20.44	0\\
20.45	0\\
20.46	0\\
20.47	0\\
20.48	0\\
20.49	0\\
20.5	0\\
20.51	0\\
20.52	0\\
20.53	0\\
20.54	0\\
20.55	0\\
20.56	0\\
20.57	0\\
20.58	0\\
20.59	0\\
20.6	0\\
20.61	0\\
20.62	0\\
20.63	0\\
20.64	0\\
20.65	0\\
20.66	0\\
20.67	0\\
20.68	0\\
20.69	0\\
20.7	0\\
20.71	0\\
20.72	0\\
20.73	0\\
20.74	0\\
20.75	0\\
20.76	0\\
20.77	0\\
20.78	0\\
20.79	0\\
20.8	0\\
20.81	0\\
20.82	0\\
20.83	0\\
20.84	0\\
20.85	0\\
20.86	0\\
20.87	0\\
20.88	0\\
20.89	0\\
20.9	0\\
20.91	0\\
20.92	0\\
20.93	0\\
20.94	0\\
20.95	0\\
20.96	0\\
20.97	0\\
20.98	0\\
20.99	0\\
21	0\\
21.01	0\\
21.02	0\\
21.03	0\\
21.04	0\\
21.05	0\\
21.06	0\\
21.07	0\\
21.08	0\\
21.09	0\\
21.1	0\\
21.11	0\\
21.12	0\\
21.13	0\\
21.14	0\\
21.15	0\\
21.16	0\\
21.17	0\\
21.18	0\\
21.19	0\\
21.2	0\\
21.21	0\\
21.22	0\\
21.23	0\\
21.24	0\\
21.25	0\\
21.26	0\\
21.27	0\\
21.28	0\\
21.29	0\\
21.3	0\\
21.31	0\\
21.32	0\\
21.33	0\\
21.34	0\\
21.35	0\\
21.36	0\\
21.37	0\\
21.38	0\\
21.39	0\\
21.4	0\\
21.41	0\\
21.42	0\\
21.43	0\\
21.44	0\\
21.45	0\\
21.46	0\\
21.47	0\\
21.48	0\\
21.49	0\\
21.5	0\\
21.51	0\\
21.52	0\\
21.53	0\\
21.54	0\\
21.55	0\\
21.56	0\\
21.57	0\\
21.58	0\\
21.59	0\\
21.6	0\\
21.61	0\\
21.62	0\\
21.63	0\\
21.64	0\\
21.65	0\\
21.66	0\\
21.67	0\\
21.68	0\\
21.69	0\\
21.7	0\\
21.71	0\\
21.72	0\\
21.73	0\\
21.74	0\\
21.75	0\\
21.76	0\\
21.77	0\\
21.78	0\\
21.79	0\\
21.8	0\\
21.81	0\\
21.82	0\\
21.83	0\\
21.84	0\\
21.85	0\\
21.86	0\\
21.87	0\\
21.88	0\\
21.89	0\\
21.9	0\\
21.91	0\\
21.92	0\\
21.93	0\\
21.94	0\\
21.95	0\\
21.96	0\\
21.97	0\\
21.98	0\\
21.99	0\\
22	0\\
22.01	0\\
22.02	0\\
22.03	0\\
22.04	0\\
22.05	0\\
22.06	0\\
22.07	0\\
22.08	0\\
22.09	0\\
22.1	0\\
22.11	0\\
22.12	0\\
22.13	0\\
22.14	0\\
22.15	0\\
22.16	0\\
22.17	0\\
22.18	0\\
22.19	0\\
22.2	0\\
22.21	0\\
22.22	0\\
22.23	0\\
22.24	0\\
22.25	0\\
22.26	0\\
22.27	0\\
22.28	0\\
22.29	0\\
22.3	0\\
22.31	0\\
22.32	0\\
22.33	0\\
22.34	0\\
22.35	0\\
22.36	0\\
22.37	0\\
22.38	0\\
22.39	0\\
22.4	0\\
22.41	0\\
22.42	0\\
22.43	0\\
22.44	0\\
22.45	0\\
22.46	0\\
22.47	0\\
22.48	0\\
22.49	0\\
22.5	0\\
22.51	0\\
22.52	0\\
22.53	0\\
22.54	0\\
22.55	0\\
22.56	0\\
22.57	0\\
22.58	0\\
22.59	0\\
22.6	0\\
22.61	0\\
22.62	0\\
22.63	0\\
22.64	0\\
22.65	0\\
22.66	0\\
22.67	0\\
22.68	0\\
22.69	0\\
22.7	0\\
22.71	0\\
22.72	0\\
22.73	0\\
22.74	0\\
22.75	0\\
22.76	0\\
22.77	0\\
22.78	0\\
22.79	0\\
22.8	0\\
22.81	0\\
22.82	0\\
22.83	0\\
22.84	0\\
22.85	0\\
22.86	0\\
22.87	0\\
22.88	0\\
22.89	0\\
22.9	0\\
22.91	0\\
22.92	0\\
22.93	0\\
22.94	0\\
22.95	0\\
22.96	0\\
22.97	0\\
22.98	0\\
22.99	0\\
23	0\\
23.01	0\\
23.02	0\\
23.03	0\\
23.04	0\\
23.05	0\\
23.06	0\\
23.07	0\\
23.08	0\\
23.09	0\\
23.1	0\\
23.11	0\\
23.12	0\\
23.13	0\\
23.14	0\\
23.15	0\\
23.16	0\\
23.17	0\\
23.18	0\\
23.19	0\\
23.2	0\\
23.21	0\\
23.22	0\\
23.23	0\\
23.24	0\\
23.25	0\\
23.26	0\\
23.27	0\\
23.28	0\\
23.29	0\\
23.3	0\\
23.31	0\\
23.32	0\\
23.33	0\\
23.34	0\\
23.35	0\\
23.36	0\\
23.37	0\\
23.38	0\\
23.39	0\\
23.4	0\\
23.41	0\\
23.42	0\\
23.43	0\\
23.44	0\\
23.45	0\\
23.46	0\\
23.47	0\\
23.48	0\\
23.49	0\\
23.5	0\\
23.51	0\\
23.52	0\\
23.53	0\\
23.54	0\\
23.55	0\\
23.56	0\\
23.57	0\\
23.58	0\\
23.59	0\\
23.6	0\\
23.61	0\\
23.62	0\\
23.63	0\\
23.64	0\\
23.65	0\\
23.66	0\\
23.67	0\\
23.68	0\\
23.69	0\\
23.7	0\\
23.71	0\\
23.72	0\\
23.73	0\\
23.74	0\\
23.75	0\\
23.76	0\\
23.77	0\\
23.78	0\\
23.79	0\\
23.8	0\\
23.81	0\\
23.82	0\\
23.83	0\\
23.84	0\\
23.85	0\\
23.86	0\\
23.87	0\\
23.88	0\\
23.89	0\\
23.9	0\\
23.91	0\\
23.92	0\\
23.93	0\\
23.94	0\\
23.95	0\\
23.96	0\\
23.97	0\\
23.98	0\\
23.99	0\\
24	0\\
24.01	0\\
24.02	0\\
24.03	0\\
24.04	0\\
24.05	0\\
24.06	0\\
24.07	0\\
24.08	0\\
24.09	0\\
24.1	0\\
24.11	0\\
24.12	0\\
24.13	0\\
24.14	0\\
24.15	0\\
24.16	0\\
24.17	0\\
24.18	0\\
24.19	0\\
24.2	0\\
24.21	0\\
24.22	0\\
24.23	0\\
24.24	0\\
24.25	0\\
24.26	0\\
24.27	0\\
24.28	0\\
24.29	0\\
24.3	0\\
24.31	0\\
24.32	0\\
24.33	0\\
24.34	0\\
24.35	0\\
24.36	0\\
24.37	0\\
24.38	0\\
24.39	0\\
24.4	0\\
24.41	0\\
24.42	0\\
24.43	0\\
24.44	0\\
24.45	0\\
24.46	0\\
24.47	0\\
24.48	0\\
24.49	0\\
24.5	0\\
24.51	0\\
24.52	0\\
24.53	0\\
24.54	0\\
24.55	0\\
24.56	0\\
24.57	0\\
24.58	0\\
24.59	0\\
24.6	0\\
24.61	0\\
24.62	0\\
24.63	0\\
24.64	0\\
24.65	0\\
24.66	0\\
24.67	0\\
24.68	0\\
24.69	0\\
24.7	0\\
24.71	0\\
24.72	0\\
24.73	0\\
24.74	0\\
24.75	0\\
24.76	0\\
24.77	0\\
24.78	0\\
24.79	0\\
24.8	0\\
24.81	0\\
24.82	0\\
24.83	0\\
24.84	0\\
24.85	0\\
24.86	0\\
24.87	0\\
24.88	0\\
24.89	0\\
24.9	0\\
24.91	0\\
24.92	0\\
24.93	0\\
24.94	0\\
24.95	0\\
24.96	0\\
24.97	0\\
24.98	0\\
24.99	0\\
25	0\\
25.01	0\\
25.02	0\\
25.03	0\\
25.04	0\\
25.05	0\\
25.06	0\\
25.07	0\\
25.08	0\\
25.09	0\\
25.1	0\\
25.11	0\\
25.12	0\\
25.13	0\\
25.14	0\\
25.15	0\\
25.16	0\\
25.17	0\\
25.18	0\\
25.19	0\\
25.2	0\\
25.21	0\\
25.22	0\\
25.23	0\\
25.24	0\\
25.25	0\\
25.26	0\\
25.27	0\\
25.28	0\\
25.29	0\\
25.3	0\\
25.31	0\\
25.32	0\\
25.33	0\\
25.34	0\\
25.35	0\\
25.36	0\\
25.37	0\\
25.38	0\\
25.39	0\\
25.4	0\\
25.41	0\\
25.42	0\\
25.43	0\\
25.44	0\\
25.45	0\\
25.46	0\\
25.47	0\\
25.48	0\\
25.49	0\\
25.5	0\\
25.51	0\\
25.52	0\\
25.53	0\\
25.54	0\\
25.55	0\\
25.56	0\\
25.57	0\\
25.58	0\\
25.59	0\\
25.6	0\\
25.61	0\\
25.62	0\\
25.63	0\\
25.64	0\\
25.65	0\\
25.66	0\\
25.67	0\\
25.68	0\\
25.69	0\\
25.7	0\\
25.71	0\\
25.72	0\\
25.73	0\\
25.74	0\\
25.75	0\\
25.76	0\\
25.77	0\\
25.78	0\\
25.79	0\\
25.8	0\\
25.81	0\\
25.82	0\\
25.83	0\\
25.84	0\\
25.85	0\\
25.86	0\\
25.87	0\\
25.88	0\\
25.89	0\\
25.9	0\\
25.91	0\\
25.92	0\\
25.93	0\\
25.94	0\\
25.95	0\\
25.96	0\\
25.97	0\\
25.98	0\\
25.99	0\\
26	0\\
26.01	0\\
26.02	0\\
26.03	0\\
26.04	0\\
26.05	0\\
26.06	0\\
26.07	0\\
26.08	0\\
26.09	0\\
26.1	0\\
26.11	0\\
26.12	0\\
26.13	0\\
26.14	0\\
26.15	0\\
26.16	0\\
26.17	0\\
26.18	0\\
26.19	0\\
26.2	0\\
26.21	0\\
26.22	0\\
26.23	0\\
26.24	0\\
26.25	0\\
26.26	0\\
26.27	0\\
26.28	0\\
26.29	0\\
26.3	0\\
26.31	0\\
26.32	0\\
26.33	0\\
26.34	0\\
26.35	0\\
26.36	0\\
26.37	0\\
26.38	0\\
26.39	0\\
26.4	0\\
26.41	0\\
26.42	0\\
26.43	0\\
26.44	0\\
26.45	0\\
26.46	0\\
26.47	0\\
26.48	0\\
26.49	0\\
26.5	0\\
26.51	0\\
26.52	0\\
26.53	0\\
26.54	0\\
26.55	0\\
26.56	0\\
26.57	0\\
26.58	0\\
26.59	0\\
26.6	0\\
26.61	0\\
26.62	0\\
26.63	0\\
26.64	0\\
26.65	0\\
26.66	0\\
26.67	0\\
26.68	0\\
26.69	0\\
26.7	0\\
26.71	0\\
26.72	0\\
26.73	0\\
26.74	0\\
26.75	0\\
26.76	0\\
26.77	0\\
26.78	0\\
26.79	0\\
26.8	0\\
26.81	0\\
26.82	0\\
26.83	0\\
26.84	0\\
26.85	0\\
26.86	0\\
26.87	0\\
26.88	0\\
26.89	0\\
26.9	0\\
26.91	0\\
26.92	0\\
26.93	0\\
26.94	0\\
26.95	0\\
26.96	0\\
26.97	0\\
26.98	0\\
26.99	0\\
27	0\\
27.01	0\\
27.02	0\\
27.03	0\\
27.04	0\\
27.05	0\\
27.06	0\\
27.07	0\\
27.08	0\\
27.09	0\\
27.1	0\\
27.11	0\\
27.12	0\\
27.13	0\\
27.14	0\\
27.15	0\\
27.16	0\\
27.17	0\\
27.18	0\\
27.19	0\\
27.2	0\\
27.21	0\\
27.22	0\\
27.23	0\\
27.24	0\\
27.25	0\\
27.26	0\\
27.27	0\\
27.28	0\\
27.29	0\\
27.3	0\\
27.31	0\\
27.32	0\\
27.33	0\\
27.34	0\\
27.35	0\\
27.36	0\\
27.37	0\\
27.38	0\\
27.39	0\\
27.4	0\\
27.41	0\\
27.42	0\\
27.43	0\\
27.44	0\\
27.45	0\\
27.46	0\\
27.47	0\\
27.48	0\\
27.49	0\\
27.5	0\\
27.51	0\\
27.52	0\\
27.53	0\\
27.54	0\\
27.55	0\\
27.56	0\\
27.57	0\\
27.58	0\\
27.59	0\\
27.6	0\\
27.61	0\\
27.62	0\\
27.63	0\\
27.64	0\\
27.65	0\\
27.66	0\\
27.67	0\\
27.68	0\\
27.69	0\\
27.7	0\\
27.71	0\\
27.72	0\\
27.73	0\\
27.74	0\\
27.75	0\\
27.76	0\\
27.77	0\\
27.78	0\\
27.79	0\\
27.8	0\\
27.81	0\\
27.82	0\\
27.83	0\\
27.84	0\\
27.85	0\\
27.86	0\\
27.87	0\\
27.88	0\\
27.89	0\\
27.9	0\\
27.91	0\\
27.92	0\\
27.93	0\\
27.94	0\\
27.95	0\\
27.96	0\\
27.97	0\\
27.98	0\\
27.99	0\\
28	0\\
28.01	0\\
28.02	0\\
28.03	0\\
28.04	0\\
28.05	0\\
28.06	0\\
28.07	0\\
28.08	0\\
28.09	0\\
28.1	0\\
28.11	0\\
28.12	0\\
28.13	0\\
28.14	0\\
28.15	0\\
28.16	0\\
28.17	0\\
28.18	0\\
28.19	0\\
28.2	0\\
28.21	0\\
28.22	0\\
28.23	0\\
28.24	0\\
28.25	0\\
28.26	0\\
28.27	0\\
28.28	0\\
28.29	0\\
28.3	0\\
28.31	0\\
28.32	0\\
28.33	0\\
28.34	0\\
28.35	0\\
28.36	0\\
28.37	0\\
28.38	0\\
28.39	0\\
28.4	0\\
28.41	0\\
28.42	0\\
28.43	0\\
28.44	0\\
28.45	0\\
28.46	0\\
28.47	0\\
28.48	0\\
28.49	0\\
28.5	0\\
28.51	0\\
28.52	0\\
28.53	0\\
28.54	0\\
28.55	0\\
28.56	0\\
28.57	0\\
28.58	0\\
28.59	0\\
28.6	0\\
28.61	0\\
28.62	0\\
28.63	0\\
28.64	0\\
28.65	0\\
28.66	0\\
28.67	0\\
28.68	0\\
28.69	0\\
28.7	0\\
28.71	0\\
28.72	0\\
28.73	0\\
28.74	0\\
28.75	0\\
28.76	0\\
28.77	0\\
28.78	0\\
28.79	0\\
28.8	0\\
28.81	0\\
28.82	0\\
28.83	0\\
28.84	0\\
28.85	0\\
28.86	0\\
28.87	0\\
28.88	0\\
28.89	0\\
28.9	0\\
28.91	0\\
28.92	0\\
28.93	0\\
28.94	0\\
28.95	0\\
28.96	0\\
28.97	0\\
28.98	0\\
28.99	0\\
29	0\\
29.01	0\\
29.02	0\\
29.03	0\\
29.04	0\\
29.05	0\\
29.06	0\\
29.07	0\\
29.08	0\\
29.09	0\\
29.1	0\\
29.11	0\\
29.12	0\\
29.13	0\\
29.14	0\\
29.15	0\\
29.16	0\\
29.17	0\\
29.18	0\\
29.19	0\\
29.2	0\\
29.21	0\\
29.22	0\\
29.23	0\\
29.24	0\\
29.25	0\\
29.26	0\\
29.27	0\\
29.28	0\\
29.29	0\\
29.3	0\\
29.31	0\\
29.32	0\\
29.33	0\\
29.34	0\\
29.35	0\\
29.36	0\\
29.37	0\\
29.38	0\\
29.39	0\\
29.4	0\\
29.41	0\\
29.42	0\\
29.43	0\\
29.44	0\\
29.45	0\\
29.46	0\\
29.47	0\\
29.48	0\\
29.49	0\\
29.5	0\\
29.51	0\\
29.52	0\\
29.53	0\\
29.54	0\\
29.55	0\\
29.56	0\\
29.57	0\\
29.58	0\\
29.59	0\\
29.6	0\\
29.61	0\\
29.62	0\\
29.63	0\\
29.64	0\\
29.65	0\\
29.66	0\\
29.67	0\\
29.68	0\\
29.69	0\\
29.7	0\\
29.71	0\\
29.72	0\\
29.73	0\\
29.74	0\\
29.75	0\\
29.76	0\\
29.77	0\\
29.78	0\\
29.79	0\\
29.8	0\\
29.81	0\\
29.82	0\\
29.83	0\\
29.84	0\\
29.85	0\\
29.86	0\\
29.87	0\\
29.88	0\\
29.89	0\\
29.9	0\\
29.91	0\\
29.92	0\\
29.93	0\\
29.94	0\\
29.95	0\\
29.96	0\\
29.97	0\\
29.98	0\\
29.99	0\\
30	0\\
30.01	0\\
30.02	0\\
30.03	0\\
30.04	0\\
30.05	0\\
30.06	0\\
30.07	0\\
30.08	0\\
30.09	0\\
30.1	0\\
30.11	0\\
30.12	0\\
30.13	0\\
30.14	0\\
30.15	0\\
30.16	0\\
30.17	0\\
30.18	0\\
30.19	0\\
30.2	0\\
30.21	0\\
30.22	0\\
30.23	0\\
30.24	0\\
30.25	0\\
30.26	0\\
30.27	0\\
30.28	0\\
30.29	0\\
30.3	0\\
30.31	0\\
30.32	0\\
30.33	0\\
30.34	0\\
30.35	0\\
30.36	0\\
30.37	0\\
30.38	0\\
30.39	0\\
30.4	0\\
30.41	0\\
30.42	0\\
30.43	0\\
30.44	0\\
30.45	0\\
30.46	0\\
30.47	0\\
30.48	0\\
30.49	0\\
30.5	0\\
30.51	0\\
30.52	0\\
30.53	0\\
30.54	0\\
30.55	0\\
30.56	0\\
30.57	0\\
30.58	0\\
30.59	0\\
30.6	0\\
30.61	0\\
30.62	0\\
30.63	0\\
30.64	0\\
30.65	0\\
30.66	0\\
30.67	0\\
30.68	0\\
30.69	0\\
30.7	0\\
30.71	0\\
30.72	0\\
30.73	0\\
30.74	0\\
30.75	0\\
30.76	0\\
30.77	0\\
30.78	0\\
30.79	0\\
30.8	0\\
30.81	0\\
30.82	0\\
30.83	0\\
30.84	0\\
30.85	0\\
30.86	0\\
30.87	0\\
30.88	0\\
30.89	0\\
30.9	0\\
30.91	0\\
30.92	0\\
30.93	0\\
30.94	0\\
30.95	0\\
30.96	0\\
30.97	0\\
30.98	0\\
30.99	0\\
31	0\\
31.01	0\\
31.02	0\\
31.03	0\\
31.04	0\\
31.05	0\\
31.06	0\\
31.07	0\\
31.08	0\\
31.09	0\\
31.1	0\\
31.11	0\\
31.12	0\\
31.13	0\\
31.14	0\\
31.15	0\\
31.16	0\\
31.17	0\\
31.18	0\\
31.19	0\\
31.2	0\\
31.21	0\\
31.22	0\\
31.23	0\\
31.24	0\\
31.25	0\\
31.26	0\\
31.27	0\\
31.28	0\\
31.29	0\\
31.3	0\\
31.31	0\\
31.32	0\\
31.33	0\\
31.34	0\\
31.35	0\\
31.36	0\\
31.37	0\\
31.38	0\\
31.39	0\\
31.4	0\\
31.41	0\\
31.42	0\\
31.43	0\\
31.44	0\\
31.45	0\\
31.46	0\\
31.47	0\\
31.48	0\\
31.49	0\\
31.5	0\\
31.51	0\\
31.52	0\\
31.53	0\\
31.54	0\\
31.55	0\\
31.56	0\\
31.57	0\\
31.58	0\\
31.59	0\\
31.6	0\\
31.61	0\\
31.62	0\\
31.63	0\\
31.64	0\\
31.65	0\\
31.66	0\\
31.67	0\\
31.68	0\\
31.69	0\\
31.7	0\\
31.71	0\\
31.72	0\\
31.73	0\\
31.74	0\\
31.75	0\\
31.76	0\\
31.77	0\\
31.78	0\\
31.79	0\\
31.8	0\\
31.81	0\\
31.82	0\\
31.83	0\\
31.84	0\\
31.85	0\\
31.86	0\\
31.87	0\\
31.88	0\\
31.89	0\\
31.9	0\\
31.91	0\\
31.92	0\\
31.93	0\\
31.94	0\\
31.95	0\\
31.96	0\\
31.97	0\\
31.98	0\\
31.99	0\\
32	0\\
32.01	0\\
32.02	0\\
32.03	0\\
32.04	0\\
32.05	0\\
32.06	0\\
32.07	0\\
32.08	0\\
32.09	0\\
32.1	0\\
32.11	0\\
32.12	0\\
32.13	0\\
32.14	0\\
32.15	0\\
32.16	0\\
32.17	0\\
32.18	0\\
32.19	0\\
32.2	0\\
32.21	0\\
32.22	0\\
32.23	0\\
32.24	0\\
32.25	0\\
32.26	0\\
32.27	0\\
32.28	0\\
32.29	0\\
32.3	0\\
32.31	0\\
32.32	0\\
32.33	0\\
32.34	0\\
32.35	0\\
32.36	0\\
32.37	0\\
32.38	0\\
32.39	0\\
32.4	0\\
32.41	0\\
32.42	0\\
32.43	0\\
32.44	0\\
32.45	0\\
32.46	0\\
32.47	0\\
32.48	0\\
32.49	0\\
32.5	0\\
32.51	0\\
32.52	0\\
32.53	0\\
32.54	0\\
32.55	0\\
32.56	0\\
32.57	0\\
32.58	0\\
32.59	0\\
32.6	0\\
32.61	0\\
32.62	0\\
32.63	0\\
32.64	0\\
32.65	0\\
32.66	0\\
32.67	0\\
32.68	0\\
32.69	0\\
32.7	0\\
32.71	0\\
32.72	0\\
32.73	0\\
32.74	0\\
32.75	0\\
32.76	0\\
32.77	0\\
32.78	0\\
32.79	0\\
32.8	0\\
32.81	0\\
32.82	0\\
32.83	0\\
32.84	0\\
32.85	0\\
32.86	0\\
32.87	0\\
32.88	0\\
32.89	0\\
32.9	0\\
32.91	0\\
32.92	0\\
32.93	0\\
32.94	0\\
32.95	0\\
32.96	0\\
32.97	0\\
32.98	0\\
32.99	0\\
33	0\\
33.01	0\\
33.02	0\\
33.03	0\\
33.04	0\\
33.05	0\\
33.06	0\\
33.07	0\\
33.08	0\\
33.09	0\\
33.1	0\\
33.11	0\\
33.12	0\\
33.13	0\\
33.14	0\\
33.15	0\\
33.16	0\\
33.17	0\\
33.18	0\\
33.19	0\\
33.2	0\\
33.21	0\\
33.22	0\\
33.23	0\\
33.24	0\\
33.25	0\\
33.26	0\\
33.27	0\\
33.28	0\\
33.29	0\\
33.3	0\\
33.31	0\\
33.32	0\\
33.33	0\\
33.34	0\\
33.35	0\\
33.36	0\\
33.37	0\\
33.38	0\\
33.39	0\\
33.4	0\\
33.41	0\\
33.42	0\\
33.43	0\\
33.44	0\\
33.45	0\\
33.46	0\\
33.47	0\\
33.48	0\\
33.49	0\\
33.5	0\\
33.51	0\\
33.52	0\\
33.53	0\\
33.54	0\\
33.55	0\\
33.56	0\\
33.57	0\\
33.58	0\\
33.59	0\\
33.6	0\\
33.61	0\\
33.62	0\\
33.63	0\\
33.64	0\\
33.65	0\\
33.66	0\\
33.67	0\\
33.68	0\\
33.69	0\\
33.7	0\\
33.71	0\\
33.72	0\\
33.73	0\\
33.74	0\\
33.75	0\\
33.76	0\\
33.77	0\\
33.78	0\\
33.79	0\\
33.8	0\\
33.81	0\\
33.82	0\\
33.83	0\\
33.84	0\\
33.85	0\\
33.86	0\\
33.87	0\\
33.88	0\\
33.89	0\\
33.9	0\\
33.91	0\\
33.92	0\\
33.93	0\\
33.94	0\\
33.95	0\\
33.96	0\\
33.97	0\\
33.98	0\\
33.99	0\\
34	0\\
34.01	0\\
34.02	0\\
34.03	0\\
34.04	0\\
34.05	0\\
34.06	0\\
34.07	0\\
34.08	0\\
34.09	0\\
34.1	0\\
34.11	0\\
34.12	0\\
34.13	0\\
34.14	0\\
34.15	0\\
34.16	0\\
34.17	0\\
34.18	0\\
34.19	0\\
34.2	0\\
34.21	0\\
34.22	0\\
34.23	0\\
34.24	0\\
34.25	0\\
34.26	0\\
34.27	0\\
34.28	0\\
34.29	0\\
34.3	0\\
34.31	0\\
34.32	0\\
34.33	0\\
34.34	0\\
34.35	0\\
34.36	0\\
34.37	0\\
34.38	0\\
34.39	0\\
34.4	0\\
34.41	0\\
34.42	0\\
34.43	0\\
34.44	0\\
34.45	0\\
34.46	0\\
34.47	0\\
34.48	0\\
34.49	0\\
34.5	0\\
34.51	0\\
34.52	0\\
34.53	0\\
34.54	0\\
34.55	0\\
34.56	0\\
34.57	0\\
34.58	0\\
34.59	0\\
34.6	0\\
34.61	0\\
34.62	0\\
34.63	0\\
34.64	0\\
34.65	0\\
34.66	0\\
34.67	0\\
34.68	0\\
34.69	0\\
34.7	0\\
34.71	0\\
34.72	0\\
34.73	0\\
34.74	0\\
34.75	0\\
34.76	0\\
34.77	0\\
34.78	0\\
34.79	0\\
34.8	0\\
34.81	0\\
34.82	0\\
34.83	0\\
34.84	0\\
34.85	0\\
34.86	0\\
34.87	0\\
34.88	0\\
34.89	0\\
34.9	0\\
34.91	0\\
34.92	0\\
34.93	0\\
34.94	0\\
34.95	0\\
34.96	0\\
34.97	0\\
34.98	0\\
34.99	0\\
35	0\\
35.01	0\\
35.02	0\\
35.03	0\\
35.04	0\\
35.05	0\\
35.06	0\\
35.07	0\\
35.08	0\\
35.09	0\\
35.1	0\\
35.11	0\\
35.12	0\\
35.13	0\\
35.14	0\\
35.15	0\\
35.16	0\\
35.17	0\\
35.18	0\\
35.19	0\\
35.2	0\\
35.21	0\\
35.22	0\\
35.23	0\\
35.24	0\\
35.25	0\\
35.26	0\\
35.27	0\\
35.28	0\\
35.29	0\\
35.3	0\\
35.31	0\\
35.32	0\\
35.33	0\\
35.34	0\\
35.35	0\\
35.36	0\\
35.37	0\\
35.38	0\\
35.39	0\\
35.4	0\\
35.41	0\\
35.42	0\\
35.43	0\\
35.44	0\\
35.45	0\\
35.46	0\\
35.47	0\\
35.48	0\\
35.49	0\\
35.5	0\\
35.51	0\\
35.52	0\\
35.53	0\\
35.54	0\\
35.55	0\\
35.56	0\\
35.57	0\\
35.58	0\\
35.59	0\\
35.6	0\\
35.61	0\\
35.62	0\\
35.63	0\\
35.64	0\\
35.65	0\\
35.66	0\\
35.67	0\\
35.68	0\\
35.69	0\\
35.7	0\\
35.71	0\\
35.72	0\\
35.73	0\\
35.74	0\\
35.75	0\\
35.76	0\\
35.77	0\\
35.78	0\\
35.79	0\\
35.8	0\\
35.81	0\\
35.82	0\\
35.83	0\\
35.84	0\\
35.85	0\\
35.86	0\\
35.87	0\\
35.88	0\\
35.89	0\\
35.9	0\\
35.91	0\\
35.92	0\\
35.93	0\\
35.94	0\\
35.95	0\\
35.96	0\\
35.97	0\\
35.98	0\\
35.99	0\\
36	0\\
36.01	0\\
36.02	0\\
36.03	0\\
36.04	0\\
36.05	0\\
36.06	0\\
36.07	0\\
36.08	0\\
36.09	0\\
36.1	0\\
36.11	0\\
36.12	0\\
36.13	0\\
36.14	0\\
36.15	0\\
36.16	0\\
36.17	0\\
36.18	0\\
36.19	0\\
36.2	0\\
36.21	0\\
36.22	0\\
36.23	0\\
36.24	0\\
36.25	0\\
36.26	0\\
36.27	0\\
36.28	0\\
36.29	0\\
36.3	0\\
36.31	0\\
36.32	0\\
36.33	0\\
36.34	0\\
36.35	0\\
36.36	0\\
36.37	0\\
36.38	0\\
36.39	0\\
36.4	0\\
36.41	0\\
36.42	0\\
36.43	0\\
36.44	0\\
36.45	0\\
36.46	0\\
36.47	0\\
36.48	0\\
36.49	0\\
36.5	0\\
36.51	0\\
36.52	0\\
36.53	0\\
36.54	0\\
36.55	0\\
36.56	0\\
36.57	0\\
36.58	0\\
36.59	0\\
36.6	0\\
36.61	0\\
36.62	0\\
36.63	0\\
36.64	0\\
36.65	0\\
36.66	0\\
36.67	0\\
36.68	0\\
36.69	0\\
36.7	0\\
36.71	0\\
36.72	0\\
36.73	0\\
36.74	0\\
36.75	0\\
36.76	0\\
36.77	0\\
36.78	0\\
36.79	0\\
36.8	0\\
36.81	0\\
36.82	0\\
36.83	0\\
36.84	0\\
36.85	0\\
36.86	0\\
36.87	0\\
36.88	0\\
36.89	0\\
36.9	0\\
36.91	0\\
36.92	0\\
36.93	0\\
36.94	0\\
36.95	0\\
36.96	0\\
36.97	0\\
36.98	0\\
36.99	0\\
37	0\\
37.01	0\\
37.02	0\\
37.03	0\\
37.04	0\\
37.05	0\\
37.06	0\\
37.07	0\\
37.08	0\\
37.09	0\\
37.1	0\\
37.11	0\\
37.12	0\\
37.13	0\\
37.14	0\\
37.15	0\\
37.16	0\\
37.17	0\\
37.18	0\\
37.19	0\\
37.2	0\\
37.21	0\\
37.22	0\\
37.23	0\\
37.24	0\\
37.25	0\\
37.26	0\\
37.27	0\\
37.28	0\\
37.29	0\\
37.3	0\\
37.31	0\\
37.32	0\\
37.33	0\\
37.34	0\\
37.35	0\\
37.36	0\\
37.37	0\\
37.38	0\\
37.39	0\\
37.4	0\\
37.41	0\\
37.42	0\\
37.43	0\\
37.44	0\\
37.45	0\\
37.46	0\\
37.47	0\\
37.48	0\\
37.49	0\\
37.5	0\\
37.51	0\\
37.52	0\\
37.53	0\\
37.54	0\\
37.55	0\\
37.56	0\\
37.57	0\\
37.58	0\\
37.59	0\\
37.6	0\\
37.61	0\\
37.62	0\\
37.63	0\\
37.64	0\\
37.65	0\\
37.66	0\\
37.67	0\\
37.68	0\\
37.69	0\\
37.7	0\\
37.71	0\\
37.72	0\\
37.73	0\\
37.74	0\\
37.75	0\\
37.76	0\\
37.77	0\\
37.78	0\\
37.79	0\\
37.8	0\\
37.81	0\\
37.82	0\\
37.83	0\\
37.84	0\\
37.85	0\\
37.86	0\\
37.87	0\\
37.88	0\\
37.89	0\\
37.9	0\\
37.91	0\\
37.92	0\\
37.93	0\\
37.94	0\\
37.95	0\\
37.96	0\\
37.97	0\\
37.98	0\\
37.99	0\\
38	0\\
38.01	0\\
38.02	0\\
38.03	0\\
38.04	0\\
38.05	0\\
38.06	0\\
38.07	0\\
38.08	0\\
38.09	0\\
38.1	0\\
38.11	0\\
38.12	0\\
38.13	0\\
38.14	0\\
38.15	0\\
38.16	0\\
38.17	0\\
38.18	0\\
38.19	0\\
38.2	0\\
38.21	0\\
38.22	0\\
38.23	0\\
38.24	0\\
38.25	0\\
38.26	0\\
38.27	0\\
38.28	0\\
38.29	0\\
38.3	0\\
38.31	0\\
38.32	0\\
38.33	0\\
38.34	0\\
38.35	0\\
38.36	0\\
38.37	0\\
38.38	0\\
38.39	0\\
38.4	0\\
38.41	0\\
38.42	0\\
38.43	0\\
38.44	0\\
38.45	0\\
38.46	0\\
38.47	0\\
38.48	0\\
38.49	0\\
38.5	0\\
38.51	0\\
38.52	0\\
38.53	0\\
38.54	0\\
38.55	0\\
38.56	0\\
38.57	0\\
38.58	0\\
38.59	0\\
38.6	0\\
38.61	0\\
38.62	0\\
38.63	0\\
38.64	0\\
38.65	0\\
38.66	0\\
38.67	0\\
38.68	0\\
38.69	0\\
38.7	0\\
38.71	0\\
38.72	0\\
38.73	0\\
38.74	0\\
38.75	0\\
38.76	0\\
38.77	0\\
38.78	0\\
38.79	0\\
38.8	0\\
38.81	0\\
38.82	0\\
38.83	0\\
38.84	0\\
38.85	0\\
38.86	0\\
38.87	0\\
38.88	0\\
38.89	0\\
38.9	0\\
38.91	0\\
38.92	0\\
38.93	0\\
38.94	0\\
38.95	0\\
38.96	0\\
38.97	0\\
38.98	0\\
38.99	0\\
39	0\\
39.01	0\\
39.02	0\\
39.03	0\\
39.04	0\\
39.05	0\\
39.06	0\\
39.07	0\\
39.08	0\\
39.09	0\\
39.1	0\\
39.11	0\\
39.12	0\\
39.13	0\\
39.14	0\\
39.15	0\\
39.16	0\\
39.17	0\\
39.18	0\\
39.19	0\\
39.2	0\\
39.21	0\\
39.22	0\\
39.23	0\\
39.24	0\\
39.25	0\\
39.26	0\\
39.27	0\\
39.28	0\\
39.29	0\\
39.3	0\\
39.31	0\\
39.32	0\\
39.33	0\\
39.34	0\\
39.35	0\\
39.36	0\\
39.37	0\\
39.38	0\\
39.39	0\\
39.4	0\\
39.41	0\\
39.42	0\\
39.43	0\\
39.44	0\\
39.45	0\\
39.46	0\\
39.47	0\\
39.48	0\\
39.49	0\\
39.5	0\\
39.51	0\\
39.52	0\\
39.53	0\\
39.54	0\\
39.55	0\\
39.56	0\\
39.57	0\\
39.58	0\\
39.59	0\\
39.6	0\\
39.61	0\\
39.62	0\\
39.63	0\\
39.64	0\\
39.65	0\\
39.66	0\\
39.67	0\\
39.68	0\\
39.69	0\\
39.7	0\\
39.71	0\\
39.72	0\\
39.73	0\\
39.74	0\\
39.75	0\\
39.76	0\\
39.77	0\\
39.78	0\\
39.79	0\\
39.8	0\\
39.81	0\\
39.82	0\\
39.83	0\\
39.84	0\\
39.85	0\\
39.86	0\\
39.87	0\\
39.88	0\\
39.89	0\\
39.9	0\\
39.91	0\\
39.92	0\\
39.93	0\\
39.94	0\\
39.95	0\\
39.96	0\\
39.97	0\\
39.98	0\\
39.99	0\\
40	0\\
40.01	0\\
};
\addplot [color=mycolor1,dashed,forget plot]
  table[row sep=crcr]{%
40.01	0\\
40.02	0\\
40.03	0\\
40.04	0\\
40.05	0\\
40.06	0\\
40.07	0\\
40.08	0\\
40.09	0\\
40.1	0\\
40.11	0\\
40.12	0\\
40.13	0\\
40.14	0\\
40.15	0\\
40.16	0\\
40.17	0\\
40.18	0\\
40.19	0\\
40.2	0\\
40.21	0\\
40.22	0\\
40.23	0\\
40.24	0\\
40.25	0\\
40.26	0\\
40.27	0\\
40.28	0\\
40.29	0\\
40.3	0\\
40.31	0\\
40.32	0\\
40.33	0\\
40.34	0\\
40.35	0\\
40.36	0\\
40.37	0\\
40.38	0\\
40.39	0\\
40.4	0\\
40.41	0\\
40.42	0\\
40.43	0\\
40.44	0\\
40.45	0\\
40.46	0\\
40.47	0\\
40.48	0\\
40.49	0\\
40.5	0\\
40.51	0\\
40.52	0\\
40.53	0\\
40.54	0\\
40.55	0\\
40.56	0\\
40.57	0\\
40.58	0\\
40.59	0\\
40.6	0\\
40.61	0\\
40.62	0\\
40.63	0\\
40.64	0\\
40.65	0\\
40.66	0\\
40.67	0\\
40.68	0\\
40.69	0\\
40.7	0\\
40.71	0\\
40.72	0\\
40.73	0\\
40.74	0\\
40.75	0\\
40.76	0\\
40.77	0\\
40.78	0\\
40.79	0\\
40.8	0\\
40.81	0\\
40.82	0\\
40.83	0\\
40.84	0\\
40.85	0\\
40.86	0\\
40.87	0\\
40.88	0\\
40.89	0\\
40.9	0\\
40.91	0\\
40.92	0\\
40.93	0\\
40.94	0\\
40.95	0\\
40.96	0\\
40.97	0\\
40.98	0\\
40.99	0\\
41	0\\
41.01	0\\
41.02	0\\
41.03	0\\
41.04	0\\
41.05	0\\
41.06	0\\
41.07	0\\
41.08	0\\
41.09	0\\
41.1	0\\
41.11	0\\
41.12	0\\
41.13	0\\
41.14	0\\
41.15	0\\
41.16	0\\
41.17	0\\
41.18	0\\
41.19	0\\
41.2	0\\
41.21	0\\
41.22	0\\
41.23	0\\
41.24	0\\
41.25	0\\
41.26	0\\
41.27	0\\
41.28	0\\
41.29	0\\
41.3	0\\
41.31	0\\
41.32	0\\
41.33	0\\
41.34	0\\
41.35	0\\
41.36	0\\
41.37	0\\
41.38	0\\
41.39	0\\
41.4	0\\
41.41	0\\
41.42	0\\
41.43	0\\
41.44	0\\
41.45	0\\
41.46	0\\
41.47	0\\
41.48	0\\
41.49	0\\
41.5	0\\
41.51	0\\
41.52	0\\
41.53	0\\
41.54	0\\
41.55	0\\
41.56	0\\
41.57	0\\
41.58	0\\
41.59	0\\
41.6	0\\
41.61	0\\
41.62	0\\
41.63	0\\
41.64	0\\
41.65	0\\
41.66	0\\
41.67	0\\
41.68	0\\
41.69	0\\
41.7	0\\
41.71	0\\
41.72	0\\
41.73	0\\
41.74	0\\
41.75	0\\
41.76	0\\
41.77	0\\
41.78	0\\
41.79	0\\
41.8	0\\
41.81	0\\
41.82	0\\
41.83	0\\
41.84	0\\
41.85	0\\
41.86	0\\
41.87	0\\
41.88	0\\
41.89	0\\
41.9	0\\
41.91	0\\
41.92	0\\
41.93	0\\
41.94	0\\
41.95	0\\
41.96	0\\
41.97	0\\
41.98	0\\
41.99	0\\
42	0\\
42.01	0\\
42.02	0\\
42.03	0\\
42.04	0\\
42.05	0\\
42.06	0\\
42.07	0\\
42.08	0\\
42.09	0\\
42.1	0\\
42.11	0\\
42.12	0\\
42.13	0\\
42.14	0\\
42.15	0\\
42.16	0\\
42.17	0\\
42.18	0\\
42.19	0\\
42.2	0\\
42.21	0\\
42.22	0\\
42.23	0\\
42.24	0\\
42.25	0\\
42.26	0\\
42.27	0\\
42.28	0\\
42.29	0\\
42.3	0\\
42.31	0\\
42.32	0\\
42.33	0\\
42.34	0\\
42.35	0\\
42.36	0\\
42.37	0\\
42.38	0\\
42.39	0\\
42.4	0\\
42.41	0\\
42.42	0\\
42.43	0\\
42.44	0\\
42.45	0\\
42.46	0\\
42.47	0\\
42.48	0\\
42.49	0\\
42.5	0\\
42.51	0\\
42.52	0\\
42.53	0\\
42.54	0\\
42.55	0\\
42.56	0\\
42.57	0\\
42.58	0\\
42.59	0\\
42.6	0\\
42.61	0\\
42.62	0\\
42.63	0\\
42.64	0\\
42.65	0\\
42.66	0\\
42.67	0\\
42.68	0\\
42.69	0\\
42.7	0\\
42.71	0\\
42.72	0\\
42.73	0\\
42.74	0\\
42.75	0\\
42.76	0\\
42.77	0\\
42.78	0\\
42.79	0\\
42.8	0\\
42.81	0\\
42.82	0\\
42.83	0\\
42.84	0\\
42.85	0\\
42.86	0\\
42.87	0\\
42.88	0\\
42.89	0\\
42.9	0\\
42.91	0\\
42.92	0\\
42.93	0\\
42.94	0\\
42.95	0\\
42.96	0\\
42.97	0\\
42.98	0\\
42.99	0\\
43	0\\
43.01	0\\
43.02	0\\
43.03	0\\
43.04	0\\
43.05	0\\
43.06	0\\
43.07	0\\
43.08	0\\
43.09	0\\
43.1	0\\
43.11	0\\
43.12	0\\
43.13	0\\
43.14	0\\
43.15	0\\
43.16	0\\
43.17	0\\
43.18	0\\
43.19	0\\
43.2	0\\
43.21	0\\
43.22	0\\
43.23	0\\
43.24	0\\
43.25	0\\
43.26	0\\
43.27	0\\
43.28	0\\
43.29	0\\
43.3	0\\
43.31	0\\
43.32	0\\
43.33	0\\
43.34	0\\
43.35	0\\
43.36	0\\
43.37	0\\
43.38	0\\
43.39	0\\
43.4	0\\
43.41	0\\
43.42	0\\
43.43	0\\
43.44	0\\
43.45	0\\
43.46	0\\
43.47	0\\
43.48	0\\
43.49	0\\
43.5	0\\
43.51	0\\
43.52	0\\
43.53	0\\
43.54	0\\
43.55	0\\
43.56	0\\
43.57	0\\
43.58	0\\
43.59	0\\
43.6	0\\
43.61	0\\
43.62	0\\
43.63	0\\
43.64	0\\
43.65	0\\
43.66	0\\
43.67	0\\
43.68	0\\
43.69	0\\
43.7	0\\
43.71	0\\
43.72	0\\
43.73	0\\
43.74	0\\
43.75	0\\
43.76	0\\
43.77	0\\
43.78	0\\
43.79	0\\
43.8	0\\
43.81	0\\
43.82	0\\
43.83	0\\
43.84	0\\
43.85	0\\
43.86	0\\
43.87	0\\
43.88	0\\
43.89	0\\
43.9	0\\
43.91	0\\
43.92	0\\
43.93	0\\
43.94	0\\
43.95	0\\
43.96	0\\
43.97	0\\
43.98	0\\
43.99	0\\
44	0\\
44.01	0\\
44.02	0\\
44.03	0\\
44.04	0\\
44.05	0\\
44.06	0\\
44.07	0\\
44.08	0\\
44.09	0\\
44.1	0\\
44.11	0\\
44.12	0\\
44.13	0\\
44.14	0\\
44.15	0\\
44.16	0\\
44.17	0\\
44.18	0\\
44.19	0\\
44.2	0\\
44.21	0\\
44.22	0\\
44.23	0\\
44.24	0\\
44.25	0\\
44.26	0\\
44.27	0\\
44.28	0\\
44.29	0\\
44.3	0\\
44.31	0\\
44.32	0\\
44.33	0\\
44.34	0\\
44.35	0\\
44.36	0\\
44.37	0\\
44.38	0\\
44.39	0\\
44.4	0\\
44.41	0\\
44.42	0\\
44.43	0\\
44.44	0\\
44.45	0\\
44.46	0\\
44.47	0\\
44.48	0\\
44.49	0\\
44.5	0\\
44.51	0\\
44.52	0\\
44.53	0\\
44.54	0\\
44.55	0\\
44.56	0\\
44.57	0\\
44.58	0\\
44.59	0\\
44.6	0\\
44.61	0\\
44.62	0\\
44.63	0\\
44.64	0\\
44.65	0\\
44.66	0\\
44.67	0\\
44.68	0\\
44.69	0\\
44.7	0\\
44.71	0\\
44.72	0\\
44.73	0\\
44.74	0\\
44.75	0\\
44.76	0\\
44.77	0\\
44.78	0\\
44.79	0\\
44.8	0\\
44.81	0\\
44.82	0\\
44.83	0\\
44.84	0\\
44.85	0\\
44.86	0\\
44.87	0\\
44.88	0\\
44.89	0\\
44.9	0\\
44.91	0\\
44.92	0\\
44.93	0\\
44.94	0\\
44.95	0\\
44.96	0\\
44.97	0\\
44.98	0\\
44.99	0\\
45	0\\
45.01	0\\
45.02	0\\
45.03	0\\
45.04	0\\
45.05	0\\
45.06	0\\
45.07	0\\
45.08	0\\
45.09	0\\
45.1	0\\
45.11	0\\
45.12	0\\
45.13	0\\
45.14	0\\
45.15	0\\
45.16	0\\
45.17	0\\
45.18	0\\
45.19	0\\
45.2	0\\
45.21	0\\
45.22	0\\
45.23	0\\
45.24	0\\
45.25	0\\
45.26	0\\
45.27	0\\
45.28	0\\
45.29	0\\
45.3	0\\
45.31	0\\
45.32	0\\
45.33	0\\
45.34	0\\
45.35	0\\
45.36	0\\
45.37	0\\
45.38	0\\
45.39	0\\
45.4	0\\
45.41	0\\
45.42	0\\
45.43	0\\
45.44	0\\
45.45	0\\
45.46	0\\
45.47	0\\
45.48	0\\
45.49	0\\
45.5	0\\
45.51	0\\
45.52	0\\
45.53	0\\
45.54	0\\
45.55	0\\
45.56	0\\
45.57	0\\
45.58	0\\
45.59	0\\
45.6	0\\
45.61	0\\
45.62	0\\
45.63	0\\
45.64	0\\
45.65	0\\
45.66	0\\
45.67	0\\
45.68	0\\
45.69	0\\
45.7	0\\
45.71	0\\
45.72	0\\
45.73	0\\
45.74	0\\
45.75	0\\
45.76	0\\
45.77	0\\
45.78	0\\
45.79	0\\
45.8	0\\
45.81	0\\
45.82	0\\
45.83	0\\
45.84	0\\
45.85	0\\
45.86	0\\
45.87	0\\
45.88	0\\
45.89	0\\
45.9	0\\
45.91	0\\
45.92	0\\
45.93	0\\
45.94	0\\
45.95	0\\
45.96	0\\
45.97	0\\
45.98	0\\
45.99	0\\
46	0\\
46.01	0\\
46.02	0\\
46.03	0\\
46.04	0\\
46.05	0\\
46.06	0\\
46.07	0\\
46.08	0\\
46.09	0\\
46.1	0\\
46.11	0\\
46.12	0\\
46.13	0\\
46.14	0\\
46.15	0\\
46.16	0\\
46.17	0\\
46.18	0\\
46.19	0\\
46.2	0\\
46.21	0\\
46.22	0\\
46.23	0\\
46.24	0\\
46.25	0\\
46.26	0\\
46.27	0\\
46.28	0\\
46.29	0\\
46.3	0\\
46.31	0\\
46.32	0\\
46.33	0\\
46.34	0\\
46.35	0\\
46.36	0\\
46.37	0\\
46.38	0\\
46.39	0\\
46.4	0\\
46.41	0\\
46.42	0\\
46.43	0\\
46.44	0\\
46.45	0\\
46.46	0\\
46.47	0\\
46.48	0\\
46.49	0\\
46.5	0\\
46.51	0\\
46.52	0\\
46.53	0\\
46.54	0\\
46.55	0\\
46.56	0\\
46.57	0\\
46.58	0\\
46.59	0\\
46.6	0\\
46.61	0\\
46.62	0\\
46.63	0\\
46.64	0\\
46.65	0\\
46.66	0\\
46.67	0\\
46.68	0\\
46.69	0\\
46.7	0\\
46.71	0\\
46.72	0\\
46.73	0\\
46.74	0\\
46.75	0\\
46.76	0\\
46.77	0\\
46.78	0\\
46.79	0\\
46.8	0\\
46.81	0\\
46.82	0\\
46.83	0\\
46.84	0\\
46.85	0\\
46.86	0\\
46.87	0\\
46.88	0\\
46.89	0\\
46.9	0\\
46.91	0\\
46.92	0\\
46.93	0\\
46.94	0\\
46.95	0\\
46.96	0\\
46.97	0\\
46.98	0\\
46.99	0\\
47	0\\
47.01	0\\
47.02	0\\
47.03	0\\
47.04	0\\
47.05	0\\
47.06	0\\
47.07	0\\
47.08	0\\
47.09	0\\
47.1	0\\
47.11	0\\
47.12	0\\
47.13	0\\
47.14	0\\
47.15	0\\
47.16	0\\
47.17	0\\
47.18	0\\
47.19	0\\
47.2	0\\
47.21	0\\
47.22	0\\
47.23	0\\
47.24	0\\
47.25	0\\
47.26	0\\
47.27	0\\
47.28	0\\
47.29	0\\
47.3	0\\
47.31	0\\
47.32	0\\
47.33	0\\
47.34	0\\
47.35	0\\
47.36	0\\
47.37	0\\
47.38	0\\
47.39	0\\
47.4	0\\
47.41	0\\
47.42	0\\
47.43	0\\
47.44	0\\
47.45	0\\
47.46	0\\
47.47	0\\
47.48	0\\
47.49	0\\
47.5	0\\
47.51	0\\
47.52	0\\
47.53	0\\
47.54	0\\
47.55	0\\
47.56	0\\
47.57	0\\
47.58	0\\
47.59	0\\
47.6	0\\
47.61	0\\
47.62	0\\
47.63	0\\
47.64	0\\
47.65	0\\
47.66	0\\
47.67	0\\
47.68	0\\
47.69	0\\
47.7	0\\
47.71	0\\
47.72	0\\
47.73	0\\
47.74	0\\
47.75	0\\
47.76	0\\
47.77	0\\
47.78	0\\
47.79	0\\
47.8	0\\
47.81	0\\
47.82	0\\
47.83	0\\
47.84	0\\
47.85	0\\
47.86	0\\
47.87	0\\
47.88	0\\
47.89	0\\
47.9	0\\
47.91	0\\
47.92	0\\
47.93	0\\
47.94	0\\
47.95	0\\
47.96	0\\
47.97	0\\
47.98	0\\
47.99	0\\
48	0\\
48.01	0\\
48.02	0\\
48.03	0\\
48.04	0\\
48.05	0\\
48.06	0\\
48.07	0\\
48.08	0\\
48.09	0\\
48.1	0\\
48.11	0\\
48.12	0\\
48.13	0\\
48.14	0\\
48.15	0\\
48.16	0\\
48.17	0\\
48.18	0\\
48.19	0\\
48.2	0\\
48.21	0\\
48.22	0\\
48.23	0\\
48.24	0\\
48.25	0\\
48.26	0\\
48.27	0\\
48.28	0\\
48.29	0\\
48.3	0\\
48.31	0\\
48.32	0\\
48.33	0\\
48.34	0\\
48.35	0\\
48.36	0\\
48.37	0\\
48.38	0\\
48.39	0\\
48.4	0\\
48.41	0\\
48.42	0\\
48.43	0\\
48.44	0\\
48.45	0\\
48.46	0\\
48.47	0\\
48.48	0\\
48.49	0\\
48.5	0\\
48.51	0\\
48.52	0\\
48.53	0\\
48.54	0\\
48.55	0\\
48.56	0\\
48.57	0\\
48.58	0\\
48.59	0\\
48.6	0\\
48.61	0\\
48.62	0\\
48.63	0\\
48.64	0\\
48.65	0\\
48.66	0\\
48.67	0\\
48.68	0\\
48.69	0\\
48.7	0\\
48.71	0\\
48.72	0\\
48.73	0\\
48.74	0\\
48.75	0\\
48.76	0\\
48.77	0\\
48.78	0\\
48.79	0\\
48.8	0\\
48.81	0\\
48.82	0\\
48.83	0\\
48.84	0\\
48.85	0\\
48.86	0\\
48.87	0\\
48.88	0\\
48.89	0\\
48.9	0\\
48.91	0\\
48.92	0\\
48.93	0\\
48.94	0\\
48.95	0\\
48.96	0\\
48.97	0\\
48.98	0\\
48.99	0\\
49	0\\
49.01	0\\
49.02	0\\
49.03	0\\
49.04	0\\
49.05	0\\
49.06	0\\
49.07	0\\
49.08	0\\
49.09	0\\
49.1	0\\
49.11	0\\
49.12	0\\
49.13	0\\
49.14	0\\
49.15	0\\
49.16	0\\
49.17	0\\
49.18	0\\
49.19	0\\
49.2	0\\
49.21	0\\
49.22	0\\
49.23	0\\
49.24	0\\
49.25	0\\
49.26	0\\
49.27	0\\
49.28	0\\
49.29	0\\
49.3	0\\
49.31	0\\
49.32	0\\
49.33	0\\
49.34	0\\
49.35	0\\
49.36	0\\
49.37	0\\
49.38	0\\
49.39	0\\
49.4	0\\
49.41	0\\
49.42	0\\
49.43	0\\
49.44	0\\
49.45	0\\
49.46	0\\
49.47	0\\
49.48	0\\
49.49	0\\
49.5	0\\
49.51	0\\
49.52	0\\
49.53	0\\
49.54	0\\
49.55	0\\
49.56	0\\
49.57	0\\
49.58	0\\
49.59	0\\
49.6	0\\
49.61	0\\
49.62	0\\
49.63	0\\
49.64	0\\
49.65	0\\
49.66	0\\
49.67	0\\
49.68	0\\
49.69	0\\
49.7	0\\
49.71	0\\
49.72	0\\
49.73	0\\
49.74	0\\
49.75	0\\
49.76	0\\
49.77	0\\
49.78	0\\
49.79	0\\
49.8	0\\
49.81	0\\
49.82	0\\
49.83	0\\
49.84	0\\
49.85	0\\
49.86	0\\
49.87	0\\
49.88	0\\
49.89	0\\
49.9	0\\
49.91	0\\
49.92	0\\
49.93	0\\
49.94	0\\
49.95	0\\
49.96	0\\
49.97	0\\
49.98	0\\
49.99	0\\
50	0\\
50.01	0\\
50.02	0\\
50.03	0\\
50.04	0\\
50.05	0\\
50.06	0\\
50.07	0\\
50.08	0\\
50.09	0\\
50.1	0\\
50.11	0\\
50.12	0\\
50.13	0\\
50.14	0\\
50.15	0\\
50.16	0\\
50.17	0\\
50.18	0\\
50.19	0\\
50.2	0\\
50.21	0\\
50.22	0\\
50.23	0\\
50.24	0\\
50.25	0\\
50.26	0\\
50.27	0\\
50.28	0\\
50.29	0\\
50.3	0\\
50.31	0\\
50.32	0\\
50.33	0\\
50.34	0\\
50.35	0\\
50.36	0\\
50.37	0\\
50.38	0\\
50.39	0\\
50.4	0\\
50.41	0\\
50.42	0\\
50.43	0\\
50.44	0\\
50.45	0\\
50.46	0\\
50.47	0\\
50.48	0\\
50.49	0\\
50.5	0\\
50.51	0\\
50.52	0\\
50.53	0\\
50.54	0\\
50.55	0\\
50.56	0\\
50.57	0\\
50.58	0\\
50.59	0\\
50.6	0\\
50.61	0\\
50.62	0\\
50.63	0\\
50.64	0\\
50.65	0\\
50.66	0\\
50.67	0\\
50.68	0\\
50.69	0\\
50.7	0\\
50.71	0\\
50.72	0\\
50.73	0\\
50.74	0\\
50.75	0\\
50.76	0\\
50.77	0\\
50.78	0\\
50.79	0\\
50.8	0\\
50.81	0\\
50.82	0\\
50.83	0\\
50.84	0\\
50.85	0\\
50.86	0\\
50.87	0\\
50.88	0\\
50.89	0\\
50.9	0\\
50.91	0\\
50.92	0\\
50.93	0\\
50.94	0\\
50.95	0\\
50.96	0\\
50.97	0\\
50.98	0\\
50.99	0\\
51	0\\
51.01	0\\
51.02	0\\
51.03	0\\
51.04	0\\
51.05	0\\
51.06	0\\
51.07	0\\
51.08	0\\
51.09	0\\
51.1	0\\
51.11	0\\
51.12	0\\
51.13	0\\
51.14	0\\
51.15	0\\
51.16	0\\
51.17	0\\
51.18	0\\
51.19	0\\
51.2	0\\
51.21	0\\
51.22	0\\
51.23	0\\
51.24	0\\
51.25	0\\
51.26	0\\
51.27	0\\
51.28	0\\
51.29	0\\
51.3	0\\
51.31	0\\
51.32	0\\
51.33	0\\
51.34	0\\
51.35	0\\
51.36	0\\
51.37	0\\
51.38	0\\
51.39	0\\
51.4	0\\
51.41	0\\
51.42	0\\
51.43	0\\
51.44	0\\
51.45	0\\
51.46	0\\
51.47	0\\
51.48	0\\
51.49	0\\
51.5	0\\
51.51	0\\
51.52	0\\
51.53	0\\
51.54	0\\
51.55	0\\
51.56	0\\
51.57	0\\
51.58	0\\
51.59	0\\
51.6	0\\
51.61	0\\
51.62	0\\
51.63	0\\
51.64	0\\
51.65	0\\
51.66	0\\
51.67	0\\
51.68	0\\
51.69	0\\
51.7	0\\
51.71	0\\
51.72	0\\
51.73	0\\
51.74	0\\
51.75	0\\
51.76	0\\
51.77	0\\
51.78	0\\
51.79	0\\
51.8	0\\
51.81	0\\
51.82	0\\
51.83	0\\
51.84	0\\
51.85	0\\
51.86	0\\
51.87	0\\
51.88	0\\
51.89	0\\
51.9	0\\
51.91	0\\
51.92	0\\
51.93	0\\
51.94	0\\
51.95	0\\
51.96	0\\
51.97	0\\
51.98	0\\
51.99	0\\
52	0\\
52.01	0\\
52.02	0\\
52.03	0\\
52.04	0\\
52.05	0\\
52.06	0\\
52.07	0\\
52.08	0\\
52.09	0\\
52.1	0\\
52.11	0\\
52.12	0\\
52.13	0\\
52.14	0\\
52.15	0\\
52.16	0\\
52.17	0\\
52.18	0\\
52.19	0\\
52.2	0\\
52.21	0\\
52.22	0\\
52.23	0\\
52.24	0\\
52.25	0\\
52.26	0\\
52.27	0\\
52.28	0\\
52.29	0\\
52.3	0\\
52.31	0\\
52.32	0\\
52.33	0\\
52.34	0\\
52.35	0\\
52.36	0\\
52.37	0\\
52.38	0\\
52.39	0\\
52.4	0\\
52.41	0\\
52.42	0\\
52.43	0\\
52.44	0\\
52.45	0\\
52.46	0\\
52.47	0\\
52.48	0\\
52.49	0\\
52.5	0\\
52.51	0\\
52.52	0\\
52.53	0\\
52.54	0\\
52.55	0\\
52.56	0\\
52.57	0\\
52.58	0\\
52.59	0\\
52.6	0\\
52.61	0\\
52.62	0\\
52.63	0\\
52.64	0\\
52.65	0\\
52.66	0\\
52.67	0\\
52.68	0\\
52.69	0\\
52.7	0\\
52.71	0\\
52.72	0\\
52.73	0\\
52.74	0\\
52.75	0\\
52.76	0\\
52.77	0\\
52.78	0\\
52.79	0\\
52.8	0\\
52.81	0\\
52.82	0\\
52.83	0\\
52.84	0\\
52.85	0\\
52.86	0\\
52.87	0\\
52.88	0\\
52.89	0\\
52.9	0\\
52.91	0\\
52.92	0\\
52.93	0\\
52.94	0\\
52.95	0\\
52.96	0\\
52.97	0\\
52.98	0\\
52.99	0\\
53	0\\
53.01	0\\
53.02	0\\
53.03	0\\
53.04	0\\
53.05	0\\
53.06	0\\
53.07	0\\
53.08	0\\
53.09	0\\
53.1	0\\
53.11	0\\
53.12	0\\
53.13	0\\
53.14	0\\
53.15	0\\
53.16	0\\
53.17	0\\
53.18	0\\
53.19	0\\
53.2	0\\
53.21	0\\
53.22	0\\
53.23	0\\
53.24	0\\
53.25	0\\
53.26	0\\
53.27	0\\
53.28	0\\
53.29	0\\
53.3	0\\
53.31	0\\
53.32	0\\
53.33	0\\
53.34	0\\
53.35	0\\
53.36	0\\
53.37	0\\
53.38	0\\
53.39	0\\
53.4	0\\
53.41	0\\
53.42	0\\
53.43	0\\
53.44	0\\
53.45	0\\
53.46	0\\
53.47	0\\
53.48	0\\
53.49	0\\
53.5	0\\
53.51	0\\
53.52	0\\
53.53	0\\
53.54	0\\
53.55	0\\
53.56	0\\
53.57	0\\
53.58	0\\
53.59	0\\
53.6	0\\
53.61	0\\
53.62	0\\
53.63	0\\
53.64	0\\
53.65	0\\
53.66	0\\
53.67	0\\
53.68	0\\
53.69	0\\
53.7	0\\
53.71	0\\
53.72	0\\
53.73	0\\
53.74	0\\
53.75	0\\
53.76	0\\
53.77	0\\
53.78	0\\
53.79	0\\
53.8	0\\
53.81	0\\
53.82	0\\
53.83	0\\
53.84	0\\
53.85	0\\
53.86	0\\
53.87	0\\
53.88	0\\
53.89	0\\
53.9	0\\
53.91	0\\
53.92	0\\
53.93	0\\
53.94	0\\
53.95	0\\
53.96	0\\
53.97	0\\
53.98	0\\
53.99	0\\
54	0\\
54.01	0\\
54.02	0\\
54.03	0\\
54.04	0\\
54.05	0\\
54.06	0\\
54.07	0\\
54.08	0\\
54.09	0\\
54.1	0\\
54.11	0\\
54.12	0\\
54.13	0\\
54.14	0\\
54.15	0\\
54.16	0\\
54.17	0\\
54.18	0\\
54.19	0\\
54.2	0\\
54.21	0\\
54.22	0\\
54.23	0\\
54.24	0\\
54.25	0\\
54.26	0\\
54.27	0\\
54.28	0\\
54.29	0\\
54.3	0\\
54.31	0\\
54.32	0\\
54.33	0\\
54.34	0\\
54.35	0\\
54.36	0\\
54.37	0\\
54.38	0\\
54.39	0\\
54.4	0\\
54.41	0\\
54.42	0\\
54.43	0\\
54.44	0\\
54.45	0\\
54.46	0\\
54.47	0\\
54.48	0\\
54.49	0\\
54.5	0\\
54.51	0\\
54.52	0\\
54.53	0\\
54.54	0\\
54.55	0\\
54.56	0\\
54.57	0\\
54.58	0\\
54.59	0\\
54.6	0\\
54.61	0\\
54.62	0\\
54.63	0\\
54.64	0\\
54.65	0\\
54.66	0\\
54.67	0\\
54.68	0\\
54.69	0\\
54.7	0\\
54.71	0\\
54.72	0\\
54.73	0\\
54.74	0\\
54.75	0\\
54.76	0\\
54.77	0\\
54.78	0\\
54.79	0\\
54.8	0\\
54.81	0\\
54.82	0\\
54.83	0\\
54.84	0\\
54.85	0\\
54.86	0\\
54.87	0\\
54.88	0\\
54.89	0\\
54.9	0\\
54.91	0\\
54.92	0\\
54.93	0\\
54.94	0\\
54.95	0\\
54.96	0\\
54.97	0\\
54.98	0\\
54.99	0\\
55	0\\
55.01	0\\
55.02	0\\
55.03	0\\
55.04	0\\
55.05	0\\
55.06	0\\
55.07	0\\
55.08	0\\
55.09	0\\
55.1	0\\
55.11	0\\
55.12	0\\
55.13	0\\
55.14	0\\
55.15	0\\
55.16	0\\
55.17	0\\
55.18	0\\
55.19	0\\
55.2	0\\
55.21	0\\
55.22	0\\
55.23	0\\
55.24	0\\
55.25	0\\
55.26	0\\
55.27	0\\
55.28	0\\
55.29	0\\
55.3	0\\
55.31	0\\
55.32	0\\
55.33	0\\
55.34	0\\
55.35	0\\
55.36	0\\
55.37	0\\
55.38	0\\
55.39	0\\
55.4	0\\
55.41	0\\
55.42	0\\
55.43	0\\
55.44	0\\
55.45	0\\
55.46	0\\
55.47	0\\
55.48	0\\
55.49	0\\
55.5	0\\
55.51	0\\
55.52	0\\
55.53	0\\
55.54	0\\
55.55	0\\
55.56	0\\
55.57	0\\
55.58	0\\
55.59	0\\
55.6	0\\
55.61	0\\
55.62	0\\
55.63	0\\
55.64	0\\
55.65	0\\
55.66	0\\
55.67	0\\
55.68	0\\
55.69	0\\
55.7	0\\
55.71	0\\
55.72	0\\
55.73	0\\
55.74	0\\
55.75	0\\
55.76	0\\
55.77	0\\
55.78	0\\
55.79	0\\
55.8	0\\
55.81	0\\
55.82	0\\
55.83	0\\
55.84	0\\
55.85	0\\
55.86	0\\
55.87	0\\
55.88	0\\
55.89	0\\
55.9	0\\
55.91	0\\
55.92	0\\
55.93	0\\
55.94	0\\
55.95	0\\
55.96	0\\
55.97	0\\
55.98	0\\
55.99	0\\
56	0\\
56.01	0\\
56.02	0\\
56.03	0\\
56.04	0\\
56.05	0\\
56.06	0\\
56.07	0\\
56.08	0\\
56.09	0\\
56.1	0\\
56.11	0\\
56.12	0\\
56.13	0\\
56.14	0\\
56.15	0\\
56.16	0\\
56.17	0\\
56.18	0\\
56.19	0\\
56.2	0\\
56.21	0\\
56.22	0\\
56.23	0\\
56.24	0\\
56.25	0\\
56.26	0\\
56.27	0\\
56.28	0\\
56.29	0\\
56.3	0\\
56.31	0\\
56.32	0\\
56.33	0\\
56.34	0\\
56.35	0\\
56.36	0\\
56.37	0\\
56.38	0\\
56.39	0\\
56.4	0\\
56.41	0\\
56.42	0\\
56.43	0\\
56.44	0\\
56.45	0\\
56.46	0\\
56.47	0\\
56.48	0\\
56.49	0\\
56.5	0\\
56.51	0\\
56.52	0\\
56.53	0\\
56.54	0\\
56.55	0\\
56.56	0\\
56.57	0\\
56.58	0\\
56.59	0\\
56.6	0\\
56.61	0\\
56.62	0\\
56.63	0\\
56.64	0\\
56.65	0\\
56.66	0\\
56.67	0\\
56.68	0\\
56.69	0\\
56.7	0\\
56.71	0\\
56.72	0\\
56.73	0\\
56.74	0\\
56.75	0\\
56.76	0\\
56.77	0\\
56.78	0\\
56.79	0\\
56.8	0\\
56.81	0\\
56.82	0\\
56.83	0\\
56.84	0\\
56.85	0\\
56.86	0\\
56.87	0\\
56.88	0\\
56.89	0\\
56.9	0\\
56.91	0\\
56.92	0\\
56.93	0\\
56.94	0\\
56.95	0\\
56.96	0\\
56.97	0\\
56.98	0\\
56.99	0\\
57	0\\
57.01	0\\
57.02	0\\
57.03	0\\
57.04	0\\
57.05	0\\
57.06	0\\
57.07	0\\
57.08	0\\
57.09	0\\
57.1	0\\
57.11	0\\
57.12	0\\
57.13	0\\
57.14	0\\
57.15	0\\
57.16	0\\
57.17	0\\
57.18	0\\
57.19	0\\
57.2	0\\
57.21	0\\
57.22	0\\
57.23	0\\
57.24	0\\
57.25	0\\
57.26	0\\
57.27	0\\
57.28	0\\
57.29	0\\
57.3	0\\
57.31	0\\
57.32	0\\
57.33	0\\
57.34	0\\
57.35	0\\
57.36	0\\
57.37	0\\
57.38	0\\
57.39	0\\
57.4	0\\
57.41	0\\
57.42	0\\
57.43	0\\
57.44	0\\
57.45	0\\
57.46	0\\
57.47	0\\
57.48	0\\
57.49	0\\
57.5	0\\
57.51	0\\
57.52	0\\
57.53	0\\
57.54	0\\
57.55	0\\
57.56	0\\
57.57	0\\
57.58	0\\
57.59	0\\
57.6	0\\
57.61	0\\
57.62	0\\
57.63	0\\
57.64	0\\
57.65	0\\
57.66	0\\
57.67	0\\
57.68	0\\
57.69	0\\
57.7	0\\
57.71	0\\
57.72	0\\
57.73	0\\
57.74	0\\
57.75	0\\
57.76	0\\
57.77	0\\
57.78	0\\
57.79	0\\
57.8	0\\
57.81	0\\
57.82	0\\
57.83	0\\
57.84	0\\
57.85	0\\
57.86	0\\
57.87	0\\
57.88	0\\
57.89	0\\
57.9	0\\
57.91	0\\
57.92	0\\
57.93	0\\
57.94	0\\
57.95	0\\
57.96	0\\
57.97	0\\
57.98	0\\
57.99	0\\
58	0\\
58.01	0\\
58.02	0\\
58.03	0\\
58.04	0\\
58.05	0\\
58.06	0\\
58.07	0\\
58.08	0\\
58.09	0\\
58.1	0\\
58.11	0\\
58.12	0\\
58.13	0\\
58.14	0\\
58.15	0\\
58.16	0\\
58.17	0\\
58.18	0\\
58.19	0\\
58.2	0\\
58.21	0\\
58.22	0\\
58.23	0\\
58.24	0\\
58.25	0\\
58.26	0\\
58.27	0\\
58.28	0\\
58.29	0\\
58.3	0\\
58.31	0\\
58.32	0\\
58.33	0\\
58.34	0\\
58.35	0\\
58.36	0\\
58.37	0\\
58.38	0\\
58.39	0\\
58.4	0\\
58.41	0\\
58.42	0\\
58.43	0\\
58.44	0\\
58.45	0\\
58.46	0\\
58.47	0\\
58.48	0\\
58.49	0\\
58.5	0\\
58.51	0\\
58.52	0\\
58.53	0\\
58.54	0\\
58.55	0\\
58.56	0\\
58.57	0\\
58.58	0\\
58.59	0\\
58.6	0\\
58.61	0\\
58.62	0\\
58.63	0\\
58.64	0\\
58.65	0\\
58.66	0\\
58.67	0\\
58.68	0\\
58.69	0\\
58.7	0\\
58.71	0\\
58.72	0\\
58.73	0\\
58.74	0\\
58.75	0\\
58.76	0\\
58.77	0\\
58.78	0\\
58.79	0\\
58.8	0\\
58.81	0\\
58.82	0\\
58.83	0\\
58.84	0\\
58.85	0\\
58.86	0\\
58.87	0\\
58.88	0\\
58.89	0\\
58.9	0\\
58.91	0\\
58.92	0\\
58.93	0\\
58.94	0\\
58.95	0\\
58.96	0\\
58.97	0\\
58.98	0\\
58.99	0\\
59	0\\
59.01	0\\
59.02	0\\
59.03	0\\
59.04	0\\
59.05	0\\
59.06	0\\
59.07	0\\
59.08	0\\
59.09	0\\
59.1	0\\
59.11	0\\
59.12	0\\
59.13	0\\
59.14	0\\
59.15	0\\
59.16	0\\
59.17	0\\
59.18	0\\
59.19	0\\
59.2	0\\
59.21	0\\
59.22	0\\
59.23	0\\
59.24	0\\
59.25	0\\
59.26	0\\
59.27	0\\
59.28	0\\
59.29	0\\
59.3	0\\
59.31	0\\
59.32	0\\
59.33	0\\
59.34	0\\
59.35	0\\
59.36	0\\
59.37	0\\
59.38	0\\
59.39	0\\
59.4	0\\
59.41	0\\
59.42	0\\
59.43	0\\
59.44	0\\
59.45	0\\
59.46	0\\
59.47	0\\
59.48	0\\
59.49	0\\
59.5	0\\
59.51	0\\
59.52	0\\
59.53	0\\
59.54	0\\
59.55	0\\
59.56	0\\
59.57	0\\
59.58	0\\
59.59	0\\
59.6	0\\
59.61	0\\
59.62	0\\
59.63	0\\
59.64	0\\
59.65	0\\
59.66	0\\
59.67	0\\
59.68	0\\
59.69	0\\
59.7	0\\
59.71	0\\
59.72	0\\
59.73	0\\
59.74	0\\
59.75	0\\
59.76	0\\
59.77	0\\
59.78	0\\
59.79	0\\
59.8	0\\
59.81	0\\
59.82	0\\
59.83	0\\
59.84	0\\
59.85	0\\
59.86	0\\
59.87	0\\
59.88	0\\
59.89	0\\
59.9	0\\
59.91	0\\
59.92	0\\
59.93	0\\
59.94	0\\
59.95	0\\
59.96	0\\
59.97	0\\
59.98	0\\
59.99	0\\
60	0\\
60.01	0\\
60.02	0\\
60.03	0\\
60.04	0\\
60.05	0\\
60.06	0\\
60.07	0\\
60.08	0\\
60.09	0\\
60.1	0\\
60.11	0\\
60.12	0\\
60.13	0\\
60.14	0\\
60.15	0\\
60.16	0\\
60.17	0\\
60.18	0\\
60.19	0\\
60.2	0\\
60.21	0\\
60.22	0\\
60.23	0\\
60.24	0\\
60.25	0\\
60.26	0\\
60.27	0\\
60.28	0\\
60.29	0\\
60.3	0\\
60.31	0\\
60.32	0\\
60.33	0\\
60.34	0\\
60.35	0\\
60.36	0\\
60.37	0\\
60.38	0\\
60.39	0\\
60.4	0\\
60.41	0\\
60.42	0\\
60.43	0\\
60.44	0\\
60.45	0\\
60.46	0\\
60.47	0\\
60.48	0\\
60.49	0\\
60.5	0\\
60.51	0\\
60.52	0\\
60.53	0\\
60.54	0\\
60.55	0\\
60.56	0\\
60.57	0\\
60.58	0\\
60.59	0\\
60.6	0\\
60.61	0\\
60.62	0\\
60.63	0\\
60.64	0\\
60.65	0\\
60.66	0\\
60.67	0\\
60.68	0\\
60.69	0\\
60.7	0\\
60.71	0\\
60.72	0\\
60.73	0\\
60.74	0\\
60.75	0\\
60.76	0\\
60.77	0\\
60.78	0\\
60.79	0\\
60.8	0\\
60.81	0\\
60.82	0\\
60.83	0\\
60.84	0\\
60.85	0\\
60.86	0\\
60.87	0\\
60.88	0\\
60.89	0\\
60.9	0\\
60.91	0\\
60.92	0\\
60.93	0\\
60.94	0\\
60.95	0\\
60.96	0\\
60.97	0\\
60.98	0\\
60.99	0\\
61	0\\
61.01	0\\
61.02	0\\
61.03	0\\
61.04	0\\
61.05	0\\
61.06	0\\
61.07	0\\
61.08	0\\
61.09	0\\
61.1	0\\
61.11	0\\
61.12	0\\
61.13	0\\
61.14	0\\
61.15	0\\
61.16	0\\
61.17	0\\
61.18	0\\
61.19	0\\
61.2	0\\
61.21	0\\
61.22	0\\
61.23	0\\
61.24	0\\
61.25	0\\
61.26	0\\
61.27	0\\
61.28	0\\
61.29	0\\
61.3	0\\
61.31	0\\
61.32	0\\
61.33	0\\
61.34	0\\
61.35	0\\
61.36	0\\
61.37	0\\
61.38	0\\
61.39	0\\
61.4	0\\
61.41	0\\
61.42	0\\
61.43	0\\
61.44	0\\
61.45	0\\
61.46	0\\
61.47	0\\
61.48	0\\
61.49	0\\
61.5	0\\
61.51	0\\
61.52	0\\
61.53	0\\
61.54	0\\
61.55	0\\
61.56	0\\
61.57	0\\
61.58	0\\
61.59	0\\
61.6	0\\
61.61	0\\
61.62	0\\
61.63	0\\
61.64	0\\
61.65	0\\
61.66	0\\
61.67	0\\
61.68	0\\
61.69	0\\
61.7	0\\
61.71	0\\
61.72	0\\
61.73	0\\
61.74	0\\
61.75	0\\
61.76	0\\
61.77	0\\
61.78	0\\
61.79	0\\
61.8	0\\
61.81	0\\
61.82	0\\
61.83	0\\
61.84	0\\
61.85	0\\
61.86	0\\
61.87	0\\
61.88	0\\
61.89	0\\
61.9	0\\
61.91	0\\
61.92	0\\
61.93	0\\
61.94	0\\
61.95	0\\
61.96	0\\
61.97	0\\
61.98	0\\
61.99	0\\
62	0\\
62.01	0\\
62.02	0\\
62.03	0\\
62.04	0\\
62.05	0\\
62.06	0\\
62.07	0\\
62.08	0\\
62.09	0\\
62.1	0\\
62.11	0\\
62.12	0\\
62.13	0\\
62.14	0\\
62.15	0\\
62.16	0\\
62.17	0\\
62.18	0\\
62.19	0\\
62.2	0\\
62.21	0\\
62.22	0\\
62.23	0\\
62.24	0\\
62.25	0\\
62.26	0\\
62.27	0\\
62.28	0\\
62.29	0\\
62.3	0\\
62.31	0\\
62.32	0\\
62.33	0\\
62.34	0\\
62.35	0\\
62.36	0\\
62.37	0\\
62.38	0\\
62.39	0\\
62.4	0\\
62.41	0\\
62.42	0\\
62.43	0\\
62.44	0\\
62.45	0\\
62.46	0\\
62.47	0\\
62.48	0\\
62.49	0\\
62.5	0\\
62.51	0\\
62.52	0\\
62.53	0\\
62.54	0\\
62.55	0\\
62.56	0\\
62.57	0\\
62.58	0\\
62.59	0\\
62.6	0\\
62.61	0\\
62.62	0\\
62.63	0\\
62.64	0\\
62.65	0\\
62.66	0\\
62.67	0\\
62.68	0\\
62.69	0\\
62.7	0\\
62.71	0\\
62.72	0\\
62.73	0\\
62.74	0\\
62.75	0\\
62.76	0\\
62.77	0\\
62.78	0\\
62.79	0\\
62.8	0\\
62.81	0\\
62.82	0\\
62.83	0\\
62.84	0\\
62.85	0\\
62.86	0\\
62.87	0\\
62.88	0\\
62.89	0\\
62.9	0\\
62.91	0\\
62.92	0\\
62.93	0\\
62.94	0\\
62.95	0\\
62.96	0\\
62.97	0\\
62.98	0\\
62.99	0\\
63	0\\
63.01	0\\
63.02	0\\
63.03	0\\
63.04	0\\
63.05	0\\
63.06	0\\
63.07	0\\
63.08	0\\
63.09	0\\
63.1	0\\
63.11	0\\
63.12	0\\
63.13	0\\
63.14	0\\
63.15	0\\
63.16	0\\
63.17	0\\
63.18	0\\
63.19	0\\
63.2	0\\
63.21	0\\
63.22	0\\
63.23	0\\
63.24	0\\
63.25	0\\
63.26	0\\
63.27	0\\
63.28	0\\
63.29	0\\
63.3	0\\
63.31	0\\
63.32	0\\
63.33	0\\
63.34	0\\
63.35	0\\
63.36	0\\
63.37	0\\
63.38	0\\
63.39	0\\
63.4	0\\
63.41	0\\
63.42	0\\
63.43	0\\
63.44	0\\
63.45	0\\
63.46	0\\
63.47	0\\
63.48	0\\
63.49	0\\
63.5	0\\
63.51	0\\
63.52	0\\
63.53	0\\
63.54	0\\
63.55	0\\
63.56	0\\
63.57	0\\
63.58	0\\
63.59	0\\
63.6	0\\
63.61	0\\
63.62	0\\
63.63	0\\
63.64	0\\
63.65	0\\
63.66	0\\
63.67	0\\
63.68	0\\
63.69	0\\
63.7	0\\
63.71	0\\
63.72	0\\
63.73	0\\
63.74	0\\
63.75	0\\
63.76	0\\
63.77	0\\
63.78	0\\
63.79	0\\
63.8	0\\
63.81	0\\
63.82	0\\
63.83	0\\
63.84	0\\
63.85	0\\
63.86	0\\
63.87	0\\
63.88	0\\
63.89	0\\
63.9	0\\
63.91	0\\
63.92	0\\
63.93	0\\
63.94	0\\
63.95	0\\
63.96	0\\
63.97	0\\
63.98	0\\
63.99	0\\
64	0\\
64.01	0\\
64.02	0\\
64.03	0\\
64.04	0\\
64.05	0\\
64.06	0\\
64.07	0\\
64.08	0\\
64.09	0\\
64.1	0\\
64.11	0\\
64.12	0\\
64.13	0\\
64.14	0\\
64.15	0\\
64.16	0\\
64.17	0\\
64.18	0\\
64.19	0\\
64.2	0\\
64.21	0\\
64.22	0\\
64.23	0\\
64.24	0\\
64.25	0\\
64.26	0\\
64.27	0\\
64.28	0\\
64.29	0\\
64.3	0\\
64.31	0\\
64.32	0\\
64.33	0\\
64.34	0\\
64.35	0\\
64.36	0\\
64.37	0\\
64.38	0\\
64.39	0\\
64.4	0\\
64.41	0\\
64.42	0\\
64.43	0\\
64.44	0\\
64.45	0\\
64.46	0\\
64.47	0\\
64.48	0\\
64.49	0\\
64.5	0\\
64.51	0\\
64.52	0\\
64.53	0\\
64.54	0\\
64.55	0\\
64.56	0\\
64.57	0\\
64.58	0\\
64.59	0\\
64.6	0\\
64.61	0\\
64.62	0\\
64.63	0\\
64.64	0\\
64.65	0\\
64.66	0\\
64.67	0\\
64.68	0\\
64.69	0\\
64.7	0\\
64.71	0\\
64.72	0\\
64.73	0\\
64.74	0\\
64.75	0\\
64.76	0\\
64.77	0\\
64.78	0\\
64.79	0\\
64.8	0\\
64.81	0\\
64.82	0\\
64.83	0\\
64.84	0\\
64.85	0\\
64.86	0\\
64.87	0\\
64.88	0\\
64.89	0\\
64.9	0\\
64.91	0\\
64.92	0\\
64.93	0\\
64.94	0\\
64.95	0\\
64.96	0\\
64.97	0\\
64.98	0\\
64.99	0\\
65	0\\
65.01	0\\
65.02	0\\
65.03	0\\
65.04	0\\
65.05	0\\
65.06	0\\
65.07	0\\
65.08	0\\
65.09	0\\
65.1	0\\
65.11	0\\
65.12	0\\
65.13	0\\
65.14	0\\
65.15	0\\
65.16	0\\
65.17	0\\
65.18	0\\
65.19	0\\
65.2	0\\
65.21	0\\
65.22	0\\
65.23	0\\
65.24	0\\
65.25	0\\
65.26	0\\
65.27	0\\
65.28	0\\
65.29	0\\
65.3	0\\
65.31	0\\
65.32	0\\
65.33	0\\
65.34	0\\
65.35	0\\
65.36	0\\
65.37	0\\
65.38	0\\
65.39	0\\
65.4	0\\
65.41	0\\
65.42	0\\
65.43	0\\
65.44	0\\
65.45	0\\
65.46	0\\
65.47	0\\
65.48	0\\
65.49	0\\
65.5	0\\
65.51	0\\
65.52	0\\
65.53	0\\
65.54	0\\
65.55	0\\
65.56	0\\
65.57	0\\
65.58	0\\
65.59	0\\
65.6	0\\
65.61	0\\
65.62	0\\
65.63	0\\
65.64	0\\
65.65	0\\
65.66	0\\
65.67	0\\
65.68	0\\
65.69	0\\
65.7	0\\
65.71	0\\
65.72	0\\
65.73	0\\
65.74	0\\
65.75	0\\
65.76	0\\
65.77	0\\
65.78	0\\
65.79	0\\
65.8	0\\
65.81	0\\
65.82	0\\
65.83	0\\
65.84	0\\
65.85	0\\
65.86	0\\
65.87	0\\
65.88	0\\
65.89	0\\
65.9	0\\
65.91	0\\
65.92	0\\
65.93	0\\
65.94	0\\
65.95	0\\
65.96	0\\
65.97	0\\
65.98	0\\
65.99	0\\
66	0\\
66.01	0\\
66.02	0\\
66.03	0\\
66.04	0\\
66.05	0\\
66.06	0\\
66.07	0\\
66.08	0\\
66.09	0\\
66.1	0\\
66.11	0\\
66.12	0\\
66.13	0\\
66.14	0\\
66.15	0\\
66.16	0\\
66.17	0\\
66.18	0\\
66.19	0\\
66.2	0\\
66.21	0\\
66.22	0\\
66.23	0\\
66.24	0\\
66.25	0\\
66.26	0\\
66.27	0\\
66.28	0\\
66.29	0\\
66.3	0\\
66.31	0\\
66.32	0\\
66.33	0\\
66.34	0\\
66.35	0\\
66.36	0\\
66.37	0\\
66.38	0\\
66.39	0\\
66.4	0\\
66.41	0\\
66.42	0\\
66.43	0\\
66.44	0\\
66.45	0\\
66.46	0\\
66.47	0\\
66.48	0\\
66.49	0\\
66.5	0\\
66.51	0\\
66.52	0\\
66.53	0\\
66.54	0\\
66.55	0\\
66.56	0\\
66.57	0\\
66.58	0\\
66.59	0\\
66.6	0\\
66.61	0\\
66.62	0\\
66.63	0\\
66.64	0\\
66.65	0\\
66.66	0\\
66.67	0\\
66.68	0\\
66.69	0\\
66.7	0\\
66.71	0\\
66.72	0\\
66.73	0\\
66.74	0\\
66.75	0\\
66.76	0\\
66.77	0\\
66.78	0\\
66.79	0\\
66.8	0\\
66.81	0\\
66.82	0\\
66.83	0\\
66.84	0\\
66.85	0\\
66.86	0\\
66.87	0\\
66.88	0\\
66.89	0\\
66.9	0\\
66.91	0\\
66.92	0\\
66.93	0\\
66.94	0\\
66.95	0\\
66.96	0\\
66.97	0\\
66.98	0\\
66.99	0\\
67	0\\
67.01	0\\
67.02	0\\
67.03	0\\
67.04	0\\
67.05	0\\
67.06	0\\
67.07	0\\
67.08	0\\
67.09	0\\
67.1	0\\
67.11	0\\
67.12	0\\
67.13	0\\
67.14	0\\
67.15	0\\
67.16	0\\
67.17	0\\
67.18	0\\
67.19	0\\
67.2	0\\
67.21	0\\
67.22	0\\
67.23	0\\
67.24	0\\
67.25	0\\
67.26	0\\
67.27	0\\
67.28	0\\
67.29	0\\
67.3	0\\
67.31	0\\
67.32	0\\
67.33	0\\
67.34	0\\
67.35	0\\
67.36	0\\
67.37	0\\
67.38	0\\
67.39	0\\
67.4	0\\
67.41	0\\
67.42	0\\
67.43	0\\
67.44	0\\
67.45	0\\
67.46	0\\
67.47	0\\
67.48	0\\
67.49	0\\
67.5	0\\
67.51	0\\
67.52	0\\
67.53	0\\
67.54	0\\
67.55	0\\
67.56	0\\
67.57	0\\
67.58	0\\
67.59	0\\
67.6	0\\
67.61	0\\
67.62	0\\
67.63	0\\
67.64	0\\
67.65	0\\
67.66	0\\
67.67	0\\
67.68	0\\
67.69	0\\
67.7	0\\
67.71	0\\
67.72	0\\
67.73	0\\
67.74	0\\
67.75	0\\
67.76	0\\
67.77	0\\
67.78	0\\
67.79	0\\
67.8	0\\
67.81	0\\
67.82	0\\
67.83	0\\
67.84	0\\
67.85	0\\
67.86	0\\
67.87	0\\
67.88	0\\
67.89	0\\
67.9	0\\
67.91	0\\
67.92	0\\
67.93	0\\
67.94	0\\
67.95	0\\
67.96	0\\
67.97	0\\
67.98	0\\
67.99	0\\
68	0\\
68.01	0\\
68.02	0\\
68.03	0\\
68.04	0\\
68.05	0\\
68.06	0\\
68.07	0\\
68.08	0\\
68.09	0\\
68.1	0\\
68.11	0\\
68.12	0\\
68.13	0\\
68.14	0\\
68.15	0\\
68.16	0\\
68.17	0\\
68.18	0\\
68.19	0\\
68.2	0\\
68.21	0\\
68.22	0\\
68.23	0\\
68.24	0\\
68.25	0\\
68.26	0\\
68.27	0\\
68.28	0\\
68.29	0\\
68.3	0\\
68.31	0\\
68.32	0\\
68.33	0\\
68.34	0\\
68.35	0\\
68.36	0\\
68.37	0\\
68.38	0\\
68.39	0\\
68.4	0\\
68.41	0\\
68.42	0\\
68.43	0\\
68.44	0\\
68.45	0\\
68.46	0\\
68.47	0\\
68.48	0\\
68.49	0\\
68.5	0\\
68.51	0\\
68.52	0\\
68.53	0\\
68.54	0\\
68.55	0\\
68.56	0\\
68.57	0\\
68.58	0\\
68.59	0\\
68.6	0\\
68.61	0\\
68.62	0\\
68.63	0\\
68.64	0\\
68.65	0\\
68.66	0\\
68.67	0\\
68.68	0\\
68.69	0\\
68.7	0\\
68.71	0\\
68.72	0\\
68.73	0\\
68.74	0\\
68.75	0\\
68.76	0\\
68.77	0\\
68.78	0\\
68.79	0\\
68.8	0\\
68.81	0\\
68.82	0\\
68.83	0\\
68.84	0\\
68.85	0\\
68.86	0\\
68.87	0\\
68.88	0\\
68.89	0\\
68.9	0\\
68.91	0\\
68.92	0\\
68.93	0\\
68.94	0\\
68.95	0\\
68.96	0\\
68.97	0\\
68.98	0\\
68.99	0\\
69	0\\
69.01	0\\
69.02	0\\
69.03	0\\
69.04	0\\
69.05	0\\
69.06	0\\
69.07	0\\
69.08	0\\
69.09	0\\
69.1	0\\
69.11	0\\
69.12	0\\
69.13	0\\
69.14	0\\
69.15	0\\
69.16	0\\
69.17	0\\
69.18	0\\
69.19	0\\
69.2	0\\
69.21	0\\
69.22	0\\
69.23	0\\
69.24	0\\
69.25	0\\
69.26	0\\
69.27	0\\
69.28	0\\
69.29	0\\
69.3	0\\
69.31	0\\
69.32	0\\
69.33	0\\
69.34	0\\
69.35	0\\
69.36	0\\
69.37	0\\
69.38	0\\
69.39	0\\
69.4	0\\
69.41	0\\
69.42	0\\
69.43	0\\
69.44	0\\
69.45	0\\
69.46	0\\
69.47	0\\
69.48	0\\
69.49	0\\
69.5	0\\
69.51	0\\
69.52	0\\
69.53	0\\
69.54	0\\
69.55	0\\
69.56	0\\
69.57	0\\
69.58	0\\
69.59	0\\
69.6	0\\
69.61	0\\
69.62	0\\
69.63	0\\
69.64	0\\
69.65	0\\
69.66	0\\
69.67	0\\
69.68	0\\
69.69	0\\
69.7	0\\
69.71	0\\
69.72	0\\
69.73	0\\
69.74	0\\
69.75	0\\
69.76	0\\
69.77	0\\
69.78	0\\
69.79	0\\
69.8	0\\
69.81	0\\
69.82	0\\
69.83	0\\
69.84	0\\
69.85	0\\
69.86	0\\
69.87	0\\
69.88	0\\
69.89	0\\
69.9	0\\
69.91	0\\
69.92	0\\
69.93	0\\
69.94	0\\
69.95	0\\
69.96	0\\
69.97	0\\
69.98	0\\
69.99	0\\
70	0\\
70.01	0\\
70.02	0\\
70.03	0\\
70.04	0\\
70.05	0\\
70.06	0\\
70.07	0\\
70.08	0\\
70.09	0\\
70.1	0\\
70.11	0\\
70.12	0\\
70.13	0\\
70.14	0\\
70.15	0\\
70.16	0\\
70.17	0\\
70.18	0\\
70.19	0\\
70.2	0\\
70.21	0\\
70.22	0\\
70.23	0\\
70.24	0\\
70.25	0\\
70.26	0\\
70.27	0\\
70.28	0\\
70.29	0\\
70.3	0\\
70.31	0\\
70.32	0\\
70.33	0\\
70.34	0\\
70.35	0\\
70.36	0\\
70.37	0\\
70.38	0\\
70.39	0\\
70.4	0\\
70.41	0\\
70.42	0\\
70.43	0\\
70.44	0\\
70.45	0\\
70.46	0\\
70.47	0\\
70.48	0\\
70.49	0\\
70.5	0\\
70.51	0\\
70.52	0\\
70.53	0\\
70.54	0\\
70.55	0\\
70.56	0\\
70.57	0\\
70.58	0\\
70.59	0\\
70.6	0\\
70.61	0\\
70.62	0\\
70.63	0\\
70.64	0\\
70.65	0\\
70.66	0\\
70.67	0\\
70.68	0\\
70.69	0\\
70.7	0\\
70.71	0\\
70.72	0\\
70.73	0\\
70.74	0\\
70.75	0\\
70.76	0\\
70.77	0\\
70.78	0\\
70.79	0\\
70.8	0\\
70.81	0\\
70.82	0\\
70.83	0\\
70.84	0\\
70.85	0\\
70.86	0\\
70.87	0\\
70.88	0\\
70.89	0\\
70.9	0\\
70.91	0\\
70.92	0\\
70.93	0\\
70.94	0\\
70.95	0\\
70.96	0\\
70.97	0\\
70.98	0\\
70.99	0\\
71	0\\
71.01	0\\
71.02	0\\
71.03	0\\
71.04	0\\
71.05	0\\
71.06	0\\
71.07	0\\
71.08	0\\
71.09	0\\
71.1	0\\
71.11	0\\
71.12	0\\
71.13	0\\
71.14	0\\
71.15	0\\
71.16	0\\
71.17	0\\
71.18	0\\
71.19	0\\
71.2	0\\
71.21	0\\
71.22	0\\
71.23	0\\
71.24	0\\
71.25	0\\
71.26	0\\
71.27	0\\
71.28	0\\
71.29	0\\
71.3	0\\
71.31	0\\
71.32	0\\
71.33	0\\
71.34	0\\
71.35	0\\
71.36	0\\
71.37	0\\
71.38	0\\
71.39	0\\
71.4	0\\
71.41	0\\
71.42	0\\
71.43	0\\
71.44	0\\
71.45	0\\
71.46	0\\
71.47	0\\
71.48	0\\
71.49	0\\
71.5	0\\
71.51	0\\
71.52	0\\
71.53	0\\
71.54	0\\
71.55	0\\
71.56	0\\
71.57	0\\
71.58	0\\
71.59	0\\
71.6	0\\
71.61	0\\
71.62	0\\
71.63	0\\
71.64	0\\
71.65	0\\
71.66	0\\
71.67	0\\
71.68	0\\
71.69	0\\
71.7	0\\
71.71	0\\
71.72	0\\
71.73	0\\
71.74	0\\
71.75	0\\
71.76	0\\
71.77	0\\
71.78	0\\
71.79	0\\
71.8	0\\
71.81	0\\
71.82	0\\
71.83	0\\
71.84	0\\
71.85	0\\
71.86	0\\
71.87	0\\
71.88	0\\
71.89	0\\
71.9	0\\
71.91	0\\
71.92	0\\
71.93	0\\
71.94	0\\
71.95	0\\
71.96	0\\
71.97	0\\
71.98	0\\
71.99	0\\
72	0\\
72.01	0\\
72.02	0\\
72.03	0\\
72.04	0\\
72.05	0\\
72.06	0\\
72.07	0\\
72.08	0\\
72.09	0\\
72.1	0\\
72.11	0\\
72.12	0\\
72.13	0\\
72.14	0\\
72.15	0\\
72.16	0\\
72.17	0\\
72.18	0\\
72.19	0\\
72.2	0\\
72.21	0\\
72.22	0\\
72.23	0\\
72.24	0\\
72.25	0\\
72.26	0\\
72.27	0\\
72.28	0\\
72.29	0\\
72.3	0\\
72.31	0\\
72.32	0\\
72.33	0\\
72.34	0\\
72.35	0\\
72.36	0\\
72.37	0\\
72.38	0\\
72.39	0\\
72.4	0\\
72.41	0\\
72.42	0\\
72.43	0\\
72.44	0\\
72.45	0\\
72.46	0\\
72.47	0\\
72.48	0\\
72.49	0\\
72.5	0\\
72.51	0\\
72.52	0\\
72.53	0\\
72.54	0\\
72.55	0\\
72.56	0\\
72.57	0\\
72.58	0\\
72.59	0\\
72.6	0\\
72.61	0\\
72.62	0\\
72.63	0\\
72.64	0\\
72.65	0\\
72.66	0\\
72.67	0\\
72.68	0\\
72.69	0\\
72.7	0\\
72.71	0\\
72.72	0\\
72.73	0\\
72.74	0\\
72.75	0\\
72.76	0\\
72.77	0\\
72.78	0\\
72.79	0\\
72.8	0\\
72.81	0\\
72.82	0\\
72.83	0\\
72.84	0\\
72.85	0\\
72.86	0\\
72.87	0\\
72.88	0\\
72.89	0\\
72.9	0\\
72.91	0\\
72.92	0\\
72.93	0\\
72.94	0\\
72.95	0\\
72.96	0\\
72.97	0\\
72.98	0\\
72.99	0\\
73	0\\
73.01	0\\
73.02	0\\
73.03	0\\
73.04	0\\
73.05	0\\
73.06	0\\
73.07	0\\
73.08	0\\
73.09	0\\
73.1	0\\
73.11	0\\
73.12	0\\
73.13	0\\
73.14	0\\
73.15	0\\
73.16	0\\
73.17	0\\
73.18	0\\
73.19	0\\
73.2	0\\
73.21	0\\
73.22	0\\
73.23	0\\
73.24	0\\
73.25	0\\
73.26	0\\
73.27	0\\
73.28	0\\
73.29	0\\
73.3	0\\
73.31	0\\
73.32	0\\
73.33	0\\
73.34	0\\
73.35	0\\
73.36	0\\
73.37	0\\
73.38	0\\
73.39	0\\
73.4	0\\
73.41	0\\
73.42	0\\
73.43	0\\
73.44	0\\
73.45	0\\
73.46	0\\
73.47	0\\
73.48	0\\
73.49	0\\
73.5	0\\
73.51	0\\
73.52	0\\
73.53	0\\
73.54	0\\
73.55	0\\
73.56	0\\
73.57	0\\
73.58	0\\
73.59	0\\
73.6	0\\
73.61	0\\
73.62	0\\
73.63	0\\
73.64	0\\
73.65	0\\
73.66	0\\
73.67	0\\
73.68	0\\
73.69	0\\
73.7	0\\
73.71	0\\
73.72	0\\
73.73	0\\
73.74	0\\
73.75	0\\
73.76	0\\
73.77	0\\
73.78	0\\
73.79	0\\
73.8	0\\
73.81	0\\
73.82	0\\
73.83	0\\
73.84	0\\
73.85	0\\
73.86	0\\
73.87	0\\
73.88	0\\
73.89	0\\
73.9	0\\
73.91	0\\
73.92	0\\
73.93	0\\
73.94	0\\
73.95	0\\
73.96	0\\
73.97	0\\
73.98	0\\
73.99	0\\
74	0\\
74.01	0\\
74.02	0\\
74.03	0\\
74.04	0\\
74.05	0\\
74.06	0\\
74.07	0\\
74.08	0\\
74.09	0\\
74.1	0\\
74.11	0\\
74.12	0\\
74.13	0\\
74.14	0\\
74.15	0\\
74.16	0\\
74.17	0\\
74.18	0\\
74.19	0\\
74.2	0\\
74.21	0\\
74.22	0\\
74.23	0\\
74.24	0\\
74.25	0\\
74.26	0\\
74.27	0\\
74.28	0\\
74.29	0\\
74.3	0\\
74.31	0\\
74.32	0\\
74.33	0\\
74.34	0\\
74.35	0\\
74.36	0\\
74.37	0\\
74.38	0\\
74.39	0\\
74.4	0\\
74.41	0\\
74.42	0\\
74.43	0\\
74.44	0\\
74.45	0\\
74.46	0\\
74.47	0\\
74.48	0\\
74.49	0\\
74.5	0\\
74.51	0\\
74.52	0\\
74.53	0\\
74.54	0\\
74.55	0\\
74.56	0\\
74.57	0\\
74.58	0\\
74.59	0\\
74.6	0\\
74.61	0\\
74.62	0\\
74.63	0\\
74.64	0\\
74.65	0\\
74.66	0\\
74.67	0\\
74.68	0\\
74.69	0\\
74.7	0\\
74.71	0\\
74.72	0\\
74.73	0\\
74.74	0\\
74.75	0\\
74.76	0\\
74.77	0\\
74.78	0\\
74.79	0\\
74.8	0\\
74.81	0\\
74.82	0\\
74.83	0\\
74.84	0\\
74.85	0\\
74.86	0\\
74.87	0\\
74.88	0\\
74.89	0\\
74.9	0\\
74.91	0\\
74.92	0\\
74.93	0\\
74.94	0\\
74.95	0\\
74.96	0\\
74.97	0\\
74.98	0\\
74.99	0\\
75	0\\
75.01	0\\
75.02	0\\
75.03	0\\
75.04	0\\
75.05	0\\
75.06	0\\
75.07	0\\
75.08	0\\
75.09	0\\
75.1	0\\
75.11	0\\
75.12	0\\
75.13	0\\
75.14	0\\
75.15	0\\
75.16	0\\
75.17	0\\
75.18	0\\
75.19	0\\
75.2	0\\
75.21	0\\
75.22	0\\
75.23	0\\
75.24	0\\
75.25	0\\
75.26	0\\
75.27	0\\
75.28	0\\
75.29	0\\
75.3	0\\
75.31	0\\
75.32	0\\
75.33	0\\
75.34	0\\
75.35	0\\
75.36	0\\
75.37	0\\
75.38	0\\
75.39	0\\
75.4	0\\
75.41	0\\
75.42	0\\
75.43	0\\
75.44	0\\
75.45	0\\
75.46	0\\
75.47	0\\
75.48	0\\
75.49	0\\
75.5	0\\
75.51	0\\
75.52	0\\
75.53	0\\
75.54	0\\
75.55	0\\
75.56	0\\
75.57	0\\
75.58	0\\
75.59	0\\
75.6	0\\
75.61	0\\
75.62	0\\
75.63	0\\
75.64	0\\
75.65	0\\
75.66	0\\
75.67	0\\
75.68	0\\
75.69	0\\
75.7	0\\
75.71	0\\
75.72	0\\
75.73	0\\
75.74	0\\
75.75	0\\
75.76	0\\
75.77	0\\
75.78	0\\
75.79	0\\
75.8	0\\
75.81	0\\
75.82	0\\
75.83	0\\
75.84	0\\
75.85	0\\
75.86	0\\
75.87	0\\
75.88	0\\
75.89	0\\
75.9	0\\
75.91	0\\
75.92	0\\
75.93	0\\
75.94	0\\
75.95	0\\
75.96	0\\
75.97	0\\
75.98	0\\
75.99	0\\
76	0\\
76.01	0\\
76.02	0\\
76.03	0\\
76.04	0\\
76.05	0\\
76.06	0\\
76.07	0\\
76.08	0\\
76.09	0\\
76.1	0\\
76.11	0\\
76.12	0\\
76.13	0\\
76.14	0\\
76.15	0\\
76.16	0\\
76.17	0\\
76.18	0\\
76.19	0\\
76.2	0\\
76.21	0\\
76.22	0\\
76.23	0\\
76.24	0\\
76.25	0\\
76.26	0\\
76.27	0\\
76.28	0\\
76.29	0\\
76.3	0\\
76.31	0\\
76.32	0\\
76.33	0\\
76.34	0\\
76.35	0\\
76.36	0\\
76.37	0\\
76.38	0\\
76.39	0\\
76.4	0\\
76.41	0\\
76.42	0\\
76.43	0\\
76.44	0\\
76.45	0\\
76.46	0\\
76.47	0\\
76.48	0\\
76.49	0\\
76.5	0\\
76.51	0\\
76.52	0\\
76.53	0\\
76.54	0\\
76.55	0\\
76.56	0\\
76.57	0\\
76.58	0\\
76.59	0\\
76.6	0\\
76.61	0\\
76.62	0\\
76.63	0\\
76.64	0\\
76.65	0\\
76.66	0\\
76.67	0\\
76.68	0\\
76.69	0\\
76.7	0\\
76.71	0\\
76.72	0\\
76.73	0\\
76.74	0\\
76.75	0\\
76.76	0\\
76.77	0\\
76.78	0\\
76.79	0\\
76.8	0\\
76.81	0\\
76.82	0\\
76.83	0\\
76.84	0\\
76.85	0\\
76.86	0\\
76.87	0\\
76.88	0\\
76.89	0\\
76.9	0\\
76.91	0\\
76.92	0\\
76.93	0\\
76.94	0\\
76.95	0\\
76.96	0\\
76.97	0\\
76.98	0\\
76.99	0\\
77	0\\
77.01	0\\
77.02	0\\
77.03	0\\
77.04	0\\
77.05	0\\
77.06	0\\
77.07	0\\
77.08	0\\
77.09	0\\
77.1	0\\
77.11	0\\
77.12	0\\
77.13	0\\
77.14	0\\
77.15	0\\
77.16	0\\
77.17	0\\
77.18	0\\
77.19	0\\
77.2	0\\
77.21	0\\
77.22	0\\
77.23	0\\
77.24	0\\
77.25	0\\
77.26	0\\
77.27	0\\
77.28	0\\
77.29	0\\
77.3	0\\
77.31	0\\
77.32	0\\
77.33	0\\
77.34	0\\
77.35	0\\
77.36	0\\
77.37	0\\
77.38	0\\
77.39	0\\
77.4	0\\
77.41	0\\
77.42	0\\
77.43	0\\
77.44	0\\
77.45	0\\
77.46	0\\
77.47	0\\
77.48	0\\
77.49	0\\
77.5	0\\
77.51	0\\
77.52	0\\
77.53	0\\
77.54	0\\
77.55	0\\
77.56	0\\
77.57	0\\
77.58	0\\
77.59	0\\
77.6	0\\
77.61	0\\
77.62	0\\
77.63	0\\
77.64	0\\
77.65	0\\
77.66	0\\
77.67	0\\
77.68	0\\
77.69	0\\
77.7	0\\
77.71	0\\
77.72	0\\
77.73	0\\
77.74	0\\
77.75	0\\
77.76	0\\
77.77	0\\
77.78	0\\
77.79	0\\
77.8	0\\
77.81	0\\
77.82	0\\
77.83	0\\
77.84	0\\
77.85	0\\
77.86	0\\
77.87	0\\
77.88	0\\
77.89	0\\
77.9	0\\
77.91	0\\
77.92	0\\
77.93	0\\
77.94	0\\
77.95	0\\
77.96	0\\
77.97	0\\
77.98	0\\
77.99	0\\
78	0\\
78.01	0\\
78.02	0\\
78.03	0\\
78.04	0\\
78.05	0\\
78.06	0\\
78.07	0\\
78.08	0\\
78.09	0\\
78.1	0\\
78.11	0\\
78.12	0\\
78.13	0\\
78.14	0\\
78.15	0\\
78.16	0\\
78.17	0\\
78.18	0\\
78.19	0\\
78.2	0\\
78.21	0\\
78.22	0\\
78.23	0\\
78.24	0\\
78.25	0\\
78.26	0\\
78.27	0\\
78.28	0\\
78.29	0\\
78.3	0\\
78.31	0\\
78.32	0\\
78.33	0\\
78.34	0\\
78.35	0\\
78.36	0\\
78.37	0\\
78.38	0\\
78.39	0\\
78.4	0\\
78.41	0\\
78.42	0\\
78.43	0\\
78.44	0\\
78.45	0\\
78.46	0\\
78.47	0\\
78.48	0\\
78.49	0\\
78.5	0\\
78.51	0\\
78.52	0\\
78.53	0\\
78.54	0\\
78.55	0\\
78.56	0\\
78.57	0\\
78.58	0\\
78.59	0\\
78.6	0\\
78.61	0\\
78.62	0\\
78.63	0\\
78.64	0\\
78.65	0\\
78.66	0\\
78.67	0\\
78.68	0\\
78.69	0\\
78.7	0\\
78.71	0\\
78.72	0\\
78.73	0\\
78.74	0\\
78.75	0\\
78.76	0\\
78.77	0\\
78.78	0\\
78.79	0\\
78.8	0\\
78.81	0\\
78.82	0\\
78.83	0\\
78.84	0\\
78.85	0\\
78.86	0\\
78.87	0\\
78.88	0\\
78.89	0\\
78.9	0\\
78.91	0\\
78.92	0\\
78.93	0\\
78.94	0\\
78.95	0\\
78.96	0\\
78.97	0\\
78.98	0\\
78.99	0\\
79	0\\
79.01	0\\
79.02	0\\
79.03	0\\
79.04	0\\
79.05	0\\
79.06	0\\
79.07	0\\
79.08	0\\
79.09	0\\
79.1	0\\
79.11	0\\
79.12	0\\
79.13	0\\
79.14	0\\
79.15	0\\
79.16	0\\
79.17	0\\
79.18	0\\
79.19	0\\
79.2	0\\
79.21	0\\
79.22	0\\
79.23	0\\
79.24	0\\
79.25	0\\
79.26	0\\
79.27	0\\
79.28	0\\
79.29	0\\
79.3	0\\
79.31	0\\
79.32	0\\
79.33	0\\
79.34	0\\
79.35	0\\
79.36	0\\
79.37	0\\
79.38	0\\
79.39	0\\
79.4	0\\
79.41	0\\
79.42	0\\
79.43	0\\
79.44	0\\
79.45	0\\
79.46	0\\
79.47	0\\
79.48	0\\
79.49	0\\
79.5	0\\
79.51	0\\
79.52	0\\
79.53	0\\
79.54	0\\
79.55	0\\
79.56	0\\
79.57	0\\
79.58	0\\
79.59	0\\
79.6	0\\
79.61	0\\
79.62	0\\
79.63	0\\
79.64	0\\
79.65	0\\
79.66	0\\
79.67	0\\
79.68	0\\
79.69	0\\
79.7	0\\
79.71	0\\
79.72	0\\
79.73	0\\
79.74	0\\
79.75	0\\
79.76	0\\
79.77	0\\
79.78	0\\
79.79	0\\
79.8	0\\
79.81	0\\
79.82	0\\
79.83	0\\
79.84	0\\
79.85	0\\
79.86	0\\
79.87	0\\
79.88	0\\
79.89	0\\
79.9	0\\
79.91	0\\
79.92	0\\
79.93	0\\
79.94	0\\
79.95	0\\
79.96	0\\
79.97	0\\
79.98	0\\
79.99	0\\
80	0\\
80.01	0\\
};
\addplot [color=mycolor1,dashed]
  table[row sep=crcr]{%
80.01	0\\
80.02	0\\
80.03	0\\
80.04	0\\
80.05	0\\
80.06	0\\
80.07	0\\
80.08	0\\
80.09	0\\
80.1	0\\
80.11	0\\
80.12	0\\
80.13	0\\
80.14	0\\
80.15	0\\
80.16	0\\
80.17	0\\
80.18	0\\
80.19	0\\
80.2	0\\
80.21	0\\
80.22	0\\
80.23	0\\
80.24	0\\
80.25	0\\
80.26	0\\
80.27	0\\
80.28	0\\
80.29	0\\
80.3	0\\
80.31	0\\
80.32	0\\
80.33	0\\
80.34	0\\
80.35	0\\
80.36	0\\
80.37	0\\
80.38	0\\
80.39	0\\
80.4	0\\
80.41	0\\
80.42	0\\
80.43	0\\
80.44	0\\
80.45	0\\
80.46	0\\
80.47	0\\
80.48	0\\
80.49	0\\
80.5	0\\
80.51	0\\
80.52	0\\
80.53	0\\
80.54	0\\
80.55	0\\
80.56	0\\
80.57	0\\
80.58	0\\
80.59	0\\
80.6	0\\
80.61	0\\
80.62	0\\
80.63	0\\
80.64	0\\
80.65	0\\
80.66	0\\
80.67	0\\
80.68	0\\
80.69	0\\
80.7	0\\
80.71	0\\
80.72	0\\
80.73	0\\
80.74	0\\
80.75	0\\
80.76	0\\
80.77	0\\
80.78	0\\
80.79	0\\
80.8	0\\
80.81	0\\
80.82	0\\
80.83	0\\
80.84	0\\
80.85	0\\
80.86	0\\
80.87	0\\
80.88	0\\
80.89	0\\
80.9	0\\
80.91	0\\
80.92	0\\
80.93	0\\
80.94	0\\
80.95	0\\
80.96	0\\
80.97	0\\
80.98	0\\
80.99	0\\
81	0\\
81.01	0\\
81.02	0\\
81.03	0\\
81.04	0\\
81.05	0\\
81.06	0\\
81.07	0\\
81.08	0\\
81.09	0\\
81.1	0\\
81.11	0\\
81.12	0\\
81.13	0\\
81.14	0\\
81.15	0\\
81.16	0\\
81.17	0\\
81.18	0\\
81.19	0\\
81.2	0\\
81.21	0\\
81.22	0\\
81.23	0\\
81.24	0\\
81.25	0\\
81.26	0\\
81.27	0\\
81.28	0\\
81.29	0\\
81.3	0\\
81.31	0\\
81.32	0\\
81.33	0\\
81.34	0\\
81.35	0\\
81.36	0\\
81.37	0\\
81.38	0\\
81.39	0\\
81.4	0\\
81.41	0\\
81.42	0\\
81.43	0\\
81.44	0\\
81.45	0\\
81.46	0\\
81.47	0\\
81.48	0\\
81.49	0\\
81.5	0\\
81.51	0\\
81.52	0\\
81.53	0\\
81.54	0\\
81.55	0\\
81.56	0\\
81.57	0\\
81.58	0\\
81.59	0\\
81.6	0\\
81.61	0\\
81.62	0\\
81.63	0\\
81.64	0\\
81.65	0\\
81.66	0\\
81.67	0\\
81.68	0\\
81.69	0\\
81.7	0\\
81.71	0\\
81.72	0\\
81.73	0\\
81.74	0\\
81.75	0\\
81.76	0\\
81.77	0\\
81.78	0\\
81.79	0\\
81.8	0\\
81.81	0\\
81.82	0\\
81.83	0\\
81.84	0\\
81.85	0\\
81.86	0\\
81.87	0\\
81.88	0\\
81.89	0\\
81.9	0\\
81.91	0\\
81.92	0\\
81.93	0\\
81.94	0\\
81.95	0\\
81.96	0\\
81.97	0\\
81.98	0\\
81.99	0\\
82	0\\
82.01	0\\
82.02	0\\
82.03	0\\
82.04	0\\
82.05	0\\
82.06	0\\
82.07	0\\
82.08	0\\
82.09	0\\
82.1	0\\
82.11	0\\
82.12	0\\
82.13	0\\
82.14	0\\
82.15	0\\
82.16	0\\
82.17	0\\
82.18	0\\
82.19	0\\
82.2	0\\
82.21	0\\
82.22	0\\
82.23	0\\
82.24	0\\
82.25	0\\
82.26	0\\
82.27	0\\
82.28	0\\
82.29	0\\
82.3	0\\
82.31	0\\
82.32	0\\
82.33	0\\
82.34	0\\
82.35	0\\
82.36	0\\
82.37	0\\
82.38	0\\
82.39	0\\
82.4	0\\
82.41	0\\
82.42	0\\
82.43	0\\
82.44	0\\
82.45	0\\
82.46	0\\
82.47	0\\
82.48	0\\
82.49	0\\
82.5	0\\
82.51	0\\
82.52	0\\
82.53	0\\
82.54	0\\
82.55	0\\
82.56	0\\
82.57	0\\
82.58	0\\
82.59	0\\
82.6	0\\
82.61	0\\
82.62	0\\
82.63	0\\
82.64	0\\
82.65	0\\
82.66	0\\
82.67	0\\
82.68	0\\
82.69	0\\
82.7	0\\
82.71	0\\
82.72	0\\
82.73	0\\
82.74	0\\
82.75	0\\
82.76	0\\
82.77	0\\
82.78	0\\
82.79	0\\
82.8	0\\
82.81	0\\
82.82	0\\
82.83	0\\
82.84	0\\
82.85	0\\
82.86	0\\
82.87	0\\
82.88	0\\
82.89	0\\
82.9	0\\
82.91	0\\
82.92	0\\
82.93	0\\
82.94	0\\
82.95	0\\
82.96	0\\
82.97	0\\
82.98	0\\
82.99	0\\
83	0\\
83.01	0\\
83.02	0\\
83.03	0\\
83.04	0\\
83.05	0\\
83.06	0\\
83.07	0\\
83.08	0\\
83.09	0\\
83.1	0\\
83.11	0\\
83.12	0\\
83.13	0\\
83.14	0\\
83.15	0\\
83.16	0\\
83.17	0\\
83.18	0\\
83.19	0\\
83.2	0\\
83.21	0\\
83.22	0\\
83.23	0\\
83.24	0\\
83.25	0\\
83.26	0\\
83.27	0\\
83.28	0\\
83.29	0\\
83.3	0\\
83.31	0\\
83.32	0\\
83.33	0\\
83.34	0\\
83.35	0\\
83.36	0\\
83.37	0\\
83.38	0\\
83.39	0\\
83.4	0\\
83.41	0\\
83.42	0\\
83.43	0\\
83.44	0\\
83.45	0\\
83.46	0\\
83.47	0\\
83.48	0\\
83.49	0\\
83.5	0\\
83.51	0\\
83.52	0\\
83.53	0\\
83.54	0\\
83.55	0\\
83.56	0\\
83.57	0\\
83.58	0\\
83.59	0\\
83.6	0\\
83.61	0\\
83.62	0\\
83.63	0\\
83.64	0\\
83.65	0\\
83.66	0\\
83.67	0\\
83.68	0\\
83.69	0\\
83.7	0\\
83.71	0\\
83.72	0\\
83.73	0\\
83.74	0\\
83.75	0\\
83.76	0\\
83.77	0\\
83.78	0\\
83.79	0\\
83.8	0\\
83.81	0\\
83.82	0\\
83.83	0\\
83.84	0\\
83.85	0\\
83.86	0\\
83.87	0\\
83.88	0\\
83.89	0\\
83.9	0\\
83.91	0\\
83.92	0\\
83.93	0\\
83.94	0\\
83.95	0\\
83.96	0\\
83.97	0\\
83.98	0\\
83.99	0\\
84	0\\
84.01	0\\
84.02	0\\
84.03	0\\
84.04	0\\
84.05	0\\
84.06	0\\
84.07	0\\
84.08	0\\
84.09	0\\
84.1	0\\
84.11	0\\
84.12	0\\
84.13	0\\
84.14	0\\
84.15	0\\
84.16	0\\
84.17	0\\
84.18	0\\
84.19	0\\
84.2	0\\
84.21	0\\
84.22	0\\
84.23	0\\
84.24	0\\
84.25	0\\
84.26	0\\
84.27	0\\
84.28	0\\
84.29	0\\
84.3	0\\
84.31	0\\
84.32	0\\
84.33	0\\
84.34	0\\
84.35	0\\
84.36	0\\
84.37	0\\
84.38	0\\
84.39	0\\
84.4	0\\
84.41	0\\
84.42	0\\
84.43	0\\
84.44	0\\
84.45	0\\
84.46	0\\
84.47	0\\
84.48	0\\
84.49	0\\
84.5	0\\
84.51	0\\
84.52	0\\
84.53	0\\
84.54	0\\
84.55	0\\
84.56	0\\
84.57	0\\
84.58	0\\
84.59	0\\
84.6	0\\
84.61	0\\
84.62	0\\
84.63	0\\
84.64	0\\
84.65	0\\
84.66	0\\
84.67	0\\
84.68	0\\
84.69	0\\
84.7	0\\
84.71	0\\
84.72	0\\
84.73	0\\
84.74	0\\
84.75	0\\
84.76	0\\
84.77	0\\
84.78	0\\
84.79	0\\
84.8	0\\
84.81	0\\
84.82	0\\
84.83	0\\
84.84	0\\
84.85	0\\
84.86	0\\
84.87	0\\
84.88	0\\
84.89	0\\
84.9	0\\
84.91	0\\
84.92	0\\
84.93	0\\
84.94	0\\
84.95	0\\
84.96	0\\
84.97	0\\
84.98	0\\
84.99	0\\
85	0\\
85.01	0\\
85.02	0\\
85.03	0\\
85.04	0\\
85.05	0\\
85.06	0\\
85.07	0\\
85.08	0\\
85.09	0\\
85.1	0\\
85.11	0\\
85.12	0\\
85.13	0\\
85.14	0\\
85.15	0\\
85.16	0\\
85.17	0\\
85.18	0\\
85.19	0\\
85.2	0\\
85.21	0\\
85.22	0\\
85.23	0\\
85.24	0\\
85.25	0\\
85.26	0\\
85.27	0\\
85.28	0\\
85.29	0\\
85.3	0\\
85.31	0\\
85.32	0\\
85.33	0\\
85.34	0\\
85.35	0\\
85.36	0\\
85.37	0\\
85.38	0\\
85.39	0\\
85.4	0\\
85.41	0\\
85.42	0\\
85.43	0\\
85.44	0\\
85.45	0\\
85.46	0\\
85.47	0\\
85.48	0\\
85.49	0\\
85.5	0\\
85.51	0\\
85.52	0\\
85.53	0\\
85.54	0\\
85.55	0\\
85.56	0\\
85.57	0\\
85.58	0\\
85.59	0\\
85.6	0\\
85.61	0\\
85.62	0\\
85.63	0\\
85.64	0\\
85.65	0\\
85.66	0\\
85.67	0\\
85.68	0\\
85.69	0\\
85.7	0\\
85.71	0\\
85.72	0\\
85.73	0\\
85.74	0\\
85.75	0\\
85.76	0\\
85.77	0\\
85.78	0\\
85.79	0\\
85.8	0\\
85.81	0\\
85.82	0\\
85.83	0\\
85.84	0\\
85.85	0\\
85.86	0\\
85.87	0\\
85.88	0\\
85.89	0\\
85.9	0\\
85.91	0\\
85.92	0\\
85.93	0\\
85.94	0\\
85.95	0\\
85.96	0\\
85.97	0\\
85.98	0\\
85.99	0\\
86	0\\
86.01	0\\
86.02	0\\
86.03	0\\
86.04	0\\
86.05	0\\
86.06	0\\
86.07	0\\
86.08	0\\
86.09	0\\
86.1	0\\
86.11	0\\
86.12	0\\
86.13	0\\
86.14	0\\
86.15	0\\
86.16	0\\
86.17	0\\
86.18	0\\
86.19	0\\
86.2	0\\
86.21	0\\
86.22	0\\
86.23	0\\
86.24	0\\
86.25	0\\
86.26	0\\
86.27	0\\
86.28	0\\
86.29	0\\
86.3	0\\
86.31	0\\
86.32	0\\
86.33	0\\
86.34	0\\
86.35	0\\
86.36	0\\
86.37	0\\
86.38	0\\
86.39	0\\
86.4	0\\
86.41	0\\
86.42	0\\
86.43	0\\
86.44	0\\
86.45	0\\
86.46	0\\
86.47	0\\
86.48	0\\
86.49	0\\
86.5	0\\
86.51	0\\
86.52	0\\
86.53	0\\
86.54	0\\
86.55	0\\
86.56	0\\
86.57	0\\
86.58	0\\
86.59	0\\
86.6	0\\
86.61	0\\
86.62	0\\
86.63	0\\
86.64	0\\
86.65	0\\
86.66	0\\
86.67	0\\
86.68	0\\
86.69	0\\
86.7	0\\
86.71	0\\
86.72	0\\
86.73	0\\
86.74	0\\
86.75	0\\
86.76	0\\
86.77	0\\
86.78	0\\
86.79	0\\
86.8	0\\
86.81	0\\
86.82	0\\
86.83	0\\
86.84	0\\
86.85	0\\
86.86	0\\
86.87	0\\
86.88	0\\
86.89	0\\
86.9	0\\
86.91	0\\
86.92	0\\
86.93	0\\
86.94	0\\
86.95	0\\
86.96	0\\
86.97	0\\
86.98	0\\
86.99	0\\
87	0\\
87.01	0\\
87.02	0\\
87.03	0\\
87.04	0\\
87.05	0\\
87.06	0\\
87.07	0\\
87.08	0\\
87.09	0\\
87.1	0\\
87.11	0\\
87.12	0\\
87.13	0\\
87.14	0\\
87.15	0\\
87.16	0\\
87.17	0\\
87.18	0\\
87.19	0\\
87.2	0\\
87.21	0\\
87.22	0\\
87.23	0\\
87.24	0\\
87.25	0\\
87.26	0\\
87.27	0\\
87.28	0\\
87.29	0\\
87.3	0\\
87.31	0\\
87.32	0\\
87.33	0\\
87.34	0\\
87.35	0\\
87.36	0\\
87.37	0\\
87.38	0\\
87.39	0\\
87.4	0\\
87.41	0\\
87.42	0\\
87.43	0\\
87.44	0\\
87.45	0\\
87.46	0\\
87.47	0\\
87.48	0\\
87.49	0\\
87.5	0\\
87.51	0\\
87.52	0\\
87.53	0\\
87.54	0\\
87.55	0\\
87.56	0\\
87.57	0\\
87.58	0\\
87.59	0\\
87.6	0\\
87.61	0\\
87.62	0\\
87.63	0\\
87.64	0\\
87.65	0\\
87.66	0\\
87.67	0\\
87.68	0\\
87.69	0\\
87.7	0\\
87.71	0\\
87.72	0\\
87.73	0\\
87.74	0\\
87.75	0\\
87.76	0\\
87.77	0\\
87.78	0\\
87.79	0\\
87.8	0\\
87.81	0\\
87.82	0\\
87.83	0\\
87.84	0\\
87.85	0\\
87.86	0\\
87.87	0\\
87.88	0\\
87.89	0\\
87.9	0\\
87.91	0\\
87.92	0\\
87.93	0\\
87.94	0\\
87.95	0\\
87.96	0\\
87.97	0\\
87.98	0\\
87.99	0\\
88	0\\
88.01	0\\
88.02	0\\
88.03	0\\
88.04	0\\
88.05	0\\
88.06	0\\
88.07	0\\
88.08	0\\
88.09	0\\
88.1	0\\
88.11	0\\
88.12	0\\
88.13	0\\
88.14	0\\
88.15	0\\
88.16	0\\
88.17	0\\
88.18	0\\
88.19	0\\
88.2	0\\
88.21	0\\
88.22	0\\
88.23	0\\
88.24	0\\
88.25	0\\
88.26	0\\
88.27	0\\
88.28	0\\
88.29	0\\
88.3	0\\
88.31	0\\
88.32	0\\
88.33	0\\
88.34	0\\
88.35	0\\
88.36	0\\
88.37	0\\
88.38	0\\
88.39	0\\
88.4	0\\
88.41	0\\
88.42	0\\
88.43	0\\
88.44	0\\
88.45	0\\
88.46	0\\
88.47	0\\
88.48	0\\
88.49	0\\
88.5	0\\
88.51	0\\
88.52	0\\
88.53	0\\
88.54	0\\
88.55	0\\
88.56	0\\
88.57	0\\
88.58	0\\
88.59	0\\
88.6	0\\
88.61	0\\
88.62	0\\
88.63	0\\
88.64	0\\
88.65	0\\
88.66	0\\
88.67	0\\
88.68	0\\
88.69	0\\
88.7	0\\
88.71	0\\
88.72	0\\
88.73	0\\
88.74	0\\
88.75	0\\
88.76	0\\
88.77	0\\
88.78	0\\
88.79	0\\
88.8	0\\
88.81	0\\
88.82	0\\
88.83	0\\
88.84	0\\
88.85	0\\
88.86	0\\
88.87	0\\
88.88	0\\
88.89	0\\
88.9	0\\
88.91	0\\
88.92	0\\
88.93	0\\
88.94	0\\
88.95	0\\
88.96	0\\
88.97	0\\
88.98	0\\
88.99	0\\
89	0\\
89.01	0\\
89.02	0\\
89.03	0\\
89.04	0\\
89.05	0\\
89.06	0\\
89.07	0\\
89.08	0\\
89.09	0\\
89.1	0\\
89.11	0\\
89.12	0\\
89.13	0\\
89.14	0\\
89.15	0\\
89.16	0\\
89.17	0\\
89.18	0\\
89.19	0\\
89.2	0\\
89.21	0\\
89.22	0\\
89.23	0\\
89.24	0\\
89.25	0\\
89.26	0\\
89.27	0\\
89.28	0\\
89.29	0\\
89.3	0\\
89.31	0\\
89.32	0\\
89.33	0\\
89.34	0\\
89.35	0\\
89.36	0\\
89.37	0\\
89.38	0\\
89.39	0\\
89.4	0\\
89.41	0\\
89.42	0\\
89.43	0\\
89.44	0\\
89.45	0\\
89.46	0\\
89.47	0\\
89.48	0\\
89.49	0\\
89.5	0\\
89.51	0\\
89.52	0\\
89.53	0\\
89.54	0\\
89.55	0\\
89.56	0\\
89.57	0\\
89.58	0\\
89.59	0\\
89.6	0\\
89.61	0\\
89.62	0\\
89.63	0\\
89.64	0\\
89.65	0\\
89.66	0\\
89.67	0\\
89.68	0\\
89.69	0\\
89.7	0\\
89.71	0\\
89.72	0\\
89.73	0\\
89.74	0\\
89.75	0\\
89.76	0\\
89.77	0\\
89.78	0\\
89.79	0\\
89.8	0\\
89.81	0\\
89.82	0\\
89.83	0\\
89.84	0\\
89.85	0\\
89.86	0\\
89.87	0\\
89.88	0\\
89.89	0\\
89.9	0\\
89.91	0\\
89.92	0\\
89.93	0\\
89.94	0\\
89.95	0\\
89.96	0\\
89.97	0\\
89.98	0\\
89.99	0\\
90	0\\
90.01	0\\
90.02	0\\
90.03	0\\
90.04	0\\
90.05	0\\
90.06	0\\
90.07	0\\
90.08	0\\
90.09	0\\
90.1	0\\
90.11	0\\
90.12	0\\
90.13	0\\
90.14	0\\
90.15	0\\
90.16	0\\
90.17	0\\
90.18	0\\
90.19	0\\
90.2	0\\
90.21	0\\
90.22	0\\
90.23	0\\
90.24	0\\
90.25	0\\
90.26	0\\
90.27	0\\
90.28	0\\
90.29	0\\
90.3	0\\
90.31	0\\
90.32	0\\
90.33	0\\
90.34	0\\
90.35	0\\
90.36	0\\
90.37	0\\
90.38	0\\
90.39	0\\
90.4	0\\
90.41	0\\
90.42	0\\
90.43	0\\
90.44	0\\
90.45	0\\
90.46	0\\
90.47	0\\
90.48	0\\
90.49	0\\
90.5	0\\
90.51	0\\
90.52	0\\
90.53	0\\
90.54	0\\
90.55	0\\
90.56	0\\
90.57	0\\
90.58	0\\
90.59	0\\
90.6	0\\
90.61	0\\
90.62	0\\
90.63	0\\
90.64	0\\
90.65	0\\
90.66	0\\
90.67	0\\
90.68	0\\
90.69	0\\
90.7	0\\
90.71	0\\
90.72	0\\
90.73	0\\
90.74	0\\
90.75	0\\
90.76	0\\
90.77	0\\
90.78	0\\
90.79	0\\
90.8	0\\
90.81	0\\
90.82	0\\
90.83	0\\
90.84	0\\
90.85	0\\
90.86	0\\
90.87	0\\
90.88	0\\
90.89	0\\
90.9	0\\
90.91	0\\
90.92	0\\
90.93	0\\
90.94	0\\
90.95	0\\
90.96	0\\
90.97	0\\
90.98	0\\
90.99	0\\
91	0\\
91.01	0\\
91.02	0\\
91.03	0\\
91.04	0\\
91.05	0\\
91.06	0\\
91.07	0\\
91.08	0\\
91.09	0\\
91.1	0\\
91.11	0\\
91.12	0\\
91.13	0\\
91.14	0\\
91.15	0\\
91.16	0\\
91.17	0\\
91.18	0\\
91.19	0\\
91.2	0\\
91.21	0\\
91.22	0\\
91.23	0\\
91.24	0\\
91.25	0\\
91.26	0\\
91.27	0\\
91.28	0\\
91.29	0\\
91.3	0\\
91.31	0\\
91.32	0\\
91.33	0\\
91.34	0\\
91.35	0\\
91.36	0\\
91.37	0\\
91.38	0\\
91.39	0\\
91.4	0\\
91.41	0\\
91.42	0\\
91.43	0\\
91.44	0\\
91.45	0\\
91.46	0\\
91.47	0\\
91.48	0\\
91.49	0\\
91.5	0\\
91.51	0\\
91.52	0\\
91.53	0\\
91.54	0\\
91.55	0\\
91.56	0\\
91.57	0\\
91.58	0\\
91.59	0\\
91.6	0\\
91.61	0\\
91.62	0\\
91.63	0\\
91.64	0\\
91.65	0\\
91.66	0\\
91.67	0\\
91.68	0\\
91.69	0\\
91.7	0\\
91.71	0\\
91.72	0\\
91.73	0\\
91.74	0\\
91.75	0\\
91.76	0\\
91.77	0\\
91.78	0\\
91.79	0\\
91.8	0\\
91.81	0\\
91.82	0\\
91.83	0\\
91.84	0\\
91.85	0\\
91.86	0\\
91.87	0\\
91.88	0\\
91.89	0\\
91.9	0\\
91.91	0\\
91.92	0\\
91.93	0\\
91.94	0\\
91.95	0\\
91.96	0\\
91.97	0\\
91.98	0\\
91.99	0\\
92	0\\
92.01	0\\
92.02	0\\
92.03	0\\
92.04	0\\
92.05	0\\
92.06	0\\
92.07	0\\
92.08	0\\
92.09	0\\
92.1	0\\
92.11	0\\
92.12	0\\
92.13	0\\
92.14	0\\
92.15	0\\
92.16	0\\
92.17	0\\
92.18	0\\
92.19	0\\
92.2	0\\
92.21	0\\
92.22	0\\
92.23	0\\
92.24	0\\
92.25	0\\
92.26	0\\
92.27	0\\
92.28	0\\
92.29	0\\
92.3	0\\
92.31	0\\
92.32	0\\
92.33	0\\
92.34	0\\
92.35	0\\
92.36	0\\
92.37	0\\
92.38	0\\
92.39	0\\
92.4	0\\
92.41	0\\
92.42	0\\
92.43	0\\
92.44	0\\
92.45	0\\
92.46	0\\
92.47	0\\
92.48	0\\
92.49	0\\
92.5	0\\
92.51	0\\
92.52	0\\
92.53	0\\
92.54	0\\
92.55	0\\
92.56	0\\
92.57	0\\
92.58	0\\
92.59	0\\
92.6	0\\
92.61	0\\
92.62	0\\
92.63	0\\
92.64	0\\
92.65	0\\
92.66	0\\
92.67	0\\
92.68	0\\
92.69	0\\
92.7	0\\
92.71	0\\
92.72	0\\
92.73	0\\
92.74	0\\
92.75	0\\
92.76	0\\
92.77	0\\
92.78	0\\
92.79	0\\
92.8	0\\
92.81	0\\
92.82	0\\
92.83	0\\
92.84	0\\
92.85	0\\
92.86	0\\
92.87	0\\
92.88	0\\
92.89	0\\
92.9	0\\
92.91	0\\
92.92	0\\
92.93	0\\
92.94	0\\
92.95	0\\
92.96	0\\
92.97	0\\
92.98	0\\
92.99	0\\
93	0\\
93.01	0\\
93.02	0\\
93.03	0\\
93.04	0\\
93.05	0\\
93.06	0\\
93.07	0\\
93.08	0\\
93.09	0\\
93.1	0\\
93.11	0\\
93.12	0\\
93.13	0\\
93.14	0\\
93.15	0\\
93.16	0\\
93.17	0\\
93.18	0\\
93.19	0\\
93.2	0\\
93.21	0\\
93.22	0\\
93.23	0\\
93.24	0\\
93.25	0\\
93.26	0\\
93.27	0\\
93.28	0\\
93.29	0\\
93.3	0\\
93.31	0\\
93.32	0\\
93.33	0\\
93.34	0\\
93.35	0\\
93.36	0\\
93.37	0\\
93.38	0\\
93.39	0\\
93.4	0\\
93.41	0\\
93.42	0\\
93.43	0\\
93.44	0\\
93.45	0\\
93.46	0\\
93.47	0\\
93.48	0\\
93.49	0\\
93.5	0\\
93.51	0\\
93.52	0\\
93.53	0\\
93.54	0\\
93.55	0\\
93.56	0\\
93.57	0\\
93.58	0\\
93.59	0\\
93.6	0\\
93.61	0\\
93.62	0\\
93.63	0\\
93.64	0\\
93.65	0\\
93.66	0\\
93.67	0\\
93.68	0\\
93.69	0\\
93.7	0\\
93.71	0\\
93.72	0\\
93.73	0\\
93.74	0\\
93.75	0\\
93.76	0\\
93.77	0\\
93.78	0\\
93.79	0\\
93.8	0\\
93.81	0\\
93.82	0\\
93.83	0\\
93.84	0\\
93.85	0\\
93.86	0\\
93.87	0\\
93.88	0\\
93.89	0\\
93.9	0\\
93.91	0\\
93.92	0\\
93.93	0\\
93.94	0\\
93.95	0\\
93.96	0\\
93.97	0\\
93.98	0\\
93.99	0\\
94	0\\
94.01	0\\
94.02	0\\
94.03	0\\
94.04	0\\
94.05	0\\
94.06	0\\
94.07	0\\
94.08	0\\
94.09	0\\
94.1	0\\
94.11	0\\
94.12	0\\
94.13	0\\
94.14	0\\
94.15	0\\
94.16	0\\
94.17	0\\
94.18	0\\
94.19	0\\
94.2	0\\
94.21	0\\
94.22	0\\
94.23	0\\
94.24	0\\
94.25	0\\
94.26	0\\
94.27	0\\
94.28	0\\
94.29	0\\
94.3	0\\
94.31	0\\
94.32	0\\
94.33	0\\
94.34	0\\
94.35	0\\
94.36	0\\
94.37	0\\
94.38	0\\
94.39	0\\
94.4	0\\
94.41	0\\
94.42	0\\
94.43	0\\
94.44	0\\
94.45	0\\
94.46	0\\
94.47	0\\
94.48	0\\
94.49	0\\
94.5	0\\
94.51	0\\
94.52	0\\
94.53	0\\
94.54	0\\
94.55	0\\
94.56	0\\
94.57	0\\
94.58	0\\
94.59	0\\
94.6	0\\
94.61	0\\
94.62	0\\
94.63	0\\
94.64	0\\
94.65	0\\
94.66	0\\
94.67	0\\
94.68	0\\
94.69	0\\
94.7	0\\
94.71	0\\
94.72	0\\
94.73	0\\
94.74	0\\
94.75	0\\
94.76	0\\
94.77	0\\
94.78	0\\
94.79	0\\
94.8	0\\
94.81	0\\
94.82	0\\
94.83	0\\
94.84	0\\
94.85	0\\
94.86	0\\
94.87	0\\
94.88	0\\
94.89	0\\
94.9	0\\
94.91	0\\
94.92	0\\
94.93	0\\
94.94	0\\
94.95	0\\
94.96	0\\
94.97	0\\
94.98	0\\
94.99	0\\
95	0\\
95.01	0\\
95.02	0\\
95.03	0\\
95.04	0\\
95.05	0\\
95.06	0\\
95.07	0\\
95.08	0\\
95.09	0\\
95.1	0\\
95.11	0\\
95.12	0\\
95.13	0\\
95.14	0\\
95.15	0\\
95.16	0\\
95.17	0\\
95.18	0\\
95.19	0\\
95.2	0\\
95.21	0\\
95.22	0\\
95.23	0\\
95.24	0\\
95.25	0\\
95.26	0\\
95.27	0\\
95.28	0\\
95.29	0\\
95.3	0\\
95.31	0\\
95.32	0\\
95.33	0\\
95.34	0\\
95.35	0\\
95.36	0\\
95.37	0\\
95.38	0\\
95.39	0\\
95.4	0\\
95.41	0\\
95.42	0\\
95.43	0\\
95.44	0\\
95.45	0\\
95.46	0\\
95.47	0\\
95.48	0\\
95.49	0\\
95.5	0\\
95.51	0\\
95.52	0\\
95.53	0\\
95.54	0\\
95.55	0\\
95.56	0\\
95.57	0\\
95.58	0\\
95.59	0\\
95.6	0\\
95.61	0\\
95.62	0\\
95.63	0\\
95.64	0\\
95.65	0\\
95.66	0\\
95.67	0\\
95.68	0\\
95.69	0\\
95.7	0\\
95.71	0\\
95.72	0\\
95.73	0\\
95.74	0\\
95.75	0\\
95.76	0\\
95.77	0\\
95.78	0\\
95.79	0\\
95.8	0\\
95.81	0\\
95.82	0\\
95.83	0\\
95.84	0\\
95.85	0\\
95.86	0\\
95.87	0\\
95.88	0\\
95.89	0\\
95.9	0\\
95.91	0\\
95.92	0\\
95.93	0\\
95.94	0\\
95.95	0\\
95.96	0\\
95.97	0\\
95.98	0\\
95.99	0\\
96	0\\
96.01	0\\
96.02	0\\
96.03	0\\
96.04	0\\
96.05	0\\
96.06	0\\
96.07	0\\
96.08	0\\
96.09	0\\
96.1	0\\
96.11	0\\
96.12	0\\
96.13	0\\
96.14	0\\
96.15	0\\
96.16	0\\
96.17	0\\
96.18	0\\
96.19	0\\
96.2	0\\
96.21	0\\
96.22	0\\
96.23	0\\
96.24	0\\
96.25	0\\
96.26	0\\
96.27	0\\
96.28	0\\
96.29	0\\
96.3	0\\
96.31	0\\
96.32	0\\
96.33	0\\
96.34	0\\
96.35	0\\
96.36	0\\
96.37	0\\
96.38	0\\
96.39	0\\
96.4	0\\
96.41	0\\
96.42	0\\
96.43	0\\
96.44	0\\
96.45	0\\
96.46	0\\
96.47	0\\
96.48	0\\
96.49	0\\
96.5	0\\
96.51	0\\
96.52	0\\
96.53	0\\
96.54	0\\
96.55	0\\
96.56	0\\
96.57	0\\
96.58	0\\
96.59	0\\
96.6	0\\
96.61	0\\
96.62	0\\
96.63	0\\
96.64	0\\
96.65	0\\
96.66	0\\
96.67	0\\
96.68	0\\
96.69	0\\
96.7	0\\
96.71	0\\
96.72	0\\
96.73	0\\
96.74	0\\
96.75	0\\
96.76	0\\
96.77	0\\
96.78	0\\
96.79	0\\
96.8	0\\
96.81	0\\
96.82	0\\
96.83	0\\
96.84	0\\
96.85	0\\
96.86	0\\
96.87	0\\
96.88	0\\
96.89	0\\
96.9	0\\
96.91	0\\
96.92	0\\
96.93	0\\
96.94	0\\
96.95	0\\
96.96	0\\
96.97	0\\
96.98	0\\
96.99	0\\
97	0\\
97.01	0\\
97.02	0\\
97.03	0\\
97.04	0\\
97.05	0\\
97.06	0\\
97.07	0\\
97.08	0\\
97.09	0\\
97.1	0\\
97.11	0\\
97.12	0\\
97.13	0\\
97.14	0\\
97.15	0\\
97.16	0\\
97.17	0\\
97.18	0\\
97.19	0\\
97.2	0\\
97.21	0\\
97.22	0\\
97.23	0\\
97.24	0\\
97.25	0\\
97.26	0\\
97.27	0\\
97.28	0\\
97.29	0\\
97.3	0\\
97.31	0\\
97.32	0\\
97.33	0\\
97.34	0\\
97.35	0\\
97.36	0\\
97.37	0\\
97.38	0\\
97.39	0\\
97.4	0\\
97.41	0\\
97.42	0\\
97.43	0\\
97.44	0\\
97.45	0\\
97.46	0\\
97.47	0\\
97.48	0\\
97.49	0\\
97.5	0\\
97.51	0\\
97.52	0\\
97.53	0\\
97.54	0\\
97.55	0\\
97.56	0\\
97.57	0\\
97.58	0\\
97.59	0\\
97.6	0\\
97.61	0\\
97.62	0\\
97.63	0\\
97.64	0\\
97.65	0\\
97.66	0\\
97.67	0\\
97.68	0\\
97.69	0\\
97.7	0\\
97.71	0\\
97.72	0\\
97.73	0\\
97.74	0\\
97.75	0\\
97.76	0\\
97.77	0\\
97.78	0\\
97.79	0\\
97.8	0\\
97.81	0\\
97.82	0\\
97.83	0\\
97.84	0\\
97.85	0\\
97.86	0\\
97.87	0\\
97.88	0\\
97.89	0\\
97.9	0\\
97.91	0\\
97.92	0\\
97.93	0\\
97.94	0\\
97.95	0\\
97.96	0\\
97.97	0\\
97.98	0\\
97.99	0\\
98	0\\
98.01	0\\
98.02	0\\
98.03	0\\
98.04	0\\
98.05	3.62724723310531e-05\\
98.06	0.000120003855284247\\
98.07	0.000204379720968137\\
98.08	0.000289406336152986\\
98.09	0.000375090034574817\\
98.1	0.00046143722446261\\
98.11	0.000548454391553085\\
98.12	0.000636148091755454\\
98.13	0.000724524952073445\\
98.14	0.000813591671543767\\
98.15	0.000903355022191549\\
98.16	0.00094558642743571\\
98.17	0.000964904203007898\\
98.18	0.000984377791717249\\
98.19	0.00100401234294328\\
98.2	0.00102380914862103\\
98.21	0.00104376951060383\\
98.22	0.00106389473755049\\
98.23	0.00108418614485106\\
98.24	0.00110464505454799\\
98.25	0.00112527279525247\\
98.26	0.0011460707020558\\
98.27	0.00116704011643543\\
98.28	0.00118818238615577\\
98.29	0.00120949886516327\\
98.3	0.0012309909134758\\
98.31	0.00125265989706593\\
98.32	0.00127450718773804\\
98.33	0.00129655088801074\\
98.34	0.00131879397499367\\
98.35	0.00134123836552923\\
98.36	0.00136388594673886\\
98.37	0.00138673862359563\\
98.38	0.00140979831909745\\
98.39	0.00143306697444223\\
98.4	0.00145654654920784\\
98.41	0.00148023902086371\\
98.42	0.00150414638013303\\
98.43	0.00152827063641009\\
98.44	0.00155261381793702\\
98.45	0.00157717797198227\\
98.46	0.00160196516502085\\
98.47	0.00162697748291628\\
98.48	0.00165221703110441\\
98.49	0.00167768593477895\\
98.5	0.00170338633907891\\
98.51	0.0017293204092778\\
98.52	0.00175549033097478\\
98.53	0.00178189831028761\\
98.54	0.0018085465740476\\
98.55	0.00183543736999636\\
98.56	0.00186257296698598\\
98.57	0.00188995565518988\\
98.58	0.00191758774630612\\
98.59	0.00194547157376284\\
98.6	0.00197360949292573\\
98.61	0.00200200388130751\\
98.62	0.00203065713877966\\
98.63	0.00205957168725982\\
98.64	0.00208874996727528\\
98.65	0.00211819444212324\\
98.66	0.00214790759808243\\
98.67	0.00217789194462653\\
98.68	0.00220815001463968\\
98.69	0.00223868436463388\\
98.7	0.00226949757496833\\
98.71	0.00230059225007091\\
98.72	0.0023319710186616\\
98.73	0.00236363653397786\\
98.74	0.00239559147401015\\
98.75	0.00242783854174076\\
98.76	0.00246038046537646\\
98.77	0.00249321999858351\\
98.78	0.00252635992072465\\
98.79	0.00255980303709871\\
98.8	0.00259355217918758\\
98.81	0.00262761020490035\\
98.82	0.00266197999881968\\
98.83	0.00269666447245046\\
98.84	0.00273166656447082\\
98.85	0.00276698924098544\\
98.86	0.00280263549578123\\
98.87	0.00283860835058542\\
98.88	0.002874910855326\\
98.89	0.00291154608839463\\
98.9	0.00294851715691199\\
98.91	0.00298582719699561\\
98.92	0.00302347937403009\\
98.93	0.00306147688294\\
98.94	0.00309982294846515\\
98.95	0.0031385208254385\\
98.96	0.00317757379906671\\
98.97	0.00321698518521315\\
98.98	0.0032567583306837\\
98.99	0.00329689661351513\\
99	0.00333740344326616\\
99.01	0.00337828226131128\\
99.02	0.00341953654113725\\
99.03	0.00346116978864239\\
99.04	0.00350318554243863\\
99.05	0.00354558737415644\\
99.06	0.00358837888875248\\
99.07	0.00363156372482022\\
99.08	0.00367514555490337\\
99.09	0.00371912808581235\\
99.1	0.00376351505894353\\
99.11	0.00380831025060157\\
99.12	0.00385351747232472\\
99.13	0.00389914057121316\\
99.14	0.00394518343026037\\
99.15	0.00399164996868758\\
99.16	0.00403854414228137\\
99.17	0.00408586994373441\\
99.18	0.0041336314029893\\
99.19	0.00418183254997663\\
99.2	0.0042304774492322\\
99.21	0.00427957020281343\\
99.22	0.00432911495064515\\
99.23	0.00437911587086857\\
99.24	0.00442957718019346\\
99.25	0.0044805031342536\\
99.26	0.00453189802796542\\
99.27	0.00458376619589004\\
99.28	0.00463611201259857\\
99.29	0.00468893989304077\\
99.3	0.00474225429291715\\
99.31	0.00479605970905447\\
99.32	0.00485036067978466\\
99.33	0.00490516178532726\\
99.34	0.00496046764817539\\
99.35	0.00501628293348524\\
99.36	0.00507261234946913\\
99.37	0.00512946064779225\\
99.38	0.00518683262397297\\
99.39	0.0052447331177869\\
99.4	0.00530316701367457\\
99.41	0.005362139241153\\
99.42	0.00542165477523084\\
99.43	0.00548171863682759\\
99.44	0.00554233589319639\\
99.45	0.00560351165835098\\
99.46	0.00566525109349632\\
99.47	0.00572755940746341\\
99.48	0.00579044185714788\\
99.49	0.00585390374795281\\
99.5	0.0059179504342355\\
99.51	0.00598258731975837\\
99.52	0.00604781985814402\\
99.53	0.00611365355333448\\
99.54	0.00618009396005464\\
99.55	0.00624714668428\\
99.56	0.00631481738370866\\
99.57	0.00638311176823765\\
99.58	0.00645203560044371\\
99.59	0.0065215946960684\\
99.6	0.00659179492450768\\
99.61	0.00666264220930608\\
99.62	0.00673414252531063\\
99.63	0.00680630189702865\\
99.64	0.00687912640447119\\
99.65	0.00695262218366352\\
99.66	0.00702679542716034\\
99.67	0.00710165238456571\\
99.68	0.00717719936305772\\
99.69	0.00725344272791803\\
99.7	0.00733038890306619\\
99.71	0.00740804437159896\\
99.72	0.00748641567633449\\
99.73	0.00756550942036152\\
99.74	0.00764533226759364\\
99.75	0.00772589094332859\\
99.76	0.0078071922348127\\
99.77	0.00788924299181055\\
99.78	0.00797205012717976\\
99.79	0.00805562061745114\\
99.8	0.0081399615034141\\
99.81	0.00822507989070742\\
99.82	0.00831098295041551\\
99.83	0.00839767791967001\\
99.84	0.00848517210225699\\
99.85	0.0085734728692297\\
99.86	0.00866258765952688\\
99.87	0.00875252398059677\\
99.88	0.00884328940902682\\
99.89	0.00893489159117915\\
99.9	0.00902733824383187\\
99.91	0.00912063715482618\\
99.92	0.00921479618371945\\
99.93	0.00930982326244423\\
99.94	0.00940572639597333\\
99.95	0.0095025136629909\\
99.96	0.0096001932165697\\
99.97	0.00969877328485451\\
99.98	0.0097982621717518\\
99.99	0.00989866825762563\\
100	0.01\\
};
\addlegendentry{$q=-3$};

\addplot [color=red,dashed,forget plot]
  table[row sep=crcr]{%
0.01	0\\
0.02	0\\
0.03	0\\
0.04	0\\
0.05	0\\
0.06	0\\
0.07	0\\
0.08	0\\
0.09	0\\
0.1	0\\
0.11	0\\
0.12	0\\
0.13	0\\
0.14	0\\
0.15	0\\
0.16	0\\
0.17	0\\
0.18	0\\
0.19	0\\
0.2	0\\
0.21	0\\
0.22	0\\
0.23	0\\
0.24	0\\
0.25	0\\
0.26	0\\
0.27	0\\
0.28	0\\
0.29	0\\
0.3	0\\
0.31	0\\
0.32	0\\
0.33	0\\
0.34	0\\
0.35	0\\
0.36	0\\
0.37	0\\
0.38	0\\
0.39	0\\
0.4	0\\
0.41	0\\
0.42	0\\
0.43	0\\
0.44	0\\
0.45	0\\
0.46	0\\
0.47	0\\
0.48	0\\
0.49	0\\
0.5	0\\
0.51	0\\
0.52	0\\
0.53	0\\
0.54	0\\
0.55	0\\
0.56	0\\
0.57	0\\
0.58	0\\
0.59	0\\
0.6	0\\
0.61	0\\
0.62	0\\
0.63	0\\
0.64	0\\
0.65	0\\
0.66	0\\
0.67	0\\
0.68	0\\
0.69	0\\
0.7	0\\
0.71	0\\
0.72	0\\
0.73	0\\
0.74	0\\
0.75	0\\
0.76	0\\
0.77	0\\
0.78	0\\
0.79	0\\
0.8	0\\
0.81	0\\
0.82	0\\
0.83	0\\
0.84	0\\
0.85	0\\
0.86	0\\
0.87	0\\
0.88	0\\
0.89	0\\
0.9	0\\
0.91	0\\
0.92	0\\
0.93	0\\
0.94	0\\
0.95	0\\
0.96	0\\
0.97	0\\
0.98	0\\
0.99	0\\
1	0\\
1.01	0\\
1.02	0\\
1.03	0\\
1.04	0\\
1.05	0\\
1.06	0\\
1.07	0\\
1.08	0\\
1.09	0\\
1.1	0\\
1.11	0\\
1.12	0\\
1.13	0\\
1.14	0\\
1.15	0\\
1.16	0\\
1.17	0\\
1.18	0\\
1.19	0\\
1.2	0\\
1.21	0\\
1.22	0\\
1.23	0\\
1.24	0\\
1.25	0\\
1.26	0\\
1.27	0\\
1.28	0\\
1.29	0\\
1.3	0\\
1.31	0\\
1.32	0\\
1.33	0\\
1.34	0\\
1.35	0\\
1.36	0\\
1.37	0\\
1.38	0\\
1.39	0\\
1.4	0\\
1.41	0\\
1.42	0\\
1.43	0\\
1.44	0\\
1.45	0\\
1.46	0\\
1.47	0\\
1.48	0\\
1.49	0\\
1.5	0\\
1.51	0\\
1.52	0\\
1.53	0\\
1.54	0\\
1.55	0\\
1.56	0\\
1.57	0\\
1.58	0\\
1.59	0\\
1.6	0\\
1.61	0\\
1.62	0\\
1.63	0\\
1.64	0\\
1.65	0\\
1.66	0\\
1.67	0\\
1.68	0\\
1.69	0\\
1.7	0\\
1.71	0\\
1.72	0\\
1.73	0\\
1.74	0\\
1.75	0\\
1.76	0\\
1.77	0\\
1.78	0\\
1.79	0\\
1.8	0\\
1.81	0\\
1.82	0\\
1.83	0\\
1.84	0\\
1.85	0\\
1.86	0\\
1.87	0\\
1.88	0\\
1.89	0\\
1.9	0\\
1.91	0\\
1.92	0\\
1.93	0\\
1.94	0\\
1.95	0\\
1.96	0\\
1.97	0\\
1.98	0\\
1.99	0\\
2	0\\
2.01	0\\
2.02	0\\
2.03	0\\
2.04	0\\
2.05	0\\
2.06	0\\
2.07	0\\
2.08	0\\
2.09	0\\
2.1	0\\
2.11	0\\
2.12	0\\
2.13	0\\
2.14	0\\
2.15	0\\
2.16	0\\
2.17	0\\
2.18	0\\
2.19	0\\
2.2	0\\
2.21	0\\
2.22	0\\
2.23	0\\
2.24	0\\
2.25	0\\
2.26	0\\
2.27	0\\
2.28	0\\
2.29	0\\
2.3	0\\
2.31	0\\
2.32	0\\
2.33	0\\
2.34	0\\
2.35	0\\
2.36	0\\
2.37	0\\
2.38	0\\
2.39	0\\
2.4	0\\
2.41	0\\
2.42	0\\
2.43	0\\
2.44	0\\
2.45	0\\
2.46	0\\
2.47	0\\
2.48	0\\
2.49	0\\
2.5	0\\
2.51	0\\
2.52	0\\
2.53	0\\
2.54	0\\
2.55	0\\
2.56	0\\
2.57	0\\
2.58	0\\
2.59	0\\
2.6	0\\
2.61	0\\
2.62	0\\
2.63	0\\
2.64	0\\
2.65	0\\
2.66	0\\
2.67	0\\
2.68	0\\
2.69	0\\
2.7	0\\
2.71	0\\
2.72	0\\
2.73	0\\
2.74	0\\
2.75	0\\
2.76	0\\
2.77	0\\
2.78	0\\
2.79	0\\
2.8	0\\
2.81	0\\
2.82	0\\
2.83	0\\
2.84	0\\
2.85	0\\
2.86	0\\
2.87	0\\
2.88	0\\
2.89	0\\
2.9	0\\
2.91	0\\
2.92	0\\
2.93	0\\
2.94	0\\
2.95	0\\
2.96	0\\
2.97	0\\
2.98	0\\
2.99	0\\
3	0\\
3.01	0\\
3.02	0\\
3.03	0\\
3.04	0\\
3.05	0\\
3.06	0\\
3.07	0\\
3.08	0\\
3.09	0\\
3.1	0\\
3.11	0\\
3.12	0\\
3.13	0\\
3.14	0\\
3.15	0\\
3.16	0\\
3.17	0\\
3.18	0\\
3.19	0\\
3.2	0\\
3.21	0\\
3.22	0\\
3.23	0\\
3.24	0\\
3.25	0\\
3.26	0\\
3.27	0\\
3.28	0\\
3.29	0\\
3.3	0\\
3.31	0\\
3.32	0\\
3.33	0\\
3.34	0\\
3.35	0\\
3.36	0\\
3.37	0\\
3.38	0\\
3.39	0\\
3.4	0\\
3.41	0\\
3.42	0\\
3.43	0\\
3.44	0\\
3.45	0\\
3.46	0\\
3.47	0\\
3.48	0\\
3.49	0\\
3.5	0\\
3.51	0\\
3.52	0\\
3.53	0\\
3.54	0\\
3.55	0\\
3.56	0\\
3.57	0\\
3.58	0\\
3.59	0\\
3.6	0\\
3.61	0\\
3.62	0\\
3.63	0\\
3.64	0\\
3.65	0\\
3.66	0\\
3.67	0\\
3.68	0\\
3.69	0\\
3.7	0\\
3.71	0\\
3.72	0\\
3.73	0\\
3.74	0\\
3.75	0\\
3.76	0\\
3.77	0\\
3.78	0\\
3.79	0\\
3.8	0\\
3.81	0\\
3.82	0\\
3.83	0\\
3.84	0\\
3.85	0\\
3.86	0\\
3.87	0\\
3.88	0\\
3.89	0\\
3.9	0\\
3.91	0\\
3.92	0\\
3.93	0\\
3.94	0\\
3.95	0\\
3.96	0\\
3.97	0\\
3.98	0\\
3.99	0\\
4	0\\
4.01	0\\
4.02	0\\
4.03	0\\
4.04	0\\
4.05	0\\
4.06	0\\
4.07	0\\
4.08	0\\
4.09	0\\
4.1	0\\
4.11	0\\
4.12	0\\
4.13	0\\
4.14	0\\
4.15	0\\
4.16	0\\
4.17	0\\
4.18	0\\
4.19	0\\
4.2	0\\
4.21	0\\
4.22	0\\
4.23	0\\
4.24	0\\
4.25	0\\
4.26	0\\
4.27	0\\
4.28	0\\
4.29	0\\
4.3	0\\
4.31	0\\
4.32	0\\
4.33	0\\
4.34	0\\
4.35	0\\
4.36	0\\
4.37	0\\
4.38	0\\
4.39	0\\
4.4	0\\
4.41	0\\
4.42	0\\
4.43	0\\
4.44	0\\
4.45	0\\
4.46	0\\
4.47	0\\
4.48	0\\
4.49	0\\
4.5	0\\
4.51	0\\
4.52	0\\
4.53	0\\
4.54	0\\
4.55	0\\
4.56	0\\
4.57	0\\
4.58	0\\
4.59	0\\
4.6	0\\
4.61	0\\
4.62	0\\
4.63	0\\
4.64	0\\
4.65	0\\
4.66	0\\
4.67	0\\
4.68	0\\
4.69	0\\
4.7	0\\
4.71	0\\
4.72	0\\
4.73	0\\
4.74	0\\
4.75	0\\
4.76	0\\
4.77	0\\
4.78	0\\
4.79	0\\
4.8	0\\
4.81	0\\
4.82	0\\
4.83	0\\
4.84	0\\
4.85	0\\
4.86	0\\
4.87	0\\
4.88	0\\
4.89	0\\
4.9	0\\
4.91	0\\
4.92	0\\
4.93	0\\
4.94	0\\
4.95	0\\
4.96	0\\
4.97	0\\
4.98	0\\
4.99	0\\
5	0\\
5.01	0\\
5.02	0\\
5.03	0\\
5.04	0\\
5.05	0\\
5.06	0\\
5.07	0\\
5.08	0\\
5.09	0\\
5.1	0\\
5.11	0\\
5.12	0\\
5.13	0\\
5.14	0\\
5.15	0\\
5.16	0\\
5.17	0\\
5.18	0\\
5.19	0\\
5.2	0\\
5.21	0\\
5.22	0\\
5.23	0\\
5.24	0\\
5.25	0\\
5.26	0\\
5.27	0\\
5.28	0\\
5.29	0\\
5.3	0\\
5.31	0\\
5.32	0\\
5.33	0\\
5.34	0\\
5.35	0\\
5.36	0\\
5.37	0\\
5.38	0\\
5.39	0\\
5.4	0\\
5.41	0\\
5.42	0\\
5.43	0\\
5.44	0\\
5.45	0\\
5.46	0\\
5.47	0\\
5.48	0\\
5.49	0\\
5.5	0\\
5.51	0\\
5.52	0\\
5.53	0\\
5.54	0\\
5.55	0\\
5.56	0\\
5.57	0\\
5.58	0\\
5.59	0\\
5.6	0\\
5.61	0\\
5.62	0\\
5.63	0\\
5.64	0\\
5.65	0\\
5.66	0\\
5.67	0\\
5.68	0\\
5.69	0\\
5.7	0\\
5.71	0\\
5.72	0\\
5.73	0\\
5.74	0\\
5.75	0\\
5.76	0\\
5.77	0\\
5.78	0\\
5.79	0\\
5.8	0\\
5.81	0\\
5.82	0\\
5.83	0\\
5.84	0\\
5.85	0\\
5.86	0\\
5.87	0\\
5.88	0\\
5.89	0\\
5.9	0\\
5.91	0\\
5.92	0\\
5.93	0\\
5.94	0\\
5.95	0\\
5.96	0\\
5.97	0\\
5.98	0\\
5.99	0\\
6	0\\
6.01	0\\
6.02	0\\
6.03	0\\
6.04	0\\
6.05	0\\
6.06	0\\
6.07	0\\
6.08	0\\
6.09	0\\
6.1	0\\
6.11	0\\
6.12	0\\
6.13	0\\
6.14	0\\
6.15	0\\
6.16	0\\
6.17	0\\
6.18	0\\
6.19	0\\
6.2	0\\
6.21	0\\
6.22	0\\
6.23	0\\
6.24	0\\
6.25	0\\
6.26	0\\
6.27	0\\
6.28	0\\
6.29	0\\
6.3	0\\
6.31	0\\
6.32	0\\
6.33	0\\
6.34	0\\
6.35	0\\
6.36	0\\
6.37	0\\
6.38	0\\
6.39	0\\
6.4	0\\
6.41	0\\
6.42	0\\
6.43	0\\
6.44	0\\
6.45	0\\
6.46	0\\
6.47	0\\
6.48	0\\
6.49	0\\
6.5	0\\
6.51	0\\
6.52	0\\
6.53	0\\
6.54	0\\
6.55	0\\
6.56	0\\
6.57	0\\
6.58	0\\
6.59	0\\
6.6	0\\
6.61	0\\
6.62	0\\
6.63	0\\
6.64	0\\
6.65	0\\
6.66	0\\
6.67	0\\
6.68	0\\
6.69	0\\
6.7	0\\
6.71	0\\
6.72	0\\
6.73	0\\
6.74	0\\
6.75	0\\
6.76	0\\
6.77	0\\
6.78	0\\
6.79	0\\
6.8	0\\
6.81	0\\
6.82	0\\
6.83	0\\
6.84	0\\
6.85	0\\
6.86	0\\
6.87	0\\
6.88	0\\
6.89	0\\
6.9	0\\
6.91	0\\
6.92	0\\
6.93	0\\
6.94	0\\
6.95	0\\
6.96	0\\
6.97	0\\
6.98	0\\
6.99	0\\
7	0\\
7.01	0\\
7.02	0\\
7.03	0\\
7.04	0\\
7.05	0\\
7.06	0\\
7.07	0\\
7.08	0\\
7.09	0\\
7.1	0\\
7.11	0\\
7.12	0\\
7.13	0\\
7.14	0\\
7.15	0\\
7.16	0\\
7.17	0\\
7.18	0\\
7.19	0\\
7.2	0\\
7.21	0\\
7.22	0\\
7.23	0\\
7.24	0\\
7.25	0\\
7.26	0\\
7.27	0\\
7.28	0\\
7.29	0\\
7.3	0\\
7.31	0\\
7.32	0\\
7.33	0\\
7.34	0\\
7.35	0\\
7.36	0\\
7.37	0\\
7.38	0\\
7.39	0\\
7.4	0\\
7.41	0\\
7.42	0\\
7.43	0\\
7.44	0\\
7.45	0\\
7.46	0\\
7.47	0\\
7.48	0\\
7.49	0\\
7.5	0\\
7.51	0\\
7.52	0\\
7.53	0\\
7.54	0\\
7.55	0\\
7.56	0\\
7.57	0\\
7.58	0\\
7.59	0\\
7.6	0\\
7.61	0\\
7.62	0\\
7.63	0\\
7.64	0\\
7.65	0\\
7.66	0\\
7.67	0\\
7.68	0\\
7.69	0\\
7.7	0\\
7.71	0\\
7.72	0\\
7.73	0\\
7.74	0\\
7.75	0\\
7.76	0\\
7.77	0\\
7.78	0\\
7.79	0\\
7.8	0\\
7.81	0\\
7.82	0\\
7.83	0\\
7.84	0\\
7.85	0\\
7.86	0\\
7.87	0\\
7.88	0\\
7.89	0\\
7.9	0\\
7.91	0\\
7.92	0\\
7.93	0\\
7.94	0\\
7.95	0\\
7.96	0\\
7.97	0\\
7.98	0\\
7.99	0\\
8	0\\
8.01	0\\
8.02	0\\
8.03	0\\
8.04	0\\
8.05	0\\
8.06	0\\
8.07	0\\
8.08	0\\
8.09	0\\
8.1	0\\
8.11	0\\
8.12	0\\
8.13	0\\
8.14	0\\
8.15	0\\
8.16	0\\
8.17	0\\
8.18	0\\
8.19	0\\
8.2	0\\
8.21	0\\
8.22	0\\
8.23	0\\
8.24	0\\
8.25	0\\
8.26	0\\
8.27	0\\
8.28	0\\
8.29	0\\
8.3	0\\
8.31	0\\
8.32	0\\
8.33	0\\
8.34	0\\
8.35	0\\
8.36	0\\
8.37	0\\
8.38	0\\
8.39	0\\
8.4	0\\
8.41	0\\
8.42	0\\
8.43	0\\
8.44	0\\
8.45	0\\
8.46	0\\
8.47	0\\
8.48	0\\
8.49	0\\
8.5	0\\
8.51	0\\
8.52	0\\
8.53	0\\
8.54	0\\
8.55	0\\
8.56	0\\
8.57	0\\
8.58	0\\
8.59	0\\
8.6	0\\
8.61	0\\
8.62	0\\
8.63	0\\
8.64	0\\
8.65	0\\
8.66	0\\
8.67	0\\
8.68	0\\
8.69	0\\
8.7	0\\
8.71	0\\
8.72	0\\
8.73	0\\
8.74	0\\
8.75	0\\
8.76	0\\
8.77	0\\
8.78	0\\
8.79	0\\
8.8	0\\
8.81	0\\
8.82	0\\
8.83	0\\
8.84	0\\
8.85	0\\
8.86	0\\
8.87	0\\
8.88	0\\
8.89	0\\
8.9	0\\
8.91	0\\
8.92	0\\
8.93	0\\
8.94	0\\
8.95	0\\
8.96	0\\
8.97	0\\
8.98	0\\
8.99	0\\
9	0\\
9.01	0\\
9.02	0\\
9.03	0\\
9.04	0\\
9.05	0\\
9.06	0\\
9.07	0\\
9.08	0\\
9.09	0\\
9.1	0\\
9.11	0\\
9.12	0\\
9.13	0\\
9.14	0\\
9.15	0\\
9.16	0\\
9.17	0\\
9.18	0\\
9.19	0\\
9.2	0\\
9.21	0\\
9.22	0\\
9.23	0\\
9.24	0\\
9.25	0\\
9.26	0\\
9.27	0\\
9.28	0\\
9.29	0\\
9.3	0\\
9.31	0\\
9.32	0\\
9.33	0\\
9.34	0\\
9.35	0\\
9.36	0\\
9.37	0\\
9.38	0\\
9.39	0\\
9.4	0\\
9.41	0\\
9.42	0\\
9.43	0\\
9.44	0\\
9.45	0\\
9.46	0\\
9.47	0\\
9.48	0\\
9.49	0\\
9.5	0\\
9.51	0\\
9.52	0\\
9.53	0\\
9.54	0\\
9.55	0\\
9.56	0\\
9.57	0\\
9.58	0\\
9.59	0\\
9.6	0\\
9.61	0\\
9.62	0\\
9.63	0\\
9.64	0\\
9.65	0\\
9.66	0\\
9.67	0\\
9.68	0\\
9.69	0\\
9.7	0\\
9.71	0\\
9.72	0\\
9.73	0\\
9.74	0\\
9.75	0\\
9.76	0\\
9.77	0\\
9.78	0\\
9.79	0\\
9.8	0\\
9.81	0\\
9.82	0\\
9.83	0\\
9.84	0\\
9.85	0\\
9.86	0\\
9.87	0\\
9.88	0\\
9.89	0\\
9.9	0\\
9.91	0\\
9.92	0\\
9.93	0\\
9.94	0\\
9.95	0\\
9.96	0\\
9.97	0\\
9.98	0\\
9.99	0\\
10	0\\
10.01	0\\
10.02	0\\
10.03	0\\
10.04	0\\
10.05	0\\
10.06	0\\
10.07	0\\
10.08	0\\
10.09	0\\
10.1	0\\
10.11	0\\
10.12	0\\
10.13	0\\
10.14	0\\
10.15	0\\
10.16	0\\
10.17	0\\
10.18	0\\
10.19	0\\
10.2	0\\
10.21	0\\
10.22	0\\
10.23	0\\
10.24	0\\
10.25	0\\
10.26	0\\
10.27	0\\
10.28	0\\
10.29	0\\
10.3	0\\
10.31	0\\
10.32	0\\
10.33	0\\
10.34	0\\
10.35	0\\
10.36	0\\
10.37	0\\
10.38	0\\
10.39	0\\
10.4	0\\
10.41	0\\
10.42	0\\
10.43	0\\
10.44	0\\
10.45	0\\
10.46	0\\
10.47	0\\
10.48	0\\
10.49	0\\
10.5	0\\
10.51	0\\
10.52	0\\
10.53	0\\
10.54	0\\
10.55	0\\
10.56	0\\
10.57	0\\
10.58	0\\
10.59	0\\
10.6	0\\
10.61	0\\
10.62	0\\
10.63	0\\
10.64	0\\
10.65	0\\
10.66	0\\
10.67	0\\
10.68	0\\
10.69	0\\
10.7	0\\
10.71	0\\
10.72	0\\
10.73	0\\
10.74	0\\
10.75	0\\
10.76	0\\
10.77	0\\
10.78	0\\
10.79	0\\
10.8	0\\
10.81	0\\
10.82	0\\
10.83	0\\
10.84	0\\
10.85	0\\
10.86	0\\
10.87	0\\
10.88	0\\
10.89	0\\
10.9	0\\
10.91	0\\
10.92	0\\
10.93	0\\
10.94	0\\
10.95	0\\
10.96	0\\
10.97	0\\
10.98	0\\
10.99	0\\
11	0\\
11.01	0\\
11.02	0\\
11.03	0\\
11.04	0\\
11.05	0\\
11.06	0\\
11.07	0\\
11.08	0\\
11.09	0\\
11.1	0\\
11.11	0\\
11.12	0\\
11.13	0\\
11.14	0\\
11.15	0\\
11.16	0\\
11.17	0\\
11.18	0\\
11.19	0\\
11.2	0\\
11.21	0\\
11.22	0\\
11.23	0\\
11.24	0\\
11.25	0\\
11.26	0\\
11.27	0\\
11.28	0\\
11.29	0\\
11.3	0\\
11.31	0\\
11.32	0\\
11.33	0\\
11.34	0\\
11.35	0\\
11.36	0\\
11.37	0\\
11.38	0\\
11.39	0\\
11.4	0\\
11.41	0\\
11.42	0\\
11.43	0\\
11.44	0\\
11.45	0\\
11.46	0\\
11.47	0\\
11.48	0\\
11.49	0\\
11.5	0\\
11.51	0\\
11.52	0\\
11.53	0\\
11.54	0\\
11.55	0\\
11.56	0\\
11.57	0\\
11.58	0\\
11.59	0\\
11.6	0\\
11.61	0\\
11.62	0\\
11.63	0\\
11.64	0\\
11.65	0\\
11.66	0\\
11.67	0\\
11.68	0\\
11.69	0\\
11.7	0\\
11.71	0\\
11.72	0\\
11.73	0\\
11.74	0\\
11.75	0\\
11.76	0\\
11.77	0\\
11.78	0\\
11.79	0\\
11.8	0\\
11.81	0\\
11.82	0\\
11.83	0\\
11.84	0\\
11.85	0\\
11.86	0\\
11.87	0\\
11.88	0\\
11.89	0\\
11.9	0\\
11.91	0\\
11.92	0\\
11.93	0\\
11.94	0\\
11.95	0\\
11.96	0\\
11.97	0\\
11.98	0\\
11.99	0\\
12	0\\
12.01	0\\
12.02	0\\
12.03	0\\
12.04	0\\
12.05	0\\
12.06	0\\
12.07	0\\
12.08	0\\
12.09	0\\
12.1	0\\
12.11	0\\
12.12	0\\
12.13	0\\
12.14	0\\
12.15	0\\
12.16	0\\
12.17	0\\
12.18	0\\
12.19	0\\
12.2	0\\
12.21	0\\
12.22	0\\
12.23	0\\
12.24	0\\
12.25	0\\
12.26	0\\
12.27	0\\
12.28	0\\
12.29	0\\
12.3	0\\
12.31	0\\
12.32	0\\
12.33	0\\
12.34	0\\
12.35	0\\
12.36	0\\
12.37	0\\
12.38	0\\
12.39	0\\
12.4	0\\
12.41	0\\
12.42	0\\
12.43	0\\
12.44	0\\
12.45	0\\
12.46	0\\
12.47	0\\
12.48	0\\
12.49	0\\
12.5	0\\
12.51	0\\
12.52	0\\
12.53	0\\
12.54	0\\
12.55	0\\
12.56	0\\
12.57	0\\
12.58	0\\
12.59	0\\
12.6	0\\
12.61	0\\
12.62	0\\
12.63	0\\
12.64	0\\
12.65	0\\
12.66	0\\
12.67	0\\
12.68	0\\
12.69	0\\
12.7	0\\
12.71	0\\
12.72	0\\
12.73	0\\
12.74	0\\
12.75	0\\
12.76	0\\
12.77	0\\
12.78	0\\
12.79	0\\
12.8	0\\
12.81	0\\
12.82	0\\
12.83	0\\
12.84	0\\
12.85	0\\
12.86	0\\
12.87	0\\
12.88	0\\
12.89	0\\
12.9	0\\
12.91	0\\
12.92	0\\
12.93	0\\
12.94	0\\
12.95	0\\
12.96	0\\
12.97	0\\
12.98	0\\
12.99	0\\
13	0\\
13.01	0\\
13.02	0\\
13.03	0\\
13.04	0\\
13.05	0\\
13.06	0\\
13.07	0\\
13.08	0\\
13.09	0\\
13.1	0\\
13.11	0\\
13.12	0\\
13.13	0\\
13.14	0\\
13.15	0\\
13.16	0\\
13.17	0\\
13.18	0\\
13.19	0\\
13.2	0\\
13.21	0\\
13.22	0\\
13.23	0\\
13.24	0\\
13.25	0\\
13.26	0\\
13.27	0\\
13.28	0\\
13.29	0\\
13.3	0\\
13.31	0\\
13.32	0\\
13.33	0\\
13.34	0\\
13.35	0\\
13.36	0\\
13.37	0\\
13.38	0\\
13.39	0\\
13.4	0\\
13.41	0\\
13.42	0\\
13.43	0\\
13.44	0\\
13.45	0\\
13.46	0\\
13.47	0\\
13.48	0\\
13.49	0\\
13.5	0\\
13.51	0\\
13.52	0\\
13.53	0\\
13.54	0\\
13.55	0\\
13.56	0\\
13.57	0\\
13.58	0\\
13.59	0\\
13.6	0\\
13.61	0\\
13.62	0\\
13.63	0\\
13.64	0\\
13.65	0\\
13.66	0\\
13.67	0\\
13.68	0\\
13.69	0\\
13.7	0\\
13.71	0\\
13.72	0\\
13.73	0\\
13.74	0\\
13.75	0\\
13.76	0\\
13.77	0\\
13.78	0\\
13.79	0\\
13.8	0\\
13.81	0\\
13.82	0\\
13.83	0\\
13.84	0\\
13.85	0\\
13.86	0\\
13.87	0\\
13.88	0\\
13.89	0\\
13.9	0\\
13.91	0\\
13.92	0\\
13.93	0\\
13.94	0\\
13.95	0\\
13.96	0\\
13.97	0\\
13.98	0\\
13.99	0\\
14	0\\
14.01	0\\
14.02	0\\
14.03	0\\
14.04	0\\
14.05	0\\
14.06	0\\
14.07	0\\
14.08	0\\
14.09	0\\
14.1	0\\
14.11	0\\
14.12	0\\
14.13	0\\
14.14	0\\
14.15	0\\
14.16	0\\
14.17	0\\
14.18	0\\
14.19	0\\
14.2	0\\
14.21	0\\
14.22	0\\
14.23	0\\
14.24	0\\
14.25	0\\
14.26	0\\
14.27	0\\
14.28	0\\
14.29	0\\
14.3	0\\
14.31	0\\
14.32	0\\
14.33	0\\
14.34	0\\
14.35	0\\
14.36	0\\
14.37	0\\
14.38	0\\
14.39	0\\
14.4	0\\
14.41	0\\
14.42	0\\
14.43	0\\
14.44	0\\
14.45	0\\
14.46	0\\
14.47	0\\
14.48	0\\
14.49	0\\
14.5	0\\
14.51	0\\
14.52	0\\
14.53	0\\
14.54	0\\
14.55	0\\
14.56	0\\
14.57	0\\
14.58	0\\
14.59	0\\
14.6	0\\
14.61	0\\
14.62	0\\
14.63	0\\
14.64	0\\
14.65	0\\
14.66	0\\
14.67	0\\
14.68	0\\
14.69	0\\
14.7	0\\
14.71	0\\
14.72	0\\
14.73	0\\
14.74	0\\
14.75	0\\
14.76	0\\
14.77	0\\
14.78	0\\
14.79	0\\
14.8	0\\
14.81	0\\
14.82	0\\
14.83	0\\
14.84	0\\
14.85	0\\
14.86	0\\
14.87	0\\
14.88	0\\
14.89	0\\
14.9	0\\
14.91	0\\
14.92	0\\
14.93	0\\
14.94	0\\
14.95	0\\
14.96	0\\
14.97	0\\
14.98	0\\
14.99	0\\
15	0\\
15.01	0\\
15.02	0\\
15.03	0\\
15.04	0\\
15.05	0\\
15.06	0\\
15.07	0\\
15.08	0\\
15.09	0\\
15.1	0\\
15.11	0\\
15.12	0\\
15.13	0\\
15.14	0\\
15.15	0\\
15.16	0\\
15.17	0\\
15.18	0\\
15.19	0\\
15.2	0\\
15.21	0\\
15.22	0\\
15.23	0\\
15.24	0\\
15.25	0\\
15.26	0\\
15.27	0\\
15.28	0\\
15.29	0\\
15.3	0\\
15.31	0\\
15.32	0\\
15.33	0\\
15.34	0\\
15.35	0\\
15.36	0\\
15.37	0\\
15.38	0\\
15.39	0\\
15.4	0\\
15.41	0\\
15.42	0\\
15.43	0\\
15.44	0\\
15.45	0\\
15.46	0\\
15.47	0\\
15.48	0\\
15.49	0\\
15.5	0\\
15.51	0\\
15.52	0\\
15.53	0\\
15.54	0\\
15.55	0\\
15.56	0\\
15.57	0\\
15.58	0\\
15.59	0\\
15.6	0\\
15.61	0\\
15.62	0\\
15.63	0\\
15.64	0\\
15.65	0\\
15.66	0\\
15.67	0\\
15.68	0\\
15.69	0\\
15.7	0\\
15.71	0\\
15.72	0\\
15.73	0\\
15.74	0\\
15.75	0\\
15.76	0\\
15.77	0\\
15.78	0\\
15.79	0\\
15.8	0\\
15.81	0\\
15.82	0\\
15.83	0\\
15.84	0\\
15.85	0\\
15.86	0\\
15.87	0\\
15.88	0\\
15.89	0\\
15.9	0\\
15.91	0\\
15.92	0\\
15.93	0\\
15.94	0\\
15.95	0\\
15.96	0\\
15.97	0\\
15.98	0\\
15.99	0\\
16	0\\
16.01	0\\
16.02	0\\
16.03	0\\
16.04	0\\
16.05	0\\
16.06	0\\
16.07	0\\
16.08	0\\
16.09	0\\
16.1	0\\
16.11	0\\
16.12	0\\
16.13	0\\
16.14	0\\
16.15	0\\
16.16	0\\
16.17	0\\
16.18	0\\
16.19	0\\
16.2	0\\
16.21	0\\
16.22	0\\
16.23	0\\
16.24	0\\
16.25	0\\
16.26	0\\
16.27	0\\
16.28	0\\
16.29	0\\
16.3	0\\
16.31	0\\
16.32	0\\
16.33	0\\
16.34	0\\
16.35	0\\
16.36	0\\
16.37	0\\
16.38	0\\
16.39	0\\
16.4	0\\
16.41	0\\
16.42	0\\
16.43	0\\
16.44	0\\
16.45	0\\
16.46	0\\
16.47	0\\
16.48	0\\
16.49	0\\
16.5	0\\
16.51	0\\
16.52	0\\
16.53	0\\
16.54	0\\
16.55	0\\
16.56	0\\
16.57	0\\
16.58	0\\
16.59	0\\
16.6	0\\
16.61	0\\
16.62	0\\
16.63	0\\
16.64	0\\
16.65	0\\
16.66	0\\
16.67	0\\
16.68	0\\
16.69	0\\
16.7	0\\
16.71	0\\
16.72	0\\
16.73	0\\
16.74	0\\
16.75	0\\
16.76	0\\
16.77	0\\
16.78	0\\
16.79	0\\
16.8	0\\
16.81	0\\
16.82	0\\
16.83	0\\
16.84	0\\
16.85	0\\
16.86	0\\
16.87	0\\
16.88	0\\
16.89	0\\
16.9	0\\
16.91	0\\
16.92	0\\
16.93	0\\
16.94	0\\
16.95	0\\
16.96	0\\
16.97	0\\
16.98	0\\
16.99	0\\
17	0\\
17.01	0\\
17.02	0\\
17.03	0\\
17.04	0\\
17.05	0\\
17.06	0\\
17.07	0\\
17.08	0\\
17.09	0\\
17.1	0\\
17.11	0\\
17.12	0\\
17.13	0\\
17.14	0\\
17.15	0\\
17.16	0\\
17.17	0\\
17.18	0\\
17.19	0\\
17.2	0\\
17.21	0\\
17.22	0\\
17.23	0\\
17.24	0\\
17.25	0\\
17.26	0\\
17.27	0\\
17.28	0\\
17.29	0\\
17.3	0\\
17.31	0\\
17.32	0\\
17.33	0\\
17.34	0\\
17.35	0\\
17.36	0\\
17.37	0\\
17.38	0\\
17.39	0\\
17.4	0\\
17.41	0\\
17.42	0\\
17.43	0\\
17.44	0\\
17.45	0\\
17.46	0\\
17.47	0\\
17.48	0\\
17.49	0\\
17.5	0\\
17.51	0\\
17.52	0\\
17.53	0\\
17.54	0\\
17.55	0\\
17.56	0\\
17.57	0\\
17.58	0\\
17.59	0\\
17.6	0\\
17.61	0\\
17.62	0\\
17.63	0\\
17.64	0\\
17.65	0\\
17.66	0\\
17.67	0\\
17.68	0\\
17.69	0\\
17.7	0\\
17.71	0\\
17.72	0\\
17.73	0\\
17.74	0\\
17.75	0\\
17.76	0\\
17.77	0\\
17.78	0\\
17.79	0\\
17.8	0\\
17.81	0\\
17.82	0\\
17.83	0\\
17.84	0\\
17.85	0\\
17.86	0\\
17.87	0\\
17.88	0\\
17.89	0\\
17.9	0\\
17.91	0\\
17.92	0\\
17.93	0\\
17.94	0\\
17.95	0\\
17.96	0\\
17.97	0\\
17.98	0\\
17.99	0\\
18	0\\
18.01	0\\
18.02	0\\
18.03	0\\
18.04	0\\
18.05	0\\
18.06	0\\
18.07	0\\
18.08	0\\
18.09	0\\
18.1	0\\
18.11	0\\
18.12	0\\
18.13	0\\
18.14	0\\
18.15	0\\
18.16	0\\
18.17	0\\
18.18	0\\
18.19	0\\
18.2	0\\
18.21	0\\
18.22	0\\
18.23	0\\
18.24	0\\
18.25	0\\
18.26	0\\
18.27	0\\
18.28	0\\
18.29	0\\
18.3	0\\
18.31	0\\
18.32	0\\
18.33	0\\
18.34	0\\
18.35	0\\
18.36	0\\
18.37	0\\
18.38	0\\
18.39	0\\
18.4	0\\
18.41	0\\
18.42	0\\
18.43	0\\
18.44	0\\
18.45	0\\
18.46	0\\
18.47	0\\
18.48	0\\
18.49	0\\
18.5	0\\
18.51	0\\
18.52	0\\
18.53	0\\
18.54	0\\
18.55	0\\
18.56	0\\
18.57	0\\
18.58	0\\
18.59	0\\
18.6	0\\
18.61	0\\
18.62	0\\
18.63	0\\
18.64	0\\
18.65	0\\
18.66	0\\
18.67	0\\
18.68	0\\
18.69	0\\
18.7	0\\
18.71	0\\
18.72	0\\
18.73	0\\
18.74	0\\
18.75	0\\
18.76	0\\
18.77	0\\
18.78	0\\
18.79	0\\
18.8	0\\
18.81	0\\
18.82	0\\
18.83	0\\
18.84	0\\
18.85	0\\
18.86	0\\
18.87	0\\
18.88	0\\
18.89	0\\
18.9	0\\
18.91	0\\
18.92	0\\
18.93	0\\
18.94	0\\
18.95	0\\
18.96	0\\
18.97	0\\
18.98	0\\
18.99	0\\
19	0\\
19.01	0\\
19.02	0\\
19.03	0\\
19.04	0\\
19.05	0\\
19.06	0\\
19.07	0\\
19.08	0\\
19.09	0\\
19.1	0\\
19.11	0\\
19.12	0\\
19.13	0\\
19.14	0\\
19.15	0\\
19.16	0\\
19.17	0\\
19.18	0\\
19.19	0\\
19.2	0\\
19.21	0\\
19.22	0\\
19.23	0\\
19.24	0\\
19.25	0\\
19.26	0\\
19.27	0\\
19.28	0\\
19.29	0\\
19.3	0\\
19.31	0\\
19.32	0\\
19.33	0\\
19.34	0\\
19.35	0\\
19.36	0\\
19.37	0\\
19.38	0\\
19.39	0\\
19.4	0\\
19.41	0\\
19.42	0\\
19.43	0\\
19.44	0\\
19.45	0\\
19.46	0\\
19.47	0\\
19.48	0\\
19.49	0\\
19.5	0\\
19.51	0\\
19.52	0\\
19.53	0\\
19.54	0\\
19.55	0\\
19.56	0\\
19.57	0\\
19.58	0\\
19.59	0\\
19.6	0\\
19.61	0\\
19.62	0\\
19.63	0\\
19.64	0\\
19.65	0\\
19.66	0\\
19.67	0\\
19.68	0\\
19.69	0\\
19.7	0\\
19.71	0\\
19.72	0\\
19.73	0\\
19.74	0\\
19.75	0\\
19.76	0\\
19.77	0\\
19.78	0\\
19.79	0\\
19.8	0\\
19.81	0\\
19.82	0\\
19.83	0\\
19.84	0\\
19.85	0\\
19.86	0\\
19.87	0\\
19.88	0\\
19.89	0\\
19.9	0\\
19.91	0\\
19.92	0\\
19.93	0\\
19.94	0\\
19.95	0\\
19.96	0\\
19.97	0\\
19.98	0\\
19.99	0\\
20	0\\
20.01	0\\
20.02	0\\
20.03	0\\
20.04	0\\
20.05	0\\
20.06	0\\
20.07	0\\
20.08	0\\
20.09	0\\
20.1	0\\
20.11	0\\
20.12	0\\
20.13	0\\
20.14	0\\
20.15	0\\
20.16	0\\
20.17	0\\
20.18	0\\
20.19	0\\
20.2	0\\
20.21	0\\
20.22	0\\
20.23	0\\
20.24	0\\
20.25	0\\
20.26	0\\
20.27	0\\
20.28	0\\
20.29	0\\
20.3	0\\
20.31	0\\
20.32	0\\
20.33	0\\
20.34	0\\
20.35	0\\
20.36	0\\
20.37	0\\
20.38	0\\
20.39	0\\
20.4	0\\
20.41	0\\
20.42	0\\
20.43	0\\
20.44	0\\
20.45	0\\
20.46	0\\
20.47	0\\
20.48	0\\
20.49	0\\
20.5	0\\
20.51	0\\
20.52	0\\
20.53	0\\
20.54	0\\
20.55	0\\
20.56	0\\
20.57	0\\
20.58	0\\
20.59	0\\
20.6	0\\
20.61	0\\
20.62	0\\
20.63	0\\
20.64	0\\
20.65	0\\
20.66	0\\
20.67	0\\
20.68	0\\
20.69	0\\
20.7	0\\
20.71	0\\
20.72	0\\
20.73	0\\
20.74	0\\
20.75	0\\
20.76	0\\
20.77	0\\
20.78	0\\
20.79	0\\
20.8	0\\
20.81	0\\
20.82	0\\
20.83	0\\
20.84	0\\
20.85	0\\
20.86	0\\
20.87	0\\
20.88	0\\
20.89	0\\
20.9	0\\
20.91	0\\
20.92	0\\
20.93	0\\
20.94	0\\
20.95	0\\
20.96	0\\
20.97	0\\
20.98	0\\
20.99	0\\
21	0\\
21.01	0\\
21.02	0\\
21.03	0\\
21.04	0\\
21.05	0\\
21.06	0\\
21.07	0\\
21.08	0\\
21.09	0\\
21.1	0\\
21.11	0\\
21.12	0\\
21.13	0\\
21.14	0\\
21.15	0\\
21.16	0\\
21.17	0\\
21.18	0\\
21.19	0\\
21.2	0\\
21.21	0\\
21.22	0\\
21.23	0\\
21.24	0\\
21.25	0\\
21.26	0\\
21.27	0\\
21.28	0\\
21.29	0\\
21.3	0\\
21.31	0\\
21.32	0\\
21.33	0\\
21.34	0\\
21.35	0\\
21.36	0\\
21.37	0\\
21.38	0\\
21.39	0\\
21.4	0\\
21.41	0\\
21.42	0\\
21.43	0\\
21.44	0\\
21.45	0\\
21.46	0\\
21.47	0\\
21.48	0\\
21.49	0\\
21.5	0\\
21.51	0\\
21.52	0\\
21.53	0\\
21.54	0\\
21.55	0\\
21.56	0\\
21.57	0\\
21.58	0\\
21.59	0\\
21.6	0\\
21.61	0\\
21.62	0\\
21.63	0\\
21.64	0\\
21.65	0\\
21.66	0\\
21.67	0\\
21.68	0\\
21.69	0\\
21.7	0\\
21.71	0\\
21.72	0\\
21.73	0\\
21.74	0\\
21.75	0\\
21.76	0\\
21.77	0\\
21.78	0\\
21.79	0\\
21.8	0\\
21.81	0\\
21.82	0\\
21.83	0\\
21.84	0\\
21.85	0\\
21.86	0\\
21.87	0\\
21.88	0\\
21.89	0\\
21.9	0\\
21.91	0\\
21.92	0\\
21.93	0\\
21.94	0\\
21.95	0\\
21.96	0\\
21.97	0\\
21.98	0\\
21.99	0\\
22	0\\
22.01	0\\
22.02	0\\
22.03	0\\
22.04	0\\
22.05	0\\
22.06	0\\
22.07	0\\
22.08	0\\
22.09	0\\
22.1	0\\
22.11	0\\
22.12	0\\
22.13	0\\
22.14	0\\
22.15	0\\
22.16	0\\
22.17	0\\
22.18	0\\
22.19	0\\
22.2	0\\
22.21	0\\
22.22	0\\
22.23	0\\
22.24	0\\
22.25	0\\
22.26	0\\
22.27	0\\
22.28	0\\
22.29	0\\
22.3	0\\
22.31	0\\
22.32	0\\
22.33	0\\
22.34	0\\
22.35	0\\
22.36	0\\
22.37	0\\
22.38	0\\
22.39	0\\
22.4	0\\
22.41	0\\
22.42	0\\
22.43	0\\
22.44	0\\
22.45	0\\
22.46	0\\
22.47	0\\
22.48	0\\
22.49	0\\
22.5	0\\
22.51	0\\
22.52	0\\
22.53	0\\
22.54	0\\
22.55	0\\
22.56	0\\
22.57	0\\
22.58	0\\
22.59	0\\
22.6	0\\
22.61	0\\
22.62	0\\
22.63	0\\
22.64	0\\
22.65	0\\
22.66	0\\
22.67	0\\
22.68	0\\
22.69	0\\
22.7	0\\
22.71	0\\
22.72	0\\
22.73	0\\
22.74	0\\
22.75	0\\
22.76	0\\
22.77	0\\
22.78	0\\
22.79	0\\
22.8	0\\
22.81	0\\
22.82	0\\
22.83	0\\
22.84	0\\
22.85	0\\
22.86	0\\
22.87	0\\
22.88	0\\
22.89	0\\
22.9	0\\
22.91	0\\
22.92	0\\
22.93	0\\
22.94	0\\
22.95	0\\
22.96	0\\
22.97	0\\
22.98	0\\
22.99	0\\
23	0\\
23.01	0\\
23.02	0\\
23.03	0\\
23.04	0\\
23.05	0\\
23.06	0\\
23.07	0\\
23.08	0\\
23.09	0\\
23.1	0\\
23.11	0\\
23.12	0\\
23.13	0\\
23.14	0\\
23.15	0\\
23.16	0\\
23.17	0\\
23.18	0\\
23.19	0\\
23.2	0\\
23.21	0\\
23.22	0\\
23.23	0\\
23.24	0\\
23.25	0\\
23.26	0\\
23.27	0\\
23.28	0\\
23.29	0\\
23.3	0\\
23.31	0\\
23.32	0\\
23.33	0\\
23.34	0\\
23.35	0\\
23.36	0\\
23.37	0\\
23.38	0\\
23.39	0\\
23.4	0\\
23.41	0\\
23.42	0\\
23.43	0\\
23.44	0\\
23.45	0\\
23.46	0\\
23.47	0\\
23.48	0\\
23.49	0\\
23.5	0\\
23.51	0\\
23.52	0\\
23.53	0\\
23.54	0\\
23.55	0\\
23.56	0\\
23.57	0\\
23.58	0\\
23.59	0\\
23.6	0\\
23.61	0\\
23.62	0\\
23.63	0\\
23.64	0\\
23.65	0\\
23.66	0\\
23.67	0\\
23.68	0\\
23.69	0\\
23.7	0\\
23.71	0\\
23.72	0\\
23.73	0\\
23.74	0\\
23.75	0\\
23.76	0\\
23.77	0\\
23.78	0\\
23.79	0\\
23.8	0\\
23.81	0\\
23.82	0\\
23.83	0\\
23.84	0\\
23.85	0\\
23.86	0\\
23.87	0\\
23.88	0\\
23.89	0\\
23.9	0\\
23.91	0\\
23.92	0\\
23.93	0\\
23.94	0\\
23.95	0\\
23.96	0\\
23.97	0\\
23.98	0\\
23.99	0\\
24	0\\
24.01	0\\
24.02	0\\
24.03	0\\
24.04	0\\
24.05	0\\
24.06	0\\
24.07	0\\
24.08	0\\
24.09	0\\
24.1	0\\
24.11	0\\
24.12	0\\
24.13	0\\
24.14	0\\
24.15	0\\
24.16	0\\
24.17	0\\
24.18	0\\
24.19	0\\
24.2	0\\
24.21	0\\
24.22	0\\
24.23	0\\
24.24	0\\
24.25	0\\
24.26	0\\
24.27	0\\
24.28	0\\
24.29	0\\
24.3	0\\
24.31	0\\
24.32	0\\
24.33	0\\
24.34	0\\
24.35	0\\
24.36	0\\
24.37	0\\
24.38	0\\
24.39	0\\
24.4	0\\
24.41	0\\
24.42	0\\
24.43	0\\
24.44	0\\
24.45	0\\
24.46	0\\
24.47	0\\
24.48	0\\
24.49	0\\
24.5	0\\
24.51	0\\
24.52	0\\
24.53	0\\
24.54	0\\
24.55	0\\
24.56	0\\
24.57	0\\
24.58	0\\
24.59	0\\
24.6	0\\
24.61	0\\
24.62	0\\
24.63	0\\
24.64	0\\
24.65	0\\
24.66	0\\
24.67	0\\
24.68	0\\
24.69	0\\
24.7	0\\
24.71	0\\
24.72	0\\
24.73	0\\
24.74	0\\
24.75	0\\
24.76	0\\
24.77	0\\
24.78	0\\
24.79	0\\
24.8	0\\
24.81	0\\
24.82	0\\
24.83	0\\
24.84	0\\
24.85	0\\
24.86	0\\
24.87	0\\
24.88	0\\
24.89	0\\
24.9	0\\
24.91	0\\
24.92	0\\
24.93	0\\
24.94	0\\
24.95	0\\
24.96	0\\
24.97	0\\
24.98	0\\
24.99	0\\
25	0\\
25.01	0\\
25.02	0\\
25.03	0\\
25.04	0\\
25.05	0\\
25.06	0\\
25.07	0\\
25.08	0\\
25.09	0\\
25.1	0\\
25.11	0\\
25.12	0\\
25.13	0\\
25.14	0\\
25.15	0\\
25.16	0\\
25.17	0\\
25.18	0\\
25.19	0\\
25.2	0\\
25.21	0\\
25.22	0\\
25.23	0\\
25.24	0\\
25.25	0\\
25.26	0\\
25.27	0\\
25.28	0\\
25.29	0\\
25.3	0\\
25.31	0\\
25.32	0\\
25.33	0\\
25.34	0\\
25.35	0\\
25.36	0\\
25.37	0\\
25.38	0\\
25.39	0\\
25.4	0\\
25.41	0\\
25.42	0\\
25.43	0\\
25.44	0\\
25.45	0\\
25.46	0\\
25.47	0\\
25.48	0\\
25.49	0\\
25.5	0\\
25.51	0\\
25.52	0\\
25.53	0\\
25.54	0\\
25.55	0\\
25.56	0\\
25.57	0\\
25.58	0\\
25.59	0\\
25.6	0\\
25.61	0\\
25.62	0\\
25.63	0\\
25.64	0\\
25.65	0\\
25.66	0\\
25.67	0\\
25.68	0\\
25.69	0\\
25.7	0\\
25.71	0\\
25.72	0\\
25.73	0\\
25.74	0\\
25.75	0\\
25.76	0\\
25.77	0\\
25.78	0\\
25.79	0\\
25.8	0\\
25.81	0\\
25.82	0\\
25.83	0\\
25.84	0\\
25.85	0\\
25.86	0\\
25.87	0\\
25.88	0\\
25.89	0\\
25.9	0\\
25.91	0\\
25.92	0\\
25.93	0\\
25.94	0\\
25.95	0\\
25.96	0\\
25.97	0\\
25.98	0\\
25.99	0\\
26	0\\
26.01	0\\
26.02	0\\
26.03	0\\
26.04	0\\
26.05	0\\
26.06	0\\
26.07	0\\
26.08	0\\
26.09	0\\
26.1	0\\
26.11	0\\
26.12	0\\
26.13	0\\
26.14	0\\
26.15	0\\
26.16	0\\
26.17	0\\
26.18	0\\
26.19	0\\
26.2	0\\
26.21	0\\
26.22	0\\
26.23	0\\
26.24	0\\
26.25	0\\
26.26	0\\
26.27	0\\
26.28	0\\
26.29	0\\
26.3	0\\
26.31	0\\
26.32	0\\
26.33	0\\
26.34	0\\
26.35	0\\
26.36	0\\
26.37	0\\
26.38	0\\
26.39	0\\
26.4	0\\
26.41	0\\
26.42	0\\
26.43	0\\
26.44	0\\
26.45	0\\
26.46	0\\
26.47	0\\
26.48	0\\
26.49	0\\
26.5	0\\
26.51	0\\
26.52	0\\
26.53	0\\
26.54	0\\
26.55	0\\
26.56	0\\
26.57	0\\
26.58	0\\
26.59	0\\
26.6	0\\
26.61	0\\
26.62	0\\
26.63	0\\
26.64	0\\
26.65	0\\
26.66	0\\
26.67	0\\
26.68	0\\
26.69	0\\
26.7	0\\
26.71	0\\
26.72	0\\
26.73	0\\
26.74	0\\
26.75	0\\
26.76	0\\
26.77	0\\
26.78	0\\
26.79	0\\
26.8	0\\
26.81	0\\
26.82	0\\
26.83	0\\
26.84	0\\
26.85	0\\
26.86	0\\
26.87	0\\
26.88	0\\
26.89	0\\
26.9	0\\
26.91	0\\
26.92	0\\
26.93	0\\
26.94	0\\
26.95	0\\
26.96	0\\
26.97	0\\
26.98	0\\
26.99	0\\
27	0\\
27.01	0\\
27.02	0\\
27.03	0\\
27.04	0\\
27.05	0\\
27.06	0\\
27.07	0\\
27.08	0\\
27.09	0\\
27.1	0\\
27.11	0\\
27.12	0\\
27.13	0\\
27.14	0\\
27.15	0\\
27.16	0\\
27.17	0\\
27.18	0\\
27.19	0\\
27.2	0\\
27.21	0\\
27.22	0\\
27.23	0\\
27.24	0\\
27.25	0\\
27.26	0\\
27.27	0\\
27.28	0\\
27.29	0\\
27.3	0\\
27.31	0\\
27.32	0\\
27.33	0\\
27.34	0\\
27.35	0\\
27.36	0\\
27.37	0\\
27.38	0\\
27.39	0\\
27.4	0\\
27.41	0\\
27.42	0\\
27.43	0\\
27.44	0\\
27.45	0\\
27.46	0\\
27.47	0\\
27.48	0\\
27.49	0\\
27.5	0\\
27.51	0\\
27.52	0\\
27.53	0\\
27.54	0\\
27.55	0\\
27.56	0\\
27.57	0\\
27.58	0\\
27.59	0\\
27.6	0\\
27.61	0\\
27.62	0\\
27.63	0\\
27.64	0\\
27.65	0\\
27.66	0\\
27.67	0\\
27.68	0\\
27.69	0\\
27.7	0\\
27.71	0\\
27.72	0\\
27.73	0\\
27.74	0\\
27.75	0\\
27.76	0\\
27.77	0\\
27.78	0\\
27.79	0\\
27.8	0\\
27.81	0\\
27.82	0\\
27.83	0\\
27.84	0\\
27.85	0\\
27.86	0\\
27.87	0\\
27.88	0\\
27.89	0\\
27.9	0\\
27.91	0\\
27.92	0\\
27.93	0\\
27.94	0\\
27.95	0\\
27.96	0\\
27.97	0\\
27.98	0\\
27.99	0\\
28	0\\
28.01	0\\
28.02	0\\
28.03	0\\
28.04	0\\
28.05	0\\
28.06	0\\
28.07	0\\
28.08	0\\
28.09	0\\
28.1	0\\
28.11	0\\
28.12	0\\
28.13	0\\
28.14	0\\
28.15	0\\
28.16	0\\
28.17	0\\
28.18	0\\
28.19	0\\
28.2	0\\
28.21	0\\
28.22	0\\
28.23	0\\
28.24	0\\
28.25	0\\
28.26	0\\
28.27	0\\
28.28	0\\
28.29	0\\
28.3	0\\
28.31	0\\
28.32	0\\
28.33	0\\
28.34	0\\
28.35	0\\
28.36	0\\
28.37	0\\
28.38	0\\
28.39	0\\
28.4	0\\
28.41	0\\
28.42	0\\
28.43	0\\
28.44	0\\
28.45	0\\
28.46	0\\
28.47	0\\
28.48	0\\
28.49	0\\
28.5	0\\
28.51	0\\
28.52	0\\
28.53	0\\
28.54	0\\
28.55	0\\
28.56	0\\
28.57	0\\
28.58	0\\
28.59	0\\
28.6	0\\
28.61	0\\
28.62	0\\
28.63	0\\
28.64	0\\
28.65	0\\
28.66	0\\
28.67	0\\
28.68	0\\
28.69	0\\
28.7	0\\
28.71	0\\
28.72	0\\
28.73	0\\
28.74	0\\
28.75	0\\
28.76	0\\
28.77	0\\
28.78	0\\
28.79	0\\
28.8	0\\
28.81	0\\
28.82	0\\
28.83	0\\
28.84	0\\
28.85	0\\
28.86	0\\
28.87	0\\
28.88	0\\
28.89	0\\
28.9	0\\
28.91	0\\
28.92	0\\
28.93	0\\
28.94	0\\
28.95	0\\
28.96	0\\
28.97	0\\
28.98	0\\
28.99	0\\
29	0\\
29.01	0\\
29.02	0\\
29.03	0\\
29.04	0\\
29.05	0\\
29.06	0\\
29.07	0\\
29.08	0\\
29.09	0\\
29.1	0\\
29.11	0\\
29.12	0\\
29.13	0\\
29.14	0\\
29.15	0\\
29.16	0\\
29.17	0\\
29.18	0\\
29.19	0\\
29.2	0\\
29.21	0\\
29.22	0\\
29.23	0\\
29.24	0\\
29.25	0\\
29.26	0\\
29.27	0\\
29.28	0\\
29.29	0\\
29.3	0\\
29.31	0\\
29.32	0\\
29.33	0\\
29.34	0\\
29.35	0\\
29.36	0\\
29.37	0\\
29.38	0\\
29.39	0\\
29.4	0\\
29.41	0\\
29.42	0\\
29.43	0\\
29.44	0\\
29.45	0\\
29.46	0\\
29.47	0\\
29.48	0\\
29.49	0\\
29.5	0\\
29.51	0\\
29.52	0\\
29.53	0\\
29.54	0\\
29.55	0\\
29.56	0\\
29.57	0\\
29.58	0\\
29.59	0\\
29.6	0\\
29.61	0\\
29.62	0\\
29.63	0\\
29.64	0\\
29.65	0\\
29.66	0\\
29.67	0\\
29.68	0\\
29.69	0\\
29.7	0\\
29.71	0\\
29.72	0\\
29.73	0\\
29.74	0\\
29.75	0\\
29.76	0\\
29.77	0\\
29.78	0\\
29.79	0\\
29.8	0\\
29.81	0\\
29.82	0\\
29.83	0\\
29.84	0\\
29.85	0\\
29.86	0\\
29.87	0\\
29.88	0\\
29.89	0\\
29.9	0\\
29.91	0\\
29.92	0\\
29.93	0\\
29.94	0\\
29.95	0\\
29.96	0\\
29.97	0\\
29.98	0\\
29.99	0\\
30	0\\
30.01	0\\
30.02	0\\
30.03	0\\
30.04	0\\
30.05	0\\
30.06	0\\
30.07	0\\
30.08	0\\
30.09	0\\
30.1	0\\
30.11	0\\
30.12	0\\
30.13	0\\
30.14	0\\
30.15	0\\
30.16	0\\
30.17	0\\
30.18	0\\
30.19	0\\
30.2	0\\
30.21	0\\
30.22	0\\
30.23	0\\
30.24	0\\
30.25	0\\
30.26	0\\
30.27	0\\
30.28	0\\
30.29	0\\
30.3	0\\
30.31	0\\
30.32	0\\
30.33	0\\
30.34	0\\
30.35	0\\
30.36	0\\
30.37	0\\
30.38	0\\
30.39	0\\
30.4	0\\
30.41	0\\
30.42	0\\
30.43	0\\
30.44	0\\
30.45	0\\
30.46	0\\
30.47	0\\
30.48	0\\
30.49	0\\
30.5	0\\
30.51	0\\
30.52	0\\
30.53	0\\
30.54	0\\
30.55	0\\
30.56	0\\
30.57	0\\
30.58	0\\
30.59	0\\
30.6	0\\
30.61	0\\
30.62	0\\
30.63	0\\
30.64	0\\
30.65	0\\
30.66	0\\
30.67	0\\
30.68	0\\
30.69	0\\
30.7	0\\
30.71	0\\
30.72	0\\
30.73	0\\
30.74	0\\
30.75	0\\
30.76	0\\
30.77	0\\
30.78	0\\
30.79	0\\
30.8	0\\
30.81	0\\
30.82	0\\
30.83	0\\
30.84	0\\
30.85	0\\
30.86	0\\
30.87	0\\
30.88	0\\
30.89	0\\
30.9	0\\
30.91	0\\
30.92	0\\
30.93	0\\
30.94	0\\
30.95	0\\
30.96	0\\
30.97	0\\
30.98	0\\
30.99	0\\
31	0\\
31.01	0\\
31.02	0\\
31.03	0\\
31.04	0\\
31.05	0\\
31.06	0\\
31.07	0\\
31.08	0\\
31.09	0\\
31.1	0\\
31.11	0\\
31.12	0\\
31.13	0\\
31.14	0\\
31.15	0\\
31.16	0\\
31.17	0\\
31.18	0\\
31.19	0\\
31.2	0\\
31.21	0\\
31.22	0\\
31.23	0\\
31.24	0\\
31.25	0\\
31.26	0\\
31.27	0\\
31.28	0\\
31.29	0\\
31.3	0\\
31.31	0\\
31.32	0\\
31.33	0\\
31.34	0\\
31.35	0\\
31.36	0\\
31.37	0\\
31.38	0\\
31.39	0\\
31.4	0\\
31.41	0\\
31.42	0\\
31.43	0\\
31.44	0\\
31.45	0\\
31.46	0\\
31.47	0\\
31.48	0\\
31.49	0\\
31.5	0\\
31.51	0\\
31.52	0\\
31.53	0\\
31.54	0\\
31.55	0\\
31.56	0\\
31.57	0\\
31.58	0\\
31.59	0\\
31.6	0\\
31.61	0\\
31.62	0\\
31.63	0\\
31.64	0\\
31.65	0\\
31.66	0\\
31.67	0\\
31.68	0\\
31.69	0\\
31.7	0\\
31.71	0\\
31.72	0\\
31.73	0\\
31.74	0\\
31.75	0\\
31.76	0\\
31.77	0\\
31.78	0\\
31.79	0\\
31.8	0\\
31.81	0\\
31.82	0\\
31.83	0\\
31.84	0\\
31.85	0\\
31.86	0\\
31.87	0\\
31.88	0\\
31.89	0\\
31.9	0\\
31.91	0\\
31.92	0\\
31.93	0\\
31.94	0\\
31.95	0\\
31.96	0\\
31.97	0\\
31.98	0\\
31.99	0\\
32	0\\
32.01	0\\
32.02	0\\
32.03	0\\
32.04	0\\
32.05	0\\
32.06	0\\
32.07	0\\
32.08	0\\
32.09	0\\
32.1	0\\
32.11	0\\
32.12	0\\
32.13	0\\
32.14	0\\
32.15	0\\
32.16	0\\
32.17	0\\
32.18	0\\
32.19	0\\
32.2	0\\
32.21	0\\
32.22	0\\
32.23	0\\
32.24	0\\
32.25	0\\
32.26	0\\
32.27	0\\
32.28	0\\
32.29	0\\
32.3	0\\
32.31	0\\
32.32	0\\
32.33	0\\
32.34	0\\
32.35	0\\
32.36	0\\
32.37	0\\
32.38	0\\
32.39	0\\
32.4	0\\
32.41	0\\
32.42	0\\
32.43	0\\
32.44	0\\
32.45	0\\
32.46	0\\
32.47	0\\
32.48	0\\
32.49	0\\
32.5	0\\
32.51	0\\
32.52	0\\
32.53	0\\
32.54	0\\
32.55	0\\
32.56	0\\
32.57	0\\
32.58	0\\
32.59	0\\
32.6	0\\
32.61	0\\
32.62	0\\
32.63	0\\
32.64	0\\
32.65	0\\
32.66	0\\
32.67	0\\
32.68	0\\
32.69	0\\
32.7	0\\
32.71	0\\
32.72	0\\
32.73	0\\
32.74	0\\
32.75	0\\
32.76	0\\
32.77	0\\
32.78	0\\
32.79	0\\
32.8	0\\
32.81	0\\
32.82	0\\
32.83	0\\
32.84	0\\
32.85	0\\
32.86	0\\
32.87	0\\
32.88	0\\
32.89	0\\
32.9	0\\
32.91	0\\
32.92	0\\
32.93	0\\
32.94	0\\
32.95	0\\
32.96	0\\
32.97	0\\
32.98	0\\
32.99	0\\
33	0\\
33.01	0\\
33.02	0\\
33.03	0\\
33.04	0\\
33.05	0\\
33.06	0\\
33.07	0\\
33.08	0\\
33.09	0\\
33.1	0\\
33.11	0\\
33.12	0\\
33.13	0\\
33.14	0\\
33.15	0\\
33.16	0\\
33.17	0\\
33.18	0\\
33.19	0\\
33.2	0\\
33.21	0\\
33.22	0\\
33.23	0\\
33.24	0\\
33.25	0\\
33.26	0\\
33.27	0\\
33.28	0\\
33.29	0\\
33.3	0\\
33.31	0\\
33.32	0\\
33.33	0\\
33.34	0\\
33.35	0\\
33.36	0\\
33.37	0\\
33.38	0\\
33.39	0\\
33.4	0\\
33.41	0\\
33.42	0\\
33.43	0\\
33.44	0\\
33.45	0\\
33.46	0\\
33.47	0\\
33.48	0\\
33.49	0\\
33.5	0\\
33.51	0\\
33.52	0\\
33.53	0\\
33.54	0\\
33.55	0\\
33.56	0\\
33.57	0\\
33.58	0\\
33.59	0\\
33.6	0\\
33.61	0\\
33.62	0\\
33.63	0\\
33.64	0\\
33.65	0\\
33.66	0\\
33.67	0\\
33.68	0\\
33.69	0\\
33.7	0\\
33.71	0\\
33.72	0\\
33.73	0\\
33.74	0\\
33.75	0\\
33.76	0\\
33.77	0\\
33.78	0\\
33.79	0\\
33.8	0\\
33.81	0\\
33.82	0\\
33.83	0\\
33.84	0\\
33.85	0\\
33.86	0\\
33.87	0\\
33.88	0\\
33.89	0\\
33.9	0\\
33.91	0\\
33.92	0\\
33.93	0\\
33.94	0\\
33.95	0\\
33.96	0\\
33.97	0\\
33.98	0\\
33.99	0\\
34	0\\
34.01	0\\
34.02	0\\
34.03	0\\
34.04	0\\
34.05	0\\
34.06	0\\
34.07	0\\
34.08	0\\
34.09	0\\
34.1	0\\
34.11	0\\
34.12	0\\
34.13	0\\
34.14	0\\
34.15	0\\
34.16	0\\
34.17	0\\
34.18	0\\
34.19	0\\
34.2	0\\
34.21	0\\
34.22	0\\
34.23	0\\
34.24	0\\
34.25	0\\
34.26	0\\
34.27	0\\
34.28	0\\
34.29	0\\
34.3	0\\
34.31	0\\
34.32	0\\
34.33	0\\
34.34	0\\
34.35	0\\
34.36	0\\
34.37	0\\
34.38	0\\
34.39	0\\
34.4	0\\
34.41	0\\
34.42	0\\
34.43	0\\
34.44	0\\
34.45	0\\
34.46	0\\
34.47	0\\
34.48	0\\
34.49	0\\
34.5	0\\
34.51	0\\
34.52	0\\
34.53	0\\
34.54	0\\
34.55	0\\
34.56	0\\
34.57	0\\
34.58	0\\
34.59	0\\
34.6	0\\
34.61	0\\
34.62	0\\
34.63	0\\
34.64	0\\
34.65	0\\
34.66	0\\
34.67	0\\
34.68	0\\
34.69	0\\
34.7	0\\
34.71	0\\
34.72	0\\
34.73	0\\
34.74	0\\
34.75	0\\
34.76	0\\
34.77	0\\
34.78	0\\
34.79	0\\
34.8	0\\
34.81	0\\
34.82	0\\
34.83	0\\
34.84	0\\
34.85	0\\
34.86	0\\
34.87	0\\
34.88	0\\
34.89	0\\
34.9	0\\
34.91	0\\
34.92	0\\
34.93	0\\
34.94	0\\
34.95	0\\
34.96	0\\
34.97	0\\
34.98	0\\
34.99	0\\
35	0\\
35.01	0\\
35.02	0\\
35.03	0\\
35.04	0\\
35.05	0\\
35.06	0\\
35.07	0\\
35.08	0\\
35.09	0\\
35.1	0\\
35.11	0\\
35.12	0\\
35.13	0\\
35.14	0\\
35.15	0\\
35.16	0\\
35.17	0\\
35.18	0\\
35.19	0\\
35.2	0\\
35.21	0\\
35.22	0\\
35.23	0\\
35.24	0\\
35.25	0\\
35.26	0\\
35.27	0\\
35.28	0\\
35.29	0\\
35.3	0\\
35.31	0\\
35.32	0\\
35.33	0\\
35.34	0\\
35.35	0\\
35.36	0\\
35.37	0\\
35.38	0\\
35.39	0\\
35.4	0\\
35.41	0\\
35.42	0\\
35.43	0\\
35.44	0\\
35.45	0\\
35.46	0\\
35.47	0\\
35.48	0\\
35.49	0\\
35.5	0\\
35.51	0\\
35.52	0\\
35.53	0\\
35.54	0\\
35.55	0\\
35.56	0\\
35.57	0\\
35.58	0\\
35.59	0\\
35.6	0\\
35.61	0\\
35.62	0\\
35.63	0\\
35.64	0\\
35.65	0\\
35.66	0\\
35.67	0\\
35.68	0\\
35.69	0\\
35.7	0\\
35.71	0\\
35.72	0\\
35.73	0\\
35.74	0\\
35.75	0\\
35.76	0\\
35.77	0\\
35.78	0\\
35.79	0\\
35.8	0\\
35.81	0\\
35.82	0\\
35.83	0\\
35.84	0\\
35.85	0\\
35.86	0\\
35.87	0\\
35.88	0\\
35.89	0\\
35.9	0\\
35.91	0\\
35.92	0\\
35.93	0\\
35.94	0\\
35.95	0\\
35.96	0\\
35.97	0\\
35.98	0\\
35.99	0\\
36	0\\
36.01	0\\
36.02	0\\
36.03	0\\
36.04	0\\
36.05	0\\
36.06	0\\
36.07	0\\
36.08	0\\
36.09	0\\
36.1	0\\
36.11	0\\
36.12	0\\
36.13	0\\
36.14	0\\
36.15	0\\
36.16	0\\
36.17	0\\
36.18	0\\
36.19	0\\
36.2	0\\
36.21	0\\
36.22	0\\
36.23	0\\
36.24	0\\
36.25	0\\
36.26	0\\
36.27	0\\
36.28	0\\
36.29	0\\
36.3	0\\
36.31	0\\
36.32	0\\
36.33	0\\
36.34	0\\
36.35	0\\
36.36	0\\
36.37	0\\
36.38	0\\
36.39	0\\
36.4	0\\
36.41	0\\
36.42	0\\
36.43	0\\
36.44	0\\
36.45	0\\
36.46	0\\
36.47	0\\
36.48	0\\
36.49	0\\
36.5	0\\
36.51	0\\
36.52	0\\
36.53	0\\
36.54	0\\
36.55	0\\
36.56	0\\
36.57	0\\
36.58	0\\
36.59	0\\
36.6	0\\
36.61	0\\
36.62	0\\
36.63	0\\
36.64	0\\
36.65	0\\
36.66	0\\
36.67	0\\
36.68	0\\
36.69	0\\
36.7	0\\
36.71	0\\
36.72	0\\
36.73	0\\
36.74	0\\
36.75	0\\
36.76	0\\
36.77	0\\
36.78	0\\
36.79	0\\
36.8	0\\
36.81	0\\
36.82	0\\
36.83	0\\
36.84	0\\
36.85	0\\
36.86	0\\
36.87	0\\
36.88	0\\
36.89	0\\
36.9	0\\
36.91	0\\
36.92	0\\
36.93	0\\
36.94	0\\
36.95	0\\
36.96	0\\
36.97	0\\
36.98	0\\
36.99	0\\
37	0\\
37.01	0\\
37.02	0\\
37.03	0\\
37.04	0\\
37.05	0\\
37.06	0\\
37.07	0\\
37.08	0\\
37.09	0\\
37.1	0\\
37.11	0\\
37.12	0\\
37.13	0\\
37.14	0\\
37.15	0\\
37.16	0\\
37.17	0\\
37.18	0\\
37.19	0\\
37.2	0\\
37.21	0\\
37.22	0\\
37.23	0\\
37.24	0\\
37.25	0\\
37.26	0\\
37.27	0\\
37.28	0\\
37.29	0\\
37.3	0\\
37.31	0\\
37.32	0\\
37.33	0\\
37.34	0\\
37.35	0\\
37.36	0\\
37.37	0\\
37.38	0\\
37.39	0\\
37.4	0\\
37.41	0\\
37.42	0\\
37.43	0\\
37.44	0\\
37.45	0\\
37.46	0\\
37.47	0\\
37.48	0\\
37.49	0\\
37.5	0\\
37.51	0\\
37.52	0\\
37.53	0\\
37.54	0\\
37.55	0\\
37.56	0\\
37.57	0\\
37.58	0\\
37.59	0\\
37.6	0\\
37.61	0\\
37.62	0\\
37.63	0\\
37.64	0\\
37.65	0\\
37.66	0\\
37.67	0\\
37.68	0\\
37.69	0\\
37.7	0\\
37.71	0\\
37.72	0\\
37.73	0\\
37.74	0\\
37.75	0\\
37.76	0\\
37.77	0\\
37.78	0\\
37.79	0\\
37.8	0\\
37.81	0\\
37.82	0\\
37.83	0\\
37.84	0\\
37.85	0\\
37.86	0\\
37.87	0\\
37.88	0\\
37.89	0\\
37.9	0\\
37.91	0\\
37.92	0\\
37.93	0\\
37.94	0\\
37.95	0\\
37.96	0\\
37.97	0\\
37.98	0\\
37.99	0\\
38	0\\
38.01	0\\
38.02	0\\
38.03	0\\
38.04	0\\
38.05	0\\
38.06	0\\
38.07	0\\
38.08	0\\
38.09	0\\
38.1	0\\
38.11	0\\
38.12	0\\
38.13	0\\
38.14	0\\
38.15	0\\
38.16	0\\
38.17	0\\
38.18	0\\
38.19	0\\
38.2	0\\
38.21	0\\
38.22	0\\
38.23	0\\
38.24	0\\
38.25	0\\
38.26	0\\
38.27	0\\
38.28	0\\
38.29	0\\
38.3	0\\
38.31	0\\
38.32	0\\
38.33	0\\
38.34	0\\
38.35	0\\
38.36	0\\
38.37	0\\
38.38	0\\
38.39	0\\
38.4	0\\
38.41	0\\
38.42	0\\
38.43	0\\
38.44	0\\
38.45	0\\
38.46	0\\
38.47	0\\
38.48	0\\
38.49	0\\
38.5	0\\
38.51	0\\
38.52	0\\
38.53	0\\
38.54	0\\
38.55	0\\
38.56	0\\
38.57	0\\
38.58	0\\
38.59	0\\
38.6	0\\
38.61	0\\
38.62	0\\
38.63	0\\
38.64	0\\
38.65	0\\
38.66	0\\
38.67	0\\
38.68	0\\
38.69	0\\
38.7	0\\
38.71	0\\
38.72	0\\
38.73	0\\
38.74	0\\
38.75	0\\
38.76	0\\
38.77	0\\
38.78	0\\
38.79	0\\
38.8	0\\
38.81	0\\
38.82	0\\
38.83	0\\
38.84	0\\
38.85	0\\
38.86	0\\
38.87	0\\
38.88	0\\
38.89	0\\
38.9	0\\
38.91	0\\
38.92	0\\
38.93	0\\
38.94	0\\
38.95	0\\
38.96	0\\
38.97	0\\
38.98	0\\
38.99	0\\
39	0\\
39.01	0\\
39.02	0\\
39.03	0\\
39.04	0\\
39.05	0\\
39.06	0\\
39.07	0\\
39.08	0\\
39.09	0\\
39.1	0\\
39.11	0\\
39.12	0\\
39.13	0\\
39.14	0\\
39.15	0\\
39.16	0\\
39.17	0\\
39.18	0\\
39.19	0\\
39.2	0\\
39.21	0\\
39.22	0\\
39.23	0\\
39.24	0\\
39.25	0\\
39.26	0\\
39.27	0\\
39.28	0\\
39.29	0\\
39.3	0\\
39.31	0\\
39.32	0\\
39.33	0\\
39.34	0\\
39.35	0\\
39.36	0\\
39.37	0\\
39.38	0\\
39.39	0\\
39.4	0\\
39.41	0\\
39.42	0\\
39.43	0\\
39.44	0\\
39.45	0\\
39.46	0\\
39.47	0\\
39.48	0\\
39.49	0\\
39.5	0\\
39.51	0\\
39.52	0\\
39.53	0\\
39.54	0\\
39.55	0\\
39.56	0\\
39.57	0\\
39.58	0\\
39.59	0\\
39.6	0\\
39.61	0\\
39.62	0\\
39.63	0\\
39.64	0\\
39.65	0\\
39.66	0\\
39.67	0\\
39.68	0\\
39.69	0\\
39.7	0\\
39.71	0\\
39.72	0\\
39.73	0\\
39.74	0\\
39.75	0\\
39.76	0\\
39.77	0\\
39.78	0\\
39.79	0\\
39.8	0\\
39.81	0\\
39.82	0\\
39.83	0\\
39.84	0\\
39.85	0\\
39.86	0\\
39.87	0\\
39.88	0\\
39.89	0\\
39.9	0\\
39.91	0\\
39.92	0\\
39.93	0\\
39.94	0\\
39.95	0\\
39.96	0\\
39.97	0\\
39.98	0\\
39.99	0\\
40	0\\
40.01	0\\
};
\addplot [color=red,dashed,forget plot]
  table[row sep=crcr]{%
40.01	0\\
40.02	0\\
40.03	0\\
40.04	0\\
40.05	0\\
40.06	0\\
40.07	0\\
40.08	0\\
40.09	0\\
40.1	0\\
40.11	0\\
40.12	0\\
40.13	0\\
40.14	0\\
40.15	0\\
40.16	0\\
40.17	0\\
40.18	0\\
40.19	0\\
40.2	0\\
40.21	0\\
40.22	0\\
40.23	0\\
40.24	0\\
40.25	0\\
40.26	0\\
40.27	0\\
40.28	0\\
40.29	0\\
40.3	0\\
40.31	0\\
40.32	0\\
40.33	0\\
40.34	0\\
40.35	0\\
40.36	0\\
40.37	0\\
40.38	0\\
40.39	0\\
40.4	0\\
40.41	0\\
40.42	0\\
40.43	0\\
40.44	0\\
40.45	0\\
40.46	0\\
40.47	0\\
40.48	0\\
40.49	0\\
40.5	0\\
40.51	0\\
40.52	0\\
40.53	0\\
40.54	0\\
40.55	0\\
40.56	0\\
40.57	0\\
40.58	0\\
40.59	0\\
40.6	0\\
40.61	0\\
40.62	0\\
40.63	0\\
40.64	0\\
40.65	0\\
40.66	0\\
40.67	0\\
40.68	0\\
40.69	0\\
40.7	0\\
40.71	0\\
40.72	0\\
40.73	0\\
40.74	0\\
40.75	0\\
40.76	0\\
40.77	0\\
40.78	0\\
40.79	0\\
40.8	0\\
40.81	0\\
40.82	0\\
40.83	0\\
40.84	0\\
40.85	0\\
40.86	0\\
40.87	0\\
40.88	0\\
40.89	0\\
40.9	0\\
40.91	0\\
40.92	0\\
40.93	0\\
40.94	0\\
40.95	0\\
40.96	0\\
40.97	0\\
40.98	0\\
40.99	0\\
41	0\\
41.01	0\\
41.02	0\\
41.03	0\\
41.04	0\\
41.05	0\\
41.06	0\\
41.07	0\\
41.08	0\\
41.09	0\\
41.1	0\\
41.11	0\\
41.12	0\\
41.13	0\\
41.14	0\\
41.15	0\\
41.16	0\\
41.17	0\\
41.18	0\\
41.19	0\\
41.2	0\\
41.21	0\\
41.22	0\\
41.23	0\\
41.24	0\\
41.25	0\\
41.26	0\\
41.27	0\\
41.28	0\\
41.29	0\\
41.3	0\\
41.31	0\\
41.32	0\\
41.33	0\\
41.34	0\\
41.35	0\\
41.36	0\\
41.37	0\\
41.38	0\\
41.39	0\\
41.4	0\\
41.41	0\\
41.42	0\\
41.43	0\\
41.44	0\\
41.45	0\\
41.46	0\\
41.47	0\\
41.48	0\\
41.49	0\\
41.5	0\\
41.51	0\\
41.52	0\\
41.53	0\\
41.54	0\\
41.55	0\\
41.56	0\\
41.57	0\\
41.58	0\\
41.59	0\\
41.6	0\\
41.61	0\\
41.62	0\\
41.63	0\\
41.64	0\\
41.65	0\\
41.66	0\\
41.67	0\\
41.68	0\\
41.69	0\\
41.7	0\\
41.71	0\\
41.72	0\\
41.73	0\\
41.74	0\\
41.75	0\\
41.76	0\\
41.77	0\\
41.78	0\\
41.79	0\\
41.8	0\\
41.81	0\\
41.82	0\\
41.83	0\\
41.84	0\\
41.85	0\\
41.86	0\\
41.87	0\\
41.88	0\\
41.89	0\\
41.9	0\\
41.91	0\\
41.92	0\\
41.93	0\\
41.94	0\\
41.95	0\\
41.96	0\\
41.97	0\\
41.98	0\\
41.99	0\\
42	0\\
42.01	0\\
42.02	0\\
42.03	0\\
42.04	0\\
42.05	0\\
42.06	0\\
42.07	0\\
42.08	0\\
42.09	0\\
42.1	0\\
42.11	0\\
42.12	0\\
42.13	0\\
42.14	0\\
42.15	0\\
42.16	0\\
42.17	0\\
42.18	0\\
42.19	0\\
42.2	0\\
42.21	0\\
42.22	0\\
42.23	0\\
42.24	0\\
42.25	0\\
42.26	0\\
42.27	0\\
42.28	0\\
42.29	0\\
42.3	0\\
42.31	0\\
42.32	0\\
42.33	0\\
42.34	0\\
42.35	0\\
42.36	0\\
42.37	0\\
42.38	0\\
42.39	0\\
42.4	0\\
42.41	0\\
42.42	0\\
42.43	0\\
42.44	0\\
42.45	0\\
42.46	0\\
42.47	0\\
42.48	0\\
42.49	0\\
42.5	0\\
42.51	0\\
42.52	0\\
42.53	0\\
42.54	0\\
42.55	0\\
42.56	0\\
42.57	0\\
42.58	0\\
42.59	0\\
42.6	0\\
42.61	0\\
42.62	0\\
42.63	0\\
42.64	0\\
42.65	0\\
42.66	0\\
42.67	0\\
42.68	0\\
42.69	0\\
42.7	0\\
42.71	0\\
42.72	0\\
42.73	0\\
42.74	0\\
42.75	0\\
42.76	0\\
42.77	0\\
42.78	0\\
42.79	0\\
42.8	0\\
42.81	0\\
42.82	0\\
42.83	0\\
42.84	0\\
42.85	0\\
42.86	0\\
42.87	0\\
42.88	0\\
42.89	0\\
42.9	0\\
42.91	0\\
42.92	0\\
42.93	0\\
42.94	0\\
42.95	0\\
42.96	0\\
42.97	0\\
42.98	0\\
42.99	0\\
43	0\\
43.01	0\\
43.02	0\\
43.03	0\\
43.04	0\\
43.05	0\\
43.06	0\\
43.07	0\\
43.08	0\\
43.09	0\\
43.1	0\\
43.11	0\\
43.12	0\\
43.13	0\\
43.14	0\\
43.15	0\\
43.16	0\\
43.17	0\\
43.18	0\\
43.19	0\\
43.2	0\\
43.21	0\\
43.22	0\\
43.23	0\\
43.24	0\\
43.25	0\\
43.26	0\\
43.27	0\\
43.28	0\\
43.29	0\\
43.3	0\\
43.31	0\\
43.32	0\\
43.33	0\\
43.34	0\\
43.35	0\\
43.36	0\\
43.37	0\\
43.38	0\\
43.39	0\\
43.4	0\\
43.41	0\\
43.42	0\\
43.43	0\\
43.44	0\\
43.45	0\\
43.46	0\\
43.47	0\\
43.48	0\\
43.49	0\\
43.5	0\\
43.51	0\\
43.52	0\\
43.53	0\\
43.54	0\\
43.55	0\\
43.56	0\\
43.57	0\\
43.58	0\\
43.59	0\\
43.6	0\\
43.61	0\\
43.62	0\\
43.63	0\\
43.64	0\\
43.65	0\\
43.66	0\\
43.67	0\\
43.68	0\\
43.69	0\\
43.7	0\\
43.71	0\\
43.72	0\\
43.73	0\\
43.74	0\\
43.75	0\\
43.76	0\\
43.77	0\\
43.78	0\\
43.79	0\\
43.8	0\\
43.81	0\\
43.82	0\\
43.83	0\\
43.84	0\\
43.85	0\\
43.86	0\\
43.87	0\\
43.88	0\\
43.89	0\\
43.9	0\\
43.91	0\\
43.92	0\\
43.93	0\\
43.94	0\\
43.95	0\\
43.96	0\\
43.97	0\\
43.98	0\\
43.99	0\\
44	0\\
44.01	0\\
44.02	0\\
44.03	0\\
44.04	0\\
44.05	0\\
44.06	0\\
44.07	0\\
44.08	0\\
44.09	0\\
44.1	0\\
44.11	0\\
44.12	0\\
44.13	0\\
44.14	0\\
44.15	0\\
44.16	0\\
44.17	0\\
44.18	0\\
44.19	0\\
44.2	0\\
44.21	0\\
44.22	0\\
44.23	0\\
44.24	0\\
44.25	0\\
44.26	0\\
44.27	0\\
44.28	0\\
44.29	0\\
44.3	0\\
44.31	0\\
44.32	0\\
44.33	0\\
44.34	0\\
44.35	0\\
44.36	0\\
44.37	0\\
44.38	0\\
44.39	0\\
44.4	0\\
44.41	0\\
44.42	0\\
44.43	0\\
44.44	0\\
44.45	0\\
44.46	0\\
44.47	0\\
44.48	0\\
44.49	0\\
44.5	0\\
44.51	0\\
44.52	0\\
44.53	0\\
44.54	0\\
44.55	0\\
44.56	0\\
44.57	0\\
44.58	0\\
44.59	0\\
44.6	0\\
44.61	0\\
44.62	0\\
44.63	0\\
44.64	0\\
44.65	0\\
44.66	0\\
44.67	0\\
44.68	0\\
44.69	0\\
44.7	0\\
44.71	0\\
44.72	0\\
44.73	0\\
44.74	0\\
44.75	0\\
44.76	0\\
44.77	0\\
44.78	0\\
44.79	0\\
44.8	0\\
44.81	0\\
44.82	0\\
44.83	0\\
44.84	0\\
44.85	0\\
44.86	0\\
44.87	0\\
44.88	0\\
44.89	0\\
44.9	0\\
44.91	0\\
44.92	0\\
44.93	0\\
44.94	0\\
44.95	0\\
44.96	0\\
44.97	0\\
44.98	0\\
44.99	0\\
45	0\\
45.01	0\\
45.02	0\\
45.03	0\\
45.04	0\\
45.05	0\\
45.06	0\\
45.07	0\\
45.08	0\\
45.09	0\\
45.1	0\\
45.11	0\\
45.12	0\\
45.13	0\\
45.14	0\\
45.15	0\\
45.16	0\\
45.17	0\\
45.18	0\\
45.19	0\\
45.2	0\\
45.21	0\\
45.22	0\\
45.23	0\\
45.24	0\\
45.25	0\\
45.26	0\\
45.27	0\\
45.28	0\\
45.29	0\\
45.3	0\\
45.31	0\\
45.32	0\\
45.33	0\\
45.34	0\\
45.35	0\\
45.36	0\\
45.37	0\\
45.38	0\\
45.39	0\\
45.4	0\\
45.41	0\\
45.42	0\\
45.43	0\\
45.44	0\\
45.45	0\\
45.46	0\\
45.47	0\\
45.48	0\\
45.49	0\\
45.5	0\\
45.51	0\\
45.52	0\\
45.53	0\\
45.54	0\\
45.55	0\\
45.56	0\\
45.57	0\\
45.58	0\\
45.59	0\\
45.6	0\\
45.61	0\\
45.62	0\\
45.63	0\\
45.64	0\\
45.65	0\\
45.66	0\\
45.67	0\\
45.68	0\\
45.69	0\\
45.7	0\\
45.71	0\\
45.72	0\\
45.73	0\\
45.74	0\\
45.75	0\\
45.76	0\\
45.77	0\\
45.78	0\\
45.79	0\\
45.8	0\\
45.81	0\\
45.82	0\\
45.83	0\\
45.84	0\\
45.85	0\\
45.86	0\\
45.87	0\\
45.88	0\\
45.89	0\\
45.9	0\\
45.91	0\\
45.92	0\\
45.93	0\\
45.94	0\\
45.95	0\\
45.96	0\\
45.97	0\\
45.98	0\\
45.99	0\\
46	0\\
46.01	0\\
46.02	0\\
46.03	0\\
46.04	0\\
46.05	0\\
46.06	0\\
46.07	0\\
46.08	0\\
46.09	0\\
46.1	0\\
46.11	0\\
46.12	0\\
46.13	0\\
46.14	0\\
46.15	0\\
46.16	0\\
46.17	0\\
46.18	0\\
46.19	0\\
46.2	0\\
46.21	0\\
46.22	0\\
46.23	0\\
46.24	0\\
46.25	0\\
46.26	0\\
46.27	0\\
46.28	0\\
46.29	0\\
46.3	0\\
46.31	0\\
46.32	0\\
46.33	0\\
46.34	0\\
46.35	0\\
46.36	0\\
46.37	0\\
46.38	0\\
46.39	0\\
46.4	0\\
46.41	0\\
46.42	0\\
46.43	0\\
46.44	0\\
46.45	0\\
46.46	0\\
46.47	0\\
46.48	0\\
46.49	0\\
46.5	0\\
46.51	0\\
46.52	0\\
46.53	0\\
46.54	0\\
46.55	0\\
46.56	0\\
46.57	0\\
46.58	0\\
46.59	0\\
46.6	0\\
46.61	0\\
46.62	0\\
46.63	0\\
46.64	0\\
46.65	0\\
46.66	0\\
46.67	0\\
46.68	0\\
46.69	0\\
46.7	0\\
46.71	0\\
46.72	0\\
46.73	0\\
46.74	0\\
46.75	0\\
46.76	0\\
46.77	0\\
46.78	0\\
46.79	0\\
46.8	0\\
46.81	0\\
46.82	0\\
46.83	0\\
46.84	0\\
46.85	0\\
46.86	0\\
46.87	0\\
46.88	0\\
46.89	0\\
46.9	0\\
46.91	0\\
46.92	0\\
46.93	0\\
46.94	0\\
46.95	0\\
46.96	0\\
46.97	0\\
46.98	0\\
46.99	0\\
47	0\\
47.01	0\\
47.02	0\\
47.03	0\\
47.04	0\\
47.05	0\\
47.06	0\\
47.07	0\\
47.08	0\\
47.09	0\\
47.1	0\\
47.11	0\\
47.12	0\\
47.13	0\\
47.14	0\\
47.15	0\\
47.16	0\\
47.17	0\\
47.18	0\\
47.19	0\\
47.2	0\\
47.21	0\\
47.22	0\\
47.23	0\\
47.24	0\\
47.25	0\\
47.26	0\\
47.27	0\\
47.28	0\\
47.29	0\\
47.3	0\\
47.31	0\\
47.32	0\\
47.33	0\\
47.34	0\\
47.35	0\\
47.36	0\\
47.37	0\\
47.38	0\\
47.39	0\\
47.4	0\\
47.41	0\\
47.42	0\\
47.43	0\\
47.44	0\\
47.45	0\\
47.46	0\\
47.47	0\\
47.48	0\\
47.49	0\\
47.5	0\\
47.51	0\\
47.52	0\\
47.53	0\\
47.54	0\\
47.55	0\\
47.56	0\\
47.57	0\\
47.58	0\\
47.59	0\\
47.6	0\\
47.61	0\\
47.62	0\\
47.63	0\\
47.64	0\\
47.65	0\\
47.66	0\\
47.67	0\\
47.68	0\\
47.69	0\\
47.7	0\\
47.71	0\\
47.72	0\\
47.73	0\\
47.74	0\\
47.75	0\\
47.76	0\\
47.77	0\\
47.78	0\\
47.79	0\\
47.8	0\\
47.81	0\\
47.82	0\\
47.83	0\\
47.84	0\\
47.85	0\\
47.86	0\\
47.87	0\\
47.88	0\\
47.89	0\\
47.9	0\\
47.91	0\\
47.92	0\\
47.93	0\\
47.94	0\\
47.95	0\\
47.96	0\\
47.97	0\\
47.98	0\\
47.99	0\\
48	0\\
48.01	0\\
48.02	0\\
48.03	0\\
48.04	0\\
48.05	0\\
48.06	0\\
48.07	0\\
48.08	0\\
48.09	0\\
48.1	0\\
48.11	0\\
48.12	0\\
48.13	0\\
48.14	0\\
48.15	0\\
48.16	0\\
48.17	0\\
48.18	0\\
48.19	0\\
48.2	0\\
48.21	0\\
48.22	0\\
48.23	0\\
48.24	0\\
48.25	0\\
48.26	0\\
48.27	0\\
48.28	0\\
48.29	0\\
48.3	0\\
48.31	0\\
48.32	0\\
48.33	0\\
48.34	0\\
48.35	0\\
48.36	0\\
48.37	0\\
48.38	0\\
48.39	0\\
48.4	0\\
48.41	0\\
48.42	0\\
48.43	0\\
48.44	0\\
48.45	0\\
48.46	0\\
48.47	0\\
48.48	0\\
48.49	0\\
48.5	0\\
48.51	0\\
48.52	0\\
48.53	0\\
48.54	0\\
48.55	0\\
48.56	0\\
48.57	0\\
48.58	0\\
48.59	0\\
48.6	0\\
48.61	0\\
48.62	0\\
48.63	0\\
48.64	0\\
48.65	0\\
48.66	0\\
48.67	0\\
48.68	0\\
48.69	0\\
48.7	0\\
48.71	0\\
48.72	0\\
48.73	0\\
48.74	0\\
48.75	0\\
48.76	0\\
48.77	0\\
48.78	0\\
48.79	0\\
48.8	0\\
48.81	0\\
48.82	0\\
48.83	0\\
48.84	0\\
48.85	0\\
48.86	0\\
48.87	0\\
48.88	0\\
48.89	0\\
48.9	0\\
48.91	0\\
48.92	0\\
48.93	0\\
48.94	0\\
48.95	0\\
48.96	0\\
48.97	0\\
48.98	0\\
48.99	0\\
49	0\\
49.01	0\\
49.02	0\\
49.03	0\\
49.04	0\\
49.05	0\\
49.06	0\\
49.07	0\\
49.08	0\\
49.09	0\\
49.1	0\\
49.11	0\\
49.12	0\\
49.13	0\\
49.14	0\\
49.15	0\\
49.16	0\\
49.17	0\\
49.18	0\\
49.19	0\\
49.2	0\\
49.21	0\\
49.22	0\\
49.23	0\\
49.24	0\\
49.25	0\\
49.26	0\\
49.27	0\\
49.28	0\\
49.29	0\\
49.3	0\\
49.31	0\\
49.32	0\\
49.33	0\\
49.34	0\\
49.35	0\\
49.36	0\\
49.37	0\\
49.38	0\\
49.39	0\\
49.4	0\\
49.41	0\\
49.42	0\\
49.43	0\\
49.44	0\\
49.45	0\\
49.46	0\\
49.47	0\\
49.48	0\\
49.49	0\\
49.5	0\\
49.51	0\\
49.52	0\\
49.53	0\\
49.54	0\\
49.55	0\\
49.56	0\\
49.57	0\\
49.58	0\\
49.59	0\\
49.6	0\\
49.61	0\\
49.62	0\\
49.63	0\\
49.64	0\\
49.65	0\\
49.66	0\\
49.67	0\\
49.68	0\\
49.69	0\\
49.7	0\\
49.71	0\\
49.72	0\\
49.73	0\\
49.74	0\\
49.75	0\\
49.76	0\\
49.77	0\\
49.78	0\\
49.79	0\\
49.8	0\\
49.81	0\\
49.82	0\\
49.83	0\\
49.84	0\\
49.85	0\\
49.86	0\\
49.87	0\\
49.88	0\\
49.89	0\\
49.9	0\\
49.91	0\\
49.92	0\\
49.93	0\\
49.94	0\\
49.95	0\\
49.96	0\\
49.97	0\\
49.98	0\\
49.99	0\\
50	0\\
50.01	0\\
50.02	0\\
50.03	0\\
50.04	0\\
50.05	0\\
50.06	0\\
50.07	0\\
50.08	0\\
50.09	0\\
50.1	0\\
50.11	0\\
50.12	0\\
50.13	0\\
50.14	0\\
50.15	0\\
50.16	0\\
50.17	0\\
50.18	0\\
50.19	0\\
50.2	0\\
50.21	0\\
50.22	0\\
50.23	0\\
50.24	0\\
50.25	0\\
50.26	0\\
50.27	0\\
50.28	0\\
50.29	0\\
50.3	0\\
50.31	0\\
50.32	0\\
50.33	0\\
50.34	0\\
50.35	0\\
50.36	0\\
50.37	0\\
50.38	0\\
50.39	0\\
50.4	0\\
50.41	0\\
50.42	0\\
50.43	0\\
50.44	0\\
50.45	0\\
50.46	0\\
50.47	0\\
50.48	0\\
50.49	0\\
50.5	0\\
50.51	0\\
50.52	0\\
50.53	0\\
50.54	0\\
50.55	0\\
50.56	0\\
50.57	0\\
50.58	0\\
50.59	0\\
50.6	0\\
50.61	0\\
50.62	0\\
50.63	0\\
50.64	0\\
50.65	0\\
50.66	0\\
50.67	0\\
50.68	0\\
50.69	0\\
50.7	0\\
50.71	0\\
50.72	0\\
50.73	0\\
50.74	0\\
50.75	0\\
50.76	0\\
50.77	0\\
50.78	0\\
50.79	0\\
50.8	0\\
50.81	0\\
50.82	0\\
50.83	0\\
50.84	0\\
50.85	0\\
50.86	0\\
50.87	0\\
50.88	0\\
50.89	0\\
50.9	0\\
50.91	0\\
50.92	0\\
50.93	0\\
50.94	0\\
50.95	0\\
50.96	0\\
50.97	0\\
50.98	0\\
50.99	0\\
51	0\\
51.01	0\\
51.02	0\\
51.03	0\\
51.04	0\\
51.05	0\\
51.06	0\\
51.07	0\\
51.08	0\\
51.09	0\\
51.1	0\\
51.11	0\\
51.12	0\\
51.13	0\\
51.14	0\\
51.15	0\\
51.16	0\\
51.17	0\\
51.18	0\\
51.19	0\\
51.2	0\\
51.21	0\\
51.22	0\\
51.23	0\\
51.24	0\\
51.25	0\\
51.26	0\\
51.27	0\\
51.28	0\\
51.29	0\\
51.3	0\\
51.31	0\\
51.32	0\\
51.33	0\\
51.34	0\\
51.35	0\\
51.36	0\\
51.37	0\\
51.38	0\\
51.39	0\\
51.4	0\\
51.41	0\\
51.42	0\\
51.43	0\\
51.44	0\\
51.45	0\\
51.46	0\\
51.47	0\\
51.48	0\\
51.49	0\\
51.5	0\\
51.51	0\\
51.52	0\\
51.53	0\\
51.54	0\\
51.55	0\\
51.56	0\\
51.57	0\\
51.58	0\\
51.59	0\\
51.6	0\\
51.61	0\\
51.62	0\\
51.63	0\\
51.64	0\\
51.65	0\\
51.66	0\\
51.67	0\\
51.68	0\\
51.69	0\\
51.7	0\\
51.71	0\\
51.72	0\\
51.73	0\\
51.74	0\\
51.75	0\\
51.76	0\\
51.77	0\\
51.78	0\\
51.79	0\\
51.8	0\\
51.81	0\\
51.82	0\\
51.83	0\\
51.84	0\\
51.85	0\\
51.86	0\\
51.87	0\\
51.88	0\\
51.89	0\\
51.9	0\\
51.91	0\\
51.92	0\\
51.93	0\\
51.94	0\\
51.95	0\\
51.96	0\\
51.97	0\\
51.98	0\\
51.99	0\\
52	0\\
52.01	0\\
52.02	0\\
52.03	0\\
52.04	0\\
52.05	0\\
52.06	0\\
52.07	0\\
52.08	0\\
52.09	0\\
52.1	0\\
52.11	0\\
52.12	0\\
52.13	0\\
52.14	0\\
52.15	0\\
52.16	0\\
52.17	0\\
52.18	0\\
52.19	0\\
52.2	0\\
52.21	0\\
52.22	0\\
52.23	0\\
52.24	0\\
52.25	0\\
52.26	0\\
52.27	0\\
52.28	0\\
52.29	0\\
52.3	0\\
52.31	0\\
52.32	0\\
52.33	0\\
52.34	0\\
52.35	0\\
52.36	0\\
52.37	0\\
52.38	0\\
52.39	0\\
52.4	0\\
52.41	0\\
52.42	0\\
52.43	0\\
52.44	0\\
52.45	0\\
52.46	0\\
52.47	0\\
52.48	0\\
52.49	0\\
52.5	0\\
52.51	0\\
52.52	0\\
52.53	0\\
52.54	0\\
52.55	0\\
52.56	0\\
52.57	0\\
52.58	0\\
52.59	0\\
52.6	0\\
52.61	0\\
52.62	0\\
52.63	0\\
52.64	0\\
52.65	0\\
52.66	0\\
52.67	0\\
52.68	0\\
52.69	0\\
52.7	0\\
52.71	0\\
52.72	0\\
52.73	0\\
52.74	0\\
52.75	0\\
52.76	0\\
52.77	0\\
52.78	0\\
52.79	0\\
52.8	0\\
52.81	0\\
52.82	0\\
52.83	0\\
52.84	0\\
52.85	0\\
52.86	0\\
52.87	0\\
52.88	0\\
52.89	0\\
52.9	0\\
52.91	0\\
52.92	0\\
52.93	0\\
52.94	0\\
52.95	0\\
52.96	0\\
52.97	0\\
52.98	0\\
52.99	0\\
53	0\\
53.01	0\\
53.02	0\\
53.03	0\\
53.04	0\\
53.05	0\\
53.06	0\\
53.07	0\\
53.08	0\\
53.09	0\\
53.1	0\\
53.11	0\\
53.12	0\\
53.13	0\\
53.14	0\\
53.15	0\\
53.16	0\\
53.17	0\\
53.18	0\\
53.19	0\\
53.2	0\\
53.21	0\\
53.22	0\\
53.23	0\\
53.24	0\\
53.25	0\\
53.26	0\\
53.27	0\\
53.28	0\\
53.29	0\\
53.3	0\\
53.31	0\\
53.32	0\\
53.33	0\\
53.34	0\\
53.35	0\\
53.36	0\\
53.37	0\\
53.38	0\\
53.39	0\\
53.4	0\\
53.41	0\\
53.42	0\\
53.43	0\\
53.44	0\\
53.45	0\\
53.46	0\\
53.47	0\\
53.48	0\\
53.49	0\\
53.5	0\\
53.51	0\\
53.52	0\\
53.53	0\\
53.54	0\\
53.55	0\\
53.56	0\\
53.57	0\\
53.58	0\\
53.59	0\\
53.6	0\\
53.61	0\\
53.62	0\\
53.63	0\\
53.64	0\\
53.65	0\\
53.66	0\\
53.67	0\\
53.68	0\\
53.69	0\\
53.7	0\\
53.71	0\\
53.72	0\\
53.73	0\\
53.74	0\\
53.75	0\\
53.76	0\\
53.77	0\\
53.78	0\\
53.79	0\\
53.8	0\\
53.81	0\\
53.82	0\\
53.83	0\\
53.84	0\\
53.85	0\\
53.86	0\\
53.87	0\\
53.88	0\\
53.89	0\\
53.9	0\\
53.91	0\\
53.92	0\\
53.93	0\\
53.94	0\\
53.95	0\\
53.96	0\\
53.97	0\\
53.98	0\\
53.99	0\\
54	0\\
54.01	0\\
54.02	0\\
54.03	0\\
54.04	0\\
54.05	0\\
54.06	0\\
54.07	0\\
54.08	0\\
54.09	0\\
54.1	0\\
54.11	0\\
54.12	0\\
54.13	0\\
54.14	0\\
54.15	0\\
54.16	0\\
54.17	0\\
54.18	0\\
54.19	0\\
54.2	0\\
54.21	0\\
54.22	0\\
54.23	0\\
54.24	0\\
54.25	0\\
54.26	0\\
54.27	0\\
54.28	0\\
54.29	0\\
54.3	0\\
54.31	0\\
54.32	0\\
54.33	0\\
54.34	0\\
54.35	0\\
54.36	0\\
54.37	0\\
54.38	0\\
54.39	0\\
54.4	0\\
54.41	0\\
54.42	0\\
54.43	0\\
54.44	0\\
54.45	0\\
54.46	0\\
54.47	0\\
54.48	0\\
54.49	0\\
54.5	0\\
54.51	0\\
54.52	0\\
54.53	0\\
54.54	0\\
54.55	0\\
54.56	0\\
54.57	0\\
54.58	0\\
54.59	0\\
54.6	0\\
54.61	0\\
54.62	0\\
54.63	0\\
54.64	0\\
54.65	0\\
54.66	0\\
54.67	0\\
54.68	0\\
54.69	0\\
54.7	0\\
54.71	0\\
54.72	0\\
54.73	0\\
54.74	0\\
54.75	0\\
54.76	0\\
54.77	0\\
54.78	0\\
54.79	0\\
54.8	0\\
54.81	0\\
54.82	0\\
54.83	0\\
54.84	0\\
54.85	0\\
54.86	0\\
54.87	0\\
54.88	0\\
54.89	0\\
54.9	0\\
54.91	0\\
54.92	0\\
54.93	0\\
54.94	0\\
54.95	0\\
54.96	0\\
54.97	0\\
54.98	0\\
54.99	0\\
55	0\\
55.01	0\\
55.02	0\\
55.03	0\\
55.04	0\\
55.05	0\\
55.06	0\\
55.07	0\\
55.08	0\\
55.09	0\\
55.1	0\\
55.11	0\\
55.12	0\\
55.13	0\\
55.14	0\\
55.15	0\\
55.16	0\\
55.17	0\\
55.18	0\\
55.19	0\\
55.2	0\\
55.21	0\\
55.22	0\\
55.23	0\\
55.24	0\\
55.25	0\\
55.26	0\\
55.27	0\\
55.28	0\\
55.29	0\\
55.3	0\\
55.31	0\\
55.32	0\\
55.33	0\\
55.34	0\\
55.35	0\\
55.36	0\\
55.37	0\\
55.38	0\\
55.39	0\\
55.4	0\\
55.41	0\\
55.42	0\\
55.43	0\\
55.44	0\\
55.45	0\\
55.46	0\\
55.47	0\\
55.48	0\\
55.49	0\\
55.5	0\\
55.51	0\\
55.52	0\\
55.53	0\\
55.54	0\\
55.55	0\\
55.56	0\\
55.57	0\\
55.58	0\\
55.59	0\\
55.6	0\\
55.61	0\\
55.62	0\\
55.63	0\\
55.64	0\\
55.65	0\\
55.66	0\\
55.67	0\\
55.68	0\\
55.69	0\\
55.7	0\\
55.71	0\\
55.72	0\\
55.73	0\\
55.74	0\\
55.75	0\\
55.76	0\\
55.77	0\\
55.78	0\\
55.79	0\\
55.8	0\\
55.81	0\\
55.82	0\\
55.83	0\\
55.84	0\\
55.85	0\\
55.86	0\\
55.87	0\\
55.88	0\\
55.89	0\\
55.9	0\\
55.91	0\\
55.92	0\\
55.93	0\\
55.94	0\\
55.95	0\\
55.96	0\\
55.97	0\\
55.98	0\\
55.99	0\\
56	0\\
56.01	0\\
56.02	0\\
56.03	0\\
56.04	0\\
56.05	0\\
56.06	0\\
56.07	0\\
56.08	0\\
56.09	0\\
56.1	0\\
56.11	0\\
56.12	0\\
56.13	0\\
56.14	0\\
56.15	0\\
56.16	0\\
56.17	0\\
56.18	0\\
56.19	0\\
56.2	0\\
56.21	0\\
56.22	0\\
56.23	0\\
56.24	0\\
56.25	0\\
56.26	0\\
56.27	0\\
56.28	0\\
56.29	0\\
56.3	0\\
56.31	0\\
56.32	0\\
56.33	0\\
56.34	0\\
56.35	0\\
56.36	0\\
56.37	0\\
56.38	0\\
56.39	0\\
56.4	0\\
56.41	0\\
56.42	0\\
56.43	0\\
56.44	0\\
56.45	0\\
56.46	0\\
56.47	0\\
56.48	0\\
56.49	0\\
56.5	0\\
56.51	0\\
56.52	0\\
56.53	0\\
56.54	0\\
56.55	0\\
56.56	0\\
56.57	0\\
56.58	0\\
56.59	0\\
56.6	0\\
56.61	0\\
56.62	0\\
56.63	0\\
56.64	0\\
56.65	0\\
56.66	0\\
56.67	0\\
56.68	0\\
56.69	0\\
56.7	0\\
56.71	0\\
56.72	0\\
56.73	0\\
56.74	0\\
56.75	0\\
56.76	0\\
56.77	0\\
56.78	0\\
56.79	0\\
56.8	0\\
56.81	0\\
56.82	0\\
56.83	0\\
56.84	0\\
56.85	0\\
56.86	0\\
56.87	0\\
56.88	0\\
56.89	0\\
56.9	0\\
56.91	0\\
56.92	0\\
56.93	0\\
56.94	0\\
56.95	0\\
56.96	0\\
56.97	0\\
56.98	0\\
56.99	0\\
57	0\\
57.01	0\\
57.02	0\\
57.03	0\\
57.04	0\\
57.05	0\\
57.06	0\\
57.07	0\\
57.08	0\\
57.09	0\\
57.1	0\\
57.11	0\\
57.12	0\\
57.13	0\\
57.14	0\\
57.15	0\\
57.16	0\\
57.17	0\\
57.18	0\\
57.19	0\\
57.2	0\\
57.21	0\\
57.22	0\\
57.23	0\\
57.24	0\\
57.25	0\\
57.26	0\\
57.27	0\\
57.28	0\\
57.29	0\\
57.3	0\\
57.31	0\\
57.32	0\\
57.33	0\\
57.34	0\\
57.35	0\\
57.36	0\\
57.37	0\\
57.38	0\\
57.39	0\\
57.4	0\\
57.41	0\\
57.42	0\\
57.43	0\\
57.44	0\\
57.45	0\\
57.46	0\\
57.47	0\\
57.48	0\\
57.49	0\\
57.5	0\\
57.51	0\\
57.52	0\\
57.53	0\\
57.54	0\\
57.55	0\\
57.56	0\\
57.57	0\\
57.58	0\\
57.59	0\\
57.6	0\\
57.61	0\\
57.62	0\\
57.63	0\\
57.64	0\\
57.65	0\\
57.66	0\\
57.67	0\\
57.68	0\\
57.69	0\\
57.7	0\\
57.71	0\\
57.72	0\\
57.73	0\\
57.74	0\\
57.75	0\\
57.76	0\\
57.77	0\\
57.78	0\\
57.79	0\\
57.8	0\\
57.81	0\\
57.82	0\\
57.83	0\\
57.84	0\\
57.85	0\\
57.86	0\\
57.87	0\\
57.88	0\\
57.89	0\\
57.9	0\\
57.91	0\\
57.92	0\\
57.93	0\\
57.94	0\\
57.95	0\\
57.96	0\\
57.97	0\\
57.98	0\\
57.99	0\\
58	0\\
58.01	0\\
58.02	0\\
58.03	0\\
58.04	0\\
58.05	0\\
58.06	0\\
58.07	0\\
58.08	0\\
58.09	0\\
58.1	0\\
58.11	0\\
58.12	0\\
58.13	0\\
58.14	0\\
58.15	0\\
58.16	0\\
58.17	0\\
58.18	0\\
58.19	0\\
58.2	0\\
58.21	0\\
58.22	0\\
58.23	0\\
58.24	0\\
58.25	0\\
58.26	0\\
58.27	0\\
58.28	0\\
58.29	0\\
58.3	0\\
58.31	0\\
58.32	0\\
58.33	0\\
58.34	0\\
58.35	0\\
58.36	0\\
58.37	0\\
58.38	0\\
58.39	0\\
58.4	0\\
58.41	0\\
58.42	0\\
58.43	0\\
58.44	0\\
58.45	0\\
58.46	0\\
58.47	0\\
58.48	0\\
58.49	0\\
58.5	0\\
58.51	0\\
58.52	0\\
58.53	0\\
58.54	0\\
58.55	0\\
58.56	0\\
58.57	0\\
58.58	0\\
58.59	0\\
58.6	0\\
58.61	0\\
58.62	0\\
58.63	0\\
58.64	0\\
58.65	0\\
58.66	0\\
58.67	0\\
58.68	0\\
58.69	0\\
58.7	0\\
58.71	0\\
58.72	0\\
58.73	0\\
58.74	0\\
58.75	0\\
58.76	0\\
58.77	0\\
58.78	0\\
58.79	0\\
58.8	0\\
58.81	0\\
58.82	0\\
58.83	0\\
58.84	0\\
58.85	0\\
58.86	0\\
58.87	0\\
58.88	0\\
58.89	0\\
58.9	0\\
58.91	0\\
58.92	0\\
58.93	0\\
58.94	0\\
58.95	0\\
58.96	0\\
58.97	0\\
58.98	0\\
58.99	0\\
59	0\\
59.01	0\\
59.02	0\\
59.03	0\\
59.04	0\\
59.05	0\\
59.06	0\\
59.07	0\\
59.08	0\\
59.09	0\\
59.1	0\\
59.11	0\\
59.12	0\\
59.13	0\\
59.14	0\\
59.15	0\\
59.16	0\\
59.17	0\\
59.18	0\\
59.19	0\\
59.2	0\\
59.21	0\\
59.22	0\\
59.23	0\\
59.24	0\\
59.25	0\\
59.26	0\\
59.27	0\\
59.28	0\\
59.29	0\\
59.3	0\\
59.31	0\\
59.32	0\\
59.33	0\\
59.34	0\\
59.35	0\\
59.36	0\\
59.37	0\\
59.38	0\\
59.39	0\\
59.4	0\\
59.41	0\\
59.42	0\\
59.43	0\\
59.44	0\\
59.45	0\\
59.46	0\\
59.47	0\\
59.48	0\\
59.49	0\\
59.5	0\\
59.51	0\\
59.52	0\\
59.53	0\\
59.54	0\\
59.55	0\\
59.56	0\\
59.57	0\\
59.58	0\\
59.59	0\\
59.6	0\\
59.61	0\\
59.62	0\\
59.63	0\\
59.64	0\\
59.65	0\\
59.66	0\\
59.67	0\\
59.68	0\\
59.69	0\\
59.7	0\\
59.71	0\\
59.72	0\\
59.73	0\\
59.74	0\\
59.75	0\\
59.76	0\\
59.77	0\\
59.78	0\\
59.79	0\\
59.8	0\\
59.81	0\\
59.82	0\\
59.83	0\\
59.84	0\\
59.85	0\\
59.86	0\\
59.87	0\\
59.88	0\\
59.89	0\\
59.9	0\\
59.91	0\\
59.92	0\\
59.93	0\\
59.94	0\\
59.95	0\\
59.96	0\\
59.97	0\\
59.98	0\\
59.99	0\\
60	0\\
60.01	0\\
60.02	0\\
60.03	0\\
60.04	0\\
60.05	0\\
60.06	0\\
60.07	0\\
60.08	0\\
60.09	0\\
60.1	0\\
60.11	0\\
60.12	0\\
60.13	0\\
60.14	0\\
60.15	0\\
60.16	0\\
60.17	0\\
60.18	0\\
60.19	0\\
60.2	0\\
60.21	0\\
60.22	0\\
60.23	0\\
60.24	0\\
60.25	0\\
60.26	0\\
60.27	0\\
60.28	0\\
60.29	0\\
60.3	0\\
60.31	0\\
60.32	0\\
60.33	0\\
60.34	0\\
60.35	0\\
60.36	0\\
60.37	0\\
60.38	0\\
60.39	0\\
60.4	0\\
60.41	0\\
60.42	0\\
60.43	0\\
60.44	0\\
60.45	0\\
60.46	0\\
60.47	0\\
60.48	0\\
60.49	0\\
60.5	0\\
60.51	0\\
60.52	0\\
60.53	0\\
60.54	0\\
60.55	0\\
60.56	0\\
60.57	0\\
60.58	0\\
60.59	0\\
60.6	0\\
60.61	0\\
60.62	0\\
60.63	0\\
60.64	0\\
60.65	0\\
60.66	0\\
60.67	0\\
60.68	0\\
60.69	0\\
60.7	0\\
60.71	0\\
60.72	0\\
60.73	0\\
60.74	0\\
60.75	0\\
60.76	0\\
60.77	0\\
60.78	0\\
60.79	0\\
60.8	0\\
60.81	0\\
60.82	0\\
60.83	0\\
60.84	0\\
60.85	0\\
60.86	0\\
60.87	0\\
60.88	0\\
60.89	0\\
60.9	0\\
60.91	0\\
60.92	0\\
60.93	0\\
60.94	0\\
60.95	0\\
60.96	0\\
60.97	0\\
60.98	0\\
60.99	0\\
61	0\\
61.01	0\\
61.02	0\\
61.03	0\\
61.04	0\\
61.05	0\\
61.06	0\\
61.07	0\\
61.08	0\\
61.09	0\\
61.1	0\\
61.11	0\\
61.12	0\\
61.13	0\\
61.14	0\\
61.15	0\\
61.16	0\\
61.17	0\\
61.18	0\\
61.19	0\\
61.2	0\\
61.21	0\\
61.22	0\\
61.23	0\\
61.24	0\\
61.25	0\\
61.26	0\\
61.27	0\\
61.28	0\\
61.29	0\\
61.3	0\\
61.31	0\\
61.32	0\\
61.33	0\\
61.34	0\\
61.35	0\\
61.36	0\\
61.37	0\\
61.38	0\\
61.39	0\\
61.4	0\\
61.41	0\\
61.42	0\\
61.43	0\\
61.44	0\\
61.45	0\\
61.46	0\\
61.47	0\\
61.48	0\\
61.49	0\\
61.5	0\\
61.51	0\\
61.52	0\\
61.53	0\\
61.54	0\\
61.55	0\\
61.56	0\\
61.57	0\\
61.58	0\\
61.59	0\\
61.6	0\\
61.61	0\\
61.62	0\\
61.63	0\\
61.64	0\\
61.65	0\\
61.66	0\\
61.67	0\\
61.68	0\\
61.69	0\\
61.7	0\\
61.71	0\\
61.72	0\\
61.73	0\\
61.74	0\\
61.75	0\\
61.76	0\\
61.77	0\\
61.78	0\\
61.79	0\\
61.8	0\\
61.81	0\\
61.82	0\\
61.83	0\\
61.84	0\\
61.85	0\\
61.86	0\\
61.87	0\\
61.88	0\\
61.89	0\\
61.9	0\\
61.91	0\\
61.92	0\\
61.93	0\\
61.94	0\\
61.95	0\\
61.96	0\\
61.97	0\\
61.98	0\\
61.99	0\\
62	0\\
62.01	0\\
62.02	0\\
62.03	0\\
62.04	0\\
62.05	0\\
62.06	0\\
62.07	0\\
62.08	0\\
62.09	0\\
62.1	0\\
62.11	0\\
62.12	0\\
62.13	0\\
62.14	0\\
62.15	0\\
62.16	0\\
62.17	0\\
62.18	0\\
62.19	0\\
62.2	0\\
62.21	0\\
62.22	0\\
62.23	0\\
62.24	0\\
62.25	0\\
62.26	0\\
62.27	0\\
62.28	0\\
62.29	0\\
62.3	0\\
62.31	0\\
62.32	0\\
62.33	0\\
62.34	0\\
62.35	0\\
62.36	0\\
62.37	0\\
62.38	0\\
62.39	0\\
62.4	0\\
62.41	0\\
62.42	0\\
62.43	0\\
62.44	0\\
62.45	0\\
62.46	0\\
62.47	0\\
62.48	0\\
62.49	0\\
62.5	0\\
62.51	0\\
62.52	0\\
62.53	0\\
62.54	0\\
62.55	0\\
62.56	0\\
62.57	0\\
62.58	0\\
62.59	0\\
62.6	0\\
62.61	0\\
62.62	0\\
62.63	0\\
62.64	0\\
62.65	0\\
62.66	0\\
62.67	0\\
62.68	0\\
62.69	0\\
62.7	0\\
62.71	0\\
62.72	0\\
62.73	0\\
62.74	0\\
62.75	0\\
62.76	0\\
62.77	0\\
62.78	0\\
62.79	0\\
62.8	0\\
62.81	0\\
62.82	0\\
62.83	0\\
62.84	0\\
62.85	0\\
62.86	0\\
62.87	0\\
62.88	0\\
62.89	0\\
62.9	0\\
62.91	0\\
62.92	0\\
62.93	0\\
62.94	0\\
62.95	0\\
62.96	0\\
62.97	0\\
62.98	0\\
62.99	0\\
63	0\\
63.01	0\\
63.02	0\\
63.03	0\\
63.04	0\\
63.05	0\\
63.06	0\\
63.07	0\\
63.08	0\\
63.09	0\\
63.1	0\\
63.11	0\\
63.12	0\\
63.13	0\\
63.14	0\\
63.15	0\\
63.16	0\\
63.17	0\\
63.18	0\\
63.19	0\\
63.2	0\\
63.21	0\\
63.22	0\\
63.23	0\\
63.24	0\\
63.25	0\\
63.26	0\\
63.27	0\\
63.28	0\\
63.29	0\\
63.3	0\\
63.31	0\\
63.32	0\\
63.33	0\\
63.34	0\\
63.35	0\\
63.36	0\\
63.37	0\\
63.38	0\\
63.39	0\\
63.4	0\\
63.41	0\\
63.42	0\\
63.43	0\\
63.44	0\\
63.45	0\\
63.46	0\\
63.47	0\\
63.48	0\\
63.49	0\\
63.5	0\\
63.51	0\\
63.52	0\\
63.53	0\\
63.54	0\\
63.55	0\\
63.56	0\\
63.57	0\\
63.58	0\\
63.59	0\\
63.6	0\\
63.61	0\\
63.62	0\\
63.63	0\\
63.64	0\\
63.65	0\\
63.66	0\\
63.67	0\\
63.68	0\\
63.69	0\\
63.7	0\\
63.71	0\\
63.72	0\\
63.73	0\\
63.74	0\\
63.75	0\\
63.76	0\\
63.77	0\\
63.78	0\\
63.79	0\\
63.8	0\\
63.81	0\\
63.82	0\\
63.83	0\\
63.84	0\\
63.85	0\\
63.86	0\\
63.87	0\\
63.88	0\\
63.89	0\\
63.9	0\\
63.91	0\\
63.92	0\\
63.93	0\\
63.94	0\\
63.95	0\\
63.96	0\\
63.97	0\\
63.98	0\\
63.99	0\\
64	0\\
64.01	0\\
64.02	0\\
64.03	0\\
64.04	0\\
64.05	0\\
64.06	0\\
64.07	0\\
64.08	0\\
64.09	0\\
64.1	0\\
64.11	0\\
64.12	0\\
64.13	0\\
64.14	0\\
64.15	0\\
64.16	0\\
64.17	0\\
64.18	0\\
64.19	0\\
64.2	0\\
64.21	0\\
64.22	0\\
64.23	0\\
64.24	0\\
64.25	0\\
64.26	0\\
64.27	0\\
64.28	0\\
64.29	0\\
64.3	0\\
64.31	0\\
64.32	0\\
64.33	0\\
64.34	0\\
64.35	0\\
64.36	0\\
64.37	0\\
64.38	0\\
64.39	0\\
64.4	0\\
64.41	0\\
64.42	0\\
64.43	0\\
64.44	0\\
64.45	0\\
64.46	0\\
64.47	0\\
64.48	0\\
64.49	0\\
64.5	0\\
64.51	0\\
64.52	0\\
64.53	0\\
64.54	0\\
64.55	0\\
64.56	0\\
64.57	0\\
64.58	0\\
64.59	0\\
64.6	0\\
64.61	0\\
64.62	0\\
64.63	0\\
64.64	0\\
64.65	0\\
64.66	0\\
64.67	0\\
64.68	0\\
64.69	0\\
64.7	0\\
64.71	0\\
64.72	0\\
64.73	0\\
64.74	0\\
64.75	0\\
64.76	0\\
64.77	0\\
64.78	0\\
64.79	0\\
64.8	0\\
64.81	0\\
64.82	0\\
64.83	0\\
64.84	0\\
64.85	0\\
64.86	0\\
64.87	0\\
64.88	0\\
64.89	0\\
64.9	0\\
64.91	0\\
64.92	0\\
64.93	0\\
64.94	0\\
64.95	0\\
64.96	0\\
64.97	0\\
64.98	0\\
64.99	0\\
65	0\\
65.01	0\\
65.02	0\\
65.03	0\\
65.04	0\\
65.05	0\\
65.06	0\\
65.07	0\\
65.08	0\\
65.09	0\\
65.1	0\\
65.11	0\\
65.12	0\\
65.13	0\\
65.14	0\\
65.15	0\\
65.16	0\\
65.17	0\\
65.18	0\\
65.19	0\\
65.2	0\\
65.21	0\\
65.22	0\\
65.23	0\\
65.24	0\\
65.25	0\\
65.26	0\\
65.27	0\\
65.28	0\\
65.29	0\\
65.3	0\\
65.31	0\\
65.32	0\\
65.33	0\\
65.34	0\\
65.35	0\\
65.36	0\\
65.37	0\\
65.38	0\\
65.39	0\\
65.4	0\\
65.41	0\\
65.42	0\\
65.43	0\\
65.44	0\\
65.45	0\\
65.46	0\\
65.47	0\\
65.48	0\\
65.49	0\\
65.5	0\\
65.51	0\\
65.52	0\\
65.53	0\\
65.54	0\\
65.55	0\\
65.56	0\\
65.57	0\\
65.58	0\\
65.59	0\\
65.6	0\\
65.61	0\\
65.62	0\\
65.63	0\\
65.64	0\\
65.65	0\\
65.66	0\\
65.67	0\\
65.68	0\\
65.69	0\\
65.7	0\\
65.71	0\\
65.72	0\\
65.73	0\\
65.74	0\\
65.75	0\\
65.76	0\\
65.77	0\\
65.78	0\\
65.79	0\\
65.8	0\\
65.81	0\\
65.82	0\\
65.83	0\\
65.84	0\\
65.85	0\\
65.86	0\\
65.87	0\\
65.88	0\\
65.89	0\\
65.9	0\\
65.91	0\\
65.92	0\\
65.93	0\\
65.94	0\\
65.95	0\\
65.96	0\\
65.97	0\\
65.98	0\\
65.99	0\\
66	0\\
66.01	0\\
66.02	0\\
66.03	0\\
66.04	0\\
66.05	0\\
66.06	0\\
66.07	0\\
66.08	0\\
66.09	0\\
66.1	0\\
66.11	0\\
66.12	0\\
66.13	0\\
66.14	0\\
66.15	0\\
66.16	0\\
66.17	0\\
66.18	0\\
66.19	0\\
66.2	0\\
66.21	0\\
66.22	0\\
66.23	0\\
66.24	0\\
66.25	0\\
66.26	0\\
66.27	0\\
66.28	0\\
66.29	0\\
66.3	0\\
66.31	0\\
66.32	0\\
66.33	0\\
66.34	0\\
66.35	0\\
66.36	0\\
66.37	0\\
66.38	0\\
66.39	0\\
66.4	0\\
66.41	0\\
66.42	0\\
66.43	0\\
66.44	0\\
66.45	0\\
66.46	0\\
66.47	0\\
66.48	0\\
66.49	0\\
66.5	0\\
66.51	0\\
66.52	0\\
66.53	0\\
66.54	0\\
66.55	0\\
66.56	0\\
66.57	0\\
66.58	0\\
66.59	0\\
66.6	0\\
66.61	0\\
66.62	0\\
66.63	0\\
66.64	0\\
66.65	0\\
66.66	0\\
66.67	0\\
66.68	0\\
66.69	0\\
66.7	0\\
66.71	0\\
66.72	0\\
66.73	0\\
66.74	0\\
66.75	0\\
66.76	0\\
66.77	0\\
66.78	0\\
66.79	0\\
66.8	0\\
66.81	0\\
66.82	0\\
66.83	0\\
66.84	0\\
66.85	0\\
66.86	0\\
66.87	0\\
66.88	0\\
66.89	0\\
66.9	0\\
66.91	0\\
66.92	0\\
66.93	0\\
66.94	0\\
66.95	0\\
66.96	0\\
66.97	0\\
66.98	0\\
66.99	0\\
67	0\\
67.01	0\\
67.02	0\\
67.03	0\\
67.04	0\\
67.05	0\\
67.06	0\\
67.07	0\\
67.08	0\\
67.09	0\\
67.1	0\\
67.11	0\\
67.12	0\\
67.13	0\\
67.14	0\\
67.15	0\\
67.16	0\\
67.17	0\\
67.18	0\\
67.19	0\\
67.2	0\\
67.21	0\\
67.22	0\\
67.23	0\\
67.24	0\\
67.25	0\\
67.26	0\\
67.27	0\\
67.28	0\\
67.29	0\\
67.3	0\\
67.31	0\\
67.32	0\\
67.33	0\\
67.34	0\\
67.35	0\\
67.36	0\\
67.37	0\\
67.38	0\\
67.39	0\\
67.4	0\\
67.41	0\\
67.42	0\\
67.43	0\\
67.44	0\\
67.45	0\\
67.46	0\\
67.47	0\\
67.48	0\\
67.49	0\\
67.5	0\\
67.51	0\\
67.52	0\\
67.53	0\\
67.54	0\\
67.55	0\\
67.56	0\\
67.57	0\\
67.58	0\\
67.59	0\\
67.6	0\\
67.61	0\\
67.62	0\\
67.63	0\\
67.64	0\\
67.65	0\\
67.66	0\\
67.67	0\\
67.68	0\\
67.69	0\\
67.7	0\\
67.71	0\\
67.72	0\\
67.73	0\\
67.74	0\\
67.75	0\\
67.76	0\\
67.77	0\\
67.78	0\\
67.79	0\\
67.8	0\\
67.81	0\\
67.82	0\\
67.83	0\\
67.84	0\\
67.85	0\\
67.86	0\\
67.87	0\\
67.88	0\\
67.89	0\\
67.9	0\\
67.91	0\\
67.92	0\\
67.93	0\\
67.94	0\\
67.95	0\\
67.96	0\\
67.97	0\\
67.98	0\\
67.99	0\\
68	0\\
68.01	0\\
68.02	0\\
68.03	0\\
68.04	0\\
68.05	0\\
68.06	0\\
68.07	0\\
68.08	0\\
68.09	0\\
68.1	0\\
68.11	0\\
68.12	0\\
68.13	0\\
68.14	0\\
68.15	0\\
68.16	0\\
68.17	0\\
68.18	0\\
68.19	0\\
68.2	0\\
68.21	0\\
68.22	0\\
68.23	0\\
68.24	0\\
68.25	0\\
68.26	0\\
68.27	0\\
68.28	0\\
68.29	0\\
68.3	0\\
68.31	0\\
68.32	0\\
68.33	0\\
68.34	0\\
68.35	0\\
68.36	0\\
68.37	0\\
68.38	0\\
68.39	0\\
68.4	0\\
68.41	0\\
68.42	0\\
68.43	0\\
68.44	0\\
68.45	0\\
68.46	0\\
68.47	0\\
68.48	0\\
68.49	0\\
68.5	0\\
68.51	0\\
68.52	0\\
68.53	0\\
68.54	0\\
68.55	0\\
68.56	0\\
68.57	0\\
68.58	0\\
68.59	0\\
68.6	0\\
68.61	0\\
68.62	0\\
68.63	0\\
68.64	0\\
68.65	0\\
68.66	0\\
68.67	0\\
68.68	0\\
68.69	0\\
68.7	0\\
68.71	0\\
68.72	0\\
68.73	0\\
68.74	0\\
68.75	0\\
68.76	0\\
68.77	0\\
68.78	0\\
68.79	0\\
68.8	0\\
68.81	0\\
68.82	0\\
68.83	0\\
68.84	0\\
68.85	0\\
68.86	0\\
68.87	0\\
68.88	0\\
68.89	0\\
68.9	0\\
68.91	0\\
68.92	0\\
68.93	0\\
68.94	0\\
68.95	0\\
68.96	0\\
68.97	0\\
68.98	0\\
68.99	0\\
69	0\\
69.01	0\\
69.02	0\\
69.03	0\\
69.04	0\\
69.05	0\\
69.06	0\\
69.07	0\\
69.08	0\\
69.09	0\\
69.1	0\\
69.11	0\\
69.12	0\\
69.13	0\\
69.14	0\\
69.15	0\\
69.16	0\\
69.17	0\\
69.18	0\\
69.19	0\\
69.2	0\\
69.21	0\\
69.22	0\\
69.23	0\\
69.24	0\\
69.25	0\\
69.26	0\\
69.27	0\\
69.28	0\\
69.29	0\\
69.3	0\\
69.31	0\\
69.32	0\\
69.33	0\\
69.34	0\\
69.35	0\\
69.36	0\\
69.37	0\\
69.38	0\\
69.39	0\\
69.4	0\\
69.41	0\\
69.42	0\\
69.43	0\\
69.44	0\\
69.45	0\\
69.46	0\\
69.47	0\\
69.48	0\\
69.49	0\\
69.5	0\\
69.51	0\\
69.52	0\\
69.53	0\\
69.54	0\\
69.55	0\\
69.56	0\\
69.57	0\\
69.58	0\\
69.59	0\\
69.6	0\\
69.61	0\\
69.62	0\\
69.63	0\\
69.64	0\\
69.65	0\\
69.66	0\\
69.67	0\\
69.68	0\\
69.69	0\\
69.7	0\\
69.71	0\\
69.72	0\\
69.73	0\\
69.74	0\\
69.75	0\\
69.76	0\\
69.77	0\\
69.78	0\\
69.79	0\\
69.8	0\\
69.81	0\\
69.82	0\\
69.83	0\\
69.84	0\\
69.85	0\\
69.86	0\\
69.87	0\\
69.88	0\\
69.89	0\\
69.9	0\\
69.91	0\\
69.92	0\\
69.93	0\\
69.94	0\\
69.95	0\\
69.96	0\\
69.97	0\\
69.98	0\\
69.99	0\\
70	0\\
70.01	0\\
70.02	0\\
70.03	0\\
70.04	0\\
70.05	0\\
70.06	0\\
70.07	0\\
70.08	0\\
70.09	0\\
70.1	0\\
70.11	0\\
70.12	0\\
70.13	0\\
70.14	0\\
70.15	0\\
70.16	0\\
70.17	0\\
70.18	0\\
70.19	0\\
70.2	0\\
70.21	0\\
70.22	0\\
70.23	0\\
70.24	0\\
70.25	0\\
70.26	0\\
70.27	0\\
70.28	0\\
70.29	0\\
70.3	0\\
70.31	0\\
70.32	0\\
70.33	0\\
70.34	0\\
70.35	0\\
70.36	0\\
70.37	0\\
70.38	0\\
70.39	0\\
70.4	0\\
70.41	0\\
70.42	0\\
70.43	0\\
70.44	0\\
70.45	0\\
70.46	0\\
70.47	0\\
70.48	0\\
70.49	0\\
70.5	0\\
70.51	0\\
70.52	0\\
70.53	0\\
70.54	0\\
70.55	0\\
70.56	0\\
70.57	0\\
70.58	0\\
70.59	0\\
70.6	0\\
70.61	0\\
70.62	0\\
70.63	0\\
70.64	0\\
70.65	0\\
70.66	0\\
70.67	0\\
70.68	0\\
70.69	0\\
70.7	0\\
70.71	0\\
70.72	0\\
70.73	0\\
70.74	0\\
70.75	0\\
70.76	0\\
70.77	0\\
70.78	0\\
70.79	0\\
70.8	0\\
70.81	0\\
70.82	0\\
70.83	0\\
70.84	0\\
70.85	0\\
70.86	0\\
70.87	0\\
70.88	0\\
70.89	0\\
70.9	0\\
70.91	0\\
70.92	0\\
70.93	0\\
70.94	0\\
70.95	0\\
70.96	0\\
70.97	0\\
70.98	0\\
70.99	0\\
71	0\\
71.01	0\\
71.02	0\\
71.03	0\\
71.04	0\\
71.05	0\\
71.06	0\\
71.07	0\\
71.08	0\\
71.09	0\\
71.1	0\\
71.11	0\\
71.12	0\\
71.13	0\\
71.14	0\\
71.15	0\\
71.16	0\\
71.17	0\\
71.18	0\\
71.19	0\\
71.2	0\\
71.21	0\\
71.22	0\\
71.23	0\\
71.24	0\\
71.25	0\\
71.26	0\\
71.27	0\\
71.28	0\\
71.29	0\\
71.3	0\\
71.31	0\\
71.32	0\\
71.33	0\\
71.34	0\\
71.35	0\\
71.36	0\\
71.37	0\\
71.38	0\\
71.39	0\\
71.4	0\\
71.41	0\\
71.42	0\\
71.43	0\\
71.44	0\\
71.45	0\\
71.46	0\\
71.47	0\\
71.48	0\\
71.49	0\\
71.5	0\\
71.51	0\\
71.52	0\\
71.53	0\\
71.54	0\\
71.55	0\\
71.56	0\\
71.57	0\\
71.58	0\\
71.59	0\\
71.6	0\\
71.61	0\\
71.62	0\\
71.63	0\\
71.64	0\\
71.65	0\\
71.66	0\\
71.67	0\\
71.68	0\\
71.69	0\\
71.7	0\\
71.71	0\\
71.72	0\\
71.73	0\\
71.74	0\\
71.75	0\\
71.76	0\\
71.77	0\\
71.78	0\\
71.79	0\\
71.8	0\\
71.81	0\\
71.82	0\\
71.83	0\\
71.84	0\\
71.85	0\\
71.86	0\\
71.87	0\\
71.88	0\\
71.89	0\\
71.9	0\\
71.91	0\\
71.92	0\\
71.93	0\\
71.94	0\\
71.95	0\\
71.96	0\\
71.97	0\\
71.98	0\\
71.99	0\\
72	0\\
72.01	0\\
72.02	0\\
72.03	0\\
72.04	0\\
72.05	0\\
72.06	0\\
72.07	0\\
72.08	0\\
72.09	0\\
72.1	0\\
72.11	0\\
72.12	0\\
72.13	0\\
72.14	0\\
72.15	0\\
72.16	0\\
72.17	0\\
72.18	0\\
72.19	0\\
72.2	0\\
72.21	0\\
72.22	0\\
72.23	0\\
72.24	0\\
72.25	0\\
72.26	0\\
72.27	0\\
72.28	0\\
72.29	0\\
72.3	0\\
72.31	0\\
72.32	0\\
72.33	0\\
72.34	0\\
72.35	0\\
72.36	0\\
72.37	0\\
72.38	0\\
72.39	0\\
72.4	0\\
72.41	0\\
72.42	0\\
72.43	0\\
72.44	0\\
72.45	0\\
72.46	0\\
72.47	0\\
72.48	0\\
72.49	0\\
72.5	0\\
72.51	0\\
72.52	0\\
72.53	0\\
72.54	0\\
72.55	0\\
72.56	0\\
72.57	0\\
72.58	0\\
72.59	0\\
72.6	0\\
72.61	0\\
72.62	0\\
72.63	0\\
72.64	0\\
72.65	0\\
72.66	0\\
72.67	0\\
72.68	0\\
72.69	0\\
72.7	0\\
72.71	0\\
72.72	0\\
72.73	0\\
72.74	0\\
72.75	0\\
72.76	0\\
72.77	0\\
72.78	0\\
72.79	0\\
72.8	0\\
72.81	0\\
72.82	0\\
72.83	0\\
72.84	0\\
72.85	0\\
72.86	0\\
72.87	0\\
72.88	0\\
72.89	0\\
72.9	0\\
72.91	0\\
72.92	0\\
72.93	0\\
72.94	0\\
72.95	0\\
72.96	0\\
72.97	0\\
72.98	0\\
72.99	0\\
73	0\\
73.01	0\\
73.02	0\\
73.03	0\\
73.04	0\\
73.05	0\\
73.06	0\\
73.07	0\\
73.08	0\\
73.09	0\\
73.1	0\\
73.11	0\\
73.12	0\\
73.13	0\\
73.14	0\\
73.15	0\\
73.16	0\\
73.17	0\\
73.18	0\\
73.19	0\\
73.2	0\\
73.21	0\\
73.22	0\\
73.23	0\\
73.24	0\\
73.25	0\\
73.26	0\\
73.27	0\\
73.28	0\\
73.29	0\\
73.3	0\\
73.31	0\\
73.32	0\\
73.33	0\\
73.34	0\\
73.35	0\\
73.36	0\\
73.37	0\\
73.38	0\\
73.39	0\\
73.4	0\\
73.41	0\\
73.42	0\\
73.43	0\\
73.44	0\\
73.45	0\\
73.46	0\\
73.47	0\\
73.48	0\\
73.49	0\\
73.5	0\\
73.51	0\\
73.52	0\\
73.53	0\\
73.54	0\\
73.55	0\\
73.56	0\\
73.57	0\\
73.58	0\\
73.59	0\\
73.6	0\\
73.61	0\\
73.62	0\\
73.63	0\\
73.64	0\\
73.65	0\\
73.66	0\\
73.67	0\\
73.68	0\\
73.69	0\\
73.7	0\\
73.71	0\\
73.72	0\\
73.73	0\\
73.74	0\\
73.75	0\\
73.76	0\\
73.77	0\\
73.78	0\\
73.79	0\\
73.8	0\\
73.81	0\\
73.82	0\\
73.83	0\\
73.84	0\\
73.85	0\\
73.86	0\\
73.87	0\\
73.88	0\\
73.89	0\\
73.9	0\\
73.91	0\\
73.92	0\\
73.93	0\\
73.94	0\\
73.95	0\\
73.96	0\\
73.97	0\\
73.98	0\\
73.99	0\\
74	0\\
74.01	0\\
74.02	0\\
74.03	0\\
74.04	0\\
74.05	0\\
74.06	0\\
74.07	0\\
74.08	0\\
74.09	0\\
74.1	0\\
74.11	0\\
74.12	0\\
74.13	0\\
74.14	0\\
74.15	0\\
74.16	0\\
74.17	0\\
74.18	0\\
74.19	0\\
74.2	0\\
74.21	0\\
74.22	0\\
74.23	0\\
74.24	0\\
74.25	0\\
74.26	0\\
74.27	0\\
74.28	0\\
74.29	0\\
74.3	0\\
74.31	0\\
74.32	0\\
74.33	0\\
74.34	0\\
74.35	0\\
74.36	0\\
74.37	0\\
74.38	0\\
74.39	0\\
74.4	0\\
74.41	0\\
74.42	0\\
74.43	0\\
74.44	0\\
74.45	0\\
74.46	0\\
74.47	0\\
74.48	0\\
74.49	0\\
74.5	0\\
74.51	0\\
74.52	0\\
74.53	0\\
74.54	0\\
74.55	0\\
74.56	0\\
74.57	0\\
74.58	0\\
74.59	0\\
74.6	0\\
74.61	0\\
74.62	0\\
74.63	0\\
74.64	0\\
74.65	0\\
74.66	0\\
74.67	0\\
74.68	0\\
74.69	0\\
74.7	0\\
74.71	0\\
74.72	0\\
74.73	0\\
74.74	0\\
74.75	0\\
74.76	0\\
74.77	0\\
74.78	0\\
74.79	0\\
74.8	0\\
74.81	0\\
74.82	0\\
74.83	0\\
74.84	0\\
74.85	0\\
74.86	0\\
74.87	0\\
74.88	0\\
74.89	0\\
74.9	0\\
74.91	0\\
74.92	0\\
74.93	0\\
74.94	0\\
74.95	0\\
74.96	0\\
74.97	0\\
74.98	0\\
74.99	0\\
75	0\\
75.01	0\\
75.02	0\\
75.03	0\\
75.04	0\\
75.05	0\\
75.06	0\\
75.07	0\\
75.08	0\\
75.09	0\\
75.1	0\\
75.11	0\\
75.12	0\\
75.13	0\\
75.14	0\\
75.15	0\\
75.16	0\\
75.17	0\\
75.18	0\\
75.19	0\\
75.2	0\\
75.21	0\\
75.22	0\\
75.23	0\\
75.24	0\\
75.25	0\\
75.26	0\\
75.27	0\\
75.28	0\\
75.29	0\\
75.3	0\\
75.31	0\\
75.32	0\\
75.33	0\\
75.34	0\\
75.35	0\\
75.36	0\\
75.37	0\\
75.38	0\\
75.39	0\\
75.4	0\\
75.41	0\\
75.42	0\\
75.43	0\\
75.44	0\\
75.45	0\\
75.46	0\\
75.47	0\\
75.48	0\\
75.49	0\\
75.5	0\\
75.51	0\\
75.52	0\\
75.53	0\\
75.54	0\\
75.55	0\\
75.56	0\\
75.57	0\\
75.58	0\\
75.59	0\\
75.6	0\\
75.61	0\\
75.62	0\\
75.63	0\\
75.64	0\\
75.65	0\\
75.66	0\\
75.67	0\\
75.68	0\\
75.69	0\\
75.7	0\\
75.71	0\\
75.72	0\\
75.73	0\\
75.74	0\\
75.75	0\\
75.76	0\\
75.77	0\\
75.78	0\\
75.79	0\\
75.8	0\\
75.81	0\\
75.82	0\\
75.83	0\\
75.84	0\\
75.85	0\\
75.86	0\\
75.87	0\\
75.88	0\\
75.89	0\\
75.9	0\\
75.91	0\\
75.92	0\\
75.93	0\\
75.94	0\\
75.95	0\\
75.96	0\\
75.97	0\\
75.98	0\\
75.99	0\\
76	0\\
76.01	0\\
76.02	0\\
76.03	0\\
76.04	0\\
76.05	0\\
76.06	0\\
76.07	0\\
76.08	0\\
76.09	0\\
76.1	0\\
76.11	0\\
76.12	0\\
76.13	0\\
76.14	0\\
76.15	0\\
76.16	0\\
76.17	0\\
76.18	0\\
76.19	0\\
76.2	0\\
76.21	0\\
76.22	0\\
76.23	0\\
76.24	0\\
76.25	0\\
76.26	0\\
76.27	0\\
76.28	0\\
76.29	0\\
76.3	0\\
76.31	0\\
76.32	0\\
76.33	0\\
76.34	0\\
76.35	0\\
76.36	0\\
76.37	0\\
76.38	0\\
76.39	0\\
76.4	0\\
76.41	0\\
76.42	0\\
76.43	0\\
76.44	0\\
76.45	0\\
76.46	0\\
76.47	0\\
76.48	0\\
76.49	0\\
76.5	0\\
76.51	0\\
76.52	0\\
76.53	0\\
76.54	0\\
76.55	0\\
76.56	0\\
76.57	0\\
76.58	0\\
76.59	0\\
76.6	0\\
76.61	0\\
76.62	0\\
76.63	0\\
76.64	0\\
76.65	0\\
76.66	0\\
76.67	0\\
76.68	0\\
76.69	0\\
76.7	0\\
76.71	0\\
76.72	0\\
76.73	0\\
76.74	0\\
76.75	0\\
76.76	0\\
76.77	0\\
76.78	0\\
76.79	0\\
76.8	0\\
76.81	0\\
76.82	0\\
76.83	0\\
76.84	0\\
76.85	0\\
76.86	0\\
76.87	0\\
76.88	0\\
76.89	0\\
76.9	0\\
76.91	0\\
76.92	0\\
76.93	0\\
76.94	0\\
76.95	0\\
76.96	0\\
76.97	0\\
76.98	0\\
76.99	0\\
77	0\\
77.01	0\\
77.02	0\\
77.03	0\\
77.04	0\\
77.05	0\\
77.06	0\\
77.07	0\\
77.08	0\\
77.09	0\\
77.1	0\\
77.11	0\\
77.12	0\\
77.13	0\\
77.14	0\\
77.15	0\\
77.16	0\\
77.17	0\\
77.18	0\\
77.19	0\\
77.2	0\\
77.21	0\\
77.22	0\\
77.23	0\\
77.24	0\\
77.25	0\\
77.26	0\\
77.27	0\\
77.28	0\\
77.29	0\\
77.3	0\\
77.31	0\\
77.32	0\\
77.33	0\\
77.34	0\\
77.35	0\\
77.36	0\\
77.37	0\\
77.38	0\\
77.39	0\\
77.4	0\\
77.41	0\\
77.42	0\\
77.43	0\\
77.44	0\\
77.45	0\\
77.46	0\\
77.47	0\\
77.48	0\\
77.49	0\\
77.5	0\\
77.51	0\\
77.52	0\\
77.53	0\\
77.54	0\\
77.55	0\\
77.56	0\\
77.57	0\\
77.58	0\\
77.59	0\\
77.6	0\\
77.61	0\\
77.62	0\\
77.63	0\\
77.64	0\\
77.65	0\\
77.66	0\\
77.67	0\\
77.68	0\\
77.69	0\\
77.7	0\\
77.71	0\\
77.72	0\\
77.73	0\\
77.74	0\\
77.75	0\\
77.76	0\\
77.77	0\\
77.78	0\\
77.79	0\\
77.8	0\\
77.81	0\\
77.82	0\\
77.83	0\\
77.84	0\\
77.85	0\\
77.86	0\\
77.87	0\\
77.88	0\\
77.89	0\\
77.9	0\\
77.91	0\\
77.92	0\\
77.93	0\\
77.94	0\\
77.95	0\\
77.96	0\\
77.97	0\\
77.98	0\\
77.99	0\\
78	0\\
78.01	0\\
78.02	0\\
78.03	0\\
78.04	0\\
78.05	0\\
78.06	0\\
78.07	0\\
78.08	0\\
78.09	0\\
78.1	0\\
78.11	0\\
78.12	0\\
78.13	0\\
78.14	0\\
78.15	0\\
78.16	0\\
78.17	0\\
78.18	0\\
78.19	0\\
78.2	0\\
78.21	0\\
78.22	0\\
78.23	0\\
78.24	0\\
78.25	0\\
78.26	0\\
78.27	0\\
78.28	0\\
78.29	0\\
78.3	0\\
78.31	0\\
78.32	0\\
78.33	0\\
78.34	0\\
78.35	0\\
78.36	0\\
78.37	0\\
78.38	0\\
78.39	0\\
78.4	0\\
78.41	0\\
78.42	0\\
78.43	0\\
78.44	0\\
78.45	0\\
78.46	0\\
78.47	0\\
78.48	0\\
78.49	0\\
78.5	0\\
78.51	0\\
78.52	0\\
78.53	0\\
78.54	0\\
78.55	0\\
78.56	0\\
78.57	0\\
78.58	0\\
78.59	0\\
78.6	0\\
78.61	0\\
78.62	0\\
78.63	0\\
78.64	0\\
78.65	0\\
78.66	0\\
78.67	0\\
78.68	0\\
78.69	0\\
78.7	0\\
78.71	0\\
78.72	0\\
78.73	0\\
78.74	0\\
78.75	0\\
78.76	0\\
78.77	0\\
78.78	0\\
78.79	0\\
78.8	0\\
78.81	0\\
78.82	0\\
78.83	0\\
78.84	0\\
78.85	0\\
78.86	0\\
78.87	0\\
78.88	0\\
78.89	0\\
78.9	0\\
78.91	0\\
78.92	0\\
78.93	0\\
78.94	0\\
78.95	0\\
78.96	0\\
78.97	0\\
78.98	0\\
78.99	0\\
79	0\\
79.01	0\\
79.02	0\\
79.03	0\\
79.04	0\\
79.05	0\\
79.06	0\\
79.07	0\\
79.08	0\\
79.09	0\\
79.1	0\\
79.11	0\\
79.12	0\\
79.13	0\\
79.14	0\\
79.15	0\\
79.16	0\\
79.17	0\\
79.18	0\\
79.19	0\\
79.2	0\\
79.21	0\\
79.22	0\\
79.23	0\\
79.24	0\\
79.25	0\\
79.26	0\\
79.27	0\\
79.28	0\\
79.29	0\\
79.3	0\\
79.31	0\\
79.32	0\\
79.33	0\\
79.34	0\\
79.35	0\\
79.36	0\\
79.37	0\\
79.38	0\\
79.39	0\\
79.4	0\\
79.41	0\\
79.42	0\\
79.43	0\\
79.44	0\\
79.45	0\\
79.46	0\\
79.47	0\\
79.48	0\\
79.49	0\\
79.5	0\\
79.51	0\\
79.52	0\\
79.53	0\\
79.54	0\\
79.55	0\\
79.56	0\\
79.57	0\\
79.58	0\\
79.59	0\\
79.6	0\\
79.61	0\\
79.62	0\\
79.63	0\\
79.64	0\\
79.65	0\\
79.66	0\\
79.67	0\\
79.68	0\\
79.69	0\\
79.7	0\\
79.71	0\\
79.72	0\\
79.73	0\\
79.74	0\\
79.75	0\\
79.76	0\\
79.77	0\\
79.78	0\\
79.79	0\\
79.8	0\\
79.81	0\\
79.82	0\\
79.83	0\\
79.84	0\\
79.85	0\\
79.86	0\\
79.87	0\\
79.88	0\\
79.89	0\\
79.9	0\\
79.91	0\\
79.92	0\\
79.93	0\\
79.94	0\\
79.95	0\\
79.96	0\\
79.97	0\\
79.98	0\\
79.99	0\\
80	0\\
80.01	0\\
};
\addplot [color=red,dashed]
  table[row sep=crcr]{%
80.01	0\\
80.02	0\\
80.03	0\\
80.04	0\\
80.05	0\\
80.06	0\\
80.07	0\\
80.08	0\\
80.09	0\\
80.1	0\\
80.11	0\\
80.12	0\\
80.13	0\\
80.14	0\\
80.15	0\\
80.16	0\\
80.17	0\\
80.18	0\\
80.19	0\\
80.2	0\\
80.21	0\\
80.22	0\\
80.23	0\\
80.24	0\\
80.25	0\\
80.26	0\\
80.27	0\\
80.28	0\\
80.29	0\\
80.3	0\\
80.31	0\\
80.32	0\\
80.33	0\\
80.34	0\\
80.35	0\\
80.36	0\\
80.37	0\\
80.38	0\\
80.39	0\\
80.4	0\\
80.41	0\\
80.42	0\\
80.43	0\\
80.44	0\\
80.45	0\\
80.46	0\\
80.47	0\\
80.48	0\\
80.49	0\\
80.5	0\\
80.51	0\\
80.52	0\\
80.53	0\\
80.54	0\\
80.55	0\\
80.56	0\\
80.57	0\\
80.58	0\\
80.59	0\\
80.6	0\\
80.61	0\\
80.62	0\\
80.63	0\\
80.64	0\\
80.65	0\\
80.66	0\\
80.67	0\\
80.68	0\\
80.69	0\\
80.7	0\\
80.71	0\\
80.72	0\\
80.73	0\\
80.74	0\\
80.75	0\\
80.76	0\\
80.77	0\\
80.78	0\\
80.79	0\\
80.8	0\\
80.81	0\\
80.82	0\\
80.83	0\\
80.84	0\\
80.85	0\\
80.86	0\\
80.87	0\\
80.88	0\\
80.89	0\\
80.9	0\\
80.91	0\\
80.92	0\\
80.93	0\\
80.94	0\\
80.95	0\\
80.96	0\\
80.97	0\\
80.98	0\\
80.99	0\\
81	0\\
81.01	0\\
81.02	0\\
81.03	0\\
81.04	0\\
81.05	0\\
81.06	0\\
81.07	0\\
81.08	0\\
81.09	0\\
81.1	0\\
81.11	0\\
81.12	0\\
81.13	0\\
81.14	0\\
81.15	0\\
81.16	0\\
81.17	0\\
81.18	0\\
81.19	0\\
81.2	0\\
81.21	0\\
81.22	0\\
81.23	0\\
81.24	0\\
81.25	0\\
81.26	0\\
81.27	0\\
81.28	0\\
81.29	0\\
81.3	0\\
81.31	0\\
81.32	0\\
81.33	0\\
81.34	0\\
81.35	0\\
81.36	0\\
81.37	0\\
81.38	0\\
81.39	0\\
81.4	0\\
81.41	0\\
81.42	0\\
81.43	0\\
81.44	0\\
81.45	0\\
81.46	0\\
81.47	0\\
81.48	0\\
81.49	0\\
81.5	0\\
81.51	0\\
81.52	0\\
81.53	0\\
81.54	0\\
81.55	0\\
81.56	0\\
81.57	0\\
81.58	0\\
81.59	0\\
81.6	0\\
81.61	0\\
81.62	0\\
81.63	0\\
81.64	0\\
81.65	0\\
81.66	0\\
81.67	0\\
81.68	0\\
81.69	0\\
81.7	0\\
81.71	0\\
81.72	0\\
81.73	0\\
81.74	0\\
81.75	0\\
81.76	0\\
81.77	0\\
81.78	0\\
81.79	0\\
81.8	0\\
81.81	0\\
81.82	0\\
81.83	0\\
81.84	0\\
81.85	0\\
81.86	0\\
81.87	0\\
81.88	0\\
81.89	0\\
81.9	0\\
81.91	0\\
81.92	0\\
81.93	0\\
81.94	0\\
81.95	0\\
81.96	0\\
81.97	0\\
81.98	0\\
81.99	0\\
82	0\\
82.01	0\\
82.02	0\\
82.03	0\\
82.04	0\\
82.05	0\\
82.06	0\\
82.07	0\\
82.08	0\\
82.09	0\\
82.1	0\\
82.11	0\\
82.12	0\\
82.13	0\\
82.14	0\\
82.15	0\\
82.16	0\\
82.17	0\\
82.18	0\\
82.19	0\\
82.2	0\\
82.21	0\\
82.22	0\\
82.23	0\\
82.24	0\\
82.25	0\\
82.26	0\\
82.27	0\\
82.28	0\\
82.29	0\\
82.3	0\\
82.31	0\\
82.32	0\\
82.33	0\\
82.34	0\\
82.35	0\\
82.36	0\\
82.37	0\\
82.38	0\\
82.39	0\\
82.4	0\\
82.41	0\\
82.42	0\\
82.43	0\\
82.44	0\\
82.45	0\\
82.46	0\\
82.47	0\\
82.48	0\\
82.49	0\\
82.5	0\\
82.51	0\\
82.52	0\\
82.53	0\\
82.54	0\\
82.55	0\\
82.56	0\\
82.57	0\\
82.58	0\\
82.59	0\\
82.6	0\\
82.61	0\\
82.62	0\\
82.63	0\\
82.64	0\\
82.65	0\\
82.66	0\\
82.67	0\\
82.68	0\\
82.69	0\\
82.7	0\\
82.71	0\\
82.72	0\\
82.73	0\\
82.74	0\\
82.75	0\\
82.76	0\\
82.77	0\\
82.78	0\\
82.79	0\\
82.8	0\\
82.81	0\\
82.82	0\\
82.83	0\\
82.84	0\\
82.85	0\\
82.86	0\\
82.87	0\\
82.88	0\\
82.89	0\\
82.9	0\\
82.91	0\\
82.92	0\\
82.93	0\\
82.94	0\\
82.95	0\\
82.96	0\\
82.97	0\\
82.98	0\\
82.99	0\\
83	0\\
83.01	0\\
83.02	0\\
83.03	0\\
83.04	0\\
83.05	0\\
83.06	0\\
83.07	0\\
83.08	0\\
83.09	0\\
83.1	0\\
83.11	0\\
83.12	0\\
83.13	0\\
83.14	0\\
83.15	0\\
83.16	0\\
83.17	0\\
83.18	0\\
83.19	0\\
83.2	0\\
83.21	0\\
83.22	0\\
83.23	0\\
83.24	0\\
83.25	0\\
83.26	0\\
83.27	0\\
83.28	0\\
83.29	0\\
83.3	0\\
83.31	0\\
83.32	0\\
83.33	0\\
83.34	0\\
83.35	0\\
83.36	0\\
83.37	0\\
83.38	0\\
83.39	0\\
83.4	0\\
83.41	0\\
83.42	0\\
83.43	0\\
83.44	0\\
83.45	0\\
83.46	0\\
83.47	0\\
83.48	0\\
83.49	0\\
83.5	0\\
83.51	0\\
83.52	0\\
83.53	0\\
83.54	0\\
83.55	0\\
83.56	0\\
83.57	0\\
83.58	0\\
83.59	0\\
83.6	0\\
83.61	0\\
83.62	0\\
83.63	0\\
83.64	0\\
83.65	0\\
83.66	0\\
83.67	0\\
83.68	0\\
83.69	0\\
83.7	0\\
83.71	0\\
83.72	0\\
83.73	0\\
83.74	0\\
83.75	0\\
83.76	0\\
83.77	0\\
83.78	0\\
83.79	0\\
83.8	0\\
83.81	0\\
83.82	0\\
83.83	0\\
83.84	0\\
83.85	0\\
83.86	0\\
83.87	0\\
83.88	0\\
83.89	0\\
83.9	0\\
83.91	0\\
83.92	0\\
83.93	0\\
83.94	0\\
83.95	0\\
83.96	0\\
83.97	0\\
83.98	0\\
83.99	0\\
84	0\\
84.01	0\\
84.02	0\\
84.03	0\\
84.04	0\\
84.05	0\\
84.06	0\\
84.07	0\\
84.08	0\\
84.09	0\\
84.1	0\\
84.11	0\\
84.12	0\\
84.13	0\\
84.14	0\\
84.15	0\\
84.16	0\\
84.17	0\\
84.18	0\\
84.19	0\\
84.2	0\\
84.21	0\\
84.22	0\\
84.23	0\\
84.24	0\\
84.25	0\\
84.26	0\\
84.27	0\\
84.28	0\\
84.29	0\\
84.3	0\\
84.31	0\\
84.32	0\\
84.33	0\\
84.34	0\\
84.35	0\\
84.36	0\\
84.37	0\\
84.38	0\\
84.39	0\\
84.4	0\\
84.41	0\\
84.42	0\\
84.43	0\\
84.44	0\\
84.45	0\\
84.46	0\\
84.47	0\\
84.48	0\\
84.49	0\\
84.5	0\\
84.51	0\\
84.52	0\\
84.53	0\\
84.54	0\\
84.55	0\\
84.56	0\\
84.57	0\\
84.58	0\\
84.59	0\\
84.6	0\\
84.61	0\\
84.62	0\\
84.63	0\\
84.64	0\\
84.65	0\\
84.66	0\\
84.67	0\\
84.68	0\\
84.69	0\\
84.7	0\\
84.71	0\\
84.72	0\\
84.73	0\\
84.74	0\\
84.75	0\\
84.76	0\\
84.77	0\\
84.78	0\\
84.79	0\\
84.8	0\\
84.81	0\\
84.82	0\\
84.83	0\\
84.84	0\\
84.85	0\\
84.86	0\\
84.87	0\\
84.88	0\\
84.89	0\\
84.9	0\\
84.91	0\\
84.92	0\\
84.93	0\\
84.94	0\\
84.95	0\\
84.96	0\\
84.97	0\\
84.98	0\\
84.99	0\\
85	0\\
85.01	0\\
85.02	0\\
85.03	0\\
85.04	0\\
85.05	0\\
85.06	0\\
85.07	0\\
85.08	0\\
85.09	0\\
85.1	0\\
85.11	0\\
85.12	0\\
85.13	0\\
85.14	0\\
85.15	0\\
85.16	0\\
85.17	0\\
85.18	0\\
85.19	0\\
85.2	0\\
85.21	0\\
85.22	0\\
85.23	0\\
85.24	0\\
85.25	0\\
85.26	0\\
85.27	0\\
85.28	0\\
85.29	0\\
85.3	0\\
85.31	0\\
85.32	0\\
85.33	0\\
85.34	0\\
85.35	0\\
85.36	0\\
85.37	0\\
85.38	0\\
85.39	0\\
85.4	0\\
85.41	0\\
85.42	0\\
85.43	0\\
85.44	0\\
85.45	0\\
85.46	0\\
85.47	0\\
85.48	0\\
85.49	0\\
85.5	0\\
85.51	0\\
85.52	0\\
85.53	0\\
85.54	0\\
85.55	0\\
85.56	0\\
85.57	0\\
85.58	0\\
85.59	0\\
85.6	0\\
85.61	0\\
85.62	0\\
85.63	0\\
85.64	0\\
85.65	0\\
85.66	0\\
85.67	0\\
85.68	0\\
85.69	0\\
85.7	0\\
85.71	0\\
85.72	0\\
85.73	0\\
85.74	0\\
85.75	0\\
85.76	0\\
85.77	0\\
85.78	0\\
85.79	0\\
85.8	0\\
85.81	0\\
85.82	0\\
85.83	0\\
85.84	0\\
85.85	0\\
85.86	0\\
85.87	0\\
85.88	0\\
85.89	0\\
85.9	0\\
85.91	0\\
85.92	0\\
85.93	0\\
85.94	0\\
85.95	0\\
85.96	0\\
85.97	0\\
85.98	0\\
85.99	0\\
86	0\\
86.01	0\\
86.02	0\\
86.03	0\\
86.04	0\\
86.05	0\\
86.06	0\\
86.07	0\\
86.08	0\\
86.09	0\\
86.1	0\\
86.11	0\\
86.12	0\\
86.13	0\\
86.14	0\\
86.15	0\\
86.16	0\\
86.17	0\\
86.18	0\\
86.19	0\\
86.2	0\\
86.21	0\\
86.22	0\\
86.23	0\\
86.24	0\\
86.25	0\\
86.26	0\\
86.27	0\\
86.28	0\\
86.29	0\\
86.3	0\\
86.31	0\\
86.32	0\\
86.33	0\\
86.34	0\\
86.35	0\\
86.36	0\\
86.37	0\\
86.38	0\\
86.39	0\\
86.4	0\\
86.41	0\\
86.42	0\\
86.43	0\\
86.44	0\\
86.45	0\\
86.46	0\\
86.47	0\\
86.48	0\\
86.49	0\\
86.5	0\\
86.51	0\\
86.52	0\\
86.53	0\\
86.54	0\\
86.55	0\\
86.56	0\\
86.57	0\\
86.58	0\\
86.59	0\\
86.6	0\\
86.61	0\\
86.62	0\\
86.63	0\\
86.64	0\\
86.65	0\\
86.66	0\\
86.67	0\\
86.68	0\\
86.69	0\\
86.7	0\\
86.71	0\\
86.72	0\\
86.73	0\\
86.74	0\\
86.75	0\\
86.76	0\\
86.77	0\\
86.78	0\\
86.79	0\\
86.8	0\\
86.81	0\\
86.82	0\\
86.83	0\\
86.84	0\\
86.85	0\\
86.86	0\\
86.87	0\\
86.88	0\\
86.89	0\\
86.9	0\\
86.91	0\\
86.92	0\\
86.93	0\\
86.94	0\\
86.95	0\\
86.96	0\\
86.97	0\\
86.98	0\\
86.99	0\\
87	0\\
87.01	0\\
87.02	0\\
87.03	0\\
87.04	0\\
87.05	0\\
87.06	0\\
87.07	0\\
87.08	0\\
87.09	0\\
87.1	0\\
87.11	0\\
87.12	0\\
87.13	0\\
87.14	0\\
87.15	0\\
87.16	0\\
87.17	0\\
87.18	0\\
87.19	0\\
87.2	0\\
87.21	0\\
87.22	0\\
87.23	0\\
87.24	0\\
87.25	0\\
87.26	0\\
87.27	0\\
87.28	0\\
87.29	0\\
87.3	0\\
87.31	0\\
87.32	0\\
87.33	0\\
87.34	0\\
87.35	0\\
87.36	0\\
87.37	0\\
87.38	0\\
87.39	0\\
87.4	0\\
87.41	0\\
87.42	0\\
87.43	0\\
87.44	0\\
87.45	0\\
87.46	0\\
87.47	0\\
87.48	0\\
87.49	0\\
87.5	0\\
87.51	0\\
87.52	0\\
87.53	0\\
87.54	0\\
87.55	0\\
87.56	0\\
87.57	0\\
87.58	0\\
87.59	0\\
87.6	0\\
87.61	0\\
87.62	0\\
87.63	0\\
87.64	0\\
87.65	0\\
87.66	0\\
87.67	0\\
87.68	0\\
87.69	0\\
87.7	0\\
87.71	0\\
87.72	0\\
87.73	0\\
87.74	0\\
87.75	0\\
87.76	0\\
87.77	0\\
87.78	0\\
87.79	0\\
87.8	0\\
87.81	0\\
87.82	0\\
87.83	0\\
87.84	0\\
87.85	0\\
87.86	0\\
87.87	0\\
87.88	0\\
87.89	0\\
87.9	0\\
87.91	0\\
87.92	0\\
87.93	0\\
87.94	0\\
87.95	0\\
87.96	0\\
87.97	0\\
87.98	0\\
87.99	0\\
88	0\\
88.01	0\\
88.02	0\\
88.03	0\\
88.04	0\\
88.05	0\\
88.06	0\\
88.07	0\\
88.08	0\\
88.09	0\\
88.1	0\\
88.11	0\\
88.12	0\\
88.13	0\\
88.14	0\\
88.15	0\\
88.16	0\\
88.17	0\\
88.18	0\\
88.19	0\\
88.2	0\\
88.21	0\\
88.22	0\\
88.23	0\\
88.24	0\\
88.25	0\\
88.26	0\\
88.27	0\\
88.28	0\\
88.29	0\\
88.3	0\\
88.31	0\\
88.32	0\\
88.33	0\\
88.34	0\\
88.35	0\\
88.36	0\\
88.37	0\\
88.38	0\\
88.39	0\\
88.4	0\\
88.41	0\\
88.42	0\\
88.43	0\\
88.44	0\\
88.45	0\\
88.46	0\\
88.47	0\\
88.48	0\\
88.49	0\\
88.5	0\\
88.51	0\\
88.52	0\\
88.53	0\\
88.54	0\\
88.55	0\\
88.56	0\\
88.57	0\\
88.58	0\\
88.59	0\\
88.6	0\\
88.61	0\\
88.62	0\\
88.63	0\\
88.64	0\\
88.65	0\\
88.66	0\\
88.67	0\\
88.68	0\\
88.69	0\\
88.7	0\\
88.71	0\\
88.72	0\\
88.73	0\\
88.74	0\\
88.75	0\\
88.76	0\\
88.77	0\\
88.78	0\\
88.79	0\\
88.8	0\\
88.81	0\\
88.82	0\\
88.83	0\\
88.84	0\\
88.85	0\\
88.86	0\\
88.87	0\\
88.88	0\\
88.89	0\\
88.9	0\\
88.91	0\\
88.92	0\\
88.93	0\\
88.94	0\\
88.95	0\\
88.96	0\\
88.97	0\\
88.98	0\\
88.99	0\\
89	0\\
89.01	0\\
89.02	0\\
89.03	0\\
89.04	0\\
89.05	0\\
89.06	0\\
89.07	0\\
89.08	0\\
89.09	0\\
89.1	0\\
89.11	0\\
89.12	0\\
89.13	0\\
89.14	0\\
89.15	0\\
89.16	0\\
89.17	0\\
89.18	0\\
89.19	0\\
89.2	0\\
89.21	0\\
89.22	0\\
89.23	0\\
89.24	0\\
89.25	0\\
89.26	0\\
89.27	0\\
89.28	0\\
89.29	0\\
89.3	0\\
89.31	0\\
89.32	0\\
89.33	0\\
89.34	0\\
89.35	0\\
89.36	0\\
89.37	0\\
89.38	0\\
89.39	0\\
89.4	0\\
89.41	0\\
89.42	0\\
89.43	0\\
89.44	0\\
89.45	0\\
89.46	0\\
89.47	0\\
89.48	0\\
89.49	0\\
89.5	0\\
89.51	0\\
89.52	0\\
89.53	0\\
89.54	0\\
89.55	0\\
89.56	0\\
89.57	0\\
89.58	0\\
89.59	0\\
89.6	0\\
89.61	0\\
89.62	0\\
89.63	0\\
89.64	0\\
89.65	0\\
89.66	0\\
89.67	0\\
89.68	0\\
89.69	0\\
89.7	0\\
89.71	0\\
89.72	0\\
89.73	0\\
89.74	0\\
89.75	0\\
89.76	0\\
89.77	0\\
89.78	0\\
89.79	0\\
89.8	0\\
89.81	0\\
89.82	0\\
89.83	0\\
89.84	0\\
89.85	0\\
89.86	0\\
89.87	0\\
89.88	0\\
89.89	0\\
89.9	0\\
89.91	0\\
89.92	0\\
89.93	0\\
89.94	0\\
89.95	0\\
89.96	0\\
89.97	0\\
89.98	0\\
89.99	0\\
90	0\\
90.01	0\\
90.02	0\\
90.03	0\\
90.04	0\\
90.05	0\\
90.06	0\\
90.07	0\\
90.08	0\\
90.09	0\\
90.1	0\\
90.11	0\\
90.12	0\\
90.13	0\\
90.14	0\\
90.15	0\\
90.16	0\\
90.17	0\\
90.18	0\\
90.19	0\\
90.2	0\\
90.21	0\\
90.22	0\\
90.23	0\\
90.24	0\\
90.25	0\\
90.26	0\\
90.27	0\\
90.28	0\\
90.29	0\\
90.3	0\\
90.31	0\\
90.32	0\\
90.33	0\\
90.34	0\\
90.35	0\\
90.36	0\\
90.37	0\\
90.38	0\\
90.39	0\\
90.4	0\\
90.41	0\\
90.42	0\\
90.43	0\\
90.44	0\\
90.45	0\\
90.46	0\\
90.47	0\\
90.48	0\\
90.49	0\\
90.5	0\\
90.51	0\\
90.52	0\\
90.53	0\\
90.54	0\\
90.55	0\\
90.56	0\\
90.57	0\\
90.58	0\\
90.59	0\\
90.6	0\\
90.61	0\\
90.62	0\\
90.63	0\\
90.64	0\\
90.65	0\\
90.66	0\\
90.67	0\\
90.68	0\\
90.69	0\\
90.7	0\\
90.71	0\\
90.72	0\\
90.73	0\\
90.74	0\\
90.75	0\\
90.76	0\\
90.77	0\\
90.78	0\\
90.79	0\\
90.8	0\\
90.81	0\\
90.82	0\\
90.83	0\\
90.84	0\\
90.85	0\\
90.86	0\\
90.87	0\\
90.88	0\\
90.89	0\\
90.9	0\\
90.91	0\\
90.92	0\\
90.93	0\\
90.94	0\\
90.95	0\\
90.96	0\\
90.97	0\\
90.98	0\\
90.99	0\\
91	0\\
91.01	0\\
91.02	0\\
91.03	0\\
91.04	0\\
91.05	0\\
91.06	0\\
91.07	0\\
91.08	0\\
91.09	0\\
91.1	0\\
91.11	0\\
91.12	0\\
91.13	0\\
91.14	0\\
91.15	0\\
91.16	0\\
91.17	0\\
91.18	0\\
91.19	0\\
91.2	0\\
91.21	0\\
91.22	0\\
91.23	0\\
91.24	0\\
91.25	0\\
91.26	0\\
91.27	0\\
91.28	0\\
91.29	0\\
91.3	0\\
91.31	0\\
91.32	0\\
91.33	0\\
91.34	0\\
91.35	0\\
91.36	0\\
91.37	0\\
91.38	0\\
91.39	0\\
91.4	0\\
91.41	0\\
91.42	0\\
91.43	0\\
91.44	0\\
91.45	0\\
91.46	0\\
91.47	0\\
91.48	0\\
91.49	0\\
91.5	0\\
91.51	0\\
91.52	0\\
91.53	0\\
91.54	0\\
91.55	0\\
91.56	0\\
91.57	0\\
91.58	0\\
91.59	0\\
91.6	0\\
91.61	0\\
91.62	0\\
91.63	0\\
91.64	0\\
91.65	0\\
91.66	0\\
91.67	0\\
91.68	0\\
91.69	0\\
91.7	0\\
91.71	0\\
91.72	0\\
91.73	0\\
91.74	0\\
91.75	0\\
91.76	0\\
91.77	0\\
91.78	0\\
91.79	0\\
91.8	0\\
91.81	0\\
91.82	0\\
91.83	0\\
91.84	0\\
91.85	0\\
91.86	0\\
91.87	0\\
91.88	0\\
91.89	0\\
91.9	0\\
91.91	0\\
91.92	0\\
91.93	0\\
91.94	0\\
91.95	0\\
91.96	0\\
91.97	0\\
91.98	0\\
91.99	0\\
92	0\\
92.01	0\\
92.02	0\\
92.03	0\\
92.04	0\\
92.05	0\\
92.06	0\\
92.07	0\\
92.08	0\\
92.09	0\\
92.1	0\\
92.11	0\\
92.12	0\\
92.13	0\\
92.14	0\\
92.15	0\\
92.16	0\\
92.17	0\\
92.18	0\\
92.19	0\\
92.2	0\\
92.21	0\\
92.22	0\\
92.23	0\\
92.24	0\\
92.25	0\\
92.26	0\\
92.27	0\\
92.28	0\\
92.29	0\\
92.3	0\\
92.31	0\\
92.32	0\\
92.33	0\\
92.34	0\\
92.35	0\\
92.36	0\\
92.37	0\\
92.38	0\\
92.39	0\\
92.4	0\\
92.41	0\\
92.42	0\\
92.43	0\\
92.44	0\\
92.45	0\\
92.46	0\\
92.47	0\\
92.48	0\\
92.49	0\\
92.5	0\\
92.51	0\\
92.52	0\\
92.53	0\\
92.54	0\\
92.55	0\\
92.56	0\\
92.57	0\\
92.58	0\\
92.59	0\\
92.6	0\\
92.61	0\\
92.62	0\\
92.63	0\\
92.64	0\\
92.65	0\\
92.66	0\\
92.67	0\\
92.68	0\\
92.69	0\\
92.7	0\\
92.71	0\\
92.72	0\\
92.73	0\\
92.74	0\\
92.75	0\\
92.76	0\\
92.77	0\\
92.78	0\\
92.79	0\\
92.8	0\\
92.81	0\\
92.82	0\\
92.83	0\\
92.84	0\\
92.85	0\\
92.86	0\\
92.87	0\\
92.88	0\\
92.89	0\\
92.9	0\\
92.91	0\\
92.92	0\\
92.93	0\\
92.94	0\\
92.95	0\\
92.96	0\\
92.97	0\\
92.98	0\\
92.99	0\\
93	0\\
93.01	0\\
93.02	0\\
93.03	0\\
93.04	0\\
93.05	0\\
93.06	0\\
93.07	0\\
93.08	0\\
93.09	0\\
93.1	0\\
93.11	0\\
93.12	0\\
93.13	0\\
93.14	0\\
93.15	0\\
93.16	0\\
93.17	0\\
93.18	0\\
93.19	0\\
93.2	0\\
93.21	0\\
93.22	0\\
93.23	0\\
93.24	0\\
93.25	0\\
93.26	0\\
93.27	0\\
93.28	0\\
93.29	0\\
93.3	0\\
93.31	0\\
93.32	0\\
93.33	0\\
93.34	0\\
93.35	0\\
93.36	0\\
93.37	0\\
93.38	0\\
93.39	0\\
93.4	0\\
93.41	0\\
93.42	0\\
93.43	0\\
93.44	0\\
93.45	0\\
93.46	0\\
93.47	0\\
93.48	0\\
93.49	0\\
93.5	0\\
93.51	0\\
93.52	0\\
93.53	0\\
93.54	0\\
93.55	0\\
93.56	0\\
93.57	0\\
93.58	0\\
93.59	0\\
93.6	0\\
93.61	0\\
93.62	0\\
93.63	0\\
93.64	0\\
93.65	0\\
93.66	0\\
93.67	0\\
93.68	0\\
93.69	0\\
93.7	0\\
93.71	0\\
93.72	0\\
93.73	0\\
93.74	0\\
93.75	0\\
93.76	0\\
93.77	0\\
93.78	0\\
93.79	0\\
93.8	0\\
93.81	0\\
93.82	0\\
93.83	0\\
93.84	0\\
93.85	0\\
93.86	0\\
93.87	0\\
93.88	0\\
93.89	0\\
93.9	0\\
93.91	0\\
93.92	0\\
93.93	0\\
93.94	0\\
93.95	0\\
93.96	0\\
93.97	0\\
93.98	0\\
93.99	0\\
94	0\\
94.01	0\\
94.02	0\\
94.03	0\\
94.04	0\\
94.05	0\\
94.06	0\\
94.07	0\\
94.08	0\\
94.09	0\\
94.1	0\\
94.11	0\\
94.12	0\\
94.13	0\\
94.14	0\\
94.15	0\\
94.16	0\\
94.17	0\\
94.18	0\\
94.19	0\\
94.2	0\\
94.21	0\\
94.22	0\\
94.23	0\\
94.24	0\\
94.25	0\\
94.26	0\\
94.27	0\\
94.28	0\\
94.29	0\\
94.3	0\\
94.31	0\\
94.32	0\\
94.33	0\\
94.34	0\\
94.35	0\\
94.36	0\\
94.37	0\\
94.38	0\\
94.39	0\\
94.4	0\\
94.41	0\\
94.42	0\\
94.43	0\\
94.44	0\\
94.45	0\\
94.46	0\\
94.47	0\\
94.48	0\\
94.49	0\\
94.5	0\\
94.51	0\\
94.52	0\\
94.53	0\\
94.54	0\\
94.55	0\\
94.56	0\\
94.57	0\\
94.58	0\\
94.59	0\\
94.6	0\\
94.61	0\\
94.62	0\\
94.63	0\\
94.64	0\\
94.65	0\\
94.66	0\\
94.67	0\\
94.68	0\\
94.69	0\\
94.7	0\\
94.71	0\\
94.72	0\\
94.73	0\\
94.74	0\\
94.75	0\\
94.76	0\\
94.77	0\\
94.78	0\\
94.79	0\\
94.8	0\\
94.81	0\\
94.82	0\\
94.83	0\\
94.84	0\\
94.85	0\\
94.86	0\\
94.87	0\\
94.88	0\\
94.89	0\\
94.9	0\\
94.91	0\\
94.92	0\\
94.93	0\\
94.94	0\\
94.95	0\\
94.96	0\\
94.97	0\\
94.98	0\\
94.99	0\\
95	0\\
95.01	0\\
95.02	0\\
95.03	0\\
95.04	0\\
95.05	0\\
95.06	0\\
95.07	0\\
95.08	0\\
95.09	0\\
95.1	0\\
95.11	0\\
95.12	0\\
95.13	0\\
95.14	0\\
95.15	0\\
95.16	0\\
95.17	0\\
95.18	0\\
95.19	0\\
95.2	0\\
95.21	0\\
95.22	0\\
95.23	0\\
95.24	0\\
95.25	0\\
95.26	0\\
95.27	0\\
95.28	0\\
95.29	0\\
95.3	0\\
95.31	0\\
95.32	0\\
95.33	0\\
95.34	0\\
95.35	0\\
95.36	0\\
95.37	0\\
95.38	0\\
95.39	0\\
95.4	0\\
95.41	0\\
95.42	0\\
95.43	0\\
95.44	0\\
95.45	0\\
95.46	0\\
95.47	0\\
95.48	0\\
95.49	0\\
95.5	0\\
95.51	0\\
95.52	0\\
95.53	0\\
95.54	0\\
95.55	0\\
95.56	0\\
95.57	0\\
95.58	0\\
95.59	0\\
95.6	0\\
95.61	0\\
95.62	0\\
95.63	0\\
95.64	0\\
95.65	0\\
95.66	0\\
95.67	0\\
95.68	0\\
95.69	0\\
95.7	0\\
95.71	0\\
95.72	0\\
95.73	0\\
95.74	0\\
95.75	0\\
95.76	0\\
95.77	0\\
95.78	0\\
95.79	0\\
95.8	0\\
95.81	0\\
95.82	0\\
95.83	0\\
95.84	0\\
95.85	0\\
95.86	0\\
95.87	0\\
95.88	0\\
95.89	0\\
95.9	0\\
95.91	0\\
95.92	0\\
95.93	0\\
95.94	0\\
95.95	0\\
95.96	0\\
95.97	0\\
95.98	0\\
95.99	0\\
96	0\\
96.01	0\\
96.02	0\\
96.03	0\\
96.04	0\\
96.05	0\\
96.06	0\\
96.07	0\\
96.08	0\\
96.09	0\\
96.1	0\\
96.11	0\\
96.12	0\\
96.13	0\\
96.14	0\\
96.15	0\\
96.16	0\\
96.17	0\\
96.18	0\\
96.19	0\\
96.2	0\\
96.21	0\\
96.22	0\\
96.23	0\\
96.24	0\\
96.25	0\\
96.26	0\\
96.27	0\\
96.28	0\\
96.29	0\\
96.3	0\\
96.31	0\\
96.32	0\\
96.33	0\\
96.34	0\\
96.35	0\\
96.36	0\\
96.37	0\\
96.38	0\\
96.39	0\\
96.4	0\\
96.41	0\\
96.42	0\\
96.43	0\\
96.44	0\\
96.45	0\\
96.46	0\\
96.47	0\\
96.48	0\\
96.49	0\\
96.5	0\\
96.51	0\\
96.52	0\\
96.53	0\\
96.54	0\\
96.55	0\\
96.56	0\\
96.57	0\\
96.58	0\\
96.59	0\\
96.6	0\\
96.61	0\\
96.62	0\\
96.63	0\\
96.64	0\\
96.65	0\\
96.66	0\\
96.67	0\\
96.68	0\\
96.69	0\\
96.7	0\\
96.71	0\\
96.72	0\\
96.73	0\\
96.74	0\\
96.75	0\\
96.76	0\\
96.77	0\\
96.78	0\\
96.79	0\\
96.8	0\\
96.81	0\\
96.82	0\\
96.83	0\\
96.84	0\\
96.85	0\\
96.86	0\\
96.87	0\\
96.88	0\\
96.89	0\\
96.9	0\\
96.91	0\\
96.92	0\\
96.93	0\\
96.94	0\\
96.95	0\\
96.96	0\\
96.97	0\\
96.98	0\\
96.99	0\\
97	0\\
97.01	0\\
97.02	0\\
97.03	0\\
97.04	0\\
97.05	0\\
97.06	0\\
97.07	0\\
97.08	0\\
97.09	0\\
97.1	0\\
97.11	0\\
97.12	0\\
97.13	0\\
97.14	0\\
97.15	0\\
97.16	0\\
97.17	0\\
97.18	0\\
97.19	0\\
97.2	0\\
97.21	0\\
97.22	0\\
97.23	0\\
97.24	0\\
97.25	0\\
97.26	0\\
97.27	0\\
97.28	0\\
97.29	0\\
97.3	0\\
97.31	0\\
97.32	0\\
97.33	0\\
97.34	0\\
97.35	0\\
97.36	0\\
97.37	0\\
97.38	0\\
97.39	0\\
97.4	0\\
97.41	0\\
97.42	0\\
97.43	0\\
97.44	0\\
97.45	0\\
97.46	0\\
97.47	0\\
97.48	0\\
97.49	0\\
97.5	0\\
97.51	0\\
97.52	0\\
97.53	0\\
97.54	0\\
97.55	0\\
97.56	0\\
97.57	0\\
97.58	0\\
97.59	0\\
97.6	0\\
97.61	0\\
97.62	0\\
97.63	0\\
97.64	0\\
97.65	0\\
97.66	0\\
97.67	0\\
97.68	0\\
97.69	0\\
97.7	0\\
97.71	0\\
97.72	0\\
97.73	0\\
97.74	0\\
97.75	0\\
97.76	0\\
97.77	0\\
97.78	0\\
97.79	0\\
97.8	0\\
97.81	0\\
97.82	0\\
97.83	0\\
97.84	0\\
97.85	0\\
97.86	0\\
97.87	0\\
97.88	0\\
97.89	0\\
97.9	0\\
97.91	0\\
97.92	0\\
97.93	0\\
97.94	0\\
97.95	0\\
97.96	0\\
97.97	0\\
97.98	0\\
97.99	0\\
98	0\\
98.01	0\\
98.02	0\\
98.03	0\\
98.04	0\\
98.05	0\\
98.06	0\\
98.07	0\\
98.08	0\\
98.09	0\\
98.1	0\\
98.11	0\\
98.12	0\\
98.13	0\\
98.14	0\\
98.15	0\\
98.16	4.82212896919486e-05\\
98.17	0.000120045188160105\\
98.18	0.000192408590472284\\
98.19	0.000265312749195942\\
98.2	0.000338762836400275\\
98.21	0.00041276407838637\\
98.22	0.000487321756357278\\
98.23	0.000562441207098793\\
98.24	0.000638127823671193\\
98.25	0.000714387056112143\\
98.26	0.000791224412151063\\
98.27	0.000868645457935174\\
98.28	0.000946655818767503\\
98.29	0.00102526117985712\\
98.3	0.00110446728708188\\
98.31	0.00118427994776397\\
98.32	0.00126470503145857\\
98.33	0.00129063335896428\\
98.34	0.00131320264556022\\
98.35	0.00133596245513487\\
98.36	0.00135891443389092\\
98.37	0.00138206018901843\\
98.38	0.00140540133905432\\
98.39	0.00142893951390897\\
98.4	0.00145267635355033\\
98.41	0.00147661407642034\\
98.42	0.0015007589650365\\
98.43	0.00152511278879065\\
98.44	0.00154967733010087\\
98.45	0.00157445438445211\\
98.46	0.00159944576043506\\
98.47	0.00162465327978319\\
98.48	0.00165007877740776\\
98.49	0.00167572410143088\\
98.5	0.00170159111321638\\
98.51	0.00172768168739846\\
98.52	0.00175399771190806\\
98.53	0.00178054108799671\\
98.54	0.00180731373025793\\
98.55	0.00183431756664591\\
98.56	0.00186155453807235\\
98.57	0.00188902659564803\\
98.58	0.00191673570377021\\
98.59	0.00194468384012683\\
98.6	0.00197287299569721\\
98.61	0.00200130517474939\\
98.62	0.00202998239483375\\
98.63	0.00205890829765318\\
98.64	0.00208809764632848\\
98.65	0.00211755291960702\\
98.66	0.00214727661987198\\
98.67	0.00217727127337655\\
98.68	0.00220753943048065\\
98.69	0.00223808366589022\\
98.7	0.00226890657892397\\
98.71	0.00230001079376467\\
98.72	0.00233139895970779\\
98.73	0.002363073751413\\
98.74	0.00239503786060027\\
98.75	0.00242729399520301\\
98.76	0.00245984488856546\\
98.77	0.00249269329968366\\
98.78	0.00252584201345006\\
98.79	0.00255929384066778\\
98.8	0.00259305161330027\\
98.81	0.0026271181897285\\
98.82	0.00266149645499687\\
98.83	0.00269618932106149\\
98.84	0.0027311997270407\\
98.85	0.00276653063946793\\
98.86	0.002802185052547\\
98.87	0.00283816598840958\\
98.88	0.00287447649737527\\
98.89	0.00291111965821392\\
98.9	0.0029480985784105\\
98.91	0.00298541639443233\\
98.92	0.00302307627200764\\
98.93	0.00306108140639934\\
98.94	0.00309943502268039\\
98.95	0.00313814037601164\\
98.96	0.00317720075192241\\
98.97	0.00321661946659351\\
98.98	0.00325639986714306\\
98.99	0.00329654533191486\\
99	0.00333705927076955\\
99.01	0.00337794512537842\\
99.02	0.00341920636951997\\
99.03	0.00346084650937926\\
99.04	0.00350286908385008\\
99.05	0.00354527766483987\\
99.06	0.00358807585757756\\
99.07	0.00363126730092422\\
99.08	0.00367485566768671\\
99.09	0.00371884466493414\\
99.1	0.00376323803431739\\
99.11	0.00380803955239153\\
99.12	0.00385325303094139\\
99.13	0.00389888231730995\\
99.14	0.00394493129473008\\
99.15	0.00399140388265912\\
99.16	0.00403830403711683\\
99.17	0.00408563575102626\\
99.18	0.00413340305455804\\
99.19	0.00418160998596475\\
99.2	0.00423026060940814\\
99.21	0.00427935902655147\\
99.22	0.00432890937690489\\
99.23	0.00437891583817394\\
99.24	0.00442938262661121\\
99.25	0.00448031399737134\\
99.26	0.00453171424486916\\
99.27	0.00458358770314124\\
99.28	0.00463593874621069\\
99.29	0.00468877178845529\\
99.3	0.00474209128497913\\
99.31	0.00479590173198754\\
99.32	0.0048502076671655\\
99.33	0.0049050136700596\\
99.34	0.00496032436246343\\
99.35	0.00501614440880655\\
99.36	0.00507247851654706\\
99.37	0.00512933143656773\\
99.38	0.00518670796357587\\
99.39	0.00524461293650739\\
99.4	0.00530305123896248\\
99.41	0.00536202779961808\\
99.42	0.00542154759264429\\
99.43	0.00548161563812457\\
99.44	0.00554223700248002\\
99.45	0.00560341679889745\\
99.46	0.00566516018776163\\
99.47	0.00572747237709148\\
99.48	0.0057903586229804\\
99.49	0.00585382423004071\\
99.5	0.00591787455185232\\
99.51	0.00598251499141552\\
99.52	0.00604775100160814\\
99.53	0.00611358808564699\\
99.54	0.00618003179755359\\
99.55	0.00624708774262441\\
99.56	0.00631476157790545\\
99.57	0.00638305901267143\\
99.58	0.00645198580890939\\
99.59	0.00652154778180702\\
99.6	0.00659175080024555\\
99.61	0.00666260078729736\\
99.62	0.00673410371788326\\
99.63	0.00680626561604403\\
99.64	0.006879092561375\\
99.65	0.00695259068954072\\
99.66	0.00702676619279462\\
99.67	0.00710162532050356\\
99.68	0.00717717437967758\\
99.69	0.00725341973550465\\
99.7	0.00733036781189061\\
99.71	0.00740802509200435\\
99.72	0.00748639811882837\\
99.73	0.00756549349571853\\
99.74	0.0076453178869692\\
99.75	0.00772587801838046\\
99.76	0.00780718067783092\\
99.77	0.00788923271585624\\
99.78	0.00797204104623336\\
99.79	0.00805561264657055\\
99.8	0.00813995455890341\\
99.81	0.00822507389029681\\
99.82	0.00831097781345295\\
99.83	0.00839767356732553\\
99.84	0.0084851684577402\\
99.85	0.00857346985802142\\
99.86	0.00866258520962567\\
99.87	0.00875252202278125\\
99.88	0.00884328787713478\\
99.89	0.00893489042240442\\
99.9	0.00902733737903997\\
99.91	0.00912063653888998\\
99.92	0.009214795765876\\
99.93	0.00930982299667403\\
99.94	0.00940572624140342\\
99.95	0.00950251358432325\\
99.96	0.00960019318453634\\
99.97	0.00969877327670119\\
99.98	0.00979826217175179\\
99.99	0.00989866825762563\\
100	0.01\\
};
\addlegendentry{$q=-2$};

\addplot [color=blue,dashed,forget plot]
  table[row sep=crcr]{%
0.01	0\\
0.02	0\\
0.03	0\\
0.04	0\\
0.05	0\\
0.06	0\\
0.07	0\\
0.08	0\\
0.09	0\\
0.1	0\\
0.11	0\\
0.12	0\\
0.13	0\\
0.14	0\\
0.15	0\\
0.16	0\\
0.17	0\\
0.18	0\\
0.19	0\\
0.2	0\\
0.21	0\\
0.22	0\\
0.23	0\\
0.24	0\\
0.25	0\\
0.26	0\\
0.27	0\\
0.28	0\\
0.29	0\\
0.3	0\\
0.31	0\\
0.32	0\\
0.33	0\\
0.34	0\\
0.35	0\\
0.36	0\\
0.37	0\\
0.38	0\\
0.39	0\\
0.4	0\\
0.41	0\\
0.42	0\\
0.43	0\\
0.44	0\\
0.45	0\\
0.46	0\\
0.47	0\\
0.48	0\\
0.49	0\\
0.5	0\\
0.51	0\\
0.52	0\\
0.53	0\\
0.54	0\\
0.55	0\\
0.56	0\\
0.57	0\\
0.58	0\\
0.59	0\\
0.6	0\\
0.61	0\\
0.62	0\\
0.63	0\\
0.64	0\\
0.65	0\\
0.66	0\\
0.67	0\\
0.68	0\\
0.69	0\\
0.7	0\\
0.71	0\\
0.72	0\\
0.73	0\\
0.74	0\\
0.75	0\\
0.76	0\\
0.77	0\\
0.78	0\\
0.79	0\\
0.8	0\\
0.81	0\\
0.82	0\\
0.83	0\\
0.84	0\\
0.85	0\\
0.86	0\\
0.87	0\\
0.88	0\\
0.89	0\\
0.9	0\\
0.91	0\\
0.92	0\\
0.93	0\\
0.94	0\\
0.95	0\\
0.96	0\\
0.97	0\\
0.98	0\\
0.99	0\\
1	0\\
1.01	0\\
1.02	0\\
1.03	0\\
1.04	0\\
1.05	0\\
1.06	0\\
1.07	0\\
1.08	0\\
1.09	0\\
1.1	0\\
1.11	0\\
1.12	0\\
1.13	0\\
1.14	0\\
1.15	0\\
1.16	0\\
1.17	0\\
1.18	0\\
1.19	0\\
1.2	0\\
1.21	0\\
1.22	0\\
1.23	0\\
1.24	0\\
1.25	0\\
1.26	0\\
1.27	0\\
1.28	0\\
1.29	0\\
1.3	0\\
1.31	0\\
1.32	0\\
1.33	0\\
1.34	0\\
1.35	0\\
1.36	0\\
1.37	0\\
1.38	0\\
1.39	0\\
1.4	0\\
1.41	0\\
1.42	0\\
1.43	0\\
1.44	0\\
1.45	0\\
1.46	0\\
1.47	0\\
1.48	0\\
1.49	0\\
1.5	0\\
1.51	0\\
1.52	0\\
1.53	0\\
1.54	0\\
1.55	0\\
1.56	0\\
1.57	0\\
1.58	0\\
1.59	0\\
1.6	0\\
1.61	0\\
1.62	0\\
1.63	0\\
1.64	0\\
1.65	0\\
1.66	0\\
1.67	0\\
1.68	0\\
1.69	0\\
1.7	0\\
1.71	0\\
1.72	0\\
1.73	0\\
1.74	0\\
1.75	0\\
1.76	0\\
1.77	0\\
1.78	0\\
1.79	0\\
1.8	0\\
1.81	0\\
1.82	0\\
1.83	0\\
1.84	0\\
1.85	0\\
1.86	0\\
1.87	0\\
1.88	0\\
1.89	0\\
1.9	0\\
1.91	0\\
1.92	0\\
1.93	0\\
1.94	0\\
1.95	0\\
1.96	0\\
1.97	0\\
1.98	0\\
1.99	0\\
2	0\\
2.01	0\\
2.02	0\\
2.03	0\\
2.04	0\\
2.05	0\\
2.06	0\\
2.07	0\\
2.08	0\\
2.09	0\\
2.1	0\\
2.11	0\\
2.12	0\\
2.13	0\\
2.14	0\\
2.15	0\\
2.16	0\\
2.17	0\\
2.18	0\\
2.19	0\\
2.2	0\\
2.21	0\\
2.22	0\\
2.23	0\\
2.24	0\\
2.25	0\\
2.26	0\\
2.27	0\\
2.28	0\\
2.29	0\\
2.3	0\\
2.31	0\\
2.32	0\\
2.33	0\\
2.34	0\\
2.35	0\\
2.36	0\\
2.37	0\\
2.38	0\\
2.39	0\\
2.4	0\\
2.41	0\\
2.42	0\\
2.43	0\\
2.44	0\\
2.45	0\\
2.46	0\\
2.47	0\\
2.48	0\\
2.49	0\\
2.5	0\\
2.51	0\\
2.52	0\\
2.53	0\\
2.54	0\\
2.55	0\\
2.56	0\\
2.57	0\\
2.58	0\\
2.59	0\\
2.6	0\\
2.61	0\\
2.62	0\\
2.63	0\\
2.64	0\\
2.65	0\\
2.66	0\\
2.67	0\\
2.68	0\\
2.69	0\\
2.7	0\\
2.71	0\\
2.72	0\\
2.73	0\\
2.74	0\\
2.75	0\\
2.76	0\\
2.77	0\\
2.78	0\\
2.79	0\\
2.8	0\\
2.81	0\\
2.82	0\\
2.83	0\\
2.84	0\\
2.85	0\\
2.86	0\\
2.87	0\\
2.88	0\\
2.89	0\\
2.9	0\\
2.91	0\\
2.92	0\\
2.93	0\\
2.94	0\\
2.95	0\\
2.96	0\\
2.97	0\\
2.98	0\\
2.99	0\\
3	0\\
3.01	0\\
3.02	0\\
3.03	0\\
3.04	0\\
3.05	0\\
3.06	0\\
3.07	0\\
3.08	0\\
3.09	0\\
3.1	0\\
3.11	0\\
3.12	0\\
3.13	0\\
3.14	0\\
3.15	0\\
3.16	0\\
3.17	0\\
3.18	0\\
3.19	0\\
3.2	0\\
3.21	0\\
3.22	0\\
3.23	0\\
3.24	0\\
3.25	0\\
3.26	0\\
3.27	0\\
3.28	0\\
3.29	0\\
3.3	0\\
3.31	0\\
3.32	0\\
3.33	0\\
3.34	0\\
3.35	0\\
3.36	0\\
3.37	0\\
3.38	0\\
3.39	0\\
3.4	0\\
3.41	0\\
3.42	0\\
3.43	0\\
3.44	0\\
3.45	0\\
3.46	0\\
3.47	0\\
3.48	0\\
3.49	0\\
3.5	0\\
3.51	0\\
3.52	0\\
3.53	0\\
3.54	0\\
3.55	0\\
3.56	0\\
3.57	0\\
3.58	0\\
3.59	0\\
3.6	0\\
3.61	0\\
3.62	0\\
3.63	0\\
3.64	0\\
3.65	0\\
3.66	0\\
3.67	0\\
3.68	0\\
3.69	0\\
3.7	0\\
3.71	0\\
3.72	0\\
3.73	0\\
3.74	0\\
3.75	0\\
3.76	0\\
3.77	0\\
3.78	0\\
3.79	0\\
3.8	0\\
3.81	0\\
3.82	0\\
3.83	0\\
3.84	0\\
3.85	0\\
3.86	0\\
3.87	0\\
3.88	0\\
3.89	0\\
3.9	0\\
3.91	0\\
3.92	0\\
3.93	0\\
3.94	0\\
3.95	0\\
3.96	0\\
3.97	0\\
3.98	0\\
3.99	0\\
4	0\\
4.01	0\\
4.02	0\\
4.03	0\\
4.04	0\\
4.05	0\\
4.06	0\\
4.07	0\\
4.08	0\\
4.09	0\\
4.1	0\\
4.11	0\\
4.12	0\\
4.13	0\\
4.14	0\\
4.15	0\\
4.16	0\\
4.17	0\\
4.18	0\\
4.19	0\\
4.2	0\\
4.21	0\\
4.22	0\\
4.23	0\\
4.24	0\\
4.25	0\\
4.26	0\\
4.27	0\\
4.28	0\\
4.29	0\\
4.3	0\\
4.31	0\\
4.32	0\\
4.33	0\\
4.34	0\\
4.35	0\\
4.36	0\\
4.37	0\\
4.38	0\\
4.39	0\\
4.4	0\\
4.41	0\\
4.42	0\\
4.43	0\\
4.44	0\\
4.45	0\\
4.46	0\\
4.47	0\\
4.48	0\\
4.49	0\\
4.5	0\\
4.51	0\\
4.52	0\\
4.53	0\\
4.54	0\\
4.55	0\\
4.56	0\\
4.57	0\\
4.58	0\\
4.59	0\\
4.6	0\\
4.61	0\\
4.62	0\\
4.63	0\\
4.64	0\\
4.65	0\\
4.66	0\\
4.67	0\\
4.68	0\\
4.69	0\\
4.7	0\\
4.71	0\\
4.72	0\\
4.73	0\\
4.74	0\\
4.75	0\\
4.76	0\\
4.77	0\\
4.78	0\\
4.79	0\\
4.8	0\\
4.81	0\\
4.82	0\\
4.83	0\\
4.84	0\\
4.85	0\\
4.86	0\\
4.87	0\\
4.88	0\\
4.89	0\\
4.9	0\\
4.91	0\\
4.92	0\\
4.93	0\\
4.94	0\\
4.95	0\\
4.96	0\\
4.97	0\\
4.98	0\\
4.99	0\\
5	0\\
5.01	0\\
5.02	0\\
5.03	0\\
5.04	0\\
5.05	0\\
5.06	0\\
5.07	0\\
5.08	0\\
5.09	0\\
5.1	0\\
5.11	0\\
5.12	0\\
5.13	0\\
5.14	0\\
5.15	0\\
5.16	0\\
5.17	0\\
5.18	0\\
5.19	0\\
5.2	0\\
5.21	0\\
5.22	0\\
5.23	0\\
5.24	0\\
5.25	0\\
5.26	0\\
5.27	0\\
5.28	0\\
5.29	0\\
5.3	0\\
5.31	0\\
5.32	0\\
5.33	0\\
5.34	0\\
5.35	0\\
5.36	0\\
5.37	0\\
5.38	0\\
5.39	0\\
5.4	0\\
5.41	0\\
5.42	0\\
5.43	0\\
5.44	0\\
5.45	0\\
5.46	0\\
5.47	0\\
5.48	0\\
5.49	0\\
5.5	0\\
5.51	0\\
5.52	0\\
5.53	0\\
5.54	0\\
5.55	0\\
5.56	0\\
5.57	0\\
5.58	0\\
5.59	0\\
5.6	0\\
5.61	0\\
5.62	0\\
5.63	0\\
5.64	0\\
5.65	0\\
5.66	0\\
5.67	0\\
5.68	0\\
5.69	0\\
5.7	0\\
5.71	0\\
5.72	0\\
5.73	0\\
5.74	0\\
5.75	0\\
5.76	0\\
5.77	0\\
5.78	0\\
5.79	0\\
5.8	0\\
5.81	0\\
5.82	0\\
5.83	0\\
5.84	0\\
5.85	0\\
5.86	0\\
5.87	0\\
5.88	0\\
5.89	0\\
5.9	0\\
5.91	0\\
5.92	0\\
5.93	0\\
5.94	0\\
5.95	0\\
5.96	0\\
5.97	0\\
5.98	0\\
5.99	0\\
6	0\\
6.01	0\\
6.02	0\\
6.03	0\\
6.04	0\\
6.05	0\\
6.06	0\\
6.07	0\\
6.08	0\\
6.09	0\\
6.1	0\\
6.11	0\\
6.12	0\\
6.13	0\\
6.14	0\\
6.15	0\\
6.16	0\\
6.17	0\\
6.18	0\\
6.19	0\\
6.2	0\\
6.21	0\\
6.22	0\\
6.23	0\\
6.24	0\\
6.25	0\\
6.26	0\\
6.27	0\\
6.28	0\\
6.29	0\\
6.3	0\\
6.31	0\\
6.32	0\\
6.33	0\\
6.34	0\\
6.35	0\\
6.36	0\\
6.37	0\\
6.38	0\\
6.39	0\\
6.4	0\\
6.41	0\\
6.42	0\\
6.43	0\\
6.44	0\\
6.45	0\\
6.46	0\\
6.47	0\\
6.48	0\\
6.49	0\\
6.5	0\\
6.51	0\\
6.52	0\\
6.53	0\\
6.54	0\\
6.55	0\\
6.56	0\\
6.57	0\\
6.58	0\\
6.59	0\\
6.6	0\\
6.61	0\\
6.62	0\\
6.63	0\\
6.64	0\\
6.65	0\\
6.66	0\\
6.67	0\\
6.68	0\\
6.69	0\\
6.7	0\\
6.71	0\\
6.72	0\\
6.73	0\\
6.74	0\\
6.75	0\\
6.76	0\\
6.77	0\\
6.78	0\\
6.79	0\\
6.8	0\\
6.81	0\\
6.82	0\\
6.83	0\\
6.84	0\\
6.85	0\\
6.86	0\\
6.87	0\\
6.88	0\\
6.89	0\\
6.9	0\\
6.91	0\\
6.92	0\\
6.93	0\\
6.94	0\\
6.95	0\\
6.96	0\\
6.97	0\\
6.98	0\\
6.99	0\\
7	0\\
7.01	0\\
7.02	0\\
7.03	0\\
7.04	0\\
7.05	0\\
7.06	0\\
7.07	0\\
7.08	0\\
7.09	0\\
7.1	0\\
7.11	0\\
7.12	0\\
7.13	0\\
7.14	0\\
7.15	0\\
7.16	0\\
7.17	0\\
7.18	0\\
7.19	0\\
7.2	0\\
7.21	0\\
7.22	0\\
7.23	0\\
7.24	0\\
7.25	0\\
7.26	0\\
7.27	0\\
7.28	0\\
7.29	0\\
7.3	0\\
7.31	0\\
7.32	0\\
7.33	0\\
7.34	0\\
7.35	0\\
7.36	0\\
7.37	0\\
7.38	0\\
7.39	0\\
7.4	0\\
7.41	0\\
7.42	0\\
7.43	0\\
7.44	0\\
7.45	0\\
7.46	0\\
7.47	0\\
7.48	0\\
7.49	0\\
7.5	0\\
7.51	0\\
7.52	0\\
7.53	0\\
7.54	0\\
7.55	0\\
7.56	0\\
7.57	0\\
7.58	0\\
7.59	0\\
7.6	0\\
7.61	0\\
7.62	0\\
7.63	0\\
7.64	0\\
7.65	0\\
7.66	0\\
7.67	0\\
7.68	0\\
7.69	0\\
7.7	0\\
7.71	0\\
7.72	0\\
7.73	0\\
7.74	0\\
7.75	0\\
7.76	0\\
7.77	0\\
7.78	0\\
7.79	0\\
7.8	0\\
7.81	0\\
7.82	0\\
7.83	0\\
7.84	0\\
7.85	0\\
7.86	0\\
7.87	0\\
7.88	0\\
7.89	0\\
7.9	0\\
7.91	0\\
7.92	0\\
7.93	0\\
7.94	0\\
7.95	0\\
7.96	0\\
7.97	0\\
7.98	0\\
7.99	0\\
8	0\\
8.01	0\\
8.02	0\\
8.03	0\\
8.04	0\\
8.05	0\\
8.06	0\\
8.07	0\\
8.08	0\\
8.09	0\\
8.1	0\\
8.11	0\\
8.12	0\\
8.13	0\\
8.14	0\\
8.15	0\\
8.16	0\\
8.17	0\\
8.18	0\\
8.19	0\\
8.2	0\\
8.21	0\\
8.22	0\\
8.23	0\\
8.24	0\\
8.25	0\\
8.26	0\\
8.27	0\\
8.28	0\\
8.29	0\\
8.3	0\\
8.31	0\\
8.32	0\\
8.33	0\\
8.34	0\\
8.35	0\\
8.36	0\\
8.37	0\\
8.38	0\\
8.39	0\\
8.4	0\\
8.41	0\\
8.42	0\\
8.43	0\\
8.44	0\\
8.45	0\\
8.46	0\\
8.47	0\\
8.48	0\\
8.49	0\\
8.5	0\\
8.51	0\\
8.52	0\\
8.53	0\\
8.54	0\\
8.55	0\\
8.56	0\\
8.57	0\\
8.58	0\\
8.59	0\\
8.6	0\\
8.61	0\\
8.62	0\\
8.63	0\\
8.64	0\\
8.65	0\\
8.66	0\\
8.67	0\\
8.68	0\\
8.69	0\\
8.7	0\\
8.71	0\\
8.72	0\\
8.73	0\\
8.74	0\\
8.75	0\\
8.76	0\\
8.77	0\\
8.78	0\\
8.79	0\\
8.8	0\\
8.81	0\\
8.82	0\\
8.83	0\\
8.84	0\\
8.85	0\\
8.86	0\\
8.87	0\\
8.88	0\\
8.89	0\\
8.9	0\\
8.91	0\\
8.92	0\\
8.93	0\\
8.94	0\\
8.95	0\\
8.96	0\\
8.97	0\\
8.98	0\\
8.99	0\\
9	0\\
9.01	0\\
9.02	0\\
9.03	0\\
9.04	0\\
9.05	0\\
9.06	0\\
9.07	0\\
9.08	0\\
9.09	0\\
9.1	0\\
9.11	0\\
9.12	0\\
9.13	0\\
9.14	0\\
9.15	0\\
9.16	0\\
9.17	0\\
9.18	0\\
9.19	0\\
9.2	0\\
9.21	0\\
9.22	0\\
9.23	0\\
9.24	0\\
9.25	0\\
9.26	0\\
9.27	0\\
9.28	0\\
9.29	0\\
9.3	0\\
9.31	0\\
9.32	0\\
9.33	0\\
9.34	0\\
9.35	0\\
9.36	0\\
9.37	0\\
9.38	0\\
9.39	0\\
9.4	0\\
9.41	0\\
9.42	0\\
9.43	0\\
9.44	0\\
9.45	0\\
9.46	0\\
9.47	0\\
9.48	0\\
9.49	0\\
9.5	0\\
9.51	0\\
9.52	0\\
9.53	0\\
9.54	0\\
9.55	0\\
9.56	0\\
9.57	0\\
9.58	0\\
9.59	0\\
9.6	0\\
9.61	0\\
9.62	0\\
9.63	0\\
9.64	0\\
9.65	0\\
9.66	0\\
9.67	0\\
9.68	0\\
9.69	0\\
9.7	0\\
9.71	0\\
9.72	0\\
9.73	0\\
9.74	0\\
9.75	0\\
9.76	0\\
9.77	0\\
9.78	0\\
9.79	0\\
9.8	0\\
9.81	0\\
9.82	0\\
9.83	0\\
9.84	0\\
9.85	0\\
9.86	0\\
9.87	0\\
9.88	0\\
9.89	0\\
9.9	0\\
9.91	0\\
9.92	0\\
9.93	0\\
9.94	0\\
9.95	0\\
9.96	0\\
9.97	0\\
9.98	0\\
9.99	0\\
10	0\\
10.01	0\\
10.02	0\\
10.03	0\\
10.04	0\\
10.05	0\\
10.06	0\\
10.07	0\\
10.08	0\\
10.09	0\\
10.1	0\\
10.11	0\\
10.12	0\\
10.13	0\\
10.14	0\\
10.15	0\\
10.16	0\\
10.17	0\\
10.18	0\\
10.19	0\\
10.2	0\\
10.21	0\\
10.22	0\\
10.23	0\\
10.24	0\\
10.25	0\\
10.26	0\\
10.27	0\\
10.28	0\\
10.29	0\\
10.3	0\\
10.31	0\\
10.32	0\\
10.33	0\\
10.34	0\\
10.35	0\\
10.36	0\\
10.37	0\\
10.38	0\\
10.39	0\\
10.4	0\\
10.41	0\\
10.42	0\\
10.43	0\\
10.44	0\\
10.45	0\\
10.46	0\\
10.47	0\\
10.48	0\\
10.49	0\\
10.5	0\\
10.51	0\\
10.52	0\\
10.53	0\\
10.54	0\\
10.55	0\\
10.56	0\\
10.57	0\\
10.58	0\\
10.59	0\\
10.6	0\\
10.61	0\\
10.62	0\\
10.63	0\\
10.64	0\\
10.65	0\\
10.66	0\\
10.67	0\\
10.68	0\\
10.69	0\\
10.7	0\\
10.71	0\\
10.72	0\\
10.73	0\\
10.74	0\\
10.75	0\\
10.76	0\\
10.77	0\\
10.78	0\\
10.79	0\\
10.8	0\\
10.81	0\\
10.82	0\\
10.83	0\\
10.84	0\\
10.85	0\\
10.86	0\\
10.87	0\\
10.88	0\\
10.89	0\\
10.9	0\\
10.91	0\\
10.92	0\\
10.93	0\\
10.94	0\\
10.95	0\\
10.96	0\\
10.97	0\\
10.98	0\\
10.99	0\\
11	0\\
11.01	0\\
11.02	0\\
11.03	0\\
11.04	0\\
11.05	0\\
11.06	0\\
11.07	0\\
11.08	0\\
11.09	0\\
11.1	0\\
11.11	0\\
11.12	0\\
11.13	0\\
11.14	0\\
11.15	0\\
11.16	0\\
11.17	0\\
11.18	0\\
11.19	0\\
11.2	0\\
11.21	0\\
11.22	0\\
11.23	0\\
11.24	0\\
11.25	0\\
11.26	0\\
11.27	0\\
11.28	0\\
11.29	0\\
11.3	0\\
11.31	0\\
11.32	0\\
11.33	0\\
11.34	0\\
11.35	0\\
11.36	0\\
11.37	0\\
11.38	0\\
11.39	0\\
11.4	0\\
11.41	0\\
11.42	0\\
11.43	0\\
11.44	0\\
11.45	0\\
11.46	0\\
11.47	0\\
11.48	0\\
11.49	0\\
11.5	0\\
11.51	0\\
11.52	0\\
11.53	0\\
11.54	0\\
11.55	0\\
11.56	0\\
11.57	0\\
11.58	0\\
11.59	0\\
11.6	0\\
11.61	0\\
11.62	0\\
11.63	0\\
11.64	0\\
11.65	0\\
11.66	0\\
11.67	0\\
11.68	0\\
11.69	0\\
11.7	0\\
11.71	0\\
11.72	0\\
11.73	0\\
11.74	0\\
11.75	0\\
11.76	0\\
11.77	0\\
11.78	0\\
11.79	0\\
11.8	0\\
11.81	0\\
11.82	0\\
11.83	0\\
11.84	0\\
11.85	0\\
11.86	0\\
11.87	0\\
11.88	0\\
11.89	0\\
11.9	0\\
11.91	0\\
11.92	0\\
11.93	0\\
11.94	0\\
11.95	0\\
11.96	0\\
11.97	0\\
11.98	0\\
11.99	0\\
12	0\\
12.01	0\\
12.02	0\\
12.03	0\\
12.04	0\\
12.05	0\\
12.06	0\\
12.07	0\\
12.08	0\\
12.09	0\\
12.1	0\\
12.11	0\\
12.12	0\\
12.13	0\\
12.14	0\\
12.15	0\\
12.16	0\\
12.17	0\\
12.18	0\\
12.19	0\\
12.2	0\\
12.21	0\\
12.22	0\\
12.23	0\\
12.24	0\\
12.25	0\\
12.26	0\\
12.27	0\\
12.28	0\\
12.29	0\\
12.3	0\\
12.31	0\\
12.32	0\\
12.33	0\\
12.34	0\\
12.35	0\\
12.36	0\\
12.37	0\\
12.38	0\\
12.39	0\\
12.4	0\\
12.41	0\\
12.42	0\\
12.43	0\\
12.44	0\\
12.45	0\\
12.46	0\\
12.47	0\\
12.48	0\\
12.49	0\\
12.5	0\\
12.51	0\\
12.52	0\\
12.53	0\\
12.54	0\\
12.55	0\\
12.56	0\\
12.57	0\\
12.58	0\\
12.59	0\\
12.6	0\\
12.61	0\\
12.62	0\\
12.63	0\\
12.64	0\\
12.65	0\\
12.66	0\\
12.67	0\\
12.68	0\\
12.69	0\\
12.7	0\\
12.71	0\\
12.72	0\\
12.73	0\\
12.74	0\\
12.75	0\\
12.76	0\\
12.77	0\\
12.78	0\\
12.79	0\\
12.8	0\\
12.81	0\\
12.82	0\\
12.83	0\\
12.84	0\\
12.85	0\\
12.86	0\\
12.87	0\\
12.88	0\\
12.89	0\\
12.9	0\\
12.91	0\\
12.92	0\\
12.93	0\\
12.94	0\\
12.95	0\\
12.96	0\\
12.97	0\\
12.98	0\\
12.99	0\\
13	0\\
13.01	0\\
13.02	0\\
13.03	0\\
13.04	0\\
13.05	0\\
13.06	0\\
13.07	0\\
13.08	0\\
13.09	0\\
13.1	0\\
13.11	0\\
13.12	0\\
13.13	0\\
13.14	0\\
13.15	0\\
13.16	0\\
13.17	0\\
13.18	0\\
13.19	0\\
13.2	0\\
13.21	0\\
13.22	0\\
13.23	0\\
13.24	0\\
13.25	0\\
13.26	0\\
13.27	0\\
13.28	0\\
13.29	0\\
13.3	0\\
13.31	0\\
13.32	0\\
13.33	0\\
13.34	0\\
13.35	0\\
13.36	0\\
13.37	0\\
13.38	0\\
13.39	0\\
13.4	0\\
13.41	0\\
13.42	0\\
13.43	0\\
13.44	0\\
13.45	0\\
13.46	0\\
13.47	0\\
13.48	0\\
13.49	0\\
13.5	0\\
13.51	0\\
13.52	0\\
13.53	0\\
13.54	0\\
13.55	0\\
13.56	0\\
13.57	0\\
13.58	0\\
13.59	0\\
13.6	0\\
13.61	0\\
13.62	0\\
13.63	0\\
13.64	0\\
13.65	0\\
13.66	0\\
13.67	0\\
13.68	0\\
13.69	0\\
13.7	0\\
13.71	0\\
13.72	0\\
13.73	0\\
13.74	0\\
13.75	0\\
13.76	0\\
13.77	0\\
13.78	0\\
13.79	0\\
13.8	0\\
13.81	0\\
13.82	0\\
13.83	0\\
13.84	0\\
13.85	0\\
13.86	0\\
13.87	0\\
13.88	0\\
13.89	0\\
13.9	0\\
13.91	0\\
13.92	0\\
13.93	0\\
13.94	0\\
13.95	0\\
13.96	0\\
13.97	0\\
13.98	0\\
13.99	0\\
14	0\\
14.01	0\\
14.02	0\\
14.03	0\\
14.04	0\\
14.05	0\\
14.06	0\\
14.07	0\\
14.08	0\\
14.09	0\\
14.1	0\\
14.11	0\\
14.12	0\\
14.13	0\\
14.14	0\\
14.15	0\\
14.16	0\\
14.17	0\\
14.18	0\\
14.19	0\\
14.2	0\\
14.21	0\\
14.22	0\\
14.23	0\\
14.24	0\\
14.25	0\\
14.26	0\\
14.27	0\\
14.28	0\\
14.29	0\\
14.3	0\\
14.31	0\\
14.32	0\\
14.33	0\\
14.34	0\\
14.35	0\\
14.36	0\\
14.37	0\\
14.38	0\\
14.39	0\\
14.4	0\\
14.41	0\\
14.42	0\\
14.43	0\\
14.44	0\\
14.45	0\\
14.46	0\\
14.47	0\\
14.48	0\\
14.49	0\\
14.5	0\\
14.51	0\\
14.52	0\\
14.53	0\\
14.54	0\\
14.55	0\\
14.56	0\\
14.57	0\\
14.58	0\\
14.59	0\\
14.6	0\\
14.61	0\\
14.62	0\\
14.63	0\\
14.64	0\\
14.65	0\\
14.66	0\\
14.67	0\\
14.68	0\\
14.69	0\\
14.7	0\\
14.71	0\\
14.72	0\\
14.73	0\\
14.74	0\\
14.75	0\\
14.76	0\\
14.77	0\\
14.78	0\\
14.79	0\\
14.8	0\\
14.81	0\\
14.82	0\\
14.83	0\\
14.84	0\\
14.85	0\\
14.86	0\\
14.87	0\\
14.88	0\\
14.89	0\\
14.9	0\\
14.91	0\\
14.92	0\\
14.93	0\\
14.94	0\\
14.95	0\\
14.96	0\\
14.97	0\\
14.98	0\\
14.99	0\\
15	0\\
15.01	0\\
15.02	0\\
15.03	0\\
15.04	0\\
15.05	0\\
15.06	0\\
15.07	0\\
15.08	0\\
15.09	0\\
15.1	0\\
15.11	0\\
15.12	0\\
15.13	0\\
15.14	0\\
15.15	0\\
15.16	0\\
15.17	0\\
15.18	0\\
15.19	0\\
15.2	0\\
15.21	0\\
15.22	0\\
15.23	0\\
15.24	0\\
15.25	0\\
15.26	0\\
15.27	0\\
15.28	0\\
15.29	0\\
15.3	0\\
15.31	0\\
15.32	0\\
15.33	0\\
15.34	0\\
15.35	0\\
15.36	0\\
15.37	0\\
15.38	0\\
15.39	0\\
15.4	0\\
15.41	0\\
15.42	0\\
15.43	0\\
15.44	0\\
15.45	0\\
15.46	0\\
15.47	0\\
15.48	0\\
15.49	0\\
15.5	0\\
15.51	0\\
15.52	0\\
15.53	0\\
15.54	0\\
15.55	0\\
15.56	0\\
15.57	0\\
15.58	0\\
15.59	0\\
15.6	0\\
15.61	0\\
15.62	0\\
15.63	0\\
15.64	0\\
15.65	0\\
15.66	0\\
15.67	0\\
15.68	0\\
15.69	0\\
15.7	0\\
15.71	0\\
15.72	0\\
15.73	0\\
15.74	0\\
15.75	0\\
15.76	0\\
15.77	0\\
15.78	0\\
15.79	0\\
15.8	0\\
15.81	0\\
15.82	0\\
15.83	0\\
15.84	0\\
15.85	0\\
15.86	0\\
15.87	0\\
15.88	0\\
15.89	0\\
15.9	0\\
15.91	0\\
15.92	0\\
15.93	0\\
15.94	0\\
15.95	0\\
15.96	0\\
15.97	0\\
15.98	0\\
15.99	0\\
16	0\\
16.01	0\\
16.02	0\\
16.03	0\\
16.04	0\\
16.05	0\\
16.06	0\\
16.07	0\\
16.08	0\\
16.09	0\\
16.1	0\\
16.11	0\\
16.12	0\\
16.13	0\\
16.14	0\\
16.15	0\\
16.16	0\\
16.17	0\\
16.18	0\\
16.19	0\\
16.2	0\\
16.21	0\\
16.22	0\\
16.23	0\\
16.24	0\\
16.25	0\\
16.26	0\\
16.27	0\\
16.28	0\\
16.29	0\\
16.3	0\\
16.31	0\\
16.32	0\\
16.33	0\\
16.34	0\\
16.35	0\\
16.36	0\\
16.37	0\\
16.38	0\\
16.39	0\\
16.4	0\\
16.41	0\\
16.42	0\\
16.43	0\\
16.44	0\\
16.45	0\\
16.46	0\\
16.47	0\\
16.48	0\\
16.49	0\\
16.5	0\\
16.51	0\\
16.52	0\\
16.53	0\\
16.54	0\\
16.55	0\\
16.56	0\\
16.57	0\\
16.58	0\\
16.59	0\\
16.6	0\\
16.61	0\\
16.62	0\\
16.63	0\\
16.64	0\\
16.65	0\\
16.66	0\\
16.67	0\\
16.68	0\\
16.69	0\\
16.7	0\\
16.71	0\\
16.72	0\\
16.73	0\\
16.74	0\\
16.75	0\\
16.76	0\\
16.77	0\\
16.78	0\\
16.79	0\\
16.8	0\\
16.81	0\\
16.82	0\\
16.83	0\\
16.84	0\\
16.85	0\\
16.86	0\\
16.87	0\\
16.88	0\\
16.89	0\\
16.9	0\\
16.91	0\\
16.92	0\\
16.93	0\\
16.94	0\\
16.95	0\\
16.96	0\\
16.97	0\\
16.98	0\\
16.99	0\\
17	0\\
17.01	0\\
17.02	0\\
17.03	0\\
17.04	0\\
17.05	0\\
17.06	0\\
17.07	0\\
17.08	0\\
17.09	0\\
17.1	0\\
17.11	0\\
17.12	0\\
17.13	0\\
17.14	0\\
17.15	0\\
17.16	0\\
17.17	0\\
17.18	0\\
17.19	0\\
17.2	0\\
17.21	0\\
17.22	0\\
17.23	0\\
17.24	0\\
17.25	0\\
17.26	0\\
17.27	0\\
17.28	0\\
17.29	0\\
17.3	0\\
17.31	0\\
17.32	0\\
17.33	0\\
17.34	0\\
17.35	0\\
17.36	0\\
17.37	0\\
17.38	0\\
17.39	0\\
17.4	0\\
17.41	0\\
17.42	0\\
17.43	0\\
17.44	0\\
17.45	0\\
17.46	0\\
17.47	0\\
17.48	0\\
17.49	0\\
17.5	0\\
17.51	0\\
17.52	0\\
17.53	0\\
17.54	0\\
17.55	0\\
17.56	0\\
17.57	0\\
17.58	0\\
17.59	0\\
17.6	0\\
17.61	0\\
17.62	0\\
17.63	0\\
17.64	0\\
17.65	0\\
17.66	0\\
17.67	0\\
17.68	0\\
17.69	0\\
17.7	0\\
17.71	0\\
17.72	0\\
17.73	0\\
17.74	0\\
17.75	0\\
17.76	0\\
17.77	0\\
17.78	0\\
17.79	0\\
17.8	0\\
17.81	0\\
17.82	0\\
17.83	0\\
17.84	0\\
17.85	0\\
17.86	0\\
17.87	0\\
17.88	0\\
17.89	0\\
17.9	0\\
17.91	0\\
17.92	0\\
17.93	0\\
17.94	0\\
17.95	0\\
17.96	0\\
17.97	0\\
17.98	0\\
17.99	0\\
18	0\\
18.01	0\\
18.02	0\\
18.03	0\\
18.04	0\\
18.05	0\\
18.06	0\\
18.07	0\\
18.08	0\\
18.09	0\\
18.1	0\\
18.11	0\\
18.12	0\\
18.13	0\\
18.14	0\\
18.15	0\\
18.16	0\\
18.17	0\\
18.18	0\\
18.19	0\\
18.2	0\\
18.21	0\\
18.22	0\\
18.23	0\\
18.24	0\\
18.25	0\\
18.26	0\\
18.27	0\\
18.28	0\\
18.29	0\\
18.3	0\\
18.31	0\\
18.32	0\\
18.33	0\\
18.34	0\\
18.35	0\\
18.36	0\\
18.37	0\\
18.38	0\\
18.39	0\\
18.4	0\\
18.41	0\\
18.42	0\\
18.43	0\\
18.44	0\\
18.45	0\\
18.46	0\\
18.47	0\\
18.48	0\\
18.49	0\\
18.5	0\\
18.51	0\\
18.52	0\\
18.53	0\\
18.54	0\\
18.55	0\\
18.56	0\\
18.57	0\\
18.58	0\\
18.59	0\\
18.6	0\\
18.61	0\\
18.62	0\\
18.63	0\\
18.64	0\\
18.65	0\\
18.66	0\\
18.67	0\\
18.68	0\\
18.69	0\\
18.7	0\\
18.71	0\\
18.72	0\\
18.73	0\\
18.74	0\\
18.75	0\\
18.76	0\\
18.77	0\\
18.78	0\\
18.79	0\\
18.8	0\\
18.81	0\\
18.82	0\\
18.83	0\\
18.84	0\\
18.85	0\\
18.86	0\\
18.87	0\\
18.88	0\\
18.89	0\\
18.9	0\\
18.91	0\\
18.92	0\\
18.93	0\\
18.94	0\\
18.95	0\\
18.96	0\\
18.97	0\\
18.98	0\\
18.99	0\\
19	0\\
19.01	0\\
19.02	0\\
19.03	0\\
19.04	0\\
19.05	0\\
19.06	0\\
19.07	0\\
19.08	0\\
19.09	0\\
19.1	0\\
19.11	0\\
19.12	0\\
19.13	0\\
19.14	0\\
19.15	0\\
19.16	0\\
19.17	0\\
19.18	0\\
19.19	0\\
19.2	0\\
19.21	0\\
19.22	0\\
19.23	0\\
19.24	0\\
19.25	0\\
19.26	0\\
19.27	0\\
19.28	0\\
19.29	0\\
19.3	0\\
19.31	0\\
19.32	0\\
19.33	0\\
19.34	0\\
19.35	0\\
19.36	0\\
19.37	0\\
19.38	0\\
19.39	0\\
19.4	0\\
19.41	0\\
19.42	0\\
19.43	0\\
19.44	0\\
19.45	0\\
19.46	0\\
19.47	0\\
19.48	0\\
19.49	0\\
19.5	0\\
19.51	0\\
19.52	0\\
19.53	0\\
19.54	0\\
19.55	0\\
19.56	0\\
19.57	0\\
19.58	0\\
19.59	0\\
19.6	0\\
19.61	0\\
19.62	0\\
19.63	0\\
19.64	0\\
19.65	0\\
19.66	0\\
19.67	0\\
19.68	0\\
19.69	0\\
19.7	0\\
19.71	0\\
19.72	0\\
19.73	0\\
19.74	0\\
19.75	0\\
19.76	0\\
19.77	0\\
19.78	0\\
19.79	0\\
19.8	0\\
19.81	0\\
19.82	0\\
19.83	0\\
19.84	0\\
19.85	0\\
19.86	0\\
19.87	0\\
19.88	0\\
19.89	0\\
19.9	0\\
19.91	0\\
19.92	0\\
19.93	0\\
19.94	0\\
19.95	0\\
19.96	0\\
19.97	0\\
19.98	0\\
19.99	0\\
20	0\\
20.01	0\\
20.02	0\\
20.03	0\\
20.04	0\\
20.05	0\\
20.06	0\\
20.07	0\\
20.08	0\\
20.09	0\\
20.1	0\\
20.11	0\\
20.12	0\\
20.13	0\\
20.14	0\\
20.15	0\\
20.16	0\\
20.17	0\\
20.18	0\\
20.19	0\\
20.2	0\\
20.21	0\\
20.22	0\\
20.23	0\\
20.24	0\\
20.25	0\\
20.26	0\\
20.27	0\\
20.28	0\\
20.29	0\\
20.3	0\\
20.31	0\\
20.32	0\\
20.33	0\\
20.34	0\\
20.35	0\\
20.36	0\\
20.37	0\\
20.38	0\\
20.39	0\\
20.4	0\\
20.41	0\\
20.42	0\\
20.43	0\\
20.44	0\\
20.45	0\\
20.46	0\\
20.47	0\\
20.48	0\\
20.49	0\\
20.5	0\\
20.51	0\\
20.52	0\\
20.53	0\\
20.54	0\\
20.55	0\\
20.56	0\\
20.57	0\\
20.58	0\\
20.59	0\\
20.6	0\\
20.61	0\\
20.62	0\\
20.63	0\\
20.64	0\\
20.65	0\\
20.66	0\\
20.67	0\\
20.68	0\\
20.69	0\\
20.7	0\\
20.71	0\\
20.72	0\\
20.73	0\\
20.74	0\\
20.75	0\\
20.76	0\\
20.77	0\\
20.78	0\\
20.79	0\\
20.8	0\\
20.81	0\\
20.82	0\\
20.83	0\\
20.84	0\\
20.85	0\\
20.86	0\\
20.87	0\\
20.88	0\\
20.89	0\\
20.9	0\\
20.91	0\\
20.92	0\\
20.93	0\\
20.94	0\\
20.95	0\\
20.96	0\\
20.97	0\\
20.98	0\\
20.99	0\\
21	0\\
21.01	0\\
21.02	0\\
21.03	0\\
21.04	0\\
21.05	0\\
21.06	0\\
21.07	0\\
21.08	0\\
21.09	0\\
21.1	0\\
21.11	0\\
21.12	0\\
21.13	0\\
21.14	0\\
21.15	0\\
21.16	0\\
21.17	0\\
21.18	0\\
21.19	0\\
21.2	0\\
21.21	0\\
21.22	0\\
21.23	0\\
21.24	0\\
21.25	0\\
21.26	0\\
21.27	0\\
21.28	0\\
21.29	0\\
21.3	0\\
21.31	0\\
21.32	0\\
21.33	0\\
21.34	0\\
21.35	0\\
21.36	0\\
21.37	0\\
21.38	0\\
21.39	0\\
21.4	0\\
21.41	0\\
21.42	0\\
21.43	0\\
21.44	0\\
21.45	0\\
21.46	0\\
21.47	0\\
21.48	0\\
21.49	0\\
21.5	0\\
21.51	0\\
21.52	0\\
21.53	0\\
21.54	0\\
21.55	0\\
21.56	0\\
21.57	0\\
21.58	0\\
21.59	0\\
21.6	0\\
21.61	0\\
21.62	0\\
21.63	0\\
21.64	0\\
21.65	0\\
21.66	0\\
21.67	0\\
21.68	0\\
21.69	0\\
21.7	0\\
21.71	0\\
21.72	0\\
21.73	0\\
21.74	0\\
21.75	0\\
21.76	0\\
21.77	0\\
21.78	0\\
21.79	0\\
21.8	0\\
21.81	0\\
21.82	0\\
21.83	0\\
21.84	0\\
21.85	0\\
21.86	0\\
21.87	0\\
21.88	0\\
21.89	0\\
21.9	0\\
21.91	0\\
21.92	0\\
21.93	0\\
21.94	0\\
21.95	0\\
21.96	0\\
21.97	0\\
21.98	0\\
21.99	0\\
22	0\\
22.01	0\\
22.02	0\\
22.03	0\\
22.04	0\\
22.05	0\\
22.06	0\\
22.07	0\\
22.08	0\\
22.09	0\\
22.1	0\\
22.11	0\\
22.12	0\\
22.13	0\\
22.14	0\\
22.15	0\\
22.16	0\\
22.17	0\\
22.18	0\\
22.19	0\\
22.2	0\\
22.21	0\\
22.22	0\\
22.23	0\\
22.24	0\\
22.25	0\\
22.26	0\\
22.27	0\\
22.28	0\\
22.29	0\\
22.3	0\\
22.31	0\\
22.32	0\\
22.33	0\\
22.34	0\\
22.35	0\\
22.36	0\\
22.37	0\\
22.38	0\\
22.39	0\\
22.4	0\\
22.41	0\\
22.42	0\\
22.43	0\\
22.44	0\\
22.45	0\\
22.46	0\\
22.47	0\\
22.48	0\\
22.49	0\\
22.5	0\\
22.51	0\\
22.52	0\\
22.53	0\\
22.54	0\\
22.55	0\\
22.56	0\\
22.57	0\\
22.58	0\\
22.59	0\\
22.6	0\\
22.61	0\\
22.62	0\\
22.63	0\\
22.64	0\\
22.65	0\\
22.66	0\\
22.67	0\\
22.68	0\\
22.69	0\\
22.7	0\\
22.71	0\\
22.72	0\\
22.73	0\\
22.74	0\\
22.75	0\\
22.76	0\\
22.77	0\\
22.78	0\\
22.79	0\\
22.8	0\\
22.81	0\\
22.82	0\\
22.83	0\\
22.84	0\\
22.85	0\\
22.86	0\\
22.87	0\\
22.88	0\\
22.89	0\\
22.9	0\\
22.91	0\\
22.92	0\\
22.93	0\\
22.94	0\\
22.95	0\\
22.96	0\\
22.97	0\\
22.98	0\\
22.99	0\\
23	0\\
23.01	0\\
23.02	0\\
23.03	0\\
23.04	0\\
23.05	0\\
23.06	0\\
23.07	0\\
23.08	0\\
23.09	0\\
23.1	0\\
23.11	0\\
23.12	0\\
23.13	0\\
23.14	0\\
23.15	0\\
23.16	0\\
23.17	0\\
23.18	0\\
23.19	0\\
23.2	0\\
23.21	0\\
23.22	0\\
23.23	0\\
23.24	0\\
23.25	0\\
23.26	0\\
23.27	0\\
23.28	0\\
23.29	0\\
23.3	0\\
23.31	0\\
23.32	0\\
23.33	0\\
23.34	0\\
23.35	0\\
23.36	0\\
23.37	0\\
23.38	0\\
23.39	0\\
23.4	0\\
23.41	0\\
23.42	0\\
23.43	0\\
23.44	0\\
23.45	0\\
23.46	0\\
23.47	0\\
23.48	0\\
23.49	0\\
23.5	0\\
23.51	0\\
23.52	0\\
23.53	0\\
23.54	0\\
23.55	0\\
23.56	0\\
23.57	0\\
23.58	0\\
23.59	0\\
23.6	0\\
23.61	0\\
23.62	0\\
23.63	0\\
23.64	0\\
23.65	0\\
23.66	0\\
23.67	0\\
23.68	0\\
23.69	0\\
23.7	0\\
23.71	0\\
23.72	0\\
23.73	0\\
23.74	0\\
23.75	0\\
23.76	0\\
23.77	0\\
23.78	0\\
23.79	0\\
23.8	0\\
23.81	0\\
23.82	0\\
23.83	0\\
23.84	0\\
23.85	0\\
23.86	0\\
23.87	0\\
23.88	0\\
23.89	0\\
23.9	0\\
23.91	0\\
23.92	0\\
23.93	0\\
23.94	0\\
23.95	0\\
23.96	0\\
23.97	0\\
23.98	0\\
23.99	0\\
24	0\\
24.01	0\\
24.02	0\\
24.03	0\\
24.04	0\\
24.05	0\\
24.06	0\\
24.07	0\\
24.08	0\\
24.09	0\\
24.1	0\\
24.11	0\\
24.12	0\\
24.13	0\\
24.14	0\\
24.15	0\\
24.16	0\\
24.17	0\\
24.18	0\\
24.19	0\\
24.2	0\\
24.21	0\\
24.22	0\\
24.23	0\\
24.24	0\\
24.25	0\\
24.26	0\\
24.27	0\\
24.28	0\\
24.29	0\\
24.3	0\\
24.31	0\\
24.32	0\\
24.33	0\\
24.34	0\\
24.35	0\\
24.36	0\\
24.37	0\\
24.38	0\\
24.39	0\\
24.4	0\\
24.41	0\\
24.42	0\\
24.43	0\\
24.44	0\\
24.45	0\\
24.46	0\\
24.47	0\\
24.48	0\\
24.49	0\\
24.5	0\\
24.51	0\\
24.52	0\\
24.53	0\\
24.54	0\\
24.55	0\\
24.56	0\\
24.57	0\\
24.58	0\\
24.59	0\\
24.6	0\\
24.61	0\\
24.62	0\\
24.63	0\\
24.64	0\\
24.65	0\\
24.66	0\\
24.67	0\\
24.68	0\\
24.69	0\\
24.7	0\\
24.71	0\\
24.72	0\\
24.73	0\\
24.74	0\\
24.75	0\\
24.76	0\\
24.77	0\\
24.78	0\\
24.79	0\\
24.8	0\\
24.81	0\\
24.82	0\\
24.83	0\\
24.84	0\\
24.85	0\\
24.86	0\\
24.87	0\\
24.88	0\\
24.89	0\\
24.9	0\\
24.91	0\\
24.92	0\\
24.93	0\\
24.94	0\\
24.95	0\\
24.96	0\\
24.97	0\\
24.98	0\\
24.99	0\\
25	0\\
25.01	0\\
25.02	0\\
25.03	0\\
25.04	0\\
25.05	0\\
25.06	0\\
25.07	0\\
25.08	0\\
25.09	0\\
25.1	0\\
25.11	0\\
25.12	0\\
25.13	0\\
25.14	0\\
25.15	0\\
25.16	0\\
25.17	0\\
25.18	0\\
25.19	0\\
25.2	0\\
25.21	0\\
25.22	0\\
25.23	0\\
25.24	0\\
25.25	0\\
25.26	0\\
25.27	0\\
25.28	0\\
25.29	0\\
25.3	0\\
25.31	0\\
25.32	0\\
25.33	0\\
25.34	0\\
25.35	0\\
25.36	0\\
25.37	0\\
25.38	0\\
25.39	0\\
25.4	0\\
25.41	0\\
25.42	0\\
25.43	0\\
25.44	0\\
25.45	0\\
25.46	0\\
25.47	0\\
25.48	0\\
25.49	0\\
25.5	0\\
25.51	0\\
25.52	0\\
25.53	0\\
25.54	0\\
25.55	0\\
25.56	0\\
25.57	0\\
25.58	0\\
25.59	0\\
25.6	0\\
25.61	0\\
25.62	0\\
25.63	0\\
25.64	0\\
25.65	0\\
25.66	0\\
25.67	0\\
25.68	0\\
25.69	0\\
25.7	0\\
25.71	0\\
25.72	0\\
25.73	0\\
25.74	0\\
25.75	0\\
25.76	0\\
25.77	0\\
25.78	0\\
25.79	0\\
25.8	0\\
25.81	0\\
25.82	0\\
25.83	0\\
25.84	0\\
25.85	0\\
25.86	0\\
25.87	0\\
25.88	0\\
25.89	0\\
25.9	0\\
25.91	0\\
25.92	0\\
25.93	0\\
25.94	0\\
25.95	0\\
25.96	0\\
25.97	0\\
25.98	0\\
25.99	0\\
26	0\\
26.01	0\\
26.02	0\\
26.03	0\\
26.04	0\\
26.05	0\\
26.06	0\\
26.07	0\\
26.08	0\\
26.09	0\\
26.1	0\\
26.11	0\\
26.12	0\\
26.13	0\\
26.14	0\\
26.15	0\\
26.16	0\\
26.17	0\\
26.18	0\\
26.19	0\\
26.2	0\\
26.21	0\\
26.22	0\\
26.23	0\\
26.24	0\\
26.25	0\\
26.26	0\\
26.27	0\\
26.28	0\\
26.29	0\\
26.3	0\\
26.31	0\\
26.32	0\\
26.33	0\\
26.34	0\\
26.35	0\\
26.36	0\\
26.37	0\\
26.38	0\\
26.39	0\\
26.4	0\\
26.41	0\\
26.42	0\\
26.43	0\\
26.44	0\\
26.45	0\\
26.46	0\\
26.47	0\\
26.48	0\\
26.49	0\\
26.5	0\\
26.51	0\\
26.52	0\\
26.53	0\\
26.54	0\\
26.55	0\\
26.56	0\\
26.57	0\\
26.58	0\\
26.59	0\\
26.6	0\\
26.61	0\\
26.62	0\\
26.63	0\\
26.64	0\\
26.65	0\\
26.66	0\\
26.67	0\\
26.68	0\\
26.69	0\\
26.7	0\\
26.71	0\\
26.72	0\\
26.73	0\\
26.74	0\\
26.75	0\\
26.76	0\\
26.77	0\\
26.78	0\\
26.79	0\\
26.8	0\\
26.81	0\\
26.82	0\\
26.83	0\\
26.84	0\\
26.85	0\\
26.86	0\\
26.87	0\\
26.88	0\\
26.89	0\\
26.9	0\\
26.91	0\\
26.92	0\\
26.93	0\\
26.94	0\\
26.95	0\\
26.96	0\\
26.97	0\\
26.98	0\\
26.99	0\\
27	0\\
27.01	0\\
27.02	0\\
27.03	0\\
27.04	0\\
27.05	0\\
27.06	0\\
27.07	0\\
27.08	0\\
27.09	0\\
27.1	0\\
27.11	0\\
27.12	0\\
27.13	0\\
27.14	0\\
27.15	0\\
27.16	0\\
27.17	0\\
27.18	0\\
27.19	0\\
27.2	0\\
27.21	0\\
27.22	0\\
27.23	0\\
27.24	0\\
27.25	0\\
27.26	0\\
27.27	0\\
27.28	0\\
27.29	0\\
27.3	0\\
27.31	0\\
27.32	0\\
27.33	0\\
27.34	0\\
27.35	0\\
27.36	0\\
27.37	0\\
27.38	0\\
27.39	0\\
27.4	0\\
27.41	0\\
27.42	0\\
27.43	0\\
27.44	0\\
27.45	0\\
27.46	0\\
27.47	0\\
27.48	0\\
27.49	0\\
27.5	0\\
27.51	0\\
27.52	0\\
27.53	0\\
27.54	0\\
27.55	0\\
27.56	0\\
27.57	0\\
27.58	0\\
27.59	0\\
27.6	0\\
27.61	0\\
27.62	0\\
27.63	0\\
27.64	0\\
27.65	0\\
27.66	0\\
27.67	0\\
27.68	0\\
27.69	0\\
27.7	0\\
27.71	0\\
27.72	0\\
27.73	0\\
27.74	0\\
27.75	0\\
27.76	0\\
27.77	0\\
27.78	0\\
27.79	0\\
27.8	0\\
27.81	0\\
27.82	0\\
27.83	0\\
27.84	0\\
27.85	0\\
27.86	0\\
27.87	0\\
27.88	0\\
27.89	0\\
27.9	0\\
27.91	0\\
27.92	0\\
27.93	0\\
27.94	0\\
27.95	0\\
27.96	0\\
27.97	0\\
27.98	0\\
27.99	0\\
28	0\\
28.01	0\\
28.02	0\\
28.03	0\\
28.04	0\\
28.05	0\\
28.06	0\\
28.07	0\\
28.08	0\\
28.09	0\\
28.1	0\\
28.11	0\\
28.12	0\\
28.13	0\\
28.14	0\\
28.15	0\\
28.16	0\\
28.17	0\\
28.18	0\\
28.19	0\\
28.2	0\\
28.21	0\\
28.22	0\\
28.23	0\\
28.24	0\\
28.25	0\\
28.26	0\\
28.27	0\\
28.28	0\\
28.29	0\\
28.3	0\\
28.31	0\\
28.32	0\\
28.33	0\\
28.34	0\\
28.35	0\\
28.36	0\\
28.37	0\\
28.38	0\\
28.39	0\\
28.4	0\\
28.41	0\\
28.42	0\\
28.43	0\\
28.44	0\\
28.45	0\\
28.46	0\\
28.47	0\\
28.48	0\\
28.49	0\\
28.5	0\\
28.51	0\\
28.52	0\\
28.53	0\\
28.54	0\\
28.55	0\\
28.56	0\\
28.57	0\\
28.58	0\\
28.59	0\\
28.6	0\\
28.61	0\\
28.62	0\\
28.63	0\\
28.64	0\\
28.65	0\\
28.66	0\\
28.67	0\\
28.68	0\\
28.69	0\\
28.7	0\\
28.71	0\\
28.72	0\\
28.73	0\\
28.74	0\\
28.75	0\\
28.76	0\\
28.77	0\\
28.78	0\\
28.79	0\\
28.8	0\\
28.81	0\\
28.82	0\\
28.83	0\\
28.84	0\\
28.85	0\\
28.86	0\\
28.87	0\\
28.88	0\\
28.89	0\\
28.9	0\\
28.91	0\\
28.92	0\\
28.93	0\\
28.94	0\\
28.95	0\\
28.96	0\\
28.97	0\\
28.98	0\\
28.99	0\\
29	0\\
29.01	0\\
29.02	0\\
29.03	0\\
29.04	0\\
29.05	0\\
29.06	0\\
29.07	0\\
29.08	0\\
29.09	0\\
29.1	0\\
29.11	0\\
29.12	0\\
29.13	0\\
29.14	0\\
29.15	0\\
29.16	0\\
29.17	0\\
29.18	0\\
29.19	0\\
29.2	0\\
29.21	0\\
29.22	0\\
29.23	0\\
29.24	0\\
29.25	0\\
29.26	0\\
29.27	0\\
29.28	0\\
29.29	0\\
29.3	0\\
29.31	0\\
29.32	0\\
29.33	0\\
29.34	0\\
29.35	0\\
29.36	0\\
29.37	0\\
29.38	0\\
29.39	0\\
29.4	0\\
29.41	0\\
29.42	0\\
29.43	0\\
29.44	0\\
29.45	0\\
29.46	0\\
29.47	0\\
29.48	0\\
29.49	0\\
29.5	0\\
29.51	0\\
29.52	0\\
29.53	0\\
29.54	0\\
29.55	0\\
29.56	0\\
29.57	0\\
29.58	0\\
29.59	0\\
29.6	0\\
29.61	0\\
29.62	0\\
29.63	0\\
29.64	0\\
29.65	0\\
29.66	0\\
29.67	0\\
29.68	0\\
29.69	0\\
29.7	0\\
29.71	0\\
29.72	0\\
29.73	0\\
29.74	0\\
29.75	0\\
29.76	0\\
29.77	0\\
29.78	0\\
29.79	0\\
29.8	0\\
29.81	0\\
29.82	0\\
29.83	0\\
29.84	0\\
29.85	0\\
29.86	0\\
29.87	0\\
29.88	0\\
29.89	0\\
29.9	0\\
29.91	0\\
29.92	0\\
29.93	0\\
29.94	0\\
29.95	0\\
29.96	0\\
29.97	0\\
29.98	0\\
29.99	0\\
30	0\\
30.01	0\\
30.02	0\\
30.03	0\\
30.04	0\\
30.05	0\\
30.06	0\\
30.07	0\\
30.08	0\\
30.09	0\\
30.1	0\\
30.11	0\\
30.12	0\\
30.13	0\\
30.14	0\\
30.15	0\\
30.16	0\\
30.17	0\\
30.18	0\\
30.19	0\\
30.2	0\\
30.21	0\\
30.22	0\\
30.23	0\\
30.24	0\\
30.25	0\\
30.26	0\\
30.27	0\\
30.28	0\\
30.29	0\\
30.3	0\\
30.31	0\\
30.32	0\\
30.33	0\\
30.34	0\\
30.35	0\\
30.36	0\\
30.37	0\\
30.38	0\\
30.39	0\\
30.4	0\\
30.41	0\\
30.42	0\\
30.43	0\\
30.44	0\\
30.45	0\\
30.46	0\\
30.47	0\\
30.48	0\\
30.49	0\\
30.5	0\\
30.51	0\\
30.52	0\\
30.53	0\\
30.54	0\\
30.55	0\\
30.56	0\\
30.57	0\\
30.58	0\\
30.59	0\\
30.6	0\\
30.61	0\\
30.62	0\\
30.63	0\\
30.64	0\\
30.65	0\\
30.66	0\\
30.67	0\\
30.68	0\\
30.69	0\\
30.7	0\\
30.71	0\\
30.72	0\\
30.73	0\\
30.74	0\\
30.75	0\\
30.76	0\\
30.77	0\\
30.78	0\\
30.79	0\\
30.8	0\\
30.81	0\\
30.82	0\\
30.83	0\\
30.84	0\\
30.85	0\\
30.86	0\\
30.87	0\\
30.88	0\\
30.89	0\\
30.9	0\\
30.91	0\\
30.92	0\\
30.93	0\\
30.94	0\\
30.95	0\\
30.96	0\\
30.97	0\\
30.98	0\\
30.99	0\\
31	0\\
31.01	0\\
31.02	0\\
31.03	0\\
31.04	0\\
31.05	0\\
31.06	0\\
31.07	0\\
31.08	0\\
31.09	0\\
31.1	0\\
31.11	0\\
31.12	0\\
31.13	0\\
31.14	0\\
31.15	0\\
31.16	0\\
31.17	0\\
31.18	0\\
31.19	0\\
31.2	0\\
31.21	0\\
31.22	0\\
31.23	0\\
31.24	0\\
31.25	0\\
31.26	0\\
31.27	0\\
31.28	0\\
31.29	0\\
31.3	0\\
31.31	0\\
31.32	0\\
31.33	0\\
31.34	0\\
31.35	0\\
31.36	0\\
31.37	0\\
31.38	0\\
31.39	0\\
31.4	0\\
31.41	0\\
31.42	0\\
31.43	0\\
31.44	0\\
31.45	0\\
31.46	0\\
31.47	0\\
31.48	0\\
31.49	0\\
31.5	0\\
31.51	0\\
31.52	0\\
31.53	0\\
31.54	0\\
31.55	0\\
31.56	0\\
31.57	0\\
31.58	0\\
31.59	0\\
31.6	0\\
31.61	0\\
31.62	0\\
31.63	0\\
31.64	0\\
31.65	0\\
31.66	0\\
31.67	0\\
31.68	0\\
31.69	0\\
31.7	0\\
31.71	0\\
31.72	0\\
31.73	0\\
31.74	0\\
31.75	0\\
31.76	0\\
31.77	0\\
31.78	0\\
31.79	0\\
31.8	0\\
31.81	0\\
31.82	0\\
31.83	0\\
31.84	0\\
31.85	0\\
31.86	0\\
31.87	0\\
31.88	0\\
31.89	0\\
31.9	0\\
31.91	0\\
31.92	0\\
31.93	0\\
31.94	0\\
31.95	0\\
31.96	0\\
31.97	0\\
31.98	0\\
31.99	0\\
32	0\\
32.01	0\\
32.02	0\\
32.03	0\\
32.04	0\\
32.05	0\\
32.06	0\\
32.07	0\\
32.08	0\\
32.09	0\\
32.1	0\\
32.11	0\\
32.12	0\\
32.13	0\\
32.14	0\\
32.15	0\\
32.16	0\\
32.17	0\\
32.18	0\\
32.19	0\\
32.2	0\\
32.21	0\\
32.22	0\\
32.23	0\\
32.24	0\\
32.25	0\\
32.26	0\\
32.27	0\\
32.28	0\\
32.29	0\\
32.3	0\\
32.31	0\\
32.32	0\\
32.33	0\\
32.34	0\\
32.35	0\\
32.36	0\\
32.37	0\\
32.38	0\\
32.39	0\\
32.4	0\\
32.41	0\\
32.42	0\\
32.43	0\\
32.44	0\\
32.45	0\\
32.46	0\\
32.47	0\\
32.48	0\\
32.49	0\\
32.5	0\\
32.51	0\\
32.52	0\\
32.53	0\\
32.54	0\\
32.55	0\\
32.56	0\\
32.57	0\\
32.58	0\\
32.59	0\\
32.6	0\\
32.61	0\\
32.62	0\\
32.63	0\\
32.64	0\\
32.65	0\\
32.66	0\\
32.67	0\\
32.68	0\\
32.69	0\\
32.7	0\\
32.71	0\\
32.72	0\\
32.73	0\\
32.74	0\\
32.75	0\\
32.76	0\\
32.77	0\\
32.78	0\\
32.79	0\\
32.8	0\\
32.81	0\\
32.82	0\\
32.83	0\\
32.84	0\\
32.85	0\\
32.86	0\\
32.87	0\\
32.88	0\\
32.89	0\\
32.9	0\\
32.91	0\\
32.92	0\\
32.93	0\\
32.94	0\\
32.95	0\\
32.96	0\\
32.97	0\\
32.98	0\\
32.99	0\\
33	0\\
33.01	0\\
33.02	0\\
33.03	0\\
33.04	0\\
33.05	0\\
33.06	0\\
33.07	0\\
33.08	0\\
33.09	0\\
33.1	0\\
33.11	0\\
33.12	0\\
33.13	0\\
33.14	0\\
33.15	0\\
33.16	0\\
33.17	0\\
33.18	0\\
33.19	0\\
33.2	0\\
33.21	0\\
33.22	0\\
33.23	0\\
33.24	0\\
33.25	0\\
33.26	0\\
33.27	0\\
33.28	0\\
33.29	0\\
33.3	0\\
33.31	0\\
33.32	0\\
33.33	0\\
33.34	0\\
33.35	0\\
33.36	0\\
33.37	0\\
33.38	0\\
33.39	0\\
33.4	0\\
33.41	0\\
33.42	0\\
33.43	0\\
33.44	0\\
33.45	0\\
33.46	0\\
33.47	0\\
33.48	0\\
33.49	0\\
33.5	0\\
33.51	0\\
33.52	0\\
33.53	0\\
33.54	0\\
33.55	0\\
33.56	0\\
33.57	0\\
33.58	0\\
33.59	0\\
33.6	0\\
33.61	0\\
33.62	0\\
33.63	0\\
33.64	0\\
33.65	0\\
33.66	0\\
33.67	0\\
33.68	0\\
33.69	0\\
33.7	0\\
33.71	0\\
33.72	0\\
33.73	0\\
33.74	0\\
33.75	0\\
33.76	0\\
33.77	0\\
33.78	0\\
33.79	0\\
33.8	0\\
33.81	0\\
33.82	0\\
33.83	0\\
33.84	0\\
33.85	0\\
33.86	0\\
33.87	0\\
33.88	0\\
33.89	0\\
33.9	0\\
33.91	0\\
33.92	0\\
33.93	0\\
33.94	0\\
33.95	0\\
33.96	0\\
33.97	0\\
33.98	0\\
33.99	0\\
34	0\\
34.01	0\\
34.02	0\\
34.03	0\\
34.04	0\\
34.05	0\\
34.06	0\\
34.07	0\\
34.08	0\\
34.09	0\\
34.1	0\\
34.11	0\\
34.12	0\\
34.13	0\\
34.14	0\\
34.15	0\\
34.16	0\\
34.17	0\\
34.18	0\\
34.19	0\\
34.2	0\\
34.21	0\\
34.22	0\\
34.23	0\\
34.24	0\\
34.25	0\\
34.26	0\\
34.27	0\\
34.28	0\\
34.29	0\\
34.3	0\\
34.31	0\\
34.32	0\\
34.33	0\\
34.34	0\\
34.35	0\\
34.36	0\\
34.37	0\\
34.38	0\\
34.39	0\\
34.4	0\\
34.41	0\\
34.42	0\\
34.43	0\\
34.44	0\\
34.45	0\\
34.46	0\\
34.47	0\\
34.48	0\\
34.49	0\\
34.5	0\\
34.51	0\\
34.52	0\\
34.53	0\\
34.54	0\\
34.55	0\\
34.56	0\\
34.57	0\\
34.58	0\\
34.59	0\\
34.6	0\\
34.61	0\\
34.62	0\\
34.63	0\\
34.64	0\\
34.65	0\\
34.66	0\\
34.67	0\\
34.68	0\\
34.69	0\\
34.7	0\\
34.71	0\\
34.72	0\\
34.73	0\\
34.74	0\\
34.75	0\\
34.76	0\\
34.77	0\\
34.78	0\\
34.79	0\\
34.8	0\\
34.81	0\\
34.82	0\\
34.83	0\\
34.84	0\\
34.85	0\\
34.86	0\\
34.87	0\\
34.88	0\\
34.89	0\\
34.9	0\\
34.91	0\\
34.92	0\\
34.93	0\\
34.94	0\\
34.95	0\\
34.96	0\\
34.97	0\\
34.98	0\\
34.99	0\\
35	0\\
35.01	0\\
35.02	0\\
35.03	0\\
35.04	0\\
35.05	0\\
35.06	0\\
35.07	0\\
35.08	0\\
35.09	0\\
35.1	0\\
35.11	0\\
35.12	0\\
35.13	0\\
35.14	0\\
35.15	0\\
35.16	0\\
35.17	0\\
35.18	0\\
35.19	0\\
35.2	0\\
35.21	0\\
35.22	0\\
35.23	0\\
35.24	0\\
35.25	0\\
35.26	0\\
35.27	0\\
35.28	0\\
35.29	0\\
35.3	0\\
35.31	0\\
35.32	0\\
35.33	0\\
35.34	0\\
35.35	0\\
35.36	0\\
35.37	0\\
35.38	0\\
35.39	0\\
35.4	0\\
35.41	0\\
35.42	0\\
35.43	0\\
35.44	0\\
35.45	0\\
35.46	0\\
35.47	0\\
35.48	0\\
35.49	0\\
35.5	0\\
35.51	0\\
35.52	0\\
35.53	0\\
35.54	0\\
35.55	0\\
35.56	0\\
35.57	0\\
35.58	0\\
35.59	0\\
35.6	0\\
35.61	0\\
35.62	0\\
35.63	0\\
35.64	0\\
35.65	0\\
35.66	0\\
35.67	0\\
35.68	0\\
35.69	0\\
35.7	0\\
35.71	0\\
35.72	0\\
35.73	0\\
35.74	0\\
35.75	0\\
35.76	0\\
35.77	0\\
35.78	0\\
35.79	0\\
35.8	0\\
35.81	0\\
35.82	0\\
35.83	0\\
35.84	0\\
35.85	0\\
35.86	0\\
35.87	0\\
35.88	0\\
35.89	0\\
35.9	0\\
35.91	0\\
35.92	0\\
35.93	0\\
35.94	0\\
35.95	0\\
35.96	0\\
35.97	0\\
35.98	0\\
35.99	0\\
36	0\\
36.01	0\\
36.02	0\\
36.03	0\\
36.04	0\\
36.05	0\\
36.06	0\\
36.07	0\\
36.08	0\\
36.09	0\\
36.1	0\\
36.11	0\\
36.12	0\\
36.13	0\\
36.14	0\\
36.15	0\\
36.16	0\\
36.17	0\\
36.18	0\\
36.19	0\\
36.2	0\\
36.21	0\\
36.22	0\\
36.23	0\\
36.24	0\\
36.25	0\\
36.26	0\\
36.27	0\\
36.28	0\\
36.29	0\\
36.3	0\\
36.31	0\\
36.32	0\\
36.33	0\\
36.34	0\\
36.35	0\\
36.36	0\\
36.37	0\\
36.38	0\\
36.39	0\\
36.4	0\\
36.41	0\\
36.42	0\\
36.43	0\\
36.44	0\\
36.45	0\\
36.46	0\\
36.47	0\\
36.48	0\\
36.49	0\\
36.5	0\\
36.51	0\\
36.52	0\\
36.53	0\\
36.54	0\\
36.55	0\\
36.56	0\\
36.57	0\\
36.58	0\\
36.59	0\\
36.6	0\\
36.61	0\\
36.62	0\\
36.63	0\\
36.64	0\\
36.65	0\\
36.66	0\\
36.67	0\\
36.68	0\\
36.69	0\\
36.7	0\\
36.71	0\\
36.72	0\\
36.73	0\\
36.74	0\\
36.75	0\\
36.76	0\\
36.77	0\\
36.78	0\\
36.79	0\\
36.8	0\\
36.81	0\\
36.82	0\\
36.83	0\\
36.84	0\\
36.85	0\\
36.86	0\\
36.87	0\\
36.88	0\\
36.89	0\\
36.9	0\\
36.91	0\\
36.92	0\\
36.93	0\\
36.94	0\\
36.95	0\\
36.96	0\\
36.97	0\\
36.98	0\\
36.99	0\\
37	0\\
37.01	0\\
37.02	0\\
37.03	0\\
37.04	0\\
37.05	0\\
37.06	0\\
37.07	0\\
37.08	0\\
37.09	0\\
37.1	0\\
37.11	0\\
37.12	0\\
37.13	0\\
37.14	0\\
37.15	0\\
37.16	0\\
37.17	0\\
37.18	0\\
37.19	0\\
37.2	0\\
37.21	0\\
37.22	0\\
37.23	0\\
37.24	0\\
37.25	0\\
37.26	0\\
37.27	0\\
37.28	0\\
37.29	0\\
37.3	0\\
37.31	0\\
37.32	0\\
37.33	0\\
37.34	0\\
37.35	0\\
37.36	0\\
37.37	0\\
37.38	0\\
37.39	0\\
37.4	0\\
37.41	0\\
37.42	0\\
37.43	0\\
37.44	0\\
37.45	0\\
37.46	0\\
37.47	0\\
37.48	0\\
37.49	0\\
37.5	0\\
37.51	0\\
37.52	0\\
37.53	0\\
37.54	0\\
37.55	0\\
37.56	0\\
37.57	0\\
37.58	0\\
37.59	0\\
37.6	0\\
37.61	0\\
37.62	0\\
37.63	0\\
37.64	0\\
37.65	0\\
37.66	0\\
37.67	0\\
37.68	0\\
37.69	0\\
37.7	0\\
37.71	0\\
37.72	0\\
37.73	0\\
37.74	0\\
37.75	0\\
37.76	0\\
37.77	0\\
37.78	0\\
37.79	0\\
37.8	0\\
37.81	0\\
37.82	0\\
37.83	0\\
37.84	0\\
37.85	0\\
37.86	0\\
37.87	0\\
37.88	0\\
37.89	0\\
37.9	0\\
37.91	0\\
37.92	0\\
37.93	0\\
37.94	0\\
37.95	0\\
37.96	0\\
37.97	0\\
37.98	0\\
37.99	0\\
38	0\\
38.01	0\\
38.02	0\\
38.03	0\\
38.04	0\\
38.05	0\\
38.06	0\\
38.07	0\\
38.08	0\\
38.09	0\\
38.1	0\\
38.11	0\\
38.12	0\\
38.13	0\\
38.14	0\\
38.15	0\\
38.16	0\\
38.17	0\\
38.18	0\\
38.19	0\\
38.2	0\\
38.21	0\\
38.22	0\\
38.23	0\\
38.24	0\\
38.25	0\\
38.26	0\\
38.27	0\\
38.28	0\\
38.29	0\\
38.3	0\\
38.31	0\\
38.32	0\\
38.33	0\\
38.34	0\\
38.35	0\\
38.36	0\\
38.37	0\\
38.38	0\\
38.39	0\\
38.4	0\\
38.41	0\\
38.42	0\\
38.43	0\\
38.44	0\\
38.45	0\\
38.46	0\\
38.47	0\\
38.48	0\\
38.49	0\\
38.5	0\\
38.51	0\\
38.52	0\\
38.53	0\\
38.54	0\\
38.55	0\\
38.56	0\\
38.57	0\\
38.58	0\\
38.59	0\\
38.6	0\\
38.61	0\\
38.62	0\\
38.63	0\\
38.64	0\\
38.65	0\\
38.66	0\\
38.67	0\\
38.68	0\\
38.69	0\\
38.7	0\\
38.71	0\\
38.72	0\\
38.73	0\\
38.74	0\\
38.75	0\\
38.76	0\\
38.77	0\\
38.78	0\\
38.79	0\\
38.8	0\\
38.81	0\\
38.82	0\\
38.83	0\\
38.84	0\\
38.85	0\\
38.86	0\\
38.87	0\\
38.88	0\\
38.89	0\\
38.9	0\\
38.91	0\\
38.92	0\\
38.93	0\\
38.94	0\\
38.95	0\\
38.96	0\\
38.97	0\\
38.98	0\\
38.99	0\\
39	0\\
39.01	0\\
39.02	0\\
39.03	0\\
39.04	0\\
39.05	0\\
39.06	0\\
39.07	0\\
39.08	0\\
39.09	0\\
39.1	0\\
39.11	0\\
39.12	0\\
39.13	0\\
39.14	0\\
39.15	0\\
39.16	0\\
39.17	0\\
39.18	0\\
39.19	0\\
39.2	0\\
39.21	0\\
39.22	0\\
39.23	0\\
39.24	0\\
39.25	0\\
39.26	0\\
39.27	0\\
39.28	0\\
39.29	0\\
39.3	0\\
39.31	0\\
39.32	0\\
39.33	0\\
39.34	0\\
39.35	0\\
39.36	0\\
39.37	0\\
39.38	0\\
39.39	0\\
39.4	0\\
39.41	0\\
39.42	0\\
39.43	0\\
39.44	0\\
39.45	0\\
39.46	0\\
39.47	0\\
39.48	0\\
39.49	0\\
39.5	0\\
39.51	0\\
39.52	0\\
39.53	0\\
39.54	0\\
39.55	0\\
39.56	0\\
39.57	0\\
39.58	0\\
39.59	0\\
39.6	0\\
39.61	0\\
39.62	0\\
39.63	0\\
39.64	0\\
39.65	0\\
39.66	0\\
39.67	0\\
39.68	0\\
39.69	0\\
39.7	0\\
39.71	0\\
39.72	0\\
39.73	0\\
39.74	0\\
39.75	0\\
39.76	0\\
39.77	0\\
39.78	0\\
39.79	0\\
39.8	0\\
39.81	0\\
39.82	0\\
39.83	0\\
39.84	0\\
39.85	0\\
39.86	0\\
39.87	0\\
39.88	0\\
39.89	0\\
39.9	0\\
39.91	0\\
39.92	0\\
39.93	0\\
39.94	0\\
39.95	0\\
39.96	0\\
39.97	0\\
39.98	0\\
39.99	0\\
40	0\\
40.01	0\\
};
\addplot [color=blue,dashed,forget plot]
  table[row sep=crcr]{%
40.01	0\\
40.02	0\\
40.03	0\\
40.04	0\\
40.05	0\\
40.06	0\\
40.07	0\\
40.08	0\\
40.09	0\\
40.1	0\\
40.11	0\\
40.12	0\\
40.13	0\\
40.14	0\\
40.15	0\\
40.16	0\\
40.17	0\\
40.18	0\\
40.19	0\\
40.2	0\\
40.21	0\\
40.22	0\\
40.23	0\\
40.24	0\\
40.25	0\\
40.26	0\\
40.27	0\\
40.28	0\\
40.29	0\\
40.3	0\\
40.31	0\\
40.32	0\\
40.33	0\\
40.34	0\\
40.35	0\\
40.36	0\\
40.37	0\\
40.38	0\\
40.39	0\\
40.4	0\\
40.41	0\\
40.42	0\\
40.43	0\\
40.44	0\\
40.45	0\\
40.46	0\\
40.47	0\\
40.48	0\\
40.49	0\\
40.5	0\\
40.51	0\\
40.52	0\\
40.53	0\\
40.54	0\\
40.55	0\\
40.56	0\\
40.57	0\\
40.58	0\\
40.59	0\\
40.6	0\\
40.61	0\\
40.62	0\\
40.63	0\\
40.64	0\\
40.65	0\\
40.66	0\\
40.67	0\\
40.68	0\\
40.69	0\\
40.7	0\\
40.71	0\\
40.72	0\\
40.73	0\\
40.74	0\\
40.75	0\\
40.76	0\\
40.77	0\\
40.78	0\\
40.79	0\\
40.8	0\\
40.81	0\\
40.82	0\\
40.83	0\\
40.84	0\\
40.85	0\\
40.86	0\\
40.87	0\\
40.88	0\\
40.89	0\\
40.9	0\\
40.91	0\\
40.92	0\\
40.93	0\\
40.94	0\\
40.95	0\\
40.96	0\\
40.97	0\\
40.98	0\\
40.99	0\\
41	0\\
41.01	0\\
41.02	0\\
41.03	0\\
41.04	0\\
41.05	0\\
41.06	0\\
41.07	0\\
41.08	0\\
41.09	0\\
41.1	0\\
41.11	0\\
41.12	0\\
41.13	0\\
41.14	0\\
41.15	0\\
41.16	0\\
41.17	0\\
41.18	0\\
41.19	0\\
41.2	0\\
41.21	0\\
41.22	0\\
41.23	0\\
41.24	0\\
41.25	0\\
41.26	0\\
41.27	0\\
41.28	0\\
41.29	0\\
41.3	0\\
41.31	0\\
41.32	0\\
41.33	0\\
41.34	0\\
41.35	0\\
41.36	0\\
41.37	0\\
41.38	0\\
41.39	0\\
41.4	0\\
41.41	0\\
41.42	0\\
41.43	0\\
41.44	0\\
41.45	0\\
41.46	0\\
41.47	0\\
41.48	0\\
41.49	0\\
41.5	0\\
41.51	0\\
41.52	0\\
41.53	0\\
41.54	0\\
41.55	0\\
41.56	0\\
41.57	0\\
41.58	0\\
41.59	0\\
41.6	0\\
41.61	0\\
41.62	0\\
41.63	0\\
41.64	0\\
41.65	0\\
41.66	0\\
41.67	0\\
41.68	0\\
41.69	0\\
41.7	0\\
41.71	0\\
41.72	0\\
41.73	0\\
41.74	0\\
41.75	0\\
41.76	0\\
41.77	0\\
41.78	0\\
41.79	0\\
41.8	0\\
41.81	0\\
41.82	0\\
41.83	0\\
41.84	0\\
41.85	0\\
41.86	0\\
41.87	0\\
41.88	0\\
41.89	0\\
41.9	0\\
41.91	0\\
41.92	0\\
41.93	0\\
41.94	0\\
41.95	0\\
41.96	0\\
41.97	0\\
41.98	0\\
41.99	0\\
42	0\\
42.01	0\\
42.02	0\\
42.03	0\\
42.04	0\\
42.05	0\\
42.06	0\\
42.07	0\\
42.08	0\\
42.09	0\\
42.1	0\\
42.11	0\\
42.12	0\\
42.13	0\\
42.14	0\\
42.15	0\\
42.16	0\\
42.17	0\\
42.18	0\\
42.19	0\\
42.2	0\\
42.21	0\\
42.22	0\\
42.23	0\\
42.24	0\\
42.25	0\\
42.26	0\\
42.27	0\\
42.28	0\\
42.29	0\\
42.3	0\\
42.31	0\\
42.32	0\\
42.33	0\\
42.34	0\\
42.35	0\\
42.36	0\\
42.37	0\\
42.38	0\\
42.39	0\\
42.4	0\\
42.41	0\\
42.42	0\\
42.43	0\\
42.44	0\\
42.45	0\\
42.46	0\\
42.47	0\\
42.48	0\\
42.49	0\\
42.5	0\\
42.51	0\\
42.52	0\\
42.53	0\\
42.54	0\\
42.55	0\\
42.56	0\\
42.57	0\\
42.58	0\\
42.59	0\\
42.6	0\\
42.61	0\\
42.62	0\\
42.63	0\\
42.64	0\\
42.65	0\\
42.66	0\\
42.67	0\\
42.68	0\\
42.69	0\\
42.7	0\\
42.71	0\\
42.72	0\\
42.73	0\\
42.74	0\\
42.75	0\\
42.76	0\\
42.77	0\\
42.78	0\\
42.79	0\\
42.8	0\\
42.81	0\\
42.82	0\\
42.83	0\\
42.84	0\\
42.85	0\\
42.86	0\\
42.87	0\\
42.88	0\\
42.89	0\\
42.9	0\\
42.91	0\\
42.92	0\\
42.93	0\\
42.94	0\\
42.95	0\\
42.96	0\\
42.97	0\\
42.98	0\\
42.99	0\\
43	0\\
43.01	0\\
43.02	0\\
43.03	0\\
43.04	0\\
43.05	0\\
43.06	0\\
43.07	0\\
43.08	0\\
43.09	0\\
43.1	0\\
43.11	0\\
43.12	0\\
43.13	0\\
43.14	0\\
43.15	0\\
43.16	0\\
43.17	0\\
43.18	0\\
43.19	0\\
43.2	0\\
43.21	0\\
43.22	0\\
43.23	0\\
43.24	0\\
43.25	0\\
43.26	0\\
43.27	0\\
43.28	0\\
43.29	0\\
43.3	0\\
43.31	0\\
43.32	0\\
43.33	0\\
43.34	0\\
43.35	0\\
43.36	0\\
43.37	0\\
43.38	0\\
43.39	0\\
43.4	0\\
43.41	0\\
43.42	0\\
43.43	0\\
43.44	0\\
43.45	0\\
43.46	0\\
43.47	0\\
43.48	0\\
43.49	0\\
43.5	0\\
43.51	0\\
43.52	0\\
43.53	0\\
43.54	0\\
43.55	0\\
43.56	0\\
43.57	0\\
43.58	0\\
43.59	0\\
43.6	0\\
43.61	0\\
43.62	0\\
43.63	0\\
43.64	0\\
43.65	0\\
43.66	0\\
43.67	0\\
43.68	0\\
43.69	0\\
43.7	0\\
43.71	0\\
43.72	0\\
43.73	0\\
43.74	0\\
43.75	0\\
43.76	0\\
43.77	0\\
43.78	0\\
43.79	0\\
43.8	0\\
43.81	0\\
43.82	0\\
43.83	0\\
43.84	0\\
43.85	0\\
43.86	0\\
43.87	0\\
43.88	0\\
43.89	0\\
43.9	0\\
43.91	0\\
43.92	0\\
43.93	0\\
43.94	0\\
43.95	0\\
43.96	0\\
43.97	0\\
43.98	0\\
43.99	0\\
44	0\\
44.01	0\\
44.02	0\\
44.03	0\\
44.04	0\\
44.05	0\\
44.06	0\\
44.07	0\\
44.08	0\\
44.09	0\\
44.1	0\\
44.11	0\\
44.12	0\\
44.13	0\\
44.14	0\\
44.15	0\\
44.16	0\\
44.17	0\\
44.18	0\\
44.19	0\\
44.2	0\\
44.21	0\\
44.22	0\\
44.23	0\\
44.24	0\\
44.25	0\\
44.26	0\\
44.27	0\\
44.28	0\\
44.29	0\\
44.3	0\\
44.31	0\\
44.32	0\\
44.33	0\\
44.34	0\\
44.35	0\\
44.36	0\\
44.37	0\\
44.38	0\\
44.39	0\\
44.4	0\\
44.41	0\\
44.42	0\\
44.43	0\\
44.44	0\\
44.45	0\\
44.46	0\\
44.47	0\\
44.48	0\\
44.49	0\\
44.5	0\\
44.51	0\\
44.52	0\\
44.53	0\\
44.54	0\\
44.55	0\\
44.56	0\\
44.57	0\\
44.58	0\\
44.59	0\\
44.6	0\\
44.61	0\\
44.62	0\\
44.63	0\\
44.64	0\\
44.65	0\\
44.66	0\\
44.67	0\\
44.68	0\\
44.69	0\\
44.7	0\\
44.71	0\\
44.72	0\\
44.73	0\\
44.74	0\\
44.75	0\\
44.76	0\\
44.77	0\\
44.78	0\\
44.79	0\\
44.8	0\\
44.81	0\\
44.82	0\\
44.83	0\\
44.84	0\\
44.85	0\\
44.86	0\\
44.87	0\\
44.88	0\\
44.89	0\\
44.9	0\\
44.91	0\\
44.92	0\\
44.93	0\\
44.94	0\\
44.95	0\\
44.96	0\\
44.97	0\\
44.98	0\\
44.99	0\\
45	0\\
45.01	0\\
45.02	0\\
45.03	0\\
45.04	0\\
45.05	0\\
45.06	0\\
45.07	0\\
45.08	0\\
45.09	0\\
45.1	0\\
45.11	0\\
45.12	0\\
45.13	0\\
45.14	0\\
45.15	0\\
45.16	0\\
45.17	0\\
45.18	0\\
45.19	0\\
45.2	0\\
45.21	0\\
45.22	0\\
45.23	0\\
45.24	0\\
45.25	0\\
45.26	0\\
45.27	0\\
45.28	0\\
45.29	0\\
45.3	0\\
45.31	0\\
45.32	0\\
45.33	0\\
45.34	0\\
45.35	0\\
45.36	0\\
45.37	0\\
45.38	0\\
45.39	0\\
45.4	0\\
45.41	0\\
45.42	0\\
45.43	0\\
45.44	0\\
45.45	0\\
45.46	0\\
45.47	0\\
45.48	0\\
45.49	0\\
45.5	0\\
45.51	0\\
45.52	0\\
45.53	0\\
45.54	0\\
45.55	0\\
45.56	0\\
45.57	0\\
45.58	0\\
45.59	0\\
45.6	0\\
45.61	0\\
45.62	0\\
45.63	0\\
45.64	0\\
45.65	0\\
45.66	0\\
45.67	0\\
45.68	0\\
45.69	0\\
45.7	0\\
45.71	0\\
45.72	0\\
45.73	0\\
45.74	0\\
45.75	0\\
45.76	0\\
45.77	0\\
45.78	0\\
45.79	0\\
45.8	0\\
45.81	0\\
45.82	0\\
45.83	0\\
45.84	0\\
45.85	0\\
45.86	0\\
45.87	0\\
45.88	0\\
45.89	0\\
45.9	0\\
45.91	0\\
45.92	0\\
45.93	0\\
45.94	0\\
45.95	0\\
45.96	0\\
45.97	0\\
45.98	0\\
45.99	0\\
46	0\\
46.01	0\\
46.02	0\\
46.03	0\\
46.04	0\\
46.05	0\\
46.06	0\\
46.07	0\\
46.08	0\\
46.09	0\\
46.1	0\\
46.11	0\\
46.12	0\\
46.13	0\\
46.14	0\\
46.15	0\\
46.16	0\\
46.17	0\\
46.18	0\\
46.19	0\\
46.2	0\\
46.21	0\\
46.22	0\\
46.23	0\\
46.24	0\\
46.25	0\\
46.26	0\\
46.27	0\\
46.28	0\\
46.29	0\\
46.3	0\\
46.31	0\\
46.32	0\\
46.33	0\\
46.34	0\\
46.35	0\\
46.36	0\\
46.37	0\\
46.38	0\\
46.39	0\\
46.4	0\\
46.41	0\\
46.42	0\\
46.43	0\\
46.44	0\\
46.45	0\\
46.46	0\\
46.47	0\\
46.48	0\\
46.49	0\\
46.5	0\\
46.51	0\\
46.52	0\\
46.53	0\\
46.54	0\\
46.55	0\\
46.56	0\\
46.57	0\\
46.58	0\\
46.59	0\\
46.6	0\\
46.61	0\\
46.62	0\\
46.63	0\\
46.64	0\\
46.65	0\\
46.66	0\\
46.67	0\\
46.68	0\\
46.69	0\\
46.7	0\\
46.71	0\\
46.72	0\\
46.73	0\\
46.74	0\\
46.75	0\\
46.76	0\\
46.77	0\\
46.78	0\\
46.79	0\\
46.8	0\\
46.81	0\\
46.82	0\\
46.83	0\\
46.84	0\\
46.85	0\\
46.86	0\\
46.87	0\\
46.88	0\\
46.89	0\\
46.9	0\\
46.91	0\\
46.92	0\\
46.93	0\\
46.94	0\\
46.95	0\\
46.96	0\\
46.97	0\\
46.98	0\\
46.99	0\\
47	0\\
47.01	0\\
47.02	0\\
47.03	0\\
47.04	0\\
47.05	0\\
47.06	0\\
47.07	0\\
47.08	0\\
47.09	0\\
47.1	0\\
47.11	0\\
47.12	0\\
47.13	0\\
47.14	0\\
47.15	0\\
47.16	0\\
47.17	0\\
47.18	0\\
47.19	0\\
47.2	0\\
47.21	0\\
47.22	0\\
47.23	0\\
47.24	0\\
47.25	0\\
47.26	0\\
47.27	0\\
47.28	0\\
47.29	0\\
47.3	0\\
47.31	0\\
47.32	0\\
47.33	0\\
47.34	0\\
47.35	0\\
47.36	0\\
47.37	0\\
47.38	0\\
47.39	0\\
47.4	0\\
47.41	0\\
47.42	0\\
47.43	0\\
47.44	0\\
47.45	0\\
47.46	0\\
47.47	0\\
47.48	0\\
47.49	0\\
47.5	0\\
47.51	0\\
47.52	0\\
47.53	0\\
47.54	0\\
47.55	0\\
47.56	0\\
47.57	0\\
47.58	0\\
47.59	0\\
47.6	0\\
47.61	0\\
47.62	0\\
47.63	0\\
47.64	0\\
47.65	0\\
47.66	0\\
47.67	0\\
47.68	0\\
47.69	0\\
47.7	0\\
47.71	0\\
47.72	0\\
47.73	0\\
47.74	0\\
47.75	0\\
47.76	0\\
47.77	0\\
47.78	0\\
47.79	0\\
47.8	0\\
47.81	0\\
47.82	0\\
47.83	0\\
47.84	0\\
47.85	0\\
47.86	0\\
47.87	0\\
47.88	0\\
47.89	0\\
47.9	0\\
47.91	0\\
47.92	0\\
47.93	0\\
47.94	0\\
47.95	0\\
47.96	0\\
47.97	0\\
47.98	0\\
47.99	0\\
48	0\\
48.01	0\\
48.02	0\\
48.03	0\\
48.04	0\\
48.05	0\\
48.06	0\\
48.07	0\\
48.08	0\\
48.09	0\\
48.1	0\\
48.11	0\\
48.12	0\\
48.13	0\\
48.14	0\\
48.15	0\\
48.16	0\\
48.17	0\\
48.18	0\\
48.19	0\\
48.2	0\\
48.21	0\\
48.22	0\\
48.23	0\\
48.24	0\\
48.25	0\\
48.26	0\\
48.27	0\\
48.28	0\\
48.29	0\\
48.3	0\\
48.31	0\\
48.32	0\\
48.33	0\\
48.34	0\\
48.35	0\\
48.36	0\\
48.37	0\\
48.38	0\\
48.39	0\\
48.4	0\\
48.41	0\\
48.42	0\\
48.43	0\\
48.44	0\\
48.45	0\\
48.46	0\\
48.47	0\\
48.48	0\\
48.49	0\\
48.5	0\\
48.51	0\\
48.52	0\\
48.53	0\\
48.54	0\\
48.55	0\\
48.56	0\\
48.57	0\\
48.58	0\\
48.59	0\\
48.6	0\\
48.61	0\\
48.62	0\\
48.63	0\\
48.64	0\\
48.65	0\\
48.66	0\\
48.67	0\\
48.68	0\\
48.69	0\\
48.7	0\\
48.71	0\\
48.72	0\\
48.73	0\\
48.74	0\\
48.75	0\\
48.76	0\\
48.77	0\\
48.78	0\\
48.79	0\\
48.8	0\\
48.81	0\\
48.82	0\\
48.83	0\\
48.84	0\\
48.85	0\\
48.86	0\\
48.87	0\\
48.88	0\\
48.89	0\\
48.9	0\\
48.91	0\\
48.92	0\\
48.93	0\\
48.94	0\\
48.95	0\\
48.96	0\\
48.97	0\\
48.98	0\\
48.99	0\\
49	0\\
49.01	0\\
49.02	0\\
49.03	0\\
49.04	0\\
49.05	0\\
49.06	0\\
49.07	0\\
49.08	0\\
49.09	0\\
49.1	0\\
49.11	0\\
49.12	0\\
49.13	0\\
49.14	0\\
49.15	0\\
49.16	0\\
49.17	0\\
49.18	0\\
49.19	0\\
49.2	0\\
49.21	0\\
49.22	0\\
49.23	0\\
49.24	0\\
49.25	0\\
49.26	0\\
49.27	0\\
49.28	0\\
49.29	0\\
49.3	0\\
49.31	0\\
49.32	0\\
49.33	0\\
49.34	0\\
49.35	0\\
49.36	0\\
49.37	0\\
49.38	0\\
49.39	0\\
49.4	0\\
49.41	0\\
49.42	0\\
49.43	0\\
49.44	0\\
49.45	0\\
49.46	0\\
49.47	0\\
49.48	0\\
49.49	0\\
49.5	0\\
49.51	0\\
49.52	0\\
49.53	0\\
49.54	0\\
49.55	0\\
49.56	0\\
49.57	0\\
49.58	0\\
49.59	0\\
49.6	0\\
49.61	0\\
49.62	0\\
49.63	0\\
49.64	0\\
49.65	0\\
49.66	0\\
49.67	0\\
49.68	0\\
49.69	0\\
49.7	0\\
49.71	0\\
49.72	0\\
49.73	0\\
49.74	0\\
49.75	0\\
49.76	0\\
49.77	0\\
49.78	0\\
49.79	0\\
49.8	0\\
49.81	0\\
49.82	0\\
49.83	0\\
49.84	0\\
49.85	0\\
49.86	0\\
49.87	0\\
49.88	0\\
49.89	0\\
49.9	0\\
49.91	0\\
49.92	0\\
49.93	0\\
49.94	0\\
49.95	0\\
49.96	0\\
49.97	0\\
49.98	0\\
49.99	0\\
50	0\\
50.01	0\\
50.02	0\\
50.03	0\\
50.04	0\\
50.05	0\\
50.06	0\\
50.07	0\\
50.08	0\\
50.09	0\\
50.1	0\\
50.11	0\\
50.12	0\\
50.13	0\\
50.14	0\\
50.15	0\\
50.16	0\\
50.17	0\\
50.18	0\\
50.19	0\\
50.2	0\\
50.21	0\\
50.22	0\\
50.23	0\\
50.24	0\\
50.25	0\\
50.26	0\\
50.27	0\\
50.28	0\\
50.29	0\\
50.3	0\\
50.31	0\\
50.32	0\\
50.33	0\\
50.34	0\\
50.35	0\\
50.36	0\\
50.37	0\\
50.38	0\\
50.39	0\\
50.4	0\\
50.41	0\\
50.42	0\\
50.43	0\\
50.44	0\\
50.45	0\\
50.46	0\\
50.47	0\\
50.48	0\\
50.49	0\\
50.5	0\\
50.51	0\\
50.52	0\\
50.53	0\\
50.54	0\\
50.55	0\\
50.56	0\\
50.57	0\\
50.58	0\\
50.59	0\\
50.6	0\\
50.61	0\\
50.62	0\\
50.63	0\\
50.64	0\\
50.65	0\\
50.66	0\\
50.67	0\\
50.68	0\\
50.69	0\\
50.7	0\\
50.71	0\\
50.72	0\\
50.73	0\\
50.74	0\\
50.75	0\\
50.76	0\\
50.77	0\\
50.78	0\\
50.79	0\\
50.8	0\\
50.81	0\\
50.82	0\\
50.83	0\\
50.84	0\\
50.85	0\\
50.86	0\\
50.87	0\\
50.88	0\\
50.89	0\\
50.9	0\\
50.91	0\\
50.92	0\\
50.93	0\\
50.94	0\\
50.95	0\\
50.96	0\\
50.97	0\\
50.98	0\\
50.99	0\\
51	0\\
51.01	0\\
51.02	0\\
51.03	0\\
51.04	0\\
51.05	0\\
51.06	0\\
51.07	0\\
51.08	0\\
51.09	0\\
51.1	0\\
51.11	0\\
51.12	0\\
51.13	0\\
51.14	0\\
51.15	0\\
51.16	0\\
51.17	0\\
51.18	0\\
51.19	0\\
51.2	0\\
51.21	0\\
51.22	0\\
51.23	0\\
51.24	0\\
51.25	0\\
51.26	0\\
51.27	0\\
51.28	0\\
51.29	0\\
51.3	0\\
51.31	0\\
51.32	0\\
51.33	0\\
51.34	0\\
51.35	0\\
51.36	0\\
51.37	0\\
51.38	0\\
51.39	0\\
51.4	0\\
51.41	0\\
51.42	0\\
51.43	0\\
51.44	0\\
51.45	0\\
51.46	0\\
51.47	0\\
51.48	0\\
51.49	0\\
51.5	0\\
51.51	0\\
51.52	0\\
51.53	0\\
51.54	0\\
51.55	0\\
51.56	0\\
51.57	0\\
51.58	0\\
51.59	0\\
51.6	0\\
51.61	0\\
51.62	0\\
51.63	0\\
51.64	0\\
51.65	0\\
51.66	0\\
51.67	0\\
51.68	0\\
51.69	0\\
51.7	0\\
51.71	0\\
51.72	0\\
51.73	0\\
51.74	0\\
51.75	0\\
51.76	0\\
51.77	0\\
51.78	0\\
51.79	0\\
51.8	0\\
51.81	0\\
51.82	0\\
51.83	0\\
51.84	0\\
51.85	0\\
51.86	0\\
51.87	0\\
51.88	0\\
51.89	0\\
51.9	0\\
51.91	0\\
51.92	0\\
51.93	0\\
51.94	0\\
51.95	0\\
51.96	0\\
51.97	0\\
51.98	0\\
51.99	0\\
52	0\\
52.01	0\\
52.02	0\\
52.03	0\\
52.04	0\\
52.05	0\\
52.06	0\\
52.07	0\\
52.08	0\\
52.09	0\\
52.1	0\\
52.11	0\\
52.12	0\\
52.13	0\\
52.14	0\\
52.15	0\\
52.16	0\\
52.17	0\\
52.18	0\\
52.19	0\\
52.2	0\\
52.21	0\\
52.22	0\\
52.23	0\\
52.24	0\\
52.25	0\\
52.26	0\\
52.27	0\\
52.28	0\\
52.29	0\\
52.3	0\\
52.31	0\\
52.32	0\\
52.33	0\\
52.34	0\\
52.35	0\\
52.36	0\\
52.37	0\\
52.38	0\\
52.39	0\\
52.4	0\\
52.41	0\\
52.42	0\\
52.43	0\\
52.44	0\\
52.45	0\\
52.46	0\\
52.47	0\\
52.48	0\\
52.49	0\\
52.5	0\\
52.51	0\\
52.52	0\\
52.53	0\\
52.54	0\\
52.55	0\\
52.56	0\\
52.57	0\\
52.58	0\\
52.59	0\\
52.6	0\\
52.61	0\\
52.62	0\\
52.63	0\\
52.64	0\\
52.65	0\\
52.66	0\\
52.67	0\\
52.68	0\\
52.69	0\\
52.7	0\\
52.71	0\\
52.72	0\\
52.73	0\\
52.74	0\\
52.75	0\\
52.76	0\\
52.77	0\\
52.78	0\\
52.79	0\\
52.8	0\\
52.81	0\\
52.82	0\\
52.83	0\\
52.84	0\\
52.85	0\\
52.86	0\\
52.87	0\\
52.88	0\\
52.89	0\\
52.9	0\\
52.91	0\\
52.92	0\\
52.93	0\\
52.94	0\\
52.95	0\\
52.96	0\\
52.97	0\\
52.98	0\\
52.99	0\\
53	0\\
53.01	0\\
53.02	0\\
53.03	0\\
53.04	0\\
53.05	0\\
53.06	0\\
53.07	0\\
53.08	0\\
53.09	0\\
53.1	0\\
53.11	0\\
53.12	0\\
53.13	0\\
53.14	0\\
53.15	0\\
53.16	0\\
53.17	0\\
53.18	0\\
53.19	0\\
53.2	0\\
53.21	0\\
53.22	0\\
53.23	0\\
53.24	0\\
53.25	0\\
53.26	0\\
53.27	0\\
53.28	0\\
53.29	0\\
53.3	0\\
53.31	0\\
53.32	0\\
53.33	0\\
53.34	0\\
53.35	0\\
53.36	0\\
53.37	0\\
53.38	0\\
53.39	0\\
53.4	0\\
53.41	0\\
53.42	0\\
53.43	0\\
53.44	0\\
53.45	0\\
53.46	0\\
53.47	0\\
53.48	0\\
53.49	0\\
53.5	0\\
53.51	0\\
53.52	0\\
53.53	0\\
53.54	0\\
53.55	0\\
53.56	0\\
53.57	0\\
53.58	0\\
53.59	0\\
53.6	0\\
53.61	0\\
53.62	0\\
53.63	0\\
53.64	0\\
53.65	0\\
53.66	0\\
53.67	0\\
53.68	0\\
53.69	0\\
53.7	0\\
53.71	0\\
53.72	0\\
53.73	0\\
53.74	0\\
53.75	0\\
53.76	0\\
53.77	0\\
53.78	0\\
53.79	0\\
53.8	0\\
53.81	0\\
53.82	0\\
53.83	0\\
53.84	0\\
53.85	0\\
53.86	0\\
53.87	0\\
53.88	0\\
53.89	0\\
53.9	0\\
53.91	0\\
53.92	0\\
53.93	0\\
53.94	0\\
53.95	0\\
53.96	0\\
53.97	0\\
53.98	0\\
53.99	0\\
54	0\\
54.01	0\\
54.02	0\\
54.03	0\\
54.04	0\\
54.05	0\\
54.06	0\\
54.07	0\\
54.08	0\\
54.09	0\\
54.1	0\\
54.11	0\\
54.12	0\\
54.13	0\\
54.14	0\\
54.15	0\\
54.16	0\\
54.17	0\\
54.18	0\\
54.19	0\\
54.2	0\\
54.21	0\\
54.22	0\\
54.23	0\\
54.24	0\\
54.25	0\\
54.26	0\\
54.27	0\\
54.28	0\\
54.29	0\\
54.3	0\\
54.31	0\\
54.32	0\\
54.33	0\\
54.34	0\\
54.35	0\\
54.36	0\\
54.37	0\\
54.38	0\\
54.39	0\\
54.4	0\\
54.41	0\\
54.42	0\\
54.43	0\\
54.44	0\\
54.45	0\\
54.46	0\\
54.47	0\\
54.48	0\\
54.49	0\\
54.5	0\\
54.51	0\\
54.52	0\\
54.53	0\\
54.54	0\\
54.55	0\\
54.56	0\\
54.57	0\\
54.58	0\\
54.59	0\\
54.6	0\\
54.61	0\\
54.62	0\\
54.63	0\\
54.64	0\\
54.65	0\\
54.66	0\\
54.67	0\\
54.68	0\\
54.69	0\\
54.7	0\\
54.71	0\\
54.72	0\\
54.73	0\\
54.74	0\\
54.75	0\\
54.76	0\\
54.77	0\\
54.78	0\\
54.79	0\\
54.8	0\\
54.81	0\\
54.82	0\\
54.83	0\\
54.84	0\\
54.85	0\\
54.86	0\\
54.87	0\\
54.88	0\\
54.89	0\\
54.9	0\\
54.91	0\\
54.92	0\\
54.93	0\\
54.94	0\\
54.95	0\\
54.96	0\\
54.97	0\\
54.98	0\\
54.99	0\\
55	0\\
55.01	0\\
55.02	0\\
55.03	0\\
55.04	0\\
55.05	0\\
55.06	0\\
55.07	0\\
55.08	0\\
55.09	0\\
55.1	0\\
55.11	0\\
55.12	0\\
55.13	0\\
55.14	0\\
55.15	0\\
55.16	0\\
55.17	0\\
55.18	0\\
55.19	0\\
55.2	0\\
55.21	0\\
55.22	0\\
55.23	0\\
55.24	0\\
55.25	0\\
55.26	0\\
55.27	0\\
55.28	0\\
55.29	0\\
55.3	0\\
55.31	0\\
55.32	0\\
55.33	0\\
55.34	0\\
55.35	0\\
55.36	0\\
55.37	0\\
55.38	0\\
55.39	0\\
55.4	0\\
55.41	0\\
55.42	0\\
55.43	0\\
55.44	0\\
55.45	0\\
55.46	0\\
55.47	0\\
55.48	0\\
55.49	0\\
55.5	0\\
55.51	0\\
55.52	0\\
55.53	0\\
55.54	0\\
55.55	0\\
55.56	0\\
55.57	0\\
55.58	0\\
55.59	0\\
55.6	0\\
55.61	0\\
55.62	0\\
55.63	0\\
55.64	0\\
55.65	0\\
55.66	0\\
55.67	0\\
55.68	0\\
55.69	0\\
55.7	0\\
55.71	0\\
55.72	0\\
55.73	0\\
55.74	0\\
55.75	0\\
55.76	0\\
55.77	0\\
55.78	0\\
55.79	0\\
55.8	0\\
55.81	0\\
55.82	0\\
55.83	0\\
55.84	0\\
55.85	0\\
55.86	0\\
55.87	0\\
55.88	0\\
55.89	0\\
55.9	0\\
55.91	0\\
55.92	0\\
55.93	0\\
55.94	0\\
55.95	0\\
55.96	0\\
55.97	0\\
55.98	0\\
55.99	0\\
56	0\\
56.01	0\\
56.02	0\\
56.03	0\\
56.04	0\\
56.05	0\\
56.06	0\\
56.07	0\\
56.08	0\\
56.09	0\\
56.1	0\\
56.11	0\\
56.12	0\\
56.13	0\\
56.14	0\\
56.15	0\\
56.16	0\\
56.17	0\\
56.18	0\\
56.19	0\\
56.2	0\\
56.21	0\\
56.22	0\\
56.23	0\\
56.24	0\\
56.25	0\\
56.26	0\\
56.27	0\\
56.28	0\\
56.29	0\\
56.3	0\\
56.31	0\\
56.32	0\\
56.33	0\\
56.34	0\\
56.35	0\\
56.36	0\\
56.37	0\\
56.38	0\\
56.39	0\\
56.4	0\\
56.41	0\\
56.42	0\\
56.43	0\\
56.44	0\\
56.45	0\\
56.46	0\\
56.47	0\\
56.48	0\\
56.49	0\\
56.5	0\\
56.51	0\\
56.52	0\\
56.53	0\\
56.54	0\\
56.55	0\\
56.56	0\\
56.57	0\\
56.58	0\\
56.59	0\\
56.6	0\\
56.61	0\\
56.62	0\\
56.63	0\\
56.64	0\\
56.65	0\\
56.66	0\\
56.67	0\\
56.68	0\\
56.69	0\\
56.7	0\\
56.71	0\\
56.72	0\\
56.73	0\\
56.74	0\\
56.75	0\\
56.76	0\\
56.77	0\\
56.78	0\\
56.79	0\\
56.8	0\\
56.81	0\\
56.82	0\\
56.83	0\\
56.84	0\\
56.85	0\\
56.86	0\\
56.87	0\\
56.88	0\\
56.89	0\\
56.9	0\\
56.91	0\\
56.92	0\\
56.93	0\\
56.94	0\\
56.95	0\\
56.96	0\\
56.97	0\\
56.98	0\\
56.99	0\\
57	0\\
57.01	0\\
57.02	0\\
57.03	0\\
57.04	0\\
57.05	0\\
57.06	0\\
57.07	0\\
57.08	0\\
57.09	0\\
57.1	0\\
57.11	0\\
57.12	0\\
57.13	0\\
57.14	0\\
57.15	0\\
57.16	0\\
57.17	0\\
57.18	0\\
57.19	0\\
57.2	0\\
57.21	0\\
57.22	0\\
57.23	0\\
57.24	0\\
57.25	0\\
57.26	0\\
57.27	0\\
57.28	0\\
57.29	0\\
57.3	0\\
57.31	0\\
57.32	0\\
57.33	0\\
57.34	0\\
57.35	0\\
57.36	0\\
57.37	0\\
57.38	0\\
57.39	0\\
57.4	0\\
57.41	0\\
57.42	0\\
57.43	0\\
57.44	0\\
57.45	0\\
57.46	0\\
57.47	0\\
57.48	0\\
57.49	0\\
57.5	0\\
57.51	0\\
57.52	0\\
57.53	0\\
57.54	0\\
57.55	0\\
57.56	0\\
57.57	0\\
57.58	0\\
57.59	0\\
57.6	0\\
57.61	0\\
57.62	0\\
57.63	0\\
57.64	0\\
57.65	0\\
57.66	0\\
57.67	0\\
57.68	0\\
57.69	0\\
57.7	0\\
57.71	0\\
57.72	0\\
57.73	0\\
57.74	0\\
57.75	0\\
57.76	0\\
57.77	0\\
57.78	0\\
57.79	0\\
57.8	0\\
57.81	0\\
57.82	0\\
57.83	0\\
57.84	0\\
57.85	0\\
57.86	0\\
57.87	0\\
57.88	0\\
57.89	0\\
57.9	0\\
57.91	0\\
57.92	0\\
57.93	0\\
57.94	0\\
57.95	0\\
57.96	0\\
57.97	0\\
57.98	0\\
57.99	0\\
58	0\\
58.01	0\\
58.02	0\\
58.03	0\\
58.04	0\\
58.05	0\\
58.06	0\\
58.07	0\\
58.08	0\\
58.09	0\\
58.1	0\\
58.11	0\\
58.12	0\\
58.13	0\\
58.14	0\\
58.15	0\\
58.16	0\\
58.17	0\\
58.18	0\\
58.19	0\\
58.2	0\\
58.21	0\\
58.22	0\\
58.23	0\\
58.24	0\\
58.25	0\\
58.26	0\\
58.27	0\\
58.28	0\\
58.29	0\\
58.3	0\\
58.31	0\\
58.32	0\\
58.33	0\\
58.34	0\\
58.35	0\\
58.36	0\\
58.37	0\\
58.38	0\\
58.39	0\\
58.4	0\\
58.41	0\\
58.42	0\\
58.43	0\\
58.44	0\\
58.45	0\\
58.46	0\\
58.47	0\\
58.48	0\\
58.49	0\\
58.5	0\\
58.51	0\\
58.52	0\\
58.53	0\\
58.54	0\\
58.55	0\\
58.56	0\\
58.57	0\\
58.58	0\\
58.59	0\\
58.6	0\\
58.61	0\\
58.62	0\\
58.63	0\\
58.64	0\\
58.65	0\\
58.66	0\\
58.67	0\\
58.68	0\\
58.69	0\\
58.7	0\\
58.71	0\\
58.72	0\\
58.73	0\\
58.74	0\\
58.75	0\\
58.76	0\\
58.77	0\\
58.78	0\\
58.79	0\\
58.8	0\\
58.81	0\\
58.82	0\\
58.83	0\\
58.84	0\\
58.85	0\\
58.86	0\\
58.87	0\\
58.88	0\\
58.89	0\\
58.9	0\\
58.91	0\\
58.92	0\\
58.93	0\\
58.94	0\\
58.95	0\\
58.96	0\\
58.97	0\\
58.98	0\\
58.99	0\\
59	0\\
59.01	0\\
59.02	0\\
59.03	0\\
59.04	0\\
59.05	0\\
59.06	0\\
59.07	0\\
59.08	0\\
59.09	0\\
59.1	0\\
59.11	0\\
59.12	0\\
59.13	0\\
59.14	0\\
59.15	0\\
59.16	0\\
59.17	0\\
59.18	0\\
59.19	0\\
59.2	0\\
59.21	0\\
59.22	0\\
59.23	0\\
59.24	0\\
59.25	0\\
59.26	0\\
59.27	0\\
59.28	0\\
59.29	0\\
59.3	0\\
59.31	0\\
59.32	0\\
59.33	0\\
59.34	0\\
59.35	0\\
59.36	0\\
59.37	0\\
59.38	0\\
59.39	0\\
59.4	0\\
59.41	0\\
59.42	0\\
59.43	0\\
59.44	0\\
59.45	0\\
59.46	0\\
59.47	0\\
59.48	0\\
59.49	0\\
59.5	0\\
59.51	0\\
59.52	0\\
59.53	0\\
59.54	0\\
59.55	0\\
59.56	0\\
59.57	0\\
59.58	0\\
59.59	0\\
59.6	0\\
59.61	0\\
59.62	0\\
59.63	0\\
59.64	0\\
59.65	0\\
59.66	0\\
59.67	0\\
59.68	0\\
59.69	0\\
59.7	0\\
59.71	0\\
59.72	0\\
59.73	0\\
59.74	0\\
59.75	0\\
59.76	0\\
59.77	0\\
59.78	0\\
59.79	0\\
59.8	0\\
59.81	0\\
59.82	0\\
59.83	0\\
59.84	0\\
59.85	0\\
59.86	0\\
59.87	0\\
59.88	0\\
59.89	0\\
59.9	0\\
59.91	0\\
59.92	0\\
59.93	0\\
59.94	0\\
59.95	0\\
59.96	0\\
59.97	0\\
59.98	0\\
59.99	0\\
60	0\\
60.01	0\\
60.02	0\\
60.03	0\\
60.04	0\\
60.05	0\\
60.06	0\\
60.07	0\\
60.08	0\\
60.09	0\\
60.1	0\\
60.11	0\\
60.12	0\\
60.13	0\\
60.14	0\\
60.15	0\\
60.16	0\\
60.17	0\\
60.18	0\\
60.19	0\\
60.2	0\\
60.21	0\\
60.22	0\\
60.23	0\\
60.24	0\\
60.25	0\\
60.26	0\\
60.27	0\\
60.28	0\\
60.29	0\\
60.3	0\\
60.31	0\\
60.32	0\\
60.33	0\\
60.34	0\\
60.35	0\\
60.36	0\\
60.37	0\\
60.38	0\\
60.39	0\\
60.4	0\\
60.41	0\\
60.42	0\\
60.43	0\\
60.44	0\\
60.45	0\\
60.46	0\\
60.47	0\\
60.48	0\\
60.49	0\\
60.5	0\\
60.51	0\\
60.52	0\\
60.53	0\\
60.54	0\\
60.55	0\\
60.56	0\\
60.57	0\\
60.58	0\\
60.59	0\\
60.6	0\\
60.61	0\\
60.62	0\\
60.63	0\\
60.64	0\\
60.65	0\\
60.66	0\\
60.67	0\\
60.68	0\\
60.69	0\\
60.7	0\\
60.71	0\\
60.72	0\\
60.73	0\\
60.74	0\\
60.75	0\\
60.76	0\\
60.77	0\\
60.78	0\\
60.79	0\\
60.8	0\\
60.81	0\\
60.82	0\\
60.83	0\\
60.84	0\\
60.85	0\\
60.86	0\\
60.87	0\\
60.88	0\\
60.89	0\\
60.9	0\\
60.91	0\\
60.92	0\\
60.93	0\\
60.94	0\\
60.95	0\\
60.96	0\\
60.97	0\\
60.98	0\\
60.99	0\\
61	0\\
61.01	0\\
61.02	0\\
61.03	0\\
61.04	0\\
61.05	0\\
61.06	0\\
61.07	0\\
61.08	0\\
61.09	0\\
61.1	0\\
61.11	0\\
61.12	0\\
61.13	0\\
61.14	0\\
61.15	0\\
61.16	0\\
61.17	0\\
61.18	0\\
61.19	0\\
61.2	0\\
61.21	0\\
61.22	0\\
61.23	0\\
61.24	0\\
61.25	0\\
61.26	0\\
61.27	0\\
61.28	0\\
61.29	0\\
61.3	0\\
61.31	0\\
61.32	0\\
61.33	0\\
61.34	0\\
61.35	0\\
61.36	0\\
61.37	0\\
61.38	0\\
61.39	0\\
61.4	0\\
61.41	0\\
61.42	0\\
61.43	0\\
61.44	0\\
61.45	0\\
61.46	0\\
61.47	0\\
61.48	0\\
61.49	0\\
61.5	0\\
61.51	0\\
61.52	0\\
61.53	0\\
61.54	0\\
61.55	0\\
61.56	0\\
61.57	0\\
61.58	0\\
61.59	0\\
61.6	0\\
61.61	0\\
61.62	0\\
61.63	0\\
61.64	0\\
61.65	0\\
61.66	0\\
61.67	0\\
61.68	0\\
61.69	0\\
61.7	0\\
61.71	0\\
61.72	0\\
61.73	0\\
61.74	0\\
61.75	0\\
61.76	0\\
61.77	0\\
61.78	0\\
61.79	0\\
61.8	0\\
61.81	0\\
61.82	0\\
61.83	0\\
61.84	0\\
61.85	0\\
61.86	0\\
61.87	0\\
61.88	0\\
61.89	0\\
61.9	0\\
61.91	0\\
61.92	0\\
61.93	0\\
61.94	0\\
61.95	0\\
61.96	0\\
61.97	0\\
61.98	0\\
61.99	0\\
62	0\\
62.01	0\\
62.02	0\\
62.03	0\\
62.04	0\\
62.05	0\\
62.06	0\\
62.07	0\\
62.08	0\\
62.09	0\\
62.1	0\\
62.11	0\\
62.12	0\\
62.13	0\\
62.14	0\\
62.15	0\\
62.16	0\\
62.17	0\\
62.18	0\\
62.19	0\\
62.2	0\\
62.21	0\\
62.22	0\\
62.23	0\\
62.24	0\\
62.25	0\\
62.26	0\\
62.27	0\\
62.28	0\\
62.29	0\\
62.3	0\\
62.31	0\\
62.32	0\\
62.33	0\\
62.34	0\\
62.35	0\\
62.36	0\\
62.37	0\\
62.38	0\\
62.39	0\\
62.4	0\\
62.41	0\\
62.42	0\\
62.43	0\\
62.44	0\\
62.45	0\\
62.46	0\\
62.47	0\\
62.48	0\\
62.49	0\\
62.5	0\\
62.51	0\\
62.52	0\\
62.53	0\\
62.54	0\\
62.55	0\\
62.56	0\\
62.57	0\\
62.58	0\\
62.59	0\\
62.6	0\\
62.61	0\\
62.62	0\\
62.63	0\\
62.64	0\\
62.65	0\\
62.66	0\\
62.67	0\\
62.68	0\\
62.69	0\\
62.7	0\\
62.71	0\\
62.72	0\\
62.73	0\\
62.74	0\\
62.75	0\\
62.76	0\\
62.77	0\\
62.78	0\\
62.79	0\\
62.8	0\\
62.81	0\\
62.82	0\\
62.83	0\\
62.84	0\\
62.85	0\\
62.86	0\\
62.87	0\\
62.88	0\\
62.89	0\\
62.9	0\\
62.91	0\\
62.92	0\\
62.93	0\\
62.94	0\\
62.95	0\\
62.96	0\\
62.97	0\\
62.98	0\\
62.99	0\\
63	0\\
63.01	0\\
63.02	0\\
63.03	0\\
63.04	0\\
63.05	0\\
63.06	0\\
63.07	0\\
63.08	0\\
63.09	0\\
63.1	0\\
63.11	0\\
63.12	0\\
63.13	0\\
63.14	0\\
63.15	0\\
63.16	0\\
63.17	0\\
63.18	0\\
63.19	0\\
63.2	0\\
63.21	0\\
63.22	0\\
63.23	0\\
63.24	0\\
63.25	0\\
63.26	0\\
63.27	0\\
63.28	0\\
63.29	0\\
63.3	0\\
63.31	0\\
63.32	0\\
63.33	0\\
63.34	0\\
63.35	0\\
63.36	0\\
63.37	0\\
63.38	0\\
63.39	0\\
63.4	0\\
63.41	0\\
63.42	0\\
63.43	0\\
63.44	0\\
63.45	0\\
63.46	0\\
63.47	0\\
63.48	0\\
63.49	0\\
63.5	0\\
63.51	0\\
63.52	0\\
63.53	0\\
63.54	0\\
63.55	0\\
63.56	0\\
63.57	0\\
63.58	0\\
63.59	0\\
63.6	0\\
63.61	0\\
63.62	0\\
63.63	0\\
63.64	0\\
63.65	0\\
63.66	0\\
63.67	0\\
63.68	0\\
63.69	0\\
63.7	0\\
63.71	0\\
63.72	0\\
63.73	0\\
63.74	0\\
63.75	0\\
63.76	0\\
63.77	0\\
63.78	0\\
63.79	0\\
63.8	0\\
63.81	0\\
63.82	0\\
63.83	0\\
63.84	0\\
63.85	0\\
63.86	0\\
63.87	0\\
63.88	0\\
63.89	0\\
63.9	0\\
63.91	0\\
63.92	0\\
63.93	0\\
63.94	0\\
63.95	0\\
63.96	0\\
63.97	0\\
63.98	0\\
63.99	0\\
64	0\\
64.01	0\\
64.02	0\\
64.03	0\\
64.04	0\\
64.05	0\\
64.06	0\\
64.07	0\\
64.08	0\\
64.09	0\\
64.1	0\\
64.11	0\\
64.12	0\\
64.13	0\\
64.14	0\\
64.15	0\\
64.16	0\\
64.17	0\\
64.18	0\\
64.19	0\\
64.2	0\\
64.21	0\\
64.22	0\\
64.23	0\\
64.24	0\\
64.25	0\\
64.26	0\\
64.27	0\\
64.28	0\\
64.29	0\\
64.3	0\\
64.31	0\\
64.32	0\\
64.33	0\\
64.34	0\\
64.35	0\\
64.36	0\\
64.37	0\\
64.38	0\\
64.39	0\\
64.4	0\\
64.41	0\\
64.42	0\\
64.43	0\\
64.44	0\\
64.45	0\\
64.46	0\\
64.47	0\\
64.48	0\\
64.49	0\\
64.5	0\\
64.51	0\\
64.52	0\\
64.53	0\\
64.54	0\\
64.55	0\\
64.56	0\\
64.57	0\\
64.58	0\\
64.59	0\\
64.6	0\\
64.61	0\\
64.62	0\\
64.63	0\\
64.64	0\\
64.65	0\\
64.66	0\\
64.67	0\\
64.68	0\\
64.69	0\\
64.7	0\\
64.71	0\\
64.72	0\\
64.73	0\\
64.74	0\\
64.75	0\\
64.76	0\\
64.77	0\\
64.78	0\\
64.79	0\\
64.8	0\\
64.81	0\\
64.82	0\\
64.83	0\\
64.84	0\\
64.85	0\\
64.86	0\\
64.87	0\\
64.88	0\\
64.89	0\\
64.9	0\\
64.91	0\\
64.92	0\\
64.93	0\\
64.94	0\\
64.95	0\\
64.96	0\\
64.97	0\\
64.98	0\\
64.99	0\\
65	0\\
65.01	0\\
65.02	0\\
65.03	0\\
65.04	0\\
65.05	0\\
65.06	0\\
65.07	0\\
65.08	0\\
65.09	0\\
65.1	0\\
65.11	0\\
65.12	0\\
65.13	0\\
65.14	0\\
65.15	0\\
65.16	0\\
65.17	0\\
65.18	0\\
65.19	0\\
65.2	0\\
65.21	0\\
65.22	0\\
65.23	0\\
65.24	0\\
65.25	0\\
65.26	0\\
65.27	0\\
65.28	0\\
65.29	0\\
65.3	0\\
65.31	0\\
65.32	0\\
65.33	0\\
65.34	0\\
65.35	0\\
65.36	0\\
65.37	0\\
65.38	0\\
65.39	0\\
65.4	0\\
65.41	0\\
65.42	0\\
65.43	0\\
65.44	0\\
65.45	0\\
65.46	0\\
65.47	0\\
65.48	0\\
65.49	0\\
65.5	0\\
65.51	0\\
65.52	0\\
65.53	0\\
65.54	0\\
65.55	0\\
65.56	0\\
65.57	0\\
65.58	0\\
65.59	0\\
65.6	0\\
65.61	0\\
65.62	0\\
65.63	0\\
65.64	0\\
65.65	0\\
65.66	0\\
65.67	0\\
65.68	0\\
65.69	0\\
65.7	0\\
65.71	0\\
65.72	0\\
65.73	0\\
65.74	0\\
65.75	0\\
65.76	0\\
65.77	0\\
65.78	0\\
65.79	0\\
65.8	0\\
65.81	0\\
65.82	0\\
65.83	0\\
65.84	0\\
65.85	0\\
65.86	0\\
65.87	0\\
65.88	0\\
65.89	0\\
65.9	0\\
65.91	0\\
65.92	0\\
65.93	0\\
65.94	0\\
65.95	0\\
65.96	0\\
65.97	0\\
65.98	0\\
65.99	0\\
66	0\\
66.01	0\\
66.02	0\\
66.03	0\\
66.04	0\\
66.05	0\\
66.06	0\\
66.07	0\\
66.08	0\\
66.09	0\\
66.1	0\\
66.11	0\\
66.12	0\\
66.13	0\\
66.14	0\\
66.15	0\\
66.16	0\\
66.17	0\\
66.18	0\\
66.19	0\\
66.2	0\\
66.21	0\\
66.22	0\\
66.23	0\\
66.24	0\\
66.25	0\\
66.26	0\\
66.27	0\\
66.28	0\\
66.29	0\\
66.3	0\\
66.31	0\\
66.32	0\\
66.33	0\\
66.34	0\\
66.35	0\\
66.36	0\\
66.37	0\\
66.38	0\\
66.39	0\\
66.4	0\\
66.41	0\\
66.42	0\\
66.43	0\\
66.44	0\\
66.45	0\\
66.46	0\\
66.47	0\\
66.48	0\\
66.49	0\\
66.5	0\\
66.51	0\\
66.52	0\\
66.53	0\\
66.54	0\\
66.55	0\\
66.56	0\\
66.57	0\\
66.58	0\\
66.59	0\\
66.6	0\\
66.61	0\\
66.62	0\\
66.63	0\\
66.64	0\\
66.65	0\\
66.66	0\\
66.67	0\\
66.68	0\\
66.69	0\\
66.7	0\\
66.71	0\\
66.72	0\\
66.73	0\\
66.74	0\\
66.75	0\\
66.76	0\\
66.77	0\\
66.78	0\\
66.79	0\\
66.8	0\\
66.81	0\\
66.82	0\\
66.83	0\\
66.84	0\\
66.85	0\\
66.86	0\\
66.87	0\\
66.88	0\\
66.89	0\\
66.9	0\\
66.91	0\\
66.92	0\\
66.93	0\\
66.94	0\\
66.95	0\\
66.96	0\\
66.97	0\\
66.98	0\\
66.99	0\\
67	0\\
67.01	0\\
67.02	0\\
67.03	0\\
67.04	0\\
67.05	0\\
67.06	0\\
67.07	0\\
67.08	0\\
67.09	0\\
67.1	0\\
67.11	0\\
67.12	0\\
67.13	0\\
67.14	0\\
67.15	0\\
67.16	0\\
67.17	0\\
67.18	0\\
67.19	0\\
67.2	0\\
67.21	0\\
67.22	0\\
67.23	0\\
67.24	0\\
67.25	0\\
67.26	0\\
67.27	0\\
67.28	0\\
67.29	0\\
67.3	0\\
67.31	0\\
67.32	0\\
67.33	0\\
67.34	0\\
67.35	0\\
67.36	0\\
67.37	0\\
67.38	0\\
67.39	0\\
67.4	0\\
67.41	0\\
67.42	0\\
67.43	0\\
67.44	0\\
67.45	0\\
67.46	0\\
67.47	0\\
67.48	0\\
67.49	0\\
67.5	0\\
67.51	0\\
67.52	0\\
67.53	0\\
67.54	0\\
67.55	0\\
67.56	0\\
67.57	0\\
67.58	0\\
67.59	0\\
67.6	0\\
67.61	0\\
67.62	0\\
67.63	0\\
67.64	0\\
67.65	0\\
67.66	0\\
67.67	0\\
67.68	0\\
67.69	0\\
67.7	0\\
67.71	0\\
67.72	0\\
67.73	0\\
67.74	0\\
67.75	0\\
67.76	0\\
67.77	0\\
67.78	0\\
67.79	0\\
67.8	0\\
67.81	0\\
67.82	0\\
67.83	0\\
67.84	0\\
67.85	0\\
67.86	0\\
67.87	0\\
67.88	0\\
67.89	0\\
67.9	0\\
67.91	0\\
67.92	0\\
67.93	0\\
67.94	0\\
67.95	0\\
67.96	0\\
67.97	0\\
67.98	0\\
67.99	0\\
68	0\\
68.01	0\\
68.02	0\\
68.03	0\\
68.04	0\\
68.05	0\\
68.06	0\\
68.07	0\\
68.08	0\\
68.09	0\\
68.1	0\\
68.11	0\\
68.12	0\\
68.13	0\\
68.14	0\\
68.15	0\\
68.16	0\\
68.17	0\\
68.18	0\\
68.19	0\\
68.2	0\\
68.21	0\\
68.22	0\\
68.23	0\\
68.24	0\\
68.25	0\\
68.26	0\\
68.27	0\\
68.28	0\\
68.29	0\\
68.3	0\\
68.31	0\\
68.32	0\\
68.33	0\\
68.34	0\\
68.35	0\\
68.36	0\\
68.37	0\\
68.38	0\\
68.39	0\\
68.4	0\\
68.41	0\\
68.42	0\\
68.43	0\\
68.44	0\\
68.45	0\\
68.46	0\\
68.47	0\\
68.48	0\\
68.49	0\\
68.5	0\\
68.51	0\\
68.52	0\\
68.53	0\\
68.54	0\\
68.55	0\\
68.56	0\\
68.57	0\\
68.58	0\\
68.59	0\\
68.6	0\\
68.61	0\\
68.62	0\\
68.63	0\\
68.64	0\\
68.65	0\\
68.66	0\\
68.67	0\\
68.68	0\\
68.69	0\\
68.7	0\\
68.71	0\\
68.72	0\\
68.73	0\\
68.74	0\\
68.75	0\\
68.76	0\\
68.77	0\\
68.78	0\\
68.79	0\\
68.8	0\\
68.81	0\\
68.82	0\\
68.83	0\\
68.84	0\\
68.85	0\\
68.86	0\\
68.87	0\\
68.88	0\\
68.89	0\\
68.9	0\\
68.91	0\\
68.92	0\\
68.93	0\\
68.94	0\\
68.95	0\\
68.96	0\\
68.97	0\\
68.98	0\\
68.99	0\\
69	0\\
69.01	0\\
69.02	0\\
69.03	0\\
69.04	0\\
69.05	0\\
69.06	0\\
69.07	0\\
69.08	0\\
69.09	0\\
69.1	0\\
69.11	0\\
69.12	0\\
69.13	0\\
69.14	0\\
69.15	0\\
69.16	0\\
69.17	0\\
69.18	0\\
69.19	0\\
69.2	0\\
69.21	0\\
69.22	0\\
69.23	0\\
69.24	0\\
69.25	0\\
69.26	0\\
69.27	0\\
69.28	0\\
69.29	0\\
69.3	0\\
69.31	0\\
69.32	0\\
69.33	0\\
69.34	0\\
69.35	0\\
69.36	0\\
69.37	0\\
69.38	0\\
69.39	0\\
69.4	0\\
69.41	0\\
69.42	0\\
69.43	0\\
69.44	0\\
69.45	0\\
69.46	0\\
69.47	0\\
69.48	0\\
69.49	0\\
69.5	0\\
69.51	0\\
69.52	0\\
69.53	0\\
69.54	0\\
69.55	0\\
69.56	0\\
69.57	0\\
69.58	0\\
69.59	0\\
69.6	0\\
69.61	0\\
69.62	0\\
69.63	0\\
69.64	0\\
69.65	0\\
69.66	0\\
69.67	0\\
69.68	0\\
69.69	0\\
69.7	0\\
69.71	0\\
69.72	0\\
69.73	0\\
69.74	0\\
69.75	0\\
69.76	0\\
69.77	0\\
69.78	0\\
69.79	0\\
69.8	0\\
69.81	0\\
69.82	0\\
69.83	0\\
69.84	0\\
69.85	0\\
69.86	0\\
69.87	0\\
69.88	0\\
69.89	0\\
69.9	0\\
69.91	0\\
69.92	0\\
69.93	0\\
69.94	0\\
69.95	0\\
69.96	0\\
69.97	0\\
69.98	0\\
69.99	0\\
70	0\\
70.01	0\\
70.02	0\\
70.03	0\\
70.04	0\\
70.05	0\\
70.06	0\\
70.07	0\\
70.08	0\\
70.09	0\\
70.1	0\\
70.11	0\\
70.12	0\\
70.13	0\\
70.14	0\\
70.15	0\\
70.16	0\\
70.17	0\\
70.18	0\\
70.19	0\\
70.2	0\\
70.21	0\\
70.22	0\\
70.23	0\\
70.24	0\\
70.25	0\\
70.26	0\\
70.27	0\\
70.28	0\\
70.29	0\\
70.3	0\\
70.31	0\\
70.32	0\\
70.33	0\\
70.34	0\\
70.35	0\\
70.36	0\\
70.37	0\\
70.38	0\\
70.39	0\\
70.4	0\\
70.41	0\\
70.42	0\\
70.43	0\\
70.44	0\\
70.45	0\\
70.46	0\\
70.47	0\\
70.48	0\\
70.49	0\\
70.5	0\\
70.51	0\\
70.52	0\\
70.53	0\\
70.54	0\\
70.55	0\\
70.56	0\\
70.57	0\\
70.58	0\\
70.59	0\\
70.6	0\\
70.61	0\\
70.62	0\\
70.63	0\\
70.64	0\\
70.65	0\\
70.66	0\\
70.67	0\\
70.68	0\\
70.69	0\\
70.7	0\\
70.71	0\\
70.72	0\\
70.73	0\\
70.74	0\\
70.75	0\\
70.76	0\\
70.77	0\\
70.78	0\\
70.79	0\\
70.8	0\\
70.81	0\\
70.82	0\\
70.83	0\\
70.84	0\\
70.85	0\\
70.86	0\\
70.87	0\\
70.88	0\\
70.89	0\\
70.9	0\\
70.91	0\\
70.92	0\\
70.93	0\\
70.94	0\\
70.95	0\\
70.96	0\\
70.97	0\\
70.98	0\\
70.99	0\\
71	0\\
71.01	0\\
71.02	0\\
71.03	0\\
71.04	0\\
71.05	0\\
71.06	0\\
71.07	0\\
71.08	0\\
71.09	0\\
71.1	0\\
71.11	0\\
71.12	0\\
71.13	0\\
71.14	0\\
71.15	0\\
71.16	0\\
71.17	0\\
71.18	0\\
71.19	0\\
71.2	0\\
71.21	0\\
71.22	0\\
71.23	0\\
71.24	0\\
71.25	0\\
71.26	0\\
71.27	0\\
71.28	0\\
71.29	0\\
71.3	0\\
71.31	0\\
71.32	0\\
71.33	0\\
71.34	0\\
71.35	0\\
71.36	0\\
71.37	0\\
71.38	0\\
71.39	0\\
71.4	0\\
71.41	0\\
71.42	0\\
71.43	0\\
71.44	0\\
71.45	0\\
71.46	0\\
71.47	0\\
71.48	0\\
71.49	0\\
71.5	0\\
71.51	0\\
71.52	0\\
71.53	0\\
71.54	0\\
71.55	0\\
71.56	0\\
71.57	0\\
71.58	0\\
71.59	0\\
71.6	0\\
71.61	0\\
71.62	0\\
71.63	0\\
71.64	0\\
71.65	0\\
71.66	0\\
71.67	0\\
71.68	0\\
71.69	0\\
71.7	0\\
71.71	0\\
71.72	0\\
71.73	0\\
71.74	0\\
71.75	0\\
71.76	0\\
71.77	0\\
71.78	0\\
71.79	0\\
71.8	0\\
71.81	0\\
71.82	0\\
71.83	0\\
71.84	0\\
71.85	0\\
71.86	0\\
71.87	0\\
71.88	0\\
71.89	0\\
71.9	0\\
71.91	0\\
71.92	0\\
71.93	0\\
71.94	0\\
71.95	0\\
71.96	0\\
71.97	0\\
71.98	0\\
71.99	0\\
72	0\\
72.01	0\\
72.02	0\\
72.03	0\\
72.04	0\\
72.05	0\\
72.06	0\\
72.07	0\\
72.08	0\\
72.09	0\\
72.1	0\\
72.11	0\\
72.12	0\\
72.13	0\\
72.14	0\\
72.15	0\\
72.16	0\\
72.17	0\\
72.18	0\\
72.19	0\\
72.2	0\\
72.21	0\\
72.22	0\\
72.23	0\\
72.24	0\\
72.25	0\\
72.26	0\\
72.27	0\\
72.28	0\\
72.29	0\\
72.3	0\\
72.31	0\\
72.32	0\\
72.33	0\\
72.34	0\\
72.35	0\\
72.36	0\\
72.37	0\\
72.38	0\\
72.39	0\\
72.4	0\\
72.41	0\\
72.42	0\\
72.43	0\\
72.44	0\\
72.45	0\\
72.46	0\\
72.47	0\\
72.48	0\\
72.49	0\\
72.5	0\\
72.51	0\\
72.52	0\\
72.53	0\\
72.54	0\\
72.55	0\\
72.56	0\\
72.57	0\\
72.58	0\\
72.59	0\\
72.6	0\\
72.61	0\\
72.62	0\\
72.63	0\\
72.64	0\\
72.65	0\\
72.66	0\\
72.67	0\\
72.68	0\\
72.69	0\\
72.7	0\\
72.71	0\\
72.72	0\\
72.73	0\\
72.74	0\\
72.75	0\\
72.76	0\\
72.77	0\\
72.78	0\\
72.79	0\\
72.8	0\\
72.81	0\\
72.82	0\\
72.83	0\\
72.84	0\\
72.85	0\\
72.86	0\\
72.87	0\\
72.88	0\\
72.89	0\\
72.9	0\\
72.91	0\\
72.92	0\\
72.93	0\\
72.94	0\\
72.95	0\\
72.96	0\\
72.97	0\\
72.98	0\\
72.99	0\\
73	0\\
73.01	0\\
73.02	0\\
73.03	0\\
73.04	0\\
73.05	0\\
73.06	0\\
73.07	0\\
73.08	0\\
73.09	0\\
73.1	0\\
73.11	0\\
73.12	0\\
73.13	0\\
73.14	0\\
73.15	0\\
73.16	0\\
73.17	0\\
73.18	0\\
73.19	0\\
73.2	0\\
73.21	0\\
73.22	0\\
73.23	0\\
73.24	0\\
73.25	0\\
73.26	0\\
73.27	0\\
73.28	0\\
73.29	0\\
73.3	0\\
73.31	0\\
73.32	0\\
73.33	0\\
73.34	0\\
73.35	0\\
73.36	0\\
73.37	0\\
73.38	0\\
73.39	0\\
73.4	0\\
73.41	0\\
73.42	0\\
73.43	0\\
73.44	0\\
73.45	0\\
73.46	0\\
73.47	0\\
73.48	0\\
73.49	0\\
73.5	0\\
73.51	0\\
73.52	0\\
73.53	0\\
73.54	0\\
73.55	0\\
73.56	0\\
73.57	0\\
73.58	0\\
73.59	0\\
73.6	0\\
73.61	0\\
73.62	0\\
73.63	0\\
73.64	0\\
73.65	0\\
73.66	0\\
73.67	0\\
73.68	0\\
73.69	0\\
73.7	0\\
73.71	0\\
73.72	0\\
73.73	0\\
73.74	0\\
73.75	0\\
73.76	0\\
73.77	0\\
73.78	0\\
73.79	0\\
73.8	0\\
73.81	0\\
73.82	0\\
73.83	0\\
73.84	0\\
73.85	0\\
73.86	0\\
73.87	0\\
73.88	0\\
73.89	0\\
73.9	0\\
73.91	0\\
73.92	0\\
73.93	0\\
73.94	0\\
73.95	0\\
73.96	0\\
73.97	0\\
73.98	0\\
73.99	0\\
74	0\\
74.01	0\\
74.02	0\\
74.03	0\\
74.04	0\\
74.05	0\\
74.06	0\\
74.07	0\\
74.08	0\\
74.09	0\\
74.1	0\\
74.11	0\\
74.12	0\\
74.13	0\\
74.14	0\\
74.15	0\\
74.16	0\\
74.17	0\\
74.18	0\\
74.19	0\\
74.2	0\\
74.21	0\\
74.22	0\\
74.23	0\\
74.24	0\\
74.25	0\\
74.26	0\\
74.27	0\\
74.28	0\\
74.29	0\\
74.3	0\\
74.31	0\\
74.32	0\\
74.33	0\\
74.34	0\\
74.35	0\\
74.36	0\\
74.37	0\\
74.38	0\\
74.39	0\\
74.4	0\\
74.41	0\\
74.42	0\\
74.43	0\\
74.44	0\\
74.45	0\\
74.46	0\\
74.47	0\\
74.48	0\\
74.49	0\\
74.5	0\\
74.51	0\\
74.52	0\\
74.53	0\\
74.54	0\\
74.55	0\\
74.56	0\\
74.57	0\\
74.58	0\\
74.59	0\\
74.6	0\\
74.61	0\\
74.62	0\\
74.63	0\\
74.64	0\\
74.65	0\\
74.66	0\\
74.67	0\\
74.68	0\\
74.69	0\\
74.7	0\\
74.71	0\\
74.72	0\\
74.73	0\\
74.74	0\\
74.75	0\\
74.76	0\\
74.77	0\\
74.78	0\\
74.79	0\\
74.8	0\\
74.81	0\\
74.82	0\\
74.83	0\\
74.84	0\\
74.85	0\\
74.86	0\\
74.87	0\\
74.88	0\\
74.89	0\\
74.9	0\\
74.91	0\\
74.92	0\\
74.93	0\\
74.94	0\\
74.95	0\\
74.96	0\\
74.97	0\\
74.98	0\\
74.99	0\\
75	0\\
75.01	0\\
75.02	0\\
75.03	0\\
75.04	0\\
75.05	0\\
75.06	0\\
75.07	0\\
75.08	0\\
75.09	0\\
75.1	0\\
75.11	0\\
75.12	0\\
75.13	0\\
75.14	0\\
75.15	0\\
75.16	0\\
75.17	0\\
75.18	0\\
75.19	0\\
75.2	0\\
75.21	0\\
75.22	0\\
75.23	0\\
75.24	0\\
75.25	0\\
75.26	0\\
75.27	0\\
75.28	0\\
75.29	0\\
75.3	0\\
75.31	0\\
75.32	0\\
75.33	0\\
75.34	0\\
75.35	0\\
75.36	0\\
75.37	0\\
75.38	0\\
75.39	0\\
75.4	0\\
75.41	0\\
75.42	0\\
75.43	0\\
75.44	0\\
75.45	0\\
75.46	0\\
75.47	0\\
75.48	0\\
75.49	0\\
75.5	0\\
75.51	0\\
75.52	0\\
75.53	0\\
75.54	0\\
75.55	0\\
75.56	0\\
75.57	0\\
75.58	0\\
75.59	0\\
75.6	0\\
75.61	0\\
75.62	0\\
75.63	0\\
75.64	0\\
75.65	0\\
75.66	0\\
75.67	0\\
75.68	0\\
75.69	0\\
75.7	0\\
75.71	0\\
75.72	0\\
75.73	0\\
75.74	0\\
75.75	0\\
75.76	0\\
75.77	0\\
75.78	0\\
75.79	0\\
75.8	0\\
75.81	0\\
75.82	0\\
75.83	0\\
75.84	0\\
75.85	0\\
75.86	0\\
75.87	0\\
75.88	0\\
75.89	0\\
75.9	0\\
75.91	0\\
75.92	0\\
75.93	0\\
75.94	0\\
75.95	0\\
75.96	0\\
75.97	0\\
75.98	0\\
75.99	0\\
76	0\\
76.01	0\\
76.02	0\\
76.03	0\\
76.04	0\\
76.05	0\\
76.06	0\\
76.07	0\\
76.08	0\\
76.09	0\\
76.1	0\\
76.11	0\\
76.12	0\\
76.13	0\\
76.14	0\\
76.15	0\\
76.16	0\\
76.17	0\\
76.18	0\\
76.19	0\\
76.2	0\\
76.21	0\\
76.22	0\\
76.23	0\\
76.24	0\\
76.25	0\\
76.26	0\\
76.27	0\\
76.28	0\\
76.29	0\\
76.3	0\\
76.31	0\\
76.32	0\\
76.33	0\\
76.34	0\\
76.35	0\\
76.36	0\\
76.37	0\\
76.38	0\\
76.39	0\\
76.4	0\\
76.41	0\\
76.42	0\\
76.43	0\\
76.44	0\\
76.45	0\\
76.46	0\\
76.47	0\\
76.48	0\\
76.49	0\\
76.5	0\\
76.51	0\\
76.52	0\\
76.53	0\\
76.54	0\\
76.55	0\\
76.56	0\\
76.57	0\\
76.58	0\\
76.59	0\\
76.6	0\\
76.61	0\\
76.62	0\\
76.63	0\\
76.64	0\\
76.65	0\\
76.66	0\\
76.67	0\\
76.68	0\\
76.69	0\\
76.7	0\\
76.71	0\\
76.72	0\\
76.73	0\\
76.74	0\\
76.75	0\\
76.76	0\\
76.77	0\\
76.78	0\\
76.79	0\\
76.8	0\\
76.81	0\\
76.82	0\\
76.83	0\\
76.84	0\\
76.85	0\\
76.86	0\\
76.87	0\\
76.88	0\\
76.89	0\\
76.9	0\\
76.91	0\\
76.92	0\\
76.93	0\\
76.94	0\\
76.95	0\\
76.96	0\\
76.97	0\\
76.98	0\\
76.99	0\\
77	0\\
77.01	0\\
77.02	0\\
77.03	0\\
77.04	0\\
77.05	0\\
77.06	0\\
77.07	0\\
77.08	0\\
77.09	0\\
77.1	0\\
77.11	0\\
77.12	0\\
77.13	0\\
77.14	0\\
77.15	0\\
77.16	0\\
77.17	0\\
77.18	0\\
77.19	0\\
77.2	0\\
77.21	0\\
77.22	0\\
77.23	0\\
77.24	0\\
77.25	0\\
77.26	0\\
77.27	0\\
77.28	0\\
77.29	0\\
77.3	0\\
77.31	0\\
77.32	0\\
77.33	0\\
77.34	0\\
77.35	0\\
77.36	0\\
77.37	0\\
77.38	0\\
77.39	0\\
77.4	0\\
77.41	0\\
77.42	0\\
77.43	0\\
77.44	0\\
77.45	0\\
77.46	0\\
77.47	0\\
77.48	0\\
77.49	0\\
77.5	0\\
77.51	0\\
77.52	0\\
77.53	0\\
77.54	0\\
77.55	0\\
77.56	0\\
77.57	0\\
77.58	0\\
77.59	0\\
77.6	0\\
77.61	0\\
77.62	0\\
77.63	0\\
77.64	0\\
77.65	0\\
77.66	0\\
77.67	0\\
77.68	0\\
77.69	0\\
77.7	0\\
77.71	0\\
77.72	0\\
77.73	0\\
77.74	0\\
77.75	0\\
77.76	0\\
77.77	0\\
77.78	0\\
77.79	0\\
77.8	0\\
77.81	0\\
77.82	0\\
77.83	0\\
77.84	0\\
77.85	0\\
77.86	0\\
77.87	0\\
77.88	0\\
77.89	0\\
77.9	0\\
77.91	0\\
77.92	0\\
77.93	0\\
77.94	0\\
77.95	0\\
77.96	0\\
77.97	0\\
77.98	0\\
77.99	0\\
78	0\\
78.01	0\\
78.02	0\\
78.03	0\\
78.04	0\\
78.05	0\\
78.06	0\\
78.07	0\\
78.08	0\\
78.09	0\\
78.1	0\\
78.11	0\\
78.12	0\\
78.13	0\\
78.14	0\\
78.15	0\\
78.16	0\\
78.17	0\\
78.18	0\\
78.19	0\\
78.2	0\\
78.21	0\\
78.22	0\\
78.23	0\\
78.24	0\\
78.25	0\\
78.26	0\\
78.27	0\\
78.28	0\\
78.29	0\\
78.3	0\\
78.31	0\\
78.32	0\\
78.33	0\\
78.34	0\\
78.35	0\\
78.36	0\\
78.37	0\\
78.38	0\\
78.39	0\\
78.4	0\\
78.41	0\\
78.42	0\\
78.43	0\\
78.44	0\\
78.45	0\\
78.46	0\\
78.47	0\\
78.48	0\\
78.49	0\\
78.5	0\\
78.51	0\\
78.52	0\\
78.53	0\\
78.54	0\\
78.55	0\\
78.56	0\\
78.57	0\\
78.58	0\\
78.59	0\\
78.6	0\\
78.61	0\\
78.62	0\\
78.63	0\\
78.64	0\\
78.65	0\\
78.66	0\\
78.67	0\\
78.68	0\\
78.69	0\\
78.7	0\\
78.71	0\\
78.72	0\\
78.73	0\\
78.74	0\\
78.75	0\\
78.76	0\\
78.77	0\\
78.78	0\\
78.79	0\\
78.8	0\\
78.81	0\\
78.82	0\\
78.83	0\\
78.84	0\\
78.85	0\\
78.86	0\\
78.87	0\\
78.88	0\\
78.89	0\\
78.9	0\\
78.91	0\\
78.92	0\\
78.93	0\\
78.94	0\\
78.95	0\\
78.96	0\\
78.97	0\\
78.98	0\\
78.99	0\\
79	0\\
79.01	0\\
79.02	0\\
79.03	0\\
79.04	0\\
79.05	0\\
79.06	0\\
79.07	0\\
79.08	0\\
79.09	0\\
79.1	0\\
79.11	0\\
79.12	0\\
79.13	0\\
79.14	0\\
79.15	0\\
79.16	0\\
79.17	0\\
79.18	0\\
79.19	0\\
79.2	0\\
79.21	0\\
79.22	0\\
79.23	0\\
79.24	0\\
79.25	0\\
79.26	0\\
79.27	0\\
79.28	0\\
79.29	0\\
79.3	0\\
79.31	0\\
79.32	0\\
79.33	0\\
79.34	0\\
79.35	0\\
79.36	0\\
79.37	0\\
79.38	0\\
79.39	0\\
79.4	0\\
79.41	0\\
79.42	0\\
79.43	0\\
79.44	0\\
79.45	0\\
79.46	0\\
79.47	0\\
79.48	0\\
79.49	0\\
79.5	0\\
79.51	0\\
79.52	0\\
79.53	0\\
79.54	0\\
79.55	0\\
79.56	0\\
79.57	0\\
79.58	0\\
79.59	0\\
79.6	0\\
79.61	0\\
79.62	0\\
79.63	0\\
79.64	0\\
79.65	0\\
79.66	0\\
79.67	0\\
79.68	0\\
79.69	0\\
79.7	0\\
79.71	0\\
79.72	0\\
79.73	0\\
79.74	0\\
79.75	0\\
79.76	0\\
79.77	0\\
79.78	0\\
79.79	0\\
79.8	0\\
79.81	0\\
79.82	0\\
79.83	0\\
79.84	0\\
79.85	0\\
79.86	0\\
79.87	0\\
79.88	0\\
79.89	0\\
79.9	0\\
79.91	0\\
79.92	0\\
79.93	0\\
79.94	0\\
79.95	0\\
79.96	0\\
79.97	0\\
79.98	0\\
79.99	0\\
80	0\\
80.01	0\\
};
\addplot [color=blue,dashed]
  table[row sep=crcr]{%
80.01	0\\
80.02	0\\
80.03	0\\
80.04	0\\
80.05	0\\
80.06	0\\
80.07	0\\
80.08	0\\
80.09	0\\
80.1	0\\
80.11	0\\
80.12	0\\
80.13	0\\
80.14	0\\
80.15	0\\
80.16	0\\
80.17	0\\
80.18	0\\
80.19	0\\
80.2	0\\
80.21	0\\
80.22	0\\
80.23	0\\
80.24	0\\
80.25	0\\
80.26	0\\
80.27	0\\
80.28	0\\
80.29	0\\
80.3	0\\
80.31	0\\
80.32	0\\
80.33	0\\
80.34	0\\
80.35	0\\
80.36	0\\
80.37	0\\
80.38	0\\
80.39	0\\
80.4	0\\
80.41	0\\
80.42	0\\
80.43	0\\
80.44	0\\
80.45	0\\
80.46	0\\
80.47	0\\
80.48	0\\
80.49	0\\
80.5	0\\
80.51	0\\
80.52	0\\
80.53	0\\
80.54	0\\
80.55	0\\
80.56	0\\
80.57	0\\
80.58	0\\
80.59	0\\
80.6	0\\
80.61	0\\
80.62	0\\
80.63	0\\
80.64	0\\
80.65	0\\
80.66	0\\
80.67	0\\
80.68	0\\
80.69	0\\
80.7	0\\
80.71	0\\
80.72	0\\
80.73	0\\
80.74	0\\
80.75	0\\
80.76	0\\
80.77	0\\
80.78	0\\
80.79	0\\
80.8	0\\
80.81	0\\
80.82	0\\
80.83	0\\
80.84	0\\
80.85	0\\
80.86	0\\
80.87	0\\
80.88	0\\
80.89	0\\
80.9	0\\
80.91	0\\
80.92	0\\
80.93	0\\
80.94	0\\
80.95	0\\
80.96	0\\
80.97	0\\
80.98	0\\
80.99	0\\
81	0\\
81.01	0\\
81.02	0\\
81.03	0\\
81.04	0\\
81.05	0\\
81.06	0\\
81.07	0\\
81.08	0\\
81.09	0\\
81.1	0\\
81.11	0\\
81.12	0\\
81.13	0\\
81.14	0\\
81.15	0\\
81.16	0\\
81.17	0\\
81.18	0\\
81.19	0\\
81.2	0\\
81.21	0\\
81.22	0\\
81.23	0\\
81.24	0\\
81.25	0\\
81.26	0\\
81.27	0\\
81.28	0\\
81.29	0\\
81.3	0\\
81.31	0\\
81.32	0\\
81.33	0\\
81.34	0\\
81.35	0\\
81.36	0\\
81.37	0\\
81.38	0\\
81.39	0\\
81.4	0\\
81.41	0\\
81.42	0\\
81.43	0\\
81.44	0\\
81.45	0\\
81.46	0\\
81.47	0\\
81.48	0\\
81.49	0\\
81.5	0\\
81.51	0\\
81.52	0\\
81.53	0\\
81.54	0\\
81.55	0\\
81.56	0\\
81.57	0\\
81.58	0\\
81.59	0\\
81.6	0\\
81.61	0\\
81.62	0\\
81.63	0\\
81.64	0\\
81.65	0\\
81.66	0\\
81.67	0\\
81.68	0\\
81.69	0\\
81.7	0\\
81.71	0\\
81.72	0\\
81.73	0\\
81.74	0\\
81.75	0\\
81.76	0\\
81.77	0\\
81.78	0\\
81.79	0\\
81.8	0\\
81.81	0\\
81.82	0\\
81.83	0\\
81.84	0\\
81.85	0\\
81.86	0\\
81.87	0\\
81.88	0\\
81.89	0\\
81.9	0\\
81.91	0\\
81.92	0\\
81.93	0\\
81.94	0\\
81.95	0\\
81.96	0\\
81.97	0\\
81.98	0\\
81.99	0\\
82	0\\
82.01	0\\
82.02	0\\
82.03	0\\
82.04	0\\
82.05	0\\
82.06	0\\
82.07	0\\
82.08	0\\
82.09	0\\
82.1	0\\
82.11	0\\
82.12	0\\
82.13	0\\
82.14	0\\
82.15	0\\
82.16	0\\
82.17	0\\
82.18	0\\
82.19	0\\
82.2	0\\
82.21	0\\
82.22	0\\
82.23	0\\
82.24	0\\
82.25	0\\
82.26	0\\
82.27	0\\
82.28	0\\
82.29	0\\
82.3	0\\
82.31	0\\
82.32	0\\
82.33	0\\
82.34	0\\
82.35	0\\
82.36	0\\
82.37	0\\
82.38	0\\
82.39	0\\
82.4	0\\
82.41	0\\
82.42	0\\
82.43	0\\
82.44	0\\
82.45	0\\
82.46	0\\
82.47	0\\
82.48	0\\
82.49	0\\
82.5	0\\
82.51	0\\
82.52	0\\
82.53	0\\
82.54	0\\
82.55	0\\
82.56	0\\
82.57	0\\
82.58	0\\
82.59	0\\
82.6	0\\
82.61	0\\
82.62	0\\
82.63	0\\
82.64	0\\
82.65	0\\
82.66	0\\
82.67	0\\
82.68	0\\
82.69	0\\
82.7	0\\
82.71	0\\
82.72	0\\
82.73	0\\
82.74	0\\
82.75	0\\
82.76	0\\
82.77	0\\
82.78	0\\
82.79	0\\
82.8	0\\
82.81	0\\
82.82	0\\
82.83	0\\
82.84	0\\
82.85	0\\
82.86	0\\
82.87	0\\
82.88	0\\
82.89	0\\
82.9	0\\
82.91	0\\
82.92	0\\
82.93	0\\
82.94	0\\
82.95	0\\
82.96	0\\
82.97	0\\
82.98	0\\
82.99	0\\
83	0\\
83.01	0\\
83.02	0\\
83.03	0\\
83.04	0\\
83.05	0\\
83.06	0\\
83.07	0\\
83.08	0\\
83.09	0\\
83.1	0\\
83.11	0\\
83.12	0\\
83.13	0\\
83.14	0\\
83.15	0\\
83.16	0\\
83.17	0\\
83.18	0\\
83.19	0\\
83.2	0\\
83.21	0\\
83.22	0\\
83.23	0\\
83.24	0\\
83.25	0\\
83.26	0\\
83.27	0\\
83.28	0\\
83.29	0\\
83.3	0\\
83.31	0\\
83.32	0\\
83.33	0\\
83.34	0\\
83.35	0\\
83.36	0\\
83.37	0\\
83.38	0\\
83.39	0\\
83.4	0\\
83.41	0\\
83.42	0\\
83.43	0\\
83.44	0\\
83.45	0\\
83.46	0\\
83.47	0\\
83.48	0\\
83.49	0\\
83.5	0\\
83.51	0\\
83.52	0\\
83.53	0\\
83.54	0\\
83.55	0\\
83.56	0\\
83.57	0\\
83.58	0\\
83.59	0\\
83.6	0\\
83.61	0\\
83.62	0\\
83.63	0\\
83.64	0\\
83.65	0\\
83.66	0\\
83.67	0\\
83.68	0\\
83.69	0\\
83.7	0\\
83.71	0\\
83.72	0\\
83.73	0\\
83.74	0\\
83.75	0\\
83.76	0\\
83.77	0\\
83.78	0\\
83.79	0\\
83.8	0\\
83.81	0\\
83.82	0\\
83.83	0\\
83.84	0\\
83.85	0\\
83.86	0\\
83.87	0\\
83.88	0\\
83.89	0\\
83.9	0\\
83.91	0\\
83.92	0\\
83.93	0\\
83.94	0\\
83.95	0\\
83.96	0\\
83.97	0\\
83.98	0\\
83.99	0\\
84	0\\
84.01	0\\
84.02	0\\
84.03	0\\
84.04	0\\
84.05	0\\
84.06	0\\
84.07	0\\
84.08	0\\
84.09	0\\
84.1	0\\
84.11	0\\
84.12	0\\
84.13	0\\
84.14	0\\
84.15	0\\
84.16	0\\
84.17	0\\
84.18	0\\
84.19	0\\
84.2	0\\
84.21	0\\
84.22	0\\
84.23	0\\
84.24	0\\
84.25	0\\
84.26	0\\
84.27	0\\
84.28	0\\
84.29	0\\
84.3	0\\
84.31	0\\
84.32	0\\
84.33	0\\
84.34	0\\
84.35	0\\
84.36	0\\
84.37	0\\
84.38	0\\
84.39	0\\
84.4	0\\
84.41	0\\
84.42	0\\
84.43	0\\
84.44	0\\
84.45	0\\
84.46	0\\
84.47	0\\
84.48	0\\
84.49	0\\
84.5	0\\
84.51	0\\
84.52	0\\
84.53	0\\
84.54	0\\
84.55	0\\
84.56	0\\
84.57	0\\
84.58	0\\
84.59	0\\
84.6	0\\
84.61	0\\
84.62	0\\
84.63	0\\
84.64	0\\
84.65	0\\
84.66	0\\
84.67	0\\
84.68	0\\
84.69	0\\
84.7	0\\
84.71	0\\
84.72	0\\
84.73	0\\
84.74	0\\
84.75	0\\
84.76	0\\
84.77	0\\
84.78	0\\
84.79	0\\
84.8	0\\
84.81	0\\
84.82	0\\
84.83	0\\
84.84	0\\
84.85	0\\
84.86	0\\
84.87	0\\
84.88	0\\
84.89	0\\
84.9	0\\
84.91	0\\
84.92	0\\
84.93	0\\
84.94	0\\
84.95	0\\
84.96	0\\
84.97	0\\
84.98	0\\
84.99	0\\
85	0\\
85.01	0\\
85.02	0\\
85.03	0\\
85.04	0\\
85.05	0\\
85.06	0\\
85.07	0\\
85.08	0\\
85.09	0\\
85.1	0\\
85.11	0\\
85.12	0\\
85.13	0\\
85.14	0\\
85.15	0\\
85.16	0\\
85.17	0\\
85.18	0\\
85.19	0\\
85.2	0\\
85.21	0\\
85.22	0\\
85.23	0\\
85.24	0\\
85.25	0\\
85.26	0\\
85.27	0\\
85.28	0\\
85.29	0\\
85.3	0\\
85.31	0\\
85.32	0\\
85.33	0\\
85.34	0\\
85.35	0\\
85.36	0\\
85.37	0\\
85.38	0\\
85.39	0\\
85.4	0\\
85.41	0\\
85.42	0\\
85.43	0\\
85.44	0\\
85.45	0\\
85.46	0\\
85.47	0\\
85.48	0\\
85.49	0\\
85.5	0\\
85.51	0\\
85.52	0\\
85.53	0\\
85.54	0\\
85.55	0\\
85.56	0\\
85.57	0\\
85.58	0\\
85.59	0\\
85.6	0\\
85.61	0\\
85.62	0\\
85.63	0\\
85.64	0\\
85.65	0\\
85.66	0\\
85.67	0\\
85.68	0\\
85.69	0\\
85.7	0\\
85.71	0\\
85.72	0\\
85.73	0\\
85.74	0\\
85.75	0\\
85.76	0\\
85.77	0\\
85.78	0\\
85.79	0\\
85.8	0\\
85.81	0\\
85.82	0\\
85.83	0\\
85.84	0\\
85.85	0\\
85.86	0\\
85.87	0\\
85.88	0\\
85.89	0\\
85.9	0\\
85.91	0\\
85.92	0\\
85.93	0\\
85.94	0\\
85.95	0\\
85.96	0\\
85.97	0\\
85.98	0\\
85.99	0\\
86	0\\
86.01	0\\
86.02	0\\
86.03	0\\
86.04	0\\
86.05	0\\
86.06	0\\
86.07	0\\
86.08	0\\
86.09	0\\
86.1	0\\
86.11	0\\
86.12	0\\
86.13	0\\
86.14	0\\
86.15	0\\
86.16	0\\
86.17	0\\
86.18	0\\
86.19	0\\
86.2	0\\
86.21	0\\
86.22	0\\
86.23	0\\
86.24	0\\
86.25	0\\
86.26	0\\
86.27	0\\
86.28	0\\
86.29	0\\
86.3	0\\
86.31	0\\
86.32	0\\
86.33	0\\
86.34	0\\
86.35	0\\
86.36	0\\
86.37	0\\
86.38	0\\
86.39	0\\
86.4	0\\
86.41	0\\
86.42	0\\
86.43	0\\
86.44	0\\
86.45	0\\
86.46	0\\
86.47	0\\
86.48	0\\
86.49	0\\
86.5	0\\
86.51	0\\
86.52	0\\
86.53	0\\
86.54	0\\
86.55	0\\
86.56	0\\
86.57	0\\
86.58	0\\
86.59	0\\
86.6	0\\
86.61	0\\
86.62	0\\
86.63	0\\
86.64	0\\
86.65	0\\
86.66	0\\
86.67	0\\
86.68	0\\
86.69	0\\
86.7	0\\
86.71	0\\
86.72	0\\
86.73	0\\
86.74	0\\
86.75	0\\
86.76	0\\
86.77	0\\
86.78	0\\
86.79	0\\
86.8	0\\
86.81	0\\
86.82	0\\
86.83	0\\
86.84	0\\
86.85	0\\
86.86	0\\
86.87	0\\
86.88	0\\
86.89	0\\
86.9	0\\
86.91	0\\
86.92	0\\
86.93	0\\
86.94	0\\
86.95	0\\
86.96	0\\
86.97	0\\
86.98	0\\
86.99	0\\
87	0\\
87.01	0\\
87.02	0\\
87.03	0\\
87.04	0\\
87.05	0\\
87.06	0\\
87.07	0\\
87.08	0\\
87.09	0\\
87.1	0\\
87.11	0\\
87.12	0\\
87.13	0\\
87.14	0\\
87.15	0\\
87.16	0\\
87.17	0\\
87.18	0\\
87.19	0\\
87.2	0\\
87.21	0\\
87.22	0\\
87.23	0\\
87.24	0\\
87.25	0\\
87.26	0\\
87.27	0\\
87.28	0\\
87.29	0\\
87.3	0\\
87.31	0\\
87.32	0\\
87.33	0\\
87.34	0\\
87.35	0\\
87.36	0\\
87.37	0\\
87.38	0\\
87.39	0\\
87.4	0\\
87.41	0\\
87.42	0\\
87.43	0\\
87.44	0\\
87.45	0\\
87.46	0\\
87.47	0\\
87.48	0\\
87.49	0\\
87.5	0\\
87.51	0\\
87.52	0\\
87.53	0\\
87.54	0\\
87.55	0\\
87.56	0\\
87.57	0\\
87.58	0\\
87.59	0\\
87.6	0\\
87.61	0\\
87.62	0\\
87.63	0\\
87.64	0\\
87.65	0\\
87.66	0\\
87.67	0\\
87.68	0\\
87.69	0\\
87.7	0\\
87.71	0\\
87.72	0\\
87.73	0\\
87.74	0\\
87.75	0\\
87.76	0\\
87.77	0\\
87.78	0\\
87.79	0\\
87.8	0\\
87.81	0\\
87.82	0\\
87.83	0\\
87.84	0\\
87.85	0\\
87.86	0\\
87.87	0\\
87.88	0\\
87.89	0\\
87.9	0\\
87.91	0\\
87.92	0\\
87.93	0\\
87.94	0\\
87.95	0\\
87.96	0\\
87.97	0\\
87.98	0\\
87.99	0\\
88	0\\
88.01	0\\
88.02	0\\
88.03	0\\
88.04	0\\
88.05	0\\
88.06	0\\
88.07	0\\
88.08	0\\
88.09	0\\
88.1	0\\
88.11	0\\
88.12	0\\
88.13	0\\
88.14	0\\
88.15	0\\
88.16	0\\
88.17	0\\
88.18	0\\
88.19	0\\
88.2	0\\
88.21	0\\
88.22	0\\
88.23	0\\
88.24	0\\
88.25	0\\
88.26	0\\
88.27	0\\
88.28	0\\
88.29	0\\
88.3	0\\
88.31	0\\
88.32	0\\
88.33	0\\
88.34	0\\
88.35	0\\
88.36	0\\
88.37	0\\
88.38	0\\
88.39	0\\
88.4	0\\
88.41	0\\
88.42	0\\
88.43	0\\
88.44	0\\
88.45	0\\
88.46	0\\
88.47	0\\
88.48	0\\
88.49	0\\
88.5	0\\
88.51	0\\
88.52	0\\
88.53	0\\
88.54	0\\
88.55	0\\
88.56	0\\
88.57	0\\
88.58	0\\
88.59	0\\
88.6	0\\
88.61	0\\
88.62	0\\
88.63	0\\
88.64	0\\
88.65	0\\
88.66	0\\
88.67	0\\
88.68	0\\
88.69	0\\
88.7	0\\
88.71	0\\
88.72	0\\
88.73	0\\
88.74	0\\
88.75	0\\
88.76	0\\
88.77	0\\
88.78	0\\
88.79	0\\
88.8	0\\
88.81	0\\
88.82	0\\
88.83	0\\
88.84	0\\
88.85	0\\
88.86	0\\
88.87	0\\
88.88	0\\
88.89	0\\
88.9	0\\
88.91	0\\
88.92	0\\
88.93	0\\
88.94	0\\
88.95	0\\
88.96	0\\
88.97	0\\
88.98	0\\
88.99	0\\
89	0\\
89.01	0\\
89.02	0\\
89.03	0\\
89.04	0\\
89.05	0\\
89.06	0\\
89.07	0\\
89.08	0\\
89.09	0\\
89.1	0\\
89.11	0\\
89.12	0\\
89.13	0\\
89.14	0\\
89.15	0\\
89.16	0\\
89.17	0\\
89.18	0\\
89.19	0\\
89.2	0\\
89.21	0\\
89.22	0\\
89.23	0\\
89.24	0\\
89.25	0\\
89.26	0\\
89.27	0\\
89.28	0\\
89.29	0\\
89.3	0\\
89.31	0\\
89.32	0\\
89.33	0\\
89.34	0\\
89.35	0\\
89.36	0\\
89.37	0\\
89.38	0\\
89.39	0\\
89.4	0\\
89.41	0\\
89.42	0\\
89.43	0\\
89.44	0\\
89.45	0\\
89.46	0\\
89.47	0\\
89.48	0\\
89.49	0\\
89.5	0\\
89.51	0\\
89.52	0\\
89.53	0\\
89.54	0\\
89.55	0\\
89.56	0\\
89.57	0\\
89.58	0\\
89.59	0\\
89.6	0\\
89.61	0\\
89.62	0\\
89.63	0\\
89.64	0\\
89.65	0\\
89.66	0\\
89.67	0\\
89.68	0\\
89.69	0\\
89.7	0\\
89.71	0\\
89.72	0\\
89.73	0\\
89.74	0\\
89.75	0\\
89.76	0\\
89.77	0\\
89.78	0\\
89.79	0\\
89.8	0\\
89.81	0\\
89.82	0\\
89.83	0\\
89.84	0\\
89.85	0\\
89.86	0\\
89.87	0\\
89.88	0\\
89.89	0\\
89.9	0\\
89.91	0\\
89.92	0\\
89.93	0\\
89.94	0\\
89.95	0\\
89.96	0\\
89.97	0\\
89.98	0\\
89.99	0\\
90	0\\
90.01	0\\
90.02	0\\
90.03	0\\
90.04	0\\
90.05	0\\
90.06	0\\
90.07	0\\
90.08	0\\
90.09	0\\
90.1	0\\
90.11	0\\
90.12	0\\
90.13	0\\
90.14	0\\
90.15	0\\
90.16	0\\
90.17	0\\
90.18	0\\
90.19	0\\
90.2	0\\
90.21	0\\
90.22	0\\
90.23	0\\
90.24	0\\
90.25	0\\
90.26	0\\
90.27	0\\
90.28	0\\
90.29	0\\
90.3	0\\
90.31	0\\
90.32	0\\
90.33	0\\
90.34	0\\
90.35	0\\
90.36	0\\
90.37	0\\
90.38	0\\
90.39	0\\
90.4	0\\
90.41	0\\
90.42	0\\
90.43	0\\
90.44	0\\
90.45	0\\
90.46	0\\
90.47	0\\
90.48	0\\
90.49	0\\
90.5	0\\
90.51	0\\
90.52	0\\
90.53	0\\
90.54	0\\
90.55	0\\
90.56	0\\
90.57	0\\
90.58	0\\
90.59	0\\
90.6	0\\
90.61	0\\
90.62	0\\
90.63	0\\
90.64	0\\
90.65	0\\
90.66	0\\
90.67	0\\
90.68	0\\
90.69	0\\
90.7	0\\
90.71	0\\
90.72	0\\
90.73	0\\
90.74	0\\
90.75	0\\
90.76	0\\
90.77	0\\
90.78	0\\
90.79	0\\
90.8	0\\
90.81	0\\
90.82	0\\
90.83	0\\
90.84	0\\
90.85	0\\
90.86	0\\
90.87	0\\
90.88	0\\
90.89	0\\
90.9	0\\
90.91	0\\
90.92	0\\
90.93	0\\
90.94	0\\
90.95	0\\
90.96	0\\
90.97	0\\
90.98	0\\
90.99	0\\
91	0\\
91.01	0\\
91.02	0\\
91.03	0\\
91.04	0\\
91.05	0\\
91.06	0\\
91.07	0\\
91.08	0\\
91.09	0\\
91.1	0\\
91.11	0\\
91.12	0\\
91.13	0\\
91.14	0\\
91.15	0\\
91.16	0\\
91.17	0\\
91.18	0\\
91.19	0\\
91.2	0\\
91.21	0\\
91.22	0\\
91.23	0\\
91.24	0\\
91.25	0\\
91.26	0\\
91.27	0\\
91.28	0\\
91.29	0\\
91.3	0\\
91.31	0\\
91.32	0\\
91.33	0\\
91.34	0\\
91.35	0\\
91.36	0\\
91.37	0\\
91.38	0\\
91.39	0\\
91.4	0\\
91.41	0\\
91.42	0\\
91.43	0\\
91.44	0\\
91.45	0\\
91.46	0\\
91.47	0\\
91.48	0\\
91.49	0\\
91.5	0\\
91.51	0\\
91.52	0\\
91.53	0\\
91.54	0\\
91.55	0\\
91.56	0\\
91.57	0\\
91.58	0\\
91.59	0\\
91.6	0\\
91.61	0\\
91.62	0\\
91.63	0\\
91.64	0\\
91.65	0\\
91.66	0\\
91.67	0\\
91.68	0\\
91.69	0\\
91.7	0\\
91.71	0\\
91.72	0\\
91.73	0\\
91.74	0\\
91.75	0\\
91.76	0\\
91.77	0\\
91.78	0\\
91.79	0\\
91.8	0\\
91.81	0\\
91.82	0\\
91.83	0\\
91.84	0\\
91.85	0\\
91.86	0\\
91.87	0\\
91.88	0\\
91.89	0\\
91.9	0\\
91.91	0\\
91.92	0\\
91.93	0\\
91.94	0\\
91.95	0\\
91.96	0\\
91.97	0\\
91.98	0\\
91.99	0\\
92	0\\
92.01	0\\
92.02	0\\
92.03	0\\
92.04	0\\
92.05	0\\
92.06	0\\
92.07	0\\
92.08	0\\
92.09	0\\
92.1	0\\
92.11	0\\
92.12	0\\
92.13	0\\
92.14	0\\
92.15	0\\
92.16	0\\
92.17	0\\
92.18	0\\
92.19	0\\
92.2	0\\
92.21	0\\
92.22	0\\
92.23	0\\
92.24	0\\
92.25	0\\
92.26	0\\
92.27	0\\
92.28	0\\
92.29	0\\
92.3	0\\
92.31	0\\
92.32	0\\
92.33	0\\
92.34	0\\
92.35	0\\
92.36	0\\
92.37	0\\
92.38	0\\
92.39	0\\
92.4	0\\
92.41	0\\
92.42	0\\
92.43	0\\
92.44	0\\
92.45	0\\
92.46	0\\
92.47	0\\
92.48	0\\
92.49	0\\
92.5	0\\
92.51	0\\
92.52	0\\
92.53	0\\
92.54	0\\
92.55	0\\
92.56	0\\
92.57	0\\
92.58	0\\
92.59	0\\
92.6	0\\
92.61	0\\
92.62	0\\
92.63	0\\
92.64	0\\
92.65	0\\
92.66	0\\
92.67	0\\
92.68	0\\
92.69	0\\
92.7	0\\
92.71	0\\
92.72	0\\
92.73	0\\
92.74	0\\
92.75	0\\
92.76	0\\
92.77	0\\
92.78	0\\
92.79	0\\
92.8	0\\
92.81	0\\
92.82	0\\
92.83	0\\
92.84	0\\
92.85	0\\
92.86	0\\
92.87	0\\
92.88	0\\
92.89	0\\
92.9	0\\
92.91	0\\
92.92	0\\
92.93	0\\
92.94	0\\
92.95	0\\
92.96	0\\
92.97	0\\
92.98	0\\
92.99	0\\
93	0\\
93.01	0\\
93.02	0\\
93.03	0\\
93.04	0\\
93.05	0\\
93.06	0\\
93.07	0\\
93.08	0\\
93.09	0\\
93.1	0\\
93.11	0\\
93.12	0\\
93.13	0\\
93.14	0\\
93.15	0\\
93.16	0\\
93.17	0\\
93.18	0\\
93.19	0\\
93.2	0\\
93.21	0\\
93.22	0\\
93.23	0\\
93.24	0\\
93.25	0\\
93.26	0\\
93.27	0\\
93.28	0\\
93.29	0\\
93.3	0\\
93.31	0\\
93.32	0\\
93.33	0\\
93.34	0\\
93.35	0\\
93.36	0\\
93.37	0\\
93.38	0\\
93.39	0\\
93.4	0\\
93.41	0\\
93.42	0\\
93.43	0\\
93.44	0\\
93.45	0\\
93.46	0\\
93.47	0\\
93.48	0\\
93.49	0\\
93.5	0\\
93.51	0\\
93.52	0\\
93.53	0\\
93.54	0\\
93.55	0\\
93.56	0\\
93.57	0\\
93.58	0\\
93.59	0\\
93.6	0\\
93.61	0\\
93.62	0\\
93.63	0\\
93.64	0\\
93.65	0\\
93.66	0\\
93.67	0\\
93.68	0\\
93.69	0\\
93.7	0\\
93.71	0\\
93.72	0\\
93.73	0\\
93.74	0\\
93.75	0\\
93.76	0\\
93.77	0\\
93.78	0\\
93.79	0\\
93.8	0\\
93.81	0\\
93.82	0\\
93.83	0\\
93.84	0\\
93.85	0\\
93.86	0\\
93.87	0\\
93.88	0\\
93.89	0\\
93.9	0\\
93.91	0\\
93.92	0\\
93.93	0\\
93.94	0\\
93.95	0\\
93.96	0\\
93.97	0\\
93.98	0\\
93.99	0\\
94	0\\
94.01	0\\
94.02	0\\
94.03	0\\
94.04	0\\
94.05	0\\
94.06	0\\
94.07	0\\
94.08	0\\
94.09	0\\
94.1	0\\
94.11	0\\
94.12	0\\
94.13	0\\
94.14	0\\
94.15	0\\
94.16	0\\
94.17	0\\
94.18	0\\
94.19	0\\
94.2	0\\
94.21	0\\
94.22	0\\
94.23	0\\
94.24	0\\
94.25	0\\
94.26	0\\
94.27	0\\
94.28	0\\
94.29	0\\
94.3	0\\
94.31	0\\
94.32	0\\
94.33	0\\
94.34	0\\
94.35	0\\
94.36	0\\
94.37	0\\
94.38	0\\
94.39	0\\
94.4	0\\
94.41	0\\
94.42	0\\
94.43	0\\
94.44	0\\
94.45	0\\
94.46	0\\
94.47	0\\
94.48	0\\
94.49	0\\
94.5	0\\
94.51	0\\
94.52	0\\
94.53	0\\
94.54	0\\
94.55	0\\
94.56	0\\
94.57	0\\
94.58	0\\
94.59	0\\
94.6	0\\
94.61	0\\
94.62	0\\
94.63	0\\
94.64	0\\
94.65	0\\
94.66	0\\
94.67	0\\
94.68	0\\
94.69	0\\
94.7	0\\
94.71	0\\
94.72	0\\
94.73	0\\
94.74	0\\
94.75	0\\
94.76	0\\
94.77	0\\
94.78	0\\
94.79	0\\
94.8	0\\
94.81	0\\
94.82	0\\
94.83	0\\
94.84	0\\
94.85	0\\
94.86	0\\
94.87	0\\
94.88	0\\
94.89	0\\
94.9	0\\
94.91	0\\
94.92	0\\
94.93	0\\
94.94	0\\
94.95	0\\
94.96	0\\
94.97	0\\
94.98	0\\
94.99	0\\
95	0\\
95.01	0\\
95.02	0\\
95.03	0\\
95.04	0\\
95.05	0\\
95.06	0\\
95.07	0\\
95.08	0\\
95.09	0\\
95.1	0\\
95.11	0\\
95.12	0\\
95.13	0\\
95.14	0\\
95.15	0\\
95.16	0\\
95.17	0\\
95.18	0\\
95.19	0\\
95.2	0\\
95.21	0\\
95.22	0\\
95.23	0\\
95.24	0\\
95.25	0\\
95.26	0\\
95.27	0\\
95.28	0\\
95.29	0\\
95.3	0\\
95.31	0\\
95.32	0\\
95.33	0\\
95.34	0\\
95.35	0\\
95.36	0\\
95.37	0\\
95.38	0\\
95.39	0\\
95.4	0\\
95.41	0\\
95.42	0\\
95.43	0\\
95.44	0\\
95.45	0\\
95.46	0\\
95.47	0\\
95.48	0\\
95.49	0\\
95.5	0\\
95.51	0\\
95.52	0\\
95.53	0\\
95.54	0\\
95.55	0\\
95.56	0\\
95.57	0\\
95.58	0\\
95.59	0\\
95.6	0\\
95.61	0\\
95.62	0\\
95.63	0\\
95.64	0\\
95.65	0\\
95.66	0\\
95.67	0\\
95.68	0\\
95.69	0\\
95.7	0\\
95.71	0\\
95.72	0\\
95.73	0\\
95.74	0\\
95.75	0\\
95.76	0\\
95.77	0\\
95.78	0\\
95.79	0\\
95.8	0\\
95.81	0\\
95.82	0\\
95.83	0\\
95.84	0\\
95.85	0\\
95.86	0\\
95.87	0\\
95.88	0\\
95.89	0\\
95.9	0\\
95.91	0\\
95.92	0\\
95.93	0\\
95.94	0\\
95.95	0\\
95.96	0\\
95.97	0\\
95.98	0\\
95.99	0\\
96	0\\
96.01	0\\
96.02	0\\
96.03	0\\
96.04	0\\
96.05	0\\
96.06	0\\
96.07	0\\
96.08	0\\
96.09	0\\
96.1	0\\
96.11	0\\
96.12	0\\
96.13	0\\
96.14	0\\
96.15	0\\
96.16	0\\
96.17	0\\
96.18	0\\
96.19	0\\
96.2	0\\
96.21	0\\
96.22	0\\
96.23	0\\
96.24	0\\
96.25	0\\
96.26	0\\
96.27	0\\
96.28	0\\
96.29	0\\
96.3	0\\
96.31	0\\
96.32	0\\
96.33	0\\
96.34	0\\
96.35	0\\
96.36	0\\
96.37	0\\
96.38	0\\
96.39	0\\
96.4	0\\
96.41	0\\
96.42	0\\
96.43	0\\
96.44	0\\
96.45	0\\
96.46	0\\
96.47	0\\
96.48	0\\
96.49	0\\
96.5	0\\
96.51	0\\
96.52	0\\
96.53	0\\
96.54	0\\
96.55	0\\
96.56	0\\
96.57	0\\
96.58	0\\
96.59	0\\
96.6	0\\
96.61	0\\
96.62	0\\
96.63	0\\
96.64	0\\
96.65	0\\
96.66	0\\
96.67	0\\
96.68	0\\
96.69	0\\
96.7	0\\
96.71	0\\
96.72	0\\
96.73	0\\
96.74	0\\
96.75	0\\
96.76	0\\
96.77	0\\
96.78	0\\
96.79	0\\
96.8	0\\
96.81	0\\
96.82	0\\
96.83	0\\
96.84	0\\
96.85	0\\
96.86	0\\
96.87	0\\
96.88	0\\
96.89	0\\
96.9	0\\
96.91	0\\
96.92	0\\
96.93	0\\
96.94	0\\
96.95	0\\
96.96	0\\
96.97	0\\
96.98	0\\
96.99	0\\
97	0\\
97.01	0\\
97.02	0\\
97.03	0\\
97.04	0\\
97.05	0\\
97.06	0\\
97.07	0\\
97.08	0\\
97.09	0\\
97.1	0\\
97.11	0\\
97.12	0\\
97.13	0\\
97.14	0\\
97.15	0\\
97.16	0\\
97.17	0\\
97.18	0\\
97.19	0\\
97.2	0\\
97.21	0\\
97.22	0\\
97.23	0\\
97.24	0\\
97.25	0\\
97.26	0\\
97.27	0\\
97.28	0\\
97.29	0\\
97.3	0\\
97.31	0\\
97.32	0\\
97.33	0\\
97.34	0\\
97.35	0\\
97.36	0\\
97.37	0\\
97.38	0\\
97.39	0\\
97.4	0\\
97.41	0\\
97.42	0\\
97.43	0\\
97.44	0\\
97.45	0\\
97.46	0\\
97.47	0\\
97.48	0\\
97.49	0\\
97.5	0\\
97.51	0\\
97.52	0\\
97.53	0\\
97.54	0\\
97.55	0\\
97.56	0\\
97.57	0\\
97.58	0\\
97.59	0\\
97.6	0\\
97.61	0\\
97.62	0\\
97.63	0\\
97.64	0\\
97.65	0\\
97.66	0\\
97.67	0\\
97.68	0\\
97.69	0\\
97.7	0\\
97.71	0\\
97.72	0\\
97.73	0\\
97.74	0\\
97.75	0\\
97.76	0\\
97.77	0\\
97.78	0\\
97.79	0\\
97.8	0\\
97.81	0\\
97.82	0\\
97.83	0\\
97.84	0\\
97.85	0\\
97.86	0\\
97.87	0\\
97.88	0\\
97.89	0\\
97.9	0\\
97.91	0\\
97.92	0\\
97.93	0\\
97.94	0\\
97.95	0\\
97.96	0\\
97.97	0\\
97.98	0\\
97.99	0\\
98	0\\
98.01	0\\
98.02	0\\
98.03	0\\
98.04	0\\
98.05	0\\
98.06	0\\
98.07	0\\
98.08	0\\
98.09	0\\
98.1	0\\
98.11	0\\
98.12	0\\
98.13	0\\
98.14	0\\
98.15	0\\
98.16	0\\
98.17	0\\
98.18	0\\
98.19	0\\
98.2	0\\
98.21	0\\
98.22	0\\
98.23	0\\
98.24	0\\
98.25	0\\
98.26	0\\
98.27	0\\
98.28	0\\
98.29	0\\
98.3	0\\
98.31	0\\
98.32	0\\
98.33	5.50983918425306e-05\\
98.34	0.00011416186829323\\
98.35	0.000173646399913577\\
98.36	0.000233555925247167\\
98.37	0.000293894423663059\\
98.38	0.00035466591584736\\
98.39	0.000415874464302052\\
98.4	0.00047752417518835\\
98.41	0.000539618631111172\\
98.42	0.000602157410859082\\
98.43	0.000665144661918159\\
98.44	0.000728584574881119\\
98.45	0.000792481383973774\\
98.46	0.000856839367589467\\
98.47	0.000921662848831623\\
98.48	0.000986956196064568\\
98.49	0.00105272382347279\\
98.5	0.00111897019162882\\
98.51	0.00118569980806984\\
98.52	0.00125291722788331\\
98.53	0.00132062705430171\\
98.54	0.00138883393930652\\
98.55	0.00145754258424184\\
98.56	0.00152675774085534\\
98.57	0.00159648421469233\\
98.58	0.00166672686266103\\
98.59	0.00173749059368834\\
98.6	0.00180878036938612\\
98.61	0.00188060120472835\\
98.62	0.00195295816873927\\
98.63	0.0020209086568889\\
98.64	0.00205065084748465\\
98.65	0.00208064183415551\\
98.66	0.00211088389512269\\
98.67	0.00214137932892571\\
98.68	0.00217213045459118\\
98.69	0.00220313961180256\\
98.7	0.00223440908654092\\
98.71	0.0022659411674283\\
98.72	0.00229773816245183\\
98.73	0.00232980239911189\\
98.74	0.00236214359092752\\
98.75	0.00239477221723108\\
98.76	0.00242769093464588\\
98.77	0.0024609024240391\\
98.78	0.00249440939015167\\
98.79	0.00252821474185148\\
98.8	0.0025623252373108\\
98.81	0.00259674372866936\\
98.82	0.00263147309453995\\
98.83	0.00266651624025044\\
98.84	0.00270187609808785\\
98.85	0.00273755562754466\\
98.86	0.00277355781556714\\
98.87	0.00280988567680589\\
98.88	0.00284654225386852\\
98.89	0.0028835306175745\\
98.9	0.00292085386721226\\
98.91	0.0029585151307984\\
98.92	0.00299651756279268\\
98.93	0.00303486434652039\\
98.94	0.00307355869473684\\
98.95	0.0031126038498935\\
98.96	0.00315200308440636\\
98.97	0.00319175970092647\\
98.98	0.00323187703261279\\
98.99	0.0032723584434073\\
99	0.00331320732831235\\
99.01	0.00335442711367033\\
99.02	0.00339602125744558\\
99.03	0.00343799324950865\\
99.04	0.0034803466119229\\
99.05	0.00352308489923335\\
99.06	0.0035662116987579\\
99.07	0.00360973063088092\\
99.08	0.00365364534934916\\
99.09	0.00369795954156998\\
99.1	0.00374267692891206\\
99.11	0.00378780126700835\\
99.12	0.00383333634606152\\
99.13	0.00387928599115167\\
99.14	0.00392565406254658\\
99.15	0.0039724444560142\\
99.16	0.00401966110313765\\
99.17	0.0040673079716326\\
99.18	0.00411538906566701\\
99.19	0.00416390842624637\\
99.2	0.0042128701315636\\
99.21	0.00426227829732716\\
99.22	0.00431213707709156\\
99.23	0.00436245066259037\\
99.24	0.00441322328407146\\
99.25	0.00446445921063472\\
99.26	0.00451616275057213\\
99.27	0.00456833825171019\\
99.28	0.00462099010175469\\
99.29	0.00467412272863787\\
99.3	0.00472774060086789\\
99.31	0.00478184822788059\\
99.32	0.00483645016039364\\
99.33	0.00489155099076289\\
99.34	0.00494715535334105\\
99.35	0.00500326792483865\\
99.36	0.00505989342468713\\
99.37	0.00511703661540424\\
99.38	0.00517470230296157\\
99.39	0.00523289533568309\\
99.4	0.00529162054313569\\
99.41	0.00535088279818377\\
99.42	0.00541068701734764\\
99.43	0.0054710381611638\\
99.44	0.00553194123454689\\
99.45	0.00559340128715346\\
99.46	0.00565542341374731\\
99.47	0.00571801275456649\\
99.48	0.00578117449569182\\
99.49	0.00584491386941689\\
99.5	0.00590923615461952\\
99.51	0.00597414667713447\\
99.52	0.00603965081012756\\
99.53	0.00610575397447076\\
99.54	0.00617246163911862\\
99.55	0.00623977932148552\\
99.56	0.00630771258782391\\
99.57	0.00637626705360334\\
99.58	0.00644544838389017\\
99.59	0.00651526229372785\\
99.6	0.00658571454851766\\
99.61	0.00665681096439979\\
99.62	0.00672855740863903\\
99.63	0.00680095980001549\\
99.64	0.00687402410920575\\
99.65	0.00694775635916377\\
99.66	0.00702216262550119\\
99.67	0.00709724903686717\\
99.68	0.00717302177532716\\
99.69	0.0072494870767408\\
99.7	0.00732665123113838\\
99.71	0.00740452058309589\\
99.72	0.00748310153210824\\
99.73	0.00756240052588584\\
99.74	0.00764242406140549\\
99.75	0.00772317869143584\\
99.76	0.00780467102489812\\
99.77	0.00788690772722355\\
99.78	0.00796989552070725\\
99.79	0.00805364118485809\\
99.8	0.00813815155674424\\
99.81	0.00822343353133399\\
99.82	0.00830949406183134\\
99.83	0.00839634016000594\\
99.84	0.00848397889651701\\
99.85	0.00857241740123047\\
99.86	0.00866166286352905\\
99.87	0.00875172253261461\\
99.88	0.00884260371780208\\
99.89	0.00893431378880451\\
99.9	0.0090268601760084\\
99.91	0.00912025037073873\\
99.92	0.00921449192551288\\
99.93	0.00930959245428256\\
99.94	0.00940555963266315\\
99.95	0.00950240119814921\\
99.96	0.00960012495031559\\
99.97	0.00969873875100284\\
99.98	0.00979825052448599\\
99.99	0.00989866825762563\\
100	0.01\\
};
\addlegendentry{$q=-1$};

\addplot [color=black,solid,forget plot]
  table[row sep=crcr]{%
0.01	0\\
0.02	0\\
0.03	0\\
0.04	0\\
0.05	0\\
0.06	0\\
0.07	0\\
0.08	0\\
0.09	0\\
0.1	0\\
0.11	0\\
0.12	0\\
0.13	0\\
0.14	0\\
0.15	0\\
0.16	0\\
0.17	0\\
0.18	0\\
0.19	0\\
0.2	0\\
0.21	0\\
0.22	0\\
0.23	0\\
0.24	0\\
0.25	0\\
0.26	0\\
0.27	0\\
0.28	0\\
0.29	0\\
0.3	0\\
0.31	0\\
0.32	0\\
0.33	0\\
0.34	0\\
0.35	0\\
0.36	0\\
0.37	0\\
0.38	0\\
0.39	0\\
0.4	0\\
0.41	0\\
0.42	0\\
0.43	0\\
0.44	0\\
0.45	0\\
0.46	0\\
0.47	0\\
0.48	0\\
0.49	0\\
0.5	0\\
0.51	0\\
0.52	0\\
0.53	0\\
0.54	0\\
0.55	0\\
0.56	0\\
0.57	0\\
0.58	0\\
0.59	0\\
0.6	0\\
0.61	0\\
0.62	0\\
0.63	0\\
0.64	0\\
0.65	0\\
0.66	0\\
0.67	0\\
0.68	0\\
0.69	0\\
0.7	0\\
0.71	0\\
0.72	0\\
0.73	0\\
0.74	0\\
0.75	0\\
0.76	0\\
0.77	0\\
0.78	0\\
0.79	0\\
0.8	0\\
0.81	0\\
0.82	0\\
0.83	0\\
0.84	0\\
0.85	0\\
0.86	0\\
0.87	0\\
0.88	0\\
0.89	0\\
0.9	0\\
0.91	0\\
0.92	0\\
0.93	0\\
0.94	0\\
0.95	0\\
0.96	0\\
0.97	0\\
0.98	0\\
0.99	0\\
1	0\\
1.01	0\\
1.02	0\\
1.03	0\\
1.04	0\\
1.05	0\\
1.06	0\\
1.07	0\\
1.08	0\\
1.09	0\\
1.1	0\\
1.11	0\\
1.12	0\\
1.13	0\\
1.14	0\\
1.15	0\\
1.16	0\\
1.17	0\\
1.18	0\\
1.19	0\\
1.2	0\\
1.21	0\\
1.22	0\\
1.23	0\\
1.24	0\\
1.25	0\\
1.26	0\\
1.27	0\\
1.28	0\\
1.29	0\\
1.3	0\\
1.31	0\\
1.32	0\\
1.33	0\\
1.34	0\\
1.35	0\\
1.36	0\\
1.37	0\\
1.38	0\\
1.39	0\\
1.4	0\\
1.41	0\\
1.42	0\\
1.43	0\\
1.44	0\\
1.45	0\\
1.46	0\\
1.47	0\\
1.48	0\\
1.49	0\\
1.5	0\\
1.51	0\\
1.52	0\\
1.53	0\\
1.54	0\\
1.55	0\\
1.56	0\\
1.57	0\\
1.58	0\\
1.59	0\\
1.6	0\\
1.61	0\\
1.62	0\\
1.63	0\\
1.64	0\\
1.65	0\\
1.66	0\\
1.67	0\\
1.68	0\\
1.69	0\\
1.7	0\\
1.71	0\\
1.72	0\\
1.73	0\\
1.74	0\\
1.75	0\\
1.76	0\\
1.77	0\\
1.78	0\\
1.79	0\\
1.8	0\\
1.81	0\\
1.82	0\\
1.83	0\\
1.84	0\\
1.85	0\\
1.86	0\\
1.87	0\\
1.88	0\\
1.89	0\\
1.9	0\\
1.91	0\\
1.92	0\\
1.93	0\\
1.94	0\\
1.95	0\\
1.96	0\\
1.97	0\\
1.98	0\\
1.99	0\\
2	0\\
2.01	0\\
2.02	0\\
2.03	0\\
2.04	0\\
2.05	0\\
2.06	0\\
2.07	0\\
2.08	0\\
2.09	0\\
2.1	0\\
2.11	0\\
2.12	0\\
2.13	0\\
2.14	0\\
2.15	0\\
2.16	0\\
2.17	0\\
2.18	0\\
2.19	0\\
2.2	0\\
2.21	0\\
2.22	0\\
2.23	0\\
2.24	0\\
2.25	0\\
2.26	0\\
2.27	0\\
2.28	0\\
2.29	0\\
2.3	0\\
2.31	0\\
2.32	0\\
2.33	0\\
2.34	0\\
2.35	0\\
2.36	0\\
2.37	0\\
2.38	0\\
2.39	0\\
2.4	0\\
2.41	0\\
2.42	0\\
2.43	0\\
2.44	0\\
2.45	0\\
2.46	0\\
2.47	0\\
2.48	0\\
2.49	0\\
2.5	0\\
2.51	0\\
2.52	0\\
2.53	0\\
2.54	0\\
2.55	0\\
2.56	0\\
2.57	0\\
2.58	0\\
2.59	0\\
2.6	0\\
2.61	0\\
2.62	0\\
2.63	0\\
2.64	0\\
2.65	0\\
2.66	0\\
2.67	0\\
2.68	0\\
2.69	0\\
2.7	0\\
2.71	0\\
2.72	0\\
2.73	0\\
2.74	0\\
2.75	0\\
2.76	0\\
2.77	0\\
2.78	0\\
2.79	0\\
2.8	0\\
2.81	0\\
2.82	0\\
2.83	0\\
2.84	0\\
2.85	0\\
2.86	0\\
2.87	0\\
2.88	0\\
2.89	0\\
2.9	0\\
2.91	0\\
2.92	0\\
2.93	0\\
2.94	0\\
2.95	0\\
2.96	0\\
2.97	0\\
2.98	0\\
2.99	0\\
3	0\\
3.01	0\\
3.02	0\\
3.03	0\\
3.04	0\\
3.05	0\\
3.06	0\\
3.07	0\\
3.08	0\\
3.09	0\\
3.1	0\\
3.11	0\\
3.12	0\\
3.13	0\\
3.14	0\\
3.15	0\\
3.16	0\\
3.17	0\\
3.18	0\\
3.19	0\\
3.2	0\\
3.21	0\\
3.22	0\\
3.23	0\\
3.24	0\\
3.25	0\\
3.26	0\\
3.27	0\\
3.28	0\\
3.29	0\\
3.3	0\\
3.31	0\\
3.32	0\\
3.33	0\\
3.34	0\\
3.35	0\\
3.36	0\\
3.37	0\\
3.38	0\\
3.39	0\\
3.4	0\\
3.41	0\\
3.42	0\\
3.43	0\\
3.44	0\\
3.45	0\\
3.46	0\\
3.47	0\\
3.48	0\\
3.49	0\\
3.5	0\\
3.51	0\\
3.52	0\\
3.53	0\\
3.54	0\\
3.55	0\\
3.56	0\\
3.57	0\\
3.58	0\\
3.59	0\\
3.6	0\\
3.61	0\\
3.62	0\\
3.63	0\\
3.64	0\\
3.65	0\\
3.66	0\\
3.67	0\\
3.68	0\\
3.69	0\\
3.7	0\\
3.71	0\\
3.72	0\\
3.73	0\\
3.74	0\\
3.75	0\\
3.76	0\\
3.77	0\\
3.78	0\\
3.79	0\\
3.8	0\\
3.81	0\\
3.82	0\\
3.83	0\\
3.84	0\\
3.85	0\\
3.86	0\\
3.87	0\\
3.88	0\\
3.89	0\\
3.9	0\\
3.91	0\\
3.92	0\\
3.93	0\\
3.94	0\\
3.95	0\\
3.96	0\\
3.97	0\\
3.98	0\\
3.99	0\\
4	0\\
4.01	0\\
4.02	0\\
4.03	0\\
4.04	0\\
4.05	0\\
4.06	0\\
4.07	0\\
4.08	0\\
4.09	0\\
4.1	0\\
4.11	0\\
4.12	0\\
4.13	0\\
4.14	0\\
4.15	0\\
4.16	0\\
4.17	0\\
4.18	0\\
4.19	0\\
4.2	0\\
4.21	0\\
4.22	0\\
4.23	0\\
4.24	0\\
4.25	0\\
4.26	0\\
4.27	0\\
4.28	0\\
4.29	0\\
4.3	0\\
4.31	0\\
4.32	0\\
4.33	0\\
4.34	0\\
4.35	0\\
4.36	0\\
4.37	0\\
4.38	0\\
4.39	0\\
4.4	0\\
4.41	0\\
4.42	0\\
4.43	0\\
4.44	0\\
4.45	0\\
4.46	0\\
4.47	0\\
4.48	0\\
4.49	0\\
4.5	0\\
4.51	0\\
4.52	0\\
4.53	0\\
4.54	0\\
4.55	0\\
4.56	0\\
4.57	0\\
4.58	0\\
4.59	0\\
4.6	0\\
4.61	0\\
4.62	0\\
4.63	0\\
4.64	0\\
4.65	0\\
4.66	0\\
4.67	0\\
4.68	0\\
4.69	0\\
4.7	0\\
4.71	0\\
4.72	0\\
4.73	0\\
4.74	0\\
4.75	0\\
4.76	0\\
4.77	0\\
4.78	0\\
4.79	0\\
4.8	0\\
4.81	0\\
4.82	0\\
4.83	0\\
4.84	0\\
4.85	0\\
4.86	0\\
4.87	0\\
4.88	0\\
4.89	0\\
4.9	0\\
4.91	0\\
4.92	0\\
4.93	0\\
4.94	0\\
4.95	0\\
4.96	0\\
4.97	0\\
4.98	0\\
4.99	0\\
5	0\\
5.01	0\\
5.02	0\\
5.03	0\\
5.04	0\\
5.05	0\\
5.06	0\\
5.07	0\\
5.08	0\\
5.09	0\\
5.1	0\\
5.11	0\\
5.12	0\\
5.13	0\\
5.14	0\\
5.15	0\\
5.16	0\\
5.17	0\\
5.18	0\\
5.19	0\\
5.2	0\\
5.21	0\\
5.22	0\\
5.23	0\\
5.24	0\\
5.25	0\\
5.26	0\\
5.27	0\\
5.28	0\\
5.29	0\\
5.3	0\\
5.31	0\\
5.32	0\\
5.33	0\\
5.34	0\\
5.35	0\\
5.36	0\\
5.37	0\\
5.38	0\\
5.39	0\\
5.4	0\\
5.41	0\\
5.42	0\\
5.43	0\\
5.44	0\\
5.45	0\\
5.46	0\\
5.47	0\\
5.48	0\\
5.49	0\\
5.5	0\\
5.51	0\\
5.52	0\\
5.53	0\\
5.54	0\\
5.55	0\\
5.56	0\\
5.57	0\\
5.58	0\\
5.59	0\\
5.6	0\\
5.61	0\\
5.62	0\\
5.63	0\\
5.64	0\\
5.65	0\\
5.66	0\\
5.67	0\\
5.68	0\\
5.69	0\\
5.7	0\\
5.71	0\\
5.72	0\\
5.73	0\\
5.74	0\\
5.75	0\\
5.76	0\\
5.77	0\\
5.78	0\\
5.79	0\\
5.8	0\\
5.81	0\\
5.82	0\\
5.83	0\\
5.84	0\\
5.85	0\\
5.86	0\\
5.87	0\\
5.88	0\\
5.89	0\\
5.9	0\\
5.91	0\\
5.92	0\\
5.93	0\\
5.94	0\\
5.95	0\\
5.96	0\\
5.97	0\\
5.98	0\\
5.99	0\\
6	0\\
6.01	0\\
6.02	0\\
6.03	0\\
6.04	0\\
6.05	0\\
6.06	0\\
6.07	0\\
6.08	0\\
6.09	0\\
6.1	0\\
6.11	0\\
6.12	0\\
6.13	0\\
6.14	0\\
6.15	0\\
6.16	0\\
6.17	0\\
6.18	0\\
6.19	0\\
6.2	0\\
6.21	0\\
6.22	0\\
6.23	0\\
6.24	0\\
6.25	0\\
6.26	0\\
6.27	0\\
6.28	0\\
6.29	0\\
6.3	0\\
6.31	0\\
6.32	0\\
6.33	0\\
6.34	0\\
6.35	0\\
6.36	0\\
6.37	0\\
6.38	0\\
6.39	0\\
6.4	0\\
6.41	0\\
6.42	0\\
6.43	0\\
6.44	0\\
6.45	0\\
6.46	0\\
6.47	0\\
6.48	0\\
6.49	0\\
6.5	0\\
6.51	0\\
6.52	0\\
6.53	0\\
6.54	0\\
6.55	0\\
6.56	0\\
6.57	0\\
6.58	0\\
6.59	0\\
6.6	0\\
6.61	0\\
6.62	0\\
6.63	0\\
6.64	0\\
6.65	0\\
6.66	0\\
6.67	0\\
6.68	0\\
6.69	0\\
6.7	0\\
6.71	0\\
6.72	0\\
6.73	0\\
6.74	0\\
6.75	0\\
6.76	0\\
6.77	0\\
6.78	0\\
6.79	0\\
6.8	0\\
6.81	0\\
6.82	0\\
6.83	0\\
6.84	0\\
6.85	0\\
6.86	0\\
6.87	0\\
6.88	0\\
6.89	0\\
6.9	0\\
6.91	0\\
6.92	0\\
6.93	0\\
6.94	0\\
6.95	0\\
6.96	0\\
6.97	0\\
6.98	0\\
6.99	0\\
7	0\\
7.01	0\\
7.02	0\\
7.03	0\\
7.04	0\\
7.05	0\\
7.06	0\\
7.07	0\\
7.08	0\\
7.09	0\\
7.1	0\\
7.11	0\\
7.12	0\\
7.13	0\\
7.14	0\\
7.15	0\\
7.16	0\\
7.17	0\\
7.18	0\\
7.19	0\\
7.2	0\\
7.21	0\\
7.22	0\\
7.23	0\\
7.24	0\\
7.25	0\\
7.26	0\\
7.27	0\\
7.28	0\\
7.29	0\\
7.3	0\\
7.31	0\\
7.32	0\\
7.33	0\\
7.34	0\\
7.35	0\\
7.36	0\\
7.37	0\\
7.38	0\\
7.39	0\\
7.4	0\\
7.41	0\\
7.42	0\\
7.43	0\\
7.44	0\\
7.45	0\\
7.46	0\\
7.47	0\\
7.48	0\\
7.49	0\\
7.5	0\\
7.51	0\\
7.52	0\\
7.53	0\\
7.54	0\\
7.55	0\\
7.56	0\\
7.57	0\\
7.58	0\\
7.59	0\\
7.6	0\\
7.61	0\\
7.62	0\\
7.63	0\\
7.64	0\\
7.65	0\\
7.66	0\\
7.67	0\\
7.68	0\\
7.69	0\\
7.7	0\\
7.71	0\\
7.72	0\\
7.73	0\\
7.74	0\\
7.75	0\\
7.76	0\\
7.77	0\\
7.78	0\\
7.79	0\\
7.8	0\\
7.81	0\\
7.82	0\\
7.83	0\\
7.84	0\\
7.85	0\\
7.86	0\\
7.87	0\\
7.88	0\\
7.89	0\\
7.9	0\\
7.91	0\\
7.92	0\\
7.93	0\\
7.94	0\\
7.95	0\\
7.96	0\\
7.97	0\\
7.98	0\\
7.99	0\\
8	0\\
8.01	0\\
8.02	0\\
8.03	0\\
8.04	0\\
8.05	0\\
8.06	0\\
8.07	0\\
8.08	0\\
8.09	0\\
8.1	0\\
8.11	0\\
8.12	0\\
8.13	0\\
8.14	0\\
8.15	0\\
8.16	0\\
8.17	0\\
8.18	0\\
8.19	0\\
8.2	0\\
8.21	0\\
8.22	0\\
8.23	0\\
8.24	0\\
8.25	0\\
8.26	0\\
8.27	0\\
8.28	0\\
8.29	0\\
8.3	0\\
8.31	0\\
8.32	0\\
8.33	0\\
8.34	0\\
8.35	0\\
8.36	0\\
8.37	0\\
8.38	0\\
8.39	0\\
8.4	0\\
8.41	0\\
8.42	0\\
8.43	0\\
8.44	0\\
8.45	0\\
8.46	0\\
8.47	0\\
8.48	0\\
8.49	0\\
8.5	0\\
8.51	0\\
8.52	0\\
8.53	0\\
8.54	0\\
8.55	0\\
8.56	0\\
8.57	0\\
8.58	0\\
8.59	0\\
8.6	0\\
8.61	0\\
8.62	0\\
8.63	0\\
8.64	0\\
8.65	0\\
8.66	0\\
8.67	0\\
8.68	0\\
8.69	0\\
8.7	0\\
8.71	0\\
8.72	0\\
8.73	0\\
8.74	0\\
8.75	0\\
8.76	0\\
8.77	0\\
8.78	0\\
8.79	0\\
8.8	0\\
8.81	0\\
8.82	0\\
8.83	0\\
8.84	0\\
8.85	0\\
8.86	0\\
8.87	0\\
8.88	0\\
8.89	0\\
8.9	0\\
8.91	0\\
8.92	0\\
8.93	0\\
8.94	0\\
8.95	0\\
8.96	0\\
8.97	0\\
8.98	0\\
8.99	0\\
9	0\\
9.01	0\\
9.02	0\\
9.03	0\\
9.04	0\\
9.05	0\\
9.06	0\\
9.07	0\\
9.08	0\\
9.09	0\\
9.1	0\\
9.11	0\\
9.12	0\\
9.13	0\\
9.14	0\\
9.15	0\\
9.16	0\\
9.17	0\\
9.18	0\\
9.19	0\\
9.2	0\\
9.21	0\\
9.22	0\\
9.23	0\\
9.24	0\\
9.25	0\\
9.26	0\\
9.27	0\\
9.28	0\\
9.29	0\\
9.3	0\\
9.31	0\\
9.32	0\\
9.33	0\\
9.34	0\\
9.35	0\\
9.36	0\\
9.37	0\\
9.38	0\\
9.39	0\\
9.4	0\\
9.41	0\\
9.42	0\\
9.43	0\\
9.44	0\\
9.45	0\\
9.46	0\\
9.47	0\\
9.48	0\\
9.49	0\\
9.5	0\\
9.51	0\\
9.52	0\\
9.53	0\\
9.54	0\\
9.55	0\\
9.56	0\\
9.57	0\\
9.58	0\\
9.59	0\\
9.6	0\\
9.61	0\\
9.62	0\\
9.63	0\\
9.64	0\\
9.65	0\\
9.66	0\\
9.67	0\\
9.68	0\\
9.69	0\\
9.7	0\\
9.71	0\\
9.72	0\\
9.73	0\\
9.74	0\\
9.75	0\\
9.76	0\\
9.77	0\\
9.78	0\\
9.79	0\\
9.8	0\\
9.81	0\\
9.82	0\\
9.83	0\\
9.84	0\\
9.85	0\\
9.86	0\\
9.87	0\\
9.88	0\\
9.89	0\\
9.9	0\\
9.91	0\\
9.92	0\\
9.93	0\\
9.94	0\\
9.95	0\\
9.96	0\\
9.97	0\\
9.98	0\\
9.99	0\\
10	0\\
10.01	0\\
10.02	0\\
10.03	0\\
10.04	0\\
10.05	0\\
10.06	0\\
10.07	0\\
10.08	0\\
10.09	0\\
10.1	0\\
10.11	0\\
10.12	0\\
10.13	0\\
10.14	0\\
10.15	0\\
10.16	0\\
10.17	0\\
10.18	0\\
10.19	0\\
10.2	0\\
10.21	0\\
10.22	0\\
10.23	0\\
10.24	0\\
10.25	0\\
10.26	0\\
10.27	0\\
10.28	0\\
10.29	0\\
10.3	0\\
10.31	0\\
10.32	0\\
10.33	0\\
10.34	0\\
10.35	0\\
10.36	0\\
10.37	0\\
10.38	0\\
10.39	0\\
10.4	0\\
10.41	0\\
10.42	0\\
10.43	0\\
10.44	0\\
10.45	0\\
10.46	0\\
10.47	0\\
10.48	0\\
10.49	0\\
10.5	0\\
10.51	0\\
10.52	0\\
10.53	0\\
10.54	0\\
10.55	0\\
10.56	0\\
10.57	0\\
10.58	0\\
10.59	0\\
10.6	0\\
10.61	0\\
10.62	0\\
10.63	0\\
10.64	0\\
10.65	0\\
10.66	0\\
10.67	0\\
10.68	0\\
10.69	0\\
10.7	0\\
10.71	0\\
10.72	0\\
10.73	0\\
10.74	0\\
10.75	0\\
10.76	0\\
10.77	0\\
10.78	0\\
10.79	0\\
10.8	0\\
10.81	0\\
10.82	0\\
10.83	0\\
10.84	0\\
10.85	0\\
10.86	0\\
10.87	0\\
10.88	0\\
10.89	0\\
10.9	0\\
10.91	0\\
10.92	0\\
10.93	0\\
10.94	0\\
10.95	0\\
10.96	0\\
10.97	0\\
10.98	0\\
10.99	0\\
11	0\\
11.01	0\\
11.02	0\\
11.03	0\\
11.04	0\\
11.05	0\\
11.06	0\\
11.07	0\\
11.08	0\\
11.09	0\\
11.1	0\\
11.11	0\\
11.12	0\\
11.13	0\\
11.14	0\\
11.15	0\\
11.16	0\\
11.17	0\\
11.18	0\\
11.19	0\\
11.2	0\\
11.21	0\\
11.22	0\\
11.23	0\\
11.24	0\\
11.25	0\\
11.26	0\\
11.27	0\\
11.28	0\\
11.29	0\\
11.3	0\\
11.31	0\\
11.32	0\\
11.33	0\\
11.34	0\\
11.35	0\\
11.36	0\\
11.37	0\\
11.38	0\\
11.39	0\\
11.4	0\\
11.41	0\\
11.42	0\\
11.43	0\\
11.44	0\\
11.45	0\\
11.46	0\\
11.47	0\\
11.48	0\\
11.49	0\\
11.5	0\\
11.51	0\\
11.52	0\\
11.53	0\\
11.54	0\\
11.55	0\\
11.56	0\\
11.57	0\\
11.58	0\\
11.59	0\\
11.6	0\\
11.61	0\\
11.62	0\\
11.63	0\\
11.64	0\\
11.65	0\\
11.66	0\\
11.67	0\\
11.68	0\\
11.69	0\\
11.7	0\\
11.71	0\\
11.72	0\\
11.73	0\\
11.74	0\\
11.75	0\\
11.76	0\\
11.77	0\\
11.78	0\\
11.79	0\\
11.8	0\\
11.81	0\\
11.82	0\\
11.83	0\\
11.84	0\\
11.85	0\\
11.86	0\\
11.87	0\\
11.88	0\\
11.89	0\\
11.9	0\\
11.91	0\\
11.92	0\\
11.93	0\\
11.94	0\\
11.95	0\\
11.96	0\\
11.97	0\\
11.98	0\\
11.99	0\\
12	0\\
12.01	0\\
12.02	0\\
12.03	0\\
12.04	0\\
12.05	0\\
12.06	0\\
12.07	0\\
12.08	0\\
12.09	0\\
12.1	0\\
12.11	0\\
12.12	0\\
12.13	0\\
12.14	0\\
12.15	0\\
12.16	0\\
12.17	0\\
12.18	0\\
12.19	0\\
12.2	0\\
12.21	0\\
12.22	0\\
12.23	0\\
12.24	0\\
12.25	0\\
12.26	0\\
12.27	0\\
12.28	0\\
12.29	0\\
12.3	0\\
12.31	0\\
12.32	0\\
12.33	0\\
12.34	0\\
12.35	0\\
12.36	0\\
12.37	0\\
12.38	0\\
12.39	0\\
12.4	0\\
12.41	0\\
12.42	0\\
12.43	0\\
12.44	0\\
12.45	0\\
12.46	0\\
12.47	0\\
12.48	0\\
12.49	0\\
12.5	0\\
12.51	0\\
12.52	0\\
12.53	0\\
12.54	0\\
12.55	0\\
12.56	0\\
12.57	0\\
12.58	0\\
12.59	0\\
12.6	0\\
12.61	0\\
12.62	0\\
12.63	0\\
12.64	0\\
12.65	0\\
12.66	0\\
12.67	0\\
12.68	0\\
12.69	0\\
12.7	0\\
12.71	0\\
12.72	0\\
12.73	0\\
12.74	0\\
12.75	0\\
12.76	0\\
12.77	0\\
12.78	0\\
12.79	0\\
12.8	0\\
12.81	0\\
12.82	0\\
12.83	0\\
12.84	0\\
12.85	0\\
12.86	0\\
12.87	0\\
12.88	0\\
12.89	0\\
12.9	0\\
12.91	0\\
12.92	0\\
12.93	0\\
12.94	0\\
12.95	0\\
12.96	0\\
12.97	0\\
12.98	0\\
12.99	0\\
13	0\\
13.01	0\\
13.02	0\\
13.03	0\\
13.04	0\\
13.05	0\\
13.06	0\\
13.07	0\\
13.08	0\\
13.09	0\\
13.1	0\\
13.11	0\\
13.12	0\\
13.13	0\\
13.14	0\\
13.15	0\\
13.16	0\\
13.17	0\\
13.18	0\\
13.19	0\\
13.2	0\\
13.21	0\\
13.22	0\\
13.23	0\\
13.24	0\\
13.25	0\\
13.26	0\\
13.27	0\\
13.28	0\\
13.29	0\\
13.3	0\\
13.31	0\\
13.32	0\\
13.33	0\\
13.34	0\\
13.35	0\\
13.36	0\\
13.37	0\\
13.38	0\\
13.39	0\\
13.4	0\\
13.41	0\\
13.42	0\\
13.43	0\\
13.44	0\\
13.45	0\\
13.46	0\\
13.47	0\\
13.48	0\\
13.49	0\\
13.5	0\\
13.51	0\\
13.52	0\\
13.53	0\\
13.54	0\\
13.55	0\\
13.56	0\\
13.57	0\\
13.58	0\\
13.59	0\\
13.6	0\\
13.61	0\\
13.62	0\\
13.63	0\\
13.64	0\\
13.65	0\\
13.66	0\\
13.67	0\\
13.68	0\\
13.69	0\\
13.7	0\\
13.71	0\\
13.72	0\\
13.73	0\\
13.74	0\\
13.75	0\\
13.76	0\\
13.77	0\\
13.78	0\\
13.79	0\\
13.8	0\\
13.81	0\\
13.82	0\\
13.83	0\\
13.84	0\\
13.85	0\\
13.86	0\\
13.87	0\\
13.88	0\\
13.89	0\\
13.9	0\\
13.91	0\\
13.92	0\\
13.93	0\\
13.94	0\\
13.95	0\\
13.96	0\\
13.97	0\\
13.98	0\\
13.99	0\\
14	0\\
14.01	0\\
14.02	0\\
14.03	0\\
14.04	0\\
14.05	0\\
14.06	0\\
14.07	0\\
14.08	0\\
14.09	0\\
14.1	0\\
14.11	0\\
14.12	0\\
14.13	0\\
14.14	0\\
14.15	0\\
14.16	0\\
14.17	0\\
14.18	0\\
14.19	0\\
14.2	0\\
14.21	0\\
14.22	0\\
14.23	0\\
14.24	0\\
14.25	0\\
14.26	0\\
14.27	0\\
14.28	0\\
14.29	0\\
14.3	0\\
14.31	0\\
14.32	0\\
14.33	0\\
14.34	0\\
14.35	0\\
14.36	0\\
14.37	0\\
14.38	0\\
14.39	0\\
14.4	0\\
14.41	0\\
14.42	0\\
14.43	0\\
14.44	0\\
14.45	0\\
14.46	0\\
14.47	0\\
14.48	0\\
14.49	0\\
14.5	0\\
14.51	0\\
14.52	0\\
14.53	0\\
14.54	0\\
14.55	0\\
14.56	0\\
14.57	0\\
14.58	0\\
14.59	0\\
14.6	0\\
14.61	0\\
14.62	0\\
14.63	0\\
14.64	0\\
14.65	0\\
14.66	0\\
14.67	0\\
14.68	0\\
14.69	0\\
14.7	0\\
14.71	0\\
14.72	0\\
14.73	0\\
14.74	0\\
14.75	0\\
14.76	0\\
14.77	0\\
14.78	0\\
14.79	0\\
14.8	0\\
14.81	0\\
14.82	0\\
14.83	0\\
14.84	0\\
14.85	0\\
14.86	0\\
14.87	0\\
14.88	0\\
14.89	0\\
14.9	0\\
14.91	0\\
14.92	0\\
14.93	0\\
14.94	0\\
14.95	0\\
14.96	0\\
14.97	0\\
14.98	0\\
14.99	0\\
15	0\\
15.01	0\\
15.02	0\\
15.03	0\\
15.04	0\\
15.05	0\\
15.06	0\\
15.07	0\\
15.08	0\\
15.09	0\\
15.1	0\\
15.11	0\\
15.12	0\\
15.13	0\\
15.14	0\\
15.15	0\\
15.16	0\\
15.17	0\\
15.18	0\\
15.19	0\\
15.2	0\\
15.21	0\\
15.22	0\\
15.23	0\\
15.24	0\\
15.25	0\\
15.26	0\\
15.27	0\\
15.28	0\\
15.29	0\\
15.3	0\\
15.31	0\\
15.32	0\\
15.33	0\\
15.34	0\\
15.35	0\\
15.36	0\\
15.37	0\\
15.38	0\\
15.39	0\\
15.4	0\\
15.41	0\\
15.42	0\\
15.43	0\\
15.44	0\\
15.45	0\\
15.46	0\\
15.47	0\\
15.48	0\\
15.49	0\\
15.5	0\\
15.51	0\\
15.52	0\\
15.53	0\\
15.54	0\\
15.55	0\\
15.56	0\\
15.57	0\\
15.58	0\\
15.59	0\\
15.6	0\\
15.61	0\\
15.62	0\\
15.63	0\\
15.64	0\\
15.65	0\\
15.66	0\\
15.67	0\\
15.68	0\\
15.69	0\\
15.7	0\\
15.71	0\\
15.72	0\\
15.73	0\\
15.74	0\\
15.75	0\\
15.76	0\\
15.77	0\\
15.78	0\\
15.79	0\\
15.8	0\\
15.81	0\\
15.82	0\\
15.83	0\\
15.84	0\\
15.85	0\\
15.86	0\\
15.87	0\\
15.88	0\\
15.89	0\\
15.9	0\\
15.91	0\\
15.92	0\\
15.93	0\\
15.94	0\\
15.95	0\\
15.96	0\\
15.97	0\\
15.98	0\\
15.99	0\\
16	0\\
16.01	0\\
16.02	0\\
16.03	0\\
16.04	0\\
16.05	0\\
16.06	0\\
16.07	0\\
16.08	0\\
16.09	0\\
16.1	0\\
16.11	0\\
16.12	0\\
16.13	0\\
16.14	0\\
16.15	0\\
16.16	0\\
16.17	0\\
16.18	0\\
16.19	0\\
16.2	0\\
16.21	0\\
16.22	0\\
16.23	0\\
16.24	0\\
16.25	0\\
16.26	0\\
16.27	0\\
16.28	0\\
16.29	0\\
16.3	0\\
16.31	0\\
16.32	0\\
16.33	0\\
16.34	0\\
16.35	0\\
16.36	0\\
16.37	0\\
16.38	0\\
16.39	0\\
16.4	0\\
16.41	0\\
16.42	0\\
16.43	0\\
16.44	0\\
16.45	0\\
16.46	0\\
16.47	0\\
16.48	0\\
16.49	0\\
16.5	0\\
16.51	0\\
16.52	0\\
16.53	0\\
16.54	0\\
16.55	0\\
16.56	0\\
16.57	0\\
16.58	0\\
16.59	0\\
16.6	0\\
16.61	0\\
16.62	0\\
16.63	0\\
16.64	0\\
16.65	0\\
16.66	0\\
16.67	0\\
16.68	0\\
16.69	0\\
16.7	0\\
16.71	0\\
16.72	0\\
16.73	0\\
16.74	0\\
16.75	0\\
16.76	0\\
16.77	0\\
16.78	0\\
16.79	0\\
16.8	0\\
16.81	0\\
16.82	0\\
16.83	0\\
16.84	0\\
16.85	0\\
16.86	0\\
16.87	0\\
16.88	0\\
16.89	0\\
16.9	0\\
16.91	0\\
16.92	0\\
16.93	0\\
16.94	0\\
16.95	0\\
16.96	0\\
16.97	0\\
16.98	0\\
16.99	0\\
17	0\\
17.01	0\\
17.02	0\\
17.03	0\\
17.04	0\\
17.05	0\\
17.06	0\\
17.07	0\\
17.08	0\\
17.09	0\\
17.1	0\\
17.11	0\\
17.12	0\\
17.13	0\\
17.14	0\\
17.15	0\\
17.16	0\\
17.17	0\\
17.18	0\\
17.19	0\\
17.2	0\\
17.21	0\\
17.22	0\\
17.23	0\\
17.24	0\\
17.25	0\\
17.26	0\\
17.27	0\\
17.28	0\\
17.29	0\\
17.3	0\\
17.31	0\\
17.32	0\\
17.33	0\\
17.34	0\\
17.35	0\\
17.36	0\\
17.37	0\\
17.38	0\\
17.39	0\\
17.4	0\\
17.41	0\\
17.42	0\\
17.43	0\\
17.44	0\\
17.45	0\\
17.46	0\\
17.47	0\\
17.48	0\\
17.49	0\\
17.5	0\\
17.51	0\\
17.52	0\\
17.53	0\\
17.54	0\\
17.55	0\\
17.56	0\\
17.57	0\\
17.58	0\\
17.59	0\\
17.6	0\\
17.61	0\\
17.62	0\\
17.63	0\\
17.64	0\\
17.65	0\\
17.66	0\\
17.67	0\\
17.68	0\\
17.69	0\\
17.7	0\\
17.71	0\\
17.72	0\\
17.73	0\\
17.74	0\\
17.75	0\\
17.76	0\\
17.77	0\\
17.78	0\\
17.79	0\\
17.8	0\\
17.81	0\\
17.82	0\\
17.83	0\\
17.84	0\\
17.85	0\\
17.86	0\\
17.87	0\\
17.88	0\\
17.89	0\\
17.9	0\\
17.91	0\\
17.92	0\\
17.93	0\\
17.94	0\\
17.95	0\\
17.96	0\\
17.97	0\\
17.98	0\\
17.99	0\\
18	0\\
18.01	0\\
18.02	0\\
18.03	0\\
18.04	0\\
18.05	0\\
18.06	0\\
18.07	0\\
18.08	0\\
18.09	0\\
18.1	0\\
18.11	0\\
18.12	0\\
18.13	0\\
18.14	0\\
18.15	0\\
18.16	0\\
18.17	0\\
18.18	0\\
18.19	0\\
18.2	0\\
18.21	0\\
18.22	0\\
18.23	0\\
18.24	0\\
18.25	0\\
18.26	0\\
18.27	0\\
18.28	0\\
18.29	0\\
18.3	0\\
18.31	0\\
18.32	0\\
18.33	0\\
18.34	0\\
18.35	0\\
18.36	0\\
18.37	0\\
18.38	0\\
18.39	0\\
18.4	0\\
18.41	0\\
18.42	0\\
18.43	0\\
18.44	0\\
18.45	0\\
18.46	0\\
18.47	0\\
18.48	0\\
18.49	0\\
18.5	0\\
18.51	0\\
18.52	0\\
18.53	0\\
18.54	0\\
18.55	0\\
18.56	0\\
18.57	0\\
18.58	0\\
18.59	0\\
18.6	0\\
18.61	0\\
18.62	0\\
18.63	0\\
18.64	0\\
18.65	0\\
18.66	0\\
18.67	0\\
18.68	0\\
18.69	0\\
18.7	0\\
18.71	0\\
18.72	0\\
18.73	0\\
18.74	0\\
18.75	0\\
18.76	0\\
18.77	0\\
18.78	0\\
18.79	0\\
18.8	0\\
18.81	0\\
18.82	0\\
18.83	0\\
18.84	0\\
18.85	0\\
18.86	0\\
18.87	0\\
18.88	0\\
18.89	0\\
18.9	0\\
18.91	0\\
18.92	0\\
18.93	0\\
18.94	0\\
18.95	0\\
18.96	0\\
18.97	0\\
18.98	0\\
18.99	0\\
19	0\\
19.01	0\\
19.02	0\\
19.03	0\\
19.04	0\\
19.05	0\\
19.06	0\\
19.07	0\\
19.08	0\\
19.09	0\\
19.1	0\\
19.11	0\\
19.12	0\\
19.13	0\\
19.14	0\\
19.15	0\\
19.16	0\\
19.17	0\\
19.18	0\\
19.19	0\\
19.2	0\\
19.21	0\\
19.22	0\\
19.23	0\\
19.24	0\\
19.25	0\\
19.26	0\\
19.27	0\\
19.28	0\\
19.29	0\\
19.3	0\\
19.31	0\\
19.32	0\\
19.33	0\\
19.34	0\\
19.35	0\\
19.36	0\\
19.37	0\\
19.38	0\\
19.39	0\\
19.4	0\\
19.41	0\\
19.42	0\\
19.43	0\\
19.44	0\\
19.45	0\\
19.46	0\\
19.47	0\\
19.48	0\\
19.49	0\\
19.5	0\\
19.51	0\\
19.52	0\\
19.53	0\\
19.54	0\\
19.55	0\\
19.56	0\\
19.57	0\\
19.58	0\\
19.59	0\\
19.6	0\\
19.61	0\\
19.62	0\\
19.63	0\\
19.64	0\\
19.65	0\\
19.66	0\\
19.67	0\\
19.68	0\\
19.69	0\\
19.7	0\\
19.71	0\\
19.72	0\\
19.73	0\\
19.74	0\\
19.75	0\\
19.76	0\\
19.77	0\\
19.78	0\\
19.79	0\\
19.8	0\\
19.81	0\\
19.82	0\\
19.83	0\\
19.84	0\\
19.85	0\\
19.86	0\\
19.87	0\\
19.88	0\\
19.89	0\\
19.9	0\\
19.91	0\\
19.92	0\\
19.93	0\\
19.94	0\\
19.95	0\\
19.96	0\\
19.97	0\\
19.98	0\\
19.99	0\\
20	0\\
20.01	0\\
20.02	0\\
20.03	0\\
20.04	0\\
20.05	0\\
20.06	0\\
20.07	0\\
20.08	0\\
20.09	0\\
20.1	0\\
20.11	0\\
20.12	0\\
20.13	0\\
20.14	0\\
20.15	0\\
20.16	0\\
20.17	0\\
20.18	0\\
20.19	0\\
20.2	0\\
20.21	0\\
20.22	0\\
20.23	0\\
20.24	0\\
20.25	0\\
20.26	0\\
20.27	0\\
20.28	0\\
20.29	0\\
20.3	0\\
20.31	0\\
20.32	0\\
20.33	0\\
20.34	0\\
20.35	0\\
20.36	0\\
20.37	0\\
20.38	0\\
20.39	0\\
20.4	0\\
20.41	0\\
20.42	0\\
20.43	0\\
20.44	0\\
20.45	0\\
20.46	0\\
20.47	0\\
20.48	0\\
20.49	0\\
20.5	0\\
20.51	0\\
20.52	0\\
20.53	0\\
20.54	0\\
20.55	0\\
20.56	0\\
20.57	0\\
20.58	0\\
20.59	0\\
20.6	0\\
20.61	0\\
20.62	0\\
20.63	0\\
20.64	0\\
20.65	0\\
20.66	0\\
20.67	0\\
20.68	0\\
20.69	0\\
20.7	0\\
20.71	0\\
20.72	0\\
20.73	0\\
20.74	0\\
20.75	0\\
20.76	0\\
20.77	0\\
20.78	0\\
20.79	0\\
20.8	0\\
20.81	0\\
20.82	0\\
20.83	0\\
20.84	0\\
20.85	0\\
20.86	0\\
20.87	0\\
20.88	0\\
20.89	0\\
20.9	0\\
20.91	0\\
20.92	0\\
20.93	0\\
20.94	0\\
20.95	0\\
20.96	0\\
20.97	0\\
20.98	0\\
20.99	0\\
21	0\\
21.01	0\\
21.02	0\\
21.03	0\\
21.04	0\\
21.05	0\\
21.06	0\\
21.07	0\\
21.08	0\\
21.09	0\\
21.1	0\\
21.11	0\\
21.12	0\\
21.13	0\\
21.14	0\\
21.15	0\\
21.16	0\\
21.17	0\\
21.18	0\\
21.19	0\\
21.2	0\\
21.21	0\\
21.22	0\\
21.23	0\\
21.24	0\\
21.25	0\\
21.26	0\\
21.27	0\\
21.28	0\\
21.29	0\\
21.3	0\\
21.31	0\\
21.32	0\\
21.33	0\\
21.34	0\\
21.35	0\\
21.36	0\\
21.37	0\\
21.38	0\\
21.39	0\\
21.4	0\\
21.41	0\\
21.42	0\\
21.43	0\\
21.44	0\\
21.45	0\\
21.46	0\\
21.47	0\\
21.48	0\\
21.49	0\\
21.5	0\\
21.51	0\\
21.52	0\\
21.53	0\\
21.54	0\\
21.55	0\\
21.56	0\\
21.57	0\\
21.58	0\\
21.59	0\\
21.6	0\\
21.61	0\\
21.62	0\\
21.63	0\\
21.64	0\\
21.65	0\\
21.66	0\\
21.67	0\\
21.68	0\\
21.69	0\\
21.7	0\\
21.71	0\\
21.72	0\\
21.73	0\\
21.74	0\\
21.75	0\\
21.76	0\\
21.77	0\\
21.78	0\\
21.79	0\\
21.8	0\\
21.81	0\\
21.82	0\\
21.83	0\\
21.84	0\\
21.85	0\\
21.86	0\\
21.87	0\\
21.88	0\\
21.89	0\\
21.9	0\\
21.91	0\\
21.92	0\\
21.93	0\\
21.94	0\\
21.95	0\\
21.96	0\\
21.97	0\\
21.98	0\\
21.99	0\\
22	0\\
22.01	0\\
22.02	0\\
22.03	0\\
22.04	0\\
22.05	0\\
22.06	0\\
22.07	0\\
22.08	0\\
22.09	0\\
22.1	0\\
22.11	0\\
22.12	0\\
22.13	0\\
22.14	0\\
22.15	0\\
22.16	0\\
22.17	0\\
22.18	0\\
22.19	0\\
22.2	0\\
22.21	0\\
22.22	0\\
22.23	0\\
22.24	0\\
22.25	0\\
22.26	0\\
22.27	0\\
22.28	0\\
22.29	0\\
22.3	0\\
22.31	0\\
22.32	0\\
22.33	0\\
22.34	0\\
22.35	0\\
22.36	0\\
22.37	0\\
22.38	0\\
22.39	0\\
22.4	0\\
22.41	0\\
22.42	0\\
22.43	0\\
22.44	0\\
22.45	0\\
22.46	0\\
22.47	0\\
22.48	0\\
22.49	0\\
22.5	0\\
22.51	0\\
22.52	0\\
22.53	0\\
22.54	0\\
22.55	0\\
22.56	0\\
22.57	0\\
22.58	0\\
22.59	0\\
22.6	0\\
22.61	0\\
22.62	0\\
22.63	0\\
22.64	0\\
22.65	0\\
22.66	0\\
22.67	0\\
22.68	0\\
22.69	0\\
22.7	0\\
22.71	0\\
22.72	0\\
22.73	0\\
22.74	0\\
22.75	0\\
22.76	0\\
22.77	0\\
22.78	0\\
22.79	0\\
22.8	0\\
22.81	0\\
22.82	0\\
22.83	0\\
22.84	0\\
22.85	0\\
22.86	0\\
22.87	0\\
22.88	0\\
22.89	0\\
22.9	0\\
22.91	0\\
22.92	0\\
22.93	0\\
22.94	0\\
22.95	0\\
22.96	0\\
22.97	0\\
22.98	0\\
22.99	0\\
23	0\\
23.01	0\\
23.02	0\\
23.03	0\\
23.04	0\\
23.05	0\\
23.06	0\\
23.07	0\\
23.08	0\\
23.09	0\\
23.1	0\\
23.11	0\\
23.12	0\\
23.13	0\\
23.14	0\\
23.15	0\\
23.16	0\\
23.17	0\\
23.18	0\\
23.19	0\\
23.2	0\\
23.21	0\\
23.22	0\\
23.23	0\\
23.24	0\\
23.25	0\\
23.26	0\\
23.27	0\\
23.28	0\\
23.29	0\\
23.3	0\\
23.31	0\\
23.32	0\\
23.33	0\\
23.34	0\\
23.35	0\\
23.36	0\\
23.37	0\\
23.38	0\\
23.39	0\\
23.4	0\\
23.41	0\\
23.42	0\\
23.43	0\\
23.44	0\\
23.45	0\\
23.46	0\\
23.47	0\\
23.48	0\\
23.49	0\\
23.5	0\\
23.51	0\\
23.52	0\\
23.53	0\\
23.54	0\\
23.55	0\\
23.56	0\\
23.57	0\\
23.58	0\\
23.59	0\\
23.6	0\\
23.61	0\\
23.62	0\\
23.63	0\\
23.64	0\\
23.65	0\\
23.66	0\\
23.67	0\\
23.68	0\\
23.69	0\\
23.7	0\\
23.71	0\\
23.72	0\\
23.73	0\\
23.74	0\\
23.75	0\\
23.76	0\\
23.77	0\\
23.78	0\\
23.79	0\\
23.8	0\\
23.81	0\\
23.82	0\\
23.83	0\\
23.84	0\\
23.85	0\\
23.86	0\\
23.87	0\\
23.88	0\\
23.89	0\\
23.9	0\\
23.91	0\\
23.92	0\\
23.93	0\\
23.94	0\\
23.95	0\\
23.96	0\\
23.97	0\\
23.98	0\\
23.99	0\\
24	0\\
24.01	0\\
24.02	0\\
24.03	0\\
24.04	0\\
24.05	0\\
24.06	0\\
24.07	0\\
24.08	0\\
24.09	0\\
24.1	0\\
24.11	0\\
24.12	0\\
24.13	0\\
24.14	0\\
24.15	0\\
24.16	0\\
24.17	0\\
24.18	0\\
24.19	0\\
24.2	0\\
24.21	0\\
24.22	0\\
24.23	0\\
24.24	0\\
24.25	0\\
24.26	0\\
24.27	0\\
24.28	0\\
24.29	0\\
24.3	0\\
24.31	0\\
24.32	0\\
24.33	0\\
24.34	0\\
24.35	0\\
24.36	0\\
24.37	0\\
24.38	0\\
24.39	0\\
24.4	0\\
24.41	0\\
24.42	0\\
24.43	0\\
24.44	0\\
24.45	0\\
24.46	0\\
24.47	0\\
24.48	0\\
24.49	0\\
24.5	0\\
24.51	0\\
24.52	0\\
24.53	0\\
24.54	0\\
24.55	0\\
24.56	0\\
24.57	0\\
24.58	0\\
24.59	0\\
24.6	0\\
24.61	0\\
24.62	0\\
24.63	0\\
24.64	0\\
24.65	0\\
24.66	0\\
24.67	0\\
24.68	0\\
24.69	0\\
24.7	0\\
24.71	0\\
24.72	0\\
24.73	0\\
24.74	0\\
24.75	0\\
24.76	0\\
24.77	0\\
24.78	0\\
24.79	0\\
24.8	0\\
24.81	0\\
24.82	0\\
24.83	0\\
24.84	0\\
24.85	0\\
24.86	0\\
24.87	0\\
24.88	0\\
24.89	0\\
24.9	0\\
24.91	0\\
24.92	0\\
24.93	0\\
24.94	0\\
24.95	0\\
24.96	0\\
24.97	0\\
24.98	0\\
24.99	0\\
25	0\\
25.01	0\\
25.02	0\\
25.03	0\\
25.04	0\\
25.05	0\\
25.06	0\\
25.07	0\\
25.08	0\\
25.09	0\\
25.1	0\\
25.11	0\\
25.12	0\\
25.13	0\\
25.14	0\\
25.15	0\\
25.16	0\\
25.17	0\\
25.18	0\\
25.19	0\\
25.2	0\\
25.21	0\\
25.22	0\\
25.23	0\\
25.24	0\\
25.25	0\\
25.26	0\\
25.27	0\\
25.28	0\\
25.29	0\\
25.3	0\\
25.31	0\\
25.32	0\\
25.33	0\\
25.34	0\\
25.35	0\\
25.36	0\\
25.37	0\\
25.38	0\\
25.39	0\\
25.4	0\\
25.41	0\\
25.42	0\\
25.43	0\\
25.44	0\\
25.45	0\\
25.46	0\\
25.47	0\\
25.48	0\\
25.49	0\\
25.5	0\\
25.51	0\\
25.52	0\\
25.53	0\\
25.54	0\\
25.55	0\\
25.56	0\\
25.57	0\\
25.58	0\\
25.59	0\\
25.6	0\\
25.61	0\\
25.62	0\\
25.63	0\\
25.64	0\\
25.65	0\\
25.66	0\\
25.67	0\\
25.68	0\\
25.69	0\\
25.7	0\\
25.71	0\\
25.72	0\\
25.73	0\\
25.74	0\\
25.75	0\\
25.76	0\\
25.77	0\\
25.78	0\\
25.79	0\\
25.8	0\\
25.81	0\\
25.82	0\\
25.83	0\\
25.84	0\\
25.85	0\\
25.86	0\\
25.87	0\\
25.88	0\\
25.89	0\\
25.9	0\\
25.91	0\\
25.92	0\\
25.93	0\\
25.94	0\\
25.95	0\\
25.96	0\\
25.97	0\\
25.98	0\\
25.99	0\\
26	0\\
26.01	0\\
26.02	0\\
26.03	0\\
26.04	0\\
26.05	0\\
26.06	0\\
26.07	0\\
26.08	0\\
26.09	0\\
26.1	0\\
26.11	0\\
26.12	0\\
26.13	0\\
26.14	0\\
26.15	0\\
26.16	0\\
26.17	0\\
26.18	0\\
26.19	0\\
26.2	0\\
26.21	0\\
26.22	0\\
26.23	0\\
26.24	0\\
26.25	0\\
26.26	0\\
26.27	0\\
26.28	0\\
26.29	0\\
26.3	0\\
26.31	0\\
26.32	0\\
26.33	0\\
26.34	0\\
26.35	0\\
26.36	0\\
26.37	0\\
26.38	0\\
26.39	0\\
26.4	0\\
26.41	0\\
26.42	0\\
26.43	0\\
26.44	0\\
26.45	0\\
26.46	0\\
26.47	0\\
26.48	0\\
26.49	0\\
26.5	0\\
26.51	0\\
26.52	0\\
26.53	0\\
26.54	0\\
26.55	0\\
26.56	0\\
26.57	0\\
26.58	0\\
26.59	0\\
26.6	0\\
26.61	0\\
26.62	0\\
26.63	0\\
26.64	0\\
26.65	0\\
26.66	0\\
26.67	0\\
26.68	0\\
26.69	0\\
26.7	0\\
26.71	0\\
26.72	0\\
26.73	0\\
26.74	0\\
26.75	0\\
26.76	0\\
26.77	0\\
26.78	0\\
26.79	0\\
26.8	0\\
26.81	0\\
26.82	0\\
26.83	0\\
26.84	0\\
26.85	0\\
26.86	0\\
26.87	0\\
26.88	0\\
26.89	0\\
26.9	0\\
26.91	0\\
26.92	0\\
26.93	0\\
26.94	0\\
26.95	0\\
26.96	0\\
26.97	0\\
26.98	0\\
26.99	0\\
27	0\\
27.01	0\\
27.02	0\\
27.03	0\\
27.04	0\\
27.05	0\\
27.06	0\\
27.07	0\\
27.08	0\\
27.09	0\\
27.1	0\\
27.11	0\\
27.12	0\\
27.13	0\\
27.14	0\\
27.15	0\\
27.16	0\\
27.17	0\\
27.18	0\\
27.19	0\\
27.2	0\\
27.21	0\\
27.22	0\\
27.23	0\\
27.24	0\\
27.25	0\\
27.26	0\\
27.27	0\\
27.28	0\\
27.29	0\\
27.3	0\\
27.31	0\\
27.32	0\\
27.33	0\\
27.34	0\\
27.35	0\\
27.36	0\\
27.37	0\\
27.38	0\\
27.39	0\\
27.4	0\\
27.41	0\\
27.42	0\\
27.43	0\\
27.44	0\\
27.45	0\\
27.46	0\\
27.47	0\\
27.48	0\\
27.49	0\\
27.5	0\\
27.51	0\\
27.52	0\\
27.53	0\\
27.54	0\\
27.55	0\\
27.56	0\\
27.57	0\\
27.58	0\\
27.59	0\\
27.6	0\\
27.61	0\\
27.62	0\\
27.63	0\\
27.64	0\\
27.65	0\\
27.66	0\\
27.67	0\\
27.68	0\\
27.69	0\\
27.7	0\\
27.71	0\\
27.72	0\\
27.73	0\\
27.74	0\\
27.75	0\\
27.76	0\\
27.77	0\\
27.78	0\\
27.79	0\\
27.8	0\\
27.81	0\\
27.82	0\\
27.83	0\\
27.84	0\\
27.85	0\\
27.86	0\\
27.87	0\\
27.88	0\\
27.89	0\\
27.9	0\\
27.91	0\\
27.92	0\\
27.93	0\\
27.94	0\\
27.95	0\\
27.96	0\\
27.97	0\\
27.98	0\\
27.99	0\\
28	0\\
28.01	0\\
28.02	0\\
28.03	0\\
28.04	0\\
28.05	0\\
28.06	0\\
28.07	0\\
28.08	0\\
28.09	0\\
28.1	0\\
28.11	0\\
28.12	0\\
28.13	0\\
28.14	0\\
28.15	0\\
28.16	0\\
28.17	0\\
28.18	0\\
28.19	0\\
28.2	0\\
28.21	0\\
28.22	0\\
28.23	0\\
28.24	0\\
28.25	0\\
28.26	0\\
28.27	0\\
28.28	0\\
28.29	0\\
28.3	0\\
28.31	0\\
28.32	0\\
28.33	0\\
28.34	0\\
28.35	0\\
28.36	0\\
28.37	0\\
28.38	0\\
28.39	0\\
28.4	0\\
28.41	0\\
28.42	0\\
28.43	0\\
28.44	0\\
28.45	0\\
28.46	0\\
28.47	0\\
28.48	0\\
28.49	0\\
28.5	0\\
28.51	0\\
28.52	0\\
28.53	0\\
28.54	0\\
28.55	0\\
28.56	0\\
28.57	0\\
28.58	0\\
28.59	0\\
28.6	0\\
28.61	0\\
28.62	0\\
28.63	0\\
28.64	0\\
28.65	0\\
28.66	0\\
28.67	0\\
28.68	0\\
28.69	0\\
28.7	0\\
28.71	0\\
28.72	0\\
28.73	0\\
28.74	0\\
28.75	0\\
28.76	0\\
28.77	0\\
28.78	0\\
28.79	0\\
28.8	0\\
28.81	0\\
28.82	0\\
28.83	0\\
28.84	0\\
28.85	0\\
28.86	0\\
28.87	0\\
28.88	0\\
28.89	0\\
28.9	0\\
28.91	0\\
28.92	0\\
28.93	0\\
28.94	0\\
28.95	0\\
28.96	0\\
28.97	0\\
28.98	0\\
28.99	0\\
29	0\\
29.01	0\\
29.02	0\\
29.03	0\\
29.04	0\\
29.05	0\\
29.06	0\\
29.07	0\\
29.08	0\\
29.09	0\\
29.1	0\\
29.11	0\\
29.12	0\\
29.13	0\\
29.14	0\\
29.15	0\\
29.16	0\\
29.17	0\\
29.18	0\\
29.19	0\\
29.2	0\\
29.21	0\\
29.22	0\\
29.23	0\\
29.24	0\\
29.25	0\\
29.26	0\\
29.27	0\\
29.28	0\\
29.29	0\\
29.3	0\\
29.31	0\\
29.32	0\\
29.33	0\\
29.34	0\\
29.35	0\\
29.36	0\\
29.37	0\\
29.38	0\\
29.39	0\\
29.4	0\\
29.41	0\\
29.42	0\\
29.43	0\\
29.44	0\\
29.45	0\\
29.46	0\\
29.47	0\\
29.48	0\\
29.49	0\\
29.5	0\\
29.51	0\\
29.52	0\\
29.53	0\\
29.54	0\\
29.55	0\\
29.56	0\\
29.57	0\\
29.58	0\\
29.59	0\\
29.6	0\\
29.61	0\\
29.62	0\\
29.63	0\\
29.64	0\\
29.65	0\\
29.66	0\\
29.67	0\\
29.68	0\\
29.69	0\\
29.7	0\\
29.71	0\\
29.72	0\\
29.73	0\\
29.74	0\\
29.75	0\\
29.76	0\\
29.77	0\\
29.78	0\\
29.79	0\\
29.8	0\\
29.81	0\\
29.82	0\\
29.83	0\\
29.84	0\\
29.85	0\\
29.86	0\\
29.87	0\\
29.88	0\\
29.89	0\\
29.9	0\\
29.91	0\\
29.92	0\\
29.93	0\\
29.94	0\\
29.95	0\\
29.96	0\\
29.97	0\\
29.98	0\\
29.99	0\\
30	0\\
30.01	0\\
30.02	0\\
30.03	0\\
30.04	0\\
30.05	0\\
30.06	0\\
30.07	0\\
30.08	0\\
30.09	0\\
30.1	0\\
30.11	0\\
30.12	0\\
30.13	0\\
30.14	0\\
30.15	0\\
30.16	0\\
30.17	0\\
30.18	0\\
30.19	0\\
30.2	0\\
30.21	0\\
30.22	0\\
30.23	0\\
30.24	0\\
30.25	0\\
30.26	0\\
30.27	0\\
30.28	0\\
30.29	0\\
30.3	0\\
30.31	0\\
30.32	0\\
30.33	0\\
30.34	0\\
30.35	0\\
30.36	0\\
30.37	0\\
30.38	0\\
30.39	0\\
30.4	0\\
30.41	0\\
30.42	0\\
30.43	0\\
30.44	0\\
30.45	0\\
30.46	0\\
30.47	0\\
30.48	0\\
30.49	0\\
30.5	0\\
30.51	0\\
30.52	0\\
30.53	0\\
30.54	0\\
30.55	0\\
30.56	0\\
30.57	0\\
30.58	0\\
30.59	0\\
30.6	0\\
30.61	0\\
30.62	0\\
30.63	0\\
30.64	0\\
30.65	0\\
30.66	0\\
30.67	0\\
30.68	0\\
30.69	0\\
30.7	0\\
30.71	0\\
30.72	0\\
30.73	0\\
30.74	0\\
30.75	0\\
30.76	0\\
30.77	0\\
30.78	0\\
30.79	0\\
30.8	0\\
30.81	0\\
30.82	0\\
30.83	0\\
30.84	0\\
30.85	0\\
30.86	0\\
30.87	0\\
30.88	0\\
30.89	0\\
30.9	0\\
30.91	0\\
30.92	0\\
30.93	0\\
30.94	0\\
30.95	0\\
30.96	0\\
30.97	0\\
30.98	0\\
30.99	0\\
31	0\\
31.01	0\\
31.02	0\\
31.03	0\\
31.04	0\\
31.05	0\\
31.06	0\\
31.07	0\\
31.08	0\\
31.09	0\\
31.1	0\\
31.11	0\\
31.12	0\\
31.13	0\\
31.14	0\\
31.15	0\\
31.16	0\\
31.17	0\\
31.18	0\\
31.19	0\\
31.2	0\\
31.21	0\\
31.22	0\\
31.23	0\\
31.24	0\\
31.25	0\\
31.26	0\\
31.27	0\\
31.28	0\\
31.29	0\\
31.3	0\\
31.31	0\\
31.32	0\\
31.33	0\\
31.34	0\\
31.35	0\\
31.36	0\\
31.37	0\\
31.38	0\\
31.39	0\\
31.4	0\\
31.41	0\\
31.42	0\\
31.43	0\\
31.44	0\\
31.45	0\\
31.46	0\\
31.47	0\\
31.48	0\\
31.49	0\\
31.5	0\\
31.51	0\\
31.52	0\\
31.53	0\\
31.54	0\\
31.55	0\\
31.56	0\\
31.57	0\\
31.58	0\\
31.59	0\\
31.6	0\\
31.61	0\\
31.62	0\\
31.63	0\\
31.64	0\\
31.65	0\\
31.66	0\\
31.67	0\\
31.68	0\\
31.69	0\\
31.7	0\\
31.71	0\\
31.72	0\\
31.73	0\\
31.74	0\\
31.75	0\\
31.76	0\\
31.77	0\\
31.78	0\\
31.79	0\\
31.8	0\\
31.81	0\\
31.82	0\\
31.83	0\\
31.84	0\\
31.85	0\\
31.86	0\\
31.87	0\\
31.88	0\\
31.89	0\\
31.9	0\\
31.91	0\\
31.92	0\\
31.93	0\\
31.94	0\\
31.95	0\\
31.96	0\\
31.97	0\\
31.98	0\\
31.99	0\\
32	0\\
32.01	0\\
32.02	0\\
32.03	0\\
32.04	0\\
32.05	0\\
32.06	0\\
32.07	0\\
32.08	0\\
32.09	0\\
32.1	0\\
32.11	0\\
32.12	0\\
32.13	0\\
32.14	0\\
32.15	0\\
32.16	0\\
32.17	0\\
32.18	0\\
32.19	0\\
32.2	0\\
32.21	0\\
32.22	0\\
32.23	0\\
32.24	0\\
32.25	0\\
32.26	0\\
32.27	0\\
32.28	0\\
32.29	0\\
32.3	0\\
32.31	0\\
32.32	0\\
32.33	0\\
32.34	0\\
32.35	0\\
32.36	0\\
32.37	0\\
32.38	0\\
32.39	0\\
32.4	0\\
32.41	0\\
32.42	0\\
32.43	0\\
32.44	0\\
32.45	0\\
32.46	0\\
32.47	0\\
32.48	0\\
32.49	0\\
32.5	0\\
32.51	0\\
32.52	0\\
32.53	0\\
32.54	0\\
32.55	0\\
32.56	0\\
32.57	0\\
32.58	0\\
32.59	0\\
32.6	0\\
32.61	0\\
32.62	0\\
32.63	0\\
32.64	0\\
32.65	0\\
32.66	0\\
32.67	0\\
32.68	0\\
32.69	0\\
32.7	0\\
32.71	0\\
32.72	0\\
32.73	0\\
32.74	0\\
32.75	0\\
32.76	0\\
32.77	0\\
32.78	0\\
32.79	0\\
32.8	0\\
32.81	0\\
32.82	0\\
32.83	0\\
32.84	0\\
32.85	0\\
32.86	0\\
32.87	0\\
32.88	0\\
32.89	0\\
32.9	0\\
32.91	0\\
32.92	0\\
32.93	0\\
32.94	0\\
32.95	0\\
32.96	0\\
32.97	0\\
32.98	0\\
32.99	0\\
33	0\\
33.01	0\\
33.02	0\\
33.03	0\\
33.04	0\\
33.05	0\\
33.06	0\\
33.07	0\\
33.08	0\\
33.09	0\\
33.1	0\\
33.11	0\\
33.12	0\\
33.13	0\\
33.14	0\\
33.15	0\\
33.16	0\\
33.17	0\\
33.18	0\\
33.19	0\\
33.2	0\\
33.21	0\\
33.22	0\\
33.23	0\\
33.24	0\\
33.25	0\\
33.26	0\\
33.27	0\\
33.28	0\\
33.29	0\\
33.3	0\\
33.31	0\\
33.32	0\\
33.33	0\\
33.34	0\\
33.35	0\\
33.36	0\\
33.37	0\\
33.38	0\\
33.39	0\\
33.4	0\\
33.41	0\\
33.42	0\\
33.43	0\\
33.44	0\\
33.45	0\\
33.46	0\\
33.47	0\\
33.48	0\\
33.49	0\\
33.5	0\\
33.51	0\\
33.52	0\\
33.53	0\\
33.54	0\\
33.55	0\\
33.56	0\\
33.57	0\\
33.58	0\\
33.59	0\\
33.6	0\\
33.61	0\\
33.62	0\\
33.63	0\\
33.64	0\\
33.65	0\\
33.66	0\\
33.67	0\\
33.68	0\\
33.69	0\\
33.7	0\\
33.71	0\\
33.72	0\\
33.73	0\\
33.74	0\\
33.75	0\\
33.76	0\\
33.77	0\\
33.78	0\\
33.79	0\\
33.8	0\\
33.81	0\\
33.82	0\\
33.83	0\\
33.84	0\\
33.85	0\\
33.86	0\\
33.87	0\\
33.88	0\\
33.89	0\\
33.9	0\\
33.91	0\\
33.92	0\\
33.93	0\\
33.94	0\\
33.95	0\\
33.96	0\\
33.97	0\\
33.98	0\\
33.99	0\\
34	0\\
34.01	0\\
34.02	0\\
34.03	0\\
34.04	0\\
34.05	0\\
34.06	0\\
34.07	0\\
34.08	0\\
34.09	0\\
34.1	0\\
34.11	0\\
34.12	0\\
34.13	0\\
34.14	0\\
34.15	0\\
34.16	0\\
34.17	0\\
34.18	0\\
34.19	0\\
34.2	0\\
34.21	0\\
34.22	0\\
34.23	0\\
34.24	0\\
34.25	0\\
34.26	0\\
34.27	0\\
34.28	0\\
34.29	0\\
34.3	0\\
34.31	0\\
34.32	0\\
34.33	0\\
34.34	0\\
34.35	0\\
34.36	0\\
34.37	0\\
34.38	0\\
34.39	0\\
34.4	0\\
34.41	0\\
34.42	0\\
34.43	0\\
34.44	0\\
34.45	0\\
34.46	0\\
34.47	0\\
34.48	0\\
34.49	0\\
34.5	0\\
34.51	0\\
34.52	0\\
34.53	0\\
34.54	0\\
34.55	0\\
34.56	0\\
34.57	0\\
34.58	0\\
34.59	0\\
34.6	0\\
34.61	0\\
34.62	0\\
34.63	0\\
34.64	0\\
34.65	0\\
34.66	0\\
34.67	0\\
34.68	0\\
34.69	0\\
34.7	0\\
34.71	0\\
34.72	0\\
34.73	0\\
34.74	0\\
34.75	0\\
34.76	0\\
34.77	0\\
34.78	0\\
34.79	0\\
34.8	0\\
34.81	0\\
34.82	0\\
34.83	0\\
34.84	0\\
34.85	0\\
34.86	0\\
34.87	0\\
34.88	0\\
34.89	0\\
34.9	0\\
34.91	0\\
34.92	0\\
34.93	0\\
34.94	0\\
34.95	0\\
34.96	0\\
34.97	0\\
34.98	0\\
34.99	0\\
35	0\\
35.01	0\\
35.02	0\\
35.03	0\\
35.04	0\\
35.05	0\\
35.06	0\\
35.07	0\\
35.08	0\\
35.09	0\\
35.1	0\\
35.11	0\\
35.12	0\\
35.13	0\\
35.14	0\\
35.15	0\\
35.16	0\\
35.17	0\\
35.18	0\\
35.19	0\\
35.2	0\\
35.21	0\\
35.22	0\\
35.23	0\\
35.24	0\\
35.25	0\\
35.26	0\\
35.27	0\\
35.28	0\\
35.29	0\\
35.3	0\\
35.31	0\\
35.32	0\\
35.33	0\\
35.34	0\\
35.35	0\\
35.36	0\\
35.37	0\\
35.38	0\\
35.39	0\\
35.4	0\\
35.41	0\\
35.42	0\\
35.43	0\\
35.44	0\\
35.45	0\\
35.46	0\\
35.47	0\\
35.48	0\\
35.49	0\\
35.5	0\\
35.51	0\\
35.52	0\\
35.53	0\\
35.54	0\\
35.55	0\\
35.56	0\\
35.57	0\\
35.58	0\\
35.59	0\\
35.6	0\\
35.61	0\\
35.62	0\\
35.63	0\\
35.64	0\\
35.65	0\\
35.66	0\\
35.67	0\\
35.68	0\\
35.69	0\\
35.7	0\\
35.71	0\\
35.72	0\\
35.73	0\\
35.74	0\\
35.75	0\\
35.76	0\\
35.77	0\\
35.78	0\\
35.79	0\\
35.8	0\\
35.81	0\\
35.82	0\\
35.83	0\\
35.84	0\\
35.85	0\\
35.86	0\\
35.87	0\\
35.88	0\\
35.89	0\\
35.9	0\\
35.91	0\\
35.92	0\\
35.93	0\\
35.94	0\\
35.95	0\\
35.96	0\\
35.97	0\\
35.98	0\\
35.99	0\\
36	0\\
36.01	0\\
36.02	0\\
36.03	0\\
36.04	0\\
36.05	0\\
36.06	0\\
36.07	0\\
36.08	0\\
36.09	0\\
36.1	0\\
36.11	0\\
36.12	0\\
36.13	0\\
36.14	0\\
36.15	0\\
36.16	0\\
36.17	0\\
36.18	0\\
36.19	0\\
36.2	0\\
36.21	0\\
36.22	0\\
36.23	0\\
36.24	0\\
36.25	0\\
36.26	0\\
36.27	0\\
36.28	0\\
36.29	0\\
36.3	0\\
36.31	0\\
36.32	0\\
36.33	0\\
36.34	0\\
36.35	0\\
36.36	0\\
36.37	0\\
36.38	0\\
36.39	0\\
36.4	0\\
36.41	0\\
36.42	0\\
36.43	0\\
36.44	0\\
36.45	0\\
36.46	0\\
36.47	0\\
36.48	0\\
36.49	0\\
36.5	0\\
36.51	0\\
36.52	0\\
36.53	0\\
36.54	0\\
36.55	0\\
36.56	0\\
36.57	0\\
36.58	0\\
36.59	0\\
36.6	0\\
36.61	0\\
36.62	0\\
36.63	0\\
36.64	0\\
36.65	0\\
36.66	0\\
36.67	0\\
36.68	0\\
36.69	0\\
36.7	0\\
36.71	0\\
36.72	0\\
36.73	0\\
36.74	0\\
36.75	0\\
36.76	0\\
36.77	0\\
36.78	0\\
36.79	0\\
36.8	0\\
36.81	0\\
36.82	0\\
36.83	0\\
36.84	0\\
36.85	0\\
36.86	0\\
36.87	0\\
36.88	0\\
36.89	0\\
36.9	0\\
36.91	0\\
36.92	0\\
36.93	0\\
36.94	0\\
36.95	0\\
36.96	0\\
36.97	0\\
36.98	0\\
36.99	0\\
37	0\\
37.01	0\\
37.02	0\\
37.03	0\\
37.04	0\\
37.05	0\\
37.06	0\\
37.07	0\\
37.08	0\\
37.09	0\\
37.1	0\\
37.11	0\\
37.12	0\\
37.13	0\\
37.14	0\\
37.15	0\\
37.16	0\\
37.17	0\\
37.18	0\\
37.19	0\\
37.2	0\\
37.21	0\\
37.22	0\\
37.23	0\\
37.24	0\\
37.25	0\\
37.26	0\\
37.27	0\\
37.28	0\\
37.29	0\\
37.3	0\\
37.31	0\\
37.32	0\\
37.33	0\\
37.34	0\\
37.35	0\\
37.36	0\\
37.37	0\\
37.38	0\\
37.39	0\\
37.4	0\\
37.41	0\\
37.42	0\\
37.43	0\\
37.44	0\\
37.45	0\\
37.46	0\\
37.47	0\\
37.48	0\\
37.49	0\\
37.5	0\\
37.51	0\\
37.52	0\\
37.53	0\\
37.54	0\\
37.55	0\\
37.56	0\\
37.57	0\\
37.58	0\\
37.59	0\\
37.6	0\\
37.61	0\\
37.62	0\\
37.63	0\\
37.64	0\\
37.65	0\\
37.66	0\\
37.67	0\\
37.68	0\\
37.69	0\\
37.7	0\\
37.71	0\\
37.72	0\\
37.73	0\\
37.74	0\\
37.75	0\\
37.76	0\\
37.77	0\\
37.78	0\\
37.79	0\\
37.8	0\\
37.81	0\\
37.82	0\\
37.83	0\\
37.84	0\\
37.85	0\\
37.86	0\\
37.87	0\\
37.88	0\\
37.89	0\\
37.9	0\\
37.91	0\\
37.92	0\\
37.93	0\\
37.94	0\\
37.95	0\\
37.96	0\\
37.97	0\\
37.98	0\\
37.99	0\\
38	0\\
38.01	0\\
38.02	0\\
38.03	0\\
38.04	0\\
38.05	0\\
38.06	0\\
38.07	0\\
38.08	0\\
38.09	0\\
38.1	0\\
38.11	0\\
38.12	0\\
38.13	0\\
38.14	0\\
38.15	0\\
38.16	0\\
38.17	0\\
38.18	0\\
38.19	0\\
38.2	0\\
38.21	0\\
38.22	0\\
38.23	0\\
38.24	0\\
38.25	0\\
38.26	0\\
38.27	0\\
38.28	0\\
38.29	0\\
38.3	0\\
38.31	0\\
38.32	0\\
38.33	0\\
38.34	0\\
38.35	0\\
38.36	0\\
38.37	0\\
38.38	0\\
38.39	0\\
38.4	0\\
38.41	0\\
38.42	0\\
38.43	0\\
38.44	0\\
38.45	0\\
38.46	0\\
38.47	0\\
38.48	0\\
38.49	0\\
38.5	0\\
38.51	0\\
38.52	0\\
38.53	0\\
38.54	0\\
38.55	0\\
38.56	0\\
38.57	0\\
38.58	0\\
38.59	0\\
38.6	0\\
38.61	0\\
38.62	0\\
38.63	0\\
38.64	0\\
38.65	0\\
38.66	0\\
38.67	0\\
38.68	0\\
38.69	0\\
38.7	0\\
38.71	0\\
38.72	0\\
38.73	0\\
38.74	0\\
38.75	0\\
38.76	0\\
38.77	0\\
38.78	0\\
38.79	0\\
38.8	0\\
38.81	0\\
38.82	0\\
38.83	0\\
38.84	0\\
38.85	0\\
38.86	0\\
38.87	0\\
38.88	0\\
38.89	0\\
38.9	0\\
38.91	0\\
38.92	0\\
38.93	0\\
38.94	0\\
38.95	0\\
38.96	0\\
38.97	0\\
38.98	0\\
38.99	0\\
39	0\\
39.01	0\\
39.02	0\\
39.03	0\\
39.04	0\\
39.05	0\\
39.06	0\\
39.07	0\\
39.08	0\\
39.09	0\\
39.1	0\\
39.11	0\\
39.12	0\\
39.13	0\\
39.14	0\\
39.15	0\\
39.16	0\\
39.17	0\\
39.18	0\\
39.19	0\\
39.2	0\\
39.21	0\\
39.22	0\\
39.23	0\\
39.24	0\\
39.25	0\\
39.26	0\\
39.27	0\\
39.28	0\\
39.29	0\\
39.3	0\\
39.31	0\\
39.32	0\\
39.33	0\\
39.34	0\\
39.35	0\\
39.36	0\\
39.37	0\\
39.38	0\\
39.39	0\\
39.4	0\\
39.41	0\\
39.42	0\\
39.43	0\\
39.44	0\\
39.45	0\\
39.46	0\\
39.47	0\\
39.48	0\\
39.49	0\\
39.5	0\\
39.51	0\\
39.52	0\\
39.53	0\\
39.54	0\\
39.55	0\\
39.56	0\\
39.57	0\\
39.58	0\\
39.59	0\\
39.6	0\\
39.61	0\\
39.62	0\\
39.63	0\\
39.64	0\\
39.65	0\\
39.66	0\\
39.67	0\\
39.68	0\\
39.69	0\\
39.7	0\\
39.71	0\\
39.72	0\\
39.73	0\\
39.74	0\\
39.75	0\\
39.76	0\\
39.77	0\\
39.78	0\\
39.79	0\\
39.8	0\\
39.81	0\\
39.82	0\\
39.83	0\\
39.84	0\\
39.85	0\\
39.86	0\\
39.87	0\\
39.88	0\\
39.89	0\\
39.9	0\\
39.91	0\\
39.92	0\\
39.93	0\\
39.94	0\\
39.95	0\\
39.96	0\\
39.97	0\\
39.98	0\\
39.99	0\\
40	0\\
40.01	0\\
};
\addplot [color=black,solid,forget plot]
  table[row sep=crcr]{%
40.01	0\\
40.02	0\\
40.03	0\\
40.04	0\\
40.05	0\\
40.06	0\\
40.07	0\\
40.08	0\\
40.09	0\\
40.1	0\\
40.11	0\\
40.12	0\\
40.13	0\\
40.14	0\\
40.15	0\\
40.16	0\\
40.17	0\\
40.18	0\\
40.19	0\\
40.2	0\\
40.21	0\\
40.22	0\\
40.23	0\\
40.24	0\\
40.25	0\\
40.26	0\\
40.27	0\\
40.28	0\\
40.29	0\\
40.3	0\\
40.31	0\\
40.32	0\\
40.33	0\\
40.34	0\\
40.35	0\\
40.36	0\\
40.37	0\\
40.38	0\\
40.39	0\\
40.4	0\\
40.41	0\\
40.42	0\\
40.43	0\\
40.44	0\\
40.45	0\\
40.46	0\\
40.47	0\\
40.48	0\\
40.49	0\\
40.5	0\\
40.51	0\\
40.52	0\\
40.53	0\\
40.54	0\\
40.55	0\\
40.56	0\\
40.57	0\\
40.58	0\\
40.59	0\\
40.6	0\\
40.61	0\\
40.62	0\\
40.63	0\\
40.64	0\\
40.65	0\\
40.66	0\\
40.67	0\\
40.68	0\\
40.69	0\\
40.7	0\\
40.71	0\\
40.72	0\\
40.73	0\\
40.74	0\\
40.75	0\\
40.76	0\\
40.77	0\\
40.78	0\\
40.79	0\\
40.8	0\\
40.81	0\\
40.82	0\\
40.83	0\\
40.84	0\\
40.85	0\\
40.86	0\\
40.87	0\\
40.88	0\\
40.89	0\\
40.9	0\\
40.91	0\\
40.92	0\\
40.93	0\\
40.94	0\\
40.95	0\\
40.96	0\\
40.97	0\\
40.98	0\\
40.99	0\\
41	0\\
41.01	0\\
41.02	0\\
41.03	0\\
41.04	0\\
41.05	0\\
41.06	0\\
41.07	0\\
41.08	0\\
41.09	0\\
41.1	0\\
41.11	0\\
41.12	0\\
41.13	0\\
41.14	0\\
41.15	0\\
41.16	0\\
41.17	0\\
41.18	0\\
41.19	0\\
41.2	0\\
41.21	0\\
41.22	0\\
41.23	0\\
41.24	0\\
41.25	0\\
41.26	0\\
41.27	0\\
41.28	0\\
41.29	0\\
41.3	0\\
41.31	0\\
41.32	0\\
41.33	0\\
41.34	0\\
41.35	0\\
41.36	0\\
41.37	0\\
41.38	0\\
41.39	0\\
41.4	0\\
41.41	0\\
41.42	0\\
41.43	0\\
41.44	0\\
41.45	0\\
41.46	0\\
41.47	0\\
41.48	0\\
41.49	0\\
41.5	0\\
41.51	0\\
41.52	0\\
41.53	0\\
41.54	0\\
41.55	0\\
41.56	0\\
41.57	0\\
41.58	0\\
41.59	0\\
41.6	0\\
41.61	0\\
41.62	0\\
41.63	0\\
41.64	0\\
41.65	0\\
41.66	0\\
41.67	0\\
41.68	0\\
41.69	0\\
41.7	0\\
41.71	0\\
41.72	0\\
41.73	0\\
41.74	0\\
41.75	0\\
41.76	0\\
41.77	0\\
41.78	0\\
41.79	0\\
41.8	0\\
41.81	0\\
41.82	0\\
41.83	0\\
41.84	0\\
41.85	0\\
41.86	0\\
41.87	0\\
41.88	0\\
41.89	0\\
41.9	0\\
41.91	0\\
41.92	0\\
41.93	0\\
41.94	0\\
41.95	0\\
41.96	0\\
41.97	0\\
41.98	0\\
41.99	0\\
42	0\\
42.01	0\\
42.02	0\\
42.03	0\\
42.04	0\\
42.05	0\\
42.06	0\\
42.07	0\\
42.08	0\\
42.09	0\\
42.1	0\\
42.11	0\\
42.12	0\\
42.13	0\\
42.14	0\\
42.15	0\\
42.16	0\\
42.17	0\\
42.18	0\\
42.19	0\\
42.2	0\\
42.21	0\\
42.22	0\\
42.23	0\\
42.24	0\\
42.25	0\\
42.26	0\\
42.27	0\\
42.28	0\\
42.29	0\\
42.3	0\\
42.31	0\\
42.32	0\\
42.33	0\\
42.34	0\\
42.35	0\\
42.36	0\\
42.37	0\\
42.38	0\\
42.39	0\\
42.4	0\\
42.41	0\\
42.42	0\\
42.43	0\\
42.44	0\\
42.45	0\\
42.46	0\\
42.47	0\\
42.48	0\\
42.49	0\\
42.5	0\\
42.51	0\\
42.52	0\\
42.53	0\\
42.54	0\\
42.55	0\\
42.56	0\\
42.57	0\\
42.58	0\\
42.59	0\\
42.6	0\\
42.61	0\\
42.62	0\\
42.63	0\\
42.64	0\\
42.65	0\\
42.66	0\\
42.67	0\\
42.68	0\\
42.69	0\\
42.7	0\\
42.71	0\\
42.72	0\\
42.73	0\\
42.74	0\\
42.75	0\\
42.76	0\\
42.77	0\\
42.78	0\\
42.79	0\\
42.8	0\\
42.81	0\\
42.82	0\\
42.83	0\\
42.84	0\\
42.85	0\\
42.86	0\\
42.87	0\\
42.88	0\\
42.89	0\\
42.9	0\\
42.91	0\\
42.92	0\\
42.93	0\\
42.94	0\\
42.95	0\\
42.96	0\\
42.97	0\\
42.98	0\\
42.99	0\\
43	0\\
43.01	0\\
43.02	0\\
43.03	0\\
43.04	0\\
43.05	0\\
43.06	0\\
43.07	0\\
43.08	0\\
43.09	0\\
43.1	0\\
43.11	0\\
43.12	0\\
43.13	0\\
43.14	0\\
43.15	0\\
43.16	0\\
43.17	0\\
43.18	0\\
43.19	0\\
43.2	0\\
43.21	0\\
43.22	0\\
43.23	0\\
43.24	0\\
43.25	0\\
43.26	0\\
43.27	0\\
43.28	0\\
43.29	0\\
43.3	0\\
43.31	0\\
43.32	0\\
43.33	0\\
43.34	0\\
43.35	0\\
43.36	0\\
43.37	0\\
43.38	0\\
43.39	0\\
43.4	0\\
43.41	0\\
43.42	0\\
43.43	0\\
43.44	0\\
43.45	0\\
43.46	0\\
43.47	0\\
43.48	0\\
43.49	0\\
43.5	0\\
43.51	0\\
43.52	0\\
43.53	0\\
43.54	0\\
43.55	0\\
43.56	0\\
43.57	0\\
43.58	0\\
43.59	0\\
43.6	0\\
43.61	0\\
43.62	0\\
43.63	0\\
43.64	0\\
43.65	0\\
43.66	0\\
43.67	0\\
43.68	0\\
43.69	0\\
43.7	0\\
43.71	0\\
43.72	0\\
43.73	0\\
43.74	0\\
43.75	0\\
43.76	0\\
43.77	0\\
43.78	0\\
43.79	0\\
43.8	0\\
43.81	0\\
43.82	0\\
43.83	0\\
43.84	0\\
43.85	0\\
43.86	0\\
43.87	0\\
43.88	0\\
43.89	0\\
43.9	0\\
43.91	0\\
43.92	0\\
43.93	0\\
43.94	0\\
43.95	0\\
43.96	0\\
43.97	0\\
43.98	0\\
43.99	0\\
44	0\\
44.01	0\\
44.02	0\\
44.03	0\\
44.04	0\\
44.05	0\\
44.06	0\\
44.07	0\\
44.08	0\\
44.09	0\\
44.1	0\\
44.11	0\\
44.12	0\\
44.13	0\\
44.14	0\\
44.15	0\\
44.16	0\\
44.17	0\\
44.18	0\\
44.19	0\\
44.2	0\\
44.21	0\\
44.22	0\\
44.23	0\\
44.24	0\\
44.25	0\\
44.26	0\\
44.27	0\\
44.28	0\\
44.29	0\\
44.3	0\\
44.31	0\\
44.32	0\\
44.33	0\\
44.34	0\\
44.35	0\\
44.36	0\\
44.37	0\\
44.38	0\\
44.39	0\\
44.4	0\\
44.41	0\\
44.42	0\\
44.43	0\\
44.44	0\\
44.45	0\\
44.46	0\\
44.47	0\\
44.48	0\\
44.49	0\\
44.5	0\\
44.51	0\\
44.52	0\\
44.53	0\\
44.54	0\\
44.55	0\\
44.56	0\\
44.57	0\\
44.58	0\\
44.59	0\\
44.6	0\\
44.61	0\\
44.62	0\\
44.63	0\\
44.64	0\\
44.65	0\\
44.66	0\\
44.67	0\\
44.68	0\\
44.69	0\\
44.7	0\\
44.71	0\\
44.72	0\\
44.73	0\\
44.74	0\\
44.75	0\\
44.76	0\\
44.77	0\\
44.78	0\\
44.79	0\\
44.8	0\\
44.81	0\\
44.82	0\\
44.83	0\\
44.84	0\\
44.85	0\\
44.86	0\\
44.87	0\\
44.88	0\\
44.89	0\\
44.9	0\\
44.91	0\\
44.92	0\\
44.93	0\\
44.94	0\\
44.95	0\\
44.96	0\\
44.97	0\\
44.98	0\\
44.99	0\\
45	0\\
45.01	0\\
45.02	0\\
45.03	0\\
45.04	0\\
45.05	0\\
45.06	0\\
45.07	0\\
45.08	0\\
45.09	0\\
45.1	0\\
45.11	0\\
45.12	0\\
45.13	0\\
45.14	0\\
45.15	0\\
45.16	0\\
45.17	0\\
45.18	0\\
45.19	0\\
45.2	0\\
45.21	0\\
45.22	0\\
45.23	0\\
45.24	0\\
45.25	0\\
45.26	0\\
45.27	0\\
45.28	0\\
45.29	0\\
45.3	0\\
45.31	0\\
45.32	0\\
45.33	0\\
45.34	0\\
45.35	0\\
45.36	0\\
45.37	0\\
45.38	0\\
45.39	0\\
45.4	0\\
45.41	0\\
45.42	0\\
45.43	0\\
45.44	0\\
45.45	0\\
45.46	0\\
45.47	0\\
45.48	0\\
45.49	0\\
45.5	0\\
45.51	0\\
45.52	0\\
45.53	0\\
45.54	0\\
45.55	0\\
45.56	0\\
45.57	0\\
45.58	0\\
45.59	0\\
45.6	0\\
45.61	0\\
45.62	0\\
45.63	0\\
45.64	0\\
45.65	0\\
45.66	0\\
45.67	0\\
45.68	0\\
45.69	0\\
45.7	0\\
45.71	0\\
45.72	0\\
45.73	0\\
45.74	0\\
45.75	0\\
45.76	0\\
45.77	0\\
45.78	0\\
45.79	0\\
45.8	0\\
45.81	0\\
45.82	0\\
45.83	0\\
45.84	0\\
45.85	0\\
45.86	0\\
45.87	0\\
45.88	0\\
45.89	0\\
45.9	0\\
45.91	0\\
45.92	0\\
45.93	0\\
45.94	0\\
45.95	0\\
45.96	0\\
45.97	0\\
45.98	0\\
45.99	0\\
46	0\\
46.01	0\\
46.02	0\\
46.03	0\\
46.04	0\\
46.05	0\\
46.06	0\\
46.07	0\\
46.08	0\\
46.09	0\\
46.1	0\\
46.11	0\\
46.12	0\\
46.13	0\\
46.14	0\\
46.15	0\\
46.16	0\\
46.17	0\\
46.18	0\\
46.19	0\\
46.2	0\\
46.21	0\\
46.22	0\\
46.23	0\\
46.24	0\\
46.25	0\\
46.26	0\\
46.27	0\\
46.28	0\\
46.29	0\\
46.3	0\\
46.31	0\\
46.32	0\\
46.33	0\\
46.34	0\\
46.35	0\\
46.36	0\\
46.37	0\\
46.38	0\\
46.39	0\\
46.4	0\\
46.41	0\\
46.42	0\\
46.43	0\\
46.44	0\\
46.45	0\\
46.46	0\\
46.47	0\\
46.48	0\\
46.49	0\\
46.5	0\\
46.51	0\\
46.52	0\\
46.53	0\\
46.54	0\\
46.55	0\\
46.56	0\\
46.57	0\\
46.58	0\\
46.59	0\\
46.6	0\\
46.61	0\\
46.62	0\\
46.63	0\\
46.64	0\\
46.65	0\\
46.66	0\\
46.67	0\\
46.68	0\\
46.69	0\\
46.7	0\\
46.71	0\\
46.72	0\\
46.73	0\\
46.74	0\\
46.75	0\\
46.76	0\\
46.77	0\\
46.78	0\\
46.79	0\\
46.8	0\\
46.81	0\\
46.82	0\\
46.83	0\\
46.84	0\\
46.85	0\\
46.86	0\\
46.87	0\\
46.88	0\\
46.89	0\\
46.9	0\\
46.91	0\\
46.92	0\\
46.93	0\\
46.94	0\\
46.95	0\\
46.96	0\\
46.97	0\\
46.98	0\\
46.99	0\\
47	0\\
47.01	0\\
47.02	0\\
47.03	0\\
47.04	0\\
47.05	0\\
47.06	0\\
47.07	0\\
47.08	0\\
47.09	0\\
47.1	0\\
47.11	0\\
47.12	0\\
47.13	0\\
47.14	0\\
47.15	0\\
47.16	0\\
47.17	0\\
47.18	0\\
47.19	0\\
47.2	0\\
47.21	0\\
47.22	0\\
47.23	0\\
47.24	0\\
47.25	0\\
47.26	0\\
47.27	0\\
47.28	0\\
47.29	0\\
47.3	0\\
47.31	0\\
47.32	0\\
47.33	0\\
47.34	0\\
47.35	0\\
47.36	0\\
47.37	0\\
47.38	0\\
47.39	0\\
47.4	0\\
47.41	0\\
47.42	0\\
47.43	0\\
47.44	0\\
47.45	0\\
47.46	0\\
47.47	0\\
47.48	0\\
47.49	0\\
47.5	0\\
47.51	0\\
47.52	0\\
47.53	0\\
47.54	0\\
47.55	0\\
47.56	0\\
47.57	0\\
47.58	0\\
47.59	0\\
47.6	0\\
47.61	0\\
47.62	0\\
47.63	0\\
47.64	0\\
47.65	0\\
47.66	0\\
47.67	0\\
47.68	0\\
47.69	0\\
47.7	0\\
47.71	0\\
47.72	0\\
47.73	0\\
47.74	0\\
47.75	0\\
47.76	0\\
47.77	0\\
47.78	0\\
47.79	0\\
47.8	0\\
47.81	0\\
47.82	0\\
47.83	0\\
47.84	0\\
47.85	0\\
47.86	0\\
47.87	0\\
47.88	0\\
47.89	0\\
47.9	0\\
47.91	0\\
47.92	0\\
47.93	0\\
47.94	0\\
47.95	0\\
47.96	0\\
47.97	0\\
47.98	0\\
47.99	0\\
48	0\\
48.01	0\\
48.02	0\\
48.03	0\\
48.04	0\\
48.05	0\\
48.06	0\\
48.07	0\\
48.08	0\\
48.09	0\\
48.1	0\\
48.11	0\\
48.12	0\\
48.13	0\\
48.14	0\\
48.15	0\\
48.16	0\\
48.17	0\\
48.18	0\\
48.19	0\\
48.2	0\\
48.21	0\\
48.22	0\\
48.23	0\\
48.24	0\\
48.25	0\\
48.26	0\\
48.27	0\\
48.28	0\\
48.29	0\\
48.3	0\\
48.31	0\\
48.32	0\\
48.33	0\\
48.34	0\\
48.35	0\\
48.36	0\\
48.37	0\\
48.38	0\\
48.39	0\\
48.4	0\\
48.41	0\\
48.42	0\\
48.43	0\\
48.44	0\\
48.45	0\\
48.46	0\\
48.47	0\\
48.48	0\\
48.49	0\\
48.5	0\\
48.51	0\\
48.52	0\\
48.53	0\\
48.54	0\\
48.55	0\\
48.56	0\\
48.57	0\\
48.58	0\\
48.59	0\\
48.6	0\\
48.61	0\\
48.62	0\\
48.63	0\\
48.64	0\\
48.65	0\\
48.66	0\\
48.67	0\\
48.68	0\\
48.69	0\\
48.7	0\\
48.71	0\\
48.72	0\\
48.73	0\\
48.74	0\\
48.75	0\\
48.76	0\\
48.77	0\\
48.78	0\\
48.79	0\\
48.8	0\\
48.81	0\\
48.82	0\\
48.83	0\\
48.84	0\\
48.85	0\\
48.86	0\\
48.87	0\\
48.88	0\\
48.89	0\\
48.9	0\\
48.91	0\\
48.92	0\\
48.93	0\\
48.94	0\\
48.95	0\\
48.96	0\\
48.97	0\\
48.98	0\\
48.99	0\\
49	0\\
49.01	0\\
49.02	0\\
49.03	0\\
49.04	0\\
49.05	0\\
49.06	0\\
49.07	0\\
49.08	0\\
49.09	0\\
49.1	0\\
49.11	0\\
49.12	0\\
49.13	0\\
49.14	0\\
49.15	0\\
49.16	0\\
49.17	0\\
49.18	0\\
49.19	0\\
49.2	0\\
49.21	0\\
49.22	0\\
49.23	0\\
49.24	0\\
49.25	0\\
49.26	0\\
49.27	0\\
49.28	0\\
49.29	0\\
49.3	0\\
49.31	0\\
49.32	0\\
49.33	0\\
49.34	0\\
49.35	0\\
49.36	0\\
49.37	0\\
49.38	0\\
49.39	0\\
49.4	0\\
49.41	0\\
49.42	0\\
49.43	0\\
49.44	0\\
49.45	0\\
49.46	0\\
49.47	0\\
49.48	0\\
49.49	0\\
49.5	0\\
49.51	0\\
49.52	0\\
49.53	0\\
49.54	0\\
49.55	0\\
49.56	0\\
49.57	0\\
49.58	0\\
49.59	0\\
49.6	0\\
49.61	0\\
49.62	0\\
49.63	0\\
49.64	0\\
49.65	0\\
49.66	0\\
49.67	0\\
49.68	0\\
49.69	0\\
49.7	0\\
49.71	0\\
49.72	0\\
49.73	0\\
49.74	0\\
49.75	0\\
49.76	0\\
49.77	0\\
49.78	0\\
49.79	0\\
49.8	0\\
49.81	0\\
49.82	0\\
49.83	0\\
49.84	0\\
49.85	0\\
49.86	0\\
49.87	0\\
49.88	0\\
49.89	0\\
49.9	0\\
49.91	0\\
49.92	0\\
49.93	0\\
49.94	0\\
49.95	0\\
49.96	0\\
49.97	0\\
49.98	0\\
49.99	0\\
50	0\\
50.01	0\\
50.02	0\\
50.03	0\\
50.04	0\\
50.05	0\\
50.06	0\\
50.07	0\\
50.08	0\\
50.09	0\\
50.1	0\\
50.11	0\\
50.12	0\\
50.13	0\\
50.14	0\\
50.15	0\\
50.16	0\\
50.17	0\\
50.18	0\\
50.19	0\\
50.2	0\\
50.21	0\\
50.22	0\\
50.23	0\\
50.24	0\\
50.25	0\\
50.26	0\\
50.27	0\\
50.28	0\\
50.29	0\\
50.3	0\\
50.31	0\\
50.32	0\\
50.33	0\\
50.34	0\\
50.35	0\\
50.36	0\\
50.37	0\\
50.38	0\\
50.39	0\\
50.4	0\\
50.41	0\\
50.42	0\\
50.43	0\\
50.44	0\\
50.45	0\\
50.46	0\\
50.47	0\\
50.48	0\\
50.49	0\\
50.5	0\\
50.51	0\\
50.52	0\\
50.53	0\\
50.54	0\\
50.55	0\\
50.56	0\\
50.57	0\\
50.58	0\\
50.59	0\\
50.6	0\\
50.61	0\\
50.62	0\\
50.63	0\\
50.64	0\\
50.65	0\\
50.66	0\\
50.67	0\\
50.68	0\\
50.69	0\\
50.7	0\\
50.71	0\\
50.72	0\\
50.73	0\\
50.74	0\\
50.75	0\\
50.76	0\\
50.77	0\\
50.78	0\\
50.79	0\\
50.8	0\\
50.81	0\\
50.82	0\\
50.83	0\\
50.84	0\\
50.85	0\\
50.86	0\\
50.87	0\\
50.88	0\\
50.89	0\\
50.9	0\\
50.91	0\\
50.92	0\\
50.93	0\\
50.94	0\\
50.95	0\\
50.96	0\\
50.97	0\\
50.98	0\\
50.99	0\\
51	0\\
51.01	0\\
51.02	0\\
51.03	0\\
51.04	0\\
51.05	0\\
51.06	0\\
51.07	0\\
51.08	0\\
51.09	0\\
51.1	0\\
51.11	0\\
51.12	0\\
51.13	0\\
51.14	0\\
51.15	0\\
51.16	0\\
51.17	0\\
51.18	0\\
51.19	0\\
51.2	0\\
51.21	0\\
51.22	0\\
51.23	0\\
51.24	0\\
51.25	0\\
51.26	0\\
51.27	0\\
51.28	0\\
51.29	0\\
51.3	0\\
51.31	0\\
51.32	0\\
51.33	0\\
51.34	0\\
51.35	0\\
51.36	0\\
51.37	0\\
51.38	0\\
51.39	0\\
51.4	0\\
51.41	0\\
51.42	0\\
51.43	0\\
51.44	0\\
51.45	0\\
51.46	0\\
51.47	0\\
51.48	0\\
51.49	0\\
51.5	0\\
51.51	0\\
51.52	0\\
51.53	0\\
51.54	0\\
51.55	0\\
51.56	0\\
51.57	0\\
51.58	0\\
51.59	0\\
51.6	0\\
51.61	0\\
51.62	0\\
51.63	0\\
51.64	0\\
51.65	0\\
51.66	0\\
51.67	0\\
51.68	0\\
51.69	0\\
51.7	0\\
51.71	0\\
51.72	0\\
51.73	0\\
51.74	0\\
51.75	0\\
51.76	0\\
51.77	0\\
51.78	0\\
51.79	0\\
51.8	0\\
51.81	0\\
51.82	0\\
51.83	0\\
51.84	0\\
51.85	0\\
51.86	0\\
51.87	0\\
51.88	0\\
51.89	0\\
51.9	0\\
51.91	0\\
51.92	0\\
51.93	0\\
51.94	0\\
51.95	0\\
51.96	0\\
51.97	0\\
51.98	0\\
51.99	0\\
52	0\\
52.01	0\\
52.02	0\\
52.03	0\\
52.04	0\\
52.05	0\\
52.06	0\\
52.07	0\\
52.08	0\\
52.09	0\\
52.1	0\\
52.11	0\\
52.12	0\\
52.13	0\\
52.14	0\\
52.15	0\\
52.16	0\\
52.17	0\\
52.18	0\\
52.19	0\\
52.2	0\\
52.21	0\\
52.22	0\\
52.23	0\\
52.24	0\\
52.25	0\\
52.26	0\\
52.27	0\\
52.28	0\\
52.29	0\\
52.3	0\\
52.31	0\\
52.32	0\\
52.33	0\\
52.34	0\\
52.35	0\\
52.36	0\\
52.37	0\\
52.38	0\\
52.39	0\\
52.4	0\\
52.41	0\\
52.42	0\\
52.43	0\\
52.44	0\\
52.45	0\\
52.46	0\\
52.47	0\\
52.48	0\\
52.49	0\\
52.5	0\\
52.51	0\\
52.52	0\\
52.53	0\\
52.54	0\\
52.55	0\\
52.56	0\\
52.57	0\\
52.58	0\\
52.59	0\\
52.6	0\\
52.61	0\\
52.62	0\\
52.63	0\\
52.64	0\\
52.65	0\\
52.66	0\\
52.67	0\\
52.68	0\\
52.69	0\\
52.7	0\\
52.71	0\\
52.72	0\\
52.73	0\\
52.74	0\\
52.75	0\\
52.76	0\\
52.77	0\\
52.78	0\\
52.79	0\\
52.8	0\\
52.81	0\\
52.82	0\\
52.83	0\\
52.84	0\\
52.85	0\\
52.86	0\\
52.87	0\\
52.88	0\\
52.89	0\\
52.9	0\\
52.91	0\\
52.92	0\\
52.93	0\\
52.94	0\\
52.95	0\\
52.96	0\\
52.97	0\\
52.98	0\\
52.99	0\\
53	0\\
53.01	0\\
53.02	0\\
53.03	0\\
53.04	0\\
53.05	0\\
53.06	0\\
53.07	0\\
53.08	0\\
53.09	0\\
53.1	0\\
53.11	0\\
53.12	0\\
53.13	0\\
53.14	0\\
53.15	0\\
53.16	0\\
53.17	0\\
53.18	0\\
53.19	0\\
53.2	0\\
53.21	0\\
53.22	0\\
53.23	0\\
53.24	0\\
53.25	0\\
53.26	0\\
53.27	0\\
53.28	0\\
53.29	0\\
53.3	0\\
53.31	0\\
53.32	0\\
53.33	0\\
53.34	0\\
53.35	0\\
53.36	0\\
53.37	0\\
53.38	0\\
53.39	0\\
53.4	0\\
53.41	0\\
53.42	0\\
53.43	0\\
53.44	0\\
53.45	0\\
53.46	0\\
53.47	0\\
53.48	0\\
53.49	0\\
53.5	0\\
53.51	0\\
53.52	0\\
53.53	0\\
53.54	0\\
53.55	0\\
53.56	0\\
53.57	0\\
53.58	0\\
53.59	0\\
53.6	0\\
53.61	0\\
53.62	0\\
53.63	0\\
53.64	0\\
53.65	0\\
53.66	0\\
53.67	0\\
53.68	0\\
53.69	0\\
53.7	0\\
53.71	0\\
53.72	0\\
53.73	0\\
53.74	0\\
53.75	0\\
53.76	0\\
53.77	0\\
53.78	0\\
53.79	0\\
53.8	0\\
53.81	0\\
53.82	0\\
53.83	0\\
53.84	0\\
53.85	0\\
53.86	0\\
53.87	0\\
53.88	0\\
53.89	0\\
53.9	0\\
53.91	0\\
53.92	0\\
53.93	0\\
53.94	0\\
53.95	0\\
53.96	0\\
53.97	0\\
53.98	0\\
53.99	0\\
54	0\\
54.01	0\\
54.02	0\\
54.03	0\\
54.04	0\\
54.05	0\\
54.06	0\\
54.07	0\\
54.08	0\\
54.09	0\\
54.1	0\\
54.11	0\\
54.12	0\\
54.13	0\\
54.14	0\\
54.15	0\\
54.16	0\\
54.17	0\\
54.18	0\\
54.19	0\\
54.2	0\\
54.21	0\\
54.22	0\\
54.23	0\\
54.24	0\\
54.25	0\\
54.26	0\\
54.27	0\\
54.28	0\\
54.29	0\\
54.3	0\\
54.31	0\\
54.32	0\\
54.33	0\\
54.34	0\\
54.35	0\\
54.36	0\\
54.37	0\\
54.38	0\\
54.39	0\\
54.4	0\\
54.41	0\\
54.42	0\\
54.43	0\\
54.44	0\\
54.45	0\\
54.46	0\\
54.47	0\\
54.48	0\\
54.49	0\\
54.5	0\\
54.51	0\\
54.52	0\\
54.53	0\\
54.54	0\\
54.55	0\\
54.56	0\\
54.57	0\\
54.58	0\\
54.59	0\\
54.6	0\\
54.61	0\\
54.62	0\\
54.63	0\\
54.64	0\\
54.65	0\\
54.66	0\\
54.67	0\\
54.68	0\\
54.69	0\\
54.7	0\\
54.71	0\\
54.72	0\\
54.73	0\\
54.74	0\\
54.75	0\\
54.76	0\\
54.77	0\\
54.78	0\\
54.79	0\\
54.8	0\\
54.81	0\\
54.82	0\\
54.83	0\\
54.84	0\\
54.85	0\\
54.86	0\\
54.87	0\\
54.88	0\\
54.89	0\\
54.9	0\\
54.91	0\\
54.92	0\\
54.93	0\\
54.94	0\\
54.95	0\\
54.96	0\\
54.97	0\\
54.98	0\\
54.99	0\\
55	0\\
55.01	0\\
55.02	0\\
55.03	0\\
55.04	0\\
55.05	0\\
55.06	0\\
55.07	0\\
55.08	0\\
55.09	0\\
55.1	0\\
55.11	0\\
55.12	0\\
55.13	0\\
55.14	0\\
55.15	0\\
55.16	0\\
55.17	0\\
55.18	0\\
55.19	0\\
55.2	0\\
55.21	0\\
55.22	0\\
55.23	0\\
55.24	0\\
55.25	0\\
55.26	0\\
55.27	0\\
55.28	0\\
55.29	0\\
55.3	0\\
55.31	0\\
55.32	0\\
55.33	0\\
55.34	0\\
55.35	0\\
55.36	0\\
55.37	0\\
55.38	0\\
55.39	0\\
55.4	0\\
55.41	0\\
55.42	0\\
55.43	0\\
55.44	0\\
55.45	0\\
55.46	0\\
55.47	0\\
55.48	0\\
55.49	0\\
55.5	0\\
55.51	0\\
55.52	0\\
55.53	0\\
55.54	0\\
55.55	0\\
55.56	0\\
55.57	0\\
55.58	0\\
55.59	0\\
55.6	0\\
55.61	0\\
55.62	0\\
55.63	0\\
55.64	0\\
55.65	0\\
55.66	0\\
55.67	0\\
55.68	0\\
55.69	0\\
55.7	0\\
55.71	0\\
55.72	0\\
55.73	0\\
55.74	0\\
55.75	0\\
55.76	0\\
55.77	0\\
55.78	0\\
55.79	0\\
55.8	0\\
55.81	0\\
55.82	0\\
55.83	0\\
55.84	0\\
55.85	0\\
55.86	0\\
55.87	0\\
55.88	0\\
55.89	0\\
55.9	0\\
55.91	0\\
55.92	0\\
55.93	0\\
55.94	0\\
55.95	0\\
55.96	0\\
55.97	0\\
55.98	0\\
55.99	0\\
56	0\\
56.01	0\\
56.02	0\\
56.03	0\\
56.04	0\\
56.05	0\\
56.06	0\\
56.07	0\\
56.08	0\\
56.09	0\\
56.1	0\\
56.11	0\\
56.12	0\\
56.13	0\\
56.14	0\\
56.15	0\\
56.16	0\\
56.17	0\\
56.18	0\\
56.19	0\\
56.2	0\\
56.21	0\\
56.22	0\\
56.23	0\\
56.24	0\\
56.25	0\\
56.26	0\\
56.27	0\\
56.28	0\\
56.29	0\\
56.3	0\\
56.31	0\\
56.32	0\\
56.33	0\\
56.34	0\\
56.35	0\\
56.36	0\\
56.37	0\\
56.38	0\\
56.39	0\\
56.4	0\\
56.41	0\\
56.42	0\\
56.43	0\\
56.44	0\\
56.45	0\\
56.46	0\\
56.47	0\\
56.48	0\\
56.49	0\\
56.5	0\\
56.51	0\\
56.52	0\\
56.53	0\\
56.54	0\\
56.55	0\\
56.56	0\\
56.57	0\\
56.58	0\\
56.59	0\\
56.6	0\\
56.61	0\\
56.62	0\\
56.63	0\\
56.64	0\\
56.65	0\\
56.66	0\\
56.67	0\\
56.68	0\\
56.69	0\\
56.7	0\\
56.71	0\\
56.72	0\\
56.73	0\\
56.74	0\\
56.75	0\\
56.76	0\\
56.77	0\\
56.78	0\\
56.79	0\\
56.8	0\\
56.81	0\\
56.82	0\\
56.83	0\\
56.84	0\\
56.85	0\\
56.86	0\\
56.87	0\\
56.88	0\\
56.89	0\\
56.9	0\\
56.91	0\\
56.92	0\\
56.93	0\\
56.94	0\\
56.95	0\\
56.96	0\\
56.97	0\\
56.98	0\\
56.99	0\\
57	0\\
57.01	0\\
57.02	0\\
57.03	0\\
57.04	0\\
57.05	0\\
57.06	0\\
57.07	0\\
57.08	0\\
57.09	0\\
57.1	0\\
57.11	0\\
57.12	0\\
57.13	0\\
57.14	0\\
57.15	0\\
57.16	0\\
57.17	0\\
57.18	0\\
57.19	0\\
57.2	0\\
57.21	0\\
57.22	0\\
57.23	0\\
57.24	0\\
57.25	0\\
57.26	0\\
57.27	0\\
57.28	0\\
57.29	0\\
57.3	0\\
57.31	0\\
57.32	0\\
57.33	0\\
57.34	0\\
57.35	0\\
57.36	0\\
57.37	0\\
57.38	0\\
57.39	0\\
57.4	0\\
57.41	0\\
57.42	0\\
57.43	0\\
57.44	0\\
57.45	0\\
57.46	0\\
57.47	0\\
57.48	0\\
57.49	0\\
57.5	0\\
57.51	0\\
57.52	0\\
57.53	0\\
57.54	0\\
57.55	0\\
57.56	0\\
57.57	0\\
57.58	0\\
57.59	0\\
57.6	0\\
57.61	0\\
57.62	0\\
57.63	0\\
57.64	0\\
57.65	0\\
57.66	0\\
57.67	0\\
57.68	0\\
57.69	0\\
57.7	0\\
57.71	0\\
57.72	0\\
57.73	0\\
57.74	0\\
57.75	0\\
57.76	0\\
57.77	0\\
57.78	0\\
57.79	0\\
57.8	0\\
57.81	0\\
57.82	0\\
57.83	0\\
57.84	0\\
57.85	0\\
57.86	0\\
57.87	0\\
57.88	0\\
57.89	0\\
57.9	0\\
57.91	0\\
57.92	0\\
57.93	0\\
57.94	0\\
57.95	0\\
57.96	0\\
57.97	0\\
57.98	0\\
57.99	0\\
58	0\\
58.01	0\\
58.02	0\\
58.03	0\\
58.04	0\\
58.05	0\\
58.06	0\\
58.07	0\\
58.08	0\\
58.09	0\\
58.1	0\\
58.11	0\\
58.12	0\\
58.13	0\\
58.14	0\\
58.15	0\\
58.16	0\\
58.17	0\\
58.18	0\\
58.19	0\\
58.2	0\\
58.21	0\\
58.22	0\\
58.23	0\\
58.24	0\\
58.25	0\\
58.26	0\\
58.27	0\\
58.28	0\\
58.29	0\\
58.3	0\\
58.31	0\\
58.32	0\\
58.33	0\\
58.34	0\\
58.35	0\\
58.36	0\\
58.37	0\\
58.38	0\\
58.39	0\\
58.4	0\\
58.41	0\\
58.42	0\\
58.43	0\\
58.44	0\\
58.45	0\\
58.46	0\\
58.47	0\\
58.48	0\\
58.49	0\\
58.5	0\\
58.51	0\\
58.52	0\\
58.53	0\\
58.54	0\\
58.55	0\\
58.56	0\\
58.57	0\\
58.58	0\\
58.59	0\\
58.6	0\\
58.61	0\\
58.62	0\\
58.63	0\\
58.64	0\\
58.65	0\\
58.66	0\\
58.67	0\\
58.68	0\\
58.69	0\\
58.7	0\\
58.71	0\\
58.72	0\\
58.73	0\\
58.74	0\\
58.75	0\\
58.76	0\\
58.77	0\\
58.78	0\\
58.79	0\\
58.8	0\\
58.81	0\\
58.82	0\\
58.83	0\\
58.84	0\\
58.85	0\\
58.86	0\\
58.87	0\\
58.88	0\\
58.89	0\\
58.9	0\\
58.91	0\\
58.92	0\\
58.93	0\\
58.94	0\\
58.95	0\\
58.96	0\\
58.97	0\\
58.98	0\\
58.99	0\\
59	0\\
59.01	0\\
59.02	0\\
59.03	0\\
59.04	0\\
59.05	0\\
59.06	0\\
59.07	0\\
59.08	0\\
59.09	0\\
59.1	0\\
59.11	0\\
59.12	0\\
59.13	0\\
59.14	0\\
59.15	0\\
59.16	0\\
59.17	0\\
59.18	0\\
59.19	0\\
59.2	0\\
59.21	0\\
59.22	0\\
59.23	0\\
59.24	0\\
59.25	0\\
59.26	0\\
59.27	0\\
59.28	0\\
59.29	0\\
59.3	0\\
59.31	0\\
59.32	0\\
59.33	0\\
59.34	0\\
59.35	0\\
59.36	0\\
59.37	0\\
59.38	0\\
59.39	0\\
59.4	0\\
59.41	0\\
59.42	0\\
59.43	0\\
59.44	0\\
59.45	0\\
59.46	0\\
59.47	0\\
59.48	0\\
59.49	0\\
59.5	0\\
59.51	0\\
59.52	0\\
59.53	0\\
59.54	0\\
59.55	0\\
59.56	0\\
59.57	0\\
59.58	0\\
59.59	0\\
59.6	0\\
59.61	0\\
59.62	0\\
59.63	0\\
59.64	0\\
59.65	0\\
59.66	0\\
59.67	0\\
59.68	0\\
59.69	0\\
59.7	0\\
59.71	0\\
59.72	0\\
59.73	0\\
59.74	0\\
59.75	0\\
59.76	0\\
59.77	0\\
59.78	0\\
59.79	0\\
59.8	0\\
59.81	0\\
59.82	0\\
59.83	0\\
59.84	0\\
59.85	0\\
59.86	0\\
59.87	0\\
59.88	0\\
59.89	0\\
59.9	0\\
59.91	0\\
59.92	0\\
59.93	0\\
59.94	0\\
59.95	0\\
59.96	0\\
59.97	0\\
59.98	0\\
59.99	0\\
60	0\\
60.01	0\\
60.02	0\\
60.03	0\\
60.04	0\\
60.05	0\\
60.06	0\\
60.07	0\\
60.08	0\\
60.09	0\\
60.1	0\\
60.11	0\\
60.12	0\\
60.13	0\\
60.14	0\\
60.15	0\\
60.16	0\\
60.17	0\\
60.18	0\\
60.19	0\\
60.2	0\\
60.21	0\\
60.22	0\\
60.23	0\\
60.24	0\\
60.25	0\\
60.26	0\\
60.27	0\\
60.28	0\\
60.29	0\\
60.3	0\\
60.31	0\\
60.32	0\\
60.33	0\\
60.34	0\\
60.35	0\\
60.36	0\\
60.37	0\\
60.38	0\\
60.39	0\\
60.4	0\\
60.41	0\\
60.42	0\\
60.43	0\\
60.44	0\\
60.45	0\\
60.46	0\\
60.47	0\\
60.48	0\\
60.49	0\\
60.5	0\\
60.51	0\\
60.52	0\\
60.53	0\\
60.54	0\\
60.55	0\\
60.56	0\\
60.57	0\\
60.58	0\\
60.59	0\\
60.6	0\\
60.61	0\\
60.62	0\\
60.63	0\\
60.64	0\\
60.65	0\\
60.66	0\\
60.67	0\\
60.68	0\\
60.69	0\\
60.7	0\\
60.71	0\\
60.72	0\\
60.73	0\\
60.74	0\\
60.75	0\\
60.76	0\\
60.77	0\\
60.78	0\\
60.79	0\\
60.8	0\\
60.81	0\\
60.82	0\\
60.83	0\\
60.84	0\\
60.85	0\\
60.86	0\\
60.87	0\\
60.88	0\\
60.89	0\\
60.9	0\\
60.91	0\\
60.92	0\\
60.93	0\\
60.94	0\\
60.95	0\\
60.96	0\\
60.97	0\\
60.98	0\\
60.99	0\\
61	0\\
61.01	0\\
61.02	0\\
61.03	0\\
61.04	0\\
61.05	0\\
61.06	0\\
61.07	0\\
61.08	0\\
61.09	0\\
61.1	0\\
61.11	0\\
61.12	0\\
61.13	0\\
61.14	0\\
61.15	0\\
61.16	0\\
61.17	0\\
61.18	0\\
61.19	0\\
61.2	0\\
61.21	0\\
61.22	0\\
61.23	0\\
61.24	0\\
61.25	0\\
61.26	0\\
61.27	0\\
61.28	0\\
61.29	0\\
61.3	0\\
61.31	0\\
61.32	0\\
61.33	0\\
61.34	0\\
61.35	0\\
61.36	0\\
61.37	0\\
61.38	0\\
61.39	0\\
61.4	0\\
61.41	0\\
61.42	0\\
61.43	0\\
61.44	0\\
61.45	0\\
61.46	0\\
61.47	0\\
61.48	0\\
61.49	0\\
61.5	0\\
61.51	0\\
61.52	0\\
61.53	0\\
61.54	0\\
61.55	0\\
61.56	0\\
61.57	0\\
61.58	0\\
61.59	0\\
61.6	0\\
61.61	0\\
61.62	0\\
61.63	0\\
61.64	0\\
61.65	0\\
61.66	0\\
61.67	0\\
61.68	0\\
61.69	0\\
61.7	0\\
61.71	0\\
61.72	0\\
61.73	0\\
61.74	0\\
61.75	0\\
61.76	0\\
61.77	0\\
61.78	0\\
61.79	0\\
61.8	0\\
61.81	0\\
61.82	0\\
61.83	0\\
61.84	0\\
61.85	0\\
61.86	0\\
61.87	0\\
61.88	0\\
61.89	0\\
61.9	0\\
61.91	0\\
61.92	0\\
61.93	0\\
61.94	0\\
61.95	0\\
61.96	0\\
61.97	0\\
61.98	0\\
61.99	0\\
62	0\\
62.01	0\\
62.02	0\\
62.03	0\\
62.04	0\\
62.05	0\\
62.06	0\\
62.07	0\\
62.08	0\\
62.09	0\\
62.1	0\\
62.11	0\\
62.12	0\\
62.13	0\\
62.14	0\\
62.15	0\\
62.16	0\\
62.17	0\\
62.18	0\\
62.19	0\\
62.2	0\\
62.21	0\\
62.22	0\\
62.23	0\\
62.24	0\\
62.25	0\\
62.26	0\\
62.27	0\\
62.28	0\\
62.29	0\\
62.3	0\\
62.31	0\\
62.32	0\\
62.33	0\\
62.34	0\\
62.35	0\\
62.36	0\\
62.37	0\\
62.38	0\\
62.39	0\\
62.4	0\\
62.41	0\\
62.42	0\\
62.43	0\\
62.44	0\\
62.45	0\\
62.46	0\\
62.47	0\\
62.48	0\\
62.49	0\\
62.5	0\\
62.51	0\\
62.52	0\\
62.53	0\\
62.54	0\\
62.55	0\\
62.56	0\\
62.57	0\\
62.58	0\\
62.59	0\\
62.6	0\\
62.61	0\\
62.62	0\\
62.63	0\\
62.64	0\\
62.65	0\\
62.66	0\\
62.67	0\\
62.68	0\\
62.69	0\\
62.7	0\\
62.71	0\\
62.72	0\\
62.73	0\\
62.74	0\\
62.75	0\\
62.76	0\\
62.77	0\\
62.78	0\\
62.79	0\\
62.8	0\\
62.81	0\\
62.82	0\\
62.83	0\\
62.84	0\\
62.85	0\\
62.86	0\\
62.87	0\\
62.88	0\\
62.89	0\\
62.9	0\\
62.91	0\\
62.92	0\\
62.93	0\\
62.94	0\\
62.95	0\\
62.96	0\\
62.97	0\\
62.98	0\\
62.99	0\\
63	0\\
63.01	0\\
63.02	0\\
63.03	0\\
63.04	0\\
63.05	0\\
63.06	0\\
63.07	0\\
63.08	0\\
63.09	0\\
63.1	0\\
63.11	0\\
63.12	0\\
63.13	0\\
63.14	0\\
63.15	0\\
63.16	0\\
63.17	0\\
63.18	0\\
63.19	0\\
63.2	0\\
63.21	0\\
63.22	0\\
63.23	0\\
63.24	0\\
63.25	0\\
63.26	0\\
63.27	0\\
63.28	0\\
63.29	0\\
63.3	0\\
63.31	0\\
63.32	0\\
63.33	0\\
63.34	0\\
63.35	0\\
63.36	0\\
63.37	0\\
63.38	0\\
63.39	0\\
63.4	0\\
63.41	0\\
63.42	0\\
63.43	0\\
63.44	0\\
63.45	0\\
63.46	0\\
63.47	0\\
63.48	0\\
63.49	0\\
63.5	0\\
63.51	0\\
63.52	0\\
63.53	0\\
63.54	0\\
63.55	0\\
63.56	0\\
63.57	0\\
63.58	0\\
63.59	0\\
63.6	0\\
63.61	0\\
63.62	0\\
63.63	0\\
63.64	0\\
63.65	0\\
63.66	0\\
63.67	0\\
63.68	0\\
63.69	0\\
63.7	0\\
63.71	0\\
63.72	0\\
63.73	0\\
63.74	0\\
63.75	0\\
63.76	0\\
63.77	0\\
63.78	0\\
63.79	0\\
63.8	0\\
63.81	0\\
63.82	0\\
63.83	0\\
63.84	0\\
63.85	0\\
63.86	0\\
63.87	0\\
63.88	0\\
63.89	0\\
63.9	0\\
63.91	0\\
63.92	0\\
63.93	0\\
63.94	0\\
63.95	0\\
63.96	0\\
63.97	0\\
63.98	0\\
63.99	0\\
64	0\\
64.01	0\\
64.02	0\\
64.03	0\\
64.04	0\\
64.05	0\\
64.06	0\\
64.07	0\\
64.08	0\\
64.09	0\\
64.1	0\\
64.11	0\\
64.12	0\\
64.13	0\\
64.14	0\\
64.15	0\\
64.16	0\\
64.17	0\\
64.18	0\\
64.19	0\\
64.2	0\\
64.21	0\\
64.22	0\\
64.23	0\\
64.24	0\\
64.25	0\\
64.26	0\\
64.27	0\\
64.28	0\\
64.29	0\\
64.3	0\\
64.31	0\\
64.32	0\\
64.33	0\\
64.34	0\\
64.35	0\\
64.36	0\\
64.37	0\\
64.38	0\\
64.39	0\\
64.4	0\\
64.41	0\\
64.42	0\\
64.43	0\\
64.44	0\\
64.45	0\\
64.46	0\\
64.47	0\\
64.48	0\\
64.49	0\\
64.5	0\\
64.51	0\\
64.52	0\\
64.53	0\\
64.54	0\\
64.55	0\\
64.56	0\\
64.57	0\\
64.58	0\\
64.59	0\\
64.6	0\\
64.61	0\\
64.62	0\\
64.63	0\\
64.64	0\\
64.65	0\\
64.66	0\\
64.67	0\\
64.68	0\\
64.69	0\\
64.7	0\\
64.71	0\\
64.72	0\\
64.73	0\\
64.74	0\\
64.75	0\\
64.76	0\\
64.77	0\\
64.78	0\\
64.79	0\\
64.8	0\\
64.81	0\\
64.82	0\\
64.83	0\\
64.84	0\\
64.85	0\\
64.86	0\\
64.87	0\\
64.88	0\\
64.89	0\\
64.9	0\\
64.91	0\\
64.92	0\\
64.93	0\\
64.94	0\\
64.95	0\\
64.96	0\\
64.97	0\\
64.98	0\\
64.99	0\\
65	0\\
65.01	0\\
65.02	0\\
65.03	0\\
65.04	0\\
65.05	0\\
65.06	0\\
65.07	0\\
65.08	0\\
65.09	0\\
65.1	0\\
65.11	0\\
65.12	0\\
65.13	0\\
65.14	0\\
65.15	0\\
65.16	0\\
65.17	0\\
65.18	0\\
65.19	0\\
65.2	0\\
65.21	0\\
65.22	0\\
65.23	0\\
65.24	0\\
65.25	0\\
65.26	0\\
65.27	0\\
65.28	0\\
65.29	0\\
65.3	0\\
65.31	0\\
65.32	0\\
65.33	0\\
65.34	0\\
65.35	0\\
65.36	0\\
65.37	0\\
65.38	0\\
65.39	0\\
65.4	0\\
65.41	0\\
65.42	0\\
65.43	0\\
65.44	0\\
65.45	0\\
65.46	0\\
65.47	0\\
65.48	0\\
65.49	0\\
65.5	0\\
65.51	0\\
65.52	0\\
65.53	0\\
65.54	0\\
65.55	0\\
65.56	0\\
65.57	0\\
65.58	0\\
65.59	0\\
65.6	0\\
65.61	0\\
65.62	0\\
65.63	0\\
65.64	0\\
65.65	0\\
65.66	0\\
65.67	0\\
65.68	0\\
65.69	0\\
65.7	0\\
65.71	0\\
65.72	0\\
65.73	0\\
65.74	0\\
65.75	0\\
65.76	0\\
65.77	0\\
65.78	0\\
65.79	0\\
65.8	0\\
65.81	0\\
65.82	0\\
65.83	0\\
65.84	0\\
65.85	0\\
65.86	0\\
65.87	0\\
65.88	0\\
65.89	0\\
65.9	0\\
65.91	0\\
65.92	0\\
65.93	0\\
65.94	0\\
65.95	0\\
65.96	0\\
65.97	0\\
65.98	0\\
65.99	0\\
66	0\\
66.01	0\\
66.02	0\\
66.03	0\\
66.04	0\\
66.05	0\\
66.06	0\\
66.07	0\\
66.08	0\\
66.09	0\\
66.1	0\\
66.11	0\\
66.12	0\\
66.13	0\\
66.14	0\\
66.15	0\\
66.16	0\\
66.17	0\\
66.18	0\\
66.19	0\\
66.2	0\\
66.21	0\\
66.22	0\\
66.23	0\\
66.24	0\\
66.25	0\\
66.26	0\\
66.27	0\\
66.28	0\\
66.29	0\\
66.3	0\\
66.31	0\\
66.32	0\\
66.33	0\\
66.34	0\\
66.35	0\\
66.36	0\\
66.37	0\\
66.38	0\\
66.39	0\\
66.4	0\\
66.41	0\\
66.42	0\\
66.43	0\\
66.44	0\\
66.45	0\\
66.46	0\\
66.47	0\\
66.48	0\\
66.49	0\\
66.5	0\\
66.51	0\\
66.52	0\\
66.53	0\\
66.54	0\\
66.55	0\\
66.56	0\\
66.57	0\\
66.58	0\\
66.59	0\\
66.6	0\\
66.61	0\\
66.62	0\\
66.63	0\\
66.64	0\\
66.65	0\\
66.66	0\\
66.67	0\\
66.68	0\\
66.69	0\\
66.7	0\\
66.71	0\\
66.72	0\\
66.73	0\\
66.74	0\\
66.75	0\\
66.76	0\\
66.77	0\\
66.78	0\\
66.79	0\\
66.8	0\\
66.81	0\\
66.82	0\\
66.83	0\\
66.84	0\\
66.85	0\\
66.86	0\\
66.87	0\\
66.88	0\\
66.89	0\\
66.9	0\\
66.91	0\\
66.92	0\\
66.93	0\\
66.94	0\\
66.95	0\\
66.96	0\\
66.97	0\\
66.98	0\\
66.99	0\\
67	0\\
67.01	0\\
67.02	0\\
67.03	0\\
67.04	0\\
67.05	0\\
67.06	0\\
67.07	0\\
67.08	0\\
67.09	0\\
67.1	0\\
67.11	0\\
67.12	0\\
67.13	0\\
67.14	0\\
67.15	0\\
67.16	0\\
67.17	0\\
67.18	0\\
67.19	0\\
67.2	0\\
67.21	0\\
67.22	0\\
67.23	0\\
67.24	0\\
67.25	0\\
67.26	0\\
67.27	0\\
67.28	0\\
67.29	0\\
67.3	0\\
67.31	0\\
67.32	0\\
67.33	0\\
67.34	0\\
67.35	0\\
67.36	0\\
67.37	0\\
67.38	0\\
67.39	0\\
67.4	0\\
67.41	0\\
67.42	0\\
67.43	0\\
67.44	0\\
67.45	0\\
67.46	0\\
67.47	0\\
67.48	0\\
67.49	0\\
67.5	0\\
67.51	0\\
67.52	0\\
67.53	0\\
67.54	0\\
67.55	0\\
67.56	0\\
67.57	0\\
67.58	0\\
67.59	0\\
67.6	0\\
67.61	0\\
67.62	0\\
67.63	0\\
67.64	0\\
67.65	0\\
67.66	0\\
67.67	0\\
67.68	0\\
67.69	0\\
67.7	0\\
67.71	0\\
67.72	0\\
67.73	0\\
67.74	0\\
67.75	0\\
67.76	0\\
67.77	0\\
67.78	0\\
67.79	0\\
67.8	0\\
67.81	0\\
67.82	0\\
67.83	0\\
67.84	0\\
67.85	0\\
67.86	0\\
67.87	0\\
67.88	0\\
67.89	0\\
67.9	0\\
67.91	0\\
67.92	0\\
67.93	0\\
67.94	0\\
67.95	0\\
67.96	0\\
67.97	0\\
67.98	0\\
67.99	0\\
68	0\\
68.01	0\\
68.02	0\\
68.03	0\\
68.04	0\\
68.05	0\\
68.06	0\\
68.07	0\\
68.08	0\\
68.09	0\\
68.1	0\\
68.11	0\\
68.12	0\\
68.13	0\\
68.14	0\\
68.15	0\\
68.16	0\\
68.17	0\\
68.18	0\\
68.19	0\\
68.2	0\\
68.21	0\\
68.22	0\\
68.23	0\\
68.24	0\\
68.25	0\\
68.26	0\\
68.27	0\\
68.28	0\\
68.29	0\\
68.3	0\\
68.31	0\\
68.32	0\\
68.33	0\\
68.34	0\\
68.35	0\\
68.36	0\\
68.37	0\\
68.38	0\\
68.39	0\\
68.4	0\\
68.41	0\\
68.42	0\\
68.43	0\\
68.44	0\\
68.45	0\\
68.46	0\\
68.47	0\\
68.48	0\\
68.49	0\\
68.5	0\\
68.51	0\\
68.52	0\\
68.53	0\\
68.54	0\\
68.55	0\\
68.56	0\\
68.57	0\\
68.58	0\\
68.59	0\\
68.6	0\\
68.61	0\\
68.62	0\\
68.63	0\\
68.64	0\\
68.65	0\\
68.66	0\\
68.67	0\\
68.68	0\\
68.69	0\\
68.7	0\\
68.71	0\\
68.72	0\\
68.73	0\\
68.74	0\\
68.75	0\\
68.76	0\\
68.77	0\\
68.78	0\\
68.79	0\\
68.8	0\\
68.81	0\\
68.82	0\\
68.83	0\\
68.84	0\\
68.85	0\\
68.86	0\\
68.87	0\\
68.88	0\\
68.89	0\\
68.9	0\\
68.91	0\\
68.92	0\\
68.93	0\\
68.94	0\\
68.95	0\\
68.96	0\\
68.97	0\\
68.98	0\\
68.99	0\\
69	0\\
69.01	0\\
69.02	0\\
69.03	0\\
69.04	0\\
69.05	0\\
69.06	0\\
69.07	0\\
69.08	0\\
69.09	0\\
69.1	0\\
69.11	0\\
69.12	0\\
69.13	0\\
69.14	0\\
69.15	0\\
69.16	0\\
69.17	0\\
69.18	0\\
69.19	0\\
69.2	0\\
69.21	0\\
69.22	0\\
69.23	0\\
69.24	0\\
69.25	0\\
69.26	0\\
69.27	0\\
69.28	0\\
69.29	0\\
69.3	0\\
69.31	0\\
69.32	0\\
69.33	0\\
69.34	0\\
69.35	0\\
69.36	0\\
69.37	0\\
69.38	0\\
69.39	0\\
69.4	0\\
69.41	0\\
69.42	0\\
69.43	0\\
69.44	0\\
69.45	0\\
69.46	0\\
69.47	0\\
69.48	0\\
69.49	0\\
69.5	0\\
69.51	0\\
69.52	0\\
69.53	0\\
69.54	0\\
69.55	0\\
69.56	0\\
69.57	0\\
69.58	0\\
69.59	0\\
69.6	0\\
69.61	0\\
69.62	0\\
69.63	0\\
69.64	0\\
69.65	0\\
69.66	0\\
69.67	0\\
69.68	0\\
69.69	0\\
69.7	0\\
69.71	0\\
69.72	0\\
69.73	0\\
69.74	0\\
69.75	0\\
69.76	0\\
69.77	0\\
69.78	0\\
69.79	0\\
69.8	0\\
69.81	0\\
69.82	0\\
69.83	0\\
69.84	0\\
69.85	0\\
69.86	0\\
69.87	0\\
69.88	0\\
69.89	0\\
69.9	0\\
69.91	0\\
69.92	0\\
69.93	0\\
69.94	0\\
69.95	0\\
69.96	0\\
69.97	0\\
69.98	0\\
69.99	0\\
70	0\\
70.01	0\\
70.02	0\\
70.03	0\\
70.04	0\\
70.05	0\\
70.06	0\\
70.07	0\\
70.08	0\\
70.09	0\\
70.1	0\\
70.11	0\\
70.12	0\\
70.13	0\\
70.14	0\\
70.15	0\\
70.16	0\\
70.17	0\\
70.18	0\\
70.19	0\\
70.2	0\\
70.21	0\\
70.22	0\\
70.23	0\\
70.24	0\\
70.25	0\\
70.26	0\\
70.27	0\\
70.28	0\\
70.29	0\\
70.3	0\\
70.31	0\\
70.32	0\\
70.33	0\\
70.34	0\\
70.35	0\\
70.36	0\\
70.37	0\\
70.38	0\\
70.39	0\\
70.4	0\\
70.41	0\\
70.42	0\\
70.43	0\\
70.44	0\\
70.45	0\\
70.46	0\\
70.47	0\\
70.48	0\\
70.49	0\\
70.5	0\\
70.51	0\\
70.52	0\\
70.53	0\\
70.54	0\\
70.55	0\\
70.56	0\\
70.57	0\\
70.58	0\\
70.59	0\\
70.6	0\\
70.61	0\\
70.62	0\\
70.63	0\\
70.64	0\\
70.65	0\\
70.66	0\\
70.67	0\\
70.68	0\\
70.69	0\\
70.7	0\\
70.71	0\\
70.72	0\\
70.73	0\\
70.74	0\\
70.75	0\\
70.76	0\\
70.77	0\\
70.78	0\\
70.79	0\\
70.8	0\\
70.81	0\\
70.82	0\\
70.83	0\\
70.84	0\\
70.85	0\\
70.86	0\\
70.87	0\\
70.88	0\\
70.89	0\\
70.9	0\\
70.91	0\\
70.92	0\\
70.93	0\\
70.94	0\\
70.95	0\\
70.96	0\\
70.97	0\\
70.98	0\\
70.99	0\\
71	0\\
71.01	0\\
71.02	0\\
71.03	0\\
71.04	0\\
71.05	0\\
71.06	0\\
71.07	0\\
71.08	0\\
71.09	0\\
71.1	0\\
71.11	0\\
71.12	0\\
71.13	0\\
71.14	0\\
71.15	0\\
71.16	0\\
71.17	0\\
71.18	0\\
71.19	0\\
71.2	0\\
71.21	0\\
71.22	0\\
71.23	0\\
71.24	0\\
71.25	0\\
71.26	0\\
71.27	0\\
71.28	0\\
71.29	0\\
71.3	0\\
71.31	0\\
71.32	0\\
71.33	0\\
71.34	0\\
71.35	0\\
71.36	0\\
71.37	0\\
71.38	0\\
71.39	0\\
71.4	0\\
71.41	0\\
71.42	0\\
71.43	0\\
71.44	0\\
71.45	0\\
71.46	0\\
71.47	0\\
71.48	0\\
71.49	0\\
71.5	0\\
71.51	0\\
71.52	0\\
71.53	0\\
71.54	0\\
71.55	0\\
71.56	0\\
71.57	0\\
71.58	0\\
71.59	0\\
71.6	0\\
71.61	0\\
71.62	0\\
71.63	0\\
71.64	0\\
71.65	0\\
71.66	0\\
71.67	0\\
71.68	0\\
71.69	0\\
71.7	0\\
71.71	0\\
71.72	0\\
71.73	0\\
71.74	0\\
71.75	0\\
71.76	0\\
71.77	0\\
71.78	0\\
71.79	0\\
71.8	0\\
71.81	0\\
71.82	0\\
71.83	0\\
71.84	0\\
71.85	0\\
71.86	0\\
71.87	0\\
71.88	0\\
71.89	0\\
71.9	0\\
71.91	0\\
71.92	0\\
71.93	0\\
71.94	0\\
71.95	0\\
71.96	0\\
71.97	0\\
71.98	0\\
71.99	0\\
72	0\\
72.01	0\\
72.02	0\\
72.03	0\\
72.04	0\\
72.05	0\\
72.06	0\\
72.07	0\\
72.08	0\\
72.09	0\\
72.1	0\\
72.11	0\\
72.12	0\\
72.13	0\\
72.14	0\\
72.15	0\\
72.16	0\\
72.17	0\\
72.18	0\\
72.19	0\\
72.2	0\\
72.21	0\\
72.22	0\\
72.23	0\\
72.24	0\\
72.25	0\\
72.26	0\\
72.27	0\\
72.28	0\\
72.29	0\\
72.3	0\\
72.31	0\\
72.32	0\\
72.33	0\\
72.34	0\\
72.35	0\\
72.36	0\\
72.37	0\\
72.38	0\\
72.39	0\\
72.4	0\\
72.41	0\\
72.42	0\\
72.43	0\\
72.44	0\\
72.45	0\\
72.46	0\\
72.47	0\\
72.48	0\\
72.49	0\\
72.5	0\\
72.51	0\\
72.52	0\\
72.53	0\\
72.54	0\\
72.55	0\\
72.56	0\\
72.57	0\\
72.58	0\\
72.59	0\\
72.6	0\\
72.61	0\\
72.62	0\\
72.63	0\\
72.64	0\\
72.65	0\\
72.66	0\\
72.67	0\\
72.68	0\\
72.69	0\\
72.7	0\\
72.71	0\\
72.72	0\\
72.73	0\\
72.74	0\\
72.75	0\\
72.76	0\\
72.77	0\\
72.78	0\\
72.79	0\\
72.8	0\\
72.81	0\\
72.82	0\\
72.83	0\\
72.84	0\\
72.85	0\\
72.86	0\\
72.87	0\\
72.88	0\\
72.89	0\\
72.9	0\\
72.91	0\\
72.92	0\\
72.93	0\\
72.94	0\\
72.95	0\\
72.96	0\\
72.97	0\\
72.98	0\\
72.99	0\\
73	0\\
73.01	0\\
73.02	0\\
73.03	0\\
73.04	0\\
73.05	0\\
73.06	0\\
73.07	0\\
73.08	0\\
73.09	0\\
73.1	0\\
73.11	0\\
73.12	0\\
73.13	0\\
73.14	0\\
73.15	0\\
73.16	0\\
73.17	0\\
73.18	0\\
73.19	0\\
73.2	0\\
73.21	0\\
73.22	0\\
73.23	0\\
73.24	0\\
73.25	0\\
73.26	0\\
73.27	0\\
73.28	0\\
73.29	0\\
73.3	0\\
73.31	0\\
73.32	0\\
73.33	0\\
73.34	0\\
73.35	0\\
73.36	0\\
73.37	0\\
73.38	0\\
73.39	0\\
73.4	0\\
73.41	0\\
73.42	0\\
73.43	0\\
73.44	0\\
73.45	0\\
73.46	0\\
73.47	0\\
73.48	0\\
73.49	0\\
73.5	0\\
73.51	0\\
73.52	0\\
73.53	0\\
73.54	0\\
73.55	0\\
73.56	0\\
73.57	0\\
73.58	0\\
73.59	0\\
73.6	0\\
73.61	0\\
73.62	0\\
73.63	0\\
73.64	0\\
73.65	0\\
73.66	0\\
73.67	0\\
73.68	0\\
73.69	0\\
73.7	0\\
73.71	0\\
73.72	0\\
73.73	0\\
73.74	0\\
73.75	0\\
73.76	0\\
73.77	0\\
73.78	0\\
73.79	0\\
73.8	0\\
73.81	0\\
73.82	0\\
73.83	0\\
73.84	0\\
73.85	0\\
73.86	0\\
73.87	0\\
73.88	0\\
73.89	0\\
73.9	0\\
73.91	0\\
73.92	0\\
73.93	0\\
73.94	0\\
73.95	0\\
73.96	0\\
73.97	0\\
73.98	0\\
73.99	0\\
74	0\\
74.01	0\\
74.02	0\\
74.03	0\\
74.04	0\\
74.05	0\\
74.06	0\\
74.07	0\\
74.08	0\\
74.09	0\\
74.1	0\\
74.11	0\\
74.12	0\\
74.13	0\\
74.14	0\\
74.15	0\\
74.16	0\\
74.17	0\\
74.18	0\\
74.19	0\\
74.2	0\\
74.21	0\\
74.22	0\\
74.23	0\\
74.24	0\\
74.25	0\\
74.26	0\\
74.27	0\\
74.28	0\\
74.29	0\\
74.3	0\\
74.31	0\\
74.32	0\\
74.33	0\\
74.34	0\\
74.35	0\\
74.36	0\\
74.37	0\\
74.38	0\\
74.39	0\\
74.4	0\\
74.41	0\\
74.42	0\\
74.43	0\\
74.44	0\\
74.45	0\\
74.46	0\\
74.47	0\\
74.48	0\\
74.49	0\\
74.5	0\\
74.51	0\\
74.52	0\\
74.53	0\\
74.54	0\\
74.55	0\\
74.56	0\\
74.57	0\\
74.58	0\\
74.59	0\\
74.6	0\\
74.61	0\\
74.62	0\\
74.63	0\\
74.64	0\\
74.65	0\\
74.66	0\\
74.67	0\\
74.68	0\\
74.69	0\\
74.7	0\\
74.71	0\\
74.72	0\\
74.73	0\\
74.74	0\\
74.75	0\\
74.76	0\\
74.77	0\\
74.78	0\\
74.79	0\\
74.8	0\\
74.81	0\\
74.82	0\\
74.83	0\\
74.84	0\\
74.85	0\\
74.86	0\\
74.87	0\\
74.88	0\\
74.89	0\\
74.9	0\\
74.91	0\\
74.92	0\\
74.93	0\\
74.94	0\\
74.95	0\\
74.96	0\\
74.97	0\\
74.98	0\\
74.99	0\\
75	0\\
75.01	0\\
75.02	0\\
75.03	0\\
75.04	0\\
75.05	0\\
75.06	0\\
75.07	0\\
75.08	0\\
75.09	0\\
75.1	0\\
75.11	0\\
75.12	0\\
75.13	0\\
75.14	0\\
75.15	0\\
75.16	0\\
75.17	0\\
75.18	0\\
75.19	0\\
75.2	0\\
75.21	0\\
75.22	0\\
75.23	0\\
75.24	0\\
75.25	0\\
75.26	0\\
75.27	0\\
75.28	0\\
75.29	0\\
75.3	0\\
75.31	0\\
75.32	0\\
75.33	0\\
75.34	0\\
75.35	0\\
75.36	0\\
75.37	0\\
75.38	0\\
75.39	0\\
75.4	0\\
75.41	0\\
75.42	0\\
75.43	0\\
75.44	0\\
75.45	0\\
75.46	0\\
75.47	0\\
75.48	0\\
75.49	0\\
75.5	0\\
75.51	0\\
75.52	0\\
75.53	0\\
75.54	0\\
75.55	0\\
75.56	0\\
75.57	0\\
75.58	0\\
75.59	0\\
75.6	0\\
75.61	0\\
75.62	0\\
75.63	0\\
75.64	0\\
75.65	0\\
75.66	0\\
75.67	0\\
75.68	0\\
75.69	0\\
75.7	0\\
75.71	0\\
75.72	0\\
75.73	0\\
75.74	0\\
75.75	0\\
75.76	0\\
75.77	0\\
75.78	0\\
75.79	0\\
75.8	0\\
75.81	0\\
75.82	0\\
75.83	0\\
75.84	0\\
75.85	0\\
75.86	0\\
75.87	0\\
75.88	0\\
75.89	0\\
75.9	0\\
75.91	0\\
75.92	0\\
75.93	0\\
75.94	0\\
75.95	0\\
75.96	0\\
75.97	0\\
75.98	0\\
75.99	0\\
76	0\\
76.01	0\\
76.02	0\\
76.03	0\\
76.04	0\\
76.05	0\\
76.06	0\\
76.07	0\\
76.08	0\\
76.09	0\\
76.1	0\\
76.11	0\\
76.12	0\\
76.13	0\\
76.14	0\\
76.15	0\\
76.16	0\\
76.17	0\\
76.18	0\\
76.19	0\\
76.2	0\\
76.21	0\\
76.22	0\\
76.23	0\\
76.24	0\\
76.25	0\\
76.26	0\\
76.27	0\\
76.28	0\\
76.29	0\\
76.3	0\\
76.31	0\\
76.32	0\\
76.33	0\\
76.34	0\\
76.35	0\\
76.36	0\\
76.37	0\\
76.38	0\\
76.39	0\\
76.4	0\\
76.41	0\\
76.42	0\\
76.43	0\\
76.44	0\\
76.45	0\\
76.46	0\\
76.47	0\\
76.48	0\\
76.49	0\\
76.5	0\\
76.51	0\\
76.52	0\\
76.53	0\\
76.54	0\\
76.55	0\\
76.56	0\\
76.57	0\\
76.58	0\\
76.59	0\\
76.6	0\\
76.61	0\\
76.62	0\\
76.63	0\\
76.64	0\\
76.65	0\\
76.66	0\\
76.67	0\\
76.68	0\\
76.69	0\\
76.7	0\\
76.71	0\\
76.72	0\\
76.73	0\\
76.74	0\\
76.75	0\\
76.76	0\\
76.77	0\\
76.78	0\\
76.79	0\\
76.8	0\\
76.81	0\\
76.82	0\\
76.83	0\\
76.84	0\\
76.85	0\\
76.86	0\\
76.87	0\\
76.88	0\\
76.89	0\\
76.9	0\\
76.91	0\\
76.92	0\\
76.93	0\\
76.94	0\\
76.95	0\\
76.96	0\\
76.97	0\\
76.98	0\\
76.99	0\\
77	0\\
77.01	0\\
77.02	0\\
77.03	0\\
77.04	0\\
77.05	0\\
77.06	0\\
77.07	0\\
77.08	0\\
77.09	0\\
77.1	0\\
77.11	0\\
77.12	0\\
77.13	0\\
77.14	0\\
77.15	0\\
77.16	0\\
77.17	0\\
77.18	0\\
77.19	0\\
77.2	0\\
77.21	0\\
77.22	0\\
77.23	0\\
77.24	0\\
77.25	0\\
77.26	0\\
77.27	0\\
77.28	0\\
77.29	0\\
77.3	0\\
77.31	0\\
77.32	0\\
77.33	0\\
77.34	0\\
77.35	0\\
77.36	0\\
77.37	0\\
77.38	0\\
77.39	0\\
77.4	0\\
77.41	0\\
77.42	0\\
77.43	0\\
77.44	0\\
77.45	0\\
77.46	0\\
77.47	0\\
77.48	0\\
77.49	0\\
77.5	0\\
77.51	0\\
77.52	0\\
77.53	0\\
77.54	0\\
77.55	0\\
77.56	0\\
77.57	0\\
77.58	0\\
77.59	0\\
77.6	0\\
77.61	0\\
77.62	0\\
77.63	0\\
77.64	0\\
77.65	0\\
77.66	0\\
77.67	0\\
77.68	0\\
77.69	0\\
77.7	0\\
77.71	0\\
77.72	0\\
77.73	0\\
77.74	0\\
77.75	0\\
77.76	0\\
77.77	0\\
77.78	0\\
77.79	0\\
77.8	0\\
77.81	0\\
77.82	0\\
77.83	0\\
77.84	0\\
77.85	0\\
77.86	0\\
77.87	0\\
77.88	0\\
77.89	0\\
77.9	0\\
77.91	0\\
77.92	0\\
77.93	0\\
77.94	0\\
77.95	0\\
77.96	0\\
77.97	0\\
77.98	0\\
77.99	0\\
78	0\\
78.01	0\\
78.02	0\\
78.03	0\\
78.04	0\\
78.05	0\\
78.06	0\\
78.07	0\\
78.08	0\\
78.09	0\\
78.1	0\\
78.11	0\\
78.12	0\\
78.13	0\\
78.14	0\\
78.15	0\\
78.16	0\\
78.17	0\\
78.18	0\\
78.19	0\\
78.2	0\\
78.21	0\\
78.22	0\\
78.23	0\\
78.24	0\\
78.25	0\\
78.26	0\\
78.27	0\\
78.28	0\\
78.29	0\\
78.3	0\\
78.31	0\\
78.32	0\\
78.33	0\\
78.34	0\\
78.35	0\\
78.36	0\\
78.37	0\\
78.38	0\\
78.39	0\\
78.4	0\\
78.41	0\\
78.42	0\\
78.43	0\\
78.44	0\\
78.45	0\\
78.46	0\\
78.47	0\\
78.48	0\\
78.49	0\\
78.5	0\\
78.51	0\\
78.52	0\\
78.53	0\\
78.54	0\\
78.55	0\\
78.56	0\\
78.57	0\\
78.58	0\\
78.59	0\\
78.6	0\\
78.61	0\\
78.62	0\\
78.63	0\\
78.64	0\\
78.65	0\\
78.66	0\\
78.67	0\\
78.68	0\\
78.69	0\\
78.7	0\\
78.71	0\\
78.72	0\\
78.73	0\\
78.74	0\\
78.75	0\\
78.76	0\\
78.77	0\\
78.78	0\\
78.79	0\\
78.8	0\\
78.81	0\\
78.82	0\\
78.83	0\\
78.84	0\\
78.85	0\\
78.86	0\\
78.87	0\\
78.88	0\\
78.89	0\\
78.9	0\\
78.91	0\\
78.92	0\\
78.93	0\\
78.94	0\\
78.95	0\\
78.96	0\\
78.97	0\\
78.98	0\\
78.99	0\\
79	0\\
79.01	0\\
79.02	0\\
79.03	0\\
79.04	0\\
79.05	0\\
79.06	0\\
79.07	0\\
79.08	0\\
79.09	0\\
79.1	0\\
79.11	0\\
79.12	0\\
79.13	0\\
79.14	0\\
79.15	0\\
79.16	0\\
79.17	0\\
79.18	0\\
79.19	0\\
79.2	0\\
79.21	0\\
79.22	0\\
79.23	0\\
79.24	0\\
79.25	0\\
79.26	0\\
79.27	0\\
79.28	0\\
79.29	0\\
79.3	0\\
79.31	0\\
79.32	0\\
79.33	0\\
79.34	0\\
79.35	0\\
79.36	0\\
79.37	0\\
79.38	0\\
79.39	0\\
79.4	0\\
79.41	0\\
79.42	0\\
79.43	0\\
79.44	0\\
79.45	0\\
79.46	0\\
79.47	0\\
79.48	0\\
79.49	0\\
79.5	0\\
79.51	0\\
79.52	0\\
79.53	0\\
79.54	0\\
79.55	0\\
79.56	0\\
79.57	0\\
79.58	0\\
79.59	0\\
79.6	0\\
79.61	0\\
79.62	0\\
79.63	0\\
79.64	0\\
79.65	0\\
79.66	0\\
79.67	0\\
79.68	0\\
79.69	0\\
79.7	0\\
79.71	0\\
79.72	0\\
79.73	0\\
79.74	0\\
79.75	0\\
79.76	0\\
79.77	0\\
79.78	0\\
79.79	0\\
79.8	0\\
79.81	0\\
79.82	0\\
79.83	0\\
79.84	0\\
79.85	0\\
79.86	0\\
79.87	0\\
79.88	0\\
79.89	0\\
79.9	0\\
79.91	0\\
79.92	0\\
79.93	0\\
79.94	0\\
79.95	0\\
79.96	0\\
79.97	0\\
79.98	0\\
79.99	0\\
80	0\\
80.01	0\\
};
\addplot [color=black,solid]
  table[row sep=crcr]{%
80.01	0\\
80.02	0\\
80.03	0\\
80.04	0\\
80.05	0\\
80.06	0\\
80.07	0\\
80.08	0\\
80.09	0\\
80.1	0\\
80.11	0\\
80.12	0\\
80.13	0\\
80.14	0\\
80.15	0\\
80.16	0\\
80.17	0\\
80.18	0\\
80.19	0\\
80.2	0\\
80.21	0\\
80.22	0\\
80.23	0\\
80.24	0\\
80.25	0\\
80.26	0\\
80.27	0\\
80.28	0\\
80.29	0\\
80.3	0\\
80.31	0\\
80.32	0\\
80.33	0\\
80.34	0\\
80.35	0\\
80.36	0\\
80.37	0\\
80.38	0\\
80.39	0\\
80.4	0\\
80.41	0\\
80.42	0\\
80.43	0\\
80.44	0\\
80.45	0\\
80.46	0\\
80.47	0\\
80.48	0\\
80.49	0\\
80.5	0\\
80.51	0\\
80.52	0\\
80.53	0\\
80.54	0\\
80.55	0\\
80.56	0\\
80.57	0\\
80.58	0\\
80.59	0\\
80.6	0\\
80.61	0\\
80.62	0\\
80.63	0\\
80.64	0\\
80.65	0\\
80.66	0\\
80.67	0\\
80.68	0\\
80.69	0\\
80.7	0\\
80.71	0\\
80.72	0\\
80.73	0\\
80.74	0\\
80.75	0\\
80.76	0\\
80.77	0\\
80.78	0\\
80.79	0\\
80.8	0\\
80.81	0\\
80.82	0\\
80.83	0\\
80.84	0\\
80.85	0\\
80.86	0\\
80.87	0\\
80.88	0\\
80.89	0\\
80.9	0\\
80.91	0\\
80.92	0\\
80.93	0\\
80.94	0\\
80.95	0\\
80.96	0\\
80.97	0\\
80.98	0\\
80.99	0\\
81	0\\
81.01	0\\
81.02	0\\
81.03	0\\
81.04	0\\
81.05	0\\
81.06	0\\
81.07	0\\
81.08	0\\
81.09	0\\
81.1	0\\
81.11	0\\
81.12	0\\
81.13	0\\
81.14	0\\
81.15	0\\
81.16	0\\
81.17	0\\
81.18	0\\
81.19	0\\
81.2	0\\
81.21	0\\
81.22	0\\
81.23	0\\
81.24	0\\
81.25	0\\
81.26	0\\
81.27	0\\
81.28	0\\
81.29	0\\
81.3	0\\
81.31	0\\
81.32	0\\
81.33	0\\
81.34	0\\
81.35	0\\
81.36	0\\
81.37	0\\
81.38	0\\
81.39	0\\
81.4	0\\
81.41	0\\
81.42	0\\
81.43	0\\
81.44	0\\
81.45	0\\
81.46	0\\
81.47	0\\
81.48	0\\
81.49	0\\
81.5	0\\
81.51	0\\
81.52	0\\
81.53	0\\
81.54	0\\
81.55	0\\
81.56	0\\
81.57	0\\
81.58	0\\
81.59	0\\
81.6	0\\
81.61	0\\
81.62	0\\
81.63	0\\
81.64	0\\
81.65	0\\
81.66	0\\
81.67	0\\
81.68	0\\
81.69	0\\
81.7	0\\
81.71	0\\
81.72	0\\
81.73	0\\
81.74	0\\
81.75	0\\
81.76	0\\
81.77	0\\
81.78	0\\
81.79	0\\
81.8	0\\
81.81	0\\
81.82	0\\
81.83	0\\
81.84	0\\
81.85	0\\
81.86	0\\
81.87	0\\
81.88	0\\
81.89	0\\
81.9	0\\
81.91	0\\
81.92	0\\
81.93	0\\
81.94	0\\
81.95	0\\
81.96	0\\
81.97	0\\
81.98	0\\
81.99	0\\
82	0\\
82.01	0\\
82.02	0\\
82.03	0\\
82.04	0\\
82.05	0\\
82.06	0\\
82.07	0\\
82.08	0\\
82.09	0\\
82.1	0\\
82.11	0\\
82.12	0\\
82.13	0\\
82.14	0\\
82.15	0\\
82.16	0\\
82.17	0\\
82.18	0\\
82.19	0\\
82.2	0\\
82.21	0\\
82.22	0\\
82.23	0\\
82.24	0\\
82.25	0\\
82.26	0\\
82.27	0\\
82.28	0\\
82.29	0\\
82.3	0\\
82.31	0\\
82.32	0\\
82.33	0\\
82.34	0\\
82.35	0\\
82.36	0\\
82.37	0\\
82.38	0\\
82.39	0\\
82.4	0\\
82.41	0\\
82.42	0\\
82.43	0\\
82.44	0\\
82.45	0\\
82.46	0\\
82.47	0\\
82.48	0\\
82.49	0\\
82.5	0\\
82.51	0\\
82.52	0\\
82.53	0\\
82.54	0\\
82.55	0\\
82.56	0\\
82.57	0\\
82.58	0\\
82.59	0\\
82.6	0\\
82.61	0\\
82.62	0\\
82.63	0\\
82.64	0\\
82.65	0\\
82.66	0\\
82.67	0\\
82.68	0\\
82.69	0\\
82.7	0\\
82.71	0\\
82.72	0\\
82.73	0\\
82.74	0\\
82.75	0\\
82.76	0\\
82.77	0\\
82.78	0\\
82.79	0\\
82.8	0\\
82.81	0\\
82.82	0\\
82.83	0\\
82.84	0\\
82.85	0\\
82.86	0\\
82.87	0\\
82.88	0\\
82.89	0\\
82.9	0\\
82.91	0\\
82.92	0\\
82.93	0\\
82.94	0\\
82.95	0\\
82.96	0\\
82.97	0\\
82.98	0\\
82.99	0\\
83	0\\
83.01	0\\
83.02	0\\
83.03	0\\
83.04	0\\
83.05	0\\
83.06	0\\
83.07	0\\
83.08	0\\
83.09	0\\
83.1	0\\
83.11	0\\
83.12	0\\
83.13	0\\
83.14	0\\
83.15	0\\
83.16	0\\
83.17	0\\
83.18	0\\
83.19	0\\
83.2	0\\
83.21	0\\
83.22	0\\
83.23	0\\
83.24	0\\
83.25	0\\
83.26	0\\
83.27	0\\
83.28	0\\
83.29	0\\
83.3	0\\
83.31	0\\
83.32	0\\
83.33	0\\
83.34	0\\
83.35	0\\
83.36	0\\
83.37	0\\
83.38	0\\
83.39	0\\
83.4	0\\
83.41	0\\
83.42	0\\
83.43	0\\
83.44	0\\
83.45	0\\
83.46	0\\
83.47	0\\
83.48	0\\
83.49	0\\
83.5	0\\
83.51	0\\
83.52	0\\
83.53	0\\
83.54	0\\
83.55	0\\
83.56	0\\
83.57	0\\
83.58	0\\
83.59	0\\
83.6	0\\
83.61	0\\
83.62	0\\
83.63	0\\
83.64	0\\
83.65	0\\
83.66	0\\
83.67	0\\
83.68	0\\
83.69	0\\
83.7	0\\
83.71	0\\
83.72	0\\
83.73	0\\
83.74	0\\
83.75	0\\
83.76	0\\
83.77	0\\
83.78	0\\
83.79	0\\
83.8	0\\
83.81	0\\
83.82	0\\
83.83	0\\
83.84	0\\
83.85	0\\
83.86	0\\
83.87	0\\
83.88	0\\
83.89	0\\
83.9	0\\
83.91	0\\
83.92	0\\
83.93	0\\
83.94	0\\
83.95	0\\
83.96	0\\
83.97	0\\
83.98	0\\
83.99	0\\
84	0\\
84.01	0\\
84.02	0\\
84.03	0\\
84.04	0\\
84.05	0\\
84.06	0\\
84.07	0\\
84.08	0\\
84.09	0\\
84.1	0\\
84.11	0\\
84.12	0\\
84.13	0\\
84.14	0\\
84.15	0\\
84.16	0\\
84.17	0\\
84.18	0\\
84.19	0\\
84.2	0\\
84.21	0\\
84.22	0\\
84.23	0\\
84.24	0\\
84.25	0\\
84.26	0\\
84.27	0\\
84.28	0\\
84.29	0\\
84.3	0\\
84.31	0\\
84.32	0\\
84.33	0\\
84.34	0\\
84.35	0\\
84.36	0\\
84.37	0\\
84.38	0\\
84.39	0\\
84.4	0\\
84.41	0\\
84.42	0\\
84.43	0\\
84.44	0\\
84.45	0\\
84.46	0\\
84.47	0\\
84.48	0\\
84.49	0\\
84.5	0\\
84.51	0\\
84.52	0\\
84.53	0\\
84.54	0\\
84.55	0\\
84.56	0\\
84.57	0\\
84.58	0\\
84.59	0\\
84.6	0\\
84.61	0\\
84.62	0\\
84.63	0\\
84.64	0\\
84.65	0\\
84.66	0\\
84.67	0\\
84.68	0\\
84.69	0\\
84.7	0\\
84.71	0\\
84.72	0\\
84.73	0\\
84.74	0\\
84.75	0\\
84.76	0\\
84.77	0\\
84.78	0\\
84.79	0\\
84.8	0\\
84.81	0\\
84.82	0\\
84.83	0\\
84.84	0\\
84.85	0\\
84.86	0\\
84.87	0\\
84.88	0\\
84.89	0\\
84.9	0\\
84.91	0\\
84.92	0\\
84.93	0\\
84.94	0\\
84.95	0\\
84.96	0\\
84.97	0\\
84.98	0\\
84.99	0\\
85	0\\
85.01	0\\
85.02	0\\
85.03	0\\
85.04	0\\
85.05	0\\
85.06	0\\
85.07	0\\
85.08	0\\
85.09	0\\
85.1	0\\
85.11	0\\
85.12	0\\
85.13	0\\
85.14	0\\
85.15	0\\
85.16	0\\
85.17	0\\
85.18	0\\
85.19	0\\
85.2	0\\
85.21	0\\
85.22	0\\
85.23	0\\
85.24	0\\
85.25	0\\
85.26	0\\
85.27	0\\
85.28	0\\
85.29	0\\
85.3	0\\
85.31	0\\
85.32	0\\
85.33	0\\
85.34	0\\
85.35	0\\
85.36	0\\
85.37	0\\
85.38	0\\
85.39	0\\
85.4	0\\
85.41	0\\
85.42	0\\
85.43	0\\
85.44	0\\
85.45	0\\
85.46	0\\
85.47	0\\
85.48	0\\
85.49	0\\
85.5	0\\
85.51	0\\
85.52	0\\
85.53	0\\
85.54	0\\
85.55	0\\
85.56	0\\
85.57	0\\
85.58	0\\
85.59	0\\
85.6	0\\
85.61	0\\
85.62	0\\
85.63	0\\
85.64	0\\
85.65	0\\
85.66	0\\
85.67	0\\
85.68	0\\
85.69	0\\
85.7	0\\
85.71	0\\
85.72	0\\
85.73	0\\
85.74	0\\
85.75	0\\
85.76	0\\
85.77	0\\
85.78	0\\
85.79	0\\
85.8	0\\
85.81	0\\
85.82	0\\
85.83	0\\
85.84	0\\
85.85	0\\
85.86	0\\
85.87	0\\
85.88	0\\
85.89	0\\
85.9	0\\
85.91	0\\
85.92	0\\
85.93	0\\
85.94	0\\
85.95	0\\
85.96	0\\
85.97	0\\
85.98	0\\
85.99	0\\
86	0\\
86.01	0\\
86.02	0\\
86.03	0\\
86.04	0\\
86.05	0\\
86.06	0\\
86.07	0\\
86.08	0\\
86.09	0\\
86.1	0\\
86.11	0\\
86.12	0\\
86.13	0\\
86.14	0\\
86.15	0\\
86.16	0\\
86.17	0\\
86.18	0\\
86.19	0\\
86.2	0\\
86.21	0\\
86.22	0\\
86.23	0\\
86.24	0\\
86.25	0\\
86.26	0\\
86.27	0\\
86.28	0\\
86.29	0\\
86.3	0\\
86.31	0\\
86.32	0\\
86.33	0\\
86.34	0\\
86.35	0\\
86.36	0\\
86.37	0\\
86.38	0\\
86.39	0\\
86.4	0\\
86.41	0\\
86.42	0\\
86.43	0\\
86.44	0\\
86.45	0\\
86.46	0\\
86.47	0\\
86.48	0\\
86.49	0\\
86.5	0\\
86.51	0\\
86.52	0\\
86.53	0\\
86.54	0\\
86.55	0\\
86.56	0\\
86.57	0\\
86.58	0\\
86.59	0\\
86.6	0\\
86.61	0\\
86.62	0\\
86.63	0\\
86.64	0\\
86.65	0\\
86.66	0\\
86.67	0\\
86.68	0\\
86.69	0\\
86.7	0\\
86.71	0\\
86.72	0\\
86.73	0\\
86.74	0\\
86.75	0\\
86.76	0\\
86.77	0\\
86.78	0\\
86.79	0\\
86.8	0\\
86.81	0\\
86.82	0\\
86.83	0\\
86.84	0\\
86.85	0\\
86.86	0\\
86.87	0\\
86.88	0\\
86.89	0\\
86.9	0\\
86.91	0\\
86.92	0\\
86.93	0\\
86.94	0\\
86.95	0\\
86.96	0\\
86.97	0\\
86.98	0\\
86.99	0\\
87	0\\
87.01	0\\
87.02	0\\
87.03	0\\
87.04	0\\
87.05	0\\
87.06	0\\
87.07	0\\
87.08	0\\
87.09	0\\
87.1	0\\
87.11	0\\
87.12	0\\
87.13	0\\
87.14	0\\
87.15	0\\
87.16	0\\
87.17	0\\
87.18	0\\
87.19	0\\
87.2	0\\
87.21	0\\
87.22	0\\
87.23	0\\
87.24	0\\
87.25	0\\
87.26	0\\
87.27	0\\
87.28	0\\
87.29	0\\
87.3	0\\
87.31	0\\
87.32	0\\
87.33	0\\
87.34	0\\
87.35	0\\
87.36	0\\
87.37	0\\
87.38	0\\
87.39	0\\
87.4	0\\
87.41	0\\
87.42	0\\
87.43	0\\
87.44	0\\
87.45	0\\
87.46	0\\
87.47	0\\
87.48	0\\
87.49	0\\
87.5	0\\
87.51	0\\
87.52	0\\
87.53	0\\
87.54	0\\
87.55	0\\
87.56	0\\
87.57	0\\
87.58	0\\
87.59	0\\
87.6	0\\
87.61	0\\
87.62	0\\
87.63	0\\
87.64	0\\
87.65	0\\
87.66	0\\
87.67	0\\
87.68	0\\
87.69	0\\
87.7	0\\
87.71	0\\
87.72	0\\
87.73	0\\
87.74	0\\
87.75	0\\
87.76	0\\
87.77	0\\
87.78	0\\
87.79	0\\
87.8	0\\
87.81	0\\
87.82	0\\
87.83	0\\
87.84	0\\
87.85	0\\
87.86	0\\
87.87	0\\
87.88	0\\
87.89	0\\
87.9	0\\
87.91	0\\
87.92	0\\
87.93	0\\
87.94	0\\
87.95	0\\
87.96	0\\
87.97	0\\
87.98	0\\
87.99	0\\
88	0\\
88.01	0\\
88.02	0\\
88.03	0\\
88.04	0\\
88.05	0\\
88.06	0\\
88.07	0\\
88.08	0\\
88.09	0\\
88.1	0\\
88.11	0\\
88.12	0\\
88.13	0\\
88.14	0\\
88.15	0\\
88.16	0\\
88.17	0\\
88.18	0\\
88.19	0\\
88.2	0\\
88.21	0\\
88.22	0\\
88.23	0\\
88.24	0\\
88.25	0\\
88.26	0\\
88.27	0\\
88.28	0\\
88.29	0\\
88.3	0\\
88.31	0\\
88.32	0\\
88.33	0\\
88.34	0\\
88.35	0\\
88.36	0\\
88.37	0\\
88.38	0\\
88.39	0\\
88.4	0\\
88.41	0\\
88.42	0\\
88.43	0\\
88.44	0\\
88.45	0\\
88.46	0\\
88.47	0\\
88.48	0\\
88.49	0\\
88.5	0\\
88.51	0\\
88.52	0\\
88.53	0\\
88.54	0\\
88.55	0\\
88.56	0\\
88.57	0\\
88.58	0\\
88.59	0\\
88.6	0\\
88.61	0\\
88.62	0\\
88.63	0\\
88.64	0\\
88.65	0\\
88.66	0\\
88.67	0\\
88.68	0\\
88.69	0\\
88.7	0\\
88.71	0\\
88.72	0\\
88.73	0\\
88.74	0\\
88.75	0\\
88.76	0\\
88.77	0\\
88.78	0\\
88.79	0\\
88.8	0\\
88.81	0\\
88.82	0\\
88.83	0\\
88.84	0\\
88.85	0\\
88.86	0\\
88.87	0\\
88.88	0\\
88.89	0\\
88.9	0\\
88.91	0\\
88.92	0\\
88.93	0\\
88.94	0\\
88.95	0\\
88.96	0\\
88.97	0\\
88.98	0\\
88.99	0\\
89	0\\
89.01	0\\
89.02	0\\
89.03	0\\
89.04	0\\
89.05	0\\
89.06	0\\
89.07	0\\
89.08	0\\
89.09	0\\
89.1	0\\
89.11	0\\
89.12	0\\
89.13	0\\
89.14	0\\
89.15	0\\
89.16	0\\
89.17	0\\
89.18	0\\
89.19	0\\
89.2	0\\
89.21	0\\
89.22	0\\
89.23	0\\
89.24	0\\
89.25	0\\
89.26	0\\
89.27	0\\
89.28	0\\
89.29	0\\
89.3	0\\
89.31	0\\
89.32	0\\
89.33	0\\
89.34	0\\
89.35	0\\
89.36	0\\
89.37	0\\
89.38	0\\
89.39	0\\
89.4	0\\
89.41	0\\
89.42	0\\
89.43	0\\
89.44	0\\
89.45	0\\
89.46	0\\
89.47	0\\
89.48	0\\
89.49	0\\
89.5	0\\
89.51	0\\
89.52	0\\
89.53	0\\
89.54	0\\
89.55	0\\
89.56	0\\
89.57	0\\
89.58	0\\
89.59	0\\
89.6	0\\
89.61	0\\
89.62	0\\
89.63	0\\
89.64	0\\
89.65	0\\
89.66	0\\
89.67	0\\
89.68	0\\
89.69	0\\
89.7	0\\
89.71	0\\
89.72	0\\
89.73	0\\
89.74	0\\
89.75	0\\
89.76	0\\
89.77	0\\
89.78	0\\
89.79	0\\
89.8	0\\
89.81	0\\
89.82	0\\
89.83	0\\
89.84	0\\
89.85	0\\
89.86	0\\
89.87	0\\
89.88	0\\
89.89	0\\
89.9	0\\
89.91	0\\
89.92	0\\
89.93	0\\
89.94	0\\
89.95	0\\
89.96	0\\
89.97	0\\
89.98	0\\
89.99	0\\
90	0\\
90.01	0\\
90.02	0\\
90.03	0\\
90.04	0\\
90.05	0\\
90.06	0\\
90.07	0\\
90.08	0\\
90.09	0\\
90.1	0\\
90.11	0\\
90.12	0\\
90.13	0\\
90.14	0\\
90.15	0\\
90.16	0\\
90.17	0\\
90.18	0\\
90.19	0\\
90.2	0\\
90.21	0\\
90.22	0\\
90.23	0\\
90.24	0\\
90.25	0\\
90.26	0\\
90.27	0\\
90.28	0\\
90.29	0\\
90.3	0\\
90.31	0\\
90.32	0\\
90.33	0\\
90.34	0\\
90.35	0\\
90.36	0\\
90.37	0\\
90.38	0\\
90.39	0\\
90.4	0\\
90.41	0\\
90.42	0\\
90.43	0\\
90.44	0\\
90.45	0\\
90.46	0\\
90.47	0\\
90.48	0\\
90.49	0\\
90.5	0\\
90.51	0\\
90.52	0\\
90.53	0\\
90.54	0\\
90.55	0\\
90.56	0\\
90.57	0\\
90.58	0\\
90.59	0\\
90.6	0\\
90.61	0\\
90.62	0\\
90.63	0\\
90.64	0\\
90.65	0\\
90.66	0\\
90.67	0\\
90.68	0\\
90.69	0\\
90.7	0\\
90.71	0\\
90.72	0\\
90.73	0\\
90.74	0\\
90.75	0\\
90.76	0\\
90.77	0\\
90.78	0\\
90.79	0\\
90.8	0\\
90.81	0\\
90.82	0\\
90.83	0\\
90.84	0\\
90.85	0\\
90.86	0\\
90.87	0\\
90.88	0\\
90.89	0\\
90.9	0\\
90.91	0\\
90.92	0\\
90.93	0\\
90.94	0\\
90.95	0\\
90.96	0\\
90.97	0\\
90.98	0\\
90.99	0\\
91	0\\
91.01	0\\
91.02	0\\
91.03	0\\
91.04	0\\
91.05	0\\
91.06	0\\
91.07	0\\
91.08	0\\
91.09	0\\
91.1	0\\
91.11	0\\
91.12	0\\
91.13	0\\
91.14	0\\
91.15	0\\
91.16	0\\
91.17	0\\
91.18	0\\
91.19	0\\
91.2	0\\
91.21	0\\
91.22	0\\
91.23	0\\
91.24	0\\
91.25	0\\
91.26	0\\
91.27	0\\
91.28	0\\
91.29	0\\
91.3	0\\
91.31	0\\
91.32	0\\
91.33	0\\
91.34	0\\
91.35	0\\
91.36	0\\
91.37	0\\
91.38	0\\
91.39	0\\
91.4	0\\
91.41	0\\
91.42	0\\
91.43	0\\
91.44	0\\
91.45	0\\
91.46	0\\
91.47	0\\
91.48	0\\
91.49	0\\
91.5	0\\
91.51	0\\
91.52	0\\
91.53	0\\
91.54	0\\
91.55	0\\
91.56	0\\
91.57	0\\
91.58	0\\
91.59	0\\
91.6	0\\
91.61	0\\
91.62	0\\
91.63	0\\
91.64	0\\
91.65	0\\
91.66	0\\
91.67	0\\
91.68	0\\
91.69	0\\
91.7	0\\
91.71	0\\
91.72	0\\
91.73	0\\
91.74	0\\
91.75	0\\
91.76	0\\
91.77	0\\
91.78	0\\
91.79	0\\
91.8	0\\
91.81	0\\
91.82	0\\
91.83	0\\
91.84	0\\
91.85	0\\
91.86	0\\
91.87	0\\
91.88	0\\
91.89	0\\
91.9	0\\
91.91	0\\
91.92	0\\
91.93	0\\
91.94	0\\
91.95	0\\
91.96	0\\
91.97	0\\
91.98	0\\
91.99	0\\
92	0\\
92.01	0\\
92.02	0\\
92.03	0\\
92.04	0\\
92.05	0\\
92.06	0\\
92.07	0\\
92.08	0\\
92.09	0\\
92.1	0\\
92.11	0\\
92.12	0\\
92.13	0\\
92.14	0\\
92.15	0\\
92.16	0\\
92.17	0\\
92.18	0\\
92.19	0\\
92.2	0\\
92.21	0\\
92.22	0\\
92.23	0\\
92.24	0\\
92.25	0\\
92.26	0\\
92.27	0\\
92.28	0\\
92.29	0\\
92.3	0\\
92.31	0\\
92.32	0\\
92.33	0\\
92.34	0\\
92.35	0\\
92.36	0\\
92.37	0\\
92.38	0\\
92.39	0\\
92.4	0\\
92.41	0\\
92.42	0\\
92.43	0\\
92.44	0\\
92.45	0\\
92.46	0\\
92.47	0\\
92.48	0\\
92.49	0\\
92.5	0\\
92.51	0\\
92.52	0\\
92.53	0\\
92.54	0\\
92.55	0\\
92.56	0\\
92.57	0\\
92.58	0\\
92.59	0\\
92.6	0\\
92.61	0\\
92.62	0\\
92.63	0\\
92.64	0\\
92.65	0\\
92.66	0\\
92.67	0\\
92.68	0\\
92.69	0\\
92.7	0\\
92.71	0\\
92.72	0\\
92.73	0\\
92.74	0\\
92.75	0\\
92.76	0\\
92.77	0\\
92.78	0\\
92.79	0\\
92.8	0\\
92.81	0\\
92.82	0\\
92.83	0\\
92.84	0\\
92.85	0\\
92.86	0\\
92.87	0\\
92.88	0\\
92.89	0\\
92.9	0\\
92.91	0\\
92.92	0\\
92.93	0\\
92.94	0\\
92.95	0\\
92.96	0\\
92.97	0\\
92.98	0\\
92.99	0\\
93	0\\
93.01	0\\
93.02	0\\
93.03	0\\
93.04	0\\
93.05	0\\
93.06	0\\
93.07	0\\
93.08	0\\
93.09	0\\
93.1	0\\
93.11	0\\
93.12	0\\
93.13	0\\
93.14	0\\
93.15	0\\
93.16	0\\
93.17	0\\
93.18	0\\
93.19	0\\
93.2	0\\
93.21	0\\
93.22	0\\
93.23	0\\
93.24	0\\
93.25	0\\
93.26	0\\
93.27	0\\
93.28	0\\
93.29	0\\
93.3	0\\
93.31	0\\
93.32	0\\
93.33	0\\
93.34	0\\
93.35	0\\
93.36	0\\
93.37	0\\
93.38	0\\
93.39	0\\
93.4	0\\
93.41	0\\
93.42	0\\
93.43	0\\
93.44	0\\
93.45	0\\
93.46	0\\
93.47	0\\
93.48	0\\
93.49	0\\
93.5	0\\
93.51	0\\
93.52	0\\
93.53	0\\
93.54	0\\
93.55	0\\
93.56	0\\
93.57	0\\
93.58	0\\
93.59	0\\
93.6	0\\
93.61	0\\
93.62	0\\
93.63	0\\
93.64	0\\
93.65	0\\
93.66	0\\
93.67	0\\
93.68	0\\
93.69	0\\
93.7	0\\
93.71	0\\
93.72	0\\
93.73	0\\
93.74	0\\
93.75	0\\
93.76	0\\
93.77	0\\
93.78	0\\
93.79	0\\
93.8	0\\
93.81	0\\
93.82	0\\
93.83	0\\
93.84	0\\
93.85	0\\
93.86	0\\
93.87	0\\
93.88	0\\
93.89	0\\
93.9	0\\
93.91	0\\
93.92	0\\
93.93	0\\
93.94	0\\
93.95	0\\
93.96	0\\
93.97	0\\
93.98	0\\
93.99	0\\
94	0\\
94.01	0\\
94.02	0\\
94.03	0\\
94.04	0\\
94.05	0\\
94.06	0\\
94.07	0\\
94.08	0\\
94.09	0\\
94.1	0\\
94.11	0\\
94.12	0\\
94.13	0\\
94.14	0\\
94.15	0\\
94.16	0\\
94.17	0\\
94.18	0\\
94.19	0\\
94.2	0\\
94.21	0\\
94.22	0\\
94.23	0\\
94.24	0\\
94.25	0\\
94.26	0\\
94.27	0\\
94.28	0\\
94.29	0\\
94.3	0\\
94.31	0\\
94.32	0\\
94.33	0\\
94.34	0\\
94.35	0\\
94.36	0\\
94.37	0\\
94.38	0\\
94.39	0\\
94.4	0\\
94.41	0\\
94.42	0\\
94.43	0\\
94.44	0\\
94.45	0\\
94.46	0\\
94.47	0\\
94.48	0\\
94.49	0\\
94.5	0\\
94.51	0\\
94.52	0\\
94.53	0\\
94.54	0\\
94.55	0\\
94.56	0\\
94.57	0\\
94.58	0\\
94.59	0\\
94.6	0\\
94.61	0\\
94.62	0\\
94.63	0\\
94.64	0\\
94.65	0\\
94.66	0\\
94.67	0\\
94.68	0\\
94.69	0\\
94.7	0\\
94.71	0\\
94.72	0\\
94.73	0\\
94.74	0\\
94.75	0\\
94.76	0\\
94.77	0\\
94.78	0\\
94.79	0\\
94.8	0\\
94.81	0\\
94.82	0\\
94.83	0\\
94.84	0\\
94.85	0\\
94.86	0\\
94.87	0\\
94.88	0\\
94.89	0\\
94.9	0\\
94.91	0\\
94.92	0\\
94.93	0\\
94.94	0\\
94.95	0\\
94.96	0\\
94.97	0\\
94.98	0\\
94.99	0\\
95	0\\
95.01	0\\
95.02	0\\
95.03	0\\
95.04	0\\
95.05	0\\
95.06	0\\
95.07	0\\
95.08	0\\
95.09	0\\
95.1	0\\
95.11	0\\
95.12	0\\
95.13	0\\
95.14	0\\
95.15	0\\
95.16	0\\
95.17	0\\
95.18	0\\
95.19	0\\
95.2	0\\
95.21	0\\
95.22	0\\
95.23	0\\
95.24	0\\
95.25	0\\
95.26	0\\
95.27	0\\
95.28	0\\
95.29	0\\
95.3	0\\
95.31	0\\
95.32	0\\
95.33	0\\
95.34	0\\
95.35	0\\
95.36	0\\
95.37	0\\
95.38	0\\
95.39	0\\
95.4	0\\
95.41	0\\
95.42	0\\
95.43	0\\
95.44	0\\
95.45	0\\
95.46	0\\
95.47	0\\
95.48	0\\
95.49	0\\
95.5	0\\
95.51	0\\
95.52	0\\
95.53	0\\
95.54	0\\
95.55	0\\
95.56	0\\
95.57	0\\
95.58	0\\
95.59	0\\
95.6	0\\
95.61	0\\
95.62	0\\
95.63	0\\
95.64	0\\
95.65	0\\
95.66	0\\
95.67	0\\
95.68	0\\
95.69	0\\
95.7	0\\
95.71	0\\
95.72	0\\
95.73	0\\
95.74	0\\
95.75	0\\
95.76	0\\
95.77	0\\
95.78	0\\
95.79	0\\
95.8	0\\
95.81	0\\
95.82	0\\
95.83	0\\
95.84	0\\
95.85	0\\
95.86	0\\
95.87	0\\
95.88	0\\
95.89	0\\
95.9	0\\
95.91	0\\
95.92	0\\
95.93	0\\
95.94	0\\
95.95	0\\
95.96	0\\
95.97	0\\
95.98	0\\
95.99	0\\
96	0\\
96.01	0\\
96.02	0\\
96.03	0\\
96.04	0\\
96.05	0\\
96.06	0\\
96.07	0\\
96.08	0\\
96.09	0\\
96.1	0\\
96.11	0\\
96.12	0\\
96.13	0\\
96.14	0\\
96.15	0\\
96.16	0\\
96.17	0\\
96.18	0\\
96.19	0\\
96.2	0\\
96.21	0\\
96.22	0\\
96.23	0\\
96.24	0\\
96.25	0\\
96.26	0\\
96.27	0\\
96.28	0\\
96.29	0\\
96.3	0\\
96.31	0\\
96.32	0\\
96.33	0\\
96.34	0\\
96.35	0\\
96.36	0\\
96.37	0\\
96.38	0\\
96.39	0\\
96.4	0\\
96.41	0\\
96.42	0\\
96.43	0\\
96.44	0\\
96.45	0\\
96.46	0\\
96.47	0\\
96.48	0\\
96.49	0\\
96.5	0\\
96.51	0\\
96.52	0\\
96.53	0\\
96.54	0\\
96.55	0\\
96.56	0\\
96.57	0\\
96.58	0\\
96.59	0\\
96.6	0\\
96.61	0\\
96.62	0\\
96.63	0\\
96.64	0\\
96.65	0\\
96.66	0\\
96.67	0\\
96.68	0\\
96.69	0\\
96.7	0\\
96.71	0\\
96.72	0\\
96.73	0\\
96.74	0\\
96.75	0\\
96.76	0\\
96.77	0\\
96.78	0\\
96.79	0\\
96.8	0\\
96.81	0\\
96.82	0\\
96.83	0\\
96.84	0\\
96.85	0\\
96.86	0\\
96.87	0\\
96.88	0\\
96.89	0\\
96.9	0\\
96.91	0\\
96.92	0\\
96.93	0\\
96.94	0\\
96.95	0\\
96.96	0\\
96.97	0\\
96.98	0\\
96.99	0\\
97	0\\
97.01	0\\
97.02	0\\
97.03	0\\
97.04	0\\
97.05	0\\
97.06	0\\
97.07	0\\
97.08	0\\
97.09	0\\
97.1	0\\
97.11	0\\
97.12	0\\
97.13	0\\
97.14	0\\
97.15	0\\
97.16	0\\
97.17	0\\
97.18	0\\
97.19	0\\
97.2	0\\
97.21	0\\
97.22	0\\
97.23	0\\
97.24	0\\
97.25	0\\
97.26	0\\
97.27	0\\
97.28	0\\
97.29	0\\
97.3	0\\
97.31	0\\
97.32	0\\
97.33	0\\
97.34	0\\
97.35	0\\
97.36	0\\
97.37	0\\
97.38	0\\
97.39	0\\
97.4	0\\
97.41	0\\
97.42	0\\
97.43	0\\
97.44	0\\
97.45	0\\
97.46	0\\
97.47	0\\
97.48	0\\
97.49	0\\
97.5	0\\
97.51	0\\
97.52	0\\
97.53	0\\
97.54	0\\
97.55	0\\
97.56	0\\
97.57	0\\
97.58	0\\
97.59	0\\
97.6	0\\
97.61	0\\
97.62	0\\
97.63	0\\
97.64	0\\
97.65	0\\
97.66	0\\
97.67	0\\
97.68	0\\
97.69	0\\
97.7	0\\
97.71	0\\
97.72	0\\
97.73	0\\
97.74	0\\
97.75	0\\
97.76	0\\
97.77	0\\
97.78	0\\
97.79	0\\
97.8	0\\
97.81	0\\
97.82	0\\
97.83	0\\
97.84	0\\
97.85	0\\
97.86	0\\
97.87	0\\
97.88	0\\
97.89	0\\
97.9	0\\
97.91	0\\
97.92	0\\
97.93	0\\
97.94	0\\
97.95	0\\
97.96	0\\
97.97	0\\
97.98	0\\
97.99	0\\
98	0\\
98.01	0\\
98.02	0\\
98.03	0\\
98.04	0\\
98.05	0\\
98.06	0\\
98.07	0\\
98.08	0\\
98.09	0\\
98.1	0\\
98.11	0\\
98.12	0\\
98.13	0\\
98.14	0\\
98.15	0\\
98.16	0\\
98.17	0\\
98.18	0\\
98.19	0\\
98.2	0\\
98.21	0\\
98.22	0\\
98.23	0\\
98.24	0\\
98.25	0\\
98.26	0\\
98.27	0\\
98.28	0\\
98.29	0\\
98.3	0\\
98.31	0\\
98.32	0\\
98.33	0\\
98.34	0\\
98.35	0\\
98.36	0\\
98.37	0\\
98.38	0\\
98.39	0\\
98.4	0\\
98.41	0\\
98.42	0\\
98.43	0\\
98.44	0\\
98.45	0\\
98.46	0\\
98.47	0\\
98.48	0\\
98.49	0\\
98.5	0\\
98.51	0\\
98.52	0\\
98.53	0\\
98.54	0\\
98.55	0\\
98.56	0\\
98.57	0\\
98.58	0\\
98.59	0\\
98.6	0\\
98.61	0\\
98.62	0\\
98.63	4.94611794170548e-06\\
98.64	4.86326403356298e-05\\
98.65	9.26072923284407e-05\\
98.66	0.000136872661141966\\
98.67	0.000181431360839864\\
98.68	0.000226286032657277\\
98.69	0.000271439345335158\\
98.7	0.000316893995648279\\
98.71	0.000362652708795118\\
98.72	0.000408718238750657\\
98.73	0.000455093368624159\\
98.74	0.000501773555020202\\
98.75	0.000548753550006406\\
98.76	0.000596035979263025\\
98.77	0.000643623495679556\\
98.78	0.000691518780279354\\
98.79	0.000739724362759035\\
98.8	0.000788238981861179\\
98.81	0.000837065336280141\\
98.82	0.000886206152745978\\
98.83	0.000935664186379102\\
98.84	0.000985442221050094\\
98.85	0.00103554306974477\\
98.86	0.00108596957493457\\
98.87	0.00113672460895238\\
98.88	0.00118781107437385\\
98.89	0.00123923190440427\\
98.9	0.00129099006327118\\
98.91	0.00134308854662269\\
98.92	0.00139553038446909\\
98.93	0.00144831863943283\\
98.94	0.00150145640686559\\
98.95	0.00155494681527221\\
98.96	0.00160879302674092\\
98.97	0.00166299823737984\\
98.98	0.00171756567776007\\
98.99	0.00177249861336525\\
99	0.00182780034504796\\
99.01	0.00188347420949276\\
99.02	0.00193952357968626\\
99.03	0.00199595186539415\\
99.04	0.00205276251364543\\
99.05	0.00210995900922383\\
99.06	0.0021675448751667\\
99.07	0.00222552367327132\\
99.08	0.00228389900460891\\
99.09	0.00234267451004639\\
99.1	0.002401853870776\\
99.11	0.00246144080885295\\
99.12	0.00252143908774134\\
99.13	0.00258185251286822\\
99.14	0.00264268493218622\\
99.15	0.00270394023674476\\
99.16	0.00276562236126998\\
99.17	0.00282773528475357\\
99.18	0.0028902830310507\\
99.19	0.00295326966948699\\
99.2	0.00301669931547491\\
99.21	0.00308057613113965\\
99.22	0.00314490432595475\\
99.23	0.00320968815738764\\
99.24	0.00327493193155525\\
99.25	0.0033406400038899\\
99.26	0.00340681677981562\\
99.27	0.00347346671543517\\
99.28	0.00354059431822787\\
99.29	0.00360820414775853\\
99.3	0.00367630081639761\\
99.31	0.00374488899005289\\
99.32	0.00381397338891286\\
99.33	0.003883558788202\\
99.34	0.00395365001894827\\
99.35	0.00402425196876301\\
99.36	0.00409536958263347\\
99.37	0.00416700786372826\\
99.38	0.00423917187421598\\
99.39	0.00431186673610018\\
99.4	0.00438509763218818\\
99.41	0.00445886980695187\\
99.42	0.00453318856740286\\
99.43	0.00460805928398217\\
99.44	0.00468348739146472\\
99.45	0.00475947838987908\\
99.46	0.00483603784544274\\
99.47	0.00491317139151319\\
99.48	0.00499088472955523\\
99.49	0.00506918363012486\\
99.5	0.00514807393387008\\
99.51	0.00522756155254905\\
99.52	0.00530765247006595\\
99.53	0.00538835274352498\\
99.54	0.00546966850430286\\
99.55	0.00555160595914044\\
99.56	0.00563417139125364\\
99.57	0.00571737116146438\\
99.58	0.00580121170935183\\
99.59	0.00588569955442464\\
99.6	0.00597084129731456\\
99.61	0.00605664362099196\\
99.62	0.00614311329200395\\
99.63	0.00623025716173545\\
99.64	0.00631808216769411\\
99.65	0.00640659533481932\\
99.66	0.00649580377681639\\
99.67	0.00658571469751614\\
99.68	0.00667633539226089\\
99.69	0.00676767324931748\\
99.7	0.00685973575131798\\
99.71	0.00695253047672899\\
99.72	0.0070460651013502\\
99.73	0.0071403473998536\\
99.74	0.00723538524735233\\
99.75	0.00733118662099236\\
99.76	0.00742775960157701\\
99.77	0.0075251123752253\\
99.78	0.00762325323506514\\
99.79	0.00772219058296238\\
99.8	0.00782193293128693\\
99.81	0.00792248890471694\\
99.82	0.00802386724208233\\
99.83	0.0081260767982489\\
99.84	0.00822912654604423\\
99.85	0.00833302557822688\\
99.86	0.00843778310950011\\
99.87	0.00854340847857167\\
99.88	0.00864991115026125\\
99.89	0.00875730071765702\\
99.9	0.00886558690432315\\
99.91	0.00897477956655976\\
99.92	0.00908488869571742\\
99.93	0.00919592442056783\\
99.94	0.00930789700973289\\
99.95	0.009420816874174\\
99.96	0.00953469456974397\\
99.97	0.00964954079980366\\
99.98	0.00976536641790577\\
99.99	0.00988218243054826\\
100	0.01\\
};
\addlegendentry{$q=0$};

\addplot [color=blue,solid,forget plot]
  table[row sep=crcr]{%
0.01	0.00162143504819226\\
0.02	0.00162143504819226\\
0.03	0.00162143504819226\\
0.04	0.00162143504819226\\
0.05	0.00162143504819226\\
0.06	0.00162143504819226\\
0.07	0.00162143504819226\\
0.08	0.00162143504819226\\
0.09	0.00162143504819226\\
0.1	0.00162143504819226\\
0.11	0.00162143504819226\\
0.12	0.00162143504819226\\
0.13	0.00162143504819226\\
0.14	0.00162143504819226\\
0.15	0.00162143504819226\\
0.16	0.00162143504819226\\
0.17	0.00162143504819226\\
0.18	0.00162143504819226\\
0.19	0.00162143504819226\\
0.2	0.00162143504819226\\
0.21	0.00162143504819226\\
0.22	0.00162143504819226\\
0.23	0.00162143504819226\\
0.24	0.00162143504819226\\
0.25	0.00162143504819226\\
0.26	0.00162143504819226\\
0.27	0.00162143504819226\\
0.28	0.00162143504819226\\
0.29	0.00162143504819226\\
0.3	0.00162143504819226\\
0.31	0.00162143504819226\\
0.32	0.00162143504819226\\
0.33	0.00162143504819226\\
0.34	0.00162143504819226\\
0.35	0.00162143504819226\\
0.36	0.00162143504819226\\
0.37	0.00162143504819226\\
0.38	0.00162143504819226\\
0.39	0.00162143504819226\\
0.4	0.00162143504819226\\
0.41	0.00162143504819226\\
0.42	0.00162143504819226\\
0.43	0.00162143504819226\\
0.44	0.00162143504819226\\
0.45	0.00162143504819226\\
0.46	0.00162143504819226\\
0.47	0.00162143504819226\\
0.48	0.00162143504819226\\
0.49	0.00162143504819226\\
0.5	0.00162143504819226\\
0.51	0.00162143504819226\\
0.52	0.00162143504819226\\
0.53	0.00162143504819226\\
0.54	0.00162143504819226\\
0.55	0.00162143504819226\\
0.56	0.00162143504819226\\
0.57	0.00162143504819226\\
0.58	0.00162143504819226\\
0.59	0.00162143504819226\\
0.6	0.00162143504819226\\
0.61	0.00162143504819226\\
0.62	0.00162143504819226\\
0.63	0.00162143504819226\\
0.64	0.00162143504819226\\
0.65	0.00162143504819226\\
0.66	0.00162143504819226\\
0.67	0.00162143504819226\\
0.68	0.00162143504819226\\
0.69	0.00162143504819226\\
0.7	0.00162143504819226\\
0.71	0.00162143504819226\\
0.72	0.00162143504819226\\
0.73	0.00162143504819226\\
0.74	0.00162143504819226\\
0.75	0.00162143504819226\\
0.76	0.00162143504819226\\
0.77	0.00162143504819226\\
0.78	0.00162143504819226\\
0.79	0.00162143504819226\\
0.8	0.00162143504819226\\
0.81	0.00162143504819226\\
0.82	0.00162143504819226\\
0.83	0.00162143504819226\\
0.84	0.00162143504819226\\
0.85	0.00162143504819226\\
0.86	0.00162143504819226\\
0.87	0.00162143504819226\\
0.88	0.00162143504819226\\
0.89	0.00162143504819226\\
0.9	0.00162143504819226\\
0.91	0.00162143504819226\\
0.92	0.00162143504819226\\
0.93	0.00162143504819226\\
0.94	0.00162143504819226\\
0.95	0.00162143504819226\\
0.96	0.00162143504819226\\
0.97	0.00162143504819226\\
0.98	0.00162143504819226\\
0.99	0.00162143504819226\\
1	0.00162143504819226\\
1.01	0.00162143504819226\\
1.02	0.00162143504819226\\
1.03	0.00162143504819226\\
1.04	0.00162143504819226\\
1.05	0.00162143504819226\\
1.06	0.00162143504819226\\
1.07	0.00162143504819226\\
1.08	0.00162143504819226\\
1.09	0.00162143504819226\\
1.1	0.00162143504819226\\
1.11	0.00162143504819226\\
1.12	0.00162143504819226\\
1.13	0.00162143504819226\\
1.14	0.00162143504819226\\
1.15	0.00162143504819226\\
1.16	0.00162143504819226\\
1.17	0.00162143504819226\\
1.18	0.00162143504819226\\
1.19	0.00162143504819226\\
1.2	0.00162143504819226\\
1.21	0.00162143504819226\\
1.22	0.00162143504819226\\
1.23	0.00162143504819226\\
1.24	0.00162143504819226\\
1.25	0.00162143504819226\\
1.26	0.00162143504819226\\
1.27	0.00162143504819226\\
1.28	0.00162143504819226\\
1.29	0.00162143504819226\\
1.3	0.00162143504819226\\
1.31	0.00162143504819226\\
1.32	0.00162143504819226\\
1.33	0.00162143504819226\\
1.34	0.00162143504819226\\
1.35	0.00162143504819226\\
1.36	0.00162143504819226\\
1.37	0.00162143504819226\\
1.38	0.00162143504819226\\
1.39	0.00162143504819226\\
1.4	0.00162143504819226\\
1.41	0.00162143504819226\\
1.42	0.00162143504819226\\
1.43	0.00162143504819226\\
1.44	0.00162143504819226\\
1.45	0.00162143504819226\\
1.46	0.00162143504819226\\
1.47	0.00162143504819226\\
1.48	0.00162143504819226\\
1.49	0.00162143504819226\\
1.5	0.00162143504819226\\
1.51	0.00162143504819226\\
1.52	0.00162143504819226\\
1.53	0.00162143504819226\\
1.54	0.00162143504819226\\
1.55	0.00162143504819226\\
1.56	0.00162143504819226\\
1.57	0.00162143504819226\\
1.58	0.00162143504819226\\
1.59	0.00162143504819226\\
1.6	0.00162143504819226\\
1.61	0.00162143504819226\\
1.62	0.00162143504819226\\
1.63	0.00162143504819226\\
1.64	0.00162143504819226\\
1.65	0.00162143504819226\\
1.66	0.00162143504819226\\
1.67	0.00162143504819226\\
1.68	0.00162143504819226\\
1.69	0.00162143504819226\\
1.7	0.00162143504819226\\
1.71	0.00162143504819226\\
1.72	0.00162143504819226\\
1.73	0.00162143504819226\\
1.74	0.00162143504819226\\
1.75	0.00162143504819226\\
1.76	0.00162143504819226\\
1.77	0.00162143504819226\\
1.78	0.00162143504819226\\
1.79	0.00162143504819226\\
1.8	0.00162143504819226\\
1.81	0.00162143504819226\\
1.82	0.00162143504819226\\
1.83	0.00162143504819226\\
1.84	0.00162143504819226\\
1.85	0.00162143504819226\\
1.86	0.00162143504819226\\
1.87	0.00162143504819226\\
1.88	0.00162143504819226\\
1.89	0.00162143504819226\\
1.9	0.00162143504819226\\
1.91	0.00162143504819226\\
1.92	0.00162143504819226\\
1.93	0.00162143504819226\\
1.94	0.00162143504819226\\
1.95	0.00162143504819226\\
1.96	0.00162143504819226\\
1.97	0.00162143504819226\\
1.98	0.00162143504819226\\
1.99	0.00162143504819226\\
2	0.00162143504819226\\
2.01	0.00162143504819226\\
2.02	0.00162143504819226\\
2.03	0.00162143504819226\\
2.04	0.00162143504819226\\
2.05	0.00162143504819226\\
2.06	0.00162143504819226\\
2.07	0.00162143504819226\\
2.08	0.00162143504819226\\
2.09	0.00162143504819226\\
2.1	0.00162143504819226\\
2.11	0.00162143504819226\\
2.12	0.00162143504819226\\
2.13	0.00162143504819226\\
2.14	0.00162143504819226\\
2.15	0.00162143504819226\\
2.16	0.00162143504819226\\
2.17	0.00162143504819226\\
2.18	0.00162143504819226\\
2.19	0.00162143504819226\\
2.2	0.00162143504819226\\
2.21	0.00162143504819226\\
2.22	0.00162143504819226\\
2.23	0.00162143504819226\\
2.24	0.00162143504819226\\
2.25	0.00162143504819226\\
2.26	0.00162143504819226\\
2.27	0.00162143504819226\\
2.28	0.00162143504819226\\
2.29	0.00162143504819226\\
2.3	0.00162143504819226\\
2.31	0.00162143504819226\\
2.32	0.00162143504819226\\
2.33	0.00162143504819226\\
2.34	0.00162143504819226\\
2.35	0.00162143504819226\\
2.36	0.00162143504819226\\
2.37	0.00162143504819226\\
2.38	0.00162143504819226\\
2.39	0.00162143504819226\\
2.4	0.00162143504819226\\
2.41	0.00162143504819226\\
2.42	0.00162143504819226\\
2.43	0.00162143504819226\\
2.44	0.00162143504819226\\
2.45	0.00162143504819226\\
2.46	0.00162143504819226\\
2.47	0.00162143504819226\\
2.48	0.00162143504819226\\
2.49	0.00162143504819226\\
2.5	0.00162143504819226\\
2.51	0.00162143504819226\\
2.52	0.00162143504819226\\
2.53	0.00162143504819226\\
2.54	0.00162143504819226\\
2.55	0.00162143504819226\\
2.56	0.00162143504819226\\
2.57	0.00162143504819226\\
2.58	0.00162143504819226\\
2.59	0.00162143504819226\\
2.6	0.00162143504819226\\
2.61	0.00162143504819226\\
2.62	0.00162143504819226\\
2.63	0.00162143504819226\\
2.64	0.00162143504819226\\
2.65	0.00162143504819226\\
2.66	0.00162143504819226\\
2.67	0.00162143504819226\\
2.68	0.00162143504819226\\
2.69	0.00162143504819226\\
2.7	0.00162143504819226\\
2.71	0.00162143504819226\\
2.72	0.00162143504819226\\
2.73	0.00162143504819226\\
2.74	0.00162143504819226\\
2.75	0.00162143504819226\\
2.76	0.00162143504819226\\
2.77	0.00162143504819226\\
2.78	0.00162143504819226\\
2.79	0.00162143504819226\\
2.8	0.00162143504819226\\
2.81	0.00162143504819226\\
2.82	0.00162143504819226\\
2.83	0.00162143504819226\\
2.84	0.00162143504819226\\
2.85	0.00162143504819226\\
2.86	0.00162143504819226\\
2.87	0.00162143504819226\\
2.88	0.00162143504819226\\
2.89	0.00162143504819226\\
2.9	0.00162143504819226\\
2.91	0.00162143504819226\\
2.92	0.00162143504819226\\
2.93	0.00162143504819226\\
2.94	0.00162143504819226\\
2.95	0.00162143504819226\\
2.96	0.00162143504819226\\
2.97	0.00162143504819226\\
2.98	0.00162143504819226\\
2.99	0.00162143504819226\\
3	0.00162143504819226\\
3.01	0.00162143504819226\\
3.02	0.00162143504819226\\
3.03	0.00162143504819226\\
3.04	0.00162143504819226\\
3.05	0.00162143504819226\\
3.06	0.00162143504819226\\
3.07	0.00162143504819226\\
3.08	0.00162143504819226\\
3.09	0.00162143504819226\\
3.1	0.00162143504819226\\
3.11	0.00162143504819226\\
3.12	0.00162143504819226\\
3.13	0.00162143504819226\\
3.14	0.00162143504819226\\
3.15	0.00162143504819226\\
3.16	0.00162143504819226\\
3.17	0.00162143504819226\\
3.18	0.00162143504819226\\
3.19	0.00162143504819226\\
3.2	0.00162143504819226\\
3.21	0.00162143504819226\\
3.22	0.00162143504819226\\
3.23	0.00162143504819226\\
3.24	0.00162143504819226\\
3.25	0.00162143504819226\\
3.26	0.00162143504819226\\
3.27	0.00162143504819226\\
3.28	0.00162143504819226\\
3.29	0.00162143504819226\\
3.3	0.00162143504819226\\
3.31	0.00162143504819226\\
3.32	0.00162143504819226\\
3.33	0.00162143504819226\\
3.34	0.00162143504819226\\
3.35	0.00162143504819226\\
3.36	0.00162143504819226\\
3.37	0.00162143504819226\\
3.38	0.00162143504819226\\
3.39	0.00162143504819226\\
3.4	0.00162143504819226\\
3.41	0.00162143504819226\\
3.42	0.00162143504819226\\
3.43	0.00162143504819226\\
3.44	0.00162143504819226\\
3.45	0.00162143504819226\\
3.46	0.00162143504819226\\
3.47	0.00162143504819226\\
3.48	0.00162143504819226\\
3.49	0.00162143504819226\\
3.5	0.00162143504819226\\
3.51	0.00162143504819226\\
3.52	0.00162143504819226\\
3.53	0.00162143504819226\\
3.54	0.00162143504819226\\
3.55	0.00162143504819226\\
3.56	0.00162143504819226\\
3.57	0.00162143504819226\\
3.58	0.00162143504819226\\
3.59	0.00162143504819226\\
3.6	0.00162143504819226\\
3.61	0.00162143504819226\\
3.62	0.00162143504819226\\
3.63	0.00162143504819226\\
3.64	0.00162143504819226\\
3.65	0.00162143504819226\\
3.66	0.00162143504819226\\
3.67	0.00162143504819226\\
3.68	0.00162143504819226\\
3.69	0.00162143504819226\\
3.7	0.00162143504819226\\
3.71	0.00162143504819226\\
3.72	0.00162143504819226\\
3.73	0.00162143504819226\\
3.74	0.00162143504819226\\
3.75	0.00162143504819226\\
3.76	0.00162143504819226\\
3.77	0.00162143504819226\\
3.78	0.00162143504819226\\
3.79	0.00162143504819226\\
3.8	0.00162143504819226\\
3.81	0.00162143504819226\\
3.82	0.00162143504819226\\
3.83	0.00162143504819226\\
3.84	0.00162143504819226\\
3.85	0.00162143504819226\\
3.86	0.00162143504819226\\
3.87	0.00162143504819226\\
3.88	0.00162143504819226\\
3.89	0.00162143504819226\\
3.9	0.00162143504819226\\
3.91	0.00162143504819226\\
3.92	0.00162143504819226\\
3.93	0.00162143504819226\\
3.94	0.00162143504819226\\
3.95	0.00162143504819226\\
3.96	0.00162143504819226\\
3.97	0.00162143504819226\\
3.98	0.00162143504819226\\
3.99	0.00162143504819226\\
4	0.00162143504819226\\
4.01	0.00162143504819226\\
4.02	0.00162143504819226\\
4.03	0.00162143504819226\\
4.04	0.00162143504819226\\
4.05	0.00162143504819226\\
4.06	0.00162143504819226\\
4.07	0.00162143504819226\\
4.08	0.00162143504819226\\
4.09	0.00162143504819226\\
4.1	0.00162143504819226\\
4.11	0.00162143504819226\\
4.12	0.00162143504819226\\
4.13	0.00162143504819226\\
4.14	0.00162143504819226\\
4.15	0.00162143504819226\\
4.16	0.00162143504819226\\
4.17	0.00162143504819226\\
4.18	0.00162143504819226\\
4.19	0.00162143504819226\\
4.2	0.00162143504819226\\
4.21	0.00162143504819226\\
4.22	0.00162143504819226\\
4.23	0.00162143504819226\\
4.24	0.00162143504819226\\
4.25	0.00162143504819226\\
4.26	0.00162143504819226\\
4.27	0.00162143504819226\\
4.28	0.00162143504819226\\
4.29	0.00162143504819226\\
4.3	0.00162143504819226\\
4.31	0.00162143504819226\\
4.32	0.00162143504819226\\
4.33	0.00162143504819226\\
4.34	0.00162143504819226\\
4.35	0.00162143504819226\\
4.36	0.00162143504819226\\
4.37	0.00162143504819226\\
4.38	0.00162143504819226\\
4.39	0.00162143504819226\\
4.4	0.00162143504819226\\
4.41	0.00162143504819226\\
4.42	0.00162143504819226\\
4.43	0.00162143504819226\\
4.44	0.00162143504819226\\
4.45	0.00162143504819226\\
4.46	0.00162143504819226\\
4.47	0.00162143504819226\\
4.48	0.00162143504819226\\
4.49	0.00162143504819226\\
4.5	0.00162143504819226\\
4.51	0.00162143504819226\\
4.52	0.00162143504819226\\
4.53	0.00162143504819226\\
4.54	0.00162143504819226\\
4.55	0.00162143504819226\\
4.56	0.00162143504819226\\
4.57	0.00162143504819226\\
4.58	0.00162143504819226\\
4.59	0.00162143504819226\\
4.6	0.00162143504819226\\
4.61	0.00162143504819226\\
4.62	0.00162143504819226\\
4.63	0.00162143504819226\\
4.64	0.00162143504819226\\
4.65	0.00162143504819226\\
4.66	0.00162143504819226\\
4.67	0.00162143504819226\\
4.68	0.00162143504819226\\
4.69	0.00162143504819226\\
4.7	0.00162143504819226\\
4.71	0.00162143504819226\\
4.72	0.00162143504819226\\
4.73	0.00162143504819226\\
4.74	0.00162143504819226\\
4.75	0.00162143504819226\\
4.76	0.00162143504819226\\
4.77	0.00162143504819226\\
4.78	0.00162143504819226\\
4.79	0.00162143504819226\\
4.8	0.00162143504819226\\
4.81	0.00162143504819226\\
4.82	0.00162143504819226\\
4.83	0.00162143504819226\\
4.84	0.00162143504819226\\
4.85	0.00162143504819226\\
4.86	0.00162143504819226\\
4.87	0.00162143504819226\\
4.88	0.00162143504819226\\
4.89	0.00162143504819226\\
4.9	0.00162143504819226\\
4.91	0.00162143504819226\\
4.92	0.00162143504819226\\
4.93	0.00162143504819226\\
4.94	0.00162143504819226\\
4.95	0.00162143504819226\\
4.96	0.00162143504819226\\
4.97	0.00162143504819226\\
4.98	0.00162143504819226\\
4.99	0.00162143504819226\\
5	0.00162143504819226\\
5.01	0.00162143504819226\\
5.02	0.00162143504819226\\
5.03	0.00162143504819226\\
5.04	0.00162143504819226\\
5.05	0.00162143504819226\\
5.06	0.00162143504819226\\
5.07	0.00162143504819226\\
5.08	0.00162143504819226\\
5.09	0.00162143504819226\\
5.1	0.00162143504819226\\
5.11	0.00162143504819226\\
5.12	0.00162143504819226\\
5.13	0.00162143504819226\\
5.14	0.00162143504819226\\
5.15	0.00162143504819226\\
5.16	0.00162143504819226\\
5.17	0.00162143504819226\\
5.18	0.00162143504819226\\
5.19	0.00162143504819226\\
5.2	0.00162143504819226\\
5.21	0.00162143504819226\\
5.22	0.00162143504819226\\
5.23	0.00162143504819226\\
5.24	0.00162143504819226\\
5.25	0.00162143504819226\\
5.26	0.00162143504819226\\
5.27	0.00162143504819226\\
5.28	0.00162143504819226\\
5.29	0.00162143504819226\\
5.3	0.00162143504819226\\
5.31	0.00162143504819226\\
5.32	0.00162143504819226\\
5.33	0.00162143504819226\\
5.34	0.00162143504819226\\
5.35	0.00162143504819226\\
5.36	0.00162143504819226\\
5.37	0.00162143504819226\\
5.38	0.00162143504819226\\
5.39	0.00162143504819226\\
5.4	0.00162143504819226\\
5.41	0.00162143504819226\\
5.42	0.00162143504819226\\
5.43	0.00162143504819226\\
5.44	0.00162143504819226\\
5.45	0.00162143504819226\\
5.46	0.00162143504819226\\
5.47	0.00162143504819226\\
5.48	0.00162143504819226\\
5.49	0.00162143504819226\\
5.5	0.00162143504819226\\
5.51	0.00162143504819226\\
5.52	0.00162143504819226\\
5.53	0.00162143504819226\\
5.54	0.00162143504819226\\
5.55	0.00162143504819226\\
5.56	0.00162143504819226\\
5.57	0.00162143504819226\\
5.58	0.00162143504819226\\
5.59	0.00162143504819226\\
5.6	0.00162143504819226\\
5.61	0.00162143504819226\\
5.62	0.00162143504819226\\
5.63	0.00162143504819226\\
5.64	0.00162143504819226\\
5.65	0.00162143504819226\\
5.66	0.00162143504819226\\
5.67	0.00162143504819226\\
5.68	0.00162143504819226\\
5.69	0.00162143504819226\\
5.7	0.00162143504819226\\
5.71	0.00162143504819226\\
5.72	0.00162143504819226\\
5.73	0.00162143504819226\\
5.74	0.00162143504819226\\
5.75	0.00162143504819226\\
5.76	0.00162143504819226\\
5.77	0.00162143504819226\\
5.78	0.00162143504819226\\
5.79	0.00162143504819226\\
5.8	0.00162143504819226\\
5.81	0.00162143504819226\\
5.82	0.00162143504819226\\
5.83	0.00162143504819226\\
5.84	0.00162143504819226\\
5.85	0.00162143504819226\\
5.86	0.00162143504819226\\
5.87	0.00162143504819226\\
5.88	0.00162143504819226\\
5.89	0.00162143504819226\\
5.9	0.00162143504819226\\
5.91	0.00162143504819226\\
5.92	0.00162143504819226\\
5.93	0.00162143504819226\\
5.94	0.00162143504819226\\
5.95	0.00162143504819226\\
5.96	0.00162143504819226\\
5.97	0.00162143504819226\\
5.98	0.00162143504819226\\
5.99	0.00162143504819226\\
6	0.00162143504819226\\
6.01	0.00162143504819226\\
6.02	0.00162143504819226\\
6.03	0.00162143504819226\\
6.04	0.00162143504819226\\
6.05	0.00162143504819226\\
6.06	0.00162143504819226\\
6.07	0.00162143504819226\\
6.08	0.00162143504819226\\
6.09	0.00162143504819226\\
6.1	0.00162143504819226\\
6.11	0.00162143504819226\\
6.12	0.00162143504819226\\
6.13	0.00162143504819226\\
6.14	0.00162143504819226\\
6.15	0.00162143504819226\\
6.16	0.00162143504819226\\
6.17	0.00162143504819226\\
6.18	0.00162143504819226\\
6.19	0.00162143504819226\\
6.2	0.00162143504819226\\
6.21	0.00162143504819226\\
6.22	0.00162143504819226\\
6.23	0.00162143504819226\\
6.24	0.00162143504819226\\
6.25	0.00162143504819226\\
6.26	0.00162143504819226\\
6.27	0.00162143504819226\\
6.28	0.00162143504819226\\
6.29	0.00162143504819226\\
6.3	0.00162143504819226\\
6.31	0.00162143504819226\\
6.32	0.00162143504819226\\
6.33	0.00162143504819226\\
6.34	0.00162143504819226\\
6.35	0.00162143504819226\\
6.36	0.00162143504819226\\
6.37	0.00162143504819226\\
6.38	0.00162143504819226\\
6.39	0.00162143504819226\\
6.4	0.00162143504819226\\
6.41	0.00162143504819226\\
6.42	0.00162143504819226\\
6.43	0.00162143504819226\\
6.44	0.00162143504819226\\
6.45	0.00162143504819226\\
6.46	0.00162143504819226\\
6.47	0.00162143504819226\\
6.48	0.00162143504819226\\
6.49	0.00162143504819226\\
6.5	0.00162143504819226\\
6.51	0.00162143504819226\\
6.52	0.00162143504819226\\
6.53	0.00162143504819226\\
6.54	0.00162143504819226\\
6.55	0.00162143504819226\\
6.56	0.00162143504819226\\
6.57	0.00162143504819226\\
6.58	0.00162143504819226\\
6.59	0.00162143504819226\\
6.6	0.00162143504819226\\
6.61	0.00162143504819226\\
6.62	0.00162143504819226\\
6.63	0.00162143504819226\\
6.64	0.00162143504819226\\
6.65	0.00162143504819226\\
6.66	0.00162143504819226\\
6.67	0.00162143504819226\\
6.68	0.00162143504819226\\
6.69	0.00162143504819226\\
6.7	0.00162143504819226\\
6.71	0.00162143504819226\\
6.72	0.00162143504819226\\
6.73	0.00162143504819226\\
6.74	0.00162143504819226\\
6.75	0.00162143504819226\\
6.76	0.00162143504819226\\
6.77	0.00162143504819226\\
6.78	0.00162143504819226\\
6.79	0.00162143504819226\\
6.8	0.00162143504819226\\
6.81	0.00162143504819226\\
6.82	0.00162143504819226\\
6.83	0.00162143504819226\\
6.84	0.00162143504819226\\
6.85	0.00162143504819226\\
6.86	0.00162143504819226\\
6.87	0.00162143504819226\\
6.88	0.00162143504819226\\
6.89	0.00162143504819226\\
6.9	0.00162143504819226\\
6.91	0.00162143504819226\\
6.92	0.00162143504819226\\
6.93	0.00162143504819226\\
6.94	0.00162143504819226\\
6.95	0.00162143504819226\\
6.96	0.00162143504819226\\
6.97	0.00162143504819226\\
6.98	0.00162143504819226\\
6.99	0.00162143504819226\\
7	0.00162143504819226\\
7.01	0.00162143504819226\\
7.02	0.00162143504819226\\
7.03	0.00162143504819226\\
7.04	0.00162143504819226\\
7.05	0.00162143504819226\\
7.06	0.00162143504819226\\
7.07	0.00162143504819226\\
7.08	0.00162143504819226\\
7.09	0.00162143504819226\\
7.1	0.00162143504819226\\
7.11	0.00162143504819226\\
7.12	0.00162143504819226\\
7.13	0.00162143504819226\\
7.14	0.00162143504819226\\
7.15	0.00162143504819226\\
7.16	0.00162143504819226\\
7.17	0.00162143504819226\\
7.18	0.00162143504819226\\
7.19	0.00162143504819226\\
7.2	0.00162143504819226\\
7.21	0.00162143504819226\\
7.22	0.00162143504819226\\
7.23	0.00162143504819226\\
7.24	0.00162143504819226\\
7.25	0.00162143504819226\\
7.26	0.00162143504819226\\
7.27	0.00162143504819226\\
7.28	0.00162143504819226\\
7.29	0.00162143504819226\\
7.3	0.00162143504819226\\
7.31	0.00162143504819226\\
7.32	0.00162143504819226\\
7.33	0.00162143504819226\\
7.34	0.00162143504819226\\
7.35	0.00162143504819226\\
7.36	0.00162143504819226\\
7.37	0.00162143504819226\\
7.38	0.00162143504819226\\
7.39	0.00162143504819226\\
7.4	0.00162143504819226\\
7.41	0.00162143504819226\\
7.42	0.00162143504819226\\
7.43	0.00162143504819226\\
7.44	0.00162143504819226\\
7.45	0.00162143504819226\\
7.46	0.00162143504819226\\
7.47	0.00162143504819226\\
7.48	0.00162143504819226\\
7.49	0.00162143504819226\\
7.5	0.00162143504819226\\
7.51	0.00162143504819226\\
7.52	0.00162143504819226\\
7.53	0.00162143504819226\\
7.54	0.00162143504819226\\
7.55	0.00162143504819226\\
7.56	0.00162143504819226\\
7.57	0.00162143504819226\\
7.58	0.00162143504819226\\
7.59	0.00162143504819226\\
7.6	0.00162143504819226\\
7.61	0.00162143504819226\\
7.62	0.00162143504819226\\
7.63	0.00162143504819226\\
7.64	0.00162143504819226\\
7.65	0.00162143504819226\\
7.66	0.00162143504819226\\
7.67	0.00162143504819226\\
7.68	0.00162143504819226\\
7.69	0.00162143504819226\\
7.7	0.00162143504819226\\
7.71	0.00162143504819226\\
7.72	0.00162143504819226\\
7.73	0.00162143504819226\\
7.74	0.00162143504819226\\
7.75	0.00162143504819226\\
7.76	0.00162143504819226\\
7.77	0.00162143504819226\\
7.78	0.00162143504819226\\
7.79	0.00162143504819226\\
7.8	0.00162143504819226\\
7.81	0.00162143504819226\\
7.82	0.00162143504819226\\
7.83	0.00162143504819226\\
7.84	0.00162143504819226\\
7.85	0.00162143504819226\\
7.86	0.00162143504819226\\
7.87	0.00162143504819226\\
7.88	0.00162143504819226\\
7.89	0.00162143504819226\\
7.9	0.00162143504819226\\
7.91	0.00162143504819226\\
7.92	0.00162143504819226\\
7.93	0.00162143504819226\\
7.94	0.00162143504819226\\
7.95	0.00162143504819226\\
7.96	0.00162143504819226\\
7.97	0.00162143504819226\\
7.98	0.00162143504819226\\
7.99	0.00162143504819226\\
8	0.00162143504819226\\
8.01	0.00162143504819226\\
8.02	0.00162143504819226\\
8.03	0.00162143504819226\\
8.04	0.00162143504819226\\
8.05	0.00162143504819226\\
8.06	0.00162143504819226\\
8.07	0.00162143504819226\\
8.08	0.00162143504819226\\
8.09	0.00162143504819226\\
8.1	0.00162143504819226\\
8.11	0.00162143504819226\\
8.12	0.00162143504819226\\
8.13	0.00162143504819226\\
8.14	0.00162143504819226\\
8.15	0.00162143504819226\\
8.16	0.00162143504819226\\
8.17	0.00162143504819226\\
8.18	0.00162143504819226\\
8.19	0.00162143504819226\\
8.2	0.00162143504819226\\
8.21	0.00162143504819226\\
8.22	0.00162143504819226\\
8.23	0.00162143504819226\\
8.24	0.00162143504819226\\
8.25	0.00162143504819226\\
8.26	0.00162143504819226\\
8.27	0.00162143504819226\\
8.28	0.00162143504819226\\
8.29	0.00162143504819226\\
8.3	0.00162143504819226\\
8.31	0.00162143504819226\\
8.32	0.00162143504819226\\
8.33	0.00162143504819226\\
8.34	0.00162143504819226\\
8.35	0.00162143504819226\\
8.36	0.00162143504819226\\
8.37	0.00162143504819226\\
8.38	0.00162143504819226\\
8.39	0.00162143504819226\\
8.4	0.00162143504819226\\
8.41	0.00162143504819226\\
8.42	0.00162143504819226\\
8.43	0.00162143504819226\\
8.44	0.00162143504819226\\
8.45	0.00162143504819226\\
8.46	0.00162143504819226\\
8.47	0.00162143504819226\\
8.48	0.00162143504819226\\
8.49	0.00162143504819226\\
8.5	0.00162143504819226\\
8.51	0.00162143504819226\\
8.52	0.00162143504819226\\
8.53	0.00162143504819226\\
8.54	0.00162143504819226\\
8.55	0.00162143504819226\\
8.56	0.00162143504819226\\
8.57	0.00162143504819226\\
8.58	0.00162143504819226\\
8.59	0.00162143504819226\\
8.6	0.00162143504819226\\
8.61	0.00162143504819226\\
8.62	0.00162143504819226\\
8.63	0.00162143504819226\\
8.64	0.00162143504819226\\
8.65	0.00162143504819226\\
8.66	0.00162143504819226\\
8.67	0.00162143504819226\\
8.68	0.00162143504819226\\
8.69	0.00162143504819226\\
8.7	0.00162143504819226\\
8.71	0.00162143504819226\\
8.72	0.00162143504819226\\
8.73	0.00162143504819226\\
8.74	0.00162143504819226\\
8.75	0.00162143504819226\\
8.76	0.00162143504819226\\
8.77	0.00162143504819226\\
8.78	0.00162143504819226\\
8.79	0.00162143504819226\\
8.8	0.00162143504819226\\
8.81	0.00162143504819226\\
8.82	0.00162143504819226\\
8.83	0.00162143504819226\\
8.84	0.00162143504819226\\
8.85	0.00162143504819226\\
8.86	0.00162143504819226\\
8.87	0.00162143504819226\\
8.88	0.00162143504819226\\
8.89	0.00162143504819226\\
8.9	0.00162143504819226\\
8.91	0.00162143504819226\\
8.92	0.00162143504819226\\
8.93	0.00162143504819226\\
8.94	0.00162143504819226\\
8.95	0.00162143504819226\\
8.96	0.00162143504819226\\
8.97	0.00162143504819226\\
8.98	0.00162143504819226\\
8.99	0.00162143504819226\\
9	0.00162143504819226\\
9.01	0.00162143504819226\\
9.02	0.00162143504819226\\
9.03	0.00162143504819226\\
9.04	0.00162143504819226\\
9.05	0.00162143504819226\\
9.06	0.00162143504819226\\
9.07	0.00162143504819226\\
9.08	0.00162143504819226\\
9.09	0.00162143504819226\\
9.1	0.00162143504819226\\
9.11	0.00162143504819226\\
9.12	0.00162143504819226\\
9.13	0.00162143504819226\\
9.14	0.00162143504819226\\
9.15	0.00162143504819226\\
9.16	0.00162143504819226\\
9.17	0.00162143504819226\\
9.18	0.00162143504819226\\
9.19	0.00162143504819226\\
9.2	0.00162143504819226\\
9.21	0.00162143504819226\\
9.22	0.00162143504819226\\
9.23	0.00162143504819226\\
9.24	0.00162143504819226\\
9.25	0.00162143504819226\\
9.26	0.00162143504819226\\
9.27	0.00162143504819226\\
9.28	0.00162143504819226\\
9.29	0.00162143504819226\\
9.3	0.00162143504819226\\
9.31	0.00162143504819226\\
9.32	0.00162143504819226\\
9.33	0.00162143504819226\\
9.34	0.00162143504819226\\
9.35	0.00162143504819226\\
9.36	0.00162143504819226\\
9.37	0.00162143504819226\\
9.38	0.00162143504819226\\
9.39	0.00162143504819226\\
9.4	0.00162143504819226\\
9.41	0.00162143504819226\\
9.42	0.00162143504819226\\
9.43	0.00162143504819226\\
9.44	0.00162143504819226\\
9.45	0.00162143504819226\\
9.46	0.00162143504819226\\
9.47	0.00162143504819226\\
9.48	0.00162143504819226\\
9.49	0.00162143504819226\\
9.5	0.00162143504819226\\
9.51	0.00162143504819226\\
9.52	0.00162143504819226\\
9.53	0.00162143504819226\\
9.54	0.00162143504819226\\
9.55	0.00162143504819226\\
9.56	0.00162143504819226\\
9.57	0.00162143504819226\\
9.58	0.00162143504819226\\
9.59	0.00162143504819226\\
9.6	0.00162143504819226\\
9.61	0.00162143504819226\\
9.62	0.00162143504819226\\
9.63	0.00162143504819226\\
9.64	0.00162143504819226\\
9.65	0.00162143504819226\\
9.66	0.00162143504819226\\
9.67	0.00162143504819226\\
9.68	0.00162143504819226\\
9.69	0.00162143504819226\\
9.7	0.00162143504819226\\
9.71	0.00162143504819226\\
9.72	0.00162143504819226\\
9.73	0.00162143504819226\\
9.74	0.00162143504819226\\
9.75	0.00162143504819226\\
9.76	0.00162143504819226\\
9.77	0.00162143504819226\\
9.78	0.00162143504819226\\
9.79	0.00162143504819226\\
9.8	0.00162143504819226\\
9.81	0.00162143504819226\\
9.82	0.00162143504819226\\
9.83	0.00162143504819226\\
9.84	0.00162143504819226\\
9.85	0.00162143504819226\\
9.86	0.00162143504819226\\
9.87	0.00162143504819226\\
9.88	0.00162143504819226\\
9.89	0.00162143504819226\\
9.9	0.00162143504819226\\
9.91	0.00162143504819226\\
9.92	0.00162143504819226\\
9.93	0.00162143504819226\\
9.94	0.00162143504819226\\
9.95	0.00162143504819226\\
9.96	0.00162143504819226\\
9.97	0.00162143504819226\\
9.98	0.00162143504819226\\
9.99	0.00162143504819226\\
10	0.00162143504819226\\
10.01	0.00162143504819226\\
10.02	0.00162143504819226\\
10.03	0.00162143504819226\\
10.04	0.00162143504819226\\
10.05	0.00162143504819226\\
10.06	0.00162143504819226\\
10.07	0.00162143504819226\\
10.08	0.00162143504819226\\
10.09	0.00162143504819226\\
10.1	0.00162143504819226\\
10.11	0.00162143504819226\\
10.12	0.00162143504819226\\
10.13	0.00162143504819226\\
10.14	0.00162143504819226\\
10.15	0.00162143504819226\\
10.16	0.00162143504819226\\
10.17	0.00162143504819226\\
10.18	0.00162143504819226\\
10.19	0.00162143504819226\\
10.2	0.00162143504819226\\
10.21	0.00162143504819226\\
10.22	0.00162143504819226\\
10.23	0.00162143504819226\\
10.24	0.00162143504819226\\
10.25	0.00162143504819226\\
10.26	0.00162143504819226\\
10.27	0.00162143504819226\\
10.28	0.00162143504819226\\
10.29	0.00162143504819226\\
10.3	0.00162143504819226\\
10.31	0.00162143504819226\\
10.32	0.00162143504819226\\
10.33	0.00162143504819226\\
10.34	0.00162143504819226\\
10.35	0.00162143504819226\\
10.36	0.00162143504819226\\
10.37	0.00162143504819226\\
10.38	0.00162143504819226\\
10.39	0.00162143504819226\\
10.4	0.00162143504819226\\
10.41	0.00162143504819226\\
10.42	0.00162143504819226\\
10.43	0.00162143504819226\\
10.44	0.00162143504819226\\
10.45	0.00162143504819226\\
10.46	0.00162143504819226\\
10.47	0.00162143504819226\\
10.48	0.00162143504819226\\
10.49	0.00162143504819226\\
10.5	0.00162143504819226\\
10.51	0.00162143504819226\\
10.52	0.00162143504819226\\
10.53	0.00162143504819226\\
10.54	0.00162143504819226\\
10.55	0.00162143504819226\\
10.56	0.00162143504819226\\
10.57	0.00162143504819226\\
10.58	0.00162143504819226\\
10.59	0.00162143504819226\\
10.6	0.00162143504819226\\
10.61	0.00162143504819226\\
10.62	0.00162143504819226\\
10.63	0.00162143504819226\\
10.64	0.00162143504819226\\
10.65	0.00162143504819226\\
10.66	0.00162143504819226\\
10.67	0.00162143504819226\\
10.68	0.00162143504819226\\
10.69	0.00162143504819226\\
10.7	0.00162143504819226\\
10.71	0.00162143504819226\\
10.72	0.00162143504819226\\
10.73	0.00162143504819226\\
10.74	0.00162143504819226\\
10.75	0.00162143504819226\\
10.76	0.00162143504819226\\
10.77	0.00162143504819226\\
10.78	0.00162143504819226\\
10.79	0.00162143504819226\\
10.8	0.00162143504819226\\
10.81	0.00162143504819226\\
10.82	0.00162143504819226\\
10.83	0.00162143504819226\\
10.84	0.00162143504819226\\
10.85	0.00162143504819226\\
10.86	0.00162143504819226\\
10.87	0.00162143504819226\\
10.88	0.00162143504819226\\
10.89	0.00162143504819226\\
10.9	0.00162143504819226\\
10.91	0.00162143504819226\\
10.92	0.00162143504819226\\
10.93	0.00162143504819226\\
10.94	0.00162143504819226\\
10.95	0.00162143504819226\\
10.96	0.00162143504819226\\
10.97	0.00162143504819226\\
10.98	0.00162143504819226\\
10.99	0.00162143504819226\\
11	0.00162143504819226\\
11.01	0.00162143504819226\\
11.02	0.00162143504819226\\
11.03	0.00162143504819226\\
11.04	0.00162143504819226\\
11.05	0.00162143504819226\\
11.06	0.00162143504819226\\
11.07	0.00162143504819226\\
11.08	0.00162143504819226\\
11.09	0.00162143504819226\\
11.1	0.00162143504819226\\
11.11	0.00162143504819226\\
11.12	0.00162143504819226\\
11.13	0.00162143504819226\\
11.14	0.00162143504819226\\
11.15	0.00162143504819226\\
11.16	0.00162143504819226\\
11.17	0.00162143504819226\\
11.18	0.00162143504819226\\
11.19	0.00162143504819226\\
11.2	0.00162143504819226\\
11.21	0.00162143504819226\\
11.22	0.00162143504819226\\
11.23	0.00162143504819226\\
11.24	0.00162143504819226\\
11.25	0.00162143504819226\\
11.26	0.00162143504819226\\
11.27	0.00162143504819226\\
11.28	0.00162143504819226\\
11.29	0.00162143504819226\\
11.3	0.00162143504819226\\
11.31	0.00162143504819226\\
11.32	0.00162143504819226\\
11.33	0.00162143504819226\\
11.34	0.00162143504819226\\
11.35	0.00162143504819226\\
11.36	0.00162143504819226\\
11.37	0.00162143504819226\\
11.38	0.00162143504819226\\
11.39	0.00162143504819226\\
11.4	0.00162143504819226\\
11.41	0.00162143504819226\\
11.42	0.00162143504819226\\
11.43	0.00162143504819226\\
11.44	0.00162143504819226\\
11.45	0.00162143504819226\\
11.46	0.00162143504819226\\
11.47	0.00162143504819226\\
11.48	0.00162143504819226\\
11.49	0.00162143504819226\\
11.5	0.00162143504819226\\
11.51	0.00162143504819226\\
11.52	0.00162143504819226\\
11.53	0.00162143504819226\\
11.54	0.00162143504819226\\
11.55	0.00162143504819226\\
11.56	0.00162143504819226\\
11.57	0.00162143504819226\\
11.58	0.00162143504819226\\
11.59	0.00162143504819226\\
11.6	0.00162143504819226\\
11.61	0.00162143504819226\\
11.62	0.00162143504819226\\
11.63	0.00162143504819226\\
11.64	0.00162143504819226\\
11.65	0.00162143504819226\\
11.66	0.00162143504819226\\
11.67	0.00162143504819226\\
11.68	0.00162143504819226\\
11.69	0.00162143504819226\\
11.7	0.00162143504819226\\
11.71	0.00162143504819226\\
11.72	0.00162143504819226\\
11.73	0.00162143504819226\\
11.74	0.00162143504819226\\
11.75	0.00162143504819226\\
11.76	0.00162143504819226\\
11.77	0.00162143504819226\\
11.78	0.00162143504819226\\
11.79	0.00162143504819226\\
11.8	0.00162143504819226\\
11.81	0.00162143504819226\\
11.82	0.00162143504819226\\
11.83	0.00162143504819226\\
11.84	0.00162143504819226\\
11.85	0.00162143504819226\\
11.86	0.00162143504819226\\
11.87	0.00162143504819226\\
11.88	0.00162143504819226\\
11.89	0.00162143504819226\\
11.9	0.00162143504819226\\
11.91	0.00162143504819226\\
11.92	0.00162143504819226\\
11.93	0.00162143504819226\\
11.94	0.00162143504819226\\
11.95	0.00162143504819226\\
11.96	0.00162143504819226\\
11.97	0.00162143504819226\\
11.98	0.00162143504819226\\
11.99	0.00162143504819226\\
12	0.00162143504819226\\
12.01	0.00162143504819226\\
12.02	0.00162143504819226\\
12.03	0.00162143504819226\\
12.04	0.00162143504819226\\
12.05	0.00162143504819226\\
12.06	0.00162143504819226\\
12.07	0.00162143504819226\\
12.08	0.00162143504819226\\
12.09	0.00162143504819226\\
12.1	0.00162143504819226\\
12.11	0.00162143504819226\\
12.12	0.00162143504819226\\
12.13	0.00162143504819226\\
12.14	0.00162143504819226\\
12.15	0.00162143504819226\\
12.16	0.00162143504819226\\
12.17	0.00162143504819226\\
12.18	0.00162143504819226\\
12.19	0.00162143504819226\\
12.2	0.00162143504819226\\
12.21	0.00162143504819226\\
12.22	0.00162143504819226\\
12.23	0.00162143504819226\\
12.24	0.00162143504819226\\
12.25	0.00162143504819226\\
12.26	0.00162143504819226\\
12.27	0.00162143504819226\\
12.28	0.00162143504819226\\
12.29	0.00162143504819226\\
12.3	0.00162143504819226\\
12.31	0.00162143504819226\\
12.32	0.00162143504819226\\
12.33	0.00162143504819226\\
12.34	0.00162143504819226\\
12.35	0.00162143504819226\\
12.36	0.00162143504819226\\
12.37	0.00162143504819226\\
12.38	0.00162143504819226\\
12.39	0.00162143504819226\\
12.4	0.00162143504819226\\
12.41	0.00162143504819226\\
12.42	0.00162143504819226\\
12.43	0.00162143504819226\\
12.44	0.00162143504819226\\
12.45	0.00162143504819226\\
12.46	0.00162143504819226\\
12.47	0.00162143504819226\\
12.48	0.00162143504819226\\
12.49	0.00162143504819226\\
12.5	0.00162143504819226\\
12.51	0.00162143504819226\\
12.52	0.00162143504819226\\
12.53	0.00162143504819226\\
12.54	0.00162143504819226\\
12.55	0.00162143504819226\\
12.56	0.00162143504819226\\
12.57	0.00162143504819226\\
12.58	0.00162143504819226\\
12.59	0.00162143504819226\\
12.6	0.00162143504819226\\
12.61	0.00162143504819226\\
12.62	0.00162143504819226\\
12.63	0.00162143504819226\\
12.64	0.00162143504819226\\
12.65	0.00162143504819226\\
12.66	0.00162143504819226\\
12.67	0.00162143504819226\\
12.68	0.00162143504819226\\
12.69	0.00162143504819226\\
12.7	0.00162143504819226\\
12.71	0.00162143504819226\\
12.72	0.00162143504819226\\
12.73	0.00162143504819226\\
12.74	0.00162143504819226\\
12.75	0.00162143504819226\\
12.76	0.00162143504819226\\
12.77	0.00162143504819226\\
12.78	0.00162143504819226\\
12.79	0.00162143504819226\\
12.8	0.00162143504819226\\
12.81	0.00162143504819226\\
12.82	0.00162143504819226\\
12.83	0.00162143504819226\\
12.84	0.00162143504819226\\
12.85	0.00162143504819226\\
12.86	0.00162143504819226\\
12.87	0.00162143504819226\\
12.88	0.00162143504819226\\
12.89	0.00162143504819226\\
12.9	0.00162143504819226\\
12.91	0.00162143504819226\\
12.92	0.00162143504819226\\
12.93	0.00162143504819226\\
12.94	0.00162143504819226\\
12.95	0.00162143504819226\\
12.96	0.00162143504819226\\
12.97	0.00162143504819226\\
12.98	0.00162143504819226\\
12.99	0.00162143504819226\\
13	0.00162143504819226\\
13.01	0.00162143504819226\\
13.02	0.00162143504819226\\
13.03	0.00162143504819226\\
13.04	0.00162143504819226\\
13.05	0.00162143504819226\\
13.06	0.00162143504819226\\
13.07	0.00162143504819226\\
13.08	0.00162143504819226\\
13.09	0.00162143504819226\\
13.1	0.00162143504819226\\
13.11	0.00162143504819226\\
13.12	0.00162143504819226\\
13.13	0.00162143504819226\\
13.14	0.00162143504819226\\
13.15	0.00162143504819226\\
13.16	0.00162143504819226\\
13.17	0.00162143504819226\\
13.18	0.00162143504819226\\
13.19	0.00162143504819226\\
13.2	0.00162143504819226\\
13.21	0.00162143504819226\\
13.22	0.00162143504819226\\
13.23	0.00162143504819226\\
13.24	0.00162143504819226\\
13.25	0.00162143504819226\\
13.26	0.00162143504819226\\
13.27	0.00162143504819226\\
13.28	0.00162143504819226\\
13.29	0.00162143504819226\\
13.3	0.00162143504819226\\
13.31	0.00162143504819226\\
13.32	0.00162143504819226\\
13.33	0.00162143504819226\\
13.34	0.00162143504819226\\
13.35	0.00162143504819226\\
13.36	0.00162143504819226\\
13.37	0.00162143504819226\\
13.38	0.00162143504819226\\
13.39	0.00162143504819226\\
13.4	0.00162143504819226\\
13.41	0.00162143504819226\\
13.42	0.00162143504819226\\
13.43	0.00162143504819226\\
13.44	0.00162143504819226\\
13.45	0.00162143504819226\\
13.46	0.00162143504819226\\
13.47	0.00162143504819226\\
13.48	0.00162143504819226\\
13.49	0.00162143504819226\\
13.5	0.00162143504819226\\
13.51	0.00162143504819226\\
13.52	0.00162143504819226\\
13.53	0.00162143504819226\\
13.54	0.00162143504819226\\
13.55	0.00162143504819226\\
13.56	0.00162143504819226\\
13.57	0.00162143504819226\\
13.58	0.00162143504819226\\
13.59	0.00162143504819226\\
13.6	0.00162143504819226\\
13.61	0.00162143504819226\\
13.62	0.00162143504819226\\
13.63	0.00162143504819226\\
13.64	0.00162143504819226\\
13.65	0.00162143504819226\\
13.66	0.00162143504819226\\
13.67	0.00162143504819226\\
13.68	0.00162143504819226\\
13.69	0.00162143504819226\\
13.7	0.00162143504819226\\
13.71	0.00162143504819226\\
13.72	0.00162143504819226\\
13.73	0.00162143504819226\\
13.74	0.00162143504819226\\
13.75	0.00162143504819226\\
13.76	0.00162143504819226\\
13.77	0.00162143504819226\\
13.78	0.00162143504819226\\
13.79	0.00162143504819226\\
13.8	0.00162143504819226\\
13.81	0.00162143504819226\\
13.82	0.00162143504819226\\
13.83	0.00162143504819226\\
13.84	0.00162143504819226\\
13.85	0.00162143504819226\\
13.86	0.00162143504819226\\
13.87	0.00162143504819226\\
13.88	0.00162143504819226\\
13.89	0.00162143504819226\\
13.9	0.00162143504819226\\
13.91	0.00162143504819226\\
13.92	0.00162143504819226\\
13.93	0.00162143504819226\\
13.94	0.00162143504819226\\
13.95	0.00162143504819226\\
13.96	0.00162143504819226\\
13.97	0.00162143504819226\\
13.98	0.00162143504819226\\
13.99	0.00162143504819226\\
14	0.00162143504819226\\
14.01	0.00162143504819226\\
14.02	0.00162143504819226\\
14.03	0.00162143504819226\\
14.04	0.00162143504819226\\
14.05	0.00162143504819226\\
14.06	0.00162143504819226\\
14.07	0.00162143504819226\\
14.08	0.00162143504819226\\
14.09	0.00162143504819226\\
14.1	0.00162143504819226\\
14.11	0.00162143504819226\\
14.12	0.00162143504819226\\
14.13	0.00162143504819226\\
14.14	0.00162143504819226\\
14.15	0.00162143504819226\\
14.16	0.00162143504819226\\
14.17	0.00162143504819226\\
14.18	0.00162143504819226\\
14.19	0.00162143504819226\\
14.2	0.00162143504819226\\
14.21	0.00162143504819226\\
14.22	0.00162143504819226\\
14.23	0.00162143504819226\\
14.24	0.00162143504819226\\
14.25	0.00162143504819226\\
14.26	0.00162143504819226\\
14.27	0.00162143504819226\\
14.28	0.00162143504819226\\
14.29	0.00162143504819226\\
14.3	0.00162143504819226\\
14.31	0.00162143504819226\\
14.32	0.00162143504819226\\
14.33	0.00162143504819226\\
14.34	0.00162143504819226\\
14.35	0.00162143504819226\\
14.36	0.00162143504819226\\
14.37	0.00162143504819226\\
14.38	0.00162143504819226\\
14.39	0.00162143504819226\\
14.4	0.00162143504819226\\
14.41	0.00162143504819226\\
14.42	0.00162143504819226\\
14.43	0.00162143504819226\\
14.44	0.00162143504819226\\
14.45	0.00162143504819226\\
14.46	0.00162143504819226\\
14.47	0.00162143504819226\\
14.48	0.00162143504819226\\
14.49	0.00162143504819226\\
14.5	0.00162143504819226\\
14.51	0.00162143504819226\\
14.52	0.00162143504819226\\
14.53	0.00162143504819226\\
14.54	0.00162143504819226\\
14.55	0.00162143504819226\\
14.56	0.00162143504819226\\
14.57	0.00162143504819226\\
14.58	0.00162143504819226\\
14.59	0.00162143504819226\\
14.6	0.00162143504819226\\
14.61	0.00162143504819226\\
14.62	0.00162143504819226\\
14.63	0.00162143504819226\\
14.64	0.00162143504819226\\
14.65	0.00162143504819226\\
14.66	0.00162143504819226\\
14.67	0.00162143504819226\\
14.68	0.00162143504819226\\
14.69	0.00162143504819226\\
14.7	0.00162143504819226\\
14.71	0.00162143504819226\\
14.72	0.00162143504819226\\
14.73	0.00162143504819226\\
14.74	0.00162143504819226\\
14.75	0.00162143504819226\\
14.76	0.00162143504819226\\
14.77	0.00162143504819226\\
14.78	0.00162143504819226\\
14.79	0.00162143504819226\\
14.8	0.00162143504819226\\
14.81	0.00162143504819226\\
14.82	0.00162143504819226\\
14.83	0.00162143504819226\\
14.84	0.00162143504819226\\
14.85	0.00162143504819226\\
14.86	0.00162143504819226\\
14.87	0.00162143504819226\\
14.88	0.00162143504819226\\
14.89	0.00162143504819226\\
14.9	0.00162143504819226\\
14.91	0.00162143504819226\\
14.92	0.00162143504819226\\
14.93	0.00162143504819226\\
14.94	0.00162143504819226\\
14.95	0.00162143504819226\\
14.96	0.00162143504819226\\
14.97	0.00162143504819226\\
14.98	0.00162143504819226\\
14.99	0.00162143504819226\\
15	0.00162143504819226\\
15.01	0.00162143504819226\\
15.02	0.00162143504819226\\
15.03	0.00162143504819226\\
15.04	0.00162143504819226\\
15.05	0.00162143504819226\\
15.06	0.00162143504819226\\
15.07	0.00162143504819226\\
15.08	0.00162143504819226\\
15.09	0.00162143504819226\\
15.1	0.00162143504819226\\
15.11	0.00162143504819226\\
15.12	0.00162143504819226\\
15.13	0.00162143504819226\\
15.14	0.00162143504819226\\
15.15	0.00162143504819226\\
15.16	0.00162143504819226\\
15.17	0.00162143504819226\\
15.18	0.00162143504819226\\
15.19	0.00162143504819226\\
15.2	0.00162143504819226\\
15.21	0.00162143504819226\\
15.22	0.00162143504819226\\
15.23	0.00162143504819226\\
15.24	0.00162143504819226\\
15.25	0.00162143504819226\\
15.26	0.00162143504819226\\
15.27	0.00162143504819226\\
15.28	0.00162143504819226\\
15.29	0.00162143504819226\\
15.3	0.00162143504819226\\
15.31	0.00162143504819226\\
15.32	0.00162143504819226\\
15.33	0.00162143504819226\\
15.34	0.00162143504819226\\
15.35	0.00162143504819226\\
15.36	0.00162143504819226\\
15.37	0.00162143504819226\\
15.38	0.00162143504819226\\
15.39	0.00162143504819226\\
15.4	0.00162143504819226\\
15.41	0.00162143504819226\\
15.42	0.00162143504819226\\
15.43	0.00162143504819226\\
15.44	0.00162143504819226\\
15.45	0.00162143504819226\\
15.46	0.00162143504819226\\
15.47	0.00162143504819226\\
15.48	0.00162143504819226\\
15.49	0.00162143504819226\\
15.5	0.00162143504819226\\
15.51	0.00162143504819226\\
15.52	0.00162143504819226\\
15.53	0.00162143504819226\\
15.54	0.00162143504819226\\
15.55	0.00162143504819226\\
15.56	0.00162143504819226\\
15.57	0.00162143504819226\\
15.58	0.00162143504819226\\
15.59	0.00162143504819226\\
15.6	0.00162143504819226\\
15.61	0.00162143504819226\\
15.62	0.00162143504819226\\
15.63	0.00162143504819226\\
15.64	0.00162143504819226\\
15.65	0.00162143504819226\\
15.66	0.00162143504819226\\
15.67	0.00162143504819226\\
15.68	0.00162143504819226\\
15.69	0.00162143504819226\\
15.7	0.00162143504819226\\
15.71	0.00162143504819226\\
15.72	0.00162143504819226\\
15.73	0.00162143504819226\\
15.74	0.00162143504819226\\
15.75	0.00162143504819226\\
15.76	0.00162143504819226\\
15.77	0.00162143504819226\\
15.78	0.00162143504819226\\
15.79	0.00162143504819226\\
15.8	0.00162143504819226\\
15.81	0.00162143504819226\\
15.82	0.00162143504819226\\
15.83	0.00162143504819226\\
15.84	0.00162143504819226\\
15.85	0.00162143504819226\\
15.86	0.00162143504819226\\
15.87	0.00162143504819226\\
15.88	0.00162143504819226\\
15.89	0.00162143504819226\\
15.9	0.00162143504819226\\
15.91	0.00162143504819226\\
15.92	0.00162143504819226\\
15.93	0.00162143504819226\\
15.94	0.00162143504819226\\
15.95	0.00162143504819226\\
15.96	0.00162143504819226\\
15.97	0.00162143504819226\\
15.98	0.00162143504819226\\
15.99	0.00162143504819226\\
16	0.00162143504819226\\
16.01	0.00162143504819226\\
16.02	0.00162143504819226\\
16.03	0.00162143504819226\\
16.04	0.00162143504819226\\
16.05	0.00162143504819226\\
16.06	0.00162143504819226\\
16.07	0.00162143504819226\\
16.08	0.00162143504819226\\
16.09	0.00162143504819226\\
16.1	0.00162143504819226\\
16.11	0.00162143504819226\\
16.12	0.00162143504819226\\
16.13	0.00162143504819226\\
16.14	0.00162143504819226\\
16.15	0.00162143504819226\\
16.16	0.00162143504819226\\
16.17	0.00162143504819226\\
16.18	0.00162143504819226\\
16.19	0.00162143504819226\\
16.2	0.00162143504819226\\
16.21	0.00162143504819226\\
16.22	0.00162143504819226\\
16.23	0.00162143504819226\\
16.24	0.00162143504819226\\
16.25	0.00162143504819226\\
16.26	0.00162143504819226\\
16.27	0.00162143504819226\\
16.28	0.00162143504819226\\
16.29	0.00162143504819226\\
16.3	0.00162143504819226\\
16.31	0.00162143504819226\\
16.32	0.00162143504819226\\
16.33	0.00162143504819226\\
16.34	0.00162143504819226\\
16.35	0.00162143504819226\\
16.36	0.00162143504819226\\
16.37	0.00162143504819226\\
16.38	0.00162143504819226\\
16.39	0.00162143504819226\\
16.4	0.00162143504819226\\
16.41	0.00162143504819226\\
16.42	0.00162143504819226\\
16.43	0.00162143504819226\\
16.44	0.00162143504819226\\
16.45	0.00162143504819226\\
16.46	0.00162143504819226\\
16.47	0.00162143504819226\\
16.48	0.00162143504819226\\
16.49	0.00162143504819226\\
16.5	0.00162143504819226\\
16.51	0.00162143504819226\\
16.52	0.00162143504819226\\
16.53	0.00162143504819226\\
16.54	0.00162143504819226\\
16.55	0.00162143504819226\\
16.56	0.00162143504819226\\
16.57	0.00162143504819226\\
16.58	0.00162143504819226\\
16.59	0.00162143504819226\\
16.6	0.00162143504819226\\
16.61	0.00162143504819226\\
16.62	0.00162143504819226\\
16.63	0.00162143504819226\\
16.64	0.00162143504819226\\
16.65	0.00162143504819226\\
16.66	0.00162143504819226\\
16.67	0.00162143504819226\\
16.68	0.00162143504819226\\
16.69	0.00162143504819226\\
16.7	0.00162143504819226\\
16.71	0.00162143504819226\\
16.72	0.00162143504819226\\
16.73	0.00162143504819226\\
16.74	0.00162143504819226\\
16.75	0.00162143504819226\\
16.76	0.00162143504819226\\
16.77	0.00162143504819226\\
16.78	0.00162143504819226\\
16.79	0.00162143504819226\\
16.8	0.00162143504819226\\
16.81	0.00162143504819226\\
16.82	0.00162143504819226\\
16.83	0.00162143504819226\\
16.84	0.00162143504819226\\
16.85	0.00162143504819226\\
16.86	0.00162143504819226\\
16.87	0.00162143504819226\\
16.88	0.00162143504819226\\
16.89	0.00162143504819226\\
16.9	0.00162143504819226\\
16.91	0.00162143504819226\\
16.92	0.00162143504819226\\
16.93	0.00162143504819226\\
16.94	0.00162143504819226\\
16.95	0.00162143504819226\\
16.96	0.00162143504819226\\
16.97	0.00162143504819226\\
16.98	0.00162143504819226\\
16.99	0.00162143504819226\\
17	0.00162143504819226\\
17.01	0.00162143504819226\\
17.02	0.00162143504819226\\
17.03	0.00162143504819226\\
17.04	0.00162143504819226\\
17.05	0.00162143504819226\\
17.06	0.00162143504819226\\
17.07	0.00162143504819226\\
17.08	0.00162143504819226\\
17.09	0.00162143504819226\\
17.1	0.00162143504819226\\
17.11	0.00162143504819226\\
17.12	0.00162143504819226\\
17.13	0.00162143504819226\\
17.14	0.00162143504819226\\
17.15	0.00162143504819226\\
17.16	0.00162143504819226\\
17.17	0.00162143504819226\\
17.18	0.00162143504819226\\
17.19	0.00162143504819226\\
17.2	0.00162143504819226\\
17.21	0.00162143504819226\\
17.22	0.00162143504819226\\
17.23	0.00162143504819226\\
17.24	0.00162143504819226\\
17.25	0.00162143504819226\\
17.26	0.00162143504819226\\
17.27	0.00162143504819226\\
17.28	0.00162143504819226\\
17.29	0.00162143504819226\\
17.3	0.00162143504819226\\
17.31	0.00162143504819226\\
17.32	0.00162143504819226\\
17.33	0.00162143504819226\\
17.34	0.00162143504819226\\
17.35	0.00162143504819226\\
17.36	0.00162143504819226\\
17.37	0.00162143504819226\\
17.38	0.00162143504819226\\
17.39	0.00162143504819226\\
17.4	0.00162143504819226\\
17.41	0.00162143504819226\\
17.42	0.00162143504819226\\
17.43	0.00162143504819226\\
17.44	0.00162143504819226\\
17.45	0.00162143504819226\\
17.46	0.00162143504819226\\
17.47	0.00162143504819226\\
17.48	0.00162143504819226\\
17.49	0.00162143504819226\\
17.5	0.00162143504819226\\
17.51	0.00162143504819226\\
17.52	0.00162143504819226\\
17.53	0.00162143504819226\\
17.54	0.00162143504819226\\
17.55	0.00162143504819226\\
17.56	0.00162143504819226\\
17.57	0.00162143504819226\\
17.58	0.00162143504819226\\
17.59	0.00162143504819226\\
17.6	0.00162143504819226\\
17.61	0.00162143504819226\\
17.62	0.00162143504819226\\
17.63	0.00162143504819226\\
17.64	0.00162143504819226\\
17.65	0.00162143504819226\\
17.66	0.00162143504819226\\
17.67	0.00162143504819226\\
17.68	0.00162143504819226\\
17.69	0.00162143504819226\\
17.7	0.00162143504819226\\
17.71	0.00162143504819226\\
17.72	0.00162143504819226\\
17.73	0.00162143504819226\\
17.74	0.00162143504819226\\
17.75	0.00162143504819226\\
17.76	0.00162143504819226\\
17.77	0.00162143504819226\\
17.78	0.00162143504819226\\
17.79	0.00162143504819226\\
17.8	0.00162143504819226\\
17.81	0.00162143504819226\\
17.82	0.00162143504819226\\
17.83	0.00162143504819226\\
17.84	0.00162143504819226\\
17.85	0.00162143504819226\\
17.86	0.00162143504819226\\
17.87	0.00162143504819226\\
17.88	0.00162143504819226\\
17.89	0.00162143504819226\\
17.9	0.00162143504819226\\
17.91	0.00162143504819226\\
17.92	0.00162143504819226\\
17.93	0.00162143504819226\\
17.94	0.00162143504819226\\
17.95	0.00162143504819226\\
17.96	0.00162143504819226\\
17.97	0.00162143504819226\\
17.98	0.00162143504819226\\
17.99	0.00162143504819226\\
18	0.00162143504819226\\
18.01	0.00162143504819226\\
18.02	0.00162143504819226\\
18.03	0.00162143504819226\\
18.04	0.00162143504819226\\
18.05	0.00162143504819226\\
18.06	0.00162143504819226\\
18.07	0.00162143504819226\\
18.08	0.00162143504819226\\
18.09	0.00162143504819226\\
18.1	0.00162143504819226\\
18.11	0.00162143504819226\\
18.12	0.00162143504819226\\
18.13	0.00162143504819226\\
18.14	0.00162143504819226\\
18.15	0.00162143504819226\\
18.16	0.00162143504819226\\
18.17	0.00162143504819226\\
18.18	0.00162143504819226\\
18.19	0.00162143504819226\\
18.2	0.00162143504819226\\
18.21	0.00162143504819226\\
18.22	0.00162143504819226\\
18.23	0.00162143504819226\\
18.24	0.00162143504819226\\
18.25	0.00162143504819226\\
18.26	0.00162143504819226\\
18.27	0.00162143504819226\\
18.28	0.00162143504819226\\
18.29	0.00162143504819226\\
18.3	0.00162143504819226\\
18.31	0.00162143504819226\\
18.32	0.00162143504819226\\
18.33	0.00162143504819226\\
18.34	0.00162143504819226\\
18.35	0.00162143504819226\\
18.36	0.00162143504819226\\
18.37	0.00162143504819226\\
18.38	0.00162143504819226\\
18.39	0.00162143504819226\\
18.4	0.00162143504819226\\
18.41	0.00162143504819226\\
18.42	0.00162143504819226\\
18.43	0.00162143504819226\\
18.44	0.00162143504819226\\
18.45	0.00162143504819226\\
18.46	0.00162143504819226\\
18.47	0.00162143504819226\\
18.48	0.00162143504819226\\
18.49	0.00162143504819226\\
18.5	0.00162143504819226\\
18.51	0.00162143504819226\\
18.52	0.00162143504819226\\
18.53	0.00162143504819226\\
18.54	0.00162143504819226\\
18.55	0.00162143504819226\\
18.56	0.00162143504819226\\
18.57	0.00162143504819226\\
18.58	0.00162143504819226\\
18.59	0.00162143504819226\\
18.6	0.00162143504819226\\
18.61	0.00162143504819226\\
18.62	0.00162143504819226\\
18.63	0.00162143504819226\\
18.64	0.00162143504819226\\
18.65	0.00162143504819226\\
18.66	0.00162143504819226\\
18.67	0.00162143504819226\\
18.68	0.00162143504819226\\
18.69	0.00162143504819226\\
18.7	0.00162143504819226\\
18.71	0.00162143504819226\\
18.72	0.00162143504819226\\
18.73	0.00162143504819226\\
18.74	0.00162143504819226\\
18.75	0.00162143504819226\\
18.76	0.00162143504819226\\
18.77	0.00162143504819226\\
18.78	0.00162143504819226\\
18.79	0.00162143504819226\\
18.8	0.00162143504819226\\
18.81	0.00162143504819226\\
18.82	0.00162143504819226\\
18.83	0.00162143504819226\\
18.84	0.00162143504819226\\
18.85	0.00162143504819226\\
18.86	0.00162143504819226\\
18.87	0.00162143504819226\\
18.88	0.00162143504819226\\
18.89	0.00162143504819226\\
18.9	0.00162143504819226\\
18.91	0.00162143504819226\\
18.92	0.00162143504819226\\
18.93	0.00162143504819226\\
18.94	0.00162143504819226\\
18.95	0.00162143504819226\\
18.96	0.00162143504819226\\
18.97	0.00162143504819226\\
18.98	0.00162143504819226\\
18.99	0.00162143504819226\\
19	0.00162143504819226\\
19.01	0.00162143504819226\\
19.02	0.00162143504819226\\
19.03	0.00162143504819226\\
19.04	0.00162143504819226\\
19.05	0.00162143504819226\\
19.06	0.00162143504819226\\
19.07	0.00162143504819226\\
19.08	0.00162143504819226\\
19.09	0.00162143504819226\\
19.1	0.00162143504819226\\
19.11	0.00162143504819226\\
19.12	0.00162143504819226\\
19.13	0.00162143504819226\\
19.14	0.00162143504819226\\
19.15	0.00162143504819226\\
19.16	0.00162143504819226\\
19.17	0.00162143504819226\\
19.18	0.00162143504819226\\
19.19	0.00162143504819226\\
19.2	0.00162143504819226\\
19.21	0.00162143504819226\\
19.22	0.00162143504819226\\
19.23	0.00162143504819226\\
19.24	0.00162143504819226\\
19.25	0.00162143504819226\\
19.26	0.00162143504819226\\
19.27	0.00162143504819226\\
19.28	0.00162143504819226\\
19.29	0.00162143504819226\\
19.3	0.00162143504819226\\
19.31	0.00162143504819226\\
19.32	0.00162143504819226\\
19.33	0.00162143504819226\\
19.34	0.00162143504819226\\
19.35	0.00162143504819226\\
19.36	0.00162143504819226\\
19.37	0.00162143504819226\\
19.38	0.00162143504819226\\
19.39	0.00162143504819226\\
19.4	0.00162143504819226\\
19.41	0.00162143504819226\\
19.42	0.00162143504819226\\
19.43	0.00162143504819226\\
19.44	0.00162143504819226\\
19.45	0.00162143504819226\\
19.46	0.00162143504819226\\
19.47	0.00162143504819226\\
19.48	0.00162143504819226\\
19.49	0.00162143504819226\\
19.5	0.00162143504819226\\
19.51	0.00162143504819226\\
19.52	0.00162143504819226\\
19.53	0.00162143504819226\\
19.54	0.00162143504819226\\
19.55	0.00162143504819226\\
19.56	0.00162143504819226\\
19.57	0.00162143504819226\\
19.58	0.00162143504819226\\
19.59	0.00162143504819226\\
19.6	0.00162143504819226\\
19.61	0.00162143504819226\\
19.62	0.00162143504819226\\
19.63	0.00162143504819226\\
19.64	0.00162143504819226\\
19.65	0.00162143504819226\\
19.66	0.00162143504819226\\
19.67	0.00162143504819226\\
19.68	0.00162143504819226\\
19.69	0.00162143504819226\\
19.7	0.00162143504819226\\
19.71	0.00162143504819226\\
19.72	0.00162143504819226\\
19.73	0.00162143504819226\\
19.74	0.00162143504819226\\
19.75	0.00162143504819226\\
19.76	0.00162143504819226\\
19.77	0.00162143504819226\\
19.78	0.00162143504819226\\
19.79	0.00162143504819226\\
19.8	0.00162143504819226\\
19.81	0.00162143504819226\\
19.82	0.00162143504819226\\
19.83	0.00162143504819226\\
19.84	0.00162143504819226\\
19.85	0.00162143504819226\\
19.86	0.00162143504819226\\
19.87	0.00162143504819226\\
19.88	0.00162143504819226\\
19.89	0.00162143504819226\\
19.9	0.00162143504819226\\
19.91	0.00162143504819226\\
19.92	0.00162143504819226\\
19.93	0.00162143504819226\\
19.94	0.00162143504819226\\
19.95	0.00162143504819226\\
19.96	0.00162143504819226\\
19.97	0.00162143504819226\\
19.98	0.00162143504819226\\
19.99	0.00162143504819226\\
20	0.00162143504819226\\
20.01	0.00162143504819226\\
20.02	0.00162143504819226\\
20.03	0.00162143504819226\\
20.04	0.00162143504819226\\
20.05	0.00162143504819226\\
20.06	0.00162143504819226\\
20.07	0.00162143504819226\\
20.08	0.00162143504819226\\
20.09	0.00162143504819226\\
20.1	0.00162143504819226\\
20.11	0.00162143504819226\\
20.12	0.00162143504819226\\
20.13	0.00162143504819226\\
20.14	0.00162143504819226\\
20.15	0.00162143504819226\\
20.16	0.00162143504819226\\
20.17	0.00162143504819226\\
20.18	0.00162143504819226\\
20.19	0.00162143504819226\\
20.2	0.00162143504819226\\
20.21	0.00162143504819226\\
20.22	0.00162143504819226\\
20.23	0.00162143504819226\\
20.24	0.00162143504819226\\
20.25	0.00162143504819226\\
20.26	0.00162143504819226\\
20.27	0.00162143504819226\\
20.28	0.00162143504819226\\
20.29	0.00162143504819226\\
20.3	0.00162143504819226\\
20.31	0.00162143504819226\\
20.32	0.00162143504819226\\
20.33	0.00162143504819226\\
20.34	0.00162143504819226\\
20.35	0.00162143504819226\\
20.36	0.00162143504819226\\
20.37	0.00162143504819226\\
20.38	0.00162143504819226\\
20.39	0.00162143504819226\\
20.4	0.00162143504819226\\
20.41	0.00162143504819226\\
20.42	0.00162143504819226\\
20.43	0.00162143504819226\\
20.44	0.00162143504819226\\
20.45	0.00162143504819226\\
20.46	0.00162143504819226\\
20.47	0.00162143504819226\\
20.48	0.00162143504819226\\
20.49	0.00162143504819226\\
20.5	0.00162143504819226\\
20.51	0.00162143504819226\\
20.52	0.00162143504819226\\
20.53	0.00162143504819226\\
20.54	0.00162143504819226\\
20.55	0.00162143504819226\\
20.56	0.00162143504819226\\
20.57	0.00162143504819226\\
20.58	0.00162143504819226\\
20.59	0.00162143504819226\\
20.6	0.00162143504819226\\
20.61	0.00162143504819226\\
20.62	0.00162143504819226\\
20.63	0.00162143504819226\\
20.64	0.00162143504819226\\
20.65	0.00162143504819226\\
20.66	0.00162143504819226\\
20.67	0.00162143504819226\\
20.68	0.00162143504819226\\
20.69	0.00162143504819226\\
20.7	0.00162143504819226\\
20.71	0.00162143504819226\\
20.72	0.00162143504819226\\
20.73	0.00162143504819226\\
20.74	0.00162143504819226\\
20.75	0.00162143504819226\\
20.76	0.00162143504819226\\
20.77	0.00162143504819226\\
20.78	0.00162143504819226\\
20.79	0.00162143504819226\\
20.8	0.00162143504819226\\
20.81	0.00162143504819226\\
20.82	0.00162143504819226\\
20.83	0.00162143504819226\\
20.84	0.00162143504819226\\
20.85	0.00162143504819226\\
20.86	0.00162143504819226\\
20.87	0.00162143504819226\\
20.88	0.00162143504819226\\
20.89	0.00162143504819226\\
20.9	0.00162143504819226\\
20.91	0.00162143504819226\\
20.92	0.00162143504819226\\
20.93	0.00162143504819226\\
20.94	0.00162143504819226\\
20.95	0.00162143504819226\\
20.96	0.00162143504819226\\
20.97	0.00162143504819226\\
20.98	0.00162143504819226\\
20.99	0.00162143504819226\\
21	0.00162143504819226\\
21.01	0.00162143504819226\\
21.02	0.00162143504819226\\
21.03	0.00162143504819226\\
21.04	0.00162143504819226\\
21.05	0.00162143504819226\\
21.06	0.00162143504819226\\
21.07	0.00162143504819226\\
21.08	0.00162143504819226\\
21.09	0.00162143504819226\\
21.1	0.00162143504819226\\
21.11	0.00162143504819226\\
21.12	0.00162143504819226\\
21.13	0.00162143504819226\\
21.14	0.00162143504819226\\
21.15	0.00162143504819226\\
21.16	0.00162143504819226\\
21.17	0.00162143504819226\\
21.18	0.00162143504819226\\
21.19	0.00162143504819226\\
21.2	0.00162143504819226\\
21.21	0.00162143504819226\\
21.22	0.00162143504819226\\
21.23	0.00162143504819226\\
21.24	0.00162143504819226\\
21.25	0.00162143504819226\\
21.26	0.00162143504819226\\
21.27	0.00162143504819226\\
21.28	0.00162143504819226\\
21.29	0.00162143504819226\\
21.3	0.00162143504819226\\
21.31	0.00162143504819226\\
21.32	0.00162143504819226\\
21.33	0.00162143504819226\\
21.34	0.00162143504819226\\
21.35	0.00162143504819226\\
21.36	0.00162143504819226\\
21.37	0.00162143504819226\\
21.38	0.00162143504819226\\
21.39	0.00162143504819226\\
21.4	0.00162143504819226\\
21.41	0.00162143504819226\\
21.42	0.00162143504819226\\
21.43	0.00162143504819226\\
21.44	0.00162143504819226\\
21.45	0.00162143504819226\\
21.46	0.00162143504819226\\
21.47	0.00162143504819226\\
21.48	0.00162143504819226\\
21.49	0.00162143504819226\\
21.5	0.00162143504819226\\
21.51	0.00162143504819226\\
21.52	0.00162143504819226\\
21.53	0.00162143504819226\\
21.54	0.00162143504819226\\
21.55	0.00162143504819226\\
21.56	0.00162143504819226\\
21.57	0.00162143504819226\\
21.58	0.00162143504819226\\
21.59	0.00162143504819226\\
21.6	0.00162143504819226\\
21.61	0.00162143504819226\\
21.62	0.00162143504819226\\
21.63	0.00162143504819226\\
21.64	0.00162143504819226\\
21.65	0.00162143504819226\\
21.66	0.00162143504819226\\
21.67	0.00162143504819226\\
21.68	0.00162143504819226\\
21.69	0.00162143504819226\\
21.7	0.00162143504819226\\
21.71	0.00162143504819226\\
21.72	0.00162143504819226\\
21.73	0.00162143504819226\\
21.74	0.00162143504819226\\
21.75	0.00162143504819226\\
21.76	0.00162143504819226\\
21.77	0.00162143504819226\\
21.78	0.00162143504819226\\
21.79	0.00162143504819226\\
21.8	0.00162143504819226\\
21.81	0.00162143504819226\\
21.82	0.00162143504819226\\
21.83	0.00162143504819226\\
21.84	0.00162143504819226\\
21.85	0.00162143504819226\\
21.86	0.00162143504819226\\
21.87	0.00162143504819226\\
21.88	0.00162143504819226\\
21.89	0.00162143504819226\\
21.9	0.00162143504819226\\
21.91	0.00162143504819226\\
21.92	0.00162143504819226\\
21.93	0.00162143504819226\\
21.94	0.00162143504819226\\
21.95	0.00162143504819226\\
21.96	0.00162143504819226\\
21.97	0.00162143504819226\\
21.98	0.00162143504819226\\
21.99	0.00162143504819226\\
22	0.00162143504819226\\
22.01	0.00162143504819226\\
22.02	0.00162143504819226\\
22.03	0.00162143504819226\\
22.04	0.00162143504819226\\
22.05	0.00162143504819226\\
22.06	0.00162143504819226\\
22.07	0.00162143504819226\\
22.08	0.00162143504819226\\
22.09	0.00162143504819226\\
22.1	0.00162143504819226\\
22.11	0.00162143504819226\\
22.12	0.00162143504819226\\
22.13	0.00162143504819226\\
22.14	0.00162143504819226\\
22.15	0.00162143504819226\\
22.16	0.00162143504819226\\
22.17	0.00162143504819226\\
22.18	0.00162143504819226\\
22.19	0.00162143504819226\\
22.2	0.00162143504819226\\
22.21	0.00162143504819226\\
22.22	0.00162143504819226\\
22.23	0.00162143504819226\\
22.24	0.00162143504819226\\
22.25	0.00162143504819226\\
22.26	0.00162143504819226\\
22.27	0.00162143504819226\\
22.28	0.00162143504819226\\
22.29	0.00162143504819226\\
22.3	0.00162143504819226\\
22.31	0.00162143504819226\\
22.32	0.00162143504819226\\
22.33	0.00162143504819226\\
22.34	0.00162143504819226\\
22.35	0.00162143504819226\\
22.36	0.00162143504819226\\
22.37	0.00162143504819226\\
22.38	0.00162143504819226\\
22.39	0.00162143504819226\\
22.4	0.00162143504819226\\
22.41	0.00162143504819226\\
22.42	0.00162143504819226\\
22.43	0.00162143504819226\\
22.44	0.00162143504819226\\
22.45	0.00162143504819226\\
22.46	0.00162143504819226\\
22.47	0.00162143504819226\\
22.48	0.00162143504819226\\
22.49	0.00162143504819226\\
22.5	0.00162143504819226\\
22.51	0.00162143504819226\\
22.52	0.00162143504819226\\
22.53	0.00162143504819226\\
22.54	0.00162143504819226\\
22.55	0.00162143504819226\\
22.56	0.00162143504819226\\
22.57	0.00162143504819226\\
22.58	0.00162143504819226\\
22.59	0.00162143504819226\\
22.6	0.00162143504819226\\
22.61	0.00162143504819226\\
22.62	0.00162143504819226\\
22.63	0.00162143504819226\\
22.64	0.00162143504819226\\
22.65	0.00162143504819226\\
22.66	0.00162143504819226\\
22.67	0.00162143504819226\\
22.68	0.00162143504819226\\
22.69	0.00162143504819226\\
22.7	0.00162143504819226\\
22.71	0.00162143504819226\\
22.72	0.00162143504819226\\
22.73	0.00162143504819226\\
22.74	0.00162143504819226\\
22.75	0.00162143504819226\\
22.76	0.00162143504819226\\
22.77	0.00162143504819226\\
22.78	0.00162143504819226\\
22.79	0.00162143504819226\\
22.8	0.00162143504819226\\
22.81	0.00162143504819226\\
22.82	0.00162143504819226\\
22.83	0.00162143504819226\\
22.84	0.00162143504819226\\
22.85	0.00162143504819226\\
22.86	0.00162143504819226\\
22.87	0.00162143504819226\\
22.88	0.00162143504819226\\
22.89	0.00162143504819226\\
22.9	0.00162143504819226\\
22.91	0.00162143504819226\\
22.92	0.00162143504819226\\
22.93	0.00162143504819226\\
22.94	0.00162143504819226\\
22.95	0.00162143504819226\\
22.96	0.00162143504819226\\
22.97	0.00162143504819226\\
22.98	0.00162143504819226\\
22.99	0.00162143504819226\\
23	0.00162143504819226\\
23.01	0.00162143504819226\\
23.02	0.00162143504819226\\
23.03	0.00162143504819226\\
23.04	0.00162143504819226\\
23.05	0.00162143504819226\\
23.06	0.00162143504819226\\
23.07	0.00162143504819226\\
23.08	0.00162143504819226\\
23.09	0.00162143504819226\\
23.1	0.00162143504819226\\
23.11	0.00162143504819226\\
23.12	0.00162143504819226\\
23.13	0.00162143504819226\\
23.14	0.00162143504819226\\
23.15	0.00162143504819226\\
23.16	0.00162143504819226\\
23.17	0.00162143504819226\\
23.18	0.00162143504819226\\
23.19	0.00162143504819226\\
23.2	0.00162143504819226\\
23.21	0.00162143504819226\\
23.22	0.00162143504819226\\
23.23	0.00162143504819226\\
23.24	0.00162143504819226\\
23.25	0.00162143504819226\\
23.26	0.00162143504819226\\
23.27	0.00162143504819226\\
23.28	0.00162143504819226\\
23.29	0.00162143504819226\\
23.3	0.00162143504819226\\
23.31	0.00162143504819226\\
23.32	0.00162143504819226\\
23.33	0.00162143504819226\\
23.34	0.00162143504819226\\
23.35	0.00162143504819226\\
23.36	0.00162143504819226\\
23.37	0.00162143504819226\\
23.38	0.00162143504819226\\
23.39	0.00162143504819226\\
23.4	0.00162143504819226\\
23.41	0.00162143504819226\\
23.42	0.00162143504819226\\
23.43	0.00162143504819226\\
23.44	0.00162143504819226\\
23.45	0.00162143504819226\\
23.46	0.00162143504819226\\
23.47	0.00162143504819226\\
23.48	0.00162143504819226\\
23.49	0.00162143504819226\\
23.5	0.00162143504819226\\
23.51	0.00162143504819226\\
23.52	0.00162143504819226\\
23.53	0.00162143504819226\\
23.54	0.00162143504819226\\
23.55	0.00162143504819226\\
23.56	0.00162143504819226\\
23.57	0.00162143504819226\\
23.58	0.00162143504819226\\
23.59	0.00162143504819226\\
23.6	0.00162143504819226\\
23.61	0.00162143504819226\\
23.62	0.00162143504819226\\
23.63	0.00162143504819226\\
23.64	0.00162143504819226\\
23.65	0.00162143504819226\\
23.66	0.00162143504819226\\
23.67	0.00162143504819226\\
23.68	0.00162143504819226\\
23.69	0.00162143504819226\\
23.7	0.00162143504819226\\
23.71	0.00162143504819226\\
23.72	0.00162143504819226\\
23.73	0.00162143504819226\\
23.74	0.00162143504819226\\
23.75	0.00162143504819226\\
23.76	0.00162143504819226\\
23.77	0.00162143504819226\\
23.78	0.00162143504819226\\
23.79	0.00162143504819226\\
23.8	0.00162143504819226\\
23.81	0.00162143504819226\\
23.82	0.00162143504819226\\
23.83	0.00162143504819226\\
23.84	0.00162143504819226\\
23.85	0.00162143504819226\\
23.86	0.00162143504819226\\
23.87	0.00162143504819226\\
23.88	0.00162143504819226\\
23.89	0.00162143504819226\\
23.9	0.00162143504819226\\
23.91	0.00162143504819226\\
23.92	0.00162143504819226\\
23.93	0.00162143504819226\\
23.94	0.00162143504819226\\
23.95	0.00162143504819226\\
23.96	0.00162143504819226\\
23.97	0.00162143504819226\\
23.98	0.00162143504819226\\
23.99	0.00162143504819226\\
24	0.00162143504819226\\
24.01	0.00162143504819226\\
24.02	0.00162143504819226\\
24.03	0.00162143504819226\\
24.04	0.00162143504819226\\
24.05	0.00162143504819226\\
24.06	0.00162143504819226\\
24.07	0.00162143504819226\\
24.08	0.00162143504819226\\
24.09	0.00162143504819226\\
24.1	0.00162143504819226\\
24.11	0.00162143504819226\\
24.12	0.00162143504819226\\
24.13	0.00162143504819226\\
24.14	0.00162143504819226\\
24.15	0.00162143504819226\\
24.16	0.00162143504819226\\
24.17	0.00162143504819226\\
24.18	0.00162143504819226\\
24.19	0.00162143504819226\\
24.2	0.00162143504819226\\
24.21	0.00162143504819226\\
24.22	0.00162143504819226\\
24.23	0.00162143504819226\\
24.24	0.00162143504819226\\
24.25	0.00162143504819226\\
24.26	0.00162143504819226\\
24.27	0.00162143504819226\\
24.28	0.00162143504819226\\
24.29	0.00162143504819226\\
24.3	0.00162143504819226\\
24.31	0.00162143504819226\\
24.32	0.00162143504819226\\
24.33	0.00162143504819226\\
24.34	0.00162143504819226\\
24.35	0.00162143504819226\\
24.36	0.00162143504819226\\
24.37	0.00162143504819226\\
24.38	0.00162143504819226\\
24.39	0.00162143504819226\\
24.4	0.00162143504819226\\
24.41	0.00162143504819226\\
24.42	0.00162143504819226\\
24.43	0.00162143504819226\\
24.44	0.00162143504819226\\
24.45	0.00162143504819226\\
24.46	0.00162143504819226\\
24.47	0.00162143504819226\\
24.48	0.00162143504819226\\
24.49	0.00162143504819226\\
24.5	0.00162143504819226\\
24.51	0.00162143504819226\\
24.52	0.00162143504819226\\
24.53	0.00162143504819226\\
24.54	0.00162143504819226\\
24.55	0.00162143504819226\\
24.56	0.00162143504819226\\
24.57	0.00162143504819226\\
24.58	0.00162143504819226\\
24.59	0.00162143504819226\\
24.6	0.00162143504819226\\
24.61	0.00162143504819226\\
24.62	0.00162143504819226\\
24.63	0.00162143504819226\\
24.64	0.00162143504819226\\
24.65	0.00162143504819226\\
24.66	0.00162143504819226\\
24.67	0.00162143504819226\\
24.68	0.00162143504819226\\
24.69	0.00162143504819226\\
24.7	0.00162143504819226\\
24.71	0.00162143504819226\\
24.72	0.00162143504819226\\
24.73	0.00162143504819226\\
24.74	0.00162143504819226\\
24.75	0.00162143504819226\\
24.76	0.00162143504819226\\
24.77	0.00162143504819226\\
24.78	0.00162143504819226\\
24.79	0.00162143504819226\\
24.8	0.00162143504819226\\
24.81	0.00162143504819226\\
24.82	0.00162143504819226\\
24.83	0.00162143504819226\\
24.84	0.00162143504819226\\
24.85	0.00162143504819226\\
24.86	0.00162143504819226\\
24.87	0.00162143504819226\\
24.88	0.00162143504819226\\
24.89	0.00162143504819226\\
24.9	0.00162143504819226\\
24.91	0.00162143504819226\\
24.92	0.00162143504819226\\
24.93	0.00162143504819226\\
24.94	0.00162143504819226\\
24.95	0.00162143504819226\\
24.96	0.00162143504819226\\
24.97	0.00162143504819226\\
24.98	0.00162143504819226\\
24.99	0.00162143504819226\\
25	0.00162143504819226\\
25.01	0.00162143504819226\\
25.02	0.00162143504819226\\
25.03	0.00162143504819226\\
25.04	0.00162143504819226\\
25.05	0.00162143504819226\\
25.06	0.00162143504819226\\
25.07	0.00162143504819226\\
25.08	0.00162143504819226\\
25.09	0.00162143504819226\\
25.1	0.00162143504819226\\
25.11	0.00162143504819226\\
25.12	0.00162143504819226\\
25.13	0.00162143504819226\\
25.14	0.00162143504819226\\
25.15	0.00162143504819226\\
25.16	0.00162143504819226\\
25.17	0.00162143504819226\\
25.18	0.00162143504819226\\
25.19	0.00162143504819226\\
25.2	0.00162143504819226\\
25.21	0.00162143504819226\\
25.22	0.00162143504819226\\
25.23	0.00162143504819226\\
25.24	0.00162143504819226\\
25.25	0.00162143504819226\\
25.26	0.00162143504819226\\
25.27	0.00162143504819226\\
25.28	0.00162143504819226\\
25.29	0.00162143504819226\\
25.3	0.00162143504819226\\
25.31	0.00162143504819226\\
25.32	0.00162143504819226\\
25.33	0.00162143504819226\\
25.34	0.00162143504819226\\
25.35	0.00162143504819226\\
25.36	0.00162143504819226\\
25.37	0.00162143504819226\\
25.38	0.00162143504819226\\
25.39	0.00162143504819226\\
25.4	0.00162143504819226\\
25.41	0.00162143504819226\\
25.42	0.00162143504819226\\
25.43	0.00162143504819226\\
25.44	0.00162143504819226\\
25.45	0.00162143504819226\\
25.46	0.00162143504819226\\
25.47	0.00162143504819226\\
25.48	0.00162143504819226\\
25.49	0.00162143504819226\\
25.5	0.00162143504819226\\
25.51	0.00162143504819226\\
25.52	0.00162143504819226\\
25.53	0.00162143504819226\\
25.54	0.00162143504819226\\
25.55	0.00162143504819226\\
25.56	0.00162143504819226\\
25.57	0.00162143504819226\\
25.58	0.00162143504819226\\
25.59	0.00162143504819226\\
25.6	0.00162143504819226\\
25.61	0.00162143504819226\\
25.62	0.00162143504819226\\
25.63	0.00162143504819226\\
25.64	0.00162143504819226\\
25.65	0.00162143504819226\\
25.66	0.00162143504819226\\
25.67	0.00162143504819226\\
25.68	0.00162143504819226\\
25.69	0.00162143504819226\\
25.7	0.00162143504819226\\
25.71	0.00162143504819226\\
25.72	0.00162143504819226\\
25.73	0.00162143504819226\\
25.74	0.00162143504819226\\
25.75	0.00162143504819226\\
25.76	0.00162143504819226\\
25.77	0.00162143504819226\\
25.78	0.00162143504819226\\
25.79	0.00162143504819226\\
25.8	0.00162143504819226\\
25.81	0.00162143504819226\\
25.82	0.00162143504819226\\
25.83	0.00162143504819226\\
25.84	0.00162143504819226\\
25.85	0.00162143504819226\\
25.86	0.00162143504819226\\
25.87	0.00162143504819226\\
25.88	0.00162143504819226\\
25.89	0.00162143504819226\\
25.9	0.00162143504819226\\
25.91	0.00162143504819226\\
25.92	0.00162143504819226\\
25.93	0.00162143504819226\\
25.94	0.00162143504819226\\
25.95	0.00162143504819226\\
25.96	0.00162143504819226\\
25.97	0.00162143504819226\\
25.98	0.00162143504819226\\
25.99	0.00162143504819226\\
26	0.00162143504819226\\
26.01	0.00162143504819226\\
26.02	0.00162143504819226\\
26.03	0.00162143504819226\\
26.04	0.00162143504819226\\
26.05	0.00162143504819226\\
26.06	0.00162143504819226\\
26.07	0.00162143504819226\\
26.08	0.00162143504819226\\
26.09	0.00162143504819226\\
26.1	0.00162143504819226\\
26.11	0.00162143504819226\\
26.12	0.00162143504819226\\
26.13	0.00162143504819226\\
26.14	0.00162143504819226\\
26.15	0.00162143504819226\\
26.16	0.00162143504819226\\
26.17	0.00162143504819226\\
26.18	0.00162143504819226\\
26.19	0.00162143504819226\\
26.2	0.00162143504819226\\
26.21	0.00162143504819226\\
26.22	0.00162143504819226\\
26.23	0.00162143504819226\\
26.24	0.00162143504819226\\
26.25	0.00162143504819226\\
26.26	0.00162143504819226\\
26.27	0.00162143504819226\\
26.28	0.00162143504819226\\
26.29	0.00162143504819226\\
26.3	0.00162143504819226\\
26.31	0.00162143504819226\\
26.32	0.00162143504819226\\
26.33	0.00162143504819226\\
26.34	0.00162143504819226\\
26.35	0.00162143504819226\\
26.36	0.00162143504819226\\
26.37	0.00162143504819226\\
26.38	0.00162143504819226\\
26.39	0.00162143504819226\\
26.4	0.00162143504819226\\
26.41	0.00162143504819226\\
26.42	0.00162143504819226\\
26.43	0.00162143504819226\\
26.44	0.00162143504819226\\
26.45	0.00162143504819226\\
26.46	0.00162143504819226\\
26.47	0.00162143504819226\\
26.48	0.00162143504819226\\
26.49	0.00162143504819226\\
26.5	0.00162143504819226\\
26.51	0.00162143504819226\\
26.52	0.00162143504819226\\
26.53	0.00162143504819226\\
26.54	0.00162143504819226\\
26.55	0.00162143504819226\\
26.56	0.00162143504819226\\
26.57	0.00162143504819226\\
26.58	0.00162143504819226\\
26.59	0.00162143504819226\\
26.6	0.00162143504819226\\
26.61	0.00162143504819226\\
26.62	0.00162143504819226\\
26.63	0.00162143504819226\\
26.64	0.00162143504819226\\
26.65	0.00162143504819226\\
26.66	0.00162143504819226\\
26.67	0.00162143504819226\\
26.68	0.00162143504819226\\
26.69	0.00162143504819226\\
26.7	0.00162143504819226\\
26.71	0.00162143504819226\\
26.72	0.00162143504819226\\
26.73	0.00162143504819226\\
26.74	0.00162143504819226\\
26.75	0.00162143504819226\\
26.76	0.00162143504819226\\
26.77	0.00162143504819226\\
26.78	0.00162143504819226\\
26.79	0.00162143504819226\\
26.8	0.00162143504819226\\
26.81	0.00162143504819226\\
26.82	0.00162143504819226\\
26.83	0.00162143504819226\\
26.84	0.00162143504819226\\
26.85	0.00162143504819226\\
26.86	0.00162143504819226\\
26.87	0.00162143504819226\\
26.88	0.00162143504819226\\
26.89	0.00162143504819226\\
26.9	0.00162143504819226\\
26.91	0.00162143504819226\\
26.92	0.00162143504819226\\
26.93	0.00162143504819226\\
26.94	0.00162143504819226\\
26.95	0.00162143504819226\\
26.96	0.00162143504819226\\
26.97	0.00162143504819226\\
26.98	0.00162143504819226\\
26.99	0.00162143504819226\\
27	0.00162143504819226\\
27.01	0.00162143504819226\\
27.02	0.00162143504819226\\
27.03	0.00162143504819226\\
27.04	0.00162143504819226\\
27.05	0.00162143504819226\\
27.06	0.00162143504819226\\
27.07	0.00162143504819226\\
27.08	0.00162143504819226\\
27.09	0.00162143504819226\\
27.1	0.00162143504819226\\
27.11	0.00162143504819226\\
27.12	0.00162143504819226\\
27.13	0.00162143504819226\\
27.14	0.00162143504819226\\
27.15	0.00162143504819226\\
27.16	0.00162143504819226\\
27.17	0.00162143504819226\\
27.18	0.00162143504819226\\
27.19	0.00162143504819226\\
27.2	0.00162143504819226\\
27.21	0.00162143504819226\\
27.22	0.00162143504819226\\
27.23	0.00162143504819226\\
27.24	0.00162143504819226\\
27.25	0.00162143504819226\\
27.26	0.00162143504819226\\
27.27	0.00162143504819226\\
27.28	0.00162143504819226\\
27.29	0.00162143504819226\\
27.3	0.00162143504819226\\
27.31	0.00162143504819226\\
27.32	0.00162143504819226\\
27.33	0.00162143504819226\\
27.34	0.00162143504819226\\
27.35	0.00162143504819226\\
27.36	0.00162143504819226\\
27.37	0.00162143504819226\\
27.38	0.00162143504819226\\
27.39	0.00162143504819226\\
27.4	0.00162143504819226\\
27.41	0.00162143504819226\\
27.42	0.00162143504819226\\
27.43	0.00162143504819226\\
27.44	0.00162143504819226\\
27.45	0.00162143504819226\\
27.46	0.00162143504819226\\
27.47	0.00162143504819226\\
27.48	0.00162143504819226\\
27.49	0.00162143504819226\\
27.5	0.00162143504819226\\
27.51	0.00162143504819226\\
27.52	0.00162143504819226\\
27.53	0.00162143504819226\\
27.54	0.00162143504819226\\
27.55	0.00162143504819226\\
27.56	0.00162143504819226\\
27.57	0.00162143504819226\\
27.58	0.00162143504819226\\
27.59	0.00162143504819226\\
27.6	0.00162143504819226\\
27.61	0.00162143504819226\\
27.62	0.00162143504819226\\
27.63	0.00162143504819226\\
27.64	0.00162143504819226\\
27.65	0.00162143504819226\\
27.66	0.00162143504819226\\
27.67	0.00162143504819226\\
27.68	0.00162143504819226\\
27.69	0.00162143504819226\\
27.7	0.00162143504819226\\
27.71	0.00162143504819226\\
27.72	0.00162143504819226\\
27.73	0.00162143504819226\\
27.74	0.00162143504819226\\
27.75	0.00162143504819226\\
27.76	0.00162143504819226\\
27.77	0.00162143504819226\\
27.78	0.00162143504819226\\
27.79	0.00162143504819226\\
27.8	0.00162143504819226\\
27.81	0.00162143504819226\\
27.82	0.00162143504819226\\
27.83	0.00162143504819226\\
27.84	0.00162143504819226\\
27.85	0.00162143504819226\\
27.86	0.00162143504819226\\
27.87	0.00162143504819226\\
27.88	0.00162143504819226\\
27.89	0.00162143504819226\\
27.9	0.00162143504819226\\
27.91	0.00162143504819226\\
27.92	0.00162143504819226\\
27.93	0.00162143504819226\\
27.94	0.00162143504819226\\
27.95	0.00162143504819226\\
27.96	0.00162143504819226\\
27.97	0.00162143504819226\\
27.98	0.00162143504819226\\
27.99	0.00162143504819226\\
28	0.00162143504819226\\
28.01	0.00162143504819226\\
28.02	0.00162143504819226\\
28.03	0.00162143504819226\\
28.04	0.00162143504819226\\
28.05	0.00162143504819226\\
28.06	0.00162143504819226\\
28.07	0.00162143504819226\\
28.08	0.00162143504819226\\
28.09	0.00162143504819226\\
28.1	0.00162143504819226\\
28.11	0.00162143504819226\\
28.12	0.00162143504819226\\
28.13	0.00162143504819226\\
28.14	0.00162143504819226\\
28.15	0.00162143504819226\\
28.16	0.00162143504819226\\
28.17	0.00162143504819226\\
28.18	0.00162143504819226\\
28.19	0.00162143504819226\\
28.2	0.00162143504819226\\
28.21	0.00162143504819226\\
28.22	0.00162143504819226\\
28.23	0.00162143504819226\\
28.24	0.00162143504819226\\
28.25	0.00162143504819226\\
28.26	0.00162143504819226\\
28.27	0.00162143504819226\\
28.28	0.00162143504819226\\
28.29	0.00162143504819226\\
28.3	0.00162143504819226\\
28.31	0.00162143504819226\\
28.32	0.00162143504819226\\
28.33	0.00162143504819226\\
28.34	0.00162143504819226\\
28.35	0.00162143504819226\\
28.36	0.00162143504819226\\
28.37	0.00162143504819226\\
28.38	0.00162143504819226\\
28.39	0.00162143504819226\\
28.4	0.00162143504819226\\
28.41	0.00162143504819226\\
28.42	0.00162143504819226\\
28.43	0.00162143504819226\\
28.44	0.00162143504819226\\
28.45	0.00162143504819226\\
28.46	0.00162143504819226\\
28.47	0.00162143504819226\\
28.48	0.00162143504819226\\
28.49	0.00162143504819226\\
28.5	0.00162143504819226\\
28.51	0.00162143504819226\\
28.52	0.00162143504819226\\
28.53	0.00162143504819226\\
28.54	0.00162143504819226\\
28.55	0.00162143504819226\\
28.56	0.00162143504819226\\
28.57	0.00162143504819226\\
28.58	0.00162143504819226\\
28.59	0.00162143504819226\\
28.6	0.00162143504819226\\
28.61	0.00162143504819226\\
28.62	0.00162143504819226\\
28.63	0.00162143504819226\\
28.64	0.00162143504819226\\
28.65	0.00162143504819226\\
28.66	0.00162143504819226\\
28.67	0.00162143504819226\\
28.68	0.00162143504819226\\
28.69	0.00162143504819226\\
28.7	0.00162143504819226\\
28.71	0.00162143504819226\\
28.72	0.00162143504819226\\
28.73	0.00162143504819226\\
28.74	0.00162143504819226\\
28.75	0.00162143504819226\\
28.76	0.00162143504819226\\
28.77	0.00162143504819226\\
28.78	0.00162143504819226\\
28.79	0.00162143504819226\\
28.8	0.00162143504819226\\
28.81	0.00162143504819226\\
28.82	0.00162143504819226\\
28.83	0.00162143504819226\\
28.84	0.00162143504819226\\
28.85	0.00162143504819226\\
28.86	0.00162143504819226\\
28.87	0.00162143504819226\\
28.88	0.00162143504819226\\
28.89	0.00162143504819226\\
28.9	0.00162143504819226\\
28.91	0.00162143504819226\\
28.92	0.00162143504819226\\
28.93	0.00162143504819226\\
28.94	0.00162143504819226\\
28.95	0.00162143504819226\\
28.96	0.00162143504819226\\
28.97	0.00162143504819226\\
28.98	0.00162143504819226\\
28.99	0.00162143504819226\\
29	0.00162143504819226\\
29.01	0.00162143504819226\\
29.02	0.00162143504819226\\
29.03	0.00162143504819226\\
29.04	0.00162143504819226\\
29.05	0.00162143504819226\\
29.06	0.00162143504819226\\
29.07	0.00162143504819226\\
29.08	0.00162143504819226\\
29.09	0.00162143504819226\\
29.1	0.00162143504819226\\
29.11	0.00162143504819226\\
29.12	0.00162143504819226\\
29.13	0.00162143504819226\\
29.14	0.00162143504819226\\
29.15	0.00162143504819226\\
29.16	0.00162143504819226\\
29.17	0.00162143504819226\\
29.18	0.00162143504819226\\
29.19	0.00162143504819226\\
29.2	0.00162143504819226\\
29.21	0.00162143504819226\\
29.22	0.00162143504819226\\
29.23	0.00162143504819226\\
29.24	0.00162143504819226\\
29.25	0.00162143504819226\\
29.26	0.00162143504819226\\
29.27	0.00162143504819226\\
29.28	0.00162143504819226\\
29.29	0.00162143504819226\\
29.3	0.00162143504819226\\
29.31	0.00162143504819226\\
29.32	0.00162143504819226\\
29.33	0.00162143504819226\\
29.34	0.00162143504819226\\
29.35	0.00162143504819226\\
29.36	0.00162143504819226\\
29.37	0.00162143504819226\\
29.38	0.00162143504819226\\
29.39	0.00162143504819226\\
29.4	0.00162143504819226\\
29.41	0.00162143504819226\\
29.42	0.00162143504819226\\
29.43	0.00162143504819226\\
29.44	0.00162143504819226\\
29.45	0.00162143504819226\\
29.46	0.00162143504819226\\
29.47	0.00162143504819226\\
29.48	0.00162143504819226\\
29.49	0.00162143504819226\\
29.5	0.00162143504819226\\
29.51	0.00162143504819226\\
29.52	0.00162143504819226\\
29.53	0.00162143504819226\\
29.54	0.00162143504819226\\
29.55	0.00162143504819226\\
29.56	0.00162143504819226\\
29.57	0.00162143504819226\\
29.58	0.00162143504819226\\
29.59	0.00162143504819226\\
29.6	0.00162143504819226\\
29.61	0.00162143504819226\\
29.62	0.00162143504819226\\
29.63	0.00162143504819226\\
29.64	0.00162143504819226\\
29.65	0.00162143504819226\\
29.66	0.00162143504819226\\
29.67	0.00162143504819226\\
29.68	0.00162143504819226\\
29.69	0.00162143504819226\\
29.7	0.00162143504819226\\
29.71	0.00162143504819226\\
29.72	0.00162143504819226\\
29.73	0.00162143504819226\\
29.74	0.00162143504819226\\
29.75	0.00162143504819226\\
29.76	0.00162143504819226\\
29.77	0.00162143504819226\\
29.78	0.00162143504819226\\
29.79	0.00162143504819226\\
29.8	0.00162143504819226\\
29.81	0.00162143504819226\\
29.82	0.00162143504819226\\
29.83	0.00162143504819226\\
29.84	0.00162143504819226\\
29.85	0.00162143504819226\\
29.86	0.00162143504819226\\
29.87	0.00162143504819226\\
29.88	0.00162143504819226\\
29.89	0.00162143504819226\\
29.9	0.00162143504819226\\
29.91	0.00162143504819226\\
29.92	0.00162143504819226\\
29.93	0.00162143504819226\\
29.94	0.00162143504819226\\
29.95	0.00162143504819226\\
29.96	0.00162143504819226\\
29.97	0.00162143504819226\\
29.98	0.00162143504819226\\
29.99	0.00162143504819226\\
30	0.00162143504819226\\
30.01	0.00162143504819226\\
30.02	0.00162143504819226\\
30.03	0.00162143504819226\\
30.04	0.00162143504819226\\
30.05	0.00162143504819226\\
30.06	0.00162143504819226\\
30.07	0.00162143504819226\\
30.08	0.00162143504819226\\
30.09	0.00162143504819226\\
30.1	0.00162143504819226\\
30.11	0.00162143504819226\\
30.12	0.00162143504819226\\
30.13	0.00162143504819226\\
30.14	0.00162143504819226\\
30.15	0.00162143504819226\\
30.16	0.00162143504819226\\
30.17	0.00162143504819226\\
30.18	0.00162143504819226\\
30.19	0.00162143504819226\\
30.2	0.00162143504819226\\
30.21	0.00162143504819226\\
30.22	0.00162143504819226\\
30.23	0.00162143504819226\\
30.24	0.00162143504819226\\
30.25	0.00162143504819226\\
30.26	0.00162143504819226\\
30.27	0.00162143504819226\\
30.28	0.00162143504819226\\
30.29	0.00162143504819226\\
30.3	0.00162143504819226\\
30.31	0.00162143504819226\\
30.32	0.00162143504819226\\
30.33	0.00162143504819226\\
30.34	0.00162143504819226\\
30.35	0.00162143504819226\\
30.36	0.00162143504819226\\
30.37	0.00162143504819226\\
30.38	0.00162143504819226\\
30.39	0.00162143504819226\\
30.4	0.00162143504819226\\
30.41	0.00162143504819226\\
30.42	0.00162143504819226\\
30.43	0.00162143504819226\\
30.44	0.00162143504819226\\
30.45	0.00162143504819226\\
30.46	0.00162143504819226\\
30.47	0.00162143504819226\\
30.48	0.00162143504819226\\
30.49	0.00162143504819226\\
30.5	0.00162143504819226\\
30.51	0.00162143504819226\\
30.52	0.00162143504819226\\
30.53	0.00162143504819226\\
30.54	0.00162143504819226\\
30.55	0.00162143504819226\\
30.56	0.00162143504819226\\
30.57	0.00162143504819226\\
30.58	0.00162143504819226\\
30.59	0.00162143504819226\\
30.6	0.00162143504819226\\
30.61	0.00162143504819226\\
30.62	0.00162143504819226\\
30.63	0.00162143504819226\\
30.64	0.00162143504819226\\
30.65	0.00162143504819226\\
30.66	0.00162143504819226\\
30.67	0.00162143504819226\\
30.68	0.00162143504819226\\
30.69	0.00162143504819226\\
30.7	0.00162143504819226\\
30.71	0.00162143504819226\\
30.72	0.00162143504819226\\
30.73	0.00162143504819226\\
30.74	0.00162143504819226\\
30.75	0.00162143504819226\\
30.76	0.00162143504819226\\
30.77	0.00162143504819226\\
30.78	0.00162143504819226\\
30.79	0.00162143504819226\\
30.8	0.00162143504819226\\
30.81	0.00162143504819226\\
30.82	0.00162143504819226\\
30.83	0.00162143504819226\\
30.84	0.00162143504819226\\
30.85	0.00162143504819226\\
30.86	0.00162143504819226\\
30.87	0.00162143504819226\\
30.88	0.00162143504819226\\
30.89	0.00162143504819226\\
30.9	0.00162143504819226\\
30.91	0.00162143504819226\\
30.92	0.00162143504819226\\
30.93	0.00162143504819226\\
30.94	0.00162143504819226\\
30.95	0.00162143504819226\\
30.96	0.00162143504819226\\
30.97	0.00162143504819226\\
30.98	0.00162143504819226\\
30.99	0.00162143504819226\\
31	0.00162143504819226\\
31.01	0.00162143504819226\\
31.02	0.00162143504819226\\
31.03	0.00162143504819226\\
31.04	0.00162143504819226\\
31.05	0.00162143504819226\\
31.06	0.00162143504819226\\
31.07	0.00162143504819226\\
31.08	0.00162143504819226\\
31.09	0.00162143504819226\\
31.1	0.00162143504819226\\
31.11	0.00162143504819226\\
31.12	0.00162143504819226\\
31.13	0.00162143504819226\\
31.14	0.00162143504819226\\
31.15	0.00162143504819226\\
31.16	0.00162143504819226\\
31.17	0.00162143504819226\\
31.18	0.00162143504819226\\
31.19	0.00162143504819226\\
31.2	0.00162143504819226\\
31.21	0.00162143504819226\\
31.22	0.00162143504819226\\
31.23	0.00162143504819226\\
31.24	0.00162143504819226\\
31.25	0.00162143504819226\\
31.26	0.00162143504819226\\
31.27	0.00162143504819226\\
31.28	0.00162143504819226\\
31.29	0.00162143504819226\\
31.3	0.00162143504819226\\
31.31	0.00162143504819226\\
31.32	0.00162143504819226\\
31.33	0.00162143504819226\\
31.34	0.00162143504819226\\
31.35	0.00162143504819226\\
31.36	0.00162143504819226\\
31.37	0.00162143504819226\\
31.38	0.00162143504819226\\
31.39	0.00162143504819226\\
31.4	0.00162143504819226\\
31.41	0.00162143504819226\\
31.42	0.00162143504819226\\
31.43	0.00162143504819226\\
31.44	0.00162143504819226\\
31.45	0.00162143504819226\\
31.46	0.00162143504819226\\
31.47	0.00162143504819226\\
31.48	0.00162143504819226\\
31.49	0.00162143504819226\\
31.5	0.00162143504819226\\
31.51	0.00162143504819226\\
31.52	0.00162143504819226\\
31.53	0.00162143504819226\\
31.54	0.00162143504819226\\
31.55	0.00162143504819226\\
31.56	0.00162143504819226\\
31.57	0.00162143504819226\\
31.58	0.00162143504819226\\
31.59	0.00162143504819226\\
31.6	0.00162143504819226\\
31.61	0.00162143504819226\\
31.62	0.00162143504819226\\
31.63	0.00162143504819226\\
31.64	0.00162143504819226\\
31.65	0.00162143504819226\\
31.66	0.00162143504819226\\
31.67	0.00162143504819226\\
31.68	0.00162143504819226\\
31.69	0.00162143504819226\\
31.7	0.00162143504819226\\
31.71	0.00162143504819226\\
31.72	0.00162143504819226\\
31.73	0.00162143504819226\\
31.74	0.00162143504819226\\
31.75	0.00162143504819226\\
31.76	0.00162143504819226\\
31.77	0.00162143504819226\\
31.78	0.00162143504819226\\
31.79	0.00162143504819226\\
31.8	0.00162143504819226\\
31.81	0.00162143504819226\\
31.82	0.00162143504819226\\
31.83	0.00162143504819226\\
31.84	0.00162143504819226\\
31.85	0.00162143504819226\\
31.86	0.00162143504819226\\
31.87	0.00162143504819226\\
31.88	0.00162143504819226\\
31.89	0.00162143504819226\\
31.9	0.00162143504819226\\
31.91	0.00162143504819226\\
31.92	0.00162143504819226\\
31.93	0.00162143504819226\\
31.94	0.00162143504819226\\
31.95	0.00162143504819226\\
31.96	0.00162143504819226\\
31.97	0.00162143504819226\\
31.98	0.00162143504819226\\
31.99	0.00162143504819226\\
32	0.00162143504819226\\
32.01	0.00162143504819226\\
32.02	0.00162143504819226\\
32.03	0.00162143504819226\\
32.04	0.00162143504819226\\
32.05	0.00162143504819226\\
32.06	0.00162143504819226\\
32.07	0.00162143504819226\\
32.08	0.00162143504819226\\
32.09	0.00162143504819226\\
32.1	0.00162143504819226\\
32.11	0.00162143504819226\\
32.12	0.00162143504819226\\
32.13	0.00162143504819226\\
32.14	0.00162143504819226\\
32.15	0.00162143504819226\\
32.16	0.00162143504819226\\
32.17	0.00162143504819226\\
32.18	0.00162143504819226\\
32.19	0.00162143504819226\\
32.2	0.00162143504819226\\
32.21	0.00162143504819226\\
32.22	0.00162143504819226\\
32.23	0.00162143504819226\\
32.24	0.00162143504819226\\
32.25	0.00162143504819226\\
32.26	0.00162143504819226\\
32.27	0.00162143504819226\\
32.28	0.00162143504819226\\
32.29	0.00162143504819226\\
32.3	0.00162143504819226\\
32.31	0.00162143504819226\\
32.32	0.00162143504819226\\
32.33	0.00162143504819226\\
32.34	0.00162143504819226\\
32.35	0.00162143504819226\\
32.36	0.00162143504819226\\
32.37	0.00162143504819226\\
32.38	0.00162143504819226\\
32.39	0.00162143504819226\\
32.4	0.00162143504819226\\
32.41	0.00162143504819226\\
32.42	0.00162143504819226\\
32.43	0.00162143504819226\\
32.44	0.00162143504819226\\
32.45	0.00162143504819226\\
32.46	0.00162143504819226\\
32.47	0.00162143504819226\\
32.48	0.00162143504819226\\
32.49	0.00162143504819226\\
32.5	0.00162143504819226\\
32.51	0.00162143504819226\\
32.52	0.00162143504819226\\
32.53	0.00162143504819226\\
32.54	0.00162143504819226\\
32.55	0.00162143504819226\\
32.56	0.00162143504819226\\
32.57	0.00162143504819226\\
32.58	0.00162143504819226\\
32.59	0.00162143504819226\\
32.6	0.00162143504819226\\
32.61	0.00162143504819226\\
32.62	0.00162143504819226\\
32.63	0.00162143504819226\\
32.64	0.00162143504819226\\
32.65	0.00162143504819226\\
32.66	0.00162143504819226\\
32.67	0.00162143504819226\\
32.68	0.00162143504819226\\
32.69	0.00162143504819226\\
32.7	0.00162143504819226\\
32.71	0.00162143504819226\\
32.72	0.00162143504819226\\
32.73	0.00162143504819226\\
32.74	0.00162143504819226\\
32.75	0.00162143504819226\\
32.76	0.00162143504819226\\
32.77	0.00162143504819226\\
32.78	0.00162143504819226\\
32.79	0.00162143504819226\\
32.8	0.00162143504819226\\
32.81	0.00162143504819226\\
32.82	0.00162143504819226\\
32.83	0.00162143504819226\\
32.84	0.00162143504819226\\
32.85	0.00162143504819226\\
32.86	0.00162143504819226\\
32.87	0.00162143504819226\\
32.88	0.00162143504819226\\
32.89	0.00162143504819226\\
32.9	0.00162143504819226\\
32.91	0.00162143504819226\\
32.92	0.00162143504819226\\
32.93	0.00162143504819226\\
32.94	0.00162143504819226\\
32.95	0.00162143504819226\\
32.96	0.00162143504819226\\
32.97	0.00162143504819226\\
32.98	0.00162143504819226\\
32.99	0.00162143504819226\\
33	0.00162143504819226\\
33.01	0.00162143504819226\\
33.02	0.00162143504819226\\
33.03	0.00162143504819226\\
33.04	0.00162143504819226\\
33.05	0.00162143504819226\\
33.06	0.00162143504819226\\
33.07	0.00162143504819226\\
33.08	0.00162143504819226\\
33.09	0.00162143504819226\\
33.1	0.00162143504819226\\
33.11	0.00162143504819226\\
33.12	0.00162143504819226\\
33.13	0.00162143504819226\\
33.14	0.00162143504819226\\
33.15	0.00162143504819226\\
33.16	0.00162143504819226\\
33.17	0.00162143504819226\\
33.18	0.00162143504819226\\
33.19	0.00162143504819226\\
33.2	0.00162143504819226\\
33.21	0.00162143504819226\\
33.22	0.00162143504819226\\
33.23	0.00162143504819226\\
33.24	0.00162143504819226\\
33.25	0.00162143504819226\\
33.26	0.00162143504819226\\
33.27	0.00162143504819226\\
33.28	0.00162143504819226\\
33.29	0.00162143504819226\\
33.3	0.00162143504819226\\
33.31	0.00162143504819226\\
33.32	0.00162143504819226\\
33.33	0.00162143504819226\\
33.34	0.00162143504819226\\
33.35	0.00162143504819226\\
33.36	0.00162143504819226\\
33.37	0.00162143504819226\\
33.38	0.00162143504819226\\
33.39	0.00162143504819226\\
33.4	0.00162143504819226\\
33.41	0.00162143504819226\\
33.42	0.00162143504819226\\
33.43	0.00162143504819226\\
33.44	0.00162143504819226\\
33.45	0.00162143504819226\\
33.46	0.00162143504819226\\
33.47	0.00162143504819226\\
33.48	0.00162143504819226\\
33.49	0.00162143504819226\\
33.5	0.00162143504819226\\
33.51	0.00162143504819226\\
33.52	0.00162143504819226\\
33.53	0.00162143504819226\\
33.54	0.00162143504819226\\
33.55	0.00162143504819226\\
33.56	0.00162143504819226\\
33.57	0.00162143504819226\\
33.58	0.00162143504819226\\
33.59	0.00162143504819226\\
33.6	0.00162143504819226\\
33.61	0.00162143504819226\\
33.62	0.00162143504819226\\
33.63	0.00162143504819226\\
33.64	0.00162143504819226\\
33.65	0.00162143504819226\\
33.66	0.00162143504819226\\
33.67	0.00162143504819226\\
33.68	0.00162143504819226\\
33.69	0.00162143504819226\\
33.7	0.00162143504819226\\
33.71	0.00162143504819226\\
33.72	0.00162143504819226\\
33.73	0.00162143504819226\\
33.74	0.00162143504819226\\
33.75	0.00162143504819226\\
33.76	0.00162143504819226\\
33.77	0.00162143504819226\\
33.78	0.00162143504819226\\
33.79	0.00162143504819226\\
33.8	0.00162143504819226\\
33.81	0.00162143504819226\\
33.82	0.00162143504819226\\
33.83	0.00162143504819226\\
33.84	0.00162143504819226\\
33.85	0.00162143504819226\\
33.86	0.00162143504819226\\
33.87	0.00162143504819226\\
33.88	0.00162143504819226\\
33.89	0.00162143504819226\\
33.9	0.00162143504819226\\
33.91	0.00162143504819226\\
33.92	0.00162143504819226\\
33.93	0.00162143504819226\\
33.94	0.00162143504819226\\
33.95	0.00162143504819226\\
33.96	0.00162143504819226\\
33.97	0.00162143504819226\\
33.98	0.00162143504819226\\
33.99	0.00162143504819226\\
34	0.00162143504819226\\
34.01	0.00162143504819226\\
34.02	0.00162143504819226\\
34.03	0.00162143504819226\\
34.04	0.00162143504819226\\
34.05	0.00162143504819226\\
34.06	0.00162143504819226\\
34.07	0.00162143504819226\\
34.08	0.00162143504819226\\
34.09	0.00162143504819226\\
34.1	0.00162143504819226\\
34.11	0.00162143504819226\\
34.12	0.00162143504819226\\
34.13	0.00162143504819226\\
34.14	0.00162143504819226\\
34.15	0.00162143504819226\\
34.16	0.00162143504819226\\
34.17	0.00162143504819226\\
34.18	0.00162143504819226\\
34.19	0.00162143504819226\\
34.2	0.00162143504819226\\
34.21	0.00162143504819226\\
34.22	0.00162143504819226\\
34.23	0.00162143504819226\\
34.24	0.00162143504819226\\
34.25	0.00162143504819226\\
34.26	0.00162143504819226\\
34.27	0.00162143504819226\\
34.28	0.00162143504819226\\
34.29	0.00162143504819226\\
34.3	0.00162143504819226\\
34.31	0.00162143504819226\\
34.32	0.00162143504819226\\
34.33	0.00162143504819226\\
34.34	0.00162143504819226\\
34.35	0.00162143504819226\\
34.36	0.00162143504819226\\
34.37	0.00162143504819226\\
34.38	0.00162143504819226\\
34.39	0.00162143504819226\\
34.4	0.00162143504819226\\
34.41	0.00162143504819226\\
34.42	0.00162143504819226\\
34.43	0.00162143504819226\\
34.44	0.00162143504819226\\
34.45	0.00162143504819226\\
34.46	0.00162143504819226\\
34.47	0.00162143504819226\\
34.48	0.00162143504819226\\
34.49	0.00162143504819226\\
34.5	0.00162143504819226\\
34.51	0.00162143504819226\\
34.52	0.00162143504819226\\
34.53	0.00162143504819226\\
34.54	0.00162143504819226\\
34.55	0.00162143504819226\\
34.56	0.00162143504819226\\
34.57	0.00162143504819226\\
34.58	0.00162143504819226\\
34.59	0.00162143504819226\\
34.6	0.00162143504819226\\
34.61	0.00162143504819226\\
34.62	0.00162143504819226\\
34.63	0.00162143504819226\\
34.64	0.00162143504819226\\
34.65	0.00162143504819226\\
34.66	0.00162143504819226\\
34.67	0.00162143504819226\\
34.68	0.00162143504819226\\
34.69	0.00162143504819226\\
34.7	0.00162143504819226\\
34.71	0.00162143504819226\\
34.72	0.00162143504819226\\
34.73	0.00162143504819226\\
34.74	0.00162143504819226\\
34.75	0.00162143504819226\\
34.76	0.00162143504819226\\
34.77	0.00162143504819226\\
34.78	0.00162143504819226\\
34.79	0.00162143504819226\\
34.8	0.00162143504819226\\
34.81	0.00162143504819226\\
34.82	0.00162143504819226\\
34.83	0.00162143504819226\\
34.84	0.00162143504819226\\
34.85	0.00162143504819226\\
34.86	0.00162143504819226\\
34.87	0.00162143504819226\\
34.88	0.00162143504819226\\
34.89	0.00162143504819226\\
34.9	0.00162143504819226\\
34.91	0.00162143504819226\\
34.92	0.00162143504819226\\
34.93	0.00162143504819226\\
34.94	0.00162143504819226\\
34.95	0.00162143504819226\\
34.96	0.00162143504819226\\
34.97	0.00162143504819226\\
34.98	0.00162143504819226\\
34.99	0.00162143504819226\\
35	0.00162143504819226\\
35.01	0.00162143504819226\\
35.02	0.00162143504819226\\
35.03	0.00162143504819226\\
35.04	0.00162143504819226\\
35.05	0.00162143504819226\\
35.06	0.00162143504819226\\
35.07	0.00162143504819226\\
35.08	0.00162143504819226\\
35.09	0.00162143504819226\\
35.1	0.00162143504819226\\
35.11	0.00162143504819226\\
35.12	0.00162143504819226\\
35.13	0.00162143504819226\\
35.14	0.00162143504819226\\
35.15	0.00162143504819226\\
35.16	0.00162143504819226\\
35.17	0.00162143504819226\\
35.18	0.00162143504819226\\
35.19	0.00162143504819226\\
35.2	0.00162143504819226\\
35.21	0.00162143504819226\\
35.22	0.00162143504819226\\
35.23	0.00162143504819226\\
35.24	0.00162143504819226\\
35.25	0.00162143504819226\\
35.26	0.00162143504819226\\
35.27	0.00162143504819226\\
35.28	0.00162143504819226\\
35.29	0.00162143504819226\\
35.3	0.00162143504819226\\
35.31	0.00162143504819226\\
35.32	0.00162143504819226\\
35.33	0.00162143504819226\\
35.34	0.00162143504819226\\
35.35	0.00162143504819226\\
35.36	0.00162143504819226\\
35.37	0.00162143504819226\\
35.38	0.00162143504819226\\
35.39	0.00162143504819226\\
35.4	0.00162143504819226\\
35.41	0.00162143504819226\\
35.42	0.00162143504819226\\
35.43	0.00162143504819226\\
35.44	0.00162143504819226\\
35.45	0.00162143504819226\\
35.46	0.00162143504819226\\
35.47	0.00162143504819226\\
35.48	0.00162143504819226\\
35.49	0.00162143504819226\\
35.5	0.00162143504819226\\
35.51	0.00162143504819226\\
35.52	0.00162143504819226\\
35.53	0.00162143504819226\\
35.54	0.00162143504819226\\
35.55	0.00162143504819226\\
35.56	0.00162143504819226\\
35.57	0.00162143504819226\\
35.58	0.00162143504819226\\
35.59	0.00162143504819226\\
35.6	0.00162143504819226\\
35.61	0.00162143504819226\\
35.62	0.00162143504819226\\
35.63	0.00162143504819226\\
35.64	0.00162143504819226\\
35.65	0.00162143504819226\\
35.66	0.00162143504819226\\
35.67	0.00162143504819226\\
35.68	0.00162143504819226\\
35.69	0.00162143504819226\\
35.7	0.00162143504819226\\
35.71	0.00162143504819226\\
35.72	0.00162143504819226\\
35.73	0.00162143504819226\\
35.74	0.00162143504819226\\
35.75	0.00162143504819226\\
35.76	0.00162143504819226\\
35.77	0.00162143504819226\\
35.78	0.00162143504819226\\
35.79	0.00162143504819226\\
35.8	0.00162143504819226\\
35.81	0.00162143504819226\\
35.82	0.00162143504819226\\
35.83	0.00162143504819226\\
35.84	0.00162143504819226\\
35.85	0.00162143504819226\\
35.86	0.00162143504819226\\
35.87	0.00162143504819226\\
35.88	0.00162143504819226\\
35.89	0.00162143504819226\\
35.9	0.00162143504819226\\
35.91	0.00162143504819226\\
35.92	0.00162143504819226\\
35.93	0.00162143504819226\\
35.94	0.00162143504819226\\
35.95	0.00162143504819226\\
35.96	0.00162143504819226\\
35.97	0.00162143504819226\\
35.98	0.00162143504819226\\
35.99	0.00162143504819226\\
36	0.00162143504819226\\
36.01	0.00162143504819226\\
36.02	0.00162143504819226\\
36.03	0.00162143504819226\\
36.04	0.00162143504819226\\
36.05	0.00162143504819226\\
36.06	0.00162143504819226\\
36.07	0.00162143504819226\\
36.08	0.00162143504819226\\
36.09	0.00162143504819226\\
36.1	0.00162143504819226\\
36.11	0.00162143504819226\\
36.12	0.00162143504819226\\
36.13	0.00162143504819226\\
36.14	0.00162143504819226\\
36.15	0.00162143504819226\\
36.16	0.00162143504819226\\
36.17	0.00162143504819226\\
36.18	0.00162143504819226\\
36.19	0.00162143504819226\\
36.2	0.00162143504819226\\
36.21	0.00162143504819226\\
36.22	0.00162143504819226\\
36.23	0.00162143504819226\\
36.24	0.00162143504819226\\
36.25	0.00162143504819226\\
36.26	0.00162143504819226\\
36.27	0.00162143504819226\\
36.28	0.00162143504819226\\
36.29	0.00162143504819226\\
36.3	0.00162143504819226\\
36.31	0.00162143504819226\\
36.32	0.00162143504819226\\
36.33	0.00162143504819226\\
36.34	0.00162143504819226\\
36.35	0.00162143504819226\\
36.36	0.00162143504819226\\
36.37	0.00162143504819226\\
36.38	0.00162143504819226\\
36.39	0.00162143504819226\\
36.4	0.00162143504819226\\
36.41	0.00162143504819226\\
36.42	0.00162143504819226\\
36.43	0.00162143504819226\\
36.44	0.00162143504819226\\
36.45	0.00162143504819226\\
36.46	0.00162143504819226\\
36.47	0.00162143504819226\\
36.48	0.00162143504819226\\
36.49	0.00162143504819226\\
36.5	0.00162143504819226\\
36.51	0.00162143504819226\\
36.52	0.00162143504819226\\
36.53	0.00162143504819226\\
36.54	0.00162143504819226\\
36.55	0.00162143504819226\\
36.56	0.00162143504819226\\
36.57	0.00162143504819226\\
36.58	0.00162143504819226\\
36.59	0.00162143504819226\\
36.6	0.00162143504819226\\
36.61	0.00162143504819226\\
36.62	0.00162143504819226\\
36.63	0.00162143504819226\\
36.64	0.00162143504819226\\
36.65	0.00162143504819226\\
36.66	0.00162143504819226\\
36.67	0.00162143504819226\\
36.68	0.00162143504819226\\
36.69	0.00162143504819226\\
36.7	0.00162143504819226\\
36.71	0.00162143504819226\\
36.72	0.00162143504819226\\
36.73	0.00162143504819226\\
36.74	0.00162143504819226\\
36.75	0.00162143504819226\\
36.76	0.00162143504819226\\
36.77	0.00162143504819226\\
36.78	0.00162143504819226\\
36.79	0.00162143504819226\\
36.8	0.00162143504819226\\
36.81	0.00162143504819226\\
36.82	0.00162143504819226\\
36.83	0.00162143504819226\\
36.84	0.00162143504819226\\
36.85	0.00162143504819226\\
36.86	0.00162143504819226\\
36.87	0.00162143504819226\\
36.88	0.00162143504819226\\
36.89	0.00162143504819226\\
36.9	0.00162143504819226\\
36.91	0.00162143504819226\\
36.92	0.00162143504819226\\
36.93	0.00162143504819226\\
36.94	0.00162143504819226\\
36.95	0.00162143504819226\\
36.96	0.00162143504819226\\
36.97	0.00162143504819226\\
36.98	0.00162143504819226\\
36.99	0.00162143504819226\\
37	0.00162143504819226\\
37.01	0.00162143504819226\\
37.02	0.00162143504819226\\
37.03	0.00162143504819226\\
37.04	0.00162143504819226\\
37.05	0.00162143504819226\\
37.06	0.00162143504819226\\
37.07	0.00162143504819226\\
37.08	0.00162143504819226\\
37.09	0.00162143504819226\\
37.1	0.00162143504819226\\
37.11	0.00162143504819226\\
37.12	0.00162143504819226\\
37.13	0.00162143504819226\\
37.14	0.00162143504819226\\
37.15	0.00162143504819226\\
37.16	0.00162143504819226\\
37.17	0.00162143504819226\\
37.18	0.00162143504819226\\
37.19	0.00162143504819226\\
37.2	0.00162143504819226\\
37.21	0.00162143504819226\\
37.22	0.00162143504819226\\
37.23	0.00162143504819226\\
37.24	0.00162143504819226\\
37.25	0.00162143504819226\\
37.26	0.00162143504819226\\
37.27	0.00162143504819226\\
37.28	0.00162143504819226\\
37.29	0.00162143504819226\\
37.3	0.00162143504819226\\
37.31	0.00162143504819226\\
37.32	0.00162143504819226\\
37.33	0.00162143504819226\\
37.34	0.00162143504819226\\
37.35	0.00162143504819226\\
37.36	0.00162143504819226\\
37.37	0.00162143504819226\\
37.38	0.00162143504819226\\
37.39	0.00162143504819226\\
37.4	0.00162143504819226\\
37.41	0.00162143504819226\\
37.42	0.00162143504819226\\
37.43	0.00162143504819226\\
37.44	0.00162143504819226\\
37.45	0.00162143504819226\\
37.46	0.00162143504819226\\
37.47	0.00162143504819226\\
37.48	0.00162143504819226\\
37.49	0.00162143504819226\\
37.5	0.00162143504819226\\
37.51	0.00162143504819226\\
37.52	0.00162143504819226\\
37.53	0.00162143504819226\\
37.54	0.00162143504819226\\
37.55	0.00162143504819226\\
37.56	0.00162143504819226\\
37.57	0.00162143504819226\\
37.58	0.00162143504819226\\
37.59	0.00162143504819226\\
37.6	0.00162143504819226\\
37.61	0.00162143504819226\\
37.62	0.00162143504819226\\
37.63	0.00162143504819226\\
37.64	0.00162143504819226\\
37.65	0.00162143504819226\\
37.66	0.00162143504819226\\
37.67	0.00162143504819226\\
37.68	0.00162143504819226\\
37.69	0.00162143504819226\\
37.7	0.00162143504819226\\
37.71	0.00162143504819226\\
37.72	0.00162143504819226\\
37.73	0.00162143504819226\\
37.74	0.00162143504819226\\
37.75	0.00162143504819226\\
37.76	0.00162143504819226\\
37.77	0.00162143504819226\\
37.78	0.00162143504819226\\
37.79	0.00162143504819226\\
37.8	0.00162143504819226\\
37.81	0.00162143504819226\\
37.82	0.00162143504819226\\
37.83	0.00162143504819226\\
37.84	0.00162143504819226\\
37.85	0.00162143504819226\\
37.86	0.00162143504819226\\
37.87	0.00162143504819226\\
37.88	0.00162143504819226\\
37.89	0.00162143504819226\\
37.9	0.00162143504819226\\
37.91	0.00162143504819226\\
37.92	0.00162143504819226\\
37.93	0.00162143504819226\\
37.94	0.00162143504819226\\
37.95	0.00162143504819226\\
37.96	0.00162143504819226\\
37.97	0.00162143504819226\\
37.98	0.00162143504819226\\
37.99	0.00162143504819226\\
38	0.00162143504819226\\
38.01	0.00162143504819226\\
38.02	0.00162143504819226\\
38.03	0.00162143504819226\\
38.04	0.00162143504819226\\
38.05	0.00162143504819226\\
38.06	0.00162143504819226\\
38.07	0.00162143504819226\\
38.08	0.00162143504819226\\
38.09	0.00162143504819226\\
38.1	0.00162143504819226\\
38.11	0.00162143504819226\\
38.12	0.00162143504819226\\
38.13	0.00162143504819226\\
38.14	0.00162143504819226\\
38.15	0.00162143504819226\\
38.16	0.00162143504819226\\
38.17	0.00162143504819226\\
38.18	0.00162143504819226\\
38.19	0.00162143504819226\\
38.2	0.00162143504819226\\
38.21	0.00162143504819226\\
38.22	0.00162143504819226\\
38.23	0.00162143504819226\\
38.24	0.00162143504819226\\
38.25	0.00162143504819226\\
38.26	0.00162143504819226\\
38.27	0.00162143504819226\\
38.28	0.00162143504819226\\
38.29	0.00162143504819226\\
38.3	0.00162143504819226\\
38.31	0.00162143504819226\\
38.32	0.00162143504819226\\
38.33	0.00162143504819226\\
38.34	0.00162143504819226\\
38.35	0.00162143504819226\\
38.36	0.00162143504819226\\
38.37	0.00162143504819226\\
38.38	0.00162143504819226\\
38.39	0.00162143504819226\\
38.4	0.00162143504819226\\
38.41	0.00162143504819226\\
38.42	0.00162143504819226\\
38.43	0.00162143504819226\\
38.44	0.00162143504819226\\
38.45	0.00162143504819226\\
38.46	0.00162143504819226\\
38.47	0.00162143504819226\\
38.48	0.00162143504819226\\
38.49	0.00162143504819226\\
38.5	0.00162143504819226\\
38.51	0.00162143504819226\\
38.52	0.00162143504819226\\
38.53	0.00162143504819226\\
38.54	0.00162143504819226\\
38.55	0.00162143504819226\\
38.56	0.00162143504819226\\
38.57	0.00162143504819226\\
38.58	0.00162143504819226\\
38.59	0.00162143504819226\\
38.6	0.00162143504819226\\
38.61	0.00162143504819226\\
38.62	0.00162143504819226\\
38.63	0.00162143504819226\\
38.64	0.00162143504819226\\
38.65	0.00162143504819226\\
38.66	0.00162143504819226\\
38.67	0.00162143504819226\\
38.68	0.00162143504819226\\
38.69	0.00162143504819225\\
38.7	0.00162143504819225\\
38.71	0.00162143504819225\\
38.72	0.00162143504819225\\
38.73	0.00162143504819225\\
38.74	0.00162143504819225\\
38.75	0.00162143504819225\\
38.76	0.00162143504819225\\
38.77	0.00162143504819225\\
38.78	0.00162143504819225\\
38.79	0.00162143504819225\\
38.8	0.00162143504819225\\
38.81	0.00162143504819225\\
38.82	0.00162143504819225\\
38.83	0.00162143504819225\\
38.84	0.00162143504819225\\
38.85	0.00162143504819225\\
38.86	0.00162143504819225\\
38.87	0.00162143504819225\\
38.88	0.00162143504819225\\
38.89	0.00162143504819225\\
38.9	0.00162143504819225\\
38.91	0.00162143504819225\\
38.92	0.00162143504819225\\
38.93	0.00162143504819225\\
38.94	0.00162143504819225\\
38.95	0.00162143504819225\\
38.96	0.00162143504819225\\
38.97	0.00162143504819225\\
38.98	0.00162143504819225\\
38.99	0.00162143504819225\\
39	0.00162143504819225\\
39.01	0.00162143504819225\\
39.02	0.00162143504819225\\
39.03	0.00162143504819225\\
39.04	0.00162143504819225\\
39.05	0.00162143504819225\\
39.06	0.00162143504819225\\
39.07	0.00162143504819225\\
39.08	0.00162143504819225\\
39.09	0.00162143504819225\\
39.1	0.00162143504819225\\
39.11	0.00162143504819225\\
39.12	0.00162143504819225\\
39.13	0.00162143504819225\\
39.14	0.00162143504819225\\
39.15	0.00162143504819225\\
39.16	0.00162143504819225\\
39.17	0.00162143504819224\\
39.18	0.00162143504819224\\
39.19	0.00162143504819224\\
39.2	0.00162143504819224\\
39.21	0.00162143504819224\\
39.22	0.00162143504819224\\
39.23	0.00162143504819224\\
39.24	0.00162143504819224\\
39.25	0.00162143504819224\\
39.26	0.00162143504819224\\
39.27	0.00162143504819224\\
39.28	0.00162143504819224\\
39.29	0.00162143504819224\\
39.3	0.00162143504819224\\
39.31	0.00162143504819224\\
39.32	0.00162143504819224\\
39.33	0.00162143504819224\\
39.34	0.00162143504819224\\
39.35	0.00162143504819224\\
39.36	0.00162143504819224\\
39.37	0.00162143504819224\\
39.38	0.00162143504819224\\
39.39	0.00162143504819224\\
39.4	0.00162143504819224\\
39.41	0.00162143504819224\\
39.42	0.00162143504819224\\
39.43	0.00162143504819224\\
39.44	0.00162143504819224\\
39.45	0.00162143504819224\\
39.46	0.00162143504819224\\
39.47	0.00162143504819224\\
39.48	0.00162143504819224\\
39.49	0.00162143504819224\\
39.5	0.00162143504819224\\
39.51	0.00162143504819224\\
39.52	0.00162143504819223\\
39.53	0.00162143504819223\\
39.54	0.00162143504819223\\
39.55	0.00162143504819223\\
39.56	0.00162143504819223\\
39.57	0.00162143504819223\\
39.58	0.00162143504819223\\
39.59	0.00162143504819223\\
39.6	0.00162143504819223\\
39.61	0.00162143504819223\\
39.62	0.00162143504819223\\
39.63	0.00162143504819223\\
39.64	0.00162143504819223\\
39.65	0.00162143504819223\\
39.66	0.00162143504819223\\
39.67	0.00162143504819223\\
39.68	0.00162143504819223\\
39.69	0.00162143504819223\\
39.7	0.00162143504819223\\
39.71	0.00162143504819223\\
39.72	0.00162143504819223\\
39.73	0.00162143504819223\\
39.74	0.00162143504819223\\
39.75	0.00162143504819223\\
39.76	0.00162143504819222\\
39.77	0.00162143504819222\\
39.78	0.00162143504819222\\
39.79	0.00162143504819222\\
39.8	0.00162143504819222\\
39.81	0.00162143504819222\\
39.82	0.00162143504819222\\
39.83	0.00162143504819222\\
39.84	0.00162143504819222\\
39.85	0.00162143504819222\\
39.86	0.00162143504819222\\
39.87	0.00162143504819222\\
39.88	0.00162143504819222\\
39.89	0.00162143504819222\\
39.9	0.00162143504819222\\
39.91	0.00162143504819222\\
39.92	0.00162143504819222\\
39.93	0.00162143504819222\\
39.94	0.00162143504819222\\
39.95	0.00162143504819221\\
39.96	0.00162143504819221\\
39.97	0.00162143504819221\\
39.98	0.00162143504819221\\
39.99	0.00162143504819221\\
40	0.00162143504819221\\
40.01	0.00162143504819221\\
};
\addplot [color=blue,solid,forget plot]
  table[row sep=crcr]{%
40.01	0.00162143504819221\\
40.02	0.00162143504819221\\
40.03	0.00162143504819221\\
40.04	0.00162143504819221\\
40.05	0.00162143504819221\\
40.06	0.00162143504819221\\
40.07	0.00162143504819221\\
40.08	0.0016214350481922\\
40.09	0.0016214350481922\\
40.1	0.0016214350481922\\
40.11	0.0016214350481922\\
40.12	0.0016214350481922\\
40.13	0.0016214350481922\\
40.14	0.0016214350481922\\
40.15	0.0016214350481922\\
40.16	0.0016214350481922\\
40.17	0.0016214350481922\\
40.18	0.0016214350481922\\
40.19	0.0016214350481922\\
40.2	0.0016214350481922\\
40.21	0.0016214350481922\\
40.22	0.0016214350481922\\
40.23	0.00162143504819219\\
40.24	0.00162143504819219\\
40.25	0.00162143504819219\\
40.26	0.00162143504819219\\
40.27	0.00162143504819219\\
40.28	0.00162143504819219\\
40.29	0.00162143504819219\\
40.3	0.00162143504819219\\
40.31	0.00162143504819219\\
40.32	0.00162143504819219\\
40.33	0.00162143504819219\\
40.34	0.00162143504819219\\
40.35	0.00162143504819218\\
40.36	0.00162143504819218\\
40.37	0.00162143504819218\\
40.38	0.00162143504819218\\
40.39	0.00162143504819218\\
40.4	0.00162143504819218\\
40.41	0.00162143504819218\\
40.42	0.00162143504819218\\
40.43	0.00162143504819218\\
40.44	0.00162143504819218\\
40.45	0.00162143504819217\\
40.46	0.00162143504819217\\
40.47	0.00162143504819217\\
40.48	0.00162143504819217\\
40.49	0.00162143504819217\\
40.5	0.00162143504819217\\
40.51	0.00162143504819217\\
40.52	0.00162143504819217\\
40.53	0.00162143504819216\\
40.54	0.00162143504819216\\
40.55	0.00162143504819216\\
40.56	0.00162143504819216\\
40.57	0.00162143504819216\\
40.58	0.00162143504819216\\
40.59	0.00162143504819216\\
40.6	0.00162143504819216\\
40.61	0.00162143504819216\\
40.62	0.00162143504819216\\
40.63	0.00162143504819215\\
40.64	0.00162143504819215\\
40.65	0.00162143504819215\\
40.66	0.00162143504819215\\
40.67	0.00162143504819215\\
40.68	0.00162143504819215\\
40.69	0.00162143504819215\\
40.7	0.00162143504819215\\
40.71	0.00162143504819214\\
40.72	0.00162143504819214\\
40.73	0.00162143504819214\\
40.74	0.00162143504819214\\
40.75	0.00162143504819214\\
40.76	0.00162143504819214\\
40.77	0.00162143504819214\\
40.78	0.00162143504819213\\
40.79	0.00162143504819213\\
40.8	0.00162143504819213\\
40.81	0.00162143504819213\\
40.82	0.00162143504819213\\
40.83	0.00162143504819213\\
40.84	0.00162143504819212\\
40.85	0.00162143504819212\\
40.86	0.00162143504819212\\
40.87	0.00162143504819212\\
40.88	0.00162143504819212\\
40.89	0.00162143504819212\\
40.9	0.00162143504819212\\
40.91	0.00162143504819211\\
40.92	0.00162143504819211\\
40.93	0.00162143504819211\\
40.94	0.00162143504819211\\
40.95	0.00162143504819211\\
40.96	0.00162143504819211\\
40.97	0.0016214350481921\\
40.98	0.0016214350481921\\
40.99	0.0016214350481921\\
41	0.0016214350481921\\
41.01	0.0016214350481921\\
41.02	0.0016214350481921\\
41.03	0.00162143504819209\\
41.04	0.00162143504819209\\
41.05	0.00162143504819209\\
41.06	0.00162143504819209\\
41.07	0.00162143504819208\\
41.08	0.00162143504819208\\
41.09	0.00162143504819208\\
41.1	0.00162143504819208\\
41.11	0.00162143504819208\\
41.12	0.00162143504819208\\
41.13	0.00162143504819207\\
41.14	0.00162143504819207\\
41.15	0.00162143504819207\\
41.16	0.00162143504819207\\
41.17	0.00162143504819207\\
41.18	0.00162143504819206\\
41.19	0.00162143504819206\\
41.2	0.00162143504819206\\
41.21	0.00162143504819206\\
41.22	0.00162143504819205\\
41.23	0.00162143504819205\\
41.24	0.00162143504819205\\
41.25	0.00162143504819205\\
41.26	0.00162143504819205\\
41.27	0.00162143504819204\\
41.28	0.00162143504819204\\
41.29	0.00162143504819204\\
41.3	0.00162143504819203\\
41.31	0.00162143504819203\\
41.32	0.00162143504819203\\
41.33	0.00162143504819203\\
41.34	0.00162143504819203\\
41.35	0.00162143504819202\\
41.36	0.00162143504819202\\
41.37	0.00162143504819202\\
41.38	0.00162143504819202\\
41.39	0.00162143504819201\\
41.4	0.00162143504819201\\
41.41	0.00162143504819201\\
41.42	0.001621435048192\\
41.43	0.001621435048192\\
41.44	0.001621435048192\\
41.45	0.00162143504819199\\
41.46	0.00162143504819199\\
41.47	0.00162143504819199\\
41.48	0.00162143504819199\\
41.49	0.00162143504819198\\
41.5	0.00162143504819198\\
41.51	0.00162143504819198\\
41.52	0.00162143504819197\\
41.53	0.00162143504819197\\
41.54	0.00162143504819197\\
41.55	0.00162143504819196\\
41.56	0.00162143504819196\\
41.57	0.00162143504819196\\
41.58	0.00162143504819195\\
41.59	0.00162143504819195\\
41.6	0.00162143504819195\\
41.61	0.00162143504819194\\
41.62	0.00162143504819194\\
41.63	0.00162143504819194\\
41.64	0.00162143504819193\\
41.65	0.00162143504819193\\
41.66	0.00162143504819193\\
41.67	0.00162143504819192\\
41.68	0.00162143504819192\\
41.69	0.00162143504819191\\
41.7	0.00162143504819191\\
41.71	0.00162143504819191\\
41.72	0.0016214350481919\\
41.73	0.0016214350481919\\
41.74	0.0016214350481919\\
41.75	0.00162143504819189\\
41.76	0.00162143504819189\\
41.77	0.00162143504819189\\
41.78	0.00162143504819188\\
41.79	0.00162143504819188\\
41.8	0.00162143504819187\\
41.81	0.00162143504819187\\
41.82	0.00162143504819186\\
41.83	0.00162143504819186\\
41.84	0.00162143504819186\\
41.85	0.00162143504819185\\
41.86	0.00162143504819185\\
41.87	0.00162143504819184\\
41.88	0.00162143504819184\\
41.89	0.00162143504819183\\
41.9	0.00162143504819183\\
41.91	0.00162143504819182\\
41.92	0.00162143504819182\\
41.93	0.00162143504819181\\
41.94	0.00162143504819181\\
41.95	0.0016214350481918\\
41.96	0.0016214350481918\\
41.97	0.00162143504819179\\
41.98	0.00162143504819179\\
41.99	0.00162143504819178\\
42	0.00162143504819178\\
42.01	0.00162143504819177\\
42.02	0.00162143504819177\\
42.03	0.00162143504819176\\
42.04	0.00162143504819176\\
42.05	0.00162143504819175\\
42.06	0.00162143504819175\\
42.07	0.00162143504819174\\
42.08	0.00162143504819174\\
42.09	0.00162143504819173\\
42.1	0.00162143504819172\\
42.11	0.00162143504819172\\
42.12	0.00162143504819171\\
42.13	0.00162143504819171\\
42.14	0.0016214350481917\\
42.15	0.00162143504819169\\
42.16	0.00162143504819169\\
42.17	0.00162143504819168\\
42.18	0.00162143504819168\\
42.19	0.00162143504819167\\
42.2	0.00162143504819166\\
42.21	0.00162143504819166\\
42.22	0.00162143504819165\\
42.23	0.00162143504819164\\
42.24	0.00162143504819164\\
42.25	0.00162143504819163\\
42.26	0.00162143504819163\\
42.27	0.00162143504819162\\
42.28	0.00162143504819161\\
42.29	0.0016214350481916\\
42.3	0.0016214350481916\\
42.31	0.00162143504819159\\
42.32	0.00162143504819158\\
42.33	0.00162143504819157\\
42.34	0.00162143504819157\\
42.35	0.00162143504819156\\
42.36	0.00162143504819155\\
42.37	0.00162143504819155\\
42.38	0.00162143504819154\\
42.39	0.00162143504819153\\
42.4	0.00162143504819152\\
42.41	0.00162143504819151\\
42.42	0.00162143504819151\\
42.43	0.0016214350481915\\
42.44	0.00162143504819149\\
42.45	0.00162143504819148\\
42.46	0.00162143504819147\\
42.47	0.00162143504819147\\
42.48	0.00162143504819146\\
42.49	0.00162143504819145\\
42.5	0.00162143504819144\\
42.51	0.00162143504819143\\
42.52	0.00162143504819142\\
42.53	0.00162143504819141\\
42.54	0.0016214350481914\\
42.55	0.00162143504819139\\
42.56	0.00162143504819139\\
42.57	0.00162143504819138\\
42.58	0.00162143504819137\\
42.59	0.00162143504819136\\
42.6	0.00162143504819135\\
42.61	0.00162143504819134\\
42.62	0.00162143504819133\\
42.63	0.00162143504819132\\
42.64	0.00162143504819131\\
42.65	0.0016214350481913\\
42.66	0.00162143504819129\\
42.67	0.00162143504819128\\
42.68	0.00162143504819127\\
42.69	0.00162143504819126\\
42.7	0.00162143504819125\\
42.71	0.00162143504819123\\
42.72	0.00162143504819122\\
42.73	0.00162143504819121\\
42.74	0.0016214350481912\\
42.75	0.00162143504819119\\
42.76	0.00162143504819118\\
42.77	0.00162143504819117\\
42.78	0.00162143504819116\\
42.79	0.00162143504819114\\
42.8	0.00162143504819113\\
42.81	0.00162143504819112\\
42.82	0.00162143504819111\\
42.83	0.0016214350481911\\
42.84	0.00162143504819108\\
42.85	0.00162143504819107\\
42.86	0.00162143504819106\\
42.87	0.00162143504819105\\
42.88	0.00162143504819103\\
42.89	0.00162143504819102\\
42.9	0.00162143504819101\\
42.91	0.00162143504819099\\
42.92	0.00162143504819098\\
42.93	0.00162143504819097\\
42.94	0.00162143504819095\\
42.95	0.00162143504819094\\
42.96	0.00162143504819093\\
42.97	0.00162143504819091\\
42.98	0.0016214350481909\\
42.99	0.00162143504819088\\
43	0.00162143504819087\\
43.01	0.00162143504819085\\
43.02	0.00162143504819084\\
43.03	0.00162143504819082\\
43.04	0.00162143504819081\\
43.05	0.00162143504819079\\
43.06	0.00162143504819078\\
43.07	0.00162143504819076\\
43.08	0.00162143504819075\\
43.09	0.00162143504819073\\
43.1	0.00162143504819071\\
43.11	0.0016214350481907\\
43.12	0.00162143504819068\\
43.13	0.00162143504819066\\
43.14	0.00162143504819065\\
43.15	0.00162143504819063\\
43.16	0.00162143504819061\\
43.17	0.00162143504819059\\
43.18	0.00162143504819058\\
43.19	0.00162143504819056\\
43.2	0.00162143504819054\\
43.21	0.00162143504819052\\
43.22	0.0016214350481905\\
43.23	0.00162143504819049\\
43.24	0.00162143504819047\\
43.25	0.00162143504819045\\
43.26	0.00162143504819043\\
43.27	0.00162143504819041\\
43.28	0.00162143504819039\\
43.29	0.00162143504819037\\
43.3	0.00162143504819035\\
43.31	0.00162143504819033\\
43.32	0.00162143504819031\\
43.33	0.00162143504819029\\
43.34	0.00162143504819027\\
43.35	0.00162143504819025\\
43.36	0.00162143504819023\\
43.37	0.0016214350481902\\
43.38	0.00162143504819018\\
43.39	0.00162143504819016\\
43.4	0.00162143504819014\\
43.41	0.00162143504819012\\
43.42	0.0016214350481901\\
43.43	0.00162143504819007\\
43.44	0.00162143504819005\\
43.45	0.00162143504819003\\
43.46	0.00162143504819\\
43.47	0.00162143504818998\\
43.48	0.00162143504818995\\
43.49	0.00162143504818993\\
43.5	0.0016214350481899\\
43.51	0.00162143504818988\\
43.52	0.00162143504818985\\
43.53	0.00162143504818983\\
43.54	0.0016214350481898\\
43.55	0.00162143504818978\\
43.56	0.00162143504818975\\
43.57	0.00162143504818973\\
43.58	0.0016214350481897\\
43.59	0.00162143504818967\\
43.6	0.00162143504818964\\
43.61	0.00162143504818962\\
43.62	0.00162143504818959\\
43.63	0.00162143504818956\\
43.64	0.00162143504818953\\
43.65	0.0016214350481895\\
43.66	0.00162143504818947\\
43.67	0.00162143504818944\\
43.68	0.00162143504818942\\
43.69	0.00162143504818939\\
43.7	0.00162143504818935\\
43.71	0.00162143504818932\\
43.72	0.00162143504818929\\
43.73	0.00162143504818926\\
43.74	0.00162143504818923\\
43.75	0.0016214350481892\\
43.76	0.00162143504818917\\
43.77	0.00162143504818913\\
43.78	0.0016214350481891\\
43.79	0.00162143504818907\\
43.8	0.00162143504818903\\
43.81	0.001621435048189\\
43.82	0.00162143504818897\\
43.83	0.00162143504818893\\
43.84	0.0016214350481889\\
43.85	0.00162143504818886\\
43.86	0.00162143504818882\\
43.87	0.00162143504818879\\
43.88	0.00162143504818875\\
43.89	0.00162143504818871\\
43.9	0.00162143504818868\\
43.91	0.00162143504818864\\
43.92	0.0016214350481886\\
43.93	0.00162143504818856\\
43.94	0.00162143504818853\\
43.95	0.00162143504818849\\
43.96	0.00162143504818845\\
43.97	0.00162143504818841\\
43.98	0.00162143504818837\\
43.99	0.00162143504818832\\
44	0.00162143504818828\\
44.01	0.00162143504818824\\
44.02	0.0016214350481882\\
44.03	0.00162143504818816\\
44.04	0.00162143504818811\\
44.05	0.00162143504818807\\
44.06	0.00162143504818803\\
44.07	0.00162143504818798\\
44.08	0.00162143504818794\\
44.09	0.00162143504818789\\
44.1	0.00162143504818784\\
44.11	0.0016214350481878\\
44.12	0.00162143504818775\\
44.13	0.0016214350481877\\
44.14	0.00162143504818766\\
44.15	0.00162143504818761\\
44.16	0.00162143504818756\\
44.17	0.00162143504818751\\
44.18	0.00162143504818746\\
44.19	0.00162143504818741\\
44.2	0.00162143504818736\\
44.21	0.00162143504818731\\
44.22	0.00162143504818725\\
44.23	0.0016214350481872\\
44.24	0.00162143504818715\\
44.25	0.0016214350481871\\
44.26	0.00162143504818704\\
44.27	0.00162143504818699\\
44.28	0.00162143504818693\\
44.29	0.00162143504818687\\
44.3	0.00162143504818682\\
44.31	0.00162143504818676\\
44.32	0.0016214350481867\\
44.33	0.00162143504818664\\
44.34	0.00162143504818659\\
44.35	0.00162143504818653\\
44.36	0.00162143504818647\\
44.37	0.00162143504818641\\
44.38	0.00162143504818634\\
44.39	0.00162143504818628\\
44.4	0.00162143504818622\\
44.41	0.00162143504818616\\
44.42	0.00162143504818609\\
44.43	0.00162143504818603\\
44.44	0.00162143504818596\\
44.45	0.0016214350481859\\
44.46	0.00162143504818583\\
44.47	0.00162143504818576\\
44.48	0.00162143504818569\\
44.49	0.00162143504818562\\
44.5	0.00162143504818556\\
44.51	0.00162143504818549\\
44.52	0.00162143504818541\\
44.53	0.00162143504818534\\
44.54	0.00162143504818527\\
44.55	0.0016214350481852\\
44.56	0.00162143504818512\\
44.57	0.00162143504818505\\
44.58	0.00162143504818497\\
44.59	0.0016214350481849\\
44.6	0.00162143504818482\\
44.61	0.00162143504818474\\
44.62	0.00162143504818466\\
44.63	0.00162143504818458\\
44.64	0.0016214350481845\\
44.65	0.00162143504818442\\
44.66	0.00162143504818434\\
44.67	0.00162143504818425\\
44.68	0.00162143504818417\\
44.69	0.00162143504818409\\
44.7	0.001621435048184\\
44.71	0.00162143504818391\\
44.72	0.00162143504818383\\
44.73	0.00162143504818374\\
44.74	0.00162143504818365\\
44.75	0.00162143504818356\\
44.76	0.00162143504818347\\
44.77	0.00162143504818338\\
44.78	0.00162143504818328\\
44.79	0.00162143504818319\\
44.8	0.00162143504818309\\
44.81	0.001621435048183\\
44.82	0.0016214350481829\\
44.83	0.0016214350481828\\
44.84	0.00162143504818271\\
44.85	0.0016214350481826\\
44.86	0.0016214350481825\\
44.87	0.0016214350481824\\
44.88	0.0016214350481823\\
44.89	0.00162143504818219\\
44.9	0.00162143504818209\\
44.91	0.00162143504818198\\
44.92	0.00162143504818187\\
44.93	0.00162143504818177\\
44.94	0.00162143504818166\\
44.95	0.00162143504818154\\
44.96	0.00162143504818143\\
44.97	0.00162143504818132\\
44.98	0.00162143504818121\\
44.99	0.00162143504818109\\
45	0.00162143504818097\\
45.01	0.00162143504818086\\
45.02	0.00162143504818074\\
45.03	0.00162143504818061\\
45.04	0.00162143504818049\\
45.05	0.00162143504818037\\
45.06	0.00162143504818025\\
45.07	0.00162143504818012\\
45.08	0.00162143504817999\\
45.09	0.00162143504817987\\
45.1	0.00162143504817974\\
45.11	0.00162143504817961\\
45.12	0.00162143504817947\\
45.13	0.00162143504817934\\
45.14	0.0016214350481792\\
45.15	0.00162143504817907\\
45.16	0.00162143504817893\\
45.17	0.00162143504817879\\
45.18	0.00162143504817865\\
45.19	0.00162143504817851\\
45.2	0.00162143504817837\\
45.21	0.00162143504817822\\
45.22	0.00162143504817807\\
45.23	0.00162143504817792\\
45.24	0.00162143504817778\\
45.25	0.00162143504817762\\
45.26	0.00162143504817747\\
45.27	0.00162143504817732\\
45.28	0.00162143504817716\\
45.29	0.001621435048177\\
45.3	0.00162143504817684\\
45.31	0.00162143504817668\\
45.32	0.00162143504817652\\
45.33	0.00162143504817636\\
45.34	0.00162143504817619\\
45.35	0.00162143504817602\\
45.36	0.00162143504817585\\
45.37	0.00162143504817568\\
45.38	0.00162143504817551\\
45.39	0.00162143504817533\\
45.4	0.00162143504817516\\
45.41	0.00162143504817498\\
45.42	0.0016214350481748\\
45.43	0.00162143504817462\\
45.44	0.00162143504817443\\
45.45	0.00162143504817425\\
45.46	0.00162143504817406\\
45.47	0.00162143504817387\\
45.48	0.00162143504817368\\
45.49	0.00162143504817348\\
45.5	0.00162143504817329\\
45.51	0.00162143504817309\\
45.52	0.00162143504817289\\
45.53	0.00162143504817269\\
45.54	0.00162143504817248\\
45.55	0.00162143504817228\\
45.56	0.00162143504817207\\
45.57	0.00162143504817186\\
45.58	0.00162143504817164\\
45.59	0.00162143504817143\\
45.6	0.00162143504817121\\
45.61	0.00162143504817099\\
45.62	0.00162143504817077\\
45.63	0.00162143504817055\\
45.64	0.00162143504817032\\
45.65	0.00162143504817009\\
45.66	0.00162143504816986\\
45.67	0.00162143504816963\\
45.68	0.00162143504816939\\
45.69	0.00162143504816915\\
45.7	0.00162143504816891\\
45.71	0.00162143504816867\\
45.72	0.00162143504816842\\
45.73	0.00162143504816818\\
45.74	0.00162143504816792\\
45.75	0.00162143504816767\\
45.76	0.00162143504816742\\
45.77	0.00162143504816716\\
45.78	0.0016214350481669\\
45.79	0.00162143504816663\\
45.8	0.00162143504816636\\
45.81	0.00162143504816609\\
45.82	0.00162143504816582\\
45.83	0.00162143504816555\\
45.84	0.00162143504816527\\
45.85	0.00162143504816499\\
45.86	0.0016214350481647\\
45.87	0.00162143504816442\\
45.88	0.00162143504816413\\
45.89	0.00162143504816383\\
45.9	0.00162143504816354\\
45.91	0.00162143504816324\\
45.92	0.00162143504816294\\
45.93	0.00162143504816263\\
45.94	0.00162143504816232\\
45.95	0.00162143504816201\\
45.96	0.0016214350481617\\
45.97	0.00162143504816138\\
45.98	0.00162143504816106\\
45.99	0.00162143504816073\\
46	0.0016214350481604\\
46.01	0.00162143504816007\\
46.02	0.00162143504815974\\
46.03	0.0016214350481594\\
46.04	0.00162143504815906\\
46.05	0.00162143504815871\\
46.06	0.00162143504815836\\
46.07	0.00162143504815801\\
46.08	0.00162143504815765\\
46.09	0.0016214350481573\\
46.1	0.00162143504815693\\
46.11	0.00162143504815656\\
46.12	0.00162143504815619\\
46.13	0.00162143504815582\\
46.14	0.00162143504815544\\
46.15	0.00162143504815506\\
46.16	0.00162143504815467\\
46.17	0.00162143504815428\\
46.18	0.00162143504815388\\
46.19	0.00162143504815349\\
46.2	0.00162143504815308\\
46.21	0.00162143504815268\\
46.22	0.00162143504815226\\
46.23	0.00162143504815185\\
46.24	0.00162143504815143\\
46.25	0.001621435048151\\
46.26	0.00162143504815057\\
46.27	0.00162143504815014\\
46.28	0.0016214350481497\\
46.29	0.00162143504814926\\
46.3	0.00162143504814882\\
46.31	0.00162143504814836\\
46.32	0.00162143504814791\\
46.33	0.00162143504814745\\
46.34	0.00162143504814698\\
46.35	0.00162143504814651\\
46.36	0.00162143504814604\\
46.37	0.00162143504814556\\
46.38	0.00162143504814507\\
46.39	0.00162143504814458\\
46.4	0.00162143504814409\\
46.41	0.00162143504814359\\
46.42	0.00162143504814308\\
46.43	0.00162143504814257\\
46.44	0.00162143504814206\\
46.45	0.00162143504814154\\
46.46	0.00162143504814101\\
46.47	0.00162143504814048\\
46.48	0.00162143504813994\\
46.49	0.0016214350481394\\
46.5	0.00162143504813885\\
46.51	0.00162143504813829\\
46.52	0.00162143504813773\\
46.53	0.00162143504813717\\
46.54	0.0016214350481366\\
46.55	0.00162143504813602\\
46.56	0.00162143504813544\\
46.57	0.00162143504813485\\
46.58	0.00162143504813425\\
46.59	0.00162143504813365\\
46.6	0.00162143504813304\\
46.61	0.00162143504813243\\
46.62	0.00162143504813181\\
46.63	0.00162143504813118\\
46.64	0.00162143504813055\\
46.65	0.00162143504812991\\
46.66	0.00162143504812926\\
46.67	0.00162143504812861\\
46.68	0.00162143504812795\\
46.69	0.00162143504812728\\
46.7	0.00162143504812661\\
46.71	0.00162143504812593\\
46.72	0.00162143504812524\\
46.73	0.00162143504812455\\
46.74	0.00162143504812384\\
46.75	0.00162143504812313\\
46.76	0.00162143504812242\\
46.77	0.0016214350481217\\
46.78	0.00162143504812096\\
46.79	0.00162143504812023\\
46.8	0.00162143504811948\\
46.81	0.00162143504811873\\
46.82	0.00162143504811796\\
46.83	0.00162143504811719\\
46.84	0.00162143504811642\\
46.85	0.00162143504811563\\
46.86	0.00162143504811484\\
46.87	0.00162143504811404\\
46.88	0.00162143504811323\\
46.89	0.00162143504811241\\
46.9	0.00162143504811158\\
46.91	0.00162143504811075\\
46.92	0.0016214350481099\\
46.93	0.00162143504810905\\
46.94	0.00162143504810819\\
46.95	0.00162143504810732\\
46.96	0.00162143504810644\\
46.97	0.00162143504810555\\
46.98	0.00162143504810465\\
46.99	0.00162143504810375\\
47	0.00162143504810283\\
47.01	0.00162143504810191\\
47.02	0.00162143504810097\\
47.03	0.00162143504810002\\
47.04	0.00162143504809907\\
47.05	0.00162143504809811\\
47.06	0.00162143504809713\\
47.07	0.00162143504809615\\
47.08	0.00162143504809516\\
47.09	0.00162143504809415\\
47.1	0.00162143504809314\\
47.11	0.00162143504809211\\
47.12	0.00162143504809108\\
47.13	0.00162143504809003\\
47.14	0.00162143504808897\\
47.15	0.0016214350480879\\
47.16	0.00162143504808683\\
47.17	0.00162143504808573\\
47.18	0.00162143504808463\\
47.19	0.00162143504808352\\
47.2	0.0016214350480824\\
47.21	0.00162143504808126\\
47.22	0.00162143504808012\\
47.23	0.00162143504807895\\
47.24	0.00162143504807778\\
47.25	0.0016214350480766\\
47.26	0.00162143504807541\\
47.27	0.0016214350480742\\
47.28	0.00162143504807298\\
47.29	0.00162143504807175\\
47.3	0.0016214350480705\\
47.31	0.00162143504806925\\
47.32	0.00162143504806797\\
47.33	0.00162143504806669\\
47.34	0.00162143504806539\\
47.35	0.00162143504806408\\
47.36	0.00162143504806276\\
47.37	0.00162143504806142\\
47.38	0.00162143504806007\\
47.39	0.00162143504805871\\
47.4	0.00162143504805733\\
47.41	0.00162143504805594\\
47.42	0.00162143504805453\\
47.43	0.00162143504805311\\
47.44	0.00162143504805167\\
47.45	0.00162143504805022\\
47.46	0.00162143504804875\\
47.47	0.00162143504804727\\
47.48	0.00162143504804577\\
47.49	0.00162143504804426\\
47.5	0.00162143504804274\\
47.51	0.0016214350480412\\
47.52	0.00162143504803964\\
47.53	0.00162143504803806\\
47.54	0.00162143504803647\\
47.55	0.00162143504803486\\
47.56	0.00162143504803324\\
47.57	0.0016214350480316\\
47.58	0.00162143504802994\\
47.59	0.00162143504802827\\
47.6	0.00162143504802658\\
47.61	0.00162143504802487\\
47.62	0.00162143504802315\\
47.63	0.0016214350480214\\
47.64	0.00162143504801964\\
47.65	0.00162143504801786\\
47.66	0.00162143504801606\\
47.67	0.00162143504801425\\
47.68	0.00162143504801241\\
47.69	0.00162143504801056\\
47.7	0.00162143504800869\\
47.71	0.00162143504800679\\
47.72	0.00162143504800488\\
47.73	0.00162143504800295\\
47.74	0.001621435048001\\
47.75	0.00162143504799903\\
47.76	0.00162143504799704\\
47.77	0.00162143504799503\\
47.78	0.001621435047993\\
47.79	0.00162143504799095\\
47.8	0.00162143504798887\\
47.81	0.00162143504798678\\
47.82	0.00162143504798466\\
47.83	0.00162143504798253\\
47.84	0.00162143504798037\\
47.85	0.00162143504797818\\
47.86	0.00162143504797598\\
47.87	0.00162143504797375\\
47.88	0.0016214350479715\\
47.89	0.00162143504796923\\
47.9	0.00162143504796694\\
47.91	0.00162143504796462\\
47.92	0.00162143504796228\\
47.93	0.00162143504795991\\
47.94	0.00162143504795752\\
47.95	0.0016214350479551\\
47.96	0.00162143504795266\\
47.97	0.0016214350479502\\
47.98	0.00162143504794771\\
47.99	0.00162143504794519\\
48	0.00162143504794265\\
48.01	0.00162143504794009\\
48.02	0.00162143504793749\\
48.03	0.00162143504793487\\
48.04	0.00162143504793222\\
48.05	0.00162143504792955\\
48.06	0.00162143504792685\\
48.07	0.00162143504792412\\
48.08	0.00162143504792136\\
48.09	0.00162143504791858\\
48.1	0.00162143504791577\\
48.11	0.00162143504791292\\
48.12	0.00162143504791005\\
48.13	0.00162143504790715\\
48.14	0.00162143504790422\\
48.15	0.00162143504790126\\
48.16	0.00162143504789827\\
48.17	0.00162143504789525\\
48.18	0.0016214350478922\\
48.19	0.00162143504788912\\
48.2	0.00162143504788601\\
48.21	0.00162143504788286\\
48.22	0.00162143504787968\\
48.23	0.00162143504787647\\
48.24	0.00162143504787323\\
48.25	0.00162143504786995\\
48.26	0.00162143504786664\\
48.27	0.0016214350478633\\
48.28	0.00162143504785992\\
48.29	0.00162143504785651\\
48.3	0.00162143504785306\\
48.31	0.00162143504784958\\
48.32	0.00162143504784607\\
48.33	0.00162143504784251\\
48.34	0.00162143504783892\\
48.35	0.0016214350478353\\
48.36	0.00162143504783164\\
48.37	0.00162143504782794\\
48.38	0.0016214350478242\\
48.39	0.00162143504782042\\
48.4	0.00162143504781661\\
48.41	0.00162143504781275\\
48.42	0.00162143504780886\\
48.43	0.00162143504780493\\
48.44	0.00162143504780096\\
48.45	0.00162143504779694\\
48.46	0.00162143504779289\\
48.47	0.0016214350477888\\
48.48	0.00162143504778466\\
48.49	0.00162143504778048\\
48.5	0.00162143504777626\\
48.51	0.001621435047772\\
48.52	0.00162143504776769\\
48.53	0.00162143504776334\\
48.54	0.00162143504775894\\
48.55	0.0016214350477545\\
48.56	0.00162143504775001\\
48.57	0.00162143504774548\\
48.58	0.00162143504774091\\
48.59	0.00162143504773628\\
48.6	0.00162143504773161\\
48.61	0.00162143504772689\\
48.62	0.00162143504772213\\
48.63	0.00162143504771731\\
48.64	0.00162143504771245\\
48.65	0.00162143504770754\\
48.66	0.00162143504770257\\
48.67	0.00162143504769756\\
48.68	0.00162143504769249\\
48.69	0.00162143504768738\\
48.7	0.00162143504768221\\
48.71	0.00162143504767699\\
48.72	0.00162143504767171\\
48.73	0.00162143504766639\\
48.74	0.001621435047661\\
48.75	0.00162143504765557\\
48.76	0.00162143504765008\\
48.77	0.00162143504764453\\
48.78	0.00162143504763892\\
48.79	0.00162143504763326\\
48.8	0.00162143504762755\\
48.81	0.00162143504762177\\
48.82	0.00162143504761593\\
48.83	0.00162143504761004\\
48.84	0.00162143504760409\\
48.85	0.00162143504759807\\
48.86	0.001621435047592\\
48.87	0.00162143504758586\\
48.88	0.00162143504757966\\
48.89	0.0016214350475734\\
48.9	0.00162143504756707\\
48.91	0.00162143504756068\\
48.92	0.00162143504755423\\
48.93	0.00162143504754771\\
48.94	0.00162143504754112\\
48.95	0.00162143504753447\\
48.96	0.00162143504752775\\
48.97	0.00162143504752096\\
48.98	0.0016214350475141\\
48.99	0.00162143504750717\\
49	0.00162143504750018\\
49.01	0.00162143504749311\\
49.02	0.00162143504748597\\
49.03	0.00162143504747876\\
49.04	0.00162143504747147\\
49.05	0.00162143504746411\\
49.06	0.00162143504745668\\
49.07	0.00162143504744917\\
49.08	0.00162143504744158\\
49.09	0.00162143504743392\\
49.1	0.00162143504742618\\
49.11	0.00162143504741836\\
49.12	0.00162143504741047\\
49.13	0.00162143504740249\\
49.14	0.00162143504739443\\
49.15	0.00162143504738629\\
49.16	0.00162143504737807\\
49.17	0.00162143504736976\\
49.18	0.00162143504736137\\
49.19	0.0016214350473529\\
49.2	0.00162143504734434\\
49.21	0.00162143504733569\\
49.22	0.00162143504732696\\
49.23	0.00162143504731814\\
49.24	0.00162143504730923\\
49.25	0.00162143504730022\\
49.26	0.00162143504729113\\
49.27	0.00162143504728194\\
49.28	0.00162143504727267\\
49.29	0.0016214350472633\\
49.3	0.00162143504725383\\
49.31	0.00162143504724427\\
49.32	0.00162143504723461\\
49.33	0.00162143504722485\\
49.34	0.001621435047215\\
49.35	0.00162143504720504\\
49.36	0.00162143504719499\\
49.37	0.00162143504718483\\
49.38	0.00162143504717457\\
49.39	0.00162143504716421\\
49.4	0.00162143504715374\\
49.41	0.00162143504714317\\
49.42	0.00162143504713249\\
49.43	0.0016214350471217\\
49.44	0.0016214350471108\\
49.45	0.0016214350470998\\
49.46	0.00162143504708868\\
49.47	0.00162143504707745\\
49.48	0.0016214350470661\\
49.49	0.00162143504705464\\
49.5	0.00162143504704307\\
49.51	0.00162143504703138\\
49.52	0.00162143504701957\\
49.53	0.00162143504700764\\
49.54	0.0016214350469956\\
49.55	0.00162143504698343\\
49.56	0.00162143504697113\\
49.57	0.00162143504695872\\
49.58	0.00162143504694618\\
49.59	0.00162143504693351\\
49.6	0.00162143504692071\\
49.61	0.00162143504690779\\
49.62	0.00162143504689473\\
49.63	0.00162143504688155\\
49.64	0.00162143504686823\\
49.65	0.00162143504685477\\
49.66	0.00162143504684118\\
49.67	0.00162143504682746\\
49.68	0.00162143504681359\\
49.69	0.00162143504679959\\
49.7	0.00162143504678544\\
49.71	0.00162143504677116\\
49.72	0.00162143504675672\\
49.73	0.00162143504674215\\
49.74	0.00162143504672742\\
49.75	0.00162143504671255\\
49.76	0.00162143504669753\\
49.77	0.00162143504668236\\
49.78	0.00162143504666703\\
49.79	0.00162143504665155\\
49.8	0.00162143504663592\\
49.81	0.00162143504662013\\
49.82	0.00162143504660417\\
49.83	0.00162143504658806\\
49.84	0.00162143504657179\\
49.85	0.00162143504655535\\
49.86	0.00162143504653875\\
49.87	0.00162143504652198\\
49.88	0.00162143504650504\\
49.89	0.00162143504648793\\
49.9	0.00162143504647065\\
49.91	0.0016214350464532\\
49.92	0.00162143504643557\\
49.93	0.00162143504641776\\
49.94	0.00162143504639977\\
49.95	0.00162143504638161\\
49.96	0.00162143504636326\\
49.97	0.00162143504634473\\
49.98	0.001621435046326\\
49.99	0.0016214350463071\\
50	0.001621435046288\\
50.01	0.00162143504626871\\
50.02	0.00162143504624923\\
50.03	0.00162143504622955\\
50.04	0.00162143504620968\\
50.05	0.0016214350461896\\
50.06	0.00162143504616932\\
50.07	0.00162143504614884\\
50.08	0.00162143504612816\\
50.09	0.00162143504610726\\
50.1	0.00162143504608616\\
50.11	0.00162143504606485\\
50.12	0.00162143504604332\\
50.13	0.00162143504602158\\
50.14	0.00162143504599961\\
50.15	0.00162143504597743\\
50.16	0.00162143504595503\\
50.17	0.0016214350459324\\
50.18	0.00162143504590954\\
50.19	0.00162143504588646\\
50.2	0.00162143504586314\\
50.21	0.0016214350458396\\
50.22	0.00162143504581581\\
50.23	0.00162143504579179\\
50.24	0.00162143504576752\\
50.25	0.00162143504574301\\
50.26	0.00162143504571826\\
50.27	0.00162143504569326\\
50.28	0.00162143504566801\\
50.29	0.00162143504564251\\
50.3	0.00162143504561675\\
50.31	0.00162143504559073\\
50.32	0.00162143504556446\\
50.33	0.00162143504553792\\
50.34	0.00162143504551111\\
50.35	0.00162143504548404\\
50.36	0.00162143504545669\\
50.37	0.00162143504542908\\
50.38	0.00162143504540118\\
50.39	0.00162143504537301\\
50.4	0.00162143504534456\\
50.41	0.00162143504531582\\
50.42	0.00162143504528679\\
50.43	0.00162143504525748\\
50.44	0.00162143504522787\\
50.45	0.00162143504519796\\
50.46	0.00162143504516776\\
50.47	0.00162143504513725\\
50.48	0.00162143504510645\\
50.49	0.00162143504507533\\
50.5	0.0016214350450439\\
50.51	0.00162143504501216\\
50.52	0.0016214350449801\\
50.53	0.00162143504494772\\
50.54	0.00162143504491502\\
50.55	0.00162143504488199\\
50.56	0.00162143504484863\\
50.57	0.00162143504481494\\
50.58	0.00162143504478092\\
50.59	0.00162143504474655\\
50.6	0.00162143504471184\\
50.61	0.00162143504467679\\
50.62	0.00162143504464138\\
50.63	0.00162143504460563\\
50.64	0.00162143504456951\\
50.65	0.00162143504453304\\
50.66	0.0016214350444962\\
50.67	0.001621435044459\\
50.68	0.00162143504442142\\
50.69	0.00162143504438347\\
50.7	0.00162143504434514\\
50.71	0.00162143504430643\\
50.72	0.00162143504426734\\
50.73	0.00162143504422786\\
50.74	0.00162143504418798\\
50.75	0.00162143504414771\\
50.76	0.00162143504410703\\
50.77	0.00162143504406595\\
50.78	0.00162143504402446\\
50.79	0.00162143504398256\\
50.8	0.00162143504394024\\
50.81	0.0016214350438975\\
50.82	0.00162143504385434\\
50.83	0.00162143504381074\\
50.84	0.00162143504376672\\
50.85	0.00162143504372225\\
50.86	0.00162143504367734\\
50.87	0.00162143504363199\\
50.88	0.00162143504358619\\
50.89	0.00162143504353993\\
50.9	0.00162143504349321\\
50.91	0.00162143504344602\\
50.92	0.00162143504339837\\
50.93	0.00162143504335025\\
50.94	0.00162143504330164\\
50.95	0.00162143504325256\\
50.96	0.00162143504320298\\
50.97	0.00162143504315292\\
50.98	0.00162143504310235\\
50.99	0.00162143504305129\\
51	0.00162143504299972\\
51.01	0.00162143504294763\\
51.02	0.00162143504289504\\
51.03	0.00162143504284191\\
51.04	0.00162143504278827\\
51.05	0.00162143504273408\\
51.06	0.00162143504267937\\
51.07	0.00162143504262411\\
51.08	0.0016214350425683\\
51.09	0.00162143504251194\\
51.1	0.00162143504245502\\
51.11	0.00162143504239753\\
51.12	0.00162143504233948\\
51.13	0.00162143504228085\\
51.14	0.00162143504222164\\
51.15	0.00162143504216184\\
51.16	0.00162143504210146\\
51.17	0.00162143504204047\\
51.18	0.00162143504197888\\
51.19	0.00162143504191668\\
51.2	0.00162143504185386\\
51.21	0.00162143504179043\\
51.22	0.00162143504172636\\
51.23	0.00162143504166166\\
51.24	0.00162143504159632\\
51.25	0.00162143504153034\\
51.26	0.0016214350414637\\
51.27	0.0016214350413964\\
51.28	0.00162143504132844\\
51.29	0.00162143504125981\\
51.3	0.0016214350411905\\
51.31	0.0016214350411205\\
51.32	0.00162143504104981\\
51.33	0.00162143504097842\\
51.34	0.00162143504090633\\
51.35	0.00162143504083352\\
51.36	0.00162143504076\\
51.37	0.00162143504068575\\
51.38	0.00162143504061077\\
51.39	0.00162143504053505\\
51.4	0.00162143504045857\\
51.41	0.00162143504038135\\
51.42	0.00162143504030336\\
51.43	0.0016214350402246\\
51.44	0.00162143504014507\\
51.45	0.00162143504006475\\
51.46	0.00162143503998364\\
51.47	0.00162143503990173\\
51.48	0.00162143503981901\\
51.49	0.00162143503973548\\
51.5	0.00162143503965112\\
51.51	0.00162143503956593\\
51.52	0.0016214350394799\\
51.53	0.00162143503939303\\
51.54	0.0016214350393053\\
51.55	0.0016214350392167\\
51.56	0.00162143503912723\\
51.57	0.00162143503903688\\
51.58	0.00162143503894564\\
51.59	0.00162143503885351\\
51.6	0.00162143503876046\\
51.61	0.0016214350386665\\
51.62	0.00162143503857162\\
51.63	0.0016214350384758\\
51.64	0.00162143503837904\\
51.65	0.00162143503828133\\
51.66	0.00162143503818266\\
51.67	0.00162143503808302\\
51.68	0.00162143503798239\\
51.69	0.00162143503788078\\
51.7	0.00162143503777817\\
51.71	0.00162143503767455\\
51.72	0.00162143503756992\\
51.73	0.00162143503746425\\
51.74	0.00162143503735755\\
51.75	0.0016214350372498\\
51.76	0.00162143503714099\\
51.77	0.00162143503703112\\
51.78	0.00162143503692016\\
51.79	0.00162143503680811\\
51.8	0.00162143503669497\\
51.81	0.00162143503658072\\
51.82	0.00162143503646534\\
51.83	0.00162143503634884\\
51.84	0.00162143503623119\\
51.85	0.00162143503611238\\
51.86	0.00162143503599242\\
51.87	0.00162143503587127\\
51.88	0.00162143503574894\\
51.89	0.00162143503562541\\
51.9	0.00162143503550067\\
51.91	0.00162143503537471\\
51.92	0.00162143503524751\\
51.93	0.00162143503511907\\
51.94	0.00162143503498937\\
51.95	0.0016214350348584\\
51.96	0.00162143503472615\\
51.97	0.0016214350345926\\
51.98	0.00162143503445775\\
51.99	0.00162143503432158\\
52	0.00162143503418407\\
52.01	0.00162143503404522\\
52.02	0.00162143503390502\\
52.03	0.00162143503376344\\
52.04	0.00162143503362048\\
52.05	0.00162143503347612\\
52.06	0.00162143503333034\\
52.07	0.00162143503318315\\
52.08	0.00162143503303452\\
52.09	0.00162143503288443\\
52.1	0.00162143503273288\\
52.11	0.00162143503257985\\
52.12	0.00162143503242532\\
52.13	0.00162143503226929\\
52.14	0.00162143503211173\\
52.15	0.00162143503195264\\
52.16	0.00162143503179199\\
52.17	0.00162143503162977\\
52.18	0.00162143503146598\\
52.19	0.00162143503130058\\
52.2	0.00162143503113358\\
52.21	0.00162143503096494\\
52.22	0.00162143503079466\\
52.23	0.00162143503062273\\
52.24	0.00162143503044911\\
52.25	0.00162143503027381\\
52.26	0.00162143503009679\\
52.27	0.00162143502991806\\
52.28	0.00162143502973758\\
52.29	0.00162143502955534\\
52.3	0.00162143502937133\\
52.31	0.00162143502918553\\
52.32	0.00162143502899793\\
52.33	0.00162143502880849\\
52.34	0.00162143502861722\\
52.35	0.00162143502842408\\
52.36	0.00162143502822907\\
52.37	0.00162143502803216\\
52.38	0.00162143502783333\\
52.39	0.00162143502763257\\
52.4	0.00162143502742986\\
52.41	0.00162143502722518\\
52.42	0.00162143502701852\\
52.43	0.00162143502680985\\
52.44	0.00162143502659915\\
52.45	0.0016214350263864\\
52.46	0.00162143502617159\\
52.47	0.00162143502595469\\
52.48	0.00162143502573569\\
52.49	0.00162143502551457\\
52.5	0.0016214350252913\\
52.51	0.00162143502506586\\
52.52	0.00162143502483824\\
52.53	0.00162143502460841\\
52.54	0.00162143502437635\\
52.55	0.00162143502414204\\
52.56	0.00162143502390547\\
52.57	0.0016214350236666\\
52.58	0.00162143502342541\\
52.59	0.0016214350231819\\
52.6	0.00162143502293602\\
52.61	0.00162143502268776\\
52.62	0.0016214350224371\\
52.63	0.00162143502218401\\
52.64	0.00162143502192848\\
52.65	0.00162143502167047\\
52.66	0.00162143502140997\\
52.67	0.00162143502114695\\
52.68	0.00162143502088138\\
52.69	0.00162143502061325\\
52.7	0.00162143502034253\\
52.71	0.00162143502006919\\
52.72	0.0016214350197932\\
52.73	0.00162143501951456\\
52.74	0.00162143501923322\\
52.75	0.00162143501894916\\
52.76	0.00162143501866237\\
52.77	0.0016214350183728\\
52.78	0.00162143501808044\\
52.79	0.00162143501778526\\
52.8	0.00162143501748723\\
52.81	0.00162143501718633\\
52.82	0.00162143501688252\\
52.83	0.00162143501657579\\
52.84	0.0016214350162661\\
52.85	0.00162143501595342\\
52.86	0.00162143501563773\\
52.87	0.001621435015319\\
52.88	0.0016214350149972\\
52.89	0.0016214350146723\\
52.9	0.00162143501434427\\
52.91	0.00162143501401309\\
52.92	0.00162143501367871\\
52.93	0.00162143501334112\\
52.94	0.00162143501300028\\
52.95	0.00162143501265617\\
52.96	0.00162143501230874\\
52.97	0.00162143501195797\\
52.98	0.00162143501160383\\
52.99	0.00162143501124629\\
53	0.00162143501088531\\
53.01	0.00162143501052086\\
53.02	0.00162143501015292\\
53.03	0.00162143500978144\\
53.04	0.00162143500940639\\
53.05	0.00162143500902774\\
53.06	0.00162143500864546\\
53.07	0.00162143500825951\\
53.08	0.00162143500786986\\
53.09	0.00162143500747646\\
53.1	0.0016214350070793\\
53.11	0.00162143500667833\\
53.12	0.00162143500627351\\
53.13	0.00162143500586482\\
53.14	0.00162143500545221\\
53.15	0.00162143500503564\\
53.16	0.00162143500461509\\
53.17	0.00162143500419051\\
53.18	0.00162143500376187\\
53.19	0.00162143500332912\\
53.2	0.00162143500289223\\
53.21	0.00162143500245116\\
53.22	0.00162143500200587\\
53.23	0.00162143500155632\\
53.24	0.00162143500110247\\
53.25	0.00162143500064429\\
53.26	0.00162143500018173\\
53.27	0.00162143499971474\\
53.28	0.0016214349992433\\
53.29	0.00162143499876735\\
53.3	0.00162143499828685\\
53.31	0.00162143499780177\\
53.32	0.00162143499731206\\
53.33	0.00162143499681768\\
53.34	0.00162143499631857\\
53.35	0.00162143499581471\\
53.36	0.00162143499530604\\
53.37	0.00162143499479253\\
53.38	0.00162143499427411\\
53.39	0.00162143499375076\\
53.4	0.00162143499322242\\
53.41	0.00162143499268904\\
53.42	0.00162143499215059\\
53.43	0.001621434991607\\
53.44	0.00162143499105825\\
53.45	0.00162143499050427\\
53.46	0.00162143498994502\\
53.47	0.00162143498938045\\
53.48	0.0016214349888105\\
53.49	0.00162143498823514\\
53.5	0.00162143498765431\\
53.51	0.00162143498706795\\
53.52	0.00162143498647603\\
53.53	0.00162143498587848\\
53.54	0.00162143498527526\\
53.55	0.0016214349846663\\
53.56	0.00162143498405157\\
53.57	0.001621434983431\\
53.58	0.00162143498280454\\
53.59	0.00162143498217214\\
53.6	0.00162143498153374\\
53.61	0.00162143498088929\\
53.62	0.00162143498023872\\
53.63	0.00162143497958199\\
53.64	0.00162143497891903\\
53.65	0.0016214349782498\\
53.66	0.00162143497757422\\
53.67	0.00162143497689225\\
53.68	0.00162143497620382\\
53.69	0.00162143497550888\\
53.7	0.00162143497480736\\
53.71	0.0016214349740992\\
53.72	0.00162143497338434\\
53.73	0.00162143497266273\\
53.74	0.00162143497193429\\
53.75	0.00162143497119896\\
53.76	0.00162143497045669\\
53.77	0.00162143496970741\\
53.78	0.00162143496895105\\
53.79	0.00162143496818755\\
53.8	0.00162143496741684\\
53.81	0.00162143496663886\\
53.82	0.00162143496585353\\
53.83	0.0016214349650608\\
53.84	0.00162143496426059\\
53.85	0.00162143496345284\\
53.86	0.00162143496263747\\
53.87	0.00162143496181442\\
53.88	0.00162143496098362\\
53.89	0.00162143496014499\\
53.9	0.00162143495929846\\
53.91	0.00162143495844396\\
53.92	0.00162143495758142\\
53.93	0.00162143495671077\\
53.94	0.00162143495583192\\
53.95	0.00162143495494481\\
53.96	0.00162143495404936\\
53.97	0.00162143495314549\\
53.98	0.00162143495223313\\
53.99	0.0016214349513122\\
54	0.00162143495038261\\
54.01	0.0016214349494443\\
54.02	0.00162143494849718\\
54.03	0.00162143494754117\\
54.04	0.00162143494657619\\
54.05	0.00162143494560215\\
54.06	0.00162143494461899\\
54.07	0.0016214349436266\\
54.08	0.00162143494262492\\
54.09	0.00162143494161384\\
54.1	0.0016214349405933\\
54.11	0.0016214349395632\\
54.12	0.00162143493852345\\
54.13	0.00162143493747397\\
54.14	0.00162143493641467\\
54.15	0.00162143493534546\\
54.16	0.00162143493426625\\
54.17	0.00162143493317695\\
54.18	0.00162143493207747\\
54.19	0.00162143493096771\\
54.2	0.00162143492984758\\
54.21	0.001621434928717\\
54.22	0.00162143492757585\\
54.23	0.00162143492642405\\
54.24	0.0016214349252615\\
54.25	0.00162143492408811\\
54.26	0.00162143492290377\\
54.27	0.00162143492170839\\
54.28	0.00162143492050186\\
54.29	0.00162143491928409\\
54.3	0.00162143491805498\\
54.31	0.00162143491681441\\
54.32	0.0016214349155623\\
54.33	0.00162143491429853\\
54.34	0.00162143491302299\\
54.35	0.00162143491173559\\
54.36	0.00162143491043622\\
54.37	0.00162143490912477\\
54.38	0.00162143490780112\\
54.39	0.00162143490646517\\
54.4	0.00162143490511681\\
54.41	0.00162143490375593\\
54.42	0.00162143490238242\\
54.43	0.00162143490099615\\
54.44	0.00162143489959702\\
54.45	0.00162143489818491\\
54.46	0.0016214348967597\\
54.47	0.00162143489532128\\
54.48	0.00162143489386953\\
54.49	0.00162143489240433\\
54.5	0.00162143489092555\\
54.51	0.00162143488943308\\
54.52	0.0016214348879268\\
54.53	0.00162143488640657\\
54.54	0.00162143488487228\\
54.55	0.00162143488332379\\
54.56	0.00162143488176099\\
54.57	0.00162143488018375\\
54.58	0.00162143487859192\\
54.59	0.0016214348769854\\
54.6	0.00162143487536403\\
54.61	0.0016214348737277\\
54.62	0.00162143487207626\\
54.63	0.00162143487040959\\
54.64	0.00162143486872754\\
54.65	0.00162143486702999\\
54.66	0.00162143486531678\\
54.67	0.00162143486358779\\
54.68	0.00162143486184287\\
54.69	0.00162143486008188\\
54.7	0.00162143485830467\\
54.71	0.00162143485651111\\
54.72	0.00162143485470105\\
54.73	0.00162143485287434\\
54.74	0.00162143485103083\\
54.75	0.00162143484917037\\
54.76	0.00162143484729283\\
54.77	0.00162143484539803\\
54.78	0.00162143484348583\\
54.79	0.00162143484155608\\
54.8	0.00162143483960862\\
54.81	0.00162143483764329\\
54.82	0.00162143483565994\\
54.83	0.0016214348336584\\
54.84	0.00162143483163852\\
54.85	0.00162143482960013\\
54.86	0.00162143482754307\\
54.87	0.00162143482546718\\
54.88	0.00162143482337229\\
54.89	0.00162143482125822\\
54.9	0.00162143481912482\\
54.91	0.00162143481697191\\
54.92	0.00162143481479932\\
54.93	0.00162143481260688\\
54.94	0.00162143481039442\\
54.95	0.00162143480816174\\
54.96	0.00162143480590869\\
54.97	0.00162143480363508\\
54.98	0.00162143480134072\\
54.99	0.00162143479902544\\
55	0.00162143479668906\\
55.01	0.00162143479433139\\
55.02	0.00162143479195223\\
55.03	0.00162143478955141\\
55.04	0.00162143478712874\\
55.05	0.00162143478468401\\
55.06	0.00162143478221705\\
55.07	0.00162143477972765\\
55.08	0.00162143477721562\\
55.09	0.00162143477468076\\
55.1	0.00162143477212288\\
55.11	0.00162143476954177\\
55.12	0.00162143476693722\\
55.13	0.00162143476430904\\
55.14	0.00162143476165702\\
55.15	0.00162143475898095\\
55.16	0.00162143475628062\\
55.17	0.00162143475355583\\
55.18	0.00162143475080634\\
55.19	0.00162143474803197\\
55.2	0.00162143474523248\\
55.21	0.00162143474240766\\
55.22	0.00162143473955729\\
55.23	0.00162143473668114\\
55.24	0.001621434733779\\
55.25	0.00162143473085064\\
55.26	0.00162143472789584\\
55.27	0.00162143472491435\\
55.28	0.00162143472190596\\
55.29	0.00162143471887042\\
55.3	0.00162143471580751\\
55.31	0.00162143471271699\\
55.32	0.00162143470959862\\
55.33	0.00162143470645216\\
55.34	0.00162143470327737\\
55.35	0.001621434700074\\
55.36	0.00162143469684181\\
55.37	0.00162143469358055\\
55.38	0.00162143469028996\\
55.39	0.00162143468696981\\
55.4	0.00162143468361982\\
55.41	0.00162143468023976\\
55.42	0.00162143467682935\\
55.43	0.00162143467338834\\
55.44	0.00162143466991646\\
55.45	0.00162143466641346\\
55.46	0.00162143466287906\\
55.47	0.00162143465931299\\
55.48	0.00162143465571499\\
55.49	0.00162143465208478\\
55.5	0.00162143464842209\\
55.51	0.00162143464472663\\
55.52	0.00162143464099813\\
55.53	0.00162143463723632\\
55.54	0.00162143463344089\\
55.55	0.00162143462961157\\
55.56	0.00162143462574807\\
55.57	0.00162143462185009\\
55.58	0.00162143461791735\\
55.59	0.00162143461394954\\
55.6	0.00162143460994638\\
55.61	0.00162143460590755\\
55.62	0.00162143460183277\\
55.63	0.00162143459772171\\
55.64	0.00162143459357408\\
55.65	0.00162143458938956\\
55.66	0.00162143458516784\\
55.67	0.00162143458090862\\
55.68	0.00162143457661156\\
55.69	0.00162143457227635\\
55.7	0.00162143456790267\\
55.71	0.0016214345634902\\
55.72	0.0016214345590386\\
55.73	0.00162143455454754\\
55.74	0.00162143455001671\\
55.75	0.00162143454544575\\
55.76	0.00162143454083433\\
55.77	0.00162143453618211\\
55.78	0.00162143453148875\\
55.79	0.00162143452675391\\
55.8	0.00162143452197723\\
55.81	0.00162143451715836\\
55.82	0.00162143451229695\\
55.83	0.00162143450739265\\
55.84	0.00162143450244509\\
55.85	0.00162143449745391\\
55.86	0.00162143449241875\\
55.87	0.00162143448733925\\
55.88	0.00162143448221502\\
55.89	0.0016214344770457\\
55.9	0.00162143447183091\\
55.91	0.00162143446657027\\
55.92	0.00162143446126341\\
55.93	0.00162143445590993\\
55.94	0.00162143445050945\\
55.95	0.00162143444506157\\
55.96	0.00162143443956591\\
55.97	0.00162143443402207\\
55.98	0.00162143442842965\\
55.99	0.00162143442278824\\
56	0.00162143441709744\\
56.01	0.00162143441135684\\
56.02	0.00162143440556603\\
56.03	0.0016214343997246\\
56.04	0.00162143439383212\\
56.05	0.00162143438788818\\
56.06	0.00162143438189234\\
56.07	0.00162143437584419\\
56.08	0.00162143436974329\\
56.09	0.00162143436358922\\
56.1	0.00162143435738152\\
56.11	0.00162143435111976\\
56.12	0.00162143434480349\\
56.13	0.00162143433843228\\
56.14	0.00162143433200567\\
56.15	0.0016214343255232\\
56.16	0.00162143431898442\\
56.17	0.00162143431238886\\
56.18	0.00162143430573607\\
56.19	0.00162143429902557\\
56.2	0.0016214342922569\\
56.21	0.00162143428542958\\
56.22	0.00162143427854313\\
56.23	0.00162143427159707\\
56.24	0.00162143426459092\\
56.25	0.00162143425752418\\
56.26	0.00162143425039637\\
56.27	0.00162143424320699\\
56.28	0.00162143423595553\\
56.29	0.00162143422864151\\
56.3	0.0016214342212644\\
56.31	0.0016214342138237\\
56.32	0.0016214342063189\\
56.33	0.00162143419874948\\
56.34	0.00162143419111491\\
56.35	0.00162143418341467\\
56.36	0.00162143417564824\\
56.37	0.00162143416781507\\
56.38	0.00162143415991463\\
56.39	0.00162143415194638\\
56.4	0.00162143414390978\\
56.41	0.00162143413580428\\
56.42	0.00162143412762932\\
56.43	0.00162143411938435\\
56.44	0.0016214341110688\\
56.45	0.00162143410268212\\
56.46	0.00162143409422374\\
56.47	0.00162143408569307\\
56.48	0.00162143407708955\\
56.49	0.0016214340684126\\
56.5	0.00162143405966163\\
56.51	0.00162143405083604\\
56.52	0.00162143404193526\\
56.53	0.00162143403295867\\
56.54	0.00162143402390568\\
56.55	0.00162143401477569\\
56.56	0.00162143400556807\\
56.57	0.00162143399628223\\
56.58	0.00162143398691753\\
56.59	0.00162143397747337\\
56.6	0.0016214339679491\\
56.61	0.0016214339583441\\
56.62	0.00162143394865773\\
56.63	0.00162143393888935\\
56.64	0.00162143392903832\\
56.65	0.00162143391910398\\
56.66	0.00162143390908569\\
56.67	0.00162143389898278\\
56.68	0.0016214338887946\\
56.69	0.00162143387852046\\
56.7	0.00162143386815971\\
56.71	0.00162143385771167\\
56.72	0.00162143384717565\\
56.73	0.00162143383655096\\
56.74	0.00162143382583692\\
56.75	0.00162143381503284\\
56.76	0.001621433804138\\
56.77	0.00162143379315172\\
56.78	0.00162143378207327\\
56.79	0.00162143377090194\\
56.8	0.00162143375963702\\
56.81	0.00162143374827778\\
56.82	0.00162143373682348\\
56.83	0.00162143372527341\\
56.84	0.00162143371362681\\
56.85	0.00162143370188295\\
56.86	0.00162143369004107\\
56.87	0.00162143367810043\\
56.88	0.00162143366606026\\
56.89	0.0016214336539198\\
56.9	0.00162143364167829\\
56.91	0.00162143362933494\\
56.92	0.00162143361688898\\
56.93	0.00162143360433963\\
56.94	0.00162143359168609\\
56.95	0.00162143357892758\\
56.96	0.00162143356606329\\
56.97	0.00162143355309242\\
56.98	0.00162143354001416\\
56.99	0.00162143352682769\\
57	0.0016214335135322\\
57.01	0.00162143350012686\\
57.02	0.00162143348661083\\
57.03	0.00162143347298328\\
57.04	0.00162143345924338\\
57.05	0.00162143344539026\\
57.06	0.00162143343142309\\
57.07	0.001621433417341\\
57.08	0.00162143340314313\\
57.09	0.0016214333888286\\
57.1	0.00162143337439656\\
57.11	0.00162143335984611\\
57.12	0.00162143334517638\\
57.13	0.00162143333038646\\
57.14	0.00162143331547547\\
57.15	0.0016214333004425\\
57.16	0.00162143328528665\\
57.17	0.001621433270007\\
57.18	0.00162143325460263\\
57.19	0.00162143323907262\\
57.2	0.00162143322341604\\
57.21	0.00162143320763195\\
57.22	0.00162143319171941\\
57.23	0.00162143317567747\\
57.24	0.00162143315950519\\
57.25	0.00162143314320159\\
57.26	0.00162143312676572\\
57.27	0.00162143311019661\\
57.28	0.00162143309349328\\
57.29	0.00162143307665474\\
57.3	0.00162143305968001\\
57.31	0.0016214330425681\\
57.32	0.001621433025318\\
57.33	0.0016214330079287\\
57.34	0.00162143299039921\\
57.35	0.00162143297272849\\
57.36	0.00162143295491553\\
57.37	0.00162143293695929\\
57.38	0.00162143291885874\\
57.39	0.00162143290061284\\
57.4	0.00162143288222053\\
57.41	0.00162143286368077\\
57.42	0.00162143284499249\\
57.43	0.00162143282615463\\
57.44	0.00162143280716612\\
57.45	0.00162143278802587\\
57.46	0.0016214327687328\\
57.47	0.00162143274928582\\
57.48	0.00162143272968384\\
57.49	0.00162143270992574\\
57.5	0.00162143269001043\\
57.51	0.00162143266993678\\
57.52	0.00162143264970368\\
57.53	0.00162143262930998\\
57.54	0.00162143260875457\\
57.55	0.0016214325880363\\
57.56	0.00162143256715402\\
57.57	0.00162143254610657\\
57.58	0.00162143252489281\\
57.59	0.00162143250351156\\
57.6	0.00162143248196165\\
57.61	0.0016214324602419\\
57.62	0.00162143243835113\\
57.63	0.00162143241628814\\
57.64	0.00162143239405175\\
57.65	0.00162143237164073\\
57.66	0.00162143234905389\\
57.67	0.00162143232629001\\
57.68	0.00162143230334785\\
57.69	0.00162143228022621\\
57.7	0.00162143225692383\\
57.71	0.00162143223343947\\
57.72	0.00162143220977189\\
57.73	0.00162143218591983\\
57.74	0.00162143216188203\\
57.75	0.00162143213765723\\
57.76	0.00162143211324413\\
57.77	0.00162143208864148\\
57.78	0.00162143206384797\\
57.79	0.00162143203886232\\
57.8	0.00162143201368322\\
57.81	0.00162143198830936\\
57.82	0.00162143196273944\\
57.83	0.00162143193697213\\
57.84	0.00162143191100612\\
57.85	0.00162143188484006\\
57.86	0.00162143185847261\\
57.87	0.00162143183190244\\
57.88	0.00162143180512818\\
57.89	0.00162143177814849\\
57.9	0.001621431750962\\
57.91	0.00162143172356732\\
57.92	0.0016214316959631\\
57.93	0.00162143166814794\\
57.94	0.00162143164012046\\
57.95	0.00162143161187926\\
57.96	0.00162143158342293\\
57.97	0.00162143155475006\\
57.98	0.00162143152585924\\
57.99	0.00162143149674906\\
58	0.00162143146741807\\
58.01	0.00162143143786484\\
58.02	0.00162143140808794\\
58.03	0.00162143137808592\\
58.04	0.00162143134785732\\
58.05	0.00162143131740068\\
58.06	0.00162143128671454\\
58.07	0.00162143125579742\\
58.08	0.00162143122464785\\
58.09	0.00162143119326435\\
58.1	0.00162143116164541\\
58.11	0.00162143112978954\\
58.12	0.00162143109769525\\
58.13	0.00162143106536102\\
58.14	0.00162143103278534\\
58.15	0.00162143099996668\\
58.16	0.00162143096690351\\
58.17	0.00162143093359431\\
58.18	0.00162143090003754\\
58.19	0.00162143086623165\\
58.2	0.00162143083217508\\
58.21	0.00162143079786629\\
58.22	0.0016214307633037\\
58.23	0.00162143072848575\\
58.24	0.00162143069341087\\
58.25	0.00162143065807747\\
58.26	0.00162143062248397\\
58.27	0.00162143058662877\\
58.28	0.00162143055051029\\
58.29	0.0016214305141269\\
58.3	0.00162143047747702\\
58.31	0.00162143044055902\\
58.32	0.00162143040337128\\
58.33	0.00162143036591218\\
58.34	0.00162143032818009\\
58.35	0.00162143029017337\\
58.36	0.00162143025189038\\
58.37	0.00162143021332948\\
58.38	0.001621430174489\\
58.39	0.0016214301353673\\
58.4	0.00162143009596272\\
58.41	0.00162143005627358\\
58.42	0.00162143001629821\\
58.43	0.00162142997603495\\
58.44	0.00162142993548209\\
58.45	0.00162142989463797\\
58.46	0.00162142985350088\\
58.47	0.00162142981206913\\
58.48	0.00162142977034102\\
58.49	0.00162142972831484\\
58.5	0.00162142968598889\\
58.51	0.00162142964336145\\
58.52	0.00162142960043079\\
58.53	0.00162142955719521\\
58.54	0.00162142951365296\\
58.55	0.00162142946980232\\
58.56	0.00162142942564155\\
58.57	0.00162142938116891\\
58.58	0.00162142933638266\\
58.59	0.00162142929128105\\
58.6	0.00162142924586232\\
58.61	0.00162142920012473\\
58.62	0.00162142915406651\\
58.63	0.00162142910768591\\
58.64	0.00162142906098114\\
58.65	0.00162142901395046\\
58.66	0.00162142896659207\\
58.67	0.00162142891890422\\
58.68	0.00162142887088511\\
58.69	0.00162142882253296\\
58.7	0.001621428773846\\
58.71	0.00162142872482243\\
58.72	0.00162142867546046\\
58.73	0.0016214286257583\\
58.74	0.00162142857571416\\
58.75	0.00162142852532623\\
58.76	0.00162142847459271\\
58.77	0.0016214284235118\\
58.78	0.00162142837208171\\
58.79	0.00162142832030061\\
58.8	0.00162142826816671\\
58.81	0.00162142821567818\\
58.82	0.00162142816283323\\
58.83	0.00162142810963003\\
58.84	0.00162142805606677\\
58.85	0.00162142800214164\\
58.86	0.00162142794785281\\
58.87	0.00162142789319847\\
58.88	0.0016214278381768\\
58.89	0.00162142778278599\\
58.9	0.0016214277270242\\
58.91	0.00162142767088962\\
58.92	0.00162142761438042\\
58.93	0.0016214275574948\\
58.94	0.00162142750023091\\
58.95	0.00162142744258695\\
58.96	0.00162142738456108\\
58.97	0.0016214273261515\\
58.98	0.00162142726735637\\
58.99	0.00162142720817389\\
59	0.00162142714860222\\
59.01	0.00162142708863956\\
59.02	0.00162142702828408\\
59.03	0.00162142696753398\\
59.04	0.00162142690638742\\
59.05	0.00162142684484261\\
59.06	0.00162142678289774\\
59.07	0.00162142672055099\\
59.08	0.00162142665780056\\
59.09	0.00162142659464464\\
59.1	0.00162142653108144\\
59.11	0.00162142646710915\\
59.12	0.00162142640272598\\
59.13	0.00162142633793013\\
59.14	0.00162142627271982\\
59.15	0.00162142620709326\\
59.16	0.00162142614104867\\
59.17	0.00162142607458427\\
59.18	0.00162142600769828\\
59.19	0.00162142594038894\\
59.2	0.00162142587265447\\
59.21	0.00162142580449312\\
59.22	0.00162142573590314\\
59.23	0.00162142566688276\\
59.24	0.00162142559743025\\
59.25	0.00162142552754386\\
59.26	0.00162142545722187\\
59.27	0.00162142538646253\\
59.28	0.00162142531526413\\
59.29	0.00162142524362495\\
59.3	0.00162142517154329\\
59.31	0.00162142509901743\\
59.32	0.00162142502604569\\
59.33	0.00162142495262637\\
59.34	0.0016214248787578\\
59.35	0.00162142480443831\\
59.36	0.00162142472966622\\
59.37	0.00162142465443988\\
59.38	0.00162142457875765\\
59.39	0.00162142450261788\\
59.4	0.00162142442601895\\
59.41	0.00162142434895923\\
59.42	0.00162142427143712\\
59.43	0.001621424193451\\
59.44	0.0016214241149993\\
59.45	0.00162142403608043\\
59.46	0.00162142395669283\\
59.47	0.00162142387683492\\
59.48	0.00162142379650517\\
59.49	0.00162142371570204\\
59.5	0.001621423634424\\
59.51	0.00162142355266954\\
59.52	0.00162142347043717\\
59.53	0.00162142338772539\\
59.54	0.00162142330453273\\
59.55	0.00162142322085773\\
59.56	0.00162142313669895\\
59.57	0.00162142305205494\\
59.58	0.00162142296692429\\
59.59	0.0016214228813056\\
59.6	0.00162142279519747\\
59.61	0.00162142270859853\\
59.62	0.00162142262150742\\
59.63	0.0016214225339228\\
59.64	0.00162142244584333\\
59.65	0.00162142235726772\\
59.66	0.00162142226819465\\
59.67	0.00162142217862287\\
59.68	0.00162142208855109\\
59.69	0.00162142199797809\\
59.7	0.00162142190690264\\
59.71	0.00162142181532353\\
59.72	0.00162142172323958\\
59.73	0.00162142163064961\\
59.74	0.00162142153755248\\
59.75	0.00162142144394706\\
59.76	0.00162142134983225\\
59.77	0.00162142125520694\\
59.78	0.00162142116007008\\
59.79	0.00162142106442062\\
59.8	0.00162142096825754\\
59.81	0.00162142087157983\\
59.82	0.00162142077438651\\
59.83	0.00162142067667662\\
59.84	0.00162142057844924\\
59.85	0.00162142047970344\\
59.86	0.00162142038043834\\
59.87	0.00162142028065309\\
59.88	0.00162142018034683\\
59.89	0.00162142007951876\\
59.9	0.00162141997816809\\
59.91	0.00162141987629405\\
59.92	0.00162141977389591\\
59.93	0.00162141967097296\\
59.94	0.00162141956752452\\
59.95	0.00162141946354993\\
59.96	0.00162141935904856\\
59.97	0.00162141925401981\\
59.98	0.00162141914846311\\
59.99	0.00162141904237792\\
60	0.00162141893576373\\
60.01	0.00162141882862004\\
60.02	0.00162141872094641\\
60.03	0.00162141861274242\\
60.04	0.00162141850400767\\
60.05	0.0016214183947418\\
60.06	0.00162141828494448\\
60.07	0.00162141817461541\\
60.08	0.00162141806375434\\
60.09	0.00162141795236102\\
60.1	0.00162141784043526\\
60.11	0.00162141772797689\\
60.12	0.00162141761498579\\
60.13	0.00162141750146185\\
60.14	0.00162141738740502\\
60.15	0.00162141727281527\\
60.16	0.00162141715769262\\
60.17	0.0016214170420371\\
60.18	0.0016214169258488\\
60.19	0.00162141680912784\\
60.2	0.00162141669187438\\
60.21	0.00162141657408861\\
60.22	0.00162141645577076\\
60.23	0.00162141633692112\\
60.24	0.00162141621753998\\
60.25	0.00162141609762771\\
60.26	0.00162141597718468\\
60.27	0.00162141585621134\\
60.28	0.00162141573470814\\
60.29	0.00162141561267562\\
60.3	0.00162141549011431\\
60.31	0.00162141536702482\\
60.32	0.00162141524340778\\
60.33	0.00162141511926388\\
60.34	0.00162141499459383\\
60.35	0.00162141486939841\\
60.36	0.00162141474367842\\
60.37	0.00162141461743472\\
60.38	0.00162141449066822\\
60.39	0.00162141436337985\\
60.4	0.0016214142355706\\
60.41	0.00162141410724151\\
60.42	0.00162141397839367\\
60.43	0.00162141384902819\\
60.44	0.00162141371914625\\
60.45	0.00162141358874908\\
60.46	0.00162141345783795\\
60.47	0.00162141332641417\\
60.48	0.00162141319447909\\
60.49	0.00162141306203415\\
60.5	0.0016214129290808\\
60.51	0.00162141279562054\\
60.52	0.00162141266165495\\
60.53	0.00162141252718562\\
60.54	0.00162141239221423\\
60.55	0.00162141225674246\\
60.56	0.0016214121207721\\
60.57	0.00162141198430494\\
60.58	0.00162141184734286\\
60.59	0.00162141170988775\\
60.6	0.00162141157194159\\
60.61	0.00162141143350639\\
60.62	0.00162141129458421\\
60.63	0.00162141115517719\\
60.64	0.00162141101528748\\
60.65	0.00162141087491731\\
60.66	0.00162141073406897\\
60.67	0.00162141059274477\\
60.68	0.0016214104509471\\
60.69	0.0016214103086784\\
60.7	0.00162141016594115\\
60.71	0.0016214100227379\\
60.72	0.00162140987907123\\
60.73	0.00162140973494381\\
60.74	0.00162140959035833\\
60.75	0.00162140944531754\\
60.76	0.00162140929982426\\
60.77	0.00162140915388134\\
60.78	0.00162140900749171\\
60.79	0.00162140886065834\\
60.8	0.00162140871338424\\
60.81	0.00162140856567249\\
60.82	0.00162140841752623\\
60.83	0.00162140826894864\\
60.84	0.00162140811994295\\
60.85	0.00162140797051246\\
60.86	0.0016214078206605\\
60.87	0.00162140767039047\\
60.88	0.00162140751970583\\
60.89	0.00162140736861006\\
60.9	0.00162140721710673\\
60.91	0.00162140706519944\\
60.92	0.00162140691289184\\
60.93	0.00162140676018764\\
60.94	0.0016214066070906\\
60.95	0.00162140645360453\\
60.96	0.00162140629973328\\
60.97	0.00162140614548077\\
60.98	0.00162140599085095\\
60.99	0.00162140583584783\\
61	0.00162140568047546\\
61.01	0.00162140552473795\\
61.02	0.00162140536863944\\
61.03	0.00162140521218412\\
61.04	0.00162140505537625\\
61.05	0.00162140489822011\\
61.06	0.00162140474072003\\
61.07	0.00162140458288038\\
61.08	0.00162140442470558\\
61.09	0.00162140426620009\\
61.1	0.00162140410736842\\
61.11	0.00162140394821509\\
61.12	0.00162140378874471\\
61.13	0.00162140362896187\\
61.14	0.00162140346887124\\
61.15	0.00162140330847751\\
61.16	0.00162140314778542\\
61.17	0.00162140298679971\\
61.18	0.00162140282552518\\
61.19	0.00162140266396667\\
61.2	0.00162140250212901\\
61.21	0.00162140234001711\\
61.22	0.00162140217763587\\
61.23	0.00162140201499023\\
61.24	0.00162140185208515\\
61.25	0.00162140168892561\\
61.26	0.00162140152551662\\
61.27	0.00162140136186321\\
61.28	0.00162140119797042\\
61.29	0.0016214010338433\\
61.3	0.00162140086948693\\
61.31	0.0016214007049064\\
61.32	0.00162140054010679\\
61.33	0.00162140037509322\\
61.34	0.00162140020987078\\
61.35	0.00162140004444459\\
61.36	0.00162139987881977\\
61.37	0.00162139971300142\\
61.38	0.00162139954699466\\
61.39	0.00162139938080458\\
61.4	0.00162139921443628\\
61.41	0.00162139904789485\\
61.42	0.00162139888118534\\
61.43	0.00162139871431282\\
61.44	0.0016213985472823\\
61.45	0.00162139838009881\\
61.46	0.00162139821276733\\
61.47	0.00162139804529281\\
61.48	0.00162139787768016\\
61.49	0.00162139770993429\\
61.5	0.00162139754206003\\
61.51	0.00162139737406219\\
61.52	0.00162139720594552\\
61.53	0.00162139703771475\\
61.54	0.00162139686937451\\
61.55	0.0016213967009294\\
61.56	0.00162139653238397\\
61.57	0.00162139636374267\\
61.58	0.00162139619500991\\
61.59	0.00162139602619\\
61.6	0.0016213958572872\\
61.61	0.00162139568830566\\
61.62	0.00162139551924945\\
61.63	0.00162139535012254\\
61.64	0.00162139518092883\\
61.65	0.00162139501167207\\
61.66	0.00162139484235594\\
61.67	0.00162139467298399\\
61.68	0.00162139450355965\\
61.69	0.00162139433408621\\
61.7	0.00162139416456686\\
61.71	0.00162139399500463\\
61.72	0.00162139382540242\\
61.73	0.00162139365576295\\
61.74	0.00162139348608883\\
61.75	0.00162139331638247\\
61.76	0.00162139314664612\\
61.77	0.00162139297688186\\
61.78	0.00162139280709158\\
61.79	0.00162139263727699\\
61.8	0.00162139246743957\\
61.81	0.00162139229758065\\
61.82	0.00162139212770128\\
61.83	0.00162139195780233\\
61.84	0.00162139178788444\\
61.85	0.001621391617948\\
61.86	0.00162139144799315\\
61.87	0.00162139127801978\\
61.88	0.00162139110802751\\
61.89	0.0016213909380157\\
61.9	0.00162139076798342\\
61.91	0.00162139059792971\\
61.92	0.00162139042785361\\
61.93	0.00162139025775414\\
61.94	0.00162139008763032\\
61.95	0.00162138991748115\\
61.96	0.00162138974730563\\
61.97	0.00162138957710272\\
61.98	0.00162138940687141\\
61.99	0.00162138923661065\\
62	0.00162138906631938\\
62.01	0.00162138889599654\\
62.02	0.00162138872564106\\
62.03	0.00162138855525184\\
62.04	0.00162138838482779\\
62.05	0.00162138821436781\\
62.06	0.00162138804387077\\
62.07	0.00162138787333554\\
62.08	0.00162138770276098\\
62.09	0.00162138753214594\\
62.1	0.00162138736148925\\
62.11	0.00162138719078975\\
62.12	0.00162138702004625\\
62.13	0.00162138684925755\\
62.14	0.00162138667842244\\
62.15	0.00162138650753973\\
62.16	0.00162138633660817\\
62.17	0.00162138616562653\\
62.18	0.00162138599459357\\
62.19	0.00162138582350803\\
62.2	0.00162138565236865\\
62.21	0.00162138548117415\\
62.22	0.00162138530992324\\
62.23	0.00162138513861463\\
62.24	0.00162138496724701\\
62.25	0.00162138479581907\\
62.26	0.00162138462432949\\
62.27	0.00162138445277694\\
62.28	0.00162138428116006\\
62.29	0.00162138410947751\\
62.3	0.00162138393772794\\
62.31	0.00162138376590996\\
62.32	0.00162138359402221\\
62.33	0.0016213834220633\\
62.34	0.00162138325003183\\
62.35	0.00162138307792641\\
62.36	0.00162138290574561\\
62.37	0.00162138273348802\\
62.38	0.00162138256115222\\
62.39	0.00162138238873677\\
62.4	0.00162138221624023\\
62.41	0.00162138204366115\\
62.42	0.00162138187099807\\
62.43	0.00162138169824952\\
62.44	0.00162138152541405\\
62.45	0.00162138135249017\\
62.46	0.00162138117947639\\
62.47	0.00162138100637124\\
62.48	0.0016213808331732\\
62.49	0.00162138065988078\\
62.5	0.00162138048649247\\
62.51	0.00162138031300676\\
62.52	0.00162138013942212\\
62.53	0.00162137996573704\\
62.54	0.00162137979194997\\
62.55	0.00162137961805939\\
62.56	0.00162137944406376\\
62.57	0.00162137926996154\\
62.58	0.00162137909575116\\
62.59	0.00162137892143109\\
62.6	0.00162137874699977\\
62.61	0.00162137857245564\\
62.62	0.00162137839779712\\
62.63	0.00162137822302267\\
62.64	0.0016213780481307\\
62.65	0.00162137787311964\\
62.66	0.00162137769798793\\
62.67	0.00162137752273398\\
62.68	0.00162137734735622\\
62.69	0.00162137717185306\\
62.7	0.00162137699622292\\
62.71	0.00162137682046421\\
62.72	0.00162137664457536\\
62.73	0.00162137646855478\\
62.74	0.00162137629240087\\
62.75	0.00162137611611205\\
62.76	0.00162137593968674\\
62.77	0.00162137576312335\\
62.78	0.00162137558642029\\
62.79	0.00162137540957598\\
62.8	0.00162137523258882\\
62.81	0.00162137505545724\\
62.82	0.00162137487817966\\
62.83	0.00162137470075448\\
62.84	0.00162137452318013\\
62.85	0.00162137434545504\\
62.86	0.00162137416757762\\
62.87	0.00162137398954631\\
62.88	0.00162137381135954\\
62.89	0.00162137363301573\\
62.9	0.00162137345451332\\
62.91	0.00162137327585076\\
62.92	0.00162137309702648\\
62.93	0.00162137291803894\\
62.94	0.00162137273888659\\
62.95	0.00162137255956787\\
62.96	0.00162137238008126\\
62.97	0.00162137220042521\\
62.98	0.0016213720205982\\
62.99	0.0016213718405987\\
63	0.00162137166042519\\
63.01	0.00162137148007616\\
63.02	0.00162137129955011\\
63.03	0.00162137111884553\\
63.04	0.00162137093796092\\
63.05	0.00162137075689479\\
63.06	0.00162137057564567\\
63.07	0.00162137039421208\\
63.08	0.00162137021259255\\
63.09	0.00162137003078561\\
63.1	0.00162136984878982\\
63.11	0.00162136966660372\\
63.12	0.00162136948422588\\
63.13	0.00162136930165485\\
63.14	0.00162136911888923\\
63.15	0.00162136893592758\\
63.16	0.00162136875276851\\
63.17	0.00162136856941059\\
63.18	0.00162136838585246\\
63.19	0.0016213682020927\\
63.2	0.00162136801812996\\
63.21	0.00162136783396285\\
63.22	0.00162136764959001\\
63.23	0.00162136746501009\\
63.24	0.00162136728022174\\
63.25	0.00162136709522362\\
63.26	0.00162136691001439\\
63.27	0.00162136672459273\\
63.28	0.00162136653895733\\
63.29	0.00162136635310686\\
63.3	0.00162136616704003\\
63.31	0.00162136598075554\\
63.32	0.0016213657942521\\
63.33	0.00162136560752841\\
63.34	0.00162136542058321\\
63.35	0.00162136523341522\\
63.36	0.00162136504602316\\
63.37	0.00162136485840578\\
63.38	0.00162136467056181\\
63.39	0.00162136448249\\
63.4	0.00162136429418909\\
63.41	0.00162136410565784\\
63.42	0.001621363916895\\
63.43	0.00162136372789932\\
63.44	0.00162136353866956\\
63.45	0.00162136334920448\\
63.46	0.00162136315950282\\
63.47	0.00162136296956335\\
63.48	0.00162136277938482\\
63.49	0.00162136258896597\\
63.5	0.00162136239830556\\
63.51	0.00162136220740232\\
63.52	0.001621362016255\\
63.53	0.00162136182486232\\
63.54	0.001621361633223\\
63.55	0.00162136144133575\\
63.56	0.00162136124919929\\
63.57	0.0016213610568123\\
63.58	0.00162136086417347\\
63.59	0.00162136067128148\\
63.6	0.00162136047813498\\
63.61	0.00162136028473263\\
63.62	0.00162136009107308\\
63.63	0.00162135989715496\\
63.64	0.0016213597029769\\
63.65	0.00162135950853751\\
63.66	0.00162135931383539\\
63.67	0.00162135911886915\\
63.68	0.00162135892363736\\
63.69	0.0016213587281386\\
63.7	0.00162135853237142\\
63.71	0.00162135833633439\\
63.72	0.00162135814002605\\
63.73	0.00162135794344492\\
63.74	0.00162135774658953\\
63.75	0.00162135754945838\\
63.76	0.00162135735204998\\
63.77	0.00162135715436281\\
63.78	0.00162135695639535\\
63.79	0.00162135675814607\\
63.8	0.00162135655961342\\
63.81	0.00162135636079584\\
63.82	0.00162135616169177\\
63.83	0.00162135596229962\\
63.84	0.0016213557626178\\
63.85	0.00162135556264473\\
63.86	0.00162135536237877\\
63.87	0.0016213551618183\\
63.88	0.00162135496096169\\
63.89	0.00162135475980729\\
63.9	0.00162135455835344\\
63.91	0.00162135435659846\\
63.92	0.00162135415454068\\
63.93	0.00162135395217838\\
63.94	0.00162135374950987\\
63.95	0.00162135354653343\\
63.96	0.00162135334324731\\
63.97	0.00162135313964978\\
63.98	0.00162135293573907\\
63.99	0.00162135273151342\\
64	0.00162135252697104\\
64.01	0.00162135232211013\\
64.02	0.0016213521169289\\
64.03	0.00162135191142551\\
64.04	0.00162135170559813\\
64.05	0.00162135149944491\\
64.06	0.001621351292964\\
64.07	0.00162135108615352\\
64.08	0.00162135087901158\\
64.09	0.00162135067153629\\
64.1	0.00162135046372573\\
64.11	0.00162135025557797\\
64.12	0.00162135004709107\\
64.13	0.00162134983826309\\
64.14	0.00162134962909205\\
64.15	0.00162134941957597\\
64.16	0.00162134920971286\\
64.17	0.00162134899950071\\
64.18	0.0016213487889375\\
64.19	0.00162134857802118\\
64.2	0.00162134836674972\\
64.21	0.00162134815512104\\
64.22	0.00162134794313307\\
64.23	0.00162134773078371\\
64.24	0.00162134751807086\\
64.25	0.00162134730499239\\
64.26	0.00162134709154618\\
64.27	0.00162134687773005\\
64.28	0.00162134666354186\\
64.29	0.00162134644897942\\
64.3	0.00162134623404054\\
64.31	0.001621346018723\\
64.32	0.00162134580302458\\
64.33	0.00162134558694304\\
64.34	0.00162134537047613\\
64.35	0.00162134515362157\\
64.36	0.00162134493637708\\
64.37	0.00162134471874036\\
64.38	0.00162134450070909\\
64.39	0.00162134428228094\\
64.4	0.00162134406345356\\
64.41	0.00162134384422459\\
64.42	0.00162134362459166\\
64.43	0.00162134340455236\\
64.44	0.00162134318410428\\
64.45	0.00162134296324501\\
64.46	0.00162134274197209\\
64.47	0.00162134252028307\\
64.48	0.00162134229817548\\
64.49	0.00162134207564682\\
64.5	0.00162134185269459\\
64.51	0.00162134162931626\\
64.52	0.0016213414055093\\
64.53	0.00162134118127114\\
64.54	0.00162134095659923\\
64.55	0.00162134073149096\\
64.56	0.00162134050594372\\
64.57	0.00162134027995491\\
64.58	0.00162134005352188\\
64.59	0.00162133982664196\\
64.6	0.0016213395993125\\
64.61	0.00162133937153079\\
64.62	0.00162133914329413\\
64.63	0.0016213389145998\\
64.64	0.00162133868544506\\
64.65	0.00162133845582713\\
64.66	0.00162133822574326\\
64.67	0.00162133799519064\\
64.68	0.00162133776416646\\
64.69	0.00162133753266789\\
64.7	0.0016213373006921\\
64.71	0.0016213370682362\\
64.72	0.00162133683529733\\
64.73	0.00162133660187257\\
64.74	0.00162133636795902\\
64.75	0.00162133613355374\\
64.76	0.00162133589865377\\
64.77	0.00162133566325614\\
64.78	0.00162133542735786\\
64.79	0.00162133519095593\\
64.8	0.00162133495404732\\
64.81	0.00162133471662898\\
64.82	0.00162133447869786\\
64.83	0.00162133424025086\\
64.84	0.0016213340012849\\
64.85	0.00162133376179684\\
64.86	0.00162133352178357\\
64.87	0.00162133328124192\\
64.88	0.00162133304016871\\
64.89	0.00162133279856077\\
64.9	0.00162133255641487\\
64.91	0.00162133231372778\\
64.92	0.00162133207049626\\
64.93	0.00162133182671705\\
64.94	0.00162133158238684\\
64.95	0.00162133133750235\\
64.96	0.00162133109206024\\
64.97	0.00162133084605717\\
64.98	0.00162133059948978\\
64.99	0.00162133035235469\\
65	0.0016213301046485\\
65.01	0.00162132985636778\\
65.02	0.00162132960750911\\
65.03	0.00162132935806901\\
65.04	0.00162132910804402\\
65.05	0.00162132885743064\\
65.06	0.00162132860622536\\
65.07	0.00162132835442463\\
65.08	0.0016213281020249\\
65.09	0.0016213278490226\\
65.1	0.00162132759541414\\
65.11	0.0016213273411959\\
65.12	0.00162132708636425\\
65.13	0.00162132683091554\\
65.14	0.0016213265748461\\
65.15	0.00162132631815223\\
65.16	0.00162132606083024\\
65.17	0.00162132580287638\\
65.18	0.00162132554428692\\
65.19	0.00162132528505807\\
65.2	0.00162132502518605\\
65.21	0.00162132476466706\\
65.22	0.00162132450349727\\
65.23	0.00162132424167283\\
65.24	0.00162132397918987\\
65.25	0.00162132371604451\\
65.26	0.00162132345223284\\
65.27	0.00162132318775094\\
65.28	0.00162132292259487\\
65.29	0.00162132265676065\\
65.3	0.00162132239024431\\
65.31	0.00162132212304185\\
65.32	0.00162132185514923\\
65.33	0.00162132158656242\\
65.34	0.00162132131727736\\
65.35	0.00162132104728997\\
65.36	0.00162132077659614\\
65.37	0.00162132050519176\\
65.38	0.0016213202330727\\
65.39	0.00162131996023478\\
65.4	0.00162131968667383\\
65.41	0.00162131941238567\\
65.42	0.00162131913736606\\
65.43	0.00162131886161079\\
65.44	0.00162131858511559\\
65.45	0.00162131830787619\\
65.46	0.0016213180298883\\
65.47	0.00162131775114761\\
65.48	0.00162131747164979\\
65.49	0.00162131719139049\\
65.5	0.00162131691036535\\
65.51	0.00162131662856999\\
65.52	0.00162131634599998\\
65.53	0.00162131606265093\\
65.54	0.00162131577851837\\
65.55	0.00162131549359786\\
65.56	0.00162131520788492\\
65.57	0.00162131492137506\\
65.58	0.00162131463406375\\
65.59	0.00162131434594646\\
65.6	0.00162131405701866\\
65.61	0.00162131376727577\\
65.62	0.0016213134767132\\
65.63	0.00162131318532635\\
65.64	0.00162131289311061\\
65.65	0.00162131260006134\\
65.66	0.00162131230617388\\
65.67	0.00162131201144356\\
65.68	0.0016213117158657\\
65.69	0.00162131141943558\\
65.7	0.00162131112214849\\
65.71	0.00162131082399968\\
65.72	0.00162131052498441\\
65.73	0.0016213102250979\\
65.74	0.00162130992433536\\
65.75	0.001621309622692\\
65.76	0.00162130932016298\\
65.77	0.00162130901674349\\
65.78	0.00162130871242866\\
65.79	0.00162130840721363\\
65.8	0.00162130810109353\\
65.81	0.00162130779406345\\
65.82	0.00162130748611849\\
65.83	0.00162130717725372\\
65.84	0.0016213068674642\\
65.85	0.00162130655674498\\
65.86	0.00162130624509109\\
65.87	0.00162130593249756\\
65.88	0.00162130561895938\\
65.89	0.00162130530447155\\
65.9	0.00162130498902904\\
65.91	0.00162130467262683\\
65.92	0.00162130435525987\\
65.93	0.0016213040369231\\
65.94	0.00162130371761144\\
65.95	0.00162130339731982\\
65.96	0.00162130307604314\\
65.97	0.00162130275377629\\
65.98	0.00162130243051417\\
65.99	0.00162130210625163\\
66	0.00162130178098355\\
66.01	0.00162130145470477\\
66.02	0.00162130112741014\\
66.03	0.00162130079909448\\
66.04	0.00162130046975262\\
66.05	0.00162130013937938\\
66.06	0.00162129980796955\\
66.07	0.00162129947551794\\
66.08	0.00162129914201933\\
66.09	0.00162129880746851\\
66.1	0.00162129847186024\\
66.11	0.00162129813518929\\
66.12	0.00162129779745042\\
66.13	0.00162129745863839\\
66.14	0.00162129711874794\\
66.15	0.00162129677777381\\
66.16	0.00162129643571074\\
66.17	0.00162129609255346\\
66.18	0.0016212957482967\\
66.19	0.00162129540293517\\
66.2	0.0016212950564636\\
66.21	0.0016212947088767\\
66.22	0.0016212943601692\\
66.23	0.00162129401033578\\
66.24	0.00162129365937117\\
66.25	0.00162129330727008\\
66.26	0.0016212929540272\\
66.27	0.00162129259963725\\
66.28	0.00162129224409492\\
66.29	0.00162129188739493\\
66.3	0.00162129152953198\\
66.31	0.00162129117050077\\
66.32	0.00162129081029603\\
66.33	0.00162129044891245\\
66.34	0.00162129008634475\\
66.35	0.00162128972258766\\
66.36	0.00162128935763589\\
66.37	0.00162128899148416\\
66.38	0.00162128862412722\\
66.39	0.00162128825555979\\
66.4	0.00162128788577662\\
66.41	0.00162128751477246\\
66.42	0.00162128714254206\\
66.43	0.00162128676908019\\
66.44	0.00162128639438162\\
66.45	0.00162128601844113\\
66.46	0.00162128564125351\\
66.47	0.00162128526281356\\
66.48	0.0016212848831161\\
66.49	0.00162128450215595\\
66.5	0.00162128411992794\\
66.51	0.00162128373642692\\
66.52	0.00162128335164775\\
66.53	0.00162128296558532\\
66.54	0.0016212825782345\\
66.55	0.00162128218959021\\
66.56	0.00162128179964736\\
66.57	0.0016212814084009\\
66.58	0.00162128101584578\\
66.59	0.00162128062197698\\
66.6	0.00162128022678949\\
66.61	0.00162127983027832\\
66.62	0.00162127943243851\\
66.63	0.00162127903326512\\
66.64	0.00162127863275321\\
66.65	0.0016212782308979\\
66.66	0.0016212778276943\\
66.67	0.00162127742313756\\
66.68	0.00162127701722286\\
66.69	0.0016212766099454\\
66.7	0.0016212762013004\\
66.71	0.00162127579128312\\
66.72	0.00162127537988883\\
66.73	0.00162127496711286\\
66.74	0.00162127455295055\\
66.75	0.00162127413739726\\
66.76	0.00162127372044841\\
66.77	0.00162127330209942\\
66.78	0.00162127288234579\\
66.79	0.00162127246118301\\
66.8	0.00162127203860662\\
66.81	0.00162127161461222\\
66.82	0.0016212711891954\\
66.83	0.00162127076235183\\
66.84	0.00162127033407721\\
66.85	0.00162126990436727\\
66.86	0.00162126947321779\\
66.87	0.00162126904062458\\
66.88	0.00162126860658351\\
66.89	0.0016212681710905\\
66.9	0.00162126773414148\\
66.91	0.00162126729573246\\
66.92	0.0016212668558595\\
66.93	0.00162126641451867\\
66.94	0.00162126597170614\\
66.95	0.00162126552741809\\
66.96	0.00162126508165077\\
66.97	0.00162126463440048\\
66.98	0.00162126418566358\\
66.99	0.00162126373543648\\
67	0.00162126328371565\\
67.01	0.0016212628304976\\
67.02	0.00162126237577891\\
67.03	0.00162126191955624\\
67.04	0.00162126146182627\\
67.05	0.00162126100258578\\
67.06	0.00162126054183159\\
67.07	0.00162126007956059\\
67.08	0.00162125961576973\\
67.09	0.00162125915045605\\
67.1	0.00162125868361662\\
67.11	0.00162125821524861\\
67.12	0.00162125774534924\\
67.13	0.00162125727391582\\
67.14	0.00162125680094573\\
67.15	0.0016212563264364\\
67.16	0.00162125585038535\\
67.17	0.00162125537279019\\
67.18	0.0016212548936486\\
67.19	0.00162125441295831\\
67.2	0.00162125393071718\\
67.21	0.00162125344692311\\
67.22	0.0016212529615741\\
67.23	0.00162125247466823\\
67.24	0.00162125198620367\\
67.25	0.00162125149617868\\
67.26	0.00162125100459159\\
67.27	0.00162125051144082\\
67.28	0.00162125001672491\\
67.29	0.00162124952044246\\
67.3	0.00162124902259217\\
67.31	0.00162124852317284\\
67.32	0.00162124802218337\\
67.33	0.00162124751962275\\
67.34	0.00162124701549005\\
67.35	0.00162124650978448\\
67.36	0.00162124600250531\\
67.37	0.00162124549365194\\
67.38	0.00162124498322385\\
67.39	0.00162124447122065\\
67.4	0.00162124395764204\\
67.41	0.00162124344248782\\
67.42	0.00162124292575793\\
67.43	0.00162124240745237\\
67.44	0.0016212418875713\\
67.45	0.00162124136611496\\
67.46	0.00162124084308372\\
67.47	0.00162124031847806\\
67.48	0.00162123979229857\\
67.49	0.00162123926454597\\
67.5	0.00162123873522109\\
67.51	0.00162123820432488\\
67.52	0.00162123767185842\\
67.53	0.00162123713782289\\
67.54	0.00162123660221963\\
67.55	0.00162123606505007\\
67.56	0.00162123552631578\\
67.57	0.00162123498601846\\
67.58	0.00162123444415994\\
67.59	0.00162123390074218\\
67.6	0.00162123335576725\\
67.61	0.00162123280923738\\
67.62	0.00162123226115492\\
67.63	0.00162123171152236\\
67.64	0.00162123116034232\\
67.65	0.00162123060761755\\
67.66	0.00162123005335095\\
67.67	0.00162122949754556\\
67.68	0.00162122894020455\\
67.69	0.00162122838133122\\
67.7	0.00162122782092905\\
67.71	0.00162122725900162\\
67.72	0.00162122669555269\\
67.73	0.00162122613058612\\
67.74	0.00162122556410597\\
67.75	0.0016212249961164\\
67.76	0.00162122442662173\\
67.77	0.00162122385562645\\
67.78	0.00162122328313517\\
67.79	0.00162122270915267\\
67.8	0.00162122213368385\\
67.81	0.0016212215567338\\
67.82	0.00162122097830773\\
67.83	0.00162122039841102\\
67.84	0.0016212198170492\\
67.85	0.00162121923422793\\
67.86	0.00162121864995307\\
67.87	0.00162121806423058\\
67.88	0.00162121747706661\\
67.89	0.00162121688846746\\
67.9	0.00162121629843957\\
67.91	0.00162121570698955\\
67.92	0.00162121511412416\\
67.93	0.0016212145198503\\
67.94	0.00162121392417504\\
67.95	0.00162121332710562\\
67.96	0.00162121272864941\\
67.97	0.00162121212881395\\
67.98	0.00162121152760691\\
67.99	0.00162121092503615\\
68	0.00162121032110967\\
68.01	0.0016212097158356\\
68.02	0.00162120910922227\\
68.03	0.00162120850127811\\
68.04	0.00162120789201175\\
68.05	0.00162120728143194\\
68.06	0.00162120666954758\\
68.07	0.00162120605636775\\
68.08	0.00162120544190165\\
68.09	0.00162120482615863\\
68.1	0.00162120420914819\\
68.11	0.00162120359087999\\
68.12	0.00162120297136381\\
68.13	0.00162120235060958\\
68.14	0.00162120172862739\\
68.15	0.00162120110542743\\
68.16	0.00162120048102006\\
68.17	0.00162119985541577\\
68.18	0.00162119922862517\\
68.19	0.00162119860065901\\
68.2	0.00162119797152817\\
68.21	0.00162119734124365\\
68.22	0.00162119670981658\\
68.23	0.0016211960772582\\
68.24	0.00162119544357989\\
68.25	0.00162119480879312\\
68.26	0.0016211941729095\\
68.27	0.00162119353594071\\
68.28	0.00162119289789858\\
68.29	0.00162119225879501\\
68.3	0.00162119161864202\\
68.31	0.0016211909774517\\
68.32	0.00162119033523624\\
68.33	0.00162118969200793\\
68.34	0.00162118904777914\\
68.35	0.00162118840256228\\
68.36	0.00162118775636988\\
68.37	0.00162118710921452\\
68.38	0.00162118646110883\\
68.39	0.0016211858120655\\
68.4	0.00162118516209729\\
68.41	0.00162118451121698\\
68.42	0.0016211838594374\\
68.43	0.00162118320677142\\
68.44	0.00162118255323192\\
68.45	0.00162118189883182\\
68.46	0.00162118124358403\\
68.47	0.00162118058750147\\
68.48	0.00162117993059708\\
68.49	0.00162117927288376\\
68.5	0.0016211786143744\\
68.51	0.00162117795508188\\
68.52	0.00162117729501903\\
68.53	0.00162117663419863\\
68.54	0.00162117597263342\\
68.55	0.00162117531033606\\
68.56	0.00162117464731917\\
68.57	0.00162117398359525\\
68.58	0.00162117331917672\\
68.59	0.00162117265407591\\
68.6	0.00162117198830502\\
68.61	0.00162117132187613\\
68.62	0.00162117065480116\\
68.63	0.00162116998709192\\
68.64	0.00162116931876001\\
68.65	0.0016211686498169\\
68.66	0.00162116798027384\\
68.67	0.00162116731014187\\
68.68	0.00162116663943184\\
68.69	0.00162116596815436\\
68.7	0.00162116529631977\\
68.71	0.00162116462393817\\
68.72	0.00162116395101938\\
68.73	0.00162116327757292\\
68.74	0.00162116260360799\\
68.75	0.00162116192913347\\
68.76	0.00162116125415789\\
68.77	0.00162116057868943\\
68.78	0.00162115990273586\\
68.79	0.00162115922630456\\
68.8	0.00162115854940248\\
68.81	0.00162115787203614\\
68.82	0.00162115719421159\\
68.83	0.00162115651593437\\
68.84	0.00162115583720954\\
68.85	0.00162115515804163\\
68.86	0.00162115447843459\\
68.87	0.00162115379839181\\
68.88	0.00162115311791607\\
68.89	0.00162115243700952\\
68.9	0.00162115175567366\\
68.91	0.00162115107390931\\
68.92	0.00162115039171657\\
68.93	0.00162114970909482\\
68.94	0.00162114902604331\\
68.95	0.0016211483425613\\
68.96	0.00162114765864803\\
68.97	0.00162114697430276\\
68.98	0.00162114628952474\\
68.99	0.00162114560431321\\
69	0.00162114491866741\\
69.01	0.00162114423258658\\
69.02	0.00162114354606996\\
69.03	0.00162114285911679\\
69.04	0.0016211421717263\\
69.05	0.00162114148389771\\
69.06	0.00162114079563027\\
69.07	0.00162114010692318\\
69.08	0.00162113941777569\\
69.09	0.00162113872818699\\
69.1	0.00162113803815633\\
69.11	0.0016211373476829\\
69.12	0.00162113665676593\\
69.13	0.00162113596540461\\
69.14	0.00162113527359818\\
69.15	0.00162113458134582\\
69.16	0.00162113388864674\\
69.17	0.00162113319550014\\
69.18	0.00162113250190522\\
69.19	0.00162113180786118\\
69.2	0.0016211311133672\\
69.21	0.00162113041842248\\
69.22	0.00162112972302621\\
69.23	0.00162112902717756\\
69.24	0.00162112833087574\\
69.25	0.0016211276341199\\
69.26	0.00162112693690924\\
69.27	0.00162112623924292\\
69.28	0.00162112554112012\\
69.29	0.00162112484254\\
69.3	0.00162112414350175\\
69.31	0.0016211234440045\\
69.32	0.00162112274404744\\
69.33	0.00162112204362972\\
69.34	0.00162112134275049\\
69.35	0.00162112064140891\\
69.36	0.00162111993960413\\
69.37	0.00162111923733529\\
69.38	0.00162111853460155\\
69.39	0.00162111783140204\\
69.4	0.00162111712773591\\
69.41	0.00162111642360228\\
69.42	0.00162111571900031\\
69.43	0.00162111501392911\\
69.44	0.00162111430838781\\
69.45	0.00162111360237555\\
69.46	0.00162111289589145\\
69.47	0.00162111218893462\\
69.48	0.00162111148150418\\
69.49	0.00162111077359925\\
69.5	0.00162111006521893\\
69.51	0.00162110935636235\\
69.52	0.0016211086470286\\
69.53	0.00162110793721679\\
69.54	0.00162110722692602\\
69.55	0.00162110651615538\\
69.56	0.00162110580490397\\
69.57	0.00162110509317088\\
69.58	0.0016211043809552\\
69.59	0.00162110366825602\\
69.6	0.00162110295507241\\
69.61	0.00162110224140346\\
69.62	0.00162110152724824\\
69.63	0.00162110081260584\\
69.64	0.00162110009747531\\
69.65	0.00162109938185573\\
69.66	0.00162109866574615\\
69.67	0.00162109794914565\\
69.68	0.00162109723205328\\
69.69	0.0016210965144681\\
69.7	0.00162109579638915\\
69.71	0.00162109507781549\\
69.72	0.00162109435874616\\
69.73	0.00162109363918021\\
69.74	0.00162109291911668\\
69.75	0.00162109219855459\\
69.76	0.00162109147749299\\
69.77	0.00162109075593091\\
69.78	0.00162109003386737\\
69.79	0.0016210893113014\\
69.8	0.00162108858823201\\
69.81	0.00162108786465822\\
69.82	0.00162108714057905\\
69.83	0.00162108641599351\\
69.84	0.00162108569090061\\
69.85	0.00162108496529935\\
69.86	0.00162108423918872\\
69.87	0.00162108351256773\\
69.88	0.00162108278543538\\
69.89	0.00162108205779064\\
69.9	0.00162108132963251\\
69.91	0.00162108060095997\\
69.92	0.00162107987177201\\
69.93	0.0016210791420676\\
69.94	0.0016210784118457\\
69.95	0.0016210776811053\\
69.96	0.00162107694984536\\
69.97	0.00162107621806483\\
69.98	0.00162107548576269\\
69.99	0.00162107475293788\\
70	0.00162107401958936\\
70.01	0.00162107328571607\\
70.02	0.00162107255131696\\
70.03	0.00162107181639098\\
70.04	0.00162107108093705\\
70.05	0.00162107034495411\\
70.06	0.00162106960844109\\
70.07	0.00162106887139692\\
70.08	0.00162106813382052\\
70.09	0.00162106739571081\\
70.1	0.0016210666570667\\
70.11	0.0016210659178871\\
70.12	0.00162106517817093\\
70.13	0.00162106443791708\\
70.14	0.00162106369712445\\
70.15	0.00162106295579195\\
70.16	0.00162106221391845\\
70.17	0.00162106147150285\\
70.18	0.00162106072854403\\
70.19	0.00162105998504087\\
70.2	0.00162105924099224\\
70.21	0.00162105849639703\\
70.22	0.00162105775125408\\
70.23	0.00162105700556228\\
70.24	0.00162105625932047\\
70.25	0.00162105551252752\\
70.26	0.00162105476518226\\
70.27	0.00162105401728356\\
70.28	0.00162105326883025\\
70.29	0.00162105251982116\\
70.3	0.00162105177025514\\
70.31	0.00162105102013101\\
70.32	0.0016210502694476\\
70.33	0.00162104951820373\\
70.34	0.00162104876639821\\
70.35	0.00162104801402986\\
70.36	0.00162104726109748\\
70.37	0.00162104650759988\\
70.38	0.00162104575353586\\
70.39	0.00162104499890421\\
70.4	0.00162104424370372\\
70.41	0.00162104348793317\\
70.42	0.00162104273159134\\
70.43	0.00162104197467702\\
70.44	0.00162104121718897\\
70.45	0.00162104045912596\\
70.46	0.00162103970048674\\
70.47	0.00162103894127009\\
70.48	0.00162103818147474\\
70.49	0.00162103742109945\\
70.5	0.00162103666014297\\
70.51	0.00162103589860402\\
70.52	0.00162103513648133\\
70.53	0.00162103437377365\\
70.54	0.00162103361047969\\
70.55	0.00162103284659817\\
70.56	0.0016210320821278\\
70.57	0.00162103131706729\\
70.58	0.00162103055141535\\
70.59	0.00162102978517067\\
70.6	0.00162102901833195\\
70.61	0.00162102825089787\\
70.62	0.00162102748286711\\
70.63	0.00162102671423836\\
70.64	0.00162102594501028\\
70.65	0.00162102517518155\\
70.66	0.00162102440475082\\
70.67	0.00162102363371675\\
70.68	0.001621022862078\\
70.69	0.0016210220898332\\
70.7	0.00162102131698101\\
70.71	0.00162102054352005\\
70.72	0.00162101976944895\\
70.73	0.00162101899476634\\
70.74	0.00162101821947083\\
70.75	0.00162101744356105\\
70.76	0.0016210166670356\\
70.77	0.00162101588989307\\
70.78	0.00162101511213208\\
70.79	0.0016210143337512\\
70.8	0.00162101355474902\\
70.81	0.00162101277512413\\
70.82	0.0016210119948751\\
70.83	0.00162101121400049\\
70.84	0.00162101043249887\\
70.85	0.00162100965036879\\
70.86	0.00162100886760882\\
70.87	0.00162100808421748\\
70.88	0.00162100730019333\\
70.89	0.00162100651553489\\
70.9	0.00162100573024069\\
70.91	0.00162100494430927\\
70.92	0.00162100415773912\\
70.93	0.00162100337052876\\
70.94	0.00162100258267671\\
70.95	0.00162100179418145\\
70.96	0.00162100100504147\\
70.97	0.00162100021525528\\
70.98	0.00162099942482134\\
70.99	0.00162099863373812\\
71	0.00162099784200411\\
71.01	0.00162099704961776\\
71.02	0.00162099625657753\\
71.03	0.00162099546288187\\
71.04	0.00162099466852923\\
71.05	0.00162099387351804\\
71.06	0.00162099307784673\\
71.07	0.00162099228151374\\
71.08	0.00162099148451748\\
71.09	0.00162099068685637\\
71.1	0.00162098988852882\\
71.11	0.00162098908953323\\
71.12	0.001620988289868\\
71.13	0.00162098748953151\\
71.14	0.00162098668852215\\
71.15	0.00162098588683831\\
71.16	0.00162098508447834\\
71.17	0.00162098428144063\\
71.18	0.00162098347772352\\
71.19	0.00162098267332537\\
71.2	0.00162098186824454\\
71.21	0.00162098106247935\\
71.22	0.00162098025602815\\
71.23	0.00162097944888927\\
71.24	0.00162097864106103\\
71.25	0.00162097783254174\\
71.26	0.00162097702332973\\
71.27	0.00162097621342329\\
71.28	0.00162097540282072\\
71.29	0.00162097459152032\\
71.3	0.00162097377952036\\
71.31	0.00162097296681915\\
71.32	0.00162097215341494\\
71.33	0.00162097133930601\\
71.34	0.00162097052449062\\
71.35	0.00162096970896702\\
71.36	0.00162096889273347\\
71.37	0.00162096807578822\\
71.38	0.00162096725812949\\
71.39	0.00162096643975553\\
71.4	0.00162096562066455\\
71.41	0.00162096480085478\\
71.42	0.00162096398032444\\
71.43	0.00162096315907173\\
71.44	0.00162096233709486\\
71.45	0.00162096151439202\\
71.46	0.0016209606909614\\
71.47	0.00162095986680119\\
71.48	0.00162095904190956\\
71.49	0.0016209582162847\\
71.5	0.00162095738992476\\
71.51	0.00162095656282792\\
71.52	0.00162095573499232\\
71.53	0.00162095490641611\\
71.54	0.00162095407709744\\
71.55	0.00162095324703446\\
71.56	0.00162095241622528\\
71.57	0.00162095158466804\\
71.58	0.00162095075236086\\
71.59	0.00162094991930186\\
71.6	0.00162094908548914\\
71.61	0.00162094825092081\\
71.62	0.00162094741559498\\
71.63	0.00162094657950973\\
71.64	0.00162094574266315\\
71.65	0.00162094490505334\\
71.66	0.00162094406667835\\
71.67	0.00162094322753628\\
71.68	0.00162094238762518\\
71.69	0.00162094154694311\\
71.7	0.00162094070548814\\
71.71	0.00162093986325831\\
71.72	0.00162093902025168\\
71.73	0.00162093817646628\\
71.74	0.00162093733190015\\
71.75	0.00162093648655132\\
71.76	0.00162093564041782\\
71.77	0.00162093479349767\\
71.78	0.00162093394578889\\
71.79	0.0016209330972895\\
71.8	0.00162093224799749\\
71.81	0.00162093139791087\\
71.82	0.00162093054702764\\
71.83	0.0016209296953458\\
71.84	0.00162092884286333\\
71.85	0.00162092798957822\\
71.86	0.00162092713548846\\
71.87	0.00162092628059201\\
71.88	0.00162092542488686\\
71.89	0.00162092456837096\\
71.9	0.0016209237110423\\
71.91	0.00162092285289881\\
71.92	0.00162092199393848\\
71.93	0.00162092113415924\\
71.94	0.00162092027355905\\
71.95	0.00162091941213585\\
71.96	0.00162091854988759\\
71.97	0.00162091768681221\\
71.98	0.00162091682290764\\
71.99	0.00162091595817182\\
72	0.00162091509260267\\
72.01	0.00162091422619813\\
72.02	0.00162091335895611\\
72.03	0.00162091249087454\\
72.04	0.00162091162195134\\
72.05	0.00162091075218443\\
72.06	0.00162090988157171\\
72.07	0.0016209090101111\\
72.08	0.0016209081378005\\
72.09	0.00162090726463784\\
72.1	0.001620906390621\\
72.11	0.0016209055157479\\
72.12	0.00162090464001643\\
72.13	0.0016209037634245\\
72.14	0.00162090288597\\
72.15	0.00162090200765083\\
72.16	0.00162090112846489\\
72.17	0.00162090024841007\\
72.18	0.00162089936748427\\
72.19	0.00162089848568537\\
72.2	0.00162089760301128\\
72.21	0.00162089671945988\\
72.22	0.00162089583502907\\
72.23	0.00162089494971673\\
72.24	0.00162089406352077\\
72.25	0.00162089317643906\\
72.26	0.00162089228846951\\
72.27	0.00162089139961001\\
72.28	0.00162089050985845\\
72.29	0.00162088961921272\\
72.3	0.00162088872767072\\
72.31	0.00162088783523035\\
72.32	0.00162088694188951\\
72.33	0.00162088604764609\\
72.34	0.001620885152498\\
72.35	0.00162088425644314\\
72.36	0.00162088335947941\\
72.37	0.00162088246160474\\
72.38	0.00162088156281702\\
72.39	0.00162088066311417\\
72.4	0.00162087976249411\\
72.41	0.00162087886095475\\
72.42	0.00162087795849403\\
72.43	0.00162087705510987\\
72.44	0.0016208761508002\\
72.45	0.00162087524556296\\
72.46	0.00162087433939609\\
72.47	0.00162087343229753\\
72.48	0.00162087252426524\\
72.49	0.00162087161529716\\
72.5	0.00162087070539127\\
72.51	0.00162086979454552\\
72.52	0.00162086888275789\\
72.53	0.00162086797002636\\
72.54	0.00162086705634892\\
72.55	0.00162086614172356\\
72.56	0.00162086522614827\\
72.57	0.00162086430962106\\
72.58	0.00162086339213996\\
72.59	0.00162086247370298\\
72.6	0.00162086155430816\\
72.61	0.00162086063395353\\
72.62	0.00162085971263715\\
72.63	0.00162085879035707\\
72.64	0.00162085786711137\\
72.65	0.00162085694289811\\
72.66	0.0016208560177154\\
72.67	0.00162085509156132\\
72.68	0.001620854164434\\
72.69	0.00162085323633155\\
72.7	0.0016208523072521\\
72.71	0.00162085137719381\\
72.72	0.00162085044615483\\
72.73	0.00162084951413334\\
72.74	0.00162084858112751\\
72.75	0.00162084764713555\\
72.76	0.00162084671215567\\
72.77	0.0016208457761861\\
72.78	0.00162084483922508\\
72.79	0.00162084390127087\\
72.8	0.00162084296232174\\
72.81	0.00162084202237599\\
72.82	0.00162084108143191\\
72.83	0.00162084013948784\\
72.84	0.00162083919654211\\
72.85	0.00162083825259309\\
72.86	0.00162083730763916\\
72.87	0.0016208363616787\\
72.88	0.00162083541471014\\
72.89	0.00162083446673191\\
72.9	0.00162083351774247\\
72.91	0.00162083256774031\\
72.92	0.00162083161672391\\
72.93	0.00162083066469181\\
72.94	0.00162082971164254\\
72.95	0.00162082875757468\\
72.96	0.00162082780248681\\
72.97	0.00162082684637756\\
72.98	0.00162082588924556\\
72.99	0.00162082493108947\\
73	0.001620823971908\\
73.01	0.00162082301169986\\
73.02	0.00162082205046379\\
73.03	0.00162082108819857\\
73.04	0.001620820124903\\
73.05	0.00162081916057592\\
73.06	0.00162081819521617\\
73.07	0.00162081722882266\\
73.08	0.00162081626139431\\
73.09	0.00162081529293007\\
73.1	0.00162081432342892\\
73.11	0.00162081335288989\\
73.12	0.00162081238131203\\
73.13	0.00162081140869442\\
73.14	0.00162081043503619\\
73.15	0.0016208094603365\\
73.16	0.00162080848459454\\
73.17	0.00162080750780954\\
73.18	0.00162080652998078\\
73.19	0.00162080555110756\\
73.2	0.00162080457118923\\
73.21	0.00162080359022519\\
73.22	0.00162080260821485\\
73.23	0.0016208016251577\\
73.24	0.00162080064105326\\
73.25	0.00162079965590107\\
73.26	0.00162079866970075\\
73.27	0.00162079768245194\\
73.28	0.00162079669415434\\
73.29	0.00162079570480769\\
73.3	0.00162079471441178\\
73.31	0.00162079372296646\\
73.32	0.00162079273047161\\
73.33	0.00162079173692717\\
73.34	0.00162079074233314\\
73.35	0.00162078974668956\\
73.36	0.00162078874999652\\
73.37	0.0016207877522542\\
73.38	0.00162078675346278\\
73.39	0.00162078575362254\\
73.4	0.00162078475273381\\
73.41	0.00162078375079696\\
73.42	0.00162078274781244\\
73.43	0.00162078174378075\\
73.44	0.00162078073870245\\
73.45	0.00162077973257818\\
73.46	0.00162077872540863\\
73.47	0.00162077771719455\\
73.48	0.00162077670793677\\
73.49	0.00162077569763618\\
73.5	0.00162077468629374\\
73.51	0.00162077367391049\\
73.52	0.00162077266048752\\
73.53	0.00162077164602601\\
73.54	0.00162077063052721\\
73.55	0.00162076961399243\\
73.56	0.00162076859642309\\
73.57	0.00162076757782065\\
73.58	0.00162076655818666\\
73.59	0.00162076553752278\\
73.6	0.0016207645158307\\
73.61	0.00162076349311223\\
73.62	0.00162076246936926\\
73.63	0.00162076144460375\\
73.64	0.00162076041881775\\
73.65	0.00162075939201342\\
73.66	0.00162075836419297\\
73.67	0.00162075733535875\\
73.68	0.00162075630551316\\
73.69	0.00162075527465872\\
73.7	0.00162075424279803\\
73.71	0.0016207532099338\\
73.72	0.00162075217606883\\
73.73	0.00162075114120603\\
73.74	0.00162075010534839\\
73.75	0.00162074906849903\\
73.76	0.00162074803066116\\
73.77	0.0016207469918381\\
73.78	0.00162074595203327\\
73.79	0.00162074491125022\\
73.8	0.00162074386949261\\
73.81	0.00162074282676418\\
73.82	0.00162074178306882\\
73.83	0.00162074073841053\\
73.84	0.00162073969279342\\
73.85	0.00162073864622173\\
73.86	0.00162073759869981\\
73.87	0.00162073655023215\\
73.88	0.00162073550082335\\
73.89	0.00162073445047816\\
73.9	0.00162073339920143\\
73.91	0.00162073234699818\\
73.92	0.00162073129387352\\
73.93	0.00162073023983274\\
73.94	0.00162072918488123\\
73.95	0.00162072812902453\\
73.96	0.00162072707226836\\
73.97	0.00162072601461852\\
73.98	0.001620724956081\\
73.99	0.00162072389666193\\
74	0.00162072283636759\\
74.01	0.0016207217752044\\
74.02	0.00162072071317895\\
74.03	0.00162071965029799\\
74.04	0.00162071858656841\\
74.05	0.00162071752199728\\
74.06	0.00162071645659183\\
74.07	0.00162071539035945\\
74.08	0.0016207143233077\\
74.09	0.00162071325544432\\
74.1	0.00162071218677721\\
74.11	0.00162071111731446\\
74.12	0.00162071004706433\\
74.13	0.00162070897603526\\
74.14	0.00162070790423587\\
74.15	0.00162070683167499\\
74.16	0.00162070575836161\\
74.17	0.00162070468430492\\
74.18	0.00162070360951431\\
74.19	0.00162070253399936\\
74.2	0.00162070145776985\\
74.21	0.00162070038083577\\
74.22	0.0016206993032073\\
74.23	0.00162069822489485\\
74.24	0.00162069714590901\\
74.25	0.00162069606626061\\
74.26	0.00162069498596068\\
74.27	0.00162069390502048\\
74.28	0.00162069282345149\\
74.29	0.0016206917412654\\
74.3	0.00162069065847414\\
74.31	0.00162068957508988\\
74.32	0.00162068849112502\\
74.33	0.00162068740659217\\
74.34	0.00162068632150422\\
74.35	0.00162068523587428\\
74.36	0.00162068414971571\\
74.37	0.00162068306304213\\
74.38	0.0016206819758674\\
74.39	0.00162068088820564\\
74.4	0.00162067980007123\\
74.41	0.00162067871147882\\
74.42	0.00162067762244333\\
74.43	0.00162067653297992\\
74.44	0.00162067544310406\\
74.45	0.00162067435283147\\
74.46	0.00162067326217818\\
74.47	0.00162067217116048\\
74.48	0.00162067107979495\\
74.49	0.00162066998809847\\
74.5	0.00162066889608821\\
74.51	0.00162066780378164\\
74.52	0.00162066671119653\\
74.53	0.00162066561835096\\
74.54	0.00162066452526333\\
74.55	0.00162066343195233\\
74.56	0.00162066233843699\\
74.57	0.00162066124473666\\
74.58	0.00162066015087099\\
74.59	0.00162065905685999\\
74.6	0.001620657962724\\
74.61	0.00162065686848369\\
74.62	0.00162065577416007\\
74.63	0.0016206546797745\\
74.64	0.0016206535853487\\
74.65	0.00162065249090474\\
74.66	0.00162065139646503\\
74.67	0.00162065030205238\\
74.68	0.00162064920768994\\
74.69	0.00162064811340124\\
74.7	0.00162064701921019\\
74.71	0.00162064592514108\\
74.72	0.00162064483121859\\
74.73	0.00162064373746779\\
74.74	0.00162064264391413\\
74.75	0.00162064155058348\\
74.76	0.00162064045750211\\
74.77	0.0016206393646967\\
74.78	0.00162063827219433\\
74.79	0.00162063718002252\\
74.8	0.0016206360882092\\
74.81	0.00162063499678274\\
74.82	0.00162063390577194\\
74.83	0.00162063281520603\\
74.84	0.0016206317251147\\
74.85	0.00162063063552808\\
74.86	0.00162062954647675\\
74.87	0.00162062845799176\\
74.88	0.00162062737010462\\
74.89	0.00162062628284731\\
74.9	0.0016206251962523\\
74.91	0.00162062411035251\\
74.92	0.00162062302518138\\
74.93	0.00162062194077281\\
74.94	0.00162062085716124\\
74.95	0.00162061977438157\\
74.96	0.00162061869246922\\
74.97	0.00162061761146015\\
74.98	0.00162061653139082\\
74.99	0.0016206154522982\\
75	0.00162061437421983\\
75.01	0.00162061329719376\\
75.02	0.00162061222125859\\
75.03	0.00162061114645348\\
75.04	0.00162061007281812\\
75.05	0.00162060900039279\\
75.06	0.00162060792921833\\
75.07	0.00162060685933614\\
75.08	0.00162060579078822\\
75.09	0.00162060472361714\\
75.1	0.00162060365786609\\
75.11	0.00162060259357835\\
75.12	0.00162060153079714\\
75.13	0.00162060046956554\\
75.14	0.00162059940992648\\
75.15	0.00162059835192274\\
75.16	0.0016205972955969\\
75.17	0.00162059624099131\\
75.18	0.00162059518814813\\
75.19	0.0016205941371092\\
75.2	0.0016205930879161\\
75.21	0.00162059204061009\\
75.22	0.00162059099523208\\
75.23	0.00162058995182262\\
75.24	0.00162058891042184\\
75.25	0.00162058787106944\\
75.26	0.00162058683380467\\
75.27	0.00162058579866626\\
75.28	0.00162058476569243\\
75.29	0.00162058373492083\\
75.3	0.00162058270638852\\
75.31	0.00162058168013191\\
75.32	0.00162058065618675\\
75.33	0.00162057963458807\\
75.34	0.00162057861537018\\
75.35	0.00162057759856656\\
75.36	0.00162057658420991\\
75.37	0.00162057557233201\\
75.38	0.00162057456296376\\
75.39	0.00162057355613508\\
75.4	0.00162057255187491\\
75.41	0.0016205715502111\\
75.42	0.00162057055117044\\
75.43	0.00162056955477854\\
75.44	0.00162056856105983\\
75.45	0.00162056757003748\\
75.46	0.00162056658173334\\
75.47	0.00162056559616793\\
75.48	0.00162056461336032\\
75.49	0.00162056363332812\\
75.5	0.00162056265608741\\
75.51	0.00162056168165267\\
75.52	0.00162056071003765\\
75.53	0.001620559741256\\
75.54	0.00162055877532126\\
75.55	0.00162055781224686\\
75.56	0.00162055685204613\\
75.57	0.00162055589473226\\
75.58	0.00162055494031835\\
75.59	0.00162055398881735\\
75.6	0.0016205530402421\\
75.61	0.00162055209460529\\
75.62	0.00162055115191951\\
75.63	0.00162055021219719\\
75.64	0.00162054927545061\\
75.65	0.00162054834169191\\
75.66	0.0016205474109331\\
75.67	0.00162054648318602\\
75.68	0.00162054555846235\\
75.69	0.00162054463677361\\
75.7	0.00162054371813116\\
75.71	0.00162054280254619\\
75.72	0.00162054189002972\\
75.73	0.00162054098059258\\
75.74	0.00162054007424543\\
75.75	0.00162053917099872\\
75.76	0.00162053827086275\\
75.77	0.00162053737384759\\
75.78	0.00162053647996312\\
75.79	0.00162053558921901\\
75.8	0.00162053470162474\\
75.81	0.00162053381718955\\
75.82	0.00162053293592248\\
75.83	0.00162053205783235\\
75.84	0.00162053118292772\\
75.85	0.00162053031121696\\
75.86	0.00162052944270816\\
75.87	0.0016205285774092\\
75.88	0.00162052771532769\\
75.89	0.00162052685647101\\
75.9	0.00162052600084625\\
75.91	0.00162052514846026\\
75.92	0.00162052429931961\\
75.93	0.0016205234534306\\
75.94	0.00162052261079925\\
75.95	0.00162052177143128\\
75.96	0.00162052093533214\\
75.97	0.00162052010250698\\
75.98	0.00162051927296063\\
75.99	0.00162051844669762\\
76	0.00162051762372219\\
76.01	0.00162051680403822\\
76.02	0.00162051598764929\\
76.03	0.00162051517455864\\
76.04	0.00162051436476918\\
76.05	0.00162051355828346\\
76.06	0.0016205127551037\\
76.07	0.00162051195523174\\
76.08	0.00162051115866908\\
76.09	0.00162051036541685\\
76.1	0.00162050957547578\\
76.11	0.00162050878884625\\
76.12	0.00162050800552823\\
76.13	0.00162050722552129\\
76.14	0.00162050644882464\\
76.15	0.00162050567543702\\
76.16	0.00162050490535681\\
76.17	0.00162050413858193\\
76.18	0.0016205033751099\\
76.19	0.00162050261493778\\
76.2	0.00162050185806221\\
76.21	0.00162050110447936\\
76.22	0.00162050035418496\\
76.23	0.00162049960717426\\
76.24	0.00162049886344205\\
76.25	0.00162049812298265\\
76.26	0.00162049738578987\\
76.27	0.00162049665185704\\
76.28	0.00162049592117699\\
76.29	0.00162049519374205\\
76.3	0.00162049446954402\\
76.31	0.00162049374857418\\
76.32	0.00162049303082328\\
76.33	0.00162049231628154\\
76.34	0.00162049160493862\\
76.35	0.00162049089678364\\
76.36	0.00162049019180514\\
76.37	0.0016204894899911\\
76.38	0.00162048879132893\\
76.39	0.00162048809580543\\
76.4	0.00162048740340684\\
76.41	0.00162048671411876\\
76.42	0.0016204860279262\\
76.43	0.00162048534481356\\
76.44	0.0016204846647646\\
76.45	0.00162048398776244\\
76.46	0.00162048331378956\\
76.47	0.0016204826428278\\
76.48	0.00162048197485832\\
76.49	0.00162048130986162\\
76.5	0.00162048064781754\\
76.51	0.0016204799887052\\
76.52	0.00162047933250304\\
76.53	0.00162047867918881\\
76.54	0.00162047802873953\\
76.55	0.00162047738113151\\
76.56	0.00162047673634033\\
76.57	0.00162047609434081\\
76.58	0.00162047545510706\\
76.59	0.00162047481861241\\
76.6	0.00162047418482942\\
76.61	0.0016204735537299\\
76.62	0.00162047292528487\\
76.63	0.00162047229946454\\
76.64	0.00162047167623834\\
76.65	0.00162047105557488\\
76.66	0.00162047043744197\\
76.67	0.00162046982180657\\
76.68	0.00162046920863481\\
76.69	0.001620468597892\\
76.7	0.00162046798954256\\
76.71	0.00162046738355006\\
76.72	0.00162046677987722\\
76.73	0.00162046617848586\\
76.74	0.0016204655793369\\
76.75	0.00162046498239039\\
76.76	0.00162046438760545\\
76.77	0.0016204637949403\\
76.78	0.00162046320435222\\
76.79	0.00162046261579758\\
76.8	0.00162046202923178\\
76.81	0.00162046144460929\\
76.82	0.00162046086188361\\
76.83	0.00162046028100728\\
76.84	0.00162045970193185\\
76.85	0.0016204591246079\\
76.86	0.00162045854898502\\
76.87	0.00162045797501177\\
76.88	0.00162045740263573\\
76.89	0.00162045683180344\\
76.9	0.00162045626246044\\
76.91	0.00162045569455119\\
76.92	0.00162045512801916\\
76.93	0.00162045456280673\\
76.94	0.00162045399885523\\
76.95	0.00162045343610493\\
76.96	0.00162045287449503\\
76.97	0.00162045231396363\\
76.98	0.00162045175444776\\
76.99	0.00162045119588335\\
77	0.0016204506382052\\
77.01	0.00162045008134704\\
77.02	0.00162044952524145\\
77.03	0.00162044896981989\\
77.04	0.00162044841501272\\
77.05	0.00162044786074911\\
77.06	0.00162044730695714\\
77.07	0.00162044675356369\\
77.08	0.00162044620049451\\
77.09	0.0016204456476742\\
77.1	0.00162044509502616\\
77.11	0.00162044454247264\\
77.12	0.00162044398993471\\
77.13	0.00162044343733225\\
77.14	0.00162044288458395\\
77.15	0.00162044233160732\\
77.16	0.00162044177831866\\
77.17	0.00162044122463308\\
77.18	0.00162044067046449\\
77.19	0.00162044011572557\\
77.2	0.00162043956032781\\
77.21	0.00162043900418148\\
77.22	0.00162043844719564\\
77.23	0.00162043788927813\\
77.24	0.00162043733033555\\
77.25	0.00162043677027331\\
77.26	0.00162043620899558\\
77.27	0.00162043564640531\\
77.28	0.00162043508240422\\
77.29	0.00162043451689282\\
77.3	0.00162043394977039\\
77.31	0.00162043338093497\\
77.32	0.00162043281028341\\
77.33	0.00162043223771132\\
77.34	0.00162043166311309\\
77.35	0.0016204310863819\\
77.36	0.00162043050740974\\
77.37	0.00162042992608735\\
77.38	0.00162042934230429\\
77.39	0.00162042875594892\\
77.4	0.0016204281669084\\
77.41	0.0016204275750687\\
77.42	0.00162042698031461\\
77.43	0.00162042638252976\\
77.44	0.00162042578159657\\
77.45	0.00162042517739634\\
77.46	0.00162042456980921\\
77.47	0.00162042395871416\\
77.48	0.00162042334398905\\
77.49	0.00162042272551063\\
77.5	0.00162042210315453\\
77.51	0.00162042147679527\\
77.52	0.00162042084630632\\
77.53	0.00162042021156004\\
77.54	0.00162041957242776\\
77.55	0.00162041892877977\\
77.56	0.00162041828048533\\
77.57	0.0016204176274127\\
77.58	0.00162041696942913\\
77.59	0.00162041630640094\\
77.6	0.00162041563819346\\
77.61	0.00162041496467113\\
77.62	0.00162041428569745\\
77.63	0.00162041360113507\\
77.64	0.00162041291084576\\
77.65	0.00162041221469046\\
77.66	0.00162041151252931\\
77.67	0.00162041080422166\\
77.68	0.00162041008962612\\
77.69	0.00162040936860058\\
77.7	0.00162040864100223\\
77.71	0.00162040790668761\\
77.72	0.00162040716551264\\
77.73	0.00162040641733264\\
77.74	0.00162040566200238\\
77.75	0.00162040489937613\\
77.76	0.00162040412930768\\
77.77	0.00162040335165036\\
77.78	0.00162040256625714\\
77.79	0.00162040177298063\\
77.8	0.00162040097167313\\
77.81	0.0016204001621867\\
77.82	0.00162039934437317\\
77.83	0.00162039851808422\\
77.84	0.00162039768317144\\
77.85	0.00162039683948636\\
77.86	0.00162039598688049\\
77.87	0.00162039512520544\\
77.88	0.00162039425431292\\
77.89	0.00162039337405481\\
77.9	0.00162039248428327\\
77.91	0.00162039158485075\\
77.92	0.00162039067561007\\
77.93	0.00162038975641452\\
77.94	0.00162038882711789\\
77.95	0.00162038788757458\\
77.96	0.00162038693763964\\
77.97	0.00162038597716891\\
77.98	0.00162038500601902\\
77.99	0.00162038402404754\\
78	0.00162038303111302\\
78.01	0.00162038202707512\\
78.02	0.00162038101179467\\
78.03	0.00162037998513379\\
78.04	0.00162037894695595\\
78.05	0.00162037789712612\\
78.06	0.00162037683551082\\
78.07	0.00162037576197795\\
78.08	0.00162037467639395\\
78.09	0.00162037357862373\\
78.1	0.00162037246853072\\
78.11	0.0016203713459768\\
78.12	0.00162037021082233\\
78.13	0.00162036906292607\\
78.14	0.00162036790214523\\
78.15	0.00162036672833541\\
78.16	0.0016203655413506\\
78.17	0.00162036434104316\\
78.18	0.00162036312726377\\
78.19	0.00162036189986149\\
78.2	0.00162036065868363\\
78.21	0.00162035940357585\\
78.22	0.00162035813438205\\
78.23	0.00162035685094438\\
78.24	0.00162035555310324\\
78.25	0.00162035424069724\\
78.26	0.00162035291356318\\
78.27	0.00162035157153604\\
78.28	0.00162035021444896\\
78.29	0.00162034884213319\\
78.3	0.00162034745441811\\
78.31	0.0016203460511312\\
78.32	0.00162034463209798\\
78.33	0.00162034319714205\\
78.34	0.00162034174608503\\
78.35	0.00162034027874652\\
78.36	0.00162033879494413\\
78.37	0.00162033729449342\\
78.38	0.00162033577720788\\
78.39	0.00162033424289894\\
78.4	0.00162033269137588\\
78.41	0.00162033112244589\\
78.42	0.00162032953591397\\
78.43	0.00162032793158295\\
78.44	0.00162032630925346\\
78.45	0.0016203246687239\\
78.46	0.00162032300979041\\
78.47	0.00162032133224686\\
78.48	0.00162031963588479\\
78.49	0.00162031792049345\\
78.5	0.00162031618585968\\
78.51	0.00162031443176799\\
78.52	0.00162031265800045\\
78.53	0.0016203108643367\\
78.54	0.00162030905055391\\
78.55	0.00162030721642676\\
78.56	0.00162030536172742\\
78.57	0.00162030348622551\\
78.58	0.00162030158968807\\
78.59	0.00162029967187953\\
78.6	0.0016202977325617\\
78.61	0.0016202957714937\\
78.62	0.00162029378843201\\
78.63	0.00162029178313032\\
78.64	0.00162028975533962\\
78.65	0.00162028770480808\\
78.66	0.00162028563128109\\
78.67	0.00162028353450116\\
78.68	0.00162028141420793\\
78.69	0.00162027927013815\\
78.7	0.00162027710202561\\
78.71	0.00162027490960111\\
78.72	0.00162027269259248\\
78.73	0.00162027045072446\\
78.74	0.00162026818371876\\
78.75	0.00162026589129394\\
78.76	0.00162026357316544\\
78.77	0.0016202612290455\\
78.78	0.00162025885864317\\
78.79	0.00162025646166422\\
78.8	0.00162025403781115\\
78.81	0.00162025158678313\\
78.82	0.00162024910827595\\
78.83	0.00162024660198203\\
78.84	0.00162024406759034\\
78.85	0.00162024150478636\\
78.86	0.00162023891325208\\
78.87	0.00162023629266592\\
78.88	0.0016202336427027\\
78.89	0.00162023096303363\\
78.9	0.00162022825332622\\
78.91	0.00162022551324429\\
78.92	0.00162022274244787\\
78.93	0.00162021994059322\\
78.94	0.00162021710733276\\
78.95	0.00162021424231499\\
78.96	0.00162021134518453\\
78.97	0.001620208415582\\
78.98	0.001620205453144\\
78.99	0.0016202024575031\\
79	0.00162019942828773\\
79.01	0.00162019636512219\\
79.02	0.00162019326762657\\
79.03	0.00162019013541673\\
79.04	0.00162018696810423\\
79.05	0.00162018376529629\\
79.06	0.00162018052659573\\
79.07	0.00162017725160096\\
79.08	0.00162017393990587\\
79.09	0.00162017059109986\\
79.1	0.00162016720476769\\
79.11	0.00162016378048951\\
79.12	0.00162016031784078\\
79.13	0.00162015681639222\\
79.14	0.00162015327570973\\
79.15	0.0016201496953544\\
79.16	0.00162014607488238\\
79.17	0.00162014241384489\\
79.18	0.00162013871178811\\
79.19	0.00162013496825319\\
79.2	0.00162013118277611\\
79.21	0.0016201273548877\\
79.22	0.00162012348411353\\
79.23	0.00162011956997389\\
79.24	0.0016201156119837\\
79.25	0.00162011160965246\\
79.26	0.00162010756248421\\
79.27	0.00162010346997742\\
79.28	0.00162009933162498\\
79.29	0.00162009514691412\\
79.3	0.00162009091532633\\
79.31	0.00162008663633733\\
79.32	0.00162008230941696\\
79.33	0.00162007793402915\\
79.34	0.00162007350963185\\
79.35	0.00162006903567695\\
79.36	0.00162006451161023\\
79.37	0.00162005993687126\\
79.38	0.00162005531089338\\
79.39	0.00162005063310358\\
79.4	0.00162004590292246\\
79.41	0.00162004111976415\\
79.42	0.00162003628303625\\
79.43	0.00162003139213973\\
79.44	0.00162002644646887\\
79.45	0.0016200214454112\\
79.46	0.0016200163883474\\
79.47	0.00162001127465126\\
79.48	0.00162000610368955\\
79.49	0.00162000087482198\\
79.5	0.00161999558740111\\
79.51	0.00161999024077228\\
79.52	0.00161998483427352\\
79.53	0.00161997936723547\\
79.54	0.0016199738389813\\
79.55	0.0016199682488266\\
79.56	0.00161996259607935\\
79.57	0.00161995688003981\\
79.58	0.00161995110000039\\
79.59	0.00161994525524565\\
79.6	0.00161993934505212\\
79.61	0.00161993336868829\\
79.62	0.00161992732541446\\
79.63	0.0016199212144827\\
79.64	0.0016199150351367\\
79.65	0.00161990878661173\\
79.66	0.00161990246813453\\
79.67	0.00161989607892318\\
79.68	0.00161988961818707\\
79.69	0.00161988308512675\\
79.7	0.00161987647893384\\
79.71	0.00161986979879094\\
79.72	0.00161986304387155\\
79.73	0.00161985621333992\\
79.74	0.00161984930635099\\
79.75	0.00161984232205025\\
79.76	0.00161983525957369\\
79.77	0.00161982811804762\\
79.78	0.00161982089658862\\
79.79	0.00161981359430341\\
79.8	0.00161980621028875\\
79.81	0.0016197987436313\\
79.82	0.00161979119340755\\
79.83	0.00161978355868368\\
79.84	0.00161977583851544\\
79.85	0.00161976803194806\\
79.86	0.0016197601380161\\
79.87	0.00161975215574337\\
79.88	0.00161974408414277\\
79.89	0.0016197359222162\\
79.9	0.00161972766895442\\
79.91	0.00161971932333691\\
79.92	0.00161971088433181\\
79.93	0.00161970235089571\\
79.94	0.00161969372197357\\
79.95	0.00161968499649858\\
79.96	0.00161967617339204\\
79.97	0.00161966725156319\\
79.98	0.00161965822990915\\
79.99	0.00161964910731468\\
80	0.00161963988265214\\
80.01	0.00161963055478129\\
};
\addplot [color=blue,solid]
  table[row sep=crcr]{%
80.01	0.00161963055478129\\
80.02	0.00161962112254919\\
80.03	0.00161961158479003\\
80.04	0.00161960194032498\\
80.05	0.00161959218796208\\
80.06	0.00161958232649608\\
80.07	0.00161957235470828\\
80.08	0.00161956227136638\\
80.09	0.00161955207522434\\
80.1	0.00161954176502222\\
80.11	0.00161953133948605\\
80.12	0.00161952079732761\\
80.13	0.00161951013724435\\
80.14	0.00161949935791918\\
80.15	0.00161948845802031\\
80.16	0.00161947743620112\\
80.17	0.00161946629109996\\
80.18	0.00161945502133999\\
80.19	0.00161944362552902\\
80.2	0.00161943210225934\\
80.21	0.00161942045010755\\
80.22	0.00161940866763435\\
80.23	0.00161939675338442\\
80.24	0.00161938470588618\\
80.25	0.00161937252365169\\
80.26	0.00161936020517636\\
80.27	0.00161934774893886\\
80.28	0.00161933515340089\\
80.29	0.001619322417007\\
80.3	0.00161930953818439\\
80.31	0.00161929651534273\\
80.32	0.00161928334687395\\
80.33	0.00161927003115207\\
80.34	0.00161925656653295\\
80.35	0.00161924295135415\\
80.36	0.00161922918393468\\
80.37	0.00161921526257482\\
80.38	0.00161920118555589\\
80.39	0.00161918695114004\\
80.4	0.00161917255757008\\
80.41	0.0016191580030692\\
80.42	0.00161914328584079\\
80.43	0.00161912840406822\\
80.44	0.0016191133559146\\
80.45	0.00161909813952256\\
80.46	0.00161908275301405\\
80.47	0.00161906719449006\\
80.48	0.00161905146203041\\
80.49	0.00161903555369354\\
80.5	0.00161901946751622\\
80.51	0.00161900320151337\\
80.52	0.00161898675367774\\
80.53	0.00161897012197976\\
80.54	0.00161895330436719\\
80.55	0.00161893629876494\\
80.56	0.0016189191030748\\
80.57	0.00161890171517514\\
80.58	0.00161888413292073\\
80.59	0.0016188663541424\\
80.6	0.00161884837664681\\
80.61	0.00161883019821618\\
80.62	0.00161881181660802\\
80.63	0.00161879322955482\\
80.64	0.00161877443476384\\
80.65	0.00161875542991676\\
80.66	0.00161873621266943\\
80.67	0.00161871678065159\\
80.68	0.00161869713146656\\
80.69	0.00161867726269093\\
80.7	0.00161865717187433\\
80.71	0.00161863685653905\\
80.72	0.00161861631417977\\
80.73	0.00161859554226327\\
80.74	0.0016185745382281\\
80.75	0.00161855329948427\\
80.76	0.00161853182341291\\
80.77	0.00161851010736599\\
80.78	0.00161848814866598\\
80.79	0.0016184659446055\\
80.8	0.00161844349244701\\
80.81	0.00161842078942248\\
80.82	0.00161839783273301\\
80.83	0.00161837461954856\\
80.84	0.00161835114700751\\
80.85	0.00161832741221641\\
80.86	0.00161830341224954\\
80.87	0.00161827914414858\\
80.88	0.00161825460492228\\
80.89	0.00161822979154605\\
80.9	0.00161820470096159\\
80.91	0.00161817933007656\\
80.92	0.00161815367576415\\
80.93	0.00161812773486273\\
80.94	0.00161810150417543\\
80.95	0.00161807498046979\\
80.96	0.00161804816047733\\
80.97	0.00161802104089316\\
80.98	0.00161799361837559\\
80.99	0.00161796588954568\\
81	0.00161793785098688\\
81.01	0.00161790949924456\\
81.02	0.00161788083082563\\
81.03	0.00161785184219807\\
81.04	0.00161782252979054\\
81.05	0.00161779288999192\\
81.06	0.00161776291915084\\
81.07	0.0016177326135753\\
81.08	0.00161770196953216\\
81.09	0.00161767098324668\\
81.1	0.00161763965090213\\
81.11	0.00161760796863922\\
81.12	0.0016175759325557\\
81.13	0.00161754353870587\\
81.14	0.00161751078310006\\
81.15	0.00161747766170419\\
81.16	0.00161744417043924\\
81.17	0.00161741030518078\\
81.18	0.00161737606175842\\
81.19	0.00161734143595535\\
81.2	0.0016173064235078\\
81.21	0.00161727102010449\\
81.22	0.00161723522138617\\
81.23	0.00161719902294502\\
81.24	0.00161716242032414\\
81.25	0.00161712540901699\\
81.26	0.00161708798446687\\
81.27	0.00161705014206632\\
81.28	0.00161701187715659\\
81.29	0.00161697318502704\\
81.3	0.00161693406091457\\
81.31	0.00161689450000308\\
81.32	0.00161685449742279\\
81.33	0.00161681404824975\\
81.34	0.00161677314750515\\
81.35	0.00161673179015475\\
81.36	0.00161668997110827\\
81.37	0.00161664768521873\\
81.38	0.00161660492728186\\
81.39	0.00161656169203546\\
81.4	0.0016165179741587\\
81.41	0.00161647376827155\\
81.42	0.00161642906893406\\
81.43	0.00161638387064572\\
81.44	0.00161633816784477\\
81.45	0.00161629195490756\\
81.46	0.00161624522614779\\
81.47	0.00161619797581589\\
81.48	0.00161615019809826\\
81.49	0.00161610188711657\\
81.5	0.00161605303692708\\
81.51	0.00161600364151985\\
81.52	0.00161595369481806\\
81.53	0.00161590319067721\\
81.54	0.00161585212288443\\
81.55	0.00161580048515766\\
81.56	0.00161574827114493\\
81.57	0.00161569547442355\\
81.58	0.00161564208849933\\
81.59	0.00161558810680579\\
81.6	0.00161553352270337\\
81.61	0.00161547832947857\\
81.62	0.00161542252034319\\
81.63	0.00161536608843345\\
81.64	0.00161530902680917\\
81.65	0.00161525132845293\\
81.66	0.00161519298626919\\
81.67	0.00161513399308343\\
81.68	0.00161507434164127\\
81.69	0.00161501402460758\\
81.7	0.0016149530345656\\
81.71	0.001614891364016\\
81.72	0.00161482900537599\\
81.73	0.00161476595097839\\
81.74	0.00161470219307064\\
81.75	0.00161463772381395\\
81.76	0.00161457253528225\\
81.77	0.00161450661946125\\
81.78	0.00161443996824748\\
81.79	0.00161437257344726\\
81.8	0.00161430442677574\\
81.81	0.00161423551985584\\
81.82	0.00161416584421726\\
81.83	0.00161409539129542\\
81.84	0.00161402415243043\\
81.85	0.00161395211886602\\
81.86	0.00161387928174847\\
81.87	0.00161380563212552\\
81.88	0.00161373116094528\\
81.89	0.00161365585905512\\
81.9	0.00161357971720057\\
81.91	0.00161350272602414\\
81.92	0.00161342487606422\\
81.93	0.00161334615775389\\
81.94	0.00161326656141979\\
81.95	0.00161318607728089\\
81.96	0.0016131046954473\\
81.97	0.0016130224059191\\
81.98	0.00161293919858507\\
81.99	0.00161285506322146\\
82	0.00161276998949077\\
82.01	0.00161268396694046\\
82.02	0.00161259698500165\\
82.03	0.00161250903298786\\
82.04	0.00161242010009371\\
82.05	0.00161233017539356\\
82.06	0.0016122392478402\\
82.07	0.00161214730626347\\
82.08	0.00161205433936892\\
82.09	0.00161196033573645\\
82.1	0.00161186528381883\\
82.11	0.00161176917194037\\
82.12	0.00161167198829545\\
82.13	0.00161157372094708\\
82.14	0.00161147435782545\\
82.15	0.00161137388672644\\
82.16	0.00161127229531012\\
82.17	0.00161116957109926\\
82.18	0.0016110657014778\\
82.19	0.00161096067368929\\
82.2	0.00161085447483534\\
82.21	0.00161074709187402\\
82.22	0.0016106385116183\\
82.23	0.00161052872073442\\
82.24	0.00161041770574025\\
82.25	0.00161030545300363\\
82.26	0.00161019194874073\\
82.27	0.00161007717901435\\
82.28	0.00160996112973221\\
82.29	0.00160984378664521\\
82.3	0.00160972513534573\\
82.31	0.00160960516126583\\
82.32	0.00160948384967545\\
82.33	0.00160936118568067\\
82.34	0.00160923715422182\\
82.35	0.00160911174007168\\
82.36	0.00160898492783361\\
82.37	0.00160885790019364\\
82.38	0.00160873067942973\\
82.39	0.00160860326421506\\
82.4	0.00160847565320959\\
82.41	0.00160834784505985\\
82.42	0.00160821983839891\\
82.43	0.00160809163184611\\
82.44	0.00160796322400705\\
82.45	0.00160783461347336\\
82.46	0.00160770579882258\\
82.47	0.00160757677861803\\
82.48	0.00160744755140862\\
82.49	0.00160731811572878\\
82.5	0.00160718847009823\\
82.51	0.00160705861302187\\
82.52	0.00160692854298961\\
82.53	0.00160679825847624\\
82.54	0.00160666775794125\\
82.55	0.00160653703982869\\
82.56	0.00160640610256697\\
82.57	0.00160627494456878\\
82.58	0.00160614356423085\\
82.59	0.0016060119599338\\
82.6	0.00160588013004204\\
82.61	0.0016057480729035\\
82.62	0.00160561578684954\\
82.63	0.00160548327019476\\
82.64	0.0016053505212368\\
82.65	0.0016052175382562\\
82.66	0.0016050843195162\\
82.67	0.0016049508632626\\
82.68	0.00160481716772352\\
82.69	0.00160468323110927\\
82.7	0.00160454905161215\\
82.71	0.00160441462740627\\
82.72	0.00160427995664735\\
82.73	0.00160414503747254\\
82.74	0.00160400986800024\\
82.75	0.00160387444632989\\
82.76	0.00160373877054177\\
82.77	0.00160360283869686\\
82.78	0.00160346664883655\\
82.79	0.00160333019898252\\
82.8	0.00160319348713649\\
82.81	0.00160305651128004\\
82.82	0.00160291926937441\\
82.83	0.00160278175936026\\
82.84	0.00160264397915748\\
82.85	0.00160250592666499\\
82.86	0.00160236759976049\\
82.87	0.00160222899630028\\
82.88	0.00160209011411904\\
82.89	0.00160195095102956\\
82.9	0.00160181150482258\\
82.91	0.00160167177326652\\
82.92	0.00160153175410729\\
82.93	0.00160139144506802\\
82.94	0.00160125084384885\\
82.95	0.0016011099481267\\
82.96	0.00160096875555503\\
82.97	0.00160082726376357\\
82.98	0.00160068547035815\\
82.99	0.00160054337292037\\
83	0.00160040096900742\\
83.01	0.0016002582561518\\
83.02	0.00160011523186107\\
83.03	0.0015999718936176\\
83.04	0.00159982823887833\\
83.05	0.00159968426507449\\
83.06	0.00159953996961135\\
83.07	0.00159939534986793\\
83.08	0.00159925040319678\\
83.09	0.00159910512692368\\
83.1	0.00159895951834738\\
83.11	0.0015988135747393\\
83.12	0.00159866729334329\\
83.13	0.00159852067137533\\
83.14	0.00159837370602326\\
83.15	0.00159822639444646\\
83.16	0.00159807873377563\\
83.17	0.00159793072111241\\
83.18	0.00159778235352915\\
83.19	0.00159763362806862\\
83.2	0.00159748454174365\\
83.21	0.00159733509153689\\
83.22	0.00159718527440047\\
83.23	0.00159703508725568\\
83.24	0.00159688452699272\\
83.25	0.00159673359047031\\
83.26	0.00159658227451541\\
83.27	0.00159643057592292\\
83.28	0.00159627849145531\\
83.29	0.00159612601784231\\
83.3	0.0015959731517806\\
83.31	0.00159581988993347\\
83.32	0.00159566622893043\\
83.33	0.00159551216536697\\
83.34	0.00159535769580413\\
83.35	0.00159520281676818\\
83.36	0.0015950475247503\\
83.37	0.00159489181620619\\
83.38	0.0015947356875557\\
83.39	0.00159457913518255\\
83.4	0.00159442215543385\\
83.41	0.00159426474461983\\
83.42	0.00159410689901343\\
83.43	0.00159394861484992\\
83.44	0.00159378988832653\\
83.45	0.00159363071560207\\
83.46	0.00159347109279656\\
83.47	0.0015933110159908\\
83.48	0.00159315048122602\\
83.49	0.00159298948450345\\
83.5	0.00159282802178394\\
83.51	0.00159266608898756\\
83.52	0.00159250368199318\\
83.53	0.00159234079663804\\
83.54	0.00159217742871738\\
83.55	0.00159201357398398\\
83.56	0.00159184922814775\\
83.57	0.00159168438687532\\
83.58	0.00159151904578956\\
83.59	0.00159135320046919\\
83.6	0.00159118684644833\\
83.61	0.00159101997921604\\
83.62	0.00159085259421588\\
83.63	0.00159068468684544\\
83.64	0.00159051625245592\\
83.65	0.00159034728635165\\
83.66	0.0015901777837896\\
83.67	0.00159000773997893\\
83.68	0.00158983715008053\\
83.69	0.00158966600920651\\
83.7	0.00158949431241974\\
83.71	0.00158932205473334\\
83.72	0.0015891492311102\\
83.73	0.00158897583646248\\
83.74	0.0015888018656511\\
83.75	0.00158862731348525\\
83.76	0.00158845217472184\\
83.77	0.00158827644406504\\
83.78	0.0015881001161657\\
83.79	0.00158792318562086\\
83.8	0.0015877456469732\\
83.81	0.0015875674947105\\
83.82	0.00158738872326513\\
83.83	0.00158720932701344\\
83.84	0.00158702930027529\\
83.85	0.00158684863731341\\
83.86	0.00158666733233289\\
83.87	0.00158648537948061\\
83.88	0.00158630277284461\\
83.89	0.00158611950645361\\
83.9	0.00158593557427632\\
83.91	0.00158575097022092\\
83.92	0.00158556568813446\\
83.93	0.0015853797218022\\
83.94	0.00158519306494708\\
83.95	0.00158500571122906\\
83.96	0.00158481765424452\\
83.97	0.00158462888752563\\
83.98	0.00158443940453972\\
83.99	0.00158424919868864\\
84	0.00158405826330814\\
84.01	0.00158386659166719\\
84.02	0.00158367417696736\\
84.03	0.00158348101234214\\
84.04	0.00158328709085627\\
84.05	0.00158309240550509\\
84.06	0.00158289694921385\\
84.07	0.00158270071483702\\
84.08	0.00158250369515759\\
84.09	0.00158230588288642\\
84.1	0.00158210727066149\\
84.11	0.00158190785104719\\
84.12	0.00158170761653366\\
84.13	0.00158150655953598\\
84.14	0.00158130467239353\\
84.15	0.0015811019473692\\
84.16	0.00158089837664865\\
84.17	0.00158069395233957\\
84.18	0.00158048866647095\\
84.19	0.00158028251099227\\
84.2	0.00158007547777276\\
84.21	0.0015798675586006\\
84.22	0.00157965874518215\\
84.23	0.00157944902914118\\
84.24	0.00157923840201804\\
84.25	0.00157902685526883\\
84.26	0.00157881438026466\\
84.27	0.00157860096829078\\
84.28	0.00157838661054573\\
84.29	0.00157817129814058\\
84.3	0.00157795502209801\\
84.31	0.0015777377733515\\
84.32	0.00157751954274446\\
84.33	0.00157730032102936\\
84.34	0.00157708009886688\\
84.35	0.00157685886682499\\
84.36	0.00157663661537808\\
84.37	0.00157641333490609\\
84.38	0.00157618901569355\\
84.39	0.00157596364792869\\
84.4	0.00157573722170253\\
84.41	0.00157550972700795\\
84.42	0.00157528115373873\\
84.43	0.00157505149168861\\
84.44	0.00157482073055035\\
84.45	0.00157458885991475\\
84.46	0.00157435586926967\\
84.47	0.0015741217479991\\
84.48	0.00157388648538208\\
84.49	0.0015736500705918\\
84.5	0.00157341249269452\\
84.51	0.00157317374064859\\
84.52	0.00157293380330341\\
84.53	0.00157269266939839\\
84.54	0.00157245032756195\\
84.55	0.0015722067663104\\
84.56	0.00157196197404694\\
84.57	0.00157171593906054\\
84.58	0.0015714686495249\\
84.59	0.00157122009349733\\
84.6	0.00157097025891767\\
84.61	0.00157071913360717\\
84.62	0.00157046670526738\\
84.63	0.00157021296147902\\
84.64	0.00156995788970082\\
84.65	0.00156970147726842\\
84.66	0.00156944371139317\\
84.67	0.00156918457916097\\
84.68	0.00156892406753111\\
84.69	0.00156866216333506\\
84.7	0.0015683988532753\\
84.71	0.00156813412392407\\
84.72	0.00156786796172218\\
84.73	0.00156760035297779\\
84.74	0.00156733128386515\\
84.75	0.00156706074042335\\
84.76	0.00156678870855506\\
84.77	0.00156651517402527\\
84.78	0.00156624012246\\
84.79	0.001565963539345\\
84.8	0.00156568541002446\\
84.81	0.00156540571969969\\
84.82	0.00156512445342777\\
84.83	0.00156484159612024\\
84.84	0.00156455713254176\\
84.85	0.00156427104730871\\
84.86	0.00156398332488785\\
84.87	0.00156369394959492\\
84.88	0.00156340290559328\\
84.89	0.00156311017689243\\
84.9	0.00156281574734667\\
84.91	0.00156251960065364\\
84.92	0.00156222258065645\\
84.93	0.00156192480073423\\
84.94	0.00156162625401971\\
84.95	0.00156132693357291\\
84.96	0.00156102683238039\\
84.97	0.00156072594335442\\
84.98	0.00156042425933221\\
84.99	0.00156012177307513\\
85	0.00155981847726787\\
85.01	0.00155951436451762\\
85.02	0.00155920942735328\\
85.03	0.00155890365822458\\
85.04	0.00155859704950127\\
85.05	0.00155828959347225\\
85.06	0.00155798128234473\\
85.07	0.00155767210824333\\
85.08	0.00155736206320919\\
85.09	0.00155705113919916\\
85.1	0.00155673932808479\\
85.11	0.00155642662165152\\
85.12	0.00155611301159769\\
85.13	0.00155579848953367\\
85.14	0.00155548304698087\\
85.15	0.00155516667537085\\
85.16	0.00155484936604432\\
85.17	0.00155453111025018\\
85.18	0.00155421189914457\\
85.19	0.00155389172378985\\
85.2	0.00155357057515362\\
85.21	0.00155324844410771\\
85.22	0.00155292532142718\\
85.23	0.00155260119778924\\
85.24	0.00155227606377226\\
85.25	0.00155194990985471\\
85.26	0.00155162272641409\\
85.27	0.00155129450372585\\
85.28	0.00155096523196232\\
85.29	0.0015506349011916\\
85.3	0.00155030350137647\\
85.31	0.00154997102237329\\
85.32	0.00154963745393081\\
85.33	0.00154930278568907\\
85.34	0.00154896700717827\\
85.35	0.00154863010781753\\
85.36	0.00154829207691379\\
85.37	0.00154795290366059\\
85.38	0.00154761257713684\\
85.39	0.00154727108630564\\
85.4	0.00154692842001306\\
85.41	0.00154658456698686\\
85.42	0.00154623951583526\\
85.43	0.00154589325504568\\
85.44	0.00154554577298346\\
85.45	0.00154519705789055\\
85.46	0.00154484709788421\\
85.47	0.0015444958809557\\
85.48	0.00154414339496893\\
85.49	0.00154378962765914\\
85.5	0.00154343456663151\\
85.51	0.00154307819935979\\
85.52	0.00154272051318493\\
85.53	0.00154236149531364\\
85.54	0.001542001132817\\
85.55	0.00154163941262904\\
85.56	0.00154127632154523\\
85.57	0.00154091184622108\\
85.58	0.00154054597317063\\
85.59	0.00154017868876497\\
85.6	0.0015398099792307\\
85.61	0.00153943983064846\\
85.62	0.00153906822895133\\
85.63	0.00153869515992331\\
85.64	0.00153832060919772\\
85.65	0.00153794456225566\\
85.66	0.00153756700442433\\
85.67	0.00153718792087547\\
85.68	0.00153680729662369\\
85.69	0.00153642511652483\\
85.7	0.00153604136527428\\
85.71	0.00153565602740526\\
85.72	0.00153526908728717\\
85.73	0.00153488052912381\\
85.74	0.00153449033695168\\
85.75	0.00153409849463819\\
85.76	0.00153370498587987\\
85.77	0.00153330979420061\\
85.78	0.00153291290294981\\
85.79	0.00153251429530056\\
85.8	0.00153211395424779\\
85.81	0.00153171186260638\\
85.82	0.00153130800300927\\
85.83	0.00153090235790556\\
85.84	0.00153049490955857\\
85.85	0.00153008564004388\\
85.86	0.00152967453124739\\
85.87	0.00152926156486327\\
85.88	0.00152884672239201\\
85.89	0.00152842998513833\\
85.9	0.00152801133420919\\
85.91	0.00152759075051163\\
85.92	0.00152716821475077\\
85.93	0.0015267437074276\\
85.94	0.0015263172088369\\
85.95	0.00152588869906507\\
85.96	0.0015254581579879\\
85.97	0.00152502556526844\\
85.98	0.0015245909003547\\
85.99	0.00152415414247744\\
86	0.00152371527064786\\
86.01	0.00152327426365535\\
86.02	0.00152283110006513\\
86.03	0.00152238575821591\\
86.04	0.00152193821621751\\
86.05	0.00152148845194852\\
86.06	0.0015210364430538\\
86.07	0.0015205821669421\\
86.08	0.00152012560078356\\
86.09	0.00151966672150724\\
86.1	0.00151920550579855\\
86.11	0.0015187419300968\\
86.12	0.00151827597059251\\
86.13	0.00151780760322492\\
86.14	0.00151733680367929\\
86.15	0.00151686354738431\\
86.16	0.00151638780950935\\
86.17	0.00151590956496185\\
86.18	0.0015154287883845\\
86.19	0.00151494545415252\\
86.2	0.00151445953637088\\
86.21	0.00151397100887146\\
86.22	0.00151347984521025\\
86.23	0.00151298601866439\\
86.24	0.00151248950222935\\
86.25	0.00151199026861598\\
86.26	0.00151148829024751\\
86.27	0.00151098353925659\\
86.28	0.00151047598748228\\
86.29	0.00150996560646693\\
86.3	0.00150945236745319\\
86.31	0.00150893624138082\\
86.32	0.00150841719888358\\
86.33	0.00150789521028602\\
86.34	0.00150737024560031\\
86.35	0.00150684227452297\\
86.36	0.00150631126643159\\
86.37	0.00150577719038152\\
86.38	0.00150524001510256\\
86.39	0.00150469970899553\\
86.4	0.0015041562401289\\
86.41	0.00150360957623533\\
86.42	0.00150305968470818\\
86.43	0.00150250653259801\\
86.44	0.00150195008660902\\
86.45	0.00150139031309547\\
86.46	0.00150082717805805\\
86.47	0.00150026064714021\\
86.48	0.00149969068562451\\
86.49	0.00149911725842882\\
86.5	0.0014985403301026\\
86.51	0.00149795986482305\\
86.52	0.00149737582639132\\
86.53	0.00149678817822855\\
86.54	0.00149619688337201\\
86.55	0.0014956019044711\\
86.56	0.00149500320378335\\
86.57	0.00149440074317039\\
86.58	0.00149379448409385\\
86.59	0.00149318438761122\\
86.6	0.00149257041437174\\
86.61	0.00149195252461211\\
86.62	0.0014913306781523\\
86.63	0.00149070483439124\\
86.64	0.00149007495230246\\
86.65	0.00148944099042975\\
86.66	0.00148880290688269\\
86.67	0.00148816065933222\\
86.68	0.00148751420500612\\
86.69	0.00148686350068443\\
86.7	0.00148620850269487\\
86.71	0.00148554916690816\\
86.72	0.00148488544873337\\
86.73	0.00148421730311314\\
86.74	0.00148354468451892\\
86.75	0.00148286754694607\\
86.76	0.00148218584390907\\
86.77	0.00148149952843651\\
86.78	0.00148080855306612\\
86.79	0.00148011286983977\\
86.8	0.00147941243029834\\
86.81	0.00147870718547664\\
86.82	0.00147799918903405\\
86.83	0.00147728932369116\\
86.84	0.00147657757191568\\
86.85	0.00147586391597907\\
86.86	0.00147514833795425\\
86.87	0.00147443081971334\\
86.88	0.00147371134292536\\
86.89	0.00147298988905393\\
86.9	0.00147226643935485\\
86.91	0.0014715409748738\\
86.92	0.00147081347644387\\
86.93	0.00147008392468315\\
86.94	0.00146935229999228\\
86.95	0.0014686185825519\\
86.96	0.00146788275232022\\
86.97	0.00146714478903039\\
86.98	0.00146640467218796\\
86.99	0.00146566238106825\\
87	0.00146491789471374\\
87.01	0.00146417119193138\\
87.02	0.00146342225128989\\
87.03	0.00146267105111703\\
87.04	0.00146191756949685\\
87.05	0.00146116178426688\\
87.06	0.00146040367301532\\
87.07	0.00145964321307817\\
87.08	0.00145888038153631\\
87.09	0.00145811515521266\\
87.1	0.00145734751066912\\
87.11	0.00145657742420365\\
87.12	0.0014558048718472\\
87.13	0.00145502982936067\\
87.14	0.00145425227223182\\
87.15	0.00145347217567211\\
87.16	0.00145268951461356\\
87.17	0.0014519042637055\\
87.18	0.0014511163973114\\
87.19	0.00145032588950553\\
87.2	0.00144953271406965\\
87.21	0.00144873684448967\\
87.22	0.00144793825395228\\
87.23	0.00144713691534145\\
87.24	0.00144633280123502\\
87.25	0.00144552588390117\\
87.26	0.00144471613529483\\
87.27	0.00144390352705414\\
87.28	0.00144308803049679\\
87.29	0.00144226961661634\\
87.3	0.0014414482560785\\
87.31	0.00144062391921738\\
87.32	0.00143979657603168\\
87.33	0.00143896619618081\\
87.34	0.00143813274898106\\
87.35	0.0014372962034016\\
87.36	0.00143645652806053\\
87.37	0.00143561369122082\\
87.38	0.00143476766078629\\
87.39	0.00143391840429742\\
87.4	0.00143306588892723\\
87.41	0.00143221008147702\\
87.42	0.00143135094837214\\
87.43	0.00143048845565764\\
87.44	0.00142962256899394\\
87.45	0.00142875325365239\\
87.46	0.00142788047451078\\
87.47	0.00142700419604889\\
87.48	0.00142612438234383\\
87.49	0.0014252409970655\\
87.5	0.00142435400347185\\
87.51	0.00142346336440418\\
87.52	0.00142256904228236\\
87.53	0.00142167099909996\\
87.54	0.00142076919641938\\
87.55	0.00141986359536689\\
87.56	0.00141895415662762\\
87.57	0.00141804084044047\\
87.58	0.00141712360659303\\
87.59	0.00141620241441635\\
87.6	0.00141527722277972\\
87.61	0.00141434799008537\\
87.62	0.00141341467426307\\
87.63	0.00141247723276473\\
87.64	0.0014115356225589\\
87.65	0.00141058980012522\\
87.66	0.00140963972144879\\
87.67	0.00140868534201451\\
87.68	0.00140772661680129\\
87.69	0.00140676350027628\\
87.7	0.00140579594638896\\
87.71	0.00140482390856517\\
87.72	0.00140384733970117\\
87.73	0.00140286619215744\\
87.74	0.00140188041775263\\
87.75	0.00140088996775723\\
87.76	0.00139989479288736\\
87.77	0.00139889484329832\\
87.78	0.00139789006857819\\
87.79	0.00139688041774129\\
87.8	0.00139586583922157\\
87.81	0.00139484628086596\\
87.82	0.00139382168992761\\
87.83	0.00139279201305904\\
87.84	0.00139175719630526\\
87.85	0.00139071718509678\\
87.86	0.0013896719242425\\
87.87	0.00138862135792263\\
87.88	0.00138756542968139\\
87.89	0.00138650408241973\\
87.9	0.00138543725838792\\
87.91	0.00138436489917805\\
87.92	0.00138328694571648\\
87.93	0.00138220333825618\\
87.94	0.00138111401636893\\
87.95	0.00138001891893756\\
87.96	0.00137891798414795\\
87.97	0.00137781114948104\\
87.98	0.00137669835170472\\
87.99	0.0013755795268656\\
88	0.00137445461028073\\
88.01	0.0013733235365292\\
88.02	0.00137218623944359\\
88.03	0.00137104265210147\\
88.04	0.00136989270681662\\
88.05	0.00136873633513027\\
88.06	0.00136757346780221\\
88.07	0.00136640403480177\\
88.08	0.00136522796529874\\
88.09	0.00136404518765412\\
88.1	0.00136285562941084\\
88.11	0.00136165921728434\\
88.12	0.00136045587715299\\
88.13	0.0013592455340485\\
88.14	0.00135802811214611\\
88.15	0.00135680353475478\\
88.16	0.00135557172430716\\
88.17	0.00135433260234949\\
88.18	0.00135308608953144\\
88.19	0.00135183210559568\\
88.2	0.00135057056936753\\
88.21	0.00134930139874428\\
88.22	0.00134802451068459\\
88.23	0.00134673982119759\\
88.24	0.00134544724533197\\
88.25	0.00134414669716488\\
88.26	0.00134283808979074\\
88.27	0.00134152133530993\\
88.28	0.00134019634481725\\
88.29	0.00133886302839041\\
88.3	0.00133752129507822\\
88.31	0.00133617105288875\\
88.32	0.00133481220877735\\
88.33	0.00133344466863443\\
88.34	0.00133206833727324\\
88.35	0.0013306831184174\\
88.36	0.00132928891468832\\
88.37	0.00132788562759249\\
88.38	0.00132647315750861\\
88.39	0.00132505140367455\\
88.4	0.0013236202641742\\
88.41	0.0013221796359241\\
88.42	0.00132072941466003\\
88.43	0.00131926949492329\\
88.44	0.00131779977004696\\
88.45	0.0013163201321419\\
88.46	0.00131483047208267\\
88.47	0.00131333067949321\\
88.48	0.00131182064273238\\
88.49	0.00131030024887938\\
88.5	0.00130876938371892\\
88.51	0.00130722793172626\\
88.52	0.0013056757760521\\
88.53	0.00130411279850722\\
88.54	0.001302538879547\\
88.55	0.00130095389825576\\
88.56	0.00129935773233089\\
88.57	0.0012977502580668\\
88.58	0.00129613135033867\\
88.59	0.00129450088258607\\
88.6	0.00129285872679632\\
88.61	0.00129120475348767\\
88.62	0.00128953883169231\\
88.63	0.00128786082893919\\
88.64	0.00128617061123654\\
88.65	0.00128446804305433\\
88.66	0.00128275298730644\\
88.67	0.0012810253053326\\
88.68	0.0012792848568802\\
88.69	0.00127753150008579\\
88.7	0.00127576509145647\\
88.71	0.00127398548585094\\
88.72	0.00127219253646046\\
88.73	0.00127038609478946\\
88.74	0.00126856601063603\\
88.75	0.00126673213207207\\
88.76	0.00126488430542334\\
88.77	0.00126302237524916\\
88.78	0.00126114618432191\\
88.79	0.00125925557360634\\
88.8	0.00125735038223852\\
88.81	0.00125543044750468\\
88.82	0.00125349560481971\\
88.83	0.0012515456877054\\
88.84	0.00124958052776851\\
88.85	0.00124759995467848\\
88.86	0.00124560379614495\\
88.87	0.00124359187789497\\
88.88	0.00124156402364996\\
88.89	0.00123952005510243\\
88.9	0.00123745979189233\\
88.91	0.00123538305158326\\
88.92	0.00123328964963824\\
88.93	0.00123117939939534\\
88.94	0.00122905211204292\\
88.95	0.00122690759659463\\
88.96	0.00122474565986411\\
88.97	0.00122256610643934\\
88.98	0.00122036873865676\\
88.99	0.00121815335657504\\
89	0.00121591975794852\\
89.01	0.00121366773820042\\
89.02	0.00121139709039562\\
89.03	0.00120910760521319\\
89.04	0.00120679907091859\\
89.05	0.00120447127333548\\
89.06	0.00120212399581731\\
89.07	0.00119975701921844\\
89.08	0.001197370121865\\
89.09	0.00119496307952538\\
89.1	0.00119253566538038\\
89.11	0.00119008764999297\\
89.12	0.00118761880127776\\
89.13	0.00118512888446999\\
89.14	0.00118261766209427\\
89.15	0.0011800848939329\\
89.16	0.00117753033699378\\
89.17	0.00117495374547798\\
89.18	0.00117235487074688\\
89.19	0.00116973992422297\\
89.2	0.00116712005680812\\
89.21	0.00116449522790771\\
89.22	0.00116186539646808\\
89.23	0.00115923052097104\\
89.24	0.0011565905594283\\
89.25	0.00115394546937585\\
89.26	0.00115129520786821\\
89.27	0.00114863973147269\\
89.28	0.00114597899626355\\
89.29	0.00114331295781603\\
89.3	0.00114064157120042\\
89.31	0.00113796479097595\\
89.32	0.00113528257118469\\
89.33	0.0011325948653453\\
89.34	0.00112990162644677\\
89.35	0.00112720280694201\\
89.36	0.00112449835874149\\
89.37	0.00112178823320661\\
89.38	0.00111907238114319\\
89.39	0.00111635075279471\\
89.4	0.00111362329783561\\
89.41	0.0011108899653644\\
89.42	0.00110815070389672\\
89.43	0.00110540546135837\\
89.44	0.00110265418507815\\
89.45	0.00109989682178071\\
89.46	0.00109713331757925\\
89.47	0.00109436361796816\\
89.48	0.00109158766781556\\
89.49	0.00108880541135574\\
89.5	0.00108601679218153\\
89.51	0.00108322175323655\\
89.52	0.0010804202368074\\
89.53	0.00107761218451571\\
89.54	0.0010747975373101\\
89.55	0.00107197623545812\\
89.56	0.00106914821853794\\
89.57	0.0010663134254301\\
89.58	0.00106347179430903\\
89.59	0.00106062326263455\\
89.6	0.00105776776714321\\
89.61	0.00105490524383957\\
89.62	0.00105203562798733\\
89.63	0.00104915885410039\\
89.64	0.00104627485593377\\
89.65	0.00104338356647443\\
89.66	0.00104048491793196\\
89.67	0.0010375788417292\\
89.68	0.00103466526849271\\
89.69	0.0010317441280431\\
89.7	0.0010288153493853\\
89.71	0.00102587886069867\\
89.72	0.00102293458932699\\
89.73	0.00101998246176834\\
89.74	0.00101702240366482\\
89.75	0.00101405433979225\\
89.76	0.00101107819404955\\
89.77	0.00100809388944821\\
89.78	0.00100510134810146\\
89.79	0.00100210049121337\\
89.8	0.000999091239067855\\
89.81	0.000996073511017466\\
89.82	0.000993047225472089\\
89.83	0.000990012299887507\\
89.84	0.000986968650753789\\
89.85	0.000983916193583572\\
89.86	0.000980854842900168\\
89.87	0.000977784512225546\\
89.88	0.000974705114068148\\
89.89	0.000971616559910564\\
89.9	0.000968518760197053\\
89.91	0.000965411624320907\\
89.92	0.000962295060611658\\
89.93	0.000959168976322131\\
89.94	0.000956033277615327\\
89.95	0.000952887869551156\\
89.96	0.000949732656072995\\
89.97	0.00094656753999408\\
89.98	0.000943392422983734\\
89.99	0.000940207205553423\\
90	0.000937011787042636\\
90.01	0.000933806065604585\\
90.02	0.000930589938191742\\
90.03	0.000927363300541178\\
90.04	0.000924126047159734\\
90.05	0.000920878071308997\\
90.06	0.000917619264990097\\
90.07	0.000914349518928308\\
90.08	0.000911068722557465\\
90.09	0.000907776764004171\\
90.1	0.000904473530071837\\
90.11	0.000901158906224482\\
90.12	0.000897832776570366\\
90.13	0.000894495023845405\\
90.14	0.000891145529396377\\
90.15	0.000887784173163915\\
90.16	0.000884410833665304\\
90.17	0.000881025387977043\\
90.18	0.000877627711717208\\
90.19	0.000874217679027572\\
90.2	0.00087079516255552\\
90.21	0.000867360033435736\\
90.22	0.000863912161271648\\
90.23	0.000860451414116654\\
90.24	0.000856977658455105\\
90.25	0.000853490759183052\\
90.26	0.000849990579588754\\
90.27	0.000846476981332936\\
90.28	0.000842949824428804\\
90.29	0.000839408967221811\\
90.3	0.000835854266369156\\
90.31	0.000832285576819048\\
90.32	0.000828702751789678\\
90.33	0.000825105642747969\\
90.34	0.000821494099388009\\
90.35	0.000817867969609259\\
90.36	0.000814227099494452\\
90.37	0.000810571333287232\\
90.38	0.000806900513369511\\
90.39	0.00080321448023853\\
90.4	0.000799513072483635\\
90.41	0.000795796126762774\\
90.42	0.000792063477778668\\
90.43	0.000788314958254717\\
90.44	0.000784550398910566\\
90.45	0.000780769628437391\\
90.46	0.000776972473472857\\
90.47	0.000773158758575759\\
90.48	0.000769328306200351\\
90.49	0.000765480936670346\\
90.5	0.000761616468152579\\
90.51	0.00075773471663035\\
90.52	0.000753835495876419\\
90.53	0.000749918617425656\\
90.54	0.000745983890547362\\
90.55	0.000742031122217207\\
90.56	0.000738060117088849\\
90.57	0.000734070677465164\\
90.58	0.000730062603269117\\
90.59	0.000726035692014267\\
90.6	0.0007219897387749\\
90.61	0.000717924536155767\\
90.62	0.000713839874261454\\
90.63	0.000709735540665338\\
90.64	0.000705611320378184\\
90.65	0.000701466995816295\\
90.66	0.000697302346769294\\
90.67	0.000693117150367477\\
90.68	0.000688911181048742\\
90.69	0.000684684210525116\\
90.7	0.000680436007748831\\
90.71	0.000676166338877978\\
90.72	0.000671874967241725\\
90.73	0.000667561653305074\\
90.74	0.000663226154633174\\
90.75	0.000658868225855182\\
90.76	0.00065448761862765\\
90.77	0.000650094623876368\\
90.78	0.000645704532660156\\
90.79	0.000641317404645986\\
90.8	0.000636933300274603\\
90.81	0.000632552280770323\\
90.82	0.000628174408150942\\
90.83	0.000623799745237776\\
90.84	0.000619428355665823\\
90.85	0.000615060303894074\\
90.86	0.000610695655215919\\
90.87	0.000606334475769715\\
90.88	0.00060197683254948\\
90.89	0.000597622793415711\\
90.9	0.000593272427106351\\
90.91	0.000588925803247896\\
90.92	0.000584582992366628\\
90.93	0.00058024406590001\\
90.94	0.000575909096208206\\
90.95	0.000571578156585761\\
90.96	0.000567251321273428\\
90.97	0.000562928665470126\\
90.98	0.000558610265345077\\
90.99	0.000554296198050077\\
91	0.000549986541731931\\
91.01	0.000545681375545037\\
91.02	0.000541380779664143\\
91.03	0.00053708483529725\\
91.04	0.000532793624698689\\
91.05	0.000528507231182354\\
91.06	0.000524225739135119\\
91.07	0.000519949234030401\\
91.08	0.000515677802441916\\
91.09	0.000511411532057597\\
91.1	0.000507150511693687\\
91.11	0.000502894831309024\\
91.12	0.000498644582019484\\
91.13	0.000494399856112632\\
91.14	0.000490160747062533\\
91.15	0.000485927349544774\\
91.16	0.000481699759451656\\
91.17	0.000477478073907592\\
91.18	0.000473262391284689\\
91.19	0.000469052811218536\\
91.2	0.000464849434624185\\
91.21	0.000460652363712339\\
91.22	0.000456461702005733\\
91.23	0.000452277554355743\\
91.24	0.000448100026959181\\
91.25	0.000443929227375323\\
91.26	0.000439765264543132\\
91.27	0.000435608248798715\\
91.28	0.000431458291892995\\
91.29	0.0004273155070096\\
91.3	0.000423180008782987\\
91.31	0.000419051913316787\\
91.32	0.000414931338202385\\
91.33	0.00041081840253774\\
91.34	0.000406713226946428\\
91.35	0.00040261593359694\\
91.36	0.000398526646222214\\
91.37	0.000394445490139419\\
91.38	0.000390372592269978\\
91.39	0.000386308081159864\\
91.4	0.000382252087000127\\
91.41	0.000378204741647698\\
91.42	0.000374166178646444\\
91.43	0.000370136533248496\\
91.44	0.000366115942435843\\
91.45	0.000362104544942196\\
91.46	0.000358102481275123\\
91.47	0.000354109893738477\\
91.48	0.000350126926455096\\
91.49	0.000346153725389781\\
91.5	0.00034219043837258\\
91.51	0.000338237215122359\\
91.52	0.000334294207270656\\
91.53	0.000330361568385862\\
91.54	0.000326439453997674\\
91.55	0.000322528021621882\\
91.56	0.000318627430785453\\
91.57	0.000314737843051935\\
91.58	0.000310859422047177\\
91.59	0.000306992333485377\\
91.6	0.000303136745195451\\
91.61	0.000299292827147741\\
91.62	0.00029546075148104\\
91.63	0.000291640692529987\\
91.64	0.000287832826852774\\
91.65	0.000284037333259218\\
91.66	0.000280254392839184\\
91.67	0.000276484188991353\\
91.68	0.000272726907452366\\
91.69	0.000268982736326324\\
91.7	0.000265251866114652\\
91.71	0.000261534489746354\\
91.72	0.000257830802608628\\
91.73	0.000254141002577871\\
91.74	0.000250465290051082\\
91.75	0.000246803867977641\\
91.76	0.000243156941891491\\
91.77	0.000239524719943729\\
91.78	0.000235907412935596\\
91.79	0.000232305234351884\\
91.8	0.000228718400394748\\
91.81	0.000225147130017965\\
91.82	0.000221591644961595\\
91.83	0.000218052169787078\\
91.84	0.000214528931912792\\
91.85	0.000211022161650016\\
91.86	0.000207532092239379\\
91.87	0.000204058959887732\\
91.88	0.000200603003805498\\
91.89	0.00019716446624448\\
91.9	0.000193743592536136\\
91.91	0.000190340631130343\\
91.92	0.000186955833634617\\
91.93	0.00018358945485385\\
91.94	0.000180241752830523\\
91.95	0.000176912988885414\\
91.96	0.000173603427658832\\
91.97	0.000170313337152329\\
91.98	0.000167042988770971\\
91.99	0.00016379265736609\\
92	0.000160562621278596\\
92.01	0.000157353162382811\\
92.02	0.000154164566130849\\
92.03	0.000150997121597539\\
92.04	0.000147851121525915\\
92.05	0.000144726862373254\\
92.06	0.000141624644357692\\
92.07	0.000138544771505413\\
92.08	0.000135487551698409\\
92.09	0.000132453296722854\\
92.1	0.000129442322318039\\
92.11	0.000126454948225945\\
92.12	0.000123491498241394\\
92.13	0.00012055230026284\\
92.14	0.000117637686343772\\
92.15	0.000114747992744751\\
92.16	0.000111883559986089\\
92.17	0.000109044732901162\\
92.18	0.000106231860690397\\
92.19	0.000103445296975896\\
92.2	0.000100685399855782\\
92.21	9.79525319583525e-05\\
92.22	9.52470604986648e-05\\
92.23	9.2569357335797e-05\\
92.24	8.99197990308076e-05\\
92.25	8.7298766905412e-05\\
92.26	8.47066471013607e-05\\
92.27	8.21438306405475e-05\\
92.28	7.96107134858441e-05\\
92.29	7.71076966026765e-05\\
92.3	7.46351860213424e-05\\
92.31	7.219359290009e-05\\
92.32	6.97833335889624e-05\\
92.33	6.74048296944053e-05\\
92.34	6.50585081446615e-05\\
92.35	6.27448012559643e-05\\
92.36	6.04641467995114e-05\\
92.37	5.82169880692599e-05\\
92.38	5.60037739505363e-05\\
92.39	5.38249589894686e-05\\
92.4	5.16810034632571e-05\\
92.41	4.95723734512867e-05\\
92.42	4.74995409070981e-05\\
92.43	4.54629837312159e-05\\
92.44	4.34631858448487e-05\\
92.45	4.15006372644806e-05\\
92.46	3.95758341773488e-05\\
92.47	3.76892790178128e-05\\
92.48	3.58414805446596e-05\\
92.49	3.40329539193027e-05\\
92.5	3.22642207849343e-05\\
92.51	3.05358093466036e-05\\
92.52	2.88482544522541e-05\\
92.53	2.72020976747264e-05\\
92.54	2.55978873947189e-05\\
92.55	2.40361788847489e-05\\
92.56	2.25175343940912e-05\\
92.57	2.10425232347305e-05\\
92.58	1.96117218683213e-05\\
92.59	1.82257139941765e-05\\
92.6	1.68850906382915e-05\\
92.61	1.55904502434079e-05\\
92.62	1.43423987601504e-05\\
92.63	1.31415497392068e-05\\
92.64	1.19885244246158e-05\\
92.65	1.08839518481244e-05\\
92.66	9.82846892465597e-06\\
92.67	8.82272054889e-06\\
92.68	7.86735969296633e-06\\
92.69	6.96304750532652e-06\\
92.7	6.11045341069802e-06\\
92.71	5.31025521124419e-06\\
92.72	4.56313918887653e-06\\
92.73	3.86980020876566e-06\\
92.74	3.23094182401971e-06\\
92.75	2.64727638159788e-06\\
92.76	2.11952512942613e-06\\
92.77	1.64841832474109e-06\\
92.78	1.23469534368467e-06\\
92.79	8.79104792147686e-07\\
92.8	5.82404617888513e-07\\
92.81	3.45362223909407e-07\\
92.82	1.68754583156419e-07\\
92.83	5.33683544926e-08\\
92.84	0\\
92.85	0\\
92.86	0\\
92.87	0\\
92.88	0\\
92.89	0\\
92.9	0\\
92.91	0\\
92.92	0\\
92.93	0\\
92.94	0\\
92.95	0\\
92.96	0\\
92.97	0\\
92.98	0\\
92.99	0\\
93	0\\
93.01	0\\
93.02	0\\
93.03	0\\
93.04	0\\
93.05	0\\
93.06	0\\
93.07	0\\
93.08	0\\
93.09	0\\
93.1	0\\
93.11	0\\
93.12	0\\
93.13	0\\
93.14	0\\
93.15	0\\
93.16	0\\
93.17	0\\
93.18	0\\
93.19	0\\
93.2	0\\
93.21	0\\
93.22	0\\
93.23	0\\
93.24	0\\
93.25	0\\
93.26	0\\
93.27	0\\
93.28	0\\
93.29	0\\
93.3	0\\
93.31	0\\
93.32	0\\
93.33	0\\
93.34	0\\
93.35	0\\
93.36	0\\
93.37	0\\
93.38	0\\
93.39	0\\
93.4	0\\
93.41	0\\
93.42	0\\
93.43	0\\
93.44	0\\
93.45	0\\
93.46	0\\
93.47	0\\
93.48	0\\
93.49	0\\
93.5	0\\
93.51	0\\
93.52	0\\
93.53	0\\
93.54	0\\
93.55	0\\
93.56	0\\
93.57	0\\
93.58	0\\
93.59	0\\
93.6	0\\
93.61	0\\
93.62	0\\
93.63	0\\
93.64	0\\
93.65	0\\
93.66	0\\
93.67	0\\
93.68	0\\
93.69	0\\
93.7	0\\
93.71	0\\
93.72	0\\
93.73	0\\
93.74	0\\
93.75	0\\
93.76	0\\
93.77	0\\
93.78	0\\
93.79	0\\
93.8	0\\
93.81	0\\
93.82	0\\
93.83	0\\
93.84	0\\
93.85	0\\
93.86	0\\
93.87	0\\
93.88	0\\
93.89	0\\
93.9	0\\
93.91	0\\
93.92	0\\
93.93	0\\
93.94	0\\
93.95	0\\
93.96	0\\
93.97	0\\
93.98	0\\
93.99	0\\
94	0\\
94.01	0\\
94.02	0\\
94.03	0\\
94.04	0\\
94.05	0\\
94.06	0\\
94.07	0\\
94.08	0\\
94.09	0\\
94.1	0\\
94.11	0\\
94.12	0\\
94.13	0\\
94.14	0\\
94.15	0\\
94.16	0\\
94.17	0\\
94.18	0\\
94.19	0\\
94.2	0\\
94.21	0\\
94.22	0\\
94.23	0\\
94.24	0\\
94.25	0\\
94.26	0\\
94.27	0\\
94.28	0\\
94.29	0\\
94.3	0\\
94.31	0\\
94.32	0\\
94.33	0\\
94.34	0\\
94.35	0\\
94.36	0\\
94.37	0\\
94.38	0\\
94.39	0\\
94.4	0\\
94.41	0\\
94.42	0\\
94.43	0\\
94.44	0\\
94.45	0\\
94.46	0\\
94.47	0\\
94.48	0\\
94.49	0\\
94.5	0\\
94.51	0\\
94.52	0\\
94.53	0\\
94.54	0\\
94.55	0\\
94.56	0\\
94.57	0\\
94.58	0\\
94.59	0\\
94.6	0\\
94.61	0\\
94.62	0\\
94.63	0\\
94.64	0\\
94.65	0\\
94.66	0\\
94.67	0\\
94.68	0\\
94.69	0\\
94.7	0\\
94.71	0\\
94.72	0\\
94.73	0\\
94.74	0\\
94.75	0\\
94.76	0\\
94.77	0\\
94.78	0\\
94.79	0\\
94.8	0\\
94.81	0\\
94.82	0\\
94.83	0\\
94.84	0\\
94.85	0\\
94.86	0\\
94.87	0\\
94.88	0\\
94.89	0\\
94.9	0\\
94.91	0\\
94.92	0\\
94.93	0\\
94.94	0\\
94.95	0\\
94.96	0\\
94.97	0\\
94.98	0\\
94.99	0\\
95	0\\
95.01	0\\
95.02	0\\
95.03	0\\
95.04	0\\
95.05	0\\
95.06	0\\
95.07	0\\
95.08	0\\
95.09	0\\
95.1	0\\
95.11	0\\
95.12	0\\
95.13	0\\
95.14	0\\
95.15	0\\
95.16	0\\
95.17	0\\
95.18	0\\
95.19	0\\
95.2	0\\
95.21	0\\
95.22	0\\
95.23	0\\
95.24	0\\
95.25	0\\
95.26	0\\
95.27	0\\
95.28	0\\
95.29	0\\
95.3	0\\
95.31	0\\
95.32	0\\
95.33	0\\
95.34	0\\
95.35	0\\
95.36	0\\
95.37	0\\
95.38	0\\
95.39	0\\
95.4	0\\
95.41	0\\
95.42	0\\
95.43	0\\
95.44	0\\
95.45	0\\
95.46	0\\
95.47	0\\
95.48	0\\
95.49	0\\
95.5	0\\
95.51	0\\
95.52	0\\
95.53	0\\
95.54	0\\
95.55	0\\
95.56	0\\
95.57	0\\
95.58	0\\
95.59	0\\
95.6	0\\
95.61	0\\
95.62	0\\
95.63	0\\
95.64	0\\
95.65	0\\
95.66	0\\
95.67	0\\
95.68	0\\
95.69	0\\
95.7	0\\
95.71	0\\
95.72	0\\
95.73	0\\
95.74	0\\
95.75	0\\
95.76	0\\
95.77	0\\
95.78	0\\
95.79	0\\
95.8	0\\
95.81	0\\
95.82	0\\
95.83	0\\
95.84	0\\
95.85	0\\
95.86	0\\
95.87	0\\
95.88	0\\
95.89	0\\
95.9	0\\
95.91	0\\
95.92	0\\
95.93	0\\
95.94	0\\
95.95	0\\
95.96	0\\
95.97	0\\
95.98	0\\
95.99	0\\
96	0\\
96.01	0\\
96.02	0\\
96.03	0\\
96.04	0\\
96.05	0\\
96.06	0\\
96.07	0\\
96.08	0\\
96.09	0\\
96.1	0\\
96.11	0\\
96.12	0\\
96.13	0\\
96.14	0\\
96.15	0\\
96.16	0\\
96.17	0\\
96.18	0\\
96.19	0\\
96.2	0\\
96.21	0\\
96.22	0\\
96.23	0\\
96.24	0\\
96.25	0\\
96.26	0\\
96.27	0\\
96.28	0\\
96.29	0\\
96.3	0\\
96.31	0\\
96.32	0\\
96.33	0\\
96.34	0\\
96.35	0\\
96.36	0\\
96.37	0\\
96.38	0\\
96.39	0\\
96.4	0\\
96.41	0\\
96.42	0\\
96.43	0\\
96.44	0\\
96.45	0\\
96.46	0\\
96.47	0\\
96.48	0\\
96.49	0\\
96.5	0\\
96.51	0\\
96.52	0\\
96.53	0\\
96.54	0\\
96.55	0\\
96.56	0\\
96.57	0\\
96.58	0\\
96.59	0\\
96.6	0\\
96.61	0\\
96.62	0\\
96.63	0\\
96.64	0\\
96.65	0\\
96.66	0\\
96.67	0\\
96.68	0\\
96.69	0\\
96.7	0\\
96.71	0\\
96.72	0\\
96.73	0\\
96.74	0\\
96.75	0\\
96.76	0\\
96.77	0\\
96.78	0\\
96.79	0\\
96.8	0\\
96.81	0\\
96.82	0\\
96.83	0\\
96.84	0\\
96.85	0\\
96.86	0\\
96.87	0\\
96.88	0\\
96.89	0\\
96.9	0\\
96.91	0\\
96.92	0\\
96.93	0\\
96.94	0\\
96.95	0\\
96.96	0\\
96.97	0\\
96.98	0\\
96.99	0\\
97	0\\
97.01	0\\
97.02	0\\
97.03	0\\
97.04	0\\
97.05	0\\
97.06	0\\
97.07	0\\
97.08	0\\
97.09	0\\
97.1	0\\
97.11	0\\
97.12	0\\
97.13	0\\
97.14	0\\
97.15	0\\
97.16	0\\
97.17	0\\
97.18	0\\
97.19	0\\
97.2	0\\
97.21	0\\
97.22	0\\
97.23	0\\
97.24	0\\
97.25	0\\
97.26	0\\
97.27	0\\
97.28	0\\
97.29	0\\
97.3	0\\
97.31	0\\
97.32	0\\
97.33	0\\
97.34	0\\
97.35	0\\
97.36	0\\
97.37	0\\
97.38	0\\
97.39	0\\
97.4	0\\
97.41	0\\
97.42	0\\
97.43	0\\
97.44	0\\
97.45	0\\
97.46	0\\
97.47	0\\
97.48	0\\
97.49	0\\
97.5	0\\
97.51	0\\
97.52	0\\
97.53	0\\
97.54	0\\
97.55	0\\
97.56	0\\
97.57	0\\
97.58	0\\
97.59	0\\
97.6	0\\
97.61	0\\
97.62	0\\
97.63	0\\
97.64	0\\
97.65	0\\
97.66	0\\
97.67	0\\
97.68	0\\
97.69	0\\
97.7	0\\
97.71	0\\
97.72	0\\
97.73	0\\
97.74	0\\
97.75	0\\
97.76	0\\
97.77	0\\
97.78	0\\
97.79	0\\
97.8	0\\
97.81	0\\
97.82	0\\
97.83	0\\
97.84	0\\
97.85	0\\
97.86	0\\
97.87	0\\
97.88	0\\
97.89	0\\
97.9	0\\
97.91	0\\
97.92	0\\
97.93	0\\
97.94	0\\
97.95	0\\
97.96	0\\
97.97	0\\
97.98	0\\
97.99	0\\
98	0\\
98.01	0\\
98.02	0\\
98.03	0\\
98.04	0\\
98.05	0\\
98.06	0\\
98.07	0\\
98.08	0\\
98.09	0\\
98.1	0\\
98.11	0\\
98.12	0\\
98.13	0\\
98.14	0\\
98.15	0\\
98.16	0\\
98.17	0\\
98.18	0\\
98.19	0\\
98.2	0\\
98.21	0\\
98.22	0\\
98.23	0\\
98.24	0\\
98.25	0\\
98.26	0\\
98.27	0\\
98.28	0\\
98.29	0\\
98.3	0\\
98.31	0\\
98.32	0\\
98.33	0\\
98.34	0\\
98.35	0\\
98.36	0\\
98.37	0\\
98.38	0\\
98.39	0\\
98.4	0\\
98.41	0\\
98.42	0\\
98.43	0\\
98.44	0\\
98.45	0\\
98.46	0\\
98.47	0\\
98.48	0\\
98.49	0\\
98.5	0\\
98.51	0\\
98.52	0\\
98.53	0\\
98.54	0\\
98.55	0\\
98.56	0\\
98.57	0\\
98.58	0\\
98.59	0\\
98.6	0\\
98.61	0\\
98.62	0\\
98.63	0\\
98.64	0\\
98.65	0\\
98.66	0\\
98.67	0\\
98.68	0\\
98.69	0\\
98.7	0\\
98.71	0\\
98.72	0\\
98.73	0\\
98.74	0\\
98.75	0\\
98.76	0\\
98.77	0\\
98.78	0\\
98.79	0\\
98.8	0\\
98.81	0\\
98.82	0\\
98.83	0\\
98.84	0\\
98.85	0\\
98.86	0\\
98.87	0\\
98.88	0\\
98.89	0\\
98.9	0\\
98.91	0\\
98.92	0\\
98.93	0\\
98.94	0\\
98.95	0\\
98.96	0\\
98.97	0\\
98.98	0\\
98.99	0\\
99	0\\
99.01	0\\
99.02	0\\
99.03	0\\
99.04	0\\
99.05	0\\
99.06	0\\
99.07	0\\
99.08	0\\
99.09	0\\
99.1	0\\
99.11	0\\
99.12	0\\
99.13	0\\
99.14	0\\
99.15	0\\
99.16	0\\
99.17	0\\
99.18	0\\
99.19	0\\
99.2	0\\
99.21	0\\
99.22	0\\
99.23	0\\
99.24	0\\
99.25	0\\
99.26	0\\
99.27	0\\
99.28	0\\
99.29	0\\
99.3	0\\
99.31	0\\
99.32	0\\
99.33	0\\
99.34	0\\
99.35	0\\
99.36	0\\
99.37	0\\
99.38	0\\
99.39	0\\
99.4	0\\
99.41	0\\
99.42	0\\
99.43	0\\
99.44	0\\
99.45	0\\
99.46	0\\
99.47	0\\
99.48	0\\
99.49	0\\
99.5	0\\
99.51	0\\
99.52	0\\
99.53	0\\
99.54	0\\
99.55	0\\
99.56	0\\
99.57	0\\
99.58	0\\
99.59	0\\
99.6	0\\
99.61	0\\
99.62	0\\
99.63	0\\
99.64	0\\
99.65	0\\
99.66	0\\
99.67	0\\
99.68	0\\
99.69	0\\
99.7	0\\
99.71	0\\
99.72	0\\
99.73	0\\
99.74	0\\
99.75	0\\
99.76	0\\
99.77	0\\
99.78	0\\
99.79	0\\
99.8	0\\
99.81	0\\
99.82	0\\
99.83	0\\
99.84	0\\
99.85	0\\
99.86	0\\
99.87	0\\
99.88	0\\
99.89	0\\
99.9	0\\
99.91	0\\
99.92	0\\
99.93	0\\
99.94	0\\
99.95	0\\
99.96	0\\
99.97	0\\
99.98	0\\
99.99	0\\
100	0\\
};
\addlegendentry{$q=1$};

\addplot [color=red,solid,forget plot]
  table[row sep=crcr]{%
0.01	0\\
0.02	0\\
0.03	0\\
0.04	0\\
0.05	0\\
0.06	0\\
0.07	0\\
0.08	0\\
0.09	0\\
0.1	0\\
0.11	0\\
0.12	0\\
0.13	0\\
0.14	0\\
0.15	0\\
0.16	0\\
0.17	0\\
0.18	0\\
0.19	0\\
0.2	0\\
0.21	0\\
0.22	0\\
0.23	0\\
0.24	0\\
0.25	0\\
0.26	0\\
0.27	0\\
0.28	0\\
0.29	0\\
0.3	0\\
0.31	0\\
0.32	0\\
0.33	0\\
0.34	0\\
0.35	0\\
0.36	0\\
0.37	0\\
0.38	0\\
0.39	0\\
0.4	0\\
0.41	0\\
0.42	0\\
0.43	0\\
0.44	0\\
0.45	0\\
0.46	0\\
0.47	0\\
0.48	0\\
0.49	0\\
0.5	0\\
0.51	0\\
0.52	0\\
0.53	0\\
0.54	0\\
0.55	0\\
0.56	0\\
0.57	0\\
0.58	0\\
0.59	0\\
0.6	0\\
0.61	0\\
0.62	0\\
0.63	0\\
0.64	0\\
0.65	0\\
0.66	0\\
0.67	0\\
0.68	0\\
0.69	0\\
0.7	0\\
0.71	0\\
0.72	0\\
0.73	0\\
0.74	0\\
0.75	0\\
0.76	0\\
0.77	0\\
0.78	0\\
0.79	0\\
0.8	0\\
0.81	0\\
0.82	0\\
0.83	0\\
0.84	0\\
0.85	0\\
0.86	0\\
0.87	0\\
0.88	0\\
0.89	0\\
0.9	0\\
0.91	0\\
0.92	0\\
0.93	0\\
0.94	0\\
0.95	0\\
0.96	0\\
0.97	0\\
0.98	0\\
0.99	0\\
1	0\\
1.01	0\\
1.02	0\\
1.03	0\\
1.04	0\\
1.05	0\\
1.06	0\\
1.07	0\\
1.08	0\\
1.09	0\\
1.1	0\\
1.11	0\\
1.12	0\\
1.13	0\\
1.14	0\\
1.15	0\\
1.16	0\\
1.17	0\\
1.18	0\\
1.19	0\\
1.2	0\\
1.21	0\\
1.22	0\\
1.23	0\\
1.24	0\\
1.25	0\\
1.26	0\\
1.27	0\\
1.28	0\\
1.29	0\\
1.3	0\\
1.31	0\\
1.32	0\\
1.33	0\\
1.34	0\\
1.35	0\\
1.36	0\\
1.37	0\\
1.38	0\\
1.39	0\\
1.4	0\\
1.41	0\\
1.42	0\\
1.43	0\\
1.44	0\\
1.45	0\\
1.46	0\\
1.47	0\\
1.48	0\\
1.49	0\\
1.5	0\\
1.51	0\\
1.52	0\\
1.53	0\\
1.54	0\\
1.55	0\\
1.56	0\\
1.57	0\\
1.58	0\\
1.59	0\\
1.6	0\\
1.61	0\\
1.62	0\\
1.63	0\\
1.64	0\\
1.65	0\\
1.66	0\\
1.67	0\\
1.68	0\\
1.69	0\\
1.7	0\\
1.71	0\\
1.72	0\\
1.73	0\\
1.74	0\\
1.75	0\\
1.76	0\\
1.77	0\\
1.78	0\\
1.79	0\\
1.8	0\\
1.81	0\\
1.82	0\\
1.83	0\\
1.84	0\\
1.85	0\\
1.86	0\\
1.87	0\\
1.88	0\\
1.89	0\\
1.9	0\\
1.91	0\\
1.92	0\\
1.93	0\\
1.94	0\\
1.95	0\\
1.96	0\\
1.97	0\\
1.98	0\\
1.99	0\\
2	0\\
2.01	0\\
2.02	0\\
2.03	0\\
2.04	0\\
2.05	0\\
2.06	0\\
2.07	0\\
2.08	0\\
2.09	0\\
2.1	0\\
2.11	0\\
2.12	0\\
2.13	0\\
2.14	0\\
2.15	0\\
2.16	0\\
2.17	0\\
2.18	0\\
2.19	0\\
2.2	0\\
2.21	0\\
2.22	0\\
2.23	0\\
2.24	0\\
2.25	0\\
2.26	0\\
2.27	0\\
2.28	0\\
2.29	0\\
2.3	0\\
2.31	0\\
2.32	0\\
2.33	0\\
2.34	0\\
2.35	0\\
2.36	0\\
2.37	0\\
2.38	0\\
2.39	0\\
2.4	0\\
2.41	0\\
2.42	0\\
2.43	0\\
2.44	0\\
2.45	0\\
2.46	0\\
2.47	0\\
2.48	0\\
2.49	0\\
2.5	0\\
2.51	0\\
2.52	0\\
2.53	0\\
2.54	0\\
2.55	0\\
2.56	0\\
2.57	0\\
2.58	0\\
2.59	0\\
2.6	0\\
2.61	0\\
2.62	0\\
2.63	0\\
2.64	0\\
2.65	0\\
2.66	0\\
2.67	0\\
2.68	0\\
2.69	0\\
2.7	0\\
2.71	0\\
2.72	0\\
2.73	0\\
2.74	0\\
2.75	0\\
2.76	0\\
2.77	0\\
2.78	0\\
2.79	0\\
2.8	0\\
2.81	0\\
2.82	0\\
2.83	0\\
2.84	0\\
2.85	0\\
2.86	0\\
2.87	0\\
2.88	0\\
2.89	0\\
2.9	0\\
2.91	0\\
2.92	0\\
2.93	0\\
2.94	0\\
2.95	0\\
2.96	0\\
2.97	0\\
2.98	0\\
2.99	0\\
3	0\\
3.01	0\\
3.02	0\\
3.03	0\\
3.04	0\\
3.05	0\\
3.06	0\\
3.07	0\\
3.08	0\\
3.09	0\\
3.1	0\\
3.11	0\\
3.12	0\\
3.13	0\\
3.14	0\\
3.15	0\\
3.16	0\\
3.17	0\\
3.18	0\\
3.19	0\\
3.2	0\\
3.21	0\\
3.22	0\\
3.23	0\\
3.24	0\\
3.25	0\\
3.26	0\\
3.27	0\\
3.28	0\\
3.29	0\\
3.3	0\\
3.31	0\\
3.32	0\\
3.33	0\\
3.34	0\\
3.35	0\\
3.36	0\\
3.37	0\\
3.38	0\\
3.39	0\\
3.4	0\\
3.41	0\\
3.42	0\\
3.43	0\\
3.44	0\\
3.45	0\\
3.46	0\\
3.47	0\\
3.48	0\\
3.49	0\\
3.5	0\\
3.51	0\\
3.52	0\\
3.53	0\\
3.54	0\\
3.55	0\\
3.56	0\\
3.57	0\\
3.58	0\\
3.59	0\\
3.6	0\\
3.61	0\\
3.62	0\\
3.63	0\\
3.64	0\\
3.65	0\\
3.66	0\\
3.67	0\\
3.68	0\\
3.69	0\\
3.7	0\\
3.71	0\\
3.72	0\\
3.73	0\\
3.74	0\\
3.75	0\\
3.76	0\\
3.77	0\\
3.78	0\\
3.79	0\\
3.8	0\\
3.81	0\\
3.82	0\\
3.83	0\\
3.84	0\\
3.85	0\\
3.86	0\\
3.87	0\\
3.88	0\\
3.89	0\\
3.9	0\\
3.91	0\\
3.92	0\\
3.93	0\\
3.94	0\\
3.95	0\\
3.96	0\\
3.97	0\\
3.98	0\\
3.99	0\\
4	0\\
4.01	0\\
4.02	0\\
4.03	0\\
4.04	0\\
4.05	0\\
4.06	0\\
4.07	0\\
4.08	0\\
4.09	0\\
4.1	0\\
4.11	0\\
4.12	0\\
4.13	0\\
4.14	0\\
4.15	0\\
4.16	0\\
4.17	0\\
4.18	0\\
4.19	0\\
4.2	0\\
4.21	0\\
4.22	0\\
4.23	0\\
4.24	0\\
4.25	0\\
4.26	0\\
4.27	0\\
4.28	0\\
4.29	0\\
4.3	0\\
4.31	0\\
4.32	0\\
4.33	0\\
4.34	0\\
4.35	0\\
4.36	0\\
4.37	0\\
4.38	0\\
4.39	0\\
4.4	0\\
4.41	0\\
4.42	0\\
4.43	0\\
4.44	0\\
4.45	0\\
4.46	0\\
4.47	0\\
4.48	0\\
4.49	0\\
4.5	0\\
4.51	0\\
4.52	0\\
4.53	0\\
4.54	0\\
4.55	0\\
4.56	0\\
4.57	0\\
4.58	0\\
4.59	0\\
4.6	0\\
4.61	0\\
4.62	0\\
4.63	0\\
4.64	0\\
4.65	0\\
4.66	0\\
4.67	0\\
4.68	0\\
4.69	0\\
4.7	0\\
4.71	0\\
4.72	0\\
4.73	0\\
4.74	0\\
4.75	0\\
4.76	0\\
4.77	0\\
4.78	0\\
4.79	0\\
4.8	0\\
4.81	0\\
4.82	0\\
4.83	0\\
4.84	0\\
4.85	0\\
4.86	0\\
4.87	0\\
4.88	0\\
4.89	0\\
4.9	0\\
4.91	0\\
4.92	0\\
4.93	0\\
4.94	0\\
4.95	0\\
4.96	0\\
4.97	0\\
4.98	0\\
4.99	0\\
5	0\\
5.01	0\\
5.02	0\\
5.03	0\\
5.04	0\\
5.05	0\\
5.06	0\\
5.07	0\\
5.08	0\\
5.09	0\\
5.1	0\\
5.11	0\\
5.12	0\\
5.13	0\\
5.14	0\\
5.15	0\\
5.16	0\\
5.17	0\\
5.18	0\\
5.19	0\\
5.2	0\\
5.21	0\\
5.22	0\\
5.23	0\\
5.24	0\\
5.25	0\\
5.26	0\\
5.27	0\\
5.28	0\\
5.29	0\\
5.3	0\\
5.31	0\\
5.32	0\\
5.33	0\\
5.34	0\\
5.35	0\\
5.36	0\\
5.37	0\\
5.38	0\\
5.39	0\\
5.4	0\\
5.41	0\\
5.42	0\\
5.43	0\\
5.44	0\\
5.45	0\\
5.46	0\\
5.47	0\\
5.48	0\\
5.49	0\\
5.5	0\\
5.51	0\\
5.52	0\\
5.53	0\\
5.54	0\\
5.55	0\\
5.56	0\\
5.57	0\\
5.58	0\\
5.59	0\\
5.6	0\\
5.61	0\\
5.62	0\\
5.63	0\\
5.64	0\\
5.65	0\\
5.66	0\\
5.67	0\\
5.68	0\\
5.69	0\\
5.7	0\\
5.71	0\\
5.72	0\\
5.73	0\\
5.74	0\\
5.75	0\\
5.76	0\\
5.77	0\\
5.78	0\\
5.79	0\\
5.8	0\\
5.81	0\\
5.82	0\\
5.83	0\\
5.84	0\\
5.85	0\\
5.86	0\\
5.87	0\\
5.88	0\\
5.89	0\\
5.9	0\\
5.91	0\\
5.92	0\\
5.93	0\\
5.94	0\\
5.95	0\\
5.96	0\\
5.97	0\\
5.98	0\\
5.99	0\\
6	0\\
6.01	0\\
6.02	0\\
6.03	0\\
6.04	0\\
6.05	0\\
6.06	0\\
6.07	0\\
6.08	0\\
6.09	0\\
6.1	0\\
6.11	0\\
6.12	0\\
6.13	0\\
6.14	0\\
6.15	0\\
6.16	0\\
6.17	0\\
6.18	0\\
6.19	0\\
6.2	0\\
6.21	0\\
6.22	0\\
6.23	0\\
6.24	0\\
6.25	0\\
6.26	0\\
6.27	0\\
6.28	0\\
6.29	0\\
6.3	0\\
6.31	0\\
6.32	0\\
6.33	0\\
6.34	0\\
6.35	0\\
6.36	0\\
6.37	0\\
6.38	0\\
6.39	0\\
6.4	0\\
6.41	0\\
6.42	0\\
6.43	0\\
6.44	0\\
6.45	0\\
6.46	0\\
6.47	0\\
6.48	0\\
6.49	0\\
6.5	0\\
6.51	0\\
6.52	0\\
6.53	0\\
6.54	0\\
6.55	0\\
6.56	0\\
6.57	0\\
6.58	0\\
6.59	0\\
6.6	0\\
6.61	0\\
6.62	0\\
6.63	0\\
6.64	0\\
6.65	0\\
6.66	0\\
6.67	0\\
6.68	0\\
6.69	0\\
6.7	0\\
6.71	0\\
6.72	0\\
6.73	0\\
6.74	0\\
6.75	0\\
6.76	0\\
6.77	0\\
6.78	0\\
6.79	0\\
6.8	0\\
6.81	0\\
6.82	0\\
6.83	0\\
6.84	0\\
6.85	0\\
6.86	0\\
6.87	0\\
6.88	0\\
6.89	0\\
6.9	0\\
6.91	0\\
6.92	0\\
6.93	0\\
6.94	0\\
6.95	0\\
6.96	0\\
6.97	0\\
6.98	0\\
6.99	0\\
7	0\\
7.01	0\\
7.02	0\\
7.03	0\\
7.04	0\\
7.05	0\\
7.06	0\\
7.07	0\\
7.08	0\\
7.09	0\\
7.1	0\\
7.11	0\\
7.12	0\\
7.13	0\\
7.14	0\\
7.15	0\\
7.16	0\\
7.17	0\\
7.18	0\\
7.19	0\\
7.2	0\\
7.21	0\\
7.22	0\\
7.23	0\\
7.24	0\\
7.25	0\\
7.26	0\\
7.27	0\\
7.28	0\\
7.29	0\\
7.3	0\\
7.31	0\\
7.32	0\\
7.33	0\\
7.34	0\\
7.35	0\\
7.36	0\\
7.37	0\\
7.38	0\\
7.39	0\\
7.4	0\\
7.41	0\\
7.42	0\\
7.43	0\\
7.44	0\\
7.45	0\\
7.46	0\\
7.47	0\\
7.48	0\\
7.49	0\\
7.5	0\\
7.51	0\\
7.52	0\\
7.53	0\\
7.54	0\\
7.55	0\\
7.56	0\\
7.57	0\\
7.58	0\\
7.59	0\\
7.6	0\\
7.61	0\\
7.62	0\\
7.63	0\\
7.64	0\\
7.65	0\\
7.66	0\\
7.67	0\\
7.68	0\\
7.69	0\\
7.7	0\\
7.71	0\\
7.72	0\\
7.73	0\\
7.74	0\\
7.75	0\\
7.76	0\\
7.77	0\\
7.78	0\\
7.79	0\\
7.8	0\\
7.81	0\\
7.82	0\\
7.83	0\\
7.84	0\\
7.85	0\\
7.86	0\\
7.87	0\\
7.88	0\\
7.89	0\\
7.9	0\\
7.91	0\\
7.92	0\\
7.93	0\\
7.94	0\\
7.95	0\\
7.96	0\\
7.97	0\\
7.98	0\\
7.99	0\\
8	0\\
8.01	0\\
8.02	0\\
8.03	0\\
8.04	0\\
8.05	0\\
8.06	0\\
8.07	0\\
8.08	0\\
8.09	0\\
8.1	0\\
8.11	0\\
8.12	0\\
8.13	0\\
8.14	0\\
8.15	0\\
8.16	0\\
8.17	0\\
8.18	0\\
8.19	0\\
8.2	0\\
8.21	0\\
8.22	0\\
8.23	0\\
8.24	0\\
8.25	0\\
8.26	0\\
8.27	0\\
8.28	0\\
8.29	0\\
8.3	0\\
8.31	0\\
8.32	0\\
8.33	0\\
8.34	0\\
8.35	0\\
8.36	0\\
8.37	0\\
8.38	0\\
8.39	0\\
8.4	0\\
8.41	0\\
8.42	0\\
8.43	0\\
8.44	0\\
8.45	0\\
8.46	0\\
8.47	0\\
8.48	0\\
8.49	0\\
8.5	0\\
8.51	0\\
8.52	0\\
8.53	0\\
8.54	0\\
8.55	0\\
8.56	0\\
8.57	0\\
8.58	0\\
8.59	0\\
8.6	0\\
8.61	0\\
8.62	0\\
8.63	0\\
8.64	0\\
8.65	0\\
8.66	0\\
8.67	0\\
8.68	0\\
8.69	0\\
8.7	0\\
8.71	0\\
8.72	0\\
8.73	0\\
8.74	0\\
8.75	0\\
8.76	0\\
8.77	0\\
8.78	0\\
8.79	0\\
8.8	0\\
8.81	0\\
8.82	0\\
8.83	0\\
8.84	0\\
8.85	0\\
8.86	0\\
8.87	0\\
8.88	0\\
8.89	0\\
8.9	0\\
8.91	0\\
8.92	0\\
8.93	0\\
8.94	0\\
8.95	0\\
8.96	0\\
8.97	0\\
8.98	0\\
8.99	0\\
9	0\\
9.01	0\\
9.02	0\\
9.03	0\\
9.04	0\\
9.05	0\\
9.06	0\\
9.07	0\\
9.08	0\\
9.09	0\\
9.1	0\\
9.11	0\\
9.12	0\\
9.13	0\\
9.14	0\\
9.15	0\\
9.16	0\\
9.17	0\\
9.18	0\\
9.19	0\\
9.2	0\\
9.21	0\\
9.22	0\\
9.23	0\\
9.24	0\\
9.25	0\\
9.26	0\\
9.27	0\\
9.28	0\\
9.29	0\\
9.3	0\\
9.31	0\\
9.32	0\\
9.33	0\\
9.34	0\\
9.35	0\\
9.36	0\\
9.37	0\\
9.38	0\\
9.39	0\\
9.4	0\\
9.41	0\\
9.42	0\\
9.43	0\\
9.44	0\\
9.45	0\\
9.46	0\\
9.47	0\\
9.48	0\\
9.49	0\\
9.5	0\\
9.51	0\\
9.52	0\\
9.53	0\\
9.54	0\\
9.55	0\\
9.56	0\\
9.57	0\\
9.58	0\\
9.59	0\\
9.6	0\\
9.61	0\\
9.62	0\\
9.63	0\\
9.64	0\\
9.65	0\\
9.66	0\\
9.67	0\\
9.68	0\\
9.69	0\\
9.7	0\\
9.71	0\\
9.72	0\\
9.73	0\\
9.74	0\\
9.75	0\\
9.76	0\\
9.77	0\\
9.78	0\\
9.79	0\\
9.8	0\\
9.81	0\\
9.82	0\\
9.83	0\\
9.84	0\\
9.85	0\\
9.86	0\\
9.87	0\\
9.88	0\\
9.89	0\\
9.9	0\\
9.91	0\\
9.92	0\\
9.93	0\\
9.94	0\\
9.95	0\\
9.96	0\\
9.97	0\\
9.98	0\\
9.99	0\\
10	0\\
10.01	0\\
10.02	0\\
10.03	0\\
10.04	0\\
10.05	0\\
10.06	0\\
10.07	0\\
10.08	0\\
10.09	0\\
10.1	0\\
10.11	0\\
10.12	0\\
10.13	0\\
10.14	0\\
10.15	0\\
10.16	0\\
10.17	0\\
10.18	0\\
10.19	0\\
10.2	0\\
10.21	0\\
10.22	0\\
10.23	0\\
10.24	0\\
10.25	0\\
10.26	0\\
10.27	0\\
10.28	0\\
10.29	0\\
10.3	0\\
10.31	0\\
10.32	0\\
10.33	0\\
10.34	0\\
10.35	0\\
10.36	0\\
10.37	0\\
10.38	0\\
10.39	0\\
10.4	0\\
10.41	0\\
10.42	0\\
10.43	0\\
10.44	0\\
10.45	0\\
10.46	0\\
10.47	0\\
10.48	0\\
10.49	0\\
10.5	0\\
10.51	0\\
10.52	0\\
10.53	0\\
10.54	0\\
10.55	0\\
10.56	0\\
10.57	0\\
10.58	0\\
10.59	0\\
10.6	0\\
10.61	0\\
10.62	0\\
10.63	0\\
10.64	0\\
10.65	0\\
10.66	0\\
10.67	0\\
10.68	0\\
10.69	0\\
10.7	0\\
10.71	0\\
10.72	0\\
10.73	0\\
10.74	0\\
10.75	0\\
10.76	0\\
10.77	0\\
10.78	0\\
10.79	0\\
10.8	0\\
10.81	0\\
10.82	0\\
10.83	0\\
10.84	0\\
10.85	0\\
10.86	0\\
10.87	0\\
10.88	0\\
10.89	0\\
10.9	0\\
10.91	0\\
10.92	0\\
10.93	0\\
10.94	0\\
10.95	0\\
10.96	0\\
10.97	0\\
10.98	0\\
10.99	0\\
11	0\\
11.01	0\\
11.02	0\\
11.03	0\\
11.04	0\\
11.05	0\\
11.06	0\\
11.07	0\\
11.08	0\\
11.09	0\\
11.1	0\\
11.11	0\\
11.12	0\\
11.13	0\\
11.14	0\\
11.15	0\\
11.16	0\\
11.17	0\\
11.18	0\\
11.19	0\\
11.2	0\\
11.21	0\\
11.22	0\\
11.23	0\\
11.24	0\\
11.25	0\\
11.26	0\\
11.27	0\\
11.28	0\\
11.29	0\\
11.3	0\\
11.31	0\\
11.32	0\\
11.33	0\\
11.34	0\\
11.35	0\\
11.36	0\\
11.37	0\\
11.38	0\\
11.39	0\\
11.4	0\\
11.41	0\\
11.42	0\\
11.43	0\\
11.44	0\\
11.45	0\\
11.46	0\\
11.47	0\\
11.48	0\\
11.49	0\\
11.5	0\\
11.51	0\\
11.52	0\\
11.53	0\\
11.54	0\\
11.55	0\\
11.56	0\\
11.57	0\\
11.58	0\\
11.59	0\\
11.6	0\\
11.61	0\\
11.62	0\\
11.63	0\\
11.64	0\\
11.65	0\\
11.66	0\\
11.67	0\\
11.68	0\\
11.69	0\\
11.7	0\\
11.71	0\\
11.72	0\\
11.73	0\\
11.74	0\\
11.75	0\\
11.76	0\\
11.77	0\\
11.78	0\\
11.79	0\\
11.8	0\\
11.81	0\\
11.82	0\\
11.83	0\\
11.84	0\\
11.85	0\\
11.86	0\\
11.87	0\\
11.88	0\\
11.89	0\\
11.9	0\\
11.91	0\\
11.92	0\\
11.93	0\\
11.94	0\\
11.95	0\\
11.96	0\\
11.97	0\\
11.98	0\\
11.99	0\\
12	0\\
12.01	0\\
12.02	0\\
12.03	0\\
12.04	0\\
12.05	0\\
12.06	0\\
12.07	0\\
12.08	0\\
12.09	0\\
12.1	0\\
12.11	0\\
12.12	0\\
12.13	0\\
12.14	0\\
12.15	0\\
12.16	0\\
12.17	0\\
12.18	0\\
12.19	0\\
12.2	0\\
12.21	0\\
12.22	0\\
12.23	0\\
12.24	0\\
12.25	0\\
12.26	0\\
12.27	0\\
12.28	0\\
12.29	0\\
12.3	0\\
12.31	0\\
12.32	0\\
12.33	0\\
12.34	0\\
12.35	0\\
12.36	0\\
12.37	0\\
12.38	0\\
12.39	0\\
12.4	0\\
12.41	0\\
12.42	0\\
12.43	0\\
12.44	0\\
12.45	0\\
12.46	0\\
12.47	0\\
12.48	0\\
12.49	0\\
12.5	0\\
12.51	0\\
12.52	0\\
12.53	0\\
12.54	0\\
12.55	0\\
12.56	0\\
12.57	0\\
12.58	0\\
12.59	0\\
12.6	0\\
12.61	0\\
12.62	0\\
12.63	0\\
12.64	0\\
12.65	0\\
12.66	0\\
12.67	0\\
12.68	0\\
12.69	0\\
12.7	0\\
12.71	0\\
12.72	0\\
12.73	0\\
12.74	0\\
12.75	0\\
12.76	0\\
12.77	0\\
12.78	0\\
12.79	0\\
12.8	0\\
12.81	0\\
12.82	0\\
12.83	0\\
12.84	0\\
12.85	0\\
12.86	0\\
12.87	0\\
12.88	0\\
12.89	0\\
12.9	0\\
12.91	0\\
12.92	0\\
12.93	0\\
12.94	0\\
12.95	0\\
12.96	0\\
12.97	0\\
12.98	0\\
12.99	0\\
13	0\\
13.01	0\\
13.02	0\\
13.03	0\\
13.04	0\\
13.05	0\\
13.06	0\\
13.07	0\\
13.08	0\\
13.09	0\\
13.1	0\\
13.11	0\\
13.12	0\\
13.13	0\\
13.14	0\\
13.15	0\\
13.16	0\\
13.17	0\\
13.18	0\\
13.19	0\\
13.2	0\\
13.21	0\\
13.22	0\\
13.23	0\\
13.24	0\\
13.25	0\\
13.26	0\\
13.27	0\\
13.28	0\\
13.29	0\\
13.3	0\\
13.31	0\\
13.32	0\\
13.33	0\\
13.34	0\\
13.35	0\\
13.36	0\\
13.37	0\\
13.38	0\\
13.39	0\\
13.4	0\\
13.41	0\\
13.42	0\\
13.43	0\\
13.44	0\\
13.45	0\\
13.46	0\\
13.47	0\\
13.48	0\\
13.49	0\\
13.5	0\\
13.51	0\\
13.52	0\\
13.53	0\\
13.54	0\\
13.55	0\\
13.56	0\\
13.57	0\\
13.58	0\\
13.59	0\\
13.6	0\\
13.61	0\\
13.62	0\\
13.63	0\\
13.64	0\\
13.65	0\\
13.66	0\\
13.67	0\\
13.68	0\\
13.69	0\\
13.7	0\\
13.71	0\\
13.72	0\\
13.73	0\\
13.74	0\\
13.75	0\\
13.76	0\\
13.77	0\\
13.78	0\\
13.79	0\\
13.8	0\\
13.81	0\\
13.82	0\\
13.83	0\\
13.84	0\\
13.85	0\\
13.86	0\\
13.87	0\\
13.88	0\\
13.89	0\\
13.9	0\\
13.91	0\\
13.92	0\\
13.93	0\\
13.94	0\\
13.95	0\\
13.96	0\\
13.97	0\\
13.98	0\\
13.99	0\\
14	0\\
14.01	0\\
14.02	0\\
14.03	0\\
14.04	0\\
14.05	0\\
14.06	0\\
14.07	0\\
14.08	0\\
14.09	0\\
14.1	0\\
14.11	0\\
14.12	0\\
14.13	0\\
14.14	0\\
14.15	0\\
14.16	0\\
14.17	0\\
14.18	0\\
14.19	0\\
14.2	0\\
14.21	0\\
14.22	0\\
14.23	0\\
14.24	0\\
14.25	0\\
14.26	0\\
14.27	0\\
14.28	0\\
14.29	0\\
14.3	0\\
14.31	0\\
14.32	0\\
14.33	0\\
14.34	0\\
14.35	0\\
14.36	0\\
14.37	0\\
14.38	0\\
14.39	0\\
14.4	0\\
14.41	0\\
14.42	0\\
14.43	0\\
14.44	0\\
14.45	0\\
14.46	0\\
14.47	0\\
14.48	0\\
14.49	0\\
14.5	0\\
14.51	0\\
14.52	0\\
14.53	0\\
14.54	0\\
14.55	0\\
14.56	0\\
14.57	0\\
14.58	0\\
14.59	0\\
14.6	0\\
14.61	0\\
14.62	0\\
14.63	0\\
14.64	0\\
14.65	0\\
14.66	0\\
14.67	0\\
14.68	0\\
14.69	0\\
14.7	0\\
14.71	0\\
14.72	0\\
14.73	0\\
14.74	0\\
14.75	0\\
14.76	0\\
14.77	0\\
14.78	0\\
14.79	0\\
14.8	0\\
14.81	0\\
14.82	0\\
14.83	0\\
14.84	0\\
14.85	0\\
14.86	0\\
14.87	0\\
14.88	0\\
14.89	0\\
14.9	0\\
14.91	0\\
14.92	0\\
14.93	0\\
14.94	0\\
14.95	0\\
14.96	0\\
14.97	0\\
14.98	0\\
14.99	0\\
15	0\\
15.01	0\\
15.02	0\\
15.03	0\\
15.04	0\\
15.05	0\\
15.06	0\\
15.07	0\\
15.08	0\\
15.09	0\\
15.1	0\\
15.11	0\\
15.12	0\\
15.13	0\\
15.14	0\\
15.15	0\\
15.16	0\\
15.17	0\\
15.18	0\\
15.19	0\\
15.2	0\\
15.21	0\\
15.22	0\\
15.23	0\\
15.24	0\\
15.25	0\\
15.26	0\\
15.27	0\\
15.28	0\\
15.29	0\\
15.3	0\\
15.31	0\\
15.32	0\\
15.33	0\\
15.34	0\\
15.35	0\\
15.36	0\\
15.37	0\\
15.38	0\\
15.39	0\\
15.4	0\\
15.41	0\\
15.42	0\\
15.43	0\\
15.44	0\\
15.45	0\\
15.46	0\\
15.47	0\\
15.48	0\\
15.49	0\\
15.5	0\\
15.51	0\\
15.52	0\\
15.53	0\\
15.54	0\\
15.55	0\\
15.56	0\\
15.57	0\\
15.58	0\\
15.59	0\\
15.6	0\\
15.61	0\\
15.62	0\\
15.63	0\\
15.64	0\\
15.65	0\\
15.66	0\\
15.67	0\\
15.68	0\\
15.69	0\\
15.7	0\\
15.71	0\\
15.72	0\\
15.73	0\\
15.74	0\\
15.75	0\\
15.76	0\\
15.77	0\\
15.78	0\\
15.79	0\\
15.8	0\\
15.81	0\\
15.82	0\\
15.83	0\\
15.84	0\\
15.85	0\\
15.86	0\\
15.87	0\\
15.88	0\\
15.89	0\\
15.9	0\\
15.91	0\\
15.92	0\\
15.93	0\\
15.94	0\\
15.95	0\\
15.96	0\\
15.97	0\\
15.98	0\\
15.99	0\\
16	0\\
16.01	0\\
16.02	0\\
16.03	0\\
16.04	0\\
16.05	0\\
16.06	0\\
16.07	0\\
16.08	0\\
16.09	0\\
16.1	0\\
16.11	0\\
16.12	0\\
16.13	0\\
16.14	0\\
16.15	0\\
16.16	0\\
16.17	0\\
16.18	0\\
16.19	0\\
16.2	0\\
16.21	0\\
16.22	0\\
16.23	0\\
16.24	0\\
16.25	0\\
16.26	0\\
16.27	0\\
16.28	0\\
16.29	0\\
16.3	0\\
16.31	0\\
16.32	0\\
16.33	0\\
16.34	0\\
16.35	0\\
16.36	0\\
16.37	0\\
16.38	0\\
16.39	0\\
16.4	0\\
16.41	0\\
16.42	0\\
16.43	0\\
16.44	0\\
16.45	0\\
16.46	0\\
16.47	0\\
16.48	0\\
16.49	0\\
16.5	0\\
16.51	0\\
16.52	0\\
16.53	0\\
16.54	0\\
16.55	0\\
16.56	0\\
16.57	0\\
16.58	0\\
16.59	0\\
16.6	0\\
16.61	0\\
16.62	0\\
16.63	0\\
16.64	0\\
16.65	0\\
16.66	0\\
16.67	0\\
16.68	0\\
16.69	0\\
16.7	0\\
16.71	0\\
16.72	0\\
16.73	0\\
16.74	0\\
16.75	0\\
16.76	0\\
16.77	0\\
16.78	0\\
16.79	0\\
16.8	0\\
16.81	0\\
16.82	0\\
16.83	0\\
16.84	0\\
16.85	0\\
16.86	0\\
16.87	0\\
16.88	0\\
16.89	0\\
16.9	0\\
16.91	0\\
16.92	0\\
16.93	0\\
16.94	0\\
16.95	0\\
16.96	0\\
16.97	0\\
16.98	0\\
16.99	0\\
17	0\\
17.01	0\\
17.02	0\\
17.03	0\\
17.04	0\\
17.05	0\\
17.06	0\\
17.07	0\\
17.08	0\\
17.09	0\\
17.1	0\\
17.11	0\\
17.12	0\\
17.13	0\\
17.14	0\\
17.15	0\\
17.16	0\\
17.17	0\\
17.18	0\\
17.19	0\\
17.2	0\\
17.21	0\\
17.22	0\\
17.23	0\\
17.24	0\\
17.25	0\\
17.26	0\\
17.27	0\\
17.28	0\\
17.29	0\\
17.3	0\\
17.31	0\\
17.32	0\\
17.33	0\\
17.34	0\\
17.35	0\\
17.36	0\\
17.37	0\\
17.38	0\\
17.39	0\\
17.4	0\\
17.41	0\\
17.42	0\\
17.43	0\\
17.44	0\\
17.45	0\\
17.46	0\\
17.47	0\\
17.48	0\\
17.49	0\\
17.5	0\\
17.51	0\\
17.52	0\\
17.53	0\\
17.54	0\\
17.55	0\\
17.56	0\\
17.57	0\\
17.58	0\\
17.59	0\\
17.6	0\\
17.61	0\\
17.62	0\\
17.63	0\\
17.64	0\\
17.65	0\\
17.66	0\\
17.67	0\\
17.68	0\\
17.69	0\\
17.7	0\\
17.71	0\\
17.72	0\\
17.73	0\\
17.74	0\\
17.75	0\\
17.76	0\\
17.77	0\\
17.78	0\\
17.79	0\\
17.8	0\\
17.81	0\\
17.82	0\\
17.83	0\\
17.84	0\\
17.85	0\\
17.86	0\\
17.87	0\\
17.88	0\\
17.89	0\\
17.9	0\\
17.91	0\\
17.92	0\\
17.93	0\\
17.94	0\\
17.95	0\\
17.96	0\\
17.97	0\\
17.98	0\\
17.99	0\\
18	0\\
18.01	0\\
18.02	0\\
18.03	0\\
18.04	0\\
18.05	0\\
18.06	0\\
18.07	0\\
18.08	0\\
18.09	0\\
18.1	0\\
18.11	0\\
18.12	0\\
18.13	0\\
18.14	0\\
18.15	0\\
18.16	0\\
18.17	0\\
18.18	0\\
18.19	0\\
18.2	0\\
18.21	0\\
18.22	0\\
18.23	0\\
18.24	0\\
18.25	0\\
18.26	0\\
18.27	0\\
18.28	0\\
18.29	0\\
18.3	0\\
18.31	0\\
18.32	0\\
18.33	0\\
18.34	0\\
18.35	0\\
18.36	0\\
18.37	0\\
18.38	0\\
18.39	0\\
18.4	0\\
18.41	0\\
18.42	0\\
18.43	0\\
18.44	0\\
18.45	0\\
18.46	0\\
18.47	0\\
18.48	0\\
18.49	0\\
18.5	0\\
18.51	0\\
18.52	0\\
18.53	0\\
18.54	0\\
18.55	0\\
18.56	0\\
18.57	0\\
18.58	0\\
18.59	0\\
18.6	0\\
18.61	0\\
18.62	0\\
18.63	0\\
18.64	0\\
18.65	0\\
18.66	0\\
18.67	0\\
18.68	0\\
18.69	0\\
18.7	0\\
18.71	0\\
18.72	0\\
18.73	0\\
18.74	0\\
18.75	0\\
18.76	0\\
18.77	0\\
18.78	0\\
18.79	0\\
18.8	0\\
18.81	0\\
18.82	0\\
18.83	0\\
18.84	0\\
18.85	0\\
18.86	0\\
18.87	0\\
18.88	0\\
18.89	0\\
18.9	0\\
18.91	0\\
18.92	0\\
18.93	0\\
18.94	0\\
18.95	0\\
18.96	0\\
18.97	0\\
18.98	0\\
18.99	0\\
19	0\\
19.01	0\\
19.02	0\\
19.03	0\\
19.04	0\\
19.05	0\\
19.06	0\\
19.07	0\\
19.08	0\\
19.09	0\\
19.1	0\\
19.11	0\\
19.12	0\\
19.13	0\\
19.14	0\\
19.15	0\\
19.16	0\\
19.17	0\\
19.18	0\\
19.19	0\\
19.2	0\\
19.21	0\\
19.22	0\\
19.23	0\\
19.24	0\\
19.25	0\\
19.26	0\\
19.27	0\\
19.28	0\\
19.29	0\\
19.3	0\\
19.31	0\\
19.32	0\\
19.33	0\\
19.34	0\\
19.35	0\\
19.36	0\\
19.37	0\\
19.38	0\\
19.39	0\\
19.4	0\\
19.41	0\\
19.42	0\\
19.43	0\\
19.44	0\\
19.45	0\\
19.46	0\\
19.47	0\\
19.48	0\\
19.49	0\\
19.5	0\\
19.51	0\\
19.52	0\\
19.53	0\\
19.54	0\\
19.55	0\\
19.56	0\\
19.57	0\\
19.58	0\\
19.59	0\\
19.6	0\\
19.61	0\\
19.62	0\\
19.63	0\\
19.64	0\\
19.65	0\\
19.66	0\\
19.67	0\\
19.68	0\\
19.69	0\\
19.7	0\\
19.71	0\\
19.72	0\\
19.73	0\\
19.74	0\\
19.75	0\\
19.76	0\\
19.77	0\\
19.78	0\\
19.79	0\\
19.8	0\\
19.81	0\\
19.82	0\\
19.83	0\\
19.84	0\\
19.85	0\\
19.86	0\\
19.87	0\\
19.88	0\\
19.89	0\\
19.9	0\\
19.91	0\\
19.92	0\\
19.93	0\\
19.94	0\\
19.95	0\\
19.96	0\\
19.97	0\\
19.98	0\\
19.99	0\\
20	0\\
20.01	0\\
20.02	0\\
20.03	0\\
20.04	0\\
20.05	0\\
20.06	0\\
20.07	0\\
20.08	0\\
20.09	0\\
20.1	0\\
20.11	0\\
20.12	0\\
20.13	0\\
20.14	0\\
20.15	0\\
20.16	0\\
20.17	0\\
20.18	0\\
20.19	0\\
20.2	0\\
20.21	0\\
20.22	0\\
20.23	0\\
20.24	0\\
20.25	0\\
20.26	0\\
20.27	0\\
20.28	0\\
20.29	0\\
20.3	0\\
20.31	0\\
20.32	0\\
20.33	0\\
20.34	0\\
20.35	0\\
20.36	0\\
20.37	0\\
20.38	0\\
20.39	0\\
20.4	0\\
20.41	0\\
20.42	0\\
20.43	0\\
20.44	0\\
20.45	0\\
20.46	0\\
20.47	0\\
20.48	0\\
20.49	0\\
20.5	0\\
20.51	0\\
20.52	0\\
20.53	0\\
20.54	0\\
20.55	0\\
20.56	0\\
20.57	0\\
20.58	0\\
20.59	0\\
20.6	0\\
20.61	0\\
20.62	0\\
20.63	0\\
20.64	0\\
20.65	0\\
20.66	0\\
20.67	0\\
20.68	0\\
20.69	0\\
20.7	0\\
20.71	0\\
20.72	0\\
20.73	0\\
20.74	0\\
20.75	0\\
20.76	0\\
20.77	0\\
20.78	0\\
20.79	0\\
20.8	0\\
20.81	0\\
20.82	0\\
20.83	0\\
20.84	0\\
20.85	0\\
20.86	0\\
20.87	0\\
20.88	0\\
20.89	0\\
20.9	0\\
20.91	0\\
20.92	0\\
20.93	0\\
20.94	0\\
20.95	0\\
20.96	0\\
20.97	0\\
20.98	0\\
20.99	0\\
21	0\\
21.01	0\\
21.02	0\\
21.03	0\\
21.04	0\\
21.05	0\\
21.06	0\\
21.07	0\\
21.08	0\\
21.09	0\\
21.1	0\\
21.11	0\\
21.12	0\\
21.13	0\\
21.14	0\\
21.15	0\\
21.16	0\\
21.17	0\\
21.18	0\\
21.19	0\\
21.2	0\\
21.21	0\\
21.22	0\\
21.23	0\\
21.24	0\\
21.25	0\\
21.26	0\\
21.27	0\\
21.28	0\\
21.29	0\\
21.3	0\\
21.31	0\\
21.32	0\\
21.33	0\\
21.34	0\\
21.35	0\\
21.36	0\\
21.37	0\\
21.38	0\\
21.39	0\\
21.4	0\\
21.41	0\\
21.42	0\\
21.43	0\\
21.44	0\\
21.45	0\\
21.46	0\\
21.47	0\\
21.48	0\\
21.49	0\\
21.5	0\\
21.51	0\\
21.52	0\\
21.53	0\\
21.54	0\\
21.55	0\\
21.56	0\\
21.57	0\\
21.58	0\\
21.59	0\\
21.6	0\\
21.61	0\\
21.62	0\\
21.63	0\\
21.64	0\\
21.65	0\\
21.66	0\\
21.67	0\\
21.68	0\\
21.69	0\\
21.7	0\\
21.71	0\\
21.72	0\\
21.73	0\\
21.74	0\\
21.75	0\\
21.76	0\\
21.77	0\\
21.78	0\\
21.79	0\\
21.8	0\\
21.81	0\\
21.82	0\\
21.83	0\\
21.84	0\\
21.85	0\\
21.86	0\\
21.87	0\\
21.88	0\\
21.89	0\\
21.9	0\\
21.91	0\\
21.92	0\\
21.93	0\\
21.94	0\\
21.95	0\\
21.96	0\\
21.97	0\\
21.98	0\\
21.99	0\\
22	0\\
22.01	0\\
22.02	0\\
22.03	0\\
22.04	0\\
22.05	0\\
22.06	0\\
22.07	0\\
22.08	0\\
22.09	0\\
22.1	0\\
22.11	0\\
22.12	0\\
22.13	0\\
22.14	0\\
22.15	0\\
22.16	0\\
22.17	0\\
22.18	0\\
22.19	0\\
22.2	0\\
22.21	0\\
22.22	0\\
22.23	0\\
22.24	0\\
22.25	0\\
22.26	0\\
22.27	0\\
22.28	0\\
22.29	0\\
22.3	0\\
22.31	0\\
22.32	0\\
22.33	0\\
22.34	0\\
22.35	0\\
22.36	0\\
22.37	0\\
22.38	0\\
22.39	0\\
22.4	0\\
22.41	0\\
22.42	0\\
22.43	0\\
22.44	0\\
22.45	0\\
22.46	0\\
22.47	0\\
22.48	0\\
22.49	0\\
22.5	0\\
22.51	0\\
22.52	0\\
22.53	0\\
22.54	0\\
22.55	0\\
22.56	0\\
22.57	0\\
22.58	0\\
22.59	0\\
22.6	0\\
22.61	0\\
22.62	0\\
22.63	0\\
22.64	0\\
22.65	0\\
22.66	0\\
22.67	0\\
22.68	0\\
22.69	0\\
22.7	0\\
22.71	0\\
22.72	0\\
22.73	0\\
22.74	0\\
22.75	0\\
22.76	0\\
22.77	0\\
22.78	0\\
22.79	0\\
22.8	0\\
22.81	0\\
22.82	0\\
22.83	0\\
22.84	0\\
22.85	0\\
22.86	0\\
22.87	0\\
22.88	0\\
22.89	0\\
22.9	0\\
22.91	0\\
22.92	0\\
22.93	0\\
22.94	0\\
22.95	0\\
22.96	0\\
22.97	0\\
22.98	0\\
22.99	0\\
23	0\\
23.01	0\\
23.02	0\\
23.03	0\\
23.04	0\\
23.05	0\\
23.06	0\\
23.07	0\\
23.08	0\\
23.09	0\\
23.1	0\\
23.11	0\\
23.12	0\\
23.13	0\\
23.14	0\\
23.15	0\\
23.16	0\\
23.17	0\\
23.18	0\\
23.19	0\\
23.2	0\\
23.21	0\\
23.22	0\\
23.23	0\\
23.24	0\\
23.25	0\\
23.26	0\\
23.27	0\\
23.28	0\\
23.29	0\\
23.3	0\\
23.31	0\\
23.32	0\\
23.33	0\\
23.34	0\\
23.35	0\\
23.36	0\\
23.37	0\\
23.38	0\\
23.39	0\\
23.4	0\\
23.41	0\\
23.42	0\\
23.43	0\\
23.44	0\\
23.45	0\\
23.46	0\\
23.47	0\\
23.48	0\\
23.49	0\\
23.5	0\\
23.51	0\\
23.52	0\\
23.53	0\\
23.54	0\\
23.55	0\\
23.56	0\\
23.57	0\\
23.58	0\\
23.59	0\\
23.6	0\\
23.61	0\\
23.62	0\\
23.63	0\\
23.64	0\\
23.65	0\\
23.66	0\\
23.67	0\\
23.68	0\\
23.69	0\\
23.7	0\\
23.71	0\\
23.72	0\\
23.73	0\\
23.74	0\\
23.75	0\\
23.76	0\\
23.77	0\\
23.78	0\\
23.79	0\\
23.8	0\\
23.81	0\\
23.82	0\\
23.83	0\\
23.84	0\\
23.85	0\\
23.86	0\\
23.87	0\\
23.88	0\\
23.89	0\\
23.9	0\\
23.91	0\\
23.92	0\\
23.93	0\\
23.94	0\\
23.95	0\\
23.96	0\\
23.97	0\\
23.98	0\\
23.99	0\\
24	0\\
24.01	0\\
24.02	0\\
24.03	0\\
24.04	0\\
24.05	0\\
24.06	0\\
24.07	0\\
24.08	0\\
24.09	0\\
24.1	0\\
24.11	0\\
24.12	0\\
24.13	0\\
24.14	0\\
24.15	0\\
24.16	0\\
24.17	0\\
24.18	0\\
24.19	0\\
24.2	0\\
24.21	0\\
24.22	0\\
24.23	0\\
24.24	0\\
24.25	0\\
24.26	0\\
24.27	0\\
24.28	0\\
24.29	0\\
24.3	0\\
24.31	0\\
24.32	0\\
24.33	0\\
24.34	0\\
24.35	0\\
24.36	0\\
24.37	0\\
24.38	0\\
24.39	0\\
24.4	0\\
24.41	0\\
24.42	0\\
24.43	0\\
24.44	0\\
24.45	0\\
24.46	0\\
24.47	0\\
24.48	0\\
24.49	0\\
24.5	0\\
24.51	0\\
24.52	0\\
24.53	0\\
24.54	0\\
24.55	0\\
24.56	0\\
24.57	0\\
24.58	0\\
24.59	0\\
24.6	0\\
24.61	0\\
24.62	0\\
24.63	0\\
24.64	0\\
24.65	0\\
24.66	0\\
24.67	0\\
24.68	0\\
24.69	0\\
24.7	0\\
24.71	0\\
24.72	0\\
24.73	0\\
24.74	0\\
24.75	0\\
24.76	0\\
24.77	0\\
24.78	0\\
24.79	0\\
24.8	0\\
24.81	0\\
24.82	0\\
24.83	0\\
24.84	0\\
24.85	0\\
24.86	0\\
24.87	0\\
24.88	0\\
24.89	0\\
24.9	0\\
24.91	0\\
24.92	0\\
24.93	0\\
24.94	0\\
24.95	0\\
24.96	0\\
24.97	0\\
24.98	0\\
24.99	0\\
25	0\\
25.01	0\\
25.02	0\\
25.03	0\\
25.04	0\\
25.05	0\\
25.06	0\\
25.07	0\\
25.08	0\\
25.09	0\\
25.1	0\\
25.11	0\\
25.12	0\\
25.13	0\\
25.14	0\\
25.15	0\\
25.16	0\\
25.17	0\\
25.18	0\\
25.19	0\\
25.2	0\\
25.21	0\\
25.22	0\\
25.23	0\\
25.24	0\\
25.25	0\\
25.26	0\\
25.27	0\\
25.28	0\\
25.29	0\\
25.3	0\\
25.31	0\\
25.32	0\\
25.33	0\\
25.34	0\\
25.35	0\\
25.36	0\\
25.37	0\\
25.38	0\\
25.39	0\\
25.4	0\\
25.41	0\\
25.42	0\\
25.43	0\\
25.44	0\\
25.45	0\\
25.46	0\\
25.47	0\\
25.48	0\\
25.49	0\\
25.5	0\\
25.51	0\\
25.52	0\\
25.53	0\\
25.54	0\\
25.55	0\\
25.56	0\\
25.57	0\\
25.58	0\\
25.59	0\\
25.6	0\\
25.61	0\\
25.62	0\\
25.63	0\\
25.64	0\\
25.65	0\\
25.66	0\\
25.67	0\\
25.68	0\\
25.69	0\\
25.7	0\\
25.71	0\\
25.72	0\\
25.73	0\\
25.74	0\\
25.75	0\\
25.76	0\\
25.77	0\\
25.78	0\\
25.79	0\\
25.8	0\\
25.81	0\\
25.82	0\\
25.83	0\\
25.84	0\\
25.85	0\\
25.86	0\\
25.87	0\\
25.88	0\\
25.89	0\\
25.9	0\\
25.91	0\\
25.92	0\\
25.93	0\\
25.94	0\\
25.95	0\\
25.96	0\\
25.97	0\\
25.98	0\\
25.99	0\\
26	0\\
26.01	0\\
26.02	0\\
26.03	0\\
26.04	0\\
26.05	0\\
26.06	0\\
26.07	0\\
26.08	0\\
26.09	0\\
26.1	0\\
26.11	0\\
26.12	0\\
26.13	0\\
26.14	0\\
26.15	0\\
26.16	0\\
26.17	0\\
26.18	0\\
26.19	0\\
26.2	0\\
26.21	0\\
26.22	0\\
26.23	0\\
26.24	0\\
26.25	0\\
26.26	0\\
26.27	0\\
26.28	0\\
26.29	0\\
26.3	0\\
26.31	0\\
26.32	0\\
26.33	0\\
26.34	0\\
26.35	0\\
26.36	0\\
26.37	0\\
26.38	0\\
26.39	0\\
26.4	0\\
26.41	0\\
26.42	0\\
26.43	0\\
26.44	0\\
26.45	0\\
26.46	0\\
26.47	0\\
26.48	0\\
26.49	0\\
26.5	0\\
26.51	0\\
26.52	0\\
26.53	0\\
26.54	0\\
26.55	0\\
26.56	0\\
26.57	0\\
26.58	0\\
26.59	0\\
26.6	0\\
26.61	0\\
26.62	0\\
26.63	0\\
26.64	0\\
26.65	0\\
26.66	0\\
26.67	0\\
26.68	0\\
26.69	0\\
26.7	0\\
26.71	0\\
26.72	0\\
26.73	0\\
26.74	0\\
26.75	0\\
26.76	0\\
26.77	0\\
26.78	0\\
26.79	0\\
26.8	0\\
26.81	0\\
26.82	0\\
26.83	0\\
26.84	0\\
26.85	0\\
26.86	0\\
26.87	0\\
26.88	0\\
26.89	0\\
26.9	0\\
26.91	0\\
26.92	0\\
26.93	0\\
26.94	0\\
26.95	0\\
26.96	0\\
26.97	0\\
26.98	0\\
26.99	0\\
27	0\\
27.01	0\\
27.02	0\\
27.03	0\\
27.04	0\\
27.05	0\\
27.06	0\\
27.07	0\\
27.08	0\\
27.09	0\\
27.1	0\\
27.11	0\\
27.12	0\\
27.13	0\\
27.14	0\\
27.15	0\\
27.16	0\\
27.17	0\\
27.18	0\\
27.19	0\\
27.2	0\\
27.21	0\\
27.22	0\\
27.23	0\\
27.24	0\\
27.25	0\\
27.26	0\\
27.27	0\\
27.28	0\\
27.29	0\\
27.3	0\\
27.31	0\\
27.32	0\\
27.33	0\\
27.34	0\\
27.35	0\\
27.36	0\\
27.37	0\\
27.38	0\\
27.39	0\\
27.4	0\\
27.41	0\\
27.42	0\\
27.43	0\\
27.44	0\\
27.45	0\\
27.46	0\\
27.47	0\\
27.48	0\\
27.49	0\\
27.5	0\\
27.51	0\\
27.52	0\\
27.53	0\\
27.54	0\\
27.55	0\\
27.56	0\\
27.57	0\\
27.58	0\\
27.59	0\\
27.6	0\\
27.61	0\\
27.62	0\\
27.63	0\\
27.64	0\\
27.65	0\\
27.66	0\\
27.67	0\\
27.68	0\\
27.69	0\\
27.7	0\\
27.71	0\\
27.72	0\\
27.73	0\\
27.74	0\\
27.75	0\\
27.76	0\\
27.77	0\\
27.78	0\\
27.79	0\\
27.8	0\\
27.81	0\\
27.82	0\\
27.83	0\\
27.84	0\\
27.85	0\\
27.86	0\\
27.87	0\\
27.88	0\\
27.89	0\\
27.9	0\\
27.91	0\\
27.92	0\\
27.93	0\\
27.94	0\\
27.95	0\\
27.96	0\\
27.97	0\\
27.98	0\\
27.99	0\\
28	0\\
28.01	0\\
28.02	0\\
28.03	0\\
28.04	0\\
28.05	0\\
28.06	0\\
28.07	0\\
28.08	0\\
28.09	0\\
28.1	0\\
28.11	0\\
28.12	0\\
28.13	0\\
28.14	0\\
28.15	0\\
28.16	0\\
28.17	0\\
28.18	0\\
28.19	0\\
28.2	0\\
28.21	0\\
28.22	0\\
28.23	0\\
28.24	0\\
28.25	0\\
28.26	0\\
28.27	0\\
28.28	0\\
28.29	0\\
28.3	0\\
28.31	0\\
28.32	0\\
28.33	0\\
28.34	0\\
28.35	0\\
28.36	0\\
28.37	0\\
28.38	0\\
28.39	0\\
28.4	0\\
28.41	0\\
28.42	0\\
28.43	0\\
28.44	0\\
28.45	0\\
28.46	0\\
28.47	0\\
28.48	0\\
28.49	0\\
28.5	0\\
28.51	0\\
28.52	0\\
28.53	0\\
28.54	0\\
28.55	0\\
28.56	0\\
28.57	0\\
28.58	0\\
28.59	0\\
28.6	0\\
28.61	0\\
28.62	0\\
28.63	0\\
28.64	0\\
28.65	0\\
28.66	0\\
28.67	0\\
28.68	0\\
28.69	0\\
28.7	0\\
28.71	0\\
28.72	0\\
28.73	0\\
28.74	0\\
28.75	0\\
28.76	0\\
28.77	0\\
28.78	0\\
28.79	0\\
28.8	0\\
28.81	0\\
28.82	0\\
28.83	0\\
28.84	0\\
28.85	0\\
28.86	0\\
28.87	0\\
28.88	0\\
28.89	0\\
28.9	0\\
28.91	0\\
28.92	0\\
28.93	0\\
28.94	0\\
28.95	0\\
28.96	0\\
28.97	0\\
28.98	0\\
28.99	0\\
29	0\\
29.01	0\\
29.02	0\\
29.03	0\\
29.04	0\\
29.05	0\\
29.06	0\\
29.07	0\\
29.08	0\\
29.09	0\\
29.1	0\\
29.11	0\\
29.12	0\\
29.13	0\\
29.14	0\\
29.15	0\\
29.16	0\\
29.17	0\\
29.18	0\\
29.19	0\\
29.2	0\\
29.21	0\\
29.22	0\\
29.23	0\\
29.24	0\\
29.25	0\\
29.26	0\\
29.27	0\\
29.28	0\\
29.29	0\\
29.3	0\\
29.31	0\\
29.32	0\\
29.33	0\\
29.34	0\\
29.35	0\\
29.36	0\\
29.37	0\\
29.38	0\\
29.39	0\\
29.4	0\\
29.41	0\\
29.42	0\\
29.43	0\\
29.44	0\\
29.45	0\\
29.46	0\\
29.47	0\\
29.48	0\\
29.49	0\\
29.5	0\\
29.51	0\\
29.52	0\\
29.53	0\\
29.54	0\\
29.55	0\\
29.56	0\\
29.57	0\\
29.58	0\\
29.59	0\\
29.6	0\\
29.61	0\\
29.62	0\\
29.63	0\\
29.64	0\\
29.65	0\\
29.66	0\\
29.67	0\\
29.68	0\\
29.69	0\\
29.7	0\\
29.71	0\\
29.72	0\\
29.73	0\\
29.74	0\\
29.75	0\\
29.76	0\\
29.77	0\\
29.78	0\\
29.79	0\\
29.8	0\\
29.81	0\\
29.82	0\\
29.83	0\\
29.84	0\\
29.85	0\\
29.86	0\\
29.87	0\\
29.88	0\\
29.89	0\\
29.9	0\\
29.91	0\\
29.92	0\\
29.93	0\\
29.94	0\\
29.95	0\\
29.96	0\\
29.97	0\\
29.98	0\\
29.99	0\\
30	0\\
30.01	0\\
30.02	0\\
30.03	0\\
30.04	0\\
30.05	0\\
30.06	0\\
30.07	0\\
30.08	0\\
30.09	0\\
30.1	0\\
30.11	0\\
30.12	0\\
30.13	0\\
30.14	0\\
30.15	0\\
30.16	0\\
30.17	0\\
30.18	0\\
30.19	0\\
30.2	0\\
30.21	0\\
30.22	0\\
30.23	0\\
30.24	0\\
30.25	0\\
30.26	0\\
30.27	0\\
30.28	0\\
30.29	0\\
30.3	0\\
30.31	0\\
30.32	0\\
30.33	0\\
30.34	0\\
30.35	0\\
30.36	0\\
30.37	0\\
30.38	0\\
30.39	0\\
30.4	0\\
30.41	0\\
30.42	0\\
30.43	0\\
30.44	0\\
30.45	0\\
30.46	0\\
30.47	0\\
30.48	0\\
30.49	0\\
30.5	0\\
30.51	0\\
30.52	0\\
30.53	0\\
30.54	0\\
30.55	0\\
30.56	0\\
30.57	0\\
30.58	0\\
30.59	0\\
30.6	0\\
30.61	0\\
30.62	0\\
30.63	0\\
30.64	0\\
30.65	0\\
30.66	0\\
30.67	0\\
30.68	0\\
30.69	0\\
30.7	0\\
30.71	0\\
30.72	0\\
30.73	0\\
30.74	0\\
30.75	0\\
30.76	0\\
30.77	0\\
30.78	0\\
30.79	0\\
30.8	0\\
30.81	0\\
30.82	0\\
30.83	0\\
30.84	0\\
30.85	0\\
30.86	0\\
30.87	0\\
30.88	0\\
30.89	0\\
30.9	0\\
30.91	0\\
30.92	0\\
30.93	0\\
30.94	0\\
30.95	0\\
30.96	0\\
30.97	0\\
30.98	0\\
30.99	0\\
31	0\\
31.01	0\\
31.02	0\\
31.03	0\\
31.04	0\\
31.05	0\\
31.06	0\\
31.07	0\\
31.08	0\\
31.09	0\\
31.1	0\\
31.11	0\\
31.12	0\\
31.13	0\\
31.14	0\\
31.15	0\\
31.16	0\\
31.17	0\\
31.18	0\\
31.19	0\\
31.2	0\\
31.21	0\\
31.22	0\\
31.23	0\\
31.24	0\\
31.25	0\\
31.26	0\\
31.27	0\\
31.28	0\\
31.29	0\\
31.3	0\\
31.31	0\\
31.32	0\\
31.33	0\\
31.34	0\\
31.35	0\\
31.36	0\\
31.37	0\\
31.38	0\\
31.39	0\\
31.4	0\\
31.41	0\\
31.42	0\\
31.43	0\\
31.44	0\\
31.45	0\\
31.46	0\\
31.47	0\\
31.48	0\\
31.49	0\\
31.5	0\\
31.51	0\\
31.52	0\\
31.53	0\\
31.54	0\\
31.55	0\\
31.56	0\\
31.57	0\\
31.58	0\\
31.59	0\\
31.6	0\\
31.61	0\\
31.62	0\\
31.63	0\\
31.64	0\\
31.65	0\\
31.66	0\\
31.67	0\\
31.68	0\\
31.69	0\\
31.7	0\\
31.71	0\\
31.72	0\\
31.73	0\\
31.74	0\\
31.75	0\\
31.76	0\\
31.77	0\\
31.78	0\\
31.79	0\\
31.8	0\\
31.81	0\\
31.82	0\\
31.83	0\\
31.84	0\\
31.85	0\\
31.86	0\\
31.87	0\\
31.88	0\\
31.89	0\\
31.9	0\\
31.91	0\\
31.92	0\\
31.93	0\\
31.94	0\\
31.95	0\\
31.96	0\\
31.97	0\\
31.98	0\\
31.99	0\\
32	0\\
32.01	0\\
32.02	0\\
32.03	0\\
32.04	0\\
32.05	0\\
32.06	0\\
32.07	0\\
32.08	0\\
32.09	0\\
32.1	0\\
32.11	0\\
32.12	0\\
32.13	0\\
32.14	0\\
32.15	0\\
32.16	0\\
32.17	0\\
32.18	0\\
32.19	0\\
32.2	0\\
32.21	0\\
32.22	0\\
32.23	0\\
32.24	0\\
32.25	0\\
32.26	0\\
32.27	0\\
32.28	0\\
32.29	0\\
32.3	0\\
32.31	0\\
32.32	0\\
32.33	0\\
32.34	0\\
32.35	0\\
32.36	0\\
32.37	0\\
32.38	0\\
32.39	0\\
32.4	0\\
32.41	0\\
32.42	0\\
32.43	0\\
32.44	0\\
32.45	0\\
32.46	0\\
32.47	0\\
32.48	0\\
32.49	0\\
32.5	0\\
32.51	0\\
32.52	0\\
32.53	0\\
32.54	0\\
32.55	0\\
32.56	0\\
32.57	0\\
32.58	0\\
32.59	0\\
32.6	0\\
32.61	0\\
32.62	0\\
32.63	0\\
32.64	0\\
32.65	0\\
32.66	0\\
32.67	0\\
32.68	0\\
32.69	0\\
32.7	0\\
32.71	0\\
32.72	0\\
32.73	0\\
32.74	0\\
32.75	0\\
32.76	0\\
32.77	0\\
32.78	0\\
32.79	0\\
32.8	0\\
32.81	0\\
32.82	0\\
32.83	0\\
32.84	0\\
32.85	0\\
32.86	0\\
32.87	0\\
32.88	0\\
32.89	0\\
32.9	0\\
32.91	0\\
32.92	0\\
32.93	0\\
32.94	0\\
32.95	0\\
32.96	0\\
32.97	0\\
32.98	0\\
32.99	0\\
33	0\\
33.01	0\\
33.02	0\\
33.03	0\\
33.04	0\\
33.05	0\\
33.06	0\\
33.07	0\\
33.08	0\\
33.09	0\\
33.1	0\\
33.11	0\\
33.12	0\\
33.13	0\\
33.14	0\\
33.15	0\\
33.16	0\\
33.17	0\\
33.18	0\\
33.19	0\\
33.2	0\\
33.21	0\\
33.22	0\\
33.23	0\\
33.24	0\\
33.25	0\\
33.26	0\\
33.27	0\\
33.28	0\\
33.29	0\\
33.3	0\\
33.31	0\\
33.32	0\\
33.33	0\\
33.34	0\\
33.35	0\\
33.36	0\\
33.37	0\\
33.38	0\\
33.39	0\\
33.4	0\\
33.41	0\\
33.42	0\\
33.43	0\\
33.44	0\\
33.45	0\\
33.46	0\\
33.47	0\\
33.48	0\\
33.49	0\\
33.5	0\\
33.51	0\\
33.52	0\\
33.53	0\\
33.54	0\\
33.55	0\\
33.56	0\\
33.57	0\\
33.58	0\\
33.59	0\\
33.6	0\\
33.61	0\\
33.62	0\\
33.63	0\\
33.64	0\\
33.65	0\\
33.66	0\\
33.67	0\\
33.68	0\\
33.69	0\\
33.7	0\\
33.71	0\\
33.72	0\\
33.73	0\\
33.74	0\\
33.75	0\\
33.76	0\\
33.77	0\\
33.78	0\\
33.79	0\\
33.8	0\\
33.81	0\\
33.82	0\\
33.83	0\\
33.84	0\\
33.85	0\\
33.86	0\\
33.87	0\\
33.88	0\\
33.89	0\\
33.9	0\\
33.91	0\\
33.92	0\\
33.93	0\\
33.94	0\\
33.95	0\\
33.96	0\\
33.97	0\\
33.98	0\\
33.99	0\\
34	0\\
34.01	0\\
34.02	0\\
34.03	0\\
34.04	0\\
34.05	0\\
34.06	0\\
34.07	0\\
34.08	0\\
34.09	0\\
34.1	0\\
34.11	0\\
34.12	0\\
34.13	0\\
34.14	0\\
34.15	0\\
34.16	0\\
34.17	0\\
34.18	0\\
34.19	0\\
34.2	0\\
34.21	0\\
34.22	0\\
34.23	0\\
34.24	0\\
34.25	0\\
34.26	0\\
34.27	0\\
34.28	0\\
34.29	0\\
34.3	0\\
34.31	0\\
34.32	0\\
34.33	0\\
34.34	0\\
34.35	0\\
34.36	0\\
34.37	0\\
34.38	0\\
34.39	0\\
34.4	0\\
34.41	0\\
34.42	0\\
34.43	0\\
34.44	0\\
34.45	0\\
34.46	0\\
34.47	0\\
34.48	0\\
34.49	0\\
34.5	0\\
34.51	0\\
34.52	0\\
34.53	0\\
34.54	0\\
34.55	0\\
34.56	0\\
34.57	0\\
34.58	0\\
34.59	0\\
34.6	0\\
34.61	0\\
34.62	0\\
34.63	0\\
34.64	0\\
34.65	0\\
34.66	0\\
34.67	0\\
34.68	0\\
34.69	0\\
34.7	0\\
34.71	0\\
34.72	0\\
34.73	0\\
34.74	0\\
34.75	0\\
34.76	0\\
34.77	0\\
34.78	0\\
34.79	0\\
34.8	0\\
34.81	0\\
34.82	0\\
34.83	0\\
34.84	0\\
34.85	0\\
34.86	0\\
34.87	0\\
34.88	0\\
34.89	0\\
34.9	0\\
34.91	0\\
34.92	0\\
34.93	0\\
34.94	0\\
34.95	0\\
34.96	0\\
34.97	0\\
34.98	0\\
34.99	0\\
35	0\\
35.01	0\\
35.02	0\\
35.03	0\\
35.04	0\\
35.05	0\\
35.06	0\\
35.07	0\\
35.08	0\\
35.09	0\\
35.1	0\\
35.11	0\\
35.12	0\\
35.13	0\\
35.14	0\\
35.15	0\\
35.16	0\\
35.17	0\\
35.18	0\\
35.19	0\\
35.2	0\\
35.21	0\\
35.22	0\\
35.23	0\\
35.24	0\\
35.25	0\\
35.26	0\\
35.27	0\\
35.28	0\\
35.29	0\\
35.3	0\\
35.31	0\\
35.32	0\\
35.33	0\\
35.34	0\\
35.35	0\\
35.36	0\\
35.37	0\\
35.38	0\\
35.39	0\\
35.4	0\\
35.41	0\\
35.42	0\\
35.43	0\\
35.44	0\\
35.45	0\\
35.46	0\\
35.47	0\\
35.48	0\\
35.49	0\\
35.5	0\\
35.51	0\\
35.52	0\\
35.53	0\\
35.54	0\\
35.55	0\\
35.56	0\\
35.57	0\\
35.58	0\\
35.59	0\\
35.6	0\\
35.61	0\\
35.62	0\\
35.63	0\\
35.64	0\\
35.65	0\\
35.66	0\\
35.67	0\\
35.68	0\\
35.69	0\\
35.7	0\\
35.71	0\\
35.72	0\\
35.73	0\\
35.74	0\\
35.75	0\\
35.76	0\\
35.77	0\\
35.78	0\\
35.79	0\\
35.8	0\\
35.81	0\\
35.82	0\\
35.83	0\\
35.84	0\\
35.85	0\\
35.86	0\\
35.87	0\\
35.88	0\\
35.89	0\\
35.9	0\\
35.91	0\\
35.92	0\\
35.93	0\\
35.94	0\\
35.95	0\\
35.96	0\\
35.97	0\\
35.98	0\\
35.99	0\\
36	0\\
36.01	0\\
36.02	0\\
36.03	0\\
36.04	0\\
36.05	0\\
36.06	0\\
36.07	0\\
36.08	0\\
36.09	0\\
36.1	0\\
36.11	0\\
36.12	0\\
36.13	0\\
36.14	0\\
36.15	0\\
36.16	0\\
36.17	0\\
36.18	0\\
36.19	0\\
36.2	0\\
36.21	0\\
36.22	0\\
36.23	0\\
36.24	0\\
36.25	0\\
36.26	0\\
36.27	0\\
36.28	0\\
36.29	0\\
36.3	0\\
36.31	0\\
36.32	0\\
36.33	0\\
36.34	0\\
36.35	0\\
36.36	0\\
36.37	0\\
36.38	0\\
36.39	0\\
36.4	0\\
36.41	0\\
36.42	0\\
36.43	0\\
36.44	0\\
36.45	0\\
36.46	0\\
36.47	0\\
36.48	0\\
36.49	0\\
36.5	0\\
36.51	0\\
36.52	0\\
36.53	0\\
36.54	0\\
36.55	0\\
36.56	0\\
36.57	0\\
36.58	0\\
36.59	0\\
36.6	0\\
36.61	0\\
36.62	0\\
36.63	0\\
36.64	0\\
36.65	0\\
36.66	0\\
36.67	0\\
36.68	0\\
36.69	0\\
36.7	0\\
36.71	0\\
36.72	0\\
36.73	0\\
36.74	0\\
36.75	0\\
36.76	0\\
36.77	0\\
36.78	0\\
36.79	0\\
36.8	0\\
36.81	0\\
36.82	0\\
36.83	0\\
36.84	0\\
36.85	0\\
36.86	0\\
36.87	0\\
36.88	0\\
36.89	0\\
36.9	0\\
36.91	0\\
36.92	0\\
36.93	0\\
36.94	0\\
36.95	0\\
36.96	0\\
36.97	0\\
36.98	0\\
36.99	0\\
37	0\\
37.01	0\\
37.02	0\\
37.03	0\\
37.04	0\\
37.05	0\\
37.06	0\\
37.07	0\\
37.08	0\\
37.09	0\\
37.1	0\\
37.11	0\\
37.12	0\\
37.13	0\\
37.14	0\\
37.15	0\\
37.16	0\\
37.17	0\\
37.18	0\\
37.19	0\\
37.2	0\\
37.21	0\\
37.22	0\\
37.23	0\\
37.24	0\\
37.25	0\\
37.26	0\\
37.27	0\\
37.28	0\\
37.29	0\\
37.3	0\\
37.31	0\\
37.32	0\\
37.33	0\\
37.34	0\\
37.35	0\\
37.36	0\\
37.37	0\\
37.38	0\\
37.39	0\\
37.4	0\\
37.41	0\\
37.42	0\\
37.43	0\\
37.44	0\\
37.45	0\\
37.46	0\\
37.47	0\\
37.48	0\\
37.49	0\\
37.5	0\\
37.51	0\\
37.52	0\\
37.53	0\\
37.54	0\\
37.55	0\\
37.56	0\\
37.57	0\\
37.58	0\\
37.59	0\\
37.6	0\\
37.61	0\\
37.62	0\\
37.63	0\\
37.64	0\\
37.65	0\\
37.66	0\\
37.67	0\\
37.68	0\\
37.69	0\\
37.7	0\\
37.71	0\\
37.72	0\\
37.73	0\\
37.74	0\\
37.75	0\\
37.76	0\\
37.77	0\\
37.78	0\\
37.79	0\\
37.8	0\\
37.81	0\\
37.82	0\\
37.83	0\\
37.84	0\\
37.85	0\\
37.86	0\\
37.87	0\\
37.88	0\\
37.89	0\\
37.9	0\\
37.91	0\\
37.92	0\\
37.93	0\\
37.94	0\\
37.95	0\\
37.96	0\\
37.97	0\\
37.98	0\\
37.99	0\\
38	0\\
38.01	0\\
38.02	0\\
38.03	0\\
38.04	0\\
38.05	0\\
38.06	0\\
38.07	0\\
38.08	0\\
38.09	0\\
38.1	0\\
38.11	0\\
38.12	0\\
38.13	0\\
38.14	0\\
38.15	0\\
38.16	0\\
38.17	0\\
38.18	0\\
38.19	0\\
38.2	0\\
38.21	0\\
38.22	0\\
38.23	0\\
38.24	0\\
38.25	0\\
38.26	0\\
38.27	0\\
38.28	0\\
38.29	0\\
38.3	0\\
38.31	0\\
38.32	0\\
38.33	0\\
38.34	0\\
38.35	0\\
38.36	0\\
38.37	0\\
38.38	0\\
38.39	0\\
38.4	0\\
38.41	0\\
38.42	0\\
38.43	0\\
38.44	0\\
38.45	0\\
38.46	0\\
38.47	0\\
38.48	0\\
38.49	0\\
38.5	0\\
38.51	0\\
38.52	0\\
38.53	0\\
38.54	0\\
38.55	0\\
38.56	0\\
38.57	0\\
38.58	0\\
38.59	0\\
38.6	0\\
38.61	1.73472347597681e-18\\
38.62	0\\
38.63	0\\
38.64	0\\
38.65	0\\
38.66	0\\
38.67	0\\
38.68	0\\
38.69	0\\
38.7	0\\
38.71	0\\
38.72	0\\
38.73	0\\
38.74	0\\
38.75	0\\
38.76	0\\
38.77	1.73472347597681e-18\\
38.78	0\\
38.79	0\\
38.8	0\\
38.81	0\\
38.82	0\\
38.83	0\\
38.84	0\\
38.85	0\\
38.86	0\\
38.87	0\\
38.88	0\\
38.89	0\\
38.9	0\\
38.91	0\\
38.92	0\\
38.93	1.73472347597681e-18\\
38.94	0\\
38.95	0\\
38.96	0\\
38.97	0\\
38.98	0\\
38.99	0\\
39	0\\
39.01	0\\
39.02	0\\
39.03	0\\
39.04	0\\
39.05	0\\
39.06	0\\
39.07	0\\
39.08	0\\
39.09	1.73472347597681e-18\\
39.1	0\\
39.11	0\\
39.12	0\\
39.13	0\\
39.14	0\\
39.15	0\\
39.16	0\\
39.17	0\\
39.18	0\\
39.19	0\\
39.2	0\\
39.21	0\\
39.22	0\\
39.23	0\\
39.24	0\\
39.25	1.73472347597681e-18\\
39.26	0\\
39.27	0\\
39.28	0\\
39.29	0\\
39.3	0\\
39.31	0\\
39.32	0\\
39.33	0\\
39.34	0\\
39.35	0\\
39.36	0\\
39.37	0\\
39.38	0\\
39.39	0\\
39.4	0\\
39.41	0\\
39.42	0\\
39.43	0\\
39.44	0\\
39.45	0\\
39.46	0\\
39.47	0\\
39.48	0\\
39.49	0\\
39.5	0\\
39.51	0\\
39.52	0\\
39.53	0\\
39.54	0\\
39.55	0\\
39.56	0\\
39.57	0\\
39.58	0\\
39.59	0\\
39.6	0\\
39.61	0\\
39.62	0\\
39.63	0\\
39.64	0\\
39.65	0\\
39.66	0\\
39.67	0\\
39.68	0\\
39.69	0\\
39.7	0\\
39.71	0\\
39.72	0\\
39.73	0\\
39.74	0\\
39.75	0\\
39.76	0\\
39.77	0\\
39.78	0\\
39.79	0\\
39.8	0\\
39.81	0\\
39.82	0\\
39.83	0\\
39.84	0\\
39.85	0\\
39.86	0\\
39.87	0\\
39.88	0\\
39.89	0\\
39.9	0\\
39.91	0\\
39.92	0\\
39.93	0\\
39.94	0\\
39.95	0\\
39.96	0\\
39.97	1.73472347597681e-18\\
39.98	0\\
39.99	0\\
40	0\\
40.01	0\\
};
\addplot [color=red,solid,forget plot]
  table[row sep=crcr]{%
40.01	0\\
40.02	0\\
40.03	0\\
40.04	0\\
40.05	0\\
40.06	0\\
40.07	0\\
40.08	0\\
40.09	0\\
40.1	0\\
40.11	0\\
40.12	0\\
40.13	1.73472347597681e-18\\
40.14	0\\
40.15	0\\
40.16	0\\
40.17	0\\
40.18	1.73472347597681e-18\\
40.19	0\\
40.2	0\\
40.21	0\\
40.22	1.73472347597681e-18\\
40.23	0\\
40.24	0\\
40.25	0\\
40.26	1.73472347597681e-18\\
40.27	0\\
40.28	0\\
40.29	0\\
40.3	1.73472347597681e-18\\
40.31	0\\
40.32	0\\
40.33	0\\
40.34	1.73472347597681e-18\\
40.35	0\\
40.36	0\\
40.37	0\\
40.38	1.73472347597681e-18\\
40.39	0\\
40.4	0\\
40.41	0\\
40.42	0\\
40.43	0\\
40.44	0\\
40.45	0\\
40.46	0\\
40.47	0\\
40.48	1.73472347597681e-18\\
40.49	0\\
40.5	0\\
40.51	0\\
40.52	0\\
40.53	0\\
40.54	0\\
40.55	0\\
40.56	0\\
40.57	0\\
40.58	0\\
40.59	0\\
40.6	0\\
40.61	0\\
40.62	0\\
40.63	0\\
40.64	0\\
40.65	0\\
40.66	0\\
40.67	0\\
40.68	0\\
40.69	0\\
40.7	0\\
40.71	0\\
40.72	0\\
40.73	0\\
40.74	0\\
40.75	0\\
40.76	0\\
40.77	0\\
40.78	0\\
40.79	0\\
40.8	0\\
40.81	0\\
40.82	0\\
40.83	0\\
40.84	0\\
40.85	0\\
40.86	1.73472347597681e-18\\
40.87	0\\
40.88	0\\
40.89	0\\
40.9	0\\
40.91	0\\
40.92	0\\
40.93	0\\
40.94	0\\
40.95	0\\
40.96	0\\
40.97	0\\
40.98	0\\
40.99	0\\
41	0\\
41.01	0\\
41.02	0\\
41.03	0\\
41.04	0\\
41.05	0\\
41.06	1.73472347597681e-18\\
41.07	0\\
41.08	0\\
41.09	0\\
41.1	0\\
41.11	0\\
41.12	0\\
41.13	0\\
41.14	0\\
41.15	0\\
41.16	0\\
41.17	0\\
41.18	0\\
41.19	0\\
41.2	0\\
41.21	0\\
41.22	0\\
41.23	0\\
41.24	0\\
41.25	0\\
41.26	0\\
41.27	1.73472347597681e-18\\
41.28	0\\
41.29	0\\
41.3	0\\
41.31	0\\
41.32	0\\
41.33	0\\
41.34	1.73472347597681e-18\\
41.35	0\\
41.36	0\\
41.37	0\\
41.38	1.73472347597681e-18\\
41.39	0\\
41.4	0\\
41.41	0\\
41.42	0\\
41.43	1.73472347597681e-18\\
41.44	0\\
41.45	0\\
41.46	0\\
41.47	0\\
41.48	0\\
41.49	0\\
41.5	0\\
41.51	0\\
41.52	0\\
41.53	0\\
41.54	0\\
41.55	0\\
41.56	0\\
41.57	0\\
41.58	0\\
41.59	0\\
41.6	0\\
41.61	0\\
41.62	0\\
41.63	0\\
41.64	0\\
41.65	0\\
41.66	0\\
41.67	0\\
41.68	0\\
41.69	0\\
41.7	0\\
41.71	0\\
41.72	0\\
41.73	0\\
41.74	1.73472347597681e-18\\
41.75	0\\
41.76	0\\
41.77	0\\
41.78	0\\
41.79	0\\
41.8	1.73472347597681e-18\\
41.81	0\\
41.82	0\\
41.83	0\\
41.84	0\\
41.85	0\\
41.86	0\\
41.87	0\\
41.88	0\\
41.89	0\\
41.9	0\\
41.91	0\\
41.92	0\\
41.93	0\\
41.94	0\\
41.95	0\\
41.96	0\\
41.97	0\\
41.98	0\\
41.99	0\\
42	0\\
42.01	0\\
42.02	0\\
42.03	0\\
42.04	1.73472347597681e-18\\
42.05	0\\
42.06	0\\
42.07	0\\
42.08	0\\
42.09	1.73472347597681e-18\\
42.1	0\\
42.11	0\\
42.12	1.73472347597681e-18\\
42.13	0\\
42.14	0\\
42.15	0\\
42.16	0\\
42.17	0\\
42.18	0\\
42.19	0\\
42.2	0\\
42.21	0\\
42.22	0\\
42.23	0\\
42.24	0\\
42.25	0\\
42.26	0\\
42.27	0\\
42.28	0\\
42.29	0\\
42.3	0\\
42.31	0\\
42.32	0\\
42.33	0\\
42.34	0\\
42.35	0\\
42.36	1.73472347597681e-18\\
42.37	0\\
42.38	0\\
42.39	0\\
42.4	0\\
42.41	0\\
42.42	0\\
42.43	0\\
42.44	0\\
42.45	0\\
42.46	0\\
42.47	0\\
42.48	0\\
42.49	0\\
42.5	0\\
42.51	0\\
42.52	0\\
42.53	0\\
42.54	0\\
42.55	0\\
42.56	0\\
42.57	0\\
42.58	0\\
42.59	0\\
42.6	0\\
42.61	0\\
42.62	0\\
42.63	0\\
42.64	0\\
42.65	0\\
42.66	0\\
42.67	0\\
42.68	0\\
42.69	0\\
42.7	0\\
42.71	0\\
42.72	0\\
42.73	0\\
42.74	0\\
42.75	0\\
42.76	0\\
42.77	0\\
42.78	0\\
42.79	0\\
42.8	0\\
42.81	0\\
42.82	0\\
42.83	0\\
42.84	0\\
42.85	0\\
42.86	0\\
42.87	0\\
42.88	0\\
42.89	0\\
42.9	0\\
42.91	0\\
42.92	1.73472347597681e-18\\
42.93	0\\
42.94	0\\
42.95	0\\
42.96	0\\
42.97	1.73472347597681e-18\\
42.98	0\\
42.99	0\\
43	0\\
43.01	0\\
43.02	0\\
43.03	0\\
43.04	0\\
43.05	0\\
43.06	0\\
43.07	0\\
43.08	0\\
43.09	0\\
43.1	0\\
43.11	0\\
43.12	0\\
43.13	0\\
43.14	0\\
43.15	0\\
43.16	0\\
43.17	0\\
43.18	0\\
43.19	0\\
43.2	0\\
43.21	0\\
43.22	0\\
43.23	0\\
43.24	1.73472347597681e-18\\
43.25	0\\
43.26	0\\
43.27	0\\
43.28	0\\
43.29	0\\
43.3	0\\
43.31	0\\
43.32	0\\
43.33	0\\
43.34	0\\
43.35	0\\
43.36	0\\
43.37	0\\
43.38	0\\
43.39	0\\
43.4	0\\
43.41	0\\
43.42	0\\
43.43	1.73472347597681e-18\\
43.44	0\\
43.45	0\\
43.46	0\\
43.47	0\\
43.48	0\\
43.49	0\\
43.5	0\\
43.51	0\\
43.52	0\\
43.53	0\\
43.54	0\\
43.55	0\\
43.56	0\\
43.57	0\\
43.58	0\\
43.59	0\\
43.6	0\\
43.61	0\\
43.62	0\\
43.63	0\\
43.64	0\\
43.65	0\\
43.66	0\\
43.67	0\\
43.68	0\\
43.69	0\\
43.7	0\\
43.71	0\\
43.72	0\\
43.73	0\\
43.74	0\\
43.75	0\\
43.76	0\\
43.77	0\\
43.78	0\\
43.79	0\\
43.8	0\\
43.81	1.73472347597681e-18\\
43.82	0\\
43.83	0\\
43.84	1.73472347597681e-18\\
43.85	0\\
43.86	0\\
43.87	0\\
43.88	0\\
43.89	0\\
43.9	0\\
43.91	0\\
43.92	0\\
43.93	0\\
43.94	1.73472347597681e-18\\
43.95	0\\
43.96	0\\
43.97	0\\
43.98	1.73472347597681e-18\\
43.99	0\\
44	0\\
44.01	0\\
44.02	1.73472347597681e-18\\
44.03	0\\
44.04	0\\
44.05	0\\
44.06	1.73472347597681e-18\\
44.07	0\\
44.08	0\\
44.09	0\\
44.1	0\\
44.11	0\\
44.12	1.73472347597681e-18\\
44.13	0\\
44.14	0\\
44.15	0\\
44.16	0\\
44.17	0\\
44.18	0\\
44.19	0\\
44.2	0\\
44.21	0\\
44.22	0\\
44.23	0\\
44.24	0\\
44.25	1.73472347597681e-18\\
44.26	0\\
44.27	0\\
44.28	0\\
44.29	0\\
44.3	0\\
44.31	0\\
44.32	0\\
44.33	0\\
44.34	0\\
44.35	0\\
44.36	0\\
44.37	0\\
44.38	0\\
44.39	0\\
44.4	0\\
44.41	0\\
44.42	1.73472347597681e-18\\
44.43	0\\
44.44	0\\
44.45	0\\
44.46	0\\
44.47	0\\
44.48	0\\
44.49	0\\
44.5	0\\
44.51	1.73472347597681e-18\\
44.52	0\\
44.53	0\\
44.54	0\\
44.55	0\\
44.56	0\\
44.57	0\\
44.58	0\\
44.59	0\\
44.6	0\\
44.61	0\\
44.62	0\\
44.63	0\\
44.64	0\\
44.65	0\\
44.66	0\\
44.67	0\\
44.68	0\\
44.69	0\\
44.7	0\\
44.71	0\\
44.72	0\\
44.73	0\\
44.74	0\\
44.75	0\\
44.76	0\\
44.77	0\\
44.78	0\\
44.79	0\\
44.8	0\\
44.81	0\\
44.82	0\\
44.83	0\\
44.84	0\\
44.85	0\\
44.86	0\\
44.87	0\\
44.88	0\\
44.89	0\\
44.9	0\\
44.91	0\\
44.92	0\\
44.93	0\\
44.94	0\\
44.95	0\\
44.96	0\\
44.97	0\\
44.98	0\\
44.99	0\\
45	0\\
45.01	0\\
45.02	0\\
45.03	0\\
45.04	0\\
45.05	0\\
45.06	0\\
45.07	0\\
45.08	0\\
45.09	0\\
45.1	0\\
45.11	0\\
45.12	0\\
45.13	0\\
45.14	0\\
45.15	0\\
45.16	0\\
45.17	0\\
45.18	0\\
45.19	0\\
45.2	0\\
45.21	0\\
45.22	0\\
45.23	0\\
45.24	0\\
45.25	0\\
45.26	0\\
45.27	0\\
45.28	0\\
45.29	0\\
45.3	0\\
45.31	0\\
45.32	1.73472347597681e-18\\
45.33	0\\
45.34	0\\
45.35	0\\
45.36	0\\
45.37	0\\
45.38	0\\
45.39	0\\
45.4	0\\
45.41	0\\
45.42	0\\
45.43	1.73472347597681e-18\\
45.44	0\\
45.45	0\\
45.46	0\\
45.47	0\\
45.48	0\\
45.49	0\\
45.5	0\\
45.51	0\\
45.52	0\\
45.53	0\\
45.54	1.73472347597681e-18\\
45.55	0\\
45.56	0\\
45.57	0\\
45.58	0\\
45.59	0\\
45.6	1.73472347597681e-18\\
45.61	0\\
45.62	0\\
45.63	0\\
45.64	0\\
45.65	0\\
45.66	0\\
45.67	0\\
45.68	0\\
45.69	0\\
45.7	0\\
45.71	0\\
45.72	0\\
45.73	0\\
45.74	0\\
45.75	0\\
45.76	0\\
45.77	0\\
45.78	0\\
45.79	0\\
45.8	0\\
45.81	0\\
45.82	0\\
45.83	0\\
45.84	0\\
45.85	0\\
45.86	0\\
45.87	0\\
45.88	0\\
45.89	0\\
45.9	0\\
45.91	0\\
45.92	0\\
45.93	0\\
45.94	0\\
45.95	0\\
45.96	0\\
45.97	0\\
45.98	1.73472347597681e-18\\
45.99	0\\
46	0\\
46.01	0\\
46.02	0\\
46.03	0\\
46.04	0\\
46.05	0\\
46.06	0\\
46.07	0\\
46.08	0\\
46.09	0\\
46.1	0\\
46.11	1.73472347597681e-18\\
46.12	1.73472347597681e-18\\
46.13	0\\
46.14	0\\
46.15	0\\
46.16	0\\
46.17	0\\
46.18	0\\
46.19	0\\
46.2	0\\
46.21	0\\
46.22	0\\
46.23	0\\
46.24	0\\
46.25	0\\
46.26	0\\
46.27	0\\
46.28	0\\
46.29	0\\
46.3	0\\
46.31	0\\
46.32	0\\
46.33	0\\
46.34	0\\
46.35	0\\
46.36	0\\
46.37	0\\
46.38	0\\
46.39	0\\
46.4	1.73472347597681e-18\\
46.41	0\\
46.42	0\\
46.43	0\\
46.44	0\\
46.45	0\\
46.46	0\\
46.47	0\\
46.48	0\\
46.49	0\\
46.5	0\\
46.51	1.73472347597681e-18\\
46.52	0\\
46.53	0\\
46.54	0\\
46.55	0\\
46.56	0\\
46.57	0\\
46.58	0\\
46.59	0\\
46.6	0\\
46.61	0\\
46.62	0\\
46.63	0\\
46.64	0\\
46.65	0\\
46.66	0\\
46.67	0\\
46.68	0\\
46.69	0\\
46.7	0\\
46.71	0\\
46.72	0\\
46.73	0\\
46.74	0\\
46.75	0\\
46.76	0\\
46.77	0\\
46.78	0\\
46.79	1.73472347597681e-18\\
46.8	0\\
46.81	0\\
46.82	0\\
46.83	0\\
46.84	0\\
46.85	0\\
46.86	0\\
46.87	1.73472347597681e-18\\
46.88	0\\
46.89	0\\
46.9	0\\
46.91	0\\
46.92	0\\
46.93	0\\
46.94	0\\
46.95	0\\
46.96	0\\
46.97	0\\
46.98	0\\
46.99	0\\
47	0\\
47.01	0\\
47.02	0\\
47.03	0\\
47.04	0\\
47.05	0\\
47.06	0\\
47.07	0\\
47.08	1.73472347597681e-18\\
47.09	0\\
47.1	0\\
47.11	0\\
47.12	0\\
47.13	0\\
47.14	0\\
47.15	0\\
47.16	0\\
47.17	0\\
47.18	0\\
47.19	0\\
47.2	0\\
47.21	0\\
47.22	1.73472347597681e-18\\
47.23	0\\
47.24	0\\
47.25	0\\
47.26	0\\
47.27	0\\
47.28	0\\
47.29	0\\
47.3	0\\
47.31	1.73472347597681e-18\\
47.32	0\\
47.33	0\\
47.34	0\\
47.35	0\\
47.36	0\\
47.37	0\\
47.38	1.73472347597681e-18\\
47.39	0\\
47.4	0\\
47.41	0\\
47.42	0\\
47.43	0\\
47.44	0\\
47.45	0\\
47.46	0\\
47.47	0\\
47.48	0\\
47.49	0\\
47.5	0\\
47.51	0\\
47.52	0\\
47.53	0\\
47.54	0\\
47.55	0\\
47.56	0\\
47.57	0\\
47.58	0\\
47.59	0\\
47.6	0\\
47.61	0\\
47.62	0\\
47.63	0\\
47.64	0\\
47.65	0\\
47.66	0\\
47.67	0\\
47.68	0\\
47.69	0\\
47.7	0\\
47.71	0\\
47.72	0\\
47.73	0\\
47.74	0\\
47.75	0\\
47.76	0\\
47.77	0\\
47.78	0\\
47.79	0\\
47.8	0\\
47.81	1.73472347597681e-18\\
47.82	0\\
47.83	1.73472347597681e-18\\
47.84	0\\
47.85	0\\
47.86	0\\
47.87	0\\
47.88	0\\
47.89	0\\
47.9	0\\
47.91	0\\
47.92	0\\
47.93	0\\
47.94	1.73472347597681e-18\\
47.95	0\\
47.96	0\\
47.97	0\\
47.98	0\\
47.99	0\\
48	0\\
48.01	1.73472347597681e-18\\
48.02	0\\
48.03	0\\
48.04	0\\
48.05	0\\
48.06	0\\
48.07	0\\
48.08	0\\
48.09	0\\
48.1	0\\
48.11	0\\
48.12	0\\
48.13	0\\
48.14	0\\
48.15	0\\
48.16	1.73472347597681e-18\\
48.17	0\\
48.18	0\\
48.19	0\\
48.2	1.73472347597681e-18\\
48.21	0\\
48.22	0\\
48.23	0\\
48.24	0\\
48.25	0\\
48.26	0\\
48.27	0\\
48.28	0\\
48.29	0\\
48.3	0\\
48.31	1.73472347597681e-18\\
48.32	0\\
48.33	0\\
48.34	0\\
48.35	0\\
48.36	0\\
48.37	0\\
48.38	1.73472347597681e-18\\
48.39	0\\
48.4	0\\
48.41	0\\
48.42	0\\
48.43	0\\
48.44	0\\
48.45	0\\
48.46	0\\
48.47	0\\
48.48	0\\
48.49	0\\
48.5	0\\
48.51	0\\
48.52	0\\
48.53	1.73472347597681e-18\\
48.54	1.73472347597681e-18\\
48.55	0\\
48.56	0\\
48.57	0\\
48.58	0\\
48.59	0\\
48.6	0\\
48.61	0\\
48.62	0\\
48.63	0\\
48.64	0\\
48.65	1.73472347597681e-18\\
48.66	0\\
48.67	0\\
48.68	0\\
48.69	0\\
48.7	0\\
48.71	0\\
48.72	0\\
48.73	0\\
48.74	0\\
48.75	0\\
48.76	0\\
48.77	0\\
48.78	0\\
48.79	0\\
48.8	0\\
48.81	0\\
48.82	0\\
48.83	0\\
48.84	0\\
48.85	0\\
48.86	0\\
48.87	0\\
48.88	0\\
48.89	0\\
48.9	1.73472347597681e-18\\
48.91	0\\
48.92	0\\
48.93	0\\
48.94	0\\
48.95	0\\
48.96	0\\
48.97	0\\
48.98	0\\
48.99	0\\
49	0\\
49.01	0\\
49.02	0\\
49.03	0\\
49.04	0\\
49.05	0\\
49.06	0\\
49.07	0\\
49.08	1.73472347597681e-18\\
49.09	0\\
49.1	0\\
49.11	0\\
49.12	0\\
49.13	0\\
49.14	0\\
49.15	0\\
49.16	0\\
49.17	0\\
49.18	0\\
49.19	0\\
49.2	0\\
49.21	0\\
49.22	0\\
49.23	0\\
49.24	0\\
49.25	0\\
49.26	0\\
49.27	0\\
49.28	0\\
49.29	0\\
49.3	0\\
49.31	0\\
49.32	0\\
49.33	0\\
49.34	1.73472347597681e-18\\
49.35	0\\
49.36	0\\
49.37	0\\
49.38	0\\
49.39	0\\
49.4	1.73472347597681e-18\\
49.41	0\\
49.42	0\\
49.43	0\\
49.44	0\\
49.45	0\\
49.46	0\\
49.47	0\\
49.48	0\\
49.49	0\\
49.5	0\\
49.51	0\\
49.52	0\\
49.53	0\\
49.54	0\\
49.55	1.73472347597681e-18\\
49.56	0\\
49.57	0\\
49.58	0\\
49.59	1.73472347597681e-18\\
49.6	0\\
49.61	0\\
49.62	0\\
49.63	1.73472347597681e-18\\
49.64	0\\
49.65	0\\
49.66	0\\
49.67	0\\
49.68	0\\
49.69	0\\
49.7	0\\
49.71	0\\
49.72	0\\
49.73	0\\
49.74	0\\
49.75	0\\
49.76	0\\
49.77	0\\
49.78	0\\
49.79	0\\
49.8	0\\
49.81	0\\
49.82	0\\
49.83	0\\
49.84	0\\
49.85	0\\
49.86	0\\
49.87	0\\
49.88	0\\
49.89	0\\
49.9	0\\
49.91	0\\
49.92	0\\
49.93	0\\
49.94	0\\
49.95	0\\
49.96	0\\
49.97	1.73472347597681e-18\\
49.98	0\\
49.99	0\\
50	0\\
50.01	0\\
50.02	0\\
50.03	0\\
50.04	0\\
50.05	0\\
50.06	1.73472347597681e-18\\
50.07	0\\
50.08	0\\
50.09	0\\
50.1	0\\
50.11	0\\
50.12	0\\
50.13	0\\
50.14	0\\
50.15	0\\
50.16	0\\
50.17	0\\
50.18	0\\
50.19	0\\
50.2	0\\
50.21	0\\
50.22	0\\
50.23	0\\
50.24	0\\
50.25	0\\
50.26	0\\
50.27	0\\
50.28	0\\
50.29	0\\
50.3	0\\
50.31	0\\
50.32	0\\
50.33	0\\
50.34	0\\
50.35	0\\
50.36	0\\
50.37	0\\
50.38	0\\
50.39	0\\
50.4	1.73472347597681e-18\\
50.41	0\\
50.42	0\\
50.43	0\\
50.44	0\\
50.45	0\\
50.46	0\\
50.47	0\\
50.48	0\\
50.49	0\\
50.5	0\\
50.51	0\\
50.52	0\\
50.53	0\\
50.54	0\\
50.55	0\\
50.56	0\\
50.57	0\\
50.58	1.73472347597681e-18\\
50.59	0\\
50.6	0\\
50.61	0\\
50.62	0\\
50.63	0\\
50.64	0\\
50.65	0\\
50.66	0\\
50.67	0\\
50.68	0\\
50.69	0\\
50.7	0\\
50.71	0\\
50.72	0\\
50.73	0\\
50.74	0\\
50.75	1.73472347597681e-18\\
50.76	0\\
50.77	0\\
50.78	0\\
50.79	0\\
50.8	0\\
50.81	0\\
50.82	0\\
50.83	0\\
50.84	0\\
50.85	0\\
50.86	0\\
50.87	0\\
50.88	0\\
50.89	0\\
50.9	0\\
50.91	0\\
50.92	0\\
50.93	0\\
50.94	0\\
50.95	0\\
50.96	0\\
50.97	0\\
50.98	0\\
50.99	0\\
51	0\\
51.01	0\\
51.02	0\\
51.03	0\\
51.04	0\\
51.05	0\\
51.06	0\\
51.07	0\\
51.08	1.73472347597681e-18\\
51.09	0\\
51.1	0\\
51.11	0\\
51.12	0\\
51.13	0\\
51.14	0\\
51.15	0\\
51.16	0\\
51.17	0\\
51.18	0\\
51.19	0\\
51.2	0\\
51.21	0\\
51.22	0\\
51.23	0\\
51.24	0\\
51.25	0\\
51.26	0\\
51.27	0\\
51.28	0\\
51.29	0\\
51.3	0\\
51.31	0\\
51.32	0\\
51.33	0\\
51.34	0\\
51.35	0\\
51.36	0\\
51.37	0\\
51.38	0\\
51.39	0\\
51.4	0\\
51.41	0\\
51.42	0\\
51.43	0\\
51.44	0\\
51.45	0\\
51.46	0\\
51.47	0\\
51.48	0\\
51.49	0\\
51.5	0\\
51.51	0\\
51.52	0\\
51.53	0\\
51.54	1.73472347597681e-18\\
51.55	0\\
51.56	0\\
51.57	0\\
51.58	0\\
51.59	0\\
51.6	0\\
51.61	0\\
51.62	0\\
51.63	0\\
51.64	0\\
51.65	0\\
51.66	0\\
51.67	0\\
51.68	0\\
51.69	0\\
51.7	0\\
51.71	0\\
51.72	0\\
51.73	0\\
51.74	0\\
51.75	0\\
51.76	0\\
51.77	0\\
51.78	0\\
51.79	0\\
51.8	0\\
51.81	0\\
51.82	0\\
51.83	0\\
51.84	1.73472347597681e-18\\
51.85	0\\
51.86	0\\
51.87	0\\
51.88	0\\
51.89	0\\
51.9	0\\
51.91	1.73472347597681e-18\\
51.92	0\\
51.93	0\\
51.94	0\\
51.95	0\\
51.96	0\\
51.97	0\\
51.98	0\\
51.99	0\\
52	0\\
52.01	0\\
52.02	0\\
52.03	0\\
52.04	1.73472347597681e-18\\
52.05	0\\
52.06	0\\
52.07	0\\
52.08	0\\
52.09	0\\
52.1	0\\
52.11	0\\
52.12	0\\
52.13	0\\
52.14	0\\
52.15	1.73472347597681e-18\\
52.16	0\\
52.17	0\\
52.18	0\\
52.19	0\\
52.2	0\\
52.21	0\\
52.22	0\\
52.23	0\\
52.24	0\\
52.25	0\\
52.26	0\\
52.27	0\\
52.28	0\\
52.29	0\\
52.3	0\\
52.31	0\\
52.32	0\\
52.33	0\\
52.34	0\\
52.35	0\\
52.36	0\\
52.37	1.73472347597681e-18\\
52.38	0\\
52.39	0\\
52.4	0\\
52.41	0\\
52.42	0\\
52.43	0\\
52.44	0\\
52.45	0\\
52.46	0\\
52.47	0\\
52.48	0\\
52.49	0\\
52.5	1.73472347597681e-18\\
52.51	0\\
52.52	0\\
52.53	0\\
52.54	0\\
52.55	0\\
52.56	0\\
52.57	0\\
52.58	0\\
52.59	1.73472347597681e-18\\
52.6	0\\
52.61	0\\
52.62	0\\
52.63	0\\
52.64	0\\
52.65	1.73472347597681e-18\\
52.66	0\\
52.67	0\\
52.68	0\\
52.69	0\\
52.7	0\\
52.71	0\\
52.72	0\\
52.73	0\\
52.74	0\\
52.75	0\\
52.76	0\\
52.77	0\\
52.78	0\\
52.79	0\\
52.8	0\\
52.81	0\\
52.82	0\\
52.83	0\\
52.84	0\\
52.85	0\\
52.86	0\\
52.87	0\\
52.88	0\\
52.89	0\\
52.9	0\\
52.91	0\\
52.92	1.73472347597681e-18\\
52.93	0\\
52.94	0\\
52.95	0\\
52.96	0\\
52.97	0\\
52.98	0\\
52.99	0\\
53	0\\
53.01	0\\
53.02	0\\
53.03	0\\
53.04	0\\
53.05	0\\
53.06	0\\
53.07	0\\
53.08	0\\
53.09	0\\
53.1	0\\
53.11	0\\
53.12	0\\
53.13	0\\
53.14	0\\
53.15	0\\
53.16	0\\
53.17	0\\
53.18	0\\
53.19	0\\
53.2	0\\
53.21	0\\
53.22	0\\
53.23	0\\
53.24	0\\
53.25	0\\
53.26	0\\
53.27	0\\
53.28	0\\
53.29	0\\
53.3	0\\
53.31	0\\
53.32	0\\
53.33	0\\
53.34	0\\
53.35	0\\
53.36	0\\
53.37	0\\
53.38	0\\
53.39	0\\
53.4	0\\
53.41	0\\
53.42	0\\
53.43	0\\
53.44	0\\
53.45	0\\
53.46	0\\
53.47	1.73472347597681e-18\\
53.48	0\\
53.49	0\\
53.5	0\\
53.51	0\\
53.52	0\\
53.53	0\\
53.54	0\\
53.55	0\\
53.56	0\\
53.57	0\\
53.58	0\\
53.59	0\\
53.6	0\\
53.61	1.73472347597681e-18\\
53.62	0\\
53.63	0\\
53.64	0\\
53.65	0\\
53.66	0\\
53.67	0\\
53.68	0\\
53.69	0\\
53.7	0\\
53.71	0\\
53.72	0\\
53.73	0\\
53.74	0\\
53.75	0\\
53.76	0\\
53.77	0\\
53.78	0\\
53.79	0\\
53.8	0\\
53.81	1.73472347597681e-18\\
53.82	0\\
53.83	0\\
53.84	1.73472347597681e-18\\
53.85	0\\
53.86	0\\
53.87	0\\
53.88	0\\
53.89	0\\
53.9	0\\
53.91	0\\
53.92	0\\
53.93	0\\
53.94	0\\
53.95	0\\
53.96	0\\
53.97	1.73472347597681e-18\\
53.98	0\\
53.99	0\\
54	0\\
54.01	0\\
54.02	0\\
54.03	0\\
54.04	0\\
54.05	0\\
54.06	0\\
54.07	0\\
54.08	0\\
54.09	0\\
54.1	0\\
54.11	0\\
54.12	0\\
54.13	0\\
54.14	0\\
54.15	0\\
54.16	0\\
54.17	0\\
54.18	0\\
54.19	0\\
54.2	0\\
54.21	0\\
54.22	0\\
54.23	0\\
54.24	0\\
54.25	0\\
54.26	0\\
54.27	0\\
54.28	0\\
54.29	0\\
54.3	0\\
54.31	0\\
54.32	0\\
54.33	0\\
54.34	0\\
54.35	0\\
54.36	0\\
54.37	0\\
54.38	0\\
54.39	0\\
54.4	0\\
54.41	0\\
54.42	0\\
54.43	0\\
54.44	0\\
54.45	0\\
54.46	0\\
54.47	0\\
54.48	0\\
54.49	0\\
54.5	0\\
54.51	0\\
54.52	0\\
54.53	0\\
54.54	0\\
54.55	0\\
54.56	0\\
54.57	0\\
54.58	0\\
54.59	0\\
54.6	0\\
54.61	0\\
54.62	0\\
54.63	0\\
54.64	0\\
54.65	0\\
54.66	0\\
54.67	0\\
54.68	0\\
54.69	0\\
54.7	0\\
54.71	0\\
54.72	1.73472347597681e-18\\
54.73	0\\
54.74	0\\
54.75	0\\
54.76	0\\
54.77	0\\
54.78	0\\
54.79	0\\
54.8	0\\
54.81	0\\
54.82	0\\
54.83	0\\
54.84	0\\
54.85	0\\
54.86	0\\
54.87	0\\
54.88	0\\
54.89	0\\
54.9	0\\
54.91	0\\
54.92	0\\
54.93	0\\
54.94	0\\
54.95	0\\
54.96	0\\
54.97	0\\
54.98	0\\
54.99	0\\
55	0\\
55.01	0\\
55.02	0\\
55.03	0\\
55.04	0\\
55.05	0\\
55.06	0\\
55.07	0\\
55.08	0\\
55.09	0\\
55.1	0\\
55.11	0\\
55.12	0\\
55.13	0\\
55.14	1.73472347597681e-18\\
55.15	0\\
55.16	0\\
55.17	0\\
55.18	0\\
55.19	0\\
55.2	0\\
55.21	1.73472347597681e-18\\
55.22	0\\
55.23	0\\
55.24	0\\
55.25	0\\
55.26	0\\
55.27	0\\
55.28	0\\
55.29	0\\
55.3	0\\
55.31	0\\
55.32	0\\
55.33	0\\
55.34	0\\
55.35	0\\
55.36	0\\
55.37	0\\
55.38	0\\
55.39	0\\
55.4	0\\
55.41	0\\
55.42	0\\
55.43	1.73472347597681e-18\\
55.44	0\\
55.45	0\\
55.46	0\\
55.47	0\\
55.48	0\\
55.49	0\\
55.5	0\\
55.51	0\\
55.52	0\\
55.53	0\\
55.54	0\\
55.55	0\\
55.56	1.73472347597681e-18\\
55.57	0\\
55.58	0\\
55.59	0\\
55.6	0\\
55.61	0\\
55.62	0\\
55.63	0\\
55.64	0\\
55.65	0\\
55.66	1.73472347597681e-18\\
55.67	0\\
55.68	0\\
55.69	0\\
55.7	1.73472347597681e-18\\
55.71	0\\
55.72	0\\
55.73	0\\
55.74	0\\
55.75	0\\
55.76	1.73472347597681e-18\\
55.77	0\\
55.78	0\\
55.79	0\\
55.8	0\\
55.81	0\\
55.82	1.73472347597681e-18\\
55.83	0\\
55.84	0\\
55.85	0\\
55.86	0\\
55.87	0\\
55.88	0\\
55.89	0\\
55.9	0\\
55.91	0\\
55.92	0\\
55.93	0\\
55.94	0\\
55.95	0\\
55.96	1.73472347597681e-18\\
55.97	0\\
55.98	0\\
55.99	0\\
56	0\\
56.01	0\\
56.02	0\\
56.03	0\\
56.04	0\\
56.05	0\\
56.06	0\\
56.07	0\\
56.08	0\\
56.09	0\\
56.1	0\\
56.11	0\\
56.12	0\\
56.13	0\\
56.14	0\\
56.15	0\\
56.16	0\\
56.17	0\\
56.18	0\\
56.19	0\\
56.2	0\\
56.21	0\\
56.22	0\\
56.23	0\\
56.24	0\\
56.25	0\\
56.26	0\\
56.27	0\\
56.28	0\\
56.29	0\\
56.3	0\\
56.31	0\\
56.32	0\\
56.33	0\\
56.34	0\\
56.35	0\\
56.36	0\\
56.37	0\\
56.38	0\\
56.39	0\\
56.4	0\\
56.41	0\\
56.42	0\\
56.43	0\\
56.44	0\\
56.45	0\\
56.46	0\\
56.47	0\\
56.48	0\\
56.49	0\\
56.5	1.73472347597681e-18\\
56.51	0\\
56.52	0\\
56.53	0\\
56.54	0\\
56.55	0\\
56.56	0\\
56.57	0\\
56.58	0\\
56.59	0\\
56.6	0\\
56.61	0\\
56.62	0\\
56.63	0\\
56.64	0\\
56.65	0\\
56.66	0\\
56.67	0\\
56.68	0\\
56.69	0\\
56.7	1.73472347597681e-18\\
56.71	0\\
56.72	0\\
56.73	0\\
56.74	0\\
56.75	0\\
56.76	0\\
56.77	0\\
56.78	0\\
56.79	0\\
56.8	0\\
56.81	0\\
56.82	0\\
56.83	0\\
56.84	0\\
56.85	0\\
56.86	0\\
56.87	1.73472347597681e-18\\
56.88	0\\
56.89	0\\
56.9	1.73472347597681e-18\\
56.91	0\\
56.92	0\\
56.93	0\\
56.94	0\\
56.95	0\\
56.96	0\\
56.97	0\\
56.98	0\\
56.99	0\\
57	0\\
57.01	0\\
57.02	0\\
57.03	0\\
57.04	0\\
57.05	0\\
57.06	0\\
57.07	0\\
57.08	0\\
57.09	0\\
57.1	0\\
57.11	0\\
57.12	0\\
57.13	1.73472347597681e-18\\
57.14	0\\
57.15	0\\
57.16	0\\
57.17	0\\
57.18	0\\
57.19	0\\
57.2	0\\
57.21	0\\
57.22	0\\
57.23	0\\
57.24	0\\
57.25	0\\
57.26	0\\
57.27	0\\
57.28	0\\
57.29	0\\
57.3	0\\
57.31	0\\
57.32	0\\
57.33	0\\
57.34	0\\
57.35	1.73472347597681e-18\\
57.36	1.73472347597681e-18\\
57.37	0\\
57.38	0\\
57.39	0\\
57.4	0\\
57.41	1.73472347597681e-18\\
57.42	0\\
57.43	0\\
57.44	0\\
57.45	1.73472347597681e-18\\
57.46	0\\
57.47	0\\
57.48	0\\
57.49	0\\
57.5	0\\
57.51	0\\
57.52	0\\
57.53	0\\
57.54	0\\
57.55	0\\
57.56	0\\
57.57	0\\
57.58	0\\
57.59	0\\
57.6	0\\
57.61	0\\
57.62	0\\
57.63	0\\
57.64	0\\
57.65	1.73472347597681e-18\\
57.66	0\\
57.67	0\\
57.68	0\\
57.69	0\\
57.7	0\\
57.71	0\\
57.72	0\\
57.73	0\\
57.74	0\\
57.75	0\\
57.76	0\\
57.77	0\\
57.78	0\\
57.79	0\\
57.8	0\\
57.81	0\\
57.82	0\\
57.83	0\\
57.84	0\\
57.85	0\\
57.86	0\\
57.87	0\\
57.88	1.73472347597681e-18\\
57.89	1.73472347597681e-18\\
57.9	0\\
57.91	0\\
57.92	0\\
57.93	0\\
57.94	0\\
57.95	0\\
57.96	0\\
57.97	0\\
57.98	0\\
57.99	0\\
58	0\\
58.01	0\\
58.02	0\\
58.03	0\\
58.04	0\\
58.05	0\\
58.06	0\\
58.07	0\\
58.08	0\\
58.09	0\\
58.1	0\\
58.11	0\\
58.12	0\\
58.13	0\\
58.14	0\\
58.15	0\\
58.16	0\\
58.17	0\\
58.18	0\\
58.19	1.73472347597681e-18\\
58.2	0\\
58.21	0\\
58.22	0\\
58.23	0\\
58.24	0\\
58.25	0\\
58.26	0\\
58.27	0\\
58.28	0\\
58.29	0\\
58.3	0\\
58.31	0\\
58.32	0\\
58.33	0\\
58.34	0\\
58.35	0\\
58.36	0\\
58.37	0\\
58.38	0\\
58.39	0\\
58.4	0\\
58.41	0\\
58.42	0\\
58.43	0\\
58.44	0\\
58.45	0\\
58.46	0\\
58.47	0\\
58.48	0\\
58.49	0\\
58.5	0\\
58.51	0\\
58.52	0\\
58.53	0\\
58.54	0\\
58.55	0\\
58.56	0\\
58.57	0\\
58.58	0\\
58.59	0\\
58.6	0\\
58.61	0\\
58.62	0\\
58.63	0\\
58.64	0\\
58.65	0\\
58.66	0\\
58.67	0\\
58.68	0\\
58.69	0\\
58.7	0\\
58.71	0\\
58.72	0\\
58.73	0\\
58.74	0\\
58.75	0\\
58.76	0\\
58.77	0\\
58.78	1.73472347597681e-18\\
58.79	0\\
58.8	0\\
58.81	0\\
58.82	0\\
58.83	0\\
58.84	0\\
58.85	0\\
58.86	1.73472347597681e-18\\
58.87	0\\
58.88	0\\
58.89	0\\
58.9	0\\
58.91	0\\
58.92	0\\
58.93	0\\
58.94	0\\
58.95	0\\
58.96	0\\
58.97	0\\
58.98	0\\
58.99	0\\
59	0\\
59.01	0\\
59.02	0\\
59.03	0\\
59.04	0\\
59.05	0\\
59.06	0\\
59.07	0\\
59.08	0\\
59.09	0\\
59.1	0\\
59.11	0\\
59.12	0\\
59.13	0\\
59.14	0\\
59.15	0\\
59.16	0\\
59.17	0\\
59.18	1.73472347597681e-18\\
59.19	0\\
59.2	0\\
59.21	0\\
59.22	0\\
59.23	0\\
59.24	0\\
59.25	0\\
59.26	0\\
59.27	0\\
59.28	0\\
59.29	0\\
59.3	0\\
59.31	0\\
59.32	0\\
59.33	0\\
59.34	0\\
59.35	0\\
59.36	0\\
59.37	0\\
59.38	0\\
59.39	0\\
59.4	0\\
59.41	0\\
59.42	0\\
59.43	0\\
59.44	0\\
59.45	0\\
59.46	0\\
59.47	0\\
59.48	0\\
59.49	0\\
59.5	0\\
59.51	0\\
59.52	0\\
59.53	0\\
59.54	0\\
59.55	0\\
59.56	0\\
59.57	0\\
59.58	0\\
59.59	0\\
59.6	0\\
59.61	0\\
59.62	0\\
59.63	0\\
59.64	0\\
59.65	0\\
59.66	0\\
59.67	0\\
59.68	1.73472347597681e-18\\
59.69	0\\
59.7	0\\
59.71	0\\
59.72	0\\
59.73	0\\
59.74	0\\
59.75	0\\
59.76	0\\
59.77	0\\
59.78	0\\
59.79	0\\
59.8	0\\
59.81	0\\
59.82	0\\
59.83	0\\
59.84	0\\
59.85	0\\
59.86	0\\
59.87	0\\
59.88	0\\
59.89	0\\
59.9	0\\
59.91	0\\
59.92	0\\
59.93	0\\
59.94	1.73472347597681e-18\\
59.95	0\\
59.96	0\\
59.97	0\\
59.98	0\\
59.99	0\\
60	0\\
60.01	0\\
60.02	0\\
60.03	0\\
60.04	0\\
60.05	1.73472347597681e-18\\
60.06	0\\
60.07	0\\
60.08	1.73472347597681e-18\\
60.09	0\\
60.1	0\\
60.11	0\\
60.12	0\\
60.13	0\\
60.14	1.73472347597681e-18\\
60.15	0\\
60.16	0\\
60.17	0\\
60.18	0\\
60.19	0\\
60.2	1.73472347597681e-18\\
60.21	0\\
60.22	0\\
60.23	0\\
60.24	0\\
60.25	0\\
60.26	0\\
60.27	0\\
60.28	0\\
60.29	0\\
60.3	0\\
60.31	0\\
60.32	0\\
60.33	0\\
60.34	0\\
60.35	0\\
60.36	0\\
60.37	0\\
60.38	0\\
60.39	0\\
60.4	0\\
60.41	0\\
60.42	0\\
60.43	0\\
60.44	0\\
60.45	0\\
60.46	0\\
60.47	0\\
60.48	0\\
60.49	0\\
60.5	0\\
60.51	0\\
60.52	0\\
60.53	0\\
60.54	0\\
60.55	0\\
60.56	0\\
60.57	0\\
60.58	0\\
60.59	0\\
60.6	0\\
60.61	0\\
60.62	0\\
60.63	1.73472347597681e-18\\
60.64	0\\
60.65	0\\
60.66	0\\
60.67	0\\
60.68	0\\
60.69	0\\
60.7	0\\
60.71	0\\
60.72	0\\
60.73	0\\
60.74	0\\
60.75	0\\
60.76	0\\
60.77	0\\
60.78	0\\
60.79	0\\
60.8	0\\
60.81	0\\
60.82	0\\
60.83	0\\
60.84	0\\
60.85	0\\
60.86	0\\
60.87	0\\
60.88	0\\
60.89	0\\
60.9	0\\
60.91	0\\
60.92	1.73472347597681e-18\\
60.93	0\\
60.94	0\\
60.95	0\\
60.96	1.73472347597681e-18\\
60.97	0\\
60.98	0\\
60.99	0\\
61	0\\
61.01	0\\
61.02	0\\
61.03	0\\
61.04	0\\
61.05	0\\
61.06	0\\
61.07	0\\
61.08	0\\
61.09	0\\
61.1	0\\
61.11	0\\
61.12	0\\
61.13	0\\
61.14	0\\
61.15	0\\
61.16	0\\
61.17	0\\
61.18	0\\
61.19	0\\
61.2	0\\
61.21	0\\
61.22	0\\
61.23	0\\
61.24	0\\
61.25	0\\
61.26	0\\
61.27	0\\
61.28	0\\
61.29	0\\
61.3	0\\
61.31	0\\
61.32	0\\
61.33	0\\
61.34	0\\
61.35	0\\
61.36	0\\
61.37	0\\
61.38	0\\
61.39	0\\
61.4	0\\
61.41	0\\
61.42	0\\
61.43	0\\
61.44	0\\
61.45	0\\
61.46	0\\
61.47	0\\
61.48	0\\
61.49	0\\
61.5	0\\
61.51	0\\
61.52	0\\
61.53	0\\
61.54	0\\
61.55	0\\
61.56	0\\
61.57	0\\
61.58	0\\
61.59	0\\
61.6	0\\
61.61	0\\
61.62	0\\
61.63	0\\
61.64	0\\
61.65	0\\
61.66	0\\
61.67	1.73472347597681e-18\\
61.68	1.73472347597681e-18\\
61.69	0\\
61.7	0\\
61.71	0\\
61.72	0\\
61.73	0\\
61.74	0\\
61.75	0\\
61.76	0\\
61.77	0\\
61.78	0\\
61.79	0\\
61.8	0\\
61.81	0\\
61.82	0\\
61.83	0\\
61.84	0\\
61.85	0\\
61.86	0\\
61.87	0\\
61.88	0\\
61.89	0\\
61.9	0\\
61.91	0\\
61.92	0\\
61.93	0\\
61.94	0\\
61.95	0\\
61.96	0\\
61.97	0\\
61.98	0\\
61.99	0\\
62	0\\
62.01	0\\
62.02	0\\
62.03	0\\
62.04	0\\
62.05	0\\
62.06	0\\
62.07	0\\
62.08	0\\
62.09	0\\
62.1	0\\
62.11	0\\
62.12	0\\
62.13	0\\
62.14	0\\
62.15	0\\
62.16	0\\
62.17	0\\
62.18	0\\
62.19	0\\
62.2	0\\
62.21	0\\
62.22	0\\
62.23	1.73472347597681e-18\\
62.24	0\\
62.25	0\\
62.26	0\\
62.27	0\\
62.28	0\\
62.29	0\\
62.3	1.73472347597681e-18\\
62.31	0\\
62.32	0\\
62.33	1.73472347597681e-18\\
62.34	0\\
62.35	0\\
62.36	0\\
62.37	0\\
62.38	0\\
62.39	0\\
62.4	0\\
62.41	0\\
62.42	0\\
62.43	0\\
62.44	0\\
62.45	0\\
62.46	0\\
62.47	0\\
62.48	0\\
62.49	0\\
62.5	0\\
62.51	0\\
62.52	0\\
62.53	1.73472347597681e-18\\
62.54	0\\
62.55	0\\
62.56	0\\
62.57	0\\
62.58	0\\
62.59	0\\
62.6	0\\
62.61	1.73472347597681e-18\\
62.62	0\\
62.63	0\\
62.64	0\\
62.65	0\\
62.66	0\\
62.67	0\\
62.68	0\\
62.69	0\\
62.7	0\\
62.71	0\\
62.72	0\\
62.73	0\\
62.74	0\\
62.75	0\\
62.76	0\\
62.77	0\\
62.78	0\\
62.79	1.73472347597681e-18\\
62.8	0\\
62.81	0\\
62.82	0\\
62.83	1.73472347597681e-18\\
62.84	0\\
62.85	0\\
62.86	0\\
62.87	0\\
62.88	0\\
62.89	1.73472347597681e-18\\
62.9	0\\
62.91	0\\
62.92	0\\
62.93	0\\
62.94	0\\
62.95	0\\
62.96	0\\
62.97	0\\
62.98	0\\
62.99	0\\
63	0\\
63.01	0\\
63.02	0\\
63.03	0\\
63.04	0\\
63.05	0\\
63.06	0\\
63.07	0\\
63.08	0\\
63.09	0\\
63.1	0\\
63.11	0\\
63.12	0\\
63.13	0\\
63.14	0\\
63.15	0\\
63.16	0\\
63.17	1.73472347597681e-18\\
63.18	0\\
63.19	0\\
63.2	0\\
63.21	0\\
63.22	0\\
63.23	0\\
63.24	0\\
63.25	1.73472347597681e-18\\
63.26	0\\
63.27	1.73472347597681e-18\\
63.28	0\\
63.29	1.73472347597681e-18\\
63.3	0\\
63.31	1.73472347597681e-18\\
63.32	0\\
63.33	0\\
63.34	0\\
63.35	0\\
63.36	0\\
63.37	0\\
63.38	0\\
63.39	0\\
63.4	0\\
63.41	0\\
63.42	0\\
63.43	0\\
63.44	0\\
63.45	0\\
63.46	0\\
63.47	0\\
63.48	0\\
63.49	0\\
63.5	0\\
63.51	0\\
63.52	0\\
63.53	0\\
63.54	0\\
63.55	0\\
63.56	0\\
63.57	1.73472347597681e-18\\
63.58	0\\
63.59	0\\
63.6	0\\
63.61	0\\
63.62	0\\
63.63	0\\
63.64	0\\
63.65	0\\
63.66	0\\
63.67	0\\
63.68	0\\
63.69	0\\
63.7	0\\
63.71	0\\
63.72	0\\
63.73	0\\
63.74	0\\
63.75	0\\
63.76	0\\
63.77	0\\
63.78	0\\
63.79	0\\
63.8	0\\
63.81	0\\
63.82	0\\
63.83	0\\
63.84	0\\
63.85	0\\
63.86	0\\
63.87	1.73472347597681e-18\\
63.88	0\\
63.89	1.73472347597681e-18\\
63.9	0\\
63.91	0\\
63.92	0\\
63.93	0\\
63.94	0\\
63.95	0\\
63.96	0\\
63.97	0\\
63.98	0\\
63.99	0\\
64	0\\
64.01	0\\
64.02	0\\
64.03	0\\
64.04	0\\
64.05	0\\
64.06	0\\
64.07	0\\
64.08	0\\
64.09	0\\
64.1	0\\
64.11	0\\
64.12	0\\
64.13	0\\
64.14	0\\
64.15	0\\
64.16	0\\
64.17	1.73472347597681e-18\\
64.18	0\\
64.19	0\\
64.2	0\\
64.21	0\\
64.22	0\\
64.23	0\\
64.24	0\\
64.25	0\\
64.26	0\\
64.27	0\\
64.28	0\\
64.29	0\\
64.3	0\\
64.31	0\\
64.32	0\\
64.33	0\\
64.34	0\\
64.35	0\\
64.36	0\\
64.37	0\\
64.38	0\\
64.39	0\\
64.4	0\\
64.41	0\\
64.42	0\\
64.43	0\\
64.44	0\\
64.45	0\\
64.46	0\\
64.47	0\\
64.48	0\\
64.49	0\\
64.5	0\\
64.51	0\\
64.52	0\\
64.53	0\\
64.54	0\\
64.55	0\\
64.56	0\\
64.57	0\\
64.58	0\\
64.59	0\\
64.6	0\\
64.61	0\\
64.62	0\\
64.63	0\\
64.64	1.73472347597681e-18\\
64.65	0\\
64.66	0\\
64.67	0\\
64.68	0\\
64.69	0\\
64.7	1.73472347597681e-18\\
64.71	0\\
64.72	0\\
64.73	0\\
64.74	0\\
64.75	0\\
64.76	0\\
64.77	1.73472347597681e-18\\
64.78	0\\
64.79	0\\
64.8	0\\
64.81	0\\
64.82	0\\
64.83	0\\
64.84	1.73472347597681e-18\\
64.85	0\\
64.86	0\\
64.87	0\\
64.88	0\\
64.89	0\\
64.9	0\\
64.91	0\\
64.92	0\\
64.93	1.73472347597681e-18\\
64.94	0\\
64.95	0\\
64.96	0\\
64.97	0\\
64.98	0\\
64.99	0\\
65	1.73472347597681e-18\\
65.01	0\\
65.02	0\\
65.03	0\\
65.04	0\\
65.05	0\\
65.06	0\\
65.07	0\\
65.08	0\\
65.09	0\\
65.1	0\\
65.11	0\\
65.12	0\\
65.13	0\\
65.14	0\\
65.15	0\\
65.16	1.73472347597681e-18\\
65.17	0\\
65.18	0\\
65.19	0\\
65.2	0\\
65.21	0\\
65.22	0\\
65.23	0\\
65.24	0\\
65.25	1.73472347597681e-18\\
65.26	1.73472347597681e-18\\
65.27	0\\
65.28	0\\
65.29	0\\
65.3	0\\
65.31	0\\
65.32	0\\
65.33	0\\
65.34	0\\
65.35	0\\
65.36	0\\
65.37	0\\
65.38	0\\
65.39	0\\
65.4	0\\
65.41	1.73472347597681e-18\\
65.42	0\\
65.43	0\\
65.44	0\\
65.45	0\\
65.46	1.73472347597681e-18\\
65.47	0\\
65.48	0\\
65.49	0\\
65.5	0\\
65.51	0\\
65.52	0\\
65.53	0\\
65.54	0\\
65.55	0\\
65.56	0\\
65.57	0\\
65.58	0\\
65.59	0\\
65.6	0\\
65.61	0\\
65.62	0\\
65.63	0\\
65.64	0\\
65.65	0\\
65.66	0\\
65.67	0\\
65.68	0\\
65.69	0\\
65.7	0\\
65.71	0\\
65.72	0\\
65.73	0\\
65.74	0\\
65.75	0\\
65.76	0\\
65.77	1.73472347597681e-18\\
65.78	0\\
65.79	0\\
65.8	0\\
65.81	0\\
65.82	1.73472347597681e-18\\
65.83	1.73472347597681e-18\\
65.84	0\\
65.85	0\\
65.86	1.73472347597681e-18\\
65.87	0\\
65.88	0\\
65.89	0\\
65.9	0\\
65.91	0\\
65.92	0\\
65.93	0\\
65.94	0\\
65.95	0\\
65.96	0\\
65.97	0\\
65.98	0\\
65.99	0\\
66	0\\
66.01	0\\
66.02	0\\
66.03	0\\
66.04	1.73472347597681e-18\\
66.05	0\\
66.06	0\\
66.07	0\\
66.08	0\\
66.09	0\\
66.1	0\\
66.11	0\\
66.12	0\\
66.13	1.73472347597681e-18\\
66.14	0\\
66.15	1.73472347597681e-18\\
66.16	0\\
66.17	0\\
66.18	0\\
66.19	0\\
66.2	0\\
66.21	0\\
66.22	0\\
66.23	0\\
66.24	0\\
66.25	0\\
66.26	0\\
66.27	0\\
66.28	0\\
66.29	0\\
66.3	0\\
66.31	0\\
66.32	0\\
66.33	0\\
66.34	0\\
66.35	0\\
66.36	0\\
66.37	0\\
66.38	0\\
66.39	0\\
66.4	1.73472347597681e-18\\
66.41	0\\
66.42	0\\
66.43	0\\
66.44	0\\
66.45	0\\
66.46	0\\
66.47	0\\
66.48	0\\
66.49	0\\
66.5	0\\
66.51	0\\
66.52	0\\
66.53	0\\
66.54	0\\
66.55	0\\
66.56	0\\
66.57	0\\
66.58	0\\
66.59	0\\
66.6	0\\
66.61	0\\
66.62	0\\
66.63	0\\
66.64	0\\
66.65	0\\
66.66	0\\
66.67	0\\
66.68	0\\
66.69	0\\
66.7	0\\
66.71	0\\
66.72	0\\
66.73	0\\
66.74	0\\
66.75	0\\
66.76	0\\
66.77	0\\
66.78	0\\
66.79	1.73472347597681e-18\\
66.8	0\\
66.81	0\\
66.82	0\\
66.83	1.73472347597681e-18\\
66.84	0\\
66.85	0\\
66.86	0\\
66.87	0\\
66.88	0\\
66.89	0\\
66.9	0\\
66.91	0\\
66.92	0\\
66.93	0\\
66.94	0\\
66.95	0\\
66.96	0\\
66.97	1.73472347597681e-18\\
66.98	0\\
66.99	0\\
67	0\\
67.01	0\\
67.02	0\\
67.03	0\\
67.04	0\\
67.05	0\\
67.06	0\\
67.07	0\\
67.08	0\\
67.09	0\\
67.1	0\\
67.11	0\\
67.12	0\\
67.13	0\\
67.14	0\\
67.15	0\\
67.16	0\\
67.17	0\\
67.18	0\\
67.19	0\\
67.2	0\\
67.21	0\\
67.22	0\\
67.23	0\\
67.24	0\\
67.25	0\\
67.26	1.73472347597681e-18\\
67.27	0\\
67.28	0\\
67.29	0\\
67.3	0\\
67.31	0\\
67.32	0\\
67.33	0\\
67.34	0\\
67.35	0\\
67.36	0\\
67.37	0\\
67.38	0\\
67.39	0\\
67.4	0\\
67.41	0\\
67.42	0\\
67.43	0\\
67.44	0\\
67.45	0\\
67.46	0\\
67.47	0\\
67.48	0\\
67.49	0\\
67.5	0\\
67.51	0\\
67.52	0\\
67.53	0\\
67.54	0\\
67.55	0\\
67.56	0\\
67.57	0\\
67.58	0\\
67.59	0\\
67.6	0\\
67.61	0\\
67.62	0\\
67.63	0\\
67.64	0\\
67.65	0\\
67.66	0\\
67.67	0\\
67.68	0\\
67.69	0\\
67.7	0\\
67.71	0\\
67.72	0\\
67.73	0\\
67.74	1.73472347597681e-18\\
67.75	0\\
67.76	0\\
67.77	0\\
67.78	0\\
67.79	0\\
67.8	0\\
67.81	0\\
67.82	0\\
67.83	0\\
67.84	0\\
67.85	0\\
67.86	0\\
67.87	0\\
67.88	0\\
67.89	0\\
67.9	0\\
67.91	0\\
67.92	1.73472347597681e-18\\
67.93	0\\
67.94	0\\
67.95	0\\
67.96	0\\
67.97	0\\
67.98	0\\
67.99	0\\
68	1.73472347597681e-18\\
68.01	0\\
68.02	0\\
68.03	0\\
68.04	0\\
68.05	0\\
68.06	0\\
68.07	0\\
68.08	0\\
68.09	0\\
68.1	0\\
68.11	0\\
68.12	0\\
68.13	0\\
68.14	0\\
68.15	0\\
68.16	0\\
68.17	0\\
68.18	0\\
68.19	0\\
68.2	0\\
68.21	0\\
68.22	0\\
68.23	0\\
68.24	1.73472347597681e-18\\
68.25	0\\
68.26	1.73472347597681e-18\\
68.27	0\\
68.28	0\\
68.29	0\\
68.3	0\\
68.31	0\\
68.32	0\\
68.33	0\\
68.34	0\\
68.35	1.73472347597681e-18\\
68.36	0\\
68.37	0\\
68.38	0\\
68.39	0\\
68.4	0\\
68.41	0\\
68.42	0\\
68.43	0\\
68.44	0\\
68.45	0\\
68.46	0\\
68.47	0\\
68.48	0\\
68.49	0\\
68.5	0\\
68.51	0\\
68.52	0\\
68.53	0\\
68.54	0\\
68.55	0\\
68.56	0\\
68.57	0\\
68.58	0\\
68.59	0\\
68.6	0\\
68.61	0\\
68.62	0\\
68.63	0\\
68.64	0\\
68.65	1.73472347597681e-18\\
68.66	0\\
68.67	0\\
68.68	0\\
68.69	0\\
68.7	0\\
68.71	0\\
68.72	0\\
68.73	0\\
68.74	0\\
68.75	0\\
68.76	0\\
68.77	0\\
68.78	0\\
68.79	0\\
68.8	0\\
68.81	1.73472347597681e-18\\
68.82	0\\
68.83	0\\
68.84	0\\
68.85	0\\
68.86	0\\
68.87	0\\
68.88	0\\
68.89	0\\
68.9	0\\
68.91	0\\
68.92	0\\
68.93	0\\
68.94	0\\
68.95	0\\
68.96	1.73472347597681e-18\\
68.97	0\\
68.98	0\\
68.99	0\\
69	0\\
69.01	0\\
69.02	0\\
69.03	0\\
69.04	0\\
69.05	0\\
69.06	0\\
69.07	0\\
69.08	0\\
69.09	0\\
69.1	0\\
69.11	0\\
69.12	0\\
69.13	0\\
69.14	0\\
69.15	0\\
69.16	1.73472347597681e-18\\
69.17	0\\
69.18	0\\
69.19	0\\
69.2	0\\
69.21	1.73472347597681e-18\\
69.22	0\\
69.23	0\\
69.24	0\\
69.25	0\\
69.26	0\\
69.27	0\\
69.28	0\\
69.29	0\\
69.3	0\\
69.31	0\\
69.32	0\\
69.33	0\\
69.34	0\\
69.35	0\\
69.36	0\\
69.37	0\\
69.38	0\\
69.39	0\\
69.4	0\\
69.41	0\\
69.42	0\\
69.43	0\\
69.44	0\\
69.45	0\\
69.46	0\\
69.47	0\\
69.48	0\\
69.49	0\\
69.5	0\\
69.51	0\\
69.52	0\\
69.53	0\\
69.54	0\\
69.55	0\\
69.56	0\\
69.57	0\\
69.58	0\\
69.59	0\\
69.6	0\\
69.61	0\\
69.62	0\\
69.63	0\\
69.64	0\\
69.65	0\\
69.66	0\\
69.67	0\\
69.68	1.73472347597681e-18\\
69.69	0\\
69.7	0\\
69.71	0\\
69.72	0\\
69.73	0\\
69.74	0\\
69.75	0\\
69.76	0\\
69.77	0\\
69.78	1.73472347597681e-18\\
69.79	0\\
69.8	0\\
69.81	0\\
69.82	0\\
69.83	0\\
69.84	0\\
69.85	0\\
69.86	0\\
69.87	0\\
69.88	0\\
69.89	0\\
69.9	0\\
69.91	0\\
69.92	0\\
69.93	0\\
69.94	0\\
69.95	0\\
69.96	0\\
69.97	0\\
69.98	0\\
69.99	0\\
70	0\\
70.01	0\\
70.02	0\\
70.03	0\\
70.04	0\\
70.05	0\\
70.06	0\\
70.07	0\\
70.08	1.73472347597681e-18\\
70.09	0\\
70.1	0\\
70.11	0\\
70.12	0\\
70.13	0\\
70.14	0\\
70.15	0\\
70.16	0\\
70.17	0\\
70.18	1.73472347597681e-18\\
70.19	0\\
70.2	1.73472347597681e-18\\
70.21	0\\
70.22	0\\
70.23	0\\
70.24	0\\
70.25	0\\
70.26	0\\
70.27	0\\
70.28	0\\
70.29	0\\
70.3	0\\
70.31	0\\
70.32	0\\
70.33	0\\
70.34	0\\
70.35	0\\
70.36	0\\
70.37	0\\
70.38	0\\
70.39	0\\
70.4	0\\
70.41	1.73472347597681e-18\\
70.42	0\\
70.43	0\\
70.44	0\\
70.45	0\\
70.46	0\\
70.47	0\\
70.48	0\\
70.49	0\\
70.5	0\\
70.51	0\\
70.52	0\\
70.53	0\\
70.54	0\\
70.55	0\\
70.56	0\\
70.57	0\\
70.58	0\\
70.59	0\\
70.6	0\\
70.61	0\\
70.62	0\\
70.63	0\\
70.64	0\\
70.65	0\\
70.66	0\\
70.67	0\\
70.68	0\\
70.69	1.73472347597681e-18\\
70.7	0\\
70.71	0\\
70.72	0\\
70.73	0\\
70.74	0\\
70.75	0\\
70.76	0\\
70.77	0\\
70.78	0\\
70.79	0\\
70.8	0\\
70.81	0\\
70.82	0\\
70.83	0\\
70.84	0\\
70.85	0\\
70.86	0\\
70.87	0\\
70.88	1.73472347597681e-18\\
70.89	0\\
70.9	0\\
70.91	1.73472347597681e-18\\
70.92	0\\
70.93	0\\
70.94	0\\
70.95	0\\
70.96	0\\
70.97	0\\
70.98	0\\
70.99	0\\
71	1.73472347597681e-18\\
71.01	0\\
71.02	0\\
71.03	0\\
71.04	0\\
71.05	0\\
71.06	0\\
71.07	0\\
71.08	0\\
71.09	0\\
71.1	0\\
71.11	0\\
71.12	0\\
71.13	0\\
71.14	0\\
71.15	0\\
71.16	0\\
71.17	1.73472347597681e-18\\
71.18	0\\
71.19	0\\
71.2	0\\
71.21	0\\
71.22	0\\
71.23	0\\
71.24	0\\
71.25	0\\
71.26	0\\
71.27	0\\
71.28	0\\
71.29	0\\
71.3	0\\
71.31	0\\
71.32	0\\
71.33	0\\
71.34	0\\
71.35	0\\
71.36	0\\
71.37	0\\
71.38	0\\
71.39	0\\
71.4	0\\
71.41	0\\
71.42	0\\
71.43	0\\
71.44	0\\
71.45	0\\
71.46	0\\
71.47	0\\
71.48	0\\
71.49	0\\
71.5	0\\
71.51	0\\
71.52	0\\
71.53	0\\
71.54	0\\
71.55	0\\
71.56	0\\
71.57	0\\
71.58	1.73472347597681e-18\\
71.59	0\\
71.6	0\\
71.61	0\\
71.62	0\\
71.63	0\\
71.64	0\\
71.65	0\\
71.66	0\\
71.67	0\\
71.68	0\\
71.69	0\\
71.7	0\\
71.71	0\\
71.72	0\\
71.73	0\\
71.74	0\\
71.75	1.73472347597681e-18\\
71.76	0\\
71.77	0\\
71.78	0\\
71.79	0\\
71.8	0\\
71.81	0\\
71.82	0\\
71.83	0\\
71.84	0\\
71.85	0\\
71.86	0\\
71.87	0\\
71.88	0\\
71.89	0\\
71.9	0\\
71.91	0\\
71.92	0\\
71.93	0\\
71.94	0\\
71.95	0\\
71.96	0\\
71.97	0\\
71.98	0\\
71.99	0\\
72	0\\
72.01	0\\
72.02	0\\
72.03	0\\
72.04	0\\
72.05	0\\
72.06	0\\
72.07	0\\
72.08	0\\
72.09	0\\
72.1	0\\
72.11	0\\
72.12	1.73472347597681e-18\\
72.13	0\\
72.14	0\\
72.15	0\\
72.16	0\\
72.17	0\\
72.18	0\\
72.19	1.73472347597681e-18\\
72.2	0\\
72.21	1.73472347597681e-18\\
72.22	0\\
72.23	0\\
72.24	0\\
72.25	0\\
72.26	0\\
72.27	1.73472347597681e-18\\
72.28	0\\
72.29	0\\
72.3	0\\
72.31	0\\
72.32	0\\
72.33	0\\
72.34	0\\
72.35	0\\
72.36	0\\
72.37	0\\
72.38	0\\
72.39	0\\
72.4	0\\
72.41	0\\
72.42	0\\
72.43	1.73472347597681e-18\\
72.44	0\\
72.45	0\\
72.46	0\\
72.47	0\\
72.48	0\\
72.49	0\\
72.5	1.73472347597681e-18\\
72.51	0\\
72.52	0\\
72.53	0\\
72.54	0\\
72.55	0\\
72.56	0\\
72.57	0\\
72.58	0\\
72.59	0\\
72.6	0\\
72.61	0\\
72.62	0\\
72.63	0\\
72.64	0\\
72.65	0\\
72.66	0\\
72.67	0\\
72.68	0\\
72.69	0\\
72.7	1.73472347597681e-18\\
72.71	0\\
72.72	0\\
72.73	0\\
72.74	0\\
72.75	0\\
72.76	0\\
72.77	0\\
72.78	0\\
72.79	0\\
72.8	0\\
72.81	1.73472347597681e-18\\
72.82	0\\
72.83	0\\
72.84	0\\
72.85	0\\
72.86	0\\
72.87	0\\
72.88	0\\
72.89	0\\
72.9	0\\
72.91	1.73472347597681e-18\\
72.92	0\\
72.93	0\\
72.94	0\\
72.95	0\\
72.96	0\\
72.97	0\\
72.98	0\\
72.99	0\\
73	0\\
73.01	0\\
73.02	0\\
73.03	0\\
73.04	0\\
73.05	1.73472347597681e-18\\
73.06	0\\
73.07	0\\
73.08	0\\
73.09	0\\
73.1	0\\
73.11	0\\
73.12	0\\
73.13	0\\
73.14	0\\
73.15	0\\
73.16	0\\
73.17	0\\
73.18	0\\
73.19	1.73472347597681e-18\\
73.2	0\\
73.21	0\\
73.22	0\\
73.23	0\\
73.24	1.73472347597681e-18\\
73.25	0\\
73.26	0\\
73.27	0\\
73.28	0\\
73.29	0\\
73.3	0\\
73.31	0\\
73.32	0\\
73.33	0\\
73.34	0\\
73.35	0\\
73.36	0\\
73.37	0\\
73.38	0\\
73.39	0\\
73.4	0\\
73.41	0\\
73.42	0\\
73.43	0\\
73.44	0\\
73.45	0\\
73.46	0\\
73.47	0\\
73.48	0\\
73.49	0\\
73.5	0\\
73.51	0\\
73.52	0\\
73.53	0\\
73.54	1.73472347597681e-18\\
73.55	0\\
73.56	0\\
73.57	0\\
73.58	0\\
73.59	0\\
73.6	0\\
73.61	0\\
73.62	0\\
73.63	0\\
73.64	1.73472347597681e-18\\
73.65	0\\
73.66	0\\
73.67	0\\
73.68	0\\
73.69	1.73472347597681e-18\\
73.7	0\\
73.71	0\\
73.72	0\\
73.73	0\\
73.74	0\\
73.75	0\\
73.76	0\\
73.77	1.73472347597681e-18\\
73.78	0\\
73.79	0\\
73.8	0\\
73.81	0\\
73.82	0\\
73.83	0\\
73.84	0\\
73.85	0\\
73.86	0\\
73.87	0\\
73.88	0\\
73.89	0\\
73.9	0\\
73.91	0\\
73.92	0\\
73.93	0\\
73.94	0\\
73.95	0\\
73.96	0\\
73.97	0\\
73.98	0\\
73.99	0\\
74	0\\
74.01	0\\
74.02	0\\
74.03	0\\
74.04	0\\
74.05	0\\
74.06	0\\
74.07	0\\
74.08	1.73472347597681e-18\\
74.09	0\\
74.1	0\\
74.11	0\\
74.12	1.73472347597681e-18\\
74.13	0\\
74.14	0\\
74.15	1.73472347597681e-18\\
74.16	1.73472347597681e-18\\
74.17	0\\
74.18	0\\
74.19	0\\
74.2	0\\
74.21	0\\
74.22	0\\
74.23	0\\
74.24	0\\
74.25	0\\
74.26	0\\
74.27	0\\
74.28	1.73472347597681e-18\\
74.29	0\\
74.3	0\\
74.31	0\\
74.32	0\\
74.33	0\\
74.34	0\\
74.35	0\\
74.36	0\\
74.37	0\\
74.38	0\\
74.39	0\\
74.4	0\\
74.41	0\\
74.42	0\\
74.43	0\\
74.44	0\\
74.45	0\\
74.46	0\\
74.47	0\\
74.48	0\\
74.49	0\\
74.5	0\\
74.51	0\\
74.52	0\\
74.53	0\\
74.54	0\\
74.55	0\\
74.56	0\\
74.57	0\\
74.58	0\\
74.59	0\\
74.6	0\\
74.61	0\\
74.62	0\\
74.63	0\\
74.64	0\\
74.65	1.73472347597681e-18\\
74.66	0\\
74.67	1.73472347597681e-18\\
74.68	0\\
74.69	0\\
74.7	0\\
74.71	0\\
74.72	0\\
74.73	0\\
74.74	0\\
74.75	0\\
74.76	0\\
74.77	0\\
74.78	0\\
74.79	0\\
74.8	0\\
74.81	0\\
74.82	0\\
74.83	0\\
74.84	0\\
74.85	0\\
74.86	0\\
74.87	0\\
74.88	0\\
74.89	1.73472347597681e-18\\
74.9	1.73472347597681e-18\\
74.91	1.73472347597681e-18\\
74.92	0\\
74.93	0\\
74.94	1.73472347597681e-18\\
74.95	0\\
74.96	0\\
74.97	0\\
74.98	0\\
74.99	0\\
75	0\\
75.01	0\\
75.02	1.73472347597681e-18\\
75.03	0\\
75.04	0\\
75.05	0\\
75.06	0\\
75.07	0\\
75.08	0\\
75.09	0\\
75.1	0\\
75.11	0\\
75.12	0\\
75.13	0\\
75.14	0\\
75.15	0\\
75.16	0\\
75.17	0\\
75.18	0\\
75.19	0\\
75.2	0\\
75.21	0\\
75.22	0\\
75.23	0\\
75.24	0\\
75.25	0\\
75.26	0\\
75.27	0\\
75.28	0\\
75.29	0\\
75.3	0\\
75.31	0\\
75.32	0\\
75.33	0\\
75.34	0\\
75.35	0\\
75.36	0\\
75.37	0\\
75.38	0\\
75.39	0\\
75.4	1.73472347597681e-18\\
75.41	0\\
75.42	0\\
75.43	0\\
75.44	0\\
75.45	0\\
75.46	0\\
75.47	0\\
75.48	0\\
75.49	0\\
75.5	0\\
75.51	0\\
75.52	0\\
75.53	0\\
75.54	0\\
75.55	0\\
75.56	0\\
75.57	0\\
75.58	0\\
75.59	0\\
75.6	0\\
75.61	0\\
75.62	0\\
75.63	0\\
75.64	0\\
75.65	0\\
75.66	0\\
75.67	1.73472347597681e-18\\
75.68	0\\
75.69	0\\
75.7	0\\
75.71	0\\
75.72	0\\
75.73	0\\
75.74	1.73472347597681e-18\\
75.75	0\\
75.76	0\\
75.77	0\\
75.78	0\\
75.79	0\\
75.8	0\\
75.81	0\\
75.82	0\\
75.83	0\\
75.84	0\\
75.85	0\\
75.86	0\\
75.87	0\\
75.88	0\\
75.89	1.73472347597681e-18\\
75.9	0\\
75.91	0\\
75.92	0\\
75.93	0\\
75.94	0\\
75.95	0\\
75.96	0\\
75.97	0\\
75.98	0\\
75.99	0\\
76	0\\
76.01	0\\
76.02	0\\
76.03	0\\
76.04	0\\
76.05	0\\
76.06	0\\
76.07	0\\
76.08	0\\
76.09	0\\
76.1	0\\
76.11	0\\
76.12	0\\
76.13	0\\
76.14	0\\
76.15	1.73472347597681e-18\\
76.16	0\\
76.17	0\\
76.18	0\\
76.19	0\\
76.2	0\\
76.21	1.73472347597681e-18\\
76.22	1.73472347597681e-18\\
76.23	0\\
76.24	0\\
76.25	0\\
76.26	0\\
76.27	0\\
76.28	0\\
76.29	0\\
76.3	0\\
76.31	0\\
76.32	0\\
76.33	0\\
76.34	0\\
76.35	0\\
76.36	0\\
76.37	0\\
76.38	0\\
76.39	0\\
76.4	0\\
76.41	0\\
76.42	0\\
76.43	0\\
76.44	0\\
76.45	0\\
76.46	1.73472347597681e-18\\
76.47	0\\
76.48	0\\
76.49	0\\
76.5	0\\
76.51	0\\
76.52	0\\
76.53	0\\
76.54	1.73472347597681e-18\\
76.55	0\\
76.56	0\\
76.57	0\\
76.58	0\\
76.59	0\\
76.6	0\\
76.61	0\\
76.62	0\\
76.63	0\\
76.64	0\\
76.65	0\\
76.66	0\\
76.67	0\\
76.68	0\\
76.69	0\\
76.7	0\\
76.71	0\\
76.72	0\\
76.73	1.73472347597681e-18\\
76.74	0\\
76.75	0\\
76.76	0\\
76.77	0\\
76.78	0\\
76.79	0\\
76.8	0\\
76.81	0\\
76.82	0\\
76.83	1.73472347597681e-18\\
76.84	0\\
76.85	0\\
76.86	0\\
76.87	0\\
76.88	0\\
76.89	0\\
76.9	0\\
76.91	0\\
76.92	1.73472347597681e-18\\
76.93	0\\
76.94	0\\
76.95	0\\
76.96	0\\
76.97	1.73472347597681e-18\\
76.98	0\\
76.99	0\\
77	1.73472347597681e-18\\
77.01	1.73472347597681e-18\\
77.02	0\\
77.03	0\\
77.04	0\\
77.05	0\\
77.06	0\\
77.07	0\\
77.08	1.73472347597681e-18\\
77.09	0\\
77.1	0\\
77.11	0\\
77.12	0\\
77.13	0\\
77.14	0\\
77.15	0\\
77.16	0\\
77.17	0\\
77.18	0\\
77.19	0\\
77.2	0\\
77.21	0\\
77.22	0\\
77.23	0\\
77.24	0\\
77.25	0\\
77.26	0\\
77.27	0\\
77.28	0\\
77.29	0\\
77.3	0\\
77.31	0\\
77.32	0\\
77.33	0\\
77.34	0\\
77.35	0\\
77.36	0\\
77.37	1.73472347597681e-18\\
77.38	0\\
77.39	0\\
77.4	0\\
77.41	1.73472347597681e-18\\
77.42	0\\
77.43	0\\
77.44	0\\
77.45	0\\
77.46	0\\
77.47	0\\
77.48	0\\
77.49	0\\
77.5	0\\
77.51	0\\
77.52	0\\
77.53	0\\
77.54	0\\
77.55	0\\
77.56	0\\
77.57	0\\
77.58	0\\
77.59	0\\
77.6	0\\
77.61	0\\
77.62	0\\
77.63	0\\
77.64	0\\
77.65	0\\
77.66	0\\
77.67	0\\
77.68	0\\
77.69	0\\
77.7	0\\
77.71	0\\
77.72	0\\
77.73	0\\
77.74	0\\
77.75	0\\
77.76	0\\
77.77	0\\
77.78	0\\
77.79	1.73472347597681e-18\\
77.8	0\\
77.81	0\\
77.82	0\\
77.83	0\\
77.84	0\\
77.85	0\\
77.86	0\\
77.87	0\\
77.88	0\\
77.89	0\\
77.9	0\\
77.91	1.73472347597681e-18\\
77.92	0\\
77.93	0\\
77.94	0\\
77.95	1.73472347597681e-18\\
77.96	0\\
77.97	1.73472347597681e-18\\
77.98	1.73472347597681e-18\\
77.99	0\\
78	0\\
78.01	0\\
78.02	0\\
78.03	0\\
78.04	0\\
78.05	0\\
78.06	0\\
78.07	0\\
78.08	0\\
78.09	0\\
78.1	0\\
78.11	0\\
78.12	0\\
78.13	0\\
78.14	0\\
78.15	0\\
78.16	0\\
78.17	0\\
78.18	0\\
78.19	0\\
78.2	1.73472347597681e-18\\
78.21	0\\
78.22	0\\
78.23	0\\
78.24	0\\
78.25	0\\
78.26	0\\
78.27	0\\
78.28	0\\
78.29	0\\
78.3	0\\
78.31	0\\
78.32	1.73472347597681e-18\\
78.33	0\\
78.34	0\\
78.35	0\\
78.36	0\\
78.37	0\\
78.38	0\\
78.39	0\\
78.4	0\\
78.41	0\\
78.42	1.73472347597681e-18\\
78.43	0\\
78.44	1.73472347597681e-18\\
78.45	0\\
78.46	0\\
78.47	0\\
78.48	0\\
78.49	0\\
78.5	0\\
78.51	0\\
78.52	0\\
78.53	0\\
78.54	0\\
78.55	0\\
78.56	0\\
78.57	0\\
78.58	0\\
78.59	0\\
78.6	0\\
78.61	0\\
78.62	0\\
78.63	0\\
78.64	0\\
78.65	0\\
78.66	0\\
78.67	0\\
78.68	1.73472347597681e-18\\
78.69	0\\
78.7	0\\
78.71	0\\
78.72	0\\
78.73	0\\
78.74	0\\
78.75	0\\
78.76	0\\
78.77	0\\
78.78	0\\
78.79	0\\
78.8	0\\
78.81	0\\
78.82	0\\
78.83	0\\
78.84	0\\
78.85	0\\
78.86	0\\
78.87	0\\
78.88	0\\
78.89	0\\
78.9	0\\
78.91	1.73472347597681e-18\\
78.92	0\\
78.93	0\\
78.94	0\\
78.95	0\\
78.96	0\\
78.97	0\\
78.98	0\\
78.99	0\\
79	0\\
79.01	0\\
79.02	0\\
79.03	0\\
79.04	0\\
79.05	0\\
79.06	0\\
79.07	0\\
79.08	0\\
79.09	0\\
79.1	0\\
79.11	0\\
79.12	0\\
79.13	0\\
79.14	0\\
79.15	0\\
79.16	0\\
79.17	0\\
79.18	0\\
79.19	0\\
79.2	0\\
79.21	0\\
79.22	0\\
79.23	0\\
79.24	0\\
79.25	0\\
79.26	0\\
79.27	0\\
79.28	0\\
79.29	0\\
79.3	0\\
79.31	0\\
79.32	0\\
79.33	0\\
79.34	0\\
79.35	0\\
79.36	0\\
79.37	0\\
79.38	0\\
79.39	0\\
79.4	0\\
79.41	0\\
79.42	0\\
79.43	0\\
79.44	0\\
79.45	0\\
79.46	0\\
79.47	0\\
79.48	0\\
79.49	0\\
79.5	0\\
79.51	0\\
79.52	0\\
79.53	0\\
79.54	0\\
79.55	1.73472347597681e-18\\
79.56	0\\
79.57	0\\
79.58	0\\
79.59	0\\
79.6	0\\
79.61	0\\
79.62	0\\
79.63	0\\
79.64	0\\
79.65	0\\
79.66	0\\
79.67	0\\
79.68	0\\
79.69	0\\
79.7	0\\
79.71	0\\
79.72	1.73472347597681e-18\\
79.73	1.73472347597681e-18\\
79.74	0\\
79.75	0\\
79.76	0\\
79.77	0\\
79.78	0\\
79.79	0\\
79.8	0\\
79.81	0\\
79.82	0\\
79.83	0\\
79.84	0\\
79.85	0\\
79.86	0\\
79.87	0\\
79.88	1.73472347597681e-18\\
79.89	0\\
79.9	0\\
79.91	0\\
79.92	0\\
79.93	0\\
79.94	0\\
79.95	0\\
79.96	1.73472347597681e-18\\
79.97	0\\
79.98	0\\
79.99	0\\
80	0\\
80.01	0\\
};
\addplot [color=red,solid]
  table[row sep=crcr]{%
80.01	0\\
80.02	0\\
80.03	0\\
80.04	0\\
80.05	0\\
80.06	0\\
80.07	0\\
80.08	0\\
80.09	1.73472347597681e-18\\
80.1	0\\
80.11	0\\
80.12	0\\
80.13	0\\
80.14	0\\
80.15	0\\
80.16	0\\
80.17	0\\
80.18	1.73472347597681e-18\\
80.19	0\\
80.2	0\\
80.21	0\\
80.22	0\\
80.23	0\\
80.24	0\\
80.25	0\\
80.26	0\\
80.27	0\\
80.28	0\\
80.29	0\\
80.3	0\\
80.31	0\\
80.32	0\\
80.33	0\\
80.34	0\\
80.35	0\\
80.36	0\\
80.37	0\\
80.38	0\\
80.39	0\\
80.4	0\\
80.41	0\\
80.42	0\\
80.43	0\\
80.44	0\\
80.45	0\\
80.46	0\\
80.47	0\\
80.48	0\\
80.49	0\\
80.5	1.73472347597681e-18\\
80.51	0\\
80.52	0\\
80.53	0\\
80.54	1.73472347597681e-18\\
80.55	0\\
80.56	0\\
80.57	0\\
80.58	0\\
80.59	0\\
80.6	0\\
80.61	0\\
80.62	0\\
80.63	0\\
80.64	0\\
80.65	0\\
80.66	0\\
80.67	0\\
80.68	0\\
80.69	0\\
80.7	0\\
80.71	0\\
80.72	0\\
80.73	0\\
80.74	0\\
80.75	0\\
80.76	0\\
80.77	0\\
80.78	0\\
80.79	0\\
80.8	0\\
80.81	1.73472347597681e-18\\
80.82	0\\
80.83	0\\
80.84	0\\
80.85	0\\
80.86	0\\
80.87	0\\
80.88	0\\
80.89	0\\
80.9	0\\
80.91	0\\
80.92	0\\
80.93	0\\
80.94	0\\
80.95	0\\
80.96	1.73472347597681e-18\\
80.97	0\\
80.98	0\\
80.99	0\\
81	0\\
81.01	0\\
81.02	0\\
81.03	0\\
81.04	0\\
81.05	0\\
81.06	0\\
81.07	0\\
81.08	0\\
81.09	0\\
81.1	0\\
81.11	0\\
81.12	0\\
81.13	1.73472347597681e-18\\
81.14	0\\
81.15	0\\
81.16	0\\
81.17	0\\
81.18	0\\
81.19	0\\
81.2	0\\
81.21	0\\
81.22	0\\
81.23	0\\
81.24	0\\
81.25	0\\
81.26	0\\
81.27	0\\
81.28	0\\
81.29	0\\
81.3	0\\
81.31	0\\
81.32	0\\
81.33	0\\
81.34	0\\
81.35	0\\
81.36	0\\
81.37	1.73472347597681e-18\\
81.38	0\\
81.39	0\\
81.4	0\\
81.41	0\\
81.42	0\\
81.43	0\\
81.44	0\\
81.45	0\\
81.46	0\\
81.47	0\\
81.48	0\\
81.49	0\\
81.5	0\\
81.51	0\\
81.52	0\\
81.53	0\\
81.54	0\\
81.55	0\\
81.56	0\\
81.57	0\\
81.58	0\\
81.59	0\\
81.6	0\\
81.61	0\\
81.62	0\\
81.63	0\\
81.64	0\\
81.65	0\\
81.66	0\\
81.67	0\\
81.68	0\\
81.69	0\\
81.7	0\\
81.71	1.73472347597681e-18\\
81.72	0\\
81.73	0\\
81.74	0\\
81.75	0\\
81.76	0\\
81.77	0\\
81.78	0\\
81.79	0\\
81.8	0\\
81.81	0\\
81.82	0\\
81.83	0\\
81.84	0\\
81.85	0\\
81.86	0\\
81.87	0\\
81.88	0\\
81.89	0\\
81.9	0\\
81.91	0\\
81.92	0\\
81.93	0\\
81.94	0\\
81.95	0\\
81.96	0\\
81.97	0\\
81.98	0\\
81.99	0\\
82	0\\
82.01	0\\
82.02	0\\
82.03	0\\
82.04	0\\
82.05	0\\
82.06	0\\
82.07	0\\
82.08	0\\
82.09	0\\
82.1	0\\
82.11	0\\
82.12	0\\
82.13	0\\
82.14	0\\
82.15	0\\
82.16	0\\
82.17	0\\
82.18	0\\
82.19	0\\
82.2	0\\
82.21	0\\
82.22	0\\
82.23	0\\
82.24	0\\
82.25	0\\
82.26	0\\
82.27	0\\
82.28	0\\
82.29	0\\
82.3	0\\
82.31	0\\
82.32	0\\
82.33	0\\
82.34	0\\
82.35	0\\
82.36	0\\
82.37	0\\
82.38	0\\
82.39	0\\
82.4	0\\
82.41	0\\
82.42	0\\
82.43	0\\
82.44	0\\
82.45	0\\
82.46	0\\
82.47	1.73472347597681e-18\\
82.48	0\\
82.49	0\\
82.5	0\\
82.51	0\\
82.52	0\\
82.53	0\\
82.54	0\\
82.55	0\\
82.56	0\\
82.57	0\\
82.58	0\\
82.59	0\\
82.6	0\\
82.61	0\\
82.62	0\\
82.63	0\\
82.64	1.73472347597681e-18\\
82.65	0\\
82.66	0\\
82.67	0\\
82.68	0\\
82.69	0\\
82.7	0\\
82.71	1.73472347597681e-18\\
82.72	0\\
82.73	0\\
82.74	0\\
82.75	0\\
82.76	0\\
82.77	0\\
82.78	0\\
82.79	0\\
82.8	0\\
82.81	0\\
82.82	0\\
82.83	0\\
82.84	0\\
82.85	0\\
82.86	0\\
82.87	0\\
82.88	0\\
82.89	0\\
82.9	0\\
82.91	0\\
82.92	0\\
82.93	0\\
82.94	0\\
82.95	0\\
82.96	0\\
82.97	0\\
82.98	0\\
82.99	0\\
83	0\\
83.01	0\\
83.02	0\\
83.03	0\\
83.04	0\\
83.05	0\\
83.06	0\\
83.07	0\\
83.08	0\\
83.09	0\\
83.1	0\\
83.11	0\\
83.12	1.73472347597681e-18\\
83.13	0\\
83.14	0\\
83.15	0\\
83.16	0\\
83.17	0\\
83.18	0\\
83.19	0\\
83.2	1.73472347597681e-18\\
83.21	0\\
83.22	0\\
83.23	0\\
83.24	0\\
83.25	0\\
83.26	0\\
83.27	0\\
83.28	0\\
83.29	0\\
83.3	0\\
83.31	0\\
83.32	0\\
83.33	0\\
83.34	0\\
83.35	0\\
83.36	0\\
83.37	0\\
83.38	0\\
83.39	0\\
83.4	0\\
83.41	0\\
83.42	0\\
83.43	0\\
83.44	0\\
83.45	0\\
83.46	1.73472347597681e-18\\
83.47	0\\
83.48	0\\
83.49	0\\
83.5	0\\
83.51	0\\
83.52	0\\
83.53	0\\
83.54	0\\
83.55	0\\
83.56	0\\
83.57	0\\
83.58	0\\
83.59	0\\
83.6	0\\
83.61	0\\
83.62	0\\
83.63	0\\
83.64	0\\
83.65	0\\
83.66	0\\
83.67	0\\
83.68	0\\
83.69	0\\
83.7	0\\
83.71	0\\
83.72	0\\
83.73	0\\
83.74	0\\
83.75	0\\
83.76	0\\
83.77	0\\
83.78	0\\
83.79	0\\
83.8	0\\
83.81	0\\
83.82	0\\
83.83	0\\
83.84	0\\
83.85	0\\
83.86	0\\
83.87	0\\
83.88	0\\
83.89	0\\
83.9	0\\
83.91	0\\
83.92	0\\
83.93	0\\
83.94	0\\
83.95	0\\
83.96	0\\
83.97	0\\
83.98	0\\
83.99	0\\
84	0\\
84.01	0\\
84.02	0\\
84.03	0\\
84.04	0\\
84.05	0\\
84.06	0\\
84.07	0\\
84.08	0\\
84.09	0\\
84.1	0\\
84.11	0\\
84.12	0\\
84.13	0\\
84.14	0\\
84.15	0\\
84.16	0\\
84.17	0\\
84.18	0\\
84.19	0\\
84.2	0\\
84.21	0\\
84.22	0\\
84.23	0\\
84.24	0\\
84.25	0\\
84.26	0\\
84.27	0\\
84.28	0\\
84.29	0\\
84.3	0\\
84.31	0\\
84.32	1.73472347597681e-18\\
84.33	0\\
84.34	0\\
84.35	1.73472347597681e-18\\
84.36	0\\
84.37	0\\
84.38	0\\
84.39	0\\
84.4	0\\
84.41	0\\
84.42	0\\
84.43	0\\
84.44	0\\
84.45	0\\
84.46	0\\
84.47	0\\
84.48	0\\
84.49	0\\
84.5	0\\
84.51	0\\
84.52	0\\
84.53	0\\
84.54	0\\
84.55	0\\
84.56	0\\
84.57	0\\
84.58	0\\
84.59	0\\
84.6	0\\
84.61	0\\
84.62	0\\
84.63	0\\
84.64	0\\
84.65	0\\
84.66	0\\
84.67	0\\
84.68	0\\
84.69	0\\
84.7	0\\
84.71	0\\
84.72	0\\
84.73	0\\
84.74	0\\
84.75	0\\
84.76	0\\
84.77	0\\
84.78	0\\
84.79	1.73472347597681e-18\\
84.8	0\\
84.81	0\\
84.82	0\\
84.83	0\\
84.84	0\\
84.85	0\\
84.86	0\\
84.87	0\\
84.88	0\\
84.89	0\\
84.9	0\\
84.91	0\\
84.92	0\\
84.93	0\\
84.94	1.73472347597681e-18\\
84.95	0\\
84.96	0\\
84.97	0\\
84.98	0\\
84.99	0\\
85	0\\
85.01	0\\
85.02	1.73472347597681e-18\\
85.03	0\\
85.04	0\\
85.05	0\\
85.06	0\\
85.07	0\\
85.08	0\\
85.09	0\\
85.1	0\\
85.11	0\\
85.12	0\\
85.13	0\\
85.14	0\\
85.15	0\\
85.16	0\\
85.17	0\\
85.18	0\\
85.19	0\\
85.2	0\\
85.21	0\\
85.22	0\\
85.23	0\\
85.24	0\\
85.25	0\\
85.26	0\\
85.27	0\\
85.28	0\\
85.29	0\\
85.3	0\\
85.31	0\\
85.32	0\\
85.33	0\\
85.34	0\\
85.35	0\\
85.36	0\\
85.37	1.73472347597681e-18\\
85.38	0\\
85.39	0\\
85.4	0\\
85.41	1.73472347597681e-18\\
85.42	0\\
85.43	0\\
85.44	0\\
85.45	0\\
85.46	0\\
85.47	0\\
85.48	0\\
85.49	0\\
85.5	0\\
85.51	0\\
85.52	0\\
85.53	0\\
85.54	1.73472347597681e-18\\
85.55	0\\
85.56	1.73472347597681e-18\\
85.57	0\\
85.58	0\\
85.59	0\\
85.6	0\\
85.61	0\\
85.62	0\\
85.63	0\\
85.64	0\\
85.65	0\\
85.66	0\\
85.67	0\\
85.68	0\\
85.69	0\\
85.7	0\\
85.71	0\\
85.72	0\\
85.73	0\\
85.74	0\\
85.75	1.73472347597681e-18\\
85.76	0\\
85.77	1.73472347597681e-18\\
85.78	0\\
85.79	0\\
85.8	0\\
85.81	1.73472347597681e-18\\
85.82	0\\
85.83	1.73472347597681e-18\\
85.84	0\\
85.85	0\\
85.86	0\\
85.87	0\\
85.88	0\\
85.89	0\\
85.9	0\\
85.91	0\\
85.92	0\\
85.93	0\\
85.94	0\\
85.95	0\\
85.96	0\\
85.97	0\\
85.98	0\\
85.99	0\\
86	0\\
86.01	0\\
86.02	0\\
86.03	0\\
86.04	0\\
86.05	0\\
86.06	0\\
86.07	0\\
86.08	1.73472347597681e-18\\
86.09	0\\
86.1	0\\
86.11	0\\
86.12	0\\
86.13	0\\
86.14	0\\
86.15	0\\
86.16	0\\
86.17	0\\
86.18	0\\
86.19	0\\
86.2	0\\
86.21	0\\
86.22	0\\
86.23	0\\
86.24	0\\
86.25	0\\
86.26	0\\
86.27	0\\
86.28	0\\
86.29	0\\
86.3	0\\
86.31	0\\
86.32	0\\
86.33	0\\
86.34	0\\
86.35	0\\
86.36	0\\
86.37	0\\
86.38	0\\
86.39	0\\
86.4	0\\
86.41	0\\
86.42	1.73472347597681e-18\\
86.43	0\\
86.44	0\\
86.45	0\\
86.46	1.73472347597681e-18\\
86.47	0\\
86.48	0\\
86.49	0\\
86.5	0\\
86.51	0\\
86.52	0\\
86.53	0\\
86.54	0\\
86.55	0\\
86.56	0\\
86.57	0\\
86.58	0\\
86.59	0\\
86.6	0\\
86.61	1.73472347597681e-18\\
86.62	0\\
86.63	0\\
86.64	0\\
86.65	0\\
86.66	0\\
86.67	0\\
86.68	0\\
86.69	0\\
86.7	0\\
86.71	0\\
86.72	0\\
86.73	0\\
86.74	0\\
86.75	1.73472347597681e-18\\
86.76	0\\
86.77	0\\
86.78	0\\
86.79	0\\
86.8	0\\
86.81	0\\
86.82	0\\
86.83	0\\
86.84	0\\
86.85	1.73472347597681e-18\\
86.86	0\\
86.87	0\\
86.88	0\\
86.89	0\\
86.9	0\\
86.91	1.73472347597681e-18\\
86.92	0\\
86.93	0\\
86.94	0\\
86.95	0\\
86.96	0\\
86.97	0\\
86.98	0\\
86.99	0\\
87	0\\
87.01	0\\
87.02	0\\
87.03	0\\
87.04	0\\
87.05	1.73472347597681e-18\\
87.06	1.73472347597681e-18\\
87.07	0\\
87.08	0\\
87.09	0\\
87.1	0\\
87.11	0\\
87.12	0\\
87.13	0\\
87.14	0\\
87.15	0\\
87.16	0\\
87.17	0\\
87.18	0\\
87.19	0\\
87.2	1.73472347597681e-18\\
87.21	0\\
87.22	0\\
87.23	0\\
87.24	0\\
87.25	0\\
87.26	0\\
87.27	0\\
87.28	0\\
87.29	0\\
87.3	0\\
87.31	0\\
87.32	0\\
87.33	0\\
87.34	0\\
87.35	0\\
87.36	0\\
87.37	0\\
87.38	0\\
87.39	0\\
87.4	0\\
87.41	0\\
87.42	0\\
87.43	0\\
87.44	0\\
87.45	0\\
87.46	0\\
87.47	0\\
87.48	0\\
87.49	0\\
87.5	0\\
87.51	0\\
87.52	0\\
87.53	0\\
87.54	0\\
87.55	0\\
87.56	0\\
87.57	0\\
87.58	0\\
87.59	0\\
87.6	0\\
87.61	0\\
87.62	0\\
87.63	0\\
87.64	0\\
87.65	0\\
87.66	0\\
87.67	0\\
87.68	0\\
87.69	0\\
87.7	0\\
87.71	0\\
87.72	0\\
87.73	0\\
87.74	0\\
87.75	0\\
87.76	0\\
87.77	0\\
87.78	0\\
87.79	0\\
87.8	0\\
87.81	0\\
87.82	0\\
87.83	0\\
87.84	0\\
87.85	0\\
87.86	0\\
87.87	0\\
87.88	0\\
87.89	0\\
87.9	0\\
87.91	0\\
87.92	0\\
87.93	0\\
87.94	0\\
87.95	0\\
87.96	0\\
87.97	1.73472347597681e-18\\
87.98	0\\
87.99	0\\
88	0\\
88.01	0\\
88.02	0\\
88.03	0\\
88.04	0\\
88.05	0\\
88.06	0\\
88.07	0\\
88.08	0\\
88.09	0\\
88.1	0\\
88.11	0\\
88.12	0\\
88.13	0\\
88.14	0\\
88.15	0\\
88.16	0\\
88.17	1.73472347597681e-18\\
88.18	0\\
88.19	0\\
88.2	0\\
88.21	0\\
88.22	0\\
88.23	0\\
88.24	0\\
88.25	0\\
88.26	0\\
88.27	0\\
88.28	0\\
88.29	0\\
88.3	0\\
88.31	0\\
88.32	0\\
88.33	0\\
88.34	1.73472347597681e-18\\
88.35	0\\
88.36	0\\
88.37	0\\
88.38	1.73472347597681e-18\\
88.39	0\\
88.4	0\\
88.41	0\\
88.42	0\\
88.43	1.73472347597681e-18\\
88.44	0\\
88.45	0\\
88.46	0\\
88.47	0\\
88.48	0\\
88.49	0\\
88.5	0\\
88.51	0\\
88.52	0\\
88.53	0\\
88.54	0\\
88.55	0\\
88.56	0\\
88.57	0\\
88.58	0\\
88.59	0\\
88.6	0\\
88.61	0\\
88.62	0\\
88.63	0\\
88.64	0\\
88.65	0\\
88.66	1.73472347597681e-18\\
88.67	0\\
88.68	0\\
88.69	1.73472347597681e-18\\
88.7	0\\
88.71	0\\
88.72	0\\
88.73	0\\
88.74	0\\
88.75	0\\
88.76	0\\
88.77	0\\
88.78	0\\
88.79	0\\
88.8	0\\
88.81	0\\
88.82	0\\
88.83	0\\
88.84	0\\
88.85	0\\
88.86	0\\
88.87	0\\
88.88	0\\
88.89	0\\
88.9	0\\
88.91	0\\
88.92	0\\
88.93	0\\
88.94	0\\
88.95	0\\
88.96	0\\
88.97	0\\
88.98	0\\
88.99	0\\
89	0\\
89.01	0\\
89.02	0\\
89.03	0\\
89.04	0\\
89.05	0\\
89.06	0\\
89.07	0\\
89.08	0\\
89.09	0\\
89.1	0\\
89.11	0\\
89.12	0\\
89.13	0\\
89.14	0\\
89.15	0\\
89.16	0\\
89.17	0\\
89.18	0\\
89.19	0\\
89.2	0\\
89.21	0\\
89.22	0\\
89.23	0\\
89.24	0\\
89.25	0\\
89.26	0\\
89.27	0\\
89.28	0\\
89.29	0\\
89.3	0\\
89.31	1.73472347597681e-18\\
89.32	0\\
89.33	0\\
89.34	0\\
89.35	0\\
89.36	0\\
89.37	0\\
89.38	0\\
89.39	0\\
89.4	0\\
89.41	0\\
89.42	0\\
89.43	1.73472347597681e-18\\
89.44	1.73472347597681e-18\\
89.45	0\\
89.46	0\\
89.47	0\\
89.48	0\\
89.49	0\\
89.5	0\\
89.51	0\\
89.52	0\\
89.53	0\\
89.54	0\\
89.55	0\\
89.56	0\\
89.57	0\\
89.58	0\\
89.59	0\\
89.6	0\\
89.61	0\\
89.62	0\\
89.63	0\\
89.64	0\\
89.65	0\\
89.66	0\\
89.67	0\\
89.68	0\\
89.69	0\\
89.7	0\\
89.71	0\\
89.72	0\\
89.73	0\\
89.74	0\\
89.75	1.73472347597681e-18\\
89.76	0\\
89.77	0\\
89.78	0\\
89.79	0\\
89.8	0\\
89.81	1.73472347597681e-18\\
89.82	0\\
89.83	0\\
89.84	0\\
89.85	0\\
89.86	0\\
89.87	0\\
89.88	1.73472347597681e-18\\
89.89	0\\
89.9	0\\
89.91	0\\
89.92	0\\
89.93	0\\
89.94	0\\
89.95	0\\
89.96	0\\
89.97	0\\
89.98	0\\
89.99	0\\
90	0\\
90.01	0\\
90.02	0\\
90.03	0\\
90.04	0\\
90.05	0\\
90.06	0\\
90.07	0\\
90.08	0\\
90.09	0\\
90.1	0\\
90.11	0\\
90.12	0\\
90.13	0\\
90.14	0\\
90.15	0\\
90.16	0\\
90.17	0\\
90.18	0\\
90.19	0\\
90.2	0\\
90.21	0\\
90.22	0\\
90.23	0\\
90.24	0\\
90.25	0\\
90.26	0\\
90.27	0\\
90.28	0\\
90.29	0\\
90.3	0\\
90.31	0\\
90.32	0\\
90.33	0\\
90.34	0\\
90.35	0\\
90.36	0\\
90.37	0\\
90.38	0\\
90.39	0\\
90.4	0\\
90.41	0\\
90.42	0\\
90.43	0\\
90.44	0\\
90.45	0\\
90.46	0\\
90.47	0\\
90.48	0\\
90.49	0\\
90.5	0\\
90.51	0\\
90.52	0\\
90.53	0\\
90.54	1.73472347597681e-18\\
90.55	0\\
90.56	0\\
90.57	0\\
90.58	0\\
90.59	0\\
90.6	0\\
90.61	0\\
90.62	0\\
90.63	0\\
90.64	0\\
90.65	0\\
90.66	0\\
90.67	0\\
90.68	0\\
90.69	0\\
90.7	1.73472347597681e-18\\
90.71	0\\
90.72	0\\
90.73	0\\
90.74	0\\
90.75	0\\
90.76	0\\
90.77	0\\
90.78	0\\
90.79	0\\
90.8	0\\
90.81	0\\
90.82	0\\
90.83	0\\
90.84	0\\
90.85	0\\
90.86	0\\
90.87	0\\
90.88	0\\
90.89	0\\
90.9	0\\
90.91	0\\
90.92	0\\
90.93	0\\
90.94	0\\
90.95	0\\
90.96	0\\
90.97	0\\
90.98	0\\
90.99	0\\
91	0\\
91.01	0\\
91.02	0\\
91.03	0\\
91.04	0\\
91.05	0\\
91.06	0\\
91.07	0\\
91.08	0\\
91.09	0\\
91.1	0\\
91.11	0\\
91.12	0\\
91.13	0\\
91.14	0\\
91.15	0\\
91.16	0\\
91.17	0\\
91.18	0\\
91.19	0\\
91.2	0\\
91.21	0\\
91.22	0\\
91.23	0\\
91.24	0\\
91.25	0\\
91.26	0\\
91.27	0\\
91.28	0\\
91.29	0\\
91.3	0\\
91.31	0\\
91.32	0\\
91.33	0\\
91.34	0\\
91.35	0\\
91.36	0\\
91.37	0\\
91.38	0\\
91.39	0\\
91.4	0\\
91.41	0\\
91.42	0\\
91.43	0\\
91.44	0\\
91.45	0\\
91.46	0\\
91.47	0\\
91.48	0\\
91.49	0\\
91.5	0\\
91.51	0\\
91.52	0\\
91.53	0\\
91.54	0\\
91.55	0\\
91.56	0\\
91.57	0\\
91.58	0\\
91.59	0\\
91.6	0\\
91.61	0\\
91.62	0\\
91.63	0\\
91.64	0\\
91.65	0\\
91.66	0\\
91.67	0\\
91.68	0\\
91.69	0\\
91.7	0\\
91.71	0\\
91.72	0\\
91.73	0\\
91.74	0\\
91.75	0\\
91.76	0\\
91.77	0\\
91.78	0\\
91.79	0\\
91.8	0\\
91.81	0\\
91.82	0\\
91.83	0\\
91.84	0\\
91.85	0\\
91.86	0\\
91.87	0\\
91.88	0\\
91.89	0\\
91.9	0\\
91.91	0\\
91.92	0\\
91.93	0\\
91.94	0\\
91.95	0\\
91.96	0\\
91.97	0\\
91.98	0\\
91.99	0\\
92	0\\
92.01	0\\
92.02	0\\
92.03	0\\
92.04	0\\
92.05	0\\
92.06	0\\
92.07	0\\
92.08	0\\
92.09	0\\
92.1	0\\
92.11	0\\
92.12	0\\
92.13	0\\
92.14	0\\
92.15	0\\
92.16	0\\
92.17	0\\
92.18	0\\
92.19	0\\
92.2	0\\
92.21	0\\
92.22	0\\
92.23	0\\
92.24	0\\
92.25	0\\
92.26	0\\
92.27	0\\
92.28	0\\
92.29	0\\
92.3	0\\
92.31	0\\
92.32	0\\
92.33	0\\
92.34	0\\
92.35	0\\
92.36	0\\
92.37	0\\
92.38	0\\
92.39	0\\
92.4	0\\
92.41	0\\
92.42	0\\
92.43	0\\
92.44	0\\
92.45	0\\
92.46	0\\
92.47	0\\
92.48	0\\
92.49	0\\
92.5	0\\
92.51	0\\
92.52	0\\
92.53	0\\
92.54	0\\
92.55	0\\
92.56	0\\
92.57	0\\
92.58	0\\
92.59	0\\
92.6	0\\
92.61	0\\
92.62	0\\
92.63	0\\
92.64	0\\
92.65	0\\
92.66	0\\
92.67	0\\
92.68	0\\
92.69	0\\
92.7	0\\
92.71	0\\
92.72	0\\
92.73	0\\
92.74	0\\
92.75	0\\
92.76	0\\
92.77	0\\
92.78	0\\
92.79	0\\
92.8	0\\
92.81	0\\
92.82	0\\
92.83	0\\
92.84	0\\
92.85	0\\
92.86	0\\
92.87	0\\
92.88	0\\
92.89	0\\
92.9	0\\
92.91	0\\
92.92	0\\
92.93	0\\
92.94	0\\
92.95	0\\
92.96	0\\
92.97	0\\
92.98	0\\
92.99	0\\
93	0\\
93.01	0\\
93.02	0\\
93.03	0\\
93.04	0\\
93.05	0\\
93.06	0\\
93.07	0\\
93.08	0\\
93.09	0\\
93.1	0\\
93.11	0\\
93.12	0\\
93.13	0\\
93.14	0\\
93.15	0\\
93.16	0\\
93.17	0\\
93.18	0\\
93.19	0\\
93.2	0\\
93.21	0\\
93.22	0\\
93.23	0\\
93.24	0\\
93.25	0\\
93.26	0\\
93.27	0\\
93.28	0\\
93.29	0\\
93.3	0\\
93.31	0\\
93.32	0\\
93.33	0\\
93.34	0\\
93.35	0\\
93.36	0\\
93.37	0\\
93.38	0\\
93.39	0\\
93.4	0\\
93.41	0\\
93.42	0\\
93.43	0\\
93.44	0\\
93.45	0\\
93.46	0\\
93.47	0\\
93.48	0\\
93.49	0\\
93.5	0\\
93.51	0\\
93.52	0\\
93.53	0\\
93.54	0\\
93.55	0\\
93.56	0\\
93.57	0\\
93.58	0\\
93.59	0\\
93.6	0\\
93.61	0\\
93.62	0\\
93.63	0\\
93.64	0\\
93.65	0\\
93.66	0\\
93.67	0\\
93.68	0\\
93.69	0\\
93.7	0\\
93.71	0\\
93.72	0\\
93.73	0\\
93.74	0\\
93.75	0\\
93.76	0\\
93.77	0\\
93.78	0\\
93.79	0\\
93.8	0\\
93.81	0\\
93.82	0\\
93.83	0\\
93.84	0\\
93.85	0\\
93.86	0\\
93.87	0\\
93.88	0\\
93.89	0\\
93.9	0\\
93.91	0\\
93.92	0\\
93.93	0\\
93.94	0\\
93.95	0\\
93.96	0\\
93.97	0\\
93.98	0\\
93.99	0\\
94	0\\
94.01	0\\
94.02	0\\
94.03	0\\
94.04	0\\
94.05	0\\
94.06	0\\
94.07	0\\
94.08	0\\
94.09	0\\
94.1	0\\
94.11	0\\
94.12	0\\
94.13	0\\
94.14	0\\
94.15	0\\
94.16	0\\
94.17	0\\
94.18	0\\
94.19	0\\
94.2	0\\
94.21	0\\
94.22	0\\
94.23	0\\
94.24	0\\
94.25	0\\
94.26	0\\
94.27	0\\
94.28	0\\
94.29	0\\
94.3	0\\
94.31	0\\
94.32	0\\
94.33	0\\
94.34	0\\
94.35	0\\
94.36	0\\
94.37	0\\
94.38	0\\
94.39	0\\
94.4	0\\
94.41	0\\
94.42	0\\
94.43	0\\
94.44	0\\
94.45	0\\
94.46	0\\
94.47	0\\
94.48	0\\
94.49	0\\
94.5	0\\
94.51	0\\
94.52	0\\
94.53	0\\
94.54	0\\
94.55	0\\
94.56	0\\
94.57	0\\
94.58	0\\
94.59	0\\
94.6	0\\
94.61	0\\
94.62	0\\
94.63	0\\
94.64	0\\
94.65	0\\
94.66	0\\
94.67	0\\
94.68	0\\
94.69	0\\
94.7	0\\
94.71	0\\
94.72	0\\
94.73	0\\
94.74	0\\
94.75	0\\
94.76	0\\
94.77	0\\
94.78	0\\
94.79	0\\
94.8	0\\
94.81	0\\
94.82	0\\
94.83	0\\
94.84	0\\
94.85	0\\
94.86	0\\
94.87	0\\
94.88	0\\
94.89	0\\
94.9	0\\
94.91	0\\
94.92	0\\
94.93	0\\
94.94	0\\
94.95	0\\
94.96	0\\
94.97	0\\
94.98	0\\
94.99	0\\
95	0\\
95.01	0\\
95.02	0\\
95.03	0\\
95.04	0\\
95.05	0\\
95.06	0\\
95.07	0\\
95.08	0\\
95.09	0\\
95.1	0\\
95.11	0\\
95.12	0\\
95.13	0\\
95.14	0\\
95.15	0\\
95.16	0\\
95.17	0\\
95.18	0\\
95.19	0\\
95.2	0\\
95.21	0\\
95.22	0\\
95.23	0\\
95.24	0\\
95.25	0\\
95.26	0\\
95.27	0\\
95.28	0\\
95.29	0\\
95.3	0\\
95.31	0\\
95.32	0\\
95.33	0\\
95.34	0\\
95.35	0\\
95.36	0\\
95.37	0\\
95.38	0\\
95.39	0\\
95.4	0\\
95.41	0\\
95.42	0\\
95.43	0\\
95.44	0\\
95.45	0\\
95.46	0\\
95.47	0\\
95.48	0\\
95.49	0\\
95.5	0\\
95.51	0\\
95.52	0\\
95.53	0\\
95.54	0\\
95.55	0\\
95.56	0\\
95.57	0\\
95.58	0\\
95.59	0\\
95.6	0\\
95.61	0\\
95.62	0\\
95.63	0\\
95.64	0\\
95.65	0\\
95.66	0\\
95.67	0\\
95.68	0\\
95.69	0\\
95.7	0\\
95.71	0\\
95.72	0\\
95.73	0\\
95.74	0\\
95.75	0\\
95.76	0\\
95.77	0\\
95.78	0\\
95.79	0\\
95.8	0\\
95.81	0\\
95.82	0\\
95.83	0\\
95.84	0\\
95.85	0\\
95.86	0\\
95.87	0\\
95.88	0\\
95.89	0\\
95.9	0\\
95.91	0\\
95.92	0\\
95.93	0\\
95.94	0\\
95.95	0\\
95.96	0\\
95.97	0\\
95.98	0\\
95.99	0\\
96	0\\
96.01	0\\
96.02	0\\
96.03	0\\
96.04	0\\
96.05	0\\
96.06	0\\
96.07	0\\
96.08	0\\
96.09	0\\
96.1	0\\
96.11	0\\
96.12	0\\
96.13	0\\
96.14	0\\
96.15	0\\
96.16	0\\
96.17	0\\
96.18	0\\
96.19	0\\
96.2	0\\
96.21	0\\
96.22	0\\
96.23	0\\
96.24	0\\
96.25	0\\
96.26	0\\
96.27	0\\
96.28	0\\
96.29	0\\
96.3	0\\
96.31	0\\
96.32	0\\
96.33	0\\
96.34	0\\
96.35	0\\
96.36	0\\
96.37	0\\
96.38	0\\
96.39	0\\
96.4	0\\
96.41	0\\
96.42	0\\
96.43	0\\
96.44	0\\
96.45	0\\
96.46	0\\
96.47	0\\
96.48	0\\
96.49	0\\
96.5	0\\
96.51	0\\
96.52	0\\
96.53	0\\
96.54	0\\
96.55	0\\
96.56	0\\
96.57	0\\
96.58	0\\
96.59	0\\
96.6	0\\
96.61	0\\
96.62	0\\
96.63	0\\
96.64	0\\
96.65	0\\
96.66	0\\
96.67	0\\
96.68	0\\
96.69	0\\
96.7	0\\
96.71	0\\
96.72	0\\
96.73	0\\
96.74	0\\
96.75	0\\
96.76	0\\
96.77	0\\
96.78	0\\
96.79	0\\
96.8	0\\
96.81	0\\
96.82	0\\
96.83	0\\
96.84	0\\
96.85	0\\
96.86	0\\
96.87	0\\
96.88	0\\
96.89	0\\
96.9	0\\
96.91	0\\
96.92	0\\
96.93	0\\
96.94	0\\
96.95	0\\
96.96	0\\
96.97	0\\
96.98	0\\
96.99	0\\
97	0\\
97.01	0\\
97.02	0\\
97.03	0\\
97.04	0\\
97.05	0\\
97.06	0\\
97.07	0\\
97.08	0\\
97.09	0\\
97.1	0\\
97.11	0\\
97.12	0\\
97.13	0\\
97.14	0\\
97.15	0\\
97.16	0\\
97.17	0\\
97.18	0\\
97.19	0\\
97.2	0\\
97.21	0\\
97.22	0\\
97.23	0\\
97.24	0\\
97.25	0\\
97.26	0\\
97.27	0\\
97.28	0\\
97.29	0\\
97.3	0\\
97.31	0\\
97.32	0\\
97.33	0\\
97.34	0\\
97.35	0\\
97.36	0\\
97.37	0\\
97.38	0\\
97.39	0\\
97.4	0\\
97.41	0\\
97.42	0\\
97.43	0\\
97.44	0\\
97.45	0\\
97.46	0\\
97.47	0\\
97.48	0\\
97.49	0\\
97.5	0\\
97.51	0\\
97.52	0\\
97.53	0\\
97.54	0\\
97.55	0\\
97.56	0\\
97.57	0\\
97.58	0\\
97.59	0\\
97.6	0\\
97.61	0\\
97.62	0\\
97.63	0\\
97.64	0\\
97.65	0\\
97.66	0\\
97.67	0\\
97.68	0\\
97.69	0\\
97.7	0\\
97.71	0\\
97.72	0\\
97.73	0\\
97.74	0\\
97.75	0\\
97.76	0\\
97.77	0\\
97.78	0\\
97.79	0\\
97.8	0\\
97.81	0\\
97.82	0\\
97.83	0\\
97.84	0\\
97.85	0\\
97.86	0\\
97.87	0\\
97.88	0\\
97.89	0\\
97.9	0\\
97.91	0\\
97.92	0\\
97.93	0\\
97.94	0\\
97.95	0\\
97.96	0\\
97.97	0\\
97.98	0\\
97.99	0\\
98	0\\
98.01	0\\
98.02	0\\
98.03	0\\
98.04	0\\
98.05	0\\
98.06	0\\
98.07	0\\
98.08	0\\
98.09	0\\
98.1	0\\
98.11	0\\
98.12	0\\
98.13	0\\
98.14	0\\
98.15	0\\
98.16	0\\
98.17	0\\
98.18	0\\
98.19	0\\
98.2	0\\
98.21	0\\
98.22	0\\
98.23	0\\
98.24	0\\
98.25	0\\
98.26	0\\
98.27	0\\
98.28	0\\
98.29	0\\
98.3	0\\
98.31	0\\
98.32	0\\
98.33	0\\
98.34	0\\
98.35	0\\
98.36	0\\
98.37	0\\
98.38	0\\
98.39	0\\
98.4	0\\
98.41	0\\
98.42	0\\
98.43	0\\
98.44	0\\
98.45	0\\
98.46	0\\
98.47	0\\
98.48	0\\
98.49	0\\
98.5	0\\
98.51	0\\
98.52	0\\
98.53	0\\
98.54	0\\
98.55	0\\
98.56	0\\
98.57	0\\
98.58	0\\
98.59	0\\
98.6	0\\
98.61	0\\
98.62	0\\
98.63	0\\
98.64	0\\
98.65	0\\
98.66	0\\
98.67	0\\
98.68	0\\
98.69	0\\
98.7	0\\
98.71	0\\
98.72	0\\
98.73	0\\
98.74	0\\
98.75	0\\
98.76	0\\
98.77	0\\
98.78	0\\
98.79	0\\
98.8	0\\
98.81	0\\
98.82	0\\
98.83	0\\
98.84	0\\
98.85	0\\
98.86	0\\
98.87	0\\
98.88	0\\
98.89	0\\
98.9	0\\
98.91	0\\
98.92	0\\
98.93	0\\
98.94	0\\
98.95	0\\
98.96	0\\
98.97	0\\
98.98	0\\
98.99	0\\
99	0\\
99.01	0\\
99.02	0\\
99.03	0\\
99.04	0\\
99.05	0\\
99.06	0\\
99.07	0\\
99.08	0\\
99.09	0\\
99.1	0\\
99.11	0\\
99.12	0\\
99.13	0\\
99.14	0\\
99.15	0\\
99.16	0\\
99.17	0\\
99.18	0\\
99.19	0\\
99.2	0\\
99.21	0\\
99.22	0\\
99.23	0\\
99.24	0\\
99.25	0\\
99.26	0\\
99.27	0\\
99.28	0\\
99.29	0\\
99.3	0\\
99.31	0\\
99.32	0\\
99.33	0\\
99.34	0\\
99.35	0\\
99.36	0\\
99.37	0\\
99.38	0\\
99.39	0\\
99.4	0\\
99.41	0\\
99.42	0\\
99.43	0\\
99.44	0\\
99.45	0\\
99.46	0\\
99.47	0\\
99.48	0\\
99.49	0\\
99.5	0\\
99.51	0\\
99.52	0\\
99.53	0\\
99.54	0\\
99.55	0\\
99.56	0\\
99.57	0\\
99.58	0\\
99.59	0\\
99.6	0\\
99.61	0\\
99.62	0\\
99.63	0\\
99.64	0\\
99.65	0\\
99.66	0\\
99.67	0\\
99.68	0\\
99.69	0\\
99.7	0\\
99.71	0\\
99.72	0\\
99.73	0\\
99.74	0\\
99.75	0\\
99.76	0\\
99.77	0\\
99.78	0\\
99.79	0\\
99.8	0\\
99.81	0\\
99.82	0\\
99.83	0\\
99.84	0\\
99.85	0\\
99.86	0\\
99.87	0\\
99.88	0\\
99.89	0\\
99.9	0\\
99.91	0\\
99.92	0\\
99.93	0\\
99.94	0\\
99.95	0\\
99.96	0\\
99.97	0\\
99.98	0\\
99.99	0\\
100	0\\
};
\addlegendentry{$q=2$};

\addplot [color=mycolor1,solid,forget plot]
  table[row sep=crcr]{%
0.01	0\\
0.02	0\\
0.03	0\\
0.04	0\\
0.05	0\\
0.06	0\\
0.07	0\\
0.08	0\\
0.09	0\\
0.1	0\\
0.11	0\\
0.12	0\\
0.13	0\\
0.14	0\\
0.15	0\\
0.16	0\\
0.17	0\\
0.18	0\\
0.19	0\\
0.2	0\\
0.21	0\\
0.22	0\\
0.23	0\\
0.24	0\\
0.25	0\\
0.26	0\\
0.27	0\\
0.28	0\\
0.29	0\\
0.3	0\\
0.31	0\\
0.32	0\\
0.33	0\\
0.34	0\\
0.35	0\\
0.36	0\\
0.37	0\\
0.38	0\\
0.39	0\\
0.4	0\\
0.41	0\\
0.42	0\\
0.43	0\\
0.44	0\\
0.45	0\\
0.46	0\\
0.47	0\\
0.48	0\\
0.49	0\\
0.5	0\\
0.51	0\\
0.52	0\\
0.53	0\\
0.54	0\\
0.55	0\\
0.56	0\\
0.57	0\\
0.58	0\\
0.59	0\\
0.6	0\\
0.61	0\\
0.62	0\\
0.63	0\\
0.64	0\\
0.65	0\\
0.66	0\\
0.67	0\\
0.68	0\\
0.69	0\\
0.7	0\\
0.71	0\\
0.72	0\\
0.73	0\\
0.74	0\\
0.75	0\\
0.76	0\\
0.77	0\\
0.78	0\\
0.79	0\\
0.8	0\\
0.81	0\\
0.82	0\\
0.83	0\\
0.84	0\\
0.85	0\\
0.86	0\\
0.87	0\\
0.88	0\\
0.89	0\\
0.9	0\\
0.91	0\\
0.92	0\\
0.93	0\\
0.94	0\\
0.95	0\\
0.96	0\\
0.97	0\\
0.98	0\\
0.99	0\\
1	0\\
1.01	0\\
1.02	0\\
1.03	0\\
1.04	0\\
1.05	0\\
1.06	0\\
1.07	0\\
1.08	0\\
1.09	0\\
1.1	0\\
1.11	0\\
1.12	0\\
1.13	0\\
1.14	0\\
1.15	0\\
1.16	0\\
1.17	0\\
1.18	0\\
1.19	0\\
1.2	0\\
1.21	0\\
1.22	0\\
1.23	0\\
1.24	0\\
1.25	0\\
1.26	0\\
1.27	0\\
1.28	0\\
1.29	0\\
1.3	0\\
1.31	0\\
1.32	0\\
1.33	0\\
1.34	0\\
1.35	0\\
1.36	0\\
1.37	0\\
1.38	0\\
1.39	0\\
1.4	0\\
1.41	0\\
1.42	0\\
1.43	0\\
1.44	0\\
1.45	0\\
1.46	0\\
1.47	0\\
1.48	0\\
1.49	0\\
1.5	0\\
1.51	0\\
1.52	0\\
1.53	0\\
1.54	0\\
1.55	0\\
1.56	0\\
1.57	0\\
1.58	0\\
1.59	0\\
1.6	0\\
1.61	0\\
1.62	0\\
1.63	0\\
1.64	0\\
1.65	0\\
1.66	0\\
1.67	0\\
1.68	0\\
1.69	0\\
1.7	0\\
1.71	0\\
1.72	0\\
1.73	0\\
1.74	0\\
1.75	0\\
1.76	0\\
1.77	0\\
1.78	0\\
1.79	0\\
1.8	0\\
1.81	0\\
1.82	0\\
1.83	0\\
1.84	0\\
1.85	0\\
1.86	0\\
1.87	0\\
1.88	0\\
1.89	0\\
1.9	0\\
1.91	0\\
1.92	0\\
1.93	0\\
1.94	0\\
1.95	0\\
1.96	0\\
1.97	0\\
1.98	0\\
1.99	0\\
2	0\\
2.01	0\\
2.02	0\\
2.03	0\\
2.04	0\\
2.05	0\\
2.06	0\\
2.07	0\\
2.08	0\\
2.09	0\\
2.1	0\\
2.11	0\\
2.12	0\\
2.13	0\\
2.14	0\\
2.15	0\\
2.16	0\\
2.17	0\\
2.18	0\\
2.19	0\\
2.2	0\\
2.21	0\\
2.22	0\\
2.23	0\\
2.24	0\\
2.25	0\\
2.26	0\\
2.27	0\\
2.28	0\\
2.29	0\\
2.3	0\\
2.31	0\\
2.32	0\\
2.33	0\\
2.34	0\\
2.35	0\\
2.36	0\\
2.37	0\\
2.38	0\\
2.39	0\\
2.4	0\\
2.41	0\\
2.42	0\\
2.43	0\\
2.44	0\\
2.45	0\\
2.46	0\\
2.47	0\\
2.48	0\\
2.49	0\\
2.5	0\\
2.51	0\\
2.52	0\\
2.53	0\\
2.54	0\\
2.55	0\\
2.56	0\\
2.57	0\\
2.58	0\\
2.59	0\\
2.6	0\\
2.61	0\\
2.62	0\\
2.63	0\\
2.64	0\\
2.65	0\\
2.66	0\\
2.67	0\\
2.68	0\\
2.69	0\\
2.7	0\\
2.71	0\\
2.72	0\\
2.73	0\\
2.74	0\\
2.75	0\\
2.76	0\\
2.77	0\\
2.78	0\\
2.79	0\\
2.8	0\\
2.81	0\\
2.82	0\\
2.83	0\\
2.84	0\\
2.85	0\\
2.86	0\\
2.87	0\\
2.88	0\\
2.89	0\\
2.9	0\\
2.91	0\\
2.92	0\\
2.93	0\\
2.94	0\\
2.95	0\\
2.96	0\\
2.97	0\\
2.98	0\\
2.99	0\\
3	0\\
3.01	0\\
3.02	0\\
3.03	0\\
3.04	0\\
3.05	0\\
3.06	0\\
3.07	0\\
3.08	0\\
3.09	0\\
3.1	0\\
3.11	0\\
3.12	0\\
3.13	0\\
3.14	0\\
3.15	0\\
3.16	0\\
3.17	0\\
3.18	0\\
3.19	0\\
3.2	0\\
3.21	0\\
3.22	0\\
3.23	0\\
3.24	0\\
3.25	0\\
3.26	0\\
3.27	0\\
3.28	0\\
3.29	0\\
3.3	0\\
3.31	0\\
3.32	0\\
3.33	0\\
3.34	0\\
3.35	0\\
3.36	0\\
3.37	0\\
3.38	0\\
3.39	0\\
3.4	0\\
3.41	0\\
3.42	0\\
3.43	0\\
3.44	0\\
3.45	0\\
3.46	0\\
3.47	0\\
3.48	0\\
3.49	0\\
3.5	0\\
3.51	0\\
3.52	0\\
3.53	0\\
3.54	0\\
3.55	0\\
3.56	0\\
3.57	0\\
3.58	0\\
3.59	0\\
3.6	0\\
3.61	0\\
3.62	0\\
3.63	0\\
3.64	0\\
3.65	0\\
3.66	0\\
3.67	0\\
3.68	0\\
3.69	0\\
3.7	0\\
3.71	0\\
3.72	0\\
3.73	0\\
3.74	0\\
3.75	0\\
3.76	0\\
3.77	0\\
3.78	0\\
3.79	0\\
3.8	0\\
3.81	0\\
3.82	0\\
3.83	0\\
3.84	0\\
3.85	0\\
3.86	0\\
3.87	0\\
3.88	0\\
3.89	0\\
3.9	0\\
3.91	0\\
3.92	0\\
3.93	0\\
3.94	0\\
3.95	0\\
3.96	0\\
3.97	0\\
3.98	0\\
3.99	0\\
4	0\\
4.01	0\\
4.02	0\\
4.03	0\\
4.04	0\\
4.05	0\\
4.06	0\\
4.07	0\\
4.08	0\\
4.09	0\\
4.1	0\\
4.11	0\\
4.12	0\\
4.13	0\\
4.14	0\\
4.15	0\\
4.16	0\\
4.17	0\\
4.18	0\\
4.19	0\\
4.2	0\\
4.21	0\\
4.22	0\\
4.23	0\\
4.24	0\\
4.25	0\\
4.26	0\\
4.27	0\\
4.28	0\\
4.29	0\\
4.3	0\\
4.31	0\\
4.32	0\\
4.33	0\\
4.34	0\\
4.35	0\\
4.36	0\\
4.37	0\\
4.38	0\\
4.39	0\\
4.4	0\\
4.41	0\\
4.42	0\\
4.43	0\\
4.44	0\\
4.45	0\\
4.46	0\\
4.47	0\\
4.48	0\\
4.49	0\\
4.5	0\\
4.51	0\\
4.52	0\\
4.53	0\\
4.54	0\\
4.55	0\\
4.56	0\\
4.57	0\\
4.58	0\\
4.59	0\\
4.6	0\\
4.61	0\\
4.62	0\\
4.63	0\\
4.64	0\\
4.65	0\\
4.66	0\\
4.67	0\\
4.68	0\\
4.69	0\\
4.7	0\\
4.71	0\\
4.72	0\\
4.73	0\\
4.74	0\\
4.75	0\\
4.76	0\\
4.77	0\\
4.78	0\\
4.79	0\\
4.8	0\\
4.81	0\\
4.82	0\\
4.83	0\\
4.84	0\\
4.85	0\\
4.86	0\\
4.87	0\\
4.88	0\\
4.89	0\\
4.9	0\\
4.91	0\\
4.92	0\\
4.93	0\\
4.94	0\\
4.95	0\\
4.96	0\\
4.97	0\\
4.98	0\\
4.99	0\\
5	0\\
5.01	0\\
5.02	0\\
5.03	0\\
5.04	0\\
5.05	0\\
5.06	0\\
5.07	0\\
5.08	0\\
5.09	0\\
5.1	0\\
5.11	0\\
5.12	0\\
5.13	0\\
5.14	0\\
5.15	0\\
5.16	0\\
5.17	0\\
5.18	0\\
5.19	0\\
5.2	0\\
5.21	0\\
5.22	0\\
5.23	0\\
5.24	0\\
5.25	0\\
5.26	0\\
5.27	0\\
5.28	0\\
5.29	0\\
5.3	0\\
5.31	0\\
5.32	0\\
5.33	0\\
5.34	0\\
5.35	0\\
5.36	0\\
5.37	0\\
5.38	0\\
5.39	0\\
5.4	0\\
5.41	0\\
5.42	0\\
5.43	0\\
5.44	0\\
5.45	0\\
5.46	0\\
5.47	0\\
5.48	0\\
5.49	0\\
5.5	0\\
5.51	0\\
5.52	0\\
5.53	0\\
5.54	0\\
5.55	0\\
5.56	0\\
5.57	0\\
5.58	0\\
5.59	0\\
5.6	0\\
5.61	0\\
5.62	0\\
5.63	0\\
5.64	0\\
5.65	0\\
5.66	0\\
5.67	0\\
5.68	0\\
5.69	0\\
5.7	0\\
5.71	0\\
5.72	0\\
5.73	0\\
5.74	0\\
5.75	0\\
5.76	0\\
5.77	0\\
5.78	0\\
5.79	0\\
5.8	0\\
5.81	0\\
5.82	0\\
5.83	0\\
5.84	0\\
5.85	0\\
5.86	0\\
5.87	0\\
5.88	0\\
5.89	0\\
5.9	0\\
5.91	0\\
5.92	0\\
5.93	0\\
5.94	0\\
5.95	0\\
5.96	0\\
5.97	0\\
5.98	0\\
5.99	0\\
6	0\\
6.01	0\\
6.02	0\\
6.03	0\\
6.04	0\\
6.05	0\\
6.06	0\\
6.07	0\\
6.08	0\\
6.09	0\\
6.1	0\\
6.11	0\\
6.12	0\\
6.13	0\\
6.14	0\\
6.15	0\\
6.16	0\\
6.17	0\\
6.18	0\\
6.19	0\\
6.2	0\\
6.21	0\\
6.22	0\\
6.23	0\\
6.24	0\\
6.25	0\\
6.26	0\\
6.27	0\\
6.28	0\\
6.29	0\\
6.3	0\\
6.31	0\\
6.32	0\\
6.33	0\\
6.34	0\\
6.35	0\\
6.36	0\\
6.37	0\\
6.38	0\\
6.39	0\\
6.4	0\\
6.41	0\\
6.42	0\\
6.43	0\\
6.44	0\\
6.45	0\\
6.46	0\\
6.47	0\\
6.48	0\\
6.49	0\\
6.5	0\\
6.51	0\\
6.52	0\\
6.53	0\\
6.54	0\\
6.55	0\\
6.56	0\\
6.57	0\\
6.58	0\\
6.59	0\\
6.6	0\\
6.61	0\\
6.62	0\\
6.63	0\\
6.64	0\\
6.65	0\\
6.66	0\\
6.67	0\\
6.68	0\\
6.69	0\\
6.7	0\\
6.71	0\\
6.72	0\\
6.73	0\\
6.74	0\\
6.75	0\\
6.76	0\\
6.77	0\\
6.78	0\\
6.79	0\\
6.8	0\\
6.81	0\\
6.82	0\\
6.83	0\\
6.84	0\\
6.85	0\\
6.86	0\\
6.87	0\\
6.88	0\\
6.89	0\\
6.9	0\\
6.91	0\\
6.92	0\\
6.93	0\\
6.94	0\\
6.95	0\\
6.96	0\\
6.97	0\\
6.98	0\\
6.99	0\\
7	0\\
7.01	0\\
7.02	0\\
7.03	0\\
7.04	0\\
7.05	0\\
7.06	0\\
7.07	0\\
7.08	0\\
7.09	0\\
7.1	0\\
7.11	0\\
7.12	0\\
7.13	0\\
7.14	0\\
7.15	0\\
7.16	0\\
7.17	0\\
7.18	0\\
7.19	0\\
7.2	0\\
7.21	0\\
7.22	0\\
7.23	0\\
7.24	0\\
7.25	0\\
7.26	0\\
7.27	0\\
7.28	0\\
7.29	0\\
7.3	0\\
7.31	0\\
7.32	0\\
7.33	0\\
7.34	0\\
7.35	0\\
7.36	0\\
7.37	0\\
7.38	0\\
7.39	0\\
7.4	0\\
7.41	0\\
7.42	0\\
7.43	0\\
7.44	0\\
7.45	0\\
7.46	0\\
7.47	0\\
7.48	0\\
7.49	0\\
7.5	0\\
7.51	0\\
7.52	0\\
7.53	0\\
7.54	0\\
7.55	0\\
7.56	0\\
7.57	0\\
7.58	0\\
7.59	0\\
7.6	0\\
7.61	0\\
7.62	0\\
7.63	0\\
7.64	0\\
7.65	0\\
7.66	0\\
7.67	0\\
7.68	0\\
7.69	0\\
7.7	0\\
7.71	0\\
7.72	0\\
7.73	0\\
7.74	0\\
7.75	0\\
7.76	0\\
7.77	0\\
7.78	0\\
7.79	0\\
7.8	0\\
7.81	0\\
7.82	0\\
7.83	0\\
7.84	0\\
7.85	0\\
7.86	0\\
7.87	0\\
7.88	0\\
7.89	0\\
7.9	0\\
7.91	0\\
7.92	0\\
7.93	0\\
7.94	0\\
7.95	0\\
7.96	0\\
7.97	0\\
7.98	0\\
7.99	0\\
8	0\\
8.01	0\\
8.02	0\\
8.03	0\\
8.04	0\\
8.05	0\\
8.06	0\\
8.07	0\\
8.08	0\\
8.09	0\\
8.1	0\\
8.11	0\\
8.12	0\\
8.13	0\\
8.14	0\\
8.15	0\\
8.16	0\\
8.17	0\\
8.18	0\\
8.19	0\\
8.2	0\\
8.21	0\\
8.22	0\\
8.23	0\\
8.24	0\\
8.25	0\\
8.26	0\\
8.27	0\\
8.28	0\\
8.29	0\\
8.3	0\\
8.31	0\\
8.32	0\\
8.33	0\\
8.34	0\\
8.35	0\\
8.36	0\\
8.37	0\\
8.38	0\\
8.39	0\\
8.4	0\\
8.41	0\\
8.42	0\\
8.43	0\\
8.44	0\\
8.45	0\\
8.46	0\\
8.47	0\\
8.48	0\\
8.49	0\\
8.5	0\\
8.51	0\\
8.52	0\\
8.53	0\\
8.54	0\\
8.55	0\\
8.56	0\\
8.57	0\\
8.58	0\\
8.59	0\\
8.6	0\\
8.61	0\\
8.62	0\\
8.63	0\\
8.64	0\\
8.65	0\\
8.66	0\\
8.67	0\\
8.68	0\\
8.69	0\\
8.7	0\\
8.71	0\\
8.72	0\\
8.73	0\\
8.74	0\\
8.75	0\\
8.76	0\\
8.77	0\\
8.78	0\\
8.79	0\\
8.8	0\\
8.81	0\\
8.82	0\\
8.83	0\\
8.84	0\\
8.85	0\\
8.86	0\\
8.87	0\\
8.88	0\\
8.89	0\\
8.9	0\\
8.91	0\\
8.92	0\\
8.93	0\\
8.94	0\\
8.95	0\\
8.96	0\\
8.97	0\\
8.98	0\\
8.99	0\\
9	0\\
9.01	0\\
9.02	0\\
9.03	0\\
9.04	0\\
9.05	0\\
9.06	0\\
9.07	0\\
9.08	0\\
9.09	0\\
9.1	0\\
9.11	0\\
9.12	0\\
9.13	0\\
9.14	0\\
9.15	0\\
9.16	0\\
9.17	0\\
9.18	0\\
9.19	0\\
9.2	0\\
9.21	0\\
9.22	0\\
9.23	0\\
9.24	0\\
9.25	0\\
9.26	0\\
9.27	0\\
9.28	0\\
9.29	0\\
9.3	0\\
9.31	0\\
9.32	0\\
9.33	0\\
9.34	0\\
9.35	0\\
9.36	0\\
9.37	0\\
9.38	0\\
9.39	0\\
9.4	0\\
9.41	0\\
9.42	0\\
9.43	0\\
9.44	0\\
9.45	0\\
9.46	0\\
9.47	0\\
9.48	0\\
9.49	0\\
9.5	0\\
9.51	0\\
9.52	0\\
9.53	0\\
9.54	0\\
9.55	0\\
9.56	0\\
9.57	0\\
9.58	0\\
9.59	0\\
9.6	0\\
9.61	0\\
9.62	0\\
9.63	0\\
9.64	0\\
9.65	0\\
9.66	0\\
9.67	0\\
9.68	0\\
9.69	0\\
9.7	0\\
9.71	0\\
9.72	0\\
9.73	0\\
9.74	0\\
9.75	0\\
9.76	0\\
9.77	0\\
9.78	0\\
9.79	0\\
9.8	0\\
9.81	0\\
9.82	0\\
9.83	0\\
9.84	0\\
9.85	0\\
9.86	0\\
9.87	0\\
9.88	0\\
9.89	0\\
9.9	0\\
9.91	0\\
9.92	0\\
9.93	0\\
9.94	0\\
9.95	0\\
9.96	0\\
9.97	0\\
9.98	0\\
9.99	0\\
10	0\\
10.01	0\\
10.02	0\\
10.03	0\\
10.04	0\\
10.05	0\\
10.06	0\\
10.07	0\\
10.08	0\\
10.09	0\\
10.1	0\\
10.11	0\\
10.12	0\\
10.13	0\\
10.14	0\\
10.15	0\\
10.16	0\\
10.17	0\\
10.18	0\\
10.19	0\\
10.2	0\\
10.21	0\\
10.22	0\\
10.23	0\\
10.24	0\\
10.25	0\\
10.26	0\\
10.27	0\\
10.28	0\\
10.29	0\\
10.3	0\\
10.31	0\\
10.32	0\\
10.33	0\\
10.34	0\\
10.35	0\\
10.36	0\\
10.37	0\\
10.38	0\\
10.39	0\\
10.4	0\\
10.41	0\\
10.42	0\\
10.43	0\\
10.44	0\\
10.45	0\\
10.46	0\\
10.47	0\\
10.48	0\\
10.49	0\\
10.5	0\\
10.51	0\\
10.52	0\\
10.53	0\\
10.54	0\\
10.55	0\\
10.56	0\\
10.57	0\\
10.58	0\\
10.59	0\\
10.6	0\\
10.61	0\\
10.62	0\\
10.63	0\\
10.64	0\\
10.65	0\\
10.66	0\\
10.67	0\\
10.68	0\\
10.69	0\\
10.7	0\\
10.71	0\\
10.72	0\\
10.73	0\\
10.74	0\\
10.75	0\\
10.76	0\\
10.77	0\\
10.78	0\\
10.79	0\\
10.8	0\\
10.81	0\\
10.82	0\\
10.83	0\\
10.84	0\\
10.85	0\\
10.86	0\\
10.87	0\\
10.88	0\\
10.89	0\\
10.9	0\\
10.91	0\\
10.92	0\\
10.93	0\\
10.94	0\\
10.95	0\\
10.96	0\\
10.97	0\\
10.98	0\\
10.99	0\\
11	0\\
11.01	0\\
11.02	0\\
11.03	0\\
11.04	0\\
11.05	0\\
11.06	0\\
11.07	0\\
11.08	0\\
11.09	0\\
11.1	0\\
11.11	0\\
11.12	0\\
11.13	0\\
11.14	0\\
11.15	0\\
11.16	0\\
11.17	0\\
11.18	0\\
11.19	0\\
11.2	0\\
11.21	0\\
11.22	0\\
11.23	0\\
11.24	0\\
11.25	0\\
11.26	0\\
11.27	0\\
11.28	0\\
11.29	0\\
11.3	0\\
11.31	0\\
11.32	0\\
11.33	0\\
11.34	0\\
11.35	0\\
11.36	0\\
11.37	0\\
11.38	0\\
11.39	0\\
11.4	0\\
11.41	0\\
11.42	0\\
11.43	0\\
11.44	0\\
11.45	0\\
11.46	0\\
11.47	0\\
11.48	0\\
11.49	0\\
11.5	0\\
11.51	0\\
11.52	0\\
11.53	0\\
11.54	0\\
11.55	0\\
11.56	0\\
11.57	0\\
11.58	0\\
11.59	0\\
11.6	0\\
11.61	0\\
11.62	0\\
11.63	0\\
11.64	0\\
11.65	0\\
11.66	0\\
11.67	0\\
11.68	0\\
11.69	0\\
11.7	0\\
11.71	0\\
11.72	0\\
11.73	0\\
11.74	0\\
11.75	0\\
11.76	0\\
11.77	0\\
11.78	0\\
11.79	0\\
11.8	0\\
11.81	0\\
11.82	0\\
11.83	0\\
11.84	0\\
11.85	0\\
11.86	0\\
11.87	0\\
11.88	0\\
11.89	0\\
11.9	0\\
11.91	0\\
11.92	0\\
11.93	0\\
11.94	0\\
11.95	0\\
11.96	0\\
11.97	0\\
11.98	0\\
11.99	0\\
12	0\\
12.01	0\\
12.02	0\\
12.03	0\\
12.04	0\\
12.05	0\\
12.06	0\\
12.07	0\\
12.08	0\\
12.09	0\\
12.1	0\\
12.11	0\\
12.12	0\\
12.13	0\\
12.14	0\\
12.15	0\\
12.16	0\\
12.17	0\\
12.18	0\\
12.19	0\\
12.2	0\\
12.21	0\\
12.22	0\\
12.23	0\\
12.24	0\\
12.25	0\\
12.26	0\\
12.27	0\\
12.28	0\\
12.29	0\\
12.3	0\\
12.31	0\\
12.32	0\\
12.33	0\\
12.34	0\\
12.35	0\\
12.36	0\\
12.37	0\\
12.38	0\\
12.39	0\\
12.4	0\\
12.41	0\\
12.42	0\\
12.43	0\\
12.44	0\\
12.45	0\\
12.46	0\\
12.47	0\\
12.48	0\\
12.49	0\\
12.5	0\\
12.51	0\\
12.52	0\\
12.53	0\\
12.54	0\\
12.55	0\\
12.56	0\\
12.57	0\\
12.58	0\\
12.59	0\\
12.6	0\\
12.61	0\\
12.62	0\\
12.63	0\\
12.64	0\\
12.65	0\\
12.66	0\\
12.67	0\\
12.68	0\\
12.69	0\\
12.7	0\\
12.71	0\\
12.72	0\\
12.73	0\\
12.74	0\\
12.75	0\\
12.76	0\\
12.77	0\\
12.78	0\\
12.79	0\\
12.8	0\\
12.81	0\\
12.82	0\\
12.83	0\\
12.84	0\\
12.85	0\\
12.86	0\\
12.87	0\\
12.88	0\\
12.89	0\\
12.9	0\\
12.91	0\\
12.92	0\\
12.93	0\\
12.94	0\\
12.95	0\\
12.96	0\\
12.97	0\\
12.98	0\\
12.99	0\\
13	0\\
13.01	0\\
13.02	0\\
13.03	0\\
13.04	0\\
13.05	0\\
13.06	0\\
13.07	0\\
13.08	0\\
13.09	0\\
13.1	0\\
13.11	0\\
13.12	0\\
13.13	0\\
13.14	0\\
13.15	0\\
13.16	0\\
13.17	0\\
13.18	0\\
13.19	0\\
13.2	0\\
13.21	0\\
13.22	0\\
13.23	0\\
13.24	0\\
13.25	0\\
13.26	0\\
13.27	0\\
13.28	0\\
13.29	0\\
13.3	0\\
13.31	0\\
13.32	0\\
13.33	0\\
13.34	0\\
13.35	0\\
13.36	0\\
13.37	0\\
13.38	0\\
13.39	0\\
13.4	0\\
13.41	0\\
13.42	0\\
13.43	0\\
13.44	0\\
13.45	0\\
13.46	0\\
13.47	0\\
13.48	0\\
13.49	0\\
13.5	0\\
13.51	0\\
13.52	0\\
13.53	0\\
13.54	0\\
13.55	0\\
13.56	0\\
13.57	0\\
13.58	0\\
13.59	0\\
13.6	0\\
13.61	0\\
13.62	0\\
13.63	0\\
13.64	0\\
13.65	0\\
13.66	0\\
13.67	0\\
13.68	0\\
13.69	0\\
13.7	0\\
13.71	0\\
13.72	0\\
13.73	0\\
13.74	0\\
13.75	0\\
13.76	0\\
13.77	0\\
13.78	0\\
13.79	0\\
13.8	0\\
13.81	0\\
13.82	0\\
13.83	0\\
13.84	0\\
13.85	0\\
13.86	0\\
13.87	0\\
13.88	0\\
13.89	0\\
13.9	0\\
13.91	0\\
13.92	0\\
13.93	0\\
13.94	0\\
13.95	0\\
13.96	0\\
13.97	0\\
13.98	0\\
13.99	0\\
14	0\\
14.01	0\\
14.02	0\\
14.03	0\\
14.04	0\\
14.05	0\\
14.06	0\\
14.07	0\\
14.08	0\\
14.09	0\\
14.1	0\\
14.11	0\\
14.12	0\\
14.13	0\\
14.14	0\\
14.15	0\\
14.16	0\\
14.17	0\\
14.18	0\\
14.19	0\\
14.2	0\\
14.21	0\\
14.22	0\\
14.23	0\\
14.24	0\\
14.25	0\\
14.26	0\\
14.27	0\\
14.28	0\\
14.29	0\\
14.3	0\\
14.31	0\\
14.32	0\\
14.33	0\\
14.34	0\\
14.35	0\\
14.36	0\\
14.37	0\\
14.38	0\\
14.39	0\\
14.4	0\\
14.41	0\\
14.42	0\\
14.43	0\\
14.44	0\\
14.45	0\\
14.46	0\\
14.47	0\\
14.48	0\\
14.49	0\\
14.5	0\\
14.51	0\\
14.52	0\\
14.53	0\\
14.54	0\\
14.55	0\\
14.56	0\\
14.57	0\\
14.58	0\\
14.59	0\\
14.6	0\\
14.61	0\\
14.62	0\\
14.63	0\\
14.64	0\\
14.65	0\\
14.66	0\\
14.67	0\\
14.68	0\\
14.69	0\\
14.7	0\\
14.71	0\\
14.72	0\\
14.73	0\\
14.74	0\\
14.75	0\\
14.76	0\\
14.77	0\\
14.78	0\\
14.79	0\\
14.8	0\\
14.81	0\\
14.82	0\\
14.83	0\\
14.84	0\\
14.85	0\\
14.86	0\\
14.87	0\\
14.88	0\\
14.89	0\\
14.9	0\\
14.91	0\\
14.92	0\\
14.93	0\\
14.94	0\\
14.95	0\\
14.96	0\\
14.97	0\\
14.98	0\\
14.99	0\\
15	0\\
15.01	0\\
15.02	0\\
15.03	0\\
15.04	0\\
15.05	0\\
15.06	0\\
15.07	0\\
15.08	0\\
15.09	0\\
15.1	0\\
15.11	0\\
15.12	0\\
15.13	0\\
15.14	0\\
15.15	0\\
15.16	0\\
15.17	0\\
15.18	0\\
15.19	0\\
15.2	0\\
15.21	0\\
15.22	0\\
15.23	0\\
15.24	0\\
15.25	0\\
15.26	0\\
15.27	0\\
15.28	0\\
15.29	0\\
15.3	0\\
15.31	0\\
15.32	0\\
15.33	0\\
15.34	0\\
15.35	0\\
15.36	0\\
15.37	0\\
15.38	0\\
15.39	0\\
15.4	0\\
15.41	0\\
15.42	0\\
15.43	0\\
15.44	0\\
15.45	0\\
15.46	0\\
15.47	0\\
15.48	0\\
15.49	0\\
15.5	0\\
15.51	0\\
15.52	0\\
15.53	0\\
15.54	0\\
15.55	0\\
15.56	0\\
15.57	0\\
15.58	0\\
15.59	0\\
15.6	0\\
15.61	0\\
15.62	0\\
15.63	0\\
15.64	0\\
15.65	0\\
15.66	0\\
15.67	0\\
15.68	0\\
15.69	0\\
15.7	0\\
15.71	0\\
15.72	0\\
15.73	0\\
15.74	0\\
15.75	0\\
15.76	0\\
15.77	0\\
15.78	0\\
15.79	0\\
15.8	0\\
15.81	0\\
15.82	0\\
15.83	0\\
15.84	0\\
15.85	0\\
15.86	0\\
15.87	0\\
15.88	0\\
15.89	0\\
15.9	0\\
15.91	0\\
15.92	0\\
15.93	0\\
15.94	0\\
15.95	0\\
15.96	0\\
15.97	0\\
15.98	0\\
15.99	0\\
16	0\\
16.01	0\\
16.02	0\\
16.03	0\\
16.04	0\\
16.05	0\\
16.06	0\\
16.07	0\\
16.08	0\\
16.09	0\\
16.1	0\\
16.11	0\\
16.12	0\\
16.13	0\\
16.14	0\\
16.15	0\\
16.16	0\\
16.17	0\\
16.18	0\\
16.19	0\\
16.2	0\\
16.21	0\\
16.22	0\\
16.23	0\\
16.24	0\\
16.25	0\\
16.26	0\\
16.27	0\\
16.28	0\\
16.29	0\\
16.3	0\\
16.31	0\\
16.32	0\\
16.33	0\\
16.34	0\\
16.35	0\\
16.36	0\\
16.37	0\\
16.38	0\\
16.39	0\\
16.4	0\\
16.41	0\\
16.42	0\\
16.43	0\\
16.44	0\\
16.45	0\\
16.46	0\\
16.47	0\\
16.48	0\\
16.49	0\\
16.5	0\\
16.51	0\\
16.52	0\\
16.53	0\\
16.54	0\\
16.55	0\\
16.56	0\\
16.57	0\\
16.58	0\\
16.59	0\\
16.6	0\\
16.61	0\\
16.62	0\\
16.63	0\\
16.64	0\\
16.65	0\\
16.66	0\\
16.67	0\\
16.68	0\\
16.69	0\\
16.7	0\\
16.71	0\\
16.72	0\\
16.73	0\\
16.74	0\\
16.75	0\\
16.76	0\\
16.77	0\\
16.78	0\\
16.79	0\\
16.8	0\\
16.81	0\\
16.82	0\\
16.83	0\\
16.84	0\\
16.85	0\\
16.86	0\\
16.87	0\\
16.88	0\\
16.89	0\\
16.9	0\\
16.91	0\\
16.92	0\\
16.93	0\\
16.94	0\\
16.95	0\\
16.96	0\\
16.97	0\\
16.98	0\\
16.99	0\\
17	0\\
17.01	0\\
17.02	0\\
17.03	0\\
17.04	0\\
17.05	0\\
17.06	0\\
17.07	0\\
17.08	0\\
17.09	0\\
17.1	0\\
17.11	0\\
17.12	0\\
17.13	0\\
17.14	0\\
17.15	0\\
17.16	0\\
17.17	0\\
17.18	0\\
17.19	0\\
17.2	0\\
17.21	0\\
17.22	0\\
17.23	0\\
17.24	0\\
17.25	0\\
17.26	0\\
17.27	0\\
17.28	0\\
17.29	0\\
17.3	0\\
17.31	0\\
17.32	0\\
17.33	0\\
17.34	0\\
17.35	0\\
17.36	0\\
17.37	0\\
17.38	0\\
17.39	0\\
17.4	0\\
17.41	0\\
17.42	0\\
17.43	0\\
17.44	0\\
17.45	0\\
17.46	0\\
17.47	0\\
17.48	0\\
17.49	0\\
17.5	0\\
17.51	0\\
17.52	0\\
17.53	0\\
17.54	0\\
17.55	0\\
17.56	0\\
17.57	0\\
17.58	0\\
17.59	0\\
17.6	0\\
17.61	0\\
17.62	0\\
17.63	0\\
17.64	0\\
17.65	0\\
17.66	0\\
17.67	0\\
17.68	0\\
17.69	0\\
17.7	0\\
17.71	0\\
17.72	0\\
17.73	0\\
17.74	0\\
17.75	0\\
17.76	0\\
17.77	0\\
17.78	0\\
17.79	0\\
17.8	0\\
17.81	0\\
17.82	0\\
17.83	0\\
17.84	0\\
17.85	0\\
17.86	0\\
17.87	0\\
17.88	0\\
17.89	0\\
17.9	0\\
17.91	0\\
17.92	0\\
17.93	0\\
17.94	0\\
17.95	0\\
17.96	0\\
17.97	0\\
17.98	0\\
17.99	0\\
18	0\\
18.01	0\\
18.02	0\\
18.03	0\\
18.04	0\\
18.05	0\\
18.06	0\\
18.07	0\\
18.08	0\\
18.09	0\\
18.1	0\\
18.11	0\\
18.12	0\\
18.13	0\\
18.14	0\\
18.15	0\\
18.16	0\\
18.17	0\\
18.18	0\\
18.19	0\\
18.2	0\\
18.21	0\\
18.22	0\\
18.23	0\\
18.24	0\\
18.25	0\\
18.26	0\\
18.27	0\\
18.28	0\\
18.29	0\\
18.3	0\\
18.31	0\\
18.32	0\\
18.33	0\\
18.34	0\\
18.35	0\\
18.36	0\\
18.37	0\\
18.38	0\\
18.39	0\\
18.4	0\\
18.41	0\\
18.42	0\\
18.43	0\\
18.44	0\\
18.45	0\\
18.46	0\\
18.47	0\\
18.48	0\\
18.49	0\\
18.5	0\\
18.51	0\\
18.52	0\\
18.53	0\\
18.54	0\\
18.55	0\\
18.56	0\\
18.57	0\\
18.58	0\\
18.59	0\\
18.6	0\\
18.61	0\\
18.62	0\\
18.63	0\\
18.64	0\\
18.65	0\\
18.66	0\\
18.67	0\\
18.68	0\\
18.69	0\\
18.7	0\\
18.71	0\\
18.72	0\\
18.73	0\\
18.74	0\\
18.75	0\\
18.76	0\\
18.77	0\\
18.78	0\\
18.79	0\\
18.8	0\\
18.81	0\\
18.82	0\\
18.83	0\\
18.84	0\\
18.85	0\\
18.86	0\\
18.87	0\\
18.88	0\\
18.89	0\\
18.9	0\\
18.91	0\\
18.92	0\\
18.93	0\\
18.94	0\\
18.95	0\\
18.96	0\\
18.97	0\\
18.98	0\\
18.99	0\\
19	0\\
19.01	0\\
19.02	0\\
19.03	0\\
19.04	0\\
19.05	0\\
19.06	0\\
19.07	0\\
19.08	0\\
19.09	0\\
19.1	0\\
19.11	0\\
19.12	0\\
19.13	0\\
19.14	0\\
19.15	0\\
19.16	0\\
19.17	0\\
19.18	0\\
19.19	0\\
19.2	0\\
19.21	0\\
19.22	0\\
19.23	0\\
19.24	0\\
19.25	0\\
19.26	0\\
19.27	0\\
19.28	0\\
19.29	0\\
19.3	0\\
19.31	0\\
19.32	0\\
19.33	0\\
19.34	0\\
19.35	0\\
19.36	0\\
19.37	0\\
19.38	0\\
19.39	0\\
19.4	0\\
19.41	0\\
19.42	0\\
19.43	0\\
19.44	0\\
19.45	0\\
19.46	0\\
19.47	0\\
19.48	0\\
19.49	0\\
19.5	0\\
19.51	0\\
19.52	0\\
19.53	0\\
19.54	0\\
19.55	0\\
19.56	0\\
19.57	0\\
19.58	0\\
19.59	0\\
19.6	0\\
19.61	0\\
19.62	0\\
19.63	0\\
19.64	0\\
19.65	0\\
19.66	0\\
19.67	0\\
19.68	0\\
19.69	0\\
19.7	0\\
19.71	0\\
19.72	0\\
19.73	0\\
19.74	0\\
19.75	0\\
19.76	0\\
19.77	0\\
19.78	0\\
19.79	0\\
19.8	0\\
19.81	0\\
19.82	0\\
19.83	0\\
19.84	0\\
19.85	0\\
19.86	0\\
19.87	0\\
19.88	0\\
19.89	0\\
19.9	0\\
19.91	0\\
19.92	0\\
19.93	0\\
19.94	0\\
19.95	0\\
19.96	0\\
19.97	0\\
19.98	0\\
19.99	0\\
20	0\\
20.01	0\\
20.02	0\\
20.03	0\\
20.04	0\\
20.05	0\\
20.06	0\\
20.07	0\\
20.08	0\\
20.09	0\\
20.1	0\\
20.11	0\\
20.12	0\\
20.13	0\\
20.14	0\\
20.15	0\\
20.16	0\\
20.17	0\\
20.18	0\\
20.19	0\\
20.2	0\\
20.21	0\\
20.22	0\\
20.23	0\\
20.24	0\\
20.25	0\\
20.26	0\\
20.27	0\\
20.28	0\\
20.29	0\\
20.3	0\\
20.31	0\\
20.32	0\\
20.33	0\\
20.34	0\\
20.35	0\\
20.36	0\\
20.37	0\\
20.38	0\\
20.39	0\\
20.4	0\\
20.41	0\\
20.42	0\\
20.43	0\\
20.44	0\\
20.45	0\\
20.46	0\\
20.47	0\\
20.48	0\\
20.49	0\\
20.5	0\\
20.51	0\\
20.52	0\\
20.53	0\\
20.54	0\\
20.55	0\\
20.56	0\\
20.57	0\\
20.58	0\\
20.59	0\\
20.6	0\\
20.61	0\\
20.62	0\\
20.63	0\\
20.64	0\\
20.65	0\\
20.66	0\\
20.67	0\\
20.68	0\\
20.69	0\\
20.7	0\\
20.71	0\\
20.72	0\\
20.73	0\\
20.74	0\\
20.75	0\\
20.76	0\\
20.77	0\\
20.78	0\\
20.79	0\\
20.8	0\\
20.81	0\\
20.82	0\\
20.83	0\\
20.84	0\\
20.85	0\\
20.86	0\\
20.87	0\\
20.88	0\\
20.89	0\\
20.9	0\\
20.91	0\\
20.92	0\\
20.93	0\\
20.94	0\\
20.95	0\\
20.96	0\\
20.97	0\\
20.98	0\\
20.99	0\\
21	0\\
21.01	0\\
21.02	0\\
21.03	0\\
21.04	0\\
21.05	0\\
21.06	0\\
21.07	0\\
21.08	0\\
21.09	0\\
21.1	0\\
21.11	0\\
21.12	0\\
21.13	0\\
21.14	0\\
21.15	0\\
21.16	0\\
21.17	0\\
21.18	0\\
21.19	0\\
21.2	0\\
21.21	0\\
21.22	0\\
21.23	0\\
21.24	0\\
21.25	0\\
21.26	0\\
21.27	0\\
21.28	0\\
21.29	0\\
21.3	0\\
21.31	0\\
21.32	0\\
21.33	0\\
21.34	0\\
21.35	0\\
21.36	0\\
21.37	0\\
21.38	0\\
21.39	0\\
21.4	0\\
21.41	0\\
21.42	0\\
21.43	0\\
21.44	0\\
21.45	0\\
21.46	0\\
21.47	0\\
21.48	0\\
21.49	0\\
21.5	0\\
21.51	0\\
21.52	0\\
21.53	0\\
21.54	0\\
21.55	0\\
21.56	0\\
21.57	0\\
21.58	0\\
21.59	0\\
21.6	0\\
21.61	0\\
21.62	0\\
21.63	0\\
21.64	0\\
21.65	0\\
21.66	0\\
21.67	0\\
21.68	0\\
21.69	0\\
21.7	0\\
21.71	0\\
21.72	0\\
21.73	0\\
21.74	0\\
21.75	0\\
21.76	0\\
21.77	0\\
21.78	0\\
21.79	0\\
21.8	0\\
21.81	0\\
21.82	0\\
21.83	0\\
21.84	0\\
21.85	0\\
21.86	0\\
21.87	0\\
21.88	0\\
21.89	0\\
21.9	0\\
21.91	0\\
21.92	0\\
21.93	0\\
21.94	0\\
21.95	0\\
21.96	0\\
21.97	0\\
21.98	0\\
21.99	0\\
22	0\\
22.01	0\\
22.02	0\\
22.03	0\\
22.04	0\\
22.05	0\\
22.06	0\\
22.07	0\\
22.08	0\\
22.09	0\\
22.1	0\\
22.11	0\\
22.12	0\\
22.13	0\\
22.14	0\\
22.15	0\\
22.16	0\\
22.17	0\\
22.18	0\\
22.19	0\\
22.2	0\\
22.21	0\\
22.22	0\\
22.23	0\\
22.24	0\\
22.25	0\\
22.26	0\\
22.27	0\\
22.28	0\\
22.29	0\\
22.3	0\\
22.31	0\\
22.32	0\\
22.33	0\\
22.34	0\\
22.35	0\\
22.36	0\\
22.37	0\\
22.38	0\\
22.39	0\\
22.4	0\\
22.41	0\\
22.42	0\\
22.43	0\\
22.44	0\\
22.45	0\\
22.46	0\\
22.47	0\\
22.48	0\\
22.49	0\\
22.5	0\\
22.51	0\\
22.52	0\\
22.53	0\\
22.54	0\\
22.55	0\\
22.56	0\\
22.57	0\\
22.58	0\\
22.59	0\\
22.6	0\\
22.61	0\\
22.62	0\\
22.63	0\\
22.64	0\\
22.65	0\\
22.66	0\\
22.67	0\\
22.68	0\\
22.69	0\\
22.7	0\\
22.71	0\\
22.72	0\\
22.73	0\\
22.74	0\\
22.75	0\\
22.76	0\\
22.77	0\\
22.78	0\\
22.79	0\\
22.8	0\\
22.81	0\\
22.82	0\\
22.83	0\\
22.84	0\\
22.85	0\\
22.86	0\\
22.87	0\\
22.88	0\\
22.89	0\\
22.9	0\\
22.91	0\\
22.92	0\\
22.93	0\\
22.94	0\\
22.95	0\\
22.96	0\\
22.97	0\\
22.98	0\\
22.99	0\\
23	0\\
23.01	0\\
23.02	0\\
23.03	0\\
23.04	0\\
23.05	0\\
23.06	0\\
23.07	0\\
23.08	0\\
23.09	0\\
23.1	0\\
23.11	0\\
23.12	0\\
23.13	0\\
23.14	0\\
23.15	0\\
23.16	0\\
23.17	0\\
23.18	0\\
23.19	0\\
23.2	0\\
23.21	0\\
23.22	0\\
23.23	0\\
23.24	0\\
23.25	0\\
23.26	0\\
23.27	0\\
23.28	0\\
23.29	0\\
23.3	0\\
23.31	0\\
23.32	0\\
23.33	0\\
23.34	0\\
23.35	0\\
23.36	0\\
23.37	0\\
23.38	0\\
23.39	0\\
23.4	0\\
23.41	0\\
23.42	0\\
23.43	0\\
23.44	0\\
23.45	0\\
23.46	0\\
23.47	0\\
23.48	0\\
23.49	0\\
23.5	0\\
23.51	0\\
23.52	0\\
23.53	0\\
23.54	0\\
23.55	0\\
23.56	0\\
23.57	0\\
23.58	0\\
23.59	0\\
23.6	0\\
23.61	0\\
23.62	0\\
23.63	0\\
23.64	0\\
23.65	0\\
23.66	0\\
23.67	0\\
23.68	0\\
23.69	0\\
23.7	0\\
23.71	0\\
23.72	0\\
23.73	0\\
23.74	0\\
23.75	0\\
23.76	0\\
23.77	0\\
23.78	0\\
23.79	0\\
23.8	0\\
23.81	0\\
23.82	0\\
23.83	0\\
23.84	0\\
23.85	0\\
23.86	0\\
23.87	0\\
23.88	0\\
23.89	0\\
23.9	0\\
23.91	0\\
23.92	0\\
23.93	0\\
23.94	0\\
23.95	0\\
23.96	0\\
23.97	0\\
23.98	0\\
23.99	0\\
24	0\\
24.01	0\\
24.02	0\\
24.03	0\\
24.04	0\\
24.05	0\\
24.06	0\\
24.07	0\\
24.08	0\\
24.09	0\\
24.1	0\\
24.11	0\\
24.12	0\\
24.13	0\\
24.14	0\\
24.15	0\\
24.16	0\\
24.17	0\\
24.18	0\\
24.19	0\\
24.2	0\\
24.21	0\\
24.22	0\\
24.23	0\\
24.24	0\\
24.25	0\\
24.26	0\\
24.27	0\\
24.28	0\\
24.29	0\\
24.3	0\\
24.31	0\\
24.32	0\\
24.33	0\\
24.34	0\\
24.35	0\\
24.36	0\\
24.37	0\\
24.38	0\\
24.39	0\\
24.4	0\\
24.41	0\\
24.42	0\\
24.43	0\\
24.44	0\\
24.45	0\\
24.46	0\\
24.47	0\\
24.48	0\\
24.49	0\\
24.5	0\\
24.51	0\\
24.52	0\\
24.53	0\\
24.54	0\\
24.55	0\\
24.56	0\\
24.57	0\\
24.58	0\\
24.59	0\\
24.6	0\\
24.61	0\\
24.62	0\\
24.63	0\\
24.64	0\\
24.65	0\\
24.66	0\\
24.67	0\\
24.68	0\\
24.69	0\\
24.7	0\\
24.71	0\\
24.72	0\\
24.73	0\\
24.74	0\\
24.75	0\\
24.76	0\\
24.77	0\\
24.78	0\\
24.79	0\\
24.8	0\\
24.81	0\\
24.82	0\\
24.83	0\\
24.84	0\\
24.85	0\\
24.86	0\\
24.87	0\\
24.88	0\\
24.89	0\\
24.9	0\\
24.91	0\\
24.92	0\\
24.93	0\\
24.94	0\\
24.95	0\\
24.96	0\\
24.97	0\\
24.98	0\\
24.99	0\\
25	0\\
25.01	0\\
25.02	0\\
25.03	0\\
25.04	0\\
25.05	0\\
25.06	0\\
25.07	0\\
25.08	0\\
25.09	0\\
25.1	0\\
25.11	0\\
25.12	0\\
25.13	0\\
25.14	0\\
25.15	0\\
25.16	0\\
25.17	0\\
25.18	0\\
25.19	0\\
25.2	0\\
25.21	0\\
25.22	0\\
25.23	0\\
25.24	0\\
25.25	0\\
25.26	0\\
25.27	0\\
25.28	0\\
25.29	0\\
25.3	0\\
25.31	0\\
25.32	0\\
25.33	0\\
25.34	0\\
25.35	0\\
25.36	0\\
25.37	0\\
25.38	0\\
25.39	0\\
25.4	0\\
25.41	0\\
25.42	0\\
25.43	0\\
25.44	0\\
25.45	0\\
25.46	0\\
25.47	0\\
25.48	0\\
25.49	0\\
25.5	0\\
25.51	0\\
25.52	0\\
25.53	0\\
25.54	0\\
25.55	0\\
25.56	0\\
25.57	0\\
25.58	0\\
25.59	0\\
25.6	0\\
25.61	0\\
25.62	0\\
25.63	0\\
25.64	0\\
25.65	0\\
25.66	0\\
25.67	0\\
25.68	0\\
25.69	0\\
25.7	0\\
25.71	0\\
25.72	0\\
25.73	0\\
25.74	0\\
25.75	0\\
25.76	0\\
25.77	0\\
25.78	0\\
25.79	0\\
25.8	0\\
25.81	0\\
25.82	0\\
25.83	0\\
25.84	0\\
25.85	0\\
25.86	0\\
25.87	0\\
25.88	0\\
25.89	0\\
25.9	0\\
25.91	0\\
25.92	0\\
25.93	0\\
25.94	0\\
25.95	0\\
25.96	0\\
25.97	0\\
25.98	0\\
25.99	0\\
26	0\\
26.01	0\\
26.02	0\\
26.03	0\\
26.04	0\\
26.05	0\\
26.06	0\\
26.07	0\\
26.08	0\\
26.09	0\\
26.1	0\\
26.11	0\\
26.12	0\\
26.13	0\\
26.14	0\\
26.15	0\\
26.16	0\\
26.17	0\\
26.18	0\\
26.19	0\\
26.2	0\\
26.21	0\\
26.22	0\\
26.23	0\\
26.24	0\\
26.25	0\\
26.26	0\\
26.27	0\\
26.28	0\\
26.29	0\\
26.3	0\\
26.31	0\\
26.32	0\\
26.33	0\\
26.34	0\\
26.35	0\\
26.36	0\\
26.37	0\\
26.38	0\\
26.39	0\\
26.4	0\\
26.41	0\\
26.42	0\\
26.43	0\\
26.44	0\\
26.45	0\\
26.46	0\\
26.47	0\\
26.48	0\\
26.49	0\\
26.5	0\\
26.51	0\\
26.52	0\\
26.53	0\\
26.54	0\\
26.55	0\\
26.56	0\\
26.57	0\\
26.58	0\\
26.59	0\\
26.6	0\\
26.61	0\\
26.62	0\\
26.63	0\\
26.64	0\\
26.65	0\\
26.66	0\\
26.67	0\\
26.68	0\\
26.69	0\\
26.7	0\\
26.71	0\\
26.72	0\\
26.73	0\\
26.74	0\\
26.75	0\\
26.76	0\\
26.77	0\\
26.78	0\\
26.79	0\\
26.8	0\\
26.81	0\\
26.82	0\\
26.83	0\\
26.84	0\\
26.85	0\\
26.86	0\\
26.87	0\\
26.88	0\\
26.89	0\\
26.9	0\\
26.91	0\\
26.92	0\\
26.93	0\\
26.94	0\\
26.95	0\\
26.96	0\\
26.97	0\\
26.98	0\\
26.99	0\\
27	0\\
27.01	0\\
27.02	0\\
27.03	0\\
27.04	0\\
27.05	0\\
27.06	0\\
27.07	0\\
27.08	0\\
27.09	0\\
27.1	0\\
27.11	0\\
27.12	0\\
27.13	0\\
27.14	0\\
27.15	0\\
27.16	0\\
27.17	0\\
27.18	0\\
27.19	0\\
27.2	0\\
27.21	0\\
27.22	0\\
27.23	0\\
27.24	0\\
27.25	0\\
27.26	0\\
27.27	0\\
27.28	0\\
27.29	0\\
27.3	0\\
27.31	0\\
27.32	0\\
27.33	0\\
27.34	0\\
27.35	0\\
27.36	0\\
27.37	0\\
27.38	0\\
27.39	0\\
27.4	0\\
27.41	0\\
27.42	0\\
27.43	0\\
27.44	0\\
27.45	0\\
27.46	0\\
27.47	0\\
27.48	0\\
27.49	0\\
27.5	0\\
27.51	0\\
27.52	0\\
27.53	0\\
27.54	0\\
27.55	0\\
27.56	0\\
27.57	0\\
27.58	0\\
27.59	0\\
27.6	0\\
27.61	0\\
27.62	0\\
27.63	0\\
27.64	0\\
27.65	0\\
27.66	0\\
27.67	0\\
27.68	0\\
27.69	0\\
27.7	0\\
27.71	0\\
27.72	0\\
27.73	0\\
27.74	0\\
27.75	0\\
27.76	0\\
27.77	0\\
27.78	0\\
27.79	0\\
27.8	0\\
27.81	0\\
27.82	0\\
27.83	0\\
27.84	0\\
27.85	0\\
27.86	0\\
27.87	0\\
27.88	0\\
27.89	0\\
27.9	0\\
27.91	0\\
27.92	0\\
27.93	0\\
27.94	0\\
27.95	0\\
27.96	0\\
27.97	0\\
27.98	0\\
27.99	0\\
28	0\\
28.01	0\\
28.02	0\\
28.03	0\\
28.04	0\\
28.05	0\\
28.06	0\\
28.07	0\\
28.08	0\\
28.09	0\\
28.1	0\\
28.11	0\\
28.12	0\\
28.13	0\\
28.14	0\\
28.15	0\\
28.16	0\\
28.17	0\\
28.18	0\\
28.19	0\\
28.2	0\\
28.21	0\\
28.22	0\\
28.23	0\\
28.24	0\\
28.25	0\\
28.26	0\\
28.27	0\\
28.28	0\\
28.29	0\\
28.3	0\\
28.31	0\\
28.32	0\\
28.33	0\\
28.34	0\\
28.35	0\\
28.36	0\\
28.37	0\\
28.38	0\\
28.39	0\\
28.4	0\\
28.41	0\\
28.42	0\\
28.43	0\\
28.44	0\\
28.45	0\\
28.46	0\\
28.47	0\\
28.48	0\\
28.49	0\\
28.5	0\\
28.51	0\\
28.52	0\\
28.53	0\\
28.54	0\\
28.55	0\\
28.56	0\\
28.57	0\\
28.58	0\\
28.59	0\\
28.6	0\\
28.61	0\\
28.62	0\\
28.63	0\\
28.64	0\\
28.65	0\\
28.66	0\\
28.67	0\\
28.68	0\\
28.69	0\\
28.7	0\\
28.71	0\\
28.72	0\\
28.73	0\\
28.74	0\\
28.75	0\\
28.76	0\\
28.77	0\\
28.78	0\\
28.79	0\\
28.8	0\\
28.81	0\\
28.82	0\\
28.83	0\\
28.84	0\\
28.85	0\\
28.86	0\\
28.87	0\\
28.88	0\\
28.89	0\\
28.9	0\\
28.91	0\\
28.92	0\\
28.93	0\\
28.94	0\\
28.95	0\\
28.96	0\\
28.97	0\\
28.98	0\\
28.99	0\\
29	0\\
29.01	0\\
29.02	0\\
29.03	0\\
29.04	0\\
29.05	0\\
29.06	0\\
29.07	0\\
29.08	0\\
29.09	0\\
29.1	0\\
29.11	0\\
29.12	0\\
29.13	0\\
29.14	0\\
29.15	0\\
29.16	0\\
29.17	0\\
29.18	0\\
29.19	0\\
29.2	0\\
29.21	0\\
29.22	0\\
29.23	0\\
29.24	0\\
29.25	0\\
29.26	0\\
29.27	0\\
29.28	0\\
29.29	0\\
29.3	0\\
29.31	0\\
29.32	0\\
29.33	0\\
29.34	0\\
29.35	0\\
29.36	0\\
29.37	0\\
29.38	0\\
29.39	0\\
29.4	0\\
29.41	0\\
29.42	0\\
29.43	0\\
29.44	0\\
29.45	0\\
29.46	0\\
29.47	0\\
29.48	0\\
29.49	0\\
29.5	0\\
29.51	0\\
29.52	0\\
29.53	0\\
29.54	0\\
29.55	0\\
29.56	0\\
29.57	0\\
29.58	0\\
29.59	0\\
29.6	0\\
29.61	0\\
29.62	0\\
29.63	0\\
29.64	0\\
29.65	0\\
29.66	0\\
29.67	0\\
29.68	0\\
29.69	0\\
29.7	0\\
29.71	0\\
29.72	0\\
29.73	0\\
29.74	0\\
29.75	0\\
29.76	0\\
29.77	0\\
29.78	0\\
29.79	0\\
29.8	0\\
29.81	0\\
29.82	0\\
29.83	0\\
29.84	0\\
29.85	0\\
29.86	0\\
29.87	0\\
29.88	0\\
29.89	0\\
29.9	0\\
29.91	0\\
29.92	0\\
29.93	0\\
29.94	0\\
29.95	0\\
29.96	0\\
29.97	0\\
29.98	0\\
29.99	0\\
30	0\\
30.01	0\\
30.02	0\\
30.03	0\\
30.04	0\\
30.05	0\\
30.06	0\\
30.07	0\\
30.08	0\\
30.09	0\\
30.1	0\\
30.11	0\\
30.12	0\\
30.13	0\\
30.14	0\\
30.15	0\\
30.16	0\\
30.17	0\\
30.18	0\\
30.19	0\\
30.2	0\\
30.21	0\\
30.22	0\\
30.23	0\\
30.24	0\\
30.25	0\\
30.26	0\\
30.27	0\\
30.28	0\\
30.29	0\\
30.3	0\\
30.31	0\\
30.32	0\\
30.33	0\\
30.34	0\\
30.35	0\\
30.36	0\\
30.37	0\\
30.38	0\\
30.39	0\\
30.4	0\\
30.41	0\\
30.42	0\\
30.43	0\\
30.44	0\\
30.45	0\\
30.46	0\\
30.47	0\\
30.48	0\\
30.49	0\\
30.5	0\\
30.51	0\\
30.52	0\\
30.53	0\\
30.54	0\\
30.55	0\\
30.56	0\\
30.57	0\\
30.58	0\\
30.59	0\\
30.6	0\\
30.61	0\\
30.62	0\\
30.63	0\\
30.64	0\\
30.65	0\\
30.66	0\\
30.67	0\\
30.68	0\\
30.69	0\\
30.7	0\\
30.71	0\\
30.72	0\\
30.73	0\\
30.74	0\\
30.75	0\\
30.76	0\\
30.77	0\\
30.78	0\\
30.79	0\\
30.8	0\\
30.81	0\\
30.82	0\\
30.83	0\\
30.84	0\\
30.85	0\\
30.86	0\\
30.87	0\\
30.88	0\\
30.89	0\\
30.9	0\\
30.91	0\\
30.92	0\\
30.93	0\\
30.94	0\\
30.95	0\\
30.96	0\\
30.97	0\\
30.98	0\\
30.99	0\\
31	0\\
31.01	0\\
31.02	0\\
31.03	0\\
31.04	0\\
31.05	0\\
31.06	0\\
31.07	0\\
31.08	0\\
31.09	0\\
31.1	0\\
31.11	0\\
31.12	0\\
31.13	0\\
31.14	0\\
31.15	0\\
31.16	0\\
31.17	0\\
31.18	0\\
31.19	0\\
31.2	0\\
31.21	0\\
31.22	0\\
31.23	0\\
31.24	0\\
31.25	0\\
31.26	0\\
31.27	0\\
31.28	0\\
31.29	0\\
31.3	0\\
31.31	0\\
31.32	0\\
31.33	0\\
31.34	0\\
31.35	0\\
31.36	0\\
31.37	0\\
31.38	0\\
31.39	0\\
31.4	0\\
31.41	0\\
31.42	0\\
31.43	0\\
31.44	0\\
31.45	0\\
31.46	0\\
31.47	0\\
31.48	0\\
31.49	0\\
31.5	0\\
31.51	0\\
31.52	0\\
31.53	0\\
31.54	0\\
31.55	0\\
31.56	0\\
31.57	0\\
31.58	0\\
31.59	0\\
31.6	0\\
31.61	0\\
31.62	0\\
31.63	0\\
31.64	0\\
31.65	0\\
31.66	0\\
31.67	0\\
31.68	0\\
31.69	0\\
31.7	0\\
31.71	0\\
31.72	0\\
31.73	0\\
31.74	0\\
31.75	0\\
31.76	0\\
31.77	0\\
31.78	0\\
31.79	0\\
31.8	0\\
31.81	0\\
31.82	0\\
31.83	0\\
31.84	0\\
31.85	0\\
31.86	0\\
31.87	0\\
31.88	0\\
31.89	0\\
31.9	0\\
31.91	0\\
31.92	0\\
31.93	0\\
31.94	0\\
31.95	0\\
31.96	0\\
31.97	0\\
31.98	0\\
31.99	0\\
32	0\\
32.01	0\\
32.02	0\\
32.03	0\\
32.04	0\\
32.05	0\\
32.06	0\\
32.07	0\\
32.08	0\\
32.09	0\\
32.1	0\\
32.11	0\\
32.12	0\\
32.13	0\\
32.14	0\\
32.15	0\\
32.16	0\\
32.17	0\\
32.18	0\\
32.19	0\\
32.2	0\\
32.21	0\\
32.22	0\\
32.23	0\\
32.24	0\\
32.25	0\\
32.26	0\\
32.27	0\\
32.28	0\\
32.29	0\\
32.3	0\\
32.31	0\\
32.32	0\\
32.33	0\\
32.34	0\\
32.35	0\\
32.36	0\\
32.37	0\\
32.38	0\\
32.39	0\\
32.4	0\\
32.41	0\\
32.42	0\\
32.43	0\\
32.44	0\\
32.45	0\\
32.46	0\\
32.47	0\\
32.48	0\\
32.49	0\\
32.5	0\\
32.51	0\\
32.52	0\\
32.53	0\\
32.54	0\\
32.55	0\\
32.56	0\\
32.57	0\\
32.58	0\\
32.59	0\\
32.6	0\\
32.61	0\\
32.62	0\\
32.63	0\\
32.64	0\\
32.65	0\\
32.66	0\\
32.67	0\\
32.68	0\\
32.69	0\\
32.7	0\\
32.71	0\\
32.72	0\\
32.73	0\\
32.74	0\\
32.75	0\\
32.76	0\\
32.77	0\\
32.78	0\\
32.79	0\\
32.8	0\\
32.81	0\\
32.82	0\\
32.83	0\\
32.84	0\\
32.85	0\\
32.86	0\\
32.87	0\\
32.88	0\\
32.89	0\\
32.9	0\\
32.91	0\\
32.92	0\\
32.93	0\\
32.94	0\\
32.95	0\\
32.96	0\\
32.97	0\\
32.98	0\\
32.99	0\\
33	0\\
33.01	0\\
33.02	0\\
33.03	0\\
33.04	0\\
33.05	0\\
33.06	0\\
33.07	0\\
33.08	0\\
33.09	0\\
33.1	0\\
33.11	0\\
33.12	0\\
33.13	0\\
33.14	0\\
33.15	0\\
33.16	0\\
33.17	0\\
33.18	0\\
33.19	0\\
33.2	0\\
33.21	0\\
33.22	0\\
33.23	0\\
33.24	0\\
33.25	0\\
33.26	0\\
33.27	0\\
33.28	0\\
33.29	0\\
33.3	0\\
33.31	0\\
33.32	0\\
33.33	0\\
33.34	0\\
33.35	0\\
33.36	0\\
33.37	0\\
33.38	0\\
33.39	0\\
33.4	0\\
33.41	0\\
33.42	0\\
33.43	0\\
33.44	0\\
33.45	0\\
33.46	0\\
33.47	0\\
33.48	0\\
33.49	0\\
33.5	0\\
33.51	0\\
33.52	0\\
33.53	0\\
33.54	0\\
33.55	0\\
33.56	0\\
33.57	0\\
33.58	0\\
33.59	0\\
33.6	0\\
33.61	0\\
33.62	0\\
33.63	0\\
33.64	0\\
33.65	0\\
33.66	0\\
33.67	0\\
33.68	0\\
33.69	0\\
33.7	0\\
33.71	0\\
33.72	0\\
33.73	0\\
33.74	0\\
33.75	0\\
33.76	0\\
33.77	0\\
33.78	0\\
33.79	0\\
33.8	0\\
33.81	0\\
33.82	0\\
33.83	0\\
33.84	0\\
33.85	0\\
33.86	0\\
33.87	0\\
33.88	0\\
33.89	0\\
33.9	0\\
33.91	0\\
33.92	0\\
33.93	0\\
33.94	0\\
33.95	0\\
33.96	0\\
33.97	0\\
33.98	0\\
33.99	0\\
34	0\\
34.01	0\\
34.02	0\\
34.03	0\\
34.04	0\\
34.05	0\\
34.06	0\\
34.07	0\\
34.08	0\\
34.09	0\\
34.1	0\\
34.11	0\\
34.12	0\\
34.13	0\\
34.14	0\\
34.15	0\\
34.16	0\\
34.17	0\\
34.18	0\\
34.19	0\\
34.2	0\\
34.21	0\\
34.22	0\\
34.23	0\\
34.24	0\\
34.25	0\\
34.26	0\\
34.27	0\\
34.28	0\\
34.29	0\\
34.3	0\\
34.31	0\\
34.32	0\\
34.33	0\\
34.34	0\\
34.35	0\\
34.36	0\\
34.37	0\\
34.38	0\\
34.39	0\\
34.4	0\\
34.41	0\\
34.42	0\\
34.43	0\\
34.44	0\\
34.45	0\\
34.46	0\\
34.47	0\\
34.48	0\\
34.49	0\\
34.5	0\\
34.51	0\\
34.52	0\\
34.53	0\\
34.54	0\\
34.55	0\\
34.56	0\\
34.57	0\\
34.58	0\\
34.59	0\\
34.6	0\\
34.61	0\\
34.62	0\\
34.63	0\\
34.64	0\\
34.65	0\\
34.66	0\\
34.67	0\\
34.68	0\\
34.69	0\\
34.7	0\\
34.71	0\\
34.72	0\\
34.73	0\\
34.74	0\\
34.75	0\\
34.76	0\\
34.77	0\\
34.78	0\\
34.79	0\\
34.8	0\\
34.81	0\\
34.82	0\\
34.83	0\\
34.84	0\\
34.85	0\\
34.86	0\\
34.87	0\\
34.88	0\\
34.89	0\\
34.9	0\\
34.91	0\\
34.92	0\\
34.93	0\\
34.94	0\\
34.95	0\\
34.96	0\\
34.97	0\\
34.98	0\\
34.99	0\\
35	0\\
35.01	0\\
35.02	0\\
35.03	0\\
35.04	0\\
35.05	0\\
35.06	0\\
35.07	0\\
35.08	0\\
35.09	0\\
35.1	0\\
35.11	0\\
35.12	0\\
35.13	0\\
35.14	0\\
35.15	0\\
35.16	0\\
35.17	0\\
35.18	0\\
35.19	0\\
35.2	0\\
35.21	0\\
35.22	0\\
35.23	0\\
35.24	0\\
35.25	0\\
35.26	0\\
35.27	0\\
35.28	0\\
35.29	0\\
35.3	0\\
35.31	0\\
35.32	0\\
35.33	0\\
35.34	0\\
35.35	0\\
35.36	0\\
35.37	0\\
35.38	0\\
35.39	0\\
35.4	0\\
35.41	0\\
35.42	0\\
35.43	0\\
35.44	0\\
35.45	0\\
35.46	0\\
35.47	0\\
35.48	0\\
35.49	0\\
35.5	0\\
35.51	0\\
35.52	0\\
35.53	0\\
35.54	0\\
35.55	0\\
35.56	0\\
35.57	0\\
35.58	0\\
35.59	0\\
35.6	0\\
35.61	0\\
35.62	0\\
35.63	0\\
35.64	0\\
35.65	0\\
35.66	0\\
35.67	0\\
35.68	0\\
35.69	0\\
35.7	0\\
35.71	0\\
35.72	0\\
35.73	0\\
35.74	0\\
35.75	0\\
35.76	0\\
35.77	0\\
35.78	0\\
35.79	0\\
35.8	0\\
35.81	0\\
35.82	0\\
35.83	0\\
35.84	0\\
35.85	0\\
35.86	0\\
35.87	0\\
35.88	0\\
35.89	0\\
35.9	0\\
35.91	0\\
35.92	0\\
35.93	0\\
35.94	0\\
35.95	0\\
35.96	0\\
35.97	0\\
35.98	0\\
35.99	0\\
36	0\\
36.01	0\\
36.02	0\\
36.03	0\\
36.04	0\\
36.05	0\\
36.06	0\\
36.07	0\\
36.08	0\\
36.09	0\\
36.1	0\\
36.11	0\\
36.12	0\\
36.13	0\\
36.14	0\\
36.15	0\\
36.16	0\\
36.17	0\\
36.18	0\\
36.19	0\\
36.2	0\\
36.21	0\\
36.22	0\\
36.23	0\\
36.24	0\\
36.25	0\\
36.26	0\\
36.27	0\\
36.28	0\\
36.29	0\\
36.3	0\\
36.31	0\\
36.32	0\\
36.33	0\\
36.34	0\\
36.35	0\\
36.36	0\\
36.37	0\\
36.38	0\\
36.39	0\\
36.4	0\\
36.41	0\\
36.42	0\\
36.43	0\\
36.44	0\\
36.45	0\\
36.46	0\\
36.47	0\\
36.48	0\\
36.49	0\\
36.5	0\\
36.51	0\\
36.52	0\\
36.53	0\\
36.54	0\\
36.55	0\\
36.56	0\\
36.57	0\\
36.58	0\\
36.59	0\\
36.6	0\\
36.61	0\\
36.62	0\\
36.63	0\\
36.64	0\\
36.65	0\\
36.66	0\\
36.67	0\\
36.68	0\\
36.69	0\\
36.7	0\\
36.71	0\\
36.72	0\\
36.73	0\\
36.74	0\\
36.75	0\\
36.76	0\\
36.77	0\\
36.78	0\\
36.79	0\\
36.8	0\\
36.81	0\\
36.82	0\\
36.83	0\\
36.84	0\\
36.85	0\\
36.86	0\\
36.87	0\\
36.88	0\\
36.89	0\\
36.9	0\\
36.91	0\\
36.92	0\\
36.93	0\\
36.94	0\\
36.95	0\\
36.96	0\\
36.97	0\\
36.98	0\\
36.99	0\\
37	0\\
37.01	0\\
37.02	0\\
37.03	0\\
37.04	0\\
37.05	0\\
37.06	0\\
37.07	0\\
37.08	0\\
37.09	0\\
37.1	0\\
37.11	0\\
37.12	0\\
37.13	0\\
37.14	0\\
37.15	0\\
37.16	0\\
37.17	0\\
37.18	0\\
37.19	0\\
37.2	0\\
37.21	0\\
37.22	0\\
37.23	0\\
37.24	0\\
37.25	0\\
37.26	0\\
37.27	0\\
37.28	0\\
37.29	0\\
37.3	0\\
37.31	0\\
37.32	0\\
37.33	0\\
37.34	0\\
37.35	0\\
37.36	0\\
37.37	0\\
37.38	0\\
37.39	0\\
37.4	0\\
37.41	0\\
37.42	0\\
37.43	0\\
37.44	0\\
37.45	0\\
37.46	0\\
37.47	0\\
37.48	0\\
37.49	0\\
37.5	0\\
37.51	0\\
37.52	0\\
37.53	0\\
37.54	0\\
37.55	0\\
37.56	0\\
37.57	0\\
37.58	0\\
37.59	0\\
37.6	0\\
37.61	0\\
37.62	0\\
37.63	0\\
37.64	0\\
37.65	0\\
37.66	0\\
37.67	0\\
37.68	0\\
37.69	0\\
37.7	0\\
37.71	0\\
37.72	0\\
37.73	0\\
37.74	0\\
37.75	0\\
37.76	0\\
37.77	0\\
37.78	0\\
37.79	0\\
37.8	0\\
37.81	0\\
37.82	0\\
37.83	0\\
37.84	0\\
37.85	0\\
37.86	0\\
37.87	0\\
37.88	0\\
37.89	0\\
37.9	0\\
37.91	0\\
37.92	0\\
37.93	0\\
37.94	0\\
37.95	0\\
37.96	0\\
37.97	0\\
37.98	0\\
37.99	0\\
38	0\\
38.01	0\\
38.02	0\\
38.03	0\\
38.04	0\\
38.05	0\\
38.06	0\\
38.07	0\\
38.08	0\\
38.09	0\\
38.1	0\\
38.11	0\\
38.12	0\\
38.13	0\\
38.14	0\\
38.15	0\\
38.16	0\\
38.17	0\\
38.18	0\\
38.19	0\\
38.2	0\\
38.21	0\\
38.22	0\\
38.23	0\\
38.24	0\\
38.25	0\\
38.26	0\\
38.27	0\\
38.28	0\\
38.29	0\\
38.3	0\\
38.31	0\\
38.32	0\\
38.33	0\\
38.34	0\\
38.35	0\\
38.36	0\\
38.37	0\\
38.38	0\\
38.39	0\\
38.4	0\\
38.41	0\\
38.42	0\\
38.43	0\\
38.44	0\\
38.45	0\\
38.46	0\\
38.47	0\\
38.48	0\\
38.49	0\\
38.5	0\\
38.51	0\\
38.52	0\\
38.53	0\\
38.54	0\\
38.55	0\\
38.56	0\\
38.57	0\\
38.58	0\\
38.59	0\\
38.6	0\\
38.61	1.73472347597681e-18\\
38.62	0\\
38.63	0\\
38.64	0\\
38.65	0\\
38.66	0\\
38.67	0\\
38.68	0\\
38.69	0\\
38.7	0\\
38.71	0\\
38.72	0\\
38.73	0\\
38.74	0\\
38.75	0\\
38.76	0\\
38.77	1.73472347597681e-18\\
38.78	0\\
38.79	0\\
38.8	0\\
38.81	0\\
38.82	0\\
38.83	0\\
38.84	0\\
38.85	0\\
38.86	0\\
38.87	0\\
38.88	0\\
38.89	0\\
38.9	0\\
38.91	0\\
38.92	0\\
38.93	1.73472347597681e-18\\
38.94	0\\
38.95	0\\
38.96	0\\
38.97	0\\
38.98	0\\
38.99	0\\
39	0\\
39.01	0\\
39.02	0\\
39.03	0\\
39.04	0\\
39.05	0\\
39.06	0\\
39.07	0\\
39.08	0\\
39.09	1.73472347597681e-18\\
39.1	0\\
39.11	0\\
39.12	0\\
39.13	0\\
39.14	0\\
39.15	0\\
39.16	0\\
39.17	0\\
39.18	0\\
39.19	0\\
39.2	0\\
39.21	0\\
39.22	0\\
39.23	0\\
39.24	0\\
39.25	1.73472347597681e-18\\
39.26	0\\
39.27	0\\
39.28	0\\
39.29	0\\
39.3	0\\
39.31	0\\
39.32	0\\
39.33	0\\
39.34	0\\
39.35	0\\
39.36	0\\
39.37	0\\
39.38	0\\
39.39	0\\
39.4	0\\
39.41	0\\
39.42	0\\
39.43	0\\
39.44	0\\
39.45	0\\
39.46	0\\
39.47	0\\
39.48	0\\
39.49	0\\
39.5	0\\
39.51	0\\
39.52	0\\
39.53	0\\
39.54	0\\
39.55	0\\
39.56	0\\
39.57	0\\
39.58	0\\
39.59	0\\
39.6	0\\
39.61	0\\
39.62	0\\
39.63	0\\
39.64	0\\
39.65	0\\
39.66	0\\
39.67	0\\
39.68	0\\
39.69	0\\
39.7	0\\
39.71	0\\
39.72	0\\
39.73	0\\
39.74	0\\
39.75	0\\
39.76	0\\
39.77	0\\
39.78	0\\
39.79	0\\
39.8	0\\
39.81	0\\
39.82	0\\
39.83	0\\
39.84	0\\
39.85	0\\
39.86	0\\
39.87	0\\
39.88	0\\
39.89	0\\
39.9	0\\
39.91	0\\
39.92	0\\
39.93	0\\
39.94	0\\
39.95	0\\
39.96	0\\
39.97	1.73472347597681e-18\\
39.98	0\\
39.99	0\\
40	0\\
40.01	0\\
};
\addplot [color=mycolor1,solid,forget plot]
  table[row sep=crcr]{%
40.01	0\\
40.02	0\\
40.03	0\\
40.04	0\\
40.05	0\\
40.06	0\\
40.07	0\\
40.08	0\\
40.09	0\\
40.1	0\\
40.11	0\\
40.12	0\\
40.13	1.73472347597681e-18\\
40.14	0\\
40.15	0\\
40.16	0\\
40.17	0\\
40.18	1.73472347597681e-18\\
40.19	0\\
40.2	0\\
40.21	0\\
40.22	1.73472347597681e-18\\
40.23	0\\
40.24	0\\
40.25	0\\
40.26	1.73472347597681e-18\\
40.27	0\\
40.28	0\\
40.29	0\\
40.3	1.73472347597681e-18\\
40.31	0\\
40.32	0\\
40.33	0\\
40.34	1.73472347597681e-18\\
40.35	0\\
40.36	0\\
40.37	0\\
40.38	1.73472347597681e-18\\
40.39	0\\
40.4	0\\
40.41	0\\
40.42	0\\
40.43	0\\
40.44	0\\
40.45	0\\
40.46	0\\
40.47	0\\
40.48	1.73472347597681e-18\\
40.49	0\\
40.5	0\\
40.51	0\\
40.52	0\\
40.53	0\\
40.54	0\\
40.55	0\\
40.56	0\\
40.57	0\\
40.58	0\\
40.59	0\\
40.6	0\\
40.61	0\\
40.62	0\\
40.63	0\\
40.64	0\\
40.65	0\\
40.66	0\\
40.67	0\\
40.68	0\\
40.69	0\\
40.7	0\\
40.71	0\\
40.72	0\\
40.73	0\\
40.74	0\\
40.75	0\\
40.76	0\\
40.77	0\\
40.78	0\\
40.79	0\\
40.8	0\\
40.81	0\\
40.82	0\\
40.83	0\\
40.84	0\\
40.85	0\\
40.86	1.73472347597681e-18\\
40.87	0\\
40.88	0\\
40.89	0\\
40.9	0\\
40.91	0\\
40.92	0\\
40.93	0\\
40.94	0\\
40.95	0\\
40.96	0\\
40.97	0\\
40.98	0\\
40.99	0\\
41	0\\
41.01	0\\
41.02	0\\
41.03	0\\
41.04	0\\
41.05	0\\
41.06	1.73472347597681e-18\\
41.07	0\\
41.08	0\\
41.09	0\\
41.1	0\\
41.11	0\\
41.12	0\\
41.13	0\\
41.14	0\\
41.15	0\\
41.16	0\\
41.17	0\\
41.18	0\\
41.19	0\\
41.2	0\\
41.21	0\\
41.22	0\\
41.23	0\\
41.24	0\\
41.25	0\\
41.26	0\\
41.27	1.73472347597681e-18\\
41.28	0\\
41.29	0\\
41.3	0\\
41.31	0\\
41.32	0\\
41.33	0\\
41.34	1.73472347597681e-18\\
41.35	0\\
41.36	0\\
41.37	0\\
41.38	1.73472347597681e-18\\
41.39	0\\
41.4	0\\
41.41	0\\
41.42	0\\
41.43	1.73472347597681e-18\\
41.44	0\\
41.45	0\\
41.46	0\\
41.47	0\\
41.48	0\\
41.49	0\\
41.5	0\\
41.51	0\\
41.52	0\\
41.53	0\\
41.54	0\\
41.55	0\\
41.56	0\\
41.57	0\\
41.58	0\\
41.59	0\\
41.6	0\\
41.61	0\\
41.62	0\\
41.63	0\\
41.64	0\\
41.65	0\\
41.66	0\\
41.67	0\\
41.68	0\\
41.69	0\\
41.7	0\\
41.71	0\\
41.72	0\\
41.73	0\\
41.74	1.73472347597681e-18\\
41.75	0\\
41.76	0\\
41.77	0\\
41.78	0\\
41.79	0\\
41.8	1.73472347597681e-18\\
41.81	0\\
41.82	0\\
41.83	0\\
41.84	0\\
41.85	0\\
41.86	0\\
41.87	0\\
41.88	0\\
41.89	0\\
41.9	0\\
41.91	0\\
41.92	0\\
41.93	0\\
41.94	0\\
41.95	0\\
41.96	0\\
41.97	0\\
41.98	0\\
41.99	0\\
42	0\\
42.01	0\\
42.02	0\\
42.03	0\\
42.04	1.73472347597681e-18\\
42.05	0\\
42.06	0\\
42.07	0\\
42.08	0\\
42.09	1.73472347597681e-18\\
42.1	0\\
42.11	0\\
42.12	1.73472347597681e-18\\
42.13	0\\
42.14	0\\
42.15	0\\
42.16	0\\
42.17	0\\
42.18	0\\
42.19	0\\
42.2	0\\
42.21	0\\
42.22	0\\
42.23	0\\
42.24	0\\
42.25	0\\
42.26	0\\
42.27	0\\
42.28	0\\
42.29	0\\
42.3	0\\
42.31	0\\
42.32	0\\
42.33	0\\
42.34	0\\
42.35	0\\
42.36	1.73472347597681e-18\\
42.37	0\\
42.38	0\\
42.39	0\\
42.4	0\\
42.41	0\\
42.42	0\\
42.43	0\\
42.44	0\\
42.45	0\\
42.46	0\\
42.47	0\\
42.48	0\\
42.49	0\\
42.5	0\\
42.51	0\\
42.52	0\\
42.53	0\\
42.54	0\\
42.55	0\\
42.56	0\\
42.57	0\\
42.58	0\\
42.59	0\\
42.6	0\\
42.61	0\\
42.62	0\\
42.63	0\\
42.64	0\\
42.65	0\\
42.66	0\\
42.67	0\\
42.68	0\\
42.69	0\\
42.7	0\\
42.71	0\\
42.72	0\\
42.73	0\\
42.74	0\\
42.75	0\\
42.76	0\\
42.77	0\\
42.78	0\\
42.79	0\\
42.8	0\\
42.81	0\\
42.82	0\\
42.83	0\\
42.84	0\\
42.85	0\\
42.86	0\\
42.87	0\\
42.88	0\\
42.89	0\\
42.9	0\\
42.91	0\\
42.92	1.73472347597681e-18\\
42.93	0\\
42.94	0\\
42.95	0\\
42.96	0\\
42.97	1.73472347597681e-18\\
42.98	0\\
42.99	0\\
43	0\\
43.01	0\\
43.02	0\\
43.03	0\\
43.04	0\\
43.05	0\\
43.06	0\\
43.07	0\\
43.08	0\\
43.09	0\\
43.1	0\\
43.11	0\\
43.12	0\\
43.13	0\\
43.14	0\\
43.15	0\\
43.16	0\\
43.17	0\\
43.18	0\\
43.19	0\\
43.2	0\\
43.21	0\\
43.22	0\\
43.23	0\\
43.24	1.73472347597681e-18\\
43.25	0\\
43.26	0\\
43.27	0\\
43.28	0\\
43.29	0\\
43.3	0\\
43.31	0\\
43.32	0\\
43.33	0\\
43.34	0\\
43.35	0\\
43.36	0\\
43.37	0\\
43.38	0\\
43.39	0\\
43.4	0\\
43.41	0\\
43.42	0\\
43.43	1.73472347597681e-18\\
43.44	0\\
43.45	0\\
43.46	0\\
43.47	0\\
43.48	0\\
43.49	0\\
43.5	0\\
43.51	0\\
43.52	0\\
43.53	0\\
43.54	0\\
43.55	0\\
43.56	0\\
43.57	0\\
43.58	0\\
43.59	0\\
43.6	0\\
43.61	0\\
43.62	0\\
43.63	0\\
43.64	0\\
43.65	0\\
43.66	0\\
43.67	0\\
43.68	0\\
43.69	0\\
43.7	0\\
43.71	0\\
43.72	0\\
43.73	0\\
43.74	0\\
43.75	0\\
43.76	0\\
43.77	0\\
43.78	0\\
43.79	0\\
43.8	0\\
43.81	1.73472347597681e-18\\
43.82	0\\
43.83	0\\
43.84	1.73472347597681e-18\\
43.85	0\\
43.86	0\\
43.87	0\\
43.88	0\\
43.89	0\\
43.9	0\\
43.91	0\\
43.92	0\\
43.93	0\\
43.94	1.73472347597681e-18\\
43.95	0\\
43.96	0\\
43.97	0\\
43.98	1.73472347597681e-18\\
43.99	0\\
44	0\\
44.01	0\\
44.02	1.73472347597681e-18\\
44.03	0\\
44.04	0\\
44.05	0\\
44.06	1.73472347597681e-18\\
44.07	0\\
44.08	0\\
44.09	0\\
44.1	0\\
44.11	0\\
44.12	1.73472347597681e-18\\
44.13	0\\
44.14	0\\
44.15	0\\
44.16	0\\
44.17	0\\
44.18	0\\
44.19	0\\
44.2	0\\
44.21	0\\
44.22	0\\
44.23	0\\
44.24	0\\
44.25	1.73472347597681e-18\\
44.26	0\\
44.27	0\\
44.28	0\\
44.29	0\\
44.3	0\\
44.31	0\\
44.32	0\\
44.33	0\\
44.34	0\\
44.35	0\\
44.36	0\\
44.37	0\\
44.38	0\\
44.39	0\\
44.4	0\\
44.41	0\\
44.42	1.73472347597681e-18\\
44.43	0\\
44.44	0\\
44.45	0\\
44.46	0\\
44.47	0\\
44.48	0\\
44.49	0\\
44.5	0\\
44.51	1.73472347597681e-18\\
44.52	0\\
44.53	0\\
44.54	0\\
44.55	0\\
44.56	0\\
44.57	0\\
44.58	0\\
44.59	0\\
44.6	0\\
44.61	0\\
44.62	0\\
44.63	0\\
44.64	0\\
44.65	0\\
44.66	0\\
44.67	0\\
44.68	0\\
44.69	0\\
44.7	0\\
44.71	0\\
44.72	0\\
44.73	0\\
44.74	0\\
44.75	0\\
44.76	0\\
44.77	0\\
44.78	0\\
44.79	0\\
44.8	0\\
44.81	0\\
44.82	0\\
44.83	0\\
44.84	0\\
44.85	0\\
44.86	0\\
44.87	0\\
44.88	0\\
44.89	0\\
44.9	0\\
44.91	0\\
44.92	0\\
44.93	0\\
44.94	0\\
44.95	0\\
44.96	0\\
44.97	0\\
44.98	0\\
44.99	0\\
45	0\\
45.01	0\\
45.02	0\\
45.03	0\\
45.04	0\\
45.05	0\\
45.06	0\\
45.07	0\\
45.08	0\\
45.09	0\\
45.1	0\\
45.11	0\\
45.12	0\\
45.13	0\\
45.14	0\\
45.15	0\\
45.16	0\\
45.17	0\\
45.18	0\\
45.19	0\\
45.2	0\\
45.21	0\\
45.22	0\\
45.23	0\\
45.24	0\\
45.25	0\\
45.26	0\\
45.27	0\\
45.28	0\\
45.29	0\\
45.3	0\\
45.31	0\\
45.32	1.73472347597681e-18\\
45.33	0\\
45.34	0\\
45.35	0\\
45.36	0\\
45.37	0\\
45.38	0\\
45.39	0\\
45.4	0\\
45.41	0\\
45.42	0\\
45.43	1.73472347597681e-18\\
45.44	0\\
45.45	0\\
45.46	0\\
45.47	0\\
45.48	0\\
45.49	0\\
45.5	0\\
45.51	0\\
45.52	0\\
45.53	0\\
45.54	1.73472347597681e-18\\
45.55	0\\
45.56	0\\
45.57	0\\
45.58	0\\
45.59	0\\
45.6	1.73472347597681e-18\\
45.61	0\\
45.62	0\\
45.63	0\\
45.64	0\\
45.65	0\\
45.66	0\\
45.67	0\\
45.68	0\\
45.69	0\\
45.7	0\\
45.71	0\\
45.72	0\\
45.73	0\\
45.74	0\\
45.75	0\\
45.76	0\\
45.77	0\\
45.78	0\\
45.79	0\\
45.8	0\\
45.81	0\\
45.82	0\\
45.83	0\\
45.84	0\\
45.85	0\\
45.86	0\\
45.87	0\\
45.88	0\\
45.89	0\\
45.9	0\\
45.91	0\\
45.92	0\\
45.93	0\\
45.94	0\\
45.95	0\\
45.96	0\\
45.97	0\\
45.98	1.73472347597681e-18\\
45.99	0\\
46	0\\
46.01	0\\
46.02	0\\
46.03	0\\
46.04	0\\
46.05	0\\
46.06	0\\
46.07	0\\
46.08	0\\
46.09	0\\
46.1	0\\
46.11	1.73472347597681e-18\\
46.12	1.73472347597681e-18\\
46.13	0\\
46.14	0\\
46.15	0\\
46.16	0\\
46.17	0\\
46.18	0\\
46.19	0\\
46.2	0\\
46.21	0\\
46.22	0\\
46.23	0\\
46.24	0\\
46.25	0\\
46.26	0\\
46.27	0\\
46.28	0\\
46.29	0\\
46.3	0\\
46.31	0\\
46.32	0\\
46.33	0\\
46.34	0\\
46.35	0\\
46.36	0\\
46.37	0\\
46.38	0\\
46.39	0\\
46.4	1.73472347597681e-18\\
46.41	0\\
46.42	0\\
46.43	0\\
46.44	0\\
46.45	0\\
46.46	0\\
46.47	0\\
46.48	0\\
46.49	0\\
46.5	0\\
46.51	1.73472347597681e-18\\
46.52	0\\
46.53	0\\
46.54	0\\
46.55	0\\
46.56	0\\
46.57	0\\
46.58	0\\
46.59	0\\
46.6	0\\
46.61	0\\
46.62	0\\
46.63	0\\
46.64	0\\
46.65	0\\
46.66	0\\
46.67	0\\
46.68	0\\
46.69	0\\
46.7	0\\
46.71	0\\
46.72	0\\
46.73	0\\
46.74	0\\
46.75	0\\
46.76	0\\
46.77	0\\
46.78	0\\
46.79	1.73472347597681e-18\\
46.8	0\\
46.81	0\\
46.82	0\\
46.83	0\\
46.84	0\\
46.85	0\\
46.86	0\\
46.87	1.73472347597681e-18\\
46.88	0\\
46.89	0\\
46.9	0\\
46.91	0\\
46.92	0\\
46.93	0\\
46.94	0\\
46.95	0\\
46.96	0\\
46.97	0\\
46.98	0\\
46.99	0\\
47	0\\
47.01	0\\
47.02	0\\
47.03	0\\
47.04	0\\
47.05	0\\
47.06	0\\
47.07	0\\
47.08	1.73472347597681e-18\\
47.09	0\\
47.1	0\\
47.11	0\\
47.12	0\\
47.13	0\\
47.14	0\\
47.15	0\\
47.16	0\\
47.17	0\\
47.18	0\\
47.19	0\\
47.2	0\\
47.21	0\\
47.22	1.73472347597681e-18\\
47.23	0\\
47.24	0\\
47.25	0\\
47.26	0\\
47.27	0\\
47.28	0\\
47.29	0\\
47.3	0\\
47.31	1.73472347597681e-18\\
47.32	0\\
47.33	0\\
47.34	0\\
47.35	0\\
47.36	0\\
47.37	0\\
47.38	1.73472347597681e-18\\
47.39	0\\
47.4	0\\
47.41	0\\
47.42	0\\
47.43	0\\
47.44	0\\
47.45	0\\
47.46	0\\
47.47	0\\
47.48	0\\
47.49	0\\
47.5	0\\
47.51	0\\
47.52	0\\
47.53	0\\
47.54	0\\
47.55	0\\
47.56	0\\
47.57	0\\
47.58	0\\
47.59	0\\
47.6	0\\
47.61	0\\
47.62	0\\
47.63	0\\
47.64	0\\
47.65	0\\
47.66	0\\
47.67	0\\
47.68	0\\
47.69	0\\
47.7	0\\
47.71	0\\
47.72	0\\
47.73	0\\
47.74	0\\
47.75	0\\
47.76	0\\
47.77	0\\
47.78	0\\
47.79	0\\
47.8	0\\
47.81	1.73472347597681e-18\\
47.82	0\\
47.83	1.73472347597681e-18\\
47.84	0\\
47.85	0\\
47.86	0\\
47.87	0\\
47.88	0\\
47.89	0\\
47.9	0\\
47.91	0\\
47.92	0\\
47.93	0\\
47.94	1.73472347597681e-18\\
47.95	0\\
47.96	0\\
47.97	0\\
47.98	0\\
47.99	0\\
48	0\\
48.01	1.73472347597681e-18\\
48.02	0\\
48.03	0\\
48.04	0\\
48.05	0\\
48.06	0\\
48.07	0\\
48.08	0\\
48.09	0\\
48.1	0\\
48.11	0\\
48.12	0\\
48.13	0\\
48.14	0\\
48.15	0\\
48.16	1.73472347597681e-18\\
48.17	0\\
48.18	0\\
48.19	0\\
48.2	1.73472347597681e-18\\
48.21	0\\
48.22	0\\
48.23	0\\
48.24	0\\
48.25	0\\
48.26	0\\
48.27	0\\
48.28	0\\
48.29	0\\
48.3	0\\
48.31	1.73472347597681e-18\\
48.32	0\\
48.33	0\\
48.34	0\\
48.35	0\\
48.36	0\\
48.37	0\\
48.38	1.73472347597681e-18\\
48.39	0\\
48.4	0\\
48.41	0\\
48.42	0\\
48.43	0\\
48.44	0\\
48.45	0\\
48.46	0\\
48.47	0\\
48.48	0\\
48.49	0\\
48.5	0\\
48.51	0\\
48.52	0\\
48.53	1.73472347597681e-18\\
48.54	1.73472347597681e-18\\
48.55	0\\
48.56	0\\
48.57	0\\
48.58	0\\
48.59	0\\
48.6	0\\
48.61	0\\
48.62	0\\
48.63	0\\
48.64	0\\
48.65	1.73472347597681e-18\\
48.66	0\\
48.67	0\\
48.68	0\\
48.69	0\\
48.7	0\\
48.71	0\\
48.72	0\\
48.73	0\\
48.74	0\\
48.75	0\\
48.76	0\\
48.77	0\\
48.78	0\\
48.79	0\\
48.8	0\\
48.81	0\\
48.82	0\\
48.83	0\\
48.84	0\\
48.85	0\\
48.86	0\\
48.87	0\\
48.88	0\\
48.89	0\\
48.9	1.73472347597681e-18\\
48.91	0\\
48.92	0\\
48.93	0\\
48.94	0\\
48.95	0\\
48.96	0\\
48.97	0\\
48.98	0\\
48.99	0\\
49	0\\
49.01	0\\
49.02	0\\
49.03	0\\
49.04	0\\
49.05	0\\
49.06	0\\
49.07	0\\
49.08	1.73472347597681e-18\\
49.09	0\\
49.1	0\\
49.11	0\\
49.12	0\\
49.13	0\\
49.14	0\\
49.15	0\\
49.16	0\\
49.17	0\\
49.18	0\\
49.19	0\\
49.2	0\\
49.21	0\\
49.22	0\\
49.23	0\\
49.24	0\\
49.25	0\\
49.26	0\\
49.27	0\\
49.28	0\\
49.29	0\\
49.3	0\\
49.31	0\\
49.32	0\\
49.33	0\\
49.34	1.73472347597681e-18\\
49.35	0\\
49.36	0\\
49.37	0\\
49.38	0\\
49.39	0\\
49.4	1.73472347597681e-18\\
49.41	0\\
49.42	0\\
49.43	0\\
49.44	0\\
49.45	0\\
49.46	0\\
49.47	0\\
49.48	0\\
49.49	0\\
49.5	0\\
49.51	0\\
49.52	0\\
49.53	0\\
49.54	0\\
49.55	1.73472347597681e-18\\
49.56	0\\
49.57	0\\
49.58	0\\
49.59	1.73472347597681e-18\\
49.6	0\\
49.61	0\\
49.62	0\\
49.63	1.73472347597681e-18\\
49.64	0\\
49.65	0\\
49.66	0\\
49.67	0\\
49.68	0\\
49.69	0\\
49.7	0\\
49.71	0\\
49.72	0\\
49.73	0\\
49.74	0\\
49.75	0\\
49.76	0\\
49.77	0\\
49.78	0\\
49.79	0\\
49.8	0\\
49.81	0\\
49.82	0\\
49.83	0\\
49.84	0\\
49.85	0\\
49.86	0\\
49.87	0\\
49.88	0\\
49.89	0\\
49.9	0\\
49.91	0\\
49.92	0\\
49.93	0\\
49.94	0\\
49.95	0\\
49.96	0\\
49.97	1.73472347597681e-18\\
49.98	0\\
49.99	0\\
50	0\\
50.01	0\\
50.02	0\\
50.03	0\\
50.04	0\\
50.05	0\\
50.06	1.73472347597681e-18\\
50.07	0\\
50.08	0\\
50.09	0\\
50.1	0\\
50.11	0\\
50.12	0\\
50.13	0\\
50.14	0\\
50.15	0\\
50.16	0\\
50.17	0\\
50.18	0\\
50.19	0\\
50.2	0\\
50.21	0\\
50.22	0\\
50.23	0\\
50.24	0\\
50.25	0\\
50.26	0\\
50.27	0\\
50.28	0\\
50.29	0\\
50.3	0\\
50.31	0\\
50.32	0\\
50.33	0\\
50.34	0\\
50.35	0\\
50.36	0\\
50.37	0\\
50.38	0\\
50.39	0\\
50.4	1.73472347597681e-18\\
50.41	0\\
50.42	0\\
50.43	0\\
50.44	0\\
50.45	0\\
50.46	0\\
50.47	0\\
50.48	0\\
50.49	0\\
50.5	0\\
50.51	0\\
50.52	0\\
50.53	0\\
50.54	0\\
50.55	0\\
50.56	0\\
50.57	0\\
50.58	1.73472347597681e-18\\
50.59	0\\
50.6	0\\
50.61	0\\
50.62	0\\
50.63	0\\
50.64	0\\
50.65	0\\
50.66	0\\
50.67	0\\
50.68	0\\
50.69	0\\
50.7	0\\
50.71	0\\
50.72	0\\
50.73	0\\
50.74	0\\
50.75	1.73472347597681e-18\\
50.76	0\\
50.77	0\\
50.78	0\\
50.79	0\\
50.8	0\\
50.81	0\\
50.82	0\\
50.83	0\\
50.84	0\\
50.85	0\\
50.86	0\\
50.87	0\\
50.88	0\\
50.89	0\\
50.9	0\\
50.91	0\\
50.92	0\\
50.93	0\\
50.94	0\\
50.95	0\\
50.96	0\\
50.97	0\\
50.98	0\\
50.99	0\\
51	0\\
51.01	0\\
51.02	0\\
51.03	0\\
51.04	0\\
51.05	0\\
51.06	0\\
51.07	0\\
51.08	1.73472347597681e-18\\
51.09	0\\
51.1	0\\
51.11	0\\
51.12	0\\
51.13	0\\
51.14	0\\
51.15	0\\
51.16	0\\
51.17	0\\
51.18	0\\
51.19	0\\
51.2	0\\
51.21	0\\
51.22	0\\
51.23	0\\
51.24	0\\
51.25	0\\
51.26	0\\
51.27	0\\
51.28	0\\
51.29	0\\
51.3	0\\
51.31	0\\
51.32	0\\
51.33	0\\
51.34	0\\
51.35	0\\
51.36	0\\
51.37	0\\
51.38	0\\
51.39	0\\
51.4	0\\
51.41	0\\
51.42	0\\
51.43	0\\
51.44	0\\
51.45	0\\
51.46	0\\
51.47	0\\
51.48	0\\
51.49	0\\
51.5	0\\
51.51	0\\
51.52	0\\
51.53	0\\
51.54	1.73472347597681e-18\\
51.55	0\\
51.56	0\\
51.57	0\\
51.58	0\\
51.59	0\\
51.6	0\\
51.61	0\\
51.62	0\\
51.63	0\\
51.64	0\\
51.65	0\\
51.66	0\\
51.67	0\\
51.68	0\\
51.69	0\\
51.7	0\\
51.71	0\\
51.72	0\\
51.73	0\\
51.74	0\\
51.75	0\\
51.76	0\\
51.77	0\\
51.78	0\\
51.79	0\\
51.8	0\\
51.81	0\\
51.82	0\\
51.83	0\\
51.84	1.73472347597681e-18\\
51.85	0\\
51.86	0\\
51.87	0\\
51.88	0\\
51.89	0\\
51.9	0\\
51.91	1.73472347597681e-18\\
51.92	0\\
51.93	0\\
51.94	0\\
51.95	0\\
51.96	0\\
51.97	0\\
51.98	0\\
51.99	0\\
52	0\\
52.01	0\\
52.02	0\\
52.03	0\\
52.04	1.73472347597681e-18\\
52.05	0\\
52.06	0\\
52.07	0\\
52.08	0\\
52.09	0\\
52.1	0\\
52.11	0\\
52.12	0\\
52.13	0\\
52.14	0\\
52.15	1.73472347597681e-18\\
52.16	0\\
52.17	0\\
52.18	0\\
52.19	0\\
52.2	0\\
52.21	0\\
52.22	0\\
52.23	0\\
52.24	0\\
52.25	0\\
52.26	0\\
52.27	0\\
52.28	0\\
52.29	0\\
52.3	0\\
52.31	0\\
52.32	0\\
52.33	0\\
52.34	0\\
52.35	0\\
52.36	0\\
52.37	1.73472347597681e-18\\
52.38	0\\
52.39	0\\
52.4	0\\
52.41	0\\
52.42	0\\
52.43	0\\
52.44	0\\
52.45	0\\
52.46	0\\
52.47	0\\
52.48	0\\
52.49	0\\
52.5	1.73472347597681e-18\\
52.51	0\\
52.52	0\\
52.53	0\\
52.54	0\\
52.55	0\\
52.56	0\\
52.57	0\\
52.58	0\\
52.59	1.73472347597681e-18\\
52.6	0\\
52.61	0\\
52.62	0\\
52.63	0\\
52.64	0\\
52.65	1.73472347597681e-18\\
52.66	0\\
52.67	0\\
52.68	0\\
52.69	0\\
52.7	0\\
52.71	0\\
52.72	0\\
52.73	0\\
52.74	0\\
52.75	0\\
52.76	0\\
52.77	0\\
52.78	0\\
52.79	0\\
52.8	0\\
52.81	0\\
52.82	0\\
52.83	0\\
52.84	0\\
52.85	0\\
52.86	0\\
52.87	0\\
52.88	0\\
52.89	0\\
52.9	0\\
52.91	0\\
52.92	1.73472347597681e-18\\
52.93	0\\
52.94	0\\
52.95	0\\
52.96	0\\
52.97	0\\
52.98	0\\
52.99	0\\
53	0\\
53.01	0\\
53.02	0\\
53.03	0\\
53.04	0\\
53.05	0\\
53.06	0\\
53.07	0\\
53.08	0\\
53.09	0\\
53.1	0\\
53.11	0\\
53.12	0\\
53.13	0\\
53.14	0\\
53.15	0\\
53.16	0\\
53.17	0\\
53.18	0\\
53.19	0\\
53.2	0\\
53.21	0\\
53.22	0\\
53.23	0\\
53.24	0\\
53.25	0\\
53.26	0\\
53.27	0\\
53.28	0\\
53.29	0\\
53.3	0\\
53.31	0\\
53.32	0\\
53.33	0\\
53.34	0\\
53.35	0\\
53.36	0\\
53.37	0\\
53.38	0\\
53.39	0\\
53.4	0\\
53.41	0\\
53.42	0\\
53.43	0\\
53.44	0\\
53.45	0\\
53.46	0\\
53.47	1.73472347597681e-18\\
53.48	0\\
53.49	0\\
53.5	0\\
53.51	0\\
53.52	0\\
53.53	0\\
53.54	0\\
53.55	0\\
53.56	0\\
53.57	0\\
53.58	0\\
53.59	0\\
53.6	0\\
53.61	1.73472347597681e-18\\
53.62	0\\
53.63	0\\
53.64	0\\
53.65	0\\
53.66	0\\
53.67	0\\
53.68	0\\
53.69	0\\
53.7	0\\
53.71	0\\
53.72	0\\
53.73	0\\
53.74	0\\
53.75	0\\
53.76	0\\
53.77	0\\
53.78	0\\
53.79	0\\
53.8	0\\
53.81	1.73472347597681e-18\\
53.82	0\\
53.83	0\\
53.84	1.73472347597681e-18\\
53.85	0\\
53.86	0\\
53.87	0\\
53.88	0\\
53.89	0\\
53.9	0\\
53.91	0\\
53.92	0\\
53.93	0\\
53.94	0\\
53.95	0\\
53.96	0\\
53.97	1.73472347597681e-18\\
53.98	0\\
53.99	0\\
54	0\\
54.01	0\\
54.02	0\\
54.03	0\\
54.04	0\\
54.05	0\\
54.06	0\\
54.07	0\\
54.08	0\\
54.09	0\\
54.1	0\\
54.11	0\\
54.12	0\\
54.13	0\\
54.14	0\\
54.15	0\\
54.16	0\\
54.17	0\\
54.18	0\\
54.19	0\\
54.2	0\\
54.21	0\\
54.22	0\\
54.23	0\\
54.24	0\\
54.25	0\\
54.26	0\\
54.27	0\\
54.28	0\\
54.29	0\\
54.3	0\\
54.31	0\\
54.32	0\\
54.33	0\\
54.34	0\\
54.35	0\\
54.36	0\\
54.37	0\\
54.38	0\\
54.39	0\\
54.4	0\\
54.41	0\\
54.42	0\\
54.43	0\\
54.44	0\\
54.45	0\\
54.46	0\\
54.47	0\\
54.48	0\\
54.49	0\\
54.5	0\\
54.51	0\\
54.52	0\\
54.53	0\\
54.54	0\\
54.55	0\\
54.56	0\\
54.57	0\\
54.58	0\\
54.59	0\\
54.6	0\\
54.61	0\\
54.62	0\\
54.63	0\\
54.64	0\\
54.65	0\\
54.66	0\\
54.67	0\\
54.68	0\\
54.69	0\\
54.7	0\\
54.71	0\\
54.72	1.73472347597681e-18\\
54.73	0\\
54.74	0\\
54.75	0\\
54.76	0\\
54.77	0\\
54.78	0\\
54.79	0\\
54.8	0\\
54.81	0\\
54.82	0\\
54.83	0\\
54.84	0\\
54.85	0\\
54.86	0\\
54.87	0\\
54.88	0\\
54.89	0\\
54.9	0\\
54.91	0\\
54.92	0\\
54.93	0\\
54.94	0\\
54.95	0\\
54.96	0\\
54.97	0\\
54.98	0\\
54.99	0\\
55	0\\
55.01	0\\
55.02	0\\
55.03	0\\
55.04	0\\
55.05	0\\
55.06	0\\
55.07	0\\
55.08	0\\
55.09	0\\
55.1	0\\
55.11	0\\
55.12	0\\
55.13	0\\
55.14	1.73472347597681e-18\\
55.15	0\\
55.16	0\\
55.17	0\\
55.18	0\\
55.19	0\\
55.2	0\\
55.21	1.73472347597681e-18\\
55.22	0\\
55.23	0\\
55.24	0\\
55.25	0\\
55.26	0\\
55.27	0\\
55.28	0\\
55.29	0\\
55.3	0\\
55.31	0\\
55.32	0\\
55.33	0\\
55.34	0\\
55.35	0\\
55.36	0\\
55.37	0\\
55.38	0\\
55.39	0\\
55.4	0\\
55.41	0\\
55.42	0\\
55.43	1.73472347597681e-18\\
55.44	0\\
55.45	0\\
55.46	0\\
55.47	0\\
55.48	0\\
55.49	0\\
55.5	0\\
55.51	0\\
55.52	0\\
55.53	0\\
55.54	0\\
55.55	0\\
55.56	1.73472347597681e-18\\
55.57	0\\
55.58	0\\
55.59	0\\
55.6	0\\
55.61	0\\
55.62	0\\
55.63	0\\
55.64	0\\
55.65	0\\
55.66	1.73472347597681e-18\\
55.67	0\\
55.68	0\\
55.69	0\\
55.7	1.73472347597681e-18\\
55.71	0\\
55.72	0\\
55.73	0\\
55.74	0\\
55.75	0\\
55.76	1.73472347597681e-18\\
55.77	0\\
55.78	0\\
55.79	0\\
55.8	0\\
55.81	0\\
55.82	1.73472347597681e-18\\
55.83	0\\
55.84	0\\
55.85	0\\
55.86	0\\
55.87	0\\
55.88	0\\
55.89	0\\
55.9	0\\
55.91	0\\
55.92	0\\
55.93	0\\
55.94	0\\
55.95	0\\
55.96	1.73472347597681e-18\\
55.97	0\\
55.98	0\\
55.99	0\\
56	0\\
56.01	0\\
56.02	0\\
56.03	0\\
56.04	0\\
56.05	0\\
56.06	0\\
56.07	0\\
56.08	0\\
56.09	0\\
56.1	0\\
56.11	0\\
56.12	0\\
56.13	0\\
56.14	0\\
56.15	0\\
56.16	0\\
56.17	0\\
56.18	0\\
56.19	0\\
56.2	0\\
56.21	0\\
56.22	0\\
56.23	0\\
56.24	0\\
56.25	0\\
56.26	0\\
56.27	0\\
56.28	0\\
56.29	0\\
56.3	0\\
56.31	0\\
56.32	0\\
56.33	0\\
56.34	0\\
56.35	0\\
56.36	0\\
56.37	0\\
56.38	0\\
56.39	0\\
56.4	0\\
56.41	0\\
56.42	0\\
56.43	0\\
56.44	0\\
56.45	0\\
56.46	0\\
56.47	0\\
56.48	0\\
56.49	0\\
56.5	1.73472347597681e-18\\
56.51	0\\
56.52	0\\
56.53	0\\
56.54	0\\
56.55	0\\
56.56	0\\
56.57	0\\
56.58	0\\
56.59	0\\
56.6	0\\
56.61	0\\
56.62	0\\
56.63	0\\
56.64	0\\
56.65	0\\
56.66	0\\
56.67	0\\
56.68	0\\
56.69	0\\
56.7	1.73472347597681e-18\\
56.71	0\\
56.72	0\\
56.73	0\\
56.74	0\\
56.75	0\\
56.76	0\\
56.77	0\\
56.78	0\\
56.79	0\\
56.8	0\\
56.81	0\\
56.82	0\\
56.83	0\\
56.84	0\\
56.85	0\\
56.86	0\\
56.87	1.73472347597681e-18\\
56.88	0\\
56.89	0\\
56.9	1.73472347597681e-18\\
56.91	0\\
56.92	0\\
56.93	0\\
56.94	0\\
56.95	0\\
56.96	0\\
56.97	0\\
56.98	0\\
56.99	0\\
57	0\\
57.01	0\\
57.02	0\\
57.03	0\\
57.04	0\\
57.05	0\\
57.06	0\\
57.07	0\\
57.08	0\\
57.09	0\\
57.1	0\\
57.11	0\\
57.12	0\\
57.13	1.73472347597681e-18\\
57.14	0\\
57.15	0\\
57.16	0\\
57.17	0\\
57.18	0\\
57.19	0\\
57.2	0\\
57.21	0\\
57.22	0\\
57.23	0\\
57.24	0\\
57.25	0\\
57.26	0\\
57.27	0\\
57.28	0\\
57.29	0\\
57.3	0\\
57.31	0\\
57.32	0\\
57.33	0\\
57.34	0\\
57.35	1.73472347597681e-18\\
57.36	1.73472347597681e-18\\
57.37	0\\
57.38	0\\
57.39	0\\
57.4	0\\
57.41	1.73472347597681e-18\\
57.42	0\\
57.43	0\\
57.44	0\\
57.45	1.73472347597681e-18\\
57.46	0\\
57.47	0\\
57.48	0\\
57.49	0\\
57.5	0\\
57.51	0\\
57.52	0\\
57.53	0\\
57.54	0\\
57.55	0\\
57.56	0\\
57.57	0\\
57.58	0\\
57.59	0\\
57.6	0\\
57.61	0\\
57.62	0\\
57.63	0\\
57.64	0\\
57.65	1.73472347597681e-18\\
57.66	0\\
57.67	0\\
57.68	0\\
57.69	0\\
57.7	0\\
57.71	0\\
57.72	0\\
57.73	0\\
57.74	0\\
57.75	0\\
57.76	0\\
57.77	0\\
57.78	0\\
57.79	0\\
57.8	0\\
57.81	0\\
57.82	0\\
57.83	0\\
57.84	0\\
57.85	0\\
57.86	0\\
57.87	0\\
57.88	1.73472347597681e-18\\
57.89	1.73472347597681e-18\\
57.9	0\\
57.91	0\\
57.92	0\\
57.93	0\\
57.94	0\\
57.95	0\\
57.96	0\\
57.97	0\\
57.98	0\\
57.99	0\\
58	0\\
58.01	0\\
58.02	0\\
58.03	0\\
58.04	0\\
58.05	0\\
58.06	0\\
58.07	0\\
58.08	0\\
58.09	0\\
58.1	0\\
58.11	0\\
58.12	0\\
58.13	0\\
58.14	0\\
58.15	0\\
58.16	0\\
58.17	0\\
58.18	0\\
58.19	1.73472347597681e-18\\
58.2	0\\
58.21	0\\
58.22	0\\
58.23	0\\
58.24	0\\
58.25	0\\
58.26	0\\
58.27	0\\
58.28	0\\
58.29	0\\
58.3	0\\
58.31	0\\
58.32	0\\
58.33	0\\
58.34	0\\
58.35	0\\
58.36	0\\
58.37	0\\
58.38	0\\
58.39	0\\
58.4	0\\
58.41	0\\
58.42	0\\
58.43	0\\
58.44	0\\
58.45	0\\
58.46	0\\
58.47	0\\
58.48	0\\
58.49	0\\
58.5	0\\
58.51	0\\
58.52	0\\
58.53	0\\
58.54	0\\
58.55	0\\
58.56	0\\
58.57	0\\
58.58	0\\
58.59	0\\
58.6	0\\
58.61	0\\
58.62	0\\
58.63	0\\
58.64	0\\
58.65	0\\
58.66	0\\
58.67	0\\
58.68	0\\
58.69	0\\
58.7	0\\
58.71	0\\
58.72	0\\
58.73	0\\
58.74	0\\
58.75	0\\
58.76	0\\
58.77	0\\
58.78	1.73472347597681e-18\\
58.79	0\\
58.8	0\\
58.81	0\\
58.82	0\\
58.83	0\\
58.84	0\\
58.85	0\\
58.86	1.73472347597681e-18\\
58.87	0\\
58.88	0\\
58.89	0\\
58.9	0\\
58.91	0\\
58.92	0\\
58.93	0\\
58.94	0\\
58.95	0\\
58.96	0\\
58.97	0\\
58.98	0\\
58.99	0\\
59	0\\
59.01	0\\
59.02	0\\
59.03	0\\
59.04	0\\
59.05	0\\
59.06	0\\
59.07	0\\
59.08	0\\
59.09	0\\
59.1	0\\
59.11	0\\
59.12	0\\
59.13	0\\
59.14	0\\
59.15	0\\
59.16	0\\
59.17	0\\
59.18	1.73472347597681e-18\\
59.19	0\\
59.2	0\\
59.21	0\\
59.22	0\\
59.23	0\\
59.24	0\\
59.25	0\\
59.26	0\\
59.27	0\\
59.28	0\\
59.29	0\\
59.3	0\\
59.31	0\\
59.32	0\\
59.33	0\\
59.34	0\\
59.35	0\\
59.36	0\\
59.37	0\\
59.38	0\\
59.39	0\\
59.4	0\\
59.41	0\\
59.42	0\\
59.43	0\\
59.44	0\\
59.45	0\\
59.46	0\\
59.47	0\\
59.48	0\\
59.49	0\\
59.5	0\\
59.51	0\\
59.52	0\\
59.53	0\\
59.54	0\\
59.55	0\\
59.56	0\\
59.57	0\\
59.58	0\\
59.59	0\\
59.6	0\\
59.61	0\\
59.62	0\\
59.63	0\\
59.64	0\\
59.65	0\\
59.66	0\\
59.67	0\\
59.68	1.73472347597681e-18\\
59.69	0\\
59.7	0\\
59.71	0\\
59.72	0\\
59.73	0\\
59.74	0\\
59.75	0\\
59.76	0\\
59.77	0\\
59.78	0\\
59.79	0\\
59.8	0\\
59.81	0\\
59.82	0\\
59.83	0\\
59.84	0\\
59.85	0\\
59.86	0\\
59.87	0\\
59.88	0\\
59.89	0\\
59.9	0\\
59.91	0\\
59.92	0\\
59.93	0\\
59.94	1.73472347597681e-18\\
59.95	0\\
59.96	0\\
59.97	0\\
59.98	0\\
59.99	0\\
60	0\\
60.01	0\\
60.02	0\\
60.03	0\\
60.04	0\\
60.05	1.73472347597681e-18\\
60.06	0\\
60.07	0\\
60.08	1.73472347597681e-18\\
60.09	0\\
60.1	0\\
60.11	0\\
60.12	0\\
60.13	0\\
60.14	1.73472347597681e-18\\
60.15	0\\
60.16	0\\
60.17	0\\
60.18	0\\
60.19	0\\
60.2	1.73472347597681e-18\\
60.21	0\\
60.22	0\\
60.23	0\\
60.24	0\\
60.25	0\\
60.26	0\\
60.27	0\\
60.28	0\\
60.29	0\\
60.3	0\\
60.31	0\\
60.32	0\\
60.33	0\\
60.34	0\\
60.35	0\\
60.36	0\\
60.37	0\\
60.38	0\\
60.39	0\\
60.4	0\\
60.41	0\\
60.42	0\\
60.43	0\\
60.44	0\\
60.45	0\\
60.46	0\\
60.47	0\\
60.48	0\\
60.49	0\\
60.5	0\\
60.51	0\\
60.52	0\\
60.53	0\\
60.54	0\\
60.55	0\\
60.56	0\\
60.57	0\\
60.58	0\\
60.59	0\\
60.6	0\\
60.61	0\\
60.62	0\\
60.63	1.73472347597681e-18\\
60.64	0\\
60.65	0\\
60.66	0\\
60.67	0\\
60.68	0\\
60.69	0\\
60.7	0\\
60.71	0\\
60.72	0\\
60.73	0\\
60.74	0\\
60.75	0\\
60.76	0\\
60.77	0\\
60.78	0\\
60.79	0\\
60.8	0\\
60.81	0\\
60.82	0\\
60.83	0\\
60.84	0\\
60.85	0\\
60.86	0\\
60.87	0\\
60.88	0\\
60.89	0\\
60.9	0\\
60.91	0\\
60.92	1.73472347597681e-18\\
60.93	0\\
60.94	0\\
60.95	0\\
60.96	1.73472347597681e-18\\
60.97	0\\
60.98	0\\
60.99	0\\
61	0\\
61.01	0\\
61.02	0\\
61.03	0\\
61.04	0\\
61.05	0\\
61.06	0\\
61.07	0\\
61.08	0\\
61.09	0\\
61.1	0\\
61.11	0\\
61.12	0\\
61.13	0\\
61.14	0\\
61.15	0\\
61.16	0\\
61.17	0\\
61.18	0\\
61.19	0\\
61.2	0\\
61.21	0\\
61.22	0\\
61.23	0\\
61.24	0\\
61.25	0\\
61.26	0\\
61.27	0\\
61.28	0\\
61.29	0\\
61.3	0\\
61.31	0\\
61.32	0\\
61.33	0\\
61.34	0\\
61.35	0\\
61.36	0\\
61.37	0\\
61.38	0\\
61.39	0\\
61.4	0\\
61.41	0\\
61.42	0\\
61.43	0\\
61.44	0\\
61.45	0\\
61.46	0\\
61.47	0\\
61.48	0\\
61.49	0\\
61.5	0\\
61.51	0\\
61.52	0\\
61.53	0\\
61.54	0\\
61.55	0\\
61.56	0\\
61.57	0\\
61.58	0\\
61.59	0\\
61.6	0\\
61.61	0\\
61.62	0\\
61.63	0\\
61.64	0\\
61.65	0\\
61.66	0\\
61.67	1.73472347597681e-18\\
61.68	1.73472347597681e-18\\
61.69	0\\
61.7	0\\
61.71	0\\
61.72	0\\
61.73	0\\
61.74	0\\
61.75	0\\
61.76	0\\
61.77	0\\
61.78	0\\
61.79	0\\
61.8	0\\
61.81	0\\
61.82	0\\
61.83	0\\
61.84	0\\
61.85	0\\
61.86	0\\
61.87	0\\
61.88	0\\
61.89	0\\
61.9	0\\
61.91	0\\
61.92	0\\
61.93	0\\
61.94	0\\
61.95	0\\
61.96	0\\
61.97	0\\
61.98	0\\
61.99	0\\
62	0\\
62.01	0\\
62.02	0\\
62.03	0\\
62.04	0\\
62.05	0\\
62.06	0\\
62.07	0\\
62.08	0\\
62.09	0\\
62.1	0\\
62.11	0\\
62.12	0\\
62.13	0\\
62.14	0\\
62.15	0\\
62.16	0\\
62.17	0\\
62.18	0\\
62.19	0\\
62.2	0\\
62.21	0\\
62.22	0\\
62.23	1.73472347597681e-18\\
62.24	0\\
62.25	0\\
62.26	0\\
62.27	0\\
62.28	0\\
62.29	0\\
62.3	1.73472347597681e-18\\
62.31	0\\
62.32	0\\
62.33	1.73472347597681e-18\\
62.34	0\\
62.35	0\\
62.36	0\\
62.37	0\\
62.38	0\\
62.39	0\\
62.4	0\\
62.41	0\\
62.42	0\\
62.43	0\\
62.44	0\\
62.45	0\\
62.46	0\\
62.47	0\\
62.48	0\\
62.49	0\\
62.5	0\\
62.51	0\\
62.52	0\\
62.53	1.73472347597681e-18\\
62.54	0\\
62.55	0\\
62.56	0\\
62.57	0\\
62.58	0\\
62.59	0\\
62.6	0\\
62.61	1.73472347597681e-18\\
62.62	0\\
62.63	0\\
62.64	0\\
62.65	0\\
62.66	0\\
62.67	0\\
62.68	0\\
62.69	0\\
62.7	0\\
62.71	0\\
62.72	0\\
62.73	0\\
62.74	0\\
62.75	0\\
62.76	0\\
62.77	0\\
62.78	0\\
62.79	1.73472347597681e-18\\
62.8	0\\
62.81	0\\
62.82	0\\
62.83	1.73472347597681e-18\\
62.84	0\\
62.85	0\\
62.86	0\\
62.87	0\\
62.88	0\\
62.89	1.73472347597681e-18\\
62.9	0\\
62.91	0\\
62.92	0\\
62.93	0\\
62.94	0\\
62.95	0\\
62.96	0\\
62.97	0\\
62.98	0\\
62.99	0\\
63	0\\
63.01	0\\
63.02	0\\
63.03	0\\
63.04	0\\
63.05	0\\
63.06	0\\
63.07	0\\
63.08	0\\
63.09	0\\
63.1	0\\
63.11	0\\
63.12	0\\
63.13	0\\
63.14	0\\
63.15	0\\
63.16	0\\
63.17	1.73472347597681e-18\\
63.18	0\\
63.19	0\\
63.2	0\\
63.21	0\\
63.22	0\\
63.23	0\\
63.24	0\\
63.25	1.73472347597681e-18\\
63.26	0\\
63.27	1.73472347597681e-18\\
63.28	0\\
63.29	1.73472347597681e-18\\
63.3	0\\
63.31	1.73472347597681e-18\\
63.32	0\\
63.33	0\\
63.34	0\\
63.35	0\\
63.36	0\\
63.37	0\\
63.38	0\\
63.39	0\\
63.4	0\\
63.41	0\\
63.42	0\\
63.43	0\\
63.44	0\\
63.45	0\\
63.46	0\\
63.47	0\\
63.48	0\\
63.49	0\\
63.5	0\\
63.51	0\\
63.52	0\\
63.53	0\\
63.54	0\\
63.55	0\\
63.56	0\\
63.57	1.73472347597681e-18\\
63.58	0\\
63.59	0\\
63.6	0\\
63.61	0\\
63.62	0\\
63.63	0\\
63.64	0\\
63.65	0\\
63.66	0\\
63.67	0\\
63.68	0\\
63.69	0\\
63.7	0\\
63.71	0\\
63.72	0\\
63.73	0\\
63.74	0\\
63.75	0\\
63.76	0\\
63.77	0\\
63.78	0\\
63.79	0\\
63.8	0\\
63.81	0\\
63.82	0\\
63.83	0\\
63.84	0\\
63.85	0\\
63.86	0\\
63.87	1.73472347597681e-18\\
63.88	0\\
63.89	1.73472347597681e-18\\
63.9	0\\
63.91	0\\
63.92	0\\
63.93	0\\
63.94	0\\
63.95	0\\
63.96	0\\
63.97	0\\
63.98	0\\
63.99	0\\
64	0\\
64.01	0\\
64.02	0\\
64.03	0\\
64.04	0\\
64.05	0\\
64.06	0\\
64.07	0\\
64.08	0\\
64.09	0\\
64.1	0\\
64.11	0\\
64.12	0\\
64.13	0\\
64.14	0\\
64.15	0\\
64.16	0\\
64.17	1.73472347597681e-18\\
64.18	0\\
64.19	0\\
64.2	0\\
64.21	0\\
64.22	0\\
64.23	0\\
64.24	0\\
64.25	0\\
64.26	0\\
64.27	0\\
64.28	0\\
64.29	0\\
64.3	0\\
64.31	0\\
64.32	0\\
64.33	0\\
64.34	0\\
64.35	0\\
64.36	0\\
64.37	0\\
64.38	0\\
64.39	0\\
64.4	0\\
64.41	0\\
64.42	0\\
64.43	0\\
64.44	0\\
64.45	0\\
64.46	0\\
64.47	0\\
64.48	0\\
64.49	0\\
64.5	0\\
64.51	0\\
64.52	0\\
64.53	0\\
64.54	0\\
64.55	0\\
64.56	0\\
64.57	0\\
64.58	0\\
64.59	0\\
64.6	0\\
64.61	0\\
64.62	0\\
64.63	0\\
64.64	1.73472347597681e-18\\
64.65	0\\
64.66	0\\
64.67	0\\
64.68	0\\
64.69	0\\
64.7	1.73472347597681e-18\\
64.71	0\\
64.72	0\\
64.73	0\\
64.74	0\\
64.75	0\\
64.76	0\\
64.77	1.73472347597681e-18\\
64.78	0\\
64.79	0\\
64.8	0\\
64.81	0\\
64.82	0\\
64.83	0\\
64.84	1.73472347597681e-18\\
64.85	0\\
64.86	0\\
64.87	0\\
64.88	0\\
64.89	0\\
64.9	0\\
64.91	0\\
64.92	0\\
64.93	1.73472347597681e-18\\
64.94	0\\
64.95	0\\
64.96	0\\
64.97	0\\
64.98	0\\
64.99	0\\
65	1.73472347597681e-18\\
65.01	0\\
65.02	0\\
65.03	0\\
65.04	0\\
65.05	0\\
65.06	0\\
65.07	0\\
65.08	0\\
65.09	0\\
65.1	0\\
65.11	0\\
65.12	0\\
65.13	0\\
65.14	0\\
65.15	0\\
65.16	1.73472347597681e-18\\
65.17	0\\
65.18	0\\
65.19	0\\
65.2	0\\
65.21	0\\
65.22	0\\
65.23	0\\
65.24	0\\
65.25	1.73472347597681e-18\\
65.26	1.73472347597681e-18\\
65.27	0\\
65.28	0\\
65.29	0\\
65.3	0\\
65.31	0\\
65.32	0\\
65.33	0\\
65.34	0\\
65.35	0\\
65.36	0\\
65.37	0\\
65.38	0\\
65.39	0\\
65.4	0\\
65.41	1.73472347597681e-18\\
65.42	0\\
65.43	0\\
65.44	0\\
65.45	0\\
65.46	1.73472347597681e-18\\
65.47	0\\
65.48	0\\
65.49	0\\
65.5	0\\
65.51	0\\
65.52	0\\
65.53	0\\
65.54	0\\
65.55	0\\
65.56	0\\
65.57	0\\
65.58	0\\
65.59	0\\
65.6	0\\
65.61	0\\
65.62	0\\
65.63	0\\
65.64	0\\
65.65	0\\
65.66	0\\
65.67	0\\
65.68	0\\
65.69	0\\
65.7	0\\
65.71	0\\
65.72	0\\
65.73	0\\
65.74	0\\
65.75	0\\
65.76	0\\
65.77	1.73472347597681e-18\\
65.78	0\\
65.79	0\\
65.8	0\\
65.81	0\\
65.82	1.73472347597681e-18\\
65.83	1.73472347597681e-18\\
65.84	0\\
65.85	0\\
65.86	1.73472347597681e-18\\
65.87	0\\
65.88	0\\
65.89	0\\
65.9	0\\
65.91	0\\
65.92	0\\
65.93	0\\
65.94	0\\
65.95	0\\
65.96	0\\
65.97	0\\
65.98	0\\
65.99	0\\
66	0\\
66.01	0\\
66.02	0\\
66.03	0\\
66.04	1.73472347597681e-18\\
66.05	0\\
66.06	0\\
66.07	0\\
66.08	0\\
66.09	0\\
66.1	0\\
66.11	0\\
66.12	0\\
66.13	1.73472347597681e-18\\
66.14	0\\
66.15	1.73472347597681e-18\\
66.16	0\\
66.17	0\\
66.18	0\\
66.19	0\\
66.2	0\\
66.21	0\\
66.22	0\\
66.23	0\\
66.24	0\\
66.25	0\\
66.26	0\\
66.27	0\\
66.28	0\\
66.29	0\\
66.3	0\\
66.31	0\\
66.32	0\\
66.33	0\\
66.34	0\\
66.35	0\\
66.36	0\\
66.37	0\\
66.38	0\\
66.39	0\\
66.4	1.73472347597681e-18\\
66.41	0\\
66.42	0\\
66.43	0\\
66.44	0\\
66.45	0\\
66.46	0\\
66.47	0\\
66.48	0\\
66.49	0\\
66.5	0\\
66.51	0\\
66.52	0\\
66.53	0\\
66.54	0\\
66.55	0\\
66.56	0\\
66.57	0\\
66.58	0\\
66.59	0\\
66.6	0\\
66.61	0\\
66.62	0\\
66.63	0\\
66.64	0\\
66.65	0\\
66.66	0\\
66.67	0\\
66.68	0\\
66.69	0\\
66.7	0\\
66.71	0\\
66.72	0\\
66.73	0\\
66.74	0\\
66.75	0\\
66.76	0\\
66.77	0\\
66.78	0\\
66.79	1.73472347597681e-18\\
66.8	0\\
66.81	0\\
66.82	0\\
66.83	1.73472347597681e-18\\
66.84	0\\
66.85	0\\
66.86	0\\
66.87	0\\
66.88	0\\
66.89	0\\
66.9	0\\
66.91	0\\
66.92	0\\
66.93	0\\
66.94	0\\
66.95	0\\
66.96	0\\
66.97	1.73472347597681e-18\\
66.98	0\\
66.99	0\\
67	0\\
67.01	0\\
67.02	0\\
67.03	0\\
67.04	0\\
67.05	0\\
67.06	0\\
67.07	0\\
67.08	0\\
67.09	0\\
67.1	0\\
67.11	0\\
67.12	0\\
67.13	0\\
67.14	0\\
67.15	0\\
67.16	0\\
67.17	0\\
67.18	0\\
67.19	0\\
67.2	0\\
67.21	0\\
67.22	0\\
67.23	0\\
67.24	0\\
67.25	0\\
67.26	1.73472347597681e-18\\
67.27	0\\
67.28	0\\
67.29	0\\
67.3	0\\
67.31	0\\
67.32	0\\
67.33	0\\
67.34	0\\
67.35	0\\
67.36	0\\
67.37	0\\
67.38	0\\
67.39	0\\
67.4	0\\
67.41	0\\
67.42	0\\
67.43	0\\
67.44	0\\
67.45	0\\
67.46	0\\
67.47	0\\
67.48	0\\
67.49	0\\
67.5	0\\
67.51	0\\
67.52	0\\
67.53	0\\
67.54	0\\
67.55	0\\
67.56	0\\
67.57	0\\
67.58	0\\
67.59	0\\
67.6	0\\
67.61	0\\
67.62	0\\
67.63	0\\
67.64	0\\
67.65	0\\
67.66	0\\
67.67	0\\
67.68	0\\
67.69	0\\
67.7	0\\
67.71	0\\
67.72	0\\
67.73	0\\
67.74	1.73472347597681e-18\\
67.75	0\\
67.76	0\\
67.77	0\\
67.78	0\\
67.79	0\\
67.8	0\\
67.81	0\\
67.82	0\\
67.83	0\\
67.84	0\\
67.85	0\\
67.86	0\\
67.87	0\\
67.88	0\\
67.89	0\\
67.9	0\\
67.91	0\\
67.92	1.73472347597681e-18\\
67.93	0\\
67.94	0\\
67.95	0\\
67.96	0\\
67.97	0\\
67.98	0\\
67.99	0\\
68	1.73472347597681e-18\\
68.01	0\\
68.02	0\\
68.03	0\\
68.04	0\\
68.05	0\\
68.06	0\\
68.07	0\\
68.08	0\\
68.09	0\\
68.1	0\\
68.11	0\\
68.12	0\\
68.13	0\\
68.14	0\\
68.15	0\\
68.16	0\\
68.17	0\\
68.18	0\\
68.19	0\\
68.2	0\\
68.21	0\\
68.22	0\\
68.23	0\\
68.24	1.73472347597681e-18\\
68.25	0\\
68.26	1.73472347597681e-18\\
68.27	0\\
68.28	0\\
68.29	0\\
68.3	0\\
68.31	0\\
68.32	0\\
68.33	0\\
68.34	0\\
68.35	1.73472347597681e-18\\
68.36	0\\
68.37	0\\
68.38	0\\
68.39	0\\
68.4	0\\
68.41	0\\
68.42	0\\
68.43	0\\
68.44	0\\
68.45	0\\
68.46	0\\
68.47	0\\
68.48	0\\
68.49	0\\
68.5	0\\
68.51	0\\
68.52	0\\
68.53	0\\
68.54	0\\
68.55	0\\
68.56	0\\
68.57	0\\
68.58	0\\
68.59	0\\
68.6	0\\
68.61	0\\
68.62	0\\
68.63	0\\
68.64	0\\
68.65	1.73472347597681e-18\\
68.66	0\\
68.67	0\\
68.68	0\\
68.69	0\\
68.7	0\\
68.71	0\\
68.72	0\\
68.73	0\\
68.74	0\\
68.75	0\\
68.76	0\\
68.77	0\\
68.78	0\\
68.79	0\\
68.8	0\\
68.81	1.73472347597681e-18\\
68.82	0\\
68.83	0\\
68.84	0\\
68.85	0\\
68.86	0\\
68.87	0\\
68.88	0\\
68.89	0\\
68.9	0\\
68.91	0\\
68.92	0\\
68.93	0\\
68.94	0\\
68.95	0\\
68.96	1.73472347597681e-18\\
68.97	0\\
68.98	0\\
68.99	0\\
69	0\\
69.01	0\\
69.02	0\\
69.03	0\\
69.04	0\\
69.05	0\\
69.06	0\\
69.07	0\\
69.08	0\\
69.09	0\\
69.1	0\\
69.11	0\\
69.12	0\\
69.13	0\\
69.14	0\\
69.15	0\\
69.16	1.73472347597681e-18\\
69.17	0\\
69.18	0\\
69.19	0\\
69.2	0\\
69.21	1.73472347597681e-18\\
69.22	0\\
69.23	0\\
69.24	0\\
69.25	0\\
69.26	0\\
69.27	0\\
69.28	0\\
69.29	0\\
69.3	0\\
69.31	0\\
69.32	0\\
69.33	0\\
69.34	0\\
69.35	0\\
69.36	0\\
69.37	0\\
69.38	0\\
69.39	0\\
69.4	0\\
69.41	0\\
69.42	0\\
69.43	0\\
69.44	0\\
69.45	0\\
69.46	0\\
69.47	0\\
69.48	0\\
69.49	0\\
69.5	0\\
69.51	0\\
69.52	0\\
69.53	0\\
69.54	0\\
69.55	0\\
69.56	0\\
69.57	0\\
69.58	0\\
69.59	0\\
69.6	0\\
69.61	0\\
69.62	0\\
69.63	0\\
69.64	0\\
69.65	0\\
69.66	0\\
69.67	0\\
69.68	1.73472347597681e-18\\
69.69	0\\
69.7	0\\
69.71	0\\
69.72	0\\
69.73	0\\
69.74	0\\
69.75	0\\
69.76	0\\
69.77	0\\
69.78	1.73472347597681e-18\\
69.79	0\\
69.8	0\\
69.81	0\\
69.82	0\\
69.83	0\\
69.84	0\\
69.85	0\\
69.86	0\\
69.87	0\\
69.88	0\\
69.89	0\\
69.9	0\\
69.91	0\\
69.92	0\\
69.93	0\\
69.94	0\\
69.95	0\\
69.96	0\\
69.97	0\\
69.98	0\\
69.99	0\\
70	0\\
70.01	0\\
70.02	0\\
70.03	0\\
70.04	0\\
70.05	0\\
70.06	0\\
70.07	0\\
70.08	1.73472347597681e-18\\
70.09	0\\
70.1	0\\
70.11	0\\
70.12	0\\
70.13	0\\
70.14	0\\
70.15	0\\
70.16	0\\
70.17	0\\
70.18	1.73472347597681e-18\\
70.19	0\\
70.2	1.73472347597681e-18\\
70.21	0\\
70.22	0\\
70.23	0\\
70.24	0\\
70.25	0\\
70.26	0\\
70.27	0\\
70.28	0\\
70.29	0\\
70.3	0\\
70.31	0\\
70.32	0\\
70.33	0\\
70.34	0\\
70.35	0\\
70.36	0\\
70.37	0\\
70.38	0\\
70.39	0\\
70.4	0\\
70.41	1.73472347597681e-18\\
70.42	0\\
70.43	0\\
70.44	0\\
70.45	0\\
70.46	0\\
70.47	0\\
70.48	0\\
70.49	0\\
70.5	0\\
70.51	0\\
70.52	0\\
70.53	0\\
70.54	0\\
70.55	0\\
70.56	0\\
70.57	0\\
70.58	0\\
70.59	0\\
70.6	0\\
70.61	0\\
70.62	0\\
70.63	0\\
70.64	0\\
70.65	0\\
70.66	0\\
70.67	0\\
70.68	0\\
70.69	1.73472347597681e-18\\
70.7	0\\
70.71	0\\
70.72	0\\
70.73	0\\
70.74	0\\
70.75	0\\
70.76	0\\
70.77	0\\
70.78	0\\
70.79	0\\
70.8	0\\
70.81	0\\
70.82	0\\
70.83	0\\
70.84	0\\
70.85	0\\
70.86	0\\
70.87	0\\
70.88	1.73472347597681e-18\\
70.89	0\\
70.9	0\\
70.91	1.73472347597681e-18\\
70.92	0\\
70.93	0\\
70.94	0\\
70.95	0\\
70.96	0\\
70.97	0\\
70.98	0\\
70.99	0\\
71	1.73472347597681e-18\\
71.01	0\\
71.02	0\\
71.03	0\\
71.04	0\\
71.05	0\\
71.06	0\\
71.07	0\\
71.08	0\\
71.09	0\\
71.1	0\\
71.11	0\\
71.12	0\\
71.13	0\\
71.14	0\\
71.15	0\\
71.16	0\\
71.17	1.73472347597681e-18\\
71.18	0\\
71.19	0\\
71.2	0\\
71.21	0\\
71.22	0\\
71.23	0\\
71.24	0\\
71.25	0\\
71.26	0\\
71.27	0\\
71.28	0\\
71.29	0\\
71.3	0\\
71.31	0\\
71.32	0\\
71.33	0\\
71.34	0\\
71.35	0\\
71.36	0\\
71.37	0\\
71.38	0\\
71.39	0\\
71.4	0\\
71.41	0\\
71.42	0\\
71.43	0\\
71.44	0\\
71.45	0\\
71.46	0\\
71.47	0\\
71.48	0\\
71.49	0\\
71.5	0\\
71.51	0\\
71.52	0\\
71.53	0\\
71.54	0\\
71.55	0\\
71.56	0\\
71.57	0\\
71.58	1.73472347597681e-18\\
71.59	0\\
71.6	0\\
71.61	0\\
71.62	0\\
71.63	0\\
71.64	0\\
71.65	0\\
71.66	0\\
71.67	0\\
71.68	0\\
71.69	0\\
71.7	0\\
71.71	0\\
71.72	0\\
71.73	0\\
71.74	0\\
71.75	1.73472347597681e-18\\
71.76	0\\
71.77	0\\
71.78	0\\
71.79	0\\
71.8	0\\
71.81	0\\
71.82	0\\
71.83	0\\
71.84	0\\
71.85	0\\
71.86	0\\
71.87	0\\
71.88	0\\
71.89	0\\
71.9	0\\
71.91	0\\
71.92	0\\
71.93	0\\
71.94	0\\
71.95	0\\
71.96	0\\
71.97	0\\
71.98	0\\
71.99	0\\
72	0\\
72.01	0\\
72.02	0\\
72.03	0\\
72.04	0\\
72.05	0\\
72.06	0\\
72.07	0\\
72.08	0\\
72.09	0\\
72.1	0\\
72.11	0\\
72.12	1.73472347597681e-18\\
72.13	0\\
72.14	0\\
72.15	0\\
72.16	0\\
72.17	0\\
72.18	0\\
72.19	1.73472347597681e-18\\
72.2	0\\
72.21	1.73472347597681e-18\\
72.22	0\\
72.23	0\\
72.24	0\\
72.25	0\\
72.26	0\\
72.27	1.73472347597681e-18\\
72.28	0\\
72.29	0\\
72.3	0\\
72.31	0\\
72.32	0\\
72.33	0\\
72.34	0\\
72.35	0\\
72.36	0\\
72.37	0\\
72.38	0\\
72.39	0\\
72.4	0\\
72.41	0\\
72.42	0\\
72.43	1.73472347597681e-18\\
72.44	0\\
72.45	0\\
72.46	0\\
72.47	0\\
72.48	0\\
72.49	0\\
72.5	1.73472347597681e-18\\
72.51	0\\
72.52	0\\
72.53	0\\
72.54	0\\
72.55	0\\
72.56	0\\
72.57	0\\
72.58	0\\
72.59	0\\
72.6	0\\
72.61	0\\
72.62	0\\
72.63	0\\
72.64	0\\
72.65	0\\
72.66	0\\
72.67	0\\
72.68	0\\
72.69	0\\
72.7	1.73472347597681e-18\\
72.71	0\\
72.72	0\\
72.73	0\\
72.74	0\\
72.75	0\\
72.76	0\\
72.77	0\\
72.78	0\\
72.79	0\\
72.8	0\\
72.81	1.73472347597681e-18\\
72.82	0\\
72.83	0\\
72.84	0\\
72.85	0\\
72.86	0\\
72.87	0\\
72.88	0\\
72.89	0\\
72.9	0\\
72.91	1.73472347597681e-18\\
72.92	0\\
72.93	0\\
72.94	0\\
72.95	0\\
72.96	0\\
72.97	0\\
72.98	0\\
72.99	0\\
73	0\\
73.01	0\\
73.02	0\\
73.03	0\\
73.04	0\\
73.05	1.73472347597681e-18\\
73.06	0\\
73.07	0\\
73.08	0\\
73.09	0\\
73.1	0\\
73.11	0\\
73.12	0\\
73.13	0\\
73.14	0\\
73.15	0\\
73.16	0\\
73.17	0\\
73.18	0\\
73.19	1.73472347597681e-18\\
73.2	0\\
73.21	0\\
73.22	0\\
73.23	0\\
73.24	1.73472347597681e-18\\
73.25	0\\
73.26	0\\
73.27	0\\
73.28	0\\
73.29	0\\
73.3	0\\
73.31	0\\
73.32	0\\
73.33	0\\
73.34	0\\
73.35	0\\
73.36	0\\
73.37	0\\
73.38	0\\
73.39	0\\
73.4	0\\
73.41	0\\
73.42	0\\
73.43	0\\
73.44	0\\
73.45	0\\
73.46	0\\
73.47	0\\
73.48	0\\
73.49	0\\
73.5	0\\
73.51	0\\
73.52	0\\
73.53	0\\
73.54	1.73472347597681e-18\\
73.55	0\\
73.56	0\\
73.57	0\\
73.58	0\\
73.59	0\\
73.6	0\\
73.61	0\\
73.62	0\\
73.63	0\\
73.64	1.73472347597681e-18\\
73.65	0\\
73.66	0\\
73.67	0\\
73.68	0\\
73.69	1.73472347597681e-18\\
73.7	0\\
73.71	0\\
73.72	0\\
73.73	0\\
73.74	0\\
73.75	0\\
73.76	0\\
73.77	1.73472347597681e-18\\
73.78	0\\
73.79	0\\
73.8	0\\
73.81	0\\
73.82	0\\
73.83	0\\
73.84	0\\
73.85	0\\
73.86	0\\
73.87	0\\
73.88	0\\
73.89	0\\
73.9	0\\
73.91	0\\
73.92	0\\
73.93	0\\
73.94	0\\
73.95	0\\
73.96	0\\
73.97	0\\
73.98	0\\
73.99	0\\
74	0\\
74.01	0\\
74.02	0\\
74.03	0\\
74.04	0\\
74.05	0\\
74.06	0\\
74.07	0\\
74.08	1.73472347597681e-18\\
74.09	0\\
74.1	0\\
74.11	0\\
74.12	1.73472347597681e-18\\
74.13	0\\
74.14	0\\
74.15	1.73472347597681e-18\\
74.16	1.73472347597681e-18\\
74.17	0\\
74.18	0\\
74.19	0\\
74.2	0\\
74.21	0\\
74.22	0\\
74.23	0\\
74.24	0\\
74.25	0\\
74.26	0\\
74.27	0\\
74.28	1.73472347597681e-18\\
74.29	0\\
74.3	0\\
74.31	0\\
74.32	0\\
74.33	0\\
74.34	0\\
74.35	0\\
74.36	0\\
74.37	0\\
74.38	0\\
74.39	0\\
74.4	0\\
74.41	0\\
74.42	0\\
74.43	0\\
74.44	0\\
74.45	0\\
74.46	0\\
74.47	0\\
74.48	0\\
74.49	0\\
74.5	0\\
74.51	0\\
74.52	0\\
74.53	0\\
74.54	0\\
74.55	0\\
74.56	0\\
74.57	0\\
74.58	0\\
74.59	0\\
74.6	0\\
74.61	0\\
74.62	0\\
74.63	0\\
74.64	0\\
74.65	1.73472347597681e-18\\
74.66	0\\
74.67	1.73472347597681e-18\\
74.68	0\\
74.69	0\\
74.7	0\\
74.71	0\\
74.72	0\\
74.73	0\\
74.74	0\\
74.75	0\\
74.76	0\\
74.77	0\\
74.78	0\\
74.79	0\\
74.8	0\\
74.81	0\\
74.82	0\\
74.83	0\\
74.84	0\\
74.85	0\\
74.86	0\\
74.87	0\\
74.88	0\\
74.89	1.73472347597681e-18\\
74.9	1.73472347597681e-18\\
74.91	1.73472347597681e-18\\
74.92	0\\
74.93	0\\
74.94	1.73472347597681e-18\\
74.95	0\\
74.96	0\\
74.97	0\\
74.98	0\\
74.99	0\\
75	0\\
75.01	0\\
75.02	1.73472347597681e-18\\
75.03	0\\
75.04	0\\
75.05	0\\
75.06	0\\
75.07	0\\
75.08	0\\
75.09	0\\
75.1	0\\
75.11	0\\
75.12	0\\
75.13	0\\
75.14	0\\
75.15	0\\
75.16	0\\
75.17	0\\
75.18	0\\
75.19	0\\
75.2	0\\
75.21	0\\
75.22	0\\
75.23	0\\
75.24	0\\
75.25	0\\
75.26	0\\
75.27	0\\
75.28	0\\
75.29	0\\
75.3	0\\
75.31	0\\
75.32	0\\
75.33	0\\
75.34	0\\
75.35	0\\
75.36	0\\
75.37	0\\
75.38	0\\
75.39	0\\
75.4	1.73472347597681e-18\\
75.41	0\\
75.42	0\\
75.43	0\\
75.44	0\\
75.45	0\\
75.46	0\\
75.47	0\\
75.48	0\\
75.49	0\\
75.5	0\\
75.51	0\\
75.52	0\\
75.53	0\\
75.54	0\\
75.55	0\\
75.56	0\\
75.57	0\\
75.58	0\\
75.59	0\\
75.6	0\\
75.61	0\\
75.62	0\\
75.63	0\\
75.64	0\\
75.65	0\\
75.66	0\\
75.67	1.73472347597681e-18\\
75.68	0\\
75.69	0\\
75.7	0\\
75.71	0\\
75.72	0\\
75.73	0\\
75.74	1.73472347597681e-18\\
75.75	0\\
75.76	0\\
75.77	0\\
75.78	0\\
75.79	0\\
75.8	0\\
75.81	0\\
75.82	0\\
75.83	0\\
75.84	0\\
75.85	0\\
75.86	0\\
75.87	0\\
75.88	0\\
75.89	1.73472347597681e-18\\
75.9	0\\
75.91	0\\
75.92	0\\
75.93	0\\
75.94	0\\
75.95	0\\
75.96	0\\
75.97	0\\
75.98	0\\
75.99	0\\
76	0\\
76.01	0\\
76.02	0\\
76.03	0\\
76.04	0\\
76.05	0\\
76.06	0\\
76.07	0\\
76.08	0\\
76.09	0\\
76.1	0\\
76.11	0\\
76.12	0\\
76.13	0\\
76.14	0\\
76.15	1.73472347597681e-18\\
76.16	0\\
76.17	0\\
76.18	0\\
76.19	0\\
76.2	0\\
76.21	1.73472347597681e-18\\
76.22	1.73472347597681e-18\\
76.23	0\\
76.24	0\\
76.25	0\\
76.26	0\\
76.27	0\\
76.28	0\\
76.29	0\\
76.3	0\\
76.31	0\\
76.32	0\\
76.33	0\\
76.34	0\\
76.35	0\\
76.36	0\\
76.37	0\\
76.38	0\\
76.39	0\\
76.4	0\\
76.41	0\\
76.42	0\\
76.43	0\\
76.44	0\\
76.45	0\\
76.46	1.73472347597681e-18\\
76.47	0\\
76.48	0\\
76.49	0\\
76.5	0\\
76.51	0\\
76.52	0\\
76.53	0\\
76.54	1.73472347597681e-18\\
76.55	0\\
76.56	0\\
76.57	0\\
76.58	0\\
76.59	0\\
76.6	0\\
76.61	0\\
76.62	0\\
76.63	0\\
76.64	0\\
76.65	0\\
76.66	0\\
76.67	0\\
76.68	0\\
76.69	0\\
76.7	0\\
76.71	0\\
76.72	0\\
76.73	1.73472347597681e-18\\
76.74	0\\
76.75	0\\
76.76	0\\
76.77	0\\
76.78	0\\
76.79	0\\
76.8	0\\
76.81	0\\
76.82	0\\
76.83	1.73472347597681e-18\\
76.84	0\\
76.85	0\\
76.86	0\\
76.87	0\\
76.88	0\\
76.89	0\\
76.9	0\\
76.91	0\\
76.92	1.73472347597681e-18\\
76.93	0\\
76.94	0\\
76.95	0\\
76.96	0\\
76.97	1.73472347597681e-18\\
76.98	0\\
76.99	0\\
77	1.73472347597681e-18\\
77.01	1.73472347597681e-18\\
77.02	0\\
77.03	0\\
77.04	0\\
77.05	0\\
77.06	0\\
77.07	0\\
77.08	1.73472347597681e-18\\
77.09	0\\
77.1	0\\
77.11	0\\
77.12	0\\
77.13	0\\
77.14	0\\
77.15	0\\
77.16	0\\
77.17	0\\
77.18	0\\
77.19	0\\
77.2	0\\
77.21	0\\
77.22	0\\
77.23	0\\
77.24	0\\
77.25	0\\
77.26	0\\
77.27	0\\
77.28	0\\
77.29	0\\
77.3	0\\
77.31	0\\
77.32	0\\
77.33	0\\
77.34	0\\
77.35	0\\
77.36	0\\
77.37	1.73472347597681e-18\\
77.38	0\\
77.39	0\\
77.4	0\\
77.41	1.73472347597681e-18\\
77.42	0\\
77.43	0\\
77.44	0\\
77.45	0\\
77.46	0\\
77.47	0\\
77.48	0\\
77.49	0\\
77.5	0\\
77.51	0\\
77.52	0\\
77.53	0\\
77.54	0\\
77.55	0\\
77.56	0\\
77.57	0\\
77.58	0\\
77.59	0\\
77.6	0\\
77.61	0\\
77.62	0\\
77.63	0\\
77.64	0\\
77.65	0\\
77.66	0\\
77.67	0\\
77.68	0\\
77.69	0\\
77.7	0\\
77.71	0\\
77.72	0\\
77.73	0\\
77.74	0\\
77.75	0\\
77.76	0\\
77.77	0\\
77.78	0\\
77.79	1.73472347597681e-18\\
77.8	0\\
77.81	0\\
77.82	0\\
77.83	0\\
77.84	0\\
77.85	0\\
77.86	0\\
77.87	0\\
77.88	0\\
77.89	0\\
77.9	0\\
77.91	1.73472347597681e-18\\
77.92	0\\
77.93	0\\
77.94	0\\
77.95	1.73472347597681e-18\\
77.96	0\\
77.97	1.73472347597681e-18\\
77.98	1.73472347597681e-18\\
77.99	0\\
78	0\\
78.01	0\\
78.02	0\\
78.03	0\\
78.04	0\\
78.05	0\\
78.06	0\\
78.07	0\\
78.08	0\\
78.09	0\\
78.1	0\\
78.11	0\\
78.12	0\\
78.13	0\\
78.14	0\\
78.15	0\\
78.16	0\\
78.17	0\\
78.18	0\\
78.19	0\\
78.2	1.73472347597681e-18\\
78.21	0\\
78.22	0\\
78.23	0\\
78.24	0\\
78.25	0\\
78.26	0\\
78.27	0\\
78.28	0\\
78.29	0\\
78.3	0\\
78.31	0\\
78.32	1.73472347597681e-18\\
78.33	0\\
78.34	0\\
78.35	0\\
78.36	0\\
78.37	0\\
78.38	0\\
78.39	0\\
78.4	0\\
78.41	0\\
78.42	1.73472347597681e-18\\
78.43	0\\
78.44	1.73472347597681e-18\\
78.45	0\\
78.46	0\\
78.47	0\\
78.48	0\\
78.49	0\\
78.5	0\\
78.51	0\\
78.52	0\\
78.53	0\\
78.54	0\\
78.55	0\\
78.56	0\\
78.57	0\\
78.58	0\\
78.59	0\\
78.6	0\\
78.61	0\\
78.62	0\\
78.63	0\\
78.64	0\\
78.65	0\\
78.66	0\\
78.67	0\\
78.68	1.73472347597681e-18\\
78.69	0\\
78.7	0\\
78.71	0\\
78.72	0\\
78.73	0\\
78.74	0\\
78.75	0\\
78.76	0\\
78.77	0\\
78.78	0\\
78.79	0\\
78.8	0\\
78.81	0\\
78.82	0\\
78.83	0\\
78.84	0\\
78.85	0\\
78.86	0\\
78.87	0\\
78.88	0\\
78.89	0\\
78.9	0\\
78.91	1.73472347597681e-18\\
78.92	0\\
78.93	0\\
78.94	0\\
78.95	0\\
78.96	0\\
78.97	0\\
78.98	0\\
78.99	0\\
79	0\\
79.01	0\\
79.02	0\\
79.03	0\\
79.04	0\\
79.05	0\\
79.06	0\\
79.07	0\\
79.08	0\\
79.09	0\\
79.1	0\\
79.11	0\\
79.12	0\\
79.13	0\\
79.14	0\\
79.15	0\\
79.16	0\\
79.17	0\\
79.18	0\\
79.19	0\\
79.2	0\\
79.21	0\\
79.22	0\\
79.23	0\\
79.24	0\\
79.25	0\\
79.26	0\\
79.27	0\\
79.28	0\\
79.29	0\\
79.3	0\\
79.31	0\\
79.32	0\\
79.33	0\\
79.34	0\\
79.35	0\\
79.36	0\\
79.37	0\\
79.38	0\\
79.39	0\\
79.4	0\\
79.41	0\\
79.42	0\\
79.43	0\\
79.44	0\\
79.45	0\\
79.46	0\\
79.47	0\\
79.48	0\\
79.49	0\\
79.5	0\\
79.51	0\\
79.52	0\\
79.53	0\\
79.54	0\\
79.55	1.73472347597681e-18\\
79.56	0\\
79.57	0\\
79.58	0\\
79.59	0\\
79.6	0\\
79.61	0\\
79.62	0\\
79.63	0\\
79.64	0\\
79.65	0\\
79.66	0\\
79.67	0\\
79.68	0\\
79.69	0\\
79.7	0\\
79.71	0\\
79.72	1.73472347597681e-18\\
79.73	1.73472347597681e-18\\
79.74	0\\
79.75	0\\
79.76	0\\
79.77	0\\
79.78	0\\
79.79	0\\
79.8	0\\
79.81	0\\
79.82	0\\
79.83	0\\
79.84	0\\
79.85	0\\
79.86	0\\
79.87	0\\
79.88	1.73472347597681e-18\\
79.89	0\\
79.9	0\\
79.91	0\\
79.92	0\\
79.93	0\\
79.94	0\\
79.95	0\\
79.96	1.73472347597681e-18\\
79.97	0\\
79.98	0\\
79.99	0\\
80	0\\
80.01	0\\
};
\addplot [color=mycolor1,solid]
  table[row sep=crcr]{%
80.01	0\\
80.02	0\\
80.03	0\\
80.04	0\\
80.05	0\\
80.06	0\\
80.07	0\\
80.08	0\\
80.09	1.73472347597681e-18\\
80.1	0\\
80.11	0\\
80.12	0\\
80.13	0\\
80.14	0\\
80.15	0\\
80.16	0\\
80.17	0\\
80.18	1.73472347597681e-18\\
80.19	0\\
80.2	0\\
80.21	0\\
80.22	0\\
80.23	0\\
80.24	0\\
80.25	0\\
80.26	0\\
80.27	0\\
80.28	0\\
80.29	0\\
80.3	0\\
80.31	0\\
80.32	0\\
80.33	0\\
80.34	0\\
80.35	0\\
80.36	0\\
80.37	0\\
80.38	0\\
80.39	0\\
80.4	0\\
80.41	0\\
80.42	0\\
80.43	0\\
80.44	0\\
80.45	0\\
80.46	0\\
80.47	0\\
80.48	0\\
80.49	0\\
80.5	1.73472347597681e-18\\
80.51	0\\
80.52	0\\
80.53	0\\
80.54	1.73472347597681e-18\\
80.55	0\\
80.56	0\\
80.57	0\\
80.58	0\\
80.59	0\\
80.6	0\\
80.61	0\\
80.62	0\\
80.63	0\\
80.64	0\\
80.65	0\\
80.66	0\\
80.67	0\\
80.68	0\\
80.69	0\\
80.7	0\\
80.71	0\\
80.72	0\\
80.73	0\\
80.74	0\\
80.75	0\\
80.76	0\\
80.77	0\\
80.78	0\\
80.79	0\\
80.8	0\\
80.81	1.73472347597681e-18\\
80.82	0\\
80.83	0\\
80.84	0\\
80.85	0\\
80.86	0\\
80.87	0\\
80.88	0\\
80.89	0\\
80.9	0\\
80.91	0\\
80.92	0\\
80.93	0\\
80.94	0\\
80.95	0\\
80.96	1.73472347597681e-18\\
80.97	0\\
80.98	0\\
80.99	0\\
81	0\\
81.01	0\\
81.02	0\\
81.03	0\\
81.04	0\\
81.05	0\\
81.06	0\\
81.07	0\\
81.08	0\\
81.09	0\\
81.1	0\\
81.11	0\\
81.12	0\\
81.13	1.73472347597681e-18\\
81.14	0\\
81.15	0\\
81.16	0\\
81.17	0\\
81.18	0\\
81.19	0\\
81.2	0\\
81.21	0\\
81.22	0\\
81.23	0\\
81.24	0\\
81.25	0\\
81.26	0\\
81.27	0\\
81.28	0\\
81.29	0\\
81.3	0\\
81.31	0\\
81.32	0\\
81.33	0\\
81.34	0\\
81.35	0\\
81.36	0\\
81.37	1.73472347597681e-18\\
81.38	0\\
81.39	0\\
81.4	0\\
81.41	0\\
81.42	0\\
81.43	0\\
81.44	0\\
81.45	0\\
81.46	0\\
81.47	0\\
81.48	0\\
81.49	0\\
81.5	0\\
81.51	0\\
81.52	0\\
81.53	0\\
81.54	0\\
81.55	0\\
81.56	0\\
81.57	0\\
81.58	0\\
81.59	0\\
81.6	0\\
81.61	0\\
81.62	0\\
81.63	0\\
81.64	0\\
81.65	0\\
81.66	0\\
81.67	0\\
81.68	0\\
81.69	0\\
81.7	0\\
81.71	1.73472347597681e-18\\
81.72	0\\
81.73	0\\
81.74	0\\
81.75	0\\
81.76	0\\
81.77	0\\
81.78	0\\
81.79	0\\
81.8	0\\
81.81	0\\
81.82	0\\
81.83	0\\
81.84	0\\
81.85	0\\
81.86	0\\
81.87	0\\
81.88	0\\
81.89	0\\
81.9	0\\
81.91	0\\
81.92	0\\
81.93	0\\
81.94	0\\
81.95	0\\
81.96	0\\
81.97	0\\
81.98	0\\
81.99	0\\
82	0\\
82.01	0\\
82.02	0\\
82.03	0\\
82.04	0\\
82.05	0\\
82.06	0\\
82.07	0\\
82.08	0\\
82.09	0\\
82.1	0\\
82.11	0\\
82.12	0\\
82.13	0\\
82.14	0\\
82.15	0\\
82.16	0\\
82.17	0\\
82.18	0\\
82.19	0\\
82.2	0\\
82.21	0\\
82.22	0\\
82.23	0\\
82.24	0\\
82.25	0\\
82.26	0\\
82.27	0\\
82.28	0\\
82.29	0\\
82.3	0\\
82.31	0\\
82.32	0\\
82.33	0\\
82.34	0\\
82.35	0\\
82.36	0\\
82.37	0\\
82.38	0\\
82.39	0\\
82.4	0\\
82.41	0\\
82.42	0\\
82.43	0\\
82.44	0\\
82.45	0\\
82.46	0\\
82.47	1.73472347597681e-18\\
82.48	0\\
82.49	0\\
82.5	0\\
82.51	0\\
82.52	0\\
82.53	0\\
82.54	0\\
82.55	0\\
82.56	0\\
82.57	0\\
82.58	0\\
82.59	0\\
82.6	0\\
82.61	0\\
82.62	0\\
82.63	0\\
82.64	1.73472347597681e-18\\
82.65	0\\
82.66	0\\
82.67	0\\
82.68	0\\
82.69	0\\
82.7	0\\
82.71	1.73472347597681e-18\\
82.72	0\\
82.73	0\\
82.74	0\\
82.75	0\\
82.76	0\\
82.77	0\\
82.78	0\\
82.79	0\\
82.8	0\\
82.81	0\\
82.82	0\\
82.83	0\\
82.84	0\\
82.85	0\\
82.86	0\\
82.87	0\\
82.88	0\\
82.89	0\\
82.9	0\\
82.91	0\\
82.92	0\\
82.93	0\\
82.94	0\\
82.95	0\\
82.96	0\\
82.97	0\\
82.98	0\\
82.99	0\\
83	0\\
83.01	0\\
83.02	0\\
83.03	0\\
83.04	0\\
83.05	0\\
83.06	0\\
83.07	0\\
83.08	0\\
83.09	0\\
83.1	0\\
83.11	0\\
83.12	1.73472347597681e-18\\
83.13	0\\
83.14	0\\
83.15	0\\
83.16	0\\
83.17	0\\
83.18	0\\
83.19	0\\
83.2	1.73472347597681e-18\\
83.21	0\\
83.22	0\\
83.23	0\\
83.24	0\\
83.25	0\\
83.26	0\\
83.27	0\\
83.28	0\\
83.29	0\\
83.3	0\\
83.31	0\\
83.32	0\\
83.33	0\\
83.34	0\\
83.35	0\\
83.36	0\\
83.37	0\\
83.38	0\\
83.39	0\\
83.4	0\\
83.41	0\\
83.42	0\\
83.43	0\\
83.44	0\\
83.45	0\\
83.46	1.73472347597681e-18\\
83.47	0\\
83.48	0\\
83.49	0\\
83.5	0\\
83.51	0\\
83.52	0\\
83.53	0\\
83.54	0\\
83.55	0\\
83.56	0\\
83.57	0\\
83.58	0\\
83.59	0\\
83.6	0\\
83.61	0\\
83.62	0\\
83.63	0\\
83.64	0\\
83.65	0\\
83.66	0\\
83.67	0\\
83.68	0\\
83.69	0\\
83.7	0\\
83.71	0\\
83.72	0\\
83.73	0\\
83.74	0\\
83.75	0\\
83.76	0\\
83.77	0\\
83.78	0\\
83.79	0\\
83.8	0\\
83.81	0\\
83.82	0\\
83.83	0\\
83.84	0\\
83.85	0\\
83.86	0\\
83.87	0\\
83.88	0\\
83.89	0\\
83.9	0\\
83.91	0\\
83.92	0\\
83.93	0\\
83.94	0\\
83.95	0\\
83.96	0\\
83.97	0\\
83.98	0\\
83.99	0\\
84	0\\
84.01	0\\
84.02	0\\
84.03	0\\
84.04	0\\
84.05	0\\
84.06	0\\
84.07	0\\
84.08	0\\
84.09	0\\
84.1	0\\
84.11	0\\
84.12	0\\
84.13	0\\
84.14	0\\
84.15	0\\
84.16	0\\
84.17	0\\
84.18	0\\
84.19	0\\
84.2	0\\
84.21	0\\
84.22	0\\
84.23	0\\
84.24	0\\
84.25	0\\
84.26	0\\
84.27	0\\
84.28	0\\
84.29	0\\
84.3	0\\
84.31	0\\
84.32	1.73472347597681e-18\\
84.33	0\\
84.34	0\\
84.35	1.73472347597681e-18\\
84.36	0\\
84.37	0\\
84.38	0\\
84.39	0\\
84.4	0\\
84.41	0\\
84.42	0\\
84.43	0\\
84.44	0\\
84.45	0\\
84.46	0\\
84.47	0\\
84.48	0\\
84.49	0\\
84.5	0\\
84.51	0\\
84.52	0\\
84.53	0\\
84.54	0\\
84.55	0\\
84.56	0\\
84.57	0\\
84.58	0\\
84.59	0\\
84.6	0\\
84.61	0\\
84.62	0\\
84.63	0\\
84.64	0\\
84.65	0\\
84.66	0\\
84.67	0\\
84.68	0\\
84.69	0\\
84.7	0\\
84.71	0\\
84.72	0\\
84.73	0\\
84.74	0\\
84.75	0\\
84.76	0\\
84.77	0\\
84.78	0\\
84.79	1.73472347597681e-18\\
84.8	0\\
84.81	0\\
84.82	0\\
84.83	0\\
84.84	0\\
84.85	0\\
84.86	0\\
84.87	0\\
84.88	0\\
84.89	0\\
84.9	0\\
84.91	0\\
84.92	0\\
84.93	0\\
84.94	1.73472347597681e-18\\
84.95	0\\
84.96	0\\
84.97	0\\
84.98	0\\
84.99	0\\
85	0\\
85.01	0\\
85.02	1.73472347597681e-18\\
85.03	0\\
85.04	0\\
85.05	0\\
85.06	0\\
85.07	0\\
85.08	0\\
85.09	0\\
85.1	0\\
85.11	0\\
85.12	0\\
85.13	0\\
85.14	0\\
85.15	0\\
85.16	0\\
85.17	0\\
85.18	0\\
85.19	0\\
85.2	0\\
85.21	0\\
85.22	0\\
85.23	0\\
85.24	0\\
85.25	0\\
85.26	0\\
85.27	0\\
85.28	0\\
85.29	0\\
85.3	0\\
85.31	0\\
85.32	0\\
85.33	0\\
85.34	0\\
85.35	0\\
85.36	0\\
85.37	1.73472347597681e-18\\
85.38	0\\
85.39	0\\
85.4	0\\
85.41	1.73472347597681e-18\\
85.42	0\\
85.43	0\\
85.44	0\\
85.45	0\\
85.46	0\\
85.47	0\\
85.48	0\\
85.49	0\\
85.5	0\\
85.51	0\\
85.52	0\\
85.53	0\\
85.54	1.73472347597681e-18\\
85.55	0\\
85.56	1.73472347597681e-18\\
85.57	0\\
85.58	0\\
85.59	0\\
85.6	0\\
85.61	0\\
85.62	0\\
85.63	0\\
85.64	0\\
85.65	0\\
85.66	0\\
85.67	0\\
85.68	0\\
85.69	0\\
85.7	0\\
85.71	0\\
85.72	0\\
85.73	0\\
85.74	0\\
85.75	1.73472347597681e-18\\
85.76	0\\
85.77	1.73472347597681e-18\\
85.78	0\\
85.79	0\\
85.8	0\\
85.81	1.73472347597681e-18\\
85.82	0\\
85.83	1.73472347597681e-18\\
85.84	0\\
85.85	0\\
85.86	0\\
85.87	0\\
85.88	0\\
85.89	0\\
85.9	0\\
85.91	0\\
85.92	0\\
85.93	0\\
85.94	0\\
85.95	0\\
85.96	0\\
85.97	0\\
85.98	0\\
85.99	0\\
86	0\\
86.01	0\\
86.02	0\\
86.03	0\\
86.04	0\\
86.05	0\\
86.06	0\\
86.07	0\\
86.08	1.73472347597681e-18\\
86.09	0\\
86.1	0\\
86.11	0\\
86.12	0\\
86.13	0\\
86.14	0\\
86.15	0\\
86.16	0\\
86.17	0\\
86.18	0\\
86.19	0\\
86.2	0\\
86.21	0\\
86.22	0\\
86.23	0\\
86.24	0\\
86.25	0\\
86.26	0\\
86.27	0\\
86.28	0\\
86.29	0\\
86.3	0\\
86.31	0\\
86.32	0\\
86.33	0\\
86.34	0\\
86.35	0\\
86.36	0\\
86.37	0\\
86.38	0\\
86.39	0\\
86.4	0\\
86.41	0\\
86.42	1.73472347597681e-18\\
86.43	0\\
86.44	0\\
86.45	0\\
86.46	1.73472347597681e-18\\
86.47	0\\
86.48	0\\
86.49	0\\
86.5	0\\
86.51	0\\
86.52	0\\
86.53	0\\
86.54	0\\
86.55	0\\
86.56	0\\
86.57	0\\
86.58	0\\
86.59	0\\
86.6	0\\
86.61	1.73472347597681e-18\\
86.62	0\\
86.63	0\\
86.64	0\\
86.65	0\\
86.66	0\\
86.67	0\\
86.68	0\\
86.69	0\\
86.7	0\\
86.71	0\\
86.72	0\\
86.73	0\\
86.74	0\\
86.75	1.73472347597681e-18\\
86.76	0\\
86.77	0\\
86.78	0\\
86.79	0\\
86.8	0\\
86.81	0\\
86.82	0\\
86.83	0\\
86.84	0\\
86.85	1.73472347597681e-18\\
86.86	0\\
86.87	0\\
86.88	0\\
86.89	0\\
86.9	0\\
86.91	1.73472347597681e-18\\
86.92	0\\
86.93	0\\
86.94	0\\
86.95	0\\
86.96	0\\
86.97	0\\
86.98	0\\
86.99	0\\
87	0\\
87.01	0\\
87.02	0\\
87.03	0\\
87.04	0\\
87.05	1.73472347597681e-18\\
87.06	1.73472347597681e-18\\
87.07	0\\
87.08	0\\
87.09	0\\
87.1	0\\
87.11	0\\
87.12	0\\
87.13	0\\
87.14	0\\
87.15	0\\
87.16	0\\
87.17	0\\
87.18	0\\
87.19	0\\
87.2	1.73472347597681e-18\\
87.21	0\\
87.22	0\\
87.23	0\\
87.24	0\\
87.25	0\\
87.26	0\\
87.27	0\\
87.28	0\\
87.29	0\\
87.3	0\\
87.31	0\\
87.32	0\\
87.33	0\\
87.34	0\\
87.35	0\\
87.36	0\\
87.37	0\\
87.38	0\\
87.39	0\\
87.4	0\\
87.41	0\\
87.42	0\\
87.43	0\\
87.44	0\\
87.45	0\\
87.46	0\\
87.47	0\\
87.48	0\\
87.49	0\\
87.5	0\\
87.51	0\\
87.52	0\\
87.53	0\\
87.54	0\\
87.55	0\\
87.56	0\\
87.57	0\\
87.58	0\\
87.59	0\\
87.6	0\\
87.61	0\\
87.62	0\\
87.63	0\\
87.64	0\\
87.65	0\\
87.66	0\\
87.67	0\\
87.68	0\\
87.69	0\\
87.7	0\\
87.71	0\\
87.72	0\\
87.73	0\\
87.74	0\\
87.75	0\\
87.76	0\\
87.77	0\\
87.78	0\\
87.79	0\\
87.8	0\\
87.81	0\\
87.82	0\\
87.83	0\\
87.84	0\\
87.85	0\\
87.86	0\\
87.87	0\\
87.88	0\\
87.89	0\\
87.9	0\\
87.91	0\\
87.92	0\\
87.93	0\\
87.94	0\\
87.95	0\\
87.96	0\\
87.97	1.73472347597681e-18\\
87.98	0\\
87.99	0\\
88	0\\
88.01	0\\
88.02	0\\
88.03	0\\
88.04	0\\
88.05	0\\
88.06	0\\
88.07	0\\
88.08	0\\
88.09	0\\
88.1	0\\
88.11	0\\
88.12	0\\
88.13	0\\
88.14	0\\
88.15	0\\
88.16	0\\
88.17	1.73472347597681e-18\\
88.18	0\\
88.19	0\\
88.2	0\\
88.21	0\\
88.22	0\\
88.23	0\\
88.24	0\\
88.25	0\\
88.26	0\\
88.27	0\\
88.28	0\\
88.29	0\\
88.3	0\\
88.31	0\\
88.32	0\\
88.33	0\\
88.34	1.73472347597681e-18\\
88.35	0\\
88.36	0\\
88.37	0\\
88.38	1.73472347597681e-18\\
88.39	0\\
88.4	0\\
88.41	0\\
88.42	0\\
88.43	1.73472347597681e-18\\
88.44	0\\
88.45	0\\
88.46	0\\
88.47	0\\
88.48	0\\
88.49	0\\
88.5	0\\
88.51	0\\
88.52	0\\
88.53	0\\
88.54	0\\
88.55	0\\
88.56	0\\
88.57	0\\
88.58	0\\
88.59	0\\
88.6	0\\
88.61	0\\
88.62	0\\
88.63	0\\
88.64	0\\
88.65	0\\
88.66	1.73472347597681e-18\\
88.67	0\\
88.68	0\\
88.69	1.73472347597681e-18\\
88.7	0\\
88.71	0\\
88.72	0\\
88.73	0\\
88.74	0\\
88.75	0\\
88.76	0\\
88.77	0\\
88.78	0\\
88.79	0\\
88.8	0\\
88.81	0\\
88.82	0\\
88.83	0\\
88.84	0\\
88.85	0\\
88.86	0\\
88.87	0\\
88.88	0\\
88.89	0\\
88.9	0\\
88.91	0\\
88.92	0\\
88.93	0\\
88.94	0\\
88.95	0\\
88.96	0\\
88.97	0\\
88.98	0\\
88.99	0\\
89	0\\
89.01	0\\
89.02	0\\
89.03	0\\
89.04	0\\
89.05	0\\
89.06	0\\
89.07	0\\
89.08	0\\
89.09	0\\
89.1	0\\
89.11	0\\
89.12	0\\
89.13	0\\
89.14	0\\
89.15	0\\
89.16	0\\
89.17	0\\
89.18	0\\
89.19	0\\
89.2	0\\
89.21	0\\
89.22	0\\
89.23	0\\
89.24	0\\
89.25	0\\
89.26	0\\
89.27	0\\
89.28	0\\
89.29	0\\
89.3	0\\
89.31	1.73472347597681e-18\\
89.32	0\\
89.33	0\\
89.34	0\\
89.35	0\\
89.36	0\\
89.37	0\\
89.38	0\\
89.39	0\\
89.4	0\\
89.41	0\\
89.42	0\\
89.43	1.73472347597681e-18\\
89.44	1.73472347597681e-18\\
89.45	0\\
89.46	0\\
89.47	0\\
89.48	0\\
89.49	0\\
89.5	0\\
89.51	0\\
89.52	0\\
89.53	0\\
89.54	0\\
89.55	0\\
89.56	0\\
89.57	0\\
89.58	0\\
89.59	0\\
89.6	0\\
89.61	0\\
89.62	0\\
89.63	0\\
89.64	0\\
89.65	0\\
89.66	0\\
89.67	0\\
89.68	0\\
89.69	0\\
89.7	0\\
89.71	0\\
89.72	0\\
89.73	0\\
89.74	0\\
89.75	1.73472347597681e-18\\
89.76	0\\
89.77	0\\
89.78	0\\
89.79	0\\
89.8	0\\
89.81	1.73472347597681e-18\\
89.82	0\\
89.83	0\\
89.84	0\\
89.85	0\\
89.86	0\\
89.87	0\\
89.88	1.73472347597681e-18\\
89.89	0\\
89.9	0\\
89.91	0\\
89.92	0\\
89.93	0\\
89.94	0\\
89.95	0\\
89.96	0\\
89.97	0\\
89.98	0\\
89.99	0\\
90	0\\
90.01	0\\
90.02	0\\
90.03	0\\
90.04	0\\
90.05	0\\
90.06	0\\
90.07	0\\
90.08	0\\
90.09	0\\
90.1	0\\
90.11	0\\
90.12	0\\
90.13	0\\
90.14	0\\
90.15	0\\
90.16	0\\
90.17	0\\
90.18	0\\
90.19	0\\
90.2	0\\
90.21	0\\
90.22	0\\
90.23	0\\
90.24	0\\
90.25	0\\
90.26	0\\
90.27	0\\
90.28	0\\
90.29	0\\
90.3	0\\
90.31	0\\
90.32	0\\
90.33	0\\
90.34	0\\
90.35	0\\
90.36	0\\
90.37	0\\
90.38	0\\
90.39	0\\
90.4	0\\
90.41	0\\
90.42	0\\
90.43	0\\
90.44	0\\
90.45	0\\
90.46	0\\
90.47	0\\
90.48	0\\
90.49	0\\
90.5	0\\
90.51	0\\
90.52	0\\
90.53	0\\
90.54	1.73472347597681e-18\\
90.55	0\\
90.56	0\\
90.57	0\\
90.58	0\\
90.59	0\\
90.6	0\\
90.61	0\\
90.62	0\\
90.63	0\\
90.64	0\\
90.65	0\\
90.66	0\\
90.67	0\\
90.68	0\\
90.69	0\\
90.7	1.73472347597681e-18\\
90.71	0\\
90.72	0\\
90.73	0\\
90.74	0\\
90.75	0\\
90.76	0\\
90.77	0\\
90.78	0\\
90.79	0\\
90.8	0\\
90.81	0\\
90.82	0\\
90.83	0\\
90.84	0\\
90.85	0\\
90.86	0\\
90.87	0\\
90.88	0\\
90.89	0\\
90.9	0\\
90.91	0\\
90.92	0\\
90.93	0\\
90.94	0\\
90.95	0\\
90.96	0\\
90.97	0\\
90.98	0\\
90.99	0\\
91	0\\
91.01	0\\
91.02	0\\
91.03	0\\
91.04	0\\
91.05	0\\
91.06	0\\
91.07	0\\
91.08	0\\
91.09	0\\
91.1	0\\
91.11	0\\
91.12	0\\
91.13	0\\
91.14	0\\
91.15	0\\
91.16	0\\
91.17	0\\
91.18	0\\
91.19	0\\
91.2	0\\
91.21	0\\
91.22	0\\
91.23	0\\
91.24	0\\
91.25	0\\
91.26	0\\
91.27	0\\
91.28	0\\
91.29	0\\
91.3	0\\
91.31	0\\
91.32	0\\
91.33	0\\
91.34	0\\
91.35	0\\
91.36	0\\
91.37	0\\
91.38	0\\
91.39	0\\
91.4	0\\
91.41	0\\
91.42	0\\
91.43	0\\
91.44	0\\
91.45	0\\
91.46	0\\
91.47	0\\
91.48	0\\
91.49	0\\
91.5	0\\
91.51	0\\
91.52	0\\
91.53	0\\
91.54	0\\
91.55	0\\
91.56	0\\
91.57	0\\
91.58	0\\
91.59	0\\
91.6	0\\
91.61	0\\
91.62	0\\
91.63	0\\
91.64	0\\
91.65	0\\
91.66	0\\
91.67	0\\
91.68	0\\
91.69	0\\
91.7	0\\
91.71	0\\
91.72	0\\
91.73	0\\
91.74	0\\
91.75	0\\
91.76	0\\
91.77	0\\
91.78	0\\
91.79	0\\
91.8	0\\
91.81	0\\
91.82	0\\
91.83	0\\
91.84	0\\
91.85	0\\
91.86	0\\
91.87	0\\
91.88	0\\
91.89	0\\
91.9	0\\
91.91	0\\
91.92	0\\
91.93	0\\
91.94	0\\
91.95	0\\
91.96	0\\
91.97	0\\
91.98	0\\
91.99	0\\
92	0\\
92.01	0\\
92.02	0\\
92.03	0\\
92.04	0\\
92.05	0\\
92.06	0\\
92.07	0\\
92.08	0\\
92.09	0\\
92.1	0\\
92.11	0\\
92.12	0\\
92.13	0\\
92.14	0\\
92.15	0\\
92.16	0\\
92.17	0\\
92.18	0\\
92.19	0\\
92.2	0\\
92.21	0\\
92.22	0\\
92.23	0\\
92.24	0\\
92.25	0\\
92.26	0\\
92.27	0\\
92.28	0\\
92.29	0\\
92.3	0\\
92.31	0\\
92.32	0\\
92.33	0\\
92.34	0\\
92.35	0\\
92.36	0\\
92.37	0\\
92.38	0\\
92.39	0\\
92.4	0\\
92.41	0\\
92.42	0\\
92.43	0\\
92.44	0\\
92.45	0\\
92.46	0\\
92.47	0\\
92.48	0\\
92.49	0\\
92.5	0\\
92.51	0\\
92.52	0\\
92.53	0\\
92.54	0\\
92.55	0\\
92.56	0\\
92.57	0\\
92.58	0\\
92.59	0\\
92.6	0\\
92.61	0\\
92.62	0\\
92.63	0\\
92.64	0\\
92.65	0\\
92.66	0\\
92.67	0\\
92.68	0\\
92.69	0\\
92.7	0\\
92.71	0\\
92.72	0\\
92.73	0\\
92.74	0\\
92.75	0\\
92.76	0\\
92.77	0\\
92.78	0\\
92.79	0\\
92.8	0\\
92.81	0\\
92.82	0\\
92.83	0\\
92.84	0\\
92.85	0\\
92.86	0\\
92.87	0\\
92.88	0\\
92.89	0\\
92.9	0\\
92.91	0\\
92.92	0\\
92.93	0\\
92.94	0\\
92.95	0\\
92.96	0\\
92.97	0\\
92.98	0\\
92.99	0\\
93	0\\
93.01	0\\
93.02	0\\
93.03	0\\
93.04	0\\
93.05	0\\
93.06	0\\
93.07	0\\
93.08	0\\
93.09	0\\
93.1	0\\
93.11	0\\
93.12	0\\
93.13	0\\
93.14	0\\
93.15	0\\
93.16	0\\
93.17	0\\
93.18	0\\
93.19	0\\
93.2	0\\
93.21	0\\
93.22	0\\
93.23	0\\
93.24	0\\
93.25	0\\
93.26	0\\
93.27	0\\
93.28	0\\
93.29	0\\
93.3	0\\
93.31	0\\
93.32	0\\
93.33	0\\
93.34	0\\
93.35	0\\
93.36	0\\
93.37	0\\
93.38	0\\
93.39	0\\
93.4	0\\
93.41	0\\
93.42	0\\
93.43	0\\
93.44	0\\
93.45	0\\
93.46	0\\
93.47	0\\
93.48	0\\
93.49	0\\
93.5	0\\
93.51	0\\
93.52	0\\
93.53	0\\
93.54	0\\
93.55	0\\
93.56	0\\
93.57	0\\
93.58	0\\
93.59	0\\
93.6	0\\
93.61	0\\
93.62	0\\
93.63	0\\
93.64	0\\
93.65	0\\
93.66	0\\
93.67	0\\
93.68	0\\
93.69	0\\
93.7	0\\
93.71	0\\
93.72	0\\
93.73	0\\
93.74	0\\
93.75	0\\
93.76	0\\
93.77	0\\
93.78	0\\
93.79	0\\
93.8	0\\
93.81	0\\
93.82	0\\
93.83	0\\
93.84	0\\
93.85	0\\
93.86	0\\
93.87	0\\
93.88	0\\
93.89	0\\
93.9	0\\
93.91	0\\
93.92	0\\
93.93	0\\
93.94	0\\
93.95	0\\
93.96	0\\
93.97	0\\
93.98	0\\
93.99	0\\
94	0\\
94.01	0\\
94.02	0\\
94.03	0\\
94.04	0\\
94.05	0\\
94.06	0\\
94.07	0\\
94.08	0\\
94.09	0\\
94.1	0\\
94.11	0\\
94.12	0\\
94.13	0\\
94.14	0\\
94.15	0\\
94.16	0\\
94.17	0\\
94.18	0\\
94.19	0\\
94.2	0\\
94.21	0\\
94.22	0\\
94.23	0\\
94.24	0\\
94.25	0\\
94.26	0\\
94.27	0\\
94.28	0\\
94.29	0\\
94.3	0\\
94.31	0\\
94.32	0\\
94.33	0\\
94.34	0\\
94.35	0\\
94.36	0\\
94.37	0\\
94.38	0\\
94.39	0\\
94.4	0\\
94.41	0\\
94.42	0\\
94.43	0\\
94.44	0\\
94.45	0\\
94.46	0\\
94.47	0\\
94.48	0\\
94.49	0\\
94.5	0\\
94.51	0\\
94.52	0\\
94.53	0\\
94.54	0\\
94.55	0\\
94.56	0\\
94.57	0\\
94.58	0\\
94.59	0\\
94.6	0\\
94.61	0\\
94.62	0\\
94.63	0\\
94.64	0\\
94.65	0\\
94.66	0\\
94.67	0\\
94.68	0\\
94.69	0\\
94.7	0\\
94.71	0\\
94.72	0\\
94.73	0\\
94.74	0\\
94.75	0\\
94.76	0\\
94.77	0\\
94.78	0\\
94.79	0\\
94.8	0\\
94.81	0\\
94.82	0\\
94.83	0\\
94.84	0\\
94.85	0\\
94.86	0\\
94.87	0\\
94.88	0\\
94.89	0\\
94.9	0\\
94.91	0\\
94.92	0\\
94.93	0\\
94.94	0\\
94.95	0\\
94.96	0\\
94.97	0\\
94.98	0\\
94.99	0\\
95	0\\
95.01	0\\
95.02	0\\
95.03	0\\
95.04	0\\
95.05	0\\
95.06	0\\
95.07	0\\
95.08	0\\
95.09	0\\
95.1	0\\
95.11	0\\
95.12	0\\
95.13	0\\
95.14	0\\
95.15	0\\
95.16	0\\
95.17	0\\
95.18	0\\
95.19	0\\
95.2	0\\
95.21	0\\
95.22	0\\
95.23	0\\
95.24	0\\
95.25	0\\
95.26	0\\
95.27	0\\
95.28	0\\
95.29	0\\
95.3	0\\
95.31	0\\
95.32	0\\
95.33	0\\
95.34	0\\
95.35	0\\
95.36	0\\
95.37	0\\
95.38	0\\
95.39	0\\
95.4	0\\
95.41	0\\
95.42	0\\
95.43	0\\
95.44	0\\
95.45	0\\
95.46	0\\
95.47	0\\
95.48	0\\
95.49	0\\
95.5	0\\
95.51	0\\
95.52	0\\
95.53	0\\
95.54	0\\
95.55	0\\
95.56	0\\
95.57	0\\
95.58	0\\
95.59	0\\
95.6	0\\
95.61	0\\
95.62	0\\
95.63	0\\
95.64	0\\
95.65	0\\
95.66	0\\
95.67	0\\
95.68	0\\
95.69	0\\
95.7	0\\
95.71	0\\
95.72	0\\
95.73	0\\
95.74	0\\
95.75	0\\
95.76	0\\
95.77	0\\
95.78	0\\
95.79	0\\
95.8	0\\
95.81	0\\
95.82	0\\
95.83	0\\
95.84	0\\
95.85	0\\
95.86	0\\
95.87	0\\
95.88	0\\
95.89	0\\
95.9	0\\
95.91	0\\
95.92	0\\
95.93	0\\
95.94	0\\
95.95	0\\
95.96	0\\
95.97	0\\
95.98	0\\
95.99	0\\
96	0\\
96.01	0\\
96.02	0\\
96.03	0\\
96.04	0\\
96.05	0\\
96.06	0\\
96.07	0\\
96.08	0\\
96.09	0\\
96.1	0\\
96.11	0\\
96.12	0\\
96.13	0\\
96.14	0\\
96.15	0\\
96.16	0\\
96.17	0\\
96.18	0\\
96.19	0\\
96.2	0\\
96.21	0\\
96.22	0\\
96.23	0\\
96.24	0\\
96.25	0\\
96.26	0\\
96.27	0\\
96.28	0\\
96.29	0\\
96.3	0\\
96.31	0\\
96.32	0\\
96.33	0\\
96.34	0\\
96.35	0\\
96.36	0\\
96.37	0\\
96.38	0\\
96.39	0\\
96.4	0\\
96.41	0\\
96.42	0\\
96.43	0\\
96.44	0\\
96.45	0\\
96.46	0\\
96.47	0\\
96.48	0\\
96.49	0\\
96.5	0\\
96.51	0\\
96.52	0\\
96.53	0\\
96.54	0\\
96.55	0\\
96.56	0\\
96.57	0\\
96.58	0\\
96.59	0\\
96.6	0\\
96.61	0\\
96.62	0\\
96.63	0\\
96.64	0\\
96.65	0\\
96.66	0\\
96.67	0\\
96.68	0\\
96.69	0\\
96.7	0\\
96.71	0\\
96.72	0\\
96.73	0\\
96.74	0\\
96.75	0\\
96.76	0\\
96.77	0\\
96.78	0\\
96.79	0\\
96.8	0\\
96.81	0\\
96.82	0\\
96.83	0\\
96.84	0\\
96.85	0\\
96.86	0\\
96.87	0\\
96.88	0\\
96.89	0\\
96.9	0\\
96.91	0\\
96.92	0\\
96.93	0\\
96.94	0\\
96.95	0\\
96.96	0\\
96.97	0\\
96.98	0\\
96.99	0\\
97	0\\
97.01	0\\
97.02	0\\
97.03	0\\
97.04	0\\
97.05	0\\
97.06	0\\
97.07	0\\
97.08	0\\
97.09	0\\
97.1	0\\
97.11	0\\
97.12	0\\
97.13	0\\
97.14	0\\
97.15	0\\
97.16	0\\
97.17	0\\
97.18	0\\
97.19	0\\
97.2	0\\
97.21	0\\
97.22	0\\
97.23	0\\
97.24	0\\
97.25	0\\
97.26	0\\
97.27	0\\
97.28	0\\
97.29	0\\
97.3	0\\
97.31	0\\
97.32	0\\
97.33	0\\
97.34	0\\
97.35	0\\
97.36	0\\
97.37	0\\
97.38	0\\
97.39	0\\
97.4	0\\
97.41	0\\
97.42	0\\
97.43	0\\
97.44	0\\
97.45	0\\
97.46	0\\
97.47	0\\
97.48	0\\
97.49	0\\
97.5	0\\
97.51	0\\
97.52	0\\
97.53	0\\
97.54	0\\
97.55	0\\
97.56	0\\
97.57	0\\
97.58	0\\
97.59	0\\
97.6	0\\
97.61	0\\
97.62	0\\
97.63	0\\
97.64	0\\
97.65	0\\
97.66	0\\
97.67	0\\
97.68	0\\
97.69	0\\
97.7	0\\
97.71	0\\
97.72	0\\
97.73	0\\
97.74	0\\
97.75	0\\
97.76	0\\
97.77	0\\
97.78	0\\
97.79	0\\
97.8	0\\
97.81	0\\
97.82	0\\
97.83	0\\
97.84	0\\
97.85	0\\
97.86	0\\
97.87	0\\
97.88	0\\
97.89	0\\
97.9	0\\
97.91	0\\
97.92	0\\
97.93	0\\
97.94	0\\
97.95	0\\
97.96	0\\
97.97	0\\
97.98	0\\
97.99	0\\
98	0\\
98.01	0\\
98.02	0\\
98.03	0\\
98.04	0\\
98.05	0\\
98.06	0\\
98.07	0\\
98.08	0\\
98.09	0\\
98.1	0\\
98.11	0\\
98.12	0\\
98.13	0\\
98.14	0\\
98.15	0\\
98.16	0\\
98.17	0\\
98.18	0\\
98.19	0\\
98.2	0\\
98.21	0\\
98.22	0\\
98.23	0\\
98.24	0\\
98.25	0\\
98.26	0\\
98.27	0\\
98.28	0\\
98.29	0\\
98.3	0\\
98.31	0\\
98.32	0\\
98.33	0\\
98.34	0\\
98.35	0\\
98.36	0\\
98.37	0\\
98.38	0\\
98.39	0\\
98.4	0\\
98.41	0\\
98.42	0\\
98.43	0\\
98.44	0\\
98.45	0\\
98.46	0\\
98.47	0\\
98.48	0\\
98.49	0\\
98.5	0\\
98.51	0\\
98.52	0\\
98.53	0\\
98.54	0\\
98.55	0\\
98.56	0\\
98.57	0\\
98.58	0\\
98.59	0\\
98.6	0\\
98.61	0\\
98.62	0\\
98.63	0\\
98.64	0\\
98.65	0\\
98.66	0\\
98.67	0\\
98.68	0\\
98.69	0\\
98.7	0\\
98.71	0\\
98.72	0\\
98.73	0\\
98.74	0\\
98.75	0\\
98.76	0\\
98.77	0\\
98.78	0\\
98.79	0\\
98.8	0\\
98.81	0\\
98.82	0\\
98.83	0\\
98.84	0\\
98.85	0\\
98.86	0\\
98.87	0\\
98.88	0\\
98.89	0\\
98.9	0\\
98.91	0\\
98.92	0\\
98.93	0\\
98.94	0\\
98.95	0\\
98.96	0\\
98.97	0\\
98.98	0\\
98.99	0\\
99	0\\
99.01	0\\
99.02	0\\
99.03	0\\
99.04	0\\
99.05	0\\
99.06	0\\
99.07	0\\
99.08	0\\
99.09	0\\
99.1	0\\
99.11	0\\
99.12	0\\
99.13	0\\
99.14	0\\
99.15	0\\
99.16	0\\
99.17	0\\
99.18	0\\
99.19	0\\
99.2	0\\
99.21	0\\
99.22	0\\
99.23	0\\
99.24	0\\
99.25	0\\
99.26	0\\
99.27	0\\
99.28	0\\
99.29	0\\
99.3	0\\
99.31	0\\
99.32	0\\
99.33	0\\
99.34	0\\
99.35	0\\
99.36	0\\
99.37	0\\
99.38	0\\
99.39	0\\
99.4	0\\
99.41	0\\
99.42	0\\
99.43	0\\
99.44	0\\
99.45	0\\
99.46	0\\
99.47	0\\
99.48	0\\
99.49	0\\
99.5	0\\
99.51	0\\
99.52	0\\
99.53	0\\
99.54	0\\
99.55	0\\
99.56	0\\
99.57	0\\
99.58	0\\
99.59	0\\
99.6	0\\
99.61	0\\
99.62	0\\
99.63	0\\
99.64	0\\
99.65	0\\
99.66	0\\
99.67	0\\
99.68	0\\
99.69	0\\
99.7	0\\
99.71	0\\
99.72	0\\
99.73	0\\
99.74	0\\
99.75	0\\
99.76	0\\
99.77	0\\
99.78	0\\
99.79	0\\
99.8	0\\
99.81	0\\
99.82	0\\
99.83	0\\
99.84	0\\
99.85	0\\
99.86	0\\
99.87	0\\
99.88	0\\
99.89	0\\
99.9	0\\
99.91	0\\
99.92	0\\
99.93	0\\
99.94	0\\
99.95	0\\
99.96	0\\
99.97	0\\
99.98	0\\
99.99	0\\
100	0\\
};
\addlegendentry{$q=3$};

\addplot [color=green,solid,forget plot]
  table[row sep=crcr]{%
0.01	0\\
0.02	0\\
0.03	0\\
0.04	0\\
0.05	0\\
0.06	0\\
0.07	0\\
0.08	0\\
0.09	0\\
0.1	0\\
0.11	0\\
0.12	0\\
0.13	0\\
0.14	0\\
0.15	0\\
0.16	0\\
0.17	0\\
0.18	0\\
0.19	0\\
0.2	0\\
0.21	0\\
0.22	0\\
0.23	0\\
0.24	0\\
0.25	0\\
0.26	0\\
0.27	0\\
0.28	0\\
0.29	0\\
0.3	0\\
0.31	0\\
0.32	0\\
0.33	0\\
0.34	0\\
0.35	0\\
0.36	0\\
0.37	0\\
0.38	0\\
0.39	0\\
0.4	0\\
0.41	0\\
0.42	0\\
0.43	0\\
0.44	0\\
0.45	0\\
0.46	0\\
0.47	0\\
0.48	0\\
0.49	0\\
0.5	0\\
0.51	0\\
0.52	0\\
0.53	0\\
0.54	0\\
0.55	0\\
0.56	0\\
0.57	0\\
0.58	0\\
0.59	0\\
0.6	0\\
0.61	0\\
0.62	0\\
0.63	0\\
0.64	0\\
0.65	0\\
0.66	0\\
0.67	0\\
0.68	0\\
0.69	0\\
0.7	0\\
0.71	0\\
0.72	0\\
0.73	0\\
0.74	0\\
0.75	0\\
0.76	0\\
0.77	0\\
0.78	0\\
0.79	0\\
0.8	0\\
0.81	0\\
0.82	0\\
0.83	0\\
0.84	0\\
0.85	0\\
0.86	0\\
0.87	0\\
0.88	0\\
0.89	0\\
0.9	0\\
0.91	0\\
0.92	0\\
0.93	0\\
0.94	0\\
0.95	0\\
0.96	0\\
0.97	0\\
0.98	0\\
0.99	0\\
1	0\\
1.01	0\\
1.02	0\\
1.03	0\\
1.04	0\\
1.05	0\\
1.06	0\\
1.07	0\\
1.08	0\\
1.09	0\\
1.1	0\\
1.11	0\\
1.12	0\\
1.13	0\\
1.14	0\\
1.15	0\\
1.16	0\\
1.17	0\\
1.18	0\\
1.19	0\\
1.2	0\\
1.21	0\\
1.22	0\\
1.23	0\\
1.24	0\\
1.25	0\\
1.26	0\\
1.27	0\\
1.28	0\\
1.29	0\\
1.3	0\\
1.31	0\\
1.32	0\\
1.33	0\\
1.34	0\\
1.35	0\\
1.36	0\\
1.37	0\\
1.38	0\\
1.39	0\\
1.4	0\\
1.41	0\\
1.42	0\\
1.43	0\\
1.44	0\\
1.45	0\\
1.46	0\\
1.47	0\\
1.48	0\\
1.49	0\\
1.5	0\\
1.51	0\\
1.52	0\\
1.53	0\\
1.54	0\\
1.55	0\\
1.56	0\\
1.57	0\\
1.58	0\\
1.59	0\\
1.6	0\\
1.61	0\\
1.62	0\\
1.63	0\\
1.64	0\\
1.65	0\\
1.66	0\\
1.67	0\\
1.68	0\\
1.69	0\\
1.7	0\\
1.71	0\\
1.72	0\\
1.73	0\\
1.74	0\\
1.75	0\\
1.76	0\\
1.77	0\\
1.78	0\\
1.79	0\\
1.8	0\\
1.81	0\\
1.82	0\\
1.83	0\\
1.84	0\\
1.85	0\\
1.86	0\\
1.87	0\\
1.88	0\\
1.89	0\\
1.9	0\\
1.91	0\\
1.92	0\\
1.93	0\\
1.94	0\\
1.95	0\\
1.96	0\\
1.97	0\\
1.98	0\\
1.99	0\\
2	0\\
2.01	0\\
2.02	0\\
2.03	0\\
2.04	0\\
2.05	0\\
2.06	0\\
2.07	0\\
2.08	0\\
2.09	0\\
2.1	0\\
2.11	0\\
2.12	0\\
2.13	0\\
2.14	0\\
2.15	0\\
2.16	0\\
2.17	0\\
2.18	0\\
2.19	0\\
2.2	0\\
2.21	0\\
2.22	0\\
2.23	0\\
2.24	0\\
2.25	0\\
2.26	0\\
2.27	0\\
2.28	0\\
2.29	0\\
2.3	0\\
2.31	0\\
2.32	0\\
2.33	0\\
2.34	0\\
2.35	0\\
2.36	0\\
2.37	0\\
2.38	0\\
2.39	0\\
2.4	0\\
2.41	0\\
2.42	0\\
2.43	0\\
2.44	0\\
2.45	0\\
2.46	0\\
2.47	0\\
2.48	0\\
2.49	0\\
2.5	0\\
2.51	0\\
2.52	0\\
2.53	0\\
2.54	0\\
2.55	0\\
2.56	0\\
2.57	0\\
2.58	0\\
2.59	0\\
2.6	0\\
2.61	0\\
2.62	0\\
2.63	0\\
2.64	0\\
2.65	0\\
2.66	0\\
2.67	0\\
2.68	0\\
2.69	0\\
2.7	0\\
2.71	0\\
2.72	0\\
2.73	0\\
2.74	0\\
2.75	0\\
2.76	0\\
2.77	0\\
2.78	0\\
2.79	0\\
2.8	0\\
2.81	0\\
2.82	0\\
2.83	0\\
2.84	0\\
2.85	0\\
2.86	0\\
2.87	0\\
2.88	0\\
2.89	0\\
2.9	0\\
2.91	0\\
2.92	0\\
2.93	0\\
2.94	0\\
2.95	0\\
2.96	0\\
2.97	0\\
2.98	0\\
2.99	0\\
3	0\\
3.01	0\\
3.02	0\\
3.03	0\\
3.04	0\\
3.05	0\\
3.06	0\\
3.07	0\\
3.08	0\\
3.09	0\\
3.1	0\\
3.11	0\\
3.12	0\\
3.13	0\\
3.14	0\\
3.15	0\\
3.16	0\\
3.17	0\\
3.18	0\\
3.19	0\\
3.2	0\\
3.21	0\\
3.22	0\\
3.23	0\\
3.24	0\\
3.25	0\\
3.26	0\\
3.27	0\\
3.28	0\\
3.29	0\\
3.3	0\\
3.31	0\\
3.32	0\\
3.33	0\\
3.34	0\\
3.35	0\\
3.36	0\\
3.37	0\\
3.38	0\\
3.39	0\\
3.4	0\\
3.41	0\\
3.42	0\\
3.43	0\\
3.44	0\\
3.45	0\\
3.46	0\\
3.47	0\\
3.48	0\\
3.49	0\\
3.5	0\\
3.51	0\\
3.52	0\\
3.53	0\\
3.54	0\\
3.55	0\\
3.56	0\\
3.57	0\\
3.58	0\\
3.59	0\\
3.6	0\\
3.61	0\\
3.62	0\\
3.63	0\\
3.64	0\\
3.65	0\\
3.66	0\\
3.67	0\\
3.68	0\\
3.69	0\\
3.7	0\\
3.71	0\\
3.72	0\\
3.73	0\\
3.74	0\\
3.75	0\\
3.76	0\\
3.77	0\\
3.78	0\\
3.79	0\\
3.8	0\\
3.81	0\\
3.82	0\\
3.83	0\\
3.84	0\\
3.85	0\\
3.86	0\\
3.87	0\\
3.88	0\\
3.89	0\\
3.9	0\\
3.91	0\\
3.92	0\\
3.93	0\\
3.94	0\\
3.95	0\\
3.96	0\\
3.97	0\\
3.98	0\\
3.99	0\\
4	0\\
4.01	0\\
4.02	0\\
4.03	0\\
4.04	0\\
4.05	0\\
4.06	0\\
4.07	0\\
4.08	0\\
4.09	0\\
4.1	0\\
4.11	0\\
4.12	0\\
4.13	0\\
4.14	0\\
4.15	0\\
4.16	0\\
4.17	0\\
4.18	0\\
4.19	0\\
4.2	0\\
4.21	0\\
4.22	0\\
4.23	0\\
4.24	0\\
4.25	0\\
4.26	0\\
4.27	0\\
4.28	0\\
4.29	0\\
4.3	0\\
4.31	0\\
4.32	0\\
4.33	0\\
4.34	0\\
4.35	0\\
4.36	0\\
4.37	0\\
4.38	0\\
4.39	0\\
4.4	0\\
4.41	0\\
4.42	0\\
4.43	0\\
4.44	0\\
4.45	0\\
4.46	0\\
4.47	0\\
4.48	0\\
4.49	0\\
4.5	0\\
4.51	0\\
4.52	0\\
4.53	0\\
4.54	0\\
4.55	0\\
4.56	0\\
4.57	0\\
4.58	0\\
4.59	0\\
4.6	0\\
4.61	0\\
4.62	0\\
4.63	0\\
4.64	0\\
4.65	0\\
4.66	0\\
4.67	0\\
4.68	0\\
4.69	0\\
4.7	0\\
4.71	0\\
4.72	0\\
4.73	0\\
4.74	0\\
4.75	0\\
4.76	0\\
4.77	0\\
4.78	0\\
4.79	0\\
4.8	0\\
4.81	0\\
4.82	0\\
4.83	0\\
4.84	0\\
4.85	0\\
4.86	0\\
4.87	0\\
4.88	0\\
4.89	0\\
4.9	0\\
4.91	0\\
4.92	0\\
4.93	0\\
4.94	0\\
4.95	0\\
4.96	0\\
4.97	0\\
4.98	0\\
4.99	0\\
5	0\\
5.01	0\\
5.02	0\\
5.03	0\\
5.04	0\\
5.05	0\\
5.06	0\\
5.07	0\\
5.08	0\\
5.09	0\\
5.1	0\\
5.11	0\\
5.12	0\\
5.13	0\\
5.14	0\\
5.15	0\\
5.16	0\\
5.17	0\\
5.18	0\\
5.19	0\\
5.2	0\\
5.21	0\\
5.22	0\\
5.23	0\\
5.24	0\\
5.25	0\\
5.26	0\\
5.27	0\\
5.28	0\\
5.29	0\\
5.3	0\\
5.31	0\\
5.32	0\\
5.33	0\\
5.34	0\\
5.35	0\\
5.36	0\\
5.37	0\\
5.38	0\\
5.39	0\\
5.4	0\\
5.41	0\\
5.42	0\\
5.43	0\\
5.44	0\\
5.45	0\\
5.46	0\\
5.47	0\\
5.48	0\\
5.49	0\\
5.5	0\\
5.51	0\\
5.52	0\\
5.53	0\\
5.54	0\\
5.55	0\\
5.56	0\\
5.57	0\\
5.58	0\\
5.59	0\\
5.6	0\\
5.61	0\\
5.62	0\\
5.63	0\\
5.64	0\\
5.65	0\\
5.66	0\\
5.67	0\\
5.68	0\\
5.69	0\\
5.7	0\\
5.71	0\\
5.72	0\\
5.73	0\\
5.74	0\\
5.75	0\\
5.76	0\\
5.77	0\\
5.78	0\\
5.79	0\\
5.8	0\\
5.81	0\\
5.82	0\\
5.83	0\\
5.84	0\\
5.85	0\\
5.86	0\\
5.87	0\\
5.88	0\\
5.89	0\\
5.9	0\\
5.91	0\\
5.92	0\\
5.93	0\\
5.94	0\\
5.95	0\\
5.96	0\\
5.97	0\\
5.98	0\\
5.99	0\\
6	0\\
6.01	0\\
6.02	0\\
6.03	0\\
6.04	0\\
6.05	0\\
6.06	0\\
6.07	0\\
6.08	0\\
6.09	0\\
6.1	0\\
6.11	0\\
6.12	0\\
6.13	0\\
6.14	0\\
6.15	0\\
6.16	0\\
6.17	0\\
6.18	0\\
6.19	0\\
6.2	0\\
6.21	0\\
6.22	0\\
6.23	0\\
6.24	0\\
6.25	0\\
6.26	0\\
6.27	0\\
6.28	0\\
6.29	0\\
6.3	0\\
6.31	0\\
6.32	0\\
6.33	0\\
6.34	0\\
6.35	0\\
6.36	0\\
6.37	0\\
6.38	0\\
6.39	0\\
6.4	0\\
6.41	0\\
6.42	0\\
6.43	0\\
6.44	0\\
6.45	0\\
6.46	0\\
6.47	0\\
6.48	0\\
6.49	0\\
6.5	0\\
6.51	0\\
6.52	0\\
6.53	0\\
6.54	0\\
6.55	0\\
6.56	0\\
6.57	0\\
6.58	0\\
6.59	0\\
6.6	0\\
6.61	0\\
6.62	0\\
6.63	0\\
6.64	0\\
6.65	0\\
6.66	0\\
6.67	0\\
6.68	0\\
6.69	0\\
6.7	0\\
6.71	0\\
6.72	0\\
6.73	0\\
6.74	0\\
6.75	0\\
6.76	0\\
6.77	0\\
6.78	0\\
6.79	0\\
6.8	0\\
6.81	0\\
6.82	0\\
6.83	0\\
6.84	0\\
6.85	0\\
6.86	0\\
6.87	0\\
6.88	0\\
6.89	0\\
6.9	0\\
6.91	0\\
6.92	0\\
6.93	0\\
6.94	0\\
6.95	0\\
6.96	0\\
6.97	0\\
6.98	0\\
6.99	0\\
7	0\\
7.01	0\\
7.02	0\\
7.03	0\\
7.04	0\\
7.05	0\\
7.06	0\\
7.07	0\\
7.08	0\\
7.09	0\\
7.1	0\\
7.11	0\\
7.12	0\\
7.13	0\\
7.14	0\\
7.15	0\\
7.16	0\\
7.17	0\\
7.18	0\\
7.19	0\\
7.2	0\\
7.21	0\\
7.22	0\\
7.23	0\\
7.24	0\\
7.25	0\\
7.26	0\\
7.27	0\\
7.28	0\\
7.29	0\\
7.3	0\\
7.31	0\\
7.32	0\\
7.33	0\\
7.34	0\\
7.35	0\\
7.36	0\\
7.37	0\\
7.38	0\\
7.39	0\\
7.4	0\\
7.41	0\\
7.42	0\\
7.43	0\\
7.44	0\\
7.45	0\\
7.46	0\\
7.47	0\\
7.48	0\\
7.49	0\\
7.5	0\\
7.51	0\\
7.52	0\\
7.53	0\\
7.54	0\\
7.55	0\\
7.56	0\\
7.57	0\\
7.58	0\\
7.59	0\\
7.6	0\\
7.61	0\\
7.62	0\\
7.63	0\\
7.64	0\\
7.65	0\\
7.66	0\\
7.67	0\\
7.68	0\\
7.69	0\\
7.7	0\\
7.71	0\\
7.72	0\\
7.73	0\\
7.74	0\\
7.75	0\\
7.76	0\\
7.77	0\\
7.78	0\\
7.79	0\\
7.8	0\\
7.81	0\\
7.82	0\\
7.83	0\\
7.84	0\\
7.85	0\\
7.86	0\\
7.87	0\\
7.88	0\\
7.89	0\\
7.9	0\\
7.91	0\\
7.92	0\\
7.93	0\\
7.94	0\\
7.95	0\\
7.96	0\\
7.97	0\\
7.98	0\\
7.99	0\\
8	0\\
8.01	0\\
8.02	0\\
8.03	0\\
8.04	0\\
8.05	0\\
8.06	0\\
8.07	0\\
8.08	0\\
8.09	0\\
8.1	0\\
8.11	0\\
8.12	0\\
8.13	0\\
8.14	0\\
8.15	0\\
8.16	0\\
8.17	0\\
8.18	0\\
8.19	0\\
8.2	0\\
8.21	0\\
8.22	0\\
8.23	0\\
8.24	0\\
8.25	0\\
8.26	0\\
8.27	0\\
8.28	0\\
8.29	0\\
8.3	0\\
8.31	0\\
8.32	0\\
8.33	0\\
8.34	0\\
8.35	0\\
8.36	0\\
8.37	0\\
8.38	0\\
8.39	0\\
8.4	0\\
8.41	0\\
8.42	0\\
8.43	0\\
8.44	0\\
8.45	0\\
8.46	0\\
8.47	0\\
8.48	0\\
8.49	0\\
8.5	0\\
8.51	0\\
8.52	0\\
8.53	0\\
8.54	0\\
8.55	0\\
8.56	0\\
8.57	0\\
8.58	0\\
8.59	0\\
8.6	0\\
8.61	0\\
8.62	0\\
8.63	0\\
8.64	0\\
8.65	0\\
8.66	0\\
8.67	0\\
8.68	0\\
8.69	0\\
8.7	0\\
8.71	0\\
8.72	0\\
8.73	0\\
8.74	0\\
8.75	0\\
8.76	0\\
8.77	0\\
8.78	0\\
8.79	0\\
8.8	0\\
8.81	0\\
8.82	0\\
8.83	0\\
8.84	0\\
8.85	0\\
8.86	0\\
8.87	0\\
8.88	0\\
8.89	0\\
8.9	0\\
8.91	0\\
8.92	0\\
8.93	0\\
8.94	0\\
8.95	0\\
8.96	0\\
8.97	0\\
8.98	0\\
8.99	0\\
9	0\\
9.01	0\\
9.02	0\\
9.03	0\\
9.04	0\\
9.05	0\\
9.06	0\\
9.07	0\\
9.08	0\\
9.09	0\\
9.1	0\\
9.11	0\\
9.12	0\\
9.13	0\\
9.14	0\\
9.15	0\\
9.16	0\\
9.17	0\\
9.18	0\\
9.19	0\\
9.2	0\\
9.21	0\\
9.22	0\\
9.23	0\\
9.24	0\\
9.25	0\\
9.26	0\\
9.27	0\\
9.28	0\\
9.29	0\\
9.3	0\\
9.31	0\\
9.32	0\\
9.33	0\\
9.34	0\\
9.35	0\\
9.36	0\\
9.37	0\\
9.38	0\\
9.39	0\\
9.4	0\\
9.41	0\\
9.42	0\\
9.43	0\\
9.44	0\\
9.45	0\\
9.46	0\\
9.47	0\\
9.48	0\\
9.49	0\\
9.5	0\\
9.51	0\\
9.52	0\\
9.53	0\\
9.54	0\\
9.55	0\\
9.56	0\\
9.57	0\\
9.58	0\\
9.59	0\\
9.6	0\\
9.61	0\\
9.62	0\\
9.63	0\\
9.64	0\\
9.65	0\\
9.66	0\\
9.67	0\\
9.68	0\\
9.69	0\\
9.7	0\\
9.71	0\\
9.72	0\\
9.73	0\\
9.74	0\\
9.75	0\\
9.76	0\\
9.77	0\\
9.78	0\\
9.79	0\\
9.8	0\\
9.81	0\\
9.82	0\\
9.83	0\\
9.84	0\\
9.85	0\\
9.86	0\\
9.87	0\\
9.88	0\\
9.89	0\\
9.9	0\\
9.91	0\\
9.92	0\\
9.93	0\\
9.94	0\\
9.95	0\\
9.96	0\\
9.97	0\\
9.98	0\\
9.99	0\\
10	0\\
10.01	0\\
10.02	0\\
10.03	0\\
10.04	0\\
10.05	0\\
10.06	0\\
10.07	0\\
10.08	0\\
10.09	0\\
10.1	0\\
10.11	0\\
10.12	0\\
10.13	0\\
10.14	0\\
10.15	0\\
10.16	0\\
10.17	0\\
10.18	0\\
10.19	0\\
10.2	0\\
10.21	0\\
10.22	0\\
10.23	0\\
10.24	0\\
10.25	0\\
10.26	0\\
10.27	0\\
10.28	0\\
10.29	0\\
10.3	0\\
10.31	0\\
10.32	0\\
10.33	0\\
10.34	0\\
10.35	0\\
10.36	0\\
10.37	0\\
10.38	0\\
10.39	0\\
10.4	0\\
10.41	0\\
10.42	0\\
10.43	0\\
10.44	0\\
10.45	0\\
10.46	0\\
10.47	0\\
10.48	0\\
10.49	0\\
10.5	0\\
10.51	0\\
10.52	0\\
10.53	0\\
10.54	0\\
10.55	0\\
10.56	0\\
10.57	0\\
10.58	0\\
10.59	0\\
10.6	0\\
10.61	0\\
10.62	0\\
10.63	0\\
10.64	0\\
10.65	0\\
10.66	0\\
10.67	0\\
10.68	0\\
10.69	0\\
10.7	0\\
10.71	0\\
10.72	0\\
10.73	0\\
10.74	0\\
10.75	0\\
10.76	0\\
10.77	0\\
10.78	0\\
10.79	0\\
10.8	0\\
10.81	0\\
10.82	0\\
10.83	0\\
10.84	0\\
10.85	0\\
10.86	0\\
10.87	0\\
10.88	0\\
10.89	0\\
10.9	0\\
10.91	0\\
10.92	0\\
10.93	0\\
10.94	0\\
10.95	0\\
10.96	0\\
10.97	0\\
10.98	0\\
10.99	0\\
11	0\\
11.01	0\\
11.02	0\\
11.03	0\\
11.04	0\\
11.05	0\\
11.06	0\\
11.07	0\\
11.08	0\\
11.09	0\\
11.1	0\\
11.11	0\\
11.12	0\\
11.13	0\\
11.14	0\\
11.15	0\\
11.16	0\\
11.17	0\\
11.18	0\\
11.19	0\\
11.2	0\\
11.21	0\\
11.22	0\\
11.23	0\\
11.24	0\\
11.25	0\\
11.26	0\\
11.27	0\\
11.28	0\\
11.29	0\\
11.3	0\\
11.31	0\\
11.32	0\\
11.33	0\\
11.34	0\\
11.35	0\\
11.36	0\\
11.37	0\\
11.38	0\\
11.39	0\\
11.4	0\\
11.41	0\\
11.42	0\\
11.43	0\\
11.44	0\\
11.45	0\\
11.46	0\\
11.47	0\\
11.48	0\\
11.49	0\\
11.5	0\\
11.51	0\\
11.52	0\\
11.53	0\\
11.54	0\\
11.55	0\\
11.56	0\\
11.57	0\\
11.58	0\\
11.59	0\\
11.6	0\\
11.61	0\\
11.62	0\\
11.63	0\\
11.64	0\\
11.65	0\\
11.66	0\\
11.67	0\\
11.68	0\\
11.69	0\\
11.7	0\\
11.71	0\\
11.72	0\\
11.73	0\\
11.74	0\\
11.75	0\\
11.76	0\\
11.77	0\\
11.78	0\\
11.79	0\\
11.8	0\\
11.81	0\\
11.82	0\\
11.83	0\\
11.84	0\\
11.85	0\\
11.86	0\\
11.87	0\\
11.88	0\\
11.89	0\\
11.9	0\\
11.91	0\\
11.92	0\\
11.93	0\\
11.94	0\\
11.95	0\\
11.96	0\\
11.97	0\\
11.98	0\\
11.99	0\\
12	0\\
12.01	0\\
12.02	0\\
12.03	0\\
12.04	0\\
12.05	0\\
12.06	0\\
12.07	0\\
12.08	0\\
12.09	0\\
12.1	0\\
12.11	0\\
12.12	0\\
12.13	0\\
12.14	0\\
12.15	0\\
12.16	0\\
12.17	0\\
12.18	0\\
12.19	0\\
12.2	0\\
12.21	0\\
12.22	0\\
12.23	0\\
12.24	0\\
12.25	0\\
12.26	0\\
12.27	0\\
12.28	0\\
12.29	0\\
12.3	0\\
12.31	0\\
12.32	0\\
12.33	0\\
12.34	0\\
12.35	0\\
12.36	0\\
12.37	0\\
12.38	0\\
12.39	0\\
12.4	0\\
12.41	0\\
12.42	0\\
12.43	0\\
12.44	0\\
12.45	0\\
12.46	0\\
12.47	0\\
12.48	0\\
12.49	0\\
12.5	0\\
12.51	0\\
12.52	0\\
12.53	0\\
12.54	0\\
12.55	0\\
12.56	0\\
12.57	0\\
12.58	0\\
12.59	0\\
12.6	0\\
12.61	0\\
12.62	0\\
12.63	0\\
12.64	0\\
12.65	0\\
12.66	0\\
12.67	0\\
12.68	0\\
12.69	0\\
12.7	0\\
12.71	0\\
12.72	0\\
12.73	0\\
12.74	0\\
12.75	0\\
12.76	0\\
12.77	0\\
12.78	0\\
12.79	0\\
12.8	0\\
12.81	0\\
12.82	0\\
12.83	0\\
12.84	0\\
12.85	0\\
12.86	0\\
12.87	0\\
12.88	0\\
12.89	0\\
12.9	0\\
12.91	0\\
12.92	0\\
12.93	0\\
12.94	0\\
12.95	0\\
12.96	0\\
12.97	0\\
12.98	0\\
12.99	0\\
13	0\\
13.01	0\\
13.02	0\\
13.03	0\\
13.04	0\\
13.05	0\\
13.06	0\\
13.07	0\\
13.08	0\\
13.09	0\\
13.1	0\\
13.11	0\\
13.12	0\\
13.13	0\\
13.14	0\\
13.15	0\\
13.16	0\\
13.17	0\\
13.18	0\\
13.19	0\\
13.2	0\\
13.21	0\\
13.22	0\\
13.23	0\\
13.24	0\\
13.25	0\\
13.26	0\\
13.27	0\\
13.28	0\\
13.29	0\\
13.3	0\\
13.31	0\\
13.32	0\\
13.33	0\\
13.34	0\\
13.35	0\\
13.36	0\\
13.37	0\\
13.38	0\\
13.39	0\\
13.4	0\\
13.41	0\\
13.42	0\\
13.43	0\\
13.44	0\\
13.45	0\\
13.46	0\\
13.47	0\\
13.48	0\\
13.49	0\\
13.5	0\\
13.51	0\\
13.52	0\\
13.53	0\\
13.54	0\\
13.55	0\\
13.56	0\\
13.57	0\\
13.58	0\\
13.59	0\\
13.6	0\\
13.61	0\\
13.62	0\\
13.63	0\\
13.64	0\\
13.65	0\\
13.66	0\\
13.67	0\\
13.68	0\\
13.69	0\\
13.7	0\\
13.71	0\\
13.72	0\\
13.73	0\\
13.74	0\\
13.75	0\\
13.76	0\\
13.77	0\\
13.78	0\\
13.79	0\\
13.8	0\\
13.81	0\\
13.82	0\\
13.83	0\\
13.84	0\\
13.85	0\\
13.86	0\\
13.87	0\\
13.88	0\\
13.89	0\\
13.9	0\\
13.91	0\\
13.92	0\\
13.93	0\\
13.94	0\\
13.95	0\\
13.96	0\\
13.97	0\\
13.98	0\\
13.99	0\\
14	0\\
14.01	0\\
14.02	0\\
14.03	0\\
14.04	0\\
14.05	0\\
14.06	0\\
14.07	0\\
14.08	0\\
14.09	0\\
14.1	0\\
14.11	0\\
14.12	0\\
14.13	0\\
14.14	0\\
14.15	0\\
14.16	0\\
14.17	0\\
14.18	0\\
14.19	0\\
14.2	0\\
14.21	0\\
14.22	0\\
14.23	0\\
14.24	0\\
14.25	0\\
14.26	0\\
14.27	0\\
14.28	0\\
14.29	0\\
14.3	0\\
14.31	0\\
14.32	0\\
14.33	0\\
14.34	0\\
14.35	0\\
14.36	0\\
14.37	0\\
14.38	0\\
14.39	0\\
14.4	0\\
14.41	0\\
14.42	0\\
14.43	0\\
14.44	0\\
14.45	0\\
14.46	0\\
14.47	0\\
14.48	0\\
14.49	0\\
14.5	0\\
14.51	0\\
14.52	0\\
14.53	0\\
14.54	0\\
14.55	0\\
14.56	0\\
14.57	0\\
14.58	0\\
14.59	0\\
14.6	0\\
14.61	0\\
14.62	0\\
14.63	0\\
14.64	0\\
14.65	0\\
14.66	0\\
14.67	0\\
14.68	0\\
14.69	0\\
14.7	0\\
14.71	0\\
14.72	0\\
14.73	0\\
14.74	0\\
14.75	0\\
14.76	0\\
14.77	0\\
14.78	0\\
14.79	0\\
14.8	0\\
14.81	0\\
14.82	0\\
14.83	0\\
14.84	0\\
14.85	0\\
14.86	0\\
14.87	0\\
14.88	0\\
14.89	0\\
14.9	0\\
14.91	0\\
14.92	0\\
14.93	0\\
14.94	0\\
14.95	0\\
14.96	0\\
14.97	0\\
14.98	0\\
14.99	0\\
15	0\\
15.01	0\\
15.02	0\\
15.03	0\\
15.04	0\\
15.05	0\\
15.06	0\\
15.07	0\\
15.08	0\\
15.09	0\\
15.1	0\\
15.11	0\\
15.12	0\\
15.13	0\\
15.14	0\\
15.15	0\\
15.16	0\\
15.17	0\\
15.18	0\\
15.19	0\\
15.2	0\\
15.21	0\\
15.22	0\\
15.23	0\\
15.24	0\\
15.25	0\\
15.26	0\\
15.27	0\\
15.28	0\\
15.29	0\\
15.3	0\\
15.31	0\\
15.32	0\\
15.33	0\\
15.34	0\\
15.35	0\\
15.36	0\\
15.37	0\\
15.38	0\\
15.39	0\\
15.4	0\\
15.41	0\\
15.42	0\\
15.43	0\\
15.44	0\\
15.45	0\\
15.46	0\\
15.47	0\\
15.48	0\\
15.49	0\\
15.5	0\\
15.51	0\\
15.52	0\\
15.53	0\\
15.54	0\\
15.55	0\\
15.56	0\\
15.57	0\\
15.58	0\\
15.59	0\\
15.6	0\\
15.61	0\\
15.62	0\\
15.63	0\\
15.64	0\\
15.65	0\\
15.66	0\\
15.67	0\\
15.68	0\\
15.69	0\\
15.7	0\\
15.71	0\\
15.72	0\\
15.73	0\\
15.74	0\\
15.75	0\\
15.76	0\\
15.77	0\\
15.78	0\\
15.79	0\\
15.8	0\\
15.81	0\\
15.82	0\\
15.83	0\\
15.84	0\\
15.85	0\\
15.86	0\\
15.87	0\\
15.88	0\\
15.89	0\\
15.9	0\\
15.91	0\\
15.92	0\\
15.93	0\\
15.94	0\\
15.95	0\\
15.96	0\\
15.97	0\\
15.98	0\\
15.99	0\\
16	0\\
16.01	0\\
16.02	0\\
16.03	0\\
16.04	0\\
16.05	0\\
16.06	0\\
16.07	0\\
16.08	0\\
16.09	0\\
16.1	0\\
16.11	0\\
16.12	0\\
16.13	0\\
16.14	0\\
16.15	0\\
16.16	0\\
16.17	0\\
16.18	0\\
16.19	0\\
16.2	0\\
16.21	0\\
16.22	0\\
16.23	0\\
16.24	0\\
16.25	0\\
16.26	0\\
16.27	0\\
16.28	0\\
16.29	0\\
16.3	0\\
16.31	0\\
16.32	0\\
16.33	0\\
16.34	0\\
16.35	0\\
16.36	0\\
16.37	0\\
16.38	0\\
16.39	0\\
16.4	0\\
16.41	0\\
16.42	0\\
16.43	0\\
16.44	0\\
16.45	0\\
16.46	0\\
16.47	0\\
16.48	0\\
16.49	0\\
16.5	0\\
16.51	0\\
16.52	0\\
16.53	0\\
16.54	0\\
16.55	0\\
16.56	0\\
16.57	0\\
16.58	0\\
16.59	0\\
16.6	0\\
16.61	0\\
16.62	0\\
16.63	0\\
16.64	0\\
16.65	0\\
16.66	0\\
16.67	0\\
16.68	0\\
16.69	0\\
16.7	0\\
16.71	0\\
16.72	0\\
16.73	0\\
16.74	0\\
16.75	0\\
16.76	0\\
16.77	0\\
16.78	0\\
16.79	0\\
16.8	0\\
16.81	0\\
16.82	0\\
16.83	0\\
16.84	0\\
16.85	0\\
16.86	0\\
16.87	0\\
16.88	0\\
16.89	0\\
16.9	0\\
16.91	0\\
16.92	0\\
16.93	0\\
16.94	0\\
16.95	0\\
16.96	0\\
16.97	0\\
16.98	0\\
16.99	0\\
17	0\\
17.01	0\\
17.02	0\\
17.03	0\\
17.04	0\\
17.05	0\\
17.06	0\\
17.07	0\\
17.08	0\\
17.09	0\\
17.1	0\\
17.11	0\\
17.12	0\\
17.13	0\\
17.14	0\\
17.15	0\\
17.16	0\\
17.17	0\\
17.18	0\\
17.19	0\\
17.2	0\\
17.21	0\\
17.22	0\\
17.23	0\\
17.24	0\\
17.25	0\\
17.26	0\\
17.27	0\\
17.28	0\\
17.29	0\\
17.3	0\\
17.31	0\\
17.32	0\\
17.33	0\\
17.34	0\\
17.35	0\\
17.36	0\\
17.37	0\\
17.38	0\\
17.39	0\\
17.4	0\\
17.41	0\\
17.42	0\\
17.43	0\\
17.44	0\\
17.45	0\\
17.46	0\\
17.47	0\\
17.48	0\\
17.49	0\\
17.5	0\\
17.51	0\\
17.52	0\\
17.53	0\\
17.54	0\\
17.55	0\\
17.56	0\\
17.57	0\\
17.58	0\\
17.59	0\\
17.6	0\\
17.61	0\\
17.62	0\\
17.63	0\\
17.64	0\\
17.65	0\\
17.66	0\\
17.67	0\\
17.68	0\\
17.69	0\\
17.7	0\\
17.71	0\\
17.72	0\\
17.73	0\\
17.74	0\\
17.75	0\\
17.76	0\\
17.77	0\\
17.78	0\\
17.79	0\\
17.8	0\\
17.81	0\\
17.82	0\\
17.83	0\\
17.84	0\\
17.85	0\\
17.86	0\\
17.87	0\\
17.88	0\\
17.89	0\\
17.9	0\\
17.91	0\\
17.92	0\\
17.93	0\\
17.94	0\\
17.95	0\\
17.96	0\\
17.97	0\\
17.98	0\\
17.99	0\\
18	0\\
18.01	0\\
18.02	0\\
18.03	0\\
18.04	0\\
18.05	0\\
18.06	0\\
18.07	0\\
18.08	0\\
18.09	0\\
18.1	0\\
18.11	0\\
18.12	0\\
18.13	0\\
18.14	0\\
18.15	0\\
18.16	0\\
18.17	0\\
18.18	0\\
18.19	0\\
18.2	0\\
18.21	0\\
18.22	0\\
18.23	0\\
18.24	0\\
18.25	0\\
18.26	0\\
18.27	0\\
18.28	0\\
18.29	0\\
18.3	0\\
18.31	0\\
18.32	0\\
18.33	0\\
18.34	0\\
18.35	0\\
18.36	0\\
18.37	0\\
18.38	0\\
18.39	0\\
18.4	0\\
18.41	0\\
18.42	0\\
18.43	0\\
18.44	0\\
18.45	0\\
18.46	0\\
18.47	0\\
18.48	0\\
18.49	0\\
18.5	0\\
18.51	0\\
18.52	0\\
18.53	0\\
18.54	0\\
18.55	0\\
18.56	0\\
18.57	0\\
18.58	0\\
18.59	0\\
18.6	0\\
18.61	0\\
18.62	0\\
18.63	0\\
18.64	0\\
18.65	0\\
18.66	0\\
18.67	0\\
18.68	0\\
18.69	0\\
18.7	0\\
18.71	0\\
18.72	0\\
18.73	0\\
18.74	0\\
18.75	0\\
18.76	0\\
18.77	0\\
18.78	0\\
18.79	0\\
18.8	0\\
18.81	0\\
18.82	0\\
18.83	0\\
18.84	0\\
18.85	0\\
18.86	0\\
18.87	0\\
18.88	0\\
18.89	0\\
18.9	0\\
18.91	0\\
18.92	0\\
18.93	0\\
18.94	0\\
18.95	0\\
18.96	0\\
18.97	0\\
18.98	0\\
18.99	0\\
19	0\\
19.01	0\\
19.02	0\\
19.03	0\\
19.04	0\\
19.05	0\\
19.06	0\\
19.07	0\\
19.08	0\\
19.09	0\\
19.1	0\\
19.11	0\\
19.12	0\\
19.13	0\\
19.14	0\\
19.15	0\\
19.16	0\\
19.17	0\\
19.18	0\\
19.19	0\\
19.2	0\\
19.21	0\\
19.22	0\\
19.23	0\\
19.24	0\\
19.25	0\\
19.26	0\\
19.27	0\\
19.28	0\\
19.29	0\\
19.3	0\\
19.31	0\\
19.32	0\\
19.33	0\\
19.34	0\\
19.35	0\\
19.36	0\\
19.37	0\\
19.38	0\\
19.39	0\\
19.4	0\\
19.41	0\\
19.42	0\\
19.43	0\\
19.44	0\\
19.45	0\\
19.46	0\\
19.47	0\\
19.48	0\\
19.49	0\\
19.5	0\\
19.51	0\\
19.52	0\\
19.53	0\\
19.54	0\\
19.55	0\\
19.56	0\\
19.57	0\\
19.58	0\\
19.59	0\\
19.6	0\\
19.61	0\\
19.62	0\\
19.63	0\\
19.64	0\\
19.65	0\\
19.66	0\\
19.67	0\\
19.68	0\\
19.69	0\\
19.7	0\\
19.71	0\\
19.72	0\\
19.73	0\\
19.74	0\\
19.75	0\\
19.76	0\\
19.77	0\\
19.78	0\\
19.79	0\\
19.8	0\\
19.81	0\\
19.82	0\\
19.83	0\\
19.84	0\\
19.85	0\\
19.86	0\\
19.87	0\\
19.88	0\\
19.89	0\\
19.9	0\\
19.91	0\\
19.92	0\\
19.93	0\\
19.94	0\\
19.95	0\\
19.96	0\\
19.97	0\\
19.98	0\\
19.99	0\\
20	0\\
20.01	0\\
20.02	0\\
20.03	0\\
20.04	0\\
20.05	0\\
20.06	0\\
20.07	0\\
20.08	0\\
20.09	0\\
20.1	0\\
20.11	0\\
20.12	0\\
20.13	0\\
20.14	0\\
20.15	0\\
20.16	0\\
20.17	0\\
20.18	0\\
20.19	0\\
20.2	0\\
20.21	0\\
20.22	0\\
20.23	0\\
20.24	0\\
20.25	0\\
20.26	0\\
20.27	0\\
20.28	0\\
20.29	0\\
20.3	0\\
20.31	0\\
20.32	0\\
20.33	0\\
20.34	0\\
20.35	0\\
20.36	0\\
20.37	0\\
20.38	0\\
20.39	0\\
20.4	0\\
20.41	0\\
20.42	0\\
20.43	0\\
20.44	0\\
20.45	0\\
20.46	0\\
20.47	0\\
20.48	0\\
20.49	0\\
20.5	0\\
20.51	0\\
20.52	0\\
20.53	0\\
20.54	0\\
20.55	0\\
20.56	0\\
20.57	0\\
20.58	0\\
20.59	0\\
20.6	0\\
20.61	0\\
20.62	0\\
20.63	0\\
20.64	0\\
20.65	0\\
20.66	0\\
20.67	0\\
20.68	0\\
20.69	0\\
20.7	0\\
20.71	0\\
20.72	0\\
20.73	0\\
20.74	0\\
20.75	0\\
20.76	0\\
20.77	0\\
20.78	0\\
20.79	0\\
20.8	0\\
20.81	0\\
20.82	0\\
20.83	0\\
20.84	0\\
20.85	0\\
20.86	0\\
20.87	0\\
20.88	0\\
20.89	0\\
20.9	0\\
20.91	0\\
20.92	0\\
20.93	0\\
20.94	0\\
20.95	0\\
20.96	0\\
20.97	0\\
20.98	0\\
20.99	0\\
21	0\\
21.01	0\\
21.02	0\\
21.03	0\\
21.04	0\\
21.05	0\\
21.06	0\\
21.07	0\\
21.08	0\\
21.09	0\\
21.1	0\\
21.11	0\\
21.12	0\\
21.13	0\\
21.14	0\\
21.15	0\\
21.16	0\\
21.17	0\\
21.18	0\\
21.19	0\\
21.2	0\\
21.21	0\\
21.22	0\\
21.23	0\\
21.24	0\\
21.25	0\\
21.26	0\\
21.27	0\\
21.28	0\\
21.29	0\\
21.3	0\\
21.31	0\\
21.32	0\\
21.33	0\\
21.34	0\\
21.35	0\\
21.36	0\\
21.37	0\\
21.38	0\\
21.39	0\\
21.4	0\\
21.41	0\\
21.42	0\\
21.43	0\\
21.44	0\\
21.45	0\\
21.46	0\\
21.47	0\\
21.48	0\\
21.49	0\\
21.5	0\\
21.51	0\\
21.52	0\\
21.53	0\\
21.54	0\\
21.55	0\\
21.56	0\\
21.57	0\\
21.58	0\\
21.59	0\\
21.6	0\\
21.61	0\\
21.62	0\\
21.63	0\\
21.64	0\\
21.65	0\\
21.66	0\\
21.67	0\\
21.68	0\\
21.69	0\\
21.7	0\\
21.71	0\\
21.72	0\\
21.73	0\\
21.74	0\\
21.75	0\\
21.76	0\\
21.77	0\\
21.78	0\\
21.79	0\\
21.8	0\\
21.81	0\\
21.82	0\\
21.83	0\\
21.84	0\\
21.85	0\\
21.86	0\\
21.87	0\\
21.88	0\\
21.89	0\\
21.9	0\\
21.91	0\\
21.92	0\\
21.93	0\\
21.94	0\\
21.95	0\\
21.96	0\\
21.97	0\\
21.98	0\\
21.99	0\\
22	0\\
22.01	0\\
22.02	0\\
22.03	0\\
22.04	0\\
22.05	0\\
22.06	0\\
22.07	0\\
22.08	0\\
22.09	0\\
22.1	0\\
22.11	0\\
22.12	0\\
22.13	0\\
22.14	0\\
22.15	0\\
22.16	0\\
22.17	0\\
22.18	0\\
22.19	0\\
22.2	0\\
22.21	0\\
22.22	0\\
22.23	0\\
22.24	0\\
22.25	0\\
22.26	0\\
22.27	0\\
22.28	0\\
22.29	0\\
22.3	0\\
22.31	0\\
22.32	0\\
22.33	0\\
22.34	0\\
22.35	0\\
22.36	0\\
22.37	0\\
22.38	0\\
22.39	0\\
22.4	0\\
22.41	0\\
22.42	0\\
22.43	0\\
22.44	0\\
22.45	0\\
22.46	0\\
22.47	0\\
22.48	0\\
22.49	0\\
22.5	0\\
22.51	0\\
22.52	0\\
22.53	0\\
22.54	0\\
22.55	0\\
22.56	0\\
22.57	0\\
22.58	0\\
22.59	0\\
22.6	0\\
22.61	0\\
22.62	0\\
22.63	0\\
22.64	0\\
22.65	0\\
22.66	0\\
22.67	0\\
22.68	0\\
22.69	0\\
22.7	0\\
22.71	0\\
22.72	0\\
22.73	0\\
22.74	0\\
22.75	0\\
22.76	0\\
22.77	0\\
22.78	0\\
22.79	0\\
22.8	0\\
22.81	0\\
22.82	0\\
22.83	0\\
22.84	0\\
22.85	0\\
22.86	0\\
22.87	0\\
22.88	0\\
22.89	0\\
22.9	0\\
22.91	0\\
22.92	0\\
22.93	0\\
22.94	0\\
22.95	0\\
22.96	0\\
22.97	0\\
22.98	0\\
22.99	0\\
23	0\\
23.01	0\\
23.02	0\\
23.03	0\\
23.04	0\\
23.05	0\\
23.06	0\\
23.07	0\\
23.08	0\\
23.09	0\\
23.1	0\\
23.11	0\\
23.12	0\\
23.13	0\\
23.14	0\\
23.15	0\\
23.16	0\\
23.17	0\\
23.18	0\\
23.19	0\\
23.2	0\\
23.21	0\\
23.22	0\\
23.23	0\\
23.24	0\\
23.25	0\\
23.26	0\\
23.27	0\\
23.28	0\\
23.29	0\\
23.3	0\\
23.31	0\\
23.32	0\\
23.33	0\\
23.34	0\\
23.35	0\\
23.36	0\\
23.37	0\\
23.38	0\\
23.39	0\\
23.4	0\\
23.41	0\\
23.42	0\\
23.43	0\\
23.44	0\\
23.45	0\\
23.46	0\\
23.47	0\\
23.48	0\\
23.49	0\\
23.5	0\\
23.51	0\\
23.52	0\\
23.53	0\\
23.54	0\\
23.55	0\\
23.56	0\\
23.57	0\\
23.58	0\\
23.59	0\\
23.6	0\\
23.61	0\\
23.62	0\\
23.63	0\\
23.64	0\\
23.65	0\\
23.66	0\\
23.67	0\\
23.68	0\\
23.69	0\\
23.7	0\\
23.71	0\\
23.72	0\\
23.73	0\\
23.74	0\\
23.75	0\\
23.76	0\\
23.77	0\\
23.78	0\\
23.79	0\\
23.8	0\\
23.81	0\\
23.82	0\\
23.83	0\\
23.84	0\\
23.85	0\\
23.86	0\\
23.87	0\\
23.88	0\\
23.89	0\\
23.9	0\\
23.91	0\\
23.92	0\\
23.93	0\\
23.94	0\\
23.95	0\\
23.96	0\\
23.97	0\\
23.98	0\\
23.99	0\\
24	0\\
24.01	0\\
24.02	0\\
24.03	0\\
24.04	0\\
24.05	0\\
24.06	0\\
24.07	0\\
24.08	0\\
24.09	0\\
24.1	0\\
24.11	0\\
24.12	0\\
24.13	0\\
24.14	0\\
24.15	0\\
24.16	0\\
24.17	0\\
24.18	0\\
24.19	0\\
24.2	0\\
24.21	0\\
24.22	0\\
24.23	0\\
24.24	0\\
24.25	0\\
24.26	0\\
24.27	0\\
24.28	0\\
24.29	0\\
24.3	0\\
24.31	0\\
24.32	0\\
24.33	0\\
24.34	0\\
24.35	0\\
24.36	0\\
24.37	0\\
24.38	0\\
24.39	0\\
24.4	0\\
24.41	0\\
24.42	0\\
24.43	0\\
24.44	0\\
24.45	0\\
24.46	0\\
24.47	0\\
24.48	0\\
24.49	0\\
24.5	0\\
24.51	0\\
24.52	0\\
24.53	0\\
24.54	0\\
24.55	0\\
24.56	0\\
24.57	0\\
24.58	0\\
24.59	0\\
24.6	0\\
24.61	0\\
24.62	0\\
24.63	0\\
24.64	0\\
24.65	0\\
24.66	0\\
24.67	0\\
24.68	0\\
24.69	0\\
24.7	0\\
24.71	0\\
24.72	0\\
24.73	0\\
24.74	0\\
24.75	0\\
24.76	0\\
24.77	0\\
24.78	0\\
24.79	0\\
24.8	0\\
24.81	0\\
24.82	0\\
24.83	0\\
24.84	0\\
24.85	0\\
24.86	0\\
24.87	0\\
24.88	0\\
24.89	0\\
24.9	0\\
24.91	0\\
24.92	0\\
24.93	0\\
24.94	0\\
24.95	0\\
24.96	0\\
24.97	0\\
24.98	0\\
24.99	0\\
25	0\\
25.01	0\\
25.02	0\\
25.03	0\\
25.04	0\\
25.05	0\\
25.06	0\\
25.07	0\\
25.08	0\\
25.09	0\\
25.1	0\\
25.11	0\\
25.12	0\\
25.13	0\\
25.14	0\\
25.15	0\\
25.16	0\\
25.17	0\\
25.18	0\\
25.19	0\\
25.2	0\\
25.21	0\\
25.22	0\\
25.23	0\\
25.24	0\\
25.25	0\\
25.26	0\\
25.27	0\\
25.28	0\\
25.29	0\\
25.3	0\\
25.31	0\\
25.32	0\\
25.33	0\\
25.34	0\\
25.35	0\\
25.36	0\\
25.37	0\\
25.38	0\\
25.39	0\\
25.4	0\\
25.41	0\\
25.42	0\\
25.43	0\\
25.44	0\\
25.45	0\\
25.46	0\\
25.47	0\\
25.48	0\\
25.49	0\\
25.5	0\\
25.51	0\\
25.52	0\\
25.53	0\\
25.54	0\\
25.55	0\\
25.56	0\\
25.57	0\\
25.58	0\\
25.59	0\\
25.6	0\\
25.61	0\\
25.62	0\\
25.63	0\\
25.64	0\\
25.65	0\\
25.66	0\\
25.67	0\\
25.68	0\\
25.69	0\\
25.7	0\\
25.71	0\\
25.72	0\\
25.73	0\\
25.74	0\\
25.75	0\\
25.76	0\\
25.77	0\\
25.78	0\\
25.79	0\\
25.8	0\\
25.81	0\\
25.82	0\\
25.83	0\\
25.84	0\\
25.85	0\\
25.86	0\\
25.87	0\\
25.88	0\\
25.89	0\\
25.9	0\\
25.91	0\\
25.92	0\\
25.93	0\\
25.94	0\\
25.95	0\\
25.96	0\\
25.97	0\\
25.98	0\\
25.99	0\\
26	0\\
26.01	0\\
26.02	0\\
26.03	0\\
26.04	0\\
26.05	0\\
26.06	0\\
26.07	0\\
26.08	0\\
26.09	0\\
26.1	0\\
26.11	0\\
26.12	0\\
26.13	0\\
26.14	0\\
26.15	0\\
26.16	0\\
26.17	0\\
26.18	0\\
26.19	0\\
26.2	0\\
26.21	0\\
26.22	0\\
26.23	0\\
26.24	0\\
26.25	0\\
26.26	0\\
26.27	0\\
26.28	0\\
26.29	0\\
26.3	0\\
26.31	0\\
26.32	0\\
26.33	0\\
26.34	0\\
26.35	0\\
26.36	0\\
26.37	0\\
26.38	0\\
26.39	0\\
26.4	0\\
26.41	0\\
26.42	0\\
26.43	0\\
26.44	0\\
26.45	0\\
26.46	0\\
26.47	0\\
26.48	0\\
26.49	0\\
26.5	0\\
26.51	0\\
26.52	0\\
26.53	0\\
26.54	0\\
26.55	0\\
26.56	0\\
26.57	0\\
26.58	0\\
26.59	0\\
26.6	0\\
26.61	0\\
26.62	0\\
26.63	0\\
26.64	0\\
26.65	0\\
26.66	0\\
26.67	0\\
26.68	0\\
26.69	0\\
26.7	0\\
26.71	0\\
26.72	0\\
26.73	0\\
26.74	0\\
26.75	0\\
26.76	0\\
26.77	0\\
26.78	0\\
26.79	0\\
26.8	0\\
26.81	0\\
26.82	0\\
26.83	0\\
26.84	0\\
26.85	0\\
26.86	0\\
26.87	0\\
26.88	0\\
26.89	0\\
26.9	0\\
26.91	0\\
26.92	0\\
26.93	0\\
26.94	0\\
26.95	0\\
26.96	0\\
26.97	0\\
26.98	0\\
26.99	0\\
27	0\\
27.01	0\\
27.02	0\\
27.03	0\\
27.04	0\\
27.05	0\\
27.06	0\\
27.07	0\\
27.08	0\\
27.09	0\\
27.1	0\\
27.11	0\\
27.12	0\\
27.13	0\\
27.14	0\\
27.15	0\\
27.16	0\\
27.17	0\\
27.18	0\\
27.19	0\\
27.2	0\\
27.21	0\\
27.22	0\\
27.23	0\\
27.24	0\\
27.25	0\\
27.26	0\\
27.27	0\\
27.28	0\\
27.29	0\\
27.3	0\\
27.31	0\\
27.32	0\\
27.33	0\\
27.34	0\\
27.35	0\\
27.36	0\\
27.37	0\\
27.38	0\\
27.39	0\\
27.4	0\\
27.41	0\\
27.42	0\\
27.43	0\\
27.44	0\\
27.45	0\\
27.46	0\\
27.47	0\\
27.48	0\\
27.49	0\\
27.5	0\\
27.51	0\\
27.52	0\\
27.53	0\\
27.54	0\\
27.55	0\\
27.56	0\\
27.57	0\\
27.58	0\\
27.59	0\\
27.6	0\\
27.61	0\\
27.62	0\\
27.63	0\\
27.64	0\\
27.65	0\\
27.66	0\\
27.67	0\\
27.68	0\\
27.69	0\\
27.7	0\\
27.71	0\\
27.72	0\\
27.73	0\\
27.74	0\\
27.75	0\\
27.76	0\\
27.77	0\\
27.78	0\\
27.79	0\\
27.8	0\\
27.81	0\\
27.82	0\\
27.83	0\\
27.84	0\\
27.85	0\\
27.86	0\\
27.87	0\\
27.88	0\\
27.89	0\\
27.9	0\\
27.91	0\\
27.92	0\\
27.93	0\\
27.94	0\\
27.95	0\\
27.96	0\\
27.97	0\\
27.98	0\\
27.99	0\\
28	0\\
28.01	0\\
28.02	0\\
28.03	0\\
28.04	0\\
28.05	0\\
28.06	0\\
28.07	0\\
28.08	0\\
28.09	0\\
28.1	0\\
28.11	0\\
28.12	0\\
28.13	0\\
28.14	0\\
28.15	0\\
28.16	0\\
28.17	0\\
28.18	0\\
28.19	0\\
28.2	0\\
28.21	0\\
28.22	0\\
28.23	0\\
28.24	0\\
28.25	0\\
28.26	0\\
28.27	0\\
28.28	0\\
28.29	0\\
28.3	0\\
28.31	0\\
28.32	0\\
28.33	0\\
28.34	0\\
28.35	0\\
28.36	0\\
28.37	0\\
28.38	0\\
28.39	0\\
28.4	0\\
28.41	0\\
28.42	0\\
28.43	0\\
28.44	0\\
28.45	0\\
28.46	0\\
28.47	0\\
28.48	0\\
28.49	0\\
28.5	0\\
28.51	0\\
28.52	0\\
28.53	0\\
28.54	0\\
28.55	0\\
28.56	0\\
28.57	0\\
28.58	0\\
28.59	0\\
28.6	0\\
28.61	0\\
28.62	0\\
28.63	0\\
28.64	0\\
28.65	0\\
28.66	0\\
28.67	0\\
28.68	0\\
28.69	0\\
28.7	0\\
28.71	0\\
28.72	0\\
28.73	0\\
28.74	0\\
28.75	0\\
28.76	0\\
28.77	0\\
28.78	0\\
28.79	0\\
28.8	0\\
28.81	0\\
28.82	0\\
28.83	0\\
28.84	0\\
28.85	0\\
28.86	0\\
28.87	0\\
28.88	0\\
28.89	0\\
28.9	0\\
28.91	0\\
28.92	0\\
28.93	0\\
28.94	0\\
28.95	0\\
28.96	0\\
28.97	0\\
28.98	0\\
28.99	0\\
29	0\\
29.01	0\\
29.02	0\\
29.03	0\\
29.04	0\\
29.05	0\\
29.06	0\\
29.07	0\\
29.08	0\\
29.09	0\\
29.1	0\\
29.11	0\\
29.12	0\\
29.13	0\\
29.14	0\\
29.15	0\\
29.16	0\\
29.17	0\\
29.18	0\\
29.19	0\\
29.2	0\\
29.21	0\\
29.22	0\\
29.23	0\\
29.24	0\\
29.25	0\\
29.26	0\\
29.27	0\\
29.28	0\\
29.29	0\\
29.3	0\\
29.31	0\\
29.32	0\\
29.33	0\\
29.34	0\\
29.35	0\\
29.36	0\\
29.37	0\\
29.38	0\\
29.39	0\\
29.4	0\\
29.41	0\\
29.42	0\\
29.43	0\\
29.44	0\\
29.45	0\\
29.46	0\\
29.47	0\\
29.48	0\\
29.49	0\\
29.5	0\\
29.51	0\\
29.52	0\\
29.53	0\\
29.54	0\\
29.55	0\\
29.56	0\\
29.57	0\\
29.58	0\\
29.59	0\\
29.6	0\\
29.61	0\\
29.62	0\\
29.63	0\\
29.64	0\\
29.65	0\\
29.66	0\\
29.67	0\\
29.68	0\\
29.69	0\\
29.7	0\\
29.71	0\\
29.72	0\\
29.73	0\\
29.74	0\\
29.75	0\\
29.76	0\\
29.77	0\\
29.78	0\\
29.79	0\\
29.8	0\\
29.81	0\\
29.82	0\\
29.83	0\\
29.84	0\\
29.85	0\\
29.86	0\\
29.87	0\\
29.88	0\\
29.89	0\\
29.9	0\\
29.91	0\\
29.92	0\\
29.93	0\\
29.94	0\\
29.95	0\\
29.96	0\\
29.97	0\\
29.98	0\\
29.99	0\\
30	0\\
30.01	0\\
30.02	0\\
30.03	0\\
30.04	0\\
30.05	0\\
30.06	0\\
30.07	0\\
30.08	0\\
30.09	0\\
30.1	0\\
30.11	0\\
30.12	0\\
30.13	0\\
30.14	0\\
30.15	0\\
30.16	0\\
30.17	0\\
30.18	0\\
30.19	0\\
30.2	0\\
30.21	0\\
30.22	0\\
30.23	0\\
30.24	0\\
30.25	0\\
30.26	0\\
30.27	0\\
30.28	0\\
30.29	0\\
30.3	0\\
30.31	0\\
30.32	0\\
30.33	0\\
30.34	0\\
30.35	0\\
30.36	0\\
30.37	0\\
30.38	0\\
30.39	0\\
30.4	0\\
30.41	0\\
30.42	0\\
30.43	0\\
30.44	0\\
30.45	0\\
30.46	0\\
30.47	0\\
30.48	0\\
30.49	0\\
30.5	0\\
30.51	0\\
30.52	0\\
30.53	0\\
30.54	0\\
30.55	0\\
30.56	0\\
30.57	0\\
30.58	0\\
30.59	0\\
30.6	0\\
30.61	0\\
30.62	0\\
30.63	0\\
30.64	0\\
30.65	0\\
30.66	0\\
30.67	0\\
30.68	0\\
30.69	0\\
30.7	0\\
30.71	0\\
30.72	0\\
30.73	0\\
30.74	0\\
30.75	0\\
30.76	0\\
30.77	0\\
30.78	0\\
30.79	0\\
30.8	0\\
30.81	0\\
30.82	0\\
30.83	0\\
30.84	0\\
30.85	0\\
30.86	0\\
30.87	0\\
30.88	0\\
30.89	0\\
30.9	0\\
30.91	0\\
30.92	0\\
30.93	0\\
30.94	0\\
30.95	0\\
30.96	0\\
30.97	0\\
30.98	0\\
30.99	0\\
31	0\\
31.01	0\\
31.02	0\\
31.03	0\\
31.04	0\\
31.05	0\\
31.06	0\\
31.07	0\\
31.08	0\\
31.09	0\\
31.1	0\\
31.11	0\\
31.12	0\\
31.13	0\\
31.14	0\\
31.15	0\\
31.16	0\\
31.17	0\\
31.18	0\\
31.19	0\\
31.2	0\\
31.21	0\\
31.22	0\\
31.23	0\\
31.24	0\\
31.25	0\\
31.26	0\\
31.27	0\\
31.28	0\\
31.29	0\\
31.3	0\\
31.31	0\\
31.32	0\\
31.33	0\\
31.34	0\\
31.35	0\\
31.36	0\\
31.37	0\\
31.38	0\\
31.39	0\\
31.4	0\\
31.41	0\\
31.42	0\\
31.43	0\\
31.44	0\\
31.45	0\\
31.46	0\\
31.47	0\\
31.48	0\\
31.49	0\\
31.5	0\\
31.51	0\\
31.52	0\\
31.53	0\\
31.54	0\\
31.55	0\\
31.56	0\\
31.57	0\\
31.58	0\\
31.59	0\\
31.6	0\\
31.61	0\\
31.62	0\\
31.63	0\\
31.64	0\\
31.65	0\\
31.66	0\\
31.67	0\\
31.68	0\\
31.69	0\\
31.7	0\\
31.71	0\\
31.72	0\\
31.73	0\\
31.74	0\\
31.75	0\\
31.76	0\\
31.77	0\\
31.78	0\\
31.79	0\\
31.8	0\\
31.81	0\\
31.82	0\\
31.83	0\\
31.84	0\\
31.85	0\\
31.86	0\\
31.87	0\\
31.88	0\\
31.89	0\\
31.9	0\\
31.91	0\\
31.92	0\\
31.93	0\\
31.94	0\\
31.95	0\\
31.96	0\\
31.97	0\\
31.98	0\\
31.99	0\\
32	0\\
32.01	0\\
32.02	0\\
32.03	0\\
32.04	0\\
32.05	0\\
32.06	0\\
32.07	0\\
32.08	0\\
32.09	0\\
32.1	0\\
32.11	0\\
32.12	0\\
32.13	0\\
32.14	0\\
32.15	0\\
32.16	0\\
32.17	0\\
32.18	0\\
32.19	0\\
32.2	0\\
32.21	0\\
32.22	0\\
32.23	0\\
32.24	0\\
32.25	0\\
32.26	0\\
32.27	0\\
32.28	0\\
32.29	0\\
32.3	0\\
32.31	0\\
32.32	0\\
32.33	0\\
32.34	0\\
32.35	0\\
32.36	0\\
32.37	0\\
32.38	0\\
32.39	0\\
32.4	0\\
32.41	0\\
32.42	0\\
32.43	0\\
32.44	0\\
32.45	0\\
32.46	0\\
32.47	0\\
32.48	0\\
32.49	0\\
32.5	0\\
32.51	0\\
32.52	0\\
32.53	0\\
32.54	0\\
32.55	0\\
32.56	0\\
32.57	0\\
32.58	0\\
32.59	0\\
32.6	0\\
32.61	0\\
32.62	0\\
32.63	0\\
32.64	0\\
32.65	0\\
32.66	0\\
32.67	0\\
32.68	0\\
32.69	0\\
32.7	0\\
32.71	0\\
32.72	0\\
32.73	0\\
32.74	0\\
32.75	0\\
32.76	0\\
32.77	0\\
32.78	0\\
32.79	0\\
32.8	0\\
32.81	0\\
32.82	0\\
32.83	0\\
32.84	0\\
32.85	0\\
32.86	0\\
32.87	0\\
32.88	0\\
32.89	0\\
32.9	0\\
32.91	0\\
32.92	0\\
32.93	0\\
32.94	0\\
32.95	0\\
32.96	0\\
32.97	0\\
32.98	0\\
32.99	0\\
33	0\\
33.01	0\\
33.02	0\\
33.03	0\\
33.04	0\\
33.05	0\\
33.06	0\\
33.07	0\\
33.08	0\\
33.09	0\\
33.1	0\\
33.11	0\\
33.12	0\\
33.13	0\\
33.14	0\\
33.15	0\\
33.16	0\\
33.17	0\\
33.18	0\\
33.19	0\\
33.2	0\\
33.21	0\\
33.22	0\\
33.23	0\\
33.24	0\\
33.25	0\\
33.26	0\\
33.27	0\\
33.28	0\\
33.29	0\\
33.3	0\\
33.31	0\\
33.32	0\\
33.33	0\\
33.34	0\\
33.35	0\\
33.36	0\\
33.37	0\\
33.38	0\\
33.39	0\\
33.4	0\\
33.41	0\\
33.42	0\\
33.43	0\\
33.44	0\\
33.45	0\\
33.46	0\\
33.47	0\\
33.48	0\\
33.49	0\\
33.5	0\\
33.51	0\\
33.52	0\\
33.53	0\\
33.54	0\\
33.55	0\\
33.56	0\\
33.57	0\\
33.58	0\\
33.59	0\\
33.6	0\\
33.61	0\\
33.62	0\\
33.63	0\\
33.64	0\\
33.65	0\\
33.66	0\\
33.67	0\\
33.68	0\\
33.69	0\\
33.7	0\\
33.71	0\\
33.72	0\\
33.73	0\\
33.74	0\\
33.75	0\\
33.76	0\\
33.77	0\\
33.78	0\\
33.79	0\\
33.8	0\\
33.81	0\\
33.82	0\\
33.83	0\\
33.84	0\\
33.85	0\\
33.86	0\\
33.87	0\\
33.88	0\\
33.89	0\\
33.9	0\\
33.91	0\\
33.92	0\\
33.93	0\\
33.94	0\\
33.95	0\\
33.96	0\\
33.97	0\\
33.98	0\\
33.99	0\\
34	0\\
34.01	0\\
34.02	0\\
34.03	0\\
34.04	0\\
34.05	0\\
34.06	0\\
34.07	0\\
34.08	0\\
34.09	0\\
34.1	0\\
34.11	0\\
34.12	0\\
34.13	0\\
34.14	0\\
34.15	0\\
34.16	0\\
34.17	0\\
34.18	0\\
34.19	0\\
34.2	0\\
34.21	0\\
34.22	0\\
34.23	0\\
34.24	0\\
34.25	0\\
34.26	0\\
34.27	0\\
34.28	0\\
34.29	0\\
34.3	0\\
34.31	0\\
34.32	0\\
34.33	0\\
34.34	0\\
34.35	0\\
34.36	0\\
34.37	0\\
34.38	0\\
34.39	0\\
34.4	0\\
34.41	0\\
34.42	0\\
34.43	0\\
34.44	0\\
34.45	0\\
34.46	0\\
34.47	0\\
34.48	0\\
34.49	0\\
34.5	0\\
34.51	0\\
34.52	0\\
34.53	0\\
34.54	0\\
34.55	0\\
34.56	0\\
34.57	0\\
34.58	0\\
34.59	0\\
34.6	0\\
34.61	0\\
34.62	0\\
34.63	0\\
34.64	0\\
34.65	0\\
34.66	0\\
34.67	0\\
34.68	0\\
34.69	0\\
34.7	0\\
34.71	0\\
34.72	0\\
34.73	0\\
34.74	0\\
34.75	0\\
34.76	0\\
34.77	0\\
34.78	0\\
34.79	0\\
34.8	0\\
34.81	0\\
34.82	0\\
34.83	0\\
34.84	0\\
34.85	0\\
34.86	0\\
34.87	0\\
34.88	0\\
34.89	0\\
34.9	0\\
34.91	0\\
34.92	0\\
34.93	0\\
34.94	0\\
34.95	0\\
34.96	0\\
34.97	0\\
34.98	0\\
34.99	0\\
35	0\\
35.01	0\\
35.02	0\\
35.03	0\\
35.04	0\\
35.05	0\\
35.06	0\\
35.07	0\\
35.08	0\\
35.09	0\\
35.1	0\\
35.11	0\\
35.12	0\\
35.13	0\\
35.14	0\\
35.15	0\\
35.16	0\\
35.17	0\\
35.18	0\\
35.19	0\\
35.2	0\\
35.21	0\\
35.22	0\\
35.23	0\\
35.24	0\\
35.25	0\\
35.26	0\\
35.27	0\\
35.28	0\\
35.29	0\\
35.3	0\\
35.31	0\\
35.32	0\\
35.33	0\\
35.34	0\\
35.35	0\\
35.36	0\\
35.37	0\\
35.38	0\\
35.39	0\\
35.4	0\\
35.41	0\\
35.42	0\\
35.43	0\\
35.44	0\\
35.45	0\\
35.46	0\\
35.47	0\\
35.48	0\\
35.49	0\\
35.5	0\\
35.51	0\\
35.52	0\\
35.53	0\\
35.54	0\\
35.55	0\\
35.56	0\\
35.57	0\\
35.58	0\\
35.59	0\\
35.6	0\\
35.61	0\\
35.62	0\\
35.63	0\\
35.64	0\\
35.65	0\\
35.66	0\\
35.67	0\\
35.68	0\\
35.69	0\\
35.7	0\\
35.71	0\\
35.72	0\\
35.73	0\\
35.74	0\\
35.75	0\\
35.76	0\\
35.77	0\\
35.78	0\\
35.79	0\\
35.8	0\\
35.81	0\\
35.82	0\\
35.83	0\\
35.84	0\\
35.85	0\\
35.86	0\\
35.87	0\\
35.88	0\\
35.89	0\\
35.9	0\\
35.91	0\\
35.92	0\\
35.93	0\\
35.94	0\\
35.95	0\\
35.96	0\\
35.97	0\\
35.98	0\\
35.99	0\\
36	0\\
36.01	0\\
36.02	0\\
36.03	0\\
36.04	0\\
36.05	0\\
36.06	0\\
36.07	0\\
36.08	0\\
36.09	0\\
36.1	0\\
36.11	0\\
36.12	0\\
36.13	0\\
36.14	0\\
36.15	0\\
36.16	0\\
36.17	0\\
36.18	0\\
36.19	0\\
36.2	0\\
36.21	0\\
36.22	0\\
36.23	0\\
36.24	0\\
36.25	0\\
36.26	0\\
36.27	0\\
36.28	0\\
36.29	0\\
36.3	0\\
36.31	0\\
36.32	0\\
36.33	0\\
36.34	0\\
36.35	0\\
36.36	0\\
36.37	0\\
36.38	0\\
36.39	0\\
36.4	0\\
36.41	0\\
36.42	0\\
36.43	0\\
36.44	0\\
36.45	0\\
36.46	0\\
36.47	0\\
36.48	0\\
36.49	0\\
36.5	0\\
36.51	0\\
36.52	0\\
36.53	0\\
36.54	0\\
36.55	0\\
36.56	0\\
36.57	0\\
36.58	0\\
36.59	0\\
36.6	0\\
36.61	0\\
36.62	0\\
36.63	0\\
36.64	0\\
36.65	0\\
36.66	0\\
36.67	0\\
36.68	0\\
36.69	0\\
36.7	0\\
36.71	0\\
36.72	0\\
36.73	0\\
36.74	0\\
36.75	0\\
36.76	0\\
36.77	0\\
36.78	0\\
36.79	0\\
36.8	0\\
36.81	0\\
36.82	0\\
36.83	0\\
36.84	0\\
36.85	0\\
36.86	0\\
36.87	0\\
36.88	0\\
36.89	0\\
36.9	0\\
36.91	0\\
36.92	0\\
36.93	0\\
36.94	0\\
36.95	0\\
36.96	0\\
36.97	0\\
36.98	0\\
36.99	0\\
37	0\\
37.01	0\\
37.02	0\\
37.03	0\\
37.04	0\\
37.05	0\\
37.06	0\\
37.07	0\\
37.08	0\\
37.09	0\\
37.1	0\\
37.11	0\\
37.12	0\\
37.13	0\\
37.14	0\\
37.15	0\\
37.16	0\\
37.17	0\\
37.18	0\\
37.19	0\\
37.2	0\\
37.21	0\\
37.22	0\\
37.23	0\\
37.24	0\\
37.25	0\\
37.26	0\\
37.27	0\\
37.28	0\\
37.29	0\\
37.3	0\\
37.31	0\\
37.32	0\\
37.33	0\\
37.34	0\\
37.35	0\\
37.36	0\\
37.37	0\\
37.38	0\\
37.39	0\\
37.4	0\\
37.41	0\\
37.42	0\\
37.43	0\\
37.44	0\\
37.45	0\\
37.46	0\\
37.47	0\\
37.48	0\\
37.49	0\\
37.5	0\\
37.51	0\\
37.52	0\\
37.53	0\\
37.54	0\\
37.55	0\\
37.56	0\\
37.57	0\\
37.58	0\\
37.59	0\\
37.6	0\\
37.61	0\\
37.62	0\\
37.63	0\\
37.64	0\\
37.65	0\\
37.66	0\\
37.67	0\\
37.68	0\\
37.69	0\\
37.7	0\\
37.71	0\\
37.72	0\\
37.73	0\\
37.74	0\\
37.75	0\\
37.76	0\\
37.77	0\\
37.78	0\\
37.79	0\\
37.8	0\\
37.81	0\\
37.82	0\\
37.83	0\\
37.84	0\\
37.85	0\\
37.86	0\\
37.87	0\\
37.88	0\\
37.89	0\\
37.9	0\\
37.91	0\\
37.92	0\\
37.93	0\\
37.94	0\\
37.95	0\\
37.96	0\\
37.97	0\\
37.98	0\\
37.99	0\\
38	0\\
38.01	0\\
38.02	0\\
38.03	0\\
38.04	0\\
38.05	0\\
38.06	0\\
38.07	0\\
38.08	0\\
38.09	0\\
38.1	0\\
38.11	0\\
38.12	0\\
38.13	0\\
38.14	0\\
38.15	0\\
38.16	0\\
38.17	0\\
38.18	0\\
38.19	0\\
38.2	0\\
38.21	0\\
38.22	0\\
38.23	0\\
38.24	0\\
38.25	0\\
38.26	0\\
38.27	0\\
38.28	0\\
38.29	0\\
38.3	0\\
38.31	0\\
38.32	0\\
38.33	0\\
38.34	0\\
38.35	0\\
38.36	0\\
38.37	0\\
38.38	0\\
38.39	0\\
38.4	0\\
38.41	0\\
38.42	0\\
38.43	0\\
38.44	0\\
38.45	0\\
38.46	0\\
38.47	0\\
38.48	0\\
38.49	0\\
38.5	0\\
38.51	0\\
38.52	0\\
38.53	0\\
38.54	0\\
38.55	0\\
38.56	0\\
38.57	0\\
38.58	0\\
38.59	0\\
38.6	0\\
38.61	1.73472347597681e-18\\
38.62	0\\
38.63	0\\
38.64	0\\
38.65	0\\
38.66	0\\
38.67	0\\
38.68	0\\
38.69	0\\
38.7	0\\
38.71	0\\
38.72	0\\
38.73	0\\
38.74	0\\
38.75	0\\
38.76	0\\
38.77	1.73472347597681e-18\\
38.78	0\\
38.79	0\\
38.8	0\\
38.81	0\\
38.82	0\\
38.83	0\\
38.84	0\\
38.85	0\\
38.86	0\\
38.87	0\\
38.88	0\\
38.89	0\\
38.9	0\\
38.91	0\\
38.92	0\\
38.93	1.73472347597681e-18\\
38.94	0\\
38.95	0\\
38.96	0\\
38.97	0\\
38.98	0\\
38.99	0\\
39	0\\
39.01	0\\
39.02	0\\
39.03	0\\
39.04	0\\
39.05	0\\
39.06	0\\
39.07	0\\
39.08	0\\
39.09	1.73472347597681e-18\\
39.1	0\\
39.11	0\\
39.12	0\\
39.13	0\\
39.14	0\\
39.15	0\\
39.16	0\\
39.17	0\\
39.18	0\\
39.19	0\\
39.2	0\\
39.21	0\\
39.22	0\\
39.23	0\\
39.24	0\\
39.25	1.73472347597681e-18\\
39.26	0\\
39.27	0\\
39.28	0\\
39.29	0\\
39.3	0\\
39.31	0\\
39.32	0\\
39.33	0\\
39.34	0\\
39.35	0\\
39.36	0\\
39.37	0\\
39.38	0\\
39.39	0\\
39.4	0\\
39.41	0\\
39.42	0\\
39.43	0\\
39.44	0\\
39.45	0\\
39.46	0\\
39.47	0\\
39.48	0\\
39.49	0\\
39.5	0\\
39.51	0\\
39.52	0\\
39.53	0\\
39.54	0\\
39.55	0\\
39.56	0\\
39.57	0\\
39.58	0\\
39.59	0\\
39.6	0\\
39.61	0\\
39.62	0\\
39.63	0\\
39.64	0\\
39.65	0\\
39.66	0\\
39.67	0\\
39.68	0\\
39.69	0\\
39.7	0\\
39.71	0\\
39.72	0\\
39.73	0\\
39.74	0\\
39.75	0\\
39.76	0\\
39.77	0\\
39.78	0\\
39.79	0\\
39.8	0\\
39.81	0\\
39.82	0\\
39.83	0\\
39.84	0\\
39.85	0\\
39.86	0\\
39.87	0\\
39.88	0\\
39.89	0\\
39.9	0\\
39.91	0\\
39.92	0\\
39.93	0\\
39.94	0\\
39.95	0\\
39.96	0\\
39.97	1.73472347597681e-18\\
39.98	0\\
39.99	0\\
40	0\\
40.01	0\\
};
\addplot [color=green,solid,forget plot]
  table[row sep=crcr]{%
40.01	0\\
40.02	0\\
40.03	0\\
40.04	0\\
40.05	0\\
40.06	0\\
40.07	0\\
40.08	0\\
40.09	0\\
40.1	0\\
40.11	0\\
40.12	0\\
40.13	1.73472347597681e-18\\
40.14	0\\
40.15	0\\
40.16	0\\
40.17	0\\
40.18	1.73472347597681e-18\\
40.19	0\\
40.2	0\\
40.21	0\\
40.22	1.73472347597681e-18\\
40.23	0\\
40.24	0\\
40.25	0\\
40.26	1.73472347597681e-18\\
40.27	0\\
40.28	0\\
40.29	0\\
40.3	1.73472347597681e-18\\
40.31	0\\
40.32	0\\
40.33	0\\
40.34	1.73472347597681e-18\\
40.35	0\\
40.36	0\\
40.37	0\\
40.38	1.73472347597681e-18\\
40.39	0\\
40.4	0\\
40.41	0\\
40.42	0\\
40.43	0\\
40.44	0\\
40.45	0\\
40.46	0\\
40.47	0\\
40.48	1.73472347597681e-18\\
40.49	0\\
40.5	0\\
40.51	0\\
40.52	0\\
40.53	0\\
40.54	0\\
40.55	0\\
40.56	0\\
40.57	0\\
40.58	0\\
40.59	0\\
40.6	0\\
40.61	0\\
40.62	0\\
40.63	0\\
40.64	0\\
40.65	0\\
40.66	0\\
40.67	0\\
40.68	0\\
40.69	0\\
40.7	0\\
40.71	0\\
40.72	0\\
40.73	0\\
40.74	0\\
40.75	0\\
40.76	0\\
40.77	0\\
40.78	0\\
40.79	0\\
40.8	0\\
40.81	0\\
40.82	0\\
40.83	0\\
40.84	0\\
40.85	0\\
40.86	1.73472347597681e-18\\
40.87	0\\
40.88	0\\
40.89	0\\
40.9	0\\
40.91	0\\
40.92	0\\
40.93	0\\
40.94	0\\
40.95	0\\
40.96	0\\
40.97	0\\
40.98	0\\
40.99	0\\
41	0\\
41.01	0\\
41.02	0\\
41.03	0\\
41.04	0\\
41.05	0\\
41.06	1.73472347597681e-18\\
41.07	0\\
41.08	0\\
41.09	0\\
41.1	0\\
41.11	0\\
41.12	0\\
41.13	0\\
41.14	0\\
41.15	0\\
41.16	0\\
41.17	0\\
41.18	0\\
41.19	0\\
41.2	0\\
41.21	0\\
41.22	0\\
41.23	0\\
41.24	0\\
41.25	0\\
41.26	0\\
41.27	1.73472347597681e-18\\
41.28	0\\
41.29	0\\
41.3	0\\
41.31	0\\
41.32	0\\
41.33	0\\
41.34	1.73472347597681e-18\\
41.35	0\\
41.36	0\\
41.37	0\\
41.38	1.73472347597681e-18\\
41.39	0\\
41.4	0\\
41.41	0\\
41.42	0\\
41.43	1.73472347597681e-18\\
41.44	0\\
41.45	0\\
41.46	0\\
41.47	0\\
41.48	0\\
41.49	0\\
41.5	0\\
41.51	0\\
41.52	0\\
41.53	0\\
41.54	0\\
41.55	0\\
41.56	0\\
41.57	0\\
41.58	0\\
41.59	0\\
41.6	0\\
41.61	0\\
41.62	0\\
41.63	0\\
41.64	0\\
41.65	0\\
41.66	0\\
41.67	0\\
41.68	0\\
41.69	0\\
41.7	0\\
41.71	0\\
41.72	0\\
41.73	0\\
41.74	1.73472347597681e-18\\
41.75	0\\
41.76	0\\
41.77	0\\
41.78	0\\
41.79	0\\
41.8	1.73472347597681e-18\\
41.81	0\\
41.82	0\\
41.83	0\\
41.84	0\\
41.85	0\\
41.86	0\\
41.87	0\\
41.88	0\\
41.89	0\\
41.9	0\\
41.91	0\\
41.92	0\\
41.93	0\\
41.94	0\\
41.95	0\\
41.96	0\\
41.97	0\\
41.98	0\\
41.99	0\\
42	0\\
42.01	0\\
42.02	0\\
42.03	0\\
42.04	1.73472347597681e-18\\
42.05	0\\
42.06	0\\
42.07	0\\
42.08	0\\
42.09	1.73472347597681e-18\\
42.1	0\\
42.11	0\\
42.12	1.73472347597681e-18\\
42.13	0\\
42.14	0\\
42.15	0\\
42.16	0\\
42.17	0\\
42.18	0\\
42.19	0\\
42.2	0\\
42.21	0\\
42.22	0\\
42.23	0\\
42.24	0\\
42.25	0\\
42.26	0\\
42.27	0\\
42.28	0\\
42.29	0\\
42.3	0\\
42.31	0\\
42.32	0\\
42.33	0\\
42.34	0\\
42.35	0\\
42.36	1.73472347597681e-18\\
42.37	0\\
42.38	0\\
42.39	0\\
42.4	0\\
42.41	0\\
42.42	0\\
42.43	0\\
42.44	0\\
42.45	0\\
42.46	0\\
42.47	0\\
42.48	0\\
42.49	0\\
42.5	0\\
42.51	0\\
42.52	0\\
42.53	0\\
42.54	0\\
42.55	0\\
42.56	0\\
42.57	0\\
42.58	0\\
42.59	0\\
42.6	0\\
42.61	0\\
42.62	0\\
42.63	0\\
42.64	0\\
42.65	0\\
42.66	0\\
42.67	0\\
42.68	0\\
42.69	0\\
42.7	0\\
42.71	0\\
42.72	0\\
42.73	0\\
42.74	0\\
42.75	0\\
42.76	0\\
42.77	0\\
42.78	0\\
42.79	0\\
42.8	0\\
42.81	0\\
42.82	0\\
42.83	0\\
42.84	0\\
42.85	0\\
42.86	0\\
42.87	0\\
42.88	0\\
42.89	0\\
42.9	0\\
42.91	0\\
42.92	1.73472347597681e-18\\
42.93	0\\
42.94	0\\
42.95	0\\
42.96	0\\
42.97	1.73472347597681e-18\\
42.98	0\\
42.99	0\\
43	0\\
43.01	0\\
43.02	0\\
43.03	0\\
43.04	0\\
43.05	0\\
43.06	0\\
43.07	0\\
43.08	0\\
43.09	0\\
43.1	0\\
43.11	0\\
43.12	0\\
43.13	0\\
43.14	0\\
43.15	0\\
43.16	0\\
43.17	0\\
43.18	0\\
43.19	0\\
43.2	0\\
43.21	0\\
43.22	0\\
43.23	0\\
43.24	1.73472347597681e-18\\
43.25	0\\
43.26	0\\
43.27	0\\
43.28	0\\
43.29	0\\
43.3	0\\
43.31	0\\
43.32	0\\
43.33	0\\
43.34	0\\
43.35	0\\
43.36	0\\
43.37	0\\
43.38	0\\
43.39	0\\
43.4	0\\
43.41	0\\
43.42	0\\
43.43	1.73472347597681e-18\\
43.44	0\\
43.45	0\\
43.46	0\\
43.47	0\\
43.48	0\\
43.49	0\\
43.5	0\\
43.51	0\\
43.52	0\\
43.53	0\\
43.54	0\\
43.55	0\\
43.56	0\\
43.57	0\\
43.58	0\\
43.59	0\\
43.6	0\\
43.61	0\\
43.62	0\\
43.63	0\\
43.64	0\\
43.65	0\\
43.66	0\\
43.67	0\\
43.68	0\\
43.69	0\\
43.7	0\\
43.71	0\\
43.72	0\\
43.73	0\\
43.74	0\\
43.75	0\\
43.76	0\\
43.77	0\\
43.78	0\\
43.79	0\\
43.8	0\\
43.81	1.73472347597681e-18\\
43.82	0\\
43.83	0\\
43.84	1.73472347597681e-18\\
43.85	0\\
43.86	0\\
43.87	0\\
43.88	0\\
43.89	0\\
43.9	0\\
43.91	0\\
43.92	0\\
43.93	0\\
43.94	1.73472347597681e-18\\
43.95	0\\
43.96	0\\
43.97	0\\
43.98	1.73472347597681e-18\\
43.99	0\\
44	0\\
44.01	0\\
44.02	1.73472347597681e-18\\
44.03	0\\
44.04	0\\
44.05	0\\
44.06	1.73472347597681e-18\\
44.07	0\\
44.08	0\\
44.09	0\\
44.1	0\\
44.11	0\\
44.12	1.73472347597681e-18\\
44.13	0\\
44.14	0\\
44.15	0\\
44.16	0\\
44.17	0\\
44.18	0\\
44.19	0\\
44.2	0\\
44.21	0\\
44.22	0\\
44.23	0\\
44.24	0\\
44.25	1.73472347597681e-18\\
44.26	0\\
44.27	0\\
44.28	0\\
44.29	0\\
44.3	0\\
44.31	0\\
44.32	0\\
44.33	0\\
44.34	0\\
44.35	0\\
44.36	0\\
44.37	0\\
44.38	0\\
44.39	0\\
44.4	0\\
44.41	0\\
44.42	1.73472347597681e-18\\
44.43	0\\
44.44	0\\
44.45	0\\
44.46	0\\
44.47	0\\
44.48	0\\
44.49	0\\
44.5	0\\
44.51	1.73472347597681e-18\\
44.52	0\\
44.53	0\\
44.54	0\\
44.55	0\\
44.56	0\\
44.57	0\\
44.58	0\\
44.59	0\\
44.6	0\\
44.61	0\\
44.62	0\\
44.63	0\\
44.64	0\\
44.65	0\\
44.66	0\\
44.67	0\\
44.68	0\\
44.69	0\\
44.7	0\\
44.71	0\\
44.72	0\\
44.73	0\\
44.74	0\\
44.75	0\\
44.76	0\\
44.77	0\\
44.78	0\\
44.79	0\\
44.8	0\\
44.81	0\\
44.82	0\\
44.83	0\\
44.84	0\\
44.85	0\\
44.86	0\\
44.87	0\\
44.88	0\\
44.89	0\\
44.9	0\\
44.91	0\\
44.92	0\\
44.93	0\\
44.94	0\\
44.95	0\\
44.96	0\\
44.97	0\\
44.98	0\\
44.99	0\\
45	0\\
45.01	0\\
45.02	0\\
45.03	0\\
45.04	0\\
45.05	0\\
45.06	0\\
45.07	0\\
45.08	0\\
45.09	0\\
45.1	0\\
45.11	0\\
45.12	0\\
45.13	0\\
45.14	0\\
45.15	0\\
45.16	0\\
45.17	0\\
45.18	0\\
45.19	0\\
45.2	0\\
45.21	0\\
45.22	0\\
45.23	0\\
45.24	0\\
45.25	0\\
45.26	0\\
45.27	0\\
45.28	0\\
45.29	0\\
45.3	0\\
45.31	0\\
45.32	1.73472347597681e-18\\
45.33	0\\
45.34	0\\
45.35	0\\
45.36	0\\
45.37	0\\
45.38	0\\
45.39	0\\
45.4	0\\
45.41	0\\
45.42	0\\
45.43	1.73472347597681e-18\\
45.44	0\\
45.45	0\\
45.46	0\\
45.47	0\\
45.48	0\\
45.49	0\\
45.5	0\\
45.51	0\\
45.52	0\\
45.53	0\\
45.54	1.73472347597681e-18\\
45.55	0\\
45.56	0\\
45.57	0\\
45.58	0\\
45.59	0\\
45.6	1.73472347597681e-18\\
45.61	0\\
45.62	0\\
45.63	0\\
45.64	0\\
45.65	0\\
45.66	0\\
45.67	0\\
45.68	0\\
45.69	0\\
45.7	0\\
45.71	0\\
45.72	0\\
45.73	0\\
45.74	0\\
45.75	0\\
45.76	0\\
45.77	0\\
45.78	0\\
45.79	0\\
45.8	0\\
45.81	0\\
45.82	0\\
45.83	0\\
45.84	0\\
45.85	0\\
45.86	0\\
45.87	0\\
45.88	0\\
45.89	0\\
45.9	0\\
45.91	0\\
45.92	0\\
45.93	0\\
45.94	0\\
45.95	0\\
45.96	0\\
45.97	0\\
45.98	1.73472347597681e-18\\
45.99	0\\
46	0\\
46.01	0\\
46.02	0\\
46.03	0\\
46.04	0\\
46.05	0\\
46.06	0\\
46.07	0\\
46.08	0\\
46.09	0\\
46.1	0\\
46.11	1.73472347597681e-18\\
46.12	1.73472347597681e-18\\
46.13	0\\
46.14	0\\
46.15	0\\
46.16	0\\
46.17	0\\
46.18	0\\
46.19	0\\
46.2	0\\
46.21	0\\
46.22	0\\
46.23	0\\
46.24	0\\
46.25	0\\
46.26	0\\
46.27	0\\
46.28	0\\
46.29	0\\
46.3	0\\
46.31	0\\
46.32	0\\
46.33	0\\
46.34	0\\
46.35	0\\
46.36	0\\
46.37	0\\
46.38	0\\
46.39	0\\
46.4	1.73472347597681e-18\\
46.41	0\\
46.42	0\\
46.43	0\\
46.44	0\\
46.45	0\\
46.46	0\\
46.47	0\\
46.48	0\\
46.49	0\\
46.5	0\\
46.51	1.73472347597681e-18\\
46.52	0\\
46.53	0\\
46.54	0\\
46.55	0\\
46.56	0\\
46.57	0\\
46.58	0\\
46.59	0\\
46.6	0\\
46.61	0\\
46.62	0\\
46.63	0\\
46.64	0\\
46.65	0\\
46.66	0\\
46.67	0\\
46.68	0\\
46.69	0\\
46.7	0\\
46.71	0\\
46.72	0\\
46.73	0\\
46.74	0\\
46.75	0\\
46.76	0\\
46.77	0\\
46.78	0\\
46.79	1.73472347597681e-18\\
46.8	0\\
46.81	0\\
46.82	0\\
46.83	0\\
46.84	0\\
46.85	0\\
46.86	0\\
46.87	1.73472347597681e-18\\
46.88	0\\
46.89	0\\
46.9	0\\
46.91	0\\
46.92	0\\
46.93	0\\
46.94	0\\
46.95	0\\
46.96	0\\
46.97	0\\
46.98	0\\
46.99	0\\
47	0\\
47.01	0\\
47.02	0\\
47.03	0\\
47.04	0\\
47.05	0\\
47.06	0\\
47.07	0\\
47.08	1.73472347597681e-18\\
47.09	0\\
47.1	0\\
47.11	0\\
47.12	0\\
47.13	0\\
47.14	0\\
47.15	0\\
47.16	0\\
47.17	0\\
47.18	0\\
47.19	0\\
47.2	0\\
47.21	0\\
47.22	1.73472347597681e-18\\
47.23	0\\
47.24	0\\
47.25	0\\
47.26	0\\
47.27	0\\
47.28	0\\
47.29	0\\
47.3	0\\
47.31	1.73472347597681e-18\\
47.32	0\\
47.33	0\\
47.34	0\\
47.35	0\\
47.36	0\\
47.37	0\\
47.38	1.73472347597681e-18\\
47.39	0\\
47.4	0\\
47.41	0\\
47.42	0\\
47.43	0\\
47.44	0\\
47.45	0\\
47.46	0\\
47.47	0\\
47.48	0\\
47.49	0\\
47.5	0\\
47.51	0\\
47.52	0\\
47.53	0\\
47.54	0\\
47.55	0\\
47.56	0\\
47.57	0\\
47.58	0\\
47.59	0\\
47.6	0\\
47.61	0\\
47.62	0\\
47.63	0\\
47.64	0\\
47.65	0\\
47.66	0\\
47.67	0\\
47.68	0\\
47.69	0\\
47.7	0\\
47.71	0\\
47.72	0\\
47.73	0\\
47.74	0\\
47.75	0\\
47.76	0\\
47.77	0\\
47.78	0\\
47.79	0\\
47.8	0\\
47.81	1.73472347597681e-18\\
47.82	0\\
47.83	1.73472347597681e-18\\
47.84	0\\
47.85	0\\
47.86	0\\
47.87	0\\
47.88	0\\
47.89	0\\
47.9	0\\
47.91	0\\
47.92	0\\
47.93	0\\
47.94	1.73472347597681e-18\\
47.95	0\\
47.96	0\\
47.97	0\\
47.98	0\\
47.99	0\\
48	0\\
48.01	1.73472347597681e-18\\
48.02	0\\
48.03	0\\
48.04	0\\
48.05	0\\
48.06	0\\
48.07	0\\
48.08	0\\
48.09	0\\
48.1	0\\
48.11	0\\
48.12	0\\
48.13	0\\
48.14	0\\
48.15	0\\
48.16	1.73472347597681e-18\\
48.17	0\\
48.18	0\\
48.19	0\\
48.2	1.73472347597681e-18\\
48.21	0\\
48.22	0\\
48.23	0\\
48.24	0\\
48.25	0\\
48.26	0\\
48.27	0\\
48.28	0\\
48.29	0\\
48.3	0\\
48.31	1.73472347597681e-18\\
48.32	0\\
48.33	0\\
48.34	0\\
48.35	0\\
48.36	0\\
48.37	0\\
48.38	1.73472347597681e-18\\
48.39	0\\
48.4	0\\
48.41	0\\
48.42	0\\
48.43	0\\
48.44	0\\
48.45	0\\
48.46	0\\
48.47	0\\
48.48	0\\
48.49	0\\
48.5	0\\
48.51	0\\
48.52	0\\
48.53	1.73472347597681e-18\\
48.54	1.73472347597681e-18\\
48.55	0\\
48.56	0\\
48.57	0\\
48.58	0\\
48.59	0\\
48.6	0\\
48.61	0\\
48.62	0\\
48.63	0\\
48.64	0\\
48.65	1.73472347597681e-18\\
48.66	0\\
48.67	0\\
48.68	0\\
48.69	0\\
48.7	0\\
48.71	0\\
48.72	0\\
48.73	0\\
48.74	0\\
48.75	0\\
48.76	0\\
48.77	0\\
48.78	0\\
48.79	0\\
48.8	0\\
48.81	0\\
48.82	0\\
48.83	0\\
48.84	0\\
48.85	0\\
48.86	0\\
48.87	0\\
48.88	0\\
48.89	0\\
48.9	1.73472347597681e-18\\
48.91	0\\
48.92	0\\
48.93	0\\
48.94	0\\
48.95	0\\
48.96	0\\
48.97	0\\
48.98	0\\
48.99	0\\
49	0\\
49.01	0\\
49.02	0\\
49.03	0\\
49.04	0\\
49.05	0\\
49.06	0\\
49.07	0\\
49.08	1.73472347597681e-18\\
49.09	0\\
49.1	0\\
49.11	0\\
49.12	0\\
49.13	0\\
49.14	0\\
49.15	0\\
49.16	0\\
49.17	0\\
49.18	0\\
49.19	0\\
49.2	0\\
49.21	0\\
49.22	0\\
49.23	0\\
49.24	0\\
49.25	0\\
49.26	0\\
49.27	0\\
49.28	0\\
49.29	0\\
49.3	0\\
49.31	0\\
49.32	0\\
49.33	0\\
49.34	1.73472347597681e-18\\
49.35	0\\
49.36	0\\
49.37	0\\
49.38	0\\
49.39	0\\
49.4	1.73472347597681e-18\\
49.41	0\\
49.42	0\\
49.43	0\\
49.44	0\\
49.45	0\\
49.46	0\\
49.47	0\\
49.48	0\\
49.49	0\\
49.5	0\\
49.51	0\\
49.52	0\\
49.53	0\\
49.54	0\\
49.55	1.73472347597681e-18\\
49.56	0\\
49.57	0\\
49.58	0\\
49.59	1.73472347597681e-18\\
49.6	0\\
49.61	0\\
49.62	0\\
49.63	1.73472347597681e-18\\
49.64	0\\
49.65	0\\
49.66	0\\
49.67	0\\
49.68	0\\
49.69	0\\
49.7	0\\
49.71	0\\
49.72	0\\
49.73	0\\
49.74	0\\
49.75	0\\
49.76	0\\
49.77	0\\
49.78	0\\
49.79	0\\
49.8	0\\
49.81	0\\
49.82	0\\
49.83	0\\
49.84	0\\
49.85	0\\
49.86	0\\
49.87	0\\
49.88	0\\
49.89	0\\
49.9	0\\
49.91	0\\
49.92	0\\
49.93	0\\
49.94	0\\
49.95	0\\
49.96	0\\
49.97	1.73472347597681e-18\\
49.98	0\\
49.99	0\\
50	0\\
50.01	0\\
50.02	0\\
50.03	0\\
50.04	0\\
50.05	0\\
50.06	1.73472347597681e-18\\
50.07	0\\
50.08	0\\
50.09	0\\
50.1	0\\
50.11	0\\
50.12	0\\
50.13	0\\
50.14	0\\
50.15	0\\
50.16	0\\
50.17	0\\
50.18	0\\
50.19	0\\
50.2	0\\
50.21	0\\
50.22	0\\
50.23	0\\
50.24	0\\
50.25	0\\
50.26	0\\
50.27	0\\
50.28	0\\
50.29	0\\
50.3	0\\
50.31	0\\
50.32	0\\
50.33	0\\
50.34	0\\
50.35	0\\
50.36	0\\
50.37	0\\
50.38	0\\
50.39	0\\
50.4	1.73472347597681e-18\\
50.41	0\\
50.42	0\\
50.43	0\\
50.44	0\\
50.45	0\\
50.46	0\\
50.47	0\\
50.48	0\\
50.49	0\\
50.5	0\\
50.51	0\\
50.52	0\\
50.53	0\\
50.54	0\\
50.55	0\\
50.56	0\\
50.57	0\\
50.58	1.73472347597681e-18\\
50.59	0\\
50.6	0\\
50.61	0\\
50.62	0\\
50.63	0\\
50.64	0\\
50.65	0\\
50.66	0\\
50.67	0\\
50.68	0\\
50.69	0\\
50.7	0\\
50.71	0\\
50.72	0\\
50.73	0\\
50.74	0\\
50.75	1.73472347597681e-18\\
50.76	0\\
50.77	0\\
50.78	0\\
50.79	0\\
50.8	0\\
50.81	0\\
50.82	0\\
50.83	0\\
50.84	0\\
50.85	0\\
50.86	0\\
50.87	0\\
50.88	0\\
50.89	0\\
50.9	0\\
50.91	0\\
50.92	0\\
50.93	0\\
50.94	0\\
50.95	0\\
50.96	0\\
50.97	0\\
50.98	0\\
50.99	0\\
51	0\\
51.01	0\\
51.02	0\\
51.03	0\\
51.04	0\\
51.05	0\\
51.06	0\\
51.07	0\\
51.08	1.73472347597681e-18\\
51.09	0\\
51.1	0\\
51.11	0\\
51.12	0\\
51.13	0\\
51.14	0\\
51.15	0\\
51.16	0\\
51.17	0\\
51.18	0\\
51.19	0\\
51.2	0\\
51.21	0\\
51.22	0\\
51.23	0\\
51.24	0\\
51.25	0\\
51.26	0\\
51.27	0\\
51.28	0\\
51.29	0\\
51.3	0\\
51.31	0\\
51.32	0\\
51.33	0\\
51.34	0\\
51.35	0\\
51.36	0\\
51.37	0\\
51.38	0\\
51.39	0\\
51.4	0\\
51.41	0\\
51.42	0\\
51.43	0\\
51.44	0\\
51.45	0\\
51.46	0\\
51.47	0\\
51.48	0\\
51.49	0\\
51.5	0\\
51.51	0\\
51.52	0\\
51.53	0\\
51.54	1.73472347597681e-18\\
51.55	0\\
51.56	0\\
51.57	0\\
51.58	0\\
51.59	0\\
51.6	0\\
51.61	0\\
51.62	0\\
51.63	0\\
51.64	0\\
51.65	0\\
51.66	0\\
51.67	0\\
51.68	0\\
51.69	0\\
51.7	0\\
51.71	0\\
51.72	0\\
51.73	0\\
51.74	0\\
51.75	0\\
51.76	0\\
51.77	0\\
51.78	0\\
51.79	0\\
51.8	0\\
51.81	0\\
51.82	0\\
51.83	0\\
51.84	1.73472347597681e-18\\
51.85	0\\
51.86	0\\
51.87	0\\
51.88	0\\
51.89	0\\
51.9	0\\
51.91	1.73472347597681e-18\\
51.92	0\\
51.93	0\\
51.94	0\\
51.95	0\\
51.96	0\\
51.97	0\\
51.98	0\\
51.99	0\\
52	0\\
52.01	0\\
52.02	0\\
52.03	0\\
52.04	1.73472347597681e-18\\
52.05	0\\
52.06	0\\
52.07	0\\
52.08	0\\
52.09	0\\
52.1	0\\
52.11	0\\
52.12	0\\
52.13	0\\
52.14	0\\
52.15	1.73472347597681e-18\\
52.16	0\\
52.17	0\\
52.18	0\\
52.19	0\\
52.2	0\\
52.21	0\\
52.22	0\\
52.23	0\\
52.24	0\\
52.25	0\\
52.26	0\\
52.27	0\\
52.28	0\\
52.29	0\\
52.3	0\\
52.31	0\\
52.32	0\\
52.33	0\\
52.34	0\\
52.35	0\\
52.36	0\\
52.37	1.73472347597681e-18\\
52.38	0\\
52.39	0\\
52.4	0\\
52.41	0\\
52.42	0\\
52.43	0\\
52.44	0\\
52.45	0\\
52.46	0\\
52.47	0\\
52.48	0\\
52.49	0\\
52.5	1.73472347597681e-18\\
52.51	0\\
52.52	0\\
52.53	0\\
52.54	0\\
52.55	0\\
52.56	0\\
52.57	0\\
52.58	0\\
52.59	1.73472347597681e-18\\
52.6	0\\
52.61	0\\
52.62	0\\
52.63	0\\
52.64	0\\
52.65	1.73472347597681e-18\\
52.66	0\\
52.67	0\\
52.68	0\\
52.69	0\\
52.7	0\\
52.71	0\\
52.72	0\\
52.73	0\\
52.74	0\\
52.75	0\\
52.76	0\\
52.77	0\\
52.78	0\\
52.79	0\\
52.8	0\\
52.81	0\\
52.82	0\\
52.83	0\\
52.84	0\\
52.85	0\\
52.86	0\\
52.87	0\\
52.88	0\\
52.89	0\\
52.9	0\\
52.91	0\\
52.92	1.73472347597681e-18\\
52.93	0\\
52.94	0\\
52.95	0\\
52.96	0\\
52.97	0\\
52.98	0\\
52.99	0\\
53	0\\
53.01	0\\
53.02	0\\
53.03	0\\
53.04	0\\
53.05	0\\
53.06	0\\
53.07	0\\
53.08	0\\
53.09	0\\
53.1	0\\
53.11	0\\
53.12	0\\
53.13	0\\
53.14	0\\
53.15	0\\
53.16	0\\
53.17	0\\
53.18	0\\
53.19	0\\
53.2	0\\
53.21	0\\
53.22	0\\
53.23	0\\
53.24	0\\
53.25	0\\
53.26	0\\
53.27	0\\
53.28	0\\
53.29	0\\
53.3	0\\
53.31	0\\
53.32	0\\
53.33	0\\
53.34	0\\
53.35	0\\
53.36	0\\
53.37	0\\
53.38	0\\
53.39	0\\
53.4	0\\
53.41	0\\
53.42	0\\
53.43	0\\
53.44	0\\
53.45	0\\
53.46	0\\
53.47	1.73472347597681e-18\\
53.48	0\\
53.49	0\\
53.5	0\\
53.51	0\\
53.52	0\\
53.53	0\\
53.54	0\\
53.55	0\\
53.56	0\\
53.57	0\\
53.58	0\\
53.59	0\\
53.6	0\\
53.61	1.73472347597681e-18\\
53.62	0\\
53.63	0\\
53.64	0\\
53.65	0\\
53.66	0\\
53.67	0\\
53.68	0\\
53.69	0\\
53.7	0\\
53.71	0\\
53.72	0\\
53.73	0\\
53.74	0\\
53.75	0\\
53.76	0\\
53.77	0\\
53.78	0\\
53.79	0\\
53.8	0\\
53.81	1.73472347597681e-18\\
53.82	0\\
53.83	0\\
53.84	1.73472347597681e-18\\
53.85	0\\
53.86	0\\
53.87	0\\
53.88	0\\
53.89	0\\
53.9	0\\
53.91	0\\
53.92	0\\
53.93	0\\
53.94	0\\
53.95	0\\
53.96	0\\
53.97	1.73472347597681e-18\\
53.98	0\\
53.99	0\\
54	0\\
54.01	0\\
54.02	0\\
54.03	0\\
54.04	0\\
54.05	0\\
54.06	0\\
54.07	0\\
54.08	0\\
54.09	0\\
54.1	0\\
54.11	0\\
54.12	0\\
54.13	0\\
54.14	0\\
54.15	0\\
54.16	0\\
54.17	0\\
54.18	0\\
54.19	0\\
54.2	0\\
54.21	0\\
54.22	0\\
54.23	0\\
54.24	0\\
54.25	0\\
54.26	0\\
54.27	0\\
54.28	0\\
54.29	0\\
54.3	0\\
54.31	0\\
54.32	0\\
54.33	0\\
54.34	0\\
54.35	0\\
54.36	0\\
54.37	0\\
54.38	0\\
54.39	0\\
54.4	0\\
54.41	0\\
54.42	0\\
54.43	0\\
54.44	0\\
54.45	0\\
54.46	0\\
54.47	0\\
54.48	0\\
54.49	0\\
54.5	0\\
54.51	0\\
54.52	0\\
54.53	0\\
54.54	0\\
54.55	0\\
54.56	0\\
54.57	0\\
54.58	0\\
54.59	0\\
54.6	0\\
54.61	0\\
54.62	0\\
54.63	0\\
54.64	0\\
54.65	0\\
54.66	0\\
54.67	0\\
54.68	0\\
54.69	0\\
54.7	0\\
54.71	0\\
54.72	1.73472347597681e-18\\
54.73	0\\
54.74	0\\
54.75	0\\
54.76	0\\
54.77	0\\
54.78	0\\
54.79	0\\
54.8	0\\
54.81	0\\
54.82	0\\
54.83	0\\
54.84	0\\
54.85	0\\
54.86	0\\
54.87	0\\
54.88	0\\
54.89	0\\
54.9	0\\
54.91	0\\
54.92	0\\
54.93	0\\
54.94	0\\
54.95	0\\
54.96	0\\
54.97	0\\
54.98	0\\
54.99	0\\
55	0\\
55.01	0\\
55.02	0\\
55.03	0\\
55.04	0\\
55.05	0\\
55.06	0\\
55.07	0\\
55.08	0\\
55.09	0\\
55.1	0\\
55.11	0\\
55.12	0\\
55.13	0\\
55.14	1.73472347597681e-18\\
55.15	0\\
55.16	0\\
55.17	0\\
55.18	0\\
55.19	0\\
55.2	0\\
55.21	1.73472347597681e-18\\
55.22	0\\
55.23	0\\
55.24	0\\
55.25	0\\
55.26	0\\
55.27	0\\
55.28	0\\
55.29	0\\
55.3	0\\
55.31	0\\
55.32	0\\
55.33	0\\
55.34	0\\
55.35	0\\
55.36	0\\
55.37	0\\
55.38	0\\
55.39	0\\
55.4	0\\
55.41	0\\
55.42	0\\
55.43	1.73472347597681e-18\\
55.44	0\\
55.45	0\\
55.46	0\\
55.47	0\\
55.48	0\\
55.49	0\\
55.5	0\\
55.51	0\\
55.52	0\\
55.53	0\\
55.54	0\\
55.55	0\\
55.56	1.73472347597681e-18\\
55.57	0\\
55.58	0\\
55.59	0\\
55.6	0\\
55.61	0\\
55.62	0\\
55.63	0\\
55.64	0\\
55.65	0\\
55.66	1.73472347597681e-18\\
55.67	0\\
55.68	0\\
55.69	0\\
55.7	1.73472347597681e-18\\
55.71	0\\
55.72	0\\
55.73	0\\
55.74	0\\
55.75	0\\
55.76	1.73472347597681e-18\\
55.77	0\\
55.78	0\\
55.79	0\\
55.8	0\\
55.81	0\\
55.82	1.73472347597681e-18\\
55.83	0\\
55.84	0\\
55.85	0\\
55.86	0\\
55.87	0\\
55.88	0\\
55.89	0\\
55.9	0\\
55.91	0\\
55.92	0\\
55.93	0\\
55.94	0\\
55.95	0\\
55.96	1.73472347597681e-18\\
55.97	0\\
55.98	0\\
55.99	0\\
56	0\\
56.01	0\\
56.02	0\\
56.03	0\\
56.04	0\\
56.05	0\\
56.06	0\\
56.07	0\\
56.08	0\\
56.09	0\\
56.1	0\\
56.11	0\\
56.12	0\\
56.13	0\\
56.14	0\\
56.15	0\\
56.16	0\\
56.17	0\\
56.18	0\\
56.19	0\\
56.2	0\\
56.21	0\\
56.22	0\\
56.23	0\\
56.24	0\\
56.25	0\\
56.26	0\\
56.27	0\\
56.28	0\\
56.29	0\\
56.3	0\\
56.31	0\\
56.32	0\\
56.33	0\\
56.34	0\\
56.35	0\\
56.36	0\\
56.37	0\\
56.38	0\\
56.39	0\\
56.4	0\\
56.41	0\\
56.42	0\\
56.43	0\\
56.44	0\\
56.45	0\\
56.46	0\\
56.47	0\\
56.48	0\\
56.49	0\\
56.5	1.73472347597681e-18\\
56.51	0\\
56.52	0\\
56.53	0\\
56.54	0\\
56.55	0\\
56.56	0\\
56.57	0\\
56.58	0\\
56.59	0\\
56.6	0\\
56.61	0\\
56.62	0\\
56.63	0\\
56.64	0\\
56.65	0\\
56.66	0\\
56.67	0\\
56.68	0\\
56.69	0\\
56.7	1.73472347597681e-18\\
56.71	0\\
56.72	0\\
56.73	0\\
56.74	0\\
56.75	0\\
56.76	0\\
56.77	0\\
56.78	0\\
56.79	0\\
56.8	0\\
56.81	0\\
56.82	0\\
56.83	0\\
56.84	0\\
56.85	0\\
56.86	0\\
56.87	1.73472347597681e-18\\
56.88	0\\
56.89	0\\
56.9	1.73472347597681e-18\\
56.91	0\\
56.92	0\\
56.93	0\\
56.94	0\\
56.95	0\\
56.96	0\\
56.97	0\\
56.98	0\\
56.99	0\\
57	0\\
57.01	0\\
57.02	0\\
57.03	0\\
57.04	0\\
57.05	0\\
57.06	0\\
57.07	0\\
57.08	0\\
57.09	0\\
57.1	0\\
57.11	0\\
57.12	0\\
57.13	1.73472347597681e-18\\
57.14	0\\
57.15	0\\
57.16	0\\
57.17	0\\
57.18	0\\
57.19	0\\
57.2	0\\
57.21	0\\
57.22	0\\
57.23	0\\
57.24	0\\
57.25	0\\
57.26	0\\
57.27	0\\
57.28	0\\
57.29	0\\
57.3	0\\
57.31	0\\
57.32	0\\
57.33	0\\
57.34	0\\
57.35	1.73472347597681e-18\\
57.36	1.73472347597681e-18\\
57.37	0\\
57.38	0\\
57.39	0\\
57.4	0\\
57.41	1.73472347597681e-18\\
57.42	0\\
57.43	0\\
57.44	0\\
57.45	1.73472347597681e-18\\
57.46	0\\
57.47	0\\
57.48	0\\
57.49	0\\
57.5	0\\
57.51	0\\
57.52	0\\
57.53	0\\
57.54	0\\
57.55	0\\
57.56	0\\
57.57	0\\
57.58	0\\
57.59	0\\
57.6	0\\
57.61	0\\
57.62	0\\
57.63	0\\
57.64	0\\
57.65	1.73472347597681e-18\\
57.66	0\\
57.67	0\\
57.68	0\\
57.69	0\\
57.7	0\\
57.71	0\\
57.72	0\\
57.73	0\\
57.74	0\\
57.75	0\\
57.76	0\\
57.77	0\\
57.78	0\\
57.79	0\\
57.8	0\\
57.81	0\\
57.82	0\\
57.83	0\\
57.84	0\\
57.85	0\\
57.86	0\\
57.87	0\\
57.88	1.73472347597681e-18\\
57.89	1.73472347597681e-18\\
57.9	0\\
57.91	0\\
57.92	0\\
57.93	0\\
57.94	0\\
57.95	0\\
57.96	0\\
57.97	0\\
57.98	0\\
57.99	0\\
58	0\\
58.01	0\\
58.02	0\\
58.03	0\\
58.04	0\\
58.05	0\\
58.06	0\\
58.07	0\\
58.08	0\\
58.09	0\\
58.1	0\\
58.11	0\\
58.12	0\\
58.13	0\\
58.14	0\\
58.15	0\\
58.16	0\\
58.17	0\\
58.18	0\\
58.19	1.73472347597681e-18\\
58.2	0\\
58.21	0\\
58.22	0\\
58.23	0\\
58.24	0\\
58.25	0\\
58.26	0\\
58.27	0\\
58.28	0\\
58.29	0\\
58.3	0\\
58.31	0\\
58.32	0\\
58.33	0\\
58.34	0\\
58.35	0\\
58.36	0\\
58.37	0\\
58.38	0\\
58.39	0\\
58.4	0\\
58.41	0\\
58.42	0\\
58.43	0\\
58.44	0\\
58.45	0\\
58.46	0\\
58.47	0\\
58.48	0\\
58.49	0\\
58.5	0\\
58.51	0\\
58.52	0\\
58.53	0\\
58.54	0\\
58.55	0\\
58.56	0\\
58.57	0\\
58.58	0\\
58.59	0\\
58.6	0\\
58.61	0\\
58.62	0\\
58.63	0\\
58.64	0\\
58.65	0\\
58.66	0\\
58.67	0\\
58.68	0\\
58.69	0\\
58.7	0\\
58.71	0\\
58.72	0\\
58.73	0\\
58.74	0\\
58.75	0\\
58.76	0\\
58.77	0\\
58.78	1.73472347597681e-18\\
58.79	0\\
58.8	0\\
58.81	0\\
58.82	0\\
58.83	0\\
58.84	0\\
58.85	0\\
58.86	1.73472347597681e-18\\
58.87	0\\
58.88	0\\
58.89	0\\
58.9	0\\
58.91	0\\
58.92	0\\
58.93	0\\
58.94	0\\
58.95	0\\
58.96	0\\
58.97	0\\
58.98	0\\
58.99	0\\
59	0\\
59.01	0\\
59.02	0\\
59.03	0\\
59.04	0\\
59.05	0\\
59.06	0\\
59.07	0\\
59.08	0\\
59.09	0\\
59.1	0\\
59.11	0\\
59.12	0\\
59.13	0\\
59.14	0\\
59.15	0\\
59.16	0\\
59.17	0\\
59.18	1.73472347597681e-18\\
59.19	0\\
59.2	0\\
59.21	0\\
59.22	0\\
59.23	0\\
59.24	0\\
59.25	0\\
59.26	0\\
59.27	0\\
59.28	0\\
59.29	0\\
59.3	0\\
59.31	0\\
59.32	0\\
59.33	0\\
59.34	0\\
59.35	0\\
59.36	0\\
59.37	0\\
59.38	0\\
59.39	0\\
59.4	0\\
59.41	0\\
59.42	0\\
59.43	0\\
59.44	0\\
59.45	0\\
59.46	0\\
59.47	0\\
59.48	0\\
59.49	0\\
59.5	0\\
59.51	0\\
59.52	0\\
59.53	0\\
59.54	0\\
59.55	0\\
59.56	0\\
59.57	0\\
59.58	0\\
59.59	0\\
59.6	0\\
59.61	0\\
59.62	0\\
59.63	0\\
59.64	0\\
59.65	0\\
59.66	0\\
59.67	0\\
59.68	1.73472347597681e-18\\
59.69	0\\
59.7	0\\
59.71	0\\
59.72	0\\
59.73	0\\
59.74	0\\
59.75	0\\
59.76	0\\
59.77	0\\
59.78	0\\
59.79	0\\
59.8	0\\
59.81	0\\
59.82	0\\
59.83	0\\
59.84	0\\
59.85	0\\
59.86	0\\
59.87	0\\
59.88	0\\
59.89	0\\
59.9	0\\
59.91	0\\
59.92	0\\
59.93	0\\
59.94	1.73472347597681e-18\\
59.95	0\\
59.96	0\\
59.97	0\\
59.98	0\\
59.99	0\\
60	0\\
60.01	0\\
60.02	0\\
60.03	0\\
60.04	0\\
60.05	1.73472347597681e-18\\
60.06	0\\
60.07	0\\
60.08	1.73472347597681e-18\\
60.09	0\\
60.1	0\\
60.11	0\\
60.12	0\\
60.13	0\\
60.14	1.73472347597681e-18\\
60.15	0\\
60.16	0\\
60.17	0\\
60.18	0\\
60.19	0\\
60.2	1.73472347597681e-18\\
60.21	0\\
60.22	0\\
60.23	0\\
60.24	0\\
60.25	0\\
60.26	0\\
60.27	0\\
60.28	0\\
60.29	0\\
60.3	0\\
60.31	0\\
60.32	0\\
60.33	0\\
60.34	0\\
60.35	0\\
60.36	0\\
60.37	0\\
60.38	0\\
60.39	0\\
60.4	0\\
60.41	0\\
60.42	0\\
60.43	0\\
60.44	0\\
60.45	0\\
60.46	0\\
60.47	0\\
60.48	0\\
60.49	0\\
60.5	0\\
60.51	0\\
60.52	0\\
60.53	0\\
60.54	0\\
60.55	0\\
60.56	0\\
60.57	0\\
60.58	0\\
60.59	0\\
60.6	0\\
60.61	0\\
60.62	0\\
60.63	1.73472347597681e-18\\
60.64	0\\
60.65	0\\
60.66	0\\
60.67	0\\
60.68	0\\
60.69	0\\
60.7	0\\
60.71	0\\
60.72	0\\
60.73	0\\
60.74	0\\
60.75	0\\
60.76	0\\
60.77	0\\
60.78	0\\
60.79	0\\
60.8	0\\
60.81	0\\
60.82	0\\
60.83	0\\
60.84	0\\
60.85	0\\
60.86	0\\
60.87	0\\
60.88	0\\
60.89	0\\
60.9	0\\
60.91	0\\
60.92	1.73472347597681e-18\\
60.93	0\\
60.94	0\\
60.95	0\\
60.96	1.73472347597681e-18\\
60.97	0\\
60.98	0\\
60.99	0\\
61	0\\
61.01	0\\
61.02	0\\
61.03	0\\
61.04	0\\
61.05	0\\
61.06	0\\
61.07	0\\
61.08	0\\
61.09	0\\
61.1	0\\
61.11	0\\
61.12	0\\
61.13	0\\
61.14	0\\
61.15	0\\
61.16	0\\
61.17	0\\
61.18	0\\
61.19	0\\
61.2	0\\
61.21	0\\
61.22	0\\
61.23	0\\
61.24	0\\
61.25	0\\
61.26	0\\
61.27	0\\
61.28	0\\
61.29	0\\
61.3	0\\
61.31	0\\
61.32	0\\
61.33	0\\
61.34	0\\
61.35	0\\
61.36	0\\
61.37	0\\
61.38	0\\
61.39	0\\
61.4	0\\
61.41	0\\
61.42	0\\
61.43	0\\
61.44	0\\
61.45	0\\
61.46	0\\
61.47	0\\
61.48	0\\
61.49	0\\
61.5	0\\
61.51	0\\
61.52	0\\
61.53	0\\
61.54	0\\
61.55	0\\
61.56	0\\
61.57	0\\
61.58	0\\
61.59	0\\
61.6	0\\
61.61	0\\
61.62	0\\
61.63	0\\
61.64	0\\
61.65	0\\
61.66	0\\
61.67	1.73472347597681e-18\\
61.68	1.73472347597681e-18\\
61.69	0\\
61.7	0\\
61.71	0\\
61.72	0\\
61.73	0\\
61.74	0\\
61.75	0\\
61.76	0\\
61.77	0\\
61.78	0\\
61.79	0\\
61.8	0\\
61.81	0\\
61.82	0\\
61.83	0\\
61.84	0\\
61.85	0\\
61.86	0\\
61.87	0\\
61.88	0\\
61.89	0\\
61.9	0\\
61.91	0\\
61.92	0\\
61.93	0\\
61.94	0\\
61.95	0\\
61.96	0\\
61.97	0\\
61.98	0\\
61.99	0\\
62	0\\
62.01	0\\
62.02	0\\
62.03	0\\
62.04	0\\
62.05	0\\
62.06	0\\
62.07	0\\
62.08	0\\
62.09	0\\
62.1	0\\
62.11	0\\
62.12	0\\
62.13	0\\
62.14	0\\
62.15	0\\
62.16	0\\
62.17	0\\
62.18	0\\
62.19	0\\
62.2	0\\
62.21	0\\
62.22	0\\
62.23	1.73472347597681e-18\\
62.24	0\\
62.25	0\\
62.26	0\\
62.27	0\\
62.28	0\\
62.29	0\\
62.3	1.73472347597681e-18\\
62.31	0\\
62.32	0\\
62.33	1.73472347597681e-18\\
62.34	0\\
62.35	0\\
62.36	0\\
62.37	0\\
62.38	0\\
62.39	0\\
62.4	0\\
62.41	0\\
62.42	0\\
62.43	0\\
62.44	0\\
62.45	0\\
62.46	0\\
62.47	0\\
62.48	0\\
62.49	0\\
62.5	0\\
62.51	0\\
62.52	0\\
62.53	1.73472347597681e-18\\
62.54	0\\
62.55	0\\
62.56	0\\
62.57	0\\
62.58	0\\
62.59	0\\
62.6	0\\
62.61	1.73472347597681e-18\\
62.62	0\\
62.63	0\\
62.64	0\\
62.65	0\\
62.66	0\\
62.67	0\\
62.68	0\\
62.69	0\\
62.7	0\\
62.71	0\\
62.72	0\\
62.73	0\\
62.74	0\\
62.75	0\\
62.76	0\\
62.77	0\\
62.78	0\\
62.79	1.73472347597681e-18\\
62.8	0\\
62.81	0\\
62.82	0\\
62.83	1.73472347597681e-18\\
62.84	0\\
62.85	0\\
62.86	0\\
62.87	0\\
62.88	0\\
62.89	1.73472347597681e-18\\
62.9	0\\
62.91	0\\
62.92	0\\
62.93	0\\
62.94	0\\
62.95	0\\
62.96	0\\
62.97	0\\
62.98	0\\
62.99	0\\
63	0\\
63.01	0\\
63.02	0\\
63.03	0\\
63.04	0\\
63.05	0\\
63.06	0\\
63.07	0\\
63.08	0\\
63.09	0\\
63.1	0\\
63.11	0\\
63.12	0\\
63.13	0\\
63.14	0\\
63.15	0\\
63.16	0\\
63.17	1.73472347597681e-18\\
63.18	0\\
63.19	0\\
63.2	0\\
63.21	0\\
63.22	0\\
63.23	0\\
63.24	0\\
63.25	1.73472347597681e-18\\
63.26	0\\
63.27	1.73472347597681e-18\\
63.28	0\\
63.29	1.73472347597681e-18\\
63.3	0\\
63.31	1.73472347597681e-18\\
63.32	0\\
63.33	0\\
63.34	0\\
63.35	0\\
63.36	0\\
63.37	0\\
63.38	0\\
63.39	0\\
63.4	0\\
63.41	0\\
63.42	0\\
63.43	0\\
63.44	0\\
63.45	0\\
63.46	0\\
63.47	0\\
63.48	0\\
63.49	0\\
63.5	0\\
63.51	0\\
63.52	0\\
63.53	0\\
63.54	0\\
63.55	0\\
63.56	0\\
63.57	1.73472347597681e-18\\
63.58	0\\
63.59	0\\
63.6	0\\
63.61	0\\
63.62	0\\
63.63	0\\
63.64	0\\
63.65	0\\
63.66	0\\
63.67	0\\
63.68	0\\
63.69	0\\
63.7	0\\
63.71	0\\
63.72	0\\
63.73	0\\
63.74	0\\
63.75	0\\
63.76	0\\
63.77	0\\
63.78	0\\
63.79	0\\
63.8	0\\
63.81	0\\
63.82	0\\
63.83	0\\
63.84	0\\
63.85	0\\
63.86	0\\
63.87	1.73472347597681e-18\\
63.88	0\\
63.89	1.73472347597681e-18\\
63.9	0\\
63.91	0\\
63.92	0\\
63.93	0\\
63.94	0\\
63.95	0\\
63.96	0\\
63.97	0\\
63.98	0\\
63.99	0\\
64	0\\
64.01	0\\
64.02	0\\
64.03	0\\
64.04	0\\
64.05	0\\
64.06	0\\
64.07	0\\
64.08	0\\
64.09	0\\
64.1	0\\
64.11	0\\
64.12	0\\
64.13	0\\
64.14	0\\
64.15	0\\
64.16	0\\
64.17	1.73472347597681e-18\\
64.18	0\\
64.19	0\\
64.2	0\\
64.21	0\\
64.22	0\\
64.23	0\\
64.24	0\\
64.25	0\\
64.26	0\\
64.27	0\\
64.28	0\\
64.29	0\\
64.3	0\\
64.31	0\\
64.32	0\\
64.33	0\\
64.34	0\\
64.35	0\\
64.36	0\\
64.37	0\\
64.38	0\\
64.39	0\\
64.4	0\\
64.41	0\\
64.42	0\\
64.43	0\\
64.44	0\\
64.45	0\\
64.46	0\\
64.47	0\\
64.48	0\\
64.49	0\\
64.5	0\\
64.51	0\\
64.52	0\\
64.53	0\\
64.54	0\\
64.55	0\\
64.56	0\\
64.57	0\\
64.58	0\\
64.59	0\\
64.6	0\\
64.61	0\\
64.62	0\\
64.63	0\\
64.64	1.73472347597681e-18\\
64.65	0\\
64.66	0\\
64.67	0\\
64.68	0\\
64.69	0\\
64.7	1.73472347597681e-18\\
64.71	0\\
64.72	0\\
64.73	0\\
64.74	0\\
64.75	0\\
64.76	0\\
64.77	1.73472347597681e-18\\
64.78	0\\
64.79	0\\
64.8	0\\
64.81	0\\
64.82	0\\
64.83	0\\
64.84	1.73472347597681e-18\\
64.85	0\\
64.86	0\\
64.87	0\\
64.88	0\\
64.89	0\\
64.9	0\\
64.91	0\\
64.92	0\\
64.93	1.73472347597681e-18\\
64.94	0\\
64.95	0\\
64.96	0\\
64.97	0\\
64.98	0\\
64.99	0\\
65	1.73472347597681e-18\\
65.01	0\\
65.02	0\\
65.03	0\\
65.04	0\\
65.05	0\\
65.06	0\\
65.07	0\\
65.08	0\\
65.09	0\\
65.1	0\\
65.11	0\\
65.12	0\\
65.13	0\\
65.14	0\\
65.15	0\\
65.16	1.73472347597681e-18\\
65.17	0\\
65.18	0\\
65.19	0\\
65.2	0\\
65.21	0\\
65.22	0\\
65.23	0\\
65.24	0\\
65.25	1.73472347597681e-18\\
65.26	1.73472347597681e-18\\
65.27	0\\
65.28	0\\
65.29	0\\
65.3	0\\
65.31	0\\
65.32	0\\
65.33	0\\
65.34	0\\
65.35	0\\
65.36	0\\
65.37	0\\
65.38	0\\
65.39	0\\
65.4	0\\
65.41	1.73472347597681e-18\\
65.42	0\\
65.43	0\\
65.44	0\\
65.45	0\\
65.46	1.73472347597681e-18\\
65.47	0\\
65.48	0\\
65.49	0\\
65.5	0\\
65.51	0\\
65.52	0\\
65.53	0\\
65.54	0\\
65.55	0\\
65.56	0\\
65.57	0\\
65.58	0\\
65.59	0\\
65.6	0\\
65.61	0\\
65.62	0\\
65.63	0\\
65.64	0\\
65.65	0\\
65.66	0\\
65.67	0\\
65.68	0\\
65.69	0\\
65.7	0\\
65.71	0\\
65.72	0\\
65.73	0\\
65.74	0\\
65.75	0\\
65.76	0\\
65.77	1.73472347597681e-18\\
65.78	0\\
65.79	0\\
65.8	0\\
65.81	0\\
65.82	1.73472347597681e-18\\
65.83	1.73472347597681e-18\\
65.84	0\\
65.85	0\\
65.86	1.73472347597681e-18\\
65.87	0\\
65.88	0\\
65.89	0\\
65.9	0\\
65.91	0\\
65.92	0\\
65.93	0\\
65.94	0\\
65.95	0\\
65.96	0\\
65.97	0\\
65.98	0\\
65.99	0\\
66	0\\
66.01	0\\
66.02	0\\
66.03	0\\
66.04	1.73472347597681e-18\\
66.05	0\\
66.06	0\\
66.07	0\\
66.08	0\\
66.09	0\\
66.1	0\\
66.11	0\\
66.12	0\\
66.13	1.73472347597681e-18\\
66.14	0\\
66.15	1.73472347597681e-18\\
66.16	0\\
66.17	0\\
66.18	0\\
66.19	0\\
66.2	0\\
66.21	0\\
66.22	0\\
66.23	0\\
66.24	0\\
66.25	0\\
66.26	0\\
66.27	0\\
66.28	0\\
66.29	0\\
66.3	0\\
66.31	0\\
66.32	0\\
66.33	0\\
66.34	0\\
66.35	0\\
66.36	0\\
66.37	0\\
66.38	0\\
66.39	0\\
66.4	1.73472347597681e-18\\
66.41	0\\
66.42	0\\
66.43	0\\
66.44	0\\
66.45	0\\
66.46	0\\
66.47	0\\
66.48	0\\
66.49	0\\
66.5	0\\
66.51	0\\
66.52	0\\
66.53	0\\
66.54	0\\
66.55	0\\
66.56	0\\
66.57	0\\
66.58	0\\
66.59	0\\
66.6	0\\
66.61	0\\
66.62	0\\
66.63	0\\
66.64	0\\
66.65	0\\
66.66	0\\
66.67	0\\
66.68	0\\
66.69	0\\
66.7	0\\
66.71	0\\
66.72	0\\
66.73	0\\
66.74	0\\
66.75	0\\
66.76	0\\
66.77	0\\
66.78	0\\
66.79	1.73472347597681e-18\\
66.8	0\\
66.81	0\\
66.82	0\\
66.83	1.73472347597681e-18\\
66.84	0\\
66.85	0\\
66.86	0\\
66.87	0\\
66.88	0\\
66.89	0\\
66.9	0\\
66.91	0\\
66.92	0\\
66.93	0\\
66.94	0\\
66.95	0\\
66.96	0\\
66.97	1.73472347597681e-18\\
66.98	0\\
66.99	0\\
67	0\\
67.01	0\\
67.02	0\\
67.03	0\\
67.04	0\\
67.05	0\\
67.06	0\\
67.07	0\\
67.08	0\\
67.09	0\\
67.1	0\\
67.11	0\\
67.12	0\\
67.13	0\\
67.14	0\\
67.15	0\\
67.16	0\\
67.17	0\\
67.18	0\\
67.19	0\\
67.2	0\\
67.21	0\\
67.22	0\\
67.23	0\\
67.24	0\\
67.25	0\\
67.26	1.73472347597681e-18\\
67.27	0\\
67.28	0\\
67.29	0\\
67.3	0\\
67.31	0\\
67.32	0\\
67.33	0\\
67.34	0\\
67.35	0\\
67.36	0\\
67.37	0\\
67.38	0\\
67.39	0\\
67.4	0\\
67.41	0\\
67.42	0\\
67.43	0\\
67.44	0\\
67.45	0\\
67.46	0\\
67.47	0\\
67.48	0\\
67.49	0\\
67.5	0\\
67.51	0\\
67.52	0\\
67.53	0\\
67.54	0\\
67.55	0\\
67.56	0\\
67.57	0\\
67.58	0\\
67.59	0\\
67.6	0\\
67.61	0\\
67.62	0\\
67.63	0\\
67.64	0\\
67.65	0\\
67.66	0\\
67.67	0\\
67.68	0\\
67.69	0\\
67.7	0\\
67.71	0\\
67.72	0\\
67.73	0\\
67.74	1.73472347597681e-18\\
67.75	0\\
67.76	0\\
67.77	0\\
67.78	0\\
67.79	0\\
67.8	0\\
67.81	0\\
67.82	0\\
67.83	0\\
67.84	0\\
67.85	0\\
67.86	0\\
67.87	0\\
67.88	0\\
67.89	0\\
67.9	0\\
67.91	0\\
67.92	1.73472347597681e-18\\
67.93	0\\
67.94	0\\
67.95	0\\
67.96	0\\
67.97	0\\
67.98	0\\
67.99	0\\
68	1.73472347597681e-18\\
68.01	0\\
68.02	0\\
68.03	0\\
68.04	0\\
68.05	0\\
68.06	0\\
68.07	0\\
68.08	0\\
68.09	0\\
68.1	0\\
68.11	0\\
68.12	0\\
68.13	0\\
68.14	0\\
68.15	0\\
68.16	0\\
68.17	0\\
68.18	0\\
68.19	0\\
68.2	0\\
68.21	0\\
68.22	0\\
68.23	0\\
68.24	1.73472347597681e-18\\
68.25	0\\
68.26	1.73472347597681e-18\\
68.27	0\\
68.28	0\\
68.29	0\\
68.3	0\\
68.31	0\\
68.32	0\\
68.33	0\\
68.34	0\\
68.35	1.73472347597681e-18\\
68.36	0\\
68.37	0\\
68.38	0\\
68.39	0\\
68.4	0\\
68.41	0\\
68.42	0\\
68.43	0\\
68.44	0\\
68.45	0\\
68.46	0\\
68.47	0\\
68.48	0\\
68.49	0\\
68.5	0\\
68.51	0\\
68.52	0\\
68.53	0\\
68.54	0\\
68.55	0\\
68.56	0\\
68.57	0\\
68.58	0\\
68.59	0\\
68.6	0\\
68.61	0\\
68.62	0\\
68.63	0\\
68.64	0\\
68.65	1.73472347597681e-18\\
68.66	0\\
68.67	0\\
68.68	0\\
68.69	0\\
68.7	0\\
68.71	0\\
68.72	0\\
68.73	0\\
68.74	0\\
68.75	0\\
68.76	0\\
68.77	0\\
68.78	0\\
68.79	0\\
68.8	0\\
68.81	1.73472347597681e-18\\
68.82	0\\
68.83	0\\
68.84	0\\
68.85	0\\
68.86	0\\
68.87	0\\
68.88	0\\
68.89	0\\
68.9	0\\
68.91	0\\
68.92	0\\
68.93	0\\
68.94	0\\
68.95	0\\
68.96	1.73472347597681e-18\\
68.97	0\\
68.98	0\\
68.99	0\\
69	0\\
69.01	0\\
69.02	0\\
69.03	0\\
69.04	0\\
69.05	0\\
69.06	0\\
69.07	0\\
69.08	0\\
69.09	0\\
69.1	0\\
69.11	0\\
69.12	0\\
69.13	0\\
69.14	0\\
69.15	0\\
69.16	1.73472347597681e-18\\
69.17	0\\
69.18	0\\
69.19	0\\
69.2	0\\
69.21	1.73472347597681e-18\\
69.22	0\\
69.23	0\\
69.24	0\\
69.25	0\\
69.26	0\\
69.27	0\\
69.28	0\\
69.29	0\\
69.3	0\\
69.31	0\\
69.32	0\\
69.33	0\\
69.34	0\\
69.35	0\\
69.36	0\\
69.37	0\\
69.38	0\\
69.39	0\\
69.4	0\\
69.41	0\\
69.42	0\\
69.43	0\\
69.44	0\\
69.45	0\\
69.46	0\\
69.47	0\\
69.48	0\\
69.49	0\\
69.5	0\\
69.51	0\\
69.52	0\\
69.53	0\\
69.54	0\\
69.55	0\\
69.56	0\\
69.57	0\\
69.58	0\\
69.59	0\\
69.6	0\\
69.61	0\\
69.62	0\\
69.63	0\\
69.64	0\\
69.65	0\\
69.66	0\\
69.67	0\\
69.68	1.73472347597681e-18\\
69.69	0\\
69.7	0\\
69.71	0\\
69.72	0\\
69.73	0\\
69.74	0\\
69.75	0\\
69.76	0\\
69.77	0\\
69.78	1.73472347597681e-18\\
69.79	0\\
69.8	0\\
69.81	0\\
69.82	0\\
69.83	0\\
69.84	0\\
69.85	0\\
69.86	0\\
69.87	0\\
69.88	0\\
69.89	0\\
69.9	0\\
69.91	0\\
69.92	0\\
69.93	0\\
69.94	0\\
69.95	0\\
69.96	0\\
69.97	0\\
69.98	0\\
69.99	0\\
70	0\\
70.01	0\\
70.02	0\\
70.03	0\\
70.04	0\\
70.05	0\\
70.06	0\\
70.07	0\\
70.08	1.73472347597681e-18\\
70.09	0\\
70.1	0\\
70.11	0\\
70.12	0\\
70.13	0\\
70.14	0\\
70.15	0\\
70.16	0\\
70.17	0\\
70.18	1.73472347597681e-18\\
70.19	0\\
70.2	1.73472347597681e-18\\
70.21	0\\
70.22	0\\
70.23	0\\
70.24	0\\
70.25	0\\
70.26	0\\
70.27	0\\
70.28	0\\
70.29	0\\
70.3	0\\
70.31	0\\
70.32	0\\
70.33	0\\
70.34	0\\
70.35	0\\
70.36	0\\
70.37	0\\
70.38	0\\
70.39	0\\
70.4	0\\
70.41	1.73472347597681e-18\\
70.42	0\\
70.43	0\\
70.44	0\\
70.45	0\\
70.46	0\\
70.47	0\\
70.48	0\\
70.49	0\\
70.5	0\\
70.51	0\\
70.52	0\\
70.53	0\\
70.54	0\\
70.55	0\\
70.56	0\\
70.57	0\\
70.58	0\\
70.59	0\\
70.6	0\\
70.61	0\\
70.62	0\\
70.63	0\\
70.64	0\\
70.65	0\\
70.66	0\\
70.67	0\\
70.68	0\\
70.69	1.73472347597681e-18\\
70.7	0\\
70.71	0\\
70.72	0\\
70.73	0\\
70.74	0\\
70.75	0\\
70.76	0\\
70.77	0\\
70.78	0\\
70.79	0\\
70.8	0\\
70.81	0\\
70.82	0\\
70.83	0\\
70.84	0\\
70.85	0\\
70.86	0\\
70.87	0\\
70.88	1.73472347597681e-18\\
70.89	0\\
70.9	0\\
70.91	1.73472347597681e-18\\
70.92	0\\
70.93	0\\
70.94	0\\
70.95	0\\
70.96	0\\
70.97	0\\
70.98	0\\
70.99	0\\
71	1.73472347597681e-18\\
71.01	0\\
71.02	0\\
71.03	0\\
71.04	0\\
71.05	0\\
71.06	0\\
71.07	0\\
71.08	0\\
71.09	0\\
71.1	0\\
71.11	0\\
71.12	0\\
71.13	0\\
71.14	0\\
71.15	0\\
71.16	0\\
71.17	1.73472347597681e-18\\
71.18	0\\
71.19	0\\
71.2	0\\
71.21	0\\
71.22	0\\
71.23	0\\
71.24	0\\
71.25	0\\
71.26	0\\
71.27	0\\
71.28	0\\
71.29	0\\
71.3	0\\
71.31	0\\
71.32	0\\
71.33	0\\
71.34	0\\
71.35	0\\
71.36	0\\
71.37	0\\
71.38	0\\
71.39	0\\
71.4	0\\
71.41	0\\
71.42	0\\
71.43	0\\
71.44	0\\
71.45	0\\
71.46	0\\
71.47	0\\
71.48	0\\
71.49	0\\
71.5	0\\
71.51	0\\
71.52	0\\
71.53	0\\
71.54	0\\
71.55	0\\
71.56	0\\
71.57	0\\
71.58	1.73472347597681e-18\\
71.59	0\\
71.6	0\\
71.61	0\\
71.62	0\\
71.63	0\\
71.64	0\\
71.65	0\\
71.66	0\\
71.67	0\\
71.68	0\\
71.69	0\\
71.7	0\\
71.71	0\\
71.72	0\\
71.73	0\\
71.74	0\\
71.75	1.73472347597681e-18\\
71.76	0\\
71.77	0\\
71.78	0\\
71.79	0\\
71.8	0\\
71.81	0\\
71.82	0\\
71.83	0\\
71.84	0\\
71.85	0\\
71.86	0\\
71.87	0\\
71.88	0\\
71.89	0\\
71.9	0\\
71.91	0\\
71.92	0\\
71.93	0\\
71.94	0\\
71.95	0\\
71.96	0\\
71.97	0\\
71.98	0\\
71.99	0\\
72	0\\
72.01	0\\
72.02	0\\
72.03	0\\
72.04	0\\
72.05	0\\
72.06	0\\
72.07	0\\
72.08	0\\
72.09	0\\
72.1	0\\
72.11	0\\
72.12	1.73472347597681e-18\\
72.13	0\\
72.14	0\\
72.15	0\\
72.16	0\\
72.17	0\\
72.18	0\\
72.19	1.73472347597681e-18\\
72.2	0\\
72.21	1.73472347597681e-18\\
72.22	0\\
72.23	0\\
72.24	0\\
72.25	0\\
72.26	0\\
72.27	1.73472347597681e-18\\
72.28	0\\
72.29	0\\
72.3	0\\
72.31	0\\
72.32	0\\
72.33	0\\
72.34	0\\
72.35	0\\
72.36	0\\
72.37	0\\
72.38	0\\
72.39	0\\
72.4	0\\
72.41	0\\
72.42	0\\
72.43	1.73472347597681e-18\\
72.44	0\\
72.45	0\\
72.46	0\\
72.47	0\\
72.48	0\\
72.49	0\\
72.5	1.73472347597681e-18\\
72.51	0\\
72.52	0\\
72.53	0\\
72.54	0\\
72.55	0\\
72.56	0\\
72.57	0\\
72.58	0\\
72.59	0\\
72.6	0\\
72.61	0\\
72.62	0\\
72.63	0\\
72.64	0\\
72.65	0\\
72.66	0\\
72.67	0\\
72.68	0\\
72.69	0\\
72.7	1.73472347597681e-18\\
72.71	0\\
72.72	0\\
72.73	0\\
72.74	0\\
72.75	0\\
72.76	0\\
72.77	0\\
72.78	0\\
72.79	0\\
72.8	0\\
72.81	1.73472347597681e-18\\
72.82	0\\
72.83	0\\
72.84	0\\
72.85	0\\
72.86	0\\
72.87	0\\
72.88	0\\
72.89	0\\
72.9	0\\
72.91	1.73472347597681e-18\\
72.92	0\\
72.93	0\\
72.94	0\\
72.95	0\\
72.96	0\\
72.97	0\\
72.98	0\\
72.99	0\\
73	0\\
73.01	0\\
73.02	0\\
73.03	0\\
73.04	0\\
73.05	1.73472347597681e-18\\
73.06	0\\
73.07	0\\
73.08	0\\
73.09	0\\
73.1	0\\
73.11	0\\
73.12	0\\
73.13	0\\
73.14	0\\
73.15	0\\
73.16	0\\
73.17	0\\
73.18	0\\
73.19	1.73472347597681e-18\\
73.2	0\\
73.21	0\\
73.22	0\\
73.23	0\\
73.24	1.73472347597681e-18\\
73.25	0\\
73.26	0\\
73.27	0\\
73.28	0\\
73.29	0\\
73.3	0\\
73.31	0\\
73.32	0\\
73.33	0\\
73.34	0\\
73.35	0\\
73.36	0\\
73.37	0\\
73.38	0\\
73.39	0\\
73.4	0\\
73.41	0\\
73.42	0\\
73.43	0\\
73.44	0\\
73.45	0\\
73.46	0\\
73.47	0\\
73.48	0\\
73.49	0\\
73.5	0\\
73.51	0\\
73.52	0\\
73.53	0\\
73.54	1.73472347597681e-18\\
73.55	0\\
73.56	0\\
73.57	0\\
73.58	0\\
73.59	0\\
73.6	0\\
73.61	0\\
73.62	0\\
73.63	0\\
73.64	1.73472347597681e-18\\
73.65	0\\
73.66	0\\
73.67	0\\
73.68	0\\
73.69	1.73472347597681e-18\\
73.7	0\\
73.71	0\\
73.72	0\\
73.73	0\\
73.74	0\\
73.75	0\\
73.76	0\\
73.77	1.73472347597681e-18\\
73.78	0\\
73.79	0\\
73.8	0\\
73.81	0\\
73.82	0\\
73.83	0\\
73.84	0\\
73.85	0\\
73.86	0\\
73.87	0\\
73.88	0\\
73.89	0\\
73.9	0\\
73.91	0\\
73.92	0\\
73.93	0\\
73.94	0\\
73.95	0\\
73.96	0\\
73.97	0\\
73.98	0\\
73.99	0\\
74	0\\
74.01	0\\
74.02	0\\
74.03	0\\
74.04	0\\
74.05	0\\
74.06	0\\
74.07	0\\
74.08	1.73472347597681e-18\\
74.09	0\\
74.1	0\\
74.11	0\\
74.12	1.73472347597681e-18\\
74.13	0\\
74.14	0\\
74.15	1.73472347597681e-18\\
74.16	1.73472347597681e-18\\
74.17	0\\
74.18	0\\
74.19	0\\
74.2	0\\
74.21	0\\
74.22	0\\
74.23	0\\
74.24	0\\
74.25	0\\
74.26	0\\
74.27	0\\
74.28	1.73472347597681e-18\\
74.29	0\\
74.3	0\\
74.31	0\\
74.32	0\\
74.33	0\\
74.34	0\\
74.35	0\\
74.36	0\\
74.37	0\\
74.38	0\\
74.39	0\\
74.4	0\\
74.41	0\\
74.42	0\\
74.43	0\\
74.44	0\\
74.45	0\\
74.46	0\\
74.47	0\\
74.48	0\\
74.49	0\\
74.5	0\\
74.51	0\\
74.52	0\\
74.53	0\\
74.54	0\\
74.55	0\\
74.56	0\\
74.57	0\\
74.58	0\\
74.59	0\\
74.6	0\\
74.61	0\\
74.62	0\\
74.63	0\\
74.64	0\\
74.65	1.73472347597681e-18\\
74.66	0\\
74.67	1.73472347597681e-18\\
74.68	0\\
74.69	0\\
74.7	0\\
74.71	0\\
74.72	0\\
74.73	0\\
74.74	0\\
74.75	0\\
74.76	0\\
74.77	0\\
74.78	0\\
74.79	0\\
74.8	0\\
74.81	0\\
74.82	0\\
74.83	0\\
74.84	0\\
74.85	0\\
74.86	0\\
74.87	0\\
74.88	0\\
74.89	1.73472347597681e-18\\
74.9	1.73472347597681e-18\\
74.91	1.73472347597681e-18\\
74.92	0\\
74.93	0\\
74.94	1.73472347597681e-18\\
74.95	0\\
74.96	0\\
74.97	0\\
74.98	0\\
74.99	0\\
75	0\\
75.01	0\\
75.02	1.73472347597681e-18\\
75.03	0\\
75.04	0\\
75.05	0\\
75.06	0\\
75.07	0\\
75.08	0\\
75.09	0\\
75.1	0\\
75.11	0\\
75.12	0\\
75.13	0\\
75.14	0\\
75.15	0\\
75.16	0\\
75.17	0\\
75.18	0\\
75.19	0\\
75.2	0\\
75.21	0\\
75.22	0\\
75.23	0\\
75.24	0\\
75.25	0\\
75.26	0\\
75.27	0\\
75.28	0\\
75.29	0\\
75.3	0\\
75.31	0\\
75.32	0\\
75.33	0\\
75.34	0\\
75.35	0\\
75.36	0\\
75.37	0\\
75.38	0\\
75.39	0\\
75.4	1.73472347597681e-18\\
75.41	0\\
75.42	0\\
75.43	0\\
75.44	0\\
75.45	0\\
75.46	0\\
75.47	0\\
75.48	0\\
75.49	0\\
75.5	0\\
75.51	0\\
75.52	0\\
75.53	0\\
75.54	0\\
75.55	0\\
75.56	0\\
75.57	0\\
75.58	0\\
75.59	0\\
75.6	0\\
75.61	0\\
75.62	0\\
75.63	0\\
75.64	0\\
75.65	0\\
75.66	0\\
75.67	1.73472347597681e-18\\
75.68	0\\
75.69	0\\
75.7	0\\
75.71	0\\
75.72	0\\
75.73	0\\
75.74	1.73472347597681e-18\\
75.75	0\\
75.76	0\\
75.77	0\\
75.78	0\\
75.79	0\\
75.8	0\\
75.81	0\\
75.82	0\\
75.83	0\\
75.84	0\\
75.85	0\\
75.86	0\\
75.87	0\\
75.88	0\\
75.89	1.73472347597681e-18\\
75.9	0\\
75.91	0\\
75.92	0\\
75.93	0\\
75.94	0\\
75.95	0\\
75.96	0\\
75.97	0\\
75.98	0\\
75.99	0\\
76	0\\
76.01	0\\
76.02	0\\
76.03	0\\
76.04	0\\
76.05	0\\
76.06	0\\
76.07	0\\
76.08	0\\
76.09	0\\
76.1	0\\
76.11	0\\
76.12	0\\
76.13	0\\
76.14	0\\
76.15	1.73472347597681e-18\\
76.16	0\\
76.17	0\\
76.18	0\\
76.19	0\\
76.2	0\\
76.21	1.73472347597681e-18\\
76.22	1.73472347597681e-18\\
76.23	0\\
76.24	0\\
76.25	0\\
76.26	0\\
76.27	0\\
76.28	0\\
76.29	0\\
76.3	0\\
76.31	0\\
76.32	0\\
76.33	0\\
76.34	0\\
76.35	0\\
76.36	0\\
76.37	0\\
76.38	0\\
76.39	0\\
76.4	0\\
76.41	0\\
76.42	0\\
76.43	0\\
76.44	0\\
76.45	0\\
76.46	1.73472347597681e-18\\
76.47	0\\
76.48	0\\
76.49	0\\
76.5	0\\
76.51	0\\
76.52	0\\
76.53	0\\
76.54	1.73472347597681e-18\\
76.55	0\\
76.56	0\\
76.57	0\\
76.58	0\\
76.59	0\\
76.6	0\\
76.61	0\\
76.62	0\\
76.63	0\\
76.64	0\\
76.65	0\\
76.66	0\\
76.67	0\\
76.68	0\\
76.69	0\\
76.7	0\\
76.71	0\\
76.72	0\\
76.73	1.73472347597681e-18\\
76.74	0\\
76.75	0\\
76.76	0\\
76.77	0\\
76.78	0\\
76.79	0\\
76.8	0\\
76.81	0\\
76.82	0\\
76.83	1.73472347597681e-18\\
76.84	0\\
76.85	0\\
76.86	0\\
76.87	0\\
76.88	0\\
76.89	0\\
76.9	0\\
76.91	0\\
76.92	1.73472347597681e-18\\
76.93	0\\
76.94	0\\
76.95	0\\
76.96	0\\
76.97	1.73472347597681e-18\\
76.98	0\\
76.99	0\\
77	1.73472347597681e-18\\
77.01	1.73472347597681e-18\\
77.02	0\\
77.03	0\\
77.04	0\\
77.05	0\\
77.06	0\\
77.07	0\\
77.08	1.73472347597681e-18\\
77.09	0\\
77.1	0\\
77.11	0\\
77.12	0\\
77.13	0\\
77.14	0\\
77.15	0\\
77.16	0\\
77.17	0\\
77.18	0\\
77.19	0\\
77.2	0\\
77.21	0\\
77.22	0\\
77.23	0\\
77.24	0\\
77.25	0\\
77.26	0\\
77.27	0\\
77.28	0\\
77.29	0\\
77.3	0\\
77.31	0\\
77.32	0\\
77.33	0\\
77.34	0\\
77.35	0\\
77.36	0\\
77.37	1.73472347597681e-18\\
77.38	0\\
77.39	0\\
77.4	0\\
77.41	1.73472347597681e-18\\
77.42	0\\
77.43	0\\
77.44	0\\
77.45	0\\
77.46	0\\
77.47	0\\
77.48	0\\
77.49	0\\
77.5	0\\
77.51	0\\
77.52	0\\
77.53	0\\
77.54	0\\
77.55	0\\
77.56	0\\
77.57	0\\
77.58	0\\
77.59	0\\
77.6	0\\
77.61	0\\
77.62	0\\
77.63	0\\
77.64	0\\
77.65	0\\
77.66	0\\
77.67	0\\
77.68	0\\
77.69	0\\
77.7	0\\
77.71	0\\
77.72	0\\
77.73	0\\
77.74	0\\
77.75	0\\
77.76	0\\
77.77	0\\
77.78	0\\
77.79	1.73472347597681e-18\\
77.8	0\\
77.81	0\\
77.82	0\\
77.83	0\\
77.84	0\\
77.85	0\\
77.86	0\\
77.87	0\\
77.88	0\\
77.89	0\\
77.9	0\\
77.91	1.73472347597681e-18\\
77.92	0\\
77.93	0\\
77.94	0\\
77.95	1.73472347597681e-18\\
77.96	0\\
77.97	1.73472347597681e-18\\
77.98	1.73472347597681e-18\\
77.99	0\\
78	0\\
78.01	0\\
78.02	0\\
78.03	0\\
78.04	0\\
78.05	0\\
78.06	0\\
78.07	0\\
78.08	0\\
78.09	0\\
78.1	0\\
78.11	0\\
78.12	0\\
78.13	0\\
78.14	0\\
78.15	0\\
78.16	0\\
78.17	0\\
78.18	0\\
78.19	0\\
78.2	1.73472347597681e-18\\
78.21	0\\
78.22	0\\
78.23	0\\
78.24	0\\
78.25	0\\
78.26	0\\
78.27	0\\
78.28	0\\
78.29	0\\
78.3	0\\
78.31	0\\
78.32	1.73472347597681e-18\\
78.33	0\\
78.34	0\\
78.35	0\\
78.36	0\\
78.37	0\\
78.38	0\\
78.39	0\\
78.4	0\\
78.41	0\\
78.42	1.73472347597681e-18\\
78.43	0\\
78.44	1.73472347597681e-18\\
78.45	0\\
78.46	0\\
78.47	0\\
78.48	0\\
78.49	0\\
78.5	0\\
78.51	0\\
78.52	0\\
78.53	0\\
78.54	0\\
78.55	0\\
78.56	0\\
78.57	0\\
78.58	0\\
78.59	0\\
78.6	0\\
78.61	0\\
78.62	0\\
78.63	0\\
78.64	0\\
78.65	0\\
78.66	0\\
78.67	0\\
78.68	1.73472347597681e-18\\
78.69	0\\
78.7	0\\
78.71	0\\
78.72	0\\
78.73	0\\
78.74	0\\
78.75	0\\
78.76	0\\
78.77	0\\
78.78	0\\
78.79	0\\
78.8	0\\
78.81	0\\
78.82	0\\
78.83	0\\
78.84	0\\
78.85	0\\
78.86	0\\
78.87	0\\
78.88	0\\
78.89	0\\
78.9	0\\
78.91	1.73472347597681e-18\\
78.92	0\\
78.93	0\\
78.94	0\\
78.95	0\\
78.96	0\\
78.97	0\\
78.98	0\\
78.99	0\\
79	0\\
79.01	0\\
79.02	0\\
79.03	0\\
79.04	0\\
79.05	0\\
79.06	0\\
79.07	0\\
79.08	0\\
79.09	0\\
79.1	0\\
79.11	0\\
79.12	0\\
79.13	0\\
79.14	0\\
79.15	0\\
79.16	0\\
79.17	0\\
79.18	0\\
79.19	0\\
79.2	0\\
79.21	0\\
79.22	0\\
79.23	0\\
79.24	0\\
79.25	0\\
79.26	0\\
79.27	0\\
79.28	0\\
79.29	0\\
79.3	0\\
79.31	0\\
79.32	0\\
79.33	0\\
79.34	0\\
79.35	0\\
79.36	0\\
79.37	0\\
79.38	0\\
79.39	0\\
79.4	0\\
79.41	0\\
79.42	0\\
79.43	0\\
79.44	0\\
79.45	0\\
79.46	0\\
79.47	0\\
79.48	0\\
79.49	0\\
79.5	0\\
79.51	0\\
79.52	0\\
79.53	0\\
79.54	0\\
79.55	1.73472347597681e-18\\
79.56	0\\
79.57	0\\
79.58	0\\
79.59	0\\
79.6	0\\
79.61	0\\
79.62	0\\
79.63	0\\
79.64	0\\
79.65	0\\
79.66	0\\
79.67	0\\
79.68	0\\
79.69	0\\
79.7	0\\
79.71	0\\
79.72	1.73472347597681e-18\\
79.73	1.73472347597681e-18\\
79.74	0\\
79.75	0\\
79.76	0\\
79.77	0\\
79.78	0\\
79.79	0\\
79.8	0\\
79.81	0\\
79.82	0\\
79.83	0\\
79.84	0\\
79.85	0\\
79.86	0\\
79.87	0\\
79.88	1.73472347597681e-18\\
79.89	0\\
79.9	0\\
79.91	0\\
79.92	0\\
79.93	0\\
79.94	0\\
79.95	0\\
79.96	1.73472347597681e-18\\
79.97	0\\
79.98	0\\
79.99	0\\
80	0\\
80.01	0\\
};
\addplot [color=green,solid]
  table[row sep=crcr]{%
80.01	0\\
80.02	0\\
80.03	0\\
80.04	0\\
80.05	0\\
80.06	0\\
80.07	0\\
80.08	0\\
80.09	1.73472347597681e-18\\
80.1	0\\
80.11	0\\
80.12	0\\
80.13	0\\
80.14	0\\
80.15	0\\
80.16	0\\
80.17	0\\
80.18	1.73472347597681e-18\\
80.19	0\\
80.2	0\\
80.21	0\\
80.22	0\\
80.23	0\\
80.24	0\\
80.25	0\\
80.26	0\\
80.27	0\\
80.28	0\\
80.29	0\\
80.3	0\\
80.31	0\\
80.32	0\\
80.33	0\\
80.34	0\\
80.35	0\\
80.36	0\\
80.37	0\\
80.38	0\\
80.39	0\\
80.4	0\\
80.41	0\\
80.42	0\\
80.43	0\\
80.44	0\\
80.45	0\\
80.46	0\\
80.47	0\\
80.48	0\\
80.49	0\\
80.5	1.73472347597681e-18\\
80.51	0\\
80.52	0\\
80.53	0\\
80.54	1.73472347597681e-18\\
80.55	0\\
80.56	0\\
80.57	0\\
80.58	0\\
80.59	0\\
80.6	0\\
80.61	0\\
80.62	0\\
80.63	0\\
80.64	0\\
80.65	0\\
80.66	0\\
80.67	0\\
80.68	0\\
80.69	0\\
80.7	0\\
80.71	0\\
80.72	0\\
80.73	0\\
80.74	0\\
80.75	0\\
80.76	0\\
80.77	0\\
80.78	0\\
80.79	0\\
80.8	0\\
80.81	1.73472347597681e-18\\
80.82	0\\
80.83	0\\
80.84	0\\
80.85	0\\
80.86	0\\
80.87	0\\
80.88	0\\
80.89	0\\
80.9	0\\
80.91	0\\
80.92	0\\
80.93	0\\
80.94	0\\
80.95	0\\
80.96	1.73472347597681e-18\\
80.97	0\\
80.98	0\\
80.99	0\\
81	0\\
81.01	0\\
81.02	0\\
81.03	0\\
81.04	0\\
81.05	0\\
81.06	0\\
81.07	0\\
81.08	0\\
81.09	0\\
81.1	0\\
81.11	0\\
81.12	0\\
81.13	1.73472347597681e-18\\
81.14	0\\
81.15	0\\
81.16	0\\
81.17	0\\
81.18	0\\
81.19	0\\
81.2	0\\
81.21	0\\
81.22	0\\
81.23	0\\
81.24	0\\
81.25	0\\
81.26	0\\
81.27	0\\
81.28	0\\
81.29	0\\
81.3	0\\
81.31	0\\
81.32	0\\
81.33	0\\
81.34	0\\
81.35	0\\
81.36	0\\
81.37	1.73472347597681e-18\\
81.38	0\\
81.39	0\\
81.4	0\\
81.41	0\\
81.42	0\\
81.43	0\\
81.44	0\\
81.45	0\\
81.46	0\\
81.47	0\\
81.48	0\\
81.49	0\\
81.5	0\\
81.51	0\\
81.52	0\\
81.53	0\\
81.54	0\\
81.55	0\\
81.56	0\\
81.57	0\\
81.58	0\\
81.59	0\\
81.6	0\\
81.61	0\\
81.62	0\\
81.63	0\\
81.64	0\\
81.65	0\\
81.66	0\\
81.67	0\\
81.68	0\\
81.69	0\\
81.7	0\\
81.71	1.73472347597681e-18\\
81.72	0\\
81.73	0\\
81.74	0\\
81.75	0\\
81.76	0\\
81.77	0\\
81.78	0\\
81.79	0\\
81.8	0\\
81.81	0\\
81.82	0\\
81.83	0\\
81.84	0\\
81.85	0\\
81.86	0\\
81.87	0\\
81.88	0\\
81.89	0\\
81.9	0\\
81.91	0\\
81.92	0\\
81.93	0\\
81.94	0\\
81.95	0\\
81.96	0\\
81.97	0\\
81.98	0\\
81.99	0\\
82	0\\
82.01	0\\
82.02	0\\
82.03	0\\
82.04	0\\
82.05	0\\
82.06	0\\
82.07	0\\
82.08	0\\
82.09	0\\
82.1	0\\
82.11	0\\
82.12	0\\
82.13	0\\
82.14	0\\
82.15	0\\
82.16	0\\
82.17	0\\
82.18	0\\
82.19	0\\
82.2	0\\
82.21	0\\
82.22	0\\
82.23	0\\
82.24	0\\
82.25	0\\
82.26	0\\
82.27	0\\
82.28	0\\
82.29	0\\
82.3	0\\
82.31	0\\
82.32	0\\
82.33	0\\
82.34	0\\
82.35	0\\
82.36	0\\
82.37	0\\
82.38	0\\
82.39	0\\
82.4	0\\
82.41	0\\
82.42	0\\
82.43	0\\
82.44	0\\
82.45	0\\
82.46	0\\
82.47	1.73472347597681e-18\\
82.48	0\\
82.49	0\\
82.5	0\\
82.51	0\\
82.52	0\\
82.53	0\\
82.54	0\\
82.55	0\\
82.56	0\\
82.57	0\\
82.58	0\\
82.59	0\\
82.6	0\\
82.61	0\\
82.62	0\\
82.63	0\\
82.64	1.73472347597681e-18\\
82.65	0\\
82.66	0\\
82.67	0\\
82.68	0\\
82.69	0\\
82.7	0\\
82.71	1.73472347597681e-18\\
82.72	0\\
82.73	0\\
82.74	0\\
82.75	0\\
82.76	0\\
82.77	0\\
82.78	0\\
82.79	0\\
82.8	0\\
82.81	0\\
82.82	0\\
82.83	0\\
82.84	0\\
82.85	0\\
82.86	0\\
82.87	0\\
82.88	0\\
82.89	0\\
82.9	0\\
82.91	0\\
82.92	0\\
82.93	0\\
82.94	0\\
82.95	0\\
82.96	0\\
82.97	0\\
82.98	0\\
82.99	0\\
83	0\\
83.01	0\\
83.02	0\\
83.03	0\\
83.04	0\\
83.05	0\\
83.06	0\\
83.07	0\\
83.08	0\\
83.09	0\\
83.1	0\\
83.11	0\\
83.12	1.73472347597681e-18\\
83.13	0\\
83.14	0\\
83.15	0\\
83.16	0\\
83.17	0\\
83.18	0\\
83.19	0\\
83.2	1.73472347597681e-18\\
83.21	0\\
83.22	0\\
83.23	0\\
83.24	0\\
83.25	0\\
83.26	0\\
83.27	0\\
83.28	0\\
83.29	0\\
83.3	0\\
83.31	0\\
83.32	0\\
83.33	0\\
83.34	0\\
83.35	0\\
83.36	0\\
83.37	0\\
83.38	0\\
83.39	0\\
83.4	0\\
83.41	0\\
83.42	0\\
83.43	0\\
83.44	0\\
83.45	0\\
83.46	1.73472347597681e-18\\
83.47	0\\
83.48	0\\
83.49	0\\
83.5	0\\
83.51	0\\
83.52	0\\
83.53	0\\
83.54	0\\
83.55	0\\
83.56	0\\
83.57	0\\
83.58	0\\
83.59	0\\
83.6	0\\
83.61	0\\
83.62	0\\
83.63	0\\
83.64	0\\
83.65	0\\
83.66	0\\
83.67	0\\
83.68	0\\
83.69	0\\
83.7	0\\
83.71	0\\
83.72	0\\
83.73	0\\
83.74	0\\
83.75	0\\
83.76	0\\
83.77	0\\
83.78	0\\
83.79	0\\
83.8	0\\
83.81	0\\
83.82	0\\
83.83	0\\
83.84	0\\
83.85	0\\
83.86	0\\
83.87	0\\
83.88	0\\
83.89	0\\
83.9	0\\
83.91	0\\
83.92	0\\
83.93	0\\
83.94	0\\
83.95	0\\
83.96	0\\
83.97	0\\
83.98	0\\
83.99	0\\
84	0\\
84.01	0\\
84.02	0\\
84.03	0\\
84.04	0\\
84.05	0\\
84.06	0\\
84.07	0\\
84.08	0\\
84.09	0\\
84.1	0\\
84.11	0\\
84.12	0\\
84.13	0\\
84.14	0\\
84.15	0\\
84.16	0\\
84.17	0\\
84.18	0\\
84.19	0\\
84.2	0\\
84.21	0\\
84.22	0\\
84.23	0\\
84.24	0\\
84.25	0\\
84.26	0\\
84.27	0\\
84.28	0\\
84.29	0\\
84.3	0\\
84.31	0\\
84.32	1.73472347597681e-18\\
84.33	0\\
84.34	0\\
84.35	1.73472347597681e-18\\
84.36	0\\
84.37	0\\
84.38	0\\
84.39	0\\
84.4	0\\
84.41	0\\
84.42	0\\
84.43	0\\
84.44	0\\
84.45	0\\
84.46	0\\
84.47	0\\
84.48	0\\
84.49	0\\
84.5	0\\
84.51	0\\
84.52	0\\
84.53	0\\
84.54	0\\
84.55	0\\
84.56	0\\
84.57	0\\
84.58	0\\
84.59	0\\
84.6	0\\
84.61	0\\
84.62	0\\
84.63	0\\
84.64	0\\
84.65	0\\
84.66	0\\
84.67	0\\
84.68	0\\
84.69	0\\
84.7	0\\
84.71	0\\
84.72	0\\
84.73	0\\
84.74	0\\
84.75	0\\
84.76	0\\
84.77	0\\
84.78	0\\
84.79	1.73472347597681e-18\\
84.8	0\\
84.81	0\\
84.82	0\\
84.83	0\\
84.84	0\\
84.85	0\\
84.86	0\\
84.87	0\\
84.88	0\\
84.89	0\\
84.9	0\\
84.91	0\\
84.92	0\\
84.93	0\\
84.94	1.73472347597681e-18\\
84.95	0\\
84.96	0\\
84.97	0\\
84.98	0\\
84.99	0\\
85	0\\
85.01	0\\
85.02	1.73472347597681e-18\\
85.03	0\\
85.04	0\\
85.05	0\\
85.06	0\\
85.07	0\\
85.08	0\\
85.09	0\\
85.1	0\\
85.11	0\\
85.12	0\\
85.13	0\\
85.14	0\\
85.15	0\\
85.16	0\\
85.17	0\\
85.18	0\\
85.19	0\\
85.2	0\\
85.21	0\\
85.22	0\\
85.23	0\\
85.24	0\\
85.25	0\\
85.26	0\\
85.27	0\\
85.28	0\\
85.29	0\\
85.3	0\\
85.31	0\\
85.32	0\\
85.33	0\\
85.34	0\\
85.35	0\\
85.36	0\\
85.37	1.73472347597681e-18\\
85.38	0\\
85.39	0\\
85.4	0\\
85.41	1.73472347597681e-18\\
85.42	0\\
85.43	0\\
85.44	0\\
85.45	0\\
85.46	0\\
85.47	0\\
85.48	0\\
85.49	0\\
85.5	0\\
85.51	0\\
85.52	0\\
85.53	0\\
85.54	1.73472347597681e-18\\
85.55	0\\
85.56	1.73472347597681e-18\\
85.57	0\\
85.58	0\\
85.59	0\\
85.6	0\\
85.61	0\\
85.62	0\\
85.63	0\\
85.64	0\\
85.65	0\\
85.66	0\\
85.67	0\\
85.68	0\\
85.69	0\\
85.7	0\\
85.71	0\\
85.72	0\\
85.73	0\\
85.74	0\\
85.75	1.73472347597681e-18\\
85.76	0\\
85.77	1.73472347597681e-18\\
85.78	0\\
85.79	0\\
85.8	0\\
85.81	1.73472347597681e-18\\
85.82	0\\
85.83	1.73472347597681e-18\\
85.84	0\\
85.85	0\\
85.86	0\\
85.87	0\\
85.88	0\\
85.89	0\\
85.9	0\\
85.91	0\\
85.92	0\\
85.93	0\\
85.94	0\\
85.95	0\\
85.96	0\\
85.97	0\\
85.98	0\\
85.99	0\\
86	0\\
86.01	0\\
86.02	0\\
86.03	0\\
86.04	0\\
86.05	0\\
86.06	0\\
86.07	0\\
86.08	1.73472347597681e-18\\
86.09	0\\
86.1	0\\
86.11	0\\
86.12	0\\
86.13	0\\
86.14	0\\
86.15	0\\
86.16	0\\
86.17	0\\
86.18	0\\
86.19	0\\
86.2	0\\
86.21	0\\
86.22	0\\
86.23	0\\
86.24	0\\
86.25	0\\
86.26	0\\
86.27	0\\
86.28	0\\
86.29	0\\
86.3	0\\
86.31	0\\
86.32	0\\
86.33	0\\
86.34	0\\
86.35	0\\
86.36	0\\
86.37	0\\
86.38	0\\
86.39	0\\
86.4	0\\
86.41	0\\
86.42	1.73472347597681e-18\\
86.43	0\\
86.44	0\\
86.45	0\\
86.46	1.73472347597681e-18\\
86.47	0\\
86.48	0\\
86.49	0\\
86.5	0\\
86.51	0\\
86.52	0\\
86.53	0\\
86.54	0\\
86.55	0\\
86.56	0\\
86.57	0\\
86.58	0\\
86.59	0\\
86.6	0\\
86.61	1.73472347597681e-18\\
86.62	0\\
86.63	0\\
86.64	0\\
86.65	0\\
86.66	0\\
86.67	0\\
86.68	0\\
86.69	0\\
86.7	0\\
86.71	0\\
86.72	0\\
86.73	0\\
86.74	0\\
86.75	1.73472347597681e-18\\
86.76	0\\
86.77	0\\
86.78	0\\
86.79	0\\
86.8	0\\
86.81	0\\
86.82	0\\
86.83	0\\
86.84	0\\
86.85	1.73472347597681e-18\\
86.86	0\\
86.87	0\\
86.88	0\\
86.89	0\\
86.9	0\\
86.91	1.73472347597681e-18\\
86.92	0\\
86.93	0\\
86.94	0\\
86.95	0\\
86.96	0\\
86.97	0\\
86.98	0\\
86.99	0\\
87	0\\
87.01	0\\
87.02	0\\
87.03	0\\
87.04	0\\
87.05	1.73472347597681e-18\\
87.06	1.73472347597681e-18\\
87.07	0\\
87.08	0\\
87.09	0\\
87.1	0\\
87.11	0\\
87.12	0\\
87.13	0\\
87.14	0\\
87.15	0\\
87.16	0\\
87.17	0\\
87.18	0\\
87.19	0\\
87.2	1.73472347597681e-18\\
87.21	0\\
87.22	0\\
87.23	0\\
87.24	0\\
87.25	0\\
87.26	0\\
87.27	0\\
87.28	0\\
87.29	0\\
87.3	0\\
87.31	0\\
87.32	0\\
87.33	0\\
87.34	0\\
87.35	0\\
87.36	0\\
87.37	0\\
87.38	0\\
87.39	0\\
87.4	0\\
87.41	0\\
87.42	0\\
87.43	0\\
87.44	0\\
87.45	0\\
87.46	0\\
87.47	0\\
87.48	0\\
87.49	0\\
87.5	0\\
87.51	0\\
87.52	0\\
87.53	0\\
87.54	0\\
87.55	0\\
87.56	0\\
87.57	0\\
87.58	0\\
87.59	0\\
87.6	0\\
87.61	0\\
87.62	0\\
87.63	0\\
87.64	0\\
87.65	0\\
87.66	0\\
87.67	0\\
87.68	0\\
87.69	0\\
87.7	0\\
87.71	0\\
87.72	0\\
87.73	0\\
87.74	0\\
87.75	0\\
87.76	0\\
87.77	0\\
87.78	0\\
87.79	0\\
87.8	0\\
87.81	0\\
87.82	0\\
87.83	0\\
87.84	0\\
87.85	0\\
87.86	0\\
87.87	0\\
87.88	0\\
87.89	0\\
87.9	0\\
87.91	0\\
87.92	0\\
87.93	0\\
87.94	0\\
87.95	0\\
87.96	0\\
87.97	1.73472347597681e-18\\
87.98	0\\
87.99	0\\
88	0\\
88.01	0\\
88.02	0\\
88.03	0\\
88.04	0\\
88.05	0\\
88.06	0\\
88.07	0\\
88.08	0\\
88.09	0\\
88.1	0\\
88.11	0\\
88.12	0\\
88.13	0\\
88.14	0\\
88.15	0\\
88.16	0\\
88.17	1.73472347597681e-18\\
88.18	0\\
88.19	0\\
88.2	0\\
88.21	0\\
88.22	0\\
88.23	0\\
88.24	0\\
88.25	0\\
88.26	0\\
88.27	0\\
88.28	0\\
88.29	0\\
88.3	0\\
88.31	0\\
88.32	0\\
88.33	0\\
88.34	1.73472347597681e-18\\
88.35	0\\
88.36	0\\
88.37	0\\
88.38	1.73472347597681e-18\\
88.39	0\\
88.4	0\\
88.41	0\\
88.42	0\\
88.43	1.73472347597681e-18\\
88.44	0\\
88.45	0\\
88.46	0\\
88.47	0\\
88.48	0\\
88.49	0\\
88.5	0\\
88.51	0\\
88.52	0\\
88.53	0\\
88.54	0\\
88.55	0\\
88.56	0\\
88.57	0\\
88.58	0\\
88.59	0\\
88.6	0\\
88.61	0\\
88.62	0\\
88.63	0\\
88.64	0\\
88.65	0\\
88.66	1.73472347597681e-18\\
88.67	0\\
88.68	0\\
88.69	1.73472347597681e-18\\
88.7	0\\
88.71	0\\
88.72	0\\
88.73	0\\
88.74	0\\
88.75	0\\
88.76	0\\
88.77	0\\
88.78	0\\
88.79	0\\
88.8	0\\
88.81	0\\
88.82	0\\
88.83	0\\
88.84	0\\
88.85	0\\
88.86	0\\
88.87	0\\
88.88	0\\
88.89	0\\
88.9	0\\
88.91	0\\
88.92	0\\
88.93	0\\
88.94	0\\
88.95	0\\
88.96	0\\
88.97	0\\
88.98	0\\
88.99	0\\
89	0\\
89.01	0\\
89.02	0\\
89.03	0\\
89.04	0\\
89.05	0\\
89.06	0\\
89.07	0\\
89.08	0\\
89.09	0\\
89.1	0\\
89.11	0\\
89.12	0\\
89.13	0\\
89.14	0\\
89.15	0\\
89.16	0\\
89.17	0\\
89.18	0\\
89.19	0\\
89.2	0\\
89.21	0\\
89.22	0\\
89.23	0\\
89.24	0\\
89.25	0\\
89.26	0\\
89.27	0\\
89.28	0\\
89.29	0\\
89.3	0\\
89.31	1.73472347597681e-18\\
89.32	0\\
89.33	0\\
89.34	0\\
89.35	0\\
89.36	0\\
89.37	0\\
89.38	0\\
89.39	0\\
89.4	0\\
89.41	0\\
89.42	0\\
89.43	1.73472347597681e-18\\
89.44	1.73472347597681e-18\\
89.45	0\\
89.46	0\\
89.47	0\\
89.48	0\\
89.49	0\\
89.5	0\\
89.51	0\\
89.52	0\\
89.53	0\\
89.54	0\\
89.55	0\\
89.56	0\\
89.57	0\\
89.58	0\\
89.59	0\\
89.6	0\\
89.61	0\\
89.62	0\\
89.63	0\\
89.64	0\\
89.65	0\\
89.66	0\\
89.67	0\\
89.68	0\\
89.69	0\\
89.7	0\\
89.71	0\\
89.72	0\\
89.73	0\\
89.74	0\\
89.75	1.73472347597681e-18\\
89.76	0\\
89.77	0\\
89.78	0\\
89.79	0\\
89.8	0\\
89.81	1.73472347597681e-18\\
89.82	0\\
89.83	0\\
89.84	0\\
89.85	0\\
89.86	0\\
89.87	0\\
89.88	1.73472347597681e-18\\
89.89	0\\
89.9	0\\
89.91	0\\
89.92	0\\
89.93	0\\
89.94	0\\
89.95	0\\
89.96	0\\
89.97	0\\
89.98	0\\
89.99	0\\
90	0\\
90.01	0\\
90.02	0\\
90.03	0\\
90.04	0\\
90.05	0\\
90.06	0\\
90.07	0\\
90.08	0\\
90.09	0\\
90.1	0\\
90.11	0\\
90.12	0\\
90.13	0\\
90.14	0\\
90.15	0\\
90.16	0\\
90.17	0\\
90.18	0\\
90.19	0\\
90.2	0\\
90.21	0\\
90.22	0\\
90.23	0\\
90.24	0\\
90.25	0\\
90.26	0\\
90.27	0\\
90.28	0\\
90.29	0\\
90.3	0\\
90.31	0\\
90.32	0\\
90.33	0\\
90.34	0\\
90.35	0\\
90.36	0\\
90.37	0\\
90.38	0\\
90.39	0\\
90.4	0\\
90.41	0\\
90.42	0\\
90.43	0\\
90.44	0\\
90.45	0\\
90.46	0\\
90.47	0\\
90.48	0\\
90.49	0\\
90.5	0\\
90.51	0\\
90.52	0\\
90.53	0\\
90.54	1.73472347597681e-18\\
90.55	0\\
90.56	0\\
90.57	0\\
90.58	0\\
90.59	0\\
90.6	0\\
90.61	0\\
90.62	0\\
90.63	0\\
90.64	0\\
90.65	0\\
90.66	0\\
90.67	0\\
90.68	0\\
90.69	0\\
90.7	1.73472347597681e-18\\
90.71	0\\
90.72	0\\
90.73	0\\
90.74	0\\
90.75	0\\
90.76	0\\
90.77	0\\
90.78	0\\
90.79	0\\
90.8	0\\
90.81	0\\
90.82	0\\
90.83	0\\
90.84	0\\
90.85	0\\
90.86	0\\
90.87	0\\
90.88	0\\
90.89	0\\
90.9	0\\
90.91	0\\
90.92	0\\
90.93	0\\
90.94	0\\
90.95	0\\
90.96	0\\
90.97	0\\
90.98	0\\
90.99	0\\
91	0\\
91.01	0\\
91.02	0\\
91.03	0\\
91.04	0\\
91.05	0\\
91.06	0\\
91.07	0\\
91.08	0\\
91.09	0\\
91.1	0\\
91.11	0\\
91.12	0\\
91.13	0\\
91.14	0\\
91.15	0\\
91.16	0\\
91.17	0\\
91.18	0\\
91.19	0\\
91.2	0\\
91.21	0\\
91.22	0\\
91.23	0\\
91.24	0\\
91.25	0\\
91.26	0\\
91.27	0\\
91.28	0\\
91.29	0\\
91.3	0\\
91.31	0\\
91.32	0\\
91.33	0\\
91.34	0\\
91.35	0\\
91.36	0\\
91.37	0\\
91.38	0\\
91.39	0\\
91.4	0\\
91.41	0\\
91.42	0\\
91.43	0\\
91.44	0\\
91.45	0\\
91.46	0\\
91.47	0\\
91.48	0\\
91.49	0\\
91.5	0\\
91.51	0\\
91.52	0\\
91.53	0\\
91.54	0\\
91.55	0\\
91.56	0\\
91.57	0\\
91.58	0\\
91.59	0\\
91.6	0\\
91.61	0\\
91.62	0\\
91.63	0\\
91.64	0\\
91.65	0\\
91.66	0\\
91.67	0\\
91.68	0\\
91.69	0\\
91.7	0\\
91.71	0\\
91.72	0\\
91.73	0\\
91.74	0\\
91.75	0\\
91.76	0\\
91.77	0\\
91.78	0\\
91.79	0\\
91.8	0\\
91.81	0\\
91.82	0\\
91.83	0\\
91.84	0\\
91.85	0\\
91.86	0\\
91.87	0\\
91.88	0\\
91.89	0\\
91.9	0\\
91.91	0\\
91.92	0\\
91.93	0\\
91.94	0\\
91.95	0\\
91.96	0\\
91.97	0\\
91.98	0\\
91.99	0\\
92	0\\
92.01	0\\
92.02	0\\
92.03	0\\
92.04	0\\
92.05	0\\
92.06	0\\
92.07	0\\
92.08	0\\
92.09	0\\
92.1	0\\
92.11	0\\
92.12	0\\
92.13	0\\
92.14	0\\
92.15	0\\
92.16	0\\
92.17	0\\
92.18	0\\
92.19	0\\
92.2	0\\
92.21	0\\
92.22	0\\
92.23	0\\
92.24	0\\
92.25	0\\
92.26	0\\
92.27	0\\
92.28	0\\
92.29	0\\
92.3	0\\
92.31	0\\
92.32	0\\
92.33	0\\
92.34	0\\
92.35	0\\
92.36	0\\
92.37	0\\
92.38	0\\
92.39	0\\
92.4	0\\
92.41	0\\
92.42	0\\
92.43	0\\
92.44	0\\
92.45	0\\
92.46	0\\
92.47	0\\
92.48	0\\
92.49	0\\
92.5	0\\
92.51	0\\
92.52	0\\
92.53	0\\
92.54	0\\
92.55	0\\
92.56	0\\
92.57	0\\
92.58	0\\
92.59	0\\
92.6	0\\
92.61	0\\
92.62	0\\
92.63	0\\
92.64	0\\
92.65	0\\
92.66	0\\
92.67	0\\
92.68	0\\
92.69	0\\
92.7	0\\
92.71	0\\
92.72	0\\
92.73	0\\
92.74	0\\
92.75	0\\
92.76	0\\
92.77	0\\
92.78	0\\
92.79	0\\
92.8	0\\
92.81	0\\
92.82	0\\
92.83	0\\
92.84	0\\
92.85	0\\
92.86	0\\
92.87	0\\
92.88	0\\
92.89	0\\
92.9	0\\
92.91	0\\
92.92	0\\
92.93	0\\
92.94	0\\
92.95	0\\
92.96	0\\
92.97	0\\
92.98	0\\
92.99	0\\
93	0\\
93.01	0\\
93.02	0\\
93.03	0\\
93.04	0\\
93.05	0\\
93.06	0\\
93.07	0\\
93.08	0\\
93.09	0\\
93.1	0\\
93.11	0\\
93.12	0\\
93.13	0\\
93.14	0\\
93.15	0\\
93.16	0\\
93.17	0\\
93.18	0\\
93.19	0\\
93.2	0\\
93.21	0\\
93.22	0\\
93.23	0\\
93.24	0\\
93.25	0\\
93.26	0\\
93.27	0\\
93.28	0\\
93.29	0\\
93.3	0\\
93.31	0\\
93.32	0\\
93.33	0\\
93.34	0\\
93.35	0\\
93.36	0\\
93.37	0\\
93.38	0\\
93.39	0\\
93.4	0\\
93.41	0\\
93.42	0\\
93.43	0\\
93.44	0\\
93.45	0\\
93.46	0\\
93.47	0\\
93.48	0\\
93.49	0\\
93.5	0\\
93.51	0\\
93.52	0\\
93.53	0\\
93.54	0\\
93.55	0\\
93.56	0\\
93.57	0\\
93.58	0\\
93.59	0\\
93.6	0\\
93.61	0\\
93.62	0\\
93.63	0\\
93.64	0\\
93.65	0\\
93.66	0\\
93.67	0\\
93.68	0\\
93.69	0\\
93.7	0\\
93.71	0\\
93.72	0\\
93.73	0\\
93.74	0\\
93.75	0\\
93.76	0\\
93.77	0\\
93.78	0\\
93.79	0\\
93.8	0\\
93.81	0\\
93.82	0\\
93.83	0\\
93.84	0\\
93.85	0\\
93.86	0\\
93.87	0\\
93.88	0\\
93.89	0\\
93.9	0\\
93.91	0\\
93.92	0\\
93.93	0\\
93.94	0\\
93.95	0\\
93.96	0\\
93.97	0\\
93.98	0\\
93.99	0\\
94	0\\
94.01	0\\
94.02	0\\
94.03	0\\
94.04	0\\
94.05	0\\
94.06	0\\
94.07	0\\
94.08	0\\
94.09	0\\
94.1	0\\
94.11	0\\
94.12	0\\
94.13	0\\
94.14	0\\
94.15	0\\
94.16	0\\
94.17	0\\
94.18	0\\
94.19	0\\
94.2	0\\
94.21	0\\
94.22	0\\
94.23	0\\
94.24	0\\
94.25	0\\
94.26	0\\
94.27	0\\
94.28	0\\
94.29	0\\
94.3	0\\
94.31	0\\
94.32	0\\
94.33	0\\
94.34	0\\
94.35	0\\
94.36	0\\
94.37	0\\
94.38	0\\
94.39	0\\
94.4	0\\
94.41	0\\
94.42	0\\
94.43	0\\
94.44	0\\
94.45	0\\
94.46	0\\
94.47	0\\
94.48	0\\
94.49	0\\
94.5	0\\
94.51	0\\
94.52	0\\
94.53	0\\
94.54	0\\
94.55	0\\
94.56	0\\
94.57	0\\
94.58	0\\
94.59	0\\
94.6	0\\
94.61	0\\
94.62	0\\
94.63	0\\
94.64	0\\
94.65	0\\
94.66	0\\
94.67	0\\
94.68	0\\
94.69	0\\
94.7	0\\
94.71	0\\
94.72	0\\
94.73	0\\
94.74	0\\
94.75	0\\
94.76	0\\
94.77	0\\
94.78	0\\
94.79	0\\
94.8	0\\
94.81	0\\
94.82	0\\
94.83	0\\
94.84	0\\
94.85	0\\
94.86	0\\
94.87	0\\
94.88	0\\
94.89	0\\
94.9	0\\
94.91	0\\
94.92	0\\
94.93	0\\
94.94	0\\
94.95	0\\
94.96	0\\
94.97	0\\
94.98	0\\
94.99	0\\
95	0\\
95.01	0\\
95.02	0\\
95.03	0\\
95.04	0\\
95.05	0\\
95.06	0\\
95.07	0\\
95.08	0\\
95.09	0\\
95.1	0\\
95.11	0\\
95.12	0\\
95.13	0\\
95.14	0\\
95.15	0\\
95.16	0\\
95.17	0\\
95.18	0\\
95.19	0\\
95.2	0\\
95.21	0\\
95.22	0\\
95.23	0\\
95.24	0\\
95.25	0\\
95.26	0\\
95.27	0\\
95.28	0\\
95.29	0\\
95.3	0\\
95.31	0\\
95.32	0\\
95.33	0\\
95.34	0\\
95.35	0\\
95.36	0\\
95.37	0\\
95.38	0\\
95.39	0\\
95.4	0\\
95.41	0\\
95.42	0\\
95.43	0\\
95.44	0\\
95.45	0\\
95.46	0\\
95.47	0\\
95.48	0\\
95.49	0\\
95.5	0\\
95.51	0\\
95.52	0\\
95.53	0\\
95.54	0\\
95.55	0\\
95.56	0\\
95.57	0\\
95.58	0\\
95.59	0\\
95.6	0\\
95.61	0\\
95.62	0\\
95.63	0\\
95.64	0\\
95.65	0\\
95.66	0\\
95.67	0\\
95.68	0\\
95.69	0\\
95.7	0\\
95.71	0\\
95.72	0\\
95.73	0\\
95.74	0\\
95.75	0\\
95.76	0\\
95.77	0\\
95.78	0\\
95.79	0\\
95.8	0\\
95.81	0\\
95.82	0\\
95.83	0\\
95.84	0\\
95.85	0\\
95.86	0\\
95.87	0\\
95.88	0\\
95.89	0\\
95.9	0\\
95.91	0\\
95.92	0\\
95.93	0\\
95.94	0\\
95.95	0\\
95.96	0\\
95.97	0\\
95.98	0\\
95.99	0\\
96	0\\
96.01	0\\
96.02	0\\
96.03	0\\
96.04	0\\
96.05	0\\
96.06	0\\
96.07	0\\
96.08	0\\
96.09	0\\
96.1	0\\
96.11	0\\
96.12	0\\
96.13	0\\
96.14	0\\
96.15	0\\
96.16	0\\
96.17	0\\
96.18	0\\
96.19	0\\
96.2	0\\
96.21	0\\
96.22	0\\
96.23	0\\
96.24	0\\
96.25	0\\
96.26	0\\
96.27	0\\
96.28	0\\
96.29	0\\
96.3	0\\
96.31	0\\
96.32	0\\
96.33	0\\
96.34	0\\
96.35	0\\
96.36	0\\
96.37	0\\
96.38	0\\
96.39	0\\
96.4	0\\
96.41	0\\
96.42	0\\
96.43	0\\
96.44	0\\
96.45	0\\
96.46	0\\
96.47	0\\
96.48	0\\
96.49	0\\
96.5	0\\
96.51	0\\
96.52	0\\
96.53	0\\
96.54	0\\
96.55	0\\
96.56	0\\
96.57	0\\
96.58	0\\
96.59	0\\
96.6	0\\
96.61	0\\
96.62	0\\
96.63	0\\
96.64	0\\
96.65	0\\
96.66	0\\
96.67	0\\
96.68	0\\
96.69	0\\
96.7	0\\
96.71	0\\
96.72	0\\
96.73	0\\
96.74	0\\
96.75	0\\
96.76	0\\
96.77	0\\
96.78	0\\
96.79	0\\
96.8	0\\
96.81	0\\
96.82	0\\
96.83	0\\
96.84	0\\
96.85	0\\
96.86	0\\
96.87	0\\
96.88	0\\
96.89	0\\
96.9	0\\
96.91	0\\
96.92	0\\
96.93	0\\
96.94	0\\
96.95	0\\
96.96	0\\
96.97	0\\
96.98	0\\
96.99	0\\
97	0\\
97.01	0\\
97.02	0\\
97.03	0\\
97.04	0\\
97.05	0\\
97.06	0\\
97.07	0\\
97.08	0\\
97.09	0\\
97.1	0\\
97.11	0\\
97.12	0\\
97.13	0\\
97.14	0\\
97.15	0\\
97.16	0\\
97.17	0\\
97.18	0\\
97.19	0\\
97.2	0\\
97.21	0\\
97.22	0\\
97.23	0\\
97.24	0\\
97.25	0\\
97.26	0\\
97.27	0\\
97.28	0\\
97.29	0\\
97.3	0\\
97.31	0\\
97.32	0\\
97.33	0\\
97.34	0\\
97.35	0\\
97.36	0\\
97.37	0\\
97.38	0\\
97.39	0\\
97.4	0\\
97.41	0\\
97.42	0\\
97.43	0\\
97.44	0\\
97.45	0\\
97.46	0\\
97.47	0\\
97.48	0\\
97.49	0\\
97.5	0\\
97.51	0\\
97.52	0\\
97.53	0\\
97.54	0\\
97.55	0\\
97.56	0\\
97.57	0\\
97.58	0\\
97.59	0\\
97.6	0\\
97.61	0\\
97.62	0\\
97.63	0\\
97.64	0\\
97.65	0\\
97.66	0\\
97.67	0\\
97.68	0\\
97.69	0\\
97.7	0\\
97.71	0\\
97.72	0\\
97.73	0\\
97.74	0\\
97.75	0\\
97.76	0\\
97.77	0\\
97.78	0\\
97.79	0\\
97.8	0\\
97.81	0\\
97.82	0\\
97.83	0\\
97.84	0\\
97.85	0\\
97.86	0\\
97.87	0\\
97.88	0\\
97.89	0\\
97.9	0\\
97.91	0\\
97.92	0\\
97.93	0\\
97.94	0\\
97.95	0\\
97.96	0\\
97.97	0\\
97.98	0\\
97.99	0\\
98	0\\
98.01	0\\
98.02	0\\
98.03	0\\
98.04	0\\
98.05	0\\
98.06	0\\
98.07	0\\
98.08	0\\
98.09	0\\
98.1	0\\
98.11	0\\
98.12	0\\
98.13	0\\
98.14	0\\
98.15	0\\
98.16	0\\
98.17	0\\
98.18	0\\
98.19	0\\
98.2	0\\
98.21	0\\
98.22	0\\
98.23	0\\
98.24	0\\
98.25	0\\
98.26	0\\
98.27	0\\
98.28	0\\
98.29	0\\
98.3	0\\
98.31	0\\
98.32	0\\
98.33	0\\
98.34	0\\
98.35	0\\
98.36	0\\
98.37	0\\
98.38	0\\
98.39	0\\
98.4	0\\
98.41	0\\
98.42	0\\
98.43	0\\
98.44	0\\
98.45	0\\
98.46	0\\
98.47	0\\
98.48	0\\
98.49	0\\
98.5	0\\
98.51	0\\
98.52	0\\
98.53	0\\
98.54	0\\
98.55	0\\
98.56	0\\
98.57	0\\
98.58	0\\
98.59	0\\
98.6	0\\
98.61	0\\
98.62	0\\
98.63	0\\
98.64	0\\
98.65	0\\
98.66	0\\
98.67	0\\
98.68	0\\
98.69	0\\
98.7	0\\
98.71	0\\
98.72	0\\
98.73	0\\
98.74	0\\
98.75	0\\
98.76	0\\
98.77	0\\
98.78	0\\
98.79	0\\
98.8	0\\
98.81	0\\
98.82	0\\
98.83	0\\
98.84	0\\
98.85	0\\
98.86	0\\
98.87	0\\
98.88	0\\
98.89	0\\
98.9	0\\
98.91	0\\
98.92	0\\
98.93	0\\
98.94	0\\
98.95	0\\
98.96	0\\
98.97	0\\
98.98	0\\
98.99	0\\
99	0\\
99.01	0\\
99.02	0\\
99.03	0\\
99.04	0\\
99.05	0\\
99.06	0\\
99.07	0\\
99.08	0\\
99.09	0\\
99.1	0\\
99.11	0\\
99.12	0\\
99.13	0\\
99.14	0\\
99.15	0\\
99.16	0\\
99.17	0\\
99.18	0\\
99.19	0\\
99.2	0\\
99.21	0\\
99.22	0\\
99.23	0\\
99.24	0\\
99.25	0\\
99.26	0\\
99.27	0\\
99.28	0\\
99.29	0\\
99.3	0\\
99.31	0\\
99.32	0\\
99.33	0\\
99.34	0\\
99.35	0\\
99.36	0\\
99.37	0\\
99.38	0\\
99.39	0\\
99.4	0\\
99.41	0\\
99.42	0\\
99.43	0\\
99.44	0\\
99.45	0\\
99.46	0\\
99.47	0\\
99.48	0\\
99.49	0\\
99.5	0\\
99.51	0\\
99.52	0\\
99.53	0\\
99.54	0\\
99.55	0\\
99.56	0\\
99.57	0\\
99.58	0\\
99.59	0\\
99.6	0\\
99.61	0\\
99.62	0\\
99.63	0\\
99.64	0\\
99.65	0\\
99.66	0\\
99.67	0\\
99.68	0\\
99.69	0\\
99.7	0\\
99.71	0\\
99.72	0\\
99.73	0\\
99.74	0\\
99.75	0\\
99.76	0\\
99.77	0\\
99.78	0\\
99.79	0\\
99.8	0\\
99.81	0\\
99.82	0\\
99.83	0\\
99.84	0\\
99.85	0\\
99.86	0\\
99.87	0\\
99.88	0\\
99.89	0\\
99.9	0\\
99.91	0\\
99.92	0\\
99.93	0\\
99.94	0\\
99.95	0\\
99.96	0\\
99.97	0\\
99.98	0\\
99.99	0\\
100	0\\
};
\addlegendentry{$q=4$};

\end{axis}
\end{tikzpicture}% 
  \caption{Continuous Time w/ nFPC}
\end{subfigure}%
\hfill%
\begin{subfigure}{.45\linewidth}
  \centering
  \setlength\figureheight{\linewidth} 
  \setlength\figurewidth{\linewidth}
  \tikzsetnextfilename{dm_dscr_nFPC_z15}
  % This file was created by matlab2tikz.
%
%The latest updates can be retrieved from
%  http://www.mathworks.com/matlabcentral/fileexchange/22022-matlab2tikz-matlab2tikz
%where you can also make suggestions and rate matlab2tikz.
%
\definecolor{mycolor1}{rgb}{1.00000,0.00000,1.00000}%
%
\begin{tikzpicture}[trim axis left, trim axis right]

\begin{axis}[%
width=\figurewidth,
height=\figureheight,
at={(0\figurewidth,0\figureheight)},
scale only axis,
every outer x axis line/.append style={black},
every x tick label/.append style={font=\color{black}},
xmin=0,
xmax=100,
%xlabel={Time},
every outer y axis line/.append style={black},
every y tick label/.append style={font=\color{black}},
ymin=0,
ymax=0.015,
%ylabel={Depth $\delta^+$},
axis background/.style={fill=white},
axis x line*=bottom,
axis y line*=left,
yticklabel style={
        /pgf/number format/fixed,
        /pgf/number format/precision=3
},
scaled y ticks=false,
legend style={legend cell align=left,align=left,draw=black,font=\footnotesize, at={(0.98,0.02)},anchor=south east},
every axis legend/.code={\renewcommand\addlegendentry[2][]{}}  %ignore legend locally
]
\addplot [color=green,dashed]
  table[row sep=crcr]{%
1	0.0110455900738313\\
2	0.0110580658419961\\
3	0.0110711002023015\\
4	0.0110847186127849\\
5	0.011098947011392\\
6	0.0111138115686288\\
7	0.0111293383533924\\
8	0.0111455528900724\\
9	0.0111624795806907\\
10	0.0111801409614611\\
11	0.0111985567600587\\
12	0.0112177427225301\\
13	0.011237709201904\\
14	0.0112584595946247\\
15	0.0112799890528045\\
16	0.0113022851250408\\
17	0.0113253364165301\\
18	0.0113490753902364\\
19	0.011373362331623\\
20	0.011398075956214\\
21	0.0114230352769138\\
22	0.0114479928477834\\
23	0.011471374969726\\
24	0.0114947398506091\\
25	0.0115189546331043\\
26	0.0115440435517189\\
27	0.0115700309248424\\
28	0.0115969413676922\\
29	0.0116248002340417\\
30	0.0116536398840924\\
31	0.0116834886735304\\
32	0.0117143689176055\\
33	0.0117463019183949\\
34	0.0117793081126445\\
35	0.0118134075452996\\
36	0.0118486212132247\\
37	0.0118849754575801\\
38	0.0119221351203645\\
39	0.0119603630217099\\
40	0.012000447587354\\
41	0.0120427263453424\\
42	0.0120891770130887\\
43	0.0121365777521125\\
44	0.0121848349365603\\
45	0.0122338121987253\\
46	0.0122833556034042\\
47	0.0123332665891129\\
48	0.0123833847464205\\
49	0.0124335044426007\\
50	0.0124833659630434\\
51	0.0125327020752989\\
52	0.0125812214612951\\
53	0.0126285171151238\\
54	0.0126739498597923\\
55	0.0127169436838526\\
56	0.0127550898468969\\
57	0.0127919097843038\\
58	0.0128267594141972\\
59	0.0128594959001294\\
60	0.0128926643343223\\
61	0.0129261447385129\\
62	0.0129755266230054\\
63	0.0130286155819347\\
64	0.01308106914823\\
65	0.0131327173915753\\
66	0.0131834776323277\\
67	0.013233398777312\\
68	0.0132714278428884\\
69	0.0133090211115002\\
70	0.0133464232949045\\
71	0.0133874495048413\\
72	0.0134268569837115\\
73	0.0134612592362577\\
74	0.0134943750174216\\
75	0.0135268935899649\\
76	0.0135590004382286\\
77	0.0135874994915551\\
78	0.0136129285920795\\
79	0.0136378832610413\\
80	0.0136617356552583\\
81	0.013683673960379\\
82	0.0137032409774594\\
83	0.0137208351231345\\
84	0.0137367768512176\\
85	0.0137511467620862\\
86	0.0137635406378348\\
87	0.0137746336380689\\
88	0.0137848781595571\\
89	0.0137944730921645\\
90	0.0138038563586677\\
91	0.0138132229079681\\
92	0.0138227275629359\\
93	0.0138326426430243\\
94	0.0138436332014765\\
95	0.0138574154315624\\
96	0.0138783790039005\\
97	0.0139179224337987\\
98	0.0140058180264177\\
99	0\\
100	0\\
};
\addlegendentry{$q=-4$};

\addplot [color=mycolor1,dashed]
  table[row sep=crcr]{%
1	0.00993725649741715\\
2	0.00994743908148045\\
3	0.00995810956404846\\
4	0.00996929519187683\\
5	0.00998102523150166\\
6	0.00999333122873422\\
7	0.0100062473246203\\
8	0.0100198106422231\\
9	0.010034061761987\\
10	0.0100490453075863\\
11	0.0100648106694385\\
12	0.0100814129003547\\
13	0.0100989138287825\\
14	0.0101173834499466\\
15	0.0101369016517615\\
16	0.0101575601408662\\
17	0.0101794630802216\\
18	0.0102027310663123\\
19	0.0102275077167328\\
20	0.010253960817155\\
21	0.0102822909185747\\
22	0.0103127551382908\\
23	0.0103469405010079\\
24	0.0103834219573124\\
25	0.0104214986876766\\
26	0.0104612225610747\\
27	0.0105026329818961\\
28	0.0105457418207953\\
29	0.0105904934466075\\
30	0.0106367925732236\\
31	0.010684886305683\\
32	0.0107349767371514\\
33	0.0107870782513646\\
34	0.0108411770875308\\
35	0.0108972207748731\\
36	0.0109551016116553\\
37	0.0110146171596997\\
38	0.0110679656416736\\
39	0.0111136633318416\\
40	0.0111611611350395\\
41	0.0112104653192837\\
42	0.0112615510488951\\
43	0.0113144945252057\\
44	0.0113693662707487\\
45	0.0114259491659851\\
46	0.0114840983088909\\
47	0.0115433802594881\\
48	0.0116047575583339\\
49	0.0116665806572984\\
50	0.0117303611817939\\
51	0.0117959660250449\\
52	0.011863128475386\\
53	0.011931355243533\\
54	0.0119942627636737\\
55	0.012060544277407\\
56	0.0121325121090892\\
57	0.0122056816426013\\
58	0.0122798155140526\\
59	0.0123546103874823\\
60	0.0124295605965819\\
61	0.0125040499180606\\
62	0.0125773193351345\\
63	0.0126484616186782\\
64	0.0127163888925465\\
65	0.0127800816693448\\
66	0.0128372911983206\\
67	0.0128819721583471\\
68	0.0129377934149165\\
69	0.0129931791207402\\
70	0.0130475756700926\\
71	0.0131005037865686\\
72	0.0131570590102849\\
73	0.0132174657469469\\
74	0.0132670203012515\\
75	0.0133135493019146\\
76	0.0133590018515661\\
77	0.0134065253902449\\
78	0.0134559425450279\\
79	0.0135042967524793\\
80	0.0135514186663528\\
81	0.0135969729039468\\
82	0.0136322586220696\\
83	0.0136635805896388\\
84	0.0136896446077703\\
85	0.0137137713083481\\
86	0.0137356710415909\\
87	0.0137550051541237\\
88	0.0137715958172534\\
89	0.0137865131110447\\
90	0.0137990331112737\\
91	0.0138101317004599\\
92	0.0138208052690158\\
93	0.0138315709036979\\
94	0.0138431850534444\\
95	0.0138573254549391\\
96	0.0138783790039005\\
97	0.0139179224337987\\
98	0.0140058180264177\\
99	0\\
100	0\\
};
\addlegendentry{$q=-3$};

\addplot [color=red,dashed]
  table[row sep=crcr]{%
1	0.00845092557785682\\
2	0.00845656823941849\\
3	0.00846249107576248\\
4	0.00846871059035578\\
5	0.00847524450565671\\
6	0.00848211186795421\\
7	0.00848933315827441\\
8	0.00849693040842963\\
9	0.00850492732109275\\
10	0.00851334939277234\\
11	0.00852222403875616\\
12	0.00853158071912726\\
13	0.00854145106361232\\
14	0.00855186898865415\\
15	0.00856287079869339\\
16	0.00857449531764242\\
17	0.00858678440978791\\
18	0.00859978386050781\\
19	0.00861354459996147\\
20	0.00862812540429216\\
21	0.00864359717802262\\
22	0.00866004806934649\\
23	0.00867748307320201\\
24	0.0086959640757178\\
25	0.00871558977544893\\
26	0.00873644734090162\\
27	0.0087585514122345\\
28	0.00878164749295886\\
29	0.00880459120053278\\
30	0.00882338650485376\\
31	0.0088404307082489\\
32	0.00885883486335252\\
33	0.00887888961031317\\
34	0.00890101799341021\\
35	0.00892590423691198\\
36	0.00895483132840351\\
37	0.0089906163796001\\
38	0.0090408605051884\\
39	0.00910378174186449\\
40	0.00916942222967661\\
41	0.00923795615880716\\
42	0.00930957852060985\\
43	0.00938449145442318\\
44	0.00946288840424228\\
45	0.00954494274252755\\
46	0.00963069465067402\\
47	0.00971976082099793\\
48	0.00980496696703615\\
49	0.00986937744349283\\
50	0.00993814592443868\\
51	0.0100119720976435\\
52	0.010091464182447\\
53	0.0101774151149637\\
54	0.0102763767387847\\
55	0.0103797217785801\\
56	0.0104873311297228\\
57	0.0105994090813687\\
58	0.0107161515061263\\
59	0.0108377369718138\\
60	0.010964323063165\\
61	0.0110960446215847\\
62	0.0112330516965564\\
63	0.0113745177345102\\
64	0.0115190868409274\\
65	0.0116689361195459\\
66	0.0118068149881645\\
67	0.011921020819691\\
68	0.0120361397725554\\
69	0.0121513860103602\\
70	0.0122653080138773\\
71	0.0123653408301589\\
72	0.0124650462912041\\
73	0.0125640957472384\\
74	0.0126715293894912\\
75	0.0127771856867971\\
76	0.0128766812873347\\
77	0.0129617370189626\\
78	0.0130429843392688\\
79	0.0131194818731913\\
80	0.0131910198144678\\
81	0.0132530081282395\\
82	0.0133214089066423\\
83	0.0133861696491701\\
84	0.0134477731984394\\
85	0.0135067464386294\\
86	0.0135631957202715\\
87	0.013616831142668\\
88	0.0136673924845396\\
89	0.0137135717499587\\
90	0.0137455637238906\\
91	0.0137691815452557\\
92	0.0137904700127987\\
93	0.0138105125141738\\
94	0.0138302301789764\\
95	0.0138515483058692\\
96	0.0138771841329783\\
97	0.0139179224337987\\
98	0.0140058180264177\\
99	0\\
100	0\\
};
\addlegendentry{$q=-2$};

\addplot [color=blue,dashed]
  table[row sep=crcr]{%
1	0.00681554850718535\\
2	0.00681588973092251\\
3	0.00681624790604493\\
4	0.00681662403229282\\
5	0.00681701918263336\\
6	0.00681743450916522\\
7	0.00681787124928804\\
8	0.00681833073211409\\
9	0.0068188143851154\\
10	0.00681932374100495\\
11	0.00681986044480288\\
12	0.00682042626094946\\
13	0.0068210230805929\\
14	0.00682165293148809\\
15	0.00682231800084379\\
16	0.00682302069294455\\
17	0.00682376371377654\\
18	0.00682455017083602\\
19	0.00682538367532876\\
20	0.00682626823522245\\
21	0.00682720746955256\\
22	0.0068282024376938\\
23	0.00682925825974799\\
24	0.0068303846220623\\
25	0.00683159975836174\\
26	0.00683294794645898\\
27	0.00683455738287878\\
28	0.00683683256702084\\
29	0.00684109416742677\\
30	0.00685170549342327\\
31	0.00686637214688786\\
32	0.006882117146803\\
33	0.00689901996077394\\
34	0.00691716712181518\\
35	0.00693666354110884\\
36	0.00695764583303007\\
37	0.00698024282228485\\
38	0.00700415259648881\\
39	0.00702896958052679\\
40	0.00705474132081956\\
41	0.00708151927823191\\
42	0.00710935815038134\\
43	0.0071383147682871\\
44	0.00716844468766515\\
45	0.00719978697145568\\
46	0.00723232259049564\\
47	0.00726586038409361\\
48	0.00730028131436559\\
49	0.00733651719634623\\
50	0.00737469959653813\\
51	0.00741498112945589\\
52	0.0074575578790814\\
53	0.00750268940781908\\
54	0.00755025072979743\\
55	0.00760052206817181\\
56	0.00765385351790057\\
57	0.00771066220684896\\
58	0.00777144807873026\\
59	0.00783681469444389\\
60	0.00790749797473147\\
61	0.00798441341844466\\
62	0.00806875988614439\\
63	0.00816244126907436\\
64	0.00826904811391383\\
65	0.00839598143236846\\
66	0.00856300219363242\\
67	0.00877525537066877\\
68	0.00899933696993162\\
69	0.00923504581303584\\
70	0.00947869461855486\\
71	0.00971000767935811\\
72	0.00990465437474219\\
73	0.0101095951680048\\
74	0.0103258548766747\\
75	0.0105546924270711\\
76	0.0107977370384361\\
77	0.0110620631704474\\
78	0.0113293632347916\\
79	0.0115421906689915\\
80	0.011763453473767\\
81	0.0119973082073932\\
82	0.0122334666990685\\
83	0.0124319372635047\\
84	0.0125612878259806\\
85	0.0126817069884855\\
86	0.0127970668994875\\
87	0.0129045895389597\\
88	0.0130013449527564\\
89	0.0131014145585904\\
90	0.0132126248377904\\
91	0.0133188642268457\\
92	0.0134158310019865\\
93	0.013511266137075\\
94	0.0136047815916573\\
95	0.0136968616847195\\
96	0.013792309010026\\
97	0.0138803621983305\\
98	0.0140058180264177\\
99	0\\
100	0\\
};
\addlegendentry{$q=-1$};

\addplot [color=black,solid]
  table[row sep=crcr]{%
1	0.00621792783154248\\
2	0.00621795363151064\\
3	0.006217980731081\\
4	0.00621800920860938\\
5	0.00621803914829472\\
6	0.00621807064061459\\
7	0.00621810378283594\\
8	0.0062181386795362\\
9	0.00621817544314337\\
10	0.00621821419451566\\
11	0.00621825506366069\\
12	0.0062182981909585\\
13	0.00621834372977078\\
14	0.00621839185148292\\
15	0.00621844275128388\\
16	0.00621849665212127\\
17	0.00621855381158176\\
18	0.0062186145199329\\
19	0.00621867908373063\\
20	0.00621874781355397\\
21	0.0062188211347773\\
22	0.00621890017299207\\
23	0.00621898736594281\\
24	0.00621908806271129\\
25	0.0062192143396067\\
26	0.00621939307734149\\
27	0.00621968087587299\\
28	0.00622018200515726\\
29	0.00622102489640633\\
30	0.006222065928412\\
31	0.00622318211632672\\
32	0.00622437914036729\\
33	0.00622566332427819\\
34	0.00622704153512148\\
35	0.00622851935407994\\
36	0.00623009471218396\\
37	0.00623174686285373\\
38	0.00623345815815957\\
39	0.00623523145880468\\
40	0.0062370697610722\\
41	0.00623897606923901\\
42	0.0062409531792925\\
43	0.00624300321551038\\
44	0.0062451266250654\\
45	0.00624732106274971\\
46	0.00624958270784438\\
47	0.00625192203351148\\
48	0.00625437951091605\\
49	0.0062569642653318\\
50	0.00625968667240331\\
51	0.00626255775449155\\
52	0.00626558564303955\\
53	0.00626876738880855\\
54	0.00627212092530033\\
55	0.00627566847937362\\
56	0.00627943664884499\\
57	0.00628345771370841\\
58	0.00628777189884787\\
59	0.0062924320428559\\
60	0.00629751206215357\\
61	0.00630312702680858\\
62	0.00630947542351325\\
63	0.00631689409173776\\
64	0.006325890501855\\
65	0.00633704058716358\\
66	0.00634993194303342\\
67	0.00636319983109054\\
68	0.00637669793633343\\
69	0.00639006327006573\\
70	0.00640259934075841\\
71	0.00641406232098269\\
72	0.00642600457558528\\
73	0.00643861422240916\\
74	0.00645234168823419\\
75	0.00646831096392316\\
76	0.00648942213072244\\
77	0.00652294593690508\\
78	0.0065899276595373\\
79	0.00672850623736217\\
80	0.00688408450938001\\
81	0.00706463096777866\\
82	0.00728639351234022\\
83	0.00758917216479368\\
84	0.00797753818783012\\
85	0.00838912380471636\\
86	0.00881866262566954\\
87	0.00926903541158117\\
88	0.00974152623661196\\
89	0.0102143184308963\\
90	0.0106757434510928\\
91	0.0110774318305254\\
92	0.0113693047466867\\
93	0.0116714488375805\\
94	0.0119840727133986\\
95	0.0123074839379101\\
96	0.0126303225374715\\
97	0.0130412667576705\\
98	0.0136048426333631\\
99	0\\
100	0\\
};
\addlegendentry{$q=0$};

\addplot [color=blue,solid]
  table[row sep=crcr]{%
1	0.0139983083805517\\
2	0.0139983082720273\\
3	0.0139983081574379\\
4	0.0139983080363954\\
5	0.0139983079090009\\
6	0.0139983077753098\\
7	0.0139983076341143\\
8	0.0139983074828703\\
9	0.0139983073165798\\
10	0.0139983071255936\\
11	0.0139983068924995\\
12	0.0139983065898676\\
13	0.0139983061849815\\
14	0.0139983056621921\\
15	0.013998305057157\\
16	0.0139983044295312\\
17	0.0139983037771064\\
18	0.0139983030960604\\
19	0.0139983023790402\\
20	0.0139983016112709\\
21	0.0139983007634654\\
22	0.0139982997814332\\
23	0.0139982985788664\\
24	0.013998297060026\\
25	0.0139982952283117\\
26	0.0139982933076538\\
27	0.0139982913154207\\
28	0.0139982892426073\\
29	0.0139982870772956\\
30	0.0139982848063752\\
31	0.0139982824215368\\
32	0.0139982799282731\\
33	0.0139982773437443\\
34	0.0139982746681239\\
35	0.0139982718935258\\
36	0.013998269010109\\
37	0.0139982660076548\\
38	0.0139982628758252\\
39	0.0139982596013694\\
40	0.0139982561655939\\
41	0.0139982525386186\\
42	0.0139982486663215\\
43	0.0139982444414905\\
44	0.0139982396436937\\
45	0.0139982338273148\\
46	0.0139982261626455\\
47	0.0139982153866421\\
48	0.0139982004346597\\
49	0.013998182158501\\
50	0.0139981625593857\\
51	0.0139981420252606\\
52	0.0139981206092228\\
53	0.013998098155511\\
54	0.0139980744000645\\
55	0.0139980488305364\\
56	0.0139980203327613\\
57	0.0139979865168737\\
58	0.0139979432389766\\
59	0.0139976318273502\\
60	0.0139972043922104\\
61	0.0139967529347835\\
62	0.0139962718635278\\
63	0.01399575394748\\
64	0.013995194357183\\
65	0.0139945963173215\\
66	0.0139939544103745\\
67	0.0139932487227211\\
68	0.0139924440877367\\
69	0.0139911937998497\\
70	0.0139898463883936\\
71	0.0139884163899688\\
72	0.0139868807363929\\
73	0.0139852082736702\\
74	0.013983356982005\\
75	0.0139812313196168\\
76	0.0139777517095134\\
77	0.013973616531471\\
78	0.0139691494821928\\
79	0.013964214158185\\
80	0.0139514536016275\\
81	0.0139344793329246\\
82	0.0139107111233564\\
83	0.0138838507138313\\
84	0.0138477349623043\\
85	0.0138081269900492\\
86	0.013736240631162\\
87	0.0136473980836543\\
88	0.0135470430361267\\
89	0.0133243650786069\\
90	0.0130820992896335\\
91	0.0126355529612964\\
92	0.0121478036924208\\
93	0.0116210555269654\\
94	0.0110409883808773\\
95	0.0103634447591698\\
96	0.00918448118066964\\
97	0.00722167411404017\\
98	0.00400581802641767\\
99	0\\
100	0\\
};
\addlegendentry{$q=1$};

\addplot [color=red,solid]
  table[row sep=crcr]{%
1	0.0139245392166005\\
2	0.0139245378203568\\
3	0.0139245363514915\\
4	0.0139245347992432\\
5	0.0139245331576809\\
6	0.0139245314317816\\
7	0.0139245296261191\\
8	0.0139245277306554\\
9	0.0139245257231864\\
10	0.0139245235575464\\
11	0.0139245211335468\\
12	0.0139245182386413\\
13	0.0139245144653401\\
14	0.0139245092097827\\
15	0.0139245021355708\\
16	0.0139244939094515\\
17	0.0139244853833913\\
18	0.0139244765389551\\
19	0.0139244673498546\\
20	0.0139244577696135\\
21	0.0139244476988419\\
22	0.013924436901818\\
23	0.0139244248103099\\
24	0.0139244101461545\\
25	0.0139243906702063\\
26	0.013924237896993\\
27	0.0139240662485454\\
28	0.0139238888741228\\
29	0.0139237054071376\\
30	0.0139235153981125\\
31	0.0139233183180689\\
32	0.0139231136539381\\
33	0.0139229011275037\\
34	0.0139226805548886\\
35	0.013922451429106\\
36	0.0139222130570585\\
37	0.0139219646638605\\
38	0.0139217054655661\\
39	0.0139214346687592\\
40	0.0139211513825356\\
41	0.0139208545924508\\
42	0.0139205431123277\\
43	0.0139202154806119\\
44	0.0139198697130677\\
45	0.0139195026790389\\
46	0.0139191085170946\\
47	0.0139186749789327\\
48	0.0139181780829358\\
49	0.0139175226027776\\
50	0.0139166661680981\\
51	0.0139157672933066\\
52	0.0139148310113012\\
53	0.0139138566709414\\
54	0.0139128405886136\\
55	0.0139117783846599\\
56	0.0139106646228975\\
57	0.0139094885576432\\
58	0.0139082219134894\\
59	0.0139035937526516\\
60	0.0138974969498809\\
61	0.0138910967847774\\
62	0.0138843455455878\\
63	0.0138771613390056\\
64	0.0138694092466516\\
65	0.0138603275814683\\
66	0.0138503406141517\\
67	0.0138397361302318\\
68	0.0138282880259625\\
69	0.0138112612958869\\
70	0.0137930111444582\\
71	0.0137737719994472\\
72	0.0137533280914355\\
73	0.0137313371446556\\
74	0.0137031626087889\\
75	0.0136697451461063\\
76	0.0136193502677612\\
77	0.0135604069971424\\
78	0.0134973267826613\\
79	0.0134292481808521\\
80	0.0133626213272515\\
81	0.0132526603301399\\
82	0.0130496754137029\\
83	0.0128320506032241\\
84	0.012603996387161\\
85	0.0123530187471663\\
86	0.0119599087030613\\
87	0.0114832829858986\\
88	0.010975074275643\\
89	0.0105572257988587\\
90	0.0101048122462807\\
91	0.00982588095647541\\
92	0.00954189884861001\\
93	0.00922397477561944\\
94	0.00880247522304508\\
95	0.00769651298244612\\
96	0.00673001832143179\\
97	0.00574637998611275\\
98	0.00400581802641767\\
99	0\\
100	0\\
};
\addlegendentry{$q=2$};

\addplot [color=mycolor1,solid]
  table[row sep=crcr]{%
1	0.0138659281825004\\
2	0.0138658303664556\\
3	0.0138657284708293\\
4	0.0138656221337992\\
5	0.0138655109202661\\
6	0.0138653944050993\\
7	0.0138652723292665\\
8	0.0138651444583232\\
9	0.0138650103605435\\
10	0.0138648694949254\\
11	0.0138647211728643\\
12	0.0138645643799941\\
13	0.0138643972884491\\
14	0.0138642160275985\\
15	0.0138640125094838\\
16	0.0138637564037386\\
17	0.0138634166197179\\
18	0.0138630640967847\\
19	0.0138626981450955\\
20	0.0138623180178758\\
21	0.0138619228880535\\
22	0.0138615117666275\\
23	0.0138610831948392\\
24	0.0138606340410762\\
25	0.0138601546144108\\
26	0.0138580041589181\\
27	0.0138556098224583\\
28	0.0138531361555401\\
29	0.0138505790042018\\
30	0.0138479335717341\\
31	0.0138451940841394\\
32	0.0138423533955504\\
33	0.013839403382497\\
34	0.0138363387669431\\
35	0.0138331568375606\\
36	0.0138298501579202\\
37	0.0138264085172796\\
38	0.0138228207826792\\
39	0.0138190758423261\\
40	0.0138151622390994\\
41	0.0138110672022985\\
42	0.0138067764237643\\
43	0.0138022737649553\\
44	0.0137975408243028\\
45	0.0137925561702599\\
46	0.0137872935733119\\
47	0.0137817157837313\\
48	0.0137757454786795\\
49	0.0137682035150454\\
50	0.0137582480484675\\
51	0.0137477773588493\\
52	0.0137362444823498\\
53	0.013724017890704\\
54	0.0137112819435204\\
55	0.013697989944474\\
56	0.0136840876981937\\
57	0.0136695166819895\\
58	0.0136542206517211\\
59	0.0136417400219856\\
60	0.0136299858319327\\
61	0.0136175957092921\\
62	0.0136044746383344\\
63	0.0135904563512442\\
64	0.0135751176689904\\
65	0.0135482951130647\\
66	0.0135139517849026\\
67	0.0134729206133208\\
68	0.0134299869659956\\
69	0.0133901687966344\\
70	0.0133483630254395\\
71	0.013303721695035\\
72	0.0132556063839647\\
73	0.0132029130739205\\
74	0.0130893343139838\\
75	0.0129335155393375\\
76	0.0127833506564155\\
77	0.0126277866300801\\
78	0.0124475946068981\\
79	0.0122503212283224\\
80	0.0120352698675127\\
81	0.0118382330606605\\
82	0.0115127760722277\\
83	0.0111610068162654\\
84	0.0107917217872007\\
85	0.0104026883932588\\
86	0.0101489457463948\\
87	0.00995666357602472\\
88	0.00976697868687867\\
89	0.00957643399279868\\
90	0.00939558684152857\\
91	0.00920822989858796\\
92	0.00897855980677334\\
93	0.00842245474939861\\
94	0.00748614997919531\\
95	0.00709933536237606\\
96	0.00659458310959697\\
97	0.00574637998611275\\
98	0.00400581802641767\\
99	0\\
100	0\\
};
\addlegendentry{$q=3$};

\addplot [color=green,solid]
  table[row sep=crcr]{%
1	0.0137934540964822\\
2	0.0137920822172213\\
3	0.0137906552941932\\
4	0.0137891692232524\\
5	0.0137876190349365\\
6	0.0137859986619449\\
7	0.0137843018531052\\
8	0.0137825244217185\\
9	0.0137806633430117\\
10	0.0137787127499051\\
11	0.0137766655642113\\
12	0.0137745137387683\\
13	0.0137722477079747\\
14	0.013769854326488\\
15	0.0137673076174875\\
16	0.0137642538154483\\
17	0.0137601599183347\\
18	0.0137559135221988\\
19	0.0137515064838808\\
20	0.0137469300826655\\
21	0.0137421750415046\\
22	0.0137372316698655\\
23	0.0137320903609994\\
24	0.0137267431809246\\
25	0.0137211891919787\\
26	0.0137172292745451\\
27	0.0137133172931002\\
28	0.0137092661503339\\
29	0.0137050702508529\\
30	0.0137007235066934\\
31	0.0136962183852499\\
32	0.0136915431315267\\
33	0.013686674884002\\
34	0.013681571007666\\
35	0.0136758920841648\\
36	0.0136699121098025\\
37	0.0136637467300657\\
38	0.0136573747939205\\
39	0.013650775046622\\
40	0.0136439353666763\\
41	0.0136368423034981\\
42	0.0136294807701638\\
43	0.0136218336291618\\
44	0.0136138811660326\\
45	0.0136056004152668\\
46	0.0135969649725736\\
47	0.0135879508647332\\
48	0.013578571937541\\
49	0.0135700573734081\\
50	0.0135632395167464\\
51	0.0135557162512302\\
52	0.0135400632882504\\
53	0.01352087067752\\
54	0.0135008948792161\\
55	0.0134800769465166\\
56	0.0134583318837203\\
57	0.0134355222927865\\
58	0.0134101765001236\\
59	0.0133800158311734\\
60	0.0133483586758046\\
61	0.0133150073984438\\
62	0.0132797599746472\\
63	0.01324241044225\\
64	0.0132028895590607\\
65	0.0131717196046901\\
66	0.013128624100755\\
67	0.0130202144267264\\
68	0.0129061707761707\\
69	0.0127855041042009\\
70	0.0126573033069973\\
71	0.0125204864924476\\
72	0.0123736180735568\\
73	0.0122151095257336\\
74	0.0121008584942676\\
75	0.0120087275463001\\
76	0.0119099762978608\\
77	0.0118014285062564\\
78	0.0115223760489935\\
79	0.0111965821677563\\
80	0.0108583570943391\\
81	0.0105063547064552\\
82	0.0103502268726332\\
83	0.0102060253919138\\
84	0.0100646509320887\\
85	0.00993094654733278\\
86	0.00979740506831434\\
87	0.00966105444861244\\
88	0.00952161592045265\\
89	0.00937462806720677\\
90	0.00921206089831327\\
91	0.00900704371056902\\
92	0.00838247713451574\\
93	0.00771699577961124\\
94	0.00743248223339164\\
95	0.00708638392997788\\
96	0.00659458310959697\\
97	0.00574637998611275\\
98	0.00400581802641767\\
99	0\\
100	0\\
};
\addlegendentry{$q=4$};

\end{axis}
\end{tikzpicture}%
 
  \caption{Discrete Time w/ nFPC}
\end{subfigure}\\

\leavevmode\smash{\makebox[0pt]{\hspace{-7em}% HORIZONTAL POSITION           
  \rotatebox[origin=l]{90}{\hspace{20em}% VERTICAL POSITION
    Depth $\delta^-$}%
}}\hspace{0pt plus 1filll}\null

Time (s)

\vspace{1cm}
\begin{subfigure}{\linewidth}
  \centering
  \tikzsetnextfilename{deltalegend}
  \documentclass{article}
\usepackage{pgfplots}
\usetikzlibrary{backgrounds}
\pgfplotsset{compat=newest}  
\newlength\figureheight 
\newlength\figurewidth 

\begin{document}
%
%\begin{figure}
%  \centering
%  \setlength\figureheight{\linewidth} 
%  \setlength\figurewidth{\linewidth}
%  \input{/home/anton/Documents/masc/ml/thesis/tikz/ORCL_comp4stoch.tikz}
%  \caption{Backtest strategy comparison}
%  \label{fig:insample}
%\end{figure}
\definecolor{mycolor1}{rgb}{1.00000,0.00000,1.00000}%
\begin{tikzpicture}[framed]
    \begingroup
    % inits/clears the lists (which might be populated from previous
    % axes):
    \csname pgfplots@init@cleared@structures\endcsname
    \pgfplotsset{legend style={at={(0,1)},anchor=north west},legend columns=-1,legend style={draw=none,column sep=1ex},legend entries={$q=-4$,$q=-3$,$q=-2$,$q=-1$}}%
    
    \csname pgfplots@addlegendimage\endcsname{thick,green,dashed,sharp plot}
    \csname pgfplots@addlegendimage\endcsname{thick,mycolor1,dashed,sharp plot}
    \csname pgfplots@addlegendimage\endcsname{thick,red,dashed,sharp plot}
    \csname pgfplots@addlegendimage\endcsname{thick,blue,dashed,sharp plot}

    % draws the legend:
    \csname pgfplots@createlegend\endcsname
    \endgroup

    \begingroup
    % inits/clears the lists (which might be populated from previous
    % axes):
    \csname pgfplots@init@cleared@structures\endcsname
    \pgfplotsset{legend style={at={(3.45,0.5)},anchor=north west},legend columns=-1,legend style={draw=none,column sep=1ex},legend entries={$q=0$}}%

    \csname pgfplots@addlegendimage\endcsname{thick,black,sharp plot}

    % draws the legend:
    \csname pgfplots@createlegend\endcsname
    \endgroup

    \begingroup
    % inits/clears the lists (which might be populated from previous
    % axes):
    \csname pgfplots@init@cleared@structures\endcsname
    \pgfplotsset{legend style={at={(0,0)},anchor=north west},legend columns=-1,legend style={draw=none,column sep=1ex},legend entries={$q=+4$,$q=+3$,$q=+2$,$q=+1$}}%
    
    \csname pgfplots@addlegendimage\endcsname{thick,green,sharp plot}
    \csname pgfplots@addlegendimage\endcsname{thick,mycolor1,sharp plot}
    \csname pgfplots@addlegendimage\endcsname{thick,red,sharp plot}
    \csname pgfplots@addlegendimage\endcsname{thick,blue,sharp plot}

    % draws the legend:
    \csname pgfplots@createlegend\endcsname
    \endgroup
\end{tikzpicture}

\end{document} 
\end{subfigure}%
  \caption{Optimal sell depths $\delta^{-}$ for Markov state $Z=(\rho = +1, \Delta S = +1)$, implying heavy imbalance in favor of buy pressure, and having previously seen an upward price change. We expect the midprice to rise.}
  \label{fig:comp_dm_z15}
\end{figure}

\FloatBarrier
\subsection{Comparing Performance}

\begin{figure}
  \centering
\begin{subfigure}{\linewidth}
  \setlength\figureheight{0.5\linewidth} 
  \setlength\figurewidth{\linewidth}
  \tikzsetnextfilename{ORCL_comp4stoch}
  % This file was created by matlab2tikz.
%
%The latest updates can be retrieved from
%  http://www.mathworks.com/matlabcentral/fileexchange/22022-matlab2tikz-matlab2tikz
%where you can also make suggestions and rate matlab2tikz.
%
%
\begin{tikzpicture}[trim axis left, trim axis right]

\begin{axis}[%
width=\figurewidth,
height=\figureheight,
at={(0\figurewidth,0\figureheight)},
scale only axis,
every outer x axis line/.append style={black},
every x tick label/.append style={font=\color{black}},
xmin=9.5,
xmax=16,
xlabel={Time (h)},
every outer y axis line/.append style={black},
every y tick label/.append style={font=\color{black}},
ymin=-0.1,
ymax=0.25,
ylabel={Normalized PnL},
title={Strategy Performance using Optimal Stochastic Control},
axis background/.style={fill=white},
axis x line*=bottom,
axis y line*=left,
yticklabel style={
        /pgf/number format/fixed,
        /pgf/number format/precision=3
},
scaled y ticks=false,
legend style={legend cell align=left,align=left,draw=black,font=\small, legend pos=north west},
]
\addplot [color=cts_plot_color,solid,line width=1.5pt]
  table[row sep=crcr]{%
9.50027777777778	0.013029315960912\\
9.50583333333333	0.027864698814415\\
9.51138888888889	0.0658672429738636\\
9.51694444444444	0.00800661523744189\\
9.5225	-0.000899886766644353\\
9.52805555555556	0.00451407618006027\\
9.53361111111111	-0.00904705816935735\\
9.53916666666667	-0.000320570733841338\\
9.54472222222222	-0.01231214894334\\
9.55027777777778	-0.00528945411466411\\
9.55583333333333	-0.00739204427562812\\
9.56138888888889	0.000106265970203148\\
9.56694444444444	0.00217911169125714\\
9.5725	0.00646505653289955\\
9.57805555555555	0.00646505653289955\\
9.58361111111111	0.00054264018703161\\
9.58916666666667	0.0096815309218638\\
9.59472222222222	0.00296710083368474\\
9.60027777777778	0.00771941328838683\\
9.60583333333333	-0.00898343481041701\\
9.61138888888889	-0.00868731399312383\\
9.61694444444444	-0.00795347759948156\\
9.6225	-0.0061767526957216\\
9.62805555555556	-0.00344893584730434\\
9.63361111111111	-0.00653368454878604\\
9.63916666666667	-0.0060742050735985\\
9.64472222222222	-0.00504622344556753\\
9.65027777777778	-0.00466585690483796\\
9.65583333333333	-0.00332466050015554\\
9.66138888888889	-0.00560834577342344\\
9.66694444444444	-0.00442386250424964\\
9.6725	-0.00396878405855884\\
9.67805555555555	-0.0061847477406465\\
9.68361111111111	-0.00304082251739974\\
9.68916666666667	-0.00352504084216268\\
9.69472222222222	-0.00468274725120326\\
9.70027777777778	-0.00409050561661647\\
9.70583333333333	-0.00352182472404664\\
9.71138888888889	-0.00352182472404664\\
9.71694444444444	-0.00559467044510063\\
9.7225	-0.00301382881289665\\
9.72805555555555	-0.00172047049919201\\
9.73361111111111	-0.003793316220246\\
9.73916666666667	-0.003793316220246\\
9.74472222222222	-0.003793316220246\\
9.75027777777778	-0.00260883295107241\\
9.75583333333333	-0.00142434968189882\\
9.76138888888889	-0.0023969578572156\\
9.76694444444444	-0.0021797023062031\\
9.7725	-0.00496464710001442\\
9.77805555555556	-0.00489037631665236\\
9.78361111111111	-0.00480617153028672\\
9.78916666666667	-0.00282633654598255\\
9.79472222222222	-0.00340977223018559\\
9.80027777777778	-0.00266751092221102\\
9.80583333333333	-0.00266751092221102\\
9.81138888888889	-0.00416309586273107\\
9.81694444444444	-0.00430984138049258\\
9.8225	-0.00182137887900532\\
9.82805555555555	-0.00525113525629357\\
9.83361111111111	-0.00377053116982658\\
9.83916666666667	-0.00317828953523979\\
9.84472222222222	-0.00244557163978926\\
9.85027777777778	-0.00228992708335939\\
9.85583333333333	-0.00244557163978926\\
9.86138888888889	-0.00228885850428042\\
9.86694444444444	-0.00196799883741226\\
9.8725	-0.00115641567568231\\
9.87805555555556	-0.0009445405818255\\
9.88361111111111	-0.00050794350366816\\
9.88916666666667	-0.00050794350366816\\
9.89472222222222	-0.000172590448823596\\
9.90027777777778	-0.000172590448823596\\
9.90583333333333	-0.000468711266116782\\
9.91138888888889	-3.45701752148681e-05\\
9.91694444444444	-3.45701752148681e-05\\
9.9225	-3.45701752148681e-05\\
9.92805555555555	-3.45701752148681e-05\\
9.93361111111111	-3.45701752148681e-05\\
9.93916666666667	0.000122142960293971\\
9.94472222222222	0.000619183666677032\\
9.95027777777778	0.000159025846504827\\
9.95583333333333	0.0020641663274063\\
9.96138888888889	0.00628731612929569\\
9.96694444444444	-0.00347149427904405\\
9.9725	-0.00435985673092445\\
9.97805555555555	-0.00272275100331465\\
9.98361111111111	-0.00272275100331465\\
9.98916666666667	0.00131491712896905\\
9.99472222222222	0.00190715876355626\\
10.0002777777778	0.00190715876355626\\
10.0058333333333	0.00190715876355626\\
10.0113888888889	0.00190715876355626\\
10.0169444444444	-0.00203761278946184\\
10.0225	-0.00347059396614383\\
10.0280555555556	-0.00347059396614383\\
10.0336111111111	-0.00317447314885065\\
10.0391666666667	-0.00317447314885065\\
10.0447222222222	-0.00232971738015875\\
10.0502777777778	-0.00203359656286556\\
10.0558333333333	-0.0018142550357224\\
10.0613888888889	-0.0018142550357224\\
10.0669444444444	-0.00373857559092975\\
10.0725	-0.00373857559092975\\
10.0780555555556	-0.00856739625456624\\
10.0836111111111	-0.00856739625456624\\
10.0891666666667	-0.00856739625456624\\
10.0947222222222	-0.00856739625456624\\
10.1002777777778	-0.000448511857814258\\
10.1058333333333	0.00110647108986034\\
10.1113888888889	0.00110647108986034\\
10.1169444444444	-0.00218001594391252\\
10.1225	-0.00218001594391252\\
10.1280555555556	0.00285403795007586\\
10.1336111111111	0.00285403795007586\\
10.1391666666667	0.00285403795007586\\
10.1447222222222	-0.000536300867729521\\
10.1502777777778	-0.000536300867729521\\
10.1558333333333	-0.000536300867729521\\
10.1613888888889	-0.000536300867729521\\
10.1669444444444	-0.00254134460225484\\
10.1725	-0.00194910296766804\\
10.1780555555556	-0.00194910296766804\\
10.1836111111111	-0.00156367267896288\\
10.1891666666667	-0.00156367267896288\\
10.1947222222222	-0.00156367267896288\\
10.2002777777778	-0.00156367267896288\\
10.2058333333333	-0.00156367267896288\\
10.2113888888889	-0.00156367267896288\\
10.2169444444444	-0.00156367267896288\\
10.2225	-0.00137557517149398\\
10.2280555555556	-0.000882121572964113\\
10.2336111111111	-0.00160644702196507\\
10.2391666666667	-0.00138919147095215\\
10.2447222222222	-0.000712704112929402\\
10.2502777777778	-0.00109307065365897\\
10.2558333333333	-0.000500829019071751\\
10.2613888888889	0.00210909637822064\\
10.2669444444444	0.00131952196239759\\
10.2725	0.00434597910215365\\
10.2780555555556	0.00138477092921842\\
10.2836111111111	-0.000194621193305707\\
10.2891666666667	-0.00155474511477684\\
10.2947222222222	-0.00155474511477684\\
10.3002777777778	-0.000389400856758182\\
10.3058333333333	-0.000389400856758182\\
10.3113888888889	-0.00135662857491843\\
10.3169444444444	0.00160457959801596\\
10.3225	0.000716217146135561\\
10.3280555555556	0.000856693406998873\\
10.3336111111111	-3.16690448815278e-05\\
10.3391666666667	-3.16690448815278e-05\\
10.3447222222222	-0.000623910679468532\\
10.3502777777778	0.000264451772411659\\
10.3558333333333	0.00129329048515537\\
10.3613888888889	0.00129329048515537\\
10.3669444444444	0.00129329048515537\\
10.3725	0.00129329048515537\\
10.3780555555556	0.00129329048515537\\
10.3836111111111	0.00129329048515537\\
10.3891666666667	0.00129329048515537\\
10.3947222222222	0.00129329048515537\\
10.4002777777778	0.00145000362066463\\
10.4058333333333	0.00145000362066463\\
10.4113888888889	0.00204224525525143\\
10.4169444444444	0.00204224525525143\\
10.4225	0.00204224525525143\\
10.4280555555556	0.00282258439730755\\
10.4336111111111	0.0036121588131306\\
10.4391666666667	0.0036121588131306\\
10.4447222222222	0.00424128333392782\\
10.4502777777778	0.00424128333392782\\
10.4558333333333	0.00424128333392782\\
10.4613888888889	0.00424128333392782\\
10.4669444444444	0.00424128333392782\\
10.4725	0.00474927924507824\\
10.4780555555556	0.00563764169695864\\
10.4836111111111	0.00563764169695864\\
10.4891666666667	0.00496862077222182\\
10.4947222222222	0.00496862077222182\\
10.5002777777778	0.00496862077222182\\
10.5058333333333	0.00666109995254624\\
10.5113888888889	0.00622535995910163\\
10.5169444444444	0.00659911715125716\\
10.5225	0.00659911715125716\\
10.5280555555556	0.00711249351956411\\
10.5336111111111	0.00827783777758298\\
10.5391666666667	0.00857395859487638\\
10.5447222222222	0.00857395859487638\\
10.5502777777778	0.00768559614299597\\
10.5558333333333	0.00827783777758319\\
10.5613888888889	0.00827783777758319\\
10.5669444444444	0.0100354236701891\\
10.5725	0.00855481958372186\\
10.5780555555556	0.00885094040101505\\
10.5836111111111	0.00885094040101505\\
10.5891666666667	0.00914706121830824\\
10.5947222222222	0.00914706121830824\\
10.6002777777778	0.00752683795569004\\
10.6058333333333	0.00752683795569004\\
10.6113888888889	0.00618671013008574\\
10.6169444444444	0.00618671013008574\\
10.6225	0.00618671013008574\\
10.6280555555556	0.00811907959027641\\
10.6336111111111	0.00811907959027641\\
10.6391666666667	0.00682657301127855\\
10.6447222222222	0.00682657301127855\\
10.6502777777778	0.00851905219160297\\
10.6558333333333	0.00851905219160297\\
10.6613888888889	0.0104353962715148\\
10.6669444444444	0.0128043628098624\\
10.6725	0.00883431202005109\\
10.6780555555556	0.0091304328373447\\
10.6836111111111	0.0115209781086832\\
10.6891666666667	0.0115209781086832\\
10.6947222222222	0.0115209781086832\\
10.7002777777778	0.00960463402877141\\
10.7058333333333	0.0118170989259764\\
10.7113888888889	0.0144630472704625\\
10.7169444444444	0.0144630472704625\\
10.7225	0.0144630472704625\\
10.7280555555556	0.0124093405605632\\
10.7336111111111	0.0107891172979446\\
10.7391666666667	0.0107891172979446\\
10.7447222222222	0.00949661071894671\\
10.7502777777778	0.00821333941371577\\
10.7558333333333	0.00821333941371577\\
10.7613888888889	0.0130500605264679\\
10.7669444444444	0.0136423021610547\\
10.7725	0.015932403433122\\
10.7780555555556	0.015932403433122\\
10.7836111111111	0.015932403433122\\
10.7891666666667	0.015932403433122\\
10.7947222222222	0.015932403433122\\
10.8002777777778	0.0162093852392607\\
10.8058333333333	0.0165055060565539\\
10.8113888888889	0.0165055060565539\\
10.8169444444444	0.0145891619766421\\
10.8225	0.0145891619766421\\
10.8280555555556	0.0151814036112289\\
10.8336111111111	0.0151814036112289\\
10.8391666666667	0.0119181831642087\\
10.8447222222222	0.0171710136531893\\
10.8502777777778	0.0171710136531893\\
10.8558333333333	0.0157704837643672\\
10.8613888888889	0.0179231813686995\\
10.8669444444444	0.0205094034580605\\
10.8725	0.0211016450926473\\
10.8780555555556	0.0211016450926473\\
10.8836111111111	0.0211016450926473\\
10.8891666666667	0.0211016450926473\\
10.8947222222222	0.0211016450926473\\
10.9002777777778	0.0211016450926473\\
10.9058333333333	0.0211016450926473\\
10.9113888888889	0.019762635765236\\
10.9169444444444	0.019762635765236\\
10.9225	0.0178302663050457\\
10.9280555555556	0.0178302663050457\\
10.9336111111111	0.0143813430606276\\
10.9391666666667	0.0200869720081789\\
10.9447222222222	0.0144303454571024\\
10.9502777777778	0.0165874197808524\\
10.9558333333333	0.0175004115538275\\
10.9613888888889	0.0186564548556034\\
10.9669444444444	0.016963488986564\\
10.9725	0.0159672802489774\\
10.9780555555556	0.0198204348899416\\
10.9836111111111	0.0220200102323855\\
10.9891666666667	0.0223454333877305\\
10.9947222222222	0.0234407163756993\\
11.0002777777778	0.0240712167345856\\
11.0058333333333	0.0240712167345856\\
11.0113888888889	0.024663458369172\\
11.0169444444444	0.0261440624556392\\
11.0225	0.0261440624556388\\
11.0280555555556	0.0261440624556388\\
11.0336111111111	0.0261440624556388\\
11.0391666666667	0.0261440624556388\\
11.0447222222222	0.0250716170597066\\
11.0502777777778	0.0234829896634147\\
11.0558333333333	0.0258519562017623\\
11.0613888888889	0.0258519562017623\\
11.0669444444444	0.0258519562017623\\
11.0725	0.0240752312980019\\
11.0780555555556	0.0240752312980019\\
11.0836111111111	0.0240752312980019\\
11.0891666666667	0.0240752312980019\\
11.0947222222222	0.0240752312980019\\
11.1002777777778	0.0240752312980019\\
11.1058333333333	0.0293211602858479\\
11.1113888888889	0.0299134019204343\\
11.1169444444444	0.0302095227377275\\
11.1225	0.0302095227377275\\
11.1280555555556	0.0302095227377275\\
11.1336111111111	0.0302095227377275\\
11.1391666666667	0.0302095227377275\\
11.1447222222222	0.0302095227377275\\
11.1502777777778	0.0305056435550207\\
11.1558333333333	0.0287470300076698\\
11.1613888888889	0.0287470300076698\\
11.1669444444444	0.0287470300076698\\
11.1725	0.0287470300076698\\
11.1780555555556	0.0287470300076698\\
11.1836111111111	0.031551736539462\\
11.1891666666667	0.029339271642257\\
11.1947222222222	0.029339271642257\\
11.2002777777778	0.0322216145489116\\
11.2058333333333	0.0290431508249638\\
11.2113888888889	0.0290431508249638\\
11.2169444444444	0.0321439781740496\\
11.2225	0.0321439781740496\\
11.2280555555556	0.0321439781740496\\
11.2336111111111	0.0321439781740496\\
11.2391666666667	0.0321439781740496\\
11.2447222222222	0.0324400989913428\\
11.2502777777778	0.0263058075516164\\
11.2558333333333	0.0266019283689096\\
11.2613888888889	0.0267900258763785\\
11.2669444444444	0.0271899984777042\\
11.2725	0.0271899984777042\\
11.2780555555556	0.0311952136589489\\
11.2836111111111	0.0311952136589489\\
11.2891666666667	0.0311952136589489\\
11.2947222222222	0.0311952136589489\\
11.3002777777778	0.0346697982001902\\
11.3058333333333	0.0346697982001902\\
11.3113888888889	0.0346697982001902\\
11.3169444444444	0.0349659190174838\\
11.3225	0.0349659190174838\\
11.3280555555556	0.0349659190174838\\
11.3336111111111	0.0382041089965569\\
11.3391666666667	0.0382041089965569\\
11.3447222222222	0.0349467800063289\\
11.3502777777778	0.0318459526572435\\
11.3558333333333	0.029072841991778\\
11.3613888888889	0.029072841991778\\
11.3669444444444	0.0326262917992988\\
11.3725	0.0361797416068204\\
11.3780555555556	0.0361797416068204\\
11.3836111111111	0.0361797416068204\\
11.3891666666667	0.0364758624241136\\
11.3947222222222	0.0333750350750278\\
11.4002777777778	0.0333750350750278\\
11.4058333333333	0.0241893982821388\\
11.4113888888889	0.0241893982821388\\
11.4169444444444	0.0241893982821388\\
11.4225	0.0351517599787894\\
11.4280555555556	0.0351517599787894\\
11.4336111111111	0.0421606553163972\\
11.4391666666667	0.0421606553163972\\
11.4447222222222	0.0392185861546175\\
11.4502777777778	0.0392185861546175\\
11.4558333333333	0.0347776309799283\\
11.4613888888889	0.0333549519929121\\
11.4669444444444	0.0333549519929121\\
11.4725	0.0333549519929121\\
11.4780555555556	0.0333549519929121\\
11.4836111111111	0.0375994314710775\\
11.4891666666667	0.0383259118246596\\
11.4947222222222	0.0319601573414044\\
11.5002777777778	0.0319601573414044\\
11.5058333333333	0.0319601573414044\\
11.5113888888889	0.0408136966265315\\
11.5169444444444	0.0374795883461534\\
11.5225	0.0374795883461534\\
11.5280555555556	0.0374795883461534\\
11.5336111111111	0.0374795883461534\\
11.5391666666667	0.0374795883461534\\
11.5447222222222	0.039468188343771\\
11.5502777777778	0.0377757091634466\\
11.5558333333333	0.0377757091634466\\
11.5613888888889	0.04205790615083\\
11.5669444444444	0.0426310087742618\\
11.5725	0.0426310087742618\\
11.5780555555556	0.0426310087742618\\
11.5836111111111	0.0426310087742618\\
11.5891666666667	0.04071466469435\\
11.5947222222222	0.04071466469435\\
11.6002777777778	0.0387822952341598\\
11.6058333333333	0.0387822952341598\\
11.6113888888889	0.0387822952341598\\
11.6169444444444	0.0387822952341598\\
11.6225	0.036357360005159\\
11.6280555555556	0.039490311285369\\
11.6336111111111	0.0418592778237166\\
11.6391666666667	0.0418592778237166\\
11.6447222222222	0.0418592778237166\\
11.6502777777778	0.039678408792838\\
11.6558333333333	0.039678408792838\\
11.6613888888889	0.039678408792838\\
11.6669444444444	0.039678408792838\\
11.6725	0.0348079648464185\\
11.6780555555556	0.0326055883306573\\
11.6836111111111	0.0326055883306573\\
11.6891666666667	0.0389930667344428\\
11.6947222222222	0.0389930667344428\\
11.7002777777778	0.0389930667344428\\
11.7058333333333	0.0389930667344428\\
11.7113888888889	0.0389930667344428\\
11.7169444444444	0.0389930667344428\\
11.7225	0.0389930667344428\\
11.7280555555556	0.0389930667344428\\
11.7336111111111	0.037029801319599\\
11.7391666666667	0.03533042063814\\
11.7447222222222	0.0337202857569661\\
11.7502777777778	0.0306584352347767\\
11.7558333333333	0.0306584352347767\\
11.7613888888889	0.0293827693031818\\
11.7669444444444	0.0283707499006521\\
11.7725	0.0263467865507101\\
11.7780555555556	0.0253673238678274\\
11.7836111111111	0.0243878618744367\\
11.7891666666667	0.0224289394797373\\
11.7947222222222	0.0384350531775655\\
11.8002777777778	0.0389192715023276\\
11.8058333333333	0.0389192715023276\\
11.8113888888889	0.0389192715023276\\
11.8169444444444	0.0389192715023276\\
11.8225	0.0389192715023276\\
11.8280555555556	0.0389192715023276\\
11.8336111111111	0.0389192715023276\\
11.8391666666667	0.0382282418316839\\
11.8447222222222	0.0368548530176112\\
11.8502777777778	0.0407981282591218\\
11.8558333333333	0.0407981282591218\\
11.8613888888889	0.0427921087138959\\
11.8669444444444	0.0410153838101351\\
11.8725	0.0395347797236675\\
11.8780555555556	0.0377972870574581\\
11.8836111111111	0.0363940271932922\\
11.8891666666667	0.0337683502835682\\
11.8947222222222	0.0396907666294387\\
11.9002777777778	0.0396907666294387\\
11.9058333333333	0.0401842202279685\\
11.9113888888889	0.038813196447712\\
11.9169444444444	0.038813196447712\\
11.9225	0.038813196447712\\
11.9280555555556	0.0375335122503342\\
11.9336111111111	0.0375335122503342\\
11.9391666666667	0.0349783862150943\\
11.9447222222222	0.0349783862150943\\
11.9502777777778	0.0349783862150943\\
11.9558333333333	0.0349783862150943\\
11.9613888888889	0.0411077230970541\\
11.9669444444444	0.0414038439143473\\
11.9725	0.0414038439143473\\
11.9780555555556	0.0397838321460953\\
11.9836111111111	0.0420280500826657\\
11.9891666666667	0.0420280500826657\\
11.9947222222222	0.0420280500826657\\
12.0002777777778	0.0420280500826657\\
12.0058333333333	0.0376134201640654\\
12.0113888888889	0.036336482014694\\
12.0169444444444	0.0350605158646766\\
12.0225	0.0350605158646766\\
12.0280555555556	0.0350605158646766\\
12.0336111111111	0.0350605158646766\\
12.0391666666667	0.0440687959151212\\
12.0447222222222	0.0462972861926051\\
12.0502777777778	0.0462972861926051\\
12.0558333333333	0.0443649167324144\\
12.0613888888889	0.0410454249411222\\
12.0669444444444	0.0362519696805269\\
12.0725	0.0423310991619054\\
12.0780555555556	0.0390527043858619\\
12.0836111111111	0.0449751207317323\\
12.0891666666667	0.04696372072935\\
12.0947222222222	0.0454831166428823\\
12.1002777777778	0.0454831166428823\\
12.1058333333333	0.0479686596410827\\
12.1113888888889	0.0482647804583759\\
12.1169444444444	0.0466060283072736\\
12.1225	0.0454607772756505\\
12.1280555555556	0.0454607772756505\\
12.1336111111111	0.0454607772756505\\
12.1391666666667	0.0466452605448241\\
12.1447222222222	0.0484219854485849\\
12.1502777777778	0.0484219854485849\\
12.1558333333333	0.0486338605424417\\
12.1613888888889	0.0486338605424417\\
12.1669444444444	0.0486338605424417\\
12.1725	0.0486338605424417\\
12.1780555555556	0.0486338605424417\\
12.1836111111111	0.0486338605424417\\
12.1891666666667	0.0506278409972159\\
12.1947222222222	0.0495222229943221\\
12.2002777777778	0.0495222229943221\\
12.2058333333333	0.0495222229943221\\
12.2113888888889	0.0495222229943221\\
12.2169444444444	0.0495222229943221\\
12.2225	0.0495222229943221\\
12.2280555555556	0.0484219854485849\\
12.2336111111111	0.0484219854485849\\
12.2391666666667	0.0484219854485849\\
12.2447222222222	0.0484219854485849\\
12.2502777777778	0.0504948311696389\\
12.2558333333333	0.0471823402210625\\
12.2613888888889	0.0471823402210625\\
12.2669444444444	0.0446812898683003\\
12.2725	0.0446812898683003\\
12.2780555555556	0.0446812898683003\\
12.2836111111111	0.0436657478397503\\
12.2891666666667	0.0436657478397503\\
12.2947222222222	0.0436657478397503\\
12.3002777777778	0.0436657478397503\\
12.3058333333333	0.0490290952018303\\
12.3113888888889	0.0490290952018303\\
12.3169444444444	0.0490290952018303\\
12.3225	0.049395564087784\\
12.3280555555556	0.049395564087784\\
12.3336111111111	0.0511722889915448\\
12.3391666666667	0.0498321611659413\\
12.3447222222222	0.0498321611659413\\
12.3502777777778	0.0488043910322766\\
12.3558333333333	0.0488043910322766\\
12.3613888888889	0.0488043910322766\\
12.3669444444444	0.0504244028005285\\
12.3725	0.0504244028005285\\
12.3780555555556	0.0504244028005285\\
12.3836111111111	0.0504244028005285\\
12.3891666666667	0.049733373129884\\
12.3947222222222	0.049733373129884\\
12.4002777777778	0.049733373129884\\
12.4058333333333	0.049733373129884\\
12.4113888888889	0.049733373129884\\
12.4169444444444	0.049733373129884\\
12.4225	0.0486721479519002\\
12.4280555555556	0.0486721479519002\\
12.4336111111111	0.0488128745119974\\
12.4391666666667	0.0520949423518013\\
12.4447222222222	0.0541193097415373\\
12.4502777777778	0.0541193097415373\\
12.4558333333333	0.0541193097415373\\
12.4613888888889	0.0541193097415373\\
12.4669444444444	0.0536835697480927\\
12.4725	0.0555999138280037\\
12.4780555555556	0.0555999138280037\\
12.4836111111111	0.0555999138280037\\
12.4891666666667	0.0547115513761233\\
12.4947222222222	0.0547115513761233\\
12.5002777777778	0.0533151930130921\\
12.5058333333333	0.0544996762822656\\
12.5113888888889	0.0544996762822656\\
12.5169444444444	0.0544996762822656\\
12.5225	0.0562764011860264\\
12.5280555555556	0.0580531260897873\\
12.5336111111111	0.0587230040992364\\
12.5391666666667	0.0590382639276849\\
12.5447222222222	0.0590382639276849\\
12.5502777777778	0.0593343847449781\\
12.5558333333333	0.0581499014758045\\
12.5613888888889	0.0581499014758045\\
12.5669444444444	0.0581499014758045\\
12.5725	0.0581499014758045\\
12.5780555555556	0.0581499014758045\\
12.5836111111111	0.0596305055622717\\
12.5891666666667	0.0596305055622717\\
12.5947222222222	0.0584460222930977\\
12.6002777777778	0.0584460222930977\\
12.6058333333333	0.0584460222930977\\
12.6113888888889	0.0567535431127733\\
12.6169444444444	0.0554296521618153\\
12.6225	0.0544460887817308\\
12.6280555555556	0.0577034177719592\\
12.6336111111111	0.0577034177719592\\
12.6391666666667	0.0577034177719592\\
12.6447222222222	0.0577034177719592\\
12.6502777777778	0.056262045923043\\
12.6558333333333	0.056262045923043\\
12.6613888888889	0.056262045923043\\
12.6669444444444	0.0585606643927296\\
12.6725	0.0609296309310772\\
12.6780555555556	0.0609296309310772\\
12.6836111111111	0.063594718286718\\
12.6891666666667	0.063594718286718\\
12.6947222222222	0.0616783742068062\\
12.7002777777778	0.0600897468105147\\
12.7058333333333	0.0600897468105147\\
12.7113888888889	0.0600897468105147\\
12.7169444444444	0.0600897468105147\\
12.7225	0.0600897468105147\\
12.7280555555556	0.0600897468105147\\
12.7336111111111	0.0600897468105147\\
12.7391666666667	0.0600897468105147\\
12.7447222222222	0.0618664717142755\\
12.7502777777778	0.0618664717142755\\
12.7558333333333	0.0618664717142755\\
12.7613888888889	0.0618664717142755\\
12.7669444444444	0.0618664717142755\\
12.7725	0.0618664717142755\\
12.7780555555556	0.0618664717142755\\
12.7836111111111	0.0618664717142755\\
12.7891666666667	0.0618664717142755\\
12.7947222222222	0.0635643313517565\\
12.8002777777778	0.0635643313517565\\
12.8058333333333	0.0635643313517565\\
12.8113888888889	0.0635643313517565\\
12.8169444444444	0.0635643313517565\\
12.8225	0.0630509549834491\\
12.8280555555556	0.0630509549834491\\
12.8336111111111	0.0630509549834491\\
12.8391666666667	0.0630509549834491\\
12.8447222222222	0.0643750574287741\\
12.8502777777778	0.0646711782460673\\
12.8558333333333	0.0646711782460673\\
12.8613888888889	0.0649672990633605\\
12.8669444444444	0.0649672990633605\\
12.8725	0.0649672990633605\\
12.8780555555556	0.0655595406979469\\
12.8836111111111	0.0636747924843618\\
12.8891666666667	0.0636747924843618\\
12.8947222222222	0.0629837628137172\\
12.9002777777778	0.0629837628137172\\
12.9058333333333	0.0629837628137172\\
12.9113888888889	0.0640586657766798\\
12.9169444444444	0.0659918923215828\\
12.9225	0.0659918923215828\\
12.9280555555556	0.0659918923215828\\
12.9336111111111	0.0651035298697019\\
12.9391666666667	0.0643718805533304\\
12.9447222222222	0.0643718805533304\\
12.9502777777778	0.0643718805533304\\
12.9558333333333	0.0644492247756313\\
12.9613888888889	0.0660619126765131\\
12.9669444444444	0.0660619126765131\\
12.9725	0.0654696710419259\\
12.9780555555556	0.0648774294073387\\
12.9836111111111	0.0654696710419259\\
12.9891666666667	0.0654696710419259\\
12.9947222222222	0.0654696710419259\\
13.0002777777778	0.0648774294073387\\
13.0058333333333	0.0648774294073387\\
13.0113888888889	0.0648774294073387\\
13.0169444444444	0.0648774294073387\\
13.0225	0.0648774294073387\\
13.0280555555556	0.0645813085900455\\
13.0336111111111	0.0645813085900455\\
13.0391666666667	0.0645813085900455\\
13.0447222222222	0.0645813085900455\\
13.0502777777778	0.0656021819116502\\
13.0558333333333	0.0656021819116502\\
13.0613888888889	0.0654156414005681\\
13.0669444444444	0.0654156414005681\\
13.0725	0.0654156414005681\\
13.0780555555556	0.0664905443635306\\
13.0836111111111	0.0664905443635306\\
13.0891666666667	0.0664905443635306\\
13.0947222222222	0.0664905443635306\\
13.1002777777778	0.0664905443635306\\
13.1058333333333	0.0657902794191191\\
13.1113888888889	0.0657902794191191\\
13.1169444444444	0.0671508021659154\\
13.1225	0.0641396627973914\\
13.1280555555556	0.0641396627973914\\
13.1336111111111	0.0641396627973914\\
13.1391666666667	0.0645762598755488\\
13.1447222222222	0.0645762598755488\\
13.1502777777778	0.0660712433400224\\
13.1558333333333	0.0660712433400224\\
13.1613888888889	0.0669596057919028\\
13.1669444444444	0.0673007229743424\\
13.1725	0.0673007229743424\\
13.1780555555556	0.0673007229743424\\
13.1836111111111	0.0673007229743424\\
13.1891666666667	0.0673007229743424\\
13.1947222222222	0.0673007229743424\\
13.2002777777778	0.0673007229743424\\
13.2058333333333	0.0673007229743424\\
13.2113888888889	0.0673007229743424\\
13.2169444444444	0.0689837432614082\\
13.2225	0.0689837432614082\\
13.2280555555556	0.0689837432614082\\
13.2336111111111	0.0695759848959954\\
13.2391666666667	0.0718524965541448\\
13.2447222222222	0.079540803969919\\
13.2502777777778	0.0846887735270579\\
13.2558333333333	0.090273612634557\\
13.2613888888889	0.0936716849762216\\
13.2669444444444	0.0936716849762216\\
13.2725	0.0936716849762216\\
13.2780555555556	0.0971340362702832\\
13.2836111111111	0.107908310533937\\
13.2891666666667	0.108591649987875\\
13.2947222222222	0.108591649987875\\
13.3002777777778	0.108591649987875\\
13.3058333333333	0.104833177182411\\
13.3113888888889	0.109480012439754\\
13.3169444444444	0.0290587622885941\\
13.3225	0.0247565692053435\\
13.3280555555556	0.0345770345073454\\
13.3336111111111	0.030222974722969\\
13.3391666666667	0.0349609077996642\\
13.3447222222222	0.0316885488376693\\
13.3502777777778	0.0360223377872467\\
13.3558333333333	0.0360223377872467\\
13.3613888888889	0.0360223377872467\\
13.3669444444444	0.0360223377872467\\
13.3725	0.0360223377872467\\
13.3780555555556	0.0448071742699933\\
13.3836111111111	0.054580397241147\\
13.3891666666667	0.054580397241147\\
13.3947222222222	0.054580397241147\\
13.4002777777778	0.0757281453894228\\
13.4058333333333	0.0930292561501905\\
13.4113888888889	0.111124962944339\\
13.4169444444444	0.0325434600868136\\
13.4225	0.0325434600868136\\
13.4280555555556	0.0328788131416573\\
13.4336111111111	0.0283527551588189\\
13.4391666666667	0.0283527551588189\\
13.4447222222222	0.0283527551588189\\
13.4502777777778	0.0283527551588189\\
13.4558333333333	0.0380695801711545\\
13.4613888888889	0.0380695801711545\\
13.4669444444444	0.0341807773087884\\
13.4725	0.034081989272731\\
13.4780555555556	0.034081989272731\\
13.4836111111111	0.034081989272731\\
13.4891666666667	0.0389295026556373\\
13.4947222222222	0.0349081889235476\\
13.5002777777778	0.0399729578574672\\
13.5058333333333	0.0445391478268876\\
13.5113888888889	0.0305129279255054\\
13.5169444444444	0.0343624985503202\\
13.5225	0.03094866791895\\
13.5280555555556	0.03094866791895\\
13.5336111111111	0.0315600485646918\\
13.5391666666667	0.0315600485646918\\
13.5447222222222	0.0315600485646918\\
13.5502777777778	0.0318370303708304\\
13.5558333333333	0.0297641846497762\\
13.5613888888889	0.0297641846497762\\
13.5669444444444	0.0321331511881236\\
13.5725	0.037034195254535\\
13.5780555555556	0.0351732246273379\\
13.5836111111111	0.0351732246273379\\
13.5891666666667	0.0351732246273379\\
13.5947222222222	0.0351732246273379\\
13.6002777777778	0.0335361188997281\\
13.6058333333333	0.0335361188997281\\
13.6113888888889	0.0338322397170217\\
13.6169444444444	0.0338322397170217\\
13.6225	0.0338322397170217\\
13.6280555555556	0.0341283605343149\\
13.6336111111111	0.0341283605343149\\
13.6391666666667	0.0386490380599967\\
13.6447222222222	0.0386490380599967\\
13.6502777777778	0.0386490380599967\\
13.6558333333333	0.0386490380599967\\
13.6613888888889	0.0386490380599967\\
13.6669444444444	0.0386490380599967\\
13.6725	0.0386490380599967\\
13.6780555555556	0.0386490380599967\\
13.6836111111111	0.0386490380599967\\
13.6891666666667	0.0368723131562359\\
13.6947222222222	0.0350955882524751\\
13.7002777777778	0.0358392836721684\\
13.7058333333333	0.0358392836721684\\
13.7113888888889	0.0358392836721684\\
13.7169444444444	0.0379909946595026\\
13.7225	0.042433875497984\\
13.7280555555556	0.044670758221917\\
13.7336111111111	0.040713164287278\\
13.7391666666667	0.040713164287278\\
13.7447222222222	0.0432794636068618\\
13.7502777777778	0.0487168535905383\\
13.7558333333333	0.0515856068832749\\
13.7613888888889	0.0515856068832749\\
13.7669444444444	0.0515856068832749\\
13.7725	0.0515856068832749\\
13.7780555555556	0.0493803375250796\\
13.7836111111111	0.0493803375250796\\
13.7891666666667	0.0426587590085327\\
13.7947222222222	0.0394261688678796\\
13.8002777777778	0.0412817590379203\\
13.8058333333333	0.04324658144915\\
13.8113888888889	0.04324658144915\\
13.8169444444444	0.04324658144915\\
13.8225	0.04324658144915\\
13.8280555555556	0.04324658144915\\
13.8336111111111	0.0453586594077554\\
13.8391666666667	0.0453586594077554\\
13.8447222222222	0.0453586594077554\\
13.8502777777778	0.0453586594077554\\
13.8558333333333	0.0435819345039946\\
13.8613888888889	0.0435819345039946\\
13.8669444444444	0.0418052096002338\\
13.8725	0.040028484696473\\
13.8780555555556	0.040028484696473\\
13.8836111111111	0.040028484696473\\
13.8891666666667	0.0421013304175269\\
13.8947222222222	0.0421013304175269\\
13.9002777777778	0.0421013304175269\\
13.9058333333333	0.0421013304175269\\
13.9113888888889	0.0421013304175269\\
13.9169444444444	0.0406207263310593\\
13.9225	0.0423974512348201\\
13.9280555555556	0.0442584218620173\\
13.9336111111111	0.0442584218620173\\
13.9391666666667	0.0463704998206227\\
13.9447222222222	0.0463704998206227\\
13.9502777777778	0.044437061781353\\
13.9558333333333	0.044437061781353\\
13.9613888888889	0.044437061781353\\
13.9669444444444	0.0491524913498443\\
13.9725	0.0421776755492156\\
13.9780555555556	0.0421776755492156\\
13.9836111111111	0.0446308878109996\\
13.9891666666667	0.0446308878109996\\
13.9947222222222	0.0471094346555579\\
14.0002777777778	0.0555646056957093\\
14.0058333333333	0.0555646056957093\\
14.0113888888889	0.0428314105520905\\
14.0169444444444	0.0453408533513019\\
14.0225	0.0530612510118603\\
14.0280555555556	0.0558155386487054\\
14.0336111111111	0.0642300916533607\\
14.0391666666667	0.0642300916533607\\
14.0447222222222	0.0670359055690983\\
14.0502777777778	0.0727110911869758\\
14.0558333333333	0.07832275203899\\
14.0613888888889	0.0811285831985364\\
14.0669444444444	0.0867402464971505\\
14.0725	0.0895460784096122\\
14.0780555555556	0.0895460784096122\\
14.0836111111111	0.0895460784096122\\
14.0891666666667	0.0895460784096122\\
14.0947222222222	0.0895460784096122\\
14.1002777777778	0.0895460784096122\\
14.1058333333333	0.0895460784096122\\
14.1113888888889	0.0895460784096122\\
14.1169444444444	0.0895460784096122\\
14.1225	0.0895460784096122\\
14.1280555555556	0.0895460784096122\\
14.1336111111111	0.0870676252176332\\
14.1391666666667	0.0380154213309564\\
14.1447222222222	0.0474409777796086\\
14.1502777777778	0.0474409777796086\\
14.1558333333333	0.05988560330801\\
14.1613888888889	0.066088911575594\\
14.1669444444444	0.0691908463086248\\
14.1725	0.0513923367152974\\
14.1780555555556	0.054181801780957\\
14.1836111111111	0.0570134791591842\\
14.1891666666667	0.0570134791591842\\
14.1947222222222	0.0598109901519667\\
14.2002777777778	0.0598109901519667\\
14.2058333333333	0.0598109901519667\\
14.2113888888889	0.0598109901519667\\
14.2169444444444	0.0598109901519667\\
14.2225	0.048366395092591\\
14.2280555555556	0.0508758378918019\\
14.2336111111111	0.0508758378918019\\
14.2391666666667	0.0508758378918019\\
14.2447222222222	0.05660426527341\\
14.2502777777778	0.0594717688296495\\
14.2558333333333	0.0594717688296495\\
14.2613888888889	0.0594717688296495\\
14.2669444444444	0.0652102474144792\\
14.2725	0.07095036464084\\
14.2780555555556	0.0766905758325669\\
14.2836111111111	0.079560683774499\\
14.2891666666667	0.0858931432359678\\
14.2947222222222	0.0893554945300294\\
14.3002777777778	0.096280198064753\\
14.3058333333333	0.0997425500529238\\
14.3113888888889	0.0997425500529238\\
14.3169444444444	0.0997425500529238\\
14.3225	0.0997425500529238\\
14.3280555555556	0.0997425500529238\\
14.3336111111111	0.0997425500529238\\
14.3391666666667	0.0997425500529238\\
14.3447222222222	0.0997425500529238\\
14.3502777777778	0.0997425500529238\\
14.3558333333333	0.0997425500529238\\
14.3613888888889	0.0997425500529238\\
14.3669444444444	0.0997425500529238\\
14.3725	0.0997425500529238\\
14.3780555555556	0.0343909472504333\\
14.3836111111111	0.0343909472504333\\
14.3891666666667	0.0311336182602053\\
14.3947222222222	0.0311336182602053\\
14.4002777777778	0.0347659333340068\\
14.4058333333333	0.0386155039588211\\
14.4113888888889	0.0386155039588211\\
14.4169444444444	0.0423256669018516\\
14.4225	0.0423256669018516\\
14.4280555555556	0.0389485076623252\\
14.4336111111111	0.0352545815939402\\
14.4391666666667	0.0315446301452765\\
14.4447222222222	0.0353942007700912\\
14.4502777777778	0.0353942007700912\\
14.4558333333333	0.0330879637245189\\
14.4613888888889	0.0332326307567051\\
14.4669444444444	0.0332326307567051\\
14.4725	0.0332326307567051\\
14.4780555555556	0.0335287515739983\\
14.4836111111111	0.0335287515739983\\
14.4891666666667	0.0335287515739983\\
14.4947222222222	0.0335287515739983\\
14.5002777777778	0.0335287515739983\\
14.5058333333333	0.0345567332020293\\
14.5113888888889	0.0351489748366157\\
14.5169444444444	0.0351489748366157\\
14.5225	0.0351489748366157\\
14.5280555555556	0.0351489748366157\\
14.5336111111111	0.0351489748366157\\
14.5391666666667	0.0354450956539089\\
14.5447222222222	0.0360373372884952\\
14.5502777777778	0.0371026725487995\\
14.5558333333333	0.0371026725487995\\
14.5613888888889	0.0371026725487995\\
14.5669444444444	0.0376949141833859\\
14.5725	0.037991035000679\\
14.5780555555556	0.0442266661231993\\
14.5836111111111	0.0421538204021453\\
14.5891666666667	0.0421538204021453\\
14.5947222222222	0.0445620191780438\\
14.6002777777778	0.0445620191780438\\
14.6058333333333	0.0500529573236207\\
14.6113888888889	0.0500529573236207\\
14.6169444444444	0.0500529573236207\\
14.6225	0.0422254004968504\\
14.6280555555556	0.0422254004968504\\
14.6336111111111	0.0427188540953803\\
14.6391666666667	0.0427188540953803\\
14.6447222222222	0.0427188540953803\\
14.6502777777778	0.0427188540953803\\
14.6558333333333	0.0406460083743263\\
14.6613888888889	0.0406460083743263\\
14.6669444444444	0.0406460083743263\\
14.6725	0.0406460083743263\\
14.6780555555556	0.0430149749126735\\
14.6836111111111	0.0396623156702373\\
14.6891666666667	0.0396623156702373\\
14.6947222222222	0.0435118862950521\\
14.7002777777778	0.0435118862950521\\
14.7058333333333	0.0435118862950521\\
14.7113888888889	0.0435118862950521\\
14.7169444444444	0.0435118862950521\\
14.7225	0.043910138965377\\
14.7280555555556	0.046772559102254\\
14.7336111111111	0.046772559102254\\
14.7391666666667	0.0568185443962179\\
14.7447222222222	0.0409551998390425\\
14.7502777777778	0.0444006263367394\\
14.7558333333333	0.0481107892797698\\
14.7613888888889	0.041102677700784\\
14.7669444444444	0.0381414695278496\\
14.7725	0.0381414695278496\\
14.7780555555556	0.0381414695278496\\
14.7836111111111	0.0388113475372987\\
14.7891666666667	0.0388113475372987\\
14.7947222222222	0.0388113475372987\\
14.8002777777778	0.0388113475372987\\
14.8058333333333	0.0391074683545919\\
14.8113888888889	0.0394812255467474\\
14.8169444444444	0.0405536709426792\\
14.8225	0.0408497917599724\\
14.8280555555556	0.058972682805587\\
14.8336111111111	0.0621061704106449\\
14.8391666666667	0.0621061704106449\\
14.8447222222222	0.044562156401867\\
14.8502777777778	0.0475625968123528\\
14.8558333333333	0.0475625968123528\\
14.8613888888889	0.0475625968123528\\
14.8669444444444	0.0475625968123528\\
14.8725	0.0475625968123528\\
14.8780555555556	0.0475625968123528\\
14.8836111111111	0.0475625968123528\\
14.8891666666667	0.0475625968123528\\
14.8947222222222	0.0475625968123528\\
14.9002777777778	0.0426273309544213\\
14.9058333333333	0.0426273309544213\\
14.9113888888889	0.0429234517717149\\
14.9169444444444	0.0412863460441051\\
14.9225	0.0437341778487326\\
14.9280555555556	0.0437341778487326\\
14.9336111111111	0.0437341778487326\\
14.9391666666667	0.0437341778487326\\
14.9447222222222	0.0437341778487326\\
14.9502777777778	0.0437341778487326\\
14.9558333333333	0.0512300948603345\\
14.9613888888889	0.0453061017738314\\
14.9669444444444	0.0433412793626017\\
14.9725	0.0411895683752674\\
14.9780555555556	0.0411895683752674\\
14.9836111111111	0.041485689192561\\
14.9891666666667	0.0417818100098542\\
14.9947222222222	0.042374051644441\\
15.0002777777778	0.0426510334505793\\
15.0058333333333	0.0426510334505793\\
15.0113888888889	0.0408551695356635\\
15.0169444444444	0.0408551695356635\\
15.0225	0.0429280152567179\\
15.0280555555556	0.0429280152567179\\
15.0336111111111	0.0429280152567179\\
15.0391666666667	0.0454534834362077\\
15.0447222222222	0.0432241360740111\\
15.0502777777778	0.0432241360740111\\
15.0558333333333	0.0432241360740111\\
15.0613888888889	0.0458892234296523\\
15.0669444444444	0.0520525986344663\\
15.0725	0.0520525986344663\\
15.0780555555556	0.0520525986344663\\
15.0836111111111	0.0523487194517603\\
15.0891666666667	0.0458340614713044\\
15.0947222222222	0.0458340614713044\\
15.1002777777778	0.0505719945479987\\
15.1058333333333	0.0505719945479987\\
15.1113888888889	0.0505719945479987\\
15.1169444444444	0.0505719945479987\\
15.1225	0.0460206019823861\\
15.1280555555556	0.0460206019823861\\
15.1336111111111	0.0445399978959189\\
15.1391666666667	0.0448361187132121\\
15.1447222222222	0.0463955880659595\\
15.1502777777778	0.0483604104771892\\
15.1558333333333	0.0470354509471519\\
15.1613888888889	0.0476276925817387\\
15.1669444444444	0.0492963941756752\\
15.1725	0.0494937269569119\\
15.1780555555556	0.0521980465501036\\
15.1836111111111	0.0521980465501036\\
15.1891666666667	0.0507947866859378\\
15.1947222222222	0.0530649651882285\\
15.2002777777778	0.0556348516156655\\
15.2058333333333	0.0556348516156655\\
15.2113888888889	0.0513151482716764\\
15.2169444444444	0.0528349845956945\\
15.2225	0.0548484555665566\\
15.2280555555556	0.0548484555665566\\
15.2336111111111	0.0593296782718758\\
15.2391666666667	0.0593296782718758\\
15.2447222222222	0.0476894835165847\\
15.2502777777778	0.0502465475624012\\
15.2558333333333	0.0464737562277364\\
15.2613888888889	0.0471436342371855\\
15.2669444444444	0.0475692802738489\\
15.2725	0.0518554479768202\\
15.2780555555556	0.0679901975492113\\
15.2836111111111	0.0541251154568561\\
15.2891666666667	0.0541251154568561\\
15.2947222222222	0.0541251154568561\\
15.3002777777778	0.0578794851534667\\
15.3058333333333	0.0473143366591061\\
15.3113888888889	0.0473143366591061\\
15.3169444444444	0.0510558839740966\\
15.3225	0.043864738635617\\
15.3280555555556	0.0474181884431386\\
15.3336111111111	0.0509716382506602\\
15.3391666666667	0.0513083787136804\\
15.3447222222222	0.0452454861069474\\
15.3502777777778	0.0452454861069474\\
15.3558333333333	0.0572478890189066\\
15.3613888888889	0.0572478890189066\\
15.3669444444444	0.0572478890189066\\
15.3725	0.0504171459168814\\
15.3780555555556	0.0468636961093602\\
15.3836111111111	0.0471598169266534\\
15.3891666666667	0.0544071928025599\\
15.3947222222222	0.070039607002322\\
15.4002777777778	0.0602676200316367\\
15.4058333333333	0.0602676200316367\\
15.4113888888889	0.0602676200316367\\
15.4169444444444	0.0638984140614593\\
15.4225	0.0675352830809744\\
15.4280555555556	0.06457407490804\\
15.4336111111111	0.0583555377448772\\
15.4391666666667	0.055690450389236\\
15.4447222222222	0.055690450389236\\
15.4502777777778	0.0533214838508884\\
15.4558333333333	0.0543494654789198\\
15.4613888888889	0.054645586296213\\
15.4669444444444	0.054645586296213\\
15.4725	0.0567972972835472\\
15.4780555555556	0.0567972972835472\\
15.4836111111111	0.0567972972835472\\
15.4891666666667	0.0567972972835472\\
15.4947222222222	0.0550205723797864\\
15.5002777777778	0.0550205723797864\\
15.5058333333333	0.0550205723797864\\
15.5113888888889	0.0550205723797864\\
15.5169444444444	0.0570934181008408\\
15.5225	0.0593543613293646\\
15.5280555555556	0.0593543613293646\\
15.5336111111111	0.0577895115194606\\
15.5391666666667	0.0565261629840067\\
15.5447222222222	0.0580856323367537\\
15.5502777777778	0.0562300421667131\\
15.5558333333333	0.0583817531540474\\
15.5613888888889	0.0614604422249396\\
15.5669444444444	0.0614604422249396\\
15.5725	0.0691680107109068\\
15.5780555555556	0.0691680107109068\\
15.5836111111111	0.0673078665951516\\
15.5891666666667	0.0631621751530436\\
15.5947222222222	0.0631621751530436\\
15.6002777777778	0.0631621751530436\\
15.6058333333333	0.0654323536553343\\
15.6113888888889	0.0699813811871304\\
15.6169444444444	0.0624117707322071\\
15.6225	0.062110269457758\\
15.6280555555556	0.0573723363810631\\
15.6336111111111	0.0594451821021171\\
15.6391666666667	0.0582104294947824\\
15.6447222222222	0.05508795392773\\
15.6502777777778	0.0568646788314904\\
15.6558333333333	0.0578991424272762\\
15.6613888888889	0.0568851075168446\\
15.6669444444444	0.0569572842966019\\
15.6725	0.0595381668658775\\
15.6780555555556	0.0613060858192541\\
15.6836111111111	0.0613060858192541\\
15.6891666666667	0.0613060858192541\\
15.6947222222222	0.0613060858192541\\
15.7002777777778	0.0597880932496532\\
15.7058333333333	0.0591958516150668\\
15.7113888888889	0.0591958516150668\\
15.7169444444444	0.0591958516150668\\
15.7225	0.0591958516150668\\
15.7280555555556	0.0591958516150668\\
15.7336111111111	0.0609725765188264\\
15.7391666666667	0.0609725765188264\\
15.7447222222222	0.062157059788\\
15.7502777777778	0.0627493014225863\\
15.7558333333333	0.0625274513886531\\
15.7613888888889	0.0654886595615871\\
15.7669444444444	0.0640080554751199\\
15.7725	0.0614172389028407\\
15.7780555555556	0.058995872387587\\
15.7836111111111	0.0612105987917702\\
15.7891666666667	0.0606810707615872\\
15.7947222222222	0.0608622141756818\\
15.8002777777778	0.0613214525496589\\
15.8058333333333	0.0598961804597198\\
15.8113888888889	0.0603198695992478\\
15.8169444444444	0.0607466891321549\\
15.8225	0.0607466891321549\\
15.8280555555556	0.0607466891321558\\
15.8336111111111	0.0613389307667422\\
15.8391666666667	0.061289466188782\\
15.8447222222222	0.061289466188782\\
15.8502777777778	0.061289466188782\\
15.8558333333333	0.061289466188782\\
15.8613888888889	0.0617901812936087\\
15.8669444444444	0.0635041925929652\\
15.8725	0.0635041925929652\\
15.8780555555556	0.0646886758621388\\
15.8836111111111	0.0664654007658992\\
15.8891666666667	0.0664654007658992\\
15.8947222222222	0.0664654007658992\\
15.9002777777778	0.0655770383140188\\
15.9058333333333	0.0647365076368971\\
15.9113888888889	0.0647365076368971\\
15.9169444444444	0.0659009198411478\\
15.9225	0.0656047990238538\\
15.9280555555556	0.0658287514580001\\
15.9336111111111	0.0623112622310733\\
15.9391666666667	0.0631996246829537\\
15.9447222222222	0.0640879871348337\\
15.9502777777778	0.0651949775644215\\
15.9558333333333	0.0657872191990078\\
15.9613888888889	0.0657872191990078\\
15.9669444444444	0.0657872191990078\\
15.9725	0.0664421744379984\\
15.9780555555556	0.0679277885589267\\
15.9836111111111	0.0695164836690324\\
15.9891666666667	0.0695164836690324\\
15.9947222222222	0.0658513744232272\\
};
\addlegendentry{Cts Stoch Ctrl};

\addplot [color=dscr_plot_color,solid,line width=1.5pt]
  table[row sep=crcr]{%
9.50027777777778	0.013029315960912\\
9.50583333333333	0.0197210736578596\\
9.51138888888889	0.0276189596341472\\
9.51694444444444	0.00598654198937404\\
9.5225	-0.00551415581790399\\
9.52805555555556	0.0104108086787287\\
9.53361111111111	-0.0122263963391346\\
9.53916666666667	-0.0127025541271545\\
9.54472222222222	-0.0140623883911364\\
9.55027777777778	-0.0137662675738428\\
9.55583333333333	-0.0137662675738424\\
9.56138888888889	-0.0139967359346436\\
9.56694444444444	-0.0139620600248174\\
9.5725	-0.0115930934864701\\
9.57805555555555	-0.0115930934864701\\
9.58361111111111	-0.014552879132049\\
9.58916666666667	-0.00377947490332764\\
9.59472222222222	-0.00526007898979525\\
9.60027777777778	0.00171714451652137\\
9.60583333333333	-0.0112031427459798\\
9.61138888888889	-0.0114992635632729\\
9.61694444444444	-0.0202815616953708\\
9.6225	-0.0211480920173618\\
9.62805555555556	-0.0154140461277317\\
9.63361111111111	-0.0156985143372483\\
9.63916666666667	-0.0176072621445683\\
9.64472222222222	-0.0175476612771446\\
9.65027777777778	-0.0163090837386354\\
9.65583333333333	-0.0150080467053805\\
9.66138888888889	-0.0204391938990349\\
9.66694444444444	-0.0201430730817413\\
9.6725	-0.0231468726930269\\
9.67805555555555	-0.0229140906479468\\
9.68361111111111	-0.021137365744186\\
9.68916666666667	-0.0214334865614801\\
9.69472222222222	-0.0225082289158951\\
9.70027777777778	-0.0222121080986019\\
9.70583333333333	-0.0233087151505697\\
9.71138888888889	-0.0218281110641025\\
9.71694444444444	-0.0230125943332765\\
9.7225	-0.0224203526986901\\
9.72805555555555	-0.02373030875174\\
9.73361111111111	-0.0240264295690332\\
9.73916666666667	-0.0240264295690332\\
9.74472222222222	-0.0240264295690332\\
9.75027777777778	-0.0228419462998596\\
9.75583333333333	-0.021657463030686\\
9.76138888888889	-0.0233875375858746\\
9.76694444444444	-0.022499175133995\\
9.7725	-0.0248187050778202\\
9.77805555555556	-0.0248187050778206\\
9.78361111111111	-0.0248187050778206\\
9.78916666666667	-0.0236342218086474\\
9.79472222222222	-0.0242264634432342\\
9.80027777777778	-0.0214623811281357\\
9.80583333333333	-0.0226468643973093\\
9.81138888888889	-0.023962821067828\\
9.81694444444444	-0.023962821067828\\
9.8225	-0.0229389766452898\\
9.82805555555555	-0.0246783966800908\\
9.83361111111111	-0.0252706383146778\\
9.83916666666667	-0.0249745174973846\\
9.84472222222222	-0.0243822758627983\\
9.85027777777778	-0.0250261875790675\\
9.85583333333333	-0.0245370433360909\\
9.86138888888889	-0.0245356208087348\\
9.86694444444444	-0.0257201040779088\\
9.8725	-0.0243863633719369\\
9.87805555555556	-0.0246469213825024\\
9.88361111111111	-0.0234624381133284\\
9.88916666666667	-0.0234624381133284\\
9.89472222222222	-0.0233634431538714\\
9.90027777777778	-0.0233634431538714\\
9.90583333333333	-0.0239556847884582\\
9.91138888888889	-0.0233634431538714\\
9.91694444444444	-0.0227712015192842\\
9.9225	-0.023363443153871\\
9.92805555555555	-0.0234949165552158\\
9.93361111111111	-0.0233918191636048\\
9.93916666666667	-0.0221261985144385\\
9.94472222222222	-0.0224223193317321\\
9.95027777777778	-0.0233106817836125\\
9.95583333333333	-0.0213539199091254\\
9.96138888888889	-0.0210577990918318\\
9.96694444444444	-0.0267747151004731\\
9.9725	-0.0293971701524183\\
9.97805555555555	-0.0220423954738222\\
9.98361111111111	-0.0220423954738222\\
9.98916666666667	-0.0220423954738222\\
9.99472222222222	-0.0241152411948766\\
10.0002777777778	-0.0241152411948766\\
10.0058333333333	-0.0241152411948766\\
10.0113888888889	-0.0241152411948766\\
10.0169444444444	-0.0241152411948766\\
10.0225	-0.0293642785261659\\
10.0280555555556	-0.0293642785261659\\
10.0336111111111	-0.0293642785261659\\
10.0391666666667	-0.0300778928289549\\
10.0447222222222	-0.0274786974006018\\
10.0502777777778	-0.0284004760490981\\
10.0558333333333	-0.0276773539541034\\
10.0613888888889	-0.0276773539541034\\
10.0669444444444	-0.0282695955886897\\
10.0725	-0.0282695955886897\\
10.0780555555556	-0.0282695955886902\\
10.0836111111111	-0.0279734747713965\\
10.0891666666667	-0.0279734747713965\\
10.0947222222222	-0.0279734747713965\\
10.1002777777778	-0.0273812331368098\\
10.1058333333333	-0.0252728246090272\\
10.1113888888889	-0.023628270016356\\
10.1169444444444	-0.0248938906655218\\
10.1225	-0.0248938906655218\\
10.1280555555556	-0.0210017422027043\\
10.1336111111111	-0.0210017422027043\\
10.1391666666667	-0.0210017422027043\\
10.1447222222222	-0.0222929703204454\\
10.1502777777778	-0.0222929703204454\\
10.1558333333333	-0.0222929703204454\\
10.1613888888889	-0.0222929703204454\\
10.1669444444444	-0.0269739022418062\\
10.1725	-0.0268749072823484\\
10.1780555555556	-0.0268749072823484\\
10.1836111111111	-0.026874907282348\\
10.1891666666667	-0.0265787864650548\\
10.1947222222222	-0.0265787864650548\\
10.2002777777778	-0.0262826656477616\\
10.2058333333333	-0.0262826656477616\\
10.2113888888889	-0.0262826656477616\\
10.2169444444444	-0.0262826656477616\\
10.2225	-0.0262826656477616\\
10.2280555555556	-0.0259865448304684\\
10.2336111111111	-0.0268749072823488\\
10.2391666666667	-0.0268749072823493\\
10.2447222222222	-0.0256904240131752\\
10.2502777777778	-0.0265787864650556\\
10.2558333333333	-0.0253943031958821\\
10.2613888888889	-0.0248020615612953\\
10.2669444444444	-0.0259865448304693\\
10.2725	-0.0245059407440017\\
10.2780555555556	-0.024183654435323\\
10.2836111111111	-0.0262565001563774\\
10.2891666666667	-0.0298099499638977\\
10.2947222222222	-0.0298099499638977\\
10.3002777777778	-0.0252018854078178\\
10.3058333333333	-0.0252018854078178\\
10.3113888888889	-0.0240676496929706\\
10.3169444444444	-0.0255482537794378\\
10.3225	-0.0234754080583834\\
10.3280555555556	-0.0231778647137341\\
10.3336111111111	-0.0231778647137341\\
10.3391666666667	-0.0245423849536344\\
10.3447222222222	-0.0245423849536344\\
10.3502777777778	-0.0257552442325418\\
10.3558333333333	-0.0237854959030992\\
10.3613888888889	-0.0237854959030992\\
10.3669444444444	-0.0237854959030992\\
10.3725	-0.0237854959030992\\
10.3780555555556	-0.0237854959030992\\
10.3836111111111	-0.0237854959030992\\
10.3891666666667	-0.024377737537686\\
10.3947222222222	-0.024377737537686\\
10.4002777777778	-0.0234893750858052\\
10.4058333333333	-0.0240816167203924\\
10.4113888888889	-0.023193254268512\\
10.4169444444444	-0.023193254268512\\
10.4225	-0.023193254268512\\
10.4280555555556	-0.023193254268512\\
10.4336111111111	-0.023193254268512\\
10.4391666666667	-0.023193254268512\\
10.4447222222222	-0.0237854959030988\\
10.4502777777778	-0.0237854959030988\\
10.4558333333333	-0.0237854959030988\\
10.4613888888889	-0.0237854959030988\\
10.4669444444444	-0.0237499330963711\\
10.4725	-0.0234538122790771\\
10.4780555555556	-0.0230545940701729\\
10.4836111111111	-0.0230545940701729\\
10.4891666666667	-0.0236876886260399\\
10.4947222222222	-0.0233915678087463\\
10.5002777777778	-0.0233915678087463\\
10.5058333333333	-0.0230598841847246\\
10.5113888888889	-0.0235490284277017\\
10.5169444444444	-0.0235490284277012\\
10.5225	-0.0238451492449944\\
10.5280555555556	-0.0238451492449948\\
10.5336111111111	-0.0229085410393746\\
10.5391666666667	-0.0222281846268966\\
10.5447222222222	-0.0215347676938532\\
10.5502777777778	-0.0201374261702265\\
10.5558333333333	-0.0268930226296646\\
10.5613888888889	-0.0268930226296646\\
10.5669444444444	-0.0248201769086098\\
10.5725	-0.0248201769086098\\
10.5780555555556	-0.023554556259444\\
10.5836111111111	-0.0195153593612217\\
10.5891666666667	-0.0162860073409905\\
10.5947222222222	-0.0162860073409905\\
10.6002777777778	-0.0294225177913962\\
10.6058333333333	-0.0294225177913962\\
10.6113888888889	-0.0294225177913962\\
10.6169444444444	-0.0294225177913962\\
10.6225	-0.0294225177913962\\
10.6280555555556	-0.027941913704929\\
10.6336111111111	-0.027941913704929\\
10.6391666666667	-0.0291263969741034\\
10.6447222222222	-0.0291263969741034\\
10.6502777777778	-0.0279419137049298\\
10.6558333333333	-0.0279419137049298\\
10.6613888888889	-0.0264613096184622\\
10.6669444444444	-0.024643731793422\\
10.6725	-0.0261243358798896\\
10.6780555555556	-0.0279010607836504\\
10.6836111111111	-0.024015927352108\\
10.6891666666667	-0.021943081631054\\
10.6947222222222	-0.021943081631054\\
10.7002777777778	-0.0290499812460959\\
10.7058333333333	-0.0272732563423355\\
10.7113888888889	-0.0252004106212815\\
10.7169444444444	-0.0252004106212815\\
10.7225	-0.0252004106212815\\
10.7280555555556	-0.0287538604288027\\
10.7336111111111	-0.0287538604288027\\
10.7391666666667	-0.0272732563423355\\
10.7447222222222	-0.0272732563423355\\
10.7502777777778	-0.0272732563423355\\
10.7558333333333	-0.0272732563423355\\
10.7613888888889	-0.0275693771596287\\
10.7669444444444	-0.0308267061498567\\
10.7725	-0.0275693771596287\\
10.7780555555556	-0.0275693771596287\\
10.7836111111111	-0.0275693771596287\\
10.7891666666667	-0.0275693771596287\\
10.7947222222222	-0.0254965314385747\\
10.8002777777778	-0.0213508399964659\\
10.8058333333333	-0.0169746975984287\\
10.8113888888889	-0.0169746975984287\\
10.8169444444444	-0.027723161798883\\
10.8225	-0.027723161798883\\
10.8280555555556	-0.0274270409815898\\
10.8336111111111	-0.0274270409815898\\
10.8391666666667	-0.0271309201642967\\
10.8447222222222	-0.0243668378491973\\
10.8502777777778	-0.0205804221294506\\
10.8558333333333	-0.027216452778108\\
10.8613888888889	-0.0257358486916408\\
10.8669444444444	-0.0216807777295982\\
10.8725	-0.0194780634879293\\
10.8780555555556	-0.0194780634879293\\
10.8836111111111	-0.0194780634879293\\
10.8891666666667	-0.0194780634879293\\
10.8947222222222	-0.0194780634879293\\
10.9002777777778	-0.0194780634879293\\
10.9058333333333	-0.0194780634879293\\
10.9113888888889	-0.027327126873126\\
10.9169444444444	-0.027327126873126\\
10.9225	-0.027327126873126\\
10.9280555555556	-0.027327126873126\\
10.9336111111111	-0.0270310060558328\\
10.9391666666667	-0.0211285576273699\\
10.9447222222222	-0.0265398697186449\\
10.9502777777778	-0.025847210597243\\
10.9558333333333	-0.0241037390759532\\
10.9613888888889	-0.0234125024819095\\
10.9669444444444	-0.0222280192127367\\
10.9725	-0.0213396567608571\\
10.9780555555556	-0.0209274520969969\\
10.9836111111111	-0.0206313312797039\\
10.9891666666667	-0.0197443913551796\\
10.9947222222222	-0.019924428325906\\
11.0002777777778	-0.0189420229290674\\
11.0058333333333	-0.0189420229290674\\
11.0113888888889	-0.018349781294481\\
11.0169444444444	-0.0162002212841406\\
11.0225	-0.0150992799148871\\
11.0280555555556	-0.0150992799148871\\
11.0336111111111	-0.0193686774643559\\
11.0391666666667	-0.0180851992357246\\
11.0447222222222	-0.0185883421952905\\
11.0502777777778	-0.0221417920028121\\
11.0558333333333	-0.0174038589261165\\
11.0613888888889	-0.0174038589261165\\
11.0669444444444	-0.0174038589261165\\
11.0725	-0.0215086974469455\\
11.0780555555556	-0.0215086974469455\\
11.0836111111111	-0.0215086974469455\\
11.0891666666667	-0.0215086974469455\\
11.0947222222222	-0.0215086974469455\\
11.1002777777778	-0.0215086974469455\\
11.1058333333333	-0.0173274431981094\\
11.1113888888889	-0.00844431680256637\\
11.1169444444444	-0.00386950983447912\\
11.1225	-0.00386950983447912\\
11.1280555555556	-0.00386950983447912\\
11.1336111111111	-0.00386950983447912\\
11.1391666666667	-0.00386950983447912\\
11.1447222222222	-0.00166211159001829\\
11.1502777777778	-0.0198853034953411\\
11.1558333333333	-0.0185515623649962\\
11.1613888888889	-0.0185515623649962\\
11.1669444444444	-0.0164787166439422\\
11.1725	-0.0164787166439422\\
11.1780555555556	-0.0164787166439422\\
11.1836111111111	-0.0141097501055937\\
11.1891666666667	-0.0141097501055937\\
11.1947222222222	-0.0141097501055937\\
11.2002777777778	-0.0116999306459672\\
11.2058333333333	-0.0202874343474776\\
11.2113888888889	-0.0199913135301844\\
11.2169444444444	-0.0173262261745436\\
11.2225	-0.0173262261745436\\
11.2280555555556	-0.0173262261745436\\
11.2336111111111	-0.0173262261745436\\
11.2391666666667	-0.0109678411306445\\
11.2447222222222	-0.00754656163421243\\
11.2502777777778	-0.0215941085676191\\
11.2558333333333	-0.018927598684622\\
11.2613888888889	-0.0194842775124819\\
11.2669444444444	-0.0194842775124819\\
11.2725	-0.0194842775124819\\
11.2780555555556	-0.0191867341678326\\
11.2836111111111	-0.0191867341678326\\
11.2891666666667	-0.0191867341678326\\
11.2947222222222	-0.0165216468121914\\
11.3002777777778	-0.010453774690168\\
11.3058333333333	-0.010453774690168\\
11.3113888888889	-0.0196335200262648\\
11.3169444444444	-0.0172645534879172\\
11.3225	-0.0172645534879172\\
11.3280555555556	-0.0172645534879172\\
11.3336111111111	-0.014599466132276\\
11.3391666666667	-0.014599466132276\\
11.3447222222222	-0.014599466132276\\
11.3502777777778	-0.0204187850865342\\
11.3558333333333	-0.020122664269241\\
11.3613888888889	-0.0198265434519478\\
11.3669444444444	-0.0168653352790138\\
11.3725	-0.0138685642993518\\
11.3780555555556	-0.0138685642993518\\
11.3836111111111	-0.0138685642993518\\
11.3891666666667	-0.0105703823878439\\
11.3947222222222	-0.0132354697434851\\
11.4002777777778	-0.0132354697434851\\
11.4058333333333	-0.0159005570991263\\
11.4113888888889	-0.0159005570991263\\
11.4169444444444	-0.0159005570991263\\
11.4225	-0.00938589911866951\\
11.4280555555556	-0.0182695236374731\\
11.4336111111111	-0.0123471072916039\\
11.4391666666667	-0.0123471072916039\\
11.4447222222222	-0.0176772820028859\\
11.4502777777778	-0.0176772820028859\\
11.4558333333333	-0.0176772820028859\\
11.4613888888889	-0.017265077339025\\
11.4669444444444	-0.017265077339025\\
11.4725	-0.017265077339025\\
11.4780555555556	-0.017265077339025\\
11.4836111111111	-0.0154883524352646\\
11.4891666666667	-0.0157757178509607\\
11.4947222222222	-0.0169602011201347\\
11.5002777777778	-0.0169602011201347\\
11.5058333333333	-0.0169602011201347\\
11.5113888888889	-0.0123537414447838\\
11.5169444444444	-0.0165402858081716\\
11.5225	-0.0165402858081716\\
11.5280555555556	-0.0165402858081716\\
11.5336111111111	-0.0165402858081716\\
11.5391666666667	-0.0165402858081716\\
11.5447222222222	-0.0150596817217044\\
11.5502777777778	-0.0162441649908784\\
11.5558333333333	-0.0162441649908784\\
11.5613888888889	-0.014895034305756\\
11.5669444444444	-0.00757253362034365\\
11.5725	-0.00561495224876022\\
11.5780555555556	-0.00561495224876022\\
11.5836111111111	-0.00561495224876022\\
11.5891666666667	-0.0162753016713238\\
11.5947222222222	-0.0144985767675634\\
11.6002777777778	-0.0159791808540306\\
11.6058333333333	-0.0159791808540306\\
11.6113888888889	-0.0159791808540306\\
11.6169444444444	-0.0159791808540306\\
11.6225	-0.015979180854031\\
11.6280555555556	-0.0147946975848574\\
11.6336111111111	-0.0112272654025276\\
11.6391666666667	-0.0089904691752691\\
11.6447222222222	-0.0107671940790291\\
11.6502777777778	-0.0125439189827899\\
11.6558333333333	-0.0125439189827899\\
11.6613888888889	-0.00998484208636497\\
11.6669444444444	-0.00998484208636497\\
11.6725	-0.0166449558429752\\
11.6780555555556	-0.0166449558429752\\
11.6836111111111	-0.0166449558429752\\
11.6891666666667	-0.0163474124983259\\
11.6947222222222	-0.0163474124983259\\
11.7002777777778	-0.0163474124983259\\
11.7058333333333	-0.0163474124983259\\
11.7113888888889	-0.0163474124983259\\
11.7169444444444	-0.0163474124983259\\
11.7225	-0.0163474124983259\\
11.7280555555556	-0.0163474124983259\\
11.7336111111111	-0.0163474124983259\\
11.7391666666667	-0.0163474124983259\\
11.7447222222222	-0.0160512916810327\\
11.7502777777778	-0.0160512916810327\\
11.7558333333333	-0.0160512916810327\\
11.7613888888889	-0.0160512916810327\\
11.7669444444444	-0.0160512916810327\\
11.7725	-0.0154590500464464\\
11.7780555555556	-0.0154590500464464\\
11.7836111111111	-0.0151629292291532\\
11.7891666666667	-0.0154590500464472\\
11.7947222222222	-0.0157551708637396\\
11.8002777777778	-0.0142745667772728\\
11.8058333333333	-0.0142745667772728\\
11.8113888888889	-0.0142745667772728\\
11.8169444444444	-0.0142745667772728\\
11.8225	-0.0142745667772728\\
11.8280555555556	-0.0142745667772728\\
11.8336111111111	-0.0139784459599796\\
11.8391666666667	-0.0139784459599796\\
11.8447222222222	-0.0136823251426864\\
11.8502777777778	-0.0127939626908064\\
11.8558333333333	-0.0130900835081\\
11.8613888888889	-0.0106180195781423\\
11.8669444444444	-0.0118380656540444\\
11.8725	-0.0113619078660249\\
11.8780555555556	-0.0113619078660249\\
11.8836111111111	-0.0122502703179053\\
11.8891666666667	-0.0119541495006117\\
11.8947222222222	-0.0107696662314377\\
11.9002777777778	-0.0116580286833181\\
11.9058333333333	-0.0116580286833181\\
11.9113888888889	-0.0113619078660244\\
11.9169444444444	-0.0113619078660244\\
11.9225	-0.0110657870487313\\
11.9280555555556	-0.0110657870487313\\
11.9336111111111	-0.0110657870487313\\
11.9391666666667	-0.0110657870487313\\
11.9447222222222	-0.0110657870487313\\
11.9502777777778	-0.0110657870487313\\
11.9558333333333	-0.0110657870487313\\
11.9613888888889	-0.00988130377955725\\
11.9669444444444	-0.00958518296226322\\
11.9725	-0.00958518296226322\\
11.9780555555556	-0.0107696662314372\\
11.9836111111111	-0.0107696662314372\\
11.9891666666667	-0.0107696662314372\\
11.9947222222222	-0.0107696662314372\\
12.0002777777778	-0.0107696662314372\\
12.0058333333333	-0.0107609108298401\\
12.0113888888889	-0.0101686691952537\\
12.0169444444444	-0.0101686691952537\\
12.0225	-0.0101686691952537\\
12.0280555555556	-0.0101686691952537\\
12.0336111111111	-0.00987254837796054\\
12.0391666666667	-0.00839194429149334\\
12.0447222222222	-0.00661521938773254\\
12.0502777777778	-0.00661521938773254\\
12.0558333333333	-0.0101686691952537\\
12.0613888888889	-0.00987254837796054\\
12.0669444444444	-0.00987254837796096\\
12.0725	-0.00868806510878737\\
12.0780555555556	-0.00928030674337417\\
12.0836111111111	-0.00809582347420016\\
12.0891666666667	-0.00657965658100486\\
12.0947222222222	-0.00598741494641765\\
12.1002777777778	-0.00598741494641765\\
12.1058333333333	-0.00539517331183043\\
12.1113888888889	-0.00865250230205759\\
12.1169444444444	-0.00776413985017761\\
12.1225	-0.0083563814847644\\
12.1280555555556	-0.0083563814847644\\
12.1336111111111	-0.0083563814847644\\
12.1391666666667	-0.00618454080425307\\
12.1447222222222	-0.00480994717700177\\
12.1502777777778	-0.00319947456992013\\
12.1558333333333	-0.00750911651823342\\
12.1613888888889	-0.00750911651823342\\
12.1669444444444	-0.00750911651823342\\
12.1725	-0.00750911651823342\\
12.1780555555556	-0.00750911651823342\\
12.1836111111111	-0.00750911651823342\\
12.1891666666667	-0.00839747897011425\\
12.1947222222222	-0.00839747897011425\\
12.2002777777778	-0.00839747897011425\\
12.2058333333333	-0.00839747897011425\\
12.2113888888889	-0.00839747897011425\\
12.2169444444444	-0.00839747897011425\\
12.2225	-0.00839747897011425\\
12.2280555555556	-0.00839747897011425\\
12.2336111111111	-0.00839747897011425\\
12.2391666666667	-0.00839747897011425\\
12.2447222222222	-0.00569682880774494\\
12.2502777777778	-0.00407076894512434\\
12.2558333333333	-0.00825731330851213\\
12.2613888888889	-0.00825731330851213\\
12.2669444444444	-0.00824855790691499\\
12.2725	-0.00824855790691499\\
12.2780555555556	-0.00824855790691499\\
12.2836111111111	-0.00824855790691499\\
12.2891666666667	-0.00795243708962181\\
12.2947222222222	-0.00795243708962181\\
12.3002777777778	-0.00795243708962181\\
12.3058333333333	-0.00676795382044864\\
12.3113888888889	-0.00676795382044864\\
12.3169444444444	-0.00676795382044864\\
12.3225	-0.00647183300315545\\
12.3280555555556	-0.00647183300315545\\
12.3336111111111	-0.00518835477452326\\
12.3391666666667	-0.00356229491190265\\
12.3447222222222	-0.00686047682341089\\
12.3502777777778	-0.00686047682341089\\
12.3558333333333	-0.00686047682341089\\
12.3613888888889	-0.00686047682341089\\
12.3669444444444	-0.00656435600611686\\
12.3725	-0.00567599355423688\\
12.3780555555556	-0.00567599355423688\\
12.3836111111111	-0.00567599355423688\\
12.3891666666667	-0.00478763110235648\\
12.3947222222222	-0.00478763110235648\\
12.4002777777778	-0.00478763110235648\\
12.4058333333333	-0.00478763110235648\\
12.4113888888889	-0.00478763110235648\\
12.4169444444444	-0.00478763110235648\\
12.4225	-0.00385841572919879\\
12.4280555555556	-0.00385841572919879\\
12.4336111111111	-0.0035622949119056\\
12.4391666666667	-0.0027799429192411\\
12.4447222222222	-0.00154721389632853\\
12.4502777777778	-0.00154721389632853\\
12.4558333333333	-0.00154721389632853\\
12.4613888888889	-0.00154721389632853\\
12.4669444444444	-0.000954972261741312\\
12.4725	0.00169504359075149\\
12.4780555555556	0.00276806655639917\\
12.4836111111111	0.00276806655639917\\
12.4891666666667	0.00381910548056626\\
12.4947222222222	0.00381910548056626\\
12.5002777777778	-0.000785383251124116\\
12.5058333333333	0.000267626616705574\\
12.5113888888889	0.000267626616705574\\
12.5169444444444	0.000267626616705574\\
12.5225	0.000267626616705574\\
12.5280555555556	0.00164222024395687\\
12.5336111111111	-0.000134504659804773\\
12.5391666666667	-0.00161510874627197\\
12.5447222222222	-0.00102286711168475\\
12.5502777777778	0.00104997860936966\\
12.5558333333333	0.000161616157489255\\
12.5613888888889	0.000161616157489255\\
12.5669444444444	0.000161616157489255\\
12.5725	-0.000726746294390725\\
12.5780555555556	-0.000726746294390725\\
12.5836111111111	0.00121462602531894\\
12.5891666666667	0.00121462602531894\\
12.5947222222222	0.000586821584004049\\
12.6002777777778	0.000586821584004049\\
12.6058333333333	0.000586821584004049\\
12.6113888888889	0.000389695726167787\\
12.6169444444444	-0.00168314999488662\\
12.6225	-0.000498666725713034\\
12.6280555555556	0.00080190039002781\\
12.6336111111111	0.00080190039002781\\
12.6391666666667	0.00080190039002781\\
12.6447222222222	0.00080190039002781\\
12.6502777777778	-8.64620618525908e-05\\
12.6558333333333	-8.64620618525908e-05\\
12.6613888888889	-8.64620618525908e-05\\
12.6669444444444	0.0013941420246146\\
12.6725	0.00326986188783359\\
12.6780555555556	0.00326986188783359\\
12.6836111111111	-9.61719386025918e-05\\
12.6891666666667	-9.61719386025918e-05\\
12.6947222222222	-9.61719386025918e-05\\
12.7002777777778	-9.61719386025918e-05\\
12.7058333333333	-9.61719386025918e-05\\
12.7113888888889	-9.61719386025918e-05\\
12.7169444444444	-0.000984534390483203\\
12.7225	-9.61719386025918e-05\\
12.7280555555556	-9.61719386025918e-05\\
12.7336111111111	-9.61719386025918e-05\\
12.7391666666667	-9.61719386025918e-05\\
12.7447222222222	0.00108831133057142\\
12.7502777777778	0.00108831133057142\\
12.7558333333333	0.00108831133057142\\
12.7613888888889	0.00108831133057142\\
12.7669444444444	0.00108831133057142\\
12.7725	0.00108831133057142\\
12.7780555555556	0.00230835740647353\\
12.7836111111111	0.00230835740647353\\
12.7891666666667	0.00230835740647353\\
12.7947222222222	0.000868606241285726\\
12.8002777777778	0.000868606241285726\\
12.8058333333333	0.000868606241285726\\
12.8113888888889	0.000868606241285726\\
12.8169444444444	0.000868606241285726\\
12.8225	0.00175696869316571\\
12.8280555555556	0.00175696869316571\\
12.8336111111111	0.00175696869316571\\
12.8391666666667	0.00175696869316571\\
12.8447222222222	0.00264533114504611\\
12.8502777777778	0.0036791493730795\\
12.8558333333333	0.0036791493730795\\
12.8613888888889	0.00502758314845756\\
12.8669444444444	0.00642095101800634\\
12.8725	0.00642095101800634\\
12.8780555555556	0.00215999522642612\\
12.8836111111111	0.00156775359183933\\
12.8891666666667	0.00156775359183933\\
12.8947222222222	0.00156775359183933\\
12.9002777777778	0.00127163277454614\\
12.9058333333333	0.00186387440913293\\
12.9113888888889	0.00249167885044783\\
12.9169444444444	0.00110714217710144\\
12.9225	0.00110714217710144\\
12.9280555555556	0.00110714217710144\\
12.9336111111111	0.00110714217710144\\
12.9391666666667	0.00110714217710144\\
12.9447222222222	0.00110714217710144\\
12.9502777777778	0.00110714217710144\\
12.9558333333333	0.00110714217710102\\
12.9613888888889	0.00154871877054804\\
12.9669444444444	0.00154871877054804\\
12.9725	0.000767789305238789\\
12.9780555555556	0.000883873151805625\\
12.9836111111111	0.00117999396909923\\
12.9891666666667	0.00117999396909923\\
12.9947222222222	0.00117999396909923\\
13.0002777777778	0.000767789305238368\\
13.0058333333333	0.000767789305238368\\
13.0113888888889	0.000767789305238368\\
13.0169444444444	0.000767789305238368\\
13.0225	0.000767789305238368\\
13.0280555555556	0.00106391012253197\\
13.0336111111111	0.00106391012253197\\
13.0391666666667	0.00106391012253197\\
13.0447222222222	0.00106391012253197\\
13.0502777777778	0.00116290508198932\\
13.0558333333333	0.00116290508198932\\
13.0613888888889	0.00142346309255483\\
13.0669444444444	0.00142346309255483\\
13.0725	0.00142346309255483\\
13.0780555555556	0.00175514671657654\\
13.0836111111111	0.00175514671657654\\
13.0891666666667	0.00175514671657654\\
13.0947222222222	0.00215436492548072\\
13.1002777777778	0.00215436492548072\\
13.1058333333333	0.000733361706436427\\
13.1113888888889	0.000733361706436427\\
13.1169444444444	0.0015980413331863\\
13.1225	-0.000222911703794373\\
13.1280555555556	0.000423424200127612\\
13.1336111111111	0.000423424200127612\\
13.1391666666667	0.000836098781502938\\
13.1447222222222	0.000836098781502938\\
13.1502777777778	0.00230762431726446\\
13.1558333333333	0.00230762431726446\\
13.1613888888889	0.00414866144600727\\
13.1669444444444	0.00446873486370231\\
13.1725	0.00446873486370231\\
13.1780555555556	0.00446873486370231\\
13.1836111111111	0.00446873486370231\\
13.1891666666667	0.00446873486370231\\
13.1947222222222	0.00446873486370231\\
13.2002777777778	0.00446873486370231\\
13.2058333333333	0.00446873486370231\\
13.2113888888889	0.006145650010137\\
13.2169444444444	0.00506097649828868\\
13.2225	0.00506097649828868\\
13.2280555555556	0.00703401246201656\\
13.2336111111111	0.0132503498680701\\
13.2391666666667	0.0180856610091299\\
13.2447222222222	0.0257551296658649\\
13.2502777777778	0.0257551296658649\\
13.2558333333333	0.0308865680693745\\
13.2613888888889	0.0308865680693745\\
13.2669444444444	0.0308865680693745\\
13.2725	0.0308865680693745\\
13.2780555555556	0.0308865680693745\\
13.2836111111111	0.0308865680693745\\
13.2891666666667	0.0315781778454274\\
13.2947222222222	0.0315781778454274\\
13.3002777777778	0.0315781778454274\\
13.3058333333333	0.0278279753620791\\
13.3113888888889	0.0324665402973103\\
13.3169444444444	-0.052689199289742\\
13.3225	-0.0523930784724488\\
13.3280555555556	-0.0378837488844701\\
13.3336111111111	-0.0513756303712435\\
13.3391666666667	-0.0386424352276247\\
13.3447222222222	-0.0456803249996331\\
13.3502777777778	-0.041405547908199\\
13.3558333333333	-0.041405547908199\\
13.3613888888889	-0.041405547908199\\
13.3669444444444	-0.041405547908199\\
13.3725	-0.041405547908199\\
13.3780555555556	-0.0326515994981216\\
13.3836111111111	-0.0179892470697555\\
13.3891666666667	-0.0179892470697555\\
13.3947222222222	-0.0131119384033289\\
13.4002777777778	0.0135092079775066\\
13.4058333333333	0.0308105405350936\\
13.4113888888889	0.0428908857111296\\
13.4169444444444	-0.0583824338032411\\
13.4225	-0.0583824338032411\\
13.4280555555556	-0.0580863129859479\\
13.4336111111111	-0.0580863129859479\\
13.4391666666667	-0.0580863129859479\\
13.4447222222222	-0.0580863129859479\\
13.4502777777778	-0.0580863129859479\\
13.4558333333333	-0.0485563525632219\\
13.4613888888889	-0.0485563525632219\\
13.4669444444444	-0.0490195085484834\\
13.4725	-0.0592172924451868\\
13.4780555555556	-0.0592172924451868\\
13.4836111111111	-0.0592172924451868\\
13.4891666666667	-0.0544793593684916\\
13.4947222222222	-0.0583289299933064\\
13.5002777777778	-0.0452529636838222\\
13.5058333333333	-0.0407137753337863\\
13.5113888888889	-0.0554643644324611\\
13.5169444444444	-0.0477652231828319\\
13.5225	-0.0547050876299068\\
13.5280555555556	-0.0547050876299068\\
13.5336111111111	-0.0514477586396792\\
13.5391666666667	-0.0514477586396792\\
13.5447222222222	-0.0514477586396792\\
13.5502777777778	-0.0535206043607332\\
13.5558333333333	-0.0535206043607332\\
13.5613888888889	-0.05322448354344\\
13.5669444444444	-0.0529283627261468\\
13.5725	-0.0457207184931966\\
13.5780555555556	-0.0519392556563586\\
13.5836111111111	-0.0519392556563586\\
13.5891666666667	-0.0519392556563586\\
13.5947222222222	-0.0519392556563586\\
13.6002777777778	-0.0519392556563586\\
13.6058333333333	-0.0519392556563586\\
13.6113888888889	-0.0516431348390654\\
13.6169444444444	-0.0516431348390654\\
13.6225	-0.0516431348390654\\
13.6280555555556	-0.0516431348390654\\
13.6336111111111	-0.0516431348390654\\
13.6391666666667	-0.0472013225796634\\
13.6447222222222	-0.0472013225796634\\
13.6502777777778	-0.0472013225796634\\
13.6558333333333	-0.0472013225796634\\
13.6613888888889	-0.0472013225796634\\
13.6669444444444	-0.0472013225796634\\
13.6725	-0.0472013225796634\\
13.6780555555556	-0.0472013225796634\\
13.6836111111111	-0.0472013225796634\\
13.6891666666667	-0.0472013225796634\\
13.6947222222222	-0.0513470140217714\\
13.7002777777778	-0.0504430702881821\\
13.7058333333333	-0.0504430702881821\\
13.7113888888889	-0.0504430702881821\\
13.7169444444444	-0.0483702245671281\\
13.7225	-0.0462432845767385\\
13.7280555555556	-0.04553448917807\\
13.7336111111111	-0.047607334899124\\
13.7391666666667	-0.0451842740914412\\
13.7447222222222	-0.0451842740914412\\
13.7502777777778	-0.0426488628445266\\
13.7558333333333	-0.0398016436737716\\
13.7613888888889	-0.0398016436737716\\
13.7669444444444	-0.0398016436737716\\
13.7725	-0.0398016436737716\\
13.7780555555556	-0.0419718654854604\\
13.7836111111111	-0.0419718654854604\\
13.7891666666667	-0.0487421368702952\\
13.7947222222222	-0.0486479772372455\\
13.8002777777778	-0.0468712523334847\\
13.8058333333333	-0.0449654417803982\\
13.8113888888889	-0.0449654417803982\\
13.8169444444444	-0.0449654417803982\\
13.8225	-0.0449654417803982\\
13.8280555555556	-0.0449654417803982\\
13.8336111111111	-0.0428925960593442\\
13.8391666666667	-0.0428925960593442\\
13.8447222222222	-0.0428925960593442\\
13.8502777777778	-0.0428925960593442\\
13.8558333333333	-0.0464460458668658\\
13.8613888888889	-0.0464460458668658\\
13.8669444444444	-0.0464460458668658\\
13.8725	-0.0482227707706266\\
13.8780555555556	-0.0482227707706266\\
13.8836111111111	-0.0479266499533334\\
13.8891666666667	-0.0455576834149863\\
13.8947222222222	-0.0455576834149863\\
13.9002777777778	-0.0455576834149863\\
13.9058333333333	-0.0455576834149863\\
13.9113888888889	-0.0455576834149863\\
13.9169444444444	-0.0470382875014539\\
13.9225	-0.0452615625976931\\
13.9280555555556	-0.0434848376939323\\
13.9336111111111	-0.0434848376939323\\
13.9391666666667	-0.039548109307631\\
13.9447222222222	-0.039548109307631\\
13.9502777777778	-0.0476865035639269\\
13.9558333333333	-0.0476865035639269\\
13.9613888888889	-0.0476865035639269\\
13.9669444444444	-0.0431905970351901\\
13.9725	-0.0476865035639273\\
13.9780555555556	-0.0476865035639273\\
13.9836111111111	-0.0456136578428729\\
13.9891666666667	-0.0456136578428729\\
13.9947222222222	-0.0432446913045249\\
14.0002777777778	-0.0348699805237027\\
14.0058333333333	-0.0348699805237027\\
14.0113888888889	-0.0481954173019079\\
14.0169444444444	-0.0434033899558771\\
14.0225	-0.035777664557351\\
14.0280555555556	-0.0302089899344119\\
14.0336111111111	-0.0217933982890594\\
14.0391666666667	-0.0217933982890594\\
14.0447222222222	-0.0217933982890594\\
14.0502777777778	-0.0133192578947086\\
14.0558333333333	-0.00490154985592733\\
14.0613888888889	-0.00209564643610781\\
14.0669444444444	0.000710257032371143\\
14.0725	0.000710257032371143\\
14.0780555555556	0.000710257032371143\\
14.0836111111111	0.000710257032371143\\
14.0891666666667	0.000710257032371143\\
14.0947222222222	0.000710257032371143\\
14.1002777777778	0.000710257032371143\\
14.1058333333333	0.000710257032371143\\
14.1113888888889	0.000710257032371143\\
14.1169444444444	0.000710257032371143\\
14.1225	0.000710257032371143\\
14.1280555555556	0.000710257032371143\\
14.1336111111111	-0.00177646578228183\\
14.1391666666667	-0.0475461068134383\\
14.1447222222222	-0.0413281601096759\\
14.1502777777778	-0.0413281601096759\\
14.1558333333333	-0.0382238197473684\\
14.1613888888889	-0.0382238197473684\\
14.1669444444444	-0.0320517373481859\\
14.1725	-0.0474755135251013\\
14.1780555555556	-0.0448104261694601\\
14.1836111111111	-0.0420912445444833\\
14.1891666666667	-0.0393096034247608\\
14.1947222222222	-0.0337261683947623\\
14.2002777777778	-0.0308978054638665\\
14.2058333333333	-0.0308978054638665\\
14.2113888888889	-0.0308978054638665\\
14.2169444444444	-0.0308978054638665\\
14.2225	-0.044704019501381\\
14.2280555555556	-0.0423350529630334\\
14.2336111111111	-0.0395408799580669\\
14.2391666666667	-0.0367093478938582\\
14.2447222222222	-0.028148600105989\\
14.2502777777778	-0.0224265705275226\\
14.2558333333333	-0.019564881819064\\
14.2613888888889	-0.0167030910786026\\
14.2669444444444	-0.00811759299017918\\
14.2725	-0.00525575353118444\\
14.2780555555556	-0.00239391372569902\\
14.2836111111111	-0.00239391372569902\\
14.2891666666667	0.00392200785926239\\
14.2947222222222	0.0073760895253175\\
14.3002777777778	0.0073760895253175\\
14.3058333333333	0.0073760895253175\\
14.3113888888889	0.0073760895253175\\
14.3169444444444	0.0073760895253175\\
14.3225	0.0073760895253175\\
14.3280555555556	0.0073760895253175\\
14.3336111111111	0.0073760895253175\\
14.3391666666667	0.0073760895253175\\
14.3447222222222	0.0073760895253175\\
14.3502777777778	0.0073760895253175\\
14.3558333333333	0.0073760895253175\\
14.3613888888889	0.0073760895253175\\
14.3669444444444	0.0073760895253175\\
14.3725	0.0073760895253175\\
14.3780555555556	-0.0618168135798721\\
14.3836111111111	-0.0618168135798721\\
14.3891666666667	-0.0618168135798721\\
14.3947222222222	-0.0618168135798721\\
14.4002777777778	-0.0582633637723505\\
14.4058333333333	-0.0615206927625785\\
14.4113888888889	-0.0615206927625785\\
14.4169444444444	-0.0579672429550573\\
14.4225	-0.0579672429550573\\
14.4280555555556	-0.057596130757774\\
14.4336111111111	-0.0617418221998824\\
14.4391666666667	-0.0614457013825888\\
14.4447222222222	-0.0611495805652956\\
14.4502777777778	-0.0611495805652956\\
14.4558333333333	-0.0596533784964781\\
14.4613888888889	-0.0596377972147692\\
14.4669444444444	-0.059341676397476\\
14.4725	-0.059341676397476\\
14.4780555555556	-0.059341676397476\\
14.4836111111111	-0.059341676397476\\
14.4891666666667	-0.059341676397476\\
14.4947222222222	-0.059341676397476\\
14.5002777777778	-0.059341676397476\\
14.5058333333333	-0.059341676397476\\
14.5113888888889	-0.059341676397476\\
14.5169444444444	-0.0592403500107703\\
14.5225	-0.0592403500107703\\
14.5280555555556	-0.0592403500107703\\
14.5336111111111	-0.0592403500107703\\
14.5391666666667	-0.0589442291934771\\
14.5447222222222	-0.0589442291934771\\
14.5502777777778	-0.0592403500107703\\
14.5558333333333	-0.0592403500107703\\
14.5613888888889	-0.0592403500107703\\
14.5669444444444	-0.0592403500107703\\
14.5725	-0.0586481083761839\\
14.5780555555556	-0.052609138266233\\
14.5836111111111	-0.058827675429395\\
14.5891666666667	-0.0563296232417221\\
14.5947222222222	-0.053960656703374\\
14.6002777777778	-0.053960656703374\\
14.6058333333333	-0.0485899745791773\\
14.6113888888889	-0.0485899745791773\\
14.6169444444444	-0.0485899745791773\\
14.6225	-0.0590515792750829\\
14.6280555555556	-0.0590515792750829\\
14.6336111111111	-0.0560903711021485\\
14.6391666666667	-0.0560903711021485\\
14.6447222222222	-0.0560903711021485\\
14.6502777777778	-0.0560903711021485\\
14.6558333333333	-0.0581632168232025\\
14.6613888888889	-0.0578670960059093\\
14.6669444444444	-0.0578670960059093\\
14.6725	-0.0602360625442569\\
14.6780555555556	-0.0560903711021485\\
14.6836111111111	-0.0693322310064605\\
14.6891666666667	-0.0693322310064605\\
14.6947222222222	-0.0651959368796947\\
14.7002777777778	-0.0651959368796947\\
14.7058333333333	-0.0651959368796947\\
14.7113888888889	-0.0651959368796947\\
14.7169444444444	-0.0651959368796947\\
14.7225	-0.0681571450526295\\
14.7280555555556	-0.0648998160624015\\
14.7336111111111	-0.0648998160624015\\
14.7391666666667	-0.0646036952451083\\
14.7447222222222	-0.0684532658699235\\
14.7502777777778	-0.0643075744278147\\
14.7558333333333	-0.0643075744278147\\
14.7613888888889	-0.0681571450526295\\
14.7669444444444	-0.0716115999006933\\
14.7725	-0.0716115999006933\\
14.7780555555556	-0.0716115999006933\\
14.7836111111111	-0.0677620292758785\\
14.7891666666667	-0.0677620292758785\\
14.7947222222222	-0.0710193582661061\\
14.8002777777778	-0.0710193582661061\\
14.8058333333333	-0.0763703294979138\\
14.8113888888889	-0.0707440339693374\\
14.8169444444444	-0.0721837851345256\\
14.8225	-0.0679082251718028\\
14.8280555555556	-0.0419796878884824\\
14.8336111111111	-0.0388552041402926\\
14.8391666666667	-0.0388552041402926\\
14.8447222222222	-0.0563861417755345\\
14.8502777777778	-0.0534249336026001\\
14.8558333333333	-0.0534249336026001\\
14.8613888888889	-0.0534249336026001\\
14.8669444444444	-0.0534249336026001\\
14.8725	-0.0534249336026001\\
14.8780555555556	-0.0534249336026001\\
14.8836111111111	-0.0534249336026001\\
14.8891666666667	-0.0534249336026001\\
14.8947222222222	-0.0534249336026001\\
14.9002777777778	-0.0557939001409477\\
14.9058333333333	-0.0557939001409477\\
14.9113888888889	-0.0534249336026001\\
14.9169444444444	-0.0531288127853069\\
14.9225	-0.0545553226024389\\
14.9280555555556	-0.0545553226024389\\
14.9336111111111	-0.0527785976986781\\
14.9391666666667	-0.0527785976986781\\
14.9447222222222	-0.0527785976986781\\
14.9502777777778	-0.0527785976986781\\
14.9558333333333	-0.0475780990961223\\
14.9613888888889	-0.054388877893871\\
14.9669444444444	-0.0564617236149254\\
14.9725	-0.0561656027976323\\
14.9780555555556	-0.0561656027976323\\
14.9836111111111	-0.0561656027976323\\
14.9891666666667	-0.0555733611630459\\
14.9947222222222	-0.0535005154419919\\
15.0002777777778	-0.0535005154419919\\
15.0058333333333	-0.0535005154419919\\
15.0113888888889	-0.0552772403457527\\
15.0169444444444	-0.0552772403457527\\
15.0225	-0.0552772403457527\\
15.0280555555556	-0.0552772403457527\\
15.0336111111111	-0.0552772403457527\\
15.0391666666667	-0.0529082738074051\\
15.0447222222222	-0.0557403963310142\\
15.0502777777778	-0.0557403963310142\\
15.0558333333333	-0.0557403963310142\\
15.0613888888889	-0.053075308975373\\
15.0669444444444	-0.0476285805000093\\
15.0725	-0.0448661170537333\\
15.0780555555556	-0.0448661170537333\\
15.0836111111111	-0.0422010296980921\\
15.0891666666667	-0.0554784524091326\\
15.0947222222222	-0.0554784524091326\\
15.1002777777778	-0.0504443985151443\\
15.1058333333333	-0.0504443985151443\\
15.1113888888889	-0.0458061946147466\\
15.1169444444444	-0.0458061946147466\\
15.1225	-0.0574734387998229\\
15.1280555555556	-0.0548239327258906\\
15.1336111111111	-0.0566006576296514\\
15.1391666666667	-0.0566006576296514\\
15.1447222222222	-0.0532142429900973\\
15.1502777777778	-0.051271073377769\\
15.1558333333333	-0.0557128856371702\\
15.1613888888889	-0.054232281550703\\
15.1669444444444	-0.0526975831949002\\
15.1725	-0.0587496756494941\\
15.1780555555556	-0.0560845882938535\\
15.1836111111111	-0.0560845882938535\\
15.1891666666667	-0.0554923466592671\\
15.1947222222222	-0.0534195009382131\\
15.2002777777778	-0.0480899166863308\\
15.2058333333333	-0.0480899166863308\\
15.2113888888889	-0.0550016512658104\\
15.2169444444444	-0.0535210471793433\\
15.2225	-0.051565345681771\\
15.2280555555556	-0.051565345681771\\
15.2336111111111	-0.0493759461966357\\
15.2391666666667	-0.0493759461966357\\
15.2447222222222	-0.0564828458116789\\
15.2502777777778	-0.0541138792733313\\
15.2558333333333	-0.0552983625425053\\
15.2613888888889	-0.0547061209079185\\
15.2669444444444	-0.054129477255683\\
15.2725	-0.0554394333087332\\
15.2780555555556	-0.0396258440522822\\
15.2836111111111	-0.0534419806063529\\
15.2891666666667	-0.0534419806063529\\
15.2947222222222	-0.0534419806063529\\
15.3002777777778	-0.0497594451495056\\
15.3058333333333	-0.0524245325051476\\
15.3113888888889	-0.0487419970483003\\
15.3169444444444	-0.0409904000922827\\
15.3225	-0.0572770450434239\\
15.3280555555556	-0.053131353601316\\
15.3336111111111	-0.053131353601316\\
15.3391666666667	-0.0560175703942602\\
15.3447222222222	-0.0560175703942602\\
15.3502777777778	-0.0560175703942602\\
15.3558333333333	-0.0449445463903218\\
15.3613888888889	-0.0449445463903218\\
15.3669444444444	-0.0406167231330992\\
15.3725	-0.0560283495538985\\
15.3780555555556	-0.0560283495538985\\
15.3836111111111	-0.0560283495538985\\
15.3891666666667	-0.0488673556695197\\
15.3947222222222	-0.0373790249083894\\
15.4002777777778	-0.0546327108383945\\
15.4058333333333	-0.0546327108383945\\
15.4113888888889	-0.0546327108383945\\
15.4169444444444	-0.0475258112233513\\
15.4225	-0.0403215355176742\\
15.4280555555556	-0.0432827436906086\\
15.4336111111111	-0.0530547306612922\\
15.4391666666667	-0.0530547306612922\\
15.4447222222222	-0.0521663682094126\\
15.4502777777778	-0.0521663682094126\\
15.4558333333333	-0.0515741265748263\\
15.4613888888889	-0.0512780057575331\\
15.4669444444444	-0.0509818849402399\\
15.4725	-0.0486129184018927\\
15.4780555555556	-0.0486129184018927\\
15.4836111111111	-0.0486129184018927\\
15.4891666666667	-0.0486129184018927\\
15.4947222222222	-0.0486129184018927\\
15.5002777777778	-0.0461898575942099\\
15.5058333333333	-0.0461898575942099\\
15.5113888888889	-0.0500394282190247\\
15.5169444444444	-0.0500394282190247\\
15.5225	-0.0466530135794706\\
15.5280555555556	-0.0466530135794706\\
15.5336111111111	-0.049910342569699\\
15.5391666666667	-0.0496142217524058\\
15.5447222222222	-0.0481336176659386\\
15.5502777777778	-0.0504484899349506\\
15.5558333333333	-0.046248704223507\\
15.5613888888889	-0.0413676389574386\\
15.5669444444444	-0.050843505110829\\
15.5725	-0.0459764863848081\\
15.5780555555556	-0.0459764863848081\\
15.5836111111111	-0.0477532112885689\\
15.5891666666667	-0.0495299361923297\\
15.5947222222222	-0.0495299361923297\\
15.6002777777778	-0.0495299361923297\\
15.6058333333333	-0.0474570904712757\\
15.6113888888889	-0.0430158686712746\\
15.6169444444444	-0.048049922565263\\
15.6225	-0.0471615601133817\\
15.6280555555556	-0.0474420829483241\\
15.6336111111111	-0.0459614788618569\\
15.6391666666667	-0.0427546312755151\\
15.6447222222222	-0.0468613230918825\\
15.6502777777778	-0.0456768398227089\\
15.6558333333333	-0.0436045845610558\\
15.6613888888889	-0.0450851886475234\\
15.6669444444444	-0.044196826195643\\
15.6725	-0.0426621278398398\\
15.6780555555556	-0.0426621278398398\\
15.6836111111111	-0.0426621278398398\\
15.6891666666667	-0.0410150790448047\\
15.6947222222222	-0.0410150790448047\\
15.7002777777778	-0.042822873015428\\
15.7058333333333	-0.0422306313808408\\
15.7113888888889	-0.0422306313808408\\
15.7169444444444	-0.0422306313808408\\
15.7225	-0.0422306313808408\\
15.7280555555556	-0.0422306313808408\\
15.7336111111111	-0.0415093040969287\\
15.7391666666667	-0.0415093040969287\\
15.7447222222222	-0.0407506177537749\\
15.7502777777778	-0.0328163091697003\\
15.7558333333333	-0.0445361293790918\\
15.7613888888889	-0.0347914887814482\\
15.7669444444444	-0.0325230992531773\\
15.7725	-0.0406851066324949\\
15.7780555555556	-0.0403889858152013\\
15.7836111111111	-0.0383167305535477\\
15.7891666666667	-0.0390754168967016\\
15.7947222222222	-0.0390754168967016\\
15.8002777777778	-0.0381870544448212\\
15.8058333333333	-0.037890933627528\\
15.8113888888889	-0.0375948128102348\\
15.8169444444444	-0.0375948128102348\\
15.8225	-0.0375948128102348\\
15.8280555555556	-0.037890933627528\\
15.8336111111111	-0.0372986919929414\\
15.8391666666667	-0.0372986919929416\\
15.8447222222222	-0.0372986919929416\\
15.8502777777778	-0.0372986919929416\\
15.8558333333333	-0.0372986919929416\\
15.8613888888889	-0.0371696063436159\\
15.8669444444444	-0.0355493134622207\\
15.8725	-0.0347615280204158\\
15.8780555555556	-0.0347615280204158\\
15.8836111111111	-0.0347615280204158\\
15.8891666666667	-0.0339729753258623\\
15.8947222222222	-0.0339729753258623\\
15.9002777777778	-0.0386110986452331\\
15.9058333333333	-0.0383149778279399\\
15.9113888888889	-0.0383149778279399\\
15.9169444444444	-0.0377227361933535\\
15.9225	-0.0377227361933535\\
15.9280555555556	-0.0374266153760604\\
15.9336111111111	-0.0363773256700509\\
15.9391666666667	-0.0360812048527577\\
15.9447222222222	-0.0369701577640381\\
15.9502777777778	-0.0350138658070646\\
15.9558333333333	-0.0350138658070646\\
15.9613888888889	-0.0350138658070646\\
15.9669444444444	-0.0356061074416518\\
15.9725	-0.0344216241724782\\
15.9780555555556	-0.0344216241724782\\
15.9836111111111	-0.0338293825378919\\
15.9891666666667	-0.0346089828130371\\
15.9947222222222	-0.0357006922207798\\
};
\addlegendentry{Cts Stoch Ctrl w nFPC};

\addplot [color=cts_nFPC_plot_color,solid,line width=1.5pt]
  table[row sep=crcr]{%
9.50027777777778	0.00651465798045599\\
9.50583333333333	0.0200445396026153\\
9.51138888888889	0.0270451721856112\\
9.51694444444444	0.0174153800991449\\
9.5225	0.0265153206998117\\
9.52805555555556	0.0447450874263107\\
9.53361111111111	0.046603099408904\\
9.53916666666667	0.0261707630156523\\
9.54472222222222	0.0258949357735249\\
9.55027777777778	0.0286412274727425\\
9.55583333333333	0.0242240013259122\\
9.56138888888889	0.0260250749775928\\
9.56694444444444	0.0291449923131466\\
9.5725	0.0290997660447632\\
9.57805555555555	0.0290997660447632\\
9.58361111111111	0.03058037013123\\
9.58916666666667	0.0335040185963986\\
9.59472222222222	0.0320670569117455\\
9.60027777777778	0.024283724091004\\
9.60583333333333	0.0237432062674186\\
9.61138888888889	0.0201389363121283\\
9.61694444444444	0.0211689122859709\\
9.6225	0.0268021191607941\\
9.62805555555556	0.0277221793725048\\
9.63361111111111	0.026529314779487\\
9.63916666666667	0.0261809076460262\\
9.64472222222222	0.0267852031711157\\
9.65027777777778	0.0261576487717202\\
9.65583333333333	0.0261576487717202\\
9.66138888888889	0.0288234232098251\\
9.66694444444444	0.0300070423924282\\
9.6725	0.0304860973839293\\
9.67805555555555	0.0293291369944308\\
9.68361111111111	0.0304149581095141\\
9.68916666666667	0.0294693825263812\\
9.69472222222222	0.030860199134917\\
9.70027777777778	0.0311650879796876\\
9.70583333333333	0.0303755713050265\\
9.71138888888889	0.0318637647476124\\
9.71694444444444	0.0309207306601339\\
9.7225	0.0321102889706195\\
9.72805555555555	0.0324632312580554\\
9.73361111111111	0.0324752968537227\\
9.73916666666667	0.0324752968537227\\
9.74472222222222	0.0324752968537227\\
9.75027777777778	0.0336597801228963\\
9.75583333333333	0.0348442633920698\\
9.76138888888889	0.0339087734052872\\
9.76694444444444	0.0338508535023585\\
9.7725	0.0354745370385622\\
9.77805555555556	0.0354745370385622\\
9.78361111111111	0.0360121070375505\\
9.78916666666667	0.0361645595245702\\
9.79472222222222	0.0360214802968167\\
9.80027777777778	0.0370815358515861\\
9.80583333333333	0.0370815358515861\\
9.81138888888889	0.0370876140518991\\
9.81694444444444	0.0373904407777211\\
9.8225	0.0384191435861967\\
9.82805555555555	0.0375030998715784\\
9.83361111111111	0.0378052448394507\\
9.83916666666667	0.0384988563190337\\
9.84472222222222	0.0384988563190337\\
9.85027777777778	0.0391991515161995\\
9.85583333333333	0.0394142663553523\\
9.86138888888889	0.037823334563378\\
9.86694444444444	0.0384154208101019\\
9.8725	0.0392088538254531\\
9.87805555555556	0.0394578471078444\\
9.88361111111111	0.0392538946577874\\
9.88916666666667	0.0392538946577874\\
9.89472222222222	0.0395516041216764\\
9.90027777777778	0.0395516041216764\\
9.90583333333333	0.0398580816130423\\
9.91138888888889	0.0401542024303355\\
9.91694444444444	0.0401542024303355\\
9.9225	0.0400995307947374\\
9.92805555555555	0.0401082988222144\\
9.93361111111111	0.0401082988222144\\
9.93916666666667	0.0401602363204419\\
9.94472222222222	0.04030243120466\\
9.95027777777778	0.04030243120466\\
9.95583333333333	0.0421249112549149\\
9.96138888888889	0.0443178215135495\\
9.96694444444444	0.0412711668161853\\
9.9725	0.042751770902652\\
9.97805555555555	0.0411530269572411\\
9.98361111111111	0.0411530269572411\\
9.98916666666667	0.0434652026822263\\
9.99472222222222	0.0440570122735278\\
10.0002777777778	0.0440570122735278\\
10.0058333333333	0.0440570122735278\\
10.0113888888889	0.0440570122735278\\
10.0169444444444	0.0400042205605599\\
10.0225	0.0403797392554138\\
10.0280555555556	0.0403797392554138\\
10.0336111111111	0.0406720578857378\\
10.0391666666667	0.0406753554539282\\
10.0447222222222	0.0422062813331845\\
10.0502777777778	0.0424477305148796\\
10.0558333333333	0.0433295566424198\\
10.0613888888889	0.0433295566424198\\
10.0669444444444	0.0407969029261894\\
10.0725	0.0407969029261894\\
10.0780555555556	0.0411478084689101\\
10.0836111111111	0.0411478084689101\\
10.0891666666667	0.0411478084689101\\
10.0947222222222	0.0411478084689101\\
10.1002777777778	0.0421892125103307\\
10.1058333333333	0.0440564516499785\\
10.1113888888889	0.0443591676036528\\
10.1169444444444	0.0410849314077858\\
10.1225	0.0410849314077858\\
10.1280555555556	0.0410849314077858\\
10.1336111111111	0.0410849314077858\\
10.1391666666667	0.0410849314077858\\
10.1447222222222	0.0378022111379179\\
10.1502777777778	0.0378022111379179\\
10.1558333333333	0.0378022111379179\\
10.1613888888889	0.0378022111379179\\
10.1669444444444	0.0413660252897712\\
10.1725	0.0413703170777\\
10.1780555555556	0.0413703170777\\
10.1836111111111	0.041301472373834\\
10.1891666666667	0.0417596925000573\\
10.1947222222222	0.0417596925000573\\
10.2002777777778	0.0417596925000573\\
10.2058333333333	0.0417596925000573\\
10.2113888888889	0.0417596925000573\\
10.2169444444444	0.0417596925000573\\
10.2225	0.0421980082015685\\
10.2280555555556	0.0434313957837689\\
10.2336111111111	0.0444612217579636\\
10.2391666666667	0.0446600184939979\\
10.2447222222222	0.0444553460365318\\
10.2502777777778	0.0454351568482393\\
10.2558333333333	0.0442490578736269\\
10.2613888888889	0.0468331392511278\\
10.2669444444444	0.0474249488424294\\
10.2725	0.0458175836313272\\
10.2780555555556	0.0478427490229996\\
10.2836111111111	0.0436970575808916\\
10.2891666666667	0.0423586483786421\\
10.2947222222222	0.0423586483786421\\
10.3002777777778	0.0435474234357445\\
10.3058333333333	0.0435474234357445\\
10.3113888888889	0.0434960493683359\\
10.3169444444444	0.0465090368383444\\
10.3225	0.0479965522710601\\
10.3280555555556	0.046500968033926\\
10.3336111111111	0.0456186837823595\\
10.3391666666667	0.0447370272390079\\
10.3447222222222	0.0441525759646425\\
10.3502777777778	0.0441525759646425\\
10.3558333333333	0.0451866937267664\\
10.3613888888889	0.0451866937267664\\
10.3669444444444	0.0451866937267664\\
10.3725	0.0451866937267664\\
10.3780555555556	0.0451866937267664\\
10.3836111111111	0.0451866937267664\\
10.3891666666667	0.045043614499013\\
10.3947222222222	0.045043614499013\\
10.4002777777778	0.0446040429477961\\
10.4058333333333	0.0446040429477961\\
10.4113888888889	0.0453477613362132\\
10.4169444444444	0.0453477613362132\\
10.4225	0.0453477613362132\\
10.4280555555556	0.046090895718674\\
10.4336111111111	0.046090895718674\\
10.4391666666667	0.046090895718674\\
10.4447222222222	0.0460362240830759\\
10.4502777777778	0.0460362240830759\\
10.4558333333333	0.0460362240830759\\
10.4613888888889	0.0460362240830759\\
10.4669444444444	0.0463328988200085\\
10.4725	0.0460868249371383\\
10.4780555555556	0.0469280598541165\\
10.4836111111111	0.0469280598541165\\
10.4891666666667	0.0463505206172089\\
10.4947222222222	0.0470689085949639\\
10.5002777777778	0.0470689085949639\\
10.5058333333333	0.047910143511942\\
10.5113888888889	0.0482796145095872\\
10.5169444444444	0.0482758123226176\\
10.5225	0.0482758123226176\\
10.5280555555556	0.0482758123226176\\
10.5336111111111	0.0488468088470052\\
10.5391666666667	0.0488468088470052\\
10.5447222222222	0.049132307109199\\
10.5502777777778	0.049132307109199\\
10.5558333333333	0.0494295357657707\\
10.5613888888889	0.0494295357657707\\
10.5669444444444	0.0494295357657707\\
10.5725	0.0494295357657707\\
10.5780555555556	0.0500141730264187\\
10.5836111111111	0.0506108200684743\\
10.5891666666667	0.0511954573291224\\
10.5947222222222	0.0511987548973127\\
10.6002777777778	0.0495729227414263\\
10.6058333333333	0.0495729227414263\\
10.6113888888889	0.0482337821772737\\
10.6169444444444	0.0482337821772737\\
10.6225	0.0482337821772737\\
10.6280555555556	0.0501635486705747\\
10.6336111111111	0.0501635486705747\\
10.6391666666667	0.0500947039667087\\
10.6447222222222	0.0489559758440293\\
10.6502777777778	0.0506855732128878\\
10.6558333333333	0.0506855732128878\\
10.6613888888889	0.0506855732128878\\
10.6669444444444	0.0529998681156369\\
10.6725	0.04923234553483\\
10.6780555555556	0.0531748158887857\\
10.6836111111111	0.0531748158887857\\
10.6891666666667	0.0534603141509794\\
10.6947222222222	0.0534603141509794\\
10.7002777777778	0.0515383611777994\\
10.7058333333333	0.0537534017830715\\
10.7113888888889	0.0540389000452653\\
10.7169444444444	0.0540389000452653\\
10.7225	0.0540389000452653\\
10.7280555555556	0.0540389000452653\\
10.7336111111111	0.0524130678893784\\
10.7391666666667	0.0543319876773573\\
10.7447222222222	0.051748948365724\\
10.7502777777778	0.0505151699414724\\
10.7558333333333	0.0505151699414724\\
10.7613888888889	0.0526415262550694\\
10.7669444444444	0.0550781112234604\\
10.7725	0.0573924061262095\\
10.7780555555556	0.0573924061262095\\
10.7836111111111	0.0573924061262095\\
10.7891666666667	0.0573924061262095\\
10.7947222222222	0.0573924061262095\\
10.8002777777778	0.057963402650597\\
10.8058333333333	0.058548039911245\\
10.8113888888889	0.058548039911245\\
10.8169444444444	0.056626086938065\\
10.8225	0.056626086938065\\
10.8280555555556	0.0572171022988287\\
10.8336111111111	0.0572171022988287\\
10.8391666666667	0.0561910504392158\\
10.8447222222222	0.0582614073621766\\
10.8502777777778	0.060625002317626\\
10.8558333333333	0.0609263462299384\\
10.8613888888889	0.0616764518219443\\
10.8669444444444	0.0616852198494213\\
10.8725	0.0622785693232866\\
10.8780555555556	0.0622785693232866\\
10.8836111111111	0.0622785693232866\\
10.8891666666667	0.0622785693232866\\
10.8947222222222	0.0622785693232866\\
10.9002777777778	0.0622785693232866\\
10.9058333333333	0.0622785693232866\\
10.9113888888889	0.0588654256723275\\
10.9169444444444	0.0588654256723275\\
10.9225	0.0587386645476921\\
10.9280555555556	0.0571312993365891\\
10.9336111111111	0.0546887532470368\\
10.9391666666667	0.0621743089913325\\
10.9447222222222	0.056544204226185\\
10.9502777777778	0.060353733632553\\
10.9558333333333	0.0602097263558239\\
10.9613888888889	0.0614181207671955\\
10.9669444444444	0.0603797453467011\\
10.9725	0.0602041054157616\\
10.9780555555556	0.0596045568575515\\
10.9836111111111	0.0609183134064565\\
10.9891666666667	0.0608240125912719\\
10.9947222222222	0.0612817887979883\\
11.0002777777778	0.06124148982751\\
11.0058333333333	0.06124148982751\\
11.0113888888889	0.0606492481929219\\
11.0169444444444	0.0620604485484612\\
11.0225	0.0620604485484612\\
11.0280555555556	0.0620604485484612\\
11.0336111111111	0.0610793554727681\\
11.0391666666667	0.0627866766456006\\
11.0447222222222	0.0828581917789997\\
11.0502777777778	0.0827893470751341\\
11.0558333333333	0.0848733993572442\\
11.0613888888889	0.0848733993572442\\
11.0669444444444	0.0848733993572442\\
11.0725	0.0828005536361903\\
11.0780555555556	0.0828005536361903\\
11.0836111111111	0.0828005536361903\\
11.0891666666667	0.0828005536361903\\
11.0947222222222	0.0828005536361903\\
11.1002777777778	0.0828005536361903\\
11.1058333333333	0.0828005536361903\\
11.1113888888889	0.0860536188827986\\
11.1169444444444	0.0860536188827986\\
11.1225	0.0860536188827986\\
11.1280555555556	0.0860536188827986\\
11.1336111111111	0.0860536188827986\\
11.1391666666667	0.0860536188827986\\
11.1447222222222	0.0863459375131226\\
11.1502777777778	0.0866498486906376\\
11.1558333333333	0.0869245011440224\\
11.1613888888889	0.0869245011440224\\
11.1669444444444	0.0869277987122123\\
11.1725	0.0869277987122123\\
11.1780555555556	0.0869277987122123\\
11.1836111111111	0.0869277987122123\\
11.1891666666667	0.0847097249217391\\
11.1947222222222	0.0847097249217391\\
11.2002777777778	0.0876162614590754\\
11.2058333333333	0.0844070677801053\\
11.2113888888889	0.0815873231294851\\
11.2169444444444	0.0846161497324629\\
11.2225	0.0846161497324629\\
11.2280555555556	0.0846161497324629\\
11.2336111111111	0.0846161497324629\\
11.2391666666667	0.0881711881865799\\
11.2447222222222	0.0881711881865799\\
11.2502777777778	0.0849181509842803\\
11.2558333333333	0.0881832703347297\\
11.2613888888889	0.0881838242543687\\
11.2669444444444	0.0847818211143601\\
11.2725	0.0847818211143601\\
11.2780555555556	0.0885854881310534\\
11.2836111111111	0.0885854881310534\\
11.2891666666667	0.0885854881310534\\
11.2947222222222	0.0885854881310534\\
11.3002777777778	0.0888778067613774\\
11.3058333333333	0.0888778067613774\\
11.3113888888889	0.0888778067613774\\
11.3169444444444	0.0891633050235707\\
11.3225	0.0891633050235707\\
11.3280555555556	0.0891633050235707\\
11.3336111111111	0.0891633050235707\\
11.3391666666667	0.0891633050235707\\
11.3447222222222	0.0891633050235707\\
11.3502777777778	0.0861338060664093\\
11.3558333333333	0.0861338060664093\\
11.3613888888889	0.0833998740069021\\
11.3669444444444	0.0899274787818029\\
11.3725	0.086970562396797\\
11.3780555555556	0.086970562396797\\
11.3836111111111	0.086970562396797\\
11.3891666666667	0.086970562396797\\
11.3947222222222	0.086970562396797\\
11.4002777777778	0.086970562396797\\
11.4058333333333	0.077927369786894\\
11.4113888888889	0.077927369786894\\
11.4169444444444	0.0719375137628995\\
11.4225	0.0913490578727498\\
11.4280555555556	0.0913490578727498\\
11.4336111111111	0.0916345561349435\\
11.4391666666667	0.0916345561349435\\
11.4447222222222	0.0916345561349435\\
11.4502777777778	0.0916345561349435\\
11.4558333333333	0.0891203615271763\\
11.4613888888889	0.0873893917699097\\
11.4669444444444	0.0873893917699097\\
11.4725	0.0858594925283649\\
11.4780555555556	0.0858594925283649\\
11.4836111111111	0.0879812013612104\\
11.4891666666667	0.0881902833135684\\
11.4947222222222	0.0882000749151273\\
11.5002777777778	0.0882000749151273\\
11.5058333333333	0.0882000749151273\\
11.5113888888889	0.0920549471020703\\
11.5169444444444	0.090330159696537\\
11.5225	0.090330159696537\\
11.5280555555556	0.090330159696537\\
11.5336111111111	0.090330159696537\\
11.5391666666667	0.090330159696537\\
11.5447222222222	0.0923976338346928\\
11.5502777777778	0.0906252007188604\\
11.5558333333333	0.0906252007188604\\
11.5613888888889	0.0910800046442494\\
11.5669444444444	0.0932950452495215\\
11.5725	0.0932950452495215\\
11.5780555555556	0.0932950452495215\\
11.5836111111111	0.0932950452495215\\
11.5891666666667	0.0913730922763415\\
11.5947222222222	0.0913730922763415\\
11.6002777777778	0.0894417100776017\\
11.6058333333333	0.0879118108360569\\
11.6113888888889	0.0879118108360569\\
11.6169444444444	0.0879118108360569\\
11.6225	0.0904827487117684\\
11.6280555555556	0.092254102032631\\
11.6336111111111	0.092262870060108\\
11.6391666666667	0.092262870060108\\
11.6447222222222	0.0879031058447143\\
11.6502777777778	0.0857809649685831\\
11.6558333333333	0.0857809649685831\\
11.6613888888889	0.0857809649685831\\
11.6669444444444	0.0857809649685831\\
11.6725	0.0808743033071966\\
11.6780555555556	0.0786993175886138\\
11.6836111111111	0.0786993175886138\\
11.6891666666667	0.0880819342541435\\
11.6947222222222	0.0880819342541435\\
11.7002777777778	0.0880819342541435\\
11.7058333333333	0.0880819342541435\\
11.7113888888889	0.0880819342541435\\
11.7169444444444	0.0880819342541435\\
11.7225	0.0880819342541435\\
11.7280555555556	0.0880819342541435\\
11.7336111111111	0.0860548436795841\\
11.7391666666667	0.0842841429264024\\
11.7447222222222	0.0826865052258992\\
11.7502777777778	0.078265218456461\\
11.7558333333333	0.0743570991498332\\
11.7613888888889	0.0743570991498332\\
11.7669444444444	0.0743570991498332\\
11.7725	0.0733746736962139\\
11.7780555555556	0.0724643667074089\\
11.7836111111111	0.071554706917513\\
11.7891666666667	0.0822150563400787\\
11.7947222222222	0.0829761002655253\\
11.8002777777778	0.0830693548705894\\
11.8058333333333	0.0830693548705894\\
11.8113888888889	0.0830693548705894\\
11.8169444444444	0.0830693548705894\\
11.8225	0.0830693548705894\\
11.8280555555556	0.0830693548705894\\
11.8336111111111	0.0830693548705894\\
11.8391666666667	0.0824082685321361\\
11.8447222222222	0.0836603702313531\\
11.8502777777778	0.0836603702313531\\
11.8558333333333	0.0829992838929003\\
11.8613888888889	0.0843451530314749\\
11.8669444444444	0.0826141832742091\\
11.8725	0.0832003148962095\\
11.8780555555556	0.0814296141430278\\
11.8836111111111	0.0799559214028091\\
11.8891666666667	0.077378386356437\\
11.8947222222222	0.0790272885738052\\
11.9002777777778	0.077674367359841\\
11.9058333333333	0.0752754827046198\\
11.9113888888889	0.073928260930088\\
11.9169444444444	0.073928260930088\\
11.9225	0.0726490493888428\\
11.9280555555556	0.0726490493888428\\
11.9336111111111	0.0726490493888428\\
11.9391666666667	0.0832648651355022\\
11.9447222222222	0.0832648651355022\\
11.9502777777778	0.0832648651355022\\
11.9558333333333	0.0832648651355022\\
11.9613888888889	0.0850431786858583\\
11.9669444444444	0.0832697513502879\\
11.9725	0.0832697513502879\\
11.9780555555556	0.0816344899688417\\
11.9836111111111	0.0839535948349169\\
11.9891666666667	0.0839535948349169\\
11.9947222222222	0.0839535948349169\\
12.0002777777778	0.0839535948349169\\
12.0058333333333	0.0810070890596618\\
12.0113888888889	0.0785614707333441\\
12.0169444444444	0.0773447713418571\\
12.0225	0.0773447713418571\\
12.0280555555556	0.0760603326762928\\
12.0336111111111	0.0760603326762928\\
12.0391666666667	0.0779369259019658\\
12.0447222222222	0.0801117940097214\\
12.0502777777778	0.0801117940097214\\
12.0558333333333	0.0854419687210038\\
12.0613888888889	0.0837566921770701\\
12.0669444444444	0.0774709621092213\\
12.0725	0.0778980901228201\\
12.0780555555556	0.0766866621708119\\
12.0836111111111	0.0782647734004501\\
12.0891666666667	0.0802076807682366\\
12.0947222222222	0.078792160560467\\
12.1002777777778	0.078792160560467\\
12.1058333333333	0.0793843542313104\\
12.1113888888889	0.0883963837595779\\
12.1169444444444	0.0877787508453512\\
12.1225	0.0866002917267567\\
12.1280555555556	0.0866002917267567\\
12.1336111111111	0.0866002917267567\\
12.1391666666667	0.0879887433802656\\
12.1447222222222	0.0897721741925552\\
12.1502777777778	0.0897721741925552\\
12.1558333333333	0.0886672315933114\\
12.1613888888889	0.0886672315933114\\
12.1669444444444	0.0886672315933114\\
12.1725	0.0886672315933114\\
12.1780555555556	0.0886672315933114\\
12.1836111111111	0.0886672315933114\\
12.1891666666667	0.0905969980866124\\
12.1947222222222	0.0905969980866124\\
12.2002777777778	0.0905969980866124\\
12.2058333333333	0.0905969980866124\\
12.2113888888889	0.0905969980866124\\
12.2169444444444	0.0905969980866124\\
12.2225	0.0905969980866124\\
12.2280555555556	0.089458269963933\\
12.2336111111111	0.089458269963933\\
12.2391666666667	0.089458269963933\\
12.2447222222222	0.0932553414709474\\
12.2502777777778	0.0932586390391377\\
12.2558333333333	0.08985094454186\\
12.2613888888889	0.08985094454186\\
12.2669444444444	0.0888316432319347\\
12.2725	0.0888316432319347\\
12.2780555555556	0.0888316432319347\\
12.2836111111111	0.0896029720697787\\
12.2891666666667	0.0896029720697787\\
12.2947222222222	0.0896029720697787\\
12.3002777777778	0.0896029720697787\\
12.3058333333333	0.0909404969284925\\
12.3113888888889	0.0909404969284925\\
12.3169444444444	0.0909404969284925\\
12.3225	0.0912798840941867\\
12.3280555555556	0.0912798840941867\\
12.3336111111111	0.0930581976445428\\
12.3391666666667	0.0916295310563031\\
12.3447222222222	0.0916295310563031\\
12.3502777777778	0.0906723239005567\\
12.3558333333333	0.0906723239005567\\
12.3613888888889	0.0906723239005567\\
12.3669444444444	0.0906723239005567\\
12.3725	0.0906723239005567\\
12.3780555555556	0.0906723239005567\\
12.3836111111111	0.0906723239005567\\
12.3891666666667	0.0906723239005567\\
12.3947222222222	0.0906723239005567\\
12.4002777777778	0.0906723239005567\\
12.4058333333333	0.0906723239005567\\
12.4113888888889	0.0906723239005567\\
12.4169444444444	0.0906723239005567\\
12.4225	0.0900958005856957\\
12.4280555555556	0.0900958005856957\\
12.4336111111111	0.0902256721890746\\
12.4391666666667	0.0905217450426232\\
12.4447222222222	0.092355122118565\\
12.4502777777778	0.092355122118565\\
12.4558333333333	0.092355122118565\\
12.4613888888889	0.092355122118565\\
12.4669444444444	0.0929472583803277\\
12.4725	0.0969965449617303\\
12.4780555555556	0.0969965449617303\\
12.4836111111111	0.0969965449617303\\
12.4891666666667	0.0962095523548458\\
12.4947222222222	0.0962095523548458\\
12.5002777777778	0.0956169215461064\\
12.5058333333333	0.0956169215461064\\
12.5113888888889	0.0956169215461064\\
12.5169444444444	0.0956169215461064\\
12.5225	0.0973963746824071\\
12.5280555555556	0.0991789728690801\\
12.5336111111111	0.100025907088804\\
12.5391666666667	0.101294587243652\\
12.5447222222222	0.101294587243652\\
12.5502777777778	0.101294587243652\\
12.5558333333333	0.100112437384709\\
12.5613888888889	0.100112437384709\\
12.5669444444444	0.100112437384709\\
12.5725	0.100112437384709\\
12.5780555555556	0.100112437384709\\
12.5836111111111	0.100112437384709\\
12.5891666666667	0.100112437384709\\
12.5947222222222	0.100112437384709\\
12.6002777777778	0.100112437384709\\
12.6058333333333	0.100112437384709\\
12.6113888888889	0.101885593695406\\
12.6169444444444	0.101730936400427\\
12.6225	0.10077372924468\\
12.6280555555556	0.10255716005697\\
12.6336111111111	0.10255716005697\\
12.6391666666667	0.10255716005697\\
12.6447222222222	0.10255716005697\\
12.6502777777778	0.101078889380733\\
12.6558333333333	0.101078889380733\\
12.6613888888889	0.101078889380733\\
12.6669444444444	0.103409767702938\\
12.6725	0.105780322887882\\
12.6780555555556	0.105780322887882\\
12.6836111111111	0.103466591415952\\
12.6891666666667	0.103466591415952\\
12.6947222222222	0.103466591415952\\
12.7002777777778	0.101917142625618\\
12.7058333333333	0.101917142625618\\
12.7113888888889	0.101917142625618\\
12.7169444444444	0.101917142625618\\
12.7225	0.101917142625618\\
12.7280555555556	0.101917142625618\\
12.7336111111111	0.101917142625618\\
12.7391666666667	0.101917142625618\\
12.7447222222222	0.101917142625618\\
12.7502777777778	0.101917142625618\\
12.7558333333333	0.101917142625618\\
12.7613888888889	0.101917142625618\\
12.7669444444444	0.101917142625618\\
12.7725	0.101917142625618\\
12.7780555555556	0.100824107716271\\
12.7836111111111	0.100824107716271\\
12.7891666666667	0.100824107716271\\
12.7947222222222	0.101595436554115\\
12.8002777777778	0.101595436554115\\
12.8058333333333	0.101595436554115\\
12.8113888888889	0.101595436554115\\
12.8169444444444	0.101595436554115\\
12.8225	0.102186062483263\\
12.8280555555556	0.102186062483263\\
12.8336111111111	0.102186062483263\\
12.8391666666667	0.102186062483263\\
12.8447222222222	0.10343816418248\\
12.8502777777778	0.104920356915543\\
12.8558333333333	0.104920356915543\\
12.8613888888889	0.105219775301026\\
12.8669444444444	0.105219775301026\\
12.8725	0.105219775301026\\
12.8780555555556	0.105631333375949\\
12.8836111111111	0.103790055556251\\
12.8891666666667	0.103790055556251\\
12.8947222222222	0.103148518766586\\
12.9002777777778	0.104333002035761\\
12.9058333333333	0.10551907395153\\
12.9113888888889	0.105519627871169\\
12.9169444444444	0.105528395898646\\
12.9225	0.105528395898646\\
12.9280555555556	0.105528395898646\\
12.9336111111111	0.104640033446766\\
12.9391666666667	0.103893134517199\\
12.9447222222222	0.103893134517199\\
12.9502777777778	0.103766373392564\\
12.9558333333333	0.103475895485072\\
12.9613888888889	0.103898162645533\\
12.9669444444444	0.103898162645533\\
12.9725	0.104200361663139\\
12.9780555555556	0.104503188388961\\
12.9836111111111	0.105389062042748\\
12.9891666666667	0.105389062042748\\
12.9947222222222	0.105389062042748\\
13.0002777777778	0.104796820408161\\
13.0058333333333	0.104796820408161\\
13.0113888888889	0.104796820408161\\
13.0169444444444	0.104796820408161\\
13.0225	0.104796820408161\\
13.0280555555556	0.104305948618571\\
13.0336111111111	0.104199520108075\\
13.0391666666667	0.104199520108075\\
13.0447222222222	0.104199520108075\\
13.0502777777778	0.105255479859885\\
13.0558333333333	0.105255479859885\\
13.0613888888889	0.105060728887588\\
13.0669444444444	0.105060728887588\\
13.0725	0.105060728887588\\
13.0780555555556	0.106307540412698\\
13.0836111111111	0.106307540412698\\
13.0891666666667	0.106307540412698\\
13.0947222222222	0.10710375043162\\
13.1002777777778	0.10710375043162\\
13.1058333333333	0.107350576093835\\
13.1113888888889	0.107350576093835\\
13.1169444444444	0.107152536501657\\
13.1225	0.108042799420442\\
13.1280555555556	0.108983487950944\\
13.1336111111111	0.108983487950944\\
13.1391666666667	0.107692576171274\\
13.1447222222222	0.107692576171274\\
13.1502777777778	0.108129008460186\\
13.1558333333333	0.108890609787431\\
13.1613888888889	0.107958084966024\\
13.1669444444444	0.107659890134022\\
13.1725	0.107659890134022\\
13.1780555555556	0.107659890134022\\
13.1836111111111	0.107659890134022\\
13.1891666666667	0.107659890134022\\
13.1947222222222	0.107659890134022\\
13.2002777777778	0.107659890134022\\
13.2058333333333	0.107659890134022\\
13.2113888888889	0.109291386968364\\
13.2169444444444	0.108249098583408\\
13.2225	0.108249098583408\\
13.2280555555556	0.110176716235043\\
13.2336111111111	0.112358698210581\\
13.2391666666667	0.112309403055504\\
13.2447222222222	0.114925195256068\\
13.2502777777778	0.114925195256068\\
13.2558333333333	0.112525332076245\\
13.2613888888889	0.115874109426301\\
13.2669444444444	0.115874109426301\\
13.2725	0.115874109426301\\
13.2780555555556	0.119386131001666\\
13.2836111111111	0.112644350905843\\
13.2891666666667	0.113612385678269\\
13.2947222222222	0.113612385678269\\
13.3002777777778	0.113612385678269\\
13.3058333333333	0.114203401039033\\
13.3113888888889	0.123809059240604\\
13.3169444444444	0.109049631454977\\
13.3225	0.109049631454977\\
13.3280555555556	0.123992225045843\\
13.3336111111111	0.109367737785436\\
13.3391666666667	0.109658487019832\\
13.3447222222222	0.11925333301821\\
13.3502777777778	0.11925333301821\\
13.3558333333333	0.11925333301821\\
13.3613888888889	0.11925333301821\\
13.3669444444444	0.11925333301821\\
13.3725	0.11925333301821\\
13.3780555555556	0.105597376018202\\
13.3836111111111	0.120126290251189\\
13.3891666666667	0.120126290251189\\
13.3947222222222	0.12496559317288\\
13.4002777777778	0.151797006450967\\
13.4058333333333	0.174842595252089\\
13.4113888888889	0.192821634812083\\
13.4169444444444	0.199611372125185\\
13.4225	0.199611372125185\\
13.4280555555556	0.19450517893564\\
13.4336111111111	0.189924666162603\\
13.4391666666667	0.189924666162603\\
13.4447222222222	0.189924666162603\\
13.4502777777778	0.189924666162603\\
13.4558333333333	0.199746540794264\\
13.4613888888889	0.199746540794264\\
13.4669444444444	0.199951616652405\\
13.4725	0.133899626399702\\
13.4780555555556	0.133899626399702\\
13.4836111111111	0.133899626399702\\
13.4891666666667	0.138738929321393\\
13.4947222222222	0.139199240679409\\
13.5002777777778	0.135554903792815\\
13.5058333333333	0.140098085897213\\
13.5113888888889	0.134042210392129\\
13.5169444444444	0.134042210392129\\
13.5225	0.130556379014651\\
13.5280555555556	0.130556379014651\\
13.5336111111111	0.13095201851241\\
13.5391666666667	0.13095201851241\\
13.5447222222222	0.13095201851241\\
13.5502777777778	0.128429126040987\\
13.5558333333333	0.128425323854018\\
13.5613888888889	0.128425323854018\\
13.5669444444444	0.128717642484342\\
13.5725	0.131141844089107\\
13.5780555555556	0.131141844089107\\
13.5836111111111	0.131141844089107\\
13.5891666666667	0.131141844089107\\
13.5947222222222	0.131141844089107\\
13.6002777777778	0.129507314069564\\
13.6058333333333	0.129507314069564\\
13.6113888888889	0.129799632699888\\
13.6169444444444	0.129799632699888\\
13.6225	0.129799632699888\\
13.6280555555556	0.130096307436821\\
13.6336111111111	0.130096307436821\\
13.6391666666667	0.134592390604681\\
13.6447222222222	0.134592390604681\\
13.6502777777778	0.134592390604681\\
13.6558333333333	0.134592390604681\\
13.6613888888889	0.134592390604681\\
13.6669444444444	0.134592390604681\\
13.6725	0.134592390604681\\
13.6780555555556	0.134592390604681\\
13.6836111111111	0.134592390604681\\
13.6891666666667	0.132822371609449\\
13.6947222222222	0.13105343706591\\
13.7002777777778	0.133143818841918\\
13.7058333333333	0.133143818841918\\
13.7113888888889	0.133143818841918\\
13.7169444444444	0.135358128085287\\
13.7225	0.139852266423311\\
13.7280555555556	0.138192193892809\\
13.7336111111111	0.133454260816114\\
13.7391666666667	0.135964690876776\\
13.7447222222222	0.13858048307734\\
13.7502777777778	0.133208880133901\\
13.7558333333333	0.133208880133901\\
13.7613888888889	0.133208880133901\\
13.7669444444444	0.133208880133901\\
13.7725	0.133208880133901\\
13.7780555555556	0.131130662829948\\
13.7836111111111	0.131130662829948\\
13.7891666666667	0.131724493111131\\
13.7947222222222	0.131736558706798\\
13.8002777777778	0.133565221108786\\
13.8058333333333	0.135569222125974\\
13.8113888888889	0.135569222125974\\
13.8169444444444	0.135569222125974\\
13.8225	0.135569222125974\\
13.8280555555556	0.135569222125974\\
13.8336111111111	0.137648091997607\\
13.8391666666667	0.137648091997607\\
13.8447222222222	0.137648091997607\\
13.8502777777778	0.137648091997607\\
13.8558333333333	0.13587744529416\\
13.8613888888889	0.13587744529416\\
13.8669444444444	0.134107426298928\\
13.8725	0.136476392837275\\
13.8780555555556	0.136476392837275\\
13.8836111111111	0.136476392837275\\
13.8891666666667	0.135602526654144\\
13.8947222222222	0.135602526654144\\
13.9002777777778	0.135602526654144\\
13.9058333333333	0.135602526654144\\
13.9113888888889	0.135602526654144\\
13.9169444444444	0.137675372375198\\
13.9225	0.139543545638786\\
13.9280555555556	0.141421640387543\\
13.9336111111111	0.141421640387543\\
13.9391666666667	0.143537940171428\\
13.9447222222222	0.143537940171428\\
13.9502777777778	0.143537940171428\\
13.9558333333333	0.143537940171428\\
13.9613888888889	0.143537940171428\\
13.9669444444444	0.135440538106976\\
13.9725	0.133224613158168\\
13.9780555555556	0.133224613158168\\
13.9836111111111	0.137803628923891\\
13.9891666666667	0.137803628923891\\
13.9947222222222	0.140273965307235\\
14.0002777777778	0.145918601724636\\
14.0058333333333	0.145918601724636\\
14.0113888888889	0.14355080107316\\
14.0169444444444	0.148678880390524\\
14.0225	0.151179661619718\\
14.0280555555556	0.156711439355605\\
14.0336111111111	0.159511306378516\\
14.0391666666667	0.159511306378516\\
14.0447222222222	0.162311790751565\\
14.0502777777778	0.17085224519403\\
14.0558333333333	0.176460863612373\\
14.0613888888889	0.17926613767482\\
14.0669444444444	0.17926613767482\\
14.0725	0.17926613767482\\
14.0780555555556	0.17926613767482\\
14.0836111111111	0.17926613767482\\
14.0891666666667	0.17926613767482\\
14.0947222222222	0.17926613767482\\
14.1002777777778	0.17926613767482\\
14.1058333333333	0.17926613767482\\
14.1113888888889	0.17926613767482\\
14.1169444444444	0.17926613767482\\
14.1225	0.13188680690787\\
14.1280555555556	0.13188680690787\\
14.1336111111111	0.134854039231384\\
14.1391666666667	0.129181497807394\\
14.1447222222222	0.132369982093756\\
14.1502777777778	0.132369982093756\\
14.1558333333333	0.141644817036567\\
14.1613888888889	0.150917810722099\\
14.1669444444444	0.160203762716599\\
14.1725	0.162868850072239\\
14.1780555555556	0.16567000193353\\
14.1836111111111	0.16567000193353\\
14.1891666666667	0.16567000193353\\
14.1947222222222	0.168471899825501\\
14.2002777777778	0.171341976228772\\
14.2058333333333	0.171341976228772\\
14.2113888888889	0.171341976228772\\
14.2169444444444	0.171341976228772\\
14.2225	0.174078355599488\\
14.2280555555556	0.17665466522698\\
14.2336111111111	0.179597129854155\\
14.2391666666667	0.179597129854155\\
14.2447222222222	0.179597129854155\\
14.2502777777778	0.179597129854155\\
14.2558333333333	0.179597129854155\\
14.2613888888889	0.179597129854155\\
14.2669444444444	0.179597129854155\\
14.2725	0.179597129854155\\
14.2780555555556	0.179597129854155\\
14.2836111111111	0.12925659091427\\
14.2891666666667	0.129138451055326\\
14.2947222222222	0.136127210811426\\
14.3002777777778	0.1431574586385\\
14.3058333333333	0.146678692323709\\
14.3113888888889	0.150202224014358\\
14.3169444444444	0.16078801383563\\
14.3225	0.16078801383563\\
14.3280555555556	0.171382532244277\\
14.3336111111111	0.185517411857769\\
14.3391666666667	0.185517411857769\\
14.3447222222222	0.18905211811953\\
14.3502777777778	0.18905211811953\\
14.3558333333333	0.18905211811953\\
14.3613888888889	0.128643471391673\\
14.3669444444444	0.128643471391673\\
14.3725	0.132910352978768\\
14.3780555555556	0.129128400783996\\
14.3836111111111	0.125872660440364\\
14.3891666666667	0.125872660440364\\
14.3947222222222	0.125872660440364\\
14.4002777777778	0.129478047746113\\
14.4058333333333	0.126225604970671\\
14.4113888888889	0.126225604970671\\
14.4169444444444	0.12983099227642\\
14.4225	0.12983099227642\\
14.4280555555556	0.12983099227642\\
14.4336111111111	0.12983099227642\\
14.4391666666667	0.1261323845633\\
14.4447222222222	0.126437273408071\\
14.4502777777778	0.126437273408071\\
14.4558333333333	0.127621784721553\\
14.4613888888889	0.127921203107037\\
14.4669444444444	0.131477950482748\\
14.4725	0.131477950482748\\
14.4780555555556	0.131477950482748\\
14.4836111111111	0.131477950482748\\
14.4891666666667	0.131477950482748\\
14.4947222222222	0.131477950482748\\
14.5002777777778	0.131477950482748\\
14.5058333333333	0.129548888292508\\
14.5113888888889	0.129851604246182\\
14.5169444444444	0.130263354246191\\
14.5225	0.130263354246191\\
14.5280555555556	0.130263354246191\\
14.5336111111111	0.130263354246191\\
14.5391666666667	0.130263354246191\\
14.5447222222222	0.130862191017159\\
14.5502777777778	0.13324869361046\\
14.5558333333333	0.13324869361046\\
14.5613888888889	0.13324869361046\\
14.5669444444444	0.132012836273878\\
14.5725	0.132617143504132\\
14.5780555555556	0.137109013612899\\
14.5836111111111	0.135329799911045\\
14.5891666666667	0.135329799911045\\
14.5947222222222	0.137701099859623\\
14.6002777777778	0.137701099859623\\
14.6058333333333	0.143044210067733\\
14.6113888888889	0.143044210067733\\
14.6169444444444	0.143044210067733\\
14.6225	0.140573381244599\\
14.6280555555556	0.140573381244599\\
14.6336111111111	0.138056463319801\\
14.6391666666667	0.138056463319801\\
14.6447222222222	0.138056463319801\\
14.6502777777778	0.135936490063845\\
14.6558333333333	0.133865232989387\\
14.6613888888889	0.13416465137487\\
14.6669444444444	0.13416465137487\\
14.6725	0.134173419402347\\
14.6780555555556	0.136542385940695\\
14.6836111111111	0.135188963620343\\
14.6891666666667	0.135188963620343\\
14.6947222222222	0.139238381140955\\
14.7002777777778	0.139238381140955\\
14.7058333333333	0.139238381140955\\
14.7113888888889	0.139238381140955\\
14.7169444444444	0.139238381140955\\
14.7225	0.136731563680472\\
14.7280555555556	0.134504792026342\\
14.7336111111111	0.134504792026342\\
14.7391666666667	0.14100254280116\\
14.7447222222222	0.137449092993638\\
14.7502777777778	0.134630079704922\\
14.7558333333333	0.138166622306805\\
14.7613888888889	0.134755367383501\\
14.7669444444444	0.134758664951692\\
14.7725	0.134758664951692\\
14.7780555555556	0.134758664951692\\
14.7836111111111	0.138166886689794\\
14.7891666666667	0.138166886689794\\
14.7947222222222	0.135351170969267\\
14.8002777777778	0.135351170969267\\
14.8058333333333	0.135351170969267\\
14.8113888888889	0.135743512898836\\
14.8169444444444	0.139239881848569\\
14.8225	0.141358482716117\\
14.8280555555556	0.145193263191686\\
14.8336111111111	0.142644828209305\\
14.8391666666667	0.145785123585497\\
14.8447222222222	0.140750832961835\\
14.8502777777778	0.143718065285348\\
14.8558333333333	0.143718065285348\\
14.8613888888889	0.143718065285348\\
14.8669444444444	0.143718065285348\\
14.8725	0.143718065285348\\
14.8780555555556	0.143718065285348\\
14.8836111111111	0.143718065285348\\
14.8891666666667	0.143718065285348\\
14.8947222222222	0.143718065285348\\
14.9002777777778	0.136659128630599\\
14.9058333333333	0.136659128630599\\
14.9113888888889	0.14110756608816\\
14.9169444444444	0.14342723747143\\
14.9225	0.14342723747143\\
14.9280555555556	0.14342723747143\\
14.9336111111111	0.14131296351822\\
14.9391666666667	0.14131296351822\\
14.9447222222222	0.14131296351822\\
14.9502777777778	0.14131296351822\\
14.9558333333333	0.138825857120929\\
14.9613888888889	0.138830362597414\\
14.9669444444444	0.136768723437416\\
14.9725	0.137161065366985\\
14.9780555555556	0.137161065366985\\
14.9836111111111	0.137460483752468\\
14.9891666666667	0.137757158489401\\
14.9947222222222	0.139715161919702\\
15.0002777777778	0.139715161919702\\
15.0058333333333	0.139715161919702\\
15.0113888888889	0.139715161919702\\
15.0169444444444	0.139715161919702\\
15.0225	0.139715161919702\\
15.0280555555556	0.139715161919702\\
15.0336111111111	0.139715161919702\\
15.0391666666667	0.139646317215836\\
15.0447222222222	0.137428313589183\\
15.0502777777778	0.137428313589183\\
15.0558333333333	0.137428313589183\\
15.0613888888889	0.140093400944824\\
15.0669444444444	0.143473692341246\\
15.0725	0.146230228056714\\
15.0780555555556	0.146230228056714\\
15.0836111111111	0.148895315412356\\
15.0891666666667	0.140128044926088\\
15.0947222222222	0.14042471966302\\
15.1002777777778	0.14714431203075\\
15.1058333333333	0.14714431203075\\
15.1113888888889	0.141265349747711\\
15.1169444444444	0.141265349747711\\
15.1225	0.139145376491755\\
15.1280555555556	0.141218222212809\\
15.1336111111111	0.138500635739368\\
15.1391666666667	0.138500635739368\\
15.1447222222222	0.142037178341251\\
15.1502777777778	0.142037178341251\\
15.1558333333333	0.140709615844324\\
15.1613888888889	0.142235975077285\\
15.1669444444444	0.141119110238155\\
15.1725	0.147730055108595\\
15.1780555555556	0.1522164873177\\
15.1836111111111	0.1522164873177\\
15.1891666666667	0.14866693162959\\
15.1947222222222	0.141560032014546\\
15.2002777777778	0.144452395483614\\
15.2058333333333	0.144452395483614\\
15.2113888888889	0.142349153072781\\
15.2169444444444	0.143832090569479\\
15.2225	0.142386388213608\\
15.2280555555556	0.142386388213608\\
15.2336111111111	0.149094835795001\\
15.2391666666667	0.149094835795001\\
15.2447222222222	0.141032134049951\\
15.2502777777778	0.140963289346085\\
15.2558333333333	0.139617275769896\\
15.2613888888889	0.142429453686442\\
15.2669444444444	0.14257164857066\\
15.2725	0.147603858141472\\
15.2780555555556	0.163619653673509\\
15.2836111111111	0.1640478319742\\
15.2891666666667	0.1640478319742\\
15.2947222222222	0.1640478319742\\
15.3002777777778	0.137076507545129\\
15.3058333333333	0.14124281220484\\
15.3113888888889	0.145043087674578\\
15.3169444444444	0.137275101721082\\
15.3225	0.133776965034005\\
15.3280555555556	0.137644071713774\\
15.3336111111111	0.137934820948169\\
15.3391666666667	0.134921326459067\\
15.3447222222222	0.135234983331314\\
15.3502777777778	0.135234983331314\\
15.3558333333333	0.142945858382823\\
15.3613888888889	0.142945858382823\\
15.3669444444444	0.142945858382823\\
15.3725	0.143629653603179\\
15.3780555555556	0.140379030521479\\
15.3836111111111	0.137363150712947\\
15.3891666666667	0.140920892308397\\
15.3947222222222	0.152327882384689\\
15.4002777777778	0.156472537777146\\
15.4058333333333	0.156472537777146\\
15.4113888888889	0.156472537777146\\
15.4169444444444	0.160088477660287\\
15.4225	0.163707011346506\\
15.4280555555556	0.160746764388019\\
15.4336111111111	0.157786722101957\\
15.4391666666667	0.152653534155721\\
15.4447222222222	0.147928385187868\\
15.4502777777778	0.147928385187868\\
15.4558333333333	0.148590578662729\\
15.4613888888889	0.147108454481193\\
15.4669444444444	0.147413343325963\\
15.4725	0.149538126545246\\
15.4780555555556	0.151838248379727\\
15.4836111111111	0.151838248379727\\
15.4891666666667	0.150059034677872\\
15.4947222222222	0.150059034677872\\
15.5002777777778	0.150745612299871\\
15.5058333333333	0.150745612299871\\
15.5113888888889	0.148970476042706\\
15.5169444444444	0.151342740149244\\
15.5225	0.152819804227128\\
15.5280555555556	0.152819804227128\\
15.5336111111111	0.151292072605759\\
15.5391666666667	0.151292072605759\\
15.5447222222222	0.15291414021454\\
15.5502777777778	0.152787379089905\\
15.5558333333333	0.150291651426922\\
15.5613888888889	0.15075697654901\\
15.5669444444444	0.15075697654901\\
15.5725	0.158592670524834\\
15.5780555555556	0.158592670524834\\
15.5836111111111	0.156808292164216\\
15.5891666666667	0.154939414827497\\
15.5947222222222	0.154939414827497\\
15.6002777777778	0.154939414827497\\
15.6058333333333	0.157259086210768\\
15.6113888888889	0.161920367499494\\
15.6169444444444	0.162578767267453\\
15.6225	0.163468433378136\\
15.6280555555556	0.162313954699102\\
15.6336111111111	0.15876050489158\\
15.6391666666667	0.158819564338011\\
15.6447222222222	0.155754389374401\\
15.6502777777778	0.157089765391449\\
15.6558333333333	0.15790679434137\\
15.6613888888889	0.159093611020775\\
15.6669444444444	0.158404635492715\\
15.6725	0.160177595849371\\
15.6780555555556	0.158570230638269\\
15.6836111111111	0.158570230638269\\
15.6891666666667	0.160278110838163\\
15.6947222222222	0.160278110838163\\
15.7002777777778	0.160340705405093\\
15.7058333333333	0.160340705405093\\
15.7113888888889	0.160340705405093\\
15.7169444444444	0.160340705405093\\
15.7225	0.160335333822195\\
15.7280555555556	0.160335333822195\\
15.7336111111111	0.160782347387362\\
15.7391666666667	0.160782347387362\\
15.7447222222222	0.161829141248804\\
15.7502777777778	0.160535830023612\\
15.7558333333333	0.161933049257854\\
15.7613888888889	0.16713727702609\\
15.7669444444444	0.167729421195022\\
15.7725	0.165803908199743\\
15.7780555555556	0.166318251821486\\
15.7836111111111	0.16716518604121\\
15.7891666666667	0.166910235721576\\
15.7947222222222	0.167935520527753\\
15.8002777777778	0.167935520527753\\
15.8058333333333	0.169194461816557\\
15.8113888888889	0.169486780446881\\
15.8169444444444	0.169779099077205\\
15.8225	0.169779099077205\\
15.8280555555556	0.16956502630392\\
15.8336111111111	0.169638376484271\\
15.8391666666667	0.170658865184909\\
15.8447222222222	0.170658865184909\\
15.8502777777778	0.170658865184909\\
15.8558333333333	0.170658865184909\\
15.8613888888889	0.170809757932783\\
15.8669444444444	0.170672068525051\\
15.8725	0.171491586273065\\
15.8780555555556	0.172330653569869\\
15.8836111111111	0.172330653569869\\
15.8891666666667	0.170512500433951\\
15.8947222222222	0.170512500433951\\
15.9002777777778	0.170062453683582\\
15.9058333333333	0.170367896447992\\
15.9113888888889	0.170367896447992\\
15.9169444444444	0.170982284769016\\
15.9225	0.170697370512778\\
15.9280555555556	0.171413580102545\\
15.9336111111111	0.171367117972545\\
15.9391666666667	0.173039618351133\\
15.9447222222222	0.172529424760554\\
15.9502777777778	0.174108398800963\\
15.9558333333333	0.174108398800963\\
15.9613888888889	0.174108398800963\\
15.9669444444444	0.17366704991425\\
15.9725	0.17366704991425\\
15.9780555555556	0.17366704991425\\
15.9836111111111	0.174829018509867\\
15.9891666666667	0.174829018509867\\
15.9947222222222	0.176509773505735\\
};
\addlegendentry{Dscr Stoch Ctrl};

\addplot [color=dscr_nFPC_plot_color,solid,line width=1.5pt]
  table[row sep=crcr]{%
9.50027777777778	-0.0142137992300856\\
9.50583333333333	-0.0153761969429108\\
9.51138888888889	-0.0290905628591349\\
9.51694444444444	-0.0260964080325325\\
9.5225	-0.0203035170035144\\
9.52805555555556	-0.0192954959882675\\
9.53361111111111	-0.0241575453023504\\
9.53916666666667	-0.0245737665998322\\
9.54472222222222	-0.0222485590679709\\
9.55027777777778	-0.0212359473770928\\
9.55583333333333	-0.0286389678094288\\
9.56138888888889	-0.0234575260954245\\
9.56694444444444	-0.0234575260954245\\
9.5725	-0.0224527305087399\\
9.57805555555555	-0.0224527305087399\\
9.58361111111111	-0.0236257867281079\\
9.58916666666667	-0.0230411985847551\\
9.59472222222222	-0.0271209470504013\\
9.60027777777778	-0.0218699550765727\\
9.60583333333333	-0.0206798480245503\\
9.61138888888889	-0.0295523366250158\\
9.61694444444444	-0.0255144063555439\\
9.6225	-0.0249600334174756\\
9.62805555555556	-0.0245293236452927\\
9.63361111111111	-0.0242808074467895\\
9.63916666666667	-0.0232286034798215\\
9.64472222222222	-0.0229302006781965\\
9.65027777777778	-0.0220469444639478\\
9.65583333333333	-0.0223434126105447\\
9.66138888888889	-0.0251242760451779\\
9.66694444444444	-0.024234162221201\\
9.6725	-0.0229058703526148\\
9.67805555555555	-0.0263684823328728\\
9.68361111111111	-0.0276964965553402\\
9.68916666666667	-0.0300587109285411\\
9.69472222222222	-0.0192853612367702\\
9.70027777777778	-0.020227224776873\\
9.70583333333333	-0.0222117302422664\\
9.71138888888889	-0.0207311261557996\\
9.71694444444444	-0.0195379945731772\\
9.7225	-0.0236843604249808\\
9.72805555555555	-0.0145048166760908\\
9.73361111111111	-0.0116719318488218\\
9.73916666666667	-0.0116719318488218\\
9.74472222222222	-0.0116719318488218\\
9.75027777777778	-0.0100193231824488\\
9.75583333333333	-0.0110855097481426\\
9.76138888888889	-0.0136778813075275\\
9.76694444444444	-0.0171109216687139\\
9.7725	-0.0124571935505693\\
9.77805555555556	-0.0124512970811373\\
9.78361111111111	-0.0124426487676894\\
9.78916666666667	-0.010380127122673\\
9.79472222222222	-0.0106747089328214\\
9.80027777777778	-0.00854176698333869\\
9.80583333333333	-0.00854176698333869\\
9.81138888888889	-0.00953880286560122\\
9.81694444444444	-0.00973412987661639\\
9.8225	-0.0133392789183861\\
9.82805555555555	-0.0108431786624353\\
9.83361111111111	-0.0100976600475517\\
9.83916666666667	-0.00891500364414681\\
9.84472222222222	-0.00773132702906194\\
9.85027777777778	-0.00841695609905393\\
9.85583333333333	-0.00841695609905393\\
9.86138888888889	-0.00820224754450059\\
9.86694444444444	-0.00790644227818264\\
9.8725	-0.00675070067257796\\
9.87805555555556	-0.00703850966293226\\
9.88361111111111	-0.00526114061009051\\
9.88916666666667	-0.00496239371001459\\
9.89472222222222	-0.0052642907567088\\
9.90027777777778	-0.0052642907567088\\
9.90583333333333	-0.00579380264300988\\
9.91138888888889	-0.00527081256904745\\
9.91694444444444	-0.00438427698293577\\
9.9225	-0.00497497961037735\\
9.92805555555555	-0.00505020614120394\\
9.93361111111111	-0.00504189431426504\\
9.93916666666667	-0.0050359978448331\\
9.94472222222222	-0.00414588402085703\\
9.95027777777778	-0.00502835000330548\\
9.95583333333333	-0.00230219371136086\\
9.96138888888889	6.59661728984359e-05\\
9.96694444444444	-0.00567006335417183\\
9.9725	-0.00572356444239335\\
9.97805555555555	-0.00186996046115068\\
9.98361111111111	-0.00186996046115068\\
9.98916666666667	0.000501477092046578\\
9.99472222222222	-0.00157136862900783\\
10.0002777777778	-0.00157136862900783\\
10.0058333333333	-0.00157136862900783\\
10.0113888888889	-0.00157136862900783\\
10.0169444444444	-0.00304969073114281\\
10.0225	-0.00658859575578534\\
10.0280555555556	-0.00573959536319075\\
10.0336111111111	-0.00747135521543832\\
10.0391666666667	-0.00638336257927563\\
10.0447222222222	-0.00814142759445333\\
10.0502777777778	-0.00907726486614728\\
10.0558333333333	-0.00879138193439901\\
10.0613888888889	-0.00879138193439901\\
10.0669444444444	-0.0147894124456029\\
10.0725	-0.0147894124456029\\
10.0780555555556	-0.0153698975433476\\
10.0836111111111	-0.0153698975433476\\
10.0891666666667	-0.0153698975433476\\
10.0947222222222	-0.0153698975433476\\
10.1002777777778	-0.0147769103258223\\
10.1058333333333	-0.012422629508851\\
10.1113888888889	-0.0107742279507548\\
10.1169444444444	-0.0119689598022938\\
10.1225	-0.0119689598022938\\
10.1280555555556	-0.0100244374269044\\
10.1336111111111	-0.0100244374269044\\
10.1391666666667	-0.0100244374269044\\
10.1447222222222	-0.012471211881614\\
10.1502777777778	-0.012471211881614\\
10.1558333333333	-0.012471211881614\\
10.1613888888889	-0.012471211881614\\
10.1669444444444	-0.0132612623772216\\
10.1725	-0.0131220043265578\\
10.1780555555556	-0.0131220043265578\\
10.1836111111111	-0.0125280113198749\\
10.1891666666667	-0.011934018313192\\
10.1947222222222	-0.011934018313192\\
10.2002777777778	-0.011934018313192\\
10.2058333333333	-0.011934018313192\\
10.2113888888889	-0.011934018313192\\
10.2169444444444	-0.011934018313192\\
10.2225	-0.011339494694273\\
10.2280555555556	-0.00985676692225945\\
10.2336111111111	-0.0107443837912014\\
10.2391666666667	-0.0104583473702193\\
10.2447222222222	-0.0104583473702193\\
10.2502777777778	-0.0113383979951608\\
10.2558333333333	-0.0107464456230549\\
10.2613888888889	-0.0083076072338097\\
10.2669444444444	-0.00949209050298329\\
10.2725	-0.00742107164769802\\
10.2780555555556	-0.009493917368752\\
10.2836111111111	-0.00918720274018771\\
10.2891666666667	-0.0127406525477085\\
10.2947222222222	-0.0127406525477085\\
10.3002777777778	-0.012571412340311\\
10.3058333333333	-0.012571412340311\\
10.3113888888889	-0.0138872642763697\\
10.3169444444444	-0.0120020063443886\\
10.3225	-0.0121119061585628\\
10.3280555555556	-0.00732249731389231\\
10.3336111111111	-0.00732249731389231\\
10.3391666666667	-0.00751782432490833\\
10.3447222222222	-0.00763792480509733\\
10.3502777777778	-0.00851207896060732\\
10.3558333333333	-0.00769127584272394\\
10.3613888888889	-0.00769127584272394\\
10.3669444444444	-0.00769127584272394\\
10.3725	-0.00769127584272394\\
10.3780555555556	-0.00769127584272394\\
10.3836111111111	-0.00769127584272394\\
10.3891666666667	-0.00827762100787921\\
10.3947222222222	-0.00827762100787921\\
10.4002777777778	-0.00769156510520526\\
10.4058333333333	-0.00650533046393516\\
10.4113888888889	-0.00887429700228234\\
10.4169444444444	-0.00887429700228234\\
10.4225	-0.00887429700228234\\
10.4280555555556	-0.00769169645435704\\
10.4336111111111	-0.00650725077191536\\
10.4391666666667	-0.00650725077191536\\
10.4447222222222	-0.00706317566843647\\
10.4502777777778	-0.00706317566843647\\
10.4558333333333	-0.00706317566843647\\
10.4613888888889	-0.00706317566843647\\
10.4669444444444	-0.00693792470080506\\
10.4725	-0.00696327616863116\\
10.4780555555556	-0.00770612023867588\\
10.4836111111111	-0.00770612023867588\\
10.4891666666667	-0.00770612023867588\\
10.4947222222222	-0.0073736826833166\\
10.5002777777778	-0.0073736826833166\\
10.5058333333333	-0.0064829222630978\\
10.5113888888889	-0.00648051943183353\\
10.5169444444444	-0.00639137850201736\\
10.5225	-0.00639137850201736\\
10.5280555555556	-0.00620493787899683\\
10.5336111111111	-0.00463013119393866\\
10.5391666666667	-0.00396384052721108\\
10.5447222222222	-0.00262989824137346\\
10.5502777777778	-0.00262989824137346\\
10.5558333333333	-0.00629230571553153\\
10.5613888888889	-0.00629230571553153\\
10.5669444444444	-0.0183021106015747\\
10.5725	-0.0183021106015747\\
10.5780555555556	-0.0170421829375775\\
10.5836111111111	-0.0204227902002728\\
10.5891666666667	-0.0173498208944504\\
10.5947222222222	-0.0156633244637713\\
10.6002777777778	-0.0168500734368413\\
10.6058333333333	-0.0168500734368413\\
10.6113888888889	-0.017815540029884\\
10.6169444444444	-0.017815540029884\\
10.6225	-0.017815540029884\\
10.6280555555556	-0.0162594915485944\\
10.6336111111111	-0.0162594915485944\\
10.6391666666667	-0.0171497367217231\\
10.6447222222222	-0.0171497367217231\\
10.6502777777778	-0.0158197350707122\\
10.6558333333333	-0.0158197350707122\\
10.6613888888889	-0.0143368845416395\\
10.6669444444444	-0.01247316679789\\
10.6725	-0.0139561831491617\\
10.6780555555556	-0.0121770554141367\\
10.6836111111111	-0.00798113852523238\\
10.6891666666667	-0.00590547541762244\\
10.6947222222222	-0.00590547541762244\\
10.7002777777778	-0.00738891183211118\\
10.7058333333333	-0.00560936954179443\\
10.7113888888889	-0.00353346918192913\\
10.7169444444444	-0.00353346918192913\\
10.7225	-0.00353346918192913\\
10.7280555555556	-0.00501715002521968\\
10.7336111111111	-0.00620446562241524\\
10.7391666666667	-0.00472104414939167\\
10.7447222222222	-0.00650326018385218\\
10.7502777777778	-0.00739414924930908\\
10.7558333333333	-0.00739414924930908\\
10.7613888888889	-0.00698618374464226\\
10.7669444444444	-0.00491082028045927\\
10.7725	-0.00304249664737047\\
10.7780555555556	-0.00304249664737047\\
10.7836111111111	-0.00304249664737047\\
10.7891666666667	-0.00304249664737047\\
10.7947222222222	-0.000966833539760536\\
10.8002777777778	0.00741875194304429\\
10.8058333333333	0.0118270387829356\\
10.8113888888889	0.0118270387829356\\
10.8169444444444	0.0102717494822276\\
10.8225	0.0102717494822276\\
10.8280555555556	0.0108637860156155\\
10.8336111111111	0.0108637860156155\\
10.8391666666667	0.0101972902476894\\
10.8447222222222	-0.00598241489643954\\
10.8502777777778	-0.00215819573945\\
10.8558333333333	-0.00807097737893331\\
10.8613888888889	-0.00770140727281371\\
10.8669444444444	-0.00561991323831175\\
10.8725	-0.00120345244673584\\
10.8780555555556	-0.00120345244673584\\
10.8836111111111	-0.00120345244673584\\
10.8891666666667	-0.00120345244673584\\
10.8947222222222	-0.00120345244673584\\
10.9002777777778	-0.00120345244673584\\
10.9058333333333	-0.00120345244673584\\
10.9113888888889	-0.0112394715577492\\
10.9169444444444	-0.0112394715577492\\
10.9225	-0.0127208822983051\\
10.9280555555556	-0.012721229627608\\
10.9336111111111	-0.0118484652285839\\
10.9391666666667	-0.00874630391574263\\
10.9447222222222	-0.0112860813303385\\
10.9502777777778	-0.00846878672713571\\
10.9558333333333	-0.00719040427172507\\
10.9613888888889	-0.00665478726613479\\
10.9669444444444	-0.00672484548084828\\
10.9725	-0.00613648643619224\\
10.9780555555556	-0.0054345895963848\\
10.9836111111111	-0.0041244344522553\\
10.9891666666667	-0.0041244344522553\\
10.9947222222222	-0.00431188068528873\\
11.0002777777778	-0.00286207053681678\\
11.0058333333333	-0.00286207053681678\\
11.0113888888889	-0.0030421785689739\\
11.0169444444444	-0.00209108636880737\\
11.0225	-0.000981829630460366\\
11.0280555555556	-0.000981829630460366\\
11.0336111111111	-0.00158025699696\\
11.0391666666667	-0.000256515677122606\\
11.0447222222222	0.00173497613730264\\
11.0502777777778	0.00173497613730264\\
11.0558333333333	0.00529022779879853\\
11.0613888888889	0.00529022779879853\\
11.0669444444444	0.00529022779879853\\
11.0725	0.00785879096353189\\
11.0780555555556	0.00785879096353189\\
11.0836111111111	0.00637592117316864\\
11.0891666666667	0.00637592117316864\\
11.0947222222222	0.00637592117316864\\
11.1002777777778	0.00637592117316864\\
11.1058333333333	0.00860176729801769\\
11.1113888888889	0.0154191172545937\\
11.1169444444444	0.0201394370824234\\
11.1225	0.0201394370824234\\
11.1280555555556	0.0201394370824234\\
11.1336111111111	0.022440529452734\\
11.1391666666667	0.022440529452734\\
11.1447222222222	0.0246326666936496\\
11.1502777777778	-0.00448849670039961\\
11.1558333333333	-0.00406072245569584\\
11.1613888888889	-0.00406072245569584\\
11.1669444444444	-0.00182007926301323\\
11.1725	0.000674138242966199\\
11.1780555555556	0.000674138242966199\\
11.1836111111111	0.0030449066352885\\
11.1891666666667	0.00126629901027941\\
11.1947222222222	0.00126629901027941\\
11.2002777777778	-0.00221170640241942\\
11.2058333333333	-0.00488704234042602\\
11.2113888888889	-0.00191718585404366\\
11.2169444444444	0.000747901501597543\\
11.2225	0.000747901501597543\\
11.2280555555556	0.000747901501597543\\
11.2336111111111	0.000747901501597543\\
11.2391666666667	0.000784218239662801\\
11.2447222222222	0.00424117741417435\\
11.2502777777778	0.00172569710973976\\
11.2558333333333	-0.00948855138789692\\
11.2613888888889	-0.00918183675933178\\
11.2669444444444	-0.00591619594216405\\
11.2725	-0.00591619594216405\\
11.2780555555556	0.00268465811175524\\
11.2836111111111	0.00268465811175524\\
11.2891666666667	0.00268465811175524\\
11.2947222222222	-0.000427152496636654\\
11.3002777777778	0.00272027667941675\\
11.3058333333333	0.00272027667941675\\
11.3113888888889	0.00272027667941675\\
11.3169444444444	0.00538716588903224\\
11.3225	0.00538716588903224\\
11.3280555555556	0.00538716588903224\\
11.3336111111111	0.0080544996872776\\
11.3391666666667	0.0080544996872776\\
11.3447222222222	0.0080544996872776\\
11.3502777777778	-0.00746968215416511\\
11.3558333333333	-0.00746968215416511\\
11.3613888888889	-0.00983636670818091\\
11.3669444444444	-0.0095344032103177\\
11.3725	-0.00656488321044442\\
11.3780555555556	-0.00922768858175341\\
11.3836111111111	-0.00922768858175341\\
11.3891666666667	-0.0059688205843802\\
11.3947222222222	-0.0059688205843802\\
11.4002777777778	-0.0059688205843802\\
11.4058333333333	-0.0112942422964807\\
11.4113888888889	-0.0112942422964807\\
11.4169444444444	-0.0106967484963289\\
11.4225	-0.000922135442862176\\
11.4280555555556	-0.000922135442862176\\
11.4336111111111	0.00500892921645458\\
11.4391666666667	0.00500892921645458\\
11.4447222222222	0.00500892921645458\\
11.4502777777778	0.00500892921645458\\
11.4558333333333	0.00293604590866783\\
11.4613888888889	0.00115932100490787\\
11.4669444444444	0.00115932100490787\\
11.4725	-2.34108921700438e-05\\
11.4780555555556	-2.34108921700438e-05\\
11.4836111111111	0.00175331401159076\\
11.4891666666667	0.00241217346093718\\
11.4947222222222	0.0033105838145921\\
11.5002777777778	0.0033105838145921\\
11.5058333333333	0.0033105838145921\\
11.5113888888889	-0.00042469042681475\\
11.5169444444444	-0.00161032767937991\\
11.5225	-0.00161032767937991\\
11.5280555555556	-0.00161032767937991\\
11.5336111111111	-0.00161032767937991\\
11.5391666666667	-0.00161032767937991\\
11.5447222222222	-0.000125253580650324\\
11.5502777777778	-0.00131449575815271\\
11.5558333333333	-0.00131449575815271\\
11.5613888888889	0.00538557645579517\\
11.5669444444444	0.0108297587425714\\
11.5725	0.00245176079748638\\
11.5780555555556	0.00245176079748638\\
11.5836111111111	0.00245176079748638\\
11.5891666666667	0.000972908083115279\\
11.5947222222222	0.00275552945630803\\
11.6002777777778	0.00126873963792799\\
11.6058333333333	8.65383530861918e-05\\
11.6113888888889	8.65383530861918e-05\\
11.6169444444444	8.65383530861918e-05\\
11.6225	-0.000812915569023966\\
11.6280555555556	0.000376037712412435\\
11.6336111111111	0.000376037712412435\\
11.6391666666667	0.00251161318175255\\
11.6447222222222	0.000737170262323962\\
11.6502777777778	-0.00103708362658717\\
11.6558333333333	-0.00103708362658717\\
11.6613888888889	0.00145017961524061\\
11.6669444444444	0.00145017961524061\\
11.6725	-0.000346424184942465\\
11.6780555555556	-0.000346424184942465\\
11.6836111111111	-0.000346424184942465\\
11.6891666666667	0.00172716711905\\
11.6947222222222	0.00172716711905\\
11.7002777777778	0.00172716711905\\
11.7058333333333	0.00172716711905\\
11.7113888888889	0.00172716711905\\
11.7169444444444	0.00172716711905\\
11.7225	0.00172716711905\\
11.7280555555556	0.00172716711905\\
11.7336111111111	0.000241804124254008\\
11.7391666666667	0.000241804124254008\\
11.7447222222222	0.000833351100234667\\
11.7502777777778	0.000829815839607196\\
11.7558333333333	0.00112558932759752\\
11.7613888888889	0.00112558932759752\\
11.7669444444444	0.00112558932759752\\
11.7725	0.00112558932759752\\
11.7780555555556	0.00142433622767302\\
11.7836111111111	0.00142433622767302\\
11.7891666666667	0.00172308312774894\\
11.7947222222222	0.00350437460592522\\
11.8002777777778	0.00380277740755061\\
11.8058333333333	0.00380277740755061\\
11.8113888888889	0.00380277740755061\\
11.8169444444444	0.00380277740755061\\
11.8225	0.00380277740755061\\
11.8280555555556	0.00380277740755061\\
11.8336111111111	0.00380277740755061\\
11.8391666666667	0.00350912760510667\\
11.8447222222222	0.00351175368788898\\
11.8502777777778	0.00440011613976896\\
11.8558333333333	0.00410646633732502\\
11.8613888888889	0.00492079128199163\\
11.8669444444444	0.00373154910448925\\
11.8725	0.00522358024076249\\
11.8780555555556	0.00402275333579411\\
11.8836111111111	0.0043122574153923\\
11.8891666666667	0.00431191008608902\\
11.8947222222222	0.00550086336752542\\
11.9002777777778	0.00550086336752542\\
11.9058333333333	0.00550086336752542\\
11.9113888888889	0.00550086336752542\\
11.9169444444444	0.00550086336752542\\
11.9225	0.0046116942615563\\
11.9280555555556	0.0046116942615563\\
11.9336111111111	0.0046116942615563\\
11.9391666666667	0.00491044116163222\\
11.9447222222222	0.00491044116163222\\
11.9502777777778	0.00491044116163222\\
11.9558333333333	0.00491044116163222\\
11.9613888888889	0.00610635148061227\\
11.9669444444444	0.00764968531536602\\
11.9725	0.00764968531536602\\
11.9780555555556	0.00646337518042371\\
11.9836111111111	0.00973491246702297\\
11.9891666666667	0.00973491246702297\\
11.9947222222222	0.00973491246702297\\
12.0002777777778	0.00973491246702297\\
12.0058333333333	0.00765405414828699\\
12.0113888888889	0.00676816271125626\\
12.0169444444444	0.00706690961133217\\
12.0225	0.00706690961133217\\
12.0280555555556	0.00706690961133217\\
12.0336111111111	0.00706690961133217\\
12.0391666666667	0.00854825928073784\\
12.0447222222222	0.0103265231916439\\
12.0502777777778	0.0103265231916439\\
12.0558333333333	0.00884409223940794\\
12.0613888888889	0.00765485006190514\\
12.0669444444444	0.0085419560066371\\
12.0725	0.00884869721964803\\
12.0780555555556	0.00884869721964803\\
12.0836111111111	0.0100339260717601\\
12.0891666666667	0.0115228419851666\\
12.0947222222222	0.0121114978334201\\
12.1002777777778	0.0121114978334201\\
12.1058333333333	0.0137313599705506\\
12.1113888888889	0.0138197333209858\\
12.1169444444444	0.0129901968698105\\
12.1225	0.0123287036745985\\
12.1280555555556	0.0123287036745985\\
12.1336111111111	0.0123287036745985\\
12.1391666666667	0.0145491192732552\\
12.1447222222222	0.0144956181850337\\
12.1502777777778	0.0143005837773695\\
12.1558333333333	0.0145182647269834\\
12.1613888888889	0.0145182647269834\\
12.1669444444444	0.0145182647269834\\
12.1725	0.0145182647269834\\
12.1780555555556	0.0145182647269834\\
12.1836111111111	0.0145182647269834\\
12.1891666666667	0.0160075171268981\\
12.1947222222222	0.0154050269099463\\
12.2002777777778	0.0154050269099463\\
12.2058333333333	0.0154050269099463\\
12.2113888888889	0.0154050269099463\\
12.2169444444444	0.0154050269099463\\
12.2225	0.0154050269099463\\
12.2280555555556	0.0154050269099463\\
12.2336111111111	0.0154050269099463\\
12.2391666666667	0.0154050269099463\\
12.2447222222222	0.0166905965887481\\
12.2502777777778	0.0149138716849881\\
12.2558333333333	0.0137232026839013\\
12.2613888888889	0.0137232026839013\\
12.2669444444444	0.0137232026839013\\
12.2725	0.0137232026839013\\
12.2780555555556	0.0137232026839013\\
12.2836111111111	0.0131291341835458\\
12.2891666666667	0.0128354843811018\\
12.2947222222222	0.0128354843811018\\
12.3002777777778	0.0128354843811018\\
12.3058333333333	0.0137245924159207\\
12.3113888888889	0.0137245924159207\\
12.3169444444444	0.0137245924159207\\
12.3225	0.0140273813746916\\
12.3280555555556	0.0140273813746916\\
12.3336111111111	0.0137939903056837\\
12.3391666666667	0.0129048211997145\\
12.3447222222222	0.0129044738704113\\
12.3502777777778	0.0134989942584771\\
12.3558333333333	0.0134989942584771\\
12.3613888888889	0.0134989942584771\\
12.3669444444444	0.0146842231105892\\
12.3725	0.0155808973894085\\
12.3780555555556	0.0155808973894085\\
12.3836111111111	0.0155808973894085\\
12.3891666666667	0.0164692598412889\\
12.3947222222222	0.0164692598412889\\
12.4002777777778	0.0164692598412889\\
12.4058333333333	0.0164692598412889\\
12.4113888888889	0.0164692598412889\\
12.4169444444444	0.0164692598412889\\
12.4225	0.0170461633553557\\
12.4280555555556	0.0170461633553557\\
12.4336111111111	0.0170461633553557\\
12.4391666666667	0.0176259082074016\\
12.4447222222222	0.0188103914765752\\
12.4502777777778	0.0188103914765752\\
12.4558333333333	0.0188103914765752\\
12.4613888888889	0.0188103914765752\\
12.4669444444444	0.0194025955244297\\
12.4725	0.0213341226859607\\
12.4780555555556	0.0213341226859607\\
12.4836111111111	0.0213341226859607\\
12.4891666666667	0.0223790413812779\\
12.4947222222222	0.0223790413812779\\
12.5002777777778	0.0175181607672678\\
12.5058333333333	0.0186274175056148\\
12.5113888888889	0.0186274175056148\\
12.5169444444444	0.0186274175056148\\
12.5225	0.0186274175056148\\
12.5280555555556	0.02005452050386\\
12.5336111111111	0.0206568100402217\\
12.5391666666667	0.0209892100088483\\
12.5447222222222	0.0209892100088483\\
12.5502777777778	0.0208183401591857\\
12.5558333333333	0.0197684855683028\\
12.5613888888889	0.0197684855683028\\
12.5669444444444	0.0197684855683028\\
12.5725	0.0191700582018032\\
12.5780555555556	0.0191700582018032\\
12.5836111111111	0.0201976787043474\\
12.5891666666667	0.0201976787043474\\
12.5947222222222	0.019537877735764\\
12.6002777777778	0.019537877735764\\
12.6058333333333	0.019537877735764\\
12.6113888888889	0.0208594916082065\\
12.6169444444444	0.0199649434244128\\
12.6225	0.0199649434244128\\
12.6280555555556	0.0184843393379452\\
12.6336111111111	0.0184843393379452\\
12.6391666666667	0.0184843393379452\\
12.6447222222222	0.0184843393379452\\
12.6502777777778	0.020027673172699\\
12.6558333333333	0.020027673172699\\
12.6613888888889	0.020027673172699\\
12.6669444444444	0.0219468524952462\\
12.6725	0.0219468524952462\\
12.6780555555556	0.0219468524952462\\
12.6836111111111	0.0217086293354075\\
12.6891666666667	0.0217086293354075\\
12.6947222222222	0.0217086293354075\\
12.7002777777778	0.0205258974383296\\
12.7058333333333	0.0205258974383296\\
12.7113888888889	0.0205258974383296\\
12.7169444444444	0.0205258974383296\\
12.7225	0.0205258974383296\\
12.7280555555556	0.0205258974383296\\
12.7336111111111	0.0205258974383296\\
12.7391666666667	0.0205258974383296\\
12.7447222222222	0.021849638758167\\
12.7502777777778	0.021849638758167\\
12.7558333333333	0.021849638758167\\
12.7613888888889	0.021849638758167\\
12.7669444444444	0.021849638758167\\
12.7725	0.021849638758167\\
12.7780555555556	0.0231352084369687\\
12.7836111111111	0.0231352084369687\\
12.7891666666667	0.0231352084369687\\
12.7947222222222	0.0243560084442084\\
12.8002777777778	0.0243560084442084\\
12.8058333333333	0.0243560084442084\\
12.8113888888889	0.0243560084442084\\
12.8169444444444	0.0243560084442084\\
12.8225	0.0249465903324562\\
12.8280555555556	0.0249465903324562\\
12.8336111111111	0.0249465903324562\\
12.8391666666667	0.0249465903324562\\
12.8447222222222	0.02583675463831\\
12.8502777777778	0.0269310094344027\\
12.8558333333333	0.0198052516929772\\
12.8613888888889	0.0210524647104377\\
12.8669444444444	0.0210524647104377\\
12.8725	0.0210524647104377\\
12.8780555555556	0.0226860934126623\\
12.8836111111111	0.0220262924440789\\
12.8891666666667	0.0220262924440789\\
12.8947222222222	0.0220262924440789\\
12.9002777777778	0.0220262924440789\\
12.9058333333333	0.0220262924440789\\
12.9113888888889	0.0226829380479827\\
12.9169444444444	0.0225900749004729\\
12.9225	0.0225900749004729\\
12.9280555555556	0.0225900749004729\\
12.9336111111111	0.0225900749004729\\
12.9391666666667	0.0222877683512664\\
12.9447222222222	0.0222877683512664\\
12.9502777777778	0.0222877683512664\\
12.9558333333333	0.0231742428662279\\
12.9613888888889	0.0231757818733735\\
12.9669444444444	0.0231757818733735\\
12.9725	0.023467814995615\\
12.9780555555556	0.0228755733610278\\
12.9836111111111	0.0232344239266079\\
12.9891666666667	0.0232344239266079\\
12.9947222222222	0.0232344239266079\\
13.0002777777778	0.0230543158944508\\
13.0058333333333	0.0230543158944508\\
13.0113888888889	0.0230543158944508\\
13.0169444444444	0.0230543158944508\\
13.0225	0.0230543158944508\\
13.0280555555556	0.0233456778034156\\
13.0336111111111	0.0233456778034156\\
13.0391666666667	0.0233456778034156\\
13.0447222222222	0.0233456778034156\\
13.0502777777778	0.0234128775639642\\
13.0558333333333	0.0234128775639642\\
13.0613888888889	0.0234128775639642\\
13.0669444444444	0.0234128775639642\\
13.0725	0.0234128775639642\\
13.0780555555556	0.0247512278086278\\
13.0836111111111	0.0247512278086278\\
13.0891666666667	0.0247512278086278\\
13.0947222222222	0.0251725995935533\\
13.1002777777778	0.0251725995935533\\
13.1058333333333	0.0251725620068205\\
13.1113888888889	0.0251725620068205\\
13.1169444444444	0.0259574213391957\\
13.1225	0.0268954785825211\\
13.1280555555556	0.0268954785825211\\
13.1336111111111	0.0268954785825211\\
13.1391666666667	0.0277776553024882\\
13.1447222222222	0.0277776553024882\\
13.1502777777778	0.028964420555994\\
13.1558333333333	0.028964420555994\\
13.1613888888889	0.0298524951492513\\
13.1669444444444	0.0273349776712128\\
13.1725	0.0273349776712128\\
13.1780555555556	0.0273349776712128\\
13.1836111111111	0.0273349776712128\\
13.1891666666667	0.0273349776712128\\
13.1947222222222	0.0273349776712128\\
13.2002777777778	0.0273349776712128\\
13.2058333333333	0.0273349776712128\\
13.2113888888889	0.0273349776712128\\
13.2169444444444	0.0258543735847452\\
13.2225	0.0258543735847452\\
13.2280555555556	0.0258543735847452\\
13.2336111111111	0.0282247397114198\\
13.2391666666667	0.0308915784391575\\
13.2447222222222	0.0338550685964241\\
13.2502777777778	0.0338550685964241\\
13.2558333333333	0.0335666709595448\\
13.2613888888889	0.0335666709595448\\
13.2669444444444	0.0335666709595448\\
13.2725	0.0374187125992092\\
13.2780555555556	0.0380162063993611\\
13.2836111111111	0.0387209849685169\\
13.2891666666667	0.0348714143437017\\
13.2947222222222	0.0399105653353224\\
13.3002777777778	0.0399105653353224\\
13.3058333333333	0.035764873893214\\
13.3113888888889	0.0313230616338129\\
13.3169444444444	0.0262949042092556\\
13.3225	0.0215569711325596\\
13.3280555555556	0.0268885454321685\\
13.3336111111111	0.0172418094291155\\
13.3391666666667	0.025805104255399\\
13.3447222222222	0.0360393929170313\\
13.3502777777778	0.0407797970085761\\
13.3558333333333	0.0407797970085761\\
13.3613888888889	0.0407797970085761\\
13.3669444444444	0.0407797970085761\\
13.3725	0.0407797970085761\\
13.3780555555556	0.0360444900146633\\
13.3836111111111	0.0416656793056072\\
13.3891666666667	0.0416656793056072\\
13.3947222222222	0.0416656793056072\\
13.4002777777778	0.0419476590803905\\
13.4058333333333	0.0422367093764287\\
13.4113888888889	0.0425262134560269\\
13.4169444444444	0.0366152241599638\\
13.4225	0.0366152241599638\\
13.4280555555556	0.0310516583796747\\
13.4336111111111	0.025665707311392\\
13.4391666666667	0.025665707311392\\
13.4447222222222	0.025665707311392\\
13.4502777777778	0.025665707311392\\
13.4558333333333	0.036256670534191\\
13.4613888888889	0.036256670534191\\
13.4669444444444	0.0275325100603679\\
13.4725	0.0373021607080963\\
13.4780555555556	0.0373021607080963\\
13.4836111111111	0.0373021607080963\\
13.4891666666667	0.0373021607080963\\
13.4947222222222	0.0378896434343543\\
13.5002777777778	0.0430502945968702\\
13.5058333333333	0.0430502945968702\\
13.5113888888889	0.0392993473123029\\
13.5169444444444	0.0437403529176162\\
13.5225	0.039890782292801\\
13.5280555555556	0.039890782292801\\
13.5336111111111	0.0296673507453388\\
13.5391666666667	0.0296673507453388\\
13.5447222222222	0.0296673507453388\\
13.5502777777778	0.0208695876002725\\
13.5558333333333	0.0159250442894011\\
13.5613888888889	0.0135587255856004\\
13.5669444444444	0.00913892487369177\\
13.5725	0.0345093832751608\\
13.5780555555556	0.0321487285637521\\
13.5836111111111	0.0321487285637521\\
13.5891666666667	0.0321487285637521\\
13.5947222222222	0.0321487285637521\\
13.6002777777778	0.0300758828426981\\
13.6058333333333	0.0300758828426981\\
13.6113888888889	0.0280116854350912\\
13.6169444444444	0.0280116854350912\\
13.6225	0.0280116854350912\\
13.6280555555556	0.0260640906816686\\
13.6336111111111	0.0260640906816686\\
13.6391666666667	0.028690150658833\\
13.6447222222222	0.028690150658833\\
13.6502777777778	0.028690150658833\\
13.6558333333333	0.028690150658833\\
13.6613888888889	0.028690150658833\\
13.6669444444444	0.028690150658833\\
13.6725	0.028690150658833\\
13.6780555555556	0.028690150658833\\
13.6836111111111	0.028690150658833\\
13.6891666666667	0.0265779428784923\\
13.6947222222222	0.0354999099568104\\
13.7002777777778	0.0338377454244208\\
13.7058333333333	0.0338377454244208\\
13.7113888888889	0.0338377454244208\\
13.7169444444444	0.0365010059142933\\
13.7225	0.0365010059142933\\
13.7280555555556	0.0323568534793309\\
13.7336111111111	0.0323568534793309\\
13.7391666666667	0.0353118759203529\\
13.7447222222222	0.0382753660776195\\
13.7502777777778	0.0415377921654795\\
13.7558333333333	0.0418365390655554\\
13.7613888888889	0.0418365390655554\\
13.7669444444444	0.0418365390655554\\
13.7725	0.0418365390655554\\
13.7780555555556	0.0391759217221766\\
13.7836111111111	0.0391759217221766\\
13.7891666666667	0.0365735641148219\\
13.7947222222222	0.0343471258624954\\
13.8002777777778	0.0343471258624954\\
13.8058333333333	0.0367178437729391\\
13.8113888888889	0.0367178437729391\\
13.8169444444444	0.0367178437729391\\
13.8225	0.0367178437729391\\
13.8280555555556	0.0367178437729391\\
13.8336111111111	0.0393136795679555\\
13.8391666666667	0.0393136795679555\\
13.8447222222222	0.0393136795679555\\
13.8502777777778	0.0393136795679555\\
13.8558333333333	0.0349726611148324\\
13.8613888888889	0.0349726611148324\\
13.8669444444444	0.032846314305556\\
13.8725	0.0348797979673242\\
13.8780555555556	0.0348797979673242\\
13.8836111111111	0.0348797979673242\\
13.8891666666667	0.038379746686623\\
13.8947222222222	0.038379746686623\\
13.9002777777778	0.038379746686623\\
13.9058333333333	0.038379746686623\\
13.9113888888889	0.038379746686623\\
13.9169444444444	0.0409207178048765\\
13.9225	0.0412133033615426\\
13.9280555555556	0.0412133033615426\\
13.9336111111111	0.0412133033615426\\
13.9391666666667	0.0412097681009151\\
13.9447222222222	0.0412097681009151\\
13.9502777777778	0.0388415471455056\\
13.9558333333333	0.0388415471455056\\
13.9613888888889	0.0388415471455056\\
13.9669444444444	0.0415015282635151\\
13.9725	0.0388371864908124\\
13.9780555555556	0.0388371864908124\\
13.9836111111111	0.0364765317794041\\
13.9891666666667	0.0364765317794041\\
13.9947222222222	0.0364765317794041\\
14.0002777777778	0.0429813857850969\\
14.0058333333333	0.0429813857850969\\
14.0113888888889	0.0376494072971066\\
14.0169444444444	0.0376494072971066\\
14.0225	0.0435554413311902\\
14.0280555555556	0.0438480268878559\\
14.0336111111111	0.0441406124445216\\
14.0391666666667	0.0441406124445216\\
14.0447222222222	0.0444331980011873\\
14.0502777777778	0.044725783557853\\
14.0558333333333	0.0450183691145188\\
14.0613888888889	0.0456035402278502\\
14.0669444444444	0.0461887113411816\\
14.0725	0.0441234100800025\\
14.0780555555556	0.0473643954344357\\
14.0836111111111	0.0476569809911014\\
14.0891666666667	0.0476569809911014\\
14.0947222222222	0.0476569809911014\\
14.1002777777778	0.0476569809911014\\
14.1058333333333	0.0476569809911014\\
14.1113888888889	0.0476569809911014\\
14.1169444444444	0.0479527544790917\\
14.1225	0.0479527544790917\\
14.1280555555556	0.0479527544790917\\
14.1336111111111	0.0479461377413963\\
14.1391666666667	0.045428789347164\\
14.1447222222222	0.0492834570696108\\
14.1502777777778	0.0492834570696108\\
14.1558333333333	0.0492834570696108\\
14.1613888888889	0.0492834570696108\\
14.1669444444444	0.0492834570696108\\
14.1725	0.0493786586329425\\
14.1780555555556	0.0493786586329425\\
14.1836111111111	0.0493786586329425\\
14.1891666666667	0.0496744321209328\\
14.1947222222222	0.0496744321209328\\
14.2002777777778	0.0499702056089232\\
14.2058333333333	0.0499702056089232\\
14.2113888888889	0.0499702056089232\\
14.2169444444444	0.0499702056089232\\
14.2225	0.0532209178614558\\
14.2280555555556	0.0505554831765117\\
14.2336111111111	0.0505554831765117\\
14.2391666666667	0.0505554831765117\\
14.2447222222222	0.0505554831765117\\
14.2502777777778	0.0505554831765117\\
14.2558333333333	0.0508542300765876\\
14.2613888888889	0.0508542300765876\\
14.2669444444444	0.0511529769766635\\
14.2725	0.0511529769766635\\
14.2780555555556	0.0514517238767394\\
14.2836111111111	0.0514543499595213\\
14.2891666666667	0.0514543499595213\\
14.2947222222222	0.0520518437596732\\
14.3002777777778	0.0520518437596732\\
14.3058333333333	0.052649337559825\\
14.3113888888889	0.0529480844599009\\
14.3169444444444	0.0535455782600528\\
14.3225	0.0541430720602046\\
14.3280555555556	0.0550393127604323\\
14.3336111111111	0.0550393127604323\\
14.3391666666667	0.0550393127604323\\
14.3447222222222	0.0553380596605083\\
14.3502777777778	0.0559355534606601\\
14.3558333333333	0.0559355534606601\\
14.3613888888889	0.056234300360736\\
14.3669444444444	0.0565330472608119\\
14.3725	0.0569364528471099\\
14.3780555555556	0.0490039205284729\\
14.3836111111111	0.0452936079543219\\
14.3891666666667	0.0416866570585779\\
14.3947222222222	0.0416866570585779\\
14.4002777777778	0.0458220999183205\\
14.4058333333333	0.0420304269509602\\
14.4113888888889	0.0420304269509602\\
14.4169444444444	0.0461658698107027\\
14.4225	0.0461658698107027\\
14.4280555555556	0.0420288266820419\\
14.4336111111111	0.0420288266820419\\
14.4391666666667	0.0455527224506431\\
14.4447222222222	0.0382781374306487\\
14.4502777777778	0.0382781374306487\\
14.4558333333333	0.032357304235359\\
14.4613888888889	0.0472085534961583\\
14.4669444444444	0.0530774687538056\\
14.4725	0.0530774687538056\\
14.4780555555556	0.0499961601006826\\
14.4836111111111	0.0499961601006826\\
14.4891666666667	0.0499961601006826\\
14.4947222222222	0.0499961601006826\\
14.5002777777778	0.0499961601006826\\
14.5058333333333	0.0437885576425182\\
14.5113888888889	0.0433420344571993\\
14.5169444444444	0.0410026430779785\\
14.5225	0.0435171279981622\\
14.5280555555556	0.0435171279981622\\
14.5336111111111	0.0435171279981622\\
14.5391666666667	0.0435171279981622\\
14.5447222222222	0.0435171279981622\\
14.5502777777778	0.0520753952266145\\
14.5558333333333	0.0520753952266145\\
14.5613888888889	0.0520753952266145\\
14.5669444444444	0.0461412276811429\\
14.5725	0.0442361794317176\\
14.5780555555556	0.0538030723544636\\
14.5836111111111	0.051445532865922\\
14.5891666666667	0.051752274078933\\
14.5947222222222	0.051752274078933\\
14.6002777777778	0.051752274078933\\
14.6058333333333	0.0549866426956712\\
14.6113888888889	0.0549866426956712\\
14.6169444444444	0.0549866426956712\\
14.6225	0.0520299045349988\\
14.6280555555556	0.0520299045349988\\
14.6336111111111	0.0524467930706218\\
14.6391666666667	0.0524467930706218\\
14.6447222222222	0.0524467930706218\\
14.6502777777778	0.0471825613358012\\
14.6558333333333	0.0471825613358012\\
14.6613888888889	0.0446852714517886\\
14.6669444444444	0.0422769428541548\\
14.6725	0.0511070662847364\\
14.6780555555556	0.051801260987009\\
14.6836111111111	0.0481902567028894\\
14.6891666666667	0.0481902567028894\\
14.6947222222222	0.0532214311787972\\
14.7002777777778	0.0532214311787972\\
14.7058333333333	0.0532214311787972\\
14.7113888888889	0.0532214311787972\\
14.7169444444444	0.0532214311787972\\
14.7225	0.053524493023202\\
14.7280555555556	0.0508594056675608\\
14.7336111111111	0.0508594056675608\\
14.7391666666667	0.0582630702489782\\
14.7447222222222	0.0550064868416882\\
14.7502777777778	0.0517491578514607\\
14.7558333333333	0.0600411848847584\\
14.7613888888889	0.0561916142599432\\
14.7669444444444	0.0527008942007077\\
14.7725	0.0527008942007077\\
14.7780555555556	0.0527008942007077\\
14.7836111111111	0.0532969568267719\\
14.7891666666667	0.0532969568267719\\
14.7947222222222	0.0500396278365443\\
14.8002777777778	0.0500396278365443\\
14.8058333333333	0.0474756268905313\\
14.8113888888889	0.0448468562729554\\
14.8169444444444	0.0409198621345701\\
14.8225	0.0390035726569538\\
14.8280555555556	0.0562460189260852\\
14.8336111111111	0.0597976418678381\\
14.8391666666667	0.0597976418678381\\
14.8447222222222	0.0538943830924445\\
14.8502777777778	0.0573049890629694\\
14.8558333333333	0.0573049890629694\\
14.8613888888889	0.0573049890629694\\
14.8669444444444	0.0573049890629694\\
14.8725	0.0573049890629694\\
14.8780555555556	0.0573049890629694\\
14.8836111111111	0.0573049890629694\\
14.8891666666667	0.0573049890629694\\
14.8947222222222	0.0573049890629694\\
14.9002777777778	0.0517181676603578\\
14.9058333333333	0.0517181676603578\\
14.9113888888889	0.0561423870503954\\
14.9169444444444	0.0517064712604245\\
14.9225	0.0517064712604245\\
14.9280555555556	0.0517064712604245\\
14.9336111111111	0.0498198465424895\\
14.9391666666667	0.0498198465424895\\
14.9447222222222	0.0498198465424895\\
14.9502777777778	0.0498198465424895\\
14.9558333333333	0.0524848963113988\\
14.9613888888889	0.0626146685033625\\
14.9669444444444	0.0578767354266677\\
14.9725	0.0552175445404585\\
14.9780555555556	0.0552175445404585\\
14.9836111111111	0.0552175445404585\\
14.9891666666667	0.0526535435944454\\
14.9947222222222	0.050566308186419\\
15.0002777777778	0.0481991435020461\\
15.0058333333333	0.0481991435020461\\
15.0113888888889	0.0458324234063039\\
15.0169444444444	0.0458324234063039\\
15.0225	0.0458324234063039\\
15.0280555555556	0.0458324234063039\\
15.0336111111111	0.0458324234063039\\
15.0391666666667	0.043465859699221\\
15.0447222222222	0.0602222335399576\\
15.0502777777778	0.0602222335399576\\
15.0558333333333	0.057778040470783\\
15.0613888888889	0.057778040470783\\
15.0669444444444	0.064815264989994\\
15.0725	0.064815264989994\\
15.0780555555556	0.064815264989994\\
15.0836111111111	0.0651055662733801\\
15.0891666666667	0.0598572788808594\\
15.0947222222222	0.0637685738560158\\
15.1002777777778	0.0646812686524612\\
15.1058333333333	0.0646812686524612\\
15.1113888888889	0.0646812686524612\\
15.1169444444444	0.0646812686524612\\
15.1225	0.0620206513090824\\
15.1280555555556	0.0597144145190205\\
15.1336111111111	0.0560303342731504\\
15.1391666666667	0.0560303342731504\\
15.1447222222222	0.058400700399825\\
15.1502777777778	0.0607714183102687\\
15.1558333333333	0.0590005898759398\\
15.1613888888889	0.0590005898759398\\
15.1669444444444	0.0575199857894722\\
15.1725	0.0644048565153094\\
15.1780555555556	0.0672115765637862\\
15.1836111111111	0.0672115765637862\\
15.1891666666667	0.0647655797179032\\
15.1947222222222	0.0647655797179032\\
15.2002777777778	0.0680280058057632\\
15.2058333333333	0.0680280058057632\\
15.2113888888889	0.0675763159949857\\
15.2169444444444	0.0678658200745839\\
15.2225	0.0643139092742071\\
15.2280555555556	0.0643139092742071\\
15.2336111111111	0.0696414504656321\\
15.2391666666667	0.0696414504656321\\
15.2447222222222	0.0696440765484144\\
15.2502777777778	0.0696467026311967\\
15.2558333333333	0.0601869202678306\\
15.2613888888889	0.0555607483240243\\
15.2669444444444	0.0617934457893742\\
15.2725	0.0672742639610957\\
15.2780555555556	0.0708097213848981\\
15.2836111111111	0.0669715778098889\\
15.2891666666667	0.0669715778098889\\
15.2947222222222	0.0669715778098889\\
15.3002777777778	0.0711197402668465\\
15.3058333333333	0.0712589642470859\\
15.3113888888889	0.0715577111471618\\
15.3169444444444	0.0715629633127264\\
15.3225	0.0588957111455703\\
15.3280555555556	0.0628056956031716\\
15.3336111111111	0.0630233765527847\\
15.3391666666667	0.0667086784907404\\
15.3447222222222	0.0664140966805925\\
15.3502777777778	0.0664140966805925\\
15.3558333333333	0.070857573300755\\
15.3613888888889	0.070857573300755\\
15.3669444444444	0.070857573300755\\
15.3725	0.0673127473245192\\
15.3780555555556	0.0635190965566204\\
15.3836111111111	0.0599715432185308\\
15.3891666666667	0.0640496753266426\\
15.3947222222222	0.0729085126061877\\
15.4002777777778	0.0729596192811435\\
15.4058333333333	0.0729596192811435\\
15.4113888888889	0.0729596192811435\\
15.4169444444444	0.0732491233607417\\
15.4225	0.0738312129970056\\
15.4280555555556	0.0738312129970056\\
15.4336111111111	0.0703960598929646\\
15.4391666666667	0.0644179470362238\\
15.4447222222222	0.0644179470362238\\
15.4502777777778	0.0617256227445463\\
15.4558333333333	0.0624442579393636\\
15.4613888888889	0.0621837556897328\\
15.4669444444444	0.0603676687266853\\
15.4725	0.0629937287038496\\
15.4780555555556	0.0629937287038496\\
15.4836111111111	0.0629937287038496\\
15.4891666666667	0.0608109831686215\\
15.4947222222222	0.0608109831686215\\
15.5002777777778	0.0585875350121108\\
15.5058333333333	0.0585875350121108\\
15.5113888888889	0.0626429979811388\\
15.5169444444444	0.0650105977335329\\
15.5225	0.0673278145557551\\
15.5280555555556	0.0673278145557551\\
15.5336111111111	0.0653193728040177\\
15.5391666666667	0.0618108880480093\\
15.5447222222222	0.0587831369593792\\
15.5502777777778	0.0591477881209108\\
15.5558333333333	0.0620243246470798\\
15.5613888888889	0.0677072793115827\\
15.5669444444444	0.0677072793115827\\
15.5725	0.0736328466178754\\
15.5780555555556	0.0736328466178754\\
15.5836111111111	0.0714031381301912\\
15.5891666666667	0.0712081037225261\\
15.5947222222222	0.0712081037225261\\
15.6002777777778	0.0712081037225261\\
15.6058333333333	0.0712081037225261\\
15.6113888888889	0.0712081037225261\\
15.6169444444444	0.0740054900903276\\
15.6225	0.0692637165500594\\
15.6280555555556	0.0707451354610328\\
15.6336111111111	0.0728197325541833\\
15.6391666666667	0.0765846740394875\\
15.6447222222222	0.0733390285419192\\
15.6502777777778	0.0751180354300122\\
15.6558333333333	0.0757241751292547\\
15.6613888888889	0.0757241751292547\\
15.6669444444444	0.0761418175350287\\
15.6725	0.0776205947557276\\
15.6780555555556	0.0793965130053996\\
15.6836111111111	0.0793965130053996\\
15.6891666666667	0.0793965130053996\\
15.6947222222222	0.0793965130053996\\
15.7002777777778	0.078335664328614\\
15.7058333333333	0.07893037697805\\
15.7113888888889	0.07893037697805\\
15.7169444444444	0.07893037697805\\
15.7225	0.0783464471704026\\
15.7280555555556	0.0783464471704026\\
15.7336111111111	0.0801279250733452\\
15.7391666666667	0.0801279250733452\\
15.7447222222222	0.0801279250733452\\
15.7502777777778	0.0801331772389099\\
15.7558333333333	0.0807445470359121\\
15.7613888888889	0.0834173575719676\\
15.7669444444444	0.0822355003855764\\
15.7725	0.0795719520370808\\
15.7780555555556	0.0770186258143284\\
15.7836111111111	0.0782012263622538\\
15.7891666666667	0.0776107865816408\\
15.7947222222222	0.0782094073162901\\
15.8002777777778	0.0786510465154206\\
15.8058333333333	0.079848884747259\\
15.8113888888889	0.0802341464943683\\
15.8169444444444	0.0810128287519752\\
15.8225	0.0810128287519752\\
15.8280555555556	0.0813061172412473\\
15.8336111111111	0.0818983479077033\\
15.8391666666667	0.0823060383069964\\
15.8447222222222	0.0823060383069964\\
15.8502777777778	0.0823060383069964\\
15.8558333333333	0.0823060383069964\\
15.8613888888889	0.0828982423548509\\
15.8669444444444	0.0855687626552473\\
15.8725	0.0855687626552473\\
15.8780555555556	0.0855687626552473\\
15.8836111111111	0.0870544638393465\\
15.8891666666667	0.0870570899221284\\
15.8947222222222	0.0870570899221284\\
15.9002777777778	0.086168727470248\\
15.9058333333333	0.0852134503144795\\
15.9113888888889	0.0852134503144795\\
15.9169444444444	0.0852651246532888\\
15.9225	0.0843785640553835\\
15.9280555555556	0.0840860983223698\\
15.9336111111111	0.0798425776674964\\
15.9391666666667	0.0810633776747361\\
15.9447222222222	0.0813594609052973\\
15.9502777777778	0.0820634636727734\\
15.9558333333333	0.0826579517499648\\
15.9613888888889	0.0826579517499648\\
15.9669444444444	0.0826579517499648\\
15.9725	0.0832525962158154\\
15.9780555555556	0.0839467365128878\\
15.9836111111111	0.079421853148477\\
15.9891666666667	0.0801111199462783\\
15.9947222222222	0.0804716536219057\\
};
\addlegendentry{Dscr Stoch Ctrl w nFPC};

\addplot [color=black,dashed,line width=1.5pt]
  table[row sep=crcr]{%
9.50000138888889	0\\
9.50111194444444	0.00118448326917386\\
9.50449111111111	0.000444181225939921\\
9.50566027777778	-0.000444181225940254\\
9.50606805555556	-0.000444181225940254\\
9.50631	-0.000444181225940254\\
9.50650638888889	-0.000444181225940254\\
9.51661222222222	0.000444181225939921\\
9.52385472222222	0.00192478531240736\\
9.54045444444444	0.00148060408646722\\
9.54540055555556	0.00325732899022801\\
9.55986555555556	0.00488599348534202\\
9.57838833333333	0.00458987266804844\\
9.58957805555556	0.00414569144210808\\
9.60126472222222	0.00222090612970072\\
9.61190555555556	0.00340538939887458\\
9.619735	0.00503405389398859\\
9.62554638888889	0.00488599348534202\\
9.64076333333333	0.00621853716316245\\
9.65760166666667	0.00488599348534202\\
9.66701305555556	0.00621853716316245\\
9.67307194444444	0.00518211430263538\\
9.67669166666667	0.00458987266804844\\
9.6815325	0.00518211430263538\\
9.68680777777778	0.00444181225940188\\
9.69267194444445	0.00488599348534202\\
9.70423222222222	0.00399763103346151\\
9.71677611111111	0.00458987266804844\\
9.72511361111111	0.00458987266804844\\
9.72996777777778	0.00429375185075509\\
9.75026555555556	0.00488599348534202\\
9.76833111111111	0.00533017471128217\\
9.78471444444444	0.00607047675451589\\
9.79037527777778	0.00547823511992873\\
9.81136361111111	0.00562629552857552\\
9.82086055555556	0.0066627183891026\\
9.82731972222222	0.00651465798045603\\
9.84485055555556	0.00577435593722231\\
9.85223111111111	0.00577435593722231\\
9.86587111111111	0.00533017471128217\\
9.88255583333333	0.00577435593722231\\
9.89415277777778	0.00547823511992873\\
9.91960944444445	0.00577435593722231\\
9.94306083333333	0.00547823511992873\\
9.95475	0.00444181225940188\\
9.96290444444444	0.00414569144210808\\
9.98785638888889	0.00414569144210808\\
10.0118361111111	0.00384957062481472\\
10.0204633333333	0.00473793307669523\\
10.0404466666667	0.00547823511992873\\
10.0483366666667	0.00607047675451589\\
10.0616297222222	0.0066627183891026\\
10.0668508333333	0.00695883920639617\\
10.0777827777778	0.0066627183891026\\
10.1034741666667	0.00740302043233632\\
10.1108475	0.00784720165827646\\
10.1550372222222	0.00784720165827646\\
10.16618	0.0066627183891026\\
10.1707808333333	0.0059224163458691\\
10.2189616666667	0.0059224163458691\\
10.2282147222222	0.00518211430263538\\
10.240315	0.00547823511992873\\
10.2543530555556	0.00533017471128217\\
10.2621152777778	0.00488599348534202\\
10.2734305555556	0.00429375185075509\\
10.2854647222222	0.00458987266804844\\
10.3064977777778	0.00488599348534202\\
10.3092838888889	0.00458987266804844\\
10.3233986111111	0.00458987266804844\\
10.3397141666667	0.00488599348534202\\
10.3514469444444	0.00577435593722231\\
10.4036355555556	0.00547823511992873\\
10.4414047222222	0.00533017471128217\\
10.4481336111111	0.00607047675451589\\
10.4776477777778	0.00636659757180924\\
10.5069688888889	0.00651465798045603\\
10.5134291666667	0.00636659757180924\\
10.5302111111111	0.00651465798045603\\
10.5530877777778	0.00710689961504274\\
10.5634966666667	0.00725496002368953\\
10.5904716666667	0.00784720165827646\\
10.63591	0.00725496002368953\\
10.6638466666667	0.00799526206692325\\
10.6850736111111	0.00814332247557004\\
10.7270925	0.00814332247557004\\
10.7423130555556	0.00755108084098288\\
10.7578022222222	0.00784720165827646\\
10.7705291666667	0.00799526206692325\\
10.8023508333333	0.0082913828842166\\
10.8253030555556	0.00769914124962989\\
10.8364372222222	0.00695883920639617\\
10.8500147222222	0.00784720165827646\\
10.8563391666667	0.00784720165827646\\
10.8578588888889	0.00784720165827646\\
10.8637102777778	0.00784720165827646\\
10.9066405555556	0.00843944329286339\\
10.9104805555556	0.00814332247557004\\
10.9281575	0.00725496002368953\\
10.93547	0.00725496002368953\\
10.9427863888889	0.00725496002368953\\
10.9527866666667	0.00681077879774938\\
10.9581725	0.0066627183891026\\
10.9717638888889	0.00636659757180924\\
10.9817988888889	0.00636659757180924\\
10.9891697222222	0.0059224163458691\\
10.9943269444444	0.0059224163458691\\
10.9978458333333	0.00651465798045603\\
11.0110175	0.00695883920639617\\
11.0310494444444	0.00710689961504274\\
11.0401572222222	0.00784720165827646\\
11.0407494444444	0.00814332247557004\\
11.0514430555556	0.00784720165827646\\
11.0756444444444	0.00814332247557004\\
11.1090888888889	0.00843944329286339\\
11.1482363888889	0.0082913828842166\\
11.1502294444444	0.00858750370151018\\
11.1523738888889	0.0082913828842166\\
11.1590180555556	0.00814332247557004\\
11.19621	0.00858750370151018\\
11.2022375	0.00917974533609689\\
11.2217275	0.00903168492745032\\
11.2496625	0.00873556411015675\\
11.2578869444444	0.00932780574474368\\
11.2664038888889	0.00903168492745032\\
11.2764038888889	0.00903168492745032\\
11.2937172222222	0.00903168492745032\\
11.3126269444444	0.00873556411015675\\
11.3320022222222	0.00903168492745032\\
11.3483225	0.00917974533609689\\
11.3697255555556	0.00903168492745032\\
11.4184836111111	0.00917974533609689\\
11.4332438888889	0.00932780574474368\\
11.4529455555556	0.00843944329286339\\
11.4572761111111	0.00755108084098288\\
11.4834966666667	0.00814332247557004\\
11.4896191666667	0.00784720165827646\\
11.507555	0.00755108084098288\\
11.5398627777778	0.00755108084098288\\
11.5581236111111	0.00769914124962989\\
11.5624258333333	0.00769914124962989\\
11.5875011111111	0.00784720165827646\\
11.6045205555556	0.00755108084098288\\
11.6202886111111	0.00725496002368953\\
11.6376652777778	0.00814332247557004\\
11.6450027777778	0.0082913828842166\\
11.6699430555556	0.00843944329286339\\
11.6889369444444	0.00799526206692325\\
11.7334233333333	0.00784720165827646\\
11.7474783333333	0.00755108084098288\\
11.7712441666667	0.00725496002368953\\
11.7946125	0.00710689961504274\\
11.8165530555556	0.00695883920639617\\
11.8568216666667	0.00710689961504274\\
11.8634066666667	0.00784720165827646\\
11.8742052777778	0.00755108084098288\\
11.9017461111111	0.00755108084098288\\
11.9382172222222	0.00725496002368953\\
11.9668552777778	0.00784720165827646\\
11.9831680555556	0.00814332247557004\\
12.0152302777778	0.00755108084098288\\
12.0410402777778	0.00784720165827646\\
12.0659138888889	0.00740302043233632\\
12.0802863888889	0.00755108084098288\\
12.1096377777778	0.00784720165827646\\
12.1141016666667	0.00725496002368953\\
12.1285986111111	0.00695883920639617\\
12.1501658333333	0.00784720165827646\\
12.1707508333333	0.00755108084098288\\
12.2163308333333	0.00725496002368953\\
12.2483086111111	0.00784720165827646\\
12.2638022222222	0.00725496002368953\\
12.3048261111111	0.00695883920639617\\
12.3189922222222	0.00725496002368953\\
12.3421622222222	0.00725496002368953\\
12.4179988888889	0.00695883920639617\\
12.4183472222222	0.00636659757180924\\
12.4358361111111	0.00636659757180924\\
12.4645227777778	0.00695883920639617\\
12.4886102777778	0.00695883920639617\\
12.5043041666667	0.00725496002368953\\
12.5381361111111	0.00725496002368953\\
12.5463455555556	0.00725496002368953\\
12.5686047222222	0.00695883920639617\\
12.6062816666667	0.00710689961504274\\
12.6082305555556	0.00784720165827646\\
12.6273311111111	0.00755108084098288\\
12.6641488888889	0.00784720165827646\\
12.6708919444444	0.00814332247557004\\
12.6928519444444	0.00799526206692325\\
12.7142405555556	0.00725496002368953\\
12.7739005555556	0.00740302043233632\\
12.8030041666667	0.00695883920639617\\
12.8443927777778	0.00725496002368953\\
12.8551388888889	0.00725496002368953\\
12.8969294444444	0.00681077879774938\\
12.9146408333333	0.00725496002368953\\
12.9491175	0.0066627183891026\\
12.9549816666667	0.0059224163458691\\
12.9668175	0.0066627183891026\\
12.9781347222222	0.00636659757180924\\
13.0114336111111	0.00636659757180924\\
13.0486619444444	0.00607047675451589\\
13.0732966666667	0.00621853716316245\\
13.1032852777778	0.00636659757180924\\
13.116925	0.00562629552857552\\
13.1348447222222	0.00562629552857552\\
13.1584116666667	0.00533017471128217\\
13.1660561111111	0.00458987266804844\\
13.2154888888889	0.00444181225940188\\
13.2304272222222	0.00370151021616816\\
13.2503633333333	0.00370151021616816\\
13.2549833333333	0.00281314776428787\\
13.2839944444444	0.00222090612970072\\
13.2933194444444	0.00192478531240736\\
13.3080980555556	0.00148060408646722\\
13.3115980555556	0.00103642286052708\\
13.3331252777778	0.00118448326917386\\
13.3376194444444	0.00162866449511401\\
13.3424838888889	0.00192478531240736\\
13.3746313888889	0.00133254367782043\\
13.398155	0.000740302043233498\\
13.4147963888889	0.000444181225939921\\
13.4325194444444	0.00103642286052708\\
13.462665	0.00148060408646722\\
13.4690611111111	0.00177672490376057\\
13.4701041666667	0.00118448326917386\\
13.4955783333333	0.00148060408646722\\
13.4979116666667	0.00162866449511401\\
13.5115202777778	0.00222090612970072\\
13.5307294444444	0.00281314776428787\\
13.5484705555556	0.00340538939887458\\
13.5777555555556	0.00384957062481472\\
13.6270827777778	0.00399763103346151\\
13.6723563888889	0.00370151021616816\\
13.6992080555556	0.00399763103346151\\
13.7310766666667	0.00384957062481472\\
13.7689858333333	0.00340538939887458\\
13.7937802777778	0.00429375185075509\\
13.8282872222222	0.00370151021616816\\
13.8549211111111	0.00384957062481472\\
13.8717405555556	0.00384957062481472\\
13.9166705555556	0.00399763103346151\\
13.9217688888889	0.00429375185075509\\
13.9476483333333	0.00384957062481472\\
13.9660916666667	0.00340538939887458\\
13.9812091666667	0.00399763103346151\\
13.9997763888889	0.00325732899022801\\
14.0106	0.00370151021616816\\
14.0627572222222	0.00340538939887458\\
14.1176233333333	0.00310926858158123\\
14.1365577777778	0.00296120817293444\\
14.17165	0.00340538939887458\\
14.2231411111111	0.00370151021616816\\
14.2760219444444	0.00340538939887458\\
14.2885908333333	0.00281314776428787\\
14.3411905555556	0.00281314776428787\\
14.368325	0.00192478531240736\\
14.3840313888889	0.00236896653834751\\
14.4053222222222	0.00236896653834751\\
14.4225655555556	0.00192478531240736\\
14.4367030555556	0.00222090612970072\\
14.4520211111111	0.00236896653834751\\
14.4533102777778	0.00236896653834751\\
14.4578727777778	0.00236896653834751\\
14.4599613888889	0.00281314776428787\\
14.5024269444444	0.00281314776428787\\
14.507665	0.00325732899022801\\
14.5165019444444	0.00355344980752137\\
14.5168766666667	0.00355344980752137\\
14.5488716666667	0.00384957062481472\\
14.5726927777778	0.00414569144210808\\
14.5799011111111	0.00399763103346151\\
14.6048725	0.00340538939887458\\
14.6189983333333	0.00340538939887458\\
14.66641	0.00340538939887458\\
14.6784186111111	0.00399763103346151\\
14.6920227777778	0.00384957062481472\\
14.7191602777778	0.00310926858158123\\
14.7349805555556	0.00281314776428787\\
14.7480891666667	0.00266508735564108\\
14.7585238888889	0.00236896653834751\\
14.7803658333333	0.00266508735564108\\
14.8047269444444	0.00370151021616816\\
14.8113877777778	0.00340538939887458\\
14.8155066666667	0.00384957062481472\\
14.8235269444444	0.00384957062481472\\
14.8320288888889	0.00296120817293444\\
14.8568383333333	0.00310926858158123\\
14.8972325	0.00355344980752137\\
14.9078869444444	0.00370151021616816\\
14.9220372222222	0.00384957062481472\\
14.9338605555556	0.00399763103346151\\
14.9539852777778	0.00310926858158123\\
14.9585955555556	0.00370151021616816\\
14.9717413888889	0.00340538939887458\\
14.9892997222222	0.00340538939887458\\
14.9914952777778	0.00384957062481472\\
15.0314038888889	0.00370151021616816\\
15.0412358333333	0.00340538939887458\\
15.0798611111111	0.00340538939887458\\
15.0870244444444	0.00370151021616816\\
15.0985155555556	0.00414569144210808\\
15.1000338888889	0.00399763103346151\\
15.1191172222222	0.00370151021616816\\
15.1300683333333	0.00429375185075509\\
15.1524230555556	0.00429375185075509\\
15.1694680555556	0.00399763103346151\\
15.1745933333333	0.00399763103346151\\
15.1849508333333	0.00384957062481472\\
15.1979472222222	0.00340538939887458\\
15.2080444444444	0.00429375185075509\\
15.2189502777778	0.00429375185075509\\
15.2220713888889	0.00399763103346151\\
15.2415544444444	0.00355344980752137\\
15.2552663888889	0.00355344980752137\\
15.2646333333333	0.00414569144210808\\
15.2699113888889	0.00340538939887458\\
15.2731491666667	0.00281314776428787\\
15.2786141666667	0.00236896653834751\\
15.3051338888889	0.00236896653834751\\
15.316735	0.00192478531240736\\
15.3242872222222	0.00251702694699429\\
15.3341147222222	0.00222090612970072\\
15.3505097222222	0.00266508735564108\\
15.3694041666667	0.00192478531240736\\
15.3756863888889	0.00251702694699429\\
15.3963508333333	0.00222090612970072\\
15.4294527777778	0.00266508735564108\\
15.4454355555556	0.00340538939887458\\
15.4588930555556	0.00399763103346151\\
15.4733547222222	0.00370151021616816\\
15.5135641666667	0.00429375185075509\\
15.5217897222222	0.00429375185075509\\
15.5437788888889	0.00429375185075509\\
15.5500508333333	0.00355344980752137\\
15.5597272222222	0.00355344980752137\\
15.5888180555556	0.00399763103346151\\
15.6159077777778	0.00458987266804844\\
15.6229730555556	0.00458987266804844\\
15.6363211111111	0.00473793307669523\\
15.6427986111111	0.00488599348534202\\
15.6605063888889	0.00518211430263538\\
15.6657394444444	0.00518211430263538\\
15.6722908333333	0.00488599348534202\\
15.6980930555556	0.00503405389398859\\
15.7221111111111	0.00577435593722231\\
15.7326002777778	0.00562629552857552\\
15.7433066666667	0.00562629552857552\\
15.7480433333333	0.00488599348534202\\
15.7532744444444	0.00458987266804844\\
15.7581419444444	0.00458987266804844\\
15.7654947222222	0.00473793307669523\\
15.7747011111111	0.00518211430263538\\
15.7921936111111	0.00577435593722231\\
15.7990469444444	0.00681077879774938\\
15.8197083333333	0.00636659757180924\\
15.8295508333333	0.00577435593722231\\
15.8390375	0.00607047675451589\\
15.86611	0.00547823511992873\\
15.8901183333333	0.00518211430263538\\
15.9027716666667	0.00547823511992873\\
15.9148186111111	0.00533017471128217\\
15.9162177777778	0.00562629552857552\\
15.9256441666667	0.00607047675451589\\
15.9288980555556	0.00695883920639617\\
15.9360080555556	0.00636659757180924\\
15.9495427777778	0.00636659757180924\\
15.9634108333333	0.0066627183891026\\
15.9772219444444	0.0066627183891026\\
15.9899863888889	0.00681077879774938\\
15.9945927777778	0.00636659757180924\\
15.9986611111111	0.00607047675451589\\
};
\addlegendentry{Midprice};

\end{axis}
\end{tikzpicture}%

  \caption{Performance comparison of the four stochastic control methods.}
  \label{fig:ORCL_comp4stoch}
\end{subfigure}\\
\vspace{1cm}
\begin{subfigure}{\linewidth}
  \centering
  \setlength\figureheight{0.5\linewidth} 
  \setlength\figurewidth{\linewidth}
  \tikzsetnextfilename{ORCL_comp4stoch_inv}
  % This file was created by matlab2tikz.
%
%The latest updates can be retrieved from
%  http://www.mathworks.com/matlabcentral/fileexchange/22022-matlab2tikz-matlab2tikz
%where you can also make suggestions and rate matlab2tikz.
%
%
\begin{tikzpicture}[trim axis left, trim axis right]

\begin{axis}[%
width=\figurewidth,
height=\figureheight,
at={(0\figurewidth,0\figureheight)},
scale only axis,
every outer x axis line/.append style={black},
every x tick label/.append style={font=\color{black}},
xmin=9.5,
xmax=16,
xlabel={Time (h)},
every outer y axis line/.append style={black},
every y tick label/.append style={font=\color{black}},
ymin=-20,
ymax=20,
ylabel={Inventory},
title={Strategy Inventory using Optimal Stochastic Control},
axis background/.style={fill=white},
axis x line*=bottom,
axis y line*=left,
yticklabel style={
        /pgf/number format/fixed,
        /pgf/number format/precision=3
},
scaled y ticks=false,
legend style={legend cell align=left,align=left,draw=black,font=\small, legend pos=north west},
]
\addplot [color=cts_plot_color,solid,line width=1.5pt]
  table[row sep=crcr]{%
9.50027777777778	2\\
9.50583333333333	4\\
9.51138888888889	9\\
9.51694444444444	1\\
9.5225	-2\\
9.52805555555556	-1\\
9.53361111111111	-4\\
9.53916666666667	-2\\
9.54472222222222	-5\\
9.55027777777778	-3\\
9.55583333333333	-4\\
9.56138888888889	2\\
9.56694444444444	3\\
9.5725	7\\
9.57805555555555	7\\
9.58361111111111	2\\
9.58916666666667	8\\
9.59472222222222	4\\
9.60027777777778	6\\
9.60583333333333	2\\
9.61138888888889	2\\
9.61694444444444	2\\
9.6225	3\\
9.62805555555556	5\\
9.63361111111111	2\\
9.63916666666667	4\\
9.64472222222222	3\\
9.65027777777778	4\\
9.65583333333333	5\\
9.66138888888889	1\\
9.66694444444444	1\\
9.6725	4\\
9.67805555555555	2\\
9.68361111111111	4\\
9.68916666666667	3\\
9.69472222222222	2\\
9.70027777777778	2\\
9.70583333333333	3\\
9.71138888888889	3\\
9.71694444444444	1\\
9.7225	3\\
9.72805555555555	4\\
9.73361111111111	2\\
9.73916666666667	2\\
9.74472222222222	2\\
9.75027777777778	3\\
9.75583333333333	4\\
9.76138888888889	3\\
9.76694444444444	2\\
9.7725	-3\\
9.77805555555556	-4\\
9.78361111111111	-5\\
9.78916666666667	-2\\
9.79472222222222	-3\\
9.80027777777778	-2\\
9.80583333333333	-2\\
9.81138888888889	-5\\
9.81694444444444	-7\\
9.8225	-10\\
9.82805555555555	-1\\
9.83361111111111	2\\
9.83916666666667	3\\
9.84472222222222	4\\
9.85027777777778	2\\
9.85583333333333	4\\
9.86138888888889	4\\
9.86694444444444	3\\
9.8725	4\\
9.87805555555556	3\\
9.88361111111111	4\\
9.88916666666667	4\\
9.89472222222222	5\\
9.90027777777778	5\\
9.90583333333333	4\\
9.91138888888889	4\\
9.91694444444444	4\\
9.9225	4\\
9.92805555555555	4\\
9.93361111111111	4\\
9.93916666666667	5\\
9.94472222222222	6\\
9.95027777777778	5\\
9.95583333333333	7\\
9.96138888888889	9\\
9.96694444444444	3\\
9.9725	2\\
9.97805555555555	3\\
9.98361111111111	3\\
9.98916666666667	5\\
9.99472222222222	5\\
10.0002777777778	5\\
10.0058333333333	5\\
10.0113888888889	5\\
10.0169444444444	3\\
10.0225	2\\
10.0280555555556	2\\
10.0336111111111	2\\
10.0391666666667	2\\
10.0447222222222	2\\
10.0502777777778	2\\
10.0558333333333	3\\
10.0613888888889	3\\
10.0669444444444	5\\
10.0725	5\\
10.0780555555556	12\\
10.0836111111111	12\\
10.0891666666667	12\\
10.0947222222222	12\\
10.1002777777778	3\\
10.1058333333333	2\\
10.1113888888889	2\\
10.1169444444444	4\\
10.1225	4\\
10.1280555555556	2\\
10.1336111111111	2\\
10.1391666666667	2\\
10.1447222222222	4\\
10.1502777777778	4\\
10.1558333333333	4\\
10.1613888888889	4\\
10.1669444444444	6\\
10.1725	3\\
10.1780555555556	3\\
10.1836111111111	5\\
10.1891666666667	3\\
10.1947222222222	3\\
10.2002777777778	3\\
10.2058333333333	3\\
10.2113888888889	3\\
10.2169444444444	3\\
10.2225	4\\
10.2280555555556	5\\
10.2336111111111	3\\
10.2391666666667	3\\
10.2447222222222	4\\
10.2502777777778	3\\
10.2558333333333	3\\
10.2613888888889	5\\
10.2669444444444	4\\
10.2725	6\\
10.2780555555556	4\\
10.2836111111111	3\\
10.2891666666667	2\\
10.2947222222222	2\\
10.3002777777778	3\\
10.3058333333333	3\\
10.3113888888889	2\\
10.3169444444444	4\\
10.3225	3\\
10.3280555555556	3\\
10.3336111111111	2\\
10.3391666666667	2\\
10.3447222222222	1\\
10.3502777777778	2\\
10.3558333333333	4\\
10.3613888888889	4\\
10.3669444444444	4\\
10.3725	4\\
10.3780555555556	4\\
10.3836111111111	4\\
10.3891666666667	4\\
10.3947222222222	4\\
10.4002777777778	3\\
10.4058333333333	3\\
10.4113888888889	3\\
10.4169444444444	3\\
10.4225	3\\
10.4280555555556	4\\
10.4336111111111	5\\
10.4391666666667	5\\
10.4447222222222	4\\
10.4502777777778	4\\
10.4558333333333	4\\
10.4613888888889	4\\
10.4669444444444	4\\
10.4725	3\\
10.4780555555556	2\\
10.4836111111111	2\\
10.4891666666667	4\\
10.4947222222222	4\\
10.5002777777778	4\\
10.5058333333333	2\\
10.5113888888889	3\\
10.5169444444444	3\\
10.5225	3\\
10.5280555555556	2\\
10.5336111111111	1\\
10.5391666666667	1\\
10.5447222222222	1\\
10.5502777777778	2\\
10.5558333333333	2\\
10.5613888888889	2\\
10.5669444444444	1\\
10.5725	2\\
10.5780555555556	2\\
10.5836111111111	2\\
10.5891666666667	2\\
10.5947222222222	2\\
10.6002777777778	3\\
10.6058333333333	3\\
10.6113888888889	4\\
10.6169444444444	4\\
10.6225	4\\
10.6280555555556	3\\
10.6336111111111	3\\
10.6391666666667	4\\
10.6447222222222	4\\
10.6502777777778	3\\
10.6558333333333	3\\
10.6613888888889	2\\
10.6669444444444	1\\
10.6725	3\\
10.6780555555556	3\\
10.6836111111111	2\\
10.6891666666667	2\\
10.6947222222222	2\\
10.7002777777778	3\\
10.7058333333333	2\\
10.7113888888889	1\\
10.7169444444444	1\\
10.7225	1\\
10.7280555555556	2\\
10.7336111111111	3\\
10.7391666666667	3\\
10.7447222222222	4\\
10.7502777777778	5\\
10.7558333333333	5\\
10.7613888888889	3\\
10.7669444444444	3\\
10.7725	2\\
10.7780555555556	2\\
10.7836111111111	2\\
10.7891666666667	2\\
10.7947222222222	2\\
10.8002777777778	2\\
10.8058333333333	2\\
10.8113888888889	2\\
10.8169444444444	3\\
10.8225	3\\
10.8280555555556	3\\
10.8336111111111	3\\
10.8391666666667	6\\
10.8447222222222	3\\
10.8502777777778	3\\
10.8558333333333	4\\
10.8613888888889	3\\
10.8669444444444	2\\
10.8725	2\\
10.8780555555556	2\\
10.8836111111111	2\\
10.8891666666667	2\\
10.8947222222222	2\\
10.9002777777778	2\\
10.9058333333333	2\\
10.9113888888889	3\\
10.9169444444444	3\\
10.9225	4\\
10.9280555555556	4\\
10.9336111111111	7\\
10.9391666666667	2\\
10.9447222222222	6\\
10.9502777777778	5\\
10.9558333333333	5\\
10.9613888888889	5\\
10.9669444444444	8\\
10.9725	11\\
10.9780555555556	6\\
10.9836111111111	5\\
10.9891666666667	8\\
10.9947222222222	4\\
11.0002777777778	3\\
11.0058333333333	3\\
11.0113888888889	3\\
11.0169444444444	2\\
11.0225	2\\
11.0280555555556	2\\
11.0336111111111	2\\
11.0391666666667	2\\
11.0447222222222	3\\
11.0502777777778	4\\
11.0558333333333	3\\
11.0613888888889	3\\
11.0669444444444	3\\
11.0725	4\\
11.0780555555556	4\\
11.0836111111111	4\\
11.0891666666667	4\\
11.0947222222222	4\\
11.1002777777778	4\\
11.1058333333333	2\\
11.1113888888889	2\\
11.1169444444444	2\\
11.1225	2\\
11.1280555555556	2\\
11.1336111111111	2\\
11.1391666666667	2\\
11.1447222222222	2\\
11.1502777777778	2\\
11.1558333333333	3\\
11.1613888888889	3\\
11.1669444444444	3\\
11.1725	3\\
11.1780555555556	3\\
11.1836111111111	2\\
11.1891666666667	3\\
11.1947222222222	3\\
11.2002777777778	2\\
11.2058333333333	3\\
11.2113888888889	3\\
11.2169444444444	2\\
11.2225	2\\
11.2280555555556	2\\
11.2336111111111	2\\
11.2391666666667	2\\
11.2447222222222	2\\
11.2502777777778	4\\
11.2558333333333	4\\
11.2613888888889	4\\
11.2669444444444	4\\
11.2725	4\\
11.2780555555556	3\\
11.2836111111111	3\\
11.2891666666667	3\\
11.2947222222222	3\\
11.3002777777778	2\\
11.3058333333333	2\\
11.3113888888889	2\\
11.3169444444444	2\\
11.3225	2\\
11.3280555555556	2\\
11.3336111111111	1\\
11.3391666666667	1\\
11.3447222222222	2\\
11.3502777777778	3\\
11.3558333333333	4\\
11.3613888888889	4\\
11.3669444444444	3\\
11.3725	2\\
11.3780555555556	2\\
11.3836111111111	2\\
11.3891666666667	2\\
11.3947222222222	3\\
11.4002777777778	3\\
11.4058333333333	6\\
11.4113888888889	6\\
11.4169444444444	6\\
11.4225	3\\
11.4280555555556	3\\
11.4336111111111	1\\
11.4391666666667	1\\
11.4447222222222	2\\
11.4502777777778	2\\
11.4558333333333	4\\
11.4613888888889	5\\
11.4669444444444	5\\
11.4725	5\\
11.4780555555556	5\\
11.4836111111111	3\\
11.4891666666667	3\\
11.4947222222222	7\\
11.5002777777778	7\\
11.5058333333333	7\\
11.5113888888889	2\\
11.5169444444444	4\\
11.5225	4\\
11.5280555555556	4\\
11.5336111111111	4\\
11.5391666666667	4\\
11.5447222222222	3\\
11.5502777777778	4\\
11.5558333333333	4\\
11.5613888888889	2\\
11.5669444444444	2\\
11.5725	2\\
11.5780555555556	2\\
11.5836111111111	2\\
11.5891666666667	3\\
11.5947222222222	3\\
11.6002777777778	4\\
11.6058333333333	4\\
11.6113888888889	4\\
11.6169444444444	4\\
11.6225	6\\
11.6280555555556	4\\
11.6336111111111	3\\
11.6391666666667	3\\
11.6447222222222	3\\
11.6502777777778	4\\
11.6558333333333	4\\
11.6613888888889	4\\
11.6669444444444	4\\
11.6725	6\\
11.6780555555556	7\\
11.6836111111111	7\\
11.6891666666667	4\\
11.6947222222222	4\\
11.7002777777778	4\\
11.7058333333333	4\\
11.7113888888889	4\\
11.7169444444444	4\\
11.7225	4\\
11.7280555555556	4\\
11.7336111111111	5\\
11.7391666666667	6\\
11.7447222222222	7\\
11.7502777777778	9\\
11.7558333333333	9\\
11.7613888888889	10\\
11.7669444444444	11\\
11.7725	13\\
11.7780555555556	14\\
11.7836111111111	15\\
11.7891666666667	17\\
11.7947222222222	4\\
11.8002777777778	4\\
11.8058333333333	4\\
11.8113888888889	4\\
11.8169444444444	4\\
11.8225	4\\
11.8280555555556	4\\
11.8336111111111	4\\
11.8391666666667	5\\
11.8447222222222	7\\
11.8502777777778	3\\
11.8558333333333	3\\
11.8613888888889	2\\
11.8669444444444	3\\
11.8725	4\\
11.8780555555556	5\\
11.8836111111111	6\\
11.8891666666667	8\\
11.8947222222222	4\\
11.9002777777778	4\\
11.9058333333333	4\\
11.9113888888889	5\\
11.9169444444444	5\\
11.9225	5\\
11.9280555555556	6\\
11.9336111111111	6\\
11.9391666666667	8\\
11.9447222222222	8\\
11.9502777777778	8\\
11.9558333333333	8\\
11.9613888888889	4\\
11.9669444444444	4\\
11.9725	4\\
11.9780555555556	5\\
11.9836111111111	4\\
11.9891666666667	4\\
11.9947222222222	4\\
12.0002777777778	4\\
12.0058333333333	7\\
12.0113888888889	8\\
12.0169444444444	9\\
12.0225	9\\
12.0280555555556	9\\
12.0336111111111	9\\
12.0391666666667	4\\
12.0447222222222	3\\
12.0502777777778	3\\
12.0558333333333	4\\
12.0613888888889	6\\
12.0669444444444	9\\
12.0725	5\\
12.0780555555556	8\\
12.0836111111111	4\\
12.0891666666667	3\\
12.0947222222222	4\\
12.1002777777778	4\\
12.1058333333333	3\\
12.1113888888889	3\\
12.1169444444444	4\\
12.1225	5\\
12.1280555555556	5\\
12.1336111111111	5\\
12.1391666666667	4\\
12.1447222222222	3\\
12.1502777777778	3\\
12.1558333333333	3\\
12.1613888888889	3\\
12.1669444444444	3\\
12.1725	3\\
12.1780555555556	3\\
12.1836111111111	3\\
12.1891666666667	2\\
12.1947222222222	3\\
12.2002777777778	3\\
12.2058333333333	3\\
12.2113888888889	3\\
12.2169444444444	3\\
12.2225	3\\
12.2280555555556	4\\
12.2336111111111	4\\
12.2391666666667	4\\
12.2447222222222	4\\
12.2502777777778	3\\
12.2558333333333	5\\
12.2613888888889	5\\
12.2669444444444	7\\
12.2725	7\\
12.2780555555556	7\\
12.2836111111111	8\\
12.2891666666667	8\\
12.2947222222222	8\\
12.3002777777778	8\\
12.3058333333333	4\\
12.3113888888889	4\\
12.3169444444444	4\\
12.3225	4\\
12.3280555555556	4\\
12.3336111111111	3\\
12.3391666666667	4\\
12.3447222222222	4\\
12.3502777777778	5\\
12.3558333333333	5\\
12.3613888888889	5\\
12.3669444444444	4\\
12.3725	4\\
12.3780555555556	4\\
12.3836111111111	4\\
12.3891666666667	5\\
12.3947222222222	5\\
12.4002777777778	5\\
12.4058333333333	5\\
12.4113888888889	5\\
12.4169444444444	5\\
12.4225	11\\
12.4280555555556	11\\
12.4336111111111	12\\
12.4391666666667	4\\
12.4447222222222	2\\
12.4502777777778	2\\
12.4558333333333	2\\
12.4613888888889	2\\
12.4669444444444	3\\
12.4725	2\\
12.4780555555556	2\\
12.4836111111111	2\\
12.4891666666667	3\\
12.4947222222222	3\\
12.5002777777778	5\\
12.5058333333333	4\\
12.5113888888889	4\\
12.5169444444444	4\\
12.5225	3\\
12.5280555555556	2\\
12.5336111111111	2\\
12.5391666666667	2\\
12.5447222222222	2\\
12.5502777777778	2\\
12.5558333333333	3\\
12.5613888888889	3\\
12.5669444444444	3\\
12.5725	3\\
12.5780555555556	3\\
12.5836111111111	2\\
12.5891666666667	2\\
12.5947222222222	3\\
12.6002777777778	3\\
12.6058333333333	3\\
12.6113888888889	4\\
12.6169444444444	5\\
12.6225	6\\
12.6280555555556	4\\
12.6336111111111	4\\
12.6391666666667	4\\
12.6447222222222	4\\
12.6502777777778	5\\
12.6558333333333	5\\
12.6613888888889	5\\
12.6669444444444	4\\
12.6725	3\\
12.6780555555556	3\\
12.6836111111111	2\\
12.6891666666667	2\\
12.6947222222222	3\\
12.7002777777778	4\\
12.7058333333333	4\\
12.7113888888889	4\\
12.7169444444444	4\\
12.7225	4\\
12.7280555555556	4\\
12.7336111111111	4\\
12.7391666666667	4\\
12.7447222222222	3\\
12.7502777777778	3\\
12.7558333333333	3\\
12.7613888888889	3\\
12.7669444444444	3\\
12.7725	3\\
12.7780555555556	3\\
12.7836111111111	3\\
12.7891666666667	3\\
12.7947222222222	2\\
12.8002777777778	2\\
12.8058333333333	2\\
12.8113888888889	2\\
12.8169444444444	2\\
12.8225	3\\
12.8280555555556	3\\
12.8336111111111	3\\
12.8391666666667	3\\
12.8447222222222	2\\
12.8502777777778	2\\
12.8558333333333	2\\
12.8613888888889	2\\
12.8669444444444	2\\
12.8725	2\\
12.8780555555556	2\\
12.8836111111111	4\\
12.8891666666667	4\\
12.8947222222222	5\\
12.9002777777778	5\\
12.9058333333333	5\\
12.9113888888889	4\\
12.9169444444444	3\\
12.9225	3\\
12.9280555555556	3\\
12.9336111111111	4\\
12.9391666666667	5\\
12.9447222222222	5\\
12.9502777777778	5\\
12.9558333333333	6\\
12.9613888888889	3\\
12.9669444444444	3\\
12.9725	4\\
12.9780555555556	5\\
12.9836111111111	4\\
12.9891666666667	4\\
12.9947222222222	4\\
13.0002777777778	5\\
13.0058333333333	5\\
13.0113888888889	5\\
13.0169444444444	5\\
13.0225	5\\
13.0280555555556	6\\
13.0336111111111	6\\
13.0391666666667	6\\
13.0447222222222	6\\
13.0502777777778	4\\
13.0558333333333	4\\
13.0613888888889	5\\
13.0669444444444	5\\
13.0725	5\\
13.0780555555556	3\\
13.0836111111111	3\\
13.0891666666667	3\\
13.0947222222222	3\\
13.1002777777778	3\\
13.1058333333333	6\\
13.1113888888889	6\\
13.1169444444444	10\\
13.1225	3\\
13.1280555555556	3\\
13.1336111111111	3\\
13.1391666666667	4\\
13.1447222222222	4\\
13.1502777777778	6\\
13.1558333333333	6\\
13.1613888888889	6\\
13.1669444444444	6\\
13.1725	6\\
13.1780555555556	6\\
13.1836111111111	6\\
13.1891666666667	6\\
13.1947222222222	6\\
13.2002777777778	6\\
13.2058333333333	6\\
13.2113888888889	6\\
13.2169444444444	7\\
13.2225	7\\
13.2280555555556	7\\
13.2336111111111	7\\
13.2391666666667	8\\
13.2447222222222	11\\
13.2502777777778	13\\
13.2558333333333	15\\
13.2613888888889	16\\
13.2669444444444	16\\
13.2725	16\\
13.2780555555556	17\\
13.2836111111111	20\\
13.2891666666667	20\\
13.2947222222222	20\\
13.3002777777778	20\\
13.3058333333333	19\\
13.3113888888889	20\\
13.3169444444444	3\\
13.3225	2\\
13.3280555555556	4\\
13.3336111111111	3\\
13.3391666666667	4\\
13.3447222222222	3\\
13.3502777777778	4\\
13.3558333333333	4\\
13.3613888888889	4\\
13.3669444444444	4\\
13.3725	4\\
13.3780555555556	6\\
13.3836111111111	8\\
13.3891666666667	8\\
13.3947222222222	8\\
13.4002777777778	12\\
13.4058333333333	15\\
13.4113888888889	18\\
13.4169444444444	4\\
13.4225	4\\
13.4280555555556	4\\
13.4336111111111	3\\
13.4391666666667	3\\
13.4447222222222	3\\
13.4502777777778	3\\
13.4558333333333	5\\
13.4613888888889	5\\
13.4669444444444	4\\
13.4725	4\\
13.4780555555556	4\\
13.4836111111111	4\\
13.4891666666667	5\\
13.4947222222222	4\\
13.5002777777778	5\\
13.5058333333333	6\\
13.5113888888889	2\\
13.5169444444444	3\\
13.5225	2\\
13.5280555555556	2\\
13.5336111111111	2\\
13.5391666666667	2\\
13.5447222222222	2\\
13.5502777777778	2\\
13.5558333333333	1\\
13.5613888888889	1\\
13.5669444444444	2\\
13.5725	4\\
13.5780555555556	3\\
13.5836111111111	3\\
13.5891666666667	3\\
13.5947222222222	3\\
13.6002777777778	2\\
13.6058333333333	2\\
13.6113888888889	2\\
13.6169444444444	2\\
13.6225	2\\
13.6280555555556	2\\
13.6336111111111	2\\
13.6391666666667	4\\
13.6447222222222	4\\
13.6502777777778	4\\
13.6558333333333	4\\
13.6613888888889	4\\
13.6669444444444	4\\
13.6725	4\\
13.6780555555556	4\\
13.6836111111111	4\\
13.6891666666667	3\\
13.6947222222222	2\\
13.7002777777778	2\\
13.7058333333333	2\\
13.7113888888889	2\\
13.7169444444444	3\\
13.7225	5\\
13.7280555555556	6\\
13.7336111111111	4\\
13.7391666666667	4\\
13.7447222222222	5\\
13.7502777777778	7\\
13.7558333333333	8\\
13.7613888888889	8\\
13.7669444444444	8\\
13.7725	8\\
13.7780555555556	7\\
13.7836111111111	7\\
13.7891666666667	4\\
13.7947222222222	2\\
13.8002777777778	3\\
13.8058333333333	4\\
13.8113888888889	4\\
13.8169444444444	4\\
13.8225	4\\
13.8280555555556	4\\
13.8336111111111	5\\
13.8391666666667	5\\
13.8447222222222	5\\
13.8502777777778	5\\
13.8558333333333	4\\
13.8613888888889	4\\
13.8669444444444	3\\
13.8725	2\\
13.8780555555556	2\\
13.8836111111111	2\\
13.8891666666667	3\\
13.8947222222222	3\\
13.9002777777778	3\\
13.9058333333333	3\\
13.9113888888889	3\\
13.9169444444444	2\\
13.9225	3\\
13.9280555555556	4\\
13.9336111111111	4\\
13.9391666666667	5\\
13.9447222222222	5\\
13.9502777777778	4\\
13.9558333333333	4\\
13.9613888888889	4\\
13.9669444444444	6\\
13.9725	3\\
13.9780555555556	3\\
13.9836111111111	4\\
13.9891666666667	4\\
13.9947222222222	5\\
14.0002777777778	8\\
14.0058333333333	8\\
14.0113888888889	3\\
14.0169444444444	4\\
14.0225	7\\
14.0280555555556	8\\
14.0336111111111	11\\
14.0391666666667	11\\
14.0447222222222	12\\
14.0502777777778	14\\
14.0558333333333	16\\
14.0613888888889	17\\
14.0669444444444	19\\
14.0725	20\\
14.0780555555556	20\\
14.0836111111111	20\\
14.0891666666667	20\\
14.0947222222222	20\\
14.1002777777778	20\\
14.1058333333333	20\\
14.1113888888889	20\\
14.1169444444444	20\\
14.1225	20\\
14.1280555555556	20\\
14.1336111111111	19\\
14.1391666666667	2\\
14.1447222222222	5\\
14.1502777777778	5\\
14.1558333333333	9\\
14.1613888888889	11\\
14.1669444444444	12\\
14.1725	5\\
14.1780555555556	6\\
14.1836111111111	7\\
14.1891666666667	7\\
14.1947222222222	8\\
14.2002777777778	8\\
14.2058333333333	8\\
14.2113888888889	8\\
14.2169444444444	8\\
14.2225	3\\
14.2280555555556	4\\
14.2336111111111	4\\
14.2391666666667	4\\
14.2447222222222	6\\
14.2502777777778	7\\
14.2558333333333	7\\
14.2613888888889	7\\
14.2669444444444	9\\
14.2725	11\\
14.2780555555556	13\\
14.2836111111111	14\\
14.2891666666667	16\\
14.2947222222222	17\\
14.3002777777778	19\\
14.3058333333333	20\\
14.3113888888889	20\\
14.3169444444444	20\\
14.3225	20\\
14.3280555555556	20\\
14.3336111111111	20\\
14.3391666666667	20\\
14.3447222222222	20\\
14.3502777777778	20\\
14.3558333333333	20\\
14.3613888888889	20\\
14.3669444444444	20\\
14.3725	20\\
14.3780555555556	3\\
14.3836111111111	3\\
14.3891666666667	2\\
14.3947222222222	2\\
14.4002777777778	3\\
14.4058333333333	4\\
14.4113888888889	4\\
14.4169444444444	5\\
14.4225	5\\
14.4280555555556	4\\
14.4336111111111	3\\
14.4391666666667	2\\
14.4447222222222	3\\
14.4502777777778	3\\
14.4558333333333	2\\
14.4613888888889	2\\
14.4669444444444	2\\
14.4725	2\\
14.4780555555556	2\\
14.4836111111111	2\\
14.4891666666667	2\\
14.4947222222222	2\\
14.5002777777778	2\\
14.5058333333333	2\\
14.5113888888889	2\\
14.5169444444444	2\\
14.5225	2\\
14.5280555555556	2\\
14.5336111111111	2\\
14.5391666666667	2\\
14.5447222222222	2\\
14.5502777777778	2\\
14.5558333333333	2\\
14.5613888888889	2\\
14.5669444444444	2\\
14.5725	2\\
14.5780555555556	5\\
14.5836111111111	4\\
14.5891666666667	4\\
14.5947222222222	5\\
14.6002777777778	5\\
14.6058333333333	7\\
14.6113888888889	7\\
14.6169444444444	7\\
14.6225	4\\
14.6280555555556	4\\
14.6336111111111	4\\
14.6391666666667	4\\
14.6447222222222	4\\
14.6502777777778	4\\
14.6558333333333	3\\
14.6613888888889	3\\
14.6669444444444	3\\
14.6725	3\\
14.6780555555556	4\\
14.6836111111111	2\\
14.6891666666667	2\\
14.6947222222222	4\\
14.7002777777778	4\\
14.7058333333333	4\\
14.7113888888889	4\\
14.7169444444444	4\\
14.7225	4\\
14.7280555555556	5\\
14.7336111111111	5\\
14.7391666666667	8\\
14.7447222222222	3\\
14.7502777777778	4\\
14.7558333333333	5\\
14.7613888888889	3\\
14.7669444444444	2\\
14.7725	2\\
14.7780555555556	2\\
14.7836111111111	2\\
14.7891666666667	2\\
14.7947222222222	2\\
14.8002777777778	2\\
14.8058333333333	2\\
14.8113888888889	2\\
14.8169444444444	2\\
14.8225	2\\
14.8280555555556	9\\
14.8336111111111	10\\
14.8391666666667	10\\
14.8447222222222	4\\
14.8502777777778	5\\
14.8558333333333	5\\
14.8613888888889	5\\
14.8669444444444	5\\
14.8725	5\\
14.8780555555556	5\\
14.8836111111111	5\\
14.8891666666667	5\\
14.8947222222222	5\\
14.9002777777778	3\\
14.9058333333333	3\\
14.9113888888889	3\\
14.9169444444444	2\\
14.9225	3\\
14.9280555555556	3\\
14.9336111111111	3\\
14.9391666666667	3\\
14.9447222222222	3\\
14.9502777777778	3\\
14.9558333333333	6\\
14.9613888888889	4\\
14.9669444444444	3\\
14.9725	2\\
14.9780555555556	2\\
14.9836111111111	2\\
14.9891666666667	2\\
14.9947222222222	2\\
15.0002777777778	2\\
15.0058333333333	2\\
15.0113888888889	1\\
15.0169444444444	1\\
15.0225	2\\
15.0280555555556	2\\
15.0336111111111	2\\
15.0391666666667	3\\
15.0447222222222	2\\
15.0502777777778	2\\
15.0558333333333	2\\
15.0613888888889	3\\
15.0669444444444	5\\
15.0725	5\\
15.0780555555556	5\\
15.0836111111111	5\\
15.0891666666667	2\\
15.0947222222222	2\\
15.1002777777778	5\\
15.1058333333333	5\\
15.1113888888889	5\\
15.1169444444444	5\\
15.1225	3\\
15.1280555555556	3\\
15.1336111111111	2\\
15.1391666666667	2\\
15.1447222222222	3\\
15.1502777777778	4\\
15.1558333333333	3\\
15.1613888888889	3\\
15.1669444444444	4\\
15.1725	4\\
15.1780555555556	5\\
15.1836111111111	5\\
15.1891666666667	4\\
15.1947222222222	5\\
15.2002777777778	6\\
15.2058333333333	6\\
15.2113888888889	4\\
15.2169444444444	5\\
15.2225	6\\
15.2280555555556	6\\
15.2336111111111	8\\
15.2391666666667	8\\
15.2447222222222	3\\
15.2502777777778	4\\
15.2558333333333	2\\
15.2613888888889	2\\
15.2669444444444	2\\
15.2725	4\\
15.2780555555556	9\\
15.2836111111111	5\\
15.2891666666667	5\\
15.2947222222222	5\\
15.3002777777778	6\\
15.3058333333333	3\\
15.3113888888889	3\\
15.3169444444444	4\\
15.3225	2\\
15.3280555555556	3\\
15.3336111111111	4\\
15.3391666666667	4\\
15.3447222222222	2\\
15.3502777777778	2\\
15.3558333333333	5\\
15.3613888888889	5\\
15.3669444444444	5\\
15.3725	3\\
15.3780555555556	2\\
15.3836111111111	2\\
15.3891666666667	4\\
15.3947222222222	8\\
15.4002777777778	5\\
15.4058333333333	5\\
15.4113888888889	5\\
15.4169444444444	6\\
15.4225	7\\
15.4280555555556	6\\
15.4336111111111	4\\
15.4391666666667	3\\
15.4447222222222	3\\
15.4502777777778	2\\
15.4558333333333	2\\
15.4613888888889	2\\
15.4669444444444	2\\
15.4725	3\\
15.4780555555556	3\\
15.4836111111111	3\\
15.4891666666667	3\\
15.4947222222222	2\\
15.5002777777778	2\\
15.5058333333333	2\\
15.5113888888889	2\\
15.5169444444444	3\\
15.5225	4\\
15.5280555555556	4\\
15.5336111111111	3\\
15.5391666666667	2\\
15.5447222222222	3\\
15.5502777777778	2\\
15.5558333333333	3\\
15.5613888888889	4\\
15.5669444444444	4\\
15.5725	7\\
15.5780555555556	7\\
15.5836111111111	6\\
15.5891666666667	4\\
15.5947222222222	4\\
15.6002777777778	4\\
15.6058333333333	5\\
15.6113888888889	7\\
15.6169444444444	2\\
15.6225	1\\
15.6280555555556	-2\\
15.6336111111111	-1\\
15.6391666666667	-2\\
15.6447222222222	-4\\
15.6502777777778	-3\\
15.6558333333333	-3\\
15.6613888888889	-4\\
15.6669444444444	-5\\
15.6725	-3\\
15.6780555555556	-2\\
15.6836111111111	-2\\
15.6891666666667	-2\\
15.6947222222222	-2\\
15.7002777777778	-4\\
15.7058333333333	-5\\
15.7113888888889	-5\\
15.7169444444444	-5\\
15.7225	-5\\
15.7280555555556	-5\\
15.7336111111111	-3\\
15.7391666666667	-3\\
15.7447222222222	-2\\
15.7502777777778	-2\\
15.7558333333333	-3\\
15.7613888888889	-2\\
15.7669444444444	-3\\
15.7725	-5\\
15.7780555555556	-8\\
15.7836111111111	-6\\
15.7891666666667	-7\\
15.7947222222222	-12\\
15.8002777777778	-13\\
15.8058333333333	-7\\
15.8113888888889	-8\\
15.8169444444444	-9\\
15.8225	-9\\
15.8280555555556	-5\\
15.8336111111111	-5\\
15.8391666666667	-8\\
15.8447222222222	-8\\
15.8502777777778	-8\\
15.8558333333333	-8\\
15.8613888888889	-7\\
15.8669444444444	-4\\
15.8725	-4\\
15.8780555555556	-3\\
15.8836111111111	-2\\
15.8891666666667	-2\\
15.8947222222222	-2\\
15.9002777777778	-3\\
15.9058333333333	-5\\
15.9113888888889	-5\\
15.9169444444444	-5\\
15.9225	-6\\
15.9280555555556	-11\\
15.9336111111111	-4\\
15.9391666666667	-6\\
15.9447222222222	-2\\
15.9502777777778	-5\\
15.9558333333333	-6\\
15.9613888888889	-6\\
15.9669444444444	-6\\
15.9725	-7\\
15.9780555555556	-9\\
15.9836111111111	-11\\
15.9891666666667	-11\\
15.9947222222222	0\\
};
\addlegendentry{Cts Stoch Ctrl};

\addplot [color=dscr_plot_color,solid,line width=1.5pt]
  table[row sep=crcr]{%
9.50027777777778	2\\
9.50583333333333	3\\
9.51138888888889	4\\
9.51694444444444	1\\
9.5225	-1\\
9.52805555555556	2\\
9.53361111111111	-3\\
9.53916666666667	-3\\
9.54472222222222	-2\\
9.55027777777778	-2\\
9.55583333333333	-2\\
9.56138888888889	-2\\
9.56694444444444	-2\\
9.5725	-1\\
9.57805555555555	-1\\
9.58361111111111	-3\\
9.58916666666667	3\\
9.59472222222222	2\\
9.60027777777778	5\\
9.60583333333333	2\\
9.61138888888889	2\\
9.61694444444444	-2\\
9.6225	-2\\
9.62805555555556	2\\
9.63361111111111	2\\
9.63916666666667	-2\\
9.64472222222222	2\\
9.65027777777778	4\\
9.65583333333333	5\\
9.66138888888889	2\\
9.66694444444444	3\\
9.6725	2\\
9.67805555555555	2\\
9.68361111111111	3\\
9.68916666666667	2\\
9.69472222222222	2\\
9.70027777777778	2\\
9.70583333333333	2\\
9.71138888888889	3\\
9.71694444444444	2\\
9.7225	2\\
9.72805555555555	2\\
9.73361111111111	2\\
9.73916666666667	2\\
9.74472222222222	2\\
9.75027777777778	3\\
9.75583333333333	4\\
9.76138888888889	2\\
9.76694444444444	2\\
9.7725	-2\\
9.77805555555556	-3\\
9.78361111111111	-4\\
9.78916666666667	-2\\
9.79472222222222	-3\\
9.80027777777778	-1\\
9.80583333333333	-2\\
9.81138888888889	-4\\
9.81694444444444	-4\\
9.8225	-6\\
9.82805555555555	-1\\
9.83361111111111	-2\\
9.83916666666667	-2\\
9.84472222222222	-2\\
9.85027777777778	-5\\
9.85583333333333	-2\\
9.86138888888889	-2\\
9.86694444444444	-4\\
9.8725	-3\\
9.87805555555556	-4\\
9.88361111111111	-2\\
9.88916666666667	-2\\
9.89472222222222	-2\\
9.90027777777778	-2\\
9.90583333333333	-3\\
9.91138888888889	-3\\
9.91694444444444	-2\\
9.9225	-3\\
9.92805555555555	-4\\
9.93361111111111	-5\\
9.93916666666667	-2\\
9.94472222222222	-2\\
9.95027777777778	-3\\
9.95583333333333	-2\\
9.96138888888889	-2\\
9.96694444444444	-5\\
9.9725	-7\\
9.97805555555555	-2\\
9.98361111111111	-2\\
9.98916666666667	-2\\
9.99472222222222	-3\\
10.0002777777778	-3\\
10.0058333333333	-3\\
10.0113888888889	-3\\
10.0169444444444	-3\\
10.0225	-6\\
10.0280555555556	-6\\
10.0336111111111	-6\\
10.0391666666667	-7\\
10.0447222222222	-5\\
10.0502777777778	-8\\
10.0558333333333	-2\\
10.0613888888889	-2\\
10.0669444444444	-2\\
10.0725	-2\\
10.0780555555556	-1\\
10.0836111111111	-2\\
10.0891666666667	-2\\
10.0947222222222	-2\\
10.1002777777778	-3\\
10.1058333333333	-5\\
10.1113888888889	-6\\
10.1169444444444	-5\\
10.1225	-5\\
10.1280555555556	-7\\
10.1336111111111	-7\\
10.1391666666667	-7\\
10.1447222222222	-6\\
10.1502777777778	-6\\
10.1558333333333	-6\\
10.1613888888889	-6\\
10.1669444444444	-2\\
10.1725	-4\\
10.1780555555556	-4\\
10.1836111111111	-1\\
10.1891666666667	-2\\
10.1947222222222	-2\\
10.2002777777778	-2\\
10.2058333333333	-2\\
10.2113888888889	-2\\
10.2169444444444	-2\\
10.2225	-2\\
10.2280555555556	-2\\
10.2336111111111	-3\\
10.2391666666667	-3\\
10.2447222222222	-2\\
10.2502777777778	-3\\
10.2558333333333	-2\\
10.2613888888889	-2\\
10.2669444444444	-3\\
10.2725	-2\\
10.2780555555556	-2\\
10.2836111111111	-3\\
10.2891666666667	-5\\
10.2947222222222	-5\\
10.3002777777778	-2\\
10.3058333333333	-2\\
10.3113888888889	-2\\
10.3169444444444	-3\\
10.3225	-2\\
10.3280555555556	-2\\
10.3336111111111	-2\\
10.3391666666667	-3\\
10.3447222222222	-3\\
10.3502777777778	-5\\
10.3558333333333	-2\\
10.3613888888889	-2\\
10.3669444444444	-2\\
10.3725	-2\\
10.3780555555556	-2\\
10.3836111111111	-2\\
10.3891666666667	-3\\
10.3947222222222	-3\\
10.4002777777778	-2\\
10.4058333333333	-2\\
10.4113888888889	-2\\
10.4169444444444	-2\\
10.4225	-2\\
10.4280555555556	-2\\
10.4336111111111	-2\\
10.4391666666667	-2\\
10.4447222222222	-3\\
10.4502777777778	-3\\
10.4558333333333	-3\\
10.4613888888889	-3\\
10.4669444444444	-4\\
10.4725	-4\\
10.4780555555556	-5\\
10.4836111111111	-5\\
10.4891666666667	-2\\
10.4947222222222	-3\\
10.5002777777778	-3\\
10.5058333333333	-4\\
10.5113888888889	-3\\
10.5169444444444	-1\\
10.5225	-2\\
10.5280555555556	-4\\
10.5336111111111	-7\\
10.5391666666667	-8\\
10.5447222222222	-9\\
10.5502777777778	-11\\
10.5558333333333	-3\\
10.5613888888889	-3\\
10.5669444444444	-5\\
10.5725	-5\\
10.5780555555556	-6\\
10.5836111111111	-9\\
10.5891666666667	-11\\
10.5947222222222	-11\\
10.6002777777778	-2\\
10.6058333333333	-2\\
10.6113888888889	-2\\
10.6169444444444	-2\\
10.6225	-2\\
10.6280555555556	-3\\
10.6336111111111	-3\\
10.6391666666667	-2\\
10.6447222222222	-2\\
10.6502777777778	-3\\
10.6558333333333	-3\\
10.6613888888889	-4\\
10.6669444444444	-5\\
10.6725	-4\\
10.6780555555556	-3\\
10.6836111111111	-5\\
10.6891666666667	-6\\
10.6947222222222	-6\\
10.7002777777778	-2\\
10.7058333333333	-3\\
10.7113888888889	-4\\
10.7169444444444	-4\\
10.7225	-4\\
10.7280555555556	-2\\
10.7336111111111	-2\\
10.7391666666667	-3\\
10.7447222222222	-3\\
10.7502777777778	-3\\
10.7558333333333	-3\\
10.7613888888889	-3\\
10.7669444444444	-1\\
10.7725	-3\\
10.7780555555556	-3\\
10.7836111111111	-3\\
10.7891666666667	-3\\
10.7947222222222	-4\\
10.8002777777778	-6\\
10.8058333333333	-8\\
10.8113888888889	-8\\
10.8169444444444	-2\\
10.8225	-2\\
10.8280555555556	-2\\
10.8336111111111	-2\\
10.8391666666667	-2\\
10.8447222222222	-4\\
10.8502777777778	-6\\
10.8558333333333	-2\\
10.8613888888889	-3\\
10.8669444444444	-5\\
10.8725	-6\\
10.8780555555556	-6\\
10.8836111111111	-6\\
10.8891666666667	-6\\
10.8947222222222	-6\\
10.9002777777778	-6\\
10.9058333333333	-6\\
10.9113888888889	-2\\
10.9169444444444	-2\\
10.9225	-2\\
10.9280555555556	-2\\
10.9336111111111	-2\\
10.9391666666667	-6\\
10.9447222222222	-2\\
10.9502777777778	-2\\
10.9558333333333	-3\\
10.9613888888889	-2\\
10.9669444444444	-2\\
10.9725	-2\\
10.9780555555556	-1\\
10.9836111111111	-2\\
10.9891666666667	-2\\
10.9947222222222	-3\\
11.0002777777778	-5\\
11.0058333333333	-5\\
11.0113888888889	-5\\
11.0169444444444	-7\\
11.0225	-8\\
11.0280555555556	-8\\
11.0336111111111	-3\\
11.0391666666667	-4\\
11.0447222222222	-3\\
11.0502777777778	-1\\
11.0558333333333	-4\\
11.0613888888889	-4\\
11.0669444444444	-4\\
11.0725	-2\\
11.0780555555556	-2\\
11.0836111111111	-2\\
11.0891666666667	-2\\
11.0947222222222	-2\\
11.1002777777778	-2\\
11.1058333333333	-4\\
11.1113888888889	-8\\
11.1169444444444	-10\\
11.1225	-10\\
11.1280555555556	-10\\
11.1336111111111	-10\\
11.1391666666667	-10\\
11.1447222222222	-11\\
11.1502777777778	-2\\
11.1558333333333	-2\\
11.1613888888889	-2\\
11.1669444444444	-3\\
11.1725	-3\\
11.1780555555556	-3\\
11.1836111111111	-4\\
11.1891666666667	-4\\
11.1947222222222	-4\\
11.2002777777778	-5\\
11.2058333333333	-2\\
11.2113888888889	-2\\
11.2169444444444	-3\\
11.2225	-3\\
11.2280555555556	-3\\
11.2336111111111	-3\\
11.2391666666667	-5\\
11.2447222222222	-6\\
11.2502777777778	-1\\
11.2558333333333	-2\\
11.2613888888889	-2\\
11.2669444444444	-2\\
11.2725	-2\\
11.2780555555556	-2\\
11.2836111111111	-2\\
11.2891666666667	-2\\
11.2947222222222	-3\\
11.3002777777778	-5\\
11.3058333333333	-5\\
11.3113888888889	-2\\
11.3169444444444	-3\\
11.3225	-3\\
11.3280555555556	-3\\
11.3336111111111	-4\\
11.3391666666667	-4\\
11.3447222222222	-4\\
11.3502777777778	-2\\
11.3558333333333	-2\\
11.3613888888889	-2\\
11.3669444444444	-3\\
11.3725	-4\\
11.3780555555556	-4\\
11.3836111111111	-4\\
11.3891666666667	-5\\
11.3947222222222	-4\\
11.4002777777778	-4\\
11.4058333333333	-3\\
11.4113888888889	-3\\
11.4169444444444	-3\\
11.4225	-5\\
11.4280555555556	-2\\
11.4336111111111	-4\\
11.4391666666667	-4\\
11.4447222222222	-2\\
11.4502777777778	-2\\
11.4558333333333	-2\\
11.4613888888889	-2\\
11.4669444444444	-2\\
11.4725	-2\\
11.4780555555556	-2\\
11.4836111111111	-3\\
11.4891666666667	-3\\
11.4947222222222	-2\\
11.5002777777778	-2\\
11.5058333333333	-2\\
11.5113888888889	-5\\
11.5169444444444	-2\\
11.5225	-2\\
11.5280555555556	-2\\
11.5336111111111	-2\\
11.5391666666667	-2\\
11.5447222222222	-3\\
11.5502777777778	-2\\
11.5558333333333	-2\\
11.5613888888889	-3\\
11.5669444444444	-7\\
11.5725	-8\\
11.5780555555556	-8\\
11.5836111111111	-8\\
11.5891666666667	-2\\
11.5947222222222	-3\\
11.6002777777778	-2\\
11.6058333333333	-2\\
11.6113888888889	-2\\
11.6169444444444	-2\\
11.6225	-2\\
11.6280555555556	-3\\
11.6336111111111	-5\\
11.6391666666667	-6\\
11.6447222222222	-5\\
11.6502777777778	-4\\
11.6558333333333	-4\\
11.6613888888889	-5\\
11.6669444444444	-5\\
11.6725	-2\\
11.6780555555556	-2\\
11.6836111111111	-2\\
11.6891666666667	-2\\
11.6947222222222	-2\\
11.7002777777778	-2\\
11.7058333333333	-2\\
11.7113888888889	-2\\
11.7169444444444	-2\\
11.7225	-2\\
11.7280555555556	-2\\
11.7336111111111	-2\\
11.7391666666667	-2\\
11.7447222222222	-2\\
11.7502777777778	-2\\
11.7558333333333	-2\\
11.7613888888889	-2\\
11.7669444444444	-2\\
11.7725	-2\\
11.7780555555556	-2\\
11.7836111111111	-2\\
11.7891666666667	-2\\
11.7947222222222	-1\\
11.8002777777778	-2\\
11.8058333333333	-2\\
11.8113888888889	-2\\
11.8169444444444	-2\\
11.8225	-2\\
11.8280555555556	-2\\
11.8336111111111	-2\\
11.8391666666667	-2\\
11.8447222222222	-2\\
11.8502777777778	-3\\
11.8558333333333	-2\\
11.8613888888889	-4\\
11.8669444444444	-3\\
11.8725	-3\\
11.8780555555556	-3\\
11.8836111111111	-2\\
11.8891666666667	-2\\
11.8947222222222	-3\\
11.9002777777778	-2\\
11.9058333333333	-2\\
11.9113888888889	-2\\
11.9169444444444	-2\\
11.9225	-2\\
11.9280555555556	-2\\
11.9336111111111	-2\\
11.9391666666667	-2\\
11.9447222222222	-2\\
11.9502777777778	-2\\
11.9558333333333	-2\\
11.9613888888889	-3\\
11.9669444444444	-3\\
11.9725	-3\\
11.9780555555556	-2\\
11.9836111111111	-2\\
11.9891666666667	-2\\
11.9947222222222	-2\\
12.0002777777778	-2\\
12.0058333333333	-2\\
12.0113888888889	-2\\
12.0169444444444	-2\\
12.0225	-2\\
12.0280555555556	-2\\
12.0336111111111	-2\\
12.0391666666667	-3\\
12.0447222222222	-4\\
12.0502777777778	-4\\
12.0558333333333	-2\\
12.0613888888889	-2\\
12.0669444444444	-2\\
12.0725	-3\\
12.0780555555556	-2\\
12.0836111111111	-3\\
12.0891666666667	-4\\
12.0947222222222	-4\\
12.1002777777778	-4\\
12.1058333333333	-4\\
12.1113888888889	-2\\
12.1169444444444	-3\\
12.1225	-2\\
12.1280555555556	-2\\
12.1336111111111	-2\\
12.1391666666667	-4\\
12.1447222222222	-5\\
12.1502777777778	-6\\
12.1558333333333	-3\\
12.1613888888889	-3\\
12.1669444444444	-3\\
12.1725	-3\\
12.1780555555556	-3\\
12.1836111111111	-3\\
12.1891666666667	-2\\
12.1947222222222	-2\\
12.2002777777778	-2\\
12.2058333333333	-2\\
12.2113888888889	-2\\
12.2169444444444	-2\\
12.2225	-2\\
12.2280555555556	-2\\
12.2336111111111	-2\\
12.2391666666667	-2\\
12.2447222222222	-4\\
12.2502777777778	-5\\
12.2558333333333	-2\\
12.2613888888889	-2\\
12.2669444444444	-2\\
12.2725	-2\\
12.2780555555556	-2\\
12.2836111111111	-2\\
12.2891666666667	-2\\
12.2947222222222	-2\\
12.3002777777778	-2\\
12.3058333333333	-3\\
12.3113888888889	-3\\
12.3169444444444	-3\\
12.3225	-3\\
12.3280555555556	-3\\
12.3336111111111	-4\\
12.3391666666667	-5\\
12.3447222222222	-2\\
12.3502777777778	-2\\
12.3558333333333	-2\\
12.3613888888889	-2\\
12.3669444444444	-2\\
12.3725	-3\\
12.3780555555556	-3\\
12.3836111111111	-3\\
12.3891666666667	-4\\
12.3947222222222	-4\\
12.4002777777778	-4\\
12.4058333333333	-4\\
12.4113888888889	-4\\
12.4169444444444	-4\\
12.4225	-2\\
12.4280555555556	-2\\
12.4336111111111	-2\\
12.4391666666667	-5\\
12.4447222222222	-7\\
12.4502777777778	-7\\
12.4558333333333	-7\\
12.4613888888889	-7\\
12.4669444444444	-7\\
12.4725	-9\\
12.4780555555556	-10\\
12.4836111111111	-10\\
12.4891666666667	-11\\
12.4947222222222	-11\\
12.5002777777778	-3\\
12.5058333333333	-4\\
12.5113888888889	-4\\
12.5169444444444	-4\\
12.5225	-4\\
12.5280555555556	-5\\
12.5336111111111	-3\\
12.5391666666667	-1\\
12.5447222222222	-2\\
12.5502777777778	-4\\
12.5558333333333	-3\\
12.5613888888889	-3\\
12.5669444444444	-3\\
12.5725	-2\\
12.5780555555556	-2\\
12.5836111111111	-4\\
12.5891666666667	-4\\
12.5947222222222	-3\\
12.6002777777778	-3\\
12.6058333333333	-3\\
12.6113888888889	-3\\
12.6169444444444	-1\\
12.6225	-2\\
12.6280555555556	-3\\
12.6336111111111	-3\\
12.6391666666667	-3\\
12.6447222222222	-3\\
12.6502777777778	-2\\
12.6558333333333	-2\\
12.6613888888889	-2\\
12.6669444444444	-3\\
12.6725	-4\\
12.6780555555556	-4\\
12.6836111111111	-2\\
12.6891666666667	-2\\
12.6947222222222	-2\\
12.7002777777778	-2\\
12.7058333333333	-2\\
12.7113888888889	-2\\
12.7169444444444	-1\\
12.7225	-2\\
12.7280555555556	-2\\
12.7336111111111	-2\\
12.7391666666667	-2\\
12.7447222222222	-3\\
12.7502777777778	-3\\
12.7558333333333	-3\\
12.7613888888889	-3\\
12.7669444444444	-3\\
12.7725	-3\\
12.7780555555556	-4\\
12.7836111111111	-4\\
12.7891666666667	-4\\
12.7947222222222	-2\\
12.8002777777778	-2\\
12.8058333333333	-2\\
12.8113888888889	-2\\
12.8169444444444	-2\\
12.8225	-3\\
12.8280555555556	-3\\
12.8336111111111	-3\\
12.8391666666667	-3\\
12.8447222222222	-4\\
12.8502777777778	-5\\
12.8558333333333	-5\\
12.8613888888889	-6\\
12.8669444444444	-7\\
12.8725	-7\\
12.8780555555556	-3\\
12.8836111111111	-2\\
12.8891666666667	-2\\
12.8947222222222	-2\\
12.9002777777778	-2\\
12.9058333333333	-3\\
12.9113888888889	-4\\
12.9169444444444	-2\\
12.9225	-2\\
12.9280555555556	-2\\
12.9336111111111	-2\\
12.9391666666667	-2\\
12.9447222222222	-2\\
12.9502777777778	-2\\
12.9558333333333	-3\\
12.9613888888889	-5\\
12.9669444444444	-5\\
12.9725	-2\\
12.9780555555556	-2\\
12.9836111111111	-3\\
12.9891666666667	-3\\
12.9947222222222	-3\\
13.0002777777778	-2\\
13.0058333333333	-2\\
13.0113888888889	-2\\
13.0169444444444	-2\\
13.0225	-2\\
13.0280555555556	-2\\
13.0336111111111	-2\\
13.0391666666667	-2\\
13.0447222222222	-2\\
13.0502777777778	-4\\
13.0558333333333	-4\\
13.0613888888889	-3\\
13.0669444444444	-3\\
13.0725	-3\\
13.0780555555556	-4\\
13.0836111111111	-4\\
13.0891666666667	-4\\
13.0947222222222	-5\\
13.1002777777778	-5\\
13.1058333333333	3\\
13.1113888888889	3\\
13.1169444444444	7\\
13.1225	2\\
13.1280555555556	4\\
13.1336111111111	4\\
13.1391666666667	5\\
13.1447222222222	5\\
13.1502777777778	7\\
13.1558333333333	7\\
13.1613888888889	9\\
13.1669444444444	9\\
13.1725	9\\
13.1780555555556	9\\
13.1836111111111	9\\
13.1891666666667	9\\
13.1947222222222	9\\
13.2002777777778	9\\
13.2058333333333	9\\
13.2113888888889	10\\
13.2169444444444	9\\
13.2225	9\\
13.2280555555556	10\\
13.2336111111111	13\\
13.2391666666667	15\\
13.2447222222222	18\\
13.2502777777778	18\\
13.2558333333333	20\\
13.2613888888889	20\\
13.2669444444444	20\\
13.2725	20\\
13.2780555555556	20\\
13.2836111111111	20\\
13.2891666666667	20\\
13.2947222222222	20\\
13.3002777777778	20\\
13.3058333333333	19\\
13.3113888888889	20\\
13.3169444444444	2\\
13.3225	2\\
13.3280555555556	5\\
13.3336111111111	2\\
13.3391666666667	5\\
13.3447222222222	3\\
13.3502777777778	4\\
13.3558333333333	4\\
13.3613888888889	4\\
13.3669444444444	4\\
13.3725	4\\
13.3780555555556	6\\
13.3836111111111	9\\
13.3891666666667	9\\
13.3947222222222	10\\
13.4002777777778	15\\
13.4058333333333	18\\
13.4113888888889	20\\
13.4169444444444	2\\
13.4225	2\\
13.4280555555556	2\\
13.4336111111111	2\\
13.4391666666667	2\\
13.4447222222222	2\\
13.4502777777778	2\\
13.4558333333333	4\\
13.4613888888889	4\\
13.4669444444444	4\\
13.4725	2\\
13.4780555555556	2\\
13.4836111111111	2\\
13.4891666666667	3\\
13.4947222222222	2\\
13.5002777777778	5\\
13.5058333333333	6\\
13.5113888888889	2\\
13.5169444444444	4\\
13.5225	2\\
13.5280555555556	2\\
13.5336111111111	3\\
13.5391666666667	3\\
13.5447222222222	3\\
13.5502777777778	2\\
13.5558333333333	2\\
13.5613888888889	2\\
13.5669444444444	2\\
13.5725	5\\
13.5780555555556	2\\
13.5836111111111	2\\
13.5891666666667	2\\
13.5947222222222	2\\
13.6002777777778	2\\
13.6058333333333	2\\
13.6113888888889	2\\
13.6169444444444	2\\
13.6225	2\\
13.6280555555556	2\\
13.6336111111111	2\\
13.6391666666667	4\\
13.6447222222222	4\\
13.6502777777778	4\\
13.6558333333333	4\\
13.6613888888889	4\\
13.6669444444444	4\\
13.6725	4\\
13.6780555555556	4\\
13.6836111111111	4\\
13.6891666666667	4\\
13.6947222222222	2\\
13.7002777777778	2\\
13.7058333333333	2\\
13.7113888888889	2\\
13.7169444444444	3\\
13.7225	4\\
13.7280555555556	4\\
13.7336111111111	3\\
13.7391666666667	4\\
13.7447222222222	4\\
13.7502777777778	5\\
13.7558333333333	6\\
13.7613888888889	6\\
13.7669444444444	6\\
13.7725	6\\
13.7780555555556	5\\
13.7836111111111	5\\
13.7891666666667	2\\
13.7947222222222	2\\
13.8002777777778	3\\
13.8058333333333	4\\
13.8113888888889	4\\
13.8169444444444	4\\
13.8225	4\\
13.8280555555556	4\\
13.8336111111111	5\\
13.8391666666667	5\\
13.8447222222222	5\\
13.8502777777778	5\\
13.8558333333333	3\\
13.8613888888889	3\\
13.8669444444444	3\\
13.8725	2\\
13.8780555555556	2\\
13.8836111111111	2\\
13.8891666666667	3\\
13.8947222222222	3\\
13.9002777777778	3\\
13.9058333333333	3\\
13.9113888888889	3\\
13.9169444444444	2\\
13.9225	3\\
13.9280555555556	4\\
13.9336111111111	4\\
13.9391666666667	6\\
13.9447222222222	6\\
13.9502777777778	2\\
13.9558333333333	2\\
13.9613888888889	2\\
13.9669444444444	4\\
13.9725	2\\
13.9780555555556	2\\
13.9836111111111	3\\
13.9891666666667	3\\
13.9947222222222	4\\
14.0002777777778	7\\
14.0058333333333	7\\
14.0113888888889	2\\
14.0169444444444	4\\
14.0225	7\\
14.0280555555556	9\\
14.0336111111111	12\\
14.0391666666667	12\\
14.0447222222222	12\\
14.0502777777778	15\\
14.0558333333333	18\\
14.0613888888889	19\\
14.0669444444444	20\\
14.0725	20\\
14.0780555555556	20\\
14.0836111111111	20\\
14.0891666666667	20\\
14.0947222222222	20\\
14.1002777777778	20\\
14.1058333333333	20\\
14.1113888888889	20\\
14.1169444444444	20\\
14.1225	20\\
14.1280555555556	20\\
14.1336111111111	19\\
14.1391666666667	3\\
14.1447222222222	5\\
14.1502777777778	5\\
14.1558333333333	6\\
14.1613888888889	6\\
14.1669444444444	8\\
14.1725	2\\
14.1780555555556	3\\
14.1836111111111	4\\
14.1891666666667	5\\
14.1947222222222	7\\
14.2002777777778	8\\
14.2058333333333	8\\
14.2113888888889	8\\
14.2169444444444	8\\
14.2225	2\\
14.2280555555556	3\\
14.2336111111111	4\\
14.2391666666667	5\\
14.2447222222222	8\\
14.2502777777778	10\\
14.2558333333333	11\\
14.2613888888889	12\\
14.2669444444444	15\\
14.2725	16\\
14.2780555555556	17\\
14.2836111111111	17\\
14.2891666666667	19\\
14.2947222222222	20\\
14.3002777777778	20\\
14.3058333333333	20\\
14.3113888888889	20\\
14.3169444444444	20\\
14.3225	20\\
14.3280555555556	20\\
14.3336111111111	20\\
14.3391666666667	20\\
14.3447222222222	20\\
14.3502777777778	20\\
14.3558333333333	20\\
14.3613888888889	20\\
14.3669444444444	20\\
14.3725	20\\
14.3780555555556	2\\
14.3836111111111	2\\
14.3891666666667	2\\
14.3947222222222	2\\
14.4002777777778	3\\
14.4058333333333	2\\
14.4113888888889	2\\
14.4169444444444	3\\
14.4225	3\\
14.4280555555556	3\\
14.4336111111111	2\\
14.4391666666667	2\\
14.4447222222222	2\\
14.4502777777778	2\\
14.4558333333333	2\\
14.4613888888889	2\\
14.4669444444444	2\\
14.4725	2\\
14.4780555555556	2\\
14.4836111111111	2\\
14.4891666666667	2\\
14.4947222222222	2\\
14.5002777777778	2\\
14.5058333333333	2\\
14.5113888888889	2\\
14.5169444444444	2\\
14.5225	2\\
14.5280555555556	2\\
14.5336111111111	2\\
14.5391666666667	2\\
14.5447222222222	2\\
14.5502777777778	2\\
14.5558333333333	2\\
14.5613888888889	2\\
14.5669444444444	2\\
14.5725	2\\
14.5780555555556	5\\
14.5836111111111	2\\
14.5891666666667	3\\
14.5947222222222	4\\
14.6002777777778	4\\
14.6058333333333	6\\
14.6113888888889	6\\
14.6169444444444	6\\
14.6225	2\\
14.6280555555556	2\\
14.6336111111111	3\\
14.6391666666667	3\\
14.6447222222222	3\\
14.6502777777778	3\\
14.6558333333333	2\\
14.6613888888889	2\\
14.6669444444444	2\\
14.6725	1\\
14.6780555555556	3\\
14.6836111111111	-4\\
14.6891666666667	-4\\
14.6947222222222	-2\\
14.7002777777778	-2\\
14.7058333333333	-2\\
14.7113888888889	-2\\
14.7169444444444	-2\\
14.7225	-3\\
14.7280555555556	-2\\
14.7336111111111	-2\\
14.7391666666667	-2\\
14.7447222222222	-3\\
14.7502777777778	-2\\
14.7558333333333	-2\\
14.7613888888889	-3\\
14.7669444444444	-4\\
14.7725	-4\\
14.7780555555556	-4\\
14.7836111111111	-3\\
14.7891666666667	-3\\
14.7947222222222	-4\\
14.8002777777778	-4\\
14.8058333333333	-6\\
14.8113888888889	-4\\
14.8169444444444	-5\\
14.8225	-3\\
14.8280555555556	7\\
14.8336111111111	8\\
14.8391666666667	8\\
14.8447222222222	2\\
14.8502777777778	3\\
14.8558333333333	3\\
14.8613888888889	3\\
14.8669444444444	3\\
14.8725	3\\
14.8780555555556	3\\
14.8836111111111	3\\
14.8891666666667	3\\
14.8947222222222	3\\
14.9002777777778	2\\
14.9058333333333	2\\
14.9113888888889	3\\
14.9169444444444	3\\
14.9225	2\\
14.9280555555556	2\\
14.9336111111111	3\\
14.9391666666667	3\\
14.9447222222222	3\\
14.9502777777778	3\\
14.9558333333333	5\\
14.9613888888889	3\\
14.9669444444444	2\\
14.9725	2\\
14.9780555555556	2\\
14.9836111111111	2\\
14.9891666666667	2\\
14.9947222222222	3\\
15.0002777777778	3\\
15.0058333333333	3\\
15.0113888888889	2\\
15.0169444444444	2\\
15.0225	2\\
15.0280555555556	2\\
15.0336111111111	2\\
15.0391666666667	3\\
15.0447222222222	2\\
15.0502777777778	2\\
15.0558333333333	2\\
15.0613888888889	3\\
15.0669444444444	5\\
15.0725	6\\
15.0780555555556	6\\
15.0836111111111	7\\
15.0891666666667	2\\
15.0947222222222	2\\
15.1002777777778	5\\
15.1058333333333	5\\
15.1113888888889	7\\
15.1169444444444	7\\
15.1225	2\\
15.1280555555556	3\\
15.1336111111111	2\\
15.1391666666667	2\\
15.1447222222222	4\\
15.1502777777778	5\\
15.1558333333333	2\\
15.1613888888889	3\\
15.1669444444444	4\\
15.1725	1\\
15.1780555555556	2\\
15.1836111111111	2\\
15.1891666666667	2\\
15.1947222222222	3\\
15.2002777777778	5\\
15.2058333333333	5\\
15.2113888888889	2\\
15.2169444444444	3\\
15.2225	4\\
15.2280555555556	4\\
15.2336111111111	5\\
15.2391666666667	5\\
15.2447222222222	2\\
15.2502777777778	3\\
15.2558333333333	2\\
15.2613888888889	2\\
15.2669444444444	2\\
15.2725	2\\
15.2780555555556	7\\
15.2836111111111	3\\
15.2891666666667	3\\
15.2947222222222	3\\
15.3002777777778	4\\
15.3058333333333	3\\
15.3113888888889	4\\
15.3169444444444	6\\
15.3225	2\\
15.3280555555556	3\\
15.3336111111111	3\\
15.3391666666667	2\\
15.3447222222222	2\\
15.3502777777778	2\\
15.3558333333333	5\\
15.3613888888889	5\\
15.3669444444444	6\\
15.3725	2\\
15.3780555555556	2\\
15.3836111111111	2\\
15.3891666666667	4\\
15.3947222222222	7\\
15.4002777777778	2\\
15.4058333333333	2\\
15.4113888888889	2\\
15.4169444444444	4\\
15.4225	6\\
15.4280555555556	5\\
15.4336111111111	2\\
15.4391666666667	2\\
15.4447222222222	2\\
15.4502777777778	2\\
15.4558333333333	2\\
15.4613888888889	2\\
15.4669444444444	2\\
15.4725	3\\
15.4780555555556	3\\
15.4836111111111	3\\
15.4891666666667	3\\
15.4947222222222	3\\
15.5002777777778	4\\
15.5058333333333	4\\
15.5113888888889	2\\
15.5169444444444	2\\
15.5225	4\\
15.5280555555556	4\\
15.5336111111111	2\\
15.5391666666667	2\\
15.5447222222222	3\\
15.5502777777778	2\\
15.5558333333333	4\\
15.5613888888889	6\\
15.5669444444444	2\\
15.5725	4\\
15.5780555555556	4\\
15.5836111111111	3\\
15.5891666666667	2\\
15.5947222222222	2\\
15.6002777777778	2\\
15.6058333333333	3\\
15.6113888888889	5\\
15.6169444444444	2\\
15.6225	2\\
15.6280555555556	2\\
15.6336111111111	3\\
15.6391666666667	5\\
15.6447222222222	2\\
15.6502777777778	3\\
15.6558333333333	4\\
15.6613888888889	2\\
15.6669444444444	2\\
15.6725	4\\
15.6780555555556	4\\
15.6836111111111	4\\
15.6891666666667	5\\
15.6947222222222	5\\
15.7002777777778	3\\
15.7058333333333	3\\
15.7113888888889	3\\
15.7169444444444	3\\
15.7225	2\\
15.7280555555556	2\\
15.7336111111111	4\\
15.7391666666667	4\\
15.7447222222222	5\\
15.7502777777778	11\\
15.7558333333333	2\\
15.7613888888889	8\\
15.7669444444444	9\\
15.7725	2\\
15.7780555555556	2\\
15.7836111111111	5\\
15.7891666666667	2\\
15.7947222222222	3\\
15.8002777777778	2\\
15.8058333333333	2\\
15.8113888888889	2\\
15.8169444444444	2\\
15.8225	2\\
15.8280555555556	3\\
15.8336111111111	1\\
15.8391666666667	2\\
15.8447222222222	2\\
15.8502777777778	3\\
15.8558333333333	3\\
15.8613888888889	4\\
15.8669444444444	8\\
15.8725	9\\
15.8780555555556	9\\
15.8836111111111	9\\
15.8891666666667	10\\
15.8947222222222	10\\
15.9002777777778	2\\
15.9058333333333	2\\
15.9113888888889	2\\
15.9169444444444	3\\
15.9225	2\\
15.9280555555556	2\\
15.9336111111111	2\\
15.9391666666667	2\\
15.9447222222222	5\\
15.9502777777778	2\\
15.9558333333333	2\\
15.9613888888889	2\\
15.9669444444444	3\\
15.9725	2\\
15.9780555555556	2\\
15.9836111111111	2\\
15.9891666666667	3\\
15.9947222222222	7\\
};
\addlegendentry{Cts Stoch Ctrl w nFPC};

\addplot [color=cts_nFPC_plot_color,solid,line width=1.5pt]
  table[row sep=crcr]{%
9.50027777777778	1\\
9.50583333333333	3\\
9.51138888888889	4\\
9.51694444444444	3\\
9.5225	5\\
9.52805555555556	9\\
9.53361111111111	9\\
9.53916666666667	4\\
9.54472222222222	3\\
9.55027777777778	4\\
9.55583333333333	2\\
9.56138888888889	3\\
9.56694444444444	5\\
9.5725	5\\
9.57805555555555	5\\
9.58361111111111	6\\
9.58916666666667	7\\
9.59472222222222	6\\
9.60027777777778	3\\
9.60583333333333	3\\
9.61138888888889	2\\
9.61694444444444	2\\
9.6225	6\\
9.62805555555556	5\\
9.63361111111111	3\\
9.63916666666667	3\\
9.64472222222222	4\\
9.65027777777778	4\\
9.65583333333333	4\\
9.66138888888889	4\\
9.66694444444444	4\\
9.6725	3\\
9.67805555555555	2\\
9.68361111111111	3\\
9.68916666666667	2\\
9.69472222222222	2\\
9.70027777777778	2\\
9.70583333333333	2\\
9.71138888888889	3\\
9.71694444444444	2\\
9.7225	2\\
9.72805555555555	2\\
9.73361111111111	2\\
9.73916666666667	2\\
9.74472222222222	2\\
9.75027777777778	3\\
9.75583333333333	4\\
9.76138888888889	3\\
9.76694444444444	2\\
9.7725	3\\
9.77805555555556	3\\
9.78361111111111	2\\
9.78916666666667	4\\
9.79472222222222	3\\
9.80027777777778	5\\
9.80583333333333	5\\
9.81138888888889	4\\
9.81694444444444	3\\
9.8225	2\\
9.82805555555555	4\\
9.83361111111111	5\\
9.83916666666667	6\\
9.84472222222222	6\\
9.85027777777778	3\\
9.85583333333333	5\\
9.86138888888889	3\\
9.86694444444444	3\\
9.8725	4\\
9.87805555555556	3\\
9.88361111111111	3\\
9.88916666666667	3\\
9.89472222222222	3\\
9.90027777777778	3\\
9.90583333333333	2\\
9.91138888888889	3\\
9.91694444444444	3\\
9.9225	2\\
9.92805555555555	2\\
9.93361111111111	2\\
9.93916666666667	3\\
9.94472222222222	2\\
9.95027777777778	2\\
9.95583333333333	4\\
9.96138888888889	5\\
9.96694444444444	3\\
9.9725	4\\
9.97805555555555	3\\
9.98361111111111	3\\
9.98916666666667	4\\
9.99472222222222	4\\
10.0002777777778	4\\
10.0058333333333	4\\
10.0113888888889	4\\
10.0169444444444	2\\
10.0225	2\\
10.0280555555556	2\\
10.0336111111111	2\\
10.0391666666667	2\\
10.0447222222222	3\\
10.0502777777778	2\\
10.0558333333333	2\\
10.0613888888889	2\\
10.0669444444444	5\\
10.0725	5\\
10.0780555555556	4\\
10.0836111111111	4\\
10.0891666666667	4\\
10.0947222222222	4\\
10.1002777777778	3\\
10.1058333333333	2\\
10.1113888888889	2\\
10.1169444444444	4\\
10.1225	4\\
10.1280555555556	4\\
10.1336111111111	4\\
10.1391666666667	4\\
10.1447222222222	6\\
10.1502777777778	6\\
10.1558333333333	6\\
10.1613888888889	6\\
10.1669444444444	2\\
10.1725	3\\
10.1780555555556	3\\
10.1836111111111	3\\
10.1891666666667	3\\
10.1947222222222	3\\
10.2002777777778	3\\
10.2058333333333	3\\
10.2113888888889	3\\
10.2169444444444	3\\
10.2225	2\\
10.2280555555556	3\\
10.2336111111111	4\\
10.2391666666667	4\\
10.2447222222222	4\\
10.2502777777778	5\\
10.2558333333333	3\\
10.2613888888889	5\\
10.2669444444444	5\\
10.2725	4\\
10.2780555555556	5\\
10.2836111111111	3\\
10.2891666666667	2\\
10.2947222222222	2\\
10.3002777777778	3\\
10.3058333333333	3\\
10.3113888888889	3\\
10.3169444444444	5\\
10.3225	6\\
10.3280555555556	5\\
10.3336111111111	4\\
10.3391666666667	3\\
10.3447222222222	2\\
10.3502777777778	2\\
10.3558333333333	4\\
10.3613888888889	4\\
10.3669444444444	4\\
10.3725	4\\
10.3780555555556	4\\
10.3836111111111	4\\
10.3891666666667	3\\
10.3947222222222	3\\
10.4002777777778	2\\
10.4058333333333	2\\
10.4113888888889	2\\
10.4169444444444	2\\
10.4225	2\\
10.4280555555556	3\\
10.4336111111111	3\\
10.4391666666667	3\\
10.4447222222222	2\\
10.4502777777778	2\\
10.4558333333333	2\\
10.4613888888889	2\\
10.4669444444444	2\\
10.4725	4\\
10.4780555555556	3\\
10.4836111111111	3\\
10.4891666666667	5\\
10.4947222222222	4\\
10.5002777777778	4\\
10.5058333333333	3\\
10.5113888888889	2\\
10.5169444444444	2\\
10.5225	2\\
10.5280555555556	2\\
10.5336111111111	2\\
10.5391666666667	2\\
10.5447222222222	2\\
10.5502777777778	2\\
10.5558333333333	2\\
10.5613888888889	2\\
10.5669444444444	2\\
10.5725	2\\
10.5780555555556	2\\
10.5836111111111	2\\
10.5891666666667	2\\
10.5947222222222	2\\
10.6002777777778	3\\
10.6058333333333	3\\
10.6113888888889	4\\
10.6169444444444	4\\
10.6225	4\\
10.6280555555556	3\\
10.6336111111111	3\\
10.6391666666667	3\\
10.6447222222222	4\\
10.6502777777778	3\\
10.6558333333333	3\\
10.6613888888889	3\\
10.6669444444444	2\\
10.6725	4\\
10.6780555555556	2\\
10.6836111111111	2\\
10.6891666666667	2\\
10.6947222222222	2\\
10.7002777777778	3\\
10.7058333333333	2\\
10.7113888888889	2\\
10.7169444444444	2\\
10.7225	2\\
10.7280555555556	2\\
10.7336111111111	3\\
10.7391666666667	2\\
10.7447222222222	4\\
10.7502777777778	5\\
10.7558333333333	5\\
10.7613888888889	4\\
10.7669444444444	3\\
10.7725	2\\
10.7780555555556	2\\
10.7836111111111	2\\
10.7891666666667	2\\
10.7947222222222	2\\
10.8002777777778	2\\
10.8058333333333	2\\
10.8113888888889	2\\
10.8169444444444	3\\
10.8225	3\\
10.8280555555556	3\\
10.8336111111111	3\\
10.8391666666667	4\\
10.8447222222222	3\\
10.8502777777778	2\\
10.8558333333333	2\\
10.8613888888889	2\\
10.8669444444444	2\\
10.8725	2\\
10.8780555555556	2\\
10.8836111111111	2\\
10.8891666666667	2\\
10.8947222222222	2\\
10.9002777777778	2\\
10.9058333333333	2\\
10.9113888888889	4\\
10.9169444444444	4\\
10.9225	4\\
10.9280555555556	5\\
10.9336111111111	7\\
10.9391666666667	3\\
10.9447222222222	7\\
10.9502777777778	5\\
10.9558333333333	6\\
10.9613888888889	4\\
10.9669444444444	6\\
10.9725	7\\
10.9780555555556	10\\
10.9836111111111	10\\
10.9891666666667	13\\
10.9947222222222	13\\
11.0002777777778	14\\
11.0058333333333	14\\
11.0113888888889	15\\
11.0169444444444	14\\
11.0225	14\\
11.0280555555556	14\\
11.0336111111111	15\\
11.0391666666667	14\\
11.0447222222222	3\\
11.0502777777778	3\\
11.0558333333333	2\\
11.0613888888889	2\\
11.0669444444444	2\\
11.0725	3\\
11.0780555555556	3\\
11.0836111111111	3\\
11.0891666666667	3\\
11.0947222222222	3\\
11.1002777777778	3\\
11.1058333333333	3\\
11.1113888888889	2\\
11.1169444444444	2\\
11.1225	2\\
11.1280555555556	2\\
11.1336111111111	2\\
11.1391666666667	2\\
11.1447222222222	2\\
11.1502777777778	2\\
11.1558333333333	2\\
11.1613888888889	2\\
11.1669444444444	2\\
11.1725	2\\
11.1780555555556	2\\
11.1836111111111	2\\
11.1891666666667	3\\
11.1947222222222	3\\
11.2002777777778	2\\
11.2058333333333	3\\
11.2113888888889	4\\
11.2169444444444	3\\
11.2225	3\\
11.2280555555556	3\\
11.2336111111111	3\\
11.2391666666667	2\\
11.2447222222222	2\\
11.2502777777778	3\\
11.2558333333333	2\\
11.2613888888889	2\\
11.2669444444444	3\\
11.2725	3\\
11.2780555555556	2\\
11.2836111111111	2\\
11.2891666666667	2\\
11.2947222222222	2\\
11.3002777777778	2\\
11.3058333333333	2\\
11.3113888888889	2\\
11.3169444444444	2\\
11.3225	2\\
11.3280555555556	2\\
11.3336111111111	2\\
11.3391666666667	2\\
11.3447222222222	2\\
11.3502777777778	3\\
11.3558333333333	3\\
11.3613888888889	4\\
11.3669444444444	2\\
11.3725	3\\
11.3780555555556	3\\
11.3836111111111	3\\
11.3891666666667	3\\
11.3947222222222	3\\
11.4002777777778	3\\
11.4058333333333	6\\
11.4113888888889	6\\
11.4169444444444	8\\
11.4225	2\\
11.4280555555556	2\\
11.4336111111111	2\\
11.4391666666667	2\\
11.4447222222222	2\\
11.4502777777778	2\\
11.4558333333333	3\\
11.4613888888889	4\\
11.4669444444444	4\\
11.4725	5\\
11.4780555555556	5\\
11.4836111111111	4\\
11.4891666666667	4\\
11.4947222222222	4\\
11.5002777777778	4\\
11.5058333333333	4\\
11.5113888888889	2\\
11.5169444444444	3\\
11.5225	3\\
11.5280555555556	3\\
11.5336111111111	3\\
11.5391666666667	3\\
11.5447222222222	2\\
11.5502777777778	3\\
11.5558333333333	3\\
11.5613888888889	3\\
11.5669444444444	2\\
11.5725	2\\
11.5780555555556	2\\
11.5836111111111	2\\
11.5891666666667	3\\
11.5947222222222	3\\
11.6002777777778	4\\
11.6058333333333	5\\
11.6113888888889	5\\
11.6169444444444	5\\
11.6225	3\\
11.6280555555556	2\\
11.6336111111111	2\\
11.6391666666667	2\\
11.6447222222222	4\\
11.6502777777778	5\\
11.6558333333333	5\\
11.6613888888889	5\\
11.6669444444444	5\\
11.6725	7\\
11.6780555555556	8\\
11.6836111111111	8\\
11.6891666666667	3\\
11.6947222222222	3\\
11.7002777777778	3\\
11.7058333333333	3\\
11.7113888888889	3\\
11.7169444444444	3\\
11.7225	3\\
11.7280555555556	3\\
11.7336111111111	4\\
11.7391666666667	5\\
11.7447222222222	6\\
11.7502777777778	9\\
11.7558333333333	12\\
11.7613888888889	12\\
11.7669444444444	12\\
11.7725	13\\
11.7780555555556	14\\
11.7836111111111	15\\
11.7891666666667	3\\
11.7947222222222	3\\
11.8002777777778	3\\
11.8058333333333	3\\
11.8113888888889	3\\
11.8169444444444	3\\
11.8225	3\\
11.8280555555556	3\\
11.8336111111111	3\\
11.8391666666667	4\\
11.8447222222222	3\\
11.8502777777778	3\\
11.8558333333333	4\\
11.8613888888889	3\\
11.8669444444444	4\\
11.8725	4\\
11.8780555555556	5\\
11.8836111111111	6\\
11.8891666666667	8\\
11.8947222222222	7\\
11.9002777777778	8\\
11.9058333333333	10\\
11.9113888888889	11\\
11.9169444444444	11\\
11.9225	12\\
11.9280555555556	12\\
11.9336111111111	12\\
11.9391666666667	3\\
11.9447222222222	3\\
11.9502777777778	3\\
11.9558333333333	3\\
11.9613888888889	2\\
11.9669444444444	3\\
11.9725	3\\
11.9780555555556	4\\
11.9836111111111	3\\
11.9891666666667	3\\
11.9947222222222	3\\
12.0002777777778	3\\
12.0058333333333	5\\
12.0113888888889	7\\
12.0169444444444	8\\
12.0225	8\\
12.0280555555556	9\\
12.0336111111111	9\\
12.0391666666667	8\\
12.0447222222222	7\\
12.0502777777778	7\\
12.0558333333333	4\\
12.0613888888889	5\\
12.0669444444444	9\\
12.0725	9\\
12.0780555555556	10\\
12.0836111111111	9\\
12.0891666666667	8\\
12.0947222222222	9\\
12.1002777777778	9\\
12.1058333333333	9\\
12.1113888888889	4\\
12.1169444444444	4\\
12.1225	5\\
12.1280555555556	5\\
12.1336111111111	5\\
12.1391666666667	4\\
12.1447222222222	3\\
12.1502777777778	3\\
12.1558333333333	4\\
12.1613888888889	4\\
12.1669444444444	4\\
12.1725	4\\
12.1780555555556	4\\
12.1836111111111	4\\
12.1891666666667	3\\
12.1947222222222	3\\
12.2002777777778	3\\
12.2058333333333	3\\
12.2113888888889	3\\
12.2169444444444	3\\
12.2225	3\\
12.2280555555556	4\\
12.2336111111111	4\\
12.2391666666667	4\\
12.2447222222222	2\\
12.2502777777778	2\\
12.2558333333333	4\\
12.2613888888889	4\\
12.2669444444444	5\\
12.2725	5\\
12.2780555555556	5\\
12.2836111111111	4\\
12.2891666666667	4\\
12.2947222222222	4\\
12.3002777777778	4\\
12.3058333333333	3\\
12.3113888888889	3\\
12.3169444444444	3\\
12.3225	3\\
12.3280555555556	3\\
12.3336111111111	2\\
12.3391666666667	3\\
12.3447222222222	3\\
12.3502777777778	4\\
12.3558333333333	4\\
12.3613888888889	4\\
12.3669444444444	4\\
12.3725	4\\
12.3780555555556	4\\
12.3836111111111	4\\
12.3891666666667	4\\
12.3947222222222	4\\
12.4002777777778	4\\
12.4058333333333	4\\
12.4113888888889	4\\
12.4169444444444	4\\
12.4225	8\\
12.4280555555556	8\\
12.4336111111111	9\\
12.4391666666667	10\\
12.4447222222222	8\\
12.4502777777778	8\\
12.4558333333333	8\\
12.4613888888889	8\\
12.4669444444444	8\\
12.4725	5\\
12.4780555555556	5\\
12.4836111111111	5\\
12.4891666666667	6\\
12.4947222222222	6\\
12.5002777777778	7\\
12.5058333333333	7\\
12.5113888888889	7\\
12.5169444444444	7\\
12.5225	6\\
12.5280555555556	5\\
12.5336111111111	4\\
12.5391666666667	3\\
12.5447222222222	3\\
12.5502777777778	3\\
12.5558333333333	4\\
12.5613888888889	4\\
12.5669444444444	4\\
12.5725	4\\
12.5780555555556	4\\
12.5836111111111	4\\
12.5891666666667	4\\
12.5947222222222	4\\
12.6002777777778	4\\
12.6058333333333	4\\
12.6113888888889	3\\
12.6169444444444	3\\
12.6225	4\\
12.6280555555556	3\\
12.6336111111111	3\\
12.6391666666667	3\\
12.6447222222222	3\\
12.6502777777778	4\\
12.6558333333333	4\\
12.6613888888889	4\\
12.6669444444444	3\\
12.6725	2\\
12.6780555555556	2\\
12.6836111111111	3\\
12.6891666666667	3\\
12.6947222222222	3\\
12.7002777777778	4\\
12.7058333333333	4\\
12.7113888888889	4\\
12.7169444444444	4\\
12.7225	4\\
12.7280555555556	4\\
12.7336111111111	4\\
12.7391666666667	4\\
12.7447222222222	4\\
12.7502777777778	4\\
12.7558333333333	4\\
12.7613888888889	4\\
12.7669444444444	4\\
12.7725	4\\
12.7780555555556	5\\
12.7836111111111	5\\
12.7891666666667	5\\
12.7947222222222	4\\
12.8002777777778	4\\
12.8058333333333	4\\
12.8113888888889	4\\
12.8169444444444	4\\
12.8225	4\\
12.8280555555556	4\\
12.8336111111111	4\\
12.8391666666667	4\\
12.8447222222222	3\\
12.8502777777778	2\\
12.8558333333333	2\\
12.8613888888889	2\\
12.8669444444444	2\\
12.8725	2\\
12.8780555555556	2\\
12.8836111111111	4\\
12.8891666666667	4\\
12.8947222222222	5\\
12.9002777777778	3\\
12.9058333333333	2\\
12.9113888888889	2\\
12.9169444444444	2\\
12.9225	2\\
12.9280555555556	2\\
12.9336111111111	3\\
12.9391666666667	4\\
12.9447222222222	4\\
12.9502777777778	4\\
12.9558333333333	5\\
12.9613888888889	4\\
12.9669444444444	4\\
12.9725	4\\
12.9780555555556	4\\
12.9836111111111	3\\
12.9891666666667	3\\
12.9947222222222	3\\
13.0002777777778	4\\
13.0058333333333	4\\
13.0113888888889	4\\
13.0169444444444	4\\
13.0225	4\\
13.0280555555556	6\\
13.0336111111111	7\\
13.0391666666667	7\\
13.0447222222222	7\\
13.0502777777778	5\\
13.0558333333333	5\\
13.0613888888889	6\\
13.0669444444444	6\\
13.0725	6\\
13.0780555555556	5\\
13.0836111111111	5\\
13.0891666666667	5\\
13.0947222222222	4\\
13.1002777777778	4\\
13.1058333333333	3\\
13.1113888888889	3\\
13.1169444444444	4\\
13.1225	4\\
13.1280555555556	6\\
13.1336111111111	6\\
13.1391666666667	4\\
13.1447222222222	4\\
13.1502777777778	4\\
13.1558333333333	5\\
13.1613888888889	3\\
13.1669444444444	2\\
13.1725	2\\
13.1780555555556	2\\
13.1836111111111	2\\
13.1891666666667	2\\
13.1947222222222	2\\
13.2002777777778	2\\
13.2058333333333	2\\
13.2113888888889	3\\
13.2169444444444	2\\
13.2225	2\\
13.2280555555556	3\\
13.2336111111111	4\\
13.2391666666667	4\\
13.2447222222222	5\\
13.2502777777778	5\\
13.2558333333333	4\\
13.2613888888889	5\\
13.2669444444444	5\\
13.2725	5\\
13.2780555555556	6\\
13.2836111111111	4\\
13.2891666666667	4\\
13.2947222222222	4\\
13.3002777777778	4\\
13.3058333333333	4\\
13.3113888888889	6\\
13.3169444444444	3\\
13.3225	3\\
13.3280555555556	6\\
13.3336111111111	3\\
13.3391666666667	3\\
13.3447222222222	5\\
13.3502777777778	5\\
13.3558333333333	5\\
13.3613888888889	5\\
13.3669444444444	5\\
13.3725	5\\
13.3780555555556	2\\
13.3836111111111	5\\
13.3891666666667	5\\
13.3947222222222	6\\
13.4002777777778	11\\
13.4058333333333	15\\
13.4113888888889	18\\
13.4169444444444	19\\
13.4225	19\\
13.4280555555556	18\\
13.4336111111111	17\\
13.4391666666667	17\\
13.4447222222222	17\\
13.4502777777778	17\\
13.4558333333333	19\\
13.4613888888889	19\\
13.4669444444444	19\\
13.4725	5\\
13.4780555555556	5\\
13.4836111111111	5\\
13.4891666666667	6\\
13.4947222222222	6\\
13.5002777777778	5\\
13.5058333333333	6\\
13.5113888888889	4\\
13.5169444444444	4\\
13.5225	3\\
13.5280555555556	3\\
13.5336111111111	3\\
13.5391666666667	3\\
13.5447222222222	3\\
13.5502777777778	2\\
13.5558333333333	2\\
13.5613888888889	2\\
13.5669444444444	2\\
13.5725	3\\
13.5780555555556	3\\
13.5836111111111	3\\
13.5891666666667	3\\
13.5947222222222	3\\
13.6002777777778	2\\
13.6058333333333	2\\
13.6113888888889	2\\
13.6169444444444	2\\
13.6225	2\\
13.6280555555556	2\\
13.6336111111111	2\\
13.6391666666667	4\\
13.6447222222222	4\\
13.6502777777778	4\\
13.6558333333333	4\\
13.6613888888889	4\\
13.6669444444444	4\\
13.6725	4\\
13.6780555555556	4\\
13.6836111111111	4\\
13.6891666666667	3\\
13.6947222222222	2\\
13.7002777777778	3\\
13.7058333333333	3\\
13.7113888888889	3\\
13.7169444444444	4\\
13.7225	6\\
13.7280555555556	5\\
13.7336111111111	3\\
13.7391666666667	4\\
13.7447222222222	5\\
13.7502777777778	3\\
13.7558333333333	3\\
13.7613888888889	3\\
13.7669444444444	3\\
13.7725	3\\
13.7780555555556	2\\
13.7836111111111	2\\
13.7891666666667	2\\
13.7947222222222	2\\
13.8002777777778	3\\
13.8058333333333	4\\
13.8113888888889	4\\
13.8169444444444	4\\
13.8225	4\\
13.8280555555556	4\\
13.8336111111111	5\\
13.8391666666667	5\\
13.8447222222222	5\\
13.8502777777778	5\\
13.8558333333333	4\\
13.8613888888889	4\\
13.8669444444444	3\\
13.8725	4\\
13.8780555555556	4\\
13.8836111111111	4\\
13.8891666666667	3\\
13.8947222222222	3\\
13.9002777777778	3\\
13.9058333333333	3\\
13.9113888888889	3\\
13.9169444444444	4\\
13.9225	5\\
13.9280555555556	6\\
13.9336111111111	6\\
13.9391666666667	7\\
13.9447222222222	7\\
13.9502777777778	7\\
13.9558333333333	7\\
13.9613888888889	7\\
13.9669444444444	4\\
13.9725	3\\
13.9780555555556	3\\
13.9836111111111	5\\
13.9891666666667	5\\
13.9947222222222	6\\
14.0002777777778	8\\
14.0058333333333	8\\
14.0113888888889	7\\
14.0169444444444	9\\
14.0225	10\\
14.0280555555556	12\\
14.0336111111111	13\\
14.0391666666667	13\\
14.0447222222222	14\\
14.0502777777778	17\\
14.0558333333333	19\\
14.0613888888889	20\\
14.0669444444444	20\\
14.0725	20\\
14.0780555555556	20\\
14.0836111111111	20\\
14.0891666666667	20\\
14.0947222222222	20\\
14.1002777777778	20\\
14.1058333333333	20\\
14.1113888888889	20\\
14.1169444444444	20\\
14.1225	4\\
14.1280555555556	4\\
14.1336111111111	5\\
14.1391666666667	3\\
14.1447222222222	4\\
14.1502777777778	4\\
14.1558333333333	7\\
14.1613888888889	10\\
14.1669444444444	13\\
14.1725	14\\
14.1780555555556	15\\
14.1836111111111	15\\
14.1891666666667	15\\
14.1947222222222	16\\
14.2002777777778	17\\
14.2058333333333	17\\
14.2113888888889	17\\
14.2169444444444	17\\
14.2225	18\\
14.2280555555556	19\\
14.2336111111111	20\\
14.2391666666667	20\\
14.2447222222222	20\\
14.2502777777778	20\\
14.2558333333333	20\\
14.2613888888889	20\\
14.2669444444444	20\\
14.2725	20\\
14.2780555555556	20\\
14.2836111111111	3\\
14.2891666666667	3\\
14.2947222222222	5\\
14.3002777777778	7\\
14.3058333333333	8\\
14.3113888888889	9\\
14.3169444444444	12\\
14.3225	12\\
14.3280555555556	15\\
14.3336111111111	19\\
14.3391666666667	19\\
14.3447222222222	20\\
14.3502777777778	20\\
14.3558333333333	20\\
14.3613888888889	3\\
14.3669444444444	3\\
14.3725	4\\
14.3780555555556	3\\
14.3836111111111	2\\
14.3891666666667	2\\
14.3947222222222	2\\
14.4002777777778	3\\
14.4058333333333	2\\
14.4113888888889	2\\
14.4169444444444	3\\
14.4225	3\\
14.4280555555556	3\\
14.4336111111111	3\\
14.4391666666667	2\\
14.4447222222222	2\\
14.4502777777778	2\\
14.4558333333333	2\\
14.4613888888889	2\\
14.4669444444444	3\\
14.4725	3\\
14.4780555555556	3\\
14.4836111111111	3\\
14.4891666666667	3\\
14.4947222222222	3\\
14.5002777777778	3\\
14.5058333333333	2\\
14.5113888888889	2\\
14.5169444444444	2\\
14.5225	2\\
14.5280555555556	2\\
14.5336111111111	2\\
14.5391666666667	2\\
14.5447222222222	2\\
14.5502777777778	3\\
14.5558333333333	3\\
14.5613888888889	3\\
14.5669444444444	2\\
14.5725	2\\
14.5780555555556	4\\
14.5836111111111	3\\
14.5891666666667	3\\
14.5947222222222	4\\
14.6002777777778	4\\
14.6058333333333	6\\
14.6113888888889	6\\
14.6169444444444	6\\
14.6225	5\\
14.6280555555556	5\\
14.6336111111111	4\\
14.6391666666667	4\\
14.6447222222222	4\\
14.6502777777778	3\\
14.6558333333333	2\\
14.6613888888889	2\\
14.6669444444444	2\\
14.6725	2\\
14.6780555555556	3\\
14.6836111111111	2\\
14.6891666666667	2\\
14.6947222222222	4\\
14.7002777777778	4\\
14.7058333333333	4\\
14.7113888888889	4\\
14.7169444444444	4\\
14.7225	3\\
14.7280555555556	2\\
14.7336111111111	2\\
14.7391666666667	4\\
14.7447222222222	3\\
14.7502777777778	2\\
14.7558333333333	3\\
14.7613888888889	2\\
14.7669444444444	2\\
14.7725	2\\
14.7780555555556	2\\
14.7836111111111	3\\
14.7891666666667	3\\
14.7947222222222	2\\
14.8002777777778	2\\
14.8058333333333	2\\
14.8113888888889	2\\
14.8169444444444	3\\
14.8225	4\\
14.8280555555556	6\\
14.8336111111111	5\\
14.8391666666667	6\\
14.8447222222222	4\\
14.8502777777778	5\\
14.8558333333333	5\\
14.8613888888889	5\\
14.8669444444444	5\\
14.8725	5\\
14.8780555555556	5\\
14.8836111111111	5\\
14.8891666666667	5\\
14.8947222222222	5\\
14.9002777777778	2\\
14.9058333333333	2\\
14.9113888888889	4\\
14.9169444444444	5\\
14.9225	5\\
14.9280555555556	5\\
14.9336111111111	4\\
14.9391666666667	4\\
14.9447222222222	4\\
14.9502777777778	4\\
14.9558333333333	3\\
14.9613888888889	3\\
14.9669444444444	2\\
14.9725	2\\
14.9780555555556	2\\
14.9836111111111	2\\
14.9891666666667	2\\
14.9947222222222	3\\
15.0002777777778	3\\
15.0058333333333	3\\
15.0113888888889	3\\
15.0169444444444	3\\
15.0225	3\\
15.0280555555556	3\\
15.0336111111111	3\\
15.0391666666667	3\\
15.0447222222222	2\\
15.0502777777778	2\\
15.0558333333333	2\\
15.0613888888889	3\\
15.0669444444444	4\\
15.0725	5\\
15.0780555555556	5\\
15.0836111111111	6\\
15.0891666666667	2\\
15.0947222222222	2\\
15.1002777777778	6\\
15.1058333333333	6\\
15.1113888888889	4\\
15.1169444444444	4\\
15.1225	3\\
15.1280555555556	4\\
15.1336111111111	2\\
15.1391666666667	2\\
15.1447222222222	4\\
15.1502777777778	4\\
15.1558333333333	3\\
15.1613888888889	4\\
15.1669444444444	3\\
15.1725	6\\
15.1780555555556	8\\
15.1836111111111	8\\
15.1891666666667	6\\
15.1947222222222	3\\
15.2002777777778	4\\
15.2058333333333	4\\
15.2113888888889	3\\
15.2169444444444	4\\
15.2225	3\\
15.2280555555556	3\\
15.2336111111111	6\\
15.2391666666667	6\\
15.2447222222222	3\\
15.2502777777778	3\\
15.2558333333333	2\\
15.2613888888889	3\\
15.2669444444444	3\\
15.2725	5\\
15.2780555555556	10\\
15.2836111111111	10\\
15.2891666666667	10\\
15.2947222222222	10\\
15.3002777777778	3\\
15.3058333333333	4\\
15.3113888888889	5\\
15.3169444444444	3\\
15.3225	2\\
15.3280555555556	3\\
15.3336111111111	3\\
15.3391666666667	2\\
15.3447222222222	2\\
15.3502777777778	2\\
15.3558333333333	4\\
15.3613888888889	4\\
15.3669444444444	4\\
15.3725	4\\
15.3780555555556	3\\
15.3836111111111	2\\
15.3891666666667	3\\
15.3947222222222	6\\
15.4002777777778	7\\
15.4058333333333	7\\
15.4113888888889	7\\
15.4169444444444	8\\
15.4225	9\\
15.4280555555556	8\\
15.4336111111111	7\\
15.4391666666667	5\\
15.4447222222222	3\\
15.4502777777778	3\\
15.4558333333333	3\\
15.4613888888889	2\\
15.4669444444444	2\\
15.4725	3\\
15.4780555555556	4\\
15.4836111111111	4\\
15.4891666666667	3\\
15.4947222222222	3\\
15.5002777777778	3\\
15.5058333333333	3\\
15.5113888888889	2\\
15.5169444444444	3\\
15.5225	4\\
15.5280555555556	4\\
15.5336111111111	3\\
15.5391666666667	3\\
15.5447222222222	4\\
15.5502777777778	4\\
15.5558333333333	3\\
15.5613888888889	3\\
15.5669444444444	3\\
15.5725	6\\
15.5780555555556	6\\
15.5836111111111	5\\
15.5891666666667	4\\
15.5947222222222	4\\
15.6002777777778	4\\
15.6058333333333	5\\
15.6113888888889	7\\
15.6169444444444	7\\
15.6225	7\\
15.6280555555556	6\\
15.6336111111111	4\\
15.6391666666667	4\\
15.6447222222222	2\\
15.6502777777778	3\\
15.6558333333333	3\\
15.6613888888889	4\\
15.6669444444444	2\\
15.6725	4\\
15.6780555555556	3\\
15.6836111111111	3\\
15.6891666666667	4\\
15.6947222222222	4\\
15.7002777777778	3\\
15.7058333333333	3\\
15.7113888888889	3\\
15.7169444444444	3\\
15.7225	2\\
15.7280555555556	2\\
15.7336111111111	3\\
15.7391666666667	3\\
15.7447222222222	4\\
15.7502777777778	3\\
15.7558333333333	4\\
15.7613888888889	7\\
15.7669444444444	7\\
15.7725	5\\
15.7780555555556	5\\
15.7836111111111	6\\
15.7891666666667	5\\
15.7947222222222	3\\
15.8002777777778	3\\
15.8058333333333	2\\
15.8113888888889	2\\
15.8169444444444	2\\
15.8225	2\\
15.8280555555556	3\\
15.8336111111111	2\\
15.8391666666667	2\\
15.8447222222222	2\\
15.8502777777778	2\\
15.8558333333333	2\\
15.8613888888889	3\\
15.8669444444444	3\\
15.8725	4\\
15.8780555555556	5\\
15.8836111111111	5\\
15.8891666666667	3\\
15.8947222222222	3\\
15.9002777777778	2\\
15.9058333333333	2\\
15.9113888888889	2\\
15.9169444444444	3\\
15.9225	2\\
15.9280555555556	2\\
15.9336111111111	4\\
15.9391666666667	2\\
15.9447222222222	4\\
15.9502777777778	2\\
15.9558333333333	2\\
15.9613888888889	2\\
15.9669444444444	3\\
15.9725	3\\
15.9780555555556	3\\
15.9836111111111	2\\
15.9891666666667	2\\
15.9947222222222	2\\
};
\addlegendentry{Dscr Stoch Ctrl};

\addplot [color=dscr_nFPC_plot_color,solid,line width=1.5pt]
  table[row sep=crcr]{%
9.50027777777778	-2\\
9.50583333333333	-2\\
9.51138888888889	-4\\
9.51694444444444	-3\\
9.5225	-2\\
9.52805555555556	-2\\
9.53361111111111	-3\\
9.53916666666667	-3\\
9.54472222222222	-2\\
9.55027777777778	-2\\
9.55583333333333	-5\\
9.56138888888889	-3\\
9.56694444444444	-3\\
9.5725	-2\\
9.57805555555555	-2\\
9.58361111111111	-3\\
9.58916666666667	-3\\
9.59472222222222	-5\\
9.60027777777778	-3\\
9.60583333333333	-3\\
9.61138888888889	-5\\
9.61694444444444	-3\\
9.6225	-3\\
9.62805555555556	-4\\
9.63361111111111	-3\\
9.63916666666667	-2\\
9.64472222222222	-3\\
9.65027777777778	-2\\
9.65583333333333	-2\\
9.66138888888889	-5\\
9.66694444444444	-6\\
9.6725	-5\\
9.67805555555555	-8\\
9.68361111111111	-9\\
9.68916666666667	-11\\
9.69472222222222	-5\\
9.70027777777778	-6\\
9.70583333333333	-7\\
9.71138888888889	-6\\
9.71694444444444	-5\\
9.7225	-9\\
9.72805555555555	-5\\
9.73361111111111	-3\\
9.73916666666667	-3\\
9.74472222222222	-3\\
9.75027777777778	-2\\
9.75583333333333	-3\\
9.76138888888889	-5\\
9.76694444444444	-8\\
9.7725	-4\\
9.77805555555556	-5\\
9.78361111111111	-6\\
9.78916666666667	-3\\
9.79472222222222	-3\\
9.80027777777778	-2\\
9.80583333333333	-2\\
9.81138888888889	-3\\
9.81694444444444	-5\\
9.8225	-3\\
9.82805555555555	-5\\
9.83361111111111	-4\\
9.83916666666667	-3\\
9.84472222222222	-2\\
9.85027777777778	-5\\
9.85583333333333	-5\\
9.86138888888889	-5\\
9.86694444444444	-6\\
9.8725	-3\\
9.87805555555556	-4\\
9.88361111111111	-2\\
9.88916666666667	-2\\
9.89472222222222	-3\\
9.90027777777778	-3\\
9.90583333333333	-4\\
9.91138888888889	-4\\
9.91694444444444	-3\\
9.9225	-4\\
9.92805555555555	-3\\
9.93361111111111	-4\\
9.93916666666667	-5\\
9.94472222222222	-4\\
9.95027777777778	-5\\
9.95583333333333	-3\\
9.96138888888889	-2\\
9.96694444444444	-5\\
9.9725	-5\\
9.97805555555555	-3\\
9.98361111111111	-3\\
9.98916666666667	-2\\
9.99472222222222	-3\\
10.0002777777778	-3\\
10.0058333333333	-3\\
10.0113888888889	-3\\
10.0169444444444	-4\\
10.0225	-6\\
10.0280555555556	-5\\
10.0336111111111	-7\\
10.0391666666667	-5\\
10.0447222222222	-8\\
10.0502777777778	-11\\
10.0558333333333	-14\\
10.0613888888889	-14\\
10.0669444444444	-5\\
10.0725	-5\\
10.0780555555556	-2\\
10.0836111111111	-2\\
10.0891666666667	-2\\
10.0947222222222	-2\\
10.1002777777778	-3\\
10.1058333333333	-5\\
10.1113888888889	-6\\
10.1169444444444	-5\\
10.1225	-5\\
10.1280555555556	-6\\
10.1336111111111	-6\\
10.1391666666667	-6\\
10.1447222222222	-4\\
10.1502777777778	-4\\
10.1558333333333	-4\\
10.1613888888889	-4\\
10.1669444444444	-4\\
10.1725	-5\\
10.1780555555556	-5\\
10.1836111111111	-4\\
10.1891666666667	-4\\
10.1947222222222	-4\\
10.2002777777778	-4\\
10.2058333333333	-4\\
10.2113888888889	-4\\
10.2169444444444	-4\\
10.2225	-3\\
10.2280555555556	-2\\
10.2336111111111	-3\\
10.2391666666667	-3\\
10.2447222222222	-3\\
10.2502777777778	-4\\
10.2558333333333	-4\\
10.2613888888889	-3\\
10.2669444444444	-4\\
10.2725	-3\\
10.2780555555556	-4\\
10.2836111111111	-4\\
10.2891666666667	-6\\
10.2947222222222	-6\\
10.3002777777778	-6\\
10.3058333333333	-6\\
10.3113888888889	-7\\
10.3169444444444	-6\\
10.3225	-6\\
10.3280555555556	-3\\
10.3336111111111	-3\\
10.3391666666667	-3\\
10.3447222222222	-3\\
10.3502777777778	-5\\
10.3558333333333	-4\\
10.3613888888889	-4\\
10.3669444444444	-4\\
10.3725	-4\\
10.3780555555556	-4\\
10.3836111111111	-4\\
10.3891666666667	-5\\
10.3947222222222	-5\\
10.4002777777778	-5\\
10.4058333333333	-4\\
10.4113888888889	-8\\
10.4169444444444	-8\\
10.4225	-8\\
10.4280555555556	-7\\
10.4336111111111	-6\\
10.4391666666667	-6\\
10.4447222222222	-6\\
10.4502777777778	-6\\
10.4558333333333	-6\\
10.4613888888889	-6\\
10.4669444444444	-7\\
10.4725	-7\\
10.4780555555556	-6\\
10.4836111111111	-6\\
10.4891666666667	-6\\
10.4947222222222	-7\\
10.5002777777778	-7\\
10.5058333333333	-9\\
10.5113888888889	-10\\
10.5169444444444	-11\\
10.5225	-11\\
10.5280555555556	-13\\
10.5336111111111	-17\\
10.5391666666667	-18\\
10.5447222222222	-20\\
10.5502777777778	-20\\
10.5558333333333	-17\\
10.5613888888889	-17\\
10.5669444444444	-7\\
10.5725	-7\\
10.5780555555556	-8\\
10.5836111111111	-6\\
10.5891666666667	-8\\
10.5947222222222	-9\\
10.6002777777778	-8\\
10.6058333333333	-8\\
10.6113888888889	-7\\
10.6169444444444	-7\\
10.6225	-7\\
10.6280555555556	-8\\
10.6336111111111	-8\\
10.6391666666667	-7\\
10.6447222222222	-7\\
10.6502777777778	-8\\
10.6558333333333	-8\\
10.6613888888889	-9\\
10.6669444444444	-10\\
10.6725	-9\\
10.6780555555556	-10\\
10.6836111111111	-12\\
10.6891666666667	-13\\
10.6947222222222	-13\\
10.7002777777778	-12\\
10.7058333333333	-13\\
10.7113888888889	-14\\
10.7169444444444	-14\\
10.7225	-14\\
10.7280555555556	-13\\
10.7336111111111	-12\\
10.7391666666667	-13\\
10.7447222222222	-11\\
10.7502777777778	-10\\
10.7558333333333	-10\\
10.7613888888889	-10\\
10.7669444444444	-11\\
10.7725	-12\\
10.7780555555556	-12\\
10.7836111111111	-12\\
10.7891666666667	-12\\
10.7947222222222	-13\\
10.8002777777778	-17\\
10.8058333333333	-19\\
10.8113888888889	-19\\
10.8169444444444	-18\\
10.8225	-18\\
10.8280555555556	-18\\
10.8336111111111	-18\\
10.8391666666667	-17\\
10.8447222222222	-6\\
10.8502777777778	-8\\
10.8558333333333	-5\\
10.8613888888889	-5\\
10.8669444444444	-6\\
10.8725	-8\\
10.8780555555556	-8\\
10.8836111111111	-8\\
10.8891666666667	-8\\
10.8947222222222	-8\\
10.9002777777778	-8\\
10.9058333333333	-8\\
10.9113888888889	-3\\
10.9169444444444	-3\\
10.9225	-2\\
10.9280555555556	-2\\
10.9336111111111	-2\\
10.9391666666667	-5\\
10.9447222222222	-2\\
10.9502777777778	-4\\
10.9558333333333	-4\\
10.9613888888889	-4\\
10.9669444444444	-2\\
10.9725	-2\\
10.9780555555556	-2\\
10.9836111111111	-2\\
10.9891666666667	-2\\
10.9947222222222	-3\\
11.0002777777778	-4\\
11.0058333333333	-4\\
11.0113888888889	-3\\
11.0169444444444	-4\\
11.0225	-5\\
11.0280555555556	-5\\
11.0336111111111	-4\\
11.0391666666667	-5\\
11.0447222222222	-6\\
11.0502777777778	-6\\
11.0558333333333	-8\\
11.0613888888889	-8\\
11.0669444444444	-8\\
11.0725	-9\\
11.0780555555556	-9\\
11.0836111111111	-8\\
11.0891666666667	-8\\
11.0947222222222	-8\\
11.1002777777778	-8\\
11.1058333333333	-9\\
11.1113888888889	-12\\
11.1169444444444	-14\\
11.1225	-14\\
11.1280555555556	-14\\
11.1336111111111	-15\\
11.1391666666667	-15\\
11.1447222222222	-16\\
11.1502777777778	-5\\
11.1558333333333	-5\\
11.1613888888889	-5\\
11.1669444444444	-6\\
11.1725	-7\\
11.1780555555556	-7\\
11.1836111111111	-8\\
11.1891666666667	-7\\
11.1947222222222	-7\\
11.2002777777778	-6\\
11.2058333333333	-5\\
11.2113888888889	-6\\
11.2169444444444	-7\\
11.2225	-7\\
11.2280555555556	-7\\
11.2336111111111	-7\\
11.2391666666667	-7\\
11.2447222222222	-8\\
11.2502777777778	-7\\
11.2558333333333	-3\\
11.2613888888889	-3\\
11.2669444444444	-4\\
11.2725	-4\\
11.2780555555556	-7\\
11.2836111111111	-7\\
11.2891666666667	-7\\
11.2947222222222	-6\\
11.3002777777778	-7\\
11.3058333333333	-7\\
11.3113888888889	-7\\
11.3169444444444	-8\\
11.3225	-8\\
11.3280555555556	-8\\
11.3336111111111	-9\\
11.3391666666667	-9\\
11.3447222222222	-9\\
11.3502777777778	-4\\
11.3558333333333	-4\\
11.3613888888889	-3\\
11.3669444444444	-3\\
11.3725	-4\\
11.3780555555556	-3\\
11.3836111111111	-3\\
11.3891666666667	-4\\
11.3947222222222	-4\\
11.4002777777778	-4\\
11.4058333333333	-2\\
11.4113888888889	-2\\
11.4169444444444	-2\\
11.4225	-5\\
11.4280555555556	-5\\
11.4336111111111	-7\\
11.4391666666667	-7\\
11.4447222222222	-7\\
11.4502777777778	-7\\
11.4558333333333	-6\\
11.4613888888889	-5\\
11.4669444444444	-5\\
11.4725	-4\\
11.4780555555556	-4\\
11.4836111111111	-5\\
11.4891666666667	-5\\
11.4947222222222	-5\\
11.5002777777778	-5\\
11.5058333333333	-5\\
11.5113888888889	-3\\
11.5169444444444	-2\\
11.5225	-2\\
11.5280555555556	-2\\
11.5336111111111	-2\\
11.5391666666667	-2\\
11.5447222222222	-3\\
11.5502777777778	-2\\
11.5558333333333	-2\\
11.5613888888889	-6\\
11.5669444444444	-9\\
11.5725	-5\\
11.5780555555556	-5\\
11.5836111111111	-5\\
11.5891666666667	-4\\
11.5947222222222	-5\\
11.6002777777778	-4\\
11.6058333333333	-3\\
11.6113888888889	-3\\
11.6169444444444	-3\\
11.6225	-2\\
11.6280555555556	-3\\
11.6336111111111	-3\\
11.6391666666667	-4\\
11.6447222222222	-3\\
11.6502777777778	-2\\
11.6558333333333	-2\\
11.6613888888889	-3\\
11.6669444444444	-3\\
11.6725	-2\\
11.6780555555556	-2\\
11.6836111111111	-2\\
11.6891666666667	-3\\
11.6947222222222	-3\\
11.7002777777778	-3\\
11.7058333333333	-3\\
11.7113888888889	-3\\
11.7169444444444	-3\\
11.7225	-3\\
11.7280555555556	-3\\
11.7336111111111	-2\\
11.7391666666667	-2\\
11.7447222222222	-2\\
11.7502777777778	-2\\
11.7558333333333	-2\\
11.7613888888889	-2\\
11.7669444444444	-2\\
11.7725	-2\\
11.7780555555556	-2\\
11.7836111111111	-2\\
11.7891666666667	-2\\
11.7947222222222	-3\\
11.8002777777778	-3\\
11.8058333333333	-3\\
11.8113888888889	-3\\
11.8169444444444	-3\\
11.8225	-3\\
11.8280555555556	-3\\
11.8336111111111	-3\\
11.8391666666667	-2\\
11.8447222222222	-2\\
11.8502777777778	-3\\
11.8558333333333	-2\\
11.8613888888889	-3\\
11.8669444444444	-2\\
11.8725	-3\\
11.8780555555556	-2\\
11.8836111111111	-2\\
11.8891666666667	-2\\
11.8947222222222	-3\\
11.9002777777778	-3\\
11.9058333333333	-3\\
11.9113888888889	-3\\
11.9169444444444	-3\\
11.9225	-2\\
11.9280555555556	-2\\
11.9336111111111	-2\\
11.9391666666667	-2\\
11.9447222222222	-2\\
11.9502777777778	-2\\
11.9558333333333	-2\\
11.9613888888889	-3\\
11.9669444444444	-4\\
11.9725	-4\\
11.9780555555556	-3\\
11.9836111111111	-5\\
11.9891666666667	-5\\
11.9947222222222	-5\\
12.0002777777778	-5\\
12.0058333333333	-3\\
12.0113888888889	-2\\
12.0169444444444	-2\\
12.0225	-2\\
12.0280555555556	-2\\
12.0336111111111	-2\\
12.0391666666667	-3\\
12.0447222222222	-4\\
12.0502777777778	-4\\
12.0558333333333	-3\\
12.0613888888889	-2\\
12.0669444444444	-2\\
12.0725	-2\\
12.0780555555556	-2\\
12.0836111111111	-3\\
12.0891666666667	-4\\
12.0947222222222	-4\\
12.1002777777778	-4\\
12.1058333333333	-5\\
12.1113888888889	-5\\
12.1169444444444	-4\\
12.1225	-3\\
12.1280555555556	-3\\
12.1336111111111	-3\\
12.1391666666667	-5\\
12.1447222222222	-5\\
12.1502777777778	-5\\
12.1558333333333	-5\\
12.1613888888889	-5\\
12.1669444444444	-5\\
12.1725	-5\\
12.1780555555556	-5\\
12.1836111111111	-5\\
12.1891666666667	-6\\
12.1947222222222	-5\\
12.2002777777778	-5\\
12.2058333333333	-5\\
12.2113888888889	-5\\
12.2169444444444	-5\\
12.2225	-5\\
12.2280555555556	-5\\
12.2336111111111	-5\\
12.2391666666667	-5\\
12.2447222222222	-6\\
12.2502777777778	-5\\
12.2558333333333	-4\\
12.2613888888889	-4\\
12.2669444444444	-4\\
12.2725	-4\\
12.2780555555556	-4\\
12.2836111111111	-3\\
12.2891666666667	-2\\
12.2947222222222	-2\\
12.3002777777778	-2\\
12.3058333333333	-3\\
12.3113888888889	-3\\
12.3169444444444	-3\\
12.3225	-3\\
12.3280555555556	-3\\
12.3336111111111	-3\\
12.3391666666667	-2\\
12.3447222222222	-2\\
12.3502777777778	-2\\
12.3558333333333	-2\\
12.3613888888889	-2\\
12.3669444444444	-3\\
12.3725	-4\\
12.3780555555556	-4\\
12.3836111111111	-4\\
12.3891666666667	-5\\
12.3947222222222	-5\\
12.4002777777778	-5\\
12.4058333333333	-5\\
12.4113888888889	-5\\
12.4169444444444	-5\\
12.4225	-3\\
12.4280555555556	-3\\
12.4336111111111	-3\\
12.4391666666667	-5\\
12.4447222222222	-7\\
12.4502777777778	-7\\
12.4558333333333	-7\\
12.4613888888889	-7\\
12.4669444444444	-7\\
12.4725	-9\\
12.4780555555556	-9\\
12.4836111111111	-9\\
12.4891666666667	-10\\
12.4947222222222	-10\\
12.5002777777778	-4\\
12.5058333333333	-5\\
12.5113888888889	-5\\
12.5169444444444	-5\\
12.5225	-5\\
12.5280555555556	-6\\
12.5336111111111	-6\\
12.5391666666667	-6\\
12.5447222222222	-6\\
12.5502777777778	-6\\
12.5558333333333	-5\\
12.5613888888889	-5\\
12.5669444444444	-5\\
12.5725	-4\\
12.5780555555556	-4\\
12.5836111111111	-5\\
12.5891666666667	-5\\
12.5947222222222	-4\\
12.6002777777778	-4\\
12.6058333333333	-4\\
12.6113888888889	-5\\
12.6169444444444	-4\\
12.6225	-4\\
12.6280555555556	-3\\
12.6336111111111	-3\\
12.6391666666667	-3\\
12.6447222222222	-3\\
12.6502777777778	-4\\
12.6558333333333	-4\\
12.6613888888889	-4\\
12.6669444444444	-5\\
12.6725	-5\\
12.6780555555556	-5\\
12.6836111111111	-5\\
12.6891666666667	-5\\
12.6947222222222	-5\\
12.7002777777778	-4\\
12.7058333333333	-4\\
12.7113888888889	-4\\
12.7169444444444	-4\\
12.7225	-4\\
12.7280555555556	-4\\
12.7336111111111	-4\\
12.7391666666667	-4\\
12.7447222222222	-5\\
12.7502777777778	-5\\
12.7558333333333	-5\\
12.7613888888889	-5\\
12.7669444444444	-5\\
12.7725	-5\\
12.7780555555556	-6\\
12.7836111111111	-6\\
12.7891666666667	-6\\
12.7947222222222	-7\\
12.8002777777778	-7\\
12.8058333333333	-7\\
12.8113888888889	-7\\
12.8169444444444	-7\\
12.8225	-7\\
12.8280555555556	-7\\
12.8336111111111	-7\\
12.8391666666667	-7\\
12.8447222222222	-8\\
12.8502777777778	-9\\
12.8558333333333	-3\\
12.8613888888889	-4\\
12.8669444444444	-4\\
12.8725	-4\\
12.8780555555556	-5\\
12.8836111111111	-4\\
12.8891666666667	-4\\
12.8947222222222	-4\\
12.9002777777778	-4\\
12.9058333333333	-4\\
12.9113888888889	-5\\
12.9169444444444	-5\\
12.9225	-5\\
12.9280555555556	-5\\
12.9336111111111	-5\\
12.9391666666667	-4\\
12.9447222222222	-4\\
12.9502777777778	-4\\
12.9558333333333	-3\\
12.9613888888889	-4\\
12.9669444444444	-4\\
12.9725	-4\\
12.9780555555556	-3\\
12.9836111111111	-4\\
12.9891666666667	-4\\
12.9947222222222	-4\\
13.0002777777778	-3\\
13.0058333333333	-3\\
13.0113888888889	-3\\
13.0169444444444	-3\\
13.0225	-3\\
13.0280555555556	-2\\
13.0336111111111	-2\\
13.0391666666667	-2\\
13.0447222222222	-2\\
13.0502777777778	-4\\
13.0558333333333	-4\\
13.0613888888889	-4\\
13.0669444444444	-4\\
13.0725	-4\\
13.0780555555556	-6\\
13.0836111111111	-6\\
13.0891666666667	-6\\
13.0947222222222	-7\\
13.1002777777778	-7\\
13.1058333333333	-6\\
13.1113888888889	-6\\
13.1169444444444	-5\\
13.1225	-5\\
13.1280555555556	-5\\
13.1336111111111	-5\\
13.1391666666667	-4\\
13.1447222222222	-4\\
13.1502777777778	-3\\
13.1558333333333	-3\\
13.1613888888889	-3\\
13.1669444444444	-5\\
13.1725	-5\\
13.1780555555556	-5\\
13.1836111111111	-5\\
13.1891666666667	-5\\
13.1947222222222	-5\\
13.2002777777778	-5\\
13.2058333333333	-5\\
13.2113888888889	-5\\
13.2169444444444	-6\\
13.2225	-6\\
13.2280555555556	-6\\
13.2336111111111	-5\\
13.2391666666667	-4\\
13.2447222222222	-3\\
13.2502777777778	-3\\
13.2558333333333	-3\\
13.2613888888889	-3\\
13.2669444444444	-3\\
13.2725	-2\\
13.2780555555556	-2\\
13.2836111111111	-2\\
13.2891666666667	-3\\
13.2947222222222	-2\\
13.3002777777778	-2\\
13.3058333333333	-3\\
13.3113888888889	-4\\
13.3169444444444	-5\\
13.3225	-6\\
13.3280555555556	-5\\
13.3336111111111	-7\\
13.3391666666667	-5\\
13.3447222222222	-3\\
13.3502777777778	-2\\
13.3558333333333	-2\\
13.3613888888889	-2\\
13.3669444444444	-2\\
13.3725	-2\\
13.3780555555556	-3\\
13.3836111111111	-2\\
13.3891666666667	-2\\
13.3947222222222	-2\\
13.4002777777778	-2\\
13.4058333333333	-2\\
13.4113888888889	-2\\
13.4169444444444	-3\\
13.4225	-3\\
13.4280555555556	-4\\
13.4336111111111	-5\\
13.4391666666667	-5\\
13.4447222222222	-5\\
13.4502777777778	-5\\
13.4558333333333	-3\\
13.4613888888889	-3\\
13.4669444444444	-5\\
13.4725	-3\\
13.4780555555556	-3\\
13.4836111111111	-3\\
13.4891666666667	-3\\
13.4947222222222	-3\\
13.5002777777778	-2\\
13.5058333333333	-2\\
13.5113888888889	-3\\
13.5169444444444	-2\\
13.5225	-3\\
13.5280555555556	-3\\
13.5336111111111	-6\\
13.5391666666667	-6\\
13.5447222222222	-6\\
13.5502777777778	-9\\
13.5558333333333	-11\\
13.5613888888889	-12\\
13.5669444444444	-14\\
13.5725	-3\\
13.5780555555556	-4\\
13.5836111111111	-4\\
13.5891666666667	-4\\
13.5947222222222	-4\\
13.6002777777778	-5\\
13.6058333333333	-5\\
13.6113888888889	-6\\
13.6169444444444	-6\\
13.6225	-6\\
13.6280555555556	-7\\
13.6336111111111	-7\\
13.6391666666667	-6\\
13.6447222222222	-6\\
13.6502777777778	-6\\
13.6558333333333	-6\\
13.6613888888889	-6\\
13.6669444444444	-6\\
13.6725	-6\\
13.6780555555556	-6\\
13.6836111111111	-6\\
13.6891666666667	-7\\
13.6947222222222	-3\\
13.7002777777778	-4\\
13.7058333333333	-4\\
13.7113888888889	-4\\
13.7169444444444	-3\\
13.7225	-3\\
13.7280555555556	-5\\
13.7336111111111	-5\\
13.7391666666667	-4\\
13.7447222222222	-3\\
13.7502777777778	-2\\
13.7558333333333	-2\\
13.7613888888889	-2\\
13.7669444444444	-2\\
13.7725	-2\\
13.7780555555556	-3\\
13.7836111111111	-3\\
13.7891666666667	-4\\
13.7947222222222	-5\\
13.8002777777778	-5\\
13.8058333333333	-4\\
13.8113888888889	-4\\
13.8169444444444	-4\\
13.8225	-4\\
13.8280555555556	-4\\
13.8336111111111	-3\\
13.8391666666667	-3\\
13.8447222222222	-3\\
13.8502777777778	-3\\
13.8558333333333	-5\\
13.8613888888889	-5\\
13.8669444444444	-6\\
13.8725	-5\\
13.8780555555556	-5\\
13.8836111111111	-5\\
13.8891666666667	-3\\
13.8947222222222	-3\\
13.9002777777778	-3\\
13.9058333333333	-3\\
13.9113888888889	-3\\
13.9169444444444	-2\\
13.9225	-2\\
13.9280555555556	-2\\
13.9336111111111	-2\\
13.9391666666667	-2\\
13.9447222222222	-2\\
13.9502777777778	-3\\
13.9558333333333	-3\\
13.9613888888889	-3\\
13.9669444444444	-2\\
13.9725	-3\\
13.9780555555556	-3\\
13.9836111111111	-4\\
13.9891666666667	-4\\
13.9947222222222	-4\\
14.0002777777778	-2\\
14.0058333333333	-2\\
14.0113888888889	-4\\
14.0169444444444	-4\\
14.0225	-2\\
14.0280555555556	-2\\
14.0336111111111	-2\\
14.0391666666667	-2\\
14.0447222222222	-2\\
14.0502777777778	-2\\
14.0558333333333	-2\\
14.0613888888889	-2\\
14.0669444444444	-2\\
14.0725	-3\\
14.0780555555556	-2\\
14.0836111111111	-2\\
14.0891666666667	-2\\
14.0947222222222	-2\\
14.1002777777778	-2\\
14.1058333333333	-2\\
14.1113888888889	-2\\
14.1169444444444	-2\\
14.1225	-2\\
14.1280555555556	-2\\
14.1336111111111	-2\\
14.1391666666667	-3\\
14.1447222222222	-2\\
14.1502777777778	-2\\
14.1558333333333	-2\\
14.1613888888889	-2\\
14.1669444444444	-2\\
14.1725	-2\\
14.1780555555556	-2\\
14.1836111111111	-2\\
14.1891666666667	-2\\
14.1947222222222	-2\\
14.2002777777778	-2\\
14.2058333333333	-2\\
14.2113888888889	-2\\
14.2169444444444	-2\\
14.2225	-1\\
14.2280555555556	-2\\
14.2336111111111	-2\\
14.2391666666667	-2\\
14.2447222222222	-2\\
14.2502777777778	-2\\
14.2558333333333	-2\\
14.2613888888889	-2\\
14.2669444444444	-2\\
14.2725	-2\\
14.2780555555556	-2\\
14.2836111111111	-2\\
14.2891666666667	-2\\
14.2947222222222	-2\\
14.3002777777778	-2\\
14.3058333333333	-2\\
14.3113888888889	-2\\
14.3169444444444	-2\\
14.3225	-2\\
14.3280555555556	-2\\
14.3336111111111	-2\\
14.3391666666667	-2\\
14.3447222222222	-2\\
14.3502777777778	-2\\
14.3558333333333	-2\\
14.3613888888889	-2\\
14.3669444444444	-2\\
14.3725	-2\\
14.3780555555556	-4\\
14.3836111111111	-5\\
14.3891666666667	-6\\
14.3947222222222	-6\\
14.4002777777778	-5\\
14.4058333333333	-6\\
14.4113888888889	-6\\
14.4169444444444	-5\\
14.4225	-5\\
14.4280555555556	-6\\
14.4336111111111	-6\\
14.4391666666667	-5\\
14.4447222222222	-7\\
14.4502777777778	-7\\
14.4558333333333	-9\\
14.4613888888889	-5\\
14.4669444444444	-3\\
14.4725	-3\\
14.4780555555556	-4\\
14.4836111111111	-4\\
14.4891666666667	-4\\
14.4947222222222	-4\\
14.5002777777778	-4\\
14.5058333333333	-6\\
14.5113888888889	-6\\
14.5169444444444	-7\\
14.5225	-6\\
14.5280555555556	-6\\
14.5336111111111	-6\\
14.5391666666667	-6\\
14.5447222222222	-6\\
14.5502777777778	-2\\
14.5558333333333	-2\\
14.5613888888889	-2\\
14.5669444444444	-5\\
14.5725	-6\\
14.5780555555556	-2\\
14.5836111111111	-3\\
14.5891666666667	-3\\
14.5947222222222	-3\\
14.6002777777778	-3\\
14.6058333333333	-2\\
14.6113888888889	-2\\
14.6169444444444	-2\\
14.6225	-3\\
14.6280555555556	-3\\
14.6336111111111	-3\\
14.6391666666667	-3\\
14.6447222222222	-3\\
14.6502777777778	-5\\
14.6558333333333	-5\\
14.6613888888889	-6\\
14.6669444444444	-7\\
14.6725	-3\\
14.6780555555556	-3\\
14.6836111111111	-5\\
14.6891666666667	-5\\
14.6947222222222	-3\\
14.7002777777778	-3\\
14.7058333333333	-3\\
14.7113888888889	-3\\
14.7169444444444	-3\\
14.7225	-3\\
14.7280555555556	-4\\
14.7336111111111	-4\\
14.7391666666667	-2\\
14.7447222222222	-3\\
14.7502777777778	-4\\
14.7558333333333	-2\\
14.7613888888889	-3\\
14.7669444444444	-4\\
14.7725	-4\\
14.7780555555556	-4\\
14.7836111111111	-4\\
14.7891666666667	-4\\
14.7947222222222	-5\\
14.8002777777778	-5\\
14.8058333333333	-6\\
14.8113888888889	-7\\
14.8169444444444	-9\\
14.8225	-10\\
14.8280555555556	-4\\
14.8336111111111	-3\\
14.8391666666667	-3\\
14.8447222222222	-5\\
14.8502777777778	-4\\
14.8558333333333	-4\\
14.8613888888889	-4\\
14.8669444444444	-4\\
14.8725	-4\\
14.8780555555556	-4\\
14.8836111111111	-4\\
14.8891666666667	-4\\
14.8947222222222	-4\\
14.9002777777778	-6\\
14.9058333333333	-6\\
14.9113888888889	-4\\
14.9169444444444	-6\\
14.9225	-6\\
14.9280555555556	-6\\
14.9336111111111	-7\\
14.9391666666667	-7\\
14.9447222222222	-7\\
14.9502777777778	-7\\
14.9558333333333	-6\\
14.9613888888889	-2\\
14.9669444444444	-4\\
14.9725	-5\\
14.9780555555556	-5\\
14.9836111111111	-5\\
14.9891666666667	-6\\
14.9947222222222	-7\\
15.0002777777778	-8\\
15.0058333333333	-8\\
15.0113888888889	-9\\
15.0169444444444	-9\\
15.0225	-9\\
15.0280555555556	-9\\
15.0336111111111	-9\\
15.0391666666667	-10\\
15.0447222222222	-4\\
15.0502777777778	-4\\
15.0558333333333	-5\\
15.0613888888889	-5\\
15.0669444444444	-3\\
15.0725	-3\\
15.0780555555556	-3\\
15.0836111111111	-3\\
15.0891666666667	-5\\
15.0947222222222	-3\\
15.1002777777778	-2\\
15.1058333333333	-2\\
15.1113888888889	-2\\
15.1169444444444	-2\\
15.1225	-3\\
15.1280555555556	-4\\
15.1336111111111	-6\\
15.1391666666667	-6\\
15.1447222222222	-5\\
15.1502777777778	-4\\
15.1558333333333	-5\\
15.1613888888889	-5\\
15.1669444444444	-6\\
15.1725	-3\\
15.1780555555556	-2\\
15.1836111111111	-2\\
15.1891666666667	-3\\
15.1947222222222	-3\\
15.2002777777778	-2\\
15.2058333333333	-2\\
15.2113888888889	-2\\
15.2169444444444	-2\\
15.2225	-4\\
15.2280555555556	-4\\
15.2336111111111	-2\\
15.2391666666667	-2\\
15.2447222222222	-2\\
15.2502777777778	-2\\
15.2558333333333	-6\\
15.2613888888889	-8\\
15.2669444444444	-5\\
15.2725	-3\\
15.2780555555556	-2\\
15.2836111111111	-3\\
15.2891666666667	-3\\
15.2947222222222	-3\\
15.3002777777778	-2\\
15.3058333333333	-2\\
15.3113888888889	-2\\
15.3169444444444	-2\\
15.3225	-5\\
15.3280555555556	-4\\
15.3336111111111	-4\\
15.3391666666667	-3\\
15.3447222222222	-3\\
15.3502777777778	-3\\
15.3558333333333	-2\\
15.3613888888889	-2\\
15.3669444444444	-2\\
15.3725	-3\\
15.3780555555556	-4\\
15.3836111111111	-5\\
15.3891666666667	-4\\
15.3947222222222	-2\\
15.4002777777778	-2\\
15.4058333333333	-2\\
15.4113888888889	-2\\
15.4169444444444	-2\\
15.4225	-2\\
15.4280555555556	-2\\
15.4336111111111	-3\\
15.4391666666667	-5\\
15.4447222222222	-5\\
15.4502777777778	-6\\
15.4558333333333	-6\\
15.4613888888889	-6\\
15.4669444444444	-7\\
15.4725	-6\\
15.4780555555556	-6\\
15.4836111111111	-6\\
15.4891666666667	-7\\
15.4947222222222	-7\\
15.5002777777778	-8\\
15.5058333333333	-8\\
15.5113888888889	-6\\
15.5169444444444	-5\\
15.5225	-4\\
15.5280555555556	-4\\
15.5336111111111	-5\\
15.5391666666667	-7\\
15.5447222222222	-9\\
15.5502777777778	-9\\
15.5558333333333	-8\\
15.5613888888889	-6\\
15.5669444444444	-6\\
15.5725	-4\\
15.5780555555556	-4\\
15.5836111111111	-5\\
15.5891666666667	-5\\
15.5947222222222	-5\\
15.6002777777778	-5\\
15.6058333333333	-5\\
15.6113888888889	-5\\
15.6169444444444	-3\\
15.6225	-6\\
15.6280555555556	-5\\
15.6336111111111	-4\\
15.6391666666667	-2\\
15.6447222222222	-4\\
15.6502777777778	-3\\
15.6558333333333	-3\\
15.6613888888889	-3\\
15.6669444444444	-4\\
15.6725	-3\\
15.6780555555556	-2\\
15.6836111111111	-2\\
15.6891666666667	-2\\
15.6947222222222	-2\\
15.7002777777778	-3\\
15.7058333333333	-3\\
15.7113888888889	-3\\
15.7169444444444	-3\\
15.7225	-4\\
15.7280555555556	-4\\
15.7336111111111	-2\\
15.7391666666667	-2\\
15.7447222222222	-2\\
15.7502777777778	-2\\
15.7558333333333	-3\\
15.7613888888889	-2\\
15.7669444444444	-3\\
15.7725	-5\\
15.7780555555556	-8\\
15.7836111111111	-7\\
15.7891666666667	-8\\
15.7947222222222	-7\\
15.8002777777778	-8\\
15.8058333333333	-10\\
15.8113888888889	-11\\
15.8169444444444	-13\\
15.8225	-13\\
15.8280555555556	-12\\
15.8336111111111	-12\\
15.8391666666667	-7\\
15.8447222222222	-7\\
15.8502777777778	-7\\
15.8558333333333	-7\\
15.8613888888889	-6\\
15.8669444444444	-3\\
15.8725	-3\\
15.8780555555556	-3\\
15.8836111111111	-2\\
15.8891666666667	-2\\
15.8947222222222	-2\\
15.9002777777778	-3\\
15.9058333333333	-5\\
15.9113888888889	-5\\
15.9169444444444	-6\\
15.9225	-8\\
15.9280555555556	-13\\
15.9336111111111	-4\\
15.9391666666667	-7\\
15.9447222222222	-6\\
15.9502777777778	-8\\
15.9558333333333	-9\\
15.9613888888889	-9\\
15.9669444444444	-9\\
15.9725	-10\\
15.9780555555556	-11\\
15.9836111111111	-5\\
15.9891666666667	-5\\
15.9947222222222	-2\\
};
\addlegendentry{Dscr Stoch Ctrl w nFPC};

\end{axis}
\end{tikzpicture}%

  \caption{Inventory comparison of the four stochastic control methods.}
  \label{fig:ORCL_comp4stoch_inv}
\end{subfigure}%
\caption{Comparison of the four stochastic control methods.}
\end{figure}
In \autoref{fig:ORCL_comp4stoch} we plot the normalized PnL for the four strategies, calibrated and backtested using data for \texttt{ORCL} from 2013-05-15. At a glance the four plots show obvious similarities in trajectory, as well as in the distinct spikes between 13h and 14.5h. Nevertheless the correlation of arithmetic returns, \autoref{tbl:ORCL_comp4stoch_corr}, shows that the strategies' returns were uncorrelated. Indeed, while the overall paths are similar, on close inspection the individual returns do show markedly different behavior.
\begin{table}[H]
\centering
\ra{1.2}
\begin{tabular}{@{} r *{4}{c} @{}}
\toprule
& Cts & \cellbreak{t}{c}{Cts \\ w nFPC} & Dscr & \cellbreak{t}{c}{Dscr \\ w nFPC} \\
\cmidrule{2-5}
Cts          &  1.0000  & & & \\
Cts w nFPC   & -0.0109  &  1.0000 &  & \\
Dscr         & -0.0120  &  0.0122 &   1.0000 &  \\
Dscr w nFPC  & -0.0015  &  0.0034 &  -0.0165 &   1.0000 \\
\bottomrule
\end{tabular}
\caption{Correlation of returns}
\label{tbl:ORCL_comp4stoch_corr}
\end{table}
However, we find instead that the returns are co-integrated. On running the Engle-Granger cointegration test with statistics computed using an augmented Dickey-Fuller test of residuals, the $\tau$-test and $z$-test both returned $p$-values of 0.001, thus rejecting the null hypothesis of no co-integration. Indeed, the co-integration relation plotted in \autoref{fig:cointeg_relation} displays stationarity, thus confirming the existence of a co-integration relation.
\begin{figure}[H]
  \centering
  \setlength\figureheight{0.2\linewidth} 
  \setlength\figurewidth{0.35\linewidth}
  \tikzsetnextfilename{cointeg_relation}
  % This file was created by matlab2tikz.
%
%The latest updates can be retrieved from
%  http://www.mathworks.com/matlabcentral/fileexchange/22022-matlab2tikz-matlab2tikz
%where you can also make suggestions and rate matlab2tikz.
%
\begin{tikzpicture}[trim axis left, trim axis right]

\begin{axis}[%
width=\figurewidth,
height=\figureheight,
at={(0\figurewidth,0\figureheight)},
scale only axis,
every outer x axis line/.append style={black},
every x tick label/.append style={font=\color{black}},
xmin=9.5,
xmax=16,
xlabel={Time (h)},
every outer y axis line/.append style={black},
every y tick label/.append style={font=\color{black}},
ymin=-0.03,
ymax=0.04,
ylabel={},
title={Cointegrating Relation},
axis background/.style={fill=white},
axis x line*=bottom,
axis y line*=left,
yticklabel style={
        /pgf/number format/fixed,
        /pgf/number format/precision=3
},
scaled y ticks=false,
legend style={legend cell align=left,align=left,draw=black,font=\small, legend pos=north west},
]
\addplot [color=blue,solid,line width=1.5pt,forget plot]
  table[row sep=crcr]{%
9.50027777777778	-0.00649349697383166\\
9.50305555555556	-0.00538515058711511\\
9.50583333333333	-0.000975442708040742\\
9.50861111111111	-0.00112799142180334\\
9.51138888888889	0.033729101479292\\
9.51416666666667	-0.00110771530981036\\
9.51694444444444	-0.00584896295802729\\
9.51972222222222	-0.00588648571818438\\
9.5225	-0.0116231692842139\\
9.52527777777778	-0.0038711819602893\\
9.52805555555556	-0.02480029916492\\
9.53083333333333	-0.0277847545105595\\
9.53361111111111	-0.0205461321681931\\
9.53638888888889	-0.00988560832347582\\
9.53916666666667	-0.00405778892009148\\
9.54194444444444	-0.00970791148360503\\
9.54472222222222	-0.0157985369079592\\
9.5475	-0.00658850531093523\\
9.55027777777778	-0.0103401195680479\\
9.55305555555555	-0.00816667039863216\\
9.55583333333333	-0.00816667039863216\\
9.55861111111111	-0.0101722716437027\\
9.56138888888889	-0.00303273107411231\\
9.56416666666667	-0.00215686989180905\\
9.56694444444444	-0.00209379193385206\\
9.56972222222222	-0.0011596755536112\\
9.5725	9.3435083667005e-05\\
9.57527777777778	9.3435083667005e-05\\
9.57805555555555	9.3435083667005e-05\\
9.58083333333333	-0.00374285038001263\\
9.58361111111111	-0.00374285038001263\\
9.58638888888889	-0.00139099666894348\\
9.58916666666667	-0.00380295827271597\\
9.59194444444444	-0.00742307463604285\\
9.59472222222222	-0.00742307463604285\\
9.5975	-0.00890684895506191\\
9.60027777777778	-0.00697228352898695\\
9.60305555555555	-0.00568865009693123\\
9.60583333333333	-0.0143880487683933\\
9.60861111111111	-0.0128752517477776\\
9.61138888888889	-0.00934896370826435\\
9.61416666666667	-0.00397948339594045\\
9.61694444444444	-0.00397948339594045\\
9.61972222222222	0.00101295422161863\\
9.6225	-0.00376745066858546\\
9.62527777777778	-0.00276375326382203\\
9.62805555555556	-0.00575349229346011\\
9.63083333333333	-0.0080918727077899\\
9.63361111111111	-0.00829552338694106\\
9.63638888888889	-0.00695175341036102\\
9.63916666666667	-0.00668907964820036\\
9.64194444444444	-0.00647065598072599\\
9.64472222222222	-0.00602884082137385\\
9.6475	-0.00602884082137385\\
9.65027777777778	-0.0066620904138623\\
9.65305555555556	-0.0061721596748715\\
9.65583333333333	-0.0061721596748715\\
9.65861111111111	-0.00504968986139192\\
9.66138888888889	-0.00437993924044874\\
9.66416666666667	-0.00452990000234495\\
9.66694444444444	-0.00415981457511554\\
9.66972222222222	-0.00428455285920534\\
9.6725	-0.00214494268584359\\
9.67527777777778	-0.00178879935937043\\
9.67805555555555	-0.00285553404383201\\
9.68083333333333	-0.00238670202173334\\
9.68361111111111	-0.000922304366886534\\
9.68638888888889	-0.000930588407873039\\
9.68916666666667	1.15333855854865e-05\\
9.69194444444444	-0.00339960761853483\\
9.69472222222222	-0.00478668506822644\\
9.6975	-0.00417679250844341\\
9.70027777777778	-0.00417679250844341\\
9.70305555555555	-0.00417679250844341\\
9.70583333333333	-0.00179314834201113\\
9.70861111111111	-0.00179314834201113\\
9.71138888888889	-0.00395530716453507\\
9.71416666666667	-0.00395530716453507\\
9.71694444444444	-0.00525571010900612\\
9.71972222222222	-0.00294732894062307\\
9.7225	-0.00201821541193409\\
9.72527777777778	-0.000695618940231374\\
9.72805555555555	-0.00324041257404949\\
9.73083333333333	-0.00324041257404949\\
9.73361111111111	-0.0061349406472448\\
9.73638888888889	-0.0061349406472448\\
9.73916666666667	-0.0061349406472448\\
9.74194444444444	-0.0061349406472448\\
9.74472222222222	-0.0061349406472448\\
9.7475	-0.0061349406472448\\
9.75027777777778	-0.00684919357992936\\
9.75305555555556	-0.00656934258469412\\
9.75583333333333	-0.00656934258469412\\
9.75861111111111	-0.00498561427943606\\
9.76138888888889	-0.00498561427943606\\
9.76416666666667	-0.00460330572331415\\
9.76694444444444	-0.00414779261303325\\
9.76972222222222	-0.00769271381746917\\
9.7725	-0.00750021104793655\\
9.77527777777778	-0.00742809625755005\\
9.77805555555556	-0.00742809625755005\\
9.78083333333333	-0.00742809625755005\\
9.78361111111111	-0.00753802167530077\\
9.78638888888889	-0.00724018222732601\\
9.78916666666667	-0.00724018222732601\\
9.79194444444444	-0.00722823138337043\\
9.79472222222222	-0.00722823138337043\\
9.7975	-0.00679810045739406\\
9.80027777777778	-0.0096812725722636\\
9.80305555555555	-0.00880757790236553\\
9.80583333333333	-0.00880757790236553\\
9.80861111111111	-0.00857392437367246\\
9.81138888888889	-0.0089700928972851\\
9.81416666666667	-0.0089700928972851\\
9.81694444444444	-0.00915299588466386\\
9.81972222222222	-0.009479693564149\\
9.8225	-0.00646698618787172\\
9.82527777777778	-0.00665422869736743\\
9.82805555555555	-0.00920097650195285\\
9.83083333333333	-0.00876412916700366\\
9.83361111111111	-0.00766345241438942\\
9.83638888888889	-0.00796846319736431\\
9.83916666666667	-0.00796846319736431\\
9.84194444444444	-0.00810539373882271\\
9.84472222222222	-0.00810539373882271\\
9.8475	-0.0071051550956986\\
9.85027777777778	-0.0074728698242274\\
9.85305555555555	-0.00806573341368087\\
9.85583333333333	-0.00806573341368087\\
9.85861111111111	-0.00806573341368087\\
9.86138888888889	-0.00742340836845611\\
9.86416666666667	-0.00751979133793359\\
9.86694444444444	-0.00654734720829235\\
9.86972222222222	-0.00742400199558782\\
9.8725	-0.00742400199558782\\
9.87527777777778	-0.00700315323183133\\
9.87805555555556	-0.00700315323183133\\
9.88083333333333	-0.00700315323183133\\
9.88361111111111	-0.00801767772623417\\
9.88638888888889	-0.00812691194266817\\
9.88916666666667	-0.00812691194266817\\
9.89194444444444	-0.00747378919964214\\
9.89472222222222	-0.00785995241283687\\
9.8975	-0.00785995241283687\\
9.90027777777778	-0.00785995241283687\\
9.90305555555556	-0.00752617881600386\\
9.90583333333333	-0.00763448846887603\\
9.90861111111111	-0.00763448846887603\\
9.91138888888889	-0.00793361637538804\\
9.91416666666667	-0.00793361637538804\\
9.91694444444444	-0.00869461776872314\\
9.91972222222222	-0.00802236341406048\\
9.9225	-0.00802236341406048\\
9.92527777777778	-0.00802236341406048\\
9.92805555555555	-0.00790099528600495\\
9.93083333333333	-0.00798008079389526\\
9.93361111111111	-0.00798008079389526\\
9.93638888888889	-0.00877751704671036\\
9.93916666666667	-0.00877751704671036\\
9.94194444444444	-0.00877751704671036\\
9.94472222222222	-0.00843802883182967\\
9.9475	-0.00843802883182967\\
9.95027777777778	-0.00792024960652324\\
9.95305555555555	-0.00881364009303782\\
9.95583333333333	-0.00910266827678017\\
9.95861111111111	-0.00674285549111066\\
9.96138888888889	-0.00674285549111066\\
9.96416666666667	-0.00970807727294066\\
9.96694444444444	-0.00910514051704815\\
9.96972222222222	-0.00856554841122204\\
9.9725	-0.00856554841122204\\
9.97527777777778	-0.00856554841122204\\
9.97805555555555	-0.0131945416871447\\
9.98083333333333	-0.0131945416871447\\
9.98361111111111	-0.0131945416871447\\
9.98638888888889	-0.0131945416871447\\
9.98916666666667	-0.0108453535867021\\
9.99194444444444	-0.0108453535867021\\
9.99472222222222	-0.00817646445886957\\
9.9975	-0.00817646445886957\\
10.0002777777778	-0.00817646445886957\\
10.0030555555556	-0.00817646445886957\\
10.0058333333333	-0.00817646445886957\\
10.0086111111111	-0.00817646445886957\\
10.0113888888889	-0.00817646445886957\\
10.0141666666667	-0.00678997619591088\\
10.0169444444444	-0.0101409741209769\\
10.0197222222222	-0.00948767442491721\\
10.0225	-0.00654160801987926\\
10.0252777777778	-0.00685203765986437\\
10.0280555555556	-0.00685203765986437\\
10.0308333333333	-0.00653037780473897\\
10.0336111111111	-0.0060265580232845\\
10.0363888888889	-0.0060265580232845\\
10.0391666666667	-0.00589917070041958\\
10.0419444444444	-0.00687265453356445\\
10.0447222222222	-0.00687265453356445\\
10.0475	-0.00579358018561683\\
10.0502777777778	-0.00564020672708305\\
10.0530555555556	-0.00564020672708305\\
10.0558333333333	-0.00637204540060725\\
10.0586111111111	-0.00637204540060725\\
10.0613888888889	-0.00637204540060725\\
10.0641666666667	-0.00474127014894181\\
10.0669444444444	-0.00476668300433708\\
10.0697222222222	-0.00476668300433708\\
10.0725	-0.00476668300433708\\
10.0752777777778	-0.00661303649942249\\
10.0780555555556	-0.00950791108629428\\
10.0808333333333	-0.00972633475376895\\
10.0836111111111	-0.00972633475376895\\
10.0863888888889	-0.00972633475376895\\
10.0891666666667	-0.00972633475376895\\
10.0919444444444	-0.00972633475376895\\
10.0947222222222	-0.00972633475376895\\
10.0975	-0.00972633475376895\\
10.1002777777778	-0.00263106991317219\\
10.1030555555556	-0.00263106991317219\\
10.1058333333333	-0.00415543053484159\\
10.1086111111111	-0.00415543053484159\\
10.1113888888889	-0.0060787430081996\\
10.1141666666667	-0.00683169397285731\\
10.1169444444444	-0.00683169397285731\\
10.1197222222222	-0.00683169397285731\\
10.1225	-0.00683169397285731\\
10.1252777777778	-0.0066051998390105\\
10.1280555555556	-0.00537955164566052\\
10.1308333333333	-0.00537955164566052\\
10.1336111111111	-0.00537955164566052\\
10.1363888888889	-0.00537955164566052\\
10.1391666666667	-0.00537955164566052\\
10.1419444444444	-0.00511713110277952\\
10.1447222222222	-0.00575665345529167\\
10.1475	-0.00575665345529167\\
10.1502777777778	-0.00575665345529167\\
10.1530555555556	-0.00575665345529167\\
10.1558333333333	-0.00575665345529167\\
10.1586111111111	-0.00575665345529167\\
10.1613888888889	-0.00575665345529167\\
10.1641666666667	-0.00575665345529167\\
10.1669444444444	-0.00528610871784833\\
10.1697222222222	-0.00528610871784833\\
10.1725	-0.00481933054907513\\
10.1752777777778	-0.00481933054907513\\
10.1780555555556	-0.00481933054907513\\
10.1808333333333	-0.00484842143966381\\
10.1836111111111	-0.00462663205365013\\
10.1863888888889	-0.00522502364903915\\
10.1891666666667	-0.00522502364903915\\
10.1919444444444	-0.00522502364903915\\
10.1947222222222	-0.00522502364903915\\
10.1975	-0.00522502364903915\\
10.2002777777778	-0.00544344731651352\\
10.2030555555556	-0.00544344731651352\\
10.2058333333333	-0.00544344731651352\\
10.2086111111111	-0.00544344731651352\\
10.2113888888889	-0.00544344731651352\\
10.2141666666667	-0.00544344731651352\\
10.2169444444444	-0.00544344731651352\\
10.2197222222222	-0.00562844084154815\\
10.2225	-0.00562844084154815\\
10.2252777777778	-0.00633370988372447\\
10.2280555555556	-0.00633370988372447\\
10.2308333333333	-0.00644405346317211\\
10.2336111111111	-0.00644405346317211\\
10.2363888888889	-0.00644405346317211\\
10.2391666666667	-0.00640200581719614\\
10.2419444444444	-0.00652650465448875\\
10.2447222222222	-0.00652650465448875\\
10.2475	-0.00652650465448875\\
10.2502777777778	-0.00627788831678242\\
10.2530555555556	-0.00608356642112058\\
10.2558333333333	-0.00635443015453961\\
10.2586111111111	-0.00509855793664975\\
10.2613888888889	-0.00599106925795246\\
10.2641666666667	-0.00517562662295784\\
10.2669444444444	-0.0056840892189849\\
10.2697222222222	-0.0056840892189849\\
10.2725	-0.00393599515159796\\
10.2752777777778	-0.00709643500722194\\
10.2780555555556	-0.00709643500722194\\
10.2808333333333	-0.00451667560135383\\
10.2836111111111	-0.00578628080000925\\
10.2863888888889	-0.00275057195773\\
10.2891666666667	-0.00275057195773\\
10.2919444444444	-0.00275057195773\\
10.2947222222222	-0.00275057195773\\
10.2975	-0.00113622257146076\\
10.3002777777778	-0.00546839851745495\\
10.3030555555556	-0.00546839851745495\\
10.3058333333333	-0.00546839851745495\\
10.3086111111111	-0.00443882559353184\\
10.3113888888889	-0.00677287732376407\\
10.3141666666667	-0.00677287732376407\\
10.3169444444444	-0.00447922200199391\\
10.3197222222222	-0.00738479577781859\\
10.3225	-0.00738479577781859\\
10.3252777777778	-0.00719899404313226\\
10.3280555555556	-0.0086837023715358\\
10.3308333333333	-0.00925863944672306\\
10.3336111111111	-0.00925863944672306\\
10.3363888888889	-0.00925863944672306\\
10.3391666666667	-0.00786752447863063\\
10.3419444444444	-0.00786752447863063\\
10.3447222222222	-0.00820823014091753\\
10.3475	-0.00708216419164919\\
10.3502777777778	-0.00610561550424188\\
10.3530555555556	-0.007669253049363\\
10.3558333333333	-0.0071971788351018\\
10.3586111111111	-0.0071971788351018\\
10.3613888888889	-0.0071971788351018\\
10.3641666666667	-0.0071971788351018\\
10.3669444444444	-0.0071971788351018\\
10.3697222222222	-0.0071971788351018\\
10.3725	-0.0071971788351018\\
10.3752777777778	-0.0071971788351018\\
10.3780555555556	-0.0071971788351018\\
10.3808333333333	-0.0071971788351018\\
10.3836111111111	-0.0071971788351018\\
10.3863888888889	-0.0071971788351018\\
10.3891666666667	-0.00649511156327611\\
10.3919444444444	-0.00649511156327611\\
10.3947222222222	-0.00649511156327611\\
10.3975	-0.00705180094231762\\
10.4002777777778	-0.00705180094231762\\
10.4030555555556	-0.00705180094231762\\
10.4058333333333	-0.00704869002944691\\
10.4086111111111	-0.00599982223520678\\
10.4113888888889	-0.00650972811361511\\
10.4141666666667	-0.00650972811361511\\
10.4169444444444	-0.00650972811361511\\
10.4197222222222	-0.00650972811361511\\
10.4225	-0.00650972811361511\\
10.4252777777778	-0.00642578996721999\\
10.4280555555556	-0.00642578996721999\\
10.4308333333333	-0.00606929785617492\\
10.4336111111111	-0.00606929785617492\\
10.4363888888889	-0.00606929785617492\\
10.4391666666667	-0.00606929785617492\\
10.4419444444444	-0.00527613633249867\\
10.4447222222222	-0.00478063515680082\\
10.4475	-0.00478063515680082\\
10.4502777777778	-0.00478063515680082\\
10.4530555555556	-0.00478063515680082\\
10.4558333333333	-0.00478063515680082\\
10.4586111111111	-0.00478063515680082\\
10.4613888888889	-0.00478063515680082\\
10.4641666666667	-0.00478063515680082\\
10.4669444444444	-0.00495805545718207\\
10.4697222222222	-0.00495805545718207\\
10.4725	-0.00457179762433508\\
10.4752777777778	-0.00457179762433508\\
10.4780555555556	-0.00400513358474151\\
10.4808333333333	-0.00400513358474151\\
10.4836111111111	-0.00400513358474151\\
10.4863888888889	-0.00389139301421087\\
10.4891666666667	-0.00400200655805228\\
10.4919444444444	-0.0045971854975179\\
10.4947222222222	-0.0045971854975179\\
10.4975	-0.0045971854975179\\
10.5002777777778	-0.0045971854975179\\
10.5030555555556	-0.00379306876907388\\
10.5058333333333	-0.00377390342264628\\
10.5086111111111	-0.00406837345048314\\
10.5113888888889	-0.00398097303184232\\
10.5141666666667	-0.00349883957181686\\
10.5169444444444	-0.00363845874796816\\
10.5197222222222	-0.0034200350804938\\
10.5225	-0.0034200350804938\\
10.5252777777778	-0.00297482911114948\\
10.5280555555556	-0.00297482911114948\\
10.5308333333333	-0.00323227862924958\\
10.5336111111111	-0.0032789997198349\\
10.5363888888889	-0.0032789997198349\\
10.5391666666667	-0.00372834458263956\\
10.5419444444444	-0.00397219211245276\\
10.5447222222222	-0.00482898623540353\\
10.5475	-0.00482898623540353\\
10.5502777777778	-0.00674805116573405\\
10.5530555555556	-0.00647106935959538\\
10.5558333333333	6.07705522919846e-05\\
10.5586111111111	6.07705522919846e-05\\
10.5613888888889	6.07705522919846e-05\\
10.5641666666667	0.00422987673227056\\
10.5669444444444	0.00468067190693647\\
10.5697222222222	0.00320006782046928\\
10.5725	0.00320006782046928\\
10.5752777777778	0.00216268069091\\
10.5780555555556	0.00189427569950093\\
10.5808333333333	-0.00175753460082482\\
10.5836111111111	-6.09684578351498e-05\\
10.5863888888889	-0.00347816465440904\\
10.5891666666667	-0.00347816465440904\\
10.5919444444444	-0.00409598890508385\\
10.5947222222222	-0.00409598890508385\\
10.5975	0.00520341142579466\\
10.6002777777778	0.0049849877583203\\
10.6030555555556	0.0049849877583203\\
10.6058333333333	0.0049849877583203\\
10.6086111111111	0.00447359485131983\\
10.6113888888889	0.00447359485131983\\
10.6141666666667	0.00447359485131983\\
10.6169444444444	0.00447359485131983\\
10.6197222222222	0.00447359485131983\\
10.6225	0.00447359485131983\\
10.6252777777778	0.0040593541280294\\
10.6280555555556	0.0040593541280294\\
10.6308333333333	0.0040593541280294\\
10.6336111111111	0.0040593541280294\\
10.6363888888889	0.00399050926045652\\
10.6391666666667	0.00399050926045652\\
10.6419444444444	0.00439503454827412\\
10.6447222222222	0.00439503454827412\\
10.6475	0.00411308797487269\\
10.6502777777778	0.00411308797487269\\
10.6530555555556	0.00411308797487269\\
10.6558333333333	0.00411308797487269\\
10.6586111111111	0.00439512226507893\\
10.6613888888889	0.00439512226507893\\
10.6641666666667	0.00391982349623386\\
10.6669444444444	0.00391982349623386\\
10.6697222222222	0.0036409956212862\\
10.6725	0.00292252946054679\\
10.6752777777778	0.00292252946054679\\
10.6780555555556	0.00247813431710287\\
10.6808333333333	0.000433430834555593\\
10.6836111111111	0.000468739522499902\\
10.6863888888889	0.000468739522499902\\
10.6891666666667	-0.00192059566944173\\
10.6919444444444	-0.00192059566944173\\
10.6947222222222	-0.00192059566944173\\
10.6975	0.00263039444387428\\
10.7002777777778	0.00263039444387428\\
10.7030555555556	0.00209476525989089\\
10.7058333333333	0.00209476525989089\\
10.7086111111111	0.00235129166320309\\
10.7113888888889	0.00235129166320309\\
10.7141666666667	0.00235129166320309\\
10.7169444444444	0.00235129166320309\\
10.7197222222222	0.00235129166320309\\
10.7225	0.00235129166320309\\
10.7252777777778	0.00235129166320309\\
10.7280555555556	0.00346116401253293\\
10.7308333333333	0.00285263803997485\\
10.7336111111111	0.00285263803997485\\
10.7363888888889	0.000536436510039368\\
10.7391666666667	0.000536436510039368\\
10.7419444444444	0.000536436510039368\\
10.7447222222222	0.000813188633376294\\
10.7475	0.000813188633376294\\
10.7502777777778	0.000293954448435817\\
10.7530555555556	0.000293954448435817\\
10.7558333333333	0.000293954448435817\\
10.7586111111111	0.00374202473751246\\
10.7613888888889	0.00444455680525832\\
10.7641666666667	0.00444455680525832\\
10.7669444444444	0.00581504015339188\\
10.7697222222222	0.00472292181601881\\
10.7725	0.00419720747535385\\
10.7752777777778	0.00419720747535385\\
10.7780555555556	0.00419720747535385\\
10.7808333333333	0.00419720747535385\\
10.7836111111111	0.00419720747535385\\
10.7863888888889	0.00419720747535385\\
10.7891666666667	0.00419720747535385\\
10.7919444444444	0.00419720747535385\\
10.7947222222222	0.00190929356596877\\
10.7975	-0.000480128375205714\\
10.8002777777778	-0.00414061523818057\\
10.8030555555556	-0.00889195048812161\\
10.8058333333333	-0.00889195048812161\\
10.8086111111111	-0.00889195048812161\\
10.8113888888889	-0.00889195048812161\\
10.8141666666667	-0.00162860904571695\\
10.8169444444444	-0.00162860904571695\\
10.8197222222222	-0.00162860904571695\\
10.8225	-0.00162860904571695\\
10.8252777777778	-0.00242498271711365\\
10.8280555555556	-0.00168121822739273\\
10.8308333333333	-0.00168121822739273\\
10.8336111111111	-0.00168121822739273\\
10.8363888888889	-0.00359637746946731\\
10.8391666666667	-0.00455466611229721\\
10.8419444444444	0.00427031678429906\\
10.8447222222222	0.00383982010392844\\
10.8475	-0.00119104682698011\\
10.8502777777778	-0.00119104682698011\\
10.8530555555556	-0.000356731737055055\\
10.8558333333333	0.00435817753179403\\
10.8586111111111	0.00691492134572371\\
10.8613888888889	0.00501715686815961\\
10.8641666666667	0.0035820533944668\\
10.8669444444444	0.00384809574852551\\
10.8697222222222	0.00323510990239859\\
10.8725	0.000989954586625857\\
10.8752777777778	0.000989954586625857\\
10.8780555555556	0.000989954586625857\\
10.8808333333333	0.000989954586625857\\
10.8836111111111	0.000989954586625857\\
10.8863888888889	0.000989954586625857\\
10.8891666666667	0.000989954586625857\\
10.8919444444444	0.000989954586625857\\
10.8947222222222	0.000989954586625857\\
10.8975	0.000989954586625857\\
10.9002777777778	0.000989954586625857\\
10.9030555555556	0.000989954586625857\\
10.9058333333333	0.000989954586625857\\
10.9086111111111	0.0110884570718523\\
10.9113888888889	0.0103226248195016\\
10.9141666666667	0.0103226248195016\\
10.9169444444444	0.0103226248195016\\
10.9197222222222	0.0089769513780318\\
10.9225	0.0089769513780318\\
10.9252777777778	0.0089769513780318\\
10.9280555555556	0.00954808374741396\\
10.9308333333333	0.00744720893969062\\
10.9336111111111	0.00642931569310305\\
10.9363888888889	0.00771298550401771\\
10.9391666666667	0.00398772965335148\\
10.9419444444444	0.00650096516170836\\
10.9447222222222	0.00525128078738206\\
10.9475	0.00226887974968632\\
10.9502777777778	0.00451400982144688\\
10.9530555555556	0.00437764566756866\\
10.9558333333333	0.00372471595727394\\
10.9586111111111	0.00387382567732883\\
10.9613888888889	0.00374577411250833\\
10.9641666666667	0.00214530222403609\\
10.9669444444444	0.00157360543758654\\
10.9697222222222	0.000187872537181378\\
10.9725	-0.0002306078206817\\
10.9752777777778	0.00375435109349142\\
10.9780555555556	0.00327484074363805\\
10.9808333333333	0.004334169080169\\
10.9836111111111	0.0043102426126077\\
10.9863888888889	0.00425921097954145\\
10.9891666666667	0.00401494375771642\\
10.9919444444444	0.00464543395012847\\
10.9947222222222	0.00514894128745372\\
10.9975	0.00484491960100878\\
11.0002777777778	0.00453900840197054\\
11.0030555555556	0.00453900840197054\\
11.0058333333333	0.00453900840197054\\
11.0086111111111	0.00436324003260934\\
11.0113888888889	0.00497064738710863\\
11.0141666666667	0.00497064738710863\\
11.0169444444444	0.00401662226969571\\
11.0197222222222	0.00361103215108132\\
11.0225	0.00279895938380733\\
11.0252777777778	0.00279895938380733\\
11.0280555555556	0.00279895938380733\\
11.0308333333333	0.00651547473345048\\
11.0336111111111	0.00651547473345048\\
11.0363888888889	0.00447823114233263\\
11.0391666666667	0.00447823114233263\\
11.0419444444444	-0.00187081310701281\\
11.0447222222222	-0.0040815262199202\\
11.0475	-0.0040815262199202\\
11.0502777777778	-0.00302461300173925\\
11.0530555555556	-0.00488017527022671\\
11.0558333333333	-0.00619071727507413\\
11.0586111111111	-0.00619071727507413\\
11.0613888888889	-0.00619071727507413\\
11.0641666666667	-0.00619071727507413\\
11.0669444444444	-0.00619071727507413\\
11.0697222222222	-0.00932819672732551\\
11.0725	-0.00514245340354024\\
11.0752777777778	-0.00514245340354024\\
11.0780555555556	-0.00514245340354024\\
11.0808333333333	-0.00460025490848364\\
11.0836111111111	-0.00460025490848364\\
11.0863888888889	-0.00460025490848364\\
11.0891666666667	-0.00460025490848364\\
11.0919444444444	-0.00460025490848364\\
11.0947222222222	-0.00460025490848364\\
11.0975	-0.00460025490848364\\
11.1002777777778	-0.00460025490848364\\
11.1030555555556	-0.0035483789486006\\
11.1058333333333	-0.00325235034282935\\
11.1086111111111	-0.00580220800376616\\
11.1113888888889	-0.0128607851960059\\
11.1141666666667	-0.0176650625725748\\
11.1169444444444	-0.0176650625725748\\
11.1197222222222	-0.0176650625725748\\
11.1225	-0.0176650625725748\\
11.1252777777778	-0.0176650625725748\\
11.1280555555556	-0.0176650625725748\\
11.1308333333333	-0.0185064370670022\\
11.1336111111111	-0.0185064370670022\\
11.1363888888889	-0.0185064370670022\\
11.1391666666667	-0.0185064370670022\\
11.1419444444444	-0.0185064370670022\\
11.1447222222222	-0.021040031097212\\
11.1475	-0.021040031097212\\
11.1502777777778	0.00323775967389646\\
11.1530555555556	-0.000397622364041266\\
11.1558333333333	0.000241376030457353\\
11.1586111111111	0.000241376030457353\\
11.1613888888889	0.000241376030457353\\
11.1641666666667	0.000241376030457353\\
11.1669444444444	-0.002108032856356\\
11.1697222222222	-0.002108032856356\\
11.1725	-0.00302002188389873\\
11.1752777777778	-0.00302002188389873\\
11.1780555555556	-0.00302002188389873\\
11.1808333333333	-0.00302002188389873\\
11.1836111111111	-0.00282955562028487\\
11.1863888888889	-0.00282955562028487\\
11.1891666666667	-0.00360373267536378\\
11.1919444444444	-0.00360373267536378\\
11.1947222222222	-0.00360373267536378\\
11.1975	-0.00225973734925416\\
11.2002777777778	-0.00225973734925416\\
11.2030555555556	0.00301434255015807\\
11.2058333333333	0.00301434255015807\\
11.2086111111111	0.00301434255015807\\
11.2113888888889	0.00271171136170833\\
11.2141666666667	0.00271171136170833\\
11.2169444444444	0.00179628989826021\\
11.2197222222222	0.00179628989826021\\
11.2225	0.00179628989826021\\
11.2252777777778	0.00179628989826021\\
11.2280555555556	0.00179628989826021\\
11.2308333333333	0.00179628989826021\\
11.2336111111111	0.00179628989826021\\
11.2363888888889	-0.000532855799099535\\
11.2391666666667	-0.00416994297215397\\
11.2419444444444	-0.00416994297215397\\
11.2447222222222	-0.00766142231245038\\
11.2475	-0.00766142231245038\\
11.2502777777778	-0.0013586256408946\\
11.2530555555556	-0.0013586256408946\\
11.2558333333333	-8.88853402647664e-05\\
11.2586111111111	0.000397483457657517\\
11.2613888888889	0.000397483457657517\\
11.2641666666667	0.000811940859643779\\
11.2669444444444	0.000811940859643779\\
11.2697222222222	0.000811940859643779\\
11.2725	0.000811940859643779\\
11.2752777777778	0.00591023593046435\\
11.2780555555556	0.000101628910653906\\
11.2808333333333	0.000101628910653906\\
11.2836111111111	0.000101628910653906\\
11.2863888888889	0.000101628910653906\\
11.2891666666667	0.000101628910653906\\
11.2919444444444	0.000101628910653906\\
11.2947222222222	-0.000726377500423809\\
11.2975	-0.00298223131002392\\
11.3002777777778	-0.00298223131002392\\
11.3030555555556	-0.00298223131002392\\
11.3058333333333	-0.00298223131002392\\
11.3086111111111	0.00575471538895851\\
11.3113888888889	0.00378890238168722\\
11.3141666666667	0.00120261951736938\\
11.3169444444444	0.00126108763623655\\
11.3197222222222	0.00126108763623655\\
11.3225	0.00126108763623655\\
11.3252777777778	0.00126108763623655\\
11.3280555555556	0.00126108763623655\\
11.3308333333333	0.00155817710766533\\
11.3336111111111	0.00155817710766533\\
11.3363888888889	0.00155817710766533\\
11.3391666666667	0.00155817710766533\\
11.3419444444444	-0.00169915188256267\\
11.3447222222222	-0.00169915188256267\\
11.3475	-0.00044075359161878\\
11.3502777777778	0.00624493893235682\\
11.3530555555556	0.00624493893235682\\
11.3558333333333	0.00325340459941697\\
11.3586111111111	0.00487154909091347\\
11.3613888888889	0.00487154909091347\\
11.3641666666667	0.00275453862264555\\
11.3669444444444	0.00381146544806796\\
11.3697222222222	0.00381146544806796\\
11.3725	0.00511909168189512\\
11.3752777777778	0.00511909168189512\\
11.3780555555556	0.0060927234021817\\
11.3808333333333	0.0060927234021817\\
11.3836111111111	0.0060927234021817\\
11.3863888888889	0.00276447322250563\\
11.3891666666667	0.00276447322250563\\
11.3919444444444	0.00162945888069094\\
11.3947222222222	0.00162945888069094\\
11.3975	0.00162945888069094\\
11.4002777777778	0.00162945888069094\\
11.4030555555556	-0.000430638968206\\
11.4058333333333	-0.000430638968206\\
11.4086111111111	-0.000430638968206\\
11.4113888888889	-0.000430638968206\\
11.4141666666667	0.000524588457569366\\
11.4169444444444	0.00147874751339923\\
11.4197222222222	-0.00154611934995153\\
11.4225	-0.00283403083038067\\
11.4252777777778	0.00371867919385611\\
11.4280555555556	0.00371867919385611\\
11.4308333333333	0.00392312475934403\\
11.4336111111111	0.0040890374783056\\
11.4363888888889	0.0040890374783056\\
11.4391666666667	0.0040890374783056\\
11.4419444444444	0.00507859433106788\\
11.4447222222222	0.00507859433106788\\
11.4475	0.00507859433106788\\
11.4502777777778	0.00507859433106788\\
11.4530555555556	0.00422109068502012\\
11.4558333333333	0.00228872122482944\\
11.4586111111111	0.00132424413365912\\
11.4613888888889	0.00182655228701707\\
11.4641666666667	0.00182655228701707\\
11.4669444444444	0.00182655228701707\\
11.4697222222222	0.00182655228701707\\
11.4725	0.00280249408267856\\
11.4752777777778	0.00280249408267856\\
11.4780555555556	0.00280249408267856\\
11.4808333333333	0.00280249408267856\\
11.4836111111111	0.00433306509912886\\
11.4863888888889	0.00433306509912886\\
11.4891666666667	0.00495632982141329\\
11.4919444444444	-0.000867704354943887\\
11.4947222222222	-0.000867704354943887\\
11.4975	-0.000867704354943887\\
11.5002777777778	-0.000867704354943887\\
11.5030555555556	-0.000867704354943887\\
11.5058333333333	-0.000867704354943887\\
11.5086111111111	0.00232975005743049\\
11.5113888888889	0.00458438722757441\\
11.5141666666667	0.00422374705211696\\
11.5169444444444	0.00538458087558762\\
11.5197222222222	0.00538458087558762\\
11.5225	0.00538458087558762\\
11.5252777777778	0.00538458087558762\\
11.5280555555556	0.00538458087558762\\
11.5308333333333	0.00538458087558762\\
11.5336111111111	0.00538458087558762\\
11.5363888888889	0.00538458087558762\\
11.5391666666667	0.00538458087558762\\
11.5419444444444	0.00538458087558762\\
11.5447222222222	0.00500360216979872\\
11.5475	0.0052492983636651\\
11.5502777777778	0.0052492983636651\\
11.5530555555556	0.0052492983636651\\
11.5558333333333	0.0052492983636651\\
11.5586111111111	0.00506232556074922\\
11.5613888888889	0.00592496467251714\\
11.5641666666667	0.00193359423046751\\
11.5669444444444	-0.00168062762886841\\
11.5697222222222	-0.00168062762886841\\
11.5725	-6.1230006134845e-05\\
11.5752777777778	-6.1230006134845e-05\\
11.5780555555556	-6.1230006134845e-05\\
11.5808333333333	-6.1230006134845e-05\\
11.5836111111111	-6.1230006134845e-05\\
11.5863888888889	-6.1230006134845e-05\\
11.5891666666667	0.00710916812662943\\
11.5919444444444	0.00710916812662943\\
11.5947222222222	0.00514682605674487\\
11.5975	0.00514682605674487\\
11.6002777777778	0.00553631691800234\\
11.6030555555556	0.00553631691800234\\
11.6058333333333	0.00651206469989556\\
11.6086111111111	0.00651206469989556\\
11.6113888888889	0.00651206469989556\\
11.6141666666667	0.00651206469989556\\
11.6169444444444	0.00651206469989556\\
11.6197222222222	0.00454046324199463\\
11.6225	0.00350269911244632\\
11.6252777777778	0.00350269911244632\\
11.6280555555556	0.00469796424050311\\
11.6308333333333	0.00348128443428465\\
11.6336111111111	0.00443241835453757\\
11.6363888888889	0.00443241835453757\\
11.6391666666667	0.00200166537355452\\
11.6419444444444	0.00550979313916179\\
11.6447222222222	0.00550979313916179\\
11.6475	0.00604208254520955\\
11.6502777777778	0.00604208254520955\\
11.6530555555556	0.00604208254520955\\
11.6558333333333	0.00604208254520955\\
11.6586111111111	0.00604208254520955\\
11.6613888888889	0.0032450183645982\\
11.6641666666667	0.0032450183645982\\
11.6669444444444	0.0032450183645982\\
11.6697222222222	0.0032450183645982\\
11.6725	0.00568715594496214\\
11.6752777777778	0.00425742805909396\\
11.6780555555556	0.00425742805909396\\
11.6808333333333	0.00425742805909396\\
11.6836111111111	0.00425742805909396\\
11.6863888888889	0.000198282756649834\\
11.6891666666667	0.00633413315428792\\
11.6919444444444	0.00633413315428792\\
11.6947222222222	0.00633413315428792\\
11.6975	0.00633413315428792\\
11.7002777777778	0.00633413315428792\\
11.7030555555556	0.00633413315428792\\
11.7058333333333	0.00633413315428792\\
11.7086111111111	0.00633413315428792\\
11.7113888888889	0.00633413315428792\\
11.7141666666667	0.00633413315428792\\
11.7169444444444	0.00633413315428792\\
11.7197222222222	0.00633413315428792\\
11.7225	0.00633413315428792\\
11.7252777777778	0.00633413315428792\\
11.7280555555556	0.00633413315428792\\
11.7308333333333	0.00633413315428792\\
11.7336111111111	0.00563408775622971\\
11.7363888888889	0.00456373626756365\\
11.7391666666667	0.00456373626756365\\
11.7419444444444	0.00456373626756365\\
11.7447222222222	0.00308643343138309\\
11.7475	0.00114021913468897\\
11.7502777777778	0.00159650708483538\\
11.7530555555556	0.00207565518665361\\
11.7558333333333	0.0028766924179386\\
11.7586111111111	0.00160102648634367\\
11.7613888888889	0.00160102648634367\\
11.7641666666667	0.00160102648634367\\
11.7669444444444	0.00058900708381399\\
11.7697222222222	0.00058900708381399\\
11.7725	-0.00152280375513498\\
11.7752777777778	-0.00228812039421675\\
11.7780555555556	-0.00228812039421675\\
11.7808333333333	-0.00228812039421675\\
11.7836111111111	-0.0031628557077822\\
11.7863888888889	-0.00446997502104709\\
11.7891666666667	-0.00879960403510613\\
11.7919444444444	0.00529366645360901\\
11.7947222222222	0.00650326394016813\\
11.7975	0.0057531274744317\\
11.8002777777778	0.0057531274744317\\
11.8030555555556	0.0057531274744317\\
11.8058333333333	0.0057531274744317\\
11.8086111111111	0.0057531274744317\\
11.8113888888889	0.0057531274744317\\
11.8141666666667	0.0057531274744317\\
11.8169444444444	0.0057531274744317\\
11.8197222222222	0.0057531274744317\\
11.8225	0.0057531274744317\\
11.8252777777778	0.0057531274744317\\
11.8280555555556	0.0057531274744317\\
11.8308333333333	0.00553470380695733\\
11.8336111111111	0.00553470380695733\\
11.8363888888889	0.00553470380695733\\
11.8391666666667	0.0051858909914619\\
11.8419444444444	0.0051858909914619\\
11.8447222222222	0.00314831783893128\\
11.8475	0.00611150004198784\\
11.8502777777778	0.00611150004198784\\
11.8530555555556	0.00611150004198784\\
11.8558333333333	0.00667214056461065\\
11.8586111111111	0.00531398496034208\\
11.8613888888889	0.00606682375573188\\
11.8641666666667	0.00606682375573188\\
11.8669444444444	0.00623977637590897\\
11.8697222222222	0.00602604303744504\\
11.8725	0.00365418287555679\\
11.8752777777778	0.0029847913590502\\
11.8780555555556	0.0029847913590502\\
11.8808333333333	0.00265446698902612\\
11.8836111111111	0.00265446698902612\\
11.8863888888889	0.00265446698902612\\
11.8891666666667	0.000726144903517432\\
11.8919444444444	0.00475437500462933\\
11.8947222222222	0.00475437500462933\\
11.8975	0.00475437500462933\\
11.9002777777778	0.00589026190875366\\
11.9030555555556	0.00589026190875366\\
11.9058333333333	0.00723590268547597\\
11.9086111111111	0.00723590268547597\\
11.9113888888889	0.00612504645301591\\
11.9141666666667	0.00612504645301591\\
11.9169444444444	0.00612504645301591\\
11.9197222222222	0.00668617081804655\\
11.9225	0.00668617081804655\\
11.9252777777778	0.00668617081804655\\
11.9280555555556	0.00540648662066881\\
11.9308333333333	0.00540648662066881\\
11.9336111111111	0.00540648662066881\\
11.9363888888889	0.00401906451836541\\
11.9391666666667	-0.00102906873439055\\
11.9419444444444	-0.00102906873439055\\
11.9447222222222	-0.00102906873439055\\
11.9475	-0.00102906873439055\\
11.9502777777778	-0.00102906873439055\\
11.9530555555556	-0.00102906873439055\\
11.9558333333333	-0.00102906873439055\\
11.9586111111111	0.00315756564151932\\
11.9613888888889	0.00315756564151932\\
11.9641666666667	0.00349979459993827\\
11.9669444444444	0.00330095393447926\\
11.9697222222222	0.00330095393447926\\
11.9725	0.00330095393447926\\
11.9752777777778	0.00330095393447926\\
11.9780555555556	0.00356931615477027\\
11.9808333333333	0.00356931615477027\\
11.9836111111111	0.00379347885870606\\
11.9863888888889	0.00379347885870606\\
11.9891666666667	0.00379347885870606\\
11.9919444444444	0.00379347885870606\\
11.9947222222222	0.00379347885870606\\
11.9975	0.00379347885870606\\
12.0002777777778	0.00379347885870606\\
12.0030555555556	0.00379347885870606\\
12.0058333333333	0.00117996442308744\\
12.0086111111111	0.00139703733555873\\
12.0113888888889	0.000658886461165064\\
12.0141666666667	-0.000294089863633025\\
12.0169444444444	-0.000294089863633025\\
12.0197222222222	-0.000294089863633025\\
12.0225	-0.000294089863633025\\
12.0252777777778	0.000162198086513377\\
12.0280555555556	0.000162198086513377\\
12.0308333333333	0.000162198086513377\\
12.0336111111111	-5.62255809609859e-05\\
12.0363888888889	-5.62255809609859e-05\\
12.0391666666667	0.00665164669503861\\
12.0419444444444	0.00614678132152057\\
12.0447222222222	0.00614678132152057\\
12.0475	0.00614678132152057\\
12.0502777777778	0.00614678132152057\\
12.0530555555556	0.00614678132152057\\
12.0558333333333	0.00548402621688812\\
12.0586111111111	0.00548402621688812\\
12.0613888888889	0.00297962969319959\\
12.0641666666667	0.000911834686163361\\
12.0669444444444	9.47739492141875e-05\\
12.0697222222222	0.00503631722103495\\
12.0725	0.00503631722103495\\
12.0752777777778	0.00382436217873982\\
12.0780555555556	0.00262512117918696\\
12.0808333333333	0.00667986109427873\\
12.0836111111111	0.00667986109427873\\
12.0863888888889	0.00631549742461216\\
12.0891666666667	0.00631549742461216\\
12.0919444444444	0.00452379991389094\\
12.0947222222222	0.0046856628520772\\
12.0975	0.0046856628520772\\
12.1002777777778	0.0046856628520772\\
12.1030555555556	0.0059316972504784\\
12.1058333333333	0.0059316972504784\\
12.1086111111111	0.0059316972504784\\
12.1113888888889	0.00539670431548958\\
12.1141666666667	0.00202375325334897\\
12.1169444444444	0.003605403800326\\
12.1197222222222	0.003605403800326\\
12.1225	0.00355750881025263\\
12.1252777777778	0.00355750881025263\\
12.1280555555556	0.00355750881025263\\
12.1308333333333	0.00355750881025263\\
12.1336111111111	0.00355750881025263\\
12.1363888888889	0.00326562204889629\\
12.1391666666667	0.00183489249311463\\
12.1419444444444	0.00198370489853946\\
12.1447222222222	0.00198370489853946\\
12.1475	0.00198370489853946\\
12.1502777777778	0.000867106010185817\\
12.1530555555556	0.00447702938647501\\
12.1558333333333	0.00457077514469662\\
12.1586111111111	0.00457077514469662\\
12.1613888888889	0.00457077514469662\\
12.1641666666667	0.00457077514469662\\
12.1669444444444	0.00457077514469662\\
12.1697222222222	0.00457077514469662\\
12.1725	0.00457077514469662\\
12.1752777777778	0.00457077514469662\\
12.1780555555556	0.00457077514469662\\
12.1808333333333	0.00457077514469662\\
12.1836111111111	0.00457077514469662\\
12.1863888888889	0.00598995816439077\\
12.1891666666667	0.00598995816439077\\
12.1919444444444	0.00598995816439077\\
12.1947222222222	0.00510463549089176\\
12.1975	0.00510463549089176\\
12.2002777777778	0.00510463549089176\\
12.2030555555556	0.00510463549089176\\
12.2058333333333	0.00510463549089176\\
12.2086111111111	0.00510463549089176\\
12.2113888888889	0.00510463549089176\\
12.2141666666667	0.00510463549089176\\
12.2169444444444	0.00510463549089176\\
12.2197222222222	0.00510463549089176\\
12.2225	0.00510463549089176\\
12.2252777777778	0.00440892323297215\\
12.2280555555556	0.00440892323297215\\
12.2308333333333	0.00440892323297215\\
12.2336111111111	0.00440892323297215\\
12.2363888888889	0.00440892323297215\\
12.2391666666667	0.00440892323297215\\
12.2419444444444	0.00059793770191835\\
12.2447222222222	0.00059793770191835\\
12.2475	0.00059793770191835\\
12.2502777777778	0.00211984710612638\\
12.2530555555556	0.00105701252844185\\
12.2558333333333	0.0035413390031332\\
12.2586111111111	0.0035413390031332\\
12.2613888888889	0.0035413390031332\\
12.2641666666667	0.00139593025850426\\
12.2669444444444	0.00139593025850426\\
12.2697222222222	0.00139593025850426\\
12.2725	0.00139593025850426\\
12.2752777777778	0.00139593025850426\\
12.2780555555556	0.00139593025850426\\
12.2808333333333	0.00139593025850426\\
12.2836111111111	0.000323594998057723\\
12.2863888888889	0.000212541837416449\\
12.2891666666667	0.000212541837416449\\
12.2919444444444	0.000212541837416449\\
12.2947222222222	0.000212541837416449\\
12.2975	0.000212541837416449\\
12.3002777777778	0.000212541837416449\\
12.3030555555556	-0.000689606345487235\\
12.3058333333333	0.00390195342461383\\
12.3086111111111	0.00390195342461383\\
12.3113888888889	0.00390195342461383\\
12.3141666666667	0.00390195342461383\\
12.3169444444444	0.00390195342461383\\
12.3197222222222	0.00381872154018782\\
12.3225	0.00381872154018782\\
12.3252777777778	0.00381872154018782\\
12.3280555555556	0.00381872154018782\\
12.3308333333333	0.00381872154018782\\
12.3336111111111	0.00410233527281322\\
12.3363888888889	0.0029029263221929\\
12.3391666666667	0.00239543940060359\\
12.3419444444444	0.00239543940060359\\
12.3447222222222	0.00482836053877905\\
12.3475	0.00392325039470508\\
12.3502777777778	0.00392325039470508\\
12.3530555555556	0.00392325039470508\\
12.3558333333333	0.00392325039470508\\
12.3586111111111	0.00392325039470508\\
12.3613888888889	0.00392325039470508\\
12.3641666666667	0.00392325039470508\\
12.3669444444444	0.00489146983149748\\
12.3697222222222	0.00390833764551548\\
12.3725	0.00390833764551548\\
12.3752777777778	0.00390833764551548\\
12.3780555555556	0.00390833764551548\\
12.3808333333333	0.00390833764551548\\
12.3836111111111	0.00390833764551548\\
12.3863888888889	0.00390833764551548\\
12.3891666666667	0.00223721493641641\\
12.3919444444444	0.00223721493641641\\
12.3947222222222	0.00223721493641641\\
12.3975	0.00223721493641641\\
12.4002777777778	0.00223721493641641\\
12.4030555555556	0.00223721493641641\\
12.4058333333333	0.00223721493641641\\
12.4086111111111	0.00223721493641641\\
12.4113888888889	0.00223721493641641\\
12.4141666666667	0.00223721493641641\\
12.4169444444444	0.00223721493641641\\
12.4197222222222	0.000576102680043013\\
12.4225	0.000484451104098818\\
12.4252777777778	0.000484451104098818\\
12.4280555555556	0.000484451104098818\\
12.4308333333333	0.000360618008078005\\
12.4336111111111	0.000360618008078005\\
12.4363888888889	0.00348707370382108\\
12.4391666666667	0.00274845341650275\\
12.4419444444444	0.00276632468276482\\
12.4447222222222	0.00277914869057244\\
12.4475	0.00277914869057244\\
12.4502777777778	0.00277914869057244\\
12.4530555555556	0.00277914869057244\\
12.4558333333333	0.00277914869057244\\
12.4586111111111	0.00277914869057244\\
12.4613888888889	0.00277914869057244\\
12.4641666666667	0.00249483239107479\\
12.4669444444444	0.00147967477085102\\
12.4697222222222	0.00165240950598957\\
12.4725	-0.000703404385445056\\
12.4752777777778	-0.00149488407128198\\
12.4780555555556	-0.00149488407128198\\
12.4808333333333	-0.00149488407128198\\
12.4836111111111	-0.00149488407128198\\
12.4863888888889	-0.00149488407128198\\
12.4891666666667	-0.00326100220360845\\
12.4919444444444	-0.00326100220360845\\
12.4947222222222	-0.00326100220360845\\
12.4975	-0.00326100220360845\\
12.5002777777778	0.000726854221455554\\
12.5030555555556	0.000726854221455554\\
12.5058333333333	0.000729029681848667\\
12.5086111111111	0.000729029681848667\\
12.5113888888889	0.000729029681848667\\
12.5141666666667	0.000729029681848667\\
12.5169444444444	0.000729029681848667\\
12.5197222222222	0.0018736161692628\\
12.5225	0.0018736161692628\\
12.5252777777778	0.0018736161692628\\
12.5280555555556	0.00148135429912668\\
12.5308333333333	0.00294068483871479\\
12.5336111111111	0.00294068483871479\\
12.5363888888889	0.00294068483871479\\
12.5391666666667	0.00377583398683833\\
12.5419444444444	0.00333898665188898\\
12.5447222222222	0.00333898665188898\\
12.5475	0.00216861887741003\\
12.5502777777778	0.00216861887741003\\
12.5530555555556	0.00244322740779447\\
12.5558333333333	0.00244322740779447\\
12.5586111111111	0.00244322740779447\\
12.5613888888889	0.00244322740779447\\
12.5641666666667	0.00244322740779447\\
12.5669444444444	0.00244322740779447\\
12.5697222222222	0.00331730819353493\\
12.5725	0.00331730819353493\\
12.5752777777778	0.00331730819353493\\
12.5780555555556	0.00331730819353493\\
12.5808333333333	0.00331730819353493\\
12.5836111111111	0.00299018305133206\\
12.5863888888889	0.00299018305133206\\
12.5891666666667	0.00299018305133206\\
12.5919444444444	0.00299018305133206\\
12.5947222222222	0.00251002934747501\\
12.5975	0.00251002934747501\\
12.6002777777778	0.00251002934747501\\
12.6030555555556	0.00251002934747501\\
12.6058333333333	0.00251002934747501\\
12.6086111111111	-0.000150184681157477\\
12.6113888888889	-0.000150184681157477\\
12.6141666666667	-0.000150184681157477\\
12.6169444444444	0.000436914769650463\\
12.6197222222222	-0.00108030206024858\\
12.6225	-0.00108030206024858\\
12.6252777777778	-0.00108030206024858\\
12.6280555555556	0.0011255254751717\\
12.6308333333333	0.0011255254751717\\
12.6336111111111	0.0011255254751717\\
12.6363888888889	0.0011255254751717\\
12.6391666666667	0.0011255254751717\\
12.6419444444444	0.0011255254751717\\
12.6447222222222	0.0011255254751717\\
12.6475	0.0011255254751717\\
12.6502777777778	0.000300263410497622\\
12.6530555555556	0.000300263410497622\\
12.6558333333333	0.000300263410497622\\
12.6586111111111	0.000300263410497622\\
12.6613888888889	0.000300263410497622\\
12.6641666666667	-0.00011313069726847\\
12.6669444444444	-2.29961524776685e-05\\
12.6697222222222	0.000120284713368756\\
12.6725	0.000120284713368756\\
12.6752777777778	0.000120284713368756\\
12.6780555555556	0.000120284713368756\\
12.6808333333333	0.000329844007235704\\
12.6836111111111	0.0061772562504723\\
12.6863888888889	0.0061772562504723\\
12.6891666666667	0.0061772562504723\\
12.6919444444444	0.0061772562504723\\
12.6947222222222	0.0042609121705605\\
12.6975	0.0042609121705605\\
12.7002777777778	0.00365517141189325\\
12.7030555555556	0.00365517141189325\\
12.7058333333333	0.00365517141189325\\
12.7086111111111	0.00365517141189325\\
12.7113888888889	0.00365517141189325\\
12.7141666666667	0.00365517141189325\\
12.7169444444444	0.00431044241431711\\
12.7197222222222	0.00365517141189325\\
12.7225	0.00365517141189325\\
12.7252777777778	0.00365517141189325\\
12.7280555555556	0.00365517141189325\\
12.7308333333333	0.00365517141189325\\
12.7336111111111	0.00365517141189325\\
12.7363888888889	0.00365517141189325\\
12.7391666666667	0.00365517141189325\\
12.7419444444444	0.00407418709766493\\
12.7447222222222	0.00407418709766493\\
12.7475	0.00407418709766493\\
12.7502777777778	0.00407418709766493\\
12.7530555555556	0.00407418709766493\\
12.7558333333333	0.00407418709766493\\
12.7586111111111	0.00407418709766493\\
12.7613888888889	0.00407418709766493\\
12.7641666666667	0.00407418709766493\\
12.7669444444444	0.00407418709766493\\
12.7697222222222	0.00407418709766493\\
12.7725	0.00407418709766493\\
12.7752777777778	0.00309249637845472\\
12.7780555555556	0.00309249637845472\\
12.7808333333333	0.00309249637845472\\
12.7836111111111	0.00309249637845472\\
12.7863888888889	0.00309249637845472\\
12.7891666666667	0.00309249637845472\\
12.7919444444444	0.0051319563729836\\
12.7947222222222	0.0051319563729836\\
12.7975	0.0051319563729836\\
12.8002777777778	0.0051319563729836\\
12.8030555555556	0.0051319563729836\\
12.8058333333333	0.0051319563729836\\
12.8086111111111	0.0051319563729836\\
12.8113888888889	0.0051319563729836\\
12.8141666666667	0.0051319563729836\\
12.8169444444444	0.0051319563729836\\
12.8197222222222	0.0035375520747483\\
12.8225	0.0035375520747483\\
12.8252777777778	0.0035375520747483\\
12.8280555555556	0.0035375520747483\\
12.8308333333333	0.0035375520747483\\
12.8336111111111	0.0035375520747483\\
12.8363888888889	0.0035375520747483\\
12.8391666666667	0.0035375520747483\\
12.8419444444444	0.0035375520747483\\
12.8447222222222	0.0034361021827602\\
12.8475	0.00204301788145984\\
12.8502777777778	0.00204301788145984\\
12.8530555555556	0.00204301788145984\\
12.8558333333333	0.00464848948027933\\
12.8586111111111	0.00464848948027933\\
12.8613888888889	0.0033875840449882\\
12.8641666666667	0.00235981260024558\\
12.8669444444444	0.00235981260024558\\
12.8697222222222	0.00235981260024558\\
12.8725	0.00235981260024558\\
12.8752777777778	0.00535148114711861\\
12.8780555555556	0.00535148114711861\\
12.8808333333333	0.00535148114711861\\
12.8836111111111	0.00479893198523574\\
12.8863888888889	0.00479893198523574\\
12.8891666666667	0.00479893198523574\\
12.8919444444444	0.00433580382094365\\
12.8947222222222	0.00433580382094365\\
12.8975	0.0041334480013439\\
12.9002777777778	0.0041334480013439\\
12.9030555555556	0.00327525682360822\\
12.9058333333333	0.00327525682360822\\
12.9086111111111	0.00327525682360822\\
12.9113888888889	0.00364678717689901\\
12.9141666666667	0.00391835784268795\\
12.9169444444444	0.00663211095297823\\
12.9197222222222	0.00663211095297823\\
12.9225	0.00663211095297823\\
12.9252777777778	0.00663211095297823\\
12.9280555555556	0.00663211095297823\\
12.9308333333333	0.00663211095297823\\
12.9336111111111	0.00605933311640283\\
12.9363888888889	0.00605933311640283\\
12.9391666666667	0.00570355024899148\\
12.9419444444444	0.00570355024899148\\
12.9447222222222	0.00570355024899148\\
12.9475	0.00574858126157994\\
12.9502777777778	0.00574858126157994\\
12.9530555555556	0.00544193355501364\\
12.9558333333333	0.0056049840227319\\
12.9586111111111	0.00617873925960697\\
12.9613888888889	0.00674138743076513\\
12.9641666666667	0.00674138743076513\\
12.9669444444444	0.00674138743076513\\
12.9697222222222	0.00651103893878321\\
12.9725	0.00651103893878321\\
12.9752777777778	0.00610557106567342\\
12.9780555555556	0.00594214284209794\\
12.9808333333333	0.00594214284209794\\
12.9836111111111	0.00587004972106894\\
12.9863888888889	0.00587004972106894\\
12.9891666666667	0.00587004972106894\\
12.9919444444444	0.00587004972106894\\
12.9947222222222	0.00587004972106894\\
12.9975	0.00587004972106894\\
13.0002777777778	0.00585810182619388\\
13.0030555555556	0.00585810182619388\\
13.0058333333333	0.00585810182619388\\
13.0086111111111	0.00585810182619388\\
13.0113888888889	0.00585810182619388\\
13.0141666666667	0.00585810182619388\\
13.0169444444444	0.00585810182619388\\
13.0197222222222	0.00585810182619388\\
13.0225	0.00585810182619388\\
13.0252777777778	0.00566717588066908\\
13.0280555555556	0.00541140219040953\\
13.0308333333333	0.00541140219040953\\
13.0336111111111	0.0054492101824868\\
13.0363888888889	0.0054492101824868\\
13.0391666666667	0.0054492101824868\\
13.0419444444444	0.0054492101824868\\
13.0447222222222	0.0054492101824868\\
13.0475	0.0054492101824868\\
13.0502777777778	0.00599736975924\\
13.0530555555556	0.00599736975924\\
13.0558333333333	0.00599736975924\\
13.0586111111111	0.00568782123733002\\
13.0613888888889	0.00568782123733002\\
13.0641666666667	0.00568782123733002\\
13.0669444444444	0.00568782123733002\\
13.0697222222222	0.00568782123733002\\
13.0725	0.00568782123733002\\
13.0752777777778	0.00558579146806008\\
13.0780555555556	0.00558579146806008\\
13.0808333333333	0.00558579146806008\\
13.0836111111111	0.00558579146806008\\
13.0863888888889	0.00558579146806008\\
13.0891666666667	0.00558579146806008\\
13.0919444444444	0.00558579146806008\\
13.0947222222222	0.00485440239574106\\
13.0975	0.00485440239574106\\
13.1002777777778	0.00485440239574106\\
13.1030555555556	0.00485440239574106\\
13.1058333333333	0.00511462382637362\\
13.1086111111111	0.00511462382637362\\
13.1113888888889	0.00511462382637362\\
13.1141666666667	0.00511462382637362\\
13.1169444444444	0.00562071959686531\\
13.1197222222222	0.00550612981685622\\
13.1225	0.00329349339485012\\
13.1252777777778	0.00329349339485012\\
13.1280555555556	0.00248257212936756\\
13.1308333333333	0.00248257212936756\\
13.1336111111111	0.00248257212936756\\
13.1363888888889	0.00275080073671169\\
13.1391666666667	0.00275080073671169\\
13.1419444444444	0.00275080073671169\\
13.1447222222222	0.00275080073671169\\
13.1475	0.00275080073671169\\
13.1502777777778	0.00257139236742486\\
13.1530555555556	0.00257139236742486\\
13.1558333333333	0.00230083876470788\\
13.1586111111111	0.0024383804610772\\
13.1613888888889	0.00183777764751254\\
13.1641666666667	0.00183777764751254\\
13.1669444444444	0.00296924343547271\\
13.1697222222222	0.00296924343547271\\
13.1725	0.00296924343547271\\
13.1752777777778	0.00296924343547271\\
13.1780555555556	0.00296924343547271\\
13.1808333333333	0.00296924343547271\\
13.1836111111111	0.00296924343547271\\
13.1863888888889	0.00296924343547271\\
13.1891666666667	0.00296924343547271\\
13.1919444444444	0.00296924343547271\\
13.1947222222222	0.00296924343547271\\
13.1975	0.00296924343547271\\
13.2002777777778	0.00296924343547271\\
13.2030555555556	0.00296924343547271\\
13.2058333333333	0.00296924343547271\\
13.2086111111111	0.00296924343547271\\
13.2113888888889	0.00115274481331039\\
13.2141666666667	0.00115274481331039\\
13.2169444444444	0.00454747422216335\\
13.2197222222222	0.00454747422216335\\
13.2225	0.00454747422216335\\
13.2252777777778	0.00454747422216335\\
13.2280555555556	0.00240735706075813\\
13.2308333333333	-0.000736328942714751\\
13.2336111111111	-0.00322751358919298\\
13.2363888888889	-0.00358268679098103\\
13.2391666666667	-0.00547520301888064\\
13.2419444444444	-0.00505471959882195\\
13.2447222222222	-0.00545684054765904\\
13.2475	-0.00288285710188681\\
13.2502777777778	-0.000308870990520129\\
13.2530555555556	-0.00164308163007005\\
13.2558333333333	0.00244891821720027\\
13.2586111111111	0.00465736057222611\\
13.2613888888889	0.00465736057222611\\
13.2641666666667	0.00465736057222611\\
13.2669444444444	0.00465736057222611\\
13.2697222222222	0.00465736057222611\\
13.2725	0.00324889491091555\\
13.2752777777778	0.00535439062019299\\
13.2780555555556	0.00524515640375899\\
13.2808333333333	0.0196837267142269\\
13.2836111111111	0.0181567052271508\\
13.2863888888889	0.0193935762431854\\
13.2891666666667	0.0193935762431854\\
13.2919444444444	0.0175510543294101\\
13.2947222222222	0.0175510543294101\\
13.2975	0.0175510543294101\\
13.3002777777778	0.0175510543294101\\
13.3030555555556	0.0194181421196792\\
13.3058333333333	0.0178646756479301\\
13.3086111111111	0.0190485318234364\\
13.3113888888889	0.0173017943851157\\
13.3141666666667	0.00677452881411743\\
13.3169444444444	0.00677452881411743\\
13.3197222222222	0.00677452881411743\\
13.3225	0.00398629625555693\\
13.3252777777778	0.00398629625555693\\
13.3280555555556	-0.0041532595792259\\
13.3308333333333	-0.0041532595792259\\
13.3336111111111	0.0101670114649716\\
13.3363888888889	0.0101670114649716\\
13.3391666666667	0.00227834560340112\\
13.3419444444444	-0.00257032629787179\\
13.3447222222222	-0.0029533298224713\\
13.3475	-0.0035059756487873\\
13.3502777777778	-0.0035059756487873\\
13.3530555555556	-0.0035059756487873\\
13.3558333333333	-0.0035059756487873\\
13.3586111111111	-0.0035059756487873\\
13.3613888888889	-0.0035059756487873\\
13.3641666666667	-0.0035059756487873\\
13.3669444444444	-0.0035059756487873\\
13.3697222222222	-0.0035059756487873\\
13.3725	-0.0035059756487873\\
13.3752777777778	0.00540441031592567\\
13.3780555555556	0.00540441031592567\\
13.3808333333333	0.003056714136547\\
13.3836111111111	-0.00285419853915299\\
13.3863888888889	-0.00285419853915299\\
13.3891666666667	-0.00285419853915299\\
13.3919444444444	-0.00817091186924699\\
13.3947222222222	-0.00817091186924699\\
13.3975	-0.0086174774140342\\
13.4002777777778	-0.0162941439403942\\
13.4030555555556	-0.0195146347168337\\
13.4058333333333	-0.0200472601684522\\
13.4086111111111	-0.0173550021038588\\
13.4113888888889	-0.0173550021038588\\
13.4141666666667	-0.0216862545303849\\
13.4169444444444	-0.0214863158554788\\
13.4197222222222	-0.0214863158554788\\
13.4225	-0.0214863158554788\\
13.4252777777778	-0.0175211770232975\\
13.4280555555556	-0.0175211770232975\\
13.4308333333333	-0.0156231633071625\\
13.4336111111111	-0.0184507131736571\\
13.4363888888889	-0.0184507131736571\\
13.4391666666667	-0.0184507131736571\\
13.4419444444444	-0.0184507131736571\\
13.4447222222222	-0.0184507131736571\\
13.4475	-0.0184507131736571\\
13.4502777777778	-0.0184507131736571\\
13.4530555555556	-0.0231249933425059\\
13.4558333333333	-0.0231249933425059\\
13.4586111111111	-0.0231249933425059\\
13.4613888888889	-0.0231249933425059\\
13.4641666666667	-0.0183303480929262\\
13.4669444444444	-0.0235551026702636\\
13.4697222222222	-0.00163389928491944\\
13.4725	0.00376048970146869\\
13.4752777777778	0.00376048970146869\\
13.4780555555556	0.00376048970146869\\
13.4808333333333	0.00376048970146869\\
13.4836111111111	0.00376048970146869\\
13.4863888888889	0.00339409552033214\\
13.4891666666667	0.00339409552033214\\
13.4919444444444	0.00183395906951881\\
13.4947222222222	0.00183395906951881\\
13.4975	-3.68228112807446e-05\\
13.5002777777778	-0.0033386453743738\\
13.5030555555556	-0.0033386453743738\\
13.5058333333333	-0.00373457066282618\\
13.5086111111111	-0.00425583370567932\\
13.5113888888889	-0.00335770007611736\\
13.5141666666667	-0.00681096003971388\\
13.5169444444444	-0.00681096003971388\\
13.5197222222222	-0.00245995094739211\\
13.5225	-0.00245995094739211\\
13.5252777777778	-0.00245995094739211\\
13.5280555555556	-0.00245995094739211\\
13.5308333333333	-0.00172616790722044\\
13.5336111111111	-0.00065366964272346\\
13.5363888888889	-0.00065366964272346\\
13.5391666666667	-0.00065366964272346\\
13.5419444444444	-0.00065366964272346\\
13.5447222222222	-0.00065366964272346\\
13.5475	0.00421333934065045\\
13.5502777777778	0.00526534382542839\\
13.5530555555556	0.00526534382542839\\
13.5558333333333	0.00500177823973965\\
13.5586111111111	0.00500177823973965\\
13.5613888888889	0.00564857851057934\\
13.5641666666667	0.00564857851057934\\
13.5669444444444	0.00931133905876583\\
13.5697222222222	-0.000414751016054643\\
13.5725	-0.0012417650284747\\
13.5752777777778	-0.0012417650284747\\
13.5780555555556	0.00234731430984772\\
13.5808333333333	0.00234731430984772\\
13.5836111111111	0.00234731430984772\\
13.5863888888889	0.00234731430984772\\
13.5891666666667	0.00234731430984772\\
13.5919444444444	0.00234731430984772\\
13.5947222222222	0.00234731430984772\\
13.5975	0.00234731430984772\\
13.6002777777778	0.00204878214801173\\
13.6030555555556	0.00204878214801173\\
13.6058333333333	0.00204878214801173\\
13.6086111111111	0.00204878214801173\\
13.6113888888889	0.00277739102984638\\
13.6141666666667	0.00277739102984638\\
13.6169444444444	0.00277739102984638\\
13.6197222222222	0.00277739102984638\\
13.6225	0.00277739102984638\\
13.6252777777778	0.00368024135186254\\
13.6280555555556	0.00368024135186254\\
13.6308333333333	0.00368024135186254\\
13.6336111111111	0.00368024135186254\\
13.6363888888889	0.00236716533893134\\
13.6391666666667	0.00236716533893134\\
13.6419444444444	0.00236716533893134\\
13.6447222222222	0.00236716533893134\\
13.6475	0.00236716533893134\\
13.6502777777778	0.00236716533893134\\
13.6530555555556	0.00236716533893134\\
13.6558333333333	0.00236716533893134\\
13.6586111111111	0.00236716533893134\\
13.6613888888889	0.00236716533893134\\
13.6641666666667	0.00236716533893134\\
13.6669444444444	0.00236716533893134\\
13.6697222222222	0.00236716533893134\\
13.6725	0.00236716533893134\\
13.6752777777778	0.00236716533893134\\
13.6780555555556	0.00236716533893134\\
13.6808333333333	0.00236716533893134\\
13.6836111111111	0.00236716533893134\\
13.6863888888889	0.00236716533893134\\
13.6891666666667	0.00199153791828603\\
13.6919444444444	0.00199153791828603\\
13.6947222222222	0.000638906133112509\\
13.6975	0.00141016105533062\\
13.7002777777778	0.000580999936336766\\
13.7030555555556	0.000580999936336766\\
13.7058333333333	0.000580999936336766\\
13.7086111111111	0.000580999936336766\\
13.7113888888889	0.000580999936336766\\
13.7141666666667	0.000580999936336766\\
13.7169444444444	-0.000556670919129762\\
13.7197222222222	0.000720831855274292\\
13.7225	0.000720831855274292\\
13.7252777777778	0.000720831855274292\\
13.7280555555556	0.00453989796402942\\
13.7308333333333	0.00137929214551102\\
13.7336111111111	0.00379438765000782\\
13.7363888888889	0.00379438765000782\\
13.7391666666667	3.48061998137281e-05\\
13.7419444444444	0.000588288928388364\\
13.7447222222222	0.000588288928388364\\
13.7475	0.0048708641144217\\
13.7502777777778	0.0048708641144217\\
13.7530555555556	0.00553022671045944\\
13.7558333333333	0.00553022671045944\\
13.7586111111111	0.00553022671045944\\
13.7613888888889	0.00553022671045944\\
13.7641666666667	0.00553022671045944\\
13.7669444444444	0.00553022671045944\\
13.7697222222222	0.00553022671045944\\
13.7725	0.00553022671045944\\
13.7752777777778	0.00553022671045944\\
13.7780555555556	0.00663685326452774\\
13.7808333333333	0.00663685326452774\\
13.7836111111111	0.00663685326452774\\
13.7863888888889	0.00663685326452774\\
13.7891666666667	0.00564971557435265\\
13.7919444444444	0.00777435973694557\\
13.7947222222222	0.00315746335694485\\
13.7975	0.00315746335694485\\
13.8002777777778	0.0030528918545874\\
13.8030555555556	0.0030528918545874\\
13.8058333333333	0.00203321655934091\\
13.8086111111111	0.00203321655934091\\
13.8113888888889	0.00203321655934091\\
13.8141666666667	0.00203321655934091\\
13.8169444444444	0.00203321655934091\\
13.8197222222222	0.00203321655934091\\
13.8225	0.00203321655934091\\
13.8252777777778	0.00203321655934091\\
13.8280555555556	0.00203321655934091\\
13.8308333333333	0.000928679830659083\\
13.8336111111111	0.000928679830659083\\
13.8363888888889	0.000928679830659083\\
13.8391666666667	0.000928679830659083\\
13.8419444444444	0.000928679830659083\\
13.8447222222222	0.000928679830659083\\
13.8475	0.000928679830659083\\
13.8502777777778	0.000928679830659083\\
13.8530555555556	0.000928679830659083\\
13.8558333333333	0.00398930472649974\\
13.8586111111111	0.00398930472649974\\
13.8613888888889	0.00398930472649974\\
13.8641666666667	0.0036188471193204\\
13.8669444444444	0.0036188471193204\\
13.8697222222222	0.0036188471193204\\
13.8725	0.00156757955823333\\
13.8752777777778	0.00156757955823333\\
13.8780555555556	0.00156757955823333\\
13.8808333333333	0.00134915589075897\\
13.8836111111111	0.00134915589075897\\
13.8863888888889	0.000897805307170612\\
13.8891666666667	0.000705321253523567\\
13.8919444444444	0.000705321253523567\\
13.8947222222222	0.000705321253523567\\
13.8975	0.000705321253523567\\
13.9002777777778	0.000705321253523567\\
13.9030555555556	0.000705321253523567\\
13.9058333333333	0.000705321253523567\\
13.9086111111111	0.000705321253523567\\
13.9113888888889	0.000705321253523567\\
13.9141666666667	0.000705321253523567\\
13.9169444444444	-0.00134861267886728\\
13.9197222222222	-0.00134861267886728\\
13.9225	-0.00165306678233246\\
13.9252777777778	-0.00176981832674794\\
13.9280555555556	-0.00176981832674794\\
13.9308333333333	-0.00176981832674794\\
13.9336111111111	-0.00176981832674794\\
13.9363888888889	-0.00176981832674794\\
13.9391666666667	-0.00331204528805838\\
13.9419444444444	-0.00331204528805838\\
13.9447222222222	-0.00331204528805838\\
13.9475	-0.00331204528805838\\
13.9502777777778	0.00162345187139966\\
13.9530555555556	0.00162345187139966\\
13.9558333333333	0.00162345187139966\\
13.9586111111111	0.00162345187139966\\
13.9613888888889	0.00162345187139966\\
13.9641666666667	0.00162345187139966\\
13.9669444444444	0.00492657259871394\\
13.9697222222222	0.00492657259871394\\
13.9725	0.00302939814835552\\
13.9752777777778	0.00302939814835552\\
13.9780555555556	0.00302939814835552\\
13.9808333333333	0.00302939814835552\\
13.9836111111111	0.00319013401340277\\
13.9863888888889	0.00319013401340277\\
13.9891666666667	0.00319013401340277\\
13.9919444444444	0.00304372160786194\\
13.9947222222222	0.00304372160786194\\
13.9975	0.00165664560629711\\
14.0002777777778	0.000937904345281872\\
14.0030555555556	0.000937904345281872\\
14.0058333333333	0.000937904345281872\\
14.0086111111111	0.0019872507150694\\
14.0113888888889	0.00082451079043515\\
14.0141666666667	-0.00202244062898492\\
14.0169444444444	-0.00202244062898492\\
14.0197222222222	-0.00240418332486366\\
14.0225	-0.0029747809840734\\
14.0252777777778	-0.0029747809840734\\
14.0280555555556	-0.00640014803174247\\
14.0308333333333	-0.00872922047663286\\
14.0336111111111	-0.00529469094087328\\
14.0363888888889	-0.00529469094087328\\
14.0391666666667	-0.00529469094087328\\
14.0419444444444	-0.00529469094087328\\
14.0447222222222	-0.00359071109938649\\
14.0475	-0.00692763340692137\\
14.0502777777778	-0.00730711189898279\\
14.0530555555556	-0.0107400470629649\\
14.0558333333333	-0.0100038975501232\\
14.0586111111111	-0.0101108789247053\\
14.0613888888889	-0.0104782646347996\\
14.0641666666667	-0.00984909581597903\\
14.0669444444444	-0.00715024541260823\\
14.0697222222222	-0.00455954188878772\\
14.0725	-0.0035892539824822\\
14.0752777777778	-0.0035892539824822\\
14.0780555555556	-0.00477429220572614\\
14.0808333333333	-0.00488127358030832\\
14.0836111111111	-0.00488127358030832\\
14.0863888888889	-0.00488127358030832\\
14.0891666666667	-0.00488127358030832\\
14.0919444444444	-0.00488127358030832\\
14.0947222222222	-0.00488127358030832\\
14.0975	-0.00488127358030832\\
14.1002777777778	-0.00488127358030832\\
14.1030555555556	-0.00488127358030832\\
14.1058333333333	-0.00488127358030832\\
14.1086111111111	-0.00488127358030832\\
14.1113888888889	-0.00488127358030832\\
14.1141666666667	-0.00488127358030832\\
14.1169444444444	-0.00498942059436726\\
14.1197222222222	0.0118417588885876\\
14.1225	0.0118417588885876\\
14.1252777777778	0.0118417588885876\\
14.1280555555556	0.0118417588885876\\
14.1308333333333	0.0118417588885876\\
14.1336111111111	0.0101458846326138\\
14.1363888888889	-0.000278780154585982\\
14.1391666666667	-0.00221028879491506\\
14.1419444444444	-0.00611191909904994\\
14.1447222222222	8.66933190912592e-05\\
14.1475	8.66933190912592e-05\\
14.1502777777778	8.66933190912592e-05\\
14.1530555555556	0.00494182605631271\\
14.1558333333333	0.00694668452619643\\
14.1586111111111	0.00584937735137019\\
14.1613888888889	0.0098558259379169\\
14.1641666666667	0.00648482598713342\\
14.1669444444444	0.00510635925937376\\
14.1697222222222	-0.00229687853521979\\
14.1725	-0.00229687853521979\\
14.1752777777778	-0.00229687853521979\\
14.1780555555556	-0.00246831629708046\\
14.1808333333333	-0.00246831629708046\\
14.1836111111111	-0.0016423527639163\\
14.1863888888889	-0.00380228479100163\\
14.1891666666667	-0.00380228479100163\\
14.1919444444444	-0.00380228479100163\\
14.1947222222222	-0.00611856370212784\\
14.1975	-0.00611856370212784\\
14.2002777777778	-0.0093325333055245\\
14.2030555555556	-0.0093325333055245\\
14.2058333333333	-0.0093325333055245\\
14.2086111111111	-0.0093325333055245\\
14.2113888888889	-0.0093325333055245\\
14.2141666666667	-0.0093325333055245\\
14.2169444444444	-0.0093325333055245\\
14.2197222222222	-0.0093325333055245\\
14.2225	-0.012754108601405\\
14.2252777777778	-0.0117796424933126\\
14.2280555555556	-0.0119326782081059\\
14.2308333333333	-0.0119326782081059\\
14.2336111111111	-0.0150389971520069\\
14.2363888888889	-0.0150389971520069\\
14.2391666666667	-0.0171275825601154\\
14.2419444444444	-0.0192277390403802\\
14.2447222222222	-0.0177137059258736\\
14.2475	-0.0190668670880082\\
14.2502777777778	-0.0190668670880082\\
14.2530555555556	-0.0212869307577212\\
14.2558333333333	-0.0212869307577212\\
14.2586111111111	-0.0212869307577212\\
14.2613888888889	-0.0233978354715119\\
14.2641666666667	-0.0233978354715119\\
14.2669444444444	-0.0241013980861217\\
14.2697222222222	-0.0204722215091566\\
14.2725	-0.0204722215091566\\
14.2752777777778	-0.0225831624141297\\
14.2780555555556	-0.0169521854388367\\
14.2808333333333	-0.0169521854388367\\
14.2836111111111	0.00380009049931134\\
14.2863888888889	0.0035042682889217\\
14.2891666666667	0.00551578893312872\\
14.2919444444444	0.00508323574770207\\
14.2947222222222	0.00372917483286177\\
14.2975	0.00594390511518362\\
14.3002777777778	0.00815643146253784\\
14.3030555555556	0.0101494211111984\\
14.3058333333333	0.0101494211111984\\
14.3086111111111	0.00889771085402088\\
14.3113888888889	0.00878847663758687\\
14.3141666666667	0.00606354627830205\\
14.3169444444444	0.00480947960360692\\
14.3197222222222	0.00459101117073891\\
14.3225	0.00459101117073891\\
14.3252777777778	0.0031182289119984\\
14.3280555555556	0.000499679150044442\\
14.3308333333333	-0.00201056311224114\\
14.3336111111111	-0.00452163904701642\\
14.3363888888889	-0.00452163904701642\\
14.3391666666667	-0.00452163904701642\\
14.3419444444444	-0.00463087326345042\\
14.3447222222222	-0.00588655320970928\\
14.3475	-0.00588655320970928\\
14.3502777777778	-0.00610502164257729\\
14.3530555555556	-0.00610502164257729\\
14.3558333333333	-0.00610502164257729\\
14.3586111111111	-0.00610502164257729\\
14.3613888888889	0.0152454979817546\\
14.3641666666667	0.0152454979817546\\
14.3669444444444	0.0151362637653205\\
14.3697222222222	0.0150264483131463\\
14.3725	0.0134729818413974\\
14.3752777777778	0.00340319571351147\\
14.3780555555556	0.00340319571351147\\
14.3808333333333	0.00591641860333547\\
14.3836111111111	0.00591641860333547\\
14.3863888888889	0.00397793996674262\\
14.3891666666667	0.00397793996674262\\
14.3919444444444	0.00397793996674262\\
14.3947222222222	0.00397793996674262\\
14.3975	0.00219629326991875\\
14.4002777777778	0.00219629326991875\\
14.4030555555556	0.00219629326991875\\
14.4058333333333	0.0109903244115691\\
14.4086111111111	0.0109903244115691\\
14.4113888888889	0.0109903244115691\\
14.4141666666667	0.0109903244115691\\
14.4169444444444	0.00928652558397435\\
14.4197222222222	0.00928652558397435\\
14.4225	0.00928652558397435\\
14.4252777777778	0.0104022926868017\\
14.4280555555556	0.00714830176809584\\
14.4308333333333	0.0065123070443548\\
14.4336111111111	0.0065123070443548\\
14.4363888888889	0.00541678768044115\\
14.4391666666667	0.00260935476542149\\
14.4419444444444	0.00395556322942178\\
14.4447222222222	0.00879208107405929\\
14.4475	0.00879208107405929\\
14.4502777777778	0.00879208107405929\\
14.4530555555556	0.00520684028476593\\
14.4558333333333	0.00712633228374302\\
14.4586111111111	0.00976038720160498\\
14.4613888888889	0.00172290933290073\\
14.4641666666667	0.00172290933290073\\
14.4669444444444	-0.00190494226477562\\
14.4697222222222	-0.00190494226477562\\
14.4725	-0.00190494226477562\\
14.4752777777778	-0.00190494226477562\\
14.4780555555556	-0.000482167627042202\\
14.4808333333333	-0.000482167627042202\\
14.4836111111111	-0.000482167627042202\\
14.4863888888889	-0.000482167627042202\\
14.4891666666667	-0.000482167627042202\\
14.4919444444444	-0.000482167627042202\\
14.4947222222222	-0.000482167627042202\\
14.4975	-0.000482167627042202\\
14.5002777777778	-0.000482167627042202\\
14.5030555555556	0.00190023831251776\\
14.5058333333333	0.00350085607078979\\
14.5086111111111	0.00414882729221133\\
14.5113888888889	0.00414882729221133\\
14.5141666666667	0.00507749646884319\\
14.5169444444444	0.00478319411099675\\
14.5197222222222	0.00386379447899525\\
14.5225	0.00386379447899525\\
14.5252777777778	0.00386379447899525\\
14.5280555555556	0.00386379447899525\\
14.5308333333333	0.00386379447899525\\
14.5336111111111	0.00386379447899525\\
14.5363888888889	0.00386379447899525\\
14.5391666666667	0.00394149162881408\\
14.5419444444444	0.00413124613605189\\
14.5447222222222	0.0043210006432897\\
14.5475	0.00491324227787607\\
14.5502777777778	0.00162771474955936\\
14.5530555555556	0.00162771474955936\\
14.5558333333333	0.00162771474955936\\
14.5586111111111	0.00162771474955936\\
14.5613888888889	0.00162771474955936\\
14.5641666666667	0.00415161311632802\\
14.5669444444444	0.00482876311781616\\
14.5697222222222	0.00482876311781616\\
14.5725	0.00516992502917185\\
14.5752777777778	0.00452028095711026\\
14.5780555555556	0.00185735348111299\\
14.5808333333333	0.00586547203023558\\
14.5836111111111	0.00586547203023558\\
14.5863888888889	0.00336233519590217\\
14.5891666666667	0.00391070968881493\\
14.5919444444444	0.00391070968881493\\
14.5947222222222	0.00372913122292709\\
14.5975	0.00372913122292709\\
14.6002777777778	0.00372913122292709\\
14.6030555555556	0.00372913122292709\\
14.6058333333333	0.00217784260069869\\
14.6086111111111	0.00217784260069869\\
14.6113888888889	0.00217784260069869\\
14.6141666666667	0.00217784260069869\\
14.6169444444444	0.00217784260069869\\
14.6197222222222	0.00402579091357796\\
14.6225	0.00402579091357796\\
14.6252777777778	0.00402579091357796\\
14.6280555555556	0.00402579091357796\\
14.6308333333333	0.00603951462665754\\
14.6336111111111	0.00307669383581814\\
14.6363888888889	0.00307669383581814\\
14.6391666666667	0.00307669383581814\\
14.6419444444444	0.00307669383581814\\
14.6447222222222	0.00307669383581814\\
14.6475	0.00307669383581814\\
14.6502777777778	0.00575462040694407\\
14.6530555555556	0.00594654010487842\\
14.6558333333333	0.00594654010487842\\
14.6586111111111	0.00653486254330803\\
14.6613888888889	0.00653486254330803\\
14.6641666666667	0.00653486254330803\\
14.6669444444444	0.00741544703543594\\
14.6697222222222	0.00568306284327181\\
14.6725	0.00593106344333279\\
14.6752777777778	0.00332377545165446\\
14.6780555555556	0.00414671338272896\\
14.6808333333333	0.0105575838784621\\
14.6836111111111	0.0123625978329006\\
14.6863888888889	0.0123625978329006\\
14.6891666666667	0.0123625978329006\\
14.6919444444444	0.00941501597622642\\
14.6947222222222	0.0098830357999721\\
14.6975	0.0098830357999721\\
14.7002777777778	0.0098830357999721\\
14.7030555555556	0.0098830357999721\\
14.7058333333333	0.0098830357999721\\
14.7086111111111	0.0098830357999721\\
14.7113888888889	0.0098830357999721\\
14.7141666666667	0.0098830357999721\\
14.7169444444444	0.0098830357999721\\
14.7197222222222	0.0144517743999764\\
14.7225	0.0132452427671671\\
14.7252777777778	0.0132452427671671\\
14.7280555555556	0.0154705138951553\\
14.7308333333333	0.0154705138951553\\
14.7336111111111	0.0154705138951553\\
14.7363888888889	0.0170401374229377\\
14.7391666666667	0.020282708682788\\
14.7419444444444	0.00971195178017322\\
14.7447222222222	0.00971195178017322\\
14.7475	0.00687244410300372\\
14.7502777777778	0.0122918960341146\\
14.7530555555556	0.0117138188199422\\
14.7558333333333	0.0117138188199422\\
14.7586111111111	0.0101646017865719\\
14.7613888888889	0.0101646017865719\\
14.7641666666667	0.011026637402245\\
14.7669444444444	0.011026637402245\\
14.7697222222222	0.011026637402245\\
14.7725	0.011026637402245\\
14.7752777777778	0.011026637402245\\
14.7780555555556	0.011026637402245\\
14.7808333333333	0.00657915627603297\\
14.7836111111111	0.0074283154038123\\
14.7863888888889	0.0074283154038123\\
14.7891666666667	0.0074283154038123\\
14.7919444444444	0.0120234248469208\\
14.7947222222222	0.012022253408634\\
14.7975	0.012022253408634\\
14.8002777777778	0.012022253408634\\
14.8030555555556	0.0143953202874254\\
14.8058333333333	0.0172028448397978\\
14.8086111111111	0.0172028448397978\\
14.8113888888889	0.0142483628010671\\
14.8141666666667	0.00955363027006105\\
14.8169444444444	0.0165766032482601\\
14.8197222222222	0.0165766032482601\\
14.8225	0.0136670558760711\\
14.8252777777778	0.00269611028606476\\
14.8280555555556	0.00499779379646524\\
14.8308333333333	0.00499779379646524\\
14.8336111111111	0.00543330366251847\\
14.8363888888889	0.00431773545480857\\
14.8391666666667	0.00431773545480857\\
14.8419444444444	0.00177615547503041\\
14.8447222222222	0.00365170660893266\\
14.8475	0.00365170660893266\\
14.8502777777778	0.00216676305681151\\
14.8530555555556	0.00216676305681151\\
14.8558333333333	0.00216676305681151\\
14.8586111111111	0.00216676305681151\\
14.8613888888889	0.00216676305681151\\
14.8641666666667	0.00216676305681151\\
14.8669444444444	0.00216676305681151\\
14.8697222222222	0.00216676305681151\\
14.8725	0.00216676305681151\\
14.8752777777778	0.00216676305681151\\
14.8780555555556	0.00216676305681151\\
14.8808333333333	0.00216676305681151\\
14.8836111111111	0.00216676305681151\\
14.8863888888889	0.00216676305681151\\
14.8891666666667	0.00216676305681151\\
14.8919444444444	0.00216676305681151\\
14.8947222222222	0.00216676305681151\\
14.8975	0.00505826347954613\\
14.9002777777778	0.0035292978072242\\
14.9030555555556	0.0035292978072242\\
14.9058333333333	0.0035292978072242\\
14.9086111111111	-0.00111992484746892\\
14.9113888888889	-0.00111992484746892\\
14.9141666666667	0.00105465326048484\\
14.9169444444444	-0.00217754726670719\\
14.9197222222222	-0.00151700563966783\\
14.9225	0.00132250203750137\\
14.9252777777778	0.00132250203750137\\
14.9280555555556	0.00132250203750137\\
14.9308333333333	0.00201233002472495\\
14.9336111111111	0.00145286921512833\\
14.9363888888889	0.00145286921512833\\
14.9391666666667	0.00145286921512833\\
14.9419444444444	0.00145286921512833\\
14.9447222222222	0.00145286921512833\\
14.9475	0.00145286921512833\\
14.9502777777778	0.00145286921512833\\
14.9530555555556	0.00502188645573776\\
14.9558333333333	0.00502188645573776\\
14.9586111111111	0.00219465170503617\\
14.9613888888889	0.000416173718232909\\
14.9641666666667	0.00244508422058087\\
14.9669444444444	0.00244508422058087\\
14.9697222222222	-0.000197467698163839\\
14.9725	0.000907882924645682\\
14.9752777777778	0.000907882924645682\\
14.9780555555556	0.000907882924645682\\
14.9808333333333	0.000907882924645682\\
14.9836111111111	0.00109763743188393\\
14.9863888888889	0.00200744767178034\\
14.9891666666667	0.00178902400430596\\
14.9919444444444	0.00246899492087275\\
14.9947222222222	0.000919912341976109\\
14.9975	0.000919912341976109\\
15.0002777777778	0.00206242741189722\\
15.0030555555556	0.00206242741189722\\
15.0058333333333	0.00206242741189722\\
15.0086111111111	0.00206242741189722\\
15.0113888888889	0.00244247620562974\\
15.0141666666667	0.00244247620562974\\
15.0169444444444	0.00244247620562974\\
15.0197222222222	0.00451532192668414\\
15.0225	0.00451532192668414\\
15.0252777777778	0.00451532192668414\\
15.0280555555556	0.00451532192668414\\
15.0308333333333	0.00451532192668414\\
15.0336111111111	0.00451532192668414\\
15.0363888888889	0.00618317089279789\\
15.0391666666667	0.00618317089279789\\
15.0419444444444	-0.000206153841361481\\
15.0447222222222	0.000703951955058277\\
15.0475	0.000703951955058277\\
15.0502777777778	0.000703951955058277\\
15.0530555555556	0.00159764998056705\\
15.0558333333333	0.00159764998056705\\
15.0586111111111	0.00135017048302091\\
15.0613888888889	0.00135017048302091\\
15.0641666666667	-0.000277977658999763\\
15.0669444444444	-0.000277977658999763\\
15.0697222222222	-0.000277977658999763\\
15.0725	-0.00329485714244587\\
15.0752777777778	-0.00329485714244587\\
15.0780555555556	-0.00329485714244587\\
15.0808333333333	-0.00468796874492963\\
15.0836111111111	-0.00601744932818183\\
15.0863888888889	0.00436789048507152\\
15.0891666666667	0.00229504476401754\\
15.0919444444444	0.00280046017461059\\
15.0947222222222	0.000759521985915404\\
15.0975	0.00220973127135182\\
15.1002777777778	-0.000936554933766744\\
15.1030555555556	-0.000936554933766744\\
15.1058333333333	-0.000936554933766744\\
15.1086111111111	-0.00324720040892632\\
15.1113888888889	-0.00226931102478031\\
15.1141666666667	-0.00226931102478031\\
15.1169444444444	-0.00226931102478031\\
15.1197222222222	0.00351118837584558\\
15.1225	0.00351118837584558\\
15.1252777777778	0.00351118837584558\\
15.1280555555556	0.00166375978713299\\
15.1308333333333	0.00277078312562187\\
15.1336111111111	0.00380615353229455\\
15.1363888888889	0.00410227434958774\\
15.1391666666667	0.00410227434958774\\
15.1419444444444	0.0040252005691803\\
15.1447222222222	0.00104083148939188\\
15.1475	0.000705507024817222\\
15.1502777777778	0.000705507024817222\\
15.1530555555556	0.000705507024817222\\
15.1558333333333	0.00377599797888938\\
15.1586111111111	0.00288763552700898\\
15.1613888888889	0.00273389271800472\\
15.1641666666667	0.00420870370441589\\
15.1669444444444	0.00420870370441589\\
15.1697222222222	0.00330660737665563\\
15.1725	0.00400427532332464\\
15.1752777777778	0.00155655454795401\\
15.1780555555556	0.00212275502878484\\
15.1808333333333	0.00212275502878484\\
15.1836111111111	0.00212275502878484\\
15.1863888888889	0.00243796049274954\\
15.1891666666667	0.00243796049274954\\
15.1919444444444	0.00570385024516203\\
15.1947222222222	0.00570385024516203\\
15.1975	0.00570385024516203\\
15.2002777777778	0.0021221762361656\\
15.2030555555556	0.0021221762361656\\
15.2058333333333	0.0021221762361656\\
15.2086111111111	0.00168514394769716\\
15.2113888888889	0.00381300270035104\\
15.2141666666667	0.00381300270035104\\
15.2169444444444	0.00360806274112978\\
15.2197222222222	0.00529230369888332\\
15.2225	0.00599127675814465\\
15.2252777777778	0.00599127675814465\\
15.2280555555556	0.00599127675814465\\
15.2308333333333	0.00405883069167032\\
15.2336111111111	0.00452646289893257\\
15.2363888888889	0.00452646289893257\\
15.2391666666667	0.00452646289893257\\
15.2419444444444	0.00996332618615935\\
15.2447222222222	0.000991694975460498\\
15.2475	0.000991694975460498\\
15.2502777777778	0.00182486608183502\\
15.2530555555556	0.00182486608183502\\
15.2558333333333	0.00286281891405134\\
15.2586111111111	0.00361188793826374\\
15.2613888888889	0.00378836254038758\\
15.2641666666667	0.00378836254038758\\
15.2669444444444	0.00145922098179085\\
15.2697222222222	1.43288453295434e-05\\
15.2725	0.00291996284886856\\
15.2752777777778	0.00135408451351521\\
15.2780555555556	0.000408134552714368\\
15.2808333333333	-0.00177087110526126\\
15.2836111111111	-0.00201465756644368\\
15.2863888888889	-0.00201465756644368\\
15.2891666666667	-0.00201465756644368\\
15.2919444444444	-0.00201465756644368\\
15.2947222222222	-0.00201465756644368\\
15.2975	-0.00201465756644368\\
15.3002777777778	0.00708804912802718\\
15.3030555555556	0.00267749745004028\\
15.3058333333333	-0.00304224330926485\\
15.3086111111111	-0.00304224330926485\\
15.3113888888889	-0.00721779883947648\\
15.3141666666667	-0.00721779883947648\\
15.3169444444444	-0.00643635622326994\\
15.3197222222222	0.010065581055514\\
15.3225	0.00426016004545327\\
15.3252777777778	0.00374584288208707\\
15.3280555555556	0.00195226366231295\\
15.3308333333333	0.00711641288170866\\
15.3336111111111	0.00532283366193468\\
15.3363888888889	0.00530062104169345\\
15.3391666666667	0.00751152023978181\\
15.3419444444444	0.00549013755799956\\
15.3447222222222	0.00144491448645067\\
15.3475	0.00144491448645067\\
15.3502777777778	0.00144491448645067\\
15.3530555555556	-0.00111353251943922\\
15.3558333333333	0.000915715938923031\\
15.3586111111111	0.000915715938923031\\
15.3613888888889	0.000915715938923031\\
15.3641666666667	0.000915715938923031\\
15.3669444444444	-0.00227655887871488\\
15.3697222222222	-0.00621669245237842\\
15.3725	0.0033137928210408\\
15.3752777777778	-0.000195743286142688\\
15.3780555555556	0.00230221992015826\\
15.3808333333333	0.00496684332660936\\
15.3836111111111	0.00496684332660936\\
15.3863888888889	0.00496684332660936\\
15.3891666666667	0.00417715357331273\\
15.3919444444444	0.00640299042542527\\
15.3947222222222	0.0040441714290457\\
15.3975	0.00550774490096449\\
15.4002777777778	0.00550774490096449\\
15.4030555555556	0.00550774490096449\\
15.4058333333333	0.00550774490096449\\
15.4086111111111	0.00550774490096449\\
15.4113888888889	0.00550774490096449\\
15.4141666666667	0.00550774490096449\\
15.4169444444444	0.00250597862393827\\
15.4197222222222	0.00250597862393827\\
15.4225	-0.000669441701650061\\
15.4252777777778	-0.000669441701650061\\
15.4280555555556	-0.000394805946936954\\
15.4308333333333	0.00290220644011499\\
15.4336111111111	0.00290220644011499\\
15.4363888888889	0.00230955756509451\\
15.4391666666667	0.00424649377777867\\
15.4419444444444	0.00548822292490022\\
15.4447222222222	0.00526979925742586\\
15.4475	0.00294775303895538\\
15.4502777777778	0.00388525777692123\\
15.4530555555556	0.00455150467797309\\
15.4558333333333	0.00397838968457087\\
15.4586111111111	0.00538089997184267\\
15.4613888888889	0.00467785158774169\\
15.4641666666667	0.00467785158774169\\
15.4669444444444	0.00501515473550281\\
15.4697222222222	0.00479673106802845\\
15.4725	0.00370446575853605\\
15.4752777777778	0.00288736338916658\\
15.4780555555556	0.00288736338916658\\
15.4808333333333	0.00288736338916658\\
15.4836111111111	0.00288736338916658\\
15.4863888888889	0.00288736338916658\\
15.4891666666667	0.00431751874731633\\
15.4919444444444	0.00254079384355552\\
15.4947222222222	0.00254079384355552\\
15.4975	-0.000138309245832448\\
15.5002777777778	0.00132258627568469\\
15.5030555555556	0.00132258627568469\\
15.5058333333333	0.00132258627568469\\
15.5086111111111	0.00132258627568469\\
15.5113888888889	0.00330985391928158\\
15.5141666666667	0.00367427688526486\\
15.5169444444444	0.00367427688526486\\
15.5197222222222	0.00250935582560294\\
15.5225	0.00206535698176881\\
15.5252777777778	0.00206535698176881\\
15.5280555555556	0.00206535698176881\\
15.5308333333333	0.00418025293419394\\
15.5336111111111	0.00418025293419394\\
15.5363888888889	0.00418025293419394\\
15.5391666666667	0.00398132779376525\\
15.5419444444444	0.00398132779376525\\
15.5447222222222	0.00497952147351419\\
15.5475	0.00497952147351419\\
15.5502777777778	0.00474311927925714\\
15.5530555555556	0.00557750319688053\\
15.5558333333333	0.00363180740505087\\
15.5586111111111	0.00205081820098794\\
15.5613888888889	0.00086691441859179\\
15.5641666666667	0.0078564717777776\\
15.5669444444444	0.0078564717777776\\
15.5697222222222	0.00739711400270751\\
15.5725	0.0070238374113679\\
15.5752777777778	0.0070238374113679\\
15.5780555555556	0.0070238374113679\\
15.5808333333333	0.00792339696470501\\
15.5836111111111	0.00792339696470501\\
15.5863888888889	0.00582346591423322\\
15.5891666666667	0.00582346591423322\\
15.5919444444444	0.00582346591423322\\
15.5947222222222	0.00582346591423322\\
15.5975	0.00582346591423322\\
15.6002777777778	0.00582346591423322\\
15.6030555555556	0.00582346591423322\\
15.6058333333333	0.00574063153286961\\
15.6086111111111	0.00535785165437738\\
15.6113888888889	0.00535785165437738\\
15.6141666666667	0.000244711487799528\\
15.6169444444444	0.000244711487799528\\
15.6197222222222	0.000700939201943532\\
15.6225	0.000705679904255598\\
15.6252777777778	-0.00395688226719898\\
15.6280555555556	-0.00395688226719898\\
15.6308333333333	-0.00395688226719898\\
15.6336111111111	-0.00247237488040544\\
15.6363888888889	-0.00388633317958285\\
15.6391666666667	-0.00747015060619995\\
15.6419444444444	-0.00804889162756441\\
15.6447222222222	-0.00528783723978241\\
15.6475	-0.00528783723978241\\
15.6502777777778	-0.00550966855766194\\
15.6530555555556	-0.00655426564245269\\
15.6558333333333	-0.00651560871467522\\
15.6586111111111	-0.00685913370264967\\
15.6613888888889	-0.00685913370264967\\
15.6641666666667	-0.00853949344714965\\
15.6669444444444	-0.00735018146361757\\
15.6697222222222	-0.00735018146361757\\
15.6725	-0.00707185205320858\\
15.6752777777778	-0.00707185205320858\\
15.6780555555556	-0.00538227685438197\\
15.6808333333333	-0.00538227685438197\\
15.6836111111111	-0.00538227685438197\\
15.6863888888889	-0.00720388021842494\\
15.6891666666667	-0.00720388021842494\\
15.6919444444444	-0.00720388021842494\\
15.6947222222222	-0.00720388021842494\\
15.6975	-0.00528706792508394\\
15.7002777777778	-0.00702275987651073\\
15.7030555555556	-0.00798882249858634\\
15.7058333333333	-0.00826930037524399\\
15.7086111111111	-0.00826930037524399\\
15.7113888888889	-0.00826930037524399\\
15.7141666666667	-0.00826930037524399\\
15.7169444444444	-0.00826930037524399\\
15.7197222222222	-0.00826930037524399\\
15.7225	-0.00805388328108772\\
15.7252777777778	-0.00805388328108772\\
15.7280555555556	-0.00805388328108772\\
15.7308333333333	-0.00805388328108772\\
15.7336111111111	-0.00761940188779862\\
15.7363888888889	-0.00761940188779862\\
15.7391666666667	-0.00761940188779862\\
15.7419444444444	-0.00761940188779862\\
15.7447222222222	-0.00736640463192861\\
15.7475	-0.00688085342804291\\
15.7502777777778	-0.0121691222540295\\
15.7530555555556	-0.0186138218895313\\
15.7558333333333	-0.00446613142467966\\
15.7586111111111	-0.0067771004893813\\
15.7613888888889	-0.0115187890700201\\
15.7641666666667	-0.0147210233410633\\
15.7669444444444	-0.0144508144874143\\
15.7697222222222	-0.00891284121207034\\
15.7725	-0.00936326935237203\\
15.7752777777778	-0.0112521749306121\\
15.7780555555556	-0.0112521749306121\\
15.7808333333333	-0.011989889787995\\
15.7836111111111	-0.0112992538357789\\
15.7863888888889	-0.0109627036011603\\
15.7891666666667	-0.0109627036011603\\
15.7919444444444	-0.0110666631295696\\
15.7947222222222	-0.011364666008207\\
15.7975	-0.011364666008207\\
15.8002777777778	-0.0117221801845837\\
15.8030555555556	-0.0135065467977001\\
15.8058333333333	-0.0142510852999072\\
15.8086111111111	-0.0142510852999072\\
15.8113888888889	-0.0142905316199134\\
15.8141666666667	-0.0144322979139253\\
15.8169444444444	-0.0142522746811408\\
15.8197222222222	-0.0142522746811408\\
15.8225	-0.0142522746811408\\
15.8252777777778	-0.0140894981424989\\
15.8280555555556	-0.0140650415377209\\
15.8308333333333	-0.0140650415377209\\
15.8336111111111	-0.0141522483958324\\
15.8363888888889	-0.0147034082354335\\
15.8391666666667	-0.0147133029882295\\
15.8419444444444	-0.0147133029882295\\
15.8447222222222	-0.0147133029882295\\
15.8475	-0.0147133029882295\\
15.8502777777778	-0.0147133029882295\\
15.8530555555556	-0.0147133029882295\\
15.8558333333333	-0.0147133029882295\\
15.8586111111111	-0.0145779415028689\\
15.8613888888889	-0.0145779415028689\\
15.8641666666667	-0.0145779415028689\\
15.8669444444444	-0.0149866247400564\\
15.8697222222222	-0.0149866247400564\\
15.8725	-0.0158588364682189\\
15.8752777777778	-0.0158588364682189\\
15.8780555555556	-0.0149724260515353\\
15.8808333333333	-0.0149724260515353\\
15.8836111111111	-0.0137389349174272\\
15.8863888888889	-0.0136756584544633\\
15.8891666666667	-0.0136756584544633\\
15.8919444444444	-0.0136756584544633\\
15.8947222222222	-0.0136756584544633\\
15.8975	-0.0106581655424187\\
15.9002777777778	-0.0106581655424187\\
15.9030555555556	-0.0112346661430836\\
15.9058333333333	-0.011476337498077\\
15.9086111111111	-0.011476337498077\\
15.9113888888889	-0.011476337498077\\
15.9141666666667	-0.011476337498077\\
15.9169444444444	-0.0109859241038684\\
15.9197222222222	-0.0108566678989503\\
15.9225	-0.0108566678989503\\
15.9252777777778	-0.0111950720045715\\
15.9280555555556	-0.0109986300721761\\
15.9308333333333	-0.0108728855093979\\
15.9336111111111	-0.0137219810129858\\
15.9363888888889	-0.0140925613664116\\
15.9391666666667	-0.0140925613664116\\
15.9419444444444	-0.0133110673441797\\
15.9447222222222	-0.0124755099046115\\
15.9475	-0.0124755099046115\\
15.9502777777778	-0.0136298452602929\\
15.9530555555556	-0.013254973042008\\
15.9558333333333	-0.013254973042008\\
15.9586111111111	-0.013254973042008\\
15.9613888888889	-0.013254973042008\\
15.9641666666667	-0.013254973042008\\
15.9669444444444	-0.0126613395683223\\
15.9697222222222	-0.0130975055976899\\
15.9725	-0.0130975055976899\\
15.9752777777778	-0.0130975055976899\\
15.9780555555556	-0.0118656978641306\\
15.9808333333333	-0.0118656978641306\\
15.9836111111111	-0.00947214698383887\\
15.9863888888889	-0.00983479984970423\\
15.9891666666667	-0.00914912521660121\\
15.9919444444444	-0.0132319087945714\\
15.9947222222222	-0.0127378739448555\\
15.9975	-0.0142912960173226\\
};
\end{axis}
\end{tikzpicture}%

  \caption{Co-integration relation of the four stochastic control methods.}
  \label{fig:cointeg_relation}
\end{figure}
From the inventory plot in \autoref{fig:ORCL_comp4stoch_inv} we see that the strategies always avoid maintaining zero inventory. This is not surprising: in both the discrete and continuous cases we found that the value function ansatz $h(t,z,q)$ was non-negative, and was equal to zero at zero inventory. The interpretation is that there is no added value to having zero inventory, whereas non-zero inventory can at worst have zero value. Thus it is always profitable, from a value-function standpoint, to have non-zero inventory. Further, we see that the strategies rarely cross the zero-inventory barrier. This is likely attributed to the backtesting algorithm itself, which gives priority to executing buy market orders above sell market orders - we suspect that once the strategy gets into either positive or negative inventory territory, the ansatz function $h$ rarely produces the circumstances to cross the inventory sign barrier by virtue of the non-linear mark-to-market behavior on either side of zero inventory.

Concerning trade execution, the number of executed market orders and filled limit orders generated by each strategy are presented in \autoref{tbl:ORCL_comp4stoch_numt}. The surge in market orders seen for the Cts w nFPC strategy can be explained by the difference in the $\delta^\pm$ plots. As mentioned already, from the stochastic analysis chapter, we know that if $q < 0 $ and $\delta^+ =0$, or if $q > 0$ and $\delta^+ = 1/\kappa$, then we execute a buy MO. Likewise, if $q < 0 $ and $\delta^- =1/\kappa$, or if $q > 0$ and $\delta^- = 0$, then we execute a sell MO. In \autoref{fig:comp_dp_z1} we see that we have $\delta^+ = 0$ for almost all inventory values, and in \autoref{fig:comp_dp_z15} we have  $\delta^+ = 1/\kappa$ for almost all inventory values. This tells us that when the Markov chain state is in one of the non-neutral states, the Cts w nFPC strategy will execute market orders when possible, as it expects prices to move in the corresponding direction. Regarding overall number of trades, it should be noted that we did not include the cost of market order execution in the stochastic control problem. Thus, actual performance would have been negatively affected in proportion to the number of market orders listed.
\begin{table}
\centering
\ra{1.2}
\begin{tabular}{@{} r *{2}{c} @{}}
\toprule
& Market Orders & Limit Orders \\
\midrule
Cts          &  536 & 1280 \\
Cts w nFPC   & 1010 & 1306 \\
Dscr         &  559 & 1285 \\
Dscr w nFPC  &  523 & 1287 \\
\bottomrule
\end{tabular}
\caption{Number of trades comparison of the four stochastic control methods.}
\label{tbl:ORCL_comp4stoch_numt}
\end{table}

To better see how the strategies differ in behaviour, in the figures that follow we show a short sample path on a fine timescale spanning about 2 minutes. In \autoref{fig:samplepath_paths} we plot the midprice path (black line), the optimal posting depths on either side of the real bid/ask prices (gray lines), our execution of market orders (dark blue and dark green), and track incoming external market orders (light blue and light green) that either fill our limit orders (solid lines) or do not (dashed lines). \autoref{fig:samplepath_depths} plots just the optimal depths as they react to the changing Markov state, allowing a better comparison of the behaviours, as well as highlighting the almost-symmetric behaviour between $\delta^+$ and $delta^-$. \autoref{fig:samplepath_pnl} and \autoref{fig:samplepath_inv} show the effect on PnL and inventory, respectively. 

\begin{figure}
\centering
\begin{subfigure}{.45\linewidth}
  \centering
  \setlength\figureheight{\linewidth} 
  \setlength\figurewidth{\linewidth}
  \tikzsetnextfilename{samplepath_cts_paths}
  % This file was created by matlab2tikz.
%
%The latest updates can be retrieved from
%  http://www.mathworks.com/matlabcentral/fileexchange/22022-matlab2tikz-matlab2tikz
%where you can also make suggestions and rate matlab2tikz.
%
\definecolor{mycolor1}{rgb}{0.65098,0.80784,0.89020}%
\definecolor{mycolor2}{rgb}{0.69804,0.87451,0.54118}%
\definecolor{mycolor3}{rgb}{0.20000,0.62745,0.17255}%
\definecolor{mycolor4}{rgb}{0.12157,0.47059,0.70588}%
%
\begin{tikzpicture}[trim axis left, trim axis right]

\begin{axis}[%
width=\figurewidth,
height=\figureheight,
at={(0\figurewidth,0\figureheight)},
scale only axis,
every outer x axis line/.append style={black},
every x tick label/.append style={font=\color{black}},
xmin=10.975,
xmax=11,
every outer y axis line/.append style={black},
every y tick label/.append style={font=\color{black}},
ymin=33.93,
ymax=34.03,
axis background/.style={fill=white},
axis x line*=bottom,
axis y line*=left,
legend style={legend cell align=left,align=left,draw=black,font=\footnotesize,legend pos=south west},
every axis legend/.code={\renewcommand\addlegendentry[2][]{}}  %ignore legend locally
]
\addplot [color=black,solid,line width=3.0pt]
  table[row sep=crcr]{%
10.975	33.995\\
10.9752777777778	33.995\\
10.9755555555556	33.995\\
10.9758333333333	33.995\\
10.9761111111111	33.995\\
10.9763888888889	33.995\\
10.9766666666667	33.995\\
10.9769444444444	33.995\\
10.9772222222222	33.995\\
10.9775	33.995\\
10.9777777777778	33.995\\
10.9780555555556	33.99\\
10.9783333333333	33.99\\
10.9786111111111	33.99\\
10.9788888888889	33.99\\
10.9791666666667	33.99\\
10.9794444444444	33.995\\
10.9797222222222	33.995\\
10.98	33.995\\
10.9802777777778	33.99\\
10.9805555555556	33.99\\
10.9808333333333	33.99\\
10.9811111111111	33.995\\
10.9813888888889	33.985\\
10.9816666666667	33.985\\
10.9819444444444	33.985\\
10.9822222222222	33.985\\
10.9825	33.985\\
10.9827777777778	33.985\\
10.9830555555556	33.985\\
10.9833333333333	33.985\\
10.9836111111111	33.985\\
10.9838888888889	33.985\\
10.9841666666667	33.985\\
10.9844444444444	33.98\\
10.9847222222222	33.98\\
10.985	33.985\\
10.9852777777778	33.985\\
10.9855555555556	33.98\\
10.9858333333333	33.985\\
10.9861111111111	33.985\\
10.9863888888889	33.975\\
10.9866666666667	33.975\\
10.9869444444444	33.975\\
10.9872222222222	33.975\\
10.9875	33.975\\
10.9877777777778	33.975\\
10.9880555555556	33.975\\
10.9883333333333	33.975\\
10.9886111111111	33.975\\
10.9888888888889	33.975\\
10.9891666666667	33.97\\
10.9894444444444	33.97\\
10.9897222222222	33.97\\
10.99	33.97\\
10.9902777777778	33.97\\
10.9905555555556	33.97\\
10.9908333333333	33.97\\
10.9911111111111	33.975\\
10.9913888888889	33.97\\
10.9916666666667	33.97\\
10.9919444444444	33.97\\
10.9922222222222	33.965\\
10.9925	33.965\\
10.9927777777778	33.965\\
10.9930555555556	33.965\\
10.9933333333333	33.955\\
10.9936111111111	33.965\\
10.9938888888889	33.965\\
10.9941666666667	33.965\\
10.9944444444444	33.975\\
10.9947222222222	33.975\\
10.995	33.965\\
10.9952777777778	33.975\\
10.9955555555556	33.975\\
10.9958333333333	33.975\\
10.9961111111111	33.975\\
10.9963888888889	33.975\\
10.9966666666667	33.975\\
10.9969444444444	33.97\\
10.9972222222222	33.97\\
10.9975	33.975\\
10.9977777777778	33.985\\
10.9980555555556	33.995\\
10.9983333333333	33.995\\
10.9986111111111	33.995\\
10.9988888888889	33.995\\
10.9991666666667	33.995\\
10.9994444444444	33.995\\
10.9997222222222	33.995\\
11	33.995\\
};
\addlegendentry{$S$};

\addplot [color=gray,solid,line width=2.0pt,forget plot]
  table[row sep=crcr]{%
10.975	34.01\\
10.9752777777778	34.01\\
10.9755555555556	34.01\\
10.9758333333333	34.01\\
10.9761111111111	34.01\\
10.9763888888889	34.01\\
10.9766666666667	34.01\\
10.9769444444444	34.01\\
10.9772222222222	34.01\\
10.9775	34.0052561166706\\
10.9777777777778	34.0052561166706\\
10.9780555555556	34.01\\
10.9783333333333	34.0062994730593\\
10.9786111111111	34.0062994730593\\
10.9788888888889	34.0062994730593\\
10.9791666666667	34.0071550219196\\
10.9794444444444	34.0001000871007\\
10.9797222222222	34.0071550219196\\
10.98	34.0036479471728\\
10.9802777777778	34.01\\
10.9805555555556	34.0071550219196\\
10.9808333333333	34.0071550219196\\
10.9811111111111	34.0011202389371\\
10.9813888888889	34\\
10.9816666666667	34\\
10.9819444444444	34\\
10.9822222222222	34\\
10.9825	34\\
10.9827777777778	34\\
10.9830555555556	34\\
10.9833333333333	33.9971550219196\\
10.9836111111111	33.9971550219196\\
10.9838888888889	33.9971550219196\\
10.9841666666667	33.9971550219196\\
10.9844444444444	34\\
10.9847222222222	34\\
10.985	33.9908354309502\\
10.9852777777778	33.9971550219196\\
10.9855555555556	34\\
10.9858333333333	33.9908354309502\\
10.9861111111111	33.9971550219196\\
10.9863888888889	33.99\\
10.9866666666667	33.99\\
10.9869444444444	33.99\\
10.9872222222222	33.99\\
10.9875	33.99\\
10.9877777777778	33.99\\
10.9880555555556	33.99\\
10.9883333333333	33.9886751273379\\
10.9886111111111	33.9873880856129\\
10.9888888888889	33.9857997647327\\
10.9891666666667	33.99\\
10.9894444444444	33.9857997647327\\
10.9897222222222	33.9857997647327\\
10.99	33.9873880856129\\
10.9902777777778	33.9873880856129\\
10.9905555555556	33.9873880856129\\
10.9908333333333	33.9873880856129\\
10.9911111111111	33.9811202389371\\
10.9913888888889	33.99\\
10.9916666666667	33.99\\
10.9919444444444	33.99\\
10.9922222222222	33.98\\
10.9925	33.98\\
10.9927777777778	33.98\\
10.9930555555556	33.98\\
10.9933333333333	33.97\\
10.9936111111111	33.9740236282595\\
10.9938888888889	33.98\\
10.9941666666667	33.98\\
10.9944444444444	33.989227427322\\
10.9947222222222	33.99\\
10.995	33.98\\
10.9952777777778	33.9818010884478\\
10.9955555555556	33.99\\
10.9958333333333	33.99\\
10.9961111111111	33.9873367199577\\
10.9963888888889	33.9852561166706\\
10.9966666666667	33.9852561166706\\
10.9969444444444	33.99\\
10.9972222222222	33.99\\
10.9975	33.9801000871007\\
10.9977777777778	33.992511774143\\
10.9980555555556	34.0049796089352\\
10.9983333333333	34.01\\
10.9986111111111	34.01\\
10.9988888888889	34.01\\
10.9991666666667	34.01\\
10.9994444444444	34.01\\
10.9997222222222	34.01\\
11	34.01\\
};
\addplot [color=gray,solid,line width=2.0pt]
  table[row sep=crcr]{%
10.975	33.99\\
10.9752777777778	33.99\\
10.9755555555556	33.99\\
10.9758333333333	33.99\\
10.9761111111111	33.99\\
10.9763888888889	33.99\\
10.9766666666667	33.99\\
10.9769444444444	33.9871550219196\\
10.9772222222222	33.9871550219196\\
10.9775	33.9847077974139\\
10.9777777777778	33.9847077974139\\
10.9780555555556	33.98\\
10.9783333333333	33.9757997647327\\
10.9786111111111	33.9757997647327\\
10.9788888888889	33.9757997647327\\
10.9791666666667	33.9762994730593\\
10.9794444444444	33.98\\
10.9797222222222	33.9862994730593\\
10.98	33.9833360719777\\
10.9802777777778	33.98\\
10.9805555555556	33.9762994730593\\
10.9808333333333	33.9762994730593\\
10.9811111111111	33.98\\
10.9813888888889	33.98\\
10.9816666666667	33.98\\
10.9819444444444	33.98\\
10.9822222222222	33.98\\
10.9825	33.98\\
10.9827777777778	33.98\\
10.9830555555556	33.98\\
10.9833333333333	33.9762994730593\\
10.9836111111111	33.9762994730593\\
10.9838888888889	33.9762994730593\\
10.9841666666667	33.9762994730593\\
10.9844444444444	33.97\\
10.9847222222222	33.97\\
10.985	33.97\\
10.9852777777778	33.9762994730593\\
10.9855555555556	33.97\\
10.9858333333333	33.97\\
10.9861111111111	33.9762994730593\\
10.9863888888889	33.97\\
10.9866666666667	33.97\\
10.9869444444444	33.97\\
10.9872222222222	33.97\\
10.9875	33.97\\
10.9877777777778	33.97\\
10.9880555555556	33.97\\
10.9883333333333	33.9673880856129\\
10.9886111111111	33.967182933211\\
10.9888888888889	33.96566340596\\
10.9891666666667	33.96\\
10.9894444444444	33.95566340596\\
10.9897222222222	33.95566340596\\
10.99	33.957182933211\\
10.9902777777778	33.957182933211\\
10.9905555555556	33.957182933211\\
10.9908333333333	33.957182933211\\
10.9911111111111	33.96\\
10.9913888888889	33.96\\
10.9916666666667	33.96\\
10.9919444444444	33.96\\
10.9922222222222	33.96\\
10.9925	33.96\\
10.9927777777778	33.96\\
10.9930555555556	33.96\\
10.9933333333333	33.95\\
10.9936111111111	33.9518010884478\\
10.9938888888889	33.96\\
10.9941666666667	33.96\\
10.9944444444444	33.9640236282595\\
10.9947222222222	33.97\\
10.995	33.96\\
10.9952777777778	33.96043693944\\
10.9955555555556	33.9671550219196\\
10.9958333333333	33.9671550219196\\
10.9961111111111	33.9652561166706\\
10.9963888888889	33.9647077974139\\
10.9966666666667	33.9647077974139\\
10.9969444444444	33.96\\
10.9972222222222	33.9586751273379\\
10.9975	33.96\\
10.9977777777778	33.9708354309501\\
10.9980555555556	33.982118374971\\
10.9983333333333	33.99\\
10.9986111111111	33.99\\
10.9988888888889	33.99\\
10.9991666666667	33.9871550219196\\
10.9994444444444	33.9871550219196\\
10.9997222222222	33.9871550219196\\
11	33.9871550219196\\
};
\addlegendentry{$S \pm \delta^\pm$};

\addplot [color=mycolor1,solid,line width=2.0pt,forget plot]
  table[row sep=crcr]{%
10.9791666666667	33.99\\
10.9791666666667	34.0071550219196\\
};
\addplot [color=mycolor1,solid,line width=2.0pt,mark=*,mark options={solid,fill=mycolor1},forget plot]
  table[row sep=crcr]{%
10.9791666666667	34.0071550219196\\
};
\addplot [color=mycolor1,solid,line width=2.0pt,forget plot]
  table[row sep=crcr]{%
10.9811111111111	33.995\\
10.9811111111111	34.0011202389371\\
};
\addplot [color=mycolor1,solid,line width=2.0pt,mark=*,mark options={solid,fill=mycolor1},forget plot]
  table[row sep=crcr]{%
10.9811111111111	34.0011202389371\\
};
\addplot [color=mycolor1,solid,line width=2.0pt,forget plot]
  table[row sep=crcr]{%
10.9913888888889	33.97\\
10.9913888888889	33.99\\
};
\addplot [color=mycolor1,solid,line width=2.0pt,mark=*,mark options={solid,fill=mycolor1},forget plot]
  table[row sep=crcr]{%
10.9913888888889	33.99\\
};
\addplot [color=mycolor1,solid,line width=2.0pt,forget plot]
  table[row sep=crcr]{%
10.9936111111111	33.965\\
10.9936111111111	33.9740236282595\\
};
\addplot [color=mycolor1,solid,line width=2.0pt,mark=*,mark options={solid,fill=mycolor1},forget plot]
  table[row sep=crcr]{%
10.9936111111111	33.9740236282595\\
};
\addplot [color=mycolor1,solid,line width=2.0pt,forget plot]
  table[row sep=crcr]{%
10.9944444444444	33.975\\
10.9944444444444	33.989227427322\\
};
\addplot [color=mycolor1,solid,line width=2.0pt,mark=*,mark options={solid,fill=mycolor1},forget plot]
  table[row sep=crcr]{%
10.9944444444444	33.989227427322\\
};
\addplot [color=mycolor1,solid,line width=2.0pt,forget plot]
  table[row sep=crcr]{%
10.9977777777778	33.985\\
10.9977777777778	33.992511774143\\
};
\addplot [color=mycolor1,solid,line width=2.0pt,mark=*,mark options={solid,fill=mycolor1},forget plot]
  table[row sep=crcr]{%
10.9977777777778	33.992511774143\\
};
\addplot [color=mycolor1,solid,line width=2.0pt,forget plot]
  table[row sep=crcr]{%
10.9980555555556	33.995\\
10.9980555555556	34.0049796089352\\
};
\addplot [color=mycolor1,solid,line width=2.0pt,mark=*,mark options={solid,fill=mycolor1},forget plot]
  table[row sep=crcr]{%
10.9980555555556	34.0049796089352\\
};
\addplot [color=mycolor1,solid,line width=2.0pt,forget plot]
  table[row sep=crcr]{%
10.9983333333333	33.995\\
10.9983333333333	34.01\\
};
\addplot [color=mycolor1,solid,line width=2.0pt,mark=*,mark options={solid,fill=mycolor1}]
  table[row sep=crcr]{%
10.9983333333333	34.01\\
};
\addlegendentry{Ext Buy MO lifts our sell LO};

\addplot [color=mycolor1,dashed,line width=2.0pt,forget plot]
  table[row sep=crcr]{%
10.9911111111111	33.975\\
10.9911111111111	33.98\\
};
\addplot [color=mycolor1,dashed,line width=2.0pt,mark=o,mark options={solid},forget plot]
  table[row sep=crcr]{%
10.9911111111111	33.98\\
};
\addplot [color=mycolor1,dashed,line width=2.0pt,forget plot]
  table[row sep=crcr]{%
10.9952777777778	33.975\\
10.9952777777778	33.97\\
};
\addplot [color=mycolor1,dashed,line width=2.0pt,mark=o,mark options={solid},forget plot]
  table[row sep=crcr]{%
10.9952777777778	33.97\\
};
\addplot [color=mycolor1,dashed,line width=2.0pt,forget plot]
  table[row sep=crcr]{%
10.9975	33.975\\
10.9975	33.98\\
};
\addplot [color=mycolor1,dashed,line width=2.0pt,mark=o,mark options={solid}]
  table[row sep=crcr]{%
10.9975	33.98\\
};
\addlegendentry{Ext Buy MO arrives};

\addplot [color=mycolor2,solid,line width=2.0pt,forget plot]
  table[row sep=crcr]{%
10.9769444444444	33.995\\
10.9769444444444	33.9871550219196\\
};
\addplot [color=mycolor2,solid,line width=2.0pt,mark=*,mark options={solid,fill=mycolor2},forget plot]
  table[row sep=crcr]{%
10.9769444444444	33.9871550219196\\
};
\addplot [color=mycolor2,solid,line width=2.0pt,forget plot]
  table[row sep=crcr]{%
10.9775	33.995\\
10.9775	33.9847077974139\\
};
\addplot [color=mycolor2,solid,line width=2.0pt,mark=*,mark options={solid,fill=mycolor2},forget plot]
  table[row sep=crcr]{%
10.9775	33.9847077974139\\
};
\addplot [color=mycolor2,solid,line width=2.0pt,forget plot]
  table[row sep=crcr]{%
10.9780555555556	33.99\\
10.9780555555556	33.98\\
};
\addplot [color=mycolor2,solid,line width=2.0pt,mark=*,mark options={solid,fill=mycolor2},forget plot]
  table[row sep=crcr]{%
10.9780555555556	33.98\\
};
\addplot [color=mycolor2,solid,line width=2.0pt,forget plot]
  table[row sep=crcr]{%
10.9833333333333	33.985\\
10.9833333333333	33.9762994730593\\
};
\addplot [color=mycolor2,solid,line width=2.0pt,mark=*,mark options={solid,fill=mycolor2},forget plot]
  table[row sep=crcr]{%
10.9833333333333	33.9762994730593\\
};
\addplot [color=mycolor2,solid,line width=2.0pt,forget plot]
  table[row sep=crcr]{%
10.9855555555556	33.98\\
10.9855555555556	33.97\\
};
\addplot [color=mycolor2,solid,line width=2.0pt,mark=*,mark options={solid,fill=mycolor2},forget plot]
  table[row sep=crcr]{%
10.9855555555556	33.97\\
};
\addplot [color=mycolor2,solid,line width=2.0pt,forget plot]
  table[row sep=crcr]{%
10.9880555555556	33.975\\
10.9880555555556	33.97\\
};
\addplot [color=mycolor2,solid,line width=2.0pt,mark=*,mark options={solid,fill=mycolor2},forget plot]
  table[row sep=crcr]{%
10.9880555555556	33.97\\
};
\addplot [color=mycolor2,solid,line width=2.0pt,forget plot]
  table[row sep=crcr]{%
10.9886111111111	33.975\\
10.9886111111111	33.967182933211\\
};
\addplot [color=mycolor2,solid,line width=2.0pt,mark=*,mark options={solid,fill=mycolor2},forget plot]
  table[row sep=crcr]{%
10.9886111111111	33.967182933211\\
};
\addplot [color=mycolor2,solid,line width=2.0pt,forget plot]
  table[row sep=crcr]{%
10.9933333333333	33.955\\
10.9933333333333	33.95\\
};
\addplot [color=mycolor2,solid,line width=2.0pt,mark=*,mark options={solid,fill=mycolor2},forget plot]
  table[row sep=crcr]{%
10.9933333333333	33.95\\
};
\addplot [color=mycolor2,solid,line width=2.0pt,forget plot]
  table[row sep=crcr]{%
10.995	33.965\\
10.995	33.96\\
};
\addplot [color=mycolor2,solid,line width=2.0pt,mark=*,mark options={solid,fill=mycolor2},forget plot]
  table[row sep=crcr]{%
10.995	33.96\\
};
\addplot [color=mycolor2,solid,line width=2.0pt,forget plot]
  table[row sep=crcr]{%
10.9963888888889	33.975\\
10.9963888888889	33.9647077974139\\
};
\addplot [color=mycolor2,solid,line width=2.0pt,mark=*,mark options={solid,fill=mycolor2},forget plot]
  table[row sep=crcr]{%
10.9963888888889	33.9647077974139\\
};
\addplot [color=mycolor2,solid,line width=2.0pt,forget plot]
  table[row sep=crcr]{%
10.9988888888889	33.995\\
10.9988888888889	33.99\\
};
\addplot [color=mycolor2,solid,line width=2.0pt,mark=*,mark options={solid,fill=mycolor2}]
  table[row sep=crcr]{%
10.9988888888889	33.99\\
};
\addlegendentry{Ext Sell MO lifts our buy LO};

\addplot [color=mycolor2,dashed,line width=2.0pt,forget plot]
  table[row sep=crcr]{%
10.9802777777778	33.99\\
10.9802777777778	33.99\\
};
\addplot [color=mycolor2,dashed,line width=2.0pt,mark=o,mark options={solid},forget plot]
  table[row sep=crcr]{%
10.9802777777778	33.99\\
};
\addplot [color=mycolor2,dashed,line width=2.0pt,forget plot]
  table[row sep=crcr]{%
10.9813888888889	33.985\\
10.9813888888889	33.99\\
};
\addplot [color=mycolor2,dashed,line width=2.0pt,mark=o,mark options={solid},forget plot]
  table[row sep=crcr]{%
10.9813888888889	33.99\\
};
\addplot [color=mycolor2,dashed,line width=2.0pt,forget plot]
  table[row sep=crcr]{%
10.9841666666667	33.985\\
10.9841666666667	33.98\\
};
\addplot [color=mycolor2,dashed,line width=2.0pt,mark=o,mark options={solid},forget plot]
  table[row sep=crcr]{%
10.9841666666667	33.98\\
};
\addplot [color=mycolor2,dashed,line width=2.0pt,forget plot]
  table[row sep=crcr]{%
10.9863888888889	33.975\\
10.9863888888889	33.98\\
};
\addplot [color=mycolor2,dashed,line width=2.0pt,mark=o,mark options={solid},forget plot]
  table[row sep=crcr]{%
10.9863888888889	33.98\\
};
\addplot [color=mycolor2,dashed,line width=2.0pt,forget plot]
  table[row sep=crcr]{%
10.9891666666667	33.97\\
10.9891666666667	33.97\\
};
\addplot [color=mycolor2,dashed,line width=2.0pt,mark=o,mark options={solid},forget plot]
  table[row sep=crcr]{%
10.9891666666667	33.97\\
};
\addplot [color=mycolor2,dashed,line width=2.0pt,forget plot]
  table[row sep=crcr]{%
10.9913888888889	33.97\\
10.9913888888889	33.97\\
};
\addplot [color=mycolor2,dashed,line width=2.0pt,mark=o,mark options={solid},forget plot]
  table[row sep=crcr]{%
10.9913888888889	33.97\\
};
\addplot [color=mycolor2,dashed,line width=2.0pt,forget plot]
  table[row sep=crcr]{%
10.9969444444444	33.97\\
10.9969444444444	33.97\\
};
\addplot [color=mycolor2,dashed,line width=2.0pt,mark=o,mark options={solid},forget plot]
  table[row sep=crcr]{%
10.9969444444444	33.97\\
};
\addplot [color=mycolor2,dashed,line width=2.0pt,forget plot]
  table[row sep=crcr]{%
10.9980555555556	33.995\\
10.9980555555556	34\\
};
\addplot [color=mycolor2,dashed,line width=2.0pt,mark=o,mark options={solid}]
  table[row sep=crcr]{%
10.9980555555556	34\\
};
\addlegendentry{Ext Sell MO arrives};

\addplot [color=mycolor3,solid,line width=2.0pt,forget plot]
  table[row sep=crcr]{%
10.975	33.995\\
10.975	34\\
};
\addplot [color=mycolor3,solid,line width=2.0pt,mark=*,mark options={solid,fill=mycolor3},forget plot]
  table[row sep=crcr]{%
10.975	34\\
};
\addplot [color=mycolor3,solid,line width=2.0pt,forget plot]
  table[row sep=crcr]{%
10.9947222222222	33.975\\
10.9947222222222	33.98\\
};
\addplot [color=mycolor3,solid,line width=2.0pt,mark=*,mark options={solid,fill=mycolor3},forget plot]
  table[row sep=crcr]{%
10.9947222222222	33.98\\
};
\addplot [color=mycolor3,solid,line width=2.0pt,forget plot]
  table[row sep=crcr]{%
10.9983333333333	33.995\\
10.9983333333333	34\\
};
\addplot [color=mycolor3,solid,line width=2.0pt,mark=*,mark options={solid,fill=mycolor3}]
  table[row sep=crcr]{%
10.9983333333333	34\\
};
\addlegendentry{Our Buy MO};

\addplot [color=mycolor4,solid,line width=2.0pt,forget plot]
  table[row sep=crcr]{%
10.9858333333333	33.985\\
10.9858333333333	33.98\\
};
\addplot [color=mycolor4,solid,line width=2.0pt,mark=*,mark options={solid,fill=mycolor4},forget plot]
  table[row sep=crcr]{%
10.9858333333333	33.98\\
};
\addplot [color=mycolor4,solid,line width=2.0pt,forget plot]
  table[row sep=crcr]{%
10.9911111111111	33.975\\
10.9911111111111	33.97\\
};
\addplot [color=mycolor4,solid,line width=2.0pt,mark=*,mark options={solid,fill=mycolor4}]
  table[row sep=crcr]{%
10.9911111111111	33.97\\
};
\addlegendentry{Our Sell MO};

\end{axis}
\end{tikzpicture}%

\end{subfigure}%
\hfill%
\begin{subfigure}{.45\linewidth}
  \centering
  \setlength\figureheight{\linewidth} 
  \setlength\figurewidth{\linewidth}
  \tikzsetnextfilename{samplepath_dscr_paths}
  % This file was created by matlab2tikz.
%
%The latest updates can be retrieved from
%  http://www.mathworks.com/matlabcentral/fileexchange/22022-matlab2tikz-matlab2tikz
%where you can also make suggestions and rate matlab2tikz.
%
\definecolor{mycolor1}{rgb}{0.65098,0.80784,0.89020}%
\definecolor{mycolor2}{rgb}{0.69804,0.87451,0.54118}%
\definecolor{mycolor3}{rgb}{0.20000,0.62745,0.17255}%
\definecolor{mycolor4}{rgb}{0.12157,0.47059,0.70588}%
%
\begin{tikzpicture}[trim axis left, trim axis right]

\begin{axis}[%
width=\figurewidth,
height=\figureheight,
at={(0\figurewidth,0\figureheight)},
scale only axis,
every outer x axis line/.append style={black},
every x tick label/.append style={font=\color{black}},
xmin=10.975,
xmax=11,
every outer y axis line/.append style={black},
every y tick label/.append style={font=\color{black}},
ymin=33.93,
ymax=34.03,
axis background/.style={fill=white},
axis x line*=bottom,
axis y line*=left,
legend style={legend cell align=left,align=left,draw=black,legend pos = south west},
every axis legend/.code={\renewcommand\addlegendentry[2][]{}}  %ignore legend locally
]
\addplot [color=black,solid,line width=3.0pt]
  table[row sep=crcr]{%
10.975	33.995\\
10.9752777777778	33.995\\
10.9755555555556	33.995\\
10.9758333333333	33.995\\
10.9761111111111	33.995\\
10.9763888888889	33.995\\
10.9766666666667	33.995\\
10.9769444444444	33.995\\
10.9772222222222	33.995\\
10.9775	33.995\\
10.9777777777778	33.995\\
10.9780555555556	33.99\\
10.9783333333333	33.99\\
10.9786111111111	33.99\\
10.9788888888889	33.99\\
10.9791666666667	33.99\\
10.9794444444444	33.995\\
10.9797222222222	33.995\\
10.98	33.995\\
10.9802777777778	33.99\\
10.9805555555556	33.99\\
10.9808333333333	33.99\\
10.9811111111111	33.995\\
10.9813888888889	33.985\\
10.9816666666667	33.985\\
10.9819444444444	33.985\\
10.9822222222222	33.985\\
10.9825	33.985\\
10.9827777777778	33.985\\
10.9830555555556	33.985\\
10.9833333333333	33.985\\
10.9836111111111	33.985\\
10.9838888888889	33.985\\
10.9841666666667	33.985\\
10.9844444444444	33.98\\
10.9847222222222	33.98\\
10.985	33.985\\
10.9852777777778	33.985\\
10.9855555555556	33.98\\
10.9858333333333	33.985\\
10.9861111111111	33.985\\
10.9863888888889	33.975\\
10.9866666666667	33.975\\
10.9869444444444	33.975\\
10.9872222222222	33.975\\
10.9875	33.975\\
10.9877777777778	33.975\\
10.9880555555556	33.975\\
10.9883333333333	33.975\\
10.9886111111111	33.975\\
10.9888888888889	33.975\\
10.9891666666667	33.97\\
10.9894444444444	33.97\\
10.9897222222222	33.97\\
10.99	33.97\\
10.9902777777778	33.97\\
10.9905555555556	33.97\\
10.9908333333333	33.97\\
10.9911111111111	33.975\\
10.9913888888889	33.97\\
10.9916666666667	33.97\\
10.9919444444444	33.97\\
10.9922222222222	33.965\\
10.9925	33.965\\
10.9927777777778	33.965\\
10.9930555555556	33.965\\
10.9933333333333	33.955\\
10.9936111111111	33.965\\
10.9938888888889	33.965\\
10.9941666666667	33.965\\
10.9944444444444	33.975\\
10.9947222222222	33.975\\
10.995	33.965\\
10.9952777777778	33.975\\
10.9955555555556	33.975\\
10.9958333333333	33.975\\
10.9961111111111	33.975\\
10.9963888888889	33.975\\
10.9966666666667	33.975\\
10.9969444444444	33.97\\
10.9972222222222	33.97\\
10.9975	33.975\\
10.9977777777778	33.985\\
10.9980555555556	33.995\\
10.9983333333333	33.995\\
10.9986111111111	33.995\\
10.9988888888889	33.995\\
10.9991666666667	33.995\\
10.9994444444444	33.995\\
10.9997222222222	33.995\\
11	33.995\\
};
\addlegendentry{$S$};

\addplot [color=gray,solid,line width=2.0pt,forget plot]
  table[row sep=crcr]{%
10.975	34.0059182180567\\
10.9752777777778	34.0059182180567\\
10.9755555555556	34.0059182180567\\
10.9758333333333	34.0059182180567\\
10.9761111111111	34.0059182180567\\
10.9763888888889	34.0059182180567\\
10.9766666666667	34.0059182180567\\
10.9769444444444	34.0056834278805\\
10.9772222222222	34.0056834278805\\
10.9775	34.003372553231\\
10.9777777777778	34.003372553231\\
10.9780555555556	34.0011849949641\\
10.9783333333333	34.0056119818102\\
10.9786111111111	34.0056119818102\\
10.9788888888889	34.0056119818102\\
10.9791666666667	34.0056119818102\\
10.9794444444444	34.0108655723145\\
10.9797222222222	34.0056119818102\\
10.98	34.0010067763665\\
10.9802777777778	34.0011380381246\\
10.9805555555556	34.005546685923\\
10.9808333333333	34.005546685923\\
10.9811111111111	34.0090040506672\\
10.9813888888889	33.9911380381246\\
10.9816666666667	34.0000275388031\\
10.9819444444444	34.0000275388031\\
10.9822222222222	33.9977551732797\\
10.9825	33.9977551732797\\
10.9827777777778	33.9977551732797\\
10.9830555555556	33.9977551732797\\
10.9833333333333	33.9954944216281\\
10.9836111111111	33.9954944216281\\
10.9838888888889	33.9954944216281\\
10.9841666666667	33.9954684661003\\
10.9844444444444	33.993101665781\\
10.9847222222222	34.0000186652223\\
10.985	34.0011119245755\\
10.9852777777778	33.9954684661003\\
10.9855555555556	33.9910686277899\\
10.9858333333333	34.0011006501926\\
10.9861111111111	33.9954478023027\\
10.9863888888889	33.9810514999642\\
10.9866666666667	33.9900156366446\\
10.9869444444444	33.9900156366446\\
10.9872222222222	33.9900156366446\\
10.9875	33.9900156366446\\
10.9877777777778	33.9900156366446\\
10.9880555555556	33.9900140341327\\
10.9883333333333	33.9876562360065\\
10.9886111111111	33.9876375213784\\
10.9888888888889	33.9853790049406\\
10.9891666666667	33.9810116677959\\
10.9894444444444	33.9853790049406\\
10.9897222222222	33.9853790049406\\
10.99	33.9876375213784\\
10.9902777777778	33.9876375213784\\
10.9905555555556	33.9876375213784\\
10.9908333333333	33.9876375213784\\
10.9911111111111	33.9888198884016\\
10.9913888888889	33.9809874902617\\
10.9916666666667	33.9876161066837\\
10.9919444444444	33.9876161066837\\
10.9922222222222	33.9730379405805\\
10.9925	33.9800102121667\\
10.9927777777778	33.9800102121667\\
10.9930555555556	33.9800102121667\\
10.9933333333333	33.9642875309228\\
10.9936111111111	33.9826749211683\\
10.9938888888889	33.9800102121667\\
10.9941666666667	33.9800102121667\\
10.9944444444444	33.9926749211683\\
10.9947222222222	33.9853499342462\\
10.995	33.9709596128336\\
10.9952777777778	33.9926749211683\\
10.9955555555556	33.9853499342462\\
10.9958333333333	33.9853499342462\\
10.9961111111111	33.9830762963274\\
10.9963888888889	33.9830762963274\\
10.9966666666667	33.9830762963274\\
10.9969444444444	33.9809874902617\\
10.9972222222222	33.9876161066837\\
10.9975	33.9907069403664\\
10.9977777777778	34.0010643450663\\
10.9980555555556	34.0106903541472\\
10.9983333333333	34.0076375213784\\
10.9986111111111	34.0100122475993\\
10.9988888888889	34.0076161066837\\
10.9991666666667	34.0053499342462\\
10.9994444444444	34.0053499342462\\
10.9997222222222	34.0053499342462\\
11	34.0053499342462\\
};
\addplot [color=gray,solid,line width=2.0pt]
  table[row sep=crcr]{%
10.975	33.9856881493956\\
10.9752777777778	33.9856881493956\\
10.9755555555556	33.9856881493956\\
10.9758333333333	33.9856881493956\\
10.9761111111111	33.9856881493956\\
10.9763888888889	33.9856881493956\\
10.9766666666667	33.9856881493956\\
10.9769444444444	33.9856142359539\\
10.9772222222222	33.9856142359539\\
10.9775	33.9832955920654\\
10.9777777777778	33.9832955920654\\
10.9780555555556	33.9711624114765\\
10.9783333333333	33.9755486178901\\
10.9786111111111	33.9755486178901\\
10.9788888888889	33.9755486178901\\
10.9791666666667	33.9755486178901\\
10.9794444444444	33.99\\
10.9797222222222	33.9855486178901\\
10.98	33.9809099219393\\
10.9802777777778	33.9711208098915\\
10.9805555555556	33.9754956793259\\
10.9808333333333	33.9754956793259\\
10.9811111111111	33.9889849747738\\
10.9813888888889	33.9711208098915\\
10.9816666666667	33.98\\
10.9819444444444	33.98\\
10.9822222222222	33.9777226346755\\
10.9825	33.9777226346755\\
10.9827777777778	33.9777226346755\\
10.9830555555556	33.9777226346755\\
10.9833333333333	33.9754692311835\\
10.9836111111111	33.9754692311835\\
10.9838888888889	33.9754692311835\\
10.9841666666667	33.9754484906363\\
10.9844444444444	33.9630985298242\\
10.9847222222222	33.97\\
10.985	33.98\\
10.9852777777778	33.9754484906363\\
10.9855555555556	33.9610612302116\\
10.9858333333333	33.98\\
10.9861111111111	33.9754276427221\\
10.9863888888889	33.9610434964238\\
10.9866666666667	33.97\\
10.9869444444444	33.97\\
10.9872222222222	33.97\\
10.9875	33.97\\
10.9877777777778	33.97\\
10.9880555555556	33.97\\
10.9883333333333	33.9676383300123\\
10.9886111111111	33.9676170353958\\
10.9888888888889	33.965350979661\\
10.9891666666667	33.9510014369277\\
10.9894444444444	33.955350979661\\
10.9897222222222	33.955350979661\\
10.99	33.9576170353958\\
10.9902777777778	33.9576170353958\\
10.9905555555556	33.9576170353958\\
10.9908333333333	33.9576170353958\\
10.9911111111111	33.968814477513\\
10.9913888888889	33.9509756900264\\
10.9916666666667	33.9575924686509\\
10.9919444444444	33.9575924686509\\
10.9922222222222	33.9530335411951\\
10.9925	33.96\\
10.9927777777778	33.96\\
10.9930555555556	33.96\\
10.9933333333333	33.9442861734524\\
10.9936111111111	33.96\\
10.9938888888889	33.96\\
10.9941666666667	33.96\\
10.9944444444444	33.97\\
10.9947222222222	33.9653172698093\\
10.995	33.9509459789529\\
10.9952777777778	33.97\\
10.9955555555556	33.9653172698093\\
10.9958333333333	33.9653172698093\\
10.9961111111111	33.9630430896799\\
10.9963888888889	33.9630430896799\\
10.9966666666667	33.9630430896799\\
10.9969444444444	33.9509756900264\\
10.9972222222222	33.9575924686509\\
10.9975	33.97\\
10.9977777777778	33.98\\
10.9980555555556	33.99\\
10.9983333333333	33.9876170353958\\
10.9986111111111	33.99\\
10.9988888888889	33.9875924686509\\
10.9991666666667	33.9853172698093\\
10.9994444444444	33.9853172698093\\
10.9997222222222	33.9853172698093\\
11	33.9853172698093\\
};
\addlegendentry{$S \pm \delta^\pm$};

\addplot [color=mycolor1,solid,line width=2.0pt,forget plot]
  table[row sep=crcr]{%
10.9811111111111	33.995\\
10.9811111111111	34.0090040506672\\
};
\addplot [color=mycolor1,solid,line width=2.0pt,mark=*,mark options={solid,fill=mycolor1},forget plot]
  table[row sep=crcr]{%
10.9811111111111	34.0090040506672\\
};
\addplot [color=mycolor1,solid,line width=2.0pt,forget plot]
  table[row sep=crcr]{%
10.9936111111111	33.965\\
10.9936111111111	33.9826749211683\\
};
\addplot [color=mycolor1,solid,line width=2.0pt,mark=*,mark options={solid,fill=mycolor1},forget plot]
  table[row sep=crcr]{%
10.9936111111111	33.9826749211683\\
};
\addplot [color=mycolor1,solid,line width=2.0pt,forget plot]
  table[row sep=crcr]{%
10.9952777777778	33.975\\
10.9952777777778	33.9926749211683\\
};
\addplot [color=mycolor1,solid,line width=2.0pt,mark=*,mark options={solid,fill=mycolor1},forget plot]
  table[row sep=crcr]{%
10.9952777777778	33.9926749211683\\
};
\addplot [color=mycolor1,solid,line width=2.0pt,forget plot]
  table[row sep=crcr]{%
10.9975	33.975\\
10.9975	33.9907069403664\\
};
\addplot [color=mycolor1,solid,line width=2.0pt,mark=*,mark options={solid,fill=mycolor1},forget plot]
  table[row sep=crcr]{%
10.9975	33.9907069403664\\
};
\addplot [color=mycolor1,solid,line width=2.0pt,forget plot]
  table[row sep=crcr]{%
10.9983333333333	33.995\\
10.9983333333333	34.0076375213784\\
};
\addplot [color=mycolor1,solid,line width=2.0pt,mark=*,mark options={solid,fill=mycolor1}]
  table[row sep=crcr]{%
10.9983333333333	34.0076375213784\\
};
\addlegendentry{Ext Buy MO lifts our sell LO};

\addplot [color=mycolor1,dashed,line width=2.0pt,forget plot]
  table[row sep=crcr]{%
10.9791666666667	33.99\\
10.9791666666667	34\\
};
\addplot [color=mycolor1,dashed,line width=2.0pt,mark=o,mark options={solid},forget plot]
  table[row sep=crcr]{%
10.9791666666667	34\\
};
\addplot [color=mycolor1,dashed,line width=2.0pt,forget plot]
  table[row sep=crcr]{%
10.9911111111111	33.975\\
10.9911111111111	33.98\\
};
\addplot [color=mycolor1,dashed,line width=2.0pt,mark=o,mark options={solid},forget plot]
  table[row sep=crcr]{%
10.9911111111111	33.98\\
};
\addplot [color=mycolor1,dashed,line width=2.0pt,forget plot]
  table[row sep=crcr]{%
10.9913888888889	33.97\\
10.9913888888889	33.98\\
};
\addplot [color=mycolor1,dashed,line width=2.0pt,mark=o,mark options={solid},forget plot]
  table[row sep=crcr]{%
10.9913888888889	33.98\\
};
\addplot [color=mycolor1,dashed,line width=2.0pt,forget plot]
  table[row sep=crcr]{%
10.9944444444444	33.975\\
10.9944444444444	33.97\\
};
\addplot [color=mycolor1,dashed,line width=2.0pt,mark=o,mark options={solid},forget plot]
  table[row sep=crcr]{%
10.9944444444444	33.97\\
};
\addplot [color=mycolor1,dashed,line width=2.0pt,forget plot]
  table[row sep=crcr]{%
10.9977777777778	33.985\\
10.9977777777778	33.98\\
};
\addplot [color=mycolor1,dashed,line width=2.0pt,mark=o,mark options={solid},forget plot]
  table[row sep=crcr]{%
10.9977777777778	33.98\\
};
\addplot [color=mycolor1,dashed,line width=2.0pt,forget plot]
  table[row sep=crcr]{%
10.9980555555556	33.995\\
10.9980555555556	34\\
};
\addplot [color=mycolor1,dashed,line width=2.0pt,mark=o,mark options={solid}]
  table[row sep=crcr]{%
10.9980555555556	34\\
};
\addlegendentry{Ext Buy MO arrives};

\addplot [color=mycolor2,solid,line width=2.0pt,forget plot]
  table[row sep=crcr]{%
10.9769444444444	33.995\\
10.9769444444444	33.9856142359539\\
};
\addplot [color=mycolor2,solid,line width=2.0pt,mark=*,mark options={solid,fill=mycolor2},forget plot]
  table[row sep=crcr]{%
10.9769444444444	33.9856142359539\\
};
\addplot [color=mycolor2,solid,line width=2.0pt,forget plot]
  table[row sep=crcr]{%
10.9775	33.995\\
10.9775	33.9832955920654\\
};
\addplot [color=mycolor2,solid,line width=2.0pt,mark=*,mark options={solid,fill=mycolor2},forget plot]
  table[row sep=crcr]{%
10.9775	33.9832955920654\\
};
\addplot [color=mycolor2,solid,line width=2.0pt,forget plot]
  table[row sep=crcr]{%
10.9802777777778	33.99\\
10.9802777777778	33.9711208098915\\
};
\addplot [color=mycolor2,solid,line width=2.0pt,mark=*,mark options={solid,fill=mycolor2},forget plot]
  table[row sep=crcr]{%
10.9802777777778	33.9711208098915\\
};
\addplot [color=mycolor2,solid,line width=2.0pt,forget plot]
  table[row sep=crcr]{%
10.9813888888889	33.985\\
10.9813888888889	33.9711208098915\\
};
\addplot [color=mycolor2,solid,line width=2.0pt,mark=*,mark options={solid,fill=mycolor2},forget plot]
  table[row sep=crcr]{%
10.9813888888889	33.9711208098915\\
};
\addplot [color=mycolor2,solid,line width=2.0pt,forget plot]
  table[row sep=crcr]{%
10.9833333333333	33.985\\
10.9833333333333	33.9754692311835\\
};
\addplot [color=mycolor2,solid,line width=2.0pt,mark=*,mark options={solid,fill=mycolor2},forget plot]
  table[row sep=crcr]{%
10.9833333333333	33.9754692311835\\
};
\addplot [color=mycolor2,solid,line width=2.0pt,forget plot]
  table[row sep=crcr]{%
10.9841666666667	33.985\\
10.9841666666667	33.9754484906363\\
};
\addplot [color=mycolor2,solid,line width=2.0pt,mark=*,mark options={solid,fill=mycolor2},forget plot]
  table[row sep=crcr]{%
10.9841666666667	33.9754484906363\\
};
\addplot [color=mycolor2,solid,line width=2.0pt,forget plot]
  table[row sep=crcr]{%
10.9855555555556	33.98\\
10.9855555555556	33.9610612302116\\
};
\addplot [color=mycolor2,solid,line width=2.0pt,mark=*,mark options={solid,fill=mycolor2},forget plot]
  table[row sep=crcr]{%
10.9855555555556	33.9610612302116\\
};
\addplot [color=mycolor2,solid,line width=2.0pt,forget plot]
  table[row sep=crcr]{%
10.9863888888889	33.975\\
10.9863888888889	33.9610434964238\\
};
\addplot [color=mycolor2,solid,line width=2.0pt,mark=*,mark options={solid,fill=mycolor2},forget plot]
  table[row sep=crcr]{%
10.9863888888889	33.9610434964238\\
};
\addplot [color=mycolor2,solid,line width=2.0pt,forget plot]
  table[row sep=crcr]{%
10.9880555555556	33.975\\
10.9880555555556	33.97\\
};
\addplot [color=mycolor2,solid,line width=2.0pt,mark=*,mark options={solid,fill=mycolor2},forget plot]
  table[row sep=crcr]{%
10.9880555555556	33.97\\
};
\addplot [color=mycolor2,solid,line width=2.0pt,forget plot]
  table[row sep=crcr]{%
10.9886111111111	33.975\\
10.9886111111111	33.9676170353958\\
};
\addplot [color=mycolor2,solid,line width=2.0pt,mark=*,mark options={solid,fill=mycolor2},forget plot]
  table[row sep=crcr]{%
10.9886111111111	33.9676170353958\\
};
\addplot [color=mycolor2,solid,line width=2.0pt,forget plot]
  table[row sep=crcr]{%
10.9913888888889	33.97\\
10.9913888888889	33.9509756900264\\
};
\addplot [color=mycolor2,solid,line width=2.0pt,mark=*,mark options={solid,fill=mycolor2},forget plot]
  table[row sep=crcr]{%
10.9913888888889	33.9509756900264\\
};
\addplot [color=mycolor2,solid,line width=2.0pt,forget plot]
  table[row sep=crcr]{%
10.9933333333333	33.955\\
10.9933333333333	33.9442861734524\\
};
\addplot [color=mycolor2,solid,line width=2.0pt,mark=*,mark options={solid,fill=mycolor2},forget plot]
  table[row sep=crcr]{%
10.9933333333333	33.9442861734524\\
};
\addplot [color=mycolor2,solid,line width=2.0pt,forget plot]
  table[row sep=crcr]{%
10.995	33.965\\
10.995	33.9509459789529\\
};
\addplot [color=mycolor2,solid,line width=2.0pt,mark=*,mark options={solid,fill=mycolor2},forget plot]
  table[row sep=crcr]{%
10.995	33.9509459789529\\
};
\addplot [color=mycolor2,solid,line width=2.0pt,forget plot]
  table[row sep=crcr]{%
10.9980555555556	33.995\\
10.9980555555556	33.99\\
};
\addplot [color=mycolor2,solid,line width=2.0pt,mark=*,mark options={solid,fill=mycolor2},forget plot]
  table[row sep=crcr]{%
10.9980555555556	33.99\\
};
\addplot [color=mycolor2,solid,line width=2.0pt,forget plot]
  table[row sep=crcr]{%
10.9988888888889	33.995\\
10.9988888888889	33.9875924686509\\
};
\addplot [color=mycolor2,solid,line width=2.0pt,mark=*,mark options={solid,fill=mycolor2}]
  table[row sep=crcr]{%
10.9988888888889	33.9875924686509\\
};
\addlegendentry{Ext Sell MO lifts our buy LO};

\addplot [color=mycolor2,dashed,line width=2.0pt,forget plot]
  table[row sep=crcr]{%
10.9780555555556	33.99\\
10.9780555555556	33.99\\
};
\addplot [color=mycolor2,dashed,line width=2.0pt,mark=o,mark options={solid},forget plot]
  table[row sep=crcr]{%
10.9780555555556	33.99\\
};
\addplot [color=mycolor2,dashed,line width=2.0pt,forget plot]
  table[row sep=crcr]{%
10.9891666666667	33.97\\
10.9891666666667	33.97\\
};
\addplot [color=mycolor2,dashed,line width=2.0pt,mark=o,mark options={solid},forget plot]
  table[row sep=crcr]{%
10.9891666666667	33.97\\
};
\addplot [color=mycolor2,dashed,line width=2.0pt,forget plot]
  table[row sep=crcr]{%
10.9963888888889	33.975\\
10.9963888888889	33.97\\
};
\addplot [color=mycolor2,dashed,line width=2.0pt,mark=o,mark options={solid},forget plot]
  table[row sep=crcr]{%
10.9963888888889	33.97\\
};
\addplot [color=mycolor2,dashed,line width=2.0pt,forget plot]
  table[row sep=crcr]{%
10.9969444444444	33.97\\
10.9969444444444	33.97\\
};
\addplot [color=mycolor2,dashed,line width=2.0pt,mark=o,mark options={solid}]
  table[row sep=crcr]{%
10.9969444444444	33.97\\
};
\addlegendentry{Ext Sell MO arrives};

\end{axis}
\end{tikzpicture}%
 
\end{subfigure}\\
\vspace{1cm}
\begin{subfigure}{.45\linewidth}
  \centering
  \setlength\figureheight{\linewidth} 
  \setlength\figurewidth{\linewidth}
  \tikzsetnextfilename{samplepath_cts_nFPC_paths}
  % This file was created by matlab2tikz.
%
%The latest updates can be retrieved from
%  http://www.mathworks.com/matlabcentral/fileexchange/22022-matlab2tikz-matlab2tikz
%where you can also make suggestions and rate matlab2tikz.
%
\definecolor{mycolor1}{rgb}{0.65098,0.80784,0.89020}%
\definecolor{mycolor2}{rgb}{0.69804,0.87451,0.54118}%
\definecolor{mycolor3}{rgb}{0.20000,0.62745,0.17255}%
\definecolor{mycolor4}{rgb}{0.12157,0.47059,0.70588}%
%
\begin{tikzpicture}[trim axis left, trim axis right]

\begin{axis}[%
width=\figurewidth,
height=\figureheight,
at={(0\figurewidth,0\figureheight)},
scale only axis,
every outer x axis line/.append style={black},
every x tick label/.append style={font=\color{black}},
xmin=10.975,
xmax=11,
xlabel={Time (h)},
every outer y axis line/.append style={black},
every y tick label/.append style={font=\color{black}},
ymin=33.93,
ymax=34.03,
ylabel={Price},
axis background/.style={fill=white},
axis x line*=bottom,
axis y line*=left,
legend style={legend cell align=left,align=left,draw=black,legend pos = south west},
every axis legend/.code={\renewcommand\addlegendentry[2][]{}}  %ignore legend locally
]
\addplot [color=black,solid,line width=3.0pt]
  table[row sep=crcr]{%
10.975	33.995\\
10.9752777777778	33.995\\
10.9755555555556	33.995\\
10.9758333333333	33.995\\
10.9761111111111	33.995\\
10.9763888888889	33.995\\
10.9766666666667	33.995\\
10.9769444444444	33.995\\
10.9772222222222	33.995\\
10.9775	33.995\\
10.9777777777778	33.995\\
10.9780555555556	33.99\\
10.9783333333333	33.99\\
10.9786111111111	33.99\\
10.9788888888889	33.99\\
10.9791666666667	33.99\\
10.9794444444444	33.995\\
10.9797222222222	33.995\\
10.98	33.995\\
10.9802777777778	33.99\\
10.9805555555556	33.99\\
10.9808333333333	33.99\\
10.9811111111111	33.995\\
10.9813888888889	33.985\\
10.9816666666667	33.985\\
10.9819444444444	33.985\\
10.9822222222222	33.985\\
10.9825	33.985\\
10.9827777777778	33.985\\
10.9830555555556	33.985\\
10.9833333333333	33.985\\
10.9836111111111	33.985\\
10.9838888888889	33.985\\
10.9841666666667	33.985\\
10.9844444444444	33.98\\
10.9847222222222	33.98\\
10.985	33.985\\
10.9852777777778	33.985\\
10.9855555555556	33.98\\
10.9858333333333	33.985\\
10.9861111111111	33.985\\
10.9863888888889	33.975\\
10.9866666666667	33.975\\
10.9869444444444	33.975\\
10.9872222222222	33.975\\
10.9875	33.975\\
10.9877777777778	33.975\\
10.9880555555556	33.975\\
10.9883333333333	33.975\\
10.9886111111111	33.975\\
10.9888888888889	33.975\\
10.9891666666667	33.97\\
10.9894444444444	33.97\\
10.9897222222222	33.97\\
10.99	33.97\\
10.9902777777778	33.97\\
10.9905555555556	33.97\\
10.9908333333333	33.97\\
10.9911111111111	33.975\\
10.9913888888889	33.97\\
10.9916666666667	33.97\\
10.9919444444444	33.97\\
10.9922222222222	33.965\\
10.9925	33.965\\
10.9927777777778	33.965\\
10.9930555555556	33.965\\
10.9933333333333	33.955\\
10.9936111111111	33.965\\
10.9938888888889	33.965\\
10.9941666666667	33.965\\
10.9944444444444	33.975\\
10.9947222222222	33.975\\
10.995	33.965\\
10.9952777777778	33.975\\
10.9955555555556	33.975\\
10.9958333333333	33.975\\
10.9961111111111	33.975\\
10.9963888888889	33.975\\
10.9966666666667	33.975\\
10.9969444444444	33.97\\
10.9972222222222	33.97\\
10.9975	33.975\\
10.9977777777778	33.985\\
10.9980555555556	33.995\\
10.9983333333333	33.995\\
10.9986111111111	33.995\\
10.9988888888889	33.995\\
10.9991666666667	33.995\\
10.9994444444444	33.995\\
10.9997222222222	33.995\\
11	33.995\\
};
\addlegendentry{$S$};

\addplot [color=gray,solid,line width=2.0pt,forget plot]
  table[row sep=crcr]{%
10.975	34.01\\
10.9752777777778	34.01\\
10.9755555555556	34.01\\
10.9758333333333	34.01\\
10.9761111111111	34.01\\
10.9763888888889	34.01\\
10.9766666666667	34.01\\
10.9769444444444	34.01\\
10.9772222222222	34.01\\
10.9775	34.0081732365245\\
10.9777777777778	34.01\\
10.9780555555556	34.0084246177287\\
10.9783333333333	34.0084246177287\\
10.9786111111111	34.0084246177287\\
10.9788888888889	34.0084246177287\\
10.9791666666667	34.000526743864\\
10.9794444444444	34.01\\
10.9797222222222	34.01\\
10.98	34.01\\
10.9802777777778	34.01\\
10.9805555555556	34.01\\
10.9808333333333	34.000526743864\\
10.9811111111111	34.01\\
10.9813888888889	34\\
10.9816666666667	34\\
10.9819444444444	34\\
10.9822222222222	34\\
10.9825	34\\
10.9827777777778	34\\
10.9830555555556	34\\
10.9833333333333	34\\
10.9836111111111	34\\
10.9838888888889	34\\
10.9841666666667	34\\
10.9844444444444	34\\
10.9847222222222	33.9916214350482\\
10.985	34\\
10.9852777777778	34\\
10.9855555555556	33.990526743864\\
10.9858333333333	34\\
10.9861111111111	34\\
10.9863888888889	33.99\\
10.9866666666667	33.99\\
10.9869444444444	33.99\\
10.9872222222222	33.99\\
10.9875	33.99\\
10.9877777777778	33.99\\
10.9880555555556	33.99\\
10.9883333333333	33.99\\
10.9886111111111	33.99\\
10.9888888888889	33.99\\
10.9891666666667	33.9867116094193\\
10.9894444444444	33.9867116094193\\
10.9897222222222	33.9867116094193\\
10.99	33.9886320646659\\
10.9902777777778	33.9886320646659\\
10.9905555555556	33.9886320646659\\
10.9908333333333	33.9805261798833\\
10.9911111111111	33.99\\
10.9913888888889	33.99\\
10.9916666666667	33.99\\
10.9919444444444	33.99\\
10.9922222222222	33.98\\
10.9925	33.98\\
10.9927777777778	33.98\\
10.9930555555556	33.98\\
10.9933333333333	33.9616214350482\\
10.9936111111111	33.98\\
10.9938888888889	33.98\\
10.9941666666667	33.9716214350482\\
10.9944444444444	33.99\\
10.9947222222222	33.99\\
10.995	33.970526743864\\
10.9952777777778	33.99\\
10.9955555555556	33.99\\
10.9958333333333	33.99\\
10.9961111111111	33.99\\
10.9963888888889	33.99\\
10.9966666666667	33.99\\
10.9969444444444	33.99\\
10.9972222222222	33.9805261798833\\
10.9975	33.983421792079\\
10.9977777777778	33.99\\
10.9980555555556	34.0012009559832\\
10.9983333333333	34.0049120415607\\
10.9986111111111	34.0064200267923\\
10.9988888888889	34.0033430597809\\
10.9991666666667	34.0012009559832\\
10.9994444444444	34.0012009559832\\
10.9997222222222	34.0012009559832\\
11	34.0012009559832\\
};
\addplot [color=gray,solid,line width=2.0pt]
  table[row sep=crcr]{%
10.975	33.99\\
10.9752777777778	33.99\\
10.9755555555556	33.99\\
10.9758333333333	33.99\\
10.9761111111111	33.99\\
10.9763888888889	33.99\\
10.9766666666667	33.99\\
10.9769444444444	33.99\\
10.9772222222222	33.99\\
10.9775	33.986063979387\\
10.9777777777778	33.99\\
10.9780555555556	33.9767116094193\\
10.9783333333333	33.9767116094193\\
10.9786111111111	33.9767116094193\\
10.9788888888889	33.9767116094193\\
10.9791666666667	33.97\\
10.9794444444444	33.99\\
10.9797222222222	33.99\\
10.98	33.99\\
10.9802777777778	33.98\\
10.9805555555556	33.98\\
10.9808333333333	33.97\\
10.9811111111111	33.99\\
10.9813888888889	33.98\\
10.9816666666667	33.98\\
10.9819444444444	33.98\\
10.9822222222222	33.98\\
10.9825	33.98\\
10.9827777777778	33.98\\
10.9830555555556	33.98\\
10.9833333333333	33.9784246177287\\
10.9836111111111	33.9784246177287\\
10.9838888888889	33.9784246177287\\
10.9841666666667	33.98\\
10.9844444444444	33.9686320646659\\
10.9847222222222	33.96\\
10.985	33.98\\
10.9852777777778	33.98\\
10.9855555555556	33.96\\
10.9858333333333	33.98\\
10.9861111111111	33.98\\
10.9863888888889	33.97\\
10.9866666666667	33.97\\
10.9869444444444	33.97\\
10.9872222222222	33.97\\
10.9875	33.97\\
10.9877777777778	33.97\\
10.9880555555556	33.97\\
10.9883333333333	33.97\\
10.9886111111111	33.9686320646659\\
10.9888888888889	33.97\\
10.9891666666667	33.95599716387\\
10.9894444444444	33.95599716387\\
10.9897222222222	33.95599716387\\
10.99	33.9573430261438\\
10.9902777777778	33.9573430261438\\
10.9905555555556	33.9573430261438\\
10.9908333333333	33.95\\
10.9911111111111	33.97\\
10.9913888888889	33.96\\
10.9916666666667	33.96\\
10.9919444444444	33.96\\
10.9922222222222	33.96\\
10.9925	33.96\\
10.9927777777778	33.96\\
10.9930555555556	33.96\\
10.9933333333333	33.94\\
10.9936111111111	33.96\\
10.9938888888889	33.96\\
10.9941666666667	33.95\\
10.9944444444444	33.97\\
10.9947222222222	33.97\\
10.995	33.95\\
10.9952777777778	33.97\\
10.9955555555556	33.97\\
10.9958333333333	33.97\\
10.9961111111111	33.97\\
10.9963888888889	33.9681732365245\\
10.9966666666667	33.97\\
10.9969444444444	33.9584246177287\\
10.9972222222222	33.95\\
10.9975	33.960526743864\\
10.9977777777778	33.97\\
10.9980555555556	33.98\\
10.9983333333333	33.9833430597809\\
10.9986111111111	33.9855601432366\\
10.9988888888889	33.98\\
10.9991666666667	33.98\\
10.9994444444444	33.98\\
10.9997222222222	33.98\\
11	33.98\\
};
\addlegendentry{$S \pm \delta^\pm$};

\addplot [color=mycolor1,solid,line width=2.0pt,forget plot]
  table[row sep=crcr]{%
10.9811111111111	33.995\\
10.9811111111111	34.01\\
};
\addplot [color=mycolor1,solid,line width=2.0pt,mark=*,mark options={solid,fill=mycolor1},forget plot]
  table[row sep=crcr]{%
10.9811111111111	34.01\\
};
\addplot [color=mycolor1,solid,line width=2.0pt,forget plot]
  table[row sep=crcr]{%
10.9911111111111	33.975\\
10.9911111111111	33.99\\
};
\addplot [color=mycolor1,solid,line width=2.0pt,mark=*,mark options={solid,fill=mycolor1},forget plot]
  table[row sep=crcr]{%
10.9911111111111	33.99\\
};
\addplot [color=mycolor1,solid,line width=2.0pt,forget plot]
  table[row sep=crcr]{%
10.9936111111111	33.965\\
10.9936111111111	33.98\\
};
\addplot [color=mycolor1,solid,line width=2.0pt,mark=*,mark options={solid,fill=mycolor1},forget plot]
  table[row sep=crcr]{%
10.9936111111111	33.98\\
};
\addplot [color=mycolor1,solid,line width=2.0pt,forget plot]
  table[row sep=crcr]{%
10.9944444444444	33.975\\
10.9944444444444	33.99\\
};
\addplot [color=mycolor1,solid,line width=2.0pt,mark=*,mark options={solid,fill=mycolor1},forget plot]
  table[row sep=crcr]{%
10.9944444444444	33.99\\
};
\addplot [color=mycolor1,solid,line width=2.0pt,forget plot]
  table[row sep=crcr]{%
10.9952777777778	33.975\\
10.9952777777778	33.99\\
};
\addplot [color=mycolor1,solid,line width=2.0pt,mark=*,mark options={solid,fill=mycolor1},forget plot]
  table[row sep=crcr]{%
10.9952777777778	33.99\\
};
\addplot [color=mycolor1,solid,line width=2.0pt,forget plot]
  table[row sep=crcr]{%
10.9975	33.975\\
10.9975	33.983421792079\\
};
\addplot [color=mycolor1,solid,line width=2.0pt,mark=*,mark options={solid,fill=mycolor1},forget plot]
  table[row sep=crcr]{%
10.9975	33.983421792079\\
};
\addplot [color=mycolor1,solid,line width=2.0pt,forget plot]
  table[row sep=crcr]{%
10.9977777777778	33.985\\
10.9977777777778	33.99\\
};
\addplot [color=mycolor1,solid,line width=2.0pt,mark=*,mark options={solid,fill=mycolor1},forget plot]
  table[row sep=crcr]{%
10.9977777777778	33.99\\
};
\addplot [color=mycolor1,solid,line width=2.0pt,forget plot]
  table[row sep=crcr]{%
10.9980555555556	33.995\\
10.9980555555556	34.0012009559832\\
};
\addplot [color=mycolor1,solid,line width=2.0pt,mark=*,mark options={solid,fill=mycolor1},forget plot]
  table[row sep=crcr]{%
10.9980555555556	34.0012009559832\\
};
\addplot [color=mycolor1,solid,line width=2.0pt,forget plot]
  table[row sep=crcr]{%
10.9983333333333	33.995\\
10.9983333333333	34.0049120415607\\
};
\addplot [color=mycolor1,solid,line width=2.0pt,mark=*,mark options={solid,fill=mycolor1}]
  table[row sep=crcr]{%
10.9983333333333	34.0049120415607\\
};
\addlegendentry{Ext Buy MO lifts our sell LO};

\addplot [color=mycolor1,dashed,line width=2.0pt,forget plot]
  table[row sep=crcr]{%
10.9791666666667	33.99\\
10.9791666666667	34\\
};
\addplot [color=mycolor1,dashed,line width=2.0pt,mark=o,mark options={solid},forget plot]
  table[row sep=crcr]{%
10.9791666666667	34\\
};
\addplot [color=mycolor1,dashed,line width=2.0pt,forget plot]
  table[row sep=crcr]{%
10.9913888888889	33.97\\
10.9913888888889	33.98\\
};
\addplot [color=mycolor1,dashed,line width=2.0pt,mark=o,mark options={solid}]
  table[row sep=crcr]{%
10.9913888888889	33.98\\
};
\addlegendentry{Ext Buy MO arrives};

\addplot [color=mycolor2,solid,line width=2.0pt,forget plot]
  table[row sep=crcr]{%
10.9769444444444	33.995\\
10.9769444444444	33.99\\
};
\addplot [color=mycolor2,solid,line width=2.0pt,mark=*,mark options={solid,fill=mycolor2},forget plot]
  table[row sep=crcr]{%
10.9769444444444	33.99\\
};
\addplot [color=mycolor2,solid,line width=2.0pt,forget plot]
  table[row sep=crcr]{%
10.9775	33.995\\
10.9775	33.986063979387\\
};
\addplot [color=mycolor2,solid,line width=2.0pt,mark=*,mark options={solid,fill=mycolor2},forget plot]
  table[row sep=crcr]{%
10.9775	33.986063979387\\
};
\addplot [color=mycolor2,solid,line width=2.0pt,forget plot]
  table[row sep=crcr]{%
10.9780555555556	33.99\\
10.9780555555556	33.9767116094193\\
};
\addplot [color=mycolor2,solid,line width=2.0pt,mark=*,mark options={solid,fill=mycolor2},forget plot]
  table[row sep=crcr]{%
10.9780555555556	33.9767116094193\\
};
\addplot [color=mycolor2,solid,line width=2.0pt,forget plot]
  table[row sep=crcr]{%
10.9802777777778	33.99\\
10.9802777777778	33.98\\
};
\addplot [color=mycolor2,solid,line width=2.0pt,mark=*,mark options={solid,fill=mycolor2},forget plot]
  table[row sep=crcr]{%
10.9802777777778	33.98\\
};
\addplot [color=mycolor2,solid,line width=2.0pt,forget plot]
  table[row sep=crcr]{%
10.9813888888889	33.985\\
10.9813888888889	33.98\\
};
\addplot [color=mycolor2,solid,line width=2.0pt,mark=*,mark options={solid,fill=mycolor2},forget plot]
  table[row sep=crcr]{%
10.9813888888889	33.98\\
};
\addplot [color=mycolor2,solid,line width=2.0pt,forget plot]
  table[row sep=crcr]{%
10.9833333333333	33.985\\
10.9833333333333	33.9784246177287\\
};
\addplot [color=mycolor2,solid,line width=2.0pt,mark=*,mark options={solid,fill=mycolor2},forget plot]
  table[row sep=crcr]{%
10.9833333333333	33.9784246177287\\
};
\addplot [color=mycolor2,solid,line width=2.0pt,forget plot]
  table[row sep=crcr]{%
10.9841666666667	33.985\\
10.9841666666667	33.98\\
};
\addplot [color=mycolor2,solid,line width=2.0pt,mark=*,mark options={solid,fill=mycolor2},forget plot]
  table[row sep=crcr]{%
10.9841666666667	33.98\\
};
\addplot [color=mycolor2,solid,line width=2.0pt,forget plot]
  table[row sep=crcr]{%
10.9855555555556	33.98\\
10.9855555555556	33.96\\
};
\addplot [color=mycolor2,solid,line width=2.0pt,mark=*,mark options={solid,fill=mycolor2},forget plot]
  table[row sep=crcr]{%
10.9855555555556	33.96\\
};
\addplot [color=mycolor2,solid,line width=2.0pt,forget plot]
  table[row sep=crcr]{%
10.9863888888889	33.975\\
10.9863888888889	33.97\\
};
\addplot [color=mycolor2,solid,line width=2.0pt,mark=*,mark options={solid,fill=mycolor2},forget plot]
  table[row sep=crcr]{%
10.9863888888889	33.97\\
};
\addplot [color=mycolor2,solid,line width=2.0pt,forget plot]
  table[row sep=crcr]{%
10.9880555555556	33.975\\
10.9880555555556	33.97\\
};
\addplot [color=mycolor2,solid,line width=2.0pt,mark=*,mark options={solid,fill=mycolor2},forget plot]
  table[row sep=crcr]{%
10.9880555555556	33.97\\
};
\addplot [color=mycolor2,solid,line width=2.0pt,forget plot]
  table[row sep=crcr]{%
10.9886111111111	33.975\\
10.9886111111111	33.9686320646659\\
};
\addplot [color=mycolor2,solid,line width=2.0pt,mark=*,mark options={solid,fill=mycolor2},forget plot]
  table[row sep=crcr]{%
10.9886111111111	33.9686320646659\\
};
\addplot [color=mycolor2,solid,line width=2.0pt,forget plot]
  table[row sep=crcr]{%
10.9891666666667	33.97\\
10.9891666666667	33.95599716387\\
};
\addplot [color=mycolor2,solid,line width=2.0pt,mark=*,mark options={solid,fill=mycolor2},forget plot]
  table[row sep=crcr]{%
10.9891666666667	33.95599716387\\
};
\addplot [color=mycolor2,solid,line width=2.0pt,forget plot]
  table[row sep=crcr]{%
10.9913888888889	33.97\\
10.9913888888889	33.96\\
};
\addplot [color=mycolor2,solid,line width=2.0pt,mark=*,mark options={solid,fill=mycolor2},forget plot]
  table[row sep=crcr]{%
10.9913888888889	33.96\\
};
\addplot [color=mycolor2,solid,line width=2.0pt,forget plot]
  table[row sep=crcr]{%
10.9933333333333	33.955\\
10.9933333333333	33.94\\
};
\addplot [color=mycolor2,solid,line width=2.0pt,mark=*,mark options={solid,fill=mycolor2},forget plot]
  table[row sep=crcr]{%
10.9933333333333	33.94\\
};
\addplot [color=mycolor2,solid,line width=2.0pt,forget plot]
  table[row sep=crcr]{%
10.995	33.965\\
10.995	33.95\\
};
\addplot [color=mycolor2,solid,line width=2.0pt,mark=*,mark options={solid,fill=mycolor2},forget plot]
  table[row sep=crcr]{%
10.995	33.95\\
};
\addplot [color=mycolor2,solid,line width=2.0pt,forget plot]
  table[row sep=crcr]{%
10.9963888888889	33.975\\
10.9963888888889	33.9681732365245\\
};
\addplot [color=mycolor2,solid,line width=2.0pt,mark=*,mark options={solid,fill=mycolor2},forget plot]
  table[row sep=crcr]{%
10.9963888888889	33.9681732365245\\
};
\addplot [color=mycolor2,solid,line width=2.0pt,forget plot]
  table[row sep=crcr]{%
10.9969444444444	33.97\\
10.9969444444444	33.9584246177287\\
};
\addplot [color=mycolor2,solid,line width=2.0pt,mark=*,mark options={solid,fill=mycolor2},forget plot]
  table[row sep=crcr]{%
10.9969444444444	33.9584246177287\\
};
\addplot [color=mycolor2,solid,line width=2.0pt,forget plot]
  table[row sep=crcr]{%
10.9988888888889	33.995\\
10.9988888888889	33.98\\
};
\addplot [color=mycolor2,solid,line width=2.0pt,mark=*,mark options={solid,fill=mycolor2}]
  table[row sep=crcr]{%
10.9988888888889	33.98\\
};
\addlegendentry{Ext Sell MO lifts our buy LO};

\addplot [color=mycolor2,dashed,line width=2.0pt,forget plot]
  table[row sep=crcr]{%
10.9980555555556	33.995\\
10.9980555555556	34\\
};
\addplot [color=mycolor2,dashed,line width=2.0pt,mark=o,mark options={solid}]
  table[row sep=crcr]{%
10.9980555555556	34\\
};
\addlegendentry{Ext Sell MO arrives};

\addplot [color=mycolor3,solid,line width=2.0pt,forget plot]
  table[row sep=crcr]{%
10.9811111111111	33.995\\
10.9811111111111	34\\
};
\addplot [color=mycolor3,solid,line width=2.0pt,mark=*,mark options={solid,fill=mycolor3},forget plot]
  table[row sep=crcr]{%
10.9811111111111	34\\
};
\addplot [color=mycolor3,solid,line width=2.0pt,forget plot]
  table[row sep=crcr]{%
10.985	33.985\\
10.985	33.99\\
};
\addplot [color=mycolor3,solid,line width=2.0pt,mark=*,mark options={solid,fill=mycolor3},forget plot]
  table[row sep=crcr]{%
10.985	33.99\\
};
\addplot [color=mycolor3,solid,line width=2.0pt,forget plot]
  table[row sep=crcr]{%
10.9911111111111	33.975\\
10.9911111111111	33.98\\
};
\addplot [color=mycolor3,solid,line width=2.0pt,mark=*,mark options={solid,fill=mycolor3},forget plot]
  table[row sep=crcr]{%
10.9911111111111	33.98\\
};
\addplot [color=mycolor3,solid,line width=2.0pt,forget plot]
  table[row sep=crcr]{%
10.9936111111111	33.965\\
10.9936111111111	33.97\\
};
\addplot [color=mycolor3,solid,line width=2.0pt,mark=*,mark options={solid,fill=mycolor3},forget plot]
  table[row sep=crcr]{%
10.9936111111111	33.97\\
};
\addplot [color=mycolor3,solid,line width=2.0pt,forget plot]
  table[row sep=crcr]{%
10.9944444444444	33.975\\
10.9944444444444	33.98\\
};
\addplot [color=mycolor3,solid,line width=2.0pt,mark=*,mark options={solid,fill=mycolor3},forget plot]
  table[row sep=crcr]{%
10.9944444444444	33.98\\
};
\addplot [color=mycolor3,solid,line width=2.0pt,forget plot]
  table[row sep=crcr]{%
10.9952777777778	33.975\\
10.9952777777778	33.98\\
};
\addplot [color=mycolor3,solid,line width=2.0pt,mark=*,mark options={solid,fill=mycolor3}]
  table[row sep=crcr]{%
10.9952777777778	33.98\\
};
\addlegendentry{Our Buy MO};

\addplot [color=mycolor4,solid,line width=2.0pt,forget plot]
  table[row sep=crcr]{%
10.9791666666667	33.99\\
10.9791666666667	33.98\\
};
\addplot [color=mycolor4,solid,line width=2.0pt,mark=*,mark options={solid,fill=mycolor4},forget plot]
  table[row sep=crcr]{%
10.9791666666667	33.98\\
};
\addplot [color=mycolor4,solid,line width=2.0pt,forget plot]
  table[row sep=crcr]{%
10.9808333333333	33.99\\
10.9808333333333	33.98\\
};
\addplot [color=mycolor4,solid,line width=2.0pt,mark=*,mark options={solid,fill=mycolor4},forget plot]
  table[row sep=crcr]{%
10.9808333333333	33.98\\
};
\addplot [color=mycolor4,solid,line width=2.0pt,forget plot]
  table[row sep=crcr]{%
10.9847222222222	33.98\\
10.9847222222222	33.97\\
};
\addplot [color=mycolor4,solid,line width=2.0pt,mark=*,mark options={solid,fill=mycolor4},forget plot]
  table[row sep=crcr]{%
10.9847222222222	33.97\\
};
\addplot [color=mycolor4,solid,line width=2.0pt,forget plot]
  table[row sep=crcr]{%
10.9855555555556	33.98\\
10.9855555555556	33.97\\
};
\addplot [color=mycolor4,solid,line width=2.0pt,mark=*,mark options={solid,fill=mycolor4},forget plot]
  table[row sep=crcr]{%
10.9855555555556	33.97\\
};
\addplot [color=mycolor4,solid,line width=2.0pt,forget plot]
  table[row sep=crcr]{%
10.9908333333333	33.97\\
10.9908333333333	33.96\\
};
\addplot [color=mycolor4,solid,line width=2.0pt,mark=*,mark options={solid,fill=mycolor4},forget plot]
  table[row sep=crcr]{%
10.9908333333333	33.96\\
};
\addplot [color=mycolor4,solid,line width=2.0pt,forget plot]
  table[row sep=crcr]{%
10.9933333333333	33.955\\
10.9933333333333	33.95\\
};
\addplot [color=mycolor4,solid,line width=2.0pt,mark=*,mark options={solid,fill=mycolor4},forget plot]
  table[row sep=crcr]{%
10.9933333333333	33.95\\
};
\addplot [color=mycolor4,solid,line width=2.0pt,forget plot]
  table[row sep=crcr]{%
10.9941666666667	33.965\\
10.9941666666667	33.96\\
};
\addplot [color=mycolor4,solid,line width=2.0pt,mark=*,mark options={solid,fill=mycolor4},forget plot]
  table[row sep=crcr]{%
10.9941666666667	33.96\\
};
\addplot [color=mycolor4,solid,line width=2.0pt,forget plot]
  table[row sep=crcr]{%
10.995	33.965\\
10.995	33.96\\
};
\addplot [color=mycolor4,solid,line width=2.0pt,mark=*,mark options={solid,fill=mycolor4},forget plot]
  table[row sep=crcr]{%
10.995	33.96\\
};
\addplot [color=mycolor4,solid,line width=2.0pt,forget plot]
  table[row sep=crcr]{%
10.9972222222222	33.97\\
10.9972222222222	33.96\\
};
\addplot [color=mycolor4,solid,line width=2.0pt,mark=*,mark options={solid,fill=mycolor4},forget plot]
  table[row sep=crcr]{%
10.9972222222222	33.96\\
};
\addplot [color=mycolor4,solid,line width=2.0pt,forget plot]
  table[row sep=crcr]{%
10.9977777777778	33.985\\
10.9977777777778	33.98\\
};
\addplot [color=mycolor4,solid,line width=2.0pt,mark=*,mark options={solid,fill=mycolor4}]
  table[row sep=crcr]{%
10.9977777777778	33.98\\
};
\addlegendentry{Our Sell MO};

\end{axis}
\end{tikzpicture}%

\end{subfigure}%
\hfill%
\begin{subfigure}{.45\linewidth}
  \centering
  \setlength\figureheight{\linewidth} 
  \setlength\figurewidth{\linewidth}
  \tikzsetnextfilename{samplepath_dscr_nFPC_paths}
  % This file was created by matlab2tikz.
%
%The latest updates can be retrieved from
%  http://www.mathworks.com/matlabcentral/fileexchange/22022-matlab2tikz-matlab2tikz
%where you can also make suggestions and rate matlab2tikz.
%
\definecolor{mycolor1}{rgb}{0.65098,0.80784,0.89020}%
\definecolor{mycolor2}{rgb}{0.69804,0.87451,0.54118}%
\definecolor{mycolor3}{rgb}{0.20000,0.62745,0.17255}%
\definecolor{mycolor4}{rgb}{0.12157,0.47059,0.70588}%
%
\begin{tikzpicture}[trim axis left, trim axis right]

\begin{axis}[%
width=\figurewidth,
height=\figureheight,
at={(0\figurewidth,0\figureheight)},
scale only axis,
every outer x axis line/.append style={black},
every x tick label/.append style={font=\color{black}},
xmin=10.975,
xmax=11,
xlabel={Time (h)},
every outer y axis line/.append style={black},
every y tick label/.append style={font=\color{black}},
ymin=33.93,
ymax=34.03,
ylabel={Price},
axis background/.style={fill=white},
axis x line*=bottom,
axis y line*=left,
legend style={legend cell align=left,align=left,draw=black,legend pos = south west},
every axis legend/.code={\renewcommand\addlegendentry[2][]{}}  %ignore legend locally
]
\addplot [color=black,solid,line width=3.0pt]
  table[row sep=crcr]{%
10.975	33.995\\
10.9752777777778	33.995\\
10.9755555555556	33.995\\
10.9758333333333	33.995\\
10.9761111111111	33.995\\
10.9763888888889	33.995\\
10.9766666666667	33.995\\
10.9769444444444	33.995\\
10.9772222222222	33.995\\
10.9775	33.995\\
10.9777777777778	33.995\\
10.9780555555556	33.99\\
10.9783333333333	33.99\\
10.9786111111111	33.99\\
10.9788888888889	33.99\\
10.9791666666667	33.99\\
10.9794444444444	33.995\\
10.9797222222222	33.995\\
10.98	33.995\\
10.9802777777778	33.99\\
10.9805555555556	33.99\\
10.9808333333333	33.99\\
10.9811111111111	33.995\\
10.9813888888889	33.985\\
10.9816666666667	33.985\\
10.9819444444444	33.985\\
10.9822222222222	33.985\\
10.9825	33.985\\
10.9827777777778	33.985\\
10.9830555555556	33.985\\
10.9833333333333	33.985\\
10.9836111111111	33.985\\
10.9838888888889	33.985\\
10.9841666666667	33.985\\
10.9844444444444	33.98\\
10.9847222222222	33.98\\
10.985	33.985\\
10.9852777777778	33.985\\
10.9855555555556	33.98\\
10.9858333333333	33.985\\
10.9861111111111	33.985\\
10.9863888888889	33.975\\
10.9866666666667	33.975\\
10.9869444444444	33.975\\
10.9872222222222	33.975\\
10.9875	33.975\\
10.9877777777778	33.975\\
10.9880555555556	33.975\\
10.9883333333333	33.975\\
10.9886111111111	33.975\\
10.9888888888889	33.975\\
10.9891666666667	33.97\\
10.9894444444444	33.97\\
10.9897222222222	33.97\\
10.99	33.97\\
10.9902777777778	33.97\\
10.9905555555556	33.97\\
10.9908333333333	33.97\\
10.9911111111111	33.975\\
10.9913888888889	33.97\\
10.9916666666667	33.97\\
10.9919444444444	33.97\\
10.9922222222222	33.965\\
10.9925	33.965\\
10.9927777777778	33.965\\
10.9930555555556	33.965\\
10.9933333333333	33.955\\
10.9936111111111	33.965\\
10.9938888888889	33.965\\
10.9941666666667	33.965\\
10.9944444444444	33.975\\
10.9947222222222	33.975\\
10.995	33.965\\
10.9952777777778	33.975\\
10.9955555555556	33.975\\
10.9958333333333	33.975\\
10.9961111111111	33.975\\
10.9963888888889	33.975\\
10.9966666666667	33.975\\
10.9969444444444	33.97\\
10.9972222222222	33.97\\
10.9975	33.975\\
10.9977777777778	33.985\\
10.9980555555556	33.995\\
10.9983333333333	33.995\\
10.9986111111111	33.995\\
10.9988888888889	33.995\\
10.9991666666667	33.995\\
10.9994444444444	33.995\\
10.9997222222222	33.995\\
11	33.995\\
};
\addlegendentry{$S$};

\addplot [color=gray,solid,line width=2.0pt,forget plot]
  table[row sep=crcr]{%
10.975	34.0095591449678\\
10.9752777777778	34.0095591449678\\
10.9755555555556	34.0095591449678\\
10.9758333333333	34.0095591449678\\
10.9761111111111	34.0095591449678\\
10.9763888888889	34.0095591449678\\
10.9766666666667	34.0095591449678\\
10.9769444444444	34.0077907684168\\
10.9772222222222	34.0077907684168\\
10.9775	34.0044883451971\\
10.9777777777778	34\\
10.9780555555556	34.0070144313823\\
10.9783333333333	34.0070144313823\\
10.9786111111111	34.0070144313823\\
10.9788888888889	34.0070144313823\\
10.9791666666667	34.0118829629196\\
10.9794444444444	34.0070144313823\\
10.9797222222222	34.0070144313823\\
10.98	34\\
10.9802777777778	34.0095591449678\\
10.9805555555556	34.0095591449678\\
10.9808333333333	34.012650854911\\
10.9811111111111	34.0021523461235\\
10.9813888888889	33.9997593349026\\
10.9816666666667	34.000221604306\\
10.9819444444444	34.000221604306\\
10.9822222222222	33.9999930529003\\
10.9825	33.9999930529003\\
10.9827777777778	33.9999930529003\\
10.9830555555556	33.9999930529003\\
10.9833333333333	33.9995591449678\\
10.9836111111111	33.9995591449678\\
10.9838888888889	33.9995591449678\\
10.9841666666667	33.9924887877916\\
10.9844444444444	33.9997969182872\\
10.9847222222222	34.0034558419258\\
10.985	33.9997969182872\\
10.9852777777778	33.9924887877916\\
10.9855555555556	34.0018829629196\\
10.9858333333333	33.9997258918278\\
10.9861111111111	33.9920935862006\\
10.9863888888889	33.9870144313823\\
10.9866666666667	33.9901671828515\\
10.9869444444444	33.9901671828515\\
10.9872222222222	33.9901671828515\\
10.9875	33.9901671828515\\
10.9877777777778	33.9901671828515\\
10.9880555555556	33.9901581955578\\
10.9883333333333	33.9896650872252\\
10.9886111111111	33.9888653999139\\
10.9888888888889	33.9814256768415\\
10.9891666666667	33.9858026788123\\
10.9894444444444	33.9858026788123\\
10.9897222222222	33.9858026788123\\
10.99	33.987668687197\\
10.9902777777778	33.987668687197\\
10.9905555555556	33.987668687197\\
10.9908333333333	33.9915804064569\\
10.9911111111111	33.98\\
10.9913888888889	33.9877907684168\\
10.9916666666667	33.9897969182872\\
10.9919444444444	33.9841301013182\\
10.9922222222222	33.9797969182872\\
10.9925	33.98018095728\\
10.9927777777778	33.98018095728\\
10.9930555555556	33.9750986475475\\
10.9933333333333	33.9733875791588\\
10.9936111111111	33.98018095728\\
10.9938888888889	33.98018095728\\
10.9941666666667	33.9834558419258\\
10.9944444444444	33.9902110954469\\
10.9947222222222	33.9833086626523\\
10.995	33.9821207619446\\
10.9952777777778	33.99018095728\\
10.9955555555556	33.9877907684168\\
10.9958333333333	33.9877907684168\\
10.9961111111111	33.9849396889425\\
10.9963888888889	33.9844883451971\\
10.9966666666667	33.98\\
10.9969444444444	33.9867367765128\\
10.9972222222222	33.9923883907621\\
10.9975	33.9917432433565\\
10.9977777777778	34.0025065370642\\
10.9980555555556	34.0067367765128\\
10.9983333333333	34.0096650872252\\
10.9986111111111	34.0101581955578\\
10.9988888888889	34.0088653999139\\
10.9991666666667	34.0062335570531\\
10.9994444444444	34.0062335570531\\
10.9997222222222	34.0062335570531\\
11	34.0062335570531\\
};
\addplot [color=gray,solid,line width=2.0pt]
  table[row sep=crcr]{%
10.975	33.9878764041058\\
10.9752777777778	33.9878764041058\\
10.9755555555556	33.9878764041058\\
10.9758333333333	33.9878764041058\\
10.9761111111111	33.9878764041058\\
10.9763888888889	33.9878764041058\\
10.9766666666667	33.9878764041058\\
10.9769444444444	33.9870520397391\\
10.9772222222222	33.9870520397391\\
10.9775	33.9842616086803\\
10.9777777777778	33.9798455400393\\
10.9780555555556	33.9767675356892\\
10.9783333333333	33.9767675356892\\
10.9786111111111	33.9767675356892\\
10.9788888888889	33.9767675356892\\
10.9791666666667	33.98\\
10.9794444444444	33.9867675356893\\
10.9797222222222	33.9867675356893\\
10.98	33.9795147998989\\
10.9802777777778	33.9778764041057\\
10.9805555555556	33.9778764041057\\
10.9808333333333	33.98\\
10.9811111111111	33.9815306001365\\
10.9813888888889	33.9796529109682\\
10.9816666666667	33.98\\
10.9819444444444	33.98\\
10.9822222222222	33.9799665589495\\
10.9825	33.9799665589495\\
10.9827777777778	33.9799665589495\\
10.9830555555556	33.9799665589495\\
10.9833333333333	33.9778764041057\\
10.9836111111111	33.9778764041057\\
10.9838888888889	33.9778764041057\\
10.9841666666667	33.9721754260084\\
10.9844444444444	33.9697303226573\\
10.9847222222222	33.97\\
10.985	33.9797303226573\\
10.9852777777778	33.9721754260084\\
10.9855555555556	33.97\\
10.9858333333333	33.9797017484381\\
10.9861111111111	33.9719521233684\\
10.9863888888889	33.9667675356893\\
10.9866666666667	33.97\\
10.9869444444444	33.97\\
10.9872222222222	33.97\\
10.9875	33.97\\
10.9877777777778	33.97\\
10.9880555555556	33.97\\
10.9883333333333	33.9689490076283\\
10.9886111111111	33.9677351358409\\
10.9888888888889	33.9610819402726\\
10.9891666666667	33.9556856460776\\
10.9894444444444	33.9556856460776\\
10.9897222222222	33.9556856460776\\
10.99	33.9574744593794\\
10.9902777777778	33.9574744593794\\
10.9905555555556	33.9574744593794\\
10.9908333333333	33.96\\
10.9911111111111	33.9598455400393\\
10.9913888888889	33.9570520397391\\
10.9916666666667	33.9597303226574\\
10.9919444444444	33.9539290070915\\
10.9922222222222	33.9597303226574\\
10.9925	33.96\\
10.9927777777778	33.96\\
10.9930555555556	33.9550224734259\\
10.9933333333333	33.95\\
10.9936111111111	33.96\\
10.9938888888889	33.96\\
10.9941666666667	33.96\\
10.9944444444444	33.97\\
10.9947222222222	33.9626552154609\\
10.995	33.96\\
10.9952777777778	33.97\\
10.9955555555556	33.9670520397391\\
10.9958333333333	33.9670520397391\\
10.9961111111111	33.9645163776871\\
10.9963888888889	33.9642616086803\\
10.9966666666667	33.9598455400393\\
10.9969444444444	33.9562695844837\\
10.9972222222222	33.96\\
10.9975	33.97\\
10.9977777777778	33.98\\
10.9980555555556	33.9862695844837\\
10.9983333333333	33.9889490076283\\
10.9986111111111	33.99\\
10.9988888888889	33.9877351358409\\
10.9991666666667	33.9858243720109\\
10.9994444444444	33.9858243720109\\
10.9997222222222	33.9858243720109\\
11	33.9858243720109\\
};
\addlegendentry{$S \pm \delta^\pm$};

\addplot [color=mycolor1,solid,line width=2.0pt,forget plot]
  table[row sep=crcr]{%
10.9811111111111	33.995\\
10.9811111111111	34.0021523461235\\
};
\addplot [color=mycolor1,solid,line width=2.0pt,mark=*,mark options={solid,fill=mycolor1},forget plot]
  table[row sep=crcr]{%
10.9811111111111	34.0021523461235\\
};
\addplot [color=mycolor1,solid,line width=2.0pt,forget plot]
  table[row sep=crcr]{%
10.9911111111111	33.975\\
10.9911111111111	33.98\\
};
\addplot [color=mycolor1,solid,line width=2.0pt,mark=*,mark options={solid,fill=mycolor1},forget plot]
  table[row sep=crcr]{%
10.9911111111111	33.98\\
};
\addplot [color=mycolor1,solid,line width=2.0pt,forget plot]
  table[row sep=crcr]{%
10.9913888888889	33.97\\
10.9913888888889	33.9877907684168\\
};
\addplot [color=mycolor1,solid,line width=2.0pt,mark=*,mark options={solid,fill=mycolor1},forget plot]
  table[row sep=crcr]{%
10.9913888888889	33.9877907684168\\
};
\addplot [color=mycolor1,solid,line width=2.0pt,forget plot]
  table[row sep=crcr]{%
10.9936111111111	33.965\\
10.9936111111111	33.98018095728\\
};
\addplot [color=mycolor1,solid,line width=2.0pt,mark=*,mark options={solid,fill=mycolor1},forget plot]
  table[row sep=crcr]{%
10.9936111111111	33.98018095728\\
};
\addplot [color=mycolor1,solid,line width=2.0pt,forget plot]
  table[row sep=crcr]{%
10.9944444444444	33.975\\
10.9944444444444	33.9902110954469\\
};
\addplot [color=mycolor1,solid,line width=2.0pt,mark=*,mark options={solid,fill=mycolor1},forget plot]
  table[row sep=crcr]{%
10.9944444444444	33.9902110954469\\
};
\addplot [color=mycolor1,solid,line width=2.0pt,forget plot]
  table[row sep=crcr]{%
10.9977777777778	33.985\\
10.9977777777778	34.0025065370642\\
};
\addplot [color=mycolor1,solid,line width=2.0pt,mark=*,mark options={solid,fill=mycolor1}]
  table[row sep=crcr]{%
10.9977777777778	34.0025065370642\\
};
\addlegendentry{Ext Buy MO lifts our sell LO};

\addplot [color=mycolor1,dashed,line width=2.0pt,forget plot]
  table[row sep=crcr]{%
10.9791666666667	33.99\\
10.9791666666667	34\\
};
\addplot [color=mycolor1,dashed,line width=2.0pt,mark=o,mark options={solid},forget plot]
  table[row sep=crcr]{%
10.9791666666667	34\\
};
\addplot [color=mycolor1,dashed,line width=2.0pt,forget plot]
  table[row sep=crcr]{%
10.9952777777778	33.975\\
10.9952777777778	33.97\\
};
\addplot [color=mycolor1,dashed,line width=2.0pt,mark=o,mark options={solid},forget plot]
  table[row sep=crcr]{%
10.9952777777778	33.97\\
};
\addplot [color=mycolor1,dashed,line width=2.0pt,forget plot]
  table[row sep=crcr]{%
10.9975	33.975\\
10.9975	33.98\\
};
\addplot [color=mycolor1,dashed,line width=2.0pt,mark=o,mark options={solid},forget plot]
  table[row sep=crcr]{%
10.9975	33.98\\
};
\addplot [color=mycolor1,dashed,line width=2.0pt,forget plot]
  table[row sep=crcr]{%
10.9980555555556	33.995\\
10.9980555555556	34\\
};
\addplot [color=mycolor1,dashed,line width=2.0pt,mark=o,mark options={solid},forget plot]
  table[row sep=crcr]{%
10.9980555555556	34\\
};
\addplot [color=mycolor1,dashed,line width=2.0pt,forget plot]
  table[row sep=crcr]{%
10.9983333333333	33.995\\
10.9983333333333	34\\
};
\addplot [color=mycolor1,dashed,line width=2.0pt,mark=o,mark options={solid}]
  table[row sep=crcr]{%
10.9983333333333	34\\
};
\addlegendentry{Ext Buy MO arrives};

\addplot [color=mycolor2,solid,line width=2.0pt,forget plot]
  table[row sep=crcr]{%
10.9769444444444	33.995\\
10.9769444444444	33.9870520397391\\
};
\addplot [color=mycolor2,solid,line width=2.0pt,mark=*,mark options={solid,fill=mycolor2},forget plot]
  table[row sep=crcr]{%
10.9769444444444	33.9870520397391\\
};
\addplot [color=mycolor2,solid,line width=2.0pt,forget plot]
  table[row sep=crcr]{%
10.9775	33.995\\
10.9775	33.9842616086803\\
};
\addplot [color=mycolor2,solid,line width=2.0pt,mark=*,mark options={solid,fill=mycolor2},forget plot]
  table[row sep=crcr]{%
10.9775	33.9842616086803\\
};
\addplot [color=mycolor2,solid,line width=2.0pt,forget plot]
  table[row sep=crcr]{%
10.9802777777778	33.99\\
10.9802777777778	33.9778764041057\\
};
\addplot [color=mycolor2,solid,line width=2.0pt,mark=*,mark options={solid,fill=mycolor2},forget plot]
  table[row sep=crcr]{%
10.9802777777778	33.9778764041057\\
};
\addplot [color=mycolor2,solid,line width=2.0pt,forget plot]
  table[row sep=crcr]{%
10.9833333333333	33.985\\
10.9833333333333	33.9778764041057\\
};
\addplot [color=mycolor2,solid,line width=2.0pt,mark=*,mark options={solid,fill=mycolor2},forget plot]
  table[row sep=crcr]{%
10.9833333333333	33.9778764041057\\
};
\addplot [color=mycolor2,solid,line width=2.0pt,forget plot]
  table[row sep=crcr]{%
10.9841666666667	33.985\\
10.9841666666667	33.9721754260084\\
};
\addplot [color=mycolor2,solid,line width=2.0pt,mark=*,mark options={solid,fill=mycolor2},forget plot]
  table[row sep=crcr]{%
10.9841666666667	33.9721754260084\\
};
\addplot [color=mycolor2,solid,line width=2.0pt,forget plot]
  table[row sep=crcr]{%
10.9855555555556	33.98\\
10.9855555555556	33.97\\
};
\addplot [color=mycolor2,solid,line width=2.0pt,mark=*,mark options={solid,fill=mycolor2},forget plot]
  table[row sep=crcr]{%
10.9855555555556	33.97\\
};
\addplot [color=mycolor2,solid,line width=2.0pt,forget plot]
  table[row sep=crcr]{%
10.9880555555556	33.975\\
10.9880555555556	33.97\\
};
\addplot [color=mycolor2,solid,line width=2.0pt,mark=*,mark options={solid,fill=mycolor2},forget plot]
  table[row sep=crcr]{%
10.9880555555556	33.97\\
};
\addplot [color=mycolor2,solid,line width=2.0pt,forget plot]
  table[row sep=crcr]{%
10.9886111111111	33.975\\
10.9886111111111	33.9677351358409\\
};
\addplot [color=mycolor2,solid,line width=2.0pt,mark=*,mark options={solid,fill=mycolor2},forget plot]
  table[row sep=crcr]{%
10.9886111111111	33.9677351358409\\
};
\addplot [color=mycolor2,solid,line width=2.0pt,forget plot]
  table[row sep=crcr]{%
10.9891666666667	33.97\\
10.9891666666667	33.9556856460776\\
};
\addplot [color=mycolor2,solid,line width=2.0pt,mark=*,mark options={solid,fill=mycolor2},forget plot]
  table[row sep=crcr]{%
10.9891666666667	33.9556856460776\\
};
\addplot [color=mycolor2,solid,line width=2.0pt,forget plot]
  table[row sep=crcr]{%
10.9933333333333	33.955\\
10.9933333333333	33.95\\
};
\addplot [color=mycolor2,solid,line width=2.0pt,mark=*,mark options={solid,fill=mycolor2},forget plot]
  table[row sep=crcr]{%
10.9933333333333	33.95\\
};
\addplot [color=mycolor2,solid,line width=2.0pt,forget plot]
  table[row sep=crcr]{%
10.995	33.965\\
10.995	33.96\\
};
\addplot [color=mycolor2,solid,line width=2.0pt,mark=*,mark options={solid,fill=mycolor2},forget plot]
  table[row sep=crcr]{%
10.995	33.96\\
};
\addplot [color=mycolor2,solid,line width=2.0pt,forget plot]
  table[row sep=crcr]{%
10.9963888888889	33.975\\
10.9963888888889	33.9642616086803\\
};
\addplot [color=mycolor2,solid,line width=2.0pt,mark=*,mark options={solid,fill=mycolor2},forget plot]
  table[row sep=crcr]{%
10.9963888888889	33.9642616086803\\
};
\addplot [color=mycolor2,solid,line width=2.0pt,forget plot]
  table[row sep=crcr]{%
10.9969444444444	33.97\\
10.9969444444444	33.9562695844837\\
};
\addplot [color=mycolor2,solid,line width=2.0pt,mark=*,mark options={solid,fill=mycolor2},forget plot]
  table[row sep=crcr]{%
10.9969444444444	33.9562695844837\\
};
\addplot [color=mycolor2,solid,line width=2.0pt,forget plot]
  table[row sep=crcr]{%
10.9980555555556	33.995\\
10.9980555555556	33.9862695844837\\
};
\addplot [color=mycolor2,solid,line width=2.0pt,mark=*,mark options={solid,fill=mycolor2},forget plot]
  table[row sep=crcr]{%
10.9980555555556	33.9862695844837\\
};
\addplot [color=mycolor2,solid,line width=2.0pt,forget plot]
  table[row sep=crcr]{%
10.9988888888889	33.995\\
10.9988888888889	33.9877351358409\\
};
\addplot [color=mycolor2,solid,line width=2.0pt,mark=*,mark options={solid,fill=mycolor2}]
  table[row sep=crcr]{%
10.9988888888889	33.9877351358409\\
};
\addlegendentry{Ext Sell MO lifts our buy LO};

\addplot [color=mycolor2,dashed,line width=2.0pt,forget plot]
  table[row sep=crcr]{%
10.9780555555556	33.99\\
10.9780555555556	33.99\\
};
\addplot [color=mycolor2,dashed,line width=2.0pt,mark=o,mark options={solid},forget plot]
  table[row sep=crcr]{%
10.9780555555556	33.99\\
};
\addplot [color=mycolor2,dashed,line width=2.0pt,forget plot]
  table[row sep=crcr]{%
10.9813888888889	33.985\\
10.9813888888889	33.99\\
};
\addplot [color=mycolor2,dashed,line width=2.0pt,mark=o,mark options={solid},forget plot]
  table[row sep=crcr]{%
10.9813888888889	33.99\\
};
\addplot [color=mycolor2,dashed,line width=2.0pt,forget plot]
  table[row sep=crcr]{%
10.9863888888889	33.975\\
10.9863888888889	33.98\\
};
\addplot [color=mycolor2,dashed,line width=2.0pt,mark=o,mark options={solid},forget plot]
  table[row sep=crcr]{%
10.9863888888889	33.98\\
};
\addplot [color=mycolor2,dashed,line width=2.0pt,forget plot]
  table[row sep=crcr]{%
10.9913888888889	33.97\\
10.9913888888889	33.97\\
};
\addplot [color=mycolor2,dashed,line width=2.0pt,mark=o,mark options={solid}]
  table[row sep=crcr]{%
10.9913888888889	33.97\\
};
\addlegendentry{Ext Sell MO arrives};

\addplot [color=mycolor4,solid,line width=2.0pt,forget plot]
  table[row sep=crcr]{%
10.98	33.995\\
10.98	33.99\\
};
\addplot [color=mycolor4,solid,line width=2.0pt,mark=*,mark options={solid,fill=mycolor4},forget plot]
  table[row sep=crcr]{%
10.98	33.99\\
};
\addplot [color=mycolor4,solid,line width=2.0pt,forget plot]
  table[row sep=crcr]{%
10.9911111111111	33.975\\
10.9911111111111	33.97\\
};
\addplot [color=mycolor4,solid,line width=2.0pt,mark=*,mark options={solid,fill=mycolor4},forget plot]
  table[row sep=crcr]{%
10.9911111111111	33.97\\
};

\end{axis}
\end{tikzpicture}%
 
\end{subfigure}\\
\leavevmode\smash{\makebox[0pt]{\hspace{-7em}% HORIZONTAL POSITION           
  \rotatebox[origin=l]{90}{\hspace{20em}% VERTICAL POSITION
    Price}%
}}\hspace{0pt plus 1filll}\null

Time (h)

\vspace{1cm}
\begin{subfigure}{\linewidth}
  \centering
  \setlength\figureheight{\linewidth} 
  \setlength\figurewidth{\linewidth}
  \tikzsetnextfilename{samplepathslegend}
  \definecolor{mycolor1}{rgb}{0.65098,0.80784,0.89020}%
\definecolor{mycolor2}{rgb}{0.69804,0.87451,0.54118}%
\definecolor{mycolor3}{rgb}{0.20000,0.62745,0.17255}%
\definecolor{mycolor4}{rgb}{0.12157,0.47059,0.70588}%
\begin{tikzpicture}
    \begingroup
    % inits/clears the lists (which might be populated from previous
    % axes):
    \csname pgfplots@init@cleared@structures\endcsname
    \pgfplotsset{legend cell align=left,legend columns = 2,legend style={at={(0,1)},anchor=north west},legend style={draw=black,column sep=1ex},
    legend entries={Midprice,
    				Midprice $\pm \delta^\pm$,
    				Our Sell MO,
    				Our Buy MO,
    				Ext Buy MO lifts our sell LO,
    				Ext Sell MO lifts our buy LO,
    				Ext Buy MO arrives,
    				Ext Sell MO arrives}}%
    \csname pgfplots@addlegendimage\endcsname{line width=2pt,black,solid,sharp plot}
    \csname pgfplots@addlegendimage\endcsname{line width=2pt,gray,solid,sharp plot}
    \csname pgfplots@addlegendimage\endcsname{line width=1.5pt,mycolor4,solid,mark=*,mark options={solid,fill=mycolor4},sharp plot}%sell
    \csname pgfplots@addlegendimage\endcsname{line width=1.5pt,mycolor3,solid,mark=*,mark options={solid,fill=mycolor3},sharp plot}%buy
    \csname pgfplots@addlegendimage\endcsname{line width=1.5pt,mycolor1,solid,mark=*,mark options={solid,fill=mycolor1},sharp plot}% ext buy lifts
    \csname pgfplots@addlegendimage\endcsname{line width=1.5pt,mycolor2,solid,mark=*,mark options={solid,fill=mycolor2},sharp plot}%ext sell lifts    
    \csname pgfplots@addlegendimage\endcsname{line width=1pt,mycolor1,dashed,mark=o,mark options={solid},sharp plot}%ext buy
    \csname pgfplots@addlegendimage\endcsname{line width=1pt,mycolor2,dashed,mark=o,mark options={solid},sharp plot}%ext sell


    % draws the legend:
    \csname pgfplots@createlegend\endcsname
    \endgroup
\end{tikzpicture}
\end{subfigure}%
  \caption{Sample paths of the optimal trading strategies, showing price, limit order posting depths, executed market orders, and filled limit orders.}
  \label{fig:samplepath_paths}
\end{figure}

\begin{figure}
\centering
\begin{subfigure}{.45\linewidth}
  \centering
  \setlength\figureheight{\linewidth} 
  \setlength\figurewidth{\linewidth}
  \tikzsetnextfilename{samplepath_cts_depths}
  % This file was created by matlab2tikz.
%
%The latest updates can be retrieved from
%  http://www.mathworks.com/matlabcentral/fileexchange/22022-matlab2tikz-matlab2tikz
%where you can also make suggestions and rate matlab2tikz.
%
\begin{tikzpicture}[trim axis left, trim axis right]

\begin{axis}[%
width=\figurewidth,
height=\figureheight,
at={(0\figurewidth,0\figureheight)},
scale only axis,
every outer x axis line/.append style={black},
every x tick label/.append style={font=\color{black}},
xmin=10.975,
xmax=11,
every outer y axis line/.append style={black},
every y tick label/.append style={font=\color{black}},
ymin=-0.01,
ymax=0.015,
ytick={-0.01,-0.005,0,0.005,0.01,0.015},
yticklabels={{ 0.01},{0.005},{    0},{0.005},{ 0.01},{0.015}},
xlabel={Time [h]},
ylabel={Depth [\$]},
axis background/.style={fill=white},
axis x line*=bottom,
axis y line*=left,
yticklabel style={
        /pgf/number format/fixed,
        /pgf/number format/precision=3
},
scaled y ticks=false,
legend style={legend cell align=left,align=left,draw=black}
]
\addplot [color=white!60!black,solid,line width=2.0pt]
  table[row sep=crcr]{%
10.975	0.01\\
10.9752777777778	0.01\\
10.9755555555556	0.01\\
10.9758333333333	0.01\\
10.9761111111111	0.01\\
10.9763888888889	0.01\\
10.9766666666667	0.01\\
10.9769444444444	0.01\\
10.9772222222222	0.01\\
10.9775	0.00525611667063562\\
10.9777777777778	0.00525611667063562\\
10.9780555555556	0.01\\
10.9783333333333	0.00629947305926094\\
10.9786111111111	0.00629947305926094\\
10.9788888888889	0.00629947305926094\\
10.9791666666667	0.00715502191955105\\
10.9794444444444	0.000100087100726632\\
10.9797222222222	0.00715502191955105\\
10.98	0.0036479471727682\\
10.9802777777778	0.01\\
10.9805555555556	0.00715502191955105\\
10.9808333333333	0.00715502191955105\\
10.9811111111111	0.00112023893709131\\
10.9813888888889	0.01\\
10.9816666666667	0.01\\
10.9819444444444	0.01\\
10.9822222222222	0.01\\
10.9825	0.01\\
10.9827777777778	0.01\\
10.9830555555556	0.01\\
10.9833333333333	0.00715502191955105\\
10.9836111111111	0.00715502191955105\\
10.9838888888889	0.00715502191955105\\
10.9841666666667	0.00715502191955105\\
10.9844444444444	0.01\\
10.9847222222222	0.01\\
10.985	0.000835430950151666\\
10.9852777777778	0.00715502191955105\\
10.9855555555556	0.01\\
10.9858333333333	0.000835430950151666\\
10.9861111111111	0.00715502191955105\\
10.9863888888889	0.01\\
10.9866666666667	0.01\\
10.9869444444444	0.01\\
10.9872222222222	0.01\\
10.9875	0.01\\
10.9877777777778	0.01\\
10.9880555555556	0.01\\
10.9883333333333	0.00867512733790384\\
10.9886111111111	0.00738808561287795\\
10.9888888888889	0.00579976473268557\\
10.9891666666667	0.01\\
10.9894444444444	0.00579976473268557\\
10.9897222222222	0.00579976473268557\\
10.99	0.00738808561287795\\
10.9902777777778	0.00738808561287795\\
10.9905555555556	0.00738808561287795\\
10.9908333333333	0.00738808561287795\\
10.9911111111111	0.00112023893709131\\
10.9913888888889	0.01\\
10.9916666666667	0.01\\
10.9919444444444	0.01\\
10.9922222222222	0.01\\
10.9925	0.01\\
10.9927777777778	0.01\\
10.9930555555556	0.01\\
10.9933333333333	0.01\\
10.9936111111111	0.00402362825950001\\
10.9938888888889	0.01\\
10.9941666666667	0.01\\
10.9944444444444	0.00922742732202166\\
10.9947222222222	0.01\\
10.995	0.01\\
10.9952777777778	0.00180108844777369\\
10.9955555555556	0.01\\
10.9958333333333	0.01\\
10.9961111111111	0.00733671995772021\\
10.9963888888889	0.00525611667063562\\
10.9966666666667	0.00525611667063562\\
10.9969444444444	0.01\\
10.9972222222222	0.01\\
10.9975	0.000100087100726632\\
10.9977777777778	0.0025117741429678\\
10.9980555555556	0.00497960893516648\\
10.9983333333333	0.01\\
10.9986111111111	0.01\\
10.9988888888889	0.01\\
10.9991666666667	0.01\\
10.9994444444444	0.01\\
10.9997222222222	0.01\\
11	0.01\\
};
\addlegendentry{$\delta^-$ (sell depth)};

\addplot [color=white!40!black,solid,line width=2.0pt]
  table[row sep=crcr]{%
10.975	-0\\
10.9752777777778	-0\\
10.9755555555556	-0\\
10.9758333333333	-0\\
10.9761111111111	-0\\
10.9763888888889	-0\\
10.9766666666667	-0\\
10.9769444444444	-0.00284497808044895\\
10.9772222222222	-0.00284497808044895\\
10.9775	-0.00529220258614084\\
10.9777777777778	-0.00529220258614084\\
10.9780555555556	-0\\
10.9783333333333	-0.00420023526731443\\
10.9786111111111	-0.00420023526731443\\
10.9788888888889	-0.00420023526731443\\
10.9791666666667	-0.00370052694073906\\
10.9794444444444	-0.01\\
10.9797222222222	-0.00370052694073906\\
10.98	-0.00666392802235069\\
10.9802777777778	-0\\
10.9805555555556	-0.00370052694073906\\
10.9808333333333	-0.00370052694073906\\
10.9811111111111	-0.01\\
10.9813888888889	-0\\
10.9816666666667	-0\\
10.9819444444444	-0\\
10.9822222222222	-0\\
10.9825	-0\\
10.9827777777778	-0\\
10.9830555555556	-0\\
10.9833333333333	-0.00370052694073906\\
10.9836111111111	-0.00370052694073906\\
10.9838888888889	-0.00370052694073906\\
10.9841666666667	-0.00370052694073906\\
10.9844444444444	-0\\
10.9847222222222	-0\\
10.985	-0.01\\
10.9852777777778	-0.00370052694073906\\
10.9855555555556	-0\\
10.9858333333333	-0.01\\
10.9861111111111	-0.00370052694073906\\
10.9863888888889	-0\\
10.9866666666667	-0\\
10.9869444444444	-0\\
10.9872222222222	-0\\
10.9875	-0\\
10.9877777777778	-0\\
10.9880555555556	-0\\
10.9883333333333	-0.00261191438712204\\
10.9886111111111	-0.0028170667890354\\
10.9888888888889	-0.00433659404002523\\
10.9891666666667	-0\\
10.9894444444444	-0.00433659404002523\\
10.9897222222222	-0.00433659404002523\\
10.99	-0.0028170667890354\\
10.9902777777778	-0.0028170667890354\\
10.9905555555556	-0.0028170667890354\\
10.9908333333333	-0.0028170667890354\\
10.9911111111111	-0.01\\
10.9913888888889	-0\\
10.9916666666667	-0\\
10.9919444444444	-0\\
10.9922222222222	-0\\
10.9925	-0\\
10.9927777777778	-0\\
10.9930555555556	-0\\
10.9933333333333	-0\\
10.9936111111111	-0.00819891155222631\\
10.9938888888889	-0\\
10.9941666666667	-0\\
10.9944444444444	-0.00597637174049999\\
10.9947222222222	-0\\
10.995	-0\\
10.9952777777778	-0.00956306055999642\\
10.9955555555556	-0.00284497808044895\\
10.9958333333333	-0.00284497808044895\\
10.9961111111111	-0.00474388332936438\\
10.9963888888889	-0.00529220258614084\\
10.9966666666667	-0.00529220258614084\\
10.9969444444444	-0\\
10.9972222222222	-0.00132487266209616\\
10.9975	-0.01\\
10.9977777777778	-0.00916456904984833\\
10.9980555555556	-0.00788162502896627\\
10.9983333333333	-0\\
10.9986111111111	-0\\
10.9988888888889	-0\\
10.9991666666667	-0.00284497808044895\\
10.9994444444444	-0.00284497808044895\\
10.9997222222222	-0.00284497808044895\\
11	-0.00284497808044895\\
};
\addlegendentry{$\delta^+$ (buy depth)};

\end{axis}
\end{tikzpicture}%

\end{subfigure}%
\hfill%
\begin{subfigure}{.45\linewidth}
  \centering
  \setlength\figureheight{\linewidth} 
  \setlength\figurewidth{\linewidth}
  \tikzsetnextfilename{samplepath_dscr_depths}
  % This file was created by matlab2tikz.
%
%The latest updates can be retrieved from
%  http://www.mathworks.com/matlabcentral/fileexchange/22022-matlab2tikz-matlab2tikz
%where you can also make suggestions and rate matlab2tikz.
%
\begin{tikzpicture}

\begin{axis}[%
width=3.742in,
height=3.694in,
at={(1.889in,0.622in)},
scale only axis,
every outer x axis line/.append style={black},
every x tick label/.append style={font=\color{black}},
xmin=10.975,
xmax=11,
xlabel={Time (h)},
every outer y axis line/.append style={black},
every y tick label/.append style={font=\color{black}},
ymin=-0.01,
ymax=0.015,
ytick={-0.01,-0.005,0,0.005,0.01,0.015},
yticklabels={{ 0.01},{0.005},{    0},{0.005},{ 0.01},{0.015}},
ylabel={LO Posting Depths},
axis background/.style={fill=white},
axis x line*=bottom,
axis y line*=left,
legend style={legend cell align=left,align=left,draw=black}
]
\addplot [color=white!60!black,solid,line width=2.0pt]
  table[row sep=crcr]{%
10.975	0.00591821805671686\\
10.9752777777778	0.00591821805671686\\
10.9755555555556	0.00591821805671686\\
10.9758333333333	0.00591821805671686\\
10.9761111111111	0.00591821805671686\\
10.9763888888889	0.00591821805671686\\
10.9766666666667	0.00591821805671686\\
10.9769444444444	0.00568342788052975\\
10.9772222222222	0.00568342788052975\\
10.9775	0.00337255323101976\\
10.9777777777778	0.00337255323101976\\
10.9780555555556	0.00118499496413139\\
10.9783333333333	0.00561198181015415\\
10.9786111111111	0.00561198181015415\\
10.9788888888889	0.00561198181015415\\
10.9791666666667	0.00561198181015415\\
10.9794444444444	0.0108655723144592\\
10.9797222222222	0.00561198181015415\\
10.98	0.00100677636654667\\
10.9802777777778	0.00113803812455127\\
10.9805555555556	0.00554668592302211\\
10.9808333333333	0.00554668592302211\\
10.9811111111111	0.00900405066718764\\
10.9813888888889	0.00113803812455127\\
10.9816666666667	0.0100275388030968\\
10.9819444444444	0.0100275388030968\\
10.9822222222222	0.00775517327970128\\
10.9825	0.00775517327970128\\
10.9827777777778	0.00775517327970128\\
10.9830555555556	0.00775517327970128\\
10.9833333333333	0.00549442162806638\\
10.9836111111111	0.00549442162806638\\
10.9838888888889	0.00549442162806638\\
10.9841666666667	0.00546846610034147\\
10.9844444444444	0.0031016657810417\\
10.9847222222222	0.0100186652222556\\
10.985	0.0111119245754859\\
10.9852777777778	0.00546846610034147\\
10.9855555555556	0.00106862778989292\\
10.9858333333333	0.0111006501925519\\
10.9861111111111	0.00544780230268703\\
10.9863888888889	0.00105149996421414\\
10.9866666666667	0.010015636644651\\
10.9869444444444	0.010015636644651\\
10.9872222222222	0.010015636644651\\
10.9875	0.010015636644651\\
10.9877777777778	0.010015636644651\\
10.9880555555556	0.0100140341327421\\
10.9883333333333	0.00765623600652009\\
10.9886111111111	0.00763752137837441\\
10.9888888888889	0.00537900494058132\\
10.9891666666667	0.00101166779590929\\
10.9894444444444	0.00537900494058132\\
10.9897222222222	0.00537900494058132\\
10.99	0.00763752137837441\\
10.9902777777778	0.00763752137837441\\
10.9905555555556	0.00763752137837441\\
10.9908333333333	0.00763752137837441\\
10.9911111111111	0.00881988840160116\\
10.9913888888889	0.000987490261742568\\
10.9916666666667	0.0076161066836826\\
10.9919444444444	0.0076161066836826\\
10.9922222222222	0.00303794058045924\\
10.9925	0.0100102121666927\\
10.9927777777778	0.0100102121666927\\
10.9930555555556	0.0100102121666927\\
10.9933333333333	0.00428753092284626\\
10.9936111111111	0.0126749211683205\\
10.9938888888889	0.0100102121666927\\
10.9941666666667	0.0100102121666927\\
10.9944444444444	0.0126749211683205\\
10.9947222222222	0.00534993424615531\\
10.995	0.000959612833554843\\
10.9952777777778	0.0126749211683205\\
10.9955555555556	0.00534993424615531\\
10.9958333333333	0.00534993424615531\\
10.9961111111111	0.00307629632740271\\
10.9963888888889	0.00307629632740271\\
10.9966666666667	0.00307629632740271\\
10.9969444444444	0.000987490261742568\\
10.9972222222222	0.0076161066836826\\
10.9975	0.0107069403663601\\
10.9977777777778	0.0110643450662585\\
10.9980555555556	0.0106903541471763\\
10.9983333333333	0.00763752137837441\\
10.9986111111111	0.0100122475992815\\
10.9988888888889	0.0076161066836826\\
10.9991666666667	0.00534993424615531\\
10.9994444444444	0.00534993424615531\\
10.9997222222222	0.00534993424615531\\
11	0.00534993424615531\\
};
\addlegendentry{$\delta^-$ (sell depth)};

\addplot [color=white!40!black,solid,line width=2.0pt]
  table[row sep=crcr]{%
10.975	-0.0043118506044314\\
10.9752777777778	-0.0043118506044314\\
10.9755555555556	-0.0043118506044314\\
10.9758333333333	-0.0043118506044314\\
10.9761111111111	-0.0043118506044314\\
10.9763888888889	-0.0043118506044314\\
10.9766666666667	-0.0043118506044314\\
10.9769444444444	-0.00438576404607996\\
10.9772222222222	-0.00438576404607996\\
10.9775	-0.00670440793461694\\
10.9777777777778	-0.00670440793461694\\
10.9780555555556	-0.00883758852350922\\
10.9783333333333	-0.00445138210990432\\
10.9786111111111	-0.00445138210990432\\
10.9788888888889	-0.00445138210990432\\
10.9791666666667	-0.00445138210990432\\
10.9794444444444	-0\\
10.9797222222222	-0.00445138210990432\\
10.98	-0.00909007806073707\\
10.9802777777778	-0.00887919010845352\\
10.9805555555556	-0.00450432067409909\\
10.9808333333333	-0.00450432067409909\\
10.9811111111111	-0.00101502522624212\\
10.9813888888889	-0.00887919010845352\\
10.9816666666667	-0\\
10.9819444444444	-0\\
10.9822222222222	-0.00227736532449263\\
10.9825	-0.00227736532449263\\
10.9827777777778	-0.00227736532449263\\
10.9830555555556	-0.00227736532449263\\
10.9833333333333	-0.00453076881645172\\
10.9836111111111	-0.00453076881645172\\
10.9838888888889	-0.00453076881645172\\
10.9841666666667	-0.00455150936370967\\
10.9844444444444	-0.00690147017578766\\
10.9847222222222	-0\\
10.985	-0\\
10.9852777777778	-0.00455150936370967\\
10.9855555555556	-0.0089387697884026\\
10.9858333333333	-0\\
10.9861111111111	-0.00457235727789788\\
10.9863888888889	-0.0089565035762392\\
10.9866666666667	-0\\
10.9869444444444	-0\\
10.9872222222222	-0\\
10.9875	-0\\
10.9877777777778	-0\\
10.9880555555556	-0\\
10.9883333333333	-0.00236166998765341\\
10.9886111111111	-0.00238296460419384\\
10.9888888888889	-0.00464902033895635\\
10.9891666666667	-0.00899856307228236\\
10.9894444444444	-0.00464902033895635\\
10.9897222222222	-0.00464902033895635\\
10.99	-0.00238296460419384\\
10.9902777777778	-0.00238296460419384\\
10.9905555555556	-0.00238296460419384\\
10.9908333333333	-0.00238296460419384\\
10.9911111111111	-0.00118552248698985\\
10.9913888888889	-0.00902430997363722\\
10.9916666666667	-0.00240753134907238\\
10.9919444444444	-0.00240753134907238\\
10.9922222222222	-0.00696645880494782\\
10.9925	-0\\
10.9927777777778	-0\\
10.9930555555556	-0\\
10.9933333333333	-0.00571382654755853\\
10.9936111111111	-0\\
10.9938888888889	-0\\
10.9941666666667	-0\\
10.9944444444444	-0\\
10.9947222222222	-0.00468273019067273\\
10.995	-0.00905402104706181\\
10.9952777777778	-0\\
10.9955555555556	-0.00468273019067273\\
10.9958333333333	-0.00468273019067273\\
10.9961111111111	-0.0069569103200494\\
10.9963888888889	-0.0069569103200494\\
10.9966666666667	-0.0069569103200494\\
10.9969444444444	-0.00902430997363722\\
10.9972222222222	-0.00240753134907238\\
10.9975	-0\\
10.9977777777778	-0\\
10.9980555555556	-0\\
10.9983333333333	-0.00238296460419384\\
10.9986111111111	-0\\
10.9988888888889	-0.00240753134907238\\
10.9991666666667	-0.00468273019067273\\
10.9994444444444	-0.00468273019067273\\
10.9997222222222	-0.00468273019067273\\
11	-0.00468273019067273\\
};
\addlegendentry{$\delta^+$ (buy depth)};

\end{axis}
\end{tikzpicture}% 
\end{subfigure}\\
\vspace{1cm}
\begin{subfigure}{.45\linewidth}
  \centering
  \setlength\figureheight{\linewidth} 
  \setlength\figurewidth{\linewidth}
  \tikzsetnextfilename{samplepath_cts_nFPC_depths}
  % This file was created by matlab2tikz.
%
%The latest updates can be retrieved from
%  http://www.mathworks.com/matlabcentral/fileexchange/22022-matlab2tikz-matlab2tikz
%where you can also make suggestions and rate matlab2tikz.
%
\begin{tikzpicture}[trim axis left, trim axis right]

\begin{axis}[%
width=\figurewidth,
height=\figureheight,
at={(0\figurewidth,0\figureheight)},
scale only axis,
every outer x axis line/.append style={black},
every x tick label/.append style={font=\color{black}},
xmin=10.975,
xmax=11,
every outer y axis line/.append style={black},
every y tick label/.append style={font=\color{black}},
ymin=-0.01,
ymax=0.015,
ytick={-0.01,-0.005,0,0.005,0.01,0.015},
yticklabels={{ 0.01},{0.005},{    0},{0.005},{ 0.01},{0.015}},
axis background/.style={fill=white},
axis x line*=bottom,
axis y line*=left,
yticklabel style={
        /pgf/number format/fixed,
        /pgf/number format/precision=3
},
scaled y ticks=false,
legend style={legend cell align=left,align=left,draw=black}
]
\addplot [color=white!60!black,solid,line width=2.0pt]
  table[row sep=crcr]{%
10.975	0.01\\
10.9752777777778	0.01\\
10.9755555555556	0.01\\
10.9758333333333	0.01\\
10.9761111111111	0.01\\
10.9763888888889	0.01\\
10.9766666666667	0.01\\
10.9769444444444	0.01\\
10.9772222222222	0.01\\
10.9775	0.00817323652453748\\
10.9777777777778	0.01\\
10.9780555555556	0.00842461772871728\\
10.9783333333333	0.00842461772871728\\
10.9786111111111	0.00842461772871728\\
10.9788888888889	0.00842461772871728\\
10.9791666666667	0.000526743864001341\\
10.9794444444444	0.01\\
10.9797222222222	0.01\\
10.98	0.01\\
10.9802777777778	0.01\\
10.9805555555556	0.01\\
10.9808333333333	0.000526743864001341\\
10.9811111111111	0.01\\
10.9813888888889	0.01\\
10.9816666666667	0.01\\
10.9819444444444	0.01\\
10.9822222222222	0.01\\
10.9825	0.01\\
10.9827777777778	0.01\\
10.9830555555556	0.01\\
10.9833333333333	0.01\\
10.9836111111111	0.01\\
10.9838888888889	0.01\\
10.9841666666667	0.01\\
10.9844444444444	0.01\\
10.9847222222222	0.00162143504819226\\
10.985	0.01\\
10.9852777777778	0.01\\
10.9855555555556	0.000526743864001341\\
10.9858333333333	0.01\\
10.9861111111111	0.01\\
10.9863888888889	0.01\\
10.9866666666667	0.01\\
10.9869444444444	0.01\\
10.9872222222222	0.01\\
10.9875	0.01\\
10.9877777777778	0.01\\
10.9880555555556	0.01\\
10.9883333333333	0.01\\
10.9886111111111	0.01\\
10.9888888888889	0.01\\
10.9891666666667	0.00671160941927011\\
10.9894444444444	0.00671160941927011\\
10.9897222222222	0.00671160941927011\\
10.99	0.00863206466588298\\
10.9902777777778	0.00863206466588298\\
10.9905555555556	0.00863206466588298\\
10.9908333333333	0.000526179883314139\\
10.9911111111111	0.01\\
10.9913888888889	0.01\\
10.9916666666667	0.01\\
10.9919444444444	0.01\\
10.9922222222222	0.01\\
10.9925	0.01\\
10.9927777777778	0.01\\
10.9930555555556	0.01\\
10.9933333333333	0.00162143504819226\\
10.9936111111111	0.01\\
10.9938888888889	0.01\\
10.9941666666667	0.00162143504819226\\
10.9944444444444	0.01\\
10.9947222222222	0.01\\
10.995	0.000526743864001341\\
10.9952777777778	0.01\\
10.9955555555556	0.01\\
10.9958333333333	0.01\\
10.9961111111111	0.01\\
10.9963888888889	0.01\\
10.9966666666667	0.01\\
10.9969444444444	0.01\\
10.9972222222222	0.000526179883314139\\
10.9975	0.00342179207905144\\
10.9977777777778	0\\
10.9980555555556	0.00120095598319226\\
10.9983333333333	0.00491204156068829\\
10.9986111111111	0.00642002679226435\\
10.9988888888889	0.00334305978087883\\
10.9991666666667	0.00120095598319226\\
10.9994444444444	0.00120095598319226\\
10.9997222222222	0.00120095598319226\\
11	0.00120095598319226\\
};
\addlegendentry{$\delta^-$ (sell depth)};

\addplot [color=white!40!black,solid,line width=2.0pt]
  table[row sep=crcr]{%
10.975	-0\\
10.9752777777778	-0\\
10.9755555555556	-0\\
10.9758333333333	-0\\
10.9761111111111	-0\\
10.9763888888889	-0\\
10.9766666666667	-0\\
10.9769444444444	-0\\
10.9772222222222	-0\\
10.9775	-0.00393602061303001\\
10.9777777777778	-0\\
10.9780555555556	-0.0032883905807299\\
10.9783333333333	-0.0032883905807299\\
10.9786111111111	-0.0032883905807299\\
10.9788888888889	-0.0032883905807299\\
10.9791666666667	-0.01\\
10.9794444444444	-0\\
10.9797222222222	-0\\
10.98	-0\\
10.9802777777778	-0\\
10.9805555555556	-0\\
10.9808333333333	-0.01\\
10.9811111111111	-0\\
10.9813888888889	-0\\
10.9816666666667	-0\\
10.9819444444444	-0\\
10.9822222222222	-0\\
10.9825	-0\\
10.9827777777778	-0\\
10.9830555555556	-0\\
10.9833333333333	-0.00157538227128272\\
10.9836111111111	-0.00157538227128272\\
10.9838888888889	-0.00157538227128272\\
10.9841666666667	-0\\
10.9844444444444	-0.00136793533411702\\
10.9847222222222	-0.01\\
10.985	-0\\
10.9852777777778	-0\\
10.9855555555556	-0.01\\
10.9858333333333	-0\\
10.9861111111111	-0\\
10.9863888888889	-0\\
10.9866666666667	-0\\
10.9869444444444	-0\\
10.9872222222222	-0\\
10.9875	-0\\
10.9877777777778	-0\\
10.9880555555556	-0\\
10.9883333333333	-0\\
10.9886111111111	-0.00136793533411702\\
10.9888888888889	-0\\
10.9891666666667	-0.00400283612999854\\
10.9894444444444	-0.00400283612999854\\
10.9897222222222	-0.00400283612999854\\
10.99	-0.00265697385624915\\
10.9902777777778	-0.00265697385624915\\
10.9905555555556	-0.00265697385624915\\
10.9908333333333	-0.01\\
10.9911111111111	-0\\
10.9913888888889	-0\\
10.9916666666667	-0\\
10.9919444444444	-0\\
10.9922222222222	-0\\
10.9925	-0\\
10.9927777777778	-0\\
10.9930555555556	-0\\
10.9933333333333	-0.01\\
10.9936111111111	-0\\
10.9938888888889	-0\\
10.9941666666667	-0.01\\
10.9944444444444	-0\\
10.9947222222222	-0\\
10.995	-0.01\\
10.9952777777778	-0\\
10.9955555555556	-0\\
10.9958333333333	-0\\
10.9961111111111	-0\\
10.9963888888889	-0.00182676347546253\\
10.9966666666667	-0\\
10.9969444444444	-0.00157538227128272\\
10.9972222222222	-0.01\\
10.9975	-0.00947325613599866\\
10.9977777777778	-0.01\\
10.9980555555556	-0.01\\
10.9983333333333	-0.00665694021912117\\
10.9986111111111	-0.00443985676341491\\
10.9988888888889	-0.01\\
10.9991666666667	-0.01\\
10.9994444444444	-0.01\\
10.9997222222222	-0.01\\
11	-0.01\\
};
\addlegendentry{$\delta^+$ (buy depth)};

\end{axis}
\end{tikzpicture}%

\end{subfigure}%
\hfill%
\begin{subfigure}{.45\linewidth}
  \centering
  \setlength\figureheight{\linewidth} 
  \setlength\figurewidth{\linewidth}
  \tikzsetnextfilename{samplepath_dscr_nFPC_depths}
  % This file was created by matlab2tikz.
%
%The latest updates can be retrieved from
%  http://www.mathworks.com/matlabcentral/fileexchange/22022-matlab2tikz-matlab2tikz
%where you can also make suggestions and rate matlab2tikz.
%
\begin{tikzpicture}[trim axis left, trim axis right]

\begin{axis}[%
width=\figurewidth,
height=\figureheight,
at={(0\figurewidth,0\figureheight)},
scale only axis,
every outer x axis line/.append style={black},
every x tick label/.append style={font=\color{black}},
xmin=10.975,
xmax=11,
every outer y axis line/.append style={black},
every y tick label/.append style={font=\color{black}},
ymin=-0.015,
ymax=0.015,
ytick={-0.015,-0.01,-0.005,0,0.005,0.01,0.015},
yticklabels={{0.015},{ 0.01},{0.005},{    0},{0.005},{ 0.01},{0.015}},
axis background/.style={fill=white},
axis x line*=bottom,
axis y line*=left,
yticklabel style={
        /pgf/number format/fixed,
        /pgf/number format/precision=3
},
scaled y ticks=false,
legend style={legend cell align=left,align=left,draw=black}
]
\addplot [color=white!60!black,solid,line width=2.0pt]
  table[row sep=crcr]{%
10.975	0.00955914496776727\\
10.9752777777778	0.00955914496776727\\
10.9755555555556	0.00955914496776727\\
10.9758333333333	0.00955914496776727\\
10.9761111111111	0.00955914496776727\\
10.9763888888889	0.00955914496776727\\
10.9766666666667	0.00955914496776727\\
10.9769444444444	0.00779076841678636\\
10.9772222222222	0.00779076841678636\\
10.9775	0.0044883451970511\\
10.9777777777778	0\\
10.9780555555556	0.00701443138225722\\
10.9783333333333	0.00701443138225722\\
10.9786111111111	0.00701443138225722\\
10.9788888888889	0.00701443138225722\\
10.9791666666667	0.0118829629195527\\
10.9794444444444	0.00701443138225722\\
10.9797222222222	0.00701443138225722\\
10.98	0\\
10.9802777777778	0.00955914496776727\\
10.9805555555556	0.00955914496776727\\
10.9808333333333	0.01265085491098\\
10.9811111111111	0.00215234612348661\\
10.9813888888889	0.00975933490261401\\
10.9816666666667	0.0102216043060101\\
10.9819444444444	0.0102216043060101\\
10.9822222222222	0.0099930529003391\\
10.9825	0.0099930529003391\\
10.9827777777778	0.0099930529003391\\
10.9830555555556	0.0099930529003391\\
10.9833333333333	0.00955914496776727\\
10.9836111111111	0.00955914496776727\\
10.9838888888889	0.00955914496776727\\
10.9841666666667	0.00248878779155287\\
10.9844444444444	0.00979691828718088\\
10.9847222222222	0.0134558419257991\\
10.985	0.00979691828718088\\
10.9852777777778	0.00248878779155287\\
10.9855555555556	0.0118829629195527\\
10.9858333333333	0.00972589182777017\\
10.9861111111111	0.00209358620056745\\
10.9863888888889	0.00701443138225722\\
10.9866666666667	0.0101671828515352\\
10.9869444444444	0.0101671828515352\\
10.9872222222222	0.0101671828515352\\
10.9875	0.0101671828515352\\
10.9877777777778	0.0101671828515352\\
10.9880555555556	0.0101581955578191\\
10.9883333333333	0.00966508722521721\\
10.9886111111111	0.00886539991386342\\
10.9888888888889	0.00142567684150565\\
10.9891666666667	0.00580267881232323\\
10.9894444444444	0.00580267881232323\\
10.9897222222222	0.00580267881232323\\
10.99	0.00766868719702822\\
10.9902777777778	0.00766868719702822\\
10.9905555555556	0.00766868719702822\\
10.9908333333333	0.0115804064569313\\
10.9911111111111	0\\
10.9913888888889	0.00779076841678636\\
10.9916666666667	0.00979691828718088\\
10.9919444444444	0.00413010131820344\\
10.9922222222222	0.00979691828718088\\
10.9925	0.0101809572800362\\
10.9927777777778	0.0101809572800362\\
10.9930555555556	0.00509864754750646\\
10.9933333333333	0.0133875791588378\\
10.9936111111111	0.0101809572800362\\
10.9938888888889	0.0101809572800362\\
10.9941666666667	0.0134558419257991\\
10.9944444444444	0.0102110954468975\\
10.9947222222222	0.00330866265227938\\
10.995	0.0121207619446429\\
10.9952777777778	0.0101809572800362\\
10.9955555555556	0.00779076841678636\\
10.9958333333333	0.00779076841678636\\
10.9961111111111	0.00493968894246695\\
10.9963888888889	0.0044883451970511\\
10.9966666666667	0\\
10.9969444444444	0.00673677651284548\\
10.9972222222222	0.0123883907620873\\
10.9975	0.0117432433564966\\
10.9977777777778	0.0125065370642039\\
10.9980555555556	0.00673677651284548\\
10.9983333333333	0.00966508722521721\\
10.9986111111111	0.0101581955578191\\
10.9988888888889	0.00886539991386342\\
10.9991666666667	0.00623355705305608\\
10.9994444444444	0.00623355705305608\\
10.9997222222222	0.00623355705305608\\
11	0.00623355705305608\\
};
\addlegendentry{$\delta^-$ (sell depth)};

\addplot [color=white!40!black,solid,line width=2.0pt]
  table[row sep=crcr]{%
10.975	-0.00212359589424925\\
10.9752777777778	-0.00212359589424925\\
10.9755555555556	-0.00212359589424925\\
10.9758333333333	-0.00212359589424925\\
10.9761111111111	-0.00212359589424925\\
10.9763888888889	-0.00212359589424925\\
10.9766666666667	-0.00212359589424925\\
10.9769444444444	-0.00294796026086679\\
10.9772222222222	-0.00294796026086679\\
10.9775	-0.00573839131972788\\
10.9777777777778	-0.0101544599606959\\
10.9780555555556	-0.003232464310747\\
10.9783333333333	-0.003232464310747\\
10.9786111111111	-0.003232464310747\\
10.9788888888889	-0.003232464310747\\
10.9791666666667	-0\\
10.9794444444444	-0.003232464310747\\
10.9797222222222	-0.003232464310747\\
10.98	-0.0104852001010911\\
10.9802777777778	-0.00212359589424925\\
10.9805555555556	-0.00212359589424925\\
10.9808333333333	-0\\
10.9811111111111	-0.00846939986349565\\
10.9813888888889	-0.000347089031771669\\
10.9816666666667	-0\\
10.9819444444444	-0\\
10.9822222222222	-3.34410504625313e-05\\
10.9825	-3.34410504625313e-05\\
10.9827777777778	-3.34410504625313e-05\\
10.9830555555556	-3.34410504625313e-05\\
10.9833333333333	-0.00212359589424925\\
10.9836111111111	-0.00212359589424925\\
10.9838888888889	-0.00212359589424925\\
10.9841666666667	-0.00782457399156521\\
10.9844444444444	-0.000269677342648533\\
10.9847222222222	-0\\
10.985	-0.000269677342648533\\
10.9852777777778	-0.00782457399156521\\
10.9855555555556	-0\\
10.9858333333333	-0.000298251561890983\\
10.9861111111111	-0.00804787663159079\\
10.9863888888889	-0.003232464310747\\
10.9866666666667	-0\\
10.9869444444444	-0\\
10.9872222222222	-0\\
10.9875	-0\\
10.9877777777778	-0\\
10.9880555555556	-0\\
10.9883333333333	-0.00105099237168575\\
10.9886111111111	-0.00226486415908192\\
10.9888888888889	-0.00891805972735326\\
10.9891666666667	-0.00431435392241137\\
10.9894444444444	-0.00431435392241137\\
10.9897222222222	-0.00431435392241137\\
10.99	-0.0025255406205822\\
10.9902777777778	-0.0025255406205822\\
10.9905555555556	-0.0025255406205822\\
10.9908333333333	-0\\
10.9911111111111	-0.0101544599606959\\
10.9913888888889	-0.00294796026086679\\
10.9916666666667	-0.000269677342648533\\
10.9919444444444	-0.00607099290853057\\
10.9922222222222	-0.000269677342648533\\
10.9925	-0\\
10.9927777777778	-0\\
10.9930555555556	-0.00497752657405822\\
10.9933333333333	-0\\
10.9936111111111	-0\\
10.9938888888889	-0\\
10.9941666666667	-0\\
10.9944444444444	-0\\
10.9947222222222	-0.00734478453907611\\
10.995	-0\\
10.9952777777778	-0\\
10.9955555555556	-0.00294796026086679\\
10.9958333333333	-0.00294796026086679\\
10.9961111111111	-0.00548362231286444\\
10.9963888888889	-0.00573839131972788\\
10.9966666666667	-0.0101544599606959\\
10.9969444444444	-0.00373041551627003\\
10.9972222222222	-0\\
10.9975	-0\\
10.9977777777778	-0\\
10.9980555555556	-0.00373041551627003\\
10.9983333333333	-0.00105099237168575\\
10.9986111111111	-0\\
10.9988888888889	-0.00226486415908192\\
10.9991666666667	-0.00417562798910466\\
10.9994444444444	-0.00417562798910466\\
10.9997222222222	-0.00417562798910466\\
11	-0.00417562798910466\\
};
\addlegendentry{$\delta^+$ (buy depth)};

\end{axis}
\end{tikzpicture}%
 
\end{subfigure}\\
\leavevmode\smash{\makebox[0pt]{\hspace{-7em}% HORIZONTAL POSITION           
  \rotatebox[origin=l]{90}{\hspace{20em}% VERTICAL POSITION
    PnL}%
}}\hspace{0pt plus 1filll}\null

Time (h)

\vspace{1cm}
\begin{subfigure}{\linewidth}
  \centering
  \setlength\figureheight{\linewidth} 
  \setlength\figurewidth{\linewidth}
  \tikzsetnextfilename{samplepathslegend}
  \definecolor{mycolor1}{rgb}{0.65098,0.80784,0.89020}%
\definecolor{mycolor2}{rgb}{0.69804,0.87451,0.54118}%
\definecolor{mycolor3}{rgb}{0.20000,0.62745,0.17255}%
\definecolor{mycolor4}{rgb}{0.12157,0.47059,0.70588}%
\begin{tikzpicture}
    \begingroup
    % inits/clears the lists (which might be populated from previous
    % axes):
    \csname pgfplots@init@cleared@structures\endcsname
    \pgfplotsset{legend cell align=left,legend columns = 2,legend style={at={(0,1)},anchor=north west},legend style={draw=black,column sep=1ex},
    legend entries={Midprice,
    				Midprice $\pm \delta^\pm$,
    				Our Sell MO,
    				Our Buy MO,
    				Ext Buy MO lifts our sell LO,
    				Ext Sell MO lifts our buy LO,
    				Ext Buy MO arrives,
    				Ext Sell MO arrives}}%
    \csname pgfplots@addlegendimage\endcsname{line width=2pt,black,solid,sharp plot}
    \csname pgfplots@addlegendimage\endcsname{line width=2pt,gray,solid,sharp plot}
    \csname pgfplots@addlegendimage\endcsname{line width=1.5pt,mycolor4,solid,mark=*,mark options={solid,fill=mycolor4},sharp plot}%sell
    \csname pgfplots@addlegendimage\endcsname{line width=1.5pt,mycolor3,solid,mark=*,mark options={solid,fill=mycolor3},sharp plot}%buy
    \csname pgfplots@addlegendimage\endcsname{line width=1.5pt,mycolor1,solid,mark=*,mark options={solid,fill=mycolor1},sharp plot}% ext buy lifts
    \csname pgfplots@addlegendimage\endcsname{line width=1.5pt,mycolor2,solid,mark=*,mark options={solid,fill=mycolor2},sharp plot}%ext sell lifts    
    \csname pgfplots@addlegendimage\endcsname{line width=1pt,mycolor1,dashed,mark=o,mark options={solid},sharp plot}%ext buy
    \csname pgfplots@addlegendimage\endcsname{line width=1pt,mycolor2,dashed,mark=o,mark options={solid},sharp plot}%ext sell


    % draws the legend:
    \csname pgfplots@createlegend\endcsname
    \endgroup
\end{tikzpicture}
\end{subfigure}%
  \caption{Sample paths of the optimal trading strategies, showing price, limit order posting depths, executed market orders, and filled limit orders.}
  \label{fig:samplepath_depths}
\end{figure}

\begin{figure}
\centering
\begin{subfigure}{.45\linewidth}
  \centering
  \setlength\figureheight{\linewidth} 
  \setlength\figurewidth{\linewidth}
  \tikzsetnextfilename{samplepath_cts_bookvalue}
  % This file was created by matlab2tikz.
%
%The latest updates can be retrieved from
%  http://www.mathworks.com/matlabcentral/fileexchange/22022-matlab2tikz-matlab2tikz
%where you can also make suggestions and rate matlab2tikz.
%
\begin{tikzpicture}

\begin{axis}[%
width=4.376in,
height=3.694in,
at={(1.256in,0.622in)},
scale only axis,
separate axis lines,
every outer x axis line/.append style={black},
every x tick label/.append style={font=\color{black}},
xmin=10.975,
xmax=11,
xlabel={Time (h)},
every outer y axis line/.append style={black},
every y tick label/.append style={font=\color{black}},
ymin=1.56,
ymax=1.72,
ylabel={Book Value},
axis background/.style={fill=white}
]
\addplot [color=black,solid,line width=2.0pt,forget plot]
  table[row sep=crcr]{%
10.975	1.62135739067691\\
10.9752777777778	1.62135739067691\\
10.9755555555556	1.62135739067691\\
10.9758333333333	1.62135739067691\\
10.9761111111111	1.62135739067691\\
10.9763888888889	1.62135739067691\\
10.9766666666667	1.62135739067691\\
10.9769444444444	1.6013573906769\\
10.9772222222222	1.6013573906769\\
10.9775	1.58420236875733\\
10.9777777777778	1.58420236875733\\
10.9780555555556	1.56949457134348\\
10.9783333333333	1.56949457134348\\
10.9786111111111	1.56949457134348\\
10.9788888888889	1.56949457134348\\
10.9791666666667	1.60579404440273\\
10.9794444444444	1.60579404440273\\
10.9797222222222	1.60579404440273\\
10.98	1.60579404440273\\
10.9802777777778	1.60579404440273\\
10.9805555555556	1.60579404440273\\
10.9808333333333	1.60579404440273\\
10.9811111111111	1.64294906632227\\
10.9813888888889	1.64294906632227\\
10.9816666666667	1.64294906632227\\
10.9819444444444	1.64294906632227\\
10.9822222222222	1.64294906632227\\
10.9825	1.64294906632227\\
10.9827777777778	1.64294906632227\\
10.9830555555556	1.64294906632227\\
10.9833333333333	1.63294906632228\\
10.9836111111111	1.63294906632228\\
10.9838888888889	1.63294906632228\\
10.9841666666667	1.63294906632228\\
10.9844444444444	1.63294906632228\\
10.9847222222222	1.63294906632228\\
10.985	1.63294906632228\\
10.9852777777778	1.63294906632228\\
10.9855555555556	1.62664959326301\\
10.9858333333333	1.636649593263\\
10.9861111111111	1.636649593263\\
10.9863888888889	1.636649593263\\
10.9866666666667	1.636649593263\\
10.9869444444444	1.636649593263\\
10.9872222222222	1.636649593263\\
10.9875	1.636649593263\\
10.9877777777778	1.636649593263\\
10.9880555555556	1.636649593263\\
10.9883333333333	1.636649593263\\
10.9886111111111	1.63926150765013\\
10.9888888888889	1.63926150765013\\
10.9891666666667	1.63926150765013\\
10.9894444444444	1.63926150765013\\
10.9897222222222	1.63926150765013\\
10.99	1.63926150765013\\
10.9902777777778	1.63926150765013\\
10.9905555555556	1.63926150765013\\
10.9908333333333	1.63926150765013\\
10.9911111111111	1.63926150765013\\
10.9913888888889	1.65038174658721\\
10.9916666666667	1.65038174658721\\
10.9919444444444	1.65038174658721\\
10.9922222222222	1.65038174658721\\
10.9925	1.65038174658721\\
10.9927777777778	1.65038174658721\\
10.9930555555556	1.65038174658721\\
10.9933333333333	1.66038174658721\\
10.9936111111111	1.66038174658721\\
10.9938888888889	1.66038174658721\\
10.9941666666667	1.66038174658721\\
10.9944444444444	1.6703817465872\\
10.9947222222222	1.66038174658721\\
10.995	1.66038174658721\\
10.9952777777778	1.66038174658721\\
10.9955555555556	1.66038174658721\\
10.9958333333333	1.66038174658721\\
10.9961111111111	1.66038174658721\\
10.9963888888889	1.66512562991656\\
10.9966666666667	1.66512562991656\\
10.9969444444444	1.66512562991656\\
10.9972222222222	1.66512562991656\\
10.9975	1.66512562991656\\
10.9977777777778	1.67522571701728\\
10.9980555555556	1.69773749116025\\
10.9983333333333	1.70271710009541\\
10.9986111111111	1.70271710009541\\
10.9988888888889	1.6827171000954\\
10.9991666666667	1.6827171000954\\
10.9994444444444	1.6827171000954\\
10.9997222222222	1.6827171000954\\
11	1.6827171000954\\
};
\end{axis}
\end{tikzpicture}%
\end{subfigure}%
\hfill%
\begin{subfigure}{.45\linewidth}
  \centering
  \setlength\figureheight{\linewidth} 
  \setlength\figurewidth{\linewidth}
  \tikzsetnextfilename{samplepath_dscr_bookvalue}
  % This file was created by matlab2tikz.
%
%The latest updates can be retrieved from
%  http://www.mathworks.com/matlabcentral/fileexchange/22022-matlab2tikz-matlab2tikz
%where you can also make suggestions and rate matlab2tikz.
%
\begin{tikzpicture}[trim axis left, trim axis right]

\begin{axis}[%
width=\figurewidth,
height=\figureheight,
at={(0\figurewidth,0\figureheight)},
scale only axis,
separate axis lines,
every outer x axis line/.append style={black},
every x tick label/.append style={font=\color{black}},
xmin=10.975,
xmax=11,
every outer y axis line/.append style={black},
every y tick label/.append style={font=\color{black}},
ymin=2.22,
ymax=2.29,
axis background/.style={fill=white}
]
\addplot [color=black,solid,line width=2.0pt,forget plot]
  table[row sep=crcr]{%
10.975	2.27140001455481\\
10.9752777777778	2.27140001455481\\
10.9755555555556	2.27140001455481\\
10.9758333333333	2.27140001455481\\
10.9761111111111	2.27140001455481\\
10.9763888888889	2.27140001455481\\
10.9766666666667	2.27140001455481\\
10.9769444444444	2.25571186515924\\
10.9772222222222	2.25571186515924\\
10.9775	2.24009762920537\\
10.9777777777778	2.24009762920537\\
10.9780555555556	2.24009762920537\\
10.9783333333333	2.24009762920537\\
10.9786111111111	2.24009762920537\\
10.9788888888889	2.24009762920537\\
10.9791666666667	2.24009762920537\\
10.9794444444444	2.24009762920537\\
10.9797222222222	2.24009762920537\\
10.98	2.24009762920537\\
10.9802777777778	2.22918770726608\\
10.9805555555556	2.22918770726608\\
10.9808333333333	2.22918770726608\\
10.9811111111111	2.26473439318914\\
10.9813888888889	2.24574941841536\\
10.9816666666667	2.24574941841536\\
10.9819444444444	2.24574941841536\\
10.9822222222222	2.24574941841536\\
10.9825	2.24574941841536\\
10.9827777777778	2.24574941841536\\
10.9830555555556	2.24574941841536\\
10.9833333333333	2.23802678373983\\
10.9836111111111	2.23802678373983\\
10.9838888888889	2.23802678373983\\
10.9841666666667	2.23255755255633\\
10.9844444444444	2.23255755255633\\
10.9847222222222	2.23255755255633\\
10.985	2.23255755255633\\
10.9852777777778	2.23255755255633\\
10.9855555555556	2.22710906192009\\
10.9858333333333	2.22710906192009\\
10.9861111111111	2.22710906192009\\
10.9863888888889	2.22168141919798\\
10.9866666666667	2.22168141919798\\
10.9869444444444	2.22168141919798\\
10.9872222222222	2.22168141919798\\
10.9875	2.22168141919798\\
10.9877777777778	2.22168141919798\\
10.9880555555556	2.22168141919792\\
10.9883333333333	2.22168141919792\\
10.9886111111111	2.22404308918556\\
10.9888888888889	2.22404308918556\\
10.9891666666667	2.22404308918556\\
10.9894444444444	2.22404308918556\\
10.9897222222222	2.22404308918556\\
10.99	2.22404308918556\\
10.9902777777778	2.22404308918556\\
10.9905555555556	2.22404308918556\\
10.9908333333333	2.22404308918556\\
10.9911111111111	2.22404308918556\\
10.9913888888889	2.22522861167261\\
10.9916666666667	2.22522861167261\\
10.9919444444444	2.22522861167261\\
10.9922222222222	2.22522861167261\\
10.9925	2.22522861167261\\
10.9927777777778	2.22522861167261\\
10.9930555555556	2.22522861167261\\
10.9933333333333	2.2352286116726\\
10.9936111111111	2.22951614259546\\
10.9938888888889	2.22951614259546\\
10.9941666666667	2.22951614259546\\
10.9944444444444	2.22951614259546\\
10.9947222222222	2.22951614259546\\
10.995	2.23419887278612\\
10.9952777777778	2.23515848561965\\
10.9955555555556	2.23515848561965\\
10.9958333333333	2.23515848561965\\
10.9961111111111	2.23515848561965\\
10.9963888888889	2.23515848561965\\
10.9966666666667	2.23515848561965\\
10.9969444444444	2.23515848561965\\
10.9972222222222	2.23515848561965\\
10.9975	2.25277459230335\\
10.9977777777778	2.25277459230335\\
10.9980555555556	2.24277459230336\\
10.9983333333333	2.28346494645052\\
10.9986111111111	2.28346494645052\\
10.9988888888889	2.26346494645054\\
10.9991666666667	2.26346494645054\\
10.9994444444444	2.26346494645054\\
10.9997222222222	2.26346494645054\\
11	2.26346494645054\\
};
\end{axis}
\end{tikzpicture}%
 
\end{subfigure}\\
\vspace{1cm}
\begin{subfigure}{.45\linewidth}
  \centering
  \setlength\figureheight{\linewidth} 
  \setlength\figurewidth{\linewidth}
  \tikzsetnextfilename{samplepath_cts_nFPC_bookvalue}
  % This file was created by matlab2tikz.
%
%The latest updates can be retrieved from
%  http://www.mathworks.com/matlabcentral/fileexchange/22022-matlab2tikz-matlab2tikz
%where you can also make suggestions and rate matlab2tikz.
%
\begin{tikzpicture}[trim axis left, trim axis right]

\begin{axis}[%
width=\figurewidth,
height=\figureheight,
at={(0\figurewidth,0\figureheight)},
scale only axis,
separate axis lines,
every outer x axis line/.append style={black},
every x tick label/.append style={font=\color{black}},
xtick = {10.97,10.98,10.99,11},
xmin=10.975,
xmax=11,
every outer y axis line/.append style={black},
every y tick label/.append style={font=\color{black}},
ymin=-7.4,
ymax=-7.05,
axis background/.style={fill=white}
]
\addplot [color=black,solid,line width=2.0pt,forget plot]
  table[row sep=crcr]{%
10.975	-7.07965620945679\\
10.9752777777778	-7.07965620945679\\
10.9755555555556	-7.07965620945679\\
10.9758333333333	-7.07965620945679\\
10.9761111111111	-7.07965620945679\\
10.9763888888889	-7.07965620945679\\
10.9766666666667	-7.07965620945679\\
10.9769444444444	-7.09965620945678\\
10.9772222222222	-7.09965620945678\\
10.9775	-7.11965620945679\\
10.9777777777778	-7.11965620945679\\
10.9780555555556	-7.1396562094568\\
10.9783333333333	-7.1396562094568\\
10.9786111111111	-7.1396562094568\\
10.9788888888889	-7.1396562094568\\
10.9791666666667	-7.10965620945683\\
10.9794444444444	-7.10965620945683\\
10.9797222222222	-7.10965620945683\\
10.98	-7.10965620945683\\
10.9802777777778	-7.12965620945684\\
10.9805555555556	-7.12965620945684\\
10.9808333333333	-7.11965620945685\\
10.9811111111111	-7.11912946559285\\
10.9813888888889	-7.13912946559284\\
10.9816666666667	-7.13912946559284\\
10.9819444444444	-7.13912946559284\\
10.9822222222222	-7.13912946559284\\
10.9825	-7.13912946559284\\
10.9827777777778	-7.13912946559284\\
10.9830555555556	-7.13912946559284\\
10.9833333333333	-7.14912946559284\\
10.9836111111111	-7.14912946559284\\
10.9838888888889	-7.14912946559284\\
10.9841666666667	-7.15755408332154\\
10.9844444444444	-7.15755408332154\\
10.9847222222222	-7.15755408332154\\
10.985	-7.17755408332155\\
10.9852777777778	-7.17755408332155\\
10.9855555555556	-7.18755408332154\\
10.9858333333333	-7.18755408332154\\
10.9861111111111	-7.18755408332154\\
10.9863888888889	-7.19755408332153\\
10.9866666666667	-7.19755408332153\\
10.9869444444444	-7.19755408332153\\
10.9872222222222	-7.19755408332153\\
10.9875	-7.19755408332153\\
10.9877777777778	-7.19755408332153\\
10.9880555555556	-7.19755408332153\\
10.9883333333333	-7.19755408332153\\
10.9886111111111	-7.19755408332153\\
10.9888888888889	-7.19755408332153\\
10.9891666666667	-7.19755408332153\\
10.9894444444444	-7.19755408332153\\
10.9897222222222	-7.19755408332153\\
10.99	-7.19755408332153\\
10.9902777777778	-7.19755408332153\\
10.9905555555556	-7.19755408332153\\
10.9908333333333	-7.2375540833215\\
10.9911111111111	-7.23702790343819\\
10.9913888888889	-7.23702790343819\\
10.9916666666667	-7.23702790343819\\
10.9919444444444	-7.23702790343819\\
10.9922222222222	-7.23702790343819\\
10.9925	-7.23702790343819\\
10.9927777777778	-7.23702790343819\\
10.9930555555556	-7.23702790343819\\
10.9933333333333	-7.28702790343819\\
10.9936111111111	-7.29540646838998\\
10.9938888888889	-7.29540646838998\\
10.9941666666667	-7.30540646838998\\
10.9944444444444	-7.32378503334178\\
10.9947222222222	-7.32378503334178\\
10.995	-7.33378503334177\\
10.9952777777778	-7.34325828947777\\
10.9955555555556	-7.34325828947777\\
10.9958333333333	-7.34325828947777\\
10.9961111111111	-7.34325828947777\\
10.9963888888889	-7.34325828947777\\
10.9966666666667	-7.34325828947777\\
10.9969444444444	-7.34325828947777\\
10.9972222222222	-7.36325828947776\\
10.9975	-7.35273210959444\\
10.9977777777778	-7.33931031751539\\
10.9980555555556	-7.32931031751539\\
10.9983333333333	-7.30810936153219\\
10.9986111111111	-7.30810936153219\\
10.9988888888889	-7.3136695047688\\
10.9991666666667	-7.3136695047688\\
10.9994444444444	-7.3136695047688\\
10.9997222222222	-7.3136695047688\\
11	-7.3136695047688\\
};
\end{axis}
\end{tikzpicture}%

\end{subfigure}%
\hfill%
\begin{subfigure}{.45\linewidth}
  \centering
  \setlength\figureheight{\linewidth} 
  \setlength\figurewidth{\linewidth}
  \tikzsetnextfilename{samplepath_dscr_nFPC_bookvalue}
  % This file was created by matlab2tikz.
%
%The latest updates can be retrieved from
%  http://www.mathworks.com/matlabcentral/fileexchange/22022-matlab2tikz-matlab2tikz
%where you can also make suggestions and rate matlab2tikz.
%
\begin{tikzpicture}[trim axis left, trim axis right]

\begin{axis}[%
width=\figurewidth,
height=\figureheight,
at={(0\figurewidth,0\figureheight)},
scale only axis,
separate axis lines,
every outer x axis line/.append style={black},
every x tick label/.append style={font=\color{black}},
xtick = {10.97,10.98,10.99,11},
xmin=10.975,
xmax=11,
every outer y axis line/.append style={black},
every y tick label/.append style={font=\color{black}},
ymin=4.25,
ymax=4.5,
axis background/.style={fill=white}
]
\addplot [color=black,solid,line width=2.0pt,forget plot]
  table[row sep=crcr]{%
10.975	4.31785157278387\\
10.9752777777778	4.31785157278387\\
10.9755555555556	4.31785157278387\\
10.9758333333333	4.31785157278387\\
10.9761111111111	4.31785157278387\\
10.9763888888889	4.31785157278387\\
10.9766666666667	4.31785157278387\\
10.9769444444444	4.29997516867812\\
10.9772222222222	4.29997516867812\\
10.9775	4.28292312893899\\
10.9777777777778	4.28292312893899\\
10.9780555555556	4.28292312893899\\
10.9783333333333	4.28292312893899\\
10.9786111111111	4.28292312893899\\
10.9788888888889	4.28292312893899\\
10.9791666666667	4.28292312893899\\
10.9794444444444	4.28292312893899\\
10.9797222222222	4.28292312893899\\
10.98	4.34292312893902\\
10.9802777777778	4.33340832904011\\
10.9805555555556	4.33340832904011\\
10.9808333333333	4.33340832904011\\
10.9811111111111	4.3760591839511\\
10.9813888888889	4.3760591839511\\
10.9816666666667	4.3760591839511\\
10.9819444444444	4.3760591839511\\
10.9822222222222	4.3760591839511\\
10.9825	4.3760591839511\\
10.9827777777778	4.3760591839511\\
10.9830555555556	4.3760591839511\\
10.9833333333333	4.36609262500156\\
10.9836111111111	4.36609262500156\\
10.9838888888889	4.36609262500156\\
10.9841666666667	4.3582162208958\\
10.9844444444444	4.3582162208958\\
10.9847222222222	4.3582162208958\\
10.985	4.3582162208958\\
10.9852777777778	4.3582162208958\\
10.9855555555556	4.35604079488735\\
10.9858333333333	4.35604079488735\\
10.9861111111111	4.35604079488735\\
10.9863888888889	4.35604079488735\\
10.9866666666667	4.35604079488735\\
10.9869444444444	4.35604079488735\\
10.9872222222222	4.35604079488735\\
10.9875	4.35604079488735\\
10.9877777777778	4.35604079488735\\
10.9880555555556	4.35604079488735\\
10.9883333333333	4.35604079488735\\
10.9886111111111	4.35709178725898\\
10.9888888888889	4.35709178725898\\
10.9891666666667	4.36600984698629\\
10.9894444444444	4.36600984698629\\
10.9897222222222	4.36600984698629\\
10.99	4.36600984698629\\
10.9902777777778	4.36600984698629\\
10.9905555555556	4.36600984698629\\
10.9908333333333	4.36600984698629\\
10.9911111111111	4.38759025344325\\
10.9913888888889	4.39759025344324\\
10.9916666666667	4.39759025344324\\
10.9919444444444	4.39759025344324\\
10.9922222222222	4.39759025344324\\
10.9925	4.39759025344324\\
10.9927777777778	4.39759025344324\\
10.9930555555556	4.39759025344324\\
10.9933333333333	4.4125677800173\\
10.9936111111111	4.41595535917614\\
10.9938888888889	4.41595535917614\\
10.9941666666667	4.41595535917614\\
10.9944444444444	4.42941120110194\\
10.9947222222222	4.42941120110194\\
10.995	4.43675598564101\\
10.9952777777778	4.43675598564101\\
10.9955555555556	4.43675598564101\\
10.9958333333333	4.43675598564101\\
10.9961111111111	4.43675598564101\\
10.9963888888889	4.44223960795387\\
10.9966666666667	4.44223960795387\\
10.9969444444444	4.45239406791461\\
10.9972222222222	4.45239406791461\\
10.9975	4.45239406791461\\
10.9977777777778	4.47413731127108\\
10.9980555555556	4.46413731127109\\
10.9983333333333	4.46413731127109\\
10.9986111111111	4.46413731127109\\
10.9988888888889	4.44413731127105\\
10.9991666666667	4.44413731127105\\
10.9994444444444	4.44413731127105\\
10.9997222222222	4.44413731127105\\
11	4.44413731127105\\
};
\end{axis}
\end{tikzpicture}%
 
\end{subfigure}\\
\leavevmode\smash{\makebox[0pt]{\hspace{-7em}% HORIZONTAL POSITION           
  \rotatebox[origin=l]{90}{\hspace{20em}% VERTICAL POSITION
    PnL}%
}}\hspace{0pt plus 1filll}\null

Time (h)

\vspace{1cm}%
  \caption{Sample paths of the optimal trading strategies, showing price, limit order posting depths, executed market orders, and filled limit orders.}
  \label{fig:samplepath_pnl}
\end{figure}

\begin{figure}
\centering
\begin{subfigure}{.45\linewidth}
  \centering
  \setlength\figureheight{\linewidth} 
  \setlength\figurewidth{\linewidth}
  \tikzsetnextfilename{samplepath_cts_inventory}
  % This file was created by matlab2tikz.
%
%The latest updates can be retrieved from
%  http://www.mathworks.com/matlabcentral/fileexchange/22022-matlab2tikz-matlab2tikz
%where you can also make suggestions and rate matlab2tikz.
%
\begin{tikzpicture}[trim axis left, trim axis right]

\begin{axis}[%
width=\figurewidth,
height=\figureheight,
at={(0\figurewidth,0\figureheight)},
scale only axis,
separate axis lines,
every outer x axis line/.append style={black},
every x tick label/.append style={font=\color{black}},
xmin=10.975,
xmax=11,
every outer y axis line/.append style={black},
every y tick label/.append style={font=\color{black}},
ymin=1,
ymax=7,
axis background/.style={fill=white}
]
\addplot [color=black,solid,line width=2.0pt,forget plot]
  table[row sep=crcr]{%
10.975	2\\
10.9752777777778	2\\
10.9755555555556	2\\
10.9758333333333	2\\
10.9761111111111	2\\
10.9763888888889	2\\
10.9766666666667	2\\
10.9769444444444	3\\
10.9772222222222	3\\
10.9775	4\\
10.9777777777778	4\\
10.9780555555556	5\\
10.9783333333333	5\\
10.9786111111111	5\\
10.9788888888889	5\\
10.9791666666667	4\\
10.9794444444444	4\\
10.9797222222222	4\\
10.98	4\\
10.9802777777778	4\\
10.9805555555556	4\\
10.9808333333333	4\\
10.9811111111111	3\\
10.9813888888889	3\\
10.9816666666667	3\\
10.9819444444444	3\\
10.9822222222222	3\\
10.9825	3\\
10.9827777777778	3\\
10.9830555555556	3\\
10.9833333333333	4\\
10.9836111111111	4\\
10.9838888888889	4\\
10.9841666666667	4\\
10.9844444444444	4\\
10.9847222222222	4\\
10.985	4\\
10.9852777777778	4\\
10.9855555555556	5\\
10.9858333333333	4\\
10.9861111111111	4\\
10.9863888888889	4\\
10.9866666666667	4\\
10.9869444444444	4\\
10.9872222222222	4\\
10.9875	4\\
10.9877777777778	4\\
10.9880555555556	5\\
10.9883333333333	5\\
10.9886111111111	6\\
10.9888888888889	6\\
10.9891666666667	6\\
10.9894444444444	6\\
10.9897222222222	6\\
10.99	6\\
10.9902777777778	6\\
10.9905555555556	6\\
10.9908333333333	6\\
10.9911111111111	3\\
10.9913888888889	2\\
10.9916666666667	2\\
10.9919444444444	2\\
10.9922222222222	2\\
10.9925	2\\
10.9927777777778	2\\
10.9930555555556	2\\
10.9933333333333	3\\
10.9936111111111	2\\
10.9938888888889	2\\
10.9941666666667	2\\
10.9944444444444	1\\
10.9947222222222	2\\
10.995	3\\
10.9952777777778	3\\
10.9955555555556	3\\
10.9958333333333	3\\
10.9961111111111	3\\
10.9963888888889	4\\
10.9966666666667	4\\
10.9969444444444	4\\
10.9972222222222	4\\
10.9975	4\\
10.9977777777778	3\\
10.9980555555556	2\\
10.9983333333333	2\\
10.9986111111111	2\\
10.9988888888889	3\\
10.9991666666667	3\\
10.9994444444444	3\\
10.9997222222222	3\\
11	3\\
};
\end{axis}
\end{tikzpicture}%

\end{subfigure}%
\hfill%
\begin{subfigure}{.45\linewidth}
  \centering
  \setlength\figureheight{\linewidth} 
  \setlength\figurewidth{\linewidth}
  \tikzsetnextfilename{samplepath_dscr_inventory}
  % This file was created by matlab2tikz.
%
%The latest updates can be retrieved from
%  http://www.mathworks.com/matlabcentral/fileexchange/22022-matlab2tikz-matlab2tikz
%where you can also make suggestions and rate matlab2tikz.
%
\begin{tikzpicture}[trim axis left, trim axis right]

\begin{axis}[%
width=\figurewidth,
height=\figureheight,
at={(0\figurewidth,0\figureheight)},
scale only axis,
separate axis lines,
every outer x axis line/.append style={black},
every x tick label/.append style={font=\color{black}},
xtick = {10.97,10.98,10.99,11},
xmin=10.975,
xmax=11,
ylabel near ticks,
yticklabel pos=right,
every outer y axis line/.append style={black},
every y tick label/.append style={font=\color{black}},
ymin=6,
ymax=18,
axis background/.style={fill=white}
]
\addplot [color=black,solid,line width=2.0pt,forget plot]
  table[row sep=crcr]{%
10.975	7\\
10.9752777777778	7\\
10.9755555555556	7\\
10.9758333333333	7\\
10.9761111111111	7\\
10.9763888888889	7\\
10.9766666666667	7\\
10.9769444444444	8\\
10.9772222222222	8\\
10.9775	9\\
10.9777777777778	9\\
10.9780555555556	9\\
10.9783333333333	9\\
10.9786111111111	9\\
10.9788888888889	9\\
10.9791666666667	9\\
10.9794444444444	9\\
10.9797222222222	9\\
10.98	9\\
10.9802777777778	10\\
10.9805555555556	10\\
10.9808333333333	10\\
10.9811111111111	9\\
10.9813888888889	10\\
10.9816666666667	10\\
10.9819444444444	10\\
10.9822222222222	10\\
10.9825	10\\
10.9827777777778	10\\
10.9830555555556	10\\
10.9833333333333	11\\
10.9836111111111	11\\
10.9838888888889	11\\
10.9841666666667	12\\
10.9844444444444	12\\
10.9847222222222	12\\
10.985	12\\
10.9852777777778	12\\
10.9855555555556	13\\
10.9858333333333	13\\
10.9861111111111	13\\
10.9863888888889	14\\
10.9866666666667	14\\
10.9869444444444	14\\
10.9872222222222	14\\
10.9875	14\\
10.9877777777778	14\\
10.9880555555556	15\\
10.9883333333333	15\\
10.9886111111111	16\\
10.9888888888889	16\\
10.9891666666667	16\\
10.9894444444444	16\\
10.9897222222222	16\\
10.99	16\\
10.9902777777778	16\\
10.9905555555556	16\\
10.9908333333333	16\\
10.9911111111111	16\\
10.9913888888889	17\\
10.9916666666667	17\\
10.9919444444444	17\\
10.9922222222222	17\\
10.9925	17\\
10.9927777777778	17\\
10.9930555555556	17\\
10.9933333333333	18\\
10.9936111111111	17\\
10.9938888888889	17\\
10.9941666666667	17\\
10.9944444444444	17\\
10.9947222222222	17\\
10.995	18\\
10.9952777777778	17\\
10.9955555555556	17\\
10.9958333333333	17\\
10.9961111111111	17\\
10.9963888888889	17\\
10.9966666666667	17\\
10.9969444444444	17\\
10.9972222222222	17\\
10.9975	16\\
10.9977777777778	16\\
10.9980555555556	17\\
10.9983333333333	16\\
10.9986111111111	16\\
10.9988888888889	17\\
10.9991666666667	17\\
10.9994444444444	17\\
10.9997222222222	17\\
11	17\\
};
\end{axis}
\end{tikzpicture}%
 
\end{subfigure}\\
\vspace{1cm}
\begin{subfigure}{.45\linewidth}
  \centering
  \setlength\figureheight{\linewidth} 
  \setlength\figurewidth{\linewidth}
  \tikzsetnextfilename{samplepath_cts_nFPC_inventory}
  \input{Figs/samplepath_cts_nFPC_inventory.tikz}
\end{subfigure}%
\hfill%
\begin{subfigure}{.45\linewidth}
  \centering
  \setlength\figureheight{\linewidth} 
  \setlength\figurewidth{\linewidth}
  \tikzsetnextfilename{samplepath_dscr_nFPC_inventory}
  % This file was created by matlab2tikz.
%
%The latest updates can be retrieved from
%  http://www.mathworks.com/matlabcentral/fileexchange/22022-matlab2tikz-matlab2tikz
%where you can also make suggestions and rate matlab2tikz.
%
\begin{tikzpicture}[trim axis left, trim axis right]

\begin{axis}[%
width=\figurewidth,
height=\figureheight,
at={(0\figurewidth,0\figureheight)},
scale only axis,
separate axis lines,
every outer x axis line/.append style={black},
every x tick label/.append style={font=\color{black}},
xtick = {10.97,10.98,10.99,11},
xmin=10.975,
xmax=11,
ylabel near ticks,
yticklabel pos=right,
every outer y axis line/.append style={black},
every y tick label/.append style={font=\color{black}},
ymin=5,
ymax=11,
axis background/.style={fill=white}
]
\addplot [color=black,solid,line width=2.0pt,forget plot]
  table[row sep=crcr]{%
10.975	6\\
10.9752777777778	6\\
10.9755555555556	6\\
10.9758333333333	6\\
10.9761111111111	6\\
10.9763888888889	6\\
10.9766666666667	6\\
10.9769444444444	7\\
10.9772222222222	7\\
10.9775	8\\
10.9777777777778	8\\
10.9780555555556	8\\
10.9783333333333	8\\
10.9786111111111	8\\
10.9788888888889	8\\
10.9791666666667	8\\
10.9794444444444	8\\
10.9797222222222	8\\
10.98	5\\
10.9802777777778	6\\
10.9805555555556	6\\
10.9808333333333	6\\
10.9811111111111	5\\
10.9813888888889	5\\
10.9816666666667	5\\
10.9819444444444	5\\
10.9822222222222	5\\
10.9825	5\\
10.9827777777778	5\\
10.9830555555556	5\\
10.9833333333333	6\\
10.9836111111111	6\\
10.9838888888889	6\\
10.9841666666667	7\\
10.9844444444444	7\\
10.9847222222222	7\\
10.985	7\\
10.9852777777778	7\\
10.9855555555556	8\\
10.9858333333333	8\\
10.9861111111111	8\\
10.9863888888889	8\\
10.9866666666667	8\\
10.9869444444444	8\\
10.9872222222222	8\\
10.9875	8\\
10.9877777777778	8\\
10.9880555555556	9\\
10.9883333333333	9\\
10.9886111111111	10\\
10.9888888888889	10\\
10.9891666666667	11\\
10.9894444444444	11\\
10.9897222222222	11\\
10.99	11\\
10.9902777777778	11\\
10.9905555555556	11\\
10.9908333333333	11\\
10.9911111111111	8\\
10.9913888888889	7\\
10.9916666666667	7\\
10.9919444444444	7\\
10.9922222222222	7\\
10.9925	7\\
10.9927777777778	7\\
10.9930555555556	7\\
10.9933333333333	8\\
10.9936111111111	7\\
10.9938888888889	7\\
10.9941666666667	7\\
10.9944444444444	6\\
10.9947222222222	6\\
10.995	7\\
10.9952777777778	7\\
10.9955555555556	7\\
10.9958333333333	7\\
10.9961111111111	7\\
10.9963888888889	8\\
10.9966666666667	8\\
10.9969444444444	9\\
10.9972222222222	9\\
10.9975	9\\
10.9977777777778	8\\
10.9980555555556	9\\
10.9983333333333	9\\
10.9986111111111	9\\
10.9988888888889	10\\
10.9991666666667	10\\
10.9994444444444	10\\
10.9997222222222	10\\
11	10\\
};
\end{axis}
\end{tikzpicture}%
 
\end{subfigure}\\
\leavevmode\smash{\makebox[0pt]{\hspace{-7em}% HORIZONTAL POSITION           
  \rotatebox[origin=l]{90}{\hspace{19em}% VERTICAL POSITION
    Inventory}%
}}\hspace{0pt plus 1filll}\null

Time (h)

\vspace{1cm}%
  \caption{Sample paths of the optimal trading strategies, showing price, limit order posting depths, executed market orders, and filled limit orders.}
  \label{fig:samplepath_inv}
\end{figure}

\FloatBarrier
\section{In-Sample Backtesting}

\subsection{Same-Day Calibration}
We begin our in-sample backtesting same-day calibration: calibration was run for each ticker and each trading day of 2013, and backtesting was then done for each strategy using the same day's calibration. Each backtest would yield the end of day PnL, average inventory held during the day, and the number of executed market orders and filled limit orders. In \autoref{tbl:IS_sameday} we show performance values for several metrics of interest, while \autoref{fig:IS_sameday_comp} compares the day-over-day performance of the various strategies. 

Since we are calibrating and backtesting using the same underlying data, the calibration should be best attuned to the price dynamics for that praticular day, and hence we expect the performance using same-day calibration to exceed that of the weekly offset calibration and the annual calibration (detailed in the sections that follow). Looking at the \% Win column in \autoref{tbl:IS_sameday} we see that trading on \texttt{FARO} very rarely produces positive PnL. This is not surprising and was mentioned at the conclusion of the exploratory data analysis chapter: \texttt{FARO} is highly illiquid, with daily volume hovering around 200k, and its bid-ask spread averages approximately 20 cents. This makes it never profitable to execute market orders (due to crossing the spread), and because our optimal strategies still force us to post limit orders at depths between 0 and $1/\kappa = 0.01 = 1\text{cent}$, the most probable occurrence is that our limit orders are lifted adversely. \texttt{NTAP} seems to be a borderline case for liquidity, with average volumes around 4m, and here the strategies exhibit weak regularity of profits. The most liquid stocks, \texttt{ORCL} and \texttt{NTAP}, with average volumes around 15m and 30m respectively, post extremely promising results: the stochastic control strategies produce positive EOD PnL more than 90\% of the time. The discrete time controller outperforms its continuous time counterpart, and in particular we highlight that in the case of \texttt{INTC}, we attain a very good Sharpe ratio of 2.5.

\begin{figure}
\centering
\begin{subfigure}{.45\linewidth}
  \centering
  \setlength\figureheight{\linewidth} 
  \setlength\figurewidth{\linewidth}
  \tikzsetnextfilename{IS_sameday_FARO}
  % This file was created by matlab2tikz.
%
%The latest updates can be retrieved from
%  http://www.mathworks.com/matlabcentral/fileexchange/22022-matlab2tikz-matlab2tikz
%where you can also make suggestions and rate matlab2tikz.
%
\definecolor{mycolor1}{rgb}{0.25098,0.00000,0.38824}%
\definecolor{mycolor2}{rgb}{0.00000,0.46275,0.00000}%
\definecolor{mycolor3}{rgb}{0.00000,0.34902,0.34902}%
\definecolor{mycolor4}{rgb}{0.58039,0.26275,0.00000}%
%
\begin{tikzpicture}[trim axis left, trim axis right]

\begin{axis}[%
width=\figurewidth,
height=\figureheight,
at={(0\figurewidth,0\figureheight)},
scale only axis,
every outer x axis line/.append style={black},
every x tick label/.append style={font=\color{black}},
xmin=1,
xmax=252,
%xlabel={Time (h)},
every outer y axis line/.append style={black},
every y tick label/.append style={font=\color{black}},
ymin=-1.1,
ymax=1.1,
%ylabel={Normalized PnL},
title={FARO},
axis background/.style={fill=white},
axis x line*=bottom,
axis y line*=left,
yticklabel style={
        /pgf/number format/fixed,
        /pgf/number format/precision=3
},
scaled y ticks=false,
]
\addplot [color=mycolor1,solid,line width=1.5pt,forget plot]
  table[row sep=crcr]{%
1	-0.0171582644099111\\
2	0.00198553845329328\\
3	-0.0918381341353484\\
4	-0.0986434275244136\\
5	-0.10053545379216\\
6	-0.0765475019464891\\
7	-0.0619222370274102\\
8	-0.708629331666439\\
9	-0.0932142611000022\\
10	-0.0941648419858469\\
11	-0.0324252434646838\\
12	-0.0464054629407883\\
13	-0.0459735160777416\\
14	-0.0679795826088153\\
15	-0.0396135878814431\\
16	-0.0274015640561488\\
17	-0.110112508170455\\
18	-0.0207975175600572\\
19	0.00664822964171117\\
20	0.0305409285433767\\
21	0.0124858716902851\\
22	-0.0469516904294046\\
23	-0.035362251491487\\
24	0.00640515053588226\\
25	-0.0564561570035896\\
26	-0.0640620435107844\\
27	-0.0466163810411621\\
28	-0.0636460269021971\\
29	-0.0590713368383734\\
30	-0.010462513018786\\
31	-0.0251229925079348\\
32	-0.0557200425544463\\
33	-0.103341621432614\\
34	-0.0773956403892493\\
35	-0.0365092147868603\\
36	0.0100489072284386\\
37	-0.0890099138389983\\
38	0.0336076766630253\\
39	-0.103949730541853\\
40	-0.9116126651753\\
41	-0.532670609912645\\
42	-0.351849497654309\\
43	0.00645839220614087\\
44	-0.257313121546475\\
45	-0.0635082838868881\\
46	-0.0531319967624851\\
47	0.013153055583196\\
48	-0.111001208924059\\
49	-0.0605525101196774\\
50	-0.016332751886972\\
51	-0.0491571843311811\\
52	0.0223958997868287\\
53	-0.0772677491286859\\
54	-0.00639802756883161\\
55	-0.0646782974058186\\
56	-0.0527780538430975\\
57	-0.0250415322516236\\
58	0.0134292871432345\\
59	-0.015126888078404\\
60	-0.0835960651984423\\
61	-0.086372312057223\\
62	0.00471824400921696\\
63	-0.0839530150416437\\
64	-0.0504905221235172\\
65	-0.00514006616616673\\
66	-0.108323298106884\\
67	-0.0155128701854274\\
68	0.0124478974112718\\
69	0.0141068263254035\\
70	0.00375355550701115\\
71	-0.205160939015571\\
72	-0.0100435249402997\\
73	-0.0200745061305122\\
74	-0.115347831271105\\
75	-0.0237109391031726\\
76	-0.135479522427706\\
77	0.00136345324069474\\
78	-0.0252246426626556\\
79	0.0562445274565356\\
80	-0.060622245960738\\
81	0.0714540967726391\\
82	-0.0357462251970239\\
83	-0.14184563478958\\
84	-0.0288407456551952\\
85	-0.0161809618453014\\
86	-0.00270174938537434\\
87	-0.16710314270281\\
88	-0.344546544008865\\
89	-0.0822310439453183\\
90	0.00184115954419039\\
91	-0.0901620515392069\\
92	-0.02814712928248\\
93	-0.0642351253433988\\
94	-0.0543397269401892\\
95	-0.127588246935\\
96	-0.12962178832341\\
97	-0.0934037887833329\\
98	-0.247396463887623\\
99	-0.0676838074453423\\
100	-0.0339345272701973\\
101	-0.0256718812675493\\
102	0.0142415822097371\\
103	-0.0596969405389897\\
104	-0.127715473920977\\
105	-0.085726574917314\\
106	-0.0483658503841326\\
107	-0.00795737318767066\\
108	-0.193668543730335\\
109	-0.229030841457497\\
110	-0.0687365352216405\\
111	0.0304523542243592\\
112	-0.00877099369150027\\
113	-0.0773206687783234\\
114	-0.0253415848181428\\
115	0.0581194867286839\\
116	-0.0142253085159781\\
117	-0.189137185984256\\
118	-0.0917052152473808\\
119	-0.0779583437011491\\
120	-0.0995032279734559\\
121	0.0263299830333238\\
122	-0.066827365652015\\
123	0.0135215893583277\\
124	0.0432558243268529\\
125	0.05736020086017\\
126	-0.0391372604231726\\
127	nan\\
128	-0.0434157192895887\\
129	-0.0102160832983049\\
130	-0.0592146150877287\\
131	-0.00566252535470318\\
132	-0.0507383647620117\\
133	-0.0287858549492007\\
134	-0.047826363634428\\
135	-0.0805724317409333\\
136	-0.0210173086284954\\
137	-0.0319044331325967\\
138	-0.0304715635181761\\
139	0.0042972322618874\\
140	-0.0101214222501411\\
141	-0.0146749057198633\\
142	-0.0504943830128512\\
143	-0.0645585909497196\\
144	-0.0776329979733335\\
145	-0.0971405013331328\\
146	-0.217125860947905\\
147	-0.0506980931886626\\
148	0.000643074417361048\\
149	-0.0266954743946673\\
150	-0.00488551393582025\\
151	-0.020341720430395\\
152	-0.0600976650527583\\
153	-0.0162281149997243\\
154	-0.00616110101063212\\
155	-0.0339752895706021\\
156	-0.00388022564675624\\
157	-0.0243355058151993\\
158	-0.00227270934430147\\
159	-0.0125325964452041\\
160	-0.0339467567985541\\
161	-0.0430175913719791\\
162	-0.0504876244347099\\
163	-0.0229053235371861\\
164	-0.0178023478383072\\
165	-0.0282115837509881\\
166	-0.000331926056490367\\
167	-0.0021869080185209\\
168	0.0189294409664905\\
169	-0.0808254202616795\\
170	-0.0651728287440223\\
171	-0.0116019260016085\\
172	-0.0333900082557554\\
173	-0.0203953257171169\\
174	-0.0131235963338227\\
175	-0.00557681283044115\\
176	-0.0137813869077116\\
177	-0.0163558795336526\\
178	-0.0233167203181002\\
179	-0.00672705190154837\\
180	-0.22506014020993\\
181	-0.0422140102548055\\
182	-0.111830932994618\\
183	-0.0770119129734678\\
184	-0.033865470656763\\
185	-0.0309516758892761\\
186	0.0017030471483138\\
187	0.00122428991185019\\
188	-0.129030064251804\\
189	-0.146425169717127\\
190	-0.165472509132876\\
191	0.029299834942611\\
192	-0.00806656440187077\\
193	0.00629841165134728\\
194	-0.0305743127447412\\
195	-0.0257328488148134\\
196	0.00520985929153253\\
197	0.0151726383005547\\
198	-0.0117093589421445\\
199	-0.00184518493792325\\
200	-0.0199496013094712\\
201	0.0150463734629994\\
202	-0.0817453192951282\\
203	-0.0455764355931173\\
204	0.0366252996905885\\
205	0.0243659707317058\\
206	0.0404684951656568\\
207	-0.0183992890629788\\
208	-0.0162286922500947\\
209	0.0316530234813035\\
210	-0.0915995685229851\\
211	-0.571450089369681\\
212	-0.522800509713424\\
213	-0.110285731204592\\
214	-0.0636330295621417\\
215	-0.113831964775362\\
216	-0.196734629325433\\
217	-0.0992367600758085\\
218	0.0409386988386661\\
219	-0.160882625082066\\
220	-0.0142251178772514\\
221	0.0141150152601961\\
222	0.0119435929361839\\
223	-0.0521911400526267\\
224	-0.0986690004686093\\
225	-0.00349592015040395\\
226	0.0159080741143208\\
227	0.0323391649223707\\
228	-0.0102717989347697\\
229	-0.0602873068144885\\
230	-0.0431071489260421\\
231	nan\\
232	-0.0530180516963446\\
233	-0.0207328360728421\\
234	-0.0565243675288807\\
235	-0.0192230995284595\\
236	-0.0166270880962157\\
237	-0.0979943053035664\\
238	-0.0169414260554841\\
239	-0.0217924199887439\\
240	-0.000568558703685773\\
241	-0.019621456236181\\
242	-0.129287753892855\\
243	-0.0723576800320373\\
244	-0.00299039452103355\\
245	0.0216702226468757\\
246	-0.0274519273566285\\
247	-0.0862383725984549\\
248	nan\\
249	-0.114382117250106\\
250	-0.189145284673407\\
251	-0.0214157588924525\\
252	-0.00544348336715652\\
};
\addplot [color=mycolor2,solid,line width=1.5pt,forget plot]
  table[row sep=crcr]{%
1	-0.0856203607850693\\
2	-0.0707747680009662\\
3	-0.104141767525771\\
4	-0.199497981634198\\
5	-0.202681998710269\\
6	-0.124272945470926\\
7	-0.0993665627651743\\
8	-0.183401978355264\\
9	-0.0100345814990514\\
10	-0.0351430873842089\\
11	-0.034657838205582\\
12	-0.0239896214341985\\
13	-0.0215512017589147\\
14	-0.0367101149924947\\
15	-0.0224441555157761\\
16	-0.0675490454834694\\
17	-0.108555182425727\\
18	-0.0722263810848897\\
19	-0.00771450276650055\\
20	-0.016270732313766\\
21	-0.0465005274262002\\
22	-0.0650023862255114\\
23	-0.0239450712140652\\
24	-0.0682625897914213\\
25	-0.0289944212941058\\
26	-0.127588160982961\\
27	-0.024696177113977\\
28	-0.0590517227982054\\
29	-0.0550493487370463\\
30	-0.0773557542425015\\
31	-0.0574895482736575\\
32	-0.0376033956692175\\
33	-0.236411619367621\\
34	-0.0415619297297856\\
35	-0.0774282743612988\\
36	-0.00417096847419357\\
37	-0.0584864611637903\\
38	-0.059937989907866\\
39	-0.258529585311911\\
40	-0.913876532859668\\
41	-0.241128583708994\\
42	-0.308320541558404\\
43	0.0221913295446278\\
44	-0.166382789956429\\
45	-0.0746443572646033\\
46	-0.083504386724256\\
47	-0.0280890090267684\\
48	-0.0812589613421778\\
49	-0.116398106578273\\
50	-0.0481030665944524\\
51	-0.0391416075350832\\
52	-0.0545576809466638\\
53	-0.156039215761556\\
54	-0.0678036386983275\\
55	-0.118155941332162\\
56	-0.0735552449067791\\
57	-0.0371177887306459\\
58	-0.0176973207914295\\
59	-0.0263150638058797\\
60	-0.0714150906462965\\
61	-0.0516524626348779\\
62	-0.104627480490001\\
63	0.046317926814774\\
64	-0.0625564482050183\\
65	-0.0418152620529059\\
66	-0.0382930905093672\\
67	-0.0550099913859691\\
68	-0.0618335286862652\\
69	-0.0752666300786891\\
70	-0.0281333125739904\\
71	0.00807330571377673\\
72	-0.0229210637247318\\
73	-0.0669069953283607\\
74	-0.0737350731420096\\
75	-0.0777053928440509\\
76	-0.136486911745524\\
77	-0.0366913682070886\\
78	-0.0698376112609589\\
79	-0.00337707611656408\\
80	-0.114332186553646\\
81	-0.0115305221444353\\
82	-0.115209215970164\\
83	0.439650394556965\\
84	0.00612820399059535\\
85	-0.0615294028012233\\
86	-0.0338466468133894\\
87	-0.136516720514761\\
88	-0.145849664598147\\
89	-0.0566279996114591\\
90	-0.011033454052542\\
91	-0.0779248342818201\\
92	-0.0605750749077891\\
93	-0.00706762663330628\\
94	-0.0475068278957096\\
95	-0.042571755182104\\
96	-0.0620456353807177\\
97	-0.168971234019066\\
98	-0.134466411627154\\
99	-0.107370959617784\\
100	-0.0355136658602324\\
101	-0.0826875458012084\\
102	-0.012497329406777\\
103	-0.0885485171161874\\
104	-0.0784231848579368\\
105	-0.148875945742619\\
106	-0.0540032742362043\\
107	-0.0192702772363494\\
108	-0.100003725668135\\
109	-0.0880473248130928\\
110	-0.0629689384357705\\
111	0.0271013275987113\\
112	0.0267954796082197\\
113	-0.00706077752500883\\
114	0.0128907769675273\\
115	0.079774740717241\\
116	0.00709653171822195\\
117	-0.0348048497337723\\
118	0.0137361248936539\\
119	-0.11069698020023\\
120	-0.0397754309647627\\
121	-0.0439333532540539\\
122	-0.0433208106039542\\
123	0.0108010198370059\\
124	0.0412562754054816\\
125	0.0495524503980965\\
126	-0.0230646480102991\\
127	nan\\
128	-0.0825196433645509\\
129	-0.0426437891756898\\
130	-0.0789593568157475\\
131	0.00171891840661759\\
132	-0.0439384492123908\\
133	-8.64259799509436e-05\\
134	-0.0226745101498006\\
135	-0.0893874893231492\\
136	-0.0649509389243762\\
137	-0.0610984608998195\\
138	0.0215272301610553\\
139	-0.0692698722440269\\
140	-0.0195680422436206\\
141	-0.110076383365166\\
142	-0.0456384545240186\\
143	-0.0149081746520216\\
144	-0.0960757775549577\\
145	-0.00629160189492709\\
146	-0.374685438067206\\
147	-0.0683852020732083\\
148	-0.016805086047098\\
149	-0.0293643955733932\\
150	-0.0109455526971296\\
151	-0.0332981944801438\\
152	-0.0379064416954155\\
153	-0.0279380860146927\\
154	-0.0187273673479832\\
155	-0.0162547540706838\\
156	-0.0878741281253934\\
157	-0.0439976559598474\\
158	-0.0429758734059628\\
159	-0.0221036345308733\\
160	-0.0548862232653352\\
161	-0.0484284016576985\\
162	-0.0522623362336953\\
163	-0.0490208043745991\\
164	-0.0123963327824419\\
165	-0.076196737440846\\
166	-0.0415989240108919\\
167	-0.0225447362634346\\
168	-0.0509916412046076\\
169	-0.117639659966885\\
170	-0.0363932875760041\\
171	-0.026816628718428\\
172	-0.0532919333856116\\
173	-0.0385311885296545\\
174	-0.0239080218807847\\
175	-0.0318015339407009\\
176	-0.0574657445212645\\
177	-0.00979371527250017\\
178	-0.0295716879791661\\
179	-0.0306115437632438\\
180	-0.0965882847989252\\
181	-0.0457428528654177\\
182	-0.00841085885731082\\
183	-0.0585946596159777\\
184	-0.00873880794793503\\
185	0.0272599193223205\\
186	-0.0432600087181072\\
187	-0.0264348076179931\\
188	-0.0652251652492641\\
189	-0.113779963847615\\
190	-0.156487555067705\\
191	-0.0516262787892315\\
192	0.00669748999338609\\
193	-0.0506072739666934\\
194	-0.0952110369833532\\
195	-0.0401777848420842\\
196	-0.0807671958922854\\
197	-0.0381874581366996\\
198	-0.0506746193639166\\
199	-0.0418030336911066\\
200	-0.0798617004081724\\
201	-0.0475251100419537\\
202	-0.0717503092598897\\
203	-0.0878079773248392\\
204	-0.0996163870508437\\
205	-0.0777651969959645\\
206	-0.0190252615980488\\
207	-0.0447288543526097\\
208	-0.0705843737607429\\
209	-0.0248370801208491\\
210	-0.127457303804752\\
211	-0.604418315878085\\
212	-0.291205074120635\\
213	0.0584940114227124\\
214	-0.0681439495242273\\
215	-0.159294342792978\\
216	-0.189538872966937\\
217	-0.0913258339765485\\
218	-0.0990296458364101\\
219	-0.116301064003621\\
220	-0.0966356484637969\\
221	-0.0351315812712979\\
222	-0.0261504556467591\\
223	-0.062852785198048\\
224	-0.170329954761776\\
225	-0.0582206542418083\\
226	0.0350050108017376\\
227	-0.0179826980976183\\
228	-0.0921585953559416\\
229	-0.0573584738895998\\
230	-0.0826404156044176\\
231	nan\\
232	-0.0205025824630308\\
233	-0.010472547699454\\
234	-0.101950489112216\\
235	-0.0766498745566366\\
236	-0.0593906968680247\\
237	-0.0793761003808785\\
238	-0.0560834056017775\\
239	-0.0654466634150648\\
240	-0.0687733104517934\\
241	-0.0449674093725098\\
242	-0.136999348902649\\
243	-0.0990008446675353\\
244	-0.0473733622107678\\
245	-0.0276760569690386\\
246	0.09745752787986\\
247	-0.147801343392415\\
248	nan\\
249	-0.0772661059886298\\
250	-0.0338821430775791\\
251	-0.0930978658052126\\
252	-0.028861778877178\\
};
\addplot [color=mycolor3,solid,line width=1.5pt,forget plot]
  table[row sep=crcr]{%
1	-0.0325100186027445\\
2	0.00499071688286546\\
3	-0.0852566445186826\\
4	-0.128535812985994\\
5	-0.126430414437586\\
6	-0.0756068325374878\\
7	-0.0538770823587646\\
8	-0.531721353957638\\
9	-0.0708225479627843\\
10	-0.119629808342178\\
11	-0.0561264147152898\\
12	-0.0533832179700141\\
13	-0.0498798625465059\\
14	-0.0719390766891468\\
15	-0.0429492169957707\\
16	-0.0382271953511934\\
17	-0.124846453570998\\
18	-0.0487833452673584\\
19	-0.0126515153877701\\
20	-0.00596509339016379\\
21	0.00925214405352224\\
22	-0.0543759280174415\\
23	-0.0246462118601773\\
24	-0.00474002932294398\\
25	-0.0441995006162151\\
26	-0.0547268005867932\\
27	-0.0420869121125324\\
28	-0.0595788926705136\\
29	-0.0632721249170167\\
30	-0.0269424411484572\\
31	-0.0352636697056688\\
32	-0.0591175241909321\\
33	-0.0774958445608846\\
34	-0.0795424262984346\\
35	-0.0480154443853273\\
36	0.00729087441898886\\
37	-0.105513532189232\\
38	0.0104821499505265\\
39	-0.117417090163649\\
40	-1.16087436964866\\
41	-0.450240024031655\\
42	-0.362511252113198\\
43	0.00253674922364051\\
44	-0.232219955047903\\
45	-0.0618511195038171\\
46	-0.0657308906530964\\
47	-0.00154341978978814\\
48	-0.0914535442590734\\
49	-0.0708698213294756\\
50	-0.019896182315092\\
51	-0.0492654348784088\\
52	0.00689967392370656\\
53	-0.0881481112551058\\
54	-0.032251385455512\\
55	-0.0645320177175673\\
56	-0.0556703886598468\\
57	-0.0270177890059823\\
58	0.00681735123120174\\
59	-0.00711459545958231\\
60	-0.0673024944336585\\
61	-0.0983070088896275\\
62	0.00824195941551491\\
63	-0.0820803854091777\\
64	-0.0605120601427456\\
65	-0.0208801293947695\\
66	-0.100844400164873\\
67	-0.0199379004327238\\
68	-0.0197793621417604\\
69	0.0174249481803315\\
70	0.00511681446935318\\
71	-0.207955344653329\\
72	-0.0214743855952195\\
73	-0.0449543766786345\\
74	-0.11663155533071\\
75	-0.0452597783201444\\
76	-0.127120023372661\\
77	0.0134075363327031\\
78	-0.0373553717434991\\
79	0.0322795907227259\\
80	-0.0659634130057415\\
81	0.063873091925299\\
82	-0.0553045738236964\\
83	-0.445143988821365\\
84	-0.0351244628148961\\
85	-0.0337146072651949\\
86	-0.0168501539127626\\
87	-0.168770320946046\\
88	-0.326141157109719\\
89	-0.0711067284362476\\
90	0.025136257067558\\
91	-0.0839147215733757\\
92	-0.0304584294714067\\
93	-0.065847661597966\\
94	-0.0553607015174642\\
95	-0.0964286893116665\\
96	-0.148385653282368\\
97	-0.11250924989955\\
98	-0.243201151236821\\
99	-0.0765028605896402\\
100	-0.0472184640709103\\
101	-0.0285262451235031\\
102	0.0151693078845256\\
103	-0.0600461727505705\\
104	-0.143224829065629\\
105	-0.10240229704266\\
106	-0.0548469475561593\\
107	-0.0173263081257618\\
108	-0.194189114583288\\
109	-0.146668309612629\\
110	-0.0783409137968528\\
111	0.0103246460058126\\
112	-0.0162356167413733\\
113	-0.0742648241775433\\
114	-0.0228181338709292\\
115	0.0260919976537978\\
116	-0.00824281311377484\\
117	-0.161457767517951\\
118	-0.0607936211132628\\
119	-0.0607122690762398\\
120	-0.0809099450505293\\
121	0.0151417891219907\\
122	-0.0582003299236888\\
123	0.00838915937655658\\
124	0.0328128854927279\\
125	0.0774547674302478\\
126	-0.0385892495767562\\
127	nan\\
128	-0.0504536046952919\\
129	-0.0183504982275549\\
130	-0.0545839997487897\\
131	-0.0033753513913728\\
132	-0.0338200504699539\\
133	-0.0304633685792612\\
134	-0.0332670993045126\\
135	-0.0534531211665265\\
136	-0.0444908058306384\\
137	-0.0324152882979398\\
138	-0.0367574902755926\\
139	0.0100006452739671\\
140	-0.0209455992891978\\
141	-0.0449060916024155\\
142	-0.0464680620917089\\
143	-0.0647606324716471\\
144	-0.0698059030936675\\
145	-0.0898400797659559\\
146	-0.193773187382333\\
147	-0.0186641424485681\\
148	0.00149580181566159\\
149	-0.0209297538160674\\
150	-0.00788979612878293\\
151	-0.0235582163011212\\
152	-0.0631215744085287\\
153	-0.0323508828645202\\
154	-0.0091303912188726\\
155	-0.033484960670348\\
156	-0.030396352128917\\
157	-0.0233276584988491\\
158	-0.00118789986850866\\
159	0.00345145667368415\\
160	-0.0345997498649027\\
161	-0.050138520735859\\
162	-0.0305551660095176\\
163	-0.0377359594650799\\
164	-0.0216983067651871\\
165	-0.0147785040235029\\
166	-0.0100912132086347\\
167	-0.000701390630210771\\
168	0.0098851367776351\\
169	-0.0924709707647595\\
170	-0.0563595480316546\\
171	-0.0124263278692122\\
172	-0.0221704130169706\\
173	-0.0221389455308899\\
174	-0.0171248785605318\\
175	-0.0107670654365861\\
176	-0.0125576959180354\\
177	-0.0134986533562417\\
178	-0.0167150236524654\\
179	-0.019728820316897\\
180	-0.153845242829659\\
181	-0.0396045744308492\\
182	-0.121507464052203\\
183	-0.107953671166394\\
184	-0.0237759297794944\\
185	-0.0326840755200716\\
186	-0.024306742888668\\
187	0.00292971611045153\\
188	-0.119092188456988\\
189	-0.15063649222065\\
190	-0.161002339383097\\
191	0.0338426364268917\\
192	-0.0054904172481059\\
193	0.0028448384348453\\
194	-0.0327301587023332\\
195	-0.0301497301114387\\
196	-0.0314348449158078\\
197	-0.00406469927148007\\
198	-0.0218253714970635\\
199	-0.0126471305489269\\
200	-0.0291676838754158\\
201	-0.00197921820880713\\
202	-0.0943460647947407\\
203	-0.0516507457586046\\
204	-0.00527532551066649\\
205	0.0154303085609422\\
206	0.0312347081868182\\
207	-0.0223158646075264\\
208	-0.0257662721930381\\
209	0.0252244562619509\\
210	-0.0743939428939186\\
211	-0.517811023063384\\
212	-0.480860080640541\\
213	-0.14066362713826\\
214	-0.0513391312381968\\
215	-0.110695575262891\\
216	-0.254960425034872\\
217	-0.100422858811898\\
218	0.0465248030180825\\
219	-0.168680130529138\\
220	-0.0225095117080068\\
221	-0.00106343297125548\\
222	0.0050177638227106\\
223	-0.0493416422209153\\
224	-0.0946918654901592\\
225	-0.0125458380134338\\
226	0.0469300862314703\\
227	0.0285000683526303\\
228	-0.00433467588116343\\
229	-0.0477687697144216\\
230	-0.0486051770547526\\
231	nan\\
232	-0.0611896769542466\\
233	-0.0117401651206573\\
234	-0.0466409178607115\\
235	-0.0340311045304937\\
236	-0.014574508177385\\
237	-0.081373367229482\\
238	-0.0179718313667354\\
239	-0.0159782856994813\\
240	-0.00317863495006516\\
241	-0.0356328237653631\\
242	-0.147350844523739\\
243	-0.0648150138084743\\
244	-0.0270338813110732\\
245	0.0129733570280646\\
246	-0.0323933530834971\\
247	-0.0894666557187541\\
248	nan\\
249	-0.111514129059471\\
250	-0.180058701160543\\
251	-0.0566108077699478\\
252	-0.0119279514298396\\
};
\addplot [color=mycolor4,solid,line width=1.5pt,forget plot]
  table[row sep=crcr]{%
1	-0.0772538560126161\\
2	-0.0487663806382359\\
3	-0.0995878386535013\\
4	-0.157039252752378\\
5	-0.155954965253489\\
6	-0.1017185443399\\
7	-0.125639905725187\\
8	-0.181654451310473\\
9	-0.0440025253326762\\
10	-0.00994264585330307\\
11	-0.0123795390126177\\
12	0.0788118452540155\\
13	-0.0435783745467516\\
14	-0.0642452120456839\\
15	-0.0219952026611009\\
16	-0.0474258407491828\\
17	-0.0905389213253349\\
18	-0.0533781055306928\\
19	0.000359855304758824\\
20	-0.0344969867867687\\
21	-0.0401698534279172\\
22	-0.0487567635568958\\
23	-0.0238892468923298\\
24	-0.0666232030207952\\
25	-0.0407890322268669\\
26	-0.104970479938002\\
27	-0.0236147782336466\\
28	-0.0607980424170458\\
29	-0.0555068034283317\\
30	-0.0774287796883188\\
31	-0.049138143932718\\
32	-0.0522260033186819\\
33	-0.189282852132894\\
34	-0.0328192420575641\\
35	-0.103372512784294\\
36	0.00498475053095889\\
37	0.329562491250644\\
38	-0.0323123540531002\\
39	-0.201009233049357\\
40	-0.716131976300465\\
41	-0.278139840653107\\
42	-0.320253212724183\\
43	0.0203601980201811\\
44	-0.172658880891437\\
45	-0.0447067801485141\\
46	-0.0593586765558188\\
47	-0.0149041438446444\\
48	-0.0731028566197628\\
49	-0.101943854933178\\
50	-0.0456367498315522\\
51	-0.0488950430817485\\
52	-0.0550089096320184\\
53	-0.0981461396597604\\
54	-0.0648742991789584\\
55	-0.117944324145923\\
56	-0.150944158551876\\
57	-0.034077891769953\\
58	-0.0550494949217167\\
59	-0.0322821339298258\\
60	0.00401846431606009\\
61	-0.0481181439006338\\
62	0.0133009045122569\\
63	0.0350441784052174\\
64	-0.0593319229365297\\
65	-0.0529940020969334\\
66	-0.0495275013937246\\
67	-0.0423220305753306\\
68	-0.0370759107265927\\
69	-0.0505337942644965\\
70	-0.0314051029149161\\
71	-0.00163753441109454\\
72	-0.0309281751018605\\
73	-0.0474279186149383\\
74	-0.0695188763270507\\
75	-0.0696380205340643\\
76	-0.130975676538113\\
77	-0.0785090684947946\\
78	-0.0804868878131417\\
79	0.00707339171628024\\
80	-0.0960045026717784\\
81	-0.00540077502783183\\
82	-0.0510521702582195\\
83	0.538837573666183\\
84	0.014358075808554\\
85	-0.0525077527322873\\
86	-0.0422992308608391\\
87	-0.12844297456689\\
88	-0.152795929405942\\
89	-0.0661209323204958\\
90	-0.00636353180036207\\
91	-0.0784302682475388\\
92	-0.0724413586266111\\
93	0.00679372805307658\\
94	-0.0359795590314484\\
95	-0.0289246614781889\\
96	-0.0462546345950233\\
97	-0.104294408805889\\
98	-0.195552680959809\\
99	-0.128297371866996\\
100	-0.0586787119142817\\
101	-0.0431536576590735\\
102	-0.00348452136081325\\
103	-0.0760628260968697\\
104	-0.0687245700787149\\
105	-0.131630097177762\\
106	-0.066856432252672\\
107	-0.0764328034015351\\
108	-0.0987506455691217\\
109	-0.0819510835955211\\
110	-0.0639284689433902\\
111	0.0345001595390402\\
112	0.0263698310461612\\
113	0.0243226854861509\\
114	0.00564551992382162\\
115	0.0485078109559253\\
116	-0.000119918449584913\\
117	-0.00890716060067398\\
118	-0.0171328596110289\\
119	-0.137819303108701\\
120	-0.0535083583698354\\
121	-0.0158429075775827\\
122	-0.0543100878572725\\
123	0.0240170738000622\\
124	0.00245797975176231\\
125	0.0424685835361417\\
126	-0.0270946795989594\\
127	nan\\
128	-0.0884490246860671\\
129	-0.01979352205602\\
130	-0.0602209425028455\\
131	-0.047390182858586\\
132	-0.00763040618558268\\
133	-0.0183942120197465\\
134	-0.018524605807832\\
135	-0.101818549201356\\
136	-0.0441827685465075\\
137	-0.0603019168075178\\
138	0.0106773139402215\\
139	-0.060134771308082\\
140	-0.0180420801104241\\
141	-0.0344071960710348\\
142	-0.0364838995842449\\
143	-0.0248453006625892\\
144	-0.0469324765491856\\
145	-0.0126376129666439\\
146	-0.325455703332194\\
147	-0.0974231984998782\\
148	-0.0209135001537699\\
149	-0.0228853857280722\\
150	-0.0111692977900301\\
151	-0.0272122272873205\\
152	-0.0355248017379502\\
153	-0.00719711559512561\\
154	-0.0147237214070352\\
155	-0.0194375470338417\\
156	-0.0846836984499125\\
157	-0.0547371498014214\\
158	-0.0451454071552599\\
159	-0.0321358202410631\\
160	-0.035721366820431\\
161	-0.0366210025217901\\
162	-0.043417480992881\\
163	-0.0715093319742274\\
164	-0.0105724490039661\\
165	-0.0640378778906103\\
166	-0.0070971258966624\\
167	-0.0102372292760136\\
168	-0.0668849024409976\\
169	-0.112004510426872\\
170	-0.0550437927387341\\
171	-0.0113708820647503\\
172	-0.044096804374606\\
173	-0.0414595416991836\\
174	0.00436535979800067\\
175	-0.0259001975317621\\
176	-0.038506748418225\\
177	-0.0197660370139826\\
178	-0.0409584252365702\\
179	-0.022123631327268\\
180	-0.0590341745343367\\
181	-0.0407592269820797\\
182	-0.0178894107077691\\
183	-0.0570585816082112\\
184	-0.0157515388879908\\
185	-0.0507842872328195\\
186	-0.0508017650246672\\
187	-0.0274600610542058\\
188	-0.11448340349484\\
189	-0.126498340267413\\
190	-0.166891557063536\\
191	-0.0624470421368618\\
192	-0.00312345797821631\\
193	-0.0365341283977062\\
194	-0.0864226769558292\\
195	-0.033297829277073\\
196	-0.0931575860883022\\
197	-0.0180807039183585\\
198	-0.0384201418606434\\
199	-0.0503471590960953\\
200	-0.0688841809215643\\
201	-0.0421318421198584\\
202	-0.0581141246155688\\
203	-0.0938432452427879\\
204	-0.0628536425622035\\
205	-0.0634822056026895\\
206	-0.0185695161655531\\
207	-0.0437535387441696\\
208	-0.0529967823525217\\
209	-0.0332262021893026\\
210	-0.0843441610302172\\
211	-0.448453757861155\\
212	-0.480672263477085\\
213	-0.10154146855894\\
214	-0.0901049318841872\\
215	-0.161506065537761\\
216	-0.196046565322263\\
217	-0.12929137635563\\
218	-0.0935565777276325\\
219	-0.123092636211245\\
220	-0.109638109821853\\
221	-0.0383269909900374\\
222	-0.00518380287887244\\
223	-0.0670863506228746\\
224	-0.150065305203484\\
225	-0.0687504059741254\\
226	-0.0359091376169781\\
227	-0.0396195571702265\\
228	-0.0835084658962133\\
229	-0.0609935393359201\\
230	-0.104173308186193\\
231	nan\\
232	-0.0285122764177975\\
233	-0.041851454588042\\
234	-0.113780873715656\\
235	-0.0997152866933102\\
236	-0.0772546548995375\\
237	-0.101285386367683\\
238	-0.0520203611978263\\
239	-0.0719311313284583\\
240	-0.0628166253878295\\
241	-0.0371527205573437\\
242	-0.0419428482393087\\
243	-0.0846577230990247\\
244	-0.0301119457497231\\
245	-0.0417833931591006\\
246	0.103713190789445\\
247	-0.166255659948543\\
248	nan\\
249	-0.0710819768332999\\
250	-0.0267233799613964\\
251	-0.0886930899149551\\
252	-0.055034461224752\\
};
\end{axis}
\end{tikzpicture}%

\end{subfigure}%
\hfill%
\begin{subfigure}{.45\linewidth}
  \centering
  \setlength\figureheight{\linewidth} 
  \setlength\figurewidth{\linewidth}
  \tikzsetnextfilename{IS_sameday_NTAP}
  % This file was created by matlab2tikz.
%
%The latest updates can be retrieved from
%  http://www.mathworks.com/matlabcentral/fileexchange/22022-matlab2tikz-matlab2tikz
%where you can also make suggestions and rate matlab2tikz.
%
\definecolor{mycolor1}{rgb}{0.25098,0.00000,0.38824}%
\definecolor{mycolor2}{rgb}{0.00000,0.46275,0.00000}%
\definecolor{mycolor3}{rgb}{0.00000,0.34902,0.34902}%
\definecolor{mycolor4}{rgb}{0.58039,0.26275,0.00000}%
%
\begin{tikzpicture}[trim axis left, trim axis right]

\begin{axis}[%
width=\figurewidth,
height=\figureheight,
at={(0\figurewidth,0\figureheight)},
scale only axis,
every outer x axis line/.append style={black},
every x tick label/.append style={font=\color{black}},
xmin=1,
xmax=252,
%xlabel={Time (h)},
every outer y axis line/.append style={black},
every y tick label/.append style={font=\color{black}},
ymin=-1.1,
ymax=1.1,
%ylabel={Normalized PnL},
title={NTAP},
axis background/.style={fill=white},
axis x line*=bottom,
axis y line*=left,
yticklabel style={
        /pgf/number format/fixed,
        /pgf/number format/precision=3
},
scaled y ticks=false,
]
\addplot [color=mycolor1,solid,line width=1.5pt,forget plot]
  table[row sep=crcr]{%
1	-0.146334170010575\\
2	-0.144423708590969\\
3	-0.113628960428223\\
4	-0.111609761510288\\
5	-0.110024540277312\\
6	0.140994376775195\\
7	-0.0332346473748883\\
8	0.0296803479021415\\
9	-0.118898954471748\\
10	-0.190455353230526\\
11	0.0454943820496442\\
12	0.103455191170325\\
13	-0.0311975823263387\\
14	-0.0822526439771115\\
15	-0.205542229063115\\
16	0.0420613010293632\\
17	0.0581810057255279\\
18	0.0934843564082873\\
19	-0.10407702781347\\
20	0.0495365578162272\\
21	-0.0651968426695856\\
22	-0.0362594470922687\\
23	0.0295027433549918\\
24	0.0402863872513736\\
25	-0.0304396733253954\\
26	0.0417105612437457\\
27	-0.0757320937702446\\
28	-0.133328387690426\\
29	-0.0105532138988593\\
30	0.00734746971436523\\
31	-0.271014263488796\\
32	-0.0225360665226154\\
33	0.0270826314618481\\
34	-0.11036823607362\\
35	-0.0669363295449461\\
36	-0.0684361096624793\\
37	-0.0726888900256057\\
38	-0.152911284350321\\
39	0.0115639181478459\\
40	0.0391571897768147\\
41	-0.0276917904604929\\
42	-0.0683789280157995\\
43	-0.00530031523669075\\
44	0.0219871415084311\\
45	-0.0129985277813901\\
46	-0.0401921637228632\\
47	0.0643914125935195\\
48	0.043394485507103\\
49	-0.0711739964749996\\
50	0.110457599167972\\
51	0.0100403529626161\\
52	-0.000832245014777877\\
53	0.0424172810391523\\
54	0.0434608810096843\\
55	-0.0443678189886408\\
56	0.111106607577829\\
57	0.119456010136082\\
58	-0.0320339589783253\\
59	0.0918476493070952\\
60	0.0686293798300882\\
61	0.0759622015320213\\
62	-0.0542835635266948\\
63	0.0686233034746546\\
64	-0.0118677391997202\\
65	-0.0113275428555239\\
66	0.0444465804486523\\
67	0.14811082690168\\
68	-0.0192722667572909\\
69	0.124365938609198\\
70	-0.441102148058369\\
71	-0.117697471659442\\
72	-0.0294122088199094\\
73	-0.233151140569622\\
74	-0.0593052258912813\\
75	0.020792681529729\\
76	0.0711701663125001\\
77	0.0283796709671788\\
78	0.160795073761954\\
79	-0.124474251962115\\
80	-0.17755262720032\\
81	0.0611409103664031\\
82	0.0809662461599879\\
83	-0.21441705287873\\
84	-0.00730563552259109\\
85	0.0570135157456549\\
86	0.0723259059553574\\
87	0.0258750433987791\\
88	-0.0925193045532482\\
89	0.015324672857035\\
90	0.0799573384106038\\
91	-0.231020564515516\\
92	0.0470950719066619\\
93	0.0388064634304863\\
94	0.128481093339731\\
95	0.183622826299882\\
96	0.0255416421079278\\
97	0.0711187932026343\\
98	0.208882847356938\\
99	0.0696713353354581\\
100	0.0718035162559606\\
101	0.0102230241949068\\
102	0.0365797660163733\\
103	-0.103865110790456\\
104	0.030650033039141\\
105	0.0410558602789736\\
106	-0.309681077456261\\
107	-0.0119963771443947\\
108	0.0738055865058521\\
109	-0.0638892439528518\\
110	0.0355662508557962\\
111	0.00149083921598245\\
112	0.034974244210311\\
113	-0.0145454016341785\\
114	-0.0337232241322727\\
115	0.00814165644218772\\
116	0.0973219229186651\\
117	-0.104761297194103\\
118	-0.419237465072863\\
119	-0.0523189998296781\\
120	-0.0768411737612388\\
121	-0.139236977502743\\
122	0.0276395965518297\\
123	-0.0140285426197914\\
124	-0.0806214971063888\\
125	0.0263867511605079\\
126	0.0460580304566267\\
127	nan\\
128	-0.173811622638303\\
129	-0.10974086476214\\
130	-0.0507990945628054\\
131	-0.0728025878038892\\
132	0.0222118388324187\\
133	0.0363185993848166\\
134	0.0734376842153292\\
135	0.0336667922929843\\
136	-0.00991920319171624\\
137	0.00937863012335889\\
138	-0.00761756646769313\\
139	0.0430429318019514\\
140	-0.0306077288933996\\
141	-0.419941923197977\\
142	-0.057246909240319\\
143	-0.0384410902577501\\
144	0.0883684527555557\\
145	0.0111990845774435\\
146	0.0108542440234786\\
147	-0.0738308480389174\\
148	0.0295952773286623\\
149	0.00673140392301664\\
150	-0.0167476548085527\\
151	-0.00477607857846891\\
152	0.123967067201216\\
153	0.051698844506967\\
154	0.0381213374465553\\
155	0.020716777358153\\
156	0.0235163562779327\\
157	0.215470792206479\\
158	0.0559155521377211\\
159	0.036605144719932\\
160	-0.039776044782485\\
161	-0.0665321773332405\\
162	-0.0700413707222049\\
163	0.00978652665770475\\
164	0.0436755401320482\\
165	0.0228987415853389\\
166	-0.0154884600655549\\
167	0.0184169731807707\\
168	-0.0401208134678286\\
169	0.0411759538028637\\
170	-0.0237934135621796\\
171	-0.0123871426362652\\
172	0.0739620296686734\\
173	0.0195615906981918\\
174	0.0651260876811813\\
175	-0.0303899452706762\\
176	0.0436287452408109\\
177	0.0191688803649383\\
178	0.00840355103418225\\
179	0.0958795337938052\\
180	0.0140446658526974\\
181	0.0676158648691198\\
182	0.00640791975164628\\
183	0.0303009828790512\\
184	-0.0994653235826126\\
185	0.0712246515022809\\
186	-0.0808762441478835\\
187	0.0155932579789001\\
188	-0.00931717780727907\\
189	0.0150731451006184\\
190	0.0188591527404255\\
191	0.00968067939096548\\
192	0.100237421294454\\
193	0.0525604985243461\\
194	0.000638530076900374\\
195	0.0986389076436885\\
196	0.0914807345213507\\
197	0.125061157001554\\
198	-0.016015464386295\\
199	0.0510021451385751\\
200	-0.0258011756544677\\
201	0.0459394562766339\\
202	-0.00941871633395888\\
203	-7.55520003988876e-05\\
204	-0.0901083796304973\\
205	-0.0935858887981964\\
206	-0.0177398883251711\\
207	0.00706535085873735\\
208	0.0371564549558536\\
209	0.00215051615313052\\
210	0.0318752027796627\\
211	0.105000940332606\\
212	0.0197618518867625\\
213	0.0936925777053488\\
214	0.0776326682904282\\
215	0.129981106334259\\
216	0.104501994143678\\
217	0.125550090491769\\
218	0.0355674462123794\\
219	0.0533954104517503\\
220	0.0139076815454674\\
221	-0.0810052530656197\\
222	0.0287692464938397\\
223	0.0261627527076821\\
224	-0.0655168760210977\\
225	-0.0508405112929092\\
226	-0.0675880657603052\\
227	-0.00671118307397814\\
228	0.0434520149266367\\
229	0.0612387726738441\\
230	0.00569900039774141\\
231	nan\\
232	0.0655467788499101\\
233	-0.0196756770261993\\
234	-0.0418391737939853\\
235	0.0736224743590198\\
236	0.0216778620711751\\
237	-0.00190011317706467\\
238	0.0537005389197896\\
239	0.0242413869476011\\
240	-0.113996554880806\\
241	-0.169723127166857\\
242	0.0409965470764547\\
243	-0.0171390619206672\\
244	-0.295443260283733\\
245	-0.0941032767048099\\
246	-0.0270864439245175\\
247	0.00205563752126393\\
248	nan\\
249	0.0385430468655989\\
250	0.0458987125601525\\
251	0.0291309838401741\\
252	-0.0863508709436403\\
};
\addplot [color=mycolor2,solid,line width=1.5pt,forget plot]
  table[row sep=crcr]{%
1	0.0609275984551955\\
2	0.0950875034786651\\
3	-0.0171970693855623\\
4	0.0287749011815951\\
5	-0.00283688707041341\\
6	0.0555567487150617\\
7	0.0392869024075223\\
8	0.139976500341321\\
9	0.0906270023399637\\
10	0.0788664216530345\\
11	0.186313902380164\\
12	0.118506059519284\\
13	0.0301503457485284\\
14	0.13810005370944\\
15	0.0149860502373101\\
16	0.242167005827668\\
17	0.270649058113191\\
18	0.00152634842132761\\
19	-0.0424654509653028\\
20	0.0878302300898151\\
21	0.102956420640893\\
22	0.0170364926120804\\
23	-0.0534108804068856\\
24	0.114603615277228\\
25	0.117790824239874\\
26	0.072921939554139\\
27	0.0822530911206107\\
28	-0.0804882784749721\\
29	0.00107511764534981\\
30	0.239778491201471\\
31	0.224803502576105\\
32	0.113306746069711\\
33	0.0774034777100572\\
34	0.168385496479386\\
35	-0.033864300218401\\
36	-0.0411067939793546\\
37	-0.115648386636437\\
38	-0.0329113001965352\\
39	0.28292061161448\\
40	-0.0595391674234815\\
41	0.12648111600364\\
42	-0.00514071295008419\\
43	0.00396858621873165\\
44	0.111405936137243\\
45	-0.0209933476646561\\
46	0.00168009156635855\\
47	0.147158729873515\\
48	0.129572716471885\\
49	0.0548857562411508\\
50	0.165286372348446\\
51	0.0460633080491916\\
52	0.00742233808868739\\
53	0.0833581977307227\\
54	0.055425800453132\\
55	0.0390369778020091\\
56	0.0833721716900087\\
57	-0.0147107927370948\\
58	-0.0769346690440123\\
59	0.128980552397636\\
60	0.124655826947206\\
61	0.1306429932187\\
62	0.282038950133202\\
63	0.0417678517697073\\
64	0.00858961249779559\\
65	0.083873558429835\\
66	0.0802231534094313\\
67	0.140531047212986\\
68	0.150242012120184\\
69	0.241449521372468\\
70	0.270482989442111\\
71	0.0285657836968452\\
72	0.481369378366072\\
73	-0.104948428312733\\
74	-0.0541622011243189\\
75	0.22844345369596\\
76	0.399036238060357\\
77	0.292317909222674\\
78	0.247558173713441\\
79	-0.0306423072126751\\
80	0.298193027534852\\
81	0.0224741761338023\\
82	0.229317204467132\\
83	-0.107515207874432\\
84	0.0684554058759941\\
85	0.375037941168099\\
86	0.322164310817521\\
87	0.0939390734250137\\
88	0.237971591805726\\
89	0.0878030899202089\\
90	0.209256294831725\\
91	0.0750197434729755\\
92	0.0799208917754899\\
93	0.42989919425326\\
94	1.04223704402971\\
95	0.0134742073558479\\
96	0.0496027838228986\\
97	0.138381946924433\\
98	0.0790603777575076\\
99	0.0687445519884812\\
100	0.130697967786828\\
101	0.497451906971097\\
102	0.182296379355635\\
103	0.156338045993842\\
104	0.0965748238084229\\
105	0.0332511495905237\\
106	0.235527779582819\\
107	0.0940473071544629\\
108	0.217892921736545\\
109	0.0778135784284177\\
110	0.0928866581190771\\
111	0.116468856123083\\
112	0.0289182169645737\\
113	0.121005161796437\\
114	0.0156199141356453\\
115	0.100355176029493\\
116	0.19877707470414\\
117	0.00186972317330306\\
118	-0.0197838231672022\\
119	-0.104695465828348\\
120	0.0609763712451391\\
121	-0.0154013977547844\\
122	0.0830393037212593\\
123	0.044295299264677\\
124	0.0512542626286886\\
125	0.0633694787218436\\
126	0.327088551393902\\
127	nan\\
128	0.108762703693394\\
129	-0.0331187979584362\\
130	0.0161605783302232\\
131	0.0436341106679181\\
132	0.104394809531516\\
133	0.0786056495668824\\
134	0.101685868955278\\
135	0.0583805526960136\\
136	0.11990653429152\\
137	0.0611890641710924\\
138	0.128994225423853\\
139	0.163377760093438\\
140	0.30035296646803\\
141	0.0101577972613104\\
142	0.158542544930491\\
143	0.0585802179849182\\
144	0.0359168149655003\\
145	0.0561347604222076\\
146	0.0238099775565552\\
147	0.0590474125612155\\
148	0.042548754582079\\
149	0.157120351355349\\
150	0.0423260278722799\\
151	0.156414997167463\\
152	0.301620368641878\\
153	0.108746644996181\\
154	0.0329985337140441\\
155	0.138400611124417\\
156	0.0820424090692136\\
157	0.888417419446388\\
158	0.153773940083529\\
159	0.0943990042580011\\
160	0.0385683948458663\\
161	-0.0279894572048993\\
162	0.0760980130710236\\
163	-0.00957367808115706\\
164	0.0120820551230113\\
165	0.116586717969787\\
166	-0.12553876341692\\
167	0.0602045171665562\\
168	0.0182677827927335\\
169	0.142795875255276\\
170	0.0985693987018073\\
171	0.048321731189936\\
172	0.193646467807197\\
173	0.0808444492424373\\
174	0.0746833072657553\\
175	0.115658246323195\\
176	0.119148065788819\\
177	0.0863380504620934\\
178	0.00211501188468592\\
179	0.0735366365524754\\
180	0.0294108039856393\\
181	0.0690322315072054\\
182	0.00580010499653321\\
183	0.0940971452637962\\
184	0.0449619567982585\\
185	0.072686997432567\\
186	0.0683895235658366\\
187	0.00181437511091313\\
188	0.0659647441695265\\
189	0.0824313402119695\\
190	0.0926905965924258\\
191	0.0472561385423525\\
192	0.116962909059665\\
193	0.190410798314783\\
194	0.024985022164049\\
195	0.0628646265311809\\
196	0.0744729573793952\\
197	0.1005908398202\\
198	0.028693186001653\\
199	0.0656473870289947\\
200	0.027186821758811\\
201	0.272177789269849\\
202	0.106525902245672\\
203	0.0214425359878095\\
204	-0.0463844875978321\\
205	-0.0536553946432129\\
206	0.065712399236653\\
207	0.0351628287713756\\
208	0.144250905882125\\
209	0.0386007555359326\\
210	-0.0438214774575132\\
211	0.20440423781531\\
212	0.120078231799797\\
213	0.136132529038807\\
214	0.116124524754091\\
215	0.211140347605339\\
216	0.31695666981137\\
217	0.188475890604656\\
218	0.128809959505246\\
219	0.261604502121207\\
220	0.272955701250515\\
221	-0.0599373656583528\\
222	0.159493039250562\\
223	0.00512007204121944\\
224	0.0838935562615367\\
225	0.095609998226938\\
226	0.0746528434361931\\
227	0.103900449944385\\
228	0.0697632489517808\\
229	0.0763906594030649\\
230	0.222812154117533\\
231	nan\\
232	0.107424496536352\\
233	0.0322100952873769\\
234	0.214996356656436\\
235	0.304441427968825\\
236	0.0669956127574658\\
237	0.0349308505738761\\
238	0.120862133678185\\
239	-0.0440005338920748\\
240	0.00892630345809987\\
241	-0.00535660601520037\\
242	0.0545656999881955\\
243	0.0702386612435167\\
244	0.0112258666516098\\
245	0.00524987508684319\\
246	-0.00861602522280953\\
247	0.0395295871695247\\
248	nan\\
249	0.0258054019817371\\
250	0.0867563047498356\\
251	0.105287084366041\\
252	-0.00922145539284379\\
};
\addplot [color=mycolor3,solid,line width=1.5pt,forget plot]
  table[row sep=crcr]{%
1	-0.221897345898991\\
2	-0.160668840235624\\
3	-0.279885385958042\\
4	-0.458021461871051\\
5	-0.427844593226991\\
6	0.0140876063628836\\
7	-0.206510527558516\\
8	-0.206428300917142\\
9	-0.223462578915298\\
10	-0.052753174298243\\
11	-0.141998968194952\\
12	0.0172065072920903\\
13	-0.122288994152593\\
14	0.0675137017402154\\
15	-0.105228543201608\\
16	-0.0925332175132246\\
17	0.00302070982548691\\
18	-0.174930786261241\\
19	0.106073819881101\\
20	-0.251512509707802\\
21	-0.114172592314085\\
22	-0.114501590715144\\
23	-0.183735921700641\\
24	-0.0809647374496597\\
25	-0.109233311446707\\
26	-0.172859308385964\\
27	-0.273959351585184\\
28	-0.240582979518584\\
29	-0.0236744820109217\\
30	-0.023961526511399\\
31	-0.371439868264196\\
32	-0.316771831149095\\
33	-0.0975116151337195\\
34	-0.288280975634245\\
35	-0.486296309861344\\
36	-0.380735790209438\\
37	-0.359507129081605\\
38	-0.313340173346978\\
39	-0.366952929246599\\
40	-0.22360468596119\\
41	-0.347311121844036\\
42	-0.269759877867614\\
43	-0.312871137904917\\
44	-0.193453502683584\\
45	-0.273410230700954\\
46	-0.12153311108687\\
47	-0.0972222878148076\\
48	-0.26496222824649\\
49	-0.393187912229119\\
50	-0.062924003325565\\
51	-0.229454618256752\\
52	-0.104693817552017\\
53	-0.262146006499089\\
54	-0.25145298189975\\
55	-0.24991900942359\\
56	-0.0746912866368648\\
57	-0.192475187142345\\
58	-0.204338770921482\\
59	-0.11177724124841\\
60	-0.153459353138629\\
61	-0.112876080573347\\
62	-0.324310870083684\\
63	-0.260881071860624\\
64	-0.242199610896453\\
65	-0.469899873913509\\
66	-0.14149280795039\\
67	-0.0995159235285039\\
68	-0.496832201991141\\
69	-0.20459102303056\\
70	-0.934599634568009\\
71	-0.497859637771278\\
72	-0.13596962129122\\
73	-0.559296072963275\\
74	-0.446300263861607\\
75	-0.308132504226949\\
76	-0.444610482037457\\
77	-0.544940010793985\\
78	-0.210195372656718\\
79	-0.286059975944909\\
80	-0.240496423803031\\
81	-0.0105768395022788\\
82	0.243881297539726\\
83	-0.146046194126093\\
84	-0.146196977012992\\
85	0.049753090356078\\
86	-0.286069734494076\\
87	-0.209887385794233\\
88	-0.532147433899234\\
89	-0.206428929429583\\
90	-0.14899877699009\\
91	-0.383165747682687\\
92	-0.257973289124885\\
93	0.0138185561852694\\
94	0.124092915540727\\
95	-0.261134851097768\\
96	-0.126134310301268\\
97	-0.185584770803984\\
98	-0.118268909954323\\
99	-0.154238864464098\\
100	0.029679930759376\\
101	-0.0046397654097486\\
102	-0.0913352042124955\\
103	-0.176450320984024\\
104	-0.0896893281668327\\
105	-0.1452356479199\\
106	-0.337174360678847\\
107	-0.0917525093738904\\
108	0.0253060404451461\\
109	-0.0495270858659681\\
110	-0.0284976617586047\\
111	-0.0888942319517275\\
112	-0.0360799734455534\\
113	-0.210576002029873\\
114	-0.140481119340807\\
115	-0.169065717275663\\
116	-0.0152268517606645\\
117	-0.332567104532107\\
118	-0.451082739930509\\
119	-0.397449048302414\\
120	-0.158616514672985\\
121	-0.342908024729248\\
122	-0.0434280932827393\\
123	-0.13446557111096\\
124	-0.312286159419527\\
125	-0.141320499143886\\
126	-0.0960068374850642\\
127	nan\\
128	-0.198018470670532\\
129	-0.207128686273796\\
130	-0.217747794475989\\
131	-0.245912579874692\\
132	0.0118144657748489\\
133	-0.0223462292829573\\
134	-0.112271478697404\\
135	-0.140456787743348\\
136	-0.0755598349197332\\
137	-0.246919679826006\\
138	-0.120572130405933\\
139	-0.0942241508565454\\
140	-0.0798148456920532\\
141	-0.227994612582025\\
142	-0.0745363501705154\\
143	-0.142023602627177\\
144	-0.009435055148019\\
145	-0.0707421247137581\\
146	-0.0850130486290516\\
147	-0.157320870299976\\
148	-0.142833692541075\\
149	-0.0141103706127006\\
150	-0.0788826816709233\\
151	-0.050786100929248\\
152	0.0809644518103945\\
153	-0.033335789331376\\
154	-0.0417381893153845\\
155	-0.0435194348387565\\
156	-0.182188324290648\\
157	0.147776634812953\\
158	-0.0362198988986458\\
159	0.0159161737390447\\
160	-0.0575344958920988\\
161	-0.230790098411685\\
162	-0.310720241373073\\
163	-0.130926207886537\\
164	0.0192152543262268\\
165	-0.0718739031353337\\
166	-0.16648021256982\\
167	-0.0778439010531418\\
168	-0.120067847069719\\
169	0.0138877607599651\\
170	-0.00310585985525876\\
171	-0.0372699641512008\\
172	-0.0440758804487057\\
173	-0.103906360103764\\
174	-0.0201963104618272\\
175	-0.0476365110943458\\
176	-0.0326325062188797\\
177	-0.0362745844317767\\
178	-0.00948586745426994\\
179	-0.0173009186567767\\
180	-0.122218548957123\\
181	0.00952649933091816\\
182	-0.189100955681924\\
183	-0.0488301695343668\\
184	-0.27331411928271\\
185	0.014679390592631\\
186	-0.144924022471503\\
187	-0.0760423226990721\\
188	-0.0859954592593946\\
189	-0.100959083711751\\
190	0.0145750868249861\\
191	-0.231798650348902\\
192	0.0639288739717163\\
193	-0.0901664453825316\\
194	-0.083539464937944\\
195	0.0470877271830689\\
196	-0.0507480920331644\\
197	0.00155213034442048\\
198	-0.189549396383355\\
199	-0.286625395390046\\
200	-0.128150457053247\\
201	-0.0412592839902325\\
202	-0.162549974568938\\
203	-0.031943905765726\\
204	-0.29773625833293\\
205	-0.325346780007507\\
206	-0.168221412130828\\
207	-0.0633227723266867\\
208	-0.117001835589108\\
209	-0.134381106726437\\
210	-0.0433483297145184\\
211	-0.0003632309699483\\
212	-0.0178307888589857\\
213	-0.114834280804606\\
214	0.0170399694552865\\
215	0.0270737923937779\\
216	-0.0959699037638495\\
217	-0.0694238716907414\\
218	-0.0161833311756387\\
219	-0.00507590309371556\\
220	-0.321434704125238\\
221	-0.265021543448214\\
222	-0.13978849243942\\
223	-0.0884519572481613\\
224	0.00173112514018595\\
225	-0.0909880378550325\\
226	-0.120763393679595\\
227	-0.0553755553259604\\
228	0.0353365240750126\\
229	0.0442624421193745\\
230	-0.0453609808381064\\
231	nan\\
232	-0.0439458818892116\\
233	-0.098423329001228\\
234	-0.162686460592699\\
235	0.0381143710994553\\
236	-0.0107080146672956\\
237	-0.0075731188465051\\
238	-0.0689378157327155\\
239	-0.0963057206277922\\
240	-0.283297652264101\\
241	-0.0672891405026618\\
242	0.000480376018287777\\
243	0.0619812667574372\\
244	0.198345472450602\\
245	0.0369578240377158\\
246	-0.111960725821409\\
247	0.0308970637992113\\
248	nan\\
249	0.00332902375952278\\
250	-0.0324401927198416\\
251	-0.00282069915454051\\
252	-0.10009411415034\\
};
\addplot [color=mycolor4,solid,line width=1.5pt,forget plot]
  table[row sep=crcr]{%
1	0.0776663767348847\\
2	0.0907141560058189\\
3	0.0454369998667038\\
4	0.184901366138002\\
5	0.10688206566116\\
6	0.31861578912269\\
7	0.0729730451922033\\
8	0.148629079977738\\
9	0.0807282085447651\\
10	0.112636765259078\\
11	-0.0729304265383075\\
12	0.143314519850656\\
13	0.0275546492612374\\
14	-0.036888453046952\\
15	0.0986327089522674\\
16	0.320057627928247\\
17	0.0311923977844148\\
18	0.198892346480778\\
19	-0.0125696648755947\\
20	0.105798659545416\\
21	0.0926620507902995\\
22	0.0549343229868392\\
23	0.168669085912424\\
24	0.0513615268880323\\
25	0.0567100336967936\\
26	0.0564280439319978\\
27	0.0821073974134711\\
28	0.0263977618174404\\
29	0.295885724292233\\
30	0.222209014156316\\
31	0.301671642633316\\
32	0.140113138300991\\
33	0.0666840027423222\\
34	0.269240433904748\\
35	0.157232494578222\\
36	0.00842305132795561\\
37	-0.13888135572351\\
38	0.0137876048954372\\
39	0.213049439032533\\
40	0.238143798649001\\
41	0.198060687778523\\
42	0.125590684203624\\
43	0.0424317611256796\\
44	0.108244275294955\\
45	0.142257591144978\\
46	0.0141186558128219\\
47	0.180176323935343\\
48	0.143820449404543\\
49	0.0349572248022134\\
50	0.217018598220828\\
51	0.0520519094901039\\
52	0.0342343908327333\\
53	0.218563665085742\\
54	0.152368891881181\\
55	0.0905876912997299\\
56	0.202451042500571\\
57	0.00752546296332642\\
58	0.0106105911060552\\
59	0.0103785619107652\\
60	0.157718186052121\\
61	0.171582539710896\\
62	0.370976143391494\\
63	0.175642975919649\\
64	0.0832387705566563\\
65	0.164668264224106\\
66	0.13707108823537\\
67	0.171821406704652\\
68	0.147935631735202\\
69	0.248181447570254\\
70	0.317021450067863\\
71	0.0848957242011071\\
72	0.540308030531828\\
73	-0.103522737041812\\
74	-0.0576962771897293\\
75	0.165189562522402\\
76	0.484005198570207\\
77	0.322791469160326\\
78	0.206461972664454\\
79	-0.0456141109591401\\
80	0.262991652579518\\
81	0.210455226196022\\
82	0.250106894140199\\
83	-0.13530868818557\\
84	0.0444441174315221\\
85	0.394214892109759\\
86	0.281215859942133\\
87	0.109748227189895\\
88	0.26364162623539\\
89	0.0587500460069281\\
90	0.209028617764171\\
91	0.0462243886058061\\
92	0.170370792820705\\
93	0.00916657634758967\\
94	0.962469782420417\\
95	0.708356047246591\\
96	0.171301391695062\\
97	0.0933137526503772\\
98	0.0159928971600543\\
99	0.441869217809132\\
100	0.0969920439870136\\
101	0.510370000209713\\
102	0.183711212036224\\
103	0.127959093365516\\
104	0.0482598455376544\\
105	0.101972072136078\\
106	0.126779293705925\\
107	0.0578144747682026\\
108	0.196216567530989\\
109	0.0348716120292757\\
110	0.0810212361258562\\
111	0.0227812853216696\\
112	-0.00790898768559478\\
113	0.0579328454514521\\
114	-0.0231021320139937\\
115	-0.0262234009538987\\
116	0.194646699024379\\
117	0.148702372396008\\
118	-0.0532389917708331\\
119	0.132590700824369\\
120	0.063995444079536\\
121	-0.0177346696411575\\
122	0.0952258138681367\\
123	0.0818859198686564\\
124	0.0515988720263589\\
125	0.0886858968332003\\
126	0.277491549324952\\
127	nan\\
128	0.0818727093008569\\
129	0.00344747290233926\\
130	0.0680292829861298\\
131	0.0560777409972058\\
132	0.0303767642564203\\
133	0.0722841731122094\\
134	0.163692085465261\\
135	0.0789787312394804\\
136	0.132075240415809\\
137	-0.0588214340247697\\
138	0.155823820197212\\
139	-0.0496488223140521\\
140	0.244844290147389\\
141	-0.0635252575949201\\
142	0.0771504696893602\\
143	0.0858840466685321\\
144	0.0581597249188646\\
145	0.0840240650747702\\
146	0.0562755164613257\\
147	0.0770581420101036\\
148	0.0620026264832715\\
149	0.193718811605707\\
150	0.0551868249721559\\
151	0.175545791949479\\
152	0.265226662285279\\
153	0.0910554779124941\\
154	0.0294235588541712\\
155	0.138893131048589\\
156	0.115875323159545\\
157	0.935671911456588\\
158	0.170399949043299\\
159	0.154007446706733\\
160	0.0459429833404734\\
161	-0.0271183257486216\\
162	0.0995257935699255\\
163	0.0439444915789618\\
164	0.0592577771226312\\
165	0.0906661460713747\\
166	-0.00934421917577832\\
167	0.0691920003913826\\
168	0.0404467400894151\\
169	0.116357079062839\\
170	0.0873829868840405\\
171	0.0689207344763225\\
172	0.115797796108756\\
173	0.102397043450317\\
174	0.102132131921362\\
175	0.0988197625440041\\
176	0.137725673675573\\
177	0.0417680894954063\\
178	0.126107383932523\\
179	0.0688883613033566\\
180	0.0301486515163635\\
181	0.112680280368023\\
182	0.0228471102546033\\
183	0.107690034234174\\
184	0.0306535256219218\\
185	0.0585133081083639\\
186	0.0991100593273178\\
187	0.00968107514038874\\
188	0.135917373184386\\
189	0.0993256221605311\\
190	0.0905046784876292\\
191	0.0566056281555663\\
192	0.144957112923868\\
193	0.191822031872813\\
194	0.058318735155575\\
195	0.0899274475943735\\
196	0.0975745966477257\\
197	0.108267845644366\\
198	0.0281725014584351\\
199	0.256188622568536\\
200	0.109476000772486\\
201	0.292313285262679\\
202	0.0647289062570355\\
203	0.0251197348948053\\
204	0.00895702473531189\\
205	0.340164948773289\\
206	0.0859461024205246\\
207	0.0634386156723113\\
208	0.170725975670945\\
209	0.0612547323867039\\
210	-0.0248490764958978\\
211	0.207818489945175\\
212	0.0964755625880126\\
213	0.154068018704167\\
214	0.123824293700171\\
215	0.210578372713573\\
216	-0.0237832666641564\\
217	0.19835660928597\\
218	0.113216170851969\\
219	0.194208040118491\\
220	0.397498485209749\\
221	-0.0876346285626362\\
222	0.186936948816548\\
223	-0.0238837197366306\\
224	0.149373899472736\\
225	0.190120750748549\\
226	0.0805772119481869\\
227	0.0273366850750235\\
228	0.0537738531530057\\
229	0.0975099375201429\\
230	0.101562787844492\\
231	nan\\
232	0.0783348603028827\\
233	0.0619807971405299\\
234	0.232791527437023\\
235	0.408532536242019\\
236	0.0664400521112427\\
237	0.1030434208326\\
238	0.125640660788505\\
239	0.541335529362101\\
240	-0.00255034936805533\\
241	-0.0038956891717937\\
242	0.0894642139762351\\
243	0.0648820468057703\\
244	-0.0134466406131337\\
245	0.151959596143075\\
246	0.0116627653897352\\
247	0.0431726523143674\\
248	nan\\
249	0.0288866414171665\\
250	0.0882512063336328\\
251	0.0821384908360475\\
252	0.0203376197223765\\
};
\end{axis}
\end{tikzpicture}%
 
\end{subfigure}\\
\vspace{1cm}
\begin{subfigure}{.45\linewidth}
  \centering
  \setlength\figureheight{\linewidth} 
  \setlength\figurewidth{\linewidth}
  \tikzsetnextfilename{IS_sameday_ORCL}
  % This file was created by matlab2tikz.
%
%The latest updates can be retrieved from
%  http://www.mathworks.com/matlabcentral/fileexchange/22022-matlab2tikz-matlab2tikz
%where you can also make suggestions and rate matlab2tikz.
%
\definecolor{mycolor1}{rgb}{0.25098,0.00000,0.38824}%
\definecolor{mycolor2}{rgb}{0.00000,0.46275,0.00000}%
\definecolor{mycolor3}{rgb}{0.00000,0.34902,0.34902}%
\definecolor{mycolor4}{rgb}{0.58039,0.26275,0.00000}%
%
\begin{tikzpicture}[trim axis left, trim axis right]

\begin{axis}[%
width=\figurewidth,
height=\figureheight,
at={(0\figurewidth,0\figureheight)},
scale only axis,
every outer x axis line/.append style={black},
every x tick label/.append style={font=\color{black}},
xmin=1,
xmax=252,
%xlabel={Time (h)},
every outer y axis line/.append style={black},
every y tick label/.append style={font=\color{black}},
ymin=-1.1,
ymax=1.1,
%ylabel={Normalized PnL},
title={ORCL},
axis background/.style={fill=white},
axis x line*=bottom,
axis y line*=left,
yticklabel style={
        /pgf/number format/fixed,
        /pgf/number format/precision=3
},
scaled y ticks=false,
]
\addplot [color=mycolor1,solid,line width=1.5pt,forget plot]
  table[row sep=crcr]{%
1	0.241223274287608\\
2	0.24745065209925\\
3	0.127746130518664\\
4	0.0203272197232222\\
5	0.060032618753344\\
6	0.137470270516756\\
7	0.223445930484044\\
8	0.164140903838792\\
9	0.154450640745265\\
10	0.053217762948819\\
11	0.23458028517867\\
12	0.14572885290245\\
13	0.521284360126471\\
14	0.177521212622025\\
15	0.125315186751598\\
16	0.209854204316187\\
17	0.210490303692473\\
18	0.159760094908392\\
19	0.197355786902324\\
20	0.13823422781648\\
21	0.128173567773223\\
22	0.110295740028735\\
23	-0.0459245507968338\\
24	0.100599672742255\\
25	0.110979164201082\\
26	0.1345583952384\\
27	0.142962944472724\\
28	0.162019958135607\\
29	0.167794129093981\\
30	0.0450090971134973\\
31	0.0602639492750183\\
32	0.139149109878398\\
33	0.248690911586315\\
34	-0.0655120377294428\\
35	0.115125661875154\\
36	0.0954162073168408\\
37	0.0669382802350141\\
38	0.168464080407859\\
39	0.0927190483618675\\
40	0.248860269652851\\
41	0.0973576369216187\\
42	0.169025168510289\\
43	0.150855917903857\\
44	0.23514066295871\\
45	0.108436124677382\\
46	0.101299615551686\\
47	0.125910215136719\\
48	0.0379509192574323\\
49	0.0558724696828833\\
50	0.156030085921235\\
51	0.139719738402309\\
52	0.136003184936705\\
53	0.191848662819897\\
54	0.206937220712009\\
55	0.290068723473809\\
56	0.18894115508058\\
57	-0.0523846010311696\\
58	0.107298264062165\\
59	0.31809637815883\\
60	0.226176092613965\\
61	0.191197845873848\\
62	0.180443154787067\\
63	0.127039685403067\\
64	0.219638567605891\\
65	0.202412509975261\\
66	0.1569711490769\\
67	0.28055513945065\\
68	0.245362139750547\\
69	0.185467343604661\\
70	0.060709656064548\\
71	-0.0285056144430903\\
72	0.14458350097571\\
73	0.207388957403149\\
74	0.230732506321745\\
75	0.257405714580662\\
76	0.228033641021165\\
77	0.147286436881983\\
78	0.240994694556747\\
79	0.145005857526444\\
80	0.221598854926346\\
81	0.157875393697151\\
82	0.126219773292884\\
83	0.178324864158897\\
84	0.195839825256848\\
85	0.0928970420706774\\
86	0.155724657821501\\
87	-0.000465347954131393\\
88	0.127258315497155\\
89	0.199381256230535\\
90	0.107231200134548\\
91	0.0301832428779352\\
92	-0.0985974903465689\\
93	0.0808793552600453\\
94	-0.00860132105104882\\
95	0.0729748833160431\\
96	0.078747488141741\\
97	0.139186196515515\\
98	-0.114958700000476\\
99	0.184070586197995\\
100	0.0369592578900176\\
101	0.0546503495155683\\
102	0.153141414163885\\
103	0.12884774990341\\
104	-0.00867147887933388\\
105	0.29183544379335\\
106	0.193418498046049\\
107	0.0920431132441799\\
108	0.163648754280062\\
109	0.133098513253821\\
110	0.237121761593614\\
111	0.0299341390334858\\
112	0.188724675552604\\
113	0.172244592484314\\
114	0.13218191213765\\
115	0.17903778535716\\
116	0.0739145224338239\\
117	0.16561557862633\\
118	0.0855004972167472\\
119	-0.216596498114913\\
120	0.271285706464782\\
121	0.19548447765328\\
122	0.142875503733172\\
123	0.0290956358200346\\
124	0.216256923540652\\
125	-0.128048224230666\\
126	0.217388801809032\\
127	nan\\
128	0.167492394919075\\
129	0.189328149615795\\
130	0.0959364006512009\\
131	0.108981973590273\\
132	0.0897098506433756\\
133	-0.14266997682801\\
134	0.147425894288815\\
135	0.0672618200057133\\
136	0.113532146342714\\
137	0.0447675775724201\\
138	-0.0210005222149592\\
139	0.0633252183434963\\
140	0.0220889265289387\\
141	0.0827438285466515\\
142	0.180875311005951\\
143	0.132219491113588\\
144	0.160563645809524\\
145	0.0890586783857221\\
146	0.110891060530117\\
147	0.0812720090836312\\
148	0.0860984505059703\\
149	0.100754722510865\\
150	0.0685619150860805\\
151	0.0487189280678173\\
152	0.0813039281390116\\
153	0.0166560683293825\\
154	0.0550293357495144\\
155	-0.000154935184961307\\
156	0.107560800217674\\
157	0.0316463098747111\\
158	0.12151499323743\\
159	0.0434969721000555\\
160	0.102875303407633\\
161	0.0988586081988907\\
162	0.0268600688505\\
163	0.156856156054873\\
164	0.078127863406937\\
165	0.0402222962530346\\
166	0.0495058588916819\\
167	0.034743352642104\\
168	0.0335260093509291\\
169	0.046336613013224\\
170	0.0209464277052586\\
171	0.0652117287117677\\
172	0.0324816015661456\\
173	0.0482661868731509\\
174	0.0497191253010622\\
175	-0.016562153628797\\
176	0.0863502930792854\\
177	0.070404280217022\\
178	0.00637605090709595\\
179	0.0393588987029311\\
180	0.142256298886192\\
181	0.16421400363218\\
182	0.120326306884496\\
183	0.0344992241089748\\
184	0.0473475310219116\\
185	0.0636802096642029\\
186	0.0631257278991188\\
187	0.12088653154436\\
188	0.124133771256082\\
189	0.0584762516333993\\
190	0.105455357015284\\
191	0.0682468045322736\\
192	0.140588009293907\\
193	0.164773444874498\\
194	0.231907967089212\\
195	0.205664160133516\\
196	0.248029570591285\\
197	0.149379353264988\\
198	0.10466505765627\\
199	0.137433651124083\\
200	0.133102222995278\\
201	0.069180818207139\\
202	0.143775577759242\\
203	0.0739502508429764\\
204	0.173104009949441\\
205	0.191529757841133\\
206	0.0716301103331933\\
207	0.163363505350456\\
208	0.0587338333643523\\
209	0.116289849557864\\
210	0.146317972320714\\
211	0.126036872851262\\
212	0.154504471955781\\
213	0.053379498807833\\
214	0.0772241375076522\\
215	0.143404427705278\\
216	0.068022502603057\\
217	0.00854996506683371\\
218	0.0341009738373644\\
219	0.0944268509716125\\
220	0.122148291575481\\
221	0.130590493220607\\
222	0.00489377884325976\\
223	0.0881278447922347\\
224	0.0231001602860589\\
225	0.0640433149208885\\
226	0.0506999679533119\\
227	0.11438020428506\\
228	0.0442678639845931\\
229	0.148451266074303\\
230	0.100813727774735\\
231	nan\\
232	-0.0234799535535712\\
233	0.141941938597241\\
234	0.141906435665674\\
235	0.101901315463783\\
236	0.176801485681818\\
237	0.0863980044614175\\
238	0.0254058724886769\\
239	0.098460953890485\\
240	0.058179083754149\\
241	0.0115479930953512\\
242	0.143984141545859\\
243	0.101118132867356\\
244	0.294196493306117\\
245	0.282158026086236\\
246	0.137386117329218\\
247	0.151718434304225\\
248	nan\\
249	0.11292797935102\\
250	0.0398592077364431\\
251	0.042703010834764\\
252	0.0483242416796185\\
};
\addplot [color=mycolor2,solid,line width=1.5pt,forget plot]
  table[row sep=crcr]{%
1	0.273346665066574\\
2	0.189212312626568\\
3	0.211205152629585\\
4	0.179013768165097\\
5	0.169064698420172\\
6	0.0964873429571397\\
7	0.17558688117321\\
8	0.122105728437492\\
9	0.164119278284694\\
10	0.175848435914647\\
11	0.188204213997302\\
12	0.121890287422419\\
13	0.515050783158148\\
14	0.188028003721372\\
15	0.124428940917841\\
16	0.2220490561916\\
17	0.221516059081375\\
18	0.228332917589034\\
19	0.209887733030315\\
20	0.0875383764195\\
21	0.220762736769412\\
22	0.127634483875145\\
23	0.0257226860606545\\
24	0.117477215430799\\
25	0.075484976001504\\
26	0.108252039055186\\
27	0.160307065992963\\
28	0.0980375962929234\\
29	0.176103032120917\\
30	0.0663116570919216\\
31	0.100264220076255\\
32	0.130377427069623\\
33	0.251174961680548\\
34	0.0718727701614224\\
35	0.0819917590646432\\
36	0.0916017618022998\\
37	0.0302330760789778\\
38	0.102931926247013\\
39	0.126051405992587\\
40	0.112587374030975\\
41	0.164423714118142\\
42	0.259079847017375\\
43	0.097595037307296\\
44	0.2418257337092\\
45	0.147104344096026\\
46	0.083169093589618\\
47	0.123792769995301\\
48	0.0633159197964904\\
49	0.0641380249115127\\
50	0.198799785600185\\
51	0.166613207334672\\
52	0.171839442460413\\
53	0.0361148384642435\\
54	0.232225143110759\\
55	0.485188252045305\\
56	0.18868442443865\\
57	0.101076171797811\\
58	0.150680913939971\\
59	0.336958755375116\\
60	0.24505226999223\\
61	0.19123162197311\\
62	0.191380349176894\\
63	0.156593554694079\\
64	0.20705005338918\\
65	0.297069140122517\\
66	0.129093895116347\\
67	0.286397382291399\\
68	0.329564105009742\\
69	0.192508713170727\\
70	0.114080020307172\\
71	0.0032055876252193\\
72	0.183312677255579\\
73	-0.0566901234503444\\
74	0.193117321765846\\
75	0.320902819989265\\
76	0.218212823777351\\
77	0.131936428895831\\
78	0.165131312391019\\
79	0.171946799426226\\
80	0.231091102315572\\
81	0.196226962892521\\
82	0.416299269499185\\
83	0.12724580623387\\
84	0.24659822718522\\
85	0.118540866460038\\
86	0.143455446634539\\
87	0.118801469887235\\
88	0.192628530541603\\
89	0.280904679770813\\
90	0.126576332970002\\
91	0.039765584493218\\
92	0.108558612643594\\
93	0.186690131816102\\
94	0.120046391751876\\
95	0.129465249564172\\
96	0.111564228059796\\
97	0.189333991136525\\
98	0.0288264833224655\\
99	0.174054847095928\\
100	0.105890654914792\\
101	0.0894025798863696\\
102	0.0771669747740801\\
103	0.0839455837165869\\
104	0.0653931761400871\\
105	0.335273437562528\\
106	0.201584907139371\\
107	0.146825567491122\\
108	0.128345921588691\\
109	0.143479640545705\\
110	0.216413939197816\\
111	0.127768664139924\\
112	0.111208610819479\\
113	0.24728612832673\\
114	0.108781785973734\\
115	0.16325065491916\\
116	0.137459933321632\\
117	0.0673502806794637\\
118	-0.00791208882767072\\
119	0.156423076414051\\
120	0.29375609055909\\
121	0.118846864947344\\
122	0.156778739682479\\
123	0.313427626122744\\
124	0.297919902909278\\
125	0.0873580997993704\\
126	0.211048138151547\\
127	nan\\
128	0.212140002709667\\
129	0.214268833264449\\
130	0.244162177162469\\
131	0.157511081632883\\
132	0.194036613983136\\
133	0.139842108130623\\
134	0.168520280553994\\
135	0.106642265173287\\
136	0.145830393290165\\
137	0.0925566426252537\\
138	0.0627152187216657\\
139	0.0651092114789767\\
140	0.0976686090058129\\
141	0.075170588027189\\
142	0.0988593821602895\\
143	0.118902858429481\\
144	0.151727088886939\\
145	0.0436928544311631\\
146	0.113561753192908\\
147	0.0832077495537323\\
148	0.0539508409658656\\
149	0.0701439256636209\\
150	0.19270004503352\\
151	0.073373174575009\\
152	0.0557939043785814\\
153	0.0651762035367486\\
154	0.0606709459560761\\
155	0.027790196372764\\
156	0.115831565008953\\
157	0.0415731376565472\\
158	0.0851811833747788\\
159	0.0709144384693512\\
160	0.0658345812929325\\
161	0.0694815446780439\\
162	0.0747291028069253\\
163	0.0324007969749028\\
164	0.172988146407423\\
165	0.015647815386357\\
166	0.0262265276423608\\
167	0.0765869873848283\\
168	0.0991924831409798\\
169	0.103949430167673\\
170	0.101799169207094\\
171	0.0509383632951249\\
172	0.0860225548288399\\
173	0.21896668727248\\
174	0.097533578774753\\
175	0.0972294515542211\\
176	0.0977980045664691\\
177	0.0440044809180337\\
178	0.0711218674448398\\
179	0.135724878737634\\
180	0.153414262473705\\
181	0.260616041982053\\
182	0.111490494211181\\
183	0.107276117911108\\
184	0.0299050229668953\\
185	0.0769459936223605\\
186	0.0877406995209833\\
187	0.0595068027496756\\
188	0.0746817717294346\\
189	0.093149089038501\\
190	0.0604099500234195\\
191	-0.0634235302427407\\
192	0.121088758042349\\
193	0.10494311384696\\
194	0.122904842841552\\
195	0.121487137284002\\
196	0.323962352476322\\
197	0.0891732130448263\\
198	0.125268227571178\\
199	0.0629518651664227\\
200	0.138620723659144\\
201	0.0858094335144247\\
202	0.061835341186842\\
203	0.0403852289768351\\
204	0.0580164473040898\\
205	0.0946614662032086\\
206	0.180788875327448\\
207	0.102963476251914\\
208	0.0982523031321061\\
209	0.135988269401859\\
210	0.0471417200606991\\
211	0.142930524157119\\
212	0.103158214732359\\
213	0.0809070237730298\\
214	0.058940334861863\\
215	0.205968113911324\\
216	0.0323278939199908\\
217	0.0601643207931574\\
218	0.0674616700757591\\
219	0.147705767227897\\
220	0.113156192179139\\
221	0.0773647658653425\\
222	0.160827683540172\\
223	0.0739187978328479\\
224	0.0784007796469039\\
225	0.0472635467184074\\
226	0.0716836030065077\\
227	0.0793792271825583\\
228	0.0969261723876889\\
229	0.158513618723575\\
230	0.103394013234269\\
231	nan\\
232	0.0287296975388596\\
233	0.179287496936059\\
234	0.139689384171674\\
235	0.12502080733179\\
236	0.171006881758098\\
237	0.0964309718920965\\
238	0.0717890901144666\\
239	0.0315646560471625\\
240	0.10627969293463\\
241	0.0980803063454243\\
242	0.0841267982951749\\
243	0.0770993047727351\\
244	0.317135474553656\\
245	0.429330303626638\\
246	0.199891083029741\\
247	0.167428558387384\\
248	nan\\
249	0.128500149202004\\
250	0.136260048789033\\
251	0.0342568208026125\\
252	0.0668284671772081\\
};
\addplot [color=mycolor3,solid,line width=1.5pt,forget plot]
  table[row sep=crcr]{%
1	-0.0846576225918833\\
2	0.155281601903694\\
3	0.0221751536407668\\
4	0.0337038384603742\\
5	0.0116085386732742\\
6	-0.0175877020692845\\
7	0.147556791679367\\
8	0.121744214406951\\
9	0.0752120937739852\\
10	0.028158031446299\\
11	0.11086890180201\\
12	0.065635175982436\\
13	0.503393570902585\\
14	-0.0381665301710048\\
15	0.0937484557313351\\
16	0.0741180141310412\\
17	0.149422872459086\\
18	0.0412212942352366\\
19	0.0395035093552537\\
20	-0.126479857853542\\
21	0.0275979621563786\\
22	-0.0477692996328487\\
23	-0.213250996409327\\
24	0.0700504263461083\\
25	-0.0622542471476072\\
26	-0.122209029044425\\
27	0.0481814026832434\\
28	0.0904226975773476\\
29	0.182607883130763\\
30	0.0611108060630259\\
31	0.00490750872288962\\
32	0.0429367998809651\\
33	0.0752582035032304\\
34	-0.190949119205558\\
35	-0.0863546057391088\\
36	-0.104020680247121\\
37	-0.138237662523695\\
38	-0.0511122550532538\\
39	-0.166307438326315\\
40	0.0836486133601431\\
41	0.176675160682967\\
42	0.0432541131287971\\
43	-0.0249965210769209\\
44	-0.0112820327496435\\
45	0.0235090691746489\\
46	-0.0283187733753653\\
47	0.0528482238468779\\
48	-0.018264867559768\\
49	-0.0289363217573902\\
50	0.0534948813879319\\
51	0.0954755842380508\\
52	0.0224398222063561\\
53	-0.00368548833273068\\
54	0.0120852123650379\\
55	0.0910526192212227\\
56	0.0799127668471845\\
57	-0.0693162163445281\\
58	0.0781683160041529\\
59	0.16649960400856\\
60	0.0778574941823046\\
61	0.0705434190131545\\
62	0.0221829696954044\\
63	-0.0379533784247016\\
64	0.0848255226525035\\
65	0.145029474590804\\
66	0.0732602800011299\\
67	-0.0524642139851114\\
68	-0.0531025130736515\\
69	0.0234836439557796\\
70	-0.0682540646419776\\
71	-0.132420767879116\\
72	-0.0428638835429499\\
73	-0.261856361126151\\
74	-0.126752828756445\\
75	0.0556289157318817\\
76	0.0434120064241584\\
77	-0.118498751536332\\
78	0.0348167372240673\\
79	-0.00329157252309795\\
80	-0.0468210966101787\\
81	0.0876246348069102\\
82	0.191871458697738\\
83	0.0373557634466459\\
84	-0.0121371214121342\\
85	-0.008343365232015\\
86	-0.0331095093905462\\
87	-0.0322958227548654\\
88	0.097098128135251\\
89	0.146093976955255\\
90	-0.0138021304815943\\
91	-0.121009711924756\\
92	-0.168487320043594\\
93	0.00986484180124042\\
94	-0.0457537434853789\\
95	0.00519213619101177\\
96	-0.0333620089540813\\
97	0.0901430146916826\\
98	-0.281331014338772\\
99	-0.157999852400096\\
100	-0.224051074956349\\
101	-0.091912764282101\\
102	-0.101582871739512\\
103	-0.0692772857726541\\
104	-0.181591250066822\\
105	0.00914738862360414\\
106	0.0710280024488038\\
107	-0.12063125893102\\
108	-0.0717977254365932\\
109	0.0293394093789232\\
110	-0.00226067818741903\\
111	-0.215804458070252\\
112	-0.0375676753045819\\
113	0.0140450866257482\\
114	-0.0175386469973769\\
115	0.0837311057976844\\
116	0.13808153812408\\
117	-0.0808118752226555\\
118	-0.206266489480464\\
119	-0.478207937042886\\
120	0.121972852644426\\
121	0.0477286903377687\\
122	0.173686494231355\\
123	-0.247103007819378\\
124	-0.0644129362762274\\
125	-0.156364581206834\\
126	0.174764318052435\\
127	nan\\
128	-0.0445714235812645\\
129	0.0861094272178511\\
130	0.0893130000150174\\
131	0.0814090499854394\\
132	0.132763938887605\\
133	-0.14753273975168\\
134	0.0254980440968309\\
135	-0.0634148125882246\\
136	0.0530169273621083\\
137	-0.0334542731792991\\
138	-0.110635913862955\\
139	0.055031731345818\\
140	-0.0213467072359347\\
141	-0.0101921442935641\\
142	-0.0104854653610358\\
143	-0.049982870049771\\
144	0.0162122832847615\\
145	0.000323636202974383\\
146	-0.113183707091048\\
147	0.0072187533983602\\
148	0.0369688155256217\\
149	0.036977467411456\\
150	0.0730615040923931\\
151	-0.0667703261663191\\
152	0.0251262446013779\\
153	0.00216973636669341\\
154	-0.0482272472139541\\
155	-0.0868730582214768\\
156	0.0653355503406741\\
157	-0.0383234539760709\\
158	-0.00108468463711581\\
159	-0.0541794247057601\\
160	0.0314131652349796\\
161	0.0504261119636444\\
162	0.00436060118335924\\
163	0.0270640076812079\\
164	-0.0210071368614834\\
165	-0.118898264773489\\
166	-0.0263563541011511\\
167	0.00696478598077608\\
168	-0.017189057992433\\
169	-0.0334184527152792\\
170	0.0339692809025743\\
171	-0.0465058854921497\\
172	-0.0402170466475977\\
173	0.0585487766266136\\
174	-0.0313781133397876\\
175	-0.121257760005896\\
176	-0.0339090820768931\\
177	0.00313004440230345\\
178	-0.066609273727825\\
179	-0.0468718897081484\\
180	-0.135170096099848\\
181	-0.255767981268249\\
182	-0.121621298945692\\
183	-0.173984423017701\\
184	-0.0605610390915512\\
185	-0.133077218584128\\
186	-0.0950356244983138\\
187	-0.0995823391090869\\
188	-0.0565236226010403\\
189	-0.0978886858219799\\
190	-0.156501853025451\\
191	-0.168322739663517\\
192	-0.0379973254617043\\
193	-0.0696391560601512\\
194	0.00455896402856274\\
195	0.0315317519874283\\
196	0.0772161557106037\\
197	-0.178778529657615\\
198	-0.0226173591654247\\
199	-0.103955509839379\\
200	-0.0382022970515123\\
201	-0.0531888661737467\\
202	-0.0067638461231929\\
203	-0.0388717218347404\\
204	-0.014229578474209\\
205	-0.00318179693005503\\
206	0.0594990759366699\\
207	0.0379238343155247\\
208	-0.0260218004275441\\
209	0.0175691806121162\\
210	-0.144790476407852\\
211	-0.00355621418883807\\
212	0.0583664576522994\\
213	0.0293825015424531\\
214	0.0464850777956799\\
215	0.0357380968930529\\
216	-0.0873788472858223\\
217	-0.0655897458475655\\
218	0.0247845237125166\\
219	-0.0142226174354622\\
220	0.0911153403224589\\
221	-0.111084180450963\\
222	0.123595853265444\\
223	0.0493089697404575\\
224	-0.0659406720165051\\
225	-0.01844582654628\\
226	0.0283824539060457\\
227	0.0425695384813175\\
228	0.0722915230285756\\
229	0.0504789694652664\\
230	0.0148724624634424\\
231	nan\\
232	-0.0608119860700085\\
233	0.0908345234576338\\
234	0.104946511449879\\
235	-0.0191603642490593\\
236	0.0830904007390058\\
237	-0.0133266661397632\\
238	0.0267752435700931\\
239	0.00878429212249776\\
240	-0.113693058333204\\
241	0.0464327740851524\\
242	-0.0545602570180026\\
243	0.0534725065018719\\
244	0.096906976517188\\
245	-0.154088322082056\\
246	0.0296050598383107\\
247	0.0750937502002928\\
248	nan\\
249	-0.00413260491543337\\
250	0.00108549800118402\\
251	0.0315328096198634\\
252	-0.0105493936908908\\
};
\addplot [color=mycolor4,solid,line width=1.5pt,forget plot]
  table[row sep=crcr]{%
1	0.307285424571827\\
2	0.196668817360928\\
3	0.235311054721116\\
4	0.184743829233353\\
5	0.209543650440518\\
6	0.128661231937815\\
7	0.211165829849492\\
8	0.126929036072532\\
9	0.176560187720251\\
10	0.196132126133114\\
11	0.21274191293403\\
12	0.174456184898567\\
13	0.572703712979235\\
14	0.322096582835897\\
15	0.184456548359416\\
16	0.171077512791466\\
17	0.123016581173625\\
18	0.240855607656172\\
19	0.223617221978723\\
20	0.092311525105119\\
21	0.151543712443178\\
22	0.0375241122849847\\
23	-0.0249626725291844\\
24	0.128432500374722\\
25	0.162439704142357\\
26	0.273526246000629\\
27	0.115437721821743\\
28	0.10513599732059\\
29	0.116937515160046\\
30	0.0535358982026782\\
31	0.0995758865662426\\
32	0.158641622242691\\
33	0.2456121908829\\
34	0.010453101695058\\
35	0.0866437001083825\\
36	0.0149861290420329\\
37	0.00740214696205961\\
38	0.128350439966792\\
39	0.185681358259925\\
40	0.287963410717822\\
41	0.179869089783602\\
42	0.293066608658142\\
43	0.128591434798244\\
44	0.2418772223055\\
45	0.142367074798148\\
46	0.165228010952215\\
47	0.157391745062771\\
48	0.14894638088329\\
49	0.0693477211394827\\
50	0.223055388648566\\
51	0.151292371649033\\
52	0.17714454462719\\
53	0.253825101458325\\
54	0.204814805173776\\
55	0.458250959253053\\
56	0.117121262800926\\
57	0.0751484741075734\\
58	0.169449344897272\\
59	0.359516561671493\\
60	0.248653564181917\\
61	0.193505802777672\\
62	0.198862693718709\\
63	0.148237051263375\\
64	0.220668635859849\\
65	0.334512495534136\\
66	0.157366619147144\\
67	0.106587072926027\\
68	0.383270727564274\\
69	0.184577521804748\\
70	0.0947805709273705\\
71	0.185044179569012\\
72	0.229267863603023\\
73	-0.0225791628539177\\
74	0.418248195298535\\
75	0.354985403548023\\
76	0.168185968911444\\
77	0.122753302189851\\
78	0.272541556765899\\
79	0.14686626464717\\
80	0.133695611235848\\
81	0.164791423363682\\
82	0.455604427542813\\
83	0.14690699834231\\
84	0.263048173024092\\
85	0.089409489762996\\
86	0.138639283744544\\
87	0.0988079958750348\\
88	0.164863828094923\\
89	0.292372170096296\\
90	0.142040274486306\\
91	0.0235983185342222\\
92	0.0566750373676545\\
93	0.0905864320161909\\
94	0.0607362801469435\\
95	0.131528118257272\\
96	0.102549382523037\\
97	0.200739353469501\\
98	0.235484344276673\\
99	0.252411325176238\\
100	0.11016282248878\\
101	0.095908983062567\\
102	0.123338851585889\\
103	0.0591781074202609\\
104	0.0388251493755226\\
105	0.464479776088965\\
106	0.153899403747447\\
107	0.162247133896345\\
108	0.313494898733717\\
109	0.183344363464974\\
110	0.215328929100844\\
111	0.0909984350513846\\
112	0.159267122409394\\
113	-0.0316582002231229\\
114	0.23093136426872\\
115	0.185409868378807\\
116	0.128344857874948\\
117	0.117662961563677\\
118	-0.0124790467832456\\
119	0.116807511327024\\
120	0.349616930350628\\
121	0.0984807943232725\\
122	0.155844982833761\\
123	0.280223838905043\\
124	0.300210581070049\\
125	0.0146208958647383\\
126	0.20932916520657\\
127	nan\\
128	0.199103491677951\\
129	0.120638911289335\\
130	0.258851138344847\\
131	0.140852647475561\\
132	0.190832634455336\\
133	0.0630070295587054\\
134	0.154781032796507\\
135	0.119844192414237\\
136	0.153248149228531\\
137	0.11548395888323\\
138	0.0536905330868437\\
139	0.0636464934007989\\
140	0.105827621322495\\
141	0.0799214990589647\\
142	0.103888019718052\\
143	0.0512625513101907\\
144	0.154113577374722\\
145	0.092318523498236\\
146	0.121029426515441\\
147	0.0901377859971579\\
148	0.0699504874455004\\
149	0.0348783431840218\\
150	0.166420248066132\\
151	0.0746463166148519\\
152	0.0912031998905761\\
153	0.0320691628387041\\
154	0.0228924044609168\\
155	0.0725241439741116\\
156	0.0796851476670739\\
157	0.158247387447405\\
158	0.0562948211631204\\
159	0.134534455960333\\
160	0.0842106066081413\\
161	0.0967467938158669\\
162	0.0660858574302106\\
163	0.0245804268173264\\
164	0.168232955722048\\
165	0.0850135652308796\\
166	0.0611008179592187\\
167	0.091185195989006\\
168	0.121902595724408\\
169	0.103652592552605\\
170	0.0298343021929557\\
171	0.0758839803205474\\
172	0.0967485937809668\\
173	0.0438856643312103\\
174	0.0969598726272626\\
175	0.0399342095009177\\
176	0.204225560407387\\
177	0.0493607021819654\\
178	0.0377742266476312\\
179	0.14099209272245\\
180	0.231308889608903\\
181	0.270558867329969\\
182	0.119964151532389\\
183	0.0937556819824004\\
184	0.081688611726089\\
185	0.063556976289782\\
186	0.0696681812845726\\
187	0.0261314132138828\\
188	0.195335402696224\\
189	0.104742809892857\\
190	0.0616857058362728\\
191	-0.0237057342954233\\
192	0.166182104706984\\
193	0.15416037199674\\
194	0.140790567550672\\
195	0.146180716896118\\
196	0.358567793747239\\
197	0.159105047433635\\
198	0.128428148945464\\
199	0.0483323438423279\\
200	0.185871189266475\\
201	0.121738494402955\\
202	0.101683570926425\\
203	0.16041813005798\\
204	0.134062442660448\\
205	0.0812758252239515\\
206	0.203226409153398\\
207	0.124056123826672\\
208	0.0972582039864547\\
209	0.186347102272649\\
210	0.0516254980297903\\
211	0.139205506868576\\
212	0.0892179945971371\\
213	0.0837651650602831\\
214	0.0481244793387642\\
215	0.282217433996174\\
216	0.028732311418992\\
217	0.0665234401348808\\
218	0.0685391920627127\\
219	0.0212470760301315\\
220	0.0552896147186778\\
221	0.131991653633165\\
222	0.0276060105381962\\
223	0.0673788238363792\\
224	0.112900847015324\\
225	0.05687194828695\\
226	0.0304754564341683\\
227	0.074048278512078\\
228	0.101169379633474\\
229	0.187547298111378\\
230	0.114790708437446\\
231	nan\\
232	0.0230624436458383\\
233	0.13865002676311\\
234	0.0887185122655012\\
235	0.0880199781819814\\
236	0.244309000173449\\
237	0.113573922025022\\
238	0.0230279663344548\\
239	0.137324963705612\\
240	0.0731413280680853\\
241	0.0885642099411079\\
242	0.108985810793237\\
243	0.0988798551165541\\
244	0.382457372710262\\
245	0.465094175049723\\
246	0.197487144557938\\
247	0.206654202096348\\
248	nan\\
249	0.151514847965492\\
250	0.127387638224365\\
251	0.0347612411880016\\
252	0.0943046211513907\\
};
\end{axis}
\end{tikzpicture}%

\end{subfigure}%
\hfill%
\begin{subfigure}{.45\linewidth}
  \centering
  \setlength\figureheight{\linewidth} 
  \setlength\figurewidth{\linewidth}
  \tikzsetnextfilename{IS_sameday_INTC}
  % This file was created by matlab2tikz.
%
%The latest updates can be retrieved from
%  http://www.mathworks.com/matlabcentral/fileexchange/22022-matlab2tikz-matlab2tikz
%where you can also make suggestions and rate matlab2tikz.
%
%
\begin{tikzpicture}[trim axis left, trim axis right]

\begin{axis}[%
width=\figurewidth,
height=\figureheight,
at={(0\figurewidth,0\figureheight)},
scale only axis,
every outer x axis line/.append style={black},
every x tick label/.append style={font=\color{black}},
xmin=1,
xmax=252,
%xlabel={Time (h)},
every outer y axis line/.append style={black},
every y tick label/.append style={font=\color{black}},
ymin=-1.1,
ymax=1.1,
%ylabel={Normalized PnL},
title={INTC},
axis background/.style={fill=white},
axis x line*=bottom,
axis y line*=left,
yticklabel style={
        /pgf/number format/fixed,
        /pgf/number format/precision=3
},
scaled y ticks=false,
]
\addplot [color=cts_plot_color,solid,line width=1.5pt,forget plot]
  table[row sep=crcr]{%
1	0.391589743909999\\
2	0.293083529682506\\
3	0.188933630622459\\
4	0.137562023861166\\
5	0.103150713225003\\
6	0.307578941910393\\
7	0.338294500475941\\
8	0.239104447513227\\
9	0.211827283956308\\
10	0.214517866655773\\
11	0.336718265752563\\
12	0.207215673870654\\
13	0.31850310303286\\
14	0.192046597654574\\
15	0.179448508958753\\
16	0.229834462662867\\
17	0.206629655940924\\
18	0.242404046396384\\
19	0.260557719208387\\
20	0.179553123507906\\
21	0.158273794341599\\
22	0.308850958221178\\
23	0.187928773474472\\
24	0.210443692503262\\
25	0.209761085362709\\
26	0.291570630073024\\
27	0.222689951228062\\
28	0.143289627521191\\
29	0.226480981385223\\
30	0.125171371227529\\
31	0.166586106296488\\
32	0.159410680897874\\
33	0.155251802870983\\
34	0.168237639261154\\
35	-0.0686538044199376\\
36	0.15770140032313\\
37	0.211210567515153\\
38	0.423660127674359\\
39	0.359910454100227\\
40	0.137236165283162\\
41	0.294336986097433\\
42	0.227326885180794\\
43	0.240341027609736\\
44	0.245485822371267\\
45	0.207293677312832\\
46	0.100200577045647\\
47	0.279989737706634\\
48	0.0928568278703446\\
49	0.173340720221563\\
50	0.182501881776688\\
51	0.132773709084778\\
52	0.216255834381277\\
53	0.249170078390131\\
54	0.178407444213397\\
55	0.260856351173351\\
56	0.251273752896795\\
57	0.260528389026425\\
58	0.448570971825313\\
59	0.34780586911256\\
60	0.121897622188909\\
61	0.157625766963898\\
62	0.173903993046632\\
63	0.118192786942004\\
64	0.282875234760071\\
65	0.271846713507021\\
66	0.245037388946468\\
67	0.296549738076209\\
68	0.408068413063436\\
69	0.363541161875419\\
70	0.173309066028106\\
71	0.226814147686051\\
72	0.472609891207332\\
73	0.342410509752256\\
74	0.456413095290415\\
75	0.444951932109454\\
76	0.385459576099748\\
77	0.453027499315521\\
78	0.452254308164672\\
79	0.0790376599986121\\
80	0.280051727140669\\
81	0.216799036380052\\
82	0.343739805470527\\
83	0.165918096320032\\
84	0.261102719099238\\
85	0.209991451932288\\
86	0.117624205199578\\
87	0.143599060028991\\
88	0.24884118900429\\
89	0.265238721313174\\
90	0.223775295905549\\
91	0.0499184605029373\\
92	0.0780738339099536\\
93	0.188837262009207\\
94	0.0927700468375558\\
95	0.214890774430686\\
96	0.219035329192683\\
97	0.27081400407282\\
98	0.210179507916276\\
99	0.20516652153545\\
100	0.276124930281051\\
101	0.219383027057866\\
102	0.357047990005264\\
103	0.180632457335695\\
104	0.21972784265787\\
105	0.302179316213804\\
106	0.241441667805397\\
107	0.279404438964498\\
108	0.233438674884992\\
109	0.357901302114471\\
110	0.361517583530466\\
111	0.155806280615884\\
112	0.219597870107508\\
113	0.379977264387118\\
114	0.248319341714596\\
115	0.297896826251599\\
116	0.220767379910672\\
117	0.265708522202168\\
118	0.229259386833002\\
119	0.28771867169118\\
120	0.133564535005963\\
121	0.286206389465077\\
122	0.218340325664506\\
123	0.301161029290859\\
124	0.486990711699025\\
125	0.142324905310972\\
126	0.215826086374088\\
127	nan\\
128	0.353981154708094\\
129	-0.0876606655766194\\
130	0.0897218957308472\\
131	0.126823732267754\\
132	0.445427602160508\\
133	0.194566159875542\\
134	0.256991519393359\\
135	0.242616077792208\\
136	0.111505500768326\\
137	0.366342534221969\\
138	0.224050507906716\\
139	0.0985754872832232\\
140	0.127024023511359\\
141	0.188686797285021\\
142	0.26359114586076\\
143	0.285506229463265\\
144	0.203257044360471\\
145	0.179764512445963\\
146	0.228174544655024\\
147	0.159415813173573\\
148	0.302615196957305\\
149	0.261725871424861\\
150	0.218591227381236\\
151	0.18113793898798\\
152	0.127977569845308\\
153	0.179001249139115\\
154	0.239555167804789\\
155	0.194784539785957\\
156	0.198471636396744\\
157	0.124605854266245\\
158	0.0589386829418371\\
159	0.230550759393515\\
160	0.236902700794932\\
161	-0.113058079575813\\
162	-0.212985845877257\\
163	0.208586353127076\\
164	0.128044538071642\\
165	0.325275299611676\\
166	0.215486122422006\\
167	0.142800169454835\\
168	0.222559479106779\\
169	0.297829803928806\\
170	0.18924898234552\\
171	0.193770509019651\\
172	0.236328431833781\\
173	0.288120799220058\\
174	0.126672227132185\\
175	0.136845165867239\\
176	0.113668634996973\\
177	0.462061822423768\\
178	0.0502408372055615\\
179	0.196139557246953\\
180	0.277954127277983\\
181	0.184504125597655\\
182	0.102103725066033\\
183	0.177909702961505\\
184	0.124227438696616\\
185	0.118452085691134\\
186	0.0806303006329809\\
187	0.124861207164269\\
188	0.132552169186704\\
189	0.137411816495894\\
190	0.153995132510856\\
191	0.227685285676196\\
192	0.241506758769529\\
193	0.199515612126498\\
194	0.188686139412486\\
195	0.228379602058814\\
196	0.237324333998516\\
197	0.300319125077605\\
198	0.281072139830337\\
199	0.276135399852095\\
200	0.253488793280004\\
201	0.285504271180965\\
202	0.171387822471255\\
203	0.27154615662202\\
204	0.215994096185234\\
205	0.118481807337922\\
206	0.168023083443109\\
207	0.248443651567233\\
208	0.326728341757265\\
209	0.221154986645938\\
210	0.133839353551871\\
211	0.136878419892747\\
212	0.100470278727643\\
213	0.265640288597569\\
214	0.131917664913351\\
215	0.18014086988717\\
216	0.177975179035934\\
217	0.205897906850193\\
218	0.0827645501757977\\
219	0.144541424398236\\
220	0.299431168272583\\
221	0.214330692316746\\
222	0.246778315161637\\
223	0.22449342903404\\
224	0.211913452007816\\
225	0.173767486513819\\
226	0.386812150512532\\
227	0.136299598668391\\
228	0.276795310724196\\
229	0.093957495086345\\
230	0.190155990631332\\
231	nan\\
232	0.13456243263902\\
233	0.266971665229883\\
234	0.262750616268233\\
235	0.347257025196628\\
236	0.242411287810825\\
237	0.157847882121487\\
238	0.161660624395257\\
239	0.00240282950344152\\
240	0.0977244784159328\\
241	0.08793921931565\\
242	0.0825696869284111\\
243	0.188293785863016\\
244	0.236325653540559\\
245	0.172480752113236\\
246	0.0157650550512185\\
247	0.19220692833729\\
248	nan\\
249	0.247991358901486\\
250	0.0546080130429217\\
251	0.228637802663637\\
252	0.136383000925083\\
};
\addplot [color=dscr_plot_color,solid,line width=1.5pt,forget plot]
  table[row sep=crcr]{%
1	0.333768911677547\\
2	0.232727472512196\\
3	0.116003418925775\\
4	0.215428056281661\\
5	0.221651367048505\\
6	0.295577594094905\\
7	0.302622948454115\\
8	0.212614229441805\\
9	0.203897782621441\\
10	0.216710737439324\\
11	0.286118838314943\\
12	0.254478154886667\\
13	0.352074738164661\\
14	0.227737436907713\\
15	0.158455395450261\\
16	0.189556668605896\\
17	0.176027631670616\\
18	0.221673890089364\\
19	0.281797309611809\\
20	0.180142385389432\\
21	0.258122804274677\\
22	0.349488141899585\\
23	0.213851550819874\\
24	0.218444959937139\\
25	0.180183290415965\\
26	0.305180751825854\\
27	0.201939433280902\\
28	0.0920272419603295\\
29	0.179214109327316\\
30	0.112978938362586\\
31	0.155220863928893\\
32	0.152879316487404\\
33	0.124965794773272\\
34	0.178794410634309\\
35	0.226640116103109\\
36	0.160800171150487\\
37	0.11772465252331\\
38	0.383917476269602\\
39	0.295235919097192\\
40	0.173348804409072\\
41	0.33930547581791\\
42	0.282282878028601\\
43	0.305449256930086\\
44	0.227467800670768\\
45	0.256480418740308\\
46	0.169325811244331\\
47	0.308568373450275\\
48	0.168721701539817\\
49	0.150541367347649\\
50	0.209728663894474\\
51	0.149670481929516\\
52	0.20380574767565\\
53	0.248590605253363\\
54	0.191283231318807\\
55	0.262394776850264\\
56	0.18835938160362\\
57	0.144135022393152\\
58	0.458719122491559\\
59	0.328322654740615\\
60	0.129241793435649\\
61	0.148231240723555\\
62	0.153984723547677\\
63	0.176412025987317\\
64	0.256585142373645\\
65	0.296707290082554\\
66	0.216615736124191\\
67	0.391861267441821\\
68	0.539551082151974\\
69	0.381932499516413\\
70	0.19374053891556\\
71	0.189265385758965\\
72	0.510769133691676\\
73	0.4860766722571\\
74	0.434901837872652\\
75	0.364289203278551\\
76	0.382263698412221\\
77	0.421916862366552\\
78	0.470976858927345\\
79	0.232856649519314\\
80	0.310555098675914\\
81	0.223615540346806\\
82	0.395051098455704\\
83	0.325245896851892\\
84	0.273031040682037\\
85	0.272365011891521\\
86	0.164684847009387\\
87	0.156780790194564\\
88	0.282912733803722\\
89	0.166633515545989\\
90	0.173040849392663\\
91	0.12208622281088\\
92	0.0947661999218445\\
93	0.261955498874767\\
94	0.125055804156315\\
95	0.196422788020447\\
96	0.26160152179177\\
97	0.22893798390484\\
98	0.238335614964464\\
99	0.228498471249047\\
100	0.238334044277545\\
101	0.201928054118012\\
102	0.246041488795959\\
103	0.214430219738578\\
104	0.299664116995078\\
105	0.366473733680578\\
106	0.302133372992863\\
107	0.232177600536365\\
108	0.248595209240429\\
109	0.406193218599909\\
110	0.396003382499736\\
111	0.211418354595269\\
112	0.168549932837974\\
113	0.338452156382061\\
114	0.194493499240721\\
115	0.292511233319038\\
116	0.268471478622731\\
117	0.119209274157966\\
118	0.268405050549451\\
119	0.293597698373012\\
120	0.229391436912554\\
121	0.281351765324802\\
122	0.190421984864482\\
123	0.320019122184802\\
124	0.500882825729437\\
125	0.258168127608347\\
126	0.288642837129454\\
127	nan\\
128	0.296250080015567\\
129	0.247537697847365\\
130	0.100860578798212\\
131	0.204441407394484\\
132	0.611465214342782\\
133	0.359205774804975\\
134	0.24601898713452\\
135	0.241720110684394\\
136	0.224423730982619\\
137	0.305988730853051\\
138	0.243633638552224\\
139	0.11921399394659\\
140	0.156562062381879\\
141	0.134113042488412\\
142	0.266611715038007\\
143	0.253770583377862\\
144	0.193610712663937\\
145	0.111908954802266\\
146	0.12152313207422\\
147	0.182478773111015\\
148	0.275369655138293\\
149	0.246346687184395\\
150	0.229053550725161\\
151	0.182776466745665\\
152	0.160007290646792\\
153	0.20970990272689\\
154	0.246817649678624\\
155	0.195739910307888\\
156	0.229146372891214\\
157	0.180610373817526\\
158	0.116506756712066\\
159	0.271253408242694\\
160	0.232723622158607\\
161	0.0851512134271443\\
162	-0.21665706489368\\
163	0.206420233748082\\
164	0.178122696492527\\
165	0.345469368560939\\
166	0.205030777797029\\
167	0.144950239179236\\
168	0.320989347666679\\
169	0.231515837871226\\
170	0.23738881401316\\
171	0.199188252526374\\
172	0.256113712468798\\
173	0.239927922835006\\
174	0.155967877072355\\
175	0.144436324329898\\
176	0.0763658414985831\\
177	0.491003226431149\\
178	0.145318751398294\\
179	0.229928911782692\\
180	0.255313601647687\\
181	0.22302287025072\\
182	0.150244371017714\\
183	0.159384706676769\\
184	0.16842033604045\\
185	0.0693389467958679\\
186	0.216596099758733\\
187	0.17573546272982\\
188	0.253109541582453\\
189	0.154064031660957\\
190	0.213617156589859\\
191	0.32524689086249\\
192	0.18647972059609\\
193	0.17315070553096\\
194	0.208826256729927\\
195	0.275883432527285\\
196	0.236016421805217\\
197	0.289122125310656\\
198	0.236519644598414\\
199	0.23995119711274\\
200	0.316622693874422\\
201	0.334667899134426\\
202	0.190002958040152\\
203	0.259840042714512\\
204	0.203776192197029\\
205	0.166724697803691\\
206	0.15486690616195\\
207	0.216464628494421\\
208	0.34002218675414\\
209	0.269044858828246\\
210	0.160120053352544\\
211	0.16711692513657\\
212	0.0949436526703116\\
213	0.166791266125258\\
214	0.18123567302268\\
215	0.214317316756454\\
216	0.173972235563062\\
217	0.202964348332083\\
218	0.138156259099779\\
219	0.218869822815176\\
220	0.303252967796926\\
221	0.20734236891722\\
222	0.225295458148032\\
223	0.171392933134449\\
224	0.211171306349371\\
225	0.185110316986111\\
226	0.361321650227918\\
227	0.234325015260041\\
228	0.238319038825123\\
229	0.145888606827339\\
230	0.189702908840084\\
231	nan\\
232	0.141105032321184\\
233	0.235711900821984\\
234	0.265969117719761\\
235	0.351458488973552\\
236	0.276610963533824\\
237	0.163527875239468\\
238	0.199369637176612\\
239	0.136369455475376\\
240	0.190326157777735\\
241	0.157375171133857\\
242	0.118804195552336\\
243	0.205899398729245\\
244	0.290032432677362\\
245	0.235785658215783\\
246	0.170257244631804\\
247	0.223844571703125\\
248	nan\\
249	0.242059363952209\\
250	0.122399936418885\\
251	0.244697683538711\\
252	0.171282281300026\\
};
\addplot [color=cts_nFPC_plot_color,solid,line width=1.5pt,forget plot]
  table[row sep=crcr]{%
1	0.208572316405276\\
2	0.14833928522974\\
3	0.112452439639978\\
4	0.266337300392211\\
5	0.0120323403425431\\
6	0.263507675247356\\
7	0.159542111451932\\
8	0.152597362358569\\
9	0.116236238787322\\
10	0.111906663812486\\
11	0.1311569919059\\
12	-0.0343539076140332\\
13	0.356571793766579\\
14	0.0627084160617568\\
15	0.212827954547176\\
16	0.222806229356876\\
17	0.0968225395783486\\
18	0.15451505562834\\
19	0.0793682455558322\\
20	0.0397822573314476\\
21	0.110762201260052\\
22	0.170844495596471\\
23	0.149005569580097\\
24	0.145570801763429\\
25	0.190640439888381\\
26	0.253043448393966\\
27	0.154751160429114\\
28	0.133345399236251\\
29	0.133746447931057\\
30	0.118200213093476\\
31	0.144103563707981\\
32	0.116215484747289\\
33	0.131497807212047\\
34	0.0996512135734099\\
35	-0.0549622985166351\\
36	0.0721231779860896\\
37	0.102196150637255\\
38	0.00287862113606428\\
39	0.0208494400265195\\
40	0.000749602455799581\\
41	0.18867244657714\\
42	0.184339612217394\\
43	0.207618866290416\\
44	0.212603832376325\\
45	0.113499174942538\\
46	0.0481654063884207\\
47	0.186856852491959\\
48	-0.0161360324000987\\
49	0.0892194608781958\\
50	0.186334719461304\\
51	0.0847005969954608\\
52	0.0702741138471873\\
53	0.138739645340319\\
54	0.138794243722763\\
55	0.211046661359078\\
56	0.163104894946067\\
57	0.0856356077735918\\
58	0.408394207216064\\
59	0.0914488368687861\\
60	0.0380296947566958\\
61	0.0294228998748581\\
62	0.166476024030666\\
63	0.257163473238519\\
64	0.135411156767448\\
65	0.259342077165369\\
66	0.210238592799974\\
67	0.0569762377215822\\
68	0.276963533820283\\
69	0.181804041040413\\
70	0.136633927347281\\
71	-0.0402782396901753\\
72	0.309016162919673\\
73	0.0302506439555826\\
74	0.138678739895202\\
75	0.307497593288483\\
76	0.157561932092823\\
77	0.18113060327347\\
78	0.314389029527146\\
79	0.00319218274424889\\
80	0.0646736506132777\\
81	0.0341700496948436\\
82	0.219765253663634\\
83	0.0168812213505443\\
84	0.00312959509183796\\
85	0.0881096242257643\\
86	0.0889515348774606\\
87	0.0578848377611993\\
88	0.151225591361362\\
89	0.0770278067728074\\
90	0.0873721787275192\\
91	0.0192649108371895\\
92	-0.0199207831327442\\
93	0.0660045344247142\\
94	-0.0900714839478612\\
95	0.0957358110783347\\
96	0.211435980662948\\
97	0.170553392991275\\
98	-0.147577545552589\\
99	0.0435593690473405\\
100	0.0499510936997321\\
101	0.0732619841892531\\
102	0.103389362818078\\
103	0.0805971442513251\\
104	0.0851609928902873\\
105	0.0427475961705217\\
106	-0.0695817687491329\\
107	0.100067474922637\\
108	-0.0234100108960775\\
109	0.115083647101147\\
110	0.0366651071495101\\
111	-0.00937704416345261\\
112	0.0673434751655782\\
113	0.137811286778276\\
114	0.172219534500304\\
115	0.0460793659112324\\
116	0.0744653430963214\\
117	-0.00771272225708336\\
118	0.103347606461473\\
119	0.0889482130516056\\
120	0.0330905249723488\\
121	0.279867457454173\\
122	0.118022330372072\\
123	0.144893113599767\\
124	0.271632379521963\\
125	0.123755428697926\\
126	0.245750514955849\\
127	nan\\
128	0.266692876001423\\
129	-0.244235681819987\\
130	0.0584428802819006\\
131	-0.0649785285550654\\
132	0.32357095219598\\
133	0.0950832914557975\\
134	0.191990251504753\\
135	0.0592509676938719\\
136	-0.12083280985335\\
137	0.331592897306785\\
138	0.195492325654146\\
139	0.0853091511933448\\
140	0.0372583421801789\\
141	0.0057495130501054\\
142	0.222978662986317\\
143	0.218238920865679\\
144	0.139552311059381\\
145	0.117306901470082\\
146	0.0720915058273515\\
147	0.158372843596323\\
148	0.174946820643204\\
149	0.232737890283557\\
150	0.204408361011538\\
151	0.14351041590253\\
152	0.100026915910576\\
153	0.107359358407358\\
154	0.220238984703566\\
155	0.11698559777673\\
156	0.141341483813451\\
157	0.00104025284951189\\
158	0.102813320260989\\
159	0.0242294312795385\\
160	0.128381336205801\\
161	0.0475288005886188\\
162	-0.218377892608336\\
163	0.17480745917009\\
164	0.168280444898345\\
165	0.161280497137455\\
166	0.0916941129866418\\
167	0.114502127716107\\
168	0.159315346414182\\
169	0.15133785175757\\
170	0.0694699544451333\\
171	0.141162050949331\\
172	0.0577416837540809\\
173	0.168479267232314\\
174	0.0929424426533464\\
175	0.0852990274274164\\
176	0.0776316049787917\\
177	0.415652016493405\\
178	0.031513855835861\\
179	0.105654741837782\\
180	0.135461668120968\\
181	0.144714562517296\\
182	0.0576025843059079\\
183	0.173600558636567\\
184	-0.00523260697827171\\
185	0.087154855783982\\
186	0.0662483112219375\\
187	-0.0071577759611377\\
188	0.0422000258068876\\
189	0.0382348798195439\\
190	0.151101356189055\\
191	0.18325449536047\\
192	0.125986542549519\\
193	0.0956149280239598\\
194	0.156143797706234\\
195	0.155043443224061\\
196	0.142346556131763\\
197	0.202670282531865\\
198	0.112836216107313\\
199	0.154964682787758\\
200	0.215752745741765\\
201	0.140914921856394\\
202	0.125518735750116\\
203	0.211421195194753\\
204	0.158945781151614\\
205	0.0603606236469995\\
206	0.0771360863332413\\
207	0.149773839856891\\
208	0.234330247526507\\
209	0.114414228458558\\
210	0.0291387543337991\\
211	0.235770065325781\\
212	0.0560025416422469\\
213	0.152738882932125\\
214	0.0460055754477534\\
215	0.135205809403947\\
216	0.109687425959324\\
217	0.0632596699055367\\
218	0.0150814666044855\\
219	-0.0174249104440113\\
220	0.108274618131137\\
221	0.163423865818873\\
222	0.21067852307861\\
223	0.186867085363917\\
224	0.155710650387992\\
225	0.0639343568153517\\
226	0.095177681638619\\
227	0.0239651780991623\\
228	0.264918260755787\\
229	0.0787859516178995\\
230	0.14036520374532\\
231	nan\\
232	0.127015693063667\\
233	0.180376110288811\\
234	0.194639865496199\\
235	0.0853837699158635\\
236	0.162784009425773\\
237	0.0952189101879141\\
238	0.241323865255618\\
239	0.0535040728906561\\
240	-0.0158618596943482\\
241	-0.00559769091562715\\
242	-0.0769981792985474\\
243	0.0614324396865506\\
244	0.0249844295403188\\
245	0.0681306089187595\\
246	0.0189679015240573\\
247	0.109719257209715\\
248	nan\\
249	0.216251968542567\\
250	0.00883984459256003\\
251	0.14948722704969\\
252	0.0751563761642884\\
};
\addplot [color=dscr_nFPC_plot_color,solid,line width=1.5pt,forget plot]
  table[row sep=crcr]{%
1	0.392998953875574\\
2	0.285368606498711\\
3	0.133775733676425\\
4	0.21359681628222\\
5	0.201399845217861\\
6	0.339070939052882\\
7	0.315383735662621\\
8	0.239336691481629\\
9	0.219384772430487\\
10	0.207565504343158\\
11	0.290470768193617\\
12	0.290423438654441\\
13	0.328122256717569\\
14	0.24587888372209\\
15	0.155476747066635\\
16	0.355657104331459\\
17	0.160744281254348\\
18	0.203565570229359\\
19	0.261012375652109\\
20	0.130010788902493\\
21	0.216354552062287\\
22	0.316665103893108\\
23	0.290370518183633\\
24	0.193269063478001\\
25	0.276208868963551\\
26	0.277597836948956\\
27	0.196126136400166\\
28	0.11349098926935\\
29	0.212652184598576\\
30	0.117545974831672\\
31	0.139699896592418\\
32	0.159847478042831\\
33	0.133223239455452\\
34	0.152081738806074\\
35	0.108932186949634\\
36	0.158272442957345\\
37	0.135098806472973\\
38	0.407475422927014\\
39	0.340596364007856\\
40	0.146528395458194\\
41	0.1234492870125\\
42	0.25654496666507\\
43	0.289233904646697\\
44	0.247735002262201\\
45	0.240336619229197\\
46	0.143454251629071\\
47	0.310608795783249\\
48	0.16908728329742\\
49	0.157904424351091\\
50	0.211968229501408\\
51	0.122311820944972\\
52	0.218908704779559\\
53	0.215069964389261\\
54	0.21363027007528\\
55	0.271944245736794\\
56	0.172953546312389\\
57	0.268530698047401\\
58	0.465879678575854\\
59	0.319761446367702\\
60	0.130171721579893\\
61	0.281651528310274\\
62	0.1514660856373\\
63	0.160234782667765\\
64	0.26024197970397\\
65	0.267193189518822\\
66	0.104453406427488\\
67	0.00925209259343471\\
68	0.602378205214577\\
69	0.348672128673846\\
70	0.167933010987475\\
71	0.202962381262112\\
72	0.564575867837199\\
73	0.526953874374029\\
74	0.483076804863291\\
75	0.40592573422906\\
76	0.450704573752827\\
77	0.478595425326174\\
78	0.524678804006999\\
79	0.160192946084838\\
80	0.330887311658275\\
81	0.224868763723493\\
82	0.402417161254912\\
83	0.262856350143555\\
84	0.298335414267989\\
85	0.186030426971146\\
86	0.218686413659908\\
87	0.161679122883166\\
88	0.298712411628303\\
89	0.187624791269267\\
90	0.210848849546735\\
91	0.111775903175298\\
92	0.0733470046105282\\
93	0.251997400631176\\
94	0.104140075851183\\
95	0.196334991050354\\
96	0.18226364843081\\
97	0.187900142499613\\
98	0.18585761354001\\
99	0.258802893195389\\
100	0.221021678152658\\
101	0.187889420231888\\
102	0.244225404564783\\
103	0.237233354539624\\
104	0.170818410167884\\
105	0.350748335622377\\
106	0.22076585341228\\
107	0.184624110460172\\
108	0.147493383880028\\
109	0.397213154052382\\
110	0.38050942841072\\
111	0.287599691147976\\
112	0.292371199387385\\
113	0.382625586115085\\
114	0.218038262062834\\
115	0.300346564068007\\
116	0.26624167279707\\
117	0.0951150743617846\\
118	0.236128349606966\\
119	0.280489454815599\\
120	0.127231324622025\\
121	0.280465625058357\\
122	0.173384669960189\\
123	0.29406996940685\\
124	0.135824863648377\\
125	0.15889807939581\\
126	0.219640188807317\\
127	nan\\
128	0.317359403967283\\
129	0.0867327775729016\\
130	0.081373840927681\\
131	0.21801929216247\\
132	0.650223143734301\\
133	0.300245621994336\\
134	0.128432197823438\\
135	0.257587675598922\\
136	0.186075432242773\\
137	0.371034112756539\\
138	0.186287192453459\\
139	0.113168531279885\\
140	0.121205323439328\\
141	0.115248498166329\\
142	0.287469105525996\\
143	0.260426548245732\\
144	0.198388642410851\\
145	0.143764244194818\\
146	0.19610083870649\\
147	0.133399046766265\\
148	0.281447435665594\\
149	0.226387144400128\\
150	0.251970447565208\\
151	0.17177546502284\\
152	0.157911648767173\\
153	0.216096103343097\\
154	0.257813156331301\\
155	0.271761369679223\\
156	0.217751631781759\\
157	0.105244542283106\\
158	0.102721124453486\\
159	0.291332892545082\\
160	0.259070393594845\\
161	0.0533171627521759\\
162	-0.220117010168094\\
163	0.189805774799252\\
164	0.166892333854392\\
165	0.319451689476314\\
166	0.203190135261707\\
167	0.133604856769013\\
168	0.376788700006355\\
169	0.311636027059345\\
170	0.247160964093518\\
171	0.173704092384282\\
172	0.31883034771726\\
173	0.248310717016367\\
174	0.179210605427053\\
175	0.124172132089872\\
176	0.0937191585444825\\
177	0.520066513706459\\
178	0.105735051500734\\
179	0.209705603744315\\
180	0.248620509281717\\
181	0.184004216722346\\
182	0.121080689274207\\
183	0.266215142127721\\
184	0.131353850513838\\
185	0.13004822913655\\
186	0.163066711670286\\
187	0.249120897411268\\
188	0.19510464035822\\
189	0.137426930058441\\
190	0.191562299748405\\
191	0.313463706996225\\
192	0.213103100447217\\
193	0.197744270395187\\
194	0.336783272859535\\
195	0.240875614514517\\
196	0.267654938880436\\
197	0.331018939960578\\
198	0.256115820408796\\
199	0.227763698143317\\
200	0.27059480946856\\
201	0.313451252411457\\
202	0.150141782483742\\
203	0.159315860196636\\
204	0.173230968063198\\
205	0.135937989592514\\
206	0.15667652721982\\
207	0.21466150600947\\
208	0.330313347043268\\
209	0.285985317619785\\
210	0.162361064419463\\
211	0.14191921846279\\
212	0.0919663256607726\\
213	0.16012823404491\\
214	0.191388696629016\\
215	0.216777385045652\\
216	0.252034413475323\\
217	0.186593841718372\\
218	0.130670628415025\\
219	0.208120387485163\\
220	0.308974178041681\\
221	0.19548874231866\\
222	0.151322423834585\\
223	0.168600042310865\\
224	0.221830038238226\\
225	0.212631475241632\\
226	0.423970353904841\\
227	0.14525549721858\\
228	0.28781921905292\\
229	0.144905861318691\\
230	0.217448528161204\\
231	nan\\
232	0.30634138144378\\
233	0.206408544757044\\
234	0.271443889056105\\
235	0.374295618746448\\
236	0.20200713095419\\
237	0.136261154465944\\
238	0.200976750226435\\
239	0.0632290851564593\\
240	0.214059118605928\\
241	0.14703172020519\\
242	0.115243106781836\\
243	0.170407332585463\\
244	0.293577089435767\\
245	0.200069357471978\\
246	0.112351472199517\\
247	0.224409256551706\\
248	nan\\
249	0.238828167014954\\
250	0.12115279753739\\
251	0.253506331650686\\
252	0.172527085171514\\
};
\end{axis}
\end{tikzpicture}%
 
\end{subfigure}\\

\leavevmode\smash{\makebox[0pt]{\hspace{-7em}% HORIZONTAL POSITION           
  \rotatebox[origin=l]{90}{\hspace{20em}% VERTICAL POSITION
    Normalized PnL}%
}}\hspace{0pt plus 1filll}\null

Trading Day Number of 2013

\vspace{1cm}
\begin{subfigure}{\linewidth}
  %\centering
  \setlength\figureheight{\linewidth} 
  \setlength\figurewidth{\linewidth}
  \tikzsetnextfilename{strategylegend}
  \resizebox{\linewidth}{!}{\definecolor{mycolor1}{rgb}{0.25098,0.00000,0.38824}%
\definecolor{mycolor2}{rgb}{0.00000,0.46275,0.00000}%
\definecolor{mycolor3}{rgb}{0.00000,0.34902,0.34902}%
\definecolor{mycolor4}{rgb}{0.58039,0.26275,0.00000}%
\begin{tikzpicture}
    \begingroup
    % inits/clears the lists (which might be populated from previous
    % axes):
    \csname pgfplots@init@cleared@structures\endcsname
    \pgfplotsset{legend style={at={(0,1)},anchor=north west},legend columns=-1,legend style={draw=black,column sep=1ex},
            legend entries={Cts Stoch Ctrl,Dscr Stoch Ctrl,Cts Stoch Ctrl w nFPC,Dscr Stoch Ctrl w nFPC}}%
    
    \csname pgfplots@addlegendimage\endcsname{line width=2pt,mycolor1,sharp plot}
    \csname pgfplots@addlegendimage\endcsname{line width=2pt,mycolor2,sharp plot}
    \csname pgfplots@addlegendimage\endcsname{line width=2pt,mycolor3,sharp plot}
    \csname pgfplots@addlegendimage\endcsname{line width=2pt,mycolor4,sharp plot}

    % draws the legend:
    \csname pgfplots@createlegend\endcsname
    \endgroup
\end{tikzpicture}
}
\end{subfigure}%
  \caption{End of day strategy performances: in-sample backtesting using same-day calibration.}
  \label{fig:IS_sameday_comp}
\end{figure}

\begin{table}
\centering
\ra{1.2}
\begin{tabular}{@{} *{8}{r} @{}}
\toprule
Strategy & Return & Sharpe & Trades & Inv & \% Win & Max Loss & Max Win \\
\midrule
\multicolumn{8}{l}{\texttt{FARO}} \\
Naive & -0.879 & -0.808 & 413 & 0.47 & 0.07 & -7.109 & 5.715 \\ 
Naive+ & 0.101 & 0.107 & 213 & 2.45 & 0.74 & -8.797 & 5.336 \\ 
Naive++ & 0.002 & 0.021 & 7 & 0.17 & 0.50 & -0.842 & 0.320 \\ 
Cont & -0.059 & -0.551 & 201 & 0.09 & 0.18 & -0.912 & 0.071 \\ 
Dscr & -0.064 & -0.695 & 210 & -0.02 & 0.08 & -0.914 & 0.440 \\ 
Cont w nFPC & -0.063 & -0.571 & 204 & 0.08 & 0.14 & -1.161 & 0.077 \\ 
Dscr w nFPC & -0.060 & -0.662 & 209 & -0.03 & 0.09 & -0.716 & 0.539 \\[2ex]
\multicolumn{8}{l}{\texttt{NTAP}} \\
Naive & -0.188 & -0.316 & 842 & -9.81 & 0.23 & -3.238 & 3.524 \\ 
Naive+ & 0.388 & 0.169 & 3562 & -9.73 & 0.74 & -19.367 & 10.201 \\ 
Naive++ & -0.005 & -0.012 & 157 & -0.90 & 0.54 & -2.888 & 2.558 \\ 
Cont & -0.006 & -0.062 & 2265 & 0.40 & 0.56 & -0.441 & 0.215 \\ 
Dscr & 0.099 & 0.767 & 1872 & 4.74 & 0.86 & -0.126 & 1.042 \\ 
Cont w nFPC & -0.141 & -0.951 & 2897 & 0.65 & 0.14 & -0.935 & 0.244 \\ 
Dscr w nFPC & 0.121 & 0.881 & 1738 & 2.82 & 0.89 & -0.139 & 0.962 \\[2ex]
\multicolumn{8}{l}{\texttt{ORCL}} \\
Naive & -0.105 & -0.253 & 484 & 1.40 & 0.28 & -1.837 & 2.180 \\ 
Naive+ & -0.034 & -0.011 & 4086 & -55.18 & 0.61 & -17.501 & 18.400 \\ 
Naive++ & 0.002 & 0.006 & 132 & 0.61 & 0.52 & -0.798 & 2.636 \\ 
Cont & 0.115 & 1.348 & 1874 & 1.94 & 0.92 & -0.217 & 0.521 \\ 
Dscr & 0.135 & 1.620 & 1898 & 3.93 & 0.98 & -0.063 & 0.515 \\ 
Cont w nFPC & -0.010 & -0.100 & 2455 & 1.32 & 0.48 & -0.478 & 0.503 \\ 
Dscr w nFPC & 0.144 & 1.501 & 1759 & 2.85 & 0.97 & -0.032 & 0.573 \\[2ex]
\multicolumn{8}{l}{\texttt{INTC}} \\
Naive & -0.082 & -0.228 & 258 & -5.21 & 0.33 & -1.465 & 1.425 \\ 
Naive+ & 0.365 & 0.134 & 3962 & -32.50 & 0.63 & -11.202 & 11.669 \\ 
Naive++ & -0.001 & -0.003 & 74 & -0.84 & 0.48 & -1.314 & 1.264 \\ 
Cont & 0.214 & 2.159 & 1577 & 5.17 & 0.97 & -0.213 & 0.487 \\ 
Dscr & 0.232 & 2.528 & 1642 & 4.48 & 0.98 & -0.217 & 0.611 \\ 
Cont w nFPC & 0.114 & 1.218 & 1894 & 2.01 & 0.90 & -0.244 & 0.416 \\ 
Dscr w nFPC &  0.226 & 2.202 & 1569 & 4.28 & 0.98 & -0.220 & 0.650 \\ 
\bottomrule
\end{tabular}
\caption{Averaged strategy performance results: in-sample backtesting using same-day calibration.}
\label{tbl:IS_sameday}
\end{table}

\fxnote{commentary needed. compare stoch methods}

\FloatBarrier
\subsection{Week Offset Calibration}
The next type of in-sample backtesting done was to calibrate for each ticker and each trading day of 2013, and to use the results to backtest on the date given by the calibration date shifted forward 7 days. Thus, the calibration obtained on Monday, January 2, 2013 would be used to backtest on Monday, January 9, 2013. Performance values are given in \autoref{tbl:IS_week}, and \autoref{fig:IS_week_comp} compares the day-over-day performance of the various strategies. 

Most of the observations from the previous section apply here. Chiefly, the illiquid stock \texttt{FARO} produces negative PnL and the low-liquidity stock \texttt{NTAP} approximately breaks even. As expected, the week offset calibration underperforms same-day calibration, but remarkably the difference is very small: in the case of \texttt{INTC}, the discrete time controller still generates a Sharpe ratio of approximately 2.5, and in this case only returned negative PnL once during the trading year. 

The similarity of the results can be interpreted in several ways. First, it is possible that trading behaviour is stable across days of the week, such that substituting one Monday for another yields a similar calibration. This is readily testable by calibrating on a given day and backtesting on the subsequent trading day, instead of a one-week offset. On the other hand, even with dissimilar data, it's possible that the calculation of $\delta^\pm$ is stable with respect to day-over-day fluctuations of data.

\begin{figure}
\centering
\begin{subfigure}{.45\linewidth}
  \centering
  \setlength\figureheight{\linewidth} 
  \setlength\figurewidth{\linewidth}
  \tikzsetnextfilename{IS_week_FARO}
  % This file was created by matlab2tikz.
%
%The latest updates can be retrieved from
%  http://www.mathworks.com/matlabcentral/fileexchange/22022-matlab2tikz-matlab2tikz
%where you can also make suggestions and rate matlab2tikz.
%
%
\begin{tikzpicture}[trim axis left, trim axis right]

\begin{axis}[%
width=\figurewidth,
height=\figureheight,
at={(0\figurewidth,0\figureheight)},
scale only axis,
every outer x axis line/.append style={black},
every x tick label/.append style={font=\color{black}},
xmin=1,
xmax=240,
%xlabel={Time (h)},
every outer y axis line/.append style={black},
every y tick label/.append style={font=\color{black}},
ymin=-1.1,
ymax=1.1,
%ylabel={Normalized PnL},
title={FARO},
axis background/.style={fill=white},
axis x line*=bottom,
axis y line*=left,
yticklabel style={
        /pgf/number format/fixed,
        /pgf/number format/precision=3
},
scaled y ticks=false,
legend style={legend cell align=left,align=left,draw=black,font=\small, legend pos=north west},
every axis legend/.code={\renewcommand\addlegendentry[2][]{}}  %ignore legend locally
]
\addplot [color=cts_plot_color,solid,line width=1.5pt]
  table[row sep=crcr]{%
1	-0.0989928992435463\\
2	-0.0688180757696504\\
3	-0.667895329957383\\
4	-0.0630984607302803\\
5	-0.0804420804978697\\
6	-0.0238517422661603\\
7	-0.0638804461775067\\
8	-0.0912660237796769\\
9	-0.0490205942512218\\
10	-0.0252396718205314\\
11	-0.0360152481078483\\
12	-0.117630685374556\\
13	-0.00905645976076215\\
14	0.0305930218094014\\
15	0.00454532675490999\\
16	-0.0474566329727204\\
17	-0.0277740698587635\\
18	0.0104523122719218\\
19	-0.0471103202144804\\
20	-0.0466657064669461\\
21	-0.0379092006431071\\
22	-0.0962090318821978\\
23	-0.0667128402210335\\
24	0.00458328159485368\\
25	-0.0340792833717887\\
26	-0.0524606181198134\\
27	-0.101686333308474\\
28	-0.0930262650903495\\
29	-0.038440417353103\\
30	0.00646484296662299\\
31	0.0236287464914939\\
32	-0.106015086663405\\
33	-0.954925663307111\\
34	-0.393668338263199\\
35	-0.352469294096596\\
36	0.0309633242908584\\
37	-0.132497830787876\\
38	-0.0872340928201544\\
39	-0.0283111451658453\\
40	0.0118032806624083\\
41	-0.116258049308536\\
42	-0.097156710139102\\
43	-0.0129087881285822\\
44	-0.0396811580752417\\
45	0.0384162219195533\\
46	-0.0932070169147991\\
47	-0.0301884358829766\\
48	-0.061222235792448\\
49	-0.0703213827773295\\
50	-0.0246499622256796\\
51	0.0137850210254537\\
52	-0.00719359648918848\\
53	-0.0999153985217064\\
54	-0.0693644053212767\\
55	-0.0233426349342\\
56	-0.116730392046449\\
57	-0.0482703093536527\\
58	-0.0741330791335301\\
59	-0.0218726798057195\\
60	-0.00162410122724674\\
61	0.0331777422327288\\
62	0.00363315681677933\\
63	-0.155468008244286\\
64	-0.000435368998228069\\
65	-0.0261484374669145\\
66	-0.102217855734003\\
67	-0.0227363366769885\\
68	-0.133455366021963\\
69	-0.0269378322323864\\
70	-0.0195586074005721\\
71	0.0761346958181762\\
72	-0.0590198958472396\\
73	0.0767273777479881\\
74	-0.0432146734715059\\
75	-0.271416961782643\\
76	-0.0377477696742255\\
77	-0.0415751562377729\\
78	-0.0114566831881186\\
79	-0.172023901074782\\
80	-0.335974961081289\\
81	-0.0892871783058809\\
82	0.0538948598614626\\
83	-0.0365041983473961\\
84	-0.032151216144696\\
85	-0.118950280616404\\
86	-0.027891574310685\\
87	-0.130260925782951\\
88	-0.137685456840875\\
89	-0.0940534331095446\\
90	-0.211819331162252\\
91	-0.0483104244035162\\
92	-0.0168887195686571\\
93	-0.0190066197018946\\
94	-0.00888131484237648\\
95	-0.0640431796960462\\
96	-0.124070638642356\\
97	-0.0253571972739885\\
98	-0.0154399222921771\\
99	-0.189194549464084\\
100	-0.140447138688782\\
101	-0.0982887168231367\\
102	-0.00717265561770279\\
103	-0.0361224416123228\\
104	-0.168915984467289\\
105	-0.0495988604924424\\
106	0.0546528494487437\\
107	-0.012350967857789\\
108	-0.168601836127504\\
109	-0.0897402428653327\\
110	-0.0672065239533574\\
111	-0.0663273135132303\\
112	0.0247811747852366\\
113	-0.0828194756305478\\
114	0.0245182018522123\\
115	0.0397933168347158\\
116	0.062974447202189\\
117	-0.0271910791886851\\
118	-0.127458547111586\\
119	-0.05678167212853\\
120	0.0131570838509638\\
121	-0.0549489467788073\\
122	0.0129310344827583\\
123	-0.0323243914044978\\
124	-0.0543232834024399\\
125	-0.070463814773802\\
126	-0.0191957261505864\\
127	-0.00260670643491442\\
128	-0.0320702661926855\\
129	0.0063671403302416\\
130	-0.0113211497883461\\
131	-0.0367713400427507\\
132	-0.0594025300019686\\
133	-0.0642285458218825\\
134	-0.087358436414643\\
135	-0.0647255398601378\\
136	-0.445433279612283\\
137	-0.0385032218541256\\
138	0.00260985539699599\\
139	-0.0373202341780704\\
140	-0.00421998109767481\\
141	-0.0621682396887828\\
142	-0.0633737754817796\\
143	-0.0265361314481719\\
144	-0.0119949039905129\\
145	-0.0247057280739619\\
146	0.00727289519933497\\
147	-0.0177507320566373\\
148	-0.00847097312293046\\
149	0.00596429862640786\\
150	-0.027882114658807\\
151	-0.0494771736983482\\
152	-0.0298202347050229\\
153	-0.0190028326438429\\
154	-0.0331645592293618\\
155	-0.0370638724578523\\
156	-0.00901903851536128\\
157	-0.00814933180142084\\
158	0.019944331897309\\
159	-0.0646841749373628\\
160	-0.0594321816054181\\
161	-0.00903343215215287\\
162	-0.0396062748802371\\
163	-0.0081865280186242\\
164	-0.016514668612421\\
165	-0.0174400832815655\\
166	-0.0118513266090668\\
167	-0.0144099633674281\\
168	0.00432651465202523\\
169	-0.151139081185435\\
170	-0.0302992601490747\\
171	-0.0979672622465193\\
172	-0.0942754486622191\\
173	-0.0182954943922059\\
174	-0.0199049988690337\\
175	-0.0227851169555195\\
176	0.00710088148873564\\
177	-0.107164761151541\\
178	-0.155143080609608\\
179	-0.21007149595768\\
180	0.0205934345070606\\
181	-0.0216214277949158\\
182	0.0337096204960389\\
183	-0.028289187238836\\
184	-0.0262910610830008\\
185	0.0004377233990683\\
186	-0.00109071305727747\\
187	-0.03023765742818\\
188	-0.0206927050990663\\
189	-0.0137475930099623\\
190	0.02665776778776\\
191	-0.0818645181863823\\
192	-0.0381153134578919\\
193	0.0287286464700551\\
194	0.0230054256027323\\
195	0.0194148115766118\\
196	-0.0407011609226884\\
197	0.00289020608165716\\
198	0.0375356378763503\\
199	-0.0738193540642478\\
200	-0.37155921794098\\
201	-0.442696758209487\\
202	-0.156069615014703\\
203	-0.0457175362190334\\
204	-0.129216531434065\\
205	-0.198209611029361\\
206	-0.0983047051566051\\
207	0.0675492324282333\\
208	-0.213569427969743\\
209	0.0187020759304281\\
210	0.0529215326435023\\
211	0.0145119890494427\\
212	-0.0621550140246297\\
213	-0.104098893949249\\
214	-0.00834901235061623\\
215	0.0376713349480656\\
216	0.0258067182602828\\
217	-0.012936941253008\\
218	-0.0670234402278327\\
219	-0.0870676977658279\\
220	-0.200004204721159\\
221	-0.0422636938307848\\
222	-0.0274776526235301\\
223	-0.0340701501302789\\
224	-0.0289366581669569\\
225	-0.0653807276375892\\
226	0.022921097403135\\
227	-0.023858132599337\\
228	0.0219377068609881\\
229	-0.0243773524026548\\
230	-0.179711176723372\\
231	-0.0682080981424351\\
232	-0.0299513217494404\\
233	0.0369695198259879\\
234	-0.0261613586243667\\
235	-0.0783588174899534\\
236	-0.125299794306574\\
237	-0.189118515394342\\
238	-0.245097122884676\\
239	-0.0588247368973214\\
240	0.00712136618536608\\
};
\addlegendentry{Cts Stoch Ctrl};

\addplot [color=dscr_plot_color,solid,line width=1.5pt]
  table[row sep=crcr]{%
1	-0.0995966207847469\\
2	-0.0955400522120556\\
3	-0.512177498035373\\
4	-0.0967261190740271\\
5	-0.211912683247003\\
6	-0.0338607079611349\\
7	-0.116836593809303\\
8	-0.0409623574512794\\
9	-0.0425852649052626\\
10	-0.00129863427129358\\
11	-0.0660318197644637\\
12	0.0178080601603774\\
13	0.00342127380272925\\
14	-0.0111056135298548\\
15	-0.00435331133027751\\
16	-0.0237677600366383\\
17	-0.0510455454596241\\
18	-0.0286322627025294\\
19	-0.061962302384862\\
20	-0.121955195592133\\
21	-0.0529510596123478\\
22	-0.0200861021004944\\
23	-0.0361574715984256\\
24	-0.0431417467617645\\
25	-0.0648805322537758\\
26	-0.0167552813269773\\
27	-0.17986021889352\\
28	-0.0789448859615183\\
29	-0.0748290530065737\\
30	-0.0077930343590699\\
31	-0.149178049209888\\
32	-0.198054736337988\\
33	-1.0038236725938\\
34	-0.262097390911982\\
35	-0.239828027358685\\
36	-0.0124696859560573\\
37	-0.226144936698071\\
38	-0.0735066525503359\\
39	-0.0700205658496726\\
40	-0.0222913865291784\\
41	-0.0589095689411235\\
42	-0.0716890864347401\\
43	-0.0270988990549223\\
44	-0.0649422858573774\\
45	-0.0200871083350892\\
46	-0.132927008950368\\
47	-0.0922801810570446\\
48	-0.140860871266921\\
49	-0.144633073706183\\
50	-0.0340617688762246\\
51	-0.0681894912495955\\
52	-0.0176941658313543\\
53	-0.0975432358447838\\
54	-0.0643685155820067\\
55	-0.0505340692711008\\
56	-0.0805893343847505\\
57	-0.0572397681277407\\
58	-0.0663337478075647\\
59	-0.0412542044959246\\
60	0.0586294576198538\\
61	-0.0407102358099261\\
62	-0.041223148810457\\
63	-0.110900958494676\\
64	-0.0218544140904249\\
65	-0.042424893896799\\
66	-0.0953951082001896\\
67	-0.0346788590652774\\
68	-0.0658652561021561\\
69	-0.0880310618608504\\
70	-0.0727069410057454\\
71	0.0120642370796955\\
72	-0.109167350631398\\
73	-0.0201956017728097\\
74	-0.103969236682258\\
75	-0.359600927095318\\
76	-0.0300705348679874\\
77	-0.097839934921383\\
78	-0.0302775675279979\\
79	-0.141431348090031\\
80	-0.247613824379903\\
81	-0.0326466026311607\\
82	-0.0125117058597183\\
83	-0.07965222658458\\
84	-0.0570558236770073\\
85	-0.0493518547689478\\
86	-0.0416719787334401\\
87	-0.0579109075868733\\
88	-0.0639626690665008\\
89	-0.113038765677644\\
90	-0.178784398205381\\
91	-0.108905080503451\\
92	-0.0328384692683799\\
93	-0.104763872903824\\
94	0.00191596037330389\\
95	-0.0888854622250821\\
96	-0.0683551412843746\\
97	-0.099180498794051\\
98	-0.0509256190405727\\
99	-0.15026328364107\\
100	-0.0742637153916431\\
101	-0.0516295715497802\\
102	0.0242770366344454\\
103	0.0839232096717357\\
104	-0.179769057501029\\
105	-0.0258275395712541\\
106	0.0424755855636558\\
107	0.00248663600060657\\
108	-0.0289275930632835\\
109	-0.00369395621058976\\
110	-0.106396466252579\\
111	-0.0982966626263425\\
112	-0.0219554961838324\\
113	-0.0717743345283962\\
114	0.00943020367554314\\
115	0.00722407034168151\\
116	0.0483846617685872\\
117	-0.0884769242981952\\
118	-0.184789645876706\\
119	-0.0647511812312117\\
120	0.0070545849437567\\
121	-0.0767806901249797\\
122	-0.0168354487167097\\
123	-0.0040624654056175\\
124	-0.018566212068315\\
125	-0.126989593539228\\
126	-0.0424639278394763\\
127	-0.0556742845038517\\
128	-0.0907343854584827\\
129	-0.0645519926408665\\
130	-0.0194634235792512\\
131	-0.0628063420991571\\
132	-0.0697900275459092\\
133	-0.119152212217465\\
134	-0.0658455382524425\\
135	-0.0691443833399853\\
136	-0.312455111232868\\
137	-0.0771505696756136\\
138	-0.0123734576878019\\
139	-0.0378580518425196\\
140	-0.022769687717831\\
141	-0.0376493770356853\\
142	-0.0481374827047047\\
143	-0.0531258653283602\\
144	-0.0133596073462276\\
145	-0.0191608129577023\\
146	-0.0877179476605288\\
147	-0.0204313519799799\\
148	-0.0434196171494139\\
149	-0.0408269411776303\\
150	-0.0249320375719999\\
151	-0.0387166586153994\\
152	-0.0377281723270208\\
153	-0.0418160700872973\\
154	-0.00210806397559851\\
155	-0.0833320090062677\\
156	-0.030795451904264\\
157	-0.0352537916958394\\
158	-0.0667973037356563\\
159	-0.105641535465567\\
160	-0.0616612955004078\\
161	-0.029680134509237\\
162	-0.0586846507048888\\
163	-0.0205017613844083\\
164	-0.0298111638411698\\
165	-0.0304415903616888\\
166	-0.0236533784645829\\
167	-0.0301779675613731\\
168	-0.0182722644251138\\
169	-0.104992252705746\\
170	-0.0498070818522068\\
171	-0.051990351056985\\
172	-0.0469572620315705\\
173	-0.0366764619292028\\
174	-0.0424399731627968\\
175	-0.049488136588733\\
176	-0.0170293407156964\\
177	-0.0742677827912091\\
178	-0.103310744403417\\
179	0.0623521135985585\\
180	-0.0637176018713269\\
181	0.00181226441765912\\
182	-0.0151405642783778\\
183	-0.112582882290347\\
184	-0.0710612531919435\\
185	-0.0826123217636468\\
186	-0.0341104834331987\\
187	-0.0624070115857106\\
188	-0.0445500207790483\\
189	-0.0286283388943652\\
190	-0.0198824777312811\\
191	-0.0520016571908172\\
192	-0.0776483417156276\\
193	-0.0231520235268523\\
194	-0.0981868548338814\\
195	-0.0908561904747453\\
196	-0.0439273688123807\\
197	-0.0614124806165277\\
198	-0.0287415819349967\\
199	-0.107678632634983\\
200	-0.388457313733346\\
201	-0.365425918679538\\
202	-0.247662127724195\\
203	-0.0886696315432309\\
204	-0.214749939117925\\
205	-0.289661531025451\\
206	-0.12464695015031\\
207	0.0710077160828214\\
208	-0.124546382275244\\
209	-0.088345351154684\\
210	-0.041134188195587\\
211	-0.0420258755838985\\
212	-0.051193064389229\\
213	-0.0945689302627227\\
214	-0.103157097432378\\
215	-0.0215759017161257\\
216	-0.0515873728442342\\
217	-0.0897554795563876\\
218	-0.0690889866451098\\
219	-0.188361031378949\\
220	-0.123656524933847\\
221	-0.0680015348507405\\
222	-0.0111379335705933\\
223	-0.0971871659578753\\
224	-0.0542251743875227\\
225	-0.0750576630075023\\
226	-0.030428502194894\\
227	-0.0943968501723815\\
228	-0.0368365474947934\\
229	-0.0329932161979989\\
230	-0.175537100210999\\
231	-0.089484283073257\\
232	-0.0608288964275909\\
233	-0.0094390721646589\\
234	-0.0552722142990639\\
235	-0.0993661819923027\\
236	-0.0519124321229057\\
237	-0.102529599538556\\
238	-0.541925451157354\\
239	-0.109779154154819\\
240	-0.00811388632014382\\
};
\addlegendentry{Dscr Stoch Ctrl};

\addplot [color=cts_nFPC_plot_color,solid,line width=1.5pt]
  table[row sep=crcr]{%
1	-0.0871127387607132\\
2	-0.0671621857356237\\
3	-0.632697555545204\\
4	-0.0694814356123973\\
5	-0.0691230412178578\\
6	-0.0422207460143579\\
7	-0.0648862465784592\\
8	-0.10752398450512\\
9	-0.0755606342417569\\
10	-0.0417598794169953\\
11	-0.0154112824444694\\
12	-0.117615331477479\\
13	-0.0125899945470681\\
14	0.0115857485707156\\
15	0.00454236419622189\\
16	-0.0534978348494885\\
17	-0.0243232067686003\\
18	0.00278687631675928\\
19	-0.0452104403771012\\
20	-0.0646148612474375\\
21	-0.046210052065509\\
22	-0.0706441967021038\\
23	-0.0526246175332645\\
24	-0.000945564443374598\\
25	-0.0228867026732935\\
26	-0.0663359752444718\\
27	-0.087656575963981\\
28	-0.127380328273489\\
29	-0.0452214790818072\\
30	0.00405322635364281\\
31	0.00052020722235808\\
32	-0.108281849441816\\
33	-1.02176778576647\\
34	-0.376114842190908\\
35	-0.349716633911133\\
36	0.0175171279812335\\
37	-0.158787510706624\\
38	-0.0712138361658776\\
39	-0.078729309131304\\
40	0.00826410467480504\\
41	-0.11846550144202\\
42	-0.074061894643235\\
43	4.91571086225418e-05\\
44	-0.0580695561279427\\
45	0.0427260606103892\\
46	-0.0944141348480918\\
47	-0.0344478691372178\\
48	-0.0648990002944799\\
49	-0.0710173787321302\\
50	-0.0401795786327721\\
51	0.0133345198721642\\
52	-0.0123745008368418\\
53	-0.0851652664658814\\
54	-0.0657593318290426\\
55	-0.0295875583738101\\
56	-0.103082843692256\\
57	-0.0574530279289594\\
58	-0.0924995990785913\\
59	-0.0279504966776894\\
60	0.0139020240926115\\
61	0.0139137337164068\\
62	-0.00880540136011325\\
63	-0.165126019098238\\
64	0.00198936235308465\\
65	-0.0331332867964299\\
66	-0.10490095539328\\
67	-0.0475032673729588\\
68	-0.122445239523586\\
69	-0.0580330203451716\\
70	-0.0258812339251961\\
71	0.0532461764032767\\
72	-0.0569133427249097\\
73	0.0715442764578818\\
74	-0.0462194251440889\\
75	-0.257988561755775\\
76	-0.0369674602733516\\
77	-0.0667766588369836\\
78	-0.0171477686121267\\
79	-0.179315557830554\\
80	-0.387902144236105\\
81	-0.161451745808743\\
82	0.0436958210296877\\
83	-0.0677623063157492\\
84	-0.0268104110002942\\
85	-0.105237717065019\\
86	-0.0575493104914281\\
87	-0.0921772900792722\\
88	-0.160883761681185\\
89	-0.102512785292145\\
90	-0.223971803290721\\
91	-0.0615615429632503\\
92	-0.0471462843051883\\
93	-0.0286134509961212\\
94	0.00288034434676267\\
95	-0.0596597759119077\\
96	-0.110177106957168\\
97	-0.0219724807973389\\
98	-0.00979946423816803\\
99	-0.21114771897182\\
100	-0.186166769397773\\
101	-0.0766812080816469\\
102	0.0365495334683822\\
103	-0.035989298634138\\
104	-0.143145743292324\\
105	-0.0395262721749439\\
106	0.024762949640771\\
107	-0.00759370235250323\\
108	-0.132806480492229\\
109	-0.0978651926587319\\
110	-0.0672445933706186\\
111	-0.0672526152871426\\
112	0.0187921810901593\\
113	-0.0526130948611329\\
114	0.0400986413741723\\
115	0.0325336500416856\\
116	0.0522228442249863\\
117	-0.0194571105392318\\
118	-0.125818245366238\\
119	-0.055880407828404\\
120	0.010470065009321\\
121	-0.0556567401080457\\
122	-0.00590207960683699\\
123	-0.0260406329063047\\
124	-0.0485734163122704\\
125	-0.0827993430414907\\
126	-0.0085292832861981\\
127	-0.0165455819391737\\
128	-0.0546306843589225\\
129	-0.0252488143973061\\
130	-0.0219398049907086\\
131	-0.0330672621956983\\
132	-0.05641001499249\\
133	-0.0832481636049818\\
134	-0.0900604090212291\\
135	-0.0751512839576212\\
136	-0.489802983209772\\
137	-0.0588776241875546\\
138	-0.00117824887415718\\
139	-0.0370864406634303\\
140	-0.0114669060927844\\
141	-0.040358701582808\\
142	-0.0378270859155574\\
143	-0.0317965137995542\\
144	-0.00678407873008303\\
145	-0.0356809437699124\\
146	0.00572698822291418\\
147	-0.0134644417909712\\
148	0.0298364096577485\\
149	0.000767762267181203\\
150	-0.0327513423351131\\
151	-0.0552578806504685\\
152	-0.0521586795031856\\
153	-0.0367150968072826\\
154	-0.0367272079292918\\
155	-0.0341657499663833\\
156	-0.00844300339264999\\
157	-0.0162302252213209\\
158	-0.00493963017913874\\
159	-0.0787881448713488\\
160	-0.0533532645279524\\
161	-0.00671417779905864\\
162	-0.0278451638859784\\
163	-0.00709899415224736\\
164	-0.0156208065681947\\
165	-0.0171413951156905\\
166	-0.0104742401795647\\
167	-0.0222141693343199\\
168	-0.00287483775355411\\
169	-0.146653230399974\\
170	-0.0374486723297366\\
171	-0.0878182495128086\\
172	-0.111458298114415\\
173	-0.0281096007281745\\
174	-0.0196866449606567\\
175	-0.0259362501514982\\
176	0.00563173359451441\\
177	-0.120868971920394\\
178	-0.154689618340419\\
179	-0.138947084341368\\
180	0.015821020206797\\
181	-0.00981265201609758\\
182	0.0190127351499457\\
183	-0.0227041517136506\\
184	-0.0116083020631266\\
185	-0.00751710074770835\\
186	-0.00203245901230065\\
187	-0.028802678906054\\
188	-0.0233346722978996\\
189	-0.0161760358572218\\
190	0.0213079510418383\\
191	-0.0794955196422548\\
192	-0.0509948621203456\\
193	0.00416952791979274\\
194	0.00729646389411873\\
195	0.0143697065852701\\
196	-0.0392807858441568\\
197	-0.0189923853623441\\
198	0.0341561861612771\\
199	-0.063761165469273\\
200	-0.50156655845565\\
201	-0.436797526921779\\
202	-0.175165257234287\\
203	-0.0648459591692155\\
204	-0.159333403348048\\
205	-0.252520595313619\\
206	-0.10174281652234\\
207	0.0705711178181963\\
208	-0.193326966545528\\
209	-0.00357441891311646\\
210	0.0164799555596676\\
211	-0.0151477664151083\\
212	-0.0556676484912089\\
213	-0.110004209554791\\
214	-0.0137605681154881\\
215	0.0359769954649446\\
216	0.0218087154165005\\
217	-0.0156747562798948\\
218	-0.0660532630932026\\
219	-0.0958202863965884\\
220	-0.191385226247613\\
221	-0.0509812988828071\\
222	-0.0171487457888097\\
223	-0.0571172512965657\\
224	-0.0209639093423886\\
225	-0.0757461600179559\\
226	-0.00235452846095781\\
227	-0.0181633609544181\\
228	0.0213935081117822\\
229	-0.0216707847904582\\
230	-0.169209394530033\\
231	-0.0624835127866373\\
232	-0.0333515408441886\\
233	0.0280423280423287\\
234	-0.00563498784643804\\
235	-0.102178169542021\\
236	-0.127159758277303\\
237	-0.116743989877684\\
238	-0.227751413776023\\
239	-0.0761921366925282\\
240	0.00073549018986007\\
};
\addlegendentry{Cts Stoch Ctrl w NMC};

\addplot [color=dscr_nFPC_plot_color,solid,line width=1.5pt]
  table[row sep=crcr]{%
1	-0.0622236403141551\\
2	-0.0902214029339275\\
3	-0.542290931891746\\
4	-0.0717159931334488\\
5	-0.201243961290449\\
6	-0.0205159854710596\\
7	-0.108018101271503\\
8	-0.0345685658457754\\
9	-0.0395623404473825\\
10	-0.00710962724638317\\
11	-0.0546198265711766\\
12	-0.105298470261885\\
13	-0.0132681269301445\\
14	-0.00820262041878009\\
15	-0.0176897572022576\\
16	-0.00347117319601139\\
17	-0.0349042621221976\\
18	-0.0391791119222073\\
19	-0.0540654180710634\\
20	-0.141267072952724\\
21	-0.0265016512326562\\
22	-0.0236207598606104\\
23	-0.0898093876323304\\
24	-0.0421938775337798\\
25	-0.044784567276272\\
26	-0.0400194988061513\\
27	-0.166538317442239\\
28	-0.0864737932784008\\
29	-0.0945690265312077\\
30	-0.00689195049311092\\
31	-0.0944950695228009\\
32	-0.186096758507663\\
33	-0.903607340688565\\
34	-0.227193131115657\\
35	-0.279884735271594\\
36	0.00971134247837628\\
37	-0.210000909954266\\
38	-0.0716067594794867\\
39	-0.0741181373152859\\
40	-0.0292473061097523\\
41	-0.097086674407452\\
42	-0.0630845875417165\\
43	-0.0171196920681851\\
44	-0.0517142166576022\\
45	-0.00206728966938795\\
46	-0.104572269152522\\
47	-0.108397275023398\\
48	-0.142576199918195\\
49	-0.152529964917434\\
50	-0.0391588230642041\\
51	-0.0252705628230447\\
52	-0.0241722807265594\\
53	-0.0905879241140987\\
54	-0.0544375219064896\\
55	-0.0814547668301678\\
56	-0.0942634872744548\\
57	0.0279637082316996\\
58	-0.0599447231913957\\
59	-0.0157381403300757\\
60	0.0417783102585255\\
61	-0.061661302172879\\
62	-0.0457520433081582\\
63	-0.0818618772971108\\
64	-0.0349052106245809\\
65	-0.0424434788780984\\
66	-0.0919713935537352\\
67	-0.0467653613516443\\
68	-0.0670277227015725\\
69	-0.126994512926019\\
70	-0.0531452931165491\\
71	0.0378043105477498\\
72	-0.109593020500796\\
73	-0.0254771827917257\\
74	-0.147394867733764\\
75	-0.364728498154195\\
76	-0.040240959519834\\
77	-0.0877259709108337\\
78	-0.0293458855617777\\
79	-0.192838749231946\\
80	-0.283023964761692\\
81	-0.0336014430708085\\
82	0.00348932110596437\\
83	-0.094830800584839\\
84	-0.0429276027984096\\
85	-0.0340383290336537\\
86	-0.0285004894319218\\
87	-0.0394006803604519\\
88	-0.107173626622257\\
89	-0.124573224481829\\
90	-0.171644995829899\\
91	-0.0685585924587923\\
92	-0.0353170007152629\\
93	-0.0391334431322789\\
94	0.00680186908474811\\
95	-0.10301852240814\\
96	-0.0789571128223681\\
97	-0.0922587816312591\\
98	-0.0488741282267409\\
99	-0.163186487179174\\
100	-0.0761392650795775\\
101	-0.0651113068472757\\
102	0.0281762900870729\\
103	-0.00807480901727355\\
104	-0.183732382719522\\
105	-0.0170101720071374\\
106	0.00525375093447478\\
107	0.00858546593338885\\
108	-0.0143810136961229\\
109	-0.0269078616776616\\
110	-0.0795656252962227\\
111	-0.0486901493672376\\
112	-0.0200663686632242\\
113	-0.0155263591138719\\
114	0.00809581309591992\\
115	-0.0302542674343119\\
116	0.0533265966555904\\
117	-0.0390713988537171\\
118	-0.19324540637989\\
119	-0.105977750696868\\
120	0.00462628013723824\\
121	-0.0732629729062215\\
122	-0.0268577487282578\\
123	0.00364658019612162\\
124	-0.0109155092309639\\
125	-0.0853037956782845\\
126	-0.0683959688525811\\
127	-0.032261124762357\\
128	-0.0740880725455547\\
129	-0.0632909429370922\\
130	-0.0135285669830839\\
131	-0.0208576001258154\\
132	-0.0777235195157482\\
133	-0.081760721788352\\
134	-0.0495625237324592\\
135	-0.0184125134061292\\
136	-0.377538664557245\\
137	-0.0521078627366559\\
138	-0.00883882224766369\\
139	-0.033764027209326\\
140	-0.0173829910371007\\
141	-0.0306354702621573\\
142	-0.0571649221326191\\
143	-0.046745546798273\\
144	-0.0041950788230853\\
145	-0.0134369556759433\\
146	-0.069554468112651\\
147	-0.00406869956106517\\
148	-0.0316823769879335\\
149	-0.0502657568826392\\
150	-0.0361906807789412\\
151	-0.0338544277081945\\
152	-0.0530489218274422\\
153	-0.0553829325534318\\
154	-0.0254645099463233\\
155	-0.0471205961204492\\
156	-0.0549166270956556\\
157	-0.0420849087961308\\
158	-0.0722496556889746\\
159	-0.0765796518791437\\
160	-0.0466696959772314\\
161	-0.0132397259525\\
162	-0.0417859034419682\\
163	-0.0286568319333279\\
164	-0.0306270322886295\\
165	-0.0392944029199278\\
166	-0.0186315364531892\\
167	-0.0424968649572698\\
168	0.000576257089235774\\
169	-0.0971774988680029\\
170	-0.0403157068411932\\
171	-0.0589031861797398\\
172	-0.0504786300445456\\
173	-0.031341609912908\\
174	-0.0451807162447258\\
175	-0.0637446493978277\\
176	-0.00384236881034699\\
177	-0.100988253518279\\
178	-0.0894755998710421\\
179	-0.22857805558768\\
180	-0.0726527386712699\\
181	0.00551852627939679\\
182	-0.0233628481454359\\
183	-0.0911163022512063\\
184	-0.063396265323932\\
185	-0.0980072653995251\\
186	-0.0254073911692917\\
187	-0.0386653237938813\\
188	-0.047339457224938\\
189	-0.0325608424450099\\
190	-0.0347837238329513\\
191	-0.0678912332272933\\
192	-0.0877696378294019\\
193	-0.033562859116245\\
194	-0.077530346724354\\
195	-0.0823316433175613\\
196	-0.0349509240569361\\
197	-0.0592287524696285\\
198	0.00601338794253177\\
199	-0.0927109166083285\\
200	-0.405672275514796\\
201	-0.351153290804939\\
202	-0.176700005644929\\
203	-0.0901474091204997\\
204	-0.190453778383595\\
205	-0.220083408628351\\
206	-0.0902196008698294\\
207	-0.0426876463195926\\
208	-0.125707831921622\\
209	-0.0820683052049124\\
210	-0.0340953391818671\\
211	-0.0312351278705099\\
212	-0.044968800985478\\
213	-0.102542419634159\\
214	-0.0634441831626672\\
215	-0.00395905323249378\\
216	-0.0433834610118744\\
217	-0.0913445654734438\\
218	-0.0687076681557595\\
219	-0.164161494195099\\
220	-0.0540132435500447\\
221	-0.0553912723358738\\
222	-0.0432155009974497\\
223	-0.0993336302941086\\
224	-0.0592135988537866\\
225	-0.0723902567796435\\
226	-0.0425758656386836\\
227	-0.093635446694427\\
228	-0.0632205143073491\\
229	-0.0496741146261264\\
230	-0.265678949422655\\
231	-0.0974478471882132\\
232	-0.077840837789503\\
233	-0.010825795914125\\
234	-0.0463316960790487\\
235	-0.103536985524458\\
236	-0.0921391128858256\\
237	-0.121690837742565\\
238	-0.578050734139789\\
239	-0.159598840346705\\
240	-0.018074250222312\\
};
\addlegendentry{Dscr Stoch Ctrl w NMC};

\end{axis}
\end{tikzpicture}%

\end{subfigure}%
\hfill%
\begin{subfigure}{.45\linewidth}
  \centering
  \setlength\figureheight{\linewidth} 
  \setlength\figurewidth{\linewidth}
  \tikzsetnextfilename{IS_week_NTAP}
  % This file was created by matlab2tikz.
%
%The latest updates can be retrieved from
%  http://www.mathworks.com/matlabcentral/fileexchange/22022-matlab2tikz-matlab2tikz
%where you can also make suggestions and rate matlab2tikz.
%
%
\begin{tikzpicture}[trim axis left, trim axis right]

\begin{axis}[%
width=\figurewidth,
height=\figureheight,
at={(0\figurewidth,0\figureheight)},
scale only axis,
every outer x axis line/.append style={black},
every x tick label/.append style={font=\color{black}},
xmin=1,
xmax=240,
%xlabel={Time (h)},
every outer y axis line/.append style={black},
every y tick label/.append style={font=\color{black}},
ymin=-1.1,
ymax=1.1,
%ylabel={Normalized PnL},
title={NTAP},
axis background/.style={fill=white},
axis x line*=bottom,
axis y line*=left,
yticklabel style={
        /pgf/number format/fixed,
        /pgf/number format/precision=3
},
scaled y ticks=false,
legend style={legend cell align=left,align=left,draw=black,font=\small, legend pos=north west},
every axis legend/.code={\renewcommand\addlegendentry[2][]{}}  %ignore legend locally
]
\addplot [color=cts_plot_color,solid,line width=1.5pt]
  table[row sep=crcr]{%
1	0.0973842315996598\\
2	-0.0708650954069955\\
3	-0.0598632120162681\\
4	0.0341523605968625\\
5	-0.00652695944036494\\
6	0.0195894381677894\\
7	0.0408599009402559\\
8	-0.00716058821124536\\
9	-0.107392035132287\\
10	0.00298464446297911\\
11	0.0471406931020282\\
12	0.108773794298955\\
13	-0.452966515536883\\
14	-0.0388028055304427\\
15	-0.0840846957730187\\
16	-0.0693172209752143\\
17	0.0493268753503404\\
18	-0.0575102653515546\\
19	0.0879208592925131\\
20	0.0291898443665743\\
21	-0.060069693165109\\
22	-0.0924867213712007\\
23	0.0102158499293169\\
24	-0.259938753763427\\
25	-0.47725819357207\\
26	-0.0828920841600036\\
27	0.0277038108307295\\
28	-0.1424138666713\\
29	-0.387827848470006\\
30	-0.110188843538406\\
31	-0.00596173213012182\\
32	0.0890686220975857\\
33	0.0832315186734875\\
34	-0.0559277181235802\\
35	-0.0924300197756936\\
36	-0.136539617751062\\
37	-0.0737087964191357\\
38	-0.0730543724262139\\
39	-0.0493033828088148\\
40	0.0187322988045535\\
41	0.0564279433419993\\
42	0.00595373998704507\\
43	0.133030267515719\\
44	-0.0127000043369285\\
45	0.0609548342088399\\
46	0.0663499375021835\\
47	-0.0285298535590834\\
48	-0.0301305337910599\\
49	0.0887161151340703\\
50	0.0473640158063604\\
51	-0.0362600550821372\\
52	0.0691194282565057\\
53	0.01141673468283\\
54	0.0174280224562928\\
55	0.0258091918218855\\
56	-0.00493745151199758\\
57	0.0473061584795541\\
58	0.0608516700823546\\
59	0.0804747254843715\\
60	0.0457808092371504\\
61	0.0286536617594221\\
62	-0.612189903637762\\
63	-0.167087974967655\\
64	-0.00969747218529174\\
65	-0.180728932238483\\
66	0.0438671552521918\\
67	-0.0500913508345933\\
68	-0.0780695756780642\\
69	0.0173881705260624\\
70	0.157911004189968\\
71	-0.0874048978518429\\
72	0.066151280261778\\
73	-0.0122414001608518\\
74	0.118879503024411\\
75	0.00107303527881092\\
76	-0.172098736336661\\
77	-0.0932679357515626\\
78	-0.169801828013107\\
79	-0.0119978407380682\\
80	-0.382475612272398\\
81	-0.00786648138056621\\
82	0.0499885009466521\\
83	-0.113410775061386\\
84	-0.00702652821288167\\
85	-0.0754342779012364\\
86	-0.304693540635337\\
87	0.0868670029577257\\
88	0.10458939703196\\
89	0.136700888298652\\
90	-0.0686251927558937\\
91	-0.0468358464385195\\
92	0.119048894819612\\
93	0.0211078453801693\\
94	0.0867679032302116\\
95	-0.0144372302542293\\
96	0.0922244543083735\\
97	0.0379171669411518\\
98	-0.0269605319573218\\
99	-0.0797527451819617\\
100	-0.116439854831078\\
101	-0.0440602322030294\\
102	-0.0876475806521321\\
103	0.00906483133371041\\
104	-0.0858587851749569\\
105	-0.0136003442744996\\
106	-0.0641346111214305\\
107	-0.000110011637300687\\
108	0.0372919471763565\\
109	-0.196224750592391\\
110	-0.139840626420907\\
111	-0.0479629633617055\\
112	-0.125685940553263\\
113	-0.0537672428487077\\
114	-0.0899824359670461\\
115	-0.088545697187311\\
116	-0.00536329371751885\\
117	0.00903009781103259\\
118	-0.177805770611549\\
119	-0.00573457681851643\\
120	-0.0323039792870895\\
121	-0.0503208559062953\\
122	-0.100681625113905\\
123	-0.0518738459796998\\
124	0.0173260218852166\\
125	0.0162083975438982\\
126	-0.0342858021622865\\
127	-0.164456995559568\\
128	-0.02769136384579\\
129	0.110179548534562\\
130	0.00554586380144403\\
131	-0.127235181686079\\
132	-0.014302810202828\\
133	-0.0150363808756104\\
134	0.0371083786406148\\
135	-0.0112709696918206\\
136	0.011418024923865\\
137	-0.0448750784287619\\
138	0.016659210915472\\
139	0.0280238927523558\\
140	0.0517541544693311\\
141	0.0472509431435074\\
142	0.0725860815201903\\
143	0.0208445393261559\\
144	0.012004775627597\\
145	-0.0440407923369719\\
146	0.0131233609097689\\
147	-0.0142313260269168\\
148	0.00151835358913453\\
149	0.0363672232634753\\
150	-0.0122267145492334\\
151	-0.0420302152254878\\
152	0.024357017844225\\
153	0.0333996547270617\\
154	0.0564529242258631\\
155	0.0979691531352013\\
156	-0.051266094379007\\
157	-0.0113941383939634\\
158	-0.0475199148250848\\
159	0.0196101199306324\\
160	0.0356842358480718\\
161	-0.0409987474021925\\
162	-0.0276228242107958\\
163	0.0696573675515952\\
164	-0.0842302317220567\\
165	0.0430481430293894\\
166	0.00207627757627633\\
167	0.0348728848560369\\
168	0.0484201417835678\\
169	0.0356939905297771\\
170	0.0606889683700428\\
171	0.102790080003422\\
172	0.0270719625968143\\
173	-0.0891425154163665\\
174	0.00665998755658641\\
175	-0.0798350273128964\\
176	0.0430239494082436\\
177	0.0404849596397555\\
178	0.00773424163131437\\
179	0.0203945601793577\\
180	0.00208754748670289\\
181	0.10936434516018\\
182	0.0163821812803336\\
183	-0.00603797725033009\\
184	0.0498808149625003\\
185	0.0746372585555441\\
186	0.0745077027351688\\
187	-0.0392155062254381\\
188	0.000721633881269832\\
189	-0.0314516428962623\\
190	0.0318774689583385\\
191	0.0395510001309161\\
192	-0.0046065009632562\\
193	0.0123540943081898\\
194	-0.137494414718074\\
195	0.0162279491866922\\
196	0.0224618978454733\\
197	0.0394943147636381\\
198	0.000730236493346396\\
199	0.0373462725164568\\
200	0.0238811394288191\\
201	0.00402512529932606\\
202	0.0564962753322664\\
203	0.11231731008455\\
204	0.143079017599241\\
205	0.0425705047122373\\
206	0.062346163114291\\
207	0.0618846006026825\\
208	0.0928790567125626\\
209	0.0354268926922381\\
210	-0.0878627553214327\\
211	0.0396297912552752\\
212	-0.0455798560871387\\
213	-0.0790917174718304\\
214	0.0401246697427942\\
215	-0.0768316976721247\\
216	0.0200646394235883\\
217	0.0334186708568887\\
218	0.0499161413844089\\
219	-0.00873390501630864\\
220	-0.0994913239564507\\
221	-0.0218355431308969\\
222	0.0119232044776062\\
223	-0.122587064806905\\
224	0.0584418982669117\\
225	-0.00392136452251621\\
226	0.00936478759172981\\
227	-0.0342997397713096\\
228	-0.0713175880221822\\
229	-0.0116268589818055\\
230	0.0264458483768565\\
231	0.0609417742133519\\
232	0.030173869175963\\
233	-0.0688490661433105\\
234	-0.0325883067354387\\
235	0.0683853891061657\\
236	-0.0240907528205583\\
237	0.0326078720422878\\
238	0.00338068277755105\\
239	-0.0585670470297425\\
240	-0.0380622041802715\\
};
\addlegendentry{Cts Stoch Ctrl};

\addplot [color=dscr_plot_color,solid,line width=1.5pt]
  table[row sep=crcr]{%
1	0.0703720049266865\\
2	0.0131570971235728\\
3	0.138204783252761\\
4	0.054913776919686\\
5	0.0886245118733281\\
6	0.168782981974309\\
7	0.146748613683705\\
8	0.0305576335210845\\
9	0.127846792970914\\
10	-0.0331845819674926\\
11	0.266919256727259\\
12	0.140609833422398\\
13	0.0197032835328913\\
14	0.0270110335132634\\
15	-0.151363094159371\\
16	-0.0309665053476047\\
17	-0.0438607592736874\\
18	0.081021793472687\\
19	0.130374023075972\\
20	0.168100107226583\\
21	0.0554268607588556\\
22	-0.120777675628993\\
23	0.0435005833594051\\
24	0.254067640596002\\
25	0.223582152624015\\
26	0.122779978744148\\
27	0.0860531211174495\\
28	-0.354740091939946\\
29	-0.355126714292391\\
30	-0.0631804308522634\\
31	-0.0327048336344858\\
32	0.118512166123868\\
33	-0.0167039732512806\\
34	0.087181492555383\\
35	-0.0149865421487683\\
36	-0.0232170297438526\\
37	0.0161697736347325\\
38	-0.0164689248728257\\
39	0.0196389470066701\\
40	-0.00693261054426208\\
41	0.0969448129590094\\
42	0.0500657884045183\\
43	0.11975200340865\\
44	0.0129281507354426\\
45	0.0500729137213891\\
46	0.0494494047779015\\
47	0.0677095968173248\\
48	0.0242826694960415\\
49	0.090531120878199\\
50	-0.0648261507830045\\
51	-0.0275720051660725\\
52	0.186643372329312\\
53	0.0778485764868438\\
54	0.118872938736678\\
55	0.237407526901494\\
56	-0.0408340326712999\\
57	0.017711190838325\\
58	0.0781740766019589\\
59	0.12444187077108\\
60	0.0851912661283341\\
61	0.139778033270261\\
62	0.341818771808098\\
63	0.0650509044970772\\
64	0.310371121284102\\
65	-0.110531828864107\\
66	-0.0860367523964\\
67	0.002420316510972\\
68	0.211405940233653\\
69	0.379847057267015\\
70	0.114214490743908\\
71	-0.0441445609205973\\
72	0.148642814541579\\
73	-0.0681327531166301\\
74	0.16052083484256\\
75	-0.125258506529542\\
76	0.031466200062119\\
77	0.380966745135204\\
78	0.0261574240493854\\
79	0.0610611710914722\\
80	0.040053103160992\\
81	0.0572662523947591\\
82	0.165802990829075\\
83	0.11443974016285\\
84	0.0801875405391441\\
85	0.154084235128611\\
86	0.503845335895026\\
87	-0.108121660741786\\
88	0.0456741717295523\\
89	0.173027832842805\\
90	-0.00895336607024053\\
91	0.017883607886326\\
92	0.0891335436901179\\
93	0.217466107071817\\
94	0.148030251192651\\
95	0.0252119114804575\\
96	0.0825536653247613\\
97	-0.00327135297004967\\
98	0.0929930497265642\\
99	0.224847413314423\\
100	0.106663123770563\\
101	0.0641076046925142\\
102	0.12651608672365\\
103	-0.0173231033191183\\
104	0.174985279681904\\
105	-0.0736593350195474\\
106	0.106752694994312\\
107	0.122440134711149\\
108	0.0810235360476938\\
109	-0.0525456071667759\\
110	-0.0398475622394987\\
111	0.0555125133539333\\
112	-0.0492517300046208\\
113	0.0251673310518787\\
114	-0.00128025087771159\\
115	0.0520316601428443\\
116	0.0946532931646031\\
117	0.0547263047033345\\
118	0.0682001471849204\\
119	0.0725394662934728\\
120	-0.0438700704818943\\
121	0.0349106889074497\\
122	0.0388626562316958\\
123	0.128572894410691\\
124	0.110933513095272\\
125	0.0746946495013565\\
126	0.147597425692192\\
127	-0.160713631569919\\
128	0.140870131762671\\
129	0.0961105931250165\\
130	0.00479671563141531\\
131	0.11012099730051\\
132	0.0673907358049508\\
133	0.033143917369162\\
134	0.0388080060804038\\
135	0.039005304931326\\
136	-0.0032686173221429\\
137	0.0884789177185559\\
138	0.0423259515873512\\
139	0.134714071372903\\
140	0.0140861738915347\\
141	0.130199764376351\\
142	0.148298521857623\\
143	0.10963107031178\\
144	-0.00100301481994817\\
145	0.0222669739083351\\
146	0.0616899541537857\\
147	0.858103801715829\\
148	0.180998242470763\\
149	0.134713181055879\\
150	0.00536507856131954\\
151	-0.0139582803641756\\
152	0.126306284956143\\
153	0.0417697811435116\\
154	0.00700168211320959\\
155	0.0661718377738596\\
156	-0.0712701599376247\\
157	0.05861661684765\\
158	-0.00432438458340827\\
159	0.0919093275241118\\
160	-0.0243343033170994\\
161	0.0304260385092162\\
162	0.112061563671839\\
163	0.0936579920591389\\
164	0.166368897263938\\
165	0.10339254630715\\
166	0.134336774363349\\
167	0.0134596614583168\\
168	0.0791195462892243\\
169	-0.00833855563794509\\
170	0.111074495240072\\
171	0.0357613720189755\\
172	0.0693815857291547\\
173	0.0548262292885271\\
174	0.0701809990456316\\
175	0.0885283655052241\\
176	-0.00132236582863772\\
177	0.107546138557534\\
178	0.0751004693834156\\
179	0.0485959942035392\\
180	-0.0151512017364843\\
181	0.0936888582035544\\
182	0.176044433462656\\
183	0.0749982538589873\\
184	0.119839649693843\\
185	0.136264843230219\\
186	0.115244970463338\\
187	0.145324854694208\\
188	0.067693120205217\\
189	0.0323687640217347\\
190	0.226325967443027\\
191	0.124518444187915\\
192	0.0413785683165608\\
193	-0.0673277822677443\\
194	-0.0127395406619523\\
195	0.102882504336133\\
196	0.0622036747140454\\
197	-0.0163633714192136\\
198	0.0183265646790919\\
199	-0.0121286035448168\\
200	0.171304349298364\\
201	0.0756299722242753\\
202	0.14268579165767\\
203	0.123876883982222\\
204	0.106299717769913\\
205	0.357810796775369\\
206	0.206137903508627\\
207	0.0854257580335576\\
208	0.1501699794034\\
209	0.457609782231183\\
210	0.0899058480398283\\
211	0.237119578694487\\
212	0.0187829546946557\\
213	0.0914304801072561\\
214	0.0580026276419761\\
215	0.026639367472526\\
216	0.0777822471417843\\
217	0.0591151086329065\\
218	0.0850248715728726\\
219	0.105704023427833\\
220	-0.179455832253954\\
221	0.132402567886143\\
222	0.050460845647744\\
223	0.24869901501101\\
224	0.0648627812921855\\
225	0.0330572876890373\\
226	0.0987520949934108\\
227	-0.32733212757619\\
228	-0.157421786003843\\
229	0.0165227569390805\\
230	0.060339698416858\\
231	0.0986272514484145\\
232	0.0465909091451621\\
233	0.0859018863053148\\
234	-0.0411104150350423\\
235	0.0699766222008862\\
236	0.0182982984295398\\
237	0.00167502174536144\\
238	0.0619841721316854\\
239	0.060638188739647\\
240	-0.0046001609284566\\
};
\addlegendentry{Dscr Stoch Ctrl};

\addplot [color=cts_nFPC_plot_color,solid,line width=1.5pt]
  table[row sep=crcr]{%
1	-0.139805058667736\\
2	-0.0451404477631651\\
3	-0.158370684869451\\
4	-0.227529350882089\\
5	-0.245508724865619\\
6	-0.0709054916273988\\
7	-0.159481710576313\\
8	-0.267538397034946\\
9	0.0521049156195567\\
10	-0.178238146718729\\
11	-0.00522146613769647\\
12	-0.0808943546013343\\
13	0.0827574746868611\\
14	-0.0960108910954503\\
15	-0.0806871987593052\\
16	-0.0528525149356751\\
17	-0.410678949658928\\
18	-0.061181405137241\\
19	0.105880802053019\\
20	-0.220005930521351\\
21	-0.185003101970468\\
22	-0.21342065955324\\
23	-0.196639402325448\\
24	-0.284449823239727\\
25	-0.62615723249344\\
26	-0.235633565476951\\
27	-0.0351387672420759\\
28	-0.345606759420644\\
29	-0.444868848128163\\
30	-0.357640752526337\\
31	-0.566076818411576\\
32	-0.389955250992929\\
33	-0.331432449312558\\
34	-0.349072016290433\\
35	-0.26000583584631\\
36	-0.222837887603184\\
37	-0.162393288520881\\
38	-0.196353950358269\\
39	-0.200925898744061\\
40	-0.124280316409587\\
41	-0.23321618584913\\
42	-0.153131406215968\\
43	-0.0646832022754805\\
44	-0.118513096528325\\
45	-0.190410595635772\\
46	-0.396068228256934\\
47	-0.381999122273798\\
48	-0.258910314452487\\
49	-0.0281677600694823\\
50	-0.168146056425567\\
51	-0.221389263464318\\
52	-0.166807851495887\\
53	-0.0906337302012409\\
54	-0.157018191564938\\
55	-0.11634201589609\\
56	-0.289591401779817\\
57	-0.304850412481734\\
58	-0.194425723176599\\
59	-0.226253713194153\\
60	-0.362417784551983\\
61	-0.175631622079742\\
62	-1.08339291676473\\
63	-0.598541588805985\\
64	-0.11364061662608\\
65	-0.806238309103878\\
66	-0.559425384372708\\
67	-0.344960670065169\\
68	-0.393272737098969\\
69	-0.299285912334562\\
70	-0.1406947830123\\
71	-0.316468720178912\\
72	-0.481412017893969\\
73	-0.325153031044975\\
74	-0.260770225929821\\
75	-0.330789703944535\\
76	-0.201032549253977\\
77	-0.139015504498741\\
78	-0.20404752942388\\
79	0.0491519807258837\\
80	-0.380707648021428\\
81	-0.195191053418403\\
82	0.0309756482068927\\
83	-0.382271259936085\\
84	-0.253206985150912\\
85	-0.328310168564215\\
86	-0.8531098372815\\
87	-0.129444819039134\\
88	-0.109971187942221\\
89	-0.216353398648089\\
90	-0.0871309653400267\\
91	0.0148561179244617\\
92	-0.251188959527083\\
93	-0.575531995266241\\
94	-0.0853015766673952\\
95	-0.139304179117453\\
96	-0.00941462402616778\\
97	-0.122796320813025\\
98	-0.237271780360567\\
99	-0.158600706324494\\
100	-0.182355265848589\\
101	-0.1639725030951\\
102	-0.0895770323041142\\
103	-0.0609674295685618\\
104	-0.136358852213361\\
105	-0.167515313486299\\
106	-0.0458931996934408\\
107	-0.202565300929382\\
108	-0.163258275148794\\
109	-0.464295777616052\\
110	-0.253947745821125\\
111	-0.210567161259449\\
112	-0.164988155548863\\
113	-0.133259844991595\\
114	-0.132500221879267\\
115	-0.42913451718449\\
116	-0.06461795653533\\
117	-0.156329294214558\\
118	-0.0776825508361103\\
119	-0.252925793551627\\
120	-0.226815343485615\\
121	-0.198884384654073\\
122	-0.159777766329815\\
123	-0.0371167163108244\\
124	-0.0594494272206225\\
125	-0.0771971952089563\\
126	-0.0951316343720054\\
127	-0.0858963469206517\\
128	-0.152775050735581\\
129	-0.148170290773805\\
130	-0.17047129066974\\
131	-0.183520076724633\\
132	-0.229329484887191\\
133	-0.136802763409496\\
134	-0.0929533091569738\\
135	-0.121382548908818\\
136	-0.0329245581069627\\
137	0.0142572918880028\\
138	-0.0988132400802492\\
139	-0.10364364052205\\
140	-0.0336228633268921\\
141	-0.0655148977077766\\
142	0.0462071874361303\\
143	-0.0584706337534911\\
144	-0.0432530997445138\\
145	-0.0636850169984925\\
146	-0.056203599847353\\
147	-0.0640876360369058\\
148	-0.04328865383036\\
149	-0.0711619757335845\\
150	-0.0541234861951094\\
151	-0.276314170760101\\
152	0.0161816682374379\\
153	-0.0489838344333206\\
154	-0.0756218734484506\\
155	-0.0249367905650375\\
156	-0.111887997467713\\
157	-0.152478222658382\\
158	-0.239539869184267\\
159	-0.0406230069141187\\
160	-0.117504138062886\\
161	-0.114719327256121\\
162	-0.0593547328900494\\
163	0.029283525163585\\
164	-0.0491937260716997\\
165	0.0263109265209291\\
166	-0.106789555619347\\
167	-0.100624037266344\\
168	-0.0171202410538076\\
169	0.0360161424788116\\
170	-0.0153844101057019\\
171	-0.114226973436493\\
172	-0.0301528770846605\\
173	-0.17748137478675\\
174	-0.0808235320593774\\
175	-0.0568600144908419\\
176	-0.0893365363075065\\
177	-0.10107433274118\\
178	-0.12073908377315\\
179	-0.0385581121709723\\
180	-0.012864186318239\\
181	-0.0165695356914907\\
182	-0.0524252979060749\\
183	-0.129241184142247\\
184	-0.0747396618411098\\
185	-0.0336547427265626\\
186	0.0624152543784383\\
187	-0.133518269646842\\
188	-0.124926116654389\\
189	-0.0242659847043819\\
190	-0.123716209586504\\
191	-0.0638767211387936\\
192	-0.113361908956059\\
193	-0.290206004994798\\
194	-0.352184540335726\\
195	-0.12780729077896\\
196	-0.0618855806306413\\
197	-0.0585759520522991\\
198	-0.147400507388389\\
199	-0.153150853445248\\
200	-0.104298701807242\\
201	-0.102345525978899\\
202	-0.105568230893709\\
203	-0.00929950971616619\\
204	0.0316890251894685\\
205	0.0660363904060208\\
206	0.0509260506138231\\
207	-0.0167802905836541\\
208	-0.032518064905848\\
209	-0.157417949112782\\
210	-0.522363238474154\\
211	-0.0864178118798642\\
212	-0.157979290439029\\
213	-0.0836034338984622\\
214	-0.302538517422969\\
215	0.0403595552492635\\
216	-0.195632526009467\\
217	0.00429088957262867\\
218	0.011948960731785\\
219	-0.0616913728362621\\
220	-0.275275702185792\\
221	-0.0181740622095131\\
222	-0.028219994469852\\
223	-0.127663475429474\\
224	-0.0198263095684604\\
225	-0.0556835640337739\\
226	-0.0574177187395385\\
227	-0.114168063780801\\
228	-0.120117543015972\\
229	-0.214764456837728\\
230	0.0171972288329242\\
231	0.0305378553274413\\
232	-0.130151195857846\\
233	-0.106397177516726\\
234	-0.174768976494117\\
235	-0.00990093967864353\\
236	-0.211169153497108\\
237	-0.0039709544119715\\
238	-0.0615590173599486\\
239	-0.0682867244033581\\
240	-0.0266869421089509\\
};
\addlegendentry{Cts Stoch Ctrl w NMC};

\addplot [color=dscr_nFPC_plot_color,solid,line width=1.5pt]
  table[row sep=crcr]{%
1	0.212369623403168\\
2	-0.00118159023845058\\
3	0.157753925724068\\
4	0.111647395220853\\
5	0.16586166405704\\
6	-0.116395038591044\\
7	0.138953124208097\\
8	0.0321024325280366\\
9	-0.052403116069919\\
10	0.114618052283324\\
11	0.297238025276683\\
12	0.0671491776788578\\
13	-0.212781307507827\\
14	0.081646922126694\\
15	-0.226056725272632\\
16	0.0404602496968807\\
17	0.000680666591977871\\
18	0.0806172507259021\\
19	0.0159552937303739\\
20	-0.0336127503876021\\
21	0.0777662642801583\\
22	-0.076734598592604\\
23	0.11696151367665\\
24	0.365734670358325\\
25	0.392356680279431\\
26	0.114699594316019\\
27	0.140170961860565\\
28	0.24524768169929\\
29	-0.416168162249947\\
30	-0.0298495309624558\\
31	-0.0106368949264504\\
32	0.215252947741464\\
33	0.136959761243266\\
34	0.00311741319488611\\
35	0.120837635351075\\
36	0.000258481185795926\\
37	0.106711786231056\\
38	0.240149044185374\\
39	-0.00408549563017066\\
40	0.11430487848279\\
41	0.156602630599312\\
42	0.273672566733029\\
43	0.179691498906968\\
44	0.0504732603950427\\
45	0.0648037114127482\\
46	0.225451636934519\\
47	0.079321164598447\\
48	0.0926756523569687\\
49	0.190666439621348\\
50	0.0143780590219546\\
51	0.0104614697628026\\
52	-0.0311075043205512\\
53	0.133149597907239\\
54	0.191040626683126\\
55	0.283460747294866\\
56	0.0421407728199236\\
57	0.074077508772687\\
58	0.0949115047389401\\
59	0.160141378763195\\
60	0.0843346907672647\\
61	0.239714689728876\\
62	0.509461052518395\\
63	0.0756898001707959\\
64	0.301257622077714\\
65	-0.172227201653601\\
66	-0.16047462863718\\
67	0.0772336895619947\\
68	0.323360869671054\\
69	0.433692375758365\\
70	0.216192659093922\\
71	-0.0177929553874671\\
72	0.211459039269082\\
73	-0.0599224602969366\\
74	0.000613001967151697\\
75	0.210190508613215\\
76	0.0134040114181168\\
77	0.361261379018139\\
78	0.0917167832687542\\
79	0.0150254480996171\\
80	0.0723362153813598\\
81	0.0867688754032171\\
82	0.188182843146294\\
83	0.0227795228827394\\
84	0.197784317898375\\
85	0.0125877564587754\\
86	-0.31662809742699\\
87	0.48401003236161\\
88	0.0921178109073896\\
89	0.181043534287218\\
90	0.681490424825109\\
91	0.00591502711937267\\
92	0.07433148514321\\
93	0.287550970575212\\
94	0.185598759934311\\
95	0.113216234262766\\
96	0.0719882940982891\\
97	0.0482479111855568\\
98	0.285480911762139\\
99	0.183848037883026\\
100	-0.159110325158905\\
101	0.0398684872933536\\
102	-0.0792744700754039\\
103	-0.0751682628023988\\
104	0.149736153157471\\
105	-0.0740033491404619\\
106	-0.0382474644734932\\
107	0.151759076662519\\
108	0.201142861857653\\
109	-0.107029488133103\\
110	0.114500842902242\\
111	0.0386783462056631\\
112	-0.0484538651733601\\
113	0.0414201638588112\\
114	0.0800991816675072\\
115	-0.0516171378841521\\
116	0.0918823367908057\\
117	0.141347491709536\\
118	0.0970385551296132\\
119	0.101346471290196\\
120	-0.0340404665642237\\
121	0.0536278217177678\\
122	0.0165288375453295\\
123	0.102572874590666\\
124	0.076131463426304\\
125	0.0939550012837272\\
126	0.176470258246533\\
127	-0.0347861082713793\\
128	0.134383264300894\\
129	0.102422867910867\\
130	0.08361697508324\\
131	0.0241053627229993\\
132	0.105942882328945\\
133	0.0549778446143145\\
134	0.0896274931422403\\
135	0.0974137627825942\\
136	8.45005460011903e-05\\
137	0.0631686222401509\\
138	0.0462973009387984\\
139	0.153288958175477\\
140	0.053128591050959\\
141	0.173374349769575\\
142	0.16805372769577\\
143	0.121204835615406\\
144	0.00688519707297414\\
145	0.116676465566799\\
146	0.103081561392267\\
147	1.04816710190263\\
148	0.0703659678364617\\
149	0.157543928575852\\
150	0.0370815541820001\\
151	0.0341633036370514\\
152	0.0938146589510169\\
153	0.0713526119601484\\
154	0.0257126438468068\\
155	0.0921744525361482\\
156	-0.0133419076438335\\
157	0.0679668488457903\\
158	0.0387547479927556\\
159	0.0454334860942742\\
160	0.0672853526299744\\
161	0.0338795240070051\\
162	0.122006783264531\\
163	0.0890457436857605\\
164	0.132770386071602\\
165	0.117362027281264\\
166	0.0305973624725745\\
167	0.122373259649513\\
168	0.0685252275744998\\
169	0.000746608023432382\\
170	0.0919017084930504\\
171	0.0547934788780824\\
172	0.0911854255101162\\
173	0.0153503553359939\\
174	0.0551879275522795\\
175	0.121754012417702\\
176	0.0287637355629404\\
177	0.160092412283716\\
178	0.0733548694147488\\
179	0.0615450437058364\\
180	0.0111449395886488\\
181	0.10070701874267\\
182	0.166610940092815\\
183	0.035861073512564\\
184	0.0905977208344743\\
185	0.0881121216659243\\
186	0.12854481584928\\
187	0.0129837999862982\\
188	0.122492096573611\\
189	0.0399920041654045\\
190	0.232237891167446\\
191	0.0762168853372158\\
192	0.0200281033117737\\
193	-0.0487205089614086\\
194	0.00311190983816331\\
195	0.103245091180965\\
196	0.109929210220211\\
197	0.148389922574749\\
198	0.00834126057065443\\
199	-0.0142926457061776\\
200	0.141882144618257\\
201	0.0871911752251949\\
202	0.13713817597731\\
203	0.117296782813323\\
204	0.174313487446561\\
205	0.36528597127677\\
206	0.232943162009742\\
207	0.13457553763455\\
208	0.170120923700974\\
209	0.438604887097346\\
210	0.143951307792897\\
211	0.0238981361581802\\
212	-0.0679737356839339\\
213	0.125707003215686\\
214	0.0768447897813109\\
215	0.0396662792630399\\
216	0.138629982343903\\
217	0.0356179356058343\\
218	0.0908604404199727\\
219	0.104401043197881\\
220	-0.115806795130364\\
221	0.0588956673446209\\
222	0.0532713405008364\\
223	0.208714599789417\\
224	0.0429985474335868\\
225	0.119813575912389\\
226	0.120468810280838\\
227	0.228373720150135\\
228	-0.145456806935693\\
229	0.0349564804506123\\
230	0.0691140386050729\\
231	0.102494497826076\\
232	0.0740934067708173\\
233	-0.123339326075203\\
234	-0.014782199130979\\
235	0.0981118486193388\\
236	0.00447697501266885\\
237	-0.0606588291369753\\
238	0.0902325955025149\\
239	0.059260665473269\\
240	0.00297640142370169\\
};
\addlegendentry{Dscr Stoch Ctrl w NMC};

\end{axis}
\end{tikzpicture}%
 
\end{subfigure}\\
\vspace{1cm}
\begin{subfigure}{.45\linewidth}
  \centering
  \setlength\figureheight{\linewidth} 
  \setlength\figurewidth{\linewidth}
  \tikzsetnextfilename{IS_week_ORCL}
  % This file was created by matlab2tikz.
%
%The latest updates can be retrieved from
%  http://www.mathworks.com/matlabcentral/fileexchange/22022-matlab2tikz-matlab2tikz
%where you can also make suggestions and rate matlab2tikz.
%
\definecolor{mycolor1}{rgb}{0.25098,0.00000,0.38824}%
\definecolor{mycolor2}{rgb}{0.00000,0.46275,0.00000}%
\definecolor{mycolor3}{rgb}{0.00000,0.34902,0.34902}%
\definecolor{mycolor4}{rgb}{0.58039,0.26275,0.00000}%
%
\begin{tikzpicture}[trim axis left, trim axis right]

\begin{axis}[%
width=\figurewidth,
height=\figureheight,
at={(0\figurewidth,0\figureheight)},
scale only axis,
every outer x axis line/.append style={black},
every x tick label/.append style={font=\color{black}},
xmin=1,
xmax=240,
%xlabel={Time (h)},
every outer y axis line/.append style={black},
every y tick label/.append style={font=\color{black}},
ymin=-1.1,
ymax=1.1,
%ylabel={Normalized PnL},
title={ORCL},
axis background/.style={fill=white},
axis x line*=bottom,
axis y line*=left,
yticklabel style={
        /pgf/number format/fixed,
        /pgf/number format/precision=3
},
scaled y ticks=false,
legend style={legend cell align=left,align=left,draw=black,font=\small, legend pos=north west},
every axis legend/.code={\renewcommand\addlegendentry[2][]{}}  %ignore legend locally
]
\addplot [color=mycolor1,solid,line width=1.5pt]
  table[row sep=crcr]{%
1	0.106540754541501\\
2	0.22519372901346\\
3	0.137704826456894\\
4	0.168538911227994\\
5	0.0300561834533785\\
6	0.13903655298721\\
7	0.126874101627404\\
8	0.237833677362031\\
9	0.130642833885921\\
10	0.101937030161955\\
11	0.156042487038076\\
12	0.241810122294663\\
13	0.161519843532851\\
14	0.0904173528628341\\
15	0.0383617227233518\\
16	0.108386493967483\\
17	-0.032647944717102\\
18	0.11826176428271\\
19	0.0865219332948834\\
20	0.148362063470914\\
21	0.175519765254118\\
22	0.108810693701721\\
23	0.193732306006846\\
24	0.0994486582297581\\
25	0.0752718234446712\\
26	0.131260170381711\\
27	0.049909224561027\\
28	-0.0366442919029737\\
29	0.0980880419790949\\
30	0.0754522444194005\\
31	0.168759918810082\\
32	0.150401317875709\\
33	0.207770742854203\\
34	0.106264370122263\\
35	-0.0108015303115805\\
36	0.148488992902124\\
37	0.169522807546045\\
38	0.00780302608514987\\
39	-0.114989717867627\\
40	0.0967554207335205\\
41	0.0208527812867942\\
42	0.0620210884541054\\
43	0.172283427442278\\
44	0.130877263274654\\
45	0.143522461323539\\
46	0.0733889442296028\\
47	0.201432577003317\\
48	0.309802582383725\\
49	0.121277788123214\\
50	-0.132300214479061\\
51	0.138173316342143\\
52	0.17565710017519\\
53	0.200976957200634\\
54	0.184688371728441\\
55	0.18607799584135\\
56	0.0550629689159508\\
57	0.191717667592766\\
58	0.192061274875198\\
59	0.258033393314315\\
60	0.265595365748545\\
61	0.078564545224155\\
62	0.139117902133239\\
63	0.0355346657015414\\
64	0.210116966047322\\
65	0.0367465719449428\\
66	0.144908813949386\\
67	0.300505847984412\\
68	0.213455634362758\\
69	0.136804229125549\\
70	0.244318534955863\\
71	0.180124989382077\\
72	0.179980606683071\\
73	0.182654914049672\\
74	0.15752146785494\\
75	0.0840143360724211\\
76	0.104615061355028\\
77	0.0484170713308239\\
78	0.167493834086675\\
79	-0.0344109502234701\\
80	0.146015856657944\\
81	0.206825097223236\\
82	0.0300674464814801\\
83	0.0608210664331526\\
84	-0.107472698835373\\
85	0.125432346243525\\
86	-0.0142956830596146\\
87	0.0407122910096369\\
88	0.112004966557289\\
89	0.10239431667985\\
90	-0.180466543256322\\
91	0.100504235088528\\
92	0.0440663122405075\\
93	0.0576425718320533\\
94	0.11093020719874\\
95	0.0840501164846748\\
96	-0.0013289362316018\\
97	0.164458704586842\\
98	0.140011164944835\\
99	0.0969858976414885\\
100	0.00836844899112041\\
101	0.237147295459091\\
102	0.0597236105534419\\
103	0.126470853540951\\
104	0.0146315140526567\\
105	0.133716930340285\\
106	0.101109854033775\\
107	0.0997633675058042\\
108	0.111676793506579\\
109	0.00355380804801269\\
110	-0.157212346042898\\
111	0.273065472967105\\
112	0.205900026021536\\
113	0.0997771262975968\\
114	-0.140400800628824\\
115	0.209978541658699\\
116	-0.203085906962451\\
117	0.206988909267803\\
118	0.147020419666974\\
119	0.161570903225032\\
120	0.195837674365191\\
121	0.108088242674058\\
122	0.0719793282939159\\
123	-0.0791268545270439\\
124	0.169441288708443\\
125	0.115612722044645\\
126	0.109637229538418\\
127	0.0640339334916849\\
128	0.0451473112000344\\
129	0.0364309227551844\\
130	0.0490332483132823\\
131	0.0437042104070921\\
132	0.157818382592803\\
133	0.0794614974287867\\
134	0.163798639187689\\
135	0.117969394594341\\
136	0.22576095036217\\
137	0.126883311502032\\
138	0.0781489835973646\\
139	0.0162281486427537\\
140	0.130748727268543\\
141	0.0511427420779248\\
142	0.077178168633474\\
143	0.054232453028918\\
144	0.151581378585396\\
145	-0.00281670124734755\\
146	0.128790874779945\\
147	-0.0269832512909599\\
148	0.0770599271074546\\
149	0.00466176316042218\\
150	0.0796886207761633\\
151	0.109268808323537\\
152	0.08773189845648\\
153	0.0778706326410299\\
154	-0.0571944272605597\\
155	-0.0123174595114863\\
156	-0.000303449326891686\\
157	0.0133312865871319\\
158	-0.0255703307017186\\
159	0.0782193822713527\\
160	-0.010557305457592\\
161	0.0645361926212708\\
162	0.118548149997695\\
163	0.00483175004533483\\
164	0.0232509846706501\\
165	0.127367418449002\\
166	0.0801217980547043\\
167	0.0221770221356782\\
168	0.0153861567969524\\
169	0.200554715140204\\
170	0.172803787809462\\
171	0.251390032188578\\
172	0.0631820688585877\\
173	0.0406025199863487\\
174	0.0751247999867299\\
175	0.0265710690877486\\
176	0.0978538979144822\\
177	0.154206005405942\\
178	0.160271917054417\\
179	0.112793232603879\\
180	0.131014144354607\\
181	0.182082003168149\\
182	0.15734682731923\\
183	0.177836859277068\\
184	0.216075653365485\\
185	0.106006125985158\\
186	0.0302139317292166\\
187	0.122603808735947\\
188	0.175198306217881\\
189	0.12038397078862\\
190	0.10797620742374\\
191	0.156454209563141\\
192	0.0857024349421306\\
193	0.110144412756735\\
194	0.0937727814791615\\
195	0.0907290933407674\\
196	0.0489790112177488\\
197	0.137430549025354\\
198	0.0367857861188821\\
199	0.117020251879886\\
200	0.124634438767712\\
201	0.105245387705113\\
202	0.0858524000999755\\
203	0.0631476246688538\\
204	0.0977850741680623\\
205	-0.013206364918819\\
206	-0.00211264214248321\\
207	0.0573987144102122\\
208	0.00484426790904826\\
209	0.115707009147385\\
210	0.154632156567978\\
211	0.0321881421261843\\
212	0.0871493925104386\\
213	0.029303984948224\\
214	0.1179825367604\\
215	-0.0145495339508366\\
216	0.0940658411903503\\
217	0.0734611839077353\\
218	0.153341423861865\\
219	0.0713073538391985\\
220	0.0437609588143221\\
221	-0.00231820108185265\\
222	0.162239565127336\\
223	0.155770854003913\\
224	0.102784958684766\\
225	0.0966195948139982\\
226	0.035157020915888\\
227	0.0958560777098545\\
228	0.0480149986944441\\
229	0.0306402568862938\\
230	0.0948192037587309\\
231	0.0986593751249876\\
232	0.285580744072392\\
233	0.132028209341729\\
234	0.0710948097076419\\
235	0.187737164656444\\
236	0.0160842048662836\\
237	0.156303332833646\\
238	0.0900505896142218\\
239	0.0566559933095659\\
240	0.0274556501064303\\
};
\addlegendentry{Cts Stoch Ctrl};

\addplot [color=mycolor2,solid,line width=1.5pt]
  table[row sep=crcr]{%
1	0.112923418391793\\
2	0.192712292915206\\
3	0.134469330581035\\
4	0.170487041797623\\
5	0.17817365170928\\
6	0.186990828628621\\
7	0.151059478170748\\
8	0.505273049610193\\
9	0.178868120574461\\
10	0.11755788465129\\
11	0.204926251447202\\
12	0.296505559149932\\
13	0.246645023489756\\
14	0.0876852876535064\\
15	0.220064166280156\\
16	0.128552760614477\\
17	-0.0950319854256295\\
18	0.10683909209394\\
19	0.0645682801076503\\
20	0.109773325585976\\
21	0.173838831675686\\
22	0.0886950556373176\\
23	0.184222369341749\\
24	0.0602171869621153\\
25	0.0554222878699282\\
26	0.144894913923229\\
27	0.259138582690354\\
28	0.0463473827036197\\
29	0.136326408813607\\
30	0.0930191959540357\\
31	0.12371609733179\\
32	0.156809166070012\\
33	0.111153983064208\\
34	0.220766724816837\\
35	0.230022579735752\\
36	0.142761604348176\\
37	0.150913532749018\\
38	0.154405151621384\\
39	-0.0294960200195824\\
40	0.157307126324037\\
41	0.0475095656302157\\
42	0.0386791181444514\\
43	0.161476224077651\\
44	0.191136250237844\\
45	0.176776143966389\\
46	0.0939732556459778\\
47	0.204200170887122\\
48	0.424550880623221\\
49	0.144020314543328\\
50	0.0698174359059105\\
51	0.127687679101662\\
52	0.316509619384463\\
53	0.248117737623612\\
54	0.124371510496666\\
55	0.208886491583142\\
56	0.214228667412326\\
57	0.188064306088318\\
58	0.155777078338759\\
59	0.322047725350016\\
60	0.280624094583949\\
61	0.19757320108001\\
62	0.0874350738906626\\
63	-0.0151023490518649\\
64	0.235424496993248\\
65	0.0326258966291267\\
66	0.165608359343947\\
67	0.286711863570512\\
68	0.139080405453994\\
69	0.0931955401458295\\
70	0.161503683616151\\
71	0.165770013565127\\
72	0.223966493837649\\
73	0.19073929887774\\
74	0.36574095675131\\
75	0.146535593130328\\
76	0.230563260896064\\
77	0.0673456566490985\\
78	0.136627217487627\\
79	0.0851349757280626\\
80	0.151514254112261\\
81	0.278531581958965\\
82	0.140633225838834\\
83	0.0447319412170896\\
84	0.104874066966397\\
85	0.187521647075365\\
86	0.138686464214135\\
87	0.117956645425514\\
88	0.104604782506671\\
89	0.163154360770709\\
90	0.0729033011745306\\
91	0.264990941368631\\
92	0.162329823321742\\
93	0.0999594668515173\\
94	0.0434662501743662\\
95	0.0532909411535066\\
96	0.0262081530643639\\
97	0.179240682874133\\
98	0.106951584691941\\
99	0.0989505790322949\\
100	0.167068836860499\\
101	0.246601702597136\\
102	0.117742479350114\\
103	0.149214656694155\\
104	0.201893183385329\\
105	0.0721716455637645\\
106	0.179009240686497\\
107	0.130747435232052\\
108	0.0968144382473444\\
109	0.00950761569782395\\
110	0.193849187502614\\
111	0.316556741988046\\
112	0.0972773580949286\\
113	0.160733278146426\\
114	0.252398944836731\\
115	0.295491438238362\\
116	-0.0968912449955398\\
117	0.211274248342138\\
118	0.204990587616484\\
119	0.210164523551879\\
120	0.210867544909957\\
121	0.213400808040063\\
122	0.159464133005557\\
123	0.198893199629936\\
124	0.171433456405851\\
125	0.140188288963804\\
126	0.135215464009214\\
127	0.152881341011124\\
128	0.0761457670972769\\
129	0.0668657437985615\\
130	0.0773760206579396\\
131	0.0610678193577542\\
132	0.0631810145409384\\
133	0.0776166488189116\\
134	0.132965822295822\\
135	0.0742394003104476\\
136	0.0937795903791547\\
137	0.0944396117072238\\
138	0.0799011385574969\\
139	0.0907755703832218\\
140	0.144447946727802\\
141	0.0811941294787575\\
142	0.0728851627935245\\
143	0.0579803719635675\\
144	0.0828800556940454\\
145	0.0470926882002771\\
146	0.112712754596622\\
147	0.050692884878265\\
148	0.0641142427090044\\
149	0.018586049573798\\
150	0.081208720921814\\
151	0.0902307697592179\\
152	0.0217764091055521\\
153	0.059315368820206\\
154	0.0303699889860479\\
155	-0.00567818102576588\\
156	-0.013855552036576\\
157	0.0667253741364553\\
158	0.0640701642930541\\
159	0.0764566858800782\\
160	0.115443883202301\\
161	0.0365955959895032\\
162	0.0966460499652526\\
163	0.0914523473754671\\
164	0.100724657014396\\
165	0.0964732018164757\\
166	0.0698258889662013\\
167	0.102635320341467\\
168	0.134113711784774\\
169	0.125577126518279\\
170	0.222178988345205\\
171	0.0866631001821743\\
172	0.0659650639069685\\
173	0.0244563480601626\\
174	0.0607816957133521\\
175	0.0608616850147566\\
176	0.077458620507703\\
177	0.0791640584904546\\
178	0.091749172578733\\
179	0.0900993690880803\\
180	0.0121419852774606\\
181	0.104941054281327\\
182	0.106226881002613\\
183	0.0765421272085126\\
184	0.123213705596731\\
185	0.123533287985628\\
186	0.055787356062088\\
187	0.115019847172415\\
188	0.0771813689532249\\
189	0.129146925356191\\
190	0.165716900397474\\
191	0.126098416984155\\
192	0.0240025889892594\\
193	0.0992562391747015\\
194	0.0886494568074968\\
195	0.143915646264521\\
196	0.0852920656956927\\
197	0.076812083818993\\
198	0.132979492358577\\
199	0.0719460591369591\\
200	0.149270069290304\\
201	0.0645629352984282\\
202	0.0635900219145438\\
203	0.0426620654723346\\
204	0.142297946982187\\
205	0.0518713609233419\\
206	0.0841803891249813\\
207	0.0637596282429846\\
208	0.139401890792643\\
209	0.133365557218228\\
210	0.103145486656201\\
211	0.116369202817268\\
212	0.0629943612364632\\
213	0.0977578389807265\\
214	0.0193980264568238\\
215	0.0723188321357767\\
216	0.0709584359826685\\
217	0.130629598835936\\
218	0.12892357535481\\
219	0.0932272323586785\\
220	0.0607947283334159\\
221	0.0369955029295741\\
222	0.140757546047835\\
223	0.130888010867174\\
224	0.127851681521066\\
225	0.105680733008434\\
226	0.0496777262960564\\
227	0.052252925431684\\
228	0.131790473369414\\
229	0.0154541041723531\\
230	0.101212899601714\\
231	0.0868216474614345\\
232	0.269855169694504\\
233	0.467827369177334\\
234	0.148344388952312\\
235	0.157478288640751\\
236	0.0797374430686309\\
237	0.127831031494324\\
238	0.135819959681923\\
239	0.028376835862097\\
240	0.0927673752145075\\
};
\addlegendentry{Dscr Stoch Ctrl};

\addplot [color=mycolor3,solid,line width=1.5pt]
  table[row sep=crcr]{%
1	0.0109418068400895\\
2	0.145878834632259\\
3	0.00840977134466853\\
4	0.0439828285924121\\
5	0.0462227863971218\\
6	0.123276975480219\\
7	0.106304551970127\\
8	0.500466130890592\\
9	0.118089643974495\\
10	0.024027008310466\\
11	0.0947657189261147\\
12	0.149428645578148\\
13	-0.0404436870807658\\
14	0.112563451462946\\
15	0.0606882933443012\\
16	-0.00661088245083104\\
17	-0.138940691109414\\
18	0.00161622733695933\\
19	-0.105407089360194\\
20	-0.060433107959002\\
21	0.0975090561589264\\
22	-0.0137701058352276\\
23	0.0821009274644704\\
24	0.00234473768295178\\
25	-0.0716433364625198\\
26	0.0306850177363901\\
27	0.179407046638373\\
28	-0.20825710135667\\
29	-0.0250649262809243\\
30	-0.046874646514069\\
31	-0.0125944901064334\\
32	-0.0743018134745574\\
33	0.0671966480045781\\
34	-0.138327646701389\\
35	0.082476923599092\\
36	-0.0449672744580462\\
37	-0.0131805321034779\\
38	0.0146329418799298\\
39	-0.121466376763797\\
40	0.0174924340375877\\
41	-0.0613204026675408\\
42	-0.0135581421155556\\
43	0.0144414173730219\\
44	0.0905268074692689\\
45	0.0519784142346992\\
46	-0.088459998838066\\
47	0.046796969253838\\
48	0.156203116512954\\
49	0.15473471433827\\
50	-0.0730330715159118\\
51	0.000944632235859662\\
52	0.0609975035451418\\
53	0.129115249593117\\
54	0.0816886026587709\\
55	0.103735744507912\\
56	-0.0987276735776261\\
57	0.180672731599778\\
58	0.0829950654879805\\
59	0.114407230948071\\
60	-0.0966680083755922\\
61	-0.0250957638902034\\
62	0.0314253272579883\\
63	-0.0840202772862022\\
64	0.0447922468224132\\
65	-0.0669029355811453\\
66	-0.0120292107397511\\
67	0.0474986216119333\\
68	0.0797352358583845\\
69	-0.158987124630962\\
70	0.0460094411566786\\
71	0.0337594795197594\\
72	0.0215496387555915\\
73	0.0988449935513265\\
74	0.132520249095029\\
75	-0.104789353351203\\
76	-0.0789238497263615\\
77	0.0195957170096018\\
78	0.0391288842926782\\
79	-0.0216270465561866\\
80	0.0523778697792018\\
81	0.0539092457677933\\
82	-0.00761281630562344\\
83	-0.074930186611399\\
84	-0.148654531689077\\
85	-0.0149304191812535\\
86	-0.0456195357660466\\
87	-0.0336696107135641\\
88	-0.0198176716035008\\
89	-0.0237195155828747\\
90	-0.335598921465332\\
91	-0.134649814258177\\
92	-0.0984485515198874\\
93	-0.0952073468963271\\
94	-0.00662518541499159\\
95	-0.0163127052484753\\
96	-0.179603525680058\\
97	0.0296975572832644\\
98	-0.032622822317478\\
99	0.0776828619047956\\
100	-0.0955040148838274\\
101	0.0440574533905225\\
102	-0.0838289339201449\\
103	-0.0862236636975325\\
104	-0.00817334818549639\\
105	0.0811713934920186\\
106	-0.019590235609851\\
107	0.0294520330682153\\
108	-0.0286529274687698\\
109	-0.328817454572584\\
110	-0.527812092324924\\
111	0.230907037070111\\
112	0.0409853877969768\\
113	0.00869906959706923\\
114	-0.214616950890463\\
115	-0.0878386738937054\\
116	-0.226810944060554\\
117	0.080558191630008\\
118	0.0787376586396832\\
119	-0.0301520248005351\\
120	0.118010003653557\\
121	-0.0171988802068232\\
122	0.0202083413283912\\
123	-0.177809684428788\\
124	-0.0608281578444691\\
125	0.0516497066578072\\
126	0.0206662602518183\\
127	0.00867666965859449\\
128	-0.145586911058566\\
129	0.0184556556871515\\
130	-0.0268410415795488\\
131	0.00908303519926278\\
132	0.121409745837948\\
133	0.0359239516831623\\
134	0.0678644035434147\\
135	0.0903842582521414\\
136	-0.0585251997721324\\
137	-0.0447806576784392\\
138	-0.0209519860753186\\
139	0.00721439443024314\\
140	-0.0447676368453893\\
141	-0.0660017964094734\\
142	-0.0275399308853287\\
143	-0.0387822206962715\\
144	-0.0458608761039165\\
145	-0.105390591321028\\
146	-0.0342478236669038\\
147	-0.0168491734330536\\
148	0.022098230198952\\
149	-0.131764616515259\\
150	-0.0700485201844154\\
151	-0.0325076628581467\\
152	-0.0179182653495157\\
153	0.0590234864036128\\
154	-0.0901377462546244\\
155	-0.0527608535680072\\
156	0.0207403445438522\\
157	0.03053987361856\\
158	-0.239092544684608\\
159	-0.0708837727828399\\
160	-0.0213406128421758\\
161	0.0215174122532895\\
162	0.00935327477704696\\
163	-0.0237277352749333\\
164	-0.102155600817695\\
165	0.00952620227369642\\
166	-0.0569473268258413\\
167	-0.13853814511086\\
168	-0.0452574574847635\\
169	-0.0894066831986427\\
170	-0.296973259070747\\
171	0.071847862859746\\
172	-0.115167160220225\\
173	-0.011577149730968\\
174	-0.148201450245767\\
175	-0.103206325428874\\
176	-0.151904818384523\\
177	-0.0505014241429798\\
178	-0.19258058808214\\
179	-0.122016976310281\\
180	-0.104626715711469\\
181	-0.0530244665156968\\
182	-0.0539844189779024\\
183	-0.111756018244114\\
184	-0.0244708380017296\\
185	0.0210803067977749\\
186	-0.062673942733755\\
187	0.017654761451592\\
188	-0.063271199506945\\
189	-0.060097204798916\\
190	-0.111202435190112\\
191	-0.0542101791684325\\
192	0.0174020116922894\\
193	-0.00138711837754103\\
194	0.0369310438569653\\
195	0.0546130069888784\\
196	-0.0332784543482989\\
197	-0.0383237014017327\\
198	-0.0221978573299285\\
199	-0.138590196388663\\
200	-0.0907861470349187\\
201	0.120435729778013\\
202	0.004132231550445\\
203	0.0296689846708275\\
204	-0.00424960309322448\\
205	-0.05266630301145\\
206	0.0427530077108709\\
207	0.0164248431113527\\
208	0.0971378954763436\\
209	-0.0189748475353819\\
210	0.0302643281886083\\
211	0.0696524218591655\\
212	0.0318520491321496\\
213	-0.0721762693711667\\
214	0.0425615578128655\\
215	-0.069847806051081\\
216	0.0375617487782085\\
217	0.043568261554469\\
218	0.0856928174385157\\
219	0.00125275551296378\\
220	-0.0427781503450971\\
221	0.000336150994249034\\
222	0.0521792135351439\\
223	0.0360084786341543\\
224	0.0498991208105158\\
225	-0.00175725938858798\\
226	0.029125300301572\\
227	-0.0196330065279931\\
228	-0.117394149463417\\
229	0.0477810863996858\\
230	0.0505065195961207\\
231	0.01334884575943\\
232	0.144975856733256\\
233	-0.268790341265156\\
234	0.07914064675577\\
235	0.0255335704288297\\
236	-0.0397899714597549\\
237	0.0554887644016089\\
238	-0.0621743058775188\\
239	0.0311680076379203\\
240	0.0558388221293866\\
};
\addlegendentry{Cts Stoch Ctrl w NMC};

\addplot [color=mycolor4,solid,line width=1.5pt]
  table[row sep=crcr]{%
1	0.117522439302889\\
2	0.193714637909439\\
3	0.123929666788155\\
4	0.197049044642514\\
5	0.196741908614173\\
6	0.203276744323998\\
7	0.151442786258534\\
8	0.515631578925498\\
9	0.174761457375557\\
10	0.113943418018702\\
11	0.17732934325146\\
12	0.357976994655604\\
13	0.260072384965373\\
14	0.0779663072622854\\
15	0.15331852717122\\
16	0.0141890393194416\\
17	-0.0985477725873703\\
18	0.119917594298331\\
19	0.184108978333709\\
20	0.231928204355273\\
21	0.160007605473972\\
22	0.11651153809355\\
23	0.0714925474706558\\
24	0.0411507665190115\\
25	0.0608148687413444\\
26	0.142368659730332\\
27	0.259574559352778\\
28	-0.000225617507666213\\
29	0.0817481942431757\\
30	0.119646443068115\\
31	0.141367471535399\\
32	-0.00271194585200402\\
33	0.264725023949842\\
34	0.235708479038162\\
35	-0.0367915336278034\\
36	0.126199239945149\\
37	0.17232339664511\\
38	0.144672170659601\\
39	-0.0509230029957569\\
40	0.153658198983958\\
41	0.0531559174436159\\
42	0.173256452518617\\
43	0.183248008774112\\
44	0.154017730793622\\
45	0.162582285547935\\
46	0.213947291144197\\
47	0.192688391917162\\
48	0.456095667365722\\
49	0.123815233021952\\
50	0.0283121951575793\\
51	0.138765021716711\\
52	0.359246118361702\\
53	0.240163035817193\\
54	0.165847841289187\\
55	0.196903878209283\\
56	0.120099563595465\\
57	0.234132857822863\\
58	0.17343793387391\\
59	0.34006916179467\\
60	0.29408677567746\\
61	0.184165853680406\\
62	0.0855733369314742\\
63	0.195588250275169\\
64	0.28447593148572\\
65	0.0102000830937004\\
66	0.101335375622015\\
67	0.316386493706967\\
68	0.0943858279056424\\
69	0.188981078462164\\
70	0.209841998632369\\
71	0.138478152135852\\
72	0.222550470727845\\
73	0.18489860809711\\
74	0.363786262449125\\
75	0.179650842731286\\
76	0.289483241634642\\
77	0.0295019247261798\\
78	0.150599675938237\\
79	0.105417826738771\\
80	0.135691261963458\\
81	0.295644285678406\\
82	0.150874944464978\\
83	0.0545669374461137\\
84	0.0873131379979612\\
85	0.19550109907606\\
86	0.119086071425362\\
87	0.106233947871963\\
88	0.0561278617824999\\
89	0.173697362259545\\
90	0.274176988551816\\
91	0.305926651371315\\
92	0.155653635886955\\
93	0.0573664571729875\\
94	0.102821300614347\\
95	0.0664114582579276\\
96	0.0282669561140941\\
97	0.186042783853509\\
98	0.146111602289072\\
99	0.322645366393515\\
100	0.172001204607572\\
101	0.178900952253933\\
102	0.0817572551815943\\
103	0.188758635781794\\
104	-0.117634568405045\\
105	-0.00187137047687233\\
106	0.216451560884194\\
107	0.0974223269735625\\
108	0.131927563954563\\
109	-0.0217154021619471\\
110	0.165111976471387\\
111	0.331079821873786\\
112	0.0657827271909845\\
113	0.136327083558669\\
114	0.251579013349062\\
115	0.291330408771043\\
116	-0.192229030409068\\
117	0.213322437441027\\
118	0.200027651337103\\
119	0.223880872923399\\
120	0.2352026106347\\
121	0.22842888734839\\
122	0.14409797354605\\
123	0.0709912760727584\\
124	0.185754939760091\\
125	0.140282997990943\\
126	0.153200693508223\\
127	0.151588996855974\\
128	0.0633639445554077\\
129	0.0546685960226589\\
130	0.0983179524532892\\
131	0.0606520765779729\\
132	0.0971383284516079\\
133	0.049631003466941\\
134	0.0951527364190741\\
135	0.0887399282346796\\
136	0.112079108429553\\
137	0.0530819187221686\\
138	0.065453954340685\\
139	0.0848063167436592\\
140	0.170373909268454\\
141	0.0965789706188669\\
142	0.0698544663643723\\
143	0.0263559498605503\\
144	0.0109615419713951\\
145	0.0476218969985001\\
146	0.0573330062587148\\
147	0.138634665850133\\
148	0.0463345241101315\\
149	0.0911891381636153\\
150	0.0532012411551162\\
151	0.149601621123438\\
152	0.109804357655505\\
153	0.062630262902223\\
154	0.138119538371798\\
155	0.0355255564446561\\
156	0.0586076575792843\\
157	0.0713423187238186\\
158	0.101763964638775\\
159	0.170611506126705\\
160	0.135508862104957\\
161	0.108134521168962\\
162	0.0939754587806221\\
163	0.12775658034352\\
164	0.130360203153521\\
165	0.19723414203133\\
166	0.204249119465143\\
167	0.0688947815331206\\
168	0.115628748792134\\
169	0.183185443841869\\
170	0.241802923080661\\
171	0.128708735109009\\
172	0.0623649121456092\\
173	0.0957188186721063\\
174	0.0788766427076832\\
175	0.0477373231714646\\
176	0.118147561798162\\
177	0.141350252646352\\
178	0.100574952960529\\
179	0.0866405299435105\\
180	0.172138509760234\\
181	0.124250831156419\\
182	0.171243673466863\\
183	0.117832392224321\\
184	0.130135808112528\\
185	0.240905574362623\\
186	0.14967302678588\\
187	0.198516613063936\\
188	0.0822103927105676\\
189	0.173489204880383\\
190	0.179844945892797\\
191	0.204214436623029\\
192	0.161743898901856\\
193	0.113713550348592\\
194	0.0767547382265253\\
195	0.178596296045314\\
196	0.107042005726179\\
197	0.0970248591337719\\
198	0.186066909464693\\
199	0.0181066151886708\\
200	0.14210058370006\\
201	0.0667126876463618\\
202	0.0636870802454166\\
203	0.0381247566042119\\
204	0.200432813975418\\
205	0.0500092295437106\\
206	-0.0380267829305514\\
207	0.0495462285338741\\
208	0.0221751660978788\\
209	0.0339316530600687\\
210	0.118316377604758\\
211	0.0452085236356668\\
212	0.0889751281956243\\
213	0.0912477817340114\\
214	0.0302893356311725\\
215	0.0803666279453793\\
216	0.0763116015300408\\
217	0.0888185395726834\\
218	0.157442893440473\\
219	0.108562899370013\\
220	0.0207624886165965\\
221	0.0804043858439082\\
222	0.12329431091289\\
223	0.0937410056368895\\
224	0.151374053547168\\
225	0.107806900632565\\
226	0.242945555862427\\
227	0.161135749395884\\
228	0.143708820165988\\
229	0.0241578259737574\\
230	0.0860202367078583\\
231	0.0952288237094876\\
232	0.284729820159629\\
233	0.480241305839283\\
234	0.119121831351708\\
235	0.203526683032668\\
236	0.0775028183851073\\
237	0.156236473007947\\
238	0.13776956097407\\
239	0.0353417569283897\\
240	0.125371603723996\\
};
\addlegendentry{Dscr Stoch Ctrl w NMC};

\end{axis}
\end{tikzpicture}%

\end{subfigure}%
\hfill%
\begin{subfigure}{.45\linewidth}
  \centering
  \setlength\figureheight{\linewidth} 
  \setlength\figurewidth{\linewidth}
  \tikzsetnextfilename{IS_week_INTC}
  % This file was created by matlab2tikz.
%
%The latest updates can be retrieved from
%  http://www.mathworks.com/matlabcentral/fileexchange/22022-matlab2tikz-matlab2tikz
%where you can also make suggestions and rate matlab2tikz.
%
\definecolor{mycolor1}{rgb}{0.25098,0.00000,0.38824}%
\definecolor{mycolor2}{rgb}{0.00000,0.46275,0.00000}%
\definecolor{mycolor3}{rgb}{0.00000,0.34902,0.34902}%
\definecolor{mycolor4}{rgb}{0.58039,0.26275,0.00000}%
%
\begin{tikzpicture}[trim axis left, trim axis right]

\begin{axis}[%
width=\figurewidth,
height=\figureheight,
at={(0\figurewidth,0\figureheight)},
scale only axis,
every outer x axis line/.append style={black},
every x tick label/.append style={font=\color{black}},
xmin=1,
xmax=240,
%xlabel={Time (h)},
every outer y axis line/.append style={black},
every y tick label/.append style={font=\color{black}},
ymin=-1.1,
ymax=1.1,
%ylabel={Normalized PnL},
title={INTC},
axis background/.style={fill=white},
axis x line*=bottom,
axis y line*=left,
yticklabel style={
        /pgf/number format/fixed,
        /pgf/number format/precision=3
},
scaled y ticks=false,
legend style={legend cell align=left,align=left,draw=black,font=\small, legend pos=north west},
every axis legend/.code={\renewcommand\addlegendentry[2][]{}}  %ignore legend locally
]
\addplot [color=mycolor1,solid,line width=1.5pt]
  table[row sep=crcr]{%
1	0.331396165929232\\
2	0.339499560527892\\
3	0.232328043849591\\
4	0.197393490825541\\
5	0.187155719449737\\
6	0.200682473766741\\
7	0.412030727012545\\
8	0.35601657438774\\
9	0.163684184463028\\
10	0.182946063259209\\
11	0.172032262672843\\
12	0.174864171813871\\
13	0.261566506485612\\
14	0.151522528504462\\
15	0.142727887295511\\
16	0.172723516061412\\
17	0.184583636968767\\
18	0.228810977335256\\
19	0.213349599109731\\
20	0.281919660012566\\
21	0.215171620276083\\
22	0.139159231125741\\
23	0.246242067987401\\
24	0.130420505342434\\
25	0.170657309778014\\
26	0.147901755594369\\
27	0.125948448738494\\
28	0.186841504058438\\
29	0.00559335767240588\\
30	0.160846370738877\\
31	0.462772567964411\\
32	0.347895542091314\\
33	0.142161610438638\\
34	0.291376159385902\\
35	0.234377829130777\\
36	0.254543649532816\\
37	0.26089771889818\\
38	0.20834383439368\\
39	0.112842876861812\\
40	0.282708033243972\\
41	0.132373693541792\\
42	0.204999204462943\\
43	0.191642904976306\\
44	0.0791157700167747\\
45	0.23253508553312\\
46	0.258140655442665\\
47	0.185172214396056\\
48	0.269169380874173\\
49	0.201375091344223\\
50	0.203393862379776\\
51	0.00366377257170145\\
52	0.35212774409079\\
53	0.147912902166732\\
54	0.1099710341438\\
55	0.17431052186199\\
56	0.11078847173706\\
57	0.27719300493022\\
58	0.235813610013044\\
59	0.297249380549499\\
60	0.381681392360562\\
61	0.37292235246887\\
62	0.310305236672697\\
63	0.192715494550387\\
64	0.513016301733233\\
65	0.39527714349713\\
66	0.473736729339163\\
67	0.414661375366238\\
68	0.303235645894483\\
69	0.454398148670253\\
70	0.450480002206538\\
71	0.0284376859931493\\
72	0.267376965048921\\
73	0.206263380914674\\
74	0.281272246948003\\
75	0.120186737962823\\
76	0.257181072375543\\
77	0.199314056667815\\
78	0.102551826069036\\
79	0.131731636196664\\
80	0.229895694997589\\
81	0.186041980900323\\
82	0.151277440577492\\
83	0.12629960725065\\
84	0.0765255571591537\\
85	0.209743306206626\\
86	0.0675038498785687\\
87	0.202461621892708\\
88	0.227436013723439\\
89	0.252333893016006\\
90	0.206217049333736\\
91	0.282054259749008\\
92	0.257341327129602\\
93	0.217129960209356\\
94	0.332322480792552\\
95	0.247922733285289\\
96	0.174838578493935\\
97	0.236392619359868\\
98	0.261828324054605\\
99	0.158428430462519\\
100	0.342854593637918\\
101	0.341170741327908\\
102	0.23907081865857\\
103	0.169479645423366\\
104	0.375611643634223\\
105	0.240994824258382\\
106	0.257928213532449\\
107	0.0546482155050077\\
108	0.184666550029001\\
109	0.205457942223012\\
110	0.29714493301848\\
111	0.0917917241698569\\
112	0.275552181780701\\
113	0.204890118860845\\
114	0.269997714887668\\
115	0.476751883158293\\
116	0.144670618224455\\
117	0.173428940452699\\
118	0.156269682517367\\
119	0.364237850831936\\
120	-0.115324549293291\\
121	0.0853628463398832\\
122	0.141094818844601\\
123	0.169126983818796\\
124	0.234607316033912\\
125	0.239551026639516\\
126	0.087215096382643\\
127	0.304838454261983\\
128	0.195457975725964\\
129	0.100972716893283\\
130	0.0863760760545377\\
131	0.149572380384725\\
132	0.276045381896443\\
133	0.280013038603723\\
134	0.186258060209268\\
135	0.200429073352436\\
136	0.195578757832808\\
137	0.148190948773964\\
138	0.28583211952428\\
139	0.251036536316024\\
140	0.230860747909843\\
141	0.158839227327434\\
142	0.0916500073104917\\
143	0.156287547561436\\
144	0.233617238564273\\
145	0.188228953514499\\
146	0.171158554219468\\
147	0.154081577052322\\
148	0.0622471289795272\\
149	0.259402956603548\\
150	0.271502326866788\\
151	-0.10060751455371\\
152	-0.139074807257041\\
153	0.213535759940408\\
154	0.0924034655681529\\
155	0.348920501470298\\
156	0.195452437404994\\
157	0.109835696601448\\
158	0.231827550536351\\
159	0.234788159073377\\
160	0.220621074150899\\
161	0.176291802370658\\
162	0.235761737355467\\
163	0.138998875283066\\
164	0.140666989609675\\
165	0.108207981570183\\
166	0.483907363104017\\
167	0.0349358815471575\\
168	0.187229953398813\\
169	0.31027402624019\\
170	0.199015927057617\\
171	0.0352144648524511\\
172	0.209246576078776\\
173	0.104267401983979\\
174	0.108414092446744\\
175	0.0859657807463441\\
176	0.0962301606047842\\
177	0.162055011565105\\
178	0.170216251394938\\
179	0.179749324478075\\
180	0.258227752002927\\
181	0.241996935939608\\
182	0.191658750598612\\
183	0.18754644678136\\
184	0.256807138275276\\
185	0.252857408032043\\
186	0.282199883870709\\
187	0.180111190075556\\
188	0.291396360105658\\
189	0.233430677193213\\
190	0.300815151407373\\
191	0.172997606444136\\
192	0.28653597005103\\
193	0.193495394545922\\
194	0.107924701857685\\
195	0.154758975897866\\
196	0.237898723751482\\
197	0.298265044594303\\
198	0.189354294478767\\
199	0.140597279265902\\
200	0.134596492053173\\
201	0.0861547112516496\\
202	0.20521091733216\\
203	0.173426599329909\\
204	0.168969798598249\\
205	0.152523540885011\\
206	0.206229119521072\\
207	0.0976579696137138\\
208	-0.00052714493799049\\
209	0.273780106347652\\
210	0.225342111888984\\
211	0.242345275420585\\
212	0.221112279550754\\
213	0.210286467652625\\
214	0.150701491463962\\
215	0.353641582992934\\
216	0.190098992939186\\
217	0.247625593672221\\
218	0.0971579894278747\\
219	0.19531994041142\\
220	0.0577919934340389\\
221	0.14909675795604\\
222	0.272251760887914\\
223	0.298766493925412\\
224	0.334485682858427\\
225	0.140286032574761\\
226	0.183874510484128\\
227	-0.0110254331231015\\
228	0.0520541985910703\\
229	0.0491171985192197\\
230	0.059079545184912\\
231	0.188934566487484\\
232	0.301985520520548\\
233	0.166160641556414\\
234	0.0240694899771623\\
235	0.155298633663994\\
236	0.0757047514639142\\
237	0.218416052396393\\
238	0.0437393522047888\\
239	0.218708176172619\\
240	0.130366145827001\\
};
\addlegendentry{Cts Stoch Ctrl};

\addplot [color=mycolor2,solid,line width=1.5pt]
  table[row sep=crcr]{%
1	0.258333332954965\\
2	0.298395915529189\\
3	0.204460619410076\\
4	0.264537072893063\\
5	0.194028835985588\\
6	0.290043130874557\\
7	0.348624890845606\\
8	0.35376605128522\\
9	0.267731020440209\\
10	0.141381343936286\\
11	0.107878453349953\\
12	0.177655570640551\\
13	0.272815031263431\\
14	0.199514888824382\\
15	0.267999262376953\\
16	0.354507053216405\\
17	0.234410193587408\\
18	0.230534176127727\\
19	0.199747528094638\\
20	0.350938655199766\\
21	0.216702474382561\\
22	0.114961760483312\\
23	0.194664585765652\\
24	0.127795872755482\\
25	0.157200473697505\\
26	0.167493121103629\\
27	0.0721925917056041\\
28	0.171111866732889\\
29	0.175354876754656\\
30	0.207560652220519\\
31	0.40887494508836\\
32	0.309137534629374\\
33	0.219826378984648\\
34	0.308612699856069\\
35	0.275133411727778\\
36	0.313865407449887\\
37	0.239210558033938\\
38	0.235689301957969\\
39	0.158216734915965\\
40	0.286676782585371\\
41	0.211368003680231\\
42	0.177475200397231\\
43	0.189314930582791\\
44	0.108170016063631\\
45	0.188442385264394\\
46	0.15584155352183\\
47	0.169407296464634\\
48	0.261543002758309\\
49	0.175792619966598\\
50	0.182953346845535\\
51	0.368667868260818\\
52	0.277245186634247\\
53	0.144979832605964\\
54	0.0574344310237376\\
55	0.131839709986538\\
56	0.152033815912936\\
57	0.193663694314966\\
58	0.212576282637062\\
59	0.363073434156807\\
60	0.521575479588552\\
61	0.410403309034656\\
62	0.232375464914921\\
63	0.19794554132765\\
64	0.548604400931705\\
65	0.559621634035289\\
66	0.415365188282705\\
67	0.298417289044937\\
68	0.337567002987758\\
69	0.468130975862967\\
70	0.414045770081976\\
71	0.163477531651022\\
72	0.368690000407711\\
73	0.243850048745596\\
74	0.420018345680921\\
75	0.3222049914706\\
76	0.323600313651368\\
77	0.143580283982501\\
78	0.123218850479158\\
79	0.134753391844057\\
80	0.279129752713564\\
81	0.161562231784907\\
82	0.206015154260378\\
83	0.123146862553987\\
84	0.059725079087019\\
85	0.255374408832185\\
86	0.115968479223038\\
87	0.200505032799341\\
88	0.275391614774295\\
89	0.221220475366963\\
90	0.217397467591595\\
91	0.228009217698381\\
92	0.194216045485652\\
93	0.211797652459087\\
94	0.258855238671246\\
95	0.197983753993186\\
96	0.281002729397494\\
97	0.335487447234812\\
98	0.225876259127117\\
99	0.244016400175303\\
100	0.392929861837949\\
101	0.392713878074444\\
102	0.206742453910765\\
103	0.143318971072785\\
104	0.38639609995595\\
105	0.222693636582825\\
106	0.287398216102626\\
107	0.259245640933127\\
108	0.164391671686726\\
109	0.215785589565327\\
110	0.296681475183561\\
111	0.150769021537656\\
112	0.345731647608159\\
113	0.153837507573348\\
114	0.329308186018068\\
115	0.517449725627535\\
116	0.252682418242897\\
117	0.196645416834867\\
118	0.199004907643989\\
119	0.264827645469955\\
120	0.254807080459137\\
121	0.170303052930315\\
122	0.237393767625636\\
123	0.325365464025871\\
124	0.239881071150188\\
125	0.262581422452126\\
126	0.200218309008649\\
127	0.292323794910911\\
128	0.170192237965674\\
129	0.111783371502943\\
130	0.124641064092572\\
131	0.124932328295501\\
132	0.303711270155845\\
133	0.263464289045227\\
134	0.165222789319563\\
135	0.158291183586955\\
136	0.144916816511215\\
137	0.143712800781857\\
138	0.264111554009897\\
139	0.213144415567275\\
140	0.235821628012592\\
141	0.189318631641625\\
142	0.157050194742795\\
143	0.229485762168262\\
144	0.252461030128294\\
145	0.213214105989781\\
146	0.216827782904002\\
147	0.225418533377847\\
148	0.142295964844881\\
149	0.293273847340325\\
150	0.230694473401839\\
151	-0.0290205589527297\\
152	0.434314809037508\\
153	0.21459631022965\\
154	0.106030697649942\\
155	0.281527526696193\\
156	0.206550133011625\\
157	0.10552874054675\\
158	0.312181138697747\\
159	0.218242480844117\\
160	0.255238413553204\\
161	0.163903650890966\\
162	0.234358318517292\\
163	0.126576310836501\\
164	0.10933802770406\\
165	0.115327257630403\\
166	0.491016467956258\\
167	0.093757686457163\\
168	0.192926585170794\\
169	0.178573401887598\\
170	0.202110611943852\\
171	0.12642510417572\\
172	0.189080339845646\\
173	0.16983946905113\\
174	0.115307657091625\\
175	0.207716283446279\\
176	0.163308247534264\\
177	0.25780752848014\\
178	0.156230348988371\\
179	0.164409323338058\\
180	0.348233303518176\\
181	0.169747414507807\\
182	0.201973744532911\\
183	0.237984198370188\\
184	0.315310723622873\\
185	0.228023350302429\\
186	0.305665418061777\\
187	0.221124458755576\\
188	0.264628158716188\\
189	0.299405388580942\\
190	0.307069658467044\\
191	0.171963812712652\\
192	0.300471630654666\\
193	0.19181371750441\\
194	0.120457413360201\\
195	0.163388899984384\\
196	0.19291988058835\\
197	0.309929306966694\\
198	0.253187510897678\\
199	0.170589840671008\\
200	0.150686101339681\\
201	0.0795666788394481\\
202	0.191510242058821\\
203	0.223195481602274\\
204	0.184635372461343\\
205	0.190701865687302\\
206	0.192178851757808\\
207	0.111384067996687\\
208	0.209319393784229\\
209	0.283444064222149\\
210	0.176259046557202\\
211	0.195980508062143\\
212	0.157087947486856\\
213	0.215830191036432\\
214	0.175110057113848\\
215	0.34003388168324\\
216	0.246261042321727\\
217	0.231807908796562\\
218	0.152812342322448\\
219	0.20432434531518\\
220	0.116735867179073\\
221	0.164762705388815\\
222	0.219568779865992\\
223	0.262637931904502\\
224	0.299628542292272\\
225	0.190477112331715\\
226	0.230601207988424\\
227	0.0475988718710717\\
228	0.123431475627929\\
229	0.15349633391494\\
230	0.137010963157796\\
231	0.232134636089899\\
232	0.324024915436665\\
233	0.222921603271774\\
234	0.268069673236259\\
235	0.228965930981914\\
236	0.138180232581363\\
237	0.228457999486799\\
238	0.100837622403079\\
239	0.23211054311038\\
240	0.161369278141647\\
};
\addlegendentry{Dscr Stoch Ctrl};

\addplot [color=mycolor3,solid,line width=1.5pt]
  table[row sep=crcr]{%
1	0.222263544979983\\
2	0.198970727955836\\
3	0.169310262332087\\
4	0.0912973472287008\\
5	0.173651820119614\\
6	0.239583164062507\\
7	-0.245476694886574\\
8	0.174925499866295\\
9	0.0680706430093557\\
10	0.157516659413538\\
11	0.0963856778844056\\
12	0.0892999584982044\\
13	-0.00176492844837237\\
14	0.115167579841054\\
15	0.0364517122605558\\
16	0.26829907687146\\
17	0.214228000551541\\
18	0.193554284591098\\
19	0.154239469944987\\
20	0.293929077121696\\
21	0.18854769543431\\
22	0.130327349308429\\
23	0.141708361014342\\
24	0.0832319728063623\\
25	0.15614472821394\\
26	0.0967042968054644\\
27	0.127208849008037\\
28	0.142079724750189\\
29	-0.0479619881322133\\
30	0.0892294670628884\\
31	0.456779897933269\\
32	0.0567962565264658\\
33	0.0583162148837402\\
34	0.189257565320723\\
35	0.17685619700167\\
36	0.261157895701715\\
37	0.0683143416369446\\
38	0.124588910574566\\
39	-0.0185815047606836\\
40	0.167890119206015\\
41	0.064563428605367\\
42	0.145843118442487\\
43	0.174248817777444\\
44	0.0378259429956586\\
45	0.21059007305075\\
46	0.0977694810039008\\
47	0.146351159366483\\
48	0.115176947943702\\
49	0.0614340856714658\\
50	0.118935770061037\\
51	0.321129814564622\\
52	0.231855707053304\\
53	0.0346677316525492\\
54	0.0723543314181321\\
55	0.098054996954259\\
56	0.0125058064020707\\
57	0.127311308232199\\
58	0.11313964609865\\
59	0.230723532227329\\
60	0.0530659720434264\\
61	0.222639583307394\\
62	0.151346094798832\\
63	0.0538002873713941\\
64	0.256445775476438\\
65	0.086102613386259\\
66	0.153692825836058\\
67	0.216198560426405\\
68	0.118637674059301\\
69	0.164226326535527\\
70	0.153463201655866\\
71	0.14544786723175\\
72	-0.0374005200029325\\
73	0.0396359688827504\\
74	0.153506821585801\\
75	-0.0285609755531888\\
76	0.190453598770038\\
77	0.201402188396788\\
78	0.00786736934922523\\
79	0.122893831598667\\
80	0.0311697740142221\\
81	0.0601400445151178\\
82	0.0859423969598438\\
83	-0.0224194326125029\\
84	-0.0600735038533097\\
85	0.0675887153718381\\
86	-0.0611846187475456\\
87	0.141916685254974\\
88	0.204401464139166\\
89	0.112019687506728\\
90	0.119988482990356\\
91	0.131665774338503\\
92	0.122983586224832\\
93	0.0865212242274805\\
94	0.110947215307438\\
95	-0.000743889725715881\\
96	-0.0671739297741392\\
97	0.0622732145249389\\
98	0.0390279660461109\\
99	-0.00481379357654755\\
100	0.118590306347889\\
101	0.128082594350603\\
102	0.0296610151694145\\
103	0.0639681453031256\\
104	0.157516505382841\\
105	0.208561931646692\\
106	0.213239520374583\\
107	0.102162200860606\\
108	-0.0527594320045033\\
109	-0.051380484793228\\
110	0.0868102464672762\\
111	-0.0691089289026354\\
112	0.271470296006428\\
113	0.217230860351698\\
114	0.108924567404139\\
115	0.215944703670225\\
116	0.0623487461021289\\
117	0.120114863901387\\
118	0.118440628177787\\
119	0.1648179163908\\
120	-0.204111674658723\\
121	0.164222719830404\\
122	-0.0465832206163647\\
123	0.37914444890461\\
124	0.222533055307076\\
125	0.107047112388434\\
126	0.0155077164104214\\
127	0.238052378971289\\
128	0.15505731929797\\
129	0.078774639017682\\
130	0.032030375820157\\
131	0.0213811545568484\\
132	0.125799470415069\\
133	0.24500326793308\\
134	0.200201285031303\\
135	0.103378102876811\\
136	0.0263338064988217\\
137	0.133048567597795\\
138	0.277908890894589\\
139	0.210647826861229\\
140	0.201419127130224\\
141	0.125624400125455\\
142	0.0710552965948081\\
143	0.0699236324174838\\
144	0.221989622609864\\
145	0.112480349000936\\
146	0.0736233962979803\\
147	0.0554132695274645\\
148	0.166062207505647\\
149	0.108744265969822\\
150	0.141065973325588\\
151	0.042384884228189\\
152	-0.187711816121461\\
153	0.164985041377612\\
154	0.0471598023015995\\
155	0.15841873469733\\
156	0.178194085904888\\
157	0.103533648435791\\
158	0.244930318815808\\
159	0.141433280301853\\
160	0.0100685663924666\\
161	0.223703250082627\\
162	0.226642172970313\\
163	0.0344924108116539\\
164	0.118934798955324\\
165	0.144287147385437\\
166	0.119690123089361\\
167	-0.054667435632899\\
168	0.131515424593767\\
169	0.0511534862318332\\
170	0.15829546431751\\
171	0.0816477239829299\\
172	0.198986572935073\\
173	0.058119449290335\\
174	0.0154527156878105\\
175	0.00432947808770741\\
176	0.0507748476349964\\
177	0.0655263846953395\\
178	-0.0243542346727458\\
179	0.0900148775303378\\
180	0.327811157135771\\
181	0.0853700157049899\\
182	0.0280877814698996\\
183	0.0443266562505952\\
184	-0.00478779993089406\\
185	0.185080069839642\\
186	0.141368505121706\\
187	0.0385564626016761\\
188	0.0107984086131524\\
189	0.309174999086628\\
190	0.142794079666083\\
191	0.0866662105881236\\
192	0.0747867724204601\\
193	0.122154897912091\\
194	-0.033878869324388\\
195	0.127752103886536\\
196	0.117556375180358\\
197	0.223381078835774\\
198	0.0636512001013263\\
199	0.0411018843320269\\
200	0.0665209616784442\\
201	0.0034822650256495\\
202	0.180577771885791\\
203	0.098043797933666\\
204	0.0982138996781343\\
205	0.143793466483495\\
206	0.206028069048729\\
207	0.0581586532885447\\
208	-0.000660918346393593\\
209	0.145873773214626\\
210	0.155977134829783\\
211	0.149805124882246\\
212	0.161150072278889\\
213	0.0632454266821189\\
214	0.124546356489698\\
215	-0.0259147557937289\\
216	0.17240477549684\\
217	0.254343278339333\\
218	0.0265230558401982\\
219	0.128974739237414\\
220	-0.0160821089086349\\
221	0.11242760220697\\
222	0.210078385990293\\
223	0.147920332022308\\
224	0.156700244992677\\
225	0.140204815123486\\
226	0.115769063010685\\
227	-0.0902461393897774\\
228	-0.0577940379838994\\
229	0.0136258129000245\\
230	-0.0038275794047052\\
231	0.0594622397943263\\
232	-0.0185177454823751\\
233	-0.021099347311368\\
234	0.00366900162136685\\
235	0.148855295962291\\
236	-0.0429862627762104\\
237	0.200662917717979\\
238	0.0145005061079729\\
239	0.213501695674985\\
240	0.0145070136869115\\
};
\addlegendentry{Cts Stoch Ctrl w NMC};

\addplot [color=mycolor4,solid,line width=1.5pt]
  table[row sep=crcr]{%
1	0.336471609378567\\
2	0.321901294212283\\
3	0.235637541051299\\
4	0.209567440381408\\
5	0.199943897443242\\
6	0.325737707160169\\
7	0.366342338959215\\
8	0.358690803445911\\
9	0.284912139158231\\
10	0.120970256205192\\
11	0.0718793308997572\\
12	0.196037788187212\\
13	0.286110669682671\\
14	0.150864562185575\\
15	0.223967996687471\\
16	0.297118293543899\\
17	0.206895412737444\\
18	0.202657897266101\\
19	0.24845499023069\\
20	0.330014448597852\\
21	0.140739592158399\\
22	0.133881057088247\\
23	0.198959963207385\\
24	0.127717038457593\\
25	0.16665259999308\\
26	0.147283365473374\\
27	0.104950386922479\\
28	0.330240919119253\\
29	0.115057256357791\\
30	0.190303890197856\\
31	0.443960766772709\\
32	0.35098610972577\\
33	0.186367214921145\\
34	0.32853383736874\\
35	0.233873946740099\\
36	0.339381876811973\\
37	0.259994707452077\\
38	0.128036514193369\\
39	0.115246714317089\\
40	0.283972916843903\\
41	0.213083135863214\\
42	0.114903358806584\\
43	0.193165214478117\\
44	0.0556806464852635\\
45	0.181773927933091\\
46	0.17394854504711\\
47	0.18309774144252\\
48	0.225072513157133\\
49	0.188569552091848\\
50	0.240912591060741\\
51	-0.117791703153207\\
52	0.204186368310798\\
53	0.123050499607986\\
54	0.0754905978466561\\
55	0.147646993405581\\
56	0.430238704229078\\
57	0.223424225258602\\
58	0.0903696073776\\
59	0.364015430420251\\
60	0.486715309260562\\
61	0.391788180484186\\
62	0.215197045617866\\
63	0.16897447583751\\
64	0.526687819575454\\
65	0.647345693571796\\
66	0.450892354260948\\
67	0.341601298346067\\
68	0.402101220385606\\
69	0.535899666427412\\
70	0.500292351444691\\
71	0.11130991247058\\
72	0.33797126780722\\
73	0.266042027761\\
74	0.453772473312595\\
75	0.315265519497982\\
76	0.0664533765119052\\
77	0.439780505268266\\
78	0.0912445196438349\\
79	0.138047523013254\\
80	0.291415697930149\\
81	0.152100645682413\\
82	0.203256320908266\\
83	0.223401154038658\\
84	0.0592868177624871\\
85	0.284406218384738\\
86	0.108826076280922\\
87	0.203368054884026\\
88	0.162811533331163\\
89	0.205339842738819\\
90	0.217146679095861\\
91	0.127711117158194\\
92	0.188616880823886\\
93	0.194290973988217\\
94	0.280702334793909\\
95	0.209514532675422\\
96	0.193473916785821\\
97	0.291554419936054\\
98	0.171912827529474\\
99	0.180459779138088\\
100	0.318739493675968\\
101	0.398703139959641\\
102	0.323188290723377\\
103	0.325320456440727\\
104	-0.00091818887207803\\
105	0.233315207242249\\
106	0.284865140598491\\
107	0.27440529907213\\
108	0.0991397943366665\\
109	0.15602470541477\\
110	0.253334717248088\\
111	0.0852614878821786\\
112	0.301336583033504\\
113	0.154614011473175\\
114	0.311064998254319\\
115	0.112089018661595\\
116	0.139321915280773\\
117	0.179933889357738\\
118	0.262349426034681\\
119	0.293060625319897\\
120	0.116884334158766\\
121	0.116181136466744\\
122	0.167734551947296\\
123	0.308227920264372\\
124	0.237265210757983\\
125	0.256604132389172\\
126	0.180037877296538\\
127	0.232202413379556\\
128	0.183201589253406\\
129	0.106547648899202\\
130	0.105046483429241\\
131	0.120939257901037\\
132	0.322220498678864\\
133	0.265203885854655\\
134	0.164748550729207\\
135	0.129343575805158\\
136	0.167127662526121\\
137	0.113708057432757\\
138	0.278475600551741\\
139	0.222309964064261\\
140	0.260274102865244\\
141	0.156198848615877\\
142	0.10816112086139\\
143	0.217704312843464\\
144	0.254454599293711\\
145	0.193981820748988\\
146	0.113682305327934\\
147	0.162646706255191\\
148	0.112161902892448\\
149	0.255496648459768\\
150	0.256116799899766\\
151	-0.0460722496585199\\
152	0.459679736144816\\
153	0.207389011750971\\
154	0.0590343649122772\\
155	0.271662389479131\\
156	0.189129631292154\\
157	0.110758706881691\\
158	0.392456629933447\\
159	0.303547945896831\\
160	0.24009414475569\\
161	0.251549095054938\\
162	0.197068162855751\\
163	0.144759067160211\\
164	0.122587145000078\\
165	0.105557013597745\\
166	0.500044066275686\\
167	0.0801768278836259\\
168	0.212238014976256\\
169	0.231437824023911\\
170	0.186113018718118\\
171	0.066590852023465\\
172	0.264249249586093\\
173	0.128818168869885\\
174	0.136928633257533\\
175	0.138323253603058\\
176	0.243815395181275\\
177	0.12571589852038\\
178	0.142408549287779\\
179	0.175069093258628\\
180	0.304949745976365\\
181	0.211449709725533\\
182	0.123026312578697\\
183	0.358392738783192\\
184	0.264683479484792\\
185	0.254724651954903\\
186	0.302951492642233\\
187	0.245464395768922\\
188	0.2229113329183\\
189	0.296411850700688\\
190	0.349401735868073\\
191	0.164826243797203\\
192	0.135606559707778\\
193	0.174499364352956\\
194	0.103898300875638\\
195	0.153645718696207\\
196	0.184919779975383\\
197	0.312815815002384\\
198	0.257011361557203\\
199	0.16599290711904\\
200	0.162844421211338\\
201	0.198670798121436\\
202	0.164096823162197\\
203	0.213928780875623\\
204	0.220455941675477\\
205	0.258098591121641\\
206	0.185514906774204\\
207	0.11007872898255\\
208	0.220027752152494\\
209	0.309911013321452\\
210	0.186712279668909\\
211	0.209909330706734\\
212	0.140409002633512\\
213	0.21218285773143\\
214	0.163341836080128\\
215	0.365074278535047\\
216	0.179371538662177\\
217	0.30634839897908\\
218	0.135259764457064\\
219	0.0801512735900101\\
220	0.0782481925200402\\
221	0.133845774965093\\
222	0.237810382581997\\
223	0.2844476036017\\
224	0.265764801532066\\
225	0.162367295591022\\
226	0.184405066416753\\
227	0.00870622503042605\\
228	0.236135869541002\\
229	0.174734406783864\\
230	0.119449335951609\\
231	0.0709973235572263\\
232	0.338610186189586\\
233	0.150710070162335\\
234	0.127081749206416\\
235	0.226708013939924\\
236	0.0730620679571756\\
237	0.22773623918712\\
238	0.0767991856584905\\
239	0.243246619476617\\
240	0.169123449736759\\
};
\addlegendentry{Dscr Stoch Ctrl w NMC};

\end{axis}
\end{tikzpicture}%
 
\end{subfigure}\\

\leavevmode\smash{\makebox[0pt]{\hspace{-7em}% HORIZONTAL POSITION           
  \rotatebox[origin=l]{90}{\hspace{20em}% VERTICAL POSITION
    Normalized PnL}%
}}\hspace{0pt plus 1filll}\null

Trading Day Number of 2013

\vspace{1cm}
\begin{subfigure}{\linewidth}
  %\centering
  \setlength\figureheight{\linewidth} 
  \setlength\figurewidth{\linewidth}
  \tikzsetnextfilename{strategylegend}
  \resizebox{\linewidth}{!}{\definecolor{mycolor1}{rgb}{0.25098,0.00000,0.38824}%
\definecolor{mycolor2}{rgb}{0.00000,0.46275,0.00000}%
\definecolor{mycolor3}{rgb}{0.00000,0.34902,0.34902}%
\definecolor{mycolor4}{rgb}{0.58039,0.26275,0.00000}%
\begin{tikzpicture}
    \begingroup
    % inits/clears the lists (which might be populated from previous
    % axes):
    \csname pgfplots@init@cleared@structures\endcsname
    \pgfplotsset{legend style={at={(0,1)},anchor=north west},legend columns=-1,legend style={draw=black,column sep=1ex},
            legend entries={Cts Stoch Ctrl,Dscr Stoch Ctrl,Cts Stoch Ctrl w nFPC,Dscr Stoch Ctrl w nFPC}}%
    
    \csname pgfplots@addlegendimage\endcsname{line width=2pt,mycolor1,sharp plot}
    \csname pgfplots@addlegendimage\endcsname{line width=2pt,mycolor2,sharp plot}
    \csname pgfplots@addlegendimage\endcsname{line width=2pt,mycolor3,sharp plot}
    \csname pgfplots@addlegendimage\endcsname{line width=2pt,mycolor4,sharp plot}

    % draws the legend:
    \csname pgfplots@createlegend\endcsname
    \endgroup
\end{tikzpicture}
}
\end{subfigure}%
  \caption{End of day strategy performances: in-sample backtesting using a one-week offset for calibration.}
  \label{fig:IS_week_comp}
\end{figure}

\begin{table}
\centering
\ra{1.2}
\begin{tabular}{@{} *{9}{r} @{}}
\toprule
Strategy & Return & Sharpe & \# MO & \# LO & Inv & \% Win & Max Loss & Max Win \\
\midrule
\multicolumn{9}{l}{\texttt{FARO}} \\ 
Naive & -1.072 & -0.444 & 435 & 0 & 1.74 & 0.08 & -34.276 & 2.984 \\ 
Naive+ & 0.045 & 0.046 & 0 & 213 & 2.26 & 0.74 & -8.764 & 5.363 \\ 
Naive++ & -0.003 & -0.027 & 0 & 6 & 0.25 & 0.50 & -1.138 & 0.444 \\ 
Cts & -0.060 & -0.565 & 53 & 149 & 0.08 & 0.20 & -0.955 & 0.077 \\ 
Dscr & -0.076 & -0.764 & 80 & 133 & -0.07 & 0.07 & -1.004 & 0.084 \\ 
Cts w nFPC & -0.065 & -0.590 & 56 & 149 & 0.09 & 0.17 & -1.022 & 0.072 \\ 
Dscr w nFPC & -0.076 & -0.778 & 78 & 133 & 0.07 & 0.07 & -0.904 & 0.053 \\[2ex]
\multicolumn{9}{l}{\texttt{NTAP}} \\ 
Naive & -0.303 & -0.316 & 854 & 0 & -10.96 & 0.20 & -9.463 & 4.349 \\ 
Naive+ & 0.290 & 0.122 & 0 & 3537 & -13.83 & 0.73 & -19.806 & 10.266 \\ 
Naive++ & -0.048 & -0.084 & 0 & 156 & -1.79 & 0.52 & -6.310 & 2.670 \\ 
Cts & -0.016 & -0.165 & 830 & 1405 & 0.37 & 0.52 & -0.612 & 0.158 \\ 
Dscr & 0.070 & 0.593 & 460 & 1388 & 5.06 & 0.79 & -0.355 & 0.858 \\ 
Cts w nFPC & -0.156 & -0.987 & 1506 & 1425 & 0.70 & 0.09 & -1.083 & 0.106 \\ 
Dscr w nFPC & 0.091 & 0.656 & 332 & 1401 & 3.16 & 0.85 & -0.416 & 1.048 \\[2ex]
\multicolumn{9}{l}{\texttt{ORCL}} \\ 
Naive & -0.112 & -0.248 & 492 & 0 & 3.66 & 0.28 & -3.197 & 2.452 \\ 
Naive+ & 0.066 & 0.022 & 0 & 4049 & -50.06 & 0.64 & -17.396 & 18.873 \\ 
Naive++ & 0.002 & 0.005 & 0 & 134 & 0.64 & 0.49 & -1.691 & 2.537 \\ 
Cts & 0.098 & 1.181 & 545 & 1318 & 1.86 & 0.90 & -0.203 & 0.310 \\ 
Dscr & 0.126 & 1.547 & 578 & 1310 & 4.01 & 0.97 & -0.097 & 0.505 \\ 
Cts w nFPC & -0.013 & -0.130 & 1069 & 1365 & 1.39 & 0.47 & -0.528 & 0.500 \\ 
Dscr w nFPC & 0.135 & 1.459 & 416 & 1338 & 3.16 & 0.96 & -0.192 & 0.516 \\[2ex]
\multicolumn{9}{l}{\texttt{INTC}} \\ 
Naive & -0.057 & -0.179 & 274 & 0 & -3.63 & 0.31 & -0.954 & 1.766 \\ 
Naive+ & 0.375 & 0.138 & 0 & 3925 & -25.43 & 0.65 & -11.060 & 11.465 \\ 
Naive++ & 0.013 & 0.055 & 0 & 77 & -0.47 & 0.53 & -1.815 & 1.126 \\ 
Cts & 0.202 & 1.995 & 423 & 1139 & 4.76 & 0.98 & -0.139 & 0.513 \\ 
Dscr & 0.226 & 2.494 & 501 & 1136 & 4.62 & 0.99 & -0.029 & 0.560 \\ 
Cts w nFPC & 0.107 & 1.111 & 681 & 1187 & 1.64 & 0.87 & -0.245 & 0.457 \\ 
Dscr w nFPC & 0.215 & 2.027 & 401 & 1156 & 3.78 & 0.99 & -0.118 & 0.647 \\ 
\bottomrule
\end{tabular}
\caption{Averaged strategy performance results: in-sample backtesting using a one-week offset for calibration.}
\label{tbl:IS_week}
\end{table}

\FloatBarrier
\subsection{Annual Calibration}
The second type of out-of-sample backtesting done was to calibrate using data amalgamated from the entire 2013 trading year. This was a very rich calibration source, as it effectively ensured that every possible state of the Markov chain would have had sufficient observations. Further, this caused us to fix the imbalance bins $\rho$ for the entire year, rather than having bins (and hence what it means to be `heavy buy imbalance' and `neutral imbalance') vary each day. Performance values are given in \autoref{tbl:IS_annual}, and \autoref{fig:IS_annual_comp} compares the day-over-day performance of the various strategies. 

Here we backtest only the more liquid of the stocks, \texttt{ORCL} and \texttt{INTC}. In comparing \autoref{tbl:IS_annual} with \autoref{tbl:IS_week}, we note some interesting observations. Again we see the most liquid stock, \texttt{INTC}, posting on average the better results using the strategies, suggesting that utilizing a liquid stock is key.  (\texttt{INTC} started the year at \$21.38 and gained 21.42\% over the year, while \texttt{ORCL} started at \$34.69 and climbed 10.29\%. However, \texttt{NTAP} started at \$34.30 and gained 19.94\%, similar in performance to \texttt{INTC}, and yet performed substantially worse.) Whereas we have seen thus far that the nFPC strategies underperform the regular calibration, here the roles were reversed in terms of performance, number of market orders used, and average inventory held. Across the strategies we see stability in the number of limit orders used, which suggests that this isn't so much strategy dependent as it is externally dependent on outside agents submitting their market orders. In the case of \texttt{ORCL} we see that the Cts Stoch Ctrl strategy was particularly susceptible to the sharp downward spikes on days 55, 100, and 119, which corresponded to large sell-offs in the market. 


\begin{figure}
\centering
\begin{subfigure}{.45\linewidth}
  \centering
  \setlength\figureheight{\linewidth} 
  \setlength\figurewidth{\linewidth}
  \tikzsetnextfilename{IS_annual_ORCL}
  % This file was created by matlab2tikz.
%
%The latest updates can be retrieved from
%  http://www.mathworks.com/matlabcentral/fileexchange/22022-matlab2tikz-matlab2tikz
%where you can also make suggestions and rate matlab2tikz.
%
%
\begin{tikzpicture}[trim axis left, trim axis right]

\begin{axis}[%
width=\figurewidth,
height=\figureheight,
at={(0\figurewidth,0\figureheight)},
scale only axis,
every outer x axis line/.append style={black},
every x tick label/.append style={font=\color{black}},
xmin=1,
xmax=252,
%xlabel={Time (h)},
every outer y axis line/.append style={black},
every y tick label/.append style={font=\color{black}},
ymin=-0.8,
ymax=0.8,
%ylabel={Normalized PnL},
title={ORCL},
axis background/.style={fill=white},
axis x line*=bottom,
axis y line*=left,
yticklabel style={
        /pgf/number format/fixed,
        /pgf/number format/precision=3
},
scaled y ticks=false,
]
\addplot [color=cts_plot_color,solid,line width=1pt,forget plot]
  table[row sep=crcr]{%
1	-0.143152744598688\\
2	0.0409566773302763\\
3	0.0225720068215832\\
4	-0.0650511708916325\\
5	0.00606454290473\\
6	-0.0254824530777331\\
7	0.0389772133852268\\
8	-0.0114111094916207\\
9	-0.0270662195722461\\
10	-0.0855628988030846\\
11	0.0649951064606233\\
12	0.00133072102315445\\
13	0.393077107457149\\
14	-0.00370215310303141\\
15	0.0122839443285884\\
16	-0.0310875346980275\\
17	0.043483201502087\\
18	0.0424801826151211\\
19	-0.147306409434344\\
20	-0.10992458583361\\
21	-0.0962793977367626\\
22	-0.0576680286191835\\
23	-0.219546062859651\\
24	0.00362378008515\\
25	-0.108928643579458\\
26	-0.131042895444256\\
27	0.0114815785053352\\
28	-0.0822703709955286\\
29	0.0201343521373436\\
30	0.0382040436018392\\
31	-0.0500538335436623\\
32	-0.0176756122811914\\
33	0.0775656899759344\\
34	-0.2764653864822\\
35	-0.161615860562715\\
36	-0.0956478820635329\\
37	-0.270159027824385\\
38	-0.183830015523017\\
39	-0.179310083118617\\
40	-0.00940138580350346\\
41	-0.180985242258304\\
42	-0.19719625310921\\
43	-0.0824227714707286\\
44	-0.0531953383210056\\
45	-0.0925788166066226\\
46	-0.0588850425026199\\
47	-0.0347490286673981\\
48	-0.100533762597442\\
49	0.00205268421010893\\
50	-0.0486530554235266\\
51	0.0597717282980598\\
52	-0.0121448160451146\\
53	-0.181976377014897\\
54	-0.0399372084786851\\
55	-0.0877379612378367\\
56	-0.0701567205115872\\
57	-0.349815535018211\\
58	-0.0480335028794422\\
59	0.0769337253679048\\
60	0.00278944024552383\\
61	-0.0473646691752706\\
62	-0.0155302822075744\\
63	-0.127111759802825\\
64	0.00921385738268186\\
65	0.00240678774855579\\
66	-0.0311700232224119\\
67	-0.0925817813805951\\
68	-0.16980536098477\\
69	-0.05954691398974\\
70	-0.135488068826615\\
71	-0.278645562400599\\
72	-0.0781971703793798\\
73	-0.270850179290615\\
74	-0.15449502674324\\
75	-0.0645404258648556\\
76	-0.0734500683439551\\
77	-0.0970099619552773\\
78	0.0343089263116042\\
79	-0.0220254386537158\\
80	-0.0442356829584443\\
81	0.0968352774717633\\
82	0.122400452743693\\
83	-0.121869683409915\\
84	-0.121181915917986\\
85	-0.0884818093938658\\
86	-0.060038117851821\\
87	-0.1569169145631\\
88	0.0138989723853876\\
89	0.0269543638594793\\
90	-0.0320964376181165\\
91	-0.155649278913946\\
92	-0.205566908993275\\
93	-0.0497061943267014\\
94	-0.0737985376265631\\
95	-0.0617678464555871\\
96	-0.0760827785560391\\
97	-0.0427962379198825\\
98	-0.48245336165393\\
99	-0.270097408331084\\
100	-0.200745587840738\\
101	-0.240067478279848\\
102	-0.180853369016357\\
103	-0.0731967785924826\\
104	-0.233633926996703\\
105	-0.0756747359577371\\
106	-0.0659827454379421\\
107	-0.174511843997212\\
108	-0.197334044158415\\
109	-0.125061437084109\\
110	0.00163813062197134\\
111	-0.232989620678163\\
112	-0.138649697715001\\
113	-0.0701052742784381\\
114	-0.0852794488629931\\
115	-0.112522439392022\\
116	-0.0317477071254898\\
117	-0.154550200882653\\
118	-0.378287738056346\\
119	-0.704683591375321\\
120	-0.023404944316434\\
121	-0.0446406264371165\\
122	-0.0058047879683146\\
123	-0.154248404465989\\
124	-0.0506591020030976\\
125	-0.169844449650414\\
126	0.0673153639089035\\
127	-0.107346566699149\\
128	-0.0802971367218684\\
129	-0.0299497705794451\\
130	-0.0595761638187059\\
131	-0.0530883940998834\\
132	-0.162185071780507\\
133	-0.115011748174835\\
134	-0.0889645884489991\\
135	-0.00188943158720356\\
136	-0.0272731277886099\\
137	-0.107747834854862\\
138	-0.227810407875211\\
139	-0.0273238988812028\\
140	-0.123337225647539\\
141	-0.0732425625456187\\
142	-0.0242438481459873\\
143	-0.0709720888898088\\
144	0.00405181146134272\\
145	-0.0556437877204816\\
146	-0.215098235804878\\
147	-0.0596423649910278\\
148	-0.0198103625232419\\
149	-0.0317729478597815\\
150	-0.0721309593712943\\
151	-0.0991151788172514\\
152	-0.0884387712389177\\
153	-0.0968651256745322\\
154	-0.107491391129115\\
155	-0.186157852539664\\
156	-0.0845809615461386\\
157	-0.233604749845871\\
158	-0.158251744371543\\
159	-0.209757519603007\\
160	-0.119598643023831\\
161	-0.2716873816551\\
162	-0.088638283444982\\
163	-0.057641899827942\\
164	-0.119637287885132\\
165	-0.270060307437976\\
166	-0.1316211013809\\
167	-0.0882053915257603\\
168	-0.212273720091381\\
169	-0.163287930562959\\
170	-0.0814100188432132\\
171	-0.103277467734153\\
172	-0.141165514663665\\
173	-0.0701834757049924\\
174	-0.0885747285781201\\
175	-0.172232724418938\\
176	-0.059724561539514\\
177	-0.125900417482458\\
178	-0.138295864979637\\
179	-0.118017132530749\\
180	-0.0817441610397176\\
181	-0.366232220109311\\
182	-0.0772019082926824\\
183	-0.191672660254052\\
184	-0.0861969923477891\\
185	-0.123284607585518\\
186	-0.137455256807513\\
187	-0.158738857772212\\
188	-0.116281588057889\\
189	-0.126799652024237\\
190	-0.178604090811128\\
191	-0.264911600566118\\
192	-0.146225201168938\\
193	-0.123713463299111\\
194	-0.164904878102206\\
195	-0.108532603092511\\
196	-0.0636764057699469\\
197	-0.0536678189814755\\
198	-0.183158397264326\\
199	-0.123283516416816\\
200	-0.139879283060345\\
201	-0.10990845652968\\
202	-0.046241660977155\\
203	-0.0706757150549093\\
204	-0.00112441208515265\\
205	-0.0241871460250395\\
206	0.00485155862698148\\
207	-0.0204329767171499\\
208	-0.0151218651688404\\
209	-0.00723833840901775\\
210	-0.139161210405707\\
211	-0.161273996041664\\
212	0.00132204213310409\\
213	-0.00745434975615357\\
214	-0.0287389723815516\\
215	-0.0526478859955804\\
216	-0.154981025371928\\
217	-0.0716134550215987\\
218	-0.0917724279642875\\
219	-0.0665064951265261\\
220	-0.0178675594637534\\
221	-0.16746116139023\\
222	0.033571211903165\\
223	-0.0575307379321925\\
224	-0.100541134147284\\
225	-0.131513682341395\\
226	-0.0459363995993681\\
227	-0.000283787111564833\\
228	-0.00027757569332935\\
229	0.0362952069415284\\
230	-0.0177461784420121\\
231	-0.083375347496053\\
232	-0.0810836709781799\\
233	-0.0488485458900137\\
234	-0.0742196521054875\\
235	-0.0448109158400766\\
236	-0.0393384448078818\\
237	-0.018157458888292\\
238	-0.0517682293819131\\
239	-0.108671614294697\\
240	-0.188868189752239\\
241	-0.0266460894078831\\
242	-0.0729181094505936\\
243	0.00641299882468128\\
244	-0.0181581957487398\\
245	-0.426311894516413\\
246	0.014583934132769\\
247	-0.0224994188023452\\
248	-0.0464222913087153\\
249	-0.0694982949806928\\
250	-0.106972192995484\\
251	-0.053550611555725\\
252	-0.034315649862337\\
};
\addplot [color=dscr_plot_color,solid,line width=1pt,forget plot]
  table[row sep=crcr]{%
1	0.30391980123581\\
2	0.279989607342037\\
3	0.104493659719645\\
4	0.0711411679674797\\
5	0.0974758416324635\\
6	0.135020030966423\\
7	0.21937719933309\\
8	0.134574186167451\\
9	0.186300080256134\\
10	0.165612719953199\\
11	0.13467833008811\\
12	0.167423759863081\\
13	0.0273710924965377\\
14	0.161202230918598\\
15	0.202008352694582\\
16	0.224383770282168\\
17	0.0778875140564405\\
18	0.231587921719369\\
19	-0.0217627123551034\\
20	0.095600070443442\\
21	0.166199510372964\\
22	-0.0202893831428722\\
23	0.221450352248142\\
24	0.126287642618672\\
25	0.190188837247395\\
26	0.291873965036333\\
27	0.124267978835074\\
28	0.100806563660975\\
29	0.0847839138166134\\
30	0.0184649111040544\\
31	0.100482289360945\\
32	0.16322775334867\\
33	0.251461720958779\\
34	-0.0327838632566798\\
35	0.167909224439716\\
36	0.0272785283611255\\
37	0.000414903580518349\\
38	0.164491640329409\\
39	0.0458338234426272\\
40	0.284592653235583\\
41	-0.00469052712511058\\
42	-0.0530235264860993\\
43	0.132213458598435\\
44	0.204612238819433\\
45	0.144748710447075\\
46	0.179901756658058\\
47	0.127701704248544\\
48	0.172020381164583\\
49	0.192321130551971\\
50	0.188266078681895\\
51	0.140330754542413\\
52	0.18785893331547\\
53	0.223954011310401\\
54	0.239709368828676\\
55	0.486706883215756\\
56	0.1082978357118\\
57	0.0998090973931565\\
58	0.172609838678833\\
59	0.298166624198851\\
60	0.239835531571686\\
61	0.189575299654216\\
62	0.197292977773833\\
63	0.202448284904525\\
64	0.254431687248642\\
65	0.316089304850331\\
66	0.0540213799023794\\
67	0.11644299390196\\
68	0.34064603090327\\
69	0.182537492085717\\
70	0.124607912690121\\
71	0.189841042030954\\
72	0.258984142277515\\
73	0.0487250763856916\\
74	0.189573268955994\\
75	0.356903731255983\\
76	0.161101512843304\\
77	0.173778379565565\\
78	0.208139671810986\\
79	0.308844330850879\\
80	0.141538332254659\\
81	0.200884667976744\\
82	0.430060445304225\\
83	0.174214190395833\\
84	0.212347003142854\\
85	0.0867898893046956\\
86	0.162241064088688\\
87	0.0693328406846551\\
88	0.183453531641591\\
89	0.195073659996617\\
90	0.123014106886756\\
91	0.0380106331322261\\
92	0.0656316116706567\\
93	0.0860833603847741\\
94	0.129749892656321\\
95	0.14792478529521\\
96	0.210088176815601\\
97	0.156258497308003\\
98	0.304256198738895\\
99	0.252164816960141\\
100	0.0470737533076383\\
101	0.130023133307429\\
102	0.129347946294781\\
103	0.0565983208316409\\
104	0.0306932829263233\\
105	0.117249818901254\\
106	0.196575424815801\\
107	0.176874117858134\\
108	0.365597873585027\\
109	0.0325231578156801\\
110	0.182150007632377\\
111	0.133743185556336\\
112	0.203613666874533\\
113	-0.0375880811795595\\
114	0.22260110900325\\
115	0.193040861680235\\
116	0.116315638975275\\
117	0.156173099150138\\
118	0.0525915009837263\\
119	0.164384087683327\\
120	0.30997565831735\\
121	0.122539354378317\\
122	0.147175061761096\\
123	0.282246833667623\\
124	0.298841277387712\\
125	-0.0884394250286415\\
126	0.213293322054991\\
127	0.141715715659852\\
128	0.243149666667379\\
129	0.123140309991338\\
130	0.262474005936294\\
131	0.174208796072005\\
132	0.188610750698997\\
133	0.0738162531852026\\
134	0.190907578740153\\
135	0.128948008096204\\
136	0.142425552244105\\
137	0.158521782566722\\
138	0.122888370221202\\
139	0.0795776118784478\\
140	0.131723683502111\\
141	0.047383308862499\\
142	0.170470022526545\\
143	0.0375614020833121\\
144	0.0617909046525308\\
145	0.107317561146912\\
146	0.158026016810783\\
147	0.0222572004844339\\
148	0.0666107895413924\\
149	0.0769429369736428\\
150	0.177912628218722\\
151	0.098541499388799\\
152	0.0974846296761735\\
153	0.00673842209381607\\
154	-0.0129916213644158\\
155	0.0326269794361613\\
156	0.0768455126175127\\
157	0.175945939976982\\
158	0.085317143846952\\
159	0.169323414048746\\
160	0.048264874903101\\
161	0.16615597552273\\
162	0.104752538805686\\
163	0.0232288833476614\\
164	0.162816446149189\\
165	0.0803752218020178\\
166	0.0888071110841557\\
167	0.111380501676856\\
168	0.116422322098607\\
169	0.0990401583308419\\
170	-0.010285782641404\\
171	0.11072438210708\\
172	0.103206811668615\\
173	-0.0192713433878968\\
174	0.127101802996817\\
175	0.0363792966517226\\
176	0.209016980770479\\
177	0.187960713636485\\
178	0.0151744102249499\\
179	0.167414848029729\\
180	0.231697582998873\\
181	0.14935345067665\\
182	0.114407949157651\\
183	0.115477621828263\\
184	0.116492715019623\\
185	0.0656513081444051\\
186	0.0697816799157596\\
187	0.110921044416324\\
188	0.177450322752631\\
189	0.159118350529589\\
190	0.0639034839441202\\
191	0.199424647280944\\
192	0.153925170278027\\
193	0.146756515987654\\
194	0.123793168501931\\
195	0.160023645195434\\
196	0.274514140833035\\
197	0.167228216438165\\
198	0.174231906711933\\
199	0.0851806523486165\\
200	0.219216833428879\\
201	0.148161637411238\\
202	0.129580344988186\\
203	0.185103320574473\\
204	0.165675996125955\\
205	0.0936761677088966\\
206	0.183987332029324\\
207	0.142261652992772\\
208	0.0877171160285701\\
209	0.180112708799717\\
210	0.0609501957840159\\
211	0.148070777786537\\
212	0.055851394988721\\
213	0.0709838521940563\\
214	0.041187241862375\\
215	0.246162596118826\\
216	0.0424474302863783\\
217	-0.0170189896255588\\
218	0.0329206564737356\\
219	0.00463526319300386\\
220	0.0538715832589857\\
221	0.178639197376422\\
222	0.043028709851543\\
223	0.0843213186930804\\
224	0.117533888698174\\
225	0.0511575883166564\\
226	0.0239408676355892\\
227	0.0873734921264623\\
228	0.114638751291501\\
229	0.202883531990841\\
230	0.112748172133995\\
231	0.0194997633866025\\
232	0.113754809343357\\
233	0.124258726004152\\
234	0.101914799678255\\
235	0.104712832530443\\
236	0.188882488383911\\
237	0.098442717105064\\
238	0.254361223677934\\
239	0.168421614203036\\
240	0.187890168720843\\
241	0.0302354516987999\\
242	0.131689780483186\\
243	0.0998327464269098\\
244	0.269633425535159\\
245	0.438931459095862\\
246	0.275483120449328\\
247	0.207058701736482\\
248	0.0874100256086577\\
249	0.133859657932863\\
250	0.0816213207678088\\
251	0.0416854316167692\\
252	0.107435330244426\\
};
\addplot [color=cts_nFPC_plot_color,solid,line width=1pt,forget plot]
  table[row sep=crcr]{%
1	0.232985180645052\\
2	0.278661369202366\\
3	0.177784445622526\\
4	0.0974783024399574\\
5	0.0327501169678138\\
6	0.115835166200431\\
7	0.228182960974266\\
8	0.133677083274036\\
9	0.117138736406273\\
10	0.0407993933299398\\
11	0.224901868201894\\
12	0.127772225688058\\
13	0.484476142631701\\
14	0.158329884387747\\
15	0.222959432027228\\
16	0.168627969078757\\
17	0.223764755537495\\
18	0.191147521777371\\
19	0.188269006298159\\
20	0.0633774412278127\\
21	0.100655573303472\\
22	0.0552923224973704\\
23	0.00296293629926227\\
24	0.146565657641003\\
25	0.0539323835085516\\
26	0.088952312466428\\
27	0.103644176185121\\
28	0.0459792583841881\\
29	0.167203997305459\\
30	0.0275495644912131\\
31	0.0385576079915185\\
32	0.140369493831539\\
33	0.0908384340048176\\
34	-0.0948591341170648\\
35	0.0864392597988126\\
36	-0.0354566217863406\\
37	-0.010317418128354\\
38	0.100727021651652\\
39	0.0185516200215174\\
40	0.0999079395420016\\
41	0.127182620672739\\
42	0.198457523865318\\
43	0.161341126958339\\
44	0.162018082003606\\
45	0.0896753081223347\\
46	0.0706439778174085\\
47	0.0862713664879718\\
48	0.0150859046368243\\
49	0.0657548690991912\\
50	0.120804460730971\\
51	0.116836047940295\\
52	0.173762024633149\\
53	0.0408085503252444\\
54	0.196231599131226\\
55	0.303153416169213\\
56	0.213132313445921\\
57	0.00328884448705292\\
58	0.0816711766253685\\
59	0.326281946274648\\
60	0.100711841711582\\
61	0.117978675757647\\
62	0.165274080760713\\
63	0.0512366903200091\\
64	0.183001481869082\\
65	0.22740384885254\\
66	0.18120229904944\\
67	0.160902639904801\\
68	0.244873158782679\\
69	0.198132558155759\\
70	0.111828439479246\\
71	-0.0292877791293691\\
72	0.175789026392167\\
73	0.0777978826200172\\
74	0.318573788431889\\
75	0.225325607071812\\
76	0.176900250172823\\
77	0.0943795291426934\\
78	0.267498491033146\\
79	0.162023593040026\\
80	0.21221640032508\\
81	0.192799867501877\\
82	0.332764168099998\\
83	0.192159036173535\\
84	0.0485391971680672\\
85	0.147201294910995\\
86	0.134205673668886\\
87	0.0413551405682738\\
88	0.163274976214799\\
89	0.181687987700759\\
90	0.0922938076229452\\
91	0.0115536515993344\\
92	-0.140280495851299\\
93	0.119840477478217\\
94	-0.00590111106983454\\
95	0.0956018488099142\\
96	0.0819497429315121\\
97	0.123005840782121\\
98	-0.0913045949948849\\
99	0.124815016842876\\
100	0.0548683207746685\\
101	0.0947713389582852\\
102	0.120853429885179\\
103	0.0977217262898269\\
104	-0.0362412098370985\\
105	0.326377020260979\\
106	0.191980074306645\\
107	0.130274208205258\\
108	0.107792579916992\\
109	-0.0195827679726336\\
110	0.239851413357581\\
111	0.0911675169453773\\
112	0.127287854404605\\
113	0.176936165888826\\
114	0.146426576735801\\
115	0.17153665839019\\
116	0.104597987250093\\
117	0.159312657578851\\
118	0.0485939492601045\\
119	-0.119306879716023\\
120	0.32113588646771\\
121	0.164477978725228\\
122	0.111832535583704\\
123	-0.0289101023348679\\
124	0.342743014178556\\
125	-0.137819552344678\\
126	0.179713536266101\\
127	0.102257290230638\\
128	0.0794973462296643\\
129	0.169663688572771\\
130	0.118037309848645\\
131	0.172209657125192\\
132	0.112773605073544\\
133	0.113802253707354\\
134	0.161743871723443\\
135	0.027181216214204\\
136	0.0992267616906097\\
137	0.0451809622425888\\
138	0.0307367073514741\\
139	0.00834368907549202\\
140	0.0438643539213626\\
141	0.0622992966938269\\
142	0.142075461545313\\
143	0.124319704085495\\
144	0.155064036520721\\
145	0.112355960956459\\
146	0.129968237608048\\
147	0.124265765213282\\
148	0.089205432552081\\
149	0.110339676984122\\
150	0.0865363232719383\\
151	0.056711945542368\\
152	0.0604724903236844\\
153	0.0132143447916864\\
154	0.146084319432537\\
155	0.0143784265959894\\
156	0.128361495663678\\
157	0.0269224735038067\\
158	0.124822343461558\\
159	0.101653009775558\\
160	0.0513006823572526\\
161	0.0968342157936758\\
162	0.0790767557675106\\
163	0.193633892620425\\
164	0.0870793980974061\\
165	0.0521911180987885\\
166	0.0428379242719558\\
167	0.10077308471489\\
168	0.038952356186051\\
169	0.113450984150026\\
170	0.0845369523509786\\
171	0.0936409666937151\\
172	0.109651245944451\\
173	0.179757522467018\\
174	0.0354537216892541\\
175	0.0151809728304655\\
176	0.0945005562075589\\
177	0.0149260439944415\\
178	0.00281069854134182\\
179	0.0335558826282202\\
180	0.155767841494501\\
181	0.0543554310967031\\
182	0.138057298858646\\
183	-0.0211471483376159\\
184	0.0267091244602764\\
185	0.0651297930594392\\
186	0.000786096804094837\\
187	0.10292943681807\\
188	0.172169738924283\\
189	0.0647197961089958\\
190	0.184515558782605\\
191	0.0511900478827466\\
192	0.14964999585032\\
193	0.127198865510193\\
194	0.260012355734284\\
195	0.227666905713059\\
196	0.248227046444054\\
197	0.12498511758634\\
198	0.159341835522063\\
199	0.148673188238109\\
200	0.167152469016004\\
201	0.14479397098422\\
202	0.174397908105001\\
203	0.0673452735247626\\
204	0.131226284747402\\
205	0.108445392792424\\
206	0.165797086956511\\
207	0.169913012420754\\
208	0.0636082840378056\\
209	0.0620139396004335\\
210	0.0562393195838818\\
211	0.0688769521494154\\
212	0.117228698734344\\
213	0.0880519489204243\\
214	0.0931417105478116\\
215	0.191311934292677\\
216	0.133069703542529\\
217	0.0847130762616071\\
218	0.0803582190909989\\
219	0.138757357023911\\
220	0.123093160196518\\
221	0.0833453108336262\\
222	0.189295334350849\\
223	0.079163850824988\\
224	0.0203166165120443\\
225	0.0989683606295565\\
226	0.00651819309494288\\
227	0.0717527520079922\\
228	0.0825436823681192\\
229	0.0870800327581871\\
230	0.125969715648678\\
231	0.0520229824858752\\
232	-0.0228189242434396\\
233	0.161682078513971\\
234	0.128832122256469\\
235	0.152389538249389\\
236	0.188106371248526\\
237	0.0909801730509801\\
238	0.00679257366476614\\
239	0.0135800578969971\\
240	0.157213526162243\\
241	0.0288904217185857\\
242	0.104194652248243\\
243	0.104010133085349\\
244	0.255648086856961\\
245	0.364429955687002\\
246	0.0673541865664417\\
247	0.176558312207671\\
248	0.0671039842472772\\
249	0.0445631345055255\\
250	0.108440838773946\\
251	0.0534500570486243\\
252	0.0917498830878987\\
};
\addplot [color=dscr_nFPC_plot_color,solid,line width=1pt,forget plot]
  table[row sep=crcr]{%
1	0.27639918926549\\
2	0.21155945953622\\
3	0.207492605393876\\
4	0.160139160324937\\
5	0.165479784839595\\
6	0.125073629404836\\
7	0.197829837790079\\
8	0.118702865642066\\
9	0.165089329716041\\
10	0.161538787011453\\
11	0.19580344812474\\
12	0.127316910070779\\
13	0.474496491286064\\
14	0.180050973300325\\
15	0.0993212969582757\\
16	0.195156699322351\\
17	0.198532647257166\\
18	0.188298656265802\\
19	0.211657684387069\\
20	0.0626875689116539\\
21	0.163931551455658\\
22	0.131771254868048\\
23	-0.00315271753168663\\
24	0.0787402723718865\\
25	0.0453231588317941\\
26	0.0894584600383547\\
27	0.166341554645226\\
28	0.0844377317306865\\
29	0.159084791145838\\
30	0.0526663675139228\\
31	0.0623373498696312\\
32	0.11945868160562\\
33	0.212381459920464\\
34	0.0209163567438843\\
35	0.0149111992343755\\
36	0.0877389429888953\\
37	0.0299015923204245\\
38	0.0676225102386918\\
39	0.119742769412006\\
40	0.0816240977702925\\
41	0.194699654303858\\
42	0.21275686985415\\
43	0.0915241793980571\\
44	0.164354192431146\\
45	0.128275075633929\\
46	0.0566021414925927\\
47	0.106947366994773\\
48	0.00885450759402141\\
49	0.0622596231879802\\
50	0.167817160381626\\
51	0.116056004094141\\
52	0.156960052608011\\
53	0.0487993586743903\\
54	0.166073155840938\\
55	0.40888991168613\\
56	0.157105693678344\\
57	0.0740134910320949\\
58	0.118001485633177\\
59	0.236909011594077\\
60	0.201353462292212\\
61	0.13557854172803\\
62	0.157627403304817\\
63	0.105009925749533\\
64	0.144682515232769\\
65	0.195943856230458\\
66	0.101357041301963\\
67	0.227810409516371\\
68	0.227654535421307\\
69	0.142485829739997\\
70	0.0844971435701694\\
71	0.0406893614329708\\
72	0.132250488974365\\
73	-0.028164811726563\\
74	0.0702665979208937\\
75	0.226417512723961\\
76	0.151366309590113\\
77	0.15186766327251\\
78	0.160882965886393\\
79	0.119989986242234\\
80	0.186278816347836\\
81	0.167459124509021\\
82	0.298649172065754\\
83	0.169794370348843\\
84	0.172102501150202\\
85	0.10402313182636\\
86	0.128497938498235\\
87	0.123765813880071\\
88	0.145672467618361\\
89	0.197300881455733\\
90	0.11954366976266\\
91	0.0378913409256639\\
92	0.0912287534477121\\
93	0.139569403217225\\
94	0.144401548157279\\
95	0.0969862124168518\\
96	0.0963345993519173\\
97	0.119569214466929\\
98	0.0120137533667836\\
99	0.162922884216401\\
100	0.0967948968883721\\
101	0.0519572488941979\\
102	0.0708806015535982\\
103	0.0569403659887177\\
104	0.0371553896048364\\
105	0.257619063542184\\
106	0.174366979201043\\
107	0.0998755760222771\\
108	0.0769646924915516\\
109	0.163258335810131\\
110	0.189208772934403\\
111	0.0870240481764104\\
112	0.0961675141419996\\
113	0.159549733883943\\
114	0.107770071084745\\
115	0.163726119614005\\
116	0.0813024799791784\\
117	0.103036055979575\\
118	0.0051992963078465\\
119	0.155133859446972\\
120	0.26573111681967\\
121	0.160423562044355\\
122	0.143375609446446\\
123	0.268451511416498\\
124	0.293980385583363\\
125	0.0408807978982486\\
126	0.204596687406779\\
127	0.187397723747391\\
128	0.161320128691823\\
129	0.132550163832802\\
130	0.222926595135384\\
131	0.108433128023631\\
132	0.197314897482464\\
133	0.142841163453813\\
134	0.155059050338522\\
135	0.120882177243395\\
136	0.118120095764545\\
137	0.103900837186136\\
138	0.0585840380984916\\
139	0.0673575673034174\\
140	0.0806967742107236\\
141	0.0761542780972822\\
142	0.111974229824667\\
143	0.109195985086925\\
144	0.131682835045461\\
145	0.0960223016899116\\
146	0.123142595616906\\
147	0.0916802495254794\\
148	0.0493302201987418\\
149	0.0809375684407428\\
150	0.148103387204611\\
151	0.0761853140612121\\
152	0.0702567139928381\\
153	0.053148628011654\\
154	0.101598604579821\\
155	0.0457749226417626\\
156	0.101303352539949\\
157	0.0355607019801478\\
158	0.0485429137702689\\
159	0.0118533548508159\\
160	0.0641164528980895\\
161	0.0752134823370411\\
162	0.0685664083529915\\
163	0.0436990463152773\\
164	0.117130029217376\\
165	0.0698969146054663\\
166	0.0315670037669296\\
167	0.0640988053172732\\
168	0.0746331568685694\\
169	0.0828505034364849\\
170	0.113497221110076\\
171	0.0644964108987353\\
172	0.0849738885158042\\
173	0.181369947577198\\
174	0.0840224549847493\\
175	0.0971815598235255\\
176	0.0966495775071051\\
177	0.0543883939075832\\
178	0.0773925270985083\\
179	0.129462014301531\\
180	0.16224732391565\\
181	0.215811512520427\\
182	0.120947380754986\\
183	0.0773099357478404\\
184	0.0438866453904282\\
185	0.073895078461553\\
186	0.0572866402734621\\
187	0.0676904147510946\\
188	0.0711673362608009\\
189	0.112812436941538\\
190	0.107596090182604\\
191	0.0425067462018149\\
192	0.118865299757302\\
193	0.100907240011908\\
194	0.10016762524532\\
195	0.137821720080934\\
196	0.218515169444105\\
197	0.129052416600099\\
198	0.135845709334879\\
199	0.066528557717827\\
200	0.121554717409691\\
201	0.0961636298085788\\
202	0.117253122533433\\
203	0.0497705231919543\\
204	0.0951939424467456\\
205	0.0879054494519945\\
206	0.154745078511307\\
207	0.110314152312057\\
208	0.0836100748894903\\
209	0.146884965031663\\
210	0.0565817460467028\\
211	0.102581251934584\\
212	0.0713936205600112\\
213	0.0728126816662681\\
214	0.0411561861587021\\
215	0.189970984626892\\
216	0.0489161925568649\\
217	0.0714902653976091\\
218	0.0686970681125924\\
219	0.127734251014221\\
220	0.0899410427676556\\
221	0.0902039767100507\\
222	0.168210094571749\\
223	0.0609878756769018\\
224	0.0699860116246424\\
225	0.0266330888162964\\
226	0.0695080629070155\\
227	0.0597590110332024\\
228	0.0980603191800073\\
229	0.129449383905655\\
230	0.0870174619857003\\
231	0.0566516008915038\\
232	0.0312845703438675\\
233	0.11888581710355\\
234	0.11658074205095\\
235	0.0653655101536513\\
236	0.140861632630684\\
237	0.0903054193630153\\
238	0.0523090028468483\\
239	-0.00650629789209997\\
240	0.108269648266976\\
241	0.0567961650517652\\
242	0.0966328430306656\\
243	0.068512438300021\\
244	0.236861122884258\\
245	0.374265369788632\\
246	0.11198907773798\\
247	0.148390098963562\\
248	0.0446932786848131\\
249	0.0960325619798381\\
250	0.127640008925875\\
251	0.0290784157024081\\
252	0.0804453804514989\\
};
\end{axis}
\end{tikzpicture}%

\end{subfigure}%
\hfill%
\begin{subfigure}{.45\linewidth}
  \centering
  \setlength\figureheight{\linewidth} 
  \setlength\figurewidth{\linewidth}
  \tikzsetnextfilename{IS_annual_INTC}
  % This file was created by matlab2tikz.
%
%The latest updates can be retrieved from
%  http://www.mathworks.com/matlabcentral/fileexchange/22022-matlab2tikz-matlab2tikz
%where you can also make suggestions and rate matlab2tikz.
%
%
\begin{tikzpicture}[trim axis left, trim axis right]

\begin{axis}[%
width=\figurewidth,
height=\figureheight,
at={(0\figurewidth,0\figureheight)},
scale only axis,
every outer x axis line/.append style={black},
every x tick label/.append style={font=\color{black}},
xmin=1,
xmax=252,
%xlabel={Time (h)},
every outer y axis line/.append style={black},
every y tick label/.append style={font=\color{black}},
ymin=-0.8,
ymax=0.8,
%ylabel={Normalized PnL},
title={INTC},
axis background/.style={fill=white},
axis x line*=bottom,
axis y line*=left,
yticklabel style={
        /pgf/number format/fixed,
        /pgf/number format/precision=3
},
scaled y ticks=false,
]
\addplot [color=cts_plot_color,solid,line width=1pt,forget plot]
  table[row sep=crcr]{%
1	0.108261969453122\\
2	0.128092766008791\\
3	0.138042602261726\\
4	0.0689827355724957\\
5	-0.0508908268687132\\
6	0.11538950403179\\
7	0.14626078138833\\
8	0.131540265756651\\
9	0.0320565799796215\\
10	0.150855952311756\\
11	0.138601678341993\\
12	-0.198193726608908\\
13	0.176304349601666\\
14	0.106431495798279\\
15	0.124571842575772\\
16	0.169655559701445\\
17	0.0755018234871014\\
18	0.129893264309841\\
19	0.116225134188291\\
20	0.0360262567184804\\
21	0.0247422126180168\\
22	0.211414718911429\\
23	0.138335248304581\\
24	0.116255681041543\\
25	0.120163088231376\\
26	0.174552526651406\\
27	0.113898655980305\\
28	0.0625369538875912\\
29	0.0828337756485767\\
30	0.078848504424959\\
31	0.0778406626193225\\
32	0.0613091077904826\\
33	0.0981156080686699\\
34	0.0437305896462215\\
35	-0.107978727945947\\
36	0.0422570380736357\\
37	0.0518807637461655\\
38	0.138515799274456\\
39	-0.0082199773927469\\
40	0.103146673228178\\
41	0.124075226268223\\
42	0.120275057214139\\
43	0.175268195805374\\
44	0.115844802852367\\
45	0.0387918535917847\\
46	0.0474052750595506\\
47	0.0829883775743784\\
48	0.00618608785344248\\
49	0.0913658967434522\\
50	0.134210002920354\\
51	0.0505766095096737\\
52	0.00543701888886731\\
53	0.0807066908067748\\
54	0.156978531580506\\
55	0.0979806277522267\\
56	0.0474474090495496\\
57	0.0374303132845947\\
58	0.0554990886649717\\
59	0.171774404775077\\
60	0.0512813079475249\\
61	0.0503689941991482\\
62	0.119893637511322\\
63	0.00707292088765692\\
64	0.115872241811019\\
65	0.17328633932529\\
66	0.135219371640266\\
67	0.141174535281971\\
68	0.2010688432868\\
69	0.227756892412636\\
70	0.0561618573866091\\
71	-0.0394852814489606\\
72	0.211159472217932\\
73	0.00122970503154915\\
74	0.149456372593392\\
75	0.208616975679732\\
76	0.0960263888191837\\
77	0.0131105074853763\\
78	0.17404792333262\\
79	-0.158162085792078\\
80	0.0374647796877087\\
81	-0.00403344021345173\\
82	0.148567930878915\\
83	-0.0514163669677407\\
84	0.102201960858149\\
85	0.10829618996525\\
86	0.00591430843927303\\
87	-0.0150519084078956\\
88	0.143416662067332\\
89	0.0230792443964536\\
90	0.0596959518151509\\
91	0.0190487985205076\\
92	-0.0514062438074665\\
93	0.0455170523808247\\
94	0.0754234224032671\\
95	0.119293461388266\\
96	0.145551946718334\\
97	0.168239562035543\\
98	-0.0200800110482956\\
99	0.0647420743323313\\
100	0.0265974945583752\\
101	0.0503832233438308\\
102	0.109468013496597\\
103	0.0365425315970992\\
104	0.0382582389951579\\
105	0.172335727977483\\
106	-0.0965722398996357\\
107	0.0268956385755057\\
108	0.0597835470846484\\
109	0.0613502089919859\\
110	0.0272245350806087\\
111	0.0134598791399228\\
112	0.0220557718365536\\
113	0.144781742916602\\
114	0.164265216477742\\
115	0.0929810910255108\\
116	-0.0171582414088864\\
117	0.00184779714519647\\
118	-0.0861258770145403\\
119	0.0992043294393057\\
120	-0.0568703746190836\\
121	0.162277518790955\\
122	0.067038771555409\\
123	-0.00698610227058504\\
124	0.22732290240923\\
125	0.0798645984348988\\
126	0.100308116841178\\
127	0.0869853473120074\\
128	0.153042019602568\\
129	-0.282473459336083\\
130	0.00167938058314608\\
131	-0.0668692982940619\\
132	0.340573246594684\\
133	0.101383537283402\\
134	0.153128721439757\\
135	-0.0225515513667713\\
136	0.0850346045170208\\
137	0.257659489851652\\
138	0.134822504798924\\
139	0.0161452839639641\\
140	-0.15772702724084\\
141	0.00416626455367591\\
142	0.156677954482728\\
143	0.195150362533003\\
144	0.00823489172142039\\
145	0.0665959869364099\\
146	0.0500482970538794\\
147	0.0392330862105301\\
148	0.216426980671116\\
149	0.160708215279891\\
150	0.175253288271324\\
151	0.171441017487954\\
152	0.0149874172867495\\
153	0.0625221919521225\\
154	0.0825572865332587\\
155	0.0547155864908492\\
156	0.058490385203267\\
157	-0.0693070489521058\\
158	0.063095140026463\\
159	-0.191704169063219\\
160	0.108825520224136\\
161	-0.156982715492995\\
162	-0.28203914522936\\
163	0.0743978044883306\\
164	-0.121542879949486\\
165	0.109117879553276\\
166	0.00988417645492044\\
167	-0.0186351783008256\\
168	0.103413433901317\\
169	0.0844681789726084\\
170	0.0241571341510587\\
171	0.0964149750578668\\
172	0.0697235101198282\\
173	0.151217538217463\\
174	0.0152478812507011\\
175	0.0469930948941769\\
176	0.0973377614788725\\
177	0.321871466660433\\
178	-0.0621356190449453\\
179	0.0547220381542117\\
180	0.0458075373278752\\
181	0.115797502019694\\
182	0.0458726525250665\\
183	0.0630251303152078\\
184	0.0289874319378473\\
185	0.0566523090266537\\
186	0.0464686055954117\\
187	0.0169041473736411\\
188	-0.0508784022615498\\
189	0.0101772592828562\\
190	0.0366323527817701\\
191	0.112460797610547\\
192	0.0335893508647951\\
193	0.0736036364243488\\
194	0.0178994386564783\\
195	-0.0307879324856051\\
196	0.140180273122916\\
197	0.0897795818278904\\
198	0.0254436555103005\\
199	0.132509305805736\\
200	0.20357032745515\\
201	0.0845321588225085\\
202	0.0958021331742496\\
203	0.00833383662002339\\
204	0.129727041431246\\
205	0.0267717333567808\\
206	0.143753897530091\\
207	0.0540151526406507\\
208	0.122186064640078\\
209	0.0761791761842384\\
210	0.0121409206149818\\
211	0.0826400930819239\\
212	0.00312684435651294\\
213	0.1407315318068\\
214	0.0472542886965665\\
215	0.0860742202844107\\
216	0.0223579836328475\\
217	0.154575549988418\\
218	-0.00386652868985407\\
219	-0.0101279877341988\\
220	0.00698156876633294\\
221	0.13803333410012\\
222	0.0362115875140923\\
223	0.112508556997716\\
224	0.00492072431518957\\
225	0.0141581289172317\\
226	-0.0197276703929077\\
227	0.0834549195547045\\
228	0.23227820315365\\
229	0.0732053400641598\\
230	0.0144336996587433\\
231	-0.0272787330211376\\
232	0.0238914469341749\\
233	0.162125941547513\\
234	0.0877290034606728\\
235	0.0417879921061364\\
236	0.155411021937718\\
237	-0.00941043830190998\\
238	0.138194685593565\\
239	0.027954220969785\\
240	0.00136754272925214\\
241	-0.0252780656416822\\
242	-0.0919386524572901\\
243	0.0665961182252593\\
244	-0.105726711005937\\
245	-0.0474528158135998\\
246	0.116695784046653\\
247	0.102962311067144\\
248	-0.0823809768660973\\
249	0.0429709001119464\\
250	0.0399148361903201\\
251	0.114042220458798\\
252	-0.026146132380294\\
};
\addplot [color=dscr_plot_color,solid,line width=1pt,forget plot]
  table[row sep=crcr]{%
1	0.400025196631865\\
2	0.289485874992995\\
3	0.143207326104979\\
4	0.243534612788485\\
5	0.23273237047548\\
6	0.321566798063289\\
7	0.353725948682474\\
8	0.240002928416267\\
9	0.25770585528097\\
10	0.219076703553454\\
11	0.298949557450466\\
12	0.39296100834028\\
13	0.45820569588233\\
14	0.292822969364162\\
15	0.164316992992746\\
16	0.34163596830614\\
17	0.172440767072072\\
18	0.20295896933454\\
19	0.0994961896218226\\
20	0.137024469542746\\
21	0.158075644492136\\
22	0.355102889629264\\
23	0.211411050710463\\
24	0.154976180694573\\
25	0.300136897018099\\
26	0.399675729145874\\
27	0.151307908407647\\
28	0.13135293513083\\
29	0.242607741538753\\
30	0.110744857252593\\
31	0.137701019293833\\
32	0.197685834163507\\
33	0.119475404105709\\
34	0.430259917507794\\
35	0.569928951324634\\
36	0.219365380469569\\
37	0.168043520159332\\
38	0.382184956949107\\
39	0.132895937597891\\
40	0.228181848396486\\
41	0.210621201201309\\
42	0.220479849762868\\
43	0.314370929058208\\
44	0.263582718270233\\
45	0.183039829840106\\
46	0.151693839090666\\
47	0.293368955598131\\
48	0.17021097670925\\
49	0.159629438891879\\
50	0.208881013474258\\
51	0.111307683897903\\
52	0.184636566297049\\
53	0.21355416240471\\
54	0.205207948532088\\
55	0.261933874424987\\
56	0.219852062839298\\
57	0.249392049128776\\
58	-0.0428184873773782\\
59	0.22287204255617\\
60	0.1320697395179\\
61	0.268097115925098\\
62	0.136297325280829\\
63	0.424003789442383\\
64	0.292833012482542\\
65	0.261727501422772\\
66	0.0886749255115792\\
67	-0.0610766578155111\\
68	0.505145055965026\\
69	0.350789225158246\\
70	0.237405158695785\\
71	0.222315992049437\\
72	0.438474496255641\\
73	0.414560734833353\\
74	0.483882398665746\\
75	0.368617517441074\\
76	0.416301586563577\\
77	0.197430177545447\\
78	0.431532323362498\\
79	0.392375703280779\\
80	0.329906495294956\\
81	0.0981611612556981\\
82	0.399812549200479\\
83	0.28243717140467\\
84	0.07633978815363\\
85	0.418005946039606\\
86	0.236327526412087\\
87	0.162709764878771\\
88	0.270992847249807\\
89	0.204143419121148\\
90	0.223875198219955\\
91	0.254775191205843\\
92	0.194024329600252\\
93	0.276556895402207\\
94	0.114518886463483\\
95	0.214271077989601\\
96	0.210659286640705\\
97	0.232079647257014\\
98	0.233110999959891\\
99	0.184291008316362\\
100	0.244636985945496\\
101	0.214831811122427\\
102	0.341040387263145\\
103	0.27435146237683\\
104	0.254566345764663\\
105	0.175157981890586\\
106	0.439605533212178\\
107	0.438615042639915\\
108	0.356455549434069\\
109	0.345698184639719\\
110	0.375471176434594\\
111	0.313113526357139\\
112	0.36757207715139\\
113	0.0709416221678153\\
114	0.268290358867691\\
115	0.276679141303827\\
116	0.275069204433356\\
117	0.093835704590033\\
118	0.428782612063659\\
119	0.293660498715605\\
120	0.503533149629406\\
121	0.308300109163703\\
122	0.179479697708016\\
123	0.338610481255123\\
124	0.203473906669954\\
125	0.190967714336239\\
126	0.230348134959505\\
127	0.0207556124967796\\
128	0.322062702413738\\
129	0.172771853905164\\
130	0.105973225770651\\
131	0.209468175513551\\
132	0.594849762799291\\
133	0.351307634057394\\
134	0.119695303316431\\
135	0.251372084375492\\
136	0.369175993778912\\
137	0.40972633504259\\
138	0.184532873559399\\
139	0.115845608055244\\
140	0.102774469028738\\
141	0.148209803836426\\
142	0.308731434451249\\
143	0.0389804948730261\\
144	0.178460586093457\\
145	0.145564931429177\\
146	0.228581298697616\\
147	0.172958927676682\\
148	0.290221498465205\\
149	0.226202489771086\\
150	0.224054262503606\\
151	0.164182116430292\\
152	0.118562806984135\\
153	0.19814779100876\\
154	0.272938211089922\\
155	0.297317450149003\\
156	0.151590062341798\\
157	0.0782426753595337\\
158	0.108338340676328\\
159	0.116780629145222\\
160	0.266310664860373\\
161	0.0375722879569656\\
162	0.334313994093827\\
163	0.193067591204357\\
164	0.0945683640197671\\
165	0.295460463182257\\
166	0.141025956010348\\
167	0.170059100252603\\
168	0.409777608871741\\
169	0.342821356730035\\
170	0.234790248838244\\
171	0.158906521124168\\
172	0.201765036954251\\
173	0.25845115186197\\
174	0.159288398134003\\
175	0.136266441463357\\
176	0.0939117395263683\\
177	0.478013239843091\\
178	0.111135213296954\\
179	0.204906635518907\\
180	0.211068461257253\\
181	0.212856584986603\\
182	0.1302657773789\\
183	0.264169912690202\\
184	0.130887314402289\\
185	0.148028749158154\\
186	0.136431043810465\\
187	0.262025212646355\\
188	0.1364443298524\\
189	0.241477403681927\\
190	0.207918187591207\\
191	0.334477331877957\\
192	0.25252338219494\\
193	0.142888169325405\\
194	0.385118980173844\\
195	0.259601950501781\\
196	0.242717423219597\\
197	0.334942807702739\\
198	0.277083386116088\\
199	0.263849276931672\\
200	0.164608737552717\\
201	0.311227764988255\\
202	0.161831156549722\\
203	0.124662159070175\\
204	0.190175905915047\\
205	0.154885138646797\\
206	0.152752678054191\\
207	0.221228556609678\\
208	0.306518282995364\\
209	0.283061067964265\\
210	0.170942426750386\\
211	0.159236096137171\\
212	0.226451057169982\\
213	0.172323019436497\\
214	0.209952635183128\\
215	0.249718523091676\\
216	0.285596128671762\\
217	0.194485128359416\\
218	0.117046625199036\\
219	0.215614699658539\\
220	0.310325721564494\\
221	0.198987490177524\\
222	0.159648581371898\\
223	0.153184993040198\\
224	0.225814155323589\\
225	0.236255134343643\\
226	0.38244016652343\\
227	0.535003461699192\\
228	0.367347870882156\\
229	0.12883689423119\\
230	0.118319169438353\\
231	0.0940884259534607\\
232	0.271043466169644\\
233	0.236413726507604\\
234	0.289174964409929\\
235	0.353370719316426\\
236	0.296373550871967\\
237	0.120196128157652\\
238	0.25854734740425\\
239	0.242399693511239\\
240	0.230987706946474\\
241	0.163328693908056\\
242	0.100431646128926\\
243	0.0969579056847306\\
244	0.337428050798324\\
245	0.16791435040434\\
246	0.134798980496439\\
247	0.230652077519846\\
248	0.0463452603567209\\
249	0.251665763229872\\
250	0.096994551905227\\
251	0.245968407349172\\
252	0.171368477784535\\
};
\addplot [color=cts_nFPC_plot_color,solid,line width=1pt,forget plot]
  table[row sep=crcr]{%
1	0.409380580106947\\
2	0.27193430943486\\
3	0.178633165957182\\
4	0.148859863916421\\
5	0.149620386058862\\
6	0.34361760692377\\
7	0.36614825032781\\
8	0.23673791757725\\
9	0.206894746722539\\
10	0.211356413735345\\
11	0.295780453536132\\
12	0.384865099253492\\
13	0.35781698684539\\
14	0.17438113070194\\
15	0.170063852865386\\
16	0.232028978916103\\
17	0.181874464902405\\
18	0.223114389515592\\
19	0.230471626661809\\
20	0.141764963810315\\
21	0.109620945485864\\
22	0.278292721953501\\
23	0.173892989797102\\
24	0.213824308912708\\
25	0.200100541272195\\
26	0.281239543824297\\
27	0.198786873667823\\
28	0.139241686630429\\
29	0.228269441644398\\
30	0.107632184443418\\
31	0.169870855548386\\
32	0.143766718059292\\
33	0.128061188216047\\
34	0.131630966596605\\
35	-0.0954846673123338\\
36	0.147181374377658\\
37	0.210790733801998\\
38	0.375070826982312\\
39	0.343891284089822\\
40	0.119193926766073\\
41	0.319590625082945\\
42	0.225559793021216\\
43	0.229237427561732\\
44	0.241921523780251\\
45	0.192909010350702\\
46	0.108443354208804\\
47	0.302865258354305\\
48	0.142924600521329\\
49	0.174855726139533\\
50	0.182751857268009\\
51	0.0918169160759578\\
52	0.24180243653747\\
53	0.228939581764732\\
54	0.185538531146522\\
55	0.272640832981793\\
56	0.223928193814315\\
57	0.18178910613155\\
58	0.414482576829753\\
59	0.354672731465507\\
60	0.111470427760016\\
61	0.133252489477692\\
62	0.181725478681812\\
63	0.0885336621433856\\
64	0.279642889218802\\
65	0.270429410455935\\
66	0.253138638387396\\
67	0.273434374873234\\
68	0.392720125825363\\
69	0.375823631244993\\
70	0.210807158008849\\
71	0.179171019038376\\
72	0.492557778980544\\
73	0.423404373501245\\
74	0.492272856813045\\
75	0.439761483108558\\
76	0.424347593855491\\
77	0.487760676427891\\
78	0.46097684216883\\
79	0.0590160939723585\\
80	0.288649108290585\\
81	0.235549722358523\\
82	0.317057606915869\\
83	0.145178019536038\\
84	0.286178880151742\\
85	0.20119686854961\\
86	0.114231095510779\\
87	0.153127713450194\\
88	0.250894245375757\\
89	0.173864482674641\\
90	0.223289013824408\\
91	0.0577492439533075\\
92	0.085000101744777\\
93	0.205718958380988\\
94	0.107677638286136\\
95	0.199032046652932\\
96	0.232388102792328\\
97	0.288248213805986\\
98	0.210344287020462\\
99	0.267256848592102\\
100	0.248611174180272\\
101	0.193721719649013\\
102	0.342035188023794\\
103	0.206343995638513\\
104	0.228219352708996\\
105	0.356067573191545\\
106	0.258668397240784\\
107	0.230974298471121\\
108	0.21193763186034\\
109	0.333248838462054\\
110	0.353555588826717\\
111	0.270876264851258\\
112	0.130311540427804\\
113	0.346860300506474\\
114	0.220663944374247\\
115	0.287830332001458\\
116	0.247232970046368\\
117	0.00411269993083423\\
118	0.201302330785844\\
119	0.244306484449321\\
120	0.110321484448345\\
121	0.286837380682217\\
122	0.192233138788586\\
123	0.286896762673856\\
124	0.467512441305329\\
125	0.136551270367038\\
126	0.200089086389807\\
127	0.139754707102902\\
128	0.376603083845216\\
129	-0.112362630923675\\
130	0.0963751543522127\\
131	0.160238131673804\\
132	0.494160971320074\\
133	0.192763598562732\\
134	0.258852819620232\\
135	0.198793739942111\\
136	0.0772412568234956\\
137	0.361338737252661\\
138	0.216348137619985\\
139	0.0799999247508229\\
140	0.0887463865524794\\
141	0.158350641839326\\
142	0.281346916587545\\
143	0.288344441948139\\
144	0.178514861911213\\
145	0.187919095374932\\
146	0.208220503424345\\
147	0.186405806747519\\
148	0.296180267554823\\
149	0.252002000595093\\
150	0.220689717195736\\
151	0.1475334615605\\
152	0.0781904373778997\\
153	0.166892732909485\\
154	0.250505436609991\\
155	0.12836360005422\\
156	0.151438178202751\\
157	0.0980722749930247\\
158	0.0428009275899794\\
159	0.203565147048779\\
160	0.23460762973018\\
161	-0.0928936078142307\\
162	0.362410744668385\\
163	0.221298677773816\\
164	0.10870159556042\\
165	0.32369033678377\\
166	0.179326842861759\\
167	0.11460198778673\\
168	0.224127949158287\\
169	0.236202271186356\\
170	0.214297383420241\\
171	0.16279919907474\\
172	0.225286230661745\\
173	0.280749481582824\\
174	0.118911143173455\\
175	0.126703391579931\\
176	0.0826144520914961\\
177	0.491622823111333\\
178	0.0493655248960054\\
179	0.168911722625979\\
180	0.30318804875908\\
181	0.201703543347544\\
182	0.0706610087682732\\
183	0.17794680869249\\
184	0.110157856966721\\
185	0.110230615842454\\
186	0.0914484597764864\\
187	0.0802714998621025\\
188	0.168402777423651\\
189	0.139017606521816\\
190	0.171176339974814\\
191	0.223574319050662\\
192	0.249414827143633\\
193	0.213222108211419\\
194	0.15198747869594\\
195	0.215945917084024\\
196	0.242437963951711\\
197	0.29537203569712\\
198	0.241708620475021\\
199	0.256598645296036\\
200	0.28042826758908\\
201	0.299629141091671\\
202	0.166764359209815\\
203	0.274088336778844\\
204	0.184108704454264\\
205	0.129093431066118\\
206	0.168039784552491\\
207	0.251668373495046\\
208	0.31561851388323\\
209	0.184706359043372\\
210	0.142017640380836\\
211	0.117052993005956\\
212	0.0869291264767735\\
213	0.1909669015156\\
214	0.189691005175916\\
215	0.183460272819179\\
216	0.173166750528578\\
217	0.202027719958232\\
218	0.0930957313862146\\
219	0.138249271373509\\
220	0.305953074424776\\
221	0.210265138386714\\
222	0.244593446158739\\
223	0.221718798607988\\
224	0.210299006473937\\
225	0.157382541847867\\
226	0.335655189075567\\
227	0.155814562687372\\
228	0.235590066510059\\
229	0.0848620520949471\\
230	0.194292567063813\\
231	0.0376709917765818\\
232	0.145787988029175\\
233	0.266685767386097\\
234	0.283173014459159\\
235	0.346831904405895\\
236	0.283766440153289\\
237	0.171911302548391\\
238	0.190836017686627\\
239	-0.018794808486488\\
240	0.0811257407635893\\
241	0.053872168241526\\
242	0.0261874708864481\\
243	0.177658788689333\\
244	0.295416479804735\\
245	0.172544246960114\\
246	0.0304703150618709\\
247	0.172742471818508\\
248	0.0832171581474543\\
249	0.233200000360545\\
250	0.0297006908133427\\
251	0.218301474930152\\
252	0.122818939325019\\
};
\addplot [color=dscr_nFPC_plot_color,solid,line width=1pt,forget plot]
  table[row sep=crcr]{%
1	0.283273087692138\\
2	0.240316778927671\\
3	0.137653392760366\\
4	0.201236934782089\\
5	0.221140275876572\\
6	0.251296212310154\\
7	0.270557769072591\\
8	0.217977026580689\\
9	0.201250083044183\\
10	0.185208306121869\\
11	0.238339465663979\\
12	0.261169771904389\\
13	0.338185918164731\\
14	0.228966183863086\\
15	0.126543002623788\\
16	0.151940509535204\\
17	0.129892623893044\\
18	0.168731034247013\\
19	0.251184515727683\\
20	0.1583458534619\\
21	0.164593960788328\\
22	0.279105868178504\\
23	0.17376133822048\\
24	0.171316029127863\\
25	0.172821809253734\\
26	0.252598091995637\\
27	0.168954357199089\\
28	0.100642958012295\\
29	0.188485610610372\\
30	0.0989764054929328\\
31	0.132799948994482\\
32	0.136071627038489\\
33	0.127254994636462\\
34	0.08087665797576\\
35	0.111354823448533\\
36	0.142328870532445\\
37	0.130205052838473\\
38	0.348205073989094\\
39	0.275314450989157\\
40	0.136299055886808\\
41	0.320090601721998\\
42	0.277624593230822\\
43	0.257289582595687\\
44	0.224395373978443\\
45	0.213228047228006\\
46	0.140799475714779\\
47	0.29921413510376\\
48	0.137393880035162\\
49	0.147384908525771\\
50	0.197323559869352\\
51	0.116622001217631\\
52	0.174138123754495\\
53	0.139537260621815\\
54	0.180161290517066\\
55	0.211560413598572\\
56	0.15125387860992\\
57	0.125787149595859\\
58	0.377173961655109\\
59	0.26554493146359\\
60	0.116986877582731\\
61	0.056453939702643\\
62	0.154792580815771\\
63	0.12999231655439\\
64	0.204947200277858\\
65	0.209665157436112\\
66	0.194888746683896\\
67	0.294524304577694\\
68	0.419785927631256\\
69	0.336487305463687\\
70	0.184217482744818\\
71	0.138335161388858\\
72	0.345454118439998\\
73	0.363480053973329\\
74	0.359592653684877\\
75	0.24855273768956\\
76	0.33459961092932\\
77	0.324792148310527\\
78	0.363512295235183\\
79	0.178902745393211\\
80	0.26282220608998\\
81	0.16998484223114\\
82	0.263019176501658\\
83	0.183664923988541\\
84	0.238934331833857\\
85	0.148155833967507\\
86	0.131405047048942\\
87	0.114923145850592\\
88	0.227926738503707\\
89	0.157054960525412\\
90	0.172325139680059\\
91	0.114043160615563\\
92	0.0675482699161709\\
93	0.259976044856538\\
94	0.12063786444581\\
95	0.178178491144231\\
96	0.250204231541912\\
97	0.213122960428201\\
98	0.205246879298343\\
99	0.19422336790515\\
100	0.214020931823415\\
101	0.18836614260077\\
102	0.268840973091279\\
103	0.176192979489467\\
104	0.220030265393452\\
105	0.385065069201689\\
106	0.297235770043008\\
107	0.190663506137877\\
108	0.199613710179333\\
109	0.345840245466668\\
110	0.301675180345023\\
111	0.210672103905532\\
112	0.110655011953572\\
113	0.347288591725608\\
114	0.191450442542822\\
115	0.245353941828268\\
116	0.228123935413177\\
117	0.126394819916494\\
118	0.256853779744032\\
119	0.293139118367735\\
120	0.214653964445232\\
121	0.294154935349773\\
122	0.124279370181123\\
123	0.276192811541769\\
124	0.45951329793021\\
125	0.175305012966312\\
126	0.186092583923094\\
127	0.165932777248396\\
128	0.278157298857628\\
129	0.112953435447527\\
130	0.150969216962796\\
131	0.218189567929002\\
132	0.49036693108004\\
133	0.28789781485678\\
134	0.22838365469535\\
135	0.205794561316854\\
136	0.164159648177851\\
137	0.367575111792948\\
138	0.210280503476175\\
139	0.0895855818902679\\
140	0.10673885401738\\
141	0.113912371979412\\
142	0.22395712697017\\
143	0.241405766611521\\
144	0.12774057546991\\
145	0.133102895071512\\
146	0.132456514168308\\
147	0.178479549178216\\
148	0.235783750522147\\
149	0.194451896430442\\
150	0.232691407344574\\
151	0.133024873179756\\
152	0.110139243817735\\
153	0.159785045004232\\
154	0.193643744646674\\
155	0.191344866554914\\
156	0.175868871562936\\
157	0.107915475541562\\
158	0.0846068408555347\\
159	0.268030891426995\\
160	0.203658402509391\\
161	0.0508564099185756\\
162	0.176993373097338\\
163	0.196957232190911\\
164	0.141260081948049\\
165	0.232706056231707\\
166	0.157594387012881\\
167	0.147582672748645\\
168	0.20574509674136\\
169	0.162553219305875\\
170	0.225110208133347\\
171	0.162450741126946\\
172	0.206768424752949\\
173	0.219456510400262\\
174	0.163042393274452\\
175	0.130001100939952\\
176	0.0761447236803113\\
177	0.459698210875096\\
178	0.128157552239729\\
179	0.186392285368906\\
180	0.212710476948183\\
181	0.178402319391698\\
182	0.146430611396294\\
183	0.148017251278876\\
184	0.0983002951631574\\
185	0.0919187014010666\\
186	0.184831650690781\\
187	0.111439416106827\\
188	0.170579944139033\\
189	0.115653536587723\\
190	0.15838335891106\\
191	0.257248991952529\\
192	0.170759971397798\\
193	0.165232398129494\\
194	0.11122537234962\\
195	0.203180169512002\\
196	0.189460428867282\\
197	0.272761983575787\\
198	0.210208070092263\\
199	0.223625344823139\\
200	0.276918743366371\\
201	0.262106504556999\\
202	0.142909868159749\\
203	0.254316477075702\\
204	0.197011312048393\\
205	0.131448404084982\\
206	0.1245363354743\\
207	0.171582931655926\\
208	0.22235204774326\\
209	0.252182392440717\\
210	0.134548510607566\\
211	0.123938888858439\\
212	0.0886640687956931\\
213	0.136045242535645\\
214	0.162052673779445\\
215	0.153550337855269\\
216	0.143325695584478\\
217	0.176495716993906\\
218	0.0793907205854168\\
219	0.193708915239853\\
220	0.256557800957156\\
221	0.196553332515899\\
222	0.192383535404911\\
223	0.153964367190734\\
224	0.206541446830496\\
225	0.145044240403608\\
226	0.325670435538243\\
227	0.207984538466702\\
228	0.23075179976574\\
229	0.12456462402548\\
230	0.173456457851575\\
231	0.0896133221556759\\
232	0.146806969809239\\
233	0.195924987979096\\
234	0.25524905466495\\
235	0.292511388073308\\
236	0.285128107836885\\
237	0.155291041618852\\
238	0.20710478613284\\
239	0.0579010948829961\\
240	0.124683825178111\\
241	0.124424328729406\\
242	0.0995039212783834\\
243	0.187818890061266\\
244	0.255363058788249\\
245	0.210114773516928\\
246	0.117299555109926\\
247	0.213112131471332\\
248	0.132084222046144\\
249	0.2059814896587\\
250	0.0935808953524389\\
251	0.235375107982328\\
252	0.144734221927228\\
};
\end{axis}
\end{tikzpicture}%
 
\end{subfigure}\\

\leavevmode\smash{\makebox[0pt]{\hspace{-7em}% HORIZONTAL POSITION           
  \rotatebox[origin=l]{90}{\hspace{7em}% VERTICAL POSITION
    Normalized PnL}%
}}\hspace{0pt plus 1filll}\null

Trading Day Number of 2013

\vspace{1cm}
\begin{subfigure}{\linewidth}
  %\centering
  \setlength\figureheight{\linewidth} 
  \setlength\figurewidth{\linewidth}
  \tikzsetnextfilename{strategylegend}
  \resizebox{\linewidth}{!}{\definecolor{mycolor1}{rgb}{0.25098,0.00000,0.38824}%
\definecolor{mycolor2}{rgb}{0.00000,0.46275,0.00000}%
\definecolor{mycolor3}{rgb}{0.00000,0.34902,0.34902}%
\definecolor{mycolor4}{rgb}{0.58039,0.26275,0.00000}%
\begin{tikzpicture}
    \begingroup
    % inits/clears the lists (which might be populated from previous
    % axes):
    \csname pgfplots@init@cleared@structures\endcsname
    \pgfplotsset{legend style={at={(0,1)},anchor=north west},legend columns=-1,legend style={draw=black,column sep=1ex},
            legend entries={Cts Stoch Ctrl,Dscr Stoch Ctrl,Cts Stoch Ctrl w nFPC,Dscr Stoch Ctrl w nFPC}}%
    
    \csname pgfplots@addlegendimage\endcsname{line width=2pt,mycolor1,sharp plot}
    \csname pgfplots@addlegendimage\endcsname{line width=2pt,mycolor2,sharp plot}
    \csname pgfplots@addlegendimage\endcsname{line width=2pt,mycolor3,sharp plot}
    \csname pgfplots@addlegendimage\endcsname{line width=2pt,mycolor4,sharp plot}

    % draws the legend:
    \csname pgfplots@createlegend\endcsname
    \endgroup
\end{tikzpicture}
}
\end{subfigure}%
  \caption{End of day strategy performances: in-sample backtesting on 2013 data, using amalgamated annual 2013 data for calibration.}
  \label{fig:IS_annual_comp}
\end{figure}

\begin{table}
\centering
\ra{1.2}
\begin{tabular}{@{} *{9}{r} @{}}
\toprule
Strategy & Return & Sharpe & \# MO & \# LO & Inv & \% Win & Max Loss & Max Win \\
\midrule
\multicolumn{9}{l}{\texttt{ORCL}} \\ 
Cts & -0.089 & -0.875 & 1540 & 1383 & 1.19 & 0.14 & -0.705 & 0.393 \\ 
Dscr & 0.140 & 1.596 & 368 & 1344 & 0.46 & 0.96 & -0.088 & 0.487 \\ 
Cts w nFPC & 0.113 & 1.327 & 476 & 1338 & 2.67 & 0.94 & -0.140 & 0.484 \\ 
Dscr w nFPC & 0.118 & 1.735 & 590 & 1337 & 3.43 & 0.99 & -0.028 & 0.474 \\[2ex]
\multicolumn{9}{l}{\texttt{INTC}} \\ 
Cts & 0.065 & 0.743 & 888 & 1207 & 1.22 & 0.84 & -0.282 & 0.341 \\ 
Dscr & 0.235 & 2.189 & 380 & 1170 & 1.19 & 0.99 & -0.061 & 0.595 \\ 
Cts w nFPC & 0.209 & 2.030 & 396 & 1160 & 5.58 & 0.98 & -0.112 & 0.494 \\ 
Dscr w nFPC & 0.197 & 2.588 & 576 & 1164 & 3.78 & 1.00 &  & 0.490 \\ 
\bottomrule
\end{tabular}
\caption{Averaged strategy performance results: in-sample backtesting on 2013 data, using amalgamated annual 2013 data for calibration.}
\label{tbl:IS_annual}
\end{table}

\FloatBarrier
\section{Out-of-Sample Backtesting}
From the in-sample backtesting we can conclude that xxx.

\begin{figure}
\centering
\begin{subfigure}{.45\linewidth}
  \centering
  \setlength\figureheight{\linewidth} 
  \setlength\figurewidth{\linewidth}
  \tikzsetnextfilename{OOS_annual_INTC}
  % This file was created by matlab2tikz.
%
%The latest updates can be retrieved from
%  http://www.mathworks.com/matlabcentral/fileexchange/22022-matlab2tikz-matlab2tikz
%where you can also make suggestions and rate matlab2tikz.
%
%
\begin{tikzpicture}[trim axis left, trim axis right]

\begin{axis}[%
width=\figurewidth,
height=\figureheight,
at={(0\figurewidth,0\figureheight)},
scale only axis,
every outer x axis line/.append style={black},
every x tick label/.append style={font=\color{black}},
xmin=1,
xmax=252,
%xlabel={Time (h)},
every outer y axis line/.append style={black},
every y tick label/.append style={font=\color{black}},
ymin=-0.5,
ymax=2.5,
%ylabel={Normalized PnL},
title={INTC},
axis background/.style={fill=white},
axis x line*=bottom,
axis y line*=left,
yticklabel style={
        /pgf/number format/fixed,
        /pgf/number format/precision=3
},
scaled y ticks=false,
]
\addplot [color=cts_plot_color,solid,line width=1pt,forget plot]
  table[row sep=crcr]{%
1	0.26963564527033\\
2	0.0812791859976301\\
3	0.258353823089972\\
4	0.15375066826374\\
5	0.228086326367345\\
6	0.00266454176664818\\
7	0.150739182461208\\
8	-0.0644458160201035\\
9	0.250525165811975\\
10	-0.0285372062002428\\
11	0.203747624534175\\
12	0.22683416062517\\
13	0.204539234207424\\
14	0.13565308087828\\
15	0.0317687692307175\\
16	0.150204777884517\\
17	0.13271364583099\\
18	0.134897480299307\\
19	0.131319491506605\\
20	0.146890840586434\\
21	0.216529851766005\\
22	0.158858310518688\\
23	0.238241452954011\\
24	0.270521049539988\\
25	0.143719556639425\\
26	0.189414497397985\\
27	0.149020596653023\\
28	0.207340316730591\\
29	0.23792092507481\\
30	0.137422051983286\\
31	0.192376382841841\\
32	0.169928619540353\\
33	0.264152170973586\\
34	0.153683635295779\\
35	0.142520021356101\\
36	0.180871890047619\\
37	0.152846540684806\\
38	0.180991992124792\\
39	0.214234918752381\\
40	0.218064673442335\\
41	0.178464442986413\\
42	0.192860129299274\\
43	0.0842375583401652\\
44	0.384279415327714\\
45	0.173144733589689\\
46	0.228212061854239\\
47	0.151652151261517\\
48	0.204532786024088\\
49	0.163170341401239\\
50	0.211462893201068\\
51	0.185078717063482\\
52	0.1987472756297\\
53	0.194047354948597\\
54	0.174244446842754\\
55	0.189119742796312\\
56	0.199030406767797\\
57	0.193003098013475\\
58	0.217258056419755\\
59	0.173391125020354\\
60	0.143682148706881\\
61	0.148468507898605\\
62	0.198986770769308\\
63	0.185457370152702\\
64	0.150938247954111\\
65	0.125606029397477\\
66	0.168335192432876\\
67	0.138808441298083\\
68	0.137400435760667\\
69	0.15187022714063\\
70	0.140644766268518\\
71	0.0929942134693372\\
72	0.215480323637217\\
73	0.262338504529912\\
74	0.145758450019345\\
75	0.151275450604712\\
76	0.0534438078011343\\
77	0.0956144844396152\\
78	0.107329466952255\\
79	0.045108910668381\\
80	0.149376952011719\\
81	0.175905502929114\\
82	0.211292600547571\\
83	0.22565364241326\\
84	0.260802075855425\\
85	0.156382146075261\\
86	0.258945342929534\\
87	0.0850168600720129\\
88	0.265570399568249\\
89	0.0854455046515042\\
90	0.215153502438407\\
91	0.202230887113168\\
92	0.488961611221297\\
93	0.210220284619878\\
94	0.216692334027058\\
95	0.154499998447018\\
96	0.141034120002053\\
97	0.196537666281983\\
98	0.0716830272821748\\
99	0.175614630399463\\
100	0.392026200357144\\
101	0.315821525763047\\
102	0.141645708196059\\
103	0.034904645034563\\
104	0.223932864294619\\
105	-0.00301139156718667\\
106	0.179475025467556\\
107	0.288106415124689\\
108	0.270322782227211\\
109	0.231763037444338\\
110	0.235044492783833\\
111	0.242151485785364\\
112	0.283142805497629\\
113	0.410239161511713\\
114	0.457329738987022\\
115	0.163295557384188\\
116	0.0576422344031585\\
117	0.357266903302322\\
118	0.248757746039589\\
119	0.242130630474179\\
120	0.157057700486511\\
121	0.309180048060807\\
122	0.283555469628606\\
123	0.251315341750423\\
124	0.356696698483964\\
125	0.458118033458421\\
126	0.352989048833637\\
127	0.435612823980314\\
128	0.323941215981985\\
129	0.359916633729929\\
130	0.472228428689651\\
131	0.245733320362786\\
132	0.290200915209536\\
133	0.0715036032832169\\
134	0.281097669088852\\
135	0.265406976307438\\
136	0.170036310559606\\
137	0.17719066740288\\
138	0.280646708738482\\
139	-0.168861468056154\\
140	0.216660049689778\\
141	0.207726978849197\\
142	0.135314904905273\\
143	0.11650868619581\\
144	0.151975250219257\\
145	0.217026410045547\\
146	0.15524410913185\\
147	0.206243551871065\\
148	0.0699133688479681\\
149	0.141622012723338\\
150	0.241747506628732\\
151	0.224252632294149\\
152	0.161701823075847\\
153	0.10739841021324\\
154	0.292867162922703\\
155	0.236723852248063\\
156	0.301778375246838\\
157	0.258859523234779\\
158	0.222663439838195\\
159	0.245020590082906\\
160	0.224907286079295\\
161	0.252239218797636\\
162	0.186127122384629\\
163	0.170460229110415\\
164	0.207290277373665\\
165	0.364478133554319\\
166	0.223792380821682\\
167	0.342369486794757\\
168	0.229921450699335\\
169	0.273706321413039\\
170	0.174287954053077\\
171	0.190691419275898\\
172	0.0941464507426687\\
173	0.240257226040656\\
174	0.294616548373612\\
175	0.32186066540851\\
176	0.485112630185237\\
177	0.190239194471975\\
178	0.279471961772849\\
179	0.271365925377265\\
180	0.195290886392595\\
181	0.231170779113027\\
182	0.184397587115079\\
183	0.134930576506312\\
184	0.224089720228972\\
185	0.278560185795872\\
186	0.121069174307033\\
187	0.133158089650564\\
188	0.50178704744723\\
189	0.362567554371031\\
190	0.240806600831905\\
191	0.020784597456922\\
192	0.384424121703955\\
193	0.208887052544037\\
194	0.165595879016578\\
195	0.311471846871098\\
196	0.180777500213411\\
197	0.229751859247128\\
198	0.059276885845721\\
199	0.102799770615309\\
200	0.174098004828106\\
201	0.254053910153812\\
202	0.128744654348108\\
203	0.490190973864465\\
204	0.321416793127848\\
205	0.268611762334539\\
206	0.292449905369116\\
207	0.378407822058957\\
208	0.346020049635607\\
209	0.233442674888068\\
210	0.250753983279198\\
211	0.174683799564852\\
212	0.164391811815535\\
213	0.190571676502179\\
214	0.135897002147125\\
215	0.313678451824508\\
216	0.336960341145604\\
217	0.213052382876897\\
218	0.300299501516183\\
219	0.215036330072671\\
220	0.219294973212254\\
221	0.382081482096947\\
222	0.381633897082985\\
223	0.24677589631451\\
224	0.21403387585827\\
225	0.0686160494410331\\
226	0.198850541000567\\
227	0.233261180209299\\
228	0.0645541918868825\\
};
\addplot [color=dscr_plot_color,solid,line width=1pt,forget plot]
  table[row sep=crcr]{%
1	0.441997056889738\\
2	0.458233304284025\\
3	0.418471389199401\\
4	0.419234295384876\\
5	0.127552429123174\\
6	0.0796121196023828\\
7	0.202368102971599\\
8	0.271997015033251\\
9	0.0857094245573307\\
10	0.263960810630823\\
11	0.0554872054849208\\
12	0.236343824454965\\
13	0.0209659235606106\\
14	0.257873000756714\\
15	0.0608117960876181\\
16	0.174978399376741\\
17	0.32004932114849\\
18	0.24266544543884\\
19	0.240914574064539\\
20	0.323668351713643\\
21	0.310469306838916\\
22	0.225709470087306\\
23	0.107706802380786\\
24	0.330137661943864\\
25	-0.0101153514236141\\
26	0.131490150492565\\
27	0.315546644353201\\
28	0.37517513504854\\
29	0.130409224088904\\
30	0.101828419323733\\
31	0.0998818754669931\\
32	0.508742884606918\\
33	0.462665254414907\\
34	0.432326193122878\\
35	0.467169477869723\\
36	0.437176474362773\\
37	0.157355344246135\\
38	0.407835337669032\\
39	0.338243433626207\\
40	0.124076619144464\\
41	-0.0405716182279245\\
42	0.232958706445838\\
43	0.167458586462093\\
44	0.504429284894597\\
45	0.349568451452751\\
46	0.615332434672446\\
47	0.447192573173332\\
48	0.451207318256972\\
49	0.585387680158007\\
50	0.0896605918252703\\
51	0.721380422989914\\
52	0.533777667296209\\
53	0.140101469021222\\
54	0.180596690307499\\
55	0.236319225699871\\
56	0.397199609860212\\
57	0.194928572246658\\
58	0.351168328545488\\
59	0.252754659627726\\
60	0.123120692509247\\
61	0.123347838899411\\
62	0.253683698135599\\
63	0.423955495912861\\
64	0.0983369543420489\\
65	0.0130255437892288\\
66	0.08318138711644\\
67	0.225741836150945\\
68	0.379246813378094\\
69	0.0895323112943477\\
70	0.267174781542726\\
71	0.245486387600415\\
72	0.301704848320736\\
73	0.222532495158896\\
74	0.289105044726425\\
75	0.1421450258803\\
76	0.0565186572378707\\
77	-0.0090057084367307\\
78	0.231017407906282\\
79	0.31270517007323\\
80	0.265077921282586\\
81	0.157779619894752\\
82	0.173797116426202\\
83	0.326594662928318\\
84	0.0175443499031492\\
85	0.236100307304464\\
86	0.242425591463638\\
87	0.199519975733049\\
88	0.513603538272743\\
89	0.209908850004939\\
90	0.444400332593423\\
91	0.210945642470453\\
92	0.652785629998448\\
93	0.297957427163893\\
94	0.290466476971488\\
95	0.152633902308132\\
96	0.162930520827283\\
97	0.148302635533775\\
98	0.108377719925966\\
99	0.369756303400631\\
100	0.197213773837795\\
101	0.363949556126501\\
102	0.309348394742462\\
103	0.292370757128491\\
104	0.417333972480264\\
105	0.291049431909279\\
106	0.271067874727222\\
107	0.364220408611214\\
108	0.389582320392795\\
109	-0.0216957169073585\\
110	0.394158027682089\\
111	0.345934500395396\\
112	0.434094443967498\\
113	0.446988486558916\\
114	0.965020714428128\\
115	0.262264857615027\\
116	0.491818002746111\\
117	0.563558199130077\\
118	0.195332712364632\\
119	0.47197638072801\\
120	0.306874655592442\\
121	0.316062120176029\\
122	0.134614102600616\\
123	0.189209550147757\\
124	0.610803997841028\\
125	0.773172715096116\\
126	0.131160295087331\\
127	0.903739306448733\\
128	0.393172845883635\\
129	0.524763642248934\\
130	0.728699231507389\\
131	0.103518770517311\\
132	0.472974223004033\\
133	-0.0133687739298978\\
134	0.214201321169624\\
135	0.18482438021321\\
136	0.385421759030822\\
137	0.298936137228685\\
138	0.467121218345748\\
139	0.312275694654746\\
140	0.231948058844622\\
141	0.345688708520898\\
142	0.30596408697264\\
143	0.229122166643363\\
144	0.290419900275788\\
145	0.211774128098993\\
146	0.380140896150045\\
147	0.396805253353331\\
148	0.114987179307665\\
149	0.443290266916919\\
150	0.615359237397012\\
151	0.322190260820541\\
152	0.448351699883985\\
153	0.291794908096244\\
154	0.478597493771408\\
155	0.531014666351418\\
156	0.231478249055541\\
157	0.457496655346694\\
158	0.42143405905386\\
159	0.55153244337475\\
160	0.386384231224012\\
161	0.422942166322309\\
162	0.252613395439516\\
163	0.318789151753325\\
164	0.394194261985352\\
165	0.651806146605628\\
166	0.495521213851066\\
167	0.680707809063964\\
168	0.69822257954559\\
169	0.539427359618056\\
170	0.62496861118153\\
171	0.573095622141557\\
172	0.429544268763001\\
173	0.950166893392434\\
174	1.41792238398612\\
175	0.814800920828198\\
176	0.968861581868494\\
177	1.12895162725433\\
178	1.26668843735552\\
179	0.884959708604826\\
180	0.801532307436512\\
181	0.367328912433657\\
182	0.679598936067132\\
183	0.648422353695511\\
184	0.194714623632761\\
185	0.745110531290363\\
186	0.556700477237455\\
187	0.391545597904276\\
188	0.955124080948623\\
189	0.421607199484748\\
190	0.42106135006137\\
191	0.677710526542774\\
192	0.322448526774788\\
193	0.482191507386851\\
194	0.50367914683191\\
195	0.535553328270742\\
196	0.391783169714067\\
197	0.464787066713261\\
198	0.253043310973074\\
199	0.357658202308609\\
200	0.560817580638952\\
201	0.634138613836285\\
202	0.270348573752544\\
203	0.347741867118189\\
204	0.776140050772969\\
205	0.3901587047626\\
206	0.325488227131837\\
207	0.0712602916953453\\
208	0.361444723650599\\
209	0.318604996768405\\
210	0.448470618353861\\
211	0.264678811250693\\
212	0.197332535165579\\
213	0.720233609765952\\
214	0.501477842815664\\
215	0.511393651859714\\
216	0.29403130395263\\
217	0.892152892925317\\
218	0.710760536886796\\
219	0.623916832187122\\
220	0.255456816336281\\
221	0.442414153165319\\
222	0.806856491594673\\
223	0.256680637304243\\
224	0.534325025619459\\
225	0.0833247533131612\\
226	0.396014089362232\\
227	0.334916075613173\\
228	0.563247735737518\\
};
\addplot [color=cts_nFPC_plot_color,solid,line width=1pt,forget plot]
  table[row sep=crcr]{%
1	0.6143130460286\\
2	0.479930045169186\\
3	0.395687042528969\\
4	0.437680042861911\\
5	0.339412958095725\\
6	0.268583810098898\\
7	0.214741031069543\\
8	0.259989868511312\\
9	0.365682632156957\\
10	0.261371395812812\\
11	0.325121448560579\\
12	0.358345956387868\\
13	0.336931285447313\\
14	0.195134682784106\\
15	0.306473805844229\\
16	0.283472686814215\\
17	0.339020300929932\\
18	0.357114288622482\\
19	0.279691256840746\\
20	0.385718894427536\\
21	0.314944909234571\\
22	0.229341799515829\\
23	0.320358906572409\\
24	0.421980101396289\\
25	0.288599031723427\\
26	0.28223425709022\\
27	0.302793797708563\\
28	0.3851793902035\\
29	0.38323615022619\\
30	0.249959668586474\\
31	0.288848474529826\\
32	0.494468165623729\\
33	0.474606410430397\\
34	0.338058625019283\\
35	0.470671354705772\\
36	0.467075685549844\\
37	0.391668357741845\\
38	0.481964455628875\\
39	0.350921896784284\\
40	0.397572497620522\\
41	0.340835024510644\\
42	0.307555999868785\\
43	0.513283632669537\\
44	0.708241722672039\\
45	0.715256802026324\\
46	0.57651944450147\\
47	0.41455306651409\\
48	0.530996918914574\\
49	0.540449524887044\\
50	0.566215379881527\\
51	0.72668052256212\\
52	0.672422060412128\\
53	0.421132753233327\\
54	0.294486786651721\\
55	0.375290488637248\\
56	0.397771752437363\\
57	0.379520390483487\\
58	0.385436348368195\\
59	0.523440446631788\\
60	0.321760128221773\\
61	0.356777677997279\\
62	0.381024224555597\\
63	0.412970324091504\\
64	0.29411464088543\\
65	0.341986464474481\\
66	0.458760927536746\\
67	0.352059362915712\\
68	0.369615974203577\\
69	0.312037162787901\\
70	0.26682291862681\\
71	0.25775898336278\\
72	0.463459629157065\\
73	0.440889359704075\\
74	0.308115343416008\\
75	0.313947832265368\\
76	0.288464284088744\\
77	0.20112925319551\\
78	0.244941812053245\\
79	0.321729251208538\\
80	0.266891123062674\\
81	0.241290245588802\\
82	0.371884435521172\\
83	0.330283611456308\\
84	0.419558393629129\\
85	0.249081168070741\\
86	0.397607253089301\\
87	0.308391157851021\\
88	0.38482423389251\\
89	0.350035763773043\\
90	0.327141943033486\\
91	0.349920348898649\\
92	0.860732190446409\\
93	0.490696014004677\\
94	0.332274720034596\\
95	0.331837771997554\\
96	0.347288929152006\\
97	0.28183379533041\\
98	0.371908043123139\\
99	0.534427972798377\\
100	0.600488041210758\\
101	0.464816979699518\\
102	0.328770456986064\\
103	0.319545216410469\\
104	0.434670651070785\\
105	0.317602655621071\\
106	0.300558370027516\\
107	0.492895826334102\\
108	0.436140561080273\\
109	0.496162960244502\\
110	0.397342679498699\\
111	0.34350110621099\\
112	0.605828989718724\\
113	1.12160114247488\\
114	0.877520174669197\\
115	0.601985857132689\\
116	0.505117046294082\\
117	0.580147029450329\\
118	0.364701368002105\\
119	0.461390103271463\\
120	0.475961791452906\\
121	0.636312675435545\\
122	0.461184839970346\\
123	0.524407487976284\\
124	0.627356385918564\\
125	0.779287921647462\\
126	0.528132719464313\\
127	0.806808394032914\\
128	0.713179706889436\\
129	0.644634376292113\\
130	0.750479293565096\\
131	0.466589214432448\\
132	0.471596066345577\\
133	0.591344207392925\\
134	0.475290201438928\\
135	0.579483599004693\\
136	0.0920632659524396\\
137	0.305517603023685\\
138	0.448485306231749\\
139	-0.156220844596798\\
140	0.377601692703897\\
141	0.321355729328189\\
142	0.269449593531328\\
143	0.230196222342016\\
144	0.297067668534244\\
145	0.400282916784218\\
146	0.29773848365603\\
147	0.405130075847202\\
148	0.212734903030616\\
149	0.435126486220765\\
150	0.602114149302921\\
151	0.503808779652743\\
152	0.42710373673571\\
153	0.507270310228639\\
154	0.688778305910511\\
155	0.56841094598978\\
156	0.624685768111744\\
157	0.483911950107121\\
158	0.43197846366263\\
159	0.514815411763773\\
160	0.594440882113467\\
161	0.446133995355554\\
162	0.470665743973826\\
163	0.614662874395886\\
164	0.578876589338569\\
165	0.79398237024307\\
166	0.616583372615095\\
167	0.753937576151404\\
168	0.513372693258185\\
169	0.570371098114885\\
170	0.628141878020989\\
171	0.541215334926928\\
172	0.819242822106041\\
173	0.974200058605348\\
174	1.1428078838766\\
175	1.19442572473451\\
176	0.99447293097858\\
177	0.934276617957967\\
178	1.15350480992241\\
179	0.81820187395044\\
180	0.540691902475096\\
181	0.267658458706957\\
182	0.704204288496644\\
183	0.676524025615314\\
184	0.473728962275772\\
185	0.506625115894345\\
186	0.340421751589954\\
187	0.343948646288413\\
188	1.01198712198112\\
189	0.797312783345937\\
190	0.650620432524172\\
191	0.181751201276961\\
192	0.644353240978866\\
193	0.48010862920146\\
194	0.547020736786603\\
195	0.429320415872187\\
196	0.415714348731314\\
197	0.486656626384004\\
198	0.576145975331464\\
199	0.23402762688151\\
200	0.574212014030322\\
201	0.629827558237091\\
202	0.547943448198315\\
203	0.758425130020845\\
204	0.613676310362735\\
205	0.706066874063763\\
206	0.546946790915805\\
207	0.644994233831756\\
208	0.858924282139581\\
209	0.382715157200909\\
210	0.706751384614788\\
211	0.60271264361827\\
212	0.452936961606829\\
213	0.641518292045949\\
214	0.941727566678151\\
215	0.834407948052902\\
216	0.864800493013229\\
217	0.710179449388589\\
218	0.993267266063656\\
219	0.836689268716272\\
220	0.647034644669472\\
221	0.552319077243002\\
222	0.680901338974365\\
223	0.346174738416085\\
224	0.41058763900373\\
225	0.288588065988688\\
226	0.436397261638319\\
227	0.405515094425726\\
228	0.450649936257121\\
};
\addplot [color=dscr_nFPC_plot_color,solid,line width=1pt,forget plot]
  table[row sep=crcr]{%
1	0.61183707372905\\
2	0.466620129440472\\
3	0.394683124208813\\
4	0.437022836894612\\
5	0.344573007145789\\
6	0.282851712672088\\
7	0.197207407774155\\
8	0.267218441654412\\
9	0.352372969236646\\
10	0.261484994028186\\
11	0.301613950547206\\
12	0.324321085967892\\
13	0.319203853570605\\
14	0.193670456555661\\
15	0.301887798736334\\
16	0.274618007001055\\
17	0.333518577183689\\
18	0.366021308168541\\
19	0.248674800961759\\
20	0.398834743605126\\
21	0.276907388594029\\
22	0.233783670958193\\
23	0.332944360032859\\
24	0.418581845809891\\
25	0.296090048655316\\
26	0.273159968348499\\
27	0.304526968525399\\
28	0.370341944333678\\
29	0.34449717965225\\
30	0.253937264852245\\
31	0.296780191115426\\
32	0.486003506011523\\
33	0.458172884369276\\
34	0.333708821644002\\
35	0.462627588633211\\
36	0.431960709590673\\
37	0.373107971250565\\
38	0.448924510940559\\
39	0.33334121923638\\
40	0.388203124130375\\
41	0.340447211926684\\
42	0.32742636934098\\
43	0.490659028835647\\
44	0.718549820702009\\
45	0.670509686346988\\
46	0.598172555645278\\
47	0.436633672612086\\
48	0.509143832143232\\
49	0.551869076374524\\
50	0.53214049759813\\
51	0.719395596354385\\
52	0.686159032286584\\
53	0.424892664560722\\
54	0.319932380990227\\
55	0.354325652330032\\
56	0.373804329198854\\
57	0.381014367797213\\
58	0.398937162641295\\
59	0.53954522058515\\
60	0.347548468617298\\
61	0.359514983246846\\
62	0.380252442821248\\
63	0.403043454867974\\
64	0.300420036702677\\
65	0.335704252356505\\
66	0.433742198071703\\
67	0.343009562421407\\
68	0.355035669913453\\
69	0.299669809463617\\
70	0.255010718434588\\
71	0.237623542440356\\
72	0.441655633110088\\
73	0.417229909972626\\
74	0.277405652787983\\
75	0.312250795184475\\
76	0.285206997869146\\
77	0.192416921825856\\
78	0.244699634034665\\
79	0.319392525372523\\
80	0.273615756350554\\
81	0.234098171453855\\
82	0.345795162830772\\
83	0.308326468758271\\
84	0.409363212153008\\
85	0.24978418167321\\
86	0.369596719215507\\
87	0.310088791961515\\
88	0.368596578058314\\
89	0.306042448086602\\
90	0.321482156500573\\
91	0.355035190359323\\
92	0.876045539876714\\
93	0.472307336675702\\
94	0.316690064588974\\
95	0.318955325077107\\
96	0.355925626333873\\
97	0.290895711414747\\
98	0.372477427031126\\
99	0.531165242461047\\
100	0.582814797433697\\
101	0.474795200066837\\
102	0.349562171000063\\
103	0.305562670786552\\
104	0.416969105461709\\
105	0.303036569756862\\
106	0.288727095077769\\
107	0.480284693747882\\
108	0.386268607123281\\
109	0.484777221701001\\
110	0.39838347963515\\
111	0.332223180031924\\
112	0.610375759876839\\
113	1.22934602650305\\
114	0.923197050723522\\
115	0.574684147374411\\
116	0.478881922705708\\
117	0.558833030217281\\
118	0.388078279830345\\
119	0.46155486135149\\
120	0.457026590166114\\
121	0.634457999430658\\
122	0.441220416437272\\
123	0.530989908958742\\
124	0.643813195859474\\
125	0.75780192173065\\
126	0.529499419613188\\
127	0.811385260812246\\
128	0.711894681611713\\
129	0.630476049675469\\
130	0.715368563828518\\
131	0.456320742191189\\
132	0.478137451812145\\
133	0.590157955531636\\
134	0.425839528538212\\
135	0.589922320978104\\
136	0.36260767905207\\
137	0.289002892665452\\
138	0.460498205436308\\
139	0.320992096460161\\
140	0.360295826079102\\
141	0.338387682884993\\
142	0.275147006254992\\
143	0.224866615019689\\
144	0.287742833153562\\
145	0.388742470207407\\
146	0.386866286895132\\
147	0.426206333167008\\
148	0.412320434805305\\
149	0.429643201720446\\
150	0.614149912256539\\
151	0.480322179227098\\
152	0.447881796988888\\
153	0.488366938099405\\
154	0.66034536132001\\
155	0.532795255170833\\
156	0.644817367688909\\
157	0.517119759577625\\
158	0.427461619817519\\
159	0.490598804493259\\
160	0.636849920990711\\
161	0.461819470358364\\
162	0.464586044099153\\
163	0.645982020528883\\
164	0.547049407391825\\
165	0.778966710029067\\
166	0.595285276132172\\
167	0.738129379116837\\
168	0.646473260784414\\
169	0.555043508355322\\
170	0.603213532473551\\
171	0.509503545879259\\
172	0.844802719081944\\
173	0.92802465980639\\
174	1.55398272353678\\
175	1.1342870808597\\
176	1.33754657326245\\
177	2.07162688110104\\
178	1.48405035960051\\
179	1.02974185784727\\
180	0.772742110300907\\
181	0.78889126881632\\
182	0.652203178397897\\
183	0.618920410516353\\
184	0.687614006461787\\
185	0.747948863885012\\
186	0.559829799337336\\
187	0.596254225566214\\
188	1.02855992251955\\
189	0.803903610425394\\
190	0.634162482408358\\
191	0.649654588813321\\
192	0.617738661328253\\
193	0.590543927761638\\
194	0.531900513415453\\
195	0.512827672932275\\
196	0.435104748886764\\
197	0.486466963956121\\
198	0.527544433379397\\
199	0.528124417781775\\
200	0.541514120653209\\
201	0.587564613594762\\
202	0.488679523640645\\
203	1.33838736344592\\
204	0.708431844757965\\
205	0.66967559080376\\
206	0.524206902858684\\
207	0.609217289937578\\
208	0.852130301356892\\
209	0.739238028597512\\
210	0.677504640390143\\
211	0.55882001008687\\
212	0.443030014053196\\
213	0.878715210639802\\
214	0.956193332481327\\
215	0.778955134733541\\
216	0.887600466724404\\
217	0.901012765523267\\
218	0.955994055705178\\
219	1.01593272600577\\
220	1.00390361795808\\
221	0.816658835565289\\
222	0.68737504970322\\
223	0.705917331394664\\
224	0.528333013403345\\
225	0.277274120974258\\
226	0.401202478704402\\
227	0.383694447047517\\
228	0.479012741293375\\
};
\end{axis}
\end{tikzpicture}%

\end{subfigure}%
\hfill%
\begin{subfigure}{.45\linewidth}
  \centering
  \setlength\figureheight{\linewidth} 
  \setlength\figurewidth{\linewidth}
%  \tikzsetnextfilename{OOS_annual_AAPL}
%  % This file was created by matlab2tikz.
%
%The latest updates can be retrieved from
%  http://www.mathworks.com/matlabcentral/fileexchange/22022-matlab2tikz-matlab2tikz
%where you can also make suggestions and rate matlab2tikz.
%
%
\begin{tikzpicture}[trim axis left, trim axis right]

\begin{axis}[%
width=\figurewidth,
height=\figureheight,
at={(0\figurewidth,0\figureheight)},
scale only axis,
every outer x axis line/.append style={black},
every x tick label/.append style={font=\color{black}},
xmin=1,
xmax=252,
%xlabel={Time (h)},
every outer y axis line/.append style={black},
every y tick label/.append style={font=\color{black}},
ymin=-0.5,
ymax=2.5,
%ylabel={Normalized PnL},
title={AAPL},
axis background/.style={fill=white},
axis x line*=bottom,
axis y line*=left,
yticklabel style={
        /pgf/number format/fixed,
        /pgf/number format/precision=3
},
scaled y ticks=false,
]
\addplot [color=mycolor1,solid,line width=0.5pt,forget plot]
  table[row sep=crcr]{%
1	0.510987910390083\\
2	0.50493899806683\\
3	0.54333235039816\\
4	0.497479959177795\\
5	0.498866792659851\\
6	0.583676323725802\\
7	0.523166608602107\\
8	0.435874820908009\\
9	0.567830213116335\\
10	0.518426123215793\\
11	0.435644759110937\\
12	0.436771436444335\\
13	0.468266810055287\\
14	0.580023023511957\\
15	0.46352318185389\\
16	0.437173201793071\\
17	0.377340237152658\\
18	0.644177546256084\\
19	0.488407638526946\\
20	0.473846235744826\\
21	0.328366867926963\\
22	0.411824769917912\\
23	0.257628422878803\\
24	0.419336926576297\\
25	0.284582981199671\\
26	0.551447511265182\\
27	0.343543349787213\\
28	0.470603633864958\\
29	0.395418892323493\\
30	0.435937969426288\\
31	0.366070562879117\\
32	0.393248546480879\\
33	0.322909586908209\\
34	0.433433353306567\\
35	0.459980964463616\\
36	0.522403580488045\\
37	0.440135131708724\\
38	0.407740874646837\\
39	0.359496854881543\\
40	0.212926998888887\\
41	0.0708760847158397\\
42	0.120855795950528\\
43	0.227491801523226\\
44	0.457576096279289\\
45	0.409559707438116\\
46	0.176433420422788\\
47	0.117898075662125\\
48	0.474786049178819\\
49	0.289103379809835\\
50	0.123284435041617\\
51	0.355734383354284\\
52	0.241006285248412\\
53	0.135990744828084\\
54	0.348533898052694\\
55	0.0374594163111564\\
56	0.416587117781412\\
57	0.507876772983342\\
58	0.748064449494588\\
59	1.02485187605485\\
60	0.573285316623427\\
61	0.567844134447056\\
62	0.473715049866141\\
63	0.419608150960151\\
64	0.53416621312512\\
65	0.404616468188424\\
66	0.454675816563084\\
67	0.399771432829398\\
68	0.486904313452766\\
69	0.434953267733016\\
70	0.370062267833206\\
71	0.33021234317118\\
72	0.376413371174028\\
73	0.480376772247229\\
74	0.595014729615327\\
75	0.359143532690909\\
76	0.381733434019907\\
77	0.331805625875993\\
78	0.489379006333125\\
79	0.607107849238533\\
80	0.525190544353949\\
81	0.60136129887222\\
82	0.736187211753486\\
83	0.590861209859749\\
84	0.54081257054675\\
85	0.549193877837785\\
86	0.513938123262339\\
87	0.563538927572976\\
88	1.39174913662952\\
89	1.06123649601527\\
90	0.909921213617733\\
91	0.870278732836959\\
92	0.869686433140742\\
93	0.670608290242773\\
94	0.599688346873008\\
95	0.591595198860895\\
96	0.62536824396781\\
97	0.599380940066415\\
98	0.608238454014307\\
99	0.572288105692664\\
100	0.684094620826021\\
101	0.264464953575519\\
102	0.242272183874215\\
103	0.286686858049356\\
104	0.440685455801696\\
105	0.433852067585813\\
106	0.254831895547666\\
107	0.864063620586768\\
108	0.227658085017016\\
109	0.00807582560044849\\
110	0.206283731387004\\
111	0.120221406652525\\
112	0.317286039208535\\
113	0.389013192673175\\
114	0.441713587462249\\
115	0.0583722396375666\\
116	0.344621790605737\\
117	0.168916486344438\\
118	-0.0265651591675015\\
119	0.154059026035574\\
120	0.0938252319507085\\
121	-0.016160260163571\\
122	0.237119467306883\\
123	0.177862639984112\\
124	0.521481897491611\\
125	0.0900766131325231\\
126	0.388269595772568\\
127	0.225595336495003\\
128	0.320536904903836\\
129	0.333620645793912\\
130	0.240874820690198\\
131	0.0503960929392621\\
132	0.239320678484777\\
133	-0.000976002136468723\\
134	0.12464431947248\\
135	0.241596975901332\\
136	0.133327137827351\\
137	0.121822078766212\\
138	0.122419301940224\\
139	0.167293373148065\\
140	0.0981874192905491\\
141	0.146526835025966\\
142	0.255848991808334\\
143	-0.0325730096281847\\
144	0.157770982340905\\
145	0.183713842621101\\
146	0.190241392691877\\
147	0.984068324753296\\
148	0.304072732178635\\
149	0.0930642553527545\\
150	0.340735276245742\\
151	0.556380221872139\\
152	-0.0170946287958654\\
153	0.0288735844085977\\
154	0.433914554861058\\
155	0.277351848471341\\
156	0.291285349057398\\
157	0.854272414320157\\
158	0.484865397465263\\
159	0.664879775022002\\
160	0.249591700864865\\
161	0.249940209786659\\
162	0.368704231617722\\
163	0.682903821955991\\
164	0.0659224181477588\\
165	0.847220557837157\\
166	0.337313521156171\\
167	0.434664951730829\\
168	0.320559975020865\\
169	0.193371944435979\\
170	0.31022860042775\\
171	0.333773847928745\\
172	0.19618191008808\\
173	0.591095055851438\\
174	0.579012455065681\\
175	0.695330152417599\\
176	0.643074899823606\\
177	1.004276615396\\
178	0.751402356130219\\
179	0.438676136204901\\
180	0.149257931471269\\
181	0.263571433492917\\
182	0.258680367087105\\
183	0.278454690789311\\
184	0.292016068969924\\
185	0.238189085057527\\
186	0.102324625661413\\
187	0.162482953670979\\
188	0.225289623547628\\
189	0.202583113576629\\
190	0.198730035172748\\
191	0.172514968095086\\
192	0.208502729628415\\
193	0.130171157342652\\
194	0.210914337667021\\
195	0.183132856647801\\
196	0.0948476083282829\\
197	-0.0151176467513982\\
198	0.25021846742453\\
199	0.0824319451460317\\
200	0.232247987500126\\
201	-0.094486593950673\\
202	0.378670181803119\\
203	0.0591114198662748\\
204	0.436557111418969\\
205	-0.0825038280628991\\
206	0.612610999445504\\
207	0.0145678775478772\\
208	0.852020514419013\\
209	0.0220201228365819\\
210	0.249433094466648\\
211	0.211775295229903\\
212	0.215543504570929\\
213	0.510812445064065\\
214	0.0424863680110536\\
215	0.640344373483863\\
216	0.466728203057596\\
217	0.710346831073175\\
218	1.0832979135758\\
219	0.706973517141022\\
220	0.618711183426892\\
221	0.405786491068403\\
222	0.416112907257908\\
223	0.31620314832479\\
224	0.258904007018663\\
225	-0.0286929094134621\\
226	0.154216793162184\\
227	0.180243562066718\\
228	0.380460190665007\\
};
\addplot [color=mycolor2,solid,line width=0.5pt,forget plot]
  table[row sep=crcr]{%
1	0.572635042165019\\
2	0.525371269501357\\
3	0.544273463843038\\
4	0.547893744695127\\
5	0.573476863400697\\
6	0.643483010002963\\
7	0.531551497826567\\
8	0.407054615353053\\
9	0.624037773736192\\
10	0.53512297544551\\
11	0.45468548974163\\
12	0.510906798837668\\
13	0.502453952383931\\
14	0.566745870574096\\
15	0.547159396725753\\
16	0.471147187418905\\
17	0.396233396559053\\
18	0.701668524082678\\
19	0.56005406423644\\
20	0.495003931317454\\
21	0.352615470588501\\
22	0.457295451043031\\
23	0.258543741797873\\
24	0.436719387826526\\
25	0.318297728040905\\
26	0.577368424738107\\
27	0.367655938271027\\
28	0.487547671081847\\
29	0.415472216315897\\
30	0.473525460460465\\
31	0.420294586361968\\
32	0.410321559384572\\
33	0.342244805015436\\
34	0.469239534803419\\
35	0.515853781871504\\
36	0.582156120922253\\
37	0.457307041263558\\
38	0.394788288771593\\
39	0.385113330389144\\
40	0.235948719101105\\
41	0.400942916606883\\
42	0.304571079423648\\
43	0.305014238699459\\
44	0.489321671155054\\
45	0.530632960497969\\
46	0.470639655579753\\
47	0.531030631836087\\
48	0.484307348082753\\
49	0.571534545518245\\
50	0.430643074971713\\
51	0.521244600778244\\
52	0.506112204817425\\
53	0.59799964193009\\
54	0.460493800920078\\
55	0.39972939372257\\
56	0.528341612534132\\
57	0.93251378411044\\
58	0.788039219503593\\
59	1.24239217102466\\
60	0.620827361725037\\
61	0.662399567897206\\
62	0.507839562313441\\
63	0.422417112958209\\
64	0.559172422393611\\
65	0.534423352350451\\
66	0.528338248908195\\
67	0.404106186040187\\
68	0.544200762071391\\
69	0.465003498754473\\
70	0.387289421212477\\
71	0.341424620413702\\
72	0.37659582977998\\
73	0.524530446070264\\
74	0.685860252446102\\
75	0.359902859577453\\
76	0.385004767468789\\
77	0.346797286265547\\
78	0.500149506443763\\
79	0.693192511741897\\
80	0.554846888224278\\
81	0.639765997564432\\
82	0.773161132834359\\
83	0.698713619656617\\
84	0.688899915413481\\
85	0.645967822859769\\
86	0.590945687821289\\
87	0.630998860581595\\
88	1.54002203675934\\
89	1.10219062938273\\
90	0.941410750300312\\
91	0.876347096642116\\
92	0.927675955264196\\
93	0.736457138499131\\
94	0.597346193606566\\
95	0.646684734793865\\
96	0.634874205297542\\
97	0.62835258551894\\
98	0.675595337205851\\
99	0.653796405358777\\
100	0.737388547687769\\
101	0.62777820059426\\
102	0.79542268298354\\
103	0.760843019957251\\
104	0.756773110365475\\
105	0.556725751476698\\
106	1.09330571320605\\
107	0.983215959480354\\
108	0.714636106396028\\
109	0.809368527850809\\
110	0.687341057001014\\
111	0.734101462980017\\
112	0.617561335357786\\
113	0.574611519573093\\
114	0.965952128940091\\
115	0.886939048224717\\
116	0.594250570789462\\
117	0.665176312112676\\
118	1.19935417898177\\
119	0.662515194071923\\
120	0.798063171721589\\
121	0.794905268038086\\
122	0.665431213690731\\
123	0.663331997098926\\
124	0.91094476579677\\
125	1.02506151872686\\
126	0.776122190650635\\
127	0.999822137144747\\
128	0.813001698992531\\
129	0.806469121232209\\
130	0.787598789362484\\
131	0.61572210610844\\
132	0.64772597630447\\
133	0.570987441888115\\
134	0.564444970523129\\
135	0.813461409503097\\
136	0.735398332580561\\
137	0.880953871432894\\
138	0.69222242115663\\
139	0.49523489812302\\
140	0.825142218352522\\
141	0.56425658731651\\
142	0.646557779219151\\
143	0.748906682165934\\
144	0.781736851212993\\
145	0.531121478290193\\
146	0.828659661972717\\
147	1.2677332950082\\
148	1.2009992569898\\
149	0.983329293004528\\
150	0.754969416845888\\
151	2.03911088916297\\
152	1.68534178230356\\
153	0.934638706885989\\
154	1.04101196580143\\
155	0.95167029581856\\
156	1.10052258309991\\
157	0.985975851774984\\
158	0.64656458366843\\
159	0.735104591313398\\
160	0.905149290819487\\
161	0.99097536566092\\
162	0.988659418150452\\
163	1.493241778725\\
164	1.07319944541052\\
165	0.90944017259873\\
166	0.888126281901074\\
167	0.910214867476074\\
168	1.01239819194476\\
169	0.783286350063079\\
170	0.786179638031346\\
171	0.849510337039679\\
172	1.2498639358168\\
173	1.38835374953801\\
174	1.2943758283755\\
175	1.16579547386066\\
176	1.40344779060007\\
177	1.92716133241095\\
178	1.55072475231388\\
179	1.17975451895814\\
180	1.06893169096079\\
181	1.14143952076062\\
182	1.05464850543528\\
183	0.936937774868218\\
184	0.771663667539874\\
185	0.716077728209522\\
186	0.832706228134726\\
187	0.919150639201475\\
188	0.857306274313929\\
189	0.725683012190071\\
190	1.00790703624562\\
191	0.755606867478911\\
192	0.695699263453031\\
193	0.634550146903165\\
194	0.67165677282236\\
195	0.522734593201175\\
196	0.627253510728576\\
197	0.767766578774296\\
198	1.09444026153036\\
199	0.802462147153355\\
200	0.89928481419554\\
201	0.746587125979502\\
202	0.889974266938327\\
203	0.867220307656495\\
204	0.911802165797712\\
205	0.88268123479031\\
206	1.08727121017108\\
207	0.812756995270985\\
208	1.43015262232194\\
209	1.24128104528242\\
210	0.87627933153593\\
211	0.806449328934028\\
212	0.753983080570651\\
213	1.21543876792055\\
214	1.53399873691462\\
215	1.05810667044878\\
216	1.10045843413441\\
217	1.18213350261358\\
218	1.50811364266658\\
219	1.47155875025947\\
220	1.46118264030364\\
221	1.25802298488489\\
222	1.16027912581942\\
223	0.921543427357968\\
224	0.648692744037811\\
225	0.778791275789721\\
226	0.727660391711022\\
227	0.661829859084177\\
228	0.876544836797172\\
};
\addplot [color=mycolor3,solid,line width=0.5pt,forget plot]
  table[row sep=crcr]{%
1	0.527013146032987\\
2	0.48628432283232\\
3	0.522973185561087\\
4	0.508828838082412\\
5	0.522409077325843\\
6	0.577311592731682\\
7	0.509378530488385\\
8	0.39896233910821\\
9	0.542667264872306\\
10	0.515872005250187\\
11	0.416546408079227\\
12	0.469539047479547\\
13	0.471262235064219\\
14	0.592222545441584\\
15	0.451203534280804\\
16	0.449823397531805\\
17	0.371180866256153\\
18	0.607399875882882\\
19	0.511206413126081\\
20	0.456925041132563\\
21	0.332134492494144\\
22	0.405654196348516\\
23	0.247929965935124\\
24	0.410925015429706\\
25	0.278497770577268\\
26	0.515211180533612\\
27	0.328237461580854\\
28	0.468354144416127\\
29	0.393698412954098\\
30	0.420735534620526\\
31	0.369983001347717\\
32	0.393406182075785\\
33	0.318810755781903\\
34	0.424333956775293\\
35	0.45197342804232\\
36	0.515579865987258\\
37	0.416400685259106\\
38	0.407839645652108\\
39	0.346114512317509\\
40	0.234109943204588\\
41	0.360818600160251\\
42	0.287161727598284\\
43	0.276328209136919\\
44	0.460010376495236\\
45	0.437908891396074\\
46	0.434171673700962\\
47	0.419241931432476\\
48	0.442441792061359\\
49	0.485111411499514\\
50	0.403264115155883\\
51	0.485031817001208\\
52	0.441481319597241\\
53	0.509139789136969\\
54	0.407626020094027\\
55	0.36275346833271\\
56	0.512151192209563\\
57	0.84456574003006\\
58	0.724620926245467\\
59	0.950683653412964\\
60	0.57845036623355\\
61	0.567699785885092\\
62	0.469751720401259\\
63	0.406687353013248\\
64	0.521256375904405\\
65	0.438986942588531\\
66	0.449350064327186\\
67	0.398197454670796\\
68	0.499971651817582\\
69	0.423758670303705\\
70	0.332857371786908\\
71	0.311940257747638\\
72	0.382665585076899\\
73	0.474071930629586\\
74	0.594359976019543\\
75	0.359279669243655\\
76	0.370916159704761\\
77	0.329997932417529\\
78	0.474826548616345\\
79	0.583579407461127\\
80	0.510933416273131\\
81	0.5897407828434\\
82	0.713589519710599\\
83	0.61465941502711\\
84	0.527333391725127\\
85	0.548473513556272\\
86	0.500555334611658\\
87	0.527047696728748\\
88	1.40479776223565\\
89	1.03198653817977\\
90	0.91771637170633\\
91	0.86744104187522\\
92	0.903990957156392\\
93	0.685572197176585\\
94	0.596676712370254\\
95	0.598635108513479\\
96	0.622123437724841\\
97	0.596881680282268\\
98	0.608757345223565\\
99	0.60256051917148\\
100	0.721969155737155\\
101	0.630528333130804\\
102	0.767320331159116\\
103	0.789767368971804\\
104	0.739379499647864\\
105	0.518871476200234\\
106	0.981071546442119\\
107	0.906496063059201\\
108	0.70360087340482\\
109	0.755562361395981\\
110	0.640021218342395\\
111	0.685232289578357\\
112	0.624939466069253\\
113	0.622322644059531\\
114	0.878197905195552\\
115	0.845055476433579\\
116	0.643890227317995\\
117	0.696685451274484\\
118	1.11639017747854\\
119	0.640210545515495\\
120	0.795721218136573\\
121	0.806656860518244\\
122	0.654060807058385\\
123	0.607271001406742\\
124	0.918422786390599\\
125	0.90609645766652\\
126	0.771032068074388\\
127	0.974015774815021\\
128	0.783591541472671\\
129	0.744971522913362\\
130	0.778007324245103\\
131	0.594474194205859\\
132	0.624254835512701\\
133	0.556729800051238\\
134	0.551638641700323\\
135	0.808062769143168\\
136	0.731988122822473\\
137	0.85103919266452\\
138	0.653369201744911\\
139	0.491849406840323\\
140	0.780605107054779\\
141	0.517237476445531\\
142	0.623673354856675\\
143	0.682511820177687\\
144	0.739958425964614\\
145	0.511453648023659\\
146	0.776831176285804\\
147	1.28130605114685\\
148	1.07548053047316\\
149	0.82310424365751\\
150	0.732531511718311\\
151	1.60500830445564\\
152	1.4261072291789\\
153	0.850466314618884\\
154	0.99534655944459\\
155	0.85045224747868\\
156	1.05313316772091\\
157	0.902488572334056\\
158	0.672516603521923\\
159	0.713571192892587\\
160	0.907579955149111\\
161	0.945679549020306\\
162	0.938556955022972\\
163	1.38130553598916\\
164	1.03287190510014\\
165	0.93015744338379\\
166	0.867973475972517\\
167	0.895542691050974\\
168	1.00415129028628\\
169	0.725210307977858\\
170	0.79879133918383\\
171	0.791557908092451\\
172	1.1561272497787\\
173	1.32944033993067\\
174	1.20339516964102\\
175	1.13725439202096\\
176	1.29222567636411\\
177	1.80303487409323\\
178	1.58299084686873\\
179	1.13856940151076\\
180	0.979020720871724\\
181	1.02680890272232\\
182	1.00943414558605\\
183	0.903116062906233\\
184	0.752471853706713\\
185	0.685979158780566\\
186	0.766120184834204\\
187	0.887134543376903\\
188	0.87404978625041\\
189	0.667394784375535\\
190	0.933887631078078\\
191	0.668007163996999\\
192	0.64975615016881\\
193	0.584845856297677\\
194	0.635870218019499\\
195	0.490961925753884\\
196	0.59669659252607\\
197	0.721791020441764\\
198	0.995935668952698\\
199	0.682791103903349\\
200	0.836700426015663\\
201	0.686017377066084\\
202	0.848252328844145\\
203	0.790383604599665\\
204	0.841922082035889\\
205	0.76831350475571\\
206	1.01145468348288\\
207	0.7224784784437\\
208	0.977310129834231\\
209	0.98428511542579\\
210	0.838339405391479\\
211	0.78819792946779\\
212	0.741045341939021\\
213	1.1482434427019\\
214	1.35642301201117\\
215	1.04440629731667\\
216	1.0495885963816\\
217	1.19694920180998\\
218	1.42580567828732\\
219	1.44711298169312\\
220	1.45726146428035\\
221	1.23322571472921\\
222	1.09817753707606\\
223	0.908983180048035\\
224	0.616097616728871\\
225	0.720872500917441\\
226	0.66214419416451\\
227	0.603137699242246\\
228	0.820104113439825\\
};
\addplot [color=mycolor4,solid,line width=0.5pt,forget plot]
  table[row sep=crcr]{%
1	0.549968256347378\\
2	0.529162917392576\\
3	0.576395197253715\\
4	0.526678335871561\\
5	0.539975706503024\\
6	0.632483232115493\\
7	0.539582045476306\\
8	0.439506059045925\\
9	0.613158090984689\\
10	0.555394311798505\\
11	0.465541103710198\\
12	0.493356656994018\\
13	0.51042874695835\\
14	0.59510944024271\\
15	0.532978506719299\\
16	0.465081493699681\\
17	0.415816050672101\\
18	0.69052704081578\\
19	0.583859503130896\\
20	0.491959776551032\\
21	0.37324536770292\\
22	0.440973661208236\\
23	0.270097545899761\\
24	0.461925295370307\\
25	0.316710046783775\\
26	0.579432884829812\\
27	0.358465875835747\\
28	0.514080265460167\\
29	0.416399192755409\\
30	0.448240410496426\\
31	0.394581799440262\\
32	0.421524185321602\\
33	0.32458897153917\\
34	0.455944832426325\\
35	0.520795109637355\\
36	0.566303946438182\\
37	0.477427066975175\\
38	0.414388774236902\\
39	0.354649151325897\\
40	0.243666051947293\\
41	0.422988960388855\\
42	0.300397187009708\\
43	0.312854643837493\\
44	0.472316110722165\\
45	0.505605616035618\\
46	0.46484248756415\\
47	0.465652120302308\\
48	0.483660377592654\\
49	0.545204440840352\\
50	0.436947018410702\\
51	0.51133002568318\\
52	0.475607811667886\\
53	0.581177363519852\\
54	0.453402850408356\\
55	0.39486082169253\\
56	0.553770968950097\\
57	0.886678039186445\\
58	0.832978135318167\\
59	1.19104418823622\\
60	0.648237349747213\\
61	0.61996907429614\\
62	0.519346529758409\\
63	0.419539773553721\\
64	0.563319786866697\\
65	0.494177348271629\\
66	0.514299860319154\\
67	0.424315287090703\\
68	0.526553092390261\\
69	0.457650465566024\\
70	0.360895162151618\\
71	0.342834617318656\\
72	0.399550602495141\\
73	0.518248197954059\\
74	0.654648032301695\\
75	0.370563382661612\\
76	0.406004172878663\\
77	0.350617779368454\\
78	0.513581330983037\\
79	0.643475399588613\\
80	0.544146811535351\\
81	0.672169760977239\\
82	0.79586575866552\\
83	0.697151958472333\\
84	0.64104783073341\\
85	0.605282791916975\\
86	0.582840652378036\\
87	0.632677011108779\\
88	1.49728435274927\\
89	1.16059734637857\\
90	0.959549778538781\\
91	0.901931511505811\\
92	0.971675566942361\\
93	0.746076643465853\\
94	0.604197514019656\\
95	0.65738008198325\\
96	0.655912739646712\\
97	0.624580041873934\\
98	0.669973175721948\\
99	0.664721383162933\\
100	0.798097053860049\\
101	0.672829511574294\\
102	0.81368515527951\\
103	0.816648115158968\\
104	0.693179790783279\\
105	0.551079044919066\\
106	1.08170781066368\\
107	0.924839040091002\\
108	0.74882756401343\\
109	0.786218879413868\\
110	0.65756151699529\\
111	0.74992348861896\\
112	0.655669026627503\\
113	0.625202631630227\\
114	0.990353636079273\\
115	0.896857526768284\\
116	0.644724670988897\\
117	0.743869904974882\\
118	1.18177348376587\\
119	0.658468910744254\\
120	0.824455372349693\\
121	0.81400181033904\\
122	0.703975438104185\\
123	0.630186426223042\\
124	0.965166479114125\\
125	0.976269236393907\\
126	0.745466808667985\\
127	1.04168193482869\\
128	0.804381241851947\\
129	0.800901865473201\\
130	0.790463516442074\\
131	0.627026346124255\\
132	0.643704767599913\\
133	0.561307399573567\\
134	0.579348336001171\\
135	0.807270925967026\\
136	0.740698784136677\\
137	0.901711607909822\\
138	0.686179287851513\\
139	0.508073153953537\\
140	0.823909870381861\\
141	0.554137141109483\\
142	0.654060847657772\\
143	0.721756343353926\\
144	0.779529226047671\\
145	0.531017504345761\\
146	0.8006212187954\\
147	1.33266777842362\\
148	1.24985569092528\\
149	0.935960180904252\\
150	0.771005895799184\\
151	2.18018428102887\\
152	1.69161368211221\\
153	1.00038171093251\\
154	1.04280617061884\\
155	0.948522403921837\\
156	1.14850712404679\\
157	0.939472361399624\\
158	0.694276811330867\\
159	0.740906915630114\\
160	0.939777070117689\\
161	1.01869003984382\\
162	0.976205995572552\\
163	1.43561690584349\\
164	1.06248323603473\\
165	0.938467914833425\\
166	0.927424439354192\\
167	0.922253188003247\\
168	1.00044375170679\\
169	0.765288308279841\\
170	0.774352022498486\\
171	0.818970218398529\\
172	1.22295451410339\\
173	1.41834361348849\\
174	1.31186379353161\\
175	1.20018216310318\\
176	1.36145158461151\\
177	1.88402908727715\\
178	1.57384216265156\\
179	1.19704471287921\\
180	1.07338403341461\\
181	1.11218570608091\\
182	1.06947436116827\\
183	0.965656166006739\\
184	0.776058806057357\\
185	0.7070261302204\\
186	0.821812090308181\\
187	0.943818787711232\\
188	0.871184159906395\\
189	0.715180653441377\\
190	1.00064359360506\\
191	0.76497218294952\\
192	0.722006515249328\\
193	0.635286545640924\\
194	0.648597611874939\\
195	0.511488062072491\\
196	0.617196157989858\\
197	0.80319428326138\\
198	1.10259133580778\\
199	0.796242832121232\\
200	0.970899529573888\\
201	0.736560379665237\\
202	0.88403753862138\\
203	0.896699858279534\\
204	0.901912016009564\\
205	0.843569634284451\\
206	1.13739703593133\\
207	0.837456812215811\\
208	1.18495691303143\\
209	1.19158121374004\\
210	0.894671909199632\\
211	0.831011460238132\\
212	0.811970015943177\\
213	1.23048187142181\\
214	1.57146604683578\\
215	1.11925007577664\\
216	1.09848564717802\\
217	1.20834252232944\\
218	1.63264444958277\\
219	1.48362999633601\\
220	1.38547278532348\\
221	1.27869662370463\\
222	1.14868649081685\\
223	0.920040030239484\\
224	0.659517992192129\\
225	0.760530726445564\\
226	0.70403284527378\\
227	0.66205098498412\\
228	0.875006123471908\\
};
\end{axis}
\end{tikzpicture}%
 
\end{subfigure}\\

\leavevmode\smash{\makebox[0pt]{\hspace{-7em}% HORIZONTAL POSITION           
  \rotatebox[origin=l]{90}{\hspace{7em}% VERTICAL POSITION
    Normalized PnL}%
}}\hspace{0pt plus 1filll}\null

Trading Day Number of 2014

\vspace{1cm}
\begin{subfigure}{\linewidth}
  %\centering
  \setlength\figureheight{\linewidth} 
  \setlength\figurewidth{\linewidth}
  \tikzsetnextfilename{strategylegend}
  \resizebox{\linewidth}{!}{\definecolor{mycolor1}{rgb}{0.25098,0.00000,0.38824}%
\definecolor{mycolor2}{rgb}{0.00000,0.46275,0.00000}%
\definecolor{mycolor3}{rgb}{0.00000,0.34902,0.34902}%
\definecolor{mycolor4}{rgb}{0.58039,0.26275,0.00000}%
\begin{tikzpicture}
    \begingroup
    % inits/clears the lists (which might be populated from previous
    % axes):
    \csname pgfplots@init@cleared@structures\endcsname
    \pgfplotsset{legend style={at={(0,1)},anchor=north west},legend columns=-1,legend style={draw=black,column sep=1ex},
            legend entries={Cts Stoch Ctrl,Dscr Stoch Ctrl,Cts Stoch Ctrl w nFPC,Dscr Stoch Ctrl w nFPC}}%
    
    \csname pgfplots@addlegendimage\endcsname{line width=2pt,mycolor1,sharp plot}
    \csname pgfplots@addlegendimage\endcsname{line width=2pt,mycolor2,sharp plot}
    \csname pgfplots@addlegendimage\endcsname{line width=2pt,mycolor3,sharp plot}
    \csname pgfplots@addlegendimage\endcsname{line width=2pt,mycolor4,sharp plot}

    % draws the legend:
    \csname pgfplots@createlegend\endcsname
    \endgroup
\end{tikzpicture}
}
\end{subfigure}%
  \caption{End of day strategy performances: out-of-sample backtesting on 2014 data, using amalgamated annual 2013 data for calibration.}
  \label{fig:OOS_annual_comp}
\end{figure}

\begin{table}
\centering
\ra{1.2}
\begin{tabular}{@{} *{9}{r} @{}}
\toprule
Strategy & Return & Sharpe & \# MO & \# LO & Inv & \% Win & Max Loss & Max Win \\
\midrule
\multicolumn{9}{l}{\texttt{INTC}} \\ 
Cts & 0.275 & 2.426 & 1617 & 1821 & -0.00 & 1.00 & -0.013 & 0.760 \\ 
Dscr & 0.394 & 1.727 & 987 & 1732 & -0.15 & 1.00 & -0.027 & 1.609 \\ 
Cts w nFPC & 0.525 & 2.056 & 397 & 1611 & 0.07 & 1.00 &  & 2.145 \\ 
Dscr w nFPC & 0.522 & 2.027 & 392 & 1611 & 0.08 & 1.00 &  & 2.277 \\ [2ex]
\multicolumn{9}{l}{\texttt{AAPL}} \\ 
Cts \\
Dscr \\
Cts w nFPC \\
Dscr w nFPC\\
\bottomrule
\end{tabular}
\caption{Averaged strategy performance results: out-of-sample backtesting on 2014 data, using amalgamated annual 2013 data for calibration.}
\label{tbl:OOS_annual}
\end{table}